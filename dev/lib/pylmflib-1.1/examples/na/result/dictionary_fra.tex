\documentclass[twoside,11pt]{book}
\author{Alexis Michaud}
% ajouté en novembre 2016, en remplacement de vmargin
\usepackage[paperwidth=185mm,paperheight=260mm,top=16mm,bottom=16mm,left=15mm,right=20mm]{geometry}
\usepackage{pifont}
\usepackage{fontspec}
\usepackage{natbib}
\usepackage{booktabs}
\usepackage{xltxtra}
\usepackage{polyglossia}
\usepackage[dvipsnames,table]{xcolor}
\usepackage{longtable}
\definecolor{darkblue}{rgb}{0,0,0.75}
\usepackage{float}
\usepackage{lineno}
\usepackage{textcomp}
\usepackage{memhfixc}
\usepackage{lscape}
\usepackage{amssymb}
\usepackage{multicol}
\setlength{\columnseprule}{1pt}
\setlength{\columnsep}{1.5cm}
%\setmainfont[Mapping=tex-text,Numbers=OldStyle,Ligatures=Common]{Charis SIL} 
\setmainfont[Mapping=tex-text,Numbers=OldStyle,Ligatures=Common,ItalicFont=*,ItalicFeatures=FakeSlant,Scale=MatchLowercase]{DoulosSIL}
\newfontfamily\phon[Mapping=tex-text,Ligatures=Common,Scale=MatchLowercase]{Charis SIL} % ou DoulosSIL
\newcommand{\ipa}[1]{{\phon #1}} % API jamais en italique
\newcommand{\grise}[1]{\cellcolor{lightgray}\textbf{#1}}
\newcommand{\bleute}[1]{\cellcolor{green}\textbf{#1}}
\newcommand{\rouge}[1]{\cellcolor{red}\textbf{#1}}
\newfontfamily\cn[Mapping=tex-text,Ligatures=Common,Scale=MatchUppercase]{SimSun} % pour le chinois
\newcommand{\zh}[1]{{\cn #1}}
\newcommand{\topic}{\textsc{dem}}
\newcommand{\tete}{\textsuperscript{\textsc{head}}}
\newcommand{\rc}{\textsubscript{\textsc{rc}}}
\XeTeXlinebreaklocale "zh" % 使用中文换行
\XeTeXlinebreakskip = 0pt plus 1pt  % CIRCG
\usepackage{fancyhdr}
\pagestyle{fancy}
\fancyheadoffset{3.4em}
\fancyhead[LE,RO]{\thepage} % Numéros de page sur les côtés extérieurs
% \setlength{\oddsidemargin}{12mm}
% \setlength{\evensidemargin}{18mm}
\usepackage[dvipdfmx,xetex,bigfiles,final,activate=onclick,deactivate=onclick,transparent,passcontext]{media9}
\usepackage{graphicx}
\usepackage[bookmarks=true,colorlinks,linkcolor=blue]{hyperref}
\hypersetup{bookmarks=false,bookmarksnumbered,bookmarksopenlevel=5,bookmarksdepth=5,xetex,colorlinks=true,linkcolor=blue,citecolor=blue}
\usepackage[all]{hypcap}
%%% retirés en novembre 2016 : pas utiles. Remplacé par: geometry.
%%\usepackage{vmargin}
%%% {marge gauche}{marge en haut}{marge droite}{marge en bas}{hauteur de l'entête}{distance entre l'entête et le texte}{hauteur du pied de page}{distance entre le texte et le pied de page}
%%\setmarginsrb{2cm}{1cm}{1.5cm}{1cm}{0.5cm}{1cm}{0.5cm}{1cm}

\def\mytextsc{\bgroup\obeyspaces\mytextscaux}
\def\mytextscaux#1{\mytextscauxii #1\relax\relax\egroup}
\def\mytextscauxii#1{%
\ifx\relax#1\else \ifcat#1\@sptoken{} \expandafter\expandafter\expandafter\mytextscauxii\else
\ifnum`#1=\uccode`#1 {\normalsize #1}\else {\footnotesize \uppercase{#1}}\fi \expandafter\expandafter\expandafter\mytextscauxii\expandafter\fi\fi}
% Ne pas afficher la numérotation des sections, sous-sections etc
\setcounter{secnumdepth}{0}
\newfontfamily\englishfont{Linux Libertine O}
\newfontfamily\frenchfont{EB Garamond}



\title{Dictionnaire na-chinois-français}
\setdefaultlanguage{french}
\setotherlanguages{chinese}

\begin{document}
\maketitle
\newpage
\markboth{INTRODUCTION}{}

	{\LARGE \textbf{Note introductive}}
	
La base de données lexicographique réalisée pour le na de Yongning est quadrilingue: les mots et exemples sont traduits en chinois, français et anglais. Pour le dictionnaire en ligne (HTML) et au format PDF, deux versions sont proposées: le présent document, na-chinois-français; et une version na-chinois-anglais. Ce choix paraissait préférable à la réalisation d'une unique version quadrilingue, car la présence du français et de l'anglais côte à côte, même distingués par la typographie, paraissait gêner la lecture.

La version française est pour l'heure essentiellement utilisée par son auteur; en conséquence, il ne paraisssait pas urgent de la doter d'une introduction en français, et on s'est contenté d'une introduction en anglais (pour la version na-chinois-anglais), à laquelle est renvoyé le lecteur souhaitant obtenir des précisions au sujet de ce dictionnaire. 

La présente version, première version publique, est numérotée 1.0. Il est prévu des mises à jour tous les ans ou tous les deux ans environ, à mesure de l'enrichissement du dictionnaire. Les lecteurs sont vivement encouragés à prendre contact (michaud.cnrs@gmail.com) pour signaler des erreurs et proposer des améliorations.



\def\mytextsc{\bgroup\obeyspaces\mytextscaux}
\def\mytextscaux#1{\mytextscauxii #1\relax\relax\egroup}
\def\mytextscauxii#1{%
\ifx\relax#1\else \ifcat#1\@sptoken{} \expandafter\expandafter\expandafter\mytextscauxii\else
\ifnum`#1=\uccode`#1 {\normalsize #1}\else {\footnotesize \uppercase{#1}}\fi \expandafter\expandafter\expandafter\mytextscauxii\expandafter\fi\fi}

\setlength\parindent{0cm}

\addmediapath{/home/benjamin/Projets/pylmflib-1.1/examples/na/mp3}
\addmediapath{/home/benjamin/Projets/pylmflib-1.1/examples/na/mp3/mp3}
\addmediapath{/home/benjamin/Projets/pylmflib-1.1/examples/na/mp3/wav}
\graphicspath{{/home/benjamin/Projets/pylmflib-1.1/examples/na/pylmflib/output/img/}}

\newpage
\begin{multicols}{2}

\newpage
\section*{\centering- \textcolor{darkblue}{\textbf{\ipa{ɑ}}} -}
\subsection{\hspace{-0.5cm} {\Large \textcolor{darkblue}{\textbf{\ipa{ɑ˩mi\#˥}}}}\hspace{0.5cm}[\kern2pt{\textcolor{darkblue}{\textbf{\ipa{ɑ˩mi˥}}}}\kern2pt]} \hypertarget{A\string_Bmi\#\string_T1}{}
\markboth{\textcolor{darkblue}{\textbf{\ipa{ɑ˩mi\#˥}}}}{}
\textcolor{teal}{\mytextsc{nom}} \hspace{4pt} Ton~: LM+\#H.
\textcolor{PineGreen}{\selectlanguage{french}Oie (femelle).} \zh{母鹅。}  ¶ \textcolor{darkblue}{\textbf{\ipa{ɑ˩mi˧-ɑ˥pʰv̩˩}}} \textcolor{PineGreen}{\selectlanguage{french}oie et jars} \zh{公鹅与母鹅}  
 \zh{量词}: \textcolor{darkblue}{\textbf{\ipa{mi˩}}}  \mytextsc{clf}~: \textcolor{darkblue}{\textbf{\ipa{mi˩}}} 
\lhead{\firstmark}
\rhead{\botmark}

\subsection{\hspace{-0.5cm} {\Large \textcolor{darkblue}{\textbf{\ipa{ɑ˩pʰo˩}}}}\hspace{0.5cm}[\kern2pt{\textcolor{darkblue}{\textbf{\ipa{ɑ˩pʰo˩˥}}}}\kern2pt]} \hypertarget{A\string_Bp\string_ho\string_B1}{}
\markboth{\textcolor{darkblue}{\textbf{\ipa{ɑ˩pʰo˩}}}}{}
\textcolor{teal}{\mytextsc{adverbe}} \hspace{4pt} Ton~: L.
\textcolor{PineGreen}{\selectlanguage{french}Dehors, à l'extérieur.} \zh{外面。}  ¶ \textcolor{darkblue}{\textbf{\ipa{ɑ˩pʰo˩ bi˩˥}}} \textcolor{PineGreen}{\selectlanguage{french}aller dehors; aller faire ses besoins} \zh{出去,解手}  
 ¶ \textcolor{darkblue}{\textbf{\ipa{ɑ˩pʰo˩ bi˩-ze˥!}}} \textcolor{PineGreen}{\selectlanguage{french}On sort! / [Je] vais faire mes besoins!} \zh{出去了! / 出去解手!}  
 ¶ \textcolor{darkblue}{\textbf{\ipa{ɑ˩pʰo˩ bi˩-ʂv̩˥ɖv̩˩!}}} \textcolor{PineGreen}{\selectlanguage{french}[Elle] a envie de sortir! (Contexte: par une journée ensoleillée, un membre de la famille sent qu'un enfant aurait envie d'aller jouer dehors.)} \zh{(她)想出去!(情景:婴儿看外边,觉得她好像想出去。)}  
 ¶ \textcolor{darkblue}{\textbf{\ipa{ə˧dɑ˥ | ə˩pʰo˩ hɯ˩-ze˥!}}} \textcolor{PineGreen}{\selectlanguage{french}Papa est sorti! (Adressé à une petite fille qui vient de se réveiller de sa sieste et cherche son père.)} \zh{爸爸出去了!}  
 ¶ \textcolor{darkblue}{\textbf{\ipa{ə˩pʰo˩-bv̩˥ | lo˧ ʝi˧}}} \textcolor{PineGreen}{\selectlanguage{french}travailler à l'extérieur, aider d'autres familles (par exemple pour la récolte); aussi : aller chercher du travail à la ville} \zh{在外边工作:去帮别人家的忙(特别是收庄稼的时候),或者到城市里面打工}  

\lhead{\firstmark}
\rhead{\botmark}

\subsection{\hspace{-0.5cm} {\Large \textcolor{darkblue}{\textbf{\ipa{ɑ˩pʰo˩-hĩ˩}}}}\hspace{0.5cm}[\kern2pt{\textcolor{darkblue}{\textbf{\ipa{xxxx non-correspondance entre le nombre de morphèmes et le nombre de tons de morphèmes}}}}\kern2pt]} \hypertarget{A\string_Bp\string_ho\string_B-hi\string_~\string_B1}{}
\markboth{\textcolor{darkblue}{\textbf{\ipa{ɑ˩pʰo˩-hĩ˩}}}}{}
\textcolor{teal}{\mytextsc{nom}} \hspace{4pt} Ton~: L.
\textcolor{PineGreen}{\selectlanguage{french}Quelqu'un d'autre; une personne extérieure à la famille.} \zh{外人,别人。农村人称呼所有城里人为外面的人。} 
\lhead{\firstmark}
\rhead{\botmark}

\subsection{\hspace{-0.5cm} {\Large \textcolor{darkblue}{\textbf{\ipa{ɑ˩pʰv̩˧˥}}}}\hspace{0.5cm}[\kern2pt{\textcolor{darkblue}{\textbf{\ipa{ɑ˩pʰv̩˧˥}}}}\kern2pt]} \hypertarget{A\string_Bp\string_hv\string_=\string_M\string_T1}{}
\markboth{\textcolor{darkblue}{\textbf{\ipa{ɑ˩pʰv̩˧˥}}}}{}
\textcolor{teal}{\mytextsc{verbe}} \hspace{4pt} Ton~: LM+MH\#.
\textcolor{PineGreen}{\selectlanguage{french}Roter.} \zh{打饱嗝儿。} 
\lhead{\firstmark}
\rhead{\botmark}

\subsection{\hspace{-0.5cm} {\Large \textcolor{darkblue}{\textbf{\ipa{ɑ˩pʰv̩\#˥}}}}\hspace{0.5cm}[\kern2pt{\textcolor{darkblue}{\textbf{\ipa{ɑ˩pʰv̩˥}}}}\kern2pt]} \hypertarget{A\string_Bp\string_hv\string_=\#\string_T1}{}
\markboth{\textcolor{darkblue}{\textbf{\ipa{ɑ˩pʰv̩\#˥}}}}{}
\textcolor{teal}{\mytextsc{nom}} \hspace{4pt} Ton~: LM+\#H.
\textcolor{PineGreen}{\selectlanguage{french}Jars: mâle de l'oie.} \zh{公鹅。}  \zh{量词}: \textcolor{darkblue}{\textbf{\ipa{mi˩}}}  \mytextsc{clf}~: \textcolor{darkblue}{\textbf{\ipa{mi˩}}} 
\lhead{\firstmark}
\rhead{\botmark}

\subsection{\hspace{-0.5cm} {\Large \textcolor{darkblue}{\textbf{\ipa{ɑ˩ʁo˧}}}}\hspace{0.5cm}[\kern2pt{\textcolor{darkblue}{\textbf{\ipa{ɑ˩ʁo˥}}}}\kern2pt]} \hypertarget{A\string_BRo\string_M1}{}
\markboth{\textcolor{darkblue}{\textbf{\ipa{ɑ˩ʁo˧}}}}{}
\textcolor{teal}{\mytextsc{nom}} \hspace{4pt} Ton~: LM.
\textcolor{PineGreen}{\selectlanguage{french}Le foyer, la pièce principale; la maison.} \zh{家、家里。}  ¶ \textcolor{darkblue}{\textbf{\ipa{ɑ˩ʁo˧-hĩ\#˥}}} \textcolor{PineGreen}{\selectlanguage{french}la maisonnée, les gens de la famille (proche: ceux qui habitent sous le même toit), la lignée} \zh{家人(住在一起的家人)}  
 ¶ \textcolor{darkblue}{\textbf{\ipa{ɑ˩ʁo˧=ɻæ˩}}} \textcolor{PineGreen}{\selectlanguage{french}la maisonnée, les gens de la famille} \zh{家人、家族(住在一起的家人)}  
 ¶ \textcolor{darkblue}{\textbf{\ipa{njɤ˧ | ɑ˩ʁo˧}}} \textcolor{PineGreen}{\selectlanguage{french}mon foyer, ma maison} \zh{我家}  
 ¶ \textcolor{darkblue}{\textbf{\ipa{njɤ˧ | ɑ˩ʁo˧=ɻ̍˩}}} \textcolor{PineGreen}{\selectlanguage{french}ma famille, ma lignée} \zh{我的家族}  
 ¶ \textcolor{darkblue}{\textbf{\ipa{no˧ | ɑ˩ʁo˧}}} \textcolor{PineGreen}{\selectlanguage{french}ton foyer, ta maison} \zh{你家}  
 ¶ \textcolor{darkblue}{\textbf{\ipa{ɑ˩ʁo˧ ʝi˧}}} \textcolor{PineGreen}{\selectlanguage{french}gérer la maisonnée, s'occuper de la famille (tâche de la personne qui répartit les travaux à accomplir, veille aux approvisionnements...)} \zh{管家里的事情(如:分配工作、家务等)}  
 \zh{量词}: \textcolor{darkblue}{\textbf{\ipa{ɭɯ˧}}}  \mytextsc{clf}~: \textcolor{darkblue}{\textbf{\ipa{ɭɯ˧}}} 
\lhead{\firstmark}
\rhead{\botmark}

\subsection{\hspace{-0.5cm} {\Large \textcolor{darkblue}{\textbf{\ipa{ɑ˩zo\#˥}}}}\hspace{0.5cm}[\kern2pt{\textcolor{darkblue}{\textbf{\ipa{ɑ˩zo˥}}}}\kern2pt]} \hypertarget{A\string_Bzo\#\string_T1}{}
\markboth{\textcolor{darkblue}{\textbf{\ipa{ɑ˩zo\#˥}}}}{}
\textcolor{teal}{\mytextsc{nom}} \hspace{4pt} Ton~: LM+\#H.
\textcolor{PineGreen}{\selectlanguage{french}Oison, petit de l'oie.} \zh{小鹅。} 
\lhead{\firstmark}
\rhead{\botmark}

\subsection{\hspace{-0.5cm} {\Large \textcolor{darkblue}{\textbf{\ipa{ɑ˩˧}}}}\hspace{0.5cm}[\kern2pt{\textcolor{darkblue}{\textbf{\ipa{ɑ˩˥}}}}\kern2pt]} \hypertarget{A\string_B\string_M1}{}
\markboth{\textcolor{darkblue}{\textbf{\ipa{ɑ˩˧}}}}{}
\textcolor{teal}{\mytextsc{nom}} \hspace{4pt} Ton~: LM.
\textcolor{PineGreen}{\selectlanguage{french}Oie.} \zh{鹅。}  ¶ \textcolor{darkblue}{\textbf{\ipa{ɑ˩ dzɯ˥-ze˩}}} \textcolor{PineGreen}{\selectlanguage{french}...a mangé (une) oie} \zh{吃了鹅}  
 ¶ \textcolor{darkblue}{\textbf{\ipa{ɑ˩ hwæ˧-ze˧}}} \textcolor{PineGreen}{\selectlanguage{french}...a acheté (une) oie} \zh{买了鹅}  
 \zh{量词}: \textcolor{darkblue}{\textbf{\ipa{mi˩}}}  \mytextsc{clf}~: \textcolor{darkblue}{\textbf{\ipa{mi˩}}} 
\lhead{\firstmark}
\rhead{\botmark}

\newpage
\section*{\centering- \textcolor{darkblue}{\textbf{\ipa{æ}}} \textcolor{darkblue}{\textbf{\ipa{æ̃}}} -}
\subsection{\hspace{-0.5cm} {\Large \textcolor{darkblue}{\textbf{\ipa{æ˧bæ˧}}}}\hspace{0.5cm}[\kern2pt{\textcolor{darkblue}{\textbf{\ipa{æ˧bæ˧}}}}\kern2pt]} \hypertarget{\{\string_Mb\{\string_M1}{}
\markboth{\textcolor{darkblue}{\textbf{\ipa{æ˧bæ˧}}}}{}
\textcolor{teal}{\mytextsc{nom}} \hspace{4pt} Ton~: M.
\textcolor{PineGreen}{\selectlanguage{french}Goître.} \zh{甲状腺肿瘤。}  \zh{量词}: \textcolor{darkblue}{\textbf{\ipa{ɭɯ˧}}}  \mytextsc{clf}~: \textcolor{darkblue}{\textbf{\ipa{ɭɯ˧}}} 
\lhead{\firstmark}
\rhead{\botmark}

\subsection{\hspace{-0.5cm} {\Large \textcolor{darkblue}{\textbf{\ipa{æ˧bæ˧-ʈʂʰæ˧ɣɯ\#˥}}}}\hspace{0.5cm}[\kern2pt{\textcolor{darkblue}{\textbf{\ipa{xxxx non-correspondance entre le nombre de morphèmes et le nombre de tons de morphèmes}}}}\kern2pt]} \hypertarget{\{\string_Mb\{\string_M-t`s`\string_h\{\string_MGM\#\string_T1}{}
\markboth{\textcolor{darkblue}{\textbf{\ipa{æ˧bæ˧-ʈʂʰæ˧ɣɯ\#˥}}}}{}
\textcolor{teal}{\mytextsc{nom}} \hspace{4pt} Ton~: \#H.
\textcolor{PineGreen}{\selectlanguage{french}Algue; littéralement “médicament contre le goître” car tel était le motif de la diffusion à Yongning de l'algue, qui contient de l'iode.} \zh{海带。} 
\lhead{\firstmark}
\rhead{\botmark}

\subsection{\hspace{-0.5cm} {\Large \textcolor{darkblue}{\textbf{\ipa{æ˧ʝi˩}}}}\hspace{0.5cm}[\kern2pt{\textcolor{darkblue}{\textbf{\ipa{æ˧ʝi˩}}}}\kern2pt]} \hypertarget{\{\string_Mj££i\string_B1}{}
\markboth{\textcolor{darkblue}{\textbf{\ipa{æ˧ʝi˩}}}}{}
\textcolor{teal}{\mytextsc{nom}} \hspace{4pt} Ton~: L\#.
\textcolor{PineGreen}{\selectlanguage{french}Cri.} \zh{叫声。}  ¶ \textcolor{darkblue}{\textbf{\ipa{æ˧ʝi˩ kʰɯ˩}}} \textcolor{PineGreen}{\selectlanguage{french}crier} \zh{叫}  
 ¶ \textcolor{darkblue}{\textbf{\ipa{no˧ | æ˧ʝi˩ tʰɑ˩-kʰɯ˩! | no˧ se˧dʑæ˩ɻæ˩-gv˩! |}}} \textcolor{PineGreen}{\selectlanguage{french}Ne fais pas tant de bruit! (Contexte: réprimande qu'on adressait aux gens qui parlaient trop fort, qui élevaient la voix.)} \zh{别那么大声!}  

\lhead{\firstmark}
\rhead{\botmark}

\subsection{\hspace{-0.5cm} {\Large \textcolor{darkblue}{\textbf{\ipa{æ˧ɲi\#˥}}}}\hspace{0.5cm}[\kern2pt{\textcolor{darkblue}{\textbf{\ipa{æ˧ɲi˧}}}}\kern2pt]} \hypertarget{\{\string_MJi\#\string_T1}{}
\markboth{\textcolor{darkblue}{\textbf{\ipa{æ˧ɲi\#˥}}}}{}
\textcolor{teal}{\mytextsc{nom}} \hspace{4pt} Ton~: \#H.
\textcolor{PineGreen}{\selectlanguage{french}Clarinette.} \zh{唢呐。}  \zh{量词}: \textcolor{darkblue}{\textbf{\ipa{ɭɯ˧}}}  \mytextsc{clf}~: \textcolor{darkblue}{\textbf{\ipa{ɭɯ˧}}} 
\lhead{\firstmark}
\rhead{\botmark}

\subsection{\hspace{-0.5cm} {\Large \textcolor{darkblue}{\textbf{\ipa{æ˧ʁwæ˧}}}}\hspace{0.5cm}[\kern2pt{\textcolor{darkblue}{\textbf{\ipa{æ˧ʁwæ˧}}}}\kern2pt]} \hypertarget{\{\string_MRw\{\string_M1}{}
\markboth{\textcolor{darkblue}{\textbf{\ipa{æ˧ʁwæ˧}}}}{}
\textcolor{teal}{\mytextsc{nom}} \hspace{4pt} Ton~: M.
\textcolor{PineGreen}{\selectlanguage{french}Abricot.} \zh{杏。}  ¶ \textcolor{darkblue}{\textbf{\ipa{æ˧ʁwæ˧ | ɖɯ˧-ɭɯ˧}}} \textcolor{PineGreen}{\selectlanguage{french}un abricot} \zh{一颗杏}  
 \zh{量词}: \textcolor{darkblue}{\textbf{\ipa{ɭɯ˧}}}  \mytextsc{clf}~: \textcolor{darkblue}{\textbf{\ipa{ɭɯ˧}}} 
\lhead{\firstmark}
\rhead{\botmark}

\subsection{\hspace{-0.5cm} {\Large \textcolor{darkblue}{\textbf{\ipa{æ˧ʂæ\#˥}}}}\hspace{0.5cm}[\kern2pt{\textcolor{darkblue}{\textbf{\ipa{æ˧ʂæ˧}}}}\kern2pt]} \hypertarget{\{\string_Ms`\{\#\string_T1}{}
\markboth{\textcolor{darkblue}{\textbf{\ipa{æ˧ʂæ\#˥}}}}{}
\textcolor{teal}{\mytextsc{adverbe}} \hspace{4pt} Ton~: \#H.
\textcolor{PineGreen}{\selectlanguage{french}Jadis, dans le passé.} \zh{从前。}  ¶ \textcolor{darkblue}{\textbf{\ipa{æ˧ʂæ˧-kɤ˥ʈʂɯ˩}}} \textcolor{PineGreen}{\selectlanguage{french}Dires du temps jadis, traditions orales} \zh{过去的说法,过去的口传文化}  

\lhead{\firstmark}
\rhead{\botmark}

\subsection{\hspace{-0.5cm} {\Large \textcolor{darkblue}{\textbf{\ipa{æ˧ʂæ˧}}}}\hspace{0.5cm}[\kern2pt{\textcolor{darkblue}{\textbf{\ipa{æ˧ʂæ˧}}}}\kern2pt]} \hypertarget{\{\string_Ms`\{\string_M1}{}
\markboth{\textcolor{darkblue}{\textbf{\ipa{æ˧ʂæ˧}}}}{}
\textcolor{teal}{\mytextsc{nom}} \hspace{4pt} Ton~: M.
\textcolor{PineGreen}{\selectlanguage{french}Nom d'une montagne: l'une des deux principales montagnes autour de la plaine de Yongning, la montagne masculine (“le jeune homme”: \textcolor{darkblue}{\textbf{\ipa{/pʰæ˧tɕi˥/}}}); l'autre étant la montagne \textcolor{darkblue}{\textbf{\ipa{/kɤ˧mv̩˧˥/}}}, montagne féminine (“la jeune femme”: \textcolor{darkblue}{\textbf{\ipa{mi˩zɯ˩˥/}}}).} \zh{一座山的名字。}  ¶ \textcolor{darkblue}{\textbf{\ipa{kɤ˧mv̩˧˥, | æ˧ʂæ˧, | ŋwɤ˧hɑ̃˩, | ʂwæ˧gv̩\#˥, | nɑ˩tsʰi˩˥ | -tɕʰɤ˧pɤ˧mi\#˥, | qv̩˧ɻ̍˧-ʈʂʰɑ˧nɑ˥ |}}} \textcolor{PineGreen}{\selectlanguage{french}Les six montagnes de Yongning qui portent un nom. Les autres sommets du voisinage n'ont pas une valeur symbolique comparable, et ne portent pas de nom communément utilisé.} \zh{永宁地区有固定名字的六座山。其它的山,因为没有重要的象征意义,因此没有取名。}  

\lhead{\firstmark}
\rhead{\botmark}

\subsection{\hspace{-0.5cm} {\Large \textcolor{darkblue}{\textbf{\ipa{æ˧ʂæ˧-pi˧mv̩˧˥}}}}\hspace{0.5cm}[\kern2pt{\textcolor{darkblue}{\textbf{\ipa{xxxx non-correspondance entre le nombre de morphèmes et le nombre de tons de morphèmes}}}}\kern2pt]} \hypertarget{\{\string_Ms`\{\string_M-pi\string_Mmv\string_=\string_M\string_T1}{}
\markboth{\textcolor{darkblue}{\textbf{\ipa{æ˧ʂæ˧-pi˧mv̩˧˥}}}}{}
\textcolor{teal}{\mytextsc{nom}} \hspace{4pt} Ton~: MH\#.
\textit{De:} \textbf{æ˧ʂæ\#˥ et pi˧mv̩˥\$} \textcolor{PineGreen}{\selectlanguage{french}Conte, récit du temps jadis; terme plus familier que \textcolor{darkblue}{\textbf{\ipa{/æ˧ʂæ˧-tɑ˩mv̩˩/}}}.} \zh{传统故事。}  \zh{量词}: \textcolor{darkblue}{\textbf{\ipa{kʰwɤ˥}}}  \mytextsc{clf}~: \textcolor{darkblue}{\textbf{\ipa{kʰwɤ˥}}} 
\lhead{\firstmark}
\rhead{\botmark}

\subsection{\hspace{-0.5cm} {\Large \textcolor{darkblue}{\textbf{\ipa{æ˧ʂæ˧-qʰwæ\#˥}}}}\hspace{0.5cm}[\kern2pt{\textcolor{darkblue}{\textbf{\ipa{xxxx non-correspondance entre le nombre de morphèmes et le nombre de tons de morphèmes}}}}\kern2pt]} \hypertarget{\{\string_Ms`\{\string_M-q\string_hw\{\#\string_T1}{}
\markboth{\textcolor{darkblue}{\textbf{\ipa{æ˧ʂæ˧-qʰwæ\#˥}}}}{}
\textcolor{teal}{\mytextsc{nom}} \hspace{4pt} Ton~: \#H.
\textit{De:} \textbf{æ˧ʂæ\#˥ et qʰwæ˧} \textcolor{PineGreen}{\selectlanguage{french}Tradition orale; littéralement: “messages du temps jadis”.} \zh{口传文化。直译:“(来自)古时候的寓意”。}  ¶ \textcolor{darkblue}{\textbf{\ipa{æ˧ʂæ˧-qʰwæ˧-ɳɯ˥ | dʑo˩-ɲi˥!}}} \textcolor{PineGreen}{\selectlanguage{french}“(C'est pas moi qui invente ça:) c'est un dicton d'autrefois/ c'est quelque chose qui existe dans la tradition!” (Commentaire de quelqu'un qui cite un proverbe/dicton, et souligne qu'il ne s'agit pas de paroles en l'air, mais de vérités.)} \zh{“(这些道理,不是我个人的意见:)传统中是这样讲的! / 咱们的口传文化中就是这么讲的!”(情景:一个人提到一个谚语,也强调这些不是空话,而是重要的一个道理。)}  

\lhead{\firstmark}
\rhead{\botmark}

\subsection{\hspace{-0.5cm} {\Large \textcolor{darkblue}{\textbf{\ipa{æ˧ʂæ˧-qʰwɤ˧˥}}}}\hspace{0.5cm}[\kern2pt{\textcolor{darkblue}{\textbf{\ipa{xxxx non-correspondance entre le nombre de morphèmes et le nombre de tons de morphèmes}}}}\kern2pt]} \hypertarget{\{\string_Ms`\{\string_M-q\string_hw7\string_M\string_T1}{}
\markboth{\textcolor{darkblue}{\textbf{\ipa{æ˧ʂæ˧-qʰwɤ˧˥}}}}{}
\textcolor{teal}{\mytextsc{nom}} \hspace{4pt} Ton~: MH\#.
\textcolor{PineGreen}{\selectlanguage{french}Histoire, conte, récit traditionnel.} \zh{故事。}  ¶ \textcolor{darkblue}{\textbf{\ipa{æ˧ʂæ˧qʰwɤ˧ ʐwɤ˧˥}}} \textcolor{PineGreen}{\selectlanguage{french}raconter une histoire} \zh{讲故事}  
 ¶ \textcolor{darkblue}{\textbf{\ipa{ə˧ʝi˧-ʂɯ˥ʝi˩, | æ˧ʂæ˧qʰwɤ˧ ʐwɤ˧-kv̩˥!}}} \textcolor{PineGreen}{\selectlanguage{french}dans le temps, (on) racontait des histoires!} \zh{在过去,大家经常讲故事!}  
 \zh{量词}: \textcolor{darkblue}{\textbf{\ipa{kʰwɤ˥}}}  \mytextsc{clf}~: \textcolor{darkblue}{\textbf{\ipa{kʰwɤ˥}}} 
\lhead{\firstmark}
\rhead{\botmark}

\subsection{\hspace{-0.5cm} {\Large \textcolor{darkblue}{\textbf{\ipa{æ˧tse˥-pʰæ˩}}}}\hspace{0.5cm}[\kern2pt{\textcolor{darkblue}{\textbf{\ipa{æ˧tse˥pʰæ˩}}}}\kern2pt]} \hypertarget{\{\string_Mtse\string_T-p\string_h\{\string_B1}{}
\markboth{\textcolor{darkblue}{\textbf{\ipa{æ˧tse˥-pʰæ˩}}}}{}
\textcolor{teal}{\mytextsc{nom}} \hspace{4pt} Ton~: H\#-L.
\textcolor{PineGreen}{\selectlanguage{french}Os du genou.} \zh{膝盖骨。}  \zh{量词}: \textcolor{darkblue}{\textbf{\ipa{ɭɯ˧}}}  \mytextsc{clf}~: \textcolor{darkblue}{\textbf{\ipa{ɭɯ˧}}} 
\lhead{\firstmark}
\rhead{\botmark}

\subsection{\hspace{-0.5cm} {\Large \textcolor{darkblue}{\textbf{\ipa{æ˧tsɯ˥-pɤ˩lv̩˩}}}}\hspace{0.5cm}[\kern2pt{\textcolor{darkblue}{\textbf{\ipa{æ˧tsɯ˥pɤ˩lv̩˩}}}}\kern2pt]} \hypertarget{\{\string_MtsM\string_T-p7\string_Blv\string_=\string_B1}{}
\markboth{\textcolor{darkblue}{\textbf{\ipa{æ˧tsɯ˥-pɤ˩lv̩˩}}}}{}
\textcolor{teal}{\mytextsc{nom}} \hspace{4pt} Ton~: H\#-L.
\textcolor{PineGreen}{\selectlanguage{french}Nuque.} \zh{项背 、项、脖颈儿。}  \zh{量词}: \textcolor{darkblue}{\textbf{\ipa{ɭɯ˧}}}  \mytextsc{clf}~: \textcolor{darkblue}{\textbf{\ipa{ɭɯ˧}}} 
\lhead{\firstmark}
\rhead{\botmark}

\subsection{\hspace{-0.5cm} {\Large \textcolor{darkblue}{\textbf{\ipa{æ˧ʈwɤ˩}}}}\hspace{0.5cm}[\kern2pt{\textcolor{darkblue}{\textbf{\ipa{æ˧ʈwɤ˩}}}}\kern2pt]} \hypertarget{\{\string_Mt`w7\string_B1}{}
\markboth{\textcolor{darkblue}{\textbf{\ipa{æ˧ʈwɤ˩}}}}{}
\textcolor{teal}{\mytextsc{nom}} \hspace{4pt} Ton~: L\#.
\textcolor{PineGreen}{\selectlanguage{french}Le petit matin.} \zh{清晨、一大早(鸡叫的时候)。} 
\lhead{\firstmark}
\rhead{\botmark}

\subsection{\hspace{-0.5cm} {\Large \textcolor{darkblue}{\textbf{\ipa{æ˩gv̩˩}}}}\hspace{0.5cm}[\kern2pt{\textcolor{darkblue}{\textbf{\ipa{æ˩gv̩˩˥}}}}\kern2pt]} \hypertarget{\{\string_Bgv\string_=\string_B1}{}
\markboth{\textcolor{darkblue}{\textbf{\ipa{æ˩gv̩˩}}}}{}
\textcolor{teal}{\mytextsc{nom}} \hspace{4pt} Ton~: L.
\textcolor{PineGreen}{\selectlanguage{french}Araire. Il n'existe pas deux termes distincts, l'un pour l'araire et l'autre pour la charrue; à la date de l'enquête, seule l'araire était en usage.} \zh{犁头。}  ¶ \textcolor{darkblue}{\textbf{\ipa{æ˩gv̩˩ tʰv̩˩-nɑ˥}}} \textcolor{PineGreen}{\selectlanguage{french}\mytextsc{n}+\mytextsc{dem}+\mytextsc{clf}} \zh{这把犁头}  
 ¶ \textcolor{darkblue}{\textbf{\ipa{æ˩mo˥}}} \textcolor{PineGreen}{\selectlanguage{french}araire usagée, vieille araire (hors d'usage du fait de l'usure)} \zh{陈旧的犁头(不能再用了)}  
 ¶ \textcolor{darkblue}{\textbf{\ipa{æ˩mo˥ tʰv̩˩-nɑ˩}}} \textcolor{PineGreen}{\selectlanguage{french}\mytextsc{n}+\mytextsc{dem}+\mytextsc{clf}} \zh{这个旧犁杆}  
 ¶ \textcolor{darkblue}{\textbf{\ipa{æ˩-ʂɯ˩˥}}} \textcolor{PineGreen}{\selectlanguage{french}araire neuve} \zh{新的犁头}  
 \zh{量词}: \textcolor{darkblue}{\textbf{\ipa{nɑ˧}}}  \mytextsc{clf}~: \textcolor{darkblue}{\textbf{\ipa{nɑ˧}}} 
\lhead{\firstmark}
\rhead{\botmark}

\subsection{\hspace{-0.5cm} {\Large \textcolor{darkblue}{\textbf{\ipa{æ˩gv̩˩-mæ˩qo˥}}}}\hspace{0.5cm}[\kern2pt{\textcolor{darkblue}{\textbf{\ipa{xxxx non-correspondance entre le nombre de morphèmes et le nombre de tons de morphèmes}}}}\kern2pt]} \hypertarget{\{\string_Bgv\string_=\string_B-m\{\string_Bqo\string_T1}{}
\markboth{\textcolor{darkblue}{\textbf{\ipa{æ˩gv̩˩-mæ˩qo˥}}}}{}
\textcolor{teal}{\mytextsc{nom}} \hspace{4pt} Ton~: L+H\#.
\textcolor{PineGreen}{\selectlanguage{french}Mancheron de l'araire, manche de l'araire.} \zh{犁把。}  ¶ \textcolor{darkblue}{\textbf{\ipa{æ̃˩gv̩˩-mæ˩ ʑi˩-hĩ˥}}} \textcolor{PineGreen}{\selectlanguage{french}la personne qui tient le mancheron de la charrue} \zh{抓着犁把的人}  
 ¶ \textcolor{darkblue}{\textbf{\ipa{æ̃˩gv̩˩-mæ˩qo˥ tʰv̩˩-nɑ˩}}} \textcolor{PineGreen}{\selectlanguage{french}\mytextsc{n}+\mytextsc{dem}+\mytextsc{clf}} \zh{这个犁把}  
 \zh{量词}: \textcolor{darkblue}{\textbf{\ipa{nɑ˧}}}  \mytextsc{clf}~: \textcolor{darkblue}{\textbf{\ipa{nɑ˧}}} 
\lhead{\firstmark}
\rhead{\botmark}

\subsection{\hspace{-0.5cm} {\Large \textcolor{darkblue}{\textbf{\ipa{æ˩mi˧-mv̩˧ʈv̩˥}}}}\hspace{0.5cm}[\kern2pt{\textcolor{darkblue}{\textbf{\ipa{xxxx non-correspondance entre le nombre de morphèmes et le nombre de tons de morphèmes}}}}\kern2pt]} \hypertarget{\{\string_Bmi\string_M-mv\string_=\string_Mt`v\string_=\string_T1}{}
\markboth{\textcolor{darkblue}{\textbf{\ipa{æ˩mi˧-mv̩˧ʈv̩˥}}}}{}
\textcolor{teal}{\mytextsc{nom}} \hspace{4pt} Ton~: LM+H\#.
\textcolor{PineGreen}{\selectlanguage{french}Astragale (os du dessus du pied; os “en boule” sur les deux côtés du pied, au-dessous de la cheville).} \zh{踝骨。}  \zh{量词}: \textcolor{darkblue}{\textbf{\ipa{ɭɯ˧}}}  \mytextsc{clf}~: \textcolor{darkblue}{\textbf{\ipa{ɭɯ˧}}} 
\lhead{\firstmark}
\rhead{\botmark}

\subsection{\hspace{-0.5cm} {\Large \textcolor{darkblue}{\textbf{\ipa{æ˩mi˧-ʁwɤ\#˥}}}}\hspace{0.5cm}[\kern2pt{\textcolor{darkblue}{\textbf{\ipa{xxxx non-correspondance entre le nombre de morphèmes et le nombre de tons de morphèmes}}}}\kern2pt]} \hypertarget{\{\string_Bmi\string_M-Rw7\#\string_T1}{}
\markboth{\textcolor{darkblue}{\textbf{\ipa{æ˩mi˧-ʁwɤ\#˥}}}}{}
\textcolor{teal}{\mytextsc{nom}} \hspace{4pt} Ton~: LM+\#H.
\textcolor{PineGreen}{\selectlanguage{french}Amiwa, premier village que l'on rencontre sur la route entre \textcolor{darkblue}{\textbf{\ipa{/qʰæ˧tɕʰi˧/}}} et \textcolor{darkblue}{\textbf{\ipa{/ʈʂo˧ʂɯ\#˥/}}}. Dans la géographie traditionnelle na, qui prend comme point d'origine le lac Lugu, Amiwa est le troisième village de la plaine de Yongning.} \zh{阿咪瓦、阿米瓦(永宁的一个村落)。}  ¶ \textcolor{darkblue}{\textbf{\ipa{dʑɤ˩bv̩˧kɤ˧-sɑ˥ʁwɤ˩, | hi˩ʁwɤ˩-lo˥, | æ˩mi˧-ʁwɤ\#˥, | lɑ˧lo˧-ʁwɤ˥, | lɑ˧ŋwɤ˧, | bɤ˧tsʰo˧gv̩˥, | ə˧lɑ˧-ʁwɤ\#˥, | gæ˧ɻæ˩, | qʰæ˧tɕʰi˧, | tʰo˧ʈɯ\#˥}}} \textcolor{PineGreen}{\selectlanguage{french}les dix villages comptant traditionnellement comme faisant partie de Yongning} \zh{摩梭传统地理概念中,属于永宁的十个村落}  

\lhead{\firstmark}
\rhead{\botmark}

\subsection{\hspace{-0.5cm} {\Large \textcolor{darkblue}{\textbf{\ipa{æ˩mo˧}}}}\hspace{0.5cm}[\kern2pt{\textcolor{darkblue}{\textbf{\ipa{æ˩mo˥}}}}\kern2pt]} \hypertarget{\{\string_Bmo\string_M1}{}
\markboth{\textcolor{darkblue}{\textbf{\ipa{æ˩mo˧}}}}{}
\textcolor{teal}{\mytextsc{nom}} \hspace{4pt} Ton~: LM.
\textcolor{PineGreen}{\selectlanguage{french}Timon (âge, haie) de l'araire.} \zh{犁杆。}  ¶ \textcolor{darkblue}{\textbf{\ipa{æ˩gv̩˩-mo˥}}} \textcolor{PineGreen}{\selectlanguage{french}même sens} \zh{同上}  
 \zh{量词}: \textcolor{darkblue}{\textbf{\ipa{nɑ˧}}}  \mytextsc{clf}~: \textcolor{darkblue}{\textbf{\ipa{nɑ˧}}} \textit{Voir~:} \hyperlink{}{\textcolor{darkblue}{\textbf{\ipa{æ˩gv̩˩}}}} 
\lhead{\firstmark}
\rhead{\botmark}

\subsection{\hspace{-0.5cm} {\Large \textcolor{darkblue}{\textbf{\ipa{æ˩pʰæ˧˥}}}}\hspace{0.5cm}[\kern2pt{\textcolor{darkblue}{\textbf{\ipa{æ˩pʰæ˧˥}}}}\kern2pt]} \hypertarget{\{\string_Bp\string_h\{\string_M\string_T1}{}
\markboth{\textcolor{darkblue}{\textbf{\ipa{æ˩pʰæ˧˥}}}}{}
\textcolor{teal}{\mytextsc{nom}} \hspace{4pt} Ton~: LM+MH\#.
\textcolor{PineGreen}{\selectlanguage{french}Falaise. Le terme désigne spécifiquement le dos d'une falaise: l'espace relativement plat en bord de précipice. Pour faire référence à la paroi (face verticale de la falaise), on ajoute \textcolor{darkblue}{\textbf{\ipa{/lɑ˧bi˧/}}} 'escarpement'.} \zh{悬崖、崖山、崖壁。}  ¶ \textcolor{darkblue}{\textbf{\ipa{æ˩pʰæ˧-lɑ˧bi˥}}} \textcolor{PineGreen}{\selectlanguage{french}même sens} \zh{同上}  
 \zh{量词}: \textcolor{darkblue}{\textbf{\ipa{pʰæ˧˥}}}  \mytextsc{clf}~: \textcolor{darkblue}{\textbf{\ipa{pʰæ˧˥}}} 
\lhead{\firstmark}
\rhead{\botmark}

\subsection{\hspace{-0.5cm} {\Large \textcolor{darkblue}{\textbf{\ipa{æ˩qʰv̩˥}}}}\hspace{0.5cm}[\kern2pt{\textcolor{darkblue}{\textbf{\ipa{æ˩qʰv̩˥}}}}\kern2pt]} \hypertarget{\{\string_Bq\string_hv\string_=\string_T1}{}
\markboth{\textcolor{darkblue}{\textbf{\ipa{æ˩qʰv̩˥}}}}{}
\textcolor{teal}{\mytextsc{nom}} \hspace{4pt} Ton~: LH.
\textcolor{PineGreen}{\selectlanguage{french}Crevasse, petite grotte (où il est difficile de pénétrer).} \zh{小山洞(难进去,或者钻不进去的山洞)。}  \zh{量词}: \textcolor{darkblue}{\textbf{\ipa{ɭɯ˧}}}  \mytextsc{clf}~: \textcolor{darkblue}{\textbf{\ipa{ɭɯ˧}}} 
\lhead{\firstmark}
\rhead{\botmark}

\subsection{\hspace{-0.5cm} {\Large \textcolor{darkblue}{\textbf{\ipa{æ˩ʈv̩˥}}}}\hspace{0.5cm}[\kern2pt{\textcolor{darkblue}{\textbf{\ipa{æ˩ʈv̩˥}}}}\kern2pt]} \hypertarget{\{\string_Bt`v\string_=\string_T1}{}
\markboth{\textcolor{darkblue}{\textbf{\ipa{æ˩ʈv̩˥}}}}{}
\textcolor{teal}{\mytextsc{nom}} \hspace{4pt} Ton~: LH.
\textcolor{PineGreen}{\selectlanguage{french}Gros rocher, roc.} \zh{大岩石。}  \zh{量词}: \textcolor{darkblue}{\textbf{\ipa{ʈv̩˩}}}  \mytextsc{clf}~: \textcolor{darkblue}{\textbf{\ipa{ʈv̩˩}}} 
\lhead{\firstmark}
\rhead{\botmark}

\subsection{\hspace{-0.5cm} {\Large \textcolor{darkblue}{\textbf{\ipa{æ̃˥}}}}\hspace{0.5cm}[\kern2pt{\textcolor{darkblue}{\textbf{\ipa{æ̃˥}}}}\kern2pt]} \hypertarget{\{\string_~\string_T1}{}
\markboth{\textcolor{darkblue}{\textbf{\ipa{æ̃˥}}}}{}
\textcolor{teal}{\mytextsc{nom}} \hspace{4pt} Ton~: \#H.
\textcolor{PineGreen}{\selectlanguage{french}Cuivre; bronze.} \zh{铜,包括黄铜、红铜、青铜。}  ¶ \textcolor{darkblue}{\textbf{\ipa{æ̃˧tso˧-æ̃˧mo˩}}} \textcolor{PineGreen}{\selectlanguage{french}instruments en cuivre, objets en cuivre} \zh{铜做的工具、物品}  
 ¶ \textcolor{darkblue}{\textbf{\ipa{æ̃˧ lɑ˩-zo˩-ɳɯ˩, | ʂe˧ mɤ˧-lɑ˧˥!}}} \textcolor{PineGreen}{\selectlanguage{french}“Celui qui travaille le cuivre, il ne doit pas travailler le fer/on ne doit pas lui confier de tâches de forgeron (=travail du fer)!” Ces deux spécialités demandent des qualités différentes: de la force physique pour le travail du fer, et du soin pour le travail du cuivre. Le dicton s'emploie pour souligner que chacun a son domaine de compétence.} \zh{“打铜的人,不打铁!”这两种工作需要不同的能力:打铁需要体力,打铜需要细致。这个谚语意指:每个人有他的专业,不能随便跨越到其它领域去。}  

\lhead{\firstmark}
\rhead{\botmark}

\subsection{\hspace{-0.5cm} {\Large \textcolor{darkblue}{\textbf{\ipa{æ̃˧qæ˩}}} \textsubscript{1}}\hspace{0.5cm}[\kern2pt{\textcolor{darkblue}{\textbf{\ipa{æ̃˧qæ˩}}}}\kern2pt]} \hypertarget{\{\string_~\string_Mq\{\string_B1}{}
\markboth{\textcolor{darkblue}{\textbf{\ipa{æ̃˧qæ˩}}} \textsubscript{1}}{}
\textcolor{teal}{\mytextsc{nom}} \hspace{4pt} Ton~: L\#.
\textcolor{PineGreen}{\selectlanguage{french}Perroquet.} \zh{鹦鹉。}  \zh{量词}: \textcolor{darkblue}{\textbf{\ipa{mi˩}}}  \mytextsc{clf}~: \textcolor{darkblue}{\textbf{\ipa{mi˩}}} \textit{Voir~:} \hyperlink{}{\textcolor{darkblue}{\textbf{\ipa{æ̃˧qæ˩}}} \textsubscript{2}} 
\lhead{\firstmark}
\rhead{\botmark}

\subsection{\hspace{-0.5cm} {\Large \textcolor{darkblue}{\textbf{\ipa{æ̃˧qæ˩}}} \textsubscript{2}}\hspace{0.5cm}[\kern2pt{\textcolor{darkblue}{\textbf{\ipa{æ̃˧qæ˩}}}}\kern2pt]} \hypertarget{\{\string_~\string_Mq\{\string_B2}{}
\markboth{\textcolor{darkblue}{\textbf{\ipa{æ̃˧qæ˩}}} \textsubscript{2}}{}
\textcolor{teal}{\mytextsc{adjectif}} \hspace{4pt} Ton~: L\#.
\textcolor{PineGreen}{\selectlanguage{french}De couleur bleue/verte; couleur un peu plus légère que le vert de la plaine; équivalent du chinois \zh{青}. Littéralement: '[couleur] perroquet'.} \zh{像鹦鹉的颜色:青色、蓝色、绿色。}  ¶ \textcolor{darkblue}{\textbf{\ipa{æ̃˧qæ˩-ni˩gv̩˩}}} \textcolor{PineGreen}{\selectlanguage{french}couleur perroquet; littéralement 'comme un perroquet'} \zh{像鹦鹉的颜色:青、蓝色、绿色}  
 ¶ \textcolor{darkblue}{\textbf{\ipa{[F5] æ̃˧qæ˩-bɑ˩lɑ˩}}} \textcolor{PineGreen}{\selectlanguage{french}vêtement bleu; littéralement vêtement 'couleur perroquet'} \zh{青、蓝色、绿色衣服}  
 ¶ \textcolor{darkblue}{\textbf{\ipa{[F5] æ̃˧qæ˩ni˩\textasciitilde{}æ̃˧qæ˩ni˩gv̩˩}}} \textcolor{PineGreen}{\selectlanguage{french}\mytextsc{red;} même sens: bleu-vert} \zh{\mytextsc{重叠。同上:青色}}  
\textit{Voir~:} \hyperlink{}{\textcolor{darkblue}{\textbf{\ipa{æ̃˧qæ˩}}} \textsubscript{1}} 
\lhead{\firstmark}
\rhead{\botmark}

\subsection{\hspace{-0.5cm} {\Large \textcolor{darkblue}{\textbf{\ipa{æ̃˧ʂwæ˥}}}}\hspace{0.5cm}[\kern2pt{\textcolor{darkblue}{\textbf{\ipa{æ̃˧ʂwæ˥}}}}\kern2pt]} \hypertarget{\{\string_~\string_Ms`w\{\string_T1}{}
\markboth{\textcolor{darkblue}{\textbf{\ipa{æ̃˧ʂwæ˥}}}}{}
\textcolor{teal}{\mytextsc{nom}} \hspace{4pt} Ton~: H\#.
\textcolor{PineGreen}{\selectlanguage{french}Coq.} \zh{公鸡。}  ¶ \textcolor{darkblue}{\textbf{\ipa{æ̃˧ʂwæ˥-æ̃˩mi˩}}} \textcolor{PineGreen}{\selectlanguage{french}coq et poule} \zh{公鸡与母鸡}  
 \zh{量词}: \textcolor{darkblue}{\textbf{\ipa{mi˩}}}  \mytextsc{clf}~: \textcolor{darkblue}{\textbf{\ipa{mi˩}}} 
\lhead{\firstmark}
\rhead{\botmark}

\subsection{\hspace{-0.5cm} {\Large \textcolor{darkblue}{\textbf{\ipa{æ̃˧tsɯ˥}}}}\hspace{0.5cm}[\kern2pt{\textcolor{darkblue}{\textbf{\ipa{æ̃˧tsɯ˥}}}}\kern2pt]} \hypertarget{\{\string_~\string_MtsM\string_T1}{}
\markboth{\textcolor{darkblue}{\textbf{\ipa{æ̃˧tsɯ˥}}}}{}
\textcolor{teal}{\mytextsc{nom}} \hspace{4pt} Ton~: H\#.
\textcolor{PineGreen}{\selectlanguage{french}Poussin.} \zh{雏鸡、稚鸡。}  \zh{量词}: \textcolor{darkblue}{\textbf{\ipa{ɭɯ˧}}}  \mytextsc{clf}~: \textcolor{darkblue}{\textbf{\ipa{ɭɯ˧}}} 
\lhead{\firstmark}
\rhead{\botmark}

\subsection{\hspace{-0.5cm} {\Large \textcolor{darkblue}{\textbf{\ipa{æ̃˧tsɯ˥-kʰɯ˩ʈʂɤ˩-mo˩}}}}\hspace{0.5cm}[\kern2pt{\textcolor{darkblue}{\textbf{\ipa{xxxx non-correspondance entre le nombre de morphèmes et le nombre de tons de morphèmes}}}}\kern2pt]} \hypertarget{\{\string_~\string_MtsM\string_T-k\string_hM\string_Bt`s`7\string_B-mo\string_B1}{}
\markboth{\textcolor{darkblue}{\textbf{\ipa{æ̃˧tsɯ˥-kʰɯ˩ʈʂɤ˩-mo˩}}}}{}
\textcolor{teal}{\mytextsc{nom}} \hspace{4pt} Ton~: H\#-.
\textcolor{PineGreen}{\selectlanguage{french}“champignon griffes-de-poulet”: champignon comestible.} \zh{扫把菌,扫帚菌(一种菌子)。} 
\lhead{\firstmark}
\rhead{\botmark}

\subsection{\hspace{-0.5cm} {\Large \textcolor{darkblue}{\textbf{\ipa{æ̃˧ʈwɤ˩-mv̩˩kʰv̩˩}}}}\hspace{0.5cm}[\kern2pt{\textcolor{darkblue}{\textbf{\ipa{æ̃˧ʈwɤ˩mv̩˩kʰv̩˩}}}}\kern2pt]} \hypertarget{\{\string_~\string_Mt`w7\string_B-mv\string_=\string_Bk\string_hv\string_=\string_B1}{}
\markboth{\textcolor{darkblue}{\textbf{\ipa{æ̃˧ʈwɤ˩-mv̩˩kʰv̩˩}}}}{}
\textcolor{teal}{\mytextsc{adverbe}} \hspace{4pt} Ton~: L\#-L.
\textcolor{PineGreen}{\selectlanguage{french}Du matin au soir, constamment.} \zh{一直不停地,从早到晚。直译:‘(从)早上(到)晚上’。} 
\lhead{\firstmark}
\rhead{\botmark}

\subsection{\hspace{-0.5cm} {\Large \textcolor{darkblue}{\textbf{\ipa{æ̃˧-v̩\#˥}}}}\hspace{0.5cm}[\kern2pt{\textcolor{darkblue}{\textbf{\ipa{xxxx non-correspondance entre le nombre de morphèmes et le nombre de tons de morphèmes}}}}\kern2pt]} \hypertarget{\{\string_~\string_M-v\string_=\#\string_T1}{}
\markboth{\textcolor{darkblue}{\textbf{\ipa{æ̃˧-v̩\#˥}}}}{}
\textcolor{teal}{\mytextsc{nom}} \hspace{4pt} Ton~: \#H.
\textcolor{PineGreen}{\selectlanguage{french}Casserole en cuivre.} \zh{铜锅。}  \zh{量词}: \textcolor{darkblue}{\textbf{\ipa{ɭɯ˧}}}  \mytextsc{clf}~: \textcolor{darkblue}{\textbf{\ipa{ɭɯ˧}}} 
\lhead{\firstmark}
\rhead{\botmark}

\subsection{\hspace{-0.5cm} {\Large \textcolor{darkblue}{\textbf{\ipa{æ̃˩\textsubscript{a}}}}}\hspace{0.5cm}[\kern2pt{\textcolor{darkblue}{\textbf{\ipa{æ̃˩˥}}}}\kern2pt]} \hypertarget{\{\string_~\string_Ba1}{}
\markboth{\textcolor{darkblue}{\textbf{\ipa{æ̃˩\textsubscript{a}}}}}{}
\textcolor{teal}{\mytextsc{classificateur}} \hspace{4pt} Ton~: L\textsubscript{a}.
\textcolor{PineGreen}{\selectlanguage{french}Classificateur des feux.} \zh{量词:火(一团)。}  ¶ \textcolor{darkblue}{\textbf{\ipa{mv̩˧ | ʈʂʰɯ˧-æ̃˥}}} \textcolor{PineGreen}{\selectlanguage{french}ce feu (ton: H\# / H\$)} \zh{这团火}  

\lhead{\firstmark}
\rhead{\botmark}

\subsection{\hspace{-0.5cm} {\Large \textcolor{darkblue}{\textbf{\ipa{æ̃˩\textsubscript{a}}}} \textsubscript{1}}\hspace{0.5cm}[\kern2pt{\textcolor{darkblue}{\textbf{\ipa{æ̃˩˥}}}}\kern2pt]} \hypertarget{\{\string_~\string_Ba1}{}
\markboth{\textcolor{darkblue}{\textbf{\ipa{æ̃˩\textsubscript{a}}}} \textsubscript{1}}{}
\textcolor{teal}{\mytextsc{verbe}} \hspace{4pt} Ton~: L\textsubscript{a}.
\textcolor{PineGreen}{\selectlanguage{french}Réfléchir, renvoyer (un miroir renvoie la lumière; une cloison/toiture étanche renvoie la pluie =est étanche à la pluie).} \zh{反射、辉映。} 
\lhead{\firstmark}
\rhead{\botmark}

\subsection{\hspace{-0.5cm} {\Large \textcolor{darkblue}{\textbf{\ipa{æ̃˩\textsubscript{a}}}} \textsubscript{2}}\hspace{0.5cm}[\kern2pt{\textcolor{darkblue}{\textbf{\ipa{æ̃˩˥}}}}\kern2pt]} \hypertarget{\{\string_~\string_Ba2}{}
\markboth{\textcolor{darkblue}{\textbf{\ipa{æ̃˩\textsubscript{a}}}} \textsubscript{2}}{}
\textcolor{teal}{\mytextsc{verbe}} \hspace{4pt} Ton~: L\textsubscript{a}.
\textcolor{PineGreen}{\selectlanguage{french}S'enliser; se coincer, se bloquer.} \zh{堵塞、塞。}  ¶ \textcolor{darkblue}{\textbf{\ipa{ʝi˩mi˩˥ | ɖʐæ˩qʰæ˧-qo˩ æ̃˩!}}} \textcolor{PineGreen}{\selectlanguage{french}La vache est enlisée dans la boue.} \zh{牛陷在泥巴里。}  

\lhead{\firstmark}
\rhead{\botmark}

\subsection{\hspace{-0.5cm} {\Large \textcolor{darkblue}{\textbf{\ipa{æ̃˩bi˩}}}}\hspace{0.5cm}[\kern2pt{\textcolor{darkblue}{\textbf{\ipa{æ̃˩bi˩˥}}}}\kern2pt]} \hypertarget{\{\string_~\string_Bbi\string_B1}{}
\markboth{\textcolor{darkblue}{\textbf{\ipa{æ̃˩bi˩}}}}{}
\textcolor{teal}{\mytextsc{nom}} \hspace{4pt} Ton~: L.
\textcolor{PineGreen}{\selectlanguage{french}Abi: village sur le chemin de Qiansuo.} \zh{从阿拉瓦村到前所路上经过的一个村落。}  ¶ \textcolor{darkblue}{\textbf{\ipa{æ̃˩bi˩-ʁwɤ˥}}} \textcolor{PineGreen}{\selectlanguage{french}même sens: le village de \textcolor{darkblue}{\textbf{\ipa{/æ̃˩bi˩/}}}} \zh{同上}  
 ¶ \textcolor{darkblue}{\textbf{\ipa{æ̃˩bi˩-hĩ˥ ɲi˩!}}} \textcolor{PineGreen}{\selectlanguage{french}C'est quelqu'un du village de \textcolor{darkblue}{\textbf{\ipa{/æ̃˩bi˩/!}}}} \zh{是\textcolor{darkblue}{\textbf{\ipa{/æ̃˩bi˩/}}}村的人!}  

\lhead{\firstmark}
\rhead{\botmark}

\subsection{\hspace{-0.5cm} {\Large \textcolor{darkblue}{\textbf{\ipa{æ̃˩bv̩˥}}}}\hspace{0.5cm}[\kern2pt{\textcolor{darkblue}{\textbf{\ipa{æ̃˩bv̩˥}}}}\kern2pt]} \hypertarget{\{\string_~\string_Bbv\string_=\string_T1}{}
\markboth{\textcolor{darkblue}{\textbf{\ipa{æ̃˩bv̩˥}}}}{}
\textcolor{teal}{\mytextsc{nom}} \hspace{4pt} Ton~: LH.
\textcolor{PineGreen}{\selectlanguage{french}Poulailler.} \zh{鸡圈。}  \zh{量词}: \textcolor{darkblue}{\textbf{\ipa{ɭɯ˧}}}  \mytextsc{clf}~: \textcolor{darkblue}{\textbf{\ipa{ɭɯ˧}}} 
\lhead{\firstmark}
\rhead{\botmark}

\subsection{\hspace{-0.5cm} {\Large \textcolor{darkblue}{\textbf{\ipa{æ̃˩-kʰv̩˧˥}}}}\hspace{0.5cm}[\kern2pt{\textcolor{darkblue}{\textbf{\ipa{xxxx non-correspondance entre le nombre de morphèmes et le nombre de tons de morphèmes}}}}\kern2pt]} \hypertarget{\{\string_~\string_B-k\string_hv\string_=\string_M\string_T1}{}
\markboth{\textcolor{darkblue}{\textbf{\ipa{æ̃˩-kʰv̩˧˥}}}}{}
\textcolor{teal}{\mytextsc{nom}} \hspace{4pt} Ton~: LM+MH\#.
\textcolor{PineGreen}{\selectlanguage{french}Année du coq.} \zh{鸡年。} 
\lhead{\firstmark}
\rhead{\botmark}

\subsection{\hspace{-0.5cm} {\Large \textcolor{darkblue}{\textbf{\ipa{æ̃˩li˧pʰæ˥}}}}\hspace{0.5cm}[\kern2pt{\textcolor{darkblue}{\textbf{\ipa{æ̃˩li˧pʰæ˥}}}}\kern2pt]} \hypertarget{\{\string_~\string_Bli\string_Mp\string_h\{\string_T1}{}
\markboth{\textcolor{darkblue}{\textbf{\ipa{æ̃˩li˧pʰæ˥}}}}{}
\textcolor{teal}{\mytextsc{nom}} \hspace{4pt} Ton~: LM+H\#.
\textcolor{PineGreen}{\selectlanguage{french}Miroir.} \zh{镜子。}  \zh{量词}: \textcolor{darkblue}{\textbf{\ipa{pʰæ˧˥}}}  \mytextsc{clf}~: \textcolor{darkblue}{\textbf{\ipa{pʰæ˧˥}}} 
\lhead{\firstmark}
\rhead{\botmark}

\subsection{\hspace{-0.5cm} {\Large \textcolor{darkblue}{\textbf{\ipa{æ̃˩ɬi\#˥}}}}\hspace{0.5cm}[\kern2pt{\textcolor{darkblue}{\textbf{\ipa{æ̃˩ɬi˥}}}}\kern2pt]} \hypertarget{\{\string_~\string_BKi\#\string_T1}{}
\markboth{\textcolor{darkblue}{\textbf{\ipa{æ̃˩ɬi\#˥}}}}{}
\textcolor{teal}{\mytextsc{nom}} \hspace{4pt} Ton~: LM+\#H.
\textcolor{PineGreen}{\selectlanguage{french}Âme.} \zh{灵魂、魂魄。}  Emprunt~: tibétain  bla (older form: brla)
 \zh{量词}: \textcolor{darkblue}{\textbf{\ipa{v̩˧}}}  \mytextsc{clf}~: \textcolor{darkblue}{\textbf{\ipa{v̩˧}}} \textcolor{PineGreen}{\selectlanguage{french}individu}
\lhead{\firstmark}
\rhead{\botmark}

\subsection{\hspace{-0.5cm} {\Large \textcolor{darkblue}{\textbf{\ipa{æ̃˩mi˧}}}}\hspace{0.5cm}[\kern2pt{\textcolor{darkblue}{\textbf{\ipa{æ̃˩mi˥}}}}\kern2pt]} \hypertarget{\{\string_~\string_Bmi\string_M1}{}
\markboth{\textcolor{darkblue}{\textbf{\ipa{æ̃˩mi˧}}}}{}
\textcolor{teal}{\mytextsc{nom}} \hspace{4pt} Ton~: LM.
\textcolor{PineGreen}{\selectlanguage{french}Poule.} \zh{母鸡。}  ¶ \textcolor{darkblue}{\textbf{\ipa{æ̃˩mi˧-æ̃˧ʂwæ˥\#}}} \textcolor{PineGreen}{\selectlanguage{french}poule et coq} \zh{母鸡与公鸡}  
 ¶ \textcolor{darkblue}{\textbf{\ipa{æ̃˩mi˧-æ̃˧tsɯ˥\#}}} \textcolor{PineGreen}{\selectlanguage{french}poule et poussins} \zh{母鸡与稚鸡}  
 \zh{量词}: \textcolor{darkblue}{\textbf{\ipa{mi˩}}}  \mytextsc{clf}~: \textcolor{darkblue}{\textbf{\ipa{mi˩}}} 
\lhead{\firstmark}
\rhead{\botmark}

\subsection{\hspace{-0.5cm} {\Large \textcolor{darkblue}{\textbf{\ipa{æ̃˩ʁv̩˩}}}}\hspace{0.5cm}[\kern2pt{\textcolor{darkblue}{\textbf{\ipa{æ̃˩ʁv̩˩˥}}}}\kern2pt]} \hypertarget{\{\string_~\string_BRv\string_=\string_B1}{}
\markboth{\textcolor{darkblue}{\textbf{\ipa{æ̃˩ʁv̩˩}}}}{}
\textcolor{teal}{\mytextsc{nom}} \hspace{4pt} Ton~: L.
\textcolor{PineGreen}{\selectlanguage{french}Œuf.} \zh{蛋。}  ¶ \textcolor{darkblue}{\textbf{\ipa{bæ˧mi˧-æ̃˩ʁv̩˩}}} \textcolor{PineGreen}{\selectlanguage{french}œuf de cane} \zh{鸭子蛋}  
 ¶ \textcolor{darkblue}{\textbf{\ipa{æ̃˩ʁv̩˩ dzɯ˩˥}}} \textcolor{PineGreen}{\selectlanguage{french}manger des œufs} \zh{吃蛋}  
 \zh{量词}: \textcolor{darkblue}{\textbf{\ipa{ɭɯ˧}}}  \mytextsc{clf}~: \textcolor{darkblue}{\textbf{\ipa{ɭɯ˧}}} 
\lhead{\firstmark}
\rhead{\botmark}

\subsection{\hspace{-0.5cm} {\Large \textcolor{darkblue}{\textbf{\ipa{æ̃˩ʂe˧li˥-mo˩}}}}\hspace{0.5cm}[\kern2pt{\textcolor{darkblue}{\textbf{\ipa{æ̃˩ʂe˧li˥mo˧}}}}\kern2pt]} \hypertarget{\{\string_~\string_Bs`e\string_Mli\string_T-mo\string_B1}{}
\markboth{\textcolor{darkblue}{\textbf{\ipa{æ̃˩ʂe˧li˥-mo˩}}}}{}
\textcolor{teal}{\mytextsc{nom}} \hspace{4pt} Ton~: LM+H\#-.
\textcolor{PineGreen}{\selectlanguage{french}“champignon viande-de-poulet”: un champignon comestible, \textit{Amanita spissa}.} \zh{麻母鸡菌:一种可以吃的菌子,块鳞灰毒鹅膏菌。} 
\lhead{\firstmark}
\rhead{\botmark}

\subsection{\hspace{-0.5cm} {\Large \textcolor{darkblue}{\textbf{\ipa{æ̃˩ʂe˩}}}}\hspace{0.5cm}[\kern2pt{\textcolor{darkblue}{\textbf{\ipa{æ̃˩ʂe˩˥}}}}\kern2pt]} \hypertarget{\{\string_~\string_Bs`e\string_B1}{}
\markboth{\textcolor{darkblue}{\textbf{\ipa{æ̃˩ʂe˩}}}}{}
\textcolor{teal}{\mytextsc{nom}} \hspace{4pt} Ton~: L.
\textcolor{PineGreen}{\selectlanguage{french}Muscle.} \zh{肌肉。}  ¶ \textcolor{darkblue}{\textbf{\ipa{æ̃˩ʂe˩ tsʰi˩˥}}} \textcolor{PineGreen}{\selectlanguage{french}avoir la fièvre} \zh{发烧}  
 \zh{量词}: \textcolor{darkblue}{\textbf{\ipa{kʰwɤ˥}}}  \mytextsc{clf}~: \textcolor{darkblue}{\textbf{\ipa{kʰwɤ˥}}} 
\lhead{\firstmark}
\rhead{\botmark}

\subsection{\hspace{-0.5cm} {\Large \textcolor{darkblue}{\textbf{\ipa{æ̃˩zɯ˩}}}}\hspace{0.5cm}[\kern2pt{\textcolor{darkblue}{\textbf{\ipa{æ̃˩zɯ˩˥}}}}\kern2pt]} \hypertarget{\{\string_~\string_BzM\string_B1}{}
\markboth{\textcolor{darkblue}{\textbf{\ipa{æ̃˩zɯ˩}}}}{}
\textcolor{teal}{\mytextsc{nom}} \hspace{4pt} Ton~: L.
\textcolor{PineGreen}{\selectlanguage{french}Agate. Des perles d'agate de diverses couleurs sont utilisées en orfèvrerie. Elles sont de la taille d'un oeuf de caille, les plus gros approchent la taille d'un oeuf de poule. Les perles d'agate étaient intégrées aux bijoux et vêtements, dans une tradition d'inspiration tibétaine.} \zh{玛瑙。}  ¶ \textcolor{darkblue}{\textbf{\ipa{sɯ˧ɻ̍˧-æ̃˩zɯ˩}}} \textcolor{PineGreen}{\selectlanguage{french}agate en forme de perle} \zh{珠子形状的玛瑙}  
 ¶ \textcolor{darkblue}{\textbf{\ipa{æ̃˩zɯ˩-ʂo˩\textasciitilde{}ʂo˥}}} \textcolor{PineGreen}{\selectlanguage{french}tout plein d'agate, plein de morceaux d'agate (d'un vêtement)} \zh{(衣服上)都镶嵌着玛瑙}  
 \zh{量词}: \textcolor{darkblue}{\textbf{\ipa{ɭɯ˧}}}  \mytextsc{clf}~: \textcolor{darkblue}{\textbf{\ipa{ɭɯ˧}}} 
\lhead{\firstmark}
\rhead{\botmark}

\subsection{\hspace{-0.5cm} {\Large \textcolor{darkblue}{\textbf{\ipa{æ̃˩˥}}}}\hspace{0.5cm}[\kern2pt{\textcolor{darkblue}{\textbf{\ipa{æ̃˩˥}}}}\kern2pt]} \hypertarget{\{\string_~\string_B\string_T1}{}
\markboth{\textcolor{darkblue}{\textbf{\ipa{æ̃˩˥}}}}{}
\textcolor{teal}{\mytextsc{nom}} \hspace{4pt} Ton~: LH.
\textcolor{PineGreen}{\selectlanguage{french}Âme (monosyllabe).} \zh{灵魂。}  \zh{量词}: \textcolor{darkblue}{\textbf{\ipa{v̩˧}}}  \mytextsc{clf}~: \textcolor{darkblue}{\textbf{\ipa{v̩˧}}} 
\lhead{\firstmark}
\rhead{\botmark}

\subsection{\hspace{-0.5cm} {\Large \textcolor{darkblue}{\textbf{\ipa{æ̃˩˧}}}}\hspace{0.5cm}[\kern2pt{\textcolor{darkblue}{\textbf{\ipa{æ̃˩˥}}}}\kern2pt]} \hypertarget{\{\string_~\string_B\string_M1}{}
\markboth{\textcolor{darkblue}{\textbf{\ipa{æ̃˩˧}}}}{}
\textcolor{teal}{\mytextsc{nom}} \hspace{4pt} Ton~: LM.
\textcolor{PineGreen}{\selectlanguage{french}Poulet, poule.} \zh{鸡。}  ¶ \textcolor{darkblue}{\textbf{\ipa{æ̃˩ dzɯ˥-ze˩}}} \textcolor{PineGreen}{\selectlanguage{french}...a mangé (un/du) poulet} \zh{吃了鸡}  
 ¶ \textcolor{darkblue}{\textbf{\ipa{æ̃˩ hwæ˧-ze˧}}} \textcolor{PineGreen}{\selectlanguage{french}...a acheté (un/du) poulet} \zh{买了鸡}  
 ¶ \textcolor{darkblue}{\textbf{\ipa{æ̃˩˥, | kʰv̩˧, | bo˩˥, | hwɤ˧˥, | ʝi˧, | lɑ˧, | tʰo˧li˧, | mv̩˧gv̩˧, | bv̩˧ʐv̩˧, | ʐwæ˧, | jo˧, | ʑi˩˥}}} \textcolor{PineGreen}{\selectlanguage{french}les douze signes astrologiques} \zh{十二个生肖}  
 ¶ \textcolor{darkblue}{\textbf{\ipa{æ̃˩-mɤ˥}}} \textcolor{PineGreen}{\selectlanguage{french}graisse de poulet} \zh{鸡油}  
 ¶ \textcolor{darkblue}{\textbf{\ipa{æ̃˩-mɤ˥ dzɯ˩}}} \textcolor{PineGreen}{\selectlanguage{french}manger de la graisse de poulet} \zh{吃鸡油}  
 \zh{量词}: \textcolor{darkblue}{\textbf{\ipa{mi˩}}}  \mytextsc{clf}~: \textcolor{darkblue}{\textbf{\ipa{mi˩}}} 
\lhead{\firstmark}
\rhead{\botmark}

\newpage
\section*{\centering- \textcolor{darkblue}{\textbf{\ipa{b}}} -}
\subsection{\hspace{-0.5cm} {\Large \textcolor{darkblue}{\textbf{\ipa{bɑ˧lɑ˧kʰɯ˧tsʰɤ˧}}}}\hspace{0.5cm}[\kern2pt{\textcolor{darkblue}{\textbf{\ipa{bɑ˩lɑ˩kʰɯ˩tsʰɤ˩˥}}}}\kern2pt]} \hypertarget{bA\string_MlA\string_Mk\string_hM\string_Mts\string_h7\string_M1}{}
\markboth{\textcolor{darkblue}{\textbf{\ipa{bɑ˧lɑ˧kʰɯ˧tsʰɤ˧}}}}{}
\textcolor{teal}{\mytextsc{nom}} \hspace{4pt} Ton~: M.
\textcolor{PineGreen}{\selectlanguage{french}Araignée.} \zh{蜘蛛。}  \zh{量词}: \textcolor{darkblue}{\textbf{\ipa{kʰɯ˩}}}  \mytextsc{clf}~: \textcolor{darkblue}{\textbf{\ipa{kʰɯ˩}}} 
\lhead{\firstmark}
\rhead{\botmark}

\subsection{\hspace{-0.5cm} {\Large \textcolor{darkblue}{\textbf{\ipa{bɑ˩lɑ˩}}}}\hspace{0.5cm}[\kern2pt{\textcolor{darkblue}{\textbf{\ipa{bɑ˩lɑ˩˥}}}}\kern2pt]} \hypertarget{bA\string_BlA\string_B1}{}
\markboth{\textcolor{darkblue}{\textbf{\ipa{bɑ˩lɑ˩}}}}{}
\textcolor{teal}{\mytextsc{nom}} \hspace{4pt} Ton~: L.
\ding{202} \textcolor{PineGreen}{\selectlanguage{french}Chemise, veste; vêtement.} \zh{上衣,衣服。}  ¶ \textcolor{darkblue}{\textbf{\ipa{ɣɯ˩-bɑ˩lɑ˥ (+ɲi˩)}}} \textcolor{PineGreen}{\selectlanguage{french}veste de cuir} \zh{皮衣}  
 \zh{量词}: \textcolor{darkblue}{\textbf{\ipa{ɭɯ˧}}} \ding{203} \textcolor{PineGreen}{\selectlanguage{french}Placenta.} \zh{胎盘、衣胞。}  \zh{量词}: \textcolor{darkblue}{\textbf{\ipa{ɭɯ˧}}}  \mytextsc{clf}~: \textcolor{darkblue}{\textbf{\ipa{ɭɯ˧}}} \textcolor{darkblue}{\textbf{\ipa{ɭɯ˧}}} 
\lhead{\firstmark}
\rhead{\botmark}

\subsection{\hspace{-0.5cm} {\Large \textcolor{darkblue}{\textbf{\ipa{bɑ˩˥}}}}\hspace{0.5cm}[\kern2pt{\textcolor{darkblue}{\textbf{\ipa{bɑ˩˥}}}}\kern2pt]} \hypertarget{bA\string_B\string_T1}{}
\markboth{\textcolor{darkblue}{\textbf{\ipa{bɑ˩˥}}}}{}
\textcolor{teal}{\mytextsc{particule}} \textcolor{teal}{\mytextsc{de}} \textcolor{teal}{\mytextsc{discours}} \hspace{4pt} Ton~: L?.
\textcolor{PineGreen}{\selectlanguage{french}Particule finale affirmative: “...n'est-ce pas”.} \zh{句尾助词,表示肯定:“……是吧。”。} 
\lhead{\firstmark}
\rhead{\botmark}

\subsection{\hspace{-0.5cm} {\Large \textcolor{darkblue}{\textbf{\ipa{bæ˧}}} \textsubscript{1}}\hspace{0.5cm}[\kern2pt{\textcolor{darkblue}{\textbf{\ipa{bæ˥}}}}\kern2pt]} \hypertarget{b\{\string_M1}{}
\markboth{\textcolor{darkblue}{\textbf{\ipa{bæ˧}}} \textsubscript{1}}{}
\textcolor{teal}{\mytextsc{adjectif}} \hspace{4pt} Ton~: M.
\textcolor{PineGreen}{\selectlanguage{french}Stupide, sot, idiot.} \zh{傻、笨、蠢。}  ¶ \textcolor{darkblue}{\textbf{\ipa{bæ˧-hĩ˧}}} \textcolor{PineGreen}{\selectlanguage{french}\mytextsc{rel}} \zh{傻的}  

\lhead{\firstmark}
\rhead{\botmark}

\subsection{\hspace{-0.5cm} {\Large \textcolor{darkblue}{\textbf{\ipa{bæ˧}}} \textsubscript{2}}\hspace{0.5cm}[\kern2pt{\textcolor{darkblue}{\textbf{\ipa{bæ˥}}}}\kern2pt]} \hypertarget{b\{\string_M2}{}
\markboth{\textcolor{darkblue}{\textbf{\ipa{bæ˧}}} \textsubscript{2}}{}
\textcolor{teal}{\mytextsc{verbe}} \hspace{4pt} Ton~: M intrans.
\textcolor{PineGreen}{\selectlanguage{french}Laisser tomber, abandonner, ne pas s'entêter.} \zh{放弃。}  ¶ \textcolor{darkblue}{\textbf{\ipa{no˧ | bæ˧-ze˩!}}} \textcolor{PineGreen}{\selectlanguage{french}Tu dois être bien déçu; allez, laisse tomber! (Ce qu'on dit à quelqu'un qui a échoué après de multiples tentatives.)} \zh{你算了吧!(感叹)}  
 ¶ \textcolor{darkblue}{\textbf{\ipa{bæ˧-ze˩ mæ˩!}}} \textcolor{PineGreen}{\selectlanguage{french}Laisse tomber, allez! (nuance d'évidence)} \zh{算了嘛!(感叹)}  

\lhead{\firstmark}
\rhead{\botmark}

\subsection{\hspace{-0.5cm} {\Large \textcolor{darkblue}{\textbf{\ipa{bæ˧\textsubscript{a}}}}}\hspace{0.5cm}[\kern2pt{\textcolor{darkblue}{\textbf{\ipa{bæ˥}}}}\kern2pt]} \hypertarget{b\{\string_Ma1}{}
\markboth{\textcolor{darkblue}{\textbf{\ipa{bæ˧\textsubscript{a}}}}}{}
\textcolor{teal}{\mytextsc{classificateur}} \hspace{4pt} Ton~: M\textsubscript{a}.
\textcolor{PineGreen}{\selectlanguage{french}Classificateur des espèces/sortes de choses. Proche de \textcolor{darkblue}{\textbf{\ipa{/ʁo˩b/}}} “sorte, variété”. S'emploie dans la construction “c'est la même chose”.} \zh{量词:东西(一样)。}  ¶ \textcolor{darkblue}{\textbf{\ipa{ɖɯ˧-bæ˧-lɑ˧ ɲi˥!}}} \textcolor{PineGreen}{\selectlanguage{french}c'est pareil!/c'est la même chose!} \zh{是一样的!}  
 ¶ \textcolor{darkblue}{\textbf{\ipa{ʝi˧kʰv̩˥-dʑo˩, | ɲi˧-bæ˧ | ʐwɤ˩-tʰɑ˩˥! | ʝi˧kʰv̩˥-dʑo˩, | ɖɯ˧-bæ˧-lɑ˧ ʐwɤ˧-tʰɑ˥!}}} \textcolor{PineGreen}{\selectlanguage{french}Il y en a certaines (=des expressions/des combinaisons de mots), on peut les prononcer de deux façons/elles ont deux schémas tonals différents! Il y en a certaines, il n'y a qu'une façon de les dire/il n'y a qu'une sorte (de réalisation tonale possible)! (commentaire au sujet d'expressions qui ont deux variantes tonales)} \zh{有些(词组)有两种说法,有些只有一种说法!(情景:讨论的是一些有两种不同变调发音的词组,发音合作人确定:确实有些有两种不同的变调,而有些只有一种声调模型。)}  
 ¶ \textcolor{darkblue}{\textbf{\ipa{ɲi˧-bæ˧-ɳɯ˧ | ɖɯ˧-bæ˧ ʝi˧}}} \textcolor{PineGreen}{\selectlanguage{french}confondre deux choses (ex.: confondre deux sons, et les noter de la même façon, alors qu'ils s'opposent entre eux)} \zh{两者混淆,例如把两个音(两个不同的音位)写成一样,混淆两者}  

\lhead{\firstmark}
\rhead{\botmark}

\subsection{\hspace{-0.5cm} {\Large \textcolor{darkblue}{\textbf{\ipa{bæ˧bv̩˥}}}}\hspace{0.5cm}[\kern2pt{\textcolor{darkblue}{\textbf{\ipa{bæ˧bv̩˥}}}}\kern2pt]} \hypertarget{b\{\string_Mbv\string_=\string_T1}{}
\markboth{\textcolor{darkblue}{\textbf{\ipa{bæ˧bv̩˥}}}}{}
\textcolor{teal}{\mytextsc{nom}} \hspace{4pt} Ton~: H\#.
\textcolor{PineGreen}{\selectlanguage{french}Goret, porcelet, cochonnet, petit cochon.} \zh{猪崽。}  ¶ \textcolor{darkblue}{\textbf{\ipa{bæ˧bv̩˥-zo˩}}} \textcolor{PineGreen}{\selectlanguage{french}même sens: goret} \zh{猪崽}  
 \zh{量词}: \textcolor{darkblue}{\textbf{\ipa{ɭɯ˧}}}  \mytextsc{clf}~: \textcolor{darkblue}{\textbf{\ipa{ɭɯ˧}}} 
\lhead{\firstmark}
\rhead{\botmark}

\subsection{\hspace{-0.5cm} {\Large \textcolor{darkblue}{\textbf{\ipa{bæ˧ɖæ˧}}}}\hspace{0.5cm}[\kern2pt{\textcolor{darkblue}{\textbf{\ipa{bæ˧ɖæ˧}}}}\kern2pt]} \hypertarget{b\{\string_Md`\{\string_M1}{}
\markboth{\textcolor{darkblue}{\textbf{\ipa{bæ˧ɖæ˧}}}}{}
\textcolor{teal}{\mytextsc{nom}} \hspace{4pt} Ton~: M.
\textcolor{PineGreen}{\selectlanguage{french}Cordelette.} \zh{细绳。}  \zh{量词}: \textcolor{darkblue}{\textbf{\ipa{kʰɯ˩}}}  \mytextsc{clf}~: \textcolor{darkblue}{\textbf{\ipa{kʰɯ˩}}} 
\lhead{\firstmark}
\rhead{\botmark}

\subsection{\hspace{-0.5cm} {\Large \textcolor{darkblue}{\textbf{\ipa{bæ˧mi˧}}} \textsubscript{1}}\hspace{0.5cm}[\kern2pt{\textcolor{darkblue}{\textbf{\ipa{xxxx non-correspondance entre le nombre de morphèmes et le nombre de tons de morphèmes}}}}\kern2pt]} \hypertarget{b\{\string_Mmi\string_M1}{}
\markboth{\textcolor{darkblue}{\textbf{\ipa{bæ˧mi˧}}} \textsubscript{1}}{}
\textcolor{teal}{\mytextsc{nom}} \hspace{4pt} Ton~: M.
\ding{202} \textcolor{PineGreen}{\selectlanguage{french}Canard (sans préciser le sexe: canard ou cane).} \zh{鸭子。}  \zh{量词}: \textcolor{darkblue}{\textbf{\ipa{mi˩}}} \ding{203} \textcolor{PineGreen}{\selectlanguage{french}Cane.} \zh{母鸭子。}  ¶ \textcolor{darkblue}{\textbf{\ipa{bæ˧mi˧-bæ˧pʰv̩\#˥}}} \textcolor{PineGreen}{\selectlanguage{french}cane et canard} \zh{母鸭子与公鸭子}  
 ¶ \textcolor{darkblue}{\textbf{\ipa{bæ˧mi˧-bæ˧zo\#˥}}} \textcolor{PineGreen}{\selectlanguage{french}cane et caneton} \zh{母鸭与小鸭子}  
 \mytextsc{clf}~: \textcolor{darkblue}{\textbf{\ipa{mi˩}}} 
\lhead{\firstmark}
\rhead{\botmark}

\subsection{\hspace{-0.5cm} {\Large \textcolor{darkblue}{\textbf{\ipa{bæ˧mi˧}}} \textsubscript{2}}\hspace{0.5cm}[\kern2pt{\textcolor{darkblue}{\textbf{\ipa{bæ˧mi˧}}}}\kern2pt]} \hypertarget{b\{\string_Mmi\string_M2}{}
\markboth{\textcolor{darkblue}{\textbf{\ipa{bæ˧mi˧}}} \textsubscript{2}}{}
\textcolor{teal}{\mytextsc{nom}} \hspace{4pt} Ton~: M.
\textcolor{PineGreen}{\selectlanguage{french}Grosse corde.} \zh{粗绳索。}  \zh{量词}: \textcolor{darkblue}{\textbf{\ipa{kʰɯ˩}}}  \mytextsc{clf}~: \textcolor{darkblue}{\textbf{\ipa{kʰɯ˩}}} 
\lhead{\firstmark}
\rhead{\botmark}

\subsection{\hspace{-0.5cm} {\Large \textcolor{darkblue}{\textbf{\ipa{bæ˧mi˧-pʰv̩\#˥}}}}\hspace{0.5cm}[\kern2pt{\textcolor{darkblue}{\textbf{\ipa{xxxx non-correspondance entre le nombre de morphèmes et le nombre de tons de morphèmes}}}}\kern2pt]} \hypertarget{b\{\string_Mmi\string_M-p\string_hv\string_=\#\string_T1}{}
\markboth{\textcolor{darkblue}{\textbf{\ipa{bæ˧mi˧-pʰv̩\#˥}}}}{}
\textcolor{teal}{\mytextsc{nom}} \hspace{4pt} Ton~: \#H.
\textcolor{PineGreen}{\selectlanguage{french}Canard (mâle).} \zh{公鸭子。}  ¶ \textcolor{darkblue}{\textbf{\ipa{bæ˧mi˧-pʰv̩˧ tʰv̩˧-mi˧˥}}} \textcolor{PineGreen}{\selectlanguage{french}\mytextsc{n}+\mytextsc{dem}+\mytextsc{clf}} \zh{这只公鸭子}  
 \zh{量词}: \textcolor{darkblue}{\textbf{\ipa{mi˩}}}  \mytextsc{clf}~: \textcolor{darkblue}{\textbf{\ipa{mi˩}}} \textit{Voir~:} \hyperlink{}{\textcolor{darkblue}{\textbf{\ipa{bæ˧pʰv̩\#˥}}}} 
\lhead{\firstmark}
\rhead{\botmark}

\subsection{\hspace{-0.5cm} {\Large \textcolor{darkblue}{\textbf{\ipa{bæ˧pʰv̩\#˥}}}}\hspace{0.5cm}[\kern2pt{\textcolor{darkblue}{\textbf{\ipa{bæ˩pʰv̩˥}}}}\kern2pt]} \hypertarget{b\{\string_Mp\string_hv\string_=\#\string_T1}{}
\markboth{\textcolor{darkblue}{\textbf{\ipa{bæ˧pʰv̩\#˥}}}}{}
\textcolor{teal}{\mytextsc{nom}} \hspace{4pt} Ton~: \#H.
\textcolor{PineGreen}{\selectlanguage{french}Canard (mâle).} \zh{公鸭子。}  ¶ \textcolor{darkblue}{\textbf{\ipa{bæ˧pʰv̩˧ tʰv̩˧-mi˧˥ / bæ˧pʰv̩˧ tʰv̩˧-mi˥\#}}} \textcolor{PineGreen}{\selectlanguage{french}\mytextsc{n}+\mytextsc{dem}+\mytextsc{clf}} \zh{这个公鸭子}  
 ¶ \textcolor{darkblue}{\textbf{\ipa{bæ˧pʰv̩˧-bæ˧mi\#˥}}} \textcolor{PineGreen}{\selectlanguage{french}canard et cane} \zh{公鸭子与母鸭子}  
 \zh{量词}: \textcolor{darkblue}{\textbf{\ipa{mi˩}}}  \mytextsc{clf}~: \textcolor{darkblue}{\textbf{\ipa{mi˩}}} \textit{Voir~:} \hyperlink{}{\textcolor{darkblue}{\textbf{\ipa{bæ˧mi˧-pʰv̩\#˥}}}} 
\lhead{\firstmark}
\rhead{\botmark}

\subsection{\hspace{-0.5cm} {\Large \textcolor{darkblue}{\textbf{\ipa{bæ˧ʁwɤ˧}}}}\hspace{0.5cm}[\kern2pt{\textcolor{darkblue}{\textbf{\ipa{bæ˩ʁwɤ˩˥}}}}\kern2pt]} \hypertarget{b\{\string_MRw7\string_M1}{}
\markboth{\textcolor{darkblue}{\textbf{\ipa{bæ˧ʁwɤ˧}}}}{}
\textcolor{teal}{\mytextsc{nom}} \hspace{4pt} Ton~: M.
\textcolor{PineGreen}{\selectlanguage{french}Un village proche des Source Chaudes.} \zh{温泉乡的一个村落。}  ¶ \textcolor{darkblue}{\textbf{\ipa{bæ˧ʁwɤ˧-ʁwɤ˧}}} \textcolor{PineGreen}{\selectlanguage{french}même sens: le village de \textcolor{darkblue}{\textbf{\ipa{/bæ˧ʁwɤ˧/}}}} \zh{同上:\textcolor{darkblue}{\textbf{\ipa{/bæ˧ʁwɤ˧/}}}村}  
 ¶ \textcolor{darkblue}{\textbf{\ipa{ə˧go˧-ʁwɤ˧, | ʁwɤ˧lɑ˩-bi˩, | bæ˧ʁwɤ˧, | tʰo˧tsʰe\#˥, | pi˧tsʰe˩-di˩, | pɤ˧dʑɤ˩-di˩, | ʁwɤ˧tv̩˧}}} \textcolor{PineGreen}{\selectlanguage{french}Villages au sortir de la plaine de Yongning; les deux premiers comportent une population na; le troisième est un village na; les suivants sont essentiellement des villages pumi/prinmi.} \zh{永宁背向泸沽湖方向经过的村落。前两个村落拥有相当大的摩梭人口比例,第三个村落是摩梭村,最后一个是普米村。}  
 ¶ \textcolor{darkblue}{\textbf{\ipa{bæ˧ʁwɤ˧: | nɑ˩˥!}}} \textcolor{PineGreen}{\selectlanguage{french}\textcolor{darkblue}{\textbf{\ipa{/bæ˧ʁwɤ˧/}}}, c'est un village na!} \zh{\textcolor{darkblue}{\textbf{\ipa{/bæ˧ʁwɤ˧/}}}是一个摩梭人村落!}  

\lhead{\firstmark}
\rhead{\botmark}

\subsection{\hspace{-0.5cm} {\Large \textcolor{darkblue}{\textbf{\ipa{bæ˧zo\#˥}}}}\hspace{0.5cm}[\kern2pt{\textcolor{darkblue}{\textbf{\ipa{bæ˩zo˩˥}}}}\kern2pt]} \hypertarget{b\{\string_Mzo\#\string_T1}{}
\markboth{\textcolor{darkblue}{\textbf{\ipa{bæ˧zo\#˥}}}}{}
\textcolor{teal}{\mytextsc{nom}} \hspace{4pt} Ton~: \#H.
\textcolor{PineGreen}{\selectlanguage{french}Caneton, petit canard.} \zh{小鸭子。}  ¶ \textcolor{darkblue}{\textbf{\ipa{bæ˧zo˧ tʰv̩˧-ɭɯ\#˥}}} \textcolor{PineGreen}{\selectlanguage{french}\mytextsc{n}+\mytextsc{dem}+\mytextsc{clf}} \zh{这只小鸭子}  
 ¶ \textcolor{darkblue}{\textbf{\ipa{bæ˧zo˧-bæ˧mi\#˥}}} \textcolor{PineGreen}{\selectlanguage{french}caneton et cane} \zh{小鸭子与母鸭}  
 \zh{量词}: \textcolor{darkblue}{\textbf{\ipa{ɭɯ˧}}}  \mytextsc{clf}~: \textcolor{darkblue}{\textbf{\ipa{ɭɯ˧}}} 
\lhead{\firstmark}
\rhead{\botmark}

\subsection{\hspace{-0.5cm} {\Large \textcolor{darkblue}{\textbf{\ipa{bæ˩}}} \textsubscript{1}}\hspace{0.5cm}[\kern2pt{\textcolor{darkblue}{\textbf{\ipa{bæ˥}}}}\kern2pt]} \hypertarget{b\{\string_B1}{}
\markboth{\textcolor{darkblue}{\textbf{\ipa{bæ˩}}} \textsubscript{1}}{}
\textcolor{teal}{\mytextsc{nom}} \hspace{4pt} Ton~: L.
\textcolor{PineGreen}{\selectlanguage{french}Corde.} \zh{绳子。}  ¶ \textcolor{darkblue}{\textbf{\ipa{bæ˩ ʈʂʰɯ˩-kʰɯ˥}}} \textcolor{PineGreen}{\selectlanguage{french}\mytextsc{n}+\mytextsc{dem}+\mytextsc{clf}} \zh{这条绳子}  
 \zh{量词}: \textcolor{darkblue}{\textbf{\ipa{ʈʰɤ˥}}} \textcolor{darkblue}{\textbf{\ipa{ɖæ˩}}} \textcolor{darkblue}{\textbf{\ipa{kʰɯ˩}}}  \mytextsc{clf}~: \textcolor{darkblue}{\textbf{\ipa{ʈʰɤ˥}}} \textcolor{darkblue}{\textbf{\ipa{ɖæ˩}}} \textcolor{darkblue}{\textbf{\ipa{kʰɯ˩}}} 
\lhead{\firstmark}
\rhead{\botmark}

\subsection{\hspace{-0.5cm} {\Large \textcolor{darkblue}{\textbf{\ipa{bæ˩}}} \textsubscript{2}}\hspace{0.5cm}[\kern2pt{\textcolor{darkblue}{\textbf{\ipa{bæ˩˥}}}}\kern2pt]} \hypertarget{b\{\string_B2}{}
\markboth{\textcolor{darkblue}{\textbf{\ipa{bæ˩}}} \textsubscript{2}}{}
\textcolor{teal}{\mytextsc{verbe}} \hspace{4pt} Ton~: L.
\textcolor{PineGreen}{\selectlanguage{french}Pus.} \zh{脓。}  ¶ \textcolor{darkblue}{\textbf{\ipa{bæ˩ bæ˧-ze˩}}} \textcolor{PineGreen}{\selectlanguage{french}la blessure donne du pus, il y a du pus} \zh{伤口在化脓}  
 ¶ \textcolor{darkblue}{\textbf{\ipa{bæ˩˥ | le˧-bæ˩-ze˩}}} \textcolor{PineGreen}{\selectlanguage{french}la blessure donne du pus, il y a du pus} \zh{伤口在化脓}  
\textit{Voir~:} \hyperlink{}{\textcolor{darkblue}{\textbf{\ipa{bæ˩˥}}}} 
\lhead{\firstmark}
\rhead{\botmark}

\subsection{\hspace{-0.5cm} {\Large \textcolor{darkblue}{\textbf{\ipa{bæ˩\textsubscript{a}}}} \textsubscript{1}}\hspace{0.5cm}[\kern2pt{\textcolor{darkblue}{\textbf{\ipa{bæ˩˥}}}}\kern2pt]} \hypertarget{b\{\string_Ba1}{}
\markboth{\textcolor{darkblue}{\textbf{\ipa{bæ˩\textsubscript{a}}}} \textsubscript{1}}{}
\textcolor{teal}{\mytextsc{verbe}} \hspace{4pt} Ton~: L\textsubscript{a}.
\textcolor{PineGreen}{\selectlanguage{french}Balayer.} \zh{扫。}  ¶ \textcolor{darkblue}{\textbf{\ipa{ɖæ˩ bæ˧}}} \textcolor{PineGreen}{\selectlanguage{french}balayer les saletés, balayer le sol} \zh{扫地}  
 ¶ \textcolor{darkblue}{\textbf{\ipa{le˧-bæ˧\textasciitilde{}bæ˥}}} \textcolor{PineGreen}{\selectlanguage{french}\mytextsc{accomp} \mytextsc{red}} \zh{扫一扫}  
 ¶ \textcolor{darkblue}{\textbf{\ipa{ɖʐɤ˩ bæ˩˥}}} \textcolor{PineGreen}{\selectlanguage{french}balayer l'escalier} \zh{扫楼梯}  
 ¶ \textcolor{darkblue}{\textbf{\ipa{njɤ˧ | ɖʐɤ˩ bæ˩-zo˩-ho˥.}}} \textcolor{PineGreen}{\selectlanguage{french}Il va falloir que je balaie l'escalier!} \zh{我要扫楼梯了!}  
 ¶ \textcolor{darkblue}{\textbf{\ipa{gi˩ bæ˩˥}}} \textcolor{PineGreen}{\selectlanguage{french}balayer le grenier à céréales} \zh{扫仓廪}  
 ¶ \textcolor{darkblue}{\textbf{\ipa{njɤ˧ | gi˩ bæ˩-zo˩-ho˥.}}} \textcolor{PineGreen}{\selectlanguage{french}Il va falloir que je balaie le grenier à céréales!} \zh{我要扫仓廪了!}  

\lhead{\firstmark}
\rhead{\botmark}

\subsection{\hspace{-0.5cm} {\Large \textcolor{darkblue}{\textbf{\ipa{bæ˩\textsubscript{a}}}} \textsubscript{2}}\hspace{0.5cm}[\kern2pt{\textcolor{darkblue}{\textbf{\ipa{bæ˩˥}}}}\kern2pt]} \hypertarget{b\{\string_Ba2}{}
\markboth{\textcolor{darkblue}{\textbf{\ipa{bæ˩\textsubscript{a}}}} \textsubscript{2}}{}
\textcolor{teal}{\mytextsc{verbe}} \hspace{4pt} Ton~: L\textsubscript{a}.
\textcolor{PineGreen}{\selectlanguage{french}S'ouvrir (fleur), fleurir.} \zh{开花。}  ¶ \textcolor{darkblue}{\textbf{\ipa{bæ˩bæ˩ bæ˥-ze˩}}} \textcolor{PineGreen}{\selectlanguage{french}La fleur a fleuri.} \zh{花开了。}  

\lhead{\firstmark}
\rhead{\botmark}

\subsection{\hspace{-0.5cm} {\Large \textcolor{darkblue}{\textbf{\ipa{bæ˩\textsubscript{a}}}} \textsubscript{3}}\hspace{0.5cm}[\kern2pt{\textcolor{darkblue}{\textbf{\ipa{bæ˩˥}}}}\kern2pt]} \hypertarget{b\{\string_Ba3}{}
\markboth{\textcolor{darkblue}{\textbf{\ipa{bæ˩\textsubscript{a}}}} \textsubscript{3}}{}
\textcolor{teal}{\mytextsc{classificateur}} \hspace{4pt} Ton~: L\textsubscript{a}.
\textcolor{PineGreen}{\selectlanguage{french}Auto-classificateur des fleurs.} \zh{量词:花(一朵)。}  ¶ \textcolor{darkblue}{\textbf{\ipa{tʰv̩˧-bæ˥}}} \textcolor{PineGreen}{\selectlanguage{french}\mytextsc{dem} \string_ (ton: H\# / H\$)} \zh{\mytextsc{指示代词} \string_ :这朵(花)}  

\lhead{\firstmark}
\rhead{\botmark}

\subsection{\hspace{-0.5cm} {\Large \textcolor{darkblue}{\textbf{\ipa{bæ˩bæ˩}}} \textsubscript{1}}\hspace{0.5cm}[\kern2pt{\textcolor{darkblue}{\textbf{\ipa{bæ˩bæ˩˥}}}}\kern2pt]} \hypertarget{b\{\string_Bb\{\string_B1}{}
\markboth{\textcolor{darkblue}{\textbf{\ipa{bæ˩bæ˩}}} \textsubscript{1}}{}
\textcolor{teal}{\mytextsc{nom}} \hspace{4pt} Ton~: L.
\textcolor{PineGreen}{\selectlanguage{french}Fleur.} \zh{花。}  \zh{量词}: \textcolor{darkblue}{\textbf{\ipa{bæ˩}}}  \mytextsc{clf}~: \textcolor{darkblue}{\textbf{\ipa{bæ˩}}} \textit{Voir~:} \hyperlink{}{\textcolor{darkblue}{\textbf{\ipa{bæ˩bæ˩}}} \textsubscript{2}} 
\lhead{\firstmark}
\rhead{\botmark}

\subsection{\hspace{-0.5cm} {\Large \textcolor{darkblue}{\textbf{\ipa{bæ˩bæ˩}}} \textsubscript{2}}\hspace{0.5cm}[\kern2pt{\textcolor{darkblue}{\textbf{\ipa{bæ˩bæ˩˥}}}}\kern2pt]} \hypertarget{b\{\string_Bb\{\string_B2}{}
\markboth{\textcolor{darkblue}{\textbf{\ipa{bæ˩bæ˩}}} \textsubscript{2}}{}
\textcolor{teal}{\mytextsc{adjectif}} \hspace{4pt} Ton~: L.
\textcolor{PineGreen}{\selectlanguage{french}Bariolé, tacheté, moucheté (ex.: un oeuf moucheté, un oiseau au pelage moucheté, une pierre ayant plusieurs couleurs).} \zh{花的(蛋、石头、鸟)。}  ¶ \textcolor{darkblue}{\textbf{\ipa{bæ˩bæ˩ tʰi˩-di˥}}} \textcolor{PineGreen}{\selectlanguage{french}même sens: bariolé, tacheté (par ex.: oeuf, oiseau, pierre)} \zh{花的,有花纹}  
\textit{Voir~:} \hyperlink{}{\textcolor{darkblue}{\textbf{\ipa{bæ˩bæ˩}}} \textsubscript{1}} 
\lhead{\firstmark}
\rhead{\botmark}

\subsection{\hspace{-0.5cm} {\Large \textcolor{darkblue}{\textbf{\ipa{bæ˩dʑɯ˥}}}}\hspace{0.5cm}[\kern2pt{\textcolor{darkblue}{\textbf{\ipa{bæ˩dʑɯ˥}}}}\kern2pt]} \hypertarget{b\{\string_Bdz£M\string_T1}{}
\markboth{\textcolor{darkblue}{\textbf{\ipa{bæ˩dʑɯ˥}}}}{}
\textcolor{teal}{\mytextsc{nom}} \hspace{4pt} Ton~: LH.
\textcolor{PineGreen}{\selectlanguage{french}Récolte; plantes que l'on a semées.} \zh{庄稼。}  ¶ \textcolor{darkblue}{\textbf{\ipa{bæ˩dʑɯ˥ | mɤ˧-dʑɤ˩!}}} \textcolor{PineGreen}{\selectlanguage{french}La récolte n'est pas bonne!} \zh{收成不好!}  
 ¶ \textcolor{darkblue}{\textbf{\ipa{bæ˩dʑɯ˧ | tv̩˧-bæ˩ le˩-mv̩˩-kʰɯ˩!}}} \textcolor{PineGreen}{\selectlanguage{french}Puissent mille récoltes venir à maturité! (Bénédiction qu'on dit aux aînés lors de cérémonies: par exemple lors du rite de passage à l'âge adulte)} \zh{祝:一千棵庄稼成熟!(成年礼、过年等节庆时的祝福用语,晚辈对长辈的祝福)}  
 \zh{量词}: \textcolor{darkblue}{\textbf{\ipa{bæ˩}}}  \mytextsc{clf}~: \textcolor{darkblue}{\textbf{\ipa{bæ˩}}} 
\lhead{\firstmark}
\rhead{\botmark}

\subsection{\hspace{-0.5cm} {\Large \textcolor{darkblue}{\textbf{\ipa{bæ˩-lɑ˩\textasciitilde{}lɑ˥}}}}\hspace{0.5cm}[\kern2pt{\textcolor{darkblue}{\textbf{\ipa{xxxx non-correspondance entre le nombre de morphèmes et le nombre de tons de morphèmes}}}}\kern2pt]} \hypertarget{b\{\string_B-lA\string_B~lA\string_T1}{}
\markboth{\textcolor{darkblue}{\textbf{\ipa{bæ˩-lɑ˩\textasciitilde{}lɑ˥}}}}{}
\textcolor{teal}{\mytextsc{adjectif}} \hspace{4pt} Ton~: L.
\textcolor{PineGreen}{\selectlanguage{french}Flasque, sans consistance.} \zh{软,柔软、软塌塌。} 
\lhead{\firstmark}
\rhead{\botmark}

\subsection{\hspace{-0.5cm} {\Large \textcolor{darkblue}{\textbf{\ipa{bæ˩-ljɤ˧\textasciitilde{}ljɤ˧}}}}\hspace{0.5cm}[\kern2pt{\textcolor{darkblue}{\textbf{\ipa{xxxx non-correspondance entre le nombre de morphèmes et le nombre de tons de morphèmes}}}}\kern2pt]} \hypertarget{b\{\string_B-lj7\string_M~lj7\string_M1}{}
\markboth{\textcolor{darkblue}{\textbf{\ipa{bæ˩-ljɤ˧\textasciitilde{}ljɤ˧}}}}{}
\textcolor{teal}{\mytextsc{nom}} \hspace{4pt} Ton~: L-.
\textcolor{PineGreen}{\selectlanguage{french}Pomme de pin, fruit du sapin.} \zh{杉树果。}  \zh{量词}: \textcolor{darkblue}{\textbf{\ipa{ɭɯ˧}}}  \mytextsc{clf}~: \textcolor{darkblue}{\textbf{\ipa{ɭɯ˧}}} 
\lhead{\firstmark}
\rhead{\botmark}

\subsection{\hspace{-0.5cm} {\Large \textcolor{darkblue}{\textbf{\ipa{bæ˩pʰv̩˥}}}}\hspace{0.5cm}[\kern2pt{\textcolor{darkblue}{\textbf{\ipa{bæ˧pʰv̩˧}}}}\kern2pt]} \hypertarget{b\{\string_Bp\string_hv\string_=\string_T1}{}
\markboth{\textcolor{darkblue}{\textbf{\ipa{bæ˩pʰv̩˥}}}}{}
\textcolor{teal}{\mytextsc{nom}} \hspace{4pt} Ton~: L+H\#.
\textcolor{PineGreen}{\selectlanguage{french}Chrysanthème couronné, chrysanthème des jardins, chrysanthème comestible ou chrysanthème à couronnes, \textit{Glebionis coronaria}.} \zh{茼蒿。}  ¶ \textcolor{darkblue}{\textbf{\ipa{bæ˩pʰv̩˥-bv̩˩ | bæ˩bæ˩˥}}} \textcolor{PineGreen}{\selectlanguage{french}fleur de chrysanthème couronné} \zh{茼蒿的顶花}  
 ¶ \textcolor{darkblue}{\textbf{\ipa{bæ˩pʰv̩˥-bæ˩bæ˩}}} \textcolor{PineGreen}{\selectlanguage{french}fleur de chrysanthème couronné} \zh{茼蒿顶花}  
 \zh{量词}: \textcolor{darkblue}{\textbf{\ipa{po˧}}}  \mytextsc{clf}~: \textcolor{darkblue}{\textbf{\ipa{po˧}}} 
\lhead{\firstmark}
\rhead{\botmark}

\subsection{\hspace{-0.5cm} {\Large \textcolor{darkblue}{\textbf{\ipa{bæ˩-ʁwæ˩\textasciitilde{}ʁwæ˥}}}}\hspace{0.5cm}[\kern2pt{\textcolor{darkblue}{\textbf{\ipa{xxxx non-correspondance entre le nombre de morphèmes et le nombre de tons de morphèmes}}}}\kern2pt]} \hypertarget{b\{\string_B-Rw\{\string_B~Rw\{\string_T1}{}
\markboth{\textcolor{darkblue}{\textbf{\ipa{bæ˩-ʁwæ˩\textasciitilde{}ʁwæ˥}}}}{}
\textcolor{teal}{\mytextsc{adjectif}} \hspace{4pt} Ton~: L.
\textcolor{PineGreen}{\selectlanguage{french}Relâché.} \zh{松。}  ¶ \textcolor{darkblue}{\textbf{\ipa{ʈʂʰɯ˧ | ɖwæ˧˥ | bæ˩ʁwæ˩\textasciitilde{}ʁwæ˥-ʝi˩!}}} \textcolor{PineGreen}{\selectlanguage{french}C'est tout relâché, ce n'est pas bien serré! (Au sujet d'une charge sur le dos d'un mulet)} \zh{(驮在马上面的货物没系好)松动了!}  

\lhead{\firstmark}
\rhead{\botmark}

\subsection{\hspace{-0.5cm} {\Large \textcolor{darkblue}{\textbf{\ipa{bæ˩ʈʂo˥}}}}\hspace{0.5cm}[\kern2pt{\textcolor{darkblue}{\textbf{\ipa{bæ˧ʈʂo˧}}}}\kern2pt]} \hypertarget{b\{\string_Bt`s`o\string_T1}{}
\markboth{\textcolor{darkblue}{\textbf{\ipa{bæ˩ʈʂo˥}}}}{}
\textcolor{teal}{\mytextsc{nom}} \hspace{4pt} Ton~: LH.
\textcolor{PineGreen}{\selectlanguage{french}Balai.} \zh{扫帚。}  \zh{量词}: \textcolor{darkblue}{\textbf{\ipa{nɑ˧}}}  \mytextsc{clf}~: \textcolor{darkblue}{\textbf{\ipa{nɑ˧}}} 
\lhead{\firstmark}
\rhead{\botmark}

\subsection{\hspace{-0.5cm} {\Large \textcolor{darkblue}{\textbf{\ipa{bæ˩ʈʂwæ˩}}}}\hspace{0.5cm}[\kern2pt{\textcolor{darkblue}{\textbf{\ipa{bæ˩ʈʂwæ˥}}}}\kern2pt]} \hypertarget{b\{\string_Bt`s`w\{\string_B1}{}
\markboth{\textcolor{darkblue}{\textbf{\ipa{bæ˩ʈʂwæ˩}}}}{}
\textcolor{teal}{\mytextsc{nom}} \hspace{4pt} Ton~: L.
\textcolor{PineGreen}{\selectlanguage{french}Rênes.} \zh{缰绳。}  ¶ \textcolor{darkblue}{\textbf{\ipa{ʐwæ˧-bæ˥ʈʂwæ˩}}} \textcolor{PineGreen}{\selectlanguage{french}rênes du cheval} \zh{马缰绳}  
 \zh{量词}: \textcolor{darkblue}{\textbf{\ipa{kʰɯ˩}}}  \mytextsc{clf}~: \textcolor{darkblue}{\textbf{\ipa{kʰɯ˩}}} 
\lhead{\firstmark}
\rhead{\botmark}

\subsection{\hspace{-0.5cm} {\Large \textcolor{darkblue}{\textbf{\ipa{bæ˧˥}}}}\hspace{0.5cm}[\kern2pt{\textcolor{darkblue}{\textbf{\ipa{bæ˧˥}}}}\kern2pt]} \hypertarget{b\{\string_M\string_T1}{}
\markboth{\textcolor{darkblue}{\textbf{\ipa{bæ˧˥}}}}{}
\textcolor{teal}{\mytextsc{verbe}} \hspace{4pt} Ton~: MH.
\textcolor{PineGreen}{\selectlanguage{french}Courir.} \zh{跑。}  ¶ \textcolor{darkblue}{\textbf{\ipa{le˧-bæ˧-ze˥}}} \textcolor{PineGreen}{\selectlanguage{french}\mytextsc{accomp} \string_ \mytextsc{pfv}} \zh{跑了}  

\lhead{\firstmark}
\rhead{\botmark}

\subsection{\hspace{-0.5cm} {\Large \textcolor{darkblue}{\textbf{\ipa{bæ˩˥}}}}\hspace{0.5cm}[\kern2pt{\textcolor{darkblue}{\textbf{\ipa{bæ˩˥}}}}\kern2pt]} \hypertarget{b\{\string_B\string_T1}{}
\markboth{\textcolor{darkblue}{\textbf{\ipa{bæ˩˥}}}}{}
\textcolor{teal}{\mytextsc{nom}} \hspace{4pt} Ton~: LH.
\textcolor{PineGreen}{\selectlanguage{french}Pus.} \zh{脓。}  ¶ \textcolor{darkblue}{\textbf{\ipa{bæ˩ bæ˧-ze˩}}} \textcolor{PineGreen}{\selectlanguage{french}la blessure donne du pus, il y a du pus} \zh{伤口在化脓}  
 ¶ \textcolor{darkblue}{\textbf{\ipa{bæ˩˥ | le˧-bæ˩-ze˩}}} \textcolor{PineGreen}{\selectlanguage{french}la blessure donne du pus, il y a du pus} \zh{伤口在化脓}  
 ¶ \textcolor{darkblue}{\textbf{\ipa{bæ˩˥ | le˧-bæ˩-ze˩}}} \textcolor{PineGreen}{\selectlanguage{french}yyyy reporter : verbe, ton ˩}  
 \zh{量词}: \textcolor{darkblue}{\textbf{\ipa{ʈʰɤ˥}}}  \mytextsc{clf}~: \textcolor{darkblue}{\textbf{\ipa{ʈʰɤ˥}}} \textit{Voir~:} \hyperlink{}{\textcolor{darkblue}{\textbf{\ipa{bæ˩}}} \textsubscript{2}} 
\lhead{\firstmark}
\rhead{\botmark}

\subsection{\hspace{-0.5cm} {\Large \textcolor{darkblue}{\textbf{\ipa{bæ˩˧}}}}\hspace{0.5cm}[\kern2pt{\textcolor{darkblue}{\textbf{\ipa{bæ˩˥}}}}\kern2pt]} \hypertarget{b\{\string_B\string_M1}{}
\markboth{\textcolor{darkblue}{\textbf{\ipa{bæ˩˧}}}}{}
\textcolor{teal}{\mytextsc{nom}} \hspace{4pt} Ton~: LM.
\textcolor{PineGreen}{\selectlanguage{french}Récolte; plantes que l'on a semées.} \zh{庄稼。}  ¶ \textcolor{darkblue}{\textbf{\ipa{bæ˩ ɲi˧}}} \textcolor{PineGreen}{\selectlanguage{french}\mytextsc{cop}} \zh{是庄稼}  
 ¶ \textcolor{darkblue}{\textbf{\ipa{ɖɯ˧-kʰv̩˧ ʈv̩˧-bæ˥ mv̩˩, | ɕi˧-kʰv̩˧ | le˧-mɤ˧-dzɯ˧!}}} \textcolor{PineGreen}{\selectlanguage{french}“Quand bien même on aurait fait une récolte fabuleuse, ça ne nous durerait pas éternellement: ça se rejoue chaque année!” Littéralement: “si, une année, mille récoltes parvenaient à maturité, on n['en] mangerait pas [pour autant pendant] cent ans =on n'aurait pas à manger pour cent ans!” Le proverbe sert à se consoler d'une mauvaise récolte, qui va obliger à une année frugale: “Si belle soit la récolte, elle n'aurait de toute façon pas duré éternellement; tout est à recommencer l'année suivante, voyons donc de l'avant!”} \zh{“一年收千棵,不够吃百年!”(这个谚语,来慰藉收成不好的年份。)}  
 \zh{量词}: \textcolor{darkblue}{\textbf{\ipa{bæ˩}}}  \mytextsc{clf}~: \textcolor{darkblue}{\textbf{\ipa{bæ˩}}} 
\lhead{\firstmark}
\rhead{\botmark}

\subsection{\hspace{-0.5cm} {\Large \textcolor{darkblue}{\textbf{\ipa{bɤ˥}}}}\hspace{0.5cm}[\kern2pt{\textcolor{darkblue}{\textbf{\ipa{bɤ˥}}}}\kern2pt]} \hypertarget{b7\string_T1}{}
\markboth{\textcolor{darkblue}{\textbf{\ipa{bɤ˥}}}}{}
\textcolor{teal}{\mytextsc{nom}} \hspace{4pt} Ton~: \#H.
\textcolor{PineGreen}{\selectlanguage{french}Pumi (Prinmi) (groupe ethnique).} \zh{普米族。}  ¶ \textcolor{darkblue}{\textbf{\ipa{bɤ˧-ʐwɤ˧ so˥}}} \textcolor{PineGreen}{\selectlanguage{french}apprendre la langue prinmi} \zh{学普米语}  
 \zh{量词}: \textcolor{darkblue}{\textbf{\ipa{v̩˧}}}  \mytextsc{clf}~: \textcolor{darkblue}{\textbf{\ipa{v̩˧}}} 
\lhead{\firstmark}
\rhead{\botmark}

\subsection{\hspace{-0.5cm} {\Large \textcolor{darkblue}{\textbf{\ipa{bɤ˧dzi˩}}}}\hspace{0.5cm}[\kern2pt{\textcolor{darkblue}{\textbf{\ipa{xxxx non-correspondance entre le nombre de morphèmes et le nombre de tons de morphèmes}}}}\kern2pt]} \hypertarget{b7\string_Mdzi\string_B1}{}
\markboth{\textcolor{darkblue}{\textbf{\ipa{bɤ˧dzi˩}}}}{}
\textcolor{teal}{\mytextsc{nom}} \hspace{4pt} Ton~: L\#.
\textcolor{PineGreen}{\selectlanguage{french}Un des villages de la plaine de Yongning.} \zh{八珠(永宁的一个村落)。}  ¶ \textcolor{darkblue}{\textbf{\ipa{ɖæ˩ʂɯ\#˥, | ʈʂo˧ʂɯ\#˥, | bɤ˩tɕʰɯ˩˥, | dɑ˧pʰo˥, | bɤ˧dzi˩, | dze˧bo˧}}} \textcolor{PineGreen}{\selectlanguage{french}les six villages de la plaine de Yongning, dans l'ordre, qui prend comme point d'origine le village le plus proche du Lac} \zh{永宁坝的六个村落,按传统排序:从距离泸沽湖最近的村落说起。}  

\lhead{\firstmark}
\rhead{\botmark}

\subsection{\hspace{-0.5cm} {\Large \textcolor{darkblue}{\textbf{\ipa{bɤ˧kɯ˧}}}}\hspace{0.5cm}[\kern2pt{\textcolor{darkblue}{\textbf{\ipa{bɤ˩kɯ˥}}}}\kern2pt]} \hypertarget{b7\string_MkM\string_M1}{}
\markboth{\textcolor{darkblue}{\textbf{\ipa{bɤ˧kɯ˧}}}}{}
\textcolor{teal}{\mytextsc{nom}} \hspace{4pt} Ton~: M.
\textcolor{PineGreen}{\selectlanguage{french}Vannerie: tamis, crible, en forme de gourde; on y met des légumes, des choses à porter. Cette vannerie est commode à porter.} \zh{筛子。}  \zh{量词}: \textcolor{darkblue}{\textbf{\ipa{nɑ˧}}}  \mytextsc{clf}~: \textcolor{darkblue}{\textbf{\ipa{nɑ˧}}} 
\lhead{\firstmark}
\rhead{\botmark}

\subsection{\hspace{-0.5cm} {\Large \textcolor{darkblue}{\textbf{\ipa{bɤ˧mi\#˥}}}}\hspace{0.5cm}[\kern2pt{\textcolor{darkblue}{\textbf{\ipa{xxxx non-correspondance entre le nombre de morphèmes et le nombre de tons de morphèmes}}}}\kern2pt]} \hypertarget{b7\string_Mmi\#\string_T1}{}
\markboth{\textcolor{darkblue}{\textbf{\ipa{bɤ˧mi\#˥}}}}{}
\textcolor{teal}{\mytextsc{nom}} \hspace{4pt} Ton~: \#H.
\textcolor{PineGreen}{\selectlanguage{french}Femme pumi.} \zh{普米族女人。}  \zh{量词}: \textcolor{darkblue}{\textbf{\ipa{v̩˧}}}  \mytextsc{clf}~: \textcolor{darkblue}{\textbf{\ipa{v̩˧}}} 
\lhead{\firstmark}
\rhead{\botmark}

\subsection{\hspace{-0.5cm} {\Large \textcolor{darkblue}{\textbf{\ipa{bɤ˧mi˥-ʂe˩}}}}\hspace{0.5cm}[\kern2pt{\textcolor{darkblue}{\textbf{\ipa{bɤ˩mi˧ʂe˧}}}}\kern2pt]} \hypertarget{b7\string_Mmi\string_T-s`e\string_B1}{}
\markboth{\textcolor{darkblue}{\textbf{\ipa{bɤ˧mi˥-ʂe˩}}}}{}
\textcolor{teal}{\mytextsc{nom}} \hspace{4pt} Ton~: H\#-.
\textcolor{PineGreen}{\selectlanguage{french}Cupronickel: alliage cuivre-nickel.} \zh{白铜。} 
\lhead{\firstmark}
\rhead{\botmark}

\subsection{\hspace{-0.5cm} {\Large \textcolor{darkblue}{\textbf{\ipa{bɤ˧ʂɯ˩}}}}\hspace{0.5cm}[\kern2pt{\textcolor{darkblue}{\textbf{\ipa{bɤ˩ʂɯ˥}}}}\kern2pt]} \hypertarget{b7\string_Ms`M\string_B1}{}
\markboth{\textcolor{darkblue}{\textbf{\ipa{bɤ˧ʂɯ˩}}}}{}
\textcolor{teal}{\mytextsc{nom}} \hspace{4pt} Ton~: L\#.
\textcolor{PineGreen}{\selectlanguage{french}Nom d'un village de la plaine de Lijiang, d'où venaient de nombreux marchands, d'où le fait que son nom soit connu à Yongning. En naxi: \textcolor{darkblue}{\textbf{\ipa{/bɤ˧ʂɯ˩/}}}.} \zh{白沙(丽江坝子里的一个村落)。} 
\lhead{\firstmark}
\rhead{\botmark}

\subsection{\hspace{-0.5cm} {\Large \textcolor{darkblue}{\textbf{\ipa{bɤ˧tʰv̩˩}}}}\hspace{0.5cm}[\kern2pt{\textcolor{darkblue}{\textbf{\ipa{bɤ˧tʰv̩˥}}}}\kern2pt]} \hypertarget{b7\string_Mt\string_hv\string_=\string_B1}{}
\markboth{\textcolor{darkblue}{\textbf{\ipa{bɤ˧tʰv̩˩}}}}{}
\textcolor{teal}{\mytextsc{nom}} \hspace{4pt} Ton~: L\#.
\textcolor{PineGreen}{\selectlanguage{french}Empreintes, traces de pas, traces de pied.} \zh{脚印。}  ¶ \textcolor{darkblue}{\textbf{\ipa{hĩ˧-bɤ˧tʰv̩˥}}} \textcolor{PineGreen}{\selectlanguage{french}empreintes (de pied) d'homme} \zh{人的脚印}  
 ¶ \textcolor{darkblue}{\textbf{\ipa{kʰv̩˩mi˩-bɤ˩tʰv̩˥}}} \textcolor{PineGreen}{\selectlanguage{french}empreintes de (pattes de) chien} \zh{狗爪印}  
 \zh{量词}: \textcolor{darkblue}{\textbf{\ipa{tʰv̩˧˥}}}  \mytextsc{clf}~: \textcolor{darkblue}{\textbf{\ipa{tʰv̩˧˥}}} 
\lhead{\firstmark}
\rhead{\botmark}

\subsection{\hspace{-0.5cm} {\Large \textcolor{darkblue}{\textbf{\ipa{bɤ˧tsʰo˧gv̩˥}}}}\hspace{0.5cm}[\kern2pt{\textcolor{darkblue}{\textbf{\ipa{bɤ˧tsʰo˧gv̩˩}}}}\kern2pt]} \hypertarget{b7\string_Mts\string_ho\string_Mgv\string_=\string_T1}{}
\markboth{\textcolor{darkblue}{\textbf{\ipa{bɤ˧tsʰo˧gv̩˥}}}}{}
\textcolor{teal}{\mytextsc{nom}} \hspace{4pt} Ton~: H\#.
\textcolor{PineGreen}{\selectlanguage{french}Un des villages de la plaine de Yongning; lieu de l'actuel marché; terme également employé pour désigner le lieu d'habitation des artisans Naxi).} \zh{巴搓古(永宁的一个村落)。}  ¶ \textcolor{darkblue}{\textbf{\ipa{bɤ˧tsʰo˧gv̩˥-hĩ˩}}} \textcolor{PineGreen}{\selectlanguage{french}quelqu'un de Bacuogu} \zh{从巴搓古来的一个人}  
 ¶ \textcolor{darkblue}{\textbf{\ipa{dʑɤ˩bv̩˧kɤ˧-sɑ˥ʁwɤ˩, | hi˩ʁwɤ˩-lo˥, | æ˩mi˧-ʁwɤ\#˥, | lɑ˧lo˧-ʁwɤ˥, | lɑ˧ŋwɤ˧, | bɤ˧tsʰo˧gv̩˥, | ə˧lɑ˧-ʁwɤ\#˥, | gæ˧ɻæ˩, | qʰæ˧tɕʰi˧, | tʰo˧ʈɯ\#˥}}} \textcolor{PineGreen}{\selectlanguage{french}les dix villages comptant traditionnellement comme faisant partie de Yongning} \zh{摩梭传统地理概念中,属于永宁的十个村落}  

\lhead{\firstmark}
\rhead{\botmark}

\subsection{\hspace{-0.5cm} {\Large \textcolor{darkblue}{\textbf{\ipa{bɤ˧zo\#˥}}}}\hspace{0.5cm}[\kern2pt{\textcolor{darkblue}{\textbf{\ipa{bɤ˧zo˧}}}}\kern2pt]} \hypertarget{b7\string_Mzo\#\string_T1}{}
\markboth{\textcolor{darkblue}{\textbf{\ipa{bɤ˧zo\#˥}}}}{}
\textcolor{teal}{\mytextsc{nom}} \hspace{4pt} Ton~: \#H.
\textcolor{PineGreen}{\selectlanguage{french}Homme pumi.} \zh{普米族男人。}  \zh{量词}: \textcolor{darkblue}{\textbf{\ipa{v̩˧}}}  \mytextsc{clf}~: \textcolor{darkblue}{\textbf{\ipa{v̩˧}}} 
\lhead{\firstmark}
\rhead{\botmark}

\subsection{\hspace{-0.5cm} {\Large \textcolor{darkblue}{\textbf{\ipa{bɤ˩\textsubscript{a}}}} \textsubscript{1}}\hspace{0.5cm}[\kern2pt{\textcolor{darkblue}{\textbf{\ipa{bɤ˩˥}}}}\kern2pt]} \hypertarget{b7\string_Ba1}{}
\markboth{\textcolor{darkblue}{\textbf{\ipa{bɤ˩\textsubscript{a}}}} \textsubscript{1}}{}
\textcolor{teal}{\mytextsc{classificateur}} \hspace{4pt} Ton~: L\textsubscript{a}.
\textcolor{PineGreen}{\selectlanguage{french}Classificateur des épis de maïs.} \zh{量词:玉米棒子(一根)。}  ¶ \textcolor{darkblue}{\textbf{\ipa{hɑ˧bɤ˥ | ɖɯ˧-bɤ˩}}} \textcolor{PineGreen}{\selectlanguage{french}un épi de maïs} \zh{一根玉米棒子}  
 ¶ \textcolor{darkblue}{\textbf{\ipa{tʰv̩˧-bɤ˥}}} \textcolor{PineGreen}{\selectlanguage{french}\mytextsc{dem} \string_ (ton: H\# / H\$)} \zh{\mytextsc{指示代词} \string_:那根(玉米棒子)}  

\lhead{\firstmark}
\rhead{\botmark}

\subsection{\hspace{-0.5cm} {\Large \textcolor{darkblue}{\textbf{\ipa{bɤ˩\textsubscript{a}}}} \textsubscript{2}}\hspace{0.5cm}[\kern2pt{\textcolor{darkblue}{\textbf{\ipa{bɤ˩˥}}}}\kern2pt]} \hypertarget{b7\string_Ba2}{}
\markboth{\textcolor{darkblue}{\textbf{\ipa{bɤ˩\textsubscript{a}}}} \textsubscript{2}}{}
\textcolor{teal}{\mytextsc{classificateur}} \hspace{4pt} Ton~: L\textsubscript{a}.
\textcolor{PineGreen}{\selectlanguage{french}Classificateur des moitiés.} \zh{量词:半。}  ¶ \textcolor{darkblue}{\textbf{\ipa{ɖɯ˧-bɤ˩-lɑ˩ tʰv̩˩-sɯ˩! | ɖɯ˧-hu˧-ɻ̍˥!}}} \textcolor{PineGreen}{\selectlanguage{french}Je n'ai fait que la moitié! Attends un peu! (Contexte: quelqu'un est en train de trier les vêtements, et signale qu'il n'a pas fini)} \zh{我才干了一半!稍等!(情景:一个在收拾衣服,告诉对方:工作没完,还需要时间。)}  
 ¶ \textcolor{darkblue}{\textbf{\ipa{ɖɯ˧-bɤ˩-lɑ˩ tʰv̩˩-ze˩!}}} \textcolor{PineGreen}{\selectlanguage{french}Tu es à mi-chemin! / Tu n'as parcouru que la moitié du chemin! (Réflexion de la consultante principale au sujet de mon apprentissage de la langue na. Cette formulation souligne le chemin qui reste à parcourir, mais l'emploi du perfectif apporte une note d'encouragement, là où l'exemple précédant insisterait essentiellement sur le chemin restant à parcourir: \textcolor{darkblue}{\textbf{\ipa{ɖɯ˧-bɤ˩-lɑ˩ tʰv̩˩-sɯ˩}}}.)} \zh{你才到了一半!(合作人对调查者学摩梭话的评定)}  
 ¶ \textcolor{darkblue}{\textbf{\ipa{ʐæ˩ʂæ˥ | ʐwæ˩˥! | le˧-se˥, | ɖɯ˧-bɤ˩-qo˩-lɑ˩ tʰv̩˩-sɯ˩!}}} \textcolor{PineGreen}{\selectlanguage{french}C'est bien loin! On a marché, et on n'est encore parvenu qu'à mi-chemin!} \zh{真远!走啊走,才走了一半的路!}  
 ¶ \textcolor{darkblue}{\textbf{\ipa{ʐɤ˩mi˩˥ | ɖɯ˧-bɤ˩}}} \textcolor{PineGreen}{\selectlanguage{french}la moitié du chemin} \zh{半路}  
 ¶ \textcolor{darkblue}{\textbf{\ipa{ə˧mi˧! | wɤ˩˥ | ɖɯ˧-bɤ˩ dʑo˩-sɯ˩-wɤ˩!}}} \textcolor{PineGreen}{\selectlanguage{french}Houlàà! (Que c'est loin!) Il reste encore la moitié du chemin (à parcourir)!} \zh{啊呀嚒!还剩一半的路啊!}  

\lhead{\firstmark}
\rhead{\botmark}

\subsection{\hspace{-0.5cm} {\Large \textcolor{darkblue}{\textbf{\ipa{bɤ˩tɕʰɯ˩}}}}\hspace{0.5cm}[\kern2pt{\textcolor{darkblue}{\textbf{\ipa{bɤ˧tɕʰɯ˩}}}}\kern2pt]} \hypertarget{b7\string_Bts£\string_hM\string_B1}{}
\markboth{\textcolor{darkblue}{\textbf{\ipa{bɤ˩tɕʰɯ˩}}}}{}
\textcolor{teal}{\mytextsc{nom}} \hspace{4pt} Ton~: L.
\textcolor{PineGreen}{\selectlanguage{french}Un des villages de la plaine de Yongning.} \zh{八七(永宁的一个村落)。}  ¶ \textcolor{darkblue}{\textbf{\ipa{ɖæ˩ʂɯ\#˥, | ʈʂo˧ʂɯ\#˥, | bɤ˩tɕʰɯ˩˥, | dɑ˧pʰo˥, | bɤ˧dzi˩, | dze˧bo˧}}} \textcolor{PineGreen}{\selectlanguage{french}les six villages de la plaine de Yongning, dans l'ordre, qui prend comme point d'origine le village le plus proche du Lac} \zh{永宁坝的六个村落,按传统排序:从距离泸沽湖最近的村落说起。}  

\lhead{\firstmark}
\rhead{\botmark}

\subsection{\hspace{-0.5cm} {\Large \textcolor{darkblue}{\textbf{\ipa{bɤ˧˥\textsubscript{a}}}}}\hspace{0.5cm}[\kern2pt{\textcolor{darkblue}{\textbf{\ipa{bɤ˩˥}}}}\kern2pt]} \hypertarget{b7\string_M\string_Ta1}{}
\markboth{\textcolor{darkblue}{\textbf{\ipa{bɤ˧˥\textsubscript{a}}}}}{}
\textcolor{teal}{\mytextsc{classificateur}} \hspace{4pt} Ton~: MH\textsubscript{a}.
\textcolor{PineGreen}{\selectlanguage{french}Classificateur des fichus et foulards.} \zh{量词:头帕(一条)。} 
\lhead{\firstmark}
\rhead{\botmark}

\subsection{\hspace{-0.5cm} {\Large \textcolor{darkblue}{\textbf{\ipa{‑bi}}}}\hspace{0.5cm}[\kern2pt{\textcolor{darkblue}{\textbf{\ipa{xxxx groupe tonal entier sans aucun ton}}}}\kern2pt]} \hypertarget{‑bi1}{}
\markboth{\textcolor{darkblue}{\textbf{\ipa{‑bi}}}}{}
\textcolor{teal}{\mytextsc{conjonction}} \hspace{4pt} Ton~: 0.
\textcolor{PineGreen}{\selectlanguage{french}\mytextsc{adversatif}: bien que, même si.} \zh{虽然……。}  ¶ \textcolor{darkblue}{\textbf{\ipa{ʈʂʰɯ˧ | nɑ˩ ɲi˥-pi˩-bi˩-bi˩, | nɑ˩-ʐwɤ˧ | mɤ˧-kv̩˧˥!}}} \textcolor{PineGreen}{\selectlanguage{french}Bien qu'elle/il soit Na, elle/il ne sait pas parler la langue na!} \zh{他虽然是摩梭人但不会讲摩梭话。}  
 ¶ \textcolor{darkblue}{\textbf{\ipa{*ʈʂʰɯ˧ | nɑ˩ ɲi˥-bi˩, … / *ʈʂʰɯ˧ | nɑ˩ ɲi˥-bi˩-bi˩}}} \textcolor{PineGreen}{\selectlanguage{french}phrase non acceptable; l'intention était de dire 'bien qu'il soit Na…'} \zh{病句:不能这样说“他虽然是摩梭人……”}  

\lhead{\firstmark}
\rhead{\botmark}

\subsection{\hspace{-0.5cm} {\Large \textcolor{darkblue}{\textbf{\ipa{bi˥}}}}\hspace{0.5cm}[\kern2pt{\textcolor{darkblue}{\textbf{\ipa{bi˥}}}}\kern2pt]} \hypertarget{bi\string_T1}{}
\markboth{\textcolor{darkblue}{\textbf{\ipa{bi˥}}}}{}
\textcolor{teal}{\mytextsc{nom}} \hspace{4pt} Ton~: \#H.
\textcolor{PineGreen}{\selectlanguage{french}Neige.} \zh{雪。}  ¶ \textcolor{darkblue}{\textbf{\ipa{bi˧ gi˧-ze˩}}} \textcolor{PineGreen}{\selectlanguage{french}il neige} \zh{下雪了}  
 \zh{量词}: \textcolor{darkblue}{\textbf{\ipa{ʁwɤ˧, etc}}}  \mytextsc{clf}~: \textcolor{darkblue}{\textbf{\ipa{ʁwɤ˧, etc}}} 
\lhead{\firstmark}
\rhead{\botmark}

\subsection{\hspace{-0.5cm} {\Large \textcolor{darkblue}{\textbf{\ipa{bi˥}}}}\hspace{0.5cm}[\kern2pt{\textcolor{darkblue}{\textbf{\ipa{bi˥}}}}\kern2pt]} \hypertarget{bi\string_T1}{}
\markboth{\textcolor{darkblue}{\textbf{\ipa{bi˥}}}}{}
\textcolor{teal}{\mytextsc{adjectif}} \hspace{4pt} Ton~: H.
\textcolor{PineGreen}{\selectlanguage{french}Mince; peu profond.} \zh{薄,浅(水浅)。}  ¶ \textcolor{darkblue}{\textbf{\ipa{bi˧ | ʐwæ˩˥!}}} \textcolor{PineGreen}{\selectlanguage{french}C'est très peu profond!} \zh{很浅!}  
 ¶ \textcolor{darkblue}{\textbf{\ipa{dʑɯ˧ | ɖɯ˧-pi˧ bi˧˥}}} \textcolor{PineGreen}{\selectlanguage{french}L'eau est assez peu profonde.} \zh{水有点浅。}  
 ¶ \textcolor{darkblue}{\textbf{\ipa{dʑɯ˧ bi˧-hĩ˧, | mɤ˧-ɖwæ˩!}}} \textcolor{PineGreen}{\selectlanguage{french}L'eau pas profonde, ça fait pas peur! / L'eau n'est pas profonde; tu vas quand même pas avoir peur!} \zh{水很浅,不用怕!}  

\lhead{\firstmark}
\rhead{\botmark}

\subsection{\hspace{-0.5cm} {\Large \textcolor{darkblue}{\textbf{\ipa{‑bi˧}}}}\hspace{0.5cm}[\kern2pt{\textcolor{darkblue}{\textbf{\ipa{bi˥}}}}\kern2pt]} \hypertarget{‑bi\string_M1}{}
\markboth{\textcolor{darkblue}{\textbf{\ipa{‑bi˧}}}}{}
\textcolor{teal}{\mytextsc{suffixe}} \hspace{4pt} Ton~: M.
\textcolor{PineGreen}{\selectlanguage{french}Futur immédiat.} \zh{要\mytextsc{近将来。}} 
\lhead{\firstmark}
\rhead{\botmark}

\subsection{\hspace{-0.5cm} {\Large \textcolor{darkblue}{\textbf{\ipa{bi˧}}} \textsubscript{2}}\hspace{0.5cm}[\kern2pt{\textcolor{darkblue}{\textbf{\ipa{bi˥}}}}\kern2pt]} \hypertarget{bi\string_M2}{}
\markboth{\textcolor{darkblue}{\textbf{\ipa{bi˧}}} \textsubscript{2}}{}
\textcolor{teal}{\mytextsc{nom}} \hspace{4pt} Ton~: M.
\textcolor{PineGreen}{\selectlanguage{french}Le village; les gens du village, les voisins.} \zh{村落,邻居、村里的人们。} 
\lhead{\firstmark}
\rhead{\botmark}

\subsection{\hspace{-0.5cm} {\Large \textcolor{darkblue}{\textbf{\ipa{bi˧}}} \textsubscript{3}}\hspace{0.5cm}[\kern2pt{\textcolor{darkblue}{\textbf{\ipa{bi˥}}}}\kern2pt]} \hypertarget{bi\string_M3}{}
\markboth{\textcolor{darkblue}{\textbf{\ipa{bi˧}}} \textsubscript{3}}{}
\textcolor{teal}{\mytextsc{verbe}} \hspace{4pt} Ton~: M.
\textcolor{PineGreen}{\selectlanguage{french}Oser.} \zh{敢。}  ¶ \textcolor{darkblue}{\textbf{\ipa{ʝi˧-mɤ˧-bi˧}}} \textcolor{PineGreen}{\selectlanguage{french}ne pas oser faire} \zh{不敢做}  

\lhead{\firstmark}
\rhead{\botmark}

\subsection{\hspace{-0.5cm} {\Large \textcolor{darkblue}{\textbf{\ipa{bi˧\textsubscript{c}}}} \textsubscript{1}}\hspace{0.5cm}[\kern2pt{\textcolor{darkblue}{\textbf{\ipa{bi˩˥}}}}\kern2pt]} \hypertarget{bi\string_Mc1}{}
\markboth{\textcolor{darkblue}{\textbf{\ipa{bi˧\textsubscript{c}}}} \textsubscript{1}}{}
\textcolor{teal}{\mytextsc{verbe}} \hspace{4pt} Ton~: M\textsubscript{c}.
\textcolor{PineGreen}{\selectlanguage{french}Aller.} \zh{去。}  ¶ \textcolor{darkblue}{\textbf{\ipa{bi˧-tʰɑ˧!}}} \textcolor{PineGreen}{\selectlanguage{french}\mytextsc{abilitive}: On peut y aller!} \zh{可以去!}  
 ¶ \textcolor{darkblue}{\textbf{\ipa{bi˧-tʰɑ˧-ze˥!}}} \textcolor{PineGreen}{\selectlanguage{french}\mytextsc{abilitive}+\mytextsc{pfv}: Ca y est, on peut y aller!} \zh{可以去了!}  
 ¶ \textcolor{darkblue}{\textbf{\ipa{le˧-bi˩}}} \textcolor{PineGreen}{\selectlanguage{french}retourner; s'en retourner} \zh{回去,返回}  

\lhead{\firstmark}
\rhead{\botmark}

\subsection{\hspace{-0.5cm} {\Large \textcolor{darkblue}{\textbf{\ipa{bi˧bv̩˥}}}}\hspace{0.5cm}[\kern2pt{\textcolor{darkblue}{\textbf{\ipa{bi˩bv̩˥}}}}\kern2pt]} \hypertarget{bi\string_Mbv\string_=\string_T1}{}
\markboth{\textcolor{darkblue}{\textbf{\ipa{bi˧bv̩˥}}}}{}
\textcolor{teal}{\mytextsc{verbe}} \hspace{4pt} Ton~: H\#.
\textcolor{PineGreen}{\selectlanguage{french}Couler à flots, couler en trombe.} \zh{流淌,冲下去,下泻,很快地流。}  ¶ \textcolor{darkblue}{\textbf{\ipa{tʰi˧-ʈwæ˧˥, | sɤ˧ | bi˧bv̩˥-ze˩!}}} \textcolor{PineGreen}{\selectlanguage{french}[il] est tombé [et s'est blessé]; le sang a coulé à flots!} \zh{(他)摔倒了,流了很多血}  
 ¶ \textcolor{darkblue}{\textbf{\ipa{dʑɯ˧ | bi˧bv̩˥-ze˩!}}} \textcolor{PineGreen}{\selectlanguage{french}L'eau coule à flots!} \zh{水流如注!}  
 ¶ \textcolor{darkblue}{\textbf{\ipa{dʑɯ˩nɑ˩mi˩ bi˩bv̩˥-ze˩-pʰæ˩di˩!}}} \textcolor{PineGreen}{\selectlanguage{french}On dirait qu'il y a une coulée de boue / un glissement de terrain sur la montagne!} \zh{山上好像有了泥石流!}  

\lhead{\firstmark}
\rhead{\botmark}

\subsection{\hspace{-0.5cm} {\Large \textcolor{darkblue}{\textbf{\ipa{bi˧ɕi˧kv̩˥}}}}\hspace{0.5cm}[\kern2pt{\textcolor{darkblue}{\textbf{\ipa{bi˧ɕi˧kv̩˧}}}}\kern2pt]} \hypertarget{bi\string_Ms£i\string_Mkv\string_=\string_T1}{}
\markboth{\textcolor{darkblue}{\textbf{\ipa{bi˧ɕi˧kv̩˥}}}}{}
\textcolor{teal}{\mytextsc{nom}} \hspace{4pt} Ton~: H\#.
\textcolor{PineGreen}{\selectlanguage{french}Joue.} \zh{腮,腮帮子。}  \zh{量词}: \textcolor{darkblue}{\textbf{\ipa{ɭɯ˧}}}  \mytextsc{clf}~: \textcolor{darkblue}{\textbf{\ipa{ɭɯ˧}}} 
\lhead{\firstmark}
\rhead{\botmark}

\subsection{\hspace{-0.5cm} {\Large \textcolor{darkblue}{\textbf{\ipa{bi˧hæ˧˥}}}}\hspace{0.5cm}[\kern2pt{\textcolor{darkblue}{\textbf{\ipa{bi˧hæ˥}}}}\kern2pt]} \hypertarget{bi\string_Mh\{\string_M\string_T1}{}
\markboth{\textcolor{darkblue}{\textbf{\ipa{bi˧hæ˧˥}}}}{}
\textcolor{teal}{\mytextsc{nom}} \hspace{4pt} Ton~: MH\#.
\textcolor{PineGreen}{\selectlanguage{french}Sangle ventrale.} \zh{马肚带。}  ¶ \textcolor{darkblue}{\textbf{\ipa{ʐwæ˧-bi˥hæ˩}}} \textcolor{PineGreen}{\selectlanguage{french}sangle de cheval} \zh{马肚带}  
 \zh{量词}: \textcolor{darkblue}{\textbf{\ipa{kʰɯ˩}}}  \mytextsc{clf}~: \textcolor{darkblue}{\textbf{\ipa{kʰɯ˩}}} 
\lhead{\firstmark}
\rhead{\botmark}

\subsection{\hspace{-0.5cm} {\Large \textcolor{darkblue}{\textbf{\ipa{bi˧-lv̩˧\textasciitilde{}lv̩˥}}}}\hspace{0.5cm}[\kern2pt{\textcolor{darkblue}{\textbf{\ipa{xxxx non-correspondance entre le nombre de morphèmes et le nombre de tons de morphèmes}}}}\kern2pt]} \hypertarget{bi\string_M-lv\string_=\string_M~lv\string_=\string_T1}{}
\markboth{\textcolor{darkblue}{\textbf{\ipa{bi˧-lv̩˧\textasciitilde{}lv̩˥}}}}{}
\textcolor{teal}{\mytextsc{nom}} \hspace{4pt} Ton~: H\#.
\textcolor{PineGreen}{\selectlanguage{french}Flocons de neige.} \zh{雪花。}  \zh{量词}: \textcolor{darkblue}{\textbf{\ipa{ɭɯ˧}}}  \mytextsc{clf}~: \textcolor{darkblue}{\textbf{\ipa{ɭɯ˧}}} 
\lhead{\firstmark}
\rhead{\botmark}

\subsection{\hspace{-0.5cm} {\Large \textcolor{darkblue}{\textbf{\ipa{bi˧mi˧}}}}\hspace{0.5cm}[\kern2pt{\textcolor{darkblue}{\textbf{\ipa{bi˩mi˩˥}}}}\kern2pt]} \hypertarget{bi\string_Mmi\string_M1}{}
\markboth{\textcolor{darkblue}{\textbf{\ipa{bi˧mi˧}}}}{}
\textcolor{teal}{\mytextsc{nom}} \hspace{4pt} Ton~: M.
\textcolor{PineGreen}{\selectlanguage{french}Ventre.} \zh{肚子。}  ¶ \textcolor{darkblue}{\textbf{\ipa{bi˧mi˧-ɖɯ˩}}} \textcolor{PineGreen}{\selectlanguage{french}avide, glouton; littéralement “qui a un gros ventre/gros estomac”} \zh{贪心不足,贪吃}  
 \zh{量词}: \textcolor{darkblue}{\textbf{\ipa{ɭɯ˧}}}  \mytextsc{clf}~: \textcolor{darkblue}{\textbf{\ipa{ɭɯ˧}}} 
\lhead{\firstmark}
\rhead{\botmark}

\subsection{\hspace{-0.5cm} {\Large \textcolor{darkblue}{\textbf{\ipa{bi˧tɑ˧}}}}\hspace{0.5cm}[\kern2pt{\textcolor{darkblue}{\textbf{\ipa{bi˩tɑ˩˥}}}}\kern2pt]} \hypertarget{bi\string_MtA\string_M1}{}
\markboth{\textcolor{darkblue}{\textbf{\ipa{bi˧tɑ˧}}}}{}
\textcolor{teal}{\mytextsc{nom}} \hspace{4pt} Ton~: M.
\textcolor{PineGreen}{\selectlanguage{french}Ceinture large, en tissu.} \zh{宽腰带。}  \zh{量词}: \textcolor{darkblue}{\textbf{\ipa{tsʰi˥}}}  \mytextsc{clf}~: \textcolor{darkblue}{\textbf{\ipa{tsʰi˥}}} 
\lhead{\firstmark}
\rhead{\botmark}

\subsection{\hspace{-0.5cm} {\Large \textcolor{darkblue}{\textbf{\ipa{bi˧tɕɤ˩}}}}\hspace{0.5cm}[\kern2pt{\textcolor{darkblue}{\textbf{\ipa{bi˧tɕɤ˧}}}}\kern2pt]} \hypertarget{bi\string_Mts£7\string_B1}{}
\markboth{\textcolor{darkblue}{\textbf{\ipa{bi˧tɕɤ˩}}}}{}
\textcolor{teal}{\mytextsc{nom}} \hspace{4pt} Ton~: L\#.
\textcolor{PineGreen}{\selectlanguage{french}Nombril.} \zh{肚脐。}  \zh{量词}: \textcolor{darkblue}{\textbf{\ipa{kʰwɤ˥}}}  \mytextsc{clf}~: \textcolor{darkblue}{\textbf{\ipa{kʰwɤ˥}}} 
\lhead{\firstmark}
\rhead{\botmark}

\subsection{\hspace{-0.5cm} {\Large \textcolor{darkblue}{\textbf{\ipa{bi˧tɕo˧}}}}\hspace{0.5cm}[\kern2pt{\textcolor{darkblue}{\textbf{\ipa{bi˧tɕo˧}}}}\kern2pt]} \hypertarget{bi\string_Mts£o\string_M1}{}
\markboth{\textcolor{darkblue}{\textbf{\ipa{bi˧tɕo˧}}}}{}
\textcolor{teal}{\mytextsc{nom}} \hspace{4pt} Ton~: M.
\textcolor{PineGreen}{\selectlanguage{french}Les villages environnants; par exemple, pour la consultante principale, le village, \textcolor{darkblue}{\textbf{\ipa{/ʁwɤ˧qo˧/}}}, c'est \textcolor{darkblue}{\textbf{\ipa{/ə˧lɑ˧-ʁwɤ˧/;}}} les villages environnants constituent le voisinage, \textcolor{darkblue}{\textbf{\ipa{/bi˧tɕo˧/}}}.} \zh{周围的村落。} 
\lhead{\firstmark}
\rhead{\botmark}

\subsection{\hspace{-0.5cm} {\Large \textcolor{darkblue}{\textbf{\ipa{bi˧zɯ˧}}}}\hspace{0.5cm}[\kern2pt{\textcolor{darkblue}{\textbf{\ipa{bi˩zɯ˥}}}}\kern2pt]} \hypertarget{bi\string_MzM\string_M1}{}
\markboth{\textcolor{darkblue}{\textbf{\ipa{bi˧zɯ˧}}}}{}
\textcolor{teal}{\mytextsc{nom}} \hspace{4pt} Ton~: M.
\textcolor{PineGreen}{\selectlanguage{french}Bas-ventre.} \zh{小肚子。}  \zh{量词}: \textcolor{darkblue}{\textbf{\ipa{ɭɯ˧}}}  \mytextsc{clf}~: \textcolor{darkblue}{\textbf{\ipa{ɭɯ˧}}} 
\lhead{\firstmark}
\rhead{\botmark}

\subsection{\hspace{-0.5cm} {\Large \textcolor{darkblue}{\textbf{\ipa{bi˩}}}}\hspace{0.5cm}[\kern2pt{\textcolor{darkblue}{\textbf{\ipa{bi˥}}}}\kern2pt]} \hypertarget{bi\string_B1}{}
\markboth{\textcolor{darkblue}{\textbf{\ipa{bi˩}}}}{}
\textcolor{teal}{\mytextsc{postposition}} \hspace{4pt} Ton~: L.
\textcolor{PineGreen}{\selectlanguage{french}Sur; vers.} \zh{向、至、往。}  ¶ \textcolor{darkblue}{\textbf{\ipa{gv̩˧mi˧-bi˩}}} \textcolor{PineGreen}{\selectlanguage{french}sur le corps} \zh{身上}  
 ¶ \textcolor{darkblue}{\textbf{\ipa{kʰɯ˧tsʰɤ˧-bi˥}}} \textcolor{PineGreen}{\selectlanguage{french}sur les pieds} \zh{脚上}  
 ¶ \textcolor{darkblue}{\textbf{\ipa{ʐæ˩sɯ˩-bi˥ | tʰi˧-ʈʂʰv̩˧˥}}} \textcolor{PineGreen}{\selectlanguage{french}[elle] a agrippé son vêtement} \zh{抓住毡子}  
 ¶ \textcolor{darkblue}{\textbf{\ipa{lo˩qʰwɤ˧ bi˩}}} \textcolor{PineGreen}{\selectlanguage{french}sur la main} \zh{手上}  
 ¶ \textcolor{darkblue}{\textbf{\ipa{pʰæ˧qʰwɤ˩ bi˩, | mɤ˩ tʰi˩-jɤ˩˥.}}} \textcolor{PineGreen}{\selectlanguage{french}s’étaler de l’huile sur le visage, se mettre de la crème solaire sur le visage} \zh{抹防晒霜}  

\lhead{\firstmark}
\rhead{\botmark}

\subsection{\hspace{-0.5cm} {\Large \textcolor{darkblue}{\textbf{\ipa{bi˩\textsubscript{c}}}}}\hspace{0.5cm}[\kern2pt{\textcolor{darkblue}{\textbf{\ipa{bi˥}}}}\kern2pt]} \hypertarget{bi\string_Bc1}{}
\markboth{\textcolor{darkblue}{\textbf{\ipa{bi˩\textsubscript{c}}}}}{}
\textcolor{teal}{\mytextsc{classificateur}} \hspace{4pt} Ton~: L\textsubscript{c}.
\textcolor{PineGreen}{\selectlanguage{french}Auto-classificateur des sabots d'animaux, et des traces qu'ils laissent sur le sol.} \zh{量词:动物的脚或脚印(一只)。} 
\lhead{\firstmark}
\rhead{\botmark}

\subsection{\hspace{-0.5cm} {\Large \textcolor{darkblue}{\textbf{\ipa{bi˩bi˧}}}}\hspace{0.5cm}[\kern2pt{\textcolor{darkblue}{\textbf{\ipa{bi˧bi˧}}}}\kern2pt]} \hypertarget{bi\string_Bbi\string_M1}{}
\markboth{\textcolor{darkblue}{\textbf{\ipa{bi˩bi˧}}}}{}
\textcolor{teal}{\mytextsc{nom}} \hspace{4pt} Ton~: LM.
\textcolor{PineGreen}{\selectlanguage{french}Cosse de haricot.} \zh{豆荚。}  ¶ \textcolor{darkblue}{\textbf{\ipa{bi˩bi˧ ɲi˩}}} \textcolor{PineGreen}{\selectlanguage{french}\mytextsc{cop}} \zh{是豆荚}  
 ¶ \textcolor{darkblue}{\textbf{\ipa{nv̩˩ɭɯ˧-bi˩bi˩}}} \textcolor{PineGreen}{\selectlanguage{french}cosses de soja} \zh{黄豆荚}  
 \zh{量词}: \textcolor{darkblue}{\textbf{\ipa{kʰwɤ˥}}}  \mytextsc{clf}~: \textcolor{darkblue}{\textbf{\ipa{kʰwɤ˥}}} 
\lhead{\firstmark}
\rhead{\botmark}

\subsection{\hspace{-0.5cm} {\Large \textcolor{darkblue}{\textbf{\ipa{bi˩mi˩}}}}\hspace{0.5cm}[\kern2pt{\textcolor{darkblue}{\textbf{\ipa{bi˧mi˥}}}}\kern2pt]} \hypertarget{bi\string_Bmi\string_B1}{}
\markboth{\textcolor{darkblue}{\textbf{\ipa{bi˩mi˩}}}}{}
\textcolor{teal}{\mytextsc{nom}} \hspace{4pt} Ton~: L.
\textcolor{PineGreen}{\selectlanguage{french}Hache.} \zh{斧头。}  \zh{量词}: \textcolor{darkblue}{\textbf{\ipa{nɑ˧}}}  \mytextsc{clf}~: \textcolor{darkblue}{\textbf{\ipa{nɑ˧}}} 
\lhead{\firstmark}
\rhead{\botmark}

\subsection{\hspace{-0.5cm} {\Large \textcolor{darkblue}{\textbf{\ipa{bi˩pʰv̩˧˥}}}}\hspace{0.5cm}[\kern2pt{\textcolor{darkblue}{\textbf{\ipa{bi˧pʰv̩˧}}}}\kern2pt]} \hypertarget{bi\string_Bp\string_hv\string_=\string_M\string_T1}{}
\markboth{\textcolor{darkblue}{\textbf{\ipa{bi˩pʰv̩˧˥}}}}{}
\textcolor{teal}{\mytextsc{nom}} \hspace{4pt} Ton~: LM+MH\#.
\textcolor{PineGreen}{\selectlanguage{french}Gourde (cucurbitacée); son fruit est la calebasse, qui devient dure en séchant.} \zh{葫芦。} 
\lhead{\firstmark}
\rhead{\botmark}

\subsection{\hspace{-0.5cm} {\Large \textcolor{darkblue}{\textbf{\ipa{bi˩pʰv̩˧-dʑɯ˧˥}}}}\hspace{0.5cm}[\kern2pt{\textcolor{darkblue}{\textbf{\ipa{xxxx non-correspondance entre le nombre de morphèmes et le nombre de tons de morphèmes}}}}\kern2pt]} \hypertarget{bi\string_Bp\string_hv\string_=\string_M-dz£M\string_M\string_T1}{}
\markboth{\textcolor{darkblue}{\textbf{\ipa{bi˩pʰv̩˧-dʑɯ˧˥}}}}{}
\textcolor{teal}{\mytextsc{nom}} \hspace{4pt} Ton~: LM+MH\#.
\textcolor{PineGreen}{\selectlanguage{french}Inondation.} \zh{洪水。} 
\lhead{\firstmark}
\rhead{\botmark}

\subsection{\hspace{-0.5cm} {\Large \textcolor{darkblue}{\textbf{\ipa{bi˩ʁo˥}}}}\hspace{0.5cm}[\kern2pt{\textcolor{darkblue}{\textbf{\ipa{bi˩ʁo˧˥}}}}\kern2pt]} \hypertarget{bi\string_BRo\string_T1}{}
\markboth{\textcolor{darkblue}{\textbf{\ipa{bi˩ʁo˥}}}}{}
\textcolor{teal}{\mytextsc{nom}} \hspace{4pt} Ton~: LH.
\textcolor{PineGreen}{\selectlanguage{french}Bourse. La bourse était autrefois attachée à la ceinture, sur sa face interne.} \zh{钱包(过去:是系在腰带上的)。}  ¶ \textcolor{darkblue}{\textbf{\ipa{bi˩ʁo˧ tʰi˧-ʂæ˧˥}}} \textcolor{PineGreen}{\selectlanguage{french}attacher sa bourse (à sa ceinture)} \zh{系上钱包(在腰带上)}  
 \zh{量词}: \textcolor{darkblue}{\textbf{\ipa{ɭɯ˧}}}  \mytextsc{clf}~: \textcolor{darkblue}{\textbf{\ipa{ɭɯ˧}}} 
\lhead{\firstmark}
\rhead{\botmark}

\subsection{\hspace{-0.5cm} {\Large \textcolor{darkblue}{\textbf{\ipa{bi˩tɑ˩}}}}\hspace{0.5cm}[\kern2pt{\textcolor{darkblue}{\textbf{\ipa{bi˩tɑ˥}}}}\kern2pt]} \hypertarget{bi\string_BtA\string_B1}{}
\markboth{\textcolor{darkblue}{\textbf{\ipa{bi˩tɑ˩}}}}{}
\textcolor{teal}{\mytextsc{verbe}} \hspace{4pt} Ton~: L.
\textcolor{PineGreen}{\selectlanguage{french}Tirer.} \zh{拖。}  ¶ \textcolor{darkblue}{\textbf{\ipa{bi˩tɑ˩-ze˥}}} \textcolor{PineGreen}{\selectlanguage{french}\mytextsc{pfv}} \zh{拖了}  

\lhead{\firstmark}
\rhead{\botmark}

\subsection{\hspace{-0.5cm} {\Large \textcolor{darkblue}{\textbf{\ipa{bi˩-tsɯ˧tsɯ˧}}}}\hspace{0.5cm}[\kern2pt{\textcolor{darkblue}{\textbf{\ipa{xxxx non-correspondance entre le nombre de morphèmes et le nombre de tons de morphèmes}}}}\kern2pt]} \hypertarget{bi\string_B-tsM\string_MtsM\string_M1}{}
\markboth{\textcolor{darkblue}{\textbf{\ipa{bi˩-tsɯ˧tsɯ˧}}}}{}
\textcolor{teal}{\mytextsc{nom}} \hspace{4pt} Ton~: L-.
\textcolor{PineGreen}{\selectlanguage{french}Fraise sauvage, \textit{Fragaria vesca}.} \zh{野草莓。} 
\lhead{\firstmark}
\rhead{\botmark}

\subsection{\hspace{-0.5cm} {\Large \textcolor{darkblue}{\textbf{\ipa{bi˩ʈʂʰɤ\#˥}}}}\hspace{0.5cm}[\kern2pt{\textcolor{darkblue}{\textbf{\ipa{bi˧ʈʂʰɤ˩}}}}\kern2pt]} \hypertarget{bi\string_Bt`s`\string_h7\#\string_T1}{}
\markboth{\textcolor{darkblue}{\textbf{\ipa{bi˩ʈʂʰɤ\#˥}}}}{}
\textcolor{teal}{\mytextsc{nom}} \hspace{4pt} Ton~: LM+\#H.
\textcolor{PineGreen}{\selectlanguage{french}Favoris, rouflaquettes.} \zh{胡须。}  \zh{量词}: \textcolor{darkblue}{\textbf{\ipa{kʰwɤ˥}}}  \mytextsc{clf}~: \textcolor{darkblue}{\textbf{\ipa{kʰwɤ˥}}} 
\lhead{\firstmark}
\rhead{\botmark}

\subsection{\hspace{-0.5cm} {\Large \textcolor{darkblue}{\textbf{\ipa{bi˩wɤ˧}}}}\hspace{0.5cm}[\kern2pt{\textcolor{darkblue}{\textbf{\ipa{xxxx non-correspondance entre le nombre de morphèmes et le nombre de tons de morphèmes}}}}\kern2pt]} \hypertarget{bi\string_Bw7\string_M1}{}
\markboth{\textcolor{darkblue}{\textbf{\ipa{bi˩wɤ˧}}}}{}
\textcolor{teal}{\mytextsc{nom}} \hspace{4pt} Ton~: LM.
\textcolor{PineGreen}{\selectlanguage{french}Services (ou argent) donnés en récompense à un moine.} \zh{酬劳。}  \zh{量词}: \textcolor{darkblue}{\textbf{\ipa{ɭɯ˧}}}  \mytextsc{clf}~: \textcolor{darkblue}{\textbf{\ipa{ɭɯ˧}}} 
\lhead{\firstmark}
\rhead{\botmark}

\subsection{\hspace{-0.5cm} {\Large \textcolor{darkblue}{\textbf{\ipa{bo}}}}\hspace{0.5cm}[\kern2pt{\textcolor{darkblue}{\textbf{\ipa{[]}}}}\kern2pt]} \hypertarget{bo1}{}
\markboth{\textcolor{darkblue}{\textbf{\ipa{bo}}}}{}
\textcolor{teal}{\mytextsc{particule}} \textcolor{teal}{\mytextsc{de}} \textcolor{teal}{\mytextsc{discours}} \hspace{4pt} Ton~: 0.
\textcolor{PineGreen}{\selectlanguage{french}Particule finale empruntée au chinois, exprimant une affirmation vigoureuse: ah que si!} \zh{句尾助词:啵(汉语借词)。}  Emprunt~: chinois  \zh{啵}

\lhead{\firstmark}
\rhead{\botmark}

\subsection{\hspace{-0.5cm} {\Large \textcolor{darkblue}{\textbf{\ipa{bo˧}}} \textsubscript{1}}\hspace{0.5cm}[\kern2pt{\textcolor{darkblue}{\textbf{\ipa{bo˩˥}}}}\kern2pt]} \hypertarget{bo\string_M1}{}
\markboth{\textcolor{darkblue}{\textbf{\ipa{bo˧}}} \textsubscript{1}}{}
\textcolor{teal}{\mytextsc{nom}} \hspace{4pt} Ton~: M.
\textcolor{PineGreen}{\selectlanguage{french}Diguette en bord de champ, sur laquelle on peut marcher.} \zh{田埂、小堤坝。}  ¶ \textcolor{darkblue}{\textbf{\ipa{ʈʂʰɯ˧-qo˧ | bo˧ ɖɯ˧-ɭɯ˧ tʰi˧-di˩.}}} \textcolor{PineGreen}{\selectlanguage{french}ici, il y a une diguette.} \zh{这里有一个小堤坝。}  
 ¶ \textcolor{darkblue}{\textbf{\ipa{bo˧-kʰi˧}}} \textcolor{PineGreen}{\selectlanguage{french}le bord de la diguette} \zh{小堤坝的边沿}  
 ¶ \textcolor{darkblue}{\textbf{\ipa{[F5] bo˧ ɖɯ˧-pʰæ˧˥}}} \textcolor{PineGreen}{\selectlanguage{french}une diguette} \zh{一个小堤坝}  
 \zh{量词}: \textcolor{darkblue}{\textbf{\ipa{ɭɯ˧}}}  \mytextsc{clf}~: \textcolor{darkblue}{\textbf{\ipa{ɭɯ˧}}} \textcolor{PineGreen}{\selectlanguage{french}objets ronds}
\lhead{\firstmark}
\rhead{\botmark}

\subsection{\hspace{-0.5cm} {\Large \textcolor{darkblue}{\textbf{\ipa{bo˧}}} \textsubscript{2}}\hspace{0.5cm}[\kern2pt{\textcolor{darkblue}{\textbf{\ipa{bo˥}}}}\kern2pt]} \hypertarget{bo\string_M2}{}
\markboth{\textcolor{darkblue}{\textbf{\ipa{bo˧}}} \textsubscript{2}}{}
\textcolor{teal}{\mytextsc{adjectif}} \hspace{4pt} Ton~: M.
\textcolor{PineGreen}{\selectlanguage{french}Lumineux.} \zh{亮,光明。}  ¶ \textcolor{darkblue}{\textbf{\ipa{tʰi˧-bo˧-dʑo˧}}} \textcolor{PineGreen}{\selectlanguage{french}ça éclaire, c'est lumineux (définition d'une lampe)} \zh{(它)在发光。(描写灯)}  
 ¶ \textcolor{darkblue}{\textbf{\ipa{bo˧-hĩ˧}}} \textcolor{PineGreen}{\selectlanguage{french}\mytextsc{rel}} \zh{发亮的}  

\lhead{\firstmark}
\rhead{\botmark}

\subsection{\hspace{-0.5cm} {\Large \textcolor{darkblue}{\textbf{\ipa{bo˧\textsubscript{b}}}}}\hspace{0.5cm}[\kern2pt{\textcolor{darkblue}{\textbf{\ipa{bo˩˥}}}}\kern2pt]} \hypertarget{bo\string_Mb1}{}
\markboth{\textcolor{darkblue}{\textbf{\ipa{bo˧\textsubscript{b}}}}}{}
\textcolor{teal}{\mytextsc{verbe}} \hspace{4pt} Ton~: M\textsubscript{b}.
\textcolor{PineGreen}{\selectlanguage{french}Faire tourner (la quenouille, pour filer le fil de lin).} \zh{纺(麻线),使旋转。}  ¶ \textcolor{darkblue}{\textbf{\ipa{tso˧\textasciitilde{}tso˧ bo˧}}} \textcolor{PineGreen}{\selectlanguage{french}faire tourner quelque chose} \zh{使东西旋转}  
\textit{Voir~:} \hyperlink{}{\textcolor{darkblue}{\textbf{\ipa{sɑ˧bo\#˥}}}} 
\lhead{\firstmark}
\rhead{\botmark}

\subsection{\hspace{-0.5cm} {\Large \textcolor{darkblue}{\textbf{\ipa{bo˧tsi˩}}}}\hspace{0.5cm}[\kern2pt{\textcolor{darkblue}{\textbf{\ipa{bo˩tsi˥}}}}\kern2pt]} \hypertarget{bo\string_Mtsi\string_B1}{}
\markboth{\textcolor{darkblue}{\textbf{\ipa{bo˧tsi˩}}}}{}
\textcolor{teal}{\mytextsc{nom}} \hspace{4pt} Ton~: L\#.
\textcolor{PineGreen}{\selectlanguage{french}Crinière.} \zh{(马)鬃。}  ¶ \textcolor{darkblue}{\textbf{\ipa{ʐwæ˧-bo˧tsi˥\#}}} \textcolor{PineGreen}{\selectlanguage{french}crinière du cheval} \zh{马鬃}  
 ¶ \textcolor{darkblue}{\textbf{\ipa{bo˩-bo˧tsi˩}}} \textcolor{PineGreen}{\selectlanguage{french}soies de porc ou de sanglier} \zh{猪鬃}  
 \zh{量词}: \textcolor{darkblue}{\textbf{\ipa{kʰwɤ˥}}} \textcolor{darkblue}{\textbf{\ipa{qɑ˩}}}  \mytextsc{clf}~: \textcolor{darkblue}{\textbf{\ipa{kʰwɤ˥}}} \textcolor{darkblue}{\textbf{\ipa{qɑ˩}}} \textcolor{PineGreen}{\selectlanguage{french}morceaux}
\lhead{\firstmark}
\rhead{\botmark}

\subsection{\hspace{-0.5cm} {\Large \textcolor{darkblue}{\textbf{\ipa{bo˧ʐæ˧}}}}\hspace{0.5cm}[\kern2pt{\textcolor{darkblue}{\textbf{\ipa{bo˩ʐæ˥}}}}\kern2pt]} \hypertarget{bo\string_Mz`\{\string_M1}{}
\markboth{\textcolor{darkblue}{\textbf{\ipa{bo˧ʐæ˧}}}}{}
\textcolor{teal}{\mytextsc{nom}} \hspace{4pt} Ton~: M.
\textcolor{PineGreen}{\selectlanguage{french}Verre (matière).} \zh{玻璃。}  ¶ \textcolor{darkblue}{\textbf{\ipa{bo˧ʐæ˧-tɕʰɤ˩ʈʂv˩}}} \textcolor{PineGreen}{\selectlanguage{french}gobelet à thé en verre} \zh{玻璃茶杯}  

\lhead{\firstmark}
\rhead{\botmark}

\subsection{\hspace{-0.5cm} {\Large \textcolor{darkblue}{\textbf{\ipa{bo˧ʐæ˧ʈæ˧qʰv̩\#˥}}}}\hspace{0.5cm}[\kern2pt{\textcolor{darkblue}{\textbf{\ipa{bo˩ʐæ˧ʈæ˧qʰv̩˧}}}}\kern2pt]} \hypertarget{bo\string_Mz`\{\string_Mt`\{\string_Mq\string_hv\string_=\#\string_T1}{}
\markboth{\textcolor{darkblue}{\textbf{\ipa{bo˧ʐæ˧ʈæ˧qʰv̩\#˥}}}}{}
\textcolor{teal}{\mytextsc{nom}} \hspace{4pt} Ton~: \#H.
\textcolor{PineGreen}{\selectlanguage{french}Écho (en certains endroits des montagnes, il y a un écho).} \zh{回音。}  ¶ \textcolor{darkblue}{\textbf{\ipa{bo˧ʐæ˧ʈæ˧qʰv̩˧ | tʰi˧-ɖʐɯ˩\textasciitilde{}ɖʐɯ˩!}}} \textcolor{PineGreen}{\selectlanguage{french}Il y a de l'écho!} \zh{有回音}  

\lhead{\firstmark}
\rhead{\botmark}

\subsection{\hspace{-0.5cm} {\Large \textcolor{darkblue}{\textbf{\ipa{bo˩\textsubscript{a}}}}}\hspace{0.5cm}[\kern2pt{\textcolor{darkblue}{\textbf{\ipa{bo˥}}}}\kern2pt]} \hypertarget{bo\string_Ba1}{}
\markboth{\textcolor{darkblue}{\textbf{\ipa{bo˩\textsubscript{a}}}}}{}
\textcolor{teal}{\mytextsc{verbe}} \hspace{4pt} Ton~: L\textsubscript{a}.
\textcolor{PineGreen}{\selectlanguage{french}Embrasser.} \zh{亲吻。}  ¶ \textcolor{darkblue}{\textbf{\ipa{ɖɯ˧-bo˩-ɻ̍˩}}} \textcolor{PineGreen}{\selectlanguage{french}\mytextsc{délimitatif} \string_ \mytextsc{inchoatif}} \zh{吻一下}  
 ¶ \textcolor{darkblue}{\textbf{\ipa{ɖɯ˧-bo˧\textasciitilde{}bo˥-ɻ̍˩}}} \textcolor{PineGreen}{\selectlanguage{french}\mytextsc{délimitatif} \string_ \mytextsc{red} \mytextsc{inchoatif}} \zh{吻一下}  

\lhead{\firstmark}
\rhead{\botmark}

\subsection{\hspace{-0.5cm} {\Large \textcolor{darkblue}{\textbf{\ipa{bo˩\textsubscript{b}}}}}\hspace{0.5cm}[\kern2pt{\textcolor{darkblue}{\textbf{\ipa{bo˧˥}}}}\kern2pt]} \hypertarget{bo\string_Bb1}{}
\markboth{\textcolor{darkblue}{\textbf{\ipa{bo˩\textsubscript{b}}}}}{}
\textcolor{teal}{\mytextsc{classificateur}} \hspace{4pt} Ton~: L\textsubscript{b}.
\textcolor{PineGreen}{\selectlanguage{french}Classificateur des coiffes (parures pour la chevelure des femmes).} \zh{量词:缎子发带(一条)。} 
\lhead{\firstmark}
\rhead{\botmark}

\subsection{\hspace{-0.5cm} {\Large \textcolor{darkblue}{\textbf{\ipa{bo˩-bi˧mi˧}}}}\hspace{0.5cm}[\kern2pt{\textcolor{darkblue}{\textbf{\ipa{xxxx non-correspondance entre le nombre de morphèmes et le nombre de tons de morphèmes}}}}\kern2pt]} \hypertarget{bo\string_B-bi\string_Mmi\string_M1}{}
\markboth{\textcolor{darkblue}{\textbf{\ipa{bo˩-bi˧mi˧}}}}{}
\textcolor{teal}{\mytextsc{nom}} \hspace{4pt} Ton~: L-.
\textcolor{PineGreen}{\selectlanguage{french}Viande du ventre du cochon.} \zh{猪肚肉。}  \zh{量词}: \textcolor{darkblue}{\textbf{\ipa{ɭɯ˧}}}  \mytextsc{clf}~: \textcolor{darkblue}{\textbf{\ipa{ɭɯ˧}}} 
\lhead{\firstmark}
\rhead{\botmark}

\subsection{\hspace{-0.5cm} {\Large \textcolor{darkblue}{\textbf{\ipa{bo˩-bv̩˥}}}}\hspace{0.5cm}[\kern2pt{\textcolor{darkblue}{\textbf{\ipa{bo˧bv̩˧}}}}\kern2pt]} \hypertarget{bo\string_B-bv\string_=\string_T1}{}
\markboth{\textcolor{darkblue}{\textbf{\ipa{bo˩-bv̩˥}}}}{}
\textcolor{teal}{\mytextsc{nom}} \hspace{4pt} Ton~: LH.
\textcolor{PineGreen}{\selectlanguage{french}Enclos des porcs.} \zh{猪圈。}  \zh{量词}: \textcolor{darkblue}{\textbf{\ipa{ɭɯ˧}}}  \mytextsc{clf}~: \textcolor{darkblue}{\textbf{\ipa{ɭɯ˧}}} 
\lhead{\firstmark}
\rhead{\botmark}

\subsection{\hspace{-0.5cm} {\Large \textcolor{darkblue}{\textbf{\ipa{bo˩dze˧}}}}\hspace{0.5cm}[\kern2pt{\textcolor{darkblue}{\textbf{\ipa{bo˩dze˥}}}}\kern2pt]} \hypertarget{bo\string_Bdze\string_M1}{}
\markboth{\textcolor{darkblue}{\textbf{\ipa{bo˩dze˧}}}}{}
\textcolor{teal}{\mytextsc{nom}} \hspace{4pt} Ton~: LM.
\textcolor{PineGreen}{\selectlanguage{french}Alouette.} \zh{百灵鸟。} \textit{Voir~:} \hyperlink{}{\textcolor{darkblue}{\textbf{\ipa{bo˩dze˧-ko˩dze˩}}}} 
\lhead{\firstmark}
\rhead{\botmark}

\subsection{\hspace{-0.5cm} {\Large \textcolor{darkblue}{\textbf{\ipa{bo˩dze˧-ko˩dze˩}}}}\hspace{0.5cm}[\kern2pt{\textcolor{darkblue}{\textbf{\ipa{xxxx non-correspondance entre le nombre de morphèmes et le nombre de tons de morphèmes}}}}\kern2pt]} \hypertarget{bo\string_Bdze\string_M-ko\string_Bdze\string_B1}{}
\markboth{\textcolor{darkblue}{\textbf{\ipa{bo˩dze˧-ko˩dze˩}}}}{}
\textcolor{teal}{\mytextsc{nom}} \hspace{4pt} Ton~: LM-L.
\textcolor{PineGreen}{\selectlanguage{french}Alouette.} \zh{百灵鸟。} \textit{Voir~:} \hyperlink{}{\textcolor{darkblue}{\textbf{\ipa{bo˩dze˧}}}} 
\lhead{\firstmark}
\rhead{\botmark}

\subsection{\hspace{-0.5cm} {\Large \textcolor{darkblue}{\textbf{\ipa{bo˩-ɣɯ˥}}}}\hspace{0.5cm}[\kern2pt{\textcolor{darkblue}{\textbf{\ipa{xxxx non-correspondance entre le nombre de morphèmes et le nombre de tons de morphèmes}}}}\kern2pt]} \hypertarget{bo\string_B-GM\string_T1}{}
\markboth{\textcolor{darkblue}{\textbf{\ipa{bo˩-ɣɯ˥}}}}{}
\textcolor{teal}{\mytextsc{nom}} \hspace{4pt} Ton~: LH.
\textcolor{PineGreen}{\selectlanguage{french}Couenne.} \zh{猪皮。}  ¶ \textcolor{darkblue}{\textbf{\ipa{bo˩-ɣɯ˧kɯ˩}}} \textcolor{PineGreen}{\selectlanguage{french}même sens: couenne} \zh{同上:猪皮}  
 \zh{量词}: \textcolor{darkblue}{\textbf{\ipa{tsʰi˥}}}  \mytextsc{clf}~: \textcolor{darkblue}{\textbf{\ipa{tsʰi˥}}} 
\lhead{\firstmark}
\rhead{\botmark}

\subsection{\hspace{-0.5cm} {\Large \textcolor{darkblue}{\textbf{\ipa{bo˩-hɑ\#˥}}}}\hspace{0.5cm}[\kern2pt{\textcolor{darkblue}{\textbf{\ipa{xxxx non-correspondance entre le nombre de morphèmes et le nombre de tons de morphèmes}}}}\kern2pt]} \hypertarget{bo\string_B-hA\#\string_T1}{}
\markboth{\textcolor{darkblue}{\textbf{\ipa{bo˩-hɑ\#˥}}}}{}
\textcolor{teal}{\mytextsc{nom}} \hspace{4pt} Ton~: LM+\#H.
\textcolor{PineGreen}{\selectlanguage{french}Nourriture du cochon/ pâtée des porcs.} \zh{猪食。}  \zh{量词}: \textcolor{darkblue}{\textbf{\ipa{kʰɤ˧˥}}}  \mytextsc{clf}~: \textcolor{darkblue}{\textbf{\ipa{kʰɤ˧˥}}} 
\lhead{\firstmark}
\rhead{\botmark}

\subsection{\hspace{-0.5cm} {\Large \textcolor{darkblue}{\textbf{\ipa{bo˩-kʰɯ˧}}}}\hspace{0.5cm}[\kern2pt{\textcolor{darkblue}{\textbf{\ipa{xxxx non-correspondance entre le nombre de morphèmes et le nombre de tons de morphèmes}}}}\kern2pt]} \hypertarget{bo\string_B-k\string_hM\string_M1}{}
\markboth{\textcolor{darkblue}{\textbf{\ipa{bo˩-kʰɯ˧}}}}{}
\textcolor{teal}{\mytextsc{nom}} \hspace{4pt} Ton~: LM.
\textcolor{PineGreen}{\selectlanguage{french}Pieds de porc (pièce de boucherie): viande séchée conservée dans la peau du pied de cochon.} \zh{猪脚腊肉:把猪腿的皮剥下来,缝成筒形,塞满瘦肉。}  \zh{量词}: \textcolor{darkblue}{\textbf{\ipa{ɭɯ˧}}}  \mytextsc{clf}~: \textcolor{darkblue}{\textbf{\ipa{ɭɯ˧}}} 
\lhead{\firstmark}
\rhead{\botmark}

\subsection{\hspace{-0.5cm} {\Large \textcolor{darkblue}{\textbf{\ipa{bo˩-kʰv̩˧˥}}}}\hspace{0.5cm}[\kern2pt{\textcolor{darkblue}{\textbf{\ipa{xxxx non-correspondance entre le nombre de morphèmes et le nombre de tons de morphèmes}}}}\kern2pt]} \hypertarget{bo\string_B-k\string_hv\string_=\string_M\string_T1}{}
\markboth{\textcolor{darkblue}{\textbf{\ipa{bo˩-kʰv̩˧˥}}}}{}
\textcolor{teal}{\mytextsc{nom}} \hspace{4pt} Ton~: LM+MH\#.
\textcolor{PineGreen}{\selectlanguage{french}Année du porc.} \zh{猪年。} 
\lhead{\firstmark}
\rhead{\botmark}

\subsection{\hspace{-0.5cm} {\Large \textcolor{darkblue}{\textbf{\ipa{bo˩lo˧}}}}\hspace{0.5cm}[\kern2pt{\textcolor{darkblue}{\textbf{\ipa{bo˩lo˥}}}}\kern2pt]} \hypertarget{bo\string_Blo\string_M1}{}
\markboth{\textcolor{darkblue}{\textbf{\ipa{bo˩lo˧}}}}{}
\textcolor{teal}{\mytextsc{nom}} \hspace{4pt} Ton~: LM.
\textcolor{PineGreen}{\selectlanguage{french}Mortaise.} \zh{榫眼。}  ¶ \textcolor{darkblue}{\textbf{\ipa{bo˩lo˧ | ɖɯ˧-ɭɯ˧}}} \textcolor{PineGreen}{\selectlanguage{french}une mortaise} \zh{一个榫眼}  
 \zh{量词}: \textcolor{darkblue}{\textbf{\ipa{ɭɯ˧}}}  \mytextsc{clf}~: \textcolor{darkblue}{\textbf{\ipa{ɭɯ˧}}} 
\lhead{\firstmark}
\rhead{\botmark}

\subsection{\hspace{-0.5cm} {\Large \textcolor{darkblue}{\textbf{\ipa{bo˩ɬɑ˥}}}}\hspace{0.5cm}[\kern2pt{\textcolor{darkblue}{\textbf{\ipa{bo˩ɬɑ˧˥}}}}\kern2pt]} \hypertarget{bo\string_BKA\string_T1}{}
\markboth{\textcolor{darkblue}{\textbf{\ipa{bo˩ɬɑ˥}}}}{}
\textcolor{teal}{\mytextsc{nom}} \hspace{4pt} Ton~: LH.
\textcolor{PineGreen}{\selectlanguage{french}Verrat.} \zh{种公猪。}  \zh{量词}: \textcolor{darkblue}{\textbf{\ipa{v̩˧}}}  \mytextsc{clf}~: \textcolor{darkblue}{\textbf{\ipa{v̩˧}}} 
\lhead{\firstmark}
\rhead{\botmark}

\subsection{\hspace{-0.5cm} {\Large \textcolor{darkblue}{\textbf{\ipa{bo˩-ɬo˥}}}}\hspace{0.5cm}[\kern2pt{\textcolor{darkblue}{\textbf{\ipa{xxxx non-correspondance entre le nombre de morphèmes et le nombre de tons de morphèmes}}}}\kern2pt]} \hypertarget{bo\string_B-Ko\string_T1}{}
\markboth{\textcolor{darkblue}{\textbf{\ipa{bo˩-ɬo˥}}}}{}
\textcolor{teal}{\mytextsc{nom}} \hspace{4pt} Ton~: LH.
\textcolor{PineGreen}{\selectlanguage{french}Côtes de porc.} \zh{猪肋骨。}  ¶ \textcolor{darkblue}{\textbf{\ipa{bo˩ɬo˥ | ɖɯ˧-do˥}}} \textcolor{PineGreen}{\selectlanguage{french}Un quartier de côtes de porc; désigne aussi, métaphoriquement, une famille, dans laquelle chaque individu est solidaire des autres} \zh{一大块猪肋骨。也来比喻一个家庭,强调家所有成员之间的密切关系。}  
 \zh{量词}: \textcolor{darkblue}{\textbf{\ipa{ɭɯ˧}}}  \mytextsc{clf}~: \textcolor{darkblue}{\textbf{\ipa{ɭɯ˧}}} 
\lhead{\firstmark}
\rhead{\botmark}

\subsection{\hspace{-0.5cm} {\Large \textcolor{darkblue}{\textbf{\ipa{bo˩-mæ˧qv̩˩}}}}\hspace{0.5cm}[\kern2pt{\textcolor{darkblue}{\textbf{\ipa{xxxx non-correspondance entre le nombre de morphèmes et le nombre de tons de morphèmes}}}}\kern2pt]} \hypertarget{bo\string_B-m\{\string_Mqv\string_=\string_B1}{}
\markboth{\textcolor{darkblue}{\textbf{\ipa{bo˩-mæ˧qv̩˩}}}}{}
\textcolor{teal}{\mytextsc{nom}} \hspace{4pt} Ton~: L-L\#.
\textcolor{PineGreen}{\selectlanguage{french}Queue du cochon.} \zh{猪尾巴。}  \zh{量词}: \textcolor{darkblue}{\textbf{\ipa{ɭɯ˧}}}  \mytextsc{clf}~: \textcolor{darkblue}{\textbf{\ipa{ɭɯ˧}}} 
\lhead{\firstmark}
\rhead{\botmark}

\subsection{\hspace{-0.5cm} {\Large \textcolor{darkblue}{\textbf{\ipa{bo˩-mɤ˥}}}}\hspace{0.5cm}[\kern2pt{\textcolor{darkblue}{\textbf{\ipa{xxxx non-correspondance entre le nombre de morphèmes et le nombre de tons de morphèmes}}}}\kern2pt]} \hypertarget{bo\string_B-m7\string_T1}{}
\markboth{\textcolor{darkblue}{\textbf{\ipa{bo˩-mɤ˥}}}}{}
\textcolor{teal}{\mytextsc{nom}} \hspace{4pt} Ton~: LH.
\textcolor{PineGreen}{\selectlanguage{french}Saindoux (gras de porc).} \zh{猪油。} 
\lhead{\firstmark}
\rhead{\botmark}

\subsection{\hspace{-0.5cm} {\Large \textcolor{darkblue}{\textbf{\ipa{bo˩mi˧}}}}\hspace{0.5cm}[\kern2pt{\textcolor{darkblue}{\textbf{\ipa{xxxx non-correspondance entre le nombre de morphèmes et le nombre de tons de morphèmes}}}}\kern2pt]} \hypertarget{bo\string_Bmi\string_M1}{}
\markboth{\textcolor{darkblue}{\textbf{\ipa{bo˩mi˧}}}}{}
\textcolor{teal}{\mytextsc{nom}} \hspace{4pt} Ton~: LM.
\textcolor{PineGreen}{\selectlanguage{french}Truie.} \zh{母猪。}  ¶ \textcolor{darkblue}{\textbf{\ipa{bo˩mi˧ ʑi˩}}} \textcolor{PineGreen}{\selectlanguage{french}attraper une truie} \zh{抓母猪}  
 ¶ \textcolor{darkblue}{\textbf{\ipa{bo˩mi˧ do˧ (+ze˩)}}} \textcolor{PineGreen}{\selectlanguage{french}...a vu (une/la) truie} \zh{见了母猪}  
 ¶ \textcolor{darkblue}{\textbf{\ipa{bo˩mi˧-bæ˧bv̩˥}}} \textcolor{PineGreen}{\selectlanguage{french}truie et porcelets} \zh{母猪与猪仔}  
 \zh{量词}: \textcolor{darkblue}{\textbf{\ipa{mi˩}}} \textcolor{darkblue}{\textbf{\ipa{v̩˧}}}  \mytextsc{clf}~: \textcolor{darkblue}{\textbf{\ipa{mi˩}}} \textcolor{darkblue}{\textbf{\ipa{v̩˧}}} 
\lhead{\firstmark}
\rhead{\botmark}

\subsection{\hspace{-0.5cm} {\Large \textcolor{darkblue}{\textbf{\ipa{bo˩mi˧-dʑɯ˩pv̩˩}}}}\hspace{0.5cm}[\kern2pt{\textcolor{darkblue}{\textbf{\ipa{xxxx non-correspondance entre le nombre de morphèmes et le nombre de tons de morphèmes}}}}\kern2pt]} \hypertarget{bo\string_Bmi\string_M-dz£M\string_Bpv\string_=\string_B1}{}
\markboth{\textcolor{darkblue}{\textbf{\ipa{bo˩mi˧-dʑɯ˩pv̩˩}}}}{}
\textcolor{teal}{\mytextsc{nom}} \hspace{4pt} Ton~: LM-L.
\textcolor{PineGreen}{\selectlanguage{french}Dytique, \textit{(Dytiscus}.} \zh{龙虱。} 
\lhead{\firstmark}
\rhead{\botmark}

\subsection{\hspace{-0.5cm} {\Large \textcolor{darkblue}{\textbf{\ipa{bo˩mi˧-dʑɯ˩pʰv̩˩}}}}\hspace{0.5cm}[\kern2pt{\textcolor{darkblue}{\textbf{\ipa{bo˩mi˧dʑɯ˩pʰv̩˩}}}}\kern2pt]} \hypertarget{bo\string_Bmi\string_M-dz£M\string_Bp\string_hv\string_=\string_B1}{}
\markboth{\textcolor{darkblue}{\textbf{\ipa{bo˩mi˧-dʑɯ˩pʰv̩˩}}}}{}
\textcolor{teal}{\mytextsc{nom}} \hspace{4pt} Ton~: LM-L.
\textcolor{PineGreen}{\selectlanguage{french}Charançon.} \zh{象鼻虫,米象。} 
\lhead{\firstmark}
\rhead{\botmark}

\subsection{\hspace{-0.5cm} {\Large \textcolor{darkblue}{\textbf{\ipa{bo˩mi˧-ɳæ˧tɕʰɯ˩}}}}\hspace{0.5cm}[\kern2pt{\textcolor{darkblue}{\textbf{\ipa{bo˩mi˧ɳæ˩tɕʰɯ˩}}}}\kern2pt]} \hypertarget{bo\string_Bmi\string_M-n`\{\string_Mts£\string_hM\string_B1}{}
\markboth{\textcolor{darkblue}{\textbf{\ipa{bo˩mi˧-ɳæ˧tɕʰɯ˩}}}}{}
\textcolor{teal}{\mytextsc{nom}} \hspace{4pt} Ton~: LM-L\#.
\textcolor{PineGreen}{\selectlanguage{french}Pissenlit.} \zh{蒲公英。}  \zh{量词}: \textcolor{darkblue}{\textbf{\ipa{po˧}}}  \mytextsc{clf}~: \textcolor{darkblue}{\textbf{\ipa{po˧}}} 
\lhead{\firstmark}
\rhead{\botmark}

\subsection{\hspace{-0.5cm} {\Large \textcolor{darkblue}{\textbf{\ipa{bo˩mi˧-ʁo˩do˩}}}}\hspace{0.5cm}[\kern2pt{\textcolor{darkblue}{\textbf{\ipa{bo˩mi˧ʁo˧do˩}}}}\kern2pt]} \hypertarget{bo\string_Bmi\string_M-Ro\string_Bdo\string_B1}{}
\markboth{\textcolor{darkblue}{\textbf{\ipa{bo˩mi˧-ʁo˩do˩}}}}{}
\textcolor{teal}{\mytextsc{nom}} \hspace{4pt} Ton~: LM-L.
\textcolor{PineGreen}{\selectlanguage{french}Yyyy littéralement “noix des truies”; était employé autrefois pour application sur les plaies qui suppuraient: on en extrayait le jus.} \zh{曼陀罗。}  \zh{量词}: \textcolor{darkblue}{\textbf{\ipa{dzi˩}}}  \mytextsc{clf}~: \textcolor{darkblue}{\textbf{\ipa{dzi˩}}} 
\lhead{\firstmark}
\rhead{\botmark}

\subsection{\hspace{-0.5cm} {\Large \textcolor{darkblue}{\textbf{\ipa{bo˩pʰv̩˧}}}}\hspace{0.5cm}[\kern2pt{\textcolor{darkblue}{\textbf{\ipa{bo˩pʰv̩˥}}}}\kern2pt]} \hypertarget{bo\string_Bp\string_hv\string_=\string_M1}{}
\markboth{\textcolor{darkblue}{\textbf{\ipa{bo˩pʰv̩˧}}}}{}
\textcolor{teal}{\mytextsc{nom}} \hspace{4pt} Ton~: LM.
\textcolor{PineGreen}{\selectlanguage{french}Verrat.} \zh{种公猪。}  \zh{量词}: \textcolor{darkblue}{\textbf{\ipa{v̩˧}}}  \mytextsc{clf}~: \textcolor{darkblue}{\textbf{\ipa{v̩˧}}} 
\lhead{\firstmark}
\rhead{\botmark}

\subsection{\hspace{-0.5cm} {\Large \textcolor{darkblue}{\textbf{\ipa{bo˩qʰæ˧-pv̩˧ʈɤ˥-ɻ̍˩}}}}\hspace{0.5cm}[\kern2pt{\textcolor{darkblue}{\textbf{\ipa{xxxx non-correspondance entre le nombre de morphèmes et le nombre de tons de morphèmes}}}}\kern2pt]} \hypertarget{bo\string_Bq\string_h\{\string_M-pv\string_=\string_Mt`7\string_T-r£`̍\string_B1}{}
\markboth{\textcolor{darkblue}{\textbf{\ipa{bo˩qʰæ˧-pv̩˧ʈɤ˥-ɻ̍˩}}}}{}
\textcolor{teal}{\mytextsc{nom}} \hspace{4pt} Ton~: LM - H\# -.
\textcolor{PineGreen}{\selectlanguage{french}Bousier: sorte de scarabée, qui prolifère dans les étables lorsqu'il fait chaud.} \zh{蜣螂。} 
\lhead{\firstmark}
\rhead{\botmark}

\subsection{\hspace{-0.5cm} {\Large \textcolor{darkblue}{\textbf{\ipa{bo˩tv̩\#˥}}}}\hspace{0.5cm}[\kern2pt{\textcolor{darkblue}{\textbf{\ipa{bo˧tv̩˩}}}}\kern2pt]} \hypertarget{bo\string_Btv\string_=\#\string_T1}{}
\markboth{\textcolor{darkblue}{\textbf{\ipa{bo˩tv̩\#˥}}}}{}
\textcolor{teal}{\mytextsc{nom}} \hspace{4pt} Ton~: LM+\#H.
\textcolor{PineGreen}{\selectlanguage{french}Sanglier.} \zh{野猪。}  ¶ \textcolor{darkblue}{\textbf{\ipa{bo˩tv̩˧ hwæ˥}}} \textcolor{PineGreen}{\selectlanguage{french}acheter un sanglier} \zh{买野猪}  
 \zh{量词}: \textcolor{darkblue}{\textbf{\ipa{mi˩}}}  \mytextsc{clf}~: \textcolor{darkblue}{\textbf{\ipa{mi˩}}} 
\lhead{\firstmark}
\rhead{\botmark}

\subsection{\hspace{-0.5cm} {\Large \textcolor{darkblue}{\textbf{\ipa{bo˩ʈʂʰæ˥}}}}\hspace{0.5cm}[\kern2pt{\textcolor{darkblue}{\textbf{\ipa{bo˩ʈʂʰæ˩˥}}}}\kern2pt]} \hypertarget{bo\string_Bt`s`\string_h\{\string_T1}{}
\markboth{\textcolor{darkblue}{\textbf{\ipa{bo˩ʈʂʰæ˥}}}}{}
\textcolor{teal}{\mytextsc{nom}} \hspace{4pt} Ton~: LH.
\textcolor{PineGreen}{\selectlanguage{french}Lard; le même terme est employé pour désigner le cochon entier désossé et conservé dans sa peau (au moyen de salpêtre et de sel), qui se conserve une décennie, appelé “viande pipa” en chinois. La glose adoptée dans les textes est: cochon-conservé-entier.} \zh{猪膘,琵琶肉。} 
\lhead{\firstmark}
\rhead{\botmark}

\subsection{\hspace{-0.5cm} {\Large \textcolor{darkblue}{\textbf{\ipa{bo˩zɑ˧mi\#˥}}}}\hspace{0.5cm}[\kern2pt{\textcolor{darkblue}{\textbf{\ipa{bo˧zɑ˧mi˧}}}}\kern2pt]} \hypertarget{bo\string_BzA\string_Mmi\#\string_T1}{}
\markboth{\textcolor{darkblue}{\textbf{\ipa{bo˩zɑ˧mi\#˥}}}}{}
\textcolor{teal}{\mytextsc{nom}} \hspace{4pt} Ton~: LM+\#H.
\textcolor{PineGreen}{\selectlanguage{french}“Petite Truie”: nom employé pour les petites filles pendant leurs premiers mois, avant qu'on leur donne un vrai nom. Le vilain nom dont on l'affuble vise à éviter que le nourrisson ne soit repéré par de mauvais esprits. (Actuellement, l'état-civil nécessite qu'un nom soit donné dès la naissance; mais celui-ci ne commence à être employé dans les conversations familiales qu'une fois passés les premiers mois.).} \zh{猪崽子(给刚出生的女孩起的名字,让鬼对她不感兴趣,不会来害小孩)。} 
\lhead{\firstmark}
\rhead{\botmark}

\subsection{\hspace{-0.5cm} {\Large \textcolor{darkblue}{\textbf{\ipa{bo˩˧}}}}\hspace{0.5cm}[\kern2pt{\textcolor{darkblue}{\textbf{\ipa{xxxx groupe tonal entier sans aucun ton}}}}\kern2pt]} \hypertarget{bo\string_B\string_M1}{}
\markboth{\textcolor{darkblue}{\textbf{\ipa{bo˩˧}}}}{}
\textcolor{teal}{\mytextsc{nom}} \hspace{4pt} Ton~: LM.
\textcolor{PineGreen}{\selectlanguage{french}Porc, cochon.} \zh{猪。}  ¶ \textcolor{darkblue}{\textbf{\ipa{bo˩ hwæ˧-ze˧}}} \textcolor{PineGreen}{\selectlanguage{french}...a acheté (du/un) porc} \zh{买了猪}  
 ¶ \textcolor{darkblue}{\textbf{\ipa{bo˩ dzɯ˥-ze˩}}} \textcolor{PineGreen}{\selectlanguage{french}...a mangé (du/un) porc} \zh{吃了猪}  
 \zh{量词}: \textcolor{darkblue}{\textbf{\ipa{mi˩}}} \textcolor{darkblue}{\textbf{\ipa{v̩˧}}}  \mytextsc{clf}~: \textcolor{darkblue}{\textbf{\ipa{mi˩}}} \textcolor{darkblue}{\textbf{\ipa{v̩˧}}} 
\lhead{\firstmark}
\rhead{\botmark}

\subsection{\hspace{-0.5cm} {\Large \textcolor{darkblue}{\textbf{\ipa{bõ}}}}\hspace{0.5cm}[\kern2pt{\textcolor{darkblue}{\textbf{\ipa{xxxx groupe tonal entier sans aucun ton}}}}\kern2pt]} \hypertarget{bo\string_~1}{}
\markboth{\textcolor{darkblue}{\textbf{\ipa{bõ}}}}{}
\textcolor{teal}{\mytextsc{idéophone}} \hspace{4pt} Ton~: 0.
\textcolor{PineGreen}{\selectlanguage{french}Bruit d'un choc entre deux objets durs, par exemple un coup de hache sur un tronc: Boum!} \zh{形声词:斧头砍树。砰! / 啪!。} 
\lhead{\firstmark}
\rhead{\botmark}

\subsection{\hspace{-0.5cm} {\Large \textcolor{darkblue}{\textbf{\ipa{bv̩˩}}}}\hspace{0.5cm}[\kern2pt{\textcolor{darkblue}{\textbf{\ipa{bv̩˥}}}}\kern2pt]} \hypertarget{bv\string_=\string_B1}{}
\markboth{\textcolor{darkblue}{\textbf{\ipa{bv̩˩}}}}{}
\textcolor{teal}{\mytextsc{nom}} \hspace{4pt} Ton~: L.
\textcolor{PineGreen}{\selectlanguage{french}Enclos (monosyllabe).} \zh{牲畜圈(单音节)。}  ¶ \textcolor{darkblue}{\textbf{\ipa{bv̩˩-qo˩}}} \textcolor{PineGreen}{\selectlanguage{french}dans l'enclos} \zh{牲畜圈里面}  
 ¶ \textcolor{darkblue}{\textbf{\ipa{bv̩˩qo˩ ʈæ˧}}} \textcolor{PineGreen}{\selectlanguage{french}enfermer dans l'étable} \zh{关在圈里}  
 \zh{量词}: \textcolor{darkblue}{\textbf{\ipa{ɭɯ˧}}}  \mytextsc{clf}~: \textcolor{darkblue}{\textbf{\ipa{ɭɯ˧}}} 
\lhead{\firstmark}
\rhead{\botmark}

\subsection{\hspace{-0.5cm} {\Large \textcolor{darkblue}{\textbf{\ipa{bv̩˩˧}}} \textsubscript{1}}\hspace{0.5cm}[\kern2pt{\textcolor{darkblue}{\textbf{\ipa{bv̩˥}}}}\kern2pt]} \hypertarget{bv\string_=\string_B\string_M1}{}
\markboth{\textcolor{darkblue}{\textbf{\ipa{bv̩˩˧}}} \textsubscript{1}}{}
\textcolor{teal}{\mytextsc{nom}} \hspace{4pt} Ton~: LM.
\textcolor{PineGreen}{\selectlanguage{french}Yak, Bos grunniens (sauvage ou domestiqué).} \zh{牦牛/野牦牛。}  ¶ \textcolor{darkblue}{\textbf{\ipa{bv̩˩-hṽ˩˥}}} \textcolor{PineGreen}{\selectlanguage{french}poil de yak} \zh{牦牛毛}  
 ¶ \textcolor{darkblue}{\textbf{\ipa{bv̩˩ dzɯ˧-ze˩}}} \textcolor{PineGreen}{\selectlanguage{french}...a mangé (du) yak} \zh{吃了牦牛}  
 ¶ \textcolor{darkblue}{\textbf{\ipa{bv̩˩ hwæ˧-ze˧}}} \textcolor{PineGreen}{\selectlanguage{french}...a acheté (du) yak} \zh{买了牦牛}  
 \zh{量词}: \textcolor{darkblue}{\textbf{\ipa{pʰo˧˥}}}  \mytextsc{clf}~: \textcolor{darkblue}{\textbf{\ipa{pʰo˧˥}}} 
\lhead{\firstmark}
\rhead{\botmark}

\subsection{\hspace{-0.5cm} {\Large \textcolor{darkblue}{\textbf{\ipa{bv̩˩˧}}} \textsubscript{2}}\hspace{0.5cm}[\kern2pt{\textcolor{darkblue}{\textbf{\ipa{bv̩˩˥}}}}\kern2pt]} \hypertarget{bv\string_=\string_B\string_M2}{}
\markboth{\textcolor{darkblue}{\textbf{\ipa{bv̩˩˧}}} \textsubscript{2}}{}
\textcolor{teal}{\mytextsc{nom}} \hspace{4pt} Ton~: LM.
\textcolor{PineGreen}{\selectlanguage{french}Étuve.} \zh{蒸笼。}  \zh{量词}: \textcolor{darkblue}{\textbf{\ipa{mi˩}}}  \mytextsc{clf}~: \textcolor{darkblue}{\textbf{\ipa{mi˩}}} \textit{Voir~:} \hyperlink{}{\textcolor{darkblue}{\textbf{\ipa{bv̩˩di˩}}}} 
\lhead{\firstmark}
\rhead{\botmark}

\subsection{\hspace{-0.5cm} {\Large \textcolor{darkblue}{\textbf{\ipa{bv̩˧}}}}\hspace{0.5cm}[\kern2pt{\textcolor{darkblue}{\textbf{\ipa{bv̩˩˥}}}}\kern2pt]} \hypertarget{bv\string_=\string_M1}{}
\markboth{\textcolor{darkblue}{\textbf{\ipa{bv̩˧}}}}{}
\textcolor{teal}{\mytextsc{nom}} \hspace{4pt} Ton~: M.
\textcolor{PineGreen}{\selectlanguage{french}Intestin.} \zh{肠子。}  \zh{量词}: \textcolor{darkblue}{\textbf{\ipa{kʰɯ˩}}}  \mytextsc{clf}~: \textcolor{darkblue}{\textbf{\ipa{kʰɯ˩}}} 
\lhead{\firstmark}
\rhead{\botmark}

\subsection{\hspace{-0.5cm} {\Large \textcolor{darkblue}{\textbf{\ipa{bv̩˥}}}}\hspace{0.5cm}[\kern2pt{\textcolor{darkblue}{\textbf{\ipa{bv̩˥}}}}\kern2pt]} \hypertarget{bv\string_=\string_T1}{}
\markboth{\textcolor{darkblue}{\textbf{\ipa{bv̩˥}}}}{}
\textcolor{teal}{\mytextsc{nom}} \hspace{4pt} Ton~: \#H.
\textcolor{PineGreen}{\selectlanguage{french}Ver; insecte.} \zh{虫。}  ¶ \textcolor{darkblue}{\textbf{\ipa{bv̩˧ tʰv̩˧-mi˧˥ / bv̩˧ tʰv̩˧-mi˥\#}}} \textcolor{PineGreen}{\selectlanguage{french}\mytextsc{n}+\mytextsc{dem}+\mytextsc{clf}} \zh{这只虫}  
 \zh{量词}: \textcolor{darkblue}{\textbf{\ipa{mi˩}}}  \mytextsc{clf}~: \textcolor{darkblue}{\textbf{\ipa{mi˩}}} 
\lhead{\firstmark}
\rhead{\botmark}

\subsection{\hspace{-0.5cm} {\Large \textcolor{darkblue}{\textbf{\ipa{bv̩˥}}} \textsubscript{1}}\hspace{0.5cm}[\kern2pt{\textcolor{darkblue}{\textbf{\ipa{bv̩˥}}}}\kern2pt]} \hypertarget{bv\string_=\string_T1}{}
\markboth{\textcolor{darkblue}{\textbf{\ipa{bv̩˥}}} \textsubscript{1}}{}
\textcolor{teal}{\mytextsc{adjectif}} \hspace{4pt} Ton~: H.
\textcolor{PineGreen}{\selectlanguage{french}Épais (tronc); grossier (farine, poudre).} \zh{粗(树粗大,粉末不精细……)。}  ¶ \textcolor{darkblue}{\textbf{\ipa{qʰɑ˧-bv̩˧-gv̩˧}}} \textcolor{PineGreen}{\selectlanguage{french}très épais} \zh{很粗}  
 ¶ \textcolor{darkblue}{\textbf{\ipa{qʰɑ˧bv̩˧\textasciitilde{}bv̩˧-gv̩˧}}} \textcolor{PineGreen}{\selectlanguage{french}très épais (idem ci-dessus)} \zh{很粗(同上)}  

\lhead{\firstmark}
\rhead{\botmark}

\subsection{\hspace{-0.5cm} {\Large \textcolor{darkblue}{\textbf{\ipa{bv̩˥}}} \textsubscript{2}}\hspace{0.5cm}[\kern2pt{\textcolor{darkblue}{\textbf{\ipa{bv̩˥}}}}\kern2pt]} \hypertarget{bv\string_=\string_T2}{}
\markboth{\textcolor{darkblue}{\textbf{\ipa{bv̩˥}}} \textsubscript{2}}{}
\textcolor{teal}{\mytextsc{verbe}} \hspace{4pt} Ton~: H.
\textcolor{PineGreen}{\selectlanguage{french}Partager, distribuer, répartir, diviser (ancien mot pour “donner”).} \zh{分东西。}  ¶ \textcolor{darkblue}{\textbf{\ipa{ɖɯ˧-v̩˧ ɖɯ˧-kʰwɤ˥ | le˧-bv̩˧\textasciitilde{}bv̩˧}}} \textcolor{PineGreen}{\selectlanguage{french}partager, un morceau par personne} \zh{分给一人一块}  
 ¶ \textcolor{darkblue}{\textbf{\ipa{le˧-bv̩˧\textasciitilde{}bv̩˧ tʰi˧-kwɤ˩}}} \textcolor{PineGreen}{\selectlanguage{french}littéralement “séparer et poser”; sens: mettre en désordre, disposer en désordre} \zh{弄乱,散开}  

\lhead{\firstmark}
\rhead{\botmark}

\subsection{\hspace{-0.5cm} {\Large \textcolor{darkblue}{\textbf{\ipa{bv̩˥}}} \textsubscript{3}}\hspace{0.5cm}[\kern2pt{\textcolor{darkblue}{\textbf{\ipa{bv̩˥}}}}\kern2pt]} \hypertarget{bv\string_=\string_T3}{}
\markboth{\textcolor{darkblue}{\textbf{\ipa{bv̩˥}}} \textsubscript{3}}{}
\textcolor{teal}{\mytextsc{verbe}} \hspace{4pt} Ton~: H.
\textcolor{PineGreen}{\selectlanguage{french}Griller, faire griller.} \zh{烤,炙。}  ¶ \textcolor{darkblue}{\textbf{\ipa{hɑ˧ tʰi˧-bv̩˥}}} \textcolor{PineGreen}{\selectlanguage{french}faire griller du riz} \zh{烤饭}  

\lhead{\firstmark}
\rhead{\botmark}

\subsection{\hspace{-0.5cm} {\Large \textcolor{darkblue}{\textbf{\ipa{bv̩˩\textsubscript{a}}}} \textsubscript{1}}\hspace{0.5cm}[\kern2pt{\textcolor{darkblue}{\textbf{\ipa{bv̩˥}}}}\kern2pt]} \hypertarget{bv\string_=\string_Ba1}{}
\markboth{\textcolor{darkblue}{\textbf{\ipa{bv̩˩\textsubscript{a}}}} \textsubscript{1}}{}
\textcolor{teal}{\mytextsc{verbe}} \hspace{4pt} Ton~: L\textsubscript{a}.
\textcolor{PineGreen}{\selectlanguage{french}Couver.} \zh{孵。}  ¶ \textcolor{darkblue}{\textbf{\ipa{æ˩mi˧ bv̩˩}}} \textcolor{PineGreen}{\selectlanguage{french}La poule couve.} \zh{母鸡孵蛋}  

\lhead{\firstmark}
\rhead{\botmark}

\subsection{\hspace{-0.5cm} {\Large \textcolor{darkblue}{\textbf{\ipa{bv̩˩\textsubscript{a}}}} \textsubscript{2}}\hspace{0.5cm}[\kern2pt{\textcolor{darkblue}{\textbf{\ipa{bv̩˩˥}}}}\kern2pt]} \hypertarget{bv\string_=\string_Ba2}{}
\markboth{\textcolor{darkblue}{\textbf{\ipa{bv̩˩\textsubscript{a}}}} \textsubscript{2}}{}
\textcolor{teal}{\mytextsc{verbe}} \hspace{4pt} Ton~: L\textsubscript{a}.
\textcolor{PineGreen}{\selectlanguage{french}Cuire à la vapeur, étuver.} \zh{蒸。}  ¶ \textcolor{darkblue}{\textbf{\ipa{le˧-bv̩˩-ze˩}}} \textcolor{PineGreen}{\selectlanguage{french}\mytextsc{accomp} \string_ \mytextsc{pfv}} \zh{蒸了}  
 ¶ \textcolor{darkblue}{\textbf{\ipa{pɤ˩jɤ˧ bv̩˥}}} \textcolor{PineGreen}{\selectlanguage{french}cuire de la pâte à pain à la vapeur, faire des petits pains à la vapeur} \zh{蒸馒头}  
 ¶ \textcolor{darkblue}{\textbf{\ipa{hɑ˧ bv̩˥\textasciitilde{}bv̩˩}}} \textcolor{PineGreen}{\selectlanguage{french}cuire du riz à la vapeur} \zh{蒸米饭}  

\lhead{\firstmark}
\rhead{\botmark}

\subsection{\hspace{-0.5cm} {\Large \textcolor{darkblue}{\textbf{\ipa{bv̩˩\textsubscript{a}}}} \textsubscript{3}}\hspace{0.5cm}[\kern2pt{\textcolor{darkblue}{\textbf{\ipa{bv̩˩˥}}}}\kern2pt]} \hypertarget{bv\string_=\string_Ba3}{}
\markboth{\textcolor{darkblue}{\textbf{\ipa{bv̩˩\textsubscript{a}}}} \textsubscript{3}}{}
\textcolor{teal}{\mytextsc{verbe}} \hspace{4pt} Ton~: L\textsubscript{a}.
\textcolor{PineGreen}{\selectlanguage{french}Vivre, couler des jours, vivre (sa vie).} \zh{过(日子)。}  ¶ \textcolor{darkblue}{\textbf{\ipa{zɯ˧ bv̩˩}}} \textcolor{PineGreen}{\selectlanguage{french}vivre sa vie} \zh{过日子}  
 ¶ \textcolor{darkblue}{\textbf{\ipa{hĩ˧-zɯ˧ bv̩˥, | lo˧hɑ˧!}}} \textcolor{PineGreen}{\selectlanguage{french}passer sa vie/vivre la vie des hommes, c'est difficile! / la vie est dure!} \zh{生活,是艰难的!}  
 ¶ \textcolor{darkblue}{\textbf{\ipa{hĩ˧-zɯ˧ | le˧-bv̩˩-ze˩}}} \textcolor{PineGreen}{\selectlanguage{french}(Sa) vie a passé! / (Sa) vie est terminée! (Réflexion après le décès de quelqu'un.)} \zh{他的日子,就结束了!(情景:一个人去世了,葬礼的时候,有人这样说。)}  
 ¶ \textcolor{darkblue}{\textbf{\ipa{qʰwɤ˧-ɭɯ˥, | ʈʂʰæ˧-mɤ˧-dʑɯ˧! | ʈʂʰɯ˧ ɖɯ˧-zɯ˧ bv̩˩-ze˩!}}} \textcolor{PineGreen}{\selectlanguage{french}Il n'a jamais eu à faire la vaisselle (de sa vie)! Voilà comment s'est passée toute sa vie! (Commentaire au sujet de la vie d'un mandarin qui s'était entièrement soustrait aux tâches manuelles.)} \zh{他从来没有洗过碗!他这辈子就是这么过来的!(关于一个官员,完全不用管家务、日常生活中的活儿:有人来管一切。)}  
 ¶ \textcolor{darkblue}{\textbf{\ipa{ɖɯ˧-ɲi˥\textasciitilde{}ɖɯ˩-ɲi˩ | bv̩˩ lo˩ fv̩˩˥!}}} \textcolor{PineGreen}{\selectlanguage{french}Les journées passent bien vite! / Comme le temps passe!} \zh{日子过得真快!}  

\lhead{\firstmark}
\rhead{\botmark}

\subsection{\hspace{-0.5cm} {\Large \textcolor{darkblue}{\textbf{\ipa{bv̩˩\textsubscript{a}}}} \textsubscript{4}}\hspace{0.5cm}[\kern2pt{\textcolor{darkblue}{\textbf{\ipa{bv̩˩˥}}}}\kern2pt]} \hypertarget{bv\string_=\string_Ba4}{}
\markboth{\textcolor{darkblue}{\textbf{\ipa{bv̩˩\textsubscript{a}}}} \textsubscript{4}}{}
\textcolor{teal}{\mytextsc{verbe}} \hspace{4pt} Ton~: L\textsubscript{a}.
\ding{202} \textcolor{PineGreen}{\selectlanguage{french}Asperger; arroser.} \zh{泼水,浇(浇菜)。}  ¶ \textcolor{darkblue}{\textbf{\ipa{le˧-bv̩˩-ze˩}}} \textcolor{PineGreen}{\selectlanguage{french}\mytextsc{accomp} \string_ \mytextsc{pfv}} \zh{泼了}  
 ¶ \textcolor{darkblue}{\textbf{\ipa{dʑɯ˩ bv̩˩˥}}} \textcolor{PineGreen}{\selectlanguage{french}asperger d'eau; arroser} \zh{泼水}  
 ¶ \textcolor{darkblue}{\textbf{\ipa{ɖɯ˧-bv̩˧\textasciitilde{}bv̩˥-ɻ̍˩}}} \textcolor{PineGreen}{\selectlanguage{french}\mytextsc{délimitatif} \mytextsc{red} \mytextsc{inchoatif}} \zh{泼一泼}  
 ¶ \textcolor{darkblue}{\textbf{\ipa{le˧-bv̩˧\textasciitilde{}bv̩˥-ze˩}}} \textcolor{PineGreen}{\selectlanguage{french}\mytextsc{accomp} \mytextsc{red} \mytextsc{pfv}} \zh{泼了一点}  
\ding{203} \textcolor{PineGreen}{\selectlanguage{french}Disperser, semer (ex.: des graines).} \zh{撒(种子)。}  ¶ \textcolor{darkblue}{\textbf{\ipa{ɻæ˩ bv̩˥}}} \textcolor{PineGreen}{\selectlanguage{french}semer à la volée, répandre des graines (pour les semailles)} \zh{撒种子}  
 ¶ \textcolor{darkblue}{\textbf{\ipa{tʰi˧-bv̩˩-ɻ̍˩}}} \textcolor{PineGreen}{\selectlanguage{french}Sème donc!} \zh{撒吧!}  
 ¶ \textcolor{darkblue}{\textbf{\ipa{tʰi˧-bv̩˩-qɑ˩!}}} \textcolor{PineGreen}{\selectlanguage{french}Sème!} \zh{撒吧!}  

\lhead{\firstmark}
\rhead{\botmark}

\subsection{\hspace{-0.5cm} {\Large \textcolor{darkblue}{\textbf{\ipa{bv̩˩\textsubscript{a}}}} \textsubscript{5}}\hspace{0.5cm}[\kern2pt{\textcolor{darkblue}{\textbf{\ipa{bv̩˩˥}}}}\kern2pt]} \hypertarget{bv\string_=\string_Ba5}{}
\markboth{\textcolor{darkblue}{\textbf{\ipa{bv̩˩\textsubscript{a}}}} \textsubscript{5}}{}
\textcolor{teal}{\mytextsc{adjectif}} \hspace{4pt} Ton~: L\textsubscript{a}.
\textcolor{PineGreen}{\selectlanguage{french}Clairsemé, à nu.} \zh{细、薄。}  ¶ \textcolor{darkblue}{\textbf{\ipa{ʁo˧ bv̩˧˥}}} \textcolor{PineGreen}{\selectlanguage{french}chauve (littéralement “la tête est à nu”)} \zh{头秃、头发很少}  
 ¶ \textcolor{darkblue}{\textbf{\ipa{ʁo˧-bv̩˧-hĩ˥}}} \textcolor{PineGreen}{\selectlanguage{french}un homme chauve} \zh{秃子}  
 ¶ \textcolor{darkblue}{\textbf{\ipa{ʁo˧-bv̩˧-zo˥}}} \textcolor{PineGreen}{\selectlanguage{french}idem} \zh{同上}  
 ¶ \textcolor{darkblue}{\textbf{\ipa{ʈʂʰɯ˧ | ʁo˧ bv̩˧-ze˥}}} \textcolor{PineGreen}{\selectlanguage{french}il a perdu ses cheveux, il est devenu chauve} \zh{他秃头了,他头发掉了}  
 ¶ \textcolor{darkblue}{\textbf{\ipa{ʁo˧qʰwɤ˩ | le˧-bv̩˩-ze˩}}} \textcolor{PineGreen}{\selectlanguage{french}(sa) tête s'est dégarnie, (sa) tête est devenue chauve} \zh{(他)秃头了。}  
 ¶ \textcolor{darkblue}{\textbf{\ipa{bv̩˩-hĩ˩˥}}} \textcolor{PineGreen}{\selectlanguage{french}\mytextsc{rel}} \zh{秃的}  

\lhead{\firstmark}
\rhead{\botmark}

\subsection{\hspace{-0.5cm} {\Large \textcolor{darkblue}{\textbf{\ipa{bv̩˧ɖæ˧}}}}\hspace{0.5cm}[\kern2pt{\textcolor{darkblue}{\textbf{\ipa{bv̩˩ɖæ˩˥}}}}\kern2pt]} \hypertarget{bv\string_=\string_Md`\{\string_M1}{}
\markboth{\textcolor{darkblue}{\textbf{\ipa{bv̩˧ɖæ˧}}}}{}
\textcolor{teal}{\mytextsc{adjectif}} \hspace{4pt} Ton~: M.
\textit{De:} \textbf{bv̩˧ et ɖæ˥} \textcolor{PineGreen}{\selectlanguage{french}De mauvaise humeur, ayant mauvais caractère.} \zh{脾气很坏。}  ¶ \textcolor{darkblue}{\textbf{\ipa{ʈʂʰɯ˧ | bv̩˧ɖæ˧-ze˩!}}} \textcolor{PineGreen}{\selectlanguage{french}Il est de mauvais poil/Il est de mauvaise humeur!} \zh{他现在脾气很坏!/ 他生气了!}  

\lhead{\firstmark}
\rhead{\botmark}

\subsection{\hspace{-0.5cm} {\Large \textcolor{darkblue}{\textbf{\ipa{bv̩˩di˩}}}}\hspace{0.5cm}[\kern2pt{\textcolor{darkblue}{\textbf{\ipa{bv̩˧di˧}}}}\kern2pt]} \hypertarget{bv\string_=\string_Bdi\string_B1}{}
\markboth{\textcolor{darkblue}{\textbf{\ipa{bv̩˩di˩}}}}{}
\textcolor{teal}{\mytextsc{nom}} \hspace{4pt} Ton~: L.
\textcolor{PineGreen}{\selectlanguage{french}Étuve.} \zh{蒸笼。}  \zh{量词}: \textcolor{darkblue}{\textbf{\ipa{ɭɯ˧}}}  \mytextsc{clf}~: \textcolor{darkblue}{\textbf{\ipa{ɭɯ˧}}} \textit{Voir~:} \hyperlink{}{\textcolor{darkblue}{\textbf{\ipa{bv̩˩˧}}} \textsubscript{2}} 
\lhead{\firstmark}
\rhead{\botmark}

\subsection{\hspace{-0.5cm} {\Large \textcolor{darkblue}{\textbf{\ipa{bv̩˩dze˩}}} \textsubscript{1}}\hspace{0.5cm}[\kern2pt{\textcolor{darkblue}{\textbf{\ipa{bv̩˩dze˩˥}}}}\kern2pt]} \hypertarget{bv\string_=\string_Bdze\string_B1}{}
\markboth{\textcolor{darkblue}{\textbf{\ipa{bv̩˩dze˩}}} \textsubscript{1}}{}
\textcolor{teal}{\mytextsc{nom}} \hspace{4pt} Ton~: L.
\textcolor{PineGreen}{\selectlanguage{french}Grosse cuillère (pour servir le riz, etc.).} \zh{大调羹。}  \zh{量词}: \textcolor{darkblue}{\textbf{\ipa{nɑ˧}}}  \mytextsc{clf}~: \textcolor{darkblue}{\textbf{\ipa{nɑ˧}}} \textit{Voir~:} \hyperlink{}{\textcolor{darkblue}{\textbf{\ipa{bv̩˩dze˩}}} \textsubscript{2}} 
\lhead{\firstmark}
\rhead{\botmark}

\subsection{\hspace{-0.5cm} {\Large \textcolor{darkblue}{\textbf{\ipa{bv̩˩dze˩}}} \textsubscript{2}}\hspace{0.5cm}[\kern2pt{\textcolor{darkblue}{\textbf{\ipa{bv̩˩dze˩˥}}}}\kern2pt]} \hypertarget{bv\string_=\string_Bdze\string_B2}{}
\markboth{\textcolor{darkblue}{\textbf{\ipa{bv̩˩dze˩}}} \textsubscript{2}}{}
\textcolor{teal}{\mytextsc{classificateur}} \hspace{4pt} Ton~: L.
\textcolor{PineGreen}{\selectlanguage{french}Classificateur des cuillerée.} \zh{量词:勺。}  ¶ \textcolor{darkblue}{\textbf{\ipa{ɖɯ˧-bv̩˩dze˩}}} \textcolor{PineGreen}{\selectlanguage{french}une louchée, une louche de} \zh{一勺}  
\textit{Voir~:} \hyperlink{}{\textcolor{darkblue}{\textbf{\ipa{bv̩˩dze˩}}} \textsubscript{1}} 
\lhead{\firstmark}
\rhead{\botmark}

\subsection{\hspace{-0.5cm} {\Large \textcolor{darkblue}{\textbf{\ipa{bv̩˧hu˧˥}}}}\hspace{0.5cm}[\kern2pt{\textcolor{darkblue}{\textbf{\ipa{bv̩˩hu˩˥}}}}\kern2pt]} \hypertarget{bv\string_=\string_Mhu\string_M\string_T1}{}
\markboth{\textcolor{darkblue}{\textbf{\ipa{bv̩˧hu˧˥}}}}{}
\textcolor{teal}{\mytextsc{nom}} \hspace{4pt} Ton~: MH\#.
\textcolor{PineGreen}{\selectlanguage{french}Tube digestif: estomac+intestin.} \zh{胃与肠。}  \zh{量词}: \textcolor{darkblue}{\textbf{\ipa{kwɤ˩}}}  \mytextsc{clf}~: \textcolor{darkblue}{\textbf{\ipa{kwɤ˩}}} 
\lhead{\firstmark}
\rhead{\botmark}

\subsection{\hspace{-0.5cm} {\Large \textcolor{darkblue}{\textbf{\ipa{bv̩˩hwɤ˩}}}}\hspace{0.5cm}[\kern2pt{\textcolor{darkblue}{\textbf{\ipa{bv̩˧hwɤ˧˥}}}}\kern2pt]} \hypertarget{bv\string_=\string_Bhw7\string_B1}{}
\markboth{\textcolor{darkblue}{\textbf{\ipa{bv̩˩hwɤ˩}}}}{}
\textcolor{teal}{\mytextsc{nom}} \hspace{4pt} Ton~: L.
\textcolor{PineGreen}{\selectlanguage{french}Cabane de berger, sur la montagne; n'est pas occupée à l'année, mais est assez solide pour être utilisée année après année, à la différence des cabanes provisoires construites lorsqu'on doit rester qq jours sur la montagne pour couper du bois.} \zh{山上放牧的人暂时住的木头小房。}  \zh{量词}: \textcolor{darkblue}{\textbf{\ipa{ɭɯ˧}}}  \mytextsc{clf}~: \textcolor{darkblue}{\textbf{\ipa{ɭɯ˧}}} 
\lhead{\firstmark}
\rhead{\botmark}

\subsection{\hspace{-0.5cm} {\Large \textcolor{darkblue}{\textbf{\ipa{bv̩˧kʰɯ˧˥}}}}\hspace{0.5cm}[\kern2pt{\textcolor{darkblue}{\textbf{\ipa{bv̩˩kʰɯ˩˥}}}}\kern2pt]} \hypertarget{bv\string_=\string_Mk\string_hM\string_M\string_T1}{}
\markboth{\textcolor{darkblue}{\textbf{\ipa{bv̩˧kʰɯ˧˥}}}}{}
\textcolor{teal}{\mytextsc{nom}} \hspace{4pt} Ton~: MH\#.
\textcolor{PineGreen}{\selectlanguage{french}Ver à soie.} \zh{蚕。}  \zh{量词}: \textcolor{darkblue}{\textbf{\ipa{kʰɯ˩}}}  \mytextsc{clf}~: \textcolor{darkblue}{\textbf{\ipa{kʰɯ˩}}} 
\lhead{\firstmark}
\rhead{\botmark}

\subsection{\hspace{-0.5cm} {\Large \textcolor{darkblue}{\textbf{\ipa{bv̩˩ɭɯ˩}}}}\hspace{0.5cm}[\kern2pt{\textcolor{darkblue}{\textbf{\ipa{bv̩˧ɭɯ˧˥}}}}\kern2pt]} \hypertarget{bv\string_=\string_Bl\string_RM\string_B1}{}
\markboth{\textcolor{darkblue}{\textbf{\ipa{bv̩˩ɭɯ˩}}}}{}
\textcolor{teal}{\mytextsc{nom}} \hspace{4pt} Ton~: L.
\textcolor{PineGreen}{\selectlanguage{french}Rein.} \zh{肾。}  \zh{量词}: \textcolor{darkblue}{\textbf{\ipa{ɭɯ˧}}}  \mytextsc{clf}~: \textcolor{darkblue}{\textbf{\ipa{ɭɯ˧}}} 
\lhead{\firstmark}
\rhead{\botmark}

\subsection{\hspace{-0.5cm} {\Large \textcolor{darkblue}{\textbf{\ipa{bv̩˧mi˧}}} \textsubscript{1}}\hspace{0.5cm}[\kern2pt{\textcolor{darkblue}{\textbf{\ipa{bv̩˩mi˩˥}}}}\kern2pt]} \hypertarget{bv\string_=\string_Mmi\string_M1}{}
\markboth{\textcolor{darkblue}{\textbf{\ipa{bv̩˧mi˧}}} \textsubscript{1}}{}
\textcolor{teal}{\mytextsc{nom}} \hspace{4pt} Ton~: M.
\textcolor{PineGreen}{\selectlanguage{french}Yak femelle, dri, drimo, nak.} \zh{母牦牛。}  ¶ \textcolor{darkblue}{\textbf{\ipa{bv̩˧mi˧-bv̩˩ʂwæ˩}}} \textcolor{PineGreen}{\selectlanguage{french}yak femelle et yak châtré} \zh{母牦牛与阉割牦牛}  
 ¶ \textcolor{darkblue}{\textbf{\ipa{bv̩˧mi˧-bv̩˧zo\#˥}}} \textcolor{PineGreen}{\selectlanguage{french}maman yack et petit yack} \zh{母牦牛与小牦牛}  
 \zh{量词}: \textcolor{darkblue}{\textbf{\ipa{mi˩}}}  \mytextsc{clf}~: \textcolor{darkblue}{\textbf{\ipa{mi˩}}} 
\lhead{\firstmark}
\rhead{\botmark}

\subsection{\hspace{-0.5cm} {\Large \textcolor{darkblue}{\textbf{\ipa{bv̩˧mi˧}}} \textsubscript{2}}\hspace{0.5cm}[\kern2pt{\textcolor{darkblue}{\textbf{\ipa{bv̩˧mi˧}}}}\kern2pt]} \hypertarget{bv\string_=\string_Mmi\string_M2}{}
\markboth{\textcolor{darkblue}{\textbf{\ipa{bv̩˧mi˧}}} \textsubscript{2}}{}
\textcolor{teal}{\mytextsc{nom}} \hspace{4pt} Ton~: M.
\textcolor{PineGreen}{\selectlanguage{french}Grande étuve.} \zh{大蒸笼。}  \zh{量词}: \textcolor{darkblue}{\textbf{\ipa{mi˩}}}  \mytextsc{clf}~: \textcolor{darkblue}{\textbf{\ipa{mi˩}}} 
\lhead{\firstmark}
\rhead{\botmark}

\subsection{\hspace{-0.5cm} {\Large \textcolor{darkblue}{\textbf{\ipa{bv̩˧-nɑ˥mi˩}}}}\hspace{0.5cm}[\kern2pt{\textcolor{darkblue}{\textbf{\ipa{xxxx non-correspondance entre le nombre de morphèmes et le nombre de tons de morphèmes}}}}\kern2pt]} \hypertarget{bv\string_=\string_M-nA\string_Tmi\string_B1}{}
\markboth{\textcolor{darkblue}{\textbf{\ipa{bv̩˧-nɑ˥mi˩}}}}{}
\textcolor{teal}{\mytextsc{nom}} \hspace{4pt} Ton~: \#H-.
\textcolor{PineGreen}{\selectlanguage{french}\textit{Mythimna separata (Walker)}.} \zh{玉米黏虫。} 
\lhead{\firstmark}
\rhead{\botmark}

\subsection{\hspace{-0.5cm} {\Large \textcolor{darkblue}{\textbf{\ipa{bv̩˧nv̩˧}}} \textsubscript{1}}\hspace{0.5cm}[\kern2pt{\textcolor{darkblue}{\textbf{\ipa{xxxx non-correspondance entre le nombre de morphèmes et le nombre de tons de morphèmes}}}}\kern2pt]} \hypertarget{bv\string_=\string_Mnv\string_=\string_M1}{}
\markboth{\textcolor{darkblue}{\textbf{\ipa{bv̩˧nv̩˧}}} \textsubscript{1}}{}
\textcolor{teal}{\mytextsc{verbe}} \hspace{4pt} Ton~: M intrans.
\textcolor{PineGreen}{\selectlanguage{french}Sentir (par l'odorat).} \zh{闻到(嗅觉)。}  ¶ \textcolor{darkblue}{\textbf{\ipa{no˧ | ɖɯ˧-bv̩˧nv̩˧-ɻ̍˩! | ɖwæ˩˥ | ɕjɤ˧!}}} \textcolor{PineGreen}{\selectlanguage{french}Sens donc! ça sent très bon/c'est très odorant/parfumé!} \zh{你闻一闻吧!好香!}  
\textit{Voir~:} \hyperlink{}{\textcolor{darkblue}{\textbf{\ipa{bv̩˧nv̩˧}}} \textsubscript{2}} 
\lhead{\firstmark}
\rhead{\botmark}

\subsection{\hspace{-0.5cm} {\Large \textcolor{darkblue}{\textbf{\ipa{bv̩˧nv̩˧}}} \textsubscript{2}}\hspace{0.5cm}[\kern2pt{\textcolor{darkblue}{\textbf{\ipa{bv̩˧nv̩˧}}}}\kern2pt]} \hypertarget{bv\string_=\string_Mnv\string_=\string_M2}{}
\markboth{\textcolor{darkblue}{\textbf{\ipa{bv̩˧nv̩˧}}} \textsubscript{2}}{}
\textcolor{teal}{\mytextsc{adjectif}} \hspace{4pt} Ton~: M intrans.
\textcolor{PineGreen}{\selectlanguage{french}Malodorant, puant, qui a une mauvaise odeur.} \zh{臭。}  ¶ \textcolor{darkblue}{\textbf{\ipa{bv̩˧nv̩˧-ze˧}}} \textcolor{PineGreen}{\selectlanguage{french}\mytextsc{pfv}} \zh{臭了}  
\textit{Voir~:} \hyperlink{}{\textcolor{darkblue}{\textbf{\ipa{bv̩˧nv̩˧}}} \textsubscript{1}} 
\lhead{\firstmark}
\rhead{\botmark}

\subsection{\hspace{-0.5cm} {\Large \textcolor{darkblue}{\textbf{\ipa{bv̩˧pʰv̩˧}}}}\hspace{0.5cm}[\kern2pt{\textcolor{darkblue}{\textbf{\ipa{bv̩˧pʰv̩˧}}}}\kern2pt]} \hypertarget{bv\string_=\string_Mp\string_hv\string_=\string_M1}{}
\markboth{\textcolor{darkblue}{\textbf{\ipa{bv̩˧pʰv̩˧}}}}{}
\textcolor{teal}{\mytextsc{nom}} \hspace{4pt} Ton~: M.
\textcolor{PineGreen}{\selectlanguage{french}Yak mâle. Ce mot est une forme élicitée; la forme usuelle est: \textcolor{darkblue}{\textbf{\ipa{/bv̩˩ʂwæ˩/}}}.} \zh{公牦牛。}  \zh{量词}: \textcolor{darkblue}{\textbf{\ipa{mi˩}}}  \mytextsc{clf}~: \textcolor{darkblue}{\textbf{\ipa{mi˩}}} 
\lhead{\firstmark}
\rhead{\botmark}

\subsection{\hspace{-0.5cm} {\Large \textcolor{darkblue}{\textbf{\ipa{bv̩˩-qʰæ˩}}}}\hspace{0.5cm}[\kern2pt{\textcolor{darkblue}{\textbf{\ipa{xxxx non-correspondance entre le nombre de morphèmes et le nombre de tons de morphèmes}}}}\kern2pt]} \hypertarget{bv\string_=\string_B-q\string_h\{\string_B1}{}
\markboth{\textcolor{darkblue}{\textbf{\ipa{bv̩˩-qʰæ˩}}}}{}
\textcolor{teal}{\mytextsc{nom}} \hspace{4pt} Ton~: L.
\textcolor{PineGreen}{\selectlanguage{french}Fumier.} \zh{肥料、粪。}  ¶ \textcolor{darkblue}{\textbf{\ipa{bv̩˩qʰæ˩ tʰv̩˩-ʁwɤ˥}}} \textcolor{PineGreen}{\selectlanguage{french}\mytextsc{n}+\mytextsc{dem}+\mytextsc{clf}} \zh{这堆肥料}  
 \zh{量词}: \textcolor{darkblue}{\textbf{\ipa{ʁwɤ˧}}}  \mytextsc{clf}~: \textcolor{darkblue}{\textbf{\ipa{ʁwɤ˧}}} \textcolor{PineGreen}{\selectlanguage{french}tas}
\lhead{\firstmark}
\rhead{\botmark}

\subsection{\hspace{-0.5cm} {\Large \textcolor{darkblue}{\textbf{\ipa{bv̩˩qo˩-bv̩˧qʰæ˩}}}}\hspace{0.5cm}[\kern2pt{\textcolor{darkblue}{\textbf{\ipa{xxxx non-correspondance entre le nombre de morphèmes et le nombre de tons de morphèmes}}}}\kern2pt]} \hypertarget{bv\string_=\string_Bqo\string_B-bv\string_=\string_Mq\string_h\{\string_B1}{}
\markboth{\textcolor{darkblue}{\textbf{\ipa{bv̩˩qo˩-bv̩˧qʰæ˩}}}}{}
\textcolor{teal}{\mytextsc{nom}} \hspace{4pt} Ton~: L-L\#.
\textcolor{PineGreen}{\selectlanguage{french}Fumier.} \zh{肥料、粪。}  \zh{量词}: \textcolor{darkblue}{\textbf{\ipa{ʁwɤ˧}}}  \mytextsc{clf}~: \textcolor{darkblue}{\textbf{\ipa{ʁwɤ˧}}} 
\lhead{\firstmark}
\rhead{\botmark}

\subsection{\hspace{-0.5cm} {\Large \textcolor{darkblue}{\textbf{\ipa{bv̩˩qo˩-qʰæ˩}}}}\hspace{0.5cm}[\kern2pt{\textcolor{darkblue}{\textbf{\ipa{bv̩˩qo˩qʰæ˩˥}}}}\kern2pt]} \hypertarget{bv\string_=\string_Bqo\string_B-q\string_h\{\string_B1}{}
\markboth{\textcolor{darkblue}{\textbf{\ipa{bv̩˩qo˩-qʰæ˩}}}}{}
\textcolor{teal}{\mytextsc{nom}} \hspace{4pt} Ton~: L.
\textcolor{PineGreen}{\selectlanguage{french}Fumier.} \zh{肥料 , 粪。}  ¶ \textcolor{darkblue}{\textbf{\ipa{bv̩˩qo˩-qʰæ˩ tʰv̩˩-ʁwɤ˥}}} \textcolor{PineGreen}{\selectlanguage{french}\mytextsc{n}+\mytextsc{dem}+\mytextsc{clf}} \zh{这堆肥料}  
 \zh{量词}: \textcolor{darkblue}{\textbf{\ipa{ʁwɤ˧}}}  \mytextsc{clf}~: \textcolor{darkblue}{\textbf{\ipa{ʁwɤ˧}}} 
\lhead{\firstmark}
\rhead{\botmark}

\subsection{\hspace{-0.5cm} {\Large \textcolor{darkblue}{\textbf{\ipa{bv̩˩qʰv̩˩}}}}\hspace{0.5cm}[\kern2pt{\textcolor{darkblue}{\textbf{\ipa{bv̩˩qʰv̩˩˥}}}}\kern2pt]} \hypertarget{bv\string_=\string_Bq\string_hv\string_=\string_B1}{}
\markboth{\textcolor{darkblue}{\textbf{\ipa{bv̩˩qʰv̩˩}}}}{}
\textcolor{teal}{\mytextsc{nom}} \hspace{4pt} Ton~: L.
\ding{202} \textcolor{PineGreen}{\selectlanguage{french}Conque, \textit{Turbinella pyrum L.}. On y souffle, comme dans une trompe, lors des cérémonies. Chaque famille en possède une paire.} \zh{法螺、海螺、螺号。}  \zh{量词}: \textcolor{darkblue}{\textbf{\ipa{dze˩}}} \ding{203} \textcolor{PineGreen}{\selectlanguage{french}Lignes de la main (dont la forme en spirale évoque les conques, objet symbolique important dans la culture na).} \zh{掌纹。}  ¶ \textcolor{darkblue}{\textbf{\ipa{lo˩qʰwɤ˧-bv̩˧ | bv̩˩qʰv̩˩˥}}} \textcolor{PineGreen}{\selectlanguage{french}les lignes de la main} \zh{掌纹}  
 \mytextsc{clf}~: \textcolor{darkblue}{\textbf{\ipa{dze˩}}} \textcolor{PineGreen}{\selectlanguage{french}paire; explication: va par paires, chaque maisonnée en a une paire}
\lhead{\firstmark}
\rhead{\botmark}

\subsection{\hspace{-0.5cm} {\Large \textcolor{darkblue}{\textbf{\ipa{bv̩˧qʰv̩˧ʑi˩-hĩ˩}}}}\hspace{0.5cm}[\kern2pt{\textcolor{darkblue}{\textbf{\ipa{xxxx non-correspondance entre le nombre de morphèmes et le nombre de tons de morphèmes}}}}\kern2pt]} \hypertarget{bv\string_=\string_Mq\string_hv\string_=\string_Mz£i\string_B-hi\string_~\string_B1}{}
\markboth{\textcolor{darkblue}{\textbf{\ipa{bv̩˧qʰv̩˧ʑi˩-hĩ˩}}}}{}
\textcolor{teal}{\mytextsc{nom}} \hspace{4pt} Ton~: L\#.
\textcolor{PineGreen}{\selectlanguage{french}Escargot.} \zh{蜗牛,螺蛳。}  \zh{量词}: \textcolor{darkblue}{\textbf{\ipa{mi˩}}}  \mytextsc{clf}~: \textcolor{darkblue}{\textbf{\ipa{mi˩}}} 
\lhead{\firstmark}
\rhead{\botmark}

\subsection{\hspace{-0.5cm} {\Large \textcolor{darkblue}{\textbf{\ipa{bv̩˧ɻ\#˥}}}}\hspace{0.5cm}[\kern2pt{\textcolor{darkblue}{\textbf{\ipa{bv̩˧ɻ˩}}}}\kern2pt]} \hypertarget{bv\string_=\string_Mr£`\#\string_T1}{}
\markboth{\textcolor{darkblue}{\textbf{\ipa{bv̩˧ɻ\#˥}}}}{}
\textcolor{teal}{\mytextsc{nom}} \hspace{4pt} Ton~: \#H.
\textcolor{PineGreen}{\selectlanguage{french}Mouche.} \zh{苍蝇。}  ¶ \textcolor{darkblue}{\textbf{\ipa{bv̩˧ɻ̍˧ ʈʂʰɯ˧-mi˧˥ / bv̩˧ɻ̍˧ ʈʂʰɯ˧-mi˥\#}}} \textcolor{PineGreen}{\selectlanguage{french}\mytextsc{n}+\mytextsc{dem}+\mytextsc{clf}} \zh{这只苍蝇}  
 \zh{量词}: \textcolor{darkblue}{\textbf{\ipa{mi˩}}}  \mytextsc{clf}~: \textcolor{darkblue}{\textbf{\ipa{mi˩}}} 
\lhead{\firstmark}
\rhead{\botmark}

\subsection{\hspace{-0.5cm} {\Large \textcolor{darkblue}{\textbf{\ipa{bv̩˧ʂæ˧}}}}\hspace{0.5cm}[\kern2pt{\textcolor{darkblue}{\textbf{\ipa{bv̩˧ʂæ˧}}}}\kern2pt]} \hypertarget{bv\string_=\string_Ms`\{\string_M1}{}
\markboth{\textcolor{darkblue}{\textbf{\ipa{bv̩˧ʂæ˧}}}}{}
\textcolor{teal}{\mytextsc{adjectif}} \hspace{4pt} Ton~: M.
\textit{De:} \textbf{bv̩˧ et ʂæ˧} \textcolor{PineGreen}{\selectlanguage{french}De bonne humeur, ayant bon caractère.} \zh{脾气好。}  ¶ \textcolor{darkblue}{\textbf{\ipa{ʈʂʰɯ˧ | bv̩˧ʂæ˧-ze˩}}} \textcolor{PineGreen}{\selectlanguage{french}il est de bonne humeur} \zh{他现在脾气好。/ 他高兴了。}  
 ¶ \textcolor{darkblue}{\textbf{\ipa{bv̩˧ʂæ˧ | ʐwæ˩˥}}} \textcolor{PineGreen}{\selectlanguage{french}de très bonne humeur} \zh{脾气很好}  

\lhead{\firstmark}
\rhead{\botmark}

\subsection{\hspace{-0.5cm} {\Large \textcolor{darkblue}{\textbf{\ipa{bv̩˩ʂwæ˩}}}}\hspace{0.5cm}[\kern2pt{\textcolor{darkblue}{\textbf{\ipa{bv̩˧ʂwæ˧}}}}\kern2pt]} \hypertarget{bv\string_=\string_Bs`w\{\string_B1}{}
\markboth{\textcolor{darkblue}{\textbf{\ipa{bv̩˩ʂwæ˩}}}}{}
\textcolor{teal}{\mytextsc{nom}} \hspace{4pt} Ton~: L.
\textcolor{PineGreen}{\selectlanguage{french}Yak châtré.} \zh{阉割过的牦牛。}  ¶ \textcolor{darkblue}{\textbf{\ipa{bv̩˩ʂwæ˩-bv̩˥mi˩}}} \textcolor{PineGreen}{\selectlanguage{french}yak châtré et yak femelle} \zh{阉割过的公牦牛与母牦牛}  
 \zh{量词}: \textcolor{darkblue}{\textbf{\ipa{pʰo˧˥}}}  \mytextsc{clf}~: \textcolor{darkblue}{\textbf{\ipa{pʰo˧˥}}} 
\lhead{\firstmark}
\rhead{\botmark}

\subsection{\hspace{-0.5cm} {\Large \textcolor{darkblue}{\textbf{\ipa{bv̩˧tɕi˧}}}}\hspace{0.5cm}[\kern2pt{\textcolor{darkblue}{\textbf{\ipa{bv̩˩tɕi˩˥}}}}\kern2pt]} \hypertarget{bv\string_=\string_Mts£i\string_M1}{}
\markboth{\textcolor{darkblue}{\textbf{\ipa{bv̩˧tɕi˧}}}}{}
\textcolor{teal}{\mytextsc{nom}} \hspace{4pt} Ton~: M.
\textcolor{PineGreen}{\selectlanguage{french}Pêche sauvage (de petite taille).} \zh{毛桃。}  \zh{量词}: \textcolor{darkblue}{\textbf{\ipa{tɕi˧˥}}} \textcolor{darkblue}{\textbf{\ipa{ɭɯ˧}}}  \mytextsc{clf}~: \textcolor{darkblue}{\textbf{\ipa{tɕi˧˥}}} \textcolor{darkblue}{\textbf{\ipa{ɭɯ˧}}} \textcolor{PineGreen}{\selectlanguage{french}livre; unité}
\lhead{\firstmark}
\rhead{\botmark}

\subsection{\hspace{-0.5cm} {\Large \textcolor{darkblue}{\textbf{\ipa{bv̩˧ʈʂɯ˥}}}}\hspace{0.5cm}[\kern2pt{\textcolor{darkblue}{\textbf{\ipa{bv̩˧ʈʂɯ˧}}}}\kern2pt]} \hypertarget{bv\string_=\string_Mt`s`M\string_T1}{}
\markboth{\textcolor{darkblue}{\textbf{\ipa{bv̩˧ʈʂɯ˥}}}}{}
\textcolor{teal}{\mytextsc{nom}} \hspace{4pt} Ton~: H\#.
\textcolor{PineGreen}{\selectlanguage{french}Vanneries: tamis où l'on fait sécher les graines de courge et autres produits de la ferme.} \zh{筛子。}  \zh{量词}: \textcolor{darkblue}{\textbf{\ipa{nɑ˧}}}  \mytextsc{clf}~: \textcolor{darkblue}{\textbf{\ipa{nɑ˧}}} 
\lhead{\firstmark}
\rhead{\botmark}

\subsection{\hspace{-0.5cm} {\Large \textcolor{darkblue}{\textbf{\ipa{bv̩˧ʈʂʰv̩˧}}}}\hspace{0.5cm}[\kern2pt{\textcolor{darkblue}{\textbf{\ipa{bv̩˧ʈʂʰv̩˥}}}}\kern2pt]} \hypertarget{bv\string_=\string_Mt`s`\string_hv\string_=\string_M1}{}
\markboth{\textcolor{darkblue}{\textbf{\ipa{bv̩˧ʈʂʰv̩˧}}}}{}
\textcolor{teal}{\mytextsc{nom}} \hspace{4pt} Ton~: M.
\textcolor{PineGreen}{\selectlanguage{french}Cymbales.} \zh{钹。}  \zh{量词}: \textcolor{darkblue}{\textbf{\ipa{nɑ˧}}}  \mytextsc{clf}~: \textcolor{darkblue}{\textbf{\ipa{nɑ˧}}} 
\lhead{\firstmark}
\rhead{\botmark}

\subsection{\hspace{-0.5cm} {\Large \textcolor{darkblue}{\textbf{\ipa{bv̩˩zo˩}}}}\hspace{0.5cm}[\kern2pt{\textcolor{darkblue}{\textbf{\ipa{bv̩˧zo˧}}}}\kern2pt]} \hypertarget{bv\string_=\string_Bzo\string_B1}{}
\markboth{\textcolor{darkblue}{\textbf{\ipa{bv̩˩zo˩}}}}{}
\textcolor{teal}{\mytextsc{nom}} \hspace{4pt} Ton~: L.
\textcolor{PineGreen}{\selectlanguage{french}Petite étuve.} \zh{小蒸笼。}  \zh{量词}: \textcolor{darkblue}{\textbf{\ipa{ɭɯ˧}}}  \mytextsc{clf}~: \textcolor{darkblue}{\textbf{\ipa{ɭɯ˧}}} 
\lhead{\firstmark}
\rhead{\botmark}

\subsection{\hspace{-0.5cm} {\Large \textcolor{darkblue}{\textbf{\ipa{bv̩˧zo\#˥}}}}\hspace{0.5cm}[\kern2pt{\textcolor{darkblue}{\textbf{\ipa{bv̩˩zo˩˥}}}}\kern2pt]} \hypertarget{bv\string_=\string_Mzo\#\string_T1}{}
\markboth{\textcolor{darkblue}{\textbf{\ipa{bv̩˧zo\#˥}}}}{}
\textcolor{teal}{\mytextsc{nom}} \hspace{4pt} Ton~: \#H.
\textcolor{PineGreen}{\selectlanguage{french}Petit du yak.} \zh{小牦牛。}  ¶ \textcolor{darkblue}{\textbf{\ipa{bv̩˧zo˧ tʰv̩˧-mi˧˥ / bv̩˧zo˧ tʰv̩˧-mi˥\#}}} \textcolor{PineGreen}{\selectlanguage{french}\mytextsc{n}+\mytextsc{dem}+\mytextsc{clf}} \zh{这头小牦牛}  
 \zh{量词}: \textcolor{darkblue}{\textbf{\ipa{mi˩}}}  \mytextsc{clf}~: \textcolor{darkblue}{\textbf{\ipa{mi˩}}} 
\lhead{\firstmark}
\rhead{\botmark}

\subsection{\hspace{-0.5cm} {\Large \textcolor{darkblue}{\textbf{\ipa{bv̩˩ʐv̩˩-dzi˩}}}}\hspace{0.5cm}[\kern2pt{\textcolor{darkblue}{\textbf{\ipa{xxxx non-correspondance entre le nombre de morphèmes et le nombre de tons de morphèmes}}}}\kern2pt]} \hypertarget{bv\string_=\string_Bz`v\string_=\string_B-dzi\string_B1}{}
\markboth{\textcolor{darkblue}{\textbf{\ipa{bv̩˩ʐv̩˩-dzi˩}}}}{}
\textcolor{teal}{\mytextsc{nom}} \hspace{4pt} Ton~: L.
\textcolor{PineGreen}{\selectlanguage{french}Lierre.} \zh{常春藤。}  \zh{量词}: \textcolor{darkblue}{\textbf{\ipa{dzi˩}}}  \mytextsc{clf}~: \textcolor{darkblue}{\textbf{\ipa{dzi˩}}} 
\lhead{\firstmark}
\rhead{\botmark}

\subsection{\hspace{-0.5cm} {\Large \textcolor{darkblue}{\textbf{\ipa{bv̩˧ʐv̩˧-kʰv̩˧˥}}}}\hspace{0.5cm}[\kern2pt{\textcolor{darkblue}{\textbf{\ipa{xxxx non-correspondance entre le nombre de morphèmes et le nombre de tons de morphèmes}}}}\kern2pt]} \hypertarget{bv\string_=\string_Mz`v\string_=\string_M-k\string_hv\string_=\string_M\string_T1}{}
\markboth{\textcolor{darkblue}{\textbf{\ipa{bv̩˧ʐv̩˧-kʰv̩˧˥}}}}{}
\textcolor{teal}{\mytextsc{nom}} \hspace{4pt} Ton~: MH\#.
\textcolor{PineGreen}{\selectlanguage{french}Année du serpent.} \zh{蛇年。} \textit{Voir~:} \hyperlink{}{\textcolor{darkblue}{\textbf{\ipa{ʐv̩˧bæ˧}}}} 
\lhead{\firstmark}
\rhead{\botmark}

\subsection{\hspace{-0.5cm} {\Large \textcolor{darkblue}{\textbf{\ipa{‑bv˧}}}}\hspace{0.5cm}[\kern2pt{\textcolor{darkblue}{\textbf{\ipa{xxxx groupe tonal entier sans aucun ton}}}}\kern2pt]} \hypertarget{‑bv\string_M1}{}
\markboth{\textcolor{darkblue}{\textbf{\ipa{‑bv˧}}}}{}
\textcolor{teal}{\mytextsc{suffixe}} \hspace{4pt} Ton~: 0.
\textcolor{PineGreen}{\selectlanguage{french}Possessif.} \zh{属式:的。} 
\lhead{\firstmark}
\rhead{\botmark}

\newpage
\section*{\centering- \textcolor{darkblue}{\textbf{\ipa{ɕ}}} -}
\subsection{\hspace{-0.5cm} {\Large \textcolor{darkblue}{\textbf{\ipa{ɕi˥\textsubscript{a}}}}}\hspace{0.5cm}[\kern2pt{\textcolor{darkblue}{\textbf{\ipa{ɕi˥}}}}\kern2pt]} \hypertarget{s£i\string_Ta1}{}
\markboth{\textcolor{darkblue}{\textbf{\ipa{ɕi˥\textsubscript{a}}}}}{}
\textcolor{teal}{\mytextsc{classificateur}} \hspace{4pt} Ton~: H\textsubscript{a}.
\textcolor{PineGreen}{\selectlanguage{french}100.} \zh{百。}  ¶ \textcolor{darkblue}{\textbf{\ipa{ɖɯ˧-ɕi˥}}} \textcolor{PineGreen}{\selectlanguage{french}cent} \zh{一百}  
 ¶ \textcolor{darkblue}{\textbf{\ipa{ɖɯ˧-ɕi˧ kʰv̩˧˥}}} \textcolor{PineGreen}{\selectlanguage{french}cent ans, un siècle} \zh{一百年}  
 ¶ \textcolor{darkblue}{\textbf{\ipa{ɖɯ˧-ɕi˧ kʰv̩˧\textasciitilde{}ɖɯ˥-ɕi˩ kʰv̩˩}}} \textcolor{PineGreen}{\selectlanguage{french}siècle après siècle} \zh{一百年又一百年}  
 ¶ \textcolor{darkblue}{\textbf{\ipa{ɕi˧-kʰv̩˧˥}}} \textcolor{PineGreen}{\selectlanguage{french}cent ans (formulation abrégée)} \zh{百年(“一百年”的省略说法)}  

\lhead{\firstmark}
\rhead{\botmark}

\subsection{\hspace{-0.5cm} {\Large \textcolor{darkblue}{\textbf{\ipa{ɕi˥\textsubscript{b}}}}}\hspace{0.5cm}[\kern2pt{\textcolor{darkblue}{\textbf{\ipa{ɕi˥}}}}\kern2pt]} \hypertarget{s£i\string_Tb1}{}
\markboth{\textcolor{darkblue}{\textbf{\ipa{ɕi˥\textsubscript{b}}}}}{}
\textcolor{teal}{\mytextsc{classificateur}} \hspace{4pt} Ton~: H\textsubscript{b}.
\textcolor{PineGreen}{\selectlanguage{french}Centième d'unité monétaire.} \zh{量词:分(一分钱)。} 
\lhead{\firstmark}
\rhead{\botmark}

\subsection{\hspace{-0.5cm} {\Large \textcolor{darkblue}{\textbf{\ipa{ɕi˧}}}}\hspace{0.5cm}[\kern2pt{\textcolor{darkblue}{\textbf{\ipa{ɕi˩˥}}}}\kern2pt]} \hypertarget{s£i\string_M1}{}
\markboth{\textcolor{darkblue}{\textbf{\ipa{ɕi˧}}}}{}
\textcolor{teal}{\mytextsc{nom}} \hspace{4pt} Ton~: M.
\textcolor{PineGreen}{\selectlanguage{french}Riz (monosyllabe).} \zh{米(单音节)。} 
\lhead{\firstmark}
\rhead{\botmark}

\subsection{\hspace{-0.5cm} {\Large \textcolor{darkblue}{\textbf{\ipa{ɕi˧ɕi˩-lo˩}}}}\hspace{0.5cm}[\kern2pt{\textcolor{darkblue}{\textbf{\ipa{xxxx non-correspondance entre le nombre de morphèmes et le nombre de tons de morphèmes}}}}\kern2pt]} \hypertarget{s£i\string_Ms£i\string_B-lo\string_B1}{}
\markboth{\textcolor{darkblue}{\textbf{\ipa{ɕi˧ɕi˩-lo˩}}}}{}
\textcolor{teal}{\mytextsc{nom}} \hspace{4pt} Ton~: L\#-.
\textcolor{PineGreen}{\selectlanguage{french}Les plus petites des côtelettes.} \zh{最细的肋骨。} 
\lhead{\firstmark}
\rhead{\botmark}

\subsection{\hspace{-0.5cm} {\Large \textcolor{darkblue}{\textbf{\ipa{ɕi˧-ho˩ʂɯ˩}}}}\hspace{0.5cm}[\kern2pt{\textcolor{darkblue}{\textbf{\ipa{xxxx non-correspondance entre le nombre de morphèmes et le nombre de tons de morphèmes}}}}\kern2pt]} \hypertarget{s£i\string_M-ho\string_Bs`M\string_B1}{}
\markboth{\textcolor{darkblue}{\textbf{\ipa{ɕi˧-ho˩ʂɯ˩}}}}{}
\textcolor{teal}{\mytextsc{nom}} \hspace{4pt} Ton~: -L.
\textcolor{PineGreen}{\selectlanguage{french}Tomate.} \zh{西红柿(汉语借词)。}  Emprunt~: chinois  \zh{西红柿}

\lhead{\firstmark}
\rhead{\botmark}

\subsection{\hspace{-0.5cm} {\Large \textcolor{darkblue}{\textbf{\ipa{ɕi˧lv̩˧}}}}\hspace{0.5cm}[\kern2pt{\textcolor{darkblue}{\textbf{\ipa{ɕi˧lv̩˧}}}}\kern2pt]} \hypertarget{s£i\string_Mlv\string_=\string_M1}{}
\markboth{\textcolor{darkblue}{\textbf{\ipa{ɕi˧lv̩˧}}}}{}
\textcolor{teal}{\mytextsc{nom}} \hspace{4pt} Ton~: M.
\textcolor{PineGreen}{\selectlanguage{french}Champs de riz.} \zh{水田。}  \zh{量词}: \textcolor{darkblue}{\textbf{\ipa{pʰæ˧˥, kɤ˧˥}}}  \mytextsc{clf}~: \textcolor{darkblue}{\textbf{\ipa{pʰæ˧˥, kɤ˧˥}}} 
\lhead{\firstmark}
\rhead{\botmark}

\subsection{\hspace{-0.5cm} {\Large \textcolor{darkblue}{\textbf{\ipa{ɕi˧ɭɯ˧}}}}\hspace{0.5cm}[\kern2pt{\textcolor{darkblue}{\textbf{\ipa{xxxx non-correspondance entre le nombre de morphèmes et le nombre de tons de morphèmes}}}}\kern2pt]} \hypertarget{s£i\string_Ml\string_RM\string_M1}{}
\markboth{\textcolor{darkblue}{\textbf{\ipa{ɕi˧ɭɯ˧}}}}{}
\textcolor{teal}{\mytextsc{nom}} \hspace{4pt} Ton~: M.
\textcolor{PineGreen}{\selectlanguage{french}Riz paddy; par extension: champ de riz.} \zh{稻子,稻田。}  \zh{量词}: \textcolor{darkblue}{\textbf{\ipa{kɤ˧˥ (pour le grain)}}} \textcolor{darkblue}{\textbf{\ipa{pʰæ˧˥ (pour un champ)}}}  \mytextsc{clf}~: \textcolor{darkblue}{\textbf{\ipa{kɤ˧˥ (pour le grain)}}} \textcolor{darkblue}{\textbf{\ipa{pʰæ˧˥ (pour un champ)}}} 
\lhead{\firstmark}
\rhead{\botmark}

\subsection{\hspace{-0.5cm} {\Large \textcolor{darkblue}{\textbf{\ipa{ɕi˧ɭɯ˧-lv̩˧pʰv̩˩}}}}\hspace{0.5cm}[\kern2pt{\textcolor{darkblue}{\textbf{\ipa{xxxx non-correspondance entre le nombre de morphèmes et le nombre de tons de morphèmes}}}}\kern2pt]} \hypertarget{s£i\string_Ml\string_RM\string_M-lv\string_=\string_Mp\string_hv\string_=\string_B1}{}
\markboth{\textcolor{darkblue}{\textbf{\ipa{ɕi˧ɭɯ˧-lv̩˧pʰv̩˩}}}}{}
\textcolor{teal}{\mytextsc{nom}} \hspace{4pt} Ton~: \mytextsc{L}\#.
\textcolor{PineGreen}{\selectlanguage{french}Champs de riz.} \zh{水田。}  \zh{量词}: \textcolor{darkblue}{\textbf{\ipa{pʰæ˧˥, kɤ˧˥}}}  \mytextsc{clf}~: \textcolor{darkblue}{\textbf{\ipa{pʰæ˧˥, kɤ˧˥}}} 
\lhead{\firstmark}
\rhead{\botmark}

\subsection{\hspace{-0.5cm} {\Large \textcolor{darkblue}{\textbf{\ipa{ɕi˧tɕʰi\#˥}}}}\hspace{0.5cm}[\kern2pt{\textcolor{darkblue}{\textbf{\ipa{ɕi˧tɕʰi˧}}}}\kern2pt]} \hypertarget{s£i\string_Mts£\string_hi\#\string_T1}{}
\markboth{\textcolor{darkblue}{\textbf{\ipa{ɕi˧tɕʰi\#˥}}}}{}
\textcolor{teal}{\mytextsc{nom}} \hspace{4pt} Ton~: \#H.
\textcolor{PineGreen}{\selectlanguage{french}Son de riz.} \zh{米糠。}  \zh{量词}: \textcolor{darkblue}{\textbf{\ipa{mɤ˩}}}  \mytextsc{clf}~: \textcolor{darkblue}{\textbf{\ipa{mɤ˩}}} 
\lhead{\firstmark}
\rhead{\botmark}

\subsection{\hspace{-0.5cm} {\Large \textcolor{darkblue}{\textbf{\ipa{ɕi˧ʈʂʰwæ˧}}}}\hspace{0.5cm}[\kern2pt{\textcolor{darkblue}{\textbf{\ipa{ɕi˧ʈʂʰwæ˧}}}}\kern2pt]} \hypertarget{s£i\string_Mt`s`\string_hw\{\string_M1}{}
\markboth{\textcolor{darkblue}{\textbf{\ipa{ɕi˧ʈʂʰwæ˧}}}}{}
\textcolor{teal}{\mytextsc{nom}} \hspace{4pt} Ton~: M.
\textcolor{PineGreen}{\selectlanguage{french}Riz décortiqué.} \zh{米。}  ¶ \textcolor{darkblue}{\textbf{\ipa{ɕi˧ʈʂʰwæ˧-hɑ˧}}} \textcolor{PineGreen}{\selectlanguage{french}riz cuit; littéralement: “nourriture-riz cuit”; formulation employée pour préciser le terme \textcolor{darkblue}{\textbf{\ipa{/hɑ˥/}}}, qui désigne toutes les nourritures.} \zh{米饭}  

\lhead{\firstmark}
\rhead{\botmark}

\subsection{\hspace{-0.5cm} {\Large \textcolor{darkblue}{\textbf{\ipa{ɕi˩}}}}\hspace{0.5cm}[\kern2pt{\textcolor{darkblue}{\textbf{\ipa{ɕi˧˥}}}}\kern2pt]} \hypertarget{s£i\string_B1}{}
\markboth{\textcolor{darkblue}{\textbf{\ipa{ɕi˩}}}}{}
\textcolor{teal}{\mytextsc{verbe}} \hspace{4pt} Ton~: L\textsubscript{a}?.
\textit{\textcolor{PineGreen}{\selectlanguage{french}archaïque}} [\zh{古语}] \textcolor{PineGreen}{\selectlanguage{french}Craindre, avoir peur de. Verbe qui paraît suranné; il ne se trouve que dans quelques expressions.} \zh{怕、害怕。}  ¶ \textcolor{darkblue}{\textbf{\ipa{njɤ˧ | no˩ ɕi˩ tʰɑ˥-mɤ˩-ʝi˩! | njɤ˧ | no˩ ɖwæ˩ tʰɑ˥-mɤ˩-ʝi˩!}}} \textcolor{PineGreen}{\selectlanguage{french}Ne va pas croire que tu me fasses peur! / Si tu crois que j'ai peur de toi! Si tu crois que je te crains/que tu me fais peur! (Formule de défi.)} \zh{不要以为我害怕你!(挑衅的话)}  
 ¶ \textcolor{darkblue}{\textbf{\ipa{njɤ˧ | no˩ ɕi˩-mɤ˩-ʝi˥!}}} \textcolor{PineGreen}{\selectlanguage{french}Tu ne me fais pas peur!} \zh{你不让我害怕 / 我不害怕你!}  
 ¶ \textcolor{darkblue}{\textbf{\ipa{njɤ˧ | tʰv̩˧ ɕi˩-mɤ˩-ʝi˩!}}} \textcolor{PineGreen}{\selectlanguage{french}Il ne me fait pas peur!} \zh{我不怕他!}  
 ¶ \textcolor{darkblue}{\textbf{\ipa{njɤ˧ | ʈʂʰɯ˧-v̩˧ do˧˥, | ʁo˧ ɕi˧˥ | ʐwæ˩˥!}}} \textcolor{PineGreen}{\selectlanguage{french}Quand je le vois, j'ai peur!/Il me fait très peur!} \zh{我见他,非常害怕!}  

\lhead{\firstmark}
\rhead{\botmark}

\subsection{\hspace{-0.5cm} {\Large \textcolor{darkblue}{\textbf{\ipa{ɕi˩dv̩˥}}}}\hspace{0.5cm}[\kern2pt{\textcolor{darkblue}{\textbf{\ipa{xxxx non-correspondance entre le nombre de morphèmes et le nombre de tons de morphèmes}}}}\kern2pt]} \hypertarget{s£i\string_Bdv\string_=\string_T1}{}
\markboth{\textcolor{darkblue}{\textbf{\ipa{ɕi˩dv̩˥}}}}{}
\textcolor{teal}{\mytextsc{nom}} \hspace{4pt} Ton~: LH.
\textcolor{PineGreen}{\selectlanguage{french}Encens; bâtonnet d'encens.} \zh{香,香火。}  ¶ \textcolor{darkblue}{\textbf{\ipa{ɕi˩dv̩˥ qæ˩}}} \textcolor{PineGreen}{\selectlanguage{french}brûler de l'encens} \zh{烧香}  

\lhead{\firstmark}
\rhead{\botmark}

\subsection{\hspace{-0.5cm} {\Large \textcolor{darkblue}{\textbf{\ipa{ɕi˩dzi˥}}}}\hspace{0.5cm}[\kern2pt{\textcolor{darkblue}{\textbf{\ipa{ɕi˩dzi˥}}}}\kern2pt]} \hypertarget{s£i\string_Bdzi\string_T1}{}
\markboth{\textcolor{darkblue}{\textbf{\ipa{ɕi˩dzi˥}}}}{}
\textcolor{teal}{\mytextsc{nom}} \hspace{4pt} Ton~: LH.
\textcolor{PineGreen}{\selectlanguage{french}Genévrier; arbre dont des branchages sont employés lors des rituels (suivant la tradition tibétaine).} \zh{柏树。}  \zh{量词}: \textcolor{darkblue}{\textbf{\ipa{dzi˩}}}  \mytextsc{clf}~: \textcolor{darkblue}{\textbf{\ipa{dzi˩}}} 
\lhead{\firstmark}
\rhead{\botmark}

\subsection{\hspace{-0.5cm} {\Large \textcolor{darkblue}{\textbf{\ipa{ɕi˩ʈʰæ˧˥}}} \textsubscript{1}}\hspace{0.5cm}[\kern2pt{\textcolor{darkblue}{\textbf{\ipa{ɕi˩ʈʰæ˧˥}}}}\kern2pt]} \hypertarget{s£i\string_Bt`\string_h\{\string_M\string_T1}{}
\markboth{\textcolor{darkblue}{\textbf{\ipa{ɕi˩ʈʰæ˧˥}}} \textsubscript{1}}{}
\textcolor{teal}{\mytextsc{adjectif}} \hspace{4pt} Ton~: LM+MH\#.
\textcolor{PineGreen}{\selectlanguage{french}Bègue, qui a un bégaiement.} \zh{结巴。}  ¶ \textcolor{darkblue}{\textbf{\ipa{ʈʂʰɯ˧ | ɖɯ˧-pi˧˥ | ɕi˩ʈʰæ˧˥}}} \textcolor{PineGreen}{\selectlanguage{french}Il/elle est un peu bègue.} \zh{他有一点结巴。}  
 ¶ \textcolor{darkblue}{\textbf{\ipa{ʈʂʰɯ˧ | ɕi˩ʈʰæ˧-zo˥.}}} \textcolor{PineGreen}{\selectlanguage{french}Il est très bègue / il bégaie beaucoup / il a un fort bégaiement.} \zh{他很结巴。}  
\textit{Voir~:} \hyperlink{}{\textcolor{darkblue}{\textbf{\ipa{ɕi˩ʈʰæ˧˥}}} \textsubscript{2}} 
\lhead{\firstmark}
\rhead{\botmark}

\subsection{\hspace{-0.5cm} {\Large \textcolor{darkblue}{\textbf{\ipa{ɕi˩ʈʰæ˧˥}}} \textsubscript{2}}\hspace{0.5cm}[\kern2pt{\textcolor{darkblue}{\textbf{\ipa{ɕi˩ʈʰæ˧˥}}}}\kern2pt]} \hypertarget{s£i\string_Bt`\string_h\{\string_M\string_T2}{}
\markboth{\textcolor{darkblue}{\textbf{\ipa{ɕi˩ʈʰæ˧˥}}} \textsubscript{2}}{}
\textcolor{teal}{\mytextsc{nom}} \hspace{4pt} Ton~: LM+MH\#.
\textcolor{PineGreen}{\selectlanguage{french}Bègue.} \zh{结巴。}  ¶ \textcolor{darkblue}{\textbf{\ipa{ʈʂʰɯ˧ ɕi˩ʈʰæ˧ ɲi˥}}} \textcolor{PineGreen}{\selectlanguage{french}Il/elle est bègue.} \zh{他是结巴。}  
\textit{Voir~:} \hyperlink{}{\textcolor{darkblue}{\textbf{\ipa{ɕi˩ʈʰæ˧˥}}} \textsubscript{1}} 
\lhead{\firstmark}
\rhead{\botmark}

\subsection{\hspace{-0.5cm} {\Large \textcolor{darkblue}{\textbf{\ipa{ɕi˩˥}}}}\hspace{0.5cm}[\kern2pt{\textcolor{darkblue}{\textbf{\ipa{ɕi˥}}}}\kern2pt]} \hypertarget{s£i\string_B\string_T1}{}
\markboth{\textcolor{darkblue}{\textbf{\ipa{ɕi˩˥}}}}{}
\textcolor{teal}{\mytextsc{nom}} \hspace{4pt} Ton~: LH.
\textcolor{PineGreen}{\selectlanguage{french}Encens (monosyllabe).} \zh{香(单音节)。}  ¶ \textcolor{darkblue}{\textbf{\ipa{ɕi˩ qæ˧˥}}} \textcolor{PineGreen}{\selectlanguage{french}brûler de l'encens} \zh{烧香}  

\lhead{\firstmark}
\rhead{\botmark}

\subsection{\hspace{-0.5cm} {\Large \textcolor{darkblue}{\textbf{\ipa{ɕjɤ˥}}}}\hspace{0.5cm}[\kern2pt{\textcolor{darkblue}{\textbf{\ipa{ɕjɤ˥}}}}\kern2pt]} \hypertarget{s£j7\string_T1}{}
\markboth{\textcolor{darkblue}{\textbf{\ipa{ɕjɤ˥}}}}{}
\textcolor{teal}{\mytextsc{verbe}} \hspace{4pt} Ton~: H.
\textcolor{PineGreen}{\selectlanguage{french}Inventer, trouver.} \zh{发明、想出、找到(办法)。}  ¶ \textcolor{darkblue}{\textbf{\ipa{le˧-ɕjɤ˥}}} \textcolor{PineGreen}{\selectlanguage{french}\mytextsc{accomp}} \zh{想出了}  
 ¶ \textcolor{darkblue}{\textbf{\ipa{ʈʂʰɯ˧ | pæ˧˥hwɤ˧ | ɕjɤ˧ ɣɯ˧!}}} \textcolor{PineGreen}{\selectlanguage{french}Il/elle a une solution à tout/ sait trouver une solution en toutes circonstances!} \zh{他很会想办法的!}  

\lhead{\firstmark}
\rhead{\botmark}

\subsection{\hspace{-0.5cm} {\Large \textcolor{darkblue}{\textbf{\ipa{ɕjɤ˧-bv̩˧nv̩˧}}}}\hspace{0.5cm}[\kern2pt{\textcolor{darkblue}{\textbf{\ipa{xxxx non-correspondance entre le nombre de morphèmes et le nombre de tons de morphèmes}}}}\kern2pt]} \hypertarget{s£j7\string_M-bv\string_=\string_Mnv\string_=\string_M1}{}
\markboth{\textcolor{darkblue}{\textbf{\ipa{ɕjɤ˧-bv̩˧nv̩˧}}}}{}
\textcolor{teal}{\mytextsc{adjectif}} \hspace{4pt} Ton~: M.
\textit{De:} \textbf{bv̩˧nv̩˧} \textcolor{PineGreen}{\selectlanguage{french}Bonne (odeur).} \zh{香(气味)。}  ¶ \textcolor{darkblue}{\textbf{\ipa{ʈʂʰɯ˧ ɕjɤ˧-bv̩˧nv̩˧ ɲi˩.}}} \textcolor{PineGreen}{\selectlanguage{french}ça sent bon!} \zh{这很香(气味香)。}  

\lhead{\firstmark}
\rhead{\botmark}

\subsection{\hspace{-0.5cm} {\Large \textcolor{darkblue}{\textbf{\ipa{ɕjɤ˩\textasciitilde{}ɕjɤ˩}}}}\hspace{0.5cm}[\kern2pt{\textcolor{darkblue}{\textbf{\ipa{ɕjɤ˩ɕjɤ˩˥}}}}\kern2pt]} \hypertarget{s£j7\string_B~s£j7\string_B1}{}
\markboth{\textcolor{darkblue}{\textbf{\ipa{ɕjɤ˩\textasciitilde{}ɕjɤ˩}}}}{}
\textcolor{teal}{\mytextsc{verbe}} \hspace{4pt} Ton~: L.
\textcolor{PineGreen}{\selectlanguage{french}Maltraiter.} \zh{欺负。}  ¶ \textcolor{darkblue}{\textbf{\ipa{hĩ˧ ɕjɤ˥\textasciitilde{}ɕjɤ˩}}} \textcolor{PineGreen}{\selectlanguage{french}maltraiter quelqu'un} \zh{欺负人}  
 ¶ \textcolor{darkblue}{\textbf{\ipa{no˧ | njɤ˩ ɕjɤ˩\textasciitilde{}ɕjɤ˩-mv̩˩-zo˩˥! / no˧ | njɤ˩ ɕjɤ˩\textasciitilde{}ɕjɤ˩˥!}}} \textcolor{PineGreen}{\selectlanguage{french}Vous me maltraitez!} \zh{你对我不好!你欺负我!}  
 ¶ \textcolor{darkblue}{\textbf{\ipa{no˧ | njɤ˩ ɕjɤ˩\textasciitilde{}ɕjɤ˩-ze˥!}}} \textcolor{PineGreen}{\selectlanguage{french}Vous m'avez maltraité!} \zh{你欺负了我!}  

\lhead{\firstmark}
\rhead{\botmark}

\subsection{\hspace{-0.5cm} {\Large \textcolor{darkblue}{\textbf{\ipa{ɕjɤ˩jo˩}}}}\hspace{0.5cm}[\kern2pt{\textcolor{darkblue}{\textbf{\ipa{ɕjɤ˩jo˩˥}}}}\kern2pt]} \hypertarget{s£j7\string_Bjo\string_B1}{}
\markboth{\textcolor{darkblue}{\textbf{\ipa{ɕjɤ˩jo˩}}}}{}
\textcolor{teal}{\mytextsc{nom}} \hspace{4pt} Ton~: L.
\textcolor{PineGreen}{\selectlanguage{french}\textit{Fritillaria cirrhosa}.} \zh{贝母。}  \zh{量词}: \textcolor{darkblue}{\textbf{\ipa{ɭɯ˧}}}  \mytextsc{clf}~: \textcolor{darkblue}{\textbf{\ipa{ɭɯ˧}}} 
\lhead{\firstmark}
\rhead{\botmark}

\subsection{\hspace{-0.5cm} {\Large \textcolor{darkblue}{\textbf{\ipa{ɕjɤ˩tʰv̩˧˥}}}}\hspace{0.5cm}[\kern2pt{\textcolor{darkblue}{\textbf{\ipa{ɕjɤ˩tʰv̩˧˥}}}}\kern2pt]} \hypertarget{s£j7\string_Bt\string_hv\string_=\string_M\string_T1}{}
\markboth{\textcolor{darkblue}{\textbf{\ipa{ɕjɤ˩tʰv̩˧˥}}}}{}
\textcolor{teal}{\mytextsc{verbe}} \hspace{4pt} Ton~: LM+MH\#.
\textcolor{PineGreen}{\selectlanguage{french}Insulter, maudire, se moquer; réprimander, gronder.} \zh{骂,批评。}  ¶ \textcolor{darkblue}{\textbf{\ipa{hĩ˧ ɕjɤ˥tʰv̩˩}}} \textcolor{PineGreen}{\selectlanguage{french}insulter quelqu'un/ critiquer quelqu'un} \zh{骂人、批评人}  

\lhead{\firstmark}
\rhead{\botmark}

\subsection{\hspace{-0.5cm} {\Large \textcolor{darkblue}{\textbf{\ipa{ɕjɤ˧˥}}}}\hspace{0.5cm}[\kern2pt{\textcolor{darkblue}{\textbf{\ipa{ɕjɤ˧˥}}}}\kern2pt]} \hypertarget{s£j7\string_M\string_T1}{}
\markboth{\textcolor{darkblue}{\textbf{\ipa{ɕjɤ˧˥}}}}{}
\textcolor{teal}{\mytextsc{verbe}} \hspace{4pt} Ton~: MH.
\textcolor{PineGreen}{\selectlanguage{french}Essayer, goûter, expérimenter.} \zh{尝试、体会、经过。}  ¶ \textcolor{darkblue}{\textbf{\ipa{le˧-ɕjɤ˧-ze˥}}} \textcolor{PineGreen}{\selectlanguage{french}\mytextsc{accomp} \string_ \mytextsc{pfv}} \zh{试了}  
 ¶ \textcolor{darkblue}{\textbf{\ipa{tso˧\textasciitilde{}tso˧ ɕjɤ˩}}} \textcolor{PineGreen}{\selectlanguage{french}goûter quelque chose} \zh{尝一个东西}  
 ¶ \textcolor{darkblue}{\textbf{\ipa{no˧ ɖɯ˧-kʰwɤ˥ ɕjɤ˩!}}} \textcolor{PineGreen}{\selectlanguage{french}goûte un peu! goûte un morceau!} \zh{你尝一口吧!}  
 ¶ \textcolor{darkblue}{\textbf{\ipa{ɖɯ˧-ɕjɤ˧-ɻ̍˥!}}} \textcolor{PineGreen}{\selectlanguage{french}Goûte voir! / Essaie voir!} \zh{尝一尝吧! / 试一试吧!}  

\lhead{\firstmark}
\rhead{\botmark}

\subsection{\hspace{-0.5cm} {\Large \textcolor{darkblue}{\textbf{\ipa{ɕjo˩li\#˥}}}}\hspace{0.5cm}[\kern2pt{\textcolor{darkblue}{\textbf{\ipa{ɕjo˩li˥}}}}\kern2pt]} \hypertarget{s£jo\string_Bli\#\string_T1}{}
\markboth{\textcolor{darkblue}{\textbf{\ipa{ɕjo˩li\#˥}}}}{}
\textcolor{teal}{\mytextsc{nom}} \hspace{4pt} Ton~: LM+\#H.
\textcolor{PineGreen}{\selectlanguage{french}Flûte (type “flûte traversière” et non “flûte à bec”).} \zh{笛子。}  \zh{量词}: \textcolor{darkblue}{\textbf{\ipa{ɭɯ˧}}}  \mytextsc{clf}~: \textcolor{darkblue}{\textbf{\ipa{ɭɯ˧}}} 
\lhead{\firstmark}
\rhead{\botmark}

\subsection{\hspace{-0.5cm} {\Large \textcolor{darkblue}{\textbf{\ipa{ɕɯ˩\textsubscript{a}}}}}\hspace{0.5cm}[\kern2pt{\textcolor{darkblue}{\textbf{\ipa{ɕɯ˩˥}}}}\kern2pt]} \hypertarget{s£M\string_Ba1}{}
\markboth{\textcolor{darkblue}{\textbf{\ipa{ɕɯ˩\textsubscript{a}}}}}{}
\textcolor{teal}{\mytextsc{verbe}} \hspace{4pt} Ton~: L\textsubscript{a}.
\textcolor{PineGreen}{\selectlanguage{french}Élever (terme plus relevé que \zh{ʐɤ˧}).} \zh{养。}  ¶ \textcolor{darkblue}{\textbf{\ipa{ɕɯ˩zo\#˥}}} \textcolor{PineGreen}{\selectlanguage{french}enfant adopté} \zh{养儿}  
 ¶ \textcolor{darkblue}{\textbf{\ipa{ho˧zo˧-ɕɯ˧zo˥, | æ̃˩ mɤ˧-tsɤ˧! | hĩ˧-zo˧mv˥, | ʐɤ˧ tʰɑ˧-mɤ˧-ʝi˧!}}} \textcolor{PineGreen}{\selectlanguage{french}Un bébé faisan qu'on élève chez soi ne devient pas un poulet (n'est pas domestiqué pour autant)! Il ne faut pas élever les enfants d'autrui! (Proverbe qui ne s'applique pas à l'adoption d'enfants qui ont perdu leurs attaches à leur famille biologique, mais à l'adoption d'enfants qui restent en contact avec leurs proches: quelque soin que l'on consacre à leur éducation, ils restent plus attachés à leur famille d'origine.)} \zh{养的小雉,不会变成鸡!人家的孩子,不要养!(指的不是领养孤儿,而是养别人的孩子:无论多么关心孩子,他还是会更爱自己原来的家人。)}  

\lhead{\firstmark}
\rhead{\botmark}

\newpage
\section*{\centering- \textcolor{darkblue}{\textbf{\ipa{d}}} -}
\subsection{\hspace{-0.5cm} {\Large \textcolor{darkblue}{\textbf{\ipa{dɑ˧ʝi˩}}}}\hspace{0.5cm}[\kern2pt{\textcolor{darkblue}{\textbf{\ipa{dɑ˧ʝi˩}}}}\kern2pt]} \hypertarget{dA\string_Mj££i\string_B1}{}
\markboth{\textcolor{darkblue}{\textbf{\ipa{dɑ˧ʝi˩}}}}{}
\textcolor{teal}{\mytextsc{nom}} \hspace{4pt} Ton~: L\#.
\textcolor{PineGreen}{\selectlanguage{french}Mule.} \zh{骡子。}  ¶ \textcolor{darkblue}{\textbf{\ipa{dɑ˧ʝi˩-dʑo˩, | ɖɯ˩mi˧ dʑo˧-kv̩˥-mæ˩! | ɖɯ˩zo˧ dʑo˧-kv̩˥-mæ˩!}}} \textcolor{PineGreen}{\selectlanguage{french}Les mules, ça se répartit en mules mâles et mules femelles! / Il existe une distinction de sexe parmi les mules! (Explication fournie à un visiteur citadin peu au fait de l'élevage des animaux.)} \zh{骡子呢,有母骡子!(也)有公骡子! / 骡子,分母的和公的!(这个说明是给一个不懂畜牧业的城里人听)}  
 \zh{量词}: \textcolor{darkblue}{\textbf{\ipa{v̩˧}}}  \mytextsc{clf}~: \textcolor{darkblue}{\textbf{\ipa{v̩˧}}} 
\lhead{\firstmark}
\rhead{\botmark}

\subsection{\hspace{-0.5cm} {\Large \textcolor{darkblue}{\textbf{\ipa{dɑ˧pɤ˧}}}}\hspace{0.5cm}[\kern2pt{\textcolor{darkblue}{\textbf{\ipa{dɑ˧pɤ˧}}}}\kern2pt]} \hypertarget{dA\string_Mp7\string_M1}{}
\markboth{\textcolor{darkblue}{\textbf{\ipa{dɑ˧pɤ˧}}}}{}
\textcolor{teal}{\mytextsc{nom}} \hspace{4pt} Ton~: M.
\textcolor{PineGreen}{\selectlanguage{french}Prêtre de la religion locale.} \zh{宗教礼师。音译:达巴。}  ¶ \textcolor{darkblue}{\textbf{\ipa{dɑ˧pɤ˧ ʝi˧-hĩ˧ hĩ˧}}} \textcolor{PineGreen}{\selectlanguage{french}prêtre, personne qui joue le rôle de prêtre/qui est prêtre} \zh{当达巴的人}  
 \zh{量词}: \textcolor{darkblue}{\textbf{\ipa{v̩˧}}}  \mytextsc{clf}~: \textcolor{darkblue}{\textbf{\ipa{v̩˧}}} 
\lhead{\firstmark}
\rhead{\botmark}

\subsection{\hspace{-0.5cm} {\Large \textcolor{darkblue}{\textbf{\ipa{dɑ˧pv̩\#˥}}}}\hspace{0.5cm}[\kern2pt{\textcolor{darkblue}{\textbf{\ipa{dɑ˧pv̩˧}}}}\kern2pt]} \hypertarget{dA\string_Mpv\string_=\#\string_T1}{}
\markboth{\textcolor{darkblue}{\textbf{\ipa{dɑ˧pv̩\#˥}}}}{}
\textcolor{teal}{\mytextsc{nom}} \hspace{4pt} Ton~: \#H.
\textcolor{PineGreen}{\selectlanguage{french}Maître de maison, hôte (personne qui accueille).} \zh{主人。}  ¶ \textcolor{darkblue}{\textbf{\ipa{ʑi˧dv̩˧ dɑ˧pv̩˧}}} \textcolor{PineGreen}{\selectlanguage{french}l'hôte de la maison} \zh{家的主人}  
 ¶ \textcolor{darkblue}{\textbf{\ipa{ʑi˧dv̩˧-ʝi˧-hĩ˧ dɑ˧pv̩˧}}} \textcolor{PineGreen}{\selectlanguage{french}idem} \zh{同上}  
 \zh{量词}: \textcolor{darkblue}{\textbf{\ipa{v̩˧}}}  \mytextsc{clf}~: \textcolor{darkblue}{\textbf{\ipa{v̩˧}}} 
\lhead{\firstmark}
\rhead{\botmark}

\subsection{\hspace{-0.5cm} {\Large \textcolor{darkblue}{\textbf{\ipa{dɑ˧pʰo˥}}}}\hspace{0.5cm}[\kern2pt{\textcolor{darkblue}{\textbf{\ipa{dɑ˩pʰo˥}}}}\kern2pt]} \hypertarget{dA\string_Mp\string_ho\string_T1}{}
\markboth{\textcolor{darkblue}{\textbf{\ipa{dɑ˧pʰo˥}}}}{}
\textcolor{teal}{\mytextsc{nom}} \hspace{4pt} Ton~: LH.
\textcolor{PineGreen}{\selectlanguage{french}Dapo (nom de village).} \zh{达坡(永宁的一个村落)。}  ¶ \textcolor{darkblue}{\textbf{\ipa{ɖæ˩ʂɯ\#˥, | ʈʂo˧ʂɯ\#˥, | bɤ˩tɕʰɯ˩˥, | dɑ˧pʰo˥, | bɤ˧dzi˩, | dze˧bo˧}}} \textcolor{PineGreen}{\selectlanguage{french}les six villages de la plaine de Yongning, dans l'ordre, qui prend comme point d'origine le village le plus proche du Lac} \zh{永宁坝的六个村落,按传统排序:从距离泸沽湖最近的村落说起。}  

\lhead{\firstmark}
\rhead{\botmark}

\subsection{\hspace{-0.5cm} {\Large \textcolor{darkblue}{\textbf{\ipa{dɑ˧ʁwɤ\#˥}}}}\hspace{0.5cm}[\kern2pt{\textcolor{darkblue}{\textbf{\ipa{dɑ˧ʁwɤ˧}}}}\kern2pt]} \hypertarget{dA\string_MRw7\#\string_T1}{}
\markboth{\textcolor{darkblue}{\textbf{\ipa{dɑ˧ʁwɤ\#˥}}}}{}
\textcolor{teal}{\mytextsc{nom}} \hspace{4pt} Ton~: \#H.
\textcolor{PineGreen}{\selectlanguage{french}Un village en aval de Qiansuo; la langue parlée là-bas serait relativement proche de celle de la plaine de Yongning.} \zh{一个村落,在前所的下游。据说那边的方言跟丽江坝比较接近。} 
\lhead{\firstmark}
\rhead{\botmark}

\subsection{\hspace{-0.5cm} {\Large \textcolor{darkblue}{\textbf{\ipa{dɑ˩}}}}\hspace{0.5cm}[\kern2pt{\textcolor{darkblue}{\textbf{\ipa{dɑ˩˥}}}}\kern2pt]} \hypertarget{dA\string_B1}{}
\markboth{\textcolor{darkblue}{\textbf{\ipa{dɑ˩}}}}{}
\textcolor{teal}{\mytextsc{adjectif}} \hspace{4pt} Ton~: L.
\textit{\textcolor{PineGreen}{\selectlanguage{french}archaïque}} [\zh{古语}] \textcolor{PineGreen}{\selectlanguage{french}Heureux.} \zh{幸福、平安、安好。}  ¶ \textcolor{darkblue}{\textbf{\ipa{mɤ˧-dɑ˩-qʰwɤ˩}}} \textcolor{PineGreen}{\selectlanguage{french}chanson mélancolique, récit des malheurs} \zh{悲情歌,讲述自己日子痛苦}  
 ¶ \textcolor{darkblue}{\textbf{\ipa{mɤ˧-dɑ˩!}}} \textcolor{PineGreen}{\selectlanguage{french}Formule employée en début de chanson mélancolique, et parfois au début d'un conte. (La même formule est en usage dans la langue lazé.)} \zh{悲情歌的开头词}  
 ¶ \textcolor{darkblue}{\textbf{\ipa{mɤ˧-dɑ˩-mi˩}}} \textcolor{PineGreen}{\selectlanguage{french}Comme ci-dessus: même sens que la forme sans suffixe \zh{/-mi/}.} \zh{同上}  
 ¶ \textcolor{darkblue}{\textbf{\ipa{ɖwæ˧˥ | hɤ˩-dɑ˥! | ɖwæ˧˥ | hɤ˩˥!}}} \textcolor{PineGreen}{\selectlanguage{french}Bravo, bravo! (Contexte: compliment saluant l'exploit d'une petite fille parvenue à grimper sur un meuble.)} \zh{很了不起啊!(情景:表扬一个小孩子成功地爬上了一个家具)}  
 ¶ \textcolor{darkblue}{\textbf{\ipa{ɖʐɯ˩dɑ˥-kʰɤ˩dɑ˩-ɻ̍˩}}} \textcolor{PineGreen}{\selectlanguage{french}Tout va bien, tout est pour le mieux. (S'emploie par exemple pour décrire une période sans disette, ni tremblement de terre, ni épidémie, ni guerre ou autre catastrophe.)} \zh{一切都安好。(如:来指一段时间没有饥荒、地震、流行病、战争等灾难)}  

\lhead{\firstmark}
\rhead{\botmark}

\subsection{\hspace{-0.5cm} {\Large \textcolor{darkblue}{\textbf{\ipa{dɑ˩\textsubscript{b}}}}}\hspace{0.5cm}[\kern2pt{\textcolor{darkblue}{\textbf{\ipa{dɑ˩˥}}}}\kern2pt]} \hypertarget{dA\string_Bb1}{}
\markboth{\textcolor{darkblue}{\textbf{\ipa{dɑ˩\textsubscript{b}}}}}{}
\textcolor{teal}{\mytextsc{verbe}} \hspace{4pt} Ton~: L\textsubscript{b}.
\textcolor{PineGreen}{\selectlanguage{french}Tisser.} \zh{织。}  ¶ \textcolor{darkblue}{\textbf{\ipa{ɣɯ˧ dɑ˩}}} \textcolor{PineGreen}{\selectlanguage{french}tisser du tissu} \zh{织布}  
 ¶ \textcolor{darkblue}{\textbf{\ipa{ɣɯ˧ | le˧-dɑ˩}}} \textcolor{PineGreen}{\selectlanguage{french}tisser du tissu} \zh{织布}  
 ¶ \textcolor{darkblue}{\textbf{\ipa{tso˧\textasciitilde{}tso˧ dɑ˩}}} \textcolor{PineGreen}{\selectlanguage{french}tisser des choses} \zh{织东西}  
 ¶ \textcolor{darkblue}{\textbf{\ipa{ɖɯ˧-dɑ˧\textasciitilde{}dɑ˩-ɻ̍˩}}} \textcolor{PineGreen}{\selectlanguage{french}\mytextsc{délimitatif} \string_ \mytextsc{red} \mytextsc{inchoatif}} \zh{织一下}  

\lhead{\firstmark}
\rhead{\botmark}

\subsection{\hspace{-0.5cm} {\Large \textcolor{darkblue}{\textbf{\ipa{dɑ˩kʰɤ˩}}}}\hspace{0.5cm}[\kern2pt{\textcolor{darkblue}{\textbf{\ipa{dɑ˩kʰɤ˩˥}}}}\kern2pt]} \hypertarget{dA\string_Bk\string_h7\string_B1}{}
\markboth{\textcolor{darkblue}{\textbf{\ipa{dɑ˩kʰɤ˩}}}}{}
\textcolor{teal}{\mytextsc{nom}} \hspace{4pt} Ton~: L.
\textcolor{PineGreen}{\selectlanguage{french}Tambour.} \zh{鼓。}  ¶ \textcolor{darkblue}{\textbf{\ipa{dɑ˩kʰɤ˩ lɑ˥(-ze˩)}}} \textcolor{PineGreen}{\selectlanguage{french}jouer du tambour} \zh{打鼓}  
 \zh{量词}: \textcolor{darkblue}{\textbf{\ipa{ɭɯ˧}}}  \mytextsc{clf}~: \textcolor{darkblue}{\textbf{\ipa{ɭɯ˧}}} 
\lhead{\firstmark}
\rhead{\botmark}

\subsection{\hspace{-0.5cm} {\Large \textcolor{darkblue}{\textbf{\ipa{dɑ˩to\#˥}}}}\hspace{0.5cm}[\kern2pt{\textcolor{darkblue}{\textbf{\ipa{dɑ˩to˥}}}}\kern2pt]} \hypertarget{dA\string_Bto\#\string_T1}{}
\markboth{\textcolor{darkblue}{\textbf{\ipa{dɑ˩to\#˥}}}}{}
\textcolor{teal}{\mytextsc{adverbe}} \hspace{4pt} Ton~: LM+\#H.
\textcolor{PineGreen}{\selectlanguage{french}Poliment.} \zh{客气地。}  ¶ \textcolor{darkblue}{\textbf{\ipa{dɑ˩to˧ ʐwɤ˧˥}}} \textcolor{PineGreen}{\selectlanguage{french}faire des politesses} \zh{说客气话}  
 ¶ \textcolor{darkblue}{\textbf{\ipa{dɑ˩to˧ ʐwɤ˧-hĩ˥-lɑ˩ ɲi˩!}}} \textcolor{PineGreen}{\selectlanguage{french}C'est juste pour être poli! / C'est juste une façon de dire! (Commentaire de quelqu'un au sujet d'une invitation lancée par un voisin, qui est une simple politesse et pas une vraie invitation; il convient de la décliner.)} \zh{这只是客气话而已!}  

\lhead{\firstmark}
\rhead{\botmark}

\subsection{\hspace{-0.5cm} {\Large \textcolor{darkblue}{\textbf{\ipa{dɑ˩to˩}}}}\hspace{0.5cm}[\kern2pt{\textcolor{darkblue}{\textbf{\ipa{dɑ˩to˩˥}}}}\kern2pt]} \hypertarget{dA\string_Bto\string_B1}{}
\markboth{\textcolor{darkblue}{\textbf{\ipa{dɑ˩to˩}}}}{}
\textcolor{teal}{\mytextsc{adverbe}} \hspace{4pt} Ton~: L.
\textcolor{PineGreen}{\selectlanguage{french}Au fond, en réalité, en définitive.} \zh{说到底,根本上,归根结底。} 
\lhead{\firstmark}
\rhead{\botmark}

\subsection{\hspace{-0.5cm} {\Large \textcolor{darkblue}{\textbf{\ipa{dɑ˧˥}}}}\hspace{0.5cm}[\kern2pt{\textcolor{darkblue}{\textbf{\ipa{dɑ˧˥}}}}\kern2pt]} \hypertarget{dA\string_M\string_T1}{}
\markboth{\textcolor{darkblue}{\textbf{\ipa{dɑ˧˥}}}}{}
\textcolor{teal}{\mytextsc{nom}} \hspace{4pt} Ton~: MH.
\textcolor{PineGreen}{\selectlanguage{french}Infortune, malheur.} Dialecte chinois local~:\zh{苦。} ¶ \textcolor{darkblue}{\textbf{\ipa{dɑ˧-ʐwɤ˧˥}}} \textcolor{PineGreen}{\selectlanguage{french}se plaindre de son infortune, gémir sur son sort} \zh{诉苦}  
 ¶ \textcolor{darkblue}{\textbf{\ipa{ɻ̃˧-ʐwɤ˧ | dɑ˧-ʐwɤ˧-ɻ̍˥}}} \textcolor{PineGreen}{\selectlanguage{french}raconter ses malheurs; se plaindre} \zh{讲自己的不幸}  
 ¶ \textcolor{darkblue}{\textbf{\ipa{ʈʂʰɯ˧ | mɑ˧dɑ˩-qʰwɤ˩, | ɻ̃˧-ʐwɤ˧ | dɑ˧-ʐwɤ˧-ɻ̍˥!}}} \textcolor{PineGreen}{\selectlanguage{french}Il est malheureux; il passe son temps à se plaindre!} \zh{他不幸福,他一直在诉苦!}  

\lhead{\firstmark}
\rhead{\botmark}

\subsection{\hspace{-0.5cm} {\Large \textcolor{darkblue}{\textbf{\ipa{dɑ˧˥}}} \textsubscript{1}}\hspace{0.5cm}[\kern2pt{\textcolor{darkblue}{\textbf{\ipa{dɑ˧˥}}}}\kern2pt]} \hypertarget{dA\string_M\string_T1}{}
\markboth{\textcolor{darkblue}{\textbf{\ipa{dɑ˧˥}}} \textsubscript{1}}{}
\textcolor{teal}{\mytextsc{verbe}} \hspace{4pt} Ton~: MH.
\textcolor{PineGreen}{\selectlanguage{french}Construire (une maison…).} \zh{建(房子)。}  ¶ \textcolor{darkblue}{\textbf{\ipa{ʑi˧mi˧ dɑ˧˥}}} \textcolor{PineGreen}{\selectlanguage{french}construire une maison} \zh{建房}  

\lhead{\firstmark}
\rhead{\botmark}

\subsection{\hspace{-0.5cm} {\Large \textcolor{darkblue}{\textbf{\ipa{dɑ˧˥}}} \textsubscript{2}}\hspace{0.5cm}[\kern2pt{\textcolor{darkblue}{\textbf{\ipa{dɑ˧˥}}}}\kern2pt]} \hypertarget{dA\string_M\string_T2}{}
\markboth{\textcolor{darkblue}{\textbf{\ipa{dɑ˧˥}}} \textsubscript{2}}{}
\textcolor{teal}{\mytextsc{verbe}} \hspace{4pt} Ton~: MH.
\textcolor{PineGreen}{\selectlanguage{french}Couper un arbre, abattre un arbre; ouvrir une brèche (dans une digue).} \zh{砍(树),割(肉)。}  ¶ \textcolor{darkblue}{\textbf{\ipa{le˧-dɑ˧-ze˥}}} \textcolor{PineGreen}{\selectlanguage{french}\mytextsc{accomp} \string_ \mytextsc{pfv}} \zh{砍了(树),割了(肉)}  
 ¶ \textcolor{darkblue}{\textbf{\ipa{ɖɯ˧-dɑ˧ tʰi˥-dɑ˩}}} \textcolor{PineGreen}{\selectlanguage{french}donner un coup} \zh{砍一下}  
 ¶ \textcolor{darkblue}{\textbf{\ipa{dɑ˩\textasciitilde{}dɑ˧˥}}} \textcolor{PineGreen}{\selectlanguage{french}\mytextsc{red}} \zh{\mytextsc{重叠}}  
 ¶ \textcolor{darkblue}{\textbf{\ipa{le˧-dɑ˩\textasciitilde{}dɑ˩(-ze˩)}}} \textcolor{PineGreen}{\selectlanguage{french}(j'ai) découpé (ex.: le poulet) en morceaux} \zh{(我把一只鸡)割成块了}  
 ¶ \textcolor{darkblue}{\textbf{\ipa{ʂe˧ dɑ˥\textasciitilde{}dɑ˩}}} \textcolor{PineGreen}{\selectlanguage{french}hacher de la viande, couper de la viande en morceaux} \zh{把肉剁碎}  

\lhead{\firstmark}
\rhead{\botmark}

\subsection{\hspace{-0.5cm} {\Large \textcolor{darkblue}{\textbf{\ipa{dɑ˧˥\textsubscript{b}}}}}\hspace{0.5cm}[\kern2pt{\textcolor{darkblue}{\textbf{\ipa{dɑ˧˥}}}}\kern2pt]} \hypertarget{dA\string_M\string_Tb1}{}
\markboth{\textcolor{darkblue}{\textbf{\ipa{dɑ˧˥\textsubscript{b}}}}}{}
\textcolor{teal}{\mytextsc{classificateur}} \hspace{4pt} Ton~: MH\textsubscript{b}.
\textcolor{PineGreen}{\selectlanguage{french}Auto-classificateur des coups.} \zh{量词:下(打一下)。}  ¶ \textcolor{darkblue}{\textbf{\ipa{ɖɯ˧-dɑ˧˥}}} \textcolor{PineGreen}{\selectlanguage{french}un coup} \zh{当头一棒}  
 ¶ \textcolor{darkblue}{\textbf{\ipa{ɖɯ˧-dɑ˧ tʰi˥-dɑ˩}}} \textcolor{PineGreen}{\selectlanguage{french}donner un coup} \zh{打一下}  

\lhead{\firstmark}
\rhead{\botmark}

\subsection{\hspace{-0.5cm} {\Large \textcolor{darkblue}{\textbf{\ipa{dɤ˧-qo˧}}}}\hspace{0.5cm}[\kern2pt{\textcolor{darkblue}{\textbf{\ipa{xxxx non-correspondance entre le nombre de morphèmes et le nombre de tons de morphèmes}}}}\kern2pt]} \hypertarget{d7\string_M-qo\string_M1}{}
\markboth{\textcolor{darkblue}{\textbf{\ipa{dɤ˧-qo˧}}}}{}
\textcolor{teal}{\mytextsc{adverbe}} \hspace{4pt} Ton~: M.
\textcolor{PineGreen}{\selectlanguage{french}Par là-bas tout au loin, tout au loin là-bas.} \zh{那里(远指)。} 
\lhead{\firstmark}
\rhead{\botmark}

\subsection{\hspace{-0.5cm} {\Large \textcolor{darkblue}{\textbf{\ipa{dɤ˧-tʰv̩˧-gi\#˥}}}}\hspace{0.5cm}[\kern2pt{\textcolor{darkblue}{\textbf{\ipa{xxxx non-correspondance entre le nombre de morphèmes et le nombre de tons de morphèmes}}}}\kern2pt]} \hypertarget{d7\string_M-t\string_hv\string_=\string_M-gi\#\string_T1}{}
\markboth{\textcolor{darkblue}{\textbf{\ipa{dɤ˧-tʰv̩˧-gi\#˥}}}}{}
\textcolor{teal}{\mytextsc{adverbe}} \hspace{4pt} Ton~: \#H.
\textcolor{PineGreen}{\selectlanguage{french}Au loin, de ce côté-là.} \zh{那边(远指)。} 
\lhead{\firstmark}
\rhead{\botmark}

\subsection{\hspace{-0.5cm} {\Large \textcolor{darkblue}{\textbf{\ipa{dɤ˧-tʰv̩˧qo˧}}}}\hspace{0.5cm}[\kern2pt{\textcolor{darkblue}{\textbf{\ipa{xxxx non-correspondance entre le nombre de morphèmes et le nombre de tons de morphèmes}}}}\kern2pt]} \hypertarget{d7\string_M-t\string_hv\string_=\string_Mqo\string_M1}{}
\markboth{\textcolor{darkblue}{\textbf{\ipa{dɤ˧-tʰv̩˧qo˧}}}}{}
\textcolor{teal}{\mytextsc{adverbe}} \hspace{4pt} Ton~: M.
\textcolor{PineGreen}{\selectlanguage{french}Par là-bas tout au loin.} \zh{那边(远指)。} \textit{Voir~:} \hyperlink{}{\textcolor{darkblue}{\textbf{\ipa{dɤ˧-ʈʂʰɯ˧qo˧}}}} 
\lhead{\firstmark}
\rhead{\botmark}

\subsection{\hspace{-0.5cm} {\Large \textcolor{darkblue}{\textbf{\ipa{dɤ˧-ʈʂʰɯ˧qo˧}}}}\hspace{0.5cm}[\kern2pt{\textcolor{darkblue}{\textbf{\ipa{xxxx non-correspondance entre le nombre de morphèmes et le nombre de tons de morphèmes}}}}\kern2pt]} \hypertarget{d7\string_M-t`s`\string_hM\string_Mqo\string_M1}{}
\markboth{\textcolor{darkblue}{\textbf{\ipa{dɤ˧-ʈʂʰɯ˧qo˧}}}}{}
\textcolor{teal}{\mytextsc{adverbe}} \hspace{4pt} Ton~: M.
\textcolor{PineGreen}{\selectlanguage{french}Par là-bas tout au loin.} \zh{那边(远指)。} \textit{Voir~:} \hyperlink{}{\textcolor{darkblue}{\textbf{\ipa{dɤ˧-tʰv̩˧qo˧}}}} 
\lhead{\firstmark}
\rhead{\botmark}

\subsection{\hspace{-0.5cm} {\Large \textcolor{darkblue}{\textbf{\ipa{di˧mi˧}}}}\hspace{0.5cm}[\kern2pt{\textcolor{darkblue}{\textbf{\ipa{di˧mi˧}}}}\kern2pt]} \hypertarget{di\string_Mmi\string_M1}{}
\markboth{\textcolor{darkblue}{\textbf{\ipa{di˧mi˧}}}}{}
\textcolor{teal}{\mytextsc{nom}} \hspace{4pt} Ton~: M.
\textcolor{PineGreen}{\selectlanguage{french}Grande plaine.} \zh{平坝。}  ¶ \textcolor{darkblue}{\textbf{\ipa{ɬi˧di˩-di˩mi˩}}} \textcolor{PineGreen}{\selectlanguage{french}la plaine de Yongning} \zh{永宁坝}  
 \zh{量词}: \textcolor{darkblue}{\textbf{\ipa{di˩}}}  \mytextsc{clf}~: \textcolor{darkblue}{\textbf{\ipa{di˩}}} 
\lhead{\firstmark}
\rhead{\botmark}

\subsection{\hspace{-0.5cm} {\Large \textcolor{darkblue}{\textbf{\ipa{di˧qo˧}}}}\hspace{0.5cm}[\kern2pt{\textcolor{darkblue}{\textbf{\ipa{di˧qo˧}}}}\kern2pt]} \hypertarget{di\string_Mqo\string_M1}{}
\markboth{\textcolor{darkblue}{\textbf{\ipa{di˧qo˧}}}}{}
\textcolor{teal}{\mytextsc{nom}} \hspace{4pt} Ton~: M.
\textcolor{PineGreen}{\selectlanguage{french}Plaine.} \zh{平坝。}  \zh{量词}: \textcolor{darkblue}{\textbf{\ipa{di˩}}}  \mytextsc{clf}~: \textcolor{darkblue}{\textbf{\ipa{di˩}}} 
\lhead{\firstmark}
\rhead{\botmark}

\subsection{\hspace{-0.5cm} {\Large \textcolor{darkblue}{\textbf{\ipa{di˧ɻæ˧}}}}\hspace{0.5cm}[\kern2pt{\textcolor{darkblue}{\textbf{\ipa{di˧ɻæ˧}}}}\kern2pt]} \hypertarget{di\string_Mr£`\{\string_M1}{}
\markboth{\textcolor{darkblue}{\textbf{\ipa{di˧ɻæ˧}}}}{}
\textcolor{teal}{\mytextsc{nom}} \hspace{4pt} Ton~: M.
\textcolor{PineGreen}{\selectlanguage{french}Plaine.} \zh{平坝。}  \zh{量词}: \textcolor{darkblue}{\textbf{\ipa{di˩}}}  \mytextsc{clf}~: \textcolor{darkblue}{\textbf{\ipa{di˩}}} 
\lhead{\firstmark}
\rhead{\botmark}

\subsection{\hspace{-0.5cm} {\Large \textcolor{darkblue}{\textbf{\ipa{‑di˩}}}}\hspace{0.5cm}[\kern2pt{\textcolor{darkblue}{\textbf{\ipa{di˩˥}}}}\kern2pt]} \hypertarget{‑di\string_B1}{}
\markboth{\textcolor{darkblue}{\textbf{\ipa{‑di˩}}}}{}
\textcolor{teal}{\mytextsc{suffixe}} \hspace{4pt} Ton~: L.
\textcolor{PineGreen}{\selectlanguage{french}Nominalisateur; locatif; purposif.} \zh{\mytextsc{名物化}/\mytextsc{处所格}/\mytextsc{目的格。}}  ¶ \textcolor{darkblue}{\textbf{\ipa{tso˧\textasciitilde{}tso˧-tɕɯ˧-di˧˥}}} \textcolor{PineGreen}{\selectlanguage{french}(meuble/objet) sur lequel on pose des choses} \zh{可以摆东西的(家具)}  

\lhead{\firstmark}
\rhead{\botmark}

\subsection{\hspace{-0.5cm} {\Large \textcolor{darkblue}{\textbf{\ipa{di˩\textsubscript{c}}}}}\hspace{0.5cm}[\kern2pt{\textcolor{darkblue}{\textbf{\ipa{di˩˥}}}}\kern2pt]} \hypertarget{di\string_Bc1}{}
\markboth{\textcolor{darkblue}{\textbf{\ipa{di˩\textsubscript{c}}}}}{}
\textcolor{teal}{\mytextsc{classificateur}} \hspace{4pt} Ton~: L\textsubscript{c}.
\textcolor{PineGreen}{\selectlanguage{french}Classificateur pour les plaines, les étendues de terre, les lieux.} \zh{量词:坝子、地方(一个)。}  ¶ \textcolor{darkblue}{\textbf{\ipa{ɖɯ˧-v̩˧ | ɖɯ˧-di˩ hɯ˩}}} \textcolor{PineGreen}{\selectlanguage{french}se séparer, partir chacun de son côté (par exemple: des frères se séparent et chacun va son chemin)} \zh{分开,每个人去不同的地方}  

\lhead{\firstmark}
\rhead{\botmark}

\subsection{\hspace{-0.5cm} {\Large \textcolor{darkblue}{\textbf{\ipa{di˩\textsubscript{a}}}}}\hspace{0.5cm}[\kern2pt{\textcolor{darkblue}{\textbf{\ipa{di˩˥}}}}\kern2pt]} \hypertarget{di\string_Ba1}{}
\markboth{\textcolor{darkblue}{\textbf{\ipa{di˩\textsubscript{a}}}}}{}
\textcolor{teal}{\mytextsc{verbe}} \hspace{4pt} Ton~: L\textsubscript{a}.
\textcolor{PineGreen}{\selectlanguage{french}Existentiel; posséder. Possession non amovible aussi bien que transitoire: avoir une maison, aussi bien que: avoir une tache de graisse sur la joue; avoir une différence de longueur (deux objets ont une différence de longueur).} \zh{存在动词:有,拥有。例如:有家,有污垢在衣服上,有长短区别(两个物品有长短区别)。}  ¶ \textcolor{darkblue}{\textbf{\ipa{ʈʰɯ˧ | ɑ˩ʁo˧ mɤ˧-di˩-hĩ˩.}}} \textcolor{PineGreen}{\selectlanguage{french}Elle/il n'a pas de maison, elle/il est sans domicile} \zh{他没有家。}  
 ¶ \textcolor{darkblue}{\textbf{\ipa{mɤ˧ tʰi˧-di˩}}} \textcolor{PineGreen}{\selectlanguage{french}il y a de la graisse (ex.: autour de la bouche de quelqu'un qui vient de croquer de la viande à belles dents)} \zh{有油(例如:一个人的嘴巴周围油乎乎,有油)}  
 ¶ \textcolor{darkblue}{\textbf{\ipa{ɖɯ˧-kʰwɤ˧ di˥}}} \textcolor{PineGreen}{\selectlanguage{french}il y a quelque chose} \zh{有一块东西}  

\lhead{\firstmark}
\rhead{\botmark}

\subsection{\hspace{-0.5cm} {\Large \textcolor{darkblue}{\textbf{\ipa{di˩-gɤ˩lɑ˥}}}}\hspace{0.5cm}[\kern2pt{\textcolor{darkblue}{\textbf{\ipa{xxxx non-correspondance entre le nombre de morphèmes et le nombre de tons de morphèmes}}}}\kern2pt]} \hypertarget{di\string_B-g7\string_BlA\string_T1}{}
\markboth{\textcolor{darkblue}{\textbf{\ipa{di˩-gɤ˩lɑ˥}}}}{}
\textcolor{teal}{\mytextsc{nom}} \hspace{4pt} Ton~: L+H\#.
\textcolor{PineGreen}{\selectlanguage{french}Esprit de la terre, Bodhisattva terrestre.} \zh{地菩萨。}  \zh{量词}: \textcolor{darkblue}{\textbf{\ipa{v̩˧}}}  \mytextsc{clf}~: \textcolor{darkblue}{\textbf{\ipa{v̩˧}}} 
\lhead{\firstmark}
\rhead{\botmark}

\subsection{\hspace{-0.5cm} {\Large \textcolor{darkblue}{\textbf{\ipa{di˩li˩}}}}\hspace{0.5cm}[\kern2pt{\textcolor{darkblue}{\textbf{\ipa{di˩li˩˥}}}}\kern2pt]} \hypertarget{di\string_Bli\string_B1}{}
\markboth{\textcolor{darkblue}{\textbf{\ipa{di˩li˩}}}}{}
\textcolor{teal}{\mytextsc{nom}} \hspace{4pt} Ton~: L.
\textcolor{PineGreen}{\selectlanguage{french}Ottélie: une plante verte comestible: \textit{Ottelia Acuminata, Boottia acuminata, Ottelia yunnanensis}.} \zh{海菜花。} Dialecte chinois local~:\zh{龙爪菜。} ¶ \textcolor{darkblue}{\textbf{\ipa{di˩li˩-ʁo˩bv̩˥ (ton: L+H\#)}}} \textcolor{PineGreen}{\selectlanguage{french}pousses d'ottélie} \zh{海菜花的萌芽}  
 ¶ \textcolor{darkblue}{\textbf{\ipa{di˩li˩-ʁo˩bv̩˥ hwæ˩}}} \textcolor{PineGreen}{\selectlanguage{french}acheter des pousses d'ottélie (verbe au ton M)} \zh{买海菜花萌芽}  
 ¶ \textcolor{darkblue}{\textbf{\ipa{di˩li˩-ʁo˩bv̩˥ tɕʰi˩}}} \textcolor{PineGreen}{\selectlanguage{french}vendre des pousses d'ottélie} \zh{卖海菜花萌芽}  
 ¶ \textcolor{darkblue}{\textbf{\ipa{di˩li˩-ʁo˩bv̩˥ dzɯ˩}}} \textcolor{PineGreen}{\selectlanguage{french}manger des pousses d'ottélie} \zh{吃海菜花萌芽}  
 ¶ \textcolor{darkblue}{\textbf{\ipa{di˩li˩-ʁo˩bv̩˥ dze˩}}} \textcolor{PineGreen}{\selectlanguage{french}couper des pousses d'ottélie} \zh{割海菜花萌芽}  
 ¶ \textcolor{darkblue}{\textbf{\ipa{di˩li˩-ʁo˩bv̩˥ tɕɤ˩}}} \textcolor{PineGreen}{\selectlanguage{french}faire bouillir des pousses d'ottélie} \zh{煮海菜花萌芽}  
 \zh{量词}: \textcolor{darkblue}{\textbf{\ipa{qɑ˩}}}  \mytextsc{clf}~: \textcolor{darkblue}{\textbf{\ipa{qɑ˩}}} 
\lhead{\firstmark}
\rhead{\botmark}

\subsection{\hspace{-0.5cm} {\Large \textcolor{darkblue}{\textbf{\ipa{di˧˥}}} \textsubscript{1}}\hspace{0.5cm}[\kern2pt{\textcolor{darkblue}{\textbf{\ipa{di˧˥}}}}\kern2pt]} \hypertarget{di\string_M\string_T1}{}
\markboth{\textcolor{darkblue}{\textbf{\ipa{di˧˥}}} \textsubscript{1}}{}
\textcolor{teal}{\mytextsc{verbe}} \hspace{4pt} Ton~: MH.
\textcolor{PineGreen}{\selectlanguage{french}Poursuivre, chasser; disperser, repousser, faire déguerpir.} \zh{打散,驱赶,撵,赶,打猎。}  ¶ \textcolor{darkblue}{\textbf{\ipa{tɕʰɯ˩di˩˥}}} \textcolor{PineGreen}{\selectlanguage{french}chasser le muntjac; chasser} \zh{赶麂子,狩猎}  
 ¶ \textcolor{darkblue}{\textbf{\ipa{tɕʰɯ˩di˩-bi˩-ni˩gv̩˩˥}}} \textcolor{PineGreen}{\selectlanguage{french}avoir l'habitude de chasser} \zh{有打猎的习惯、喜欢打猎}  
 ¶ \textcolor{darkblue}{\textbf{\ipa{di˩\textasciitilde{}di˧˥ / di˩\textasciitilde{}di˧-ze˥}}} \textcolor{PineGreen}{\selectlanguage{french}\mytextsc{red}: suivre à la trace, pister} \zh{\mytextsc{重叠:跟着、追着}}  
 ¶ \textcolor{darkblue}{\textbf{\ipa{tʰi˧-di˩\textasciitilde{}di˩}}} \textcolor{PineGreen}{\selectlanguage{french}\mytextsc{dur} \mytextsc{red}} \zh{\mytextsc{dur} \mytextsc{red}}  

\lhead{\firstmark}
\rhead{\botmark}

\subsection{\hspace{-0.5cm} {\Large \textcolor{darkblue}{\textbf{\ipa{di˧˥}}} \textsubscript{2}}\hspace{0.5cm}[\kern2pt{\textcolor{darkblue}{\textbf{\ipa{di˧˥}}}}\kern2pt]} \hypertarget{di\string_M\string_T2}{}
\markboth{\textcolor{darkblue}{\textbf{\ipa{di˧˥}}} \textsubscript{2}}{}
\textcolor{teal}{\mytextsc{verbe}} \hspace{4pt} Ton~: MH.
\textcolor{PineGreen}{\selectlanguage{french}S'écouler, couler; avoir la courante = avoir la diarrhée.} \zh{拉(肚子)。}  ¶ \textcolor{darkblue}{\textbf{\ipa{bi˧mi˧ di˧˥}}} \textcolor{PineGreen}{\selectlanguage{french}avoir la diarrhée} \zh{拉肚子}  

\lhead{\firstmark}
\rhead{\botmark}

\subsection{\hspace{-0.5cm} {\Large \textcolor{darkblue}{\textbf{\ipa{di˩˥}}}}\hspace{0.5cm}[\kern2pt{\textcolor{darkblue}{\textbf{\ipa{di˩˥}}}}\kern2pt]} \hypertarget{di\string_B\string_T1}{}
\markboth{\textcolor{darkblue}{\textbf{\ipa{di˩˥}}}}{}
\textcolor{teal}{\mytextsc{nom}} \hspace{4pt} Ton~: LH.
\textcolor{PineGreen}{\selectlanguage{french}Terre (le ciel et la terre).} \zh{地(天地的地)。}  ¶ \textcolor{darkblue}{\textbf{\ipa{di˩ dv̩˩-ze˥}}} \textcolor{PineGreen}{\selectlanguage{french}creuser la terre} \zh{挖土}  
 ¶ \textcolor{darkblue}{\textbf{\ipa{di˩ hwæ˧-ze˩}}} \textcolor{PineGreen}{\selectlanguage{french}a acheté de la terre} \zh{买了土}  
 ¶ \textcolor{darkblue}{\textbf{\ipa{ɖɯ˧-di˩ ɖɯ˩-bæ˩!}}} \textcolor{PineGreen}{\selectlanguage{french}C'est différent à chaque endroit/chaque endroit a ses choses propres (par exemple, en matière de langues, chaque village a sa prononciation, son dialecte)} \zh{一个地方,一个样! = 每个地方有自己的特色(如:每个村落有自己的摩梭方言/土语)}  
 \zh{量词}: \textcolor{darkblue}{\textbf{\ipa{di˩}}}  \mytextsc{clf}~: \textcolor{darkblue}{\textbf{\ipa{di˩}}} 
\lhead{\firstmark}
\rhead{\botmark}

\subsection{\hspace{-0.5cm} {\Large \textcolor{darkblue}{\textbf{\ipa{do˥}}} \textsubscript{1}}\hspace{0.5cm}[\kern2pt{\textcolor{darkblue}{\textbf{\ipa{do˥}}}}\kern2pt]} \hypertarget{do\string_T1}{}
\markboth{\textcolor{darkblue}{\textbf{\ipa{do˥}}} \textsubscript{1}}{}
\textcolor{teal}{\mytextsc{verbe}} \hspace{4pt} Ton~: H.
\textcolor{PineGreen}{\selectlanguage{french}Grimper, monter, escalader, gravir.} \zh{爬,上去,上山。}  ¶ \textcolor{darkblue}{\textbf{\ipa{ʈʂo˩bo˩ do˩˥}}} \textcolor{PineGreen}{\selectlanguage{french}grimper un mur, gravir un mur} \zh{爬墙}  
 ¶ \textcolor{darkblue}{\textbf{\ipa{gɤ˩-do˧}}} \textcolor{PineGreen}{\selectlanguage{french}grimper, gravir} \zh{爬上}  
 ¶ \textcolor{darkblue}{\textbf{\ipa{ʁwɤ˩ do˩˥}}} \textcolor{PineGreen}{\selectlanguage{french}gravir une montagne, grimper la montagne, faire de la montagne} \zh{爬山}  
 ¶ \textcolor{darkblue}{\textbf{\ipa{to˩ do˩˥}}} \textcolor{PineGreen}{\selectlanguage{french}grimper la pente/ grimper une pente} \zh{爬上一个山坡}  

\lhead{\firstmark}
\rhead{\botmark}

\subsection{\hspace{-0.5cm} {\Large \textcolor{darkblue}{\textbf{\ipa{do˥}}} \textsubscript{2}}\hspace{0.5cm}[\kern2pt{\textcolor{darkblue}{\textbf{\ipa{do˥}}}}\kern2pt]} \hypertarget{do\string_T2}{}
\markboth{\textcolor{darkblue}{\textbf{\ipa{do˥}}} \textsubscript{2}}{}
\textcolor{teal}{\mytextsc{verbe}} \hspace{4pt} Ton~: H.
\textcolor{PineGreen}{\selectlanguage{french}S'accoupler (animaux).} \zh{交配、交尾。}  ¶ \textcolor{darkblue}{\textbf{\ipa{bo˩ɬɑ˥ | bo˩mi˧ do˧}}} \textcolor{PineGreen}{\selectlanguage{french}Le verrat s'accouple avec la truie.} \zh{公猪与母猪交配。}  

\lhead{\firstmark}
\rhead{\botmark}

\subsection{\hspace{-0.5cm} {\Large \textcolor{darkblue}{\textbf{\ipa{do˥\textsubscript{a}}}}}\hspace{0.5cm}[\kern2pt{\textcolor{darkblue}{\textbf{\ipa{do˥}}}}\kern2pt]} \hypertarget{do\string_Ta1}{}
\markboth{\textcolor{darkblue}{\textbf{\ipa{do˥\textsubscript{a}}}}}{}
\textcolor{teal}{\mytextsc{classificateur}} \hspace{4pt} Ton~: H\textsubscript{a}.
\textcolor{PineGreen}{\selectlanguage{french}Classificateur des cloisons et murs.} \zh{\mytextsc{量词}.墙壁(一堵)。}  ¶ \textcolor{darkblue}{\textbf{\ipa{ʈʂʰɯ˧-do\#˥}}} \textcolor{PineGreen}{\selectlanguage{french}cette cloison} \zh{这堵(墙壁)}  

\lhead{\firstmark}
\rhead{\botmark}

\subsection{\hspace{-0.5cm} {\Large \textcolor{darkblue}{\textbf{\ipa{do˧}}} \textsubscript{1}}\hspace{0.5cm}[\kern2pt{\textcolor{darkblue}{\textbf{\ipa{do˥}}}}\kern2pt]} \hypertarget{do\string_M1}{}
\markboth{\textcolor{darkblue}{\textbf{\ipa{do˧}}} \textsubscript{1}}{}
\textcolor{teal}{\mytextsc{adjectif}} \hspace{4pt} Ton~: M.
\textcolor{PineGreen}{\selectlanguage{french}Bête, stupide.} \zh{笨、愚蠢。}  ¶ \textcolor{darkblue}{\textbf{\ipa{zo˩ do˩˥}}} \textcolor{PineGreen}{\selectlanguage{french}un idiot, un fou du village; perçu comme: “un homme qui n'a pas grandi”, “quelqu'un qui est resté enfant”} \zh{傻瓜}  

\lhead{\firstmark}
\rhead{\botmark}

\subsection{\hspace{-0.5cm} {\Large \textcolor{darkblue}{\textbf{\ipa{do˧}}} \textsubscript{2}}\hspace{0.5cm}[\kern2pt{\textcolor{darkblue}{\textbf{\ipa{do˥}}}}\kern2pt]} \hypertarget{do\string_M2}{}
\markboth{\textcolor{darkblue}{\textbf{\ipa{do˧}}} \textsubscript{2}}{}
\textcolor{teal}{\mytextsc{adjectif}} \hspace{4pt} Ton~: M.
\textcolor{PineGreen}{\selectlanguage{french}Stérile.} \zh{不能生育。} 
\lhead{\firstmark}
\rhead{\botmark}

\subsection{\hspace{-0.5cm} {\Large \textcolor{darkblue}{\textbf{\ipa{do˧bæ˧}}}}\hspace{0.5cm}[\kern2pt{\textcolor{darkblue}{\textbf{\ipa{do˧bæ˧}}}}\kern2pt]} \hypertarget{do\string_Mb\{\string_M1}{}
\markboth{\textcolor{darkblue}{\textbf{\ipa{do˧bæ˧}}}}{}
\textcolor{teal}{\mytextsc{nom}} \hspace{4pt} Ton~: M.
\textcolor{PineGreen}{\selectlanguage{french}Cuisse.} \zh{大腿。}  ¶ \textcolor{darkblue}{\textbf{\ipa{do˧bæ˧ | ɖɯ˩-hĩ˩˥}}} \textcolor{PineGreen}{\selectlanguage{french}cuisse} \zh{大腿}  
 ¶ \textcolor{darkblue}{\textbf{\ipa{do˧bæ˧ | tɕi˩-hĩ˩˥}}} \textcolor{PineGreen}{\selectlanguage{french}mollet} \zh{小腿}  
 \zh{量词}: \textcolor{darkblue}{\textbf{\ipa{ʈv̩˩}}}  \mytextsc{clf}~: \textcolor{darkblue}{\textbf{\ipa{ʈv̩˩}}} 
\lhead{\firstmark}
\rhead{\botmark}

\subsection{\hspace{-0.5cm} {\Large \textcolor{darkblue}{\textbf{\ipa{do˧bv̩˧}}}}\hspace{0.5cm}[\kern2pt{\textcolor{darkblue}{\textbf{\ipa{do˧bv̩˧}}}}\kern2pt]} \hypertarget{do\string_Mbv\string_=\string_M1}{}
\markboth{\textcolor{darkblue}{\textbf{\ipa{do˧bv̩˧}}}}{}
\textcolor{teal}{\mytextsc{nom}} \hspace{4pt} Ton~: M.
\textcolor{PineGreen}{\selectlanguage{french}Fesse.} \zh{屁股。}  \zh{量词}: \textcolor{darkblue}{\textbf{\ipa{ɭɯ˧}}}  \mytextsc{clf}~: \textcolor{darkblue}{\textbf{\ipa{ɭɯ˧}}} 
\lhead{\firstmark}
\rhead{\botmark}

\subsection{\hspace{-0.5cm} {\Large \textcolor{darkblue}{\textbf{\ipa{do˩}}}}\hspace{0.5cm}[\kern2pt{\textcolor{darkblue}{\textbf{\ipa{do˩˥}}}}\kern2pt]} \hypertarget{do\string_B1}{}
\markboth{\textcolor{darkblue}{\textbf{\ipa{do˩}}}}{}
\textcolor{teal}{\mytextsc{adjectif}} \hspace{4pt} Ton~: L.
\textit{\textcolor{PineGreen}{\selectlanguage{french}archaïque}} [\zh{古语}] \textcolor{PineGreen}{\selectlanguage{french}Immature.} \zh{不成熟、晚熟。}  ¶ \textcolor{darkblue}{\textbf{\ipa{ŋwɤ˩ɬi˩-mi˩˥, | ʂe˧ mɤ˧-mv̩˥, | ʂe˧ do˧˥! | tsʰe˩ŋwɤ˩ kʰv̩˥, | zo˧ mɤ˧-ti˩, | zo˧ do˧˥!}}} \textcolor{PineGreen}{\selectlanguage{french}Au cinquième mois, une céréale qui ne mûrit pas/qui ne donne pas de grain, c'est une récolte stérile/qui reste en herbe! A quinze ans, le garçon qui n'a pas encore acquis de maturité (=qui ne fréquente pas encore les filles), c'est qu'il a un problème/c'est un attardé!} \zh{五月份,谷物还是小草(还不出谷粒),算是晚熟!男人十五岁还不成熟(=还不见姑娘),算是晚熟!}  

\lhead{\firstmark}
\rhead{\botmark}

\subsection{\hspace{-0.5cm} {\Large \textcolor{darkblue}{\textbf{\ipa{do˩\textsubscript{b}}}}}\hspace{0.5cm}[\kern2pt{\textcolor{darkblue}{\textbf{\ipa{do˩˥}}}}\kern2pt]} \hypertarget{do\string_Bb1}{}
\markboth{\textcolor{darkblue}{\textbf{\ipa{do˩\textsubscript{b}}}}}{}
\textcolor{teal}{\mytextsc{verbe}} \hspace{4pt} Ton~: L\textsubscript{b}.
\textcolor{PineGreen}{\selectlanguage{french}Voir, apercevoir.} \zh{看见,遇见,见。}  ¶ \textcolor{darkblue}{\textbf{\ipa{ɖɯ˧-do˥\textasciitilde{}do˩-ɻ̍˩}}} \textcolor{PineGreen}{\selectlanguage{french}\mytextsc{délimitatif} \string_ \mytextsc{red} \mytextsc{inchoatif}} \zh{见一见}  
 ¶ \textcolor{darkblue}{\textbf{\ipa{ɖɯ˧-kʰwɤ˧ do˧˥}}} \textcolor{PineGreen}{\selectlanguage{french}apercevoir un bout/un morceau} \zh{看见一块(东西)}  
 ¶ \textcolor{darkblue}{\textbf{\ipa{tso˧\textasciitilde{}tso˧ do˧˥}}} \textcolor{PineGreen}{\selectlanguage{french}apercevoir des choses/apercevoir quelque chose} \zh{看见东西}  
 ¶ \textcolor{darkblue}{\textbf{\ipa{do˩-mɤ˩-ho˥}}} \textcolor{PineGreen}{\selectlanguage{french}\string_ \mytextsc{neg} \mytextsc{désidératif}} \zh{不许(看)见}  
 ¶ \textcolor{darkblue}{\textbf{\ipa{bo˩mi˧ do˩ (+ze˩)}}} \textcolor{PineGreen}{\selectlanguage{french}...a vu (une/la) truie} \zh{看见了母猪}  

\lhead{\firstmark}
\rhead{\botmark}

\subsection{\hspace{-0.5cm} {\Large \textcolor{darkblue}{\textbf{\ipa{do˩bv̩\#˥}}}}\hspace{0.5cm}[\kern2pt{\textcolor{darkblue}{\textbf{\ipa{do˩bv̩˥}}}}\kern2pt]} \hypertarget{do\string_Bbv\string_=\#\string_T1}{}
\markboth{\textcolor{darkblue}{\textbf{\ipa{do˩bv̩\#˥}}}}{}
\textcolor{teal}{\mytextsc{nom}} \hspace{4pt} Ton~: LM+\#H.
\textcolor{PineGreen}{\selectlanguage{french}Mur de mani (le nom désigne l'ensemble du mur, pas seulement une des tablettes qui s'y trouvent). Le mur de mani est un mur de pierre sèche et de sable, comportant des tablettes de pierre sur lesquelles est gravé une inscription: le plus souvent Om Mani Padme Hum. Un mur de mani doit être contourné dans le sens des aiguilles d'une montre: le sens de rotation de l'univers, selon la doctrine bouddhiste.} \zh{嘛呢堆。}  \zh{量词}: \textcolor{darkblue}{\textbf{\ipa{ɭɯ˧}}}  \mytextsc{clf}~: \textcolor{darkblue}{\textbf{\ipa{ɭɯ˧}}} \textit{Voir~:} \textcolor{darkblue}{\textbf{\ipa{mɑ˩ɳɯ˧-do˥bv˩, mɑ˩ɳɯ\#˥}}} 
\lhead{\firstmark}
\rhead{\botmark}

\subsection{\hspace{-0.5cm} {\Large \textcolor{darkblue}{\textbf{\ipa{do˩kv̩\#˥}}}}\hspace{0.5cm}[\kern2pt{\textcolor{darkblue}{\textbf{\ipa{do˩kv̩˥}}}}\kern2pt]} \hypertarget{do\string_Bkv\string_=\#\string_T1}{}
\markboth{\textcolor{darkblue}{\textbf{\ipa{do˩kv̩\#˥}}}}{}
\textcolor{teal}{\mytextsc{nom}} \hspace{4pt} Ton~: LM+\#H.
\textcolor{PineGreen}{\selectlanguage{french}Poutrelles soutenant le plancher du premier étage.} \zh{小梁子,作为楼上(第二层)木地板的底。}  \zh{量词}: \textcolor{darkblue}{\textbf{\ipa{ɭɯ˧}}}  \mytextsc{clf}~: \textcolor{darkblue}{\textbf{\ipa{ɭɯ˧}}} 
\lhead{\firstmark}
\rhead{\botmark}

\subsection{\hspace{-0.5cm} {\Large \textcolor{darkblue}{\textbf{\ipa{dv̩˩}}} \textsubscript{1}}\hspace{0.5cm}[\kern2pt{\textcolor{darkblue}{\textbf{\ipa{dv̩˩˥}}}}\kern2pt]} \hypertarget{dv\string_=\string_B1}{}
\markboth{\textcolor{darkblue}{\textbf{\ipa{dv̩˩}}} \textsubscript{1}}{}
\textcolor{teal}{\mytextsc{classificateur}} \hspace{4pt} Ton~: L *.
\textcolor{PineGreen}{\selectlanguage{french}Classificateur des troupeaux; ne s'utilise qu'au singulier.} \zh{量词:人、牲畜(一群、一队)。} \textit{Voir~:} \hyperlink{}{\textcolor{darkblue}{\textbf{\ipa{dɤ˧-tʰv̩˧-gi\#˥}}}} 
\lhead{\firstmark}
\rhead{\botmark}

\subsection{\hspace{-0.5cm} {\Large \textcolor{darkblue}{\textbf{\ipa{dv̩˩}}} \textsubscript{2}}\hspace{0.5cm}[\kern2pt{\textcolor{darkblue}{\textbf{\ipa{dv̩˩˥}}}}\kern2pt]} \hypertarget{dv\string_=\string_B2}{}
\markboth{\textcolor{darkblue}{\textbf{\ipa{dv̩˩}}} \textsubscript{2}}{}
\textcolor{teal}{\mytextsc{pronom}} \hspace{4pt} Ton~: L?.
\textcolor{PineGreen}{\selectlanguage{french}Démonstratif distal, qui apparaît dans l'indication de direction “par ici, dans cette direction”.} \zh{指示代词:那边(远指),从‘那个方向’这个短语提取出来的。}  ¶ \textcolor{darkblue}{\textbf{\ipa{dv̩˩-tɕo˧}}} \textcolor{PineGreen}{\selectlanguage{french}cette direction-là} \zh{那个方向}  
 ¶ \textcolor{darkblue}{\textbf{\ipa{dv̩˩tɕo˧ fæ˧}}} \textcolor{PineGreen}{\selectlanguage{french}cette direction-là} \zh{那个方向}  
\textit{Voir~:} \hyperlink{}{\textcolor{darkblue}{\textbf{\ipa{dɤ˧-tʰv̩˧-gi\#˥}}}} 
\lhead{\firstmark}
\rhead{\botmark}

\subsection{\hspace{-0.5cm} {\Large \textcolor{darkblue}{\textbf{\ipa{dv̩˩˧}}} \textsubscript{1}}\hspace{0.5cm}[\kern2pt{\textcolor{darkblue}{\textbf{\ipa{dv̩˩˥}}}}\kern2pt]} \hypertarget{dv\string_=\string_B\string_M1}{}
\markboth{\textcolor{darkblue}{\textbf{\ipa{dv̩˩˧}}} \textsubscript{1}}{}
\textcolor{teal}{\mytextsc{nom}} \hspace{4pt} Ton~: LM.
\textcolor{PineGreen}{\selectlanguage{french}Belette.} \zh{黄鼠狼,黄喉貂。}  ¶ \textcolor{darkblue}{\textbf{\ipa{dv̩˩ hwæ˧-ze˧}}} \textcolor{PineGreen}{\selectlanguage{french}...a acheté une belette} \zh{买了黄鼠狼}  
 ¶ \textcolor{darkblue}{\textbf{\ipa{dv̩˩ dzɯ˧-ze˩}}} \textcolor{PineGreen}{\selectlanguage{french}...a mangé une belette} \zh{吃了黄鼠狼}  
 \zh{量词}: \textcolor{darkblue}{\textbf{\ipa{mi˩}}}  \mytextsc{clf}~: \textcolor{darkblue}{\textbf{\ipa{mi˩}}} 
\lhead{\firstmark}
\rhead{\botmark}

\subsection{\hspace{-0.5cm} {\Large \textcolor{darkblue}{\textbf{\ipa{dv̩˩˧}}} \textsubscript{2}}\hspace{0.5cm}[\kern2pt{\textcolor{darkblue}{\textbf{\ipa{dv̩˩˥}}}}\kern2pt]} \hypertarget{dv\string_=\string_B\string_M2}{}
\markboth{\textcolor{darkblue}{\textbf{\ipa{dv̩˩˧}}} \textsubscript{2}}{}
\textcolor{teal}{\mytextsc{nom}} \hspace{4pt} Ton~: LM.
\textcolor{PineGreen}{\selectlanguage{french}Poison.} \zh{毒。} \textit{Voir~:} \hyperlink{}{\textcolor{darkblue}{\textbf{\ipa{dv̩˩\textsubscript{a}}}}} 
\lhead{\firstmark}
\rhead{\botmark}

\subsection{\hspace{-0.5cm} {\Large \textcolor{darkblue}{\textbf{\ipa{dv̩˥}}}}\hspace{0.5cm}[\kern2pt{\textcolor{darkblue}{\textbf{\ipa{dv̩˥}}}}\kern2pt]} \hypertarget{dv\string_=\string_T1}{}
\markboth{\textcolor{darkblue}{\textbf{\ipa{dv̩˥}}}}{}
\textcolor{teal}{\mytextsc{verbe}} \hspace{4pt} Ton~: H.
\textcolor{PineGreen}{\selectlanguage{french}Creuser.} \zh{挖。}  ¶ \textcolor{darkblue}{\textbf{\ipa{ʈʂe˧ dv̩˧(-ze˩)}}} \textcolor{PineGreen}{\selectlanguage{french}piocher la terre, creuser la terre} \zh{挖土}  

\lhead{\firstmark}
\rhead{\botmark}

\subsection{\hspace{-0.5cm} {\Large \textcolor{darkblue}{\textbf{\ipa{dv̩˩\textsubscript{a}}}}}\hspace{0.5cm}[\kern2pt{\textcolor{darkblue}{\textbf{\ipa{dv̩˩˥}}}}\kern2pt]} \hypertarget{dv\string_=\string_Ba1}{}
\markboth{\textcolor{darkblue}{\textbf{\ipa{dv̩˩\textsubscript{a}}}}}{}
\textcolor{teal}{\mytextsc{verbe}} \hspace{4pt} Ton~: L\textsubscript{a}.
\ding{202} \textcolor{PineGreen}{\selectlanguage{french}Empoisonner, rendre malade.} \zh{毒害、毒化。}  ¶ \textcolor{darkblue}{\textbf{\ipa{ʈʂʰɯ˧, | hĩ˧ dv̩˥-mɤ˩-kv̩˩! | ʈʂʰɯ˧, | hĩ˧ dv̩˥-kv̩˩!}}} \textcolor{PineGreen}{\selectlanguage{french}Celui-ci, il n'est pas dangereux / il est comestible! Celui-là [en revanche], il est vénéneux / il est dangereux / il peut vous rendre malade / il peut vous empoisonner / il n'est pas comestible! (Au sujet de diverses sortes de champignons.)} \zh{这个,不会让人中毒!那个(反倒)会让人中毒!(情景:谈不同菌子种类。)}  
 ¶ \textcolor{darkblue}{\textbf{\ipa{ʈʂʰɯ˧, | dv̩˩-mɤ˩-kv̩˥!}}} \textcolor{PineGreen}{\selectlanguage{french}C'est inoffensif/comestible/pas vénéneux/pas dangereux! (Au sujet d'une sorte de champignon.)} \zh{这个,不会让人中毒!(情景:谈不同菌子种类。)}  
\ding{203} \textcolor{PineGreen}{\selectlanguage{french}Détester.} \zh{讨厌、恨。}  ¶ \textcolor{darkblue}{\textbf{\ipa{le˧-dv̩˩-ze˩}}} \textcolor{PineGreen}{\selectlanguage{french}\mytextsc{accomp} \string_ \mytextsc{pfv}}  
 ¶ \textcolor{darkblue}{\textbf{\ipa{njɤ˧ | ʈʂʰɯ˧ dv̩˥ | ʐwæ˩˥!}}} \textcolor{PineGreen}{\selectlanguage{french}je le déteste!} \zh{我很讨厌他!}  
 ¶ \textcolor{darkblue}{\textbf{\ipa{dv̩˩-zo˧-mɤ˧-tʰɑ˧˥}}} \textcolor{PineGreen}{\selectlanguage{french}détester à mort} \zh{讨厌得不行}  
\textit{Voir~:} \hyperlink{}{\textcolor{darkblue}{\textbf{\ipa{dv̩˩˧}}} \textsubscript{2}} 
\lhead{\firstmark}
\rhead{\botmark}

\subsection{\hspace{-0.5cm} {\Large \textcolor{darkblue}{\textbf{\ipa{dv̩˩\textsubscript{b}}}}}\hspace{0.5cm}[\kern2pt{\textcolor{darkblue}{\textbf{\ipa{dv̩˩˥}}}}\kern2pt]} \hypertarget{dv\string_=\string_Bb1}{}
\markboth{\textcolor{darkblue}{\textbf{\ipa{dv̩˩\textsubscript{b}}}}}{}
\textcolor{teal}{\mytextsc{classificateur}} \hspace{4pt} Ton~: L\textsubscript{b}.
\textcolor{PineGreen}{\selectlanguage{french}Classificateur des petits groupes (de personnes): quelques-uns (plus de 3).} \zh{量词:人(一些)。}  ¶ \textcolor{darkblue}{\textbf{\ipa{hĩ˧ ɖɯ˧-dv̩˩}}} \textcolor{PineGreen}{\selectlanguage{french}quelques personnes, un groupe de personnes} \zh{一些人}  
 ¶ \textcolor{darkblue}{\textbf{\ipa{hĩ˧ ʈʂʰɯ˧-dv̩˥}}} \textcolor{PineGreen}{\selectlanguage{french}ce groupe de gens, ces qq personnes} \zh{这些人}  

\lhead{\firstmark}
\rhead{\botmark}

\subsection{\hspace{-0.5cm} {\Large \textcolor{darkblue}{\textbf{\ipa{dv̩˩bi˩}}}}\hspace{0.5cm}[\kern2pt{\textcolor{darkblue}{\textbf{\ipa{dv̩˩bi˩˥}}}}\kern2pt]} \hypertarget{dv\string_=\string_Bbi\string_B1}{}
\markboth{\textcolor{darkblue}{\textbf{\ipa{dv̩˩bi˩}}}}{}
\textcolor{teal}{\mytextsc{adverbe}} \hspace{4pt} Ton~: L.
\textcolor{PineGreen}{\selectlanguage{french}En face.} \zh{对面。} 
\lhead{\firstmark}
\rhead{\botmark}

\subsection{\hspace{-0.5cm} {\Large \textcolor{darkblue}{\textbf{\ipa{dv̩˩mi\#˥}}}}\hspace{0.5cm}[\kern2pt{\textcolor{darkblue}{\textbf{\ipa{dv̩˩mi˩˥}}}}\kern2pt]} \hypertarget{dv\string_=\string_Bmi\#\string_T1}{}
\markboth{\textcolor{darkblue}{\textbf{\ipa{dv̩˩mi\#˥}}}}{}
\textcolor{teal}{\mytextsc{nom}} \hspace{4pt} Ton~: LM+\#H.
\textcolor{PineGreen}{\selectlanguage{french}Belette femelle.} \zh{母黄鼠狼。}  ¶ \textcolor{darkblue}{\textbf{\ipa{dv̩˩mi˧-dv̩˥pʰv̩˩}}} \textcolor{PineGreen}{\selectlanguage{french}belette femelle et belette mâle} \zh{母黄鼠狼与公黄鼠狼}  
 \zh{量词}: \textcolor{darkblue}{\textbf{\ipa{mi˩}}}  \mytextsc{clf}~: \textcolor{darkblue}{\textbf{\ipa{mi˩}}} 
\lhead{\firstmark}
\rhead{\botmark}

\subsection{\hspace{-0.5cm} {\Large \textcolor{darkblue}{\textbf{\ipa{dv̩˩pʰæ˧}}}}\hspace{0.5cm}[\kern2pt{\textcolor{darkblue}{\textbf{\ipa{dv̩˩pʰæ˥}}}}\kern2pt]} \hypertarget{dv\string_=\string_Bp\string_h\{\string_M1}{}
\markboth{\textcolor{darkblue}{\textbf{\ipa{dv̩˩pʰæ˧}}}}{}
\textcolor{teal}{\mytextsc{nom}} \hspace{4pt} Ton~: LM.
\textcolor{PineGreen}{\selectlanguage{french}Partie du bâtiment principal dans laquelle étaient conservées les céréales: le grenier à céréales.} \zh{仓廪。摩梭话音译:‘独帕’。}  \zh{量词}: \textcolor{darkblue}{\textbf{\ipa{ɭɯ˧}}}  \mytextsc{clf}~: \textcolor{darkblue}{\textbf{\ipa{ɭɯ˧}}} 
\lhead{\firstmark}
\rhead{\botmark}

\subsection{\hspace{-0.5cm} {\Large \textcolor{darkblue}{\textbf{\ipa{dv̩˩pʰv̩\#˥}}}}\hspace{0.5cm}[\kern2pt{\textcolor{darkblue}{\textbf{\ipa{dv̩˩pʰv̩˥}}}}\kern2pt]} \hypertarget{dv\string_=\string_Bp\string_hv\string_=\#\string_T1}{}
\markboth{\textcolor{darkblue}{\textbf{\ipa{dv̩˩pʰv̩\#˥}}}}{}
\textcolor{teal}{\mytextsc{nom}} \hspace{4pt} Ton~: LM+\#H / LM.
\textcolor{PineGreen}{\selectlanguage{french}Belette mâle.} \zh{公黄鼠狼。}  \zh{量词}: \textcolor{darkblue}{\textbf{\ipa{mi˩}}}  \mytextsc{clf}~: \textcolor{darkblue}{\textbf{\ipa{mi˩}}} 
\lhead{\firstmark}
\rhead{\botmark}

\subsection{\hspace{-0.5cm} {\Large \textcolor{darkblue}{\textbf{\ipa{dv̩˩zo\#˥}}}}\hspace{0.5cm}[\kern2pt{\textcolor{darkblue}{\textbf{\ipa{dv̩˧zo˧}}}}\kern2pt]} \hypertarget{dv\string_=\string_Bzo\#\string_T1}{}
\markboth{\textcolor{darkblue}{\textbf{\ipa{dv̩˩zo\#˥}}}}{}
\textcolor{teal}{\mytextsc{nom}} \hspace{4pt} Ton~: LM+\#H / LM.
\textcolor{PineGreen}{\selectlanguage{french}Bébé belette.} \zh{黄鼠狼的崽子。} 
\lhead{\firstmark}
\rhead{\botmark}

\newpage
\section*{\centering- \textcolor{darkblue}{\textbf{\ipa{dz}}} -}
\subsection{\hspace{-0.5cm} {\Large \textcolor{darkblue}{\textbf{\ipa{dzɑ˥}}}}\hspace{0.5cm}[\kern2pt{\textcolor{darkblue}{\textbf{\ipa{dzɑ˥}}}}\kern2pt]} \hypertarget{dzA\string_T1}{}
\markboth{\textcolor{darkblue}{\textbf{\ipa{dzɑ˥}}}}{}
\textcolor{teal}{\mytextsc{adjectif}} \hspace{4pt} Ton~: H.
\ding{202} \textcolor{PineGreen}{\selectlanguage{french}Mauvais (action…), inférieur, indigne.} \zh{坏、差、下级(行为……)。}  ¶ \textcolor{darkblue}{\textbf{\ipa{ʈʂʰɯ˧-ɳɯ˧ | njɤ˧-ki˧ | dzɑ˧-ʝi˧ | ʐwæ˩˥!}}} \textcolor{PineGreen}{\selectlanguage{french}il me méprise vraiment!} \zh{他很瞧不起我!}  
 ¶ \textcolor{darkblue}{\textbf{\ipa{hĩ˧ ʈʂʰɯ˧-v̩˧ dʑo˩, | õ˧-ki˥ | dzɑ˧-ʝi˧-ze˩!}}} \textcolor{PineGreen}{\selectlanguage{french}cette personne ne se respecte pas!} \zh{这个人,不尊重自己!}  
 ¶ \textcolor{darkblue}{\textbf{\ipa{mv̩˧ dzɑ˧.}}} \textcolor{PineGreen}{\selectlanguage{french}Il fait mauvais temps.} \zh{天气很坏。}  
 ¶ \textcolor{darkblue}{\textbf{\ipa{mv̩˧ dzɑ˧-ze˩}}} \textcolor{PineGreen}{\selectlanguage{french}Le temps se met au mauvais, il commence à faire mauvais temps.} \zh{天气变坏了。}  
 ¶ \textcolor{darkblue}{\textbf{\ipa{lo˧ dzɑ˧}}} \textcolor{PineGreen}{\selectlanguage{french}bâclé, mal fait (travail)} \zh{(工作)差}  
\ding{203} \textcolor{PineGreen}{\selectlanguage{french}Indigent, pauvre (personne…).} \zh{穷(人)。}  ¶ \textcolor{darkblue}{\textbf{\ipa{dzɑ˧ | -ʐwæ˩-ze˥!}}} \textcolor{PineGreen}{\selectlanguage{french}(il/elle est) très pauvre!} \zh{他很穷!}  
 ¶ \textcolor{darkblue}{\textbf{\ipa{ɑ˩ʁo˧ | bo˩ʈʂʰæ˧ mɤ˧-dʑo˧, | dzɑ˧ ʈʂɤ˧-kv̩˩!}}} \textcolor{PineGreen}{\selectlanguage{french}Ne pas avoir de cochon-entier-conservé à la maison, ça fait vraiment mauvais effet/ça fait vraiment indigent/c'est la honte!} \zh{如果家里没有猪膘,会显得很穷!}  
 ¶ \textcolor{darkblue}{\textbf{\ipa{dzɑ˧ ʈʂɤ˧ | ʐwæ˩˥!}}} \textcolor{PineGreen}{\selectlanguage{french}C'est vraiment la honte/on paraît vraiment à la rue! (Au sujet de situations stigmatisées socialement, comme de ne pas posséder les nourritures ou vêtement nécessaires aux principaux rituels.)} \zh{真羞耻啊!}  

\lhead{\firstmark}
\rhead{\botmark}

\subsection{\hspace{-0.5cm} {\Large \textcolor{darkblue}{\textbf{\ipa{dzɑ˩qʰwɤ˩}}}}\hspace{0.5cm}[\kern2pt{\textcolor{darkblue}{\textbf{\ipa{dzɑ˧qʰwɤ˥}}}}\kern2pt]} \hypertarget{dzA\string_Bq\string_hw7\string_B1}{}
\markboth{\textcolor{darkblue}{\textbf{\ipa{dzɑ˩qʰwɤ˩}}}}{}
\textcolor{teal}{\mytextsc{nom}} \hspace{4pt} Ton~: L.
\textcolor{PineGreen}{\selectlanguage{french}Chaussure.} \zh{鞋、鞋子。}  \zh{量词}: \textcolor{darkblue}{\textbf{\ipa{dzi˧}}}  \mytextsc{clf}~: \textcolor{darkblue}{\textbf{\ipa{dzi˧}}} 
\lhead{\firstmark}
\rhead{\botmark}

\subsection{\hspace{-0.5cm} {\Large \textcolor{darkblue}{\textbf{\ipa{dze˥}}}}\hspace{0.5cm}[\kern2pt{\textcolor{darkblue}{\textbf{\ipa{dze˩˥}}}}\kern2pt]} \hypertarget{dze\string_T1}{}
\markboth{\textcolor{darkblue}{\textbf{\ipa{dze˥}}}}{}
\textcolor{teal}{\mytextsc{nom}} \hspace{4pt} Ton~: \#H.
\textcolor{PineGreen}{\selectlanguage{french}Sucre.} \zh{糖。} 
\lhead{\firstmark}
\rhead{\botmark}

\subsection{\hspace{-0.5cm} {\Large \textcolor{darkblue}{\textbf{\ipa{dze˧bɤ˩}}}}\hspace{0.5cm}[\kern2pt{\textcolor{darkblue}{\textbf{\ipa{dze˧bɤ˧}}}}\kern2pt]} \hypertarget{dze\string_Mb7\string_B1}{}
\markboth{\textcolor{darkblue}{\textbf{\ipa{dze˧bɤ˩}}}}{}
\textcolor{teal}{\mytextsc{nom}} \hspace{4pt} Ton~: L\#.
\textcolor{PineGreen}{\selectlanguage{french}Chauve-souris; s'emploie pour toutes les espèces, y compris le renard volant.} \zh{蝙蝠、飞鼠。}  ¶ \textcolor{darkblue}{\textbf{\ipa{dze˧bɤ˩-zo˩ | ɖɯ˧-ɭɯ˧}}} \textcolor{PineGreen}{\selectlanguage{french}une petite chauve-souris, un bébé chauve-souris} \zh{一只小蝙蝠}  
 ¶ \textcolor{darkblue}{\textbf{\ipa{dze˧bɤ˩-pʰv̩˩ | ɖɯ˧-mi˩}}} \textcolor{PineGreen}{\selectlanguage{french}une chauve-souris mâle} \zh{一只公蝙蝠}  
 ¶ \textcolor{darkblue}{\textbf{\ipa{dze˧bɤ˩-mi˩ | ɖɯ˧-mi˩}}} \textcolor{PineGreen}{\selectlanguage{french}une chauve-souris femelle} \zh{一只母蝙蝠}  
 \zh{量词}: \textcolor{darkblue}{\textbf{\ipa{mi˩}}}  \mytextsc{clf}~: \textcolor{darkblue}{\textbf{\ipa{mi˩}}} 
\lhead{\firstmark}
\rhead{\botmark}

\subsection{\hspace{-0.5cm} {\Large \textcolor{darkblue}{\textbf{\ipa{dze˧bo˧}}}}\hspace{0.5cm}[\kern2pt{\textcolor{darkblue}{\textbf{\ipa{dze˩bo˩˥}}}}\kern2pt]} \hypertarget{dze\string_Mbo\string_M1}{}
\markboth{\textcolor{darkblue}{\textbf{\ipa{dze˧bo˧}}}}{}
\textcolor{teal}{\mytextsc{nom}} \hspace{4pt} Ton~: M.
\ding{202} \textcolor{PineGreen}{\selectlanguage{french}Nom de clan/famille étendue qui compte 3 familles. C'est également le nom d'un village de la plaine de Yongning.} \zh{者波(姓)。这个家族有三个家庭。}  ¶ \textcolor{darkblue}{\textbf{\ipa{dze˧bo˧=ɻ̍˩}}} \textcolor{PineGreen}{\selectlanguage{french}le clan \textcolor{darkblue}{\textbf{\ipa{/dze˧bo˧/}}}, la famille \textcolor{darkblue}{\textbf{\ipa{/dze˧bo˧/}}}} \zh{者波家族}  
\ding{203} \textcolor{PineGreen}{\selectlanguage{french}Zhebo, un village de la plaine de Yongning. Il est divisé en deux parties, “du haut” et “du bas”: \textcolor{darkblue}{\textbf{\ipa{/gɤ˩ʁwɤ˧/}}} et \textcolor{darkblue}{\textbf{\ipa{/mv̩˩ʁwɤ˧/}}}.} \zh{者波(永宁的一个村落)。村落有两个部分,\textcolor{darkblue}{\textbf{\ipa{/gɤ˩ʁwɤ˧/}}}‘上村’与\textcolor{darkblue}{\textbf{\ipa{/mv̩˩ʁwɤ˧/}}}‘下村’.}  ¶ \textcolor{darkblue}{\textbf{\ipa{ɖæ˩ʂɯ\#˥, | ʈʂo˧ʂɯ\#˥, | bɤ˩tɕʰɯ˩˥, | dɑ˧pʰo˥, | bɤ˧dzi˩, | dze˧bo˧}}} \textcolor{PineGreen}{\selectlanguage{french}les six villages de la plaine de Yongning, dans l'ordre, qui prend comme point d'origine le village le plus proche du Lac} \zh{永宁坝的六个村落,按传统排序:从距离泸沽湖最近的村落说起。}  

\lhead{\firstmark}
\rhead{\botmark}

\subsection{\hspace{-0.5cm} {\Large \textcolor{darkblue}{\textbf{\ipa{dze˧dv̩˩}}}}\hspace{0.5cm}[\kern2pt{\textcolor{darkblue}{\textbf{\ipa{dze˧dv̩˩}}}}\kern2pt]} \hypertarget{dze\string_Mdv\string_=\string_B1}{}
\markboth{\textcolor{darkblue}{\textbf{\ipa{dze˧dv̩˩}}}}{}
\textcolor{teal}{\mytextsc{nom}} \hspace{4pt} Ton~: L\#.
\textcolor{PineGreen}{\selectlanguage{french}Galette de céréale (blé, avoine…), pain.} \zh{饼。}  ¶ \textcolor{darkblue}{\textbf{\ipa{dze˧dv̩˩-pɤ˩jɤ˩}}} \textcolor{PineGreen}{\selectlanguage{french}galette de céréale (même sens)} \zh{粮食饼}  

\lhead{\firstmark}
\rhead{\botmark}

\subsection{\hspace{-0.5cm} {\Large \textcolor{darkblue}{\textbf{\ipa{dze˧hi˧}}}}\hspace{0.5cm}[\kern2pt{\textcolor{darkblue}{\textbf{\ipa{dze˧hi˧}}}}\kern2pt]} \hypertarget{dze\string_Mhi\string_M1}{}
\markboth{\textcolor{darkblue}{\textbf{\ipa{dze˧hi˧}}}}{}
\textcolor{teal}{\mytextsc{nom}} \hspace{4pt} Ton~: M.
\textcolor{PineGreen}{\selectlanguage{french}Beaux-parents.} \zh{丈人。}  ¶ \textcolor{darkblue}{\textbf{\ipa{njɤ˧ | dze˧hi˧-ki˩ bi˩!}}} \textcolor{PineGreen}{\selectlanguage{french}Je vais chez mes beaux-parents!} \zh{我去我丈人(那边)!}  
 ¶ \textcolor{darkblue}{\textbf{\ipa{no˧ | dze˧hi˧ | ə˩-to˩-ze˥? - le˧-to˩-ze˩!}}} \textcolor{PineGreen}{\selectlanguage{french}Tu as une belle-famille? =Tu es marié(e)? -Oui!} \zh{你有丈人吗?(=你结婚了吗?)-有的!(=结婚了!)}  
 ¶ \textcolor{darkblue}{\textbf{\ipa{no˧ | dze˧hi˧ to˩ ə˩-bi˩?}}} \textcolor{PineGreen}{\selectlanguage{french}Tu comptes te marier? (Littéralement: Tu vas te lier avec une belle-famille?)} \zh{你打算结婚吗?}  

\lhead{\firstmark}
\rhead{\botmark}

\subsection{\hspace{-0.5cm} {\Large \textcolor{darkblue}{\textbf{\ipa{dze˧kʰɤ˧˥}}}}\hspace{0.5cm}[\kern2pt{\textcolor{darkblue}{\textbf{\ipa{dze˧kʰɤ˧}}}}\kern2pt]} \hypertarget{dze\string_Mk\string_h7\string_M\string_T1}{}
\markboth{\textcolor{darkblue}{\textbf{\ipa{dze˧kʰɤ˧˥}}}}{}
\textcolor{teal}{\mytextsc{nom}} \hspace{4pt} Ton~: MH\#.
\textcolor{PineGreen}{\selectlanguage{french}Roturier, la 2e des 3 castes de la société ancienne, intermédiaire entre la noblesse et les serfs.} \zh{百姓。音译:“责卡”。}  \zh{量词}: \textcolor{darkblue}{\textbf{\ipa{v̩˧}}}  \mytextsc{clf}~: \textcolor{darkblue}{\textbf{\ipa{v̩˧}}} 
\lhead{\firstmark}
\rhead{\botmark}

\subsection{\hspace{-0.5cm} {\Large \textcolor{darkblue}{\textbf{\ipa{dze˧ɭɯ˧}}}}\hspace{0.5cm}[\kern2pt{\textcolor{darkblue}{\textbf{\ipa{dze˧ɭɯ˧˥}}}}\kern2pt]} \hypertarget{dze\string_Ml\string_RM\string_M1}{}
\markboth{\textcolor{darkblue}{\textbf{\ipa{dze˧ɭɯ˧}}}}{}
\textcolor{teal}{\mytextsc{nom}} \hspace{4pt} Ton~: M.
\textcolor{PineGreen}{\selectlanguage{french}Blé, froment.} \zh{小麦。} 
\lhead{\firstmark}
\rhead{\botmark}

\subsection{\hspace{-0.5cm} {\Large \textcolor{darkblue}{\textbf{\ipa{dze˧ɭɯ˧-ɻ̃\#˥}}}}\hspace{0.5cm}[\kern2pt{\textcolor{darkblue}{\textbf{\ipa{xxxx non-correspondance entre le nombre de morphèmes et le nombre de tons de morphèmes}}}}\kern2pt]} \hypertarget{dze\string_Ml\string_RM\string_M-r£`\string_~\#\string_T1}{}
\markboth{\textcolor{darkblue}{\textbf{\ipa{dze˧ɭɯ˧-ɻ̃\#˥}}}}{}
\textcolor{teal}{\mytextsc{nom}} \hspace{4pt} Ton~: \#H.
\textcolor{PineGreen}{\selectlanguage{french}Paille de blé.} \zh{麦杆。} 
\lhead{\firstmark}
\rhead{\botmark}

\subsection{\hspace{-0.5cm} {\Large \textcolor{darkblue}{\textbf{\ipa{dze˧-ɻ̃\#˥}}}}\hspace{0.5cm}[\kern2pt{\textcolor{darkblue}{\textbf{\ipa{xxxx non-correspondance entre le nombre de morphèmes et le nombre de tons de morphèmes}}}}\kern2pt]} \hypertarget{dze\string_M-r£`\string_~\#\string_T1}{}
\markboth{\textcolor{darkblue}{\textbf{\ipa{dze˧-ɻ̃\#˥}}}}{}
\textcolor{teal}{\mytextsc{nom}} \hspace{4pt} Ton~: \#H.
\textcolor{PineGreen}{\selectlanguage{french}Paille de blé.} \zh{小麦秆。} 
\lhead{\firstmark}
\rhead{\botmark}

\subsection{\hspace{-0.5cm} {\Large \textcolor{darkblue}{\textbf{\ipa{dze˧-tɕʰi\#˥}}}}\hspace{0.5cm}[\kern2pt{\textcolor{darkblue}{\textbf{\ipa{xxxx non-correspondance entre le nombre de morphèmes et le nombre de tons de morphèmes}}}}\kern2pt]} \hypertarget{dze\string_M-ts£\string_hi\#\string_T1}{}
\markboth{\textcolor{darkblue}{\textbf{\ipa{dze˧-tɕʰi\#˥}}}}{}
\textcolor{teal}{\mytextsc{nom}} \hspace{4pt} Ton~: \#H.
\textcolor{PineGreen}{\selectlanguage{french}Barbe de blé.} \zh{麦芒。} 
\lhead{\firstmark}
\rhead{\botmark}

\subsection{\hspace{-0.5cm} {\Large \textcolor{darkblue}{\textbf{\ipa{dze˧-ʈʂæ˥}}}}\hspace{0.5cm}[\kern2pt{\textcolor{darkblue}{\textbf{\ipa{xxxx non-correspondance entre le nombre de morphèmes et le nombre de tons de morphèmes}}}}\kern2pt]} \hypertarget{dze\string_M-t`s`\{\string_T1}{}
\markboth{\textcolor{darkblue}{\textbf{\ipa{dze˧-ʈʂæ˥}}}}{}
\textcolor{teal}{\mytextsc{nom}} \hspace{4pt} Ton~: H\#.
\textcolor{PineGreen}{\selectlanguage{french}Dard de l'abeille.} \zh{蜜蜂的螫針。}  \zh{量词}: \textcolor{darkblue}{\textbf{\ipa{ɭɯ˧}}}  \mytextsc{clf}~: \textcolor{darkblue}{\textbf{\ipa{ɭɯ˧}}} 
\lhead{\firstmark}
\rhead{\botmark}

\subsection{\hspace{-0.5cm} {\Large \textcolor{darkblue}{\textbf{\ipa{dze˧ʈʂɯ˧}}}}\hspace{0.5cm}[\kern2pt{\textcolor{darkblue}{\textbf{\ipa{dze˧ʈʂɯ˥}}}}\kern2pt]} \hypertarget{dze\string_Mt`s`M\string_M1}{}
\markboth{\textcolor{darkblue}{\textbf{\ipa{dze˧ʈʂɯ˧}}}}{}
\textcolor{teal}{\mytextsc{nom}} \hspace{4pt} Ton~: M.
\textcolor{PineGreen}{\selectlanguage{french}Vanneries: tamis, crible.} \zh{筛子。}  \zh{量词}: \textcolor{darkblue}{\textbf{\ipa{nɑ˧}}}  \mytextsc{clf}~: \textcolor{darkblue}{\textbf{\ipa{nɑ˧}}} 
\lhead{\firstmark}
\rhead{\botmark}

\subsection{\hspace{-0.5cm} {\Large \textcolor{darkblue}{\textbf{\ipa{dze˧ʈʂʰɤ\$˥}}}}\hspace{0.5cm}[\kern2pt{\textcolor{darkblue}{\textbf{\ipa{dze˧ʈʂʰɤ˥}}}}\kern2pt]} \hypertarget{dze\string_Mt`s`\string_h7\$\string_T1}{}
\markboth{\textcolor{darkblue}{\textbf{\ipa{dze˧ʈʂʰɤ\$˥}}}}{}
\textcolor{teal}{\mytextsc{nom}} \hspace{4pt} Ton~: H\$.
\textcolor{PineGreen}{\selectlanguage{french}Céréales; la céréale traditionnelle était l'orge, mais le sens de l'expression tend actuellement à se confondre avec celui de l'expression chinoise \zh{五谷} 'les cinq céréales': le riz; deux sortes de millet; le blé; et les fèves.} \zh{粮食。现在,这个词的含义受到汉语‘五谷’这个词的影响,用来指代‘五谷杂粮’,相当于所有粮食类,如:小米类、稻谷、麦子、玉米以及豆类与薯类。} 
\lhead{\firstmark}
\rhead{\botmark}

\subsection{\hspace{-0.5cm} {\Large \textcolor{darkblue}{\textbf{\ipa{dze˩}}}}\hspace{0.5cm}[\kern2pt{\textcolor{darkblue}{\textbf{\ipa{dze˩˥}}}}\kern2pt]} \hypertarget{dze\string_B1}{}
\markboth{\textcolor{darkblue}{\textbf{\ipa{dze˩}}}}{}
\textcolor{teal}{\mytextsc{verbe}} \hspace{4pt} Ton~: L.
\textcolor{PineGreen}{\selectlanguage{french}Rester, être en trop, devenir un reste (nourriture, boisson).} \zh{剩下(饭或饮料)。}  ¶ \textcolor{darkblue}{\textbf{\ipa{dzɯ˧-dze˥-ze˩!}}} \textcolor{PineGreen}{\selectlanguage{french}il y a des restes! / on n'a pas achevé de manger (un plat)!} \zh{剩了一些饭!/ 剩了一些吃的!}  
 ¶ \textcolor{darkblue}{\textbf{\ipa{gɤ˩-dze˥ +ze˩!}}} \textcolor{PineGreen}{\selectlanguage{french}Il en reste / il y a des restes!} \zh{有剩下的!}  
 ¶ \textcolor{darkblue}{\textbf{\ipa{ʈʰɯ˩ dze˩-ze˥}}} \textcolor{PineGreen}{\selectlanguage{french}Il en reste / on n'a pas achevé de boire (un verre…); ne pas être entièrement bu} \zh{喝剩了、没喝完}  
 ¶ \textcolor{darkblue}{\textbf{\ipa{le˧-se˩-ze˩! | gɤ˩-mɤ˧-dze˩!}}} \textcolor{PineGreen}{\selectlanguage{french}on a tout fini, il n'y a pas de restes! / tout a été mangé, bu..., il n'en reste plus !} \zh{完了!(=全部吃/喝完了!)没有剩!}  

\lhead{\firstmark}
\rhead{\botmark}

\subsection{\hspace{-0.5cm} {\Large \textcolor{darkblue}{\textbf{\ipa{dze˩\textsubscript{a}}}}}\hspace{0.5cm}[\kern2pt{\textcolor{darkblue}{\textbf{\ipa{dze˥}}}}\kern2pt]} \hypertarget{dze\string_Ba1}{}
\markboth{\textcolor{darkblue}{\textbf{\ipa{dze˩\textsubscript{a}}}}}{}
\textcolor{teal}{\mytextsc{classificateur}} \hspace{4pt} Ton~: L\textsubscript{a}.
\textcolor{PineGreen}{\selectlanguage{french}Classificateur des lots de deux objets non indissociables; par ex.: lot de 2 casseroles; paire de haut-parleurs… Pour les paires non dissociables (ex.: paire de chaussures), on utilise: /dzi˧/.} \zh{量词:瓶子、锅(一对)。}  ¶ \textcolor{darkblue}{\textbf{\ipa{zo˧mv̩˥ | ɖɯ˧-dze˩}}} \textcolor{PineGreen}{\selectlanguage{french}des jumeaux (littéralement “une paire d'enfants”)} \zh{双胞胎(直译:“一对孩子”)}  
 ¶ \textcolor{darkblue}{\textbf{\ipa{ʈʂʰɯ˧-dze˥}}} \textcolor{PineGreen}{\selectlanguage{french}\mytextsc{dem} \string_ (ton: H\# / H\$)} \zh{\mytextsc{指示代词} \string_}  

\lhead{\firstmark}
\rhead{\botmark}

\subsection{\hspace{-0.5cm} {\Large \textcolor{darkblue}{\textbf{\ipa{dze˩\textsubscript{a}}}} \textsubscript{1}}\hspace{0.5cm}[\kern2pt{\textcolor{darkblue}{\textbf{\ipa{dze˩˥}}}}\kern2pt]} \hypertarget{dze\string_Ba1}{}
\markboth{\textcolor{darkblue}{\textbf{\ipa{dze˩\textsubscript{a}}}} \textsubscript{1}}{}
\textcolor{teal}{\mytextsc{verbe}} \hspace{4pt} Ton~: L\textsubscript{a}.
\textcolor{PineGreen}{\selectlanguage{french}Voler (dans les airs).} \zh{飞。}  ¶ \textcolor{darkblue}{\textbf{\ipa{le˧-dze˩-hɯ˩-ze˩}}} \textcolor{PineGreen}{\selectlanguage{french}(L'oiseau) est parti à tire-d'aile.} \zh{(鸟)飞走了。}  
 ¶ \textcolor{darkblue}{\textbf{\ipa{mv̩˧ʁo˧ dze˧˥}}} \textcolor{PineGreen}{\selectlanguage{french}voler dans le ciel} \zh{在天空中飞}  

\lhead{\firstmark}
\rhead{\botmark}

\subsection{\hspace{-0.5cm} {\Large \textcolor{darkblue}{\textbf{\ipa{dze˩\textsubscript{a}}}} \textsubscript{2}}\hspace{0.5cm}[\kern2pt{\textcolor{darkblue}{\textbf{\ipa{dze˩˥}}}}\kern2pt]} \hypertarget{dze\string_Ba2}{}
\markboth{\textcolor{darkblue}{\textbf{\ipa{dze˩\textsubscript{a}}}} \textsubscript{2}}{}
\textcolor{teal}{\mytextsc{verbe}} \hspace{4pt} Ton~: L\textsubscript{a}.
\textcolor{PineGreen}{\selectlanguage{french}Couper (avec un couteau).} \zh{切(用刀)。}  ¶ \textcolor{darkblue}{\textbf{\ipa{le˧-dze˩}}} \textcolor{PineGreen}{\selectlanguage{french}\mytextsc{accomp}} \zh{\mytextsc{accomp}}  
 ¶ \textcolor{darkblue}{\textbf{\ipa{dze˧\textasciitilde{}dze˥}}} \textcolor{PineGreen}{\selectlanguage{french}\mytextsc{red}} \zh{\mytextsc{red}}  
 ¶ \textcolor{darkblue}{\textbf{\ipa{le˧-dze˧\textasciitilde{}dze˥}}} \textcolor{PineGreen}{\selectlanguage{french}\mytextsc{accomp} \string_ \mytextsc{red}} \zh{\mytextsc{accomp} \string_ \mytextsc{red}}  
 ¶ \textcolor{darkblue}{\textbf{\ipa{v̩˩tsʰɤ˧ dze˧\textasciitilde{}dze˥}}} \textcolor{PineGreen}{\selectlanguage{french}découper des légumes} \zh{切菜}  
 ¶ \textcolor{darkblue}{\textbf{\ipa{nv̩˩dʑɯ˥ dze˩\textasciitilde{}dze˩}}} \textcolor{PineGreen}{\selectlanguage{french}découper du tofu} \zh{切豆腐}  

\lhead{\firstmark}
\rhead{\botmark}

\subsection{\hspace{-0.5cm} {\Large \textcolor{darkblue}{\textbf{\ipa{dze˩dʑɯ˧˥}}}}\hspace{0.5cm}[\kern2pt{\textcolor{darkblue}{\textbf{\ipa{dze˧dʑɯ˩}}}}\kern2pt]} \hypertarget{dze\string_Bdz£M\string_M\string_T1}{}
\markboth{\textcolor{darkblue}{\textbf{\ipa{dze˩dʑɯ˧˥}}}}{}
\textcolor{teal}{\mytextsc{adjectif}} \hspace{4pt} Ton~: LM+MH\#.
\textcolor{PineGreen}{\selectlanguage{french}Orgueilleux, arrogant.} \zh{骄傲,自以为好。}  ¶ \textcolor{darkblue}{\textbf{\ipa{ʈʂʰɯ˧ | hĩ˧-bi˥ | mɤ˧-li˧! | dze˩dʑɯ˧˥ | ʐwæ˧˥!}}} \textcolor{PineGreen}{\selectlanguage{french}il méprise les autres! il est très orgueilleux!} \zh{他看不起别人!他很骄傲!}  

\lhead{\firstmark}
\rhead{\botmark}

\subsection{\hspace{-0.5cm} {\Large \textcolor{darkblue}{\textbf{\ipa{dze˩mi˧}}}}\hspace{0.5cm}[\kern2pt{\textcolor{darkblue}{\textbf{\ipa{dze˧mi˧}}}}\kern2pt]} \hypertarget{dze\string_Bmi\string_M1}{}
\markboth{\textcolor{darkblue}{\textbf{\ipa{dze˩mi˧}}}}{}
\textcolor{teal}{\mytextsc{nom}} \hspace{4pt} Ton~: LM.
\textcolor{PineGreen}{\selectlanguage{french}Abeille.} \zh{蜜蜂。}  \zh{量词}: \textcolor{darkblue}{\textbf{\ipa{mi˩}}}  \mytextsc{clf}~: \textcolor{darkblue}{\textbf{\ipa{mi˩}}} 
\lhead{\firstmark}
\rhead{\botmark}

\subsection{\hspace{-0.5cm} {\Large \textcolor{darkblue}{\textbf{\ipa{dze˩mi˧-bæ˩bæ˩}}}}\hspace{0.5cm}[\kern2pt{\textcolor{darkblue}{\textbf{\ipa{xxxx non-correspondance entre le nombre de morphèmes et le nombre de tons de morphèmes}}}}\kern2pt]} \hypertarget{dze\string_Bmi\string_M-b\{\string_Bb\{\string_B1}{}
\markboth{\textcolor{darkblue}{\textbf{\ipa{dze˩mi˧-bæ˩bæ˩}}}}{}
\textcolor{teal}{\mytextsc{nom}} \hspace{4pt} Ton~: L\#.
\textcolor{PineGreen}{\selectlanguage{french}\textit{Artemisia suboligata}; littéralement “la fleur aux abeilles”.} \zh{茶绒蒿。}  \zh{量词}: \textcolor{darkblue}{\textbf{\ipa{bæ˩}}}  \mytextsc{clf}~: \textcolor{darkblue}{\textbf{\ipa{bæ˩}}} 
\lhead{\firstmark}
\rhead{\botmark}

\subsection{\hspace{-0.5cm} {\Large \textcolor{darkblue}{\textbf{\ipa{dze˩mi˧-dze\#˥}}}}\hspace{0.5cm}[\kern2pt{\textcolor{darkblue}{\textbf{\ipa{xxxx non-correspondance entre le nombre de morphèmes et le nombre de tons de morphèmes}}}}\kern2pt]} \hypertarget{dze\string_Bmi\string_M-dze\#\string_T1}{}
\markboth{\textcolor{darkblue}{\textbf{\ipa{dze˩mi˧-dze\#˥}}}}{}
\textcolor{teal}{\mytextsc{nom}} \hspace{4pt} Ton~: LM+\#H.
\textit{De:} \textbf{dze˩mi˧ et dze˥} \textcolor{PineGreen}{\selectlanguage{french}Miel.} \zh{蜂蜜。}  ¶ \textcolor{darkblue}{\textbf{\ipa{dze˩mi˧dze˧ dzɯ˧}}} \textcolor{PineGreen}{\selectlanguage{french}manger du miel} \zh{吃蜂蜜}  
 \zh{量词}: \textcolor{darkblue}{\textbf{\ipa{kʰwɤ˥}}}  \mytextsc{clf}~: \textcolor{darkblue}{\textbf{\ipa{kʰwɤ˥}}} 
\lhead{\firstmark}
\rhead{\botmark}

\subsection{\hspace{-0.5cm} {\Large \textcolor{darkblue}{\textbf{\ipa{dze˩mi˧-kʰv̩˩}}}}\hspace{0.5cm}[\kern2pt{\textcolor{darkblue}{\textbf{\ipa{xxxx non-correspondance entre le nombre de morphèmes et le nombre de tons de morphèmes}}}}\kern2pt]} \hypertarget{dze\string_Bmi\string_M-k\string_hv\string_=\string_B1}{}
\markboth{\textcolor{darkblue}{\textbf{\ipa{dze˩mi˧-kʰv̩˩}}}}{}
\textcolor{teal}{\mytextsc{nom}} \hspace{4pt} Ton~: LM-L.
\textcolor{PineGreen}{\selectlanguage{french}Ruche.} \zh{蜂窝。}  \zh{量词}: \textcolor{darkblue}{\textbf{\ipa{ɭɯ˧}}}  \mytextsc{clf}~: \textcolor{darkblue}{\textbf{\ipa{ɭɯ˧}}} 
\lhead{\firstmark}
\rhead{\botmark}

\subsection{\hspace{-0.5cm} {\Large \textcolor{darkblue}{\textbf{\ipa{dze˩mi˧-pv̩˥ɻ̍˩}}}}\hspace{0.5cm}[\kern2pt{\textcolor{darkblue}{\textbf{\ipa{dze˩mi˧pv̩˩ɻ̍˩}}}}\kern2pt]} \hypertarget{dze\string_Bmi\string_M-pv\string_=\string_Tr£`̍\string_B1}{}
\markboth{\textcolor{darkblue}{\textbf{\ipa{dze˩mi˧-pv̩˥ɻ̍˩}}}}{}
\textcolor{teal}{\mytextsc{nom}} \hspace{4pt} Ton~: LM+\#H-.
\textcolor{PineGreen}{\selectlanguage{french}\textit{Adenophora sp.}.} \zh{沙参。} Dialecte chinois local~:\zh{yyyy fusionner les 2 entrées; est un terme générique ; tʰi˧-pv˥ɻ˩ = tʰi˧-hɑ̃˧˥ se reposer qq part, se poser quelque part。} ¶ \textcolor{darkblue}{\textbf{\ipa{dze˩mi˧-pv̩˥ɻ̍˩-kʰɯ˩ʈɯ˩}}} \textcolor{PineGreen}{\selectlanguage{french}racine d'\textit{Adenophora sp.}} \zh{沙参根}  

\lhead{\firstmark}
\rhead{\botmark}

\subsection{\hspace{-0.5cm} {\Large \textcolor{darkblue}{\textbf{\ipa{dze˩mi˧-pv̩˥ɻ̍˩}}}}\hspace{0.5cm}[\kern2pt{\textcolor{darkblue}{\textbf{\ipa{dze˩mi˧pv̩˥ɻ̍˩}}}}\kern2pt]} \hypertarget{dze\string_Bmi\string_M-pv\string_=\string_Tr£`̍\string_B1}{}
\markboth{\textcolor{darkblue}{\textbf{\ipa{dze˩mi˧-pv̩˥ɻ̍˩}}}}{}
\textcolor{teal}{\mytextsc{nom}} \hspace{4pt} Ton~: LM+\#H-.
\textcolor{PineGreen}{\selectlanguage{french}Gentiane.} \zh{秦艽。}  ¶ \textcolor{darkblue}{\textbf{\ipa{dʑɯ˧qʰɑ˧-bæ˩bæ˩}}} \textcolor{PineGreen}{\selectlanguage{french}fleurs de gentiane} \zh{秦艽花}  
 \zh{量词}: \textcolor{darkblue}{\textbf{\ipa{qɑ˩}}}  \mytextsc{clf}~: \textcolor{darkblue}{\textbf{\ipa{qɑ˩}}} 
\lhead{\firstmark}
\rhead{\botmark}

\subsection{\hspace{-0.5cm} {\Large \textcolor{darkblue}{\textbf{\ipa{*dze˩˧}}}}\hspace{0.5cm}[\kern2pt{\textcolor{darkblue}{\textbf{\ipa{dze˩˥}}}}\kern2pt]} \hypertarget{*dze\string_B\string_M1}{}
\markboth{\textcolor{darkblue}{\textbf{\ipa{*dze˩˧}}}}{}
\textcolor{teal}{\mytextsc{nom}} \hspace{4pt} Ton~: LM.
\textcolor{PineGreen}{\selectlanguage{french}Abeille (racine déduite du disyllabe).} \zh{蜜蜂。} 
\lhead{\firstmark}
\rhead{\botmark}

\subsection{\hspace{-0.5cm} {\Large \textcolor{darkblue}{\textbf{\ipa{dze˩˧}}}}\hspace{0.5cm}[\kern2pt{\textcolor{darkblue}{\textbf{\ipa{dze˥}}}}\kern2pt]} \hypertarget{dze\string_B\string_M1}{}
\markboth{\textcolor{darkblue}{\textbf{\ipa{dze˩˧}}}}{}
\textcolor{teal}{\mytextsc{nom}} \hspace{4pt} Ton~: LM.
\textcolor{PineGreen}{\selectlanguage{french}Xanthoxyle, poivre de Chine, poivre du Sichuan.} \zh{花椒。}  \zh{量词}: \textcolor{darkblue}{\textbf{\ipa{mɤ˩}}}  \mytextsc{clf}~: \textcolor{darkblue}{\textbf{\ipa{mɤ˩}}} \textcolor{PineGreen}{\selectlanguage{french}un peu}
\lhead{\firstmark}
\rhead{\botmark}

\subsection{\hspace{-0.5cm} {\Large \textcolor{darkblue}{\textbf{\ipa{dzɤ˥\textsubscript{b}}}}}\hspace{0.5cm}[\kern2pt{\textcolor{darkblue}{\textbf{\ipa{dzɤ˥}}}}\kern2pt]} \hypertarget{dz7\string_Tb1}{}
\markboth{\textcolor{darkblue}{\textbf{\ipa{dzɤ˥\textsubscript{b}}}}}{}
\textcolor{teal}{\mytextsc{classificateur}} \hspace{4pt} Ton~: H\textsubscript{b}.
\textcolor{PineGreen}{\selectlanguage{french}Côté.} \zh{量词:面。}  ¶ \textcolor{darkblue}{\textbf{\ipa{ʈʂʰɯ˧-dzɤ˧}}} \textcolor{PineGreen}{\selectlanguage{french}ce côté-ci} \zh{这面}  
 ¶ \textcolor{darkblue}{\textbf{\ipa{ɖɯ˧-dzɤ˥}}} \textcolor{PineGreen}{\selectlanguage{french}un côté} \zh{一面}  

\lhead{\firstmark}
\rhead{\botmark}

\subsection{\hspace{-0.5cm} {\Large \textcolor{darkblue}{\textbf{\ipa{dzɤ˩\textsubscript{a}}}}}\hspace{0.5cm}[\kern2pt{\textcolor{darkblue}{\textbf{\ipa{dzɤ˧˥}}}}\kern2pt]} \hypertarget{dz7\string_Ba1}{}
\markboth{\textcolor{darkblue}{\textbf{\ipa{dzɤ˩\textsubscript{a}}}}}{}
\textcolor{teal}{\mytextsc{verbe}} \hspace{4pt} Ton~: L\textsubscript{a}.
\textcolor{PineGreen}{\selectlanguage{french}S’écrouler, s'effondrer (mur); tomber (arbre), se renverser, s'abattre.} \zh{塌毁,倒塌 ,倒。}  ¶ \textcolor{darkblue}{\textbf{\ipa{mv̩˩tɕo˧ dzɤ˩}}} \textcolor{PineGreen}{\selectlanguage{french}même sens: s'écrouler} \zh{同上:塌毁}  
 ¶ \textcolor{darkblue}{\textbf{\ipa{le˧-dzɤ˩-ze˩}}} \textcolor{PineGreen}{\selectlanguage{french}\mytextsc{accomp} \string_ \mytextsc{pfv}} \zh{塌毁了}  

\lhead{\firstmark}
\rhead{\botmark}

\subsection{\hspace{-0.5cm} {\Large \textcolor{darkblue}{\textbf{\ipa{dzi˥}}}}\hspace{0.5cm}[\kern2pt{\textcolor{darkblue}{\textbf{\ipa{dzi˧˥}}}}\kern2pt]} \hypertarget{dzi\string_T1}{}
\markboth{\textcolor{darkblue}{\textbf{\ipa{dzi˥}}}}{}
\textcolor{teal}{\mytextsc{nom}} \hspace{4pt} Ton~: \#H.
\textcolor{PineGreen}{\selectlanguage{french}Burin, ciseau.} \zh{凿子。}  \zh{量词}: \textcolor{darkblue}{\textbf{\ipa{ɭɯ˧ (*nɑ˧)}}}  \mytextsc{clf}~: \textcolor{darkblue}{\textbf{\ipa{ɭɯ˧ (*nɑ˧)}}} 
\lhead{\firstmark}
\rhead{\botmark}

\subsection{\hspace{-0.5cm} {\Large \textcolor{darkblue}{\textbf{\ipa{dzi˧\textsubscript{b}}}}}\hspace{0.5cm}[\kern2pt{\textcolor{darkblue}{\textbf{\ipa{dzi˩˥}}}}\kern2pt]} \hypertarget{dzi\string_Mb1}{}
\markboth{\textcolor{darkblue}{\textbf{\ipa{dzi˧\textsubscript{b}}}}}{}
\textcolor{teal}{\mytextsc{classificateur}} \hspace{4pt} Ton~: M\textsubscript{b}.
\textcolor{PineGreen}{\selectlanguage{french}Paire d'objets qui constitue une unité: par exemple une paire de chaussures.} \zh{量词:鞋(一双)。}  ¶ \textcolor{darkblue}{\textbf{\ipa{ɣɯ˩-dzɑ˩qʰwɤ˥ | ɖɯ˧-dzi˧}}} \textcolor{PineGreen}{\selectlanguage{french}une paire de chaussures en cuir} \zh{一双皮鞋}  

\lhead{\firstmark}
\rhead{\botmark}

\subsection{\hspace{-0.5cm} {\Large \textcolor{darkblue}{\textbf{\ipa{dzi˧dzi˧}}}}\hspace{0.5cm}[\kern2pt{\textcolor{darkblue}{\textbf{\ipa{dzi˧dzi˧}}}}\kern2pt]} \hypertarget{dzi\string_Mdzi\string_M1}{}
\markboth{\textcolor{darkblue}{\textbf{\ipa{dzi˧dzi˧}}}}{}
\textcolor{teal}{\mytextsc{nom}} \hspace{4pt} Ton~: M.
\textcolor{PineGreen}{\selectlanguage{french}Chêne blanc oriental.} \zh{青冈树、槲栎。}  ¶ \textcolor{darkblue}{\textbf{\ipa{dzi˧dzi˧, | si˧dzi˩-mv̩˩!}}} \textcolor{PineGreen}{\selectlanguage{french}\textcolor{darkblue}{\textbf{\ipa{/dzi˧dzi˧/}}}, c'est un nom d'arbre! / C'est le nom d'un arbre!} \zh{\textcolor{darkblue}{\textbf{\ipa{dzi˧dzi˧}}}是一种树的名字!}  
\textit{Syn~:} \hyperlink{}{\textcolor{darkblue}{\textbf{\ipa{dʑɯ˩si˩}}}}. 
\lhead{\firstmark}
\rhead{\botmark}

\subsection{\hspace{-0.5cm} {\Large \textcolor{darkblue}{\textbf{\ipa{dzi˧dzi˧-mo˧˥}}}}\hspace{0.5cm}[\kern2pt{\textcolor{darkblue}{\textbf{\ipa{xxxx non-correspondance entre le nombre de morphèmes et le nombre de tons de morphèmes}}}}\kern2pt]} \hypertarget{dzi\string_Mdzi\string_M-mo\string_M\string_T1}{}
\markboth{\textcolor{darkblue}{\textbf{\ipa{dzi˧dzi˧-mo˧˥}}}}{}
\textcolor{teal}{\mytextsc{nom}} \hspace{4pt} Ton~: MH\#.
\textcolor{PineGreen}{\selectlanguage{french}Champignon comestible, qui ne pousse pas sur la terre, seulement sur les arbres tombés; est utilisé comme médicament pour les maux d'estomac.} \zh{一种可以吃的菌子,长在枯木上。} 
\lhead{\firstmark}
\rhead{\botmark}

\subsection{\hspace{-0.5cm} {\Large \textcolor{darkblue}{\textbf{\ipa{dzi˧ɖæ˧}}}}\hspace{0.5cm}[\kern2pt{\textcolor{darkblue}{\textbf{\ipa{dzi˧ɖæ˧}}}}\kern2pt]} \hypertarget{dzi\string_Md`\{\string_M1}{}
\markboth{\textcolor{darkblue}{\textbf{\ipa{dzi˧ɖæ˧}}}}{}
\textcolor{teal}{\mytextsc{nom}} \hspace{4pt} Ton~: M.
\textcolor{PineGreen}{\selectlanguage{french}Emplacement, localisation (emploi typique: emplacement d'une maison).} \zh{位置、所在地。}  \zh{量词}: \textcolor{darkblue}{\textbf{\ipa{kʰwɤ˥}}}  \mytextsc{clf}~: \textcolor{darkblue}{\textbf{\ipa{kʰwɤ˥}}} 
\lhead{\firstmark}
\rhead{\botmark}

\subsection{\hspace{-0.5cm} {\Large \textcolor{darkblue}{\textbf{\ipa{dzi˩}}} \textsubscript{1}}\hspace{0.5cm}[\kern2pt{\textcolor{darkblue}{\textbf{\ipa{dzi˥}}}}\kern2pt]} \hypertarget{dzi\string_B1}{}
\markboth{\textcolor{darkblue}{\textbf{\ipa{dzi˩}}} \textsubscript{1}}{}
\textcolor{teal}{\mytextsc{verbe}} \hspace{4pt} Ton~: L.
\textcolor{PineGreen}{\selectlanguage{french}Tomber, venir (la nuit tombe, la nuit vient).} \zh{来(晚上来了)。}  ¶ \textcolor{darkblue}{\textbf{\ipa{nɑ˩˥ | le˧-dzi˩-ze˩!}}} \textcolor{PineGreen}{\selectlanguage{french}la nuit est tombée! Il fait noir!} \zh{天黑了!}  
 ¶ \textcolor{darkblue}{\textbf{\ipa{nɑ˩˥ | le˧-dzi˩ | le˧-se˩-ze˩!}}} \textcolor{PineGreen}{\selectlanguage{french}il fait tout à fait nuit!} \zh{天完全黑了!}  

\lhead{\firstmark}
\rhead{\botmark}

\subsection{\hspace{-0.5cm} {\Large \textcolor{darkblue}{\textbf{\ipa{dzi˩}}} \textsubscript{2}}\hspace{0.5cm}[\kern2pt{\textcolor{darkblue}{\textbf{\ipa{dzi˩˥}}}}\kern2pt]} \hypertarget{dzi\string_B2}{}
\markboth{\textcolor{darkblue}{\textbf{\ipa{dzi˩}}} \textsubscript{2}}{}
\textcolor{teal}{\mytextsc{classificateur}} \hspace{4pt} Ton~: L\textsubscript{c}.
\textcolor{PineGreen}{\selectlanguage{french}Un costume entier: tous les vêtements qu'on porte.} \zh{量词:衣服(一套)。}  ¶ \textcolor{darkblue}{\textbf{\ipa{dʑi˧hṽ˥ | ɖɯ˧-dzi˩}}} \textcolor{PineGreen}{\selectlanguage{french}un costume entier, un vêtement} \zh{一套衣服}  
 ¶ \textcolor{darkblue}{\textbf{\ipa{dʑi˧hṽ˧ ɖɯ˧-dzi˩}}} \textcolor{PineGreen}{\selectlanguage{french}un costume entier, un vêtement (même sens que l'exemple précédent; intégration en un seul groupe tonal)} \zh{一套衣服(同上,但将短语合在一起,构成一个单一的声调短语)}  

\lhead{\firstmark}
\rhead{\botmark}

\subsection{\hspace{-0.5cm} {\Large \textcolor{darkblue}{\textbf{\ipa{dzi˩\textsubscript{a}}}} \textsubscript{1}}\hspace{0.5cm}[\kern2pt{\textcolor{darkblue}{\textbf{\ipa{dzi˥}}}}\kern2pt]} \hypertarget{dzi\string_Ba1}{}
\markboth{\textcolor{darkblue}{\textbf{\ipa{dzi˩\textsubscript{a}}}} \textsubscript{1}}{}
\textcolor{teal}{\mytextsc{verbe}} \hspace{4pt} Ton~: L\textsubscript{a}.
\ding{202} \textcolor{PineGreen}{\selectlanguage{french}S'asseoir, être assis.} \zh{坐。}  ¶ \textcolor{darkblue}{\textbf{\ipa{tʰi˧-dzi˩!}}} \textcolor{PineGreen}{\selectlanguage{french}Asseyez-vous!} \zh{坐下!}  
 ¶ \textcolor{darkblue}{\textbf{\ipa{hĩ˧bæ˧ ʈʂʰɯ˧-qo˧ dzi˩.}}} \textcolor{PineGreen}{\selectlanguage{french}L'invité s'asseoit ici.} \zh{客人是坐在这边的。}  
 ¶ \textcolor{darkblue}{\textbf{\ipa{(ʈʂʰɯ˧ | ) tʰi˧-dzi˩-kʰɯ˩-se˩.}}} \textcolor{PineGreen}{\selectlanguage{french}Il est assis/installé, il a pris sa place.} \zh{他坐下了。}  
 ¶ \textcolor{darkblue}{\textbf{\ipa{le˧-dzi˧\textasciitilde{}dzi˥}}} \textcolor{PineGreen}{\selectlanguage{french}se tenir assis; s'emploie, par euphémisme, pour désigner la participation à une veillée funèbre} \zh{坐一坐。来指:居丧、守灵(委婉语)}  
\ding{203} \textcolor{PineGreen}{\selectlanguage{french}Demeurer, habiter.} \zh{住。}  ¶ \textcolor{darkblue}{\textbf{\ipa{dzi˩-bi˩-ni˩gv̩˩}}} \textcolor{PineGreen}{\selectlanguage{french}s'habituer (à un environnement; à des habitudes de vie: nourriture, boisson...; à quelque chose; à quelqu'un)} \zh{习惯(一个新的环境、一个地方的饮食……)}  
 ¶ \textcolor{darkblue}{\textbf{\ipa{dzi˩-bi˩-ni˩-mɤ˩-gv̩˩˥}}} \textcolor{PineGreen}{\selectlanguage{french}je n'aime pas/je ne m'y fais pas/ça ne me plaît pas (ex.: quelqu'un ne veut pas rester quelque part, il s'y trouve mal et a la nostalgie d'ailleurs)} \zh{不习惯}  
 ¶ \textcolor{darkblue}{\textbf{\ipa{njɤ˧ | ʈʂʰɯ˧-qo˧ | dzi˩-bi˩-ni˩-mɤ˩-gv̩˩˥}}} \textcolor{PineGreen}{\selectlanguage{french}Je ne me fais pas à ici! / Je ne suis pas bien ici! (Paraphrase proposée par M23: “Je ne veux pas rester ici!”)} \zh{我不习惯这里! / 我不喜欢这里! / 我不想待了!}  

\lhead{\firstmark}
\rhead{\botmark}

\subsection{\hspace{-0.5cm} {\Large \textcolor{darkblue}{\textbf{\ipa{dzi˩\textsubscript{a}}}} \textsubscript{2}}\hspace{0.5cm}[\kern2pt{\textcolor{darkblue}{\textbf{\ipa{dzi˩˥}}}}\kern2pt]} \hypertarget{dzi\string_Ba2}{}
\markboth{\textcolor{darkblue}{\textbf{\ipa{dzi˩\textsubscript{a}}}} \textsubscript{2}}{}
\textcolor{teal}{\mytextsc{verbe}} \hspace{4pt} Ton~: L\textsubscript{a}.
\textcolor{PineGreen}{\selectlanguage{french}Se rassembler (groupe de personnes).} \zh{聚集。}  ¶ \textcolor{darkblue}{\textbf{\ipa{ɖɯ˧-ʁwɤ˧ | le˧-dzi˧\textasciitilde{}dzi˥}}} \textcolor{PineGreen}{\selectlanguage{french}tout le village se rassemble} \zh{全村都聚集在一起}  
 ¶ \textcolor{darkblue}{\textbf{\ipa{hĩ˧ ɖɯ˧-v̩˧ | le˧-ʂɯ˧-ze˧! le˧-dzi˧\textasciitilde{}dzi˥ jo˩!}}} \textcolor{PineGreen}{\selectlanguage{french}Quelqu'un est mort! Venez participer à la réunion/au rassemblement!} \zh{一个人去世了!来参加丧礼吧!}  

\lhead{\firstmark}
\rhead{\botmark}

\subsection{\hspace{-0.5cm} {\Large \textcolor{darkblue}{\textbf{\ipa{dzi˩\textsubscript{a}}}} \textsubscript{3}}\hspace{0.5cm}[\kern2pt{\textcolor{darkblue}{\textbf{\ipa{dzi˩˥}}}}\kern2pt]} \hypertarget{dzi\string_Ba3}{}
\markboth{\textcolor{darkblue}{\textbf{\ipa{dzi˩\textsubscript{a}}}} \textsubscript{3}}{}
\textcolor{teal}{\mytextsc{verbe}} \hspace{4pt} Ton~: L\textsubscript{a}.
\textcolor{PineGreen}{\selectlanguage{french}Tomber, sombrer (ex.: un bateau qui sombre peu à peu dans le lac).} \zh{掉入、沉下去。}  ¶ \textcolor{darkblue}{\textbf{\ipa{mv̩˩tɕo˧ dzi˩}}} \textcolor{PineGreen}{\selectlanguage{french}\mytextsc{directionnel} \string_ ; même sens: sombrer, couler} \zh{往下掉入、沉下去}  

\lhead{\firstmark}
\rhead{\botmark}

\subsection{\hspace{-0.5cm} {\Large \textcolor{darkblue}{\textbf{\ipa{dzi˩\textsubscript{b}}}}}\hspace{0.5cm}[\kern2pt{\textcolor{darkblue}{\textbf{\ipa{dzi˩˥}}}}\kern2pt]} \hypertarget{dzi\string_Bb1}{}
\markboth{\textcolor{darkblue}{\textbf{\ipa{dzi˩\textsubscript{b}}}}}{}
\textcolor{teal}{\mytextsc{classificateur}} \hspace{4pt} Ton~: L\textsubscript{b}.
\textcolor{PineGreen}{\selectlanguage{french}Classificateur des arbres.} \zh{量词:树(一棵),竹子(一根)。}  ¶ \textcolor{darkblue}{\textbf{\ipa{si˧dzi˩ | ɖɯ˧-dzi˩}}} \textcolor{PineGreen}{\selectlanguage{french}un arbre} \zh{一棵树}  
 ¶ \textcolor{darkblue}{\textbf{\ipa{tʰv̩˧-dzi˧˥}}} \textcolor{PineGreen}{\selectlanguage{french}cet arbre} \zh{那棵树}  

\lhead{\firstmark}
\rhead{\botmark}

\subsection{\hspace{-0.5cm} {\Large \textcolor{darkblue}{\textbf{\ipa{dzi˩ʁo˩}}}}\hspace{0.5cm}[\kern2pt{\textcolor{darkblue}{\textbf{\ipa{dzi˧ʁo˧}}}}\kern2pt]} \hypertarget{dzi\string_BRo\string_B1}{}
\markboth{\textcolor{darkblue}{\textbf{\ipa{dzi˩ʁo˩}}}}{}
\textcolor{teal}{\mytextsc{nom}} \hspace{4pt} Ton~: L.
\textcolor{PineGreen}{\selectlanguage{french}Place, place assise.} \zh{座位。}  \zh{量词}: \textcolor{darkblue}{\textbf{\ipa{kʰwɤ˥}}}  \mytextsc{clf}~: \textcolor{darkblue}{\textbf{\ipa{kʰwɤ˥}}} 
\lhead{\firstmark}
\rhead{\botmark}

\subsection{\hspace{-0.5cm} {\Large \textcolor{darkblue}{\textbf{\ipa{dzi˩tsʰɤ˩}}}}\hspace{0.5cm}[\kern2pt{\textcolor{darkblue}{\textbf{\ipa{dzi˩tsʰɤ˩˥}}}}\kern2pt]} \hypertarget{dzi\string_Bts\string_h7\string_B1}{}
\markboth{\textcolor{darkblue}{\textbf{\ipa{dzi˩tsʰɤ˩}}}}{}
\textcolor{teal}{\mytextsc{nom}} \hspace{4pt} Ton~: L.
\textcolor{PineGreen}{\selectlanguage{french}Arbuste: sorte de houx, de grande taille.} \zh{永宁的一种灌木。}  \zh{量词}: \textcolor{darkblue}{\textbf{\ipa{dzi˩}}}  \mytextsc{clf}~: \textcolor{darkblue}{\textbf{\ipa{dzi˩}}} 
\lhead{\firstmark}
\rhead{\botmark}

\subsection{\hspace{-0.5cm} {\Large \textcolor{darkblue}{\textbf{\ipa{dzi˧˥}}}}\hspace{0.5cm}[\kern2pt{\textcolor{darkblue}{\textbf{\ipa{dzi˩˥}}}}\kern2pt]} \hypertarget{dzi\string_M\string_T1}{}
\markboth{\textcolor{darkblue}{\textbf{\ipa{dzi˧˥}}}}{}
\textcolor{teal}{\mytextsc{verbe}} \hspace{4pt} Ton~: MH.
\textcolor{PineGreen}{\selectlanguage{french}Trembler.} \zh{颤抖、抖动。}  ¶ \textcolor{darkblue}{\textbf{\ipa{njɤ˩ dzi˧˥}}} \textcolor{PineGreen}{\selectlanguage{french}la paupière tremble} \zh{眼皮跳}  
 ¶ \textcolor{darkblue}{\textbf{\ipa{njɤ˩ dzi˧-ze˥}}} \textcolor{PineGreen}{\selectlanguage{french}la paupière tremble} \zh{眼皮跳}  
 ¶ \textcolor{darkblue}{\textbf{\ipa{njɤ˩ dzi˧˥ | ʐwæ˩˥}}} \textcolor{PineGreen}{\selectlanguage{french}la paupière tremble très fort} \zh{眼皮跳得很厉害}  
 ¶ \textcolor{darkblue}{\textbf{\ipa{njɤ˩˥ | le˧-dzi˧-ze˥}}} \textcolor{PineGreen}{\selectlanguage{french}la paupière tremble} \zh{眼皮跳}  
 ¶ \textcolor{darkblue}{\textbf{\ipa{njɤ˩˥ | mɤ˧-dzi˧˥}}} \textcolor{PineGreen}{\selectlanguage{french}la paupière ne tremble pas} \zh{眼皮不跳}  

\lhead{\firstmark}
\rhead{\botmark}

\subsection{\hspace{-0.5cm} {\Large \textcolor{darkblue}{\textbf{\ipa{dzo˥}}}}\hspace{0.5cm}[\kern2pt{\textcolor{darkblue}{\textbf{\ipa{dzo˩˥}}}}\kern2pt]} \hypertarget{dzo\string_T1}{}
\markboth{\textcolor{darkblue}{\textbf{\ipa{dzo˥}}}}{}
\textcolor{teal}{\mytextsc{nom}} \hspace{4pt} Ton~: \#H.
\textcolor{PineGreen}{\selectlanguage{french}Grêle.} \zh{冰雹。}  ¶ \textcolor{darkblue}{\textbf{\ipa{dzo˧ lɑ˩}}} \textcolor{PineGreen}{\selectlanguage{french}il grêle} \zh{下冰雹}  
 ¶ \textcolor{darkblue}{\textbf{\ipa{dzo˧ gi˧-ze˩}}} \textcolor{PineGreen}{\selectlanguage{french}il tombe de la grêle} \zh{下冰雹了}  

\lhead{\firstmark}
\rhead{\botmark}

\subsection{\hspace{-0.5cm} {\Large \textcolor{darkblue}{\textbf{\ipa{dzo˧-lv̩˧\textasciitilde{}lv̩˥}}}}\hspace{0.5cm}[\kern2pt{\textcolor{darkblue}{\textbf{\ipa{xxxx non-correspondance entre le nombre de morphèmes et le nombre de tons de morphèmes}}}}\kern2pt]} \hypertarget{dzo\string_M-lv\string_=\string_M~lv\string_=\string_T1}{}
\markboth{\textcolor{darkblue}{\textbf{\ipa{dzo˧-lv̩˧\textasciitilde{}lv̩˥}}}}{}
\textcolor{teal}{\mytextsc{nom}} \hspace{4pt} Ton~: H\#.
\textcolor{PineGreen}{\selectlanguage{french}Grêlon.} \zh{冰雹。}  \zh{量词}: \textcolor{darkblue}{\textbf{\ipa{ɭɯ˧}}}  \mytextsc{clf}~: \textcolor{darkblue}{\textbf{\ipa{ɭɯ˧}}} 
\lhead{\firstmark}
\rhead{\botmark}

\subsection{\hspace{-0.5cm} {\Large \textcolor{darkblue}{\textbf{\ipa{dzo˧mi˧}}}}\hspace{0.5cm}[\kern2pt{\textcolor{darkblue}{\textbf{\ipa{dzo˧mi˥}}}}\kern2pt]} \hypertarget{dzo\string_Mmi\string_M1}{}
\markboth{\textcolor{darkblue}{\textbf{\ipa{dzo˧mi˧}}}}{}
\textcolor{teal}{\mytextsc{nom}} \hspace{4pt} Ton~: M.
\textcolor{PineGreen}{\selectlanguage{french}Grande cuve, grand tonneau (sens très proche du précédent, peuvent s'employer de façon interchangeable pour certains des tonneaux).} \zh{大桶。}  \zh{量词}: \textcolor{darkblue}{\textbf{\ipa{ɭɯ˧}}}  \mytextsc{clf}~: \textcolor{darkblue}{\textbf{\ipa{ɭɯ˧}}} 
\lhead{\firstmark}
\rhead{\botmark}

\subsection{\hspace{-0.5cm} {\Large \textcolor{darkblue}{\textbf{\ipa{dzo˧zo\#˥}}}}\hspace{0.5cm}[\kern2pt{\textcolor{darkblue}{\textbf{\ipa{dzo˩zo˥}}}}\kern2pt]} \hypertarget{dzo\string_Mzo\#\string_T1}{}
\markboth{\textcolor{darkblue}{\textbf{\ipa{dzo˧zo\#˥}}}}{}
\textcolor{teal}{\mytextsc{nom}} \hspace{4pt} Ton~: \#H.
\textcolor{PineGreen}{\selectlanguage{french}Petite cuve.} \zh{小桶。}  \zh{量词}: \textcolor{darkblue}{\textbf{\ipa{ɭɯ˧}}}  \mytextsc{clf}~: \textcolor{darkblue}{\textbf{\ipa{ɭɯ˧}}} 
\lhead{\firstmark}
\rhead{\botmark}

\subsection{\hspace{-0.5cm} {\Large \textcolor{darkblue}{\textbf{\ipa{dzo˩}}}}\hspace{0.5cm}[\kern2pt{\textcolor{darkblue}{\textbf{\ipa{dzo˥}}}}\kern2pt]} \hypertarget{dzo\string_B1}{}
\markboth{\textcolor{darkblue}{\textbf{\ipa{dzo˩}}}}{}
\textcolor{teal}{\mytextsc{nom}} \hspace{4pt} Ton~: L.
\textcolor{PineGreen}{\selectlanguage{french}Pont.} \zh{桥。}  ¶ \textcolor{darkblue}{\textbf{\ipa{dzo˧ | ɖɯ˧ pɤ˩}}} \textcolor{PineGreen}{\selectlanguage{french}un pont} \zh{一辆桥}  
 ¶ \textcolor{darkblue}{\textbf{\ipa{dzo˩ bæ˩˥}}} \textcolor{PineGreen}{\selectlanguage{french}balayer le pont} \zh{扫桥}  
 ¶ \textcolor{darkblue}{\textbf{\ipa{njɤ˧ | dzo˩ bæ˩-zo˩-ho˥.}}} \textcolor{PineGreen}{\selectlanguage{french}Il va falloir que je balaie le pont.} \zh{我要扫桥了。}  
 ¶ \textcolor{darkblue}{\textbf{\ipa{dzo˩ gv̩˩˥}}} \textcolor{PineGreen}{\selectlanguage{french}construire (ou réparer) un pont} \zh{建一辆桥}  
 ¶ \textcolor{darkblue}{\textbf{\ipa{njɤ˧ | dzo˩ gv̩˩-zo˩-ho˥.}}} \textcolor{PineGreen}{\selectlanguage{french}Il va falloir que je construise (/répare) un pont.}  
 \zh{量词}: \textcolor{darkblue}{\textbf{\ipa{pɤ˩}}} \textcolor{darkblue}{\textbf{\ipa{nɑ˧}}}  \mytextsc{clf}~: \textcolor{darkblue}{\textbf{\ipa{pɤ˩}}} \textcolor{darkblue}{\textbf{\ipa{nɑ˧}}} 
\lhead{\firstmark}
\rhead{\botmark}

\subsection{\hspace{-0.5cm} {\Large \textcolor{darkblue}{\textbf{\ipa{dzo˩\textasciitilde{}dzo˧˥}}}}\hspace{0.5cm}[\kern2pt{\textcolor{darkblue}{\textbf{\ipa{dzo˧dzo˧}}}}\kern2pt]} \hypertarget{dzo\string_B~dzo\string_M\string_T1}{}
\markboth{\textcolor{darkblue}{\textbf{\ipa{dzo˩\textasciitilde{}dzo˧˥}}}}{}
\textcolor{teal}{\mytextsc{verbe}} \hspace{4pt} Ton~: MH.
\textcolor{PineGreen}{\selectlanguage{french}Se moquer de, rire de, ridiculiser.} \zh{嘲笑、取笑。}  ¶ \textcolor{darkblue}{\textbf{\ipa{hĩ˧ dzo˧-dzo˥-ho˩-ni˩zo˩!}}} \textcolor{PineGreen}{\selectlanguage{french}On dirait qu'(elle/il) va tourner quelqu'un en dérision} \zh{他好像要开始取笑人家了!}  
 ¶ \textcolor{darkblue}{\textbf{\ipa{tʰɑ˧ dzo˩\textasciitilde{}dzo˩!}}} \textcolor{PineGreen}{\selectlanguage{french}Ne vous moquez pas!/ Pas de sarcasmes!} \zh{别嘲笑(人家)!}  

\lhead{\firstmark}
\rhead{\botmark}

\subsection{\hspace{-0.5cm} {\Large \textcolor{darkblue}{\textbf{\ipa{dzo˩mi\#˥}}}}\hspace{0.5cm}[\kern2pt{\textcolor{darkblue}{\textbf{\ipa{dzo˧mi˧}}}}\kern2pt]} \hypertarget{dzo\string_Bmi\#\string_T1}{}
\markboth{\textcolor{darkblue}{\textbf{\ipa{dzo˩mi\#˥}}}}{}
\textcolor{teal}{\mytextsc{nom}} \hspace{4pt} Ton~: LM+\#H.
\textcolor{PineGreen}{\selectlanguage{french}Lézard femelle.} \zh{母壁虎。}  ¶ \textcolor{darkblue}{\textbf{\ipa{dzo˧mi˧-dzo˩pʰv̩˩}}} \textcolor{PineGreen}{\selectlanguage{french}lézard femelle et lézard mâle} \zh{母壁虎与公壁虎}  
 \zh{量词}: \textcolor{darkblue}{\textbf{\ipa{mi˩}}}  \mytextsc{clf}~: \textcolor{darkblue}{\textbf{\ipa{mi˩}}} 
\lhead{\firstmark}
\rhead{\botmark}

\subsection{\hspace{-0.5cm} {\Large \textcolor{darkblue}{\textbf{\ipa{dzo˩pʰv̩˩}}}}\hspace{0.5cm}[\kern2pt{\textcolor{darkblue}{\textbf{\ipa{dzo˩pʰv̩˥}}}}\kern2pt]} \hypertarget{dzo\string_Bp\string_hv\string_=\string_B1}{}
\markboth{\textcolor{darkblue}{\textbf{\ipa{dzo˩pʰv̩˩}}}}{}
\textcolor{teal}{\mytextsc{nom}} \hspace{4pt} Ton~: L.
\textcolor{PineGreen}{\selectlanguage{french}Lézard mâle.} \zh{公壁虎。}  ¶ \textcolor{darkblue}{\textbf{\ipa{dzo˩pʰv̩˩-dzo˩mi˥}}} \textcolor{PineGreen}{\selectlanguage{french}lézard mâle et lézard femelle} \zh{公壁虎与母壁虎}  
 \zh{量词}: \textcolor{darkblue}{\textbf{\ipa{mi˩}}} \textcolor{darkblue}{\textbf{\ipa{ɭɯ˧}}}  \mytextsc{clf}~: \textcolor{darkblue}{\textbf{\ipa{mi˩}}} \textcolor{darkblue}{\textbf{\ipa{ɭɯ˧}}} 
\lhead{\firstmark}
\rhead{\botmark}

\subsection{\hspace{-0.5cm} {\Large \textcolor{darkblue}{\textbf{\ipa{dzo˩zo\#˥}}}}\hspace{0.5cm}[\kern2pt{\textcolor{darkblue}{\textbf{\ipa{dzo˩zo˥}}}}\kern2pt]} \hypertarget{dzo\string_Bzo\#\string_T1}{}
\markboth{\textcolor{darkblue}{\textbf{\ipa{dzo˩zo\#˥}}}}{}
\textcolor{teal}{\mytextsc{nom}} \hspace{4pt} Ton~: LM+\#H.
\textcolor{PineGreen}{\selectlanguage{french}Bébé lézard.} \zh{小壁虎。}  \zh{量词}: \textcolor{darkblue}{\textbf{\ipa{mi˩}}} \textcolor{darkblue}{\textbf{\ipa{ɭɯ˧}}}  \mytextsc{clf}~: \textcolor{darkblue}{\textbf{\ipa{mi˩}}} \textcolor{darkblue}{\textbf{\ipa{ɭɯ˧}}} 
\lhead{\firstmark}
\rhead{\botmark}

\subsection{\hspace{-0.5cm} {\Large \textcolor{darkblue}{\textbf{\ipa{dzo˩˧}}}}\hspace{0.5cm}[\kern2pt{\textcolor{darkblue}{\textbf{\ipa{dzo˥}}}}\kern2pt]} \hypertarget{dzo\string_B\string_M1}{}
\markboth{\textcolor{darkblue}{\textbf{\ipa{dzo˩˧}}}}{}
\textcolor{teal}{\mytextsc{nom}} \hspace{4pt} Ton~: LM.
\textcolor{PineGreen}{\selectlanguage{french}Lézard.} \zh{壁虎,蜥蜴,四脚蛇。}  ¶ \textcolor{darkblue}{\textbf{\ipa{dzo˩ hwæ˧-ze˧}}} \textcolor{PineGreen}{\selectlanguage{french}...a acheté (un) lézard} \zh{买了壁虎}  
 ¶ \textcolor{darkblue}{\textbf{\ipa{dzo˩ dzɯ˧-ze˩}}} \textcolor{PineGreen}{\selectlanguage{french}...a mangé (un) lézard} \zh{吃了壁虎}  
 \zh{量词}: \textcolor{darkblue}{\textbf{\ipa{mi˩}}}  \mytextsc{clf}~: \textcolor{darkblue}{\textbf{\ipa{mi˩}}} 
\lhead{\firstmark}
\rhead{\botmark}

\subsection{\hspace{-0.5cm} {\Large \textcolor{darkblue}{\textbf{\ipa{dzɯ˥}}}}\hspace{0.5cm}[\kern2pt{\textcolor{darkblue}{\textbf{\ipa{dzɯ˧˥}}}}\kern2pt]} \hypertarget{dzM\string_T1}{}
\markboth{\textcolor{darkblue}{\textbf{\ipa{dzɯ˥}}}}{}
\textcolor{teal}{\mytextsc{verbe}} \hspace{4pt} Ton~: H.
\textcolor{PineGreen}{\selectlanguage{french}Manger.} \zh{吃。}  ¶ \textcolor{darkblue}{\textbf{\ipa{le˧-dzɯ˥}}} \textcolor{PineGreen}{\selectlanguage{french}\mytextsc{accomp}} \zh{\mytextsc{实施}}  
 ¶ \textcolor{darkblue}{\textbf{\ipa{hɑ˧ dzɯ˧}}} \textcolor{PineGreen}{\selectlanguage{french}prendre un repas, manger de la nourriture} \zh{吃饭}  
 ¶ \textcolor{darkblue}{\textbf{\ipa{njɤ˧ | hɑ˧ le˧-dzɯ˥-ze˩}}} \textcolor{PineGreen}{\selectlanguage{french}J'ai mangé. J'ai pris mon repas.} \zh{我吃过饭了。}  
 ¶ \textcolor{darkblue}{\textbf{\ipa{dzɯ˧-di˧˥}}} \textcolor{PineGreen}{\selectlanguage{french}nourriture, chose à manger} \zh{吃的(东西)}  
 ¶ \textcolor{darkblue}{\textbf{\ipa{dzɯ˧-bi˩-ze˩!}}} \textcolor{PineGreen}{\selectlanguage{french}(On) va manger! / C'est l'heure de manger!} \zh{要吃饭了!}  

\lhead{\firstmark}
\rhead{\botmark}

\subsection{\hspace{-0.5cm} {\Large \textcolor{darkblue}{\textbf{\ipa{dzɯ˧tsɯ˧˥}}}}\hspace{0.5cm}[\kern2pt{\textcolor{darkblue}{\textbf{\ipa{dzɯ˧tsɯ˧˥}}}}\kern2pt]} \hypertarget{dzM\string_MtsM\string_M\string_T1}{}
\markboth{\textcolor{darkblue}{\textbf{\ipa{dzɯ˧tsɯ˧˥}}}}{}
\textcolor{teal}{\mytextsc{nom}} \hspace{4pt} Ton~: MH\#.
\textcolor{PineGreen}{\selectlanguage{french}Sorte de houx, petites feuilles à piquants.} \zh{永宁的一种灌木。}  \zh{量词}: \textcolor{darkblue}{\textbf{\ipa{pʰæ˧˥}}}  \mytextsc{clf}~: \textcolor{darkblue}{\textbf{\ipa{pʰæ˧˥}}} 
\lhead{\firstmark}
\rhead{\botmark}

\newpage
\section*{\centering- \textcolor{darkblue}{\textbf{\ipa{dʑ}}} -}
\subsection{\hspace{-0.5cm} {\Large \textcolor{darkblue}{\textbf{\ipa{dʑɤ˧bo˩}}}}\hspace{0.5cm}[\kern2pt{\textcolor{darkblue}{\textbf{\ipa{dʑɤ˩bo˩˥}}}}\kern2pt]} \hypertarget{dz£7\string_Mbo\string_B1}{}
\markboth{\textcolor{darkblue}{\textbf{\ipa{dʑɤ˧bo˩}}}}{}
\textcolor{teal}{\mytextsc{nom}} \hspace{4pt} Ton~: L\#.
\textcolor{PineGreen}{\selectlanguage{french}Grenier à céréale; selon M23, n'existe plus: était, dans le temps, un bâtiment exclusivement consacré à la conservation des céréales: un grenier.} \zh{粮仓。}  \zh{量词}: \textcolor{darkblue}{\textbf{\ipa{ɭɯ˧}}}  \mytextsc{clf}~: \textcolor{darkblue}{\textbf{\ipa{ɭɯ˧}}} 
\lhead{\firstmark}
\rhead{\botmark}

\subsection{\hspace{-0.5cm} {\Large \textcolor{darkblue}{\textbf{\ipa{dʑɤ˧do˩}}}}\hspace{0.5cm}[\kern2pt{\textcolor{darkblue}{\textbf{\ipa{dʑɤ˩do˩˥}}}}\kern2pt]} \hypertarget{dz£7\string_Mdo\string_B1}{}
\markboth{\textcolor{darkblue}{\textbf{\ipa{dʑɤ˧do˩}}}}{}
\textcolor{teal}{\mytextsc{nom}} \hspace{4pt} Ton~: L\#.
\textcolor{PineGreen}{\selectlanguage{french}Zhongdian (nom de lieu).} \zh{中甸。}  ¶ \textcolor{darkblue}{\textbf{\ipa{dʑɤ˧do˩-bɤ˩}}} \textcolor{PineGreen}{\selectlanguage{french}les Pumi de Zhongdian} \zh{中甸普米族}  

\lhead{\firstmark}
\rhead{\botmark}

\subsection{\hspace{-0.5cm} {\Large \textcolor{darkblue}{\textbf{\ipa{dʑɤ˩\textsubscript{b}}}}}\hspace{0.5cm}[\kern2pt{\textcolor{darkblue}{\textbf{\ipa{dʑɤ˥}}}}\kern2pt]} \hypertarget{dz£7\string_Bb1}{}
\markboth{\textcolor{darkblue}{\textbf{\ipa{dʑɤ˩\textsubscript{b}}}}}{}
\textcolor{teal}{\mytextsc{adjectif}} \hspace{4pt} Ton~: L\textsubscript{b}.
\textcolor{PineGreen}{\selectlanguage{french}Bon (bonne décision).} \zh{好。}  ¶ \textcolor{darkblue}{\textbf{\ipa{mɤ˧-dʑɤ˩}}} \textcolor{PineGreen}{\selectlanguage{french}pas bien, mauvais} \zh{坏}  
 ¶ \textcolor{darkblue}{\textbf{\ipa{dʑɤ˩-hĩ˥}}} \textcolor{PineGreen}{\selectlanguage{french}\mytextsc{rel}} \zh{\mytextsc{rel}}  
 ¶ \textcolor{darkblue}{\textbf{\ipa{(no˧) ɖwæ˧˥ | dʑɤ˩˥!}}} \textcolor{PineGreen}{\selectlanguage{french}Tu es quelqu'un de bien!} \zh{你很好!}  
 ¶ \textcolor{darkblue}{\textbf{\ipa{dʑɤ˩-kʰɯ˥!}}} \textcolor{PineGreen}{\selectlanguage{french}bénédiction dite au Nouvel An: “Bonnes (choses)!” =“Prospérité!” =“Bonne année!”} \zh{新年祝福:“祝一切好! / 万事如意!”}  
 ¶ \textcolor{darkblue}{\textbf{\ipa{no˧ | le˧-ʝi˥ | dʑɤ˩˥, | hĩ˧-ɳɯ˩ | do˩˥! | ʈʂʰɯ˧ | le˧-ʝi˥ | mɤ˧-dʑɤ˩, | hĩ˧-ɳɯ˩ | ʐwɤ˩˥!}}} \textcolor{PineGreen}{\selectlanguage{french}“Si tu travailles bien, les gens s'en rendent compte! S'il travail mal, les gens le disent!” = “Le bon travail est reconnu; le mauvais travail reçoit des critiques!” (Contexte: commentaire au sujet d'un mauvais livre. C'est ce qu'on disait autrefois: la belle ouvrage est reconnue; le mauvais travail vous attire des critiques.)} \zh{你做得好,人家(会)发现!他做的不好,人家(会)说(他)!}  

\lhead{\firstmark}
\rhead{\botmark}

\subsection{\hspace{-0.5cm} {\Large \textcolor{darkblue}{\textbf{\ipa{dʑɤ˩bv̩˥}}}}\hspace{0.5cm}[\kern2pt{\textcolor{darkblue}{\textbf{\ipa{dʑɤ˧bv̩˩}}}}\kern2pt]} \hypertarget{dz£7\string_Bbv\string_=\string_T1}{}
\markboth{\textcolor{darkblue}{\textbf{\ipa{dʑɤ˩bv̩˥}}}}{}
\textcolor{teal}{\mytextsc{verbe}} \hspace{4pt} Ton~: LH.
\textcolor{PineGreen}{\selectlanguage{french}Jouer.} \zh{玩,玩耍。}  ¶ \textcolor{darkblue}{\textbf{\ipa{dʑɤ˩bv̩˥ -bi˩/-ze˩}}} \textcolor{PineGreen}{\selectlanguage{french}\string_ \mytextsc{fut}.imm/\mytextsc{pfv}} \zh{要玩耍 / 玩耍了}  
 ¶ \textcolor{darkblue}{\textbf{\ipa{le˧-dʑɤ˩bv̩˩ +ze˩}}} \textcolor{PineGreen}{\selectlanguage{french}\mytextsc{accomp} \string_ \mytextsc{pfv}} \zh{玩耍了}  
 ¶ \textcolor{darkblue}{\textbf{\ipa{tʰi˧-dʑɤ˩bv̩˩, | tʰi˧-dʑɤ˩bv̩˩, | le˧-fv̩˧!}}} \textcolor{PineGreen}{\selectlanguage{french}ils jouent, ils jouent… ils sont contents! (Au sujet d'enfants qui jouent ensemble)} \zh{他们玩着玩着,很高兴!(情景:几个小孩子一起玩)}  

\lhead{\firstmark}
\rhead{\botmark}

\subsection{\hspace{-0.5cm} {\Large \textcolor{darkblue}{\textbf{\ipa{dʑɤ˩bv̩˥-di˩}}}}\hspace{0.5cm}[\kern2pt{\textcolor{darkblue}{\textbf{\ipa{xxxx non-correspondance entre le nombre de morphèmes et le nombre de tons de morphèmes}}}}\kern2pt]} \hypertarget{dz£7\string_Bbv\string_=\string_T-di\string_B1}{}
\markboth{\textcolor{darkblue}{\textbf{\ipa{dʑɤ˩bv̩˥-di˩}}}}{}
\textcolor{teal}{\mytextsc{nom}} \hspace{4pt} Ton~: LH-.
\textcolor{PineGreen}{\selectlanguage{french}Jouet.} \zh{玩具。} 
\lhead{\firstmark}
\rhead{\botmark}

\subsection{\hspace{-0.5cm} {\Large \textcolor{darkblue}{\textbf{\ipa{dʑɤ˩bv̩˧kɤ˧-sɑ˥ʁwɤ˩}}}}\hspace{0.5cm}[\kern2pt{\textcolor{darkblue}{\textbf{\ipa{dʑɤ˩bv̩˩kɤ˥sɑ˧ʁwɤ˧}}}}\kern2pt]} \hypertarget{dz£7\string_Bbv\string_=\string_Mk7\string_M-sA\string_TRw7\string_B1}{}
\markboth{\textcolor{darkblue}{\textbf{\ipa{dʑɤ˩bv̩˧kɤ˧-sɑ˥ʁwɤ˩}}}}{}
\textcolor{teal}{\mytextsc{nom}} \hspace{4pt} Ton~: LM+MH\#-.
\textcolor{PineGreen}{\selectlanguage{french}Gaoming, un village au nord-est de Yongning.} \zh{高明 (摩梭话名称的汉译:嘎撒瓦)(永宁的一个村落)。}  ¶ \textcolor{darkblue}{\textbf{\ipa{dʑɤ˩bv̩˧kɤ˧-sɑ˥ʁwɤ˩, | hi˩ʁwɤ˩-lo˥, | æ˩mi˧-ʁwɤ\#˥, | lɑ˧lo˧-ʁwɤ˥, | lɑ˧ŋwɤ˧, | bɤ˧tsʰo˧gv̩˥, | ə˧lɑ˧-ʁwɤ\#˥, | gæ˧ɻæ˩, | qʰæ˧tɕʰi˧, | tʰo˧ʈɯ\#˥}}} \textcolor{PineGreen}{\selectlanguage{french}les dix villages comptant traditionnellement comme faisant partie de Yongning} \zh{摩梭传统地理概念中,属于永宁的十个村落}  

\lhead{\firstmark}
\rhead{\botmark}

\subsection{\hspace{-0.5cm} {\Large \textcolor{darkblue}{\textbf{\ipa{dʑɤ˩bv̩˥-ʁwɤ˩}}}}\hspace{0.5cm}[\kern2pt{\textcolor{darkblue}{\textbf{\ipa{dʑɤ˩bv̩˧˥ʁwɤ˧}}}}\kern2pt]} \hypertarget{dz£7\string_Bbv\string_=\string_T-Rw7\string_B1}{}
\markboth{\textcolor{darkblue}{\textbf{\ipa{dʑɤ˩bv̩˥-ʁwɤ˩}}}}{}
\textcolor{teal}{\mytextsc{nom}} \hspace{4pt} Ton~: LH-.
\textcolor{PineGreen}{\selectlanguage{french}Jiabowa (nom de village).} \zh{甲波瓦(永宁的一个村落)。} 
\lhead{\firstmark}
\rhead{\botmark}

\subsection{\hspace{-0.5cm} {\Large \textcolor{darkblue}{\textbf{\ipa{dʑɤ˩ɕjɤ˩}}}}\hspace{0.5cm}[\kern2pt{\textcolor{darkblue}{\textbf{\ipa{xxxx non-correspondance entre le nombre de morphèmes et le nombre de tons de morphèmes}}}}\kern2pt]} \hypertarget{dz£7\string_Bs£j7\string_B1}{}
\markboth{\textcolor{darkblue}{\textbf{\ipa{dʑɤ˩ɕjɤ˩}}}}{}
\textcolor{teal}{\mytextsc{nom}} \hspace{4pt} Ton~: L.
\textcolor{PineGreen}{\selectlanguage{french}Semelle (en paille); chausson (en paille).} \zh{鞋垫。} 
\lhead{\firstmark}
\rhead{\botmark}

\subsection{\hspace{-0.5cm} {\Large \textcolor{darkblue}{\textbf{\ipa{dʑɤ˩kʰwɤ˧}}}}\hspace{0.5cm}[\kern2pt{\textcolor{darkblue}{\textbf{\ipa{xxxx ton non trouvé, à faire manuellement...}}}}\kern2pt]} \hypertarget{dz£7\string_Bk\string_hw7\string_M1}{}
\markboth{\textcolor{darkblue}{\textbf{\ipa{dʑɤ˩kʰwɤ˧}}}}{}
\textcolor{teal}{\mytextsc{nom}} \hspace{4pt} Ton~: LM.
\textcolor{PineGreen}{\selectlanguage{french}Distance.} \zh{距离。}  ¶ \textcolor{darkblue}{\textbf{\ipa{no˧ | ʈʂʰɯ˧ | ə˩-ʐæ˥ʂæ˩? | dʑɤ˩kʰwɤ˧ ə˩-di˩? | - dʑɤ˩˥ | dʑɤ˩kʰwɤ˧ mɤ˧-di˥! | mɤ˧-ʐæ˩ʂæ˩!}}} \textcolor{PineGreen}{\selectlanguage{french}tu es loin de lui? Y a-t-il de la distance entre vous? (=vous êtes proches/intimes, ou pas?) - Non, il n'y a guère de distance! Nous ne somme pas éloignés!} \zh{你们很熟吗?/ 你们很亲吗? - 不很熟!/ 不很亲!}  

\lhead{\firstmark}
\rhead{\botmark}

\subsection{\hspace{-0.5cm} {\Large \textcolor{darkblue}{\textbf{\ipa{dʑɤ˩pi\#˥}}}}\hspace{0.5cm}[\kern2pt{\textcolor{darkblue}{\textbf{\ipa{dʑɤ˩pi˥}}}}\kern2pt]} \hypertarget{dz£7\string_Bpi\#\string_T1}{}
\markboth{\textcolor{darkblue}{\textbf{\ipa{dʑɤ˩pi\#˥}}}}{}
\textcolor{teal}{\mytextsc{adjectif}} \hspace{4pt} Ton~: LM+\#H.
\textcolor{PineGreen}{\selectlanguage{french}Beaucoup.} \zh{多。}  ¶ \textcolor{darkblue}{\textbf{\ipa{dʑɤ˩pi˧ ʝi˧}}} \textcolor{PineGreen}{\selectlanguage{french}C'est très utile} \zh{很有用、有很多用处}  
 ¶ \textcolor{darkblue}{\textbf{\ipa{dʑɤ˩pi˧ dʑo˧!}}} \textcolor{PineGreen}{\selectlanguage{french}J'en ai beaucoup! (possession)} \zh{我有很多!}  
 ¶ \textcolor{darkblue}{\textbf{\ipa{dʑɤ˩pi˧ dʑo˧˥}}} \textcolor{PineGreen}{\selectlanguage{french}Il en existe beaucoup, il y en a beaucoup. (Note: existentiel, et non possession)} \zh{有很多。}  

\lhead{\firstmark}
\rhead{\botmark}

\subsection{\hspace{-0.5cm} {\Large \textcolor{darkblue}{\textbf{\ipa{dʑɤ˩qʰɑ˥}}}}\hspace{0.5cm}[\kern2pt{\textcolor{darkblue}{\textbf{\ipa{dʑɤ˩qʰɑ˩˥}}}}\kern2pt]} \hypertarget{dz£7\string_Bq\string_hA\string_T1}{}
\markboth{\textcolor{darkblue}{\textbf{\ipa{dʑɤ˩qʰɑ˥}}}}{}
\textcolor{teal}{\mytextsc{adverbe}} \hspace{4pt} Ton~: LH.
\textcolor{PineGreen}{\selectlanguage{french}Continuellement, par un effort soutenu.} \zh{一直、一个劲地。}  ¶ \textcolor{darkblue}{\textbf{\ipa{dʑɤ˩qʰɑ˥ ʈɤ˩}}} \textcolor{PineGreen}{\selectlanguage{french}tirer en un effort continu} \zh{一个劲地拉}  
 ¶ \textcolor{darkblue}{\textbf{\ipa{dʑɤ˩qʰɑ˥ mi˩}}} \textcolor{PineGreen}{\selectlanguage{french}appuyer en un effort continu} \zh{一个劲地推}  
 ¶ \textcolor{darkblue}{\textbf{\ipa{dʑɤ˩qʰɑ˥ lɑ˩}}} \textcolor{PineGreen}{\selectlanguage{french}frapper continuellement, en un effort continu} \zh{一个劲地打}  

\lhead{\firstmark}
\rhead{\botmark}

\subsection{\hspace{-0.5cm} {\Large \textcolor{darkblue}{\textbf{\ipa{dʑɤ˩so˧}}}}\hspace{0.5cm}[\kern2pt{\textcolor{darkblue}{\textbf{\ipa{dʑɤ˧so˧}}}}\kern2pt]} \hypertarget{dz£7\string_Bso\string_M1}{}
\markboth{\textcolor{darkblue}{\textbf{\ipa{dʑɤ˩so˧}}}}{}
\textcolor{teal}{\mytextsc{adjectif}} \hspace{4pt} Ton~: LM.
\textcolor{PineGreen}{\selectlanguage{french}En abondance, beaucoup.} \zh{好几(个)。}  ¶ \textcolor{darkblue}{\textbf{\ipa{dʑɤ˩-so˧ ɲi˧}}} \textcolor{PineGreen}{\selectlanguage{french}beaucoup de jours, longtemps} \zh{好几天}  

\lhead{\firstmark}
\rhead{\botmark}

\subsection{\hspace{-0.5cm} {\Large \textcolor{darkblue}{\textbf{\ipa{dʑɤ˩tsʰi\#˥}}}}\hspace{0.5cm}[\kern2pt{\textcolor{darkblue}{\textbf{\ipa{xxxx non-correspondance entre le nombre de morphèmes et le nombre de tons de morphèmes}}}}\kern2pt]} \hypertarget{dz£7\string_Bts\string_hi\#\string_T1}{}
\markboth{\textcolor{darkblue}{\textbf{\ipa{dʑɤ˩tsʰi\#˥}}}}{}
\textcolor{teal}{\mytextsc{nom}} \hspace{4pt} Ton~: LM+\#H.
\textcolor{PineGreen}{\selectlanguage{french}Prénom masculin.} \zh{男性名字。} 
\lhead{\firstmark}
\rhead{\botmark}

\subsection{\hspace{-0.5cm} {\Large \textcolor{darkblue}{\textbf{\ipa{dʑɤ˩tsʰi˧-ɖɯ˩mɑ˩}}}}\hspace{0.5cm}[\kern2pt{\textcolor{darkblue}{\textbf{\ipa{xxxx non-correspondance entre le nombre de morphèmes et le nombre de tons de morphèmes}}}}\kern2pt]} \hypertarget{dz£7\string_Bts\string_hi\string_M-d`M\string_BmA\string_B1}{}
\markboth{\textcolor{darkblue}{\textbf{\ipa{dʑɤ˩tsʰi˧-ɖɯ˩mɑ˩}}}}{}
\textcolor{teal}{\mytextsc{nom}} \hspace{4pt} Ton~: LM-L.
\textcolor{PineGreen}{\selectlanguage{french}Prénom féminin.} \zh{女性名字。} 
\lhead{\firstmark}
\rhead{\botmark}

\subsection{\hspace{-0.5cm} {\Large \textcolor{darkblue}{\textbf{\ipa{dʑɤ˩tsʰi˧-tsi˩mv̩˩}}}}\hspace{0.5cm}[\kern2pt{\textcolor{darkblue}{\textbf{\ipa{dʑɤ˩tsʰi˧tsi˩mv̩˩}}}}\kern2pt]} \hypertarget{dz£7\string_Bts\string_hi\string_M-tsi\string_Bmv\string_=\string_B1}{}
\markboth{\textcolor{darkblue}{\textbf{\ipa{dʑɤ˩tsʰi˧-tsi˩mv̩˩}}}}{}
\textcolor{teal}{\mytextsc{nom}} \hspace{4pt} Ton~: LM-L.
\textcolor{PineGreen}{\selectlanguage{french}Drapeau de prière; on en attache sur un arbre proche de la maison, ou sur le sommet de la maison.} \zh{经幡、风马旗(挂在家旁边的树上,或房顶上)。}  \zh{量词}: \textcolor{darkblue}{\textbf{\ipa{dzi˩}}}  \mytextsc{clf}~: \textcolor{darkblue}{\textbf{\ipa{dzi˩}}} 
\lhead{\firstmark}
\rhead{\botmark}

\subsection{\hspace{-0.5cm} {\Large \textcolor{darkblue}{\textbf{\ipa{dʑɤ˩tsʰo˧}}}}\hspace{0.5cm}[\kern2pt{\textcolor{darkblue}{\textbf{\ipa{dʑɤ˩tsʰo˥}}}}\kern2pt]} \hypertarget{dz£7\string_Bts\string_ho\string_M1}{}
\markboth{\textcolor{darkblue}{\textbf{\ipa{dʑɤ˩tsʰo˧}}}}{}
\textcolor{teal}{\mytextsc{verbe}} \hspace{4pt} Ton~: LM.
\textcolor{PineGreen}{\selectlanguage{french}Danser.} \zh{跳舞。}  ¶ \textcolor{darkblue}{\textbf{\ipa{ʈʂʰɯ˧ | dʑɤ˩tsʰo˧ mɤ˧-dʑɤ˩!}}} \textcolor{PineGreen}{\selectlanguage{french}il/elle ne danse pas bien!} \zh{他舞跳得不好!}  
 ¶ \textcolor{darkblue}{\textbf{\ipa{dʑɤ˩tsʰo˧ | ɖɯ˧-hɑ̃˧ tsʰo˧}}} \textcolor{PineGreen}{\selectlanguage{french}danser (toute) une soirée} \zh{跳一整夜舞}  
 ¶ \textcolor{darkblue}{\textbf{\ipa{ʑi˧qʰwɤ˧-ʂɯ˧-qo˧ | dʑɤ˩tsʰo˧ ʁɑ˧ ʂe˩}}} \textcolor{PineGreen}{\selectlanguage{french}organiser une pendaison de crémaillère dans une nouvelle maison} \zh{在新房子举办乔迁宴会}  

\lhead{\firstmark}
\rhead{\botmark}

\subsection{\hspace{-0.5cm} {\Large \textcolor{darkblue}{\textbf{\ipa{dʑi˥}}} \textsubscript{1}}\hspace{0.5cm}[\kern2pt{\textcolor{darkblue}{\textbf{\ipa{dʑi˥}}}}\kern2pt]} \hypertarget{dz£i\string_T1}{}
\markboth{\textcolor{darkblue}{\textbf{\ipa{dʑi˥}}} \textsubscript{1}}{}
\textcolor{teal}{\mytextsc{nom}} \hspace{4pt} Ton~: \#H.
\textcolor{PineGreen}{\selectlanguage{french}Urine.} \zh{尿。}  ¶ \textcolor{darkblue}{\textbf{\ipa{dʑi˧ bæ˥}}} \textcolor{PineGreen}{\selectlanguage{french}balayer l'urine} \zh{扫尿}  
 ¶ \textcolor{darkblue}{\textbf{\ipa{dʑi˧-lɑ˩ | qʰæ˧}}} \textcolor{PineGreen}{\selectlanguage{french}excréments: urine et fèces} \zh{大小便的统称}  

\lhead{\firstmark}
\rhead{\botmark}

\subsection{\hspace{-0.5cm} {\Large \textcolor{darkblue}{\textbf{\ipa{dʑi˥}}} \textsubscript{2}}\hspace{0.5cm}[\kern2pt{\textcolor{darkblue}{\textbf{\ipa{dʑi˥}}}}\kern2pt]} \hypertarget{dz£i\string_T2}{}
\markboth{\textcolor{darkblue}{\textbf{\ipa{dʑi˥}}} \textsubscript{2}}{}
\textcolor{teal}{\mytextsc{nom}} \hspace{4pt} Ton~: H.
\textcolor{PineGreen}{\selectlanguage{french}Habit, vêtement (mot monosyllabique).} \zh{衣服。}  ¶ \textcolor{darkblue}{\textbf{\ipa{kʰv̩˧ʂɯ˥, | dʑi˧ qæ˧!}}} \textcolor{PineGreen}{\selectlanguage{french}Au Nouvel An, on change de vêtements/ on porte des vêtements neufs!} \zh{过年,换衣服! / 过年,要穿新衣服!}  
 ¶ \textcolor{darkblue}{\textbf{\ipa{dʑi˧ qæ˧-ze˩!}}} \textcolor{PineGreen}{\selectlanguage{french}(il/elle) a changé de vêtements!} \zh{换衣服了!}  
 ¶ \textcolor{darkblue}{\textbf{\ipa{nɑ˩-dʑi\#˥}}} \textcolor{PineGreen}{\selectlanguage{french}le vêtement des Na} \zh{摩梭服装}  
 ¶ \textcolor{darkblue}{\textbf{\ipa{hæ˧-dʑi\#˥}}} \textcolor{PineGreen}{\selectlanguage{french}le vêtement des Chinois (Han)} \zh{汉族服装}  
 \zh{量词}: \textcolor{darkblue}{\textbf{\ipa{ɭɯ˧}}}  \mytextsc{clf}~: \textcolor{darkblue}{\textbf{\ipa{ɭɯ˧}}} 
\lhead{\firstmark}
\rhead{\botmark}

\subsection{\hspace{-0.5cm} {\Large \textcolor{darkblue}{\textbf{\ipa{dʑi˧hṽ˥\$}}}}\hspace{0.5cm}[\kern2pt{\textcolor{darkblue}{\textbf{\ipa{dʑi˧hṽ˧˥}}}}\kern2pt]} \hypertarget{dz£i\string_Mhv\string_~\string_T\$1}{}
\markboth{\textcolor{darkblue}{\textbf{\ipa{dʑi˧hṽ˥\$}}}}{}
\textcolor{teal}{\mytextsc{nom}} \hspace{4pt} Ton~: H\$.
\textcolor{PineGreen}{\selectlanguage{french}Habit, vêtement (terme générique).} \zh{衣服。}  \zh{量词}: \textcolor{darkblue}{\textbf{\ipa{ɭɯ˧}}}  \mytextsc{clf}~: \textcolor{darkblue}{\textbf{\ipa{ɭɯ˧}}} 
\lhead{\firstmark}
\rhead{\botmark}

\subsection{\hspace{-0.5cm} {\Large \textcolor{darkblue}{\textbf{\ipa{dʑi˧mi\#˥}}}}\hspace{0.5cm}[\kern2pt{\textcolor{darkblue}{\textbf{\ipa{dʑi˧mi˥}}}}\kern2pt]} \hypertarget{dz£i\string_Mmi\#\string_T1}{}
\markboth{\textcolor{darkblue}{\textbf{\ipa{dʑi˧mi\#˥}}}}{}
\textcolor{teal}{\mytextsc{nom}} \hspace{4pt} Ton~: \#H.
\textcolor{PineGreen}{\selectlanguage{french}Buffle (femelle).} \zh{母水牛。}  ¶ \textcolor{darkblue}{\textbf{\ipa{dʑi˧mi˧ tʰv̩˧-pʰo˩}}} \textcolor{PineGreen}{\selectlanguage{french}\mytextsc{n}+\mytextsc{dem}+\mytextsc{clf}} \zh{这头母水牛}  
 ¶ \textcolor{darkblue}{\textbf{\ipa{dʑi˧mi˧-dʑi˧zo\#˥ / dʑi˧mi˧-dʑi˥zo˩}}} \textcolor{PineGreen}{\selectlanguage{french}buffles femelle et mâle} \zh{母水牛与公水牛}  
 \zh{量词}: \textcolor{darkblue}{\textbf{\ipa{pʰo˧˥}}}  \mytextsc{clf}~: \textcolor{darkblue}{\textbf{\ipa{pʰo˧˥}}} 
\lhead{\firstmark}
\rhead{\botmark}

\subsection{\hspace{-0.5cm} {\Large \textcolor{darkblue}{\textbf{\ipa{dʑi˧zo\#˥}}}}\hspace{0.5cm}[\kern2pt{\textcolor{darkblue}{\textbf{\ipa{dʑi˩zo˩˥}}}}\kern2pt]} \hypertarget{dz£i\string_Mzo\#\string_T1}{}
\markboth{\textcolor{darkblue}{\textbf{\ipa{dʑi˧zo\#˥}}}}{}
\textcolor{teal}{\mytextsc{nom}} \hspace{4pt} Ton~: \#H.
\textcolor{PineGreen}{\selectlanguage{french}Buffle (enfant mâle).} \zh{小水牛(水牛崽子),一般指公的小水牛。}  ¶ \textcolor{darkblue}{\textbf{\ipa{dʑi˧zo˧ tʰv̩˧-ɭɯ\#˥}}} \textcolor{PineGreen}{\selectlanguage{french}\mytextsc{n}+\mytextsc{dem}+\mytextsc{clf}} \zh{这只水牛崽子}  
 \zh{量词}: \textcolor{darkblue}{\textbf{\ipa{ɭɯ˧}}}  \mytextsc{clf}~: \textcolor{darkblue}{\textbf{\ipa{ɭɯ˧}}} 
\lhead{\firstmark}
\rhead{\botmark}

\subsection{\hspace{-0.5cm} {\Large \textcolor{darkblue}{\textbf{\ipa{dʑi˩wɤ˩}}}}\hspace{0.5cm}[\kern2pt{\textcolor{darkblue}{\textbf{\ipa{dʑi˩wɤ˩˥}}}}\kern2pt]} \hypertarget{dz£i\string_Bw7\string_B1}{}
\markboth{\textcolor{darkblue}{\textbf{\ipa{dʑi˩wɤ˩}}}}{}
\textcolor{teal}{\mytextsc{nom}} \hspace{4pt} Ton~: L.
\textcolor{PineGreen}{\selectlanguage{french}Étriers.} \zh{马镫。}  \zh{量词}: \textcolor{darkblue}{\textbf{\ipa{dze˩}}}  \mytextsc{clf}~: \textcolor{darkblue}{\textbf{\ipa{dze˩}}} 
\lhead{\firstmark}
\rhead{\botmark}

\subsection{\hspace{-0.5cm} {\Large \textcolor{darkblue}{\textbf{\ipa{‑dʑo˥}}}}\hspace{0.5cm}[\kern2pt{\textcolor{darkblue}{\textbf{\ipa{dʑo˥}}}}\kern2pt]} \hypertarget{‑dz£o\string_T1}{}
\markboth{\textcolor{darkblue}{\textbf{\ipa{‑dʑo˥}}}}{}
\textcolor{teal}{\mytextsc{suffixe}} \hspace{4pt} Ton~: H.
\textcolor{PineGreen}{\selectlanguage{french}Marqueur de topic.} \zh{\mytextsc{主题。}} 
\lhead{\firstmark}
\rhead{\botmark}

\subsection{\hspace{-0.5cm} {\Large \textcolor{darkblue}{\textbf{\ipa{‑dʑo˧}}}}\hspace{0.5cm}[\kern2pt{\textcolor{darkblue}{\textbf{\ipa{dʑo˥}}}}\kern2pt]} \hypertarget{‑dz£o\string_M1}{}
\markboth{\textcolor{darkblue}{\textbf{\ipa{‑dʑo˧}}}}{}
\textcolor{teal}{\mytextsc{suffixe}} \hspace{4pt} Ton~: M.
\textcolor{PineGreen}{\selectlanguage{french}Aspect progressif.} \zh{\mytextsc{进行式。}} 
\lhead{\firstmark}
\rhead{\botmark}

\subsection{\hspace{-0.5cm} {\Large \textcolor{darkblue}{\textbf{\ipa{dʑo˧\textsubscript{b}}}}}\hspace{0.5cm}[\kern2pt{\textcolor{darkblue}{\textbf{\ipa{dʑo˩˥}}}}\kern2pt]} \hypertarget{dz£o\string_Mb1}{}
\markboth{\textcolor{darkblue}{\textbf{\ipa{dʑo˧\textsubscript{b}}}}}{}
\textcolor{teal}{\mytextsc{verbe}} \hspace{4pt} Ton~: M\textsubscript{b}.
\textcolor{PineGreen}{\selectlanguage{french}Posséder; y avoir; avoir un objet, avoir une pensée…; autrefois, il y avait une mère et sa fille...} \zh{有,拥有。}  ¶ \textcolor{darkblue}{\textbf{\ipa{mɤ˧-dʑo˧-ze˧!}}} \textcolor{PineGreen}{\selectlanguage{french}Il n'y en a plus!} \zh{没有了!}  
 ¶ \textcolor{darkblue}{\textbf{\ipa{le˧-dʑo˧-ze˧!}}} \textcolor{PineGreen}{\selectlanguage{french}ça y est, il y en a!} \zh{有了!}  
 ¶ \textcolor{darkblue}{\textbf{\ipa{ʈʂʰɯ˧ | ɑ˩ʁo˧ | ɖɯ˧-sɑ˥ | mɤ˧-dʑo˧!}}} \textcolor{PineGreen}{\selectlanguage{french}Il n'y a rien chez lui = sa maison est dans l'indigence} \zh{他家什么也没有 = 他家贫穷}  
 ¶ \textcolor{darkblue}{\textbf{\ipa{njɤ˧ | mv̩˩zɯ˩-ni˥mi˩ | ŋi˧-kv̩˧ dʑo˧˥!}}} \textcolor{PineGreen}{\selectlanguage{french}j’ai deux frères et sœurs!} \zh{我有两个兄弟姐妹!}  
 ¶ \textcolor{darkblue}{\textbf{\ipa{dʑo˧-tʰɑ˧˥!}}} \textcolor{PineGreen}{\selectlanguage{french}Cela arrive / il peut y en avoir / c'est possible qu'il y en ait!} \zh{会有的!}  
 ¶ \textcolor{darkblue}{\textbf{\ipa{tso˧\textasciitilde{}tso˧ dʑo˧}}} \textcolor{PineGreen}{\selectlanguage{french}il y a des choses} \zh{他有东西}  
 ¶ \textcolor{darkblue}{\textbf{\ipa{njɤ˧-ɻ̍˩, | ɖɯ˧-ɭɯ˧-lɑ˧ dʑo˥!}}} \textcolor{PineGreen}{\selectlanguage{french}Nous, on n'en a qu'un(, d'enfant)!} \zh{我们只有一个(孩子)!}  
 ¶ \textcolor{darkblue}{\textbf{\ipa{ɖwæ˧˥ | dʑo˧-ɲi˥!}}} \textcolor{PineGreen}{\selectlanguage{french}Il y en a des quantités! (ex.: au sujet du bois de construction qu'on prépare en vue de la construction d'une maison)} \zh{有很多!(如:准备建房,积累的木材有很多)}  

\lhead{\firstmark}
\rhead{\botmark}

\subsection{\hspace{-0.5cm} {\Large \textcolor{darkblue}{\textbf{\ipa{dʑo˩\textsubscript{b}}}}}\hspace{0.5cm}[\kern2pt{\textcolor{darkblue}{\textbf{\ipa{dʑo˥}}}}\kern2pt]} \hypertarget{dz£o\string_Bb1}{}
\markboth{\textcolor{darkblue}{\textbf{\ipa{dʑo˩\textsubscript{b}}}}}{}
\textcolor{teal}{\mytextsc{verbe}} \hspace{4pt} Ton~: L\textsubscript{b}.
\textcolor{PineGreen}{\selectlanguage{french}Existentiel pour les êtres animés (dont les personnes).} \zh{存在动词:有,存在着。如:某人在家或家里有几口人。}  ¶ \textcolor{darkblue}{\textbf{\ipa{mɤ˧-dʑo˩}}} \textcolor{PineGreen}{\selectlanguage{french}\mytextsc{neg}} \zh{没有、不在}  
 ¶ \textcolor{darkblue}{\textbf{\ipa{ʈʂʰɯ˧ | ɑ˩ʁo˧ mɤ˧-dʑo˩!}}} \textcolor{PineGreen}{\selectlanguage{french}Il/elle n'est pas à la maison!} \zh{他不在家!}  

\lhead{\firstmark}
\rhead{\botmark}

\subsection{\hspace{-0.5cm} {\Large \textcolor{darkblue}{\textbf{\ipa{‑dʑɯ˧}}}}\hspace{0.5cm}[\kern2pt{\textcolor{darkblue}{\textbf{\ipa{dʑɯ˥}}}}\kern2pt]} \hypertarget{‑dz£M\string_M1}{}
\markboth{\textcolor{darkblue}{\textbf{\ipa{‑dʑɯ˧}}}}{}
\textcolor{teal}{\mytextsc{suffixe}} \hspace{4pt} Ton~: M.
\textcolor{PineGreen}{\selectlanguage{french}\mytextsc{expérientiel}.} \zh{……过。} 
\lhead{\firstmark}
\rhead{\botmark}

\subsection{\hspace{-0.5cm} {\Large \textcolor{darkblue}{\textbf{\ipa{dʑɯ˧}}}}\hspace{0.5cm}[\kern2pt{\textcolor{darkblue}{\textbf{\ipa{dʑɯ˥}}}}\kern2pt]} \hypertarget{dz£M\string_M1}{}
\markboth{\textcolor{darkblue}{\textbf{\ipa{dʑɯ˧}}}}{}
\textcolor{teal}{\mytextsc{nom}} \hspace{4pt} Ton~: M.
\textcolor{PineGreen}{\selectlanguage{french}Le moment (de), l'heure (de).} \zh{……的时间。}  ¶ \textcolor{darkblue}{\textbf{\ipa{ʈʂʰwɤ˩ dzɯ˩-bi˩-dʑɯ˩˥}}} \textcolor{PineGreen}{\selectlanguage{french}l'heure du repas du soir} \zh{吃晚餐的时间}  
 ¶ \textcolor{darkblue}{\textbf{\ipa{ɑ˩pʰo˩ bi˩-dʑɯ˩˥}}} \textcolor{PineGreen}{\selectlanguage{french}l'heure d'aller dehors, le moment d'aller dehors} \zh{出去的(合适)时间}  
 ¶ \textcolor{darkblue}{\textbf{\ipa{le˧-ʑi˧-bi˧-dʑɯ˧ tʰv̩˧-ze˩!}}} \textcolor{PineGreen}{\selectlanguage{french}Il est l'heure d'aller dormir!} \zh{睡觉的时间到了!}  
 ¶ \textcolor{darkblue}{\textbf{\ipa{ʐo˩ dzɯ˩-bi˩-dʑɯ˩˥}}} \textcolor{PineGreen}{\selectlanguage{french}l'heure du déjeuner} \zh{午饭的时间}  

\lhead{\firstmark}
\rhead{\botmark}

\subsection{\hspace{-0.5cm} {\Large \textcolor{darkblue}{\textbf{\ipa{dʑɯ˧dv̩˧}}}}\hspace{0.5cm}[\kern2pt{\textcolor{darkblue}{\textbf{\ipa{xxxx non-correspondance entre le nombre de morphèmes et le nombre de tons de morphèmes}}}}\kern2pt]} \hypertarget{dz£M\string_Mdv\string_=\string_M1}{}
\markboth{\textcolor{darkblue}{\textbf{\ipa{dʑɯ˧dv̩˧}}}}{}
\textcolor{teal}{\mytextsc{nom}} \hspace{4pt} Ton~: M.
\textcolor{PineGreen}{\selectlanguage{french}Ver de terre.} \zh{蚯蚓。}  ¶ \textcolor{darkblue}{\textbf{\ipa{dʑɯ˧dv̩˧-mi˩, | ə˩-dʑo˩˥?}}} \textcolor{PineGreen}{\selectlanguage{french}Les vers de terre femelle, ça existe? (Cette phrase permet d'éliciter une forme associant 'ver de terre' au suffixe 'femelle', dans l'idée d'étudier les règles tonales productives en synchronie pour les suffixes de genre.)} \zh{有母蚯蚓吗?}  
 ¶ \textcolor{darkblue}{\textbf{\ipa{dʑɯ˧dv̩˧-pʰv̩˩, | ə˩-dʑo˩˥?}}} \textcolor{PineGreen}{\selectlanguage{french}Les vers de terre mâle, ça existe? (Cette phrase permet d'éliciter une forme associant 'ver de terre' au suffixe 'mâle', dans l'idée d'étudier les règles tonales productives en synchronie pour les suffixes de genre.)} \zh{有公蚯蚓吗?}  
 \zh{量词}: \textcolor{darkblue}{\textbf{\ipa{kʰɯ˩}}}  \mytextsc{clf}~: \textcolor{darkblue}{\textbf{\ipa{kʰɯ˩}}} 
\lhead{\firstmark}
\rhead{\botmark}

\subsection{\hspace{-0.5cm} {\Large \textcolor{darkblue}{\textbf{\ipa{dʑɯ˧dze˧mi\#˥}}}}\hspace{0.5cm}[\kern2pt{\textcolor{darkblue}{\textbf{\ipa{dʑɯ˩dze˩mi˩˥}}}}\kern2pt]} \hypertarget{dz£M\string_Mdze\string_Mmi\#\string_T1}{}
\markboth{\textcolor{darkblue}{\textbf{\ipa{dʑɯ˧dze˧mi\#˥}}}}{}
\textcolor{teal}{\mytextsc{nom}} \hspace{4pt} Ton~: \#H.
\textcolor{PineGreen}{\selectlanguage{french}Cigale.} \zh{蝉。}  ¶ \textcolor{darkblue}{\textbf{\ipa{dʑɯ˧dze˧-mi˧ tʰv̩˧-mi˧˥ / dʑɯ˧dze˧-mi˧ tʰv̩˧-mi˥\#}}} \textcolor{PineGreen}{\selectlanguage{french}\mytextsc{n}+\mytextsc{dem}+\mytextsc{clf}} \zh{这只蝉}  
 \zh{量词}: \textcolor{darkblue}{\textbf{\ipa{mi˩}}}  \mytextsc{clf}~: \textcolor{darkblue}{\textbf{\ipa{mi˩}}} 
\lhead{\firstmark}
\rhead{\botmark}

\subsection{\hspace{-0.5cm} {\Large \textcolor{darkblue}{\textbf{\ipa{dʑɯ˧ki˥}}}}\hspace{0.5cm}[\kern2pt{\textcolor{darkblue}{\textbf{\ipa{dʑɯ˩ki˩˥}}}}\kern2pt]} \hypertarget{dz£M\string_Mki\string_T1}{}
\markboth{\textcolor{darkblue}{\textbf{\ipa{dʑɯ˧ki˥}}}}{}
\textcolor{teal}{\mytextsc{nom}} \hspace{4pt} Ton~: H\#.
\textcolor{PineGreen}{\selectlanguage{french}Gaine: large ceinture en tissu, qui peut servir à porter un enfant; aussi: ceinture (terme générique).} \zh{布带子,用来背小孩的带子,腰带。}  \zh{量词}: \textcolor{darkblue}{\textbf{\ipa{kʰɯ˩}}}  \mytextsc{clf}~: \textcolor{darkblue}{\textbf{\ipa{kʰɯ˩}}} 
\lhead{\firstmark}
\rhead{\botmark}

\subsection{\hspace{-0.5cm} {\Large \textcolor{darkblue}{\textbf{\ipa{dʑɯ˧-li˧}}}}\hspace{0.5cm}[\kern2pt{\textcolor{darkblue}{\textbf{\ipa{xxxx non-correspondance entre le nombre de morphèmes et le nombre de tons de morphèmes}}}}\kern2pt]} \hypertarget{dz£M\string_M-li\string_M1}{}
\markboth{\textcolor{darkblue}{\textbf{\ipa{dʑɯ˧-li˧}}}}{}
\textcolor{teal}{\mytextsc{verbe}} \hspace{4pt} Ton~: M.
\textcolor{PineGreen}{\selectlanguage{french}Irriguer.} \zh{灌溉。}  ¶ \textcolor{darkblue}{\textbf{\ipa{dʑɯ˧-li˧-ze˧}}} \textcolor{PineGreen}{\selectlanguage{french}\mytextsc{pfv}} \zh{灌溉了}  
 ¶ \textcolor{darkblue}{\textbf{\ipa{dʑɯ˧-mɤ˧-li˧-hĩ˧ lv̩˧}}} \textcolor{PineGreen}{\selectlanguage{french}champ sec/pluvial: “champ qu'on n'irrigue pas”} \zh{旱田:不灌溉的田}  

\lhead{\firstmark}
\rhead{\botmark}

\subsection{\hspace{-0.5cm} {\Large \textcolor{darkblue}{\textbf{\ipa{dʑɯ˧ɭɯ˧}}}}\hspace{0.5cm}[\kern2pt{\textcolor{darkblue}{\textbf{\ipa{dʑɯ˧ɭɯ˧}}}}\kern2pt]} \hypertarget{dz£M\string_Ml\string_RM\string_M1}{}
\markboth{\textcolor{darkblue}{\textbf{\ipa{dʑɯ˧ɭɯ˧}}}}{}
\textcolor{teal}{\mytextsc{nom}} \hspace{4pt} Ton~: M.
\textcolor{PineGreen}{\selectlanguage{french}Millet, \textit{Panicum miliaceum}.} \zh{黍,小米。}  ¶ \textcolor{darkblue}{\textbf{\ipa{dʑɯ˧ɭɯ˧-ho\#˥}}} \textcolor{PineGreen}{\selectlanguage{french}gruau de millet} \zh{小米粥}  
\textit{Voir~:} \textcolor{darkblue}{\textbf{\ipa{dʑɯ˧njɤ˧, dʑɯ˧ʈʂʰwæ\#˥}}} 
\lhead{\firstmark}
\rhead{\botmark}

\subsection{\hspace{-0.5cm} {\Large \textcolor{darkblue}{\textbf{\ipa{dʑɯ˧mi˧}}}}\hspace{0.5cm}[\kern2pt{\textcolor{darkblue}{\textbf{\ipa{dʑɯ˧mi˧}}}}\kern2pt]} \hypertarget{dz£M\string_Mmi\string_M1}{}
\markboth{\textcolor{darkblue}{\textbf{\ipa{dʑɯ˧mi˧}}}}{}
\textcolor{teal}{\mytextsc{nom}} \hspace{4pt} Ton~: M.
\textcolor{PineGreen}{\selectlanguage{french}Grande rivière.} \zh{大河。}  \zh{量词}: \textcolor{darkblue}{\textbf{\ipa{kʰɯ˩}}}  \mytextsc{clf}~: \textcolor{darkblue}{\textbf{\ipa{kʰɯ˩}}} 
\lhead{\firstmark}
\rhead{\botmark}

\subsection{\hspace{-0.5cm} {\Large \textcolor{darkblue}{\textbf{\ipa{dʑɯ˧njɤ˧}}}}\hspace{0.5cm}[\kern2pt{\textcolor{darkblue}{\textbf{\ipa{dʑɯ˧njɤ˧}}}}\kern2pt]} \hypertarget{dz£M\string_Mnj7\string_M1}{}
\markboth{\textcolor{darkblue}{\textbf{\ipa{dʑɯ˧njɤ˧}}}}{}
\textcolor{teal}{\mytextsc{nom}} \hspace{4pt} Ton~: M.
\textcolor{PineGreen}{\selectlanguage{french}Millet, \textit{Panicum miliaceum}.} \zh{黍,小米。}  ¶ \textcolor{darkblue}{\textbf{\ipa{dʑɯ˧njɤ˧, | ʐɯ˧ tɕɤ˧˥!}}} \textcolor{PineGreen}{\selectlanguage{french}Le millet, on s'en sert pour faire du vin!} \zh{小米,用来酿酒!}  
 ¶ \textcolor{darkblue}{\textbf{\ipa{dʑɯ˧njɤ˧-hɑ\#˥}}} \textcolor{PineGreen}{\selectlanguage{french}millet cuit} \zh{小米饭}  
\textit{Voir~:} \textcolor{darkblue}{\textbf{\ipa{dʑɯ˧ɭɯ˧, dʑɯ˧ʈʂʰwæ\#˥}}} 
\lhead{\firstmark}
\rhead{\botmark}

\subsection{\hspace{-0.5cm} {\Large \textcolor{darkblue}{\textbf{\ipa{dʑɯ˧qʰɑ˧}}}}\hspace{0.5cm}[\kern2pt{\textcolor{darkblue}{\textbf{\ipa{dʑɯ˧qʰɑ˧}}}}\kern2pt]} \hypertarget{dz£M\string_Mq\string_hA\string_M1}{}
\markboth{\textcolor{darkblue}{\textbf{\ipa{dʑɯ˧qʰɑ˧}}}}{}
\textcolor{teal}{\mytextsc{nom}} \hspace{4pt} Ton~: M.
\textcolor{PineGreen}{\selectlanguage{french}Yyyy.} \zh{夏枯草。}  ¶ \textcolor{darkblue}{\textbf{\ipa{dʑɯ˧qʰɑ˧-bæ˩bæ˩}}} \textcolor{PineGreen}{\selectlanguage{french}fleurs de yyyy} \zh{夏枯草花}  
 \zh{量词}: \textcolor{darkblue}{\textbf{\ipa{qɑ˩}}}  \mytextsc{clf}~: \textcolor{darkblue}{\textbf{\ipa{qɑ˩}}} 
\lhead{\firstmark}
\rhead{\botmark}

\subsection{\hspace{-0.5cm} {\Large \textcolor{darkblue}{\textbf{\ipa{dʑɯ˧qʰv̩˩}}}}\hspace{0.5cm}[\kern2pt{\textcolor{darkblue}{\textbf{\ipa{dʑɯ˧qʰv̩˩}}}}\kern2pt]} \hypertarget{dz£M\string_Mq\string_hv\string_=\string_B1}{}
\markboth{\textcolor{darkblue}{\textbf{\ipa{dʑɯ˧qʰv̩˩}}}}{}
\textcolor{teal}{\mytextsc{nom}} \hspace{4pt} Ton~: L\#.
\textcolor{PineGreen}{\selectlanguage{french}Plante sauvage dont les graines forment de grosses boules de graines.} \zh{永宁的一种植物。}  ¶ \textcolor{darkblue}{\textbf{\ipa{dʑɯ˧qʰv̩˩-lv̩˩lv̩˩}}} \textcolor{PineGreen}{\selectlanguage{french}graines de la plante en question} \zh{这种植物的种子}  
 \zh{量词}: \textcolor{darkblue}{\textbf{\ipa{ɭɯ˧}}}  \mytextsc{clf}~: \textcolor{darkblue}{\textbf{\ipa{ɭɯ˧}}} 
\lhead{\firstmark}
\rhead{\botmark}

\subsection{\hspace{-0.5cm} {\Large \textcolor{darkblue}{\textbf{\ipa{dʑɯ˧qʰv̩˧}}}}\hspace{0.5cm}[\kern2pt{\textcolor{darkblue}{\textbf{\ipa{dʑɯ˧qʰv̩˧}}}}\kern2pt]} \hypertarget{dz£M\string_Mq\string_hv\string_=\string_M1}{}
\markboth{\textcolor{darkblue}{\textbf{\ipa{dʑɯ˧qʰv̩˧}}}}{}
\textcolor{teal}{\mytextsc{nom}} \hspace{4pt} Ton~: M.
\textcolor{PineGreen}{\selectlanguage{french}Puits.} \zh{井、水井。}  ¶ \textcolor{darkblue}{\textbf{\ipa{ɑ˩ʁo˥ | dʑɯ˧qʰv̩˧ tʰi˧-di˩.}}} \textcolor{PineGreen}{\selectlanguage{french}il y a un puits à la maison/dans la ferme.} \zh{家里有水井。}  
 \zh{量词}: \textcolor{darkblue}{\textbf{\ipa{ɭɯ˧}}}  \mytextsc{clf}~: \textcolor{darkblue}{\textbf{\ipa{ɭɯ˧}}} 
\lhead{\firstmark}
\rhead{\botmark}

\subsection{\hspace{-0.5cm} {\Large \textcolor{darkblue}{\textbf{\ipa{dʑɯ˧ʁo˩}}}}\hspace{0.5cm}[\kern2pt{\textcolor{darkblue}{\textbf{\ipa{dʑɯ˧ʁo˩}}}}\kern2pt]} \hypertarget{dz£M\string_MRo\string_B1}{}
\markboth{\textcolor{darkblue}{\textbf{\ipa{dʑɯ˧ʁo˩}}}}{}
\textcolor{teal}{\mytextsc{nom}} \hspace{4pt} Ton~: L\#.
\textcolor{PineGreen}{\selectlanguage{french}Pêche.} \zh{桃子。} 
\lhead{\firstmark}
\rhead{\botmark}

\subsection{\hspace{-0.5cm} {\Large \textcolor{darkblue}{\textbf{\ipa{dʑɯ˧ʈʂʰwæ\#˥}}}}\hspace{0.5cm}[\kern2pt{\textcolor{darkblue}{\textbf{\ipa{dʑɯ˧ʈʂʰwæ˧}}}}\kern2pt]} \hypertarget{dz£M\string_Mt`s`\string_hw\{\#\string_T1}{}
\markboth{\textcolor{darkblue}{\textbf{\ipa{dʑɯ˧ʈʂʰwæ\#˥}}}}{}
\textcolor{teal}{\mytextsc{nom}} \hspace{4pt} Ton~: \#H.
\textcolor{PineGreen}{\selectlanguage{french}Millet décortiqué, \textit{Panicum miliaceum}.} \zh{已碾的小米。} \textit{Voir~:} \textcolor{darkblue}{\textbf{\ipa{dʑɯ˧ɭɯ˧, dʑɯ˧njɤ˧}}} 
\lhead{\firstmark}
\rhead{\botmark}

\subsection{\hspace{-0.5cm} {\Large \textcolor{darkblue}{\textbf{\ipa{dʑɯ˩}}}}\hspace{0.5cm}[\kern2pt{\textcolor{darkblue}{\textbf{\ipa{dʑɯ˥}}}}\kern2pt]} \hypertarget{dz£M\string_B1}{}
\markboth{\textcolor{darkblue}{\textbf{\ipa{dʑɯ˩}}}}{}
\textcolor{teal}{\mytextsc{nom}} \hspace{4pt} Ton~: L.
\ding{202} \textcolor{PineGreen}{\selectlanguage{french}Eau.} \zh{水。}  ¶ \textcolor{darkblue}{\textbf{\ipa{dʑɯ˧ ʈʰɯ˧}}} \textcolor{PineGreen}{\selectlanguage{french}boire de l'eau} \zh{喝水}  
 ¶ \textcolor{darkblue}{\textbf{\ipa{ʈʂʰɯ˧ dʑɯ˧ ʈʰɯ˧-dʑo˧!}}} \textcolor{PineGreen}{\selectlanguage{french}il est en train de boire de l'eau!} \zh{他在喝水}  
 ¶ \textcolor{darkblue}{\textbf{\ipa{dʑɯ˧ | ɖɯ˧-ʈʰɤ˧ ʈʰɯ˧˥}}} \textcolor{PineGreen}{\selectlanguage{french}boire un peu d'eau (littéralement “une goutte d'eau”)} \zh{喝一点水(直译:‘一滴水’)}  
 ¶ \textcolor{darkblue}{\textbf{\ipa{dʑɯ˩ kʰɯ˩˥}}} \textcolor{PineGreen}{\selectlanguage{french}mettre de l'eau} \zh{放水}  
 ¶ \textcolor{darkblue}{\textbf{\ipa{dʑɯ˩ mæ˩˥}}} \textcolor{PineGreen}{\selectlanguage{french}irriguer, arroser, mettre de l’eau} \zh{浇灌、灌溉}  
 ¶ \textcolor{darkblue}{\textbf{\ipa{dʑɯ˩ qæ˩, | hɑ˩ qæ˩˥ |}}} \textcolor{PineGreen}{\selectlanguage{french}description du dépaysement que connaît le voyageur qui arrive en pays étranger et doit 'changer d'eau, changer de nourriture'. Ce dépaysement commande des stratégies de prévention de soucis de santé: en particulier, il était usuel de faire bouillir un peu de terre locale dans de l'eau, et de boire cette préparation de façon à s'accoutumer.} \zh{‘换水换土’:这个短语描述旅人到他人乡的情况,带来水土不服的危险。为了预防这类不良反应,摩梭旅人习惯水煮一点当地的土,喝下去,为了适应当地的水土。}  
 ¶ \textcolor{darkblue}{\textbf{\ipa{[F5] dʑɯ˧ | mv̩˩tɕo˧ dɑ˧˥}}} \textcolor{PineGreen}{\selectlanguage{french}l'eau coule vers le bas} \zh{水往下流}  
 \zh{量词}: \textcolor{darkblue}{\textbf{\ipa{kʰɯ˩}}} \ding{203} \textcolor{PineGreen}{\selectlanguage{french}Rivière.} \zh{河流。}  \mytextsc{clf}~: \textcolor{darkblue}{\textbf{\ipa{kʰɯ˩}}} 
\lhead{\firstmark}
\rhead{\botmark}

\subsection{\hspace{-0.5cm} {\Large \textcolor{darkblue}{\textbf{\ipa{dʑɯ˩\textsubscript{a}}}}}\hspace{0.5cm}[\kern2pt{\textcolor{darkblue}{\textbf{\ipa{dʑɯ˥}}}}\kern2pt]} \hypertarget{dz£M\string_Ba1}{}
\markboth{\textcolor{darkblue}{\textbf{\ipa{dʑɯ˩\textsubscript{a}}}}}{}
\textcolor{teal}{\mytextsc{verbe}} \hspace{4pt} Ton~: L\textsubscript{a}.
\textcolor{PineGreen}{\selectlanguage{french}Rouler, tordre (par exemple: rouler des brins, pour en faire une corde; ne s'emploie pas pour les fibres très fines, pour lesquelles on dit: \textcolor{darkblue}{\textbf{\ipa{/ʈʂwæ˧˥/}}}).} \zh{搓(搓绳子)。}  ¶ \textcolor{darkblue}{\textbf{\ipa{le˧-dʑɯ˩-ze˩}}} \textcolor{PineGreen}{\selectlanguage{french}\mytextsc{accomp} \string_ \mytextsc{pfv}} \zh{搓了}  
 ¶ \textcolor{darkblue}{\textbf{\ipa{bæ˩ dʑɯ˩˥}}} \textcolor{PineGreen}{\selectlanguage{french}tordre une corde} \zh{搓绳子}  
 ¶ \textcolor{darkblue}{\textbf{\ipa{qʰv̩˩ɖʐæ˩ dʑɯ˥}}} \textcolor{PineGreen}{\selectlanguage{french}faire une ficelle, une petite cordelette} \zh{搓一根小绳子}  
 ¶ \textcolor{darkblue}{\textbf{\ipa{ɖɯ˧-kʰwɤ˧ dʑɯ˥}}} \textcolor{PineGreen}{\selectlanguage{french}tordre un peu / tordre quelque chose} \zh{搓一下}  

\lhead{\firstmark}
\rhead{\botmark}

\subsection{\hspace{-0.5cm} {\Large \textcolor{darkblue}{\textbf{\ipa{dʑɯ˩-æ̃˩tsɯ˧}}}}\hspace{0.5cm}[\kern2pt{\textcolor{darkblue}{\textbf{\ipa{xxxx non-correspondance entre le nombre de morphèmes et le nombre de tons de morphèmes}}}}\kern2pt]} \hypertarget{dz£M\string_B-\{\string_~\string_BtsM\string_M1}{}
\markboth{\textcolor{darkblue}{\textbf{\ipa{dʑɯ˩-æ̃˩tsɯ˧}}}}{}
\textcolor{teal}{\mytextsc{nom}} \hspace{4pt} Ton~: L-LM.
\textcolor{PineGreen}{\selectlanguage{french}Gibier d'eau, sauvagine; employé pour divers oiseaux tels que: bécasseau, chevalier (\textit{Calidris}), avocette, marouette, et râle.} \zh{水禽,包括几种不同的小鸟,如:鹬。}  \zh{量词}: \textcolor{darkblue}{\textbf{\ipa{ɭɯ˧}}}  \mytextsc{clf}~: \textcolor{darkblue}{\textbf{\ipa{ɭɯ˧}}} 
\lhead{\firstmark}
\rhead{\botmark}

\subsection{\hspace{-0.5cm} {\Large \textcolor{darkblue}{\textbf{\ipa{dʑɯ˩dze˩}}}}\hspace{0.5cm}[\kern2pt{\textcolor{darkblue}{\textbf{\ipa{dʑɯ˧dze˧}}}}\kern2pt]} \hypertarget{dz£M\string_Bdze\string_B1}{}
\markboth{\textcolor{darkblue}{\textbf{\ipa{dʑɯ˩dze˩}}}}{}
\textcolor{teal}{\mytextsc{nom}} \hspace{4pt} Ton~: L.
\textcolor{PineGreen}{\selectlanguage{french}Louche utilisée pour faire la cuisine, distribuer la soupe.} \zh{舀汤的勺子。}  \zh{量词}: \textcolor{darkblue}{\textbf{\ipa{nɑ˧}}}  \mytextsc{clf}~: \textcolor{darkblue}{\textbf{\ipa{nɑ˧}}} 
\lhead{\firstmark}
\rhead{\botmark}

\subsection{\hspace{-0.5cm} {\Large \textcolor{darkblue}{\textbf{\ipa{dʑɯ˩gɤ˩di˩}}}}\hspace{0.5cm}[\kern2pt{\textcolor{darkblue}{\textbf{\ipa{dʑɯ˧gɤ˧di˧}}}}\kern2pt]} \hypertarget{dz£M\string_Bg7\string_Bdi\string_B1}{}
\markboth{\textcolor{darkblue}{\textbf{\ipa{dʑɯ˩gɤ˩di˩}}}}{}
\textcolor{teal}{\mytextsc{nom}} \hspace{4pt} Ton~: L.
\textcolor{PineGreen}{\selectlanguage{french}Palanche.} \zh{扁担。}  \zh{量词}: \textcolor{darkblue}{\textbf{\ipa{nɑ˧}}}  \mytextsc{clf}~: \textcolor{darkblue}{\textbf{\ipa{nɑ˧}}} 
\lhead{\firstmark}
\rhead{\botmark}

\subsection{\hspace{-0.5cm} {\Large \textcolor{darkblue}{\textbf{\ipa{dʑɯ˩gv̩˩}}}}\hspace{0.5cm}[\kern2pt{\textcolor{darkblue}{\textbf{\ipa{dʑɯ˩gv̩˩˥}}}}\kern2pt]} \hypertarget{dz£M\string_Bgv\string_=\string_B1}{}
\markboth{\textcolor{darkblue}{\textbf{\ipa{dʑɯ˩gv̩˩}}}}{}
\textcolor{teal}{\mytextsc{nom}} \hspace{4pt} Ton~: L.
\textcolor{PineGreen}{\selectlanguage{french}Cuve où l'on conserve l'eau potable, tonneau d'eau. A la date de l'enquête, il s'agissait d'un baril en fer.} \zh{大水桶,水槽。}  ¶ \textcolor{darkblue}{\textbf{\ipa{[F5] pv̩˩-dʑɯ˩gv̩˩˥}}} \textcolor{PineGreen}{\selectlanguage{french}même sens}  
 \zh{量词}: \textcolor{darkblue}{\textbf{\ipa{ɭɯ˧}}}  \mytextsc{clf}~: \textcolor{darkblue}{\textbf{\ipa{ɭɯ˧}}} 
\lhead{\firstmark}
\rhead{\botmark}

\subsection{\hspace{-0.5cm} {\Large \textcolor{darkblue}{\textbf{\ipa{dʑɯ˩gv̩˥}}}}\hspace{0.5cm}[\kern2pt{\textcolor{darkblue}{\textbf{\ipa{dʑɯ˩gv̩˩˥}}}}\kern2pt]} \hypertarget{dz£M\string_Bgv\string_=\string_T1}{}
\markboth{\textcolor{darkblue}{\textbf{\ipa{dʑɯ˩gv̩˥}}}}{}
\textcolor{teal}{\mytextsc{adjectif}} \hspace{4pt} Ton~: LH.
\textcolor{PineGreen}{\selectlanguage{french}Voûté, qui a le dos rond, courbé.} \zh{驼背。} 
\lhead{\firstmark}
\rhead{\botmark}

\subsection{\hspace{-0.5cm} {\Large \textcolor{darkblue}{\textbf{\ipa{dʑɯ˩hṽ˧˥}}}}\hspace{0.5cm}[\kern2pt{\textcolor{darkblue}{\textbf{\ipa{dʑɯ˩hṽ˥}}}}\kern2pt]} \hypertarget{dz£M\string_Bhv\string_~\string_M\string_T1}{}
\markboth{\textcolor{darkblue}{\textbf{\ipa{dʑɯ˩hṽ˧˥}}}}{}
\textcolor{teal}{\mytextsc{nom}} \hspace{4pt} Ton~: LM+MH\#.
\textcolor{PineGreen}{\selectlanguage{french}Mélange d'eau et de farine: par ex. de la pâte à pain, du tsamba avec de l'eau….} \zh{面和水和成的浆糊。} 
\lhead{\firstmark}
\rhead{\botmark}

\subsection{\hspace{-0.5cm} {\Large \textcolor{darkblue}{\textbf{\ipa{dʑɯ˩-hwæ˩tsɯ˥}}}}\hspace{0.5cm}[\kern2pt{\textcolor{darkblue}{\textbf{\ipa{xxxx non-correspondance entre le nombre de morphèmes et le nombre de tons de morphèmes}}}}\kern2pt]} \hypertarget{dz£M\string_B-hw\{\string_BtsM\string_T1}{}
\markboth{\textcolor{darkblue}{\textbf{\ipa{dʑɯ˩-hwæ˩tsɯ˥}}}}{}
\textcolor{teal}{\mytextsc{nom}} \hspace{4pt} Ton~: L+H\#.
\textcolor{PineGreen}{\selectlanguage{french}Musaraigne; la locutrice emploie une périphrase: “souris sauvage”.} \zh{尖鼠、鼩鼱。} 
\lhead{\firstmark}
\rhead{\botmark}

\subsection{\hspace{-0.5cm} {\Large \textcolor{darkblue}{\textbf{\ipa{dʑɯ˩kʰi˩}}}}\hspace{0.5cm}[\kern2pt{\textcolor{darkblue}{\textbf{\ipa{dʑɯ˩kʰi˥}}}}\kern2pt]} \hypertarget{dz£M\string_Bk\string_hi\string_B1}{}
\markboth{\textcolor{darkblue}{\textbf{\ipa{dʑɯ˩kʰi˩}}}}{}
\textcolor{teal}{\mytextsc{nom}} \hspace{4pt} Ton~: L.
\textcolor{PineGreen}{\selectlanguage{french}Bord de l'eau.} \zh{水边。} 
\lhead{\firstmark}
\rhead{\botmark}

\subsection{\hspace{-0.5cm} {\Large \textcolor{darkblue}{\textbf{\ipa{dʑɯ˩kʰv̩˩}}}}\hspace{0.5cm}[\kern2pt{\textcolor{darkblue}{\textbf{\ipa{dʑɯ˩kʰv̩˩˥}}}}\kern2pt]} \hypertarget{dz£M\string_Bk\string_hv\string_=\string_B1}{}
\markboth{\textcolor{darkblue}{\textbf{\ipa{dʑɯ˩kʰv̩˩}}}}{}
\textcolor{teal}{\mytextsc{nom}} \hspace{4pt} Ton~: L.
\textcolor{PineGreen}{\selectlanguage{french}Mousse.} \zh{青苔。} 
\lhead{\firstmark}
\rhead{\botmark}

\subsection{\hspace{-0.5cm} {\Large \textcolor{darkblue}{\textbf{\ipa{dʑɯ˩nɑ˩hæ̃˩tʰɑ˩}}}}\hspace{0.5cm}[\kern2pt{\textcolor{darkblue}{\textbf{\ipa{dʑɯ˩nɑ˩hæ̃˩tʰɑ˩˥}}}}\kern2pt]} \hypertarget{dz£M\string_BnA\string_Bh\{\string_~\string_Bt\string_hA\string_B1}{}
\markboth{\textcolor{darkblue}{\textbf{\ipa{dʑɯ˩nɑ˩hæ̃˩tʰɑ˩}}}}{}
\textcolor{teal}{\mytextsc{nom}} \hspace{4pt} Ton~: L.
\textcolor{PineGreen}{\selectlanguage{french}Moulin à eau.} \zh{水磨。}  \zh{量词}: \textcolor{darkblue}{\textbf{\ipa{pɤ˩}}}  \mytextsc{clf}~: \textcolor{darkblue}{\textbf{\ipa{pɤ˩}}} 
\lhead{\firstmark}
\rhead{\botmark}

\subsection{\hspace{-0.5cm} {\Large \textcolor{darkblue}{\textbf{\ipa{dʑɯ˩nɑ˩mi˩}}}}\hspace{0.5cm}[\kern2pt{\textcolor{darkblue}{\textbf{\ipa{dʑɯ˩nɑ˩mi˩˥}}}}\kern2pt]} \hypertarget{dz£M\string_BnA\string_Bmi\string_B1}{}
\markboth{\textcolor{darkblue}{\textbf{\ipa{dʑɯ˩nɑ˩mi˩}}}}{}
\textcolor{teal}{\mytextsc{nom}} \hspace{4pt} Ton~: L.
\textcolor{PineGreen}{\selectlanguage{french}Forêt d'altitude, régions sauvages en altitude.} \zh{深山老林、高山上的地方。} \textit{Voir~:} \textcolor{darkblue}{\textbf{\ipa{dʑɯ˩nɑ˩mi˩-ʁo˩, dʑɯ˩ʁo˩}}} 
\lhead{\firstmark}
\rhead{\botmark}

\subsection{\hspace{-0.5cm} {\Large \textcolor{darkblue}{\textbf{\ipa{dʑɯ˩nɑ˩mi˩-ʁo˩}}}}\hspace{0.5cm}[\kern2pt{\textcolor{darkblue}{\textbf{\ipa{xxxx non-correspondance entre le nombre de morphèmes et le nombre de tons de morphèmes}}}}\kern2pt]} \hypertarget{dz£M\string_BnA\string_Bmi\string_B-Ro\string_B1}{}
\markboth{\textcolor{darkblue}{\textbf{\ipa{dʑɯ˩nɑ˩mi˩-ʁo˩}}}}{}
\textcolor{teal}{\mytextsc{nom}} \hspace{4pt} Ton~: L.
\textcolor{PineGreen}{\selectlanguage{french}Forêt d'altitude, régions sauvages en altitude.} \zh{深山老林、高山上的地方。} \textit{Voir~:} \textcolor{darkblue}{\textbf{\ipa{dʑɯ˩nɑ˩mi˩, dʑɯ˩ʁo˩}}} 
\lhead{\firstmark}
\rhead{\botmark}

\subsection{\hspace{-0.5cm} {\Large \textcolor{darkblue}{\textbf{\ipa{dʑɯ˩pɤ˩-kv̩˧hĩ˩}}}}\hspace{0.5cm}[\kern2pt{\textcolor{darkblue}{\textbf{\ipa{dʑɯ˩pɤ˩kv̩˧hĩ˩}}}}\kern2pt]} \hypertarget{dz£M\string_Bp7\string_B-kv\string_=\string_Mhi\string_~\string_B1}{}
\markboth{\textcolor{darkblue}{\textbf{\ipa{dʑɯ˩pɤ˩-kv̩˧hĩ˩}}}}{}
\textcolor{teal}{\mytextsc{nom}} \hspace{4pt} Ton~: L-L\#.
\textcolor{PineGreen}{\selectlanguage{french}Source.} \zh{水泉、山泉。}  ¶ \textcolor{darkblue}{\textbf{\ipa{dʑɯ˩pɤ˩-kv̩˧hĩ˩ | tʰi˧-di˩}}} \textcolor{PineGreen}{\selectlanguage{french}il y a une source} \zh{有水泉}  
\textit{Voir~:} \textcolor{darkblue}{\textbf{\ipa{dʑɯ˩pɤ˩qʰv̩˩, dʑɯ˩pɤ˩tv̩˩qʰv̩˥}}} 
\lhead{\firstmark}
\rhead{\botmark}

\subsection{\hspace{-0.5cm} {\Large \textcolor{darkblue}{\textbf{\ipa{dʑɯ˩pɤ˩qʰv̩˩}}}}\hspace{0.5cm}[\kern2pt{\textcolor{darkblue}{\textbf{\ipa{dʑɯ˩pɤ˩qʰv̩˩˥}}}}\kern2pt]} \hypertarget{dz£M\string_Bp7\string_Bq\string_hv\string_=\string_B1}{}
\markboth{\textcolor{darkblue}{\textbf{\ipa{dʑɯ˩pɤ˩qʰv̩˩}}}}{}
\textcolor{teal}{\mytextsc{nom}} \hspace{4pt} Ton~: L.
\textcolor{PineGreen}{\selectlanguage{french}Source.} \zh{水泉、山泉。} \textit{Voir~:} \textcolor{darkblue}{\textbf{\ipa{dʑɯ˩pɤ˩-kv̩˧hĩ˩, dʑɯ˩pɤ˩tv̩˩qʰv̩˥}}} 
\lhead{\firstmark}
\rhead{\botmark}

\subsection{\hspace{-0.5cm} {\Large \textcolor{darkblue}{\textbf{\ipa{dʑɯ˩pɤ˩tv̩˩qʰv̩˥}}}}\hspace{0.5cm}[\kern2pt{\textcolor{darkblue}{\textbf{\ipa{dʑɯ˩pɤ˩tv̩˩qʰv̩˥}}}}\kern2pt]} \hypertarget{dz£M\string_Bp7\string_Btv\string_=\string_Bq\string_hv\string_=\string_T1}{}
\markboth{\textcolor{darkblue}{\textbf{\ipa{dʑɯ˩pɤ˩tv̩˩qʰv̩˥}}}}{}
\textcolor{teal}{\mytextsc{nom}} \hspace{4pt} Ton~: L+H\#.
\textcolor{PineGreen}{\selectlanguage{french}Source.} \zh{水泉、山泉。} \textit{Voir~:} \textcolor{darkblue}{\textbf{\ipa{dʑɯ˩pɤ˩qʰv̩˩, dʑɯ˩pɤ˩-kv̩˧hĩ˩}}} 
\lhead{\firstmark}
\rhead{\botmark}

\subsection{\hspace{-0.5cm} {\Large \textcolor{darkblue}{\textbf{\ipa{dʑɯ˩pʰæ˩}}}}\hspace{0.5cm}[\kern2pt{\textcolor{darkblue}{\textbf{\ipa{dʑɯ˩pʰæ˩˥}}}}\kern2pt]} \hypertarget{dz£M\string_Bp\string_h\{\string_B1}{}
\markboth{\textcolor{darkblue}{\textbf{\ipa{dʑɯ˩pʰæ˩}}}}{}
\textcolor{teal}{\mytextsc{nom}} \hspace{4pt} Ton~: L.
\textcolor{PineGreen}{\selectlanguage{french}Glace.} \zh{冰。}  \zh{量词}: \textcolor{darkblue}{\textbf{\ipa{pʰæ˧˥}}}  \mytextsc{clf}~: \textcolor{darkblue}{\textbf{\ipa{pʰæ˧˥}}} 
\lhead{\firstmark}
\rhead{\botmark}

\subsection{\hspace{-0.5cm} {\Large \textcolor{darkblue}{\textbf{\ipa{dʑɯ˩qʰæ˩}}}}\hspace{0.5cm}[\kern2pt{\textcolor{darkblue}{\textbf{\ipa{dʑɯ˩qʰæ˩˥}}}}\kern2pt]} \hypertarget{dz£M\string_Bq\string_h\{\string_B1}{}
\markboth{\textcolor{darkblue}{\textbf{\ipa{dʑɯ˩qʰæ˩}}}}{}
\textcolor{teal}{\mytextsc{nom}} \hspace{4pt} Ton~: L.
\textcolor{PineGreen}{\selectlanguage{french}Eau froide.} \zh{凉水。} 
\lhead{\firstmark}
\rhead{\botmark}

\subsection{\hspace{-0.5cm} {\Large \textcolor{darkblue}{\textbf{\ipa{dʑɯ˩qʰwɤ˩-lv̩˩}}}}\hspace{0.5cm}[\kern2pt{\textcolor{darkblue}{\textbf{\ipa{xxxx non-correspondance entre le nombre de morphèmes et le nombre de tons de morphèmes}}}}\kern2pt]} \hypertarget{dz£M\string_Bq\string_hw7\string_B-lv\string_=\string_B1}{}
\markboth{\textcolor{darkblue}{\textbf{\ipa{dʑɯ˩qʰwɤ˩-lv̩˩}}}}{}
\textcolor{teal}{\mytextsc{nom}} \hspace{4pt} Ton~: L.
\textcolor{PineGreen}{\selectlanguage{french}Marais (terre impropre à la culture).} \zh{沼泽、湿地。} Dialecte chinois local~:\zh{潮地。} \zh{量词}: \textcolor{darkblue}{\textbf{\ipa{kɤ˧˥}}}  \mytextsc{clf}~: \textcolor{darkblue}{\textbf{\ipa{kɤ˧˥}}} 
\lhead{\firstmark}
\rhead{\botmark}

\subsection{\hspace{-0.5cm} {\Large \textcolor{darkblue}{\textbf{\ipa{dʑɯ˩ʁo˩}}}}\hspace{0.5cm}[\kern2pt{\textcolor{darkblue}{\textbf{\ipa{dʑɯ˩ʁo˩˥}}}}\kern2pt]} \hypertarget{dz£M\string_BRo\string_B1}{}
\markboth{\textcolor{darkblue}{\textbf{\ipa{dʑɯ˩ʁo˩}}}}{}
\textcolor{teal}{\mytextsc{nom}} \hspace{4pt} Ton~: L.
\textcolor{PineGreen}{\selectlanguage{french}Forêt d'altitude, régions sauvages en altitude.} \zh{深山老林、高山上的地方。} \textit{Voir~:} \textcolor{darkblue}{\textbf{\ipa{dʑɯ˩nɑ˩mi˩, dʑɯ˩nɑ˩mi˩-ʁo˩}}} 
\lhead{\firstmark}
\rhead{\botmark}

\subsection{\hspace{-0.5cm} {\Large \textcolor{darkblue}{\textbf{\ipa{dʑɯ˩ʁo˩-æ̃˧}}}}\hspace{0.5cm}[\kern2pt{\textcolor{darkblue}{\textbf{\ipa{dʑɯ˩ʁo˩æ̃˥}}}}\kern2pt]} \hypertarget{dz£M\string_BRo\string_B-\{\string_~\string_M1}{}
\markboth{\textcolor{darkblue}{\textbf{\ipa{dʑɯ˩ʁo˩-æ̃˧}}}}{}
\textcolor{teal}{\mytextsc{nom}} \hspace{4pt} Ton~: L-M.
\textcolor{PineGreen}{\selectlanguage{french}Caille, \textit{Coturnix}; terme employé pour divers oiseaux, dont des râles (\textit{Crex}).} \zh{鹌鹑。}  \zh{量词}: \textcolor{darkblue}{\textbf{\ipa{mi˩}}}  \mytextsc{clf}~: \textcolor{darkblue}{\textbf{\ipa{mi˩}}} 
\lhead{\firstmark}
\rhead{\botmark}

\subsection{\hspace{-0.5cm} {\Large \textcolor{darkblue}{\textbf{\ipa{dʑɯ˩ʁo˩-bo˧}}}}\hspace{0.5cm}[\kern2pt{\textcolor{darkblue}{\textbf{\ipa{dʑɯ˩ʁo˩bo˥}}}}\kern2pt]} \hypertarget{dz£M\string_BRo\string_B-bo\string_M1}{}
\markboth{\textcolor{darkblue}{\textbf{\ipa{dʑɯ˩ʁo˩-bo˧}}}}{}
\textcolor{teal}{\mytextsc{nom}} \hspace{4pt} Ton~: L-M.
\textcolor{PineGreen}{\selectlanguage{french}Sanglier, porc sauvage.} \zh{野猪。}  \zh{量词}: \textcolor{darkblue}{\textbf{\ipa{mi˩}}}  \mytextsc{clf}~: \textcolor{darkblue}{\textbf{\ipa{mi˩}}} \textit{Syn~:} \hyperlink{}{\textcolor{darkblue}{\textbf{\ipa{bo˩tv̩\#˥}}}}. 
\lhead{\firstmark}
\rhead{\botmark}

\subsection{\hspace{-0.5cm} {\Large \textcolor{darkblue}{\textbf{\ipa{dʑɯ˩ʁo˩-dze˧}}}}\hspace{0.5cm}[\kern2pt{\textcolor{darkblue}{\textbf{\ipa{dʑɯ˩ʁo˩dze˥}}}}\kern2pt]} \hypertarget{dz£M\string_BRo\string_B-dze\string_M1}{}
\markboth{\textcolor{darkblue}{\textbf{\ipa{dʑɯ˩ʁo˩-dze˧}}}}{}
\textcolor{teal}{\mytextsc{nom}} \hspace{4pt} Ton~: L-M.
\textcolor{PineGreen}{\selectlanguage{french}Xanthoxyle sauvage.} \zh{野花椒。} 
\lhead{\firstmark}
\rhead{\botmark}

\subsection{\hspace{-0.5cm} {\Large \textcolor{darkblue}{\textbf{\ipa{dʑɯ˩ʁo˩-hwɤ˩li˧}}}}\hspace{0.5cm}[\kern2pt{\textcolor{darkblue}{\textbf{\ipa{dʑɯ˩ʁo˩hwɤ˩li˥}}}}\kern2pt]} \hypertarget{dz£M\string_BRo\string_B-hw7\string_Bli\string_M1}{}
\markboth{\textcolor{darkblue}{\textbf{\ipa{dʑɯ˩ʁo˩-hwɤ˩li˧}}}}{}
\textcolor{teal}{\mytextsc{nom}} \hspace{4pt} Ton~: L-LM.
\textcolor{PineGreen}{\selectlanguage{french}Chat sauvage, \textit{Felis temincki}.} \zh{野猫。}  \zh{量词}: \textcolor{darkblue}{\textbf{\ipa{mi˩}}}  \mytextsc{clf}~: \textcolor{darkblue}{\textbf{\ipa{mi˩}}} 
\lhead{\firstmark}
\rhead{\botmark}

\subsection{\hspace{-0.5cm} {\Large \textcolor{darkblue}{\textbf{\ipa{dʑɯ˩ʁo˩-ɬi˩bi˧}}}}\hspace{0.5cm}[\kern2pt{\textcolor{darkblue}{\textbf{\ipa{dʑɯ˩ʁo˩ɬi˩bi˥}}}}\kern2pt]} \hypertarget{dz£M\string_BRo\string_B-Ki\string_Bbi\string_M1}{}
\markboth{\textcolor{darkblue}{\textbf{\ipa{dʑɯ˩ʁo˩-ɬi˩bi˧}}}}{}
\textcolor{teal}{\mytextsc{nom}} \hspace{4pt} Ton~: L-LM.
\textcolor{PineGreen}{\selectlanguage{french}Igname de Chine (shan-yao).} \zh{山药。}  \zh{量词}: \textcolor{darkblue}{\textbf{\ipa{ɭɯ˧}}}  \mytextsc{clf}~: \textcolor{darkblue}{\textbf{\ipa{ɭɯ˧}}} 
\lhead{\firstmark}
\rhead{\botmark}

\subsection{\hspace{-0.5cm} {\Large \textcolor{darkblue}{\textbf{\ipa{dʑɯ˩ʁo˩-zɯ˩}}}}\hspace{0.5cm}[\kern2pt{\textcolor{darkblue}{\textbf{\ipa{xxxx non-correspondance entre le nombre de morphèmes et le nombre de tons de morphèmes}}}}\kern2pt]} \hypertarget{dz£M\string_BRo\string_B-zM\string_B1}{}
\markboth{\textcolor{darkblue}{\textbf{\ipa{dʑɯ˩ʁo˩-zɯ˩}}}}{}
\textcolor{teal}{\mytextsc{nom}} \hspace{4pt} Ton~: L.
\textcolor{PineGreen}{\selectlanguage{french}Herbes de la montagne, herbes sauvages, foin poussant sur l'alpage.} \zh{野草。}  ¶ \textcolor{darkblue}{\textbf{\ipa{ʈʂʰɯ˧ | dʑɯ˩ʁo˩-zɯ˩ ɲi˥.}}} \textcolor{PineGreen}{\selectlanguage{french}\mytextsc{dem} \string_ \mytextsc{cop}} \zh{\mytextsc{指示代词} \string_ \mytextsc{系词}}  
 \zh{量词}: \textcolor{darkblue}{\textbf{\ipa{qɑ˩}}} \textcolor{darkblue}{\textbf{\ipa{po˧}}}  \mytextsc{clf}~: \textcolor{darkblue}{\textbf{\ipa{qɑ˩}}} \textcolor{darkblue}{\textbf{\ipa{po˧}}} \textcolor{PineGreen}{\selectlanguage{french}botte; unité}
\lhead{\firstmark}
\rhead{\botmark}

\subsection{\hspace{-0.5cm} {\Large \textcolor{darkblue}{\textbf{\ipa{dʑɯ˩si˩}}}}\hspace{0.5cm}[\kern2pt{\textcolor{darkblue}{\textbf{\ipa{dʑɯ˩si˩˥}}}}\kern2pt]} \hypertarget{dz£M\string_Bsi\string_B1}{}
\markboth{\textcolor{darkblue}{\textbf{\ipa{dʑɯ˩si˩}}}}{}
\textcolor{teal}{\mytextsc{nom}} \hspace{4pt} Ton~: L.
\textcolor{PineGreen}{\selectlanguage{french}Chêne blanc oriental.} \zh{青冈树、槲栎。} \textit{Syn~:} \hyperlink{}{\textcolor{darkblue}{\textbf{\ipa{dzi˧dzi˧}}}}. 
\lhead{\firstmark}
\rhead{\botmark}

\subsection{\hspace{-0.5cm} {\Large \textcolor{darkblue}{\textbf{\ipa{dʑɯ˩so˩}}}}\hspace{0.5cm}[\kern2pt{\textcolor{darkblue}{\textbf{\ipa{dʑɯ˩so˩˥}}}}\kern2pt]} \hypertarget{dz£M\string_Bso\string_B1}{}
\markboth{\textcolor{darkblue}{\textbf{\ipa{dʑɯ˩so˩}}}}{}
\textcolor{teal}{\mytextsc{nom}} \hspace{4pt} Ton~: L.
\textcolor{PineGreen}{\selectlanguage{french}Vague.} \zh{波浪。}  ¶ \textcolor{darkblue}{\textbf{\ipa{dʑɯ˩so˩ pʰv̩˩˥}}} \textcolor{PineGreen}{\selectlanguage{french}il y a une vague, une vague déferle} \zh{有波浪}  
 \zh{量词}: \textcolor{darkblue}{\textbf{\ipa{pʰæ˧˥}}}  \mytextsc{clf}~: \textcolor{darkblue}{\textbf{\ipa{pʰæ˧˥}}} 
\lhead{\firstmark}
\rhead{\botmark}

\subsection{\hspace{-0.5cm} {\Large \textcolor{darkblue}{\textbf{\ipa{dʑɯ˩ʂo˥}}}}\hspace{0.5cm}[\kern2pt{\textcolor{darkblue}{\textbf{\ipa{dʑɯ˩ʂo˥}}}}\kern2pt]} \hypertarget{dz£M\string_Bs`o\string_T1}{}
\markboth{\textcolor{darkblue}{\textbf{\ipa{dʑɯ˩ʂo˥}}}}{}
\textcolor{teal}{\mytextsc{nom}} \hspace{4pt} Ton~: L+H\#.
\textcolor{PineGreen}{\selectlanguage{french}Nom d'un rituel.} \zh{一项仪式。}  ¶ \textcolor{darkblue}{\textbf{\ipa{dʑɯ˩ʂo˥-tsɑ˩bɤ˩}}} \textcolor{PineGreen}{\selectlanguage{french}farine (tsamba) pouvant servir aux cérémonies; elle ne doit pas contenir d'avoine. A la fin de la cérémonie, on la jette} \zh{做仪式时所使用的面粉。这种面粉里不要含有燕麦。仪式结束后,面粉被扔掉。}  

\lhead{\firstmark}
\rhead{\botmark}

\subsection{\hspace{-0.5cm} {\Large \textcolor{darkblue}{\textbf{\ipa{dʑɯ˩ʂwæ˩}}}}\hspace{0.5cm}[\kern2pt{\textcolor{darkblue}{\textbf{\ipa{dʑɯ˩ʂwæ˩˥}}}}\kern2pt]} \hypertarget{dz£M\string_Bs`w\{\string_B1}{}
\markboth{\textcolor{darkblue}{\textbf{\ipa{dʑɯ˩ʂwæ˩}}}}{}
\textcolor{teal}{\mytextsc{nom}} \hspace{4pt} Ton~: L.
\textcolor{PineGreen}{\selectlanguage{french}\textit{Lysimachia christinae Hance}.} \zh{过路黄。}  ¶ \textcolor{darkblue}{\textbf{\ipa{dʑɯ˩ʂwæ˩-bæ˥bæ˩}}} \textcolor{PineGreen}{\selectlanguage{french}fleur de yyyyy} \zh{过路黄花}  

\lhead{\firstmark}
\rhead{\botmark}

\subsection{\hspace{-0.5cm} {\Large \textcolor{darkblue}{\textbf{\ipa{dʑɯ˩tɤ˩ɻ̍˥}}}}\hspace{0.5cm}[\kern2pt{\textcolor{darkblue}{\textbf{\ipa{dʑɯ˩tɤ˩ɻ̍˥}}}}\kern2pt]} \hypertarget{dz£M\string_Bt7\string_Br£`̍\string_T1}{}
\markboth{\textcolor{darkblue}{\textbf{\ipa{dʑɯ˩tɤ˩ɻ̍˥}}}}{}
\textcolor{teal}{\mytextsc{adjectif}} \hspace{4pt} Ton~: L+H\#.
\textcolor{PineGreen}{\selectlanguage{french}Mouillé, humide.} \zh{湿。}  ¶ \textcolor{darkblue}{\textbf{\ipa{dʑɯ˩tɤ˩ɻ̍˥ gv̩˩-ze˩}}} \textcolor{PineGreen}{\selectlanguage{french}ça s'est mouillé} \zh{湿了!}  

\lhead{\firstmark}
\rhead{\botmark}

\subsection{\hspace{-0.5cm} {\Large \textcolor{darkblue}{\textbf{\ipa{dʑɯ˩tɕʰɯ˩lɑ˩qʰɑ˥}}}}\hspace{0.5cm}[\kern2pt{\textcolor{darkblue}{\textbf{\ipa{dʑɯ˩tɕʰɯ˩lɑ˩qʰɑ˥}}}}\kern2pt]} \hypertarget{dz£M\string_Bts£\string_hM\string_BlA\string_Bq\string_hA\string_T1}{}
\markboth{\textcolor{darkblue}{\textbf{\ipa{dʑɯ˩tɕʰɯ˩lɑ˩qʰɑ˥}}}}{}
\textcolor{teal}{\mytextsc{nom}} \hspace{4pt} Ton~: L+H\#.
\textcolor{PineGreen}{\selectlanguage{french}Sorte de prunelle très acide, qui pousse à l'état sauvage; utilisée en décoction, en association avec du gingembre et de petites pommes séchées, contre les maux de gorge.} \zh{一种梅子。} 
\lhead{\firstmark}
\rhead{\botmark}

\subsection{\hspace{-0.5cm} {\Large \textcolor{darkblue}{\textbf{\ipa{dʑɯ˩tɕʰɯ˩lɑ˩qʰæ˥}}}}\hspace{0.5cm}[\kern2pt{\textcolor{darkblue}{\textbf{\ipa{dʑɯ˩tɕʰɯ˩lɑ˩qʰæ˥}}}}\kern2pt]} \hypertarget{dz£M\string_Bts£\string_hM\string_BlA\string_Bq\string_h\{\string_T1}{}
\markboth{\textcolor{darkblue}{\textbf{\ipa{dʑɯ˩tɕʰɯ˩lɑ˩qʰæ˥}}}}{}
\textcolor{teal}{\mytextsc{nom}} \hspace{4pt} Ton~: L+H\#.
\textcolor{PineGreen}{\selectlanguage{french}Argousier, \textit{Hippophae rhamnoides Linn.}.} \zh{沙棘。} 
\lhead{\firstmark}
\rhead{\botmark}

\subsection{\hspace{-0.5cm} {\Large \textcolor{darkblue}{\textbf{\ipa{dʑɯ˩tsʰi˩}}}}\hspace{0.5cm}[\kern2pt{\textcolor{darkblue}{\textbf{\ipa{dʑɯ˩tsʰi˩˥}}}}\kern2pt]} \hypertarget{dz£M\string_Bts\string_hi\string_B1}{}
\markboth{\textcolor{darkblue}{\textbf{\ipa{dʑɯ˩tsʰi˩}}}}{}
\textcolor{teal}{\mytextsc{nom}} \hspace{4pt} Ton~: L.
\textcolor{PineGreen}{\selectlanguage{french}Eau bouillante, eau chaude.} \zh{开水,热水。} 
\lhead{\firstmark}
\rhead{\botmark}

\subsection{\hspace{-0.5cm} {\Large \textcolor{darkblue}{\textbf{\ipa{dʑɯ˩tsʰi˩ʈʰɯ˩di˩}}}}\hspace{0.5cm}[\kern2pt{\textcolor{darkblue}{\textbf{\ipa{dʑɯ˩tsʰi˩ʈʰɯ˩di˩˥}}}}\kern2pt]} \hypertarget{dz£M\string_Bts\string_hi\string_Bt`\string_hM\string_Bdi\string_B1}{}
\markboth{\textcolor{darkblue}{\textbf{\ipa{dʑɯ˩tsʰi˩ʈʰɯ˩di˩}}}}{}
\textcolor{teal}{\mytextsc{nom}} \hspace{4pt} Ton~: L.
\textcolor{PineGreen}{\selectlanguage{french}Petite gourde thermos individuelle.} \zh{口杯。}  \zh{量词}: \textcolor{darkblue}{\textbf{\ipa{ɭɯ˧}}}  \mytextsc{clf}~: \textcolor{darkblue}{\textbf{\ipa{ɭɯ˧}}} 
\lhead{\firstmark}
\rhead{\botmark}

\subsection{\hspace{-0.5cm} {\Large \textcolor{darkblue}{\textbf{\ipa{dʑɯ˩ʈv̩˧}}}}\hspace{0.5cm}[\kern2pt{\textcolor{darkblue}{\textbf{\ipa{dʑɯ˩ʈv̩˥}}}}\kern2pt]} \hypertarget{dz£M\string_Bt`v\string_=\string_M1}{}
\markboth{\textcolor{darkblue}{\textbf{\ipa{dʑɯ˩ʈv̩˧}}}}{}
\textcolor{teal}{\mytextsc{adjectif}} \hspace{4pt} Ton~: LM.
\textcolor{PineGreen}{\selectlanguage{french}Être gravement voûté, avoir une bosse.} \zh{驼背(严重的病)。}  ¶ \textcolor{darkblue}{\textbf{\ipa{dʑɯ˩ʈv̩˧-ze˩}}} \textcolor{PineGreen}{\selectlanguage{french}\mytextsc{pfv}} \zh{驼背了}  

\lhead{\firstmark}
\rhead{\botmark}

\subsection{\hspace{-0.5cm} {\Large \textcolor{darkblue}{\textbf{\ipa{dʑɯ˩zo˩}}}}\hspace{0.5cm}[\kern2pt{\textcolor{darkblue}{\textbf{\ipa{dʑɯ˩zo˩˥}}}}\kern2pt]} \hypertarget{dz£M\string_Bzo\string_B1}{}
\markboth{\textcolor{darkblue}{\textbf{\ipa{dʑɯ˩zo˩}}}}{}
\textcolor{teal}{\mytextsc{nom}} \hspace{4pt} Ton~: L.
\textcolor{PineGreen}{\selectlanguage{french}Ruisseau.} \zh{溪流。}  \zh{量词}: \textcolor{darkblue}{\textbf{\ipa{kʰɯ˩}}}  \mytextsc{clf}~: \textcolor{darkblue}{\textbf{\ipa{kʰɯ˩}}} 
\lhead{\firstmark}
\rhead{\botmark}

\subsection{\hspace{-0.5cm} {\Large \textcolor{darkblue}{\textbf{\ipa{dʑɯ˩ʐv̩˩}}}}\hspace{0.5cm}[\kern2pt{\textcolor{darkblue}{\textbf{\ipa{dʑɯ˩ʐv̩˩˥}}}}\kern2pt]} \hypertarget{dz£M\string_Bz`v\string_=\string_B1}{}
\markboth{\textcolor{darkblue}{\textbf{\ipa{dʑɯ˩ʐv̩˩}}}}{}
\textcolor{teal}{\mytextsc{verbe}} \hspace{4pt} Ton~: L.
\textcolor{PineGreen}{\selectlanguage{french}Nager.} \zh{游泳。} 
\lhead{\firstmark}
\rhead{\botmark}

\subsection{\hspace{-0.5cm} {\Large \textcolor{darkblue}{\textbf{\ipa{dʑɯ˧˥}}}}\hspace{0.5cm}[\kern2pt{\textcolor{darkblue}{\textbf{\ipa{dʑɯ˥}}}}\kern2pt]} \hypertarget{dz£M\string_M\string_T1}{}
\markboth{\textcolor{darkblue}{\textbf{\ipa{dʑɯ˧˥}}}}{}
\textcolor{teal}{\mytextsc{adjectif}} \hspace{4pt} Ton~: MH.
\textcolor{PineGreen}{\selectlanguage{french}Nombreux, beaucoup (dénombrable).} \zh{多。}  ¶ \textcolor{darkblue}{\textbf{\ipa{hĩ˧ dʑɯ˩}}} \textcolor{PineGreen}{\selectlanguage{french}les gens sont nombreux, il y a beaucoup de monde} \zh{人多。}  

\lhead{\firstmark}
\rhead{\botmark}

\newpage
\section*{\centering- \textcolor{darkblue}{\textbf{\ipa{ɖ}}} -}
\subsection{\hspace{-0.5cm} {\Large \textcolor{darkblue}{\textbf{\ipa{ɖæ˥}}}}\hspace{0.5cm}[\kern2pt{\textcolor{darkblue}{\textbf{\ipa{ɖæ˥}}}}\kern2pt]} \hypertarget{d`\{\string_T1}{}
\markboth{\textcolor{darkblue}{\textbf{\ipa{ɖæ˥}}}}{}
\textcolor{teal}{\mytextsc{adjectif}} \hspace{4pt} Ton~: H.
\textcolor{PineGreen}{\selectlanguage{french}Court.} \zh{短。} 
\lhead{\firstmark}
\rhead{\botmark}

\subsection{\hspace{-0.5cm} {\Large \textcolor{darkblue}{\textbf{\ipa{ɖæ˧\textasciitilde{}ɖæ˩}}}}\hspace{0.5cm}[\kern2pt{\textcolor{darkblue}{\textbf{\ipa{ɖæ˧ɖæ˩}}}}\kern2pt]} \hypertarget{d`\{\string_M~d`\{\string_B1}{}
\markboth{\textcolor{darkblue}{\textbf{\ipa{ɖæ˧\textasciitilde{}ɖæ˩}}}}{}
\textcolor{teal}{\mytextsc{adjectif}} \hspace{4pt} Ton~: L\#.
\textcolor{PineGreen}{\selectlanguage{french}Horizontal.} \zh{横着(横躺在路上)。}  ¶ \textcolor{darkblue}{\textbf{\ipa{ɖæ˧\textasciitilde{}ɖæ˩ | tʰi˧-tɕɯ˥}}} \textcolor{PineGreen}{\selectlanguage{french}poser à plat} \zh{横着放}  

\lhead{\firstmark}
\rhead{\botmark}

\subsection{\hspace{-0.5cm} {\Large \textcolor{darkblue}{\textbf{\ipa{ɖæ˩\textsubscript{a}}}}}\hspace{0.5cm}[\kern2pt{\textcolor{darkblue}{\textbf{\ipa{ɖæ˩˥}}}}\kern2pt]} \hypertarget{d`\{\string_Ba1}{}
\markboth{\textcolor{darkblue}{\textbf{\ipa{ɖæ˩\textsubscript{a}}}}}{}
\textcolor{teal}{\mytextsc{classificateur}} \hspace{4pt} Ton~: L\textsubscript{a}.
\textcolor{PineGreen}{\selectlanguage{french}Classificateur des sections, pour objets pouvant être divisés dans le sens de la longueur. Pour une route, cette unité correspond à 1/2 journée de marche; pour du tissu, elle correspond à une pièce.} \zh{量词:路(段)/布(匹)。}  ¶ \textcolor{darkblue}{\textbf{\ipa{ʐɤ˩mi˩˥ | ɖɯ˧-ɖæ˩}}} \textcolor{PineGreen}{\selectlanguage{french}un bout de chemin, un bout de route} \zh{一段路}  
 ¶ \textcolor{darkblue}{\textbf{\ipa{ɲi˧, ɲi˩-ɖæ˩! |}}} \textcolor{PineGreen}{\selectlanguage{french}formule figée traditionnel: un jour, ça fait deux étapes! (si on peut parvenir quelque part avant le déjeuner, la distance est considérée comme une seule étape; si on y parvient dans l'après-midi, on compte deux étapes)} \zh{一天两段路!(说明:早上出发,如果午饭前能到目的地,距离算是一段,如果下午晚上才到,算两段。)}  
 ¶ \textcolor{darkblue}{\textbf{\ipa{ʈʂʰɯ˧-ɖæ˥}}} \textcolor{PineGreen}{\selectlanguage{french}\mytextsc{dem} \string_ (ton: H\# / H\$)} \zh{\mytextsc{指示代词} \string_}  

\lhead{\firstmark}
\rhead{\botmark}

\subsection{\hspace{-0.5cm} {\Large \textcolor{darkblue}{\textbf{\ipa{ɖæ˩\textsubscript{a}}}}}\hspace{0.5cm}[\kern2pt{\textcolor{darkblue}{\textbf{\ipa{ɖæ˩˥}}}}\kern2pt]} \hypertarget{d`\{\string_Ba1}{}
\markboth{\textcolor{darkblue}{\textbf{\ipa{ɖæ˩\textsubscript{a}}}}}{}
\textcolor{teal}{\mytextsc{verbe}} \hspace{4pt} Ton~: L\textsubscript{a}.
\textcolor{PineGreen}{\selectlanguage{french}Passer, traverser (une rivière en bateau, une montagne…).} \zh{渡(坐船渡河……)。}  ¶ \textcolor{darkblue}{\textbf{\ipa{dʑɯ˩ ɖæ˩˥ / dʑɯ˩ ɖæ˩-ze˥}}} \textcolor{PineGreen}{\selectlanguage{french}passer une rivière} \zh{渡河}  
 ¶ \textcolor{darkblue}{\textbf{\ipa{dʑɯ˧ | ɖɯ˧-kʰɯ˩ ɖæ˩}}} \textcolor{PineGreen}{\selectlanguage{french}idem} \zh{同上}  

\lhead{\firstmark}
\rhead{\botmark}

\subsection{\hspace{-0.5cm} {\Large \textcolor{darkblue}{\textbf{\ipa{ɖæ˩dʑɯ˥}}}}\hspace{0.5cm}[\kern2pt{\textcolor{darkblue}{\textbf{\ipa{ɖæ˩dʑɯ˥}}}}\kern2pt]} \hypertarget{d`\{\string_Bdz£M\string_T1}{}
\markboth{\textcolor{darkblue}{\textbf{\ipa{ɖæ˩dʑɯ˥}}}}{}
\textcolor{teal}{\mytextsc{nom}} \hspace{4pt} Ton~: LH.
\textcolor{PineGreen}{\selectlanguage{french}Détritus, saletés, crasse.} \zh{污垢。}  \zh{量词}: \textcolor{darkblue}{\textbf{\ipa{ʁwɤ˧, etc}}}  \mytextsc{clf}~: \textcolor{darkblue}{\textbf{\ipa{ʁwɤ˧, etc}}} 
\lhead{\firstmark}
\rhead{\botmark}

\subsection{\hspace{-0.5cm} {\Large \textcolor{darkblue}{\textbf{\ipa{ɖæ˩-lɑ˧so˧}}}}\hspace{0.5cm}[\kern2pt{\textcolor{darkblue}{\textbf{\ipa{ɖæ˧lɑ˧so˧}}}}\kern2pt]} \hypertarget{d`\{\string_B-lA\string_Mso\string_M1}{}
\markboth{\textcolor{darkblue}{\textbf{\ipa{ɖæ˩-lɑ˧so˧}}}}{}
\textcolor{teal}{\mytextsc{nom}} \hspace{4pt} Ton~: L-.
\textcolor{PineGreen}{\selectlanguage{french}Nom de cérémonie que les moines (un ou deux) pratiquent une fois par an (pendant la première quinzaine de l'année) au domicile de la personne qui les y invite: offrande de céréales (ou de fruits) aux divinités. L'objet de la cérémonie est d'assurer la prospérité financière de la maisonnée.} \zh{一种祈福仪式,和尚在过年时主持行礼。}  Emprunt~: tibétain  'bras lha gsol

\lhead{\firstmark}
\rhead{\botmark}

\subsection{\hspace{-0.5cm} {\Large \textcolor{darkblue}{\textbf{\ipa{ɖæ˩mi˧}}}}\hspace{0.5cm}[\kern2pt{\textcolor{darkblue}{\textbf{\ipa{ɖæ˩mi˥}}}}\kern2pt]} \hypertarget{d`\{\string_Bmi\string_M1}{}
\markboth{\textcolor{darkblue}{\textbf{\ipa{ɖæ˩mi˧}}}}{}
\textcolor{teal}{\mytextsc{nom}} \hspace{4pt} Ton~: LM.
\textcolor{PineGreen}{\selectlanguage{french}Nom du temple de Yongning.} \zh{永宁大寺。}  Emprunt~: tibétain  dgra med
 ¶ \textcolor{darkblue}{\textbf{\ipa{ɖæ˩mi˧-ʈæ˩bɤ˩}}} \textcolor{PineGreen}{\selectlanguage{french}un prêtre du monastère} \zh{永宁大寺的和尚}  

\lhead{\firstmark}
\rhead{\botmark}

\subsection{\hspace{-0.5cm} {\Large \textcolor{darkblue}{\textbf{\ipa{ɖæ˩mi˧-go˧bɤ˩}}}}\hspace{0.5cm}[\kern2pt{\textcolor{darkblue}{\textbf{\ipa{ɖæ˩mi˧go˧bɤ˩}}}}\kern2pt]} \hypertarget{d`\{\string_Bmi\string_M-go\string_Mb7\string_B1}{}
\markboth{\textcolor{darkblue}{\textbf{\ipa{ɖæ˩mi˧-go˧bɤ˩}}}}{}
\textcolor{teal}{\mytextsc{nom}} \hspace{4pt} Ton~: LM-L\#.
\textcolor{PineGreen}{\selectlanguage{french}Le temple de Yongning.} \zh{永宁大寺。}  Emprunt~: tibétain  dgra med dgon pa

\lhead{\firstmark}
\rhead{\botmark}

\subsection{\hspace{-0.5cm} {\Large \textcolor{darkblue}{\textbf{\ipa{ɖæ˩pʰv̩˥}}}}\hspace{0.5cm}[\kern2pt{\textcolor{darkblue}{\textbf{\ipa{ɖæ˩pʰv̩˥}}}}\kern2pt]} \hypertarget{d`\{\string_Bp\string_hv\string_=\string_T1}{}
\markboth{\textcolor{darkblue}{\textbf{\ipa{ɖæ˩pʰv̩˥}}}}{}
\textcolor{teal}{\mytextsc{nom}} \hspace{4pt} Ton~: LH.
\textcolor{PineGreen}{\selectlanguage{french}Poussière.} \zh{灰尘。}  \zh{量词}: \textcolor{darkblue}{\textbf{\ipa{ti˧˥}}}  \mytextsc{clf}~: \textcolor{darkblue}{\textbf{\ipa{ti˧˥}}} 
\lhead{\firstmark}
\rhead{\botmark}

\subsection{\hspace{-0.5cm} {\Large \textcolor{darkblue}{\textbf{\ipa{ɖæ˩ʂɯ\#˥}}}}\hspace{0.5cm}[\kern2pt{\textcolor{darkblue}{\textbf{\ipa{ɖæ˩ʂɯ˥}}}}\kern2pt]} \hypertarget{d`\{\string_Bs`M\#\string_T1}{}
\markboth{\textcolor{darkblue}{\textbf{\ipa{ɖæ˩ʂɯ\#˥}}}}{}
\textcolor{teal}{\mytextsc{nom}} \hspace{4pt} Ton~: LM+\#H.
\textcolor{PineGreen}{\selectlanguage{french}Nom de village.} \zh{扎实(永宁的一个村落)。}  ¶ \textcolor{darkblue}{\textbf{\ipa{ɖæ˩ʂɯ˧-ʁwɤ\#˥}}} \textcolor{PineGreen}{\selectlanguage{french}même sens} \zh{同上:扎实村}  
 ¶ \textcolor{darkblue}{\textbf{\ipa{ɖæ˩ʂɯ\#˥, | ʈʂo˧ʂɯ\#˥, | bɤ˩tɕʰɯ˩˥, | dɑ˧pʰo˥, | bɤ˧dzi˩, | dze˧bo˧}}} \textcolor{PineGreen}{\selectlanguage{french}les six villages de la plaine de Yongning, dans l'ordre, qui prend comme point d'origine le village le plus proche du Lac} \zh{永宁坝的六个村落,按传统排序:从距离泸沽湖最近的村落说起。}  

\lhead{\firstmark}
\rhead{\botmark}

\subsection{\hspace{-0.5cm} {\Large \textcolor{darkblue}{\textbf{\ipa{ɖæ˩˧}}}}\hspace{0.5cm}[\kern2pt{\textcolor{darkblue}{\textbf{\ipa{ɖæ˩˥}}}}\kern2pt]} \hypertarget{d`\{\string_B\string_M1}{}
\markboth{\textcolor{darkblue}{\textbf{\ipa{ɖæ˩˧}}}}{}
\textcolor{teal}{\mytextsc{nom}} \hspace{4pt} Ton~: LM.
\ding{202} \textcolor{PineGreen}{\selectlanguage{french}Poussière.} \zh{尘土。}  ¶ \textcolor{darkblue}{\textbf{\ipa{ɖæ˩˥ | ɖɯ˧-ti˧ tʰi˧-di˥}}} \textcolor{PineGreen}{\selectlanguage{french}il y a une couche de poussière} \zh{有一层灰}  
 ¶ \textcolor{darkblue}{\textbf{\ipa{ɖæ˩ bæ˧}}} \textcolor{PineGreen}{\selectlanguage{french}balayer la poussière} \zh{扫灰}  
 \zh{量词}: \textcolor{darkblue}{\textbf{\ipa{ti˧˥}}} \ding{203} \textcolor{PineGreen}{\selectlanguage{french}Saletés.} \zh{污垢。}  \mytextsc{clf}~: \textcolor{darkblue}{\textbf{\ipa{ti˧˥}}} \textcolor{PineGreen}{\selectlanguage{french}couche}
\lhead{\firstmark}
\rhead{\botmark}

\subsection{\hspace{-0.5cm} {\Large \textcolor{darkblue}{\textbf{\ipa{ɖɤ˥}}}}\hspace{0.5cm}[\kern2pt{\textcolor{darkblue}{\textbf{\ipa{ɖɤ˥}}}}\kern2pt]} \hypertarget{d`7\string_T1}{}
\markboth{\textcolor{darkblue}{\textbf{\ipa{ɖɤ˥}}}}{}
\textcolor{teal}{\mytextsc{verbe}} \hspace{4pt} Ton~: H.
\textcolor{PineGreen}{\selectlanguage{french}Ramper.} \zh{爬行,匍匐。}  ¶ \textcolor{darkblue}{\textbf{\ipa{ɖɤ˧\textasciitilde{}ɖɤ˧ (-ze˩)}}} \textcolor{PineGreen}{\selectlanguage{french}\mytextsc{red}} \zh{\mytextsc{重叠:爬一爬}}  
 ¶ \textcolor{darkblue}{\textbf{\ipa{ʈʂʰɯ˧ | ɖɤ˧\textasciitilde{}ɖɤ˧-ʁo˧-ze˩!}}} \textcolor{PineGreen}{\selectlanguage{french}Elle sait ramper! (Au sujet d'un bébé qui se déplace en rampant.)} \zh{她会爬了!}  

\lhead{\firstmark}
\rhead{\botmark}

\subsection{\hspace{-0.5cm} {\Large \textcolor{darkblue}{\textbf{\ipa{ɖɤ˧mi˧}}}}\hspace{0.5cm}[\kern2pt{\textcolor{darkblue}{\textbf{\ipa{ɖɤ˧mi˧}}}}\kern2pt]} \hypertarget{d`7\string_Mmi\string_M1}{}
\markboth{\textcolor{darkblue}{\textbf{\ipa{ɖɤ˧mi˧}}}}{}
\textcolor{teal}{\mytextsc{nom}} \hspace{4pt} Ton~: M.
\textcolor{PineGreen}{\selectlanguage{french}Renard.} \zh{狐狸。}  ¶ \textcolor{darkblue}{\textbf{\ipa{ɖɤ˧mi˧-zo\#˥}}} \textcolor{PineGreen}{\selectlanguage{french}petit renard, renardeau} \zh{小狐狸}  
 ¶ \textcolor{darkblue}{\textbf{\ipa{ɖɤ˧mi˧-pʰv̩\#˥}}} \textcolor{PineGreen}{\selectlanguage{french}renard mâle} \zh{公狐狸}  
 ¶ \textcolor{darkblue}{\textbf{\ipa{ɖɤ˧mi˧, | mi˩ ɲi˥!}}} \textcolor{PineGreen}{\selectlanguage{french}Ce renard, c'est une femelle!} \zh{这只狐狸是母的!}  
 \zh{量词}: \textcolor{darkblue}{\textbf{\ipa{pʰo˧˥}}}  \mytextsc{clf}~: \textcolor{darkblue}{\textbf{\ipa{pʰo˧˥}}} 
\lhead{\firstmark}
\rhead{\botmark}

\subsection{\hspace{-0.5cm} {\Large \textcolor{darkblue}{\textbf{\ipa{ɖɤ˩\textsubscript{a}}}}}\hspace{0.5cm}[\kern2pt{\textcolor{darkblue}{\textbf{\ipa{ɖɤ˩˥}}}}\kern2pt]} \hypertarget{d`7\string_Ba1}{}
\markboth{\textcolor{darkblue}{\textbf{\ipa{ɖɤ˩\textsubscript{a}}}}}{}
\textcolor{teal}{\mytextsc{adjectif}} \hspace{4pt} Ton~: L\textsubscript{a}.
\textcolor{PineGreen}{\selectlanguage{french}Brûlant, ardent (soleil), chaud (temps).} \zh{很热(天气),阳光强烈。}  ¶ \textcolor{darkblue}{\textbf{\ipa{ɲi˧mi˧ | ɖɤ˩-ze˥!}}} \textcolor{PineGreen}{\selectlanguage{french}le soleil est torride, le soleil est très vif} \zh{太阳很大、很强烈}  
 ¶ \textcolor{darkblue}{\textbf{\ipa{ɖɤ˩-hĩ˩˥}}} \textcolor{PineGreen}{\selectlanguage{french}\mytextsc{rel}} \zh{热的}  

\lhead{\firstmark}
\rhead{\botmark}

\subsection{\hspace{-0.5cm} {\Large \textcolor{darkblue}{\textbf{\ipa{ɖo˧}}}}\hspace{0.5cm}[\kern2pt{\textcolor{darkblue}{\textbf{\ipa{ɖo˥}}}}\kern2pt]} \hypertarget{d`o\string_M1}{}
\markboth{\textcolor{darkblue}{\textbf{\ipa{ɖo˧}}}}{}
\textcolor{teal}{\mytextsc{verbe}} \hspace{4pt} Ton~: M intrans.
\textcolor{PineGreen}{\selectlanguage{french}Devoir, être obligé de; permettre, autoriser, accorder; ordonner, donner un ordre.} \zh{让,指使、使唤。}  ¶ \textcolor{darkblue}{\textbf{\ipa{po˧ mɤ˧-ɖo˧!}}} \textcolor{PineGreen}{\selectlanguage{french}(Tu n'as) pas le droit de le prendre! (ex.: un enfant n'est pas autorisé à jouer avec un couteau)} \zh{不许拿!}  
 ¶ \textcolor{darkblue}{\textbf{\ipa{ʈʂʰɯ˧, | po˧ ɖo˧!}}} \textcolor{PineGreen}{\selectlanguage{french}Ca, (tu) peux le prendre / tu peux jouer avec! (Même contexte que ci-dessus: on indique à un enfant ce qu'on a le droit de prendre, et ce qu'on n'a pas le droit de prendre.)} \zh{那个,是可以拿的! / 那个,是可以玩的!(情景同上:告诉一个小孩子什么东西可以拿,什么不可以拿。)}  
 ¶ \textcolor{darkblue}{\textbf{\ipa{gɤ˩ do˧ mɤ˧-ɖo˧!}}} \textcolor{PineGreen}{\selectlanguage{french}(tu) n'as pas le droit de grimper/monter sur (une table…)} \zh{不许爬上(桌子……)}  
 ¶ \textcolor{darkblue}{\textbf{\ipa{lɑ˧-kʰv̩˧˥, | ʑi˧qʰwɤ˧ tsʰi˧-mɤ˧-ɖo˧! | ʑi˩-kʰv̩˩˥, | ʑi˧qʰwɤ˧ tsʰi˧-mɤ˧-ɖo˧! |}}} \textcolor{PineGreen}{\selectlanguage{french}L'année du Tigre, l'année du Singe, il ne faut pas construire de maison/il ne faut pas se lancer dans la construction d'une maison! (Ce sont des années trop “dures”, \textcolor{darkblue}{\textbf{\ipa{/wu˧/}}}, selon l'astrologie traditionnelle)} \zh{虎年,不要建房!猴年,不要建房!(这样的年,被认为是太‘硬’的。)}  
 ¶ \textcolor{darkblue}{\textbf{\ipa{ʝi˧ mɤ˧-ɖo˧!}}} \textcolor{PineGreen}{\selectlanguage{french}Il ne faut pas faire (ça)!} \zh{不要做!}  

\lhead{\firstmark}
\rhead{\botmark}

\subsection{\hspace{-0.5cm} {\Large \textcolor{darkblue}{\textbf{\ipa{ɖɯ˧-}}}}\hspace{0.5cm}[\kern2pt{\textcolor{darkblue}{\textbf{\ipa{ɖɯ˥}}}}\kern2pt]} \hypertarget{d`M\string_M-1}{}
\markboth{\textcolor{darkblue}{\textbf{\ipa{ɖɯ˧-}}}}{}
\textcolor{teal}{\mytextsc{préposition}} \hspace{4pt} Ton~: M.
\textcolor{PineGreen}{\selectlanguage{french}\mytextsc{délimitatif}.} \zh{\mytextsc{进行时态。}} 
\lhead{\firstmark}
\rhead{\botmark}

\subsection{\hspace{-0.5cm} {\Large \textcolor{darkblue}{\textbf{\ipa{ɖɯ˧\textsubscript{b}}}}}\hspace{0.5cm}[\kern2pt{\textcolor{darkblue}{\textbf{\ipa{ɖɯ˥}}}}\kern2pt]} \hypertarget{d`M\string_Mb1}{}
\markboth{\textcolor{darkblue}{\textbf{\ipa{ɖɯ˧\textsubscript{b}}}}}{}
\textcolor{teal}{\mytextsc{verbe}} \hspace{4pt} Ton~: M\textsubscript{b}.
\textcolor{PineGreen}{\selectlanguage{french}Obtenir, trouver.} \zh{得到。}  ¶ \textcolor{darkblue}{\textbf{\ipa{le˧-ʂe˧ le˧-ɖɯ˧-ze˧!}}} \textcolor{PineGreen}{\selectlanguage{french}j'ai cherché et je l'ai trouvé! j'ai trouvé (en cherchant)!} \zh{找到了!}  
 ¶ \textcolor{darkblue}{\textbf{\ipa{ɖɯ˧-tʰɑ˧˥!}}} \textcolor{PineGreen}{\selectlanguage{french}On peut obtenir!} \zh{可以得到的!}  
 ¶ \textcolor{darkblue}{\textbf{\ipa{ɖɯ˧-tʰɑ˧-ze˥!}}} \textcolor{PineGreen}{\selectlanguage{french}On a réussi à obtenir!} \zh{(我们)成功地得到了!}  
 ¶ \textcolor{darkblue}{\textbf{\ipa{tso˧\textasciitilde{}tso˧ ɖɯ˧ (+ze˧)}}} \textcolor{PineGreen}{\selectlanguage{french}obtenir quelque chose} \zh{得到东西}  

\lhead{\firstmark}
\rhead{\botmark}

\subsection{\hspace{-0.5cm} {\Large \textcolor{darkblue}{\textbf{\ipa{ɖɯ˧-ɬi˧mi˧}}}}\hspace{0.5cm}[\kern2pt{\textcolor{darkblue}{\textbf{\ipa{xxxx non-correspondance entre le nombre de morphèmes et le nombre de tons de morphèmes}}}}\kern2pt]} \hypertarget{d`M\string_M-Ki\string_Mmi\string_M1}{}
\markboth{\textcolor{darkblue}{\textbf{\ipa{ɖɯ˧-ɬi˧mi˧}}}}{}
\textcolor{teal}{\mytextsc{nom}} \hspace{4pt} Ton~: M.
\textcolor{PineGreen}{\selectlanguage{french}1er mois.} \zh{正月。} 
\lhead{\firstmark}
\rhead{\botmark}

\subsection{\hspace{-0.5cm} {\Large \textcolor{darkblue}{\textbf{\ipa{ɖɯ˧-njɤ˧}}}}\hspace{0.5cm}[\kern2pt{\textcolor{darkblue}{\textbf{\ipa{xxxx non-correspondance entre le nombre de morphèmes et le nombre de tons de morphèmes}}}}\kern2pt]} \hypertarget{d`M\string_M-nj7\string_M1}{}
\markboth{\textcolor{darkblue}{\textbf{\ipa{ɖɯ˧-njɤ˧}}}}{}
\textcolor{teal}{\mytextsc{adverbe}} \hspace{4pt} Ton~: M.
\textcolor{PineGreen}{\selectlanguage{french}Sans cesse; sans arrêt; toujours.} \zh{一直、一直不停。}  ¶ \textcolor{darkblue}{\textbf{\ipa{ɖɯ˧-njɤ˧ | so˩˥}}} \textcolor{PineGreen}{\selectlanguage{french}étudier sans arrêt} \zh{一直不停地学习}  
 ¶ \textcolor{darkblue}{\textbf{\ipa{ɖɯ˧-njɤ˧ | lo˧ ʝi˧}}} \textcolor{PineGreen}{\selectlanguage{french}travailler sans arrêt} \zh{一直不停地工作}  
 ¶ \textcolor{darkblue}{\textbf{\ipa{ɖɯ˧-njɤ˧-zo˥}}} \textcolor{PineGreen}{\selectlanguage{french}souvent} \zh{经常、常}  
 ¶ \textcolor{darkblue}{\textbf{\ipa{ɖɯ˧-njɤ˧ hwæ˩; ɖɯ˧-njɤ˧ tɕʰi˧; ɖɯ˧-njɤ˧ dzɯ˧; ɖɯ˧-njɤ˧ dze˧˥; ɖɯ˧-njɤ˧ ʐwɤ˧˥; ɖɯ˧-njɤ˧ lɑ˧˥}}} \textcolor{PineGreen}{\selectlanguage{french}avec des verbes aux six tons, pour étudier les tons : acheter; vendre; manger; couper; parler; frapper} \zh{跟六个调类的动词结合:买,卖,吃,切,说,打}  
 ¶ \textcolor{darkblue}{\textbf{\ipa{ɖɯ˧-njɤ˧ | hwæ˧; ɖɯ˧-njɤ˧ | tɕʰi˧; ɖɯ˧-njɤ˧ | dzɯ˧; ɖɯ˧-njɤ˧ | dze˩˥; ɖɯ˧-njɤ˧ | ʐwɤ˩˥; ɖɯ˧-njɤ˧ | lɑ˧˥}}} \textcolor{PineGreen}{\selectlanguage{french}avec des verbes aux six tons, pour étudier les tons : acheter; vendre; manger; couper; parler; frapper; en séparant en groupes tonals} \zh{跟六个调类的动词结合:买,卖,吃,切,说,打}  

\lhead{\firstmark}
\rhead{\botmark}

\subsection{\hspace{-0.5cm} {\Large \textcolor{darkblue}{\textbf{\ipa{ɖɯ˧-ɲi˧-ɖɯ˥-hɑ̃˩}}}}\hspace{0.5cm}[\kern2pt{\textcolor{darkblue}{\textbf{\ipa{xxxx non-correspondance entre le nombre de morphèmes et le nombre de tons de morphèmes}}}}\kern2pt]} \hypertarget{d`M\string_M-Ji\string_M-d`M\string_T-hA\string_~\string_B1}{}
\markboth{\textcolor{darkblue}{\textbf{\ipa{ɖɯ˧-ɲi˧-ɖɯ˥-hɑ̃˩}}}}{}
\textcolor{teal}{\mytextsc{nom}} \hspace{4pt} Ton~: \#H-.
\textcolor{PineGreen}{\selectlanguage{french}Un jour et une nuit.} \zh{一天一夜。} 
\lhead{\firstmark}
\rhead{\botmark}

\subsection{\hspace{-0.5cm} {\Large \textcolor{darkblue}{\textbf{\ipa{ɖɯ˧-so˩}}}}\hspace{0.5cm}[\kern2pt{\textcolor{darkblue}{\textbf{\ipa{xxxx non-correspondance entre le nombre de morphèmes et le nombre de tons de morphèmes}}}}\kern2pt]} \hypertarget{d`M\string_M-so\string_B1}{}
\markboth{\textcolor{darkblue}{\textbf{\ipa{ɖɯ˧-so˩}}}}{}
\textcolor{teal}{\mytextsc{nom}} \hspace{4pt} Ton~: L\textsubscript{a}.
\textcolor{PineGreen}{\selectlanguage{french}Un petit nombre de, quelques-uns.} \zh{一些、两三个(直译:‘一三(个)’。}  ¶ \textcolor{darkblue}{\textbf{\ipa{hĩ˧ | ɖɯ˧-so˩ kv̩˩}}} \textcolor{PineGreen}{\selectlanguage{french}quelques personnes (deux, trois...)} \zh{几个人}  
 ¶ \textcolor{darkblue}{\textbf{\ipa{ɖɯ˧-so˩ ɲi˩}}} \textcolor{PineGreen}{\selectlanguage{french}quelques jours} \zh{几天}  

\lhead{\firstmark}
\rhead{\botmark}

\subsection{\hspace{-0.5cm} {\Large \textcolor{darkblue}{\textbf{\ipa{ɖɯ˧ʈæ˩}}}}\hspace{0.5cm}[\kern2pt{\textcolor{darkblue}{\textbf{\ipa{ɖɯ˧ʈæ˩}}}}\kern2pt]} \hypertarget{d`M\string_Mt`\{\string_B1}{}
\markboth{\textcolor{darkblue}{\textbf{\ipa{ɖɯ˧ʈæ˩}}}}{}
\textcolor{teal}{\mytextsc{nom}} \hspace{4pt} Ton~: L\#.
\textcolor{PineGreen}{\selectlanguage{french}Rite pratiqué par les moines du monastère après un décès.} \zh{一个葬礼仪式,由和尚主持。} 
\lhead{\firstmark}
\rhead{\botmark}

\subsection{\hspace{-0.5cm} {\Large \textcolor{darkblue}{\textbf{\ipa{ɖɯ˩\textsubscript{a}}}}}\hspace{0.5cm}[\kern2pt{\textcolor{darkblue}{\textbf{\ipa{ɖɯ˩˥}}}}\kern2pt]} \hypertarget{d`M\string_Ba1}{}
\markboth{\textcolor{darkblue}{\textbf{\ipa{ɖɯ˩\textsubscript{a}}}}}{}
\textcolor{teal}{\mytextsc{adjectif}} \hspace{4pt} Ton~: L\textsubscript{a}.
\textcolor{PineGreen}{\selectlanguage{french}Grand.} \zh{大。}  ¶ \textcolor{darkblue}{\textbf{\ipa{mɤ˧-ɖɯ˩}}} \textcolor{PineGreen}{\selectlanguage{french}\mytextsc{neg}} \zh{\mytextsc{neg}}  
 ¶ \textcolor{darkblue}{\textbf{\ipa{ɖɯ˩-hĩ˩˥}}} \textcolor{PineGreen}{\selectlanguage{french}\mytextsc{rel}} \zh{\mytextsc{rel}}  
 ¶ \textcolor{darkblue}{\textbf{\ipa{le˧-ɖɯ˩(-ze˩)}}} \textcolor{PineGreen}{\selectlanguage{french}\mytextsc{accomp}+\mytextsc{pfv}: ça a grandi!/ il/elle a grandi!} \zh{\mytextsc{accomp}+\mytextsc{pfv}}  
 ¶ \textcolor{darkblue}{\textbf{\ipa{ə˧pɤ˥ɖɯ˩-gv̩˩}}} \textcolor{PineGreen}{\selectlanguage{french}très grand} \zh{好大、大得很}  

\lhead{\firstmark}
\rhead{\botmark}

\subsection{\hspace{-0.5cm} {\Large \textcolor{darkblue}{\textbf{\ipa{ɖɯ˩ɖʐɯ˧}}}}\hspace{0.5cm}[\kern2pt{\textcolor{darkblue}{\textbf{\ipa{ɖɯ˩ɖʐɯ˥}}}}\kern2pt]} \hypertarget{d`M\string_Bd`z`M\string_M1}{}
\markboth{\textcolor{darkblue}{\textbf{\ipa{ɖɯ˩ɖʐɯ˧}}}}{}
\textcolor{teal}{\mytextsc{nom}} \hspace{4pt} Ton~: LM.
\textcolor{PineGreen}{\selectlanguage{french}Prénom masculin.} \zh{男性名字:独知。}  Emprunt~: tibétain

\lhead{\firstmark}
\rhead{\botmark}

\subsection{\hspace{-0.5cm} {\Large \textcolor{darkblue}{\textbf{\ipa{ɖɯ˩ɖʐɯ˧-tsʰɯ˩ɻ̍˩}}}}\hspace{0.5cm}[\kern2pt{\textcolor{darkblue}{\textbf{\ipa{ɖɯ˩ɖʐɯ˧tsʰɯ˩ɻ̍˩}}}}\kern2pt]} \hypertarget{d`M\string_Bd`z`M\string_M-ts\string_hM\string_Br£`̍\string_B1}{}
\markboth{\textcolor{darkblue}{\textbf{\ipa{ɖɯ˩ɖʐɯ˧-tsʰɯ˩ɻ̍˩}}}}{}
\textcolor{teal}{\mytextsc{nom}} \hspace{4pt} Ton~: LM-L.
\textcolor{PineGreen}{\selectlanguage{french}Prénom masculin.} \zh{男性名字。} 
\lhead{\firstmark}
\rhead{\botmark}

\subsection{\hspace{-0.5cm} {\Large \textcolor{darkblue}{\textbf{\ipa{ɖɯ˩hĩ˩}}}}\hspace{0.5cm}[\kern2pt{\textcolor{darkblue}{\textbf{\ipa{ɖɯ˩hĩ˩˥}}}}\kern2pt]} \hypertarget{d`M\string_Bhi\string_~\string_B1}{}
\markboth{\textcolor{darkblue}{\textbf{\ipa{ɖɯ˩hĩ˩}}}}{}
\textcolor{teal}{\mytextsc{nom}} \hspace{4pt} Ton~: L.
\textcolor{PineGreen}{\selectlanguage{french}Gens importants.} \zh{大人、重要的人(包括长辈)。}  \zh{量词}: \textcolor{darkblue}{\textbf{\ipa{v̩˧}}}  \mytextsc{clf}~: \textcolor{darkblue}{\textbf{\ipa{v̩˧}}} 
\lhead{\firstmark}
\rhead{\botmark}

\subsection{\hspace{-0.5cm} {\Large \textcolor{darkblue}{\textbf{\ipa{ɖɯ˩lo\#˥}}}}\hspace{0.5cm}[\kern2pt{\textcolor{darkblue}{\textbf{\ipa{ɖɯ˩lo˥}}}}\kern2pt]} \hypertarget{d`M\string_Blo\#\string_T1}{}
\markboth{\textcolor{darkblue}{\textbf{\ipa{ɖɯ˩lo\#˥}}}}{}
\textcolor{teal}{\mytextsc{nom}} \hspace{4pt} Ton~: LM+\#H.
\ding{202} \textcolor{PineGreen}{\selectlanguage{french}Coutume, tradition.} \zh{传统。}  ¶ \textcolor{darkblue}{\textbf{\ipa{ɖɯ˩lo˧ ɖɯ˧-kʰwɤ˥ | tʰi˧-so˥-ɻ̍˩}}} \textcolor{PineGreen}{\selectlanguage{french}enseigner une coutume} \zh{教授一个传统、一个习俗}  
 \zh{量词}: \textcolor{darkblue}{\textbf{\ipa{kʰwɤ˥}}} \ding{203} \textcolor{PineGreen}{\selectlanguage{french}Savoir-vivre.} \zh{礼仪、礼貌。}  ¶ \textcolor{darkblue}{\textbf{\ipa{ʈʂʰɯ˧ | ɖɯ˩lo˧ dʑɤ˥!}}} \textcolor{PineGreen}{\selectlanguage{french}Il/elle sait vivre/connaît les coutumes/a du savoir-vivre!} \zh{他懂礼貌、他会做人}  
\ding{204} \textcolor{PineGreen}{\selectlanguage{french}Ordre des choses.} \zh{道理。}  \mytextsc{clf}~: \textcolor{darkblue}{\textbf{\ipa{kʰwɤ˥}}} 
\lhead{\firstmark}
\rhead{\botmark}

\subsection{\hspace{-0.5cm} {\Large \textcolor{darkblue}{\textbf{\ipa{ɖɯ˩mɑ\#˥}}}}\hspace{0.5cm}[\kern2pt{\textcolor{darkblue}{\textbf{\ipa{ɖɯ˩mɑ˥}}}}\kern2pt]} \hypertarget{d`M\string_BmA\#\string_T1}{}
\markboth{\textcolor{darkblue}{\textbf{\ipa{ɖɯ˩mɑ\#˥}}}}{}
\textcolor{teal}{\mytextsc{nom}} \hspace{4pt} Ton~: LM+\#H.
\textcolor{PineGreen}{\selectlanguage{french}Prénom féminin.} \zh{女性名字。} 
\lhead{\firstmark}
\rhead{\botmark}

\subsection{\hspace{-0.5cm} {\Large \textcolor{darkblue}{\textbf{\ipa{ɖɯ˩mɑ˧-ɬɑ˩tsʰo˩}}}}\hspace{0.5cm}[\kern2pt{\textcolor{darkblue}{\textbf{\ipa{ɖɯ˩mɑ˧ɬɑ˩tsʰo˩}}}}\kern2pt]} \hypertarget{d`M\string_BmA\string_M-KA\string_Bts\string_ho\string_B1}{}
\markboth{\textcolor{darkblue}{\textbf{\ipa{ɖɯ˩mɑ˧-ɬɑ˩tsʰo˩}}}}{}
\textcolor{teal}{\mytextsc{nom}} \hspace{4pt} Ton~: LM-L.
\textcolor{PineGreen}{\selectlanguage{french}Prénom féminin.} \zh{女性名字。} 
\lhead{\firstmark}
\rhead{\botmark}

\subsection{\hspace{-0.5cm} {\Large \textcolor{darkblue}{\textbf{\ipa{ɖɯ˩mɑ˧-pv̩˩ʈʰɯ˩}}}}\hspace{0.5cm}[\kern2pt{\textcolor{darkblue}{\textbf{\ipa{ɖɯ˩mɑ˧pv̩˩ʈʰɯ˩}}}}\kern2pt]} \hypertarget{d`M\string_BmA\string_M-pv\string_=\string_Bt`\string_hM\string_B1}{}
\markboth{\textcolor{darkblue}{\textbf{\ipa{ɖɯ˩mɑ˧-pv̩˩ʈʰɯ˩}}}}{}
\textcolor{teal}{\mytextsc{nom}} \hspace{4pt} Ton~: LM-L.
\textcolor{PineGreen}{\selectlanguage{french}Prénom féminin.} \zh{女性名字。} 
\lhead{\firstmark}
\rhead{\botmark}

\subsection{\hspace{-0.5cm} {\Large \textcolor{darkblue}{\textbf{\ipa{ɖɯ˩mi\#˥}}}}\hspace{0.5cm}[\kern2pt{\textcolor{darkblue}{\textbf{\ipa{ɖɯ˩mi˥}}}}\kern2pt]} \hypertarget{d`M\string_Bmi\#\string_T1}{}
\markboth{\textcolor{darkblue}{\textbf{\ipa{ɖɯ˩mi\#˥}}}}{}
\textcolor{teal}{\mytextsc{nom}} \hspace{4pt} Ton~: LM+\#H.
\textcolor{PineGreen}{\selectlanguage{french}Mule femelle. C'est un animal stérile. Il est docile et fort, et peut accomplir des tâches importantes comme d'être l'animal de tête dans une caravane. C'est donc un animal de grand prix.} \zh{母骡子、母马骡。}  ¶ \textcolor{darkblue}{\textbf{\ipa{ɖɯ˩mi˧-ɖɯ˥zo˩}}} \textcolor{PineGreen}{\selectlanguage{french}mule femelle et mule mâle} \zh{母骡子与公骡子}  
 \zh{量词}: \textcolor{darkblue}{\textbf{\ipa{mi˩}}}  \mytextsc{clf}~: \textcolor{darkblue}{\textbf{\ipa{mi˩}}} 
\lhead{\firstmark}
\rhead{\botmark}

\subsection{\hspace{-0.5cm} {\Large \textcolor{darkblue}{\textbf{\ipa{ɖɯ˩zo\#˥}}}}\hspace{0.5cm}[\kern2pt{\textcolor{darkblue}{\textbf{\ipa{ɖɯ˩zo˥}}}}\kern2pt]} \hypertarget{d`M\string_Bzo\#\string_T1}{}
\markboth{\textcolor{darkblue}{\textbf{\ipa{ɖɯ˩zo\#˥}}}}{}
\textcolor{teal}{\mytextsc{nom}} \hspace{4pt} Ton~: LM+\#H.
\textcolor{PineGreen}{\selectlanguage{french}Mule mâle (animal moins prisé que la femelle).} \zh{公骡子。}  ¶ \textcolor{darkblue}{\textbf{\ipa{ɖɯ˩zo˧-ɖɯ˥mi˩}}} \textcolor{PineGreen}{\selectlanguage{french}mule mâle et mule femelle} \zh{公骡子与母骡子}  
 \zh{量词}: \textcolor{darkblue}{\textbf{\ipa{ɭɯ˧}}}  \mytextsc{clf}~: \textcolor{darkblue}{\textbf{\ipa{ɭɯ˧}}} 
\lhead{\firstmark}
\rhead{\botmark}

\subsection{\hspace{-0.5cm} {\Large \textcolor{darkblue}{\textbf{\ipa{ɖɯ˧˥}}}}\hspace{0.5cm}[\kern2pt{\textcolor{darkblue}{\textbf{\ipa{ɖɯ˧˥}}}}\kern2pt]} \hypertarget{d`M\string_M\string_T1}{}
\markboth{\textcolor{darkblue}{\textbf{\ipa{ɖɯ˧˥}}}}{}
\textcolor{teal}{\mytextsc{nombre}} \hspace{4pt} Ton~: MH.
\textcolor{PineGreen}{\selectlanguage{french}1.} \zh{1。} 
\lhead{\firstmark}
\rhead{\botmark}

\subsection{\hspace{-0.5cm} {\Large \textcolor{darkblue}{\textbf{\ipa{ɖv̩˩}}}}\hspace{0.5cm}[\kern2pt{\textcolor{darkblue}{\textbf{\ipa{ɖv̩˥}}}}\kern2pt]} \hypertarget{d`v\string_=\string_B1}{}
\markboth{\textcolor{darkblue}{\textbf{\ipa{ɖv̩˩}}}}{}
\textcolor{teal}{\mytextsc{nom}} \hspace{4pt} Ton~: L.
\textcolor{PineGreen}{\selectlanguage{french}Ailes (forme monosyllabique; la forme disyllabique est préférée).} \zh{翅膀。}  \zh{量词}: \textcolor{darkblue}{\textbf{\ipa{dze˩}}}  \mytextsc{clf}~: \textcolor{darkblue}{\textbf{\ipa{dze˩}}} \textcolor{PineGreen}{\selectlanguage{french}paire}
\lhead{\firstmark}
\rhead{\botmark}

\subsection{\hspace{-0.5cm} {\Large \textcolor{darkblue}{\textbf{\ipa{ɖv̩˧qæ˧}}}}\hspace{0.5cm}[\kern2pt{\textcolor{darkblue}{\textbf{\ipa{ɖv̩˩qæ˥}}}}\kern2pt]} \hypertarget{d`v\string_=\string_Mq\{\string_M1}{}
\markboth{\textcolor{darkblue}{\textbf{\ipa{ɖv̩˧qæ˧}}}}{}
\textcolor{teal}{\mytextsc{nom}} \hspace{4pt} Ton~: M.
\textcolor{PineGreen}{\selectlanguage{french}Ailes.} \zh{翅膀。}  ¶ \textcolor{darkblue}{\textbf{\ipa{kɤ˩nɑ˧mi˧-ɖv̩˧qæ˥}}} \textcolor{PineGreen}{\selectlanguage{french}aile d'aigle} \zh{老鹰翅膀}  
 \zh{量词}: \textcolor{darkblue}{\textbf{\ipa{dze˩}}}  \mytextsc{clf}~: \textcolor{darkblue}{\textbf{\ipa{dze˩}}} \textcolor{PineGreen}{\selectlanguage{french}paire}
\lhead{\firstmark}
\rhead{\botmark}

\subsection{\hspace{-0.5cm} {\Large \textcolor{darkblue}{\textbf{\ipa{ɖwæ˥}}}}\hspace{0.5cm}[\kern2pt{\textcolor{darkblue}{\textbf{\ipa{ɖwæ˥}}}}\kern2pt]} \hypertarget{d`w\{\string_T1}{}
\markboth{\textcolor{darkblue}{\textbf{\ipa{ɖwæ˥}}}}{}
\textcolor{teal}{\mytextsc{nom}} \hspace{4pt} Ton~: \#H.
\ding{202} \textcolor{PineGreen}{\selectlanguage{french}Mare.} \zh{池塘。}  ¶ \textcolor{darkblue}{\textbf{\ipa{[F5] ɖwæ˩ɬo˩mi˧}}} \textcolor{PineGreen}{\selectlanguage{french}grand étang} \zh{大池塘}  
 \zh{量词}: \textcolor{darkblue}{\textbf{\ipa{ɭɯ˧}}} \ding{203} \textcolor{PineGreen}{\selectlanguage{french}Réserve d'eau (artificielle).} \zh{水坑。}  \mytextsc{clf}~: \textcolor{darkblue}{\textbf{\ipa{ɭɯ˧}}} \textcolor{PineGreen}{\selectlanguage{french}objets ronds}
\lhead{\firstmark}
\rhead{\botmark}

\subsection{\hspace{-0.5cm} {\Large \textcolor{darkblue}{\textbf{\ipa{ɖwæ˥}}}}\hspace{0.5cm}[\kern2pt{\textcolor{darkblue}{\textbf{\ipa{ɖwæ˧˥}}}}\kern2pt]} \hypertarget{d`w\{\string_T1}{}
\markboth{\textcolor{darkblue}{\textbf{\ipa{ɖwæ˥}}}}{}
\textcolor{teal}{\mytextsc{adjectif}} \hspace{4pt} Ton~: H.
\textcolor{PineGreen}{\selectlanguage{french}Trouble (le même terme est employé pour l'eau vive et pour l'eau stagnante).} \zh{浑浊 (水)。}  ¶ \textcolor{darkblue}{\textbf{\ipa{dʑɯ˧ ɖwæ\#˥}}} \textcolor{PineGreen}{\selectlanguage{french}eau trouble} \zh{浑浊的水}  
 ¶ \textcolor{darkblue}{\textbf{\ipa{dʑɯ˧ | ɖwæ˧-ze˩!}}} \textcolor{PineGreen}{\selectlanguage{french}l'eau s'est troublée! l'eau est devenue trouble!} \zh{水浑浊了。}  

\lhead{\firstmark}
\rhead{\botmark}

\subsection{\hspace{-0.5cm} {\Large \textcolor{darkblue}{\textbf{\ipa{ɖwæ˥\textsubscript{a}}}}}\hspace{0.5cm}[\kern2pt{\textcolor{darkblue}{\textbf{\ipa{ɖwæ˩˥}}}}\kern2pt]} \hypertarget{d`w\{\string_Ta1}{}
\markboth{\textcolor{darkblue}{\textbf{\ipa{ɖwæ˥\textsubscript{a}}}}}{}
\textcolor{teal}{\mytextsc{classificateur}} \hspace{4pt} Ton~: H\textsubscript{a}.
\textcolor{PineGreen}{\selectlanguage{french}Classificateur des marches d'escalier.} \zh{量词:梯级、楼梯(一节)。}  ¶ \textcolor{darkblue}{\textbf{\ipa{ɖɯ˧-ɖwæ˧ ɲi˥}}} \textcolor{PineGreen}{\selectlanguage{french}c'est une marche} \zh{是一节/一节阶梯。(引出这句是为了了解这个词在不同语境的声调变化。)}  
 ¶ \textcolor{darkblue}{\textbf{\ipa{ʈʂʰɯ˧-ɖwæ\#˥}}} \textcolor{PineGreen}{\selectlanguage{french}cette marche} \zh{这节阶梯}  

\lhead{\firstmark}
\rhead{\botmark}

\subsection{\hspace{-0.5cm} {\Large \textcolor{darkblue}{\textbf{\ipa{ɖwæ˧-pɤ˧ɭɯ˥}}}}\hspace{0.5cm}[\kern2pt{\textcolor{darkblue}{\textbf{\ipa{xxxx non-correspondance entre le nombre de morphèmes et le nombre de tons de morphèmes}}}}\kern2pt]} \hypertarget{d`w\{\string_M-p7\string_Ml\string_RM\string_T1}{}
\markboth{\textcolor{darkblue}{\textbf{\ipa{ɖwæ˧-pɤ˧ɭɯ˥}}}}{}
\textcolor{teal}{\mytextsc{nom}} \hspace{4pt} Ton~: H\#.
\textcolor{PineGreen}{\selectlanguage{french}Flaque (naturelle).} \zh{水潭。}  ¶ \textcolor{darkblue}{\textbf{\ipa{ɖwæ˧ tʰi˧-pɤ˥ɭɯ˩}}} \textcolor{PineGreen}{\selectlanguage{french}une flaque/petite mare s'est formée, il y a une flaque} \zh{有水潭}  
 \zh{量词}: \textcolor{darkblue}{\textbf{\ipa{ɭɯ˧}}}  \mytextsc{clf}~: \textcolor{darkblue}{\textbf{\ipa{ɭɯ˧}}} \textcolor{PineGreen}{\selectlanguage{french}objets ronds}
\lhead{\firstmark}
\rhead{\botmark}

\subsection{\hspace{-0.5cm} {\Large \textcolor{darkblue}{\textbf{\ipa{ɖwæ˩\textsubscript{a}}}}}\hspace{0.5cm}[\kern2pt{\textcolor{darkblue}{\textbf{\ipa{ɖwæ˥}}}}\kern2pt]} \hypertarget{d`w\{\string_Ba1}{}
\markboth{\textcolor{darkblue}{\textbf{\ipa{ɖwæ˩\textsubscript{a}}}}}{}
\textcolor{teal}{\mytextsc{verbe}} \hspace{4pt} Ton~: L\textsubscript{a}.
\textcolor{PineGreen}{\selectlanguage{french}Avoir peur.} \zh{害怕。}  ¶ \textcolor{darkblue}{\textbf{\ipa{njɤ˧ | ɖwæ˩˥!}}} \textcolor{PineGreen}{\selectlanguage{french}J'ai peur!} \zh{我害怕!}  
 ¶ \textcolor{darkblue}{\textbf{\ipa{njɤ˧ | ʈʂʰɯ˧-v̩˧ | ɖwæ˩˥ | ʐwæ˩˥!}}} \textcolor{PineGreen}{\selectlanguage{french}J'ai très peur de lui!} \zh{我很害怕那个人!}  

\lhead{\firstmark}
\rhead{\botmark}

\subsection{\hspace{-0.5cm} {\Large \textcolor{darkblue}{\textbf{\ipa{ɖwæ˧˥}}} \textsubscript{1}}\hspace{0.5cm}[\kern2pt{\textcolor{darkblue}{\textbf{\ipa{ɖwæ˩˥}}}}\kern2pt]} \hypertarget{d`w\{\string_M\string_T1}{}
\markboth{\textcolor{darkblue}{\textbf{\ipa{ɖwæ˧˥}}} \textsubscript{1}}{}
\textcolor{teal}{\mytextsc{verbe}} \hspace{4pt} Ton~: MH.
\textcolor{PineGreen}{\selectlanguage{french}Fouetter, donner des coups (ex.: un tigre fouette le sol avec sa queue).} \zh{鞭打、抽打、加鞭。}  ¶ \textcolor{darkblue}{\textbf{\ipa{mæ˧qv̩˩-po˩-ɳɯ˩ | ɖwæ˧˥}}} \textcolor{PineGreen}{\selectlanguage{french}donner des coups de queue (ex.: le tigre fouette le sol de sa queue)} \zh{用尾巴来抽打(如:老虎用尾巴来抽打地面)}  

\lhead{\firstmark}
\rhead{\botmark}

\subsection{\hspace{-0.5cm} {\Large \textcolor{darkblue}{\textbf{\ipa{ɖwæ˧˥}}} \textsubscript{2}}\hspace{0.5cm}[\kern2pt{\textcolor{darkblue}{\textbf{\ipa{ɖwæ˧˥}}}}\kern2pt]} \hypertarget{d`w\{\string_M\string_T2}{}
\markboth{\textcolor{darkblue}{\textbf{\ipa{ɖwæ˧˥}}} \textsubscript{2}}{}
\textcolor{teal}{\mytextsc{adverbe}} \hspace{4pt} Ton~: MH.
\textcolor{PineGreen}{\selectlanguage{french}Intensif: très.} \zh{很、极。}  ¶ \textcolor{darkblue}{\textbf{\ipa{ʈʂʰɯ˧ | ɖwæ˧˥ | æ˧mv̩˩ fv̩˩!}}} \textcolor{PineGreen}{\selectlanguage{french}elle aime beaucoup sa grande sœur!} \zh{她很喜欢她姐姐!}  

\lhead{\firstmark}
\rhead{\botmark}

\newpage
\section*{\centering- \textcolor{darkblue}{\textbf{\ipa{ɖʐ}}} -}
\subsection{\hspace{-0.5cm} {\Large \textcolor{darkblue}{\textbf{\ipa{ɖʐæ˧\textsubscript{b}}}}}\hspace{0.5cm}[\kern2pt{\textcolor{darkblue}{\textbf{\ipa{ɖʐæ˩˥}}}}\kern2pt]} \hypertarget{d`z`\{\string_Mb1}{}
\markboth{\textcolor{darkblue}{\textbf{\ipa{ɖʐæ˧\textsubscript{b}}}}}{}
\textcolor{teal}{\mytextsc{verbe}} \hspace{4pt} Ton~: M\textsubscript{b}.
\textcolor{PineGreen}{\selectlanguage{french}Monter à cheval.} \zh{骑马。}  ¶ \textcolor{darkblue}{\textbf{\ipa{le˧-ɖʐæ˧-ze˧}}} \textcolor{PineGreen}{\selectlanguage{french}\mytextsc{accomp} \string_ \mytextsc{pfv}} \zh{\mytextsc{accomp} \string_ \mytextsc{pfv}}  
 ¶ \textcolor{darkblue}{\textbf{\ipa{ʐwæ˧ ɖʐæ˧}}} \textcolor{PineGreen}{\selectlanguage{french}monter à cheval} \zh{骑马}  
 ¶ \textcolor{darkblue}{\textbf{\ipa{ɖʐæ˧-tʰɑ˧˥!}}} \textcolor{PineGreen}{\selectlanguage{french}On peut le monter!} \zh{可以骑的!}  

\lhead{\firstmark}
\rhead{\botmark}

\subsection{\hspace{-0.5cm} {\Large \textcolor{darkblue}{\textbf{\ipa{ɖʐæ˧qʰæ˥\$}}}}\hspace{0.5cm}[\kern2pt{\textcolor{darkblue}{\textbf{\ipa{ɖʐæ˩qʰæ˩˥}}}}\kern2pt]} \hypertarget{d`z`\{\string_Mq\string_h\{\string_T\$1}{}
\markboth{\textcolor{darkblue}{\textbf{\ipa{ɖʐæ˧qʰæ˥\$}}}}{}
\textcolor{teal}{\mytextsc{nom}} \hspace{4pt} Ton~: H\$.
\textcolor{PineGreen}{\selectlanguage{french}Boue.} \zh{泥巴。}  ¶ \textcolor{darkblue}{\textbf{\ipa{ɖʐæ˧qʰæ˧ ʐæ˥(-ze˩)}}} \textcolor{PineGreen}{\selectlanguage{french}De la boue s'est formée; il y a de la boue, c'est tout boueux. (Littéralement: “de la boue s'est mélangée”.)} \zh{有泥巴了。}  
 ¶ \textcolor{darkblue}{\textbf{\ipa{[F5] ɖʐæ˧qʰæ˧ ʐæ˧\textasciitilde{}ʐæ˥}}} \textcolor{PineGreen}{\selectlanguage{french}De la boue s'est formée; il y a de la boue, c'est tout boueux. (Littéralement: “de la boue s'est mélangée”.)} \zh{有泥巴了}  

\lhead{\firstmark}
\rhead{\botmark}

\subsection{\hspace{-0.5cm} {\Large \textcolor{darkblue}{\textbf{\ipa{ɖʐæ˩\textsubscript{a}}}}}\hspace{0.5cm}[\kern2pt{\textcolor{darkblue}{\textbf{\ipa{ɖʐæ˥}}}}\kern2pt]} \hypertarget{d`z`\{\string_Ba1}{}
\markboth{\textcolor{darkblue}{\textbf{\ipa{ɖʐæ˩\textsubscript{a}}}}}{}
\textcolor{teal}{\mytextsc{verbe}} \hspace{4pt} Ton~: L\textsubscript{a}.
\textcolor{PineGreen}{\selectlanguage{french}Fondre.} \zh{融化。}  ¶ \textcolor{darkblue}{\textbf{\ipa{mɤ˧ | le˧-ɖʐæ˩-ze˩}}} \textcolor{PineGreen}{\selectlanguage{french}la graisse a fondu (ex.: du saindoux qui fond dans un chaudron)} \zh{油融化了。}  
 ¶ \textcolor{darkblue}{\textbf{\ipa{dʑi˩pʰæ˩˥ | le˧-ɖʐæ˩-ze˩}}} \textcolor{PineGreen}{\selectlanguage{french}La glace a fondu.} \zh{冰融化了。}  

\lhead{\firstmark}
\rhead{\botmark}

\subsection{\hspace{-0.5cm} {\Large \textcolor{darkblue}{\textbf{\ipa{ɖʐæ˩bv˩}}}}\hspace{0.5cm}[\kern2pt{\textcolor{darkblue}{\textbf{\ipa{ɖʐæ˧bv˧}}}}\kern2pt]} \hypertarget{d`z`\{\string_Bbv\string_B1}{}
\markboth{\textcolor{darkblue}{\textbf{\ipa{ɖʐæ˩bv˩}}}}{}
\textcolor{teal}{\mytextsc{nom}} \hspace{4pt} Ton~: L.
\textcolor{PineGreen}{\selectlanguage{french}Sorcier.} \zh{法师。}  ¶ \textcolor{darkblue}{\textbf{\ipa{ə˧pʰv˧-ɖʐæ˩bv˩}}} \textcolor{PineGreen}{\selectlanguage{french}'Grand-père sorcier': terme d'adresse respectueux pour un sorcier d'âge avancé, ou considéré comme ayant des pouvoirs considérables} \zh{‘法师爷爷’:对年龄高(或被认为本事很大)的法师的尊重称呼}  
 ¶ \textcolor{darkblue}{\textbf{\ipa{ə˧v˧-ɖʐæ˥bv˩}}} \textcolor{PineGreen}{\selectlanguage{french}'Oncle sorcier': terme d'adresse respectueux pour un sorcier} \zh{‘法师舅舅’:对法师的尊重称呼}  
 \zh{量词}: \textcolor{darkblue}{\textbf{\ipa{v̩˧}}}  \mytextsc{clf}~: \textcolor{darkblue}{\textbf{\ipa{v̩˧}}} 
\lhead{\firstmark}
\rhead{\botmark}

\subsection{\hspace{-0.5cm} {\Large \textcolor{darkblue}{\textbf{\ipa{ɖʐe˧}}}}\hspace{0.5cm}[\kern2pt{\textcolor{darkblue}{\textbf{\ipa{ɖʐe˥}}}}\kern2pt]} \hypertarget{d`z`e\string_M1}{}
\markboth{\textcolor{darkblue}{\textbf{\ipa{ɖʐe˧}}}}{}
\textcolor{teal}{\mytextsc{nom}} \hspace{4pt} Ton~: M.
\textcolor{PineGreen}{\selectlanguage{french}Argent (avoir de l'argent).} \zh{钱。} 
\lhead{\firstmark}
\rhead{\botmark}

\subsection{\hspace{-0.5cm} {\Large \textcolor{darkblue}{\textbf{\ipa{ɖʐe˧gɯ˧}}}}\hspace{0.5cm}[\kern2pt{\textcolor{darkblue}{\textbf{\ipa{ɖʐe˩gɯ˧˥}}}}\kern2pt]} \hypertarget{d`z`e\string_MgM\string_M1}{}
\markboth{\textcolor{darkblue}{\textbf{\ipa{ɖʐe˧gɯ˧}}}}{}
\textcolor{teal}{\mytextsc{nom}} \hspace{4pt} Ton~: M.
\textcolor{PineGreen}{\selectlanguage{french}Yongsheng (nom de comté).} \zh{永胜(地名)。}  ¶ \textcolor{darkblue}{\textbf{\ipa{ɖʐe˧gɯ˧-to˩mi˩}}} \textcolor{PineGreen}{\selectlanguage{french}une haute montagne située à Yongsheng} \zh{永胜的一座高山}  
 ¶ \textcolor{darkblue}{\textbf{\ipa{ɖʐe˧gɯ˧-hæ˧}}} \textcolor{PineGreen}{\selectlanguage{french}Chinois de Yongsheng (note: le comté de Yongsheng était peuplé majoritairement de Chinois (Han).)} \zh{永胜汉族}  
 ¶ \textcolor{darkblue}{\textbf{\ipa{ɖʐe˧gɯ˧-dʑo˧, | hæ˧-ʂo˧\textasciitilde{}ʂo˩!}}} \textcolor{PineGreen}{\selectlanguage{french}A Yongsheng, il y a plein de Chinois (Han)!} \zh{永胜,汉族群多!}  

\lhead{\firstmark}
\rhead{\botmark}

\subsection{\hspace{-0.5cm} {\Large \textcolor{darkblue}{\textbf{\ipa{ɖʐe˧ʁwɤ˧}}}}\hspace{0.5cm}[\kern2pt{\textcolor{darkblue}{\textbf{\ipa{xxxx non-correspondance entre le nombre de morphèmes et le nombre de tons de morphèmes}}}}\kern2pt]} \hypertarget{d`z`e\string_MRw7\string_M1}{}
\markboth{\textcolor{darkblue}{\textbf{\ipa{ɖʐe˧ʁwɤ˧}}}}{}
\textcolor{teal}{\mytextsc{nom}} \hspace{4pt} Ton~: M.
\textcolor{PineGreen}{\selectlanguage{french}Argent (monnaie); richesse.} \zh{钱。} 
\lhead{\firstmark}
\rhead{\botmark}

\subsection{\hspace{-0.5cm} {\Large \textcolor{darkblue}{\textbf{\ipa{ɖʐɤ˧qʰwɤ˧}}}}\hspace{0.5cm}[\kern2pt{\textcolor{darkblue}{\textbf{\ipa{ɖʐɤ˩qʰwɤ˥}}}}\kern2pt]} \hypertarget{d`z`7\string_Mq\string_hw7\string_M1}{}
\markboth{\textcolor{darkblue}{\textbf{\ipa{ɖʐɤ˧qʰwɤ˧}}}}{}
\textcolor{teal}{\mytextsc{nom}} \hspace{4pt} Ton~: M.
\textcolor{PineGreen}{\selectlanguage{french}Rhume.} \zh{感冒。}  ¶ \textcolor{darkblue}{\textbf{\ipa{ɖʐɤ˧qʰwɤ˧ go˩}}} \textcolor{PineGreen}{\selectlanguage{french}avoir un rhume, être enrhumé} \zh{感冒}  
 ¶ \textcolor{darkblue}{\textbf{\ipa{ɖʐɤ˧qʰwɤ˧ mɤ˧-go˩}}} \textcolor{PineGreen}{\selectlanguage{french}...n'est pas enrhumé} \zh{没感冒}  
 \zh{量词}: \textcolor{darkblue}{\textbf{\ipa{ʂɯ˩}}}  \mytextsc{clf}~: \textcolor{darkblue}{\textbf{\ipa{ʂɯ˩}}} \textcolor{PineGreen}{\selectlanguage{french}fois}
\lhead{\firstmark}
\rhead{\botmark}

\subsection{\hspace{-0.5cm} {\Large \textcolor{darkblue}{\textbf{\ipa{ɖʐɤ˧qʰwɤ˧ʈʂe\#˥}}}}\hspace{0.5cm}[\kern2pt{\textcolor{darkblue}{\textbf{\ipa{ɖʐɤ˧qʰwɤ˧ʈʂe˧}}}}\kern2pt]} \hypertarget{d`z`7\string_Mq\string_hw7\string_Mt`s`e\#\string_T1}{}
\markboth{\textcolor{darkblue}{\textbf{\ipa{ɖʐɤ˧qʰwɤ˧ʈʂe\#˥}}}}{}
\textcolor{teal}{\mytextsc{nom}} \hspace{4pt} Ton~: \#H.
\textcolor{PineGreen}{\selectlanguage{french}Poinçon, alène.} \zh{锥、锥子。}  \zh{量词}: \textcolor{darkblue}{\textbf{\ipa{ɭɯ˧}}}  \mytextsc{clf}~: \textcolor{darkblue}{\textbf{\ipa{ɭɯ˧}}} 
\lhead{\firstmark}
\rhead{\botmark}

\subsection{\hspace{-0.5cm} {\Large \textcolor{darkblue}{\textbf{\ipa{ɖʐɤ˩}}}}\hspace{0.5cm}[\kern2pt{\textcolor{darkblue}{\textbf{\ipa{ɖʐɤ˥}}}}\kern2pt]} \hypertarget{d`z`7\string_B1}{}
\markboth{\textcolor{darkblue}{\textbf{\ipa{ɖʐɤ˩}}}}{}
\textcolor{teal}{\mytextsc{nom}} \hspace{4pt} Ton~: L.
\ding{202} \textcolor{PineGreen}{\selectlanguage{french}Échelle.} \zh{梯子。}  ¶ \textcolor{darkblue}{\textbf{\ipa{ɖʐɤ˩ do˧}}} \textcolor{PineGreen}{\selectlanguage{french}gravir une échelle} \zh{爬上一个梯子}  
 ¶ \textcolor{darkblue}{\textbf{\ipa{ɖʐɤ˧ | gɤ˩-do˧}}} \textcolor{PineGreen}{\selectlanguage{french}même sens, avec ajout d'un directionnel: gravir une échelle} \zh{爬上一个梯子}  
 \zh{量词}: \textcolor{darkblue}{\textbf{\ipa{pɤ˩}}} \ding{203} \textcolor{PineGreen}{\selectlanguage{french}Escalier (en bois, sauf indication contraire).} \zh{楼梯。}  ¶ \textcolor{darkblue}{\textbf{\ipa{lv̩˧mi˧-ɖʐɤ˩ (+ɲi˩)}}} \textcolor{PineGreen}{\selectlanguage{french}escalier en pierre} \zh{石头楼梯}  
 \mytextsc{clf}~: \textcolor{darkblue}{\textbf{\ipa{pɤ˩}}} 
\lhead{\firstmark}
\rhead{\botmark}

\subsection{\hspace{-0.5cm} {\Large \textcolor{darkblue}{\textbf{\ipa{ɖʐɤ˩ɖwæ˩}}}}\hspace{0.5cm}[\kern2pt{\textcolor{darkblue}{\textbf{\ipa{ɖʐɤ˧ɖwæ˩}}}}\kern2pt]} \hypertarget{d`z`7\string_Bd`w\{\string_B1}{}
\markboth{\textcolor{darkblue}{\textbf{\ipa{ɖʐɤ˩ɖwæ˩}}}}{}
\textcolor{teal}{\mytextsc{nom}} \hspace{4pt} Ton~: L.
\textcolor{PineGreen}{\selectlanguage{french}Marche d'escalier.} \zh{台阶。}  ¶ \textcolor{darkblue}{\textbf{\ipa{lv̩˧mi˧-ɖʐɤ˩ɖwæ˩}}} \textcolor{PineGreen}{\selectlanguage{french}marche en pierre} \zh{石头台阶}  
 \zh{量词}: \textcolor{darkblue}{\textbf{\ipa{ɖwæ˥}}}  \mytextsc{clf}~: \textcolor{darkblue}{\textbf{\ipa{ɖwæ˥}}} 
\lhead{\firstmark}
\rhead{\botmark}

\subsection{\hspace{-0.5cm} {\Large \textcolor{darkblue}{\textbf{\ipa{ɖʐɤ˩kɤ˥\$}}}}\hspace{0.5cm}[\kern2pt{\textcolor{darkblue}{\textbf{\ipa{ɖʐɤ˩kɤ˩˥}}}}\kern2pt]} \hypertarget{d`z`7\string_Bk7\string_T\$1}{}
\markboth{\textcolor{darkblue}{\textbf{\ipa{ɖʐɤ˩kɤ˥\$}}}}{}
\textcolor{teal}{\mytextsc{nom}} \hspace{4pt} Ton~: LM+H\$.
\textcolor{PineGreen}{\selectlanguage{french}Nom de clan/famille étendue. Deux familles portent ce nom à Yongning.} \zh{一个姓。这个姓,永宁有两家。}  ¶ \textcolor{darkblue}{\textbf{\ipa{ɖʐɤ˩kɤ˧=ɻ̍˥\$}}} \textcolor{PineGreen}{\selectlanguage{french}le clan \textcolor{darkblue}{\textbf{\ipa{/ɖʐɤ˩kɤ˥\$/}}}} \zh{\textcolor{darkblue}{\textbf{\ipa{/ɖʐɤ˩kɤ˥\$/}}}家族}  

\lhead{\firstmark}
\rhead{\botmark}

\subsection{\hspace{-0.5cm} {\Large \textcolor{darkblue}{\textbf{\ipa{ɖʐɤ˧˥}}} \textsubscript{1}}\hspace{0.5cm}[\kern2pt{\textcolor{darkblue}{\textbf{\ipa{ɖʐɤ˩˥}}}}\kern2pt]} \hypertarget{d`z`7\string_M\string_T1}{}
\markboth{\textcolor{darkblue}{\textbf{\ipa{ɖʐɤ˧˥}}} \textsubscript{1}}{}
\textcolor{teal}{\mytextsc{verbe}} \hspace{4pt} Ton~: MH.
\ding{202} \textcolor{PineGreen}{\selectlanguage{french}Cueillir (des fruits, des légumes); arracher (des mauvaises herbes).} \zh{摘(果子、蔬菜)、扯(草)。}  ¶ \textcolor{darkblue}{\textbf{\ipa{le˧-ɖʐɤ˧-po˥-jo˩!}}} \textcolor{PineGreen}{\selectlanguage{french}cueille-m'en qq-unes!/cueilles-en et passe-les(-nous) par ici!} \zh{(你)去给摘(一些)过来吧!}  
 ¶ \textcolor{darkblue}{\textbf{\ipa{v̩˩tsʰɤ˧ ɖʐɤ˥}}} \textcolor{PineGreen}{\selectlanguage{french}cueillir des légumes} \zh{摘蔬菜}  
 ¶ \textcolor{darkblue}{\textbf{\ipa{le˧-ɖʐɤ˧˥, | mv̩˩-tɕo˧ kwɤ˩}}} \textcolor{PineGreen}{\selectlanguage{french}arracher et jeter (les mauvaises herbes)} \zh{扯(荒草),扔掉}  
\ding{203} \textcolor{PineGreen}{\selectlanguage{french}Déchirer, couper (fil); briser; broyer; détruire (une maison).} \zh{拆(线),拔,捣碎。}  ¶ \textcolor{darkblue}{\textbf{\ipa{le˧-ɖʐɤ˩\textasciitilde{}ɖʐɤ˩}}} \textcolor{PineGreen}{\selectlanguage{french}\mytextsc{red}} \zh{\mytextsc{重叠:拆拆}}  
 ¶ \textcolor{darkblue}{\textbf{\ipa{ʑi˧qʰwɤ˧ ɖʐɤ˧˥}}} \textcolor{PineGreen}{\selectlanguage{french}détruire une maison, démolir une maison} \zh{拆房子}  
 ¶ \textcolor{darkblue}{\textbf{\ipa{le˧-ɖʐɤ˧˥ | ɲi˧-gi˧ gv̩˧}}} \textcolor{PineGreen}{\selectlanguage{french}déchirer en deux morceaux} \zh{拆成两半}  

\lhead{\firstmark}
\rhead{\botmark}

\subsection{\hspace{-0.5cm} {\Large \textcolor{darkblue}{\textbf{\ipa{ɖʐɤ˧˥}}} \textsubscript{2}}\hspace{0.5cm}[\kern2pt{\textcolor{darkblue}{\textbf{\ipa{ɖʐɤ˧˥}}}}\kern2pt]} \hypertarget{d`z`7\string_M\string_T2}{}
\markboth{\textcolor{darkblue}{\textbf{\ipa{ɖʐɤ˧˥}}} \textsubscript{2}}{}
\textcolor{teal}{\mytextsc{verbe}} \hspace{4pt} Ton~: MH.
\textcolor{PineGreen}{\selectlanguage{french}Déployer, ouvrir en soutenant; ex.: déployer la tente.} \zh{撑开(帐篷)。}  ¶ \textcolor{darkblue}{\textbf{\ipa{le˧-ɖʐɤ˩\textasciitilde{}ɖʐɤ˩}}} \textcolor{PineGreen}{\selectlanguage{french}\mytextsc{red}} \zh{\mytextsc{重叠}}  

\lhead{\firstmark}
\rhead{\botmark}

\subsection{\hspace{-0.5cm} {\Large \textcolor{darkblue}{\textbf{\ipa{ɖʐo˥}}}}\hspace{0.5cm}[\kern2pt{\textcolor{darkblue}{\textbf{\ipa{ɖʐo˥}}}}\kern2pt]} \hypertarget{d`z`o\string_T1}{}
\markboth{\textcolor{darkblue}{\textbf{\ipa{ɖʐo˥}}}}{}
\textcolor{teal}{\mytextsc{nom}} \hspace{4pt} Ton~: \#H.
\textcolor{PineGreen}{\selectlanguage{french}Pièce de charpente carrée (côté: environ 18 cm), dans les parties du bâtiment qui n'ont pas de piliers: \textcolor{darkblue}{\textbf{\ipa{/gæ˩pʰæ˧/}}}, \textcolor{darkblue}{\textbf{\ipa{/mv̩˩pʰæ˧/}}}. Elles supportent la charpente. (M18 pense que ce terme désigne toute la structure du bâtiment.).} \zh{大梁。}  \zh{量词}: \textcolor{darkblue}{\textbf{\ipa{ɖʐo˥}}}  \mytextsc{clf}~: \textcolor{darkblue}{\textbf{\ipa{ɖʐo˥}}} 
\lhead{\firstmark}
\rhead{\botmark}

\subsection{\hspace{-0.5cm} {\Large \textcolor{darkblue}{\textbf{\ipa{ɖʐo˥}}}}\hspace{0.5cm}[\kern2pt{\textcolor{darkblue}{\textbf{\ipa{ɖʐo˥}}}}\kern2pt]} \hypertarget{d`z`o\string_T1}{}
\markboth{\textcolor{darkblue}{\textbf{\ipa{ɖʐo˥}}}}{}
\textcolor{teal}{\mytextsc{adjectif}} \hspace{4pt} Ton~: H.
\textcolor{PineGreen}{\selectlanguage{french}Froid.} \zh{冷(天气……)。} 
\lhead{\firstmark}
\rhead{\botmark}

\subsection{\hspace{-0.5cm} {\Large \textcolor{darkblue}{\textbf{\ipa{ɖʐo˥\textsubscript{a}}}}}\hspace{0.5cm}[\kern2pt{\textcolor{darkblue}{\textbf{\ipa{ɖʐo˩˥}}}}\kern2pt]} \hypertarget{d`z`o\string_Ta1}{}
\markboth{\textcolor{darkblue}{\textbf{\ipa{ɖʐo˥\textsubscript{a}}}}}{}
\textcolor{teal}{\mytextsc{classificateur}} \hspace{4pt} Ton~: H\textsubscript{a}.
\textcolor{PineGreen}{\selectlanguage{french}Classificateur des poutres.} \zh{量词:梁(一根)。} 
\lhead{\firstmark}
\rhead{\botmark}

\subsection{\hspace{-0.5cm} {\Large \textcolor{darkblue}{\textbf{\ipa{ɖʐo˩\textsubscript{b}}}}}\hspace{0.5cm}[\kern2pt{\textcolor{darkblue}{\textbf{\ipa{ɖʐo˩˥}}}}\kern2pt]} \hypertarget{d`z`o\string_Bb1}{}
\markboth{\textcolor{darkblue}{\textbf{\ipa{ɖʐo˩\textsubscript{b}}}}}{}
\textcolor{teal}{\mytextsc{verbe}} \hspace{4pt} Ton~: L\textsubscript{b}.
\textcolor{PineGreen}{\selectlanguage{french}Écraser (au moulin; ou avec les dents).} \zh{弄碎(用牙齿、手磨)。}  ¶ \textcolor{darkblue}{\textbf{\ipa{ʈʂo˧ɭɯ˧ ɖʐo˧˥}}} \textcolor{PineGreen}{\selectlanguage{french}écraser avec un moulin (/ʈʂu˧ɭɯ\#˥/: moulin)} \zh{用手磨弄碎}  
 ¶ \textcolor{darkblue}{\textbf{\ipa{ɖɯ˧-kʰwɤ˧ ɖʐo˧˥}}} \textcolor{PineGreen}{\selectlanguage{french}écraser un morceau (de quelque chose)} \zh{弄碎一块}  
 ¶ \textcolor{darkblue}{\textbf{\ipa{ɖɯ˧-mɤ˩ ɖʐo˩}}} \textcolor{PineGreen}{\selectlanguage{french}écraser un peu (de quelque chose)} \zh{弄碎一点(东西)}  

\lhead{\firstmark}
\rhead{\botmark}

\subsection{\hspace{-0.5cm} {\Large \textcolor{darkblue}{\textbf{\ipa{ɖʐɯ˥}}}}\hspace{0.5cm}[\kern2pt{\textcolor{darkblue}{\textbf{\ipa{ɖʐɯ˧˥}}}}\kern2pt]} \hypertarget{d`z`M\string_T1}{}
\markboth{\textcolor{darkblue}{\textbf{\ipa{ɖʐɯ˥}}}}{}
\textcolor{teal}{\mytextsc{nom}} \hspace{4pt} Ton~: \#H.
\textcolor{PineGreen}{\selectlanguage{french}Marché.} \zh{集市(圩场,街子)。}  \zh{量词}: \textcolor{darkblue}{\textbf{\ipa{ɖʐɯ˩}}}  \mytextsc{clf}~: \textcolor{darkblue}{\textbf{\ipa{ɖʐɯ˩}}} 
\lhead{\firstmark}
\rhead{\botmark}

\subsection{\hspace{-0.5cm} {\Large \textcolor{darkblue}{\textbf{\ipa{ɖʐɯ˥\textsubscript{a}}}}}\hspace{0.5cm}[\kern2pt{\textcolor{darkblue}{\textbf{\ipa{ɖʐɯ˩˥}}}}\kern2pt]} \hypertarget{d`z`M\string_Ta1}{}
\markboth{\textcolor{darkblue}{\textbf{\ipa{ɖʐɯ˥\textsubscript{a}}}}}{}
\textcolor{teal}{\mytextsc{classificateur}} \hspace{4pt} Ton~: H\textsubscript{a}.
\textcolor{PineGreen}{\selectlanguage{french}Auto-classificateur des marchés/villes.} \zh{量词:市场(一个),城市(一个)。}  ¶ \textcolor{darkblue}{\textbf{\ipa{ɖʐɯ˧ | ɖɯ˧-ɖʐɯ˥}}} \textcolor{PineGreen}{\selectlanguage{french}un marché} \zh{一个市场}  

\lhead{\firstmark}
\rhead{\botmark}

\subsection{\hspace{-0.5cm} {\Large \textcolor{darkblue}{\textbf{\ipa{ɖʐɯ˥kʰɤ˩}}}}\hspace{0.5cm}[\kern2pt{\textcolor{darkblue}{\textbf{\ipa{ɖʐɯ˩kʰɤ˥}}}}\kern2pt]} \hypertarget{d`z`M\string_Tk\string_h7\string_B1}{}
\markboth{\textcolor{darkblue}{\textbf{\ipa{ɖʐɯ˥kʰɤ˩}}}}{}
\textcolor{teal}{\mytextsc{nom}} \hspace{4pt} Ton~: .
\textcolor{PineGreen}{\selectlanguage{french}Un moment.} \zh{(一)会儿。}  ¶ \textcolor{darkblue}{\textbf{\ipa{ɖɯ˧-ɖʐɯ˥kʰɤ˩}}} \textcolor{PineGreen}{\selectlanguage{french}un moment} \zh{一会儿}  

\lhead{\firstmark}
\rhead{\botmark}

\subsection{\hspace{-0.5cm} {\Large \textcolor{darkblue}{\textbf{\ipa{ɖʐɯ˧qo˩}}}}\hspace{0.5cm}[\kern2pt{\textcolor{darkblue}{\textbf{\ipa{ɖʐɯ˧qo˩}}}}\kern2pt]} \hypertarget{d`z`M\string_Mqo\string_B1}{}
\markboth{\textcolor{darkblue}{\textbf{\ipa{ɖʐɯ˧qo˩}}}}{}
\textcolor{teal}{\mytextsc{adverbe}} \hspace{4pt} Ton~: L\#.
\textcolor{PineGreen}{\selectlanguage{french}En ville.} \zh{在城里、在市里。}  ¶ \textcolor{darkblue}{\textbf{\ipa{ɖʐɯ˧qo˩ kʰi˩}}} \textcolor{PineGreen}{\selectlanguage{french}aller dans la rue, faire un tour en ville} \zh{上街}  

\lhead{\firstmark}
\rhead{\botmark}

\subsection{\hspace{-0.5cm} {\Large \textcolor{darkblue}{\textbf{\ipa{ɖʐɯ˧ʂɯ˥}}}}\hspace{0.5cm}[\kern2pt{\textcolor{darkblue}{\textbf{\ipa{ɖʐɯ˧ʂɯ˥}}}}\kern2pt]} \hypertarget{d`z`M\string_Ms`M\string_T1}{}
\markboth{\textcolor{darkblue}{\textbf{\ipa{ɖʐɯ˧ʂɯ˥}}}}{}
\textcolor{teal}{\mytextsc{nom}} \hspace{4pt} Ton~: H\#.
\textcolor{PineGreen}{\selectlanguage{french}Baguettes.} \zh{筷子。}  \zh{量词}: \textcolor{darkblue}{\textbf{\ipa{dzi˧}}}  \mytextsc{clf}~: \textcolor{darkblue}{\textbf{\ipa{dzi˧}}} 
\lhead{\firstmark}
\rhead{\botmark}

\subsection{\hspace{-0.5cm} {\Large \textcolor{darkblue}{\textbf{\ipa{ɖʐɯ˧ʈʂɯ˥}}}}\hspace{0.5cm}[\kern2pt{\textcolor{darkblue}{\textbf{\ipa{ɖʐɯ˧ʈʂɯ˥}}}}\kern2pt]} \hypertarget{d`z`M\string_Mt`s`M\string_T1}{}
\markboth{\textcolor{darkblue}{\textbf{\ipa{ɖʐɯ˧ʈʂɯ˥}}}}{}
\textcolor{teal}{\mytextsc{nom}} \hspace{4pt} Ton~: H\#.
\textcolor{PineGreen}{\selectlanguage{french}Vannerie.} \zh{筛子。}  \zh{量词}: \textcolor{darkblue}{\textbf{\ipa{nɑ˧}}}  \mytextsc{clf}~: \textcolor{darkblue}{\textbf{\ipa{nɑ˧}}} 
\lhead{\firstmark}
\rhead{\botmark}

\subsection{\hspace{-0.5cm} {\Large \textcolor{darkblue}{\textbf{\ipa{ɖʐɯ˩\textasciitilde{}ɖʐɯ˧˥}}}}\hspace{0.5cm}[\kern2pt{\textcolor{darkblue}{\textbf{\ipa{ɖʐɯ˧ɖʐɯ˧˥}}}}\kern2pt]} \hypertarget{d`z`M\string_B~d`z`M\string_M\string_T1}{}
\markboth{\textcolor{darkblue}{\textbf{\ipa{ɖʐɯ˩\textasciitilde{}ɖʐɯ˧˥}}}}{}
\textcolor{teal}{\mytextsc{verbe}} \hspace{4pt} Ton~: MH.
\textcolor{PineGreen}{\selectlanguage{french}Secouer (la tête).} \zh{摇(头)。}  ¶ \textcolor{darkblue}{\textbf{\ipa{ʁo˧ ɖʐɯ˥\textasciitilde{}ɖʐɯ˩}}} \textcolor{PineGreen}{\selectlanguage{french}secouer la tête} \zh{摇头}  
 ¶ \textcolor{darkblue}{\textbf{\ipa{ʁo˧ | le˧-ɖʐɯ˩\textasciitilde{}ɖʐɯ˩-ze˩}}} \textcolor{PineGreen}{\selectlanguage{french}a secoué la tête} \zh{摇了头}  

\lhead{\firstmark}
\rhead{\botmark}

\subsection{\hspace{-0.5cm} {\Large \textcolor{darkblue}{\textbf{\ipa{ɖʐɯ˩kʰɤ˥}}}}\hspace{0.5cm}[\kern2pt{\textcolor{darkblue}{\textbf{\ipa{ɖʐɯ˧kʰɤ˥}}}}\kern2pt]} \hypertarget{d`z`M\string_Bk\string_h7\string_T1}{}
\markboth{\textcolor{darkblue}{\textbf{\ipa{ɖʐɯ˩kʰɤ˥}}}}{}
\textcolor{teal}{\mytextsc{nom}} \hspace{4pt} Ton~: LH.
\textit{De:} \textbf{ɖʐɯ˩a} \textcolor{PineGreen}{\selectlanguage{french}Époque, ère, état de la société.} \zh{时代。}  \zh{量词}: \textcolor{darkblue}{\textbf{\ipa{ɖʐɯ˩}}}  \mytextsc{clf}~: \textcolor{darkblue}{\textbf{\ipa{ɖʐɯ˩}}} 
\lhead{\firstmark}
\rhead{\botmark}

\subsection{\hspace{-0.5cm} {\Large \textcolor{darkblue}{\textbf{\ipa{ɖʐɯ˩tso\#˥}}}}\hspace{0.5cm}[\kern2pt{\textcolor{darkblue}{\textbf{\ipa{ɖʐɯ˩tso˥}}}}\kern2pt]} \hypertarget{d`z`M\string_Btso\#\string_T1}{}
\markboth{\textcolor{darkblue}{\textbf{\ipa{ɖʐɯ˩tso\#˥}}}}{}
\textcolor{teal}{\mytextsc{nom}} \hspace{4pt} Ton~: LM+\#H.
\textcolor{PineGreen}{\selectlanguage{french}Règles de conduite sociale, règles régissant la société (politique, société).} \zh{社会规矩。}  ¶ \textcolor{darkblue}{\textbf{\ipa{ɖʐɯ˩tso˥ | hĩ˧-qo˩-ɳɯ˩ | le˧-tsʰɯ˩-ɲi˩-tsɯ˩-mæ˩!}}} \textcolor{PineGreen}{\selectlanguage{french}ces morales (les contes, les proverbes...) ça provient des hommes! / la morale (des histoires, ...) c'est le fruit de l'expérience des hommes!} \zh{社会规矩,是通过人类的经验形成的! / 社会规矩,是人(按一代代的经验)创造的!}  
 \zh{量词}: \textcolor{darkblue}{\textbf{\ipa{kʰwɤ˥}}}  \mytextsc{clf}~: \textcolor{darkblue}{\textbf{\ipa{kʰwɤ˥}}} 
\lhead{\firstmark}
\rhead{\botmark}

\subsection{\hspace{-0.5cm} {\Large \textcolor{darkblue}{\textbf{\ipa{ɖʐv̩˧}}} \textsubscript{1}}\hspace{0.5cm}[\kern2pt{\textcolor{darkblue}{\textbf{\ipa{ɖʐv̩˥}}}}\kern2pt]} \hypertarget{d`z`v\string_=\string_M1}{}
\markboth{\textcolor{darkblue}{\textbf{\ipa{ɖʐv̩˧}}} \textsubscript{1}}{}
\textcolor{teal}{\mytextsc{verbe}} \hspace{4pt} Ton~: M intrans.
\textcolor{PineGreen}{\selectlanguage{french}Brûler; prendre feu.} \zh{燃烧。}  ¶ \textcolor{darkblue}{\textbf{\ipa{mv̩˧ ɖʐv̩˧-ze˩!}}} \textcolor{PineGreen}{\selectlanguage{french}Ca a pris feu! / Au feu!} \zh{着火了!}  
 ¶ \textcolor{darkblue}{\textbf{\ipa{mv̩˧ le˧-ɖʐv̩˧-ze˧!}}} \textcolor{PineGreen}{\selectlanguage{french}Le feu a pris!} \zh{开始着火了!}  
 ¶ \textcolor{darkblue}{\textbf{\ipa{tʰi˧-ɖʐv̩˧-dʑo˧!}}} \textcolor{PineGreen}{\selectlanguage{french}C'est en train de brûler! / Le feu est en train de brûler!} \zh{火在燃烧!}  

\lhead{\firstmark}
\rhead{\botmark}

\subsection{\hspace{-0.5cm} {\Large \textcolor{darkblue}{\textbf{\ipa{ɖʐv̩˧}}} \textsubscript{2}}\hspace{0.5cm}[\kern2pt{\textcolor{darkblue}{\textbf{\ipa{ɖʐv̩˥}}}}\kern2pt]} \hypertarget{d`z`v\string_=\string_M2}{}
\markboth{\textcolor{darkblue}{\textbf{\ipa{ɖʐv̩˧}}} \textsubscript{2}}{}
\textcolor{teal}{\mytextsc{nom}} \hspace{4pt} Ton~: M.
\textcolor{PineGreen}{\selectlanguage{french}Ami/amie, compagnon/compagne.} \zh{朋友、伙伴、伴侣。}  ¶ \textcolor{darkblue}{\textbf{\ipa{njɤ˧ | ɖʐv̩˧ ɲi˩.}}} \textcolor{PineGreen}{\selectlanguage{french}C'est mon ami(e).} \zh{是我朋友。}  
 ¶ \textcolor{darkblue}{\textbf{\ipa{õ˧ ɖʐv̩˥, õ˩ li˩! |}}} \textcolor{PineGreen}{\selectlanguage{french}'On est influencé par ses amis!' (Littéralement: 'On observe ses amis!') Le proverbe souligne l'influence des amis, en bien ou en mal selon qu'on a ou non choisi judicieusement.} \zh{“大家都容易受朋友的影响!”(直译:“自己的朋友,自己看(=自己爱学他们的习惯)”)}  
 \zh{量词}: \textcolor{darkblue}{\textbf{\ipa{v̩˧}}}  \mytextsc{clf}~: \textcolor{darkblue}{\textbf{\ipa{v̩˧}}} 
\lhead{\firstmark}
\rhead{\botmark}

\subsection{\hspace{-0.5cm} {\Large \textcolor{darkblue}{\textbf{\ipa{ɖʐv̩˧}}} \textsubscript{3}}\hspace{0.5cm}[\kern2pt{\textcolor{darkblue}{\textbf{\ipa{ɖʐv̩˥}}}}\kern2pt]} \hypertarget{d`z`v\string_=\string_M3}{}
\markboth{\textcolor{darkblue}{\textbf{\ipa{ɖʐv̩˧}}} \textsubscript{3}}{}
\textcolor{teal}{\mytextsc{nom}} \hspace{4pt} Ton~: M.
\textcolor{PineGreen}{\selectlanguage{french}Accident (grave).} \zh{事故,(不幸的)大事。}  ¶ \textcolor{darkblue}{\textbf{\ipa{ɖʐv̩˧ kʰɯ˧˥}}} \textcolor{PineGreen}{\selectlanguage{french}causer un accident, commettre une faute; il se passe quelque chose de grave} \zh{犯错误,出大事}  
 ¶ \textcolor{darkblue}{\textbf{\ipa{ɖʐv̩˧ kʰɯ˧-ze˥}}} \textcolor{PineGreen}{\selectlanguage{french}Comme ci-dessus, avec ajout du \mytextsc{pfv}} \zh{同上,加上\mytextsc{pfv语素}}  
 ¶ \textcolor{darkblue}{\textbf{\ipa{ɖʐv̩˧ ɖɯ˧-ɖʐv̩˧ | kʰɯ˧-ze˥!}}} \textcolor{PineGreen}{\selectlanguage{french}il est arrivé un accident!} \zh{出大事了!}  
 \zh{量词}: \textcolor{darkblue}{\textbf{\ipa{ɖʐv̩˧}}}  \mytextsc{clf}~: \textcolor{darkblue}{\textbf{\ipa{ɖʐv̩˧}}} 
\lhead{\firstmark}
\rhead{\botmark}

\subsection{\hspace{-0.5cm} {\Large \textcolor{darkblue}{\textbf{\ipa{ɖʐv̩˧}}} \textsubscript{4}}\hspace{0.5cm}[\kern2pt{\textcolor{darkblue}{\textbf{\ipa{ɖʐv̩˥}}}}\kern2pt]} \hypertarget{d`z`v\string_=\string_M4}{}
\markboth{\textcolor{darkblue}{\textbf{\ipa{ɖʐv̩˧}}} \textsubscript{4}}{}
\textcolor{teal}{\mytextsc{nom}} \hspace{4pt} Ton~: M.
\textcolor{PineGreen}{\selectlanguage{french}Rosée.} \zh{露水。} \textit{Voir~:} \hyperlink{}{\textcolor{darkblue}{\textbf{\ipa{ɖʐv̩˧qʰɑ˧}}}} 
\lhead{\firstmark}
\rhead{\botmark}

\subsection{\hspace{-0.5cm} {\Large \textcolor{darkblue}{\textbf{\ipa{ɖʐv̩˥}}}}\hspace{0.5cm}[\kern2pt{\textcolor{darkblue}{\textbf{\ipa{ɖʐv̩˥}}}}\kern2pt]} \hypertarget{d`z`v\string_=\string_T1}{}
\markboth{\textcolor{darkblue}{\textbf{\ipa{ɖʐv̩˥}}}}{}
\textcolor{teal}{\mytextsc{nom}} \hspace{4pt} Ton~: \#H.
\textcolor{PineGreen}{\selectlanguage{french}Artère.} \zh{动脉。}  \zh{量词}: \textcolor{darkblue}{\textbf{\ipa{kʰɯ˩}}}  \mytextsc{clf}~: \textcolor{darkblue}{\textbf{\ipa{kʰɯ˩}}} \textit{Voir~:} \hyperlink{}{\textcolor{darkblue}{\textbf{\ipa{ɖʐv̩˧tsi˥}}}} 
\lhead{\firstmark}
\rhead{\botmark}

\subsection{\hspace{-0.5cm} {\Large \textcolor{darkblue}{\textbf{\ipa{ɖʐv̩˥}}}}\hspace{0.5cm}[\kern2pt{\textcolor{darkblue}{\textbf{\ipa{ɖʐv̩˥}}}}\kern2pt]} \hypertarget{d`z`v\string_=\string_T1}{}
\markboth{\textcolor{darkblue}{\textbf{\ipa{ɖʐv̩˥}}}}{}
\textcolor{teal}{\mytextsc{adjectif}} \hspace{4pt} Ton~: H.
\textcolor{PineGreen}{\selectlanguage{french}Humide, mouillé.} \zh{湿。}  ¶ \textcolor{darkblue}{\textbf{\ipa{le˧-ɖʐv̩˥-ze˩}}} \textcolor{PineGreen}{\selectlanguage{french}\mytextsc{accomp} \string_ \mytextsc{pfv}} \zh{\mytextsc{accomp} \string_ \mytextsc{pfv}}  
 ¶ \textcolor{darkblue}{\textbf{\ipa{ɖʐv̩˧\textasciitilde{}ɖʐv̩˧}}} \textcolor{PineGreen}{\selectlanguage{french}\mytextsc{red}} \zh{\mytextsc{red}}  
 ¶ \textcolor{darkblue}{\textbf{\ipa{ʈʂe˧ ɖʐv̩˧-ze˩!}}} \textcolor{PineGreen}{\selectlanguage{french}la terre est mouillée!} \zh{土湿了。}  

\lhead{\firstmark}
\rhead{\botmark}

\subsection{\hspace{-0.5cm} {\Large \textcolor{darkblue}{\textbf{\ipa{ɖʐv̩˥}}}}\hspace{0.5cm}[\kern2pt{\textcolor{darkblue}{\textbf{\ipa{ɖʐv̩˥}}}}\kern2pt]} \hypertarget{d`z`v\string_=\string_T1}{}
\markboth{\textcolor{darkblue}{\textbf{\ipa{ɖʐv̩˥}}}}{}
\textcolor{teal}{\mytextsc{verbe}} \hspace{4pt} Ton~: H.
\textcolor{PineGreen}{\selectlanguage{french}Augmenter.} \zh{涨。}  ¶ \textcolor{darkblue}{\textbf{\ipa{hĩ˧ ɖʐv̩˧}}} \textcolor{PineGreen}{\selectlanguage{french}Les gens deviennent nombreux, se multiplient} \zh{人变多}  
 ¶ \textcolor{darkblue}{\textbf{\ipa{mo˧ ɖʐv̩˥}}} \textcolor{PineGreen}{\selectlanguage{french}les champignons se multiplient} \zh{菌子长得多}  

\lhead{\firstmark}
\rhead{\botmark}

\subsection{\hspace{-0.5cm} {\Large \textcolor{darkblue}{\textbf{\ipa{ɖʐv̩˩\textsubscript{a}}}} \textsubscript{1}}\hspace{0.5cm}[\kern2pt{\textcolor{darkblue}{\textbf{\ipa{ɖʐv̩˩˥}}}}\kern2pt]} \hypertarget{d`z`v\string_=\string_Ba1}{}
\markboth{\textcolor{darkblue}{\textbf{\ipa{ɖʐv̩˩\textsubscript{a}}}} \textsubscript{1}}{}
\textcolor{teal}{\mytextsc{adjectif}} \hspace{4pt} Ton~: L\textsubscript{a}.
\textcolor{PineGreen}{\selectlanguage{french}Laid, vilain.} \zh{丑陋。}  ¶ \textcolor{darkblue}{\textbf{\ipa{ɖʐv̩˩-hĩ˩˥}}} \textcolor{PineGreen}{\selectlanguage{french}\mytextsc{rel}/\mytextsc{nmlz}} \zh{丑的}  
 ¶ \textcolor{darkblue}{\textbf{\ipa{ʈʂʰɯ˧-v̩˧ | ɖwæ˧˥ | ɖʐv̩˩˥!}}} \textcolor{PineGreen}{\selectlanguage{french}celui-là/celle-là est vraiment méchant/mauvais} \zh{这个好丑!}  

\lhead{\firstmark}
\rhead{\botmark}

\subsection{\hspace{-0.5cm} {\Large \textcolor{darkblue}{\textbf{\ipa{ɖʐv̩˩\textsubscript{a}}}} \textsubscript{2}}\hspace{0.5cm}[\kern2pt{\textcolor{darkblue}{\textbf{\ipa{ɖʐv̩˩˥}}}}\kern2pt]} \hypertarget{d`z`v\string_=\string_Ba2}{}
\markboth{\textcolor{darkblue}{\textbf{\ipa{ɖʐv̩˩\textsubscript{a}}}} \textsubscript{2}}{}
\textcolor{teal}{\mytextsc{verbe}} \hspace{4pt} Ton~: L\textsubscript{a}.
\textcolor{PineGreen}{\selectlanguage{french}Décider, choisir.} \zh{决定、选择、拿主意。}  ¶ \textcolor{darkblue}{\textbf{\ipa{njɤ˧-ɳɯ˧ | ɖʐv̩˧ ʝi˧-bi˧!}}} \textcolor{PineGreen}{\selectlanguage{french}C'est moi qui vais décider!} \zh{我来决定吧!}  

\lhead{\firstmark}
\rhead{\botmark}

\subsection{\hspace{-0.5cm} {\Large \textcolor{darkblue}{\textbf{\ipa{ɖʐv̩˧\textsubscript{b}}}}}\hspace{0.5cm}[\kern2pt{\textcolor{darkblue}{\textbf{\ipa{ɖʐv̩˥}}}}\kern2pt]} \hypertarget{d`z`v\string_=\string_Mb1}{}
\markboth{\textcolor{darkblue}{\textbf{\ipa{ɖʐv̩˧\textsubscript{b}}}}}{}
\textcolor{teal}{\mytextsc{classificateur}} \hspace{4pt} Ton~: M\textsubscript{b}.
\textcolor{PineGreen}{\selectlanguage{french}Auto-classificateur des accidents.} \zh{量词:事故(一场)。} 
\lhead{\firstmark}
\rhead{\botmark}

\subsection{\hspace{-0.5cm} {\Large \textcolor{darkblue}{\textbf{\ipa{ɖʐv̩˧-nɑ˥mi˩}}}}\hspace{0.5cm}[\kern2pt{\textcolor{darkblue}{\textbf{\ipa{ɖʐv̩˧nɑ˥mi˩}}}}\kern2pt]} \hypertarget{d`z`v\string_=\string_M-nA\string_Tmi\string_B1}{}
\markboth{\textcolor{darkblue}{\textbf{\ipa{ɖʐv̩˧-nɑ˥mi˩}}}}{}
\textcolor{teal}{\mytextsc{nom}} \hspace{4pt} Ton~: \#H-.
\textcolor{PineGreen}{\selectlanguage{french}Héron: oiseau échassier, non migrateur.} \zh{鹳。}  \zh{量词}: \textcolor{darkblue}{\textbf{\ipa{mi˩}}}  \mytextsc{clf}~: \textcolor{darkblue}{\textbf{\ipa{mi˩}}} 
\lhead{\firstmark}
\rhead{\botmark}

\subsection{\hspace{-0.5cm} {\Large \textcolor{darkblue}{\textbf{\ipa{ɖʐv̩˧qʰɑ˧}}}}\hspace{0.5cm}[\kern2pt{\textcolor{darkblue}{\textbf{\ipa{ɖʐv̩˧qʰɑ˧}}}}\kern2pt]} \hypertarget{d`z`v\string_=\string_Mq\string_hA\string_M1}{}
\markboth{\textcolor{darkblue}{\textbf{\ipa{ɖʐv̩˧qʰɑ˧}}}}{}
\textcolor{teal}{\mytextsc{nom}} \hspace{4pt} Ton~: M.
\textcolor{PineGreen}{\selectlanguage{french}Rosée.} \zh{露水。} \textit{Voir~:} \hyperlink{}{\textcolor{darkblue}{\textbf{\ipa{ɖʐv̩˧}}} \textsubscript{4}} 
\lhead{\firstmark}
\rhead{\botmark}

\subsection{\hspace{-0.5cm} {\Large \textcolor{darkblue}{\textbf{\ipa{ɖʐv̩˩ti\#˥}}}}\hspace{0.5cm}[\kern2pt{\textcolor{darkblue}{\textbf{\ipa{ɖʐv̩˩ti˥}}}}\kern2pt]} \hypertarget{d`z`v\string_=\string_Bti\#\string_T1}{}
\markboth{\textcolor{darkblue}{\textbf{\ipa{ɖʐv̩˩ti\#˥}}}}{}
\textcolor{teal}{\mytextsc{nom}} \hspace{4pt} Ton~: LM+\#H.
\textcolor{PineGreen}{\selectlanguage{french}Lance.} \zh{矛。} 
\lhead{\firstmark}
\rhead{\botmark}

\subsection{\hspace{-0.5cm} {\Large \textcolor{darkblue}{\textbf{\ipa{ɖʐv̩˧tsi˥}}}}\hspace{0.5cm}[\kern2pt{\textcolor{darkblue}{\textbf{\ipa{ɖʐv̩˧tsi˥}}}}\kern2pt]} \hypertarget{d`z`v\string_=\string_Mtsi\string_T1}{}
\markboth{\textcolor{darkblue}{\textbf{\ipa{ɖʐv̩˧tsi˥}}}}{}
\textcolor{teal}{\mytextsc{nom}} \hspace{4pt} Ton~: H\#.
\ding{202} \textcolor{PineGreen}{\selectlanguage{french}Artère (du corps humain).} \zh{动脉。}  \zh{量词}: \textcolor{darkblue}{\textbf{\ipa{kʰɯ˩}}} \ding{203} \textcolor{PineGreen}{\selectlanguage{french}Tige (d'une plante).} \zh{茎。}  \mytextsc{clf}~: \textcolor{darkblue}{\textbf{\ipa{kʰɯ˩}}} \textit{Voir~:} \textcolor{darkblue}{\textbf{\ipa{ɖʐv̩˥}}} 
\lhead{\firstmark}
\rhead{\botmark}

\subsection{\hspace{-0.5cm} {\Large \textcolor{darkblue}{\textbf{\ipa{ɖʐv̩˧ʐv̩˧-ɖʐv̩˧mi\#˥}}}}\hspace{0.5cm}[\kern2pt{\textcolor{darkblue}{\textbf{\ipa{xxxx non-correspondance entre le nombre de morphèmes et le nombre de tons de morphèmes}}}}\kern2pt]} \hypertarget{d`z`v\string_=\string_Mz`v\string_=\string_M-d`z`v\string_=\string_Mmi\#\string_T1}{}
\markboth{\textcolor{darkblue}{\textbf{\ipa{ɖʐv̩˧ʐv̩˧-ɖʐv̩˧mi\#˥}}}}{}
\textcolor{teal}{\mytextsc{nom}} \hspace{4pt} Ton~: \#H.
\textcolor{PineGreen}{\selectlanguage{french}Ami(e).} \zh{朋友、伙伴、伴侣。} 
\lhead{\firstmark}
\rhead{\botmark}

\subsection{\hspace{-0.5cm} {\Large \textcolor{darkblue}{\textbf{\ipa{ɖʐwæ˥}}}}\hspace{0.5cm}[\kern2pt{\textcolor{darkblue}{\textbf{\ipa{ɖʐwæ˥}}}}\kern2pt]} \hypertarget{d`z`w\{\string_T1}{}
\markboth{\textcolor{darkblue}{\textbf{\ipa{ɖʐwæ˥}}}}{}
\textcolor{teal}{\mytextsc{nom}} \hspace{4pt} Ton~: \#H.
\textcolor{PineGreen}{\selectlanguage{french}Petite houe (plus petite que \textcolor{darkblue}{\textbf{\ipa{/hwæ˧pʰæ˩/}}}).} \zh{锄头。}  \zh{量词}: \textcolor{darkblue}{\textbf{\ipa{nɑ˧}}}  \mytextsc{clf}~: \textcolor{darkblue}{\textbf{\ipa{nɑ˧}}} 
\lhead{\firstmark}
\rhead{\botmark}

\subsection{\hspace{-0.5cm} {\Large \textcolor{darkblue}{\textbf{\ipa{ɖʐwæ˧lɑ˧-ʁo˧ɖɯ˧˥}}}}\hspace{0.5cm}[\kern2pt{\textcolor{darkblue}{\textbf{\ipa{xxxx non-correspondance entre le nombre de morphèmes et le nombre de tons de morphèmes}}}}\kern2pt]} \hypertarget{d`z`w\{\string_MlA\string_M-Ro\string_Md`M\string_M\string_T1}{}
\markboth{\textcolor{darkblue}{\textbf{\ipa{ɖʐwæ˧lɑ˧-ʁo˧ɖɯ˧˥}}}}{}
\textcolor{teal}{\mytextsc{nom}} \hspace{4pt} Ton~: MH\#.
\textcolor{PineGreen}{\selectlanguage{french}Oiseau ressemblant à un moineau, au corps blanc et noir; M23 croît le reconnaître dans: Pericrocotus divaricatus, mais cette espèce n'existe que dans le nord de la Chine.} \zh{雀。}  \zh{量词}: \textcolor{darkblue}{\textbf{\ipa{mi˩}}}  \mytextsc{clf}~: \textcolor{darkblue}{\textbf{\ipa{mi˩}}} 
\lhead{\firstmark}
\rhead{\botmark}

\subsection{\hspace{-0.5cm} {\Large \textcolor{darkblue}{\textbf{\ipa{ɖʐwæ˧mi˧}}}}\hspace{0.5cm}[\kern2pt{\textcolor{darkblue}{\textbf{\ipa{ɖʐwæ˧mi˧}}}}\kern2pt]} \hypertarget{d`z`w\{\string_Mmi\string_M1}{}
\markboth{\textcolor{darkblue}{\textbf{\ipa{ɖʐwæ˧mi˧}}}}{}
\textcolor{teal}{\mytextsc{nom}} \hspace{4pt} Ton~: M.
\textcolor{PineGreen}{\selectlanguage{french}Moineau.} \zh{麻雀。}  \zh{量词}: \textcolor{darkblue}{\textbf{\ipa{mi˩}}}  \mytextsc{clf}~: \textcolor{darkblue}{\textbf{\ipa{mi˩}}} 
\lhead{\firstmark}
\rhead{\botmark}

\subsection{\hspace{-0.5cm} {\Large \textcolor{darkblue}{\textbf{\ipa{ɖʐwæ˧pʰv̩\#˥}}}}\hspace{0.5cm}[\kern2pt{\textcolor{darkblue}{\textbf{\ipa{ɖʐwæ˧pʰv̩˧}}}}\kern2pt]} \hypertarget{d`z`w\{\string_Mp\string_hv\string_=\#\string_T1}{}
\markboth{\textcolor{darkblue}{\textbf{\ipa{ɖʐwæ˧pʰv̩\#˥}}}}{}
\textcolor{teal}{\mytextsc{nom}} \hspace{4pt} Ton~: \#H.
\textcolor{PineGreen}{\selectlanguage{french}Moineau mâle.} \zh{公麻雀。}  ¶ \textcolor{darkblue}{\textbf{\ipa{ɖʐwæ˧pʰv̩˧ tʰv̩˧-mi˧˥ / ɖʐwæ˧pʰv̩˧ tʰv̩˧-mi˥\#}}} \textcolor{PineGreen}{\selectlanguage{french}\mytextsc{n}+\mytextsc{dem}+\mytextsc{clf}} \zh{这只公麻雀}  
 \zh{量词}: \textcolor{darkblue}{\textbf{\ipa{mi˩}}}  \mytextsc{clf}~: \textcolor{darkblue}{\textbf{\ipa{mi˩}}} 
\lhead{\firstmark}
\rhead{\botmark}

\subsection{\hspace{-0.5cm} {\Large \textcolor{darkblue}{\textbf{\ipa{ɖʐwæ˧zo\#˥}}}}\hspace{0.5cm}[\kern2pt{\textcolor{darkblue}{\textbf{\ipa{ɖʐwæ˧zo˧}}}}\kern2pt]} \hypertarget{d`z`w\{\string_Mzo\#\string_T1}{}
\markboth{\textcolor{darkblue}{\textbf{\ipa{ɖʐwæ˧zo\#˥}}}}{}
\textcolor{teal}{\mytextsc{nom}} \hspace{4pt} Ton~: \#H.
\textcolor{PineGreen}{\selectlanguage{french}Moinillon, petit moineau, bébé moineau.} \zh{小麻雀。}  \zh{量词}: \textcolor{darkblue}{\textbf{\ipa{v̩˧, mi˩}}}  \mytextsc{clf}~: \textcolor{darkblue}{\textbf{\ipa{v̩˧, mi˩}}} 
\lhead{\firstmark}
\rhead{\botmark}

\subsection{\hspace{-0.5cm} {\Large \textcolor{darkblue}{\textbf{\ipa{ɖʐwæ˩\textsubscript{a}}}}}\hspace{0.5cm}[\kern2pt{\textcolor{darkblue}{\textbf{\ipa{ɖʐwæ˩˥}}}}\kern2pt]} \hypertarget{d`z`w\{\string_Ba1}{}
\markboth{\textcolor{darkblue}{\textbf{\ipa{ɖʐwæ˩\textsubscript{a}}}}}{}
\textcolor{teal}{\mytextsc{verbe}} \hspace{4pt} Ton~: L\textsubscript{a}.
\textcolor{PineGreen}{\selectlanguage{french}Se disputer (monosyllabe).} \zh{吵架。}  ¶ \textcolor{darkblue}{\textbf{\ipa{ɖʐwæ˧\textasciitilde{}ɖʐwæ˥}}} \textcolor{PineGreen}{\selectlanguage{french}\mytextsc{red}} \zh{\mytextsc{重叠}}  

\lhead{\firstmark}
\rhead{\botmark}

\subsection{\hspace{-0.5cm} {\Large \textcolor{darkblue}{\textbf{\ipa{ɖʐwæ˩hi˩}}}}\hspace{0.5cm}[\kern2pt{\textcolor{darkblue}{\textbf{\ipa{ɖʐwæ˩hi˩˥}}}}\kern2pt]} \hypertarget{d`z`w\{\string_Bhi\string_B1}{}
\markboth{\textcolor{darkblue}{\textbf{\ipa{ɖʐwæ˩hi˩}}}}{}
\textcolor{teal}{\mytextsc{nom}} \hspace{4pt} Ton~: L.
\ding{202} \textcolor{PineGreen}{\selectlanguage{french}Canine (dent), croc.} \zh{獠牙。}  \zh{量词}: \textcolor{darkblue}{\textbf{\ipa{ɭɯ˧}}} \ding{203} \textcolor{PineGreen}{\selectlanguage{french}Crocs (de bête).} \zh{动物的牙(犬牙)。}  \mytextsc{clf}~: \textcolor{darkblue}{\textbf{\ipa{ɭɯ˧}}} 
\lhead{\firstmark}
\rhead{\botmark}

\subsection{\hspace{-0.5cm} {\Large \textcolor{darkblue}{\textbf{\ipa{ɖʐwæ˧˥}}}}\hspace{0.5cm}[\kern2pt{\textcolor{darkblue}{\textbf{\ipa{ɖʐwæ˧˥}}}}\kern2pt]} \hypertarget{d`z`w\{\string_M\string_T1}{}
\markboth{\textcolor{darkblue}{\textbf{\ipa{ɖʐwæ˧˥}}}}{}
\textcolor{teal}{\mytextsc{verbe}} \hspace{4pt} Ton~: MH.
\textcolor{PineGreen}{\selectlanguage{french}Tomber; laisser tomber, lâcher (un objet qu'on tenait à la main).} \zh{掉下。}  ¶ \textcolor{darkblue}{\textbf{\ipa{mv̩˩tɕo˧ ɖʐwæ˧˥ / mv̩˩tɕo˧ ɖʐwæ˧-ze˥}}} \textcolor{PineGreen}{\selectlanguage{french}tomber par terre; littéralement “tomber vers le bas”} \zh{掉下去(+了)}  

\lhead{\firstmark}
\rhead{\botmark}

\subsection{\hspace{-0.5cm} {\Large \textcolor{darkblue}{\textbf{\ipa{ɖʐwæ˩˧}}}}\hspace{0.5cm}[\kern2pt{\textcolor{darkblue}{\textbf{\ipa{ɖʐwæ˩˥}}}}\kern2pt]} \hypertarget{d`z`w\{\string_B\string_M1}{}
\markboth{\textcolor{darkblue}{\textbf{\ipa{ɖʐwæ˩˧}}}}{}
\textcolor{teal}{\mytextsc{nom}} \hspace{4pt} Ton~: LM.
\textcolor{PineGreen}{\selectlanguage{french}Moineau (forme monosyllabique; n'est pas d'usage courant).} \zh{麻雀。}  \zh{量词}: \textcolor{darkblue}{\textbf{\ipa{mi˩}}}  \mytextsc{clf}~: \textcolor{darkblue}{\textbf{\ipa{mi˩}}} 
\lhead{\firstmark}
\rhead{\botmark}

\newpage
\section*{\centering- \textcolor{darkblue}{\textbf{\ipa{ə}}} -}
\subsection{\hspace{-0.5cm} {\Large \textcolor{darkblue}{\textbf{\ipa{ə˧bɑ˩-lɑ˩bɑ˩}}}}\hspace{0.5cm}[\kern2pt{\textcolor{darkblue}{\textbf{\ipa{ə˧bɑ˩lɑ˧bɑ˧}}}}\kern2pt]} \hypertarget{@\string_MbA\string_B-lA\string_BbA\string_B1}{}
\markboth{\textcolor{darkblue}{\textbf{\ipa{ə˧bɑ˩-lɑ˩bɑ˩}}}}{}
\textcolor{teal}{\mytextsc{nom}} \hspace{4pt} Ton~: L\#-.
\textcolor{PineGreen}{\selectlanguage{french}Cactus.} \zh{仙人掌。}  ¶ \textcolor{darkblue}{\textbf{\ipa{ə˧bɑ˩-lɑ˩bɑ˩ | ɖɯ˧-dzi˩}}} \textcolor{PineGreen}{\selectlanguage{french}un cactus} \zh{一棵仙人掌}  

\lhead{\firstmark}
\rhead{\botmark}

\subsection{\hspace{-0.5cm} {\Large \textcolor{darkblue}{\textbf{\ipa{ə˧bo˥\$}}}}\hspace{0.5cm}[\kern2pt{\textcolor{darkblue}{\textbf{\ipa{ə˧bo˥}}}}\kern2pt]} \hypertarget{@\string_Mbo\string_T\$1}{}
\markboth{\textcolor{darkblue}{\textbf{\ipa{ə˧bo˥\$}}}}{}
\textcolor{teal}{\mytextsc{nom}} \hspace{4pt} Ton~: H\$.
\textcolor{PineGreen}{\selectlanguage{french}Oncle paternel=frère du père (sens vérifié: renvoie à la famille du père).} \zh{父亲的兄弟。}  ¶ \textcolor{darkblue}{\textbf{\ipa{ə˧bo˧-ɖɯ˧˥}}} \textcolor{PineGreen}{\selectlanguage{french}oncle paternel aîné du père} \zh{伯父:父亲的哥哥}  
 ¶ \textcolor{darkblue}{\textbf{\ipa{ə˧bo˧-tɕi˥ (+ɲi˩)}}} \textcolor{PineGreen}{\selectlanguage{french}oncle paternel cadet du père} \zh{叔叔:父亲的弟弟}  
 \zh{量词}: \textcolor{darkblue}{\textbf{\ipa{v̩˧}}}  \mytextsc{clf}~: \textcolor{darkblue}{\textbf{\ipa{v̩˧}}} 
\lhead{\firstmark}
\rhead{\botmark}

\subsection{\hspace{-0.5cm} {\Large \textcolor{darkblue}{\textbf{\ipa{ə˧bo˧tɕo˧li˧}}}}\hspace{0.5cm}[\kern2pt{\textcolor{darkblue}{\textbf{\ipa{ə˧bo˧tɕo˧li˧}}}}\kern2pt]} \hypertarget{@\string_Mbo\string_Mts£o\string_Mli\string_M1}{}
\markboth{\textcolor{darkblue}{\textbf{\ipa{ə˧bo˧tɕo˧li˧}}}}{}
\textcolor{teal}{\mytextsc{nom}} \hspace{4pt} Ton~: M.
\textcolor{PineGreen}{\selectlanguage{french}Criquet.} \zh{蟋蟀。}  \zh{量词}: \textcolor{darkblue}{\textbf{\ipa{mi˩}}}  \mytextsc{clf}~: \textcolor{darkblue}{\textbf{\ipa{mi˩}}} 
\lhead{\firstmark}
\rhead{\botmark}

\subsection{\hspace{-0.5cm} {\Large \textcolor{darkblue}{\textbf{\ipa{ə˧bv̩˩}}}}\hspace{0.5cm}[\kern2pt{\textcolor{darkblue}{\textbf{\ipa{ə˧bv̩˩}}}}\kern2pt]} \hypertarget{@\string_Mbv\string_=\string_B1}{}
\markboth{\textcolor{darkblue}{\textbf{\ipa{ə˧bv̩˩}}}}{}
\textcolor{teal}{\mytextsc{nom}} \hspace{4pt} Ton~: L\#.
\textcolor{PineGreen}{\selectlanguage{french}Four où on cuit les briques, les objets en céramique….} \zh{烤砖、陶器等用的烤炉。}  \zh{量词}: \textcolor{darkblue}{\textbf{\ipa{ɭɯ˧}}}  \mytextsc{clf}~: \textcolor{darkblue}{\textbf{\ipa{ɭɯ˧}}} 
\lhead{\firstmark}
\rhead{\botmark}

\subsection{\hspace{-0.5cm} {\Large \textcolor{darkblue}{\textbf{\ipa{ə˧bv̩˧-ʁwɤ˧}}}}\hspace{0.5cm}[\kern2pt{\textcolor{darkblue}{\textbf{\ipa{xxxx non-correspondance entre le nombre de morphèmes et le nombre de tons de morphèmes}}}}\kern2pt]} \hypertarget{@\string_Mbv\string_=\string_M-Rw7\string_M1}{}
\markboth{\textcolor{darkblue}{\textbf{\ipa{ə˧bv̩˧-ʁwɤ˧}}}}{}
\textcolor{teal}{\mytextsc{nom}} \hspace{4pt} Ton~: M.
\textcolor{PineGreen}{\selectlanguage{french}Abuwa (nom de village).} \zh{阿布瓦村。} 
\lhead{\firstmark}
\rhead{\botmark}

\subsection{\hspace{-0.5cm} {\Large \textcolor{darkblue}{\textbf{\ipa{ə˧ɕjɤ˩}}}}\hspace{0.5cm}[\kern2pt{\textcolor{darkblue}{\textbf{\ipa{ə˧ɕjɤ˩}}}}\kern2pt]} \hypertarget{@\string_Ms£j7\string_B1}{}
\markboth{\textcolor{darkblue}{\textbf{\ipa{ə˧ɕjɤ˩}}}}{}
\textcolor{teal}{\mytextsc{nom}} \hspace{4pt} Ton~: L\#.
\textcolor{PineGreen}{\selectlanguage{french}Petit(e) ami(e), amant(e).} \zh{情人。}  \zh{量词}: \textcolor{darkblue}{\textbf{\ipa{v̩˧}}}  \mytextsc{clf}~: \textcolor{darkblue}{\textbf{\ipa{v̩˧}}} \textit{Voir~:} \hyperlink{}{\textcolor{darkblue}{\textbf{\ipa{ə˧ɖo˧}}}} 
\lhead{\firstmark}
\rhead{\botmark}

\subsection{\hspace{-0.5cm} {\Large \textcolor{darkblue}{\textbf{\ipa{ə˧ɕjo˩}}}}\hspace{0.5cm}[\kern2pt{\textcolor{darkblue}{\textbf{\ipa{ə˧ɕjo˩}}}}\kern2pt]} \hypertarget{@\string_Ms£jo\string_B1}{}
\markboth{\textcolor{darkblue}{\textbf{\ipa{ə˧ɕjo˩}}}}{}
\textcolor{teal}{\mytextsc{nom}} \hspace{4pt} Ton~: L\#.
\textcolor{PineGreen}{\selectlanguage{french}Nom de clan/famille étendue. Deux familles portent ce nom à Yongning.} \zh{一个姓。这个姓,永宁有两家。}  ¶ \textcolor{darkblue}{\textbf{\ipa{ə˧ɕjo˩=ɻ̍˩}}} \textcolor{PineGreen}{\selectlanguage{french}le clan \textcolor{darkblue}{\textbf{\ipa{/ə˧ɕjo˩/}}}} \zh{\textcolor{darkblue}{\textbf{\ipa{/ə˧ɕjo˩/}}}家族}  

\lhead{\firstmark}
\rhead{\botmark}

\subsection{\hspace{-0.5cm} {\Large \textcolor{darkblue}{\textbf{\ipa{ə˧dɑ˥\$}}}}\hspace{0.5cm}[\kern2pt{\textcolor{darkblue}{\textbf{\ipa{ə˧dɑ˥}}}}\kern2pt]} \hypertarget{@\string_MdA\string_T\$1}{}
\markboth{\textcolor{darkblue}{\textbf{\ipa{ə˧dɑ˥\$}}}}{}
\textcolor{teal}{\mytextsc{nom}} \hspace{4pt} Ton~: H\$.
\textcolor{PineGreen}{\selectlanguage{french}Père.} \zh{父亲。}  \zh{量词}: \textcolor{darkblue}{\textbf{\ipa{v̩˧}}}  \mytextsc{clf}~: \textcolor{darkblue}{\textbf{\ipa{v̩˧}}} 
\lhead{\firstmark}
\rhead{\botmark}

\subsection{\hspace{-0.5cm} {\Large \textcolor{darkblue}{\textbf{\ipa{ə˧dɑ˧-ə˧mi\#˥}}}}\hspace{0.5cm}[\kern2pt{\textcolor{darkblue}{\textbf{\ipa{xxxx non-correspondance entre le nombre de morphèmes et le nombre de tons de morphèmes}}}}\kern2pt]} \hypertarget{@\string_MdA\string_M-@\string_Mmi\#\string_T1}{}
\markboth{\textcolor{darkblue}{\textbf{\ipa{ə˧dɑ˧-ə˧mi\#˥}}}}{}
\textcolor{teal}{\mytextsc{nom}} \hspace{4pt} Ton~: \#H.
\textcolor{PineGreen}{\selectlanguage{french}Père et mère.} \zh{父母。}  ¶ \textcolor{darkblue}{\textbf{\ipa{ə˧dɑ˧-ə˧mi˧ ɲi˥-kv̩˩}}} \textcolor{PineGreen}{\selectlanguage{french}le père et la mère, tous les deux; le couple formé du père et de la mère} \zh{父母亲}  

\lhead{\firstmark}
\rhead{\botmark}

\subsection{\hspace{-0.5cm} {\Large \textcolor{darkblue}{\textbf{\ipa{ə˧dɑ˧-zo\#˥}}}}\hspace{0.5cm}[\kern2pt{\textcolor{darkblue}{\textbf{\ipa{xxxx non-correspondance entre le nombre de morphèmes et le nombre de tons de morphèmes}}}}\kern2pt]} \hypertarget{@\string_MdA\string_M-zo\#\string_T1}{}
\markboth{\textcolor{darkblue}{\textbf{\ipa{ə˧dɑ˧-zo\#˥}}}}{}
\textcolor{teal}{\mytextsc{nom}} \hspace{4pt} Ton~: \#H.
\textcolor{PineGreen}{\selectlanguage{french}Père et fils.} \zh{父子。} 
\lhead{\firstmark}
\rhead{\botmark}

\subsection{\hspace{-0.5cm} {\Large \textcolor{darkblue}{\textbf{\ipa{ə˧dze˧}}}}\hspace{0.5cm}[\kern2pt{\textcolor{darkblue}{\textbf{\ipa{ə˧dze˧}}}}\kern2pt]} \hypertarget{@\string_Mdze\string_M1}{}
\markboth{\textcolor{darkblue}{\textbf{\ipa{ə˧dze˧}}}}{}
\textcolor{teal}{\mytextsc{nom}} \hspace{4pt} Ton~: M.
\textcolor{PineGreen}{\selectlanguage{french}Grémil des teinturiers, \textit{Lithospermum erythrorhizon Sieb. et Zucc.}.} \zh{紫草。} Dialecte chinois local~:\zh{紫红草。} ¶ \textcolor{darkblue}{\textbf{\ipa{ə˧dze˧-njɤ˩hṽ˩}}} \textcolor{PineGreen}{\selectlanguage{french}même sens} \zh{紫草}  
 ¶ \textcolor{darkblue}{\textbf{\ipa{ə˧dze˧-bæ˩bæ˩}}} \textcolor{PineGreen}{\selectlanguage{french}fleurs de grémil} \zh{紫草花}  

\lhead{\firstmark}
\rhead{\botmark}

\subsection{\hspace{-0.5cm} {\Large \textcolor{darkblue}{\textbf{\ipa{ə˧-dzɤ˥\$}}}}\hspace{0.5cm}[\kern2pt{\textcolor{darkblue}{\textbf{\ipa{xxxx non-correspondance entre le nombre de morphèmes et le nombre de tons de morphèmes}}}}\kern2pt]} \hypertarget{@\string_M-dz7\string_T\$1}{}
\markboth{\textcolor{darkblue}{\textbf{\ipa{ə˧-dzɤ˥\$}}}}{}
\textcolor{teal}{\mytextsc{adverbe}} \hspace{4pt} Ton~: H\$.
\textcolor{PineGreen}{\selectlanguage{french}Lentement, doucement.} \zh{慢。}  ¶ \textcolor{darkblue}{\textbf{\ipa{ə˧-dzɤ˧ ʝi˧}}} \textcolor{PineGreen}{\selectlanguage{french}travailler lentement, faire lentement} \zh{慢慢做}  
 ¶ \textcolor{darkblue}{\textbf{\ipa{ə˧dzɤ˧ le˧-hõ˩! |}}} \textcolor{PineGreen}{\selectlanguage{french}Au revoir! (Dit par l'hôte à la personne qui s'en va. Littéralement: “Allez doucement!” / “Prenez votre temps en chemin!”)} \zh{慢走!}  
 ¶ \textcolor{darkblue}{\textbf{\ipa{ə˧dzɤ˥ | le˧-hõ˩! |}}} \textcolor{PineGreen}{\selectlanguage{french}Au revoir!} \zh{慢走!}  
 ¶ \textcolor{darkblue}{\textbf{\ipa{ə˧dzɤ˧ le˧-dzi˩! |}}} \textcolor{PineGreen}{\selectlanguage{french}Au revoir! (Dit par l'invité à son hôte. Littéralement: “Restez assis doucement = tranquillement!”} \zh{慢慢坐!}  
\textit{Voir~:} \textcolor{darkblue}{\textbf{\ipa{ə˧ze˧, ə˧-dzɤ˧\textasciitilde{}dzɤ˥}}} 
\lhead{\firstmark}
\rhead{\botmark}

\subsection{\hspace{-0.5cm} {\Large \textcolor{darkblue}{\textbf{\ipa{ə˧-dzɤ˧\textasciitilde{}dzɤ˥}}}}\hspace{0.5cm}[\kern2pt{\textcolor{darkblue}{\textbf{\ipa{xxxx non-correspondance entre le nombre de morphèmes et le nombre de tons de morphèmes}}}}\kern2pt]} \hypertarget{@\string_M-dz7\string_M~dz7\string_T1}{}
\markboth{\textcolor{darkblue}{\textbf{\ipa{ə˧-dzɤ˧\textasciitilde{}dzɤ˥}}}}{}
\textcolor{teal}{\mytextsc{adverbe}} \hspace{4pt} Ton~: H\#.
\textit{De:} \textbf{ə˧-dzɤ˥\$, ə˧ze˧} \textcolor{PineGreen}{\selectlanguage{french}Lentement, doucement.} \zh{慢慢地。}  ¶ \textcolor{darkblue}{\textbf{\ipa{ʈʂʰɯ˧ | ɖwæ˧˥ | ə˧-dzɤ˧\textasciitilde{}dzɤ˥ ʝi˩-kv̩˩!}}} \textcolor{PineGreen}{\selectlanguage{french}Il/elle travaille avec grand soin. (Le sens littéral est “Il/elle travaille très lentement”, mais cela n'est pas une critique: cela signifie qu'il/elle sait prendre le temps pour réaliser du bon travail.)} \zh{他工作很细致。(直译:‘他工作很慢’,但不是批评:意味着那个人懂得慢慢来做,做得更仔细。)}  
 ¶ \textcolor{darkblue}{\textbf{\ipa{[F5] ə˧-dzɤ˧\textasciitilde{}dzɤ˥ ʝi˩}}} \textcolor{PineGreen}{\selectlanguage{french}travailler lentement, faire lentement} \zh{慢慢地做}  
 ¶ \textcolor{darkblue}{\textbf{\ipa{[M21] ə˧-zɤ˧\textasciitilde{}zɤ˥ ʝi˩}}} \textcolor{PineGreen}{\selectlanguage{french}Travaille doucement! / Prends ton temps! / travailler lentement, faire lentement} \zh{慢慢地做}  
\textit{Voir~:} \hyperlink{}{\textcolor{darkblue}{\textbf{\ipa{ə˧-dzɤ˥\$}}}} 
\lhead{\firstmark}
\rhead{\botmark}

\subsection{\hspace{-0.5cm} {\Large \textcolor{darkblue}{\textbf{\ipa{ə˧ɖo˧}}}}\hspace{0.5cm}[\kern2pt{\textcolor{darkblue}{\textbf{\ipa{ə˧ɖo˧}}}}\kern2pt]} \hypertarget{@\string_Md`o\string_M1}{}
\markboth{\textcolor{darkblue}{\textbf{\ipa{ə˧ɖo˧}}}}{}
\textcolor{teal}{\mytextsc{nom}} \hspace{4pt} Ton~: M.
\textcolor{PineGreen}{\selectlanguage{french}Petit ami, petite amie, amant(e).} \zh{情人(音译:阿注)。}  \zh{量词}: \textcolor{darkblue}{\textbf{\ipa{v̩˧}}}  \mytextsc{clf}~: \textcolor{darkblue}{\textbf{\ipa{v̩˧}}} \textit{Voir~:} \hyperlink{}{\textcolor{darkblue}{\textbf{\ipa{ə˧ɕjɤ˩}}}} 
\lhead{\firstmark}
\rhead{\botmark}

\subsection{\hspace{-0.5cm} {\Large \textcolor{darkblue}{\textbf{\ipa{ə˧go˧}}}}\hspace{0.5cm}[\kern2pt{\textcolor{darkblue}{\textbf{\ipa{ə˧go˧}}}}\kern2pt]} \hypertarget{@\string_Mgo\string_M1}{}
\markboth{\textcolor{darkblue}{\textbf{\ipa{ə˧go˧}}}}{}
\textcolor{teal}{\mytextsc{nom}} \hspace{4pt} Ton~: M.
\textcolor{PineGreen}{\selectlanguage{french}Nom de clan/famille étendue. Trois familles portent ce nom à Yongning.} \zh{一个姓。这个姓,永宁有三家。}  ¶ \textcolor{darkblue}{\textbf{\ipa{ə˧go˧=ɻ̍˩}}} \textcolor{PineGreen}{\selectlanguage{french}le clan \textcolor{darkblue}{\textbf{\ipa{/ə˧go˧/}}}} \zh{\textcolor{darkblue}{\textbf{\ipa{/ə˧go˧/}}}家族}  
 ¶ \textcolor{darkblue}{\textbf{\ipa{ə˧go˧ | dʑɤ˩tsʰi˧}}} \textcolor{PineGreen}{\selectlanguage{french}la personne prénommée \textcolor{darkblue}{\textbf{\ipa{/dʑɤ˩tsʰi\#˥/}}}, du clan \textcolor{darkblue}{\textbf{\ipa{/ə˧go˧/}}}} \zh{\textcolor{darkblue}{\textbf{\ipa{/ə˧go˧/}}} 家族名叫\textcolor{darkblue}{\textbf{\ipa{/dʑɤ˩tsʰi\#˥/}}}那个人}  

\lhead{\firstmark}
\rhead{\botmark}

\subsection{\hspace{-0.5cm} {\Large \textcolor{darkblue}{\textbf{\ipa{ə˧go˧-ʁwɤ˧}}}}\hspace{0.5cm}[\kern2pt{\textcolor{darkblue}{\textbf{\ipa{xxxx non-correspondance entre le nombre de morphèmes et le nombre de tons de morphèmes}}}}\kern2pt]} \hypertarget{@\string_Mgo\string_M-Rw7\string_M1}{}
\markboth{\textcolor{darkblue}{\textbf{\ipa{ə˧go˧-ʁwɤ˧}}}}{}
\textcolor{teal}{\mytextsc{nom}} \hspace{4pt} Ton~: M.
\textcolor{PineGreen}{\selectlanguage{french}Un village proche de Wenquan.} \zh{温泉乡的一个村落。}  ¶ \textcolor{darkblue}{\textbf{\ipa{ə˧go˧-ʁwɤ˧, | ʁwɤ˧lɑ˩-bi˩, | bæ˧ʁwɤ˧, | tʰo˧tsʰe\#˥, | pi˧tsʰe˩-di˩, | pɤ˧dʑɤ˩-di˩, | ʁwɤ˧tv̩˧}}} \textcolor{PineGreen}{\selectlanguage{french}Villages au sortir de la plaine de Yongning; les deux premiers comportent une population na; le troisième est un village na; les suivants sont essentiellement des villages pumi/prinmi.} \zh{永宁背向泸沽湖方向经过的村落。前两个村落拥有相当大的摩梭人口比例,第三个村落是摩梭村,最后一个是普米村。}  

\lhead{\firstmark}
\rhead{\botmark}

\subsection{\hspace{-0.5cm} {\Large \textcolor{darkblue}{\textbf{\ipa{ə˧gɯ˩}}}}\hspace{0.5cm}[\kern2pt{\textcolor{darkblue}{\textbf{\ipa{ə˧gɯ˩}}}}\kern2pt]} \hypertarget{@\string_MgM\string_B1}{}
\markboth{\textcolor{darkblue}{\textbf{\ipa{ə˧gɯ˩}}}}{}
\textcolor{teal}{\mytextsc{nom}} \hspace{4pt} Ton~: L\#.
\textcolor{PineGreen}{\selectlanguage{french}Menthe.} \zh{薄荷。}  \zh{量词}: \textcolor{darkblue}{\textbf{\ipa{po˧}}}  \mytextsc{clf}~: \textcolor{darkblue}{\textbf{\ipa{po˧}}} 
\lhead{\firstmark}
\rhead{\botmark}

\subsection{\hspace{-0.5cm} {\Large \textcolor{darkblue}{\textbf{\ipa{ə˧hɑ˩-bɑ˩lɑ˩}}}}\hspace{0.5cm}[\kern2pt{\textcolor{darkblue}{\textbf{\ipa{ə˧hɑ˩bɑ˧lɑ˧}}}}\kern2pt]} \hypertarget{@\string_MhA\string_B-bA\string_BlA\string_B1}{}
\markboth{\textcolor{darkblue}{\textbf{\ipa{ə˧hɑ˩-bɑ˩lɑ˩}}}}{}
\textcolor{teal}{\mytextsc{nom}} \hspace{4pt} Ton~: L\#-.
\textcolor{PineGreen}{\selectlanguage{french}Chanson traditionnelle.} \zh{民歌。}  ¶ \textcolor{darkblue}{\textbf{\ipa{ə˧hɑ˩bɑ˩lɑ˩ | ɖɯ˧-ɖʐo˩ gwɤ˩}}} \textcolor{PineGreen}{\selectlanguage{french}chanter une chanson} \zh{唱一首摩梭歌}  
 \zh{量词}: \textcolor{darkblue}{\textbf{\ipa{ɖʐo˩}}}  \mytextsc{clf}~: \textcolor{darkblue}{\textbf{\ipa{ɖʐo˩}}} 
\lhead{\firstmark}
\rhead{\botmark}

\subsection{\hspace{-0.5cm} {\Large \textcolor{darkblue}{\textbf{\ipa{ə˧hĩ˥}}}}\hspace{0.5cm}[\kern2pt{\textcolor{darkblue}{\textbf{\ipa{ə˧hĩ˥}}}}\kern2pt]} \hypertarget{@\string_Mhi\string_~\string_T1}{}
\markboth{\textcolor{darkblue}{\textbf{\ipa{ə˧hĩ˥}}}}{}
\textcolor{teal}{\mytextsc{adverbe}} \hspace{4pt} Ton~: H\#.
\textcolor{PineGreen}{\selectlanguage{french}Dans un moment.} \zh{一会儿、待会儿、等一下。}  ¶ \textcolor{darkblue}{\textbf{\ipa{ə˧hĩ˥-ɳɯ˩, | li˧-kʰɯ˧-bi˥!}}} \textcolor{PineGreen}{\selectlanguage{french}Tout à l’heure je vais te montrer! / dans un moment, je te montrerai!} \zh{待会儿,我给你看吧!}  

\lhead{\firstmark}
\rhead{\botmark}

\subsection{\hspace{-0.5cm} {\Large \textcolor{darkblue}{\textbf{\ipa{ə˧hwɤ˧}}}}\hspace{0.5cm}[\kern2pt{\textcolor{darkblue}{\textbf{\ipa{ə˧hwɤ˧}}}}\kern2pt]} \hypertarget{@\string_Mhw7\string_M1}{}
\markboth{\textcolor{darkblue}{\textbf{\ipa{ə˧hwɤ˧}}}}{}
\textcolor{teal}{\mytextsc{adverbe}} \hspace{4pt} Ton~: M.
\textcolor{PineGreen}{\selectlanguage{french}Hier soir.} \zh{昨晚。}  ¶ \textcolor{darkblue}{\textbf{\ipa{ə˧hwɤ˧ | mv̩˩kʰv̩˧˥}}} \textcolor{PineGreen}{\selectlanguage{french}hier au soir, dans la nuit} \zh{昨晚,夜里}  

\lhead{\firstmark}
\rhead{\botmark}

\subsection{\hspace{-0.5cm} {\Large \textcolor{darkblue}{\textbf{\ipa{ə˧jɤ˩}}}}\hspace{0.5cm}[\kern2pt{\textcolor{darkblue}{\textbf{\ipa{ə˧jɤ˩}}}}\kern2pt]} \hypertarget{@\string_Mj7\string_B1}{}
\markboth{\textcolor{darkblue}{\textbf{\ipa{ə˧jɤ˩}}}}{}
\textcolor{teal}{\mytextsc{nom}} \hspace{4pt} Ton~: L\#.
\textcolor{PineGreen}{\selectlanguage{french}Tante maternelle (sœur aînée de la mère).} \zh{姨母 (比母亲大)。}  \zh{量词}: \textcolor{darkblue}{\textbf{\ipa{v̩˧}}}  \mytextsc{clf}~: \textcolor{darkblue}{\textbf{\ipa{v̩˧}}} 
\lhead{\firstmark}
\rhead{\botmark}

\subsection{\hspace{-0.5cm} {\Large \textcolor{darkblue}{\textbf{\ipa{ə˧ʝi˥\$}}}}\hspace{0.5cm}[\kern2pt{\textcolor{darkblue}{\textbf{\ipa{ə˧ʝi˥}}}}\kern2pt]} \hypertarget{@\string_Mj££i\string_T\$1}{}
\markboth{\textcolor{darkblue}{\textbf{\ipa{ə˧ʝi˥\$}}}}{}
\textcolor{teal}{\mytextsc{adverbe}} \hspace{4pt} Ton~: H\$.
\textcolor{PineGreen}{\selectlanguage{french}L'année dernière, l'année passée, l'an passé, l'an dernier.} \zh{去年。} 
\lhead{\firstmark}
\rhead{\botmark}

\subsection{\hspace{-0.5cm} {\Large \textcolor{darkblue}{\textbf{\ipa{ə˧ʝi˧-ʂɯ˥ʝi˩}}}}\hspace{0.5cm}[\kern2pt{\textcolor{darkblue}{\textbf{\ipa{ə˧ʝi˧ʂɯ˥ʝi˩}}}}\kern2pt]} \hypertarget{@\string_Mj££i\string_M-s`M\string_Tj££i\string_B1}{}
\markboth{\textcolor{darkblue}{\textbf{\ipa{ə˧ʝi˧-ʂɯ˥ʝi˩}}}}{}
\textcolor{teal}{\mytextsc{adverbe}} \hspace{4pt} Ton~: \#H-.
\textcolor{PineGreen}{\selectlanguage{french}Jadis, aux temps anciens, il était une fois.} \zh{很久以前,古时候,传说古代。} 
\lhead{\firstmark}
\rhead{\botmark}

\subsection{\hspace{-0.5cm} {\Large \textcolor{darkblue}{\textbf{\ipa{ə˧ʝi˧-tsʰi˧ʝi\#˥}}}}\hspace{0.5cm}[\kern2pt{\textcolor{darkblue}{\textbf{\ipa{xxxx non-correspondance entre le nombre de morphèmes et le nombre de tons de morphèmes}}}}\kern2pt]} \hypertarget{@\string_Mj££i\string_M-ts\string_hi\string_Mj££i\#\string_T1}{}
\markboth{\textcolor{darkblue}{\textbf{\ipa{ə˧ʝi˧-tsʰi˧ʝi\#˥}}}}{}
\textcolor{teal}{\mytextsc{adverbe}} \hspace{4pt} Ton~: \#H.
\textcolor{PineGreen}{\selectlanguage{french}Ces années-ci, actuellement.} \zh{这几年、现在这个时代。} 
\lhead{\firstmark}
\rhead{\botmark}

\subsection{\hspace{-0.5cm} {\Large \textcolor{darkblue}{\textbf{\ipa{ə˧lɑ˧}}}}\hspace{0.5cm}[\kern2pt{\textcolor{darkblue}{\textbf{\ipa{ə˧lɑ˧}}}}\kern2pt]} \hypertarget{@\string_MlA\string_M1}{}
\markboth{\textcolor{darkblue}{\textbf{\ipa{ə˧lɑ˧}}}}{}
\textcolor{teal}{\mytextsc{nom}} \hspace{4pt} Ton~: M.
\textcolor{PineGreen}{\selectlanguage{french}Nom de clan/famille étendue. Trois familles portent ce nom à Yongning. C'est l'un des trois clans qui se sont établis les premiers dans le voisinage du monastère de Yongning, les deux autres étant \textcolor{darkblue}{\textbf{\ipa{/kɤ˧˥tʰɑ˩/}}} et \textcolor{darkblue}{\textbf{\ipa{/lɑ˧tʰɑ˧mi˥\$/}}}.} \zh{一个姓。这个姓,永宁有三家。}  ¶ \textcolor{darkblue}{\textbf{\ipa{ə˧lɑ˧=ɻ̍˩}}} \textcolor{PineGreen}{\selectlanguage{french}le clan \textcolor{darkblue}{\textbf{\ipa{/ə˧lɑ˧/}}}} \zh{\textcolor{darkblue}{\textbf{\ipa{/ə˧lɑ˧/}}}家族}  

\lhead{\firstmark}
\rhead{\botmark}

\subsection{\hspace{-0.5cm} {\Large \textcolor{darkblue}{\textbf{\ipa{ə˧lɑ˧-ʁwɤ\#˥}}}}\hspace{0.5cm}[\kern2pt{\textcolor{darkblue}{\textbf{\ipa{xxxx non-correspondance entre le nombre de morphèmes et le nombre de tons de morphèmes}}}}\kern2pt]} \hypertarget{@\string_MlA\string_M-Rw7\#\string_T1}{}
\markboth{\textcolor{darkblue}{\textbf{\ipa{ə˧lɑ˧-ʁwɤ\#˥}}}}{}
\textcolor{teal}{\mytextsc{nom}} \hspace{4pt} Ton~: \#H.
\textcolor{PineGreen}{\selectlanguage{french}Un hameau de Yongning, proche du monastère (lieu de naissance de la consultante principale). Nom chinois: Alawa.} \zh{永宁寺旁边的村落(主合作人住的地方)。(音译:阿拉瓦,旧名:七家村,因为村落在1960年左右有七个家庭)。}  ¶ \textcolor{darkblue}{\textbf{\ipa{dʑɤ˩bv̩˧kɤ˧-sɑ˥ʁwɤ˩, | hi˩ʁwɤ˩-lo˥, | æ˩mi˧-ʁwɤ\#˥, | lɑ˧lo˧-ʁwɤ˥, | lɑ˧ŋwɤ˧, | bɤ˧tsʰo˧gv̩˥, | ə˧lɑ˧-ʁwɤ\#˥, | gæ˧ɻæ˩, | qʰæ˧tɕʰi˧, | tʰo˧ʈɯ\#˥}}} \textcolor{PineGreen}{\selectlanguage{french}les dix villages comptant traditionnellement comme faisant partie de Yongning} \zh{摩梭传统地理概念中,属于永宁的十个村落}  

\lhead{\firstmark}
\rhead{\botmark}

\subsection{\hspace{-0.5cm} {\Large \textcolor{darkblue}{\textbf{\ipa{ə˧mɑ˧}}}}\hspace{0.5cm}[\kern2pt{\textcolor{darkblue}{\textbf{\ipa{ə˧mɑ˧}}}}\kern2pt]} \hypertarget{@\string_MmA\string_M1}{}
\markboth{\textcolor{darkblue}{\textbf{\ipa{ə˧mɑ˧}}}}{}
\textcolor{teal}{\mytextsc{nom}} \hspace{4pt} Ton~: M.
\textcolor{PineGreen}{\selectlanguage{french}Mère (terme d'adresse).} \zh{阿妈(孩子对母亲的称呼)。}  \zh{量词}: \textcolor{darkblue}{\textbf{\ipa{v̩˧}}}  \mytextsc{clf}~: \textcolor{darkblue}{\textbf{\ipa{v̩˧}}} 
\lhead{\firstmark}
\rhead{\botmark}

\subsection{\hspace{-0.5cm} {\Large \textcolor{darkblue}{\textbf{\ipa{ə˧mi˧}}}}\hspace{0.5cm}[\kern2pt{\textcolor{darkblue}{\textbf{\ipa{ə˧mi˧}}}}\kern2pt]} \hypertarget{@\string_Mmi\string_M1}{}
\markboth{\textcolor{darkblue}{\textbf{\ipa{ə˧mi˧}}}}{}
\textcolor{teal}{\mytextsc{nom}} \hspace{4pt} Ton~: M.
\textcolor{PineGreen}{\selectlanguage{french}Mère; le terme s'emploie aussi pour désigner les tantes.} \zh{母亲、姑母、姨母、伯母、叔母、大娘、婶、大妈、姨、伯母、舅母、大婶、大姨、阿姨、妗母、妗子、舅妈、婶母、婶娘、婶子、叔母、姨妈、姨母、姨娘。}  ¶ \textcolor{darkblue}{\textbf{\ipa{ə˧mi˧=ɻæ˩}}} \textcolor{PineGreen}{\selectlanguage{french}\string_ \mytextsc{associatif}: les mères =les femmes de la génération supérieure} \zh{母亲们 =长辈女性}  
 \zh{量词}: \textcolor{darkblue}{\textbf{\ipa{v̩˧}}} \textcolor{darkblue}{\textbf{\ipa{jɤ˧˥}}}  \mytextsc{clf}~: \textcolor{darkblue}{\textbf{\ipa{v̩˧}}} \textcolor{darkblue}{\textbf{\ipa{jɤ˧˥}}} 
\lhead{\firstmark}
\rhead{\botmark}

\subsection{\hspace{-0.5cm} {\Large \textcolor{darkblue}{\textbf{\ipa{ə˧mi˧-ɖɯ˩}}}}\hspace{0.5cm}[\kern2pt{\textcolor{darkblue}{\textbf{\ipa{xxxx non-correspondance entre le nombre de morphèmes et le nombre de tons de morphèmes}}}}\kern2pt]} \hypertarget{@\string_Mmi\string_M-d`M\string_B1}{}
\markboth{\textcolor{darkblue}{\textbf{\ipa{ə˧mi˧-ɖɯ˩}}}}{}
\textcolor{teal}{\mytextsc{nom}} \hspace{4pt} Ton~: L\#.
\textcolor{PineGreen}{\selectlanguage{french}Tante maternelle (sœur aînée de la mère).} \zh{姨母 (比母亲大)。}  \zh{量词}: \textcolor{darkblue}{\textbf{\ipa{v̩˧}}}  \mytextsc{clf}~: \textcolor{darkblue}{\textbf{\ipa{v̩˧}}} 
\lhead{\firstmark}
\rhead{\botmark}

\subsection{\hspace{-0.5cm} {\Large \textcolor{darkblue}{\textbf{\ipa{ə˧mi˧-tɕi˩}}}}\hspace{0.5cm}[\kern2pt{\textcolor{darkblue}{\textbf{\ipa{xxxx non-correspondance entre le nombre de morphèmes et le nombre de tons de morphèmes}}}}\kern2pt]} \hypertarget{@\string_Mmi\string_M-ts£i\string_B1}{}
\markboth{\textcolor{darkblue}{\textbf{\ipa{ə˧mi˧-tɕi˩}}}}{}
\textcolor{teal}{\mytextsc{nom}} \hspace{4pt} Ton~: L\#.
\textcolor{PineGreen}{\selectlanguage{french}Tante (soeur cadette de la mère).} \zh{姨母 (比母亲小)。}  \zh{量词}: \textcolor{darkblue}{\textbf{\ipa{v̩˧}}}  \mytextsc{clf}~: \textcolor{darkblue}{\textbf{\ipa{v̩˧}}} 
\lhead{\firstmark}
\rhead{\botmark}

\subsection{\hspace{-0.5cm} {\Large \textcolor{darkblue}{\textbf{\ipa{ə˧mi˧-zo\#˥}}}}\hspace{0.5cm}[\kern2pt{\textcolor{darkblue}{\textbf{\ipa{xxxx non-correspondance entre le nombre de morphèmes et le nombre de tons de morphèmes}}}}\kern2pt]} \hypertarget{@\string_Mmi\string_M-zo\#\string_T1}{}
\markboth{\textcolor{darkblue}{\textbf{\ipa{ə˧mi˧-zo\#˥}}}}{}
\textcolor{teal}{\mytextsc{nom}} \hspace{4pt} Ton~: \#H.
\textcolor{PineGreen}{\selectlanguage{french}Mère et fils.} \zh{母子。} 
\lhead{\firstmark}
\rhead{\botmark}

\subsection{\hspace{-0.5cm} {\Large \textcolor{darkblue}{\textbf{\ipa{ə˧mv̩˩}}}}\hspace{0.5cm}[\kern2pt{\textcolor{darkblue}{\textbf{\ipa{ə˧mv̩˩}}}}\kern2pt]} \hypertarget{@\string_Mmv\string_=\string_B1}{}
\markboth{\textcolor{darkblue}{\textbf{\ipa{ə˧mv̩˩}}}}{}
\textcolor{teal}{\mytextsc{nom}} \hspace{4pt} Ton~: L\#.
\textcolor{PineGreen}{\selectlanguage{french}Aîné: grand frère, grande sœur (employé aussi entre cousins).} \zh{哥哥,姐姐(也指堂哥堂姐)。}  ¶ \textcolor{darkblue}{\textbf{\ipa{æ˧mv̩˩=ɻæ˩}}} \textcolor{PineGreen}{\selectlanguage{french}\mytextsc{associatif}: les aînés dans la fratrie: sœurs et frères aînés} \zh{联想复数:哥哥们、姐姐们}  
 \zh{量词}: \textcolor{darkblue}{\textbf{\ipa{v̩˧}}}  \mytextsc{clf}~: \textcolor{darkblue}{\textbf{\ipa{v̩˧}}} 
\lhead{\firstmark}
\rhead{\botmark}

\subsection{\hspace{-0.5cm} {\Large \textcolor{darkblue}{\textbf{\ipa{ə˧mv̩˧-gi˥zɯ˩}}}}\hspace{0.5cm}[\kern2pt{\textcolor{darkblue}{\textbf{\ipa{ə˧mv̩˧gi˥zɯ˩}}}}\kern2pt]} \hypertarget{@\string_Mmv\string_=\string_M-gi\string_TzM\string_B1}{}
\markboth{\textcolor{darkblue}{\textbf{\ipa{ə˧mv̩˧-gi˥zɯ˩}}}}{}
\textcolor{teal}{\mytextsc{nom}} \hspace{4pt} Ton~: \#H-.
\textcolor{PineGreen}{\selectlanguage{french}Frères, quel que soit leur âge (aînés ou cadets).} \zh{兄弟(哥哥们与弟弟们)。} 
\lhead{\firstmark}
\rhead{\botmark}

\subsection{\hspace{-0.5cm} {\Large \textcolor{darkblue}{\textbf{\ipa{ə˧mv̩˥-tɕi˩}}}}\hspace{0.5cm}[\kern2pt{\textcolor{darkblue}{\textbf{\ipa{xxxx non-correspondance entre le nombre de morphèmes et le nombre de tons de morphèmes}}}}\kern2pt]} \hypertarget{@\string_Mmv\string_=\string_T-ts£i\string_B1}{}
\markboth{\textcolor{darkblue}{\textbf{\ipa{ə˧mv̩˥-tɕi˩}}}}{}
\textcolor{teal}{\mytextsc{adjectif}} \hspace{4pt} Ton~: H\#.
\textcolor{PineGreen}{\selectlanguage{french}Petit, tout petit, riquiqui.} \zh{小、细小。}  ¶ \textcolor{darkblue}{\textbf{\ipa{ə˧mv̩˥-tɕi˩-gv̩˩}}} \textcolor{PineGreen}{\selectlanguage{french}tout petit} \zh{小、细小}  
 ¶ \textcolor{darkblue}{\textbf{\ipa{ə˧mv̩˥-tɕi˩-hĩ˩}}} \textcolor{PineGreen}{\selectlanguage{french}tout petit} \zh{细小的}  
 ¶ \textcolor{darkblue}{\textbf{\ipa{ə˧mv̩˥tɕi˩ | ɖɯ˧-kʰwɤ˥}}} \textcolor{PineGreen}{\selectlanguage{french}un petit morceau} \zh{一小块}  

\lhead{\firstmark}
\rhead{\botmark}

\subsection{\hspace{-0.5cm} {\Large \textcolor{darkblue}{\textbf{\ipa{ə˧ɲi˥\$}}}}\hspace{0.5cm}[\kern2pt{\textcolor{darkblue}{\textbf{\ipa{ə˧ɲi˥}}}}\kern2pt]} \hypertarget{@\string_MJi\string_T\$1}{}
\markboth{\textcolor{darkblue}{\textbf{\ipa{ə˧ɲi˥\$}}}}{}
\textcolor{teal}{\mytextsc{adverbe}} \hspace{4pt} Ton~: H\$.
\textcolor{PineGreen}{\selectlanguage{french}Hier.} \zh{昨天。} 
\lhead{\firstmark}
\rhead{\botmark}

\subsection{\hspace{-0.5cm} {\Large \textcolor{darkblue}{\textbf{\ipa{ə˧ɲi˥-tsæ˩qæ˩}}}}\hspace{0.5cm}[\kern2pt{\textcolor{darkblue}{\textbf{\ipa{ə˧ɲi˥tsæ˩qæ˩}}}}\kern2pt]} \hypertarget{@\string_MJi\string_T-ts\{\string_Bq\{\string_B1}{}
\markboth{\textcolor{darkblue}{\textbf{\ipa{ə˧ɲi˥-tsæ˩qæ˩}}}}{}
\textcolor{teal}{\mytextsc{nom}} \hspace{4pt} Ton~: H\#-L.
\textcolor{PineGreen}{\selectlanguage{french}Auriculaire.} \zh{小指。}  \zh{量词}: \textcolor{darkblue}{\textbf{\ipa{ɭɯ˧}}}  \mytextsc{clf}~: \textcolor{darkblue}{\textbf{\ipa{ɭɯ˧}}} 
\lhead{\firstmark}
\rhead{\botmark}

\subsection{\hspace{-0.5cm} {\Large \textcolor{darkblue}{\textbf{\ipa{ə˧ɲi˧-tsʰi˧ɲi\#˥}}}}\hspace{0.5cm}[\kern2pt{\textcolor{darkblue}{\textbf{\ipa{xxxx non-correspondance entre le nombre de morphèmes et le nombre de tons de morphèmes}}}}\kern2pt]} \hypertarget{@\string_MJi\string_M-ts\string_hi\string_MJi\#\string_T1}{}
\markboth{\textcolor{darkblue}{\textbf{\ipa{ə˧ɲi˧-tsʰi˧ɲi\#˥}}}}{}
\textcolor{teal}{\mytextsc{adverbe}} \hspace{4pt} Ton~: \#H.
\textcolor{PineGreen}{\selectlanguage{french}Ces temps-ci, ces jours-ci.} \zh{近来。} 
\lhead{\firstmark}
\rhead{\botmark}

\subsection{\hspace{-0.5cm} {\Large \textcolor{darkblue}{\textbf{\ipa{ə˧pʰv̩˧}}}}\hspace{0.5cm}[\kern2pt{\textcolor{darkblue}{\textbf{\ipa{ə˧pʰv̩˧}}}}\kern2pt]} \hypertarget{@\string_Mp\string_hv\string_=\string_M1}{}
\markboth{\textcolor{darkblue}{\textbf{\ipa{ə˧pʰv̩˧}}}}{}
\textcolor{teal}{\mytextsc{nom}} \hspace{4pt} Ton~: M.
\textcolor{PineGreen}{\selectlanguage{french}Frère de la grand-mère = oncle de l'oncle maternel = oncle de la mère; sens étendu: personnage masculin important 2 générations au-dessus de soi.} \zh{舅姥爷:姥姥的哥哥或弟弟(也就是母亲的舅舅)。泛指:“祖父”。}  \zh{量词}: \textcolor{darkblue}{\textbf{\ipa{v̩˧}}}  \mytextsc{clf}~: \textcolor{darkblue}{\textbf{\ipa{v̩˧}}} 
\lhead{\firstmark}
\rhead{\botmark}

\subsection{\hspace{-0.5cm} {\Large \textcolor{darkblue}{\textbf{\ipa{ə˧si˧}}}}\hspace{0.5cm}[\kern2pt{\textcolor{darkblue}{\textbf{\ipa{ə˧si˧}}}}\kern2pt]} \hypertarget{@\string_Msi\string_M1}{}
\markboth{\textcolor{darkblue}{\textbf{\ipa{ə˧si˧}}}}{}
\textcolor{teal}{\mytextsc{nom}} \hspace{4pt} Ton~: M.
\textcolor{PineGreen}{\selectlanguage{french}Arrière-grand-mère (bisaïeule); sens étendu: bisaïeule et ses frères et soeurs, c'est-à-dire les membres importants de la famille à la 3e génération.} \zh{祖母。泛指:祖母与其兄弟姐妹。}  \zh{量词}: \textcolor{darkblue}{\textbf{\ipa{v̩˧}}}  \mytextsc{clf}~: \textcolor{darkblue}{\textbf{\ipa{v̩˧}}} 
\lhead{\firstmark}
\rhead{\botmark}

\subsection{\hspace{-0.5cm} {\Large \textcolor{darkblue}{\textbf{\ipa{ə˧si˧-ə˧pʰv̩\#˥}}}}\hspace{0.5cm}[\kern2pt{\textcolor{darkblue}{\textbf{\ipa{xxxx non-correspondance entre le nombre de morphèmes et le nombre de tons de morphèmes}}}}\kern2pt]} \hypertarget{@\string_Msi\string_M-@\string_Mp\string_hv\string_=\#\string_T1}{}
\markboth{\textcolor{darkblue}{\textbf{\ipa{ə˧si˧-ə˧pʰv̩\#˥}}}}{}
\textcolor{teal}{\mytextsc{nom}} \hspace{4pt} Ton~: \#H.
\textcolor{PineGreen}{\selectlanguage{french}Ancêtres aux 3e et 4e générations.} \zh{祖宗(三、四代以前)。} 
\lhead{\firstmark}
\rhead{\botmark}

\subsection{\hspace{-0.5cm} {\Large \textcolor{darkblue}{\textbf{\ipa{ə˧so˧}}}}\hspace{0.5cm}[\kern2pt{\textcolor{darkblue}{\textbf{\ipa{ə˧so˧}}}}\kern2pt]} \hypertarget{@\string_Mso\string_M1}{}
\markboth{\textcolor{darkblue}{\textbf{\ipa{ə˧so˧}}}}{}
\textcolor{teal}{\mytextsc{adverbe}} \hspace{4pt} Ton~: M.
\textcolor{PineGreen}{\selectlanguage{french}Tout à l'heure, il y a un moment.} \zh{刚才。} 
\lhead{\firstmark}
\rhead{\botmark}

\subsection{\hspace{-0.5cm} {\Large \textcolor{darkblue}{\textbf{\ipa{ə˧ʂɯ˧ɲi˥-ɖɯ˧ɲi˥}}}}\hspace{0.5cm}[\kern2pt{\textcolor{darkblue}{\textbf{\ipa{ə˧ʂɯ˧ɲi˥ɖɯ˩ɲi˩}}}}\kern2pt]} \hypertarget{@\string_Ms`M\string_MJi\string_T-d`M\string_MJi\string_T1}{}
\markboth{\textcolor{darkblue}{\textbf{\ipa{ə˧ʂɯ˧ɲi˥-ɖɯ˧ɲi˥}}}}{}
\textcolor{teal}{\mytextsc{adverbe}} \hspace{4pt} Ton~: H\#-H\$.
\textcolor{PineGreen}{\selectlanguage{french}Ces derniers jours, les jours passés.} \zh{前几天。} 
\lhead{\firstmark}
\rhead{\botmark}

\subsection{\hspace{-0.5cm} {\Large \textcolor{darkblue}{\textbf{\ipa{ə˧ti˥-dzi˩}}}}\hspace{0.5cm}[\kern2pt{\textcolor{darkblue}{\textbf{\ipa{ə˧ti˥dzi˩}}}}\kern2pt]} \hypertarget{@\string_Mti\string_T-dzi\string_B1}{}
\markboth{\textcolor{darkblue}{\textbf{\ipa{ə˧ti˥-dzi˩}}}}{}
\textcolor{teal}{\mytextsc{nom}} \hspace{4pt} Ton~: H\#-.
\textcolor{PineGreen}{\selectlanguage{french}Weixi (localité du Yunnan).} \zh{维西。} 
\lhead{\firstmark}
\rhead{\botmark}

\subsection{\hspace{-0.5cm} {\Large \textcolor{darkblue}{\textbf{\ipa{ə˧tɕi˩}}}}\hspace{0.5cm}[\kern2pt{\textcolor{darkblue}{\textbf{\ipa{ə˧tɕi˩}}}}\kern2pt]} \hypertarget{@\string_Mts£i\string_B1}{}
\markboth{\textcolor{darkblue}{\textbf{\ipa{ə˧tɕi˩}}}}{}
\textcolor{teal}{\mytextsc{nom}} \hspace{4pt} Ton~: L\#.
\textcolor{PineGreen}{\selectlanguage{french}Tante (soeur cadette de la mère).} \zh{姨母 (比母亲小)。}  ¶ \textcolor{darkblue}{\textbf{\ipa{ə˧tɕi˩=ɻæ˩}}} \textcolor{PineGreen}{\selectlanguage{french}\mytextsc{associatif}: les tantes} \zh{姨母们}  
 \zh{量词}: \textcolor{darkblue}{\textbf{\ipa{v̩˧}}}  \mytextsc{clf}~: \textcolor{darkblue}{\textbf{\ipa{v̩˧}}} 
\lhead{\firstmark}
\rhead{\botmark}

\subsection{\hspace{-0.5cm} {\Large \textcolor{darkblue}{\textbf{\ipa{ə˧tse˥\$}}}}\hspace{0.5cm}[\kern2pt{\textcolor{darkblue}{\textbf{\ipa{ə˧tse˥}}}}\kern2pt]} \hypertarget{@\string_Mtse\string_T\$1}{}
\markboth{\textcolor{darkblue}{\textbf{\ipa{ə˧tse˥\$}}}}{}
\textcolor{teal}{\mytextsc{adverbe}} \hspace{4pt} Ton~: H\$.
\textcolor{PineGreen}{\selectlanguage{french}Pourquoi.} \zh{为什么。}  ¶ \textcolor{darkblue}{\textbf{\ipa{ə˧tse˧-ʝi˥ / ə˧tse˧-ʝi˧}}} \textcolor{PineGreen}{\selectlanguage{french}Pourquoi? (Deux variantes, même sens)} \zh{为什么?(有两个变体,意思一致)}  
 ¶ \textcolor{darkblue}{\textbf{\ipa{no˧ | ə˧tse˧-ʝi˥ | mɤ˧-tsʰɯ˩ ɲi˩? / no˧ | ə˧tse˧-ʝi˧-zo˥ | mɤ˧-tsʰɯ˩ ɲi˩?}}} \textcolor{PineGreen}{\selectlanguage{french}pourquoi tu ne viens pas/n'es pas venu?} \zh{你为什么没有来?}  
 ¶ \textcolor{darkblue}{\textbf{\ipa{no˧ | ə˧tse˧-ʝi˥ | mɤ˧-dzɯ˥? = no˧ | ə˧tse˧-ʝi˥ | mɤ˧-dzɯ˧-ɲi˥?}}} \textcolor{PineGreen}{\selectlanguage{french}pourquoi tu ne manges pas?} \zh{你为什么不吃?}  
 ¶ \textcolor{darkblue}{\textbf{\ipa{ʈʂʰɯ˧ | ə˧tse˧-ɲi˥-hɯ˩?}}} \textcolor{PineGreen}{\selectlanguage{french}qu'est-ce que ça veut dire?} \zh{这是怎么一回事?}  

\lhead{\firstmark}
\rhead{\botmark}

\subsection{\hspace{-0.5cm} {\Large \textcolor{darkblue}{\textbf{\ipa{ə˧tso˧}}}}\hspace{0.5cm}[\kern2pt{\textcolor{darkblue}{\textbf{\ipa{ə˧tso˧}}}}\kern2pt]} \hypertarget{@\string_Mtso\string_M1}{}
\markboth{\textcolor{darkblue}{\textbf{\ipa{ə˧tso˧}}}}{}
\textcolor{teal}{\mytextsc{pronom}} \hspace{4pt} Ton~: M.
\textcolor{PineGreen}{\selectlanguage{french}\mytextsc{interrog}.quoi (quoi, pronom interrogatif).} \zh{什么。}  ¶ \textcolor{darkblue}{\textbf{\ipa{ə˧tso˧ ɲi˩?}}} \textcolor{PineGreen}{\selectlanguage{french}Qu'est-ce que c'est?} \zh{是什么?}  
 ¶ \textcolor{darkblue}{\textbf{\ipa{no˧ | ə˧tso˧ ʝi˧ bi˧?}}} \textcolor{PineGreen}{\selectlanguage{french}Qu'est-ce que tu vas faire? (Cette phrase peut se substituer à “où tu vas?”, \textcolor{darkblue}{\textbf{\ipa{/zo˩qo˧ bi˧?/}}}, comme salutation adressée à quelqu'un qui est en chemin)} \zh{你要做什么?}  
 ¶ \textcolor{darkblue}{\textbf{\ipa{no˧ | ə˧tse˧ bi˧?}}} \textcolor{PineGreen}{\selectlanguage{french}forme contractée de 2: dans \textcolor{darkblue}{\textbf{\ipa{[tse˧]}}}, \textcolor{darkblue}{\textbf{\ipa{/tso˧/}}} et le \textcolor{darkblue}{\textbf{\ipa{/ʝi˧/}}} suivant fusionnent en une seule syllabe.} \zh{上面例子的缩短格式:\textcolor{darkblue}{\textbf{\ipa{/tso˧/}}}与\textcolor{darkblue}{\textbf{\ipa{/ʝi˧/}}}合成一个音节,\textcolor{darkblue}{\textbf{\ipa{[tse˧]}}}。}  

\lhead{\firstmark}
\rhead{\botmark}

\subsection{\hspace{-0.5cm} {\Large \textcolor{darkblue}{\textbf{\ipa{ə˧tso˧\textasciitilde{}ə˧tso˥}}}}\hspace{0.5cm}[\kern2pt{\textcolor{darkblue}{\textbf{\ipa{ə˧tso˧ə˧tso˥}}}}\kern2pt]} \hypertarget{@\string_Mtso\string_M~@\string_Mtso\string_T1}{}
\markboth{\textcolor{darkblue}{\textbf{\ipa{ə˧tso˧\textasciitilde{}ə˧tso˥}}}}{}
\textcolor{teal}{\mytextsc{pronom}} \hspace{4pt} Ton~: H\#.
\textcolor{PineGreen}{\selectlanguage{french}\mytextsc{interrog}.quoi, rédupliqué.} \zh{什么(重叠)。} 
\lhead{\firstmark}
\rhead{\botmark}

\subsection{\hspace{-0.5cm} {\Large \textcolor{darkblue}{\textbf{\ipa{ə˧tso˧-mɤ˧-ɲi˩}}}}\hspace{0.5cm}[\kern2pt{\textcolor{darkblue}{\textbf{\ipa{xxxx non-correspondance entre le nombre de morphèmes et le nombre de tons de morphèmes}}}}\kern2pt]} \hypertarget{@\string_Mtso\string_M-m7\string_M-Ji\string_B1}{}
\markboth{\textcolor{darkblue}{\textbf{\ipa{ə˧tso˧-mɤ˧-ɲi˩}}}}{}
\textcolor{teal}{\mytextsc{pronom}} \hspace{4pt} Ton~: L\#.
\textcolor{PineGreen}{\selectlanguage{french}Tout, toutes les sortes de.} \zh{各种。} 
\lhead{\firstmark}
\rhead{\botmark}

\subsection{\hspace{-0.5cm} {\Large \textcolor{darkblue}{\textbf{\ipa{ə˧v̩˧˥}}} \textsubscript{1}}\hspace{0.5cm}[\kern2pt{\textcolor{darkblue}{\textbf{\ipa{ə˧v̩˧˥}}}}\kern2pt]} \hypertarget{@\string_Mv\string_=\string_M\string_T1}{}
\markboth{\textcolor{darkblue}{\textbf{\ipa{ə˧v̩˧˥}}} \textsubscript{1}}{}
\textcolor{teal}{\mytextsc{adjectif}} \hspace{4pt} Ton~: MH\#.
\textcolor{PineGreen}{\selectlanguage{french}Beau, joli.} \zh{美,好看,美丽。}  ¶ \textcolor{darkblue}{\textbf{\ipa{ɖwæ˧˥ | ə˧v̩˧˥}}} \textcolor{PineGreen}{\selectlanguage{french}\mytextsc{intensif}.très} \zh{很好看!}  
 ¶ \textcolor{darkblue}{\textbf{\ipa{ə˧-mɤ˧-v̩˧˥}}} \textcolor{PineGreen}{\selectlanguage{french}\mytextsc{neg}: vilain, laid} \zh{丑陋}  

\lhead{\firstmark}
\rhead{\botmark}

\subsection{\hspace{-0.5cm} {\Large \textcolor{darkblue}{\textbf{\ipa{ə˧v̩˧˥}}} \textsubscript{2}}\hspace{0.5cm}[\kern2pt{\textcolor{darkblue}{\textbf{\ipa{ə˧v̩˧˥}}}}\kern2pt]} \hypertarget{@\string_Mv\string_=\string_M\string_T2}{}
\markboth{\textcolor{darkblue}{\textbf{\ipa{ə˧v̩˧˥}}} \textsubscript{2}}{}
\textcolor{teal}{\mytextsc{nom}} \hspace{4pt} Ton~: MH\#.
\textcolor{PineGreen}{\selectlanguage{french}Oncle maternel =frère de la mère (aîné ou cadet).} \zh{舅舅、舅父 (比母亲大或比母亲小不区分)。}  ¶ \textcolor{darkblue}{\textbf{\ipa{ə˧v̩˧-ɖɯ˧˥}}} \textcolor{PineGreen}{\selectlanguage{french}oncle, aîné de la mère} \zh{比母亲大的舅舅}  
 ¶ \textcolor{darkblue}{\textbf{\ipa{ə˧v̩˧-tɕi˥}}} \textcolor{PineGreen}{\selectlanguage{french}oncle, cadet de la mère} \zh{比母亲小的舅舅}  
 ¶ \textcolor{darkblue}{\textbf{\ipa{mv˧ʁo˥ | tʰi˧-dze˩, | kɤ˩-nɑ˧mi˧ ɖɯ˧˥ ! | di˧qo˧ ʈʰɯ˧-dʑo˩, | ə˧v˧ ɖɯ˧˥!}}} \textcolor{PineGreen}{\selectlanguage{french}“Parmi tout ce qui vole dans le ciel, l'aigle est le plus grand; parmi tout ce qui marche sur la terre, l'oncle est le plus grand.”} \zh{“天上飞的,是老鹰最大。天下走的,是舅舅最大。”}  
 ¶ \textcolor{darkblue}{\textbf{\ipa{mv˧ʁo˥ dze˩hĩ˩-dʑo˥, | kɤ˩-nɑ˧mi˧; | di˧qo˧ se˧-dʑo˩, | ə˧v˧˥!}}} \textcolor{PineGreen}{\selectlanguage{french}“Parmi tout ce qui vole dans le ciel, l'aigle est le plus grand; parmi tout ce qui marche sur la terre, l'oncle est le plus grand.”} \zh{“天上飞的,是老鹰最大。天下走的,是舅舅最大。”}  
 ¶ \textcolor{darkblue}{\textbf{\ipa{mv˧ʁo˥ dze˩hĩ˩˥ | -dʑo˥, | kɤ˩-nɑ˧mi˧; | di˧qo˧ se˧-dʑo˩, | ə˧v˧˥!}}} \textcolor{PineGreen}{\selectlanguage{french}“Parmi tout ce qui vole dans le ciel, l'aigle est le plus grand; parmi tout ce qui marche sur la terre, l'oncle est le plus grand.”} \zh{“天上飞的,是老鹰最大。天下走的,是舅舅最大。”}  
 ¶ \textcolor{darkblue}{\textbf{\ipa{mv˧ʁo˥ dze˩hĩ˩-dʑo˥, | kɤ˩-nɑ˧mi˧; | di˧qo˧-dʑo˧, | ə˧v˧˥!}}} \textcolor{PineGreen}{\selectlanguage{french}“Parmi tout ce qui vole dans le ciel, l'aigle est le plus grand; parmi tout ce qui marche sur la terre, l'oncle est le plus grand.”} \zh{“天上飞的,是老鹰最大。天下走的,是舅舅最大。”}  
 ¶ \textcolor{darkblue}{\textbf{\ipa{mv˧ʁo˥ | dze˩-hĩ˩-dʑo˥, | ɖɯ˩-hĩ˩-dʑo˥, | kɤ˩-nɑ˧mi˧! | mv˧di˧-qo˥ | ɖɯ˩-hĩ˩-dʑo˥, | ə˧v˧˥!}}} \textcolor{PineGreen}{\selectlanguage{french}“Parmi tout ce qui vole dans le ciel, l'aigle est le plus grand; parmi tout ce qui marche sur la terre, l'oncle est le plus grand.”} \zh{“天上飞的,是老鹰最大。天下走的,是舅舅最大。”}  
 \zh{量词}: \textcolor{darkblue}{\textbf{\ipa{v̩˧}}}  \mytextsc{clf}~: \textcolor{darkblue}{\textbf{\ipa{v̩˧}}} 
\lhead{\firstmark}
\rhead{\botmark}

\subsection{\hspace{-0.5cm} {\Large \textcolor{darkblue}{\textbf{\ipa{ə˧v̩˧-ze˥v̩˩}}}}\hspace{0.5cm}[\kern2pt{\textcolor{darkblue}{\textbf{\ipa{ə˧v̩˥ze˩v̩˩}}}}\kern2pt]} \hypertarget{@\string_Mv\string_=\string_M-ze\string_Tv\string_=\string_B1}{}
\markboth{\textcolor{darkblue}{\textbf{\ipa{ə˧v̩˧-ze˥v̩˩}}}}{}
\textcolor{teal}{\mytextsc{nom}} \hspace{4pt} Ton~: H\#-.
\textcolor{PineGreen}{\selectlanguage{french}Oncle et neveu.} \zh{叔叔侄子。} 
\lhead{\firstmark}
\rhead{\botmark}

\subsection{\hspace{-0.5cm} {\Large \textcolor{darkblue}{\textbf{\ipa{ə˧ze˧}}}}\hspace{0.5cm}[\kern2pt{\textcolor{darkblue}{\textbf{\ipa{ə˧ze˥}}}}\kern2pt]} \hypertarget{@\string_Mze\string_M1}{}
\markboth{\textcolor{darkblue}{\textbf{\ipa{ə˧ze˧}}}}{}
\textcolor{teal}{\mytextsc{adverbe}} \hspace{4pt} Ton~: H\#.
\textcolor{PineGreen}{\selectlanguage{french}Lentement, doucement.} \zh{慢慢地。}  ¶ \textcolor{darkblue}{\textbf{\ipa{ə˧ze˧ le˧-hõ˩!}}} \textcolor{PineGreen}{\selectlanguage{french}Salutations à quelqu'un qui s'en va: “Au revoir!”, littéralement “Marche doucement!”} \zh{慢走!}  
 ¶ \textcolor{darkblue}{\textbf{\ipa{ə˧ze˧ le˧-dzi˩!}}} \textcolor{PineGreen}{\selectlanguage{french}Salutation lorsqu'on quitte quelqu'un: 'Au revoir!', littéralement 'Reste tranquillement assis!'} \zh{慢慢坐!}  
\textit{Voir~:} \hyperlink{}{\textcolor{darkblue}{\textbf{\ipa{ə˧-dzɤ˥\$}}}} 
\lhead{\firstmark}
\rhead{\botmark}

\subsection{\hspace{-0.5cm} {\Large \textcolor{darkblue}{\textbf{\ipa{ə˧zo˩-ʁwɤ˩}}}}\hspace{0.5cm}[\kern2pt{\textcolor{darkblue}{\textbf{\ipa{ə˧zo˩ʁwɤ˧}}}}\kern2pt]} \hypertarget{@\string_Mzo\string_B-Rw7\string_B1}{}
\markboth{\textcolor{darkblue}{\textbf{\ipa{ə˧zo˩-ʁwɤ˩}}}}{}
\textcolor{teal}{\mytextsc{nom}} \hspace{4pt} Ton~: L\#-.
\textcolor{PineGreen}{\selectlanguage{french}Un village proche des Sources Chaudes.} \zh{温泉乡的一个村落。} 
\lhead{\firstmark}
\rhead{\botmark}

\subsection{\hspace{-0.5cm} {\Large \textcolor{darkblue}{\textbf{\ipa{ə˧=zɯ˩}}}}\hspace{0.5cm}[\kern2pt{\textcolor{darkblue}{\textbf{\ipa{ə˧zɯ˩}}}}\kern2pt]} \hypertarget{@\string_M=zM\string_B1}{}
\markboth{\textcolor{darkblue}{\textbf{\ipa{ə˧=zɯ˩}}}}{}
\textcolor{teal}{\mytextsc{pronom}} \hspace{4pt} Ton~: L\# / L.
\textcolor{PineGreen}{\selectlanguage{french}Pronom duel inclusif: nous deux (le locuteur et l'interlocuteur).} \zh{咱们两个。} 
\lhead{\firstmark}
\rhead{\botmark}

\subsection{\hspace{-0.5cm} {\Large \textcolor{darkblue}{\textbf{\ipa{ə˧ʐv̩˩}}}}\hspace{0.5cm}[\kern2pt{\textcolor{darkblue}{\textbf{\ipa{ə˧ʐv̩˩}}}}\kern2pt]} \hypertarget{@\string_Mz`v\string_=\string_B1}{}
\markboth{\textcolor{darkblue}{\textbf{\ipa{ə˧ʐv̩˩}}}}{}
\textcolor{teal}{\mytextsc{adjectif}} \hspace{4pt} Ton~: L\#.
\textcolor{PineGreen}{\selectlanguage{french}Ancien, usagé.} \zh{陈旧。}  ¶ \textcolor{darkblue}{\textbf{\ipa{ʂe˧ ʐv̩˥}}} \textcolor{PineGreen}{\selectlanguage{french}de la vieille viande, de la viande pas fraîche} \zh{陈肉、不新鲜的肉}  

\lhead{\firstmark}
\rhead{\botmark}

\subsection{\hspace{-0.5cm} {\Large \textcolor{darkblue}{\textbf{\ipa{ə˧ʑi˧-ə˧pʰv̩˧˥}}}}\hspace{0.5cm}[\kern2pt{\textcolor{darkblue}{\textbf{\ipa{xxxx non-correspondance entre le nombre de morphèmes et le nombre de tons de morphèmes}}}}\kern2pt]} \hypertarget{@\string_Mz£i\string_M-@\string_Mp\string_hv\string_=\string_M\string_T1}{}
\markboth{\textcolor{darkblue}{\textbf{\ipa{ə˧ʑi˧-ə˧pʰv̩˧˥}}}}{}
\textcolor{teal}{\mytextsc{nom}} \hspace{4pt} Ton~: MH\#.
\textcolor{PineGreen}{\selectlanguage{french}La grand-mère et ses frères: les aînés 2 génération au-dessus de soi.} \zh{奶奶与她的兄弟。} 
\lhead{\firstmark}
\rhead{\botmark}

\subsection{\hspace{-0.5cm} {\Large \textcolor{darkblue}{\textbf{\ipa{ə˧ʑi˧-ʐv̩˥mi˩}}}}\hspace{0.5cm}[\kern2pt{\textcolor{darkblue}{\textbf{\ipa{ə˧ʑi˥ʐv̩˩mi˩}}}}\kern2pt]} \hypertarget{@\string_Mz£i\string_M-z`v\string_=\string_Tmi\string_B1}{}
\markboth{\textcolor{darkblue}{\textbf{\ipa{ə˧ʑi˧-ʐv̩˥mi˩}}}}{}
\textcolor{teal}{\mytextsc{nom}} \hspace{4pt} Ton~: H\#-.
\textcolor{PineGreen}{\selectlanguage{french}Grand-mère et petite-fille.} \zh{奶奶与孙女。} 
\lhead{\firstmark}
\rhead{\botmark}

\subsection{\hspace{-0.5cm} {\Large \textcolor{darkblue}{\textbf{\ipa{ə˧ʑi˧˥}}}}\hspace{0.5cm}[\kern2pt{\textcolor{darkblue}{\textbf{\ipa{ə˧ʑi˧˥}}}}\kern2pt]} \hypertarget{@\string_Mz£i\string_M\string_T1}{}
\markboth{\textcolor{darkblue}{\textbf{\ipa{ə˧ʑi˧˥}}}}{}
\textcolor{teal}{\mytextsc{nom}} \hspace{4pt} Ton~: MH\#.
\textcolor{PineGreen}{\selectlanguage{french}Grand-mère, aïeule; vieille femme.} \zh{祖母,姥姥,老妪。}  ¶ \textcolor{darkblue}{\textbf{\ipa{ə˧ʑi˧ ʝi˧ so˥-zo˩-ho˩-ze˩!}}} \textcolor{PineGreen}{\selectlanguage{french}Il va falloir que j'apprenne à me comporter (sagement) comme une grand-mère! (Contexte: remarque teintée d'humour de la consultante principale face à l'âge qui vient et ses soucis: un médecin lui déconseille les sofas/assises molles et lui recommande une chaise haute en bois; elle se fait la réflexion qu'elle a vieilli et doit maintenant apprendre à prendre des précautions.)} \zh{我要学习当老太太了!(情景:一位医生建议合作人不要坐在小凳子或者软沙发上了,而要坐更高的木头椅子。她幽默地说:“看来我是老年人了!”)}  
 \zh{量词}: \textcolor{darkblue}{\textbf{\ipa{v̩˧}}}  \mytextsc{clf}~: \textcolor{darkblue}{\textbf{\ipa{v̩˧}}} 
\lhead{\firstmark}
\rhead{\botmark}

\subsection{\hspace{-0.5cm} {\Large \textcolor{darkblue}{\textbf{\ipa{ə˩‑}}}}\hspace{0.5cm}[\kern2pt{\textcolor{darkblue}{\textbf{\ipa{ə˩˥}}}}\kern2pt]} \hypertarget{@\string_B‑1}{}
\markboth{\textcolor{darkblue}{\textbf{\ipa{ə˩‑}}}}{}
\textcolor{teal}{\mytextsc{pronom}} \hspace{4pt} Ton~: L.
\textcolor{PineGreen}{\selectlanguage{french}Interrogation totale.} \zh{……吗?。}  ¶ \textcolor{darkblue}{\textbf{\ipa{dʑɯ˧ | ə˩-dʑo˧?}}} \textcolor{PineGreen}{\selectlanguage{french}est-ce qu’il y a de l’eau ?} \zh{有谁吗?}  
 ¶ \textcolor{darkblue}{\textbf{\ipa{ə˩-ŋi˩˥ ?}}} \textcolor{PineGreen}{\selectlanguage{french}Est-ce que c’est ça ?/ C'est bien ça? ... n’est-ce pas ?} \zh{对吗? / 对吧?}  

\lhead{\firstmark}
\rhead{\botmark}

\subsection{\hspace{-0.5cm} {\Large \textcolor{darkblue}{\textbf{\ipa{ə˩kʰɯ˩}}}}\hspace{0.5cm}[\kern2pt{\textcolor{darkblue}{\textbf{\ipa{ə˩kʰɯ˩˥}}}}\kern2pt]} \hypertarget{@\string_Bk\string_hM\string_B1}{}
\markboth{\textcolor{darkblue}{\textbf{\ipa{ə˩kʰɯ˩}}}}{}
\textcolor{teal}{\mytextsc{nom}} \hspace{4pt} Ton~: L.
\textcolor{PineGreen}{\selectlanguage{french}Navet, \textit{Brassica rapa}.} \zh{芜菁 、扁萝卜、大头菜、蔓菁。}  ¶ \textcolor{darkblue}{\textbf{\ipa{ə˩kʰɯ˩-bv̩˧ | kʰɯ˩ʈɯ˩˥}}} \textcolor{PineGreen}{\selectlanguage{french}racine de navet} \zh{芜菁的根}  
 \zh{量词}: \textcolor{darkblue}{\textbf{\ipa{ɭɯ˧}}}  \mytextsc{clf}~: \textcolor{darkblue}{\textbf{\ipa{ɭɯ˧}}} 
\lhead{\firstmark}
\rhead{\botmark}

\subsection{\hspace{-0.5cm} {\Large \textcolor{darkblue}{\textbf{\ipa{ə˩ljɤ˩hæ̃˩ʂɯ˥-mo˩}}}}\hspace{0.5cm}[\kern2pt{\textcolor{darkblue}{\textbf{\ipa{ə˩ljɤ˩hæ̃˩ʂɯ˥mo˧}}}}\kern2pt]} \hypertarget{@\string_Blj7\string_Bh\{\string_~\string_Bs`M\string_T-mo\string_B1}{}
\markboth{\textcolor{darkblue}{\textbf{\ipa{ə˩ljɤ˩hæ̃˩ʂɯ˥-mo˩}}}}{}
\textcolor{teal}{\mytextsc{nom}} \hspace{4pt} Ton~: L+H\#-.
\textcolor{PineGreen}{\selectlanguage{french}Grand champignon jaune vif, comestible: \textit{Hygrophorus lucorum Kalc hbr.}. Littéralement “champignon doré”.} \zh{柠檬黄蜡伞(一种菌子)。} Dialecte chinois local~:\zh{黄蜡伞。}
\lhead{\firstmark}
\rhead{\botmark}

\subsection{\hspace{-0.5cm} {\Large \textcolor{darkblue}{\textbf{\ipa{ə˩qo˥}}}}\hspace{0.5cm}[\kern2pt{\textcolor{darkblue}{\textbf{\ipa{ə˩qo˥}}}}\kern2pt]} \hypertarget{@\string_Bqo\string_T1}{}
\markboth{\textcolor{darkblue}{\textbf{\ipa{ə˩qo˥}}}}{}
\textcolor{teal}{\mytextsc{adverbe}} \hspace{4pt} Ton~: LH.
\textcolor{PineGreen}{\selectlanguage{french}À l'intérieur, vers l'intérieur.} \zh{往里。} 
\lhead{\firstmark}
\rhead{\botmark}

\newpage
\section*{\centering- \textcolor{darkblue}{\textbf{\ipa{f}}} -}
\subsection{\hspace{-0.5cm} {\Large \textcolor{darkblue}{\textbf{\ipa{fɑ˧tɑ˧˥}}}}\hspace{0.5cm}[\kern2pt{\textcolor{darkblue}{\textbf{\ipa{fɑ˧tɑ˧˥}}}}\kern2pt]} \hypertarget{fA\string_MtA\string_M\string_T1}{}
\markboth{\textcolor{darkblue}{\textbf{\ipa{fɑ˧tɑ˧˥}}}}{}
\textcolor{teal}{\mytextsc{adjectif}} \hspace{4pt} Ton~: MH.
\textcolor{PineGreen}{\selectlanguage{french}Développé, florissant.} \zh{发达。}  Emprunt~: chinois  \zh{发达}
 ¶ \textcolor{darkblue}{\textbf{\ipa{fɑ˧tɑ˧-ze˥}}} \textcolor{PineGreen}{\selectlanguage{french}\mytextsc{pfv}} \zh{很发达的了}  

\lhead{\firstmark}
\rhead{\botmark}

\subsection{\hspace{-0.5cm} {\Large \textcolor{darkblue}{\textbf{\ipa{fɑ˩\textsubscript{a}}}}}\hspace{0.5cm}[\kern2pt{\textcolor{darkblue}{\textbf{\ipa{fɑ˩˥}}}}\kern2pt]} \hypertarget{fA\string_Ba1}{}
\markboth{\textcolor{darkblue}{\textbf{\ipa{fɑ˩\textsubscript{a}}}}}{}
\textcolor{teal}{\mytextsc{verbe}} \hspace{4pt} Ton~: L\textsubscript{a}.
\textcolor{PineGreen}{\selectlanguage{french}Fermenter.} \zh{发酵(汉语借词:发)。}  Emprunt~: chinois  \zh{发(酵)}
 ¶ \textcolor{darkblue}{\textbf{\ipa{tsɑ˧bɤ˧ ɖɯ˧-mɤ˩ | tʰi˧-fɑ˩}}} \textcolor{PineGreen}{\selectlanguage{french}faire lever un peu de farine} \zh{发一点面}  
 ¶ \textcolor{darkblue}{\textbf{\ipa{tsɑ˧bɤ˧ tʰi˧-fɑ˩! | pɤ˩jɤ˧ gv̩˥-bi˩!}}} \textcolor{PineGreen}{\selectlanguage{french}Fais lever de la farine, on va faire des petits pains!} \zh{你发一点面吧!要做馒头!}  

\lhead{\firstmark}
\rhead{\botmark}

\subsection{\hspace{-0.5cm} {\Large \textcolor{darkblue}{\textbf{\ipa{fæ˧}}}}\hspace{0.5cm}[\kern2pt{\textcolor{darkblue}{\textbf{\ipa{fæ˥}}}}\kern2pt]} \hypertarget{f\{\string_M1}{}
\markboth{\textcolor{darkblue}{\textbf{\ipa{fæ˧}}}}{}
\textcolor{teal}{\mytextsc{nom}} \hspace{4pt} Ton~: M.
\textcolor{PineGreen}{\selectlanguage{french}Direction.} \zh{方(方向的方)(汉语借词)。}  Emprunt~: chinois  \zh{方}
 ¶ \textcolor{darkblue}{\textbf{\ipa{dv̩˩tɕo˧ fæ˧}}} \textcolor{PineGreen}{\selectlanguage{french}cette direction-là} \zh{那个方向}  
\textit{Voir~:} \hyperlink{}{\textcolor{darkblue}{\textbf{\ipa{dɤ˧-tʰv̩˧-gi\#˥}}}} 
\lhead{\firstmark}
\rhead{\botmark}

\subsection{\hspace{-0.5cm} {\Large \textcolor{darkblue}{\textbf{\ipa{fv̩˩˧}}}}\hspace{0.5cm}[\kern2pt{\textcolor{darkblue}{\textbf{\ipa{fv̩˩˥}}}}\kern2pt]} \hypertarget{fv\string_=\string_B\string_M1}{}
\markboth{\textcolor{darkblue}{\textbf{\ipa{fv̩˩˧}}}}{}
\textcolor{teal}{\mytextsc{nom}} \hspace{4pt} Ton~: LM.
\textcolor{PineGreen}{\selectlanguage{french}Le voisinage, les voisins.} \zh{邻居,村里的人们。} 
\lhead{\firstmark}
\rhead{\botmark}

\subsection{\hspace{-0.5cm} {\Large \textcolor{darkblue}{\textbf{\ipa{fv̩˧}}}}\hspace{0.5cm}[\kern2pt{\textcolor{darkblue}{\textbf{\ipa{fv̩˥}}}}\kern2pt]} \hypertarget{fv\string_=\string_M1}{}
\markboth{\textcolor{darkblue}{\textbf{\ipa{fv̩˧}}}}{}
\textcolor{teal}{\mytextsc{adjectif}} \hspace{4pt} Ton~: M.
\textcolor{PineGreen}{\selectlanguage{french}Content, joyeux; agréable; aimer, apprécier.} \zh{高兴、起劲,喜欢、爱、愿意。}  ¶ \textcolor{darkblue}{\textbf{\ipa{ɖwæ˧˥ | fv̩˧}}} \textcolor{PineGreen}{\selectlanguage{french}\mytextsc{intensif}.très: très content, tout content} \zh{很高兴}  
 ¶ \textcolor{darkblue}{\textbf{\ipa{dʑɤ˩˥ | fv̩˧}}} \textcolor{PineGreen}{\selectlanguage{french}très content, tout content} \zh{很高兴}  
 ¶ \textcolor{darkblue}{\textbf{\ipa{mɤ˧-fv̩˧ ʝi˧}}} \textcolor{PineGreen}{\selectlanguage{french}se mettre en colère, s'énerver} \zh{生气}  
 ¶ \textcolor{darkblue}{\textbf{\ipa{ʈʂʰɯ˧ mɤ˧-fv̩˧ ʝi˧!}}} \textcolor{PineGreen}{\selectlanguage{french}Il/elle est mécontent(e) / en colère.} \zh{他在生气。}  
 ¶ \textcolor{darkblue}{\textbf{\ipa{[F5] ɖwæ˧˥ | fv̩˧hĩ˧ ɖɯ˧-v̩˧ ɲi˩}}} \textcolor{PineGreen}{\selectlanguage{french}c'est quelqu'un de très agréable} \zh{他是很善良的人。}  

\lhead{\firstmark}
\rhead{\botmark}

\subsection{\hspace{-0.5cm} {\Large \textcolor{darkblue}{\textbf{\ipa{fv̩˩bi˩}}}}\hspace{0.5cm}[\kern2pt{\textcolor{darkblue}{\textbf{\ipa{fv̩˩bi˩˥}}}}\kern2pt]} \hypertarget{fv\string_=\string_Bbi\string_B1}{}
\markboth{\textcolor{darkblue}{\textbf{\ipa{fv̩˩bi˩}}}}{}
\textcolor{teal}{\mytextsc{nom}} \hspace{4pt} Ton~: L.
\textcolor{PineGreen}{\selectlanguage{french}Contrée, voisinage, ensemble de villages où habitent des gens de la famille étendue.} \zh{邻里、邻村:大家族居住的那片地方,包括几个小村落。} 
\lhead{\firstmark}
\rhead{\botmark}

\subsection{\hspace{-0.5cm} {\Large \textcolor{darkblue}{\textbf{\ipa{fv̩˧kʰo˥}}}}\hspace{0.5cm}[\kern2pt{\textcolor{darkblue}{\textbf{\ipa{fv̩˧kʰo˥}}}}\kern2pt]} \hypertarget{fv\string_=\string_Mk\string_ho\string_T1}{}
\markboth{\textcolor{darkblue}{\textbf{\ipa{fv̩˧kʰo˥}}}}{}
\textcolor{teal}{\mytextsc{nom}} \hspace{4pt} Ton~: H\#.
\textcolor{PineGreen}{\selectlanguage{french}Fengke: village situé au bord du Yang-tsé, sur la rive droite.} \zh{奉科(金沙江边的一个地区)。} 
\lhead{\firstmark}
\rhead{\botmark}

\subsection{\hspace{-0.5cm} {\Large \textcolor{darkblue}{\textbf{\ipa{fv̩˧ʂɯ˩}}}}\hspace{0.5cm}[\kern2pt{\textcolor{darkblue}{\textbf{\ipa{fv̩˧ʂɯ˩}}}}\kern2pt]} \hypertarget{fv\string_=\string_Ms`M\string_B1}{}
\markboth{\textcolor{darkblue}{\textbf{\ipa{fv̩˧ʂɯ˩}}}}{}
\textcolor{teal}{\mytextsc{nom}} \hspace{4pt} Ton~: L\#.
\textcolor{PineGreen}{\selectlanguage{french}Rhumatismes.} \zh{风湿(汉语借词)。}  Emprunt~: chinois  \zh{风湿}
 ¶ \textcolor{darkblue}{\textbf{\ipa{fv̩˧ʂɯ˩ go˩}}} \textcolor{PineGreen}{\selectlanguage{french}souffrir de rhumatismes, avoir des rhumatismes} \zh{有风湿、得风湿}  

\lhead{\firstmark}
\rhead{\botmark}

\newpage
\section*{\centering- \textcolor{darkblue}{\textbf{\ipa{g}}} -}
\subsection{\hspace{-0.5cm} {\Large \textcolor{darkblue}{\textbf{\ipa{gæ˧ɻæ˩}}}}\hspace{0.5cm}[\kern2pt{\textcolor{darkblue}{\textbf{\ipa{gæ˧ɻæ˩}}}}\kern2pt]} \hypertarget{g\{\string_Mr£`\{\string_B1}{}
\markboth{\textcolor{darkblue}{\textbf{\ipa{gæ˧ɻæ˩}}}}{}
\textcolor{teal}{\mytextsc{nom}} \hspace{4pt} Ton~: L\#.
\textcolor{PineGreen}{\selectlanguage{french}Village situé à environ 1,5 km à l'ouest de \textcolor{darkblue}{\textbf{\ipa{/ə˧lɑ˧-ʁwɤ\#˥/:}}} à main gauche en sortant de la vallée de Yongning, en direction de Eya. En chinois: Gaer.} \zh{嘎尔村。}  ¶ \textcolor{darkblue}{\textbf{\ipa{dʑɤ˩bv̩˧kɤ˧-sɑ˥ʁwɤ˩, | hi˩ʁwɤ˩-lo˥, | æ˩mi˧-ʁwɤ\#˥, | lɑ˧lo˧-ʁwɤ˥, | lɑ˧ŋwɤ˧, | bɤ˧tsʰo˧gv̩˥, | ə˧lɑ˧-ʁwɤ\#˥, | gæ˧ɻæ˩, | qʰæ˧tɕʰi˧, | tʰo˧ʈɯ\#˥}}} \textcolor{PineGreen}{\selectlanguage{french}les dix villages comptant traditionnellement comme faisant partie de Yongning} \zh{摩梭传统地理概念中,属于永宁的十个村落}  

\lhead{\firstmark}
\rhead{\botmark}

\subsection{\hspace{-0.5cm} {\Large \textcolor{darkblue}{\textbf{\ipa{gæ˩ɖæ˧}}}}\hspace{0.5cm}[\kern2pt{\textcolor{darkblue}{\textbf{\ipa{gæ˩ɖæ˥}}}}\kern2pt]} \hypertarget{g\{\string_Bd`\{\string_M1}{}
\markboth{\textcolor{darkblue}{\textbf{\ipa{gæ˩ɖæ˧}}}}{}
\textcolor{teal}{\mytextsc{nom}} \hspace{4pt} Ton~: LM.
\textcolor{PineGreen}{\selectlanguage{french}Le haut du corps.} \zh{上半身。} 
\lhead{\firstmark}
\rhead{\botmark}

\subsection{\hspace{-0.5cm} {\Large \textcolor{darkblue}{\textbf{\ipa{gæ˩pʰæ˧}}}}\hspace{0.5cm}[\kern2pt{\textcolor{darkblue}{\textbf{\ipa{gæ˩pʰæ˥}}}}\kern2pt]} \hypertarget{g\{\string_Bp\string_h\{\string_M1}{}
\markboth{\textcolor{darkblue}{\textbf{\ipa{gæ˩pʰæ˧}}}}{}
\textcolor{teal}{\mytextsc{nom}} \hspace{4pt} Ton~: LM.
\textcolor{PineGreen}{\selectlanguage{french}Resserre, pièce où on conserve certains produits: dans le même bâtiment que la cuisine-foyer-salle à manger, à sa gauche (vu depuis la cour).} \zh{储藏室、库房:存粮食、火腿的房间。}  \zh{量词}: \textcolor{darkblue}{\textbf{\ipa{tso˩}}}  \mytextsc{clf}~: \textcolor{darkblue}{\textbf{\ipa{tso˩}}} 
\lhead{\firstmark}
\rhead{\botmark}

\subsection{\hspace{-0.5cm} {\Large \textcolor{darkblue}{\textbf{\ipa{-gɤ˧}}}}\hspace{0.5cm}[\kern2pt{\textcolor{darkblue}{\textbf{\ipa{gɤ˥}}}}\kern2pt]} \hypertarget{-g7\string_M1}{}
\markboth{\textcolor{darkblue}{\textbf{\ipa{-gɤ˧}}}}{}
\textcolor{teal}{\mytextsc{nom}} \hspace{4pt} Ton~: M.
\ding{202} \textcolor{PineGreen}{\selectlanguage{french}Lieu, endroit.} \zh{地方。}  ¶ \textcolor{darkblue}{\textbf{\ipa{njɤ˧ | ɖɯ˧-ʝi˧ (-gɤ˧) bi˧-zo˧-ho˩!}}} \textcolor{PineGreen}{\selectlanguage{french}Je dois me rendre quelque part! / Je dois faire un voyage! / Je m'en vais! (Contexte: lorsqu'on se prépare réellement à un voyage; ou lors d'une dispute, lorsqu'on menace de quitter la maison.)} \zh{我要去一个别的地方! / 我要换一个地方了! / 我要走了!}  
 ¶ \textcolor{darkblue}{\textbf{\ipa{ze˩ gɤ˧}}} \textcolor{PineGreen}{\selectlanguage{french}quel endroit} \zh{什么地方}  
 ¶ \textcolor{darkblue}{\textbf{\ipa{ʈʂʰɯ˧-gɤ˧}}} \textcolor{PineGreen}{\selectlanguage{french}cet endroit-ci} \zh{这个地方}  
 ¶ \textcolor{darkblue}{\textbf{\ipa{tʰv̩˧-gɤ˧}}} \textcolor{PineGreen}{\selectlanguage{french}cet endroit-là} \zh{那个地方}  
\ding{203} \textcolor{PineGreen}{\selectlanguage{french}Moment.} \zh{时候。}  ¶ \textcolor{darkblue}{\textbf{\ipa{ʂɯ˧-ɬi˧mi˧-qo˧-gɤ˧ tʰv̩˧}}} \textcolor{PineGreen}{\selectlanguage{french}quand est venu le septième mois, quand on en est au septième mois} \zh{七月到了的时候}  
 ¶ \textcolor{darkblue}{\textbf{\ipa{ʂɯ˧-ɬi˧mi˧-qo˧-gɤ˧-dʑo˥}}} \textcolor{PineGreen}{\selectlanguage{french}pendant le septième mois, au cours du septième mois} \zh{七月的时候}  

\lhead{\firstmark}
\rhead{\botmark}

\subsection{\hspace{-0.5cm} {\Large \textcolor{darkblue}{\textbf{\ipa{gɤ˧\textsubscript{b}}}}}\hspace{0.5cm}[\kern2pt{\textcolor{darkblue}{\textbf{\ipa{gɤ˩˥}}}}\kern2pt]} \hypertarget{g7\string_Mb1}{}
\markboth{\textcolor{darkblue}{\textbf{\ipa{gɤ˧\textsubscript{b}}}}}{}
\textcolor{teal}{\mytextsc{verbe}} \hspace{4pt} Ton~: M\textsubscript{b}.
\textcolor{PineGreen}{\selectlanguage{french}Manquer de.} \zh{缺乏。}  ¶ \textcolor{darkblue}{\textbf{\ipa{mɤ˧-gɤ˧}}} \textcolor{PineGreen}{\selectlanguage{french}\mytextsc{neg}: ne pas manquer de} \zh{不缺乏}  

\lhead{\firstmark}
\rhead{\botmark}

\subsection{\hspace{-0.5cm} {\Large \textcolor{darkblue}{\textbf{\ipa{gɤ˧bɤ˧}}}}\hspace{0.5cm}[\kern2pt{\textcolor{darkblue}{\textbf{\ipa{gɤ˧bɤ˩}}}}\kern2pt]} \hypertarget{g7\string_Mb7\string_M1}{}
\markboth{\textcolor{darkblue}{\textbf{\ipa{gɤ˧bɤ˧}}}}{}
\textcolor{teal}{\mytextsc{nom}} \hspace{4pt} Ton~: M.
\textcolor{PineGreen}{\selectlanguage{french}Ombre.} \zh{影子。}  ¶ \textcolor{darkblue}{\textbf{\ipa{gɤ˧bɤ˧ li˧}}} \textcolor{PineGreen}{\selectlanguage{french}regarder la télé (néologisme)} \zh{看电视}  
 \zh{量词}: \textcolor{darkblue}{\textbf{\ipa{v̩˧}}}  \mytextsc{clf}~: \textcolor{darkblue}{\textbf{\ipa{v̩˧}}} 
\lhead{\firstmark}
\rhead{\botmark}

\subsection{\hspace{-0.5cm} {\Large \textcolor{darkblue}{\textbf{\ipa{-gɤ˧bi\#˥}}}}\hspace{0.5cm}[\kern2pt{\textcolor{darkblue}{\textbf{\ipa{gɤ˩bi˥}}}}\kern2pt]} \hypertarget{-g7\string_Mbi\#\string_T1}{}
\markboth{\textcolor{darkblue}{\textbf{\ipa{-gɤ˧bi\#˥}}}}{}
\textcolor{teal}{\mytextsc{postposition}} \hspace{4pt} Ton~: \#H.
\textcolor{PineGreen}{\selectlanguage{french}Sur, dessus.} \zh{上面。}  ¶ \textcolor{darkblue}{\textbf{\ipa{ʑi˧qʰwɤ˧-gɤ˧bi˧}}} \textcolor{PineGreen}{\selectlanguage{french}sur la maison, sur le toit} \zh{在房頂上}  

\lhead{\firstmark}
\rhead{\botmark}

\subsection{\hspace{-0.5cm} {\Large \textcolor{darkblue}{\textbf{\ipa{gɤ˧lɑ˧}}}}\hspace{0.5cm}[\kern2pt{\textcolor{darkblue}{\textbf{\ipa{gɤ˩lɑ˥}}}}\kern2pt]} \hypertarget{g7\string_MlA\string_M1}{}
\markboth{\textcolor{darkblue}{\textbf{\ipa{gɤ˧lɑ˧}}}}{}
\textcolor{teal}{\mytextsc{nom}} \hspace{4pt} Ton~: M.
\textcolor{PineGreen}{\selectlanguage{french}Dieu, bouddha, bodhisattva.} \zh{神,菩萨,佛。}  Emprunt~: tibétain  lha
 \zh{量词}: \textcolor{darkblue}{\textbf{\ipa{v̩˧}}}  \mytextsc{clf}~: \textcolor{darkblue}{\textbf{\ipa{v̩˧}}} 
\lhead{\firstmark}
\rhead{\botmark}

\subsection{\hspace{-0.5cm} {\Large \textcolor{darkblue}{\textbf{\ipa{gɤ˧lɑ˧-pɤ\#˥}}}}\hspace{0.5cm}[\kern2pt{\textcolor{darkblue}{\textbf{\ipa{xxxx non-correspondance entre le nombre de morphèmes et le nombre de tons de morphèmes}}}}\kern2pt]} \hypertarget{g7\string_MlA\string_M-p7\#\string_T1}{}
\markboth{\textcolor{darkblue}{\textbf{\ipa{gɤ˧lɑ˧-pɤ\#˥}}}}{}
\textcolor{teal}{\mytextsc{nom}} \hspace{4pt} Ton~: \#H.
\textcolor{PineGreen}{\selectlanguage{french}Image du bouddha.} \zh{佛像。}  \zh{量词}: \textcolor{darkblue}{\textbf{\ipa{pɤ˥}}}  \mytextsc{clf}~: \textcolor{darkblue}{\textbf{\ipa{pɤ˥}}} 
\lhead{\firstmark}
\rhead{\botmark}

\subsection{\hspace{-0.5cm} {\Large \textcolor{darkblue}{\textbf{\ipa{gɤ˧lɑ˧-ʑi˩}}}}\hspace{0.5cm}[\kern2pt{\textcolor{darkblue}{\textbf{\ipa{xxxx non-correspondance entre le nombre de morphèmes et le nombre de tons de morphèmes}}}}\kern2pt]} \hypertarget{g7\string_MlA\string_M-z£i\string_B1}{}
\markboth{\textcolor{darkblue}{\textbf{\ipa{gɤ˧lɑ˧-ʑi˩}}}}{}
\textcolor{teal}{\mytextsc{nom}} \hspace{4pt} Ton~: \mytextsc{L}.
\textcolor{PineGreen}{\selectlanguage{french}Pièce du culte: pièce des esprits, pièce des ancêtres, où se trouve un autel. Un rituel y est effectué chaque matin. Le nom désigne par extension l'intégralité d'un des quatre bâtiments de la ferme traditionnelle na.} \zh{经堂(拜佛、拜祖先的房间)。}  ¶ \textcolor{darkblue}{\textbf{\ipa{gɤ˧lɑ˧-ʑi˩-di˩}}} \textcolor{PineGreen}{\selectlanguage{french}même sens} \zh{同上}  
 \zh{量词}: \textcolor{darkblue}{\textbf{\ipa{ɭɯ˧}}}  \mytextsc{clf}~: \textcolor{darkblue}{\textbf{\ipa{ɭɯ˧}}} 
\lhead{\firstmark}
\rhead{\botmark}

\subsection{\hspace{-0.5cm} {\Large \textcolor{darkblue}{\textbf{\ipa{gɤ˧qo˥}}}}\hspace{0.5cm}[\kern2pt{\textcolor{darkblue}{\textbf{\ipa{gɤ˧qo˧˥}}}}\kern2pt]} \hypertarget{g7\string_Mqo\string_T1}{}
\markboth{\textcolor{darkblue}{\textbf{\ipa{gɤ˧qo˥}}}}{}
\textcolor{teal}{\mytextsc{nom}} \hspace{4pt} Ton~: MH.
\textcolor{PineGreen}{\selectlanguage{french}Haut du foyer: partie de la pièce où l’on prend les repas, autour du foyer; c'est une structure en bois, surélevée d’une vingtaine de centimètres par rapport au sol cimenté.} \zh{主屋的高处:人吃饭的地方。}  \zh{量词}: \textcolor{darkblue}{\textbf{\ipa{ɭɯ˧}}}  \mytextsc{clf}~: \textcolor{darkblue}{\textbf{\ipa{ɭɯ˧}}} 
\lhead{\firstmark}
\rhead{\botmark}

\subsection{\hspace{-0.5cm} {\Large \textcolor{darkblue}{\textbf{\ipa{gɤ˩‑}}}}\hspace{0.5cm}[\kern2pt{\textcolor{darkblue}{\textbf{\ipa{gɤ˩˥}}}}\kern2pt]} \hypertarget{g7\string_B‑1}{}
\markboth{\textcolor{darkblue}{\textbf{\ipa{gɤ˩‑}}}}{}
\textcolor{teal}{\mytextsc{adverbe}} \hspace{4pt} Ton~: L.
\textcolor{PineGreen}{\selectlanguage{french}Préfixe directionnel: vers le haut.} \zh{向上、往上。} \textit{Voir~:} \textcolor{darkblue}{\textbf{\ipa{gɤ˩-qo˧, gɤ˩-tʰv̩˧qo˧, gɤ˩-ʈʂʰɯ˧qo˧}}} 
\lhead{\firstmark}
\rhead{\botmark}

\subsection{\hspace{-0.5cm} {\Large \textcolor{darkblue}{\textbf{\ipa{gɤ˩}}}}\hspace{0.5cm}[\kern2pt{\textcolor{darkblue}{\textbf{\ipa{gɤ˥}}}}\kern2pt]} \hypertarget{g7\string_B1}{}
\markboth{\textcolor{darkblue}{\textbf{\ipa{gɤ˩}}}}{}
\textcolor{teal}{\mytextsc{adjectif}} \hspace{4pt} Ton~: L.
\textcolor{PineGreen}{\selectlanguage{french}Querelleur, belliqueux, batailleur. Ce terme s'emploie au sujet des signes astrologiques: certains sont considérés comme 'bagarreurs', comme le Tigre et le Singe, ce qui rend les personnes nées cette année-là peu appropriées pour certains rites (ex.: lors du rite de passage à l'âge adulte), et au contraire très prisés pour d'autres.} \zh{爱吵架。}  ¶ \textcolor{darkblue}{\textbf{\ipa{kʰv̩˧ gɤ˧˥}}} \textcolor{PineGreen}{\selectlanguage{french}signe belliqueux (concept astrologique: certains signes confèrent aux gens nés l'année correspondante un caractère dur/belliqueux)} \zh{爱打架的年份/生肖:十二个生肖中,虎、猴……被认为是爱打架的。}  
 ¶ \textcolor{darkblue}{\textbf{\ipa{kʰv̩˧ gɤ˧-hĩ˥}}} \textcolor{PineGreen}{\selectlanguage{french}personne d'une année batailleuse} \zh{属一个爱打架的年份/生肖的人。十二个生肖中,虎、猴……被认为是爱打架的。}  
 ¶ \textcolor{darkblue}{\textbf{\ipa{ʑi˩hṽ˥, | lɑ˧ : | kʰv̩˧ gɤ˧˥!}}} \textcolor{PineGreen}{\selectlanguage{french}Les signes astrologiques du Singe et du Tigre sont des signes batailleurs!} \zh{属猴和属虎的人很爱吵架!}  
 ¶ \textcolor{darkblue}{\textbf{\ipa{ʑi˩˥, | lɑ˧, | kʰv̩˧ gɤ˧˥!}}} \textcolor{PineGreen}{\selectlanguage{french}Même sens que ci-dessus; formulation modernisée par l'enquêteur, utilisant le terme usuel pour 'singe'.} \zh{同上}  

\lhead{\firstmark}
\rhead{\botmark}

\subsection{\hspace{-0.5cm} {\Large \textcolor{darkblue}{\textbf{\ipa{gɤ˩\textsubscript{a}}}} \textsubscript{1}}\hspace{0.5cm}[\kern2pt{\textcolor{darkblue}{\textbf{\ipa{gɤ˩˥}}}}\kern2pt]} \hypertarget{g7\string_Ba1}{}
\markboth{\textcolor{darkblue}{\textbf{\ipa{gɤ˩\textsubscript{a}}}} \textsubscript{1}}{}
\textcolor{teal}{\mytextsc{verbe}} \hspace{4pt} Ton~: L\textsubscript{a}.
\textcolor{PineGreen}{\selectlanguage{french}S’éteindre.} \zh{灭,熄。}  ¶ \textcolor{darkblue}{\textbf{\ipa{mv̩˧ | le˧-gɤ˩(-ze˩)}}} \textcolor{PineGreen}{\selectlanguage{french}Le feu s'est éteint.} \zh{火灭了。}  

\lhead{\firstmark}
\rhead{\botmark}

\subsection{\hspace{-0.5cm} {\Large \textcolor{darkblue}{\textbf{\ipa{gɤ˩\textsubscript{a}}}} \textsubscript{2}}\hspace{0.5cm}[\kern2pt{\textcolor{darkblue}{\textbf{\ipa{gɤ˩˥}}}}\kern2pt]} \hypertarget{g7\string_Ba2}{}
\markboth{\textcolor{darkblue}{\textbf{\ipa{gɤ˩\textsubscript{a}}}} \textsubscript{2}}{}
\textcolor{teal}{\mytextsc{verbe}} \hspace{4pt} Ton~: L\textsubscript{a}.
\textcolor{PineGreen}{\selectlanguage{french}Être satisfait, content (de son sort), heureux.} \zh{满意,幸福,甘心,服气。}  ¶ \textcolor{darkblue}{\textbf{\ipa{hɤ˩-zo˥, | le˧-gɤ˩-ze˩!}}} \textcolor{PineGreen}{\selectlanguage{french}(il a réussi) habilement (ce qu'il voulait faire), il est content/satisfait!} \zh{很成功,真高兴! / 他成功了,很满意!}  
 ¶ \textcolor{darkblue}{\textbf{\ipa{ʈʂʰɯ˧ | ɖwæ˧˥ | le˧-gɤ˩-ze˩!}}} \textcolor{PineGreen}{\selectlanguage{french}il est très content!} \zh{他很满意!}  
 ¶ \textcolor{darkblue}{\textbf{\ipa{no˩-se˥, | ɖwæ˧˥ | le˧-gɤ˩-ze˩: | zo˧mv̩˥ hɤ˩-zo˩!}}} \textcolor{PineGreen}{\selectlanguage{french}Vous, vous avez bien de la chance/vous avez toutes raisons d'être satisfait(e)/vous avez des sujets de satisfaction: vos enfants sont brillants/habiles!} \zh{你呢,(应该)很满意:(你的)孩子很成功!}  
 ¶ \textcolor{darkblue}{\textbf{\ipa{mɤ˧-gɤ˩}}} \textcolor{PineGreen}{\selectlanguage{french}être mécontent, ne pas se résigner, être récalcitrant} \zh{不满意、不甘心、不服气}  

\lhead{\firstmark}
\rhead{\botmark}

\subsection{\hspace{-0.5cm} {\Large \textcolor{darkblue}{\textbf{\ipa{gɤ˩\textsubscript{a}}}} \textsubscript{3}}\hspace{0.5cm}[\kern2pt{\textcolor{darkblue}{\textbf{\ipa{gɤ˩˥}}}}\kern2pt]} \hypertarget{g7\string_Ba3}{}
\markboth{\textcolor{darkblue}{\textbf{\ipa{gɤ˩\textsubscript{a}}}} \textsubscript{3}}{}
\textcolor{teal}{\mytextsc{adjectif}} \hspace{4pt} Ton~: L\textsubscript{a}.
\textcolor{PineGreen}{\selectlanguage{french}Surpris, étonné, abasourdi; terrifié.} \zh{震惊。}  ¶ \textcolor{darkblue}{\textbf{\ipa{le˧-gɤ˩-ze˩}}} \textcolor{PineGreen}{\selectlanguage{french}\mytextsc{accomp} \string_ \mytextsc{pfv}} \zh{震惊了}  
 ¶ \textcolor{darkblue}{\textbf{\ipa{no˧ | hĩ˧ gɤ˧-kʰɯ˥!}}} \textcolor{PineGreen}{\selectlanguage{french}Tu fais peur aux gens!} \zh{你让人害怕!}  

\lhead{\firstmark}
\rhead{\botmark}

\subsection{\hspace{-0.5cm} {\Large \textcolor{darkblue}{\textbf{\ipa{gɤ˩bv̩˧}}}}\hspace{0.5cm}[\kern2pt{\textcolor{darkblue}{\textbf{\ipa{gɤ˧bv̩˧}}}}\kern2pt]} \hypertarget{g7\string_Bbv\string_=\string_M1}{}
\markboth{\textcolor{darkblue}{\textbf{\ipa{gɤ˩bv̩˧}}}}{}
\textcolor{teal}{\mytextsc{verbe}} \hspace{4pt} Ton~: LM.
\textcolor{PineGreen}{\selectlanguage{french}Déborder.} \zh{溢出来。}  ¶ \textcolor{darkblue}{\textbf{\ipa{gɤ˩bv̩˧-ze˩}}} \textcolor{PineGreen}{\selectlanguage{french}\mytextsc{pfv}} \zh{溢出来了}  

\lhead{\firstmark}
\rhead{\botmark}

\subsection{\hspace{-0.5cm} {\Large \textcolor{darkblue}{\textbf{\ipa{gɤ˩dzɤ˧}}}}\hspace{0.5cm}[\kern2pt{\textcolor{darkblue}{\textbf{\ipa{gɤ˩dzɤ˥}}}}\kern2pt]} \hypertarget{g7\string_Bdz7\string_M1}{}
\markboth{\textcolor{darkblue}{\textbf{\ipa{gɤ˩dzɤ˧}}}}{}
\textcolor{teal}{\mytextsc{adverbe}} \hspace{4pt} Ton~: LM.
\textcolor{PineGreen}{\selectlanguage{french}Haut, partie supérieure, partie noble (d'une salle, d'une tablée…) (symboliquement: “la tête”).} \zh{在上部分,上座。}  ¶ \textcolor{darkblue}{\textbf{\ipa{gɤ˩dzɤ˧ dzi˧˥}}} \textcolor{PineGreen}{\selectlanguage{french}être assis à une place d'honneur} \zh{坐上座}  
 ¶ \textcolor{darkblue}{\textbf{\ipa{no˧ | gɤ˩dzɤ˧ dzi˧˥!}}} \textcolor{PineGreen}{\selectlanguage{french}Veuillez vous installer à l'une des premières places! / Veuillez prendre l'une des places d'honneur!} \zh{请您坐在上座!}  
\textit{Voir~:} \textcolor{darkblue}{\textbf{\ipa{gɤ˩-}}} 
\lhead{\firstmark}
\rhead{\botmark}

\subsection{\hspace{-0.5cm} {\Large \textcolor{darkblue}{\textbf{\ipa{gɤ˩-qo˧}}}}\hspace{0.5cm}[\kern2pt{\textcolor{darkblue}{\textbf{\ipa{xxxx non-correspondance entre le nombre de morphèmes et le nombre de tons de morphèmes}}}}\kern2pt]} \hypertarget{g7\string_B-qo\string_M1}{}
\markboth{\textcolor{darkblue}{\textbf{\ipa{gɤ˩-qo˧}}}}{}
\textcolor{teal}{\mytextsc{adverbe}} \hspace{4pt} Ton~: M.
\textcolor{PineGreen}{\selectlanguage{french}Par là-bas tout en haut.} \zh{那上面(指高处)。} \textit{Voir~:} \textcolor{darkblue}{\textbf{\ipa{gɤ˩-, gɤ˩-ʈʂʰɯ˧qo˧, gɤ˩-tʰv̩˧qo˧,}}} 
\lhead{\firstmark}
\rhead{\botmark}

\subsection{\hspace{-0.5cm} {\Large \textcolor{darkblue}{\textbf{\ipa{gɤ˩qwɤ˧}}}}\hspace{0.5cm}[\kern2pt{\textcolor{darkblue}{\textbf{\ipa{gɤ˩qwɤ˥}}}}\kern2pt]} \hypertarget{g7\string_Bqw7\string_M1}{}
\markboth{\textcolor{darkblue}{\textbf{\ipa{gɤ˩qwɤ˧}}}}{}
\textcolor{teal}{\mytextsc{nom}} \hspace{4pt} Ton~: LM.
\textcolor{PineGreen}{\selectlanguage{french}Autel en contrehaut du foyer, où on dépose les présents qu'apportent les invités/les membres de la famille, les offrant ainsi aux ancêtres.} \zh{火炉旁边的祭坛(上面摆礼物等)。}  \zh{量词}: \textcolor{darkblue}{\textbf{\ipa{ɭɯ˧}}}  \mytextsc{clf}~: \textcolor{darkblue}{\textbf{\ipa{ɭɯ˧}}} 
\lhead{\firstmark}
\rhead{\botmark}

\subsection{\hspace{-0.5cm} {\Large \textcolor{darkblue}{\textbf{\ipa{gɤ˩ʁwɤ\#˥}}}}\hspace{0.5cm}[\kern2pt{\textcolor{darkblue}{\textbf{\ipa{gɤ˩ʁwɤ˥}}}}\kern2pt]} \hypertarget{g7\string_BRw7\#\string_T1}{}
\markboth{\textcolor{darkblue}{\textbf{\ipa{gɤ˩ʁwɤ\#˥}}}}{}
\textcolor{teal}{\mytextsc{nom}} \hspace{4pt} Ton~: LM+\#H.
\textcolor{PineGreen}{\selectlanguage{french}Nom de village; en chinois: Gewa.} \zh{格瓦村:永宁的一个村落。直译:上村。音译:格瓦。} 
\lhead{\firstmark}
\rhead{\botmark}

\subsection{\hspace{-0.5cm} {\Large \textcolor{darkblue}{\textbf{\ipa{gɤ˩ʁwɤ˧}}}}\hspace{0.5cm}[\kern2pt{\textcolor{darkblue}{\textbf{\ipa{gɤ˩ʁwɤ˥}}}}\kern2pt]} \hypertarget{g7\string_BRw7\string_M1}{}
\markboth{\textcolor{darkblue}{\textbf{\ipa{gɤ˩ʁwɤ˧}}}}{}
\textcolor{teal}{\mytextsc{nom}} \hspace{4pt} Ton~: LM.
\textcolor{PineGreen}{\selectlanguage{french}Cours supérieur (d'une rivière), amont.} \zh{上游。} 
\lhead{\firstmark}
\rhead{\botmark}

\subsection{\hspace{-0.5cm} {\Large \textcolor{darkblue}{\textbf{\ipa{gɤ˩-tʰv̩˧-gi\#˥}}}}\hspace{0.5cm}[\kern2pt{\textcolor{darkblue}{\textbf{\ipa{xxxx non-correspondance entre le nombre de morphèmes et le nombre de tons de morphèmes}}}}\kern2pt]} \hypertarget{g7\string_B-t\string_hv\string_=\string_M-gi\#\string_T1}{}
\markboth{\textcolor{darkblue}{\textbf{\ipa{gɤ˩-tʰv̩˧-gi\#˥}}}}{}
\textcolor{teal}{\mytextsc{adverbe}} \hspace{4pt} Ton~: L-\#H.
\textcolor{PineGreen}{\selectlanguage{french}Au loin par là-haut, de ce côté tout là-haut.} \zh{那里(指高处)。} \textit{Voir~:} \textcolor{darkblue}{\textbf{\ipa{gɤ˩‑, gɤ˩-qo˧, gɤ˩-tʰv̩˧qo˧, gɤ˩-ʈʂʰɯ˧qo˧}}} 
\lhead{\firstmark}
\rhead{\botmark}

\subsection{\hspace{-0.5cm} {\Large \textcolor{darkblue}{\textbf{\ipa{gɤ˩-tʰv̩˧qo˧}}}}\hspace{0.5cm}[\kern2pt{\textcolor{darkblue}{\textbf{\ipa{gɤ˩tʰv̩˧qo˧}}}}\kern2pt]} \hypertarget{g7\string_B-t\string_hv\string_=\string_Mqo\string_M1}{}
\markboth{\textcolor{darkblue}{\textbf{\ipa{gɤ˩-tʰv̩˧qo˧}}}}{}
\textcolor{teal}{\mytextsc{adverbe}} \hspace{4pt} Ton~: L-\#H.
\textcolor{PineGreen}{\selectlanguage{french}Au loin par là-haut, de ce côté tout là-haut.} \zh{那里(指高处)。} \textit{Voir~:} \textcolor{darkblue}{\textbf{\ipa{gɤ˩‑, gɤ˩-qo˧, gɤ˩-ʈʂʰɯ˧qo˧}}} 
\lhead{\firstmark}
\rhead{\botmark}

\subsection{\hspace{-0.5cm} {\Large \textcolor{darkblue}{\textbf{\ipa{gɤ˩ʈʂæ˧˥}}}}\hspace{0.5cm}[\kern2pt{\textcolor{darkblue}{\textbf{\ipa{gɤ˩ʈʂæ˧˥}}}}\kern2pt]} \hypertarget{g7\string_Bt`s`\{\string_M\string_T1}{}
\markboth{\textcolor{darkblue}{\textbf{\ipa{gɤ˩ʈʂæ˧˥}}}}{}
\textcolor{teal}{\mytextsc{nom}} \hspace{4pt} Ton~: LM+MH\#.
\textcolor{PineGreen}{\selectlanguage{french}Haut du corps, partie supérieure du corps.} \zh{上半(身)。} 
\lhead{\firstmark}
\rhead{\botmark}

\subsection{\hspace{-0.5cm} {\Large \textcolor{darkblue}{\textbf{\ipa{gɤ˩ʈʂʰæ˧-hĩ˧˥}}}}\hspace{0.5cm}[\kern2pt{\textcolor{darkblue}{\textbf{\ipa{xxxx non-correspondance entre le nombre de morphèmes et le nombre de tons de morphèmes}}}}\kern2pt]} \hypertarget{g7\string_Bt`s`\string_h\{\string_M-hi\string_~\string_M\string_T1}{}
\markboth{\textcolor{darkblue}{\textbf{\ipa{gɤ˩ʈʂʰæ˧-hĩ˧˥}}}}{}
\textcolor{teal}{\mytextsc{nom}} \hspace{4pt} Ton~: LM+MH\#.
\textcolor{PineGreen}{\selectlanguage{french}Les générations passées, les ancêtres.} \zh{祖先。} 
\lhead{\firstmark}
\rhead{\botmark}

\subsection{\hspace{-0.5cm} {\Large \textcolor{darkblue}{\textbf{\ipa{gɤ˩-ʈʂʰɯ˧-gi\#˥}}}}\hspace{0.5cm}[\kern2pt{\textcolor{darkblue}{\textbf{\ipa{xxxx non-correspondance entre le nombre de morphèmes et le nombre de tons de morphèmes}}}}\kern2pt]} \hypertarget{g7\string_B-t`s`\string_hM\string_M-gi\#\string_T1}{}
\markboth{\textcolor{darkblue}{\textbf{\ipa{gɤ˩-ʈʂʰɯ˧-gi\#˥}}}}{}
\textcolor{teal}{\mytextsc{adverbe}} \hspace{4pt} Ton~: L-\#H.
\textcolor{PineGreen}{\selectlanguage{french}Au loin par là-haut, de ce côté tout là-haut.} \zh{那里(指高处)。} \textit{Voir~:} \textcolor{darkblue}{\textbf{\ipa{gɤ˩-, gɤ˩-qo˧, gɤ˩-tʰv̩˧qo˧, gɤ˩-ʈʂʰɯ˧qo˧}}} 
\lhead{\firstmark}
\rhead{\botmark}

\subsection{\hspace{-0.5cm} {\Large \textcolor{darkblue}{\textbf{\ipa{gɤ˩-ʈʂʰɯ˧qo˧}}}}\hspace{0.5cm}[\kern2pt{\textcolor{darkblue}{\textbf{\ipa{gɤ˩ʈʂʰɯ˧qo˧}}}}\kern2pt]} \hypertarget{g7\string_B-t`s`\string_hM\string_Mqo\string_M1}{}
\markboth{\textcolor{darkblue}{\textbf{\ipa{gɤ˩-ʈʂʰɯ˧qo˧}}}}{}
\textcolor{teal}{\mytextsc{adverbe}} \hspace{4pt} Ton~: L-\#H.
\textcolor{PineGreen}{\selectlanguage{french}Au loin par là-haut, de ce côté tout là-haut.} \zh{那里(指高处)。} \textit{Voir~:} \textcolor{darkblue}{\textbf{\ipa{gɤ˩-, gɤ˩-qo˧, gɤ˩-tʰv̩˧qo˧}}} 
\lhead{\firstmark}
\rhead{\botmark}

\subsection{\hspace{-0.5cm} {\Large \textcolor{darkblue}{\textbf{\ipa{gɤ˧˥}}}}\hspace{0.5cm}[\kern2pt{\textcolor{darkblue}{\textbf{\ipa{gɤ˥}}}}\kern2pt]} \hypertarget{g7\string_M\string_T1}{}
\markboth{\textcolor{darkblue}{\textbf{\ipa{gɤ˧˥}}}}{}
\textcolor{teal}{\mytextsc{verbe}} \hspace{4pt} Ton~: MH.
\textcolor{PineGreen}{\selectlanguage{french}Porter à l’épaule; porter sur une palanche.} \zh{扛,担。}  ¶ \textcolor{darkblue}{\textbf{\ipa{tʰi˧-gɤ˧˥}}} \textcolor{PineGreen}{\selectlanguage{french}\mytextsc{dur}} \zh{\mytextsc{dur}}  
 ¶ \textcolor{darkblue}{\textbf{\ipa{tʰi˧-gɤ˧-ze˥}}} \textcolor{PineGreen}{\selectlanguage{french}\mytextsc{dur} \string_ \mytextsc{pfv}} \zh{\mytextsc{dur} \string_ \mytextsc{pfv}}  
 ¶ \textcolor{darkblue}{\textbf{\ipa{le˧-gɤ˧-ze˥}}} \textcolor{PineGreen}{\selectlanguage{french}\mytextsc{accomp} \string_ \mytextsc{pfv}} \zh{扛了}  
 ¶ \textcolor{darkblue}{\textbf{\ipa{tso˧\textasciitilde{}tso˧ gɤ˩}}} \textcolor{PineGreen}{\selectlanguage{french}porter quelque chose à l'épaule} \zh{扛东西}  
 ¶ \textcolor{darkblue}{\textbf{\ipa{njɤ˧(-ɳɯ˧) | gɤ˧-bi˥!}}} \textcolor{PineGreen}{\selectlanguage{french}C'est moi qui porte!} \zh{我来扛吧!}  

\lhead{\firstmark}
\rhead{\botmark}

\subsection{\hspace{-0.5cm} {\Large \textcolor{darkblue}{\textbf{\ipa{gi˥}}} \textsubscript{1}}\hspace{0.5cm}[\kern2pt{\textcolor{darkblue}{\textbf{\ipa{gi˥}}}}\kern2pt]} \hypertarget{gi\string_T1}{}
\markboth{\textcolor{darkblue}{\textbf{\ipa{gi˥}}} \textsubscript{1}}{}
\textcolor{teal}{\mytextsc{verbe}} \hspace{4pt} Ton~: H.
\textcolor{PineGreen}{\selectlanguage{french}Tomber (neige, pluie), neiger, pleuvoir.} \zh{下(雨,雪)。}  ¶ \textcolor{darkblue}{\textbf{\ipa{bi˧ gi˧. / bi˧ gi˧-ze˩.}}} \textcolor{PineGreen}{\selectlanguage{french}Il neige. / Il a neigé.} \zh{下雪。 / 下雪了。}  
 ¶ \textcolor{darkblue}{\textbf{\ipa{hi˩ gi˩˥. / hi˩ gi˩-ze˥.}}} \textcolor{PineGreen}{\selectlanguage{french}Il pleut. / Il a plu.} \zh{下雨。 / 下雨了。}  
 ¶ \textcolor{darkblue}{\textbf{\ipa{tsʰi˧-ɲi˧-dʑo˩, | hi˩ gi˩-ze˥, | le˧-gɤ˩-ze˩!}}} \textcolor{PineGreen}{\selectlanguage{french}Aujourd'hui, il s'est mis à pleuvoir / il a plu; c'est bien! (Commentaire au sujet de la pluie qui est venue, après une longue période de sécheresse.)} \zh{今天,下雨了,真好!(情景:大旱灾过后,雨季终于来了,这对庄稼很好。)}  

\lhead{\firstmark}
\rhead{\botmark}

\subsection{\hspace{-0.5cm} {\Large \textcolor{darkblue}{\textbf{\ipa{gi˥}}} \textsubscript{2}}\hspace{0.5cm}[\kern2pt{\textcolor{darkblue}{\textbf{\ipa{gi˥}}}}\kern2pt]} \hypertarget{gi\string_T2}{}
\markboth{\textcolor{darkblue}{\textbf{\ipa{gi˥}}} \textsubscript{2}}{}
\textcolor{teal}{\mytextsc{verbe}} \hspace{4pt} Ton~: H.
\textcolor{PineGreen}{\selectlanguage{french}Devoir de l'argent, avoir des dettes.} \zh{欠(钱)。}  ¶ \textcolor{darkblue}{\textbf{\ipa{ɖʐe˧ | tʰi˧-gi˥}}} \textcolor{PineGreen}{\selectlanguage{french}devoir de l'argent} \zh{欠钱}  

\lhead{\firstmark}
\rhead{\botmark}

\subsection{\hspace{-0.5cm} {\Large \textcolor{darkblue}{\textbf{\ipa{gi˥\textsubscript{a}}}}}\hspace{0.5cm}[\kern2pt{\textcolor{darkblue}{\textbf{\ipa{gi˥}}}}\kern2pt]} \hypertarget{gi\string_Ta1}{}
\markboth{\textcolor{darkblue}{\textbf{\ipa{gi˥\textsubscript{a}}}}}{}
\textcolor{teal}{\mytextsc{classificateur}} \hspace{4pt} Ton~: H\textsubscript{a}.
\ding{202} \textcolor{PineGreen}{\selectlanguage{french}Une moitié, un demi.} \zh{量词:一半。}  ¶ \textcolor{darkblue}{\textbf{\ipa{ɖɯ˧-gi˥}}} \textcolor{PineGreen}{\selectlanguage{french}une moitié} \zh{一半}  
 ¶ \textcolor{darkblue}{\textbf{\ipa{tsʰe˩ʐv̩˩-gi˥}}} \textcolor{PineGreen}{\selectlanguage{french}quatorze moitiés (combinaison permettant de déterminer la catégorie tonale de ce classificateur: elle établit que le ton est H1 et non H2)} \zh{十四个半(注:这是为了确定调类而问的短语)}  
 ¶ \textcolor{darkblue}{\textbf{\ipa{tv̩˧tsʰɯ˧ | ɖɯ˧-gi˥}}} \textcolor{PineGreen}{\selectlanguage{french}la moitié du temps, la moitié de la durée} \zh{一半的时间}  
\ding{203} \textcolor{PineGreen}{\selectlanguage{french}Un côté (d'une pièce, d'une maison…); une direction.} \zh{量词:一面(房屋的一面)。}  ¶ \textcolor{darkblue}{\textbf{\ipa{ɖɯ˧-gi˧ hõ˧}}} \textcolor{PineGreen}{\selectlanguage{french}aller d'un certain côté, aller de son côté} \zh{往一个方向走、走自己的方向}  
 ¶ \textcolor{darkblue}{\textbf{\ipa{ɖɯ˧-v̩˧ | ɖɯ˧-gi˧ hɯ˧}}} \textcolor{PineGreen}{\selectlanguage{french}aller chacun de son côté; se séparer} \zh{分开:每个人去自己的方向}  

\lhead{\firstmark}
\rhead{\botmark}

\subsection{\hspace{-0.5cm} {\Large \textcolor{darkblue}{\textbf{\ipa{gi˧dʑɯ˧}}}}\hspace{0.5cm}[\kern2pt{\textcolor{darkblue}{\textbf{\ipa{gi˥}}}}\kern2pt]} \hypertarget{gi\string_Mdz£M\string_M1}{}
\markboth{\textcolor{darkblue}{\textbf{\ipa{gi˧dʑɯ˧}}}}{}
\textcolor{teal}{\mytextsc{nom}} \hspace{4pt} Ton~: M.
\textcolor{PineGreen}{\selectlanguage{french}Le fleuve Yangtze.} \zh{金沙江。}  ¶ \textcolor{darkblue}{\textbf{\ipa{gi˧dʑɯ˧-kʰi\#˥}}} \textcolor{PineGreen}{\selectlanguage{french}le bord du fleuve Yangtze: Fengke, Labai...} \zh{金沙江边:奉科,拉伯……}  
 ¶ \textcolor{darkblue}{\textbf{\ipa{gi˧dʑɯ˧-kʰi˧-hĩ\#˥}}} \textcolor{PineGreen}{\selectlanguage{french}les riverains du Yangtze: gens de Labai, Fengke...} \zh{金沙江边的人:奉科人,拉伯人……}  

\lhead{\firstmark}
\rhead{\botmark}

\subsection{\hspace{-0.5cm} {\Large \textcolor{darkblue}{\textbf{\ipa{gi˧-nɑ˧mi\#˥}}}}\hspace{0.5cm}[\kern2pt{\textcolor{darkblue}{\textbf{\ipa{xxxx non-correspondance entre le nombre de morphèmes et le nombre de tons de morphèmes}}}}\kern2pt]} \hypertarget{gi\string_M-nA\string_Mmi\#\string_T1}{}
\markboth{\textcolor{darkblue}{\textbf{\ipa{gi˧-nɑ˧mi\#˥}}}}{}
\textcolor{teal}{\mytextsc{nom}} \hspace{4pt} Ton~: \#H.
\textcolor{PineGreen}{\selectlanguage{french}Ours (mâle ou femelle). Il n'existe pas de terme désignant de façon non ambiguë une ourse.} \zh{熊,母熊。}  ¶ \textcolor{darkblue}{\textbf{\ipa{gi˧-nɑ˧mi˧ tʰv̩˧-pʰo˩}}} \textcolor{PineGreen}{\selectlanguage{french}\mytextsc{n}+\mytextsc{dem}+\mytextsc{clf}} \zh{这只熊}  
 \zh{量词}: \textcolor{darkblue}{\textbf{\ipa{pʰo˧˥}}}  \mytextsc{clf}~: \textcolor{darkblue}{\textbf{\ipa{pʰo˧˥}}} 
\lhead{\firstmark}
\rhead{\botmark}

\subsection{\hspace{-0.5cm} {\Large \textcolor{darkblue}{\textbf{\ipa{gi˧-nɑ˧mi˧-pʰv̩\#˥}}}}\hspace{0.5cm}[\kern2pt{\textcolor{darkblue}{\textbf{\ipa{xxxx non-correspondance entre le nombre de morphèmes et le nombre de tons de morphèmes}}}}\kern2pt]} \hypertarget{gi\string_M-nA\string_Mmi\string_M-p\string_hv\string_=\#\string_T1}{}
\markboth{\textcolor{darkblue}{\textbf{\ipa{gi˧-nɑ˧mi˧-pʰv̩\#˥}}}}{}
\textcolor{teal}{\mytextsc{nom}} \hspace{4pt} Ton~: \#H.
\textcolor{PineGreen}{\selectlanguage{french}Ours (mâle).} \zh{公熊。}  ¶ \textcolor{darkblue}{\textbf{\ipa{gi˧-nɑ˧mi˧-pʰv̩˧ tʰv̩˧-pʰo˩}}} \textcolor{PineGreen}{\selectlanguage{french}\mytextsc{n}+\mytextsc{dem}+\mytextsc{clf}} \zh{这只公熊}  
 \zh{量词}: \textcolor{darkblue}{\textbf{\ipa{pʰo˧˥}}}  \mytextsc{clf}~: \textcolor{darkblue}{\textbf{\ipa{pʰo˧˥}}} 
\lhead{\firstmark}
\rhead{\botmark}

\subsection{\hspace{-0.5cm} {\Large \textcolor{darkblue}{\textbf{\ipa{gi˧-nɑ˧mi˧-zo\#˥}}}}\hspace{0.5cm}[\kern2pt{\textcolor{darkblue}{\textbf{\ipa{xxxx non-correspondance entre le nombre de morphèmes et le nombre de tons de morphèmes}}}}\kern2pt]} \hypertarget{gi\string_M-nA\string_Mmi\string_M-zo\#\string_T1}{}
\markboth{\textcolor{darkblue}{\textbf{\ipa{gi˧-nɑ˧mi˧-zo\#˥}}}}{}
\textcolor{teal}{\mytextsc{nom}} \hspace{4pt} Ton~: \#H.
\textcolor{PineGreen}{\selectlanguage{french}Ourson (de sexe masculin).} \zh{小熊。}  ¶ \textcolor{darkblue}{\textbf{\ipa{gi˧-nɑ˧mi˧-zo˧ tʰv̩˧-ɭɯ\#˥}}} \textcolor{PineGreen}{\selectlanguage{french}\mytextsc{n}+\mytextsc{dem}+\mytextsc{clf}} \zh{这只小熊}  
 \zh{量词}: \textcolor{darkblue}{\textbf{\ipa{ɭɯ˧}}}  \mytextsc{clf}~: \textcolor{darkblue}{\textbf{\ipa{ɭɯ˧}}} 
\lhead{\firstmark}
\rhead{\botmark}

\subsection{\hspace{-0.5cm} {\Large \textcolor{darkblue}{\textbf{\ipa{gi˧zɯ\#˥}}}}\hspace{0.5cm}[\kern2pt{\textcolor{darkblue}{\textbf{\ipa{gi˧zɯ˧}}}}\kern2pt]} \hypertarget{gi\string_MzM\#\string_T1}{}
\markboth{\textcolor{darkblue}{\textbf{\ipa{gi˧zɯ\#˥}}}}{}
\textcolor{teal}{\mytextsc{nom}} \hspace{4pt} Ton~: \#H.
\textcolor{PineGreen}{\selectlanguage{french}Petit frère (employé aussi entre cousins).} \zh{弟弟(也可指更年轻的表弟)。}  ¶ \textcolor{darkblue}{\textbf{\ipa{gi˧zɯ˧=ɻæ˥}}} \textcolor{PineGreen}{\selectlanguage{french}\mytextsc{associatif}: les petits frères} \zh{联想复数:弟弟们,表弟们}  
 \zh{量词}: \textcolor{darkblue}{\textbf{\ipa{v̩˧}}}  \mytextsc{clf}~: \textcolor{darkblue}{\textbf{\ipa{v̩˧}}} 
\lhead{\firstmark}
\rhead{\botmark}

\subsection{\hspace{-0.5cm} {\Large \textcolor{darkblue}{\textbf{\ipa{gi˧zɯ˧-go˧mi\#˥}}}}\hspace{0.5cm}[\kern2pt{\textcolor{darkblue}{\textbf{\ipa{xxxx non-correspondance entre le nombre de morphèmes et le nombre de tons de morphèmes}}}}\kern2pt]} \hypertarget{gi\string_MzM\string_M-go\string_Mmi\#\string_T1}{}
\markboth{\textcolor{darkblue}{\textbf{\ipa{gi˧zɯ˧-go˧mi\#˥}}}}{}
\textcolor{teal}{\mytextsc{nom}} \hspace{4pt} Ton~: \#H.
\textcolor{PineGreen}{\selectlanguage{french}Cadets: petits frères+ petites sœurs.} \zh{弟弟妹妹。} 
\lhead{\firstmark}
\rhead{\botmark}

\subsection{\hspace{-0.5cm} {\Large \textcolor{darkblue}{\textbf{\ipa{gi˩}}}}\hspace{0.5cm}[\kern2pt{\textcolor{darkblue}{\textbf{\ipa{gi˥}}}}\kern2pt]} \hypertarget{gi\string_B1}{}
\markboth{\textcolor{darkblue}{\textbf{\ipa{gi˩}}}}{}
\textcolor{teal}{\mytextsc{nom}} \hspace{4pt} Ton~: L.
\textcolor{PineGreen}{\selectlanguage{french}Ours.} \zh{大熊。} 
\lhead{\firstmark}
\rhead{\botmark}

\subsection{\hspace{-0.5cm} {\Large \textcolor{darkblue}{\textbf{\ipa{gi˩}}}}\hspace{0.5cm}[\kern2pt{\textcolor{darkblue}{\textbf{\ipa{gi˥}}}}\kern2pt]} \hypertarget{gi\string_B1}{}
\markboth{\textcolor{darkblue}{\textbf{\ipa{gi˩}}}}{}
\textcolor{teal}{\mytextsc{nom}} \hspace{4pt} Ton~: L.
\textcolor{PineGreen}{\selectlanguage{french}Grenier à céréales; selon M23, est le lieu dans la maison où on stocke les céréales.} \zh{粮仓。}  ¶ \textcolor{darkblue}{\textbf{\ipa{gi˧mi˧}}} \textcolor{PineGreen}{\selectlanguage{french}grand grenier} \zh{大粮仓}  
 ¶ \textcolor{darkblue}{\textbf{\ipa{gi˩zo˩˥}}} \textcolor{PineGreen}{\selectlanguage{french}petit grenier} \zh{小粮仓}  
 ¶ \textcolor{darkblue}{\textbf{\ipa{njɤ˧ | gi˩ gv̩˩-zo˩-ho˥}}} \textcolor{PineGreen}{\selectlanguage{french}il va falloir que je répare le grenier à céréales!} \zh{我应该修粮仓!}  
 \zh{量词}: \textcolor{darkblue}{\textbf{\ipa{ɭɯ˧}}}  \mytextsc{clf}~: \textcolor{darkblue}{\textbf{\ipa{ɭɯ˧}}} 
\lhead{\firstmark}
\rhead{\botmark}

\subsection{\hspace{-0.5cm} {\Large \textcolor{darkblue}{\textbf{\ipa{gi˩\textsubscript{a}}}}}\hspace{0.5cm}[\kern2pt{\textcolor{darkblue}{\textbf{\ipa{gi˩˥}}}}\kern2pt]} \hypertarget{gi\string_Ba1}{}
\markboth{\textcolor{darkblue}{\textbf{\ipa{gi˩\textsubscript{a}}}}}{}
\textcolor{teal}{\mytextsc{adjectif}} \hspace{4pt} Ton~: L\textsubscript{a}.
\textit{De:} \textbf{/gɯ˩a 2/ et /ʝi˥/} \textcolor{PineGreen}{\selectlanguage{french}Vrai, vraiment.} \zh{真,真的。}  ¶ \textcolor{darkblue}{\textbf{\ipa{mɤ˧-gi˩!}}} \textcolor{PineGreen}{\selectlanguage{french}c'est pas vrai!} \zh{不是的! / 不是真的!}  
 ¶ \textcolor{darkblue}{\textbf{\ipa{ə˩-gi˩˥?}}} \textcolor{PineGreen}{\selectlanguage{french}c'est vrai? / n'est-ce pas?} \zh{对吧? / 对吗?}  
 ¶ \textcolor{darkblue}{\textbf{\ipa{ə˩-gi˩˥ ? – gi˩˥!}}} \textcolor{PineGreen}{\selectlanguage{french}N'est-ce pas? - Oui-da! (Le locuteur demande à son interlocuteur de confirmer qu’il adhère à son propos; l'autre donne son assentiment.)} \zh{对吧? -对的!}  
 ¶ \textcolor{darkblue}{\textbf{\ipa{gi˩-hĩ˩ ʐwɤ˥}}} \textcolor{PineGreen}{\selectlanguage{french}dire vrai} \zh{说实话,老实说}  
 ¶ \textcolor{darkblue}{\textbf{\ipa{gi˩˥ | -gɯ˩˥}}} \textcolor{PineGreen}{\selectlanguage{french}vraiment, véritablement} \zh{真的,真正的}  

\lhead{\firstmark}
\rhead{\botmark}

\subsection{\hspace{-0.5cm} {\Large \textcolor{darkblue}{\textbf{\ipa{gi˩kɯ˩}}}}\hspace{0.5cm}[\kern2pt{\textcolor{darkblue}{\textbf{\ipa{gi˧kɯ˧}}}}\kern2pt]} \hypertarget{gi\string_BkM\string_B1}{}
\markboth{\textcolor{darkblue}{\textbf{\ipa{gi˩kɯ˩}}}}{}
\textcolor{teal}{\mytextsc{nom}} \hspace{4pt} Ton~: L.
\textcolor{PineGreen}{\selectlanguage{french}Musc (littéralement: 'bile d'ours').} \zh{麝香(直译:大熊胆)。} 
\lhead{\firstmark}
\rhead{\botmark}

\subsection{\hspace{-0.5cm} {\Large \textcolor{darkblue}{\textbf{\ipa{‑gi˧˥}}}}\hspace{0.5cm}[\kern2pt{\textcolor{darkblue}{\textbf{\ipa{gi˧˥}}}}\kern2pt]} \hypertarget{‑gi\string_M\string_T1}{}
\markboth{\textcolor{darkblue}{\textbf{\ipa{‑gi˧˥}}}}{}
\textcolor{teal}{\mytextsc{postposition}} \hspace{4pt} Ton~: MH.
\textcolor{PineGreen}{\selectlanguage{french}Derrière.} \zh{后面,(最)后。}  ¶ \textcolor{darkblue}{\textbf{\ipa{ə˧mɑ˧-gi˧˥}}} \textcolor{PineGreen}{\selectlanguage{french}derrière maman} \zh{妈妈后面}  
 ¶ \textcolor{darkblue}{\textbf{\ipa{lɑ˧-gi˧˥}}} \textcolor{PineGreen}{\selectlanguage{french}derrière le tigre} \zh{老虎后面}  
 ¶ \textcolor{darkblue}{\textbf{\ipa{bo˩-gi˥}}} \textcolor{PineGreen}{\selectlanguage{french}derrière le cochon} \zh{猪后面}  
 ¶ \textcolor{darkblue}{\textbf{\ipa{mv̩˩-gi˥}}} \textcolor{PineGreen}{\selectlanguage{french}derrière la fille} \zh{女儿后面}  
 ¶ \textcolor{darkblue}{\textbf{\ipa{ʐwæ˧-gi˥}}} \textcolor{PineGreen}{\selectlanguage{french}derrière le cheval} \zh{马后面}  
 ¶ \textcolor{darkblue}{\textbf{\ipa{ʈʂʰɯ˧-gi˥ | tʰi˧-tɕʰo˩}}} \textcolor{PineGreen}{\selectlanguage{french}se cacher là-derrière} \zh{藏那后面}  
 ¶ \textcolor{darkblue}{\textbf{\ipa{no˧-gi˧ njɤ˥ ʈʂwæ˩!}}} \textcolor{PineGreen}{\selectlanguage{french}je te suis, je marche dans tes pas; je t'imite} \zh{我跟你走! / 我都按你说的来做吧!}  
 ¶ \textcolor{darkblue}{\textbf{\ipa{ɖɯ˧-v̩˧-gi˧˥, | ɖɯ˧-v̩˧ hwæ˧!}}} \textcolor{PineGreen}{\selectlanguage{french}en acheter un après l'autre (contexte: un caravanier achète des chevaux l'un après l'autre, afin de se constituer sa propre caravane)} \zh{一个接着一个地买(情景:一个人接二连三地买马,最后组成自己的马帮队)}  
 ¶ \textcolor{darkblue}{\textbf{\ipa{[F5] gi˧˥ | ɖɯ˧-qɑ˩ gv̩˩-bi˩!}}} \textcolor{PineGreen}{\selectlanguage{french}On va en faire une dernière botte! (contexte: on travaille le lin, botte après botte; vers la fin d'une longue séance de travail, quelqu'un annonce: “On va en faire une dernière botte! / Une dernière botte, et on s'arrête!”)} \zh{再做一捆吧!(情景:女人们在纺麻线,工作了很久,一个人就说:“再做最后一捆(就收工吧)!”)}  
 ¶ \textcolor{darkblue}{\textbf{\ipa{gi˧-se˧}}} \textcolor{PineGreen}{\selectlanguage{french}marcher derrière, suivre derrière} \zh{在后面走,在后面跟着}  

\lhead{\firstmark}
\rhead{\botmark}

\subsection{\hspace{-0.5cm} {\Large \textcolor{darkblue}{\textbf{\ipa{go˧bɤ˩}}}}\hspace{0.5cm}[\kern2pt{\textcolor{darkblue}{\textbf{\ipa{go˩bɤ˥}}}}\kern2pt]} \hypertarget{go\string_Mb7\string_B1}{}
\markboth{\textcolor{darkblue}{\textbf{\ipa{go˧bɤ˩}}}}{}
\textcolor{teal}{\mytextsc{nom}} \hspace{4pt} Ton~: L\#.
\textcolor{PineGreen}{\selectlanguage{french}Temple, monastère.} \zh{庙,寺。}  Emprunt~: tibétain  dgon pa
 \zh{量词}: \textcolor{darkblue}{\textbf{\ipa{ɭɯ˧}}}  \mytextsc{clf}~: \textcolor{darkblue}{\textbf{\ipa{ɭɯ˧}}} 
\lhead{\firstmark}
\rhead{\botmark}

\subsection{\hspace{-0.5cm} {\Large \textcolor{darkblue}{\textbf{\ipa{go˧mi˧}}}}\hspace{0.5cm}[\kern2pt{\textcolor{darkblue}{\textbf{\ipa{go˧mi˧}}}}\kern2pt]} \hypertarget{go\string_Mmi\string_M1}{}
\markboth{\textcolor{darkblue}{\textbf{\ipa{go˧mi˧}}}}{}
\textcolor{teal}{\mytextsc{nom}} \hspace{4pt} Ton~: M.
\textcolor{PineGreen}{\selectlanguage{french}Petite soeur (employé aussi pour les cousines plus jeunes).} \zh{妹妹。}  ¶ \textcolor{darkblue}{\textbf{\ipa{go˧mi˧=ɻæ˩}}} \textcolor{PineGreen}{\selectlanguage{french}\mytextsc{associatif}: les petites sœurs, les jeunes cousines} \zh{联想复数:妹妹们,表妹们}  
 \zh{量词}: \textcolor{darkblue}{\textbf{\ipa{v̩˧}}}  \mytextsc{clf}~: \textcolor{darkblue}{\textbf{\ipa{v̩˧}}} 
\lhead{\firstmark}
\rhead{\botmark}

\subsection{\hspace{-0.5cm} {\Large \textcolor{darkblue}{\textbf{\ipa{go˩\textsubscript{a}}}}}\hspace{0.5cm}[\kern2pt{\textcolor{darkblue}{\textbf{\ipa{go˧˥}}}}\kern2pt]} \hypertarget{go\string_Ba1}{}
\markboth{\textcolor{darkblue}{\textbf{\ipa{go˩\textsubscript{a}}}}}{}
\textcolor{teal}{\mytextsc{verbe}} \hspace{4pt} Ton~: L\textsubscript{a}.
\textcolor{PineGreen}{\selectlanguage{french}Souffrir, avoir mal; être malade.} \zh{痛,病 (生病)。}  ¶ \textcolor{darkblue}{\textbf{\ipa{njɤ˧ | go˩˥!}}} \textcolor{PineGreen}{\selectlanguage{french}J'ai mal!} \zh{我痛!}  
 ¶ \textcolor{darkblue}{\textbf{\ipa{njɤ˧ | go˩˥ | ʐwæ˩˥!}}} \textcolor{PineGreen}{\selectlanguage{french}J'ai très mal!} \zh{我好疼!}  
 ¶ \textcolor{darkblue}{\textbf{\ipa{go˩-hĩ˩˥}}} \textcolor{PineGreen}{\selectlanguage{french}\mytextsc{nmlz}: patient, malade} \zh{病人,病的(人)}  
 ¶ \textcolor{darkblue}{\textbf{\ipa{hĩ˧ | go˩-hĩ˩˥}}} \textcolor{PineGreen}{\selectlanguage{french}patient, personne malade, malade} \zh{病人}  
 ¶ \textcolor{darkblue}{\textbf{\ipa{bi˧mi˧ go˩}}} \textcolor{PineGreen}{\selectlanguage{french}avoir mal au ventre} \zh{肚子疼}  

\lhead{\firstmark}
\rhead{\botmark}

\subsection{\hspace{-0.5cm} {\Large \textcolor{darkblue}{\textbf{\ipa{go˩bi˧}}}}\hspace{0.5cm}[\kern2pt{\textcolor{darkblue}{\textbf{\ipa{go˧bi˧}}}}\kern2pt]} \hypertarget{go\string_Bbi\string_M1}{}
\markboth{\textcolor{darkblue}{\textbf{\ipa{go˩bi˧}}}}{}
\textcolor{teal}{\mytextsc{nom}} \hspace{4pt} Ton~: LM.
\textcolor{PineGreen}{\selectlanguage{french}Lijiang (la ville).} \zh{丽江城。}  ¶ \textcolor{darkblue}{\textbf{\ipa{go˩bi˧-ɖʐɯ˧qo˩}}} \textcolor{PineGreen}{\selectlanguage{french}la ville de Lijiang} \zh{丽江城}  

\lhead{\firstmark}
\rhead{\botmark}

\subsection{\hspace{-0.5cm} {\Large \textcolor{darkblue}{\textbf{\ipa{go˩bo˥}}}}\hspace{0.5cm}[\kern2pt{\textcolor{darkblue}{\textbf{\ipa{go˧bo˧}}}}\kern2pt]} \hypertarget{go\string_Bbo\string_T1}{}
\markboth{\textcolor{darkblue}{\textbf{\ipa{go˩bo˥}}}}{}
\textcolor{teal}{\mytextsc{nom}} \hspace{4pt} Ton~: LH.
\textcolor{PineGreen}{\selectlanguage{french}Bétail, animaux domestiques.} \zh{牲畜。}  \zh{量词}: \textcolor{darkblue}{\textbf{\ipa{pʰo˧˥}}}  \mytextsc{clf}~: \textcolor{darkblue}{\textbf{\ipa{pʰo˧˥}}} 
\lhead{\firstmark}
\rhead{\botmark}

\subsection{\hspace{-0.5cm} {\Large \textcolor{darkblue}{\textbf{\ipa{gɯ˩\textsubscript{a}}}} \textsubscript{1}}\hspace{0.5cm}[\kern2pt{\textcolor{darkblue}{\textbf{\ipa{gɯ˩˥}}}}\kern2pt]} \hypertarget{gM\string_Ba1}{}
\markboth{\textcolor{darkblue}{\textbf{\ipa{gɯ˩\textsubscript{a}}}} \textsubscript{1}}{}
\textcolor{teal}{\mytextsc{verbe}} \hspace{4pt} Ton~: L\textsubscript{a}.
\textcolor{PineGreen}{\selectlanguage{french}Croire.} \zh{相信。}  ¶ \textcolor{darkblue}{\textbf{\ipa{ʈʂʰɯ˧-ɳɯ˧ ʐwɤ˩-hĩ˩, | njɤ˧ | mɤ˧-gɯ˩!}}} \textcolor{PineGreen}{\selectlanguage{french}Je ne crois pas ce qu'il dit! / Je ne le crois pas! / Je ne crois pas un mot de ce qu'il raconte!} \zh{他说的话,我不相信!}  

\lhead{\firstmark}
\rhead{\botmark}

\subsection{\hspace{-0.5cm} {\Large \textcolor{darkblue}{\textbf{\ipa{gɯ˩\textsubscript{a}}}} \textsubscript{2}}\hspace{0.5cm}[\kern2pt{\textcolor{darkblue}{\textbf{\ipa{gɯ˩˥}}}}\kern2pt]} \hypertarget{gM\string_Ba2}{}
\markboth{\textcolor{darkblue}{\textbf{\ipa{gɯ˩\textsubscript{a}}}} \textsubscript{2}}{}
\textcolor{teal}{\mytextsc{adjectif}} \hspace{4pt} Ton~: L\textsubscript{a}.
\textcolor{PineGreen}{\selectlanguage{french}Vrai, authentique, véritable.} \zh{真,真的。}  ¶ \textcolor{darkblue}{\textbf{\ipa{mɤ˧-gɯ˩}}} \textcolor{PineGreen}{\selectlanguage{french}pas vrai} \zh{不是真的}  
 ¶ \textcolor{darkblue}{\textbf{\ipa{gɯ˩-hĩ˩˥}}} \textcolor{PineGreen}{\selectlanguage{french}\mytextsc{nmlz}} \zh{真的}  
 ¶ \textcolor{darkblue}{\textbf{\ipa{ə˩-gɯ˩˥?}}} \textcolor{PineGreen}{\selectlanguage{french}c'est vrai?} \zh{真的吗?}  
 ¶ \textcolor{darkblue}{\textbf{\ipa{gɯ˩ wɤ˩-ɻ̍˥!}}} \textcolor{PineGreen}{\selectlanguage{french}C'est bien ça! / C'est réellement ainsi!} \zh{就是真的啊! / 的确是这样啊!}  
 ¶ \textcolor{darkblue}{\textbf{\ipa{gɯ˩-ʝi˥?}}} \textcolor{PineGreen}{\selectlanguage{french}C'est vrai? Vraiment?} \zh{原来是这样吗?}  
 ¶ \textcolor{darkblue}{\textbf{\ipa{gɯ˩ ʂv̩˩ɖv̩˩˥}}} \textcolor{PineGreen}{\selectlanguage{french}croire (quelqu'un, quelque chose): littéralement “penser que c'est vrai”} \zh{相信}  
 ¶ \textcolor{darkblue}{\textbf{\ipa{gɯ˧ ʐwɤ˧}}} \textcolor{PineGreen}{\selectlanguage{french}dire la vérité} \zh{说实话}  

\lhead{\firstmark}
\rhead{\botmark}

\subsection{\hspace{-0.5cm} {\Large \textcolor{darkblue}{\textbf{\ipa{gɯ˩ɭɯ˧˥}}}}\hspace{0.5cm}[\kern2pt{\textcolor{darkblue}{\textbf{\ipa{gɯ˩ɭɯ˧˥}}}}\kern2pt]} \hypertarget{gM\string_Bl\string_RM\string_M\string_T1}{}
\markboth{\textcolor{darkblue}{\textbf{\ipa{gɯ˩ɭɯ˧˥}}}}{}
\textcolor{teal}{\mytextsc{verbe}} \hspace{4pt} Ton~: LM+MH\#.
\textcolor{PineGreen}{\selectlanguage{french}Frotter (ex.: se frotter les yeux, frotter un vêtement).} \zh{揉。}  ¶ \textcolor{darkblue}{\textbf{\ipa{gɯ˩ɭɯ˧-ze˥}}} \textcolor{PineGreen}{\selectlanguage{french}\mytextsc{pfv}} \zh{揉了}  
 ¶ \textcolor{darkblue}{\textbf{\ipa{le˧-gɯ˩ɭɯ˩+ze˩}}} \textcolor{PineGreen}{\selectlanguage{french}\mytextsc{accomp} \string_ \mytextsc{pfv}} \zh{揉了}  
 ¶ \textcolor{darkblue}{\textbf{\ipa{le˧-gɯ˩ɭɯ˩\textasciitilde{}le˧-gɯ˩ɭɯ˩}}} \textcolor{PineGreen}{\selectlanguage{french}\mytextsc{accomp} \mytextsc{red} \mytextsc{pfv}} \zh{揉一揉}  

\lhead{\firstmark}
\rhead{\botmark}

\subsection{\hspace{-0.5cm} {\Large \textcolor{darkblue}{\textbf{\ipa{gv̩˧}}} \textsubscript{1}}\hspace{0.5cm}[\kern2pt{\textcolor{darkblue}{\textbf{\ipa{gv̩˥}}}}\kern2pt]} \hypertarget{gv\string_=\string_M1}{}
\markboth{\textcolor{darkblue}{\textbf{\ipa{gv̩˧}}} \textsubscript{1}}{}
\textcolor{teal}{\mytextsc{verbe}} \hspace{4pt} Ton~: M\textsubscript{c}.
\textcolor{PineGreen}{\selectlanguage{french}S'écouler, passer (le temps passe); se passer (un événement).} \zh{过去 (时间)、过,发生。}  ¶ \textcolor{darkblue}{\textbf{\ipa{le˧-gv̩˩-ze˩}}} \textcolor{PineGreen}{\selectlanguage{french}\mytextsc{accomp} \string_ \mytextsc{pfv}} \zh{已经过去了}  
 ¶ \textcolor{darkblue}{\textbf{\ipa{ɖɯ˧-ɭɯ˧ gv̩˧}}} \textcolor{PineGreen}{\selectlanguage{french}une heure se passe} \zh{一个小时过去了}  
 ¶ \textcolor{darkblue}{\textbf{\ipa{tsʰe˩-ɲi˩ gv̩˩-ze˥!}}} \textcolor{PineGreen}{\selectlanguage{french}Dix jours ont passé!} \zh{十天过去了}  
 ¶ \textcolor{darkblue}{\textbf{\ipa{mɤ˧-gv̩˧-ze˧!}}} \textcolor{PineGreen}{\selectlanguage{french}(ah là là,) ça ne va plus!} \zh{不好了! / 不行了!}  
 ¶ \textcolor{darkblue}{\textbf{\ipa{ʈʂʰɯ˧ne˧-ʝi˥ | gv̩˧, -tsɯ˩-mv̩˩!}}} \textcolor{PineGreen}{\selectlanguage{french}Ca c'est passé comme ça, à ce qu'on raconte!} \zh{据说是这样发生的!}  

\lhead{\firstmark}
\rhead{\botmark}

\subsection{\hspace{-0.5cm} {\Large \textcolor{darkblue}{\textbf{\ipa{gv̩˧}}} \textsubscript{2}}\hspace{0.5cm}[\kern2pt{\textcolor{darkblue}{\textbf{\ipa{gv̩˥}}}}\kern2pt]} \hypertarget{gv\string_=\string_M2}{}
\markboth{\textcolor{darkblue}{\textbf{\ipa{gv̩˧}}} \textsubscript{2}}{}
\textcolor{teal}{\mytextsc{adjectif}} \hspace{4pt} Ton~: M.
\textcolor{PineGreen}{\selectlanguage{french}Bon (bon cœur).} \zh{好(心好)。}  ¶ \textcolor{darkblue}{\textbf{\ipa{ɖwæ˧˥ | gv̩˧!}}} \textcolor{PineGreen}{\selectlanguage{french}\mytextsc{intensif}.très} \zh{很好!}  
 ¶ \textcolor{darkblue}{\textbf{\ipa{mɤ˧-gv̩˧!}}} \textcolor{PineGreen}{\selectlanguage{french}\mytextsc{neg}} \zh{不好}  

\lhead{\firstmark}
\rhead{\botmark}

\subsection{\hspace{-0.5cm} {\Large \textcolor{darkblue}{\textbf{\ipa{gv̩˧}}} \textsubscript{3}}\hspace{0.5cm}[\kern2pt{\textcolor{darkblue}{\textbf{\ipa{gv̩˥}}}}\kern2pt]} \hypertarget{gv\string_=\string_M3}{}
\markboth{\textcolor{darkblue}{\textbf{\ipa{gv̩˧}}} \textsubscript{3}}{}
\textcolor{teal}{\mytextsc{nombre}} \hspace{4pt} Ton~: M? H\#? (pas L).
\textcolor{PineGreen}{\selectlanguage{french}9.} \zh{9。} 
\lhead{\firstmark}
\rhead{\botmark}

\subsection{\hspace{-0.5cm} {\Large \textcolor{darkblue}{\textbf{\ipa{gv̩˧}}} \textsubscript{4}}\hspace{0.5cm}[\kern2pt{\textcolor{darkblue}{\textbf{\ipa{gv̩˥}}}}\kern2pt]} \hypertarget{gv\string_=\string_M4}{}
\markboth{\textcolor{darkblue}{\textbf{\ipa{gv̩˧}}} \textsubscript{4}}{}
\textcolor{teal}{\mytextsc{verbe}} \hspace{4pt} Ton~: M.
\textcolor{PineGreen}{\selectlanguage{french}Être, devenir (verbe statif).} \zh{系词。}  ¶ \textcolor{darkblue}{\textbf{\ipa{ʈʂʰɯ˧ | no˧ | ɲi˧gɤ˧ | ʂwæ˧-mɤ˧-gv̩˧!}}} \textcolor{PineGreen}{\selectlanguage{french}elle a le nez moins droit que toi! (Au sujet d'une petite fille dont le nez ne ressemble pas au nez droit de son père)} \zh{她的鼻子没有你的直!(关于一个鼻子比较扁的小女孩)}  
 ¶ \textcolor{darkblue}{\textbf{\ipa{ʐæ˧ni˩ | mɤ˧-gv̩˧}}} \textcolor{PineGreen}{\selectlanguage{french}pas bien grand (en taille), pas bien impressionnant} \zh{个子不高}  

\lhead{\firstmark}
\rhead{\botmark}

\subsection{\hspace{-0.5cm} {\Large \textcolor{darkblue}{\textbf{\ipa{gv̩˥}}}}\hspace{0.5cm}[\kern2pt{\textcolor{darkblue}{\textbf{\ipa{gv̩˥}}}}\kern2pt]} \hypertarget{gv\string_=\string_T1}{}
\markboth{\textcolor{darkblue}{\textbf{\ipa{gv̩˥}}}}{}
\textcolor{teal}{\mytextsc{verbe}} \hspace{4pt} Ton~: H.
\textcolor{PineGreen}{\selectlanguage{french}Passer, traverser (un cours d'eau, un lac…).} \zh{过(一条河、一个湖……)。}  ¶ \textcolor{darkblue}{\textbf{\ipa{dʑɯ˩ gv̩˩˥}}} \textcolor{PineGreen}{\selectlanguage{french}traverser l'eau/traverser la rivière} \zh{过河}  

\lhead{\firstmark}
\rhead{\botmark}

\subsection{\hspace{-0.5cm} {\Large \textcolor{darkblue}{\textbf{\ipa{gv̩˥}}}}\hspace{0.5cm}[\kern2pt{\textcolor{darkblue}{\textbf{\ipa{gv̩˥}}}}\kern2pt]} \hypertarget{gv\string_=\string_T1}{}
\markboth{\textcolor{darkblue}{\textbf{\ipa{gv̩˥}}}}{}
\textcolor{teal}{\mytextsc{nom}} \hspace{4pt} Ton~: \#H.
\textcolor{PineGreen}{\selectlanguage{french}Auge, mangeoire.} \zh{马槽。}  ¶ \textcolor{darkblue}{\textbf{\ipa{ʐwæ˧gv̩\#˥}}} \textcolor{PineGreen}{\selectlanguage{french}auge du cheval} \zh{马槽}  
 \zh{量词}: \textcolor{darkblue}{\textbf{\ipa{ɭɯ˧}}}  \mytextsc{clf}~: \textcolor{darkblue}{\textbf{\ipa{ɭɯ˧}}} 
\lhead{\firstmark}
\rhead{\botmark}

\subsection{\hspace{-0.5cm} {\Large \textcolor{darkblue}{\textbf{\ipa{gv̩˩\textsubscript{a}}}} \textsubscript{1}}\hspace{0.5cm}[\kern2pt{\textcolor{darkblue}{\textbf{\ipa{gv̩˩˥}}}}\kern2pt]} \hypertarget{gv\string_=\string_Ba1}{}
\markboth{\textcolor{darkblue}{\textbf{\ipa{gv̩˩\textsubscript{a}}}} \textsubscript{1}}{}
\textcolor{teal}{\mytextsc{verbe}} \hspace{4pt} Ton~: L\textsubscript{a}.
\ding{202} \textcolor{PineGreen}{\selectlanguage{french}Cuisiner, préparer (un repas, de la nourriture).} \zh{做(饭)。}  ¶ \textcolor{darkblue}{\textbf{\ipa{hɑ˧ gv̩˥}}} \textcolor{PineGreen}{\selectlanguage{french}faire la cuisine, cuisiner} \zh{做饭}  
 ¶ \textcolor{darkblue}{\textbf{\ipa{le˧-gv̩˩-ze˩}}} \textcolor{PineGreen}{\selectlanguage{french}\mytextsc{accomp} \string_ \mytextsc{pfv}} \zh{做(饭)了}  
 ¶ \textcolor{darkblue}{\textbf{\ipa{njɤ˧ | hɑ˧ gv̩˥-bi˩!}}} \textcolor{PineGreen}{\selectlanguage{french}je vais faire la cuisine!} \zh{我来做饭吧!}  
\ding{203} \textcolor{PineGreen}{\selectlanguage{french}Construire (une maison).} \zh{盖、建 (房子)。}  ¶ \textcolor{darkblue}{\textbf{\ipa{ʑi˧qʰwɤ˧ gv̩˩}}} \textcolor{PineGreen}{\selectlanguage{french}construire un bâtiment} \zh{建房}  
\ding{204} \textcolor{PineGreen}{\selectlanguage{french}Fabriquer ou réparer (un outil).} \zh{修理、做出来(工具)。}  ¶ \textcolor{darkblue}{\textbf{\ipa{le˧-gv̩˧\textasciitilde{}gv̩˥}}} \textcolor{PineGreen}{\selectlanguage{french}\mytextsc{red}: réparer} \zh{\mytextsc{重叠:修理}}  
 ¶ \textcolor{darkblue}{\textbf{\ipa{le˧-gv̩˩ | le˧-tʰv̩˧-ze˧!}}} \textcolor{PineGreen}{\selectlanguage{french}Ca y est, c'est réparé/c'est fabriqué/c'est fini!} \zh{修理好了!/ 修理出来了!}  

\lhead{\firstmark}
\rhead{\botmark}

\subsection{\hspace{-0.5cm} {\Large \textcolor{darkblue}{\textbf{\ipa{gv̩˩\textsubscript{a}}}} \textsubscript{2}}\hspace{0.5cm}[\kern2pt{\textcolor{darkblue}{\textbf{\ipa{gv̩˩˥}}}}\kern2pt]} \hypertarget{gv\string_=\string_Ba2}{}
\markboth{\textcolor{darkblue}{\textbf{\ipa{gv̩˩\textsubscript{a}}}} \textsubscript{2}}{}
\textcolor{teal}{\mytextsc{verbe}} \hspace{4pt} Ton~: L\textsubscript{a}.
\textcolor{PineGreen}{\selectlanguage{french}Ranger.} \zh{收拾。}  ¶ \textcolor{darkblue}{\textbf{\ipa{tʰi˧-gv̩˧\textasciitilde{}gv̩˥}}} \textcolor{PineGreen}{\selectlanguage{french}\mytextsc{dur}} \zh{\mytextsc{dur}}  
 ¶ \textcolor{darkblue}{\textbf{\ipa{ɖɯ˧-gv̩˧\textasciitilde{}gv̩˥-ɻ̍˩}}} \textcolor{PineGreen}{\selectlanguage{french}ranger un peu} \zh{收拾一下}  
 ¶ \textcolor{darkblue}{\textbf{\ipa{le˧-gv̩˧\textasciitilde{}gv̩˥ | tʰi˧-tɕɯ˥}}} \textcolor{PineGreen}{\selectlanguage{french}ranger et bien mettre à sa place} \zh{收拾,摆好}  

\lhead{\firstmark}
\rhead{\botmark}

\subsection{\hspace{-0.5cm} {\Large \textcolor{darkblue}{\textbf{\ipa{gv̩˩\textsubscript{a}}}} \textsubscript{3}}\hspace{0.5cm}[\kern2pt{\textcolor{darkblue}{\textbf{\ipa{gv̩˩˥}}}}\kern2pt]} \hypertarget{gv\string_=\string_Ba3}{}
\markboth{\textcolor{darkblue}{\textbf{\ipa{gv̩˩\textsubscript{a}}}} \textsubscript{3}}{}
\textcolor{teal}{\mytextsc{verbe}} \hspace{4pt} Ton~: L\textsubscript{a}.
\textcolor{PineGreen}{\selectlanguage{french}Se coucher (le soleil se couche), décliner.} \zh{落下(太阳落山)。}  ¶ \textcolor{darkblue}{\textbf{\ipa{ɲi˧mi˧ gv̩˩-se˩}}} \textcolor{PineGreen}{\selectlanguage{french}à la nuit tombée, une fois la nuit tombée, après le coucher du soleil} \zh{在太阳落山之后,在太阳落山了以后}  
 ¶ \textcolor{darkblue}{\textbf{\ipa{ɲi˧mi˧ | le˧-gv̩˩-ze˩.}}} \textcolor{PineGreen}{\selectlanguage{french}Le soleil s'est couché.} \zh{太阳落山了。}  
 ¶ \textcolor{darkblue}{\textbf{\ipa{ɲi˧mi˧ | mɤ˧-gv̩˩-sɯ˩.}}} \textcolor{PineGreen}{\selectlanguage{french}Le soleil ne s'est pas encore couché.} \zh{太阳还没有落。}  

\lhead{\firstmark}
\rhead{\botmark}

\subsection{\hspace{-0.5cm} {\Large \textcolor{darkblue}{\textbf{\ipa{gv̩˧dv̩˧}}}}\hspace{0.5cm}[\kern2pt{\textcolor{darkblue}{\textbf{\ipa{gv̩˩dv̩˩˥}}}}\kern2pt]} \hypertarget{gv\string_=\string_Mdv\string_=\string_M1}{}
\markboth{\textcolor{darkblue}{\textbf{\ipa{gv̩˧dv̩˧}}}}{}
\textcolor{teal}{\mytextsc{nom}} \hspace{4pt} Ton~: M.
\textcolor{PineGreen}{\selectlanguage{french}Dos.} \zh{脊背。}  \zh{量词}: \textcolor{darkblue}{\textbf{\ipa{ʈv̩˩}}}  \mytextsc{clf}~: \textcolor{darkblue}{\textbf{\ipa{ʈv̩˩}}} 
\lhead{\firstmark}
\rhead{\botmark}

\subsection{\hspace{-0.5cm} {\Large \textcolor{darkblue}{\textbf{\ipa{gv̩˧dv̩˧-gv̩˧mi˧}}}}\hspace{0.5cm}[\kern2pt{\textcolor{darkblue}{\textbf{\ipa{xxxx non-correspondance entre le nombre de morphèmes et le nombre de tons de morphèmes}}}}\kern2pt]} \hypertarget{gv\string_=\string_Mdv\string_=\string_M-gv\string_=\string_Mmi\string_M1}{}
\markboth{\textcolor{darkblue}{\textbf{\ipa{gv̩˧dv̩˧-gv̩˧mi˧}}}}{}
\textcolor{teal}{\mytextsc{nom}} \hspace{4pt} Ton~: M.
\textit{De:} \textbf{gv̩˧dv̩˧ et gv̩˧mi˧} \textcolor{PineGreen}{\selectlanguage{french}Corps.} \zh{身体。}  \zh{量词}: \textcolor{darkblue}{\textbf{\ipa{ɭɯ˧}}}  \mytextsc{clf}~: \textcolor{darkblue}{\textbf{\ipa{ɭɯ˧}}} 
\lhead{\firstmark}
\rhead{\botmark}

\subsection{\hspace{-0.5cm} {\Large \textcolor{darkblue}{\textbf{\ipa{gv̩˩dʑɯ˩}}}}\hspace{0.5cm}[\kern2pt{\textcolor{darkblue}{\textbf{\ipa{gv̩˧dʑɯ˧}}}}\kern2pt]} \hypertarget{gv\string_=\string_Bdz£M\string_B1}{}
\markboth{\textcolor{darkblue}{\textbf{\ipa{gv̩˩dʑɯ˩}}}}{}
\textcolor{teal}{\mytextsc{adjectif}} \hspace{4pt} Ton~: L.
\textcolor{PineGreen}{\selectlanguage{french}En colère, affligé.} \zh{生气。} 
\lhead{\firstmark}
\rhead{\botmark}

\subsection{\hspace{-0.5cm} {\Large \textcolor{darkblue}{\textbf{\ipa{gv̩˧kv̩˩}}}}\hspace{0.5cm}[\kern2pt{\textcolor{darkblue}{\textbf{\ipa{gv̩˩kv̩˩˥}}}}\kern2pt]} \hypertarget{gv\string_=\string_Mkv\string_=\string_B1}{}
\markboth{\textcolor{darkblue}{\textbf{\ipa{gv̩˧kv̩˩}}}}{}
\textcolor{teal}{\mytextsc{nom}} \hspace{4pt} Ton~: L\#.
\textcolor{PineGreen}{\selectlanguage{french}Intonation, manière de s'exprimer; par extension, peut désigner les tons.} \zh{语调,声调。}  ¶ \textcolor{darkblue}{\textbf{\ipa{gv̩˧kv̩˩-gv̩˩li˩ | ʐwɤ˩˥}}} \textcolor{PineGreen}{\selectlanguage{french}parler avec une élocution soignée/agréable} \zh{说话说得好听、有口才、口若悬河、能言善辩}  

\lhead{\firstmark}
\rhead{\botmark}

\subsection{\hspace{-0.5cm} {\Large \textcolor{darkblue}{\textbf{\ipa{gv̩˩ɬi˩mi˩}}}}\hspace{0.5cm}[\kern2pt{\textcolor{darkblue}{\textbf{\ipa{gv̩˧ɬi˧mi˩}}}}\kern2pt]} \hypertarget{gv\string_=\string_BKi\string_Bmi\string_B1}{}
\markboth{\textcolor{darkblue}{\textbf{\ipa{gv̩˩ɬi˩mi˩}}}}{}
\textcolor{teal}{\mytextsc{nom}} \hspace{4pt} Ton~: L.
\textcolor{PineGreen}{\selectlanguage{french}9e mois.} \zh{九月。} 
\lhead{\firstmark}
\rhead{\botmark}

\subsection{\hspace{-0.5cm} {\Large \textcolor{darkblue}{\textbf{\ipa{gv̩˧mɑ˧}}}}\hspace{0.5cm}[\kern2pt{\textcolor{darkblue}{\textbf{\ipa{gv̩˩mɑ˩˥}}}}\kern2pt]} \hypertarget{gv\string_=\string_MmA\string_M1}{}
\markboth{\textcolor{darkblue}{\textbf{\ipa{gv̩˧mɑ˧}}}}{}
\textcolor{teal}{\mytextsc{nom}} \hspace{4pt} Ton~: M.
\textcolor{PineGreen}{\selectlanguage{french}Prénom masculin.} \zh{男性名字。}  ¶ \textcolor{darkblue}{\textbf{\ipa{hĩ˧ | ʈʂʰɯ˧-v̩˧, | gv̩˧mɑ˧ mv̩˧ʈʂæ˧˥!}}} \textcolor{PineGreen}{\selectlanguage{french}Cette personne s'appelle \textcolor{darkblue}{\textbf{\ipa{/gv̩˧mɑ˧/}}} !} \zh{这个人,名叫\textcolor{darkblue}{\textbf{\ipa{/gv̩˧mɑ˧/}}}!}  

\lhead{\firstmark}
\rhead{\botmark}

\subsection{\hspace{-0.5cm} {\Large \textcolor{darkblue}{\textbf{\ipa{gv̩˧mi˧}}}}\hspace{0.5cm}[\kern2pt{\textcolor{darkblue}{\textbf{\ipa{gv̩˧mi˧}}}}\kern2pt]} \hypertarget{gv\string_=\string_Mmi\string_M1}{}
\markboth{\textcolor{darkblue}{\textbf{\ipa{gv̩˧mi˧}}}}{}
\textcolor{teal}{\mytextsc{nom}} \hspace{4pt} Ton~: M.
\textcolor{PineGreen}{\selectlanguage{french}Corps.} \zh{身体。}  \zh{量词}: \textcolor{darkblue}{\textbf{\ipa{ɭɯ˧}}}  \mytextsc{clf}~: \textcolor{darkblue}{\textbf{\ipa{ɭɯ˧}}} 
\lhead{\firstmark}
\rhead{\botmark}

\subsection{\hspace{-0.5cm} {\Large \textcolor{darkblue}{\textbf{\ipa{gv̩˩pʰæ˩}}}}\hspace{0.5cm}[\kern2pt{\textcolor{darkblue}{\textbf{\ipa{gv̩˧pʰæ˧}}}}\kern2pt]} \hypertarget{gv\string_=\string_Bp\string_h\{\string_B1}{}
\markboth{\textcolor{darkblue}{\textbf{\ipa{gv̩˩pʰæ˩}}}}{}
\textcolor{teal}{\mytextsc{nom}} \hspace{4pt} Ton~: L.
\textcolor{PineGreen}{\selectlanguage{french}Planche de bois fine: trois ou quatre centimètres.} \zh{相当薄的木板。}  \zh{量词}: \textcolor{darkblue}{\textbf{\ipa{pʰæ˧˥}}}  \mytextsc{clf}~: \textcolor{darkblue}{\textbf{\ipa{pʰæ˧˥}}} 
\lhead{\firstmark}
\rhead{\botmark}

\subsection{\hspace{-0.5cm} {\Large \textcolor{darkblue}{\textbf{\ipa{gv̩˧sɯ˥-pv̩˩}}}}\hspace{0.5cm}[\kern2pt{\textcolor{darkblue}{\textbf{\ipa{xxxx non-correspondance entre le nombre de morphèmes et le nombre de tons de morphèmes}}}}\kern2pt]} \hypertarget{gv\string_=\string_MsM\string_T-pv\string_=\string_B1}{}
\markboth{\textcolor{darkblue}{\textbf{\ipa{gv̩˧sɯ˥-pv̩˩}}}}{}
\textcolor{teal}{\mytextsc{nom}} \hspace{4pt} Ton~: H\#-L.
\textcolor{PineGreen}{\selectlanguage{french}Omoplate.} \zh{肩胛骨。}  \zh{量词}: \textcolor{darkblue}{\textbf{\ipa{kʰwɤ˥}}}  \mytextsc{clf}~: \textcolor{darkblue}{\textbf{\ipa{kʰwɤ˥}}} 
\lhead{\firstmark}
\rhead{\botmark}

\subsection{\hspace{-0.5cm} {\Large \textcolor{darkblue}{\textbf{\ipa{gv̩˧tɕʰɯ˧˥}}}}\hspace{0.5cm}[\kern2pt{\textcolor{darkblue}{\textbf{\ipa{xxxx non-correspondance entre le nombre de morphèmes et le nombre de tons de morphèmes}}}}\kern2pt]} \hypertarget{gv\string_=\string_Mts£\string_hM\string_M\string_T1}{}
\markboth{\textcolor{darkblue}{\textbf{\ipa{gv̩˧tɕʰɯ˧˥}}}}{}
\textcolor{teal}{\mytextsc{verbe}} \hspace{4pt} Ton~: MH\#.
\textcolor{PineGreen}{\selectlanguage{french}Prendre froid, attraper un rhume, attraper froid.} \zh{着凉。} 
\lhead{\firstmark}
\rhead{\botmark}

\subsection{\hspace{-0.5cm} {\Large \textcolor{darkblue}{\textbf{\ipa{gv̩˧tsʰi˩}}}}\hspace{0.5cm}[\kern2pt{\textcolor{darkblue}{\textbf{\ipa{gv̩˧tsʰi˧˥}}}}\kern2pt]} \hypertarget{gv\string_=\string_Mts\string_hi\string_B1}{}
\markboth{\textcolor{darkblue}{\textbf{\ipa{gv̩˧tsʰi˩}}}}{}
\textcolor{teal}{\mytextsc{nombre}} \hspace{4pt} Ton~: L\#.
\textcolor{PineGreen}{\selectlanguage{french}90.} \zh{90。} 
\lhead{\firstmark}
\rhead{\botmark}

\subsection{\hspace{-0.5cm} {\Large \textcolor{darkblue}{\textbf{\ipa{gwɤ˩\textsubscript{a}}}}}\hspace{0.5cm}[\kern2pt{\textcolor{darkblue}{\textbf{\ipa{gwɤ˩˥}}}}\kern2pt]} \hypertarget{gw7\string_Ba1}{}
\markboth{\textcolor{darkblue}{\textbf{\ipa{gwɤ˩\textsubscript{a}}}}}{}
\textcolor{teal}{\mytextsc{verbe}} \hspace{4pt} Ton~: L\textsubscript{a}.
\textcolor{PineGreen}{\selectlanguage{french}Chanter.} \zh{唱、唱歌。}  ¶ \textcolor{darkblue}{\textbf{\ipa{njɤ˧ | ɖɯ˧-ɖʐo˩ | gwɤ˩-ze˥!}}} \textcolor{PineGreen}{\selectlanguage{french}j'ai chanté une chanson!} \zh{我唱了一首歌!}  
 ¶ \textcolor{darkblue}{\textbf{\ipa{no˧ | ɖɯ˧-ɖʐo˩ gwɤ˩!}}} \textcolor{PineGreen}{\selectlanguage{french}chante-nous une chanson!} \zh{你唱一首吧!}  
 ¶ \textcolor{darkblue}{\textbf{\ipa{ɖɯ˧-kʰwɤ˧ gwɤ˥}}} \textcolor{PineGreen}{\selectlanguage{french}chanter une chanson} \zh{唱一下}  
 ¶ \textcolor{darkblue}{\textbf{\ipa{ɖɯ˧-kʰwɤ˧ gwɤ˥-ɻ̍˩}}} \textcolor{PineGreen}{\selectlanguage{french}chanter une chanson} \zh{唱一下}  
 ¶ \textcolor{darkblue}{\textbf{\ipa{nɑ˩-gwɤ˥}}} \textcolor{PineGreen}{\selectlanguage{french}les chansons des Na} \zh{摩梭民歌}  
 ¶ \textcolor{darkblue}{\textbf{\ipa{ʈʂʰɯ˧ | nɑ˩-gwɤ˥ F | kv̩˧˥! | hæ˧-gwɤ˩ F | kv̩˧˥! | ʁo˧dzi˩-gwɤ˩ F | kv̩˧-ʝi˥! |}}} \textcolor{PineGreen}{\selectlanguage{french}Il sait chanter (toutes sortes de styles:) les chansons na! les chansons chinoises! les chansons tibétaines!} \zh{他会唱很多种风格的歌曲:摩梭的,会唱!汉族的,会唱!藏族的,会唱!}  

\lhead{\firstmark}
\rhead{\botmark}

\subsection{\hspace{-0.5cm} {\Large \textcolor{darkblue}{\textbf{\ipa{gwɤ˩\textasciitilde{}gwɤ˧˥}}}}\hspace{0.5cm}[\kern2pt{\textcolor{darkblue}{\textbf{\ipa{gwɤ˩gwɤ˥}}}}\kern2pt]} \hypertarget{gw7\string_B~gw7\string_M\string_T1}{}
\markboth{\textcolor{darkblue}{\textbf{\ipa{gwɤ˩\textasciitilde{}gwɤ˧˥}}}}{}
\textcolor{teal}{\mytextsc{verbe}} \hspace{4pt} Ton~: L.
\textcolor{PineGreen}{\selectlanguage{french}Se promener, se divertir, faire du tourisme.} \zh{逛,玩,游。}  ¶ \textcolor{darkblue}{\textbf{\ipa{le˧-gwɤ˩\textasciitilde{}gwɤ˩ | le˧-tsʰɯ˩-ze˩!}}} \textcolor{PineGreen}{\selectlanguage{french}(tu) reviens de promenade!/ Alors comme ça te voilà revenu de ta promenade!} \zh{你已经散步回来了!}  
 ¶ \textcolor{darkblue}{\textbf{\ipa{ʈʂʰɯ˧ | gwɤ˩\textasciitilde{}gwɤ˩-hɯ˩-ze˥!}}} \textcolor{PineGreen}{\selectlanguage{french}Il/elle est parti(e) se promener!} \zh{他散步去了!}  
 ¶ \textcolor{darkblue}{\textbf{\ipa{æ˧ʂæ˧ gwɤ˩; | qv̩˧ɻ̍˧ gwɤ˥; | nɑ˩tsʰi˩ gwɤ˥}}} \textcolor{PineGreen}{\selectlanguage{french}“faire le tour de la montagne \textcolor{darkblue}{\textbf{\ipa{/æ˧ʂæ˧/;}}} faire le tour de la montagne \textcolor{darkblue}{\textbf{\ipa{/qv̩˧ɻ\#˥/;}}} faire le tour de la montagne \textcolor{darkblue}{\textbf{\ipa{/nɑ˩tsʰi˩/”:}}} façon de désigner les rites qu'on pratiquait sur ces montagnes: pour des enfants qui tardaient à apprendre à parler, etc.} \zh{绕\textcolor{darkblue}{\textbf{\ipa{æ˧ʂæ˧}}}山,绕\textcolor{darkblue}{\textbf{\ipa{qv̩˧ɻ\#˥}}}山,绕\textcolor{darkblue}{\textbf{\ipa{nɑ˩tsʰi˩}}}山(做“绕山”仪式,为了求生子等)}  

\lhead{\firstmark}
\rhead{\botmark}

\subsection{\hspace{-0.5cm} {\Large \textcolor{darkblue}{\textbf{\ipa{gwɤ˩ʝi˧}}}}\hspace{0.5cm}[\kern2pt{\textcolor{darkblue}{\textbf{\ipa{gwɤ˩ʝi˩˥}}}}\kern2pt]} \hypertarget{gw7\string_Bj££i\string_M1}{}
\markboth{\textcolor{darkblue}{\textbf{\ipa{gwɤ˩ʝi˧}}}}{}
\textcolor{teal}{\mytextsc{adverbe}} \hspace{4pt} Ton~: LM.
\textcolor{PineGreen}{\selectlanguage{french}En ordre/rangé.} \zh{整齐。}  ¶ \textcolor{darkblue}{\textbf{\ipa{tso˧\textasciitilde{}tso˧ | gwɤ˩ʝi˧ tʰi˧-tɕɯ˥ |}}} \textcolor{PineGreen}{\selectlanguage{french}mettre des choses en ordre, ranger des choses} \zh{把东西摆整齐}  

\lhead{\firstmark}
\rhead{\botmark}

\newpage
\section*{\centering- \textcolor{darkblue}{\textbf{\ipa{ɣ}}} -}
\subsection{\hspace{-0.5cm} {\Large \textcolor{darkblue}{\textbf{\ipa{ɣɯ˥}}}}\hspace{0.5cm}[\kern2pt{\textcolor{darkblue}{\textbf{\ipa{ɣɯ˥}}}}\kern2pt]} \hypertarget{GM\string_T1}{}
\markboth{\textcolor{darkblue}{\textbf{\ipa{ɣɯ˥}}}}{}
\textcolor{teal}{\mytextsc{adjectif}} \hspace{4pt} Ton~: H.
\textcolor{PineGreen}{\selectlanguage{french}Habile, compétent, bon.} \zh{能干、好(做事情做得好)。}  ¶ \textcolor{darkblue}{\textbf{\ipa{mɤ˧-ɣɯ˥}}} \textcolor{PineGreen}{\selectlanguage{french}\mytextsc{neg}} \zh{不能干}  
 ¶ \textcolor{darkblue}{\textbf{\ipa{ʈʂʰɯ˧-ɳɯ˧, | bɑ˩lɑ˩ hwæ˧ | ɣɯ˧!}}} \textcolor{PineGreen}{\selectlanguage{french}Il/elle s'entend à acheter des vêtements! / Il/elle a du talent pour acheter des vêtements!} \zh{他很会买衣服!}  

\lhead{\firstmark}
\rhead{\botmark}

\subsection{\hspace{-0.5cm} {\Large \textcolor{darkblue}{\textbf{\ipa{ɣɯ˧}}}}\hspace{0.5cm}[\kern2pt{\textcolor{darkblue}{\textbf{\ipa{ɣɯ˥}}}}\kern2pt]} \hypertarget{GM\string_M1}{}
\markboth{\textcolor{darkblue}{\textbf{\ipa{ɣɯ˧}}}}{}
\textcolor{teal}{\mytextsc{nom}} \hspace{4pt} Ton~: \#H.
\textcolor{PineGreen}{\selectlanguage{french}Tissu.} \zh{布料。}  ¶ \textcolor{darkblue}{\textbf{\ipa{ɣɯ˧dzo˩, | ɣɯ˧ni˧˥, | ɣɯ˧, | ɖɯ˧-ʑi˩ ɲi˩-ze˩!}}} \textcolor{PineGreen}{\selectlanguage{french}Le métier à tisser, la structure en bambou qui maintient les fils, le tissu, c'est de la même famille! / ça forme une famille! (Commentaire sémantique de la locutrice, au cours d'une séance où il était question de tissus et de tissage)} \zh{织布机、竹子的框(让线不乱混)、布料,属于同一类!(直译:“都是一家的!”)}  
 \zh{量词}: \textcolor{darkblue}{\textbf{\ipa{bo˩}}}  \mytextsc{clf}~: \textcolor{darkblue}{\textbf{\ipa{bo˩}}} 
\lhead{\firstmark}
\rhead{\botmark}

\subsection{\hspace{-0.5cm} {\Large \textcolor{darkblue}{\textbf{\ipa{ɣɯ˧bo˩}}}}\hspace{0.5cm}[\kern2pt{\textcolor{darkblue}{\textbf{\ipa{ɣɯ˧bo˧˥}}}}\kern2pt]} \hypertarget{GM\string_Mbo\string_B1}{}
\markboth{\textcolor{darkblue}{\textbf{\ipa{ɣɯ˧bo˩}}}}{}
\textcolor{teal}{\mytextsc{nom}} \hspace{4pt} Ton~: MH\#.
\textcolor{PineGreen}{\selectlanguage{french}Trame (lorsqu'on tisse, il y a du fil de trame, et du fil de chaîne); la trame désigne l'ensemble des fils de trame.} \zh{纬线、纬纱。}  \zh{量词}: \textcolor{darkblue}{\textbf{\ipa{bo˩}}}  \mytextsc{clf}~: \textcolor{darkblue}{\textbf{\ipa{bo˩}}} 
\lhead{\firstmark}
\rhead{\botmark}

\subsection{\hspace{-0.5cm} {\Large \textcolor{darkblue}{\textbf{\ipa{ɣɯ˧dzo˩}}}}\hspace{0.5cm}[\kern2pt{\textcolor{darkblue}{\textbf{\ipa{ɣɯ˧dzo˩}}}}\kern2pt]} \hypertarget{GM\string_Mdzo\string_B1}{}
\markboth{\textcolor{darkblue}{\textbf{\ipa{ɣɯ˧dzo˩}}}}{}
\textcolor{teal}{\mytextsc{nom}} \hspace{4pt} Ton~: L\#.
\textcolor{PineGreen}{\selectlanguage{french}Métier à tisser, appareil à tisser.} \zh{织布机。}  \zh{量词}: \textcolor{darkblue}{\textbf{\ipa{nɑ˧}}}  \mytextsc{clf}~: \textcolor{darkblue}{\textbf{\ipa{nɑ˧}}} 
\lhead{\firstmark}
\rhead{\botmark}

\subsection{\hspace{-0.5cm} {\Large \textcolor{darkblue}{\textbf{\ipa{ɣɯ˧ni˧˥}}}}\hspace{0.5cm}[\kern2pt{\textcolor{darkblue}{\textbf{\ipa{ɣɯ˧ni˧˥}}}}\kern2pt]} \hypertarget{GM\string_Mni\string_M\string_T1}{}
\markboth{\textcolor{darkblue}{\textbf{\ipa{ɣɯ˧ni˧˥}}}}{}
\textcolor{teal}{\mytextsc{nom}} \hspace{4pt} Ton~: MH\#.
\textcolor{PineGreen}{\selectlanguage{french}Petite structure en bambou maintenant les fils du métier à tisser; ses fils (blancs) sont verticaux, et passent à travers la trame.} \zh{织布机的一部分:竹子的框,让线不乱混。}  \zh{量词}: \textcolor{darkblue}{\textbf{\ipa{dze˩}}}  \mytextsc{clf}~: \textcolor{darkblue}{\textbf{\ipa{dze˩}}} 
\lhead{\firstmark}
\rhead{\botmark}

\subsection{\hspace{-0.5cm} {\Large \textcolor{darkblue}{\textbf{\ipa{ɣɯ˩kɯ˧˥}}}}\hspace{0.5cm}[\kern2pt{\textcolor{darkblue}{\textbf{\ipa{ɣɯ˩kɯ˧˥}}}}\kern2pt]} \hypertarget{GM\string_BkM\string_M\string_T1}{}
\markboth{\textcolor{darkblue}{\textbf{\ipa{ɣɯ˩kɯ˧˥}}}}{}
\textcolor{teal}{\mytextsc{nom}} \hspace{4pt} Ton~: LM+MH\#.
\ding{202} \textcolor{PineGreen}{\selectlanguage{french}Pelure de fruit ou de légume.} \zh{皮、鸡蛋壳、麦麸。}  ¶ \textcolor{darkblue}{\textbf{\ipa{pʰi˩ko˧-ɣɯ˩kɯ˩}}} \textcolor{PineGreen}{\selectlanguage{french}pelure de pomme} \zh{苹果皮}  
 ¶ \textcolor{darkblue}{\textbf{\ipa{jɤ˩jo˧-ɣɯ˥kɯ˩}}} \textcolor{PineGreen}{\selectlanguage{french}pelure de pomme de terre} \zh{洋芋皮}  
 \zh{量词}: \textcolor{darkblue}{\textbf{\ipa{kʰwɤ˥}}} \ding{203} \textcolor{PineGreen}{\selectlanguage{french}Fourrure, peau d'animal.} \zh{皮。}  ¶ \textcolor{darkblue}{\textbf{\ipa{ʂe˧-ɣɯ˥kɯ˩}}} \textcolor{PineGreen}{\selectlanguage{french}peau de la viande (peau de poulet, couenne de porc...)} \zh{肉皮:鸡皮、猪肉的皮……}  
\ding{204} \textcolor{PineGreen}{\selectlanguage{french}Coquille (d'oeuf).} \zh{蛋壳。} \ding{205} \textcolor{PineGreen}{\selectlanguage{french}Son (de céréale).} \zh{麸。}  ¶ \textcolor{darkblue}{\textbf{\ipa{dze˧ɭɯ˧-ɣɯ˩kɯ˩}}} \textcolor{PineGreen}{\selectlanguage{french}son de blé} \zh{小麦麸}  
 \mytextsc{clf}~: \textcolor{darkblue}{\textbf{\ipa{kʰwɤ˥}}} 
\lhead{\firstmark}
\rhead{\botmark}

\subsection{\hspace{-0.5cm} {\Large \textcolor{darkblue}{\textbf{\ipa{ɣɯ˩-nɑ˥mi˩}}}}\hspace{0.5cm}[\kern2pt{\textcolor{darkblue}{\textbf{\ipa{ɣɯ˩˥nɑ˧mi˧}}}}\kern2pt]} \hypertarget{GM\string_B-nA\string_Tmi\string_B1}{}
\markboth{\textcolor{darkblue}{\textbf{\ipa{ɣɯ˩-nɑ˥mi˩}}}}{}
\textcolor{teal}{\mytextsc{nom}} \hspace{4pt} Ton~: LH-.
\textcolor{PineGreen}{\selectlanguage{french}Terme péjoratif pour désigner les Yi (groupe ethnique): “les peaux-noires”.} \zh{彝族(带偏见的说法)。}  ¶ \textcolor{darkblue}{\textbf{\ipa{ɣɯ˩-nɑ˥mi˩-zo˩}}} \textcolor{PineGreen}{\selectlanguage{french}homme yi} \zh{彝族男人}  
 ¶ \textcolor{darkblue}{\textbf{\ipa{ɣɯ˩-nɑ˥mi˩-mv̩˩}}} \textcolor{PineGreen}{\selectlanguage{french}femme yi} \zh{彝族女人}  
 \zh{量词}: \textcolor{darkblue}{\textbf{\ipa{v̩˧}}}  \mytextsc{clf}~: \textcolor{darkblue}{\textbf{\ipa{v̩˧}}} 
\lhead{\firstmark}
\rhead{\botmark}

\subsection{\hspace{-0.5cm} {\Large \textcolor{darkblue}{\textbf{\ipa{ɣɯ˩˥}}}}\hspace{0.5cm}[\kern2pt{\textcolor{darkblue}{\textbf{\ipa{ɣɯ˩˥}}}}\kern2pt]} \hypertarget{GM\string_B\string_T1}{}
\markboth{\textcolor{darkblue}{\textbf{\ipa{ɣɯ˩˥}}}}{}
\textcolor{teal}{\mytextsc{nom}} \hspace{4pt} Ton~: LH.
\textcolor{PineGreen}{\selectlanguage{french}Peau.} \zh{皮肤。}  ¶ \textcolor{darkblue}{\textbf{\ipa{ɣɯ˩ dzɯ˩˥}}} \textcolor{PineGreen}{\selectlanguage{french}manger de la peau} \zh{吃皮}  
 ¶ \textcolor{darkblue}{\textbf{\ipa{ɣɯ˩˥ | ɖɯ˧-ʂɯ˩ pʰv˩}}} \textcolor{PineGreen}{\selectlanguage{french}littéralement 'perdre sa peau'; sens: être épuisé et blessé par un travail ardu (par exemple: abattre des arbres sur la montagne et ramener les grumes jusque dans la plaine)} \zh{直译:‘脱皮一次’。意思:疲劳而受伤(因为做了很辛苦的工作,如:在深山老林砍树、扛树干回到坝子)}  
 \zh{量词}: \textcolor{darkblue}{\textbf{\ipa{tsʰi˥}}}  \mytextsc{clf}~: \textcolor{darkblue}{\textbf{\ipa{tsʰi˥}}} \textcolor{PineGreen}{\selectlanguage{french}cl des peaux d'animaux}
\lhead{\firstmark}
\rhead{\botmark}

\newpage
\section*{\centering- \textcolor{darkblue}{\textbf{\ipa{h}}} -}
\subsection{\hspace{-0.5cm} {\Large \textcolor{darkblue}{\textbf{\ipa{hɑ˥}}}}\hspace{0.5cm}[\kern2pt{\textcolor{darkblue}{\textbf{\ipa{hɑ˧˥}}}}\kern2pt]} \hypertarget{hA\string_T1}{}
\markboth{\textcolor{darkblue}{\textbf{\ipa{hɑ˥}}}}{}
\textcolor{teal}{\mytextsc{nom}} \hspace{4pt} Ton~: \#H.
\textcolor{PineGreen}{\selectlanguage{french}Nourriture.} \zh{饭,米饭。}  ¶ \textcolor{darkblue}{\textbf{\ipa{hɑ˧-ʈv̩˧\textasciitilde{}ʈv̩˥}}} \textcolor{PineGreen}{\selectlanguage{french}boule de céréale (riz ou autre)} \zh{饭坨坨、饭团}  
 ¶ \textcolor{darkblue}{\textbf{\ipa{hɑ˧ dzɯ˧}}} \textcolor{PineGreen}{\selectlanguage{french}manger} \zh{吃饭}  
 ¶ \textcolor{darkblue}{\textbf{\ipa{ʈʂʰɯ˧ | hɑ˧ dzɯ˧-dʑo˩!}}} \textcolor{PineGreen}{\selectlanguage{french}Il/elle est en train de manger!} \zh{他在吃饭!}  
 ¶ \textcolor{darkblue}{\textbf{\ipa{hɑ˧ʂɯ˩}}} \textcolor{PineGreen}{\selectlanguage{french}céréales fraîches (juste après la cueillette; on en prépare des galettes particulièrement savoureuses)} \zh{新鲜的粮食(可以用来烤很香的饼)}  

\lhead{\firstmark}
\rhead{\botmark}

\subsection{\hspace{-0.5cm} {\Large \textcolor{darkblue}{\textbf{\ipa{hɑ˧bɤ˥}}}}\hspace{0.5cm}[\kern2pt{\textcolor{darkblue}{\textbf{\ipa{xxxx non-correspondance entre le nombre de morphèmes et le nombre de tons de morphèmes}}}}\kern2pt]} \hypertarget{hA\string_Mb7\string_T1}{}
\markboth{\textcolor{darkblue}{\textbf{\ipa{hɑ˧bɤ˥}}}}{}
\textcolor{teal}{\mytextsc{nom}} \hspace{4pt} Ton~: H\#.
\textcolor{PineGreen}{\selectlanguage{french}Épi de maïs.} \zh{玉米棒子。}  ¶ \textcolor{darkblue}{\textbf{\ipa{qʰɑ˧dze˧-hɑ˧bɤ˥}}} \textcolor{PineGreen}{\selectlanguage{french}maïs en épi; épi de maïs} \zh{玉米棒子}  
 \zh{量词}: \textcolor{darkblue}{\textbf{\ipa{bɤ˩}}}  \mytextsc{clf}~: \textcolor{darkblue}{\textbf{\ipa{bɤ˩}}} 
\lhead{\firstmark}
\rhead{\botmark}

\subsection{\hspace{-0.5cm} {\Large \textcolor{darkblue}{\textbf{\ipa{hɑ˧-bv̩˥\textasciitilde{}bv̩˩-di˩}}}}\hspace{0.5cm}[\kern2pt{\textcolor{darkblue}{\textbf{\ipa{xxxx non-correspondance entre le nombre de morphèmes et le nombre de tons de morphèmes}}}}\kern2pt]} \hypertarget{hA\string_M-bv\string_=\string_T~bv\string_=\string_B-di\string_B1}{}
\markboth{\textcolor{darkblue}{\textbf{\ipa{hɑ˧-bv̩˥\textasciitilde{}bv̩˩-di˩}}}}{}
\textcolor{teal}{\mytextsc{nom}} \hspace{4pt} Ton~: \#H--.
\textcolor{PineGreen}{\selectlanguage{french}Étuve pour le riz.} \zh{甑。}  \zh{量词}: \textcolor{darkblue}{\textbf{\ipa{ɭɯ˧}}}  \mytextsc{clf}~: \textcolor{darkblue}{\textbf{\ipa{ɭɯ˧}}} 
\lhead{\firstmark}
\rhead{\botmark}

\subsection{\hspace{-0.5cm} {\Large \textcolor{darkblue}{\textbf{\ipa{hɑ˧-gv̩˥-di˩}}}}\hspace{0.5cm}[\kern2pt{\textcolor{darkblue}{\textbf{\ipa{xxxx non-correspondance entre le nombre de morphèmes et le nombre de tons de morphèmes}}}}\kern2pt]} \hypertarget{hA\string_M-gv\string_=\string_T-di\string_B1}{}
\markboth{\textcolor{darkblue}{\textbf{\ipa{hɑ˧-gv̩˥-di˩}}}}{}
\textcolor{teal}{\mytextsc{nom}} \hspace{4pt} Ton~: H\#-.
\textcolor{PineGreen}{\selectlanguage{french}Fourneau.} \zh{炉子、灶头。} 
\lhead{\firstmark}
\rhead{\botmark}

\subsection{\hspace{-0.5cm} {\Large \textcolor{darkblue}{\textbf{\ipa{hɑ˧ɭɯ\#˥}}}}\hspace{0.5cm}[\kern2pt{\textcolor{darkblue}{\textbf{\ipa{hɑ˧ɭɯ˧}}}}\kern2pt]} \hypertarget{hA\string_Ml\string_RM\#\string_T1}{}
\markboth{\textcolor{darkblue}{\textbf{\ipa{hɑ˧ɭɯ\#˥}}}}{}
\textcolor{teal}{\mytextsc{nom}} \hspace{4pt} Ton~: \#H.
\textcolor{PineGreen}{\selectlanguage{french}Céréales.} \zh{粮食。} 
\lhead{\firstmark}
\rhead{\botmark}

\subsection{\hspace{-0.5cm} {\Large \textcolor{darkblue}{\textbf{\ipa{hɑ˧mi˥}}}}\hspace{0.5cm}[\kern2pt{\textcolor{darkblue}{\textbf{\ipa{hɑ˧mi˥}}}}\kern2pt]} \hypertarget{hA\string_Mmi\string_T1}{}
\markboth{\textcolor{darkblue}{\textbf{\ipa{hɑ˧mi˥}}}}{}
\textcolor{teal}{\mytextsc{verbe}} \hspace{4pt} Ton~: H\#.
\textcolor{PineGreen}{\selectlanguage{french}Mendier.} \zh{讨饭。}  ¶ \textcolor{darkblue}{\textbf{\ipa{hɑ˧mi˥-hĩ˩}}} \textcolor{PineGreen}{\selectlanguage{french}\string_ \mytextsc{rel}: mendiant, [personne] qui mendie} \zh{要饭的、乞丐}  
\textit{Voir~:} \hyperlink{}{\textcolor{darkblue}{\textbf{\ipa{mi˩\textsubscript{a}}}}} 
\lhead{\firstmark}
\rhead{\botmark}

\subsection{\hspace{-0.5cm} {\Large \textcolor{darkblue}{\textbf{\ipa{hɑ˧pv̩˩}}}}\hspace{0.5cm}[\kern2pt{\textcolor{darkblue}{\textbf{\ipa{hɑ˧pv̩˩}}}}\kern2pt]} \hypertarget{hA\string_Mpv\string_=\string_B1}{}
\markboth{\textcolor{darkblue}{\textbf{\ipa{hɑ˧pv̩˩}}}}{}
\textcolor{teal}{\mytextsc{nom}} \hspace{4pt} Ton~: L\#.
\textcolor{PineGreen}{\selectlanguage{french}Riz cuit 'sec': le riz tel qu'il est servi aux repas, par opposition avec le gruau de riz.} \zh{干的米饭(与稀饭不同)。} 
\lhead{\firstmark}
\rhead{\botmark}

\subsection{\hspace{-0.5cm} {\Large \textcolor{darkblue}{\textbf{\ipa{hɑ˧ʂɯ˥}}}}\hspace{0.5cm}[\kern2pt{\textcolor{darkblue}{\textbf{\ipa{hɑ˧ʂɯ˥}}}}\kern2pt]} \hypertarget{hA\string_Ms`M\string_T1}{}
\markboth{\textcolor{darkblue}{\textbf{\ipa{hɑ˧ʂɯ˥}}}}{}
\textcolor{teal}{\mytextsc{connecteur}} \hspace{4pt} Ton~: H\#.
\textcolor{PineGreen}{\selectlanguage{french}Explétif, emprunté au chinois: “quand même/aussi...”.} \zh{还是(汉语借词)。}  Emprunt~: chinois  \zh{还是}

\lhead{\firstmark}
\rhead{\botmark}

\subsection{\hspace{-0.5cm} {\Large \textcolor{darkblue}{\textbf{\ipa{hɑ˧-ʐwɤ˩}}}}\hspace{0.5cm}[\kern2pt{\textcolor{darkblue}{\textbf{\ipa{xxxx non-correspondance entre le nombre de morphèmes et le nombre de tons de morphèmes}}}}\kern2pt]} \hypertarget{hA\string_M-z`w7\string_B1}{}
\markboth{\textcolor{darkblue}{\textbf{\ipa{hɑ˧-ʐwɤ˩}}}}{}
\textcolor{teal}{\mytextsc{adjectif}} \hspace{4pt} Ton~: L\#.
\textcolor{PineGreen}{\selectlanguage{french}Avoir faim.} \zh{饿(饭)。} \textit{Voir~:} \hyperlink{}{\textcolor{darkblue}{\textbf{\ipa{ʐwɤ˧}}}} 
\lhead{\firstmark}
\rhead{\botmark}

\subsection{\hspace{-0.5cm} {\Large \textcolor{darkblue}{\textbf{\ipa{hɑ˩\textsubscript{a}}}}}\hspace{0.5cm}[\kern2pt{\textcolor{darkblue}{\textbf{\ipa{hɑ˧˥}}}}\kern2pt]} \hypertarget{hA\string_Ba1}{}
\markboth{\textcolor{darkblue}{\textbf{\ipa{hɑ˩\textsubscript{a}}}}}{}
\textcolor{teal}{\mytextsc{verbe}} \hspace{4pt} Ton~: L\textsubscript{a}.
\textcolor{PineGreen}{\selectlanguage{french}Ouvrir (les yeux); s'ouvrir (un sac).} \zh{睁开(眼睛)。}  ¶ \textcolor{darkblue}{\textbf{\ipa{tʰi˧-hɑ˩}}} \textcolor{PineGreen}{\selectlanguage{french}\mytextsc{dur}} \zh{\mytextsc{dur}}  
 ¶ \textcolor{darkblue}{\textbf{\ipa{njɤ˩ɭɯ˥ | gɤ˩-hɑ˥ |}}} \textcolor{PineGreen}{\selectlanguage{french}ouvrir les yeux} \zh{睁开眼睛}  
 ¶ \textcolor{darkblue}{\textbf{\ipa{njɤ˩ɭɯ˧ hɑ˩}}} \textcolor{PineGreen}{\selectlanguage{french}ouvrir les yeux} \zh{睁开眼睛}  

\lhead{\firstmark}
\rhead{\botmark}

\subsection{\hspace{-0.5cm} {\Large \textcolor{darkblue}{\textbf{\ipa{hɑ̃˧mo˥}}}}\hspace{0.5cm}[\kern2pt{\textcolor{darkblue}{\textbf{\ipa{hɑ̃˧mo˥}}}}\kern2pt]} \hypertarget{hA\string_~\string_Mmo\string_T1}{}
\markboth{\textcolor{darkblue}{\textbf{\ipa{hɑ̃˧mo˥}}}}{}
\textcolor{teal}{\mytextsc{adjectif}} \hspace{4pt} Ton~: H\#.
\textcolor{PineGreen}{\selectlanguage{french}Âgé, vieux (personne humaine).} \zh{年老。}  ¶ \textcolor{darkblue}{\textbf{\ipa{hĩ˧ ʈʂʰɯ˧-v̩˧ | hɑ̃˧mo˥ | ʐwæ˩˥!}}} \textcolor{PineGreen}{\selectlanguage{french}Cette personne est très âgée!} \zh{这个人,年纪非常大!}  

\lhead{\firstmark}
\rhead{\botmark}

\subsection{\hspace{-0.5cm} {\Large \textcolor{darkblue}{\textbf{\ipa{hɑ̃˧˥}}} \textsubscript{1}}\hspace{0.5cm}[\kern2pt{\textcolor{darkblue}{\textbf{\ipa{hɑ̃˧˥}}}}\kern2pt]} \hypertarget{hA\string_~\string_M\string_T1}{}
\markboth{\textcolor{darkblue}{\textbf{\ipa{hɑ̃˧˥}}} \textsubscript{1}}{}
\textcolor{teal}{\mytextsc{verbe}} \hspace{4pt} Ton~: MH.
\textcolor{PineGreen}{\selectlanguage{french}Éteindre (par exemple le feu du foyer).} \zh{把火炉灭了。}  ¶ \textcolor{darkblue}{\textbf{\ipa{mv̩˧ le˧-hɑ̃˧˥}}} \textcolor{PineGreen}{\selectlanguage{french}éteindre le feu} \zh{灭火}  

\lhead{\firstmark}
\rhead{\botmark}

\subsection{\hspace{-0.5cm} {\Large \textcolor{darkblue}{\textbf{\ipa{hɑ̃˧˥}}} \textsubscript{2}}\hspace{0.5cm}[\kern2pt{\textcolor{darkblue}{\textbf{\ipa{hɑ̃˧˥}}}}\kern2pt]} \hypertarget{hA\string_~\string_M\string_T2}{}
\markboth{\textcolor{darkblue}{\textbf{\ipa{hɑ̃˧˥}}} \textsubscript{2}}{}
\textcolor{teal}{\mytextsc{verbe}} \hspace{4pt} Ton~: MH.
\textcolor{PineGreen}{\selectlanguage{french}Passer la nuit.} \zh{过夜。}  ¶ \textcolor{darkblue}{\textbf{\ipa{ɖɯ˧-hɑ̃˧ tʰi˥-hɑ̃˩ |}}} \textcolor{PineGreen}{\selectlanguage{french}se reposer une soirée, passer une soirée/nuitée (qq part)} \zh{过夜}  
 ¶ \textcolor{darkblue}{\textbf{\ipa{ʑi˧qʰwɤ˧ ɖɯ˧-ɭɯ˧-qo˧ hɑ̃˧˥}}} \textcolor{PineGreen}{\selectlanguage{french}passer la nuit dans une maison} \zh{在一个人家过夜}  

\lhead{\firstmark}
\rhead{\botmark}

\subsection{\hspace{-0.5cm} {\Large \textcolor{darkblue}{\textbf{\ipa{hɑ̃˧˥\textsubscript{a}}}}}\hspace{0.5cm}[\kern2pt{\textcolor{darkblue}{\textbf{\ipa{hɑ̃˩˥}}}}\kern2pt]} \hypertarget{hA\string_~\string_M\string_Ta1}{}
\markboth{\textcolor{darkblue}{\textbf{\ipa{hɑ̃˧˥\textsubscript{a}}}}}{}
\textcolor{teal}{\mytextsc{classificateur}} \hspace{4pt} Ton~: MH\textsubscript{a}.
\textcolor{PineGreen}{\selectlanguage{french}Nuit; par extension: utilisé pour le décompte des jours.} \zh{量词:夜。}  ¶ \textcolor{darkblue}{\textbf{\ipa{ɖɯ˧-hɑ̃˧˥}}} \textcolor{PineGreen}{\selectlanguage{french}une nuit} \zh{一夜}  
 ¶ \textcolor{darkblue}{\textbf{\ipa{tsʰe˩-hɑ̃˩˥}}} \textcolor{PineGreen}{\selectlanguage{french}dix soirées=10 jours} \zh{十夜(等于十天)}  
 ¶ \textcolor{darkblue}{\textbf{\ipa{ɖɯ˧-hɑ̃˧ lɑ˥-dʑo˩!}}} \textcolor{PineGreen}{\selectlanguage{french}Il ne reste qu'une soirée!} \zh{只有一个晚上了!}  
 ¶ \textcolor{darkblue}{\textbf{\ipa{[F5] ɖɯ˧-hɑ̃˧-ɳɯ˥ | le˧-li˧-le˧-se˩-ze˩!}}} \textcolor{PineGreen}{\selectlanguage{french}(Il) a tout lu en deux jours! (contexte imaginé: on offre un livre à quelqu'un; en deux jours il a tout lu)} \zh{一个晚上就读完了!/一天之内都读完了!(情景:送一个人一本书,他马上全部读完)}  

\lhead{\firstmark}
\rhead{\botmark}

\subsection{\hspace{-0.5cm} {\Large \textcolor{darkblue}{\textbf{\ipa{hæ˧}}}}\hspace{0.5cm}[\kern2pt{\textcolor{darkblue}{\textbf{\ipa{hæ˧˥}}}}\kern2pt]} \hypertarget{h\{\string_M1}{}
\markboth{\textcolor{darkblue}{\textbf{\ipa{hæ˧}}}}{}
\textcolor{teal}{\mytextsc{nom}} \hspace{4pt} Ton~: M.
\textcolor{PineGreen}{\selectlanguage{french}Chinois (Han).} \zh{汉人。}  ¶ \textcolor{darkblue}{\textbf{\ipa{hæ˧-mi\#˥}}} \textcolor{PineGreen}{\selectlanguage{french}une femme chinoise, une Chinoise (Han)} \zh{汉族女人}  
 ¶ \textcolor{darkblue}{\textbf{\ipa{hæ˧-mv̩˧ hæ˧-di˧˥}}} \textcolor{PineGreen}{\selectlanguage{french}le territoire des Chinois (Han): Chengdu, Kunming...} \zh{汉族地区,包括成都、昆明等等}  
 ¶ \textcolor{darkblue}{\textbf{\ipa{hæ˧-di˩}}} \textcolor{PineGreen}{\selectlanguage{french}le territoire des Chinois (Han): Chengdu, Kunming...; l'expression est employée pour désigner la direction du sud} \zh{汉族地区,包括成都、昆明等等,来代指南方}  
 ¶ \textcolor{darkblue}{\textbf{\ipa{hæ˧-zo˧bæ˩}}} \textcolor{PineGreen}{\selectlanguage{french}homme chinois han (terme péjoratif: littéralement 'idiot de Chinois')} \zh{汉男人(带偏见的称呼)}  
 \zh{量词}: \textcolor{darkblue}{\textbf{\ipa{v̩˧}}}  \mytextsc{clf}~: \textcolor{darkblue}{\textbf{\ipa{v̩˧}}} 
\lhead{\firstmark}
\rhead{\botmark}

\subsection{\hspace{-0.5cm} {\Large \textcolor{darkblue}{\textbf{\ipa{hæ˧di˩-ʈæ˩bɤ˩}}}}\hspace{0.5cm}[\kern2pt{\textcolor{darkblue}{\textbf{\ipa{xxxx non-correspondance entre le nombre de morphèmes et le nombre de tons de morphèmes}}}}\kern2pt]} \hypertarget{h\{\string_Mdi\string_B-t`\{\string_Bb7\string_B1}{}
\markboth{\textcolor{darkblue}{\textbf{\ipa{hæ˧di˩-ʈæ˩bɤ˩}}}}{}
\textcolor{teal}{\mytextsc{nom}} \hspace{4pt} Ton~: \mytextsc{L}.
\textcolor{PineGreen}{\selectlanguage{french}Bhiksu, moine mendiant.} \zh{比丘、游僧。} 
\lhead{\firstmark}
\rhead{\botmark}

\subsection{\hspace{-0.5cm} {\Large \textcolor{darkblue}{\textbf{\ipa{hæ˧ɭɯ\#˥}}}}\hspace{0.5cm}[\kern2pt{\textcolor{darkblue}{\textbf{\ipa{hæ˧ɭɯ˧}}}}\kern2pt]} \hypertarget{h\{\string_Ml\string_RM\#\string_T1}{}
\markboth{\textcolor{darkblue}{\textbf{\ipa{hæ˧ɭɯ\#˥}}}}{}
\textcolor{teal}{\mytextsc{nom}} \hspace{4pt} Ton~: \#H.
\textcolor{PineGreen}{\selectlanguage{french}Sorgho, gaoliang; céréale dont on se sert pour faire du vin.} \zh{高粱。} \textit{Voir~:} \hyperlink{}{\textcolor{darkblue}{\textbf{\ipa{kɤ˧ljɤ˩}}}} 
\lhead{\firstmark}
\rhead{\botmark}

\subsection{\hspace{-0.5cm} {\Large \textcolor{darkblue}{\textbf{\ipa{hæ˧se˧}}}}\hspace{0.5cm}[\kern2pt{\textcolor{darkblue}{\textbf{\ipa{hæ˧se˧}}}}\kern2pt]} \hypertarget{h\{\string_Mse\string_M1}{}
\markboth{\textcolor{darkblue}{\textbf{\ipa{hæ˧se˧}}}}{}
\textcolor{teal}{\mytextsc{nom}} \hspace{4pt} Ton~: M.
\textcolor{PineGreen}{\selectlanguage{french}Agar (ressemble à une algue).} \zh{石花菜、海参。}  Emprunt~: chinois  dialect \zh{海参}

\lhead{\firstmark}
\rhead{\botmark}

\subsection{\hspace{-0.5cm} {\Large \textcolor{darkblue}{\textbf{\ipa{hæ˧ʐwɤ˩}}}}\hspace{0.5cm}[\kern2pt{\textcolor{darkblue}{\textbf{\ipa{hæ˧ʐwɤ˩}}}}\kern2pt]} \hypertarget{h\{\string_Mz`w7\string_B1}{}
\markboth{\textcolor{darkblue}{\textbf{\ipa{hæ˧ʐwɤ˩}}}}{}
\textcolor{teal}{\mytextsc{nom}} \hspace{4pt} Ton~: L\#.
\textcolor{PineGreen}{\selectlanguage{french}La langue chinoise.} \zh{汉语。} 
\lhead{\firstmark}
\rhead{\botmark}

\subsection{\hspace{-0.5cm} {\Large \textcolor{darkblue}{\textbf{\ipa{hæ˩\textsubscript{a}}}}}\hspace{0.5cm}[\kern2pt{\textcolor{darkblue}{\textbf{\ipa{hæ˥}}}}\kern2pt]} \hypertarget{h\{\string_Ba1}{}
\markboth{\textcolor{darkblue}{\textbf{\ipa{hæ˩\textsubscript{a}}}}}{}
\textcolor{teal}{\mytextsc{verbe}} \hspace{4pt} Ton~: L\textsubscript{a}.
\textcolor{PineGreen}{\selectlanguage{french}Causer du tort à (note: ce mot n'est pas un emprunt, la ressemble avec le mandarin est une coïncidence).} \zh{祸害、害。}  ¶ \textcolor{darkblue}{\textbf{\ipa{hĩ˧ hæ˥}}} \textcolor{PineGreen}{\selectlanguage{french}causer du tort aux gens} \zh{害人}  
 ¶ \textcolor{darkblue}{\textbf{\ipa{hĩ˧ hæ˥-kv̩˩}}} \textcolor{PineGreen}{\selectlanguage{french}qui est susceptible de causer du tort/d'être cruel} \zh{会害人的、残忍、凶狠}  
 ¶ \textcolor{darkblue}{\textbf{\ipa{hĩ˧ hæ˥-zo˩}}} \textcolor{PineGreen}{\selectlanguage{french}homme terrifiant} \zh{可怕的人}  

\lhead{\firstmark}
\rhead{\botmark}

\subsection{\hspace{-0.5cm} {\Large \textcolor{darkblue}{\textbf{\ipa{hæ˧˥}}} \textsubscript{1}}\hspace{0.5cm}[\kern2pt{\textcolor{darkblue}{\textbf{\ipa{hæ˥}}}}\kern2pt]} \hypertarget{h\{\string_M\string_T1}{}
\markboth{\textcolor{darkblue}{\textbf{\ipa{hæ˧˥}}} \textsubscript{1}}{}
\textcolor{teal}{\mytextsc{adjectif}} \hspace{4pt} Ton~: MH.
\ding{202} \textcolor{PineGreen}{\selectlanguage{french}Souple, mou (branche…).} \zh{软、柔软(树枝……)。}  ¶ \textcolor{darkblue}{\textbf{\ipa{hæ˧njæ˧˥ | -gv̩˩}}} \textcolor{PineGreen}{\selectlanguage{french}souple} \zh{软、柔软(树枝……)}  
\ding{203} \textcolor{PineGreen}{\selectlanguage{french}Léger, clair, délayé (gruau, soupe...).} \zh{稀(粥、汤)。} 
\lhead{\firstmark}
\rhead{\botmark}

\subsection{\hspace{-0.5cm} {\Large \textcolor{darkblue}{\textbf{\ipa{hæ˧˥}}} \textsubscript{2}}\hspace{0.5cm}[\kern2pt{\textcolor{darkblue}{\textbf{\ipa{hæ˧˥}}}}\kern2pt]} \hypertarget{h\{\string_M\string_T2}{}
\markboth{\textcolor{darkblue}{\textbf{\ipa{hæ˧˥}}} \textsubscript{2}}{}
\textcolor{teal}{\mytextsc{nom}} \hspace{4pt} Ton~: MH.
\textcolor{PineGreen}{\selectlanguage{french}Chaux.} \zh{石灰。}  ¶ \textcolor{darkblue}{\textbf{\ipa{hæ˧ hwæ˥}}} \textcolor{PineGreen}{\selectlanguage{french}acheter de la chaux} \zh{买石灰}  
 ¶ \textcolor{darkblue}{\textbf{\ipa{hæ˧ tɕʰi˥}}} \textcolor{PineGreen}{\selectlanguage{french}vendre de la chaux} \zh{卖石灰}  
 ¶ \textcolor{darkblue}{\textbf{\ipa{hæ˧ ki˥}}} \textcolor{PineGreen}{\selectlanguage{french}donner de la chaux} \zh{给石灰}  
 ¶ \textcolor{darkblue}{\textbf{\ipa{hæ˧ dv̩˥}}} \textcolor{PineGreen}{\selectlanguage{french}piocher de la chaux} \zh{挖石灰}  
 ¶ \textcolor{darkblue}{\textbf{\ipa{hæ˧ bæ˥}}} \textcolor{PineGreen}{\selectlanguage{french}balayer de la chaux} \zh{扫石灰}  
 ¶ \textcolor{darkblue}{\textbf{\ipa{hæ˧ gɤ˥}}} \textcolor{PineGreen}{\selectlanguage{french}porter de la chaux} \zh{扛石灰}  

\lhead{\firstmark}
\rhead{\botmark}

\subsection{\hspace{-0.5cm} {\Large \textcolor{darkblue}{\textbf{\ipa{hæ̃˧}}}}\hspace{0.5cm}[\kern2pt{\textcolor{darkblue}{\textbf{\ipa{hæ̃˥}}}}\kern2pt]} \hypertarget{h\{\string_~\string_M1}{}
\markboth{\textcolor{darkblue}{\textbf{\ipa{hæ̃˧}}}}{}
\textcolor{teal}{\mytextsc{nom}} \hspace{4pt} Ton~: M.
\textcolor{PineGreen}{\selectlanguage{french}Vent.} \zh{风。}  ¶ \textcolor{darkblue}{\textbf{\ipa{hæ̃˧ tʰv̩˧ / hæ̃˧ tʰv̩˧-ze˧}}} \textcolor{PineGreen}{\selectlanguage{french}il y a du vent, le vent souffle} \zh{刮风了}  
 ¶ \textcolor{darkblue}{\textbf{\ipa{wɤ˩˥ | hæ̃˧ tʰv̩˧-ho˩-ze˩!}}} \textcolor{PineGreen}{\selectlanguage{french}Le vent va se lever à nouveau!/ On dirait que le vent va se remettre à souffler!} \zh{风又要刮起来了!}  
 \zh{量词}: \textcolor{darkblue}{\textbf{\ipa{kʰwɤ˥}}}  \mytextsc{clf}~: \textcolor{darkblue}{\textbf{\ipa{kʰwɤ˥}}} 
\lhead{\firstmark}
\rhead{\botmark}

\subsection{\hspace{-0.5cm} {\Large \textcolor{darkblue}{\textbf{\ipa{hæ̃˧\textsubscript{a}}}}}\hspace{0.5cm}[\kern2pt{\textcolor{darkblue}{\textbf{\ipa{hæ̃˥}}}}\kern2pt]} \hypertarget{h\{\string_~\string_Ma1}{}
\markboth{\textcolor{darkblue}{\textbf{\ipa{hæ̃˧\textsubscript{a}}}}}{}
\textcolor{teal}{\mytextsc{verbe}} \hspace{4pt} Ton~: M\textsubscript{a}.
\textcolor{PineGreen}{\selectlanguage{french}Vanner: verser doucement dans une vannerie; la balle s'envole à mesure, emportée par le vent.} \zh{扬(粮食)。}  ¶ \textcolor{darkblue}{\textbf{\ipa{hɑ˧ hæ̃˩}}} \textcolor{PineGreen}{\selectlanguage{french}vanner du grain} \zh{扬粮食}  
 ¶ \textcolor{darkblue}{\textbf{\ipa{tso˧\textasciitilde{}tso˧ hæ̃˩}}} \textcolor{PineGreen}{\selectlanguage{french}vanner des choses} \zh{扬东西}  

\lhead{\firstmark}
\rhead{\botmark}

\subsection{\hspace{-0.5cm} {\Large \textcolor{darkblue}{\textbf{\ipa{hæ̃˧do˧}}}}\hspace{0.5cm}[\kern2pt{\textcolor{darkblue}{\textbf{\ipa{hæ̃˧do˧}}}}\kern2pt]} \hypertarget{h\{\string_~\string_Mdo\string_M1}{}
\markboth{\textcolor{darkblue}{\textbf{\ipa{hæ̃˧do˧}}}}{}
\textcolor{teal}{\mytextsc{nom}} \hspace{4pt} Ton~: M.
\textcolor{PineGreen}{\selectlanguage{french}Aire à battre le grain.} \zh{打场。}  ¶ \textcolor{darkblue}{\textbf{\ipa{hæ̃˧do˧ bæ˩}}} \textcolor{PineGreen}{\selectlanguage{french}balayer l'aire à battre le grain} \zh{清扫打场}  
 \zh{量词}: \textcolor{darkblue}{\textbf{\ipa{ɭɯ˧}}}  \mytextsc{clf}~: \textcolor{darkblue}{\textbf{\ipa{ɭɯ˧}}} 
\lhead{\firstmark}
\rhead{\botmark}

\subsection{\hspace{-0.5cm} {\Large \textcolor{darkblue}{\textbf{\ipa{hæ̃˧kʰɤ˧˥}}}}\hspace{0.5cm}[\kern2pt{\textcolor{darkblue}{\textbf{\ipa{hæ̃˧kʰɤ˧˥}}}}\kern2pt]} \hypertarget{h\{\string_~\string_Mk\string_h7\string_M\string_T1}{}
\markboth{\textcolor{darkblue}{\textbf{\ipa{hæ̃˧kʰɤ˧˥}}}}{}
\textcolor{teal}{\mytextsc{nom}} \hspace{4pt} Ton~: MH\#.
\textcolor{PineGreen}{\selectlanguage{french}Pièce de charpente: poutrelles de toiture: poutrelles courtes, installées en inclinaison, dans le sens de la largeur du bâtiment, sur les poutres horizontales, ʐv̩˩ɭɯ˥. Les tuiles (autrefois: les planches) reposent sur ces poutrelles.} \zh{椽子。}  \zh{量词}: \textcolor{darkblue}{\textbf{\ipa{ɭɯ˧}}}  \mytextsc{clf}~: \textcolor{darkblue}{\textbf{\ipa{ɭɯ˧}}} 
\lhead{\firstmark}
\rhead{\botmark}

\subsection{\hspace{-0.5cm} {\Large \textcolor{darkblue}{\textbf{\ipa{hæ̃˧kʰo˧}}}}\hspace{0.5cm}[\kern2pt{\textcolor{darkblue}{\textbf{\ipa{hæ̃˧kʰo˧}}}}\kern2pt]} \hypertarget{h\{\string_~\string_Mk\string_ho\string_M1}{}
\markboth{\textcolor{darkblue}{\textbf{\ipa{hæ̃˧kʰo˧}}}}{}
\textcolor{teal}{\mytextsc{nom}} \hspace{4pt} Ton~: M.
\textcolor{PineGreen}{\selectlanguage{french}Demoiselle de la noblesse.} \zh{小姐、公主。}  ¶ \textcolor{darkblue}{\textbf{\ipa{hæ̃˧kʰo˧-mi˧}}} \textcolor{PineGreen}{\selectlanguage{french}même sens: demoiselle, princesse} \zh{同上:小姐、公主}  
 \zh{量词}: \textcolor{darkblue}{\textbf{\ipa{v̩˧}}}  \mytextsc{clf}~: \textcolor{darkblue}{\textbf{\ipa{v̩˧}}} 
\lhead{\firstmark}
\rhead{\botmark}

\subsection{\hspace{-0.5cm} {\Large \textcolor{darkblue}{\textbf{\ipa{hæ̃˧pɤ˧}}}}\hspace{0.5cm}[\kern2pt{\textcolor{darkblue}{\textbf{\ipa{hæ̃˧pɤ˧}}}}\kern2pt]} \hypertarget{h\{\string_~\string_Mp7\string_M1}{}
\markboth{\textcolor{darkblue}{\textbf{\ipa{hæ̃˧pɤ˧}}}}{}
\textcolor{teal}{\mytextsc{nom}} \hspace{4pt} Ton~: M.
\textcolor{PineGreen}{\selectlanguage{french}Tresse.} \zh{辫子。}  \zh{量词}: \textcolor{darkblue}{\textbf{\ipa{kʰɯ˩}}}  \mytextsc{clf}~: \textcolor{darkblue}{\textbf{\ipa{kʰɯ˩}}} 
\lhead{\firstmark}
\rhead{\botmark}

\subsection{\hspace{-0.5cm} {\Large \textcolor{darkblue}{\textbf{\ipa{hæ̃˧qʰv̩˥\$}}}}\hspace{0.5cm}[\kern2pt{\textcolor{darkblue}{\textbf{\ipa{hæ̃˧qʰv̩˥}}}}\kern2pt]} \hypertarget{h\{\string_~\string_Mq\string_hv\string_=\string_T\$1}{}
\markboth{\textcolor{darkblue}{\textbf{\ipa{hæ̃˧qʰv̩˥\$}}}}{}
\textcolor{teal}{\mytextsc{adverbe}} \hspace{4pt} Ton~: H\$.
\textcolor{PineGreen}{\selectlanguage{french}En pleine nuit, tard dans la nuit.} \zh{半夜。} 
\lhead{\firstmark}
\rhead{\botmark}

\subsection{\hspace{-0.5cm} {\Large \textcolor{darkblue}{\textbf{\ipa{hæ̃˧ʂɯ˩-}}}}\hspace{0.5cm}[\kern2pt{\textcolor{darkblue}{\textbf{\ipa{hæ̃˧ʂɯ˩}}}}\kern2pt]} \hypertarget{h\{\string_~\string_Ms`M\string_B-1}{}
\markboth{\textcolor{darkblue}{\textbf{\ipa{hæ̃˧ʂɯ˩-}}}}{}
\textcolor{teal}{\mytextsc{nom}} \hspace{4pt} Ton~: L\#.
\textcolor{PineGreen}{\selectlanguage{french}'précieux': préfixe ajouté à certains noms pour construire une appellation prestigieuse. ; n'est pas productif: on ne peut l'ajouter à: /ə˧mɑ˧/ 'mère', /ɑ˩ʁo˧/ 'maison', etc.}  ¶ \textcolor{darkblue}{\textbf{\ipa{hæ̃˧ʂɯ˩-to˩mi˩}}} \textcolor{PineGreen}{\selectlanguage{french}les Piliers d'Or, les Précieux Piliers: appellation solennelle pour les deux piliers de la maison} \zh{‘黄金柱’、‘宝贵柱’:对主屋两个柱子的庄严称呼}  
 \zh{量词}: \textcolor{darkblue}{\textbf{\ipa{nɑ˧}}}  \mytextsc{clf}~: \textcolor{darkblue}{\textbf{\ipa{nɑ˧}}} 
\lhead{\firstmark}
\rhead{\botmark}

\subsection{\hspace{-0.5cm} {\Large \textcolor{darkblue}{\textbf{\ipa{hæ̃˧ʂɯ˩-pæ˩pʰæ˩}}}}\hspace{0.5cm}[\kern2pt{\textcolor{darkblue}{\textbf{\ipa{hæ̃˧ʂɯ˩pæ˧pʰæ˧}}}}\kern2pt]} \hypertarget{h\{\string_~\string_Ms`M\string_B-p\{\string_Bp\string_h\{\string_B1}{}
\markboth{\textcolor{darkblue}{\textbf{\ipa{hæ̃˧ʂɯ˩-pæ˩pʰæ˩}}}}{}
\textcolor{teal}{\mytextsc{nom}} \hspace{4pt} Ton~: L\#-.
\textcolor{PineGreen}{\selectlanguage{french}Espalier en bois, dans la cour des fermes, pour faire sécher les épis de maïs avant égrenage.} \zh{粮架。}  \zh{量词}: \textcolor{darkblue}{\textbf{\ipa{pʰæ˧˥}}}  \mytextsc{clf}~: \textcolor{darkblue}{\textbf{\ipa{pʰæ˧˥}}} 
\lhead{\firstmark}
\rhead{\botmark}

\subsection{\hspace{-0.5cm} {\Large \textcolor{darkblue}{\textbf{\ipa{hæ̃˧ʂv̩˧pɤ˥}}}}\hspace{0.5cm}[\kern2pt{\textcolor{darkblue}{\textbf{\ipa{hæ̃˧ʂv̩˧pɤ˥}}}}\kern2pt]} \hypertarget{h\{\string_~\string_Ms`v\string_=\string_Mp7\string_T1}{}
\markboth{\textcolor{darkblue}{\textbf{\ipa{hæ̃˧ʂv̩˧pɤ˥}}}}{}
\textcolor{teal}{\mytextsc{nom}} \hspace{4pt} Ton~: H\#.
\textcolor{PineGreen}{\selectlanguage{french}Mari.} \zh{丈夫。} 
\lhead{\firstmark}
\rhead{\botmark}

\subsection{\hspace{-0.5cm} {\Large \textcolor{darkblue}{\textbf{\ipa{hæ̃˧ʐɤ˥}}}}\hspace{0.5cm}[\kern2pt{\textcolor{darkblue}{\textbf{\ipa{hæ̃˧ʐɤ˥}}}}\kern2pt]} \hypertarget{h\{\string_~\string_Mz`7\string_T1}{}
\markboth{\textcolor{darkblue}{\textbf{\ipa{hæ̃˧ʐɤ˥}}}}{}
\textcolor{teal}{\mytextsc{nom}} \hspace{4pt} Ton~: H\#.
\textcolor{PineGreen}{\selectlanguage{french}Trace de découpe, marque de coupure.} \zh{切割的痕迹。}  ¶ \textcolor{darkblue}{\textbf{\ipa{hæ̃˧ʐɤ˥ tʰv̩˩-kʰwɤ˩}}} \textcolor{PineGreen}{\selectlanguage{french}\mytextsc{n}+\mytextsc{dem}+\mytextsc{clf}: cette trace de découpe} \zh{这道割痕}  
 \zh{量词}: \textcolor{darkblue}{\textbf{\ipa{kʰwɤ˥}}}  \mytextsc{clf}~: \textcolor{darkblue}{\textbf{\ipa{kʰwɤ˥}}} 
\lhead{\firstmark}
\rhead{\botmark}

\subsection{\hspace{-0.5cm} {\Large \textcolor{darkblue}{\textbf{\ipa{hæ̃˩}}}}\hspace{0.5cm}[\kern2pt{\textcolor{darkblue}{\textbf{\ipa{hæ̃˥}}}}\kern2pt]} \hypertarget{h\{\string_~\string_B1}{}
\markboth{\textcolor{darkblue}{\textbf{\ipa{hæ̃˩}}}}{}
\textcolor{teal}{\mytextsc{nom}} \hspace{4pt} Ton~: L.
\textcolor{PineGreen}{\selectlanguage{french}Or (métal).} \zh{金子。}  \zh{量词}: \textcolor{darkblue}{\textbf{\ipa{ʈv̩˩}}}  \mytextsc{clf}~: \textcolor{darkblue}{\textbf{\ipa{ʈv̩˩}}} 
\lhead{\firstmark}
\rhead{\botmark}

\subsection{\hspace{-0.5cm} {\Large \textcolor{darkblue}{\textbf{\ipa{hæ̃˩-bɑ˧lɑ˩}}}}\hspace{0.5cm}[\kern2pt{\textcolor{darkblue}{\textbf{\ipa{xxxx non-correspondance entre le nombre de morphèmes et le nombre de tons de morphèmes}}}}\kern2pt]} \hypertarget{h\{\string_~\string_B-bA\string_MlA\string_B1}{}
\markboth{\textcolor{darkblue}{\textbf{\ipa{hæ̃˩-bɑ˧lɑ˩}}}}{}
\textcolor{teal}{\mytextsc{nom}} \hspace{4pt} Ton~: L-L\#.
\textcolor{PineGreen}{\selectlanguage{french}Soie.} \zh{丝绸。}  \zh{量词}: \textcolor{darkblue}{\textbf{\ipa{ɭɯ˧}}}  \mytextsc{clf}~: \textcolor{darkblue}{\textbf{\ipa{ɭɯ˧}}} 
\lhead{\firstmark}
\rhead{\botmark}

\subsection{\hspace{-0.5cm} {\Large \textcolor{darkblue}{\textbf{\ipa{hæ̃˩bæ˩}}}}\hspace{0.5cm}[\kern2pt{\textcolor{darkblue}{\textbf{\ipa{hæ̃˩bæ˩˥}}}}\kern2pt]} \hypertarget{h\{\string_~\string_Bb\{\string_B1}{}
\markboth{\textcolor{darkblue}{\textbf{\ipa{hæ̃˩bæ˩}}}}{}
\textcolor{teal}{\mytextsc{verbe}} \hspace{4pt} Ton~: L.
\textcolor{PineGreen}{\selectlanguage{french}Effectuer une danse rituelle.} \zh{跳大神。} 
\lhead{\firstmark}
\rhead{\botmark}

\subsection{\hspace{-0.5cm} {\Large \textcolor{darkblue}{\textbf{\ipa{hæ̃˩di˩}}}}\hspace{0.5cm}[\kern2pt{\textcolor{darkblue}{\textbf{\ipa{xxxx non-correspondance entre le nombre de morphèmes et le nombre de tons de morphèmes}}}}\kern2pt]} \hypertarget{h\{\string_~\string_Bdi\string_B1}{}
\markboth{\textcolor{darkblue}{\textbf{\ipa{hæ̃˩di˩}}}}{}
\textcolor{teal}{\mytextsc{nom}} \hspace{4pt} Ton~: L.
\textcolor{PineGreen}{\selectlanguage{french}Règle.} \zh{尺。}  \zh{量词}: \textcolor{darkblue}{\textbf{\ipa{nɑ˧}}}  \mytextsc{clf}~: \textcolor{darkblue}{\textbf{\ipa{nɑ˧}}} 
\lhead{\firstmark}
\rhead{\botmark}

\subsection{\hspace{-0.5cm} {\Large \textcolor{darkblue}{\textbf{\ipa{hæ̃˩qʰwɤ˩}}}}\hspace{0.5cm}[\kern2pt{\textcolor{darkblue}{\textbf{\ipa{hæ̃˩qʰwɤ˩˥}}}}\kern2pt]} \hypertarget{h\{\string_~\string_Bq\string_hw7\string_B1}{}
\markboth{\textcolor{darkblue}{\textbf{\ipa{hæ̃˩qʰwɤ˩}}}}{}
\textcolor{teal}{\mytextsc{nom}} \hspace{4pt} Ton~: L.
\textcolor{PineGreen}{\selectlanguage{french}Lavande.} \zh{薰衣草(永宁的一种植物)。}  \zh{量词}: \textcolor{darkblue}{\textbf{\ipa{ɭɯ˧}}}  \mytextsc{clf}~: \textcolor{darkblue}{\textbf{\ipa{ɭɯ˧}}} 
\lhead{\firstmark}
\rhead{\botmark}

\subsection{\hspace{-0.5cm} {\Large \textcolor{darkblue}{\textbf{\ipa{hæ̃˩sɤ˩}}}}\hspace{0.5cm}[\kern2pt{\textcolor{darkblue}{\textbf{\ipa{hæ̃˩sɤ˩˥}}}}\kern2pt]} \hypertarget{h\{\string_~\string_Bs7\string_B1}{}
\markboth{\textcolor{darkblue}{\textbf{\ipa{hæ̃˩sɤ˩}}}}{}
\textcolor{teal}{\mytextsc{nom}} \hspace{4pt} Ton~: L.
\textcolor{PineGreen}{\selectlanguage{french}Pie.} \zh{喜鹊。}  \zh{量词}: \textcolor{darkblue}{\textbf{\ipa{mi˩}}}  \mytextsc{clf}~: \textcolor{darkblue}{\textbf{\ipa{mi˩}}} 
\lhead{\firstmark}
\rhead{\botmark}

\subsection{\hspace{-0.5cm} {\Large \textcolor{darkblue}{\textbf{\ipa{hæ̃˧˥}}}}\hspace{0.5cm}[\kern2pt{\textcolor{darkblue}{\textbf{\ipa{hæ̃˥}}}}\kern2pt]} \hypertarget{h\{\string_~\string_M\string_T1}{}
\markboth{\textcolor{darkblue}{\textbf{\ipa{hæ̃˧˥}}}}{}
\textcolor{teal}{\mytextsc{verbe}} \hspace{4pt} Ton~: MH.
\ding{202} \textcolor{PineGreen}{\selectlanguage{french}Trancher, couper au moyen d'un instrument tranchant: couteau, épée…; ex.: tailler un vêtement.} \zh{切,裁。}  ¶ \textcolor{darkblue}{\textbf{\ipa{le˧-hæ̃˧-ze˥}}} \textcolor{PineGreen}{\selectlanguage{french}\mytextsc{accomp} \string_ \mytextsc{pfv}} \zh{切了}  
 ¶ \textcolor{darkblue}{\textbf{\ipa{tʰɑ˧-hæ̃˧˥!}}} \textcolor{PineGreen}{\selectlanguage{french}\mytextsc{prohib}} \zh{别切!}  
 ¶ \textcolor{darkblue}{\textbf{\ipa{bɑ˩lɑ˩˥ | le˧-hæ̃˧˥, | le˧-ʐv̩˧˥}}} \textcolor{PineGreen}{\selectlanguage{french}tailler des vêtements et les coudre} \zh{裁(布料来做)衣服,又缝(衣服) / 先裁布料,再缝衣服}  
\ding{203} \textcolor{PineGreen}{\selectlanguage{french}Castrer, châtrer.} \zh{阉割。} 
\lhead{\firstmark}
\rhead{\botmark}

\subsection{\hspace{-0.5cm} {\Large \textcolor{darkblue}{\textbf{\ipa{hɤ˧}}}}\hspace{0.5cm}[\kern2pt{\textcolor{darkblue}{\textbf{\ipa{hɤ˥}}}}\kern2pt]} \hypertarget{h7\string_M1}{}
\markboth{\textcolor{darkblue}{\textbf{\ipa{hɤ˧}}}}{}
\textcolor{teal}{\mytextsc{nom}} \hspace{4pt} Ton~: M.
\textcolor{PineGreen}{\selectlanguage{french}Tout.} \zh{全部。}  ¶ \textcolor{darkblue}{\textbf{\ipa{ɖɯ˧-hɤ˧ | mɤ˧-go˩}}} \textcolor{PineGreen}{\selectlanguage{french}n'avoir aucune maladie, ne souffrir de rien} \zh{一点也没病、没有任何痛苦}  
 ¶ \textcolor{darkblue}{\textbf{\ipa{ɖɯ˧-hɤ˧ | mɤ˧-sɯ˥}}} \textcolor{PineGreen}{\selectlanguage{french}ne pas savoir quoi que ce soit, être ignorant de tout, ne rien savoir du tout} \zh{什么也不知道}  
 ¶ \textcolor{darkblue}{\textbf{\ipa{ʈʂʰɯ˧ | ɖɯ˧-hɤ˧ hwæ˧}}} \textcolor{PineGreen}{\selectlanguage{french}(il/elle) achète (le) tout} \zh{他全部都买。/他什么都买。}  

\lhead{\firstmark}
\rhead{\botmark}

\subsection{\hspace{-0.5cm} {\Large \textcolor{darkblue}{\textbf{\ipa{hɤ˩\textsubscript{a}}}} \textsubscript{1}}\hspace{0.5cm}[\kern2pt{\textcolor{darkblue}{\textbf{\ipa{hɤ˩˥}}}}\kern2pt]} \hypertarget{h7\string_Ba1}{}
\markboth{\textcolor{darkblue}{\textbf{\ipa{hɤ˩\textsubscript{a}}}} \textsubscript{1}}{}
\textcolor{teal}{\mytextsc{verbe}} \hspace{4pt} Ton~: L\textsubscript{a}.
\textcolor{PineGreen}{\selectlanguage{french}Chauffer au feu, sécher au feu.} \zh{烘干。}  ¶ \textcolor{darkblue}{\textbf{\ipa{tʰi˧-hɤ˩}}} \textcolor{PineGreen}{\selectlanguage{french}\mytextsc{dur}} \zh{\mytextsc{dur}}  
 ¶ \textcolor{darkblue}{\textbf{\ipa{le˧-hɤ˩}}} \textcolor{PineGreen}{\selectlanguage{french}\mytextsc{accomp}} \zh{\mytextsc{accomp}}  
 ¶ \textcolor{darkblue}{\textbf{\ipa{ɖɯ˧-hɤ˩-ɻ̍˩}}} \textcolor{PineGreen}{\selectlanguage{french}\mytextsc{délimitatif} \string_ \mytextsc{inchoatif}: chauffer un coup, chauffer un peu} \zh{烘干一下}  
 ¶ \textcolor{darkblue}{\textbf{\ipa{le˧-hɤ˩-ze˩, | le˧-pv̩˧-ze˧!}}} \textcolor{PineGreen}{\selectlanguage{french}on l'a chauffé au feu, ça a séché} \zh{烘干了,(现在)干了!}  
 ¶ \textcolor{darkblue}{\textbf{\ipa{ɖɯ˧-kʰwɤ˧ hɤ˥}}} \textcolor{PineGreen}{\selectlanguage{french}chauffer quelque chose} \zh{烘干一个东西}  

\lhead{\firstmark}
\rhead{\botmark}

\subsection{\hspace{-0.5cm} {\Large \textcolor{darkblue}{\textbf{\ipa{hɤ˩\textsubscript{a}}}} \textsubscript{2}}\hspace{0.5cm}[\kern2pt{\textcolor{darkblue}{\textbf{\ipa{hɤ˩˥}}}}\kern2pt]} \hypertarget{h7\string_Ba2}{}
\markboth{\textcolor{darkblue}{\textbf{\ipa{hɤ˩\textsubscript{a}}}} \textsubscript{2}}{}
\textcolor{teal}{\mytextsc{verbe}} \hspace{4pt} Ton~: L\textsubscript{a}.
\textcolor{PineGreen}{\selectlanguage{french}Partir, forme passée perfective.} \zh{去,\mytextsc{过去式,°整体体。}} 
\lhead{\firstmark}
\rhead{\botmark}

\subsection{\hspace{-0.5cm} {\Large \textcolor{darkblue}{\textbf{\ipa{hɤ˩\textsubscript{a}}}} \textsubscript{3}}\hspace{0.5cm}[\kern2pt{\textcolor{darkblue}{\textbf{\ipa{hɤ˩˥}}}}\kern2pt]} \hypertarget{h7\string_Ba3}{}
\markboth{\textcolor{darkblue}{\textbf{\ipa{hɤ˩\textsubscript{a}}}} \textsubscript{3}}{}
\textcolor{teal}{\mytextsc{adjectif}} \hspace{4pt} Ton~: L\textsubscript{a}.
\textcolor{PineGreen}{\selectlanguage{french}Bon, approprié, bien; capable (adjectif), habile à une technique.} \zh{好(技巧好),好(表扬一个人的行为)。}  ¶ \textcolor{darkblue}{\textbf{\ipa{ɖwæ˧˥ | hɤ˩˥!}}} \textcolor{PineGreen}{\selectlanguage{french}\mytextsc{intensif}.très: c'est très bien!} \zh{很好!}  
 ¶ \textcolor{darkblue}{\textbf{\ipa{mɤ˧-hɤ˩}}} \textcolor{PineGreen}{\selectlanguage{french}\mytextsc{neg}} \zh{不好}  
 ¶ \textcolor{darkblue}{\textbf{\ipa{hɤ˩-hĩ˩˥}}} \textcolor{PineGreen}{\selectlanguage{french}\mytextsc{rel}/\mytextsc{nmlz}} \zh{好的}  

\lhead{\firstmark}
\rhead{\botmark}

\subsection{\hspace{-0.5cm} {\Large \textcolor{darkblue}{\textbf{\ipa{hi˥}}}}\hspace{0.5cm}[\kern2pt{\textcolor{darkblue}{\textbf{\ipa{hi˥}}}}\kern2pt]} \hypertarget{hi\string_T1}{}
\markboth{\textcolor{darkblue}{\textbf{\ipa{hi˥}}}}{}
\textcolor{teal}{\mytextsc{nom}} \hspace{4pt} Ton~: \#H.
\textcolor{PineGreen}{\selectlanguage{french}Dent.} \zh{牙齿。}  ¶ \textcolor{darkblue}{\textbf{\ipa{hi˧ go˧˥}}} \textcolor{PineGreen}{\selectlanguage{french}avoir mal aux dents} \zh{牙疼}  
 \zh{量词}: \textcolor{darkblue}{\textbf{\ipa{ɭɯ˧}}}  \mytextsc{clf}~: \textcolor{darkblue}{\textbf{\ipa{ɭɯ˧}}} 
\lhead{\firstmark}
\rhead{\botmark}

\subsection{\hspace{-0.5cm} {\Large \textcolor{darkblue}{\textbf{\ipa{*hi˧}}}}\hspace{0.5cm}[\kern2pt{\textcolor{darkblue}{\textbf{\ipa{hi˥}}}}\kern2pt]} \hypertarget{*hi\string_M1}{}
\markboth{\textcolor{darkblue}{\textbf{\ipa{*hi˧}}}}{}
\textcolor{teal}{\mytextsc{adjectif}} \hspace{4pt} Ton~: H?.
\textit{\textcolor{PineGreen}{\selectlanguage{french}archaïque}} [\zh{古语}] \textcolor{PineGreen}{\selectlanguage{french}Rapide, rapidement (racine extraite de la forme disyllabique).} \zh{快。}  ¶ \textcolor{darkblue}{\textbf{\ipa{hi˧le˩ ʝi˩}}} \textcolor{PineGreen}{\selectlanguage{french}faire rapidement} \zh{快速做}  
 ¶ \textcolor{darkblue}{\textbf{\ipa{hi˧le˩ | le˧-jo˩!}}} \textcolor{PineGreen}{\selectlanguage{french}viens vite!} \zh{快来!}  
 ¶ \textcolor{darkblue}{\textbf{\ipa{ʈʂʰɯ˧ | ɖwæ˧˥ | hi˧le˩ | ʝi˧-kv̩˩!}}} \textcolor{PineGreen}{\selectlanguage{french}Lui, il sait travailler vite!} \zh{他做事很麻利!}  

\lhead{\firstmark}
\rhead{\botmark}

\subsection{\hspace{-0.5cm} {\Large \textcolor{darkblue}{\textbf{\ipa{hi˧dʑi˧}}}}\hspace{0.5cm}[\kern2pt{\textcolor{darkblue}{\textbf{\ipa{hi˧dʑi˧}}}}\kern2pt]} \hypertarget{hi\string_Mdz£i\string_M1}{}
\markboth{\textcolor{darkblue}{\textbf{\ipa{hi˧dʑi˧}}}}{}
\textcolor{teal}{\mytextsc{nom}} \hspace{4pt} Ton~: M.
\textcolor{PineGreen}{\selectlanguage{french}Cape de pluie, vêtement qui protège de la pluie (en paille, écorce...).} \zh{蓑衣。}  ¶ \textcolor{darkblue}{\textbf{\ipa{hi˩ gi˩-ze˥, | hi˧dʑi˧ tʰi˧-mv̩˧.}}} \textcolor{PineGreen}{\selectlanguage{french}Il pleut, mets une cape de pluie.} \zh{下雨了,披蓑衣(雨衣)吧。}  
 \zh{量词}: \textcolor{darkblue}{\textbf{\ipa{ɭɯ˧}}}  \mytextsc{clf}~: \textcolor{darkblue}{\textbf{\ipa{ɭɯ˧}}} 
\lhead{\firstmark}
\rhead{\botmark}

\subsection{\hspace{-0.5cm} {\Large \textcolor{darkblue}{\textbf{\ipa{hi˧kʰɯ\#˥}}}}\hspace{0.5cm}[\kern2pt{\textcolor{darkblue}{\textbf{\ipa{hi˧kʰɯ˧}}}}\kern2pt]} \hypertarget{hi\string_Mk\string_hM\#\string_T1}{}
\markboth{\textcolor{darkblue}{\textbf{\ipa{hi˧kʰɯ\#˥}}}}{}
\textcolor{teal}{\mytextsc{nom}} \hspace{4pt} Ton~: \#H.
\textcolor{PineGreen}{\selectlanguage{french}Gencive.} \zh{牙龈。}  ¶ \textcolor{darkblue}{\textbf{\ipa{hi˧kʰɯ˧ ʈʂʰæ˧}}} \textcolor{PineGreen}{\selectlanguage{french}se brosser les dents; \textcolor{darkblue}{\textbf{\ipa{/hi˧kʰɯ˧/}}} peut désigner tout ce qu'on lave quand on se brosse les dents: gencives et dents.} \zh{刷牙}  
 ¶ \textcolor{darkblue}{\textbf{\ipa{hi˧kʰɯ˧-ʈv̩˥}}} \textcolor{PineGreen}{\selectlanguage{french}racine des dents} \zh{牙根}  

\lhead{\firstmark}
\rhead{\botmark}

\subsection{\hspace{-0.5cm} {\Large \textcolor{darkblue}{\textbf{\ipa{hi˧qʰwɤ˩}}}}\hspace{0.5cm}[\kern2pt{\textcolor{darkblue}{\textbf{\ipa{hi˧qʰwɤ˩}}}}\kern2pt]} \hypertarget{hi\string_Mq\string_hw7\string_B1}{}
\markboth{\textcolor{darkblue}{\textbf{\ipa{hi˧qʰwɤ˩}}}}{}
\textcolor{teal}{\mytextsc{nom}} \hspace{4pt} Ton~: L\#.
\textcolor{PineGreen}{\selectlanguage{french}Dent gâtée, dent cariée, carie.} \zh{蛀牙。}  \zh{量词}: \textcolor{darkblue}{\textbf{\ipa{ɭɯ˧}}}  \mytextsc{clf}~: \textcolor{darkblue}{\textbf{\ipa{ɭɯ˧}}} 
\lhead{\firstmark}
\rhead{\botmark}

\subsection{\hspace{-0.5cm} {\Large \textcolor{darkblue}{\textbf{\ipa{hi˧tʰɑ˩}}}}\hspace{0.5cm}[\kern2pt{\textcolor{darkblue}{\textbf{\ipa{hi˧tʰɑ˩}}}}\kern2pt]} \hypertarget{hi\string_Mt\string_hA\string_B1}{}
\markboth{\textcolor{darkblue}{\textbf{\ipa{hi˧tʰɑ˩}}}}{}
\textcolor{teal}{\mytextsc{adjectif}} \hspace{4pt} Ton~: L\#.
\textcolor{PineGreen}{\selectlanguage{french}Aiguisé, qui coupe bien, affûté.} \zh{锋利。} 
\lhead{\firstmark}
\rhead{\botmark}

\subsection{\hspace{-0.5cm} {\Large \textcolor{darkblue}{\textbf{\ipa{hi˧tʰo˧˥}}}}\hspace{0.5cm}[\kern2pt{\textcolor{darkblue}{\textbf{\ipa{hi˧tʰo˧˥}}}}\kern2pt]} \hypertarget{hi\string_Mt\string_ho\string_M\string_T1}{}
\markboth{\textcolor{darkblue}{\textbf{\ipa{hi˧tʰo˧˥}}}}{}
\textcolor{teal}{\mytextsc{nom}} \hspace{4pt} Ton~: MH\#.
\textcolor{PineGreen}{\selectlanguage{french}Dent.} \zh{牙齿。}  \zh{量词}: \textcolor{darkblue}{\textbf{\ipa{ɭɯ˧}}}  \mytextsc{clf}~: \textcolor{darkblue}{\textbf{\ipa{ɭɯ˧}}} 
\lhead{\firstmark}
\rhead{\botmark}

\subsection{\hspace{-0.5cm} {\Large \textcolor{darkblue}{\textbf{\ipa{hi˧tsɯ˩}}}}\hspace{0.5cm}[\kern2pt{\textcolor{darkblue}{\textbf{\ipa{hi˧tsɯ˩}}}}\kern2pt]} \hypertarget{hi\string_MtsM\string_B1}{}
\markboth{\textcolor{darkblue}{\textbf{\ipa{hi˧tsɯ˩}}}}{}
\textcolor{teal}{\mytextsc{nom}} \hspace{4pt} Ton~: L\#.
\textcolor{PineGreen}{\selectlanguage{french}Incisives (dents).} \zh{门牙。}  \zh{量词}: \textcolor{darkblue}{\textbf{\ipa{ɭɯ˧}}}  \mytextsc{clf}~: \textcolor{darkblue}{\textbf{\ipa{ɭɯ˧}}} 
\lhead{\firstmark}
\rhead{\botmark}

\subsection{\hspace{-0.5cm} {\Large \textcolor{darkblue}{\textbf{\ipa{hi˩}}} \textsubscript{1}}\hspace{0.5cm}[\kern2pt{\textcolor{darkblue}{\textbf{\ipa{hi˥}}}}\kern2pt]} \hypertarget{hi\string_B1}{}
\markboth{\textcolor{darkblue}{\textbf{\ipa{hi˩}}} \textsubscript{1}}{}
\textcolor{teal}{\mytextsc{nom}} \hspace{4pt} Ton~: L.
\textcolor{PineGreen}{\selectlanguage{french}Lac (monosyllabe).} \zh{湖、海(单音节)。}  \zh{量词}: \textcolor{darkblue}{\textbf{\ipa{ɭɯ˧}}}  \mytextsc{clf}~: \textcolor{darkblue}{\textbf{\ipa{ɭɯ˧}}} 
\lhead{\firstmark}
\rhead{\botmark}

\subsection{\hspace{-0.5cm} {\Large \textcolor{darkblue}{\textbf{\ipa{hi˩}}} \textsubscript{2}}\hspace{0.5cm}[\kern2pt{\textcolor{darkblue}{\textbf{\ipa{hi˩˥}}}}\kern2pt]} \hypertarget{hi\string_B2}{}
\markboth{\textcolor{darkblue}{\textbf{\ipa{hi˩}}} \textsubscript{2}}{}
\textcolor{teal}{\mytextsc{verbe}} \hspace{4pt} Ton~: L.
\textcolor{PineGreen}{\selectlanguage{french}Exister, se trouver: verbe d'existence pour objets non mobiles, par exemple le Lac existe/se trouve à un endroit.} \zh{存在动词:固定不动的物体,如:泸沽湖。}  ¶ \textcolor{darkblue}{\textbf{\ipa{hi˩nɑ˧mi˧ | tʰi˧-hi˩}}} \textcolor{PineGreen}{\selectlanguage{french}le Lac se trouve là/existe là/se trouve là, immuable} \zh{有(泸沽)湖(在那儿)}  

\lhead{\firstmark}
\rhead{\botmark}

\subsection{\hspace{-0.5cm} {\Large \textcolor{darkblue}{\textbf{\ipa{hi˩dʑɯ˩}}}}\hspace{0.5cm}[\kern2pt{\textcolor{darkblue}{\textbf{\ipa{hi˩dʑɯ˩˥}}}}\kern2pt]} \hypertarget{hi\string_Bdz£M\string_B1}{}
\markboth{\textcolor{darkblue}{\textbf{\ipa{hi˩dʑɯ˩}}}}{}
\textcolor{teal}{\mytextsc{nom}} \hspace{4pt} Ton~: L.
\textcolor{PineGreen}{\selectlanguage{french}Charbon de bois.} \zh{炭。}  \zh{量词}: \textcolor{darkblue}{\textbf{\ipa{kʰɤ˧˥}}}  \mytextsc{clf}~: \textcolor{darkblue}{\textbf{\ipa{kʰɤ˧˥}}} 
\lhead{\firstmark}
\rhead{\botmark}

\subsection{\hspace{-0.5cm} {\Large \textcolor{darkblue}{\textbf{\ipa{hi˩mi˩}}}}\hspace{0.5cm}[\kern2pt{\textcolor{darkblue}{\textbf{\ipa{hi˩mi˩˥}}}}\kern2pt]} \hypertarget{hi\string_Bmi\string_B1}{}
\markboth{\textcolor{darkblue}{\textbf{\ipa{hi˩mi˩}}}}{}
\textcolor{teal}{\mytextsc{nom}} \hspace{4pt} Ton~: L.
\textcolor{PineGreen}{\selectlanguage{french}Langue.} \zh{舌头。}  ¶ \textcolor{darkblue}{\textbf{\ipa{hi˩mi˩˥, | ɻ̃˧ mɤ˧-ʑi˧! | ə˧tso˧ ʐwɤ˩-bi˩, | õ˧-lɑ˥ ɖʐv̩˩!}}} \textcolor{PineGreen}{\selectlanguage{french}“La langue n'a pas d'os! Ce qu'on dit, soi seul sait (si c'est la vérité)!”} \zh{“舌头没有骨头。讲的是什么(=是否真的),只有自己才知道!”(谚语)}  
 \zh{量词}: \textcolor{darkblue}{\textbf{\ipa{ɭɯ˧}}}  \mytextsc{clf}~: \textcolor{darkblue}{\textbf{\ipa{ɭɯ˧}}} 
\lhead{\firstmark}
\rhead{\botmark}

\subsection{\hspace{-0.5cm} {\Large \textcolor{darkblue}{\textbf{\ipa{hi˩nɑ˧mi\#˥}}}}\hspace{0.5cm}[\kern2pt{\textcolor{darkblue}{\textbf{\ipa{hi˩nɑ˧mi˧}}}}\kern2pt]} \hypertarget{hi\string_BnA\string_Mmi\#\string_T1}{}
\markboth{\textcolor{darkblue}{\textbf{\ipa{hi˩nɑ˧mi\#˥}}}}{}
\textcolor{teal}{\mytextsc{nom}} \hspace{4pt} Ton~: LM+\#H.
\textcolor{PineGreen}{\selectlanguage{french}Lac.} \zh{湖。}  \zh{量词}: \textcolor{darkblue}{\textbf{\ipa{ɭɯ˧}}}  \mytextsc{clf}~: \textcolor{darkblue}{\textbf{\ipa{ɭɯ˧}}} 
\lhead{\firstmark}
\rhead{\botmark}

\subsection{\hspace{-0.5cm} {\Large \textcolor{darkblue}{\textbf{\ipa{hi˩ɲi˩zo˩}}}}\hspace{0.5cm}[\kern2pt{\textcolor{darkblue}{\textbf{\ipa{hi˩ɲi˩zo˩˥}}}}\kern2pt]} \hypertarget{hi\string_BJi\string_Bzo\string_B1}{}
\markboth{\textcolor{darkblue}{\textbf{\ipa{hi˩ɲi˩zo˩}}}}{}
\textcolor{teal}{\mytextsc{nom}} \hspace{4pt} Ton~: L.
\textcolor{PineGreen}{\selectlanguage{french}Salamandre.} \zh{娃娃鱼。}  \zh{量词}: \textcolor{darkblue}{\textbf{\ipa{mi˩}}}  \mytextsc{clf}~: \textcolor{darkblue}{\textbf{\ipa{mi˩}}} 
\lhead{\firstmark}
\rhead{\botmark}

\subsection{\hspace{-0.5cm} {\Large \textcolor{darkblue}{\textbf{\ipa{hi˩qʰɑ˩}}}}\hspace{0.5cm}[\kern2pt{\textcolor{darkblue}{\textbf{\ipa{hi˩qʰɑ˩˥}}}}\kern2pt]} \hypertarget{hi\string_Bq\string_hA\string_B1}{}
\markboth{\textcolor{darkblue}{\textbf{\ipa{hi˩qʰɑ˩}}}}{}
\textcolor{teal}{\mytextsc{nom}} \hspace{4pt} Ton~: L.
\textcolor{PineGreen}{\selectlanguage{french}Orage.} \zh{暴雨。}  ¶ \textcolor{darkblue}{\textbf{\ipa{hi˩qʰɑ˩ lɑ˥(-ze˩)}}} \textcolor{PineGreen}{\selectlanguage{french}l'orage éclate, il y a de l'orage} \zh{下暴雨了}  
 \zh{量词}: \textcolor{darkblue}{\textbf{\ipa{ʂɯ˩}}}  \mytextsc{clf}~: \textcolor{darkblue}{\textbf{\ipa{ʂɯ˩}}} \textcolor{PineGreen}{\selectlanguage{french}cl des fois}
\lhead{\firstmark}
\rhead{\botmark}

\subsection{\hspace{-0.5cm} {\Large \textcolor{darkblue}{\textbf{\ipa{hi˩ʁwɤ˩-lo˧}}}}\hspace{0.5cm}[\kern2pt{\textcolor{darkblue}{\textbf{\ipa{hi˩ʁwɤ˩lo˥}}}}\kern2pt]} \hypertarget{hi\string_BRw7\string_B-lo\string_M1}{}
\markboth{\textcolor{darkblue}{\textbf{\ipa{hi˩ʁwɤ˩-lo˧}}}}{}
\textcolor{teal}{\mytextsc{nom}} \hspace{4pt} Ton~: L-.
\textcolor{PineGreen}{\selectlanguage{french}Un village de la plaine de Yongning.} \zh{永宁的一个村落。}  ¶ \textcolor{darkblue}{\textbf{\ipa{dʑɤ˩bv̩˧kɤ˧-sɑ˥ʁwɤ˩, | hi˩ʁwɤ˩-lo˥, | æ˩mi˧-ʁwɤ\#˥, | lɑ˧lo˧-ʁwɤ˥, | lɑ˧ŋwɤ˧, | bɤ˧tsʰo˧gv̩˥, | ə˧lɑ˧-ʁwɤ\#˥, | gæ˧ɻæ˩, | qʰæ˧tɕʰi˧, | tʰo˧ʈɯ\#˥}}} \textcolor{PineGreen}{\selectlanguage{french}les dix villages comptant traditionnellement comme faisant partie de Yongning} \zh{摩梭传统地理概念中,属于永宁的十个村落}  

\lhead{\firstmark}
\rhead{\botmark}

\subsection{\hspace{-0.5cm} {\Large \textcolor{darkblue}{\textbf{\ipa{hi˩ʐæ˥}}}}\hspace{0.5cm}[\kern2pt{\textcolor{darkblue}{\textbf{\ipa{hi˩ʐæ˥}}}}\kern2pt]} \hypertarget{hi\string_Bz`\{\string_T1}{}
\markboth{\textcolor{darkblue}{\textbf{\ipa{hi˩ʐæ˥}}}}{}
\textcolor{teal}{\mytextsc{nom}} \hspace{4pt} Ton~: LH.
\ding{202} \textcolor{PineGreen}{\selectlanguage{french}Luette.} \zh{小舌。}  ¶ \textcolor{darkblue}{\textbf{\ipa{qv̩˧ʈʂæ˧-bv̩˥ | hi˩ʐæ˧}}} \textcolor{PineGreen}{\selectlanguage{french}la luette; la précision '...de la gorge' permet de lever l'ambiguïté lorsqu'il pourrait s'agir du tendon de la langue, lui aussi désigné comme \textcolor{darkblue}{\textbf{\ipa{/hi˩ʐæ˥/}}}.} \zh{小舌}  
 \zh{量词}: \textcolor{darkblue}{\textbf{\ipa{ɭɯ˧}}} \ding{203} \textcolor{PineGreen}{\selectlanguage{french}Tendon de la langue.} \zh{舌头的筋。}  ¶ \textcolor{darkblue}{\textbf{\ipa{hi˩mi˩-bv̩˧ | hi˩ʐæ˧}}} \textcolor{PineGreen}{\selectlanguage{french}le tendon de la langue; la précision 'de la langue' permet de lever l'ambiguïté dans les cas où il pourrait aussi s'agir de la luette, elle aussi désignée comme \textcolor{darkblue}{\textbf{\ipa{/hi˩ʐæ˥/}}}.} \zh{舌头的筋}  
 \mytextsc{clf}~: \textcolor{darkblue}{\textbf{\ipa{ɭɯ˧}}} 
\lhead{\firstmark}
\rhead{\botmark}

\subsection{\hspace{-0.5cm} {\Large \textcolor{darkblue}{\textbf{\ipa{hi˩˥}}}}\hspace{0.5cm}[\kern2pt{\textcolor{darkblue}{\textbf{\ipa{hi˩˥}}}}\kern2pt]} \hypertarget{hi\string_B\string_T1}{}
\markboth{\textcolor{darkblue}{\textbf{\ipa{hi˩˥}}}}{}
\textcolor{teal}{\mytextsc{nom}} \hspace{4pt} Ton~: LH.
\textcolor{PineGreen}{\selectlanguage{french}Pluie.} \zh{雨。}  ¶ \textcolor{darkblue}{\textbf{\ipa{hi˩ gi˩˥ / hi˩ gi˩-ze˥}}} \textcolor{PineGreen}{\selectlanguage{french}il pleut} \zh{下雨了}  
 \zh{量词}: \textcolor{darkblue}{\textbf{\ipa{ʂɯ˩}}}  \mytextsc{clf}~: \textcolor{darkblue}{\textbf{\ipa{ʂɯ˩}}} \textcolor{PineGreen}{\selectlanguage{french}fois}
\lhead{\firstmark}
\rhead{\botmark}

\subsection{\hspace{-0.5cm} {\Large \textcolor{darkblue}{\textbf{\ipa{‑hĩ˥}}}}\hspace{0.5cm}[\kern2pt{\textcolor{darkblue}{\textbf{\ipa{xxxx groupe tonal entier sans aucun ton}}}}\kern2pt]} \hypertarget{‑hi\string_~\string_T1}{}
\markboth{\textcolor{darkblue}{\textbf{\ipa{‑hĩ˥}}}}{}
\textcolor{teal}{\mytextsc{conjonction}} \hspace{4pt} Ton~: 0.
\textcolor{PineGreen}{\selectlanguage{french}Relativisateur et nominalisateur.} \zh{关系从句/名词化。} 
\lhead{\firstmark}
\rhead{\botmark}

\subsection{\hspace{-0.5cm} {\Large \textcolor{darkblue}{\textbf{\ipa{hĩ˥}}}}\hspace{0.5cm}[\kern2pt{\textcolor{darkblue}{\textbf{\ipa{hĩ˥}}}}\kern2pt]} \hypertarget{hi\string_~\string_T1}{}
\markboth{\textcolor{darkblue}{\textbf{\ipa{hĩ˥}}}}{}
\textcolor{teal}{\mytextsc{nom}} \hspace{4pt} Ton~: \#H.
\textcolor{PineGreen}{\selectlanguage{french}Personne; être humain; homme (sans indication de genre).} \zh{人。}  ¶ \textcolor{darkblue}{\textbf{\ipa{hĩ˧ | ɖɯ˧-v̩˧}}} \textcolor{PineGreen}{\selectlanguage{french}une personne} \zh{一个人}  
 ¶ \textcolor{darkblue}{\textbf{\ipa{hĩ˧-ɻ̃˧ | ɖɯ˧-lo˩}}} \textcolor{PineGreen}{\selectlanguage{french}une lignée, une famille} \zh{一个家族}  
 ¶ \textcolor{darkblue}{\textbf{\ipa{hĩ˧-mv˥ hĩ˩-di˩}}} \textcolor{PineGreen}{\selectlanguage{french}les terres étrangères, les terres d'autres gens (par opposition avec sa propre terre natale)} \zh{人家的地方,人家的故乡(不是自己的地方)}  
 ¶ \textcolor{darkblue}{\textbf{\ipa{hĩ˧-mv˥ hĩ˩-di˩ | qʰɑ˧-dʑɤ˥\textasciitilde{}dʑɤ˩, | õ˧-mv˥ õ˩-di˩ tsʰe˩ mɤ˩-gv˩!}}} \textcolor{PineGreen}{\selectlanguage{french}Si belles soient les terres d'autrui, elles n'auront jamais la beauté de ses propres terres / de la terre natale !} \zh{其他人的地方怎么好,也比不过自己的地方!}  
 \zh{量词}: \textcolor{darkblue}{\textbf{\ipa{v̩˧}}}  \mytextsc{clf}~: \textcolor{darkblue}{\textbf{\ipa{v̩˧}}} 
\lhead{\firstmark}
\rhead{\botmark}

\subsection{\hspace{-0.5cm} {\Large \textcolor{darkblue}{\textbf{\ipa{hĩ˧bæ\#˥}}}}\hspace{0.5cm}[\kern2pt{\textcolor{darkblue}{\textbf{\ipa{hĩ˧bæ˧}}}}\kern2pt]} \hypertarget{hi\string_~\string_Mb\{\#\string_T1}{}
\markboth{\textcolor{darkblue}{\textbf{\ipa{hĩ˧bæ\#˥}}}}{}
\textcolor{teal}{\mytextsc{nom}} \hspace{4pt} Ton~: \#H.
\textcolor{PineGreen}{\selectlanguage{french}Invité, visiteur, hôte.} \zh{客人。}  ¶ \textcolor{darkblue}{\textbf{\ipa{hĩ˧bæ˧ ʝi˧}}} \textcolor{PineGreen}{\selectlanguage{french}participer à une fête en tant qu'invité, se rendre à une fête/à une invitation} \zh{做客}  
 ¶ \textcolor{darkblue}{\textbf{\ipa{hĩ˧bæ˧ tsʰɯ˧-ze˥ ! |}}} \textcolor{PineGreen}{\selectlanguage{french}Un invité est arrivé!} \zh{客人来了!}  
 \zh{量词}: \textcolor{darkblue}{\textbf{\ipa{v̩˧}}}  \mytextsc{clf}~: \textcolor{darkblue}{\textbf{\ipa{v̩˧}}} 
\lhead{\firstmark}
\rhead{\botmark}

\subsection{\hspace{-0.5cm} {\Large \textcolor{darkblue}{\textbf{\ipa{hĩ˧hĩ\#˥}}}}\hspace{0.5cm}[\kern2pt{\textcolor{darkblue}{\textbf{\ipa{hĩ˧hĩ˧}}}}\kern2pt]} \hypertarget{hi\string_~\string_Mhi\string_~\#\string_T1}{}
\markboth{\textcolor{darkblue}{\textbf{\ipa{hĩ˧hĩ\#˥}}}}{}
\textcolor{teal}{\mytextsc{nom}} \hspace{4pt} Ton~: \#H.
\textcolor{PineGreen}{\selectlanguage{french}Les gens extérieurs à la famille (s'oppose à: “les gens de la famille”).} \zh{外人。}  \zh{量词}: \textcolor{darkblue}{\textbf{\ipa{v̩˧}}}  \mytextsc{clf}~: \textcolor{darkblue}{\textbf{\ipa{v̩˧}}} 
\lhead{\firstmark}
\rhead{\botmark}

\subsection{\hspace{-0.5cm} {\Large \textcolor{darkblue}{\textbf{\ipa{hĩ˧-lɑ˩-kv̩˩-hĩ˩}}}}\hspace{0.5cm}[\kern2pt{\textcolor{darkblue}{\textbf{\ipa{hĩ˧lɑ˧kv̩˧hĩ˧}}}}\kern2pt]} \hypertarget{hi\string_~\string_M-lA\string_B-kv\string_=\string_B-hi\string_~\string_B1}{}
\markboth{\textcolor{darkblue}{\textbf{\ipa{hĩ˧-lɑ˩-kv̩˩-hĩ˩}}}}{}
\textcolor{teal}{\mytextsc{nom}} \hspace{4pt} Ton~: -L--.
\textcolor{PineGreen}{\selectlanguage{french}Personne dangereuse, ennemi, bandit; littéralement: “personne susceptible de frapper les gens”.} \zh{危险的人,仇人,敌人。} 
\lhead{\firstmark}
\rhead{\botmark}

\subsection{\hspace{-0.5cm} {\Large \textcolor{darkblue}{\textbf{\ipa{hĩ˧mo˥}}}}\hspace{0.5cm}[\kern2pt{\textcolor{darkblue}{\textbf{\ipa{hĩ˧mo˥}}}}\kern2pt]} \hypertarget{hi\string_~\string_Mmo\string_T1}{}
\markboth{\textcolor{darkblue}{\textbf{\ipa{hĩ˧mo˥}}}}{}
\textcolor{teal}{\mytextsc{nom}} \hspace{4pt} Ton~: H\#.
\textcolor{PineGreen}{\selectlanguage{french}Personne âgée, vieillard, vieillarde.} \zh{老人。}  ¶ \textcolor{darkblue}{\textbf{\ipa{hĩ˧mo˥-hĩ˩}}} \textcolor{PineGreen}{\selectlanguage{french}\string_ \mytextsc{rel;} même sens} \zh{老人、老的人}  
 \zh{量词}: \textcolor{darkblue}{\textbf{\ipa{v̩˧}}}  \mytextsc{clf}~: \textcolor{darkblue}{\textbf{\ipa{v̩˧}}} 
\lhead{\firstmark}
\rhead{\botmark}

\subsection{\hspace{-0.5cm} {\Large \textcolor{darkblue}{\textbf{\ipa{hĩ˧mo˩}}}}\hspace{0.5cm}[\kern2pt{\textcolor{darkblue}{\textbf{\ipa{hĩ˧mo˩}}}}\kern2pt]} \hypertarget{hi\string_~\string_Mmo\string_B1}{}
\markboth{\textcolor{darkblue}{\textbf{\ipa{hĩ˧mo˩}}}}{}
\textcolor{teal}{\mytextsc{nom}} \hspace{4pt} Ton~: L\#.
\ding{202} \textcolor{PineGreen}{\selectlanguage{french}Cadavre.} \zh{尸体。}  ¶ \textcolor{darkblue}{\textbf{\ipa{hĩ˧mo˩-kʰɯ˩-di˩}}} \textcolor{PineGreen}{\selectlanguage{french}cercueil (périphrase: “objet (dans lequel) on met le cadavre”)} \zh{棺材}  
 \zh{量词}: \textcolor{darkblue}{\textbf{\ipa{mo˧}}} \ding{203} \textcolor{PineGreen}{\selectlanguage{french}Tombe, tombeau.} \zh{坟墓。}  \mytextsc{clf}~: \textcolor{darkblue}{\textbf{\ipa{mo˧}}} 
\lhead{\firstmark}
\rhead{\botmark}

\subsection{\hspace{-0.5cm} {\Large \textcolor{darkblue}{\textbf{\ipa{hĩ˧-tɕʰɯ\#˥}}}}\hspace{0.5cm}[\kern2pt{\textcolor{darkblue}{\textbf{\ipa{xxxx non-correspondance entre le nombre de morphèmes et le nombre de tons de morphèmes}}}}\kern2pt]} \hypertarget{hi\string_~\string_M-ts£\string_hM\#\string_T1}{}
\markboth{\textcolor{darkblue}{\textbf{\ipa{hĩ˧-tɕʰɯ\#˥}}}}{}
\textcolor{teal}{\mytextsc{nom}} \hspace{4pt} Ton~: \#H.
\textcolor{PineGreen}{\selectlanguage{french}Membre de la famille de même génération: frère, sœur, cousin(e) (du côté maternel).} \zh{同一辈的亲戚:兄弟姐妹、堂兄弟姐妹。}  ¶ \textcolor{darkblue}{\textbf{\ipa{hĩ˧-tɕʰɯ˧ - hĩ˧-ʈʂɤ\#˥}}} \textcolor{PineGreen}{\selectlanguage{french}même sens: les gens de la même génération, dans la famille: frères, sœurs, mais aussi cousins du côté maternel} \zh{同一辈的亲戚:兄弟姐妹、堂兄弟姐妹}  
 ¶ \textcolor{darkblue}{\textbf{\ipa{ʈʂʰɯ˧ | njɤ˧ | hĩ˧ tɕʰɯ˧ ɲi˥!}}} \textcolor{PineGreen}{\selectlanguage{french}C'est mon cousin/ma cousine/quelqu'un de ma fratrie!} \zh{他是跟我同一辈的亲戚!(=堂兄弟姐妹)}  
 ¶ \textcolor{darkblue}{\textbf{\ipa{hĩ˧-tɕʰɯ˧ mɤ˧-ɲi˥ F | hĩ˧-tɕʰɯ˧ ʝi˧ tʰɑ˩-kv̩˩!}}} \textcolor{PineGreen}{\selectlanguage{french}“Même si on n'est pas de la même famille (au départ), on peut le devenir!” Formule traditionnelle pour désigner les liens quasi-familiaux tissés entre amis, qui reviennent à des formes d'adoption au sein du cercle familial.} \zh{“不是亲戚,也可以变成亲戚!”这个俗语来形容朋友之间的深情,变成像家人之间的感情。}  
\textit{Syn~:} \hyperlink{}{\textcolor{darkblue}{\textbf{\ipa{hĩ˧-ʈʂɤ\#˥}}}}. 
\lhead{\firstmark}
\rhead{\botmark}

\subsection{\hspace{-0.5cm} {\Large \textcolor{darkblue}{\textbf{\ipa{hĩ˧-ʈʂɤ\#˥}}}}\hspace{0.5cm}[\kern2pt{\textcolor{darkblue}{\textbf{\ipa{xxxx non-correspondance entre le nombre de morphèmes et le nombre de tons de morphèmes}}}}\kern2pt]} \hypertarget{hi\string_~\string_M-t`s`7\#\string_T1}{}
\markboth{\textcolor{darkblue}{\textbf{\ipa{hĩ˧-ʈʂɤ\#˥}}}}{}
\textcolor{teal}{\mytextsc{nom}} \hspace{4pt} Ton~: \#H.
\textcolor{PineGreen}{\selectlanguage{french}Membre de la famille de même génération: frère, sœur, cousin(e) (du côté maternel).} \zh{同一辈的亲戚:兄弟姐妹、堂兄弟姐妹。}  ¶ \textcolor{darkblue}{\textbf{\ipa{hĩ˧-tɕʰɯ˧ - hĩ˧-ʈʂɤ\#˥}}} \textcolor{PineGreen}{\selectlanguage{french}même sens: les gens de la même génération, dans la famille: frères, sœurs, mais aussi cousins du côté maternel} \zh{同一辈的亲戚:兄弟姐妹、堂兄弟姐妹}  
\textit{Syn~:} \hyperlink{}{\textcolor{darkblue}{\textbf{\ipa{hĩ˧-tɕʰɯ\#˥}}}}. 
\lhead{\firstmark}
\rhead{\botmark}

\subsection{\hspace{-0.5cm} {\Large \textcolor{darkblue}{\textbf{\ipa{hĩ˧˥}}} \textsubscript{1}}\hspace{0.5cm}[\kern2pt{\textcolor{darkblue}{\textbf{\ipa{hĩ˧˥}}}}\kern2pt]} \hypertarget{hi\string_~\string_M\string_T1}{}
\markboth{\textcolor{darkblue}{\textbf{\ipa{hĩ˧˥}}} \textsubscript{1}}{}
\textcolor{teal}{\mytextsc{verbe}} \hspace{4pt} Ton~: MH.
\textcolor{PineGreen}{\selectlanguage{french}Être debout, se tenir debout.} \zh{站(站立)。} 
\lhead{\firstmark}
\rhead{\botmark}

\subsection{\hspace{-0.5cm} {\Large \textcolor{darkblue}{\textbf{\ipa{hĩ˧˥}}} \textsubscript{2}}\hspace{0.5cm}[\kern2pt{\textcolor{darkblue}{\textbf{\ipa{hĩ˧˥}}}}\kern2pt]} \hypertarget{hi\string_~\string_M\string_T2}{}
\markboth{\textcolor{darkblue}{\textbf{\ipa{hĩ˧˥}}} \textsubscript{2}}{}
\textcolor{teal}{\mytextsc{verbe}} \hspace{4pt} Ton~: MH.
\textcolor{PineGreen}{\selectlanguage{french}Devoir, falloir.} \zh{应该。}  ¶ \textcolor{darkblue}{\textbf{\ipa{mɤ˧-hĩ˧}}} \textcolor{PineGreen}{\selectlanguage{french}\mytextsc{neg}} \zh{\mytextsc{否定}}  
 ¶ \textcolor{darkblue}{\textbf{\ipa{no˧ | ʝi˧-hĩ˧˥!}}} \textcolor{PineGreen}{\selectlanguage{french}c'est à toi de le faire! / il faut que tu le fasses!} \zh{你应该做!}  
 ¶ \textcolor{darkblue}{\textbf{\ipa{njɤ˧ | ʝi˧-mɤ˧-hĩ˧-hĩ˥ | (ɖɯ˧-pi˧) ʝi˧-ze˩! |}}} \textcolor{PineGreen}{\selectlanguage{french}j'ai fait quelque chose que j'aurais pas dû!} \zh{我做了一件不应该做的事!}  
 ¶ \textcolor{darkblue}{\textbf{\ipa{no˧ | lo˧ ʝi˧-hĩ˧!}}} \textcolor{PineGreen}{\selectlanguage{french}Il faut que tu travailles!} \zh{你应该工作啊!}  

\lhead{\firstmark}
\rhead{\botmark}

\subsection{\hspace{-0.5cm} {\Large \textcolor{darkblue}{\textbf{\ipa{ho˥}}} \textsubscript{1}}\hspace{0.5cm}[\kern2pt{\textcolor{darkblue}{\textbf{\ipa{ho˧˥}}}}\kern2pt]} \hypertarget{ho\string_T1}{}
\markboth{\textcolor{darkblue}{\textbf{\ipa{ho˥}}} \textsubscript{1}}{}
\textcolor{teal}{\mytextsc{nom}} \hspace{4pt} Ton~: \#H.
\textcolor{PineGreen}{\selectlanguage{french}Faisan (utilisé aussi pour: cailles, et certaines poules sauvages).} \zh{雉。}  ¶ \textcolor{darkblue}{\textbf{\ipa{ho˧ tʰv̩˧-mi˧˥ / ho˧ tʰv̩˧-mi˥\#}}} \textcolor{PineGreen}{\selectlanguage{french}\mytextsc{n}+\mytextsc{dem}+\mytextsc{clf}} \zh{这只雉}  
 \zh{量词}: \textcolor{darkblue}{\textbf{\ipa{mi˩}}}  \mytextsc{clf}~: \textcolor{darkblue}{\textbf{\ipa{mi˩}}} 
\lhead{\firstmark}
\rhead{\botmark}

\subsection{\hspace{-0.5cm} {\Large \textcolor{darkblue}{\textbf{\ipa{ho˥}}} \textsubscript{2}}\hspace{0.5cm}[\kern2pt{\textcolor{darkblue}{\textbf{\ipa{ho˥}}}}\kern2pt]} \hypertarget{ho\string_T2}{}
\markboth{\textcolor{darkblue}{\textbf{\ipa{ho˥}}} \textsubscript{2}}{}
\textcolor{teal}{\mytextsc{nom}} \hspace{4pt} Ton~: \#H.
\textcolor{PineGreen}{\selectlanguage{french}Gruau.} \zh{粥。}  ¶ \textcolor{darkblue}{\textbf{\ipa{ho˧ ʈʰɯ˧˥}}} \textcolor{PineGreen}{\selectlanguage{french}boire du gruau} \zh{喝粥}  

\lhead{\firstmark}
\rhead{\botmark}

\subsection{\hspace{-0.5cm} {\Large \textcolor{darkblue}{\textbf{\ipa{ho˧ɕjæ˩}}}}\hspace{0.5cm}[\kern2pt{\textcolor{darkblue}{\textbf{\ipa{ho˩ɕjæ˥}}}}\kern2pt]} \hypertarget{ho\string_Ms£j\{\string_B1}{}
\markboth{\textcolor{darkblue}{\textbf{\ipa{ho˧ɕjæ˩}}}}{}
\textcolor{teal}{\mytextsc{nom}} \hspace{4pt} Ton~: LM.
\textcolor{PineGreen}{\selectlanguage{french}Mèche.} \zh{火绳,导火索。} Dialecte chinois local~:\zh{火线。} Emprunt~: chinois  \zh{火线}

\lhead{\firstmark}
\rhead{\botmark}

\subsection{\hspace{-0.5cm} {\Large \textcolor{darkblue}{\textbf{\ipa{ho˧di˧}}}}\hspace{0.5cm}[\kern2pt{\textcolor{darkblue}{\textbf{\ipa{ho˩di˥}}}}\kern2pt]} \hypertarget{ho\string_Mdi\string_M1}{}
\markboth{\textcolor{darkblue}{\textbf{\ipa{ho˧di˧}}}}{}
\textcolor{teal}{\mytextsc{nom}} \hspace{4pt} Ton~: M.
\textcolor{PineGreen}{\selectlanguage{french}Régions chinoises (Han) du Sichuan: Yanyuan, Yanbian, Xichang...} \zh{四川(盐源、盐边、西昌……)。} 
\lhead{\firstmark}
\rhead{\botmark}

\subsection{\hspace{-0.5cm} {\Large \textcolor{darkblue}{\textbf{\ipa{ho˧dʑɯ˧tɤ˥ɻ̍˩}}}}\hspace{0.5cm}[\kern2pt{\textcolor{darkblue}{\textbf{\ipa{ho˧dʑɯ˧tɤ˧ɻ̍˧˥}}}}\kern2pt]} \hypertarget{ho\string_Mdz£M\string_Mt7\string_Tr£`̍\string_B1}{}
\markboth{\textcolor{darkblue}{\textbf{\ipa{ho˧dʑɯ˧tɤ˥ɻ̍˩}}}}{}
\textcolor{teal}{\mytextsc{nom}} \hspace{4pt} Ton~: \#H-.
\textcolor{PineGreen}{\selectlanguage{french}Pâte, colle à base de farine, liquide visqueux.} \zh{浆糊,浆子。} \textit{Voir~:} \hyperlink{}{\textcolor{darkblue}{\textbf{\ipa{ho˧dʑɯ˧˥}}}} 
\lhead{\firstmark}
\rhead{\botmark}

\subsection{\hspace{-0.5cm} {\Large \textcolor{darkblue}{\textbf{\ipa{ho˧dʑɯ˧˥}}}}\hspace{0.5cm}[\kern2pt{\textcolor{darkblue}{\textbf{\ipa{ho˩dʑɯ˩˥}}}}\kern2pt]} \hypertarget{ho\string_Mdz£M\string_M\string_T1}{}
\markboth{\textcolor{darkblue}{\textbf{\ipa{ho˧dʑɯ˧˥}}}}{}
\textcolor{teal}{\mytextsc{nom}} \hspace{4pt} Ton~: MH\#.
\textcolor{PineGreen}{\selectlanguage{french}Pâte, colle à base de farine, liquide visqueux.} \zh{浆糊,浆子。} \textit{Voir~:} \hyperlink{}{\textcolor{darkblue}{\textbf{\ipa{ho˧dʑɯ˧tɤ˥ɻ̍˩}}}} 
\lhead{\firstmark}
\rhead{\botmark}

\subsection{\hspace{-0.5cm} {\Large \textcolor{darkblue}{\textbf{\ipa{ho˧ko˧}}}}\hspace{0.5cm}[\kern2pt{\textcolor{darkblue}{\textbf{\ipa{xxxx non-correspondance entre le nombre de morphèmes et le nombre de tons de morphèmes}}}}\kern2pt]} \hypertarget{ho\string_Mko\string_M1}{}
\markboth{\textcolor{darkblue}{\textbf{\ipa{ho˧ko˧}}}}{}
\textcolor{teal}{\mytextsc{nom}} \hspace{4pt} Ton~: M.
\textcolor{PineGreen}{\selectlanguage{french}Grand récipient pour faire la fondue mongole.} \zh{火锅(汉语借词)。}  Emprunt~: chinois  \zh{火锅}
 ¶ \textcolor{darkblue}{\textbf{\ipa{æ̃˧-ho˧ko˥}}} \textcolor{PineGreen}{\selectlanguage{french}récipient pour fondue en cuivre} \zh{铜火锅}  

\lhead{\firstmark}
\rhead{\botmark}

\subsection{\hspace{-0.5cm} {\Large \textcolor{darkblue}{\textbf{\ipa{ho˧mi\#˥}}}}\hspace{0.5cm}[\kern2pt{\textcolor{darkblue}{\textbf{\ipa{ho˩mi˥}}}}\kern2pt]} \hypertarget{ho\string_Mmi\#\string_T1}{}
\markboth{\textcolor{darkblue}{\textbf{\ipa{ho˧mi\#˥}}}}{}
\textcolor{teal}{\mytextsc{nom}} \hspace{4pt} Ton~: \#H.
\textcolor{PineGreen}{\selectlanguage{french}Faisan femelle.} \zh{母雉。}  ¶ \textcolor{darkblue}{\textbf{\ipa{ho˧mi˧ tʰv̩˧-mi˧˥ / ho˧mi˧ tʰv̩˧-mi˥\#}}} \textcolor{PineGreen}{\selectlanguage{french}\mytextsc{n}+\mytextsc{dem}+\mytextsc{clf}} \zh{这个母雉}  
 ¶ \textcolor{darkblue}{\textbf{\ipa{ho˧mi˧-ho˧pʰv̩˥ / ho˧mi˧-ho˥pʰv̩˩}}} \textcolor{PineGreen}{\selectlanguage{french}faisan femelle et faisan mâle} \zh{母雉与公雉}  
 \zh{量词}: \textcolor{darkblue}{\textbf{\ipa{mi˩}}}  \mytextsc{clf}~: \textcolor{darkblue}{\textbf{\ipa{mi˩}}} 
\lhead{\firstmark}
\rhead{\botmark}

\subsection{\hspace{-0.5cm} {\Large \textcolor{darkblue}{\textbf{\ipa{ho˧pʰv̩\#˥}}}}\hspace{0.5cm}[\kern2pt{\textcolor{darkblue}{\textbf{\ipa{ho˧pʰv̩˧}}}}\kern2pt]} \hypertarget{ho\string_Mp\string_hv\string_=\#\string_T1}{}
\markboth{\textcolor{darkblue}{\textbf{\ipa{ho˧pʰv̩\#˥}}}}{}
\textcolor{teal}{\mytextsc{nom}} \hspace{4pt} Ton~: \#H.
\textcolor{PineGreen}{\selectlanguage{french}Faisan mâle.} \zh{公雉。}  ¶ \textcolor{darkblue}{\textbf{\ipa{ho˧pʰv̩˧ tʰv̩˧-mi˧˥ / ho˧pʰv̩˧ tʰv̩˧-mi˥\#}}} \textcolor{PineGreen}{\selectlanguage{french}\mytextsc{n}+\mytextsc{dem}+\mytextsc{clf}} \zh{这只公雉}  
 \zh{量词}: \textcolor{darkblue}{\textbf{\ipa{mi˩}}}  \mytextsc{clf}~: \textcolor{darkblue}{\textbf{\ipa{mi˩}}} 
\lhead{\firstmark}
\rhead{\botmark}

\subsection{\hspace{-0.5cm} {\Large \textcolor{darkblue}{\textbf{\ipa{ho˧ʈʂɯ˧}}}}\hspace{0.5cm}[\kern2pt{\textcolor{darkblue}{\textbf{\ipa{ho˩ʈʂɯ˧˥}}}}\kern2pt]} \hypertarget{ho\string_Mt`s`M\string_M1}{}
\markboth{\textcolor{darkblue}{\textbf{\ipa{ho˧ʈʂɯ˧}}}}{}
\textcolor{teal}{\mytextsc{nom}} \hspace{4pt} Ton~: M.
\textcolor{PineGreen}{\selectlanguage{french}Armoise, \textit{Artemisia vulgaris}.} \zh{蒿(汉语借词:蒿枝)。} Dialecte chinois local~:\zh{蒿草、蒿枝。} Emprunt~: chinois  \zh{蒿枝}
\textit{Voir~:} \hyperlink{}{\textcolor{darkblue}{\textbf{\ipa{tɕɤ˧qʰɑ\#˥}}}} 
\lhead{\firstmark}
\rhead{\botmark}

\subsection{\hspace{-0.5cm} {\Large \textcolor{darkblue}{\textbf{\ipa{ho˧zo\#˥}}}}\hspace{0.5cm}[\kern2pt{\textcolor{darkblue}{\textbf{\ipa{ho˧zo˧}}}}\kern2pt]} \hypertarget{ho\string_Mzo\#\string_T1}{}
\markboth{\textcolor{darkblue}{\textbf{\ipa{ho˧zo\#˥}}}}{}
\textcolor{teal}{\mytextsc{nom}} \hspace{4pt} Ton~: \#H.
\textcolor{PineGreen}{\selectlanguage{french}Bébé faisan.} \zh{小雉。}  ¶ \textcolor{darkblue}{\textbf{\ipa{ho˧zo˧ tʰv̩˧-ɭɯ\#˥}}} \textcolor{PineGreen}{\selectlanguage{french}\mytextsc{n}+\mytextsc{dem}+\mytextsc{clf}} \zh{这只小雉}  
 \zh{量词}: \textcolor{darkblue}{\textbf{\ipa{ɭɯ˧}}}  \mytextsc{clf}~: \textcolor{darkblue}{\textbf{\ipa{ɭɯ˧}}} 
\lhead{\firstmark}
\rhead{\botmark}

\subsection{\hspace{-0.5cm} {\Large \textcolor{darkblue}{\textbf{\ipa{‑ho˩}}}}\hspace{0.5cm}[\kern2pt{\textcolor{darkblue}{\textbf{\ipa{ho˩˥}}}}\kern2pt]} \hypertarget{‑ho\string_B1}{}
\markboth{\textcolor{darkblue}{\textbf{\ipa{‑ho˩}}}}{}
\textcolor{teal}{\mytextsc{suffixe}} \hspace{4pt} Ton~: L.
\textcolor{PineGreen}{\selectlanguage{french}Future / desiderative / conjecture.} \zh{\mytextsc{未来}\string_愿望。}  ¶ \textcolor{darkblue}{\textbf{\ipa{hi˩gi˩ ə˥-ho˩? - hi˩ gi˩ ho˥!}}} \textcolor{PineGreen}{\selectlanguage{french}Va-t-il pleuvoir ? - Oui!} \zh{要下雨了吗? - 是的,要下雨了!}  
 ¶ \textcolor{darkblue}{\textbf{\ipa{ʈʂʰɯ˧ | so˧ɲi˥ | le˧-jo˩ ho˩-hĩ˩ | ə˩-ɲi˩˥ ? - ʈʂʰɯ˧ | so˧ɲi˥ | le˧-jo˩-ho˩-ɲi˩-mæ˩.}}} \textcolor{PineGreen}{\selectlanguage{french}Viendra-t-il demain ? - (Oui) je pense qu’il viendra demain. (Qd on est presque sûr)} \zh{他明天要来买? - (是的,)他明天会来的。(回答表示:比较肯定。)}  
 ¶ \textcolor{darkblue}{\textbf{\ipa{so˧ɲi˥ | le˧-ɬi˥ | mɤ˧-ho˥!}}} \textcolor{PineGreen}{\selectlanguage{french}Demain, ils ne seront plus en vacances! (contexte: on parle d'une crèche qui a été en vacances la semaine précédente à l'occasion de la Fête des enseignants)} \zh{明天就不休息了!}  
 ¶ \textcolor{darkblue}{\textbf{\ipa{tɕʰi˧ ə˧-ho˩?}}} \textcolor{PineGreen}{\selectlanguage{french}va(-t-il) vendre?} \zh{要卖吗?/ 会卖吗?}  
 ¶ \textcolor{darkblue}{\textbf{\ipa{hwæ˧ ə˧-ho˥?}}} \textcolor{PineGreen}{\selectlanguage{french}va(-t-il) acheter?} \zh{要买吗?/ 会买马?}  

\lhead{\firstmark}
\rhead{\botmark}

\subsection{\hspace{-0.5cm} {\Large \textcolor{darkblue}{\textbf{\ipa{ho˩\textsubscript{a}}}}}\hspace{0.5cm}[\kern2pt{\textcolor{darkblue}{\textbf{\ipa{ho˥}}}}\kern2pt]} \hypertarget{ho\string_Ba1}{}
\markboth{\textcolor{darkblue}{\textbf{\ipa{ho˩\textsubscript{a}}}}}{}
\textcolor{teal}{\mytextsc{adjectif}} \hspace{4pt} Ton~: L\textsubscript{a}.
\textcolor{PineGreen}{\selectlanguage{french}Exact, correct; adapté, convenable.} \zh{准确,合适。}  Emprunt~: chinois  ancien: \zh{合} ?
 ¶ \textcolor{darkblue}{\textbf{\ipa{mɤ˧-ho˩}}} \textcolor{PineGreen}{\selectlanguage{french}\mytextsc{neg}: faux, erroné, inapproprié} \zh{不合适,不准,不对}  
 ¶ \textcolor{darkblue}{\textbf{\ipa{ho˩-ze˧!}}} \textcolor{PineGreen}{\selectlanguage{french}\mytextsc{pfv}} \zh{对了! / 准确!}  
 ¶ \textcolor{darkblue}{\textbf{\ipa{ho˩-hĩ˩˥}}} \textcolor{PineGreen}{\selectlanguage{french}\mytextsc{rel}/\mytextsc{nmlz}} \zh{准确的}  

\lhead{\firstmark}
\rhead{\botmark}

\subsection{\hspace{-0.5cm} {\Large \textcolor{darkblue}{\textbf{\ipa{ho˩ɕjæ˧}}}}\hspace{0.5cm}[\kern2pt{\textcolor{darkblue}{\textbf{\ipa{ho˩ɕjæ˩˥}}}}\kern2pt]} \hypertarget{ho\string_Bs£j\{\string_M1}{}
\markboth{\textcolor{darkblue}{\textbf{\ipa{ho˩ɕjæ˧}}}}{}
\textcolor{teal}{\mytextsc{nom}} \hspace{4pt} Ton~: LM.
\textcolor{PineGreen}{\selectlanguage{french}Hysope, \textit{Elshotzia sp.}.} \zh{藿香。}  Emprunt~: chinois  \zh{藿香}

\lhead{\firstmark}
\rhead{\botmark}

\subsection{\hspace{-0.5cm} {\Large \textcolor{darkblue}{\textbf{\ipa{ho˩dʑɯ˩}}}}\hspace{0.5cm}[\kern2pt{\textcolor{darkblue}{\textbf{\ipa{ho˧dʑɯ˧}}}}\kern2pt]} \hypertarget{ho\string_Bdz£M\string_B1}{}
\markboth{\textcolor{darkblue}{\textbf{\ipa{ho˩dʑɯ˩}}}}{}
\textcolor{teal}{\mytextsc{adjectif}} \hspace{4pt} Ton~: L.
\textcolor{PineGreen}{\selectlanguage{french}Indigent.} \zh{穷苦、凋敝、寒苦、竭蹶、穷乏。}  ¶ \textcolor{darkblue}{\textbf{\ipa{ho˩dʑɯ˩-ze˥}}} \textcolor{PineGreen}{\selectlanguage{french}\mytextsc{pfv}: qui se retrouve à la rue, qui devient démuni} \zh{变穷苦了}  

\lhead{\firstmark}
\rhead{\botmark}

\subsection{\hspace{-0.5cm} {\Large \textcolor{darkblue}{\textbf{\ipa{ho˩lo˧pv̩˥}}}}\hspace{0.5cm}[\kern2pt{\textcolor{darkblue}{\textbf{\ipa{ho˧lo˧pv̩˧}}}}\kern2pt]} \hypertarget{ho\string_Blo\string_Mpv\string_=\string_T1}{}
\markboth{\textcolor{darkblue}{\textbf{\ipa{ho˩lo˧pv̩˥}}}}{}
\textcolor{teal}{\mytextsc{nom}} \hspace{4pt} Ton~: LM+H\#.
\textcolor{PineGreen}{\selectlanguage{french}Carotte.} \zh{胡萝卜。}  Emprunt~: chinois  \zh{胡萝卜}
 \zh{量词}: \textcolor{darkblue}{\textbf{\ipa{ɭɯ˧}}}  \mytextsc{clf}~: \textcolor{darkblue}{\textbf{\ipa{ɭɯ˧}}} 
\lhead{\firstmark}
\rhead{\botmark}

\subsection{\hspace{-0.5cm} {\Large \textcolor{darkblue}{\textbf{\ipa{ho˧˥}}}}\hspace{0.5cm}[\kern2pt{\textcolor{darkblue}{\textbf{\ipa{ho˧˥}}}}\kern2pt]} \hypertarget{ho\string_M\string_T1}{}
\markboth{\textcolor{darkblue}{\textbf{\ipa{ho˧˥}}}}{}
\textcolor{teal}{\mytextsc{verbe}} \hspace{4pt} Ton~: MH.
\textcolor{PineGreen}{\selectlanguage{french}Siroter, boire à petites gorgées.} \zh{小口地喝。}  ¶ \textcolor{darkblue}{\textbf{\ipa{ʐɯ˧ ho˧˥}}} \textcolor{PineGreen}{\selectlanguage{french}siroter du vin} \zh{小口地喝酒}  
 ¶ \textcolor{darkblue}{\textbf{\ipa{ʐɯ˧ ho˧\textasciitilde{}ho˥}}} \textcolor{PineGreen}{\selectlanguage{french}siroter du vin} \zh{小口地喝酒}  
 ¶ \textcolor{darkblue}{\textbf{\ipa{ʐɯ˧ | ɖɯ˧-ho˧\textasciitilde{}ho˥}}} \textcolor{PineGreen}{\selectlanguage{french}siroter du vin} \zh{喝一小口酒}  

\lhead{\firstmark}
\rhead{\botmark}

\subsection{\hspace{-0.5cm} {\Large \textcolor{darkblue}{\textbf{\ipa{hõ˧}}}}\hspace{0.5cm}[\kern2pt{\textcolor{darkblue}{\textbf{\ipa{hõ˥}}}}\kern2pt]} \hypertarget{ho\string_~\string_M1}{}
\markboth{\textcolor{darkblue}{\textbf{\ipa{hõ˧}}}}{}
\textcolor{teal}{\mytextsc{verbe}} \hspace{4pt} Ton~: M intrans.
\ding{202} \textcolor{PineGreen}{\selectlanguage{french}Partir (impératif).} \zh{走(离开),\mytextsc{命令式。}}  ¶ \textcolor{darkblue}{\textbf{\ipa{no˧ hõ˧!}}} \textcolor{PineGreen}{\selectlanguage{french}vas-y!} \zh{你走吧!}  
 ¶ \textcolor{darkblue}{\textbf{\ipa{no˧ | le˧-hõ˧!}}} \textcolor{PineGreen}{\selectlanguage{french}Marche!/Vas-y!} \zh{你走吧!}  
 ¶ \textcolor{darkblue}{\textbf{\ipa{ə˧-ze˧\textasciitilde{}ze˥ hõ˩! / ə˧-dzɤ˥ | le˧-hõ˧!}}} \textcolor{PineGreen}{\selectlanguage{french}salutation respectueuse à quelqu'un qui se met en chemin: Prends ton temps en chemin!} \zh{慢走!}  
 ¶ \textcolor{darkblue}{\textbf{\ipa{ɑ˩pʰo˩ hõ˩˥!}}} \textcolor{PineGreen}{\selectlanguage{french}Dehors! / Dégage!} \zh{出去!走开!滚出去!}  
\ding{203} \textcolor{PineGreen}{\selectlanguage{french}Impératif.} \zh{\mytextsc{命令式。}}  ¶ \textcolor{darkblue}{\textbf{\ipa{no˧ | dzɯ˧-hõ˧!}}} \textcolor{PineGreen}{\selectlanguage{french}Mange!} \zh{你吃吧!}  

\lhead{\firstmark}
\rhead{\botmark}

\subsection{\hspace{-0.5cm} {\Large \textcolor{darkblue}{\textbf{\ipa{hõ˧-ɬi˧mi\#˥}}}}\hspace{0.5cm}[\kern2pt{\textcolor{darkblue}{\textbf{\ipa{xxxx non-correspondance entre le nombre de morphèmes et le nombre de tons de morphèmes}}}}\kern2pt]} \hypertarget{ho\string_~\string_M-Ki\string_Mmi\#\string_T1}{}
\markboth{\textcolor{darkblue}{\textbf{\ipa{hõ˧-ɬi˧mi\#˥}}}}{}
\textcolor{teal}{\mytextsc{nom}} \hspace{4pt} Ton~: \#H.
\textcolor{PineGreen}{\selectlanguage{french}8e mois.} \zh{八月。} 
\lhead{\firstmark}
\rhead{\botmark}

\subsection{\hspace{-0.5cm} {\Large \textcolor{darkblue}{\textbf{\ipa{hõ˩tsʰi˧˥}}}}\hspace{0.5cm}[\kern2pt{\textcolor{darkblue}{\textbf{\ipa{hõ˧tsʰi˧}}}}\kern2pt]} \hypertarget{ho\string_~\string_Bts\string_hi\string_M\string_T1}{}
\markboth{\textcolor{darkblue}{\textbf{\ipa{hõ˩tsʰi˧˥}}}}{}
\textcolor{teal}{\mytextsc{nombre}} \hspace{4pt} Ton~: LM+MH\#.
\textcolor{PineGreen}{\selectlanguage{french}80.} \zh{80。} 
\lhead{\firstmark}
\rhead{\botmark}

\subsection{\hspace{-0.5cm} {\Large \textcolor{darkblue}{\textbf{\ipa{hõ˧˥}}}}\hspace{0.5cm}[\kern2pt{\textcolor{darkblue}{\textbf{\ipa{hõ˥}}}}\kern2pt]} \hypertarget{ho\string_~\string_M\string_T1}{}
\markboth{\textcolor{darkblue}{\textbf{\ipa{hõ˧˥}}}}{}
\textcolor{teal}{\mytextsc{nombre}} \hspace{4pt} Ton~: MH.
\textcolor{PineGreen}{\selectlanguage{french}8.} \zh{8。} 
\lhead{\firstmark}
\rhead{\botmark}

\subsection{\hspace{-0.5cm} {\Large \textcolor{darkblue}{\textbf{\ipa{hu˥}}}}\hspace{0.5cm}[\kern2pt{\textcolor{darkblue}{\textbf{\ipa{hu˧˥}}}}\kern2pt]} \hypertarget{hu\string_T1}{}
\markboth{\textcolor{darkblue}{\textbf{\ipa{hu˥}}}}{}
\textcolor{teal}{\mytextsc{verbe}} \hspace{4pt} Ton~: H.
\textcolor{PineGreen}{\selectlanguage{french}Attendre.} \zh{等候。}  ¶ \textcolor{darkblue}{\textbf{\ipa{le˧-hu˥-ze˩}}} \textcolor{PineGreen}{\selectlanguage{french}\mytextsc{accomp} \string_ \mytextsc{pfv}} \zh{等了}  
 ¶ \textcolor{darkblue}{\textbf{\ipa{ɖɯ˧-hu˥ / ɖɯ˧-hu˧-ɻ̍˥}}} \textcolor{PineGreen}{\selectlanguage{french}attendre un peu / Attends un peu!} \zh{等一下 / 请等一下!}  
 ¶ \textcolor{darkblue}{\textbf{\ipa{hĩ˧ hu˧}}} \textcolor{PineGreen}{\selectlanguage{french}attendre quelqu'un} \zh{等人}  

\lhead{\firstmark}
\rhead{\botmark}

\subsection{\hspace{-0.5cm} {\Large \textcolor{darkblue}{\textbf{\ipa{hu˧mi˥\$}}}}\hspace{0.5cm}[\kern2pt{\textcolor{darkblue}{\textbf{\ipa{hu˧mi˧}}}}\kern2pt]} \hypertarget{hu\string_Mmi\string_T\$1}{}
\markboth{\textcolor{darkblue}{\textbf{\ipa{hu˧mi˥\$}}}}{}
\textcolor{teal}{\mytextsc{nom}} \hspace{4pt} Ton~: H\$.
\textcolor{PineGreen}{\selectlanguage{french}Estomac.} \zh{胃。}  \zh{量词}: \textcolor{darkblue}{\textbf{\ipa{ɭɯ˧}}}  \mytextsc{clf}~: \textcolor{darkblue}{\textbf{\ipa{ɭɯ˧}}} 
\lhead{\firstmark}
\rhead{\botmark}

\subsection{\hspace{-0.5cm} {\Large \textcolor{darkblue}{\textbf{\ipa{hu˧˥}}} \textsubscript{1}}\hspace{0.5cm}[\kern2pt{\textcolor{darkblue}{\textbf{\ipa{hu˥}}}}\kern2pt]} \hypertarget{hu\string_M\string_T1}{}
\markboth{\textcolor{darkblue}{\textbf{\ipa{hu˧˥}}} \textsubscript{1}}{}
\textcolor{teal}{\mytextsc{verbe}} \hspace{4pt} Ton~: MH.
\textcolor{PineGreen}{\selectlanguage{french}Avoir la nostalgie de.} \zh{想念。}  ¶ \textcolor{darkblue}{\textbf{\ipa{ə˧mi˧ hu˧˥}}} \textcolor{PineGreen}{\selectlanguage{french}avoir la nostalgie de sa mère} \zh{想念母亲}  

\lhead{\firstmark}
\rhead{\botmark}

\subsection{\hspace{-0.5cm} {\Large \textcolor{darkblue}{\textbf{\ipa{hu˧˥}}} \textsubscript{2}}\hspace{0.5cm}[\kern2pt{\textcolor{darkblue}{\textbf{\ipa{hu˧˥}}}}\kern2pt]} \hypertarget{hu\string_M\string_T2}{}
\markboth{\textcolor{darkblue}{\textbf{\ipa{hu˧˥}}} \textsubscript{2}}{}
\textcolor{teal}{\mytextsc{nom}} \hspace{4pt} Ton~: MH.
\textcolor{PineGreen}{\selectlanguage{french}Estomac au sens large: entrailles, tripes (tout le système digestif).} \zh{内脏:胃、肠子等。}  \zh{量词}: \textcolor{darkblue}{\textbf{\ipa{ɭɯ˧}}}  \mytextsc{clf}~: \textcolor{darkblue}{\textbf{\ipa{ɭɯ˧}}} 
\lhead{\firstmark}
\rhead{\botmark}

\subsection{\hspace{-0.5cm} {\Large \textcolor{darkblue}{\textbf{\ipa{hɯ˧}}}}\hspace{0.5cm}[\kern2pt{\textcolor{darkblue}{\textbf{\ipa{hɯ˥}}}}\kern2pt]} \hypertarget{hM\string_M1}{}
\markboth{\textcolor{darkblue}{\textbf{\ipa{hɯ˧}}}}{}
\textcolor{teal}{\mytextsc{verbe}} \hspace{4pt} Ton~: M\textsubscript{c}.
\textcolor{PineGreen}{\selectlanguage{french}Aller, forme passée.} \zh{走(过去式)。}  ¶ \textcolor{darkblue}{\textbf{\ipa{(ki˧zo˧) | lo˧ ʝi˧-hɯ˧(-ze˩)!}}} \textcolor{PineGreen}{\selectlanguage{french}Kizo est partie travailler!} \zh{给若(人名)干活去了!}  
 ¶ \textcolor{darkblue}{\textbf{\ipa{le˧-hɯ˩-hĩ˩ hĩ˩}}} \textcolor{PineGreen}{\selectlanguage{french}personne décédée; littéralement "personne qui est partie} \zh{委婉语:‘走了的人’=去世了的人}  

\lhead{\firstmark}
\rhead{\botmark}

\subsection{\hspace{-0.5cm} {\Large \textcolor{darkblue}{\textbf{\ipa{hɯ˧\textsubscript{b}}}}}\hspace{0.5cm}[\kern2pt{\textcolor{darkblue}{\textbf{\ipa{hɯ˥}}}}\kern2pt]} \hypertarget{hM\string_Mb1}{}
\markboth{\textcolor{darkblue}{\textbf{\ipa{hɯ˧\textsubscript{b}}}}}{}
\textcolor{teal}{\mytextsc{classificateur}} \hspace{4pt} Ton~: M\textsubscript{b}.
\textcolor{PineGreen}{\selectlanguage{french}Quelques-uns, un peu, une petite quantité de.} \zh{量词:一点。}  ¶ \textcolor{darkblue}{\textbf{\ipa{ʈʂʰæ˧ɣɯ˧ ɖɯ˧-hɯ˧}}} \textcolor{PineGreen}{\selectlanguage{french}quelques médicaments} \zh{一些药}  

\lhead{\firstmark}
\rhead{\botmark}

\subsection{\hspace{-0.5cm} {\Large \textcolor{darkblue}{\textbf{\ipa{hṽ˥}}}}\hspace{0.5cm}[\kern2pt{\textcolor{darkblue}{\textbf{\ipa{hṽ˥}}}}\kern2pt]} \hypertarget{hv\string_~\string_T1}{}
\markboth{\textcolor{darkblue}{\textbf{\ipa{hṽ˥}}}}{}
\textcolor{teal}{\mytextsc{verbe}} \hspace{4pt} Ton~: H.
\textcolor{PineGreen}{\selectlanguage{french}Frire (viande, légumes...), cuire au wok.} \zh{炒(肉、菜)。}  ¶ \textcolor{darkblue}{\textbf{\ipa{hṽ˧\textasciitilde{}hṽ˧}}} \textcolor{PineGreen}{\selectlanguage{french}\mytextsc{red}} \zh{重叠}  
 ¶ \textcolor{darkblue}{\textbf{\ipa{le˧-hṽ˧\textasciitilde{}hṽ˧}}} \textcolor{PineGreen}{\selectlanguage{french}\mytextsc{accomp} \mytextsc{red}} \zh{\mytextsc{accomp} \mytextsc{red}}  
 ¶ \textcolor{darkblue}{\textbf{\ipa{hṽ˧\textasciitilde{}hṽ˧-ze˩}}} \textcolor{PineGreen}{\selectlanguage{french}\mytextsc{red} \mytextsc{pfv}} \zh{炒了}  
 ¶ \textcolor{darkblue}{\textbf{\ipa{ʂe˧ hṽ˧\textasciitilde{}hṽ˧}}} \textcolor{PineGreen}{\selectlanguage{french}frire de la viande} \zh{炒肉}  
 ¶ \textcolor{darkblue}{\textbf{\ipa{v̩˩tsʰɤ˧ hṽ˧\textasciitilde{}hṽ˧}}} \textcolor{PineGreen}{\selectlanguage{french}frire des légumes} \zh{炒蔬菜}  
 ¶ \textcolor{darkblue}{\textbf{\ipa{læ˧tsɯ˥ hṽ˩\textasciitilde{}hṽ˩}}} \textcolor{PineGreen}{\selectlanguage{french}frire des piments} \zh{炒辣椒}  
 ¶ \textcolor{darkblue}{\textbf{\ipa{hɑ˧ hṽ˧\textasciitilde{}hṽ˧}}} \textcolor{PineGreen}{\selectlanguage{french}frire un plat/faire la cuisine/frire du riz/de la nourriture} \zh{炒饭}  

\lhead{\firstmark}
\rhead{\botmark}

\subsection{\hspace{-0.5cm} {\Large \textcolor{darkblue}{\textbf{\ipa{hṽ˥}}}}\hspace{0.5cm}[\kern2pt{\textcolor{darkblue}{\textbf{\ipa{hṽ˥}}}}\kern2pt]} \hypertarget{hv\string_~\string_T1}{}
\markboth{\textcolor{darkblue}{\textbf{\ipa{hṽ˥}}}}{}
\textcolor{teal}{\mytextsc{nom}} \hspace{4pt} Ton~: \#H.
\textcolor{PineGreen}{\selectlanguage{french}Poils (pour les animaux, y compris les épines du hérisson; aussi pour les hommes).} \zh{毛、豪猪的翎。}  \zh{量词}: \textcolor{darkblue}{\textbf{\ipa{kʰɯ˩}}}  \mytextsc{clf}~: \textcolor{darkblue}{\textbf{\ipa{kʰɯ˩}}} 
\lhead{\firstmark}
\rhead{\botmark}

\subsection{\hspace{-0.5cm} {\Large \textcolor{darkblue}{\textbf{\ipa{hṽ˧dɤ˧ɻ\#˥}}}}\hspace{0.5cm}[\kern2pt{\textcolor{darkblue}{\textbf{\ipa{hṽ˩dɤ˩ɻ˩˥}}}}\kern2pt]} \hypertarget{hv\string_~\string_Md7\string_Mr£`\#\string_T1}{}
\markboth{\textcolor{darkblue}{\textbf{\ipa{hṽ˧dɤ˧ɻ\#˥}}}}{}
\textcolor{teal}{\mytextsc{adjectif}} \hspace{4pt} Ton~: \#H.
\textcolor{PineGreen}{\selectlanguage{french}Maladroit, incapable.} \zh{笨拙,经常损坏东西。}  ¶ \textcolor{darkblue}{\textbf{\ipa{hṽ˩-hĩ˩˥}}} \textcolor{PineGreen}{\selectlanguage{french}\mytextsc{rel}} \zh{笨拙的}  
 ¶ \textcolor{darkblue}{\textbf{\ipa{hṽ˧dɤ˧ɻ̍˧\textasciitilde{}hṽ˧dɤ˧ɻ̍˧-zo˥}}} \textcolor{PineGreen}{\selectlanguage{french}un maladroit} \zh{笨手笨脚的(男)人}  
 ¶ \textcolor{darkblue}{\textbf{\ipa{[F5] hṽ˩ɖɻ̍˩\textasciitilde{}hṽ˧ɖɻ̍˧-gv̩˧}}} \textcolor{PineGreen}{\selectlanguage{french}\mytextsc{red}} \zh{重叠:笨笨的}  

\lhead{\firstmark}
\rhead{\botmark}

\subsection{\hspace{-0.5cm} {\Large \textcolor{darkblue}{\textbf{\ipa{hṽ˧\textasciitilde{}hṽ˩-ɖʐæ˩\textasciitilde{}ɖʐæ˩}}}}\hspace{0.5cm}[\kern2pt{\textcolor{darkblue}{\textbf{\ipa{xxxx non-correspondance entre le nombre de morphèmes et le nombre de tons de morphèmes}}}}\kern2pt]} \hypertarget{hv\string_~\string_M~hv\string_~\string_B-d`z`\{\string_B~d`z`\{\string_B1}{}
\markboth{\textcolor{darkblue}{\textbf{\ipa{hṽ˧\textasciitilde{}hṽ˩-ɖʐæ˩\textasciitilde{}ɖʐæ˩}}}}{}
\textcolor{teal}{\mytextsc{adjectif}} \hspace{4pt} Ton~: L\#-.
\textit{De:} \textbf{hṽ˩a 1} \textcolor{PineGreen}{\selectlanguage{french}Tout rouge.} \zh{红红的。} 
\lhead{\firstmark}
\rhead{\botmark}

\subsection{\hspace{-0.5cm} {\Large \textcolor{darkblue}{\textbf{\ipa{hṽ˧nɑ˩}}}}\hspace{0.5cm}[\kern2pt{\textcolor{darkblue}{\textbf{\ipa{hṽ˩nɑ˥}}}}\kern2pt]} \hypertarget{hv\string_~\string_MnA\string_B1}{}
\markboth{\textcolor{darkblue}{\textbf{\ipa{hṽ˧nɑ˩}}}}{}
\textcolor{teal}{\mytextsc{nom}} \hspace{4pt} Ton~: L\#.
\textcolor{PineGreen}{\selectlanguage{french}Bête sauvage.} \zh{野兽。}  \zh{量词}: \textcolor{darkblue}{\textbf{\ipa{mi˩}}}  \mytextsc{clf}~: \textcolor{darkblue}{\textbf{\ipa{mi˩}}} 
\lhead{\firstmark}
\rhead{\botmark}

\subsection{\hspace{-0.5cm} {\Large \textcolor{darkblue}{\textbf{\ipa{hṽ˩\textsubscript{a}}}} \textsubscript{1}}\hspace{0.5cm}[\kern2pt{\textcolor{darkblue}{\textbf{\ipa{hṽ˥}}}}\kern2pt]} \hypertarget{hv\string_~\string_Ba1}{}
\markboth{\textcolor{darkblue}{\textbf{\ipa{hṽ˩\textsubscript{a}}}} \textsubscript{1}}{}
\textcolor{teal}{\mytextsc{adjectif}} \hspace{4pt} Ton~: L\textsubscript{a}.
\textcolor{PineGreen}{\selectlanguage{french}Rouge (ex.: vêtement rouge, sang rouge).} \zh{红色的。} 
\lhead{\firstmark}
\rhead{\botmark}

\subsection{\hspace{-0.5cm} {\Large \textcolor{darkblue}{\textbf{\ipa{hṽ˩\textsubscript{a}}}} \textsubscript{2}}\hspace{0.5cm}[\kern2pt{\textcolor{darkblue}{\textbf{\ipa{hṽ˩˥}}}}\kern2pt]} \hypertarget{hv\string_~\string_Ba2}{}
\markboth{\textcolor{darkblue}{\textbf{\ipa{hṽ˩\textsubscript{a}}}} \textsubscript{2}}{}
\textcolor{teal}{\mytextsc{adjectif}} \hspace{4pt} Ton~: L\textsubscript{a}.
\textcolor{PineGreen}{\selectlanguage{french}De petite taille, de petite stature; bas.} \zh{矮、低。} 
\lhead{\firstmark}
\rhead{\botmark}

\subsection{\hspace{-0.5cm} {\Large \textcolor{darkblue}{\textbf{\ipa{hṽ˩-ɖʐæ˩ɻæ˥}}}}\hspace{0.5cm}[\kern2pt{\textcolor{darkblue}{\textbf{\ipa{xxxx non-correspondance entre le nombre de morphèmes et le nombre de tons de morphèmes}}}}\kern2pt]} \hypertarget{hv\string_~\string_B-d`z`\{\string_Br£`\{\string_T1}{}
\markboth{\textcolor{darkblue}{\textbf{\ipa{hṽ˩-ɖʐæ˩ɻæ˥}}}}{}
\textcolor{teal}{\mytextsc{adjectif}} \hspace{4pt} Ton~: L+H\#.
\textit{De:} \textbf{hṽ˩a 1} \textcolor{PineGreen}{\selectlanguage{french}Tout rouge.} \zh{红红的。}  ¶ \textcolor{darkblue}{\textbf{\ipa{hṽ˩ɖʐæ˩ɻæ˥-gv̩˩}}} \textcolor{PineGreen}{\selectlanguage{french}tout rouge} \zh{红红的}  
 ¶ \textcolor{darkblue}{\textbf{\ipa{[F5] hṽ˩ɖʐæ˩˥ | hṽ˩ɖʐæ˩˥ gv̩˩}}} \textcolor{PineGreen}{\selectlanguage{french}\mytextsc{red;} les deux premières syllabes ont une fréquence fondamentale nettement plus haute que les deux suivantes} \zh{重叠}  

\lhead{\firstmark}
\rhead{\botmark}

\subsection{\hspace{-0.5cm} {\Large \textcolor{darkblue}{\textbf{\ipa{hṽ˩\textasciitilde{}hṽ˩}}}}\hspace{0.5cm}[\kern2pt{\textcolor{darkblue}{\textbf{\ipa{hṽ˧hṽ˩}}}}\kern2pt]} \hypertarget{hv\string_~\string_B~hv\string_~\string_B1}{}
\markboth{\textcolor{darkblue}{\textbf{\ipa{hṽ˩\textasciitilde{}hṽ˩}}}}{}
\textcolor{teal}{\mytextsc{verbe}} \hspace{4pt} Ton~: L.
\textcolor{PineGreen}{\selectlanguage{french}Ennuyer, empêcher, faire obstruction à.} \zh{干扰、防碍。}  ¶ \textcolor{darkblue}{\textbf{\ipa{hĩ˧ hṽ˥\textasciitilde{}hṽ˩}}} \textcolor{PineGreen}{\selectlanguage{french}ennuyer les gens} \zh{干扰人家}  

\lhead{\firstmark}
\rhead{\botmark}

\subsection{\hspace{-0.5cm} {\Large \textcolor{darkblue}{\textbf{\ipa{hwɑ˩kwɤ˧}}}}\hspace{0.5cm}[\kern2pt{\textcolor{darkblue}{\textbf{\ipa{hwɑ˧kwɤ˥}}}}\kern2pt]} \hypertarget{hwA\string_Bkw7\string_M1}{}
\markboth{\textcolor{darkblue}{\textbf{\ipa{hwɑ˩kwɤ˧}}}}{}
\textcolor{teal}{\mytextsc{nom}} \hspace{4pt} Ton~: LM.
\textcolor{PineGreen}{\selectlanguage{french}Concombre.} \zh{黄瓜(汉语借词)。}  Emprunt~: chinois  \zh{黄瓜}
 \zh{量词}: \textcolor{darkblue}{\textbf{\ipa{ɭɯ˧}}}  \mytextsc{clf}~: \textcolor{darkblue}{\textbf{\ipa{ɭɯ˧}}} 
\lhead{\firstmark}
\rhead{\botmark}

\subsection{\hspace{-0.5cm} {\Large \textcolor{darkblue}{\textbf{\ipa{hwæ˧\textsubscript{a}}}}}\hspace{0.5cm}[\kern2pt{\textcolor{darkblue}{\textbf{\ipa{hwæ˩˥}}}}\kern2pt]} \hypertarget{hw\{\string_Ma1}{}
\markboth{\textcolor{darkblue}{\textbf{\ipa{hwæ˧\textsubscript{a}}}}}{}
\textcolor{teal}{\mytextsc{verbe}} \hspace{4pt} Ton~: M\textsubscript{a}.
\textcolor{PineGreen}{\selectlanguage{french}Acheter.} \zh{买。}  ¶ \textcolor{darkblue}{\textbf{\ipa{le˧-hwæ˧}}} \textcolor{PineGreen}{\selectlanguage{french}\mytextsc{accomp}} \zh{\mytextsc{accomp}}  
 ¶ \textcolor{darkblue}{\textbf{\ipa{tso˧\textasciitilde{}tso˧ hwæ˩}}} \textcolor{PineGreen}{\selectlanguage{french}acheter des choses} \zh{买东西}  
 ¶ \textcolor{darkblue}{\textbf{\ipa{ɖɯ˧-kʰwɤ˥ hwæ˩}}} \textcolor{PineGreen}{\selectlanguage{french}acheter un morceau} \zh{买一块}  
 ¶ \textcolor{darkblue}{\textbf{\ipa{hwæ˧\textasciitilde{}hwæ˩}}} \textcolor{PineGreen}{\selectlanguage{french}\mytextsc{red}} \zh{\mytextsc{重叠}}  

\lhead{\firstmark}
\rhead{\botmark}

\subsection{\hspace{-0.5cm} {\Large \textcolor{darkblue}{\textbf{\ipa{hwæ˧ɖʐæ˥}}} \textsubscript{1}}\hspace{0.5cm}[\kern2pt{\textcolor{darkblue}{\textbf{\ipa{hwæ˧ɖʐæ˧}}}}\kern2pt]} \hypertarget{hw\{\string_Md`z`\{\string_T1}{}
\markboth{\textcolor{darkblue}{\textbf{\ipa{hwæ˧ɖʐæ˥}}} \textsubscript{1}}{}
\textcolor{teal}{\mytextsc{nom}} \hspace{4pt} Ton~: H\#.
\textcolor{PineGreen}{\selectlanguage{french}Écureuil.} \zh{松鼠,灰鼠。}  ¶ \textcolor{darkblue}{\textbf{\ipa{hwæ˧ɖʐæ˥-pʰv̩˩}}} \textcolor{PineGreen}{\selectlanguage{french}écureuil mâle} \zh{公松鼠}  
 ¶ \textcolor{darkblue}{\textbf{\ipa{hwæ˧ɖʐæ˥-mi˩}}} \textcolor{PineGreen}{\selectlanguage{french}écureuil femelle} \zh{母松鼠}  
 \zh{量词}: \textcolor{darkblue}{\textbf{\ipa{mi˩}}}  \mytextsc{clf}~: \textcolor{darkblue}{\textbf{\ipa{mi˩}}} 
\lhead{\firstmark}
\rhead{\botmark}

\subsection{\hspace{-0.5cm} {\Large \textcolor{darkblue}{\textbf{\ipa{hwæ˧ɖʐæ˥}}} \textsubscript{2}}\hspace{0.5cm}[\kern2pt{\textcolor{darkblue}{\textbf{\ipa{hwæ˧ɖʐæ˥}}}}\kern2pt]} \hypertarget{hw\{\string_Md`z`\{\string_T2}{}
\markboth{\textcolor{darkblue}{\textbf{\ipa{hwæ˧ɖʐæ˥}}} \textsubscript{2}}{}
\textcolor{teal}{\mytextsc{nom}} \hspace{4pt} Ton~: H\#.
\textcolor{PineGreen}{\selectlanguage{french}Verrue.} \zh{瘊子、肉赘。}  ¶ \textcolor{darkblue}{\textbf{\ipa{hwæ˧ʈʂæ˥ tʰv̩˩}}} \textcolor{PineGreen}{\selectlanguage{french}une verrue se forme; attraper une verrue} \zh{长瘊子}  
 ¶ \textcolor{darkblue}{\textbf{\ipa{hwæ˧ʈʂæ˥ | le˧-tʰv̩˧-ze˧!}}} \textcolor{PineGreen}{\selectlanguage{french}Une verrue s'est formée!} \zh{长瘊子了!}  
 \zh{量词}: \textcolor{darkblue}{\textbf{\ipa{mi˩}}}  \mytextsc{clf}~: \textcolor{darkblue}{\textbf{\ipa{mi˩}}} 
\lhead{\firstmark}
\rhead{\botmark}

\subsection{\hspace{-0.5cm} {\Large \textcolor{darkblue}{\textbf{\ipa{hwæ˧pʰæ˥}}}}\hspace{0.5cm}[\kern2pt{\textcolor{darkblue}{\textbf{\ipa{hwæ˧pʰæ˩}}}}\kern2pt]} \hypertarget{hw\{\string_Mp\string_h\{\string_T1}{}
\markboth{\textcolor{darkblue}{\textbf{\ipa{hwæ˧pʰæ˥}}}}{}
\textcolor{teal}{\mytextsc{nom}} \hspace{4pt} Ton~: H\#.
\textcolor{PineGreen}{\selectlanguage{french}Pièce de tissu.} \zh{一块布。}  \zh{量词}: \textcolor{darkblue}{\textbf{\ipa{pʰæ˧˥}}}  \mytextsc{clf}~: \textcolor{darkblue}{\textbf{\ipa{pʰæ˧˥}}} 
\lhead{\firstmark}
\rhead{\botmark}

\subsection{\hspace{-0.5cm} {\Large \textcolor{darkblue}{\textbf{\ipa{hwæ˧pʰæ˩}}}}\hspace{0.5cm}[\kern2pt{\textcolor{darkblue}{\textbf{\ipa{hwæ˩pʰæ˥}}}}\kern2pt]} \hypertarget{hw\{\string_Mp\string_h\{\string_B1}{}
\markboth{\textcolor{darkblue}{\textbf{\ipa{hwæ˧pʰæ˩}}}}{}
\textcolor{teal}{\mytextsc{nom}} \hspace{4pt} Ton~: L\#.
\textcolor{PineGreen}{\selectlanguage{french}Grosse houe.} \zh{大锄。} Dialecte chinois local~:\zh{挖锄。} ¶ \textcolor{darkblue}{\textbf{\ipa{hwæ˧pʰæ˩ tʰv̩˩-nɑ˩}}} \textcolor{PineGreen}{\selectlanguage{french}\mytextsc{n}+\mytextsc{dem}+\mytextsc{clf}} \zh{这把大锄}  
 \zh{量词}: \textcolor{darkblue}{\textbf{\ipa{nɑ˧}}}  \mytextsc{clf}~: \textcolor{darkblue}{\textbf{\ipa{nɑ˧}}} 
\lhead{\firstmark}
\rhead{\botmark}

\subsection{\hspace{-0.5cm} {\Large \textcolor{darkblue}{\textbf{\ipa{hwæ˧pʰæ˩-gv̩˩-di˩}}}}\hspace{0.5cm}[\kern2pt{\textcolor{darkblue}{\textbf{\ipa{xxxx non-correspondance entre le nombre de morphèmes et le nombre de tons de morphèmes}}}}\kern2pt]} \hypertarget{hw\{\string_Mp\string_h\{\string_B-gv\string_=\string_B-di\string_B1}{}
\markboth{\textcolor{darkblue}{\textbf{\ipa{hwæ˧pʰæ˩-gv̩˩-di˩}}}}{}
\textcolor{teal}{\mytextsc{nom}} \hspace{4pt} Ton~: L\#--.
\textcolor{PineGreen}{\selectlanguage{french}Métier à tisser.} \zh{织布机。}  ¶ \textcolor{darkblue}{\textbf{\ipa{hwæ˧pʰæ˩gv̩˩di˩-tɕi˩tɕʰi˧}}} \textcolor{PineGreen}{\selectlanguage{french}métier à tisser industriel, machine à faire du tissu (formé de 'métier à tisser' + le mot chinois pour 'machine')} \zh{工业织布机。直译:“织布机器”(在摩梭词后面加上汉语的“机器”)}  
 \zh{量词}: \textcolor{darkblue}{\textbf{\ipa{nɑ˧}}}  \mytextsc{clf}~: \textcolor{darkblue}{\textbf{\ipa{nɑ˧}}} 
\lhead{\firstmark}
\rhead{\botmark}

\subsection{\hspace{-0.5cm} {\Large \textcolor{darkblue}{\textbf{\ipa{hwæ˧tsɯ˥}}}}\hspace{0.5cm}[\kern2pt{\textcolor{darkblue}{\textbf{\ipa{xxxx non-correspondance entre le nombre de morphèmes et le nombre de tons de morphèmes}}}}\kern2pt]} \hypertarget{hw\{\string_MtsM\string_T1}{}
\markboth{\textcolor{darkblue}{\textbf{\ipa{hwæ˧tsɯ˥}}}}{}
\textcolor{teal}{\mytextsc{nom}} \hspace{4pt} Ton~: H\#.
\textcolor{PineGreen}{\selectlanguage{french}Rat.} \zh{老鼠(汉语借词)。} Dialecte chinois local~:\zh{耗子。} Emprunt~: chinois  \zh{耗子}
 ¶ \textcolor{darkblue}{\textbf{\ipa{hwæ˧tsɯ˥-pʰv̩˩}}} \textcolor{PineGreen}{\selectlanguage{french}rat mâle} \zh{公老鼠}  
 ¶ \textcolor{darkblue}{\textbf{\ipa{hwæ˧tsɯ˥-mi˩}}} \textcolor{PineGreen}{\selectlanguage{french}rat femelle} \zh{母老鼠}  
 \zh{量词}: \textcolor{darkblue}{\textbf{\ipa{mi˩}}}  \mytextsc{clf}~: \textcolor{darkblue}{\textbf{\ipa{mi˩}}} 
\lhead{\firstmark}
\rhead{\botmark}

\subsection{\hspace{-0.5cm} {\Large \textcolor{darkblue}{\textbf{\ipa{hwæ˧tsɯ˥-njɤ˩di˩}}}}\hspace{0.5cm}[\kern2pt{\textcolor{darkblue}{\textbf{\ipa{xxxx non-correspondance entre le nombre de morphèmes et le nombre de tons de morphèmes}}}}\kern2pt]} \hypertarget{hw\{\string_MtsM\string_T-nj7\string_Bdi\string_B1}{}
\markboth{\textcolor{darkblue}{\textbf{\ipa{hwæ˧tsɯ˥-njɤ˩di˩}}}}{}
\textcolor{teal}{\mytextsc{nom}} \hspace{4pt} Ton~: H\#-.
\textcolor{PineGreen}{\selectlanguage{french}Chardon.} \zh{大蓟。}  \zh{量词}: \textcolor{darkblue}{\textbf{\ipa{dzi˩}}}  \mytextsc{clf}~: \textcolor{darkblue}{\textbf{\ipa{dzi˩}}} 
\lhead{\firstmark}
\rhead{\botmark}

\subsection{\hspace{-0.5cm} {\Large \textcolor{darkblue}{\textbf{\ipa{hwæ˧tsɯ˥-njɤ˩di˩-si˩dzi˩}}}}\hspace{0.5cm}[\kern2pt{\textcolor{darkblue}{\textbf{\ipa{xxxx non-correspondance entre le nombre de morphèmes et le nombre de tons de morphèmes}}}}\kern2pt]} \hypertarget{hw\{\string_MtsM\string_T-nj7\string_Bdi\string_B-si\string_Bdzi\string_B1}{}
\markboth{\textcolor{darkblue}{\textbf{\ipa{hwæ˧tsɯ˥-njɤ˩di˩-si˩dzi˩}}}}{}
\textcolor{teal}{\mytextsc{nom}} \hspace{4pt} Ton~: H\#--.
\textcolor{PineGreen}{\selectlanguage{french}Bardane: \textit{Arctium lappa}, plante dont les graines adhèrent à la laine et la queue des moutons. On en tire un médicament contre le rhume.} \zh{牛蒡。} Dialecte chinois local~:\zh{牛蒡子。}
\lhead{\firstmark}
\rhead{\botmark}

\subsection{\hspace{-0.5cm} {\Large \textcolor{darkblue}{\textbf{\ipa{hwæ˩\textsubscript{a}}}} \textsubscript{1}}\hspace{0.5cm}[\kern2pt{\textcolor{darkblue}{\textbf{\ipa{xxxx non-correspondance entre le nombre de morphèmes et le nombre de tons de morphèmes}}}}\kern2pt]} \hypertarget{hw\{\string_Ba1}{}
\markboth{\textcolor{darkblue}{\textbf{\ipa{hwæ˩\textsubscript{a}}}} \textsubscript{1}}{}
\textcolor{teal}{\mytextsc{verbe}} \hspace{4pt} Ton~: L\textsubscript{a}.
\textcolor{PineGreen}{\selectlanguage{french}Fermer (la porte).} \zh{关(出门,就关门)。}  ¶ \textcolor{darkblue}{\textbf{\ipa{kʰi˧ | tʰi˧-hwæ˩!}}} \textcolor{PineGreen}{\selectlanguage{french}Ferme la porte!} \zh{关门吧!}  
 ¶ \textcolor{darkblue}{\textbf{\ipa{ʂe˧bæ˧ | le˧-wo˧-hwæ˥}}} \textcolor{PineGreen}{\selectlanguage{french}mettre la chaîne à la porte (quand on sort de la maison, on ferme la porte avec une chaîne de fer, et un verrou)}  
 ¶ \textcolor{darkblue}{\textbf{\ipa{kʰi˧-bi˥ di˩-hĩ˩ ʂe˩bæ˩}}}  
 ¶  
\textit{Voir~:} \hyperlink{}{\textcolor{darkblue}{\textbf{\ipa{ʈæ˩\textsubscript{a}}}}} 
\lhead{\firstmark}
\rhead{\botmark}

\subsection{\hspace{-0.5cm} {\Large \textcolor{darkblue}{\textbf{\ipa{hwæ˩\textsubscript{a}}}} \textsubscript{2}}\hspace{0.5cm}[\kern2pt{\textcolor{darkblue}{\textbf{\ipa{hwæ˩˥}}}}\kern2pt]} \hypertarget{hw\{\string_Ba2}{}
\markboth{\textcolor{darkblue}{\textbf{\ipa{hwæ˩\textsubscript{a}}}} \textsubscript{2}}{}
\textcolor{teal}{\mytextsc{verbe}} \hspace{4pt} Ton~: L\textsubscript{a}.
\textcolor{PineGreen}{\selectlanguage{french}Accrocher, suspendre; être accroché, suspendu à; se tenir à.} \zh{悬挂、挂在墙上。}  ¶ \textcolor{darkblue}{\textbf{\ipa{tso˧\textasciitilde{}tso˧ | gɤ˧bi˧ hwæ˥}}} \textcolor{PineGreen}{\selectlanguage{french}accrocher des choses en hauteur} \zh{挂东西在上面}  
 ¶ \textcolor{darkblue}{\textbf{\ipa{tso˧\textasciitilde{}tso˧ hwæ˥}}} \textcolor{PineGreen}{\selectlanguage{french}accrocher des choses} \zh{挂东西}  
 ¶ \textcolor{darkblue}{\textbf{\ipa{ʂe˧ | tʰi˧-hwæ˩}}} \textcolor{PineGreen}{\selectlanguage{french}accrocher de la viande (au-dessus du foyer, pour la fumer)} \zh{挂肉(在火塘上,为了熏肉)}  
 ¶ \textcolor{darkblue}{\textbf{\ipa{tso˧\textasciitilde{}tso˧ | tʰi˧-hwæ˩}}} \textcolor{PineGreen}{\selectlanguage{french}accrocher des choses} \zh{挂东西}  
 ¶ \textcolor{darkblue}{\textbf{\ipa{ʂe˧-hwæ˥-di˩}}} \textcolor{PineGreen}{\selectlanguage{french}Poutrelle de la pièce principale, juste en-dessous des poutres maîtresses, servant à suspendre de la viande qui sèche. Littéralement “objet pour accrocher de la viande”} \zh{主屋里面的小梁(大梁下面),用来挂肉,熏肉。直译:“挂肉的东西”。}  
 ¶ \textcolor{darkblue}{\textbf{\ipa{tso˧\textasciitilde{}tso˧-hwæ˥-di˩}}} \textcolor{PineGreen}{\selectlanguage{french}objet servant à suspendre des choses; cette périphrase peut par exemple désigner la poutrelle servant à accrocher de la viande, dans la pièce principale} \zh{挂(东西)用的(东西),如:钩子、用来挂肉的小梁……}  

\lhead{\firstmark}
\rhead{\botmark}

\subsection{\hspace{-0.5cm} {\Large \textcolor{darkblue}{\textbf{\ipa{hwɤ̃˩\textsubscript{a}}}}}\hspace{0.5cm}[\kern2pt{\textcolor{darkblue}{\textbf{\ipa{hwɤ̃˥}}}}\kern2pt]} \hypertarget{hw7\string_~\string_Ba1}{}
\markboth{\textcolor{darkblue}{\textbf{\ipa{hwɤ̃˩\textsubscript{a}}}}}{}
\textcolor{teal}{\mytextsc{adjectif}} \hspace{4pt} Ton~: L\textsubscript{a}.
\textcolor{PineGreen}{\selectlanguage{french}En retard.} \zh{迟,晚。}  ¶ \textcolor{darkblue}{\textbf{\ipa{hwɤ̃˩-hĩ˩˥}}} \textcolor{PineGreen}{\selectlanguage{french}\mytextsc{rel}/\mytextsc{nmlz}} \zh{迟的}  
 ¶ \textcolor{darkblue}{\textbf{\ipa{ʈʂʰɯ˧ ʑi˧-ʈi˥ hwɤ̃˩!}}} \textcolor{PineGreen}{\selectlanguage{french}il se lève tard} \zh{他起床起得晚!}  

\lhead{\firstmark}
\rhead{\botmark}

\subsection{\hspace{-0.5cm} {\Large \textcolor{darkblue}{\textbf{\ipa{hwɤ˥}}}}\hspace{0.5cm}[\kern2pt{\textcolor{darkblue}{\textbf{\ipa{hwɤ˥}}}}\kern2pt]} \hypertarget{hw7\string_T1}{}
\markboth{\textcolor{darkblue}{\textbf{\ipa{hwɤ˥}}}}{}
\textcolor{teal}{\mytextsc{nom}} \hspace{4pt} Ton~: \#H.
\textcolor{PineGreen}{\selectlanguage{french}Cadeau (souvent financier) qu'on offre à l'occasion des grands événements: décès, mariages.} \zh{在大事发生的时候送的礼物(近期一般给钱):婚礼、葬礼。}  ¶ \textcolor{darkblue}{\textbf{\ipa{hwɤ˧ | ɖɯ˧-kʰwɤ˥}}} \textcolor{PineGreen}{\selectlanguage{french}un cadeau, une donation} \zh{一份大礼物}  
 \zh{量词}: \textcolor{darkblue}{\textbf{\ipa{kʰwɤ˥}}}  \mytextsc{clf}~: \textcolor{darkblue}{\textbf{\ipa{kʰwɤ˥}}} 
\lhead{\firstmark}
\rhead{\botmark}

\subsection{\hspace{-0.5cm} {\Large \textcolor{darkblue}{\textbf{\ipa{hwɤ˥}}}}\hspace{0.5cm}[\kern2pt{\textcolor{darkblue}{\textbf{\ipa{hwɤ˧˥}}}}\kern2pt]} \hypertarget{hw7\string_T1}{}
\markboth{\textcolor{darkblue}{\textbf{\ipa{hwɤ˥}}}}{}
\textcolor{teal}{\mytextsc{verbe}} \hspace{4pt} Ton~: H.
\textcolor{PineGreen}{\selectlanguage{french}Participer à une cérémonie funèbre; littéralement “envoyer quelqu'un”, c'est-à-dire accompagner quelqu'un vers l'au-delà.} \zh{执绋送丧。} 
\lhead{\firstmark}
\rhead{\botmark}

\subsection{\hspace{-0.5cm} {\Large \textcolor{darkblue}{\textbf{\ipa{hwɤ˧}}}}\hspace{0.5cm}[\kern2pt{\textcolor{darkblue}{\textbf{\ipa{hwɤ˩˥}}}}\kern2pt]} \hypertarget{hw7\string_M1}{}
\markboth{\textcolor{darkblue}{\textbf{\ipa{hwɤ˧}}}}{}
\textcolor{teal}{\mytextsc{adjectif}} \hspace{4pt} Ton~: M.
\textcolor{PineGreen}{\selectlanguage{french}Vaste (plaine), étendu, grand (pièce de tissu), de grande taille (objet, légume...).} \zh{宽,辽阔,宽敞。}  ¶ \textcolor{darkblue}{\textbf{\ipa{qʰɑ˧-hwɤ˧-gv̩˧}}} \textcolor{PineGreen}{\selectlanguage{french}très vaste} \zh{非常宽敞}  

\lhead{\firstmark}
\rhead{\botmark}

\subsection{\hspace{-0.5cm} {\Large \textcolor{darkblue}{\textbf{\ipa{hwɤ˧kʰv̩˥}}}}\hspace{0.5cm}[\kern2pt{\textcolor{darkblue}{\textbf{\ipa{hwɤ˩kʰv̩˥}}}}\kern2pt]} \hypertarget{hw7\string_Mk\string_hv\string_=\string_T1}{}
\markboth{\textcolor{darkblue}{\textbf{\ipa{hwɤ˧kʰv̩˥}}}}{}
\textcolor{teal}{\mytextsc{nom}} \hspace{4pt} Ton~: H\#.
\textcolor{PineGreen}{\selectlanguage{french}Année du Chat (correspondant à l'année chinoise du Rat).} \zh{鼠年(摩梭话称作“猫年”)。} 
\lhead{\firstmark}
\rhead{\botmark}

\subsection{\hspace{-0.5cm} {\Large \textcolor{darkblue}{\textbf{\ipa{hwɤ˧li˧-bv̩˥}}}}\hspace{0.5cm}[\kern2pt{\textcolor{darkblue}{\textbf{\ipa{xxxx non-correspondance entre le nombre de morphèmes et le nombre de tons de morphèmes}}}}\kern2pt]} \hypertarget{hw7\string_Mli\string_M-bv\string_=\string_T1}{}
\markboth{\textcolor{darkblue}{\textbf{\ipa{hwɤ˧li˧-bv̩˥}}}}{}
\textcolor{teal}{\mytextsc{nom}} \hspace{4pt} Ton~: H\#.
\textcolor{PineGreen}{\selectlanguage{french}Mezzanine: espace de la pièce principale où un plancher est aménagé sous la charpente, formant comme une mezzanine, mais que les habitants humains n'utilisent pas: l'endroit étrant très enfumé, on n'y place qu'un plancher peu solide et sans rambarde; d'où le nom: “(la pièce) du chat”. On y laisse parfois des objets (vanneries par exemples), qui y sont relativement préservés des insectes par la fumée.} \zh{夹层:主屋的夹层。因为烟多,所以人不能将这个空间当卧室。只有一层薄的木地板。}  \zh{量词}: \textcolor{darkblue}{\textbf{\ipa{kɤ˧˥}}}  \mytextsc{clf}~: \textcolor{darkblue}{\textbf{\ipa{kɤ˧˥}}} \textit{Syn~:} \hyperlink{}{\textcolor{darkblue}{\textbf{\ipa{hwɤ˧li˧-se˧-di˧˥}}}}. 
\lhead{\firstmark}
\rhead{\botmark}

\subsection{\hspace{-0.5cm} {\Large \textcolor{darkblue}{\textbf{\ipa{hwɤ˧li˧-hwæ˧qʰæ\#˥}}}}\hspace{0.5cm}[\kern2pt{\textcolor{darkblue}{\textbf{\ipa{xxxx non-correspondance entre le nombre de morphèmes et le nombre de tons de morphèmes}}}}\kern2pt]} \hypertarget{hw7\string_Mli\string_M-hw\{\string_Mq\string_h\{\#\string_T1}{}
\markboth{\textcolor{darkblue}{\textbf{\ipa{hwɤ˧li˧-hwæ˧qʰæ\#˥}}}}{}
\textcolor{teal}{\mytextsc{nom}} \hspace{4pt} Ton~: \#H.
\textcolor{PineGreen}{\selectlanguage{french}Yyyy.} \zh{山萝卜。} 
\lhead{\firstmark}
\rhead{\botmark}

\subsection{\hspace{-0.5cm} {\Large \textcolor{darkblue}{\textbf{\ipa{hwɤ˧li˧-se˧-di˧˥}}}}\hspace{0.5cm}[\kern2pt{\textcolor{darkblue}{\textbf{\ipa{xxxx non-correspondance entre le nombre de morphèmes et le nombre de tons de morphèmes}}}}\kern2pt]} \hypertarget{hw7\string_Mli\string_M-se\string_M-di\string_M\string_T1}{}
\markboth{\textcolor{darkblue}{\textbf{\ipa{hwɤ˧li˧-se˧-di˧˥}}}}{}
\textcolor{teal}{\mytextsc{nom}} \hspace{4pt} Ton~: MH\#.
\textcolor{PineGreen}{\selectlanguage{french}Mezzanine: espace de la pièce principale où un plancher est aménagé sous la charpente, formant comme une mezzanine, mais que les habitants humains n'utilisent pas: l'endroit étrant très enfumé, on n'y place qu'un plancher peu solide et sans rambarde; d'où le nom: “(la pièce) du chat”. On y laisse parfois des objets (vanneries par exemples), qui y sont relativement préservés des insectes par la fumée.} \zh{夹层:主屋的夹层。因为烟多,所以人不能将这个空间当卧室。只有一层薄的木地板。}  \zh{量词}: \textcolor{darkblue}{\textbf{\ipa{kɤ˧˥}}}  \mytextsc{clf}~: \textcolor{darkblue}{\textbf{\ipa{kɤ˧˥}}} \textit{Syn~:} \hyperlink{}{\textcolor{darkblue}{\textbf{\ipa{hwɤ˧li˧-bv̩˥}}}}. 
\lhead{\firstmark}
\rhead{\botmark}

\subsection{\hspace{-0.5cm} {\Large \textcolor{darkblue}{\textbf{\ipa{hwɤ˧li˧-ʂɯ˧mo˥}}}}\hspace{0.5cm}[\kern2pt{\textcolor{darkblue}{\textbf{\ipa{xxxx non-correspondance entre le nombre de morphèmes et le nombre de tons de morphèmes}}}}\kern2pt]} \hypertarget{hw7\string_Mli\string_M-s`M\string_Mmo\string_T1}{}
\markboth{\textcolor{darkblue}{\textbf{\ipa{hwɤ˧li˧-ʂɯ˧mo˥}}}}{}
\textcolor{teal}{\mytextsc{nom}} \hspace{4pt} Ton~: H\#.
\textcolor{PineGreen}{\selectlanguage{french}Vieux chat, vieux matou (de l'un ou l'autre sexe).} \zh{老猫(不分公、母)。}  \zh{量词}: \textcolor{darkblue}{\textbf{\ipa{mi˩}}}  \mytextsc{clf}~: \textcolor{darkblue}{\textbf{\ipa{mi˩}}} 
\lhead{\firstmark}
\rhead{\botmark}

\subsection{\hspace{-0.5cm} {\Large \textcolor{darkblue}{\textbf{\ipa{hwɤ˧li˧-zo˧˥}}}}\hspace{0.5cm}[\kern2pt{\textcolor{darkblue}{\textbf{\ipa{xxxx non-correspondance entre le nombre de morphèmes et le nombre de tons de morphèmes}}}}\kern2pt]} \hypertarget{hw7\string_Mli\string_M-zo\string_M\string_T1}{}
\markboth{\textcolor{darkblue}{\textbf{\ipa{hwɤ˧li˧-zo˧˥}}}}{}
\textcolor{teal}{\mytextsc{nom}} \hspace{4pt} Ton~: MH\#.
\textcolor{PineGreen}{\selectlanguage{french}Chaton.} \zh{小猫。}  \zh{量词}: \textcolor{darkblue}{\textbf{\ipa{ɭɯ˧}}}  \mytextsc{clf}~: \textcolor{darkblue}{\textbf{\ipa{ɭɯ˧}}} 
\lhead{\firstmark}
\rhead{\botmark}

\subsection{\hspace{-0.5cm} {\Large \textcolor{darkblue}{\textbf{\ipa{hwɤ˧li˧˥}}}}\hspace{0.5cm}[\kern2pt{\textcolor{darkblue}{\textbf{\ipa{hwɤ˧li˥}}}}\kern2pt]} \hypertarget{hw7\string_Mli\string_M\string_T1}{}
\markboth{\textcolor{darkblue}{\textbf{\ipa{hwɤ˧li˧˥}}}}{}
\textcolor{teal}{\mytextsc{nom}} \hspace{4pt} Ton~: MH\#.
\textcolor{PineGreen}{\selectlanguage{french}Chat.} \zh{猫。}  \zh{量词}: \textcolor{darkblue}{\textbf{\ipa{mi˩}}}  \mytextsc{clf}~: \textcolor{darkblue}{\textbf{\ipa{mi˩}}} 
\lhead{\firstmark}
\rhead{\botmark}

\subsection{\hspace{-0.5cm} {\Large \textcolor{darkblue}{\textbf{\ipa{hwɤ˧mi˥\$}}}}\hspace{0.5cm}[\kern2pt{\textcolor{darkblue}{\textbf{\ipa{hwɤ˧mi˧˥}}}}\kern2pt]} \hypertarget{hw7\string_Mmi\string_T\$1}{}
\markboth{\textcolor{darkblue}{\textbf{\ipa{hwɤ˧mi˥\$}}}}{}
\textcolor{teal}{\mytextsc{nom}} \hspace{4pt} Ton~: H\$.
\textcolor{PineGreen}{\selectlanguage{french}Chatte.} \zh{母猫。}  ¶ \textcolor{darkblue}{\textbf{\ipa{hwɤ˧mi˧-hwɤ˥pʰv̩˩ / hwɤ˧mi˧-hwɤ˧pʰv̩˥\#}}} \textcolor{PineGreen}{\selectlanguage{french}chatte et matou} \zh{母猫与公猫}  
 \zh{量词}: \textcolor{darkblue}{\textbf{\ipa{mi˩}}}  \mytextsc{clf}~: \textcolor{darkblue}{\textbf{\ipa{mi˩}}} 
\lhead{\firstmark}
\rhead{\botmark}

\subsection{\hspace{-0.5cm} {\Large \textcolor{darkblue}{\textbf{\ipa{hwɤ˧pʰv̩\#˥}}}}\hspace{0.5cm}[\kern2pt{\textcolor{darkblue}{\textbf{\ipa{hwɤ˧pʰv̩˥}}}}\kern2pt]} \hypertarget{hw7\string_Mp\string_hv\string_=\#\string_T1}{}
\markboth{\textcolor{darkblue}{\textbf{\ipa{hwɤ˧pʰv̩\#˥}}}}{}
\textcolor{teal}{\mytextsc{nom}} \hspace{4pt} Ton~: \#H.
\textcolor{PineGreen}{\selectlanguage{french}Matou, chat mâle.} \zh{公猫。}  ¶ \textcolor{darkblue}{\textbf{\ipa{hwɤ˧pʰv̩˧ tʰv̩˧-mi˥\#}}} \textcolor{PineGreen}{\selectlanguage{french}\mytextsc{n}+\mytextsc{dem}+\mytextsc{clf}} \zh{那个公猫}  
 ¶ \textcolor{darkblue}{\textbf{\ipa{hwɤ˧pʰv̩˧-hwɤ˧mi˥}}} \textcolor{PineGreen}{\selectlanguage{french}matou et chatte} \zh{公猫与母猫}  
 \zh{量词}: \textcolor{darkblue}{\textbf{\ipa{mi˩}}}  \mytextsc{clf}~: \textcolor{darkblue}{\textbf{\ipa{mi˩}}} 
\lhead{\firstmark}
\rhead{\botmark}

\subsection{\hspace{-0.5cm} {\Large \textcolor{darkblue}{\textbf{\ipa{hwɤ˧se˧}}}}\hspace{0.5cm}[\kern2pt{\textcolor{darkblue}{\textbf{\ipa{hwɤ˧se˧}}}}\kern2pt]} \hypertarget{hw7\string_Mse\string_M1}{}
\markboth{\textcolor{darkblue}{\textbf{\ipa{hwɤ˧se˧}}}}{}
\textcolor{teal}{\mytextsc{nom}} \hspace{4pt} Ton~: M.
\textcolor{PineGreen}{\selectlanguage{french}Cacahuètes.} \zh{花生。}  Emprunt~: chinois  \zh{花生}
 ¶ \textcolor{darkblue}{\textbf{\ipa{hwɤ˧se˧-qo˧tv̩˩}}} \textcolor{PineGreen}{\selectlanguage{french}même sens; littéralement “graines de cacahuètes”} \zh{花生米}  

\lhead{\firstmark}
\rhead{\botmark}

\subsection{\hspace{-0.5cm} {\Large \textcolor{darkblue}{\textbf{\ipa{hwɤ˧tɕi˥}}}}\hspace{0.5cm}[\kern2pt{\textcolor{darkblue}{\textbf{\ipa{hwɤ˧tɕi˧}}}}\kern2pt]} \hypertarget{hw7\string_Mts£i\string_T1}{}
\markboth{\textcolor{darkblue}{\textbf{\ipa{hwɤ˧tɕi˥}}}}{}
\textcolor{teal}{\mytextsc{nom}} \hspace{4pt} Ton~: H\#.
\textcolor{PineGreen}{\selectlanguage{french}Parelle sauvage, oseille crépue, patience crépue, patience sauvage, \textit{Rumex crispus}. Cette plante constitue l'une des trois sortes de fourrage utilisées pour les cochons; elle est aussi consommée par les humains.} \zh{土大黄(学名:皱叶酸模)(喂猪的牧草)。}  ¶ \textcolor{darkblue}{\textbf{\ipa{hwɤ˧tɕi˥-bæ˩bæ˩}}} \textcolor{PineGreen}{\selectlanguage{french}même sens} \zh{同上}  
 ¶ \textcolor{darkblue}{\textbf{\ipa{hwɤ˧tɕʰi˥-ʁo˩bv̩˩}}} \textcolor{PineGreen}{\selectlanguage{french}pousses de parelle sauvage} \zh{土大黄的嫩芽}  
 \zh{量词}: \textcolor{darkblue}{\textbf{\ipa{po˧}}}  \mytextsc{clf}~: \textcolor{darkblue}{\textbf{\ipa{po˧}}} 
\lhead{\firstmark}
\rhead{\botmark}

\subsection{\hspace{-0.5cm} {\Large \textcolor{darkblue}{\textbf{\ipa{hwɤ˧zo\#˥}}}}\hspace{0.5cm}[\kern2pt{\textcolor{darkblue}{\textbf{\ipa{hwɤ˩zo˥}}}}\kern2pt]} \hypertarget{hw7\string_Mzo\#\string_T1}{}
\markboth{\textcolor{darkblue}{\textbf{\ipa{hwɤ˧zo\#˥}}}}{}
\textcolor{teal}{\mytextsc{nom}} \hspace{4pt} Ton~: \#H.
\textcolor{PineGreen}{\selectlanguage{french}Chaton.} \zh{小猫。}  ¶ \textcolor{darkblue}{\textbf{\ipa{hwɤ˧zo˧ tʰv̩˧-ɭɯ\#˥}}} \textcolor{PineGreen}{\selectlanguage{french}\mytextsc{n}+\mytextsc{dem}+\mytextsc{clf}} \zh{那个小猫}  
 ¶ \textcolor{darkblue}{\textbf{\ipa{hwɤ˧zo˧-hwɤ˧mi˥}}} \textcolor{PineGreen}{\selectlanguage{french}chats (toute la famille: chatons et parents)} \zh{猫,包括小猫、母猫和公猫}  
 \zh{量词}: \textcolor{darkblue}{\textbf{\ipa{ɭɯ˧}}}  \mytextsc{clf}~: \textcolor{darkblue}{\textbf{\ipa{ɭɯ˧}}} 
\lhead{\firstmark}
\rhead{\botmark}

\subsection{\hspace{-0.5cm} {\Large \textcolor{darkblue}{\textbf{\ipa{hwɤ˩}}}}\hspace{0.5cm}[\kern2pt{\textcolor{darkblue}{\textbf{\ipa{xxxx non-correspondance entre le nombre de morphèmes et le nombre de tons de morphèmes}}}}\kern2pt]} \hypertarget{hw7\string_B1}{}
\markboth{\textcolor{darkblue}{\textbf{\ipa{hwɤ˩}}}}{}
\textcolor{teal}{\mytextsc{verbe}} \hspace{4pt} Ton~: L\textsubscript{a}.
\ding{202} \textcolor{PineGreen}{\selectlanguage{french}Attacher, emballer, préparer un fardeau/une charge, constituer un ballot (avec de l'herbe, des objets…), mettre en botte, mettre en ballot.} \zh{捆(捆成捆儿)。}  ¶ \textcolor{darkblue}{\textbf{\ipa{zɯ˧-wɤ˧ hwɤ˥}}} \textcolor{PineGreen}{\selectlanguage{french}faire un ballot d'herbes} \zh{将草捆成一垛、捆一垛草}  
 ¶ \textcolor{darkblue}{\textbf{\ipa{si˧-wɤ˧ hwɤ˥}}} \textcolor{PineGreen}{\selectlanguage{french}faire un ballot de bois} \zh{将木头捆成一堆、捆一堆木头}  
 ¶ \textcolor{darkblue}{\textbf{\ipa{hɑ˧-wɤ˧ hwɤ˥}}} \textcolor{PineGreen}{\selectlanguage{french}faire un ballot de céréales} \zh{将粮食捆成一包、捆一包粮食}  
 ¶ \textcolor{darkblue}{\textbf{\ipa{wɤ˩ hwɤ˩˥}}} \textcolor{PineGreen}{\selectlanguage{french}préparer un fardeau, mettre en ballot / mettre (des objets, des choses) en paquet, de façon à ce qu'une personne puisse le porter} \zh{捆成一包}  
 ¶ \textcolor{darkblue}{\textbf{\ipa{wɤ˩˥ | tʰi˧-hwɤ˩}}} \textcolor{PineGreen}{\selectlanguage{french}et après, on en fait un ballot / on en fait un paquet / on attache ça ensemble!} \zh{然后,捆成一包!}  
 ¶ \textcolor{darkblue}{\textbf{\ipa{wɤ˩˥ | ɖɯ˧-wɤ˩ hwɤ˩}}} \textcolor{PineGreen}{\selectlanguage{french}faire un ballot de plus} \zh{又捆一包}  
\ding{203} \textcolor{PineGreen}{\selectlanguage{french}Sens figuré: avoir des embarras, se créer des embarras sur les bras, s'empêtrer.} \zh{有困难、像把自己捆起来一样。}  ¶ \textcolor{darkblue}{\textbf{\ipa{hĩ˧, | wɤ˩ hwɤ˧ ʝi˧-ni˥gv̩˩!}}} \textcolor{PineGreen}{\selectlanguage{french}Les gens, ils étaient accablés / ils étaient malheureux! (Image: les gens étaient comme ficelés, incapables de se mouvoir normalement, de vivre leur vie normalement.)} \zh{人家难受,像被捆一样}  
 ¶ \textcolor{darkblue}{\textbf{\ipa{wɤ˩hwɤ˧ ʝi˧-ni˥gv̩˩-ɲi˩-ze˩!}}} \textcolor{PineGreen}{\selectlanguage{french}(je me) suis mis des grosses complications sur les bras!} \zh{我给自己找麻烦了!}  

\lhead{\firstmark}
\rhead{\botmark}

\subsection{\hspace{-0.5cm} {\Large \textcolor{darkblue}{\textbf{\ipa{hwɤ˩\textsubscript{a}}}}}\hspace{0.5cm}[\kern2pt{\textcolor{darkblue}{\textbf{\ipa{hwɤ˩˥}}}}\kern2pt]} \hypertarget{hw7\string_Ba1}{}
\markboth{\textcolor{darkblue}{\textbf{\ipa{hwɤ˩\textsubscript{a}}}}}{}
\textcolor{teal}{\mytextsc{verbe}} \hspace{4pt} Ton~: L\textsubscript{a}.
\textcolor{PineGreen}{\selectlanguage{french}Passer un objet, envoyer un objet (à quelqu'un).} \zh{递过去。}  ¶ \textcolor{darkblue}{\textbf{\ipa{hĩ˧-ki˧ | tso˧\textasciitilde{}tso˧ hwɤ˥}}} \textcolor{PineGreen}{\selectlanguage{french}envoyer des choses à quelqu'un} \zh{给人家寄东西}  

\lhead{\firstmark}
\rhead{\botmark}

\subsection{\hspace{-0.5cm} {\Large \textcolor{darkblue}{\textbf{\ipa{hwɤ˩dʑɯ˩}}}}\hspace{0.5cm}[\kern2pt{\textcolor{darkblue}{\textbf{\ipa{hwɤ˩dʑɯ˩˥}}}}\kern2pt]} \hypertarget{hw7\string_Bdz£M\string_B1}{}
\markboth{\textcolor{darkblue}{\textbf{\ipa{hwɤ˩dʑɯ˩}}}}{}
\textcolor{teal}{\mytextsc{nom}} \hspace{4pt} Ton~: L.
\textcolor{PineGreen}{\selectlanguage{french}Cabane, hutte (qu'on construit sur la montagne quand on doit y passer qq nuits, par exemple pour couper du bois).} \zh{山上过夜的小木房。}  \zh{量词}: \textcolor{darkblue}{\textbf{\ipa{ɭɯ˧}}}  \mytextsc{clf}~: \textcolor{darkblue}{\textbf{\ipa{ɭɯ˧}}} 
\lhead{\firstmark}
\rhead{\botmark}

\subsection{\hspace{-0.5cm} {\Large \textcolor{darkblue}{\textbf{\ipa{hwɤ˩kæ˧}}}}\hspace{0.5cm}[\kern2pt{\textcolor{darkblue}{\textbf{\ipa{hwɤ˩kæ˩˥}}}}\kern2pt]} \hypertarget{hw7\string_Bk\{\string_M1}{}
\markboth{\textcolor{darkblue}{\textbf{\ipa{hwɤ˩kæ˧}}}}{}
\textcolor{teal}{\mytextsc{nom}} \hspace{4pt} Ton~: LM.
\textcolor{PineGreen}{\selectlanguage{french}Bouleau rouge; son bois est bon, on s'en sert pour fabriquer des araires.} \zh{红桦树。}  ¶ \textcolor{darkblue}{\textbf{\ipa{hwɤ˩kæ˧-si˧dzi˩}}} \textcolor{PineGreen}{\selectlanguage{french}même sens: bouleau rouge} \zh{同上:红桦树}  

\lhead{\firstmark}
\rhead{\botmark}

\subsection{\hspace{-0.5cm} {\Large \textcolor{darkblue}{\textbf{\ipa{hwɤ˩ʈi˥}}}}\hspace{0.5cm}[\kern2pt{\textcolor{darkblue}{\textbf{\ipa{hwɤ˧ʈi˥}}}}\kern2pt]} \hypertarget{hw7\string_Bt`i\string_T1}{}
\markboth{\textcolor{darkblue}{\textbf{\ipa{hwɤ˩ʈi˥}}}}{}
\textcolor{teal}{\mytextsc{verbe}} \hspace{4pt} Ton~: LH.
\textcolor{PineGreen}{\selectlanguage{french}Rouiller.} \zh{生锈。}  ¶ \textcolor{darkblue}{\textbf{\ipa{hwɤ˩ʈi˥-ze˩}}} \textcolor{PineGreen}{\selectlanguage{french}\mytextsc{pfv}: ça a rouillé} \zh{生锈了}  

\lhead{\firstmark}
\rhead{\botmark}

\subsection{\hspace{-0.5cm} {\Large \textcolor{darkblue}{\textbf{\ipa{hwɤ˧˥}}} \textsubscript{1}}\hspace{0.5cm}[\kern2pt{\textcolor{darkblue}{\textbf{\ipa{hwɤ˥}}}}\kern2pt]} \hypertarget{hw7\string_M\string_T1}{}
\markboth{\textcolor{darkblue}{\textbf{\ipa{hwɤ˧˥}}} \textsubscript{1}}{}
\textcolor{teal}{\mytextsc{verbe}} \hspace{4pt} Ton~: MH.
\textcolor{PineGreen}{\selectlanguage{french}Repriser, raccommoder (vêtement).} \zh{补。}  ¶ \textcolor{darkblue}{\textbf{\ipa{le˧-hwɤ˧˥}}} \textcolor{PineGreen}{\selectlanguage{french}\mytextsc{accomp}} \zh{\mytextsc{accomp}}  
 ¶ \textcolor{darkblue}{\textbf{\ipa{bɑ˩lɑ˩ hwɤ˥}}} \textcolor{PineGreen}{\selectlanguage{french}réparer un vêtement, recoudre un vêtement, rapetasser un vêtement} \zh{补衣服}  

\lhead{\firstmark}
\rhead{\botmark}

\subsection{\hspace{-0.5cm} {\Large \textcolor{darkblue}{\textbf{\ipa{hwɤ˧˥}}} \textsubscript{2}}\hspace{0.5cm}[\kern2pt{\textcolor{darkblue}{\textbf{\ipa{hwɤ˧˥}}}}\kern2pt]} \hypertarget{hw7\string_M\string_T2}{}
\markboth{\textcolor{darkblue}{\textbf{\ipa{hwɤ˧˥}}} \textsubscript{2}}{}
\textcolor{teal}{\mytextsc{nom}} \hspace{4pt} Ton~: MH.
\textcolor{PineGreen}{\selectlanguage{french}Chat (monosyllabique).} \zh{猫(单音节)。}  \zh{量词}: \textcolor{darkblue}{\textbf{\ipa{mi˩}}}  \mytextsc{clf}~: \textcolor{darkblue}{\textbf{\ipa{mi˩}}} 
\lhead{\firstmark}
\rhead{\botmark}

\subsection{\hspace{-0.5cm} {\Large \textcolor{darkblue}{\textbf{\ipa{hwɤ˧˥}}} \textsubscript{3}}\hspace{0.5cm}[\kern2pt{\textcolor{darkblue}{\textbf{\ipa{hwɤ˧˥}}}}\kern2pt]} \hypertarget{hw7\string_M\string_T3}{}
\markboth{\textcolor{darkblue}{\textbf{\ipa{hwɤ˧˥}}} \textsubscript{3}}{}
\textcolor{teal}{\mytextsc{nom}} \hspace{4pt} Ton~: MH.
\textcolor{PineGreen}{\selectlanguage{french}Rouille (monosyllabe).} \zh{锈(单音节)。} 
\lhead{\firstmark}
\rhead{\botmark}

\newpage
\section*{\centering- \textcolor{darkblue}{\textbf{\ipa{ĩ}}} -}
\subsection{\hspace{-0.5cm} {\Large \textcolor{darkblue}{\textbf{\ipa{ĩ˧}}}}\hspace{0.5cm}[\kern2pt{\textcolor{darkblue}{\textbf{\ipa{ĩ˥}}}}\kern2pt]} \hypertarget{i\string_~\string_M1}{}
\markboth{\textcolor{darkblue}{\textbf{\ipa{ĩ˧}}}}{}
\textcolor{teal}{\mytextsc{interjection}} \hspace{4pt} Ton~: M.
\textcolor{PineGreen}{\selectlanguage{french}Oui, d'accord.} \zh{是的,好的。}  ¶ \textcolor{darkblue}{\textbf{\ipa{ʈʂʰɯ˧-ɳɯ˧ | “ĩ˧! ĩ˧!” | pi˧. |}}} \textcolor{PineGreen}{\selectlanguage{french}il a dit: “Oui, oui!”} \zh{他说:“是的,是的!”}  

\lhead{\firstmark}
\rhead{\botmark}

\newpage
\section*{\centering- \textcolor{darkblue}{\textbf{\ipa{j}}} \textcolor{darkblue}{\textbf{\ipa{jæ}}} \textcolor{darkblue}{\textbf{\ipa{jɤ}}} \textcolor{darkblue}{\textbf{\ipa{jo}}} \textcolor{darkblue}{\textbf{\ipa{je}}} -}
\subsection{\hspace{-0.5cm} {\Large \textcolor{darkblue}{\textbf{\ipa{jɤ˧}}} \textsubscript{1}}\hspace{0.5cm}[\kern2pt{\textcolor{darkblue}{\textbf{\ipa{jɤ˥}}}}\kern2pt]} \hypertarget{j7\string_M1}{}
\markboth{\textcolor{darkblue}{\textbf{\ipa{jɤ˧}}} \textsubscript{1}}{}
\textcolor{teal}{\mytextsc{adjectif}} \hspace{4pt} Ton~: M.
\textcolor{PineGreen}{\selectlanguage{french}Bien (ne s'utilise qu'en tournure négative).} \zh{好(只出现在否定词后面)。}  ¶ \textcolor{darkblue}{\textbf{\ipa{mɤ˧-jɤ˧}}} \textcolor{PineGreen}{\selectlanguage{french}\mytextsc{neg}: ce n'est pas bien! / c'est pas beau, ça! (Au sujet du comportement de quelqu'un)} \zh{不好(形容一个人的行为)}  

\lhead{\firstmark}
\rhead{\botmark}

\subsection{\hspace{-0.5cm} {\Large \textcolor{darkblue}{\textbf{\ipa{jɤ˧}}} \textsubscript{2}}\hspace{0.5cm}[\kern2pt{\textcolor{darkblue}{\textbf{\ipa{jɤ˥}}}}\kern2pt]} \hypertarget{j7\string_M2}{}
\markboth{\textcolor{darkblue}{\textbf{\ipa{jɤ˧}}} \textsubscript{2}}{}
\textcolor{teal}{\mytextsc{adjectif}} \hspace{4pt} Ton~: M.
\textcolor{PineGreen}{\selectlanguage{french}Plat.} \zh{平(土地)。}  ¶ \textcolor{darkblue}{\textbf{\ipa{mɤ˧-jɤ˧}}} \textcolor{PineGreen}{\selectlanguage{french}\mytextsc{neg}: pas plat; inégal} \zh{不平}  

\lhead{\firstmark}
\rhead{\botmark}

\subsection{\hspace{-0.5cm} {\Large \textcolor{darkblue}{\textbf{\ipa{jɤ˧}}} \textsubscript{3}}\hspace{0.5cm}[\kern2pt{\textcolor{darkblue}{\textbf{\ipa{jɤ˥}}}}\kern2pt]} \hypertarget{j7\string_M3}{}
\markboth{\textcolor{darkblue}{\textbf{\ipa{jɤ˧}}} \textsubscript{3}}{}
\textcolor{teal}{\mytextsc{nom}} \hspace{4pt} Ton~: M.
\textcolor{PineGreen}{\selectlanguage{french}Tabac, cigarette.} \zh{烟。}  Emprunt~: chinois  \zh{烟?}
 ¶ \textcolor{darkblue}{\textbf{\ipa{jɤ˧ ʈʰɯ˩}}} \textcolor{PineGreen}{\selectlanguage{french}fumer} \zh{抽烟}  
 \zh{量词}: \textcolor{darkblue}{\textbf{\ipa{ko˧}}}  \mytextsc{clf}~: \textcolor{darkblue}{\textbf{\ipa{ko˧}}} 
\lhead{\firstmark}
\rhead{\botmark}

\subsection{\hspace{-0.5cm} {\Large \textcolor{darkblue}{\textbf{\ipa{jɤ˧gɯ˩}}}}\hspace{0.5cm}[\kern2pt{\textcolor{darkblue}{\textbf{\ipa{jɤ˧gɯ˩}}}}\kern2pt]} \hypertarget{j7\string_MgM\string_B1}{}
\markboth{\textcolor{darkblue}{\textbf{\ipa{jɤ˧gɯ˩}}}}{}
\textcolor{teal}{\mytextsc{nom}} \hspace{4pt} Ton~: L\#.
\textcolor{PineGreen}{\selectlanguage{french}Sarrasin, blé noir, \textit{Fagopyrum esculentum}.} \zh{甜荞/荞麦/花荞。}  \zh{量词}: \textcolor{darkblue}{\textbf{\ipa{kɤ˧˥}}}  \mytextsc{clf}~: \textcolor{darkblue}{\textbf{\ipa{kɤ˧˥}}} \textit{Voir~:} \hyperlink{}{\textcolor{darkblue}{\textbf{\ipa{jɤ˧qʰɑ\#˥}}}} 
\lhead{\firstmark}
\rhead{\botmark}

\subsection{\hspace{-0.5cm} {\Large \textcolor{darkblue}{\textbf{\ipa{jɤ˧ŋɤ˧}}}}\hspace{0.5cm}[\kern2pt{\textcolor{darkblue}{\textbf{\ipa{jɤ˧ŋɤ˧}}}}\kern2pt]} \hypertarget{j7\string_MN7\string_M1}{}
\markboth{\textcolor{darkblue}{\textbf{\ipa{jɤ˧ŋɤ˧}}}}{}
\textcolor{teal}{\mytextsc{nom}} \hspace{4pt} Ton~: M.
\textcolor{PineGreen}{\selectlanguage{french}La ville de Chengdu, dans le Sichuan.} \zh{成都。}  ¶ \textcolor{darkblue}{\textbf{\ipa{ho˧di˧-jɤ˧ŋɤ˧}}} \textcolor{PineGreen}{\selectlanguage{french}même sens} \zh{同上}  

\lhead{\firstmark}
\rhead{\botmark}

\subsection{\hspace{-0.5cm} {\Large \textcolor{darkblue}{\textbf{\ipa{jɤ˧qʰɑ\#˥}}}}\hspace{0.5cm}[\kern2pt{\textcolor{darkblue}{\textbf{\ipa{jɤ˧qʰɑ˧}}}}\kern2pt]} \hypertarget{j7\string_Mq\string_hA\#\string_T1}{}
\markboth{\textcolor{darkblue}{\textbf{\ipa{jɤ˧qʰɑ\#˥}}}}{}
\textcolor{teal}{\mytextsc{nom}} \hspace{4pt} Ton~: \#H.
\textcolor{PineGreen}{\selectlanguage{french}Sarrasin amer, \textit{Fagopyrum tataricum Gaertn}.} \zh{苦荞。}  \zh{量词}: \textcolor{darkblue}{\textbf{\ipa{kɤ˧˥}}}  \mytextsc{clf}~: \textcolor{darkblue}{\textbf{\ipa{kɤ˧˥}}} \textit{Voir~:} \hyperlink{}{\textcolor{darkblue}{\textbf{\ipa{jɤ˧gɯ˩}}}} 
\lhead{\firstmark}
\rhead{\botmark}

\subsection{\hspace{-0.5cm} {\Large \textcolor{darkblue}{\textbf{\ipa{jɤ˧qʰɑ˧-pɤ˥jɤ˩-mo˩}}}}\hspace{0.5cm}[\kern2pt{\textcolor{darkblue}{\textbf{\ipa{xxxx non-correspondance entre le nombre de morphèmes et le nombre de tons de morphèmes}}}}\kern2pt]} \hypertarget{j7\string_Mq\string_hA\string_M-p7\string_Tj7\string_B-mo\string_B1}{}
\markboth{\textcolor{darkblue}{\textbf{\ipa{jɤ˧qʰɑ˧-pɤ˥jɤ˩-mo˩}}}}{}
\textcolor{teal}{\mytextsc{nom}} \hspace{4pt} Ton~: \#H\mytextsc{}.
\textcolor{PineGreen}{\selectlanguage{french}Bolet, cèpe, \textit{Boletus edulis}; littéralement “champignon-galette de sarrasin”, du fait de sa texture.} \zh{牛肝菌。} \textit{Voir~:} \hyperlink{}{\textcolor{darkblue}{\textbf{\ipa{njo˩kæ˧tɕi˩˥}}}} 
\lhead{\firstmark}
\rhead{\botmark}

\subsection{\hspace{-0.5cm} {\Large \textcolor{darkblue}{\textbf{\ipa{jɤ˧wo˧˥}}}}\hspace{0.5cm}[\kern2pt{\textcolor{darkblue}{\textbf{\ipa{jɤ˧wo˧˥}}}}\kern2pt]} \hypertarget{j7\string_Mwo\string_M\string_T1}{}
\markboth{\textcolor{darkblue}{\textbf{\ipa{jɤ˧wo˧˥}}}}{}
\textcolor{teal}{\mytextsc{verbe}} \hspace{4pt} Ton~: MH\#.
\textcolor{PineGreen}{\selectlanguage{french}Régresser (le contraire de: faire des progrès).} \zh{倒退、退步。}  ¶ \textcolor{darkblue}{\textbf{\ipa{no˧ | jɤ˧wo˧˥ | sɯ˧ɖʐæ˧! / no˧ | le˧-wo˥ | sɯ˧ɖʐæ˧!}}} \textcolor{PineGreen}{\selectlanguage{french}tu régresses! (adressé à un petit enfant qui a fait caca dans sa culotte, alors que depuis plusieurs semaines il avait pris l'habitude du pot)} \zh{你这是在退步!(情景:一个小孩已经几个礼拜有了上厕所的习惯,那天又把屎拉在裤头里)}  

\lhead{\firstmark}
\rhead{\botmark}

\subsection{\hspace{-0.5cm} {\Large \textcolor{darkblue}{\textbf{\ipa{jɤ˩\textsubscript{a}}}} \textsubscript{1}}\hspace{0.5cm}[\kern2pt{\textcolor{darkblue}{\textbf{\ipa{jɤ˩˥}}}}\kern2pt]} \hypertarget{j7\string_Ba1}{}
\markboth{\textcolor{darkblue}{\textbf{\ipa{jɤ˩\textsubscript{a}}}} \textsubscript{1}}{}
\textcolor{teal}{\mytextsc{verbe}} \hspace{4pt} Ton~: L\textsubscript{a}.
\textcolor{PineGreen}{\selectlanguage{french}Enrouler (ex.: des fibres de lin).} \zh{盘、盘绕(线)。}  ¶ \textcolor{darkblue}{\textbf{\ipa{sɑ˧ jɤ˥}}} \textcolor{PineGreen}{\selectlanguage{french}enrouler du fil de lin} \zh{盘麻线}  
 ¶ \textcolor{darkblue}{\textbf{\ipa{sɑ˧ | le˧-jɤ˩}}} \textcolor{PineGreen}{\selectlanguage{french}enrouler du fil de lin} \zh{盘麻线}  
\textit{Voir~:} \hyperlink{}{\textcolor{darkblue}{\textbf{\ipa{tɕɯ˧ɭɯ˧}}}} 
\lhead{\firstmark}
\rhead{\botmark}

\subsection{\hspace{-0.5cm} {\Large \textcolor{darkblue}{\textbf{\ipa{jɤ˩\textsubscript{a}}}} \textsubscript{2}}\hspace{0.5cm}[\kern2pt{\textcolor{darkblue}{\textbf{\ipa{jɤ˩˥}}}}\kern2pt]} \hypertarget{j7\string_Ba2}{}
\markboth{\textcolor{darkblue}{\textbf{\ipa{jɤ˩\textsubscript{a}}}} \textsubscript{2}}{}
\textcolor{teal}{\mytextsc{adjectif}} \hspace{4pt} Ton~: L\textsubscript{a}.
\textcolor{PineGreen}{\selectlanguage{french}Trop cuit, dissous, tout décomposé: ex.: pommes de terre qui éclatent, pois qui se décomposent en purée.} \zh{(煮)烂。}  ¶ \textcolor{darkblue}{\textbf{\ipa{le˧-tɕɤ˧˥ | le˧-jɤ˩-ze˩!}}} \textcolor{PineGreen}{\selectlanguage{french}ça s'est décomposé à force de bouillir!} \zh{煮烂了!}  
 ¶ \textcolor{darkblue}{\textbf{\ipa{jɤ˩-hĩ˩˥}}} \textcolor{PineGreen}{\selectlanguage{french}\mytextsc{rel}/\mytextsc{nmlz}} \zh{烂的}  

\lhead{\firstmark}
\rhead{\botmark}

\subsection{\hspace{-0.5cm} {\Large \textcolor{darkblue}{\textbf{\ipa{jɤ˩\textsubscript{b}}}} \textsubscript{1}}\hspace{0.5cm}[\kern2pt{\textcolor{darkblue}{\textbf{\ipa{jɤ˩˥}}}}\kern2pt]} \hypertarget{j7\string_Bb1}{}
\markboth{\textcolor{darkblue}{\textbf{\ipa{jɤ˩\textsubscript{b}}}} \textsubscript{1}}{}
\textcolor{teal}{\mytextsc{verbe}} \hspace{4pt} Ton~: L\textsubscript{b}.
\textcolor{PineGreen}{\selectlanguage{french}Être fatigué, être sans entrain, être à la masse.} \zh{没精神。}  ¶ \textcolor{darkblue}{\textbf{\ipa{tʰi˧-jɤ˩-ho˩-ze˩!}}} \textcolor{PineGreen}{\selectlanguage{french}(Il/elle) est à la masse!} \zh{他没精神了!}  
 ¶ \textcolor{darkblue}{\textbf{\ipa{ɑ˩ʁo˧ ʂv̩˧ɖv̩˧ | tʰi˧-jɤ˩-ho˩-tsɯ˩!}}} \textcolor{PineGreen}{\selectlanguage{french}Quand on a la nostalgie, on est sans entrain!} \zh{想家的时候,没精神!}  
 ¶ \textcolor{darkblue}{\textbf{\ipa{ɑ˩ʁo˧ ʂv̩˧ɖv̩˧-zo˥, | tʰi˧-jɤ˩-ho˩!}}} \textcolor{PineGreen}{\selectlanguage{french}Quand on a la nostalgie, on est sans entrain!} \zh{想家的时候,没精神!}  
 ¶ \textcolor{darkblue}{\textbf{\ipa{ɲi˧mi˧ tsʰi˧-zo˩, | tʰi˧-jɤ˩-ho˩!}}} \textcolor{PineGreen}{\selectlanguage{french}Quand il fait très chaud, on est sans entrain!} \zh{天气很热,没精神!}  
 ¶ \textcolor{darkblue}{\textbf{\ipa{jɤ˩-mɤ˥-jɤ˩}}} \textcolor{PineGreen}{\selectlanguage{french}\string_ \mytextsc{neg} \string_} \zh{\string_ \mytextsc{neg} \string_}  

\lhead{\firstmark}
\rhead{\botmark}

\subsection{\hspace{-0.5cm} {\Large \textcolor{darkblue}{\textbf{\ipa{jɤ˩\textsubscript{b}}}} \textsubscript{2}}\hspace{0.5cm}[\kern2pt{\textcolor{darkblue}{\textbf{\ipa{jɤ˩˥}}}}\kern2pt]} \hypertarget{j7\string_Bb2}{}
\markboth{\textcolor{darkblue}{\textbf{\ipa{jɤ˩\textsubscript{b}}}} \textsubscript{2}}{}
\textcolor{teal}{\mytextsc{classificateur}} \hspace{4pt} Ton~: L\textsubscript{b}.
\textcolor{PineGreen}{\selectlanguage{french}Classificateur des rangées de légumes (dans un potager, dans un champ).} \zh{量词:排(一排菜)。}  ¶ \textcolor{darkblue}{\textbf{\ipa{v˩tsʰɤ˧˥ | ɖɯ˧-jɤ˩ tʰi˩-pʰo˩}}} \textcolor{PineGreen}{\selectlanguage{french}planter une rangée de légumes} \zh{种一排菜}  

\lhead{\firstmark}
\rhead{\botmark}

\subsection{\hspace{-0.5cm} {\Large \textcolor{darkblue}{\textbf{\ipa{jɤ˩ɕjo˧-dzɑ˧qʰwɤ˩}}}}\hspace{0.5cm}[\kern2pt{\textcolor{darkblue}{\textbf{\ipa{jɤ˩ɕjo˧dzɑ˧qʰwɤ˩}}}}\kern2pt]} \hypertarget{j7\string_Bs£jo\string_M-dzA\string_Mq\string_hw7\string_B1}{}
\markboth{\textcolor{darkblue}{\textbf{\ipa{jɤ˩ɕjo˧-dzɑ˧qʰwɤ˩}}}}{}
\textcolor{teal}{\mytextsc{nom}} \hspace{4pt} Ton~: LM-L\#.
\textcolor{PineGreen}{\selectlanguage{french}Sandale.} \zh{凉鞋。}  \zh{量词}: \textcolor{darkblue}{\textbf{\ipa{dzi˧}}}  \mytextsc{clf}~: \textcolor{darkblue}{\textbf{\ipa{dzi˧}}} \textcolor{PineGreen}{\selectlanguage{french}paire}
\lhead{\firstmark}
\rhead{\botmark}

\subsection{\hspace{-0.5cm} {\Large \textcolor{darkblue}{\textbf{\ipa{jɤ˩ho˧}}}}\hspace{0.5cm}[\kern2pt{\textcolor{darkblue}{\textbf{\ipa{jɤ˩ho˥}}}}\kern2pt]} \hypertarget{j7\string_Bho\string_M1}{}
\markboth{\textcolor{darkblue}{\textbf{\ipa{jɤ˩ho˧}}}}{}
\textcolor{teal}{\mytextsc{nom}} \hspace{4pt} Ton~: LM.
\textcolor{PineGreen}{\selectlanguage{french}Allumette.} \zh{火柴(洋火)。}  Emprunt~: chinois  \zh{洋火}
 \zh{量词}: \textcolor{darkblue}{\textbf{\ipa{po˩}}}  \mytextsc{clf}~: \textcolor{darkblue}{\textbf{\ipa{po˩}}} 
\lhead{\firstmark}
\rhead{\botmark}

\subsection{\hspace{-0.5cm} {\Large \textcolor{darkblue}{\textbf{\ipa{jɤ˩jo\#˥}}}}\hspace{0.5cm}[\kern2pt{\textcolor{darkblue}{\textbf{\ipa{jɤ˩jo˥}}}}\kern2pt]} \hypertarget{j7\string_Bjo\#\string_T1}{}
\markboth{\textcolor{darkblue}{\textbf{\ipa{jɤ˩jo\#˥}}}}{}
\textcolor{teal}{\mytextsc{nom}} \hspace{4pt} Ton~: LM+\#H.
\textcolor{PineGreen}{\selectlanguage{french}Pomme de terre.} \zh{洋芋、土豆 、马铃薯(汉语借词)。}  Emprunt~: chinois  \zh{洋芋}
 \zh{量词}: \textcolor{darkblue}{\textbf{\ipa{kɤ˧˥}}}  \mytextsc{clf}~: \textcolor{darkblue}{\textbf{\ipa{kɤ˧˥}}} 
\lhead{\firstmark}
\rhead{\botmark}

\subsection{\hspace{-0.5cm} {\Large \textcolor{darkblue}{\textbf{\ipa{jɤ˩jo˧-bv̩\#˥}}}}\hspace{0.5cm}[\kern2pt{\textcolor{darkblue}{\textbf{\ipa{xxxx non-correspondance entre le nombre de morphèmes et le nombre de tons de morphèmes}}}}\kern2pt]} \hypertarget{j7\string_Bjo\string_M-bv\string_=\#\string_T1}{}
\markboth{\textcolor{darkblue}{\textbf{\ipa{jɤ˩jo˧-bv̩\#˥}}}}{}
\textcolor{teal}{\mytextsc{nom}} \hspace{4pt} Ton~: LM+\#H.
\textit{De:} \textbf{jɤ˩jo\#˥ et bv̩˥} \textcolor{PineGreen}{\selectlanguage{french}Larve de taupin, ver fil de fer, \textit{Agriotes lineatus}: ver qui mange les tubercules.} \zh{蛴螬。}  \zh{量词}: \textcolor{darkblue}{\textbf{\ipa{mi˩}}}  \mytextsc{clf}~: \textcolor{darkblue}{\textbf{\ipa{mi˩}}} 
\lhead{\firstmark}
\rhead{\botmark}

\subsection{\hspace{-0.5cm} {\Large \textcolor{darkblue}{\textbf{\ipa{jɤ˩pæ˧sɯ˥\$}}}}\hspace{0.5cm}[\kern2pt{\textcolor{darkblue}{\textbf{\ipa{xxxx ton non trouvé, à faire manuellement...}}}}\kern2pt]} \hypertarget{j7\string_Bp\{\string_MsM\string_T\$1}{}
\markboth{\textcolor{darkblue}{\textbf{\ipa{jɤ˩pæ˧sɯ˥\$}}}}{}
\textcolor{teal}{\mytextsc{nom}} \hspace{4pt} Ton~: LM+H\$.
\textcolor{PineGreen}{\selectlanguage{french}Petit Chef Yang: nom de famille constitué de l'expression chinoise \zh{杨把事}, formé du patronyme \zh{杨}, suivi du terme chinois renvoyant au plus bas degré de la hiérarchie féodale: le chef de hameau, \zh{把事}. Ce nom est propre à une seule famille de Yongning.} \zh{杨把事。这个姓,由两部分组成的:‘杨’姓(汉语借词)与封建社会最小领导层次:‘把事’。}  ¶ \textcolor{darkblue}{\textbf{\ipa{jɤ˩pæ˧sɯ˧=ɻ̍˥\$}}} \textcolor{PineGreen}{\selectlanguage{french}\string_ \mytextsc{associatif}: les gens de la famille Petit Chef Yang} \zh{杨把事家族}  
 ¶ \textcolor{darkblue}{\textbf{\ipa{jɤ˩pɑ˧sɯ˥ | ʈæ˧ʂɯ˧}}} \textcolor{PineGreen}{\selectlanguage{french}nom propre d'une personne de la famille Petit Chef Yang: 'Dashi de la famille Petit Chef Yang'.} \zh{杨把事家的一个人的名字:杨把事•达石}  
\textit{Voir~:} \hyperlink{}{\textcolor{darkblue}{\textbf{\ipa{pæ˧sɯ˧}}}} 
\lhead{\firstmark}
\rhead{\botmark}

\subsection{\hspace{-0.5cm} {\Large \textcolor{darkblue}{\textbf{\ipa{jɤ˩po˧}}}}\hspace{0.5cm}[\kern2pt{\textcolor{darkblue}{\textbf{\ipa{jɤ˩po˥}}}}\kern2pt]} \hypertarget{j7\string_Bpo\string_M1}{}
\markboth{\textcolor{darkblue}{\textbf{\ipa{jɤ˩po˧}}}}{}
\textcolor{teal}{\mytextsc{verbe}} \hspace{4pt} Ton~: LM.
\textcolor{PineGreen}{\selectlanguage{french}Parier.} \textcolor{PineGreen}{\selectlanguage{french}Parier.} \zh{赌博、打赌。} 
\lhead{\firstmark}
\rhead{\botmark}

\subsection{\hspace{-0.5cm} {\Large \textcolor{darkblue}{\textbf{\ipa{jɤ˩tʰi˧-ʁæ˩bæ˩}}}}\hspace{0.5cm}[\kern2pt{\textcolor{darkblue}{\textbf{\ipa{jɤ˩tʰi˧ʁæ˩bæ˩}}}}\kern2pt]} \hypertarget{j7\string_Bt\string_hi\string_M-R\{\string_Bb\{\string_B1}{}
\markboth{\textcolor{darkblue}{\textbf{\ipa{jɤ˩tʰi˧-ʁæ˩bæ˩}}}}{}
\textcolor{teal}{\mytextsc{nom}} \hspace{4pt} Ton~: LM-L.
\textcolor{PineGreen}{\selectlanguage{french}Assiette en faïence/porcelaine.} \zh{瓷盘。}  \zh{量词}: \textcolor{darkblue}{\textbf{\ipa{ɭɯ˧}}}  \mytextsc{clf}~: \textcolor{darkblue}{\textbf{\ipa{ɭɯ˧}}} 
\lhead{\firstmark}
\rhead{\botmark}

\subsection{\hspace{-0.5cm} {\Large \textcolor{darkblue}{\textbf{\ipa{jɤ˧˥}}} \textsubscript{1}}\hspace{0.5cm}[\kern2pt{\textcolor{darkblue}{\textbf{\ipa{jɤ˧˥}}}}\kern2pt]} \hypertarget{j7\string_M\string_T1}{}
\markboth{\textcolor{darkblue}{\textbf{\ipa{jɤ˧˥}}} \textsubscript{1}}{}
\textcolor{teal}{\mytextsc{verbe}} \hspace{4pt} Ton~: MH.
\textcolor{PineGreen}{\selectlanguage{french}Lécher.} \zh{舔。}  ¶ \textcolor{darkblue}{\textbf{\ipa{tso˧\textasciitilde{}tso˧ jɤ˩}}} \textcolor{PineGreen}{\selectlanguage{french}lécher quelque chose} \zh{舔东西}  
 ¶ \textcolor{darkblue}{\textbf{\ipa{dzɯ˧-di˧ jɤ˥}}} \textcolor{PineGreen}{\selectlanguage{french}lécher de la nourriture} \zh{舔食品}  
 ¶ \textcolor{darkblue}{\textbf{\ipa{[F5] tso˧tso˧ ɖɯ˧-kʰwɤ˥ jɤ˩-ze˩}}} \textcolor{PineGreen}{\selectlanguage{french}(elle/il) a léché quelque chose} \zh{他舔了一个东西。}  

\lhead{\firstmark}
\rhead{\botmark}

\subsection{\hspace{-0.5cm} {\Large \textcolor{darkblue}{\textbf{\ipa{jɤ˧˥}}} \textsubscript{2}}\hspace{0.5cm}[\kern2pt{\textcolor{darkblue}{\textbf{\ipa{jɤ˧˥}}}}\kern2pt]} \hypertarget{j7\string_M\string_T2}{}
\markboth{\textcolor{darkblue}{\textbf{\ipa{jɤ˧˥}}} \textsubscript{2}}{}
\textcolor{teal}{\mytextsc{nom}} \hspace{4pt} Ton~: MH.
\textcolor{PineGreen}{\selectlanguage{french}Radis sauvage qui pousse en montagne; on le consomme surtout au printemps, à une époque où il n'y a pas encore de légumes. Ce radis est récoltée par les Yi et vendu dans la plaine.} \zh{红萝卜菜:一种山上的野菜。春天的时候,菜园的蔬菜还没有成熟的时候,永宁的人吃红萝卜菜。彝族在高山上采下来,在永宁卖。} Dialecte chinois local~:\zh{野山菜,\textcolor{darkblue}{\textbf{\ipa{/ʝi˧ʂæ˧tsʰɤ˩/}}}。} ¶ \textcolor{darkblue}{\textbf{\ipa{jɤ˧ dzɯ˧ | qʰɑ˧-sɯ˥\textasciitilde{}sɯ˩, | jɤ˧ ʈʂɤ˥ ŋv̩˩-ɭɯ˩\textasciitilde{}ɭɯ˩!}}} \textcolor{PineGreen}{\selectlanguage{french}“Le radis sauvage, ça a un goût amer quand on le mange, et ça vous fait pleurer pour le récolter!” (de: qʰɑ˧ 'amer'+expressif) / “récolter le radis sauvage, ça fait pleurer!” Non pas à cause de la plante elle-même, pas comme un oignon qui piquerait les yeux: mais parce qu'on s'épuisait à aller le chercher en haute montagne.} \zh{“红萝卜菜,味道苦,去摘也要流眼泪! / 红萝卜菜,吃起来苦,摘起来也苦!”摘红萝卜菜,需要爬高山,寻找时间长,永宁坝子的农民觉得这比较苦。}  
\textit{Voir~:} \hyperlink{}{\textcolor{darkblue}{\textbf{\ipa{ʝi˧ʂæ˧tsʰɤ˩}}}} 
\lhead{\firstmark}
\rhead{\botmark}

\subsection{\hspace{-0.5cm} {\Large \textcolor{darkblue}{\textbf{\ipa{jɤ˧˥}}} \textsubscript{3}}\hspace{0.5cm}[\kern2pt{\textcolor{darkblue}{\textbf{\ipa{jɤ˧˥}}}}\kern2pt]} \hypertarget{j7\string_M\string_T3}{}
\markboth{\textcolor{darkblue}{\textbf{\ipa{jɤ˧˥}}} \textsubscript{3}}{}
\textcolor{teal}{\mytextsc{verbe}} \hspace{4pt} Ton~: MH.
\textcolor{PineGreen}{\selectlanguage{french}Étendre, appliquer, mettre (ex.: appliquer un onguent).} \zh{抹、涂抹。}  ¶ \textcolor{darkblue}{\textbf{\ipa{pʰv˧ʂɯ˧ jɤ˧˥}}} \textcolor{PineGreen}{\selectlanguage{french}appliquer une crème de beauté ou de la crème solaire} \zh{抹防晒霜}  
 ¶ \textcolor{darkblue}{\textbf{\ipa{mɤ˩ jɤ˩˥}}} \textcolor{PineGreen}{\selectlanguage{french}appliquer de la graisse (ex.: sur une peau sèche)} \zh{涂抹油}  
 ¶ \textcolor{darkblue}{\textbf{\ipa{tʰi˧-jɤ˧˥}}} \textcolor{PineGreen}{\selectlanguage{french}\mytextsc{dur} \string_}  

\lhead{\firstmark}
\rhead{\botmark}

\subsection{\hspace{-0.5cm} {\Large \textcolor{darkblue}{\textbf{\ipa{jɤ˧˥\textsubscript{a}}}} \textsubscript{1}}\hspace{0.5cm}[\kern2pt{\textcolor{darkblue}{\textbf{\ipa{jɤ˧˥}}}}\kern2pt]} \hypertarget{j7\string_M\string_Ta1}{}
\markboth{\textcolor{darkblue}{\textbf{\ipa{jɤ˧˥\textsubscript{a}}}} \textsubscript{1}}{}
\textcolor{teal}{\mytextsc{classificateur}} \hspace{4pt} Ton~: MH\textsubscript{a}.
\textcolor{PineGreen}{\selectlanguage{french}Classificateur des créatures femelles; employé pour les personnes de sexe féminin (appellation qui ne marque pas de respect, mais n'est pas injurieuse), et pour certains animaux domestiques.} \zh{量词:母性、雌性(人或动物)(一个/一只)。} 
\lhead{\firstmark}
\rhead{\botmark}

\subsection{\hspace{-0.5cm} {\Large \textcolor{darkblue}{\textbf{\ipa{jɤ˧˥\textsubscript{a}}}} \textsubscript{2}}\hspace{0.5cm}[\kern2pt{\textcolor{darkblue}{\textbf{\ipa{jɤ˧˥}}}}\kern2pt]} \hypertarget{j7\string_M\string_Ta2}{}
\markboth{\textcolor{darkblue}{\textbf{\ipa{jɤ˧˥\textsubscript{a}}}} \textsubscript{2}}{}
\textcolor{teal}{\mytextsc{classificateur}} \hspace{4pt} Ton~: MH\textsubscript{a}.
\textcolor{PineGreen}{\selectlanguage{french}Classificateur pour la pâte à pain: quantité de pâte aux œufs que l'on peut préparer avec un œuf. Ce classificateur est également utilisé pour le thé tassé en galettes.} \zh{量词:面(一团),茶饼(一个)等。(一团面,是和了一个鸡蛋的面团的量。)。}  ¶ \textcolor{darkblue}{\textbf{\ipa{æ˩ʁv̩˩-pɤ˥jɤ˩ | ɖɯ˧-jɤ˧˥}}} \textcolor{PineGreen}{\selectlanguage{french}une boule de pâte à pain à l'oeuf} \zh{一个鸡蛋面团}  
 ¶ \textcolor{darkblue}{\textbf{\ipa{ʝi˧-jɤ˧˥}}} \textcolor{PineGreen}{\selectlanguage{french}une boule/galette} \zh{一个团/并}  

\lhead{\firstmark}
\rhead{\botmark}

\subsection{\hspace{-0.5cm} {\Large \textcolor{darkblue}{\textbf{\ipa{jo˥}}}}\hspace{0.5cm}[\kern2pt{\textcolor{darkblue}{\textbf{\ipa{jo˥}}}}\kern2pt]} \hypertarget{jo\string_T1}{}
\markboth{\textcolor{darkblue}{\textbf{\ipa{jo˥}}}}{}
\textcolor{teal}{\mytextsc{nom}} \hspace{4pt} Ton~: \#H.
\textcolor{PineGreen}{\selectlanguage{french}Jade (matière, pierre).} \zh{玉石。}  Emprunt~: chinois  \zh{玉}
 \zh{量词}: \textcolor{darkblue}{\textbf{\ipa{pʰo˧˥}}}  \mytextsc{clf}~: \textcolor{darkblue}{\textbf{\ipa{pʰo˧˥}}} 
\lhead{\firstmark}
\rhead{\botmark}

\subsection{\hspace{-0.5cm} {\Large \textcolor{darkblue}{\textbf{\ipa{jo˧}}}}\hspace{0.5cm}[\kern2pt{\textcolor{darkblue}{\textbf{\ipa{jo˥}}}}\kern2pt]} \hypertarget{jo\string_M1}{}
\markboth{\textcolor{darkblue}{\textbf{\ipa{jo˧}}}}{}
\textcolor{teal}{\mytextsc{verbe}} \hspace{4pt} Ton~: M intrans.
\textcolor{PineGreen}{\selectlanguage{french}Venir, entrer.} \zh{来。}  ¶ \textcolor{darkblue}{\textbf{\ipa{le˧-jo˧-ze˧!}}} \textcolor{PineGreen}{\selectlanguage{french}\mytextsc{accomp} \string_ \mytextsc{pfv}: (il/est) est arrivé(e) / est entré(e)!} \zh{来了!}  

\lhead{\firstmark}
\rhead{\botmark}

\subsection{\hspace{-0.5cm} {\Large \textcolor{darkblue}{\textbf{\ipa{jo˧gv̩˧}}}}\hspace{0.5cm}[\kern2pt{\textcolor{darkblue}{\textbf{\ipa{jo˧gv̩˧}}}}\kern2pt]} \hypertarget{jo\string_Mgv\string_=\string_M1}{}
\markboth{\textcolor{darkblue}{\textbf{\ipa{jo˧gv̩˧}}}}{}
\textcolor{teal}{\mytextsc{nom}} \hspace{4pt} Ton~: M.
\textcolor{PineGreen}{\selectlanguage{french}Lijiang (toute la région: la ville, et la plaine environnante).} \zh{丽江(包括丽江坝子)。} 
\lhead{\firstmark}
\rhead{\botmark}

\subsection{\hspace{-0.5cm} {\Large \textcolor{darkblue}{\textbf{\ipa{jo˧gv̩˧-ŋv̩˧lv̩˧}}}}\hspace{0.5cm}[\kern2pt{\textcolor{darkblue}{\textbf{\ipa{xxxx non-correspondance entre le nombre de morphèmes et le nombre de tons de morphèmes}}}}\kern2pt]} \hypertarget{jo\string_Mgv\string_=\string_M-Nv\string_=\string_Mlv\string_=\string_M1}{}
\markboth{\textcolor{darkblue}{\textbf{\ipa{jo˧gv̩˧-ŋv̩˧lv̩˧}}}}{}
\textcolor{teal}{\mytextsc{nom}} \hspace{4pt} Ton~: M.
\textcolor{PineGreen}{\selectlanguage{french}La montagne Yulong: principale montagne de Lijiang.} \zh{玉龙雪山。} 
\lhead{\firstmark}
\rhead{\botmark}

\subsection{\hspace{-0.5cm} {\Large \textcolor{darkblue}{\textbf{\ipa{jo˧mi˧}}}}\hspace{0.5cm}[\kern2pt{\textcolor{darkblue}{\textbf{\ipa{jo˧mi˧}}}}\kern2pt]} \hypertarget{jo\string_Mmi\string_M1}{}
\markboth{\textcolor{darkblue}{\textbf{\ipa{jo˧mi˧}}}}{}
\textcolor{teal}{\mytextsc{nom}} \hspace{4pt} Ton~: M.
\textcolor{PineGreen}{\selectlanguage{french}Brebis.} \zh{母绵羊。}  ¶ \textcolor{darkblue}{\textbf{\ipa{jo˧mi˧-po˧lo˧}}} \textcolor{PineGreen}{\selectlanguage{french}brebis et bélier} \zh{母绵羊与公羊}  
 \zh{量词}: \textcolor{darkblue}{\textbf{\ipa{pʰo˧˥}}}  \mytextsc{clf}~: \textcolor{darkblue}{\textbf{\ipa{pʰo˧˥}}} 
\lhead{\firstmark}
\rhead{\botmark}

\subsection{\hspace{-0.5cm} {\Large \textcolor{darkblue}{\textbf{\ipa{jo˧mi˧-ʁwɤ˧}}}}\hspace{0.5cm}[\kern2pt{\textcolor{darkblue}{\textbf{\ipa{xxxx non-correspondance entre le nombre de morphèmes et le nombre de tons de morphèmes}}}}\kern2pt]} \hypertarget{jo\string_Mmi\string_M-Rw7\string_M1}{}
\markboth{\textcolor{darkblue}{\textbf{\ipa{jo˧mi˧-ʁwɤ˧}}}}{}
\textcolor{teal}{\mytextsc{nom}} \hspace{4pt} Ton~: M.
\textcolor{PineGreen}{\selectlanguage{french}Le second village que l'on rencontre sur le trajet entre \textcolor{darkblue}{\textbf{\ipa{/qʰæ˧tɕʰi˧/}}} et \textcolor{darkblue}{\textbf{\ipa{/ʈʂo˧ʂɯ\#˥/}}}.} \zh{有米瓦村。} 
\lhead{\firstmark}
\rhead{\botmark}

\subsection{\hspace{-0.5cm} {\Large \textcolor{darkblue}{\textbf{\ipa{jo˩}}}}\hspace{0.5cm}[\kern2pt{\textcolor{darkblue}{\textbf{\ipa{jo˥}}}}\kern2pt]} \hypertarget{jo\string_B1}{}
\markboth{\textcolor{darkblue}{\textbf{\ipa{jo˩}}}}{}
\textcolor{teal}{\mytextsc{nom}} \hspace{4pt} Ton~: L.
\textcolor{PineGreen}{\selectlanguage{french}Mouton.} \zh{绵羊。}  ¶ \textcolor{darkblue}{\textbf{\ipa{jo˩-ɣɯ˩˥}}} \textcolor{PineGreen}{\selectlanguage{french}peau de mouton} \zh{羊皮}  
 \zh{量词}: \textcolor{darkblue}{\textbf{\ipa{pʰo˧˥}}}  \mytextsc{clf}~: \textcolor{darkblue}{\textbf{\ipa{pʰo˧˥}}} 
\lhead{\firstmark}
\rhead{\botmark}

\subsection{\hspace{-0.5cm} {\Large \textcolor{darkblue}{\textbf{\ipa{jo˩\textsubscript{b}}}}}\hspace{0.5cm}[\kern2pt{\textcolor{darkblue}{\textbf{\ipa{jo˩˥}}}}\kern2pt]} \hypertarget{jo\string_Bb1}{}
\markboth{\textcolor{darkblue}{\textbf{\ipa{jo˩\textsubscript{b}}}}}{}
\textcolor{teal}{\mytextsc{classificateur}} \hspace{4pt} Ton~: L\textsubscript{b}.
\textcolor{PineGreen}{\selectlanguage{french}Unité de poids: once.} \zh{量词:两(一两)。} 
\lhead{\firstmark}
\rhead{\botmark}

\subsection{\hspace{-0.5cm} {\Large \textcolor{darkblue}{\textbf{\ipa{jo˩gi˩}}}}\hspace{0.5cm}[\kern2pt{\textcolor{darkblue}{\textbf{\ipa{jo˩gi˩˥}}}}\kern2pt]} \hypertarget{jo\string_Bgi\string_B1}{}
\markboth{\textcolor{darkblue}{\textbf{\ipa{jo˩gi˩}}}}{}
\textcolor{teal}{\mytextsc{nom}} \hspace{4pt} Ton~: L.
\textcolor{PineGreen}{\selectlanguage{french}Droite (contraire de: gauche).} \zh{右边。}  ¶ \textcolor{darkblue}{\textbf{\ipa{jo˩gi˩dzɤ˩}}} \textcolor{PineGreen}{\selectlanguage{french}du côté droit, à droite} \zh{右边}  
\textit{Voir~:} \hyperlink{}{\textcolor{darkblue}{\textbf{\ipa{jo˩˧}}}} 
\lhead{\firstmark}
\rhead{\botmark}

\subsection{\hspace{-0.5cm} {\Large \textcolor{darkblue}{\textbf{\ipa{jo˩kʰv̩˩}}}}\hspace{0.5cm}[\kern2pt{\textcolor{darkblue}{\textbf{\ipa{jo˩kʰv̩˩˥}}}}\kern2pt]} \hypertarget{jo\string_Bk\string_hv\string_=\string_B1}{}
\markboth{\textcolor{darkblue}{\textbf{\ipa{jo˩kʰv̩˩}}}}{}
\textcolor{teal}{\mytextsc{nom}} \hspace{4pt} Ton~: L.
\textcolor{PineGreen}{\selectlanguage{french}Année du mouton.} \zh{羊年。} 
\lhead{\firstmark}
\rhead{\botmark}

\subsection{\hspace{-0.5cm} {\Large \textcolor{darkblue}{\textbf{\ipa{jo˩lo˩}}}}\hspace{0.5cm}[\kern2pt{\textcolor{darkblue}{\textbf{\ipa{jo˩lo˩˥}}}}\kern2pt]} \hypertarget{jo\string_Blo\string_B1}{}
\markboth{\textcolor{darkblue}{\textbf{\ipa{jo˩lo˩}}}}{}
\textcolor{teal}{\mytextsc{nom}} \hspace{4pt} Ton~: L.
\textcolor{PineGreen}{\selectlanguage{french}Droite (contraire de: gauche).} \zh{右边。} \textit{Voir~:} \hyperlink{}{\textcolor{darkblue}{\textbf{\ipa{jo˩˧}}}} 
\lhead{\firstmark}
\rhead{\botmark}

\subsection{\hspace{-0.5cm} {\Large \textcolor{darkblue}{\textbf{\ipa{jo˩pv̩˧}}}}\hspace{0.5cm}[\kern2pt{\textcolor{darkblue}{\textbf{\ipa{jo˩pv̩˥}}}}\kern2pt]} \hypertarget{jo\string_Bpv\string_=\string_M1}{}
\markboth{\textcolor{darkblue}{\textbf{\ipa{jo˩pv̩˧}}}}{}
\textcolor{teal}{\mytextsc{nom}} \hspace{4pt} Ton~: LM / LM+MH\#.
\textcolor{PineGreen}{\selectlanguage{french}Toile cirée.} \zh{油布。}  Emprunt~: chinois  \zh{油布}
 ¶ \textcolor{darkblue}{\textbf{\ipa{jo˩pv̩˧˥}}} \textcolor{PineGreen}{\selectlanguage{french}toile cirée (variante tonale)} \zh{油布(声调变体)}  
 \zh{量词}: \textcolor{darkblue}{\textbf{\ipa{tsʰi˥}}}  \mytextsc{clf}~: \textcolor{darkblue}{\textbf{\ipa{tsʰi˥}}} 
\lhead{\firstmark}
\rhead{\botmark}

\subsection{\hspace{-0.5cm} {\Large \textcolor{darkblue}{\textbf{\ipa{jo˩pʰv̩˩}}}}\hspace{0.5cm}[\kern2pt{\textcolor{darkblue}{\textbf{\ipa{jo˩pʰv̩˩˥}}}}\kern2pt]} \hypertarget{jo\string_Bp\string_hv\string_=\string_B1}{}
\markboth{\textcolor{darkblue}{\textbf{\ipa{jo˩pʰv̩˩}}}}{}
\textcolor{teal}{\mytextsc{nom}} \hspace{4pt} Ton~: L.
\textcolor{PineGreen}{\selectlanguage{french}Bélier.} \zh{公绵羊。}  ¶ \textcolor{darkblue}{\textbf{\ipa{jo˧pʰv̩˧ tʰv̩˧-mi˥\#}}} \textcolor{PineGreen}{\selectlanguage{french}\string_ \mytextsc{dem} \mytextsc{clf}: ce bélier} \zh{这头公羊}  
 \zh{量词}: \textcolor{darkblue}{\textbf{\ipa{pʰo˧˥}}} \textcolor{darkblue}{\textbf{\ipa{mi˩}}}  \mytextsc{clf}~: \textcolor{darkblue}{\textbf{\ipa{pʰo˧˥}}} \textcolor{darkblue}{\textbf{\ipa{mi˩}}} \textit{Voir~:} \hyperlink{}{\textcolor{darkblue}{\textbf{\ipa{po˧lo˧}}}} 
\lhead{\firstmark}
\rhead{\botmark}

\subsection{\hspace{-0.5cm} {\Large \textcolor{darkblue}{\textbf{\ipa{jo˩ʂwæ˩}}}}\hspace{0.5cm}[\kern2pt{\textcolor{darkblue}{\textbf{\ipa{jo˩ʂwæ˩˥}}}}\kern2pt]} \hypertarget{jo\string_Bs`w\{\string_B1}{}
\markboth{\textcolor{darkblue}{\textbf{\ipa{jo˩ʂwæ˩}}}}{}
\textcolor{teal}{\mytextsc{nom}} \hspace{4pt} Ton~: L.
\textcolor{PineGreen}{\selectlanguage{french}Bélier châtré.} \zh{阉羊。}  \zh{量词}: \textcolor{darkblue}{\textbf{\ipa{pʰo˧˥}}}  \mytextsc{clf}~: \textcolor{darkblue}{\textbf{\ipa{pʰo˧˥}}} 
\lhead{\firstmark}
\rhead{\botmark}

\subsection{\hspace{-0.5cm} {\Large \textcolor{darkblue}{\textbf{\ipa{jo˩zo˩}}}}\hspace{0.5cm}[\kern2pt{\textcolor{darkblue}{\textbf{\ipa{jo˩zo˩˥}}}}\kern2pt]} \hypertarget{jo\string_Bzo\string_B1}{}
\markboth{\textcolor{darkblue}{\textbf{\ipa{jo˩zo˩}}}}{}
\textcolor{teal}{\mytextsc{nom}} \hspace{4pt} Ton~: L.
\textcolor{PineGreen}{\selectlanguage{french}Agneau.} \zh{绵羊羔。}  \zh{量词}: \textcolor{darkblue}{\textbf{\ipa{ɭɯ˧}}}  \mytextsc{clf}~: \textcolor{darkblue}{\textbf{\ipa{ɭɯ˧}}} 
\lhead{\firstmark}
\rhead{\botmark}

\subsection{\hspace{-0.5cm} {\Large \textcolor{darkblue}{\textbf{\ipa{jo˧˥}}}}\hspace{0.5cm}[\kern2pt{\textcolor{darkblue}{\textbf{\ipa{jo˧˥}}}}\kern2pt]} \hypertarget{jo\string_M\string_T1}{}
\markboth{\textcolor{darkblue}{\textbf{\ipa{jo˧˥}}}}{}
\textcolor{teal}{\mytextsc{verbe}} \hspace{4pt} Ton~: MH.
\textcolor{PineGreen}{\selectlanguage{french}Offrir.} \zh{赠给。} 
\lhead{\firstmark}
\rhead{\botmark}

\subsection{\hspace{-0.5cm} {\Large \textcolor{darkblue}{\textbf{\ipa{jo˩˧}}}}\hspace{0.5cm}[\kern2pt{\textcolor{darkblue}{\textbf{\ipa{jo˩˥}}}}\kern2pt]} \hypertarget{jo\string_B\string_M1}{}
\markboth{\textcolor{darkblue}{\textbf{\ipa{jo˩˧}}}}{}
\textcolor{teal}{\mytextsc{nom}} \hspace{4pt} Ton~: LM.
\textcolor{PineGreen}{\selectlanguage{french}Droite (opposé de: gauche).} \zh{右边。} \textit{Voir~:} \hyperlink{}{\textcolor{darkblue}{\textbf{\ipa{jo˩gi˩}}}} 
\lhead{\firstmark}
\rhead{\botmark}

\subsection{\hspace{-0.5cm} {\Large \textcolor{darkblue}{\textbf{\ipa{je˧pʰi˧-jɤ\#˥}}}}\hspace{0.5cm}[\kern2pt{\textcolor{darkblue}{\textbf{\ipa{xxxx non-correspondance entre le nombre de morphèmes et le nombre de tons de morphèmes}}}}\kern2pt]} \hypertarget{je\string_Mp\string_hi\string_M-j7\#\string_T1}{}
\markboth{\textcolor{darkblue}{\textbf{\ipa{je˧pʰi˧-jɤ\#˥}}}}{}
\textcolor{teal}{\mytextsc{nom}} \hspace{4pt} Ton~: \#H.
\textit{De:} \textbf{\zh{鸦片} et jɤ˧} \textcolor{PineGreen}{\selectlanguage{french}Opium.} \zh{鸦片(汉语借词)。}  Emprunt~: chinois  \zh{鸦片}

\lhead{\firstmark}
\rhead{\botmark}

\subsection{\hspace{-0.5cm} {\Large \textcolor{darkblue}{\textbf{\ipa{je˩ʐe˧}}}}\hspace{0.5cm}[\kern2pt{\textcolor{darkblue}{\textbf{\ipa{je˧ʐe˧}}}}\kern2pt]} \hypertarget{je\string_Bz`e\string_M1}{}
\markboth{\textcolor{darkblue}{\textbf{\ipa{je˩ʐe˧}}}}{}
\textcolor{teal}{\mytextsc{nom}} \hspace{4pt} Ton~: LM.
\textcolor{PineGreen}{\selectlanguage{french}Occidental.} \zh{西方人(“洋人”)(汉语借词)。} Dialecte chinois local~:\zh{洋人。} Emprunt~: chinois  \zh{洋人}
 \zh{量词}: \textcolor{darkblue}{\textbf{\ipa{v̩˧}}}  \mytextsc{clf}~: \textcolor{darkblue}{\textbf{\ipa{v̩˧}}} 
\lhead{\firstmark}
\rhead{\botmark}

\newpage
\section*{\centering- \textcolor{darkblue}{\textbf{\ipa{ʝ}}} -}
\subsection{\hspace{-0.5cm} {\Large \textcolor{darkblue}{\textbf{\ipa{ʝi˥}}} \textsubscript{1}}\hspace{0.5cm}[\kern2pt{\textcolor{darkblue}{\textbf{\ipa{ʝi˥}}}}\kern2pt]} \hypertarget{j££i\string_T1}{}
\markboth{\textcolor{darkblue}{\textbf{\ipa{ʝi˥}}} \textsubscript{1}}{}
\textcolor{teal}{\mytextsc{nom}} \hspace{4pt} Ton~: \#H.
\textcolor{PineGreen}{\selectlanguage{french}Vache, boeuf.} \zh{牛。}  ¶ \textcolor{darkblue}{\textbf{\ipa{ʝi˧-ɣɯ˥}}} \textcolor{PineGreen}{\selectlanguage{french}peau de vache} \zh{牛皮}  
 ¶ \textcolor{darkblue}{\textbf{\ipa{ʝi˧ tʰv̩˧-pʰo˩}}} \textcolor{PineGreen}{\selectlanguage{french}\mytextsc{n}+\mytextsc{dem}+\mytextsc{clf}} \zh{那头牛}  
 \zh{量词}: \textcolor{darkblue}{\textbf{\ipa{pʰo˧˥}}}  \mytextsc{clf}~: \textcolor{darkblue}{\textbf{\ipa{pʰo˧˥}}} 
\lhead{\firstmark}
\rhead{\botmark}

\subsection{\hspace{-0.5cm} {\Large \textcolor{darkblue}{\textbf{\ipa{ʝi˥}}} \textsubscript{2}}\hspace{0.5cm}[\kern2pt{\textcolor{darkblue}{\textbf{\ipa{ʝi˥}}}}\kern2pt]} \hypertarget{j££i\string_T2}{}
\markboth{\textcolor{darkblue}{\textbf{\ipa{ʝi˥}}} \textsubscript{2}}{}
\textcolor{teal}{\mytextsc{verbe}} \hspace{4pt} Ton~: H.
\textcolor{PineGreen}{\selectlanguage{french}Travailler, faire.} \zh{做,工作。}  ¶ \textcolor{darkblue}{\textbf{\ipa{ɖwæ˧˥ | lo˧ ʝi˧}}} \textcolor{PineGreen}{\selectlanguage{french}travailleur, assidu, qui travaille beaucoup} \zh{勤劳、努力}  
 ¶ \textcolor{darkblue}{\textbf{\ipa{ɖɯ˧-sɑ˥ | mɤ˧-ʝi˥}}} \textcolor{PineGreen}{\selectlanguage{french}ne rien faire du tout} \zh{什么也不干}  
 ¶ \textcolor{darkblue}{\textbf{\ipa{ə˧tso˧-mɤ˧-ɲi˩ | ʝi˧-bi˧-zo˧-ho˥!}}} \textcolor{PineGreen}{\selectlanguage{french}Il faut tout faire! / On va devoir m'occuper de tout!} \zh{什么都要做! / 我什么都要干(/管)!}  
 ¶ \textcolor{darkblue}{\textbf{\ipa{ʈʂʰɯ˧ne-ʝi˥ | ʝi˧-zo˧-ho˥-ɲi˩!}}} \textcolor{PineGreen}{\selectlanguage{french}Voilà comment il faut faire! / C'est comme ça qu'on fait!} \zh{应该这样做的!}  
 ¶ \textcolor{darkblue}{\textbf{\ipa{ɑ˩ʁo˧ ʝi˧}}} \textcolor{PineGreen}{\selectlanguage{french}gérer la maisonnée, s'occuper de la famille (tâche de la personne qui répartit les travaux à accomplir, veille aux approvisionnements...)} \zh{管理家里的大小事情(如:分配工作、家务等)}  

\lhead{\firstmark}
\rhead{\botmark}

\subsection{\hspace{-0.5cm} {\Large \textcolor{darkblue}{\textbf{\ipa{ʝi˥}}} \textsubscript{3}}\hspace{0.5cm}[\kern2pt{\textcolor{darkblue}{\textbf{\ipa{ʝi˥}}}}\kern2pt]} \hypertarget{j££i\string_T3}{}
\markboth{\textcolor{darkblue}{\textbf{\ipa{ʝi˥}}} \textsubscript{3}}{}
\textcolor{teal}{\mytextsc{verbe}} \hspace{4pt} Ton~: H.
\textcolor{PineGreen}{\selectlanguage{french}Dessiner, tracer.} \zh{画。}  ¶ \textcolor{darkblue}{\textbf{\ipa{mɤ˧-ʝi˥}}} \textcolor{PineGreen}{\selectlanguage{french}\mytextsc{neg}} \zh{不画}  
 ¶ \textcolor{darkblue}{\textbf{\ipa{tʰɑ˧-ʝi˥!}}} \textcolor{PineGreen}{\selectlanguage{french}\mytextsc{prohib}} \zh{别画!}  
 ¶ \textcolor{darkblue}{\textbf{\ipa{ʈʂɑ˧tɑ˥ ʝi˩}}} \textcolor{PineGreen}{\selectlanguage{french}tracer un signe, dessiner un signe} \zh{画一个符号}  

\lhead{\firstmark}
\rhead{\botmark}

\subsection{\hspace{-0.5cm} {\Large \textcolor{darkblue}{\textbf{\ipa{ʝi˥}}} \textsubscript{4}}\hspace{0.5cm}[\kern2pt{\textcolor{darkblue}{\textbf{\ipa{ʝi˥}}}}\kern2pt]} \hypertarget{j££i\string_T4}{}
\markboth{\textcolor{darkblue}{\textbf{\ipa{ʝi˥}}} \textsubscript{4}}{}
\textcolor{teal}{\mytextsc{nom}} \hspace{4pt} Ton~: \#H.
\textcolor{PineGreen}{\selectlanguage{french}Jarre en terre cuite.} \zh{坛子,罐子 (陶器)。}  \zh{量词}: \textcolor{darkblue}{\textbf{\ipa{ɭɯ˧}}}  \mytextsc{clf}~: \textcolor{darkblue}{\textbf{\ipa{ɭɯ˧}}} 
\lhead{\firstmark}
\rhead{\botmark}

\subsection{\hspace{-0.5cm} {\Large \textcolor{darkblue}{\textbf{\ipa{ʝi˥}}} \textsubscript{5}}\hspace{0.5cm}[\kern2pt{\textcolor{darkblue}{\textbf{\ipa{ʝi˥}}}}\kern2pt]} \hypertarget{j££i\string_T5}{}
\markboth{\textcolor{darkblue}{\textbf{\ipa{ʝi˥}}} \textsubscript{5}}{}
\textcolor{teal}{\mytextsc{verbe}} \hspace{4pt} Ton~: H.
\textcolor{PineGreen}{\selectlanguage{french}Informer.} \zh{通知、告诉。}  ¶ \textcolor{darkblue}{\textbf{\ipa{le˧-ʝi˥-ze˩}}} \textcolor{PineGreen}{\selectlanguage{french}\mytextsc{accomp} \string_ \mytextsc{pfv}} \zh{通知了}  
 ¶ \textcolor{darkblue}{\textbf{\ipa{qʰwæ˧ mi˧ ʝi˧}}} \textcolor{PineGreen}{\selectlanguage{french}donner une nouvelle} \zh{告诉(一个)消息}  
 ¶ \textcolor{darkblue}{\textbf{\ipa{njɤ˧ | hĩ˧-ki˧ | qʰwæ˧mi˧ ʝi˧-ze˩}}} \textcolor{PineGreen}{\selectlanguage{french}j'ai annoncé la nouvelle aux gens / j'ai annoncé une nouvelle à quelqu'un} \zh{我告诉了人家(那个消息)。}  

\lhead{\firstmark}
\rhead{\botmark}

\subsection{\hspace{-0.5cm} {\Large \textcolor{darkblue}{\textbf{\ipa{ʝi˥}}} \textsubscript{6}}\hspace{0.5cm}[\kern2pt{\textcolor{darkblue}{\textbf{\ipa{ʝi˥}}}}\kern2pt]} \hypertarget{j££i\string_T6}{}
\markboth{\textcolor{darkblue}{\textbf{\ipa{ʝi˥}}} \textsubscript{6}}{}
\textcolor{teal}{\mytextsc{nom}} \hspace{4pt} Ton~: \#H.
\textit{\textcolor{PineGreen}{\selectlanguage{french}archaïque}} [\zh{古语}] \textcolor{PineGreen}{\selectlanguage{french}Homme \textit{(vir)}.} \zh{男人。} 
\lhead{\firstmark}
\rhead{\botmark}

\subsection{\hspace{-0.5cm} {\Large \textcolor{darkblue}{\textbf{\ipa{ʝi˥}}} \textsubscript{7}}\hspace{0.5cm}[\kern2pt{\textcolor{darkblue}{\textbf{\ipa{ʝi˥}}}}\kern2pt]} \hypertarget{j££i\string_T7}{}
\markboth{\textcolor{darkblue}{\textbf{\ipa{ʝi˥}}} \textsubscript{7}}{}
\textcolor{teal}{\mytextsc{verbe}} \hspace{4pt} Ton~: H.
\textcolor{PineGreen}{\selectlanguage{french}Verbe d'existence: choses amovibles.} \zh{存在动词:有(可移动物品)。}  ¶ \textcolor{darkblue}{\textbf{\ipa{ə˧tso˧-mɤ˧-ɲi˩, | le˧-ʂe˧, | le˧-ʝi˥!}}} \textcolor{PineGreen}{\selectlanguage{french}(En vue d'un rituel, d'une fête…) on rassemble toutes sortes de choses; on en a (sous la main)/on a fait une provision! / On prépare tout par avance (pour le rituel/la fête)!} \zh{所有(的东西都)找,(就)有了 = 所有的东西都备好了}  

\lhead{\firstmark}
\rhead{\botmark}

\subsection{\hspace{-0.5cm} {\Large \textcolor{darkblue}{\textbf{\ipa{ʝi˧}}} \textsubscript{1}}\hspace{0.5cm}[\kern2pt{\textcolor{darkblue}{\textbf{\ipa{ʝi˩˥}}}}\kern2pt]} \hypertarget{j££i\string_M1}{}
\markboth{\textcolor{darkblue}{\textbf{\ipa{ʝi˧}}} \textsubscript{1}}{}
\textcolor{teal}{\mytextsc{verbe}} \hspace{4pt} Ton~: M\textsubscript{c}.
\textcolor{PineGreen}{\selectlanguage{french}Venir.} \zh{来。}  ¶ \textcolor{darkblue}{\textbf{\ipa{lɑ˧ ʝi˧-ze˧!}}} \textcolor{PineGreen}{\selectlanguage{french}Voilà le tigre! / Un tigre arrive!} \zh{老虎来了!}  
 ¶ \textcolor{darkblue}{\textbf{\ipa{lɑ˧ le˧-ʝi˩-ze˩!}}} \textcolor{PineGreen}{\selectlanguage{french}Voilà le tigre qui revient! / Le tigre est revenu!/ Le tigre est de retour!} \zh{老虎又来了!}  
 ¶ \textcolor{darkblue}{\textbf{\ipa{mɤ˧-ʝi˧-ze˧!}}} \textcolor{PineGreen}{\selectlanguage{french}Ca ne va plus! / C'est la catastrophe!} \zh{不好了!不行了!}  

\lhead{\firstmark}
\rhead{\botmark}

\subsection{\hspace{-0.5cm} {\Large \textcolor{darkblue}{\textbf{\ipa{ʝi˧}}} \textsubscript{2}}\hspace{0.5cm}[\kern2pt{\textcolor{darkblue}{\textbf{\ipa{ʝi˥}}}}\kern2pt]} \hypertarget{j££i\string_M2}{}
\markboth{\textcolor{darkblue}{\textbf{\ipa{ʝi˧}}} \textsubscript{2}}{}
\textcolor{teal}{\mytextsc{nom}} \hspace{4pt} Ton~: M.
\textcolor{PineGreen}{\selectlanguage{french}Un (numéral, à emploi restreint; ne se combine qu'avec le classificateur /ɭɯ˧/).} \zh{一。}  ¶ \textcolor{darkblue}{\textbf{\ipa{zo˧mv̩˥ | ʝi˧-ɭɯ˧ ʂv̩˧}}} \textcolor{PineGreen}{\selectlanguage{french}s'occuper d'un enfant} \zh{管一个孩子}  

\lhead{\firstmark}
\rhead{\botmark}

\subsection{\hspace{-0.5cm} {\Large \textcolor{darkblue}{\textbf{\ipa{ʝi˧\textsubscript{b}}}}}\hspace{0.5cm}[\kern2pt{\textcolor{darkblue}{\textbf{\ipa{ʝi˥}}}}\kern2pt]} \hypertarget{j££i\string_Mb1}{}
\markboth{\textcolor{darkblue}{\textbf{\ipa{ʝi˧\textsubscript{b}}}}}{}
\textcolor{teal}{\mytextsc{classificateur}} \hspace{4pt} Ton~: M\textsubscript{b}.
\textcolor{PineGreen}{\selectlanguage{french}Classificateur des lieux.} \zh{量词:地方(一个)。}  ¶ \textcolor{darkblue}{\textbf{\ipa{ɖɯ˧-ʝi˧}}} \textcolor{PineGreen}{\selectlanguage{french}un endroit; qq part} \zh{一个地方}  
 ¶ \textcolor{darkblue}{\textbf{\ipa{ɖɯ˧-ʝi˧ dzi˩}}} \textcolor{PineGreen}{\selectlanguage{french}habiter quelque part; emménager quelque part/déménager vers quelque part} \zh{住在一个地方,搬家到一个地方}  
 ¶ \textcolor{darkblue}{\textbf{\ipa{ɖɯ˧-v˧ | ɖɯ˧-ʝi˧ hɯ˧}}} \textcolor{PineGreen}{\selectlanguage{french}chacun s'en va de son côté (contexte: les membres d'une famille vont habiter en des lieux différents pour raisons professionnelles)} \zh{个去个的地方!/ 每个人去不同的地方!(情景:由于工作原因,一家的成员经常需要去不同的城市工作。)}  

\lhead{\firstmark}
\rhead{\botmark}

\subsection{\hspace{-0.5cm} {\Large \textcolor{darkblue}{\textbf{\ipa{ʝi˧-bv̩˧˥}}}}\hspace{0.5cm}[\kern2pt{\textcolor{darkblue}{\textbf{\ipa{xxxx non-correspondance entre le nombre de morphèmes et le nombre de tons de morphèmes}}}}\kern2pt]} \hypertarget{j££i\string_M-bv\string_=\string_M\string_T1}{}
\markboth{\textcolor{darkblue}{\textbf{\ipa{ʝi˧-bv̩˧˥}}}}{}
\textcolor{teal}{\mytextsc{nom}} \hspace{4pt} Ton~: MH\#.
\textcolor{PineGreen}{\selectlanguage{french}Étable (des vaches).} \zh{牛圈。}  \zh{量词}: \textcolor{darkblue}{\textbf{\ipa{ɭɯ˧}}}  \mytextsc{clf}~: \textcolor{darkblue}{\textbf{\ipa{ɭɯ˧}}} 
\lhead{\firstmark}
\rhead{\botmark}

\subsection{\hspace{-0.5cm} {\Large \textcolor{darkblue}{\textbf{\ipa{ʝi˧kʰv̩˩}}}}\hspace{0.5cm}[\kern2pt{\textcolor{darkblue}{\textbf{\ipa{ʝi˩kʰv̩˥}}}}\kern2pt]} \hypertarget{j££i\string_Mk\string_hv\string_=\string_B1}{}
\markboth{\textcolor{darkblue}{\textbf{\ipa{ʝi˧kʰv̩˩}}}}{}
\textcolor{teal}{\mytextsc{nom}} \hspace{4pt} Ton~: L\#.
\textcolor{PineGreen}{\selectlanguage{french}Année du bœuf / année du boeuf.} \zh{牛年。} 
\lhead{\firstmark}
\rhead{\botmark}

\subsection{\hspace{-0.5cm} {\Large \textcolor{darkblue}{\textbf{\ipa{ʝi˧kʰv̩˥}}}}\hspace{0.5cm}[\kern2pt{\textcolor{darkblue}{\textbf{\ipa{ʝi˧kʰv̩˩}}}}\kern2pt]} \hypertarget{j££i\string_Mk\string_hv\string_=\string_T1}{}
\markboth{\textcolor{darkblue}{\textbf{\ipa{ʝi˧kʰv̩˥}}}}{}
\textcolor{teal}{\mytextsc{pronom}} \hspace{4pt} Ton~: H\#.
\textcolor{PineGreen}{\selectlanguage{french}Certains.} \zh{一些。}  ¶ \textcolor{darkblue}{\textbf{\ipa{hĩ˧ ʝi˧kʰv̩˥}}} \textcolor{PineGreen}{\selectlanguage{french}certaines personnes, une partie des gens} \zh{一些人}  

\lhead{\firstmark}
\rhead{\botmark}

\subsection{\hspace{-0.5cm} {\Large \textcolor{darkblue}{\textbf{\ipa{ʝi˧kʰwɤ˥\$}}}}\hspace{0.5cm}[\kern2pt{\textcolor{darkblue}{\textbf{\ipa{ʝi˧kʰwɤ˥}}}}\kern2pt]} \hypertarget{j££i\string_Mk\string_hw7\string_T\$1}{}
\markboth{\textcolor{darkblue}{\textbf{\ipa{ʝi˧kʰwɤ˥\$}}}}{}
\textcolor{teal}{\mytextsc{pronom}} \hspace{4pt} Ton~: H\$.
\textcolor{PineGreen}{\selectlanguage{french}Un peu.} \zh{一点。} 
\lhead{\firstmark}
\rhead{\botmark}

\subsection{\hspace{-0.5cm} {\Large \textcolor{darkblue}{\textbf{\ipa{ʝi˧lo\#˥}}}}\hspace{0.5cm}[\kern2pt{\textcolor{darkblue}{\textbf{\ipa{ʝi˧lo˥}}}}\kern2pt]} \hypertarget{j££i\string_Mlo\#\string_T1}{}
\markboth{\textcolor{darkblue}{\textbf{\ipa{ʝi˧lo\#˥}}}}{}
\textcolor{teal}{\mytextsc{nom}} \hspace{4pt} Ton~: \#H.
\textcolor{PineGreen}{\selectlanguage{french}Traitement (d'autrui), attitude.} \zh{态度、对待的态度。}  ¶ \textcolor{darkblue}{\textbf{\ipa{ʝi˧lo˧ dʑɤ˥!}}} \textcolor{PineGreen}{\selectlanguage{french}(Il / elle) a une attitude positive} \zh{态度积极}  
 ¶ \textcolor{darkblue}{\textbf{\ipa{ʈʂʰɯ˧ | ʝi˧lo˧ | dʑɤ˩˥! | hĩ˧-ki˧ | dʑɤ˩-ʝi˥!}}} \textcolor{PineGreen}{\selectlanguage{french}Il/elle traite bien les gens! Il/elle fait de bonnes actions!} \zh{他(对人)态度好!对人好/做好事!}  
 ¶ \textcolor{darkblue}{\textbf{\ipa{ʝi˧lo˧ dzɑ˧}}} \textcolor{PineGreen}{\selectlanguage{french}(avoir une) mauvaise attitude: paresseuse, dissipée…} \zh{态度不好}  
 ¶ \textcolor{darkblue}{\textbf{\ipa{njɤ˧-ɳɯ˧ hɑ˧ gv̩˥, | ʝi˧lo˧ dzɑ˧!}}} \textcolor{PineGreen}{\selectlanguage{french}Quand je fais la cuisine, je ne suis pas bien concentrée/je travaille n'importe comment!} \zh{我做饭,集中不了精神 / 做的乱七八糟!}  
 ¶ \textcolor{darkblue}{\textbf{\ipa{ʈʂʰɯ˧ | ə˧tso˧ ʝi˧lo˧ ɲi˥?}}} \textcolor{PineGreen}{\selectlanguage{french}Qu'est-ce que c'est que cette attitude? (critique adressée à quelqu'un qui fait n'importe quoi)} \zh{这是什么态度啊?(批评一个人的态度)}  

\lhead{\firstmark}
\rhead{\botmark}

\subsection{\hspace{-0.5cm} {\Large \textcolor{darkblue}{\textbf{\ipa{ʝi˧mi˧}}}}\hspace{0.5cm}[\kern2pt{\textcolor{darkblue}{\textbf{\ipa{ʝi˩mi˩˥}}}}\kern2pt]} \hypertarget{j££i\string_Mmi\string_M1}{}
\markboth{\textcolor{darkblue}{\textbf{\ipa{ʝi˧mi˧}}}}{}
\textcolor{teal}{\mytextsc{nom}} \hspace{4pt} Ton~: M.
\textcolor{PineGreen}{\selectlanguage{french}Jarre.} \zh{坛子,罐子 (陶器)。}  \zh{量词}: \textcolor{darkblue}{\textbf{\ipa{ɭɯ˧}}}  \mytextsc{clf}~: \textcolor{darkblue}{\textbf{\ipa{ɭɯ˧}}} 
\lhead{\firstmark}
\rhead{\botmark}

\subsection{\hspace{-0.5cm} {\Large \textcolor{darkblue}{\textbf{\ipa{ʝi˧pʰv̩\#˥}}}}\hspace{0.5cm}[\kern2pt{\textcolor{darkblue}{\textbf{\ipa{ʝi˩pʰv̩˧˥}}}}\kern2pt]} \hypertarget{j££i\string_Mp\string_hv\string_=\#\string_T1}{}
\markboth{\textcolor{darkblue}{\textbf{\ipa{ʝi˧pʰv̩\#˥}}}}{}
\textcolor{teal}{\mytextsc{nom}} \hspace{4pt} Ton~: \#H.
\textcolor{PineGreen}{\selectlanguage{french}Taureau.} \zh{公牛。}  ¶ \textcolor{darkblue}{\textbf{\ipa{ʝi˧pʰv̩˧ tʰv̩˧-mi˥\#}}} \textcolor{PineGreen}{\selectlanguage{french}\mytextsc{n}+\mytextsc{dem}+\mytextsc{clf}} \zh{那头公牛}  
 ¶ \textcolor{darkblue}{\textbf{\ipa{ʝi˧pʰv̩˧ tʰv̩˧-ɭɯ\#˥}}} \textcolor{PineGreen}{\selectlanguage{french}\mytextsc{n}+\mytextsc{dem}+\mytextsc{clf}.animaux} \zh{那头公牛}  
 \zh{量词}: \textcolor{darkblue}{\textbf{\ipa{ɭɯ˧ / mi˩}}}  \mytextsc{clf}~: \textcolor{darkblue}{\textbf{\ipa{ɭɯ˧ / mi˩}}} 
\lhead{\firstmark}
\rhead{\botmark}

\subsection{\hspace{-0.5cm} {\Large \textcolor{darkblue}{\textbf{\ipa{ʝi˧qv̩˥}}}}\hspace{0.5cm}[\kern2pt{\textcolor{darkblue}{\textbf{\ipa{ʝi˩qv̩˩˥}}}}\kern2pt]} \hypertarget{j££i\string_Mqv\string_=\string_T1}{}
\markboth{\textcolor{darkblue}{\textbf{\ipa{ʝi˧qv̩˥}}}}{}
\textcolor{teal}{\mytextsc{nom}} \hspace{4pt} Ton~: H\#.
\textcolor{PineGreen}{\selectlanguage{french}Collier: une partie du harnais utilisé pour les labours, qui maintient le joug en place; cette corde passe sous le cou de l'animal, et est fixée au joug.} \zh{轭的一个部分,将牛轭安在牛的脖子上。}  \zh{量词}: \textcolor{darkblue}{\textbf{\ipa{ɭɯ˧}}}  \mytextsc{clf}~: \textcolor{darkblue}{\textbf{\ipa{ɭɯ˧}}} 
\lhead{\firstmark}
\rhead{\botmark}

\subsection{\hspace{-0.5cm} {\Large \textcolor{darkblue}{\textbf{\ipa{ʝi˧ʁæ˥}}}}\hspace{0.5cm}[\kern2pt{\textcolor{darkblue}{\textbf{\ipa{ʝi˧ʁæ˥}}}}\kern2pt]} \hypertarget{j££i\string_MR\{\string_T1}{}
\markboth{\textcolor{darkblue}{\textbf{\ipa{ʝi˧ʁæ˥}}}}{}
\textcolor{teal}{\mytextsc{nom}} \hspace{4pt} Ton~: H\#.
\textcolor{PineGreen}{\selectlanguage{french}Vache, boeuf.} \zh{黄牛。}  \zh{量词}: \textcolor{darkblue}{\textbf{\ipa{pʰo˧˥}}}  \mytextsc{clf}~: \textcolor{darkblue}{\textbf{\ipa{pʰo˧˥}}} 
\lhead{\firstmark}
\rhead{\botmark}

\subsection{\hspace{-0.5cm} {\Large \textcolor{darkblue}{\textbf{\ipa{ʝi˧ʁo\#˥}}}}\hspace{0.5cm}[\kern2pt{\textcolor{darkblue}{\textbf{\ipa{ʝi˧ʁo˥}}}}\kern2pt]} \hypertarget{j££i\string_MRo\#\string_T1}{}
\markboth{\textcolor{darkblue}{\textbf{\ipa{ʝi˧ʁo\#˥}}}}{}
\textcolor{teal}{\mytextsc{adjectif}} \hspace{4pt} Ton~: \#H.
\textit{De:} \textbf{ʝi˥ 1 et ʁo˧ 2} \textcolor{PineGreen}{\selectlanguage{french}Capable; littéralement “qui sait faire”.} \zh{能干、不缺劳力。}  ¶ \textcolor{darkblue}{\textbf{\ipa{ʈʂʰɯ˧ | ʝi˧ʁo˧-hĩ˧ | ɖɯ˧-v̩˧ ɲi˩}}} \textcolor{PineGreen}{\selectlanguage{french}c'est quelqu'un d'habile/de capable} \zh{他是一个能干/称职的人。}  
 ¶ \textcolor{darkblue}{\textbf{\ipa{ʝi˧ʁo˧-zo˥}}} \textcolor{PineGreen}{\selectlanguage{french}un homme capable/habile, un gaillard compétent} \zh{一个能干的男人}  
 ¶ \textcolor{darkblue}{\textbf{\ipa{ʝi˧ʁo˧ ɲi˥}}} \textcolor{PineGreen}{\selectlanguage{french}\mytextsc{cop}} \zh{\mytextsc{cop}}  

\lhead{\firstmark}
\rhead{\botmark}

\subsection{\hspace{-0.5cm} {\Large \textcolor{darkblue}{\textbf{\ipa{ʝi˧sɑ˧}}}}\hspace{0.5cm}[\kern2pt{\textcolor{darkblue}{\textbf{\ipa{ʝi˧sɑ˧}}}}\kern2pt]} \hypertarget{j££i\string_MsA\string_M1}{}
\markboth{\textcolor{darkblue}{\textbf{\ipa{ʝi˧sɑ˧}}}}{}
\textcolor{teal}{\mytextsc{nom}} \hspace{4pt} Ton~: M.
\textcolor{PineGreen}{\selectlanguage{french}Parapluie (emprunt).} \zh{雨伞。}  Emprunt~: chinois  \zh{雨伞}
 \zh{量词}: \textcolor{darkblue}{\textbf{\ipa{nɑ˧}}}  \mytextsc{clf}~: \textcolor{darkblue}{\textbf{\ipa{nɑ˧}}} 
\lhead{\firstmark}
\rhead{\botmark}

\subsection{\hspace{-0.5cm} {\Large \textcolor{darkblue}{\textbf{\ipa{ʝi˧se˧}}} \textsubscript{1}}\hspace{0.5cm}[\kern2pt{\textcolor{darkblue}{\textbf{\ipa{ʝi˧se˩}}}}\kern2pt]} \hypertarget{j££i\string_Mse\string_M1}{}
\markboth{\textcolor{darkblue}{\textbf{\ipa{ʝi˧se˧}}} \textsubscript{1}}{}
\textcolor{teal}{\mytextsc{nom}} \hspace{4pt} Ton~: M.
\textcolor{PineGreen}{\selectlanguage{french}Médecin.} \zh{医生(汉语借词)。}  Emprunt~: chinois  \zh{医生}

\lhead{\firstmark}
\rhead{\botmark}

\subsection{\hspace{-0.5cm} {\Large \textcolor{darkblue}{\textbf{\ipa{ʝi˧se˧}}} \textsubscript{2}}\hspace{0.5cm}[\kern2pt{\textcolor{darkblue}{\textbf{\ipa{ʝi˧se˧}}}}\kern2pt]} \hypertarget{j££i\string_Mse\string_M2}{}
\markboth{\textcolor{darkblue}{\textbf{\ipa{ʝi˧se˧}}} \textsubscript{2}}{}
\textcolor{teal}{\mytextsc{adjectif}} \hspace{4pt} Ton~: M.
\textcolor{PineGreen}{\selectlanguage{french}Sauvage, spontané: plantes qui poussent spontanément (par opposition aux plantes cultivées), animaux sauvages.} \zh{野生(汉语借词)。}  Emprunt~: chinois  \zh{野生}
 ¶ \textcolor{darkblue}{\textbf{\ipa{ʝi˧se˧-hĩ˧}}} \textcolor{PineGreen}{\selectlanguage{french}\string_ \mytextsc{rel}/\mytextsc{nmlz}} \zh{野生的}  

\lhead{\firstmark}
\rhead{\botmark}

\subsection{\hspace{-0.5cm} {\Large \textcolor{darkblue}{\textbf{\ipa{ʝi˧sɯ˥}}}}\hspace{0.5cm}[\kern2pt{\textcolor{darkblue}{\textbf{\ipa{ʝi˧sɯ˧}}}}\kern2pt]} \hypertarget{j££i\string_MsM\string_T1}{}
\markboth{\textcolor{darkblue}{\textbf{\ipa{ʝi˧sɯ˥}}}}{}
\textcolor{teal}{\mytextsc{nom}} \hspace{4pt} Ton~: H\#.
\textcolor{PineGreen}{\selectlanguage{french}Signification, sens.} \zh{意思(汉语借词)。}  Emprunt~: chinois  \zh{意思}

\lhead{\firstmark}
\rhead{\botmark}

\subsection{\hspace{-0.5cm} {\Large \textcolor{darkblue}{\textbf{\ipa{ʝi˧ʂæ˧tsʰɤ˩}}}}\hspace{0.5cm}[\kern2pt{\textcolor{darkblue}{\textbf{\ipa{ʝi˧ʂæ˧tsʰɤ˧}}}}\kern2pt]} \hypertarget{j££i\string_Ms`\{\string_Mts\string_h7\string_B1}{}
\markboth{\textcolor{darkblue}{\textbf{\ipa{ʝi˧ʂæ˧tsʰɤ˩}}}}{}
\textcolor{teal}{\mytextsc{nom}} \hspace{4pt} Ton~: L\#.
\textcolor{PineGreen}{\selectlanguage{french}Radis sauvage qui pousse en montagne; on le consomme surtout au printemps, à une époque où il n'y a pas encore de légumes. Ce radis est récoltée par les Yi et vendu dans la plaine.} \zh{红萝卜菜(汉语借词:野山菜):一种山上的野菜。春天的时候,菜园的蔬菜还没有成熟的时候,永宁的人吃红萝卜菜。彝族人从高山上采下来,在永宁卖。} Dialecte chinois local~:\zh{野山菜。} Emprunt~: chinois  \zh{野山菜}
\textit{Voir~:} \hyperlink{}{\textcolor{darkblue}{\textbf{\ipa{jɤ˧˥}}} \textsubscript{2}} 
\lhead{\firstmark}
\rhead{\botmark}

\subsection{\hspace{-0.5cm} {\Large \textcolor{darkblue}{\textbf{\ipa{ʝi˧ʂɯ˥}}}}\hspace{0.5cm}[\kern2pt{\textcolor{darkblue}{\textbf{\ipa{ʝi˧ʂɯ˥}}}}\kern2pt]} \hypertarget{j££i\string_Ms`M\string_T1}{}
\markboth{\textcolor{darkblue}{\textbf{\ipa{ʝi˧ʂɯ˥}}}}{}
\textcolor{teal}{\mytextsc{nom}} \hspace{4pt} Ton~: H\#.
\textcolor{PineGreen}{\selectlanguage{french}Prénom masculin.} \zh{男性名字。} 
\lhead{\firstmark}
\rhead{\botmark}

\subsection{\hspace{-0.5cm} {\Large \textcolor{darkblue}{\textbf{\ipa{ʝi˧tɕi˧}}}}\hspace{0.5cm}[\kern2pt{\textcolor{darkblue}{\textbf{\ipa{ʝi˧tɕi˥}}}}\kern2pt]} \hypertarget{j££i\string_Mts£i\string_M1}{}
\markboth{\textcolor{darkblue}{\textbf{\ipa{ʝi˧tɕi˧}}}}{}
\textcolor{teal}{\mytextsc{nom}} \hspace{4pt} Ton~: M.
\textcolor{PineGreen}{\selectlanguage{french}Prénom féminin.} \zh{女性名字。} 
\lhead{\firstmark}
\rhead{\botmark}

\subsection{\hspace{-0.5cm} {\Large \textcolor{darkblue}{\textbf{\ipa{ʝi˧tɕi˧-ɖɯ˩mɑ˩}}}}\hspace{0.5cm}[\kern2pt{\textcolor{darkblue}{\textbf{\ipa{xxxx non-correspondance entre le nombre de morphèmes et le nombre de tons de morphèmes}}}}\kern2pt]} \hypertarget{j££i\string_Mts£i\string_M-d`M\string_BmA\string_B1}{}
\markboth{\textcolor{darkblue}{\textbf{\ipa{ʝi˧tɕi˧-ɖɯ˩mɑ˩}}}}{}
\textcolor{teal}{\mytextsc{nom}} \hspace{4pt} Ton~: \mytextsc{L}.
\textcolor{PineGreen}{\selectlanguage{french}Prénom féminin.} \zh{女性名字。} 
\lhead{\firstmark}
\rhead{\botmark}

\subsection{\hspace{-0.5cm} {\Large \textcolor{darkblue}{\textbf{\ipa{ʝi˧tsɯ˧}}}}\hspace{0.5cm}[\kern2pt{\textcolor{darkblue}{\textbf{\ipa{ʝi˧tsɯ˧}}}}\kern2pt]} \hypertarget{j££i\string_MtsM\string_M1}{}
\markboth{\textcolor{darkblue}{\textbf{\ipa{ʝi˧tsɯ˧}}}}{}
\textcolor{teal}{\mytextsc{nom}} \hspace{4pt} Ton~: M.
\textcolor{PineGreen}{\selectlanguage{french}Chaise (emprunt).} \zh{椅子。}  Emprunt~: chinois  \zh{椅子}
 \zh{量词}: \textcolor{darkblue}{\textbf{\ipa{nɑ˧}}}  \mytextsc{clf}~: \textcolor{darkblue}{\textbf{\ipa{nɑ˧}}} 
\lhead{\firstmark}
\rhead{\botmark}

\subsection{\hspace{-0.5cm} {\Large \textcolor{darkblue}{\textbf{\ipa{ʝi˧ʈʂʰe˥-mi˩}}}}\hspace{0.5cm}[\kern2pt{\textcolor{darkblue}{\textbf{\ipa{ʝi˧ʈʂʰe˥mi˩}}}}\kern2pt]} \hypertarget{j££i\string_Mt`s`\string_he\string_T-mi\string_B1}{}
\markboth{\textcolor{darkblue}{\textbf{\ipa{ʝi˧ʈʂʰe˥-mi˩}}}}{}
\textcolor{teal}{\mytextsc{nom}} \hspace{4pt} Ton~: H\#-.
\textcolor{PineGreen}{\selectlanguage{french}Sud.} \zh{南方。}  ¶ \textcolor{darkblue}{\textbf{\ipa{ʝi˧ʈʂʰe˥mi˩-gi˩dzɤ˩ se˩}}} \textcolor{PineGreen}{\selectlanguage{french}marcher en direction du sud} \zh{往南方走}  

\lhead{\firstmark}
\rhead{\botmark}

\subsection{\hspace{-0.5cm} {\Large \textcolor{darkblue}{\textbf{\ipa{ʝi˧zo\#˥}}}}\hspace{0.5cm}[\kern2pt{\textcolor{darkblue}{\textbf{\ipa{ʝi˧zo˧}}}}\kern2pt]} \hypertarget{j££i\string_Mzo\#\string_T1}{}
\markboth{\textcolor{darkblue}{\textbf{\ipa{ʝi˧zo\#˥}}}}{}
\textcolor{teal}{\mytextsc{nom}} \hspace{4pt} Ton~: \#H.
\textcolor{PineGreen}{\selectlanguage{french}Veau.} \zh{小牛。}  ¶ \textcolor{darkblue}{\textbf{\ipa{ʝi˧zo˧ tʰv̩˧-ɭɯ\#˥}}} \textcolor{PineGreen}{\selectlanguage{french}\mytextsc{n}+\mytextsc{dem}+\mytextsc{clf}} \zh{那头小牛}  
 \zh{量词}: \textcolor{darkblue}{\textbf{\ipa{pʰo˧˥ / ɭɯ˧}}}  \mytextsc{clf}~: \textcolor{darkblue}{\textbf{\ipa{pʰo˧˥ / ɭɯ˧}}} 
\lhead{\firstmark}
\rhead{\botmark}

\subsection{\hspace{-0.5cm} {\Large \textcolor{darkblue}{\textbf{\ipa{ʝi˩bv̩˩}}} \textsubscript{1}}\hspace{0.5cm}[\kern2pt{\textcolor{darkblue}{\textbf{\ipa{ʝi˧bv̩˧}}}}\kern2pt]} \hypertarget{j££i\string_Bbv\string_=\string_B1}{}
\markboth{\textcolor{darkblue}{\textbf{\ipa{ʝi˩bv̩˩}}} \textsubscript{1}}{}
\textcolor{teal}{\mytextsc{nom}} \hspace{4pt} Ton~: L.
\textcolor{PineGreen}{\selectlanguage{french}Grêlé.} \zh{麻子。}  ¶ \textcolor{darkblue}{\textbf{\ipa{ʝi˩bv̩˩-ʝi˧ʈv̩˩ʈv̩˩}}} \textcolor{PineGreen}{\selectlanguage{french}même sens} \zh{同上}  
 \zh{量词}: \textcolor{darkblue}{\textbf{\ipa{v̩˧}}}  \mytextsc{clf}~: \textcolor{darkblue}{\textbf{\ipa{v̩˧}}} 
\lhead{\firstmark}
\rhead{\botmark}

\subsection{\hspace{-0.5cm} {\Large \textcolor{darkblue}{\textbf{\ipa{ʝi˩bv̩˩}}} \textsubscript{2}}\hspace{0.5cm}[\kern2pt{\textcolor{darkblue}{\textbf{\ipa{ʝi˩bv̩˩˥}}}}\kern2pt]} \hypertarget{j££i\string_Bbv\string_=\string_B2}{}
\markboth{\textcolor{darkblue}{\textbf{\ipa{ʝi˩bv̩˩}}} \textsubscript{2}}{}
\textcolor{teal}{\mytextsc{nom}} \hspace{4pt} Ton~: L.
\textcolor{PineGreen}{\selectlanguage{french}Taureau.} \zh{公牛。}  \zh{量词}: \textcolor{darkblue}{\textbf{\ipa{pʰo˧˥}}}  \mytextsc{clf}~: \textcolor{darkblue}{\textbf{\ipa{pʰo˧˥}}} 
\lhead{\firstmark}
\rhead{\botmark}

\subsection{\hspace{-0.5cm} {\Large \textcolor{darkblue}{\textbf{\ipa{ʝi˩di˩-mi˥}}}}\hspace{0.5cm}[\kern2pt{\textcolor{darkblue}{\textbf{\ipa{xxxx non-correspondance entre le nombre de morphèmes et le nombre de tons de morphèmes}}}}\kern2pt]} \hypertarget{j££i\string_Bdi\string_B-mi\string_T1}{}
\markboth{\textcolor{darkblue}{\textbf{\ipa{ʝi˩di˩-mi˥}}}}{}
\textcolor{teal}{\mytextsc{nom}} \hspace{4pt} Ton~: L+H\#.
\textcolor{PineGreen}{\selectlanguage{french}Génisse; s'emploie pour une petite vache, aussi bien que pour le pianniu (tibétain: dzomo).} \zh{小牝牛(包括黄牛和小母犏牛)。}  \zh{量词}: \textcolor{darkblue}{\textbf{\ipa{ɭɯ˧}}} \textcolor{darkblue}{\textbf{\ipa{pʰo˧˥}}}  \mytextsc{clf}~: \textcolor{darkblue}{\textbf{\ipa{ɭɯ˧}}} \textcolor{darkblue}{\textbf{\ipa{pʰo˧˥}}} 
\lhead{\firstmark}
\rhead{\botmark}

\subsection{\hspace{-0.5cm} {\Large \textcolor{darkblue}{\textbf{\ipa{ʝi˩mi˩}}}}\hspace{0.5cm}[\kern2pt{\textcolor{darkblue}{\textbf{\ipa{ʝi˧mi˧}}}}\kern2pt]} \hypertarget{j££i\string_Bmi\string_B1}{}
\markboth{\textcolor{darkblue}{\textbf{\ipa{ʝi˩mi˩}}}}{}
\textcolor{teal}{\mytextsc{nom}} \hspace{4pt} Ton~: L.
\textcolor{PineGreen}{\selectlanguage{french}Vache.} \zh{母牛。}  ¶ \textcolor{darkblue}{\textbf{\ipa{ʝi˩mi˩-ʐɤ˥qo˩}}} \textcolor{PineGreen}{\selectlanguage{french}vache et veau} \zh{母牛与小牛}  
 \zh{量词}: \textcolor{darkblue}{\textbf{\ipa{pʰo˧˥}}}  \mytextsc{clf}~: \textcolor{darkblue}{\textbf{\ipa{pʰo˧˥}}} 
\lhead{\firstmark}
\rhead{\botmark}

\subsection{\hspace{-0.5cm} {\Large \textcolor{darkblue}{\textbf{\ipa{ʝi˩næ˩-se˧}}}}\hspace{0.5cm}[\kern2pt{\textcolor{darkblue}{\textbf{\ipa{xxxx non-correspondance entre le nombre de morphèmes et le nombre de tons de morphèmes}}}}\kern2pt]} \hypertarget{j££i\string_Bn\{\string_B-se\string_M1}{}
\markboth{\textcolor{darkblue}{\textbf{\ipa{ʝi˩næ˩-se˧}}}}{}
\textcolor{teal}{\mytextsc{nom}} \hspace{4pt} Ton~: L-M.
\textcolor{PineGreen}{\selectlanguage{french}Kunming et la partie orientale du Yunnan, une fois passés Lijiang et Dali.} \zh{云南,昆明……。}  Emprunt~: chinois  \zh{云南省}
 ¶ \textcolor{darkblue}{\textbf{\ipa{sɯ˧pʰi˧ | ʝi˩næ˩-se˧-qo˧ hɯ˧-ɲi˥!}}} \textcolor{PineGreen}{\selectlanguage{french}Le seigneur est parti à Kunming!} \zh{土司到昆明去了!}  

\lhead{\firstmark}
\rhead{\botmark}

\subsection{\hspace{-0.5cm} {\Large \textcolor{darkblue}{\textbf{\ipa{ʝi˩ŋɤ˧˥}}}}\hspace{0.5cm}[\kern2pt{\textcolor{darkblue}{\textbf{\ipa{xxxx non-correspondance entre le nombre de morphèmes et le nombre de tons de morphèmes}}}}\kern2pt]} \hypertarget{j££i\string_BN7\string_M\string_T1}{}
\markboth{\textcolor{darkblue}{\textbf{\ipa{ʝi˩ŋɤ˧˥}}}}{}
\textcolor{teal}{\mytextsc{verbe}} \hspace{4pt} Ton~: LM+MH\#.
\textcolor{PineGreen}{\selectlanguage{french}Se courber vers l’arrière.} \zh{往后仰。}  ¶ \textcolor{darkblue}{\textbf{\ipa{ʝi˩ŋɤ˧-ze˥}}} \textcolor{PineGreen}{\selectlanguage{french}\mytextsc{pfv}} \zh{往后仰了}  
 ¶ \textcolor{darkblue}{\textbf{\ipa{ʝi˩ŋɤ˧˥ | tʰi˧-dzi˩}}} \textcolor{PineGreen}{\selectlanguage{french}être assis en s'inclinant vers l'arrière} \zh{坐着往后仰}  

\lhead{\firstmark}
\rhead{\botmark}

\subsection{\hspace{-0.5cm} {\Large \textcolor{darkblue}{\textbf{\ipa{ʝi˩qʰv̩˩}}}}\hspace{0.5cm}[\kern2pt{\textcolor{darkblue}{\textbf{\ipa{ʝi˧qʰv̩˧}}}}\kern2pt]} \hypertarget{j££i\string_Bq\string_hv\string_=\string_B1}{}
\markboth{\textcolor{darkblue}{\textbf{\ipa{ʝi˩qʰv̩˩}}}}{}
\textcolor{teal}{\mytextsc{nom}} \hspace{4pt} Ton~: L.
\textcolor{PineGreen}{\selectlanguage{french}Manche.} \zh{袖子。}  \zh{量词}: \textcolor{darkblue}{\textbf{\ipa{ɭɯ˧}}}  \mytextsc{clf}~: \textcolor{darkblue}{\textbf{\ipa{ɭɯ˧}}} 
\lhead{\firstmark}
\rhead{\botmark}

\subsection{\hspace{-0.5cm} {\Large \textcolor{darkblue}{\textbf{\ipa{ʝi˩ʈʂæ˧˥}}}}\hspace{0.5cm}[\kern2pt{\textcolor{darkblue}{\textbf{\ipa{ʝi˩ʈʂæ˧˥}}}}\kern2pt]} \hypertarget{j££i\string_Bt`s`\{\string_M\string_T1}{}
\markboth{\textcolor{darkblue}{\textbf{\ipa{ʝi˩ʈʂæ˧˥}}}}{}
\textcolor{teal}{\mytextsc{nom}} \hspace{4pt} Ton~: LM+MH\#.
\textcolor{PineGreen}{\selectlanguage{french}Taille.} \zh{腰。}  \zh{量词}: \textcolor{darkblue}{\textbf{\ipa{ɭɯ˧}}}  \mytextsc{clf}~: \textcolor{darkblue}{\textbf{\ipa{ɭɯ˧}}} 
\lhead{\firstmark}
\rhead{\botmark}

\subsection{\hspace{-0.5cm} {\Large \textcolor{darkblue}{\textbf{\ipa{ʝi˩˥}}}}\hspace{0.5cm}[\kern2pt{\textcolor{darkblue}{\textbf{\ipa{ʝi˩˥}}}}\kern2pt]} \hypertarget{j££i\string_B\string_T1}{}
\markboth{\textcolor{darkblue}{\textbf{\ipa{ʝi˩˥}}}}{}
\textcolor{teal}{\mytextsc{nom}} \hspace{4pt} Ton~: LH.
\textcolor{PineGreen}{\selectlanguage{french}Bouton.} \zh{痘痘。}  ¶ \textcolor{darkblue}{\textbf{\ipa{ʝi˩ tʰv̩˩˥}}} \textcolor{PineGreen}{\selectlanguage{french}avoir des boutons} \zh{出痘痘}  
 \zh{量词}: \textcolor{darkblue}{\textbf{\ipa{ɭɯ˧}}}  \mytextsc{clf}~: \textcolor{darkblue}{\textbf{\ipa{ɭɯ˧}}} 
\lhead{\firstmark}
\rhead{\botmark}

\newpage
\section*{\centering- \textcolor{darkblue}{\textbf{\ipa{k}}} -}
\subsection{\hspace{-0.5cm} {\Large \textcolor{darkblue}{\textbf{\ipa{kæ˧ʈʂe˧}}}}\hspace{0.5cm}[\kern2pt{\textcolor{darkblue}{\textbf{\ipa{kæ˧ʈʂe˧}}}}\kern2pt]} \hypertarget{k\{\string_Mt`s`e\string_M1}{}
\markboth{\textcolor{darkblue}{\textbf{\ipa{kæ˧ʈʂe˧}}}}{}
\textcolor{teal}{\mytextsc{nom}} \hspace{4pt} Ton~: M.
\textcolor{PineGreen}{\selectlanguage{french}Acupuncture.} \zh{针灸(汉语借词:干针)。} Dialecte chinois local~:\zh{干针。} Emprunt~: chinois  \zh{干针}
 ¶ \textcolor{darkblue}{\textbf{\ipa{kæ˧ʈʂe˧ lɑ˧˥}}} \textcolor{PineGreen}{\selectlanguage{french}faire une séance d'acupuncture, placer des aiguilles d'acupuncture} \zh{扎针灸}  

\lhead{\firstmark}
\rhead{\botmark}

\subsection{\hspace{-0.5cm} {\Large \textcolor{darkblue}{\textbf{\ipa{kæ˧ʈʂɯ˧}}}}\hspace{0.5cm}[\kern2pt{\textcolor{darkblue}{\textbf{\ipa{kæ˧ʈʂɯ˧}}}}\kern2pt]} \hypertarget{k\{\string_Mt`s`M\string_M1}{}
\markboth{\textcolor{darkblue}{\textbf{\ipa{kæ˧ʈʂɯ˧}}}}{}
\textcolor{teal}{\mytextsc{nom}} \hspace{4pt} Ton~: M.
\textcolor{PineGreen}{\selectlanguage{french}Canne à sucre.} \zh{甘蔗。}  Emprunt~: chinois  \zh{甘蔗}

\lhead{\firstmark}
\rhead{\botmark}

\subsection{\hspace{-0.5cm} {\Large \textcolor{darkblue}{\textbf{\ipa{kɤ˧dzi˧}}}}\hspace{0.5cm}[\kern2pt{\textcolor{darkblue}{\textbf{\ipa{kɤ˧dzi˧}}}}\kern2pt]} \hypertarget{k7\string_Mdzi\string_M1}{}
\markboth{\textcolor{darkblue}{\textbf{\ipa{kɤ˧dzi˧}}}}{}
\textcolor{teal}{\mytextsc{verbe}} \hspace{4pt} Ton~: M.
\textcolor{PineGreen}{\selectlanguage{french}Prendre place (lors d'un repas, d'une cérémonie...).} \zh{坐下(在饭桌)。}  ¶ \textcolor{darkblue}{\textbf{\ipa{ɑ˩ʁo˧-hĩ˧ | kɤ˧dzi˧-ze˧.}}} \textcolor{PineGreen}{\selectlanguage{french}Les gens de la famille se sont assis/ont pris place.} \zh{家人入座了。}  

\lhead{\firstmark}
\rhead{\botmark}

\subsection{\hspace{-0.5cm} {\Large \textcolor{darkblue}{\textbf{\ipa{kɤ˧kɤ˩}}}}\hspace{0.5cm}[\kern2pt{\textcolor{darkblue}{\textbf{\ipa{kɤ˧kɤ˩}}}}\kern2pt]} \hypertarget{k7\string_Mk7\string_B1}{}
\markboth{\textcolor{darkblue}{\textbf{\ipa{kɤ˧kɤ˩}}}}{}
\textcolor{teal}{\mytextsc{adverbe}} \hspace{4pt} Ton~: L\#.
\textcolor{PineGreen}{\selectlanguage{french}Proche de, à côté de.} \zh{挨着(坐……)。}  ¶ \textcolor{darkblue}{\textbf{\ipa{(tso˧\textasciitilde{}tso˧) kɤ˧kɤ˩ | tʰi˧-tɕɯ˥}}} \textcolor{PineGreen}{\selectlanguage{french}ranger des choses en bon ordre} \zh{摆整齐、使均匀,如:一排排挨着}  
 ¶ \textcolor{darkblue}{\textbf{\ipa{[Tiger2] kɤ˧kɤ˩ | tʰi˧-se˥}}} \textcolor{PineGreen}{\selectlanguage{french}marcher en file indienne} \zh{并排走}  
 ¶ \textcolor{darkblue}{\textbf{\ipa{kɤ˧kɤ˩ | tʰi˧-dzi˩}}} \textcolor{PineGreen}{\selectlanguage{french}être assis les uns à côté des autres, proches les uns des autres} \zh{挨着坐}  

\lhead{\firstmark}
\rhead{\botmark}

\subsection{\hspace{-0.5cm} {\Large \textcolor{darkblue}{\textbf{\ipa{kɤ˧ljɤ˩}}}}\hspace{0.5cm}[\kern2pt{\textcolor{darkblue}{\textbf{\ipa{kɤ˧ljɤ˩}}}}\kern2pt]} \hypertarget{k7\string_Mlj7\string_B1}{}
\markboth{\textcolor{darkblue}{\textbf{\ipa{kɤ˧ljɤ˩}}}}{}
\textcolor{teal}{\mytextsc{nom}} \hspace{4pt} Ton~: L\#.
\textcolor{PineGreen}{\selectlanguage{french}Sorgho, gaoliang; céréale dont on se sert pour faire du vin.} \zh{高粱(汉语借词)。}  Emprunt~: chinois  \zh{高粱}
\textit{Voir~:} \hyperlink{}{\textcolor{darkblue}{\textbf{\ipa{hæ˧ɭɯ\#˥}}}} 
\lhead{\firstmark}
\rhead{\botmark}

\subsection{\hspace{-0.5cm} {\Large \textcolor{darkblue}{\textbf{\ipa{kɤ˧mi˧}}} \textsubscript{1}}\hspace{0.5cm}[\kern2pt{\textcolor{darkblue}{\textbf{\ipa{kɤ˧mi˧}}}}\kern2pt]} \hypertarget{k7\string_Mmi\string_M1}{}
\markboth{\textcolor{darkblue}{\textbf{\ipa{kɤ˧mi˧}}} \textsubscript{1}}{}
\textcolor{teal}{\mytextsc{nom}} \hspace{4pt} Ton~: M.
\textcolor{PineGreen}{\selectlanguage{french}Faucon femelle.} \zh{母隼。}  ¶ \textcolor{darkblue}{\textbf{\ipa{kɤ˩mi˩-kɤ˩pʰv̩˥}}} \textcolor{PineGreen}{\selectlanguage{french}faucon femelle et faucon mâle} \zh{母隼与公隼}  
 \zh{量词}: \textcolor{darkblue}{\textbf{\ipa{mi˩}}}  \mytextsc{clf}~: \textcolor{darkblue}{\textbf{\ipa{mi˩}}} 
\lhead{\firstmark}
\rhead{\botmark}

\subsection{\hspace{-0.5cm} {\Large \textcolor{darkblue}{\textbf{\ipa{kɤ˧mi˧}}} \textsubscript{2}}\hspace{0.5cm}[\kern2pt{\textcolor{darkblue}{\textbf{\ipa{kɤ˧mi˧}}}}\kern2pt]} \hypertarget{k7\string_Mmi\string_M2}{}
\markboth{\textcolor{darkblue}{\textbf{\ipa{kɤ˧mi˧}}} \textsubscript{2}}{}
\textcolor{teal}{\mytextsc{nom}} \hspace{4pt} Ton~: M.
\textcolor{PineGreen}{\selectlanguage{french}Grande jarre; grande bouteille.} \zh{大坛子,大瓶。}  \zh{量词}: \textcolor{darkblue}{\textbf{\ipa{ɭɯ˧}}}  \mytextsc{clf}~: \textcolor{darkblue}{\textbf{\ipa{ɭɯ˧}}} 
\lhead{\firstmark}
\rhead{\botmark}

\subsection{\hspace{-0.5cm} {\Large \textcolor{darkblue}{\textbf{\ipa{kɤ˧mv̩˧˥}}}}\hspace{0.5cm}[\kern2pt{\textcolor{darkblue}{\textbf{\ipa{kɤ˧mv̩˧˥}}}}\kern2pt]} \hypertarget{k7\string_Mmv\string_=\string_M\string_T1}{}
\markboth{\textcolor{darkblue}{\textbf{\ipa{kɤ˧mv̩˧˥}}}}{}
\textcolor{teal}{\mytextsc{nom}} \hspace{4pt} Ton~: MH\#.
\textcolor{PineGreen}{\selectlanguage{french}La montagne Gemu (Yongning).} \zh{格母山。}  ¶ \textcolor{darkblue}{\textbf{\ipa{ɬi˧di˩-kɤ˩mv̩˩}}} \textcolor{PineGreen}{\selectlanguage{french}la montagne Gemu de Yongning} \zh{永宁格姆山}  
 ¶ \textcolor{darkblue}{\textbf{\ipa{kɤ˧mv̩˧-hæ̃˧kʰo˥}}} \textcolor{PineGreen}{\selectlanguage{french}“la princesse Gemu”; autre nom de la montagne Gemu, considérée comme une divinité féminine} \zh{格姆公主:格姆山别名(格姆山被看作女神)}  
 ¶ \textcolor{darkblue}{\textbf{\ipa{kɤ˧mv̩˧˥, | æ˧ʂæ˧, | ŋwɤ˧hɑ̃˩, | ʂwæ˧gv̩\#˥, | nɑ˩tsʰi˩˥ | -tɕʰɤ˧pɤ˧mi\#˥, | qv̩˧ɻ̍˧-ʈʂʰɑ˧nɑ˥ |}}} \textcolor{PineGreen}{\selectlanguage{french}Les six montagnes de Yongning qui portent un nom. Les autres sommets du voisinage n'ont pas une valeur symbolique comparable, et ne portent pas de nom communément utilisé.} \zh{永宁地区有固定名字的六座山。其它山,没有重要的象征意义,因此也没有固定名称。}  

\lhead{\firstmark}
\rhead{\botmark}

\subsection{\hspace{-0.5cm} {\Large \textcolor{darkblue}{\textbf{\ipa{kɤ˧ʈʂɯ˩}}} \textsubscript{1}}\hspace{0.5cm}[\kern2pt{\textcolor{darkblue}{\textbf{\ipa{kɤ˧ʈʂɯ˩}}}}\kern2pt]} \hypertarget{k7\string_Mt`s`M\string_B1}{}
\markboth{\textcolor{darkblue}{\textbf{\ipa{kɤ˧ʈʂɯ˩}}} \textsubscript{1}}{}
\textcolor{teal}{\mytextsc{verbe}} \hspace{4pt} Ton~: L\#.
\textcolor{PineGreen}{\selectlanguage{french}Parler, raconter.} \zh{讲。}  ¶ \textcolor{darkblue}{\textbf{\ipa{hĩ˧-ki˧ | tʰɑ˧-kɤ˧ʈʂɯ˩!}}} \textcolor{PineGreen}{\selectlanguage{french}il ne faut pas le dire aux gens! / c'est secret!} \zh{不要告诉人家!}  
 ¶ \textcolor{darkblue}{\textbf{\ipa{kɤ˧-tʰɑ˥-ʈʂɯ˩!}}} \textcolor{PineGreen}{\selectlanguage{french}il ne faut pas le dire! / c'est secret!} \zh{不要告诉人家!}  
 ¶ \textcolor{darkblue}{\textbf{\ipa{njɤ˧-ɳɯ˧ | kɤ˧ʈʂɯ˩-bi˩!}}} \textcolor{PineGreen}{\selectlanguage{french}je vais intervenir/je vais dire quelque chose!} \zh{我要说一点事情!}  
 ¶ \textcolor{darkblue}{\textbf{\ipa{no˧ | kɤ˧ʈʂɯ˩ dʑo˩-ɲi˩!}}} \textcolor{PineGreen}{\selectlanguage{french}il faut que tu dises quelque chose!} \zh{你得说话啊!}  
 ¶ \textcolor{darkblue}{\textbf{\ipa{ʈʂʰɯ˧ | kɤ˧ʈʂɯ˩ | dʑɤ˩˥ | mɤ˧-mv̩˧-sɯ˥! / ʈʂʰɯ˧ | kɤ˧ʈʂɯ˩ dʑɤ˩˥ | mɤ˧-mv̩˧\textasciitilde{}mv̩˧-sɯ˥!}}} \textcolor{PineGreen}{\selectlanguage{french}elle ne comprend pas encore grand'chose à ce qu'on dit! / elle ne sait pas encore comprendre ce qu'on dit! (au sujet d'une fillette de moins de 2 ans qui ne parle pas encore)} \zh{她还不怎么听得懂话!(关于一个不会说话的两岁小孩)}  
 ¶ \textcolor{darkblue}{\textbf{\ipa{tʰɑ˧-kɤ˧ʈʂɯ˩! | hĩ˧ ɳv̩˧ tʰɑ˧-kʰɯ˩!}}} \textcolor{PineGreen}{\selectlanguage{french}N'en parle pas! il ne faut pas que les gens le sachent!} \zh{不要告诉(人家)!别让人家知道!}  
 ¶ \textcolor{darkblue}{\textbf{\ipa{hĩ˧-ki˧ | kɤ˧-mɤ˧-ʈʂɯ˩}}} \textcolor{PineGreen}{\selectlanguage{french}(faire quelque chose) en cachette, sans le dire à personne} \zh{不跟人家说(自己做什么事)}  
 ¶ \textcolor{darkblue}{\textbf{\ipa{kɤ˧ʈʂɯ˩ ɲi˩}}} \textcolor{PineGreen}{\selectlanguage{french}sage (au sujet d'un enfant) (littéralement: qui écoute ce qu'on lui dit)} \zh{听话,乖(来形容一个孩子)}  
\textit{Voir~:} \hyperlink{}{\textcolor{darkblue}{\textbf{\ipa{kɤ˧ʈʂɯ˩}}} \textsubscript{2}} 
\lhead{\firstmark}
\rhead{\botmark}

\subsection{\hspace{-0.5cm} {\Large \textcolor{darkblue}{\textbf{\ipa{kɤ˧ʈʂɯ˩}}} \textsubscript{2}}\hspace{0.5cm}[\kern2pt{\textcolor{darkblue}{\textbf{\ipa{kɤ˧ʈʂɯ˩}}}}\kern2pt]} \hypertarget{k7\string_Mt`s`M\string_B2}{}
\markboth{\textcolor{darkblue}{\textbf{\ipa{kɤ˧ʈʂɯ˩}}} \textsubscript{2}}{}
\textcolor{teal}{\mytextsc{nom}} \hspace{4pt} Ton~: L\#.
\textcolor{PineGreen}{\selectlanguage{french}Parole.} \zh{话。}  ¶ \textcolor{darkblue}{\textbf{\ipa{kɤ˧ʈʂɯ˩ ʝi˩}}} \textcolor{PineGreen}{\selectlanguage{french}promettre; aussi: jurer ses grands dieux, prêter serment devant les Dieux: lorsque deux personnes avaient un différend qu'elles ne parvenaient pas à trancher, elles allaient raconter chacune sa version des faits devant les Dieux (au monastère); ceux-ci punissaient ensuite le coupable (par des calamités qui frappaient la famille du coupable).} \zh{答应,誓、发誓。两个人发生矛盾的时候,如果无法协调,他们会去大寺,在神像前讲述他们各自的观点,发誓他们自己讲的是真的。神会惩罚说谎的人(他家会有祸害)。}  
 ¶ \textcolor{darkblue}{\textbf{\ipa{ʈʂʰɯ˧ | kɤ˧ʈʂɯ˩-ʝi˩}}} \textcolor{PineGreen}{\selectlanguage{french}il/elle promet} \zh{他答应}  
 ¶ \textcolor{darkblue}{\textbf{\ipa{ʈʂʰɯ˧ | kɤ˧ʈʂɯ˩ | mɤ˧-ʝi˥!}}} \textcolor{PineGreen}{\selectlanguage{french}il n'a pas promis!} \zh{他没有答应!}  
 ¶ \textcolor{darkblue}{\textbf{\ipa{hĩ˧-kɤ˧ʈʂɯ˥ ɲi˩}}} \textcolor{PineGreen}{\selectlanguage{french}écouter les conseils d'autrui, prêter attention à la parole d'autrui, écouter les bons conseils (attitude jugée positive et souhaitable par la consultante)} \zh{听别人的建议、把别人的话当回事}  
 ¶ \textcolor{darkblue}{\textbf{\ipa{hĩ˧-kɤ˧ʈʂɯ˥ | le˧-ɲi˥}}} \textcolor{PineGreen}{\selectlanguage{french}même sens} \zh{同上}  
 ¶ \textcolor{darkblue}{\textbf{\ipa{hĩ˧-kɤ˧ʈʂɯ˥ | mɤ˧-ɲi˥}}} \textcolor{PineGreen}{\selectlanguage{french}ne pas écouter les bons conseils, ne pas prêter attention à ce qu'on vous dit} \zh{听不进去别人的意见与建议}  
 \zh{量词}: \textcolor{darkblue}{\textbf{\ipa{kʰwɤ˥}}}  \mytextsc{clf}~: \textcolor{darkblue}{\textbf{\ipa{kʰwɤ˥}}} \textit{Voir~:} \hyperlink{}{\textcolor{darkblue}{\textbf{\ipa{kɤ˧ʈʂɯ˩}}} \textsubscript{1}} 
\lhead{\firstmark}
\rhead{\botmark}

\subsection{\hspace{-0.5cm} {\Large \textcolor{darkblue}{\textbf{\ipa{kɤ˧v̩\#˥}}}}\hspace{0.5cm}[\kern2pt{\textcolor{darkblue}{\textbf{\ipa{kɤ˧v̩˧}}}}\kern2pt]} \hypertarget{k7\string_Mv\string_=\#\string_T1}{}
\markboth{\textcolor{darkblue}{\textbf{\ipa{kɤ˧v̩\#˥}}}}{}
\textcolor{teal}{\mytextsc{nom}} \hspace{4pt} Ton~: \#H.
\textcolor{PineGreen}{\selectlanguage{french}Amulette.} \zh{护符,护身符。}  \zh{量词}: \textcolor{darkblue}{\textbf{\ipa{ɭɯ˧}}}  \mytextsc{clf}~: \textcolor{darkblue}{\textbf{\ipa{ɭɯ˧}}} 
\lhead{\firstmark}
\rhead{\botmark}

\subsection{\hspace{-0.5cm} {\Large \textcolor{darkblue}{\textbf{\ipa{kɤ˧wɤ\#˥}}}}\hspace{0.5cm}[\kern2pt{\textcolor{darkblue}{\textbf{\ipa{kɤ˧wɤ˧}}}}\kern2pt]} \hypertarget{k7\string_Mw7\#\string_T1}{}
\markboth{\textcolor{darkblue}{\textbf{\ipa{kɤ˧wɤ\#˥}}}}{}
\textcolor{teal}{\mytextsc{nom}} \hspace{4pt} Ton~: \#H.
\textcolor{PineGreen}{\selectlanguage{french}Destinée, affinité prédestinée.} \zh{缘分、共同命运。}  ¶ \textcolor{darkblue}{\textbf{\ipa{kɤ˧wɤ˧-ljɤ˧˥}}} \textcolor{PineGreen}{\selectlanguage{french}avoir une affinité prédestinée, avoir un destin commun} \zh{有缘分、有共同命运}  

\lhead{\firstmark}
\rhead{\botmark}

\subsection{\hspace{-0.5cm} {\Large \textcolor{darkblue}{\textbf{\ipa{kɤ˧zo\#˥}}}}\hspace{0.5cm}[\kern2pt{\textcolor{darkblue}{\textbf{\ipa{kɤ˧zo˧}}}}\kern2pt]} \hypertarget{k7\string_Mzo\#\string_T1}{}
\markboth{\textcolor{darkblue}{\textbf{\ipa{kɤ˧zo\#˥}}}}{}
\textcolor{teal}{\mytextsc{nom}} \hspace{4pt} Ton~: \#H.
\textcolor{PineGreen}{\selectlanguage{french}Prénom masculin.} \zh{男性名字。} 
\lhead{\firstmark}
\rhead{\botmark}

\subsection{\hspace{-0.5cm} {\Large \textcolor{darkblue}{\textbf{\ipa{kɤ˩}}}}\hspace{0.5cm}[\kern2pt{\textcolor{darkblue}{\textbf{\ipa{kɤ˥}}}}\kern2pt]} \hypertarget{k7\string_B1}{}
\markboth{\textcolor{darkblue}{\textbf{\ipa{kɤ˩}}}}{}
\textcolor{teal}{\mytextsc{nom}} \hspace{4pt} Ton~: L.
\textcolor{PineGreen}{\selectlanguage{french}Bouteille.} \zh{瓶子。}  ¶ \textcolor{darkblue}{\textbf{\ipa{ʐɯ˧-kɤ˩}}} \textcolor{PineGreen}{\selectlanguage{french}bouteille d'alcool} \zh{酒瓶}  
 ¶ \textcolor{darkblue}{\textbf{\ipa{ʐɯ˧ ɖɯ˧-kɤ˩}}} \textcolor{PineGreen}{\selectlanguage{french}une bouteille d'alcool} \zh{一瓶酒}  
 \zh{量词}: \textcolor{darkblue}{\textbf{\ipa{ɭɯ˧}}}  \mytextsc{clf}~: \textcolor{darkblue}{\textbf{\ipa{ɭɯ˧}}} 
\lhead{\firstmark}
\rhead{\botmark}

\subsection{\hspace{-0.5cm} {\Large \textcolor{darkblue}{\textbf{\ipa{kɤ˩\textsubscript{a}}}}}\hspace{0.5cm}[\kern2pt{\textcolor{darkblue}{\textbf{\ipa{kɤ˩˥}}}}\kern2pt]} \hypertarget{k7\string_Ba1}{}
\markboth{\textcolor{darkblue}{\textbf{\ipa{kɤ˩\textsubscript{a}}}}}{}
\textcolor{teal}{\mytextsc{classificateur}} \hspace{4pt} Ton~: L\textsubscript{a}.
\textcolor{PineGreen}{\selectlanguage{french}Classificateur des bouteilles.} \zh{量词:瓶。}  ¶ \textcolor{darkblue}{\textbf{\ipa{kɤ˩zo˩˥}}} \textcolor{PineGreen}{\selectlanguage{french}petite bouteille} \zh{一小瓶}  
 ¶ \textcolor{darkblue}{\textbf{\ipa{ʈʂʰɯ˧-kɤ˥}}} \textcolor{PineGreen}{\selectlanguage{french}\mytextsc{dem} \string_ (tone: H\# / H\$)} \zh{\mytextsc{指示代词} \string_}  

\lhead{\firstmark}
\rhead{\botmark}

\subsection{\hspace{-0.5cm} {\Large \textcolor{darkblue}{\textbf{\ipa{kɤ˩\textasciitilde{}kɤ˧˥}}}}\hspace{0.5cm}[\kern2pt{\textcolor{darkblue}{\textbf{\ipa{kɤ˧kɤ˧˥}}}}\kern2pt]} \hypertarget{k7\string_B~k7\string_M\string_T1}{}
\markboth{\textcolor{darkblue}{\textbf{\ipa{kɤ˩\textasciitilde{}kɤ˧˥}}}}{}
\textcolor{teal}{\mytextsc{verbe}} \hspace{4pt} Ton~: MH.
\textcolor{PineGreen}{\selectlanguage{french}Tapoter.} \zh{敲、拍。}  ¶ \textcolor{darkblue}{\textbf{\ipa{kʰi˧ kɤ˥\textasciitilde{}kɤ˩}}} \textcolor{PineGreen}{\selectlanguage{french}frapper à la porte} \zh{敲门}  
 ¶ \textcolor{darkblue}{\textbf{\ipa{njɤ˧-ɳɯ˧ | no˧ | kɤ˩\textasciitilde{}kɤ˧-bi˥!}}} \textcolor{PineGreen}{\selectlanguage{french}Je vais te donner une tape/une fessée! (Menace d'un adulte à un enfant)} \zh{我要打你屁股了!(大人对孩子说)}  
 ¶ \textcolor{darkblue}{\textbf{\ipa{ʈʂo˧tsɯ˥ kɤ˩\textasciitilde{}kɤ˩ (-ze˩/-bi˩)}}} \textcolor{PineGreen}{\selectlanguage{french}heurter la table, taper sur la table} \zh{拍拍桌子}  
 ¶ \textcolor{darkblue}{\textbf{\ipa{gv̩˧dv̩˧ kɤ˧\textasciitilde{}kɤ˩}}} \textcolor{PineGreen}{\selectlanguage{french}tapoter sur le dos de quelqu'un (pour soulager un mal de dos)} \zh{敲敲背}  
\textit{Voir~:} \hyperlink{}{\textcolor{darkblue}{\textbf{\ipa{kɤ˧˥}}}} 
\lhead{\firstmark}
\rhead{\botmark}

\subsection{\hspace{-0.5cm} {\Large \textcolor{darkblue}{\textbf{\ipa{kɤ˩lo˧˥}}}}\hspace{0.5cm}[\kern2pt{\textcolor{darkblue}{\textbf{\ipa{kɤ˩lo˧˥}}}}\kern2pt]} \hypertarget{k7\string_Blo\string_M\string_T1}{}
\markboth{\textcolor{darkblue}{\textbf{\ipa{kɤ˩lo˧˥}}}}{}
\textcolor{teal}{\mytextsc{nom}} \hspace{4pt} Ton~: LM+MH\#.
\textcolor{PineGreen}{\selectlanguage{french}Branche.} \zh{树枝。}  ¶ \textcolor{darkblue}{\textbf{\ipa{si˧dzi˩-kɤ˩lo˩}}} \textcolor{PineGreen}{\selectlanguage{french}branche d'arbre} \zh{树枝}  
 ¶ \textcolor{darkblue}{\textbf{\ipa{si˧-kɤ˥lo˩}}} \textcolor{PineGreen}{\selectlanguage{french}idem} \zh{同上}  
 \zh{量词}: \textcolor{darkblue}{\textbf{\ipa{kɤ˧˥}}}  \mytextsc{clf}~: \textcolor{darkblue}{\textbf{\ipa{kɤ˧˥}}} 
\lhead{\firstmark}
\rhead{\botmark}

\subsection{\hspace{-0.5cm} {\Large \textcolor{darkblue}{\textbf{\ipa{kɤ˩-nɑ˧mi˧}}}}\hspace{0.5cm}[\kern2pt{\textcolor{darkblue}{\textbf{\ipa{kɤ˧nɑ˧mi˧}}}}\kern2pt]} \hypertarget{k7\string_B-nA\string_Mmi\string_M1}{}
\markboth{\textcolor{darkblue}{\textbf{\ipa{kɤ˩-nɑ˧mi˧}}}}{}
\textcolor{teal}{\mytextsc{nom}} \hspace{4pt} Ton~: L-.
\textcolor{PineGreen}{\selectlanguage{french}Aigle.} \zh{老鹰。}  \zh{量词}: \textcolor{darkblue}{\textbf{\ipa{mi˩}}}  \mytextsc{clf}~: \textcolor{darkblue}{\textbf{\ipa{mi˩}}} 
\lhead{\firstmark}
\rhead{\botmark}

\subsection{\hspace{-0.5cm} {\Large \textcolor{darkblue}{\textbf{\ipa{kɤ˩pʰv̩˩}}}}\hspace{0.5cm}[\kern2pt{\textcolor{darkblue}{\textbf{\ipa{kɤ˩pʰv̩˩˥}}}}\kern2pt]} \hypertarget{k7\string_Bp\string_hv\string_=\string_B1}{}
\markboth{\textcolor{darkblue}{\textbf{\ipa{kɤ˩pʰv̩˩}}}}{}
\textcolor{teal}{\mytextsc{nom}} \hspace{4pt} Ton~: L.
\textcolor{PineGreen}{\selectlanguage{french}Faucon mâle.} \zh{公隼。}  ¶ \textcolor{darkblue}{\textbf{\ipa{kɤ˩pʰv̩˩-kɤ˩mi˥}}} \textcolor{PineGreen}{\selectlanguage{french}faucon mâle et faucon femelle} \zh{公隼与母隼}  
 \zh{量词}: \textcolor{darkblue}{\textbf{\ipa{mi˩}}}  \mytextsc{clf}~: \textcolor{darkblue}{\textbf{\ipa{mi˩}}} 
\lhead{\firstmark}
\rhead{\botmark}

\subsection{\hspace{-0.5cm} {\Large \textcolor{darkblue}{\textbf{\ipa{kɤ˩-tjɤ˧ljɤ\#˥}}}}\hspace{0.5cm}[\kern2pt{\textcolor{darkblue}{\textbf{\ipa{kɤ˧tjɤ˧ljɤ˧}}}}\kern2pt]} \hypertarget{k7\string_B-tj7\string_Mlj7\#\string_T1}{}
\markboth{\textcolor{darkblue}{\textbf{\ipa{kɤ˩-tjɤ˧ljɤ\#˥}}}}{}
\textcolor{teal}{\mytextsc{nom}} \hspace{4pt} Ton~: L-\#H.
\textcolor{PineGreen}{\selectlanguage{french}Clochette s'accrochant autour du cou (ex.: clochette d'un cheval).} \zh{铃铛。}  \zh{量词}: \textcolor{darkblue}{\textbf{\ipa{ɭɯ˧}}}  \mytextsc{clf}~: \textcolor{darkblue}{\textbf{\ipa{ɭɯ˧}}} 
\lhead{\firstmark}
\rhead{\botmark}

\subsection{\hspace{-0.5cm} {\Large \textcolor{darkblue}{\textbf{\ipa{kɤ˩zo˩}}} \textsubscript{1}}\hspace{0.5cm}[\kern2pt{\textcolor{darkblue}{\textbf{\ipa{kɤ˩zo˩˥}}}}\kern2pt]} \hypertarget{k7\string_Bzo\string_B1}{}
\markboth{\textcolor{darkblue}{\textbf{\ipa{kɤ˩zo˩}}} \textsubscript{1}}{}
\textcolor{teal}{\mytextsc{nom}} \hspace{4pt} Ton~: L.
\textcolor{PineGreen}{\selectlanguage{french}Bébé faucon.} \zh{小隼。} 
\lhead{\firstmark}
\rhead{\botmark}

\subsection{\hspace{-0.5cm} {\Large \textcolor{darkblue}{\textbf{\ipa{kɤ˩zo˩}}} \textsubscript{2}}\hspace{0.5cm}[\kern2pt{\textcolor{darkblue}{\textbf{\ipa{kɤ˩zo˩˥}}}}\kern2pt]} \hypertarget{k7\string_Bzo\string_B2}{}
\markboth{\textcolor{darkblue}{\textbf{\ipa{kɤ˩zo˩}}} \textsubscript{2}}{}
\textcolor{teal}{\mytextsc{nom}} \hspace{4pt} Ton~: L.
\textcolor{PineGreen}{\selectlanguage{french}Petite bouteille.} \zh{小瓶子。} 
\lhead{\firstmark}
\rhead{\botmark}

\subsection{\hspace{-0.5cm} {\Large \textcolor{darkblue}{\textbf{\ipa{kɤ˩xxxx}}}}\hspace{0.5cm}[\kern2pt{\textcolor{darkblue}{\textbf{\ipa{xxxx groupe tonal entier sans aucun ton}}}}\kern2pt]} \hypertarget{k7\string_Bxxxx1}{}
\markboth{\textcolor{darkblue}{\textbf{\ipa{kɤ˩xxxx}}}}{}
\textcolor{teal}{\mytextsc{verbe}} \hspace{4pt} Ton~: xxxx vérifier : L\textsubscript{a}? L\textsubscript{b}?.
\textcolor{PineGreen}{\selectlanguage{french}Scier (du bois).} \zh{锯(木头)。} 
\lhead{\firstmark}
\rhead{\botmark}

\subsection{\hspace{-0.5cm} {\Large \textcolor{darkblue}{\textbf{\ipa{kɤ˧˥}}}}\hspace{0.5cm}[\kern2pt{\textcolor{darkblue}{\textbf{\ipa{kɤ˧˥}}}}\kern2pt]} \hypertarget{k7\string_M\string_T1}{}
\markboth{\textcolor{darkblue}{\textbf{\ipa{kɤ˧˥}}}}{}
\textcolor{teal}{\mytextsc{verbe}} \hspace{4pt} Ton~: MH.
\textcolor{PineGreen}{\selectlanguage{french}Frapper à la porte, heurter à la porte.} \zh{敲门。}  ¶ \textcolor{darkblue}{\textbf{\ipa{tʰi˧-kɤ˧˥}}} \textcolor{PineGreen}{\selectlanguage{french}\mytextsc{dur}} \zh{\mytextsc{dur}}  
\textit{Voir~:} \textcolor{darkblue}{\textbf{\ipa{kɤ˩kɤ˧˥}}} 
\lhead{\firstmark}
\rhead{\botmark}

\subsection{\hspace{-0.5cm} {\Large \textcolor{darkblue}{\textbf{\ipa{kɤ˧˥\textsubscript{a}}}} \textsubscript{1}}\hspace{0.5cm}[\kern2pt{\textcolor{darkblue}{\textbf{\ipa{kɤ˧˥}}}}\kern2pt]} \hypertarget{k7\string_M\string_Ta1}{}
\markboth{\textcolor{darkblue}{\textbf{\ipa{kɤ˧˥\textsubscript{a}}}} \textsubscript{1}}{}
\textcolor{teal}{\mytextsc{classificateur}} \hspace{4pt} Ton~: MH\textsubscript{a}.
\textcolor{PineGreen}{\selectlanguage{french}Classificateur des bâtons.} \zh{量词:棍子、树枝(一根)。}  ¶ \textcolor{darkblue}{\textbf{\ipa{si˧-kɤ˧˥ | ɖɯ˧-kɤ˧˥}}} \textcolor{PineGreen}{\selectlanguage{french}une branche (d'arbre)} \zh{一根树枝}  

\lhead{\firstmark}
\rhead{\botmark}

\subsection{\hspace{-0.5cm} {\Large \textcolor{darkblue}{\textbf{\ipa{kɤ˧˥\textsubscript{a}}}} \textsubscript{2}}\hspace{0.5cm}[\kern2pt{\textcolor{darkblue}{\textbf{\ipa{kɤ˧˥}}}}\kern2pt]} \hypertarget{k7\string_M\string_Ta2}{}
\markboth{\textcolor{darkblue}{\textbf{\ipa{kɤ˧˥\textsubscript{a}}}} \textsubscript{2}}{}
\textcolor{teal}{\mytextsc{classificateur}} \hspace{4pt} Ton~: MH\textsubscript{a}.
\textcolor{PineGreen}{\selectlanguage{french}Une étendue de terre.} \zh{量词:地(一片)。} 
\lhead{\firstmark}
\rhead{\botmark}

\subsection{\hspace{-0.5cm} {\Large \textcolor{darkblue}{\textbf{\ipa{kɤ˧˥tʰɑ˩}}}}\hspace{0.5cm}[\kern2pt{\textcolor{darkblue}{\textbf{\ipa{kɤ˧tʰɑ˧˥}}}}\kern2pt]} \hypertarget{k7\string_M\string_Tt\string_hA\string_B1}{}
\markboth{\textcolor{darkblue}{\textbf{\ipa{kɤ˧˥tʰɑ˩}}}}{}
\textcolor{teal}{\mytextsc{nom}} \hspace{4pt} Ton~: MH+L.
\textcolor{PineGreen}{\selectlanguage{french}Nom de clan/famille étendue. Deux familles portent ce nom à Yongning. C'est l'un des trois premiers clans à s'être établis à proximité du monastère de Yongning, les deux autres étant \textcolor{darkblue}{\textbf{\ipa{/ə˧lɑ˧/}}} et \textcolor{darkblue}{\textbf{\ipa{/lɑ˧tʰɑ˧mi˥\$/}}}.} \zh{一个姓。这个姓,永宁有两家。}  ¶ \textcolor{darkblue}{\textbf{\ipa{kɤ˧˥tʰɑ˩=ɻ̍˩}}} \textcolor{PineGreen}{\selectlanguage{french}le clan \textcolor{darkblue}{\textbf{\ipa{/kɤ˧˥tʰɑ˩/}}}} \zh{\textcolor{darkblue}{\textbf{\ipa{/kɤ˧˥tʰɑ˩/}}}家族}  

\lhead{\firstmark}
\rhead{\botmark}

\subsection{\hspace{-0.5cm} {\Large \textcolor{darkblue}{\textbf{\ipa{kɤ˩˧}}}}\hspace{0.5cm}[\kern2pt{\textcolor{darkblue}{\textbf{\ipa{kɤ˩˥}}}}\kern2pt]} \hypertarget{k7\string_B\string_M1}{}
\markboth{\textcolor{darkblue}{\textbf{\ipa{kɤ˩˧}}}}{}
\textcolor{teal}{\mytextsc{nom}} \hspace{4pt} Ton~: LM.
\textcolor{PineGreen}{\selectlanguage{french}Buse, faucon.} \zh{隼、“小鹰”。}  ¶ \textcolor{darkblue}{\textbf{\ipa{kɤ˩ hwæ˧-ze˧}}} \textcolor{PineGreen}{\selectlanguage{french}...a acheté (un) faucon} \zh{买了隼}  
 ¶ \textcolor{darkblue}{\textbf{\ipa{kɤ˩ dzɯ˧-ze˩}}} \textcolor{PineGreen}{\selectlanguage{french}...a mangé (un) faucon} \zh{吃了隼}  
 \zh{量词}: \textcolor{darkblue}{\textbf{\ipa{mi˩}}}  \mytextsc{clf}~: \textcolor{darkblue}{\textbf{\ipa{mi˩}}} 
\lhead{\firstmark}
\rhead{\botmark}

\subsection{\hspace{-0.5cm} {\Large \textcolor{darkblue}{\textbf{\ipa{ki˥\textsubscript{a}}}}}\hspace{0.5cm}[\kern2pt{\textcolor{darkblue}{\textbf{\ipa{ki˥}}}}\kern2pt]} \hypertarget{ki\string_Ta1}{}
\markboth{\textcolor{darkblue}{\textbf{\ipa{ki˥\textsubscript{a}}}}}{}
\textcolor{teal}{\mytextsc{classificateur}} \hspace{4pt} Ton~: H\textsubscript{a}.
\textcolor{PineGreen}{\selectlanguage{french}En association avec le numéral 'un', ce classificateur signifie 'ensemble'.} \zh{量词:加上数词‘一’,这个量词表示‘一起’。}  ¶ \textcolor{darkblue}{\textbf{\ipa{ɖɯ˧-ki˥}}} \textcolor{PineGreen}{\selectlanguage{french}ensemble} \zh{一起(共事)}  
 ¶ \textcolor{darkblue}{\textbf{\ipa{ɖɯ˧-ki˧ tʰv̩˧}}} \textcolor{PineGreen}{\selectlanguage{french}arriver ensemble/en même temps} \zh{同时到达}  
 ¶ \textcolor{darkblue}{\textbf{\ipa{ɖɯ˧-ʝi˧-ɳɯ˧ tsʰɯ˧˥, | ɖɯ˧-ki˧ tʰv̩˧!}}} \textcolor{PineGreen}{\selectlanguage{french}Venant du même endroit, (on) arrive ensemble!} \zh{从一个地方,一起到!}  
 ¶ \textcolor{darkblue}{\textbf{\ipa{ɖɯ˧-ki˧ dzi˧˥}}} \textcolor{PineGreen}{\selectlanguage{french}habiter ensemble} \zh{住在一起}  

\lhead{\firstmark}
\rhead{\botmark}

\subsection{\hspace{-0.5cm} {\Large \textcolor{darkblue}{\textbf{\ipa{‑ki˧}}}}\hspace{0.5cm}[\kern2pt{\textcolor{darkblue}{\textbf{\ipa{ki˥}}}}\kern2pt]} \hypertarget{‑ki\string_M1}{}
\markboth{\textcolor{darkblue}{\textbf{\ipa{‑ki˧}}}}{}
\textcolor{teal}{\mytextsc{suffixe}} \hspace{4pt} Ton~: M.
\textcolor{PineGreen}{\selectlanguage{french}Datif/allatif.} \zh{对\mytextsc{与格}/向格。}  ¶ \textcolor{darkblue}{\textbf{\ipa{ʈʂʰɯ˧-ki˧ ʐwɤ˧˥}}} \textcolor{PineGreen}{\selectlanguage{french}lui dire, lui parler} \zh{给他讲}  
 ¶ \textcolor{darkblue}{\textbf{\ipa{ʈʂʰɯ˧-ki˧ ʐwɤ˧-bi˥}}} \textcolor{PineGreen}{\selectlanguage{french}idem+futur immédiat} \zh{要给他讲}  
 ¶ \textcolor{darkblue}{\textbf{\ipa{ə˧mɑ˧-ɳɯ˧ | njɤ˧-ki˧ | nɑ˩-ʐwɤ˧ so˩!}}} \textcolor{PineGreen}{\selectlanguage{french}Ama m'enseigne la langue na!} \zh{阿妈教我摩梭话!}  

\lhead{\firstmark}
\rhead{\botmark}

\subsection{\hspace{-0.5cm} {\Large \textcolor{darkblue}{\textbf{\ipa{ki˧\textsubscript{a}}}}}\hspace{0.5cm}[\kern2pt{\textcolor{darkblue}{\textbf{\ipa{ki˥}}}}\kern2pt]} \hypertarget{ki\string_Ma1}{}
\markboth{\textcolor{darkblue}{\textbf{\ipa{ki˧\textsubscript{a}}}}}{}
\textcolor{teal}{\mytextsc{verbe}} \hspace{4pt} Ton~: M\textsubscript{a}.
\textcolor{PineGreen}{\selectlanguage{french}Donner, passer, transmettre.} \zh{给、传、献给、发(工资),嫁给。}  ¶ \textcolor{darkblue}{\textbf{\ipa{ki˧\textasciitilde{}ki˩}}} \textcolor{PineGreen}{\selectlanguage{french}\mytextsc{red}} \zh{重叠}  
 ¶ \textcolor{darkblue}{\textbf{\ipa{tso˧\textasciitilde{}tso˧-ki˩}}} \textcolor{PineGreen}{\selectlanguage{french}donner des choses} \zh{给东西}  
 ¶ \textcolor{darkblue}{\textbf{\ipa{tso˧\textasciitilde{}tso˧ ki˧\textasciitilde{}ki˥}}} \textcolor{PineGreen}{\selectlanguage{french}donner des choses} \zh{给东西}  
 ¶ \textcolor{darkblue}{\textbf{\ipa{hĩ˧-ki˧ ki˩}}} \textcolor{PineGreen}{\selectlanguage{french}1. donner à quelqu'un. 2. se donner en mariage à quelqu'un (pour une femme)} \zh{1.许配给人家。2.嫁给人}  
 ¶ \textcolor{darkblue}{\textbf{\ipa{hĩ˧-ki˧ | ɖwæ˧˥ | tʰi˧-ki˧}}} \textcolor{PineGreen}{\selectlanguage{french}être généreux} \zh{大方}  
 ¶ \textcolor{darkblue}{\textbf{\ipa{hĩ˧-ki˧ ki˩ fv̩˩}}} \textcolor{PineGreen}{\selectlanguage{french}qui aime faire des cadeaux, qui aime donner des choses aux gens} \zh{爱送礼,爱给别人送东西}  
 ¶ \textcolor{darkblue}{\textbf{\ipa{pʰɤ˧bɤ˧ ki˧ (-bi˧)}}} \textcolor{PineGreen}{\selectlanguage{french}donner des cadeaux} \zh{送礼物}  
 ¶ \textcolor{darkblue}{\textbf{\ipa{hɑ˧ ki˩}}} \textcolor{PineGreen}{\selectlanguage{french}nourrir, donner à manger} \zh{喂饭}  

\lhead{\firstmark}
\rhead{\botmark}

\subsection{\hspace{-0.5cm} {\Large \textcolor{darkblue}{\textbf{\ipa{ki˧li˥}}}}\hspace{0.5cm}[\kern2pt{\textcolor{darkblue}{\textbf{\ipa{ki˧li˥}}}}\kern2pt]} \hypertarget{ki\string_Mli\string_T1}{}
\markboth{\textcolor{darkblue}{\textbf{\ipa{ki˧li˥}}}}{}
\textcolor{teal}{\mytextsc{adverbe}} \hspace{4pt} Ton~: H\#.
\textcolor{PineGreen}{\selectlanguage{french}En désordre (formulation expressive, quasi-onomatopéique).} \zh{乱七八糟。} 
\lhead{\firstmark}
\rhead{\botmark}

\subsection{\hspace{-0.5cm} {\Large \textcolor{darkblue}{\textbf{\ipa{ki˧zo\#˥}}}}\hspace{0.5cm}[\kern2pt{\textcolor{darkblue}{\textbf{\ipa{ki˧zo˧}}}}\kern2pt]} \hypertarget{ki\string_Mzo\#\string_T1}{}
\markboth{\textcolor{darkblue}{\textbf{\ipa{ki˧zo\#˥}}}}{}
\textcolor{teal}{\mytextsc{nom}} \hspace{4pt} Ton~: \#H.
\textcolor{PineGreen}{\selectlanguage{french}Prénom unisexe: prénom utilisé pour les deux sexes.} \zh{男女通用名。} 
\lhead{\firstmark}
\rhead{\botmark}

\subsection{\hspace{-0.5cm} {\Large \textcolor{darkblue}{\textbf{\ipa{ki˩\textsubscript{a}}}}}\hspace{0.5cm}[\kern2pt{\textcolor{darkblue}{\textbf{\ipa{ki˩˥}}}}\kern2pt]} \hypertarget{ki\string_Ba1}{}
\markboth{\textcolor{darkblue}{\textbf{\ipa{ki˩\textsubscript{a}}}}}{}
\textcolor{teal}{\mytextsc{verbe}} \hspace{4pt} Ton~: L\textsubscript{a}.
\textcolor{PineGreen}{\selectlanguage{french}Enfiler, porter, mettre (une robe, un pantalon...).} \zh{穿上(裤子、袜子、鞋子)。}  ¶ \textcolor{darkblue}{\textbf{\ipa{ɬi˧qʰwɤ˩ | ɖɯ˧-ɭɯ˧ ki˩}}} \textcolor{PineGreen}{\selectlanguage{french}enfiler un pantalon} \zh{穿上裤子}  
 ¶ \textcolor{darkblue}{\textbf{\ipa{dzɑ˩qʰwɤ˩˥ | ɖɯ˧-dzi˧ ki˩}}} \textcolor{PineGreen}{\selectlanguage{french}enfiler une paire de chaussures} \zh{穿上一双鞋}  
 ¶ \textcolor{darkblue}{\textbf{\ipa{ʈʰæ˩ ki˩˥}}} \textcolor{PineGreen}{\selectlanguage{french}porter la jupe; nom du rituel de passage à l'âge adulte (à treize ans révolus) pour les jeunes filles} \zh{“穿裙”:这是成年礼的名称(穿裙礼:女孩满13岁,即为成人)}  
 ¶ \textcolor{darkblue}{\textbf{\ipa{ɬi˧ ki˥}}} \textcolor{PineGreen}{\selectlanguage{french}porter le pantalon; nom du rituel de passage à l'âge adulte (à treize ans révolus) pour les jeunes gens} \zh{“穿裤”:这是成年礼的名称(穿裤礼:男孩满了13岁,即为成人)}  

\lhead{\firstmark}
\rhead{\botmark}

\subsection{\hspace{-0.5cm} {\Large \textcolor{darkblue}{\textbf{\ipa{ki˩mi˧}}}}\hspace{0.5cm}[\kern2pt{\textcolor{darkblue}{\textbf{\ipa{ki˩mi˥}}}}\kern2pt]} \hypertarget{ki\string_Bmi\string_M1}{}
\markboth{\textcolor{darkblue}{\textbf{\ipa{ki˩mi˧}}}}{}
\textcolor{teal}{\mytextsc{nom}} \hspace{4pt} Ton~: LM.
\textcolor{PineGreen}{\selectlanguage{french}Grosse mouche, dont les larves sont redoutées.} \zh{绿头苍蝇。}  \zh{量词}: \textcolor{darkblue}{\textbf{\ipa{mi˩}}}  \mytextsc{clf}~: \textcolor{darkblue}{\textbf{\ipa{mi˩}}} 
\lhead{\firstmark}
\rhead{\botmark}

\subsection{\hspace{-0.5cm} {\Large \textcolor{darkblue}{\textbf{\ipa{ki˩tɑ\#˥}}}}\hspace{0.5cm}[\kern2pt{\textcolor{darkblue}{\textbf{\ipa{ki˩tɑ˥}}}}\kern2pt]} \hypertarget{ki\string_BtA\#\string_T1}{}
\markboth{\textcolor{darkblue}{\textbf{\ipa{ki˩tɑ\#˥}}}}{}
\textcolor{teal}{\mytextsc{nom}} \hspace{4pt} Ton~: LM+\#H.
\textcolor{PineGreen}{\selectlanguage{french}Sac de cuir et de lin, dans lequel on plaçait ce que possédait la maisonnée: or, argent… qu'on enterrait quelque part dans la maison, pour se prémunir contre les voleurs; les matières choisies, cuir et lin, se conservaient très longtemps; le sac avait quatre ou cinq épaisseurs de tissu, pour le rendre plus solide.} \zh{皮袋,来装家里的财物:金币、银币……这个皮袋,埋在房子里的一个保密地方,防贼。为了让它很结实,袋子有四、五层麻布内衬。可以保存很长时间。}  ¶ \textcolor{darkblue}{\textbf{\ipa{æ˧-tse˥pʰæ˩ | ɖɯ˧-ki˩tɑ˩}}} \textcolor{PineGreen}{\selectlanguage{french}un sac de pièces de cuivre} \zh{一袋铜币}  

\lhead{\firstmark}
\rhead{\botmark}

\subsection{\hspace{-0.5cm} {\Large \textcolor{darkblue}{\textbf{\ipa{ki˩ti\#˥}}}}\hspace{0.5cm}[\kern2pt{\textcolor{darkblue}{\textbf{\ipa{ki˩ti˥}}}}\kern2pt]} \hypertarget{ki\string_Bti\#\string_T1}{}
\markboth{\textcolor{darkblue}{\textbf{\ipa{ki˩ti\#˥}}}}{}
\textcolor{teal}{\mytextsc{nom}} \hspace{4pt} Ton~: LM+\#H.
\textcolor{PineGreen}{\selectlanguage{french}Ceinture en cuir.} \zh{皮腰带。}  \zh{量词}: \textcolor{darkblue}{\textbf{\ipa{kʰɯ˩}}}  \mytextsc{clf}~: \textcolor{darkblue}{\textbf{\ipa{kʰɯ˩}}} 
\lhead{\firstmark}
\rhead{\botmark}

\subsection{\hspace{-0.5cm} {\Large \textcolor{darkblue}{\textbf{\ipa{ko˥}}}}\hspace{0.5cm}[\kern2pt{\textcolor{darkblue}{\textbf{\ipa{ko˥}}}}\kern2pt]} \hypertarget{ko\string_T1}{}
\markboth{\textcolor{darkblue}{\textbf{\ipa{ko˥}}}}{}
\textcolor{teal}{\mytextsc{nom}} \hspace{4pt} Ton~: \#H.
\textcolor{PineGreen}{\selectlanguage{french}Colline, petite montagne.} \zh{小山。}  \zh{量词}: \textcolor{darkblue}{\textbf{\ipa{ɭɯ˧}}}  \mytextsc{clf}~: \textcolor{darkblue}{\textbf{\ipa{ɭɯ˧}}} 
\lhead{\firstmark}
\rhead{\botmark}

\subsection{\hspace{-0.5cm} {\Large \textcolor{darkblue}{\textbf{\ipa{ko˧\textsubscript{a}}}}}\hspace{0.5cm}[\kern2pt{\textcolor{darkblue}{\textbf{\ipa{ko˥}}}}\kern2pt]} \hypertarget{ko\string_Ma1}{}
\markboth{\textcolor{darkblue}{\textbf{\ipa{ko˧\textsubscript{a}}}}}{}
\textcolor{teal}{\mytextsc{classificateur}} \hspace{4pt} Ton~: M\textsubscript{a}.
\textcolor{PineGreen}{\selectlanguage{french}Classificateur des petits objets, tels que des cigarettes.} \zh{量词:小东西,例如烟(一只)。} 
\lhead{\firstmark}
\rhead{\botmark}

\subsection{\hspace{-0.5cm} {\Large \textcolor{darkblue}{\textbf{\ipa{ko˧ɖæ\#˥}}}}\hspace{0.5cm}[\kern2pt{\textcolor{darkblue}{\textbf{\ipa{ko˧ɖæ˧}}}}\kern2pt]} \hypertarget{ko\string_Md`\{\#\string_T1}{}
\markboth{\textcolor{darkblue}{\textbf{\ipa{ko˧ɖæ\#˥}}}}{}
\textcolor{teal}{\mytextsc{nom}} \hspace{4pt} Ton~: \#H.
\textcolor{PineGreen}{\selectlanguage{french}Statue de bouddha.} \zh{佛像。}  Emprunt~: tibétain  sku-vdra (sku-'dra)
 ¶ \textcolor{darkblue}{\textbf{\ipa{ko˧ɖæ˧-zo˧}}} \textcolor{PineGreen}{\selectlanguage{french}statuette du Bouddha} \zh{小佛像}  
 \zh{量词}: \textcolor{darkblue}{\textbf{\ipa{ɭɯ˧}}}  \mytextsc{clf}~: \textcolor{darkblue}{\textbf{\ipa{ɭɯ˧}}} 
\lhead{\firstmark}
\rhead{\botmark}

\subsection{\hspace{-0.5cm} {\Large \textcolor{darkblue}{\textbf{\ipa{ko˧li\#˥}}}}\hspace{0.5cm}[\kern2pt{\textcolor{darkblue}{\textbf{\ipa{ko˧li˧}}}}\kern2pt]} \hypertarget{ko\string_Mli\#\string_T1}{}
\markboth{\textcolor{darkblue}{\textbf{\ipa{ko˧li\#˥}}}}{}
\textcolor{teal}{\mytextsc{nom}} \hspace{4pt} Ton~: \#H.
\textcolor{PineGreen}{\selectlanguage{french}Soufflet à bouche: un tube dans lequel on souffle pour attiser le feu.} \zh{吹火筒,用来吹火的小管子。}  \zh{量词}: \textcolor{darkblue}{\textbf{\ipa{ɭɯ˧}}}  \mytextsc{clf}~: \textcolor{darkblue}{\textbf{\ipa{ɭɯ˧}}} 
\lhead{\firstmark}
\rhead{\botmark}

\subsection{\hspace{-0.5cm} {\Large \textcolor{darkblue}{\textbf{\ipa{ko˧no˧-ʁo\#˥}}}}\hspace{0.5cm}[\kern2pt{\textcolor{darkblue}{\textbf{\ipa{xxxx non-correspondance entre le nombre de morphèmes et le nombre de tons de morphèmes}}}}\kern2pt]} \hypertarget{ko\string_Mno\string_M-Ro\#\string_T1}{}
\markboth{\textcolor{darkblue}{\textbf{\ipa{ko˧no˧-ʁo\#˥}}}}{}
\textcolor{teal}{\mytextsc{nom}} \hspace{4pt} Ton~: \#H.
\textcolor{PineGreen}{\selectlanguage{french}Crête, ligne de faîte (en montagne).} \zh{山梁。}  \zh{量词}: \textcolor{darkblue}{\textbf{\ipa{kʰwɤ˥}}}  \mytextsc{clf}~: \textcolor{darkblue}{\textbf{\ipa{kʰwɤ˥}}} 
\lhead{\firstmark}
\rhead{\botmark}

\subsection{\hspace{-0.5cm} {\Large \textcolor{darkblue}{\textbf{\ipa{ko˧sɯ\#˥}}}}\hspace{0.5cm}[\kern2pt{\textcolor{darkblue}{\textbf{\ipa{ko˧sɯ˧}}}}\kern2pt]} \hypertarget{ko\string_MsM\#\string_T1}{}
\markboth{\textcolor{darkblue}{\textbf{\ipa{ko˧sɯ\#˥}}}}{}
\textcolor{teal}{\mytextsc{nom}} \hspace{4pt} Ton~: \#H.
\textcolor{PineGreen}{\selectlanguage{french}Boutique.} \zh{商店、小卖部(汉语借词:公司)。}  Emprunt~: chinois  \zh{公司}

\lhead{\firstmark}
\rhead{\botmark}

\subsection{\hspace{-0.5cm} {\Large \textcolor{darkblue}{\textbf{\ipa{ko˩\textsubscript{a}}}}}\hspace{0.5cm}[\kern2pt{\textcolor{darkblue}{\textbf{\ipa{ko˩˥}}}}\kern2pt]} \hypertarget{ko\string_Ba1}{}
\markboth{\textcolor{darkblue}{\textbf{\ipa{ko˩\textsubscript{a}}}}}{}
\textcolor{teal}{\mytextsc{verbe}} \hspace{4pt} Ton~: L\textsubscript{a}.
\textcolor{PineGreen}{\selectlanguage{french}Se chauffer au feu ou au soleil; prendre le soleil.} \zh{烤火取暖,晒太阳。}  ¶ \textcolor{darkblue}{\textbf{\ipa{mv̩˧ ko˥}}} \textcolor{PineGreen}{\selectlanguage{french}se chauffer au feu} \zh{烤火取暖}  
 ¶ \textcolor{darkblue}{\textbf{\ipa{le˧-ko˩-ze˩}}} \textcolor{PineGreen}{\selectlanguage{french}\mytextsc{accomp} \string_ \mytextsc{pfv}} \zh{烤火了}  
 ¶ \textcolor{darkblue}{\textbf{\ipa{ɲi˧mi˧ ko˩}}} \textcolor{PineGreen}{\selectlanguage{french}se réchauffer au soleil} \zh{晒太阳}  
 ¶ \textcolor{darkblue}{\textbf{\ipa{ɲi˧mi˧ ɖɯ˧-ko˩-ɻ̍˩}}} \textcolor{PineGreen}{\selectlanguage{french}se réchauffer un moment au soleil} \zh{晒晒太阳}  
 ¶ \textcolor{darkblue}{\textbf{\ipa{ɲi˧mi˧ ɖɯ˧-ko˧\textasciitilde{}ko˥-ɻ̍˩}}} \textcolor{PineGreen}{\selectlanguage{french}se réchauffer un moment au soleil} \zh{晒晒太阳}  

\lhead{\firstmark}
\rhead{\botmark}

\subsection{\hspace{-0.5cm} {\Large \textcolor{darkblue}{\textbf{\ipa{ko˩dze˧}}}}\hspace{0.5cm}[\kern2pt{\textcolor{darkblue}{\textbf{\ipa{ko˩dze˥}}}}\kern2pt]} \hypertarget{ko\string_Bdze\string_M1}{}
\markboth{\textcolor{darkblue}{\textbf{\ipa{ko˩dze˧}}}}{}
\textcolor{teal}{\mytextsc{nom}} \hspace{4pt} Ton~: LM.
\textcolor{PineGreen}{\selectlanguage{french}Sorte de colombe.} \zh{一种鸽子。}  \zh{量词}: \textcolor{darkblue}{\textbf{\ipa{mi˩}}}  \mytextsc{clf}~: \textcolor{darkblue}{\textbf{\ipa{mi˩}}} 
\lhead{\firstmark}
\rhead{\botmark}

\subsection{\hspace{-0.5cm} {\Large \textcolor{darkblue}{\textbf{\ipa{ko˩ɖʐo˩}}}}\hspace{0.5cm}[\kern2pt{\textcolor{darkblue}{\textbf{\ipa{ko˩ɖʐo˩˥}}}}\kern2pt]} \hypertarget{ko\string_Bd`z`o\string_B1}{}
\markboth{\textcolor{darkblue}{\textbf{\ipa{ko˩ɖʐo˩}}}}{}
\textcolor{teal}{\mytextsc{nom}} \hspace{4pt} Ton~: L.
\textcolor{PineGreen}{\selectlanguage{french}Fléau pour battre le grain.} \zh{连枷。}  \zh{量词}: \textcolor{darkblue}{\textbf{\ipa{nɑ˧}}}  \mytextsc{clf}~: \textcolor{darkblue}{\textbf{\ipa{nɑ˧}}} 
\lhead{\firstmark}
\rhead{\botmark}

\subsection{\hspace{-0.5cm} {\Large \textcolor{darkblue}{\textbf{\ipa{ko˩qʰɑ˧-dʑɯ\#˥}}}}\hspace{0.5cm}[\kern2pt{\textcolor{darkblue}{\textbf{\ipa{xxxx non-correspondance entre le nombre de morphèmes et le nombre de tons de morphèmes}}}}\kern2pt]} \hypertarget{ko\string_Bq\string_hA\string_M-dz£M\#\string_T1}{}
\markboth{\textcolor{darkblue}{\textbf{\ipa{ko˩qʰɑ˧-dʑɯ\#˥}}}}{}
\textcolor{teal}{\mytextsc{nom}} \hspace{4pt} Ton~: LM+\#H.
\textcolor{PineGreen}{\selectlanguage{french}Yyy.} \zh{金梅花。}  ¶ \textcolor{darkblue}{\textbf{\ipa{ko˩qʰɑ˧-dʑɯ˧-bæ˥bæ˩}}} \textcolor{PineGreen}{\selectlanguage{french}fleur de...} \zh{金梅花的花}  

\lhead{\firstmark}
\rhead{\botmark}

\subsection{\hspace{-0.5cm} {\Large \textcolor{darkblue}{\textbf{\ipa{ko˧˥}}} \textsubscript{1}}\hspace{0.5cm}[\kern2pt{\textcolor{darkblue}{\textbf{\ipa{ko˧˥}}}}\kern2pt]} \hypertarget{ko\string_M\string_T1}{}
\markboth{\textcolor{darkblue}{\textbf{\ipa{ko˧˥}}} \textsubscript{1}}{}
\textcolor{teal}{\mytextsc{adverbe}} \hspace{4pt} Ton~: MH.
\textcolor{PineGreen}{\selectlanguage{french}Trop, excessivement.} \zh{过于,太(汉语借词)。}  Emprunt~: chinois  \zh{过}

\lhead{\firstmark}
\rhead{\botmark}

\subsection{\hspace{-0.5cm} {\Large \textcolor{darkblue}{\textbf{\ipa{ko˧˥}}} \textsubscript{2}}\hspace{0.5cm}[\kern2pt{\textcolor{darkblue}{\textbf{\ipa{ko˧˥}}}}\kern2pt]} \hypertarget{ko\string_M\string_T2}{}
\markboth{\textcolor{darkblue}{\textbf{\ipa{ko˧˥}}} \textsubscript{2}}{}
\textcolor{teal}{\mytextsc{verbe}} \hspace{4pt} Ton~: MH.
\textcolor{PineGreen}{\selectlanguage{french}Se passer, avoir lieu: les jours passent, la vie se passe; couler (des jours/des années).} \zh{过(汉语借词)。}  Emprunt~: chinois  \zh{过}
 ¶ \textcolor{darkblue}{\textbf{\ipa{se˧ʐɯ˩ ko˩}}} \textcolor{PineGreen}{\selectlanguage{french}fêter un anniversaire} \zh{过生日}  

\lhead{\firstmark}
\rhead{\botmark}

\subsection{\hspace{-0.5cm} {\Large \textcolor{darkblue}{\textbf{\ipa{kɯ˥}}}}\hspace{0.5cm}[\kern2pt{\textcolor{darkblue}{\textbf{\ipa{kɯ˥}}}}\kern2pt]} \hypertarget{kM\string_T1}{}
\markboth{\textcolor{darkblue}{\textbf{\ipa{kɯ˥}}}}{}
\textcolor{teal}{\mytextsc{nom}} \hspace{4pt} Ton~: \#H.
\ding{202} \textcolor{PineGreen}{\selectlanguage{french}Vésicule biliaire.} \zh{胆。}  \zh{量词}: \textcolor{darkblue}{\textbf{\ipa{ɭɯ˧}}} \ding{203} \textcolor{PineGreen}{\selectlanguage{french}Bile.} \zh{胆汁。}  \mytextsc{clf}~: \textcolor{darkblue}{\textbf{\ipa{ɭɯ˧}}} 
\lhead{\firstmark}
\rhead{\botmark}

\subsection{\hspace{-0.5cm} {\Large \textcolor{darkblue}{\textbf{\ipa{kɯ˥}}}}\hspace{0.5cm}[\kern2pt{\textcolor{darkblue}{\textbf{\ipa{kɯ˥}}}}\kern2pt]} \hypertarget{kM\string_T1}{}
\markboth{\textcolor{darkblue}{\textbf{\ipa{kɯ˥}}}}{}
\textcolor{teal}{\mytextsc{adjectif}} \hspace{4pt} Ton~: H.
\textcolor{PineGreen}{\selectlanguage{french}Serré, tendu.} \zh{紧。}  ¶ \textcolor{darkblue}{\textbf{\ipa{le˧-tsɯ˥ | le˧-kɯ˥-kʰɯ˩}}} \textcolor{PineGreen}{\selectlanguage{french}attacher très serré} \zh{绑紧}  
 ¶ \textcolor{darkblue}{\textbf{\ipa{le˧-kɯ˥-se˩}}} \textcolor{PineGreen}{\selectlanguage{french}\mytextsc{accomp} \string_ \mytextsc{pfv}} \zh{紧了}  

\lhead{\firstmark}
\rhead{\botmark}

\subsection{\hspace{-0.5cm} {\Large \textcolor{darkblue}{\textbf{\ipa{kɯ˧}}}}\hspace{0.5cm}[\kern2pt{\textcolor{darkblue}{\textbf{\ipa{kɯ˥}}}}\kern2pt]} \hypertarget{kM\string_M1}{}
\markboth{\textcolor{darkblue}{\textbf{\ipa{kɯ˧}}}}{}
\textcolor{teal}{\mytextsc{nom}} \hspace{4pt} Ton~: M.
\textcolor{PineGreen}{\selectlanguage{french}Étoile.} \zh{星星。}  ¶ \textcolor{darkblue}{\textbf{\ipa{mv̩˧ʁo˥ | kɯ˧}}} \textcolor{PineGreen}{\selectlanguage{french}le ciel est étoilé, on voit les étoiles du ciel} \zh{天上有星星、天上看得见星星}  
 ¶ \textcolor{darkblue}{\textbf{\ipa{nɑ˩-ʈʂʰɯ˥, | kɯ˧ mɤ˧-li˧! | di˧mi˧-lɑ˧ li˥!}}} \textcolor{PineGreen}{\selectlanguage{french}Les Na, ils ne regardent pas les étoiles! Ils ne regardent que la plaine (=leur plaine: la plaine de Yongning)! (commentaire de F4 au sujet de son ignorance des noms d'étoiles et de constellations)} \zh{摩梭呢,不看星星,只看平坝(=永宁坝子)!(合作人说明为什么她不知道星星、星座的名字:摩梭人本来对天文不太感兴趣。)}  
 \zh{量词}: \textcolor{darkblue}{\textbf{\ipa{ɭɯ˧, kɯ˧}}}  \mytextsc{clf}~: \textcolor{darkblue}{\textbf{\ipa{ɭɯ˧, kɯ˧}}} \textcolor{PineGreen}{\selectlanguage{french}objets ronds}
\lhead{\firstmark}
\rhead{\botmark}

\subsection{\hspace{-0.5cm} {\Large \textcolor{darkblue}{\textbf{\ipa{kɯ˧\textsubscript{b}}}}}\hspace{0.5cm}[\kern2pt{\textcolor{darkblue}{\textbf{\ipa{kɯ˩˥}}}}\kern2pt]} \hypertarget{kM\string_Mb1}{}
\markboth{\textcolor{darkblue}{\textbf{\ipa{kɯ˧\textsubscript{b}}}}}{}
\textcolor{teal}{\mytextsc{classificateur}} \hspace{4pt} Ton~: M\textsubscript{b}.
\textcolor{PineGreen}{\selectlanguage{french}Auto-classificateur des étoiles; classificateur des jours propices.} \zh{量词:星星(一个)。} 
\lhead{\firstmark}
\rhead{\botmark}

\subsection{\hspace{-0.5cm} {\Large \textcolor{darkblue}{\textbf{\ipa{kɯ˧ɭɯ˧}}}}\hspace{0.5cm}[\kern2pt{\textcolor{darkblue}{\textbf{\ipa{kɯ˧ɭɯ˧˥}}}}\kern2pt]} \hypertarget{kM\string_Ml\string_RM\string_M1}{}
\markboth{\textcolor{darkblue}{\textbf{\ipa{kɯ˧ɭɯ˧}}}}{}
\textcolor{teal}{\mytextsc{nom}} \hspace{4pt} Ton~: M.
\textcolor{PineGreen}{\selectlanguage{french}Esprit bienfaisant.} \zh{神。}  ¶ \textcolor{darkblue}{\textbf{\ipa{kɯ˧ɭɯ˧ | ɖɯ˧-dze˩}}} \textcolor{PineGreen}{\selectlanguage{french}deux esprits bienfaisants} \zh{两个(好)神}  
 \zh{量词}: \textcolor{darkblue}{\textbf{\ipa{dze˩}}}  \mytextsc{clf}~: \textcolor{darkblue}{\textbf{\ipa{dze˩}}} 
\lhead{\firstmark}
\rhead{\botmark}

\subsection{\hspace{-0.5cm} {\Large \textcolor{darkblue}{\textbf{\ipa{kɯ˧qʰæ˧ʂe˧˥}}}}\hspace{0.5cm}[\kern2pt{\textcolor{darkblue}{\textbf{\ipa{kɯ˧qʰæ˧ʂe˧˥}}}}\kern2pt]} \hypertarget{kM\string_Mq\string_h\{\string_Ms`e\string_M\string_T1}{}
\markboth{\textcolor{darkblue}{\textbf{\ipa{kɯ˧qʰæ˧ʂe˧˥}}}}{}
\textcolor{teal}{\mytextsc{nom}} \hspace{4pt} Ton~: MH\#.
\textcolor{PineGreen}{\selectlanguage{french}Comète (littéralement “les étoiles défèquent”).} \zh{流星。}  \zh{量词}: \textcolor{darkblue}{\textbf{\ipa{ʂɯ˩}}}  \mytextsc{clf}~: \textcolor{darkblue}{\textbf{\ipa{ʂɯ˩}}} \textcolor{PineGreen}{\selectlanguage{french}fois}
\lhead{\firstmark}
\rhead{\botmark}

\subsection{\hspace{-0.5cm} {\Large \textcolor{darkblue}{\textbf{\ipa{kɯ˩\textsubscript{a}}}}}\hspace{0.5cm}[\kern2pt{\textcolor{darkblue}{\textbf{\ipa{kɯ˧˥}}}}\kern2pt]} \hypertarget{kM\string_Ba1}{}
\markboth{\textcolor{darkblue}{\textbf{\ipa{kɯ˩\textsubscript{a}}}}}{}
\textcolor{teal}{\mytextsc{verbe}} \hspace{4pt} Ton~: L\textsubscript{a}.
\textcolor{PineGreen}{\selectlanguage{french}Laisser quelqu'un en rade, laisser quelqu'un seul au moment où il aurait besoin d'aide, faire mine d'ignorer quelqu'un qui aurait besoin d'aide, négliger d'assister quelqu'un. Il n'a pas été trouvé d'équivalent chinois simple pour ce terme, qui renvoie à la situation où un manque de réel sympathie pour quelqu'un se traduit par le fait qu'on n'est pas tenté de faire l'effort de l'aider lorsqu'il en aurait besoin: on fait alors comme si de rien n'était, comme si on n'était pas au courant de la situation de cette personne.} \zh{不理需要帮忙的人:知道一个人需要帮助,自己也有能力帮忙,但假装没看见、什么事没有。}  ¶ \textcolor{darkblue}{\textbf{\ipa{hĩ˧ kɯ˥}}} \textcolor{PineGreen}{\selectlanguage{french}même sens} \zh{同上}  
 ¶ \textcolor{darkblue}{\textbf{\ipa{hĩ˧-ɳɯ˩ | kɯ˩-kv̩˥!}}} \textcolor{PineGreen}{\selectlanguage{french}Il arrive que les gens te laissent tomber! / Il arrive que les gens ne t'apportent pas leur aide quand tu en aurais besoin!} \zh{人家在你需要帮忙的时候就会不理你的!(如果处不好关系,人家对你没有什么好感,到时候你需要帮忙他们就不理你了。)}  
 ¶ \textcolor{darkblue}{\textbf{\ipa{kɯ˩-mɤ˩-kv̩˥!}}} \textcolor{PineGreen}{\selectlanguage{french}(ils/elles) n(e t)'aideront pas!} \zh{人家在你需要帮忙的时候就会不理你的!}  
 ¶ \textcolor{darkblue}{\textbf{\ipa{hĩ˧-ɳɯ˩ | kɯ˩-tʰɑ˩-kʰɯ˥!}}} \textcolor{PineGreen}{\selectlanguage{french}Fais en sorte que les gens ne te laissent pas en rade (quand tu auras besoin d'aide)! (Explication: il faut faire en sorte de gagner une estime et une sympathie réelles de la part des gens que l'on connaît; de la sorte, ils vous aideront spontanément lorsque vous en aurez besoin. Sinon, leur manque de réelle sympathie se traduira par le fait qu'ils ne feront pas l'effort de donner un coup de main lorsqu'on en aura besoin: ils feront alors comme si de rien n'était, comme s'ils n'étaient pas conscients de notre besoin.)} \zh{别让人家(在你需要帮忙的时候)不理你!}  
 ¶ \textcolor{darkblue}{\textbf{\ipa{njɤ˧ | no˩ kɯ˩-hĩ˥ mɤ˩-ɲi˩! | njɤ˧ | no˧-ki˧ | dʑɤ˩-so˥-ɲi˩!}}} \textcolor{PineGreen}{\selectlanguage{french}Je ne te néglige pas du tout! (Bien au contraire:) je t'enseigne bien / je fais de mon mieux pour t'apprendre des choses! (Contexte imaginé: un étudiant s'estime négligé par un enseignant; celui-ci se rend compte du mécontentement de l'étudiant, et lui dit qu'il interprète mal.)} \zh{我不是不重视你!(刚好相反:)我是用心教你的 / 我努力教你最好的!(情景:一名学生认为老师忽视他,老师发现学生不高兴,就说明。)}  

\lhead{\firstmark}
\rhead{\botmark}

\subsection{\hspace{-0.5cm} {\Large \textcolor{darkblue}{\textbf{\ipa{kɯ˩ɻ̍˧}}}}\hspace{0.5cm}[\kern2pt{\textcolor{darkblue}{\textbf{\ipa{kɯ˩ɻ̍˥}}}}\kern2pt]} \hypertarget{kM\string_Br£`̍\string_M1}{}
\markboth{\textcolor{darkblue}{\textbf{\ipa{kɯ˩ɻ̍˧}}}}{}
\textcolor{teal}{\mytextsc{nom}} \hspace{4pt} Ton~: LM.
\textcolor{PineGreen}{\selectlanguage{french}Violon à deux cordes, erhu.} \zh{胡琴,二胡。}  ¶ \textcolor{darkblue}{\textbf{\ipa{kɯ˩ɻ̍˧ ʈɤ˧}}} \textcolor{PineGreen}{\selectlanguage{french}jouer du erhu} \zh{拉二胡}  
 \zh{量词}: \textcolor{darkblue}{\textbf{\ipa{nɑ˧}}}  \mytextsc{clf}~: \textcolor{darkblue}{\textbf{\ipa{nɑ˧}}} 
\lhead{\firstmark}
\rhead{\botmark}

\subsection{\hspace{-0.5cm} {\Large \textcolor{darkblue}{\textbf{\ipa{kv̩˧˥}}}}\hspace{0.5cm}[\kern2pt{\textcolor{darkblue}{\textbf{\ipa{kv̩˧˥}}}}\kern2pt]} \hypertarget{kv\string_=\string_M\string_T1}{}
\markboth{\textcolor{darkblue}{\textbf{\ipa{kv̩˧˥}}}}{}
\textcolor{teal}{\mytextsc{verbe}} \hspace{4pt} Ton~: MH.
\textcolor{PineGreen}{\selectlanguage{french}Pouvoir, être capable de, avoir la compétence pour (verbe de modalité épistémique).} \zh{会、有能力做。} 
\lhead{\firstmark}
\rhead{\botmark}

\subsection{\hspace{-0.5cm} {\Large \textcolor{darkblue}{\textbf{\ipa{kv̩˥}}}}\hspace{0.5cm}[\kern2pt{\textcolor{darkblue}{\textbf{\ipa{kv̩˥}}}}\kern2pt]} \hypertarget{kv\string_=\string_T1}{}
\markboth{\textcolor{darkblue}{\textbf{\ipa{kv̩˥}}}}{}
\textcolor{teal}{\mytextsc{nom}} \hspace{4pt} Ton~: \#H.
\textcolor{PineGreen}{\selectlanguage{french}Ail, \textit{Allium sativum}.} \zh{大蒜。}  \zh{量词}: \textcolor{darkblue}{\textbf{\ipa{ɭɯ˧}}} \textcolor{darkblue}{\textbf{\ipa{tsʰɤ˧˥}}}  \mytextsc{clf}~: \textcolor{darkblue}{\textbf{\ipa{ɭɯ˧}}} \textcolor{darkblue}{\textbf{\ipa{tsʰɤ˧˥}}} \textcolor{PineGreen}{\selectlanguage{french}gousse; tête}
\lhead{\firstmark}
\rhead{\botmark}

\subsection{\hspace{-0.5cm} {\Large \textcolor{darkblue}{\textbf{\ipa{kv̩˩\textsubscript{a}}}} \textsubscript{1}}\hspace{0.5cm}[\kern2pt{\textcolor{darkblue}{\textbf{\ipa{kv̩˥}}}}\kern2pt]} \hypertarget{kv\string_=\string_Ba1}{}
\markboth{\textcolor{darkblue}{\textbf{\ipa{kv̩˩\textsubscript{a}}}} \textsubscript{1}}{}
\textcolor{teal}{\mytextsc{verbe}} \hspace{4pt} Ton~: L\textsubscript{a}.
\ding{202} \textcolor{PineGreen}{\selectlanguage{french}Ramasser; cueillir (des baies, des choses qu'on se baisse pour cueillir).} \zh{捡起来,拾。}  ¶ \textcolor{darkblue}{\textbf{\ipa{kv̩˧\textasciitilde{}kv̩˥}}} \textcolor{PineGreen}{\selectlanguage{french}\mytextsc{red}} \zh{\mytextsc{重叠}}  
 ¶ \textcolor{darkblue}{\textbf{\ipa{gɤ˩-kv̩˧\textasciitilde{}kv̩˥}}} \textcolor{PineGreen}{\selectlanguage{french}ramasser (quelque chose qui se trouvait à terre)} \zh{捡起来(地上的东西)}  
 ¶ \textcolor{darkblue}{\textbf{\ipa{le˧-kv̩˧\textasciitilde{}kv̩˥}}} \textcolor{PineGreen}{\selectlanguage{french}ramasser (quelque chose qui se trouvait à terre)} \zh{捡起来(地上的东西)}  
\ding{203} \textcolor{PineGreen}{\selectlanguage{french}Pêcher.} \zh{钓鱼。}  ¶ \textcolor{darkblue}{\textbf{\ipa{ɲi˧zo˧ kv̩˥}}} \textcolor{PineGreen}{\selectlanguage{french}pêcher du poisson} \zh{钓鱼}  

\lhead{\firstmark}
\rhead{\botmark}

\subsection{\hspace{-0.5cm} {\Large \textcolor{darkblue}{\textbf{\ipa{kv̩˩\textsubscript{a}}}} \textsubscript{2}}\hspace{0.5cm}[\kern2pt{\textcolor{darkblue}{\textbf{\ipa{kv̩˩˥}}}}\kern2pt]} \hypertarget{kv\string_=\string_Ba2}{}
\markboth{\textcolor{darkblue}{\textbf{\ipa{kv̩˩\textsubscript{a}}}} \textsubscript{2}}{}
\textcolor{teal}{\mytextsc{verbe}} \hspace{4pt} Ton~: L\textsubscript{a}.
\textcolor{PineGreen}{\selectlanguage{french}Traverser.} \zh{过。}  ¶ \textcolor{darkblue}{\textbf{\ipa{ʈʂʰwæ˩ kv̩˥}}} \textcolor{PineGreen}{\selectlanguage{french}traverser en bateau} \zh{坐船过(河)}  

\lhead{\firstmark}
\rhead{\botmark}

\subsection{\hspace{-0.5cm} {\Large \textcolor{darkblue}{\textbf{\ipa{kv̩˧dʑɯ˧˥}}}}\hspace{0.5cm}[\kern2pt{\textcolor{darkblue}{\textbf{\ipa{kv̩˧dʑɯ˧˥}}}}\kern2pt]} \hypertarget{kv\string_=\string_Mdz£M\string_M\string_T1}{}
\markboth{\textcolor{darkblue}{\textbf{\ipa{kv̩˧dʑɯ˧˥}}}}{}
\textcolor{teal}{\mytextsc{nom}} \hspace{4pt} Ton~: MH.
\textcolor{PineGreen}{\selectlanguage{french}Tente.} \zh{帐篷。}  ¶ \textcolor{darkblue}{\textbf{\ipa{kv̩˧dʑɯ˧ lɑ˥}}} \textcolor{PineGreen}{\selectlanguage{french}déplier une tente, installer une tente} \zh{安装帐篷、搭建帐篷}  
 \zh{量词}: \textcolor{darkblue}{\textbf{\ipa{nɑ˧}}}  \mytextsc{clf}~: \textcolor{darkblue}{\textbf{\ipa{nɑ˧}}} 
\lhead{\firstmark}
\rhead{\botmark}

\subsection{\hspace{-0.5cm} {\Large \textcolor{darkblue}{\textbf{\ipa{kv̩˧ʝi˥}}}}\hspace{0.5cm}[\kern2pt{\textcolor{darkblue}{\textbf{\ipa{kv̩˧ʝi˧˥}}}}\kern2pt]} \hypertarget{kv\string_=\string_Mj££i\string_T1}{}
\markboth{\textcolor{darkblue}{\textbf{\ipa{kv̩˧ʝi˥}}}}{}
\textcolor{teal}{\mytextsc{adverbe}} \hspace{4pt} Ton~: H\#.
\textcolor{PineGreen}{\selectlanguage{french}Pour de vrai, réellement.} \zh{真的、的确、确实。} 
\lhead{\firstmark}
\rhead{\botmark}

\subsection{\hspace{-0.5cm} {\Large \textcolor{darkblue}{\textbf{\ipa{kv̩˧ʝi˥\$}}}}\hspace{0.5cm}[\kern2pt{\textcolor{darkblue}{\textbf{\ipa{kv̩˧ʝi˥}}}}\kern2pt]} \hypertarget{kv\string_=\string_Mj££i\string_T\$1}{}
\markboth{\textcolor{darkblue}{\textbf{\ipa{kv̩˧ʝi˥\$}}}}{}
\textcolor{teal}{\mytextsc{nom}} \hspace{4pt} Ton~: H\$.
\textcolor{PineGreen}{\selectlanguage{french}Vie, existence.} \zh{生命。} 
\lhead{\firstmark}
\rhead{\botmark}

\subsection{\hspace{-0.5cm} {\Large \textcolor{darkblue}{\textbf{\ipa{kv̩˩kv̩˩}}}}\hspace{0.5cm}[\kern2pt{\textcolor{darkblue}{\textbf{\ipa{kv̩˩kv̩˩˥}}}}\kern2pt]} \hypertarget{kv\string_=\string_Bkv\string_=\string_B1}{}
\markboth{\textcolor{darkblue}{\textbf{\ipa{kv̩˩kv̩˩}}}}{}
\textcolor{teal}{\mytextsc{nom}} \hspace{4pt} Ton~: L.
\textcolor{PineGreen}{\selectlanguage{french}Pommettes.} \zh{颧骨。}  \zh{量词}: \textcolor{darkblue}{\textbf{\ipa{ɭɯ˧}}}  \mytextsc{clf}~: \textcolor{darkblue}{\textbf{\ipa{ɭɯ˧}}} \textit{Voir~:} \hyperlink{}{\textcolor{darkblue}{\textbf{\ipa{njɤ˧kv̩˩}}}} 
\lhead{\firstmark}
\rhead{\botmark}

\subsection{\hspace{-0.5cm} {\Large \textcolor{darkblue}{\textbf{\ipa{kv̩˧lv̩˧lv̩˥}}}}\hspace{0.5cm}[\kern2pt{\textcolor{darkblue}{\textbf{\ipa{kv̩˧lv̩˧lv̩˧}}}}\kern2pt]} \hypertarget{kv\string_=\string_Mlv\string_=\string_Mlv\string_=\string_T1}{}
\markboth{\textcolor{darkblue}{\textbf{\ipa{kv̩˧lv̩˧lv̩˥}}}}{}
\textcolor{teal}{\mytextsc{nom}} \hspace{4pt} Ton~: H\#.
\textcolor{PineGreen}{\selectlanguage{french}Tresse d'ail, ail tressé.} \zh{蒜瓣。}  \zh{量词}: \textcolor{darkblue}{\textbf{\ipa{ɭɯ˧}}}  \mytextsc{clf}~: \textcolor{darkblue}{\textbf{\ipa{ɭɯ˧}}} 
\lhead{\firstmark}
\rhead{\botmark}

\subsection{\hspace{-0.5cm} {\Large \textcolor{darkblue}{\textbf{\ipa{kv̩˩nɑ˧˥}}}}\hspace{0.5cm}[\kern2pt{\textcolor{darkblue}{\textbf{\ipa{kv̩˧nɑ˥}}}}\kern2pt]} \hypertarget{kv\string_=\string_BnA\string_M\string_T1}{}
\markboth{\textcolor{darkblue}{\textbf{\ipa{kv̩˩nɑ˧˥}}}}{}
\textcolor{teal}{\mytextsc{nom}} \hspace{4pt} Ton~: LM+MH\#.
\textcolor{PineGreen}{\selectlanguage{french}Soie.} \zh{丝绸。}  ¶ \textcolor{darkblue}{\textbf{\ipa{kv̩˩nɑ˧-bɑ˧lɑ˥}}} \textcolor{PineGreen}{\selectlanguage{french}vêtement en soie} \zh{丝绸衣服}  
 \zh{量词}: \textcolor{darkblue}{\textbf{\ipa{tsʰi˥}}}  \mytextsc{clf}~: \textcolor{darkblue}{\textbf{\ipa{tsʰi˥}}} 
\lhead{\firstmark}
\rhead{\botmark}

\subsection{\hspace{-0.5cm} {\Large \textcolor{darkblue}{\textbf{\ipa{kv̩˧ɲi˥}}}}\hspace{0.5cm}[\kern2pt{\textcolor{darkblue}{\textbf{\ipa{kv̩˧ɲi˥}}}}\kern2pt]} \hypertarget{kv\string_=\string_MJi\string_T1}{}
\markboth{\textcolor{darkblue}{\textbf{\ipa{kv̩˧ɲi˥}}}}{}
\textcolor{teal}{\mytextsc{adjectif}} \hspace{4pt} Ton~: H\#.
\textcolor{PineGreen}{\selectlanguage{french}Vide, sans rien.} \zh{空手,空。}  ¶ \textcolor{darkblue}{\textbf{\ipa{bi˩ʁo˧ | kv̩˧ɲi˥-kʰɯ˩}}} \textcolor{PineGreen}{\selectlanguage{french}vider la bourse (de quelqu'un), c'est-à-dire lui prendre son argent)} \zh{(把一个人的)钱包弄空}  
 ¶ \textcolor{darkblue}{\textbf{\ipa{tɕʰɯ˩ di˩-hɯ˩˥, | mɤ˧-ɖɯ˧, | kv̩˧ɲi˥ | le˧-tsʰɯ˩!}}} \textcolor{PineGreen}{\selectlanguage{french}Il est parti chasser le muntjac, il n'en a pas tué, et est revenu bredouille!} \zh{他去狩猎,没得(任何猎物),空手回来!}  

\lhead{\firstmark}
\rhead{\botmark}

\subsection{\hspace{-0.5cm} {\Large \textcolor{darkblue}{\textbf{\ipa{kv̩˧ʁo˧bv̩˥}}}}\hspace{0.5cm}[\kern2pt{\textcolor{darkblue}{\textbf{\ipa{kv̩˧ʁo˧bv̩˧˥}}}}\kern2pt]} \hypertarget{kv\string_=\string_MRo\string_Mbv\string_=\string_T1}{}
\markboth{\textcolor{darkblue}{\textbf{\ipa{kv̩˧ʁo˧bv̩˥}}}}{}
\textcolor{teal}{\mytextsc{nom}} \hspace{4pt} Ton~: H\#.
\textcolor{PineGreen}{\selectlanguage{french}Pousses d'ail (aliment).} \zh{蒜苗。}  \zh{量词}: \textcolor{darkblue}{\textbf{\ipa{kɤ˧˥}}}  \mytextsc{clf}~: \textcolor{darkblue}{\textbf{\ipa{kɤ˧˥}}} 
\lhead{\firstmark}
\rhead{\botmark}

\subsection{\hspace{-0.5cm} {\Large \textcolor{darkblue}{\textbf{\ipa{kv̩˧ʂe˥\$}}}}\hspace{0.5cm}[\kern2pt{\textcolor{darkblue}{\textbf{\ipa{kv̩˧ʂe˧˥}}}}\kern2pt]} \hypertarget{kv\string_=\string_Ms`e\string_T\$1}{}
\markboth{\textcolor{darkblue}{\textbf{\ipa{kv̩˧ʂe˥\$}}}}{}
\textcolor{teal}{\mytextsc{nom}} \hspace{4pt} Ton~: H\$.
\textcolor{PineGreen}{\selectlanguage{french}Puce.} \zh{跳蚤。}  \zh{量词}: \textcolor{darkblue}{\textbf{\ipa{mi˩}}}  \mytextsc{clf}~: \textcolor{darkblue}{\textbf{\ipa{mi˩}}} 
\lhead{\firstmark}
\rhead{\botmark}

\subsection{\hspace{-0.5cm} {\Large \textcolor{darkblue}{\textbf{\ipa{kv̩˩tɑ˩}}}}\hspace{0.5cm}[\kern2pt{\textcolor{darkblue}{\textbf{\ipa{kv̩˩tɑ˩˥}}}}\kern2pt]} \hypertarget{kv\string_=\string_BtA\string_B1}{}
\markboth{\textcolor{darkblue}{\textbf{\ipa{kv̩˩tɑ˩}}}}{}
\textcolor{teal}{\mytextsc{verbe}} \hspace{4pt} Ton~: L.
\textcolor{PineGreen}{\selectlanguage{french}Regrouper, rassembler (ex.: des troncs, après leur abattage).} \zh{集中在一起(如:砍木材后,把木材堆在一起)。} 
\lhead{\firstmark}
\rhead{\botmark}

\subsection{\hspace{-0.5cm} {\Large \textcolor{darkblue}{\textbf{\ipa{kv̩˧tsʰɑ˥\$}}}}\hspace{0.5cm}[\kern2pt{\textcolor{darkblue}{\textbf{\ipa{kv̩˧tsʰɑ˥}}}}\kern2pt]} \hypertarget{kv\string_=\string_Mts\string_hA\string_T\$1}{}
\markboth{\textcolor{darkblue}{\textbf{\ipa{kv̩˧tsʰɑ˥\$}}}}{}
\textcolor{teal}{\mytextsc{nom}} \hspace{4pt} Ton~: H\$.
\textcolor{PineGreen}{\selectlanguage{french}Nom de clan/famille étendue. Ce nom était celui de la famille des seigneurs (pumi/prinmi) de Muli.} \zh{一个姓(木里土司,普米族,的姓)。}  ¶ \textcolor{darkblue}{\textbf{\ipa{kv̩˧tsʰɑ˧=ɻ̍˥\$}}} \textcolor{PineGreen}{\selectlanguage{french}le clan \textcolor{darkblue}{\textbf{\ipa{/kv̩˧tsʰɑ˥\$/}}}, la famille \textcolor{darkblue}{\textbf{\ipa{/kv̩˧tsʰɑ˥\$/}}}} \zh{\textcolor{darkblue}{\textbf{\ipa{/kv̩˧tsʰɑ˥\$/}}}家族}  
 ¶ \textcolor{darkblue}{\textbf{\ipa{kv̩˧tsʰɑ˧=ɻ̍˧ pi˥-zo˩!}}} \textcolor{PineGreen}{\selectlanguage{french}On les appelait “les \textcolor{darkblue}{\textbf{\ipa{/kv̩˧tsʰɑ˥\$/”!}}}} \zh{人家把他们称作“\textcolor{darkblue}{\textbf{\ipa{/kv̩˧tsʰɑ˥\$/}}}家族”!}  

\lhead{\firstmark}
\rhead{\botmark}

\subsection{\hspace{-0.5cm} {\Large \textcolor{darkblue}{\textbf{\ipa{kv̩˧tsʰɤ˩}}}}\hspace{0.5cm}[\kern2pt{\textcolor{darkblue}{\textbf{\ipa{kv̩˧tsʰɤ˩}}}}\kern2pt]} \hypertarget{kv\string_=\string_Mts\string_h7\string_B1}{}
\markboth{\textcolor{darkblue}{\textbf{\ipa{kv̩˧tsʰɤ˩}}}}{}
\textcolor{teal}{\mytextsc{nom}} \hspace{4pt} Ton~: L\#.
\textcolor{PineGreen}{\selectlanguage{french}Tête d'ail.} \zh{蒜头。}  \zh{量词}: \textcolor{darkblue}{\textbf{\ipa{tsʰɤ˧˥}}}  \mytextsc{clf}~: \textcolor{darkblue}{\textbf{\ipa{tsʰɤ˧˥}}} 
\lhead{\firstmark}
\rhead{\botmark}

\subsection{\hspace{-0.5cm} {\Large \textcolor{darkblue}{\textbf{\ipa{kv̩˧ʈʂɯ˧˥}}}}\hspace{0.5cm}[\kern2pt{\textcolor{darkblue}{\textbf{\ipa{kv̩˧ʈʂɯ˧˥}}}}\kern2pt]} \hypertarget{kv\string_=\string_Mt`s`M\string_M\string_T1}{}
\markboth{\textcolor{darkblue}{\textbf{\ipa{kv̩˧ʈʂɯ˧˥}}}}{}
\textcolor{teal}{\mytextsc{nom}} \hspace{4pt} Ton~: MH\#.
\textcolor{PineGreen}{\selectlanguage{french}Ongle.} \zh{指甲。}  \zh{量词}: \textcolor{darkblue}{\textbf{\ipa{ɭɯ˧}}}  \mytextsc{clf}~: \textcolor{darkblue}{\textbf{\ipa{ɭɯ˧}}} 
\lhead{\firstmark}
\rhead{\botmark}

\subsection{\hspace{-0.5cm} {\Large \textcolor{darkblue}{\textbf{\ipa{‑kv̩˧˥}}}}\hspace{0.5cm}[\kern2pt{\textcolor{darkblue}{\textbf{\ipa{kv̩˧˥}}}}\kern2pt]} \hypertarget{‑kv\string_=\string_M\string_T1}{}
\markboth{\textcolor{darkblue}{\textbf{\ipa{‑kv̩˧˥}}}}{}
\textcolor{teal}{\mytextsc{suffixe}} \hspace{4pt} Ton~: MH.
\textcolor{PineGreen}{\selectlanguage{french}\mytextsc{abilitive;} a aussi des emplois de futur.} \zh{能。} 
\lhead{\firstmark}
\rhead{\botmark}

\subsection{\hspace{-0.5cm} {\Large \textcolor{darkblue}{\textbf{\ipa{kwɑ˧fæ˩}}}}\hspace{0.5cm}[\kern2pt{\textcolor{darkblue}{\textbf{\ipa{kwɑ˧fæ˩}}}}\kern2pt]} \hypertarget{kwA\string_Mf\{\string_B1}{}
\markboth{\textcolor{darkblue}{\textbf{\ipa{kwɑ˧fæ˩}}}}{}
\textcolor{teal}{\mytextsc{nom}} \hspace{4pt} Ton~: L\#.
\textcolor{PineGreen}{\selectlanguage{french}Nom d'un hôtel.} \zh{官房(汉语借词),酒店名称。}  Emprunt~: chinois  \zh{官房}
 ¶ \textcolor{darkblue}{\textbf{\ipa{kwɑ˧fæ˩}}} \textcolor{PineGreen}{\selectlanguage{french}le nom abrégé d'un hôtel cinq étoiles où travaille l'une des filles de la consultante principale.} \zh{丽江官房大酒店的简称。注:发音合作人的女儿在丽江官房大酒店工作。}  

\lhead{\firstmark}
\rhead{\botmark}

\subsection{\hspace{-0.5cm} {\Large \textcolor{darkblue}{\textbf{\ipa{kwɑ˧tsʰɑ˧}}}}\hspace{0.5cm}[\kern2pt{\textcolor{darkblue}{\textbf{\ipa{kwɑ˧tsʰɑ˧}}}}\kern2pt]} \hypertarget{kwA\string_Mts\string_hA\string_M1}{}
\markboth{\textcolor{darkblue}{\textbf{\ipa{kwɑ˧tsʰɑ˧}}}}{}
\textcolor{teal}{\mytextsc{nom}} \hspace{4pt} Ton~: M.
\textcolor{PineGreen}{\selectlanguage{french}Cercueil.} \zh{棺材(汉语借词)。}  Emprunt~: chinois  \zh{棺材}
 ¶ \textcolor{darkblue}{\textbf{\ipa{kwɑ˧tsʰɑ˧, | hĩ˧-mo˩-kʰɯ˩-di˩ ɲi˩!}}} \textcolor{PineGreen}{\selectlanguage{french}Le cercueil, c'est là où on met le cadavre! / Le cercueil, c'est l'objet qui accueille le cadavre!} \zh{棺材,是装尸体的! / 棺材,是用来装尸体的!}  
 \zh{量词}: \textcolor{darkblue}{\textbf{\ipa{ɭɯ˧}}}  \mytextsc{clf}~: \textcolor{darkblue}{\textbf{\ipa{ɭɯ˧}}} 
\lhead{\firstmark}
\rhead{\botmark}

\subsection{\hspace{-0.5cm} {\Large \textcolor{darkblue}{\textbf{\ipa{kwæ˧}}}}\hspace{0.5cm}[\kern2pt{\textcolor{darkblue}{\textbf{\ipa{kwæ˥}}}}\kern2pt]} \hypertarget{kw\{\string_M1}{}
\markboth{\textcolor{darkblue}{\textbf{\ipa{kwæ˧}}}}{}
\textcolor{teal}{\mytextsc{verbe}} \hspace{4pt} Ton~: M.
\textcolor{PineGreen}{\selectlanguage{french}S'occuper de, se charger de.} \zh{管(汉语借词)。}  Emprunt~: chinois  \zh{管}

\lhead{\firstmark}
\rhead{\botmark}

\subsection{\hspace{-0.5cm} {\Large \textcolor{darkblue}{\textbf{\ipa{kwæ˧fæ˥}}}}\hspace{0.5cm}[\kern2pt{\textcolor{darkblue}{\textbf{\ipa{kwæ˧fæ˥}}}}\kern2pt]} \hypertarget{kw\{\string_Mf\{\string_T1}{}
\markboth{\textcolor{darkblue}{\textbf{\ipa{kwæ˧fæ˥}}}}{}
\textcolor{teal}{\mytextsc{nom}} \hspace{4pt} Ton~: H\#.
\textcolor{PineGreen}{\selectlanguage{french}Poutre intermédiaire: pièce de charpente horizontale, posée sur une poutre maîtresse, et soutenant deux des poutres du toit, /ʐv̩˩ɭɯ˧/.} \zh{中等大小的梁。}  \zh{量词}: \textcolor{darkblue}{\textbf{\ipa{pʰæ˧˥}}}  \mytextsc{clf}~: \textcolor{darkblue}{\textbf{\ipa{pʰæ˧˥}}} 
\lhead{\firstmark}
\rhead{\botmark}

\subsection{\hspace{-0.5cm} {\Large \textcolor{darkblue}{\textbf{\ipa{kwæ˧pæ˥}}}}\hspace{0.5cm}[\kern2pt{\textcolor{darkblue}{\textbf{\ipa{kwæ˧pæ˥}}}}\kern2pt]} \hypertarget{kw\{\string_Mp\{\string_T1}{}
\markboth{\textcolor{darkblue}{\textbf{\ipa{kwæ˧pæ˥}}}}{}
\textcolor{teal}{\mytextsc{nom}} \hspace{4pt} Ton~: H\#.
\textcolor{PineGreen}{\selectlanguage{french}Herse en bois (vraisemblablement un emprunt au chinois; terme emprunté: pas identifié avec certitude).} \zh{耙(可能是汉语借词。原来借来的词:刮板?? 刮耙??)。}  \zh{量词}: \textcolor{darkblue}{\textbf{\ipa{nɑ˧}}}  \mytextsc{clf}~: \textcolor{darkblue}{\textbf{\ipa{nɑ˧}}} 
\lhead{\firstmark}
\rhead{\botmark}

\subsection{\hspace{-0.5cm} {\Large \textcolor{darkblue}{\textbf{\ipa{kwæ˧tsɯ˧}}}}\hspace{0.5cm}[\kern2pt{\textcolor{darkblue}{\textbf{\ipa{kwæ˧tsɯ˧}}}}\kern2pt]} \hypertarget{kw\{\string_MtsM\string_M1}{}
\markboth{\textcolor{darkblue}{\textbf{\ipa{kwæ˧tsɯ˧}}}}{}
\textcolor{teal}{\mytextsc{nom}} \hspace{4pt} Ton~: M.
\textcolor{PineGreen}{\selectlanguage{french}Graine de tournesol.} \zh{葵花瓜籽(汉语借词)。}  Emprunt~: chinois  \zh{瓜子}

\lhead{\firstmark}
\rhead{\botmark}

\subsection{\hspace{-0.5cm} {\Large \textcolor{darkblue}{\textbf{\ipa{‑kwɤ}}}}\hspace{0.5cm}[\kern2pt{\textcolor{darkblue}{\textbf{\ipa{xxxx groupe tonal entier sans aucun ton}}}}\kern2pt]} \hypertarget{‑kw71}{}
\markboth{\textcolor{darkblue}{\textbf{\ipa{‑kwɤ}}}}{}
\textcolor{teal}{\mytextsc{conjonction}} \hspace{4pt} Ton~: 0.
\textcolor{PineGreen}{\selectlanguage{french}Lorsque; ne peut s'employer seul, mais apparaît dans la formule \textcolor{darkblue}{\textbf{\ipa{/kwɤ-tɕɯ-lɑ/}}}.} \zh{……的时候。} 
\lhead{\firstmark}
\rhead{\botmark}

\subsection{\hspace{-0.5cm} {\Large \textcolor{darkblue}{\textbf{\ipa{kwɤ˧ɭɯ˩}}}}\hspace{0.5cm}[\kern2pt{\textcolor{darkblue}{\textbf{\ipa{kwɤ˧ɭɯ˩}}}}\kern2pt]} \hypertarget{kw7\string_Ml\string_RM\string_B1}{}
\markboth{\textcolor{darkblue}{\textbf{\ipa{kwɤ˧ɭɯ˩}}}}{}
\textcolor{teal}{\mytextsc{nom}} \hspace{4pt} Ton~: L\#.
\textcolor{PineGreen}{\selectlanguage{french}Jarre; trésor, objet de grande valeur.} \zh{坛子、罐子 (陶器),宝贝。}  ¶ \textcolor{darkblue}{\textbf{\ipa{ʈʂʰɯ˧ | njɤ˧ kwɤ˧ɭɯ˩ ɲi˩!}}} \textcolor{PineGreen}{\selectlanguage{french}C'est mon petit trésor! (dit au sujet d'un enfant)} \zh{他是我宝贝!(母亲说孩子是她的宝贝)}  
 \zh{量词}: \textcolor{darkblue}{\textbf{\ipa{ɭɯ˧}}}  \mytextsc{clf}~: \textcolor{darkblue}{\textbf{\ipa{ɭɯ˧}}} 
\lhead{\firstmark}
\rhead{\botmark}

\subsection{\hspace{-0.5cm} {\Large \textcolor{darkblue}{\textbf{\ipa{kwɤ˧pɤ˧}}}}\hspace{0.5cm}[\kern2pt{\textcolor{darkblue}{\textbf{\ipa{kwɤ˧pɤ˧}}}}\kern2pt]} \hypertarget{kw7\string_Mp7\string_M1}{}
\markboth{\textcolor{darkblue}{\textbf{\ipa{kwɤ˧pɤ˧}}}}{}
\textcolor{teal}{\mytextsc{nom}} \hspace{4pt} Ton~: M.
\textcolor{PineGreen}{\selectlanguage{french}Explication, enseignement.} \zh{解释,教导、教诲。}  ¶ \textcolor{darkblue}{\textbf{\ipa{kwɤ˧pɤ˧ ɖɯ˧-kʰwɤ˥ lɑ˩}}} \textcolor{PineGreen}{\selectlanguage{french}enseigner quelque chose à quelqu'un, expliquer quelque chose à quelqu'un} \zh{解释一个道理、教一件事}  
 ¶ \textcolor{darkblue}{\textbf{\ipa{kwɤ˧pɤ˧ ɖɯ˧-kʰwɤ˥ | tʰi˧-lɑ˩-ɻ̍˩}}} \textcolor{PineGreen}{\selectlanguage{french}comme ci-dessus: enseigner quelque chose à quelqu'un, expliquer quelque chose à quelqu'un} \zh{同上:解释一个道理、教一件事}  
 ¶ \textcolor{darkblue}{\textbf{\ipa{[M23] kwɤ˧pɤ˧ lɑ˧˥}}} \textcolor{PineGreen}{\selectlanguage{french}enseigner} \zh{教、解释}  
 \zh{量词}: \textcolor{darkblue}{\textbf{\ipa{kʰwɤ˥}}}  \mytextsc{clf}~: \textcolor{darkblue}{\textbf{\ipa{kʰwɤ˥}}} 
\lhead{\firstmark}
\rhead{\botmark}

\subsection{\hspace{-0.5cm} {\Large \textcolor{darkblue}{\textbf{\ipa{‑kwɤ˧tɕɯ˥}}}}\hspace{0.5cm}[\kern2pt{\textcolor{darkblue}{\textbf{\ipa{kwɤ˧tɕɯ˥}}}}\kern2pt]} \hypertarget{‑kw7\string_Mts£M\string_T1}{}
\markboth{\textcolor{darkblue}{\textbf{\ipa{‑kwɤ˧tɕɯ˥}}}}{}
\textcolor{teal}{\mytextsc{conjonction}} \hspace{4pt} Ton~: H\#.
\textcolor{PineGreen}{\selectlanguage{french}Comme; après; puisque.} \zh{因为,由于,既然。}  ¶ \textcolor{darkblue}{\textbf{\ipa{-kwɤ˧tɕɯ˥-lɑ˩}}} \textcolor{PineGreen}{\selectlanguage{french}même sens} \zh{同上}  
 ¶ \textcolor{darkblue}{\textbf{\ipa{ʈʂʰɯ˧ | go˩-kwɤ˩tɕɯ˥-lɑ˩, | hɑ˧ mɤ˧-dzɯ˥.}}} \textcolor{PineGreen}{\selectlanguage{french}Comme il est malade, il ne mange pas.} \zh{他病了,吃不下饭。}  
 ¶ \textcolor{darkblue}{\textbf{\ipa{[M18] ʈʂʰɯ˧ne˧-ʝi˥ | pi˧-kwɤ˩tɕɯ˩-lɑ˩, | wɤ˩˥ | lɑ˧hɑ˥ | ɖɯ˧-kʰwɤ˧ ʐwɤ˧˥.}}} \textcolor{PineGreen}{\selectlanguage{french}ayant dit cela, il dit à nouveau autre chose (après avoir dit ça, il a dit ajouté autre chose!)} \zh{他这样说完以后,又讲了些其它的。}  
 ¶ \textcolor{darkblue}{\textbf{\ipa{[M18] ʈʂʰɯ˧ | tʰi˧-dzi˩-kwɤ˩-tɕɯ˩, | ɖɯ˧-kʰwɤ˧ ʐwɤ˧-ɻ̍˥: | “sɤ˧sɤ˧˥ | ʐwæ˧˥!”}}} \textcolor{PineGreen}{\selectlanguage{french}après s’être assis/lorsqu’il fut assis, il dit une phrase: “quel confort!”} \zh{他坐下后,说了这么一句:“真舒服!”}  

\lhead{\firstmark}
\rhead{\botmark}

\subsection{\hspace{-0.5cm} {\Large \textcolor{darkblue}{\textbf{\ipa{kwɤ˩\textsubscript{a}}}}}\hspace{0.5cm}[\kern2pt{\textcolor{darkblue}{\textbf{\ipa{kwɤ˩˥}}}}\kern2pt]} \hypertarget{kw7\string_Ba1}{}
\markboth{\textcolor{darkblue}{\textbf{\ipa{kwɤ˩\textsubscript{a}}}}}{}
\textcolor{teal}{\mytextsc{classificateur}} \hspace{4pt} Ton~: L\textsubscript{a}.
\textcolor{PineGreen}{\selectlanguage{french}Classificateur des objets tressés, enfilés ou liés ensemble.} \zh{量词:串。}  ¶ \textcolor{darkblue}{\textbf{\ipa{kv̩˧ | ɖɯ˧-kwɤ˩}}} \textcolor{PineGreen}{\selectlanguage{french}une tresse d'aïl} \zh{一辫大蒜}  
 ¶ \textcolor{darkblue}{\textbf{\ipa{lɑ˧tsɯ˥ | ɖɯ˧-kwɤ˩}}} \textcolor{PineGreen}{\selectlanguage{french}une ligature de piments} \zh{一辫辣椒}  
 ¶ \textcolor{darkblue}{\textbf{\ipa{ʈʂʰɯ˧-kwɤ˥}}} \textcolor{PineGreen}{\selectlanguage{french}\mytextsc{dem} \string_ (ton: H\# / H\$)} \zh{\mytextsc{指示代词} \string_}  

\lhead{\firstmark}
\rhead{\botmark}

\subsection{\hspace{-0.5cm} {\Large \textcolor{darkblue}{\textbf{\ipa{kwɤ˩\textsubscript{a}}}} \textsubscript{1}}\hspace{0.5cm}[\kern2pt{\textcolor{darkblue}{\textbf{\ipa{kwɤ˩˥}}}}\kern2pt]} \hypertarget{kw7\string_Ba1}{}
\markboth{\textcolor{darkblue}{\textbf{\ipa{kwɤ˩\textsubscript{a}}}} \textsubscript{1}}{}
\textcolor{teal}{\mytextsc{verbe}} \hspace{4pt} Ton~: L\textsubscript{a}.
\textcolor{PineGreen}{\selectlanguage{french}Jeter.} \zh{扔掉。}  ¶ \textcolor{darkblue}{\textbf{\ipa{mv̩˩tɕo˧ kwɤ˩}}} \textcolor{PineGreen}{\selectlanguage{french}jeter (détritus); littéralement: “mettre en bas”} \zh{扔掉(垃圾)}  
 ¶ \textcolor{darkblue}{\textbf{\ipa{tso˧\textasciitilde{}tso˧ kwɤ˥}}} \textcolor{PineGreen}{\selectlanguage{french}jeter des choses} \zh{扔东西}  

\lhead{\firstmark}
\rhead{\botmark}

\subsection{\hspace{-0.5cm} {\Large \textcolor{darkblue}{\textbf{\ipa{kwɤ˩\textsubscript{a}}}} \textsubscript{2}}\hspace{0.5cm}[\kern2pt{\textcolor{darkblue}{\textbf{\ipa{kwɤ˩˥}}}}\kern2pt]} \hypertarget{kw7\string_Ba2}{}
\markboth{\textcolor{darkblue}{\textbf{\ipa{kwɤ˩\textsubscript{a}}}} \textsubscript{2}}{}
\textcolor{teal}{\mytextsc{verbe}} \hspace{4pt} Ton~: L\textsubscript{a}.
\textcolor{PineGreen}{\selectlanguage{french}S'occuper de, gérer, superviser.} \zh{管(汉语借词)。}  ¶ \textcolor{darkblue}{\textbf{\ipa{ɖɯ˧-kʰwɤ˧ kwɤ˥}}} \textcolor{PineGreen}{\selectlanguage{french}s'occuper un peu/s'occuper d'une partie (combinaison élicitée pour vérifier que le ton est L et non LM)} \zh{管一些}  

\lhead{\firstmark}
\rhead{\botmark}

\subsection{\hspace{-0.5cm} {\Large \textcolor{darkblue}{\textbf{\ipa{kwɤ˩-tjɤ˧ljɤ\#˥}}}}\hspace{0.5cm}[\kern2pt{\textcolor{darkblue}{\textbf{\ipa{kwɤ˧tjɤ˧ljɤ˧}}}}\kern2pt]} \hypertarget{kw7\string_B-tj7\string_Mlj7\#\string_T1}{}
\markboth{\textcolor{darkblue}{\textbf{\ipa{kwɤ˩-tjɤ˧ljɤ\#˥}}}}{}
\textcolor{teal}{\mytextsc{nom}} \hspace{4pt} Ton~: L-\#H.
\textcolor{PineGreen}{\selectlanguage{french}Clochette s'accrochant autour du cou (ex.: clochette d'un cheval).} \zh{铃铛。}  \zh{量词}: \textcolor{darkblue}{\textbf{\ipa{ɭɯ˧}}}  \mytextsc{clf}~: \textcolor{darkblue}{\textbf{\ipa{ɭɯ˧}}} 
\lhead{\firstmark}
\rhead{\botmark}

\subsection{\hspace{-0.5cm} {\Large \textcolor{darkblue}{\textbf{\ipa{kwɤ˩to˥}}}}\hspace{0.5cm}[\kern2pt{\textcolor{darkblue}{\textbf{\ipa{kwɤ˩to˥}}}}\kern2pt]} \hypertarget{kw7\string_Bto\string_T1}{}
\markboth{\textcolor{darkblue}{\textbf{\ipa{kwɤ˩to˥}}}}{}
\textcolor{teal}{\mytextsc{nom}} \hspace{4pt} Ton~: LH.
\textcolor{PineGreen}{\selectlanguage{french}Mandibule, mâchoire inférieure.} \zh{颌骨。}  \zh{量词}: \textcolor{darkblue}{\textbf{\ipa{ɭɯ˧}}}  \mytextsc{clf}~: \textcolor{darkblue}{\textbf{\ipa{ɭɯ˧}}} 
\lhead{\firstmark}
\rhead{\botmark}

\newpage
\section*{\centering- \textcolor{darkblue}{\textbf{\ipa{kʰ}}} -}
\subsection{\hspace{-0.5cm} {\Large \textcolor{darkblue}{\textbf{\ipa{kʰɤ˧mi˥\$}}}}\hspace{0.5cm}[\kern2pt{\textcolor{darkblue}{\textbf{\ipa{xxxx ton non trouvé, à faire manuellement...}}}}\kern2pt]} \hypertarget{k\string_h7\string_Mmi\string_T\$1}{}
\markboth{\textcolor{darkblue}{\textbf{\ipa{kʰɤ˧mi˥\$}}}}{}
\textcolor{teal}{\mytextsc{nom}} \hspace{4pt} Ton~: \$H.
\textcolor{PineGreen}{\selectlanguage{french}Grande hotte.} \zh{大背篓。}  \zh{量词}: \textcolor{darkblue}{\textbf{\ipa{kʰɤ˧˥}}}  \mytextsc{clf}~: \textcolor{darkblue}{\textbf{\ipa{kʰɤ˧˥}}} 
\lhead{\firstmark}
\rhead{\botmark}

\subsection{\hspace{-0.5cm} {\Large \textcolor{darkblue}{\textbf{\ipa{kʰɤ˧ʂɯ˧}}}}\hspace{0.5cm}[\kern2pt{\textcolor{darkblue}{\textbf{\ipa{kʰɤ˧ʂɯ˧}}}}\kern2pt]} \hypertarget{k\string_h7\string_Ms`M\string_M1}{}
\markboth{\textcolor{darkblue}{\textbf{\ipa{kʰɤ˧ʂɯ˧}}}}{}
\textcolor{teal}{\mytextsc{verbe}} \hspace{4pt} Ton~: M.
\textcolor{PineGreen}{\selectlanguage{french}Commencer.} \zh{开始(汉语借词)。}  Emprunt~: chinois  \zh{开始}

\lhead{\firstmark}
\rhead{\botmark}

\subsection{\hspace{-0.5cm} {\Large \textcolor{darkblue}{\textbf{\ipa{kʰɤ˧zo˥\$}}}}\hspace{0.5cm}[\kern2pt{\textcolor{darkblue}{\textbf{\ipa{xxxx ton non trouvé, à faire manuellement...}}}}\kern2pt]} \hypertarget{k\string_h7\string_Mzo\string_T\$1}{}
\markboth{\textcolor{darkblue}{\textbf{\ipa{kʰɤ˧zo˥\$}}}}{}
\textcolor{teal}{\mytextsc{nom}} \hspace{4pt} Ton~: \$H.
\textcolor{PineGreen}{\selectlanguage{french}Petite hotte.} \zh{小背篓。}  \zh{量词}: \textcolor{darkblue}{\textbf{\ipa{kʰɤ˧˥}}}  \mytextsc{clf}~: \textcolor{darkblue}{\textbf{\ipa{kʰɤ˧˥}}} 
\lhead{\firstmark}
\rhead{\botmark}

\subsection{\hspace{-0.5cm} {\Large \textcolor{darkblue}{\textbf{\ipa{kʰɤ˩njɤ˩\textasciitilde{}kʰɤ˧njɤ˧}}}}\hspace{0.5cm}[\kern2pt{\textcolor{darkblue}{\textbf{\ipa{xxxx non-correspondance entre le nombre de morphèmes et le nombre de tons de morphèmes}}}}\kern2pt]} \hypertarget{k\string_h7\string_Bnj7\string_B~k\string_h7\string_Mnj7\string_M1}{}
\markboth{\textcolor{darkblue}{\textbf{\ipa{kʰɤ˩njɤ˩\textasciitilde{}kʰɤ˧njɤ˧}}}}{}
\textcolor{teal}{\mytextsc{adjectif}} \hspace{4pt} Ton~: L-.
\textcolor{PineGreen}{\selectlanguage{french}Souple (mouvement).} \zh{柔软(动作)。} 
\lhead{\firstmark}
\rhead{\botmark}

\subsection{\hspace{-0.5cm} {\Large \textcolor{darkblue}{\textbf{\ipa{kʰɤ˧˥}}} \textsubscript{1}}\hspace{0.5cm}[\kern2pt{\textcolor{darkblue}{\textbf{\ipa{kʰɤ˧˥}}}}\kern2pt]} \hypertarget{k\string_h7\string_M\string_T1}{}
\markboth{\textcolor{darkblue}{\textbf{\ipa{kʰɤ˧˥}}} \textsubscript{1}}{}
\textcolor{teal}{\mytextsc{verbe}} \hspace{4pt} Ton~: MH.
\textcolor{PineGreen}{\selectlanguage{french}Éteindre le foyer.} \zh{灭(火)。} \textit{Voir~:} \hyperlink{}{\textcolor{darkblue}{\textbf{\ipa{hɑ̃˧˥}}} \textsubscript{1}} 
\lhead{\firstmark}
\rhead{\botmark}

\subsection{\hspace{-0.5cm} {\Large \textcolor{darkblue}{\textbf{\ipa{kʰɤ˧˥}}} \textsubscript{2}}\hspace{0.5cm}[\kern2pt{\textcolor{darkblue}{\textbf{\ipa{kʰɤ˧˥}}}}\kern2pt]} \hypertarget{k\string_h7\string_M\string_T2}{}
\markboth{\textcolor{darkblue}{\textbf{\ipa{kʰɤ˧˥}}} \textsubscript{2}}{}
\textcolor{teal}{\mytextsc{nom}} \hspace{4pt} Ton~: MH.
\textcolor{PineGreen}{\selectlanguage{french}Hotte.} \zh{背篓。}  \zh{量词}: \textcolor{darkblue}{\textbf{\ipa{kʰɤ˧˥}}}  \mytextsc{clf}~: \textcolor{darkblue}{\textbf{\ipa{kʰɤ˧˥}}} 
\lhead{\firstmark}
\rhead{\botmark}

\subsection{\hspace{-0.5cm} {\Large \textcolor{darkblue}{\textbf{\ipa{kʰɤ˧˥\textsubscript{a}}}}}\hspace{0.5cm}[\kern2pt{\textcolor{darkblue}{\textbf{\ipa{kʰɤ˧˥}}}}\kern2pt]} \hypertarget{k\string_h7\string_M\string_Ta1}{}
\markboth{\textcolor{darkblue}{\textbf{\ipa{kʰɤ˧˥\textsubscript{a}}}}}{}
\textcolor{teal}{\mytextsc{classificateur}} \hspace{4pt} Ton~: MH\textsubscript{a}.
\textcolor{PineGreen}{\selectlanguage{french}Classificateur des cageots.} \zh{量词:筐。} 
\lhead{\firstmark}
\rhead{\botmark}

\subsection{\hspace{-0.5cm} {\Large \textcolor{darkblue}{\textbf{\ipa{kʰi˥}}}}\hspace{0.5cm}[\kern2pt{\textcolor{darkblue}{\textbf{\ipa{kʰi˥}}}}\kern2pt]} \hypertarget{k\string_hi\string_T1}{}
\markboth{\textcolor{darkblue}{\textbf{\ipa{kʰi˥}}}}{}
\textcolor{teal}{\mytextsc{verbe}} \hspace{4pt} Ton~: H.
\textcolor{PineGreen}{\selectlanguage{french}Séparer, défaire (des fibres de lin: on sépare les fibres pour faire du fil).} \zh{拆开、分离(几根线)。}  ¶ \textcolor{darkblue}{\textbf{\ipa{sɑ˧ | le˧-kʰi˥}}} \textcolor{PineGreen}{\selectlanguage{french}défaire des fibres de lin (pour fabriquer du fil)} \zh{拆开粗麻(为了纺出细麻线)}  

\lhead{\firstmark}
\rhead{\botmark}

\subsection{\hspace{-0.5cm} {\Large \textcolor{darkblue}{\textbf{\ipa{kʰi˥}}} \textsubscript{1}}\hspace{0.5cm}[\kern2pt{\textcolor{darkblue}{\textbf{\ipa{kʰi˥}}}}\kern2pt]} \hypertarget{k\string_hi\string_T1}{}
\markboth{\textcolor{darkblue}{\textbf{\ipa{kʰi˥}}} \textsubscript{1}}{}
\textcolor{teal}{\mytextsc{nom}} \hspace{4pt} Ton~: \#H.
\textcolor{PineGreen}{\selectlanguage{french}Porte.} \zh{门。}  ¶ \textcolor{darkblue}{\textbf{\ipa{kʰi˧-zo\#˥}}} \textcolor{PineGreen}{\selectlanguage{french}petite porte} \zh{小门}  
 \zh{量词}: \textcolor{darkblue}{\textbf{\ipa{̩pɤ˩}}}  \mytextsc{clf}~: \textcolor{darkblue}{\textbf{\ipa{̩pɤ˩}}} 
\lhead{\firstmark}
\rhead{\botmark}

\subsection{\hspace{-0.5cm} {\Large \textcolor{darkblue}{\textbf{\ipa{kʰi˥}}} \textsubscript{2}}\hspace{0.5cm}[\kern2pt{\textcolor{darkblue}{\textbf{\ipa{kʰi˥}}}}\kern2pt]} \hypertarget{k\string_hi\string_T2}{}
\markboth{\textcolor{darkblue}{\textbf{\ipa{kʰi˥}}} \textsubscript{2}}{}
\textcolor{teal}{\mytextsc{nom}} \hspace{4pt} Ton~: \#H.
\textcolor{PineGreen}{\selectlanguage{french}Bord (monosyllabe).} \zh{边(单音节)。} 
\lhead{\firstmark}
\rhead{\botmark}

\subsection{\hspace{-0.5cm} {\Large \textcolor{darkblue}{\textbf{\ipa{kʰi˧bɤ\#˥}}}}\hspace{0.5cm}[\kern2pt{\textcolor{darkblue}{\textbf{\ipa{kʰi˧bɤ˧}}}}\kern2pt]} \hypertarget{k\string_hi\string_Mb7\#\string_T1}{}
\markboth{\textcolor{darkblue}{\textbf{\ipa{kʰi˧bɤ\#˥}}}}{}
\textcolor{teal}{\mytextsc{nom}} \hspace{4pt} Ton~: \#H.
\textcolor{PineGreen}{\selectlanguage{french}Seuil.} \zh{门槛。}  \zh{量词}: \textcolor{darkblue}{\textbf{\ipa{ɭɯ˧}}}  \mytextsc{clf}~: \textcolor{darkblue}{\textbf{\ipa{ɭɯ˧}}} 
\lhead{\firstmark}
\rhead{\botmark}

\subsection{\hspace{-0.5cm} {\Large \textcolor{darkblue}{\textbf{\ipa{-kʰi˧\textasciitilde{}kʰi˧}}}}\hspace{0.5cm}[\kern2pt{\textcolor{darkblue}{\textbf{\ipa{kʰi˧kʰi˧}}}}\kern2pt]} \hypertarget{-k\string_hi\string_M~k\string_hi\string_M1}{}
\markboth{\textcolor{darkblue}{\textbf{\ipa{-kʰi˧\textasciitilde{}kʰi˧}}}}{}
\textcolor{teal}{\mytextsc{postposition}} \hspace{4pt} Ton~: \#H.
\textcolor{PineGreen}{\selectlanguage{french}Aux alentours de, au bord de, auprès de.} \zh{周围、左右、旁边。}  ¶ \textcolor{darkblue}{\textbf{\ipa{ʑi˧qʰwɤ˧-kʰi˧\textasciitilde{}kʰi˧}}} \textcolor{PineGreen}{\selectlanguage{french}aux alentours de la maison} \zh{房子周围}  
 ¶ \textcolor{darkblue}{\textbf{\ipa{[F5] njɤ˧-bv̩˧ | kʰi˧\textasciitilde{}kʰi˧}}} \textcolor{PineGreen}{\selectlanguage{french}à côté de moi, autour de moi} \zh{我的周围}  
 ¶ \textcolor{darkblue}{\textbf{\ipa{[M21] ʐɤ˩mi˩-kʰi˩\textasciitilde{}kʰi˩ se˩˥}}} \textcolor{PineGreen}{\selectlanguage{french}marcher au bord de la route} \zh{走在马路边}  

\lhead{\firstmark}
\rhead{\botmark}

\subsection{\hspace{-0.5cm} {\Large \textcolor{darkblue}{\textbf{\ipa{kʰi˧mi˧}}}}\hspace{0.5cm}[\kern2pt{\textcolor{darkblue}{\textbf{\ipa{kʰi˧mi˧}}}}\kern2pt]} \hypertarget{k\string_hi\string_Mmi\string_M1}{}
\markboth{\textcolor{darkblue}{\textbf{\ipa{kʰi˧mi˧}}}}{}
\textcolor{teal}{\mytextsc{nom}} \hspace{4pt} Ton~: M.
\textcolor{PineGreen}{\selectlanguage{french}Grande porte (ex.: porte d'entrée d'une maison).} \zh{大门。}  \zh{量词}: \textcolor{darkblue}{\textbf{\ipa{pɤ˩}}}  \mytextsc{clf}~: \textcolor{darkblue}{\textbf{\ipa{pɤ˩}}} 
\lhead{\firstmark}
\rhead{\botmark}

\subsection{\hspace{-0.5cm} {\Large \textcolor{darkblue}{\textbf{\ipa{kʰi˧qʰv̩\#˥}}}}\hspace{0.5cm}[\kern2pt{\textcolor{darkblue}{\textbf{\ipa{kʰi˧qʰv̩˧}}}}\kern2pt]} \hypertarget{k\string_hi\string_Mq\string_hv\string_=\#\string_T1}{}
\markboth{\textcolor{darkblue}{\textbf{\ipa{kʰi˧qʰv̩\#˥}}}}{}
\textcolor{teal}{\mytextsc{nom}} \hspace{4pt} Ton~: \#H.
\textcolor{PineGreen}{\selectlanguage{french}Porte.} \zh{门。}  ¶ \textcolor{darkblue}{\textbf{\ipa{kʰi˧qʰv˧ tʰv˧-ɲi˥}}} \textcolor{PineGreen}{\selectlanguage{french}parvenir à la porte, atteindre la porte (contexte: retour d'un lointain périple)} \zh{到达(家)门(情景:从远方回家,到达家门)}  
 ¶ \textcolor{darkblue}{\textbf{\ipa{ɑ˩ʁo˧ kʰi˧qʰv˧ tʰv˧}}} \textcolor{PineGreen}{\selectlanguage{french}parvenir à la porte de la maison} \zh{到达家门}  

\lhead{\firstmark}
\rhead{\botmark}

\subsection{\hspace{-0.5cm} {\Large \textcolor{darkblue}{\textbf{\ipa{kʰi˧-qʰwɤ˩}}}}\hspace{0.5cm}[\kern2pt{\textcolor{darkblue}{\textbf{\ipa{xxxx non-correspondance entre le nombre de morphèmes et le nombre de tons de morphèmes}}}}\kern2pt]} \hypertarget{k\string_hi\string_M-q\string_hw7\string_B1}{}
\markboth{\textcolor{darkblue}{\textbf{\ipa{kʰi˧-qʰwɤ˩}}}}{}
\textcolor{teal}{\mytextsc{nom}} \hspace{4pt} Ton~: L\#.
\textcolor{PineGreen}{\selectlanguage{french}Gonds (d'une porte).} \zh{门的合页。}  \zh{量词}: \textcolor{darkblue}{\textbf{\ipa{ɭɯ˧}}}  \mytextsc{clf}~: \textcolor{darkblue}{\textbf{\ipa{ɭɯ˧}}} \textit{Syn~:} \hyperlink{}{\textcolor{darkblue}{\textbf{\ipa{kʰi˧-bv̩˧lv̩˩}}}}. 
\lhead{\firstmark}
\rhead{\botmark}

\subsection{\hspace{-0.5cm} {\Large \textcolor{darkblue}{\textbf{\ipa{kʰi˧tɕʰɯ˩-mo˩}}}}\hspace{0.5cm}[\kern2pt{\textcolor{darkblue}{\textbf{\ipa{kʰi˧tɕʰɯ˩mo˧}}}}\kern2pt]} \hypertarget{k\string_hi\string_Mts£\string_hM\string_B-mo\string_B1}{}
\markboth{\textcolor{darkblue}{\textbf{\ipa{kʰi˧tɕʰɯ˩-mo˩}}}}{}
\textcolor{teal}{\mytextsc{nom}} \hspace{4pt} Ton~: L\#-.
\textcolor{PineGreen}{\selectlanguage{french}Un champignon vénéneux.} \zh{一种有毒的菌子。}  ¶ \textcolor{darkblue}{\textbf{\ipa{ʈʂæ˧mo˧-kʰi˧tɕʰɯ˩-mo˩}}} \textcolor{PineGreen}{\selectlanguage{french}même sens} \zh{同上}  
\textit{Syn~:} \hyperlink{}{\textcolor{darkblue}{\textbf{\ipa{ʈʂæ˧mo\#˥}}}}. 
\lhead{\firstmark}
\rhead{\botmark}

\subsection{\hspace{-0.5cm} {\Large \textcolor{darkblue}{\textbf{\ipa{kʰi˧˥}}}}\hspace{0.5cm}[\kern2pt{\textcolor{darkblue}{\textbf{\ipa{kʰi˧˥}}}}\kern2pt]} \hypertarget{k\string_hi\string_M\string_T1}{}
\markboth{\textcolor{darkblue}{\textbf{\ipa{kʰi˧˥}}}}{}
\textcolor{teal}{\mytextsc{verbe}} \hspace{4pt} Ton~: MH.
\textcolor{PineGreen}{\selectlanguage{french}Forme passée du verbe 'partir'.} \zh{走(过去式)。}  ¶ \textcolor{darkblue}{\textbf{\ipa{ʈʂʰɯ˧ | zo˩qo˧ kʰi˧?}}} \textcolor{PineGreen}{\selectlanguage{french}Elle/il est parti où?} \zh{他到哪里去了?}  
 ¶ \textcolor{darkblue}{\textbf{\ipa{[M23] no˧ | tsʰi˧ɲi˧ | ɑ˩pʰo˩˥ | ə˩-kʰi˩˥?}}} \textcolor{PineGreen}{\selectlanguage{french}tu es allé faire un tour dehors, aujourd'hui?/tu es sorti, aujourd'hui? (Contexte: question posée par un consultant alors que je le raccompagne après une séance de travail vespérale)} \zh{你今天出去了吗?}  

\lhead{\firstmark}
\rhead{\botmark}

\subsection{\hspace{-0.5cm} {\Large \textcolor{darkblue}{\textbf{\ipa{kʰo˥}}}}\hspace{0.5cm}[\kern2pt{\textcolor{darkblue}{\textbf{\ipa{kʰo˥}}}}\kern2pt]} \hypertarget{k\string_ho\string_T1}{}
\markboth{\textcolor{darkblue}{\textbf{\ipa{kʰo˥}}}}{}
\textcolor{teal}{\mytextsc{verbe}} \hspace{4pt} Ton~: H.
\textcolor{PineGreen}{\selectlanguage{french}Étendre (un matelas), étaler (des fruits, des outils... partout par terre).} \zh{铺(床……)、铺得满地(果子、工具……)。}  ¶ \textcolor{darkblue}{\textbf{\ipa{kʰwæ˧ɻæ˧ kʰo˧}}} \textcolor{PineGreen}{\selectlanguage{french}étendre une natte} \zh{铺垫子}  

\lhead{\firstmark}
\rhead{\botmark}

\subsection{\hspace{-0.5cm} {\Large \textcolor{darkblue}{\textbf{\ipa{kʰo˧bɤ˧}}}}\hspace{0.5cm}[\kern2pt{\textcolor{darkblue}{\textbf{\ipa{kʰo˧bɤ˧}}}}\kern2pt]} \hypertarget{k\string_ho\string_Mb7\string_M1}{}
\markboth{\textcolor{darkblue}{\textbf{\ipa{kʰo˧bɤ˧}}}}{}
\textcolor{teal}{\mytextsc{nom}} \hspace{4pt} Ton~: M.
\textcolor{PineGreen}{\selectlanguage{french}Foyer. Mot ancien, qui n'est utilisé que dans un registre soutenu; il désigne un espace de vie.} \zh{家(文言):母亲生活的空间:有家人,有火塘,有母亲在那里生活的那个空间。}  ¶ \textcolor{darkblue}{\textbf{\ipa{dʑi˧kʰi˧ le˧-gwɤ˩ | qo˩ tɑ˧-ze˥, | njɤ˧-ɕi˩ ə˩mɑ˩ kʰo˩bɤ˩ dʑɤ˩. |}}} \textcolor{PineGreen}{\selectlanguage{french}xxxx traduction à affiner (chanson au sujet de la nostalgie du foyer)}  
 ¶ \textcolor{darkblue}{\textbf{\ipa{dʑi˧kʰi˧ le˧-gwɤ˩ | qo˩ tɑ˧-ze˥, | njɤ˧-ɕi˩ ə˩mɑ˩ kʰo˩bɤ˩-qo˩. |}}} \textcolor{PineGreen}{\selectlanguage{french}idem ci-dessus}  
 ¶ \textcolor{darkblue}{\textbf{\ipa{dʑi˧kʰi˧ le˧-gwɤ˩ qo˩ tɑ˩-ze˩, | njɤ˧-ɕi˩ ə˩mɑ˩ kʰo˩bɤ˩ dʑɤ˩. |}}} \textcolor{PineGreen}{\selectlanguage{french}idem}  
 \zh{量词}: \textcolor{darkblue}{\textbf{\ipa{ɭɯ˧}}}  \mytextsc{clf}~: \textcolor{darkblue}{\textbf{\ipa{ɭɯ˧}}} 
\lhead{\firstmark}
\rhead{\botmark}

\subsection{\hspace{-0.5cm} {\Large \textcolor{darkblue}{\textbf{\ipa{kʰo˧lo˧}}}}\hspace{0.5cm}[\kern2pt{\textcolor{darkblue}{\textbf{\ipa{kʰo˧lo˧}}}}\kern2pt]} \hypertarget{k\string_ho\string_Mlo\string_M1}{}
\markboth{\textcolor{darkblue}{\textbf{\ipa{kʰo˧lo˧}}}}{}
\textcolor{teal}{\mytextsc{nom}} \hspace{4pt} Ton~: M.
\textcolor{PineGreen}{\selectlanguage{french}Moulin à prière (aussi bien les très grands, fixés à des axes verticaux dans les monastères, que les plus petits, tenus à la main).} \zh{转经筒。} Dialecte chinois local~:\zh{祈祷轱辘。} \zh{量词}: \textcolor{darkblue}{\textbf{\ipa{ɭɯ˧}}}  \mytextsc{clf}~: \textcolor{darkblue}{\textbf{\ipa{ɭɯ˧}}} 
\lhead{\firstmark}
\rhead{\botmark}

\subsection{\hspace{-0.5cm} {\Large \textcolor{darkblue}{\textbf{\ipa{kʰɯ˧di˧˥}}}}\hspace{0.5cm}[\kern2pt{\textcolor{darkblue}{\textbf{\ipa{kʰɯ˩di˩˥}}}}\kern2pt]} \hypertarget{k\string_hM\string_Mdi\string_M\string_T1}{}
\markboth{\textcolor{darkblue}{\textbf{\ipa{kʰɯ˧di˧˥}}}}{}
\textcolor{teal}{\mytextsc{nom}} \hspace{4pt} Ton~: MH\#.
\textcolor{PineGreen}{\selectlanguage{french}Récipient (terme générique).} \zh{容器。}  \zh{量词}: \textcolor{darkblue}{\textbf{\ipa{ɭɯ˧}}}  \mytextsc{clf}~: \textcolor{darkblue}{\textbf{\ipa{ɭɯ˧}}} 
\lhead{\firstmark}
\rhead{\botmark}

\subsection{\hspace{-0.5cm} {\Large \textcolor{darkblue}{\textbf{\ipa{kʰɯ˧dv̩\#˥}}}}\hspace{0.5cm}[\kern2pt{\textcolor{darkblue}{\textbf{\ipa{kʰɯ˧dv̩˧˥}}}}\kern2pt]} \hypertarget{k\string_hM\string_Mdv\string_=\#\string_T1}{}
\markboth{\textcolor{darkblue}{\textbf{\ipa{kʰɯ˧dv̩\#˥}}}}{}
\textcolor{teal}{\mytextsc{nom}} \hspace{4pt} Ton~: \#H.
\textcolor{PineGreen}{\selectlanguage{french}Boiteux.} \zh{跛。}  ¶ \textcolor{darkblue}{\textbf{\ipa{kʰɯ˧dv̩˧-hĩ˧}}} \textcolor{PineGreen}{\selectlanguage{french}boiteux} \zh{跛}  
 ¶ \textcolor{darkblue}{\textbf{\ipa{kʰɯ˧dv̩˧-tsʰo˧qʰwɤ˧}}} \textcolor{PineGreen}{\selectlanguage{french}démon boiteux} \zh{跛鬼}  
 \zh{量词}: \textcolor{darkblue}{\textbf{\ipa{v̩˧}}}  \mytextsc{clf}~: \textcolor{darkblue}{\textbf{\ipa{v̩˧}}} 
\lhead{\firstmark}
\rhead{\botmark}

\subsection{\hspace{-0.5cm} {\Large \textcolor{darkblue}{\textbf{\ipa{kʰɯ˧dʑɯ˧˥}}}}\hspace{0.5cm}[\kern2pt{\textcolor{darkblue}{\textbf{\ipa{kʰɯ˧dʑɯ˧}}}}\kern2pt]} \hypertarget{k\string_hM\string_Mdz£M\string_M\string_T1}{}
\markboth{\textcolor{darkblue}{\textbf{\ipa{kʰɯ˧dʑɯ˧˥}}}}{}
\textcolor{teal}{\mytextsc{nom}} \hspace{4pt} Ton~: MH\#.
\textcolor{PineGreen}{\selectlanguage{french}Bande molletière.} \zh{裹腿。}  \zh{量词}: \textcolor{darkblue}{\textbf{\ipa{dzi˧}}}  \mytextsc{clf}~: \textcolor{darkblue}{\textbf{\ipa{dzi˧}}} \textcolor{PineGreen}{\selectlanguage{french}paire}
\lhead{\firstmark}
\rhead{\botmark}

\subsection{\hspace{-0.5cm} {\Large \textcolor{darkblue}{\textbf{\ipa{kʰɯ˧pi˧}}}}\hspace{0.5cm}[\kern2pt{\textcolor{darkblue}{\textbf{\ipa{kʰɯ˧pi˧}}}}\kern2pt]} \hypertarget{k\string_hM\string_Mpi\string_M1}{}
\markboth{\textcolor{darkblue}{\textbf{\ipa{kʰɯ˧pi˧}}}}{}
\textcolor{teal}{\mytextsc{verbe}} \hspace{4pt} Ton~: .
\textcolor{PineGreen}{\selectlanguage{french}Trébucher.} \zh{绊。}  ¶ \textcolor{darkblue}{\textbf{\ipa{njɤ˧ kʰɯ˧pi˧-ze˧!}}} \textcolor{PineGreen}{\selectlanguage{french}j'ai trébuché!} \zh{我绊了一跤!}  

\lhead{\firstmark}
\rhead{\botmark}

\subsection{\hspace{-0.5cm} {\Large \textcolor{darkblue}{\textbf{\ipa{kʰɯ˧pʰv̩˩}}}}\hspace{0.5cm}[\kern2pt{\textcolor{darkblue}{\textbf{\ipa{kʰɯ˧pʰv̩˩}}}}\kern2pt]} \hypertarget{k\string_hM\string_Mp\string_hv\string_=\string_B1}{}
\markboth{\textcolor{darkblue}{\textbf{\ipa{kʰɯ˧pʰv̩˩}}}}{}
\textcolor{teal}{\mytextsc{nom}} \hspace{4pt} Ton~: L\#.
\textit{\textcolor{PineGreen}{\selectlanguage{french}archaïque}} [\zh{古语}] \textcolor{PineGreen}{\selectlanguage{french}Chinois.} \zh{汉族。}  \zh{量词}: \textcolor{darkblue}{\textbf{\ipa{v̩˧}}}  \mytextsc{clf}~: \textcolor{darkblue}{\textbf{\ipa{v̩˧}}} 
\lhead{\firstmark}
\rhead{\botmark}

\subsection{\hspace{-0.5cm} {\Large \textcolor{darkblue}{\textbf{\ipa{kʰɯ˧tʰo˧˥}}}}\hspace{0.5cm}[\kern2pt{\textcolor{darkblue}{\textbf{\ipa{kʰɯ˧tʰo˥}}}}\kern2pt]} \hypertarget{k\string_hM\string_Mt\string_ho\string_M\string_T1}{}
\markboth{\textcolor{darkblue}{\textbf{\ipa{kʰɯ˧tʰo˧˥}}}}{}
\textcolor{teal}{\mytextsc{nom}} \hspace{4pt} Ton~: H\#.
\textcolor{PineGreen}{\selectlanguage{french}Chaîne de fer, pour attacher les chevilles d'un criminel.} \zh{脚链。}  ¶ \textcolor{darkblue}{\textbf{\ipa{kʰɯ˧tʰo˧ kʰɯ˥}}} \textcolor{PineGreen}{\selectlanguage{french}mettre les chaînes (aux pieds de quelqu'un)} \zh{戴上脚链(在一个人的脚上)}  

\lhead{\firstmark}
\rhead{\botmark}

\subsection{\hspace{-0.5cm} {\Large \textcolor{darkblue}{\textbf{\ipa{kʰɯ˧tʰv̩\#˥}}}}\hspace{0.5cm}[\kern2pt{\textcolor{darkblue}{\textbf{\ipa{kʰɯ˧tʰv̩˧}}}}\kern2pt]} \hypertarget{k\string_hM\string_Mt\string_hv\string_=\#\string_T1}{}
\markboth{\textcolor{darkblue}{\textbf{\ipa{kʰɯ˧tʰv̩\#˥}}}}{}
\textcolor{teal}{\mytextsc{nom}} \hspace{4pt} Ton~: \#H.
\textcolor{PineGreen}{\selectlanguage{french}Pédale du métier à tisser (pour inverser la position verticale des fils de trame entre 2 passages du volant).} \zh{织布机的脚蹬子=踏板。}  \zh{量词}: \textcolor{darkblue}{\textbf{\ipa{dze˩}}}  \mytextsc{clf}~: \textcolor{darkblue}{\textbf{\ipa{dze˩}}} 
\lhead{\firstmark}
\rhead{\botmark}

\subsection{\hspace{-0.5cm} {\Large \textcolor{darkblue}{\textbf{\ipa{kʰɯ˧tsɯ˧bæ˥}}}}\hspace{0.5cm}[\kern2pt{\textcolor{darkblue}{\textbf{\ipa{kʰɯ˧tsɯ˧bæ˥}}}}\kern2pt]} \hypertarget{k\string_hM\string_MtsM\string_Mb\{\string_T1}{}
\markboth{\textcolor{darkblue}{\textbf{\ipa{kʰɯ˧tsɯ˧bæ˥}}}}{}
\textcolor{teal}{\mytextsc{nom}} \hspace{4pt} Ton~: H\#.
\textcolor{PineGreen}{\selectlanguage{french}Bande de tissu large d'une dizaine de centimères, utilisée autrefois pour serrer le pantalon, qui était très ample; c'était un élément fonctionnel mais également décoratif, préparé en belle étoffe; il provenait des régions tibétaines.} \zh{绑腿布:用来绑裤腿的一块缠布,也有装饰功能(从藏族地区传过来的)。} 
\lhead{\firstmark}
\rhead{\botmark}

\subsection{\hspace{-0.5cm} {\Large \textcolor{darkblue}{\textbf{\ipa{kʰɯ˧tsʰɤ˧˥}}}}\hspace{0.5cm}[\kern2pt{\textcolor{darkblue}{\textbf{\ipa{kʰɯ˧tsʰɤ˧˥}}}}\kern2pt]} \hypertarget{k\string_hM\string_Mts\string_h7\string_M\string_T1}{}
\markboth{\textcolor{darkblue}{\textbf{\ipa{kʰɯ˧tsʰɤ˧˥}}}}{}
\textcolor{teal}{\mytextsc{nom}} \hspace{4pt} Ton~: MH\#.
\textcolor{PineGreen}{\selectlanguage{french}Jambe.} \zh{腿,脚。}  \zh{量词}: \textcolor{darkblue}{\textbf{\ipa{pʰo˧˥}}}  \mytextsc{clf}~: \textcolor{darkblue}{\textbf{\ipa{pʰo˧˥}}} 
\lhead{\firstmark}
\rhead{\botmark}

\subsection{\hspace{-0.5cm} {\Large \textcolor{darkblue}{\textbf{\ipa{kʰɯ˧ʈʂæ˧˥}}}}\hspace{0.5cm}[\kern2pt{\textcolor{darkblue}{\textbf{\ipa{kʰɯ˧ʈʂæ˧˥}}}}\kern2pt]} \hypertarget{k\string_hM\string_Mt`s`\{\string_M\string_T1}{}
\markboth{\textcolor{darkblue}{\textbf{\ipa{kʰɯ˧ʈʂæ˧˥}}}}{}
\textcolor{teal}{\mytextsc{nom}} \hspace{4pt} Ton~: MH\#.
\textcolor{PineGreen}{\selectlanguage{french}Cheville.} \zh{踝关节。}  \zh{量词}: \textcolor{darkblue}{\textbf{\ipa{ʈʂæ˧˥}}}  \mytextsc{clf}~: \textcolor{darkblue}{\textbf{\ipa{ʈʂæ˧˥}}} 
\lhead{\firstmark}
\rhead{\botmark}

\subsection{\hspace{-0.5cm} {\Large \textcolor{darkblue}{\textbf{\ipa{kʰɯ˧ʈʂɤ\#˥}}}}\hspace{0.5cm}[\kern2pt{\textcolor{darkblue}{\textbf{\ipa{kʰɯ˧ʈʂɤ˧}}}}\kern2pt]} \hypertarget{k\string_hM\string_Mt`s`7\#\string_T1}{}
\markboth{\textcolor{darkblue}{\textbf{\ipa{kʰɯ˧ʈʂɤ\#˥}}}}{}
\textcolor{teal}{\mytextsc{nom}} \hspace{4pt} Ton~: \#H.
\textcolor{PineGreen}{\selectlanguage{french}Griffes de poulet.} \zh{鸡爪。}  ¶ \textcolor{darkblue}{\textbf{\ipa{kʰɯ˧ʈʂɤ˧ tʰv̩˧-ɭɯ\#˥}}} \textcolor{PineGreen}{\selectlanguage{french}\mytextsc{n}+\mytextsc{dem}+\mytextsc{clf}} \zh{这只鸡爪}  
 ¶ \textcolor{darkblue}{\textbf{\ipa{kʰɯ˧ʈʂɤ˧ tʰv̩˧-ʈv̩˥\#}}} \textcolor{PineGreen}{\selectlanguage{french}\mytextsc{n}+\mytextsc{dem}+\mytextsc{clf}} \zh{这只鸡爪}  
 \zh{量词}: \textcolor{darkblue}{\textbf{\ipa{ʈv̩˩ / ɭɯ˧}}}  \mytextsc{clf}~: \textcolor{darkblue}{\textbf{\ipa{ʈv̩˩ / ɭɯ˧}}} 
\lhead{\firstmark}
\rhead{\botmark}

\subsection{\hspace{-0.5cm} {\Large \textcolor{darkblue}{\textbf{\ipa{kʰɯ˧ʐɯ˥\$}}}}\hspace{0.5cm}[\kern2pt{\textcolor{darkblue}{\textbf{\ipa{kʰɯ˧ʐɯ˥}}}}\kern2pt]} \hypertarget{k\string_hM\string_Mz`M\string_T\$1}{}
\markboth{\textcolor{darkblue}{\textbf{\ipa{kʰɯ˧ʐɯ˥\$}}}}{}
\textcolor{teal}{\mytextsc{nom}} \hspace{4pt} Ton~: H\$.
\textcolor{PineGreen}{\selectlanguage{french}Vin de riz (faiblement alcoolisé).} \zh{黄酒。}  \zh{量词}: \textcolor{darkblue}{\textbf{\ipa{qʰwɤ˧˥}}}  \mytextsc{clf}~: \textcolor{darkblue}{\textbf{\ipa{qʰwɤ˧˥}}} \textcolor{PineGreen}{\selectlanguage{french}bol}
\lhead{\firstmark}
\rhead{\botmark}

\subsection{\hspace{-0.5cm} {\Large \textcolor{darkblue}{\textbf{\ipa{kʰɯ˩}}}}\hspace{0.5cm}[\kern2pt{\textcolor{darkblue}{\textbf{\ipa{kʰɯ˥}}}}\kern2pt]} \hypertarget{k\string_hM\string_B1}{}
\markboth{\textcolor{darkblue}{\textbf{\ipa{kʰɯ˩}}}}{}
\textcolor{teal}{\mytextsc{nom}} \hspace{4pt} Ton~: L.
\textcolor{PineGreen}{\selectlanguage{french}Fil.} \zh{线。}  \zh{量词}: \textcolor{darkblue}{\textbf{\ipa{kʰɯ˩}}}  \mytextsc{clf}~: \textcolor{darkblue}{\textbf{\ipa{kʰɯ˩}}} \textcolor{PineGreen}{\selectlanguage{french}objets longs}
\lhead{\firstmark}
\rhead{\botmark}

\subsection{\hspace{-0.5cm} {\Large \textcolor{darkblue}{\textbf{\ipa{kʰɯ˩\textsubscript{b}}}}}\hspace{0.5cm}[\kern2pt{\textcolor{darkblue}{\textbf{\ipa{kʰɯ˥}}}}\kern2pt]} \hypertarget{k\string_hM\string_Bb1}{}
\markboth{\textcolor{darkblue}{\textbf{\ipa{kʰɯ˩\textsubscript{b}}}}}{}
\textcolor{teal}{\mytextsc{classificateur}} \hspace{4pt} Ton~: L\textsubscript{b}.
\textcolor{PineGreen}{\selectlanguage{french}Brin (d'herbe, de fil, de ficelle…).} \zh{量词:线(一根、一条)。}  ¶ \textcolor{darkblue}{\textbf{\ipa{kʰɯ˧ | ɖɯ˧-kʰɯ˩}}} \textcolor{PineGreen}{\selectlanguage{french}un brin de fil} \zh{一根线}  
 ¶ \textcolor{darkblue}{\textbf{\ipa{zɯ˧ | ɖɯ˧-kʰɯ˩}}} \textcolor{PineGreen}{\selectlanguage{french}un brin d'herbe} \zh{一根草}  
 ¶ \textcolor{darkblue}{\textbf{\ipa{bæ˩ ɖɯ˥-kʰɯ˩}}} \textcolor{PineGreen}{\selectlanguage{french}un brin de corde, un bout de corde} \zh{一条绳子}  
 ¶ \textcolor{darkblue}{\textbf{\ipa{kʰɯ˧ | ʈʂʰɯ˧-kʰɯ˧˥}}} \textcolor{PineGreen}{\selectlanguage{french}ce brin (note: schéma tonal irrégulier)} \zh{这根线}  

\lhead{\firstmark}
\rhead{\botmark}

\subsection{\hspace{-0.5cm} {\Large \textcolor{darkblue}{\textbf{\ipa{kʰɯ˩pv̩˩}}}}\hspace{0.5cm}[\kern2pt{\textcolor{darkblue}{\textbf{\ipa{kʰɯ˩pv̩˩˥}}}}\kern2pt]} \hypertarget{k\string_hM\string_Bpv\string_=\string_B1}{}
\markboth{\textcolor{darkblue}{\textbf{\ipa{kʰɯ˩pv̩˩}}}}{}
\textcolor{teal}{\mytextsc{nom}} \hspace{4pt} Ton~: L.
\textcolor{PineGreen}{\selectlanguage{french}Navette du métier à tisser; elle est actuellement confectionnée au plus simple, en prenant une tige de tournesol ou un bambou fin.} \zh{梭,梭子。}  \zh{量词}: \textcolor{darkblue}{\textbf{\ipa{ɭɯ˧}}}  \mytextsc{clf}~: \textcolor{darkblue}{\textbf{\ipa{ɭɯ˧}}} \textit{Voir~:} \hyperlink{}{\textcolor{darkblue}{\textbf{\ipa{pv̩˧qʰwɤ˥}}}} 
\lhead{\firstmark}
\rhead{\botmark}

\subsection{\hspace{-0.5cm} {\Large \textcolor{darkblue}{\textbf{\ipa{kʰɯ˩ʈɯ˩}}}}\hspace{0.5cm}[\kern2pt{\textcolor{darkblue}{\textbf{\ipa{kʰɯ˩ʈɯ˩˥}}}}\kern2pt]} \hypertarget{k\string_hM\string_Bt`M\string_B1}{}
\markboth{\textcolor{darkblue}{\textbf{\ipa{kʰɯ˩ʈɯ˩}}}}{}
\textcolor{teal}{\mytextsc{nom}} \hspace{4pt} Ton~: L.
\textcolor{PineGreen}{\selectlanguage{french}Racine.} \zh{根。}  ¶ \textcolor{darkblue}{\textbf{\ipa{si˧dzi˩-kʰɯ˩ʈɯ˩}}} \textcolor{PineGreen}{\selectlanguage{french}racines d'arbre} \zh{树根}  
 \zh{量词}: \textcolor{darkblue}{\textbf{\ipa{ʈv̩˩}}}  \mytextsc{clf}~: \textcolor{darkblue}{\textbf{\ipa{ʈv̩˩}}} 
\lhead{\firstmark}
\rhead{\botmark}

\subsection{\hspace{-0.5cm} {\Large \textcolor{darkblue}{\textbf{\ipa{kʰɯ˧˥}}} \textsubscript{1}}\hspace{0.5cm}[\kern2pt{\textcolor{darkblue}{\textbf{\ipa{kʰɯ˧˥}}}}\kern2pt]} \hypertarget{k\string_hM\string_M\string_T1}{}
\markboth{\textcolor{darkblue}{\textbf{\ipa{kʰɯ˧˥}}} \textsubscript{1}}{}
\textcolor{teal}{\mytextsc{verbe}} \hspace{4pt} Ton~: MH.
\ding{202} \textcolor{PineGreen}{\selectlanguage{french}Mettre, mettre dans (ex.: mettre de la farine dans une casserole); libérer, lâcher (ex.: un poulet qu'on tenait par les pattes); semer en enfonçant les graines; ranger, remettre à sa place.} \zh{放,装(如:装进袋里),点种,收下。}  ¶ \textcolor{darkblue}{\textbf{\ipa{kʰɯ˩\textasciitilde{}kʰɯ˧˥}}} \textcolor{PineGreen}{\selectlanguage{french}\mytextsc{red}} \zh{\mytextsc{red}}  
 ¶ \textcolor{darkblue}{\textbf{\ipa{qwɤ˧-qo˧ | si˧ tʰi˧-kʰɯ˧˥}}} \textcolor{PineGreen}{\selectlanguage{french}mettre/ajouter du bois dans le feu} \zh{放木头在火中}  
\ding{203} \textcolor{PineGreen}{\selectlanguage{french}Autoriser; a aussi valeur causative, mais ce causatif issu du verbe “mettre” paraît avoir un sens plus proche de “laisser”: par exemple “laisser sécher au soleil” plutôt que “faire sécher au soleil”.} \zh{让,\mytextsc{使动。}}  ¶ \textcolor{darkblue}{\textbf{\ipa{kʰv̩˩mi˩ zɯ˩\textasciitilde{}zɯ˩˥, | le˧-ɖæ˥-kʰɯ˩! | hĩ˧-zɯ˧\textasciitilde{}zɯ˥, | le˧-ʂæ˧-kʰɯ˥!}}} \textcolor{PineGreen}{\selectlanguage{french}La vie des chiens s'en est trouvée écourtée, et celle des hommes allongée! (Résumé en quelques mots du récit “Le chien échange sa longévité avec l'homme”)} \zh{够的寿命,变短了/使得变短!(而)人的寿命,变长了/使得变长!(《狗和人交换寿命》故事的一个提要)}  
 ¶ \textcolor{darkblue}{\textbf{\ipa{hwæ˧ kʰɯ˧ ə˥-bi˩? | - hwæ˧ kʰɯ˧-bi˥!}}} \textcolor{PineGreen}{\selectlanguage{french}Tu es d'accord pour acheter? - Oui!} \zh{(你)让买吗? - 让买!}  
 ¶ \textcolor{darkblue}{\textbf{\ipa{tɕʰi˧ kʰɯ˧ ə˥-bi˩?}}} \textcolor{PineGreen}{\selectlanguage{french}Tu es d'accord pour vendre?} \zh{(你)让卖吗?}  
 ¶ \textcolor{darkblue}{\textbf{\ipa{dzɯ˧ kʰɯ˩ ə˩-bi˩?}}} \textcolor{PineGreen}{\selectlanguage{french}Tu es d'accord pour manger?} \zh{(你)让吃吗?}  
 ¶ \textcolor{darkblue}{\textbf{\ipa{tɕi˩ kʰɯ˥ ə˩-bi˩?}}} \textcolor{PineGreen}{\selectlanguage{french}Tu es d'accord pour écrire?} \zh{(你)让写吗?}  
 ¶ \textcolor{darkblue}{\textbf{\ipa{ʈʰɯ˩ kʰɯ˩ ə˥-bi˩?}}} \textcolor{PineGreen}{\selectlanguage{french}Tu es d'accord pour boire?} \zh{(你)让喝吗?}  
 ¶ \textcolor{darkblue}{\textbf{\ipa{ʐv̩˧ kʰɯ˥ ə˩-bi˩?}}} \textcolor{PineGreen}{\selectlanguage{french}Tu es d'accord pour coudre?} \zh{(你)让缝吗?}  

\lhead{\firstmark}
\rhead{\botmark}

\subsection{\hspace{-0.5cm} {\Large \textcolor{darkblue}{\textbf{\ipa{kʰɯ˧˥}}} \textsubscript{2}}\hspace{0.5cm}[\kern2pt{\textcolor{darkblue}{\textbf{\ipa{kʰɯ˧˥}}}}\kern2pt]} \hypertarget{k\string_hM\string_M\string_T2}{}
\markboth{\textcolor{darkblue}{\textbf{\ipa{kʰɯ˧˥}}} \textsubscript{2}}{}
\textcolor{teal}{\mytextsc{verbe}} \hspace{4pt} Ton~: MH.
\textcolor{PineGreen}{\selectlanguage{french}Lancer, jeter.} \zh{甩、扔(石头)。}  ¶ \textcolor{darkblue}{\textbf{\ipa{le˧-kʰɯ˧-ze˥}}} \textcolor{PineGreen}{\selectlanguage{french}\mytextsc{accomp} \string_ \mytextsc{pfv}} \zh{甩了}  
 ¶ \textcolor{darkblue}{\textbf{\ipa{lv̩˧mi˧ kʰɯ˧˥}}} \textcolor{PineGreen}{\selectlanguage{french}jeter une pierre} \zh{扔石头}  

\lhead{\firstmark}
\rhead{\botmark}

\subsection{\hspace{-0.5cm} {\Large \textcolor{darkblue}{\textbf{\ipa{kʰɯ˧˥}}} \textsubscript{3}}\hspace{0.5cm}[\kern2pt{\textcolor{darkblue}{\textbf{\ipa{kʰɯ˧˥}}}}\kern2pt]} \hypertarget{k\string_hM\string_M\string_T3}{}
\markboth{\textcolor{darkblue}{\textbf{\ipa{kʰɯ˧˥}}} \textsubscript{3}}{}
\textcolor{teal}{\mytextsc{verbe}} \hspace{4pt} Ton~: MH.
\textcolor{PineGreen}{\selectlanguage{french}Porter (un bracelet).} \zh{戴(手镯)。}  ¶ \textcolor{darkblue}{\textbf{\ipa{le˧-kʰɯ˧-ze˥}}} \textcolor{PineGreen}{\selectlanguage{french}\mytextsc{accomp} \string_ \mytextsc{pfv}} \zh{戴了}  
 ¶ \textcolor{darkblue}{\textbf{\ipa{lo˩dʑo˧ kʰɯ˩}}} \textcolor{PineGreen}{\selectlanguage{french}porter un bracelet} \zh{戴手镯}  

\lhead{\firstmark}
\rhead{\botmark}

\subsection{\hspace{-0.5cm} {\Large \textcolor{darkblue}{\textbf{\ipa{kʰv̩˧˥}}}}\hspace{0.5cm}[\kern2pt{\textcolor{darkblue}{\textbf{\ipa{kʰv̩˧˥}}}}\kern2pt]} \hypertarget{k\string_hv\string_=\string_M\string_T1}{}
\markboth{\textcolor{darkblue}{\textbf{\ipa{kʰv̩˧˥}}}}{}
\textcolor{teal}{\mytextsc{nom}} \hspace{4pt} Ton~: MH.
\ding{202} \textcolor{PineGreen}{\selectlanguage{french}Année, an.} \zh{年、岁。}  ¶ \textcolor{darkblue}{\textbf{\ipa{kʰv̩˧-mæ˥}}} \textcolor{PineGreen}{\selectlanguage{french}fin de l'année} \zh{年尾}  
 ¶ \textcolor{darkblue}{\textbf{\ipa{kʰv̩˧-mæ˥ ʂæ˩}}} \textcolor{PineGreen}{\selectlanguage{french}année longue, de 13 mois; cela a lieu tous les 4 ans environ} \zh{闰年(有13个月)}  
 ¶ \textcolor{darkblue}{\textbf{\ipa{kʰv̩˧-mæ˥ ɖæ˩}}} \textcolor{PineGreen}{\selectlanguage{french}année normale, à douze mois} \zh{正常的年份,普通年:一年十二个月}  
\ding{203} \textcolor{PineGreen}{\selectlanguage{french}Signe astrologique.} \zh{生肖。}  ¶ \textcolor{darkblue}{\textbf{\ipa{no˧ | ə˧tso˧ kʰv̩˧ ɲi˥?}}} \textcolor{PineGreen}{\selectlanguage{french}De quel signe es-tu?} \zh{你是属什么的?}  

\lhead{\firstmark}
\rhead{\botmark}

\subsection{\hspace{-0.5cm} {\Large \textcolor{darkblue}{\textbf{\ipa{kʰv̩˥}}} \textsubscript{1}}\hspace{0.5cm}[\kern2pt{\textcolor{darkblue}{\textbf{\ipa{kʰv̩˧˥}}}}\kern2pt]} \hypertarget{k\string_hv\string_=\string_T1}{}
\markboth{\textcolor{darkblue}{\textbf{\ipa{kʰv̩˥}}} \textsubscript{1}}{}
\textcolor{teal}{\mytextsc{nom}} \hspace{4pt} Ton~: \#H.
\textcolor{PineGreen}{\selectlanguage{french}Nid (monosyllabe).} \zh{(鸟)巢。}  ¶ \textcolor{darkblue}{\textbf{\ipa{kʰv̩˧ ʈʂʰɯ˧-ɭɯ\#˥}}} \textcolor{PineGreen}{\selectlanguage{french}\mytextsc{n}+\mytextsc{dem}+\mytextsc{clf}} \zh{这只鸟巢}  
 \zh{量词}: \textcolor{darkblue}{\textbf{\ipa{ɭɯ˧}}}  \mytextsc{clf}~: \textcolor{darkblue}{\textbf{\ipa{ɭɯ˧}}} 
\lhead{\firstmark}
\rhead{\botmark}

\subsection{\hspace{-0.5cm} {\Large \textcolor{darkblue}{\textbf{\ipa{kʰv̩˥}}} \textsubscript{2}}\hspace{0.5cm}[\kern2pt{\textcolor{darkblue}{\textbf{\ipa{kʰv̩˥}}}}\kern2pt]} \hypertarget{k\string_hv\string_=\string_T2}{}
\markboth{\textcolor{darkblue}{\textbf{\ipa{kʰv̩˥}}} \textsubscript{2}}{}
\textcolor{teal}{\mytextsc{verbe}} \hspace{4pt} Ton~: H.
\textcolor{PineGreen}{\selectlanguage{french}Couper (ex.: de l'herbe) pour récolter.} \zh{割(草)。}  ¶ \textcolor{darkblue}{\textbf{\ipa{le˧-kʰv̩˥-ze˩}}} \textcolor{PineGreen}{\selectlanguage{french}\mytextsc{accomp} \string_ \mytextsc{pfv}} \zh{割了}  
 ¶ \textcolor{darkblue}{\textbf{\ipa{zɯ˧-kʰv̩˧}}} \textcolor{PineGreen}{\selectlanguage{french}couper de l'herbe} \zh{割草}  

\lhead{\firstmark}
\rhead{\botmark}

\subsection{\hspace{-0.5cm} {\Large \textcolor{darkblue}{\textbf{\ipa{kʰv̩˥}}} \textsubscript{3}}\hspace{0.5cm}[\kern2pt{\textcolor{darkblue}{\textbf{\ipa{kʰv̩˥}}}}\kern2pt]} \hypertarget{k\string_hv\string_=\string_T3}{}
\markboth{\textcolor{darkblue}{\textbf{\ipa{kʰv̩˥}}} \textsubscript{3}}{}
\textcolor{teal}{\mytextsc{nom}} \hspace{4pt} Ton~: \#H.
\textcolor{PineGreen}{\selectlanguage{french}Chien (monosyllabe).} \zh{狗。}  ¶ \textcolor{darkblue}{\textbf{\ipa{kʰv̩˧-ʂe˧ dzɯ˧}}} \textcolor{PineGreen}{\selectlanguage{french}manger de la viande de chien (pratique qui va droit à l'encontre de la culture na, dans laquelle le chien est considéré comme bienfaiteur de l'homme)} \zh{吃狗肉}  
 ¶ \textcolor{darkblue}{\textbf{\ipa{kʰv̩˧-zɯ˧\textasciitilde{}zɯ˥}}} \textcolor{PineGreen}{\selectlanguage{french}l'existence du chien, la vie du chien (qu'il a échangée avec l'homme, selon la légende)} \zh{狗的生命(传说狗与人交换了生命)}  
 ¶ \textcolor{darkblue}{\textbf{\ipa{kʰv̩˧ tʰv̩˧-mi˥\#}}} \textcolor{PineGreen}{\selectlanguage{french}\mytextsc{n}+\mytextsc{dem}+\mytextsc{clf}} \zh{那条狗}  
 ¶ \textcolor{darkblue}{\textbf{\ipa{kʰv̩˧-gɤ˥ljɤ˩}}} \textcolor{PineGreen}{\selectlanguage{french}chien errant} \zh{流浪狗}  
 \zh{量词}: \textcolor{darkblue}{\textbf{\ipa{mi˩}}} \textcolor{darkblue}{\textbf{\ipa{v̩˧}}} \textcolor{darkblue}{\textbf{\ipa{jɤ˧˥}}}  \mytextsc{clf}~: \textcolor{darkblue}{\textbf{\ipa{mi˩}}} \textcolor{darkblue}{\textbf{\ipa{v̩˧}}} \textcolor{darkblue}{\textbf{\ipa{jɤ˧˥}}} 
\lhead{\firstmark}
\rhead{\botmark}

\subsection{\hspace{-0.5cm} {\Large \textcolor{darkblue}{\textbf{\ipa{kʰv̩˥}}} \textsubscript{4}}\hspace{0.5cm}[\kern2pt{\textcolor{darkblue}{\textbf{\ipa{kʰv̩˥}}}}\kern2pt]} \hypertarget{k\string_hv\string_=\string_T4}{}
\markboth{\textcolor{darkblue}{\textbf{\ipa{kʰv̩˥}}} \textsubscript{4}}{}
\textcolor{teal}{\mytextsc{verbe}} \hspace{4pt} Ton~: H.
\textcolor{PineGreen}{\selectlanguage{french}Voler.} \zh{偷。}  ¶ \textcolor{darkblue}{\textbf{\ipa{hĩ˧-bv̩˧ tso˧\textasciitilde{}tso˧ kʰv̩˧}}} \textcolor{PineGreen}{\selectlanguage{french}voler les affaires de quelqu'un} \zh{偷别人的东西}  

\lhead{\firstmark}
\rhead{\botmark}

\subsection{\hspace{-0.5cm} {\Large \textcolor{darkblue}{\textbf{\ipa{kʰv̩˧˥\textsubscript{a}}}}}\hspace{0.5cm}[\kern2pt{\textcolor{darkblue}{\textbf{\ipa{kʰv̩˩˥}}}}\kern2pt]} \hypertarget{k\string_hv\string_=\string_M\string_Ta1}{}
\markboth{\textcolor{darkblue}{\textbf{\ipa{kʰv̩˧˥\textsubscript{a}}}}}{}
\textcolor{teal}{\mytextsc{classificateur}} \hspace{4pt} Ton~: MH\textsubscript{a}.
\textcolor{PineGreen}{\selectlanguage{french}Année.} \zh{量词:年、岁。}  ¶ \textcolor{darkblue}{\textbf{\ipa{ɖɯ˧-kʰv̩˧˥}}} \textcolor{PineGreen}{\selectlanguage{french}une année} \zh{一年}  

\lhead{\firstmark}
\rhead{\botmark}

\subsection{\hspace{-0.5cm} {\Large \textcolor{darkblue}{\textbf{\ipa{kʰv̩˧bv̩˧˥}}}}\hspace{0.5cm}[\kern2pt{\textcolor{darkblue}{\textbf{\ipa{kʰv̩˧bv̩˧˥}}}}\kern2pt]} \hypertarget{k\string_hv\string_=\string_Mbv\string_=\string_M\string_T1}{}
\markboth{\textcolor{darkblue}{\textbf{\ipa{kʰv̩˧bv̩˧˥}}}}{}
\textcolor{teal}{\mytextsc{nom}} \hspace{4pt} Ton~: MH\#.
\textcolor{PineGreen}{\selectlanguage{french}Chenil.} \zh{狗窝。}  \zh{量词}: \textcolor{darkblue}{\textbf{\ipa{ɭɯ˧}}}  \mytextsc{clf}~: \textcolor{darkblue}{\textbf{\ipa{ɭɯ˧}}} 
\lhead{\firstmark}
\rhead{\botmark}

\subsection{\hspace{-0.5cm} {\Large \textcolor{darkblue}{\textbf{\ipa{kʰv̩˩-kʰɤ˩}}}}\hspace{0.5cm}[\kern2pt{\textcolor{darkblue}{\textbf{\ipa{xxxx non-correspondance entre le nombre de morphèmes et le nombre de tons de morphèmes}}}}\kern2pt]} \hypertarget{k\string_hv\string_=\string_B-k\string_h7\string_B1}{}
\markboth{\textcolor{darkblue}{\textbf{\ipa{kʰv̩˩-kʰɤ˩}}}}{}
\textcolor{teal}{\mytextsc{nom}} \hspace{4pt} Ton~: L.
\textcolor{PineGreen}{\selectlanguage{french}Nid de poule, pondoir, endroit où la poule pond.} \zh{鸡窝。}  \zh{量词}: \textcolor{darkblue}{\textbf{\ipa{ɭɯ˧}}}  \mytextsc{clf}~: \textcolor{darkblue}{\textbf{\ipa{ɭɯ˧}}} 
\lhead{\firstmark}
\rhead{\botmark}

\subsection{\hspace{-0.5cm} {\Large \textcolor{darkblue}{\textbf{\ipa{kʰv̩˧kʰv̩˩}}}}\hspace{0.5cm}[\kern2pt{\textcolor{darkblue}{\textbf{\ipa{kʰv̩˩kʰv̩˩˥}}}}\kern2pt]} \hypertarget{k\string_hv\string_=\string_Mk\string_hv\string_=\string_B1}{}
\markboth{\textcolor{darkblue}{\textbf{\ipa{kʰv̩˧kʰv̩˩}}}}{}
\textcolor{teal}{\mytextsc{nom}} \hspace{4pt} Ton~: L\#.
\textcolor{PineGreen}{\selectlanguage{french}Année du chien.} \zh{狗年。} 
\lhead{\firstmark}
\rhead{\botmark}

\subsection{\hspace{-0.5cm} {\Large \textcolor{darkblue}{\textbf{\ipa{kʰv̩˧kwæ˧}}}}\hspace{0.5cm}[\kern2pt{\textcolor{darkblue}{\textbf{\ipa{kʰv̩˧kwæ˩}}}}\kern2pt]} \hypertarget{k\string_hv\string_=\string_Mkw\{\string_M1}{}
\markboth{\textcolor{darkblue}{\textbf{\ipa{kʰv̩˧kwæ˧}}}}{}
\textcolor{teal}{\mytextsc{nom}} \hspace{4pt} Ton~: M.
\textcolor{PineGreen}{\selectlanguage{french}Concombre amer.} \zh{苦瓜。}  Emprunt~: chinois  \zh{苦瓜}
 \zh{量词}: \textcolor{darkblue}{\textbf{\ipa{ɭɯ˧}}}  \mytextsc{clf}~: \textcolor{darkblue}{\textbf{\ipa{ɭɯ˧}}} 
\lhead{\firstmark}
\rhead{\botmark}

\subsection{\hspace{-0.5cm} {\Large \textcolor{darkblue}{\textbf{\ipa{kʰv̩˧mæ˧}}}}\hspace{0.5cm}[\kern2pt{\textcolor{darkblue}{\textbf{\ipa{kʰv̩˧mæ˥}}}}\kern2pt]} \hypertarget{k\string_hv\string_=\string_Mm\{\string_M1}{}
\markboth{\textcolor{darkblue}{\textbf{\ipa{kʰv̩˧mæ˧}}}}{}
\textcolor{teal}{\mytextsc{nom}} \hspace{4pt} Ton~: M.
\textcolor{PineGreen}{\selectlanguage{french}Voleur, bandit.} \zh{强盗。}  ¶ \textcolor{darkblue}{\textbf{\ipa{kʰv̩˧mæ˧ ʝi˧-hĩ˧-hĩ˧}}} \textcolor{PineGreen}{\selectlanguage{french}personne qui est un bandit, bandit} \zh{当强盗的人=强盗}  
 ¶ \textcolor{darkblue}{\textbf{\ipa{kʰv̩˧mæ˧-ni˩-zo˩! | hĩ˧ lɑ˩-ho˩!}}} \textcolor{PineGreen}{\selectlanguage{french}Ca doit être un voleur! Il se pourrait qu'il frappe les gens!} \zh{他像强盗似的!会打人的!}  
 ¶ \textcolor{darkblue}{\textbf{\ipa{kʰv̩˧mæ˧-ʑi˩}}} \textcolor{PineGreen}{\selectlanguage{french}prison: littéralement “maison des voleurs”} \zh{监狱。直译:“贼家”}  
 ¶ \textcolor{darkblue}{\textbf{\ipa{kʰv̩˧mæ˧-ʝi˧-hĩ˧, | lo˧ʑi˥bv̩˩-qo˩ ʈæ˩!}}} \textcolor{PineGreen}{\selectlanguage{french}les voleurs, on les met en prison!} \zh{贼,被关在监狱!}  
 ¶ \textcolor{darkblue}{\textbf{\ipa{no˧ | kʰv̩˧mæ˧-pʰæ˧qʰwɤ˩-ne˩-ʝi˩-zo˩!}}} \textcolor{PineGreen}{\selectlanguage{french}Toi, tu m'as une tête de voleur! (accusation lancée à quelqu'un qu'on pense être un voleur)} \zh{你有一张贼脸!(控告一个人)}  
 \zh{量词}: \textcolor{darkblue}{\textbf{\ipa{v̩˧}}}  \mytextsc{clf}~: \textcolor{darkblue}{\textbf{\ipa{v̩˧}}} 
\lhead{\firstmark}
\rhead{\botmark}

\subsection{\hspace{-0.5cm} {\Large \textcolor{darkblue}{\textbf{\ipa{kʰv̩˩mi˩}}}}\hspace{0.5cm}[\kern2pt{\textcolor{darkblue}{\textbf{\ipa{kʰv̩˧mi˧}}}}\kern2pt]} \hypertarget{k\string_hv\string_=\string_Bmi\string_B1}{}
\markboth{\textcolor{darkblue}{\textbf{\ipa{kʰv̩˩mi˩}}}}{}
\textcolor{teal}{\mytextsc{nom}} \hspace{4pt} Ton~: L.
\textcolor{PineGreen}{\selectlanguage{french}Chien (sans spécifier le sexe).} \zh{狗。}  ¶ \textcolor{darkblue}{\textbf{\ipa{kʰv̩˩mi˩ ʈʂʰɯ˩-jɤ˧}}} \textcolor{PineGreen}{\selectlanguage{french}\mytextsc{n}+\mytextsc{dem}+\mytextsc{clf}} \zh{这条狗}  
 ¶ \textcolor{darkblue}{\textbf{\ipa{di˧qo˧-kʰv̩˩mi˩}}} \textcolor{PineGreen}{\selectlanguage{french}les chiens de la plaine (qui à la différence des chiens des petits hameaux de montagne voient beaucoup de passage et sont moins susceptibles de mordre les inconnus de passage)} \zh{平坝的狗}  
 ¶ \textcolor{darkblue}{\textbf{\ipa{kʰv̩˩mi˩-gɤ˥ljɤ˩}}} \textcolor{PineGreen}{\selectlanguage{french}chien errant} \zh{流浪的狗}  
 \zh{量词}: \textcolor{darkblue}{\textbf{\ipa{v̩˧, terme respectueux (le même que pour les humains)}}} \textcolor{darkblue}{\textbf{\ipa{on peut aussi dire: jɤ˧˥}}}  \mytextsc{clf}~: \textcolor{darkblue}{\textbf{\ipa{v̩˧, terme respectueux (le même que pour les humains)}}} \textcolor{darkblue}{\textbf{\ipa{on peut aussi dire: jɤ˧˥}}} 
\lhead{\firstmark}
\rhead{\botmark}

\subsection{\hspace{-0.5cm} {\Large \textcolor{darkblue}{\textbf{\ipa{kʰv̩˧mv̩˥}}}}\hspace{0.5cm}[\kern2pt{\textcolor{darkblue}{\textbf{\ipa{kʰv̩˩mv̩˩˥}}}}\kern2pt]} \hypertarget{k\string_hv\string_=\string_Mmv\string_=\string_T1}{}
\markboth{\textcolor{darkblue}{\textbf{\ipa{kʰv̩˧mv̩˥}}}}{}
\textcolor{teal}{\mytextsc{nom}} \hspace{4pt} Ton~: H\#.
\textcolor{PineGreen}{\selectlanguage{french}Chienne, petit chiot femelle. Le terme est également employé comme nom provisoire pour les fillettes pendant leurs premiers mois, avant qu'on ne leur donne un vrai nom. Le vilain nom dont on l'affuble vise à éviter que le nourrisson ne soit repéré par de mauvais esprits. (Actuellement, l'état-civil nécessite qu'un nom soit donné dès la naissance; mais celui-ci ne commence à être employé dans les conversations familiales qu'une fois passés les premiers mois.).} \zh{小母狗(给刚出生的女孩起的名字,让鬼对她不感兴趣,不会来害小孩)。}  \zh{量词}: \textcolor{darkblue}{\textbf{\ipa{v̩˧}}}  \mytextsc{clf}~: \textcolor{darkblue}{\textbf{\ipa{v̩˧}}} 
\lhead{\firstmark}
\rhead{\botmark}

\subsection{\hspace{-0.5cm} {\Large \textcolor{darkblue}{\textbf{\ipa{kʰv̩˧nɑ˥}}}}\hspace{0.5cm}[\kern2pt{\textcolor{darkblue}{\textbf{\ipa{kʰv̩˩nɑ˧˥}}}}\kern2pt]} \hypertarget{k\string_hv\string_=\string_MnA\string_T1}{}
\markboth{\textcolor{darkblue}{\textbf{\ipa{kʰv̩˧nɑ˥}}}}{}
\textcolor{teal}{\mytextsc{nom}} \hspace{4pt} Ton~: H\#.
\textcolor{PineGreen}{\selectlanguage{french}Chien (registre de langage relevé).} \zh{狗。}  \zh{量词}: \textcolor{darkblue}{\textbf{\ipa{mi˩}}}  \mytextsc{clf}~: \textcolor{darkblue}{\textbf{\ipa{mi˩}}} 
\lhead{\firstmark}
\rhead{\botmark}

\subsection{\hspace{-0.5cm} {\Large \textcolor{darkblue}{\textbf{\ipa{kʰv̩˧pʰæ˧}}}}\hspace{0.5cm}[\kern2pt{\textcolor{darkblue}{\textbf{\ipa{kʰv̩˧pʰæ˥}}}}\kern2pt]} \hypertarget{k\string_hv\string_=\string_Mp\string_h\{\string_M1}{}
\markboth{\textcolor{darkblue}{\textbf{\ipa{kʰv̩˧pʰæ˧}}}}{}
\textcolor{teal}{\mytextsc{nom}} \hspace{4pt} Ton~: M.
\textcolor{PineGreen}{\selectlanguage{french}Âge.} \zh{年龄。}  ¶ \textcolor{darkblue}{\textbf{\ipa{kʰv̩˧pʰæ˧ tɕi˩}}} \textcolor{PineGreen}{\selectlanguage{french}jeune} \zh{年轻}  
 ¶ \textcolor{darkblue}{\textbf{\ipa{kʰv̩˧pʰæ˧ | tɕi˩-hĩ˩˥}}} \textcolor{PineGreen}{\selectlanguage{french}jeune} \zh{年轻的}  

\lhead{\firstmark}
\rhead{\botmark}

\subsection{\hspace{-0.5cm} {\Large \textcolor{darkblue}{\textbf{\ipa{kʰv̩˧-pʰo˥}}}}\hspace{0.5cm}[\kern2pt{\textcolor{darkblue}{\textbf{\ipa{xxxx non-correspondance entre le nombre de morphèmes et le nombre de tons de morphèmes}}}}\kern2pt]} \hypertarget{k\string_hv\string_=\string_M-p\string_ho\string_T1}{}
\markboth{\textcolor{darkblue}{\textbf{\ipa{kʰv̩˧-pʰo˥}}}}{}
\textcolor{teal}{\mytextsc{nom}} \hspace{4pt} Ton~: H\#.
\textcolor{PineGreen}{\selectlanguage{french}Une demi-année.} \zh{半年。}  ¶ \textcolor{darkblue}{\textbf{\ipa{ɖɯ˧-kʰv̩˧-kʰv̩˥-pʰo˩}}} \textcolor{PineGreen}{\selectlanguage{french}un an et demi} \zh{一年半}  

\lhead{\firstmark}
\rhead{\botmark}

\subsection{\hspace{-0.5cm} {\Large \textcolor{darkblue}{\textbf{\ipa{kʰv̩˧pʰv̩\#˥}}}}\hspace{0.5cm}[\kern2pt{\textcolor{darkblue}{\textbf{\ipa{kʰv̩˧pʰv̩˥}}}}\kern2pt]} \hypertarget{k\string_hv\string_=\string_Mp\string_hv\string_=\#\string_T1}{}
\markboth{\textcolor{darkblue}{\textbf{\ipa{kʰv̩˧pʰv̩\#˥}}}}{}
\textcolor{teal}{\mytextsc{nom}} \hspace{4pt} Ton~: \#H.
\textcolor{PineGreen}{\selectlanguage{french}Chien mâle (forme élicitée).} \zh{公狗。}  ¶ \textcolor{darkblue}{\textbf{\ipa{kʰv̩˧pʰv̩˧ ʈʂʰɯ˧-ɭɯ\#˥}}} \textcolor{PineGreen}{\selectlanguage{french}\mytextsc{n}+\mytextsc{dem}+\mytextsc{clf}} \zh{这只公狗}  
 ¶ \textcolor{darkblue}{\textbf{\ipa{kʰv̩˧pʰv̩˧ tʰv̩˧-mi˧˥}}} \textcolor{PineGreen}{\selectlanguage{french}\mytextsc{n}+\mytextsc{dem}+\mytextsc{clf}} \zh{这只公狗}  
 ¶ \textcolor{darkblue}{\textbf{\ipa{kʰv̩˧pʰv̩˧ tʰv̩˧-v̩\#˥}}} \textcolor{PineGreen}{\selectlanguage{french}\mytextsc{n}+\mytextsc{dem}+\mytextsc{clf}} \zh{这个公狗}  
 \zh{量词}: \textcolor{darkblue}{\textbf{\ipa{v̩˧ / mi˩ / ɭɯ˧}}}  \mytextsc{clf}~: \textcolor{darkblue}{\textbf{\ipa{v̩˧ / mi˩ / ɭɯ˧}}} 
\lhead{\firstmark}
\rhead{\botmark}

\subsection{\hspace{-0.5cm} {\Large \textcolor{darkblue}{\textbf{\ipa{kʰv̩˧qʰwɤ˧˥}}}}\hspace{0.5cm}[\kern2pt{\textcolor{darkblue}{\textbf{\ipa{kʰv̩˧qʰwɤ˧}}}}\kern2pt]} \hypertarget{k\string_hv\string_=\string_Mq\string_hw7\string_M\string_T1}{}
\markboth{\textcolor{darkblue}{\textbf{\ipa{kʰv̩˧qʰwɤ˧˥}}}}{}
\textcolor{teal}{\mytextsc{nom}} \hspace{4pt} Ton~: MH\#.
\textcolor{PineGreen}{\selectlanguage{french}Mauvaise année, année de disette.} \zh{庄稼收成不好的(一)年。}  ¶ \textcolor{darkblue}{\textbf{\ipa{kʰv̩˧qʰwɤ˧ tʰv̩˧˥}}} \textcolor{PineGreen}{\selectlanguage{french}une mauvaise année a lieu, une année de mauvaise récolte/de disette} \zh{今年,收成不好。}  

\lhead{\firstmark}
\rhead{\botmark}

\subsection{\hspace{-0.5cm} {\Large \textcolor{darkblue}{\textbf{\ipa{kʰv̩˧ʂæ˧˥}}}}\hspace{0.5cm}[\kern2pt{\textcolor{darkblue}{\textbf{\ipa{kʰv̩˧ʂæ˥}}}}\kern2pt]} \hypertarget{k\string_hv\string_=\string_Ms`\{\string_M\string_T1}{}
\markboth{\textcolor{darkblue}{\textbf{\ipa{kʰv̩˧ʂæ˧˥}}}}{}
\textcolor{teal}{\mytextsc{verbe}} \hspace{4pt} Ton~: MH.
\textcolor{PineGreen}{\selectlanguage{french}Chasser; mener un chien de chasse.} \zh{打猎、赶走、驱逐。}  ¶ \textcolor{darkblue}{\textbf{\ipa{kʰv̩˧ʂæ˧ hɯ˧˥}}} \textcolor{PineGreen}{\selectlanguage{french}(Elle/il) est parti(e) chasser} \zh{狩猎去了}  

\lhead{\firstmark}
\rhead{\botmark}

\subsection{\hspace{-0.5cm} {\Large \textcolor{darkblue}{\textbf{\ipa{kʰv̩˧ʂɯ˥}}}}\hspace{0.5cm}[\kern2pt{\textcolor{darkblue}{\textbf{\ipa{kʰv̩˧ʂɯ˥}}}}\kern2pt]} \hypertarget{k\string_hv\string_=\string_Ms`M\string_T1}{}
\markboth{\textcolor{darkblue}{\textbf{\ipa{kʰv̩˧ʂɯ˥}}}}{}
\textcolor{teal}{\mytextsc{verbe}} \hspace{4pt} Ton~: .
\textcolor{PineGreen}{\selectlanguage{french}Fêter le Nouvel An.} \zh{过年。} 
\lhead{\firstmark}
\rhead{\botmark}

\subsection{\hspace{-0.5cm} {\Large \textcolor{darkblue}{\textbf{\ipa{kʰv̩˧sɯ˧sɯ˩}}}}\hspace{0.5cm}[\kern2pt{\textcolor{darkblue}{\textbf{\ipa{kʰv̩˧sɯ˧sɯ˩}}}}\kern2pt]} \hypertarget{k\string_hv\string_=\string_MsM\string_MsM\string_B1}{}
\markboth{\textcolor{darkblue}{\textbf{\ipa{kʰv̩˧sɯ˧sɯ˩}}}}{}
\textcolor{teal}{\mytextsc{nom}} \hspace{4pt} Ton~: L\#.
\textcolor{PineGreen}{\selectlanguage{french}Plante à fleurs, \textit{Flemingia strobilifera}, aussi appelée \textit{Moghania fruticulosa} (nom en chinois local: “oreille de souris”, du fait de la forme de la feuille).} \zh{球穗千斤拔、半灌木千斤拔、大苞千斤拔。} Dialecte chinois local~:\zh{耗子耳朵。} \zh{量词}: \textcolor{darkblue}{\textbf{\ipa{kɤ˧˥}}}  \mytextsc{clf}~: \textcolor{darkblue}{\textbf{\ipa{kɤ˧˥}}} 
\lhead{\firstmark}
\rhead{\botmark}

\subsection{\hspace{-0.5cm} {\Large \textcolor{darkblue}{\textbf{\ipa{kʰv̩˧tɕʰi˥\$}}}}\hspace{0.5cm}[\kern2pt{\textcolor{darkblue}{\textbf{\ipa{kʰv̩˧tɕʰi˥}}}}\kern2pt]} \hypertarget{k\string_hv\string_=\string_Mts£\string_hi\string_T\$1}{}
\markboth{\textcolor{darkblue}{\textbf{\ipa{kʰv̩˧tɕʰi˥\$}}}}{}
\textcolor{teal}{\mytextsc{nom}} \hspace{4pt} Ton~: H\$.
\textcolor{PineGreen}{\selectlanguage{french}Solution, méthode.} \zh{办法。}  ¶ \textcolor{darkblue}{\textbf{\ipa{kʰv̩˧tɕʰi˥ | mɤ˧-dʑo˧-ze˧! | ɻ̃˧-ɻ̍˧ tʰo˩!}}} \textcolor{PineGreen}{\selectlanguage{french}Il n'y a plus rien à faire! C'est la catastrophe!} \zh{没有办法了!糟糕了!}  

\lhead{\firstmark}
\rhead{\botmark}

\subsection{\hspace{-0.5cm} {\Large \textcolor{darkblue}{\textbf{\ipa{kʰv̩˧tsʰi˧-bo˥tsʰi˩}}}}\hspace{0.5cm}[\kern2pt{\textcolor{darkblue}{\textbf{\ipa{kʰv̩˧tsʰi˧bo˥tsʰi˩}}}}\kern2pt]} \hypertarget{k\string_hv\string_=\string_Mts\string_hi\string_M-bo\string_Tts\string_hi\string_B1}{}
\markboth{\textcolor{darkblue}{\textbf{\ipa{kʰv̩˧tsʰi˧-bo˥tsʰi˩}}}}{}
\textcolor{teal}{\mytextsc{nom}} \hspace{4pt} Ton~: \#H-.
\textcolor{PineGreen}{\selectlanguage{french}Taupe.} \zh{鼹鼠。}  \zh{量词}: \textcolor{darkblue}{\textbf{\ipa{pʰo˧˥}}} \textcolor{darkblue}{\textbf{\ipa{v̩˧}}}  \mytextsc{clf}~: \textcolor{darkblue}{\textbf{\ipa{pʰo˧˥}}} \textcolor{darkblue}{\textbf{\ipa{v̩˧}}} 
\lhead{\firstmark}
\rhead{\botmark}

\subsection{\hspace{-0.5cm} {\Large \textcolor{darkblue}{\textbf{\ipa{kʰv̩˩tsɤ˩mi˥}}}}\hspace{0.5cm}[\kern2pt{\textcolor{darkblue}{\textbf{\ipa{kʰv̩˩tsɤ˩mi˥}}}}\kern2pt]} \hypertarget{k\string_hv\string_=\string_Bts7\string_Bmi\string_T1}{}
\markboth{\textcolor{darkblue}{\textbf{\ipa{kʰv̩˩tsɤ˩mi˥}}}}{}
\textcolor{teal}{\mytextsc{nom}} \hspace{4pt} Ton~: L+H\#.
\textcolor{PineGreen}{\selectlanguage{french}Chienne.} \zh{母狗。}  \zh{量词}: \textcolor{darkblue}{\textbf{\ipa{v̩˧}}}  \mytextsc{clf}~: \textcolor{darkblue}{\textbf{\ipa{v̩˧}}} 
\lhead{\firstmark}
\rhead{\botmark}

\subsection{\hspace{-0.5cm} {\Large \textcolor{darkblue}{\textbf{\ipa{kʰv̩˧zo˥\$}}}}\hspace{0.5cm}[\kern2pt{\textcolor{darkblue}{\textbf{\ipa{kʰv̩˧zo˥}}}}\kern2pt]} \hypertarget{k\string_hv\string_=\string_Mzo\string_T\$1}{}
\markboth{\textcolor{darkblue}{\textbf{\ipa{kʰv̩˧zo˥\$}}}}{}
\textcolor{teal}{\mytextsc{nom}} \hspace{4pt} Ton~: H\$.
\textcolor{PineGreen}{\selectlanguage{french}Nom de clan/famille étendue. Deux familles portent ce nom à Yongning.} \zh{一个姓。这个姓,永宁有两家。}  ¶ \textcolor{darkblue}{\textbf{\ipa{kʰv̩˧zo˧=ɻ̍˥\$}}} \textcolor{PineGreen}{\selectlanguage{french}La famille \textcolor{darkblue}{\textbf{\ipa{/kʰv̩˧zo˥\$/}}}, les \textcolor{darkblue}{\textbf{\ipa{/kʰv̩˧zo˥\$/}}}} \zh{\textcolor{darkblue}{\textbf{\ipa{/kʰv̩˧zo˥\$/}}}家族}  
 ¶ \textcolor{darkblue}{\textbf{\ipa{kʰv̩˧zo˥-tsʰɯ˩ɻ̍˩}}} \textcolor{PineGreen}{\selectlanguage{french}nom d'une personne, comportant un nom de famille (\textcolor{darkblue}{\textbf{\ipa{/kʰv̩˧zo˥\$/}}}) et un prénom (\textcolor{darkblue}{\textbf{\ipa{/tsʰɯ˧ɻ\#˥/}}})} \zh{一个人的名字:姓为\textcolor{darkblue}{\textbf{\ipa{/kʰv̩˧zo˥\$/}}},名为\textcolor{darkblue}{\textbf{\ipa{/tsʰɯ˧ɻ\#˥/}}}}  

\lhead{\firstmark}
\rhead{\botmark}

\subsection{\hspace{-0.5cm} {\Large \textcolor{darkblue}{\textbf{\ipa{kʰv̩˧zo\#˥}}}}\hspace{0.5cm}[\kern2pt{\textcolor{darkblue}{\textbf{\ipa{kʰv̩˧zo˧}}}}\kern2pt]} \hypertarget{k\string_hv\string_=\string_Mzo\#\string_T1}{}
\markboth{\textcolor{darkblue}{\textbf{\ipa{kʰv̩˧zo\#˥}}}}{}
\textcolor{teal}{\mytextsc{nom}} \hspace{4pt} Ton~: \#H.
\textcolor{PineGreen}{\selectlanguage{french}Chien (mâle), chiot. Le terme est également employé comme nom provisoire pour les garçonnets pendant leurs premiers mois, avant qu'on ne leur donne un vrai nom. Le vilain nom dont on l'affuble vise à éviter que le nourrisson ne soit repéré par de mauvais esprits. (Actuellement, l'état-civil nécessite qu'un nom soit donné dès la naissance; mais celui-ci ne commence à être employé dans les conversations familiales qu'une fois passés les premiers mois.).} \zh{公狗(给刚出生的男孩子的名字,让鬼对他不感兴趣,不过来害小孩)。}  ¶ \textcolor{darkblue}{\textbf{\ipa{kʰv̩˧zo˧ ʈʂʰɯ˧-ɭɯ\#˥}}} \textcolor{PineGreen}{\selectlanguage{french}\mytextsc{n}+\mytextsc{dem}+\mytextsc{clf}} \zh{这只公狗}  
 ¶ \textcolor{darkblue}{\textbf{\ipa{kʰv̩˧zo˧ tʰv̩˧-mi˧˥}}} \textcolor{PineGreen}{\selectlanguage{french}\mytextsc{n}+\mytextsc{dem}+\mytextsc{clf}} \zh{这只公狗}  
 ¶ \textcolor{darkblue}{\textbf{\ipa{kʰv̩˧zo˧ tʰv̩˧-v̩\#˥}}} \textcolor{PineGreen}{\selectlanguage{french}\mytextsc{n}+\mytextsc{dem}+\mytextsc{clf}} \zh{这只公狗}  
 ¶ \textcolor{darkblue}{\textbf{\ipa{kʰv̩˧zo˥-kʰv̩˩mv̩˩}}} \textcolor{PineGreen}{\selectlanguage{french}chien et chienne} \zh{小狗与母狗}  
 \zh{量词}: \textcolor{darkblue}{\textbf{\ipa{v̩˧ / mi˩ / ɭɯ˧}}}  \mytextsc{clf}~: \textcolor{darkblue}{\textbf{\ipa{v̩˧ / mi˩ / ɭɯ˧}}} 
\lhead{\firstmark}
\rhead{\botmark}

\subsection{\hspace{-0.5cm} {\Large \textcolor{darkblue}{\textbf{\ipa{kʰwæ˧ɻæ\#˥}}}}\hspace{0.5cm}[\kern2pt{\textcolor{darkblue}{\textbf{\ipa{kʰwæ˧ɻæ˧}}}}\kern2pt]} \hypertarget{k\string_hw\{\string_Mr£`\{\#\string_T1}{}
\markboth{\textcolor{darkblue}{\textbf{\ipa{kʰwæ˧ɻæ\#˥}}}}{}
\textcolor{teal}{\mytextsc{nom}} \hspace{4pt} Ton~: \#H.
\textcolor{PineGreen}{\selectlanguage{french}Feutre; par extension: natte, tapis (même en vannerie), coussin….} \zh{毡子。也用来指席子,垫子等。}  ¶ \textcolor{darkblue}{\textbf{\ipa{kʰwæ˧ɻæ˧ tʰi˧-kʰo˥}}} \textcolor{PineGreen}{\selectlanguage{french}étendre la natte} \zh{铺席子}  
 \zh{量词}: \textcolor{darkblue}{\textbf{\ipa{tsʰi˥}}}  \mytextsc{clf}~: \textcolor{darkblue}{\textbf{\ipa{tsʰi˥}}} 
\lhead{\firstmark}
\rhead{\botmark}

\subsection{\hspace{-0.5cm} {\Large \textcolor{darkblue}{\textbf{\ipa{kʰwɤ˥\textsubscript{a}}}}}\hspace{0.5cm}[\kern2pt{\textcolor{darkblue}{\textbf{\ipa{kʰwɤ˥}}}}\kern2pt]} \hypertarget{k\string_hw7\string_Ta1}{}
\markboth{\textcolor{darkblue}{\textbf{\ipa{kʰwɤ˥\textsubscript{a}}}}}{}
\textcolor{teal}{\mytextsc{classificateur}} \hspace{4pt} Ton~: H\textsubscript{a}.
\textcolor{PineGreen}{\selectlanguage{french}Classificateur des morceaux/bouchées.} \zh{量词:块。一块肉、一口饭。}  ¶ \textcolor{darkblue}{\textbf{\ipa{ɖɯ˧-kʰwɤ˥\textasciitilde{}ɖɯ˩-kʰwɤ˩}}} \textcolor{PineGreen}{\selectlanguage{french}par petites bouchées, par petits morceaux} \zh{一块一块地}  
 ¶ \textcolor{darkblue}{\textbf{\ipa{kʰwɤ˧ | ɖɯ˧-ʂe˧-ɻ̍˩!}}} \textcolor{PineGreen}{\selectlanguage{french}Décidez! / Il faut vous décider!} \zh{你们得要做出决定!}  
 ¶ \textcolor{darkblue}{\textbf{\ipa{ɖɯ˧-kʰwɤ˧ so˧˥, | ɖɯ˧-kʰwɤ˥ fv̩˩!}}} \textcolor{PineGreen}{\selectlanguage{french}Chaque mot appris représente une joie de plus! (Commentaire de la locutrice au sujet du travail de l'enquêteur. Tenant en main le manuscrit de ce dictionnaire, elle commente: cela représente un travail immense; l'important est que l'enquêteur y trouve de l'intérêt: chaque information nouvelle – chaque “morceau” de langue – ajoutée au dictionnaire est une joie pour l'enquêteur.)} \zh{学一点,就高兴一点!(评说语言调查工作:合作人看着本词典的初稿,说:这是一项很大的工程,关键的是调查者要有兴趣,欣赏每个新学的语言信息。)}  

\lhead{\firstmark}
\rhead{\botmark}

\subsection{\hspace{-0.5cm} {\Large \textcolor{darkblue}{\textbf{\ipa{kʰwɤ˧pʰv̩˧}}}}\hspace{0.5cm}[\kern2pt{\textcolor{darkblue}{\textbf{\ipa{kʰwɤ˧pʰv̩˧}}}}\kern2pt]} \hypertarget{k\string_hw7\string_Mp\string_hv\string_=\string_M1}{}
\markboth{\textcolor{darkblue}{\textbf{\ipa{kʰwɤ˧pʰv̩˧}}}}{}
\textcolor{teal}{\mytextsc{nom}} \hspace{4pt} Ton~: M.
\textcolor{PineGreen}{\selectlanguage{french}Pré: soit prairie de plaine, soit prairie d'altitude (alpage).} \zh{草坪、草地。} 
\lhead{\firstmark}
\rhead{\botmark}

\subsection{\hspace{-0.5cm} {\Large \textcolor{darkblue}{\textbf{\ipa{kʰwɤ˧pʰv̩˧-mo˧˥}}}}\hspace{0.5cm}[\kern2pt{\textcolor{darkblue}{\textbf{\ipa{xxxx non-correspondance entre le nombre de morphèmes et le nombre de tons de morphèmes}}}}\kern2pt]} \hypertarget{k\string_hw7\string_Mp\string_hv\string_=\string_M-mo\string_M\string_T1}{}
\markboth{\textcolor{darkblue}{\textbf{\ipa{kʰwɤ˧pʰv̩˧-mo˧˥}}}}{}
\textcolor{teal}{\mytextsc{nom}} \hspace{4pt} Ton~: MH\#.
\textcolor{PineGreen}{\selectlanguage{french}Champignon des prés: une sorte de champignon comestible (pas encore identifiée): agaric champêtre ou rosé des prés, \textit{Agaricus campestris}?} \zh{可以吃的一种菌子:可能是四孢蘑菇。直译:“草坪菌”。} 
\lhead{\firstmark}
\rhead{\botmark}

\newpage
\section*{\centering- \textcolor{darkblue}{\textbf{\ipa{l}}} \textcolor{darkblue}{\textbf{\ipa{ɭ}}} -}
\subsection{\hspace{-0.5cm} {\Large \textcolor{darkblue}{\textbf{\ipa{‑lɑ˧}}} \textsubscript{1}}\hspace{0.5cm}[\kern2pt{\textcolor{darkblue}{\textbf{\ipa{xxxx groupe tonal entier sans aucun ton}}}}\kern2pt]} \hypertarget{‑lA\string_M1}{}
\markboth{\textcolor{darkblue}{\textbf{\ipa{‑lɑ˧}}} \textsubscript{1}}{}
\textcolor{teal}{\mytextsc{adverbe}} \hspace{4pt} Ton~: 0.
\textcolor{PineGreen}{\selectlanguage{french}Seulement.} \zh{只,才。}  ¶ \textcolor{darkblue}{\textbf{\ipa{ʈʂʰɯ˧-lɑ˩ ɲi˩-ze˩-mæ˩!}}} \textcolor{PineGreen}{\selectlanguage{french}C'est tout ! / Voilà tout !} \zh{就这些了! / 就这些而已! / 就这样!}  

\lhead{\firstmark}
\rhead{\botmark}

\subsection{\hspace{-0.5cm} {\Large \textcolor{darkblue}{\textbf{\ipa{‑lɑ˧}}} \textsubscript{2}}\hspace{0.5cm}[\kern2pt{\textcolor{darkblue}{\textbf{\ipa{xxxx groupe tonal entier sans aucun ton}}}}\kern2pt]} \hypertarget{‑lA\string_M2}{}
\markboth{\textcolor{darkblue}{\textbf{\ipa{‑lɑ˧}}} \textsubscript{2}}{}
\textcolor{teal}{\mytextsc{adverbe}} \hspace{4pt} Ton~: 0.
\textcolor{PineGreen}{\selectlanguage{french}Et, aussi.} \zh{和、与、跟。}  ¶ \textcolor{darkblue}{\textbf{\ipa{ɖɯ˧-kʰv̩˧-lɑ˥ | so˩-ɬi˩˥}}} \textcolor{PineGreen}{\selectlanguage{french}un an et trois mois (contexte: on indique l'âge d'un petit enfant)} \zh{一岁三个月}  
 ¶ \textcolor{darkblue}{\textbf{\ipa{ʈʂʰɯ˧-lɑ˧ | mɤ˧-bi˧, | njɤ˧-lɑ˧ mɤ˧-bi˧!}}} \textcolor{PineGreen}{\selectlanguage{french}s'il n'y va pas, moi non plus!} \zh{他不去(的话),我也不去!}  
 ¶ \textcolor{darkblue}{\textbf{\ipa{hĩ˧-lɑ˩ | dʑɤ˧˥, | mv̩˧di˧-lɑ˥ | dʑɤ˧˥! / hĩ˧-lɑ˩ | dʑɤ˧˥, | lv̩˧-lɑ˧ | dʑɤ˧˥!}}} \textcolor{PineGreen}{\selectlanguage{french}les gens (y) sont bons, (et) la terre (y) est bonne (formule de recommandation pour la famille que va rejoindre une jeune femme lors de son mariage)} \zh{人也好,田也好!(习语:将女孩嫁出去前,一家人打听对方家如何,推荐的人保证:“他们家,人也好,田也好!”)}  
 ¶ \textcolor{darkblue}{\textbf{\ipa{hĩ˧ F | dʑɤ˧˥, | mv̩˧di˧˥ F | dʑɤ˧˥!}}} \textcolor{PineGreen}{\selectlanguage{french}même sens} \zh{同上}  
 ¶ \textcolor{darkblue}{\textbf{\ipa{mɤ˧-lɑ˧ dʑɤ˧˥!}}} \textcolor{PineGreen}{\selectlanguage{french}la graisse aussi (y) est bonne! (variation élicitée à partir des exemples qui précèdent)} \zh{猪油也好!(按照上面例子的变体)}  
 ¶ \textcolor{darkblue}{\textbf{\ipa{qæ˩-lɑ˥ | dʑɤ˧˥!}}} \textcolor{PineGreen}{\selectlanguage{french}l’huile aussi (y) est bonne! (variation à partir de l'exemple qui précède)} \zh{油也好!}  
 ¶ \textcolor{darkblue}{\textbf{\ipa{ʈʂʰɯ˧-lɑ˧ | mɤ˧-bi˧, | njɤ˧ | mɤ˧-bi˧-ze˧! / ʈʰɯ˧ mɤ˧-bi˧-ze˧-dʑo˧, | njɤ˧-lɑ˧ | mɤ˧-bi˧-ze˧!}}} \textcolor{PineGreen}{\selectlanguage{french}s'il n'y va pas, moi non plus!} \zh{他如果不去,我也不去!}  

\lhead{\firstmark}
\rhead{\botmark}

\subsection{\hspace{-0.5cm} {\Large \textcolor{darkblue}{\textbf{\ipa{lɑ˧}}}}\hspace{0.5cm}[\kern2pt{\textcolor{darkblue}{\textbf{\ipa{lɑ˥}}}}\kern2pt]} \hypertarget{lA\string_M1}{}
\markboth{\textcolor{darkblue}{\textbf{\ipa{lɑ˧}}}}{}
\textcolor{teal}{\mytextsc{nom}} \hspace{4pt} Ton~: M.
\textcolor{PineGreen}{\selectlanguage{french}Tigre.} \zh{老虎。}  \zh{量词}: \textcolor{darkblue}{\textbf{\ipa{pʰo˧˥}}}  \mytextsc{clf}~: \textcolor{darkblue}{\textbf{\ipa{pʰo˧˥}}} 
\lhead{\firstmark}
\rhead{\botmark}

\subsection{\hspace{-0.5cm} {\Large \textcolor{darkblue}{\textbf{\ipa{lɑ˧bi\#˥}}}}\hspace{0.5cm}[\kern2pt{\textcolor{darkblue}{\textbf{\ipa{lɑ˧bi˧}}}}\kern2pt]} \hypertarget{lA\string_Mbi\#\string_T1}{}
\markboth{\textcolor{darkblue}{\textbf{\ipa{lɑ˧bi\#˥}}}}{}
\textcolor{teal}{\mytextsc{nom}} \hspace{4pt} Ton~: \#H.
\textcolor{PineGreen}{\selectlanguage{french}Escarpement, pente raide, terrain escarpé.} \zh{陡坡、土坡、斜坡。}  ¶ \textcolor{darkblue}{\textbf{\ipa{lɑ˧bi˧-tsɤ˧}}} \textcolor{PineGreen}{\selectlanguage{french}raide, escarpé (littéralement 'comme un escarpement')} \zh{‘像陡坡’,等于:很陡}  
 ¶ \textcolor{darkblue}{\textbf{\ipa{lɑ˧bi˧-tsɤ˧ | ʐwæ˩˥!}}} \textcolor{PineGreen}{\selectlanguage{french}C'est très pentu! (Littéralement: 'Ca ressemble vraiment à une pente raide!')} \zh{陡得很!}  

\lhead{\firstmark}
\rhead{\botmark}

\subsection{\hspace{-0.5cm} {\Large \textcolor{darkblue}{\textbf{\ipa{lɑ˧do\#˥}}}}\hspace{0.5cm}[\kern2pt{\textcolor{darkblue}{\textbf{\ipa{lɑ˧do˧}}}}\kern2pt]} \hypertarget{lA\string_Mdo\#\string_T1}{}
\markboth{\textcolor{darkblue}{\textbf{\ipa{lɑ˧do\#˥}}}}{}
\textcolor{teal}{\mytextsc{nom}} \hspace{4pt} Ton~: \#H.
\textcolor{PineGreen}{\selectlanguage{french}Palefrenier, caravanier (employé, pas chef de caravane).} \zh{马夫(参加马帮)。} 
\lhead{\firstmark}
\rhead{\botmark}

\subsection{\hspace{-0.5cm} {\Large \textcolor{darkblue}{\textbf{\ipa{lɑ˧hwɤ˩}}}}\hspace{0.5cm}[\kern2pt{\textcolor{darkblue}{\textbf{\ipa{lɑ˧hwɤ˩}}}}\kern2pt]} \hypertarget{lA\string_Mhw7\string_B1}{}
\markboth{\textcolor{darkblue}{\textbf{\ipa{lɑ˧hwɤ˩}}}}{}
\textcolor{teal}{\mytextsc{nom}} \hspace{4pt} Ton~: L\#.
\textcolor{PineGreen}{\selectlanguage{french}Village na hors de la plaine de Yongning, vers le Lac, non loin de \textcolor{darkblue}{\textbf{\ipa{/lɑ˧tʰɑ˧-di˧˥/}}}.} \zh{村落名。} 
\lhead{\firstmark}
\rhead{\botmark}

\subsection{\hspace{-0.5cm} {\Large \textcolor{darkblue}{\textbf{\ipa{lɑ˧kɤ˩}}}}\hspace{0.5cm}[\kern2pt{\textcolor{darkblue}{\textbf{\ipa{lɑ˧kɤ˩}}}}\kern2pt]} \hypertarget{lA\string_Mk7\string_B1}{}
\markboth{\textcolor{darkblue}{\textbf{\ipa{lɑ˧kɤ˩}}}}{}
\textcolor{teal}{\mytextsc{nom}} \hspace{4pt} Ton~: L\#.
\textcolor{PineGreen}{\selectlanguage{french}Petite cruche, petit pot pour l'alcool; sert pour le conserver longtemps, pas seulement pour le verser.} \zh{小坛子,用来存酒。}  \zh{量词}: \textcolor{darkblue}{\textbf{\ipa{ɭɯ˧}}}  \mytextsc{clf}~: \textcolor{darkblue}{\textbf{\ipa{ɭɯ˧}}} 
\lhead{\firstmark}
\rhead{\botmark}

\subsection{\hspace{-0.5cm} {\Large \textcolor{darkblue}{\textbf{\ipa{lɑ˧kʰv̩˧˥}}}}\hspace{0.5cm}[\kern2pt{\textcolor{darkblue}{\textbf{\ipa{lɑ˧kʰv̩˧˥}}}}\kern2pt]} \hypertarget{lA\string_Mk\string_hv\string_=\string_M\string_T1}{}
\markboth{\textcolor{darkblue}{\textbf{\ipa{lɑ˧kʰv̩˧˥}}}}{}
\textcolor{teal}{\mytextsc{nom}} \hspace{4pt} Ton~: MH\#.
\textcolor{PineGreen}{\selectlanguage{french}Année du Tigre.} \zh{虎年。} 
\lhead{\firstmark}
\rhead{\botmark}

\subsection{\hspace{-0.5cm} {\Large \textcolor{darkblue}{\textbf{\ipa{lɑ˧\textasciitilde{}lɑ˧}}}}\hspace{0.5cm}[\kern2pt{\textcolor{darkblue}{\textbf{\ipa{lɑ˧lɑ˧}}}}\kern2pt]} \hypertarget{lA\string_M~lA\string_M1}{}
\markboth{\textcolor{darkblue}{\textbf{\ipa{lɑ˧\textasciitilde{}lɑ˧}}}}{}
\textcolor{teal}{\mytextsc{adjectif}} \hspace{4pt} Ton~: M.
\textcolor{PineGreen}{\selectlanguage{french}Ballant, flasque.} \zh{松弛。} 
\lhead{\firstmark}
\rhead{\botmark}

\subsection{\hspace{-0.5cm} {\Large \textcolor{darkblue}{\textbf{\ipa{lɑ˧\textasciitilde{}lɑ˧\textsubscript{b}}}}}\hspace{0.5cm}[\kern2pt{\textcolor{darkblue}{\textbf{\ipa{lɑ˧lɑ˧}}}}\kern2pt]} \hypertarget{lA\string_M~lA\string_Mb1}{}
\markboth{\textcolor{darkblue}{\textbf{\ipa{lɑ˧\textasciitilde{}lɑ˧\textsubscript{b}}}}}{}
\textcolor{teal}{\mytextsc{verbe}} \hspace{4pt} Ton~: M\textsubscript{b}.
\textcolor{PineGreen}{\selectlanguage{french}Diluer (dans l’eau).} \zh{掺水。}  ¶ \textcolor{darkblue}{\textbf{\ipa{(dʑɯ˧-qo˧) le˧-lɑ˧\textasciitilde{}lɑ˧}}} \textcolor{PineGreen}{\selectlanguage{french}diluer dans de l’eau} \zh{掺水}  

\lhead{\firstmark}
\rhead{\botmark}

\subsection{\hspace{-0.5cm} {\Large \textcolor{darkblue}{\textbf{\ipa{lɑ˧lo˧-ʁwɤ˥}}}}\hspace{0.5cm}[\kern2pt{\textcolor{darkblue}{\textbf{\ipa{xxxx non-correspondance entre le nombre de morphèmes et le nombre de tons de morphèmes}}}}\kern2pt]} \hypertarget{lA\string_Mlo\string_M-Rw7\string_T1}{}
\markboth{\textcolor{darkblue}{\textbf{\ipa{lɑ˧lo˧-ʁwɤ˥}}}}{}
\textcolor{teal}{\mytextsc{nom}} \hspace{4pt} Ton~: H\#.
\textcolor{PineGreen}{\selectlanguage{french}Un village de Yongning; prononciation chinoise: Laluowa.} \zh{拉洛瓦村(永宁的一个村落)。}  ¶ \textcolor{darkblue}{\textbf{\ipa{dʑɤ˩bv̩˧kɤ˧-sɑ˥ʁwɤ˩, | hi˩ʁwɤ˩-lo˥, | æ˩mi˧-ʁwɤ\#˥, | lɑ˧lo˧-ʁwɤ˥, | lɑ˧ŋwɤ˧, | bɤ˧tsʰo˧gv̩˥, | ə˧lɑ˧-ʁwɤ\#˥, | gæ˧ɻæ˩, | qʰæ˧tɕʰi˧, | tʰo˧ʈɯ\#˥}}} \textcolor{PineGreen}{\selectlanguage{french}les dix villages comptant traditionnellement comme faisant partie de Yongning} \zh{摩梭传统地理概念中,属于永宁的十个村落}  

\lhead{\firstmark}
\rhead{\botmark}

\subsection{\hspace{-0.5cm} {\Large \textcolor{darkblue}{\textbf{\ipa{lɑ˧ɬɑ˧˥}}}}\hspace{0.5cm}[\kern2pt{\textcolor{darkblue}{\textbf{\ipa{lɑ˧ɬɑ˧˥}}}}\kern2pt]} \hypertarget{lA\string_MKA\string_M\string_T1}{}
\markboth{\textcolor{darkblue}{\textbf{\ipa{lɑ˧ɬɑ˧˥}}}}{}
\textcolor{teal}{\mytextsc{conjonction}} \hspace{4pt} Ton~: MH\#.
\textcolor{PineGreen}{\selectlanguage{french}À part, en dehors de.} \zh{这以外。}  ¶ \textcolor{darkblue}{\textbf{\ipa{tsɑ˧bɤ˧ mɤ˧-pʰv̩˧ɖɯ˧! | lɑ˧ɬɑ˧˥, | ə˧tso˧-mɤ˧-ɲi˩ | pʰv̩˩ɖɯ˩˥!}}} \textcolor{PineGreen}{\selectlanguage{french}La farine n'est pas chère; à part ça, tout est cher! (Réflexion au sujet du coût de la vie dans la région aujourd'hui)} \zh{面粉不贵。其它的呢,什么都贵!(题目:讲今日永宁食品物价)}  

\lhead{\firstmark}
\rhead{\botmark}

\subsection{\hspace{-0.5cm} {\Large \textcolor{darkblue}{\textbf{\ipa{lɑ˧ɬɑ˧˥}}} \textsubscript{1}}\hspace{0.5cm}[\kern2pt{\textcolor{darkblue}{\textbf{\ipa{lɑ˧ɬɑ˧˥}}}}\kern2pt]} \hypertarget{lA\string_MKA\string_M\string_T1}{}
\markboth{\textcolor{darkblue}{\textbf{\ipa{lɑ˧ɬɑ˧˥}}} \textsubscript{1}}{}
\textcolor{teal}{\mytextsc{pronom}} \hspace{4pt} Ton~: MH\#.
\textcolor{PineGreen}{\selectlanguage{french}Autre, autres.} \zh{别的。}  ¶ \textcolor{darkblue}{\textbf{\ipa{lɑ˧ɬɑ˧˥ | ɖɯ˧-tɕi˥}}} \textcolor{PineGreen}{\selectlanguage{french}quelques autres} \zh{其它一些}  
 ¶ \textcolor{darkblue}{\textbf{\ipa{lɑ˧ɬɑ˧˥ | ʈʂʰɯ˧-tɕi˩}}} \textcolor{PineGreen}{\selectlanguage{french}ces quelques autres, ceux qui restent} \zh{其它的那些}  
 ¶ \textcolor{darkblue}{\textbf{\ipa{lɑ˧ɬɑ˧˥ | ɖɯ˧-ʁo˩ ɲi˩!}}} \textcolor{PineGreen}{\selectlanguage{french}C'est autre chose! / Ca, c'est différent!} \zh{是另一回事! / 是另一码事!}  
\textit{Voir~:} \hyperlink{}{\textcolor{darkblue}{\textbf{\ipa{lɑ˧ɬɑ˧˥}}} \textsubscript{2}} 
\lhead{\firstmark}
\rhead{\botmark}

\subsection{\hspace{-0.5cm} {\Large \textcolor{darkblue}{\textbf{\ipa{lɑ˧ɬɑ˧˥}}} \textsubscript{2}}\hspace{0.5cm}[\kern2pt{\textcolor{darkblue}{\textbf{\ipa{lɑ˧ɬɑ˧˥}}}}\kern2pt]} \hypertarget{lA\string_MKA\string_M\string_T2}{}
\markboth{\textcolor{darkblue}{\textbf{\ipa{lɑ˧ɬɑ˧˥}}} \textsubscript{2}}{}
\textcolor{teal}{\mytextsc{adjectif}} \hspace{4pt} Ton~: MH\#.
\textcolor{PineGreen}{\selectlanguage{french}Autre.} \zh{别的。}  ¶ \textcolor{darkblue}{\textbf{\ipa{lɑ˧ɬɑ˧ hĩ˥}}} \textcolor{PineGreen}{\selectlanguage{french}les autres gens} \zh{其它人}  
 ¶ \textcolor{darkblue}{\textbf{\ipa{ɖɯ˧-bæ˧ | le˧-se˩, | ɖɯ˧-bæ˧ ʝi˧! / ɖɯ˧-bæ˧ | le˧-se˩, | wɤ˩˥ | lɑ˧ɬɑ˧˥ | ɖɯ˧-bæ˧ ʝi˧! |}}} \textcolor{PineGreen}{\selectlanguage{french}Quand on a fini une chose/une tâche, on en fait une autre / on passe à une autre!} \zh{做完一件事情,就轮到另一个!}  
\textit{Voir~:} \hyperlink{}{\textcolor{darkblue}{\textbf{\ipa{lɑ˧ɬɑ˧˥}}} \textsubscript{1}} 
\lhead{\firstmark}
\rhead{\botmark}

\subsection{\hspace{-0.5cm} {\Large \textcolor{darkblue}{\textbf{\ipa{lɑ˧mɑ˧}}}}\hspace{0.5cm}[\kern2pt{\textcolor{darkblue}{\textbf{\ipa{lɑ˧mɑ˧}}}}\kern2pt]} \hypertarget{lA\string_MmA\string_M1}{}
\markboth{\textcolor{darkblue}{\textbf{\ipa{lɑ˧mɑ˧}}}}{}
\textcolor{teal}{\mytextsc{nom}} \hspace{4pt} Ton~: M.
\textcolor{PineGreen}{\selectlanguage{french}Lama.} \zh{喇嘛。}  Emprunt~: tibétain  bla-ma
 ¶ \textcolor{darkblue}{\textbf{\ipa{hæ˧-lɑ˩mɑ˩}}} \textcolor{PineGreen}{\selectlanguage{french}lama chinois} \zh{汉族喇嘛}  
 \zh{量词}: \textcolor{darkblue}{\textbf{\ipa{v̩˧}}}  \mytextsc{clf}~: \textcolor{darkblue}{\textbf{\ipa{v̩˧}}} 
\lhead{\firstmark}
\rhead{\botmark}

\subsection{\hspace{-0.5cm} {\Large \textcolor{darkblue}{\textbf{\ipa{lɑ˧mi\#˥}}}}\hspace{0.5cm}[\kern2pt{\textcolor{darkblue}{\textbf{\ipa{lɑ˧mi˧}}}}\kern2pt]} \hypertarget{lA\string_Mmi\#\string_T1}{}
\markboth{\textcolor{darkblue}{\textbf{\ipa{lɑ˧mi\#˥}}}}{}
\textcolor{teal}{\mytextsc{nom}} \hspace{4pt} Ton~: \#H.
\textcolor{PineGreen}{\selectlanguage{french}Tigresse.} \zh{母老虎。}  ¶ \textcolor{darkblue}{\textbf{\ipa{lɑ˧mi˧ tʰv̩˧-mi˧˥ / lɑ˧mi˧ tʰv̩˧-mi˥\#}}} \textcolor{PineGreen}{\selectlanguage{french}\mytextsc{n}+\mytextsc{dem}+\mytextsc{clf}} \zh{那只老虎}  
 \zh{量词}: \textcolor{darkblue}{\textbf{\ipa{pʰo˧˥ / mi˩}}}  \mytextsc{clf}~: \textcolor{darkblue}{\textbf{\ipa{pʰo˧˥ / mi˩}}} 
\lhead{\firstmark}
\rhead{\botmark}

\subsection{\hspace{-0.5cm} {\Large \textcolor{darkblue}{\textbf{\ipa{lɑ˧ŋwɤ˧}}}}\hspace{0.5cm}[\kern2pt{\textcolor{darkblue}{\textbf{\ipa{lɑ˧ŋwɤ˧}}}}\kern2pt]} \hypertarget{lA\string_MNw7\string_M1}{}
\markboth{\textcolor{darkblue}{\textbf{\ipa{lɑ˧ŋwɤ˧}}}}{}
\textcolor{teal}{\mytextsc{nom}} \hspace{4pt} Ton~: M.
\textcolor{PineGreen}{\selectlanguage{french}Nom de montagne, sur le chemin de Yongning à Wujiao; est aussi le nom qui désigne les hameaux qui se trouvent sur cette montagne.} \zh{一座山的名字。}  ¶ \textcolor{darkblue}{\textbf{\ipa{dʑɤ˩bv̩˧kɤ˧-sɑ˥ʁwɤ˩, | hi˩ʁwɤ˩-lo˥, | æ˩mi˧-ʁwɤ\#˥, | lɑ˧lo˧-ʁwɤ˥, | lɑ˧ŋwɤ˧, | bɤ˧tsʰo˧gv̩˥, | ə˧lɑ˧-ʁwɤ\#˥, | gæ˧ɻæ˩, | qʰæ˧tɕʰi˧, | tʰo˧ʈɯ\#˥}}} \textcolor{PineGreen}{\selectlanguage{french}les dix villages comptant traditionnellement comme faisant partie de Yongning} \zh{摩梭传统地理概念中,属于永宁的十个村落}  

\lhead{\firstmark}
\rhead{\botmark}

\subsection{\hspace{-0.5cm} {\Large \textcolor{darkblue}{\textbf{\ipa{lɑ˧pʰv̩\#˥}}}}\hspace{0.5cm}[\kern2pt{\textcolor{darkblue}{\textbf{\ipa{lɑ˧pʰv̩˧}}}}\kern2pt]} \hypertarget{lA\string_Mp\string_hv\string_=\#\string_T1}{}
\markboth{\textcolor{darkblue}{\textbf{\ipa{lɑ˧pʰv̩\#˥}}}}{}
\textcolor{teal}{\mytextsc{nom}} \hspace{4pt} Ton~: \#H.
\textcolor{PineGreen}{\selectlanguage{french}Tigre (mâle).} \zh{公老虎。}  ¶ \textcolor{darkblue}{\textbf{\ipa{lɑ˧pʰv̩˧ tʰv̩˧-ɭɯ\#˥}}} \textcolor{PineGreen}{\selectlanguage{french}\mytextsc{n}+\mytextsc{dem}+\mytextsc{clf}} \zh{那只老虎}  
 \zh{量词}: \textcolor{darkblue}{\textbf{\ipa{pʰo˧˥ / ɭɯ˧}}}  \mytextsc{clf}~: \textcolor{darkblue}{\textbf{\ipa{pʰo˧˥ / ɭɯ˧}}} 
\lhead{\firstmark}
\rhead{\botmark}

\subsection{\hspace{-0.5cm} {\Large \textcolor{darkblue}{\textbf{\ipa{lɑ˧tʰɑ˧-di˧˥}}}}\hspace{0.5cm}[\kern2pt{\textcolor{darkblue}{\textbf{\ipa{xxxx non-correspondance entre le nombre de morphèmes et le nombre de tons de morphèmes}}}}\kern2pt]} \hypertarget{lA\string_Mt\string_hA\string_M-di\string_M\string_T1}{}
\markboth{\textcolor{darkblue}{\textbf{\ipa{lɑ˧tʰɑ˧-di˧˥}}}}{}
\textcolor{teal}{\mytextsc{nom}} \hspace{4pt} Ton~: MH\#.
\textcolor{PineGreen}{\selectlanguage{french}La région na qui entoure le lac Lugu: Zuosuo (actuel Luguhu Zhen), le village de Luoshui, et les autres localités du bord du Lac.} \zh{泸沽湖周边的摩梭地区,包括左所(今为泸沽湖镇)、洛水村等。}  ¶ \textcolor{darkblue}{\textbf{\ipa{ɬi˧ki˧, | ɲi˧se˩, | tɑ˧dzi˩, | mv̩˧qʰwæ˩, | lɑ˧tʰɑ˧-di˧˥}}} \textcolor{PineGreen}{\selectlanguage{french}Villages dans l'ordre, après la plaine de Yongning, ne comptant pas comme faisant partie de Yongning. Le dernier, \textcolor{darkblue}{\textbf{\ipa{/lɑ˧tʰɑ˧-di˧˥/}}}, désigne toute la région na au-delà du quatrième village.} \zh{从永宁往泸沽湖所经过的村落,依次是:里格、尼赛、大祖、木垮,然后到拉塔地(拉塔地指的是泸沽湖周边的摩梭地区,包括左所、洛水村等)}  

\lhead{\firstmark}
\rhead{\botmark}

\subsection{\hspace{-0.5cm} {\Large \textcolor{darkblue}{\textbf{\ipa{lɑ˧tʰɑ˧mi˥\$}}}}\hspace{0.5cm}[\kern2pt{\textcolor{darkblue}{\textbf{\ipa{lɑ˧tʰɑ˧mi˥}}}}\kern2pt]} \hypertarget{lA\string_Mt\string_hA\string_Mmi\string_T\$1}{}
\markboth{\textcolor{darkblue}{\textbf{\ipa{lɑ˧tʰɑ˧mi˥\$}}}}{}
\textcolor{teal}{\mytextsc{nom}} \hspace{4pt} Ton~: H\$.
\textcolor{PineGreen}{\selectlanguage{french}Nom de clan/famille étendue. Cinq familles portent ce nom à Yongning. C'est l'un des trois premiers clans à s'être établis à proximité du monastère de Yongning, les deux autres étant \textcolor{darkblue}{\textbf{\ipa{/kɤ˧˥tʰɑ˩/}}} et \textcolor{darkblue}{\textbf{\ipa{/ə˧lɑ˧/}}}.} \zh{一个姓。这个姓,永宁有五个家。音译:拉他咪。}  ¶ \textcolor{darkblue}{\textbf{\ipa{lɑ˧tʰɑ˧mi˧=ɻ̍˥\$}}} \textcolor{PineGreen}{\selectlanguage{french}le clan \textcolor{darkblue}{\textbf{\ipa{/lɑ˧tʰɑ˧mi˥\$/}}}, la famille \textcolor{darkblue}{\textbf{\ipa{/lɑ˧tʰɑ˧mi˥\$/}}}} \zh{\textcolor{darkblue}{\textbf{\ipa{/lɑ˧tʰɑ˧mi˥\$/}}}家族}  

\lhead{\firstmark}
\rhead{\botmark}

\subsection{\hspace{-0.5cm} {\Large \textcolor{darkblue}{\textbf{\ipa{lɑ˧tʰɑ˧mi˥-ʈæ˧ʂɯ˧-lɑ˩mv̩˩}}}}\hspace{0.5cm}[\kern2pt{\textcolor{darkblue}{\textbf{\ipa{lɑ˧tʰɑ˧mi˥ʈæ˩ʂɯ˩lɑ˩mv̩˩}}}}\kern2pt]} \hypertarget{lA\string_Mt\string_hA\string_Mmi\string_T-t`\{\string_Ms`M\string_M-lA\string_Bmv\string_=\string_B1}{}
\markboth{\textcolor{darkblue}{\textbf{\ipa{lɑ˧tʰɑ˧mi˥-ʈæ˧ʂɯ˧-lɑ˩mv̩˩}}}}{}
\textcolor{teal}{\mytextsc{nom}} \hspace{4pt} Ton~: H\#-M-L.
\textcolor{PineGreen}{\selectlanguage{french}Nom propre (nom de famille et prénom) de la consultante de référence du présent travail (code locutrice: F4).} \zh{拉他咪•达石拉么:本著作的标准发音合作人。} 
\lhead{\firstmark}
\rhead{\botmark}

\subsection{\hspace{-0.5cm} {\Large \textcolor{darkblue}{\textbf{\ipa{lɑ˧zi˥}}}}\hspace{0.5cm}[\kern2pt{\textcolor{darkblue}{\textbf{\ipa{lɑ˧zi˥}}}}\kern2pt]} \hypertarget{lA\string_Mzi\string_T1}{}
\markboth{\textcolor{darkblue}{\textbf{\ipa{lɑ˧zi˥}}}}{}
\textcolor{teal}{\mytextsc{nom}} \hspace{4pt} Ton~: H\#.
\textcolor{PineGreen}{\selectlanguage{french}Peintre (activité qui n'est pas réservée aux moines).} \zh{画家。}  ¶ \textcolor{darkblue}{\textbf{\ipa{ʈʂʰɯ˧-v̩˧, | lɑ˧zi˥ ɲi˩!}}} \textcolor{PineGreen}{\selectlanguage{french}elle/il est peintre! / elle/il sait peindre!} \zh{他是画家!}  

\lhead{\firstmark}
\rhead{\botmark}

\subsection{\hspace{-0.5cm} {\Large \textcolor{darkblue}{\textbf{\ipa{lɑ˧zo\#˥}}}}\hspace{0.5cm}[\kern2pt{\textcolor{darkblue}{\textbf{\ipa{lɑ˧zo˧}}}}\kern2pt]} \hypertarget{lA\string_Mzo\#\string_T1}{}
\markboth{\textcolor{darkblue}{\textbf{\ipa{lɑ˧zo\#˥}}}}{}
\textcolor{teal}{\mytextsc{nom}} \hspace{4pt} Ton~: \#H.
\textcolor{PineGreen}{\selectlanguage{french}Petit tigre.} \zh{小老虎。}  ¶ \textcolor{darkblue}{\textbf{\ipa{lɑ˧zo˧ tʰv̩˧-ɭɯ\#˥}}} \textcolor{PineGreen}{\selectlanguage{french}\mytextsc{n}+\mytextsc{dem}+\mytextsc{clf}} \zh{那只小老虎}  
 \zh{量词}: \textcolor{darkblue}{\textbf{\ipa{ɭɯ˧}}}  \mytextsc{clf}~: \textcolor{darkblue}{\textbf{\ipa{ɭɯ˧}}} 
\lhead{\firstmark}
\rhead{\botmark}

\subsection{\hspace{-0.5cm} {\Large \textcolor{darkblue}{\textbf{\ipa{lɑ˩gv̩˧}}}}\hspace{0.5cm}[\kern2pt{\textcolor{darkblue}{\textbf{\ipa{lɑ˩gv̩˥}}}}\kern2pt]} \hypertarget{lA\string_Bgv\string_=\string_M1}{}
\markboth{\textcolor{darkblue}{\textbf{\ipa{lɑ˩gv̩˧}}}}{}
\textcolor{teal}{\mytextsc{adjectif}} \hspace{4pt} Ton~: LM.
\textcolor{PineGreen}{\selectlanguage{french}Recourbé, tordu, courbe.} \zh{弯(树...)。}  ¶ \textcolor{darkblue}{\textbf{\ipa{si˧dzi˩ | lɑ˩-gv̩˧-ze˩}}} \textcolor{PineGreen}{\selectlanguage{french}L'arbre est devenu courbé.} \zh{树弯了。}  

\lhead{\firstmark}
\rhead{\botmark}

\subsection{\hspace{-0.5cm} {\Large \textcolor{darkblue}{\textbf{\ipa{lɑ˩gv̩˧-lɑ˩ɲi˩}}}}\hspace{0.5cm}[\kern2pt{\textcolor{darkblue}{\textbf{\ipa{lɑ˩gv̩˧lɑ˩ɲi˩}}}}\kern2pt]} \hypertarget{lA\string_Bgv\string_=\string_M-lA\string_BJi\string_B1}{}
\markboth{\textcolor{darkblue}{\textbf{\ipa{lɑ˩gv̩˧-lɑ˩ɲi˩}}}}{}
\textcolor{teal}{\mytextsc{adjectif}} \hspace{4pt} Ton~: LM-L.
\textit{De:} \textbf{lɑ˩gv̩˧} \textcolor{PineGreen}{\selectlanguage{french}Tout tordu, tout recourbé.} \zh{弯(路,植物,人的四肢)。} 
\lhead{\firstmark}
\rhead{\botmark}

\subsection{\hspace{-0.5cm} {\Large \textcolor{darkblue}{\textbf{\ipa{lɑ˩jɤ˧-ɬi˧}}}}\hspace{0.5cm}[\kern2pt{\textcolor{darkblue}{\textbf{\ipa{lɑ˩jɤ˧ɬi˧}}}}\kern2pt]} \hypertarget{lA\string_Bj7\string_M-Ki\string_M1}{}
\markboth{\textcolor{darkblue}{\textbf{\ipa{lɑ˩jɤ˧-ɬi˧}}}}{}
\textcolor{teal}{\mytextsc{nom}} \hspace{4pt} Ton~: LM-.
\textcolor{PineGreen}{\selectlanguage{french}Le douzième mois.} \zh{十二月。} 
\lhead{\firstmark}
\rhead{\botmark}

\subsection{\hspace{-0.5cm} {\Large \textcolor{darkblue}{\textbf{\ipa{lɑ˩\textasciitilde{}lɑ˧˥}}}}\hspace{0.5cm}[\kern2pt{\textcolor{darkblue}{\textbf{\ipa{lɑ˧lɑ˧˥}}}}\kern2pt]} \hypertarget{lA\string_B~lA\string_M\string_T1}{}
\markboth{\textcolor{darkblue}{\textbf{\ipa{lɑ˩\textasciitilde{}lɑ˧˥}}}}{}
\textcolor{teal}{\mytextsc{verbe}} \hspace{4pt} Ton~: MH.
\textcolor{PineGreen}{\selectlanguage{french}Se disputer, se battre.} \zh{打架、吵架。}  ¶ \textcolor{darkblue}{\textbf{\ipa{lɑ˩lɑ˧-hĩ˥ | ʈʂʰɯ˧-tɕi˩}}} \textcolor{PineGreen}{\selectlanguage{french}ces (gens) qui se disputent} \zh{打架的这些(人)}  
\textit{Voir~:} \textcolor{darkblue}{\textbf{\ipa{lɑ˧˥}}} 
\lhead{\firstmark}
\rhead{\botmark}

\subsection{\hspace{-0.5cm} {\Large \textcolor{darkblue}{\textbf{\ipa{lɑ˩mɑ˩}}}}\hspace{0.5cm}[\kern2pt{\textcolor{darkblue}{\textbf{\ipa{lɑ˩mɑ˩˥}}}}\kern2pt]} \hypertarget{lA\string_BmA\string_B1}{}
\markboth{\textcolor{darkblue}{\textbf{\ipa{lɑ˩mɑ˩}}}}{}
\textcolor{teal}{\mytextsc{nom}} \hspace{4pt} Ton~: L.
\textcolor{PineGreen}{\selectlanguage{french}Nom de clan/famille étendue. Quatre familles portent ce nom à Yongning.} \zh{一个姓。这个姓,永宁有四个家。}  ¶ \textcolor{darkblue}{\textbf{\ipa{lɑ˩mɑ˩=ɻ̍˥\$}}} \textcolor{PineGreen}{\selectlanguage{french}le clan \textcolor{darkblue}{\textbf{\ipa{/lɑ˩mɑ˩/}}}, la famille \textcolor{darkblue}{\textbf{\ipa{/lɑ˩mɑ˩/}}}} \zh{\textcolor{darkblue}{\textbf{\ipa{/lɑ˩mɑ˩/}}}家族}  
 ¶ \textcolor{darkblue}{\textbf{\ipa{lɑ˩mɑ˩-gv̩˥mɑ˩}}} \textcolor{PineGreen}{\selectlanguage{french}nom d'une personne, comportant un nom de famille (\textcolor{darkblue}{\textbf{\ipa{/lɑ˩mɑ˩/}}}) et un prénom (\textcolor{darkblue}{\textbf{\ipa{/gv̩˧mɑ˧/}}})} \zh{一个人的名字:姓为\textcolor{darkblue}{\textbf{\ipa{/lɑ˩mɑ˩/}}},名为\textcolor{darkblue}{\textbf{\ipa{/gv̩˧mɑ˧/}}}}  

\lhead{\firstmark}
\rhead{\botmark}

\subsection{\hspace{-0.5cm} {\Large \textcolor{darkblue}{\textbf{\ipa{lɑ˩tɑ˧}}}}\hspace{0.5cm}[\kern2pt{\textcolor{darkblue}{\textbf{\ipa{lɑ˩tɑ˥}}}}\kern2pt]} \hypertarget{lA\string_BtA\string_M1}{}
\markboth{\textcolor{darkblue}{\textbf{\ipa{lɑ˩tɑ˧}}}}{}
\textcolor{teal}{\mytextsc{adjectif}} \hspace{4pt} Ton~: LM.
\textcolor{PineGreen}{\selectlanguage{french}De biais, de travers (ex.: porter son chapeau de travers).} \zh{歪,偏 (帽子戴得歪)。} 
\lhead{\firstmark}
\rhead{\botmark}

\subsection{\hspace{-0.5cm} {\Large \textcolor{darkblue}{\textbf{\ipa{-lɑ˩tɑ˩}}}}\hspace{0.5cm}[\kern2pt{\textcolor{darkblue}{\textbf{\ipa{lɑ˩tɑ˩˥}}}}\kern2pt]} \hypertarget{-lA\string_BtA\string_B1}{}
\markboth{\textcolor{darkblue}{\textbf{\ipa{-lɑ˩tɑ˩}}}}{}
\textcolor{teal}{\mytextsc{postposition}} \hspace{4pt} Ton~: L.
\textcolor{PineGreen}{\selectlanguage{french}À proximité de.}  ¶ \textcolor{darkblue}{\textbf{\ipa{ɑ˩ʁo˧ | -lɑ˩tɑ˩˥}}} \textcolor{PineGreen}{\selectlanguage{french}le périmètre de la maison, là où s'étend le domaine de la maison} \zh{家的面积}  

\lhead{\firstmark}
\rhead{\botmark}

\subsection{\hspace{-0.5cm} {\Large \textcolor{darkblue}{\textbf{\ipa{lɑ˩ʈʂv̩˩}}}}\hspace{0.5cm}[\kern2pt{\textcolor{darkblue}{\textbf{\ipa{lɑ˩ʈʂv̩˩˥}}}}\kern2pt]} \hypertarget{lA\string_Bt`s`v\string_=\string_B1}{}
\markboth{\textcolor{darkblue}{\textbf{\ipa{lɑ˩ʈʂv̩˩}}}}{}
\textcolor{teal}{\mytextsc{nom}} \hspace{4pt} Ton~: L.
\textcolor{PineGreen}{\selectlanguage{french}Bougie.} \zh{蜡烛。}  Emprunt~: chinois  \zh{蜡烛}
 \zh{量词}: \textcolor{darkblue}{\textbf{\ipa{kɤ˧˥}}}  \mytextsc{clf}~: \textcolor{darkblue}{\textbf{\ipa{kɤ˧˥}}} 
\lhead{\firstmark}
\rhead{\botmark}

\subsection{\hspace{-0.5cm} {\Large \textcolor{darkblue}{\textbf{\ipa{lɑ˧˥}}} \textsubscript{1}}\hspace{0.5cm}[\kern2pt{\textcolor{darkblue}{\textbf{\ipa{lɑ˧˥}}}}\kern2pt]} \hypertarget{lA\string_M\string_T1}{}
\markboth{\textcolor{darkblue}{\textbf{\ipa{lɑ˧˥}}} \textsubscript{1}}{}
\textcolor{teal}{\mytextsc{verbe}} \hspace{4pt} Ton~: MH.
\textcolor{PineGreen}{\selectlanguage{french}Battre quelque chose, frapper quelque chose, enfoncer un clou, casser des cailloux; donner (une injonction…).} \zh{打(打人,钉钉子……)。}  ¶ \textcolor{darkblue}{\textbf{\ipa{hĩ˧ lɑ˩}}} \textcolor{PineGreen}{\selectlanguage{french}frapper quelqu'un} \zh{打人}  
 ¶ \textcolor{darkblue}{\textbf{\ipa{hɑ˧ lɑ˩}}} \textcolor{PineGreen}{\selectlanguage{french}battre le grain} \zh{打粮食}  
 ¶ \textcolor{darkblue}{\textbf{\ipa{nv̩˩ɭɯ˧ lɑ˧}}} \textcolor{PineGreen}{\selectlanguage{french}battre les cosses de soja} \zh{打大豆}  
 ¶ \textcolor{darkblue}{\textbf{\ipa{sɯ˩tʰi˩-po˥-ɳɯ˩ | lɑ˧˥}}} \textcolor{PineGreen}{\selectlanguage{french}casser au moyen d'un couteau (du thé compressé en galettes ou en briques, à l'ancienne)} \zh{用刀子来砍(沱茶、茶饼)}  
 ¶ \textcolor{darkblue}{\textbf{\ipa{ə˧ʝi˧-ʂɯ˥ʝi˩, | ɬi˧di˩-dʑo˩, | æ˧ lɑ˩-hĩ˩ F | dʑo˩˥! | ʂe˧ lɑ˧-hĩ˥ F | dʑo˩˥! | hæ̃˩ lɑ˩-hĩ˥ F | dʑo˩˥! | ŋv̩˩ lɑ˩-hĩ˥ F | dʑo˩˥!}}} \textcolor{PineGreen}{\selectlanguage{french}Autrefois, à Yongning, il y avait des artisans qui travaillaient le cuivre! Il y avait des artisans qui travaillaient le fer! Il y avait des artisans qui travaillaient l'or! Il y avait des artisans qui travaillaient l'argent!} \zh{过去,在永宁,有铜匠、铁匠、金匠、银匠。}  
 ¶ \textcolor{darkblue}{\textbf{\ipa{ə˧ʝi˧-ʂɯ˥ʝi˩, | ɬi˧di˩-dʑo˩, | æ˧ lɑ˩-hĩ˩ dʑo˩, | ʂe˧ lɑ˧-hĩ˥ dʑo˩, | hæ̃˩ lɑ˩-hĩ˥ dʑo˩, | ŋv̩˩ lɑ˩-hĩ˥ dʑo˩.}}} \textcolor{PineGreen}{\selectlanguage{french}Autrefois, à Yongning, il y avait des artisans qui travaillaient le cuivre. Il y avait des artisans qui travaillaient le fer. Il y avait des artisans qui travaillaient l'or. Il y avait des artisans qui travaillaient l'argent.} \zh{过去,在永宁,有铜匠、铁匠、金匠、银匠。}  
\textit{Voir~:} \hyperlink{}{\textcolor{darkblue}{\textbf{\ipa{lɑ˩\textasciitilde{}lɑ˧˥}}}} 
\lhead{\firstmark}
\rhead{\botmark}

\subsection{\hspace{-0.5cm} {\Large \textcolor{darkblue}{\textbf{\ipa{lɑ˧˥}}} \textsubscript{2}}\hspace{0.5cm}[\kern2pt{\textcolor{darkblue}{\textbf{\ipa{lɑ˧˥}}}}\kern2pt]} \hypertarget{lA\string_M\string_T2}{}
\markboth{\textcolor{darkblue}{\textbf{\ipa{lɑ˧˥}}} \textsubscript{2}}{}
\textcolor{teal}{\mytextsc{verbe}} \hspace{4pt} Ton~: MH.
\textcolor{PineGreen}{\selectlanguage{french}Apparaître, y avoir (de la rosée).} \zh{有,结(露水)。}  ¶ \textcolor{darkblue}{\textbf{\ipa{ɖʐv̩˧ lɑ˧˥}}} \textcolor{PineGreen}{\selectlanguage{french}Il y a de la rosée; de la rosée s'est formée} \zh{结露水了。}  
 ¶ \textcolor{darkblue}{\textbf{\ipa{ɖʐv̩˧qʰɑ˧ lɑ˧˥}}} \textcolor{PineGreen}{\selectlanguage{french}Il y a de la rosée; de la rosée s'est formée} \zh{结露水了。}  

\lhead{\firstmark}
\rhead{\botmark}

\subsection{\hspace{-0.5cm} {\Large \textcolor{darkblue}{\textbf{\ipa{‑læ˧}}}}\hspace{0.5cm}[\kern2pt{\textcolor{darkblue}{\textbf{\ipa{læ˥}}}}\kern2pt]} \hypertarget{‑l\{\string_M1}{}
\markboth{\textcolor{darkblue}{\textbf{\ipa{‑læ˧}}}}{}
\textcolor{teal}{\mytextsc{suffixe}} \hspace{4pt} Ton~: M.
\textcolor{PineGreen}{\selectlanguage{french}Topique, introduisant un élément nouveau, pas nécessairement en contraste avec ce qui précède. Gloses possibles: pour ce qui est de, en ce qui concerne, quant à.} \zh{\mytextsc{主题:……的话、关于……。}}  ¶ \textcolor{darkblue}{\textbf{\ipa{ɖɯ˩mɑ˧ | -læ˧…}}} \textcolor{PineGreen}{\selectlanguage{french}pour ce qui est de ma petite-fille \textcolor{darkblue}{\textbf{\ipa{ɖɯ˩mɑ˧}}}, eh bien…} \zh{关于独妈呢,……}  
 ¶ \textcolor{darkblue}{\textbf{\ipa{lɑ˩mv̩˩˥ | -læ˧...}}} \textcolor{PineGreen}{\selectlanguage{french}pour ce qui est de \textcolor{darkblue}{\textbf{\ipa{lɑ˩mv̩˩˥}}} [nom propre], ...} \zh{关于拉姆呢,……}  
 ¶ \textcolor{darkblue}{\textbf{\ipa{ti˧ɖo˥ | -læ˧…}}} \textcolor{PineGreen}{\selectlanguage{french}pour ce qui est de \textcolor{darkblue}{\textbf{\ipa{ti˧ɖo˥}}} [nom propre], …} \zh{关于\textcolor{darkblue}{\textbf{\ipa{ti˧ɖo˥}}}[人的名字]呢,……}  

\lhead{\firstmark}
\rhead{\botmark}

\subsection{\hspace{-0.5cm} {\Large \textcolor{darkblue}{\textbf{\ipa{læ˧dæ˧qæ˥}}}}\hspace{0.5cm}[\kern2pt{\textcolor{darkblue}{\textbf{\ipa{læ˧dæ˧qæ˥}}}}\kern2pt]} \hypertarget{l\{\string_Md\{\string_Mq\{\string_T1}{}
\markboth{\textcolor{darkblue}{\textbf{\ipa{læ˧dæ˧qæ˥}}}}{}
\textcolor{teal}{\mytextsc{nom}} \hspace{4pt} Ton~: H\#.
\textcolor{PineGreen}{\selectlanguage{french}Aisselle.} \zh{腋下。} Dialecte chinois local~:\zh{膈肢窝。} \zh{量词}: \textcolor{darkblue}{\textbf{\ipa{ɭɯ˧}}}  \mytextsc{clf}~: \textcolor{darkblue}{\textbf{\ipa{ɭɯ˧}}} 
\lhead{\firstmark}
\rhead{\botmark}

\subsection{\hspace{-0.5cm} {\Large \textcolor{darkblue}{\textbf{\ipa{læ˧ʁæ˥\$}}}}\hspace{0.5cm}[\kern2pt{\textcolor{darkblue}{\textbf{\ipa{læ˧ʁæ˥}}}}\kern2pt]} \hypertarget{l\{\string_MR\{\string_T\$1}{}
\markboth{\textcolor{darkblue}{\textbf{\ipa{læ˧ʁæ˥\$}}}}{}
\textcolor{teal}{\mytextsc{nom}} \hspace{4pt} Ton~: H\$.
\textcolor{PineGreen}{\selectlanguage{french}Corbeau.} \zh{乌鸦。}  \zh{量词}: \textcolor{darkblue}{\textbf{\ipa{mi˩}}}  \mytextsc{clf}~: \textcolor{darkblue}{\textbf{\ipa{mi˩}}} 
\lhead{\firstmark}
\rhead{\botmark}

\subsection{\hspace{-0.5cm} {\Large \textcolor{darkblue}{\textbf{\ipa{læ˧ʁæ˧mi˥\$}}}}\hspace{0.5cm}[\kern2pt{\textcolor{darkblue}{\textbf{\ipa{læ˧ʁæ˧mi˥}}}}\kern2pt]} \hypertarget{l\{\string_MR\{\string_Mmi\string_T\$1}{}
\markboth{\textcolor{darkblue}{\textbf{\ipa{læ˧ʁæ˧mi˥\$}}}}{}
\textcolor{teal}{\mytextsc{nom}} \hspace{4pt} Ton~: H\$.
\textcolor{PineGreen}{\selectlanguage{french}Corbeau femelle.} \zh{母乌鸦。}  \zh{量词}: \textcolor{darkblue}{\textbf{\ipa{mi˩}}}  \mytextsc{clf}~: \textcolor{darkblue}{\textbf{\ipa{mi˩}}} 
\lhead{\firstmark}
\rhead{\botmark}

\subsection{\hspace{-0.5cm} {\Large \textcolor{darkblue}{\textbf{\ipa{læ˧ʁæ˧-pʰv̩\#˥}}}}\hspace{0.5cm}[\kern2pt{\textcolor{darkblue}{\textbf{\ipa{xxxx non-correspondance entre le nombre de morphèmes et le nombre de tons de morphèmes}}}}\kern2pt]} \hypertarget{l\{\string_MR\{\string_M-p\string_hv\string_=\#\string_T1}{}
\markboth{\textcolor{darkblue}{\textbf{\ipa{læ˧ʁæ˧-pʰv̩\#˥}}}}{}
\textcolor{teal}{\mytextsc{nom}} \hspace{4pt} Ton~: \#H.
\textcolor{PineGreen}{\selectlanguage{french}Corbeau mâle.} \zh{公乌鸦。}  ¶ \textcolor{darkblue}{\textbf{\ipa{læ˧ʁæ˧-pʰv̩˧ tʰv̩˧-mi˥\$}}} \textcolor{PineGreen}{\selectlanguage{french}\mytextsc{n}+\mytextsc{dem}+\mytextsc{clf}} \zh{那只公乌鸦}  
 \zh{量词}: \textcolor{darkblue}{\textbf{\ipa{mi˩}}}  \mytextsc{clf}~: \textcolor{darkblue}{\textbf{\ipa{mi˩}}} 
\lhead{\firstmark}
\rhead{\botmark}

\subsection{\hspace{-0.5cm} {\Large \textcolor{darkblue}{\textbf{\ipa{læ˧tsɯ˥}}}}\hspace{0.5cm}[\kern2pt{\textcolor{darkblue}{\textbf{\ipa{læ˧tsɯ˥}}}}\kern2pt]} \hypertarget{l\{\string_MtsM\string_T1}{}
\markboth{\textcolor{darkblue}{\textbf{\ipa{læ˧tsɯ˥}}}}{}
\textcolor{teal}{\mytextsc{nom}} \hspace{4pt} Ton~: H\#.
\textcolor{PineGreen}{\selectlanguage{french}Piment.} \zh{辣椒(汉语借词:辣子)。} Dialecte chinois local~:\zh{辣子。} Emprunt~: chinois  \zh{辣子}
 ¶ \textcolor{darkblue}{\textbf{\ipa{læ˧tsɯ˥ hṽ˩\textasciitilde{}hṽ˩}}} \textcolor{PineGreen}{\selectlanguage{french}frire des piments} \zh{炒辣椒}  
 \zh{量词}: \textcolor{darkblue}{\textbf{\ipa{ɭɯ˧}}}  \mytextsc{clf}~: \textcolor{darkblue}{\textbf{\ipa{ɭɯ˧}}} 
\lhead{\firstmark}
\rhead{\botmark}

\subsection{\hspace{-0.5cm} {\Large \textcolor{darkblue}{\textbf{\ipa{le˧‑}}}}\hspace{0.5cm}[\kern2pt{\textcolor{darkblue}{\textbf{\ipa{le˥}}}}\kern2pt]} \hypertarget{le\string_M‑1}{}
\markboth{\textcolor{darkblue}{\textbf{\ipa{le˧‑}}}}{}
\textcolor{teal}{\mytextsc{préfixe}} \hspace{4pt} Ton~: M/0.
\textcolor{PineGreen}{\selectlanguage{french}\mytextsc{accomp}.} \zh{\mytextsc{实施。}} 
\lhead{\firstmark}
\rhead{\botmark}

\subsection{\hspace{-0.5cm} {\Large \textcolor{darkblue}{\textbf{\ipa{le˧-tɑ˧˥}}}}\hspace{0.5cm}[\kern2pt{\textcolor{darkblue}{\textbf{\ipa{xxxx non-correspondance entre le nombre de morphèmes et le nombre de tons de morphèmes}}}}\kern2pt]} \hypertarget{le\string_M-tA\string_M\string_T1}{}
\markboth{\textcolor{darkblue}{\textbf{\ipa{le˧-tɑ˧˥}}}}{}
\textcolor{teal}{\mytextsc{conjonction}} \hspace{4pt} Ton~: MH.
\textcolor{PineGreen}{\selectlanguage{french}Jusqu'à; même.} \zh{到……为止、一直到……、连……。} 
\lhead{\firstmark}
\rhead{\botmark}

\subsection{\hspace{-0.5cm} {\Large \textcolor{darkblue}{\textbf{\ipa{le˧-wo˥}}}}\hspace{0.5cm}[\kern2pt{\textcolor{darkblue}{\textbf{\ipa{xxxx non-correspondance entre le nombre de morphèmes et le nombre de tons de morphèmes}}}}\kern2pt]} \hypertarget{le\string_M-wo\string_T1}{}
\markboth{\textcolor{darkblue}{\textbf{\ipa{le˧-wo˥}}}}{}
\textcolor{teal}{\mytextsc{adverbe}} \hspace{4pt} Ton~: H\#.
\textcolor{PineGreen}{\selectlanguage{french}À nouveau, de nouveau.} \zh{再、又、重新。} 
\lhead{\firstmark}
\rhead{\botmark}

\subsection{\hspace{-0.5cm} {\Large \textcolor{darkblue}{\textbf{\ipa{le˧-wo˧}}}}\hspace{0.5cm}[\kern2pt{\textcolor{darkblue}{\textbf{\ipa{xxxx non-correspondance entre le nombre de morphèmes et le nombre de tons de morphèmes}}}}\kern2pt]} \hypertarget{le\string_M-wo\string_M1}{}
\markboth{\textcolor{darkblue}{\textbf{\ipa{le˧-wo˧}}}}{}
\textcolor{teal}{\mytextsc{adverbe}} \hspace{4pt} Ton~: M.
\textcolor{PineGreen}{\selectlanguage{french}À nouveau.} \zh{又,……回去。}  ¶ \textcolor{darkblue}{\textbf{\ipa{le˧-wo˧ jo˧}}} \textcolor{PineGreen}{\selectlanguage{french}revenir} \zh{回}  
 ¶ \textcolor{darkblue}{\textbf{\ipa{le˧-wo˧ le˧-gv̩˩}}} \textcolor{PineGreen}{\selectlanguage{french}recommencer} \zh{从头开始}  
 ¶ \textcolor{darkblue}{\textbf{\ipa{le˧-wo˧ le˧-gv̩˧\textasciitilde{}gv̩˥}}} \textcolor{PineGreen}{\selectlanguage{french}refaire, réinstaller, reconstruire} \zh{重新做、重新建}  
 ¶ \textcolor{darkblue}{\textbf{\ipa{le˧-wo˧ le˥-tɕo˩ ʐwɤ˩}}} \textcolor{PineGreen}{\selectlanguage{french}répéter sans arrêt} \zh{重复讲说过的话}  

\lhead{\firstmark}
\rhead{\botmark}

\subsection{\hspace{-0.5cm} {\Large \textcolor{darkblue}{\textbf{\ipa{le˩}}}}\hspace{0.5cm}[\kern2pt{\textcolor{darkblue}{\textbf{\ipa{le˩˥}}}}\kern2pt]} \hypertarget{le\string_B1}{}
\markboth{\textcolor{darkblue}{\textbf{\ipa{le˩}}}}{}
\textcolor{teal}{\mytextsc{particule}} \textcolor{teal}{\mytextsc{de}} \textcolor{teal}{\mytextsc{discours}} \hspace{4pt} Ton~: L?.
\textcolor{PineGreen}{\selectlanguage{french}Particule finale exclamative.} \zh{句尾助词:感叹。}  ¶ \textcolor{darkblue}{\textbf{\ipa{dʑɤ˩ le˥!}}} \textcolor{PineGreen}{\selectlanguage{french}Bravo!} \zh{好了!/ 太好了!}  

\lhead{\firstmark}
\rhead{\botmark}

\subsection{\hspace{-0.5cm} {\Large \textcolor{darkblue}{\textbf{\ipa{li˧\textsubscript{a}}}}}\hspace{0.5cm}[\kern2pt{\textcolor{darkblue}{\textbf{\ipa{li˥}}}}\kern2pt]} \hypertarget{li\string_Ma1}{}
\markboth{\textcolor{darkblue}{\textbf{\ipa{li˧\textsubscript{a}}}}}{}
\textcolor{teal}{\mytextsc{verbe}} \hspace{4pt} Ton~: M\textsubscript{a}.
\ding{202} \textcolor{PineGreen}{\selectlanguage{french}Regarder.} \zh{看。}  ¶ \textcolor{darkblue}{\textbf{\ipa{tʰi˧-li˧-dʑo˧}}} \textcolor{PineGreen}{\selectlanguage{french}\mytextsc{dur} \string_ \mytextsc{prog}} \zh{正在看}  
 ¶ \textcolor{darkblue}{\textbf{\ipa{tso˧\textasciitilde{}tso˧ li˩}}} \textcolor{PineGreen}{\selectlanguage{french}regarder des choses} \zh{看东西}  
\ding{203} \textcolor{PineGreen}{\selectlanguage{french}S'occuper de.} \zh{管理。}  ¶ \textcolor{darkblue}{\textbf{\ipa{ɑ˩ʁo˧ li˧}}} \textcolor{PineGreen}{\selectlanguage{french}s'occuper de la maison, veiller aux affaires de la maison; surveiller la maison} \zh{管家、管家里的事情,看守家}  
\ding{204} \textcolor{PineGreen}{\selectlanguage{french}Rendre visite à, aller voir (quelqu'un).} \zh{访问。}  ¶ \textcolor{darkblue}{\textbf{\ipa{pʰæ˧tɕi˥-zo˩-ɳɯ˩ | mv̩˩zo˩ li˥}}} \textcolor{PineGreen}{\selectlanguage{french}Le jeune homme voit (=va voir) la jeune femme. (Euphémisme pour désigner les relations sexuelles.)} \zh{小伙子去拜访年轻女人(委婉语,指性交)}  

\lhead{\firstmark}
\rhead{\botmark}

\subsection{\hspace{-0.5cm} {\Large \textcolor{darkblue}{\textbf{\ipa{li˧ʐv̩˩}}}}\hspace{0.5cm}[\kern2pt{\textcolor{darkblue}{\textbf{\ipa{li˩ʐv̩˩˥}}}}\kern2pt]} \hypertarget{li\string_Mz`v\string_=\string_B1}{}
\markboth{\textcolor{darkblue}{\textbf{\ipa{li˧ʐv̩˩}}}}{}
\textcolor{teal}{\mytextsc{nom}} \hspace{4pt} Ton~: L\#.
\textcolor{PineGreen}{\selectlanguage{french}Filet-mignon.} \zh{里脊肉。} 
\lhead{\firstmark}
\rhead{\botmark}

\subsection{\hspace{-0.5cm} {\Large \textcolor{darkblue}{\textbf{\ipa{li˩pi˥}}}}\hspace{0.5cm}[\kern2pt{\textcolor{darkblue}{\textbf{\ipa{li˧pi˥}}}}\kern2pt]} \hypertarget{li\string_Bpi\string_T1}{}
\markboth{\textcolor{darkblue}{\textbf{\ipa{li˩pi˥}}}}{}
\textcolor{teal}{\mytextsc{nom}} \hspace{4pt} Ton~: LH.
\textcolor{PineGreen}{\selectlanguage{french}Feuille de thé qui a trop infusé, qui est bonne à jeter.} \zh{已经泡了太久的茶叶。}  \zh{量词}: \textcolor{darkblue}{\textbf{\ipa{kʰwɤ˥}}}  \mytextsc{clf}~: \textcolor{darkblue}{\textbf{\ipa{kʰwɤ˥}}} 
\lhead{\firstmark}
\rhead{\botmark}

\subsection{\hspace{-0.5cm} {\Large \textcolor{darkblue}{\textbf{\ipa{li˩pʰv̩˩}}}}\hspace{0.5cm}[\kern2pt{\textcolor{darkblue}{\textbf{\ipa{li˧pʰv̩˩}}}}\kern2pt]} \hypertarget{li\string_Bp\string_hv\string_=\string_B1}{}
\markboth{\textcolor{darkblue}{\textbf{\ipa{li˩pʰv̩˩}}}}{}
\textcolor{teal}{\mytextsc{nom}} \hspace{4pt} Ton~: L.
\textcolor{PineGreen}{\selectlanguage{french}Un lichen de montagne, \textit{Thamnolia vermicularis}, employé en décoction. Au souvenir de F4, ce lichen se cueillait au septième mois, seule période où il était abondant; on allait le cueillir en haute montagne.} \zh{雪茶。}  ¶ \textcolor{darkblue}{\textbf{\ipa{ŋwɤ˧hɑ̃˩-li˩pʰv˩}}} \textcolor{PineGreen}{\selectlanguage{french}thé de lichen de la montagne \textcolor{darkblue}{\textbf{\ipa{ŋwɤ˧hɑ̃˩}}} (ce lichen y est abondant; c'est généralement là-bas qu'on le cueillait)} \zh{\textcolor{darkblue}{\textbf{\ipa{ŋwɤ˧hɑ̃˩}}} 山的雪茶(说明:这种苔藓在那座山上多,七月份人家去采)}  

\lhead{\firstmark}
\rhead{\botmark}

\subsection{\hspace{-0.5cm} {\Large \textcolor{darkblue}{\textbf{\ipa{li˩˥}}}}\hspace{0.5cm}[\kern2pt{\textcolor{darkblue}{\textbf{\ipa{li˩˥}}}}\kern2pt]} \hypertarget{li\string_B\string_T1}{}
\markboth{\textcolor{darkblue}{\textbf{\ipa{li˩˥}}}}{}
\textcolor{teal}{\mytextsc{nom}} \hspace{4pt} Ton~: LH.
\textcolor{PineGreen}{\selectlanguage{french}Thé.} \zh{茶。}  ¶ \textcolor{darkblue}{\textbf{\ipa{li˩qʰɑ˩}}} \textcolor{PineGreen}{\selectlanguage{french}'thé amer': décoction de feuilles de pivoine blanche de Chine, que l'on buvait lorsqu'il n'y avait pas de thé à la maison; cela possédait des vertus médicinales.} \zh{‘苦茶’:用白芍药来泡的饮料,没有茶的时候就喝这种‘苦茶’。它有医疗作用,帮助消化。}  
 \zh{量词}: \textcolor{darkblue}{\textbf{\ipa{qʰwɤ˧˥}}}  \mytextsc{clf}~: \textcolor{darkblue}{\textbf{\ipa{qʰwɤ˧˥}}} \textcolor{PineGreen}{\selectlanguage{french}bol}
\lhead{\firstmark}
\rhead{\botmark}

\subsection{\hspace{-0.5cm} {\Large \textcolor{darkblue}{\textbf{\ipa{ljɤ˩\textsubscript{a}}}}}\hspace{0.5cm}[\kern2pt{\textcolor{darkblue}{\textbf{\ipa{ljɤ˩˥}}}}\kern2pt]} \hypertarget{lj7\string_Ba1}{}
\markboth{\textcolor{darkblue}{\textbf{\ipa{ljɤ˩\textsubscript{a}}}}}{}
\textcolor{teal}{\mytextsc{classificateur}} \hspace{4pt} Ton~: L\textsubscript{a}.
\textcolor{PineGreen}{\selectlanguage{french}Auto-classificateur des vies/destins.} \zh{量词:命、命运。}  ¶ \textcolor{darkblue}{\textbf{\ipa{ʈʂʰɯ˧-ljɤ˥}}} \textcolor{PineGreen}{\selectlanguage{french}\mytextsc{dem} \string_ (tone: H\# / H\$)} \zh{\mytextsc{指示代词} \string_}  

\lhead{\firstmark}
\rhead{\botmark}

\subsection{\hspace{-0.5cm} {\Large \textcolor{darkblue}{\textbf{\ipa{ljɤ˩mi˥}}}}\hspace{0.5cm}[\kern2pt{\textcolor{darkblue}{\textbf{\ipa{ljɤ˩mi˩˥}}}}\kern2pt]} \hypertarget{lj7\string_Bmi\string_T1}{}
\markboth{\textcolor{darkblue}{\textbf{\ipa{ljɤ˩mi˥}}}}{}
\textcolor{teal}{\mytextsc{nom}} \hspace{4pt} Ton~: LH.
\textcolor{PineGreen}{\selectlanguage{french}Poutre importante.} \zh{大梁。}  \zh{量词}: \textcolor{darkblue}{\textbf{\ipa{pʰæ˧˥}}}  \mytextsc{clf}~: \textcolor{darkblue}{\textbf{\ipa{pʰæ˧˥}}} 
\lhead{\firstmark}
\rhead{\botmark}

\subsection{\hspace{-0.5cm} {\Large \textcolor{darkblue}{\textbf{\ipa{ljɤ˩mi˥-ʈæ˩qo˩}}}}\hspace{0.5cm}[\kern2pt{\textcolor{darkblue}{\textbf{\ipa{xxxx non-correspondance entre le nombre de morphèmes et le nombre de tons de morphèmes}}}}\kern2pt]} \hypertarget{lj7\string_Bmi\string_T-t`\{\string_Bqo\string_B1}{}
\markboth{\textcolor{darkblue}{\textbf{\ipa{ljɤ˩mi˥-ʈæ˩qo˩}}}}{}
\textcolor{teal}{\mytextsc{nom}} \hspace{4pt} Ton~: LH-.
\textcolor{PineGreen}{\selectlanguage{french}Enjolivement sous une poutre; est perçu symboliquement comme 'l'oreille' de la poutre.} \zh{大梁的装饰:大梁的‘耳朵’。}  \zh{量词}: \textcolor{darkblue}{\textbf{\ipa{pʰæ˧˥}}}  \mytextsc{clf}~: \textcolor{darkblue}{\textbf{\ipa{pʰæ˧˥}}} 
\lhead{\firstmark}
\rhead{\botmark}

\subsection{\hspace{-0.5cm} {\Large \textcolor{darkblue}{\textbf{\ipa{ljɤ˩ʂɯ˩}}}}\hspace{0.5cm}[\kern2pt{\textcolor{darkblue}{\textbf{\ipa{xxxx non-correspondance entre le nombre de morphèmes et le nombre de tons de morphèmes}}}}\kern2pt]} \hypertarget{lj7\string_Bs`M\string_B1}{}
\markboth{\textcolor{darkblue}{\textbf{\ipa{ljɤ˩ʂɯ˩}}}}{}
\textcolor{teal}{\mytextsc{nom}} \hspace{4pt} Ton~: L.
\textcolor{PineGreen}{\selectlanguage{french}Céréales.} \zh{粮食(汉语借词)。}  Emprunt~: chinois  \zh{粮食}

\lhead{\firstmark}
\rhead{\botmark}

\subsection{\hspace{-0.5cm} {\Large \textcolor{darkblue}{\textbf{\ipa{ljɤ˩˥}}} \textsubscript{1}}\hspace{0.5cm}[\kern2pt{\textcolor{darkblue}{\textbf{\ipa{ljɤ˩˥}}}}\kern2pt]} \hypertarget{lj7\string_B\string_T1}{}
\markboth{\textcolor{darkblue}{\textbf{\ipa{ljɤ˩˥}}} \textsubscript{1}}{}
\textcolor{teal}{\mytextsc{nom}} \hspace{4pt} Ton~: LH.
\textcolor{PineGreen}{\selectlanguage{french}Poutre.} \zh{梁。}  \zh{量词}: \textcolor{darkblue}{\textbf{\ipa{pʰæ˧˥}}}  \mytextsc{clf}~: \textcolor{darkblue}{\textbf{\ipa{pʰæ˧˥}}} 
\lhead{\firstmark}
\rhead{\botmark}

\subsection{\hspace{-0.5cm} {\Large \textcolor{darkblue}{\textbf{\ipa{ljɤ˩˥}}} \textsubscript{2}}\hspace{0.5cm}[\kern2pt{\textcolor{darkblue}{\textbf{\ipa{ljɤ˩˥}}}}\kern2pt]} \hypertarget{lj7\string_B\string_T2}{}
\markboth{\textcolor{darkblue}{\textbf{\ipa{ljɤ˩˥}}} \textsubscript{2}}{}
\textcolor{teal}{\mytextsc{nom}} \hspace{4pt} Ton~: LM?LH?.
\textcolor{PineGreen}{\selectlanguage{french}Sort, lot, existence, vie, destinée.} \zh{命、生命、命运。}  ¶ \textcolor{darkblue}{\textbf{\ipa{no˧ | ljɤ˩ ʈʂʰɯ˧-ljɤ˩-dʑo˩, | qʰæ˩˥ | ʐwæ˩˥!}}} \textcolor{PineGreen}{\selectlanguage{french}Tu as une belle vie! Tu as la vie belle!} \zh{你命好!}  
 ¶ \textcolor{darkblue}{\textbf{\ipa{hĩ˧-ljɤ˥}}} \textcolor{PineGreen}{\selectlanguage{french}l'existence humaine} \zh{人命、人类的命运}  
 \zh{量词}: \textcolor{darkblue}{\textbf{\ipa{ljɤ˩}}}  \mytextsc{clf}~: \textcolor{darkblue}{\textbf{\ipa{ljɤ˩}}} 
\lhead{\firstmark}
\rhead{\botmark}

\subsection{\hspace{-0.5cm} {\Large \textcolor{darkblue}{\textbf{\ipa{lje˩fe˧}}}}\hspace{0.5cm}[\kern2pt{\textcolor{darkblue}{\textbf{\ipa{lje˧fe˩}}}}\kern2pt]} \hypertarget{lje\string_Bfe\string_M1}{}
\markboth{\textcolor{darkblue}{\textbf{\ipa{lje˩fe˧}}}}{}
\textcolor{teal}{\mytextsc{nom}} \hspace{4pt} Ton~: LM.
\textcolor{PineGreen}{\selectlanguage{french}Liangfen: spécialité de Dali et des environs.} \zh{凉粉。}  Emprunt~: chinois  \zh{凉粉}

\lhead{\firstmark}
\rhead{\botmark}

\subsection{\hspace{-0.5cm} {\Large \textcolor{darkblue}{\textbf{\ipa{lo˧}}}}\hspace{0.5cm}[\kern2pt{\textcolor{darkblue}{\textbf{\ipa{lo˥}}}}\kern2pt]} \hypertarget{lo\string_M1}{}
\markboth{\textcolor{darkblue}{\textbf{\ipa{lo˧}}}}{}
\textcolor{teal}{\mytextsc{nom}} \hspace{4pt} Ton~: M.
\ding{202} \textcolor{PineGreen}{\selectlanguage{french}Occupation, travail, tâche.} \zh{事情。}  ¶ \textcolor{darkblue}{\textbf{\ipa{lo˧ dʑo˧}}} \textcolor{PineGreen}{\selectlanguage{french}avoir du travail, être occupé} \zh{忙,有活要干}  
 ¶ \textcolor{darkblue}{\textbf{\ipa{njɤ˧ | lo˧ mɤ˧-dʑo˧.}}} \textcolor{PineGreen}{\selectlanguage{french}Je ne suis pas occupé. / J'ai du temps libre. / Je suis disponible.} \zh{我不忙。}  
 \zh{量词}: \textcolor{darkblue}{\textbf{\ipa{lo˧}}} \ding{203} \textcolor{PineGreen}{\selectlanguage{french}Utilité.} \zh{用处。}  ¶ \textcolor{darkblue}{\textbf{\ipa{lo˧ mɤ˧-dʑo˧}}} \textcolor{PineGreen}{\selectlanguage{french}C'est inutile / ça n'a aucune utilité. (Contexte: discussion au sujet du lierre, plante qui n'est utilisée ni comme fourrage, ni comme médicament, ni comme combustible, ni pour la confection de cordes ou autres outils ou objets)} \zh{没有用!(情景:谈到常春藤,说它是没有用处的植物)}  
 \mytextsc{clf}~: \textcolor{darkblue}{\textbf{\ipa{lo˧}}} 
\lhead{\firstmark}
\rhead{\botmark}

\subsection{\hspace{-0.5cm} {\Large \textcolor{darkblue}{\textbf{\ipa{lo˧\textsubscript{b}}}}}\hspace{0.5cm}[\kern2pt{\textcolor{darkblue}{\textbf{\ipa{lo˩˥}}}}\kern2pt]} \hypertarget{lo\string_Mb1}{}
\markboth{\textcolor{darkblue}{\textbf{\ipa{lo˧\textsubscript{b}}}}}{}
\textcolor{teal}{\mytextsc{classificateur}} \hspace{4pt} Ton~: M\textsubscript{b}.
\textcolor{PineGreen}{\selectlanguage{french}Auto-classificateur des travaux/occupations.} \zh{量词:事情(一件)、活(一个)。} 
\lhead{\firstmark}
\rhead{\botmark}

\subsection{\hspace{-0.5cm} {\Large \textcolor{darkblue}{\textbf{\ipa{lo˧bæ˧˥}}}}\hspace{0.5cm}[\kern2pt{\textcolor{darkblue}{\textbf{\ipa{lo˧bæ˧}}}}\kern2pt]} \hypertarget{lo\string_Mb\{\string_M\string_T1}{}
\markboth{\textcolor{darkblue}{\textbf{\ipa{lo˧bæ˧˥}}}}{}
\textcolor{teal}{\mytextsc{nom}} \hspace{4pt} Ton~: MH\#.
\textcolor{PineGreen}{\selectlanguage{french}Pont suspendu; pont de corde. La corde du pont suspendu aurait été faite d'écorces d'arbres, non de chanvre, car les cordes en chanvre se détériorent rapidement quand elles sont exposées à la pluie.} \zh{索桥,溜索。} 
\lhead{\firstmark}
\rhead{\botmark}

\subsection{\hspace{-0.5cm} {\Large \textcolor{darkblue}{\textbf{\ipa{lo˧bv̩˩-ʈʂʰɯ˩}}}}\hspace{0.5cm}[\kern2pt{\textcolor{darkblue}{\textbf{\ipa{lo˩bv̩˧ʈʂʰɯ˧}}}}\kern2pt]} \hypertarget{lo\string_Mbv\string_=\string_B-t`s`\string_hM\string_B1}{}
\markboth{\textcolor{darkblue}{\textbf{\ipa{lo˧bv̩˩-ʈʂʰɯ˩}}}}{}
\textcolor{teal}{\mytextsc{nom}} \hspace{4pt} Ton~: L\#-.
\textcolor{PineGreen}{\selectlanguage{french}Éléphant.} \zh{象、大象。}  Emprunt~: tibétain
 \zh{量词}: \textcolor{darkblue}{\textbf{\ipa{pʰo˧˥}}} \textcolor{darkblue}{\textbf{\ipa{v̩˧}}}  \mytextsc{clf}~: \textcolor{darkblue}{\textbf{\ipa{pʰo˧˥}}} \textcolor{darkblue}{\textbf{\ipa{v̩˧}}} 
\lhead{\firstmark}
\rhead{\botmark}

\subsection{\hspace{-0.5cm} {\Large \textcolor{darkblue}{\textbf{\ipa{lo˧ɖʐɤ˩}}}}\hspace{0.5cm}[\kern2pt{\textcolor{darkblue}{\textbf{\ipa{lo˩ɖʐɤ˥}}}}\kern2pt]} \hypertarget{lo\string_Md`z`7\string_B1}{}
\markboth{\textcolor{darkblue}{\textbf{\ipa{lo˧ɖʐɤ˩}}}}{}
\textcolor{teal}{\mytextsc{nom}} \hspace{4pt} Ton~: L\#.
\textcolor{PineGreen}{\selectlanguage{french}Serfouette, croc à trois dents: instrument à trois dents perpendiculaires au manche, pour ameublir la terre. Les modèles actuellement utilisés ont une tête en métal.} \zh{三齿耙。}  ¶ \textcolor{darkblue}{\textbf{\ipa{lo˧ɖʐɤ˩ ʈʂʰɯ˩-nɑ˩}}} \textcolor{PineGreen}{\selectlanguage{french}\mytextsc{n}+\mytextsc{dem}+\mytextsc{clf}} \zh{这把三齿耙}  
 \zh{量词}: \textcolor{darkblue}{\textbf{\ipa{nɑ˧}}}  \mytextsc{clf}~: \textcolor{darkblue}{\textbf{\ipa{nɑ˧}}} 
\lhead{\firstmark}
\rhead{\botmark}

\subsection{\hspace{-0.5cm} {\Large \textcolor{darkblue}{\textbf{\ipa{lo˧fv̩˧}}}}\hspace{0.5cm}[\kern2pt{\textcolor{darkblue}{\textbf{\ipa{lo˧fv̩˩}}}}\kern2pt]} \hypertarget{lo\string_Mfv\string_=\string_M1}{}
\markboth{\textcolor{darkblue}{\textbf{\ipa{lo˧fv̩˧}}}}{}
\textcolor{teal}{\mytextsc{adjectif}} \hspace{4pt} Ton~: .
\textcolor{PineGreen}{\selectlanguage{french}Facile à faire.} \zh{容易,容易做。}  ¶ \textcolor{darkblue}{\textbf{\ipa{lo˧fv̩˧ | ʐwæ˩˥}}} \textcolor{PineGreen}{\selectlanguage{french}très facile} \zh{很容易}  

\lhead{\firstmark}
\rhead{\botmark}

\subsection{\hspace{-0.5cm} {\Large \textcolor{darkblue}{\textbf{\ipa{lo˧gv̩˩}}}}\hspace{0.5cm}[\kern2pt{\textcolor{darkblue}{\textbf{\ipa{lo˧gv̩˩}}}}\kern2pt]} \hypertarget{lo\string_Mgv\string_=\string_B1}{}
\markboth{\textcolor{darkblue}{\textbf{\ipa{lo˧gv̩˩}}}}{}
\textcolor{teal}{\mytextsc{nom}} \hspace{4pt} Ton~: L\#.
\textcolor{PineGreen}{\selectlanguage{french}Ninglang; actuellement utilisé pour désigner un village na du comté de Ninglang, relativement proche du centre administratif.} \zh{宁蒗。}  ¶ \textcolor{darkblue}{\textbf{\ipa{lo˧gv̩˩-di˩mi˩}}} \textcolor{PineGreen}{\selectlanguage{french}la plaine de Ninglang} \zh{宁蒗坝子}  

\lhead{\firstmark}
\rhead{\botmark}

\subsection{\hspace{-0.5cm} {\Large \textcolor{darkblue}{\textbf{\ipa{lo˧ʝi˧-hĩ˧-hĩ\#˥}}}}\hspace{0.5cm}[\kern2pt{\textcolor{darkblue}{\textbf{\ipa{xxxx non-correspondance entre le nombre de morphèmes et le nombre de tons de morphèmes}}}}\kern2pt]} \hypertarget{lo\string_Mj££i\string_M-hi\string_~\string_M-hi\string_~\#\string_T1}{}
\markboth{\textcolor{darkblue}{\textbf{\ipa{lo˧ʝi˧-hĩ˧-hĩ\#˥}}}}{}
\textcolor{teal}{\mytextsc{nom}} \hspace{4pt} Ton~: \#H.
\textcolor{PineGreen}{\selectlanguage{french}Travailleur (paysan, ouvrier…).} \zh{劳动人民,农民。}  \zh{量词}: \textcolor{darkblue}{\textbf{\ipa{v̩˧}}}  \mytextsc{clf}~: \textcolor{darkblue}{\textbf{\ipa{v̩˧}}} 
\lhead{\firstmark}
\rhead{\botmark}

\subsection{\hspace{-0.5cm} {\Large \textcolor{darkblue}{\textbf{\ipa{lo˧lo˧}}}}\hspace{0.5cm}[\kern2pt{\textcolor{darkblue}{\textbf{\ipa{lo˧lo˧}}}}\kern2pt]} \hypertarget{lo\string_Mlo\string_M1}{}
\markboth{\textcolor{darkblue}{\textbf{\ipa{lo˧lo˧}}}}{}
\textcolor{teal}{\mytextsc{nom}} \hspace{4pt} Ton~: M.
\textcolor{PineGreen}{\selectlanguage{french}Yi (groupe ethnique).} \zh{彝族。}  \zh{量词}: \textcolor{darkblue}{\textbf{\ipa{v̩˧}}}  \mytextsc{clf}~: \textcolor{darkblue}{\textbf{\ipa{v̩˧}}} 
\lhead{\firstmark}
\rhead{\botmark}

\subsection{\hspace{-0.5cm} {\Large \textcolor{darkblue}{\textbf{\ipa{lo˧ɲi˥}}}}\hspace{0.5cm}[\kern2pt{\textcolor{darkblue}{\textbf{\ipa{lo˧ɲi˥}}}}\kern2pt]} \hypertarget{lo\string_MJi\string_T1}{}
\markboth{\textcolor{darkblue}{\textbf{\ipa{lo˧ɲi˥}}}}{}
\textcolor{teal}{\mytextsc{nom}} \hspace{4pt} Ton~: H\#.
\textcolor{PineGreen}{\selectlanguage{french}Doigt.} \zh{手指。}  \zh{量词}: \textcolor{darkblue}{\textbf{\ipa{ɭɯ˧}}}  \mytextsc{clf}~: \textcolor{darkblue}{\textbf{\ipa{ɭɯ˧}}} 
\lhead{\firstmark}
\rhead{\botmark}

\subsection{\hspace{-0.5cm} {\Large \textcolor{darkblue}{\textbf{\ipa{lo˧ɲi˥ | ɖɯ˧-ɭɯ˧}}}}\hspace{0.5cm}[\kern2pt{\textcolor{darkblue}{\textbf{\ipa{xxxx non-correspondance entre le nombre de groupes tonals et le nombre de tons}}}}\kern2pt]} \hypertarget{lo\string_MJi\string_T | d`M\string_M-l\string_RM\string_M1}{}
\markboth{\textcolor{darkblue}{\textbf{\ipa{lo˧ɲi˥ | ɖɯ˧-ɭɯ˧}}}}{}
\textcolor{teal}{\mytextsc{nom}} \hspace{4pt} Ton~: H\# | M.
\textcolor{PineGreen}{\selectlanguage{french}Index.} \zh{食指。} 
\lhead{\firstmark}
\rhead{\botmark}

\subsection{\hspace{-0.5cm} {\Large \textcolor{darkblue}{\textbf{\ipa{lo˧ɲi˥ | ɲi˧-ɭɯ˧}}}}\hspace{0.5cm}[\kern2pt{\textcolor{darkblue}{\textbf{\ipa{xxxx non-correspondance entre le nombre de groupes tonals et le nombre de tons}}}}\kern2pt]} \hypertarget{lo\string_MJi\string_T | Ji\string_M-l\string_RM\string_M1}{}
\markboth{\textcolor{darkblue}{\textbf{\ipa{lo˧ɲi˥ | ɲi˧-ɭɯ˧}}}}{}
\textcolor{teal}{\mytextsc{nom}} \hspace{4pt} Ton~: H\# | M.
\textcolor{PineGreen}{\selectlanguage{french}Majeur.} \zh{中指。} 
\lhead{\firstmark}
\rhead{\botmark}

\subsection{\hspace{-0.5cm} {\Large \textcolor{darkblue}{\textbf{\ipa{lo˧ʂv̩˩}}}}\hspace{0.5cm}[\kern2pt{\textcolor{darkblue}{\textbf{\ipa{lo˧ʂv̩˩}}}}\kern2pt]} \hypertarget{lo\string_Ms`v\string_=\string_B1}{}
\markboth{\textcolor{darkblue}{\textbf{\ipa{lo˧ʂv̩˩}}}}{}
\textcolor{teal}{\mytextsc{nom}} \hspace{4pt} Ton~: L\#.
\textcolor{PineGreen}{\selectlanguage{french}Luoshui: village du bord du Lac.} \zh{洛水村。} 
\lhead{\firstmark}
\rhead{\botmark}

\subsection{\hspace{-0.5cm} {\Large \textcolor{darkblue}{\textbf{\ipa{lo˧ʂv̩˩ | -hi˩-nɑ˧mi˧}}}}\hspace{0.5cm}[\kern2pt{\textcolor{darkblue}{\textbf{\ipa{xxxx non-correspondance entre le nombre de groupes tonals et le nombre de tons}}}}\kern2pt]} \hypertarget{lo\string_Ms`v\string_=\string_B | -hi\string_B-nA\string_Mmi\string_M1}{}
\markboth{\textcolor{darkblue}{\textbf{\ipa{lo˧ʂv̩˩ | -hi˩-nɑ˧mi˧}}}}{}
\textcolor{teal}{\mytextsc{nom}} \hspace{4pt} Ton~: L\# | L-.
\textcolor{PineGreen}{\selectlanguage{french}Lac Lugu.} \zh{泸沽湖。} 
\lhead{\firstmark}
\rhead{\botmark}

\subsection{\hspace{-0.5cm} {\Large \textcolor{darkblue}{\textbf{\ipa{lo˧tɑ˧-lo˧tɕi\#˥}}}}\hspace{0.5cm}[\kern2pt{\textcolor{darkblue}{\textbf{\ipa{xxxx non-correspondance entre le nombre de morphèmes et le nombre de tons de morphèmes}}}}\kern2pt]} \hypertarget{lo\string_MtA\string_M-lo\string_Mts£i\#\string_T1}{}
\markboth{\textcolor{darkblue}{\textbf{\ipa{lo˧tɑ˧-lo˧tɕi\#˥}}}}{}
\textcolor{teal}{\mytextsc{nom}} \hspace{4pt} Ton~: \#H.
\textcolor{PineGreen}{\selectlanguage{french}Drapeau de prières.} \zh{经幡、风马旗(挂在山上)。}  Emprunt~: tibétain  rlung rta
 ¶ \textcolor{darkblue}{\textbf{\ipa{lo˧tɑ˧-lo˧tɕi˧ | le˧-lɑ˧˥}}} \textcolor{PineGreen}{\selectlanguage{french}imprimer un drapeau de prières; sens plus général: confectionner un drapeau de prières (chez soi, avant de se rendre sur le lieu où on l'installe)} \zh{直译:印出一个经幡。也来指准备经幡的工作(到山上去挂之前)}  
 \zh{量词}: \textcolor{darkblue}{\textbf{\ipa{pɤ˥}}} \textcolor{darkblue}{\textbf{\ipa{pʰæ˧˥}}}  \mytextsc{clf}~: \textcolor{darkblue}{\textbf{\ipa{pɤ˥}}} \textcolor{darkblue}{\textbf{\ipa{pʰæ˧˥}}} 
\lhead{\firstmark}
\rhead{\botmark}

\subsection{\hspace{-0.5cm} {\Large \textcolor{darkblue}{\textbf{\ipa{lo˩}}} \textsubscript{1}}\hspace{0.5cm}[\kern2pt{\textcolor{darkblue}{\textbf{\ipa{lo˩˥}}}}\kern2pt]} \hypertarget{lo\string_B1}{}
\markboth{\textcolor{darkblue}{\textbf{\ipa{lo˩}}} \textsubscript{1}}{}
\textcolor{teal}{\mytextsc{verbe}} \hspace{4pt} Ton~: L.
\textcolor{PineGreen}{\selectlanguage{french}Passer, franchir (un col).} \zh{过(垭口)。}  ¶ \textcolor{darkblue}{\textbf{\ipa{mv̩˩tɕo˧-lo˩}}} \textcolor{PineGreen}{\selectlanguage{french}descendre (après avoir passé un col)} \zh{往下过去(过了垭口以后)}  

\lhead{\firstmark}
\rhead{\botmark}

\subsection{\hspace{-0.5cm} {\Large \textcolor{darkblue}{\textbf{\ipa{lo˩}}} \textsubscript{2}}\hspace{0.5cm}[\kern2pt{\textcolor{darkblue}{\textbf{\ipa{lo˥}}}}\kern2pt]} \hypertarget{lo\string_B2}{}
\markboth{\textcolor{darkblue}{\textbf{\ipa{lo˩}}} \textsubscript{2}}{}
\textcolor{teal}{\mytextsc{nom}} \hspace{4pt} Ton~: L.
\textcolor{PineGreen}{\selectlanguage{french}Vallée de montagne.} \zh{山谷。}  ¶ \textcolor{darkblue}{\textbf{\ipa{lo˧-qo˧}}} \textcolor{PineGreen}{\selectlanguage{french}dans la vallée} \zh{山谷里}  
 \zh{量词}: \textcolor{darkblue}{\textbf{\ipa{lo˩}}}  \mytextsc{clf}~: \textcolor{darkblue}{\textbf{\ipa{lo˩}}} 
\lhead{\firstmark}
\rhead{\botmark}

\subsection{\hspace{-0.5cm} {\Large \textcolor{darkblue}{\textbf{\ipa{lo˩\textsubscript{b}}}}}\hspace{0.5cm}[\kern2pt{\textcolor{darkblue}{\textbf{\ipa{lo˥}}}}\kern2pt]} \hypertarget{lo\string_Bb1}{}
\markboth{\textcolor{darkblue}{\textbf{\ipa{lo˩\textsubscript{b}}}}}{}
\textcolor{teal}{\mytextsc{classificateur}} \hspace{4pt} Ton~: L\textsubscript{b}.
\textcolor{PineGreen}{\selectlanguage{french}Classificateur des vallées.} \zh{量词:谷。}  ¶ \textcolor{darkblue}{\textbf{\ipa{hĩ˧-ɻ̃˧ | ɖɯ˧-lo˩}}} \textcolor{PineGreen}{\selectlanguage{french}tous les membres d'une grande famille: '[toute la population d']une vallée'} \zh{住在一座山谷里的所有人(直译:‘一山谷的人’)}  
 ¶ \textcolor{darkblue}{\textbf{\ipa{si˧dzi˩ | ɖɯ˧-lo˩}}} \textcolor{PineGreen}{\selectlanguage{french}une grande quantité d'arbres: 'une vallée [couverte] d'arbres'} \zh{一山谷的树,一片森林(直译:‘一山谷的树’)}  

\lhead{\firstmark}
\rhead{\botmark}

\subsection{\hspace{-0.5cm} {\Large \textcolor{darkblue}{\textbf{\ipa{lo˩bɤ˩}}}}\hspace{0.5cm}[\kern2pt{\textcolor{darkblue}{\textbf{\ipa{lo˧bɤ˧˥}}}}\kern2pt]} \hypertarget{lo\string_Bb7\string_B1}{}
\markboth{\textcolor{darkblue}{\textbf{\ipa{lo˩bɤ˩}}}}{}
\textcolor{teal}{\mytextsc{nom}} \hspace{4pt} Ton~: L.
\textcolor{PineGreen}{\selectlanguage{french}Paume.} \zh{手掌。}  \zh{量词}: \textcolor{darkblue}{\textbf{\ipa{ɭɯ˧}}}  \mytextsc{clf}~: \textcolor{darkblue}{\textbf{\ipa{ɭɯ˧}}} 
\lhead{\firstmark}
\rhead{\botmark}

\subsection{\hspace{-0.5cm} {\Large \textcolor{darkblue}{\textbf{\ipa{lo˩bv̩˧-ɭɯ˩}}}}\hspace{0.5cm}[\kern2pt{\textcolor{darkblue}{\textbf{\ipa{xxxx non-correspondance entre le nombre de morphèmes et le nombre de tons de morphèmes}}}}\kern2pt]} \hypertarget{lo\string_Bbv\string_=\string_M-l\string_RM\string_B1}{}
\markboth{\textcolor{darkblue}{\textbf{\ipa{lo˩bv̩˧-ɭɯ˩}}}}{}
\textcolor{teal}{\mytextsc{nom}} \hspace{4pt} Ton~: LM-L.
\textcolor{PineGreen}{\selectlanguage{french}Partie saillante du coude, qd le bras est replié.} \zh{肘。}  \zh{量词}: \textcolor{darkblue}{\textbf{\ipa{ɭɯ˧}}}  \mytextsc{clf}~: \textcolor{darkblue}{\textbf{\ipa{ɭɯ˧}}} 
\lhead{\firstmark}
\rhead{\botmark}

\subsection{\hspace{-0.5cm} {\Large \textcolor{darkblue}{\textbf{\ipa{lo˩dv̩\#˥}}}}\hspace{0.5cm}[\kern2pt{\textcolor{darkblue}{\textbf{\ipa{lo˩dv̩˥}}}}\kern2pt]} \hypertarget{lo\string_Bdv\string_=\#\string_T1}{}
\markboth{\textcolor{darkblue}{\textbf{\ipa{lo˩dv̩\#˥}}}}{}
\textcolor{teal}{\mytextsc{nom}} \hspace{4pt} Ton~: LM+\#H.
\textcolor{PineGreen}{\selectlanguage{french}Manchot.} \zh{独臂人:缺一只胳膊(手)的人。}  \zh{量词}: \textcolor{darkblue}{\textbf{\ipa{v̩˧}}}  \mytextsc{clf}~: \textcolor{darkblue}{\textbf{\ipa{v̩˧}}} 
\lhead{\firstmark}
\rhead{\botmark}

\subsection{\hspace{-0.5cm} {\Large \textcolor{darkblue}{\textbf{\ipa{lo˩dzi˩}}}}\hspace{0.5cm}[\kern2pt{\textcolor{darkblue}{\textbf{\ipa{lo˩dzi˥}}}}\kern2pt]} \hypertarget{lo\string_Bdzi\string_B1}{}
\markboth{\textcolor{darkblue}{\textbf{\ipa{lo˩dzi˩}}}}{}
\textcolor{teal}{\mytextsc{classificateur}} \hspace{4pt} Ton~: L.
\textcolor{PineGreen}{\selectlanguage{french}Classificateur des poignées (à deux mains).} \zh{量词:捧(用两只手)。} 
\lhead{\firstmark}
\rhead{\botmark}

\subsection{\hspace{-0.5cm} {\Large \textcolor{darkblue}{\textbf{\ipa{lo˩dʑo˥}}}}\hspace{0.5cm}[\kern2pt{\textcolor{darkblue}{\textbf{\ipa{lo˩dʑo˩˥}}}}\kern2pt]} \hypertarget{lo\string_Bdz£o\string_T1}{}
\markboth{\textcolor{darkblue}{\textbf{\ipa{lo˩dʑo˥}}}}{}
\textcolor{teal}{\mytextsc{nom}} \hspace{4pt} Ton~: LH.
\textcolor{PineGreen}{\selectlanguage{french}Bracelet.} \zh{手镯。}  ¶ \textcolor{darkblue}{\textbf{\ipa{ŋv̩˩-lo˩dʑo˧ (+ɲi˩)}}} \textcolor{PineGreen}{\selectlanguage{french}bracelet en argent} \zh{银手镯}  
 ¶ \textcolor{darkblue}{\textbf{\ipa{hæ̃˩-lo˩dʑo˥ (+ɲi˩)}}} \textcolor{PineGreen}{\selectlanguage{french}bracelet en or} \zh{金手镯}  
 ¶ \textcolor{darkblue}{\textbf{\ipa{jo˧-lo˥dʑo˩}}} \textcolor{PineGreen}{\selectlanguage{french}bracelet en jade} \zh{玉手镯}  
 ¶ \textcolor{darkblue}{\textbf{\ipa{lo˩dʑo˥ kʰɯ˩}}} \textcolor{PineGreen}{\selectlanguage{french}mettre un bracelet} \zh{戴上手镯}  
 \zh{量词}: \textcolor{darkblue}{\textbf{\ipa{pʰo˧˥}}}  \mytextsc{clf}~: \textcolor{darkblue}{\textbf{\ipa{pʰo˧˥}}} 
\lhead{\firstmark}
\rhead{\botmark}

\subsection{\hspace{-0.5cm} {\Large \textcolor{darkblue}{\textbf{\ipa{lo˩ɖɯ˧}}}}\hspace{0.5cm}[\kern2pt{\textcolor{darkblue}{\textbf{\ipa{xxxx non-correspondance entre le nombre de morphèmes et le nombre de tons de morphèmes}}}}\kern2pt]} \hypertarget{lo\string_Bd`M\string_M1}{}
\markboth{\textcolor{darkblue}{\textbf{\ipa{lo˩ɖɯ˧}}}}{}
\textcolor{teal}{\mytextsc{adjectif}} \hspace{4pt} Ton~: LM.
\textcolor{PineGreen}{\selectlanguage{french}Généreux.} \zh{大方。} 
\lhead{\firstmark}
\rhead{\botmark}

\subsection{\hspace{-0.5cm} {\Large \textcolor{darkblue}{\textbf{\ipa{lo˩-gv̩˧dv̩˧}}}}\hspace{0.5cm}[\kern2pt{\textcolor{darkblue}{\textbf{\ipa{lo˧gv̩˧dv̩˧}}}}\kern2pt]} \hypertarget{lo\string_B-gv\string_=\string_Mdv\string_=\string_M1}{}
\markboth{\textcolor{darkblue}{\textbf{\ipa{lo˩-gv̩˧dv̩˧}}}}{}
\textcolor{teal}{\mytextsc{nom}} \hspace{4pt} Ton~: L-.
\textcolor{PineGreen}{\selectlanguage{french}Dos de la main.} \zh{手背。}  \zh{量词}: \textcolor{darkblue}{\textbf{\ipa{kʰwɤ˥}}}  \mytextsc{clf}~: \textcolor{darkblue}{\textbf{\ipa{kʰwɤ˥}}} 
\lhead{\firstmark}
\rhead{\botmark}

\subsection{\hspace{-0.5cm} {\Large \textcolor{darkblue}{\textbf{\ipa{lo˩jɤ˧}}}}\hspace{0.5cm}[\kern2pt{\textcolor{darkblue}{\textbf{\ipa{lo˩jɤ˥}}}}\kern2pt]} \hypertarget{lo\string_Bj7\string_M1}{}
\markboth{\textcolor{darkblue}{\textbf{\ipa{lo˩jɤ˧}}}}{}
\textcolor{teal}{\mytextsc{nom}} \hspace{4pt} Ton~: LM.
\textcolor{PineGreen}{\selectlanguage{french}Pièce d'argent.} \zh{银元。}  ¶ \textcolor{darkblue}{\textbf{\ipa{lo˩jɤ˧ | ɖɯ˧-pʰæ˧˥}}} \textcolor{PineGreen}{\selectlanguage{french}une pièce d'argent} \zh{一块银元}  

\lhead{\firstmark}
\rhead{\botmark}

\subsection{\hspace{-0.5cm} {\Large \textcolor{darkblue}{\textbf{\ipa{lo˩ko˧}}}}\hspace{0.5cm}[\kern2pt{\textcolor{darkblue}{\textbf{\ipa{lo˩ko˥}}}}\kern2pt]} \hypertarget{lo\string_Bko\string_M1}{}
\markboth{\textcolor{darkblue}{\textbf{\ipa{lo˩ko˧}}}}{}
\textcolor{teal}{\mytextsc{nom}} \hspace{4pt} Ton~: LM.
\textcolor{PineGreen}{\selectlanguage{french}Casserole, pour cuire les céréales, les légumes, les soupes... Elle était autrefois en cuivre.} \zh{煮饭或煮汤的锣锅。在过去,锣锅一般是铜做的。}  Emprunt~: chinois  \zh{锣锅}
 ¶ \textcolor{darkblue}{\textbf{\ipa{lo˩ko˧: | hɑ˧ tɕɤ˩-di˩! | æ˧-v̩˧, | dʑɯ˩-kʰɯ˩-di˩˥! | ʈʂʰɤ˧ho˥, | dʑɯ˩ tɕɯ˩-di˩˥! |}}} \textcolor{PineGreen}{\selectlanguage{french}La casserole (\textcolor{darkblue}{\textbf{\ipa{/lo˩ko˧/}}}), ça sert à cuire la nourriture! La casserole de cuivre (\textcolor{darkblue}{\textbf{\ipa{/æ˧-v̩˧/}}}), ça sert à mettre de l'eau! La bouilloire (\textcolor{darkblue}{\textbf{\ipa{/ʈʂʰɤ˩ho˥/}}}), ça sert à faire bouillir l'eau! (Résumé des emplois des trois sortes de casseroles qui étaient en usage à Yongning vers le milieu du XXe siècle.)} \zh{锣锅,是用来煮饭的!铜锅,是放水用的!水壶,是来煮水的!(描写永宁二十世纪中使用的三种锅)}  
 \zh{量词}: \textcolor{darkblue}{\textbf{\ipa{ɭɯ˧}}}  \mytextsc{clf}~: \textcolor{darkblue}{\textbf{\ipa{ɭɯ˧}}} 
\lhead{\firstmark}
\rhead{\botmark}

\subsection{\hspace{-0.5cm} {\Large \textcolor{darkblue}{\textbf{\ipa{lo˩mi˧}}}}\hspace{0.5cm}[\kern2pt{\textcolor{darkblue}{\textbf{\ipa{lo˩mi˥}}}}\kern2pt]} \hypertarget{lo\string_Bmi\string_M1}{}
\markboth{\textcolor{darkblue}{\textbf{\ipa{lo˩mi˧}}}}{}
\textcolor{teal}{\mytextsc{nom}} \hspace{4pt} Ton~: LM.
\textcolor{PineGreen}{\selectlanguage{french}Pouce.} \zh{大拇指。}  \zh{量词}: \textcolor{darkblue}{\textbf{\ipa{ɭɯ˧}}}  \mytextsc{clf}~: \textcolor{darkblue}{\textbf{\ipa{ɭɯ˧}}} 
\lhead{\firstmark}
\rhead{\botmark}

\subsection{\hspace{-0.5cm} {\Large \textcolor{darkblue}{\textbf{\ipa{lo˩mi˧-qɑ˩}}}}\hspace{0.5cm}[\kern2pt{\textcolor{darkblue}{\textbf{\ipa{lo˩mi˧qɑ˧}}}}\kern2pt]} \hypertarget{lo\string_Bmi\string_M-qA\string_B1}{}
\markboth{\textcolor{darkblue}{\textbf{\ipa{lo˩mi˧-qɑ˩}}}}{}
\textcolor{teal}{\mytextsc{nom}} \hspace{4pt} Ton~: LM-L.
\textcolor{PineGreen}{\selectlanguage{french}Espace entre le pouce et l'index.} \zh{虎口。}  \zh{量词}: \textcolor{darkblue}{\textbf{\ipa{ɭɯ˧}}}  \mytextsc{clf}~: \textcolor{darkblue}{\textbf{\ipa{ɭɯ˧}}} 
\lhead{\firstmark}
\rhead{\botmark}

\subsection{\hspace{-0.5cm} {\Large \textcolor{darkblue}{\textbf{\ipa{lo˩pv̩˧˥}}}}\hspace{0.5cm}[\kern2pt{\textcolor{darkblue}{\textbf{\ipa{lo˩pv̩˧˥}}}}\kern2pt]} \hypertarget{lo\string_Bpv\string_=\string_M\string_T1}{}
\markboth{\textcolor{darkblue}{\textbf{\ipa{lo˩pv̩˧˥}}}}{}
\textcolor{teal}{\mytextsc{nom}} \hspace{4pt} Ton~: LM+MH\#.
\textcolor{PineGreen}{\selectlanguage{french}Anneau.} \zh{戒指。}  ¶ \textcolor{darkblue}{\textbf{\ipa{ŋv̩˩-lo˩pv̩˩}}} \textcolor{PineGreen}{\selectlanguage{french}anneau en argent} \zh{银戒指}  
 ¶ \textcolor{darkblue}{\textbf{\ipa{hæ̃˩-lo˩pv̩˩}}} \textcolor{PineGreen}{\selectlanguage{french}anneau en or} \zh{金戒指}  
 \zh{量词}: \textcolor{darkblue}{\textbf{\ipa{ɭɯ˧}}}  \mytextsc{clf}~: \textcolor{darkblue}{\textbf{\ipa{ɭɯ˧}}} 
\lhead{\firstmark}
\rhead{\botmark}

\subsection{\hspace{-0.5cm} {\Large \textcolor{darkblue}{\textbf{\ipa{lo˩qʰv̩˩}}}}\hspace{0.5cm}[\kern2pt{\textcolor{darkblue}{\textbf{\ipa{lo˩qʰv̩˩˥}}}}\kern2pt]} \hypertarget{lo\string_Bq\string_hv\string_=\string_B1}{}
\markboth{\textcolor{darkblue}{\textbf{\ipa{lo˩qʰv̩˩}}}}{}
\textcolor{teal}{\mytextsc{nom}} \hspace{4pt} Ton~: L.
\textcolor{PineGreen}{\selectlanguage{french}Vallée, gorge, ravin.} \zh{山沟。}  \zh{量词}: \textcolor{darkblue}{\textbf{\ipa{lo˩}}}  \mytextsc{clf}~: \textcolor{darkblue}{\textbf{\ipa{lo˩}}} 
\lhead{\firstmark}
\rhead{\botmark}

\subsection{\hspace{-0.5cm} {\Large \textcolor{darkblue}{\textbf{\ipa{lo˩qʰwɤ˧}}}}\hspace{0.5cm}[\kern2pt{\textcolor{darkblue}{\textbf{\ipa{lo˩qʰwɤ˥}}}}\kern2pt]} \hypertarget{lo\string_Bq\string_hw7\string_M1}{}
\markboth{\textcolor{darkblue}{\textbf{\ipa{lo˩qʰwɤ˧}}}}{}
\textcolor{teal}{\mytextsc{nom}} \hspace{4pt} Ton~: LM.
\ding{202} \textcolor{PineGreen}{\selectlanguage{french}Bras.} \zh{胳膊。}  ¶ \textcolor{darkblue}{\textbf{\ipa{lo˩qʰwɤ˧ li˧}}} \textcolor{PineGreen}{\selectlanguage{french}regarder le bras} \zh{看胳膊}  
 \zh{量词}: \textcolor{darkblue}{\textbf{\ipa{pʰo˧˥}}} \ding{203} \textcolor{PineGreen}{\selectlanguage{french}Main.} \zh{手。}  ¶ \textcolor{darkblue}{\textbf{\ipa{lo˩qʰwɤ˧ ʈʂʰæ˧}}} \textcolor{PineGreen}{\selectlanguage{french}se laver les mains} \zh{洗手}  
 \mytextsc{clf}~: \textcolor{darkblue}{\textbf{\ipa{pʰo˧˥}}} 
\lhead{\firstmark}
\rhead{\botmark}

\subsection{\hspace{-0.5cm} {\Large \textcolor{darkblue}{\textbf{\ipa{lo˩qʰwɤ˧-kʰɯ˧ʑi˧˥}}}}\hspace{0.5cm}[\kern2pt{\textcolor{darkblue}{\textbf{\ipa{xxxx non-correspondance entre le nombre de morphèmes et le nombre de tons de morphèmes}}}}\kern2pt]} \hypertarget{lo\string_Bq\string_hw7\string_M-k\string_hM\string_Mz£i\string_M\string_T1}{}
\markboth{\textcolor{darkblue}{\textbf{\ipa{lo˩qʰwɤ˧-kʰɯ˧ʑi˧˥}}}}{}
\textcolor{teal}{\mytextsc{nom}} \hspace{4pt} Ton~: LM+MH\#.
\textcolor{PineGreen}{\selectlanguage{french}Gant.} \zh{手套。} 
\lhead{\firstmark}
\rhead{\botmark}

\subsection{\hspace{-0.5cm} {\Large \textcolor{darkblue}{\textbf{\ipa{lo˩ʁwæ\#˥}}}}\hspace{0.5cm}[\kern2pt{\textcolor{darkblue}{\textbf{\ipa{lo˩ʁwæ˥}}}}\kern2pt]} \hypertarget{lo\string_BRw\{\#\string_T1}{}
\markboth{\textcolor{darkblue}{\textbf{\ipa{lo˩ʁwæ\#˥}}}}{}
\textcolor{teal}{\mytextsc{nom}} \hspace{4pt} Ton~: LM+\#H.
\textcolor{PineGreen}{\selectlanguage{french}Gaucher.} \zh{左撇子。} 
\lhead{\firstmark}
\rhead{\botmark}

\subsection{\hspace{-0.5cm} {\Large \textcolor{darkblue}{\textbf{\ipa{lo˩tʰo˧}}}}\hspace{0.5cm}[\kern2pt{\textcolor{darkblue}{\textbf{\ipa{lo˩tʰo˥}}}}\kern2pt]} \hypertarget{lo\string_Bt\string_ho\string_M1}{}
\markboth{\textcolor{darkblue}{\textbf{\ipa{lo˩tʰo˧}}}}{}
\textcolor{teal}{\mytextsc{nom}} \hspace{4pt} Ton~: LM.
\textcolor{PineGreen}{\selectlanguage{french}Menottes: chaîne de fer pour attacher les mains d'un criminel.} \zh{手铐。}  ¶ \textcolor{darkblue}{\textbf{\ipa{lo˩tʰo˧ kʰɯ˧˥}}} \textcolor{PineGreen}{\selectlanguage{french}passer les menottes à quelqu'un} \zh{戴上手铐}  

\lhead{\firstmark}
\rhead{\botmark}

\subsection{\hspace{-0.5cm} {\Large \textcolor{darkblue}{\textbf{\ipa{lo˩tsʰɯ˥-sɑ˩}}}}\hspace{0.5cm}[\kern2pt{\textcolor{darkblue}{\textbf{\ipa{lo˩tsʰɯ˥sɑ˧}}}}\kern2pt]} \hypertarget{lo\string_Bts\string_hM\string_T-sA\string_B1}{}
\markboth{\textcolor{darkblue}{\textbf{\ipa{lo˩tsʰɯ˥-sɑ˩}}}}{}
\textcolor{teal}{\mytextsc{nom}} \hspace{4pt} Ton~: LH-.
\textcolor{PineGreen}{\selectlanguage{french}Viande des membres antérieurs.} \zh{牲畜前腿的肉。} 
\lhead{\firstmark}
\rhead{\botmark}

\subsection{\hspace{-0.5cm} {\Large \textcolor{darkblue}{\textbf{\ipa{lo˩ʈv̩˧}}}}\hspace{0.5cm}[\kern2pt{\textcolor{darkblue}{\textbf{\ipa{lo˩ʈv̩˥}}}}\kern2pt]} \hypertarget{lo\string_Bt`v\string_=\string_M1}{}
\markboth{\textcolor{darkblue}{\textbf{\ipa{lo˩ʈv̩˧}}}}{}
\textcolor{teal}{\mytextsc{nom}} \hspace{4pt} Ton~: LM.
\textcolor{PineGreen}{\selectlanguage{french}Poing.} \zh{拳。}  \zh{量词}: \textcolor{darkblue}{\textbf{\ipa{ʈv̩˩}}}  \mytextsc{clf}~: \textcolor{darkblue}{\textbf{\ipa{ʈv̩˩}}} 
\lhead{\firstmark}
\rhead{\botmark}

\subsection{\hspace{-0.5cm} {\Large \textcolor{darkblue}{\textbf{\ipa{lo˩ʈʰɯ˧}}}}\hspace{0.5cm}[\kern2pt{\textcolor{darkblue}{\textbf{\ipa{lo˩ʈʰɯ˥}}}}\kern2pt]} \hypertarget{lo\string_Bt`\string_hM\string_M1}{}
\markboth{\textcolor{darkblue}{\textbf{\ipa{lo˩ʈʰɯ˧}}}}{}
\textcolor{teal}{\mytextsc{nom}} \hspace{4pt} Ton~: LM.
\textcolor{PineGreen}{\selectlanguage{french}Coude.} \zh{肘。}  \zh{量词}: \textcolor{darkblue}{\textbf{\ipa{ʈv̩˩}}}  \mytextsc{clf}~: \textcolor{darkblue}{\textbf{\ipa{ʈv̩˩}}} 
\lhead{\firstmark}
\rhead{\botmark}

\subsection{\hspace{-0.5cm} {\Large \textcolor{darkblue}{\textbf{\ipa{lo˩ʈʂæ˧˥}}}}\hspace{0.5cm}[\kern2pt{\textcolor{darkblue}{\textbf{\ipa{lo˩ʈʂæ˧˥}}}}\kern2pt]} \hypertarget{lo\string_Bt`s`\{\string_M\string_T1}{}
\markboth{\textcolor{darkblue}{\textbf{\ipa{lo˩ʈʂæ˧˥}}}}{}
\textcolor{teal}{\mytextsc{nom}} \hspace{4pt} Ton~: LM+MH\#.
\textcolor{PineGreen}{\selectlanguage{french}Articulations du bras: le poignet, mais aussi le coude.} \zh{手臂的关节(手腕、肘弯)。}  \zh{量词}: \textcolor{darkblue}{\textbf{\ipa{ʈʂæ˧˥}}}  \mytextsc{clf}~: \textcolor{darkblue}{\textbf{\ipa{ʈʂæ˧˥}}} 
\lhead{\firstmark}
\rhead{\botmark}

\subsection{\hspace{-0.5cm} {\Large \textcolor{darkblue}{\textbf{\ipa{lo˧˥}}} \textsubscript{1}}\hspace{0.5cm}[\kern2pt{\textcolor{darkblue}{\textbf{\ipa{lo˥}}}}\kern2pt]} \hypertarget{lo\string_M\string_T1}{}
\markboth{\textcolor{darkblue}{\textbf{\ipa{lo˧˥}}} \textsubscript{1}}{}
\textcolor{teal}{\mytextsc{adjectif}} \hspace{4pt} Ton~: MH.
\textcolor{PineGreen}{\selectlanguage{french}Épais.} \zh{厚。}  ¶ \textcolor{darkblue}{\textbf{\ipa{ʈʂʰɯ˧ | lo˧-pæ˧-ɻæ˥-gv̩˩!}}} \textcolor{PineGreen}{\selectlanguage{french}c'est très épais!} \zh{很厚啊!}  

\lhead{\firstmark}
\rhead{\botmark}

\subsection{\hspace{-0.5cm} {\Large \textcolor{darkblue}{\textbf{\ipa{lo˧˥}}} \textsubscript{2}}\hspace{0.5cm}[\kern2pt{\textcolor{darkblue}{\textbf{\ipa{lo˧˥}}}}\kern2pt]} \hypertarget{lo\string_M\string_T2}{}
\markboth{\textcolor{darkblue}{\textbf{\ipa{lo˧˥}}} \textsubscript{2}}{}
\textcolor{teal}{\mytextsc{verbe}} \hspace{4pt} Ton~: MH.
\textcolor{PineGreen}{\selectlanguage{french}Joindre les mains en signe de soumission.} \zh{拱手作揖。}  ¶ \textcolor{darkblue}{\textbf{\ipa{tsʰɤ˧tsʰɤ˧ lo˧˥}}} \textcolor{PineGreen}{\selectlanguage{french}rendre hommage à, joindre les mains en signe de soumission/respect} \zh{拱手作揖}  
 ¶ \textcolor{darkblue}{\textbf{\ipa{tsʰɤ˧tsʰɤ˧ | le˧-lo˧-ze˥}}} \textcolor{PineGreen}{\selectlanguage{french}\mytextsc{accomp} \string_ \mytextsc{pfv}} \zh{\mytextsc{accomp} \string_ \mytextsc{pfv}}  

\lhead{\firstmark}
\rhead{\botmark}

\subsection{\hspace{-0.5cm} {\Large \textcolor{darkblue}{\textbf{\ipa{*lo˩˧}}}}\hspace{0.5cm}[\kern2pt{\textcolor{darkblue}{\textbf{\ipa{lo˩˥}}}}\kern2pt]} \hypertarget{*lo\string_B\string_M1}{}
\markboth{\textcolor{darkblue}{\textbf{\ipa{*lo˩˧}}}}{}
\textcolor{teal}{\mytextsc{nom}} \hspace{4pt} Ton~: LM.
\textcolor{PineGreen}{\selectlanguage{french}Pouce (forme reconstruite d'après le disyllabe).} \zh{大拇指(单音节,按照双音节词构拟出来的)。} 
\lhead{\firstmark}
\rhead{\botmark}

\subsection{\hspace{-0.5cm} {\Large \textcolor{darkblue}{\textbf{\ipa{lv̩˧}}} \textsubscript{1}}\hspace{0.5cm}[\kern2pt{\textcolor{darkblue}{\textbf{\ipa{lv̩˥}}}}\kern2pt]} \hypertarget{lv\string_=\string_M1}{}
\markboth{\textcolor{darkblue}{\textbf{\ipa{lv̩˧}}} \textsubscript{1}}{}
\textcolor{teal}{\mytextsc{nom}} \hspace{4pt} Ton~: M.
\textcolor{PineGreen}{\selectlanguage{french}Champs.} \zh{田地。}  \zh{量词}: \textcolor{darkblue}{\textbf{\ipa{kɤ˧˥}}}  \mytextsc{clf}~: \textcolor{darkblue}{\textbf{\ipa{kɤ˧˥}}} 
\lhead{\firstmark}
\rhead{\botmark}

\subsection{\hspace{-0.5cm} {\Large \textcolor{darkblue}{\textbf{\ipa{lv̩˧}}} \textsubscript{2}}\hspace{0.5cm}[\kern2pt{\textcolor{darkblue}{\textbf{\ipa{lv̩˥}}}}\kern2pt]} \hypertarget{lv\string_=\string_M2}{}
\markboth{\textcolor{darkblue}{\textbf{\ipa{lv̩˧}}} \textsubscript{2}}{}
\textcolor{teal}{\mytextsc{nom}} \hspace{4pt} Ton~: M.
\textcolor{PineGreen}{\selectlanguage{french}Grain (pour chevaux ou vaches), picotin.} \zh{喂给马或牛的粮食。}  ¶ \textcolor{darkblue}{\textbf{\ipa{ʐwæ˧-lv̩˧}}} \textcolor{PineGreen}{\selectlanguage{french}grain pour pour cheval, picotin; même sens que: \textcolor{darkblue}{\textbf{\ipa{/ʐwæ˧-ɭɯ\#˥/}}}} \zh{喂给马的粮食}  
\textit{Voir~:} \hyperlink{}{\textcolor{darkblue}{\textbf{\ipa{ʐwæ˧-ɭɯ\#˥}}}} 
\lhead{\firstmark}
\rhead{\botmark}

\subsection{\hspace{-0.5cm} {\Large \textcolor{darkblue}{\textbf{\ipa{lv̩˧˥}}}}\hspace{0.5cm}[\kern2pt{\textcolor{darkblue}{\textbf{\ipa{lv̩˧˥}}}}\kern2pt]} \hypertarget{lv\string_=\string_M\string_T1}{}
\markboth{\textcolor{darkblue}{\textbf{\ipa{lv̩˧˥}}}}{}
\textcolor{teal}{\mytextsc{nom}} \hspace{4pt} Ton~: MH.
\textcolor{PineGreen}{\selectlanguage{french}Larve.} \zh{蛆。} 
\lhead{\firstmark}
\rhead{\botmark}

\subsection{\hspace{-0.5cm} {\Large \textcolor{darkblue}{\textbf{\ipa{lv̩˧˥}}} \textsubscript{1}}\hspace{0.5cm}[\kern2pt{\textcolor{darkblue}{\textbf{\ipa{lv̩˧˥}}}}\kern2pt]} \hypertarget{lv\string_=\string_M\string_T1}{}
\markboth{\textcolor{darkblue}{\textbf{\ipa{lv̩˧˥}}} \textsubscript{1}}{}
\textcolor{teal}{\mytextsc{verbe}} \hspace{4pt} Ton~: MH.
\textcolor{PineGreen}{\selectlanguage{french}Garder les animaux, mener paître les animaux.} \zh{放牧。}  ¶ \textcolor{darkblue}{\textbf{\ipa{go˩bo˥ lv̩˩}}} \textcolor{PineGreen}{\selectlanguage{french}mener paître le bétail, garder le bétail} \zh{放牧牲畜}  
 ¶ \textcolor{darkblue}{\textbf{\ipa{ʐwæ˧ lv̩˩}}} \textcolor{PineGreen}{\selectlanguage{french}mener paître les chevaux} \zh{放马}  
 ¶ \textcolor{darkblue}{\textbf{\ipa{ʝi˧ lv̩˩}}} \textcolor{PineGreen}{\selectlanguage{french}mener paître les vaches} \zh{放牛}  
 ¶ \textcolor{darkblue}{\textbf{\ipa{bo˩ lv̩˩˥}}} \textcolor{PineGreen}{\selectlanguage{french}garder les cochons} \zh{放猪}  
 ¶ \textcolor{darkblue}{\textbf{\ipa{tsʰɯ˧ lv̩˥}}} \textcolor{PineGreen}{\selectlanguage{french}mener paître les chèvres} \zh{放山羊}  
 ¶ \textcolor{darkblue}{\textbf{\ipa{ɖɯ˧-hɤ˧ mɤ˧-lv̩˩\textasciitilde{}lv̩˩}}} \textcolor{PineGreen}{\selectlanguage{french}paresseux, qui ne s'occupe de rien} \zh{懒,什么也不管}  

\lhead{\firstmark}
\rhead{\botmark}

\subsection{\hspace{-0.5cm} {\Large \textcolor{darkblue}{\textbf{\ipa{lv̩˧˥}}} \textsubscript{2}}\hspace{0.5cm}[\kern2pt{\textcolor{darkblue}{\textbf{\ipa{lv̩˧˥}}}}\kern2pt]} \hypertarget{lv\string_=\string_M\string_T2}{}
\markboth{\textcolor{darkblue}{\textbf{\ipa{lv̩˧˥}}} \textsubscript{2}}{}
\textcolor{teal}{\mytextsc{verbe}} \hspace{4pt} Ton~: MH.
\textcolor{PineGreen}{\selectlanguage{french}S'enfuir.} \zh{逃跑,逃掉。} 
\lhead{\firstmark}
\rhead{\botmark}

\subsection{\hspace{-0.5cm} {\Large \textcolor{darkblue}{\textbf{\ipa{lv̩˥}}}}\hspace{0.5cm}[\kern2pt{\textcolor{darkblue}{\textbf{\ipa{lv̩˥}}}}\kern2pt]} \hypertarget{lv\string_=\string_T1}{}
\markboth{\textcolor{darkblue}{\textbf{\ipa{lv̩˥}}}}{}
\textcolor{teal}{\mytextsc{verbe}} \hspace{4pt} Ton~: H.
\textcolor{PineGreen}{\selectlanguage{french}Enrouler (du fil); emballer.} \zh{缠(线……)、裹(毡子……)。}  ¶ \textcolor{darkblue}{\textbf{\ipa{le˧-qo˥-lv̩˩}}} \textcolor{PineGreen}{\selectlanguage{french}enrouler} \zh{裹起来}  
 ¶ \textcolor{darkblue}{\textbf{\ipa{kʰɯ˧ qo˧-lv̩˥}}} \textcolor{PineGreen}{\selectlanguage{french}enrouler du fil} \zh{缠线}  
 ¶ \textcolor{darkblue}{\textbf{\ipa{qo˧-lv̩˩}}} \textcolor{PineGreen}{\selectlanguage{french}même sens: enrouler} \zh{裹}  

\lhead{\firstmark}
\rhead{\botmark}

\subsection{\hspace{-0.5cm} {\Large \textcolor{darkblue}{\textbf{\ipa{lv̩˩\textsubscript{a}}}} \textsubscript{1}}\hspace{0.5cm}[\kern2pt{\textcolor{darkblue}{\textbf{\ipa{lv̩˧˥}}}}\kern2pt]} \hypertarget{lv\string_=\string_Ba1}{}
\markboth{\textcolor{darkblue}{\textbf{\ipa{lv̩˩\textsubscript{a}}}} \textsubscript{1}}{}
\textcolor{teal}{\mytextsc{verbe}} \hspace{4pt} Ton~: L\textsubscript{a}.
\textcolor{PineGreen}{\selectlanguage{french}Aboyer.} \zh{狗吠。}  ¶ \textcolor{darkblue}{\textbf{\ipa{kʰv̩˩mi˩ lv̩˥ |}}} \textcolor{PineGreen}{\selectlanguage{french}le chien aboie} \zh{狗吠}  
 ¶ \textcolor{darkblue}{\textbf{\ipa{kʰv̩˩ lv̩˥-dʑo˩ |}}} \textcolor{PineGreen}{\selectlanguage{french}le chien est en train d'aboyer} \zh{狗在叫}  
 ¶ \textcolor{darkblue}{\textbf{\ipa{ɖɯ˧-lv̩˧\textasciitilde{}lv̩˥-ɻ̍˩}}} \textcolor{PineGreen}{\selectlanguage{french}\mytextsc{délimitatif} \string_ \mytextsc{red} \mytextsc{inchoatif}} \zh{叫一叫}  

\lhead{\firstmark}
\rhead{\botmark}

\subsection{\hspace{-0.5cm} {\Large \textcolor{darkblue}{\textbf{\ipa{lv̩˩\textsubscript{a}}}} \textsubscript{2}}\hspace{0.5cm}[\kern2pt{\textcolor{darkblue}{\textbf{\ipa{lv̩˩˥}}}}\kern2pt]} \hypertarget{lv\string_=\string_Ba2}{}
\markboth{\textcolor{darkblue}{\textbf{\ipa{lv̩˩\textsubscript{a}}}} \textsubscript{2}}{}
\textcolor{teal}{\mytextsc{verbe}} \hspace{4pt} Ton~: L\textsubscript{a}.
\textcolor{PineGreen}{\selectlanguage{french}Enrouler (un tissu).} \zh{把布卷起来。}  ¶ \textcolor{darkblue}{\textbf{\ipa{le˧-qæ˥-lv̩˩}}} \textcolor{PineGreen}{\selectlanguage{french}enrouler} \zh{卷起来}  
 ¶ \textcolor{darkblue}{\textbf{\ipa{le˧-lv̩˧\textasciitilde{}lv̩˧}}} \textcolor{PineGreen}{\selectlanguage{french}\mytextsc{accomp} \mytextsc{red}}  
 ¶ \textcolor{darkblue}{\textbf{\ipa{tso˧\textasciitilde{}tso˧ lv̩˧\textasciitilde{}lv̩˧}}} \textcolor{PineGreen}{\selectlanguage{french}enrouler des choses} \zh{卷东西}  
 ¶ \textcolor{darkblue}{\textbf{\ipa{ɖɯ˧-kʰwɤ˧ lv̩˥}}} \textcolor{PineGreen}{\selectlanguage{french}enrouler quelque chose} \zh{卷一块(东西)}  

\lhead{\firstmark}
\rhead{\botmark}

\subsection{\hspace{-0.5cm} {\Large \textcolor{darkblue}{\textbf{\ipa{lv̩˩\textsubscript{a}}}} \textsubscript{3}}\hspace{0.5cm}[\kern2pt{\textcolor{darkblue}{\textbf{\ipa{lv̩˩˥}}}}\kern2pt]} \hypertarget{lv\string_=\string_Ba3}{}
\markboth{\textcolor{darkblue}{\textbf{\ipa{lv̩˩\textsubscript{a}}}} \textsubscript{3}}{}
\textcolor{teal}{\mytextsc{verbe}} \hspace{4pt} Ton~: L\textsubscript{a}.
\textcolor{PineGreen}{\selectlanguage{french}Labourer.} \zh{耕种。}  ¶ \textcolor{darkblue}{\textbf{\ipa{le˧-lv̩˩-ze˩}}} \textcolor{PineGreen}{\selectlanguage{french}\mytextsc{accomp} \string_ \mytextsc{pfv}} \zh{耕种了}  
 ¶ \textcolor{darkblue}{\textbf{\ipa{ʝi˧-lv̩˧˥}}} \textcolor{PineGreen}{\selectlanguage{french}labourer} \zh{耕种}  
 ¶ \textcolor{darkblue}{\textbf{\ipa{dʑi˧mi˧ lv̩˧˥ / dʑi˧mi˧ lv̩˧-ze˥}}} \textcolor{PineGreen}{\selectlanguage{french}labourer avec un buffle} \zh{用水牛耕田}  
 ¶ \textcolor{darkblue}{\textbf{\ipa{ʝi˧ ɖɯ˧-lv̩˧\textasciitilde{}lv̩˥-ɻ̍˩}}} \textcolor{PineGreen}{\selectlanguage{french}labourer un peu} \zh{耕一耕}  

\lhead{\firstmark}
\rhead{\botmark}

\subsection{\hspace{-0.5cm} {\Large \textcolor{darkblue}{\textbf{\ipa{lv̩˩\textsubscript{a}}}} \textsubscript{4}}\hspace{0.5cm}[\kern2pt{\textcolor{darkblue}{\textbf{\ipa{lv̩˩˥}}}}\kern2pt]} \hypertarget{lv\string_=\string_Ba4}{}
\markboth{\textcolor{darkblue}{\textbf{\ipa{lv̩˩\textsubscript{a}}}} \textsubscript{4}}{}
\textcolor{teal}{\mytextsc{verbe}} \hspace{4pt} Ton~: L\textsubscript{a}.
\textcolor{PineGreen}{\selectlanguage{french}Suffire.} \zh{足够。}  ¶ \textcolor{darkblue}{\textbf{\ipa{ə˩-lv̩˩˥? / ə˩-lv̩˩-ze˥?}}} \textcolor{PineGreen}{\selectlanguage{french}est-ce que ça (te) suffit ?} \zh{够了吗?}  

\lhead{\firstmark}
\rhead{\botmark}

\subsection{\hspace{-0.5cm} {\Large \textcolor{darkblue}{\textbf{\ipa{lv̩˧dʑɯ˥}}}}\hspace{0.5cm}[\kern2pt{\textcolor{darkblue}{\textbf{\ipa{lv̩˧dʑɯ˧}}}}\kern2pt]} \hypertarget{lv\string_=\string_Mdz£M\string_T1}{}
\markboth{\textcolor{darkblue}{\textbf{\ipa{lv̩˧dʑɯ˥}}}}{}
\textcolor{teal}{\mytextsc{nom}} \hspace{4pt} Ton~: H\#.
\textcolor{PineGreen}{\selectlanguage{french}Éclats de pierre, débris de pierre, petits bouts de pierre (ne veut pas dire “sable”).} \zh{零碎的石头块。}  \zh{量词}: \textcolor{darkblue}{\textbf{\ipa{ʈʂwɤ˧}}}  \mytextsc{clf}~: \textcolor{darkblue}{\textbf{\ipa{ʈʂwɤ˧}}} 
\lhead{\firstmark}
\rhead{\botmark}

\subsection{\hspace{-0.5cm} {\Large \textcolor{darkblue}{\textbf{\ipa{lv̩˩ʝi˧}}}}\hspace{0.5cm}[\kern2pt{\textcolor{darkblue}{\textbf{\ipa{lv̩˧ʝi˧}}}}\kern2pt]} \hypertarget{lv\string_=\string_Bj££i\string_M1}{}
\markboth{\textcolor{darkblue}{\textbf{\ipa{lv̩˩ʝi˧}}}}{}
\textcolor{teal}{\mytextsc{verbe}} \hspace{4pt} Ton~: LM.
\textcolor{PineGreen}{\selectlanguage{french}Enregistrer.} \zh{录音(汉语借词)。}  Emprunt~: chinois  \zh{录音}
 ¶ \textcolor{darkblue}{\textbf{\ipa{hɑ˧ le˧-dzɯ˧-se˥, | lv̩˩ ʝi˧-bi˧ !}}} \textcolor{PineGreen}{\selectlanguage{french}Quand (on) aura fini de manger, (on) fera un enregistrement !} \zh{吃完饭,就录音吧! / 吃完饭就可以录音了!}  

\lhead{\firstmark}
\rhead{\botmark}

\subsection{\hspace{-0.5cm} {\Large \textcolor{darkblue}{\textbf{\ipa{lv̩˧mi˧}}}}\hspace{0.5cm}[\kern2pt{\textcolor{darkblue}{\textbf{\ipa{lv̩˩mi˥}}}}\kern2pt]} \hypertarget{lv\string_=\string_Mmi\string_M1}{}
\markboth{\textcolor{darkblue}{\textbf{\ipa{lv̩˧mi˧}}}}{}
\textcolor{teal}{\mytextsc{nom}} \hspace{4pt} Ton~: M.
\textcolor{PineGreen}{\selectlanguage{french}Pierre.} \zh{石头。}  ¶ \textcolor{darkblue}{\textbf{\ipa{kʰv̩˧pʰæ˧tɕi˩, | lv̩˧mi˧ dzɯ˧-bi˧-ʁo˧-ho˩!}}} \textcolor{PineGreen}{\selectlanguage{french}'Quand on est jeune, on mangerait des pierres!' (Signification: quand on est jeune, on mange de tout, on a une digestion solide; tandis que quand on est vieux, on a facilement mal au ventre, dès qu'on mange quelque chose d'un peu indigeste, une nourriture “trop dure”.)} \zh{‘年轻人,石头都能吃!’(意思:年轻人消化好,吃什么都行,而人变老就不那么容易消化了,要注意吃什么。)}  
 \zh{量词}: \textcolor{darkblue}{\textbf{\ipa{ɭɯ˧}}}  \mytextsc{clf}~: \textcolor{darkblue}{\textbf{\ipa{ɭɯ˧}}} 
\lhead{\firstmark}
\rhead{\botmark}

\subsection{\hspace{-0.5cm} {\Large \textcolor{darkblue}{\textbf{\ipa{lv̩˧mi˧-bo\#˥}}}}\hspace{0.5cm}[\kern2pt{\textcolor{darkblue}{\textbf{\ipa{xxxx non-correspondance entre le nombre de morphèmes et le nombre de tons de morphèmes}}}}\kern2pt]} \hypertarget{lv\string_=\string_Mmi\string_M-bo\#\string_T1}{}
\markboth{\textcolor{darkblue}{\textbf{\ipa{lv̩˧mi˧-bo\#˥}}}}{}
\textcolor{teal}{\mytextsc{nom}} \hspace{4pt} Ton~: \#H.
\textcolor{PineGreen}{\selectlanguage{french}Mur en pierre.} \zh{石墙。}  \zh{量词}: \textcolor{darkblue}{\textbf{\ipa{ɭɯ˧}}}  \mytextsc{clf}~: \textcolor{darkblue}{\textbf{\ipa{ɭɯ˧}}} 
\lhead{\firstmark}
\rhead{\botmark}

\subsection{\hspace{-0.5cm} {\Large \textcolor{darkblue}{\textbf{\ipa{lv̩˧mi˧-dʑɯ˧dʑɯ˩}}}}\hspace{0.5cm}[\kern2pt{\textcolor{darkblue}{\textbf{\ipa{xxxx non-correspondance entre le nombre de morphèmes et le nombre de tons de morphèmes}}}}\kern2pt]} \hypertarget{lv\string_=\string_Mmi\string_M-dz£M\string_Mdz£M\string_B1}{}
\markboth{\textcolor{darkblue}{\textbf{\ipa{lv̩˧mi˧-dʑɯ˧dʑɯ˩}}}}{}
\textcolor{teal}{\mytextsc{nom}} \hspace{4pt} Ton~: \mytextsc{L}\#.
\textcolor{PineGreen}{\selectlanguage{french}Éclats de pierre, débris de pierre, petits bouts de pierre (ne veut pas dire “sable”).} \zh{零碎的石块。}  \zh{量词}: \textcolor{darkblue}{\textbf{\ipa{kʰwɤ˥}}}  \mytextsc{clf}~: \textcolor{darkblue}{\textbf{\ipa{kʰwɤ˥}}} 
\lhead{\firstmark}
\rhead{\botmark}

\subsection{\hspace{-0.5cm} {\Large \textcolor{darkblue}{\textbf{\ipa{lv̩˧pʰv̩˩}}} \textsubscript{2}}\hspace{0.5cm}[\kern2pt{\textcolor{darkblue}{\textbf{\ipa{lv̩˧pʰv̩˩}}}}\kern2pt]} \hypertarget{lv\string_=\string_Mp\string_hv\string_=\string_B2}{}
\markboth{\textcolor{darkblue}{\textbf{\ipa{lv̩˧pʰv̩˩}}} \textsubscript{2}}{}
\textcolor{teal}{\mytextsc{nom}} \hspace{4pt} Ton~: L\#.
\textcolor{PineGreen}{\selectlanguage{french}Champs inondés.} \zh{水田。}  \zh{量词}: \textcolor{darkblue}{\textbf{\ipa{pʰv̩˩}}}  \mytextsc{clf}~: \textcolor{darkblue}{\textbf{\ipa{pʰv̩˩}}} 
\lhead{\firstmark}
\rhead{\botmark}

\subsection{\hspace{-0.5cm} {\Large \textcolor{darkblue}{\textbf{\ipa{lv̩˧qæ\#˥}}}}\hspace{0.5cm}[\kern2pt{\textcolor{darkblue}{\textbf{\ipa{xxxx groupe tonal entier sans aucun ton}}}}\kern2pt]} \hypertarget{lv\string_=\string_Mq\{\#\string_T1}{}
\markboth{\textcolor{darkblue}{\textbf{\ipa{lv̩˧qæ\#˥}}}}{}
\textcolor{teal}{\mytextsc{nom}} \hspace{4pt} Ton~: \#˥.
\textcolor{PineGreen}{\selectlanguage{french}Limite de propriété: limite entre les champs appartenant à des familles différentes. Elle est souvent matérialisée par une diguette.} \zh{地界:不同家庭田地之间的界限。} 
\lhead{\firstmark}
\rhead{\botmark}

\subsection{\hspace{-0.5cm} {\Large \textcolor{darkblue}{\textbf{\ipa{lv̩˧sɯ˥}}}}\hspace{0.5cm}[\kern2pt{\textcolor{darkblue}{\textbf{\ipa{lv̩˧sɯ˥}}}}\kern2pt]} \hypertarget{lv\string_=\string_MsM\string_T1}{}
\markboth{\textcolor{darkblue}{\textbf{\ipa{lv̩˧sɯ˥}}}}{}
\textcolor{teal}{\mytextsc{nom}} \hspace{4pt} Ton~: H\#.
\textcolor{PineGreen}{\selectlanguage{french}Lisu (groupe ethnique).} \zh{傈僳族。}  \zh{量词}: \textcolor{darkblue}{\textbf{\ipa{v̩˧}}}  \mytextsc{clf}~: \textcolor{darkblue}{\textbf{\ipa{v̩˧}}} 
\lhead{\firstmark}
\rhead{\botmark}

\subsection{\hspace{-0.5cm} {\Large \textcolor{darkblue}{\textbf{\ipa{lv̩˩tɕʰɯ˧}}}}\hspace{0.5cm}[\kern2pt{\textcolor{darkblue}{\textbf{\ipa{lv̩˩tɕʰɯ˥}}}}\kern2pt]} \hypertarget{lv\string_=\string_Bts£\string_hM\string_M1}{}
\markboth{\textcolor{darkblue}{\textbf{\ipa{lv̩˩tɕʰɯ˧}}}}{}
\textcolor{teal}{\mytextsc{nom}} \hspace{4pt} Ton~: LM.
\textcolor{PineGreen}{\selectlanguage{french}Fengke: nom chinois ancien du village de Fengke, au bord du Yangtze.} \zh{六区,今奉科乡(汉语借词)。}  Emprunt~: chinois  \zh{六区}

\lhead{\firstmark}
\rhead{\botmark}

\subsection{\hspace{-0.5cm} {\Large \textcolor{darkblue}{\textbf{\ipa{lv̩˩tɕʰɯ˧-hĩ\#˥}}}}\hspace{0.5cm}[\kern2pt{\textcolor{darkblue}{\textbf{\ipa{xxxx non-correspondance entre le nombre de morphèmes et le nombre de tons de morphèmes}}}}\kern2pt]} \hypertarget{lv\string_=\string_Bts£\string_hM\string_M-hi\string_~\#\string_T1}{}
\markboth{\textcolor{darkblue}{\textbf{\ipa{lv̩˩tɕʰɯ˧-hĩ\#˥}}}}{}
\textcolor{teal}{\mytextsc{nom}} \hspace{4pt} Ton~: LM+\#H.
\textcolor{PineGreen}{\selectlanguage{french}Gens de Fengke (Fv-kho).} \zh{奉科的人。}  Emprunt~: chinois  \zh{六区}

\lhead{\firstmark}
\rhead{\botmark}

\subsection{\hspace{-0.5cm} {\Large \textcolor{darkblue}{\textbf{\ipa{lv̩˧tsɯ˥}}}}\hspace{0.5cm}[\kern2pt{\textcolor{darkblue}{\textbf{\ipa{lv̩˧tsɯ˥}}}}\kern2pt]} \hypertarget{lv\string_=\string_MtsM\string_T1}{}
\markboth{\textcolor{darkblue}{\textbf{\ipa{lv̩˧tsɯ˥}}}}{}
\textcolor{teal}{\mytextsc{nom}} \hspace{4pt} Ton~: H\#.
\textcolor{PineGreen}{\selectlanguage{french}Four.} \zh{炉子(汉语借词)。}  Emprunt~: chinois  \zh{炉子}
 \zh{量词}: \textcolor{darkblue}{\textbf{\ipa{nɑ˧}}}  \mytextsc{clf}~: \textcolor{darkblue}{\textbf{\ipa{nɑ˧}}} 
\lhead{\firstmark}
\rhead{\botmark}

\subsection{\hspace{-0.5cm} {\Large \textcolor{darkblue}{\textbf{\ipa{lv̩˩\textasciitilde{}lv̩˧˥}}}}\hspace{0.5cm}[\kern2pt{\textcolor{darkblue}{\textbf{\ipa{lv̩˧lv̩˧˥}}}}\kern2pt]} \hypertarget{lv\string_=\string_B~lv\string_=\string_M\string_T1}{}
\markboth{\textcolor{darkblue}{\textbf{\ipa{lv̩˩\textasciitilde{}lv̩˧˥}}}}{}
\textcolor{teal}{\mytextsc{verbe}} \hspace{4pt} Ton~: MH.
\textcolor{PineGreen}{\selectlanguage{french}Bouger, faire des mouvements.} \zh{动(虫、桌子、小孩子动)。}  ¶ \textcolor{darkblue}{\textbf{\ipa{lv̩˩\textasciitilde{}lv̩˧-ze˥}}} \textcolor{PineGreen}{\selectlanguage{french}\mytextsc{pfv}} \zh{动了}  
 ¶ \textcolor{darkblue}{\textbf{\ipa{tʰi˧-lv̩˩\textasciitilde{}lv̩˩(-ze˩)}}} \textcolor{PineGreen}{\selectlanguage{french}\mytextsc{dur} \mytextsc{red}} \zh{\mytextsc{dur} \mytextsc{red}}  
 ¶ \textcolor{darkblue}{\textbf{\ipa{tʰi˧-lv̩˩\textasciitilde{}lv̩˩ | se˧}}} \textcolor{PineGreen}{\selectlanguage{french}marcher en se trémoussant, marcher de travers, marcher en se contorsionnant} \zh{歪着走、扭着走、例如:残疾人走路有困难}  
 ¶ \textcolor{darkblue}{\textbf{\ipa{kʰɯ˧tsʰɤ˧ lv̩˥\textasciitilde{}lv̩˩}}} \textcolor{PineGreen}{\selectlanguage{french}bouger la jambe, remuer la jambe} \zh{活动一下(自己的)腿}  

\lhead{\firstmark}
\rhead{\botmark}

\subsection{\hspace{-0.5cm} {\Large \textcolor{darkblue}{\textbf{\ipa{lv˧bv˧}}}}\hspace{0.5cm}[\kern2pt{\textcolor{darkblue}{\textbf{\ipa{lv˩bv˩˥}}}}\kern2pt]} \hypertarget{lv\string_Mbv\string_M1}{}
\markboth{\textcolor{darkblue}{\textbf{\ipa{lv˧bv˧}}}}{}
\textcolor{teal}{\mytextsc{nom}} \hspace{4pt} Ton~: M.
\textcolor{PineGreen}{\selectlanguage{french}Lit à légumes (dans le potager).} \zh{菜畦。}  ¶ \textcolor{darkblue}{\textbf{\ipa{v˩tsʰɤ˧-lv˧bv̩\#˥}}} \textcolor{PineGreen}{\selectlanguage{french}même sens: lit à légumes (dans le potager)} \zh{同上:菜畦}  
 ¶ \textcolor{darkblue}{\textbf{\ipa{qʰwæ˧ɭɯ˧-qo˧ | v˩tsʰɤ˧-lv˧bv˧ | le˧-gv˩, v˩tsʰɤ˧˥ | ɖɯ˧-jɤ˩ tʰi˩-pʰo˩}}} \textcolor{PineGreen}{\selectlanguage{french}bâtir un lit à légumes dans le potager, et semer une rangée de légumes} \zh{菜园里建菜畦,种一排菜}  
 \zh{量词}: \textcolor{darkblue}{\textbf{\ipa{kɤ˧˥}}}  \mytextsc{clf}~: \textcolor{darkblue}{\textbf{\ipa{kɤ˧˥}}} 
\lhead{\firstmark}
\rhead{\botmark}

\subsection{\hspace{-0.5cm} {\Large \textcolor{darkblue}{\textbf{\ipa{lwæ˩pʰv̩˩}}}}\hspace{0.5cm}[\kern2pt{\textcolor{darkblue}{\textbf{\ipa{lwæ˩pʰv̩˩˥}}}}\kern2pt]} \hypertarget{lw\{\string_Bp\string_hv\string_=\string_B1}{}
\markboth{\textcolor{darkblue}{\textbf{\ipa{lwæ˩pʰv̩˩}}}}{}
\textcolor{teal}{\mytextsc{nom}} \hspace{4pt} Ton~: L.
\textcolor{PineGreen}{\selectlanguage{french}Cendres.} \zh{灰。}  ¶ \textcolor{darkblue}{\textbf{\ipa{[F5] lwæ˩pʰv̩˩-ni˥gv̩˩}}} \textcolor{PineGreen}{\selectlanguage{french}de couleur grise; littéralement “comme de la cendre”} \zh{灰色的(直译:“像白灰”)}  

\lhead{\firstmark}
\rhead{\botmark}

\subsection{\hspace{-0.5cm} {\Large \textcolor{darkblue}{\textbf{\ipa{lwɤ˩˥}}}}\hspace{0.5cm}[\kern2pt{\textcolor{darkblue}{\textbf{\ipa{lwɤ˩˥}}}}\kern2pt]} \hypertarget{lw7\string_B\string_T1}{}
\markboth{\textcolor{darkblue}{\textbf{\ipa{lwɤ˩˥}}}}{}
\textcolor{teal}{\mytextsc{nom}} \hspace{4pt} Ton~: LH.
\textcolor{PineGreen}{\selectlanguage{french}Cendre (cendre végétale ou cendre de charbon); scories.} \zh{灰,灰烬(包括草木灰等等)。}  ¶ \textcolor{darkblue}{\textbf{\ipa{lwɤ˩-pʰæ˧di˩}}} \textcolor{PineGreen}{\selectlanguage{french}comme de la cendre} \zh{像灰烬,灰色}  
 \zh{量词}: \textcolor{darkblue}{\textbf{\ipa{ʈʂwɤ˧}}}  \mytextsc{clf}~: \textcolor{darkblue}{\textbf{\ipa{ʈʂwɤ˧}}} 
\lhead{\firstmark}
\rhead{\botmark}

\subsection{\hspace{-0.5cm} {\Large \textcolor{darkblue}{\textbf{\ipa{ɭɯ˧\textsubscript{b}}}}}\hspace{0.5cm}[\kern2pt{\textcolor{darkblue}{\textbf{\ipa{ɭɯ˥}}}}\kern2pt]} \hypertarget{l\string_RM\string_Mb1}{}
\markboth{\textcolor{darkblue}{\textbf{\ipa{ɭɯ˧\textsubscript{b}}}}}{}
\textcolor{teal}{\mytextsc{classificateur}} \hspace{4pt} Ton~: M\textsubscript{b}.
\textcolor{PineGreen}{\selectlanguage{french}Classificateur générique; à l'origine, classificateur pour les objets ronds, à l'emploi maintenant élargi.} \zh{最常用的量词,相当于汉语中‘个’的用法。本意是圆形颗粒。一粒(米……),一个(碗……),件(衣服……)。}  ¶ \textcolor{darkblue}{\textbf{\ipa{ɕi˧ ɖɯ˧-ɭɯ˧ |}}} \textcolor{PineGreen}{\selectlanguage{french}un grain de riz} \zh{一粒米}  
 ¶ \textcolor{darkblue}{\textbf{\ipa{hõ˧-ɭɯ˥}}} \textcolor{PineGreen}{\selectlanguage{french}huit grains} \zh{八粒}  
 ¶ \textcolor{darkblue}{\textbf{\ipa{ɖɯ˧-ɭɯ˧ hwæ˧-mɤ˧-ɖo˧! | le˧-qʰwæ˧-kv̩˥!}}} \textcolor{PineGreen}{\selectlanguage{french}N'en achète pas un (unique)! Ca va se casser! (Explication: il faut acheter les objets par paires: 2, 4, 6, 8, 10…, pas en nombre impair, sinon cela porte malheur et les objets cassent, se perdent…)} \zh{不要(只)买一个!会碎的!(东西要一对一对买:2、4、6、8、10……,单数不吉利,东西会碎的。)}  
 ¶ \textcolor{darkblue}{\textbf{\ipa{ʈʂʰɯ˧ | zo˧hṽ˥ | dʑɤ˩-ɭɯ˥ dʑo˩!}}} \textcolor{PineGreen}{\selectlanguage{french}Elle a un bien bel enfant! (Contexte: lors d'une sortie, la consultante principale dit quelques politesses à une voisine qui se promenait avec un petit-fils attendrissant; elle me dit ensuite: “Elle a un bien bel enfant!” Littéralement: “elle, (d')enfant(s), (elle) en a un (de) bien!”)} \zh{她有个很漂亮的孩子!}  

\lhead{\firstmark}
\rhead{\botmark}

\subsection{\hspace{-0.5cm} {\Large \textcolor{darkblue}{\textbf{\ipa{ɭɯ˧˥\textsubscript{a}}}}}\hspace{0.5cm}[\kern2pt{\textcolor{darkblue}{\textbf{\ipa{ɭɯ˧˥}}}}\kern2pt]} \hypertarget{l\string_RM\string_M\string_Ta1}{}
\markboth{\textcolor{darkblue}{\textbf{\ipa{ɭɯ˧˥\textsubscript{a}}}}}{}
\textcolor{teal}{\mytextsc{classificateur}} \hspace{4pt} Ton~: MH\textsubscript{a}.
\textcolor{PineGreen}{\selectlanguage{french}Classificateur des vêtements.} \zh{量词:衣服(一件)。}  ¶ \textcolor{darkblue}{\textbf{\ipa{ʈʰæ˧qʰwɤ˧ ɖɯ˧-ɭɯ˧˥}}} \textcolor{PineGreen}{\selectlanguage{french}une jupe} \zh{一件裙子}  
 ¶ \textcolor{darkblue}{\textbf{\ipa{bɑ˩lɑ˩˥ | ɖɯ˧-ɭɯ˧˥ |}}} \textcolor{PineGreen}{\selectlanguage{french}un vêtement} \zh{一件衣服}  
 ¶ \textcolor{darkblue}{\textbf{\ipa{*dʑi˧hṽ˥\$+ɖɯ˧-ɭɯ˧˥}}} \textcolor{PineGreen}{\selectlanguage{french}Ce classificateur ne se combine pas avec /dʑi˧hṽ˥\$/, qui prend pour classificateur: /ɖɯ˧-dzi˩/.} \zh{(这个量词不能与/dʑi˧hṽ˥\$/结合。)}  

\lhead{\firstmark}
\rhead{\botmark}

\newpage
\section*{\centering- \textcolor{darkblue}{\textbf{\ipa{ɬ}}} -}
\subsection{\hspace{-0.5cm} {\Large \textcolor{darkblue}{\textbf{\ipa{ɬɑ˧mv̩˥\$}}}}\hspace{0.5cm}[\kern2pt{\textcolor{darkblue}{\textbf{\ipa{ɬɑ˧mv̩˥}}}}\kern2pt]} \hypertarget{KA\string_Mmv\string_=\string_T\$1}{}
\markboth{\textcolor{darkblue}{\textbf{\ipa{ɬɑ˧mv̩˥\$}}}}{}
\textcolor{teal}{\mytextsc{nom}} \hspace{4pt} Ton~: H\$.
\textcolor{PineGreen}{\selectlanguage{french}Prénom féminin.} \zh{女性名字。} 
\lhead{\firstmark}
\rhead{\botmark}

\subsection{\hspace{-0.5cm} {\Large \textcolor{darkblue}{\textbf{\ipa{ɬɑ˧pɤ˩}}}}\hspace{0.5cm}[\kern2pt{\textcolor{darkblue}{\textbf{\ipa{ɬɑ˧pɤ˩}}}}\kern2pt]} \hypertarget{KA\string_Mp7\string_B1}{}
\markboth{\textcolor{darkblue}{\textbf{\ipa{ɬɑ˧pɤ˩}}}}{}
\textcolor{teal}{\mytextsc{adverbe}} \hspace{4pt} Ton~: L\#.
\textcolor{PineGreen}{\selectlanguage{french}Beaucoup.} \zh{多、使劲。}  ¶ \textcolor{darkblue}{\textbf{\ipa{ɬɑ˧pɤ˩ ʝi˩}}} \textcolor{PineGreen}{\selectlanguage{french}en faire beaucoup} \zh{使劲工作、使劲干}  
 ¶ \textcolor{darkblue}{\textbf{\ipa{ɬɑ˧pɤ˩ | ɖɯ˧-kʰwɤ˧ ʝi˧}}} \textcolor{PineGreen}{\selectlanguage{french}en mettre un coup, beaucoup travailler, bien avancer dans son travail} \zh{使劲工作一下}  
 ¶ \textcolor{darkblue}{\textbf{\ipa{ɬɑ˧pɤ˩ | ɖɯ˧-kʰwɤ˧ so˥}}} \textcolor{PineGreen}{\selectlanguage{french}beaucoup étudier, faire un bon progrès dans l'étude} \zh{努力学习一下}  

\lhead{\firstmark}
\rhead{\botmark}

\subsection{\hspace{-0.5cm} {\Large \textcolor{darkblue}{\textbf{\ipa{ɬɑ˧sɑ˧}}}}\hspace{0.5cm}[\kern2pt{\textcolor{darkblue}{\textbf{\ipa{ɬɑ˧sɑ˧}}}}\kern2pt]} \hypertarget{KA\string_MsA\string_M1}{}
\markboth{\textcolor{darkblue}{\textbf{\ipa{ɬɑ˧sɑ˧}}}}{}
\textcolor{teal}{\mytextsc{nom}} \hspace{4pt} Ton~: M.
\textcolor{PineGreen}{\selectlanguage{french}Lhasa (capitale du Tibet).} \zh{拉萨。} 
\lhead{\firstmark}
\rhead{\botmark}

\subsection{\hspace{-0.5cm} {\Large \textcolor{darkblue}{\textbf{\ipa{ɬɑ˧tɑ˥}}}}\hspace{0.5cm}[\kern2pt{\textcolor{darkblue}{\textbf{\ipa{ɬɑ˧tɑ˥}}}}\kern2pt]} \hypertarget{KA\string_MtA\string_T1}{}
\markboth{\textcolor{darkblue}{\textbf{\ipa{ɬɑ˧tɑ˥}}}}{}
\textcolor{teal}{\mytextsc{nom}} \hspace{4pt} Ton~: H\#.
\textit{\textcolor{PineGreen}{\selectlanguage{french}archaïque}} [\zh{古语}] \textcolor{PineGreen}{\selectlanguage{french}Gilet de cuir (mot sorti d'usage, n'apparaît que dans un proverbe).} \zh{皮革背心。}  \zh{量词}: \textcolor{darkblue}{\textbf{\ipa{ɭɯ˧}}}  \mytextsc{clf}~: \textcolor{darkblue}{\textbf{\ipa{ɭɯ˧}}} 
\lhead{\firstmark}
\rhead{\botmark}

\subsection{\hspace{-0.5cm} {\Large \textcolor{darkblue}{\textbf{\ipa{ɬɑ˧tsʰo\#˥}}}}\hspace{0.5cm}[\kern2pt{\textcolor{darkblue}{\textbf{\ipa{ɬɑ˧tsʰo˧}}}}\kern2pt]} \hypertarget{KA\string_Mts\string_ho\#\string_T1}{}
\markboth{\textcolor{darkblue}{\textbf{\ipa{ɬɑ˧tsʰo\#˥}}}}{}
\textcolor{teal}{\mytextsc{nom}} \hspace{4pt} Ton~: \#H.
\textcolor{PineGreen}{\selectlanguage{french}Prénom féminin.} \zh{女性名字。} 
\lhead{\firstmark}
\rhead{\botmark}

\subsection{\hspace{-0.5cm} {\Large \textcolor{darkblue}{\textbf{\ipa{ɬɑ˧˥}}}}\hspace{0.5cm}[\kern2pt{\textcolor{darkblue}{\textbf{\ipa{ɬɑ˧˥}}}}\kern2pt]} \hypertarget{KA\string_M\string_T1}{}
\markboth{\textcolor{darkblue}{\textbf{\ipa{ɬɑ˧˥}}}}{}
\textcolor{teal}{\mytextsc{adjectif}} \hspace{4pt} Ton~: MH.
\textcolor{PineGreen}{\selectlanguage{french}Abondant, nombreux.} \zh{多、丰富、充分。}  ¶ \textcolor{darkblue}{\textbf{\ipa{dʑɤ˩-hĩ˩˥, | le˧-ɳɯ˥! | mɤ˧-dʑɤ˩-hĩ˩, | le˧-ɬɑ˧˥!}}} \textcolor{PineGreen}{\selectlanguage{french}Les bons, il n'y en a guère; les médiocres, il y en a en quantité! (Contexte: au sujet des établissements universitaires entre lesquels les titulaires du baccalauréat chinois ont à choisir)} \zh{好的,不多!不好的,就很多了!(情景:谈高中学生想入大学)}  

\lhead{\firstmark}
\rhead{\botmark}

\subsection{\hspace{-0.5cm} {\Large \textcolor{darkblue}{\textbf{\ipa{ɬi˥}}}}\hspace{0.5cm}[\kern2pt{\textcolor{darkblue}{\textbf{\ipa{ɬi˥}}}}\kern2pt]} \hypertarget{Ki\string_T1}{}
\markboth{\textcolor{darkblue}{\textbf{\ipa{ɬi˥}}}}{}
\textcolor{teal}{\mytextsc{verbe}} \hspace{4pt} Ton~: H.
\textcolor{PineGreen}{\selectlanguage{french}Se reposer, se détendre.} \zh{休息,松懈。}  ¶ \textcolor{darkblue}{\textbf{\ipa{le˧-ɬi˥}}} \textcolor{PineGreen}{\selectlanguage{french}\mytextsc{accomp} \string_} \zh{\mytextsc{accomp} \string_}  

\lhead{\firstmark}
\rhead{\botmark}

\subsection{\hspace{-0.5cm} {\Large \textcolor{darkblue}{\textbf{\ipa{ɬi˧\textsubscript{b}}}}}\hspace{0.5cm}[\kern2pt{\textcolor{darkblue}{\textbf{\ipa{ɬi˩˥}}}}\kern2pt]} \hypertarget{Ki\string_Mb1}{}
\markboth{\textcolor{darkblue}{\textbf{\ipa{ɬi˧\textsubscript{b}}}}}{}
\textcolor{teal}{\mytextsc{classificateur}} \hspace{4pt} Ton~: M\textsubscript{b}.
\textcolor{PineGreen}{\selectlanguage{french}Mois.} \zh{量词:月。} 
\lhead{\firstmark}
\rhead{\botmark}

\subsection{\hspace{-0.5cm} {\Large \textcolor{darkblue}{\textbf{\ipa{ɬi˧bo\#˥}}}}\hspace{0.5cm}[\kern2pt{\textcolor{darkblue}{\textbf{\ipa{ɬi˩bo˩˥}}}}\kern2pt]} \hypertarget{Ki\string_Mbo\#\string_T1}{}
\markboth{\textcolor{darkblue}{\textbf{\ipa{ɬi˧bo\#˥}}}}{}
\textcolor{teal}{\mytextsc{nom}} \hspace{4pt} Ton~: \#H.
\textcolor{PineGreen}{\selectlanguage{french}Sourd, personne sourde.} \zh{聋子。}  ¶ \textcolor{darkblue}{\textbf{\ipa{ɬi˧bo˧-hĩ˧}}} \textcolor{PineGreen}{\selectlanguage{french}personne sourde} \zh{耳朵聋的人}  
 \zh{量词}: \textcolor{darkblue}{\textbf{\ipa{v̩˧}}}  \mytextsc{clf}~: \textcolor{darkblue}{\textbf{\ipa{v̩˧}}} 
\lhead{\firstmark}
\rhead{\botmark}

\subsection{\hspace{-0.5cm} {\Large \textcolor{darkblue}{\textbf{\ipa{ɬi˧bv̩˧}}}}\hspace{0.5cm}[\kern2pt{\textcolor{darkblue}{\textbf{\ipa{ɬi˧bv̩˧}}}}\kern2pt]} \hypertarget{Ki\string_Mbv\string_=\string_M1}{}
\markboth{\textcolor{darkblue}{\textbf{\ipa{ɬi˧bv̩˧}}}}{}
\textcolor{teal}{\mytextsc{nom}} \hspace{4pt} Ton~: M.
\textcolor{PineGreen}{\selectlanguage{french}Bai (groupe ethnique).} \zh{白族。}  \zh{量词}: \textcolor{darkblue}{\textbf{\ipa{v̩˧}}}  \mytextsc{clf}~: \textcolor{darkblue}{\textbf{\ipa{v̩˧}}} 
\lhead{\firstmark}
\rhead{\botmark}

\subsection{\hspace{-0.5cm} {\Large \textcolor{darkblue}{\textbf{\ipa{ɬi˧bv̩˩ | dʑɤ˩tsʰi˧-si\#˥}}}}\hspace{0.5cm}[\kern2pt{\textcolor{darkblue}{\textbf{\ipa{xxxx non-correspondance entre le nombre de groupes tonals et le nombre de tons}}}}\kern2pt]} \hypertarget{Ki\string_Mbv\string_=\string_B | dz£7\string_Bts\string_hi\string_M-si\#\string_T1}{}
\markboth{\textcolor{darkblue}{\textbf{\ipa{ɬi˧bv̩˩ | dʑɤ˩tsʰi˧-si\#˥}}}}{}
\textcolor{teal}{\mytextsc{nom}} \hspace{4pt} Ton~: L\# | LM+\#H.
\textcolor{PineGreen}{\selectlanguage{french}Ailante, \textit{Ailanthus chinensis}; arbre très odorant.} \zh{香椿、香椿树。} 
\lhead{\firstmark}
\rhead{\botmark}

\subsection{\hspace{-0.5cm} {\Large \textcolor{darkblue}{\textbf{\ipa{ɬi˧bv̩˧-mi\#˥}}}}\hspace{0.5cm}[\kern2pt{\textcolor{darkblue}{\textbf{\ipa{xxxx non-correspondance entre le nombre de morphèmes et le nombre de tons de morphèmes}}}}\kern2pt]} \hypertarget{Ki\string_Mbv\string_=\string_M-mi\#\string_T1}{}
\markboth{\textcolor{darkblue}{\textbf{\ipa{ɬi˧bv̩˧-mi\#˥}}}}{}
\textcolor{teal}{\mytextsc{nom}} \hspace{4pt} Ton~: \#H.
\textcolor{PineGreen}{\selectlanguage{french}Femme bai.} \zh{白族女人。}  \zh{量词}: \textcolor{darkblue}{\textbf{\ipa{v̩˧}}}  \mytextsc{clf}~: \textcolor{darkblue}{\textbf{\ipa{v̩˧}}} 
\lhead{\firstmark}
\rhead{\botmark}

\subsection{\hspace{-0.5cm} {\Large \textcolor{darkblue}{\textbf{\ipa{ɬi˧bv̩˧-zo\#˥}}}}\hspace{0.5cm}[\kern2pt{\textcolor{darkblue}{\textbf{\ipa{xxxx non-correspondance entre le nombre de morphèmes et le nombre de tons de morphèmes}}}}\kern2pt]} \hypertarget{Ki\string_Mbv\string_=\string_M-zo\#\string_T1}{}
\markboth{\textcolor{darkblue}{\textbf{\ipa{ɬi˧bv̩˧-zo\#˥}}}}{}
\textcolor{teal}{\mytextsc{nom}} \hspace{4pt} Ton~: \#H.
\textcolor{PineGreen}{\selectlanguage{french}Homme bai.} Dialecte chinois local~:\zh{白族男人。} \zh{量词}: \textcolor{darkblue}{\textbf{\ipa{v̩˧}}}  \mytextsc{clf}~: \textcolor{darkblue}{\textbf{\ipa{v̩˧}}} 
\lhead{\firstmark}
\rhead{\botmark}

\subsection{\hspace{-0.5cm} {\Large \textcolor{darkblue}{\textbf{\ipa{ɬi˧di˩}}}}\hspace{0.5cm}[\kern2pt{\textcolor{darkblue}{\textbf{\ipa{ɬi˧di˧}}}}\kern2pt]} \hypertarget{Ki\string_Mdi\string_B1}{}
\markboth{\textcolor{darkblue}{\textbf{\ipa{ɬi˧di˩}}}}{}
\textcolor{teal}{\mytextsc{nom}} \hspace{4pt} Ton~: L\#.
\textcolor{PineGreen}{\selectlanguage{french}Yongning (nom de lieu).} \zh{永宁(地名)。} 
\lhead{\firstmark}
\rhead{\botmark}

\subsection{\hspace{-0.5cm} {\Large \textcolor{darkblue}{\textbf{\ipa{ɬi˧di˩-hĩ˩}}}}\hspace{0.5cm}[\kern2pt{\textcolor{darkblue}{\textbf{\ipa{xxxx non-correspondance entre le nombre de morphèmes et le nombre de tons de morphèmes}}}}\kern2pt]} \hypertarget{Ki\string_Mdi\string_B-hi\string_~\string_B1}{}
\markboth{\textcolor{darkblue}{\textbf{\ipa{ɬi˧di˩-hĩ˩}}}}{}
\textcolor{teal}{\mytextsc{nom}} \hspace{4pt} Ton~: L\#-.
\textcolor{PineGreen}{\selectlanguage{french}Les gens de Yongning.} \zh{永宁人(纳人)。}  \zh{量词}: \textcolor{darkblue}{\textbf{\ipa{v̩˧}}}  \mytextsc{clf}~: \textcolor{darkblue}{\textbf{\ipa{v̩˧}}} 
\lhead{\firstmark}
\rhead{\botmark}

\subsection{\hspace{-0.5cm} {\Large \textcolor{darkblue}{\textbf{\ipa{ɬi˧dʑɯ˩}}}}\hspace{0.5cm}[\kern2pt{\textcolor{darkblue}{\textbf{\ipa{xxxx non-correspondance entre le nombre de morphèmes et le nombre de tons de morphèmes}}}}\kern2pt]} \hypertarget{Ki\string_Mdz£M\string_B1}{}
\markboth{\textcolor{darkblue}{\textbf{\ipa{ɬi˧dʑɯ˩}}}}{}
\textcolor{teal}{\mytextsc{nom}} \hspace{4pt} Ton~: L\#.
\textcolor{PineGreen}{\selectlanguage{french}La rivière qui traverse la plaine de Yongning.} \zh{永宁坝子的河流。}  \zh{量词}: \textcolor{darkblue}{\textbf{\ipa{kʰɯ˩}}}  \mytextsc{clf}~: \textcolor{darkblue}{\textbf{\ipa{kʰɯ˩}}} 
\lhead{\firstmark}
\rhead{\botmark}

\subsection{\hspace{-0.5cm} {\Large \textcolor{darkblue}{\textbf{\ipa{ɬi˧gv̩\#˥}}}}\hspace{0.5cm}[\kern2pt{\textcolor{darkblue}{\textbf{\ipa{ɬi˧gv̩˩}}}}\kern2pt]} \hypertarget{Ki\string_Mgv\string_=\#\string_T1}{}
\markboth{\textcolor{darkblue}{\textbf{\ipa{ɬi˧gv̩\#˥}}}}{}
\textcolor{teal}{\mytextsc{nom}} \hspace{4pt} Ton~: \#H.
\textcolor{PineGreen}{\selectlanguage{french}Partie intermédiaire, milieu; au milieu.} \zh{中部,中间。}  ¶ \textcolor{darkblue}{\textbf{\ipa{ɬi˧gv̩˧ dzi˥}}} \textcolor{PineGreen}{\selectlanguage{french}être assis au milieu} \zh{坐在中间}  

\lhead{\firstmark}
\rhead{\botmark}

\subsection{\hspace{-0.5cm} {\Large \textcolor{darkblue}{\textbf{\ipa{ɬi˧hĩ\#˥}}} \textsubscript{1}}\hspace{0.5cm}[\kern2pt{\textcolor{darkblue}{\textbf{\ipa{ɬi˧hĩ˧}}}}\kern2pt]} \hypertarget{Ki\string_Mhi\string_~\#\string_T1}{}
\markboth{\textcolor{darkblue}{\textbf{\ipa{ɬi˧hĩ\#˥}}} \textsubscript{1}}{}
\textcolor{teal}{\mytextsc{nom}} \hspace{4pt} Ton~: \#H.
\textcolor{PineGreen}{\selectlanguage{french}Homme en position intermédiaire dans la fratrie: ni aîné ni cadet; traduction littérale: “personne du milieu”.} \zh{兄弟里面夹中的男孩(上有哥哥下有弟弟的孩子)。} 
\lhead{\firstmark}
\rhead{\botmark}

\subsection{\hspace{-0.5cm} {\Large \textcolor{darkblue}{\textbf{\ipa{ɬi˧hĩ\#˥}}} \textsubscript{2}}\hspace{0.5cm}[\kern2pt{\textcolor{darkblue}{\textbf{\ipa{ɬi˧hĩ˧}}}}\kern2pt]} \hypertarget{Ki\string_Mhi\string_~\#\string_T2}{}
\markboth{\textcolor{darkblue}{\textbf{\ipa{ɬi˧hĩ\#˥}}} \textsubscript{2}}{}
\textcolor{teal}{\mytextsc{nom}} \hspace{4pt} Ton~: \#H.
\textcolor{PineGreen}{\selectlanguage{french}Habitant de Yongning. Peut désigner les Prinmi qui habitent dans la plaine, aussi bien que les Na.} \zh{永宁的人。} 
\lhead{\firstmark}
\rhead{\botmark}

\subsection{\hspace{-0.5cm} {\Large \textcolor{darkblue}{\textbf{\ipa{ɬi˧ki\#˥}}}}\hspace{0.5cm}[\kern2pt{\textcolor{darkblue}{\textbf{\ipa{ɬi˧ki˥}}}}\kern2pt]} \hypertarget{Ki\string_Mki\#\string_T1}{}
\markboth{\textcolor{darkblue}{\textbf{\ipa{ɬi˧ki\#˥}}}}{}
\textcolor{teal}{\mytextsc{nom}} \hspace{4pt} Ton~: \#H.
\textcolor{PineGreen}{\selectlanguage{french}Village na, hors de la plaine, proche du Lac.} \zh{泸沽湖附近的一个村落。}  ¶ \textcolor{darkblue}{\textbf{\ipa{ɬi˧ki˧, | ɲi˧se˩, | tɑ˧dzi˩, | mv̩˧qʰwæ˩, | lɑ˧tʰɑ˧-di˧˥}}} \textcolor{PineGreen}{\selectlanguage{french}Villages dans l'ordre, après la plaine de Yongning, ne comptant pas comme faisant partie de Yongning. Le dernier, \textcolor{darkblue}{\textbf{\ipa{/lɑ˧tʰɑ˧-di˧˥/}}}, désigne toute la région na au-delà du quatrième village.} \zh{永宁到泸沽湖所经过的村落,依次是:里格、尼赛、大祖、木垮,然后到拉塔地(拉塔地指的是泸沽湖周边的摩梭地区,包括左所、洛水村等)}  

\lhead{\firstmark}
\rhead{\botmark}

\subsection{\hspace{-0.5cm} {\Large \textcolor{darkblue}{\textbf{\ipa{ɬi˧ki˥}}}}\hspace{0.5cm}[\kern2pt{\textcolor{darkblue}{\textbf{\ipa{ɬi˧ki˧}}}}\kern2pt]} \hypertarget{Ki\string_Mki\string_T1}{}
\markboth{\textcolor{darkblue}{\textbf{\ipa{ɬi˧ki˥}}}}{}
\textcolor{teal}{\mytextsc{nom}} \hspace{4pt} Ton~: H\#.
\textcolor{PineGreen}{\selectlanguage{french}Cérémonie pour les garçons atteignant 13 ans: littéralement “porter/enfiler/mettre le pantalon”. Après cette cérémonie, l'adolescent porte un pantalon, au lieu du vêtement unisexe des enfants. (Rituel parallèle avec fv{/ʈʰæ˩ ki˩˥/}, “porter/enfiler/mettre la jupe”, pour les jeunes filles.).} \zh{男性成年礼:直译“穿裤”。} 
\lhead{\firstmark}
\rhead{\botmark}

\subsection{\hspace{-0.5cm} {\Large \textcolor{darkblue}{\textbf{\ipa{ɬi˧mi˧}}} \textsubscript{1}}\hspace{0.5cm}[\kern2pt{\textcolor{darkblue}{\textbf{\ipa{ɬi˧mi˧}}}}\kern2pt]} \hypertarget{Ki\string_Mmi\string_M1}{}
\markboth{\textcolor{darkblue}{\textbf{\ipa{ɬi˧mi˧}}} \textsubscript{1}}{}
\textcolor{teal}{\mytextsc{nom}} \hspace{4pt} Ton~: M.
\ding{202} \textcolor{PineGreen}{\selectlanguage{french}Lune (disyllabe).} \zh{月亮(双音节)。}  \zh{量词}: \textcolor{darkblue}{\textbf{\ipa{ɭɯ˧}}} \ding{203} \textcolor{PineGreen}{\selectlanguage{french}Mois.} \zh{月(双音节)。}  ¶ \textcolor{darkblue}{\textbf{\ipa{ɬi˧mi˧ ɖɯ˧-gi˥}}} \textcolor{PineGreen}{\selectlanguage{french}une quinzaine, la moitié d'un mois} \zh{半个月}  
 ¶ \textcolor{darkblue}{\textbf{\ipa{ɬi˧mi˧ le˧-gv̩˩}}} \textcolor{PineGreen}{\selectlanguage{french}le mois décroît; expression qui peut désigner la seconde période du mois} \zh{下半月份}  
 \mytextsc{clf}~: \textcolor{darkblue}{\textbf{\ipa{ɭɯ˧}}} 
\lhead{\firstmark}
\rhead{\botmark}

\subsection{\hspace{-0.5cm} {\Large \textcolor{darkblue}{\textbf{\ipa{ɬi˧mi˧}}} \textsubscript{2}}\hspace{0.5cm}[\kern2pt{\textcolor{darkblue}{\textbf{\ipa{ɬi˧mi˧}}}}\kern2pt]} \hypertarget{Ki\string_Mmi\string_M2}{}
\markboth{\textcolor{darkblue}{\textbf{\ipa{ɬi˧mi˧}}} \textsubscript{2}}{}
\textcolor{teal}{\mytextsc{nom}} \hspace{4pt} Ton~: M.
\textcolor{PineGreen}{\selectlanguage{french}Chevrotain femelle.} \zh{母獐子。}  \zh{量词}: \textcolor{darkblue}{\textbf{\ipa{v̩˧}}}  \mytextsc{clf}~: \textcolor{darkblue}{\textbf{\ipa{v̩˧}}} 
\lhead{\firstmark}
\rhead{\botmark}

\subsection{\hspace{-0.5cm} {\Large \textcolor{darkblue}{\textbf{\ipa{ɬi˧mi˧dɑ˧dzɯ\#˥}}}}\hspace{0.5cm}[\kern2pt{\textcolor{darkblue}{\textbf{\ipa{ɬi˧mi˧dɑ˧dzɯ˧}}}}\kern2pt]} \hypertarget{Ki\string_Mmi\string_MdA\string_MdzM\#\string_T1}{}
\markboth{\textcolor{darkblue}{\textbf{\ipa{ɬi˧mi˧dɑ˧dzɯ\#˥}}}}{}
\textcolor{teal}{\mytextsc{nom}} \hspace{4pt} Ton~: \#H.
\textcolor{PineGreen}{\selectlanguage{french}Éclipse de lune.} \zh{月蚀。}  ¶ \textcolor{darkblue}{\textbf{\ipa{ɬi˧mi˧dɑ˧dzɯ˧ tʰv̩˧}}} \textcolor{PineGreen}{\selectlanguage{french}il y a une éclipse de lune} \zh{有月蚀}  
 ¶ \textcolor{darkblue}{\textbf{\ipa{ʈʂʰɯ˧ | ɬi˧mi˧dɑ˧dzɯ˧ ɲi˥!}}} \textcolor{PineGreen}{\selectlanguage{french}c'est une éclipse de lune! (réponse à la question 'Qu'est-ce qui se passe?')} \zh{这是月蚀!(一个人问:‘这是怎么回事?’,另一个回答:‘这是月蚀!’)}  
 \zh{量词}: \textcolor{darkblue}{\textbf{\ipa{ʂɯ˩}}}  \mytextsc{clf}~: \textcolor{darkblue}{\textbf{\ipa{ʂɯ˩}}} \textcolor{PineGreen}{\selectlanguage{french}fois}
\lhead{\firstmark}
\rhead{\botmark}

\subsection{\hspace{-0.5cm} {\Large \textcolor{darkblue}{\textbf{\ipa{ɬi˧ɳæ˩}}}}\hspace{0.5cm}[\kern2pt{\textcolor{darkblue}{\textbf{\ipa{ɬi˧ɳæ˧}}}}\kern2pt]} \hypertarget{Ki\string_Mn`\{\string_B1}{}
\markboth{\textcolor{darkblue}{\textbf{\ipa{ɬi˧ɳæ˩}}}}{}
\textcolor{teal}{\mytextsc{nom}} \hspace{4pt} Ton~: L\#.
\textcolor{PineGreen}{\selectlanguage{french}Menstrues.} \zh{月经。}  ¶ \textcolor{darkblue}{\textbf{\ipa{ʈʂʰɯ˧ | ɬi˧ɳæ˩-ze˩}}} \textcolor{PineGreen}{\selectlanguage{french}Elle est en train d'avoir ses règles.} \zh{她来了月经。}  
 ¶ \textcolor{darkblue}{\textbf{\ipa{ɬi˧ɳæ˩ go˩}}} \textcolor{PineGreen}{\selectlanguage{french}avoir des menstrues douloureuses} \zh{来了月经,疼}  
 \zh{量词}: \textcolor{darkblue}{\textbf{\ipa{ɬi˧}}}  \mytextsc{clf}~: \textcolor{darkblue}{\textbf{\ipa{ɬi˧}}} \textcolor{PineGreen}{\selectlanguage{french}mois}
\lhead{\firstmark}
\rhead{\botmark}

\subsection{\hspace{-0.5cm} {\Large \textcolor{darkblue}{\textbf{\ipa{ɬi˧pæ˥}}}}\hspace{0.5cm}[\kern2pt{\textcolor{darkblue}{\textbf{\ipa{ɬi˧pæ˩}}}}\kern2pt]} \hypertarget{Ki\string_Mp\{\string_T1}{}
\markboth{\textcolor{darkblue}{\textbf{\ipa{ɬi˧pæ˥}}}}{}
\textcolor{teal}{\mytextsc{nom}} \hspace{4pt} Ton~: H\#.
\textcolor{PineGreen}{\selectlanguage{french}Boucle d'oreille.} \zh{耳环。}  ¶ \textcolor{darkblue}{\textbf{\ipa{ŋv̩˩-ɬi˩pæ˥ (+ɲi˩)}}} \textcolor{PineGreen}{\selectlanguage{french}boucle d'oreille en argent} \zh{银耳环}  
 ¶ \textcolor{darkblue}{\textbf{\ipa{hæ̃˩-ɬi˩pæ˥ (+ɲi˩)}}} \textcolor{PineGreen}{\selectlanguage{french}boucle d'oreille en or} \zh{金耳环}  
 \zh{量词}: \textcolor{darkblue}{\textbf{\ipa{dze˩}}}  \mytextsc{clf}~: \textcolor{darkblue}{\textbf{\ipa{dze˩}}} \textcolor{PineGreen}{\selectlanguage{french}paire}
\lhead{\firstmark}
\rhead{\botmark}

\subsection{\hspace{-0.5cm} {\Large \textcolor{darkblue}{\textbf{\ipa{ɬi˧pi˩}}}}\hspace{0.5cm}[\kern2pt{\textcolor{darkblue}{\textbf{\ipa{ɬi˩pi˥}}}}\kern2pt]} \hypertarget{Ki\string_Mpi\string_B1}{}
\markboth{\textcolor{darkblue}{\textbf{\ipa{ɬi˧pi˩}}}}{}
\textcolor{teal}{\mytextsc{nom}} \hspace{4pt} Ton~: L\#.
\textcolor{PineGreen}{\selectlanguage{french}Oreille.} \zh{耳朵。}  \zh{量词}: \textcolor{darkblue}{\textbf{\ipa{pʰo˧˥}}}  \mytextsc{clf}~: \textcolor{darkblue}{\textbf{\ipa{pʰo˧˥}}} 
\lhead{\firstmark}
\rhead{\botmark}

\subsection{\hspace{-0.5cm} {\Large \textcolor{darkblue}{\textbf{\ipa{ɬi˧pv̩˧lv̩˥}}}}\hspace{0.5cm}[\kern2pt{\textcolor{darkblue}{\textbf{\ipa{ɬi˩pv̩˩lv̩˩˥}}}}\kern2pt]} \hypertarget{Ki\string_Mpv\string_=\string_Mlv\string_=\string_T1}{}
\markboth{\textcolor{darkblue}{\textbf{\ipa{ɬi˧pv̩˧lv̩˥}}}}{}
\textcolor{teal}{\mytextsc{nom}} \hspace{4pt} Ton~: H\#.
\textcolor{PineGreen}{\selectlanguage{french}Tumeur de l'oreille, excroissance pathologique de l'oreille.} \zh{耳朵瘤。}  \zh{量词}: \textcolor{darkblue}{\textbf{\ipa{ɭɯ˧}}}  \mytextsc{clf}~: \textcolor{darkblue}{\textbf{\ipa{ɭɯ˧}}} 
\lhead{\firstmark}
\rhead{\botmark}

\subsection{\hspace{-0.5cm} {\Large \textcolor{darkblue}{\textbf{\ipa{ɬi˧pʰv̩\#˥}}}}\hspace{0.5cm}[\kern2pt{\textcolor{darkblue}{\textbf{\ipa{ɬi˧pʰv̩˥}}}}\kern2pt]} \hypertarget{Ki\string_Mp\string_hv\string_=\#\string_T1}{}
\markboth{\textcolor{darkblue}{\textbf{\ipa{ɬi˧pʰv̩\#˥}}}}{}
\textcolor{teal}{\mytextsc{nom}} \hspace{4pt} Ton~: \#H.
\textcolor{PineGreen}{\selectlanguage{french}Chevrotain mâle.} \zh{公獐子。}  ¶ \textcolor{darkblue}{\textbf{\ipa{ɬi˧pʰv̩˧ tʰv̩˧-mi˥\# / ɬi˧pʰv̩˧ tʰv̩˧-mi˧˥}}} \textcolor{PineGreen}{\selectlanguage{french}\mytextsc{n}+\mytextsc{dem}+\mytextsc{clf}} \zh{那只公獐子}  
 \zh{量词}: \textcolor{darkblue}{\textbf{\ipa{v̩˧}}} \textcolor{darkblue}{\textbf{\ipa{ɭɯ˧}}} \textcolor{darkblue}{\textbf{\ipa{mi˩}}}  \mytextsc{clf}~: \textcolor{darkblue}{\textbf{\ipa{v̩˧}}} \textcolor{darkblue}{\textbf{\ipa{ɭɯ˧}}} \textcolor{darkblue}{\textbf{\ipa{mi˩}}} 
\lhead{\firstmark}
\rhead{\botmark}

\subsection{\hspace{-0.5cm} {\Large \textcolor{darkblue}{\textbf{\ipa{ɬi˧qʰæ\#˥}}}}\hspace{0.5cm}[\kern2pt{\textcolor{darkblue}{\textbf{\ipa{ɬi˧qʰæ˧}}}}\kern2pt]} \hypertarget{Ki\string_Mq\string_h\{\#\string_T1}{}
\markboth{\textcolor{darkblue}{\textbf{\ipa{ɬi˧qʰæ\#˥}}}}{}
\textcolor{teal}{\mytextsc{nom}} \hspace{4pt} Ton~: \#H.
\textcolor{PineGreen}{\selectlanguage{french}Cérumen.} \zh{耳垢。}  \zh{量词}: \textcolor{darkblue}{\textbf{\ipa{kʰwɤ˥}}}  \mytextsc{clf}~: \textcolor{darkblue}{\textbf{\ipa{kʰwɤ˥}}} 
\lhead{\firstmark}
\rhead{\botmark}

\subsection{\hspace{-0.5cm} {\Large \textcolor{darkblue}{\textbf{\ipa{ɬi˧qʰv̩\#˥}}}}\hspace{0.5cm}[\kern2pt{\textcolor{darkblue}{\textbf{\ipa{ɬi˧qʰv̩˧}}}}\kern2pt]} \hypertarget{Ki\string_Mq\string_hv\string_=\#\string_T1}{}
\markboth{\textcolor{darkblue}{\textbf{\ipa{ɬi˧qʰv̩\#˥}}}}{}
\textcolor{teal}{\mytextsc{nom}} \hspace{4pt} Ton~: \#H.
\textcolor{PineGreen}{\selectlanguage{french}Conduit auditif.} \zh{耳孔。}  ¶ \textcolor{darkblue}{\textbf{\ipa{ʈʂʰɯ˧ | ɬi˧qʰv̩˧ | ɖɯ˧-pi˧˥ | tʰɑ˩˥!}}} \textcolor{PineGreen}{\selectlanguage{french}Elle a l'oreille fine! (Contexte: au sujet d'une petite fille de 2 ans qui se réveille aussitôt de sa sieste lorsqu'elle entend l'arrivée de visiteurs.)} \zh{她耳朵很好使! / 她耳朵很尖!(情景:一有客人到家的声音,睡午觉的两岁女孩子立即醒来。)}  
 \zh{量词}: \textcolor{darkblue}{\textbf{\ipa{ɭɯ˧}}}  \mytextsc{clf}~: \textcolor{darkblue}{\textbf{\ipa{ɭɯ˧}}} 
\lhead{\firstmark}
\rhead{\botmark}

\subsection{\hspace{-0.5cm} {\Large \textcolor{darkblue}{\textbf{\ipa{ɬi˧ʈv̩˥}}}}\hspace{0.5cm}[\kern2pt{\textcolor{darkblue}{\textbf{\ipa{ɬi˩ʈv̩˥}}}}\kern2pt]} \hypertarget{Ki\string_Mt`v\string_=\string_T1}{}
\markboth{\textcolor{darkblue}{\textbf{\ipa{ɬi˧ʈv̩˥}}}}{}
\textcolor{teal}{\mytextsc{nom}} \hspace{4pt} Ton~: H\#.
\textcolor{PineGreen}{\selectlanguage{french}Plantain (utilisé par les Na pour ses vertus médicinales; est abondant à Yongning).} \zh{车前草。}  \zh{量词}: \textcolor{darkblue}{\textbf{\ipa{po˧}}}  \mytextsc{clf}~: \textcolor{darkblue}{\textbf{\ipa{po˧}}} 
\lhead{\firstmark}
\rhead{\botmark}

\subsection{\hspace{-0.5cm} {\Large \textcolor{darkblue}{\textbf{\ipa{ɬi˩}}} \textsubscript{1}}\hspace{0.5cm}[\kern2pt{\textcolor{darkblue}{\textbf{\ipa{ɬi˩˥}}}}\kern2pt]} \hypertarget{Ki\string_B1}{}
\markboth{\textcolor{darkblue}{\textbf{\ipa{ɬi˩}}} \textsubscript{1}}{}
\textcolor{teal}{\mytextsc{verbe}} \hspace{4pt} Ton~: L\textsubscript{a}.
\textcolor{PineGreen}{\selectlanguage{french}Toiser: mesurer (une pièce de tissu…) à l'aune de la toise: distance entre les deux bras écartés.} \zh{量(一块布料……)有多长:有多少庹。}  ¶ \textcolor{darkblue}{\textbf{\ipa{ɬi˩-se˥ (-ze˩)}}} \textcolor{PineGreen}{\selectlanguage{french}\string_ \mytextsc{achev} (\mytextsc{pfv})} \zh{量完(了)}  

\lhead{\firstmark}
\rhead{\botmark}

\subsection{\hspace{-0.5cm} {\Large \textcolor{darkblue}{\textbf{\ipa{ɬi˩}}} \textsubscript{2}}\hspace{0.5cm}[\kern2pt{\textcolor{darkblue}{\textbf{\ipa{ɬi˥}}}}\kern2pt]} \hypertarget{Ki\string_B2}{}
\markboth{\textcolor{darkblue}{\textbf{\ipa{ɬi˩}}} \textsubscript{2}}{}
\textcolor{teal}{\mytextsc{nom}} \hspace{4pt} Ton~: L.
\textcolor{PineGreen}{\selectlanguage{french}Chevrotain.} \zh{獐子。}  \zh{量词}: \textcolor{darkblue}{\textbf{\ipa{pʰo˧˥}}} \textcolor{darkblue}{\textbf{\ipa{mi˩}}}  \mytextsc{clf}~: \textcolor{darkblue}{\textbf{\ipa{pʰo˧˥}}} \textcolor{darkblue}{\textbf{\ipa{mi˩}}} 
\lhead{\firstmark}
\rhead{\botmark}

\subsection{\hspace{-0.5cm} {\Large \textcolor{darkblue}{\textbf{\ipa{ɬi˩\textsubscript{b}}}}}\hspace{0.5cm}[\kern2pt{\textcolor{darkblue}{\textbf{\ipa{ɬi˥}}}}\kern2pt]} \hypertarget{Ki\string_Bb1}{}
\markboth{\textcolor{darkblue}{\textbf{\ipa{ɬi˩\textsubscript{b}}}}}{}
\textcolor{teal}{\mytextsc{classificateur}} \hspace{4pt} Ton~: L\textsubscript{b}.
\textcolor{PineGreen}{\selectlanguage{french}Toise: envergure des bras =longueur des deux bras écartés. Cette unité correspond à environ 5 pieds chinois (1 mètre 78).} \zh{量词:庹。}  ¶ \textcolor{darkblue}{\textbf{\ipa{tsʰe˧-ɬi˧}}} \textcolor{PineGreen}{\selectlanguage{french}10 toises} \zh{十庹}  

\lhead{\firstmark}
\rhead{\botmark}

\subsection{\hspace{-0.5cm} {\Large \textcolor{darkblue}{\textbf{\ipa{ɬi˩bi˩}}}}\hspace{0.5cm}[\kern2pt{\textcolor{darkblue}{\textbf{\ipa{ɬi˧bi˧}}}}\kern2pt]} \hypertarget{Ki\string_Bbi\string_B1}{}
\markboth{\textcolor{darkblue}{\textbf{\ipa{ɬi˩bi˩}}}}{}
\textcolor{teal}{\mytextsc{nom}} \hspace{4pt} Ton~: L.
\textcolor{PineGreen}{\selectlanguage{french}Navet, gros radis.} \zh{萝卜。}  \zh{量词}: \textcolor{darkblue}{\textbf{\ipa{ɭɯ˧}}}  \mytextsc{clf}~: \textcolor{darkblue}{\textbf{\ipa{ɭɯ˧}}} 
\lhead{\firstmark}
\rhead{\botmark}

\subsection{\hspace{-0.5cm} {\Large \textcolor{darkblue}{\textbf{\ipa{ɬi˩qʰwɤ˩}}}}\hspace{0.5cm}[\kern2pt{\textcolor{darkblue}{\textbf{\ipa{ɬi˧qʰwɤ˧}}}}\kern2pt]} \hypertarget{Ki\string_Bq\string_hw7\string_B1}{}
\markboth{\textcolor{darkblue}{\textbf{\ipa{ɬi˩qʰwɤ˩}}}}{}
\textcolor{teal}{\mytextsc{nom}} \hspace{4pt} Ton~: L.
\textcolor{PineGreen}{\selectlanguage{french}Pantalon.} \zh{裤子。}  \zh{量词}: \textcolor{darkblue}{\textbf{\ipa{ɭɯ˧}}}  \mytextsc{clf}~: \textcolor{darkblue}{\textbf{\ipa{ɭɯ˧}}} 
\lhead{\firstmark}
\rhead{\botmark}

\subsection{\hspace{-0.5cm} {\Large \textcolor{darkblue}{\textbf{\ipa{ɬi˩ʁɑ˩}}}}\hspace{0.5cm}[\kern2pt{\textcolor{darkblue}{\textbf{\ipa{ɬi˩ʁɑ˩˥}}}}\kern2pt]} \hypertarget{Ki\string_BRA\string_B1}{}
\markboth{\textcolor{darkblue}{\textbf{\ipa{ɬi˩ʁɑ˩}}}}{}
\textcolor{teal}{\mytextsc{adjectif}} \hspace{4pt} Ton~: L.
\textcolor{PineGreen}{\selectlanguage{french}Furieux, en rage (attitude d'une personne violente et présomptueuse).} \zh{大发雷霆。}  ¶ \textcolor{darkblue}{\textbf{\ipa{ɬi˩ʁɑ˩ ʝi˧}}} \textcolor{PineGreen}{\selectlanguage{french}se livrer au courroux} \zh{大发雷霆}  

\lhead{\firstmark}
\rhead{\botmark}

\subsection{\hspace{-0.5cm} {\Large \textcolor{darkblue}{\textbf{\ipa{ɬi˩ʈɯ˩mæ˥}}}}\hspace{0.5cm}[\kern2pt{\textcolor{darkblue}{\textbf{\ipa{ɬi˩ʈɯ˩mæ˩˥}}}}\kern2pt]} \hypertarget{Ki\string_Bt`M\string_Bm\{\string_T1}{}
\markboth{\textcolor{darkblue}{\textbf{\ipa{ɬi˩ʈɯ˩mæ˥}}}}{}
\textcolor{teal}{\mytextsc{nom}} \hspace{4pt} Ton~: L+H\#.
\textcolor{PineGreen}{\selectlanguage{french}Contrebas du foyer: place dans la salle principale entre le foyer et la porte (où les chiens sont tolérés en fin de repas; on leur y jette des os et autres débris de nourriture; dans la maison de F4, à la date de l’enquête, c’est un endroit où rien ne recouvre le sol cimenté.} \zh{主屋里面没有火铺的地方:没有木地板、小狗可以偶尔进来的地方(家人就给它扔骨头)。}  ¶ \textcolor{darkblue}{\textbf{\ipa{u˧=ɻ̍˩, | kʰv̩˩mi˩ ʈʂʰɯ˩-jɤ˥ | ɖɯ˧-njɤ˧-zo˥ | ɬi˩ʈɯ˩mæ˥ hĩ˩ dʑo˩.}}} \textcolor{PineGreen}{\selectlanguage{french}Nous (=dans notre maison), ce chien, il se tient souvent assis en contrebas du foyer.} \zh{咱们家这只狗经常呆在主屋火塘下面的地方。}  
 \zh{量词}: \textcolor{darkblue}{\textbf{\ipa{kʰwɤ˥}}}  \mytextsc{clf}~: \textcolor{darkblue}{\textbf{\ipa{kʰwɤ˥}}} 
\lhead{\firstmark}
\rhead{\botmark}

\subsection{\hspace{-0.5cm} {\Large \textcolor{darkblue}{\textbf{\ipa{ɬi˩zo˩}}}}\hspace{0.5cm}[\kern2pt{\textcolor{darkblue}{\textbf{\ipa{ɬi˧zo˥}}}}\kern2pt]} \hypertarget{Ki\string_Bzo\string_B1}{}
\markboth{\textcolor{darkblue}{\textbf{\ipa{ɬi˩zo˩}}}}{}
\textcolor{teal}{\mytextsc{nom}} \hspace{4pt} Ton~: L.
\textcolor{PineGreen}{\selectlanguage{french}Bébé chevrotain.} \zh{小獐子。} 
\lhead{\firstmark}
\rhead{\botmark}

\subsection{\hspace{-0.5cm} {\Large \textcolor{darkblue}{\textbf{\ipa{ɬi˧˥}}}}\hspace{0.5cm}[\kern2pt{\textcolor{darkblue}{\textbf{\ipa{ɬi˧˥}}}}\kern2pt]} \hypertarget{Ki\string_M\string_T1}{}
\markboth{\textcolor{darkblue}{\textbf{\ipa{ɬi˧˥}}}}{}
\textcolor{teal}{\mytextsc{verbe}} \hspace{4pt} Ton~: MH.
\textcolor{PineGreen}{\selectlanguage{french}Faire sécher au soleil.} \zh{晒干。}  ¶ \textcolor{darkblue}{\textbf{\ipa{le˧-pv̩˧ tʰi˧-ɬi˧˥}}} \textcolor{PineGreen}{\selectlanguage{french}exposer au soleil afin de faire sécher} \zh{晒干}  

\lhead{\firstmark}
\rhead{\botmark}

\subsection{\hspace{-0.5cm} {\Large \textcolor{darkblue}{\textbf{\ipa{ɬo˥}}}}\hspace{0.5cm}[\kern2pt{\textcolor{darkblue}{\textbf{\ipa{ɬo˧˥}}}}\kern2pt]} \hypertarget{Ko\string_T1}{}
\markboth{\textcolor{darkblue}{\textbf{\ipa{ɬo˥}}}}{}
\textcolor{teal}{\mytextsc{nom}} \hspace{4pt} Ton~: \#H.
\textcolor{PineGreen}{\selectlanguage{french}Côte.} \zh{肋骨。}  ¶ \textcolor{darkblue}{\textbf{\ipa{bo˩ɬo˧}}} \textcolor{PineGreen}{\selectlanguage{french}côtes de porc} \zh{猪肋骨}  
 \zh{量词}: \textcolor{darkblue}{\textbf{\ipa{ɭɯ˧}}}  \mytextsc{clf}~: \textcolor{darkblue}{\textbf{\ipa{ɭɯ˧}}} 
\lhead{\firstmark}
\rhead{\botmark}

\subsection{\hspace{-0.5cm} {\Large \textcolor{darkblue}{\textbf{\ipa{ɬo˧kʰv̩˧}}}}\hspace{0.5cm}[\kern2pt{\textcolor{darkblue}{\textbf{\ipa{ɬo˧kʰv̩˧}}}}\kern2pt]} \hypertarget{Ko\string_Mk\string_hv\string_=\string_M1}{}
\markboth{\textcolor{darkblue}{\textbf{\ipa{ɬo˧kʰv̩˧}}}}{}
\textcolor{teal}{\mytextsc{nom}} \hspace{4pt} Ton~: M.
\textcolor{PineGreen}{\selectlanguage{french}Hanche.} \zh{胯。}  \zh{量词}: \textcolor{darkblue}{\textbf{\ipa{ɭɯ˧}}}  \mytextsc{clf}~: \textcolor{darkblue}{\textbf{\ipa{ɭɯ˧}}} \textit{Syn~:} \hyperlink{}{\textcolor{darkblue}{\textbf{\ipa{ɬo˩tsʰe˩mæ˥}}}}. 
\lhead{\firstmark}
\rhead{\botmark}

\subsection{\hspace{-0.5cm} {\Large \textcolor{darkblue}{\textbf{\ipa{ɬo˧pɤ˥}}}}\hspace{0.5cm}[\kern2pt{\textcolor{darkblue}{\textbf{\ipa{ɬo˧pɤ˥}}}}\kern2pt]} \hypertarget{Ko\string_Mp7\string_T1}{}
\markboth{\textcolor{darkblue}{\textbf{\ipa{ɬo˧pɤ˥}}}}{}
\textcolor{teal}{\mytextsc{nom}} \hspace{4pt} Ton~: H\#.
\textcolor{PineGreen}{\selectlanguage{french}Ampoule.} \zh{水泡。}  ¶ \textcolor{darkblue}{\textbf{\ipa{ɬo˧pɤ˥ qʰwæ˩-ze˩!}}} \textcolor{PineGreen}{\selectlanguage{french}(Il s'est/Tu t'es/Je me suis) fait une ampoule!} \zh{起了水泡!}  
 ¶ \textcolor{darkblue}{\textbf{\ipa{ɬo˧pɤ˥ | ɖɯ˧-ɭɯ˧ | qʰwæ˧-ze˥!}}} \textcolor{PineGreen}{\selectlanguage{french}(Il s'est/Tu t'es/Je me suis) fait une ampoule!} \zh{起了一个水泡!}  
 ¶ \textcolor{darkblue}{\textbf{\ipa{ɬo˧pɤ˥ | ʁo˩-po˥-ɳɯ˩ | ʈʂe˩˥}}} \textcolor{PineGreen}{\selectlanguage{french}percer une ampoule à l'aide d'une aiguille} \zh{用针来扎水泡}  
 \zh{量词}: \textcolor{darkblue}{\textbf{\ipa{ɭɯ˧}}}  \mytextsc{clf}~: \textcolor{darkblue}{\textbf{\ipa{ɭɯ˧}}} 
\lhead{\firstmark}
\rhead{\botmark}

\subsection{\hspace{-0.5cm} {\Large \textcolor{darkblue}{\textbf{\ipa{ɬo˧pv̩˥}}}}\hspace{0.5cm}[\kern2pt{\textcolor{darkblue}{\textbf{\ipa{ɬo˧pv̩˥}}}}\kern2pt]} \hypertarget{Ko\string_Mpv\string_=\string_T1}{}
\markboth{\textcolor{darkblue}{\textbf{\ipa{ɬo˧pv̩˥}}}}{}
\textcolor{teal}{\mytextsc{nom}} \hspace{4pt} Ton~: H\#.
\textcolor{PineGreen}{\selectlanguage{french}Prosternation, kow-tow (très probablement emprunt tibétain).} \zh{跪下磕头 (叩头)。}  ¶ \textcolor{darkblue}{\textbf{\ipa{ɬo˧pv̩˥ ti˩}}} \textcolor{PineGreen}{\selectlanguage{french}se prosterner} \zh{跪下磕头}  
 ¶ \textcolor{darkblue}{\textbf{\ipa{ɬo˧pv̩˥ | le˧-ti˩}}} \textcolor{PineGreen}{\selectlanguage{french}se prosterner} \zh{跪下磕头}  
\textit{Voir~:} \hyperlink{}{\textcolor{darkblue}{\textbf{\ipa{ɬo˧˥}}}} 
\lhead{\firstmark}
\rhead{\botmark}

\subsection{\hspace{-0.5cm} {\Large \textcolor{darkblue}{\textbf{\ipa{ɬo˧tɑ˧}}}}\hspace{0.5cm}[\kern2pt{\textcolor{darkblue}{\textbf{\ipa{ɬo˧tɑ˧}}}}\kern2pt]} \hypertarget{Ko\string_MtA\string_M1}{}
\markboth{\textcolor{darkblue}{\textbf{\ipa{ɬo˧tɑ˧}}}}{}
\textcolor{teal}{\mytextsc{préposition}} \hspace{4pt} Ton~: M.
\textcolor{PineGreen}{\selectlanguage{french}À côté de, sur le côté de.} \zh{旁边。}  ¶ \textcolor{darkblue}{\textbf{\ipa{ɬo˧tɑ˧ ɻ̍˩}}} \textcolor{PineGreen}{\selectlanguage{french}se tourner vers le côté, se tourner de côté} \zh{向侧面转}  
 ¶ \textcolor{darkblue}{\textbf{\ipa{ʁo˧qʰwɤ˩ | ɬo˧tɑ˧ | go˩˥}}} \textcolor{PineGreen}{\selectlanguage{french}avoir mal sur le côté de la tête, avoir les tempes qui bourdonnent (littéralement: 'avoir mal sur les côtés de la tête')} \zh{头疼,太阳穴阵痛}  

\lhead{\firstmark}
\rhead{\botmark}

\subsection{\hspace{-0.5cm} {\Large \textcolor{darkblue}{\textbf{\ipa{ɬo˩kɤ˩}}}}\hspace{0.5cm}[\kern2pt{\textcolor{darkblue}{\textbf{\ipa{ɬo˩kɤ˩˥}}}}\kern2pt]} \hypertarget{Ko\string_Bk7\string_B1}{}
\markboth{\textcolor{darkblue}{\textbf{\ipa{ɬo˩kɤ˩}}}}{}
\textcolor{teal}{\mytextsc{nom}} \hspace{4pt} Ton~: L.
\textcolor{PineGreen}{\selectlanguage{french}Côte (partie du corps).} \zh{肋骨。}  \zh{量词}: \textcolor{darkblue}{\textbf{\ipa{kɤ˧˥}}}  \mytextsc{clf}~: \textcolor{darkblue}{\textbf{\ipa{kɤ˧˥}}} 
\lhead{\firstmark}
\rhead{\botmark}

\subsection{\hspace{-0.5cm} {\Large \textcolor{darkblue}{\textbf{\ipa{ɬo˩tsʰe˩mæ˥}}}}\hspace{0.5cm}[\kern2pt{\textcolor{darkblue}{\textbf{\ipa{ɬo˩tsʰe˩mæ˥}}}}\kern2pt]} \hypertarget{Ko\string_Bts\string_he\string_Bm\{\string_T1}{}
\markboth{\textcolor{darkblue}{\textbf{\ipa{ɬo˩tsʰe˩mæ˥}}}}{}
\textcolor{teal}{\mytextsc{nom}} \hspace{4pt} Ton~: L+H\#.
\textit{\textcolor{PineGreen}{\selectlanguage{french}archaïque}} [\zh{古语}] \textcolor{PineGreen}{\selectlanguage{french}Hanche.} \zh{胯。}  \zh{量词}: \textcolor{darkblue}{\textbf{\ipa{ɭɯ˧}}}  \mytextsc{clf}~: \textcolor{darkblue}{\textbf{\ipa{ɭɯ˧}}} \textit{Syn~:} \hyperlink{}{\textcolor{darkblue}{\textbf{\ipa{ɬo˧kʰv̩˧}}}}. 
\lhead{\firstmark}
\rhead{\botmark}

\subsection{\hspace{-0.5cm} {\Large \textcolor{darkblue}{\textbf{\ipa{ɬo˧˥}}}}\hspace{0.5cm}[\kern2pt{\textcolor{darkblue}{\textbf{\ipa{ɬo˧˥}}}}\kern2pt]} \hypertarget{Ko\string_M\string_T1}{}
\markboth{\textcolor{darkblue}{\textbf{\ipa{ɬo˧˥}}}}{}
\textcolor{teal}{\mytextsc{adjectif}} \hspace{4pt} Ton~: MH.
\textcolor{PineGreen}{\selectlanguage{french}Profond (eau).} \zh{深(水深)。} 
\lhead{\firstmark}
\rhead{\botmark}

\subsection{\hspace{-0.5cm} {\Large \textcolor{darkblue}{\textbf{\ipa{ɬv̩˧˥}}}}\hspace{0.5cm}[\kern2pt{\textcolor{darkblue}{\textbf{\ipa{ɬv̩˧˥}}}}\kern2pt]} \hypertarget{Kv\string_=\string_M\string_T1}{}
\markboth{\textcolor{darkblue}{\textbf{\ipa{ɬv̩˧˥}}}}{}
\textcolor{teal}{\mytextsc{nom}} \hspace{4pt} Ton~: MH.
\ding{202} \textcolor{PineGreen}{\selectlanguage{french}Cerveau, cervelle.} \zh{脑子、脑髓。}  \zh{量词}: \textcolor{darkblue}{\textbf{\ipa{ʈv̩˩}}} \ding{203} \textcolor{PineGreen}{\selectlanguage{french}Moëlle (des os).} \zh{骨髓。}  \mytextsc{clf}~: \textcolor{darkblue}{\textbf{\ipa{ʈv̩˩}}} 
\lhead{\firstmark}
\rhead{\botmark}

\subsection{\hspace{-0.5cm} {\Large \textcolor{darkblue}{\textbf{\ipa{ɬv̩˩\textsubscript{a}}}} \textsubscript{1}}\hspace{0.5cm}[\kern2pt{\textcolor{darkblue}{\textbf{\ipa{ɬv̩˩˥}}}}\kern2pt]} \hypertarget{Kv\string_=\string_Ba1}{}
\markboth{\textcolor{darkblue}{\textbf{\ipa{ɬv̩˩\textsubscript{a}}}} \textsubscript{1}}{}
\textcolor{teal}{\mytextsc{verbe}} \hspace{4pt} Ton~: L\textsubscript{a}.
\textcolor{PineGreen}{\selectlanguage{french}Garder dans la bouche, laisser fondre dans la bouche.} \zh{含在嘴里、在嘴巴里溶化。}  ¶ \textcolor{darkblue}{\textbf{\ipa{tso˧\textasciitilde{}tso˧ ɬv̩˥}}} \textcolor{PineGreen}{\selectlanguage{french}mettre des choses dans sa bouche (contexte: un petit enfant qui ne fait pas encore la différence entre nourriture et choses non comestibles met des choses dans sa bouche)} \zh{含在嘴里(情景:一个小孩把不能吃的东西含在嘴巴里)}  

\lhead{\firstmark}
\rhead{\botmark}

\subsection{\hspace{-0.5cm} {\Large \textcolor{darkblue}{\textbf{\ipa{ɬv̩˩\textsubscript{a}}}} \textsubscript{2}}\hspace{0.5cm}[\kern2pt{\textcolor{darkblue}{\textbf{\ipa{ɬv̩˩˥}}}}\kern2pt]} \hypertarget{Kv\string_=\string_Ba2}{}
\markboth{\textcolor{darkblue}{\textbf{\ipa{ɬv̩˩\textsubscript{a}}}} \textsubscript{2}}{}
\textcolor{teal}{\mytextsc{adjectif}} \hspace{4pt} Ton~: L\textsubscript{a}.
\textcolor{PineGreen}{\selectlanguage{french}Chaud, tiède (agréablement tiède, pas froid).} \zh{温暖,暖和。}  ¶ \textcolor{darkblue}{\textbf{\ipa{dʑɤ˩˥ | ɬv̩˩˥}}} \textcolor{PineGreen}{\selectlanguage{french}très tiède, bien tiède} \zh{温暖}  
 ¶ \textcolor{darkblue}{\textbf{\ipa{ɖwæ˧˥ | ɬv̩˩˥}}} \textcolor{PineGreen}{\selectlanguage{french}\mytextsc{intensif}.très: très tiède} \zh{很暖和}  
 ¶ \textcolor{darkblue}{\textbf{\ipa{ɬv̩˩-hĩ˩˥}}} \textcolor{PineGreen}{\selectlanguage{french}\mytextsc{rel}/\mytextsc{nmlz}} \zh{温暖的}  

\lhead{\firstmark}
\rhead{\botmark}

\subsection{\hspace{-0.5cm} {\Large \textcolor{darkblue}{\textbf{\ipa{ɬv̩˩\textsubscript{a}}}} \textsubscript{3}}\hspace{0.5cm}[\kern2pt{\textcolor{darkblue}{\textbf{\ipa{ɬv̩˩˥}}}}\kern2pt]} \hypertarget{Kv\string_=\string_Ba3}{}
\markboth{\textcolor{darkblue}{\textbf{\ipa{ɬv̩˩\textsubscript{a}}}} \textsubscript{3}}{}
\textcolor{teal}{\mytextsc{verbe}} \hspace{4pt} Ton~: L\textsubscript{a}.
\textit{De:} \textbf{ɬv̩˩a 2} \textcolor{PineGreen}{\selectlanguage{french}Réchauffer de la nourriture.} \zh{热饭。}  ¶ \textcolor{darkblue}{\textbf{\ipa{hɑ˧ ɬv̩˧˥}}} \textcolor{PineGreen}{\selectlanguage{french}réchauffer du riz / de la nourriture} \zh{热饭}  
 ¶ \textcolor{darkblue}{\textbf{\ipa{hɑ˧ | le˧-ɬv̩˩}}} \textcolor{PineGreen}{\selectlanguage{french}réchauffer du riz / de la nourriture} \zh{热饭}  
 ¶ \textcolor{darkblue}{\textbf{\ipa{hɑ˧ | ɖɯ˧-ɬv̩˧\textasciitilde{}ɬv̩˥-ɻ̍˩}}} \textcolor{PineGreen}{\selectlanguage{french}réchauffer un peu la nourriture} \zh{饭热一热}  

\lhead{\firstmark}
\rhead{\botmark}

\subsection{\hspace{-0.5cm} {\Large \textcolor{darkblue}{\textbf{\ipa{ɬv̩˧gv̩\#˥}}}}\hspace{0.5cm}[\kern2pt{\textcolor{darkblue}{\textbf{\ipa{ɬv̩˧gv̩˥}}}}\kern2pt]} \hypertarget{Kv\string_=\string_Mgv\string_=\#\string_T1}{}
\markboth{\textcolor{darkblue}{\textbf{\ipa{ɬv̩˧gv̩\#˥}}}}{}
\textcolor{teal}{\mytextsc{nom}} \hspace{4pt} Ton~: \#H.
\textcolor{PineGreen}{\selectlanguage{french}Nourriture qu'on offre rituellement au défunt, sept jours après sa crémation.} \zh{火葬后第七天的送食物仪式。} 
\lhead{\firstmark}
\rhead{\botmark}

\subsection{\hspace{-0.5cm} {\Large \textcolor{darkblue}{\textbf{\ipa{ɬv̩˧mi˧mæ˧dv̩˧mi\#˥}}}}\hspace{0.5cm}[\kern2pt{\textcolor{darkblue}{\textbf{\ipa{ɬv̩˧mi˧mæ˧dv̩˧mi˧}}}}\kern2pt]} \hypertarget{Kv\string_=\string_Mmi\string_Mm\{\string_Mdv\string_=\string_Mmi\#\string_T1}{}
\markboth{\textcolor{darkblue}{\textbf{\ipa{ɬv̩˧mi˧mæ˧dv̩˧mi\#˥}}}}{}
\textcolor{teal}{\mytextsc{nom}} \hspace{4pt} Ton~: \#H.
\textcolor{PineGreen}{\selectlanguage{french}Mante religieuse.} \zh{螳螂。}  ¶ \textcolor{darkblue}{\textbf{\ipa{ɬv̩˧mi˧mæ˧dv̩˧mi˧ tʰv̩˧-mi˧˥ / ɬv̩˧mi˧mæ˧dv̩˧mi˧ tʰv̩˧-mi˥\#}}} \textcolor{PineGreen}{\selectlanguage{french}\mytextsc{n}+\mytextsc{dem}+\mytextsc{clf}} \zh{那只螳螂}  
 \zh{量词}: \textcolor{darkblue}{\textbf{\ipa{mi˩}}}  \mytextsc{clf}~: \textcolor{darkblue}{\textbf{\ipa{mi˩}}} 
\lhead{\firstmark}
\rhead{\botmark}

\subsection{\hspace{-0.5cm} {\Large \textcolor{darkblue}{\textbf{\ipa{ɬv̩˧ʁwɤ\#˥}}}}\hspace{0.5cm}[\kern2pt{\textcolor{darkblue}{\textbf{\ipa{ɬv̩˧ʁwɤ˧}}}}\kern2pt]} \hypertarget{Kv\string_=\string_MRw7\#\string_T1}{}
\markboth{\textcolor{darkblue}{\textbf{\ipa{ɬv̩˧ʁwɤ\#˥}}}}{}
\textcolor{teal}{\mytextsc{nom}} \hspace{4pt} Ton~: \#H.
\textcolor{PineGreen}{\selectlanguage{french}Village en aval de Qiansuo; leur langue serait relativement proche de celle de la vallée de Yongning.} \zh{村落名。} 
\lhead{\firstmark}
\rhead{\botmark}

\newpage
\section*{\centering- \textcolor{darkblue}{\textbf{\ipa{m}}} -}
\subsection{\hspace{-0.5cm} {\Large \textcolor{darkblue}{\textbf{\ipa{mɑ˧pʰv̩˧}}}}\hspace{0.5cm}[\kern2pt{\textcolor{darkblue}{\textbf{\ipa{mɑ˧pʰv̩˧}}}}\kern2pt]} \hypertarget{mA\string_Mp\string_hv\string_=\string_M1}{}
\markboth{\textcolor{darkblue}{\textbf{\ipa{mɑ˧pʰv̩˧}}}}{}
\textcolor{teal}{\mytextsc{nom}} \hspace{4pt} Ton~: M.
\textcolor{PineGreen}{\selectlanguage{french}Beurre (pour la préparation du thé au beurre).} \zh{酥油。} 
\lhead{\firstmark}
\rhead{\botmark}

\subsection{\hspace{-0.5cm} {\Large \textcolor{darkblue}{\textbf{\ipa{mɑ˧tsɑ˥}}}}\hspace{0.5cm}[\kern2pt{\textcolor{darkblue}{\textbf{\ipa{mɑ˧tsɑ˧}}}}\kern2pt]} \hypertarget{mA\string_MtsA\string_T1}{}
\markboth{\textcolor{darkblue}{\textbf{\ipa{mɑ˧tsɑ˥}}}}{}
\textcolor{teal}{\mytextsc{nom}} \hspace{4pt} Ton~: H\#.
\textcolor{PineGreen}{\selectlanguage{french}Origine, cause (lointaine).} \zh{来历、发源地、深层原因/来源、来龙去脉、脉络。}  ¶ \textcolor{darkblue}{\textbf{\ipa{mɑ˧tsɑ˥ | ʈʂʰɯ˧-qo˧ le˧-tsʰɯ˩-ɲi˩! |}}} \textcolor{PineGreen}{\selectlanguage{french}(Au sujet d'un événement) Ca vient de loin! / ça a une origine/ça n'arrive pas par hasard!} \zh{这(件事情)出处很远! / 有它的来龙去脉(=不是突然一下子出现的)!}  
 ¶ \textcolor{darkblue}{\textbf{\ipa{mɑ˧tsɑ˥ ʈʂʰɯ˩-kʰwɤ˩ |}}} \textcolor{PineGreen}{\selectlanguage{french}\mytextsc{n}+\mytextsc{dem}+\mytextsc{clf}: cette cause, cette origine} \zh{这个来历}  
 \zh{量词}: \textcolor{darkblue}{\textbf{\ipa{kʰwɤ˥}}}  \mytextsc{clf}~: \textcolor{darkblue}{\textbf{\ipa{kʰwɤ˥}}} 
\lhead{\firstmark}
\rhead{\botmark}

\subsection{\hspace{-0.5cm} {\Large \textcolor{darkblue}{\textbf{\ipa{mɑ˩dzɑ˩}}}}\hspace{0.5cm}[\kern2pt{\textcolor{darkblue}{\textbf{\ipa{mɑ˧dzɑ˧}}}}\kern2pt]} \hypertarget{mA\string_BdzA\string_B1}{}
\markboth{\textcolor{darkblue}{\textbf{\ipa{mɑ˩dzɑ˩}}}}{}
\textcolor{teal}{\mytextsc{nom}} \hspace{4pt} Ton~: L.
\textcolor{PineGreen}{\selectlanguage{french}Encre (solide).} \zh{墨。}  \zh{量词}: \textcolor{darkblue}{\textbf{\ipa{qʰwɤ˧˥}}}  \mytextsc{clf}~: \textcolor{darkblue}{\textbf{\ipa{qʰwɤ˧˥}}} \textcolor{PineGreen}{\selectlanguage{french}bol}
\lhead{\firstmark}
\rhead{\botmark}

\subsection{\hspace{-0.5cm} {\Large \textcolor{darkblue}{\textbf{\ipa{mɑ˩ɳɯ\#˥}}}}\hspace{0.5cm}[\kern2pt{\textcolor{darkblue}{\textbf{\ipa{xxxx non-correspondance entre le nombre de morphèmes et le nombre de tons de morphèmes}}}}\kern2pt]} \hypertarget{mA\string_Bn`M\#\string_T1}{}
\markboth{\textcolor{darkblue}{\textbf{\ipa{mɑ˩ɳɯ\#˥}}}}{}
\textcolor{teal}{\mytextsc{nom}} \hspace{4pt} Ton~: LM+\#H.
\textcolor{PineGreen}{\selectlanguage{french}Mur de mani: mur de pierre sèche et de sable, comportant des tablettes de pierre sur lesquelles est gravé une inscription: le plus souvent Om Mani Padme Hum. Un mur de mani doit être contourné dans le sens des aiguilles d'une montre: le sens de rotation de l'univers, selon la doctrine bouddhiste.} \zh{嘛呢堆。}  \zh{量词}: \textcolor{darkblue}{\textbf{\ipa{ɭɯ˧}}}  \mytextsc{clf}~: \textcolor{darkblue}{\textbf{\ipa{ɭɯ˧}}} \textit{Voir~:} \textcolor{darkblue}{\textbf{\ipa{mɑ˩ɳɯ˧-do˥bv˩, do˩bv̩\#˥}}} 
\lhead{\firstmark}
\rhead{\botmark}

\subsection{\hspace{-0.5cm} {\Large \textcolor{darkblue}{\textbf{\ipa{mɑ˩ɳɯ˧-do˥bv̩˩}}}}\hspace{0.5cm}[\kern2pt{\textcolor{darkblue}{\textbf{\ipa{xxxx non-correspondance entre le nombre de morphèmes et le nombre de tons de morphèmes}}}}\kern2pt]} \hypertarget{mA\string_Bn`M\string_M-do\string_Tbv\string_=\string_B1}{}
\markboth{\textcolor{darkblue}{\textbf{\ipa{mɑ˩ɳɯ˧-do˥bv̩˩}}}}{}
\textcolor{teal}{\mytextsc{nom}} \hspace{4pt} Ton~: LM+\#H-.
\textcolor{PineGreen}{\selectlanguage{french}Mur de mani: mur de pierre sèche et de sable, comportant des tablettes de pierre sur lesquelles est gravé une inscription: le plus souvent Om Mani Padme Hum. Un mur de mani doit être contourné dans le sens des aiguilles d'une montre: le sens de rotation de l'univers, selon la doctrine bouddhiste.} \zh{嘛呢堆。}  \zh{量词}: \textcolor{darkblue}{\textbf{\ipa{ɭɯ˧}}}  \mytextsc{clf}~: \textcolor{darkblue}{\textbf{\ipa{ɭɯ˧}}} \textit{Voir~:} \textcolor{darkblue}{\textbf{\ipa{mɑ˩ɳɯ\#˥, do˩bv̩\#˥}}} 
\lhead{\firstmark}
\rhead{\botmark}

\subsection{\hspace{-0.5cm} {\Large \textcolor{darkblue}{\textbf{\ipa{mæ˧}}}}\hspace{0.5cm}[\kern2pt{\textcolor{darkblue}{\textbf{\ipa{mæ˥}}}}\kern2pt]} \hypertarget{m\{\string_M1}{}
\markboth{\textcolor{darkblue}{\textbf{\ipa{mæ˧}}}}{}
\textcolor{teal}{\mytextsc{verbe}} \hspace{4pt} Ton~: M.
\textcolor{PineGreen}{\selectlanguage{french}Parvenir à, réussir à.} \zh{……成、……成功。}  ¶ \textcolor{darkblue}{\textbf{\ipa{njɤ˧ ɖʐɤ˧˥ | tʰi˧-mɤ˧-mæ˧!}}} \textcolor{PineGreen}{\selectlanguage{french}je ne parviens pas à attraper (ex.: un fruit sur une branche trop élevée)} \zh{我够不着!(例如:够不着树枝上的果子)}  

\lhead{\firstmark}
\rhead{\botmark}

\subsection{\hspace{-0.5cm} {\Large \textcolor{darkblue}{\textbf{\ipa{mæ˧}}}}\hspace{0.5cm}[\kern2pt{\textcolor{darkblue}{\textbf{\ipa{mæ˥}}}}\kern2pt]} \hypertarget{m\{\string_M1}{}
\markboth{\textcolor{darkblue}{\textbf{\ipa{mæ˧}}}}{}
\textcolor{teal}{\mytextsc{particule}} \textcolor{teal}{\mytextsc{de}} \textcolor{teal}{\mytextsc{discours}} \hspace{4pt} Ton~: M.
\textcolor{PineGreen}{\selectlanguage{french}Particule finale exprimant l'évidence.} \zh{句尾助词,表示显然、理所当然:“……呗!”。}  ¶ \textcolor{darkblue}{\textbf{\ipa{[Healing.66] hu˧mi˧-ʈʂʰæ˧ɣɯ˧ | le˧-ʈʰɯ˩, | le˧-qʰwɤ˧-ze˧ mæ˧! |}}} \textcolor{PineGreen}{\selectlanguage{french}On prend des médicaments pour l'estomac, et ça guérit!} \zh{吃了胃药,就好了呗!}  

\lhead{\firstmark}
\rhead{\botmark}

\subsection{\hspace{-0.5cm} {\Large \textcolor{darkblue}{\textbf{\ipa{mæ˧}}} \textsubscript{1}}\hspace{0.5cm}[\kern2pt{\textcolor{darkblue}{\textbf{\ipa{mæ˥}}}}\kern2pt]} \hypertarget{m\{\string_M1}{}
\markboth{\textcolor{darkblue}{\textbf{\ipa{mæ˧}}} \textsubscript{1}}{}
\textcolor{teal}{\mytextsc{verbe}} \hspace{4pt} Ton~: M.
\textcolor{PineGreen}{\selectlanguage{french}Avoir le temps de, être libre.} \zh{(有)空。}  ¶ \textcolor{darkblue}{\textbf{\ipa{njɤ˧ | mɤ˧-mæ˧.}}} \textcolor{PineGreen}{\selectlanguage{french}je suis occupé, je n'ai pas le temps} \zh{我忙、我没有空}  
 ¶ \textcolor{darkblue}{\textbf{\ipa{njɤ˧ | mæ˧-mɤ˧-ho˩.}}} \textcolor{PineGreen}{\selectlanguage{french}Je ne vais pas avoir le temps.} \zh{我不会有时间。}  

\lhead{\firstmark}
\rhead{\botmark}

\subsection{\hspace{-0.5cm} {\Large \textcolor{darkblue}{\textbf{\ipa{mæ˧}}} \textsubscript{2}}\hspace{0.5cm}[\kern2pt{\textcolor{darkblue}{\textbf{\ipa{mæ˥}}}}\kern2pt]} \hypertarget{m\{\string_M2}{}
\markboth{\textcolor{darkblue}{\textbf{\ipa{mæ˧}}} \textsubscript{2}}{}
\textcolor{teal}{\mytextsc{verbe}} \hspace{4pt} Ton~: M.
\textcolor{PineGreen}{\selectlanguage{french}Parvenir à.} \zh{能够(做)。}  ¶ \textcolor{darkblue}{\textbf{\ipa{ɖɯ˩-hĩ˩ qʰɑ˥ mæ˩\textasciitilde{}mæ˩! | tɕi˩-hĩ˩ lə˥-mɤ˩-mæ˩! / ɖɯ˩-hĩ˩˥, | qʰɑ˧ mæ˥\textasciitilde{}mæ˩! | tɕi˩-hĩ˩˥, | le˧-mɤ˧-mæ˧!}}} \textcolor{PineGreen}{\selectlanguage{french}“Les adultes peuvent tout faire; les enfants, eux, n'y arrivent pas/n'en sont pas capables!” Sens: s'adresse à quelqu'un qui assigne des tâches aux enfants et adolescents: Laissez les enfants jouer! A chacun ses occupations: les adultes travaillent; les enfants, leur tâche, c'est de s'amuser entre eux, pas de travailler. Le travail des grands, c'est pas leur affaire!} \zh{“大人管干活,小孩管玩耍!”这个谚语的意思是:不要让孩子干活,每个年龄有自己的事,孩子的事就是玩。成年人的活儿,不是他们的事!}  

\lhead{\firstmark}
\rhead{\botmark}

\subsection{\hspace{-0.5cm} {\Large \textcolor{darkblue}{\textbf{\ipa{mæ˧\textsubscript{a}}}}}\hspace{0.5cm}[\kern2pt{\textcolor{darkblue}{\textbf{\ipa{mæ˥}}}}\kern2pt]} \hypertarget{m\{\string_Ma1}{}
\markboth{\textcolor{darkblue}{\textbf{\ipa{mæ˧\textsubscript{a}}}}}{}
\textcolor{teal}{\mytextsc{verbe}} \hspace{4pt} Ton~: M\textsubscript{a}.
\ding{202} \textcolor{PineGreen}{\selectlanguage{french}Attraper (un objet en hauteur).} \zh{钩住(东西)。}  ¶ \textcolor{darkblue}{\textbf{\ipa{tʰi˧-mæ˧-ze˧}}} \textcolor{PineGreen}{\selectlanguage{french}\mytextsc{dur} \string_ \mytextsc{pfv}} \zh{钩住了}  
\ding{203} \textcolor{PineGreen}{\selectlanguage{french}Rattraper, rejoindre (quelqu'un qui est plus avant sur un chemin/une route).} \zh{跟上。} 
\lhead{\firstmark}
\rhead{\botmark}

\subsection{\hspace{-0.5cm} {\Large \textcolor{darkblue}{\textbf{\ipa{mæ˧pæ˧}}}}\hspace{0.5cm}[\kern2pt{\textcolor{darkblue}{\textbf{\ipa{mæ˩pæ˥}}}}\kern2pt]} \hypertarget{m\{\string_Mp\{\string_M1}{}
\markboth{\textcolor{darkblue}{\textbf{\ipa{mæ˧pæ˧}}}}{}
\textcolor{teal}{\mytextsc{nom}} \hspace{4pt} Ton~: M.
\textcolor{PineGreen}{\selectlanguage{french}Vannerie.} \zh{大筛子。}  \zh{量词}: \textcolor{darkblue}{\textbf{\ipa{nɑ˧}}}  \mytextsc{clf}~: \textcolor{darkblue}{\textbf{\ipa{nɑ˧}}} 
\lhead{\firstmark}
\rhead{\botmark}

\subsection{\hspace{-0.5cm} {\Large \textcolor{darkblue}{\textbf{\ipa{-mæ˧qo˩}}}}\hspace{0.5cm}[\kern2pt{\textcolor{darkblue}{\textbf{\ipa{mæ˧qo˩}}}}\kern2pt]} \hypertarget{-m\{\string_Mqo\string_B1}{}
\markboth{\textcolor{darkblue}{\textbf{\ipa{-mæ˧qo˩}}}}{}
\textcolor{teal}{\mytextsc{postposition}} \hspace{4pt} Ton~: L\#.
\textcolor{PineGreen}{\selectlanguage{french}Derrière, sous.} \zh{下面,后面。} \textit{Voir~:} \hyperlink{}{\textcolor{darkblue}{\textbf{\ipa{mæ˧qo˩}}}} 
\lhead{\firstmark}
\rhead{\botmark}

\subsection{\hspace{-0.5cm} {\Large \textcolor{darkblue}{\textbf{\ipa{mæ˧qo˩}}}}\hspace{0.5cm}[\kern2pt{\textcolor{darkblue}{\textbf{\ipa{mæ˧qo˧}}}}\kern2pt]} \hypertarget{m\{\string_Mqo\string_B1}{}
\markboth{\textcolor{darkblue}{\textbf{\ipa{mæ˧qo˩}}}}{}
\textcolor{teal}{\mytextsc{adverbe}} \hspace{4pt} Ton~: L\#.
\textcolor{PineGreen}{\selectlanguage{french}En bas, au fond; à l'arrière, derrière.} \zh{在尽头、在极点,在下面、在后面。} \textit{Voir~:} \hyperlink{}{\textcolor{darkblue}{\textbf{\ipa{-mæ˧qo˩}}}} 
\lhead{\firstmark}
\rhead{\botmark}

\subsection{\hspace{-0.5cm} {\Large \textcolor{darkblue}{\textbf{\ipa{mæ˧qv̩˩}}}}\hspace{0.5cm}[\kern2pt{\textcolor{darkblue}{\textbf{\ipa{mæ˧qv̩˩}}}}\kern2pt]} \hypertarget{m\{\string_Mqv\string_=\string_B1}{}
\markboth{\textcolor{darkblue}{\textbf{\ipa{mæ˧qv̩˩}}}}{}
\textcolor{teal}{\mytextsc{nom}} \hspace{4pt} Ton~: L\#.
\textcolor{PineGreen}{\selectlanguage{french}Queue.} \zh{尾巴。}  ¶ \textcolor{darkblue}{\textbf{\ipa{ʝi˧-mæ˧qv̩˥}}} \textcolor{PineGreen}{\selectlanguage{french}queue de la vache} \zh{牛尾巴}  
 \zh{量词}: \textcolor{darkblue}{\textbf{\ipa{ɭɯ˧}}}  \mytextsc{clf}~: \textcolor{darkblue}{\textbf{\ipa{ɭɯ˧}}} 
\lhead{\firstmark}
\rhead{\botmark}

\subsection{\hspace{-0.5cm} {\Large \textcolor{darkblue}{\textbf{\ipa{mæ˧ɻæ˩}}}}\hspace{0.5cm}[\kern2pt{\textcolor{darkblue}{\textbf{\ipa{mæ˧ɻæ˩}}}}\kern2pt]} \hypertarget{m\{\string_Mr£`\{\string_B1}{}
\markboth{\textcolor{darkblue}{\textbf{\ipa{mæ˧ɻæ˩}}}}{}
\textcolor{teal}{\mytextsc{nom}} \hspace{4pt} Ton~: L\#.
\textcolor{PineGreen}{\selectlanguage{french}Huile végétale.} \zh{植物油。} 
\lhead{\firstmark}
\rhead{\botmark}

\subsection{\hspace{-0.5cm} {\Large \textcolor{darkblue}{\textbf{\ipa{mæ˧ɻ̃\#˥}}}}\hspace{0.5cm}[\kern2pt{\textcolor{darkblue}{\textbf{\ipa{mæ˧ɻ̃˩}}}}\kern2pt]} \hypertarget{m\{\string_Mr£`\string_~\#\string_T1}{}
\markboth{\textcolor{darkblue}{\textbf{\ipa{mæ˧ɻ̃\#˥}}}}{}
\textcolor{teal}{\mytextsc{nom}} \hspace{4pt} Ton~: \#H.
\textcolor{PineGreen}{\selectlanguage{french}Coccyx.} \zh{尾椎骨。} Dialecte chinois local~:\zh{尾结骨。} \zh{量词}: \textcolor{darkblue}{\textbf{\ipa{ɭɯ˧}}}  \mytextsc{clf}~: \textcolor{darkblue}{\textbf{\ipa{ɭɯ˧}}} 
\lhead{\firstmark}
\rhead{\botmark}

\subsection{\hspace{-0.5cm} {\Large \textcolor{darkblue}{\textbf{\ipa{mæ˩}}}}\hspace{0.5cm}[\kern2pt{\textcolor{darkblue}{\textbf{\ipa{mæ˩˥}}}}\kern2pt]} \hypertarget{m\{\string_B1}{}
\markboth{\textcolor{darkblue}{\textbf{\ipa{mæ˩}}}}{}
\textcolor{teal}{\mytextsc{verbe}} \hspace{4pt} Ton~: L.
\textcolor{PineGreen}{\selectlanguage{french}Irriguer (en faisant couler de l'eau dans de petites tranchées).} \zh{灌溉。}  ¶ \textcolor{darkblue}{\textbf{\ipa{dʑɯ˩ mæ˩˥}}} \textcolor{PineGreen}{\selectlanguage{french}irriguer, arroser, mettre de l’eau} \zh{浇灌}  
 ¶ \textcolor{darkblue}{\textbf{\ipa{dʑɯ˧ | le˧-mæ˩}}} \textcolor{PineGreen}{\selectlanguage{french}\mytextsc{accomp}: irriguer, arroser} \zh{浇灌}  

\lhead{\firstmark}
\rhead{\botmark}

\subsection{\hspace{-0.5cm} {\Large \textcolor{darkblue}{\textbf{\ipa{mæ˩\textsubscript{a}}}}}\hspace{0.5cm}[\kern2pt{\textcolor{darkblue}{\textbf{\ipa{mæ˩˥}}}}\kern2pt]} \hypertarget{m\{\string_Ba1}{}
\markboth{\textcolor{darkblue}{\textbf{\ipa{mæ˩\textsubscript{a}}}}}{}
\textcolor{teal}{\mytextsc{verbe}} \hspace{4pt} Ton~: L\textsubscript{a}.
\textcolor{PineGreen}{\selectlanguage{french}Viser; pointer, montrer du doigt.} \zh{瞄准,指。}  ¶ \textcolor{darkblue}{\textbf{\ipa{tʰi˧-mæ˩-ze˩, | qʰæ˧-bi˥-ze˩.}}} \textcolor{PineGreen}{\selectlanguage{french}(Il) a visé, (il) va tirer.} \zh{瞄准了,要开枪了。}  
 ¶ \textcolor{darkblue}{\textbf{\ipa{lo˧ɲi˥ mæ˩}}} \textcolor{PineGreen}{\selectlanguage{french}montrer du doigt} \zh{用手指出}  
 ¶ \textcolor{darkblue}{\textbf{\ipa{tso˧\textasciitilde{}tso˧ mæ˥}}} \textcolor{PineGreen}{\selectlanguage{french}pointer des choses du doigt} \zh{指东西}  

\lhead{\firstmark}
\rhead{\botmark}

\subsection{\hspace{-0.5cm} {\Large \textcolor{darkblue}{\textbf{\ipa{mæ˩\textsubscript{a}}}} \textsubscript{1}}\hspace{0.5cm}[\kern2pt{\textcolor{darkblue}{\textbf{\ipa{mæ˩˥}}}}\kern2pt]} \hypertarget{m\{\string_Ba1}{}
\markboth{\textcolor{darkblue}{\textbf{\ipa{mæ˩\textsubscript{a}}}} \textsubscript{1}}{}
\textcolor{teal}{\mytextsc{classificateur}} \hspace{4pt} Ton~: L\textsubscript{a}.
\textcolor{PineGreen}{\selectlanguage{french}Unité monétaire: un yuan.} \zh{量词:钱(一元)。}  ¶ \textcolor{darkblue}{\textbf{\ipa{ʈʂʰɯ˧-mæ˥}}} \textcolor{PineGreen}{\selectlanguage{french}\mytextsc{dem} \string_ (tone: H\# / H\$)} \zh{\mytextsc{指示代词} \string_}  

\lhead{\firstmark}
\rhead{\botmark}

\subsection{\hspace{-0.5cm} {\Large \textcolor{darkblue}{\textbf{\ipa{mæ˩\textsubscript{a}}}} \textsubscript{2}}\hspace{0.5cm}[\kern2pt{\textcolor{darkblue}{\textbf{\ipa{mæ˩˥}}}}\kern2pt]} \hypertarget{m\{\string_Ba2}{}
\markboth{\textcolor{darkblue}{\textbf{\ipa{mæ˩\textsubscript{a}}}} \textsubscript{2}}{}
\textcolor{teal}{\mytextsc{classificateur}} \hspace{4pt} Ton~: L\textsubscript{a}.
\textcolor{PineGreen}{\selectlanguage{french}10.000.} \zh{万(数词充当量词)。}  Emprunt~: chinois  \zh{元}, MC *mjonH (Baxter 2000)
 ¶ \textcolor{darkblue}{\textbf{\ipa{ɖɯ˧-mæ˩}}} \textcolor{PineGreen}{\selectlanguage{french}10.000} \zh{一万}  
 ¶ \textcolor{darkblue}{\textbf{\ipa{tsʰe˩-tv̩˩ mæ˥}}} \textcolor{PineGreen}{\selectlanguage{french}dix mille fois 10.000, soit cent millions} \zh{十千万,等于一亿}  
 ¶ \textcolor{darkblue}{\textbf{\ipa{ɖɯ˧-ɕi˧ mæ˩}}} \textcolor{PineGreen}{\selectlanguage{french}cent fois 10.000, soit un million} \zh{一百万}  

\lhead{\firstmark}
\rhead{\botmark}

\subsection{\hspace{-0.5cm} {\Large \textcolor{darkblue}{\textbf{\ipa{mæ˩ɖʐo˥}}}}\hspace{0.5cm}[\kern2pt{\textcolor{darkblue}{\textbf{\ipa{mæ˩ɖʐo˩˥}}}}\kern2pt]} \hypertarget{m\{\string_Bd`z`o\string_T1}{}
\markboth{\textcolor{darkblue}{\textbf{\ipa{mæ˩ɖʐo˥}}}}{}
\textcolor{teal}{\mytextsc{nom}} \hspace{4pt} Ton~: LH.
\textcolor{PineGreen}{\selectlanguage{french}Fouet.} \zh{鞭子。}  ¶ \textcolor{darkblue}{\textbf{\ipa{ʐwæ˧-mæ˥ɖʐo˩}}} \textcolor{PineGreen}{\selectlanguage{french}fouet de cheval} \zh{马鞭}  
 \zh{量词}: \textcolor{darkblue}{\textbf{\ipa{kʰɯ˩}}}  \mytextsc{clf}~: \textcolor{darkblue}{\textbf{\ipa{kʰɯ˩}}} 
\lhead{\firstmark}
\rhead{\botmark}

\subsection{\hspace{-0.5cm} {\Large \textcolor{darkblue}{\textbf{\ipa{mæ˩ko˥}}}}\hspace{0.5cm}[\kern2pt{\textcolor{darkblue}{\textbf{\ipa{mæ˩ko˥}}}}\kern2pt]} \hypertarget{m\{\string_Bko\string_T1}{}
\markboth{\textcolor{darkblue}{\textbf{\ipa{mæ˩ko˥}}}}{}
\textcolor{teal}{\mytextsc{nom}} \hspace{4pt} Ton~: LH.
\textcolor{PineGreen}{\selectlanguage{french}Harnais.} \zh{挽具,后鞧。}  ¶ \textcolor{darkblue}{\textbf{\ipa{ʐwæ˧-mæ˥ko˩}}} \textcolor{PineGreen}{\selectlanguage{french}harnais de cheval} \zh{马后鞧}  
 \zh{量词}: \textcolor{darkblue}{\textbf{\ipa{ɭɯ˧}}}  \mytextsc{clf}~: \textcolor{darkblue}{\textbf{\ipa{ɭɯ˧}}} 
\lhead{\firstmark}
\rhead{\botmark}

\subsection{\hspace{-0.5cm} {\Large \textcolor{darkblue}{\textbf{\ipa{mæ˧˥}}}}\hspace{0.5cm}[\kern2pt{\textcolor{darkblue}{\textbf{\ipa{mæ˧˥}}}}\kern2pt]} \hypertarget{m\{\string_M\string_T1}{}
\markboth{\textcolor{darkblue}{\textbf{\ipa{mæ˧˥}}}}{}
\textcolor{teal}{\mytextsc{verbe}} \hspace{4pt} Ton~: MH.
\ding{202} \textcolor{PineGreen}{\selectlanguage{french}Fermer (la bouche).} \zh{闭(嘴)。}  ¶ \textcolor{darkblue}{\textbf{\ipa{ɲi˧to˧ | tʰi˧-mæ˧˥}}} \textcolor{PineGreen}{\selectlanguage{french}fermer la bouche} \zh{闭嘴}  
 ¶ \textcolor{darkblue}{\textbf{\ipa{mæ˩\textasciitilde{}mæ˧˥}}} \textcolor{PineGreen}{\selectlanguage{french}\mytextsc{red}} \zh{\mytextsc{重叠}}  
 ¶ \textcolor{darkblue}{\textbf{\ipa{ɲi˧to˧ | tʰi˧-mæ˩\textasciitilde{}mæ˩}}} \textcolor{PineGreen}{\selectlanguage{french}fermer la bouche} \zh{闭嘴}  
\ding{203} \textcolor{PineGreen}{\selectlanguage{french}Pincer (les lèvres).} \zh{抿(嘴巴)。} 
\lhead{\firstmark}
\rhead{\botmark}

\subsection{\hspace{-0.5cm} {\Large \textcolor{darkblue}{\textbf{\ipa{mɤ˧‑}}}}\hspace{0.5cm}[\kern2pt{\textcolor{darkblue}{\textbf{\ipa{mɤ˥}}}}\kern2pt]} \hypertarget{m7\string_M‑1}{}
\markboth{\textcolor{darkblue}{\textbf{\ipa{mɤ˧‑}}}}{}
\textcolor{teal}{\mytextsc{préfixe}} \hspace{4pt} Ton~: M.
\textcolor{PineGreen}{\selectlanguage{french}Negation.} \zh{否定:不,没。} 
\lhead{\firstmark}
\rhead{\botmark}

\subsection{\hspace{-0.5cm} {\Large \textcolor{darkblue}{\textbf{\ipa{mɤ˧ʈʂʰɤ˧}}}}\hspace{0.5cm}[\kern2pt{\textcolor{darkblue}{\textbf{\ipa{mɤ˧ʈʂʰɤ˧}}}}\kern2pt]} \hypertarget{m7\string_Mt`s`\string_h7\string_M1}{}
\markboth{\textcolor{darkblue}{\textbf{\ipa{mɤ˧ʈʂʰɤ˧}}}}{}
\textcolor{teal}{\mytextsc{nom}} \hspace{4pt} Ton~: M.
\textcolor{PineGreen}{\selectlanguage{french}Charrette.} \zh{马车(汉语借词)。}  Emprunt~: chinois  \zh{马车}

\lhead{\firstmark}
\rhead{\botmark}

\subsection{\hspace{-0.5cm} {\Large \textcolor{darkblue}{\textbf{\ipa{mɤ˩}}}}\hspace{0.5cm}[\kern2pt{\textcolor{darkblue}{\textbf{\ipa{mɤ˥}}}}\kern2pt]} \hypertarget{m7\string_B1}{}
\markboth{\textcolor{darkblue}{\textbf{\ipa{mɤ˩}}}}{}
\textcolor{teal}{\mytextsc{nom}} \hspace{4pt} Ton~: L.
\textcolor{PineGreen}{\selectlanguage{french}Huile animale, graisse.} \zh{动物油。}  ¶ \textcolor{darkblue}{\textbf{\ipa{njɤ˧ | mɤ˩ mɤ˩ dzɯ˩˥!}}} \textcolor{PineGreen}{\selectlanguage{french}Je ne mange pas de graisse/de saindoux! (C'est là l'une des particularités de l'enquêteur)} \zh{我不吃猪油!(这是调查者的特点之一)}  

\lhead{\firstmark}
\rhead{\botmark}

\subsection{\hspace{-0.5cm} {\Large \textcolor{darkblue}{\textbf{\ipa{mɤ˩\textsubscript{a}}}}}\hspace{0.5cm}[\kern2pt{\textcolor{darkblue}{\textbf{\ipa{mɤ˩˥}}}}\kern2pt]} \hypertarget{m7\string_Ba1}{}
\markboth{\textcolor{darkblue}{\textbf{\ipa{mɤ˩\textsubscript{a}}}}}{}
\textcolor{teal}{\mytextsc{classificateur}} \hspace{4pt} Ton~: L\textsubscript{a}.
\textcolor{PineGreen}{\selectlanguage{french}Classificateur des petites quantités: quelques-uns, quelque peu de, un peu de.} \zh{量词:一些、一点。}  ¶ \textcolor{darkblue}{\textbf{\ipa{ɕi˧ɭɯ˧-ɻæ˩ | ɖɯ˧-mɤ˩}}} \textcolor{PineGreen}{\selectlanguage{french}quelques graines de riz} \zh{一些稻谷种子}  
 ¶ \textcolor{darkblue}{\textbf{\ipa{ɻæ˩˥ | ɖɯ˧-mɤ˩}}} \textcolor{PineGreen}{\selectlanguage{french}quelques graines} \zh{一些种子}  
 ¶ \textcolor{darkblue}{\textbf{\ipa{tsɑ˧bɤ˧ | ɖɯ˧-mɤ˩, | tsɑ˧bɤ˧ | ɲi˧-mɤ˩}}} \textcolor{PineGreen}{\selectlanguage{french}un peu de farine; deux poignées/petites quantités de farine; etc.} \zh{一小捧面粉、两小捧面粉……}  
 ¶ \textcolor{darkblue}{\textbf{\ipa{ʈʂʰɯ˧-mɤ˥}}} \textcolor{PineGreen}{\selectlanguage{french}\mytextsc{dem} \string_ (tone: H\# / H\$)} \zh{\mytextsc{指示代词} \string_}  

\lhead{\firstmark}
\rhead{\botmark}

\subsection{\hspace{-0.5cm} {\Large \textcolor{darkblue}{\textbf{\ipa{mɤ˩\textsubscript{b}}}}}\hspace{0.5cm}[\kern2pt{\textcolor{darkblue}{\textbf{\ipa{mɤ˩˥}}}}\kern2pt]} \hypertarget{m7\string_Bb1}{}
\markboth{\textcolor{darkblue}{\textbf{\ipa{mɤ˩\textsubscript{b}}}}}{}
\textcolor{teal}{\mytextsc{verbe}} \hspace{4pt} Ton~: L\textsubscript{b}.
\textcolor{PineGreen}{\selectlanguage{french}Prendre dans la bouche un aliment en poudre.} \zh{将粉状的食品放在嘴里(如:干糌粑)。}  ¶ \textcolor{darkblue}{\textbf{\ipa{tsɑ˧bɤ˧ mɤ˩}}} \textcolor{PineGreen}{\selectlanguage{french}manger du tsamba sec: on en prend une cuillère qu'on renverse dans sa bouche, et on laisse la farine s'imprégner de salive} \zh{吃干糌粑}  
 ¶ \textcolor{darkblue}{\textbf{\ipa{tsɑ˧bɤ˧ | ɖɯ˧-mɤ˧\textasciitilde{}mɤ˩-ɻ̍˩}}} \textcolor{PineGreen}{\selectlanguage{french}savourer un peu de tsamba} \zh{品干糌粑、慢慢享受一点干糌粑}  

\lhead{\firstmark}
\rhead{\botmark}

\subsection{\hspace{-0.5cm} {\Large \textcolor{darkblue}{\textbf{\ipa{mɤ˩ɬi˩}}}}\hspace{0.5cm}[\kern2pt{\textcolor{darkblue}{\textbf{\ipa{mɤ˩ɬi˩˥}}}}\kern2pt]} \hypertarget{m7\string_BKi\string_B1}{}
\markboth{\textcolor{darkblue}{\textbf{\ipa{mɤ˩ɬi˩}}}}{}
\textcolor{teal}{\mytextsc{nom}} \hspace{4pt} Ton~: L.
\textcolor{PineGreen}{\selectlanguage{french}Thé au beurre.} \zh{酥油茶。}  \zh{量词}: \textcolor{darkblue}{\textbf{\ipa{qʰwɤ˧˥}}}  \mytextsc{clf}~: \textcolor{darkblue}{\textbf{\ipa{qʰwɤ˧˥}}} 
\lhead{\firstmark}
\rhead{\botmark}

\subsection{\hspace{-0.5cm} {\Large \textcolor{darkblue}{\textbf{\ipa{mɤ˩mv̩˩}}}}\hspace{0.5cm}[\kern2pt{\textcolor{darkblue}{\textbf{\ipa{mɤ˩mv̩˩˥}}}}\kern2pt]} \hypertarget{m7\string_Bmv\string_=\string_B1}{}
\markboth{\textcolor{darkblue}{\textbf{\ipa{mɤ˩mv̩˩}}}}{}
\textcolor{teal}{\mytextsc{nom}} \hspace{4pt} Ton~: L.
\textcolor{PineGreen}{\selectlanguage{french}Porte-bougie: objet en cuivre dans lequel on verse de la paraffine fondue, ou de la graisse, et dans lequel on place une mèche; est utilisé dans les rituels.} \zh{烛台。}  \zh{量词}: \textcolor{darkblue}{\textbf{\ipa{qʰwɤ˧˥}}}  \mytextsc{clf}~: \textcolor{darkblue}{\textbf{\ipa{qʰwɤ˧˥}}} 
\lhead{\firstmark}
\rhead{\botmark}

\subsection{\hspace{-0.5cm} {\Large \textcolor{darkblue}{\textbf{\ipa{mɤ˩tʰɑ˧}}}}\hspace{0.5cm}[\kern2pt{\textcolor{darkblue}{\textbf{\ipa{mɤ˩tʰɑ˥}}}}\kern2pt]} \hypertarget{m7\string_Bt\string_hA\string_M1}{}
\markboth{\textcolor{darkblue}{\textbf{\ipa{mɤ˩tʰɑ˧}}}}{}
\textcolor{teal}{\mytextsc{nom}} \hspace{4pt} Ton~: LM.
\textcolor{PineGreen}{\selectlanguage{french}Confiserie au sésame.} \zh{麻糖(汉语借词)。}  Emprunt~: chinois  \zh{麻糖}

\lhead{\firstmark}
\rhead{\botmark}

\subsection{\hspace{-0.5cm} {\Large \textcolor{darkblue}{\textbf{\ipa{mi˧}}}}\hspace{0.5cm}[\kern2pt{\textcolor{darkblue}{\textbf{\ipa{mi˩˥}}}}\kern2pt]} \hypertarget{mi\string_M1}{}
\markboth{\textcolor{darkblue}{\textbf{\ipa{mi˧}}}}{}
\textcolor{teal}{\mytextsc{nom}} \hspace{4pt} Ton~: M.
\textcolor{PineGreen}{\selectlanguage{french}Blessure, plaie.} \zh{伤口。}  \zh{量词}: \textcolor{darkblue}{\textbf{\ipa{kʰwɤ˥}}}  \mytextsc{clf}~: \textcolor{darkblue}{\textbf{\ipa{kʰwɤ˥}}} 
\lhead{\firstmark}
\rhead{\botmark}

\subsection{\hspace{-0.5cm} {\Large \textcolor{darkblue}{\textbf{\ipa{mi˧kʰwɤ\#˥}}}}\hspace{0.5cm}[\kern2pt{\textcolor{darkblue}{\textbf{\ipa{mi˩kʰwɤ˥}}}}\kern2pt]} \hypertarget{mi\string_Mk\string_hw7\#\string_T1}{}
\markboth{\textcolor{darkblue}{\textbf{\ipa{mi˧kʰwɤ\#˥}}}}{}
\textcolor{teal}{\mytextsc{nom}} \hspace{4pt} Ton~: \#H.
\ding{202} \textcolor{PineGreen}{\selectlanguage{french}Blessure, plaie.} \zh{伤口。}  \zh{量词}: \textcolor{darkblue}{\textbf{\ipa{kʰwɤ˥}}} \ding{203} \textcolor{PineGreen}{\selectlanguage{french}Ulcère.} \zh{疮。}  \mytextsc{clf}~: \textcolor{darkblue}{\textbf{\ipa{kʰwɤ˥}}} 
\lhead{\firstmark}
\rhead{\botmark}

\subsection{\hspace{-0.5cm} {\Large \textcolor{darkblue}{\textbf{\ipa{mi˧ɬo\#˥}}}}\hspace{0.5cm}[\kern2pt{\textcolor{darkblue}{\textbf{\ipa{mi˩ɬo˥}}}}\kern2pt]} \hypertarget{mi\string_MKo\#\string_T1}{}
\markboth{\textcolor{darkblue}{\textbf{\ipa{mi˧ɬo\#˥}}}}{}
\textcolor{teal}{\mytextsc{nom}} \hspace{4pt} Ton~: \#H.
\textcolor{PineGreen}{\selectlanguage{french}Prière.} \zh{祈祷。}  ¶ \textcolor{darkblue}{\textbf{\ipa{mi˧ɬo˧ lɑ˩}}} \textcolor{PineGreen}{\selectlanguage{french}prier} \zh{祈祷}  
 \zh{量词}: \textcolor{darkblue}{\textbf{\ipa{kʰwɤ˥}}}  \mytextsc{clf}~: \textcolor{darkblue}{\textbf{\ipa{kʰwɤ˥}}} \textit{Voir~:} \hyperlink{}{\textcolor{darkblue}{\textbf{\ipa{ɬo˧˥}}}} 
\lhead{\firstmark}
\rhead{\botmark}

\subsection{\hspace{-0.5cm} {\Large \textcolor{darkblue}{\textbf{\ipa{mi˧mi˧}}}}\hspace{0.5cm}[\kern2pt{\textcolor{darkblue}{\textbf{\ipa{mi˧mi˧}}}}\kern2pt]} \hypertarget{mi\string_Mmi\string_M1}{}
\markboth{\textcolor{darkblue}{\textbf{\ipa{mi˧mi˧}}}}{}
\textcolor{teal}{\mytextsc{nom}} \hspace{4pt} Ton~: M.
\textcolor{PineGreen}{\selectlanguage{french}Amande (d'un noyau).} \zh{核,仁。}  Emprunt~: chinois  (dialectal) \zh{米米}

\lhead{\firstmark}
\rhead{\botmark}

\subsection{\hspace{-0.5cm} {\Large \textcolor{darkblue}{\textbf{\ipa{mi˧pɤ\#˥}}}}\hspace{0.5cm}[\kern2pt{\textcolor{darkblue}{\textbf{\ipa{mi˩pɤ˩˥}}}}\kern2pt]} \hypertarget{mi\string_Mp7\#\string_T1}{}
\markboth{\textcolor{darkblue}{\textbf{\ipa{mi˧pɤ\#˥}}}}{}
\textcolor{teal}{\mytextsc{nom}} \hspace{4pt} Ton~: \#H.
\textcolor{PineGreen}{\selectlanguage{french}Cicatrice.} \zh{疤。}  \zh{量词}: \textcolor{darkblue}{\textbf{\ipa{kʰwɤ˥}}}  \mytextsc{clf}~: \textcolor{darkblue}{\textbf{\ipa{kʰwɤ˥}}} 
\lhead{\firstmark}
\rhead{\botmark}

\subsection{\hspace{-0.5cm} {\Large \textcolor{darkblue}{\textbf{\ipa{mi˧tʰv̩\#˥}}}}\hspace{0.5cm}[\kern2pt{\textcolor{darkblue}{\textbf{\ipa{mi˩tʰv̩˩˥}}}}\kern2pt]} \hypertarget{mi\string_Mt\string_hv\string_=\#\string_T1}{}
\markboth{\textcolor{darkblue}{\textbf{\ipa{mi˧tʰv̩\#˥}}}}{}
\textcolor{teal}{\mytextsc{nom}} \hspace{4pt} Ton~: \#H.
\textcolor{PineGreen}{\selectlanguage{french}Bâton, canne pour marcher.} \zh{拐棍。}  \zh{量词}: \textcolor{darkblue}{\textbf{\ipa{kɤ˧˥}}}  \mytextsc{clf}~: \textcolor{darkblue}{\textbf{\ipa{kɤ˧˥}}} 
\lhead{\firstmark}
\rhead{\botmark}

\subsection{\hspace{-0.5cm} {\Large \textcolor{darkblue}{\textbf{\ipa{mi˩\textsubscript{a}}}}}\hspace{0.5cm}[\kern2pt{\textcolor{darkblue}{\textbf{\ipa{mi˧˥}}}}\kern2pt]} \hypertarget{mi\string_Ba1}{}
\markboth{\textcolor{darkblue}{\textbf{\ipa{mi˩\textsubscript{a}}}}}{}
\textcolor{teal}{\mytextsc{verbe}} \hspace{4pt} Ton~: L\textsubscript{a}.
\textcolor{PineGreen}{\selectlanguage{french}Demander, quémander.} \zh{请求、要,讨饭。}  ¶ \textcolor{darkblue}{\textbf{\ipa{hɑ˧ mi˥}}} \textcolor{PineGreen}{\selectlanguage{french}mendier (littéralement: 'demander à manger')} \zh{讨饭}  
 ¶ \textcolor{darkblue}{\textbf{\ipa{hɑ˧ | ɖɯ˧-mi˧\textasciitilde{}mi˥-ɻ̍˩}}} \textcolor{PineGreen}{\selectlanguage{french}mendier un peu} \zh{讨点饭}  

\lhead{\firstmark}
\rhead{\botmark}

\subsection{\hspace{-0.5cm} {\Large \textcolor{darkblue}{\textbf{\ipa{mi˩\textsubscript{b}}}}}\hspace{0.5cm}[\kern2pt{\textcolor{darkblue}{\textbf{\ipa{mi˩˥}}}}\kern2pt]} \hypertarget{mi\string_Bb1}{}
\markboth{\textcolor{darkblue}{\textbf{\ipa{mi˩\textsubscript{b}}}}}{}
\textcolor{teal}{\mytextsc{classificateur}} \hspace{4pt} Ton~: L\textsubscript{b}.
\textcolor{PineGreen}{\selectlanguage{french}Classificateur des petits animaux (poules…).} \zh{量词:小动物(一只)。}  ¶ \textcolor{darkblue}{\textbf{\ipa{ʈʂʰɯ˧-mi˧˥}}} \textcolor{PineGreen}{\selectlanguage{french}cet animal} \zh{这只}  

\lhead{\firstmark}
\rhead{\botmark}

\subsection{\hspace{-0.5cm} {\Large \textcolor{darkblue}{\textbf{\ipa{mi˩hwɑ˧}}}}\hspace{0.5cm}[\kern2pt{\textcolor{darkblue}{\textbf{\ipa{mi˩hwɑ˩˥}}}}\kern2pt]} \hypertarget{mi\string_BhwA\string_M1}{}
\markboth{\textcolor{darkblue}{\textbf{\ipa{mi˩hwɑ˧}}}}{}
\textcolor{teal}{\mytextsc{nom}} \hspace{4pt} Ton~: LM.
\textcolor{PineGreen}{\selectlanguage{french}Coton.} \zh{棉花(汉语借词)。}  Emprunt~: \zh{棉花}
 ¶ \textcolor{darkblue}{\textbf{\ipa{mi˩hwɑ˧-bɑ˩lɑ˩}}} \textcolor{PineGreen}{\selectlanguage{french}vêtement de coton} \zh{棉布衣服}  

\lhead{\firstmark}
\rhead{\botmark}

\subsection{\hspace{-0.5cm} {\Large \textcolor{darkblue}{\textbf{\ipa{mi˩ɬi˩}}}}\hspace{0.5cm}[\kern2pt{\textcolor{darkblue}{\textbf{\ipa{mi˧ɬi˧}}}}\kern2pt]} \hypertarget{mi\string_BKi\string_B1}{}
\markboth{\textcolor{darkblue}{\textbf{\ipa{mi˩ɬi˩}}}}{}
\textcolor{teal}{\mytextsc{nom}} \hspace{4pt} Ton~: L.
\textcolor{PineGreen}{\selectlanguage{french}Grand bambou.} \zh{大竹子。}  ¶ \textcolor{darkblue}{\textbf{\ipa{mi˩ɬi˩-bæ˩ʈʂo˥}}} \textcolor{PineGreen}{\selectlanguage{french}balai en petites tiges de bambou} \zh{竹扫帚}  
 ¶ \textcolor{darkblue}{\textbf{\ipa{mi˩ɬi˩-ʈʂæ˥do˩}}} \textcolor{PineGreen}{\selectlanguage{french}seau en bambou pour porter de l'eau (sur le dos)} \zh{竹桶,用来背水}  
 \zh{量词}: \textcolor{darkblue}{\textbf{\ipa{dzi˩}}}  \mytextsc{clf}~: \textcolor{darkblue}{\textbf{\ipa{dzi˩}}} 
\lhead{\firstmark}
\rhead{\botmark}

\subsection{\hspace{-0.5cm} {\Large \textcolor{darkblue}{\textbf{\ipa{mi˩ɬi˩-ʁo˩bv̩˥}}}}\hspace{0.5cm}[\kern2pt{\textcolor{darkblue}{\textbf{\ipa{xxxx non-correspondance entre le nombre de morphèmes et le nombre de tons de morphèmes}}}}\kern2pt]} \hypertarget{mi\string_BKi\string_B-Ro\string_Bbv\string_=\string_T1}{}
\markboth{\textcolor{darkblue}{\textbf{\ipa{mi˩ɬi˩-ʁo˩bv̩˥}}}}{}
\textcolor{teal}{\mytextsc{nom}} \hspace{4pt} Ton~: L+H\#.
\textcolor{PineGreen}{\selectlanguage{french}Pousse de bambou.} \zh{竹笋。}  \zh{量词}: \textcolor{darkblue}{\textbf{\ipa{kɤ˧˥}}}  \mytextsc{clf}~: \textcolor{darkblue}{\textbf{\ipa{kɤ˧˥}}} 
\lhead{\firstmark}
\rhead{\botmark}

\subsection{\hspace{-0.5cm} {\Large \textcolor{darkblue}{\textbf{\ipa{mi˩mo˩}}}}\hspace{0.5cm}[\kern2pt{\textcolor{darkblue}{\textbf{\ipa{mi˧mo˧}}}}\kern2pt]} \hypertarget{mi\string_Bmo\string_B1}{}
\markboth{\textcolor{darkblue}{\textbf{\ipa{mi˩mo˩}}}}{}
\textcolor{teal}{\mytextsc{nom}} \hspace{4pt} Ton~: L.
\textcolor{PineGreen}{\selectlanguage{french}Petit crible.} \zh{小筛子。}  \zh{量词}: \textcolor{darkblue}{\textbf{\ipa{nɑ˧}}}  \mytextsc{clf}~: \textcolor{darkblue}{\textbf{\ipa{nɑ˧}}} 
\lhead{\firstmark}
\rhead{\botmark}

\subsection{\hspace{-0.5cm} {\Large \textcolor{darkblue}{\textbf{\ipa{mi˩pʰv̩˩}}}}\hspace{0.5cm}[\kern2pt{\textcolor{darkblue}{\textbf{\ipa{mi˧pʰv̩˧}}}}\kern2pt]} \hypertarget{mi\string_Bp\string_hv\string_=\string_B1}{}
\markboth{\textcolor{darkblue}{\textbf{\ipa{mi˩pʰv̩˩}}}}{}
\textcolor{teal}{\mytextsc{nom}} \hspace{4pt} Ton~: L.
\textcolor{PineGreen}{\selectlanguage{french}Ortie.} \zh{荨麻。}  \zh{量词}: \textcolor{darkblue}{\textbf{\ipa{dzi˩}}}  \mytextsc{clf}~: \textcolor{darkblue}{\textbf{\ipa{dzi˩}}} 
\lhead{\firstmark}
\rhead{\botmark}

\subsection{\hspace{-0.5cm} {\Large \textcolor{darkblue}{\textbf{\ipa{mi˩zɯ˩}}}}\hspace{0.5cm}[\kern2pt{\textcolor{darkblue}{\textbf{\ipa{mi˧zɯ˧}}}}\kern2pt]} \hypertarget{mi\string_BzM\string_B1}{}
\markboth{\textcolor{darkblue}{\textbf{\ipa{mi˩zɯ˩}}}}{}
\textcolor{teal}{\mytextsc{nom}} \hspace{4pt} Ton~: L.
\textcolor{PineGreen}{\selectlanguage{french}Femme; aussi: nom du deuxième pilier de la maison (le pilier féminin).} \zh{女人。主屋的第二个柱子(代表女性的那个柱子)也是用这个名字。}  \zh{量词}: \textcolor{darkblue}{\textbf{\ipa{v̩˧}}}  \mytextsc{clf}~: \textcolor{darkblue}{\textbf{\ipa{v̩˧}}} 
\lhead{\firstmark}
\rhead{\botmark}

\subsection{\hspace{-0.5cm} {\Large \textcolor{darkblue}{\textbf{\ipa{mi˧˥}}}}\hspace{0.5cm}[\kern2pt{\textcolor{darkblue}{\textbf{\ipa{mi˥}}}}\kern2pt]} \hypertarget{mi\string_M\string_T1}{}
\markboth{\textcolor{darkblue}{\textbf{\ipa{mi˧˥}}}}{}
\textcolor{teal}{\mytextsc{verbe}} \hspace{4pt} Ton~: MH.
\textcolor{PineGreen}{\selectlanguage{french}Pousser.} \zh{推、拥挤。}  ¶ \textcolor{darkblue}{\textbf{\ipa{le˧-mi˧-ze˥}}} \textcolor{PineGreen}{\selectlanguage{french}\mytextsc{accomp} \string_ \mytextsc{pfv}} \zh{推开了}  
 ¶ \textcolor{darkblue}{\textbf{\ipa{le˧-mi˧˥}}} \textcolor{PineGreen}{\selectlanguage{french}\mytextsc{accomp}} \zh{推}  
 ¶ \textcolor{darkblue}{\textbf{\ipa{tʰi˧-mi˧˥}}} \textcolor{PineGreen}{\selectlanguage{french}\mytextsc{dur}} \zh{推}  
 ¶ \textcolor{darkblue}{\textbf{\ipa{tso˧\textasciitilde{}tso˧ mi˩}}} \textcolor{PineGreen}{\selectlanguage{french}pousser quelque chose} \zh{推开一个东西}  
 ¶ \textcolor{darkblue}{\textbf{\ipa{mi˩\textasciitilde{}mi˧˥}}} \textcolor{PineGreen}{\selectlanguage{french}\mytextsc{red}} \zh{\mytextsc{重叠:推、拥挤}}  
 ¶ \textcolor{darkblue}{\textbf{\ipa{mi˩\textasciitilde{}mi˧-ɻ̍˥}}} \textcolor{PineGreen}{\selectlanguage{french}\mytextsc{red} \mytextsc{inchoatif}} \zh{\mytextsc{重叠:推、拥挤}}  

\lhead{\firstmark}
\rhead{\botmark}

\subsection{\hspace{-0.5cm} {\Large \textcolor{darkblue}{\textbf{\ipa{-mi˩˧}}}}\hspace{0.5cm}[\kern2pt{\textcolor{darkblue}{\textbf{\ipa{mi˩˥}}}}\kern2pt]} \hypertarget{-mi\string_B\string_M1}{}
\markboth{\textcolor{darkblue}{\textbf{\ipa{-mi˩˧}}}}{}
\textcolor{teal}{\mytextsc{suffixe}} \hspace{4pt} Ton~: LM.
\ding{202} \textcolor{PineGreen}{\selectlanguage{french}Suffixe féminin.} \zh{阴性后缀。}  \zh{量词}: \textcolor{darkblue}{\textbf{\ipa{v̩˧}}} \ding{203} \textcolor{PineGreen}{\selectlanguage{french}Suffixe augmentatif.} \zh{指大词。}  \zh{量词}: \textcolor{darkblue}{\textbf{\ipa{v̩˧}}}  \mytextsc{clf}~: \textcolor{darkblue}{\textbf{\ipa{v̩˧}}} \textcolor{darkblue}{\textbf{\ipa{v̩˧}}} 
\lhead{\firstmark}
\rhead{\botmark}

\subsection{\hspace{-0.5cm} {\Large \textcolor{darkblue}{\textbf{\ipa{mi˩˧}}}}\hspace{0.5cm}[\kern2pt{\textcolor{darkblue}{\textbf{\ipa{mi˥}}}}\kern2pt]} \hypertarget{mi\string_B\string_M1}{}
\markboth{\textcolor{darkblue}{\textbf{\ipa{mi˩˧}}}}{}
\textcolor{teal}{\mytextsc{nom}} \hspace{4pt} Ton~: LM.
\textcolor{PineGreen}{\selectlanguage{french}Femelle (animal femelle).} \zh{母的(动物)。}  ¶ \textcolor{darkblue}{\textbf{\ipa{ʈʂʰɯ˧, | mi˩˥! / ʈʂʰɯ˧, | mi˩ ɲi˥!}}} \textcolor{PineGreen}{\selectlanguage{french}C'est une femelle!} \zh{是母的!}  
 \zh{量词}: \textcolor{darkblue}{\textbf{\ipa{v̩˧}}}  \mytextsc{clf}~: \textcolor{darkblue}{\textbf{\ipa{v̩˧}}} 
\lhead{\firstmark}
\rhead{\botmark}

\subsection{\hspace{-0.5cm} {\Large \textcolor{darkblue}{\textbf{\ipa{mje˧˥}}}}\hspace{0.5cm}[\kern2pt{\textcolor{darkblue}{\textbf{\ipa{mje˥}}}}\kern2pt]} \hypertarget{mje\string_M\string_T1}{}
\markboth{\textcolor{darkblue}{\textbf{\ipa{mje˧˥}}}}{}
\textcolor{teal}{\mytextsc{nom}} \hspace{4pt} Ton~: .
\textcolor{PineGreen}{\selectlanguage{french}Nouilles, pâtes alimentaires.} \zh{面条。}  Emprunt~: chinois  \zh{面}
 ¶ \textcolor{darkblue}{\textbf{\ipa{mjæ˧˥ | dzɯ˧-bi˧! \textasciitilde{} mjæ˧ dzɯ˧-bi˧! \textasciitilde{} mjæ˧ dzɯ˥-bi˩!}}} \textcolor{PineGreen}{\selectlanguage{french}On va manger des nouilles!} \zh{吃面吧!}  
 ¶ \textcolor{darkblue}{\textbf{\ipa{mjæ˧˥ | ɖɯ˧-qʰwɤ˧ tɕɤ˥}}} \textcolor{PineGreen}{\selectlanguage{french}faire cuire un bol de nouilles} \zh{煮一碗面}  
 ¶ \textcolor{darkblue}{\textbf{\ipa{mjæ˧ hwæ˥-bi˩}}} \textcolor{PineGreen}{\selectlanguage{french}(on) va acheter des nouilles} \zh{要买面}  

\lhead{\firstmark}
\rhead{\botmark}

\subsection{\hspace{-0.5cm} {\Large \textcolor{darkblue}{\textbf{\ipa{mo˥\textsubscript{a}}}}}\hspace{0.5cm}[\kern2pt{\textcolor{darkblue}{\textbf{\ipa{mo˥}}}}\kern2pt]} \hypertarget{mo\string_Ta1}{}
\markboth{\textcolor{darkblue}{\textbf{\ipa{mo˥\textsubscript{a}}}}}{}
\textcolor{teal}{\mytextsc{classificateur}} \hspace{4pt} Ton~: H\textsubscript{a}.
\textcolor{PineGreen}{\selectlanguage{french}Acre chinois: un sixième d'acre; 0,0667 hectare.} \zh{量词:地(一亩地)(汉语借词)。}  Emprunt~: chinois  \zh{亩}

\lhead{\firstmark}
\rhead{\botmark}

\subsection{\hspace{-0.5cm} {\Large \textcolor{darkblue}{\textbf{\ipa{mo˧\textsubscript{a}}}}}\hspace{0.5cm}[\kern2pt{\textcolor{darkblue}{\textbf{\ipa{mo˥}}}}\kern2pt]} \hypertarget{mo\string_Ma1}{}
\markboth{\textcolor{darkblue}{\textbf{\ipa{mo˧\textsubscript{a}}}}}{}
\textcolor{teal}{\mytextsc{classificateur}} \hspace{4pt} Ton~: M\textsubscript{a}.
\textcolor{PineGreen}{\selectlanguage{french}Classificateur des cadavres et tombeaux.} \zh{量词:尸体。} 
\lhead{\firstmark}
\rhead{\botmark}

\subsection{\hspace{-0.5cm} {\Large \textcolor{darkblue}{\textbf{\ipa{mo˧ɖʐv̩˥}}}}\hspace{0.5cm}[\kern2pt{\textcolor{darkblue}{\textbf{\ipa{mo˧ɖʐv̩˥}}}}\kern2pt]} \hypertarget{mo\string_Md`z`v\string_=\string_T1}{}
\markboth{\textcolor{darkblue}{\textbf{\ipa{mo˧ɖʐv̩˥}}}}{}
\textcolor{teal}{\mytextsc{nom}} \hspace{4pt} Ton~: H\#.
\textcolor{PineGreen}{\selectlanguage{french}Morille: champignon comestible, particulièrement apprécié pour sa texture.} \zh{羊肚菌。} Dialecte chinois local~:\zh{羊菌。} ¶ \textcolor{darkblue}{\textbf{\ipa{ʂɯ˧-ɬi˧mi˧, | mo˧ɖʐv̩˥!}}} \textcolor{PineGreen}{\selectlanguage{french}Le septième mois, c'est la saison des morilles! (cette sorte de champignon) (Il pousse des paquets de champignons si compacts qu'on n'arrive même pas à les séparer.)} \zh{七月份,是羊肚菌的季节!}  
 ¶ \textcolor{darkblue}{\textbf{\ipa{ʂɯ˧-ɬi˧mi˧ | mo˧ɖʐv̩˥-ne˩-ʝi˩-zo˩!}}} \textcolor{PineGreen}{\selectlanguage{french}“Il vous en vient comme des morilles au septième mois!” Commentaire humoristique: ce qu'on disait au sujet des gens qui avaient beaucoup d'enfants, qui avaient enfant après enfant: “Ca prolifère comme les morilles au septième mois!”} \zh{(你们家孩子)生得像七月份的羊肚菌一样!(来形容一家有很多孩子出生,一个又一个。在永宁地区,七月份羊肚菌很多。)}  

\lhead{\firstmark}
\rhead{\botmark}

\subsection{\hspace{-0.5cm} {\Large \textcolor{darkblue}{\textbf{\ipa{mo˧jo˩-mi˩}}}}\hspace{0.5cm}[\kern2pt{\textcolor{darkblue}{\textbf{\ipa{mo˧jo˩mi˧}}}}\kern2pt]} \hypertarget{mo\string_Mjo\string_B-mi\string_B1}{}
\markboth{\textcolor{darkblue}{\textbf{\ipa{mo˧jo˩-mi˩}}}}{}
\textcolor{teal}{\mytextsc{nom}} \hspace{4pt} Ton~: L\#-.
\textcolor{PineGreen}{\selectlanguage{french}Chouette, hibou (toutes les espèces de \textit{bubo} et \textit{strix}).} \zh{猫头鹰。}  \zh{量词}: \textcolor{darkblue}{\textbf{\ipa{mi˩}}}  \mytextsc{clf}~: \textcolor{darkblue}{\textbf{\ipa{mi˩}}} 
\lhead{\firstmark}
\rhead{\botmark}

\subsection{\hspace{-0.5cm} {\Large \textcolor{darkblue}{\textbf{\ipa{mo˧jo˩mi˩-pʰv̩˩}}}}\hspace{0.5cm}[\kern2pt{\textcolor{darkblue}{\textbf{\ipa{mo˧jo˧mi˩pʰv̩˧}}}}\kern2pt]} \hypertarget{mo\string_Mjo\string_Bmi\string_B-p\string_hv\string_=\string_B1}{}
\markboth{\textcolor{darkblue}{\textbf{\ipa{mo˧jo˩mi˩-pʰv̩˩}}}}{}
\textcolor{teal}{\mytextsc{nom}} \hspace{4pt} Ton~: L\#-.
\textcolor{PineGreen}{\selectlanguage{french}Hibou mâle.} \zh{公猫头鹰。}  \zh{量词}: \textcolor{darkblue}{\textbf{\ipa{mi˩}}}  \mytextsc{clf}~: \textcolor{darkblue}{\textbf{\ipa{mi˩}}} 
\lhead{\firstmark}
\rhead{\botmark}

\subsection{\hspace{-0.5cm} {\Large \textcolor{darkblue}{\textbf{\ipa{mo˧jo˩mi˩-zo˩}}}}\hspace{0.5cm}[\kern2pt{\textcolor{darkblue}{\textbf{\ipa{mo˧jo˧mi˩zo˧}}}}\kern2pt]} \hypertarget{mo\string_Mjo\string_Bmi\string_B-zo\string_B1}{}
\markboth{\textcolor{darkblue}{\textbf{\ipa{mo˧jo˩mi˩-zo˩}}}}{}
\textcolor{teal}{\mytextsc{nom}} \hspace{4pt} Ton~: L\#-.
\textcolor{PineGreen}{\selectlanguage{french}Bébé hibou.} \zh{小的猫头鹰。}  \zh{量词}: \textcolor{darkblue}{\textbf{\ipa{ɭɯ˧}}}  \mytextsc{clf}~: \textcolor{darkblue}{\textbf{\ipa{ɭɯ˧}}} 
\lhead{\firstmark}
\rhead{\botmark}

\subsection{\hspace{-0.5cm} {\Large \textcolor{darkblue}{\textbf{\ipa{mo˧kɤ˩}}}}\hspace{0.5cm}[\kern2pt{\textcolor{darkblue}{\textbf{\ipa{mo˧kɤ˩}}}}\kern2pt]} \hypertarget{mo\string_Mk7\string_B1}{}
\markboth{\textcolor{darkblue}{\textbf{\ipa{mo˧kɤ˩}}}}{}
\textcolor{teal}{\mytextsc{nom}} \hspace{4pt} Ton~: L\#.
\textcolor{PineGreen}{\selectlanguage{french}Azalée. Cette plante est perçue comme vénéneuse; on ne consomme pas les champignons qui poussent dans son voisinage.} \zh{杜鹃花、踯躅、山石榴、照山红、唐杜鹃。} Dialecte chinois local~:\zh{杨花木。} ¶ \textcolor{darkblue}{\textbf{\ipa{mo˧kɤ˩-bæ˩bæ˩}}} \textcolor{PineGreen}{\selectlanguage{french}fleurs d'azalée} \zh{杜鹃花}  

\lhead{\firstmark}
\rhead{\botmark}

\subsection{\hspace{-0.5cm} {\Large \textcolor{darkblue}{\textbf{\ipa{mo˧ɬɑ˥}}}}\hspace{0.5cm}[\kern2pt{\textcolor{darkblue}{\textbf{\ipa{mo˧ɬɑ˥}}}}\kern2pt]} \hypertarget{mo\string_MKA\string_T1}{}
\markboth{\textcolor{darkblue}{\textbf{\ipa{mo˧ɬɑ˥}}}}{}
\textcolor{teal}{\mytextsc{nom}} \hspace{4pt} Ton~: H\#.
\textcolor{PineGreen}{\selectlanguage{french}Médisance, calomnie.} \zh{诬蔑、坏话。}  ¶ \textcolor{darkblue}{\textbf{\ipa{mo˧ɬɑ˥ ʐwɤ˩}}} \textcolor{PineGreen}{\selectlanguage{french}médire de quelqu'un, calomnier quelqu'un} \zh{说人的坏话}  

\lhead{\firstmark}
\rhead{\botmark}

\subsection{\hspace{-0.5cm} {\Large \textcolor{darkblue}{\textbf{\ipa{mo˧mo˥}}}}\hspace{0.5cm}[\kern2pt{\textcolor{darkblue}{\textbf{\ipa{mo˧mo˥}}}}\kern2pt]} \hypertarget{mo\string_Mmo\string_T1}{}
\markboth{\textcolor{darkblue}{\textbf{\ipa{mo˧mo˥}}}}{}
\textcolor{teal}{\mytextsc{nom}} \hspace{4pt} Ton~: H\#.
\textcolor{PineGreen}{\selectlanguage{french}Petits pains (pouvant contenir de la farine de maïs; mais surtout farine de blé) cuits à la vapeur.} \zh{馒头、包子。}  \zh{量词}: \textcolor{darkblue}{\textbf{\ipa{ɭɯ˧}}}  \mytextsc{clf}~: \textcolor{darkblue}{\textbf{\ipa{ɭɯ˧}}} 
\lhead{\firstmark}
\rhead{\botmark}

\subsection{\hspace{-0.5cm} {\Large \textcolor{darkblue}{\textbf{\ipa{mo˧nɑ˥}}} \textsubscript{1}}\hspace{0.5cm}[\kern2pt{\textcolor{darkblue}{\textbf{\ipa{mo˧nɑ˥}}}}\kern2pt]} \hypertarget{mo\string_MnA\string_T1}{}
\markboth{\textcolor{darkblue}{\textbf{\ipa{mo˧nɑ˥}}} \textsubscript{1}}{}
\textcolor{teal}{\mytextsc{nom}} \hspace{4pt} Ton~: H\#.
\textcolor{PineGreen}{\selectlanguage{french}Médisance, ragot.} \zh{闲话。}  ¶ \textcolor{darkblue}{\textbf{\ipa{mo˧nɑ˥ ʐwɤ˩}}} \textcolor{PineGreen}{\selectlanguage{french}ragoter, médire} \zh{八卦、讲别人的坏话}  
 ¶ \textcolor{darkblue}{\textbf{\ipa{ʈʂʰɯ˧ | ɖɯ˧-ɲi˥ | mo˧nɑ˥ ʐwɤ˩-dʑo˩!}}} \textcolor{PineGreen}{\selectlanguage{french}Il/elle ragote toute la journée!} \zh{他一天到晚都在八卦!}  
 ¶ \textcolor{darkblue}{\textbf{\ipa{mo˧nɑ˥-ɕi˩mi˩}}} \textcolor{PineGreen}{\selectlanguage{french}même sens: ragot, médisance} \zh{同上:八卦、坏话}  
 ¶ \textcolor{darkblue}{\textbf{\ipa{mo˧nɑ˥-ɕi˩mi˩ ʐwɤ˩}}} \textcolor{PineGreen}{\selectlanguage{french}ragoter, médire} \zh{八卦、讲别人的坏话}  
 ¶ \textcolor{darkblue}{\textbf{\ipa{hĩ˧ | ʈʂʰɯ˧-v̩˧, | mo˧nɑ˥-ɕi˩mi˩ | ɖɯ˧-v̩˧ ɲi˩!}}} \textcolor{PineGreen}{\selectlanguage{french}C'est un ragoteur, il est médisant} \zh{他爱八卦、爱说别人坏话}  

\lhead{\firstmark}
\rhead{\botmark}

\subsection{\hspace{-0.5cm} {\Large \textcolor{darkblue}{\textbf{\ipa{mo˧nɑ˥}}} \textsubscript{2}}\hspace{0.5cm}[\kern2pt{\textcolor{darkblue}{\textbf{\ipa{mo˧nɑ˥}}}}\kern2pt]} \hypertarget{mo\string_MnA\string_T2}{}
\markboth{\textcolor{darkblue}{\textbf{\ipa{mo˧nɑ˥}}} \textsubscript{2}}{}
\textcolor{teal}{\mytextsc{nom}} \hspace{4pt} Ton~: H\#.
\textcolor{PineGreen}{\selectlanguage{french}Paille hachée, utilisée dans la préparation des légumes en saumure.} \zh{剁成粉的秸杆。}  ¶ \textcolor{darkblue}{\textbf{\ipa{mv˩zɯ˩-mo˩nɑ˥}}} \textcolor{PineGreen}{\selectlanguage{french}paille d'avoine hachée} \zh{剁成粉的燕麦秸杆}  

\lhead{\firstmark}
\rhead{\botmark}

\subsection{\hspace{-0.5cm} {\Large \textcolor{darkblue}{\textbf{\ipa{mo˧qʰwɤ˥}}}}\hspace{0.5cm}[\kern2pt{\textcolor{darkblue}{\textbf{\ipa{mo˧qʰwɤ˥}}}}\kern2pt]} \hypertarget{mo\string_Mq\string_hw7\string_T1}{}
\markboth{\textcolor{darkblue}{\textbf{\ipa{mo˧qʰwɤ˥}}}}{}
\textcolor{teal}{\mytextsc{nom}} \hspace{4pt} Ton~: H\#.
\textcolor{PineGreen}{\selectlanguage{french}Navette en bois d'un pont de corde: la navette coulisse sur la corde; passager, cheval ou chargement y sont attachés.} \zh{溜索上往返移动的木头梭。}  \zh{量词}: \textcolor{darkblue}{\textbf{\ipa{ɭɯ˧}}}  \mytextsc{clf}~: \textcolor{darkblue}{\textbf{\ipa{ɭɯ˧}}} 
\lhead{\firstmark}
\rhead{\botmark}

\subsection{\hspace{-0.5cm} {\Large \textcolor{darkblue}{\textbf{\ipa{mo˧qʰwɤ˧˥}}}}\hspace{0.5cm}[\kern2pt{\textcolor{darkblue}{\textbf{\ipa{mo˧qʰwɤ˧˥}}}}\kern2pt]} \hypertarget{mo\string_Mq\string_hw7\string_M\string_T1}{}
\markboth{\textcolor{darkblue}{\textbf{\ipa{mo˧qʰwɤ˧˥}}}}{}
\textcolor{teal}{\mytextsc{adjectif}} \hspace{4pt} Ton~: MH.
\textcolor{PineGreen}{\selectlanguage{french}Qui a bon appétit, qui a un solide appétit; gourmand, vorace (peut être neutre, ou franchement négatif).} \zh{胃口好,或:贪吃。} 
\lhead{\firstmark}
\rhead{\botmark}

\subsection{\hspace{-0.5cm} {\Large \textcolor{darkblue}{\textbf{\ipa{mo˩}}}}\hspace{0.5cm}[\kern2pt{\textcolor{darkblue}{\textbf{\ipa{mo˩˥}}}}\kern2pt]} \hypertarget{mo\string_B1}{}
\markboth{\textcolor{darkblue}{\textbf{\ipa{mo˩}}}}{}
\textcolor{teal}{\mytextsc{particule}} \textcolor{teal}{\mytextsc{de}} \textcolor{teal}{\mytextsc{discours}} \hspace{4pt} Ton~: L.
\textcolor{PineGreen}{\selectlanguage{french}Particule indiquant l'invitation à faire quelque chose.} \zh{句尾助词:请……。}  ¶ \textcolor{darkblue}{\textbf{\ipa{no˧ | ɖɯ˧-ʈʰɯ˩-ɻ̍˩ mo˩!}}} \textcolor{PineGreen}{\selectlanguage{french}Bois donc un peu!} \zh{请你喝一点!}  

\lhead{\firstmark}
\rhead{\botmark}

\subsection{\hspace{-0.5cm} {\Large \textcolor{darkblue}{\textbf{\ipa{mo˩\textsubscript{a}}}} \textsubscript{1}}\hspace{0.5cm}[\kern2pt{\textcolor{darkblue}{\textbf{\ipa{mo˩˥}}}}\kern2pt]} \hypertarget{mo\string_Ba1}{}
\markboth{\textcolor{darkblue}{\textbf{\ipa{mo˩\textsubscript{a}}}} \textsubscript{1}}{}
\textcolor{teal}{\mytextsc{adjectif}} \hspace{4pt} Ton~: L\textsubscript{a}.
\textcolor{PineGreen}{\selectlanguage{french}Vieux, âgé.} \zh{年老。}  ¶ \textcolor{darkblue}{\textbf{\ipa{mo˩ hĩ˩˥}}} \textcolor{PineGreen}{\selectlanguage{french}vieille personne} \zh{老人}  
 ¶ \textcolor{darkblue}{\textbf{\ipa{si˧ mo˥}}} \textcolor{PineGreen}{\selectlanguage{french}vieux bois, vieil arbre} \zh{老树、老木头}  
 ¶ \textcolor{darkblue}{\textbf{\ipa{le˧-mo˩-ze˩}}} \textcolor{PineGreen}{\selectlanguage{french}\mytextsc{accomp} \string_ \mytextsc{pfv}: (il/elle) a vieilli} \zh{\mytextsc{accomp} \string_ \mytextsc{pfv}}  
 ¶ \textcolor{darkblue}{\textbf{\ipa{le˧-mo˩-hĩ˩}}} \textcolor{PineGreen}{\selectlanguage{french}vieille personne, personne qui a vieilli} \zh{老了的人}  
 ¶ \textcolor{darkblue}{\textbf{\ipa{hĩ˧ mo˥, | õ˧-di˧ fv̩˥! | ʐwæ˧ mo˥, | to˩ do˩ ɖwæ˥!}}} \textcolor{PineGreen}{\selectlanguage{french}“Les vieilles personnes aiment leur chez-eux; les vieux chevaux ont peur de grimper les pentes!” (Sens: avec l'âge, on devient moins entreprenant.)} \zh{老人爱自家,老马怕山坡!(谚语,描写不爱到处跑的老年人)}  
 ¶ \textcolor{darkblue}{\textbf{\ipa{lv̩˧ mo˥ F | dʑɯ˧ | le˧-qv̩˩; | si˧ mo˥ F | le˧-dze˩-kv̩˩! | no˧ F | ə˧tse˧ | le˧-ʂɯ˧-mɤ˧-tʰɑ˧˥ | di˩!}}} \textcolor{PineGreen}{\selectlanguage{french}Les vieilles pierres, le courant les emporte; le vieux bois, on le coupe! Alors pourquoi toi te voilà qui ne veux pas mourir! (Moquerie à l'égard d'une personne très âgée.)} \zh{老石头要被河流冲走,老木头要被砍掉。你呢,怎么还不死? (嘲笑一个年龄很高的人。摩梭传统中,人的寿命是六十岁:过了七十岁的人,被认为是已经到达了命的尽头。)}  

\lhead{\firstmark}
\rhead{\botmark}

\subsection{\hspace{-0.5cm} {\Large \textcolor{darkblue}{\textbf{\ipa{mo˩\textsubscript{a}}}} \textsubscript{2}}\hspace{0.5cm}[\kern2pt{\textcolor{darkblue}{\textbf{\ipa{mo˩˥}}}}\kern2pt]} \hypertarget{mo\string_Ba2}{}
\markboth{\textcolor{darkblue}{\textbf{\ipa{mo˩\textsubscript{a}}}} \textsubscript{2}}{}
\textcolor{teal}{\mytextsc{verbe}} \hspace{4pt} Ton~: L\textsubscript{a}.
\textcolor{PineGreen}{\selectlanguage{french}Mourir.} \zh{死、去世。}  ¶ \textcolor{darkblue}{\textbf{\ipa{mɤ˧-mo˩-sɯ˩!}}} \textcolor{PineGreen}{\selectlanguage{french}(Il n'est) pas encore mort!} \zh{还没死!}  
 ¶ \textcolor{darkblue}{\textbf{\ipa{si˧ mo˩}}} \textcolor{PineGreen}{\selectlanguage{french}bois mort} \zh{老干柴(直译:死了的木头)}  

\lhead{\firstmark}
\rhead{\botmark}

\subsection{\hspace{-0.5cm} {\Large \textcolor{darkblue}{\textbf{\ipa{mo˩kv̩\#˥}}}}\hspace{0.5cm}[\kern2pt{\textcolor{darkblue}{\textbf{\ipa{mo˩kv̩˥}}}}\kern2pt]} \hypertarget{mo\string_Bkv\string_=\#\string_T1}{}
\markboth{\textcolor{darkblue}{\textbf{\ipa{mo˩kv̩\#˥}}}}{}
\textcolor{teal}{\mytextsc{nom}} \hspace{4pt} Ton~: LM+\#H.
\textcolor{PineGreen}{\selectlanguage{french}Sorte de champignon qui pousse sur les chênes (sur les troncs tombés, sur les arbres morts).} \zh{蘑菇:长在倒在地上的树(如青冈等树木)上的菌子(汉语借词)。}  Emprunt~: chinois  \zh{蘑菇}
 ¶ \textcolor{darkblue}{\textbf{\ipa{mo˩kv̩˥, | si˧dzi˩-mo˩!}}} \textcolor{PineGreen}{\selectlanguage{french}\textcolor{darkblue}{\textbf{\ipa{/mo˩kv̩\#˥/}}}, ça désigne les champignons qui pousse sur les arbres! (littéralement: “les champignons d'arbres”, par opposition aux “champignons de terre”)} \zh{\textcolor{darkblue}{\textbf{\ipa{/mo˩kv̩\#˥/}}},指的是长在(倒在地上的)树上的菌子!}  

\lhead{\firstmark}
\rhead{\botmark}

\subsection{\hspace{-0.5cm} {\Large \textcolor{darkblue}{\textbf{\ipa{mo˩ɻ\#˥}}}}\hspace{0.5cm}[\kern2pt{\textcolor{darkblue}{\textbf{\ipa{mo˩ɻ˥}}}}\kern2pt]} \hypertarget{mo\string_Br£`\#\string_T1}{}
\markboth{\textcolor{darkblue}{\textbf{\ipa{mo˩ɻ\#˥}}}}{}
\textcolor{teal}{\mytextsc{nom}} \hspace{4pt} Ton~: LM+\#H.
\textcolor{PineGreen}{\selectlanguage{french}Champignon noir.} \zh{木耳(汉语借词)。}  Emprunt~: chinois  \zh{木耳}

\lhead{\firstmark}
\rhead{\botmark}

\subsection{\hspace{-0.5cm} {\Large \textcolor{darkblue}{\textbf{\ipa{mo˩zo\#˥}}}}\hspace{0.5cm}[\kern2pt{\textcolor{darkblue}{\textbf{\ipa{mo˩zo˥}}}}\kern2pt]} \hypertarget{mo\string_Bzo\#\string_T1}{}
\markboth{\textcolor{darkblue}{\textbf{\ipa{mo˩zo\#˥}}}}{}
\textcolor{teal}{\mytextsc{nom}} \hspace{4pt} Ton~: LM+\#H.
\textcolor{PineGreen}{\selectlanguage{french}Militaire, soldat.} \zh{士兵。}  ¶ \textcolor{darkblue}{\textbf{\ipa{mo˩zo˧ ʝi˧-hɯ˧ ɲi˥!}}} \textcolor{PineGreen}{\selectlanguage{french}Il est parti à l'armée! / Il s'est fait soldat!} \zh{当兵去了!}  
 \zh{量词}: \textcolor{darkblue}{\textbf{\ipa{v̩˧}}}  \mytextsc{clf}~: \textcolor{darkblue}{\textbf{\ipa{v̩˧}}} 
\lhead{\firstmark}
\rhead{\botmark}

\subsection{\hspace{-0.5cm} {\Large \textcolor{darkblue}{\textbf{\ipa{mo˧˥}}}}\hspace{0.5cm}[\kern2pt{\textcolor{darkblue}{\textbf{\ipa{mo˧˥}}}}\kern2pt]} \hypertarget{mo\string_M\string_T1}{}
\markboth{\textcolor{darkblue}{\textbf{\ipa{mo˧˥}}}}{}
\textcolor{teal}{\mytextsc{nom}} \hspace{4pt} Ton~: MH.
\textcolor{PineGreen}{\selectlanguage{french}Champignon.} \zh{菌子、蘑菇。}  \zh{量词}: \textcolor{darkblue}{\textbf{\ipa{ɭɯ˧}}} \textcolor{darkblue}{\textbf{\ipa{mo˧˥}}}  \mytextsc{clf}~: \textcolor{darkblue}{\textbf{\ipa{ɭɯ˧}}} \textcolor{darkblue}{\textbf{\ipa{mo˧˥}}} \textit{Voir~:} \hyperlink{}{\textcolor{darkblue}{\textbf{\ipa{mo˧˥\textsubscript{a}}}}} 
\lhead{\firstmark}
\rhead{\botmark}

\subsection{\hspace{-0.5cm} {\Large \textcolor{darkblue}{\textbf{\ipa{mo˧˥\textsubscript{a}}}}}\hspace{0.5cm}[\kern2pt{\textcolor{darkblue}{\textbf{\ipa{mo˧˥}}}}\kern2pt]} \hypertarget{mo\string_M\string_Ta1}{}
\markboth{\textcolor{darkblue}{\textbf{\ipa{mo˧˥\textsubscript{a}}}}}{}
\textcolor{teal}{\mytextsc{classificateur}} \hspace{4pt} Ton~: MH\textsubscript{a}.
\textcolor{PineGreen}{\selectlanguage{french}Auto-classificateur des champignons.} \zh{量词:蘑菇(一只)。} \textit{Voir~:} \hyperlink{}{\textcolor{darkblue}{\textbf{\ipa{mo˧˥}}}} 
\lhead{\firstmark}
\rhead{\botmark}

\subsection{\hspace{-0.5cm} {\Large \textcolor{darkblue}{\textbf{\ipa{mv̩˩˥}}}}\hspace{0.5cm}[\kern2pt{\textcolor{darkblue}{\textbf{\ipa{mv̩˩˥}}}}\kern2pt]} \hypertarget{mv\string_=\string_B\string_T1}{}
\markboth{\textcolor{darkblue}{\textbf{\ipa{mv̩˩˥}}}}{}
\textcolor{teal}{\mytextsc{nom}} \hspace{4pt} Ton~: LH.
\textcolor{PineGreen}{\selectlanguage{french}Fille.} \zh{女儿。}  \zh{量词}: \textcolor{darkblue}{\textbf{\ipa{v̩˧}}}  \mytextsc{clf}~: \textcolor{darkblue}{\textbf{\ipa{v̩˧}}} 
\lhead{\firstmark}
\rhead{\botmark}

\subsection{\hspace{-0.5cm} {\Large \textcolor{darkblue}{\textbf{\ipa{mv̩˧}}}}\hspace{0.5cm}[\kern2pt{\textcolor{darkblue}{\textbf{\ipa{mv̩˥}}}}\kern2pt]} \hypertarget{mv\string_=\string_M1}{}
\markboth{\textcolor{darkblue}{\textbf{\ipa{mv̩˧}}}}{}
\textcolor{teal}{\mytextsc{nom}} \hspace{4pt} Ton~: M.
\textcolor{PineGreen}{\selectlanguage{french}Nom (nom de famille ou prénom: nom donné à un individu).} \zh{姓名。}  ¶ \textcolor{darkblue}{\textbf{\ipa{ɑ˩ʁo˧-bv̩˧ | mv̩˧ (+ɲi˩)}}} \textcolor{PineGreen}{\selectlanguage{french}c'est le nom de la famille / c'est mon nom de famille!} \zh{这是家里的姓! / 这是我家的姓!}  
 ¶ \textcolor{darkblue}{\textbf{\ipa{njɤ˧ | mv̩˧ ɖɯ˧-kʰwɤ˥ | ʂe˧-zo˧-ho˩!}}} \textcolor{PineGreen}{\selectlanguage{french}Il va falloir que j'aille chercher un nom (auprès des moines du monastère) (pour un enfant qui vient de naître)!} \zh{我得去(向大寺里的和尚)求一个名字(给刚出生的孩子起名)}  

\lhead{\firstmark}
\rhead{\botmark}

\subsection{\hspace{-0.5cm} {\Large \textcolor{darkblue}{\textbf{\ipa{mv̩˧}}}}\hspace{0.5cm}[\kern2pt{\textcolor{darkblue}{\textbf{\ipa{mv̩˥}}}}\kern2pt]} \hypertarget{mv\string_=\string_M1}{}
\markboth{\textcolor{darkblue}{\textbf{\ipa{mv̩˧}}}}{}
\textcolor{teal}{\mytextsc{particule}} \textcolor{teal}{\mytextsc{de}} \textcolor{teal}{\mytextsc{discours}} \hspace{4pt} Ton~: M.
\textcolor{PineGreen}{\selectlanguage{french}Particule finale affirmative.} \zh{句尾助词,表示肯定:“嘛”。} 
\lhead{\firstmark}
\rhead{\botmark}

\subsection{\hspace{-0.5cm} {\Large \textcolor{darkblue}{\textbf{\ipa{mv̩˥}}}}\hspace{0.5cm}[\kern2pt{\textcolor{darkblue}{\textbf{\ipa{mv̩˥}}}}\kern2pt]} \hypertarget{mv\string_=\string_T1}{}
\markboth{\textcolor{darkblue}{\textbf{\ipa{mv̩˥}}}}{}
\textcolor{teal}{\mytextsc{verbe}} \hspace{4pt} Ton~: H.
\ding{202} \textcolor{PineGreen}{\selectlanguage{french}Entendre.} \zh{懂,听见。}  ¶ \textcolor{darkblue}{\textbf{\ipa{njɤ˧ | le˧-mv̩˥-ze˩}}} \textcolor{PineGreen}{\selectlanguage{french}j'ai entendu} \zh{我听见了}  
\ding{203} \textcolor{PineGreen}{\selectlanguage{french}Comprendre.} \zh{懂。}  ¶ \textcolor{darkblue}{\textbf{\ipa{njɤ˧ | le˧-mv̩˥-ze˩}}} \textcolor{PineGreen}{\selectlanguage{french}j'ai compris} \zh{我懂了}  

\lhead{\firstmark}
\rhead{\botmark}

\subsection{\hspace{-0.5cm} {\Large \textcolor{darkblue}{\textbf{\ipa{mv̩˥}}} \textsubscript{1}}\hspace{0.5cm}[\kern2pt{\textcolor{darkblue}{\textbf{\ipa{mv̩˥}}}}\kern2pt]} \hypertarget{mv\string_=\string_T1}{}
\markboth{\textcolor{darkblue}{\textbf{\ipa{mv̩˥}}} \textsubscript{1}}{}
\textcolor{teal}{\mytextsc{nom}} \hspace{4pt} Ton~: \#H.
\textcolor{PineGreen}{\selectlanguage{french}Ciel.} \zh{天。}  ¶ \textcolor{darkblue}{\textbf{\ipa{mv̩˧tʰv̩˧(-ze˩)}}} \textcolor{PineGreen}{\selectlanguage{french}il fait clair, il fait grand jour, le ciel est clair} \zh{天晴,天色亮}  
 ¶ \textcolor{darkblue}{\textbf{\ipa{hĩ˧-ɳɯ˩ mɤ˩-do˩, | mv̩˧-ɳɯ˩ | do˩˥!}}} \textcolor{PineGreen}{\selectlanguage{french}“Ce que les hommes ne voient pas, le ciel le voit!” (Sens: une bonne action n'est jamais perdue, et une mauvaise reçoit sa punition dans le monde d'en haut.)} \zh{“人看不见的,老天能看见!”}  
 \zh{量词}: \textcolor{darkblue}{\textbf{\ipa{ɭɯ˧}}}  \mytextsc{clf}~: \textcolor{darkblue}{\textbf{\ipa{ɭɯ˧}}} 
\lhead{\firstmark}
\rhead{\botmark}

\subsection{\hspace{-0.5cm} {\Large \textcolor{darkblue}{\textbf{\ipa{mv̩˥}}} \textsubscript{2}}\hspace{0.5cm}[\kern2pt{\textcolor{darkblue}{\textbf{\ipa{mv̩˥}}}}\kern2pt]} \hypertarget{mv\string_=\string_T2}{}
\markboth{\textcolor{darkblue}{\textbf{\ipa{mv̩˥}}} \textsubscript{2}}{}
\textcolor{teal}{\mytextsc{nom}} \hspace{4pt} Ton~: \#H.
\textcolor{PineGreen}{\selectlanguage{french}Feu.} \zh{火。}  ¶ \textcolor{darkblue}{\textbf{\ipa{mv̩˧ kʰɯ˩}}} \textcolor{PineGreen}{\selectlanguage{french}allumer un feu, faire un feu} \zh{点火}  
 \zh{量词}: \textcolor{darkblue}{\textbf{\ipa{æ̃˩}}}  \mytextsc{clf}~: \textcolor{darkblue}{\textbf{\ipa{æ̃˩}}} 
\lhead{\firstmark}
\rhead{\botmark}

\subsection{\hspace{-0.5cm} {\Large \textcolor{darkblue}{\textbf{\ipa{mv̩˩\textsubscript{a}}}} \textsubscript{1}}\hspace{0.5cm}[\kern2pt{\textcolor{darkblue}{\textbf{\ipa{mv̩˥}}}}\kern2pt]} \hypertarget{mv\string_=\string_Ba1}{}
\markboth{\textcolor{darkblue}{\textbf{\ipa{mv̩˩\textsubscript{a}}}} \textsubscript{1}}{}
\textcolor{teal}{\mytextsc{verbe}} \hspace{4pt} Ton~: L\textsubscript{a}.
\textcolor{PineGreen}{\selectlanguage{french}Souffler (ex.: souffler sur le feu, attiser le feu; souffler dans un instrument à vent).} \zh{吹(灰,乐器)。}  ¶ \textcolor{darkblue}{\textbf{\ipa{mv̩˧\textasciitilde{}mv̩˥(-ze˩)}}} \textcolor{PineGreen}{\selectlanguage{french}\mytextsc{red}} \zh{\mytextsc{重叠:吹吹}}  
 ¶ \textcolor{darkblue}{\textbf{\ipa{ʝi˧qʰv̩˧ mv̩˥}}} \textcolor{PineGreen}{\selectlanguage{french}souffler dans une corne} \zh{吹号角}  

\lhead{\firstmark}
\rhead{\botmark}

\subsection{\hspace{-0.5cm} {\Large \textcolor{darkblue}{\textbf{\ipa{mv̩˩\textsubscript{a}}}} \textsubscript{2}}\hspace{0.5cm}[\kern2pt{\textcolor{darkblue}{\textbf{\ipa{mv̩˩˥}}}}\kern2pt]} \hypertarget{mv\string_=\string_Ba2}{}
\markboth{\textcolor{darkblue}{\textbf{\ipa{mv̩˩\textsubscript{a}}}} \textsubscript{2}}{}
\textcolor{teal}{\mytextsc{verbe}} \hspace{4pt} Ton~: L\textsubscript{a}.
\textcolor{PineGreen}{\selectlanguage{french}Emporter (le courant emporte un nageur), balayer (une vague balaie une épave de bateau).} \zh{冲(走)。} 
\lhead{\firstmark}
\rhead{\botmark}

\subsection{\hspace{-0.5cm} {\Large \textcolor{darkblue}{\textbf{\ipa{mv̩˩\textsubscript{a}}}} \textsubscript{3}}\hspace{0.5cm}[\kern2pt{\textcolor{darkblue}{\textbf{\ipa{mv̩˩˥}}}}\kern2pt]} \hypertarget{mv\string_=\string_Ba3}{}
\markboth{\textcolor{darkblue}{\textbf{\ipa{mv̩˩\textsubscript{a}}}} \textsubscript{3}}{}
\textcolor{teal}{\mytextsc{adjectif}} \hspace{4pt} Ton~: L\textsubscript{a}.
\ding{202} \textcolor{PineGreen}{\selectlanguage{french}Mûr (produit agricole).} \zh{熟、成熟(植物、水果)。}  ¶ \textcolor{darkblue}{\textbf{\ipa{mv̩˩-hĩ˩˥}}} \textcolor{PineGreen}{\selectlanguage{french}\mytextsc{rel}} \zh{熟的}  
\ding{203} \textcolor{PineGreen}{\selectlanguage{french}Cuit (aliment).} \zh{熟(食物)。} 
\lhead{\firstmark}
\rhead{\botmark}

\subsection{\hspace{-0.5cm} {\Large \textcolor{darkblue}{\textbf{\ipa{mv̩˩\textsubscript{a}}}} \textsubscript{4}}\hspace{0.5cm}[\kern2pt{\textcolor{darkblue}{\textbf{\ipa{mv̩˩˥}}}}\kern2pt]} \hypertarget{mv\string_=\string_Ba4}{}
\markboth{\textcolor{darkblue}{\textbf{\ipa{mv̩˩\textsubscript{a}}}} \textsubscript{4}}{}
\textcolor{teal}{\mytextsc{verbe}} \hspace{4pt} Ton~: L\textsubscript{a}.
\textcolor{PineGreen}{\selectlanguage{french}Brûler, se consumer (ex.: un corps sur le bûcher).} \zh{燃烧。} 
\lhead{\firstmark}
\rhead{\botmark}

\subsection{\hspace{-0.5cm} {\Large \textcolor{darkblue}{\textbf{\ipa{mv̩˧\textsubscript{a}}}}}\hspace{0.5cm}[\kern2pt{\textcolor{darkblue}{\textbf{\ipa{mv̩˩˥}}}}\kern2pt]} \hypertarget{mv\string_=\string_Ma1}{}
\markboth{\textcolor{darkblue}{\textbf{\ipa{mv̩˧\textsubscript{a}}}}}{}
\textcolor{teal}{\mytextsc{verbe}} \hspace{4pt} Ton~: M\textsubscript{a}.
\textcolor{PineGreen}{\selectlanguage{french}Mettre, porter, enfiler, endosser (une chemise, une veste); se vêtir d'un habit.} \zh{穿(衣服、上衣)。}  ¶ \textcolor{darkblue}{\textbf{\ipa{bɑ˩lɑ˩ mv̩˥}}} \textcolor{PineGreen}{\selectlanguage{french}mettre une chemise/veste} \zh{穿衣服}  
 ¶ \textcolor{darkblue}{\textbf{\ipa{bɑ˩lɑ˩˥ | tʰi˧-mv̩˧}}} \textcolor{PineGreen}{\selectlanguage{french}mettre une chemise/veste} \zh{穿衣服}  
 ¶ \textcolor{darkblue}{\textbf{\ipa{dʑi˧hṽ˧ mv̩˩}}} \textcolor{PineGreen}{\selectlanguage{french}enfiler un habit} \zh{穿衣服}  
\textit{Voir~:} \hyperlink{}{\textcolor{darkblue}{\textbf{\ipa{ki˩\textsubscript{a}}}}} 
\lhead{\firstmark}
\rhead{\botmark}

\subsection{\hspace{-0.5cm} {\Large \textcolor{darkblue}{\textbf{\ipa{mv̩˩-bæ˧mi˩}}}}\hspace{0.5cm}[\kern2pt{\textcolor{darkblue}{\textbf{\ipa{xxxx non-correspondance entre le nombre de morphèmes et le nombre de tons de morphèmes}}}}\kern2pt]} \hypertarget{mv\string_=\string_B-b\{\string_Mmi\string_B1}{}
\markboth{\textcolor{darkblue}{\textbf{\ipa{mv̩˩-bæ˧mi˩}}}}{}
\textcolor{teal}{\mytextsc{nom}} \hspace{4pt} Ton~: L-L\#.
\textcolor{PineGreen}{\selectlanguage{french}Imbécile, idiote.} \zh{傻女人、笨女人。}  \zh{量词}: \textcolor{darkblue}{\textbf{\ipa{v̩˧}}}  \mytextsc{clf}~: \textcolor{darkblue}{\textbf{\ipa{v̩˧}}} 
\lhead{\firstmark}
\rhead{\botmark}

\subsection{\hspace{-0.5cm} {\Large \textcolor{darkblue}{\textbf{\ipa{mv̩˧bɤ\#˥}}}}\hspace{0.5cm}[\kern2pt{\textcolor{darkblue}{\textbf{\ipa{xxxx non-correspondance entre le nombre de morphèmes et le nombre de tons de morphèmes}}}}\kern2pt]} \hypertarget{mv\string_=\string_Mb7\#\string_T1}{}
\markboth{\textcolor{darkblue}{\textbf{\ipa{mv̩˧bɤ\#˥}}}}{}
\textcolor{teal}{\mytextsc{nom}} \hspace{4pt} Ton~: \#H.
\textcolor{PineGreen}{\selectlanguage{french}Plante du pied.} \zh{脚底。}  \zh{量词}: \textcolor{darkblue}{\textbf{\ipa{kʰwɤ˥}}}  \mytextsc{clf}~: \textcolor{darkblue}{\textbf{\ipa{kʰwɤ˥}}} 
\lhead{\firstmark}
\rhead{\botmark}

\subsection{\hspace{-0.5cm} {\Large \textcolor{darkblue}{\textbf{\ipa{mv̩˧bv̩˧ʐv̩˥}}}}\hspace{0.5cm}[\kern2pt{\textcolor{darkblue}{\textbf{\ipa{mv̩˧bv̩˧ʐv̩˧}}}}\kern2pt]} \hypertarget{mv\string_=\string_Mbv\string_=\string_Mz`v\string_=\string_T1}{}
\markboth{\textcolor{darkblue}{\textbf{\ipa{mv̩˧bv̩˧ʐv̩˥}}}}{}
\textcolor{teal}{\mytextsc{nom}} \hspace{4pt} Ton~: H\#.
\textcolor{PineGreen}{\selectlanguage{french}Dragon.} \zh{龙。}  \zh{量词}: \textcolor{darkblue}{\textbf{\ipa{mi˩}}}  \mytextsc{clf}~: \textcolor{darkblue}{\textbf{\ipa{mi˩}}} 
\lhead{\firstmark}
\rhead{\botmark}

\subsection{\hspace{-0.5cm} {\Large \textcolor{darkblue}{\textbf{\ipa{mv̩˧ɕi˥}}}}\hspace{0.5cm}[\kern2pt{\textcolor{darkblue}{\textbf{\ipa{mv̩˧ɕi˥}}}}\kern2pt]} \hypertarget{mv\string_=\string_Ms£i\string_T1}{}
\markboth{\textcolor{darkblue}{\textbf{\ipa{mv̩˧ɕi˥}}}}{}
\textcolor{teal}{\mytextsc{nom}} \hspace{4pt} Ton~: H\#.
\textcolor{PineGreen}{\selectlanguage{french}Étincelle.} \zh{火花。}  \zh{量词}: \textcolor{darkblue}{\textbf{\ipa{æ̃˩}}}  \mytextsc{clf}~: \textcolor{darkblue}{\textbf{\ipa{æ̃˩}}} 
\lhead{\firstmark}
\rhead{\botmark}

\subsection{\hspace{-0.5cm} {\Large \textcolor{darkblue}{\textbf{\ipa{mv̩˧ɕi˥dʑɯ˩ʈʰɯ˩}}}}\hspace{0.5cm}[\kern2pt{\textcolor{darkblue}{\textbf{\ipa{mv̩˧ɕi˧dʑɯ˧ʈʰɯ˥}}}}\kern2pt]} \hypertarget{mv\string_=\string_Ms£i\string_Tdz£M\string_Bt`\string_hM\string_B1}{}
\markboth{\textcolor{darkblue}{\textbf{\ipa{mv̩˧ɕi˥dʑɯ˩ʈʰɯ˩}}}}{}
\textcolor{teal}{\mytextsc{nom}} \hspace{4pt} Ton~: H\#-L.
\textcolor{PineGreen}{\selectlanguage{french}Arc-en-ciel.} \zh{彩虹。}  \zh{量词}: \textcolor{darkblue}{\textbf{\ipa{kʰɯ˩}}}  \mytextsc{clf}~: \textcolor{darkblue}{\textbf{\ipa{kʰɯ˩}}} \textcolor{PineGreen}{\selectlanguage{french}on peut également dire simplement: mv̩˧ɕi˥}
\lhead{\firstmark}
\rhead{\botmark}

\subsection{\hspace{-0.5cm} {\Large \textcolor{darkblue}{\textbf{\ipa{mv̩˩ɖæ˧}}}}\hspace{0.5cm}[\kern2pt{\textcolor{darkblue}{\textbf{\ipa{xxxx non-correspondance entre le nombre de morphèmes et le nombre de tons de morphèmes}}}}\kern2pt]} \hypertarget{mv\string_=\string_Bd`\{\string_M1}{}
\markboth{\textcolor{darkblue}{\textbf{\ipa{mv̩˩ɖæ˧}}}}{}
\textcolor{teal}{\mytextsc{nom}} \hspace{4pt} Ton~: LM.
\textcolor{PineGreen}{\selectlanguage{french}Le bas du corps.} \zh{下半身。} 
\lhead{\firstmark}
\rhead{\botmark}

\subsection{\hspace{-0.5cm} {\Large \textcolor{darkblue}{\textbf{\ipa{mv̩˧di˧˥}}}}\hspace{0.5cm}[\kern2pt{\textcolor{darkblue}{\textbf{\ipa{mv̩˩di˥}}}}\kern2pt]} \hypertarget{mv\string_=\string_Mdi\string_M\string_T1}{}
\markboth{\textcolor{darkblue}{\textbf{\ipa{mv̩˧di˧˥}}}}{}
\textcolor{teal}{\mytextsc{nom}} \hspace{4pt} Ton~: MH\#.
\ding{202} \textcolor{PineGreen}{\selectlanguage{french}Champs (quel que soit ce qu'on y cultive).} \zh{田地。}  \zh{量词}: \textcolor{darkblue}{\textbf{\ipa{kɤ˧˥}}} \ding{203} \textcolor{PineGreen}{\selectlanguage{french}La Terre, là où habitent les hommes (par opposition au ciel).} \zh{天下。}  \mytextsc{clf}~: \textcolor{darkblue}{\textbf{\ipa{kɤ˧˥}}} 
\lhead{\firstmark}
\rhead{\botmark}

\subsection{\hspace{-0.5cm} {\Large \textcolor{darkblue}{\textbf{\ipa{mv̩˩do˩}}}}\hspace{0.5cm}[\kern2pt{\textcolor{darkblue}{\textbf{\ipa{mv̩˧do˧˥}}}}\kern2pt]} \hypertarget{mv\string_=\string_Bdo\string_B1}{}
\markboth{\textcolor{darkblue}{\textbf{\ipa{mv̩˩do˩}}}}{}
\textcolor{teal}{\mytextsc{verbe}} \hspace{4pt} Ton~: L.
\textcolor{PineGreen}{\selectlanguage{french}Demander.} \zh{问。}  ¶ \textcolor{darkblue}{\textbf{\ipa{le˧-mv̩˩do˩}}} \textcolor{PineGreen}{\selectlanguage{french}\mytextsc{accomp}} \zh{\mytextsc{accomp}}  
 ¶ \textcolor{darkblue}{\textbf{\ipa{mv̩˩do˩-ze˥}}} \textcolor{PineGreen}{\selectlanguage{french}\mytextsc{pfv}} \zh{问了}  
 ¶ \textcolor{darkblue}{\textbf{\ipa{ə˧tso˧ mv̩˩do˩-bi˩? |}}} \textcolor{PineGreen}{\selectlanguage{french}qu'est-ce que (tu) vas demander?} \zh{要问什么呢?}  

\lhead{\firstmark}
\rhead{\botmark}

\subsection{\hspace{-0.5cm} {\Large \textcolor{darkblue}{\textbf{\ipa{mv̩˩ɖɯ˩}}}}\hspace{0.5cm}[\kern2pt{\textcolor{darkblue}{\textbf{\ipa{mv̩˩ɖɯ˩˥}}}}\kern2pt]} \hypertarget{mv\string_=\string_Bd`M\string_B1}{}
\markboth{\textcolor{darkblue}{\textbf{\ipa{mv̩˩ɖɯ˩}}}}{}
\textcolor{teal}{\mytextsc{nom}} \hspace{4pt} Ton~: L.
\textcolor{PineGreen}{\selectlanguage{french}Fille aînée.} \zh{大女儿。}  ¶ \textcolor{darkblue}{\textbf{\ipa{zo˧ɖɯ˧-mv̩˥ɖɯ˩}}} \textcolor{PineGreen}{\selectlanguage{french}fils aîné et fille aînée: les aînés} \zh{大儿子与大女儿}  

\lhead{\firstmark}
\rhead{\botmark}

\subsection{\hspace{-0.5cm} {\Large \textcolor{darkblue}{\textbf{\ipa{mv̩˧dze˧}}}}\hspace{0.5cm}[\kern2pt{\textcolor{darkblue}{\textbf{\ipa{mv̩˩dze˩˥}}}}\kern2pt]} \hypertarget{mv\string_=\string_Mdze\string_M1}{}
\markboth{\textcolor{darkblue}{\textbf{\ipa{mv̩˧dze˧}}}}{}
\textcolor{teal}{\mytextsc{nom}} \hspace{4pt} Ton~: M.
\textcolor{PineGreen}{\selectlanguage{french}Orge, \textit{Hordeum vulgare L}.} \zh{大麦。}  \zh{量词}: \textcolor{darkblue}{\textbf{\ipa{kɤ˧˥}}}  \mytextsc{clf}~: \textcolor{darkblue}{\textbf{\ipa{kɤ˧˥}}} 
\lhead{\firstmark}
\rhead{\botmark}

\subsection{\hspace{-0.5cm} {\Large \textcolor{darkblue}{\textbf{\ipa{mv̩˧dze˧-tɕʰi\#˥}}}}\hspace{0.5cm}[\kern2pt{\textcolor{darkblue}{\textbf{\ipa{xxxx non-correspondance entre le nombre de morphèmes et le nombre de tons de morphèmes}}}}\kern2pt]} \hypertarget{mv\string_=\string_Mdze\string_M-ts£\string_hi\#\string_T1}{}
\markboth{\textcolor{darkblue}{\textbf{\ipa{mv̩˧dze˧-tɕʰi\#˥}}}}{}
\textcolor{teal}{\mytextsc{nom}} \hspace{4pt} Ton~: \#H.
\textcolor{PineGreen}{\selectlanguage{french}Barbe d'orge.} \zh{青稞芒。} 
\lhead{\firstmark}
\rhead{\botmark}

\subsection{\hspace{-0.5cm} {\Large \textcolor{darkblue}{\textbf{\ipa{mv̩˩dzɤ˧}}}}\hspace{0.5cm}[\kern2pt{\textcolor{darkblue}{\textbf{\ipa{mv̩˧dzɤ˧}}}}\kern2pt]} \hypertarget{mv\string_=\string_Bdz7\string_M1}{}
\markboth{\textcolor{darkblue}{\textbf{\ipa{mv̩˩dzɤ˧}}}}{}
\textcolor{teal}{\mytextsc{nom}} \hspace{4pt} Ton~: LM.
\textcolor{PineGreen}{\selectlanguage{french}Bas, partie inférieure (symboliquement: “la queue”).} \zh{下面部分。}  ¶ \textcolor{darkblue}{\textbf{\ipa{mv̩˩dzɤ˧ dzi˧˥}}} \textcolor{PineGreen}{\selectlanguage{french}être assis au fond de la salle} \zh{坐在(房间的)下面部分}  
 ¶ \textcolor{darkblue}{\textbf{\ipa{no˧ | mv̩˩dzɤ˧ dzi˧˥!}}} \textcolor{PineGreen}{\selectlanguage{french}Assieds-toi en bas!} \zh{你到下面去坐!}  

\lhead{\firstmark}
\rhead{\botmark}

\subsection{\hspace{-0.5cm} {\Large \textcolor{darkblue}{\textbf{\ipa{mv̩˧gɤ˥gɤ˩}}}}\hspace{0.5cm}[\kern2pt{\textcolor{darkblue}{\textbf{\ipa{mv̩˩gɤ˧gɤ˧}}}}\kern2pt]} \hypertarget{mv\string_=\string_Mg7\string_Tg7\string_B1}{}
\markboth{\textcolor{darkblue}{\textbf{\ipa{mv̩˧gɤ˥gɤ˩}}}}{}
\textcolor{teal}{\mytextsc{nom}} \hspace{4pt} Ton~: .
\textcolor{PineGreen}{\selectlanguage{french}Les descendants, la descendance.} \zh{下一代、后裔、后人。} 
\lhead{\firstmark}
\rhead{\botmark}

\subsection{\hspace{-0.5cm} {\Large \textcolor{darkblue}{\textbf{\ipa{mv̩˧-gɤ˧lɑ˥}}}}\hspace{0.5cm}[\kern2pt{\textcolor{darkblue}{\textbf{\ipa{xxxx non-correspondance entre le nombre de morphèmes et le nombre de tons de morphèmes}}}}\kern2pt]} \hypertarget{mv\string_=\string_M-g7\string_MlA\string_T1}{}
\markboth{\textcolor{darkblue}{\textbf{\ipa{mv̩˧-gɤ˧lɑ˥}}}}{}
\textcolor{teal}{\mytextsc{nom}} \hspace{4pt} Ton~: H\#.
\textcolor{PineGreen}{\selectlanguage{french}Esprit du ciel, Bodhisattva céleste.} \zh{天宫菩萨。}  \zh{量词}: \textcolor{darkblue}{\textbf{\ipa{v̩˧}}}  \mytextsc{clf}~: \textcolor{darkblue}{\textbf{\ipa{v̩˧}}} 
\lhead{\firstmark}
\rhead{\botmark}

\subsection{\hspace{-0.5cm} {\Large \textcolor{darkblue}{\textbf{\ipa{mv̩˧-gv̩˧dv̩˧}}}}\hspace{0.5cm}[\kern2pt{\textcolor{darkblue}{\textbf{\ipa{xxxx non-correspondance entre le nombre de morphèmes et le nombre de tons de morphèmes}}}}\kern2pt]} \hypertarget{mv\string_=\string_M-gv\string_=\string_Mdv\string_=\string_M1}{}
\markboth{\textcolor{darkblue}{\textbf{\ipa{mv̩˧-gv̩˧dv̩˧}}}}{}
\textcolor{teal}{\mytextsc{nom}} \hspace{4pt} Ton~: M.
\textcolor{PineGreen}{\selectlanguage{french}Partie supérieure du pied.} \zh{脚背。}  \zh{量词}: \textcolor{darkblue}{\textbf{\ipa{ɭɯ˧}}}  \mytextsc{clf}~: \textcolor{darkblue}{\textbf{\ipa{ɭɯ˧}}} 
\lhead{\firstmark}
\rhead{\botmark}

\subsection{\hspace{-0.5cm} {\Large \textcolor{darkblue}{\textbf{\ipa{mv̩˧gv̩˧-kʰv̩˩}}}}\hspace{0.5cm}[\kern2pt{\textcolor{darkblue}{\textbf{\ipa{xxxx non-correspondance entre le nombre de morphèmes et le nombre de tons de morphèmes}}}}\kern2pt]} \hypertarget{mv\string_=\string_Mgv\string_=\string_M-k\string_hv\string_=\string_B1}{}
\markboth{\textcolor{darkblue}{\textbf{\ipa{mv̩˧gv̩˧-kʰv̩˩}}}}{}
\textcolor{teal}{\mytextsc{nom}} \hspace{4pt} Ton~: L\#.
\textcolor{PineGreen}{\selectlanguage{french}Année du dragon.} \zh{龙年。} 
\lhead{\firstmark}
\rhead{\botmark}

\subsection{\hspace{-0.5cm} {\Large \textcolor{darkblue}{\textbf{\ipa{mv̩˧gv̩\#˥}}}}\hspace{0.5cm}[\kern2pt{\textcolor{darkblue}{\textbf{\ipa{mv̩˧gv̩˧}}}}\kern2pt]} \hypertarget{mv\string_=\string_Mgv\string_=\#\string_T1}{}
\markboth{\textcolor{darkblue}{\textbf{\ipa{mv̩˧gv̩\#˥}}}}{}
\textcolor{teal}{\mytextsc{nom}} \hspace{4pt} Ton~: \#H.
\textcolor{PineGreen}{\selectlanguage{french}Tonnerre.} \zh{雷、雷声。}  ¶ \textcolor{darkblue}{\textbf{\ipa{mv̩˧gv̩˧ | gv̩˧-ze˩}}} \textcolor{PineGreen}{\selectlanguage{french}le tonnerre gronde} \zh{打雷了}  
 ¶ \textcolor{darkblue}{\textbf{\ipa{mv̩˧gv̩˧ lɑ˩}}} \textcolor{PineGreen}{\selectlanguage{french}il y a un coup de tonnerre} \zh{打雷了}  
 \zh{量词}: \textcolor{darkblue}{\textbf{\ipa{ɭɯ˧}}}  \mytextsc{clf}~: \textcolor{darkblue}{\textbf{\ipa{ɭɯ˧}}} 
\lhead{\firstmark}
\rhead{\botmark}

\subsection{\hspace{-0.5cm} {\Large \textcolor{darkblue}{\textbf{\ipa{mv̩˩kʰv̩˧˥}}}}\hspace{0.5cm}[\kern2pt{\textcolor{darkblue}{\textbf{\ipa{mv̩˩kʰv̩˧˥}}}}\kern2pt]} \hypertarget{mv\string_=\string_Bk\string_hv\string_=\string_M\string_T1}{}
\markboth{\textcolor{darkblue}{\textbf{\ipa{mv̩˩kʰv̩˧˥}}}}{}
\textcolor{teal}{\mytextsc{nom}} \hspace{4pt} Ton~: LM+MH\#.
\textcolor{PineGreen}{\selectlanguage{french}Soir, soirée (dès 17h, 18h, quand approche la tombée de la nuit).} \zh{晚上。} 
\lhead{\firstmark}
\rhead{\botmark}

\subsection{\hspace{-0.5cm} {\Large \textcolor{darkblue}{\textbf{\ipa{mv̩˧kʰv̩˧˥}}}}\hspace{0.5cm}[\kern2pt{\textcolor{darkblue}{\textbf{\ipa{mv̩˧kʰv̩˧˥}}}}\kern2pt]} \hypertarget{mv\string_=\string_Mk\string_hv\string_=\string_M\string_T1}{}
\markboth{\textcolor{darkblue}{\textbf{\ipa{mv̩˧kʰv̩˧˥}}}}{}
\textcolor{teal}{\mytextsc{nom}} \hspace{4pt} Ton~: MH\#.
\textcolor{PineGreen}{\selectlanguage{french}Fumée.} \zh{烟。}  ¶ \textcolor{darkblue}{\textbf{\ipa{mv̩˧kʰv̩˧ lv̩˥}}} \textcolor{PineGreen}{\selectlanguage{french}ça enfume tout le monde} \zh{烟很多}  
 \zh{量词}: \textcolor{darkblue}{\textbf{\ipa{æ̃˩}}}  \mytextsc{clf}~: \textcolor{darkblue}{\textbf{\ipa{æ̃˩}}} 
\lhead{\firstmark}
\rhead{\botmark}

\subsection{\hspace{-0.5cm} {\Large \textcolor{darkblue}{\textbf{\ipa{mv̩˩ɬi˥}}}}\hspace{0.5cm}[\kern2pt{\textcolor{darkblue}{\textbf{\ipa{mv̩˩ɬi˥}}}}\kern2pt]} \hypertarget{mv\string_=\string_BKi\string_T1}{}
\markboth{\textcolor{darkblue}{\textbf{\ipa{mv̩˩ɬi˥}}}}{}
\textcolor{teal}{\mytextsc{nom}} \hspace{4pt} Ton~: LH.
\textcolor{PineGreen}{\selectlanguage{french}Cadette, puinée (fille deuxième née); littéralement: “fille du milieu”.} \zh{二女儿。} 
\lhead{\firstmark}
\rhead{\botmark}

\subsection{\hspace{-0.5cm} {\Large \textcolor{darkblue}{\textbf{\ipa{mv̩˧ɭɯ˩}}}}\hspace{0.5cm}[\kern2pt{\textcolor{darkblue}{\textbf{\ipa{mv̩˧ɭɯ˩}}}}\kern2pt]} \hypertarget{mv\string_=\string_Ml\string_RM\string_B1}{}
\markboth{\textcolor{darkblue}{\textbf{\ipa{mv̩˧ɭɯ˩}}}}{}
\textcolor{teal}{\mytextsc{nom}} \hspace{4pt} Ton~: L\#.
\textcolor{PineGreen}{\selectlanguage{french}Muli (localité dans le Sichuan, proche de Yongning).} \zh{木里。} 
\lhead{\firstmark}
\rhead{\botmark}

\subsection{\hspace{-0.5cm} {\Large \textcolor{darkblue}{\textbf{\ipa{mv̩˧mi˧}}}}\hspace{0.5cm}[\kern2pt{\textcolor{darkblue}{\textbf{\ipa{mv̩˧mi˧}}}}\kern2pt]} \hypertarget{mv\string_=\string_Mmi\string_M1}{}
\markboth{\textcolor{darkblue}{\textbf{\ipa{mv̩˧mi˧}}}}{}
\textcolor{teal}{\mytextsc{nom}} \hspace{4pt} Ton~: M.
\textcolor{PineGreen}{\selectlanguage{french}Femme.} \zh{女人。}  ¶ \textcolor{darkblue}{\textbf{\ipa{mv̩˧mi˧ so˩tsʰi˩-kʰv̩˩, | qʰo˧mo˥ gi˩ le˩-ʈɤ˩! | ʝi˧=ɻæ˧ qʰv̩˧tsʰi˧-kʰv̩˩, | bɤ˧di˩ lɑ˩ hṽ˩ ɖʐæ˩!}}} \textcolor{PineGreen}{\selectlanguage{french}“A trente ans, la femme est déjà comme une vieille vache qu'il faut tirer pour qu'elle avance (=à trente ans, une femme, c'est déjà une vieille); à soixante, l'homme chevauche sur une peau [littéralement: des poils] de tigre au pays des Pumi!” (=pour l'homme, soixante ans c'est un âge qui permet encore les exploits) (Dicton au sujet de la façon dont vieillissent les deux sexes, au plan de l'attirance qu'ils exercent sur l'autre sexe; employé par une femme, peut véhiculer une nuance de moquerie à l'égard d'un homme âgé)} \zh{“女人,到三十岁就算是得拉着的老牛。男人,到六十岁还能在普米山上骑老虎!”这个谚语讲男人与女人老化过程,特别描写相互吸引的程度:三十岁女人算是老了,六十岁男人还认为自己有伟大的威力。女人可以用这个谚语隐蔽地嘲弄一个老男人。}  
 \zh{量词}: \textcolor{darkblue}{\textbf{\ipa{v̩˧}}}  \mytextsc{clf}~: \textcolor{darkblue}{\textbf{\ipa{v̩˧}}} 
\lhead{\firstmark}
\rhead{\botmark}

\subsection{\hspace{-0.5cm} {\Large \textcolor{darkblue}{\textbf{\ipa{mv̩˧-mv̩˥-di˩}}}}\hspace{0.5cm}[\kern2pt{\textcolor{darkblue}{\textbf{\ipa{xxxx non-correspondance entre le nombre de morphèmes et le nombre de tons de morphèmes}}}}\kern2pt]} \hypertarget{mv\string_=\string_M-mv\string_=\string_T-di\string_B1}{}
\markboth{\textcolor{darkblue}{\textbf{\ipa{mv̩˧-mv̩˥-di˩}}}}{}
\textcolor{teal}{\mytextsc{nom}} \hspace{4pt} Ton~: H\#-.
\textcolor{PineGreen}{\selectlanguage{french}Soufflet.} \zh{风箱。}  \zh{量词}: \textcolor{darkblue}{\textbf{\ipa{ɭɯ˧}}}  \mytextsc{clf}~: \textcolor{darkblue}{\textbf{\ipa{ɭɯ˧}}} 
\lhead{\firstmark}
\rhead{\botmark}

\subsection{\hspace{-0.5cm} {\Large \textcolor{darkblue}{\textbf{\ipa{mv̩˧ɲi˧}}}}\hspace{0.5cm}[\kern2pt{\textcolor{darkblue}{\textbf{\ipa{mv̩˧ɲi˧}}}}\kern2pt]} \hypertarget{mv\string_=\string_MJi\string_M1}{}
\markboth{\textcolor{darkblue}{\textbf{\ipa{mv̩˧ɲi˧}}}}{}
\textcolor{teal}{\mytextsc{nom}} \hspace{4pt} Ton~: M.
\textcolor{PineGreen}{\selectlanguage{french}Orteil.} \zh{脚趾。}  \zh{量词}: \textcolor{darkblue}{\textbf{\ipa{ɭɯ˧}}}  \mytextsc{clf}~: \textcolor{darkblue}{\textbf{\ipa{ɭɯ˧}}} 
\lhead{\firstmark}
\rhead{\botmark}

\subsection{\hspace{-0.5cm} {\Large \textcolor{darkblue}{\textbf{\ipa{mv̩˩pʰæ˧}}}}\hspace{0.5cm}[\kern2pt{\textcolor{darkblue}{\textbf{\ipa{mv̩˩pʰæ˥}}}}\kern2pt]} \hypertarget{mv\string_=\string_Bp\string_h\{\string_M1}{}
\markboth{\textcolor{darkblue}{\textbf{\ipa{mv̩˩pʰæ˧}}}}{}
\textcolor{teal}{\mytextsc{nom}} \hspace{4pt} Ton~: LM.
\textcolor{PineGreen}{\selectlanguage{french}Office, cuisine: pièce où on cuisine la pâtée des cochons, où on distille le vin, et où on prépare certains des plats pour les humains. Elle est située dans le même bâtiment que le foyer-salle à manger, à sa droite (vu depuis la cour).} \zh{备料房:煮猪食、煮酒的地方,有时候也在那边准备人的饭。}  \zh{量词}: \textcolor{darkblue}{\textbf{\ipa{ɭɯ˧}}}  \mytextsc{clf}~: \textcolor{darkblue}{\textbf{\ipa{ɭɯ˧}}} 
\lhead{\firstmark}
\rhead{\botmark}

\subsection{\hspace{-0.5cm} {\Large \textcolor{darkblue}{\textbf{\ipa{mv̩˧qo˩}}}}\hspace{0.5cm}[\kern2pt{\textcolor{darkblue}{\textbf{\ipa{mv̩˧qo˩}}}}\kern2pt]} \hypertarget{mv\string_=\string_Mqo\string_B1}{}
\markboth{\textcolor{darkblue}{\textbf{\ipa{mv̩˧qo˩}}}}{}
\textcolor{teal}{\mytextsc{nom}} \hspace{4pt} Ton~: L\#.
\textcolor{PineGreen}{\selectlanguage{french}Papaye.} \zh{木瓜。}  ¶ \textcolor{darkblue}{\textbf{\ipa{mv̩˧qo˩-dʑɯ˩}}} \textcolor{PineGreen}{\selectlanguage{french}un liquide préparé à base de papaye, servant d'équivalent de vinaigre (le vinaigre a été introduit tardivement; il était acheté en pays chinois)} \zh{用木瓜做的一种汁,用法类似于醋。过去,永宁没有醋,醋是从内地(汉族地区)买来的。}  
 \zh{量词}: \textcolor{darkblue}{\textbf{\ipa{ɭɯ˧}}}  \mytextsc{clf}~: \textcolor{darkblue}{\textbf{\ipa{ɭɯ˧}}} 
\lhead{\firstmark}
\rhead{\botmark}

\subsection{\hspace{-0.5cm} {\Large \textcolor{darkblue}{\textbf{\ipa{mv̩˧qʰwæ˩}}}}\hspace{0.5cm}[\kern2pt{\textcolor{darkblue}{\textbf{\ipa{mv̩˧qʰwæ˩}}}}\kern2pt]} \hypertarget{mv\string_=\string_Mq\string_hw\{\string_B1}{}
\markboth{\textcolor{darkblue}{\textbf{\ipa{mv̩˧qʰwæ˩}}}}{}
\textcolor{teal}{\mytextsc{nom}} \hspace{4pt} Ton~: L\#.
\textcolor{PineGreen}{\selectlanguage{french}Village na hors de la plaine de Yongning, vers le Lac.} \zh{木垮:村落名。}  ¶ \textcolor{darkblue}{\textbf{\ipa{ɬi˧ki˧, | ɲi˧se˩, | tɑ˧dzi˩, | mv̩˧qʰwæ˩, | lɑ˧tʰɑ˧-di˧˥}}} \textcolor{PineGreen}{\selectlanguage{french}Villages dans l'ordre, après la plaine de Yongning, ne comptant pas comme faisant partie de Yongning. Le dernier, \textcolor{darkblue}{\textbf{\ipa{/lɑ˧tʰɑ˧-di˧˥/}}}, désigne toute la région na au-delà du quatrième village.} \zh{永宁到泸沽湖所经过的村落,依次是:里格、尼赛、大祖、木垮,然后到拉塔地(拉塔地指的是泸沽湖周边的摩梭地区,包括左所、洛水村等)}  

\lhead{\firstmark}
\rhead{\botmark}

\subsection{\hspace{-0.5cm} {\Large \textcolor{darkblue}{\textbf{\ipa{mv̩˧ʁo˥\$}}}}\hspace{0.5cm}[\kern2pt{\textcolor{darkblue}{\textbf{\ipa{mv̩˧ʁo˥}}}}\kern2pt]} \hypertarget{mv\string_=\string_MRo\string_T\$1}{}
\markboth{\textcolor{darkblue}{\textbf{\ipa{mv̩˧ʁo˥\$}}}}{}
\textcolor{teal}{\mytextsc{nom}} \hspace{4pt} Ton~: H\$.
\textcolor{PineGreen}{\selectlanguage{french}Le ciel, les cieux.} \zh{天空。}  \zh{量词}: \textcolor{darkblue}{\textbf{\ipa{ɭɯ˧}}}  \mytextsc{clf}~: \textcolor{darkblue}{\textbf{\ipa{ɭɯ˧}}} 
\lhead{\firstmark}
\rhead{\botmark}

\subsection{\hspace{-0.5cm} {\Large \textcolor{darkblue}{\textbf{\ipa{mv̩˩ʁwɤ˧}}} \textsubscript{1}}\hspace{0.5cm}[\kern2pt{\textcolor{darkblue}{\textbf{\ipa{mv̩˩ʁwɤ˥}}}}\kern2pt]} \hypertarget{mv\string_=\string_BRw7\string_M1}{}
\markboth{\textcolor{darkblue}{\textbf{\ipa{mv̩˩ʁwɤ˧}}} \textsubscript{1}}{}
\textcolor{teal}{\mytextsc{nom}} \hspace{4pt} Ton~: LM.
\textcolor{PineGreen}{\selectlanguage{french}Cours inférieur, aval.} \zh{下游。} 
\lhead{\firstmark}
\rhead{\botmark}

\subsection{\hspace{-0.5cm} {\Large \textcolor{darkblue}{\textbf{\ipa{mv̩˩ʁwɤ˧}}} \textsubscript{2}}\hspace{0.5cm}[\kern2pt{\textcolor{darkblue}{\textbf{\ipa{mv̩˩ʁwɤ˥}}}}\kern2pt]} \hypertarget{mv\string_=\string_BRw7\string_M2}{}
\markboth{\textcolor{darkblue}{\textbf{\ipa{mv̩˩ʁwɤ˧}}} \textsubscript{2}}{}
\textcolor{teal}{\mytextsc{nom}} \hspace{4pt} Ton~: LM.
\textcolor{PineGreen}{\selectlanguage{french}“le village du bas”: nom courant pour désigner un hameau d'un village, ou un village entier, par exemple le hameau du bas du village de Zhubo.} \zh{下村,比如者波下村(永宁的一个村落)。} 
\lhead{\firstmark}
\rhead{\botmark}

\subsection{\hspace{-0.5cm} {\Large \textcolor{darkblue}{\textbf{\ipa{mv̩˩si˧˥}}}}\hspace{0.5cm}[\kern2pt{\textcolor{darkblue}{\textbf{\ipa{mv̩˩si˧˥}}}}\kern2pt]} \hypertarget{mv\string_=\string_Bsi\string_M\string_T1}{}
\markboth{\textcolor{darkblue}{\textbf{\ipa{mv̩˩si˧˥}}}}{}
\textcolor{teal}{\mytextsc{nom}} \hspace{4pt} Ton~: LM+MH\#.
\textcolor{PineGreen}{\selectlanguage{french}Matin.} \zh{早晨。} 
\lhead{\firstmark}
\rhead{\botmark}

\subsection{\hspace{-0.5cm} {\Large \textcolor{darkblue}{\textbf{\ipa{mv̩˩si˧-njɤ˧˥}}}}\hspace{0.5cm}[\kern2pt{\textcolor{darkblue}{\textbf{\ipa{xxxx non-correspondance entre le nombre de morphèmes et le nombre de tons de morphèmes}}}}\kern2pt]} \hypertarget{mv\string_=\string_Bsi\string_M-nj7\string_M\string_T1}{}
\markboth{\textcolor{darkblue}{\textbf{\ipa{mv̩˩si˧-njɤ˧˥}}}}{}
\textcolor{teal}{\mytextsc{adverbe}} \hspace{4pt} Ton~: LM+MH\#.
\textcolor{PineGreen}{\selectlanguage{french}Tôt le matin.} \zh{一大早。} 
\lhead{\firstmark}
\rhead{\botmark}

\subsection{\hspace{-0.5cm} {\Large \textcolor{darkblue}{\textbf{\ipa{mv̩˩tɑ\#˥}}}}\hspace{0.5cm}[\kern2pt{\textcolor{darkblue}{\textbf{\ipa{mv̩˩tɑ˥}}}}\kern2pt]} \hypertarget{mv\string_=\string_BtA\#\string_T1}{}
\markboth{\textcolor{darkblue}{\textbf{\ipa{mv̩˩tɑ\#˥}}}}{}
\textcolor{teal}{\mytextsc{verbe}} \hspace{4pt} Ton~: LM+\#H.
\textcolor{PineGreen}{\selectlanguage{french}Louer, faire l'éloge de.} \zh{表扬。}  ¶ \textcolor{darkblue}{\textbf{\ipa{mv̩˩tɑ˧ ʝi˧}}} \textcolor{PineGreen}{\selectlanguage{french}louer, faire l'éloge de} \zh{表扬}  
 ¶ \textcolor{darkblue}{\textbf{\ipa{hĩ˧-ɳɯ˩ | mv̩˩tɑ˥ F | ʝi˧ le˧-hɯ˩-ze˩.}}} \textcolor{PineGreen}{\selectlanguage{french}(Il a fait de bonnes choses, et) les gens l'ont loué/ ont chanté ses louanges} \zh{(他做了好事情,于是)人家大大地表扬他了。}  

\lhead{\firstmark}
\rhead{\botmark}

\subsection{\hspace{-0.5cm} {\Large \textcolor{darkblue}{\textbf{\ipa{mv̩˧ʈʰæ\#˥}}}}\hspace{0.5cm}[\kern2pt{\textcolor{darkblue}{\textbf{\ipa{mv̩˧ʈʰæ˧}}}}\kern2pt]} \hypertarget{mv\string_=\string_Mt`\string_h\{\#\string_T1}{}
\markboth{\textcolor{darkblue}{\textbf{\ipa{mv̩˧ʈʰæ\#˥}}}}{}
\textcolor{teal}{\mytextsc{adverbe}} \hspace{4pt} Ton~: \#H.
\textcolor{PineGreen}{\selectlanguage{french}Dessous, en bas.} \zh{下面。}  ¶ \textcolor{darkblue}{\textbf{\ipa{ʈʂʰɯ˧ | mv̩˧ʈʰæ˧-lɑ˩ li˩! | gɤ˧bi˧ mɤ˧-li˩!}}} \textcolor{PineGreen}{\selectlanguage{french}Il regarde tout le temps vers le bas! il ne regarde pas vers le haut! (au sujet d'une personne constamment assise à son bureau, et qui se plaint de mots de tête)} \zh{他老低头是往下看,不往上看!(情景:有人经常脖子疼、头疼,阿妈提出,这应该跟工作姿势不对有关:那个人一直坐在办公桌前,低着头)}  

\lhead{\firstmark}
\rhead{\botmark}

\subsection{\hspace{-0.5cm} {\Large \textcolor{darkblue}{\textbf{\ipa{mv̩˩tɕi˥}}}}\hspace{0.5cm}[\kern2pt{\textcolor{darkblue}{\textbf{\ipa{mv̩˩tɕi˥}}}}\kern2pt]} \hypertarget{mv\string_=\string_Bts£i\string_T1}{}
\markboth{\textcolor{darkblue}{\textbf{\ipa{mv̩˩tɕi˥}}}}{}
\textcolor{teal}{\mytextsc{nom}} \hspace{4pt} Ton~: LH.
\textcolor{PineGreen}{\selectlanguage{french}Benjamine, plus jeune fille.} \zh{最小的女儿。} 
\lhead{\firstmark}
\rhead{\botmark}

\subsection{\hspace{-0.5cm} {\Large \textcolor{darkblue}{\textbf{\ipa{mv̩˩tɕo˧}}}}\hspace{0.5cm}[\kern2pt{\textcolor{darkblue}{\textbf{\ipa{mv̩˩tɕo˥}}}}\kern2pt]} \hypertarget{mv\string_=\string_Bts£o\string_M1}{}
\markboth{\textcolor{darkblue}{\textbf{\ipa{mv̩˩tɕo˧}}}}{}
\textcolor{teal}{\mytextsc{adverbe}} \hspace{4pt} Ton~: LM.
\textcolor{PineGreen}{\selectlanguage{french}Vers le bas.} \zh{往下。}  ¶ \textcolor{darkblue}{\textbf{\ipa{mv̩˩tɕo˧ kwɤ˩}}} \textcolor{PineGreen}{\selectlanguage{french}jeter vers le bas} \zh{往下扔}  
 ¶ \textcolor{darkblue}{\textbf{\ipa{mv̩˩tɕo˧ se˧!}}} \textcolor{PineGreen}{\selectlanguage{french}Descends! (Ce qu'on dit au chien qui monte sur la partie haute de la cuisine, contrevenant à la règle)} \zh{下去!(命令狗从主屋的地板下去:狗不准来上面)}  

\lhead{\firstmark}
\rhead{\botmark}

\subsection{\hspace{-0.5cm} {\Large \textcolor{darkblue}{\textbf{\ipa{mv̩˩tʰi˩}}}}\hspace{0.5cm}[\kern2pt{\textcolor{darkblue}{\textbf{\ipa{mv̩˩tʰi˩˥}}}}\kern2pt]} \hypertarget{mv\string_=\string_Bt\string_hi\string_B1}{}
\markboth{\textcolor{darkblue}{\textbf{\ipa{mv̩˩tʰi˩}}}}{}
\textcolor{teal}{\mytextsc{adjectif}} \hspace{4pt} Ton~: L.
\textit{De:} \textbf{mv̩˩˥ et tʰi˧} \textcolor{PineGreen}{\selectlanguage{french}Intelligente (d'une femme).} \zh{聪明。}  ¶ \textcolor{darkblue}{\textbf{\ipa{ʈʂʰɯ˧ | mv̩˩tʰi˩˥ | ʐwæ˩˥!}}} \textcolor{PineGreen}{\selectlanguage{french}elle est très intelligente!} \zh{她很聪明!}  

\lhead{\firstmark}
\rhead{\botmark}

\subsection{\hspace{-0.5cm} {\Large \textcolor{darkblue}{\textbf{\ipa{mv̩˩ʈʂæ˧˥}}}}\hspace{0.5cm}[\kern2pt{\textcolor{darkblue}{\textbf{\ipa{mv̩˩ʈʂæ˧˥}}}}\kern2pt]} \hypertarget{mv\string_=\string_Bt`s`\{\string_M\string_T1}{}
\markboth{\textcolor{darkblue}{\textbf{\ipa{mv̩˩ʈʂæ˧˥}}}}{}
\textcolor{teal}{\mytextsc{nom}} \hspace{4pt} Ton~: LM+MH\#.
\textcolor{PineGreen}{\selectlanguage{french}Bas du corps, partie inférieure du corps.} \zh{下半(身)。} 
\lhead{\firstmark}
\rhead{\botmark}

\subsection{\hspace{-0.5cm} {\Large \textcolor{darkblue}{\textbf{\ipa{mv̩˧ʈʂæ˧˥}}}}\hspace{0.5cm}[\kern2pt{\textcolor{darkblue}{\textbf{\ipa{mv̩˧ʈʂæ˧˥}}}}\kern2pt]} \hypertarget{mv\string_=\string_Mt`s`\{\string_M\string_T1}{}
\markboth{\textcolor{darkblue}{\textbf{\ipa{mv̩˧ʈʂæ˧˥}}}}{}
\textcolor{teal}{\mytextsc{verbe}} \hspace{4pt} Ton~: MH\#.
\textcolor{PineGreen}{\selectlanguage{french}S'appeler, avoir... pour nom.} \zh{叫做、称作、名叫。}  ¶ \textcolor{darkblue}{\textbf{\ipa{(ʈʂʰɯ˧ | ) ə˧tso˧ mv̩˧ʈʂæ˧˥?}}} \textcolor{PineGreen}{\selectlanguage{french}comment il s'appelle? Quel est son nom?} \zh{他叫什么名字?}  
 ¶ \textcolor{darkblue}{\textbf{\ipa{njɤ˧ | ... mv̩˧ʈʂæ˧˥}}} \textcolor{PineGreen}{\selectlanguage{french}Je m'appelle…} \zh{我名字叫……}  

\lhead{\firstmark}
\rhead{\botmark}

\subsection{\hspace{-0.5cm} {\Large \textcolor{darkblue}{\textbf{\ipa{mv̩˧tsʰi\#˥}}}}\hspace{0.5cm}[\kern2pt{\textcolor{darkblue}{\textbf{\ipa{mv̩˧tsʰi˧}}}}\kern2pt]} \hypertarget{mv\string_=\string_Mts\string_hi\#\string_T1}{}
\markboth{\textcolor{darkblue}{\textbf{\ipa{mv̩˧tsʰi\#˥}}}}{}
\textcolor{teal}{\mytextsc{nom}} \hspace{4pt} Ton~: \#H.
\textcolor{PineGreen}{\selectlanguage{french}Saison sèche (hiver et printemps; du 9e mois au 2e mois du calendrier lunaire compris).} \zh{旱季(冬天与春天:农历九月到二月)。}  ¶ \textcolor{darkblue}{\textbf{\ipa{mv̩˧tsʰi˧-qo˩}}} \textcolor{PineGreen}{\selectlanguage{french}durant la saison sèche} \zh{旱季的时候}  

\lhead{\firstmark}
\rhead{\botmark}

\subsection{\hspace{-0.5cm} {\Large \textcolor{darkblue}{\textbf{\ipa{mv̩˩tsʰo˩}}}}\hspace{0.5cm}[\kern2pt{\textcolor{darkblue}{\textbf{\ipa{mv̩˩tsʰo˩˥}}}}\kern2pt]} \hypertarget{mv\string_=\string_Bts\string_ho\string_B1}{}
\markboth{\textcolor{darkblue}{\textbf{\ipa{mv̩˩tsʰo˩}}}}{}
\textcolor{teal}{\mytextsc{nom}} \hspace{4pt} Ton~: L.
\textcolor{PineGreen}{\selectlanguage{french}Bois gorgé de résine, pour faire partir le feu (pièces de la taille d'une bûche, qu'on débite en petits morceaux pour faire partir le feu).} \zh{含很多树脂的木头,用来引火。}  \zh{量词}: \textcolor{darkblue}{\textbf{\ipa{kɤ˧˥}}}  \mytextsc{clf}~: \textcolor{darkblue}{\textbf{\ipa{kɤ˧˥}}} 
\lhead{\firstmark}
\rhead{\botmark}

\subsection{\hspace{-0.5cm} {\Large \textcolor{darkblue}{\textbf{\ipa{mv̩˧ʈʂʰɤ˩}}}}\hspace{0.5cm}[\kern2pt{\textcolor{darkblue}{\textbf{\ipa{mv̩˧ʈʂʰɤ˩}}}}\kern2pt]} \hypertarget{mv\string_=\string_Mt`s`\string_h7\string_B1}{}
\markboth{\textcolor{darkblue}{\textbf{\ipa{mv̩˧ʈʂʰɤ˩}}}}{}
\textcolor{teal}{\mytextsc{nom}} \hspace{4pt} Ton~: L\#.
\textcolor{PineGreen}{\selectlanguage{french}Menton.} \zh{下巴。}  \zh{量词}: \textcolor{darkblue}{\textbf{\ipa{kʰwɤ˥}}}  \mytextsc{clf}~: \textcolor{darkblue}{\textbf{\ipa{kʰwɤ˥}}} 
\lhead{\firstmark}
\rhead{\botmark}

\subsection{\hspace{-0.5cm} {\Large \textcolor{darkblue}{\textbf{\ipa{mv̩˧tsɯ˧˥}}}}\hspace{0.5cm}[\kern2pt{\textcolor{darkblue}{\textbf{\ipa{mv̩˧tsɯ˧˥}}}}\kern2pt]} \hypertarget{mv\string_=\string_MtsM\string_M\string_T1}{}
\markboth{\textcolor{darkblue}{\textbf{\ipa{mv̩˧tsɯ˧˥}}}}{}
\textcolor{teal}{\mytextsc{nom}} \hspace{4pt} Ton~: MH\#.
\textcolor{PineGreen}{\selectlanguage{french}Barbe.} \zh{胡子。}  ¶ \textcolor{darkblue}{\textbf{\ipa{mv̩˧tsɯ˧ ʑi˥}}} \textcolor{PineGreen}{\selectlanguage{french}avoir de la barbe} \zh{有胡子}  
 \zh{量词}: \textcolor{darkblue}{\textbf{\ipa{kʰwɤ˥}}}  \mytextsc{clf}~: \textcolor{darkblue}{\textbf{\ipa{kʰwɤ˥}}} 
\lhead{\firstmark}
\rhead{\botmark}

\subsection{\hspace{-0.5cm} {\Large \textcolor{darkblue}{\textbf{\ipa{mv̩˧ʈʂv̩˩}}}}\hspace{0.5cm}[\kern2pt{\textcolor{darkblue}{\textbf{\ipa{mv̩˧ʈʂv̩˩}}}}\kern2pt]} \hypertarget{mv\string_=\string_Mt`s`v\string_=\string_B1}{}
\markboth{\textcolor{darkblue}{\textbf{\ipa{mv̩˧ʈʂv̩˩}}}}{}
\textcolor{teal}{\mytextsc{nom}} \hspace{4pt} Ton~: L\#.
\textcolor{PineGreen}{\selectlanguage{french}Mortier.} \zh{臼。}  \zh{量词}: \textcolor{darkblue}{\textbf{\ipa{nɑ˧}}}  \mytextsc{clf}~: \textcolor{darkblue}{\textbf{\ipa{nɑ˧}}} 
\lhead{\firstmark}
\rhead{\botmark}

\subsection{\hspace{-0.5cm} {\Large \textcolor{darkblue}{\textbf{\ipa{mv̩˧ʈʂv̩˥}}} \textsubscript{1}}\hspace{0.5cm}[\kern2pt{\textcolor{darkblue}{\textbf{\ipa{mv̩˧ʈʂv̩˥}}}}\kern2pt]} \hypertarget{mv\string_=\string_Mt`s`v\string_=\string_T1}{}
\markboth{\textcolor{darkblue}{\textbf{\ipa{mv̩˧ʈʂv̩˥}}} \textsubscript{1}}{}
\textcolor{teal}{\mytextsc{adjectif}} \hspace{4pt} Ton~: H\#.
\ding{202} \textcolor{PineGreen}{\selectlanguage{french}Plissé, froissé.} \zh{皱(衣服)。} \ding{203} \textcolor{PineGreen}{\selectlanguage{french}Ridé.} \zh{(脸)有皱纹。}  ¶ \textcolor{darkblue}{\textbf{\ipa{to˧kɤ˧ | mv̩˧ʈʂv̩˥ ze˩.}}} \textcolor{PineGreen}{\selectlanguage{french}(Son) front s'est ridé / son front a pris des rides.} \zh{他的前额有了皱纹。}  
 ¶ \textcolor{darkblue}{\textbf{\ipa{to˧kɤ˧ | le˧-mv̩˧ʈʂv̩˥}}} \textcolor{PineGreen}{\selectlanguage{french}(Son) front est ridé.} \zh{他的前额有皱纹。}  
 ¶ \textcolor{darkblue}{\textbf{\ipa{[F5] æ˩ʂe˩˥ | le˧-mv̩˧ʈʂv̩˥}}} \textcolor{PineGreen}{\selectlanguage{french}La peau est ridée (littéralement: “la chair est ridée”)} \zh{皮肤有皱纹(直译:“肉有皱纹”)}  
\ding{204} \textcolor{PineGreen}{\selectlanguage{french}Fané.} \zh{谢(花谢了)。}  ¶ \textcolor{darkblue}{\textbf{\ipa{bæ˩bæ˩˥ | le˧-mv̩˧ʈʂv̩˥-ze˩}}} \textcolor{PineGreen}{\selectlanguage{french}La fleur s'est fanée.} \zh{花谢了。}  
\textit{Voir~:} \hyperlink{}{\textcolor{darkblue}{\textbf{\ipa{mv̩˧ʈʂv̩˥}}} \textsubscript{2}} 
\lhead{\firstmark}
\rhead{\botmark}

\subsection{\hspace{-0.5cm} {\Large \textcolor{darkblue}{\textbf{\ipa{mv̩˧ʈʂv̩˥}}} \textsubscript{2}}\hspace{0.5cm}[\kern2pt{\textcolor{darkblue}{\textbf{\ipa{mv̩˧ʈʂv̩˥}}}}\kern2pt]} \hypertarget{mv\string_=\string_Mt`s`v\string_=\string_T2}{}
\markboth{\textcolor{darkblue}{\textbf{\ipa{mv̩˧ʈʂv̩˥}}} \textsubscript{2}}{}
\textcolor{teal}{\mytextsc{nom}} \hspace{4pt} Ton~: H\#.
\textcolor{PineGreen}{\selectlanguage{french}Rides.} \zh{皱纹。}  \zh{量词}: \textcolor{darkblue}{\textbf{\ipa{kʰɯ˩}}}  \mytextsc{clf}~: \textcolor{darkblue}{\textbf{\ipa{kʰɯ˩}}} \textit{Voir~:} \hyperlink{}{\textcolor{darkblue}{\textbf{\ipa{mv̩˧ʈʂv̩˥}}} \textsubscript{1}} 
\lhead{\firstmark}
\rhead{\botmark}

\subsection{\hspace{-0.5cm} {\Large \textcolor{darkblue}{\textbf{\ipa{mv̩˧ʈʂv̩˩-nv̩˩mi˩}}}}\hspace{0.5cm}[\kern2pt{\textcolor{darkblue}{\textbf{\ipa{xxxx non-correspondance entre le nombre de morphèmes et le nombre de tons de morphèmes}}}}\kern2pt]} \hypertarget{mv\string_=\string_Mt`s`v\string_=\string_B-nv\string_=\string_Bmi\string_B1}{}
\markboth{\textcolor{darkblue}{\textbf{\ipa{mv̩˧ʈʂv̩˩-nv̩˩mi˩}}}}{}
\textcolor{teal}{\mytextsc{nom}} \hspace{4pt} Ton~: L\#-.
\textit{De:} \textbf{mv̩˧ʈʂv̩˩ et nv̩˩mi˩} \textcolor{PineGreen}{\selectlanguage{french}Petit pilon.} \zh{杵。}  \zh{量词}: \textcolor{darkblue}{\textbf{\ipa{nɑ˧}}}  \mytextsc{clf}~: \textcolor{darkblue}{\textbf{\ipa{nɑ˧}}} 
\lhead{\firstmark}
\rhead{\botmark}

\subsection{\hspace{-0.5cm} {\Large \textcolor{darkblue}{\textbf{\ipa{mv̩˧ʈʰɯ˧}}}}\hspace{0.5cm}[\kern2pt{\textcolor{darkblue}{\textbf{\ipa{xxxx non-correspondance entre le nombre de morphèmes et le nombre de tons de morphèmes}}}}\kern2pt]} \hypertarget{mv\string_=\string_Mt`\string_hM\string_M1}{}
\markboth{\textcolor{darkblue}{\textbf{\ipa{mv̩˧ʈʰɯ˧}}}}{}
\textcolor{teal}{\mytextsc{nom}} \hspace{4pt} Ton~: M.
\textcolor{PineGreen}{\selectlanguage{french}Talon.} \zh{脚跟。}  \zh{量词}: \textcolor{darkblue}{\textbf{\ipa{kʰwɤ˥}}}  \mytextsc{clf}~: \textcolor{darkblue}{\textbf{\ipa{kʰwɤ˥}}} 
\lhead{\firstmark}
\rhead{\botmark}

\subsection{\hspace{-0.5cm} {\Large \textcolor{darkblue}{\textbf{\ipa{mv̩˩tv̩˩}}}}\hspace{0.5cm}[\kern2pt{\textcolor{darkblue}{\textbf{\ipa{mv̩˧tv̩˧}}}}\kern2pt]} \hypertarget{mv\string_=\string_Btv\string_=\string_B1}{}
\markboth{\textcolor{darkblue}{\textbf{\ipa{mv̩˩tv̩˩}}}}{}
\textcolor{teal}{\mytextsc{nom}} \hspace{4pt} Ton~: L.
\textcolor{PineGreen}{\selectlanguage{french}Fille unique.} \zh{独生女。}  ¶ \textcolor{darkblue}{\textbf{\ipa{mv̩˩tv̩˩˥ | ɖɯ˧-v̩˧-lɑ˧ dʑo˧˥!}}} \textcolor{PineGreen}{\selectlanguage{french}(elle) n'a qu'une fille unique!} \zh{她只有一个独生女!}  

\lhead{\firstmark}
\rhead{\botmark}

\subsection{\hspace{-0.5cm} {\Large \textcolor{darkblue}{\textbf{\ipa{mv̩˧tʰv̩˧˥}}}}\hspace{0.5cm}[\kern2pt{\textcolor{darkblue}{\textbf{\ipa{mv̩˩tʰv̩˩˥}}}}\kern2pt]} \hypertarget{mv\string_=\string_Mt\string_hv\string_=\string_M\string_T1}{}
\markboth{\textcolor{darkblue}{\textbf{\ipa{mv̩˧tʰv̩˧˥}}}}{}
\textcolor{teal}{\mytextsc{nom}} \hspace{4pt} Ton~: MH\#.
\textcolor{PineGreen}{\selectlanguage{french}Torche.} \zh{火把。}  \zh{量词}: \textcolor{darkblue}{\textbf{\ipa{qɑ˩}}}  \mytextsc{clf}~: \textcolor{darkblue}{\textbf{\ipa{qɑ˩}}} 
\lhead{\firstmark}
\rhead{\botmark}

\subsection{\hspace{-0.5cm} {\Large \textcolor{darkblue}{\textbf{\ipa{mv̩˧ʐe˧˥}}} \textsubscript{1}}\hspace{0.5cm}[\kern2pt{\textcolor{darkblue}{\textbf{\ipa{mv̩˧ʐe˧˥}}}}\kern2pt]} \hypertarget{mv\string_=\string_Mz`e\string_M\string_T1}{}
\markboth{\textcolor{darkblue}{\textbf{\ipa{mv̩˧ʐe˧˥}}} \textsubscript{1}}{}
\textcolor{teal}{\mytextsc{nom}} \hspace{4pt} Ton~: MH\#.
\textcolor{PineGreen}{\selectlanguage{french}Saison des pluies (été et automne: du 3e au 8e mois du calendrier lunaire).} \zh{雨季(夏天与秋天:三月份至八月份)。}  ¶ \textcolor{darkblue}{\textbf{\ipa{mv̩˧ʐe˧-qo˥}}} \textcolor{PineGreen}{\selectlanguage{french}pendant la saison des pluies} \zh{雨季的时候}  

\lhead{\firstmark}
\rhead{\botmark}

\subsection{\hspace{-0.5cm} {\Large \textcolor{darkblue}{\textbf{\ipa{mv̩˧ʐe˧-ʈʂʰæ˧ɣɯ\#˥}}}}\hspace{0.5cm}[\kern2pt{\textcolor{darkblue}{\textbf{\ipa{xxxx non-correspondance entre le nombre de morphèmes et le nombre de tons de morphèmes}}}}\kern2pt]} \hypertarget{mv\string_=\string_Mz`e\string_M-t`s`\string_h\{\string_MGM\#\string_T1}{}
\markboth{\textcolor{darkblue}{\textbf{\ipa{mv̩˧ʐe˧-ʈʂʰæ˧ɣɯ\#˥}}}}{}
\textcolor{teal}{\mytextsc{nom}} \hspace{4pt} Ton~: \#H.
\textcolor{PineGreen}{\selectlanguage{french}Poudre à canon.} \zh{火药。}  \zh{量词}: \textcolor{darkblue}{\textbf{\ipa{po˩}}}  \mytextsc{clf}~: \textcolor{darkblue}{\textbf{\ipa{po˩}}} 
\lhead{\firstmark}
\rhead{\botmark}

\subsection{\hspace{-0.5cm} {\Large \textcolor{darkblue}{\textbf{\ipa{mv̩˧ʐe\#˥}}} \textsubscript{2}}\hspace{0.5cm}[\kern2pt{\textcolor{darkblue}{\textbf{\ipa{mv̩˧ʐe˧}}}}\kern2pt]} \hypertarget{mv\string_=\string_Mz`e\#\string_T2}{}
\markboth{\textcolor{darkblue}{\textbf{\ipa{mv̩˧ʐe\#˥}}} \textsubscript{2}}{}
\textcolor{teal}{\mytextsc{nom}} \hspace{4pt} Ton~: \#H.
\textcolor{PineGreen}{\selectlanguage{french}Arme à feu, fusil; arquebuse.} \zh{枪,明火枪。}  \zh{量词}: \textcolor{darkblue}{\textbf{\ipa{kʰɯ˩}}}  \mytextsc{clf}~: \textcolor{darkblue}{\textbf{\ipa{kʰɯ˩}}} 
\lhead{\firstmark}
\rhead{\botmark}

\subsection{\hspace{-0.5cm} {\Large \textcolor{darkblue}{\textbf{\ipa{mv̩˧ʑi˩}}}}\hspace{0.5cm}[\kern2pt{\textcolor{darkblue}{\textbf{\ipa{mv̩˧ʑi˧}}}}\kern2pt]} \hypertarget{mv\string_=\string_Mz£i\string_B1}{}
\markboth{\textcolor{darkblue}{\textbf{\ipa{mv̩˧ʑi˩}}}}{}
\textcolor{teal}{\mytextsc{nom}} \hspace{4pt} Ton~: L.
\textcolor{PineGreen}{\selectlanguage{french}Nouvelle, ragot, information, histoire.} \zh{消息、闲话、八卦。}  ¶ \textcolor{darkblue}{\textbf{\ipa{mv̩˧ʑi˩ | ɖɯ˧-kʰwɤ˥}}} \textcolor{PineGreen}{\selectlanguage{french}une nouvelle, un ragot, une information} \zh{一个八卦}  
 \zh{量词}: \textcolor{darkblue}{\textbf{\ipa{kʰwɤ˥}}}  \mytextsc{clf}~: \textcolor{darkblue}{\textbf{\ipa{kʰwɤ˥}}} 
\lhead{\firstmark}
\rhead{\botmark}

\subsection{\hspace{-0.5cm} {\Large \textcolor{darkblue}{\textbf{\ipa{mv̩˩zo˩}}}}\hspace{0.5cm}[\kern2pt{\textcolor{darkblue}{\textbf{\ipa{mv̩˩zo˩˥}}}}\kern2pt]} \hypertarget{mv\string_=\string_Bzo\string_B1}{}
\markboth{\textcolor{darkblue}{\textbf{\ipa{mv̩˩zo˩}}}}{}
\textcolor{teal}{\mytextsc{nom}} \hspace{4pt} Ton~: L.
\textcolor{PineGreen}{\selectlanguage{french}Jeune fille.} \zh{姑娘。}  ¶ \textcolor{darkblue}{\textbf{\ipa{mv̩˩zo˩=ɻæ˧}}} \textcolor{PineGreen}{\selectlanguage{french}les jeunes filles} \zh{姑娘们}  
 \zh{量词}: \textcolor{darkblue}{\textbf{\ipa{ɭɯ˧}}} \textcolor{darkblue}{\textbf{\ipa{v̩˧}}}  \mytextsc{clf}~: \textcolor{darkblue}{\textbf{\ipa{ɭɯ˧}}} \textcolor{darkblue}{\textbf{\ipa{v̩˧}}} 
\lhead{\firstmark}
\rhead{\botmark}

\subsection{\hspace{-0.5cm} {\Large \textcolor{darkblue}{\textbf{\ipa{mv̩˩ʐɤ˩}}}}\hspace{0.5cm}[\kern2pt{\textcolor{darkblue}{\textbf{\ipa{mv̩˩ʐɤ˩˥}}}}\kern2pt]} \hypertarget{mv\string_=\string_Bz`7\string_B1}{}
\markboth{\textcolor{darkblue}{\textbf{\ipa{mv̩˩ʐɤ˩}}}}{}
\textcolor{teal}{\mytextsc{nom}} \hspace{4pt} Ton~: L.
\textcolor{PineGreen}{\selectlanguage{french}Fille adoptive.} \zh{义女。} 
\lhead{\firstmark}
\rhead{\botmark}

\subsection{\hspace{-0.5cm} {\Large \textcolor{darkblue}{\textbf{\ipa{mv̩˩zo˩-ə˩mi˥}}}}\hspace{0.5cm}[\kern2pt{\textcolor{darkblue}{\textbf{\ipa{xxxx non-correspondance entre le nombre de morphèmes et le nombre de tons de morphèmes}}}}\kern2pt]} \hypertarget{mv\string_=\string_Bzo\string_B-@\string_Bmi\string_T1}{}
\markboth{\textcolor{darkblue}{\textbf{\ipa{mv̩˩zo˩-ə˩mi˥}}}}{}
\textcolor{teal}{\mytextsc{nom}} \hspace{4pt} Ton~: L+H\#.
\textcolor{PineGreen}{\selectlanguage{french}(une) jeune fille et (sa) mère.} \zh{姑娘与母亲。} 
\lhead{\firstmark}
\rhead{\botmark}

\subsection{\hspace{-0.5cm} {\Large \textcolor{darkblue}{\textbf{\ipa{mv̩˩zɯ˩}}} \textsubscript{1}}\hspace{0.5cm}[\kern2pt{\textcolor{darkblue}{\textbf{\ipa{mv̩˩zɯ˥}}}}\kern2pt]} \hypertarget{mv\string_=\string_BzM\string_B1}{}
\markboth{\textcolor{darkblue}{\textbf{\ipa{mv̩˩zɯ˩}}} \textsubscript{1}}{}
\textcolor{teal}{\mytextsc{nom}} \hspace{4pt} Ton~: L.
\textcolor{PineGreen}{\selectlanguage{french}Frères (aînés ou cadets).} \zh{兄弟(哥哥们与弟弟们)。}  ¶ \textcolor{darkblue}{\textbf{\ipa{ʈʂʰɯ˧ | nɑ˧dʑi˧-bv̩˧ | mv̩˩zɯ˩-ʝi˥-hĩ˩ ɲi˩!}}} \textcolor{PineGreen}{\selectlanguage{french}il est frère de \textcolor{darkblue}{\textbf{\ipa{nɑ˧dʑi˧/!}}}} \zh{他是\textcolor{darkblue}{\textbf{\ipa{nɑ˧dʑi˧/}}}的兄弟!}  
 \zh{量词}: \textcolor{darkblue}{\textbf{\ipa{v̩˧}}}  \mytextsc{clf}~: \textcolor{darkblue}{\textbf{\ipa{v̩˧}}} 
\lhead{\firstmark}
\rhead{\botmark}

\subsection{\hspace{-0.5cm} {\Large \textcolor{darkblue}{\textbf{\ipa{mv̩˩zɯ˩}}} \textsubscript{2}}\hspace{0.5cm}[\kern2pt{\textcolor{darkblue}{\textbf{\ipa{mv̩˩zɯ˩˥}}}}\kern2pt]} \hypertarget{mv\string_=\string_BzM\string_B2}{}
\markboth{\textcolor{darkblue}{\textbf{\ipa{mv̩˩zɯ˩}}} \textsubscript{2}}{}
\textcolor{teal}{\mytextsc{nom}} \hspace{4pt} Ton~: L.
\textcolor{PineGreen}{\selectlanguage{french}Avoine.} \zh{燕麦。}  \zh{量词}: \textcolor{darkblue}{\textbf{\ipa{kɤ˧˥}}}  \mytextsc{clf}~: \textcolor{darkblue}{\textbf{\ipa{kɤ˧˥}}} 
\lhead{\firstmark}
\rhead{\botmark}

\subsection{\hspace{-0.5cm} {\Large \textcolor{darkblue}{\textbf{\ipa{mv̩˩zɯ˩-ni˥mi˩}}}}\hspace{0.5cm}[\kern2pt{\textcolor{darkblue}{\textbf{\ipa{xxxx non-correspondance entre le nombre de morphèmes et le nombre de tons de morphèmes}}}}\kern2pt]} \hypertarget{mv\string_=\string_BzM\string_B-ni\string_Tmi\string_B1}{}
\markboth{\textcolor{darkblue}{\textbf{\ipa{mv̩˩zɯ˩-ni˥mi˩}}}}{}
\textcolor{teal}{\mytextsc{nom}} \hspace{4pt} Ton~: L+\#H-.
\textcolor{PineGreen}{\selectlanguage{french}Frères et sœurs (tous les frères et sœurs; s'applique aussi aux cousins).} \zh{兄弟姐妹,堂兄弟姐妹。} 
\lhead{\firstmark}
\rhead{\botmark}

\subsection{\hspace{-0.5cm} {\Large \textcolor{darkblue}{\textbf{\ipa{mv̩˧‑}}}}\hspace{0.5cm}[\kern2pt{\textcolor{darkblue}{\textbf{\ipa{xxxx non-correspondance entre le nombre de morphèmes et le nombre de tons de morphèmes}}}}\kern2pt]} \hypertarget{mv\string_=\string_M‑1}{}
\markboth{\textcolor{darkblue}{\textbf{\ipa{mv̩˧‑}}}}{}
\textcolor{teal}{\mytextsc{préfixe}} \hspace{4pt} Ton~: M.
\textcolor{PineGreen}{\selectlanguage{french}Aspect/mode: l'événement est imminent: sur le point de se produire.} \zh{即将、快要、马上会、立即。}  ¶ \textcolor{darkblue}{\textbf{\ipa{ʈʂʰɯ˧ | mv̩˧-dzɯ˧-kwɤ˩-tɕɯ˩!}}} \textcolor{PineGreen}{\selectlanguage{french}Mange-le donc! / Finis donc ça! (Contexte: à table, quelqu'un ne finit pas son bol; sa mère ou grand-mère lui enjoint de finir, pour ne pas gaspiller de nourriture.)} \zh{你吃完吧!}  
 ¶ \textcolor{darkblue}{\textbf{\ipa{tʰi˧-mv̩˧-dzɯ˧-kwɤ˩-tɕɯ˩!}}} \textcolor{PineGreen}{\selectlanguage{french}Comme l'exemple précédent, avec le \mytextsc{duratif}} \zh{同上}  
 ¶ \textcolor{darkblue}{\textbf{\ipa{[M18] ʈʂʰɯ˧ mv̩˧-ʂɯ˧ bi˩-ni˩gv̩˩! njɤ˧ | gv̩˩dʑɯ˩˥ | ʐwæ˩˥! |}}} \textcolor{PineGreen}{\selectlanguage{french}Il/elle va mourir! Je suis au désespoir!} \zh{他要死了!我很伤心!}  
 ¶ \textcolor{darkblue}{\textbf{\ipa{hĩ˧ ʈʂʰɯ˧-v̩˧ tʰv̩˧ mv̩˧-ʂɯ˧-kwɤ˧tɕɯ˥-lɑ˩...}}} \textcolor{PineGreen}{\selectlanguage{french}du fait que cette personne va mourir très bientôt...} \zh{因为这个人快要去世……}  
 ¶ \textcolor{darkblue}{\textbf{\ipa{mv̩˧-dzɯ˧-bi˩-ze˩!}}} \textcolor{PineGreen}{\selectlanguage{french}[On] va manger tout de suite!} \zh{马上要吃了!}  
 ¶ \textcolor{darkblue}{\textbf{\ipa{mv̩˧-hwæ˧}}} \textcolor{PineGreen}{\selectlanguage{french}sur le point d'acheter} \zh{即将买}  
 ¶ \textcolor{darkblue}{\textbf{\ipa{mv̩˧-tɕʰi˧}}} \textcolor{PineGreen}{\selectlanguage{french}sur le point de vendre} \zh{即将卖}  
 ¶ \textcolor{darkblue}{\textbf{\ipa{mv̩˧-dzɯ˧-kwɤ˧tɕɯ˥-lɑ˩...}}} \textcolor{PineGreen}{\selectlanguage{french}puisqu'elle/il est sur le point de manger...} \zh{因为马上要吃……}  
 ¶ \textcolor{darkblue}{\textbf{\ipa{mv̩˧-lɑ˩-kwɤ˩tɕɯ˩-lɑ˩...}}} \textcolor{PineGreen}{\selectlanguage{french}puisqu'elle/il est sur le point de frapper...} \zh{因为要打……}  

\lhead{\firstmark}
\rhead{\botmark}

\subsection{\hspace{-0.5cm} {\Large \textcolor{darkblue}{\textbf{\ipa{mv̩˧\textasciitilde{}mv̩\#˥}}}}\hspace{0.5cm}[\kern2pt{\textcolor{darkblue}{\textbf{\ipa{mv̩˧mv̩˧}}}}\kern2pt]} \hypertarget{mv\string_=\string_M~mv\string_=\#\string_T1}{}
\markboth{\textcolor{darkblue}{\textbf{\ipa{mv̩˧\textasciitilde{}mv̩\#˥}}}}{}
\textcolor{teal}{\mytextsc{adjectif}} \hspace{4pt} Ton~: .
\textit{De:} \textbf{mv̩˥} \textcolor{PineGreen}{\selectlanguage{french}Clair (parole, événement…).} \zh{清楚(话、事情)。}  ¶ \textcolor{darkblue}{\textbf{\ipa{ʐwɤ˧ mv̩˧\textasciitilde{}mv̩˧}}} \textcolor{PineGreen}{\selectlanguage{french}parler clairement} \zh{讲清楚}  
 ¶ \textcolor{darkblue}{\textbf{\ipa{le˧-mv̩˧\textasciitilde{}mv̩˧-kʰɯ˩}}} \textcolor{PineGreen}{\selectlanguage{french}éclaircir, tirer au clair, expliquer} \zh{弄明白、讲清楚}  

\lhead{\firstmark}
\rhead{\botmark}

\newpage
\section*{\centering- \textcolor{darkblue}{\textbf{\ipa{n}}} -}
\subsection{\hspace{-0.5cm} {\Large \textcolor{darkblue}{\textbf{\ipa{nɑ˥}}}}\hspace{0.5cm}[\kern2pt{\textcolor{darkblue}{\textbf{\ipa{nɑ˥}}}}\kern2pt]} \hypertarget{nA\string_T1}{}
\markboth{\textcolor{darkblue}{\textbf{\ipa{nɑ˥}}}}{}
\textcolor{teal}{\mytextsc{adjectif}} \hspace{4pt} Ton~: H.
\textcolor{PineGreen}{\selectlanguage{french}Grave, sérieux (ex.: une blessure).} \zh{严重,重要。}  ¶ \textcolor{darkblue}{\textbf{\ipa{mɤ˧-nɑ˥}}} \textcolor{PineGreen}{\selectlanguage{french}bénin, pas grave, sans conséquence (ex.: une écorchure)} \zh{不严重}  

\lhead{\firstmark}
\rhead{\botmark}

\subsection{\hspace{-0.5cm} {\Large \textcolor{darkblue}{\textbf{\ipa{nɑ˧\textsubscript{a}}}}}\hspace{0.5cm}[\kern2pt{\textcolor{darkblue}{\textbf{\ipa{nɑ˥}}}}\kern2pt]} \hypertarget{nA\string_Ma1}{}
\markboth{\textcolor{darkblue}{\textbf{\ipa{nɑ˧\textsubscript{a}}}}}{}
\textcolor{teal}{\mytextsc{classificateur}} \hspace{4pt} Ton~: M\textsubscript{a}.
\textcolor{PineGreen}{\selectlanguage{french}Classificateur des outils.} \zh{量词:工具(一把)。}  ¶ \textcolor{darkblue}{\textbf{\ipa{ɖɯ˧-nɑ˧ dʑo˧}}} \textcolor{PineGreen}{\selectlanguage{french}il y en a un; il y a un outil} \zh{有一把(工具)}  

\lhead{\firstmark}
\rhead{\botmark}

\subsection{\hspace{-0.5cm} {\Large \textcolor{darkblue}{\textbf{\ipa{nɑ˧dʑi\#˥}}}}\hspace{0.5cm}[\kern2pt{\textcolor{darkblue}{\textbf{\ipa{nɑ˩dʑi˥}}}}\kern2pt]} \hypertarget{nA\string_Mdz£i\#\string_T1}{}
\markboth{\textcolor{darkblue}{\textbf{\ipa{nɑ˧dʑi\#˥}}}}{}
\textcolor{teal}{\mytextsc{nom}} \hspace{4pt} Ton~: \#H.
\textcolor{PineGreen}{\selectlanguage{french}Prénom féminin.} \zh{女性名字。} 
\lhead{\firstmark}
\rhead{\botmark}

\subsection{\hspace{-0.5cm} {\Large \textcolor{darkblue}{\textbf{\ipa{nɑ˧mi\#˥}}}}\hspace{0.5cm}[\kern2pt{\textcolor{darkblue}{\textbf{\ipa{nɑ˩mi˥}}}}\kern2pt]} \hypertarget{nA\string_Mmi\#\string_T1}{}
\markboth{\textcolor{darkblue}{\textbf{\ipa{nɑ˧mi\#˥}}}}{}
\textcolor{teal}{\mytextsc{nom}} \hspace{4pt} Ton~: \#H.
\textcolor{PineGreen}{\selectlanguage{french}Épuisement, misère, difficultés.} \zh{受累、劳累、辛苦、困难、艰难、艰苦。}  ¶ \textcolor{darkblue}{\textbf{\ipa{nɑ˧mi˧ tʰv̩˧!}}} \textcolor{PineGreen}{\selectlanguage{french}Misère! / Des difficultés surviennent, on rencontre des difficultés; on est dans une période difficile} \zh{现在是艰苦的时候! / 现在很贫困!}  
 \zh{量词}: \textcolor{darkblue}{\textbf{\ipa{kʰwɤ˥}}}  \mytextsc{clf}~: \textcolor{darkblue}{\textbf{\ipa{kʰwɤ˥}}} 
\lhead{\firstmark}
\rhead{\botmark}

\subsection{\hspace{-0.5cm} {\Large \textcolor{darkblue}{\textbf{\ipa{nɑ˩\textsubscript{b}}}}}\hspace{0.5cm}[\kern2pt{\textcolor{darkblue}{\textbf{\ipa{nɑ˥}}}}\kern2pt]} \hypertarget{nA\string_Bb1}{}
\markboth{\textcolor{darkblue}{\textbf{\ipa{nɑ˩\textsubscript{b}}}}}{}
\textcolor{teal}{\mytextsc{adjectif}} \hspace{4pt} Ton~: L\textsubscript{b}.
\textcolor{PineGreen}{\selectlanguage{french}Noir, sombre.} \zh{黑,暗(颜色,天色)。}  ¶ \textcolor{darkblue}{\textbf{\ipa{nɑ˩-hĩ˥}}} \textcolor{PineGreen}{\selectlanguage{french}\mytextsc{rel}} \zh{黑的}  
 ¶ \textcolor{darkblue}{\textbf{\ipa{mɤ˧-nɑ˩}}} \textcolor{PineGreen}{\selectlanguage{french}\mytextsc{neg}} \zh{不黑}  

\lhead{\firstmark}
\rhead{\botmark}

\subsection{\hspace{-0.5cm} {\Large \textcolor{darkblue}{\textbf{\ipa{nɑ˩bɑ˧-ʁɑ˧ɭɯ\#˥}}}}\hspace{0.5cm}[\kern2pt{\textcolor{darkblue}{\textbf{\ipa{xxxx non-correspondance entre le nombre de morphèmes et le nombre de tons de morphèmes}}}}\kern2pt]} \hypertarget{nA\string_BbA\string_M-RA\string_Ml\string_RM\#\string_T1}{}
\markboth{\textcolor{darkblue}{\textbf{\ipa{nɑ˩bɑ˧-ʁɑ˧ɭɯ\#˥}}}}{}
\textcolor{teal}{\mytextsc{nom}} \hspace{4pt} Ton~: LM+\#H.
\textcolor{PineGreen}{\selectlanguage{french}Nom d'une montagne de Yongning.} \zh{一座山的名字。} 
\lhead{\firstmark}
\rhead{\botmark}

\subsection{\hspace{-0.5cm} {\Large \textcolor{darkblue}{\textbf{\ipa{nɑ˩dzi˧}}}}\hspace{0.5cm}[\kern2pt{\textcolor{darkblue}{\textbf{\ipa{nɑ˩dzi˥}}}}\kern2pt]} \hypertarget{nA\string_Bdzi\string_M1}{}
\markboth{\textcolor{darkblue}{\textbf{\ipa{nɑ˩dzi˧}}}}{}
\textcolor{teal}{\mytextsc{adverbe}} \hspace{4pt} Ton~: LM.
\textcolor{PineGreen}{\selectlanguage{french}Sombre (au crépuscule, il se met à faire sombre).} \zh{暗(黄昏/暮的时候,天变暗)。}  ¶ \textcolor{darkblue}{\textbf{\ipa{nɑ˩dzi˧-ze˩!}}} \textcolor{PineGreen}{\selectlanguage{french}le crépuscule est venu! / c'est le crépuscule!} \zh{天变暗了! / 黄昏到了!}  
 ¶ \textcolor{darkblue}{\textbf{\ipa{nɑ˩dzi˧-ho˩-ze˩!}}} \textcolor{PineGreen}{\selectlanguage{french}Il va faire sombre! La nuit va commencer à tomber!} \zh{(天)要变暗了!}  

\lhead{\firstmark}
\rhead{\botmark}

\subsection{\hspace{-0.5cm} {\Large \textcolor{darkblue}{\textbf{\ipa{nɑ˩hĩ\#˥}}}}\hspace{0.5cm}[\kern2pt{\textcolor{darkblue}{\textbf{\ipa{nɑ˧hĩ˧}}}}\kern2pt]} \hypertarget{nA\string_Bhi\string_~\#\string_T1}{}
\markboth{\textcolor{darkblue}{\textbf{\ipa{nɑ˩hĩ\#˥}}}}{}
\textcolor{teal}{\mytextsc{nom}} \hspace{4pt} Ton~: LM+\#H.
\textcolor{PineGreen}{\selectlanguage{french}Naxi (groupe ethnique).} \zh{纳西族。}  ¶ \textcolor{darkblue}{\textbf{\ipa{nɑ˩hĩ˧-mi˧ ɲi˥!}}} \textcolor{PineGreen}{\selectlanguage{french}c'est une femme naxi!} \zh{她是纳西族女人!}  
 ¶ \textcolor{darkblue}{\textbf{\ipa{nɑ˩hĩ˧-bɑ˧lɑ˥}}} \textcolor{PineGreen}{\selectlanguage{french}vêtements naxi, costume naxi} \zh{纳西族服装}  
 ¶ \textcolor{darkblue}{\textbf{\ipa{nɑ˩hĩ˧-ʐwɤ˧ so˥}}} \textcolor{PineGreen}{\selectlanguage{french}apprendre la langue naxi} \zh{学纳西语}  
 ¶ \textcolor{darkblue}{\textbf{\ipa{nɑ˩hĩ˧-tʰæ˧ɻæ˥}}} \textcolor{PineGreen}{\selectlanguage{french}livres naxi} \zh{纳西族的书}  
 \zh{量词}: \textcolor{darkblue}{\textbf{\ipa{v̩˧}}}  \mytextsc{clf}~: \textcolor{darkblue}{\textbf{\ipa{v̩˧}}} 
\lhead{\firstmark}
\rhead{\botmark}

\subsection{\hspace{-0.5cm} {\Large \textcolor{darkblue}{\textbf{\ipa{nɑ˩kwɤ˧}}}}\hspace{0.5cm}[\kern2pt{\textcolor{darkblue}{\textbf{\ipa{nɑ˧kwɤ˩}}}}\kern2pt]} \hypertarget{nA\string_Bkw7\string_M1}{}
\markboth{\textcolor{darkblue}{\textbf{\ipa{nɑ˩kwɤ˧}}}}{}
\textcolor{teal}{\mytextsc{nom}} \hspace{4pt} Ton~: LM.
\textcolor{PineGreen}{\selectlanguage{french}Potiron.} \zh{南瓜。}  \zh{量词}: \textcolor{darkblue}{\textbf{\ipa{ɭɯ˧}}}  \mytextsc{clf}~: \textcolor{darkblue}{\textbf{\ipa{ɭɯ˧}}} 
\lhead{\firstmark}
\rhead{\botmark}

\subsection{\hspace{-0.5cm} {\Large \textcolor{darkblue}{\textbf{\ipa{nɑ˩mi\#˥}}}}\hspace{0.5cm}[\kern2pt{\textcolor{darkblue}{\textbf{\ipa{nɑ˩mi˥}}}}\kern2pt]} \hypertarget{nA\string_Bmi\#\string_T1}{}
\markboth{\textcolor{darkblue}{\textbf{\ipa{nɑ˩mi\#˥}}}}{}
\textcolor{teal}{\mytextsc{nom}} \hspace{4pt} Ton~: LM+\#H.
\textcolor{PineGreen}{\selectlanguage{french}Une femme Na.} \zh{摩梭女人。} 
\lhead{\firstmark}
\rhead{\botmark}

\subsection{\hspace{-0.5cm} {\Large \textcolor{darkblue}{\textbf{\ipa{nɑ˩mv̩˥-nɑ˩dzi˩dzi˩}}}}\hspace{0.5cm}[\kern2pt{\textcolor{darkblue}{\textbf{\ipa{xxxx non-correspondance entre le nombre de morphèmes et le nombre de tons de morphèmes}}}}\kern2pt]} \hypertarget{nA\string_Bmv\string_=\string_T-nA\string_Bdzi\string_Bdzi\string_B1}{}
\markboth{\textcolor{darkblue}{\textbf{\ipa{nɑ˩mv̩˥-nɑ˩dzi˩dzi˩}}}}{}
\textcolor{teal}{\mytextsc{adjectif}} \hspace{4pt} Ton~: .
\textcolor{PineGreen}{\selectlanguage{french}Tout sombre, tout noir (il fait nuit noire).} \zh{很暗(天变得很暗)。} 
\lhead{\firstmark}
\rhead{\botmark}

\subsection{\hspace{-0.5cm} {\Large \textcolor{darkblue}{\textbf{\ipa{nɑ˩pv̩˧-qʰwɤ˧}}}}\hspace{0.5cm}[\kern2pt{\textcolor{darkblue}{\textbf{\ipa{nɑ˩pv̩˧qʰwɤ˧}}}}\kern2pt]} \hypertarget{nA\string_Bpv\string_=\string_M-q\string_hw7\string_M1}{}
\markboth{\textcolor{darkblue}{\textbf{\ipa{nɑ˩pv̩˧-qʰwɤ˧}}}}{}
\textcolor{teal}{\mytextsc{nom}} \hspace{4pt} Ton~: LM-.
\textcolor{PineGreen}{\selectlanguage{french}Empereur (emprunt au mongole?).} \zh{皇帝。}  ¶ \textcolor{darkblue}{\textbf{\ipa{ʈʂʰɯ˧ | nɑ˩pʰv̩˧-qʰwɤ˧-ni˩gv̩˩!}}} \textcolor{PineGreen}{\selectlanguage{french}Il vous prend des airs d'empereur! / Il se prend pour l'empereur! (Façon de se moquer d'un personnage qui veut en imposer à tous, qui se prend pour un grand chef.)} \zh{他摆出做皇帝的样子! / 他以为他是皇帝吧!(嘲笑一个自以为是的人)}  

\lhead{\firstmark}
\rhead{\botmark}

\subsection{\hspace{-0.5cm} {\Large \textcolor{darkblue}{\textbf{\ipa{nɑ˩tsʰi˩}}}}\hspace{0.5cm}[\kern2pt{\textcolor{darkblue}{\textbf{\ipa{nɑ˩tsʰi˩˥}}}}\kern2pt]} \hypertarget{nA\string_Bts\string_hi\string_B1}{}
\markboth{\textcolor{darkblue}{\textbf{\ipa{nɑ˩tsʰi˩}}}}{}
\textcolor{teal}{\mytextsc{nom}} \hspace{4pt} Ton~: L.
\textcolor{PineGreen}{\selectlanguage{french}Nom d'une montagne de Yongning.} \zh{一座山的名字。}  ¶ \textcolor{darkblue}{\textbf{\ipa{kɤ˧mv̩˧˥, | æ˧ʂæ˧, | ŋwɤ˧hɑ̃˩, | ʂwæ˧gv̩\#˥, | nɑ˩tsʰi˩˥ | -tɕʰɤ˧pɤ˧mi\#˥, | qv̩˧ɻ̍˧-ʈʂʰɑ˧nɑ˥ |}}} \textcolor{PineGreen}{\selectlanguage{french}Les six montagnes de Yongning qui portent un nom. Les autres sommets du voisinage n'ont pas une valeur symbolique comparable, et ne portent pas de nom communément utilisé.} \zh{永宁地区有固定名字的六座山。其它的山,因为没有重要的象征意义,因此没有取名。}  

\lhead{\firstmark}
\rhead{\botmark}

\subsection{\hspace{-0.5cm} {\Large \textcolor{darkblue}{\textbf{\ipa{nɑ˩zo\#˥}}}}\hspace{0.5cm}[\kern2pt{\textcolor{darkblue}{\textbf{\ipa{nɑ˩zo˥}}}}\kern2pt]} \hypertarget{nA\string_Bzo\#\string_T1}{}
\markboth{\textcolor{darkblue}{\textbf{\ipa{nɑ˩zo\#˥}}}}{}
\textcolor{teal}{\mytextsc{nom}} \hspace{4pt} Ton~: LM+\#H.
\textcolor{PineGreen}{\selectlanguage{french}Un homme Na.} \zh{摩梭男人。} 
\lhead{\firstmark}
\rhead{\botmark}

\subsection{\hspace{-0.5cm} {\Large \textcolor{darkblue}{\textbf{\ipa{nɑ˩-ʐwɤ˥}}}}\hspace{0.5cm}[\kern2pt{\textcolor{darkblue}{\textbf{\ipa{xxxx non-correspondance entre le nombre de morphèmes et le nombre de tons de morphèmes}}}}\kern2pt]} \hypertarget{nA\string_B-z`w7\string_T1}{}
\markboth{\textcolor{darkblue}{\textbf{\ipa{nɑ˩-ʐwɤ˥}}}}{}
\textcolor{teal}{\mytextsc{nom}} \hspace{4pt} Ton~: LH.
\textcolor{PineGreen}{\selectlanguage{french}Langue na: endonyme de la langue na.} \zh{本语言:摩梭话(纳语)。} 
\lhead{\firstmark}
\rhead{\botmark}

\subsection{\hspace{-0.5cm} {\Large \textcolor{darkblue}{\textbf{\ipa{nɑ˧˥}}}}\hspace{0.5cm}[\kern2pt{\textcolor{darkblue}{\textbf{\ipa{nɑ˧˥}}}}\kern2pt]} \hypertarget{nA\string_M\string_T1}{}
\markboth{\textcolor{darkblue}{\textbf{\ipa{nɑ˧˥}}}}{}
\textcolor{teal}{\mytextsc{verbe}} \hspace{4pt} Ton~: MH.
\textcolor{PineGreen}{\selectlanguage{french}Trembler.} \zh{发抖,颤抖。}  ¶ \textcolor{darkblue}{\textbf{\ipa{nɑ˩\textasciitilde{}nɑ˧-ze˥}}} \textcolor{PineGreen}{\selectlanguage{french}\mytextsc{red} \mytextsc{pfv}} \zh{发抖了}  
 ¶ \textcolor{darkblue}{\textbf{\ipa{le˧-nɑ˩\textasciitilde{}nɑ˩}}} \textcolor{PineGreen}{\selectlanguage{french}\mytextsc{accomp} \mytextsc{red}} \zh{\mytextsc{accomp} \mytextsc{red}}  
 ¶ \textcolor{darkblue}{\textbf{\ipa{lo˩qʰwɤ˥ | nɑ˩\textasciitilde{}nɑ˧˥}}} \textcolor{PineGreen}{\selectlanguage{french}la main tremble} \zh{手抖}  

\lhead{\firstmark}
\rhead{\botmark}

\subsection{\hspace{-0.5cm} {\Large \textcolor{darkblue}{\textbf{\ipa{nɑ˩˧}}}}\hspace{0.5cm}[\kern2pt{\textcolor{darkblue}{\textbf{\ipa{nɑ˩˥}}}}\kern2pt]} \hypertarget{nA\string_B\string_M1}{}
\markboth{\textcolor{darkblue}{\textbf{\ipa{nɑ˩˧}}}}{}
\textcolor{teal}{\mytextsc{nom}} \hspace{4pt} Ton~: LM.
\textcolor{PineGreen}{\selectlanguage{french}Endonyme: les Na.} \zh{自称:摩梭族。}  ¶ \textcolor{darkblue}{\textbf{\ipa{nɑ˩-mv̩˧ nɑ˥-di˩ |}}} \textcolor{PineGreen}{\selectlanguage{french}le territoire des Na} \zh{摩梭人地区}  
 ¶ \textcolor{darkblue}{\textbf{\ipa{ə˧ʝi˧-ʂɯ˥ʝi˩, | nɑ˩zo˧-tɑ˥mv̩˩-ɳɯ˩ | dʑo˧-ɲi˥-tsɯ˩!}}} \textcolor{PineGreen}{\selectlanguage{french}Autrefois, notre tradition, elle en parlait ! / Notre tradition, elle en parle! (Contexte: quand on fait référence à la coutume locale: ce qu'il est interdit de faire, ce qu'on est autorisé à faire…)} \zh{过去,摩梭人的传统(里)有(关于这些问题的说法)嘛!}  
 \zh{量词}: \textcolor{darkblue}{\textbf{\ipa{v̩˧}}}  \mytextsc{clf}~: \textcolor{darkblue}{\textbf{\ipa{v̩˧}}} 
\lhead{\firstmark}
\rhead{\botmark}

\subsection{\hspace{-0.5cm} {\Large \textcolor{darkblue}{\textbf{\ipa{ni˥}}}}\hspace{0.5cm}[\kern2pt{\textcolor{darkblue}{\textbf{\ipa{ni˥}}}}\kern2pt]} \hypertarget{ni\string_T1}{}
\markboth{\textcolor{darkblue}{\textbf{\ipa{ni˥}}}}{}
\textcolor{teal}{\mytextsc{nom}} \hspace{4pt} Ton~: \#H.
\textcolor{PineGreen}{\selectlanguage{french}Amaranthe, \textit{Amaranthus}: minuscule graine qui n'est pas une céréale mais a une valeur nutritionnelle comparable aux céréales.} \zh{苋米。}  \zh{量词}: \textcolor{darkblue}{\textbf{\ipa{po˧}}}  \mytextsc{clf}~: \textcolor{darkblue}{\textbf{\ipa{po˧}}} 
\lhead{\firstmark}
\rhead{\botmark}

\subsection{\hspace{-0.5cm} {\Large \textcolor{darkblue}{\textbf{\ipa{ni˧fv̩˥}}}}\hspace{0.5cm}[\kern2pt{\textcolor{darkblue}{\textbf{\ipa{ni˧fv̩˥}}}}\kern2pt]} \hypertarget{ni\string_Mfv\string_=\string_T1}{}
\markboth{\textcolor{darkblue}{\textbf{\ipa{ni˧fv̩˥}}}}{}
\textcolor{teal}{\mytextsc{nom}} \hspace{4pt} Ton~: H\#.
\textcolor{PineGreen}{\selectlanguage{french}Très grand sac/très grande poche; en cuir, pour emballer les produits que l'on transportait sur de longues distances; s'emploie aussi pour désigner le sac de toile dans lequel on place le corps d'un défunt pendant son inhumation provisoire.} \zh{大包:用来包装物品的皮包(马帮用的),或者来装尸体的麻布包(为了在火葬前暂时存放尸体)。}  ¶ \textcolor{darkblue}{\textbf{\ipa{jɤ˧ŋɤ˧-ni˧fv̩˥}}} \textcolor{PineGreen}{\selectlanguage{french}grand sac de Chengdu. Expression employée car les sacs de ce type provenaient généralement de la région de Chengdu.} \zh{成都大包。(据说这类的包一般是成都地区生产的。)}  
 \zh{量词}: \textcolor{darkblue}{\textbf{\ipa{ɭɯ˧}}}  \mytextsc{clf}~: \textcolor{darkblue}{\textbf{\ipa{ɭɯ˧}}} 
\lhead{\firstmark}
\rhead{\botmark}

\subsection{\hspace{-0.5cm} {\Large \textcolor{darkblue}{\textbf{\ipa{‑ni˧gv̩˧˥}}}}\hspace{0.5cm}[\kern2pt{\textcolor{darkblue}{\textbf{\ipa{ni˧gv̩˧˥}}}}\kern2pt]} \hypertarget{‑ni\string_Mgv\string_=\string_M\string_T1}{}
\markboth{\textcolor{darkblue}{\textbf{\ipa{‑ni˧gv̩˧˥}}}}{}
\textcolor{teal}{\mytextsc{adverbe}} \hspace{4pt} Ton~: MH\#.
\textcolor{PineGreen}{\selectlanguage{french}Comme (être comme, être semblable à).} \zh{如、像。}  ¶ \textcolor{darkblue}{\textbf{\ipa{zɯ˧hṽ˩-ni˩gv̩˩}}} \textcolor{PineGreen}{\selectlanguage{french}comme de l’herbe, c'est-à-dire vert} \zh{像草,等于绿色}  
 ¶ \textcolor{darkblue}{\textbf{\ipa{æ̃˧qæ˩-ni˩gv̩˩}}} \textcolor{PineGreen}{\selectlanguage{french}bleu-vert; littéralement “comme un perroquet”, c'est-à-dire “couleur perroquet”} \zh{像鹦鹉,等于青色}  
 ¶ \textcolor{darkblue}{\textbf{\ipa{lwæ˩pʰv̩˩-ni˥gv̩˩}}} \textcolor{PineGreen}{\selectlanguage{french}comme de la cendre = de couleur grise} \zh{灰色}  
 ¶ \textcolor{darkblue}{\textbf{\ipa{sɯ˧pv̩˩-ni˩gv̩˩}}} \textcolor{PineGreen}{\selectlanguage{french}comme une vessie, en forme de vessie} \zh{像膀胱}  
 ¶ \textcolor{darkblue}{\textbf{\ipa{(nv̩˩mi˩˥ | ) ɖɯ˧-v̩˧-ni˩gv̩˩}}} \textcolor{PineGreen}{\selectlanguage{french}comme un seul cœur, (leur) cœur est à l'unisson} \zh{一条心,想得一致}  
 ¶ \textcolor{darkblue}{\textbf{\ipa{dzi˩bi˩-ni˩gv̩˩˥}}} \textcolor{PineGreen}{\selectlanguage{french}s’habituer, s'accoutumer, prendre ses habitudes (dans un environnement)} \zh{习惯(一个环境)}  
 ¶ \textcolor{darkblue}{\textbf{\ipa{[élicitation lors de la transcription de Agriculture.70] li˩ ʈʰɯ˩-bi˩-ni˩-gv̩˩˥}}} \textcolor{PineGreen}{\selectlanguage{french}avoir l'habitude de boire du thé, être un buveur de thé} \zh{习惯喝茶、有喝茶的习惯}  
 ¶ \textcolor{darkblue}{\textbf{\ipa{ʈʂʰɯ˧ | ʂɯ˧-bi˧-ni˩gv̩˩!}}} \textcolor{PineGreen}{\selectlanguage{french}On dirait qu'il/elle va mourir!} \zh{他好像要死了!}  

\lhead{\firstmark}
\rhead{\botmark}

\subsection{\hspace{-0.5cm} {\Large \textcolor{darkblue}{\textbf{\ipa{ni˧mi\#˥}}}}\hspace{0.5cm}[\kern2pt{\textcolor{darkblue}{\textbf{\ipa{ni˧mi˧}}}}\kern2pt]} \hypertarget{ni\string_Mmi\#\string_T1}{}
\markboth{\textcolor{darkblue}{\textbf{\ipa{ni˧mi\#˥}}}}{}
\textcolor{teal}{\mytextsc{nom}} \hspace{4pt} Ton~: \#H.
\textcolor{PineGreen}{\selectlanguage{french}Soeurs (aînées ou cadettes).} \zh{姐妹。}  ¶ \textcolor{darkblue}{\textbf{\ipa{ʈʂʰɯ˧ | ʈæ˧ʂɯ˧-bv̩˧ | ni˧mi˧ ɲi˥.}}} \textcolor{PineGreen}{\selectlanguage{french}Elle est soeur de \textcolor{darkblue}{\textbf{\ipa{/ʈæ˧ʂɯ˧/}}}.} \zh{她是达石的姐姐(或妹妹)}  
 \zh{量词}: \textcolor{darkblue}{\textbf{\ipa{v̩˧}}}  \mytextsc{clf}~: \textcolor{darkblue}{\textbf{\ipa{v̩˧}}} 
\lhead{\firstmark}
\rhead{\botmark}

\subsection{\hspace{-0.5cm} {\Large \textcolor{darkblue}{\textbf{\ipa{njæ˥-qv̩˩}}}}\hspace{0.5cm}[\kern2pt{\textcolor{darkblue}{\textbf{\ipa{njæ˥qv̩˩}}}}\kern2pt]} \hypertarget{nj\{\string_T-qv\string_=\string_B1}{}
\markboth{\textcolor{darkblue}{\textbf{\ipa{njæ˥-qv̩˩}}}}{}
\textcolor{teal}{\mytextsc{verbe}} \hspace{4pt} Ton~: H\#-.
\textcolor{PineGreen}{\selectlanguage{french}Détourner le regard, détourner la tête, se détourner.} \zh{看别的方向(蔑视态度)。}  ¶ \textcolor{darkblue}{\textbf{\ipa{hĩ˧ njæ˧qv̩˥}}} \textcolor{PineGreen}{\selectlanguage{french}se détourner de quelqu'un, détourner la tête face à quelqu'un (que l'on méprise, déteste...)} \zh{看别的方向,不直接看(蔑视态度)}  
 ¶ \textcolor{darkblue}{\textbf{\ipa{mɤ˧-njæ˥qv̩˩}}} \textcolor{PineGreen}{\selectlanguage{french}\mytextsc{neg}: ne pas détourner le regard (face à quelqu'un)} \zh{\mytextsc{neg}}  

\lhead{\firstmark}
\rhead{\botmark}

\subsection{\hspace{-0.5cm} {\Large \textcolor{darkblue}{\textbf{\ipa{njæ˧bæ˥}}}}\hspace{0.5cm}[\kern2pt{\textcolor{darkblue}{\textbf{\ipa{njæ˧bæ˥}}}}\kern2pt]} \hypertarget{nj\{\string_Mb\{\string_T1}{}
\markboth{\textcolor{darkblue}{\textbf{\ipa{njæ˧bæ˥}}}}{}
\textcolor{teal}{\mytextsc{nom}} \hspace{4pt} Ton~: H\#.
\textcolor{PineGreen}{\selectlanguage{french}Larme.} \zh{眼泪。}  \zh{量词}: \textcolor{darkblue}{\textbf{\ipa{ʈʰɤ˥}}}  \mytextsc{clf}~: \textcolor{darkblue}{\textbf{\ipa{ʈʰɤ˥}}} 
\lhead{\firstmark}
\rhead{\botmark}

\subsection{\hspace{-0.5cm} {\Large \textcolor{darkblue}{\textbf{\ipa{njæ˧tsɯ˩}}}}\hspace{0.5cm}[\kern2pt{\textcolor{darkblue}{\textbf{\ipa{njæ˧tsɯ˩}}}}\kern2pt]} \hypertarget{nj\{\string_MtsM\string_B1}{}
\markboth{\textcolor{darkblue}{\textbf{\ipa{njæ˧tsɯ˩}}}}{}
\textcolor{teal}{\mytextsc{nom}} \hspace{4pt} Ton~: L\#.
\ding{202} \textcolor{PineGreen}{\selectlanguage{french}Sourcil.} \zh{眉毛。}  ¶ \textcolor{darkblue}{\textbf{\ipa{njæ˧tsɯ˩-ɖæ˩}}} \textcolor{PineGreen}{\selectlanguage{french}sourcil (formulation permettant de lever l'ambiguïté du terme, qui peut signifier 'sourcil' aussi bien que 'cil')} \zh{眉毛}  
 ¶ \textcolor{darkblue}{\textbf{\ipa{njæ˧tsɯ˩ | mv̩˩tɕo˧ kʰɯ˧˥}}} \textcolor{PineGreen}{\selectlanguage{french}froncer les sourcils (littéralement “abaisser les sourcils”)} \zh{皱眉毛}  
 \zh{量词}: \textcolor{darkblue}{\textbf{\ipa{kʰwɤ˥}}} \ding{203} \textcolor{PineGreen}{\selectlanguage{french}Cil.} \zh{睫毛、眼睫毛、眼毛。}  ¶ \textcolor{darkblue}{\textbf{\ipa{njæ˧tsɯ˩-ʂæ˩}}} \textcolor{PineGreen}{\selectlanguage{french}cil (formulation permettant de lever l'ambiguïté du terme, qui peut signifier “sourcil” aussi bien que “cil”)}  
 \mytextsc{clf}~: \textcolor{darkblue}{\textbf{\ipa{kʰwɤ˥}}} 
\lhead{\firstmark}
\rhead{\botmark}

\subsection{\hspace{-0.5cm} {\Large \textcolor{darkblue}{\textbf{\ipa{njæ˧=zɯ˩}}}}\hspace{0.5cm}[\kern2pt{\textcolor{darkblue}{\textbf{\ipa{njæ˧zɯ˩}}}}\kern2pt]} \hypertarget{nj\{\string_M=zM\string_B1}{}
\markboth{\textcolor{darkblue}{\textbf{\ipa{njæ˧=zɯ˩}}}}{}
\textcolor{teal}{\mytextsc{pronom}} \hspace{4pt} Ton~: L\#.
\textcolor{PineGreen}{\selectlanguage{french}Pronom de première personne duelle exclusive: nous deux (le locuteur et une autre personne qui n'est pas l'interlocuteur).} \zh{我们两个。}  ¶ \textcolor{darkblue}{\textbf{\ipa{ɑ˩ʁo˧(-hĩ˧) | njæ˧zɯ˩ ho˩-dʑo˩!}}} \textcolor{PineGreen}{\selectlanguage{french}(On ne peut pas rester car) les gens de la famille nous attendent!} \zh{(我们两个不能再呆在这里了,)家里在等我们!}  

\lhead{\firstmark}
\rhead{\botmark}

\subsection{\hspace{-0.5cm} {\Large \textcolor{darkblue}{\textbf{\ipa{njæ˩pʰv̩˧}}}}\hspace{0.5cm}[\kern2pt{\textcolor{darkblue}{\textbf{\ipa{njæ˩pʰv̩˥}}}}\kern2pt]} \hypertarget{nj\{\string_Bp\string_hv\string_=\string_M1}{}
\markboth{\textcolor{darkblue}{\textbf{\ipa{njæ˩pʰv̩˧}}}}{}
\textcolor{teal}{\mytextsc{nom}} \hspace{4pt} Ton~: LM.
\textcolor{PineGreen}{\selectlanguage{french}Blanc des yeux.} \zh{白眼球。}  \zh{量词}: \textcolor{darkblue}{\textbf{\ipa{ɭɯ˧}}}  \mytextsc{clf}~: \textcolor{darkblue}{\textbf{\ipa{ɭɯ˧}}} 
\lhead{\firstmark}
\rhead{\botmark}

\subsection{\hspace{-0.5cm} {\Large \textcolor{darkblue}{\textbf{\ipa{njæ˩qwæ˧˥}}}}\hspace{0.5cm}[\kern2pt{\textcolor{darkblue}{\textbf{\ipa{njæ˩qwæ˧˥}}}}\kern2pt]} \hypertarget{nj\{\string_Bqw\{\string_M\string_T1}{}
\markboth{\textcolor{darkblue}{\textbf{\ipa{njæ˩qwæ˧˥}}}}{}
\textcolor{teal}{\mytextsc{adjectif}} \hspace{4pt} Ton~: LM+MH\#.
\textcolor{PineGreen}{\selectlanguage{french}Aveugle.} \zh{眼睛瞎了。}  ¶ \textcolor{darkblue}{\textbf{\ipa{ʈʂʰɯ˧ | njæ˩qwæ˧-ze˥}}} \textcolor{PineGreen}{\selectlanguage{french}Elle/il est devenu(e) aveugle.} \zh{他眼睛瞎了。}  
 ¶ \textcolor{darkblue}{\textbf{\ipa{ʈʂʰɯ˧ | njæ˩qwæ˧ ɲi˥.}}} \textcolor{PineGreen}{\selectlanguage{french}Elle/il est aveugle.} \zh{他是瞎子。}  
 ¶ \textcolor{darkblue}{\textbf{\ipa{njæ˩qwæ˧-mi\#˥}}} \textcolor{PineGreen}{\selectlanguage{french}femme aveugle} \zh{眼睛瞎了的女人}  
 ¶ \textcolor{darkblue}{\textbf{\ipa{njæ˩qwæ˧-zo\#˥}}} \textcolor{PineGreen}{\selectlanguage{french}homme aveugle} \zh{眼睛瞎了的男人}  
 ¶ \textcolor{darkblue}{\textbf{\ipa{njæ˩qwæ˧-hĩ\#˥}}} \textcolor{PineGreen}{\selectlanguage{french}personne aveugle} \zh{瞎子}  
 \zh{量词}: \textcolor{darkblue}{\textbf{\ipa{v̩˧}}}  \mytextsc{clf}~: \textcolor{darkblue}{\textbf{\ipa{v̩˧}}} 
\lhead{\firstmark}
\rhead{\botmark}

\subsection{\hspace{-0.5cm} {\Large \textcolor{darkblue}{\textbf{\ipa{njæ˩qʰæ\#˥}}}}\hspace{0.5cm}[\kern2pt{\textcolor{darkblue}{\textbf{\ipa{njæ˩qʰæ˥}}}}\kern2pt]} \hypertarget{nj\{\string_Bq\string_h\{\#\string_T1}{}
\markboth{\textcolor{darkblue}{\textbf{\ipa{njæ˩qʰæ\#˥}}}}{}
\textcolor{teal}{\mytextsc{nom}} \hspace{4pt} Ton~: LM+\#H.
\textcolor{PineGreen}{\selectlanguage{french}Chassie.} \zh{眼屎。}  \zh{量词}: \textcolor{darkblue}{\textbf{\ipa{kʰwɤ˥}}}  \mytextsc{clf}~: \textcolor{darkblue}{\textbf{\ipa{kʰwɤ˥}}} 
\lhead{\firstmark}
\rhead{\botmark}

\subsection{\hspace{-0.5cm} {\Large \textcolor{darkblue}{\textbf{\ipa{njɤ˧di˧˥}}}}\hspace{0.5cm}[\kern2pt{\textcolor{darkblue}{\textbf{\ipa{njɤ˧di˧˥}}}}\kern2pt]} \hypertarget{nj7\string_Mdi\string_M\string_T1}{}
\markboth{\textcolor{darkblue}{\textbf{\ipa{njɤ˧di˧˥}}}}{}
\textcolor{teal}{\mytextsc{nom}} \hspace{4pt} Ton~: MH\#.
\textcolor{PineGreen}{\selectlanguage{french}Colle.} \zh{胶。}  \zh{量词}: \textcolor{darkblue}{\textbf{\ipa{kʰwɤ˥}}}  \mytextsc{clf}~: \textcolor{darkblue}{\textbf{\ipa{kʰwɤ˥}}} 
\lhead{\firstmark}
\rhead{\botmark}

\subsection{\hspace{-0.5cm} {\Large \textcolor{darkblue}{\textbf{\ipa{njɤ˧kv̩˩}}}}\hspace{0.5cm}[\kern2pt{\textcolor{darkblue}{\textbf{\ipa{njɤ˧kv̩˩}}}}\kern2pt]} \hypertarget{nj7\string_Mkv\string_=\string_B1}{}
\markboth{\textcolor{darkblue}{\textbf{\ipa{njɤ˧kv̩˩}}}}{}
\textcolor{teal}{\mytextsc{nom}} \hspace{4pt} Ton~: L\#.
\textcolor{PineGreen}{\selectlanguage{french}Pommettes.} \zh{颧骨。}  \zh{量词}: \textcolor{darkblue}{\textbf{\ipa{ɭɯ˧}}}  \mytextsc{clf}~: \textcolor{darkblue}{\textbf{\ipa{ɭɯ˧}}} \textit{Voir~:} \hyperlink{}{\textcolor{darkblue}{\textbf{\ipa{kv̩˩kv̩˩}}}} 
\lhead{\firstmark}
\rhead{\botmark}

\subsection{\hspace{-0.5cm} {\Large \textcolor{darkblue}{\textbf{\ipa{njɤ˧kv̩˩-njɤ˩tsʰɤ˩}}}}\hspace{0.5cm}[\kern2pt{\textcolor{darkblue}{\textbf{\ipa{njɤ˧kv̩˩njɤ˧tsʰɤ˧}}}}\kern2pt]} \hypertarget{nj7\string_Mkv\string_=\string_B-nj7\string_Bts\string_h7\string_B1}{}
\markboth{\textcolor{darkblue}{\textbf{\ipa{njɤ˧kv̩˩-njɤ˩tsʰɤ˩}}}}{}
\textcolor{teal}{\mytextsc{adjectif}} \hspace{4pt} Ton~: L\#-.
\textcolor{PineGreen}{\selectlanguage{french}Belle; qui a un beau visage, qui a des traits gracieux.} \zh{美丽、面貌美。}  ¶ \textcolor{darkblue}{\textbf{\ipa{ə˧mi˧! | mv̩˩zo˩ ʈʂʰɯ˩-ɭɯ˥ | njɤ˧kv̩˩-njɤ˩tsʰɤ˩! | ɖwæ˧˥ | ə˧v̩˧˥!}}} \textcolor{PineGreen}{\selectlanguage{french}Eh bien, cette jeune fille est vraiment belle! Très jolie!} \zh{啊呀,这个少女真美丽!很漂亮!}  
 ¶ \textcolor{darkblue}{\textbf{\ipa{njɤ˧kv̩˩njɤ˩tsʰɤ˩ | ʐwæ˩˥}}} \textcolor{PineGreen}{\selectlanguage{french}particulièrement belle} \zh{非常美}  
\textit{Voir~:} \textcolor{darkblue}{\textbf{\ipa{njɤ˧kv̩˩, tsʰɤ˧˥a}}} 
\lhead{\firstmark}
\rhead{\botmark}

\subsection{\hspace{-0.5cm} {\Large \textcolor{darkblue}{\textbf{\ipa{njɤ˧le˧gv̩\#˥}}}}\hspace{0.5cm}[\kern2pt{\textcolor{darkblue}{\textbf{\ipa{njɤ˧le˧gv̩˧}}}}\kern2pt]} \hypertarget{nj7\string_Mle\string_Mgv\string_=\#\string_T1}{}
\markboth{\textcolor{darkblue}{\textbf{\ipa{njɤ˧le˧gv̩\#˥}}}}{}
\textcolor{teal}{\mytextsc{nom}} \hspace{4pt} Ton~: \#H.
\textcolor{PineGreen}{\selectlanguage{french}Journée, plein jour.} \zh{白天、大白天。}  ¶ \textcolor{darkblue}{\textbf{\ipa{ɲi˧mi˧-njɤ˩le˩gv̩˩}}} \textcolor{PineGreen}{\selectlanguage{french}même sens} \zh{白天}  

\lhead{\firstmark}
\rhead{\botmark}

\subsection{\hspace{-0.5cm} {\Large \textcolor{darkblue}{\textbf{\ipa{njɤ˧mv̩˥}}}}\hspace{0.5cm}[\kern2pt{\textcolor{darkblue}{\textbf{\ipa{njɤ˧mv̩˥}}}}\kern2pt]} \hypertarget{nj7\string_Mmv\string_=\string_T1}{}
\markboth{\textcolor{darkblue}{\textbf{\ipa{njɤ˧mv̩˥}}}}{}
\textcolor{teal}{\mytextsc{nom}} \hspace{4pt} Ton~: H\#.
\textcolor{PineGreen}{\selectlanguage{french}Sorte de fourrage pour les cochons, \textit{Chenopodium album}. (Il y a en tout trois sortes de fourrage pour les cochons.).} \zh{灰条菜、灰灰菜:喂猪的牧草。} Dialecte chinois local~:\zh{灰凋。} \zh{量词}: \textcolor{darkblue}{\textbf{\ipa{qɑ˩}}}  \mytextsc{clf}~: \textcolor{darkblue}{\textbf{\ipa{qɑ˩}}} 
\lhead{\firstmark}
\rhead{\botmark}

\subsection{\hspace{-0.5cm} {\Large \textcolor{darkblue}{\textbf{\ipa{njɤ˧mv̩˥-mi˩}}}}\hspace{0.5cm}[\kern2pt{\textcolor{darkblue}{\textbf{\ipa{njɤ˧mv̩˥mi˩}}}}\kern2pt]} \hypertarget{nj7\string_Mmv\string_=\string_T-mi\string_B1}{}
\markboth{\textcolor{darkblue}{\textbf{\ipa{njɤ˧mv̩˥-mi˩}}}}{}
\textcolor{teal}{\mytextsc{nom}} \hspace{4pt} Ton~: H\#-L.
\textcolor{PineGreen}{\selectlanguage{french}Chameau.} \zh{骆驼。}  ¶ \textcolor{darkblue}{\textbf{\ipa{njɤ˧mv̩˥mi˩-zo˩}}} \textcolor{PineGreen}{\selectlanguage{french}enfant du chameau, petit chameau} \zh{小骆驼}  
 ¶ \textcolor{darkblue}{\textbf{\ipa{njɤ˧mv̩˥mi˩-pʰv̩˩}}} \textcolor{PineGreen}{\selectlanguage{french}chameau mâle} \zh{公骆驼}  
 \zh{量词}: \textcolor{darkblue}{\textbf{\ipa{mi˩}}}  \mytextsc{clf}~: \textcolor{darkblue}{\textbf{\ipa{mi˩}}} 
\lhead{\firstmark}
\rhead{\botmark}

\subsection{\hspace{-0.5cm} {\Large \textcolor{darkblue}{\textbf{\ipa{njɤ˧nɑ˩}}}}\hspace{0.5cm}[\kern2pt{\textcolor{darkblue}{\textbf{\ipa{njɤ˧nɑ˩}}}}\kern2pt]} \hypertarget{nj7\string_MnA\string_B1}{}
\markboth{\textcolor{darkblue}{\textbf{\ipa{njɤ˧nɑ˩}}}}{}
\textcolor{teal}{\mytextsc{nom}} \hspace{4pt} Ton~: L\#.
\textcolor{PineGreen}{\selectlanguage{french}Prunelle.} \zh{眼珠。}  \zh{量词}: \textcolor{darkblue}{\textbf{\ipa{ɭɯ˧}}}  \mytextsc{clf}~: \textcolor{darkblue}{\textbf{\ipa{ɭɯ˧}}} 
\lhead{\firstmark}
\rhead{\botmark}

\subsection{\hspace{-0.5cm} {\Large \textcolor{darkblue}{\textbf{\ipa{njɤ˧ʈʂɤ˥}}}}\hspace{0.5cm}[\kern2pt{\textcolor{darkblue}{\textbf{\ipa{njɤ˧ʈʂɤ˥}}}}\kern2pt]} \hypertarget{nj7\string_Mt`s`7\string_T1}{}
\markboth{\textcolor{darkblue}{\textbf{\ipa{njɤ˧ʈʂɤ˥}}}}{}
\textcolor{teal}{\mytextsc{nom}} \hspace{4pt} Ton~: H\#.
\textcolor{PineGreen}{\selectlanguage{french}Sornet, herbe à aiguilles, \textit{Bidens pilosa L.}: plante de la famille des Asteraceae, dont les graines noires, fines et allongées, de 5 à 10 mm, s'accrochent aux vêtements et aux poils d'animaux par deux piquants fins, situés à l'une de leurs extrémités.} \zh{鬼针草。} 
\lhead{\firstmark}
\rhead{\botmark}

\subsection{\hspace{-0.5cm} {\Large \textcolor{darkblue}{\textbf{\ipa{njɤ˩}}}}\hspace{0.5cm}[\kern2pt{\textcolor{darkblue}{\textbf{\ipa{njɤ˩˥}}}}\kern2pt]} \hypertarget{nj7\string_B1}{}
\markboth{\textcolor{darkblue}{\textbf{\ipa{njɤ˩}}}}{}
\textcolor{teal}{\mytextsc{pronom}} \hspace{4pt} Ton~: L.
\textcolor{PineGreen}{\selectlanguage{french}Pronom de 1e personne du singulier.} \zh{我。}  ¶ \textcolor{darkblue}{\textbf{\ipa{njɤ˩ ɲi˩˥!}}} \textcolor{PineGreen}{\selectlanguage{french}C'est moi! (Contexte: quelqu'un frappe à la porte, on demande qui c'est, et on reçoit pour réponse: “C'est moi!”)} \zh{是我!(情景:一个人敲门,里面的人问是谁,人家回答:“是我!”)}  
 ¶ \textcolor{darkblue}{\textbf{\ipa{njɤ˧ no˧ lɑ˧˥}}} \textcolor{PineGreen}{\selectlanguage{french}je te frappe} \zh{我打你}  

\lhead{\firstmark}
\rhead{\botmark}

\subsection{\hspace{-0.5cm} {\Large \textcolor{darkblue}{\textbf{\ipa{njɤ˩\textsubscript{b}}}}}\hspace{0.5cm}[\kern2pt{\textcolor{darkblue}{\textbf{\ipa{njɤ˩˥}}}}\kern2pt]} \hypertarget{nj7\string_Bb1}{}
\markboth{\textcolor{darkblue}{\textbf{\ipa{njɤ˩\textsubscript{b}}}}}{}
\textcolor{teal}{\mytextsc{verbe}} \hspace{4pt} Ton~: L\textsubscript{b}.
\textcolor{PineGreen}{\selectlanguage{french}Décortiquer le riz.} \zh{舂米。}  ¶ \textcolor{darkblue}{\textbf{\ipa{le˧-njɤ˩-ze˩}}} \textcolor{PineGreen}{\selectlanguage{french}\mytextsc{accomp} \string_ \mytextsc{pfv}} \zh{舂了}  
 ¶ \textcolor{darkblue}{\textbf{\ipa{hɑ˧ njɤ˧˥}}} \textcolor{PineGreen}{\selectlanguage{french}décortiquer du riz} \zh{舂米}  
 ¶ \textcolor{darkblue}{\textbf{\ipa{hɑ˧ | le˧-njɤ˩}}} \textcolor{PineGreen}{\selectlanguage{french}décortiquer du riz} \zh{舂米}  
 ¶ \textcolor{darkblue}{\textbf{\ipa{hɑ˧ | ɖɯ˧-njɤ˧\textasciitilde{}njɤ˩-ɻ̍˩}}} \textcolor{PineGreen}{\selectlanguage{french}riz - \mytextsc{délimitatif} \string_ \mytextsc{red} \mytextsc{inchoatif} : décortiquer un peu le riz} \zh{把米舂一舂}  

\lhead{\firstmark}
\rhead{\botmark}

\subsection{\hspace{-0.5cm} {\Large \textcolor{darkblue}{\textbf{\ipa{njɤ˩bi˥}}}}\hspace{0.5cm}[\kern2pt{\textcolor{darkblue}{\textbf{\ipa{njɤ˩bi˥}}}}\kern2pt]} \hypertarget{nj7\string_Bbi\string_T1}{}
\markboth{\textcolor{darkblue}{\textbf{\ipa{njɤ˩bi˥}}}}{}
\textcolor{teal}{\mytextsc{nom}} \hspace{4pt} Ton~: LH.
\textcolor{PineGreen}{\selectlanguage{french}Paupière supérieure.} \zh{上眼皮。}  \zh{量词}: \textcolor{darkblue}{\textbf{\ipa{ɭɯ˧}}}  \mytextsc{clf}~: \textcolor{darkblue}{\textbf{\ipa{ɭɯ˧}}} 
\lhead{\firstmark}
\rhead{\botmark}

\subsection{\hspace{-0.5cm} {\Large \textcolor{darkblue}{\textbf{\ipa{njɤ˩-gɤ˧lɑ˩}}}}\hspace{0.5cm}[\kern2pt{\textcolor{darkblue}{\textbf{\ipa{njɤ˧gɤ˧lɑ˩}}}}\kern2pt]} \hypertarget{nj7\string_B-g7\string_MlA\string_B1}{}
\markboth{\textcolor{darkblue}{\textbf{\ipa{njɤ˩-gɤ˧lɑ˩}}}}{}
\textcolor{teal}{\mytextsc{nom}} \hspace{4pt} Ton~: L-L\#.
\textcolor{PineGreen}{\selectlanguage{french}Prunelle.} \zh{眼珠。}  \zh{量词}: \textcolor{darkblue}{\textbf{\ipa{ɭɯ˧}}}  \mytextsc{clf}~: \textcolor{darkblue}{\textbf{\ipa{ɭɯ˧}}} 
\lhead{\firstmark}
\rhead{\botmark}

\subsection{\hspace{-0.5cm} {\Large \textcolor{darkblue}{\textbf{\ipa{njɤ˩kʰi\#˥}}}}\hspace{0.5cm}[\kern2pt{\textcolor{darkblue}{\textbf{\ipa{njɤ˩kʰi˥}}}}\kern2pt]} \hypertarget{nj7\string_Bk\string_hi\#\string_T1}{}
\markboth{\textcolor{darkblue}{\textbf{\ipa{njɤ˩kʰi\#˥}}}}{}
\textcolor{teal}{\mytextsc{nom}} \hspace{4pt} Ton~: LM+\#H.
\textcolor{PineGreen}{\selectlanguage{french}Paupière inférieure.} \zh{下眼皮。}  \zh{量词}: \textcolor{darkblue}{\textbf{\ipa{kʰwɤ˥}}}  \mytextsc{clf}~: \textcolor{darkblue}{\textbf{\ipa{kʰwɤ˥}}} 
\lhead{\firstmark}
\rhead{\botmark}

\subsection{\hspace{-0.5cm} {\Large \textcolor{darkblue}{\textbf{\ipa{njɤ˩ɭɯ˧}}}}\hspace{0.5cm}[\kern2pt{\textcolor{darkblue}{\textbf{\ipa{njɤ˩ɭɯ˥}}}}\kern2pt]} \hypertarget{nj7\string_Bl\string_RM\string_M1}{}
\markboth{\textcolor{darkblue}{\textbf{\ipa{njɤ˩ɭɯ˧}}}}{}
\textcolor{teal}{\mytextsc{nom}} \hspace{4pt} Ton~: LM.
\textcolor{PineGreen}{\selectlanguage{french}Œil.} \zh{眼睛。}  \zh{量词}: \textcolor{darkblue}{\textbf{\ipa{ɭɯ˧}}}  \mytextsc{clf}~: \textcolor{darkblue}{\textbf{\ipa{ɭɯ˧}}} 
\lhead{\firstmark}
\rhead{\botmark}

\subsection{\hspace{-0.5cm} {\Large \textcolor{darkblue}{\textbf{\ipa{njɤ˩qʰwɤ˧˥}}}}\hspace{0.5cm}[\kern2pt{\textcolor{darkblue}{\textbf{\ipa{njɤ˩qʰwɤ˧˥}}}}\kern2pt]} \hypertarget{nj7\string_Bq\string_hw7\string_M\string_T1}{}
\markboth{\textcolor{darkblue}{\textbf{\ipa{njɤ˩qʰwɤ˧˥}}}}{}
\textcolor{teal}{\mytextsc{nom}} \hspace{4pt} Ton~: LM+MH\#.
\textcolor{PineGreen}{\selectlanguage{french}Orbite (de l'œil).} \zh{眼眶。}  \zh{量词}: \textcolor{darkblue}{\textbf{\ipa{ɭɯ˧}}}  \mytextsc{clf}~: \textcolor{darkblue}{\textbf{\ipa{ɭɯ˧}}} 
\lhead{\firstmark}
\rhead{\botmark}

\subsection{\hspace{-0.5cm} {\Large \textcolor{darkblue}{\textbf{\ipa{njɤ˩-tse˧\textasciitilde{}tse˩}}}}\hspace{0.5cm}[\kern2pt{\textcolor{darkblue}{\textbf{\ipa{njɤ˧tse˧tse˩}}}}\kern2pt]} \hypertarget{nj7\string_B-tse\string_M~tse\string_B1}{}
\markboth{\textcolor{darkblue}{\textbf{\ipa{njɤ˩-tse˧\textasciitilde{}tse˩}}}}{}
\textcolor{teal}{\mytextsc{nom}} \hspace{4pt} Ton~: L-L\#.
\textcolor{PineGreen}{\selectlanguage{french}Prêle ramifiée, prêle rameuse, \textit{Equisetum ramosissimum Desf.} Herbe sauvage, utilisée en pharmacopée traditionnelle; sa tige est divisée en petits segments, et elle se brise à l'une de ces articulations si on l'arrache.} \zh{节节草。} Dialecte chinois local~:\zh{节节高。} \zh{量词}: \textcolor{darkblue}{\textbf{\ipa{po˧}}}  \mytextsc{clf}~: \textcolor{darkblue}{\textbf{\ipa{po˧}}} 
\lhead{\firstmark}
\rhead{\botmark}

\subsection{\hspace{-0.5cm} {\Large \textcolor{darkblue}{\textbf{\ipa{njɤ˩ʈʂv̩˧˥}}}}\hspace{0.5cm}[\kern2pt{\textcolor{darkblue}{\textbf{\ipa{njɤ˩ʈʂv̩˧˥}}}}\kern2pt]} \hypertarget{nj7\string_Bt`s`v\string_=\string_M\string_T1}{}
\markboth{\textcolor{darkblue}{\textbf{\ipa{njɤ˩ʈʂv̩˧˥}}}}{}
\textcolor{teal}{\mytextsc{nom}} \hspace{4pt} Ton~: LM+MH\#.
\textcolor{PineGreen}{\selectlanguage{french}Loche (poisson).} \zh{泥鳅。}  \zh{量词}: \textcolor{darkblue}{\textbf{\ipa{mi˩}}}  \mytextsc{clf}~: \textcolor{darkblue}{\textbf{\ipa{mi˩}}} 
\lhead{\firstmark}
\rhead{\botmark}

\subsection{\hspace{-0.5cm} {\Large \textcolor{darkblue}{\textbf{\ipa{njɤ˧˥}}} \textsubscript{1}}\hspace{0.5cm}[\kern2pt{\textcolor{darkblue}{\textbf{\ipa{njɤ˧˥}}}}\kern2pt]} \hypertarget{nj7\string_M\string_T1}{}
\markboth{\textcolor{darkblue}{\textbf{\ipa{njɤ˧˥}}} \textsubscript{1}}{}
\textcolor{teal}{\mytextsc{verbe}} \hspace{4pt} Ton~: MH.
\textcolor{PineGreen}{\selectlanguage{french}Coller (2 objets ensemble).} \zh{贴。}  ¶ \textcolor{darkblue}{\textbf{\ipa{le˧-njɤ˧-ze˥!}}} \textcolor{PineGreen}{\selectlanguage{french}C'est recollé! / Ca y est, c'est collé!} \zh{粘在一起了!}  
 ¶ \textcolor{darkblue}{\textbf{\ipa{tso˧\textasciitilde{}tso˧ le˧-ɖʐɤ˧, | le˧-njɤ˧˥!}}} \textcolor{PineGreen}{\selectlanguage{french}(pour un livre, par ex.) Quand un truc est déchiré, on le recolle!} \zh{东西撕破了,粘在一起(就好了)}  
\textit{Voir~:} \hyperlink{}{\textcolor{darkblue}{\textbf{\ipa{njɤ˧˥}}} \textsubscript{2}} 
\lhead{\firstmark}
\rhead{\botmark}

\subsection{\hspace{-0.5cm} {\Large \textcolor{darkblue}{\textbf{\ipa{njɤ˧˥}}} \textsubscript{2}}\hspace{0.5cm}[\kern2pt{\textcolor{darkblue}{\textbf{\ipa{njɤ˧˥}}}}\kern2pt]} \hypertarget{nj7\string_M\string_T2}{}
\markboth{\textcolor{darkblue}{\textbf{\ipa{njɤ˧˥}}} \textsubscript{2}}{}
\textcolor{teal}{\mytextsc{adjectif}} \hspace{4pt} Ton~: MH.
\textcolor{PineGreen}{\selectlanguage{french}Poisseux, collant, visqueux (colle, résine...).} \zh{黏(胶,树脂)。} \textit{Voir~:} \hyperlink{}{\textcolor{darkblue}{\textbf{\ipa{njɤ˧˥}}} \textsubscript{1}} 
\lhead{\firstmark}
\rhead{\botmark}

\subsection{\hspace{-0.5cm} {\Large \textcolor{darkblue}{\textbf{\ipa{njɤ˧˥}}} \textsubscript{3}}\hspace{0.5cm}[\kern2pt{\textcolor{darkblue}{\textbf{\ipa{njɤ˧˥}}}}\kern2pt]} \hypertarget{nj7\string_M\string_T3}{}
\markboth{\textcolor{darkblue}{\textbf{\ipa{njɤ˧˥}}} \textsubscript{3}}{}
\textcolor{teal}{\mytextsc{adjectif}} \hspace{4pt} Ton~: MH.
\textcolor{PineGreen}{\selectlanguage{french}Tôt.} \zh{早。} 
\lhead{\firstmark}
\rhead{\botmark}

\subsection{\hspace{-0.5cm} {\Large \textcolor{darkblue}{\textbf{\ipa{njɤ˩˥}}}}\hspace{0.5cm}[\kern2pt{\textcolor{darkblue}{\textbf{\ipa{njɤ˩˥}}}}\kern2pt]} \hypertarget{nj7\string_B\string_T1}{}
\markboth{\textcolor{darkblue}{\textbf{\ipa{njɤ˩˥}}}}{}
\textcolor{teal}{\mytextsc{nom}} \hspace{4pt} Ton~: LH.
\textcolor{PineGreen}{\selectlanguage{french}Œil (monosyllabe).} \zh{眼睛(单音节)。}  \zh{量词}: \textcolor{darkblue}{\textbf{\ipa{ɭɯ˧}}}  \mytextsc{clf}~: \textcolor{darkblue}{\textbf{\ipa{ɭɯ˧}}} 
\lhead{\firstmark}
\rhead{\botmark}

\subsection{\hspace{-0.5cm} {\Large \textcolor{darkblue}{\textbf{\ipa{njo˥}}}}\hspace{0.5cm}[\kern2pt{\textcolor{darkblue}{\textbf{\ipa{njo˥}}}}\kern2pt]} \hypertarget{njo\string_T1}{}
\markboth{\textcolor{darkblue}{\textbf{\ipa{njo˥}}}}{}
\textcolor{teal}{\mytextsc{nom}} \hspace{4pt} Ton~: \#H.
\textcolor{PineGreen}{\selectlanguage{french}Flûte à calebasse, hulusi: instrument à vent à anche libre.} \zh{葫芦丝、葫芦箫。}  ¶ \textcolor{darkblue}{\textbf{\ipa{njo˧ mv̩˥}}} \textcolor{PineGreen}{\selectlanguage{french}jouer de la flûte à calebasse} \zh{吹响葫芦丝}  

\lhead{\firstmark}
\rhead{\botmark}

\subsection{\hspace{-0.5cm} {\Large \textcolor{darkblue}{\textbf{\ipa{njo˧}}}}\hspace{0.5cm}[\kern2pt{\textcolor{darkblue}{\textbf{\ipa{njo˥}}}}\kern2pt]} \hypertarget{njo\string_M1}{}
\markboth{\textcolor{darkblue}{\textbf{\ipa{njo˧}}}}{}
\textcolor{teal}{\mytextsc{nom}} \hspace{4pt} Ton~: M.
\textcolor{PineGreen}{\selectlanguage{french}Épi (de blé, d'orge, de riz...).} \zh{谷穗。}  ¶ \textcolor{darkblue}{\textbf{\ipa{hɑ˧-njo˩}}} \textcolor{PineGreen}{\selectlanguage{french}épi de céréales} \zh{谷穗}  
 ¶ \textcolor{darkblue}{\textbf{\ipa{mv̩˧dze˧-njo˧ (+ɲi˩)}}} \textcolor{PineGreen}{\selectlanguage{french}épi d'orge} \zh{大麦穗}  
 ¶ \textcolor{darkblue}{\textbf{\ipa{tsʰi˧zi˧-hɑ˧njo˥ (+ɲi˩)}}} \textcolor{PineGreen}{\selectlanguage{french}épi d'orge d'altitude} \zh{青稞穗}  

\lhead{\firstmark}
\rhead{\botmark}

\subsection{\hspace{-0.5cm} {\Large \textcolor{darkblue}{\textbf{\ipa{njo˧bi˧li˥}}}}\hspace{0.5cm}[\kern2pt{\textcolor{darkblue}{\textbf{\ipa{njo˧bi˧li˥}}}}\kern2pt]} \hypertarget{njo\string_Mbi\string_Mli\string_T1}{}
\markboth{\textcolor{darkblue}{\textbf{\ipa{njo˧bi˧li˥}}}}{}
\textcolor{teal}{\mytextsc{nom}} \hspace{4pt} Ton~: H\#.
\textcolor{PineGreen}{\selectlanguage{french}Lèvres.} \zh{嘴唇。}  \zh{量词}: \textcolor{darkblue}{\textbf{\ipa{ɭɯ˧}}}  \mytextsc{clf}~: \textcolor{darkblue}{\textbf{\ipa{ɭɯ˧}}} 
\lhead{\firstmark}
\rhead{\botmark}

\subsection{\hspace{-0.5cm} {\Large \textcolor{darkblue}{\textbf{\ipa{njo˩bi˥}}}}\hspace{0.5cm}[\kern2pt{\textcolor{darkblue}{\textbf{\ipa{njo˩bi˥}}}}\kern2pt]} \hypertarget{njo\string_Bbi\string_T1}{}
\markboth{\textcolor{darkblue}{\textbf{\ipa{njo˩bi˥}}}}{}
\textcolor{teal}{\mytextsc{nom}} \hspace{4pt} Ton~: LH.
\textcolor{PineGreen}{\selectlanguage{french}Sein, mamelle.} \zh{乳房。}  ¶ \textcolor{darkblue}{\textbf{\ipa{ʝi˧-njo˥bi˩}}} \textcolor{PineGreen}{\selectlanguage{french}mamelle de la vache} \zh{牛的奶头}  
 ¶ \textcolor{darkblue}{\textbf{\ipa{[F5] njo˩bi˧-ʁo˧qʰwɤ˩}}} \textcolor{PineGreen}{\selectlanguage{french}téton} \zh{乳头}  
 \zh{量词}: \textcolor{darkblue}{\textbf{\ipa{ɭɯ˧}}}  \mytextsc{clf}~: \textcolor{darkblue}{\textbf{\ipa{ɭɯ˧}}} 
\lhead{\firstmark}
\rhead{\botmark}

\subsection{\hspace{-0.5cm} {\Large \textcolor{darkblue}{\textbf{\ipa{njo˩kæ˧tɕi˩˥}}}}\hspace{0.5cm}[\kern2pt{\textcolor{darkblue}{\textbf{\ipa{xxxx ton non trouvé, à faire manuellement...}}}}\kern2pt]} \hypertarget{njo\string_Bk\{\string_Mts£i\string_B\string_T1}{}
\markboth{\textcolor{darkblue}{\textbf{\ipa{njo˩kæ˧tɕi˩˥}}}}{}
\textcolor{teal}{\mytextsc{nom}} \hspace{4pt} Ton~: LM+LH.
\textcolor{PineGreen}{\selectlanguage{french}Bolet, cèpe, \textit{Boletus edulis}; littéralement “champignon-galette de sarrasin”, du fait de sa texture.} \zh{牛肝菌(汉语借词)。}  Emprunt~: chinois  \zh{牛肝菌}
\textit{Voir~:} \hyperlink{}{\textcolor{darkblue}{\textbf{\ipa{jɤ˧qʰɑ˧-pɤ˥jɤ˩-mo˩}}}} 
\lhead{\firstmark}
\rhead{\botmark}

\subsection{\hspace{-0.5cm} {\Large \textcolor{darkblue}{\textbf{\ipa{njo˩pɤ˩lv̩˥}}}}\hspace{0.5cm}[\kern2pt{\textcolor{darkblue}{\textbf{\ipa{njo˩pɤ˩lv̩˥}}}}\kern2pt]} \hypertarget{njo\string_Bp7\string_Blv\string_=\string_T1}{}
\markboth{\textcolor{darkblue}{\textbf{\ipa{njo˩pɤ˩lv̩˥}}}}{}
\textcolor{teal}{\mytextsc{nom}} \hspace{4pt} Ton~: L+H\#.
\textcolor{PineGreen}{\selectlanguage{french}Pis de la vache.} \zh{牛的奶头。}  \zh{量词}: \textcolor{darkblue}{\textbf{\ipa{ɭɯ˧}}}  \mytextsc{clf}~: \textcolor{darkblue}{\textbf{\ipa{ɭɯ˧}}} 
\lhead{\firstmark}
\rhead{\botmark}

\subsection{\hspace{-0.5cm} {\Large \textcolor{darkblue}{\textbf{\ipa{njo˩˥}}}}\hspace{0.5cm}[\kern2pt{\textcolor{darkblue}{\textbf{\ipa{njo˩˥}}}}\kern2pt]} \hypertarget{njo\string_B\string_T1}{}
\markboth{\textcolor{darkblue}{\textbf{\ipa{njo˩˥}}}}{}
\textcolor{teal}{\mytextsc{nom}} \hspace{4pt} Ton~: LH.
\textcolor{PineGreen}{\selectlanguage{french}Lait.} \zh{奶汁。}  ¶ \textcolor{darkblue}{\textbf{\ipa{njo˩ ki˧}}} \textcolor{PineGreen}{\selectlanguage{french}donner le sein, donner la tétée, nourrir (un nourrisson); littéralement '“donner du lait”} \zh{给(喂)奶}  
 ¶ \textcolor{darkblue}{\textbf{\ipa{njo˩ ʈʰɯ˩˥}}} \textcolor{PineGreen}{\selectlanguage{french}boire du lait} \zh{喝奶}  

\lhead{\firstmark}
\rhead{\botmark}

\subsection{\hspace{-0.5cm} {\Large \textcolor{darkblue}{\textbf{\ipa{no˧no˧}}}}\hspace{0.5cm}[\kern2pt{\textcolor{darkblue}{\textbf{\ipa{no˧no˧}}}}\kern2pt]} \hypertarget{no\string_Mno\string_M1}{}
\markboth{\textcolor{darkblue}{\textbf{\ipa{no˧no˧}}}}{}
\textcolor{teal}{\mytextsc{nom}} \hspace{4pt} Ton~: M.
\textcolor{PineGreen}{\selectlanguage{french}Prénom féminin.} \zh{女性名字。} 
\lhead{\firstmark}
\rhead{\botmark}

\subsection{\hspace{-0.5cm} {\Large \textcolor{darkblue}{\textbf{\ipa{no˧=ɻ̍˩}}}}\hspace{0.5cm}[\kern2pt{\textcolor{darkblue}{\textbf{\ipa{no˧ɻ̍˩}}}}\kern2pt]} \hypertarget{no\string_M=r£`̍\string_B1}{}
\markboth{\textcolor{darkblue}{\textbf{\ipa{no˧=ɻ̍˩}}}}{}
\textcolor{teal}{\mytextsc{pronom}} \hspace{4pt} Ton~: L\#.
\textcolor{PineGreen}{\selectlanguage{french}Deuxième personne du pluriel.} \zh{你们。} \textit{Voir~:} \hyperlink{}{\textcolor{darkblue}{\textbf{\ipa{no˧-sɯ˩kv̩˩}}}} 
\lhead{\firstmark}
\rhead{\botmark}

\subsection{\hspace{-0.5cm} {\Large \textcolor{darkblue}{\textbf{\ipa{no˧=zɯ˩}}}}\hspace{0.5cm}[\kern2pt{\textcolor{darkblue}{\textbf{\ipa{no˧zɯ˩}}}}\kern2pt]} \hypertarget{no\string_M=zM\string_B1}{}
\markboth{\textcolor{darkblue}{\textbf{\ipa{no˧=zɯ˩}}}}{}
\textcolor{teal}{\mytextsc{pronom}} \hspace{4pt} Ton~: L\#.
\textcolor{PineGreen}{\selectlanguage{french}Pronom personnel de deuxième personne duel: vous deux.} \zh{你们俩。} \textit{Voir~:} \hyperlink{}{\textcolor{darkblue}{\textbf{\ipa{no˩zɯ˧˥}}}} 
\lhead{\firstmark}
\rhead{\botmark}

\subsection{\hspace{-0.5cm} {\Large \textcolor{darkblue}{\textbf{\ipa{no˩}}}}\hspace{0.5cm}[\kern2pt{\textcolor{darkblue}{\textbf{\ipa{no˩˥}}}}\kern2pt]} \hypertarget{no\string_B1}{}
\markboth{\textcolor{darkblue}{\textbf{\ipa{no˩}}}}{}
\textcolor{teal}{\mytextsc{pronom}} \hspace{4pt} Ton~: L.
\textcolor{PineGreen}{\selectlanguage{french}Pronom de deuxième personne du singulier.} \zh{你。}  ¶ \textcolor{darkblue}{\textbf{\ipa{no˩ ɲi˩˥}}} \textcolor{PineGreen}{\selectlanguage{french}c'est toi!} \zh{是你!}  

\lhead{\firstmark}
\rhead{\botmark}

\subsection{\hspace{-0.5cm} {\Large \textcolor{darkblue}{\textbf{\ipa{no˩bv̩˧}}}}\hspace{0.5cm}[\kern2pt{\textcolor{darkblue}{\textbf{\ipa{no˩bv̩˥}}}}\kern2pt]} \hypertarget{no\string_Bbv\string_=\string_M1}{}
\markboth{\textcolor{darkblue}{\textbf{\ipa{no˩bv̩˧}}}}{}
\textcolor{teal}{\mytextsc{nom}} \hspace{4pt} Ton~: LM.
\textcolor{PineGreen}{\selectlanguage{french}Prénom masculin.} \zh{男性名字。} 
\lhead{\firstmark}
\rhead{\botmark}

\subsection{\hspace{-0.5cm} {\Large \textcolor{darkblue}{\textbf{\ipa{no˩qo˥}}}}\hspace{0.5cm}[\kern2pt{\textcolor{darkblue}{\textbf{\ipa{no˩qo˥}}}}\kern2pt]} \hypertarget{no\string_Bqo\string_T1}{}
\markboth{\textcolor{darkblue}{\textbf{\ipa{no˩qo˥}}}}{}
\textcolor{teal}{\mytextsc{adverbe}} \hspace{4pt} Ton~: LH.
\textcolor{PineGreen}{\selectlanguage{french}À proximité de, à côté de.} \zh{……附近。} 
\lhead{\firstmark}
\rhead{\botmark}

\subsection{\hspace{-0.5cm} {\Large \textcolor{darkblue}{\textbf{\ipa{no˩zɯ˧˥}}}}\hspace{0.5cm}[\kern2pt{\textcolor{darkblue}{\textbf{\ipa{no˩zɯ˧˥}}}}\kern2pt]} \hypertarget{no\string_BzM\string_M\string_T1}{}
\markboth{\textcolor{darkblue}{\textbf{\ipa{no˩zɯ˧˥}}}}{}
\textcolor{teal}{\mytextsc{pronom}} \hspace{4pt} Ton~: LM+MH\#.
\textcolor{PineGreen}{\selectlanguage{french}Pronom personnel de deuxième personne duel: vous deux.} \zh{你们俩。} \textit{Voir~:} \hyperlink{}{\textcolor{darkblue}{\textbf{\ipa{no˧=zɯ˩}}}} 
\lhead{\firstmark}
\rhead{\botmark}

\subsection{\hspace{-0.5cm} {\Large \textcolor{darkblue}{\textbf{\ipa{no˩xx}}}}\hspace{0.5cm}[\kern2pt{\textcolor{darkblue}{\textbf{\ipa{xxxx ton non trouvé, à faire manuellement...}}}}\kern2pt]} \hypertarget{no\string_Bxx1}{}
\markboth{\textcolor{darkblue}{\textbf{\ipa{no˩xx}}}}{}
\textcolor{teal}{\mytextsc{verbe}} \hspace{4pt} Ton~: Lxx.
\textcolor{PineGreen}{\selectlanguage{french}Mélanger, ajouter.} \zh{搀。}  ¶ \textcolor{darkblue}{\textbf{\ipa{ɳæ˧ | tsɑ˧bɤ˧-qo˧ tʰi˧-no˩}}} \textcolor{PineGreen}{\selectlanguage{french}mettre du lait dans la farine grillée, mélanger la farine grillée avec du lait} \zh{在糌粑里搀奶、糌粑里搀奶}  
 ¶ \textcolor{darkblue}{\textbf{\ipa{le˧-no˩}}} \textcolor{PineGreen}{\selectlanguage{french}\mytextsc{accomp} \string_} \zh{\mytextsc{accomp} \string_}  
 ¶ \textcolor{darkblue}{\textbf{\ipa{hɑ˧-qo˩ | tɕæ˧ɻæ˩ tʰi˩-no˩: hɑ˧-qo˩ | tɕæ˧ɻæ˩ tʰi˩-kʰɯ˩}}} \textcolor{PineGreen}{\selectlanguage{french}paraphrase pour expliquer le sens du verbe: 'ajouter des légumes en saumure dans la nourriture, ça veut dire: mettre des légumes en saumure dans la nourriture.'} \zh{关于这个动词的说明:‘饭里面搀泡菜,就是说:在饭里面放泡菜。’}  
 ¶ \textcolor{darkblue}{\textbf{\ipa{hɑ˧-qo˩ | tɕæ˧ɻæ˩ tʰi˩-no˩: tɕæ˧ɻæ˩-lɑ˩ | hɑ˧ | ɖɯ˧-tɕʰo˩ dzɯ˩}}} \textcolor{PineGreen}{\selectlanguage{french}paraphrase pour expliquer le sens du verbe: 'ajouter des légumes en saumure dans la nourriture, ça veut dire: manger des légumes en saumure et de la nourriture (du riz) ensemble.'} \zh{关于这个动词的说明:‘饭里面搀泡菜,就是说:把泡菜和饭一起吃。’}  

\lhead{\firstmark}
\rhead{\botmark}

\subsection{\hspace{-0.5cm} {\Large \textcolor{darkblue}{\textbf{\ipa{‑no˧˥}}}}\hspace{0.5cm}[\kern2pt{\textcolor{darkblue}{\textbf{\ipa{no˧˥}}}}\kern2pt]} \hypertarget{‑no\string_M\string_T1}{}
\markboth{\textcolor{darkblue}{\textbf{\ipa{‑no˧˥}}}}{}
\textcolor{teal}{\mytextsc{particule}} \textcolor{teal}{\mytextsc{de}} \textcolor{teal}{\mytextsc{discours}} \hspace{4pt} Ton~: MH.
\textcolor{PineGreen}{\selectlanguage{french}Topique contrastif. Gloses possibles: …, en revanche, ...; …, pour sa part, ...; quant à ….} \zh{\mytextsc{主题:对于、关于。}}  ¶ \textcolor{darkblue}{\textbf{\ipa{qæ˧do˧ | -no˧˥}}} \textcolor{PineGreen}{\selectlanguage{french}en ce qui concerne le bois de construction, …} \zh{关于木材,……}  
 ¶ \textcolor{darkblue}{\textbf{\ipa{ʑi˧qʰwɤ˧ | -no˧˥}}} \textcolor{PineGreen}{\selectlanguage{french}en ce qui concerne la pièce principale, …} \zh{关于主屋……}  

\lhead{\firstmark}
\rhead{\botmark}

\subsection{\hspace{-0.5cm} {\Large \textcolor{darkblue}{\textbf{\ipa{nv̩˥}}} \textsubscript{1}}\hspace{0.5cm}[\kern2pt{\textcolor{darkblue}{\textbf{\ipa{nv̩˥}}}}\kern2pt]} \hypertarget{nv\string_=\string_T1}{}
\markboth{\textcolor{darkblue}{\textbf{\ipa{nv̩˥}}} \textsubscript{1}}{}
\textcolor{teal}{\mytextsc{verbe}} \hspace{4pt} Ton~: H.
\textcolor{PineGreen}{\selectlanguage{french}Suivre à la trace, pister.} \zh{追赶。}  ¶ \textcolor{darkblue}{\textbf{\ipa{le˧-nv̩˥}}} \textcolor{PineGreen}{\selectlanguage{french}\mytextsc{accomp}} \zh{\mytextsc{accomp}}  
 ¶ \textcolor{darkblue}{\textbf{\ipa{ʈʂɤ˩nv̩˩}}} \textcolor{PineGreen}{\selectlanguage{french}suivre à la trace, pister} \zh{追赶}  
 ¶ \textcolor{darkblue}{\textbf{\ipa{le˧-ʈʂɤ˩nv̩˩}}} \textcolor{PineGreen}{\selectlanguage{french}suivre à la trace, pister} \zh{追赶}  
 ¶ \textcolor{darkblue}{\textbf{\ipa{le˧-ʈʂɤ˩nv̩˩ | le˧-hɯ˩}}} \textcolor{PineGreen}{\selectlanguage{french}Il est parti à la poursuite de / à la chasse de} \zh{追赶去了}  

\lhead{\firstmark}
\rhead{\botmark}

\subsection{\hspace{-0.5cm} {\Large \textcolor{darkblue}{\textbf{\ipa{nv̩˥}}} \textsubscript{2}}\hspace{0.5cm}[\kern2pt{\textcolor{darkblue}{\textbf{\ipa{nv̩˥}}}}\kern2pt]} \hypertarget{nv\string_=\string_T2}{}
\markboth{\textcolor{darkblue}{\textbf{\ipa{nv̩˥}}} \textsubscript{2}}{}
\textcolor{teal}{\mytextsc{verbe}} \hspace{4pt} Ton~: H.
\textcolor{PineGreen}{\selectlanguage{french}Enterrer.} \zh{埋。} 
\lhead{\firstmark}
\rhead{\botmark}

\subsection{\hspace{-0.5cm} {\Large \textcolor{darkblue}{\textbf{\ipa{nv̩˩dʑɯ˥}}}}\hspace{0.5cm}[\kern2pt{\textcolor{darkblue}{\textbf{\ipa{nv̩˩dʑɯ˩˥}}}}\kern2pt]} \hypertarget{nv\string_=\string_Bdz£M\string_T1}{}
\markboth{\textcolor{darkblue}{\textbf{\ipa{nv̩˩dʑɯ˥}}}}{}
\textcolor{teal}{\mytextsc{nom}} \hspace{4pt} Ton~: LH.
\textcolor{PineGreen}{\selectlanguage{french}Tofu.} \zh{豆腐。}  \zh{量词}: \textcolor{darkblue}{\textbf{\ipa{v̩˥}}}  \mytextsc{clf}~: \textcolor{darkblue}{\textbf{\ipa{v̩˥}}} \textcolor{PineGreen}{\selectlanguage{french}casseroles}
\lhead{\firstmark}
\rhead{\botmark}

\subsection{\hspace{-0.5cm} {\Large \textcolor{darkblue}{\textbf{\ipa{nv̩˩ho\#˥}}}}\hspace{0.5cm}[\kern2pt{\textcolor{darkblue}{\textbf{\ipa{nv̩˩ho˥}}}}\kern2pt]} \hypertarget{nv\string_=\string_Bho\#\string_T1}{}
\markboth{\textcolor{darkblue}{\textbf{\ipa{nv̩˩ho\#˥}}}}{}
\textcolor{teal}{\mytextsc{nom}} \hspace{4pt} Ton~: LM+\#H.
\textcolor{PineGreen}{\selectlanguage{french}Tofu léger, longuement bouilli.} \zh{豆花。}  \zh{量词}: \textcolor{darkblue}{\textbf{\ipa{v̩˥}}}  \mytextsc{clf}~: \textcolor{darkblue}{\textbf{\ipa{v̩˥}}} 
\lhead{\firstmark}
\rhead{\botmark}

\subsection{\hspace{-0.5cm} {\Large \textcolor{darkblue}{\textbf{\ipa{nv̩˧hṽ˩}}}}\hspace{0.5cm}[\kern2pt{\textcolor{darkblue}{\textbf{\ipa{nv̩˩hṽ˥}}}}\kern2pt]} \hypertarget{nv\string_=\string_Mhv\string_~\string_B1}{}
\markboth{\textcolor{darkblue}{\textbf{\ipa{nv̩˧hṽ˩}}}}{}
\textcolor{teal}{\mytextsc{nom}} \hspace{4pt} Ton~: L\#.
\textcolor{PineGreen}{\selectlanguage{french}Haricot: terme générique.} \zh{豆子,四季豆,花腰豆。}  \zh{量词}: \textcolor{darkblue}{\textbf{\ipa{v̩˥}}}  \mytextsc{clf}~: \textcolor{darkblue}{\textbf{\ipa{v̩˥}}} 
\lhead{\firstmark}
\rhead{\botmark}

\subsection{\hspace{-0.5cm} {\Large \textcolor{darkblue}{\textbf{\ipa{nv̩˧hṽ˩-bi˩bi˩}}}}\hspace{0.5cm}[\kern2pt{\textcolor{darkblue}{\textbf{\ipa{xxxx non-correspondance entre le nombre de morphèmes et le nombre de tons de morphèmes}}}}\kern2pt]} \hypertarget{nv\string_=\string_Mhv\string_~\string_B-bi\string_Bbi\string_B1}{}
\markboth{\textcolor{darkblue}{\textbf{\ipa{nv̩˧hṽ˩-bi˩bi˩}}}}{}
\textcolor{teal}{\mytextsc{nom}} \hspace{4pt} Ton~: L\#-.
\textcolor{PineGreen}{\selectlanguage{french}Haricot vert; on consomme la cosse fraîche et la graine qu'il contient.} \zh{四季豆、玉豆、帶莢豌豆、菜豆、刀豆、豆角、敏豆仔、敏豆。}  \zh{量词}: \textcolor{darkblue}{\textbf{\ipa{kʰɤ˧˥}}}  \mytextsc{clf}~: \textcolor{darkblue}{\textbf{\ipa{kʰɤ˧˥}}} \textcolor{PineGreen}{\selectlanguage{french}panier}
\lhead{\firstmark}
\rhead{\botmark}

\subsection{\hspace{-0.5cm} {\Large \textcolor{darkblue}{\textbf{\ipa{nv̩˩ɭɯ˧}}}}\hspace{0.5cm}[\kern2pt{\textcolor{darkblue}{\textbf{\ipa{xxxx non-correspondance entre le nombre de morphèmes et le nombre de tons de morphèmes}}}}\kern2pt]} \hypertarget{nv\string_=\string_Bl\string_RM\string_M1}{}
\markboth{\textcolor{darkblue}{\textbf{\ipa{nv̩˩ɭɯ˧}}}}{}
\textcolor{teal}{\mytextsc{nom}} \hspace{4pt} Ton~: LM.
\textcolor{PineGreen}{\selectlanguage{french}Soja.} \zh{黄豆。}  \zh{量词}: \textcolor{darkblue}{\textbf{\ipa{wɤ˩}}}  \mytextsc{clf}~: \textcolor{darkblue}{\textbf{\ipa{wɤ˩}}} \textcolor{PineGreen}{\selectlanguage{french}une charge}
\lhead{\firstmark}
\rhead{\botmark}

\subsection{\hspace{-0.5cm} {\Large \textcolor{darkblue}{\textbf{\ipa{nv̩˩mi˩}}}}\hspace{0.5cm}[\kern2pt{\textcolor{darkblue}{\textbf{\ipa{nv̩˩mi˥}}}}\kern2pt]} \hypertarget{nv\string_=\string_Bmi\string_B1}{}
\markboth{\textcolor{darkblue}{\textbf{\ipa{nv̩˩mi˩}}}}{}
\textcolor{teal}{\mytextsc{nom}} \hspace{4pt} Ton~: L.
\ding{202} \textcolor{PineGreen}{\selectlanguage{french}Cœur.} \zh{心脏。}  ¶ \textcolor{darkblue}{\textbf{\ipa{hĩ˧ ʈʂʰɯ˧-v̩˧, | nv̩˩mi˩ tɕi˥! |}}} \textcolor{PineGreen}{\selectlanguage{french}Celui-là, il manque de courage! (littéralement “(a) un petit coeur”, “son coeur est petit”)} \zh{这个人,胆小!(直译:“心小”)}  
 ¶ \textcolor{darkblue}{\textbf{\ipa{nv̩˩mi˩˥ | ɖɯ˧-ɭɯ˧ tsɤ˧ |}}} \textcolor{PineGreen}{\selectlanguage{french}être en sympathie, à l'unisson} \zh{情投意合}  
 ¶ \textcolor{darkblue}{\textbf{\ipa{nv̩˩mi˩˥ | tʰi˧-tɕɯ˥ | so˩˥}}} \textcolor{PineGreen}{\selectlanguage{french}enseigner patiemment/apprendre patiemment} \zh{耐心地学习 / 耐心地教}  
 ¶ \textcolor{darkblue}{\textbf{\ipa{nv̩˩mi˩-qo˥ | tʰi˧-ʑi˥}}} \textcolor{PineGreen}{\selectlanguage{french}se souvenir de, garder à l'esprit, avoir à l'esprit} \zh{记住、记得(直译:‘心里存着’、‘心里有’)}  
 ¶ \textcolor{darkblue}{\textbf{\ipa{nv̩˩mi˩-qo˥ | tʰi˧-kʰɯ˧˥}}} \textcolor{PineGreen}{\selectlanguage{french}faire l'effort de se souvenir, garder à l'esprit, garder en mémoire} \zh{记住(直译:‘放在心里’)}  
 \zh{量词}: \textcolor{darkblue}{\textbf{\ipa{ɭɯ˧}}} \ding{203} \textcolor{PineGreen}{\selectlanguage{french}État d'esprit.} \zh{心情。}  ¶ \textcolor{darkblue}{\textbf{\ipa{nv̩˩mi˩ dzɯ˩\textasciitilde{}dzɯ˩-ɻ̍˥}}} \textcolor{PineGreen}{\selectlanguage{french}ne pas bien s'entendre; se chamailler sans répit; s'empoisonner mutuellement la vie, se faire la vie impossible, se bouffer le nez} \zh{经常吵架、过不到一起去}  
 \mytextsc{clf}~: \textcolor{darkblue}{\textbf{\ipa{ɭɯ˧}}} 
\lhead{\firstmark}
\rhead{\botmark}

\subsection{\hspace{-0.5cm} {\Large \textcolor{darkblue}{\textbf{\ipa{nv̩˩mi˩-ɖɯ˩}}}}\hspace{0.5cm}[\kern2pt{\textcolor{darkblue}{\textbf{\ipa{xxxx non-correspondance entre le nombre de morphèmes et le nombre de tons de morphèmes}}}}\kern2pt]} \hypertarget{nv\string_=\string_Bmi\string_B-d`M\string_B1}{}
\markboth{\textcolor{darkblue}{\textbf{\ipa{nv̩˩mi˩-ɖɯ˩}}}}{}
\textcolor{teal}{\mytextsc{adjectif}} \hspace{4pt} Ton~: L.
\textit{De:} \textbf{nv̩˩mi˩ et ɖɯ˩a} \textcolor{PineGreen}{\selectlanguage{french}Courageux, audacieux.} \zh{勇敢、有勇气的。}  ¶ \textcolor{darkblue}{\textbf{\ipa{ʈʂʰɯ˧ | nv̩˩mi˩˥ | ɖwæ˧˥ | ɖɯ˩˥! | hĩ˧ | mɤ˧-ɖwæ˥!}}} \textcolor{PineGreen}{\selectlanguage{french}il est très courageux! il n'a peur de personne!} \zh{他很勇敢!谁也不怕!}  
 ¶ \textcolor{darkblue}{\textbf{\ipa{pɤ˩mi˩˥ | nv̩˩mi˩ ɖɯ˩˥, | ʝi˧-ɳɯ˧ tʰv̩˧˥; | mi˩zɯ˩ nv̩˥mi˩ ɖɯ˩ (-dʑo˩), | hĩ˧-ɳɯ˧ lɑ˧˥!}}} \textcolor{PineGreen}{\selectlanguage{french}“La grenouille courageuse, elle se fait écraser par le bœuf; la femme courageuse, elle se fait frapper!” (Explication: les créatures qui ne sont pas les plus fortes doivent se garder d'être trop courageuses: la grenouille qui n'a peur de rien et s'aventure sur le grand chemin, elle se fait écraser; la femme qui se comporte avec une mâle assurance, elle finit par entrer dans des conflits où on en vient aux mains et où elle a finalement le dessous.)} \zh{“勇敢的青蛙,被牛压死。勇敢的女人,被人家打!”(说明:青蛙太勇敢,上马路,就容易被压死,而女人太勇敢,容易跟别人发生矛盾,最后就打不过男人。)}  

\lhead{\firstmark}
\rhead{\botmark}

\subsection{\hspace{-0.5cm} {\Large \textcolor{darkblue}{\textbf{\ipa{nv̩˩mi˩-ki˧ki˩}}}}\hspace{0.5cm}[\kern2pt{\textcolor{darkblue}{\textbf{\ipa{xxxx non-correspondance entre le nombre de morphèmes et le nombre de tons de morphèmes}}}}\kern2pt]} \hypertarget{nv\string_=\string_Bmi\string_B-ki\string_Mki\string_B1}{}
\markboth{\textcolor{darkblue}{\textbf{\ipa{nv̩˩mi˩-ki˧ki˩}}}}{}
\textcolor{teal}{\mytextsc{adjectif}} \hspace{4pt} Ton~: .
\textit{De:} \textbf{nv̩˩mi˩ et ki˧a} \textcolor{PineGreen}{\selectlanguage{french}En harmonie, à l'unisson.} \zh{心意相通。} 
\lhead{\firstmark}
\rhead{\botmark}

\subsection{\hspace{-0.5cm} {\Large \textcolor{darkblue}{\textbf{\ipa{nv̩˩mi˩-ʈʰi˩}}}}\hspace{0.5cm}[\kern2pt{\textcolor{darkblue}{\textbf{\ipa{xxxx non-correspondance entre le nombre de morphèmes et le nombre de tons de morphèmes}}}}\kern2pt]} \hypertarget{nv\string_=\string_Bmi\string_B-t`\string_hi\string_B1}{}
\markboth{\textcolor{darkblue}{\textbf{\ipa{nv̩˩mi˩-ʈʰi˩}}}}{}
\textcolor{teal}{\mytextsc{adjectif}} \hspace{4pt} Ton~: L.
\textit{De:} \textbf{nv̩˩mi˩ et ʈʰi˩a} \textcolor{PineGreen}{\selectlanguage{french}Découragé, nostalgique, mélancolique.} \zh{累得没精神了。} 
\lhead{\firstmark}
\rhead{\botmark}

\subsection{\hspace{-0.5cm} {\Large \textcolor{darkblue}{\textbf{\ipa{nv̩˩pi˧}}}}\hspace{0.5cm}[\kern2pt{\textcolor{darkblue}{\textbf{\ipa{nv̩˩pi˥}}}}\kern2pt]} \hypertarget{nv\string_=\string_Bpi\string_M1}{}
\markboth{\textcolor{darkblue}{\textbf{\ipa{nv̩˩pi˧}}}}{}
\textcolor{teal}{\mytextsc{nom}} \hspace{4pt} Ton~: LM.
\textcolor{PineGreen}{\selectlanguage{french}Tourteau de soja: le reste du soja, après qu'on en a tiré le lait de soja; sert de nourriture pour les porcs.} \zh{豆粕。} 
\lhead{\firstmark}
\rhead{\botmark}

\subsection{\hspace{-0.5cm} {\Large \textcolor{darkblue}{\textbf{\ipa{nv̩˧pɤ˩}}}}\hspace{0.5cm}[\kern2pt{\textcolor{darkblue}{\textbf{\ipa{nv̩˧pɤ˩}}}}\kern2pt]} \hypertarget{nv\string_=\string_Mp7\string_B1}{}
\markboth{\textcolor{darkblue}{\textbf{\ipa{nv̩˧pɤ˩}}}}{}
\textcolor{teal}{\mytextsc{nom}} \hspace{4pt} Ton~: L\#.
\textcolor{PineGreen}{\selectlanguage{french}Fèves.} \zh{蚕豆。} 
\lhead{\firstmark}
\rhead{\botmark}

\subsection{\hspace{-0.5cm} {\Large \textcolor{darkblue}{\textbf{\ipa{nv̩˩tɕʰi\#˥}}}}\hspace{0.5cm}[\kern2pt{\textcolor{darkblue}{\textbf{\ipa{nv̩˩tɕʰi˥}}}}\kern2pt]} \hypertarget{nv\string_=\string_Bts£\string_hi\#\string_T1}{}
\markboth{\textcolor{darkblue}{\textbf{\ipa{nv̩˩tɕʰi\#˥}}}}{}
\textcolor{teal}{\mytextsc{nom}} \hspace{4pt} Ton~: LM+\#H.
\textcolor{PineGreen}{\selectlanguage{french}Balle de légumineuse (fine, pour nourrir les bovidés).} \zh{豆类的细糠秕,来喂牛。}  \zh{量词}: \textcolor{darkblue}{\textbf{\ipa{kʰɤ˧˥}}}  \mytextsc{clf}~: \textcolor{darkblue}{\textbf{\ipa{kʰɤ˧˥}}} \textcolor{PineGreen}{\selectlanguage{french}panier}
\lhead{\firstmark}
\rhead{\botmark}

\subsection{\hspace{-0.5cm} {\Large \textcolor{darkblue}{\textbf{\ipa{nv̩˩tsɑ˧˥}}}}\hspace{0.5cm}[\kern2pt{\textcolor{darkblue}{\textbf{\ipa{nv̩˩tsɑ˧˥}}}}\kern2pt]} \hypertarget{nv\string_=\string_BtsA\string_M\string_T1}{}
\markboth{\textcolor{darkblue}{\textbf{\ipa{nv̩˩tsɑ˧˥}}}}{}
\textcolor{teal}{\mytextsc{nom}} \hspace{4pt} Ton~: LM+MH\#.
\textcolor{PineGreen}{\selectlanguage{french}Balle grossière de légumineuses.} \zh{粗的豆糠。}  \zh{量词}: \textcolor{darkblue}{\textbf{\ipa{mɤ˩, etc}}}  \mytextsc{clf}~: \textcolor{darkblue}{\textbf{\ipa{mɤ˩, etc}}} 
\lhead{\firstmark}
\rhead{\botmark}

\subsection{\hspace{-0.5cm} {\Large \textcolor{darkblue}{\textbf{\ipa{nv̩˧tv̩˥}}}}\hspace{0.5cm}[\kern2pt{\textcolor{darkblue}{\textbf{\ipa{nv̩˧tv̩˥}}}}\kern2pt]} \hypertarget{nv\string_=\string_Mtv\string_=\string_T1}{}
\markboth{\textcolor{darkblue}{\textbf{\ipa{nv̩˧tv̩˥}}}}{}
\textcolor{teal}{\mytextsc{nom}} \hspace{4pt} Ton~: H\#.
\textcolor{PineGreen}{\selectlanguage{french}Musette à grain, sac à grain: sac dans lequel on donnait à manger au cheval; le sac est pendu au cou du cheval.} \zh{(挂在马脖子下面的)饲料袋子、马粮袋子。}  \zh{量词}: \textcolor{darkblue}{\textbf{\ipa{ɭɯ˧}}}  \mytextsc{clf}~: \textcolor{darkblue}{\textbf{\ipa{ɭɯ˧}}} 
\lhead{\firstmark}
\rhead{\botmark}

\subsection{\hspace{-0.5cm} {\Large \textcolor{darkblue}{\textbf{\ipa{nv̩˩ze˧}}}}\hspace{0.5cm}[\kern2pt{\textcolor{darkblue}{\textbf{\ipa{nv̩˩ze˥}}}}\kern2pt]} \hypertarget{nv\string_=\string_Bze\string_M1}{}
\markboth{\textcolor{darkblue}{\textbf{\ipa{nv̩˩ze˧}}}}{}
\textcolor{teal}{\mytextsc{nom}} \hspace{4pt} Ton~: LM.
\textcolor{PineGreen}{\selectlanguage{french}Pois chiche, \textit{Cicer arietinum}, de couleur noire, dont on prépare la spécialité de Dali: \zh{黑色凉粉}.} \zh{鹰嘴豆、桃尔豆、鸡豆、鸡心豆。} Dialecte chinois local~:\zh{鸡豆。}
\lhead{\firstmark}
\rhead{\botmark}

\newpage
\section*{\centering- \textcolor{darkblue}{\textbf{\ipa{ɳ}}} -}
\subsection{\hspace{-0.5cm} {\Large \textcolor{darkblue}{\textbf{\ipa{ɳæ˥}}}}\hspace{0.5cm}[\kern2pt{\textcolor{darkblue}{\textbf{\ipa{ɳæ˧˥}}}}\kern2pt]} \hypertarget{n`\{\string_T1}{}
\markboth{\textcolor{darkblue}{\textbf{\ipa{ɳæ˥}}}}{}
\textcolor{teal}{\mytextsc{verbe}} \hspace{4pt} Ton~: H.
\textcolor{PineGreen}{\selectlanguage{french}Se cacher.} \zh{躲藏。}  ¶ \textcolor{darkblue}{\textbf{\ipa{tʰi˧-ɳæ˥}}} \textcolor{PineGreen}{\selectlanguage{french}\mytextsc{dur}} \zh{\mytextsc{dur} \string_}  

\lhead{\firstmark}
\rhead{\botmark}

\subsection{\hspace{-0.5cm} {\Large \textcolor{darkblue}{\textbf{\ipa{ɳæ˧=ɻ̍˩}}}}\hspace{0.5cm}[\kern2pt{\textcolor{darkblue}{\textbf{\ipa{ɳæ˧ɻ̍˩}}}}\kern2pt]} \hypertarget{n`\{\string_M=r£`̍\string_B1}{}
\markboth{\textcolor{darkblue}{\textbf{\ipa{ɳæ˧=ɻ̍˩}}}}{}
\textcolor{teal}{\mytextsc{pronom}} \hspace{4pt} Ton~: L\#.
\textcolor{PineGreen}{\selectlanguage{french}Deuxième personne du pluriel: vous autres. Variante de \textcolor{darkblue}{\textbf{\ipa{/no˧=ɻ̍˩/}}}; la forme \textcolor{darkblue}{\textbf{\ipa{/no˧=ɻ̍˩/}}} est jugée plus correcte.} \zh{你们。这是\textcolor{darkblue}{\textbf{\ipa{/no˧=ɻ̍˩/}}}的一个变体。\textcolor{darkblue}{\textbf{\ipa{/no˧=ɻ̍˩/}}}被认为是更标准的。} \textit{Voir~:} \textcolor{darkblue}{\textbf{\ipa{ɳæ˩=ɻæ˧, no˧=ɻ̍˩}}} 
\lhead{\firstmark}
\rhead{\botmark}

\subsection{\hspace{-0.5cm} {\Large \textcolor{darkblue}{\textbf{\ipa{ɳæ˩=ɻæ˧}}}}\hspace{0.5cm}[\kern2pt{\textcolor{darkblue}{\textbf{\ipa{ɳæ˩ɻæ˥}}}}\kern2pt]} \hypertarget{n`\{\string_B=r£`\{\string_M1}{}
\markboth{\textcolor{darkblue}{\textbf{\ipa{ɳæ˩=ɻæ˧}}}}{}
\textcolor{teal}{\mytextsc{pronom}} \hspace{4pt} Ton~: LM.
\textcolor{PineGreen}{\selectlanguage{french}Deuxième personne, pluriel associatif: vous autres.} \zh{你们家、你们家族。} \textit{Voir~:} \textcolor{darkblue}{\textbf{\ipa{ɳæ˧=ɻ̍˩, no˧=ɻ̍˩}}} 
\lhead{\firstmark}
\rhead{\botmark}

\subsection{\hspace{-0.5cm} {\Large \textcolor{darkblue}{\textbf{\ipa{ɳæ˧˥}}}}\hspace{0.5cm}[\kern2pt{\textcolor{darkblue}{\textbf{\ipa{ɳæ˧˥}}}}\kern2pt]} \hypertarget{n`\{\string_M\string_T1}{}
\markboth{\textcolor{darkblue}{\textbf{\ipa{ɳæ˧˥}}}}{}
\textcolor{teal}{\mytextsc{verbe}} \hspace{4pt} Ton~: MH.
\ding{202} \textcolor{PineGreen}{\selectlanguage{french}Aplatir; appuyer, peser sur; presser.} \zh{按(用手)、压扁、挤压。}  ¶ \textcolor{darkblue}{\textbf{\ipa{mv̩˩tɕo˧ ɳæ˧˥}}} \textcolor{PineGreen}{\selectlanguage{french}appuyer vers le bas, peser sur} \zh{往下按}  
 ¶ \textcolor{darkblue}{\textbf{\ipa{le˧-ɳæ˩\textasciitilde{}ɳæ˩}}} \textcolor{PineGreen}{\selectlanguage{french}\mytextsc{accomp} \mytextsc{red}}  
\ding{203} \textcolor{PineGreen}{\selectlanguage{french}Opprimer, accabler, écraser de son autorité, en imposer par la violence.} \zh{压迫。}  ¶ \textcolor{darkblue}{\textbf{\ipa{hĩ˧ kʰv̩˧, | hĩ˧ ɳæ˩}}} \textcolor{PineGreen}{\selectlanguage{french}voler et oppresser (description du comportement d'un despote)} \zh{偷和迫(描述专制统治者的行为)}  

\lhead{\firstmark}
\rhead{\botmark}

\subsection{\hspace{-0.5cm} {\Large \textcolor{darkblue}{\textbf{\ipa{ɳɯ˥}}}}\hspace{0.5cm}[\kern2pt{\textcolor{darkblue}{\textbf{\ipa{ɳɯ˥}}}}\kern2pt]} \hypertarget{n`M\string_T1}{}
\markboth{\textcolor{darkblue}{\textbf{\ipa{ɳɯ˥}}}}{}
\textcolor{teal}{\mytextsc{adjectif}} \hspace{4pt} Ton~: H.
\textcolor{PineGreen}{\selectlanguage{french}Peu, peu nombreux (dénombrable).} \zh{少。}  ¶ \textcolor{darkblue}{\textbf{\ipa{hĩ˧ ɳɯ˧}}} \textcolor{PineGreen}{\selectlanguage{french}les gens sont peu nombreux} \zh{好的,不多!不好的,就很多了!}  
 ¶ \textcolor{darkblue}{\textbf{\ipa{tso˧\textasciitilde{}tso˧ | ɳɯ˧-ze˩}}} \textcolor{PineGreen}{\selectlanguage{french}il y a moins de choses, la quantité a diminué} \zh{东西(变)少了}  
 ¶ \textcolor{darkblue}{\textbf{\ipa{dʑɤ˩-hĩ˩˥, | le˧-ɳɯ˥! | mɤ˧-dʑɤ˩-hĩ˩, | le˧-dʑɯ˧˥!}}} \textcolor{PineGreen}{\selectlanguage{french}Les bons, il n'y en a guère; les médiocres, il y en a en quantité! (contexte: au sujet des établissements universitaires entre lesquels les lauréats du concours national d'entrée à l'université ont à choisir)} \zh{好的少,不好的多!(关于大学:高考后,学生要报志愿)}  

\lhead{\firstmark}
\rhead{\botmark}

\subsection{\hspace{-0.5cm} {\Large \textcolor{darkblue}{\textbf{\ipa{‑ɳɯ˧}}} \textsubscript{1}}\hspace{0.5cm}[\kern2pt{\textcolor{darkblue}{\textbf{\ipa{ɳɯ˥}}}}\kern2pt]} \hypertarget{‑n`M\string_M1}{}
\markboth{\textcolor{darkblue}{\textbf{\ipa{‑ɳɯ˧}}} \textsubscript{1}}{}
\textcolor{teal}{\mytextsc{postposition}} \hspace{4pt} Ton~: M.
\textcolor{PineGreen}{\selectlanguage{french}Ablatif, agent, et marqueur de topique.} \zh{离格,施动者,主题。接近汉语的‘由’。} 
\lhead{\firstmark}
\rhead{\botmark}

\subsection{\hspace{-0.5cm} {\Large \textcolor{darkblue}{\textbf{\ipa{ɳɯ˧˥}}}}\hspace{0.5cm}[\kern2pt{\textcolor{darkblue}{\textbf{\ipa{ɳɯ˧˥}}}}\kern2pt]} \hypertarget{n`M\string_M\string_T1}{}
\markboth{\textcolor{darkblue}{\textbf{\ipa{ɳɯ˧˥}}}}{}
\textcolor{teal}{\mytextsc{verbe}} \hspace{4pt} Ton~: MH.
\textcolor{PineGreen}{\selectlanguage{french}Serrer.} \zh{拧。}  ¶ \textcolor{darkblue}{\textbf{\ipa{le˧-ɳɯ˧-ze˥}}} \textcolor{PineGreen}{\selectlanguage{french}\mytextsc{accomp} \string_ \mytextsc{pfv}} \zh{拧了}  
 ¶ \textcolor{darkblue}{\textbf{\ipa{ʁo˧qɑ˥ | ʈʰɯ˧-ɭɯ˧ | le˧-ɳɯ˧-qɑ˥-jo˩!}}} \textcolor{PineGreen}{\selectlanguage{french}Serre donc ce couvercle! (celui d'un bocal en verre, utilisé comme verre)} \zh{(你)拧一下盖子吧!}  

\lhead{\firstmark}
\rhead{\botmark}

\subsection{\hspace{-0.5cm} {\Large \textcolor{darkblue}{\textbf{\ipa{ɳv̩˩˧}}}}\hspace{0.5cm}[\kern2pt{\textcolor{darkblue}{\textbf{\ipa{ɳv̩˩˥}}}}\kern2pt]} \hypertarget{n`v\string_=\string_B\string_M1}{}
\markboth{\textcolor{darkblue}{\textbf{\ipa{ɳv̩˩˧}}}}{}
\textcolor{teal}{\mytextsc{nom}} \hspace{4pt} Ton~: LM.
\textcolor{PineGreen}{\selectlanguage{french}Mite (insecte qui mange les vêments).} \zh{蛀虫。}  \zh{量词}: \textcolor{darkblue}{\textbf{\ipa{mi˩}}}  \mytextsc{clf}~: \textcolor{darkblue}{\textbf{\ipa{mi˩}}} 
\lhead{\firstmark}
\rhead{\botmark}

\subsection{\hspace{-0.5cm} {\Large \textcolor{darkblue}{\textbf{\ipa{ɳv̩˥}}}}\hspace{0.5cm}[\kern2pt{\textcolor{darkblue}{\textbf{\ipa{ɳv̩˥}}}}\kern2pt]} \hypertarget{n`v\string_=\string_T1}{}
\markboth{\textcolor{darkblue}{\textbf{\ipa{ɳv̩˥}}}}{}
\textcolor{teal}{\mytextsc{verbe}} \hspace{4pt} Ton~: H.
\ding{202} \textcolor{PineGreen}{\selectlanguage{french}Sentir, renifler.} \zh{闻嗅。} \ding{203} \textcolor{PineGreen}{\selectlanguage{french}Apprendre une nouvelle; être au courant de.} \zh{听到(消息)、风闻。}  ¶ \textcolor{darkblue}{\textbf{\ipa{mɤ˧-ɳv̩˥}}} \textcolor{PineGreen}{\selectlanguage{french}\mytextsc{neg}: je ne suis pas au courant!} \zh{(我)不知道这个消息!}  
 ¶ \textcolor{darkblue}{\textbf{\ipa{no˧ ə˧tso˧ ɳv̩˥?}}} \textcolor{PineGreen}{\selectlanguage{french}Quelle nouvelle as-tu apprise?} \zh{你听到了什么消息呢?}  

\lhead{\firstmark}
\rhead{\botmark}

\newpage
\section*{\centering- \textcolor{darkblue}{\textbf{\ipa{ɲ}}} -}
\subsection{\hspace{-0.5cm} {\Large \textcolor{darkblue}{\textbf{\ipa{ɲi˥}}} \textsubscript{1}}\hspace{0.5cm}[\kern2pt{\textcolor{darkblue}{\textbf{\ipa{ɲi˥}}}}\kern2pt]} \hypertarget{Ji\string_T1}{}
\markboth{\textcolor{darkblue}{\textbf{\ipa{ɲi˥}}} \textsubscript{1}}{}
\textcolor{teal}{\mytextsc{verbe}} \hspace{4pt} Ton~: H.
\textcolor{PineGreen}{\selectlanguage{french}Écouter.} \zh{听。}  ¶ \textcolor{darkblue}{\textbf{\ipa{tʰi˧-ɲi˥}}} \textcolor{PineGreen}{\selectlanguage{french}\mytextsc{dur}} \zh{\mytextsc{dur}}  
 ¶ \textcolor{darkblue}{\textbf{\ipa{tso˧\textasciitilde{}tso˧ ɲi˧}}} \textcolor{PineGreen}{\selectlanguage{french}écouter des choses} \zh{听东西}  
 ¶ \textcolor{darkblue}{\textbf{\ipa{le˧-ɲi˥-ze˩}}} \textcolor{PineGreen}{\selectlanguage{french}\mytextsc{accomp} \string_ \mytextsc{pfv}} \zh{听了}  

\lhead{\firstmark}
\rhead{\botmark}

\subsection{\hspace{-0.5cm} {\Large \textcolor{darkblue}{\textbf{\ipa{ɲi˥}}} \textsubscript{2}}\hspace{0.5cm}[\kern2pt{\textcolor{darkblue}{\textbf{\ipa{ɲi˥}}}}\kern2pt]} \hypertarget{Ji\string_T2}{}
\markboth{\textcolor{darkblue}{\textbf{\ipa{ɲi˥}}} \textsubscript{2}}{}
\textcolor{teal}{\mytextsc{verbe}} \hspace{4pt} Ton~: H.
\textcolor{PineGreen}{\selectlanguage{french}Emprunter (un objet).} \zh{向别人借。}  ¶ \textcolor{darkblue}{\textbf{\ipa{hĩ˧-ki˧ | tso˧\textasciitilde{}tso˧ ɲi˧ |}}} \textcolor{PineGreen}{\selectlanguage{french}emprunter des choses à quelqu'un} \zh{向别人借东西}  

\lhead{\firstmark}
\rhead{\botmark}

\subsection{\hspace{-0.5cm} {\Large \textcolor{darkblue}{\textbf{\ipa{ɲi˥}}} \textsubscript{3}}\hspace{0.5cm}[\kern2pt{\textcolor{darkblue}{\textbf{\ipa{ɲi˥}}}}\kern2pt]} \hypertarget{Ji\string_T3}{}
\markboth{\textcolor{darkblue}{\textbf{\ipa{ɲi˥}}} \textsubscript{3}}{}
\textcolor{teal}{\mytextsc{verbe}} \hspace{4pt} Ton~: H.
\textcolor{PineGreen}{\selectlanguage{french}Échouer, perdre.} \zh{败、输。} 
\lhead{\firstmark}
\rhead{\botmark}

\subsection{\hspace{-0.5cm} {\Large \textcolor{darkblue}{\textbf{\ipa{ɲi˥\textsubscript{b}}}}}\hspace{0.5cm}[\kern2pt{\textcolor{darkblue}{\textbf{\ipa{ɲi˥}}}}\kern2pt]} \hypertarget{Ji\string_Tb1}{}
\markboth{\textcolor{darkblue}{\textbf{\ipa{ɲi˥\textsubscript{b}}}}}{}
\textcolor{teal}{\mytextsc{classificateur}} \hspace{4pt} Ton~: H\textsubscript{b}.
\textcolor{PineGreen}{\selectlanguage{french}Un jour.} \zh{日、天。}  ¶ \textcolor{darkblue}{\textbf{\ipa{ɖɯ˧-ɲi˥}}} \textcolor{PineGreen}{\selectlanguage{french}un jour} \zh{一天}  

\lhead{\firstmark}
\rhead{\botmark}

\subsection{\hspace{-0.5cm} {\Large \textcolor{darkblue}{\textbf{\ipa{ɲi˧}}}}\hspace{0.5cm}[\kern2pt{\textcolor{darkblue}{\textbf{\ipa{ɲi˥}}}}\kern2pt]} \hypertarget{Ji\string_M1}{}
\markboth{\textcolor{darkblue}{\textbf{\ipa{ɲi˧}}}}{}
\textcolor{teal}{\mytextsc{adjectif}} \hspace{4pt} Ton~: M.
\textcolor{PineGreen}{\selectlanguage{french}Rassasié, repu.} \zh{饱。}  ¶ \textcolor{darkblue}{\textbf{\ipa{le˧-ɲi˧-ze˧}}} \textcolor{PineGreen}{\selectlanguage{french}\mytextsc{accomp} \string_ \mytextsc{pfv}} \zh{饱了}  
 ¶ \textcolor{darkblue}{\textbf{\ipa{hɑ˧-ɲi˧(-ze˩)}}} \textcolor{PineGreen}{\selectlanguage{french}(je) suis rassasié} \zh{吃饱了。 / 吃饱饭了。}  
 ¶ \textcolor{darkblue}{\textbf{\ipa{njɤ˧ | le˧-ɲi˧-ze˧!}}} \textcolor{PineGreen}{\selectlanguage{french}je suis rassasié} \zh{我饱了。}  

\lhead{\firstmark}
\rhead{\botmark}

\subsection{\hspace{-0.5cm} {\Large \textcolor{darkblue}{\textbf{\ipa{ɲi˧\textsubscript{a}}}}}\hspace{0.5cm}[\kern2pt{\textcolor{darkblue}{\textbf{\ipa{ɲi˥}}}}\kern2pt]} \hypertarget{Ji\string_Ma1}{}
\markboth{\textcolor{darkblue}{\textbf{\ipa{ɲi˧\textsubscript{a}}}}}{}
\textcolor{teal}{\mytextsc{verbe}} \hspace{4pt} Ton~: M\textsubscript{a}.
\textcolor{PineGreen}{\selectlanguage{french}Avoir besoin de, vouloir.} \zh{需要。}  ¶ \textcolor{darkblue}{\textbf{\ipa{no˧ | ə˩-ɲi˧? | mɤ˧-ɲi˧!}}} \textcolor{PineGreen}{\selectlanguage{french}Tu en veux? - (Non,) je n'en veux pas/je n'en ai pas besoin!} \zh{你要吗?- 不要!}  

\lhead{\firstmark}
\rhead{\botmark}

\subsection{\hspace{-0.5cm} {\Large \textcolor{darkblue}{\textbf{\ipa{ɲi˧dʑɯ˧}}}}\hspace{0.5cm}[\kern2pt{\textcolor{darkblue}{\textbf{\ipa{ɲi˧dʑɯ˥}}}}\kern2pt]} \hypertarget{Ji\string_Mdz£M\string_M1}{}
\markboth{\textcolor{darkblue}{\textbf{\ipa{ɲi˧dʑɯ˧}}}}{}
\textcolor{teal}{\mytextsc{nom}} \hspace{4pt} Ton~: H\#.
\textcolor{PineGreen}{\selectlanguage{french}Pénis, organe sexuel masculin.} \zh{男生殖器。}  \zh{量词}: \textcolor{darkblue}{\textbf{\ipa{ɭɯ˧}}}  \mytextsc{clf}~: \textcolor{darkblue}{\textbf{\ipa{ɭɯ˧}}} 
\lhead{\firstmark}
\rhead{\botmark}

\subsection{\hspace{-0.5cm} {\Large \textcolor{darkblue}{\textbf{\ipa{ɲi˧gɤ\#˥}}}}\hspace{0.5cm}[\kern2pt{\textcolor{darkblue}{\textbf{\ipa{ɲi˧gɤ˧}}}}\kern2pt]} \hypertarget{Ji\string_Mg7\#\string_T1}{}
\markboth{\textcolor{darkblue}{\textbf{\ipa{ɲi˧gɤ\#˥}}}}{}
\textcolor{teal}{\mytextsc{nom}} \hspace{4pt} Ton~: \#H.
\textcolor{PineGreen}{\selectlanguage{french}Nez.} \zh{鼻子。}  \zh{量词}: \textcolor{darkblue}{\textbf{\ipa{ɭɯ˧}}}  \mytextsc{clf}~: \textcolor{darkblue}{\textbf{\ipa{ɭɯ˧}}} 
\lhead{\firstmark}
\rhead{\botmark}

\subsection{\hspace{-0.5cm} {\Large \textcolor{darkblue}{\textbf{\ipa{ɲi˧gɤ˧-bæ˧˥}}}}\hspace{0.5cm}[\kern2pt{\textcolor{darkblue}{\textbf{\ipa{xxxx non-correspondance entre le nombre de morphèmes et le nombre de tons de morphèmes}}}}\kern2pt]} \hypertarget{Ji\string_Mg7\string_M-b\{\string_M\string_T1}{}
\markboth{\textcolor{darkblue}{\textbf{\ipa{ɲi˧gɤ˧-bæ˧˥}}}}{}
\textcolor{teal}{\mytextsc{nom}} \hspace{4pt} Ton~: MH\#.
\textcolor{PineGreen}{\selectlanguage{french}Corde accrochée à l'anneau nasal, longe de vache; aussi utilisé par extension pour l'anneau nasal, pour lequel aucun terme propre n'existe.} \zh{牛鼻绳。也可以来指牛鼻圈。}  \zh{量词}: \textcolor{darkblue}{\textbf{\ipa{kʰɯ˩}}}  \mytextsc{clf}~: \textcolor{darkblue}{\textbf{\ipa{kʰɯ˩}}} 
\lhead{\firstmark}
\rhead{\botmark}

\subsection{\hspace{-0.5cm} {\Large \textcolor{darkblue}{\textbf{\ipa{ɲi˧ɬi˧mi˧}}}}\hspace{0.5cm}[\kern2pt{\textcolor{darkblue}{\textbf{\ipa{ɲi˧ɬi˧mi˧}}}}\kern2pt]} \hypertarget{Ji\string_MKi\string_Mmi\string_M1}{}
\markboth{\textcolor{darkblue}{\textbf{\ipa{ɲi˧ɬi˧mi˧}}}}{}
\textcolor{teal}{\mytextsc{nom}} \hspace{4pt} Ton~: M.
\textcolor{PineGreen}{\selectlanguage{french}Le deuxième mois.} \zh{二月。} 
\lhead{\firstmark}
\rhead{\botmark}

\subsection{\hspace{-0.5cm} {\Large \textcolor{darkblue}{\textbf{\ipa{ɲi˧mi˧}}}}\hspace{0.5cm}[\kern2pt{\textcolor{darkblue}{\textbf{\ipa{ɲi˧mi˧}}}}\kern2pt]} \hypertarget{Ji\string_Mmi\string_M1}{}
\markboth{\textcolor{darkblue}{\textbf{\ipa{ɲi˧mi˧}}}}{}
\textcolor{teal}{\mytextsc{nom}} \hspace{4pt} Ton~: M.
\ding{202} \textcolor{PineGreen}{\selectlanguage{french}Soleil.} \zh{太阳。}  ¶ \textcolor{darkblue}{\textbf{\ipa{ɲi˧mi˧ tʰv̩˧}}} \textcolor{PineGreen}{\selectlanguage{french}le soleil se lève} \zh{太阳出来、日出}  
 \zh{量词}: \textcolor{darkblue}{\textbf{\ipa{ɭɯ˧}}} \ding{203} \textcolor{PineGreen}{\selectlanguage{french}La journée; le temps.} \zh{日、时间。}  \mytextsc{clf}~: \textcolor{darkblue}{\textbf{\ipa{ɭɯ˧}}} 
\lhead{\firstmark}
\rhead{\botmark}

\subsection{\hspace{-0.5cm} {\Large \textcolor{darkblue}{\textbf{\ipa{ɲi˧mi˧dɑ˧dzɯ\#˥}}}}\hspace{0.5cm}[\kern2pt{\textcolor{darkblue}{\textbf{\ipa{ɲi˧mi˧dɑ˧dzɯ˧}}}}\kern2pt]} \hypertarget{Ji\string_Mmi\string_MdA\string_MdzM\#\string_T1}{}
\markboth{\textcolor{darkblue}{\textbf{\ipa{ɲi˧mi˧dɑ˧dzɯ\#˥}}}}{}
\textcolor{teal}{\mytextsc{nom}} \hspace{4pt} Ton~: \#H.
\textcolor{PineGreen}{\selectlanguage{french}Éclipse solaire.} \zh{日蚀。}  ¶ \textcolor{darkblue}{\textbf{\ipa{ɲi˧mi˧dɑ˧dzɯ˧ tʰv̩˧}}} \textcolor{PineGreen}{\selectlanguage{french}il y a une éclipse de soleil} \zh{有日蚀}  
 ¶ \textcolor{darkblue}{\textbf{\ipa{ɲi˧mi˧dɑ˧dzɯ˧ ɲi˥!}}} \textcolor{PineGreen}{\selectlanguage{french}Oui, c'est bien une éclipse de soleil!} \zh{是的,是日蚀!}  
 \zh{量词}: \textcolor{darkblue}{\textbf{\ipa{ʂɯ˩}}}  \mytextsc{clf}~: \textcolor{darkblue}{\textbf{\ipa{ʂɯ˩}}} 
\lhead{\firstmark}
\rhead{\botmark}

\subsection{\hspace{-0.5cm} {\Large \textcolor{darkblue}{\textbf{\ipa{ɲi˧mi˧-gv̩˩}}}}\hspace{0.5cm}[\kern2pt{\textcolor{darkblue}{\textbf{\ipa{xxxx non-correspondance entre le nombre de morphèmes et le nombre de tons de morphèmes}}}}\kern2pt]} \hypertarget{Ji\string_Mmi\string_M-gv\string_=\string_B1}{}
\markboth{\textcolor{darkblue}{\textbf{\ipa{ɲi˧mi˧-gv̩˩}}}}{}
\textcolor{teal}{\mytextsc{nom}} \hspace{4pt} Ton~: \mytextsc{L}.
\textcolor{PineGreen}{\selectlanguage{french}Ouest; “[la direction dans laquelle] le soleil se couche”.} \zh{西方。}  ¶ \textcolor{darkblue}{\textbf{\ipa{ɲi˧mi˧gv̩˩-gi˩-dzɤ˩ se˩}}} \textcolor{PineGreen}{\selectlanguage{french}marcher vers l'ouest} \zh{往西边走}  

\lhead{\firstmark}
\rhead{\botmark}

\subsection{\hspace{-0.5cm} {\Large \textcolor{darkblue}{\textbf{\ipa{ɲi˧mi˧-kʰɯ˧ʂɯ˧}}}}\hspace{0.5cm}[\kern2pt{\textcolor{darkblue}{\textbf{\ipa{xxxx non-correspondance entre le nombre de morphèmes et le nombre de tons de morphèmes}}}}\kern2pt]} \hypertarget{Ji\string_Mmi\string_M-k\string_hM\string_Ms`M\string_M1}{}
\markboth{\textcolor{darkblue}{\textbf{\ipa{ɲi˧mi˧-kʰɯ˧ʂɯ˧}}}}{}
\textcolor{teal}{\mytextsc{nom}} \hspace{4pt} Ton~: M.
\textcolor{PineGreen}{\selectlanguage{french}Rayons du soleil.} \zh{太阳的光线。}  \zh{量词}: \textcolor{darkblue}{\textbf{\ipa{kʰɯ˩}}}  \mytextsc{clf}~: \textcolor{darkblue}{\textbf{\ipa{kʰɯ˩}}} 
\lhead{\firstmark}
\rhead{\botmark}

\subsection{\hspace{-0.5cm} {\Large \textcolor{darkblue}{\textbf{\ipa{ɲi˧mi˧tʰv̩\#˥}}}}\hspace{0.5cm}[\kern2pt{\textcolor{darkblue}{\textbf{\ipa{ɲi˧mi˧tʰv̩˧}}}}\kern2pt]} \hypertarget{Ji\string_Mmi\string_Mt\string_hv\string_=\#\string_T1}{}
\markboth{\textcolor{darkblue}{\textbf{\ipa{ɲi˧mi˧tʰv̩\#˥}}}}{}
\textcolor{teal}{\mytextsc{nom}} \hspace{4pt} Ton~: \#H.
\textcolor{PineGreen}{\selectlanguage{french}Est, orient.} \zh{东方。}  ¶ \textcolor{darkblue}{\textbf{\ipa{ɲi˧mi˧tʰv̩˧-gi˧}}} \textcolor{PineGreen}{\selectlanguage{french}la direction de l'est} \zh{东方方向}  
 ¶ \textcolor{darkblue}{\textbf{\ipa{ɲi˧mi˧tʰv̩˧-gi˧ | se˧}}} \textcolor{PineGreen}{\selectlanguage{french}marcher vers l'est} \zh{向东面走}  
 ¶ \textcolor{darkblue}{\textbf{\ipa{ɲi˧mi˧tʰv̩˧-gi˧ | dʑo˩˥}}} \textcolor{PineGreen}{\selectlanguage{french}se trouver à l'est, habiter en Orient (contexte: la locutrice m'imagine, depuis l'Europe, pensant à elle, et disant: “elle habite en Orient”.} \zh{住在东方(合作人想象我在欧洲,想着她说:‘她住在东方’。)}  

\lhead{\firstmark}
\rhead{\botmark}

\subsection{\hspace{-0.5cm} {\Large \textcolor{darkblue}{\textbf{\ipa{ɲi˧mi˧-ʈæ˧ʂɯ˧}}}}\hspace{0.5cm}[\kern2pt{\textcolor{darkblue}{\textbf{\ipa{xxxx non-correspondance entre le nombre de morphèmes et le nombre de tons de morphèmes}}}}\kern2pt]} \hypertarget{Ji\string_Mmi\string_M-t`\{\string_Ms`M\string_M1}{}
\markboth{\textcolor{darkblue}{\textbf{\ipa{ɲi˧mi˧-ʈæ˧ʂɯ˧}}}}{}
\textcolor{teal}{\mytextsc{nom}} \hspace{4pt} Ton~: M.
\textcolor{PineGreen}{\selectlanguage{french}Tournesol.} \zh{葵花。}  \zh{量词}: \textcolor{darkblue}{\textbf{\ipa{dzi˩}}}  \mytextsc{clf}~: \textcolor{darkblue}{\textbf{\ipa{dzi˩}}} 
\lhead{\firstmark}
\rhead{\botmark}

\subsection{\hspace{-0.5cm} {\Large \textcolor{darkblue}{\textbf{\ipa{ɲi˧nɑ˩}}}}\hspace{0.5cm}[\kern2pt{\textcolor{darkblue}{\textbf{\ipa{ɲi˧nɑ˩}}}}\kern2pt]} \hypertarget{Ji\string_MnA\string_B1}{}
\markboth{\textcolor{darkblue}{\textbf{\ipa{ɲi˧nɑ˩}}}}{}
\textcolor{teal}{\mytextsc{nom}} \hspace{4pt} Ton~: L\#.
\textcolor{PineGreen}{\selectlanguage{french}Liane, rattan, vigne vierge, lierre.} \zh{藤子。} 
\lhead{\firstmark}
\rhead{\botmark}

\subsection{\hspace{-0.5cm} {\Large \textcolor{darkblue}{\textbf{\ipa{ɲi˧pʰv̩˩}}}}\hspace{0.5cm}[\kern2pt{\textcolor{darkblue}{\textbf{\ipa{ɲi˧pʰv̩˩}}}}\kern2pt]} \hypertarget{Ji\string_Mp\string_hv\string_=\string_B1}{}
\markboth{\textcolor{darkblue}{\textbf{\ipa{ɲi˧pʰv̩˩}}}}{}
\textcolor{teal}{\mytextsc{nom}} \hspace{4pt} Ton~: L\#.
\textcolor{PineGreen}{\selectlanguage{french}Givre.} \zh{霜。}  ¶ \textcolor{darkblue}{\textbf{\ipa{ɲi˧pʰv̩˩ lɑ˩-ze˩}}} \textcolor{PineGreen}{\selectlanguage{french}il y a du givre} \zh{有霜}  

\lhead{\firstmark}
\rhead{\botmark}

\subsection{\hspace{-0.5cm} {\Large \textcolor{darkblue}{\textbf{\ipa{ɲi˧qʰv̩˧}}}}\hspace{0.5cm}[\kern2pt{\textcolor{darkblue}{\textbf{\ipa{ɲi˧qʰv̩˧}}}}\kern2pt]} \hypertarget{Ji\string_Mq\string_hv\string_=\string_M1}{}
\markboth{\textcolor{darkblue}{\textbf{\ipa{ɲi˧qʰv̩˧}}}}{}
\textcolor{teal}{\mytextsc{nom}} \hspace{4pt} Ton~: M.
\ding{202} \textcolor{PineGreen}{\selectlanguage{french}Narine.} \zh{鼻孔。}  \zh{量词}: \textcolor{darkblue}{\textbf{\ipa{ɭɯ˧}}} \ding{203} \textcolor{PineGreen}{\selectlanguage{french}Mucus, morve.} \zh{鼻涕。}  \mytextsc{clf}~: \textcolor{darkblue}{\textbf{\ipa{ɭɯ˧}}} 
\lhead{\firstmark}
\rhead{\botmark}

\subsection{\hspace{-0.5cm} {\Large \textcolor{darkblue}{\textbf{\ipa{ɲi˧se˩}}}}\hspace{0.5cm}[\kern2pt{\textcolor{darkblue}{\textbf{\ipa{ɲi˧se˩}}}}\kern2pt]} \hypertarget{Ji\string_Mse\string_B1}{}
\markboth{\textcolor{darkblue}{\textbf{\ipa{ɲi˧se˩}}}}{}
\textcolor{teal}{\mytextsc{nom}} \hspace{4pt} Ton~: L\#.
\textcolor{PineGreen}{\selectlanguage{french}Un village du bord du Lac.} \zh{小落水(村落名)。}  ¶ \textcolor{darkblue}{\textbf{\ipa{ɲi˧se˩, | nɑ˩-lɑ˧ ɲi˥!}}} \textcolor{PineGreen}{\selectlanguage{french}Nhissei, c'est un village entièrement Na!} \zh{小落水,是纯摩梭的一个村落!}  
 ¶ \textcolor{darkblue}{\textbf{\ipa{ɬi˧ki˧, | ɲi˧se˩, | tɑ˧dzi˩, | mv̩˧qʰwæ˩, | lɑ˧tʰɑ˧-di˧˥}}} \textcolor{PineGreen}{\selectlanguage{french}Villages dans l'ordre, après la plaine de Yongning, ne comptant pas comme faisant partie de Yongning. Le dernier, \textcolor{darkblue}{\textbf{\ipa{/lɑ˧tʰɑ˧-di˧˥/}}}, désigne toute la région na au-delà du quatrième village.} \zh{永宁到泸沽湖所经过的村落,依次是:里格、尼赛、大祖、木垮,然后到拉塔地(拉塔地指的是泸沽湖周边的摩梭地区,包括左所、洛水村等)}  

\lhead{\firstmark}
\rhead{\botmark}

\subsection{\hspace{-0.5cm} {\Large \textcolor{darkblue}{\textbf{\ipa{ɲi˧to˧}}}}\hspace{0.5cm}[\kern2pt{\textcolor{darkblue}{\textbf{\ipa{ɲi˧to˧}}}}\kern2pt]} \hypertarget{Ji\string_Mto\string_M1}{}
\markboth{\textcolor{darkblue}{\textbf{\ipa{ɲi˧to˧}}}}{}
\textcolor{teal}{\mytextsc{nom}} \hspace{4pt} Ton~: M.
\textcolor{PineGreen}{\selectlanguage{french}Bouche/pourtour de la bouche (autour des lèvre).} \zh{嘴巴,包括嘴周围的部位:颚等。}  ¶ \textcolor{darkblue}{\textbf{\ipa{ɲi˧to˧ ʈʂʰwæ˩}}} \textcolor{PineGreen}{\selectlanguage{french}bavard (littéralement “bouche rapide”)} \zh{多嘴、拉不断扯不断(直译:“嘴快”)}  
 \zh{量词}: \textcolor{darkblue}{\textbf{\ipa{kʰwɤ˥}}}  \mytextsc{clf}~: \textcolor{darkblue}{\textbf{\ipa{kʰwɤ˥}}} 
\lhead{\firstmark}
\rhead{\botmark}

\subsection{\hspace{-0.5cm} {\Large \textcolor{darkblue}{\textbf{\ipa{ɲi˧-ʈʂæ˧-ʑi˧˥}}}}\hspace{0.5cm}[\kern2pt{\textcolor{darkblue}{\textbf{\ipa{xxxx non-correspondance entre le nombre de morphèmes et le nombre de tons de morphèmes}}}}\kern2pt]} \hypertarget{Ji\string_M-t`s`\{\string_M-z£i\string_M\string_T1}{}
\markboth{\textcolor{darkblue}{\textbf{\ipa{ɲi˧-ʈʂæ˧-ʑi˧˥}}}}{}
\textcolor{teal}{\mytextsc{nom}} \hspace{4pt} Ton~: MH\#.
\textcolor{PineGreen}{\selectlanguage{french}Bâtiment d'habitation; littéralement 'le bâtiment à deux étages', car c'est le seul qui ait des salles sur deux étages. Ce bâtiment se trouve face à l'entrée de la ferme.} \zh{二层房:农场里面的一栋楼,正对着农场大门。}  ¶ \textcolor{darkblue}{\textbf{\ipa{ɲi˧-ʈʂæ˧-ʑi˧-di˥}}} \textcolor{PineGreen}{\selectlanguage{french}même sens} \zh{同上}  
 \zh{量词}: \textcolor{darkblue}{\textbf{\ipa{ɭɯ˧}}}  \mytextsc{clf}~: \textcolor{darkblue}{\textbf{\ipa{ɭɯ˧}}} 
\lhead{\firstmark}
\rhead{\botmark}

\subsection{\hspace{-0.5cm} {\Large \textcolor{darkblue}{\textbf{\ipa{ɲi˧zo\#˥}}}}\hspace{0.5cm}[\kern2pt{\textcolor{darkblue}{\textbf{\ipa{ɲi˧zo˧}}}}\kern2pt]} \hypertarget{Ji\string_Mzo\#\string_T1}{}
\markboth{\textcolor{darkblue}{\textbf{\ipa{ɲi˧zo\#˥}}}}{}
\textcolor{teal}{\mytextsc{nom}} \hspace{4pt} Ton~: \#H.
\textcolor{PineGreen}{\selectlanguage{french}Poisson.} \zh{鱼。}  ¶ \textcolor{darkblue}{\textbf{\ipa{ɲi˧zo˧ tʰv̩˧-mi˥\# / ɲi˧zo˧ tʰv̩˧-mi˧˥}}} \textcolor{PineGreen}{\selectlanguage{french}\mytextsc{n}+\mytextsc{dem}+\mytextsc{clf}} \zh{那条鱼}  
 ¶ \textcolor{darkblue}{\textbf{\ipa{ɲi˧zo˧-tɑ˧pv̩˥}}} \textcolor{PineGreen}{\selectlanguage{french}poisson séché} \zh{干鱼}  
 \zh{量词}: \textcolor{darkblue}{\textbf{\ipa{mi˩}}}  \mytextsc{clf}~: \textcolor{darkblue}{\textbf{\ipa{mi˩}}} 
\lhead{\firstmark}
\rhead{\botmark}

\subsection{\hspace{-0.5cm} {\Large \textcolor{darkblue}{\textbf{\ipa{ɲi˩}}} \textsubscript{1}}\hspace{0.5cm}[\kern2pt{\textcolor{darkblue}{\textbf{\ipa{ɲi˩˥}}}}\kern2pt]} \hypertarget{Ji\string_B1}{}
\markboth{\textcolor{darkblue}{\textbf{\ipa{ɲi˩}}} \textsubscript{1}}{}
\textcolor{teal}{\mytextsc{verbe}} \hspace{4pt} Ton~: L\textsubscript{a}.
\textcolor{PineGreen}{\selectlanguage{french}Serrer, tenir (ex.: tenir quelque chose serré sous le bras, serrer quelque chose entre les jambes).} \zh{夹、夹持。}  ¶ \textcolor{darkblue}{\textbf{\ipa{ɖɯ˧-ɲi˧\textasciitilde{}ɲi˥-ɻ̍˩}}} \textcolor{PineGreen}{\selectlanguage{french}serrer un peu} \zh{夹一点}  

\lhead{\firstmark}
\rhead{\botmark}

\subsection{\hspace{-0.5cm} {\Large \textcolor{darkblue}{\textbf{\ipa{ɲi˩}}} \textsubscript{2}}\hspace{0.5cm}[\kern2pt{\textcolor{darkblue}{\textbf{\ipa{ɲi˩˥}}}}\kern2pt]} \hypertarget{Ji\string_B2}{}
\markboth{\textcolor{darkblue}{\textbf{\ipa{ɲi˩}}} \textsubscript{2}}{}
\textcolor{teal}{\mytextsc{pronom}} \hspace{4pt} Ton~: L.
\textcolor{PineGreen}{\selectlanguage{french}Pronom interrogatif: qui?} \zh{谁。}  ¶ \textcolor{darkblue}{\textbf{\ipa{ɲi˩ ɲi˧?}}} \textcolor{PineGreen}{\selectlanguage{french}C'est qui?} \zh{是谁?}  
 ¶ \textcolor{darkblue}{\textbf{\ipa{no˧ | ɲi˩ ɲi˧?}}} \textcolor{PineGreen}{\selectlanguage{french}Qui êtes-vous?} \zh{你是谁?}  
 ¶ \textcolor{darkblue}{\textbf{\ipa{ʈʂʰɯ˧ | ɲi˩ ɲi˧?}}} \textcolor{PineGreen}{\selectlanguage{french}Qui est-ce?} \zh{他是谁?}  
 ¶ \textcolor{darkblue}{\textbf{\ipa{no˧ | ɲi˩˥ | -ki˩ bi˩-pi˩, | ɖɯ˧-bæ˧ lɑ˧ ɲi˥!}}} \textcolor{PineGreen}{\selectlanguage{french}Peu importe chez qui tu vas, c'est pareil partout!} \zh{无论你去谁(家),都一样!}  
 ¶ \textcolor{darkblue}{\textbf{\ipa{no˧ | ɲi˩-ki˥ bi˩-pi˩, | ɖɯ˧-bæ˧ lɑ˧ ɲi˥!}}} \textcolor{PineGreen}{\selectlanguage{french}Comme l'exemple précédent, avec une division en groupes tonals différente} \zh{同上,声调段界不同}  

\lhead{\firstmark}
\rhead{\botmark}

\subsection{\hspace{-0.5cm} {\Large \textcolor{darkblue}{\textbf{\ipa{-ɲi˩}}}}\hspace{0.5cm}[\kern2pt{\textcolor{darkblue}{\textbf{\ipa{ɲi˩˥}}}}\kern2pt]} \hypertarget{-Ji\string_B1}{}
\markboth{\textcolor{darkblue}{\textbf{\ipa{-ɲi˩}}}}{}
\textcolor{teal}{\mytextsc{particule}} \textcolor{teal}{\mytextsc{de}} \textcolor{teal}{\mytextsc{discours}} \hspace{4pt} Ton~: L.
\textcolor{PineGreen}{\selectlanguage{french}Particule indiquant la certitude; dérivée du verbe copule.} \zh{\mytextsc{肯定(°系词)。}} 
\lhead{\firstmark}
\rhead{\botmark}

\subsection{\hspace{-0.5cm} {\Large \textcolor{darkblue}{\textbf{\ipa{ɲi˩\textsubscript{a}}}} \textsubscript{1}}\hspace{0.5cm}[\kern2pt{\textcolor{darkblue}{\textbf{\ipa{ɲi˩˥}}}}\kern2pt]} \hypertarget{Ji\string_Ba1}{}
\markboth{\textcolor{darkblue}{\textbf{\ipa{ɲi˩\textsubscript{a}}}} \textsubscript{1}}{}
\textcolor{teal}{\mytextsc{verbe}} \hspace{4pt} Ton~: L\textsubscript{a}.
\textcolor{PineGreen}{\selectlanguage{french}Tordre avec les doigts, enrouler, filer (pour fabriquer du fil de lin, pour tisser des vêtements).} \zh{捻,缠线。}  ¶ \textcolor{darkblue}{\textbf{\ipa{le˧-ɲi˩}}} \textcolor{PineGreen}{\selectlanguage{french}\mytextsc{accomp}} \zh{\mytextsc{accomp}}  
 ¶ \textcolor{darkblue}{\textbf{\ipa{sɑ˧ ɲi˥}}} \textcolor{PineGreen}{\selectlanguage{french}tordre le chanvre/le lin (pour faire du fil)} \zh{捻麻}  
 ¶ \textcolor{darkblue}{\textbf{\ipa{ɖɯ˧-ɲi˧\textasciitilde{}ɲi˥-ɻ̍˩}}} \textcolor{PineGreen}{\selectlanguage{french}\mytextsc{délimitatif} \mytextsc{red} \mytextsc{inchoatif}} \zh{捻一捻}  

\lhead{\firstmark}
\rhead{\botmark}

\subsection{\hspace{-0.5cm} {\Large \textcolor{darkblue}{\textbf{\ipa{ɲi˩\textsubscript{a}}}} \textsubscript{2}}\hspace{0.5cm}[\kern2pt{\textcolor{darkblue}{\textbf{\ipa{ɲi˩˥}}}}\kern2pt]} \hypertarget{Ji\string_Ba2}{}
\markboth{\textcolor{darkblue}{\textbf{\ipa{ɲi˩\textsubscript{a}}}} \textsubscript{2}}{}
\textcolor{teal}{\mytextsc{verbe}} \hspace{4pt} Ton~: L\textsubscript{a}.
\textcolor{PineGreen}{\selectlanguage{french}S'abîmer, se casser; tomber en panne (ex.: appareil photo).} \zh{设备坏了。}  ¶ \textcolor{darkblue}{\textbf{\ipa{le˧-ɲi˩-ze˩}}} \textcolor{PineGreen}{\selectlanguage{french}\mytextsc{accomp} \string_ \mytextsc{pfv}: c'est cassé!} \zh{坏了!/破了!}  
 ¶ \textcolor{darkblue}{\textbf{\ipa{tso˧\textasciitilde{}tso˧ ɲi˥}}} \textcolor{PineGreen}{\selectlanguage{french}casser des choses} \zh{东西坏了}  

\lhead{\firstmark}
\rhead{\botmark}

\subsection{\hspace{-0.5cm} {\Large \textcolor{darkblue}{\textbf{\ipa{ɲi˩\textsubscript{a}}}} \textsubscript{3}}\hspace{0.5cm}[\kern2pt{\textcolor{darkblue}{\textbf{\ipa{ɲi˩˥}}}}\kern2pt]} \hypertarget{Ji\string_Ba3}{}
\markboth{\textcolor{darkblue}{\textbf{\ipa{ɲi˩\textsubscript{a}}}} \textsubscript{3}}{}
\textcolor{teal}{\mytextsc{verbe}} \hspace{4pt} Ton~: L\textsubscript{a}.
\textcolor{PineGreen}{\selectlanguage{french}Verbe copule.} \zh{是\mytextsc{系词。}} 
\lhead{\firstmark}
\rhead{\botmark}

\subsection{\hspace{-0.5cm} {\Large \textcolor{darkblue}{\textbf{\ipa{ɲi˩bv̩˩}}}}\hspace{0.5cm}[\kern2pt{\textcolor{darkblue}{\textbf{\ipa{ɲi˩bv̩˩˥}}}}\kern2pt]} \hypertarget{Ji\string_Bbv\string_=\string_B1}{}
\markboth{\textcolor{darkblue}{\textbf{\ipa{ɲi˩bv̩˩}}}}{}
\textcolor{teal}{\mytextsc{nom}} \hspace{4pt} Ton~: L.
\textcolor{PineGreen}{\selectlanguage{french}Criquet.} \zh{蟋蟀。}  \zh{量词}: \textcolor{darkblue}{\textbf{\ipa{mi˩}}}  \mytextsc{clf}~: \textcolor{darkblue}{\textbf{\ipa{mi˩}}} 
\lhead{\firstmark}
\rhead{\botmark}

\subsection{\hspace{-0.5cm} {\Large \textcolor{darkblue}{\textbf{\ipa{ɲi˩bv̩˩-ʂe˩sɑ˧}}}}\hspace{0.5cm}[\kern2pt{\textcolor{darkblue}{\textbf{\ipa{xxxx non-correspondance entre le nombre de morphèmes et le nombre de tons de morphèmes}}}}\kern2pt]} \hypertarget{Ji\string_Bbv\string_=\string_B-s`e\string_BsA\string_M1}{}
\markboth{\textcolor{darkblue}{\textbf{\ipa{ɲi˩bv̩˩-ʂe˩sɑ˧}}}}{}
\textcolor{teal}{\mytextsc{nom}} \hspace{4pt} Ton~: LM.
\textcolor{PineGreen}{\selectlanguage{french}Libellule.} \zh{蜻蜓。}  \zh{量词}: \textcolor{darkblue}{\textbf{\ipa{mi˩}}}  \mytextsc{clf}~: \textcolor{darkblue}{\textbf{\ipa{mi˩}}} 
\lhead{\firstmark}
\rhead{\botmark}

\subsection{\hspace{-0.5cm} {\Large \textcolor{darkblue}{\textbf{\ipa{ɲi˩mɑ\#˥}}}}\hspace{0.5cm}[\kern2pt{\textcolor{darkblue}{\textbf{\ipa{ɲi˩mɑ˥}}}}\kern2pt]} \hypertarget{Ji\string_BmA\#\string_T1}{}
\markboth{\textcolor{darkblue}{\textbf{\ipa{ɲi˩mɑ\#˥}}}}{}
\textcolor{teal}{\mytextsc{nom}} \hspace{4pt} Ton~: LM+\#H.
\textcolor{PineGreen}{\selectlanguage{french}Prénom masculin pour l'aîné des jumeaux (l'enfant né en premier).} \zh{男性名字,起给双胞胎中的老大。} 
\lhead{\firstmark}
\rhead{\botmark}

\subsection{\hspace{-0.5cm} {\Large \textcolor{darkblue}{\textbf{\ipa{ɲi˩pʰv̩˩}}}}\hspace{0.5cm}[\kern2pt{\textcolor{darkblue}{\textbf{\ipa{ɲi˩pʰv̩˩˥}}}}\kern2pt]} \hypertarget{Ji\string_Bp\string_hv\string_=\string_B1}{}
\markboth{\textcolor{darkblue}{\textbf{\ipa{ɲi˩pʰv̩˩}}}}{}
\textcolor{teal}{\mytextsc{nom}} \hspace{4pt} Ton~: L.
\textcolor{PineGreen}{\selectlanguage{french}Une plante de montagne; la locutrice pense la reconnaître sur une photo de menthe aquatique, \textit{Mentha aquatica, Mentha hirsuta Huds.} mais ce n'est vraisemblablement pas la bonne identification.} \zh{一种植物。合作人看水薄荷的图片就觉得像这种植物,但很可能不是。李达珠等(2015:98)翻译为“野牡丹”但这好像也不准确。}  ¶ \textcolor{darkblue}{\textbf{\ipa{ɲi˩pʰv̩˩-bæ˥bæ˩}}} \textcolor{PineGreen}{\selectlanguage{french}la fleur de cette plante} \zh{这种植物的花}  

\lhead{\firstmark}
\rhead{\botmark}

\subsection{\hspace{-0.5cm} {\Large \textcolor{darkblue}{\textbf{\ipa{ɲi˩=ɻ̍˥}}}}\hspace{0.5cm}[\kern2pt{\textcolor{darkblue}{\textbf{\ipa{ɲi˩ɻ̍˥}}}}\kern2pt]} \hypertarget{Ji\string_B=r£`̍\string_T1}{}
\markboth{\textcolor{darkblue}{\textbf{\ipa{ɲi˩=ɻ̍˥}}}}{}
\textcolor{teal}{\mytextsc{pronom}} \hspace{4pt} Ton~: LM+H\#.
\textcolor{PineGreen}{\selectlanguage{french}Pronom de 2e personne associatif: toi et les tiens.} \zh{第二人称,联想复数:你与周边的人(家人、家族、亲戚、朋友们……)。} 
\lhead{\firstmark}
\rhead{\botmark}

\subsection{\hspace{-0.5cm} {\Large \textcolor{darkblue}{\textbf{\ipa{ɲi˩tsɯ\#˥}}}}\hspace{0.5cm}[\kern2pt{\textcolor{darkblue}{\textbf{\ipa{ɲi˩tsɯ˥}}}}\kern2pt]} \hypertarget{Ji\string_BtsM\#\string_T1}{}
\markboth{\textcolor{darkblue}{\textbf{\ipa{ɲi˩tsɯ\#˥}}}}{}
\textcolor{teal}{\mytextsc{nom}} \hspace{4pt} Ton~: LM+\#H.
\textcolor{PineGreen}{\selectlanguage{french}Hmông (groupe ethnique).} \zh{苗族。}  \zh{量词}: \textcolor{darkblue}{\textbf{\ipa{v̩˧}}}  \mytextsc{clf}~: \textcolor{darkblue}{\textbf{\ipa{v̩˧}}} 
\lhead{\firstmark}
\rhead{\botmark}

\subsection{\hspace{-0.5cm} {\Large \textcolor{darkblue}{\textbf{\ipa{ɲi˩ʈʂe˩}}}}\hspace{0.5cm}[\kern2pt{\textcolor{darkblue}{\textbf{\ipa{ɲi˩ʈʂe˩˥}}}}\kern2pt]} \hypertarget{Ji\string_Bt`s`e\string_B1}{}
\markboth{\textcolor{darkblue}{\textbf{\ipa{ɲi˩ʈʂe˩}}}}{}
\textcolor{teal}{\mytextsc{nom}} \hspace{4pt} Ton~: L.
\textcolor{PineGreen}{\selectlanguage{french}Barre de porte: barre pour fermer la porte principale de la ferme.} \zh{门闩。}  ¶ \textcolor{darkblue}{\textbf{\ipa{ɲi˩ʈʂe˩ tʰi˥-kʰɯ˩, | tʰi˧-ʈæ˩!}}} \textcolor{PineGreen}{\selectlanguage{french}On met la barre à la porte; on verrouille! / On met la barre à la porte, et c'est fermé!} \zh{放门闩,(好好)锁(门)!}  

\lhead{\firstmark}
\rhead{\botmark}

\subsection{\hspace{-0.5cm} {\Large \textcolor{darkblue}{\textbf{\ipa{ɲi˧˥}}}}\hspace{0.5cm}[\kern2pt{\textcolor{darkblue}{\textbf{\ipa{ɲi˧˥}}}}\kern2pt]} \hypertarget{Ji\string_M\string_T1}{}
\markboth{\textcolor{darkblue}{\textbf{\ipa{ɲi˧˥}}}}{}
\textcolor{teal}{\mytextsc{nombre}} \hspace{4pt} Ton~: MH.
\textcolor{PineGreen}{\selectlanguage{french}2.} \zh{2。} 
\lhead{\firstmark}
\rhead{\botmark}

\newpage
\section*{\centering- \textcolor{darkblue}{\textbf{\ipa{ŋ}}} -}
\subsection{\hspace{-0.5cm} {\Large \textcolor{darkblue}{\textbf{\ipa{ŋæ˧ʝi˩}}}}\hspace{0.5cm}[\kern2pt{\textcolor{darkblue}{\textbf{\ipa{ŋæ˩ʝi˥}}}}\kern2pt]} \hypertarget{N\{\string_Mj££i\string_B1}{}
\markboth{\textcolor{darkblue}{\textbf{\ipa{ŋæ˧ʝi˩}}}}{}
\textcolor{teal}{\mytextsc{adjectif}} \hspace{4pt} Ton~: L\#.
\textcolor{PineGreen}{\selectlanguage{french}À l'aise, dans le confort, dans l'abondance.} \zh{安逸(汉语借词)。}  Emprunt~: chinois  \zh{安逸}

\lhead{\firstmark}
\rhead{\botmark}

\subsection{\hspace{-0.5cm} {\Large \textcolor{darkblue}{\textbf{\ipa{ŋɤ˩ŋɤ˩}}}}\hspace{0.5cm}[\kern2pt{\textcolor{darkblue}{\textbf{\ipa{ŋɤ˩ŋɤ˩˥}}}}\kern2pt]} \hypertarget{N7\string_BN7\string_B1}{}
\markboth{\textcolor{darkblue}{\textbf{\ipa{ŋɤ˩ŋɤ˩}}}}{}
\textcolor{teal}{\mytextsc{nom}} \hspace{4pt} Ton~: L.
\textcolor{PineGreen}{\selectlanguage{french}Palais.} \zh{上腭。}  \zh{量词}: \textcolor{darkblue}{\textbf{\ipa{kʰwɤ˥}}}  \mytextsc{clf}~: \textcolor{darkblue}{\textbf{\ipa{kʰwɤ˥}}} 
\lhead{\firstmark}
\rhead{\botmark}

\subsection{\hspace{-0.5cm} {\Large \textcolor{darkblue}{\textbf{\ipa{ŋv̩˩}}}}\hspace{0.5cm}[\kern2pt{\textcolor{darkblue}{\textbf{\ipa{ŋv̩˥}}}}\kern2pt]} \hypertarget{Nv\string_=\string_B1}{}
\markboth{\textcolor{darkblue}{\textbf{\ipa{ŋv̩˩}}}}{}
\textcolor{teal}{\mytextsc{nom}} \hspace{4pt} Ton~: L.
\ding{202} \textcolor{PineGreen}{\selectlanguage{french}Argent (métal).} \zh{银子。}  ¶ \textcolor{darkblue}{\textbf{\ipa{ŋv˧hæ̃˩/ or et ærgent càd ærgent, pætrimoine}}} \textcolor{PineGreen}{\selectlanguage{french}argent, patrimoine, fortune; littéralement 'or et argent'} \zh{金钱、钱财、财富。直译:‘银子与金子’}  
\ding{203} \textcolor{PineGreen}{\selectlanguage{french}Argent (argent-papier et pièces de monnaie).} \zh{钱。} 
\lhead{\firstmark}
\rhead{\botmark}

\subsection{\hspace{-0.5cm} {\Large \textcolor{darkblue}{\textbf{\ipa{ŋv̩˩\textsubscript{a}}}}}\hspace{0.5cm}[\kern2pt{\textcolor{darkblue}{\textbf{\ipa{ŋv̩˩˥}}}}\kern2pt]} \hypertarget{Nv\string_=\string_Ba1}{}
\markboth{\textcolor{darkblue}{\textbf{\ipa{ŋv̩˩\textsubscript{a}}}}}{}
\textcolor{teal}{\mytextsc{verbe}} \hspace{4pt} Ton~: L\textsubscript{a}.
\textcolor{PineGreen}{\selectlanguage{french}Pleurer.} \zh{哭。}  ¶ \textcolor{darkblue}{\textbf{\ipa{(tʰi˧-)ŋv̩˧\textasciitilde{}ŋv̩˥}}} \textcolor{PineGreen}{\selectlanguage{french}\mytextsc{dur} \mytextsc{red}} \zh{哭一场}  

\lhead{\firstmark}
\rhead{\botmark}

\subsection{\hspace{-0.5cm} {\Large \textcolor{darkblue}{\textbf{\ipa{ŋwæ˧qʰv̩˧}}}}\hspace{0.5cm}[\kern2pt{\textcolor{darkblue}{\textbf{\ipa{ŋwæ˧qʰv̩˧}}}}\kern2pt]} \hypertarget{Nw\{\string_Mq\string_hv\string_=\string_M1}{}
\markboth{\textcolor{darkblue}{\textbf{\ipa{ŋwæ˧qʰv̩˧}}}}{}
\textcolor{teal}{\mytextsc{nom}} \hspace{4pt} Ton~: M.
\textcolor{PineGreen}{\selectlanguage{french}Four où on cuit les tuiles.} \zh{烧瓦的烤炉。}  ¶ \textcolor{darkblue}{\textbf{\ipa{ŋwæ˧qʰv̩˧ ʂɯ˧-ʑi˩}}} \textcolor{PineGreen}{\selectlanguage{french}'les sept familles du Four à tuiles': expression dont on désignait autrefois les gens du village de Alawa, du temps où il n'y avait là que sept familles} \zh{‘瓦炉七家’:过去来指阿拉瓦村的人,当时那里只有七家住}  
 ¶ \textcolor{darkblue}{\textbf{\ipa{ə˧lɑ˧-ʁwɤ˧ | ŋwæ˧qʰv̩˧ | tsʰe˧ɲi˧ ʑi˩}}} \textcolor{PineGreen}{\selectlanguage{french}“les douze familles de Alawa”: expression dont on désignait autrefois les gens du village de Alawa, du temps où le nombre de familles était passé de sept à douze par l'arrivée de nouveaux venus.} \zh{‘阿拉瓦瓦炉十二家’:过去来指阿拉瓦村的人,当时那里住的人家,从七家已经增加到十二家}  
 \zh{量词}: \textcolor{darkblue}{\textbf{\ipa{ɭɯ˧}}}  \mytextsc{clf}~: \textcolor{darkblue}{\textbf{\ipa{ɭɯ˧}}} 
\lhead{\firstmark}
\rhead{\botmark}

\subsection{\hspace{-0.5cm} {\Large \textcolor{darkblue}{\textbf{\ipa{ŋwɤ˧}}}}\hspace{0.5cm}[\kern2pt{\textcolor{darkblue}{\textbf{\ipa{ŋwɤ˥}}}}\kern2pt]} \hypertarget{Nw7\string_M1}{}
\markboth{\textcolor{darkblue}{\textbf{\ipa{ŋwɤ˧}}}}{}
\textcolor{teal}{\mytextsc{nombre}} \hspace{4pt} Ton~: M? H\#? (pas L).
\textcolor{PineGreen}{\selectlanguage{french}5.} \zh{5。} 
\lhead{\firstmark}
\rhead{\botmark}

\subsection{\hspace{-0.5cm} {\Large \textcolor{darkblue}{\textbf{\ipa{ŋwɤ˧hɑ̃˩}}}}\hspace{0.5cm}[\kern2pt{\textcolor{darkblue}{\textbf{\ipa{ŋwɤ˧hɑ̃˩}}}}\kern2pt]} \hypertarget{Nw7\string_MhA\string_~\string_B1}{}
\markboth{\textcolor{darkblue}{\textbf{\ipa{ŋwɤ˧hɑ̃˩}}}}{}
\textcolor{teal}{\mytextsc{nom}} \hspace{4pt} Ton~: L\#.
\textcolor{PineGreen}{\selectlanguage{french}Nom d'une montagne au sud-ouest de Yongning.} \zh{位于永宁西南的一座山。}  ¶ \textcolor{darkblue}{\textbf{\ipa{kɤ˧mv̩˧˥, | æ˧ʂæ˧, | ŋwɤ˧hɑ̃˩, | ʂwæ˧gv̩\#˥, | nɑ˩tsʰi˩˥ | -tɕʰɤ˧pɤ˧mi\#˥, | qv̩˧ɻ̍˧-ʈʂʰɑ˧nɑ˥ |}}} \textcolor{PineGreen}{\selectlanguage{french}Les six montagnes de Yongning qui portent un nom. Les autres sommets du voisinage n'ont pas une valeur symbolique comparable, et ne portent pas de nom communément utilisé.} \zh{永宁地区有固定名字的六座山。其它的山,因为没有重要的象征意义,因此没有取名。}  

\lhead{\firstmark}
\rhead{\botmark}

\subsection{\hspace{-0.5cm} {\Large \textcolor{darkblue}{\textbf{\ipa{ŋwɤ˧pʰæ˧˥}}}}\hspace{0.5cm}[\kern2pt{\textcolor{darkblue}{\textbf{\ipa{ŋwɤ˧pʰæ˧˥}}}}\kern2pt]} \hypertarget{Nw7\string_Mp\string_h\{\string_M\string_T1}{}
\markboth{\textcolor{darkblue}{\textbf{\ipa{ŋwɤ˧pʰæ˧˥}}}}{}
\textcolor{teal}{\mytextsc{nom}} \hspace{4pt} Ton~: MH\#.
\textcolor{PineGreen}{\selectlanguage{french}Tuile.} \zh{瓦(汉语借词)。}  Emprunt~: chinois  \zh{瓦}
 \zh{量词}: \textcolor{darkblue}{\textbf{\ipa{pʰæ˧˥}}}  \mytextsc{clf}~: \textcolor{darkblue}{\textbf{\ipa{pʰæ˧˥}}} 
\lhead{\firstmark}
\rhead{\botmark}

\subsection{\hspace{-0.5cm} {\Large \textcolor{darkblue}{\textbf{\ipa{ŋwɤ˧qo˥}}}}\hspace{0.5cm}[\kern2pt{\textcolor{darkblue}{\textbf{\ipa{ŋwɤ˧qo˥}}}}\kern2pt]} \hypertarget{Nw7\string_Mqo\string_T1}{}
\markboth{\textcolor{darkblue}{\textbf{\ipa{ŋwɤ˧qo˥}}}}{}
\textcolor{teal}{\mytextsc{nom}} \hspace{4pt} Ton~: H\#.
\textcolor{PineGreen}{\selectlanguage{french}Genou.} \zh{膝盖。}  \zh{量词}: \textcolor{darkblue}{\textbf{\ipa{ɭɯ˧}}}  \mytextsc{clf}~: \textcolor{darkblue}{\textbf{\ipa{ɭɯ˧}}} \textit{Voir~:} \hyperlink{}{\textcolor{darkblue}{\textbf{\ipa{ŋwɤ˩ɬv̩˧˥}}}} 
\lhead{\firstmark}
\rhead{\botmark}

\subsection{\hspace{-0.5cm} {\Large \textcolor{darkblue}{\textbf{\ipa{ŋwɤ˧tsʰi˩}}}}\hspace{0.5cm}[\kern2pt{\textcolor{darkblue}{\textbf{\ipa{ŋwɤ˧tsʰi˩}}}}\kern2pt]} \hypertarget{Nw7\string_Mts\string_hi\string_B1}{}
\markboth{\textcolor{darkblue}{\textbf{\ipa{ŋwɤ˧tsʰi˩}}}}{}
\textcolor{teal}{\mytextsc{nombre}} \hspace{4pt} Ton~: L\#.
\textcolor{PineGreen}{\selectlanguage{french}50.} \zh{50。} 
\lhead{\firstmark}
\rhead{\botmark}

\subsection{\hspace{-0.5cm} {\Large \textcolor{darkblue}{\textbf{\ipa{ŋwɤ˩ɭɯ˧-tse˥pʰæ˩}}}}\hspace{0.5cm}[\kern2pt{\textcolor{darkblue}{\textbf{\ipa{ŋwɤ˩ɭɯ˧tse˥pʰæ˩}}}}\kern2pt]} \hypertarget{Nw7\string_Bl\string_RM\string_M-tse\string_Tp\string_h\{\string_B1}{}
\markboth{\textcolor{darkblue}{\textbf{\ipa{ŋwɤ˩ɭɯ˧-tse˥pʰæ˩}}}}{}
\textcolor{teal}{\mytextsc{nom}} \hspace{4pt} Ton~: LM+\#H-.
\textcolor{PineGreen}{\selectlanguage{french}Os du genou.} \zh{膝盖骨。}  \zh{量词}: \textcolor{darkblue}{\textbf{\ipa{ɭɯ˧}}}  \mytextsc{clf}~: \textcolor{darkblue}{\textbf{\ipa{ɭɯ˧}}} 
\lhead{\firstmark}
\rhead{\botmark}

\subsection{\hspace{-0.5cm} {\Large \textcolor{darkblue}{\textbf{\ipa{ŋwɤ˩ɬi˩mi˩}}}}\hspace{0.5cm}[\kern2pt{\textcolor{darkblue}{\textbf{\ipa{ŋwɤ˩ɬi˩mi˩˥}}}}\kern2pt]} \hypertarget{Nw7\string_BKi\string_Bmi\string_B1}{}
\markboth{\textcolor{darkblue}{\textbf{\ipa{ŋwɤ˩ɬi˩mi˩}}}}{}
\textcolor{teal}{\mytextsc{nom}} \hspace{4pt} Ton~: L.
\textcolor{PineGreen}{\selectlanguage{french}5e mois.} \zh{五月。} 
\lhead{\firstmark}
\rhead{\botmark}

\subsection{\hspace{-0.5cm} {\Large \textcolor{darkblue}{\textbf{\ipa{ŋwɤ˩ɬv̩˧˥}}}}\hspace{0.5cm}[\kern2pt{\textcolor{darkblue}{\textbf{\ipa{ŋwɤ˩ɬv̩˧˥}}}}\kern2pt]} \hypertarget{Nw7\string_BKv\string_=\string_M\string_T1}{}
\markboth{\textcolor{darkblue}{\textbf{\ipa{ŋwɤ˩ɬv̩˧˥}}}}{}
\textcolor{teal}{\mytextsc{nom}} \hspace{4pt} Ton~: LM+MH\#.
\textcolor{PineGreen}{\selectlanguage{french}Genou, cartilages du genou, articulation du genou: littéralement “moëlle du genou”. L'expression insiste sur le caractère fragile de cette articulation.} \zh{膝盖(直译:“膝盖髓”)。这个说法强调膝盖的脆弱。}  ¶ \textcolor{darkblue}{\textbf{\ipa{[M23] ŋwɤ˩ɬv̩˧-ko˧lo˥ go˩.}}} \textcolor{PineGreen}{\selectlanguage{french}avoir mal dans le genou} \zh{膝盖疼。}  
 \zh{量词}: \textcolor{darkblue}{\textbf{\ipa{ɭɯ˧}}}  \mytextsc{clf}~: \textcolor{darkblue}{\textbf{\ipa{ɭɯ˧}}} \textit{Voir~:} \hyperlink{}{\textcolor{darkblue}{\textbf{\ipa{ŋwɤ˧qo˥}}}} 
\lhead{\firstmark}
\rhead{\botmark}

\subsection{\hspace{-0.5cm} {\Large \textcolor{darkblue}{\textbf{\ipa{ŋwɤ˧˥}}}}\hspace{0.5cm}[\kern2pt{\textcolor{darkblue}{\textbf{\ipa{ŋwɤ˧˥}}}}\kern2pt]} \hypertarget{Nw7\string_M\string_T1}{}
\markboth{\textcolor{darkblue}{\textbf{\ipa{ŋwɤ˧˥}}}}{}
\textcolor{teal}{\mytextsc{verbe}} \hspace{4pt} Ton~: MH.
\textcolor{PineGreen}{\selectlanguage{french}Percer, piquer.} \zh{刺痛。}  ¶ \textcolor{darkblue}{\textbf{\ipa{tɕʰi˧ ŋwɤ˩-ze˩}}} \textcolor{PineGreen}{\selectlanguage{french}(Elle/il) a pris une écharde} \zh{(他)被刺扎疼了。}  

\lhead{\firstmark}
\rhead{\botmark}

\newpage
\section*{\centering- \textcolor{darkblue}{\textbf{\ipa{õ}}} -}
\subsection{\hspace{-0.5cm} {\Large \textcolor{darkblue}{\textbf{\ipa{õ˧dɤ˧ɻ̍˧}}}}\hspace{0.5cm}[\kern2pt{\textcolor{darkblue}{\textbf{\ipa{õ˧dɤ˧ɻ̍˧}}}}\kern2pt]} \hypertarget{o\string_~\string_Md7\string_Mr£`̍\string_M1}{}
\markboth{\textcolor{darkblue}{\textbf{\ipa{õ˧dɤ˧ɻ̍˧}}}}{}
\textcolor{teal}{\mytextsc{nom}} \hspace{4pt} Ton~: M.
\textcolor{PineGreen}{\selectlanguage{french}Fondement/fondamentalement.} \zh{根本。}  ¶ \textcolor{darkblue}{\textbf{\ipa{õ˧dɤ˧ɻ̍˧-ɳɯ˧, | hĩ˧ ʈʂʰɯ˧-v̩˧ | ʈʂʰɯ˧ne˧ gv̩˧˥ | -ɲi˩!}}} \textcolor{PineGreen}{\selectlanguage{french}Voilà comment il se comporte en réalité/au fond! (Se dit de quelqu'un dont le comportement est irrespectueux des règles de savoir-vivre)} \zh{他原来是这样做事情的! / 他原来这么不懂事!}  

\lhead{\firstmark}
\rhead{\botmark}

\subsection{\hspace{-0.5cm} {\Large \textcolor{darkblue}{\textbf{\ipa{õ˧ʈʂwɤ˧}}}}\hspace{0.5cm}[\kern2pt{\textcolor{darkblue}{\textbf{\ipa{õ˧ʈʂwɤ˧}}}}\kern2pt]} \hypertarget{o\string_~\string_Mt`s`w7\string_M1}{}
\markboth{\textcolor{darkblue}{\textbf{\ipa{õ˧ʈʂwɤ˧}}}}{}
\textcolor{teal}{\mytextsc{nom}} \hspace{4pt} Ton~: M.
\textcolor{PineGreen}{\selectlanguage{french}Moustique.} \zh{蚊子。}  ¶ \textcolor{darkblue}{\textbf{\ipa{õ˧ʈʂwɤ˧ le˧-tʰv̩˧-ze˧!}}} \textcolor{PineGreen}{\selectlanguage{french}voilà un moustique! / un moustique est entré (dans la pièce, sous la moustiquaire…)} \zh{有一只蚊子!}  
 ¶ \textcolor{darkblue}{\textbf{\ipa{ʂɯ˧-ɬi˧mi˧, | õ˧ʈʂwɤ˧! |}}} \textcolor{PineGreen}{\selectlanguage{french}Le septième mois, c'est un mois à moustiques!} \zh{七月份,蚊子多! / 七月份,是蚊子多的一个月!}  
 \zh{量词}: \textcolor{darkblue}{\textbf{\ipa{mi˩}}}  \mytextsc{clf}~: \textcolor{darkblue}{\textbf{\ipa{mi˩}}} 
\lhead{\firstmark}
\rhead{\botmark}

\subsection{\hspace{-0.5cm} {\Large \textcolor{darkblue}{\textbf{\ipa{õ˧ʈʂʰɯ˧ne˧-ʝi˥}}}}\hspace{0.5cm}[\kern2pt{\textcolor{darkblue}{\textbf{\ipa{xxxx non-correspondance entre le nombre de morphèmes et le nombre de tons de morphèmes}}}}\kern2pt]} \hypertarget{o\string_~\string_Mt`s`\string_hM\string_Mne\string_M-j££i\string_T1}{}
\markboth{\textcolor{darkblue}{\textbf{\ipa{õ˧ʈʂʰɯ˧ne˧-ʝi˥}}}}{}
\textcolor{teal}{\mytextsc{adverbe}} \hspace{4pt} Ton~: MH\#.
\textcolor{PineGreen}{\selectlanguage{french}De cette façon-là.} \zh{那样。} 
\lhead{\firstmark}
\rhead{\botmark}

\subsection{\hspace{-0.5cm} {\Large \textcolor{darkblue}{\textbf{\ipa{õ˩dv̩˧˥}}}}\hspace{0.5cm}[\kern2pt{\textcolor{darkblue}{\textbf{\ipa{õ˩dv̩˧˥}}}}\kern2pt]} \hypertarget{o\string_~\string_Bdv\string_=\string_M\string_T1}{}
\markboth{\textcolor{darkblue}{\textbf{\ipa{õ˩dv̩˧˥}}}}{}
\textcolor{teal}{\mytextsc{nom}} \hspace{4pt} Ton~: LM+MH\#.
\textcolor{PineGreen}{\selectlanguage{french}Loup.} \zh{狼。}  \zh{量词}: \textcolor{darkblue}{\textbf{\ipa{mi˩}}}  \mytextsc{clf}~: \textcolor{darkblue}{\textbf{\ipa{mi˩}}} 
\lhead{\firstmark}
\rhead{\botmark}

\subsection{\hspace{-0.5cm} {\Large \textcolor{darkblue}{\textbf{\ipa{õ˩dv̩˧-kʰv̩˥mi˩}}}}\hspace{0.5cm}[\kern2pt{\textcolor{darkblue}{\textbf{\ipa{õ˩dv̩˧˥kʰv̩˧mi˧}}}}\kern2pt]} \hypertarget{o\string_~\string_Bdv\string_=\string_M-k\string_hv\string_=\string_Tmi\string_B1}{}
\markboth{\textcolor{darkblue}{\textbf{\ipa{õ˩dv̩˧-kʰv̩˥mi˩}}}}{}
\textcolor{teal}{\mytextsc{nom}} \hspace{4pt} Ton~: LM+MH\#-.
\textcolor{PineGreen}{\selectlanguage{french}Chien-loup.} \zh{狼狗。}  \zh{量词}: \textcolor{darkblue}{\textbf{\ipa{mi˩}}}  \mytextsc{clf}~: \textcolor{darkblue}{\textbf{\ipa{mi˩}}} 
\lhead{\firstmark}
\rhead{\botmark}

\subsection{\hspace{-0.5cm} {\Large \textcolor{darkblue}{\textbf{\ipa{õ˩dv̩˧-mi˥}}}}\hspace{0.5cm}[\kern2pt{\textcolor{darkblue}{\textbf{\ipa{xxxx non-correspondance entre le nombre de morphèmes et le nombre de tons de morphèmes}}}}\kern2pt]} \hypertarget{o\string_~\string_Bdv\string_=\string_M-mi\string_T1}{}
\markboth{\textcolor{darkblue}{\textbf{\ipa{õ˩dv̩˧-mi˥}}}}{}
\textcolor{teal}{\mytextsc{nom}} \hspace{4pt} Ton~: LM+H\#.
\textcolor{PineGreen}{\selectlanguage{french}Louve.} \zh{母狼。}  \zh{量词}: \textcolor{darkblue}{\textbf{\ipa{mi˩}}}  \mytextsc{clf}~: \textcolor{darkblue}{\textbf{\ipa{mi˩}}} 
\lhead{\firstmark}
\rhead{\botmark}

\subsection{\hspace{-0.5cm} {\Large \textcolor{darkblue}{\textbf{\ipa{õ˩dv̩˧-pʰv̩\#˥}}}}\hspace{0.5cm}[\kern2pt{\textcolor{darkblue}{\textbf{\ipa{xxxx non-correspondance entre le nombre de morphèmes et le nombre de tons de morphèmes}}}}\kern2pt]} \hypertarget{o\string_~\string_Bdv\string_=\string_M-p\string_hv\string_=\#\string_T1}{}
\markboth{\textcolor{darkblue}{\textbf{\ipa{õ˩dv̩˧-pʰv̩\#˥}}}}{}
\textcolor{teal}{\mytextsc{nom}} \hspace{4pt} Ton~: LM+\#H.
\textcolor{PineGreen}{\selectlanguage{french}Loup mâle.} \zh{公狼。}  \zh{量词}: \textcolor{darkblue}{\textbf{\ipa{mi˩}}}  \mytextsc{clf}~: \textcolor{darkblue}{\textbf{\ipa{mi˩}}} 
\lhead{\firstmark}
\rhead{\botmark}

\subsection{\hspace{-0.5cm} {\Large \textcolor{darkblue}{\textbf{\ipa{õ˩dv̩˧-zo\#˥}}}}\hspace{0.5cm}[\kern2pt{\textcolor{darkblue}{\textbf{\ipa{õ˩dv̩˧zo˥}}}}\kern2pt]} \hypertarget{o\string_~\string_Bdv\string_=\string_M-zo\#\string_T1}{}
\markboth{\textcolor{darkblue}{\textbf{\ipa{õ˩dv̩˧-zo\#˥}}}}{}
\textcolor{teal}{\mytextsc{nom}} \hspace{4pt} Ton~: LM+\#H-.
\textcolor{PineGreen}{\selectlanguage{french}Louveteau.} \zh{小狼。} 
\lhead{\firstmark}
\rhead{\botmark}

\subsection{\hspace{-0.5cm} {\Large \textcolor{darkblue}{\textbf{\ipa{õ˧˥}}}}\hspace{0.5cm}[\kern2pt{\textcolor{darkblue}{\textbf{\ipa{õ˧˥}}}}\kern2pt]} \hypertarget{o\string_~\string_M\string_T1}{}
\markboth{\textcolor{darkblue}{\textbf{\ipa{õ˧˥}}}}{}
\textcolor{teal}{\mytextsc{pronom}} \hspace{4pt} Ton~: MH.
\textcolor{PineGreen}{\selectlanguage{french}Soi-même, propre.} \zh{自己。}  ¶ \textcolor{darkblue}{\textbf{\ipa{õ˧-ɑ˥ʁo˩}}} \textcolor{PineGreen}{\selectlanguage{french}sa propre maison} \zh{自己家}  
 ¶ \textcolor{darkblue}{\textbf{\ipa{õ˧-dʑɯ˥, õ˩ ʈʰɯ˩! |}}} \textcolor{PineGreen}{\selectlanguage{french}Chacun boit sa propre boisson! (Contexte: un petit enfant s'empare du biberon d'un autre et s'apprête à boire; on l'en empêche.)} \zh{自己喝自己的!(情景:一个婴儿抓另一个婴儿的奶瓶。)}  
 ¶ \textcolor{darkblue}{\textbf{\ipa{õ˧-ʂe˥, õ˩ ʈʰæ˩! |}}} \textcolor{PineGreen}{\selectlanguage{french}chacun mange son propre morceau de viande! (Description des manières de table: dans le temps, on donnait un bout de viande à chacun et chacun mangeait son morceau, pas comme la coutume chinoise qui veut qu'on prélève bouchée par bouchée, avec ses baguettes, dans les bols/assiettes posés sur la table.)} \zh{自己吃自己的(那块)肉!(关于饮食习惯:吃饭的时候,每人分得一块肉,自己吃完。当地人认为,汉族没有这种分吃的习惯。)}  
 ¶ \textcolor{darkblue}{\textbf{\ipa{õ˧-bv̩˥-õ˩ ʝi˩-ɳɯ˩ | sɯ˧-kv̩˩!}}} \textcolor{PineGreen}{\selectlanguage{french}c'est en faisant soi-même qu'on apprend!} \zh{自己做,就能学会!/ 要学会,就得自己熟练!}  
 ¶ \textcolor{darkblue}{\textbf{\ipa{õ˧-bv̩˥-õ˩ +N |}}} \textcolor{PineGreen}{\selectlanguage{french}son propre N (soi-même+\mytextsc{poss}+soi-même)} \zh{自己的(+名词)}  
 ¶ \textcolor{darkblue}{\textbf{\ipa{õ˧-bv̩˥-õ˩ ʐwæ˩}}} \textcolor{PineGreen}{\selectlanguage{french}son propre cheval} \zh{自己的马}  
 ¶ \textcolor{darkblue}{\textbf{\ipa{õ˧-bv̩˥-õ˩ ʝi˩}}} \textcolor{PineGreen}{\selectlanguage{french}sa propre vache} \zh{自己的牛}  
 ¶ \textcolor{darkblue}{\textbf{\ipa{õ˧-bv̩˥-õ˩ lv̩˩}}} \textcolor{PineGreen}{\selectlanguage{french}son propre champ} \zh{自己的田地}  
 ¶ \textcolor{darkblue}{\textbf{\ipa{õ˧-bv̩˥-õ˩ ɖʐe˩}}} \textcolor{PineGreen}{\selectlanguage{french}son propre argent} \zh{自己的钱}  
 ¶ \textcolor{darkblue}{\textbf{\ipa{õ˧mv̩˥-õ˩di˩}}} \textcolor{PineGreen}{\selectlanguage{french}lieu de naissance, lieu d'origine} \zh{出生的地方、老家、故乡}  
 ¶ \textcolor{darkblue}{\textbf{\ipa{hĩ˧-mv˥ hĩ˩-di˩ | qʰɑ˧-dʑɤ˥\textasciitilde{}dʑɤ˩, | õ˧-mv˥ õ˩-di˩ tsʰe˩ mɤ˩-gv˩!}}} \textcolor{PineGreen}{\selectlanguage{french}Si belles soient les terres d'autrui, elles n'auront jamais la beauté de ses propres terres / de la terre natale !} \zh{其他人的地方怎么好,也比不过自己的地方!}  
 ¶ \textcolor{darkblue}{\textbf{\ipa{õ˧-ə˧mv̩˥ / õ˧-ə˥mv̩˩ / õ˧-ə˧mv̩˧˥}}} \textcolor{PineGreen}{\selectlanguage{french}son propre aîné (frère ou soeur)} \zh{自家姐姐(或哥哥)}  
 ¶ \textcolor{darkblue}{\textbf{\ipa{õ˧-ə˧v̩˥ / õ˧-ə˥v̩˩}}} \textcolor{PineGreen}{\selectlanguage{french}son propre oncle} \zh{自家舅舅(母亲的兄弟)}  
 ¶ \textcolor{darkblue}{\textbf{\ipa{õ˧-ʐɤ˥mi˩, õ˩ ɲi˩! |}}} \textcolor{PineGreen}{\selectlanguage{french}Chacun a son chemin! / Chacun vit sa vie! / A chacun sa destinée!} \zh{自己的道路,就是自己!/ 每个人有自己的命运!}  

\lhead{\firstmark}
\rhead{\botmark}

\newpage
\section*{\centering- \textcolor{darkblue}{\textbf{\ipa{p}}} -}
\subsection{\hspace{-0.5cm} {\Large \textcolor{darkblue}{\textbf{\ipa{pɑ˧tɕɤ˧}}}}\hspace{0.5cm}[\kern2pt{\textcolor{darkblue}{\textbf{\ipa{pɑ˧tɕɤ˥}}}}\kern2pt]} \hypertarget{pA\string_Mts£7\string_M1}{}
\markboth{\textcolor{darkblue}{\textbf{\ipa{pɑ˧tɕɤ˧}}}}{}
\textcolor{teal}{\mytextsc{nom}} \hspace{4pt} Ton~: M.
\textcolor{PineGreen}{\selectlanguage{french}Bananier plantain.} \zh{芭蕉(汉语借词)。}  Emprunt~: chinois  \zh{芭蕉}

\lhead{\firstmark}
\rhead{\botmark}

\subsection{\hspace{-0.5cm} {\Large \textcolor{darkblue}{\textbf{\ipa{pæ˥}}}}\hspace{0.5cm}[\kern2pt{\textcolor{darkblue}{\textbf{\ipa{pæ˥}}}}\kern2pt]} \hypertarget{p\{\string_T1}{}
\markboth{\textcolor{darkblue}{\textbf{\ipa{pæ˥}}}}{}
\textcolor{teal}{\mytextsc{verbe}} \hspace{4pt} Ton~: H.
\textcolor{PineGreen}{\selectlanguage{french}Déménager.} \zh{搬(家)。}  Emprunt~: chinois  \zh{搬?}

\lhead{\firstmark}
\rhead{\botmark}

\subsection{\hspace{-0.5cm} {\Large \textcolor{darkblue}{\textbf{\ipa{pæ˥\textsubscript{a}}}}}\hspace{0.5cm}[\kern2pt{\textcolor{darkblue}{\textbf{\ipa{pæ˥}}}}\kern2pt]} \hypertarget{p\{\string_Ta1}{}
\markboth{\textcolor{darkblue}{\textbf{\ipa{pæ˥\textsubscript{a}}}}}{}
\textcolor{teal}{\mytextsc{classificateur}} \hspace{4pt} Ton~: H\textsubscript{a}.
\textcolor{PineGreen}{\selectlanguage{french}Troupe (de chevaux, de soldats…).} \zh{量词:马、军人……(一队)。} 
\lhead{\firstmark}
\rhead{\botmark}

\subsection{\hspace{-0.5cm} {\Large \textcolor{darkblue}{\textbf{\ipa{pæ˧kʰwɤ\#˥}}}}\hspace{0.5cm}[\kern2pt{\textcolor{darkblue}{\textbf{\ipa{pæ˧kʰwɤ˧˥}}}}\kern2pt]} \hypertarget{p\{\string_Mk\string_hw7\#\string_T1}{}
\markboth{\textcolor{darkblue}{\textbf{\ipa{pæ˧kʰwɤ\#˥}}}}{}
\textcolor{teal}{\mytextsc{nom}} \hspace{4pt} Ton~: \#H.
\textcolor{PineGreen}{\selectlanguage{french}Pièce d'argent de l'époque impériale.} \zh{民国之前的银币。}  ¶ \textcolor{darkblue}{\textbf{\ipa{ə˧mi˧! | pæ˧kʰwɤ˧ so˧-ɭɯ˥ ki˩-mæ˩!}}} \textcolor{PineGreen}{\selectlanguage{french}Waouuu! [Il/elle] te donne trois pièces d'argent! (D'après le souvenir qu'en a la consultante principale, c'est le type de commentaire que faisaient autrefois les tantes ou oncles d'un enfant à qui on offrait une forte somme d'argent à l'occasion de son passage à l'âge adulte, à treize ans. Cela correspondrait aujourd'hui à la moitié d'un mois de salaire. Donner une seule pièce, c'est symboliquement inapproprié: on offre par paires. Donner deux pièces, c'est un beau cadeau, approprié et suffisant. Donner trois pièces, c'est un cadeau considérable, qui dépasse les attentes.)} \zh{哇!(他)给三块银币!(在一个孩子成年时,亲戚会给银币。给一块,不合适,因为礼物不能只给一个,要给两个。给两块银币,是合适的,也是够的。给三块银币,超出期望,是大礼物了。按现在的标准/说法,三个银币等于半个月的工资左右。)}  
 ¶ \textcolor{darkblue}{\textbf{\ipa{pæ˧kʰwɤ˧ ɖɯ˧-ɭɯ˥\# ; pæ˧kʰwɤ˧ ɲi˧-ɭɯ˥\# ; pæ˧kʰwɤ˧ so˧-ɭɯ˥\#}}} \textcolor{PineGreen}{\selectlanguage{french}une pièce d'argent; deux pièces d'argent; trois pièces d'argent} \zh{一块银币,两块银币,三块银币}  
 ¶ \textcolor{darkblue}{\textbf{\ipa{pæ˧kʰwɤ˧ ɖɯ˧-ki˩tɑ˩}}} \textcolor{PineGreen}{\selectlanguage{french}un sac de pièces d'argent, destiné à être caché/enterré} \zh{一包银币(埋在地里,为了藏)}  
 \zh{量词}: \textcolor{darkblue}{\textbf{\ipa{ɭɯ˧}}}  \mytextsc{clf}~: \textcolor{darkblue}{\textbf{\ipa{ɭɯ˧}}} 
\lhead{\firstmark}
\rhead{\botmark}

\subsection{\hspace{-0.5cm} {\Large \textcolor{darkblue}{\textbf{\ipa{pæ˧li˩}}}}\hspace{0.5cm}[\kern2pt{\textcolor{darkblue}{\textbf{\ipa{pæ˧li˧}}}}\kern2pt]} \hypertarget{p\{\string_Mli\string_B1}{}
\markboth{\textcolor{darkblue}{\textbf{\ipa{pæ˧li˩}}}}{}
\textcolor{teal}{\mytextsc{nom}} \hspace{4pt} Ton~: L\#.
\textcolor{PineGreen}{\selectlanguage{french}Châtaigne.} \zh{板栗。}  Emprunt~: chinois  \zh{板栗}
 ¶ \textcolor{darkblue}{\textbf{\ipa{pæ˧li˩-si˩dzi˩}}} \textcolor{PineGreen}{\selectlanguage{french}châtaignier} \zh{板栗树}  
 ¶ \textcolor{darkblue}{\textbf{\ipa{pæ˧li˩-dzi˩}}} \textcolor{PineGreen}{\selectlanguage{french}châtaignier} \zh{板栗树}  

\lhead{\firstmark}
\rhead{\botmark}

\subsection{\hspace{-0.5cm} {\Large \textcolor{darkblue}{\textbf{\ipa{pæ˧ɻæ˩-ʈʂʰo˩}}}}\hspace{0.5cm}[\kern2pt{\textcolor{darkblue}{\textbf{\ipa{xxxx non-correspondance entre le nombre de morphèmes et le nombre de tons de morphèmes}}}}\kern2pt]} \hypertarget{p\{\string_Mr£`\{\string_B-t`s`\string_ho\string_B1}{}
\markboth{\textcolor{darkblue}{\textbf{\ipa{pæ˧ɻæ˩-ʈʂʰo˩}}}}{}
\textcolor{teal}{\mytextsc{nom}} \hspace{4pt} Ton~: L\#-.
\textcolor{PineGreen}{\selectlanguage{french}Hongqiao, village sur la route entre Ninglang et Yongning (principalement peuplé de Chinois Han).} \zh{红桥。}  ¶ \textcolor{darkblue}{\textbf{\ipa{no˧ | pæ˧ɻæ˩ʈʂʰo˩-hĩ˩-ni˩-zo˩!}}} \textcolor{PineGreen}{\selectlanguage{french}“Tu ressembles à quelqu'un de Hongqiao!” Insulte, pour dire de quelqu'un qu'il a un physique disgracieux. La géographie populaire na attribuait des traits grossiers aux gens de Hongqiao (localité que traversaient les caravanes): gros nez camus, en particulier.} \zh{解放前用的侮辱语句:“你像红桥人!”=“你很丑!”摩梭民间文化中,红桥(马帮路过的一个乡)的人被认为难看,面貌不“眉清目秀”,比如有扁鼻子。}  

\lhead{\firstmark}
\rhead{\botmark}

\subsection{\hspace{-0.5cm} {\Large \textcolor{darkblue}{\textbf{\ipa{pæ˧sɯ˧}}}}\hspace{0.5cm}[\kern2pt{\textcolor{darkblue}{\textbf{\ipa{xxxx non-correspondance entre le nombre de morphèmes et le nombre de tons de morphèmes}}}}\kern2pt]} \hypertarget{p\{\string_MsM\string_M1}{}
\markboth{\textcolor{darkblue}{\textbf{\ipa{pæ˧sɯ˧}}}}{}
\textcolor{teal}{\mytextsc{nom}} \hspace{4pt} Ton~: M.
\textcolor{PineGreen}{\selectlanguage{french}Rang (le plus bas) dans la hiérarchie des fonctionnaires féodaux.} \zh{把事(封建官员系统中的最低等级)(汉语借词)。}  Emprunt~: chinois  \zh{把事}

\lhead{\firstmark}
\rhead{\botmark}

\subsection{\hspace{-0.5cm} {\Large \textcolor{darkblue}{\textbf{\ipa{pæ˧te˩}}}}\hspace{0.5cm}[\kern2pt{\textcolor{darkblue}{\textbf{\ipa{pæ˧te˧}}}}\kern2pt]} \hypertarget{p\{\string_Mte\string_B1}{}
\markboth{\textcolor{darkblue}{\textbf{\ipa{pæ˧te˩}}}}{}
\textcolor{teal}{\mytextsc{nom}} \hspace{4pt} Ton~: L\#.
\textcolor{PineGreen}{\selectlanguage{french}Banc, tabouret.} \zh{板凳。}  Emprunt~: chinois  \zh{板凳}
 \zh{量词}: \textcolor{darkblue}{\textbf{\ipa{ɭɯ˧}}}  \mytextsc{clf}~: \textcolor{darkblue}{\textbf{\ipa{ɭɯ˧}}} 
\lhead{\firstmark}
\rhead{\botmark}

\subsection{\hspace{-0.5cm} {\Large \textcolor{darkblue}{\textbf{\ipa{pæ˩\textsubscript{a}}}}}\hspace{0.5cm}[\kern2pt{\textcolor{darkblue}{\textbf{\ipa{pæ˥}}}}\kern2pt]} \hypertarget{p\{\string_Ba1}{}
\markboth{\textcolor{darkblue}{\textbf{\ipa{pæ˩\textsubscript{a}}}}}{}
\textcolor{teal}{\mytextsc{verbe}} \hspace{4pt} Ton~: M\textsubscript{a}.
\textcolor{PineGreen}{\selectlanguage{french}Mettre (la table), servir.} \zh{摆桌子、供应饭菜。}  Emprunt~: chinois  \zh{摆?}
 ¶ \textcolor{darkblue}{\textbf{\ipa{hɑ˧ tʰi˧-pæ˩ tsæ˩-ɲi˩-ze˩! | hɑ˧ dzɯ˧-bi˧-ze˩!}}} \textcolor{PineGreen}{\selectlanguage{french}C'est servi! A table!} \zh{饭摆好了!吃饭了!}  

\lhead{\firstmark}
\rhead{\botmark}

\subsection{\hspace{-0.5cm} {\Large \textcolor{darkblue}{\textbf{\ipa{pæ˩pʰæ˧˥}}} \textsubscript{1}}\hspace{0.5cm}[\kern2pt{\textcolor{darkblue}{\textbf{\ipa{pæ˧pʰæ˩}}}}\kern2pt]} \hypertarget{p\{\string_Bp\string_h\{\string_M\string_T1}{}
\markboth{\textcolor{darkblue}{\textbf{\ipa{pæ˩pʰæ˧˥}}} \textsubscript{1}}{}
\textcolor{teal}{\mytextsc{nom}} \hspace{4pt} Ton~: LM+MH\#.
\ding{202} \textcolor{PineGreen}{\selectlanguage{french}Grosse planche de bois, épaisse d'une dizaine de centimètres, utilisée pour la charpente des maisons.} \zh{厚的木板、 木板子。}  \zh{量词}: \textcolor{darkblue}{\textbf{\ipa{pʰæ˧˥}}} \ding{203} \textcolor{PineGreen}{\selectlanguage{french}Herse en bois, qui consiste essentiellement en une grosse pièce de bois, sans dents, d'où l'emploi (par extension) du terme qui signifie “planche”.} \zh{耙。}  \mytextsc{clf}~: \textcolor{darkblue}{\textbf{\ipa{pʰæ˧˥}}} 
\lhead{\firstmark}
\rhead{\botmark}

\subsection{\hspace{-0.5cm} {\Large \textcolor{darkblue}{\textbf{\ipa{pæ˩pʰæ˧˥}}} \textsubscript{2}}\hspace{0.5cm}[\kern2pt{\textcolor{darkblue}{\textbf{\ipa{pæ˩pʰæ˧˥}}}}\kern2pt]} \hypertarget{p\{\string_Bp\string_h\{\string_M\string_T2}{}
\markboth{\textcolor{darkblue}{\textbf{\ipa{pæ˩pʰæ˧˥}}} \textsubscript{2}}{}
\textcolor{teal}{\mytextsc{nom}} \hspace{4pt} Ton~: LM+MH\#.
\textcolor{PineGreen}{\selectlanguage{french}Prénom masculin.} \zh{男性名字。} 
\lhead{\firstmark}
\rhead{\botmark}

\subsection{\hspace{-0.5cm} {\Large \textcolor{darkblue}{\textbf{\ipa{pæ˧˥}}} \textsubscript{1}}\hspace{0.5cm}[\kern2pt{\textcolor{darkblue}{\textbf{\ipa{pæ˧˥}}}}\kern2pt]} \hypertarget{p\{\string_M\string_T1}{}
\markboth{\textcolor{darkblue}{\textbf{\ipa{pæ˧˥}}} \textsubscript{1}}{}
\textcolor{teal}{\mytextsc{verbe}} \hspace{4pt} Ton~: MH.
\textcolor{PineGreen}{\selectlanguage{french}Cultiver (une terre).} \zh{种(地)。} 
\lhead{\firstmark}
\rhead{\botmark}

\subsection{\hspace{-0.5cm} {\Large \textcolor{darkblue}{\textbf{\ipa{pæ˧˥}}} \textsubscript{2}}\hspace{0.5cm}[\kern2pt{\textcolor{darkblue}{\textbf{\ipa{pæ˧˥}}}}\kern2pt]} \hypertarget{p\{\string_M\string_T2}{}
\markboth{\textcolor{darkblue}{\textbf{\ipa{pæ˧˥}}} \textsubscript{2}}{}
\textcolor{teal}{\mytextsc{verbe}} \hspace{4pt} Ton~: MH.
\textcolor{PineGreen}{\selectlanguage{french}Dépasser, outrepasser; laisser passer (une occasion).} \zh{超过,错过。}  ¶ \textcolor{darkblue}{\textbf{\ipa{pæ˧˥ | -kʰɯ˩-pi˩, | mɤ˧-tsɤ˧! |}}} \textcolor{PineGreen}{\selectlanguage{french}Ce n'est pas bien de laisser passer (un jour propice: pour la construction d'une maison, par exemple)!} \zh{错过(一个吉日),不好!}  
 ¶ \textcolor{darkblue}{\textbf{\ipa{pæ˧˥ | -tʰɑ˧-kʰɯ˩}}} \textcolor{PineGreen}{\selectlanguage{french}Il ne faut pas laisser passer/filer (une occasion/un moment propice)!} \zh{不要错过(机会)!}  
 ¶ \textcolor{darkblue}{\textbf{\ipa{le˧-pæ˧-ze˥!}}} \textcolor{PineGreen}{\selectlanguage{french}(On) a laissé filer (une occasion)/ c'est passé, c'est trop tard!} \zh{错过了!}  

\lhead{\firstmark}
\rhead{\botmark}

\subsection{\hspace{-0.5cm} {\Large \textcolor{darkblue}{\textbf{\ipa{pæ˧˥hwɤ˧}}}}\hspace{0.5cm}[\kern2pt{\textcolor{darkblue}{\textbf{\ipa{pæ˧hwɤ˧}}}}\kern2pt]} \hypertarget{p\{\string_M\string_Thw7\string_M1}{}
\markboth{\textcolor{darkblue}{\textbf{\ipa{pæ˧˥hwɤ˧}}}}{}
\textcolor{teal}{\mytextsc{nom}} \hspace{4pt} Ton~: MH.M.
\textcolor{PineGreen}{\selectlanguage{french}Solution, méthode (emprunt chinois ancien).} \zh{办法(早期汉语借词)。}  Emprunt~: chinois  \zh{办法}
 ¶ \textcolor{darkblue}{\textbf{\ipa{ʈʂʰɯ˧ | pæ˧˥hwɤ˧ | ɕjɤ˩ ɣɯ˧ (+ | ʐwæ˩˥)!}}} \textcolor{PineGreen}{\selectlanguage{french}Il/elle excelle à trouver des solutions/ il a une solution à tout!} \zh{他很会想办法的!}  
 \zh{量词}: \textcolor{darkblue}{\textbf{\ipa{kʰwɤ˥}}}  \mytextsc{clf}~: \textcolor{darkblue}{\textbf{\ipa{kʰwɤ˥}}} 
\lhead{\firstmark}
\rhead{\botmark}

\subsection{\hspace{-0.5cm} {\Large \textcolor{darkblue}{\textbf{\ipa{pe˧ʂe˧}}}}\hspace{0.5cm}[\kern2pt{\textcolor{darkblue}{\textbf{\ipa{pe˧ʂe˧}}}}\kern2pt]} \hypertarget{pe\string_Ms`e\string_M1}{}
\markboth{\textcolor{darkblue}{\textbf{\ipa{pe˧ʂe˧}}}}{}
\textcolor{teal}{\mytextsc{adverbe}} \hspace{4pt} Ton~: M.
\textcolor{PineGreen}{\selectlanguage{french}En soi.} \zh{本身(汉语借词)。}  Emprunt~: chinois  \zh{本身}

\lhead{\firstmark}
\rhead{\botmark}

\subsection{\hspace{-0.5cm} {\Large \textcolor{darkblue}{\textbf{\ipa{pɤ˥}}}}\hspace{0.5cm}[\kern2pt{\textcolor{darkblue}{\textbf{\ipa{pɤ˥}}}}\kern2pt]} \hypertarget{p7\string_T1}{}
\markboth{\textcolor{darkblue}{\textbf{\ipa{pɤ˥}}}}{}
\textcolor{teal}{\mytextsc{nom}} \hspace{4pt} Ton~: \#H.
\textcolor{PineGreen}{\selectlanguage{french}Dessin, peinture.} \zh{画。}  \zh{量词}: \textcolor{darkblue}{\textbf{\ipa{pɤ˥}}} \textcolor{darkblue}{\textbf{\ipa{pʰæ˧˥}}}  \mytextsc{clf}~: \textcolor{darkblue}{\textbf{\ipa{pɤ˥}}} \textcolor{darkblue}{\textbf{\ipa{pʰæ˧˥}}} 
\lhead{\firstmark}
\rhead{\botmark}

\subsection{\hspace{-0.5cm} {\Large \textcolor{darkblue}{\textbf{\ipa{pɤ˥}}}}\hspace{0.5cm}[\kern2pt{\textcolor{darkblue}{\textbf{\ipa{pɤ˥}}}}\kern2pt]} \hypertarget{p7\string_T1}{}
\markboth{\textcolor{darkblue}{\textbf{\ipa{pɤ˥}}}}{}
\textcolor{teal}{\mytextsc{verbe}} \hspace{4pt} Ton~: H.
\textcolor{PineGreen}{\selectlanguage{french}S'accroupir, se mettre en boule, se recroqueviller sur soi-même.} \zh{蜷曲、蜷缩。}  ¶ \textcolor{darkblue}{\textbf{\ipa{æ˩ ʈʂʰɯ˧-mi˥ | si˧dzi˩-ʈʰæ˩qo˩ | tʰi˧-pɤ˥-dʑo˩!}}} \textcolor{PineGreen}{\selectlanguage{french}La poule est recroquevillée sous l'arbre/est accroupie sous l'arbre!} \zh{那只鸡,在树下蜷缩着!}  
 ¶ \textcolor{darkblue}{\textbf{\ipa{ʈʂʰɯ˧-qo˧ ɖɯ˧-pɤ˥ ɕjɤ˩-ɻ̍˩!}}} \textcolor{PineGreen}{\selectlanguage{french}Viens t'allonger par ici (pour te reposer)!} \zh{过来这边躺一下!}  

\lhead{\firstmark}
\rhead{\botmark}

\subsection{\hspace{-0.5cm} {\Large \textcolor{darkblue}{\textbf{\ipa{pɤ˥\textsubscript{b}}}}}\hspace{0.5cm}[\kern2pt{\textcolor{darkblue}{\textbf{\ipa{pɤ˥}}}}\kern2pt]} \hypertarget{p7\string_Tb1}{}
\markboth{\textcolor{darkblue}{\textbf{\ipa{pɤ˥\textsubscript{b}}}}}{}
\textcolor{teal}{\mytextsc{classificateur}} \hspace{4pt} Ton~: H\textsubscript{b}.
\textcolor{PineGreen}{\selectlanguage{french}Classificateur des images, peintures….} \zh{量词:雕像,如:佛像(一尊)。} 
\lhead{\firstmark}
\rhead{\botmark}

\subsection{\hspace{-0.5cm} {\Large \textcolor{darkblue}{\textbf{\ipa{pɤ˧\textsubscript{a}}}}}\hspace{0.5cm}[\kern2pt{\textcolor{darkblue}{\textbf{\ipa{pɤ˥}}}}\kern2pt]} \hypertarget{p7\string_Ma1}{}
\markboth{\textcolor{darkblue}{\textbf{\ipa{pɤ˧\textsubscript{a}}}}}{}
\textcolor{teal}{\mytextsc{verbe}} \hspace{4pt} Ton~: M\textsubscript{a}.
\textcolor{PineGreen}{\selectlanguage{french}Porter sur son dos (le bois, …).} \zh{背(水、柴、孩子……)。}  ¶ \textcolor{darkblue}{\textbf{\ipa{pɤ˧\textasciitilde{}pɤ˧}}} \textcolor{PineGreen}{\selectlanguage{french}\mytextsc{red}} \zh{\mytextsc{重叠:背一背}}  
 ¶ \textcolor{darkblue}{\textbf{\ipa{tʰi˧-pɤ˥\textasciitilde{}pɤ˩}}} \textcolor{PineGreen}{\selectlanguage{french}\mytextsc{dur} \mytextsc{red}} \zh{背一背}  
 ¶ \textcolor{darkblue}{\textbf{\ipa{qʰæ˧ pɤ˧\textasciitilde{}pɤ˥}}} \textcolor{PineGreen}{\selectlanguage{french}porter des engrais/ du fumier} \zh{背肥料}  
 ¶ \textcolor{darkblue}{\textbf{\ipa{kʰɤ˧ pɤ˧\textasciitilde{}pɤ˥}}} \textcolor{PineGreen}{\selectlanguage{french}porter un panier dorsal} \zh{背背篓}  
 ¶ \textcolor{darkblue}{\textbf{\ipa{zɯ˧ pɤ˧\textasciitilde{}pɤ˥}}} \textcolor{PineGreen}{\selectlanguage{french}porter de l'herbe} \zh{背草}  
 ¶ \textcolor{darkblue}{\textbf{\ipa{tso˧\textasciitilde{}tso˧ pɤ˧\textasciitilde{}pɤ˥}}} \textcolor{PineGreen}{\selectlanguage{french}porter des choses} \zh{背东西}  
 ¶ \textcolor{darkblue}{\textbf{\ipa{*tso˧\textasciitilde{}tso˧ pɤ˩}}} \textcolor{PineGreen}{\selectlanguage{french}porter des choses (l'expression est bien formée, mais pas usitée)} \zh{背东西(语法上,这个短语没有问题,但发音合作人不那么说。)}  
 ¶ \textcolor{darkblue}{\textbf{\ipa{njɤ˧-ɳɯ˧ pɤ˧\textasciitilde{}pɤ˩ (+bi˩)!}}} \textcolor{PineGreen}{\selectlanguage{french}c'est moi qui porte!} \zh{我来背!}  
 ¶ \textcolor{darkblue}{\textbf{\ipa{dʑɯ˩ pɤ˩\textasciitilde{}pɤ˥}}} \textcolor{PineGreen}{\selectlanguage{french}porter de l'eau} \zh{背水}  
 ¶ \textcolor{darkblue}{\textbf{\ipa{zo˧mv̩˥ pɤ˩\textasciitilde{}pɤ˩}}} \textcolor{PineGreen}{\selectlanguage{french}porter un enfant sur le dos} \zh{背孩子}  

\lhead{\firstmark}
\rhead{\botmark}

\subsection{\hspace{-0.5cm} {\Large \textcolor{darkblue}{\textbf{\ipa{pɤ˧dʑɤ˩-di˩}}}}\hspace{0.5cm}[\kern2pt{\textcolor{darkblue}{\textbf{\ipa{pɤ˧dʑɤ˩di˧}}}}\kern2pt]} \hypertarget{p7\string_Mdz£7\string_B-di\string_B1}{}
\markboth{\textcolor{darkblue}{\textbf{\ipa{pɤ˧dʑɤ˩-di˩}}}}{}
\textcolor{teal}{\mytextsc{nom}} \hspace{4pt} Ton~: L\#-.
\textcolor{PineGreen}{\selectlanguage{french}Un village proche des Sources Chaudes.} \zh{温泉乡的一个村落。}  ¶ \textcolor{darkblue}{\textbf{\ipa{ə˧go˧-ʁwɤ˧, | ʁwɤ˧lɑ˩-bi˩, | bæ˧ʁwɤ˧, | tʰo˧tsʰe\#˥, | pi˧tsʰe˩-di˩, | pɤ˧dʑɤ˩-di˩, | ʁwɤ˧tv̩˧}}} \textcolor{PineGreen}{\selectlanguage{french}Villages au sortir de la plaine de Yongning; les deux premiers comportent une population na; le troisième est un village na; les suivants sont essentiellement des villages pumi/prinmi.} \zh{永宁背向泸沽湖方向经过的村落。前两个村落拥有相当大的摩梭人口比例,第三个村落是摩梭村,最后一个是普米村。}  

\lhead{\firstmark}
\rhead{\botmark}

\subsection{\hspace{-0.5cm} {\Large \textcolor{darkblue}{\textbf{\ipa{pɤ˧lɑ˩}}}}\hspace{0.5cm}[\kern2pt{\textcolor{darkblue}{\textbf{\ipa{xxxx non-correspondance entre le nombre de morphèmes et le nombre de tons de morphèmes}}}}\kern2pt]} \hypertarget{p7\string_MlA\string_B1}{}
\markboth{\textcolor{darkblue}{\textbf{\ipa{pɤ˧lɑ˩}}}}{}
\textcolor{teal}{\mytextsc{nom}} \hspace{4pt} Ton~: L\#.
\textcolor{PineGreen}{\selectlanguage{french}Photo, photographie (néologisme).} \zh{相片,照片。}  \zh{量词}: \textcolor{darkblue}{\textbf{\ipa{pʰæ˧˥}}}  \mytextsc{clf}~: \textcolor{darkblue}{\textbf{\ipa{pʰæ˧˥}}} 
\lhead{\firstmark}
\rhead{\botmark}

\subsection{\hspace{-0.5cm} {\Large \textcolor{darkblue}{\textbf{\ipa{pɤ˧ʁɑ˧}}}}\hspace{0.5cm}[\kern2pt{\textcolor{darkblue}{\textbf{\ipa{pɤ˧ʁɑ˧}}}}\kern2pt]} \hypertarget{p7\string_MRA\string_M1}{}
\markboth{\textcolor{darkblue}{\textbf{\ipa{pɤ˧ʁɑ˧}}}}{}
\textcolor{teal}{\mytextsc{classificateur}} \hspace{4pt} Ton~: M.
\textcolor{PineGreen}{\selectlanguage{french}Un grand pas.} \zh{量词:一大步。}  ¶ \textcolor{darkblue}{\textbf{\ipa{ɖɯ˧-pɤ˧ʁɑ˧\textasciitilde{}ɖɯ˧-pɤ˧ʁɑ˧}}} \textcolor{PineGreen}{\selectlanguage{french}à grands pas} \zh{大步流星地}  
 ¶ \textcolor{darkblue}{\textbf{\ipa{ɲi˧-pɤ˧ʁɑ˧}}} \textcolor{PineGreen}{\selectlanguage{french}deux grandes enjambées} \zh{两大步}  

\lhead{\firstmark}
\rhead{\botmark}

\subsection{\hspace{-0.5cm} {\Large \textcolor{darkblue}{\textbf{\ipa{‑pɤ˧to˩}}}}\hspace{0.5cm}[\kern2pt{\textcolor{darkblue}{\textbf{\ipa{pɤ˧to˩}}}}\kern2pt]} \hypertarget{‑p7\string_Mto\string_B1}{}
\markboth{\textcolor{darkblue}{\textbf{\ipa{‑pɤ˧to˩}}}}{}
\textcolor{teal}{\mytextsc{conjonction}} \hspace{4pt} Ton~: L\#.
\textcolor{PineGreen}{\selectlanguage{french}Même.} \zh{连。}  ¶ \textcolor{darkblue}{\textbf{\ipa{ʈʂʰɯ˧ | li˩-pɤ˥to˩ | ʈʰɯ˩-ɲi˥!}}} \textcolor{PineGreen}{\selectlanguage{french}Elle boit même du thé! (au sujet de l'alimentation d'un enfant d'un an)} \zh{她连茶都喝!(关于一个一岁孩子的饮食习惯)}  
 ¶ \textcolor{darkblue}{\textbf{\ipa{ʈʂʰɯ˧ | pɤ˩jɤ˧-pɤ˥to˩ | dzɯ˩-ɲi˥!}}} \textcolor{PineGreen}{\selectlanguage{french}Elle mange même du pain! (au sujet de l'alimentation d'un enfant d'un an)} \zh{她连面包都吃!}  
 ¶ \textcolor{darkblue}{\textbf{\ipa{hæ˧, | kʰv̩˩mi˩-ʂe˩-pɤ˥to˩ dzɯ˩-kv̩˩!}}} \textcolor{PineGreen}{\selectlanguage{french}les Chinois, ils mangent même du chien! (Note: l'un des interdits alimentaires na concerne la viande de chien, le chien étant un animal sacré dans la culture na.)} \zh{汉族连狗肉都吃!(注:摩梭人不吃狗肉)}  
 ¶ \textcolor{darkblue}{\textbf{\ipa{hæ˧, | kʰv̩˩mi˩-ʂe˩˥ F dzɯ˩-kv̩˩!}}} \textcolor{PineGreen}{\selectlanguage{french}même sens} \zh{同上}  
 ¶ \textcolor{darkblue}{\textbf{\ipa{bo˩-pɤ˥to˩; lɑ˧-pɤ˧to˩; mv̩˩-pɤ˥to˩; ʐwæ˧-pɤ˧to˩; ɬi˧mi˧-pɤ˧to˩; ɲi˧mi˧-pɤ˧to˩; hwɤ˧li˧-pɤ˥-to˩; hwɤ˧mi˧-pɤ˥to˩; kʰv̩˩mi˩-pɤ˥-to˩; ʁo˧dzi˩-pɤ˩to˩; ʝi˩ʈʂæ˧-pɤ˥to˩; nɑ˩hĩ˧-pɤ˧to˩; bo˩mi˧-pɤ˧to˩; bo˩ɬɑ˧-pɤ˩to˩; ʁæ˧ʈv̩˥-pɤ˩to˩}}} \textcolor{PineGreen}{\selectlanguage{french}en association avec des noms des diverses catégories tonales} \zh{与不同声调类的名词结合}  

\lhead{\firstmark}
\rhead{\botmark}

\subsection{\hspace{-0.5cm} {\Large \textcolor{darkblue}{\textbf{\ipa{pɤ˧tv̩˥}}}}\hspace{0.5cm}[\kern2pt{\textcolor{darkblue}{\textbf{\ipa{pɤ˧tv̩˥}}}}\kern2pt]} \hypertarget{p7\string_Mtv\string_=\string_T1}{}
\markboth{\textcolor{darkblue}{\textbf{\ipa{pɤ˧tv̩˥}}}}{}
\textcolor{teal}{\mytextsc{nom}} \hspace{4pt} Ton~: H\#.
\textcolor{PineGreen}{\selectlanguage{french}Panier de vannerie.} \zh{篮子、竹篮。}  ¶ \textcolor{darkblue}{\textbf{\ipa{ɖʐɯ˧ʂɯ˥-pɤ˩tv̩˩}}} \textcolor{PineGreen}{\selectlanguage{french}panier (traditionnellement: en vannerie) dans lequel on range les baguettes} \zh{筷子篮}  

\lhead{\firstmark}
\rhead{\botmark}

\subsection{\hspace{-0.5cm} {\Large \textcolor{darkblue}{\textbf{\ipa{pɤ˧tʰi˩}}}}\hspace{0.5cm}[\kern2pt{\textcolor{darkblue}{\textbf{\ipa{pɤ˧tʰi˩}}}}\kern2pt]} \hypertarget{p7\string_Mt\string_hi\string_B1}{}
\markboth{\textcolor{darkblue}{\textbf{\ipa{pɤ˧tʰi˩}}}}{}
\textcolor{teal}{\mytextsc{nom}} \hspace{4pt} Ton~: L\#.
\textcolor{PineGreen}{\selectlanguage{french}Nom de clan/famille étendue. Deux familles portent ce nom à Yongning.} \zh{一个姓。这个姓,永宁有两家。}  ¶ \textcolor{darkblue}{\textbf{\ipa{pɤ˧tʰi˩=ɻ̍˩}}} \textcolor{PineGreen}{\selectlanguage{french}le clan \textcolor{darkblue}{\textbf{\ipa{/pɤ˧tʰi˩/}}}, la famille \textcolor{darkblue}{\textbf{\ipa{/pɤ˧tʰi˩/}}}} \zh{\textcolor{darkblue}{\textbf{\ipa{/pɤ˧tʰi˩/}}}家族}  

\lhead{\firstmark}
\rhead{\botmark}

\subsection{\hspace{-0.5cm} {\Large \textcolor{darkblue}{\textbf{\ipa{pɤ˩\textsubscript{a}}}}}\hspace{0.5cm}[\kern2pt{\textcolor{darkblue}{\textbf{\ipa{pɤ˩˥}}}}\kern2pt]} \hypertarget{p7\string_Ba1}{}
\markboth{\textcolor{darkblue}{\textbf{\ipa{pɤ˩\textsubscript{a}}}}}{}
\textcolor{teal}{\mytextsc{verbe}} \hspace{4pt} Ton~: L\textsubscript{a}.
\textcolor{PineGreen}{\selectlanguage{french}Sortir, émerger, apparaître.} \zh{出现、出来、浮现。}  ¶ \textcolor{darkblue}{\textbf{\ipa{dʑɯ˩ pɤ˩˥}}} \textcolor{PineGreen}{\selectlanguage{french}de l'eau sort} \zh{涌出水来}  
 ¶ \textcolor{darkblue}{\textbf{\ipa{dʑɯ˧qʰv̩˧-qo˧ | dʑɯ˩ pɤ˩-ze˥}}} \textcolor{PineGreen}{\selectlanguage{french}De l'eau apparaît à la source / de l'eau coule à la source} \zh{水泉里面,涌出水来。}  
 ¶ \textcolor{darkblue}{\textbf{\ipa{tʰi˧-pɤ˩-dʑo˩}}} \textcolor{PineGreen}{\selectlanguage{french}\mytextsc{dur} \string_ \mytextsc{prog}: ça sort, ça coule, ça émerge (ex.: de l'eau de source)} \zh{正在涌出水来}  
 ¶ \textcolor{darkblue}{\textbf{\ipa{gɤ˩-pɤ˥}}} \textcolor{PineGreen}{\selectlanguage{french}émerger, se lever: le soleil se lève} \zh{出现、上来:太阳出来}  

\lhead{\firstmark}
\rhead{\botmark}

\subsection{\hspace{-0.5cm} {\Large \textcolor{darkblue}{\textbf{\ipa{pɤ˩\textsubscript{b}}}}}\hspace{0.5cm}[\kern2pt{\textcolor{darkblue}{\textbf{\ipa{pɤ˩˥}}}}\kern2pt]} \hypertarget{p7\string_Bb1}{}
\markboth{\textcolor{darkblue}{\textbf{\ipa{pɤ˩\textsubscript{b}}}}}{}
\textcolor{teal}{\mytextsc{classificateur}} \hspace{4pt} Ton~: L\textsubscript{b}.
\textcolor{PineGreen}{\selectlanguage{french}Classificateur des éléments de menuiserie/charpente: échelles, portes….} \zh{量词:木工件,如梯子、门等等(一扇门,一把梯子)。} 
\lhead{\firstmark}
\rhead{\botmark}

\subsection{\hspace{-0.5cm} {\Large \textcolor{darkblue}{\textbf{\ipa{pɤ˩dʑɯ˩}}}}\hspace{0.5cm}[\kern2pt{\textcolor{darkblue}{\textbf{\ipa{pɤ˩dʑɯ˩˥}}}}\kern2pt]} \hypertarget{p7\string_Bdz£M\string_B1}{}
\markboth{\textcolor{darkblue}{\textbf{\ipa{pɤ˩dʑɯ˩}}}}{}
\textcolor{teal}{\mytextsc{nom}} \hspace{4pt} Ton~: L.
\textcolor{PineGreen}{\selectlanguage{french}Eau de source.} \zh{泉水。} 
\lhead{\firstmark}
\rhead{\botmark}

\subsection{\hspace{-0.5cm} {\Large \textcolor{darkblue}{\textbf{\ipa{pɤ˩-ho˩\textasciitilde{}ho˥}}}}\hspace{0.5cm}[\kern2pt{\textcolor{darkblue}{\textbf{\ipa{xxxx non-correspondance entre le nombre de morphèmes et le nombre de tons de morphèmes}}}}\kern2pt]} \hypertarget{p7\string_B-ho\string_B~ho\string_T1}{}
\markboth{\textcolor{darkblue}{\textbf{\ipa{pɤ˩-ho˩\textasciitilde{}ho˥}}}}{}
\textcolor{teal}{\mytextsc{adjectif}} \hspace{4pt} Ton~: L+H\#.
\textcolor{PineGreen}{\selectlanguage{french}Mou.} \zh{柔软。}  ¶ \textcolor{darkblue}{\textbf{\ipa{pɤ˩-ho˩\textasciitilde{}ho˥-gv̩˩}}} \textcolor{PineGreen}{\selectlanguage{french}mou} \zh{柔软}  
 ¶ \textcolor{darkblue}{\textbf{\ipa{ʁo˧qʰwɤ˩ | pɤ˩-ho˩\textasciitilde{}ho˥-gv̩˩-hĩ˩ | tʰv̩˧-kʰwɤ˥}}} \textcolor{PineGreen}{\selectlanguage{french}l'endroit où la tête est toute molle =la fontanelle, chez les bébés} \zh{头上软软的那块 =囟门}  

\lhead{\firstmark}
\rhead{\botmark}

\subsection{\hspace{-0.5cm} {\Large \textcolor{darkblue}{\textbf{\ipa{pɤ˩jɤ˧bv̩˥-di˩}}}}\hspace{0.5cm}[\kern2pt{\textcolor{darkblue}{\textbf{\ipa{xxxx non-correspondance entre le nombre de morphèmes et le nombre de tons de morphèmes}}}}\kern2pt]} \hypertarget{p7\string_Bj7\string_Mbv\string_=\string_T-di\string_B1}{}
\markboth{\textcolor{darkblue}{\textbf{\ipa{pɤ˩jɤ˧bv̩˥-di˩}}}}{}
\textcolor{teal}{\mytextsc{nom}} \hspace{4pt} Ton~: LM+\#H-.
\textcolor{PineGreen}{\selectlanguage{french}Étuve pour cuire la pâte/le pain.} \zh{用来蒸面团(馒头等等)的蒸笼。}  \zh{量词}: \textcolor{darkblue}{\textbf{\ipa{ɭɯ˧}}}  \mytextsc{clf}~: \textcolor{darkblue}{\textbf{\ipa{ɭɯ˧}}} 
\lhead{\firstmark}
\rhead{\botmark}

\subsection{\hspace{-0.5cm} {\Large \textcolor{darkblue}{\textbf{\ipa{pɤ˩jɤ˧˥}}}}\hspace{0.5cm}[\kern2pt{\textcolor{darkblue}{\textbf{\ipa{pɤ˩jɤ˧˥}}}}\kern2pt]} \hypertarget{p7\string_Bj7\string_M\string_T1}{}
\markboth{\textcolor{darkblue}{\textbf{\ipa{pɤ˩jɤ˧˥}}}}{}
\textcolor{teal}{\mytextsc{nom}} \hspace{4pt} Ton~: LM+MH\#.
\ding{202} \textcolor{PineGreen}{\selectlanguage{french}Pâte à pain (à cuire à la vapeur, pour obtenir des petits pains blancs).} \zh{做面包的面团(可以蒸成馒头)。}  \zh{量词}: \textcolor{darkblue}{\textbf{\ipa{jɤ˧˥}}} \ding{203} \textcolor{PineGreen}{\selectlanguage{french}Galette.} \zh{饼。}  ¶ \textcolor{darkblue}{\textbf{\ipa{li˩-pɤ˥jɤ˩ | ɖɯ˧-ɭɯ˧}}} \textcolor{PineGreen}{\selectlanguage{french}une galette de thé (feuilles de thé pressées en forme de galette)} \zh{一块茶饼}  
 ¶ \textcolor{darkblue}{\textbf{\ipa{ɕi˧ʈʂʰwæ˧-pɤ˩jɤ˩}}} \textcolor{PineGreen}{\selectlanguage{french}galette de riz} \zh{米饼}  
 ¶ \textcolor{darkblue}{\textbf{\ipa{dze˧ɭɯ˧-pɤ˩jɤ˩}}} \textcolor{PineGreen}{\selectlanguage{french}galette de froment, galette à la farine de blé} \zh{小麦饼}  
 ¶ \textcolor{darkblue}{\textbf{\ipa{qʰɑ˧dze˧-pɤ˩jɤ˩}}} \textcolor{PineGreen}{\selectlanguage{french}galette de maïs, galette à la farine de maïs} \zh{玉米饼}  
 ¶ \textcolor{darkblue}{\textbf{\ipa{tsʰi˧zi˧-pɤ˥jɤ˩}}} \textcolor{PineGreen}{\selectlanguage{french}galette à l'orge d'altitude} \zh{青稞饼}  
 ¶ \textcolor{darkblue}{\textbf{\ipa{jɤ˧gɯ˩-pɤ˩jɤ˩}}} \textcolor{PineGreen}{\selectlanguage{french}galette de sarrasin} \zh{甜荞饼}  
 ¶ \textcolor{darkblue}{\textbf{\ipa{jɤ˧qʰɑ˧-pɤ˥jɤ˩}}} \textcolor{PineGreen}{\selectlanguage{french}galette de sarrasin amer} \zh{苦荞饼}  
 \zh{量词}: \textcolor{darkblue}{\textbf{\ipa{jɤ˧˥}}}  \mytextsc{clf}~: \textcolor{darkblue}{\textbf{\ipa{jɤ˧˥}}} \textcolor{darkblue}{\textbf{\ipa{jɤ˧˥}}} 
\lhead{\firstmark}
\rhead{\botmark}

\subsection{\hspace{-0.5cm} {\Large \textcolor{darkblue}{\textbf{\ipa{pɤ˩lv̩˩}}}}\hspace{0.5cm}[\kern2pt{\textcolor{darkblue}{\textbf{\ipa{pɤ˧lv̩˧}}}}\kern2pt]} \hypertarget{p7\string_Blv\string_=\string_B1}{}
\markboth{\textcolor{darkblue}{\textbf{\ipa{pɤ˩lv̩˩}}}}{}
\textcolor{teal}{\mytextsc{nom}} \hspace{4pt} Ton~: L.
\textcolor{PineGreen}{\selectlanguage{french}Nuque.} \zh{项背 、项、脖颈儿。}  \zh{量词}: \textcolor{darkblue}{\textbf{\ipa{ɭɯ˧}}}  \mytextsc{clf}~: \textcolor{darkblue}{\textbf{\ipa{ɭɯ˧}}} 
\lhead{\firstmark}
\rhead{\botmark}

\subsection{\hspace{-0.5cm} {\Large \textcolor{darkblue}{\textbf{\ipa{pɤ˩lv̩˧}}}}\hspace{0.5cm}[\kern2pt{\textcolor{darkblue}{\textbf{\ipa{pɤ˩lv̩˩˥}}}}\kern2pt]} \hypertarget{p7\string_Blv\string_=\string_M1}{}
\markboth{\textcolor{darkblue}{\textbf{\ipa{pɤ˩lv̩˧}}}}{}
\textcolor{teal}{\mytextsc{nom}} \hspace{4pt} Ton~: LM.
\textcolor{PineGreen}{\selectlanguage{french}Réserve, magasin: bâtiment à un seul étage, face au bâtiment principal (\textcolor{darkblue}{\textbf{\ipa{/ʑi˧mi˧/}}}), dans lequel on range les gros outils, tels que l'araire, et la viande séchée.} \zh{仓库:主屋对面的房子,只有一层。用来收藏大工具,例如犁,或者腊肉。} 
\lhead{\firstmark}
\rhead{\botmark}

\subsection{\hspace{-0.5cm} {\Large \textcolor{darkblue}{\textbf{\ipa{pɤ˩mi˩}}}}\hspace{0.5cm}[\kern2pt{\textcolor{darkblue}{\textbf{\ipa{pɤ˩mi˥}}}}\kern2pt]} \hypertarget{p7\string_Bmi\string_B1}{}
\markboth{\textcolor{darkblue}{\textbf{\ipa{pɤ˩mi˩}}}}{}
\textcolor{teal}{\mytextsc{nom}} \hspace{4pt} Ton~: L.
\textcolor{PineGreen}{\selectlanguage{french}Grenouille.} \zh{青蛙。}  ¶ \textcolor{darkblue}{\textbf{\ipa{pɤ˩mi˩-pɤ˥pʰv̩˩}}} \textcolor{PineGreen}{\selectlanguage{french}grenouille femelle et grenouille mâle} \zh{母青蛙与公青蛙}  
 ¶ \textcolor{darkblue}{\textbf{\ipa{pɤ˩mi˩-ʝi˥pʰv̩˩}}} \textcolor{PineGreen}{\selectlanguage{french}grosse grenouille (ou crapaud); animal très courant dans la plaine. C'est l'une des trois sortes de grenouilles que connaît la locutrice. Cet animal n'est pas consommé par les Naxi (ni par les Na, qui ne mangent aucune grenouille). La locutrice emploie ce terme pour \textit{Kaloula verrucosa} et \textit{Rana chaochiaoensis}.} \zh{一种大青蛙,在永宁坝子很常见。这是发音合作人认识的三种蛙之一。纳西族人不吃这种动物(摩梭人不吃任何蛙类动物)。}  
 ¶ \textcolor{darkblue}{\textbf{\ipa{pɤ˩mi˩-ʝi˥pʰv̩˩-mi˩}}} \textcolor{PineGreen}{\selectlanguage{french}même sens} \zh{同上}  
 ¶ \textcolor{darkblue}{\textbf{\ipa{hæ̃˧ʂɯ˩-pɤ˩mi˩}}} \textcolor{PineGreen}{\selectlanguage{french}Belle grenouille, de longue taille. Elle ne s'observe qu'en forêt, dans la montagne. C'est la deuxième des trois sortes de grenouilles que connaît la locutrice.} \zh{一种很美的青蛙,身体很长。只出现在山上森林里。这是发音合作人认识的第二种青蛙。}  
 ¶ \textcolor{darkblue}{\textbf{\ipa{dʑɯ˩-pɤ˩mi˩˥}}} \textcolor{PineGreen}{\selectlanguage{french}Grenouille ayant une petite tête et de grands yeux, qui passerait le plus clair de son temps dans l'eau. C'est la troisième des trois sortes de grenouilles que connaît la locutrice. Les Naxi la chassent, la dénichant sous les cailloux des ruisseaux, surtout au cinquième mois.} \zh{一种青蛙,头小、眼睛大。这是发音合作人认识的第三种青蛙。纳西族吃这种青蛙。}  
 ¶ \textcolor{darkblue}{\textbf{\ipa{nɑ˩hĩ˥ | pɤ˧-ʂe˧ dzɯ˧; | pɤ˧-ɣɯ˧ | ɬɑ˧tɑ˥ mv̩˩! | pɤ˧-mæ˧, | bæ˧ʈʂo˥ ʝi˩!}}} \textcolor{PineGreen}{\selectlanguage{french}“Les Naxi mangent de la viande de grenouille; ils se vêtent de gilets en peau de grenouille; ils se font des balais avec la queue des grenouilles!”} \zh{谚语:“纳西人吃青蛙,披青蛙皮衣,蛙尾巴当扫帚!”}  
 \zh{量词}: \textcolor{darkblue}{\textbf{\ipa{mi˩}}}  \mytextsc{clf}~: \textcolor{darkblue}{\textbf{\ipa{mi˩}}} 
\lhead{\firstmark}
\rhead{\botmark}

\subsection{\hspace{-0.5cm} {\Large \textcolor{darkblue}{\textbf{\ipa{pɤ˩pʰv̩˩}}}}\hspace{0.5cm}[\kern2pt{\textcolor{darkblue}{\textbf{\ipa{pɤ˧pʰv̩˧˥}}}}\kern2pt]} \hypertarget{p7\string_Bp\string_hv\string_=\string_B1}{}
\markboth{\textcolor{darkblue}{\textbf{\ipa{pɤ˩pʰv̩˩}}}}{}
\textcolor{teal}{\mytextsc{nom}} \hspace{4pt} Ton~: L.
\textcolor{PineGreen}{\selectlanguage{french}Grenouille mâle.} \zh{公青蛙。}  \zh{量词}: \textcolor{darkblue}{\textbf{\ipa{mi˩}}}  \mytextsc{clf}~: \textcolor{darkblue}{\textbf{\ipa{mi˩}}} 
\lhead{\firstmark}
\rhead{\botmark}

\subsection{\hspace{-0.5cm} {\Large \textcolor{darkblue}{\textbf{\ipa{pɤ˩ti\#˥}}}}\hspace{0.5cm}[\kern2pt{\textcolor{darkblue}{\textbf{\ipa{pɤ˩ti˥}}}}\kern2pt]} \hypertarget{p7\string_Bti\#\string_T1}{}
\markboth{\textcolor{darkblue}{\textbf{\ipa{pɤ˩ti\#˥}}}}{}
\textcolor{teal}{\mytextsc{nom}} \hspace{4pt} Ton~: LM+\#H.
\textcolor{PineGreen}{\selectlanguage{french}Tabouret, petit banc.} \zh{凳子。}  \zh{量词}: \textcolor{darkblue}{\textbf{\ipa{ɭɯ˧}}}  \mytextsc{clf}~: \textcolor{darkblue}{\textbf{\ipa{ɭɯ˧}}} 
\lhead{\firstmark}
\rhead{\botmark}

\subsection{\hspace{-0.5cm} {\Large \textcolor{darkblue}{\textbf{\ipa{pɤ˩tɕɯ˧-pɤ˥mi˩}}}}\hspace{0.5cm}[\kern2pt{\textcolor{darkblue}{\textbf{\ipa{pɤ˩tɕɯ˧pɤ˥mi˩}}}}\kern2pt]} \hypertarget{p7\string_Bts£M\string_M-p7\string_Tmi\string_B1}{}
\markboth{\textcolor{darkblue}{\textbf{\ipa{pɤ˩tɕɯ˧-pɤ˥mi˩}}}}{}
\textcolor{teal}{\mytextsc{nom}} \hspace{4pt} Ton~: LM+\#H-.
\textcolor{PineGreen}{\selectlanguage{french}Têtard.} \zh{蝌蚪。}  \zh{量词}: \textcolor{darkblue}{\textbf{\ipa{mi˩}}}  \mytextsc{clf}~: \textcolor{darkblue}{\textbf{\ipa{mi˩}}} \textit{Voir~:} \textcolor{darkblue}{\textbf{\ipa{pɤ˩tɕɯ˧˥, pɤ˩tɕɯ˧-ʁo˧ɖɯ˧˥}}} 
\lhead{\firstmark}
\rhead{\botmark}

\subsection{\hspace{-0.5cm} {\Large \textcolor{darkblue}{\textbf{\ipa{pɤ˩tɕɯ˧-ʁo˧ɖɯ˧˥}}}}\hspace{0.5cm}[\kern2pt{\textcolor{darkblue}{\textbf{\ipa{xxxx non-correspondance entre le nombre de morphèmes et le nombre de tons de morphèmes}}}}\kern2pt]} \hypertarget{p7\string_Bts£M\string_M-Ro\string_Md`M\string_M\string_T1}{}
\markboth{\textcolor{darkblue}{\textbf{\ipa{pɤ˩tɕɯ˧-ʁo˧ɖɯ˧˥}}}}{}
\textcolor{teal}{\mytextsc{nom}} \hspace{4pt} Ton~: LM+MH\#.
\textcolor{PineGreen}{\selectlanguage{french}Têtard.} \zh{蝌蚪。}  \zh{量词}: \textcolor{darkblue}{\textbf{\ipa{mi˩}}}  \mytextsc{clf}~: \textcolor{darkblue}{\textbf{\ipa{mi˩}}} \textit{Voir~:} \textcolor{darkblue}{\textbf{\ipa{pɤ˩tɕɯ˧˥, pɤ˩tɕɯ˧-pɤ˥mi˩}}} 
\lhead{\firstmark}
\rhead{\botmark}

\subsection{\hspace{-0.5cm} {\Large \textcolor{darkblue}{\textbf{\ipa{pɤ˩tɕɯ˧˥}}}}\hspace{0.5cm}[\kern2pt{\textcolor{darkblue}{\textbf{\ipa{pɤ˩tɕɯ˧˥}}}}\kern2pt]} \hypertarget{p7\string_Bts£M\string_M\string_T1}{}
\markboth{\textcolor{darkblue}{\textbf{\ipa{pɤ˩tɕɯ˧˥}}}}{}
\textcolor{teal}{\mytextsc{nom}} \hspace{4pt} Ton~: LM+MH\#.
\textcolor{PineGreen}{\selectlanguage{french}Têtard.} \zh{蝌蚪。}  \zh{量词}: \textcolor{darkblue}{\textbf{\ipa{mi˩}}}  \mytextsc{clf}~: \textcolor{darkblue}{\textbf{\ipa{mi˩}}} \textit{Voir~:} \textcolor{darkblue}{\textbf{\ipa{pɤ˩tɕɯ˧-ʁo˧ɖɯ˧˥, pɤ˩tɕɯ˧-pɤ˥mi˩}}} 
\lhead{\firstmark}
\rhead{\botmark}

\subsection{\hspace{-0.5cm} {\Large \textcolor{darkblue}{\textbf{\ipa{pɤ˧˥}}}}\hspace{0.5cm}[\kern2pt{\textcolor{darkblue}{\textbf{\ipa{pɤ˧˥}}}}\kern2pt]} \hypertarget{p7\string_M\string_T1}{}
\markboth{\textcolor{darkblue}{\textbf{\ipa{pɤ˧˥}}}}{}
\textcolor{teal}{\mytextsc{verbe}} \hspace{4pt} Ton~: MH.
\textcolor{PineGreen}{\selectlanguage{french}Passer la herse, aplanir (à l'aide d'une herse/instrument permettant de lisser le champ après labourage, afin qu'il soit prêt pour qu'on y repique le riz).} \zh{耙地。}  ¶ \textcolor{darkblue}{\textbf{\ipa{ʝi˧ pɤ˥}}} \textcolor{PineGreen}{\selectlanguage{french}passer la herse} \zh{耙地}  
 ¶ \textcolor{darkblue}{\textbf{\ipa{ɕi˧ tv̩˧-dʑo˧, | ʝi˧ le˧-pɤ˩!}}} \textcolor{PineGreen}{\selectlanguage{french}Quand on plante du riz (=avant de planter le riz), il faut passer la herse!} \zh{种稻谷,要(先)耙地!}  

\lhead{\firstmark}
\rhead{\botmark}

\subsection{\hspace{-0.5cm} {\Large \textcolor{darkblue}{\textbf{\ipa{pɤ˩˧ʐv̩˩}}}}\hspace{0.5cm}[\kern2pt{\textcolor{darkblue}{\textbf{\ipa{xxxx non-correspondance entre le nombre de morphèmes et le nombre de tons de morphèmes}}}}\kern2pt]} \hypertarget{p7\string_B\string_Mz`v\string_=\string_B1}{}
\markboth{\textcolor{darkblue}{\textbf{\ipa{pɤ˩˧ʐv̩˩}}}}{}
\textcolor{teal}{\mytextsc{nom}} \hspace{4pt} Ton~: LM-L.
\textcolor{PineGreen}{\selectlanguage{french}Matelas.} \zh{褥子(汉语借词:被褥)。} Dialecte chinois local~:\zh{被褥。} Emprunt~: chinois  \zh{被褥}
 \zh{量词}: \textcolor{darkblue}{\textbf{\ipa{tsʰi˥}}}  \mytextsc{clf}~: \textcolor{darkblue}{\textbf{\ipa{tsʰi˥}}} 
\lhead{\firstmark}
\rhead{\botmark}

\subsection{\hspace{-0.5cm} {\Large \textcolor{darkblue}{\textbf{\ipa{pi˥}}}}\hspace{0.5cm}[\kern2pt{\textcolor{darkblue}{\textbf{\ipa{pi˩˥}}}}\kern2pt]} \hypertarget{pi\string_T1}{}
\markboth{\textcolor{darkblue}{\textbf{\ipa{pi˥}}}}{}
\textcolor{teal}{\mytextsc{verbe}} \hspace{4pt} Ton~: H.
\textcolor{PineGreen}{\selectlanguage{french}Dire.} \zh{说。}  ¶ \textcolor{darkblue}{\textbf{\ipa{tʰɑ˧-pi˥!}}} \textcolor{PineGreen}{\selectlanguage{french}Il ne faut pas (le) dire!} \zh{别说!}  
 ¶ \textcolor{darkblue}{\textbf{\ipa{ə˧tso˧ pi˧?}}} \textcolor{PineGreen}{\selectlanguage{french}Que dis-tu? (employé pour demander à quelqu'un de répéter)} \zh{(你刚才)说什么?(请人家重新说一遍)}  
 ¶ \textcolor{darkblue}{\textbf{\ipa{ə˧tso˧ pi˧-ɲi˥?}}} \textcolor{PineGreen}{\selectlanguage{french}Que dis-tu? (employé pour demander à quelqu'un de répéter)} \zh{(你刚才)说什么?(请人家重新说一遍)}  

\lhead{\firstmark}
\rhead{\botmark}

\subsection{\hspace{-0.5cm} {\Large \textcolor{darkblue}{\textbf{\ipa{pi˧lv̩\#˥}}}}\hspace{0.5cm}[\kern2pt{\textcolor{darkblue}{\textbf{\ipa{pi˧lv̩˩}}}}\kern2pt]} \hypertarget{pi\string_Mlv\string_=\#\string_T1}{}
\markboth{\textcolor{darkblue}{\textbf{\ipa{pi˧lv̩\#˥}}}}{}
\textcolor{teal}{\mytextsc{nom}} \hspace{4pt} Ton~: \#H.
\textcolor{PineGreen}{\selectlanguage{french}Déchet de la distillation: ce qui reste après la production de l'alcool; grain qu'on donne aux animaux.} \zh{酒糟:煮酒剩下的渣滓(一般给猪吃)。} Dialecte chinois local~:\zh{酒糟。} ¶ \textcolor{darkblue}{\textbf{\ipa{pi˧lv̩˧, | hĩ˧ | dzɯ˧-mɤ˧-kv̩˩!}}} \textcolor{PineGreen}{\selectlanguage{french}Les grains après distillation, ça ne se mange pas! / ce n'est pas propre à la consommation humaine!} \zh{酒糟,人不能吃!}  

\lhead{\firstmark}
\rhead{\botmark}

\subsection{\hspace{-0.5cm} {\Large \textcolor{darkblue}{\textbf{\ipa{pi˧mɑ˧}}}}\hspace{0.5cm}[\kern2pt{\textcolor{darkblue}{\textbf{\ipa{pi˧mɑ˧}}}}\kern2pt]} \hypertarget{pi\string_MmA\string_M1}{}
\markboth{\textcolor{darkblue}{\textbf{\ipa{pi˧mɑ˧}}}}{}
\textcolor{teal}{\mytextsc{nom}} \hspace{4pt} Ton~: M.
\textcolor{PineGreen}{\selectlanguage{french}Prénom unisexe: prénom utilisé pour les deux sexes.} \zh{男女通用名。} 
\lhead{\firstmark}
\rhead{\botmark}

\subsection{\hspace{-0.5cm} {\Large \textcolor{darkblue}{\textbf{\ipa{pi˧mɑ˧-ɬɑ˩mv̩˩}}}}\hspace{0.5cm}[\kern2pt{\textcolor{darkblue}{\textbf{\ipa{xxxx non-correspondance entre le nombre de morphèmes et le nombre de tons de morphèmes}}}}\kern2pt]} \hypertarget{pi\string_MmA\string_M-KA\string_Bmv\string_=\string_B1}{}
\markboth{\textcolor{darkblue}{\textbf{\ipa{pi˧mɑ˧-ɬɑ˩mv̩˩}}}}{}
\textcolor{teal}{\mytextsc{nom}} \hspace{4pt} Ton~: \mytextsc{L}.
\textcolor{PineGreen}{\selectlanguage{french}Prénom féminin.} \zh{女性名字。} 
\lhead{\firstmark}
\rhead{\botmark}

\subsection{\hspace{-0.5cm} {\Large \textcolor{darkblue}{\textbf{\ipa{pi˧mv̩˥\$}}}}\hspace{0.5cm}[\kern2pt{\textcolor{darkblue}{\textbf{\ipa{pi˧mv̩˥}}}}\kern2pt]} \hypertarget{pi\string_Mmv\string_=\string_T\$1}{}
\markboth{\textcolor{darkblue}{\textbf{\ipa{pi˧mv̩˥\$}}}}{}
\textcolor{teal}{\mytextsc{nom}} \hspace{4pt} Ton~: H\$.
\textcolor{PineGreen}{\selectlanguage{french}Dicton, parole du temps jadis, adage.} \zh{成语、俗语。} 
\lhead{\firstmark}
\rhead{\botmark}

\subsection{\hspace{-0.5cm} {\Large \textcolor{darkblue}{\textbf{\ipa{pi˧tsʰe˩-di˩}}}}\hspace{0.5cm}[\kern2pt{\textcolor{darkblue}{\textbf{\ipa{pi˧tsʰe˩di˧}}}}\kern2pt]} \hypertarget{pi\string_Mts\string_he\string_B-di\string_B1}{}
\markboth{\textcolor{darkblue}{\textbf{\ipa{pi˧tsʰe˩-di˩}}}}{}
\textcolor{teal}{\mytextsc{nom}} \hspace{4pt} Ton~: L\#-.
\textcolor{PineGreen}{\selectlanguage{french}Un village proche des Sources Chaudes.} \zh{温泉乡的一个村落。}  ¶ \textcolor{darkblue}{\textbf{\ipa{ə˧go˧-ʁwɤ˧, | ʁwɤ˧lɑ˩-bi˩, | bæ˧ʁwɤ˧, | tʰo˧tsʰe\#˥, | pi˧tsʰe˩-di˩, | pɤ˧dʑɤ˩-di˩, | ʁwɤ˧tv̩˧}}} \textcolor{PineGreen}{\selectlanguage{french}Villages au sortir de la plaine de Yongning; les deux premiers comportent une population na; le troisième est un village na; les suivants sont essentiellement des villages pumi/prinmi.} \zh{永宁背向泸沽湖方向经过的村落。前两个村落拥有相当大的摩梭人口比例,第三个村落是摩梭村,最后一个是普米村。}  
 ¶ \textcolor{darkblue}{\textbf{\ipa{pi˧tsʰe˩: bɤ˩! |}}} \textcolor{PineGreen}{\selectlanguage{french}\textcolor{darkblue}{\textbf{\ipa{/pi˧tsʰe˩/}}}, c'est un village pumi!} \zh{fv:/pi˧tsʰe˩/是一个普米族村落!}  

\lhead{\firstmark}
\rhead{\botmark}

\subsection{\hspace{-0.5cm} {\Large \textcolor{darkblue}{\textbf{\ipa{pi˩ɻ̍˥}}}}\hspace{0.5cm}[\kern2pt{\textcolor{darkblue}{\textbf{\ipa{pi˩ɻ̍˥}}}}\kern2pt]} \hypertarget{pi\string_Br£`̍\string_T1}{}
\markboth{\textcolor{darkblue}{\textbf{\ipa{pi˩ɻ̍˥}}}}{}
\textcolor{teal}{\mytextsc{nom}} \hspace{4pt} Ton~: LH.
\textcolor{PineGreen}{\selectlanguage{french}Double menton, bourrelet de chair sous le menton.} \zh{双下巴。}  \zh{量词}: \textcolor{darkblue}{\textbf{\ipa{ɭɯ˧}}}  \mytextsc{clf}~: \textcolor{darkblue}{\textbf{\ipa{ɭɯ˧}}} 
\lhead{\firstmark}
\rhead{\botmark}

\subsection{\hspace{-0.5cm} {\Large \textcolor{darkblue}{\textbf{\ipa{pi˩ti\#˥}}}}\hspace{0.5cm}[\kern2pt{\textcolor{darkblue}{\textbf{\ipa{pi˩ti˥}}}}\kern2pt]} \hypertarget{pi\string_Bti\#\string_T1}{}
\markboth{\textcolor{darkblue}{\textbf{\ipa{pi˩ti\#˥}}}}{}
\textcolor{teal}{\mytextsc{nom}} \hspace{4pt} Ton~: LM+\#H.
\textcolor{PineGreen}{\selectlanguage{french}Pépite d'argent.} \zh{银块。}  \zh{量词}: \textcolor{darkblue}{\textbf{\ipa{ɭɯ˧}}}  \mytextsc{clf}~: \textcolor{darkblue}{\textbf{\ipa{ɭɯ˧}}} 
\lhead{\firstmark}
\rhead{\botmark}

\subsection{\hspace{-0.5cm} {\Large \textcolor{darkblue}{\textbf{\ipa{pi˧˥\textsubscript{a}}}}}\hspace{0.5cm}[\kern2pt{\textcolor{darkblue}{\textbf{\ipa{pi˧˥}}}}\kern2pt]} \hypertarget{pi\string_M\string_Ta1}{}
\markboth{\textcolor{darkblue}{\textbf{\ipa{pi˧˥\textsubscript{a}}}}}{}
\textcolor{teal}{\mytextsc{classificateur}} \hspace{4pt} Ton~: MH\textsubscript{a}.
\textcolor{PineGreen}{\selectlanguage{french}Peu (indénombrable), un peu; souvent employé comme hypocoristique.} \zh{量词:少。}  ¶ \textcolor{darkblue}{\textbf{\ipa{ɖɯ˧-pi˧˥}}} \textcolor{PineGreen}{\selectlanguage{french}un peu} \zh{一点}  
 ¶ \textcolor{darkblue}{\textbf{\ipa{qʰæ˧-pi˩}}} \textcolor{PineGreen}{\selectlanguage{french}un peu de crottin (ramassé comme engrais)} \zh{一点粪肥}  
 ¶ \textcolor{darkblue}{\textbf{\ipa{ŋv̩˧-pi˧}}} \textcolor{PineGreen}{\selectlanguage{french}un peu d’argent} \zh{一点钱}  
 ¶ \textcolor{darkblue}{\textbf{\ipa{ŋv̩˧ | ɖɯ˧-pi˧˥}}} \textcolor{PineGreen}{\selectlanguage{french}un peu d’argent} \zh{一点钱}  
 ¶ \textcolor{darkblue}{\textbf{\ipa{lwɤ˧˥ | ɖɯ˧ pi˧˥}}} \textcolor{PineGreen}{\selectlanguage{french}un peu de cendre; on ne peut dire: *lwɤ˧-pi˥, non plus que: *tsʰe˧-pi˩ (pour “un peu de sel”)} \zh{一点灰}  
 ¶ \textcolor{darkblue}{\textbf{\ipa{ʈʂʰɯ˧ | ɖʐe˧ ɖɯ˧-pi˧ dʑo˧!}}} \textcolor{PineGreen}{\selectlanguage{french}il a un peu d'argent!} \zh{他有一些钱!}  

\lhead{\firstmark}
\rhead{\botmark}

\subsection{\hspace{-0.5cm} {\Large \textcolor{darkblue}{\textbf{\ipa{pi˩˥}}}}\hspace{0.5cm}[\kern2pt{\textcolor{darkblue}{\textbf{\ipa{pi˥}}}}\kern2pt]} \hypertarget{pi\string_B\string_T1}{}
\markboth{\textcolor{darkblue}{\textbf{\ipa{pi˩˥}}}}{}
\textcolor{teal}{\mytextsc{nom}} \hspace{4pt} Ton~: LH.
\textcolor{PineGreen}{\selectlanguage{french}Pinceau (emprunt ancien).} \zh{笔。}  Emprunt~: chinois  \zh{笔}
 ¶ \textcolor{darkblue}{\textbf{\ipa{tʰæ˧ɻæ˩ tɕɯ˩-di˩, | pi˩˥!}}} \textcolor{PineGreen}{\selectlanguage{french}Le truc pour écrire, ça s'appelle “pinceau”!} \zh{用来写字的那个东西,(叫做)“笔”!}  

\lhead{\firstmark}
\rhead{\botmark}

\subsection{\hspace{-0.5cm} {\Large \textcolor{darkblue}{\textbf{\ipa{pjɤ˥}}}}\hspace{0.5cm}[\kern2pt{\textcolor{darkblue}{\textbf{\ipa{pjɤ˥}}}}\kern2pt]} \hypertarget{pj7\string_T1}{}
\markboth{\textcolor{darkblue}{\textbf{\ipa{pjɤ˥}}}}{}
\textcolor{teal}{\mytextsc{adjectif}} \hspace{4pt} Ton~: H.
\textcolor{PineGreen}{\selectlanguage{french}Carré/anguleux (visage, pilier…).} \zh{方形的。}  ¶ \textcolor{darkblue}{\textbf{\ipa{tɑ˧-pjɤ˧\textasciitilde{}pjɤ˥ (-zo˩)}}} \textcolor{PineGreen}{\selectlanguage{french}un peu anguleux/carré (terme péjoratif: objet / physique trop peu lisse pour être plaisant au regard ou au toucher)} \zh{(脸、物品)太方,不圆滑}  

\lhead{\firstmark}
\rhead{\botmark}

\subsection{\hspace{-0.5cm} {\Large \textcolor{darkblue}{\textbf{\ipa{po˥}}}}\hspace{0.5cm}[\kern2pt{\textcolor{darkblue}{\textbf{\ipa{po˧˥}}}}\kern2pt]} \hypertarget{po\string_T1}{}
\markboth{\textcolor{darkblue}{\textbf{\ipa{po˥}}}}{}
\textcolor{teal}{\mytextsc{verbe}} \hspace{4pt} Ton~: .
\textcolor{PineGreen}{\selectlanguage{french}Emballer.} \zh{包(量词)(汉语借词)。}  Emprunt~: chinois  \zh{包}
 ¶ \textcolor{darkblue}{\textbf{\ipa{le˧-po˥}}} \textcolor{PineGreen}{\selectlanguage{french}\mytextsc{accomp}} \zh{\mytextsc{accomp}}  

\lhead{\firstmark}
\rhead{\botmark}

\subsection{\hspace{-0.5cm} {\Large \textcolor{darkblue}{\textbf{\ipa{po˧\textsubscript{a}}}}}\hspace{0.5cm}[\kern2pt{\textcolor{darkblue}{\textbf{\ipa{po˥}}}}\kern2pt]} \hypertarget{po\string_Ma1}{}
\markboth{\textcolor{darkblue}{\textbf{\ipa{po˧\textsubscript{a}}}}}{}
\textcolor{teal}{\mytextsc{classificateur}} \hspace{4pt} Ton~: M\textsubscript{a}.
\textcolor{PineGreen}{\selectlanguage{french}Classificateur des plantes à tiges (fleurs, poireaux…); aussi utilisé pour compter les types/catégories de vêtements.} \zh{量词:有根的植物,衣服(一棵,一件)。} 
\lhead{\firstmark}
\rhead{\botmark}

\subsection{\hspace{-0.5cm} {\Large \textcolor{darkblue}{\textbf{\ipa{po˧ɖʐɯ\#˥}}}}\hspace{0.5cm}[\kern2pt{\textcolor{darkblue}{\textbf{\ipa{po˧ɖʐɯ˧}}}}\kern2pt]} \hypertarget{po\string_Md`z`M\#\string_T1}{}
\markboth{\textcolor{darkblue}{\textbf{\ipa{po˧ɖʐɯ\#˥}}}}{}
\textcolor{teal}{\mytextsc{nom}} \hspace{4pt} Ton~: \#H.
\textcolor{PineGreen}{\selectlanguage{french}Artisan.} \zh{工匠。}  ¶ \textcolor{darkblue}{\textbf{\ipa{po˧ɖʐɯ˧ ʝi˧-hĩ˧-hĩ˧}}} \textcolor{PineGreen}{\selectlanguage{french}personne qui travaille comme artisan} \zh{当工匠的人}  
 \zh{量词}: \textcolor{darkblue}{\textbf{\ipa{v̩˧}}}  \mytextsc{clf}~: \textcolor{darkblue}{\textbf{\ipa{v̩˧}}} 
\lhead{\firstmark}
\rhead{\botmark}

\subsection{\hspace{-0.5cm} {\Large \textcolor{darkblue}{\textbf{\ipa{po˧lo˧}}}}\hspace{0.5cm}[\kern2pt{\textcolor{darkblue}{\textbf{\ipa{po˩lo˧˥}}}}\kern2pt]} \hypertarget{po\string_Mlo\string_M1}{}
\markboth{\textcolor{darkblue}{\textbf{\ipa{po˧lo˧}}}}{}
\textcolor{teal}{\mytextsc{nom}} \hspace{4pt} Ton~: M.
\textcolor{PineGreen}{\selectlanguage{french}Bélier; bouc.} \zh{公绵羊。}  ¶ \textcolor{darkblue}{\textbf{\ipa{po˧lo˧ lɑ˧˥}}} \textcolor{PineGreen}{\selectlanguage{french}frapper un bélier} \zh{打公绵羊}  
 \zh{量词}: \textcolor{darkblue}{\textbf{\ipa{pʰo˧˥}}}  \mytextsc{clf}~: \textcolor{darkblue}{\textbf{\ipa{pʰo˧˥}}} 
\lhead{\firstmark}
\rhead{\botmark}

\subsection{\hspace{-0.5cm} {\Large \textcolor{darkblue}{\textbf{\ipa{po˧po˧}}}}\hspace{0.5cm}[\kern2pt{\textcolor{darkblue}{\textbf{\ipa{po˩po˩˥}}}}\kern2pt]} \hypertarget{po\string_Mpo\string_M1}{}
\markboth{\textcolor{darkblue}{\textbf{\ipa{po˧po˧}}}}{}
\textcolor{teal}{\mytextsc{nom}} \hspace{4pt} Ton~: M.
\textcolor{PineGreen}{\selectlanguage{french}Ballon.} \zh{球。}  ¶ \textcolor{darkblue}{\textbf{\ipa{[F5] po˧po˧ lɑ˧˥}}} \textcolor{PineGreen}{\selectlanguage{french}jouer au ballon} \zh{打球}  
 \zh{量词}: \textcolor{darkblue}{\textbf{\ipa{ɭɯ˧}}}  \mytextsc{clf}~: \textcolor{darkblue}{\textbf{\ipa{ɭɯ˧}}} 
\lhead{\firstmark}
\rhead{\botmark}

\subsection{\hspace{-0.5cm} {\Large \textcolor{darkblue}{\textbf{\ipa{po˧po˧tsʰɤ˧˥}}}}\hspace{0.5cm}[\kern2pt{\textcolor{darkblue}{\textbf{\ipa{po˧po˧tsʰɤ˧}}}}\kern2pt]} \hypertarget{po\string_Mpo\string_Mts\string_h7\string_M\string_T1}{}
\markboth{\textcolor{darkblue}{\textbf{\ipa{po˧po˧tsʰɤ˧˥}}}}{}
\textcolor{teal}{\mytextsc{nom}} \hspace{4pt} Ton~: MH\#.
\textcolor{PineGreen}{\selectlanguage{french}Chou.} \zh{圆白菜。} Dialecte chinois local~:\zh{包包菜。} Emprunt~: chinois  \zh{包包菜}
 \zh{量词}: \textcolor{darkblue}{\textbf{\ipa{ɭɯ˧}}}  \mytextsc{clf}~: \textcolor{darkblue}{\textbf{\ipa{ɭɯ˧}}} 
\lhead{\firstmark}
\rhead{\botmark}

\subsection{\hspace{-0.5cm} {\Large \textcolor{darkblue}{\textbf{\ipa{po˩\textsubscript{b}}}}}\hspace{0.5cm}[\kern2pt{\textcolor{darkblue}{\textbf{\ipa{po˩˥}}}}\kern2pt]} \hypertarget{po\string_Bb1}{}
\markboth{\textcolor{darkblue}{\textbf{\ipa{po˩\textsubscript{b}}}}}{}
\textcolor{teal}{\mytextsc{classificateur}} \hspace{4pt} Ton~: L\textsubscript{b}.
\textcolor{PineGreen}{\selectlanguage{french}Classificateur des paquets.} \zh{量词:包(汉语借词)。}  Emprunt~: chinois  \zh{包}

\lhead{\firstmark}
\rhead{\botmark}

\subsection{\hspace{-0.5cm} {\Large \textcolor{darkblue}{\textbf{\ipa{po˧˥}}}}\hspace{0.5cm}[\kern2pt{\textcolor{darkblue}{\textbf{\ipa{po˧˥}}}}\kern2pt]} \hypertarget{po\string_M\string_T1}{}
\markboth{\textcolor{darkblue}{\textbf{\ipa{po˧˥}}}}{}
\textcolor{teal}{\mytextsc{verbe}} \hspace{4pt} Ton~: MH.
\ding{202} \textcolor{PineGreen}{\selectlanguage{french}Amener, apporter; ramener, rapporter; faire cadeau de; envoyer (un message), transmettre; utiliser.} \zh{寄信、服送、带过来、拿、送。}  ¶ \textcolor{darkblue}{\textbf{\ipa{qʰwæ˧ po˧˥}}} \textcolor{PineGreen}{\selectlanguage{french}amener une lettre/un message} \zh{带来一封信/一个消息}  
 ¶ \textcolor{darkblue}{\textbf{\ipa{tso˧\textasciitilde{}tso˧ tʰi˧-po˧˥}}} \textcolor{PineGreen}{\selectlanguage{french}amener quelque chose} \zh{带来一个东西}  
 ¶ \textcolor{darkblue}{\textbf{\ipa{hĩ˧ ɖɯ˧-v̩˧ | tso˧\textasciitilde{}tso˧ ɖɯ˧-kʰwɤ˥ | tʰi˧-po˧˥}}} \textcolor{PineGreen}{\selectlanguage{french}quelqu'un prend/amène quelque chose} \zh{有人带东西过来}  
 ¶ \textcolor{darkblue}{\textbf{\ipa{ʈʂʰwæ˧˥ | po˧-jo˥!}}} \textcolor{PineGreen}{\selectlanguage{french}amène vite!} \zh{快拿过来吧!/ 快带过来吧!}  
\ding{203} \textcolor{PineGreen}{\selectlanguage{french}Porter un enfant, c'est-à-dire être enceinte.} \zh{怀孕。}  ¶ \textcolor{darkblue}{\textbf{\ipa{ʈʂʰɯ˧ | zo˧mv̩˥ po˩.}}} \textcolor{PineGreen}{\selectlanguage{french}Elle est enceinte.} \zh{她怀孕了。}  
 ¶ \textcolor{darkblue}{\textbf{\ipa{zo˧ po˩ (+ze˩)}}} \textcolor{PineGreen}{\selectlanguage{french}porter un enfant, être enceinte.} \zh{怀孕}  

\lhead{\firstmark}
\rhead{\botmark}

\subsection{\hspace{-0.5cm} {\Large \textcolor{darkblue}{\textbf{\ipa{pv̩˩}}}}\hspace{0.5cm}[\kern2pt{\textcolor{darkblue}{\textbf{\ipa{pv̩˩˥}}}}\kern2pt]} \hypertarget{pv\string_=\string_B1}{}
\markboth{\textcolor{darkblue}{\textbf{\ipa{pv̩˩}}}}{}
\textcolor{teal}{\mytextsc{verbe}} \hspace{4pt} Ton~: L.
\textcolor{PineGreen}{\selectlanguage{french}Passer, s'écouler: le temps passe, les jours passent.} \zh{过、过去(时间过去、日子过去)。}  ¶ \textcolor{darkblue}{\textbf{\ipa{ɲi˧mi˧ pv̩˩}}} \textcolor{PineGreen}{\selectlanguage{french}le temps passe, la journée passe} \zh{时间过去。直译:(一)天(慢慢)过(去)}  
 ¶ \textcolor{darkblue}{\textbf{\ipa{ɲi˧mi˧ | le˧-pv̩˩-ze˩}}} \textcolor{PineGreen}{\selectlanguage{french}le temps a passé, la journée a passé} \zh{时间过去了,(一)天过去了}  
 ¶ \textcolor{darkblue}{\textbf{\ipa{dʑɤ˩-dzɯ˧ qʰwɤ˧-dzɯ˥, | bi˧mi˧ ʂv̩˧-qʰwɤ˧-ɻ̍˥; | dʑɤ˩-ʐwɤ˥ qʰwɤ˩-ʐwɤ˩, | ɲi˧mi˧ ʂæ˧ pv̩˩-di˩!}}} \textcolor{PineGreen}{\selectlanguage{french}Qu'on mange bien ou mal, on arrive à se remplir le ventre / Que la nourriture soit bonne ou mauvaise, peu importe au fond, tant qu'on a le ventre plein; qu'on dise des bonnes choses (=des éloges d'autrui) ou des mauvaises (=des ragots), on arrive à passer la journée / le jour se passe agréablement! (Proverbe qui fait l'éloge des vertus du bavardage et du commérage.)} \zh{“吃好吃坏,(都)能填满肚子/(都)能吃饱!说好说坏,(都)能让一天(轻松)过去!”(这个谚语,说闲聊的好。)}  
 ¶ \textcolor{darkblue}{\textbf{\ipa{dʑɤ˩-dzɯ˧ qʰwɤ˧-dzɯ˥, | bi˧mi˧ ʂv̩˧˥; | dʑɤ˩-ʐwɤ˥ qʰwɤ˩-ʐwɤ˩, | ɲi˧mi˧ ʂæ˧-pv̩˩-di˩!}}} \textcolor{PineGreen}{\selectlanguage{french}Variante du proverbe ci-dessus.} \zh{上述谚语的变体}  

\lhead{\firstmark}
\rhead{\botmark}

\subsection{\hspace{-0.5cm} {\Large \textcolor{darkblue}{\textbf{\ipa{pv̩˧}}} \textsubscript{1}}\hspace{0.5cm}[\kern2pt{\textcolor{darkblue}{\textbf{\ipa{pv̩˥}}}}\kern2pt]} \hypertarget{pv\string_=\string_M1}{}
\markboth{\textcolor{darkblue}{\textbf{\ipa{pv̩˧}}} \textsubscript{1}}{}
\textcolor{teal}{\mytextsc{verbe}} \hspace{4pt} Ton~: M\textsubscript{c}.
\textcolor{PineGreen}{\selectlanguage{french}Faire un sacrifice, faire un rituel, psalmodier.} \zh{祭。}  ¶ \textcolor{darkblue}{\textbf{\ipa{kʰv̩˧ pv̩˥}}} \textcolor{PineGreen}{\selectlanguage{french}faire la veillée du Nouvel An (la veille de la nouvelle année), célébrer la veillée du Nouvel An} \zh{做过年的祭礼}  
 ¶ \textcolor{darkblue}{\textbf{\ipa{tsʰi˧ɲi˧, | kʰv̩˧ pv̩˥-tso˩-ɲi˩!}}} \textcolor{PineGreen}{\selectlanguage{french}ce soir, on va fêter le Nouvel An!} \zh{今天就要过年了!}  

\lhead{\firstmark}
\rhead{\botmark}

\subsection{\hspace{-0.5cm} {\Large \textcolor{darkblue}{\textbf{\ipa{pv̩˧}}} \textsubscript{2}}\hspace{0.5cm}[\kern2pt{\textcolor{darkblue}{\textbf{\ipa{pv̩˥}}}}\kern2pt]} \hypertarget{pv\string_=\string_M2}{}
\markboth{\textcolor{darkblue}{\textbf{\ipa{pv̩˧}}} \textsubscript{2}}{}
\textcolor{teal}{\mytextsc{adjectif}} \hspace{4pt} Ton~: M.
\textcolor{PineGreen}{\selectlanguage{french}Sec.} \zh{干燥。}  ¶ \textcolor{darkblue}{\textbf{\ipa{le˧-pv̩˧-ze˧}}} \textcolor{PineGreen}{\selectlanguage{french}\mytextsc{accomp} \string_ \mytextsc{pfv}} \zh{干了}  
 ¶ \textcolor{darkblue}{\textbf{\ipa{le˧-pv̩˧ le˧-ʐwæ˩-ze˩}}} \textcolor{PineGreen}{\selectlanguage{french}ça a complètement séché/c'est entièrement sec/c'est parfaitement sec} \zh{干透了}  
 ¶ \textcolor{darkblue}{\textbf{\ipa{pv̩˧-kæ˧-ɻæ˩-gv̩˩}}} \textcolor{PineGreen}{\selectlanguage{french}tout sec} \zh{全干、完全干}  

\lhead{\firstmark}
\rhead{\botmark}

\subsection{\hspace{-0.5cm} {\Large \textcolor{darkblue}{\textbf{\ipa{pv̩˧˥}}} \textsubscript{1}}\hspace{0.5cm}[\kern2pt{\textcolor{darkblue}{\textbf{\ipa{pv̩˧˥}}}}\kern2pt]} \hypertarget{pv\string_=\string_M\string_T1}{}
\markboth{\textcolor{darkblue}{\textbf{\ipa{pv̩˧˥}}} \textsubscript{1}}{}
\textcolor{teal}{\mytextsc{verbe}} \hspace{4pt} Ton~: MH.
\textcolor{PineGreen}{\selectlanguage{french}Enlever, arracher (les mauvaises herbes); couper du fourrage pour les animaux domestiques.} \zh{拔、扯(草)。}  ¶ \textcolor{darkblue}{\textbf{\ipa{zɯ˧ pv̩˩}}} \textcolor{PineGreen}{\selectlanguage{french}arracher les mauvaises herbes; couper du fourrage pour les animaux domestiques} \zh{拔草}  
 ¶ \textcolor{darkblue}{\textbf{\ipa{zɯ˧ | le˧-pv̩˧˥}}} \textcolor{PineGreen}{\selectlanguage{french}arracher les mauvaises herbes; couper du fourrage pour les animaux domestiques} \zh{拔草}  

\lhead{\firstmark}
\rhead{\botmark}

\subsection{\hspace{-0.5cm} {\Large \textcolor{darkblue}{\textbf{\ipa{pv̩˧˥}}} \textsubscript{2}}\hspace{0.5cm}[\kern2pt{\textcolor{darkblue}{\textbf{\ipa{pv̩˧˥}}}}\kern2pt]} \hypertarget{pv\string_=\string_M\string_T2}{}
\markboth{\textcolor{darkblue}{\textbf{\ipa{pv̩˧˥}}} \textsubscript{2}}{}
\textcolor{teal}{\mytextsc{verbe}} \hspace{4pt} Ton~: MH.
\textcolor{PineGreen}{\selectlanguage{french}Dégainer (une arme blanche), sortir du fourreau.} \zh{拉出(剑……)。}  ¶ \textcolor{darkblue}{\textbf{\ipa{ʁæ˧mi˧ | tʰi˧-pv̩˧˥}}} \textcolor{PineGreen}{\selectlanguage{french}dégainer une épée} \zh{拉出剑}  
 ¶ \textcolor{darkblue}{\textbf{\ipa{gæ˩-pv̩˧˥}}} \textcolor{PineGreen}{\selectlanguage{french}dégainer, sortir une arme de son fourreau} \zh{拉出(剑……)}  
 ¶ \textcolor{darkblue}{\textbf{\ipa{ʁæ˧mi˧ | gæ˩-pv̩˧˥}}} \textcolor{PineGreen}{\selectlanguage{french}dégainer une épée} \zh{拉出剑}  

\lhead{\firstmark}
\rhead{\botmark}

\subsection{\hspace{-0.5cm} {\Large \textcolor{darkblue}{\textbf{\ipa{pv̩˧˥}}} \textsubscript{3}}\hspace{0.5cm}[\kern2pt{\textcolor{darkblue}{\textbf{\ipa{pv̩˧˥}}}}\kern2pt]} \hypertarget{pv\string_=\string_M\string_T3}{}
\markboth{\textcolor{darkblue}{\textbf{\ipa{pv̩˧˥}}} \textsubscript{3}}{}
\textcolor{teal}{\mytextsc{classificateur}} \hspace{4pt} Ton~: MH\textsubscript{a}.
\textcolor{PineGreen}{\selectlanguage{french}Classificateur des pas/enjambées; emprunt au chinois.} \zh{量词:步。}  Emprunt~: chinois  \zh{步}

\lhead{\firstmark}
\rhead{\botmark}

\subsection{\hspace{-0.5cm} {\Large \textcolor{darkblue}{\textbf{\ipa{pv̩˩\textsubscript{a}}}} \textsubscript{1}}\hspace{0.5cm}[\kern2pt{\textcolor{darkblue}{\textbf{\ipa{pv̩˩˥}}}}\kern2pt]} \hypertarget{pv\string_=\string_Ba1}{}
\markboth{\textcolor{darkblue}{\textbf{\ipa{pv̩˩\textsubscript{a}}}} \textsubscript{1}}{}
\textcolor{teal}{\mytextsc{verbe}} \hspace{4pt} Ton~: L\textsubscript{a}.
\textcolor{PineGreen}{\selectlanguage{french}Raccompagner; escorter; mener, conduire (du bétail).} \zh{送行。}  ¶ \textcolor{darkblue}{\textbf{\ipa{hĩ˧bæ˧ pv̩˥}}} \textcolor{PineGreen}{\selectlanguage{french}raccompagner un invité} \zh{送客}  

\lhead{\firstmark}
\rhead{\botmark}

\subsection{\hspace{-0.5cm} {\Large \textcolor{darkblue}{\textbf{\ipa{pv̩˩\textsubscript{a}}}} \textsubscript{2}}\hspace{0.5cm}[\kern2pt{\textcolor{darkblue}{\textbf{\ipa{pv̩˩˥}}}}\kern2pt]} \hypertarget{pv\string_=\string_Ba2}{}
\markboth{\textcolor{darkblue}{\textbf{\ipa{pv̩˩\textsubscript{a}}}} \textsubscript{2}}{}
\textcolor{teal}{\mytextsc{verbe}} \hspace{4pt} Ton~: L\textsubscript{a}.
\textcolor{PineGreen}{\selectlanguage{french}Autoriser (ex.: un mariage); demander (à quelqu'un de faire quelque chose), faire faire; commanditer, être commanditaire/investisseur (ex.: pour une caravane).} \zh{让,安排,投资,要求。}  ¶ \textcolor{darkblue}{\textbf{\ipa{sɯ˧pʰi˧-ɳɯ˧ | pv̩˩-kʰɯ˥-ɲi˩!}}} \textcolor{PineGreen}{\selectlanguage{french}c'est le seigneur qui était le commanditaire!} \zh{(马帮)是土司来投资的!}  
 ¶ \textcolor{darkblue}{\textbf{\ipa{ʈʂʰɯ˧ | ɖʐe˧ ʂe˧ pv̩˩-kʰɯ˩-tso˩-ɲi˩!}}} \textcolor{PineGreen}{\selectlanguage{french}c'est elle/lui qui apporte le capital/qui commandite! (ex.: pour une caravane)} \zh{是他来投资的!(如:马帮)}  
 ¶ \textcolor{darkblue}{\textbf{\ipa{hĩ˧-ɳɯ˩ | pv̩˩-mɤ˩-kʰɯ˥!}}} \textcolor{PineGreen}{\selectlanguage{french}On n'est pas autorisé à y aller! (Contexte: discussion au sujet des difficultés pour l'enquêteur d'accès à une localité où sont parlées des langues naish, dans le Sichuan. La consultante résume: “On n'est pas autorisé à y aller! / L'accès n'est pas autorisé!”)} \zh{人家不让去!}  

\lhead{\firstmark}
\rhead{\botmark}

\subsection{\hspace{-0.5cm} {\Large \textcolor{darkblue}{\textbf{\ipa{pv̩˩\textsubscript{a}}}} \textsubscript{3}}\hspace{0.5cm}[\kern2pt{\textcolor{darkblue}{\textbf{\ipa{pv̩˩˥}}}}\kern2pt]} \hypertarget{pv\string_=\string_Ba3}{}
\markboth{\textcolor{darkblue}{\textbf{\ipa{pv̩˩\textsubscript{a}}}} \textsubscript{3}}{}
\textcolor{teal}{\mytextsc{verbe}} \hspace{4pt} Ton~: L\textsubscript{a}.
\textcolor{PineGreen}{\selectlanguage{french}Peigner.} \zh{梳。}  ¶ \textcolor{darkblue}{\textbf{\ipa{ʁo˧qʰwɤ˩ pv̩˩}}} \textcolor{PineGreen}{\selectlanguage{french}se peigner} \zh{梳头}  
 ¶ \textcolor{darkblue}{\textbf{\ipa{ʁo˧ pv̩˥}}} \textcolor{PineGreen}{\selectlanguage{french}se peigner} \zh{梳头}  

\lhead{\firstmark}
\rhead{\botmark}

\subsection{\hspace{-0.5cm} {\Large \textcolor{darkblue}{\textbf{\ipa{pv̩˩ɭɯ˥}}}}\hspace{0.5cm}[\kern2pt{\textcolor{darkblue}{\textbf{\ipa{pv̩˩ɭɯ˥}}}}\kern2pt]} \hypertarget{pv\string_=\string_Bl\string_RM\string_T1}{}
\markboth{\textcolor{darkblue}{\textbf{\ipa{pv̩˩ɭɯ˥}}}}{}
\textcolor{teal}{\mytextsc{nom}} \hspace{4pt} Ton~: LH.
\textcolor{PineGreen}{\selectlanguage{french}Bouton (sur un vêtement).} \zh{扣子。}  \zh{量词}: \textcolor{darkblue}{\textbf{\ipa{ɭɯ˧}}}  \mytextsc{clf}~: \textcolor{darkblue}{\textbf{\ipa{ɭɯ˧}}} 
\lhead{\firstmark}
\rhead{\botmark}

\subsection{\hspace{-0.5cm} {\Large \textcolor{darkblue}{\textbf{\ipa{pv̩˧lv̩˧}}}}\hspace{0.5cm}[\kern2pt{\textcolor{darkblue}{\textbf{\ipa{pv̩˧lv̩˧}}}}\kern2pt]} \hypertarget{pv\string_=\string_Mlv\string_=\string_M1}{}
\markboth{\textcolor{darkblue}{\textbf{\ipa{pv̩˧lv̩˧}}}}{}
\textcolor{teal}{\mytextsc{nom}} \hspace{4pt} Ton~: M.
\textcolor{PineGreen}{\selectlanguage{french}Champ sec/pluvial.} \zh{旱地。}  \zh{量词}: \textcolor{darkblue}{\textbf{\ipa{pʰv̩˩}}}  \mytextsc{clf}~: \textcolor{darkblue}{\textbf{\ipa{pʰv̩˩}}} 
\lhead{\firstmark}
\rhead{\botmark}

\subsection{\hspace{-0.5cm} {\Large \textcolor{darkblue}{\textbf{\ipa{pv̩˩mi˩}}}}\hspace{0.5cm}[\kern2pt{\textcolor{darkblue}{\textbf{\ipa{pv̩˩mi˩˥}}}}\kern2pt]} \hypertarget{pv\string_=\string_Bmi\string_B1}{}
\markboth{\textcolor{darkblue}{\textbf{\ipa{pv̩˩mi˩}}}}{}
\textcolor{teal}{\mytextsc{nom}} \hspace{4pt} Ton~: L.
\textcolor{PineGreen}{\selectlanguage{french}Peigne grossier, à dents relativement écartées.} \zh{粗齿梳子。}  \zh{量词}: \textcolor{darkblue}{\textbf{\ipa{nɑ˧}}}  \mytextsc{clf}~: \textcolor{darkblue}{\textbf{\ipa{nɑ˧}}} 
\lhead{\firstmark}
\rhead{\botmark}

\subsection{\hspace{-0.5cm} {\Large \textcolor{darkblue}{\textbf{\ipa{pv̩˩pv̩˧}}}}\hspace{0.5cm}[\kern2pt{\textcolor{darkblue}{\textbf{\ipa{pv̩˩pv̩˥}}}}\kern2pt]} \hypertarget{pv\string_=\string_Bpv\string_=\string_M1}{}
\markboth{\textcolor{darkblue}{\textbf{\ipa{pv̩˩pv̩˧}}}}{}
\textcolor{teal}{\mytextsc{nom}} \hspace{4pt} Ton~: LM.
\textcolor{PineGreen}{\selectlanguage{french}Poche.} \zh{衣兜。}  ¶ \textcolor{darkblue}{\textbf{\ipa{bɑ˩lɑ˩-pv̩˥pv̩˩}}} \textcolor{PineGreen}{\selectlanguage{french}poche intérieure de chemise; on y serrait de petits objets: pièces de monnaie, tabac…} \zh{上衣兜子}  
\textit{Syn~:} \hyperlink{}{\textcolor{darkblue}{\textbf{\ipa{tɑ˩dv̩˧˥}}}}. 
\lhead{\firstmark}
\rhead{\botmark}

\subsection{\hspace{-0.5cm} {\Large \textcolor{darkblue}{\textbf{\ipa{pv̩˧qʰwɤ˥}}}}\hspace{0.5cm}[\kern2pt{\textcolor{darkblue}{\textbf{\ipa{pv̩˧qʰwɤ˥}}}}\kern2pt]} \hypertarget{pv\string_=\string_Mq\string_hw7\string_T1}{}
\markboth{\textcolor{darkblue}{\textbf{\ipa{pv̩˧qʰwɤ˥}}}}{}
\textcolor{teal}{\mytextsc{nom}} \hspace{4pt} Ton~: H\#.
\textcolor{PineGreen}{\selectlanguage{french}Navette du métier à tisser: navette traditionnelle, en bois (n'est plus en usage actuellement, remplacée par une navette plus simple).} \zh{梭,梭子(传统的,木头做的)。}  ¶ \textcolor{darkblue}{\textbf{\ipa{ɣɯ˧dzo˩-bv̩˩ | pv̩˧qʰwɤ˥}}} \textcolor{PineGreen}{\selectlanguage{french}la navette du métier à tisser} \zh{织布机的梭子}  
 \zh{量词}: \textcolor{darkblue}{\textbf{\ipa{ɭɯ˧}}}  \mytextsc{clf}~: \textcolor{darkblue}{\textbf{\ipa{ɭɯ˧}}} \textit{Voir~:} \hyperlink{}{\textcolor{darkblue}{\textbf{\ipa{kʰɯ˩pv̩˩}}}} 
\lhead{\firstmark}
\rhead{\botmark}

\subsection{\hspace{-0.5cm} {\Large \textcolor{darkblue}{\textbf{\ipa{pv̩˧ɻ\#˥}}}}\hspace{0.5cm}[\kern2pt{\textcolor{darkblue}{\textbf{\ipa{pv̩˧ɻ˧}}}}\kern2pt]} \hypertarget{pv\string_=\string_Mr£`\#\string_T1}{}
\markboth{\textcolor{darkblue}{\textbf{\ipa{pv̩˧ɻ\#˥}}}}{}
\textcolor{teal}{\mytextsc{nom}} \hspace{4pt} Ton~: \#H.
\textcolor{PineGreen}{\selectlanguage{french}Habit tibétain en laine (vêtement de grand prix).} \zh{氆氇。}  \zh{量词}: \textcolor{darkblue}{\textbf{\ipa{tsʰi˥}}}  \mytextsc{clf}~: \textcolor{darkblue}{\textbf{\ipa{tsʰi˥}}} 
\lhead{\firstmark}
\rhead{\botmark}

\subsection{\hspace{-0.5cm} {\Large \textcolor{darkblue}{\textbf{\ipa{pv̩˧ʂɯ˩}}}}\hspace{0.5cm}[\kern2pt{\textcolor{darkblue}{\textbf{\ipa{pv̩˧ʂɯ˩}}}}\kern2pt]} \hypertarget{pv\string_=\string_Ms`M\string_B1}{}
\markboth{\textcolor{darkblue}{\textbf{\ipa{pv̩˧ʂɯ˩}}}}{}
\textcolor{teal}{\mytextsc{nom}} \hspace{4pt} Ton~: L\#.
\textcolor{PineGreen}{\selectlanguage{french}Ambre.} \zh{琥珀。}  \zh{量词}: \textcolor{darkblue}{\textbf{\ipa{ɭɯ˧}}}  \mytextsc{clf}~: \textcolor{darkblue}{\textbf{\ipa{ɭɯ˧}}} 
\lhead{\firstmark}
\rhead{\botmark}

\subsection{\hspace{-0.5cm} {\Large \textcolor{darkblue}{\textbf{\ipa{pv̩˩tɑ˩}}}}\hspace{0.5cm}[\kern2pt{\textcolor{darkblue}{\textbf{\ipa{pv̩˩tɑ˩˥}}}}\kern2pt]} \hypertarget{pv\string_=\string_BtA\string_B1}{}
\markboth{\textcolor{darkblue}{\textbf{\ipa{pv̩˩tɑ˩}}}}{}
\textcolor{teal}{\mytextsc{nom}} \hspace{4pt} Ton~: L.
\textcolor{PineGreen}{\selectlanguage{french}Seau.} \zh{桶。}  \zh{量词}: \textcolor{darkblue}{\textbf{\ipa{ɭɯ˧}}}  \mytextsc{clf}~: \textcolor{darkblue}{\textbf{\ipa{ɭɯ˧}}} 
\lhead{\firstmark}
\rhead{\botmark}

\subsection{\hspace{-0.5cm} {\Large \textcolor{darkblue}{\textbf{\ipa{pv̩˩tsɯ˧˥}}}}\hspace{0.5cm}[\kern2pt{\textcolor{darkblue}{\textbf{\ipa{pv̩˩tsɯ˧˥}}}}\kern2pt]} \hypertarget{pv\string_=\string_BtsM\string_M\string_T1}{}
\markboth{\textcolor{darkblue}{\textbf{\ipa{pv̩˩tsɯ˧˥}}}}{}
\textcolor{teal}{\mytextsc{nom}} \hspace{4pt} Ton~: LM+MH\#.
\ding{202} \textcolor{PineGreen}{\selectlanguage{french}Peigne fin (utilisé pour épouiller).} \zh{篦子(用来梳虱子)。}  \zh{量词}: \textcolor{darkblue}{\textbf{\ipa{nɑ˧}}} \ding{203} \textcolor{PineGreen}{\selectlanguage{french}Fils de fer dans un cadre de bois: sorte de peigne dans lequel la trame est emprisonnée, et qui sert à tasser les fils à mesure que l'on tisse.} \zh{用来夯实布料的木头架子,里面有铁丝。}  \mytextsc{clf}~: \textcolor{darkblue}{\textbf{\ipa{nɑ˧}}} 
\lhead{\firstmark}
\rhead{\botmark}

\subsection{\hspace{-0.5cm} {\Large \textcolor{darkblue}{\textbf{\ipa{pv̩˧ʈʂɯ˩}}}}\hspace{0.5cm}[\kern2pt{\textcolor{darkblue}{\textbf{\ipa{pv̩˧ʈʂɯ˩}}}}\kern2pt]} \hypertarget{pv\string_=\string_Mt`s`M\string_B1}{}
\markboth{\textcolor{darkblue}{\textbf{\ipa{pv̩˧ʈʂɯ˩}}}}{}
\textcolor{teal}{\mytextsc{verbe}} \hspace{4pt} Ton~: L\#.
\textcolor{PineGreen}{\selectlanguage{french}Presser, serrer.} \zh{挤、挤压。}  ¶ \textcolor{darkblue}{\textbf{\ipa{njɤ˧-ɳɯ˧ | pv̩˧ʈʂɯ˩-bi˩!}}} \textcolor{PineGreen}{\selectlanguage{french}Je vais presser ça! / je m'occupe de serrer ça/presser ça!} \zh{我来压吧!}  
 ¶ \textcolor{darkblue}{\textbf{\ipa{le˧-pv̩˥ʈʂɯ˩}}} \textcolor{PineGreen}{\selectlanguage{french}\mytextsc{accomp}} \zh{\mytextsc{accomp}}  

\lhead{\firstmark}
\rhead{\botmark}

\subsection{\hspace{-0.5cm} {\Large \textcolor{darkblue}{\textbf{\ipa{pv̩˩tsɯ˧-pv̩˥mi˩}}}}\hspace{0.5cm}[\kern2pt{\textcolor{darkblue}{\textbf{\ipa{pv̩˩tsɯ˧pv̩˥mi˩}}}}\kern2pt]} \hypertarget{pv\string_=\string_BtsM\string_M-pv\string_=\string_Tmi\string_B1}{}
\markboth{\textcolor{darkblue}{\textbf{\ipa{pv̩˩tsɯ˧-pv̩˥mi˩}}}}{}
\textcolor{teal}{\mytextsc{nom}} \hspace{4pt} Ton~: LM+\#H-.
\textcolor{PineGreen}{\selectlanguage{french}Peignes.} \zh{梳子(总称)。} 
\lhead{\firstmark}
\rhead{\botmark}

\subsection{\hspace{-0.5cm} {\Large \textcolor{darkblue}{\textbf{\ipa{pv̩˩ʈʰɯ˧}}}}\hspace{0.5cm}[\kern2pt{\textcolor{darkblue}{\textbf{\ipa{pv̩˩ʈʰɯ˥}}}}\kern2pt]} \hypertarget{pv\string_=\string_Bt`\string_hM\string_M1}{}
\markboth{\textcolor{darkblue}{\textbf{\ipa{pv̩˩ʈʰɯ˧}}}}{}
\textcolor{teal}{\mytextsc{nom}} \hspace{4pt} Ton~: LM.
\textcolor{PineGreen}{\selectlanguage{french}Prénom féminin.} \zh{女性名字。} 
\lhead{\firstmark}
\rhead{\botmark}

\subsection{\hspace{-0.5cm} {\Large \textcolor{darkblue}{\textbf{\ipa{pv˧tsɤ\#˥}}}}\hspace{0.5cm}[\kern2pt{\textcolor{darkblue}{\textbf{\ipa{pv˧tsɤ˧}}}}\kern2pt]} \hypertarget{pv\string_Mts7\#\string_T1}{}
\markboth{\textcolor{darkblue}{\textbf{\ipa{pv˧tsɤ\#˥}}}}{}
\textcolor{teal}{\mytextsc{nom}} \hspace{4pt} Ton~: \#H.
\textcolor{PineGreen}{\selectlanguage{french}Mortaise.} \zh{榫眼。}  ¶ \textcolor{darkblue}{\textbf{\ipa{pv˧tsɤ˧ | ɖɯ˧-ɭɯ˧}}} \textcolor{PineGreen}{\selectlanguage{french}une mortaise} \zh{一个榫}  
 \zh{量词}: \textcolor{darkblue}{\textbf{\ipa{ɭɯ˧}}}  \mytextsc{clf}~: \textcolor{darkblue}{\textbf{\ipa{ɭɯ˧}}} 
\lhead{\firstmark}
\rhead{\botmark}

\newpage
\section*{\centering- \textcolor{darkblue}{\textbf{\ipa{pʰ}}} -}
\subsection{\hspace{-0.5cm} {\Large \textcolor{darkblue}{\textbf{\ipa{pʰæ˧\textsubscript{b}}}}}\hspace{0.5cm}[\kern2pt{\textcolor{darkblue}{\textbf{\ipa{pʰæ˥}}}}\kern2pt]} \hypertarget{p\string_h\{\string_Mb1}{}
\markboth{\textcolor{darkblue}{\textbf{\ipa{pʰæ˧\textsubscript{b}}}}}{}
\textcolor{teal}{\mytextsc{verbe}} \hspace{4pt} Ton~: M\textsubscript{b}.
\ding{202} \textcolor{PineGreen}{\selectlanguage{french}Attacher (un animal).} \zh{拴(牛……)。}  ¶ \textcolor{darkblue}{\textbf{\ipa{tʰi˧-pʰæ˧+ze˧}}} \textcolor{PineGreen}{\selectlanguage{french}\mytextsc{dur} \string_ \mytextsc{pfv}} \zh{\mytextsc{dur} \string_ \mytextsc{pfv}}  
 ¶ \textcolor{darkblue}{\textbf{\ipa{pʰæ˧\textasciitilde{}pʰæ˧}}} \textcolor{PineGreen}{\selectlanguage{french}\mytextsc{red}} \zh{\mytextsc{重叠}}  
\ding{203} \textcolor{PineGreen}{\selectlanguage{french}Être lié, avoir des liens étroits: par exemple, les membres d'une famille ont des liens profonds.} \zh{有联系,有缘分,有深的关系。}  ¶ \textcolor{darkblue}{\textbf{\ipa{pʰæ˧\textasciitilde{}pʰæ˧=ɻæ˩ ɲi˩!}}} \textcolor{PineGreen}{\selectlanguage{french}[Ils] sont unis, ils sont en couple! (au sujet de deux jeunes personnes)} \zh{他们有联系了/他们成了一俩了!(关于两个年轻人)}  

\lhead{\firstmark}
\rhead{\botmark}

\subsection{\hspace{-0.5cm} {\Large \textcolor{darkblue}{\textbf{\ipa{pʰæ˧qʰwɤ˩}}}}\hspace{0.5cm}[\kern2pt{\textcolor{darkblue}{\textbf{\ipa{pʰæ˩qʰwɤ˧˥}}}}\kern2pt]} \hypertarget{p\string_h\{\string_Mq\string_hw7\string_B1}{}
\markboth{\textcolor{darkblue}{\textbf{\ipa{pʰæ˧qʰwɤ˩}}}}{}
\textcolor{teal}{\mytextsc{nom}} \hspace{4pt} Ton~: L\#.
\textcolor{PineGreen}{\selectlanguage{french}Visage.} \zh{脸。}  \zh{量词}: \textcolor{darkblue}{\textbf{\ipa{ɭɯ˧}}}  \mytextsc{clf}~: \textcolor{darkblue}{\textbf{\ipa{ɭɯ˧}}} 
\lhead{\firstmark}
\rhead{\botmark}

\subsection{\hspace{-0.5cm} {\Large \textcolor{darkblue}{\textbf{\ipa{pʰæ˧ʂv̩˧-di˧˥}}}}\hspace{0.5cm}[\kern2pt{\textcolor{darkblue}{\textbf{\ipa{xxxx non-correspondance entre le nombre de morphèmes et le nombre de tons de morphèmes}}}}\kern2pt]} \hypertarget{p\string_h\{\string_Ms`v\string_=\string_M-di\string_M\string_T1}{}
\markboth{\textcolor{darkblue}{\textbf{\ipa{pʰæ˧ʂv̩˧-di˧˥}}}}{}
\textcolor{teal}{\mytextsc{nom}} \hspace{4pt} Ton~: .
\textcolor{PineGreen}{\selectlanguage{french}Foulard (périphrase); autrefois, on utilisait un fichu, \textcolor{darkblue}{\textbf{\ipa{/qʰwæ˧ʈɯ˥/}}}.} \zh{围巾。} 
\lhead{\firstmark}
\rhead{\botmark}

\subsection{\hspace{-0.5cm} {\Large \textcolor{darkblue}{\textbf{\ipa{pʰæ˧tɕi˥}}}}\hspace{0.5cm}[\kern2pt{\textcolor{darkblue}{\textbf{\ipa{pʰæ˧tɕi˥}}}}\kern2pt]} \hypertarget{p\string_h\{\string_Mts£i\string_T1}{}
\markboth{\textcolor{darkblue}{\textbf{\ipa{pʰæ˧tɕi˥}}}}{}
\textcolor{teal}{\mytextsc{nom}} \hspace{4pt} Ton~: H\#.
\ding{202} \textcolor{PineGreen}{\selectlanguage{french}Jeune homme, petit gars.} \zh{小伙子、 青年男子。}  ¶ \textcolor{darkblue}{\textbf{\ipa{pʰæ˧tɕi˥-zo˩}}} \textcolor{PineGreen}{\selectlanguage{french}jeune homme} \zh{小伙子}  
 ¶ \textcolor{darkblue}{\textbf{\ipa{pʰæ˧tɕi˥=ɻæ˩}}} \textcolor{PineGreen}{\selectlanguage{french}jeunes hommes; les jeunes hommes} \zh{小伙子们}  
 \zh{量词}: \textcolor{darkblue}{\textbf{\ipa{v̩˧}}} \ding{203} \textcolor{PineGreen}{\selectlanguage{french}Nom du premier pilier (il y a deux grands piliers dans la maison traditionnelle), celui qui est le plus près de la porte: c'est le pilier masculin, “le jeune homme”, le second étant féminin, “la jeune femme”.} \zh{第一根柱子的名称(代表男人、男性的那根柱子)。}  \zh{量词}: \textcolor{darkblue}{\textbf{\ipa{v̩˧}}}  \mytextsc{clf}~: \textcolor{darkblue}{\textbf{\ipa{v̩˧}}} \textcolor{darkblue}{\textbf{\ipa{v̩˧}}} 
\lhead{\firstmark}
\rhead{\botmark}

\subsection{\hspace{-0.5cm} {\Large \textcolor{darkblue}{\textbf{\ipa{pʰæ˧ʈʂʰæ˧lo\#˥}}}}\hspace{0.5cm}[\kern2pt{\textcolor{darkblue}{\textbf{\ipa{pʰæ˧ʈʂʰæ˧lo˩}}}}\kern2pt]} \hypertarget{p\string_h\{\string_Mt`s`\string_h\{\string_Mlo\#\string_T1}{}
\markboth{\textcolor{darkblue}{\textbf{\ipa{pʰæ˧ʈʂʰæ˧lo\#˥}}}}{}
\textcolor{teal}{\mytextsc{nom}} \hspace{4pt} Ton~: \#H.
\textcolor{PineGreen}{\selectlanguage{french}Bassine pour se laver le visage; le même ustensile est utilisé pour servir les nourritures si un bol serait trop petit: pour servir le riz, les soupes...} \zh{脸盆,木盆。}  \zh{量词}: \textcolor{darkblue}{\textbf{\ipa{ɭɯ˧}}}  \mytextsc{clf}~: \textcolor{darkblue}{\textbf{\ipa{ɭɯ˧}}} 
\lhead{\firstmark}
\rhead{\botmark}

\subsection{\hspace{-0.5cm} {\Large \textcolor{darkblue}{\textbf{\ipa{pʰæ˧˥}}}}\hspace{0.5cm}[\kern2pt{\textcolor{darkblue}{\textbf{\ipa{pʰæ˧˥}}}}\kern2pt]} \hypertarget{p\string_h\{\string_M\string_T1}{}
\markboth{\textcolor{darkblue}{\textbf{\ipa{pʰæ˧˥}}}}{}
\textcolor{teal}{\mytextsc{verbe}} \hspace{4pt} Ton~: MH.
\textcolor{PineGreen}{\selectlanguage{french}Écarter, pousser, jouer des coudes.} \zh{推搡。}  ¶ \textcolor{darkblue}{\textbf{\ipa{ɖɯ˩-tɕo˧ pʰæ˧˥, | ʈʂʰɯ˧-tɕo˧ pʰæ˧˥}}} \textcolor{PineGreen}{\selectlanguage{french}pousser par ici, pousser par là / jouer des coudes par ci, jouer des coudes par là (ex.: à la gare, quand il y a presse pour acheter un billet de train)} \zh{东推西挤}  
 ¶ \textcolor{darkblue}{\textbf{\ipa{[Housebuilding2] ʈʂe˧ | le˧-pʰæ˩\textasciitilde{}pʰæ˩}}} \textcolor{PineGreen}{\selectlanguage{french}rejeter la terre de droite et de gauche : une poule gratte la terre à la recherche de nourriture, et fait voler de la terre de droite et de gauche} \zh{将土扔这里扔那里:一只鸡在抓地找吃的,让土飞这里飞那里}  

\lhead{\firstmark}
\rhead{\botmark}

\subsection{\hspace{-0.5cm} {\Large \textcolor{darkblue}{\textbf{\ipa{pʰæ˧˥\textsubscript{a}}}}}\hspace{0.5cm}[\kern2pt{\textcolor{darkblue}{\textbf{\ipa{pʰæ˧˥}}}}\kern2pt]} \hypertarget{p\string_h\{\string_M\string_Ta1}{}
\markboth{\textcolor{darkblue}{\textbf{\ipa{pʰæ˧˥\textsubscript{a}}}}}{}
\textcolor{teal}{\mytextsc{classificateur}} \hspace{4pt} Ton~: MH\textsubscript{a}.
\textcolor{PineGreen}{\selectlanguage{french}Classificateur des objets plats: feuilles de papier….} \zh{量词:平面的东西,如:纸(一张、一片)。} 
\lhead{\firstmark}
\rhead{\botmark}

\subsection{\hspace{-0.5cm} {\Large \textcolor{darkblue}{\textbf{\ipa{pʰe˧}}}}\hspace{0.5cm}[\kern2pt{\textcolor{darkblue}{\textbf{\ipa{pʰe˥}}}}\kern2pt]} \hypertarget{p\string_he\string_M1}{}
\markboth{\textcolor{darkblue}{\textbf{\ipa{pʰe˧}}}}{}
\textcolor{teal}{\mytextsc{interjection}} \hspace{4pt} Ton~: M.
\textcolor{PineGreen}{\selectlanguage{french}Mais pas du tout! Mais non, enfin! Interjection par laquelle le locuteur signale qu'il reprend la main: que ses interlocuteurs lui paraissent être dans l'erreur, et qu'il va rectifier.} \zh{呸!(表示唾弃的感叹词)。} 
\lhead{\firstmark}
\rhead{\botmark}

\subsection{\hspace{-0.5cm} {\Large \textcolor{darkblue}{\textbf{\ipa{pʰɤ˧bɤ˧}}}}\hspace{0.5cm}[\kern2pt{\textcolor{darkblue}{\textbf{\ipa{pʰɤ˧bɤ˧}}}}\kern2pt]} \hypertarget{p\string_h7\string_Mb7\string_M1}{}
\markboth{\textcolor{darkblue}{\textbf{\ipa{pʰɤ˧bɤ˧}}}}{}
\textcolor{teal}{\mytextsc{nom}} \hspace{4pt} Ton~: M.
\textcolor{PineGreen}{\selectlanguage{french}Cadeau (choses à manger ou boire; essentiellement: tabac, thé, bonbons, vin; on n'offre généralement pas de vêtements).} \zh{礼物。}  ¶ \textcolor{darkblue}{\textbf{\ipa{pʰɤ˧bɤ˧ po˧-tsʰɯ˧˥}}} \textcolor{PineGreen}{\selectlanguage{french}amener des cadeaux} \zh{带礼物}  
 ¶ \textcolor{darkblue}{\textbf{\ipa{ʈʂʰɯ˧ | ʈæ˧ʂɯ˧ ki˧-hĩ˧ pʰɤ˧bɤ˧ ŋi˩.}}} \textcolor{PineGreen}{\selectlanguage{french}C'est un cadeau que m'a donné Dashi!} \zh{这是达石给的礼物!}  
 ¶ \textcolor{darkblue}{\textbf{\ipa{ʈʂʰɯ˧ | ʈæ˧ʂɯ˧ tʰi˧-ki˧-hĩ˧ pʰɤ˧bɤ˧ ŋi˩.}}} \textcolor{PineGreen}{\selectlanguage{french}C'est un cadeau que Dashi te fait! Voici un cadeau de la part de Dashi!} \zh{这是达石送你的礼物!}  
 \zh{量词}: \textcolor{darkblue}{\textbf{\ipa{kʰwɤ˥}}}  \mytextsc{clf}~: \textcolor{darkblue}{\textbf{\ipa{kʰwɤ˥}}} 
\lhead{\firstmark}
\rhead{\botmark}

\subsection{\hspace{-0.5cm} {\Large \textcolor{darkblue}{\textbf{\ipa{pʰɤ˧fv̩˩}}}}\hspace{0.5cm}[\kern2pt{\textcolor{darkblue}{\textbf{\ipa{pʰɤ˧fv̩˩}}}}\kern2pt]} \hypertarget{p\string_h7\string_Mfv\string_=\string_B1}{}
\markboth{\textcolor{darkblue}{\textbf{\ipa{pʰɤ˧fv̩˩}}}}{}
\textcolor{teal}{\mytextsc{nom}} \hspace{4pt} Ton~: L\#.
\textcolor{PineGreen}{\selectlanguage{french}Théière.} \zh{茶壶。}  Emprunt~: chinois  \zh{壶?}

\lhead{\firstmark}
\rhead{\botmark}

\subsection{\hspace{-0.5cm} {\Large \textcolor{darkblue}{\textbf{\ipa{pʰɤ˧pʰv̩\#˥}}}}\hspace{0.5cm}[\kern2pt{\textcolor{darkblue}{\textbf{\ipa{pʰɤ˩pʰv̩˩˥}}}}\kern2pt]} \hypertarget{p\string_h7\string_Mp\string_hv\string_=\#\string_T1}{}
\markboth{\textcolor{darkblue}{\textbf{\ipa{pʰɤ˧pʰv̩\#˥}}}}{}
\textcolor{teal}{\mytextsc{nom}} \hspace{4pt} Ton~: \#H.
\textcolor{PineGreen}{\selectlanguage{french}Chacal mâle.} \zh{公豺。}  \zh{量词}: \textcolor{darkblue}{\textbf{\ipa{mi˩}}}  \mytextsc{clf}~: \textcolor{darkblue}{\textbf{\ipa{mi˩}}} 
\lhead{\firstmark}
\rhead{\botmark}

\subsection{\hspace{-0.5cm} {\Large \textcolor{darkblue}{\textbf{\ipa{pʰɤ˩mi˩}}}}\hspace{0.5cm}[\kern2pt{\textcolor{darkblue}{\textbf{\ipa{pʰɤ˩mi˩˥}}}}\kern2pt]} \hypertarget{p\string_h7\string_Bmi\string_B1}{}
\markboth{\textcolor{darkblue}{\textbf{\ipa{pʰɤ˩mi˩}}}}{}
\textcolor{teal}{\mytextsc{nom}} \hspace{4pt} Ton~: L.
\textcolor{PineGreen}{\selectlanguage{french}Femelle chacal.} \zh{母豺。}  \zh{量词}: \textcolor{darkblue}{\textbf{\ipa{mi˩}}}  \mytextsc{clf}~: \textcolor{darkblue}{\textbf{\ipa{mi˩}}} 
\lhead{\firstmark}
\rhead{\botmark}

\subsection{\hspace{-0.5cm} {\Large \textcolor{darkblue}{\textbf{\ipa{pʰɤ˩-so˩\textasciitilde{}so˥}}}}\hspace{0.5cm}[\kern2pt{\textcolor{darkblue}{\textbf{\ipa{xxxx non-correspondance entre le nombre de morphèmes et le nombre de tons de morphèmes}}}}\kern2pt]} \hypertarget{p\string_h7\string_B-so\string_B~so\string_T1}{}
\markboth{\textcolor{darkblue}{\textbf{\ipa{pʰɤ˩-so˩\textasciitilde{}so˥}}}}{}
\textcolor{teal}{\mytextsc{adjectif}} \hspace{4pt} Ton~: .
\textcolor{PineGreen}{\selectlanguage{french}Meuble: la terre est meuble.} \zh{松(土)。}  ¶ \textcolor{darkblue}{\textbf{\ipa{ʈʂe˧ | pʰɤ˩-so˩\textasciitilde{}so˥-gv̩˩}}} \textcolor{PineGreen}{\selectlanguage{french}la terre est meuble, la terre a été ameublie} \zh{土是松的}  

\lhead{\firstmark}
\rhead{\botmark}

\subsection{\hspace{-0.5cm} {\Large \textcolor{darkblue}{\textbf{\ipa{pʰɤ˩zo˩}}}}\hspace{0.5cm}[\kern2pt{\textcolor{darkblue}{\textbf{\ipa{pʰɤ˩zo˩˥}}}}\kern2pt]} \hypertarget{p\string_h7\string_Bzo\string_B1}{}
\markboth{\textcolor{darkblue}{\textbf{\ipa{pʰɤ˩zo˩}}}}{}
\textcolor{teal}{\mytextsc{nom}} \hspace{4pt} Ton~: L.
\textcolor{PineGreen}{\selectlanguage{french}Petit chacal.} \zh{豺崽子。}  \zh{量词}: \textcolor{darkblue}{\textbf{\ipa{ɭɯ˧}}}  \mytextsc{clf}~: \textcolor{darkblue}{\textbf{\ipa{ɭɯ˧}}} 
\lhead{\firstmark}
\rhead{\botmark}

\subsection{\hspace{-0.5cm} {\Large \textcolor{darkblue}{\textbf{\ipa{pʰɤ˩˧}}}}\hspace{0.5cm}[\kern2pt{\textcolor{darkblue}{\textbf{\ipa{pʰɤ˩˥}}}}\kern2pt]} \hypertarget{p\string_h7\string_B\string_M1}{}
\markboth{\textcolor{darkblue}{\textbf{\ipa{pʰɤ˩˧}}}}{}
\textcolor{teal}{\mytextsc{nom}} \hspace{4pt} Ton~: LM.
\textcolor{PineGreen}{\selectlanguage{french}Hyène, chacal.} \zh{豺。}  ¶ \textcolor{darkblue}{\textbf{\ipa{pʰɤ˩ hwæ˧-ze˧}}} \textcolor{PineGreen}{\selectlanguage{french}...a acheté (une) hyène} \zh{买了豺}  
 ¶ \textcolor{darkblue}{\textbf{\ipa{pʰɤ˩ dzɯ˧-ze˩}}} \textcolor{PineGreen}{\selectlanguage{french}...a mangé (une) hyène} \zh{吃了豺}  
 \zh{量词}: \textcolor{darkblue}{\textbf{\ipa{mi˩}}}  \mytextsc{clf}~: \textcolor{darkblue}{\textbf{\ipa{mi˩}}} 
\lhead{\firstmark}
\rhead{\botmark}

\subsection{\hspace{-0.5cm} {\Large \textcolor{darkblue}{\textbf{\ipa{pʰi˧}}}}\hspace{0.5cm}[\kern2pt{\textcolor{darkblue}{\textbf{\ipa{pʰi˥}}}}\kern2pt]} \hypertarget{p\string_hi\string_M1}{}
\markboth{\textcolor{darkblue}{\textbf{\ipa{pʰi˧}}}}{}
\textcolor{teal}{\mytextsc{nom}} \hspace{4pt} Ton~: M.
\textcolor{PineGreen}{\selectlanguage{french}Tissu de lin; anciennement le tissu dont étaient faits ts les vêtements (cf récit “travaux”).} \zh{麻布,亚麻布。}  ¶ \textcolor{darkblue}{\textbf{\ipa{pʰi˩ dɑ˩˥}}} \textcolor{PineGreen}{\selectlanguage{french}tisser le lin, faire du tissu de lin} \zh{织麻布}  
 \zh{量词}: \textcolor{darkblue}{\textbf{\ipa{kʰwɤ˥}}}  \mytextsc{clf}~: \textcolor{darkblue}{\textbf{\ipa{kʰwɤ˥}}} 
\lhead{\firstmark}
\rhead{\botmark}

\subsection{\hspace{-0.5cm} {\Large \textcolor{darkblue}{\textbf{\ipa{pʰi˧\textsubscript{b}}}}}\hspace{0.5cm}[\kern2pt{\textcolor{darkblue}{\textbf{\ipa{pʰi˧˥}}}}\kern2pt]} \hypertarget{p\string_hi\string_Mb1}{}
\markboth{\textcolor{darkblue}{\textbf{\ipa{pʰi˧\textsubscript{b}}}}}{}
\textcolor{teal}{\mytextsc{verbe}} \hspace{4pt} Ton~: M\textsubscript{b}.
\textcolor{PineGreen}{\selectlanguage{french}Vanner à l'aide d'un crible (vannerie ronde): on fait “sauter” le grain dans un crible, et le vent emporte la balle.} \zh{簸。}  ¶ \textcolor{darkblue}{\textbf{\ipa{hɑ˧ pʰi˧}}} \textcolor{PineGreen}{\selectlanguage{french}vanner du grain} \zh{簸粮食}  
 ¶ \textcolor{darkblue}{\textbf{\ipa{pʰi˧\textasciitilde{}pʰi˧}}} \textcolor{PineGreen}{\selectlanguage{french}\mytextsc{red}} \zh{\mytextsc{重叠:簸一簸}}  
 ¶ \textcolor{darkblue}{\textbf{\ipa{le˧-pʰi˧(-ze˧)}}} \textcolor{PineGreen}{\selectlanguage{french}\mytextsc{accomp} \string_ (\mytextsc{pfv})} \zh{簸了}  

\lhead{\firstmark}
\rhead{\botmark}

\subsection{\hspace{-0.5cm} {\Large \textcolor{darkblue}{\textbf{\ipa{pʰi˧kʰv̩˧}}}}\hspace{0.5cm}[\kern2pt{\textcolor{darkblue}{\textbf{\ipa{pʰi˩kʰv̩˥}}}}\kern2pt]} \hypertarget{p\string_hi\string_Mk\string_hv\string_=\string_M1}{}
\markboth{\textcolor{darkblue}{\textbf{\ipa{pʰi˧kʰv̩˧}}}}{}
\textcolor{teal}{\mytextsc{nom}} \hspace{4pt} Ton~: M.
\textcolor{PineGreen}{\selectlanguage{french}Pelle à poussière.} \zh{畚箕。}  \zh{量词}: \textcolor{darkblue}{\textbf{\ipa{nɑ˧}}}  \mytextsc{clf}~: \textcolor{darkblue}{\textbf{\ipa{nɑ˧}}} 
\lhead{\firstmark}
\rhead{\botmark}

\subsection{\hspace{-0.5cm} {\Large \textcolor{darkblue}{\textbf{\ipa{pʰi˧kʰv̩˧}}}}\hspace{0.5cm}[\kern2pt{\textcolor{darkblue}{\textbf{\ipa{pʰi˧kʰv̩˧}}}}\kern2pt]} \hypertarget{p\string_hi\string_Mk\string_hv\string_=\string_M1}{}
\markboth{\textcolor{darkblue}{\textbf{\ipa{pʰi˧kʰv̩˧}}}}{}
\textcolor{teal}{\mytextsc{nom}} \hspace{4pt} Ton~: M.
\textcolor{PineGreen}{\selectlanguage{french}Coquillage.} \zh{贝壳。} 
\lhead{\firstmark}
\rhead{\botmark}

\subsection{\hspace{-0.5cm} {\Large \textcolor{darkblue}{\textbf{\ipa{pʰi˧li˩}}}}\hspace{0.5cm}[\kern2pt{\textcolor{darkblue}{\textbf{\ipa{pʰi˧li˧}}}}\kern2pt]} \hypertarget{p\string_hi\string_Mli\string_B1}{}
\markboth{\textcolor{darkblue}{\textbf{\ipa{pʰi˧li˩}}}}{}
\textcolor{teal}{\mytextsc{nom}} \hspace{4pt} Ton~: L\#.
\textcolor{PineGreen}{\selectlanguage{french}Papillon.} \zh{蝴蝶。}  \zh{量词}: \textcolor{darkblue}{\textbf{\ipa{mi˩}}}  \mytextsc{clf}~: \textcolor{darkblue}{\textbf{\ipa{mi˩}}} 
\lhead{\firstmark}
\rhead{\botmark}

\subsection{\hspace{-0.5cm} {\Large \textcolor{darkblue}{\textbf{\ipa{pʰi˧mo˩}}} \textsubscript{1}}\hspace{0.5cm}[\kern2pt{\textcolor{darkblue}{\textbf{\ipa{pʰi˧mo˩}}}}\kern2pt]} \hypertarget{p\string_hi\string_Mmo\string_B1}{}
\markboth{\textcolor{darkblue}{\textbf{\ipa{pʰi˧mo˩}}} \textsubscript{1}}{}
\textcolor{teal}{\mytextsc{nom}} \hspace{4pt} Ton~: L\#.
\textcolor{PineGreen}{\selectlanguage{french}Vanneuse.} \zh{簸箕(用来簸粮食等)。}  \zh{量词}: \textcolor{darkblue}{\textbf{\ipa{nɑ˧}}}  \mytextsc{clf}~: \textcolor{darkblue}{\textbf{\ipa{nɑ˧}}} \textit{Voir~:} \hyperlink{}{\textcolor{darkblue}{\textbf{\ipa{pʰi˧mo˩}}} \textsubscript{2}} 
\lhead{\firstmark}
\rhead{\botmark}

\subsection{\hspace{-0.5cm} {\Large \textcolor{darkblue}{\textbf{\ipa{pʰi˧mo˩}}} \textsubscript{2}}\hspace{0.5cm}[\kern2pt{\textcolor{darkblue}{\textbf{\ipa{pʰi˧mo˩}}}}\kern2pt]} \hypertarget{p\string_hi\string_Mmo\string_B2}{}
\markboth{\textcolor{darkblue}{\textbf{\ipa{pʰi˧mo˩}}} \textsubscript{2}}{}
\textcolor{teal}{\mytextsc{nom}} \hspace{4pt} Ton~: L\#.
\textcolor{PineGreen}{\selectlanguage{french}Piège pour attraper des oiseaux.} \zh{抓鸟的圈套。}  ¶ \textcolor{darkblue}{\textbf{\ipa{v̩˩dze˩ qo˥-di˩, | pʰi˧mo˩!}}} \textcolor{PineGreen}{\selectlanguage{french}Ce dont on se sert pour attraper les oiseaux, on appelle ça “piège pour oiseaux”!} \zh{抓鸟的东西,(叫做)圈套!}  
\textit{Voir~:} \hyperlink{}{\textcolor{darkblue}{\textbf{\ipa{pʰi˧mo˩}}} \textsubscript{1}} 
\lhead{\firstmark}
\rhead{\botmark}

\subsection{\hspace{-0.5cm} {\Large \textcolor{darkblue}{\textbf{\ipa{pʰi˧tsʰo\#˥}}}}\hspace{0.5cm}[\kern2pt{\textcolor{darkblue}{\textbf{\ipa{pʰi˧tsʰo˧}}}}\kern2pt]} \hypertarget{p\string_hi\string_Mts\string_ho\#\string_T1}{}
\markboth{\textcolor{darkblue}{\textbf{\ipa{pʰi˧tsʰo\#˥}}}}{}
\textcolor{teal}{\mytextsc{nom}} \hspace{4pt} Ton~: \#H.
\textcolor{PineGreen}{\selectlanguage{french}Prénom masculin.} \zh{男性名字。} 
\lhead{\firstmark}
\rhead{\botmark}

\subsection{\hspace{-0.5cm} {\Large \textcolor{darkblue}{\textbf{\ipa{pʰi˧ʈʂæ˧}}}}\hspace{0.5cm}[\kern2pt{\textcolor{darkblue}{\textbf{\ipa{pʰi˧ʈʂæ˧}}}}\kern2pt]} \hypertarget{p\string_hi\string_Mt`s`\{\string_M1}{}
\markboth{\textcolor{darkblue}{\textbf{\ipa{pʰi˧ʈʂæ˧}}}}{}
\textcolor{teal}{\mytextsc{verbe}} \hspace{4pt} Ton~: M.
\textcolor{PineGreen}{\selectlanguage{french}Se draper d'un feutre.} \zh{披毡(汉语借词)。}  Emprunt~: chinois  \zh{披毡}

\lhead{\firstmark}
\rhead{\botmark}

\subsection{\hspace{-0.5cm} {\Large \textcolor{darkblue}{\textbf{\ipa{pʰi˩}}}}\hspace{0.5cm}[\kern2pt{\textcolor{darkblue}{\textbf{\ipa{pʰi˥}}}}\kern2pt]} \hypertarget{p\string_hi\string_B1}{}
\markboth{\textcolor{darkblue}{\textbf{\ipa{pʰi˩}}}}{}
\textcolor{teal}{\mytextsc{adjectif}} \hspace{4pt} Ton~: L.
\textcolor{PineGreen}{\selectlanguage{french}Plat.} \zh{平(汉语借词)。}  Emprunt~: chinois  \zh{平}

\lhead{\firstmark}
\rhead{\botmark}

\subsection{\hspace{-0.5cm} {\Large \textcolor{darkblue}{\textbf{\ipa{pʰi˩hæ˩}}}}\hspace{0.5cm}[\kern2pt{\textcolor{darkblue}{\textbf{\ipa{pʰi˧hæ˧}}}}\kern2pt]} \hypertarget{p\string_hi\string_Bh\{\string_B1}{}
\markboth{\textcolor{darkblue}{\textbf{\ipa{pʰi˩hæ˩}}}}{}
\textcolor{teal}{\mytextsc{nom}} \hspace{4pt} Ton~: L.
\textcolor{PineGreen}{\selectlanguage{french}Sandale.} \zh{凉鞋。汉语借词:第一个音节:皮,第二个音节:不明确,同\textcolor{darkblue}{\textbf{\ipa{/tɕæ˧hæ˩/}}}。}  Emprunt~: chinois  \zh{皮} (second syllable: not identified)
 \zh{量词}: \textcolor{darkblue}{\textbf{\ipa{dzi˧}}}  \mytextsc{clf}~: \textcolor{darkblue}{\textbf{\ipa{dzi˧}}} \textcolor{PineGreen}{\selectlanguage{french}paire}
\lhead{\firstmark}
\rhead{\botmark}

\subsection{\hspace{-0.5cm} {\Large \textcolor{darkblue}{\textbf{\ipa{pʰi˩ko˧}}}}\hspace{0.5cm}[\kern2pt{\textcolor{darkblue}{\textbf{\ipa{pʰi˩ko˩˥}}}}\kern2pt]} \hypertarget{p\string_hi\string_Bko\string_M1}{}
\markboth{\textcolor{darkblue}{\textbf{\ipa{pʰi˩ko˧}}}}{}
\textcolor{teal}{\mytextsc{nom}} \hspace{4pt} Ton~: LM.
\textcolor{PineGreen}{\selectlanguage{french}Pomme.} \zh{苹果。}  Emprunt~: chinois  \zh{苹果}
 \zh{量词}: \textcolor{darkblue}{\textbf{\ipa{ɭɯ˧}}}  \mytextsc{clf}~: \textcolor{darkblue}{\textbf{\ipa{ɭɯ˧}}} 
\lhead{\firstmark}
\rhead{\botmark}

\subsection{\hspace{-0.5cm} {\Large \textcolor{darkblue}{\textbf{\ipa{pʰi˩tʰɑ˩}}}}\hspace{0.5cm}[\kern2pt{\textcolor{darkblue}{\textbf{\ipa{pʰi˩tʰɑ˩˥}}}}\kern2pt]} \hypertarget{p\string_hi\string_Bt\string_hA\string_B1}{}
\markboth{\textcolor{darkblue}{\textbf{\ipa{pʰi˩tʰɑ˩}}}}{}
\textcolor{teal}{\mytextsc{nom}} \hspace{4pt} Ton~: L.
\textcolor{PineGreen}{\selectlanguage{french}Amadou.} \zh{火草。}  \zh{量词}: \textcolor{darkblue}{\textbf{\ipa{po˧}}}  \mytextsc{clf}~: \textcolor{darkblue}{\textbf{\ipa{po˧}}} 
\lhead{\firstmark}
\rhead{\botmark}

\subsection{\hspace{-0.5cm} {\Large \textcolor{darkblue}{\textbf{\ipa{pʰi˧˥}}}}\hspace{0.5cm}[\kern2pt{\textcolor{darkblue}{\textbf{\ipa{pʰi˥}}}}\kern2pt]} \hypertarget{p\string_hi\string_M\string_T1}{}
\markboth{\textcolor{darkblue}{\textbf{\ipa{pʰi˧˥}}}}{}
\textcolor{teal}{\mytextsc{verbe}} \hspace{4pt} Ton~: MH.
\textcolor{PineGreen}{\selectlanguage{french}Vomir.} \zh{呕吐。}  ¶ \textcolor{darkblue}{\textbf{\ipa{le˧-pʰi˧-ze˥}}} \textcolor{PineGreen}{\selectlanguage{french}\mytextsc{accomp} \string_ \mytextsc{pfv}} \zh{呕吐了}  

\lhead{\firstmark}
\rhead{\botmark}

\subsection{\hspace{-0.5cm} {\Large \textcolor{darkblue}{\textbf{\ipa{pʰo˥}}}}\hspace{0.5cm}[\kern2pt{\textcolor{darkblue}{\textbf{\ipa{pʰo˥}}}}\kern2pt]} \hypertarget{p\string_ho\string_T1}{}
\markboth{\textcolor{darkblue}{\textbf{\ipa{pʰo˥}}}}{}
\textcolor{teal}{\mytextsc{verbe}} \hspace{4pt} Ton~: H.
\textcolor{PineGreen}{\selectlanguage{french}Se draper de, endosser, mettre sur son dos. Le fait de porter un vêtement sur les épaules sans le boutonner était considéré comme mal élevé: seuls les voleurs gardent la veste ouverte, pour y fourrer subrepticement leur butin.} \zh{披上(不系扣子)。}  ¶ \textcolor{darkblue}{\textbf{\ipa{mɤ˧-pʰo˥}}} \textcolor{PineGreen}{\selectlanguage{french}\mytextsc{neg}} \zh{不披}  
 ¶ \textcolor{darkblue}{\textbf{\ipa{bɑ˩lɑ˩ qɑ˩-pʰo˩˥}}} \textcolor{PineGreen}{\selectlanguage{french}endosser un habit, se mettre un habit sur les épaules (sans boutonner)} \zh{披上衣服(不系扣子)}  

\lhead{\firstmark}
\rhead{\botmark}

\subsection{\hspace{-0.5cm} {\Large \textcolor{darkblue}{\textbf{\ipa{pʰo˧\textsubscript{b}}}}}\hspace{0.5cm}[\kern2pt{\textcolor{darkblue}{\textbf{\ipa{pʰo˥}}}}\kern2pt]} \hypertarget{p\string_ho\string_Mb1}{}
\markboth{\textcolor{darkblue}{\textbf{\ipa{pʰo˧\textsubscript{b}}}}}{}
\textcolor{teal}{\mytextsc{verbe}} \hspace{4pt} Ton~: M\textsubscript{b}.
\textcolor{PineGreen}{\selectlanguage{french}Ouvrir (ex.: porte).} \zh{打开(例如:开门)。}  ¶ \textcolor{darkblue}{\textbf{\ipa{gɤ˩-pʰo˧ (-ze˧)}}} \textcolor{PineGreen}{\selectlanguage{french}ouvrir} \zh{打开}  
 ¶ \textcolor{darkblue}{\textbf{\ipa{kʰi˧ pʰo˧}}} \textcolor{PineGreen}{\selectlanguage{french}ouvrir la porte} \zh{开门}  
 ¶ \textcolor{darkblue}{\textbf{\ipa{kʰi˧mi˧ le˧-pʰo˧}}} \textcolor{PineGreen}{\selectlanguage{french}ouvrir la porte} \zh{开门}  
 ¶ \textcolor{darkblue}{\textbf{\ipa{tso˧\textasciitilde{}tso˧ pʰo˧}}} \textcolor{PineGreen}{\selectlanguage{french}ouvrir quelque chose} \zh{打开东西}  

\lhead{\firstmark}
\rhead{\botmark}

\subsection{\hspace{-0.5cm} {\Large \textcolor{darkblue}{\textbf{\ipa{pʰo˩\textsubscript{a}}}}}\hspace{0.5cm}[\kern2pt{\textcolor{darkblue}{\textbf{\ipa{pʰo˩˥}}}}\kern2pt]} \hypertarget{p\string_ho\string_Ba1}{}
\markboth{\textcolor{darkblue}{\textbf{\ipa{pʰo˩\textsubscript{a}}}}}{}
\textcolor{teal}{\mytextsc{verbe}} \hspace{4pt} Ton~: L\textsubscript{a}.
\textcolor{PineGreen}{\selectlanguage{french}S'échapper, s'enfuir; détaler.} \zh{逃,逃跑,逃掉。}  ¶ \textcolor{darkblue}{\textbf{\ipa{le˧-pʰo˩-ze˩}}} \textcolor{PineGreen}{\selectlanguage{french}\mytextsc{accomp} \string_ \mytextsc{pfv}} \zh{逃跑了}  
 ¶ \textcolor{darkblue}{\textbf{\ipa{le˧-pʰo˩-hɯ˩-ze˩!}}} \textcolor{PineGreen}{\selectlanguage{french}(Elle/il) s'est enfui(e)!} \zh{(他)逃跑了!}  

\lhead{\firstmark}
\rhead{\botmark}

\subsection{\hspace{-0.5cm} {\Large \textcolor{darkblue}{\textbf{\ipa{pʰo˩lɑ˧˥}}}}\hspace{0.5cm}[\kern2pt{\textcolor{darkblue}{\textbf{\ipa{pʰo˧lɑ˩}}}}\kern2pt]} \hypertarget{p\string_ho\string_BlA\string_M\string_T1}{}
\markboth{\textcolor{darkblue}{\textbf{\ipa{pʰo˩lɑ˧˥}}}}{}
\textcolor{teal}{\mytextsc{verbe}} \hspace{4pt} Ton~: LM+MH\#.
\textcolor{PineGreen}{\selectlanguage{french}Faire la guerre.} \zh{战争、打仗。}  ¶ \textcolor{darkblue}{\textbf{\ipa{mɤ˧-pʰo˩lɑ˩}}} \textcolor{PineGreen}{\selectlanguage{french}\mytextsc{neg}} \zh{不打仗}  
 ¶ \textcolor{darkblue}{\textbf{\ipa{pʰo˩lɑ˧˥ | ɖɯ˧-kʰv̩˧˥}}} \textcolor{PineGreen}{\selectlanguage{french}une année de guerre} \zh{打仗的一年}  

\lhead{\firstmark}
\rhead{\botmark}

\subsection{\hspace{-0.5cm} {\Large \textcolor{darkblue}{\textbf{\ipa{pʰo˧˥}}}}\hspace{0.5cm}[\kern2pt{\textcolor{darkblue}{\textbf{\ipa{pʰo˧˥}}}}\kern2pt]} \hypertarget{p\string_ho\string_M\string_T1}{}
\markboth{\textcolor{darkblue}{\textbf{\ipa{pʰo˧˥}}}}{}
\textcolor{teal}{\mytextsc{verbe}} \hspace{4pt} Ton~: MH.
\textcolor{PineGreen}{\selectlanguage{french}Semer à la volée.} \zh{撒 (撒种子)、播(种子)。}  ¶ \textcolor{darkblue}{\textbf{\ipa{ɻæ˩ pʰo˧˥}}} \textcolor{PineGreen}{\selectlanguage{french}semer des graines à la volée} \zh{撒种子}  

\lhead{\firstmark}
\rhead{\botmark}

\subsection{\hspace{-0.5cm} {\Large \textcolor{darkblue}{\textbf{\ipa{pʰo˧˥\textsubscript{a}}}}}\hspace{0.5cm}[\kern2pt{\textcolor{darkblue}{\textbf{\ipa{pʰo˧˥}}}}\kern2pt]} \hypertarget{p\string_ho\string_M\string_Ta1}{}
\markboth{\textcolor{darkblue}{\textbf{\ipa{pʰo˧˥\textsubscript{a}}}}}{}
\textcolor{teal}{\mytextsc{classificateur}} \hspace{4pt} Ton~: MH\textsubscript{a}.
\textcolor{PineGreen}{\selectlanguage{french}Classificateur des membres d'une paire. Par exemple: une chaussure, pas une paire. Ce classificateur est également employé pour le gros bétail: vaches, buffles….} \zh{量词:一对中的一只(例如一只鞋),一头大牲畜(牛……)。} 
\lhead{\firstmark}
\rhead{\botmark}

\subsection{\hspace{-0.5cm} {\Large \textcolor{darkblue}{\textbf{\ipa{pʰv̩˧}}} \textsubscript{1}}\hspace{0.5cm}[\kern2pt{\textcolor{darkblue}{\textbf{\ipa{pʰv̩˥}}}}\kern2pt]} \hypertarget{p\string_hv\string_=\string_M1}{}
\markboth{\textcolor{darkblue}{\textbf{\ipa{pʰv̩˧}}} \textsubscript{1}}{}
\textcolor{teal}{\mytextsc{nom}} \hspace{4pt} Ton~: M.
\textcolor{PineGreen}{\selectlanguage{french}Mâle.} \zh{公的。}  ¶ \textcolor{darkblue}{\textbf{\ipa{ʈʂʰɯ˧, | pʰv̩˧ ɲi˩!}}} \textcolor{PineGreen}{\selectlanguage{french}C'est un mâle!} \zh{这(只动物)是公的!}  
 ¶ \textcolor{darkblue}{\textbf{\ipa{ʈʂʰɯ˧, | pʰv̩˧!}}} \textcolor{PineGreen}{\selectlanguage{french}C'est un mâle!} \zh{这(只动物)是公的!}  
 \zh{量词}: \textcolor{darkblue}{\textbf{\ipa{v̩˧}}}  \mytextsc{clf}~: \textcolor{darkblue}{\textbf{\ipa{v̩˧}}} 
\lhead{\firstmark}
\rhead{\botmark}

\subsection{\hspace{-0.5cm} {\Large \textcolor{darkblue}{\textbf{\ipa{pʰv̩˧}}} \textsubscript{2}}\hspace{0.5cm}[\kern2pt{\textcolor{darkblue}{\textbf{\ipa{pʰv̩˥}}}}\kern2pt]} \hypertarget{p\string_hv\string_=\string_M2}{}
\markboth{\textcolor{darkblue}{\textbf{\ipa{pʰv̩˧}}} \textsubscript{2}}{}
\textcolor{teal}{\mytextsc{nom}} \hspace{4pt} Ton~: M.
\textcolor{PineGreen}{\selectlanguage{french}Prix.} \zh{价格。} 
\lhead{\firstmark}
\rhead{\botmark}

\subsection{\hspace{-0.5cm} {\Large \textcolor{darkblue}{\textbf{\ipa{pʰv̩˧˥}}}}\hspace{0.5cm}[\kern2pt{\textcolor{darkblue}{\textbf{\ipa{pʰv̩˧˥}}}}\kern2pt]} \hypertarget{p\string_hv\string_=\string_M\string_T1}{}
\markboth{\textcolor{darkblue}{\textbf{\ipa{pʰv̩˧˥}}}}{}
\textcolor{teal}{\mytextsc{verbe}} \hspace{4pt} Ton~: MH.
\textcolor{PineGreen}{\selectlanguage{french}Ôter, retirer (un vêtement).} \zh{脱(衣服)。} 
\lhead{\firstmark}
\rhead{\botmark}

\subsection{\hspace{-0.5cm} {\Large \textcolor{darkblue}{\textbf{\ipa{pʰv̩˧˥}}} \textsubscript{1}}\hspace{0.5cm}[\kern2pt{\textcolor{darkblue}{\textbf{\ipa{pʰv̩˧˥}}}}\kern2pt]} \hypertarget{p\string_hv\string_=\string_M\string_T1}{}
\markboth{\textcolor{darkblue}{\textbf{\ipa{pʰv̩˧˥}}} \textsubscript{1}}{}
\textcolor{teal}{\mytextsc{verbe}} \hspace{4pt} Ton~: MH.
\textcolor{PineGreen}{\selectlanguage{french}Faire bouillir, faire cuire à l'eau (œuf, patates…).} \zh{煮(鸡蛋、洋芋……)。}  ¶ \textcolor{darkblue}{\textbf{\ipa{jɤ˩jo˥ F | pʰv̩˧˥! | æ˩ʁv̩˩˥ F | pʰv̩˧˥!}}} \textcolor{PineGreen}{\selectlanguage{french}Les pommes de terre, ça se cuit à l'eau! Les œufs, ça se cuit à l'eau!} \zh{洋芋,是(可以)煮的!鸡蛋,是(可以)煮的!}  
 ¶ \textcolor{darkblue}{\textbf{\ipa{æ˩ʁv̩˩ pʰv̩˥}}} \textcolor{PineGreen}{\selectlanguage{french}cuire des œufs à l'eau, faire des oeufs durs} \zh{煮鸡蛋}  
 ¶ \textcolor{darkblue}{\textbf{\ipa{jɤ˩jo˥ pʰv̩˩}}} \textcolor{PineGreen}{\selectlanguage{french}cuire des pommes de terre à l'eau} \zh{煮洋芋}  
 ¶ \textcolor{darkblue}{\textbf{\ipa{le˧-pʰv̩˧ | le˧-mv̩˩-ze˩!}}} \textcolor{PineGreen}{\selectlanguage{french}C'est cuit (à l'eau)! Résultatif: ça a été suffisamment bouilli pour que ce soit maintenant cuit} \zh{煮熟了!}  

\lhead{\firstmark}
\rhead{\botmark}

\subsection{\hspace{-0.5cm} {\Large \textcolor{darkblue}{\textbf{\ipa{pʰv̩˧˥}}} \textsubscript{2}}\hspace{0.5cm}[\kern2pt{\textcolor{darkblue}{\textbf{\ipa{pʰv̩˧˥}}}}\kern2pt]} \hypertarget{p\string_hv\string_=\string_M\string_T2}{}
\markboth{\textcolor{darkblue}{\textbf{\ipa{pʰv̩˧˥}}} \textsubscript{2}}{}
\textcolor{teal}{\mytextsc{verbe}} \hspace{4pt} Ton~: MH.
\textcolor{PineGreen}{\selectlanguage{french}Verser; renverser; répandre; jeter.} \zh{倒(酒……),倒出来。}  ¶ \textcolor{darkblue}{\textbf{\ipa{ʐɯ˧ pʰv̩˧˥}}} \textcolor{PineGreen}{\selectlanguage{french}verser du vin, servir du vin} \zh{倒酒}  
 ¶ \textcolor{darkblue}{\textbf{\ipa{dʑɯ˩ pʰv̩˩˥}}} \textcolor{PineGreen}{\selectlanguage{french}verser de l'eau} \zh{倒水}  
 ¶ \textcolor{darkblue}{\textbf{\ipa{mv̩˩tɕo˧ pʰv̩˧˥}}} \textcolor{PineGreen}{\selectlanguage{french}renverser, verser à terre, jeter à terre} \zh{往外倒}  
 ¶ \textcolor{darkblue}{\textbf{\ipa{[F5] ɖæ˩˥ | mv̩˩tɕo˧ pʰv̩˥}}} \textcolor{PineGreen}{\selectlanguage{french}jeter des ordures} \zh{倒垃圾}  

\lhead{\firstmark}
\rhead{\botmark}

\subsection{\hspace{-0.5cm} {\Large \textcolor{darkblue}{\textbf{\ipa{pʰv̩˧˥}}} \textsubscript{3}}\hspace{0.5cm}[\kern2pt{\textcolor{darkblue}{\textbf{\ipa{pʰv̩˧˥}}}}\kern2pt]} \hypertarget{p\string_hv\string_=\string_M\string_T3}{}
\markboth{\textcolor{darkblue}{\textbf{\ipa{pʰv̩˧˥}}} \textsubscript{3}}{}
\textcolor{teal}{\mytextsc{verbe}} \hspace{4pt} Ton~: MH.
\textcolor{PineGreen}{\selectlanguage{french}Retourner; se retourner (quelqu'un est allongé et se retourne).} \zh{翻身、翻来翻去。}  ¶ \textcolor{darkblue}{\textbf{\ipa{le˧-wo˧ tsɤ˥-pʰv̩˩ |}}} \textcolor{PineGreen}{\selectlanguage{french}se retourner} \zh{翻身}  
 ¶ \textcolor{darkblue}{\textbf{\ipa{ɖɯ˧-tɕo˥ tsɤ˩-pʰv̩˩, | ʈʂʰɯ˧-tɕo˥ tsɤ˩-pʰv̩˩}}} \textcolor{PineGreen}{\selectlanguage{french}se retourner par-ci, se retourner par-là} \zh{翻来翻去}  

\lhead{\firstmark}
\rhead{\botmark}

\subsection{\hspace{-0.5cm} {\Large \textcolor{darkblue}{\textbf{\ipa{pʰv̩˩\textsubscript{a}}}}}\hspace{0.5cm}[\kern2pt{\textcolor{darkblue}{\textbf{\ipa{pʰv̩˩˥}}}}\kern2pt]} \hypertarget{p\string_hv\string_=\string_Ba1}{}
\markboth{\textcolor{darkblue}{\textbf{\ipa{pʰv̩˩\textsubscript{a}}}}}{}
\textcolor{teal}{\mytextsc{adjectif}} \hspace{4pt} Ton~: L\textsubscript{a}.
\textcolor{PineGreen}{\selectlanguage{french}Blanc (visage, habits, cheveux...).} \zh{白色(脸、衣服)。}  ¶ \textcolor{darkblue}{\textbf{\ipa{pʰv̩˩-hĩ˩˥}}} \textcolor{PineGreen}{\selectlanguage{french}\mytextsc{rel}} \zh{白的}  

\lhead{\firstmark}
\rhead{\botmark}

\subsection{\hspace{-0.5cm} {\Large \textcolor{darkblue}{\textbf{\ipa{pʰv̩˩\textsubscript{a}}}}}\hspace{0.5cm}[\kern2pt{\textcolor{darkblue}{\textbf{\ipa{pʰv̩˩˥}}}}\kern2pt]} \hypertarget{p\string_hv\string_=\string_Ba1}{}
\markboth{\textcolor{darkblue}{\textbf{\ipa{pʰv̩˩\textsubscript{a}}}}}{}
\textcolor{teal}{\mytextsc{classificateur}} \hspace{4pt} Ton~: L\textsubscript{b}.
\textcolor{PineGreen}{\selectlanguage{french}Classificateur des parcelles de terre, des champs.} \zh{量词:田地(一块)。}  ¶ \textcolor{darkblue}{\textbf{\ipa{lv̩˧ | ɖɯ˧-pʰv̩˩}}} \textcolor{PineGreen}{\selectlanguage{french}un champ; une parcelle} \zh{一块田}  

\lhead{\firstmark}
\rhead{\botmark}

\subsection{\hspace{-0.5cm} {\Large \textcolor{darkblue}{\textbf{\ipa{pʰv̩˩\textsubscript{b}}}} \textsubscript{1}}\hspace{0.5cm}[\kern2pt{\textcolor{darkblue}{\textbf{\ipa{pʰv̩˩˥}}}}\kern2pt]} \hypertarget{p\string_hv\string_=\string_Bb1}{}
\markboth{\textcolor{darkblue}{\textbf{\ipa{pʰv̩˩\textsubscript{b}}}} \textsubscript{1}}{}
\textcolor{teal}{\mytextsc{verbe}} \hspace{4pt} Ton~: L\textsubscript{b}.
\textcolor{PineGreen}{\selectlanguage{french}S'agiter.} \zh{摇动、翻滚。}  ¶ \textcolor{darkblue}{\textbf{\ipa{bo˩˥ | tʰi˧-pʰv̩˩-dʑo˩}}} \textcolor{PineGreen}{\selectlanguage{french}le cochon s'agite dans son box} \zh{猪在翻滚}  
 ¶ \textcolor{darkblue}{\textbf{\ipa{bo˩-ɳɯ˧ | pʰv̩˧\textasciitilde{}pʰv̩˩}}} \textcolor{PineGreen}{\selectlanguage{french}même sens} \zh{猪在翻滚}  

\lhead{\firstmark}
\rhead{\botmark}

\subsection{\hspace{-0.5cm} {\Large \textcolor{darkblue}{\textbf{\ipa{pʰv̩˩\textsubscript{b}}}} \textsubscript{2}}\hspace{0.5cm}[\kern2pt{\textcolor{darkblue}{\textbf{\ipa{pʰv̩˩˥}}}}\kern2pt]} \hypertarget{p\string_hv\string_=\string_Bb2}{}
\markboth{\textcolor{darkblue}{\textbf{\ipa{pʰv̩˩\textsubscript{b}}}} \textsubscript{2}}{}
\textcolor{teal}{\mytextsc{verbe}} \hspace{4pt} Ton~: L\textsubscript{b}.
\textcolor{PineGreen}{\selectlanguage{french}Connaître une expansion, se répandre, s'étendre.} \zh{扩散、发展。}  ¶ \textcolor{darkblue}{\textbf{\ipa{zo˧mv̩˥ | tʰi˧-pʰv̩˩}}} \textcolor{PineGreen}{\selectlanguage{french}les enfants se répandent, la famille essaime} \zh{孩子们扩散(到新的地方)}  

\lhead{\firstmark}
\rhead{\botmark}

\subsection{\hspace{-0.5cm} {\Large \textcolor{darkblue}{\textbf{\ipa{pʰv̩˧ɖɯ˧˥}}}}\hspace{0.5cm}[\kern2pt{\textcolor{darkblue}{\textbf{\ipa{pʰv̩˧ɖɯ˧˥}}}}\kern2pt]} \hypertarget{p\string_hv\string_=\string_Md`M\string_M\string_T1}{}
\markboth{\textcolor{darkblue}{\textbf{\ipa{pʰv̩˧ɖɯ˧˥}}}}{}
\textcolor{teal}{\mytextsc{adjectif}} \hspace{4pt} Ton~: MH\#.
\textit{De:} \textbf{pʰv̩˧ 2 et ɖɯ˩a} \textcolor{PineGreen}{\selectlanguage{french}Coûteux, cher.} \zh{贵。}  ¶ \textcolor{darkblue}{\textbf{\ipa{pʰv̩˧ɖɯ˧ ʝi˥}}} \textcolor{PineGreen}{\selectlanguage{french}prêter attention à} \zh{关心}  

\lhead{\firstmark}
\rhead{\botmark}

\subsection{\hspace{-0.5cm} {\Large \textcolor{darkblue}{\textbf{\ipa{pʰv̩˧dʑo˧-hĩ\#˥}}}}\hspace{0.5cm}[\kern2pt{\textcolor{darkblue}{\textbf{\ipa{xxxx non-correspondance entre le nombre de morphèmes et le nombre de tons de morphèmes}}}}\kern2pt]} \hypertarget{p\string_hv\string_=\string_Mdz£o\string_M-hi\string_~\#\string_T1}{}
\markboth{\textcolor{darkblue}{\textbf{\ipa{pʰv̩˧dʑo˧-hĩ\#˥}}}}{}
\textcolor{teal}{\mytextsc{nom}} \hspace{4pt} Ton~: \#H.
\textcolor{PineGreen}{\selectlanguage{french}Personne de Labai, gens de Labai.} \zh{拉伯的人。} 
\lhead{\firstmark}
\rhead{\botmark}

\subsection{\hspace{-0.5cm} {\Large \textcolor{darkblue}{\textbf{\ipa{pʰv̩˧dʑo\#˥}}}}\hspace{0.5cm}[\kern2pt{\textcolor{darkblue}{\textbf{\ipa{pʰv̩˧dʑo˧}}}}\kern2pt]} \hypertarget{p\string_hv\string_=\string_Mdz£o\#\string_T1}{}
\markboth{\textcolor{darkblue}{\textbf{\ipa{pʰv̩˧dʑo\#˥}}}}{}
\textcolor{teal}{\mytextsc{nom}} \hspace{4pt} Ton~: \#H.
\textcolor{PineGreen}{\selectlanguage{french}Village de Labai.} \zh{拉伯。}  ¶ \textcolor{darkblue}{\textbf{\ipa{pʰv̩˧dʑo˧ dzi˧˥}}} \textcolor{PineGreen}{\selectlanguage{french}habiter à Labai} \zh{在拉柏住}  

\lhead{\firstmark}
\rhead{\botmark}

\subsection{\hspace{-0.5cm} {\Large \textcolor{darkblue}{\textbf{\ipa{pʰv̩˧kɤ˧}}}}\hspace{0.5cm}[\kern2pt{\textcolor{darkblue}{\textbf{\ipa{pʰv̩˧kɤ˧}}}}\kern2pt]} \hypertarget{p\string_hv\string_=\string_Mk7\string_M1}{}
\markboth{\textcolor{darkblue}{\textbf{\ipa{pʰv̩˧kɤ˧}}}}{}
\textcolor{teal}{\mytextsc{nom}} \hspace{4pt} Ton~: M.
\textcolor{PineGreen}{\selectlanguage{french}Couverture.} \zh{被子。}  \zh{量词}: \textcolor{darkblue}{\textbf{\ipa{ɭɯ˧}}}  \mytextsc{clf}~: \textcolor{darkblue}{\textbf{\ipa{ɭɯ˧}}} 
\lhead{\firstmark}
\rhead{\botmark}

\subsection{\hspace{-0.5cm} {\Large \textcolor{darkblue}{\textbf{\ipa{pʰv̩˧ɭɯ˧-ʈʰæ˧qʰwɤ˥}}}}\hspace{0.5cm}[\kern2pt{\textcolor{darkblue}{\textbf{\ipa{xxxx non-correspondance entre le nombre de morphèmes et le nombre de tons de morphèmes}}}}\kern2pt]} \hypertarget{p\string_hv\string_=\string_Ml\string_RM\string_M-t`\string_h\{\string_Mq\string_hw7\string_T1}{}
\markboth{\textcolor{darkblue}{\textbf{\ipa{pʰv̩˧ɭɯ˧-ʈʰæ˧qʰwɤ˥}}}}{}
\textcolor{teal}{\mytextsc{nom}} \hspace{4pt} Ton~: H\#.
\textcolor{PineGreen}{\selectlanguage{french}Jupe de laine. (Ce n'est pas un vêtement courant à Yongning.).} \zh{羊毛裙子。}  \zh{量词}: \textcolor{darkblue}{\textbf{\ipa{ɭɯ˧}}}  \mytextsc{clf}~: \textcolor{darkblue}{\textbf{\ipa{ɭɯ˧}}} 
\lhead{\firstmark}
\rhead{\botmark}

\subsection{\hspace{-0.5cm} {\Large \textcolor{darkblue}{\textbf{\ipa{pʰv̩˧ɭɯ\#˥}}}}\hspace{0.5cm}[\kern2pt{\textcolor{darkblue}{\textbf{\ipa{pʰv̩˧ɭɯ˧}}}}\kern2pt]} \hypertarget{p\string_hv\string_=\string_Ml\string_RM\#\string_T1}{}
\markboth{\textcolor{darkblue}{\textbf{\ipa{pʰv̩˧ɭɯ\#˥}}}}{}
\textcolor{teal}{\mytextsc{nom}} \hspace{4pt} Ton~: \#H.
\textcolor{PineGreen}{\selectlanguage{french}Tissu de laine tibétain.} \zh{氆氇。}  \zh{量词}: \textcolor{darkblue}{\textbf{\ipa{ɭɯ˧}}}  \mytextsc{clf}~: \textcolor{darkblue}{\textbf{\ipa{ɭɯ˧}}} 
\lhead{\firstmark}
\rhead{\botmark}

\subsection{\hspace{-0.5cm} {\Large \textcolor{darkblue}{\textbf{\ipa{pʰv̩˧ʂɯ˧}}}}\hspace{0.5cm}[\kern2pt{\textcolor{darkblue}{\textbf{\ipa{pʰv̩˧ʂɯ˧}}}}\kern2pt]} \hypertarget{p\string_hv\string_=\string_Ms`M\string_M1}{}
\markboth{\textcolor{darkblue}{\textbf{\ipa{pʰv̩˧ʂɯ˧}}}}{}
\textcolor{teal}{\mytextsc{nom}} \hspace{4pt} Ton~: M.
\textcolor{PineGreen}{\selectlanguage{french}Crème de beauté; s'emploie aussi pour la crème solaire.} \zh{美容膏。也来指防晒霜。}  ¶ \textcolor{darkblue}{\textbf{\ipa{pʰv˧ʂɯ˧ jɤ˧˥}}} \textcolor{PineGreen}{\selectlanguage{french}étaler de la crème solaire, mettre de la crème solaire} \zh{抹美容膏,抹防晒霜}  
 ¶ \textcolor{darkblue}{\textbf{\ipa{pʰv˧ʂɯ˧ lɑ˧˥}}} \textcolor{PineGreen}{\selectlanguage{french}étaler de la crème solaire, mettre de la crème solaire} \zh{抹美容膏,抹防晒霜}  

\lhead{\firstmark}
\rhead{\botmark}

\subsection{\hspace{-0.5cm} {\Large \textcolor{darkblue}{\textbf{\ipa{pʰv̩˩-tɕæ˩ɻæ˥}}}}\hspace{0.5cm}[\kern2pt{\textcolor{darkblue}{\textbf{\ipa{xxxx non-correspondance entre le nombre de morphèmes et le nombre de tons de morphèmes}}}}\kern2pt]} \hypertarget{p\string_hv\string_=\string_B-ts£\{\string_Br£`\{\string_T1}{}
\markboth{\textcolor{darkblue}{\textbf{\ipa{pʰv̩˩-tɕæ˩ɻæ˥}}}}{}
\textcolor{teal}{\mytextsc{adjectif}} \hspace{4pt} Ton~: L+H\#.
\textit{De:} \textbf{pʰv̩˩a} \textcolor{PineGreen}{\selectlanguage{french}Blanc (visage, habits, cheveux...).} \zh{很白(脸、衣服、头发)。}  ¶ \textcolor{darkblue}{\textbf{\ipa{pʰv̩˩tɕæ˩ɻæ˥-gv̩˩}}} \textcolor{PineGreen}{\selectlanguage{french}tout blanc} \zh{很白}  
 ¶ \textcolor{darkblue}{\textbf{\ipa{pʰv̩˩↑tɕæ˥ɻæ˩-gv̩˩}}} \textcolor{PineGreen}{\selectlanguage{french}tout blanc} \zh{很白}  
 ¶ \textcolor{darkblue}{\textbf{\ipa{pʰæ˧qʰwɤ˩ | pʰv̩˩tɕæ˩ɻæ˥-gv̩˩}}} \textcolor{PineGreen}{\selectlanguage{french}le visage est très blanc} \zh{脸很白}  

\lhead{\firstmark}
\rhead{\botmark}

\subsection{\hspace{-0.5cm} {\Large \textcolor{darkblue}{\textbf{\ipa{pʰv̩˥ʈʂʰe˩}}}}\hspace{0.5cm}[\kern2pt{\textcolor{darkblue}{\textbf{\ipa{xxxx ton non trouvé, à faire manuellement...}}}}\kern2pt]} \hypertarget{p\string_hv\string_=\string_Tt`s`\string_he\string_B1}{}
\markboth{\textcolor{darkblue}{\textbf{\ipa{pʰv̩˥ʈʂʰe˩}}}}{}
\textcolor{teal}{\mytextsc{verbe}} \hspace{4pt} Ton~: HL.
\textcolor{PineGreen}{\selectlanguage{french}Distinguer, voir la différence (par exemple entre diverses espèces de champignons).} \zh{分开、区分、区别开来。}  ¶ \textcolor{darkblue}{\textbf{\ipa{le˧-pʰv̩˥ʈʂʰe˩}}} \textcolor{PineGreen}{\selectlanguage{french}distinguer, voir la différence (par exemple entre diverses espèces de champignons)} \zh{分开、区分、区别开来}  
 ¶ \textcolor{darkblue}{\textbf{\ipa{ɖɯ˧-pʰv̩˥ʈʂʰe˩=ɻ̍˩}}} \textcolor{PineGreen}{\selectlanguage{french}\mytextsc{délimitatif} \string_ \mytextsc{inchoatif}} \zh{试着区分}  
 ¶ \textcolor{darkblue}{\textbf{\ipa{mɤ˧-pʰv̩˥ʈʂʰe˩}}} \textcolor{PineGreen}{\selectlanguage{french}ne pas distinguer, ne pas faire de différence, ne pas voir la différence (par exemple entre diverses espèces de champignons)} \zh{不分开,不分,不区分}  

\lhead{\firstmark}
\rhead{\botmark}

\subsection{\hspace{-0.5cm} {\Large \textcolor{darkblue}{\textbf{\ipa{pʰv̩˧tv̩˥}}}}\hspace{0.5cm}[\kern2pt{\textcolor{darkblue}{\textbf{\ipa{pʰv̩˧tv̩˥}}}}\kern2pt]} \hypertarget{p\string_hv\string_=\string_Mtv\string_=\string_T1}{}
\markboth{\textcolor{darkblue}{\textbf{\ipa{pʰv̩˧tv̩˥}}}}{}
\textcolor{teal}{\mytextsc{nom}} \hspace{4pt} Ton~: H\#.
\textcolor{PineGreen}{\selectlanguage{french}Buffle mâle.} \zh{公水牛。}  ¶ \textcolor{darkblue}{\textbf{\ipa{dʑi˧mi˧-pʰv̩˩tv̩˩}}} \textcolor{PineGreen}{\selectlanguage{french}même sens: buffle mâle} \zh{同上:公水牛}  
 ¶ \textcolor{darkblue}{\textbf{\ipa{dʑi˧mi˧ ʈʂʰɯ˧-pʰo˩ dʑo˩, | pʰv̩˧tv̩˥ ɲi˩!}}} \textcolor{PineGreen}{\selectlanguage{french}ce buffle, c'est un mâle!} \zh{这头水牛是公的/是公水牛!}  
 \zh{量词}: \textcolor{darkblue}{\textbf{\ipa{pʰo˧˥}}}  \mytextsc{clf}~: \textcolor{darkblue}{\textbf{\ipa{pʰo˧˥}}} 
\lhead{\firstmark}
\rhead{\botmark}

\subsection{\hspace{-0.5cm} {\Large \textcolor{darkblue}{\textbf{\ipa{pʰv̩˧ʐo˧˥}}}}\hspace{0.5cm}[\kern2pt{\textcolor{darkblue}{\textbf{\ipa{pʰv̩˧ʐo˧˥}}}}\kern2pt]} \hypertarget{p\string_hv\string_=\string_Mz`o\string_M\string_T1}{}
\markboth{\textcolor{darkblue}{\textbf{\ipa{pʰv̩˧ʐo˧˥}}}}{}
\textcolor{teal}{\mytextsc{adjectif}} \hspace{4pt} Ton~: MH\#.
\textit{De:} \textbf{pʰv̩˧ 2 et ʐo˩a} \textcolor{PineGreen}{\selectlanguage{french}Bon marché.} \zh{便宜。} \textit{Voir~:} \textcolor{darkblue}{\textbf{\ipa{pʰv̩˧2; ʐo˩a2}}} 
\lhead{\firstmark}
\rhead{\botmark}

\newpage
\section*{\centering- \textcolor{darkblue}{\textbf{\ipa{q}}} -}
\subsection{\hspace{-0.5cm} {\Large \textcolor{darkblue}{\textbf{\ipa{qɑ˩\textsubscript{c}}}}}\hspace{0.5cm}[\kern2pt{\textcolor{darkblue}{\textbf{\ipa{qɑ˩˥}}}}\kern2pt]} \hypertarget{qA\string_Bc1}{}
\markboth{\textcolor{darkblue}{\textbf{\ipa{qɑ˩\textsubscript{c}}}}}{}
\textcolor{teal}{\mytextsc{classificateur}} \hspace{4pt} Ton~: L\textsubscript{c}.
\ding{202} \textcolor{PineGreen}{\selectlanguage{french}Classificateur : une brassée (de bois coupé pour le feu, d'objets...).} \zh{量词:抱。}  ¶ \textcolor{darkblue}{\textbf{\ipa{ʈʂʰɯ˧-qɑ˥}}} \textcolor{PineGreen}{\selectlanguage{french}cette brassée} \zh{这一抱}  
\ding{203} \textcolor{PineGreen}{\selectlanguage{french}Classificateur des bottes de céréales coupées, faites d'une dizaine de gerbes. Chaque gerbe est nouée à l'aide d'une tige, puis les gerbes sont liées ensemble avec de la ficelle. Une mule peut porter 4 bottes.} \zh{量词:粮食垛、干草垛。}  ¶ \textcolor{darkblue}{\textbf{\ipa{dze˧ɭɯ˧ ɖɯ˧-qɑ˩}}} \textcolor{PineGreen}{\selectlanguage{french}une botte de blé (lors de la récolte)} \zh{一垛小麦(收割时,将十束绑在一起成一垛)}  

\lhead{\firstmark}
\rhead{\botmark}

\subsection{\hspace{-0.5cm} {\Large \textcolor{darkblue}{\textbf{\ipa{qɑ˩\textsubscript{a}}}}}\hspace{0.5cm}[\kern2pt{\textcolor{darkblue}{\textbf{\ipa{qɑ˩˥}}}}\kern2pt]} \hypertarget{qA\string_Ba1}{}
\markboth{\textcolor{darkblue}{\textbf{\ipa{qɑ˩\textsubscript{a}}}}}{}
\textcolor{teal}{\mytextsc{verbe}} \hspace{4pt} Ton~: L\textsubscript{a}.
\ding{202} \textcolor{PineGreen}{\selectlanguage{french}Couvrir: par exemple mettre un couvercle, ou couvrir un plat d'une coupelle pour éviter que les mouches ne s'y posent.} \zh{盖、覆盖。}  ¶ \textcolor{darkblue}{\textbf{\ipa{le˧-qɑ˩-ze˩}}} \textcolor{PineGreen}{\selectlanguage{french}\mytextsc{accomp} \string_ \mytextsc{pfv}} \zh{盖了}  
 ¶ \textcolor{darkblue}{\textbf{\ipa{tʰi˧-qɑ˩-ze˩}}} \textcolor{PineGreen}{\selectlanguage{french}\mytextsc{dur} \string_ \mytextsc{pfv}} \zh{\mytextsc{dur} \string_ \mytextsc{pfv}}  
 ¶ \textcolor{darkblue}{\textbf{\ipa{ɖɯ˧-kʰwɤ˥ | tʰi˧-qɑ˥}}} \textcolor{PineGreen}{\selectlanguage{french}couvrir (un téléviseur) d'un morceau (de tissu) (pour le préserver de la poussière)} \zh{用一块(布料)来盖(电视机,为了防灰)}  
 ¶ \textcolor{darkblue}{\textbf{\ipa{hæ̃˧qʰv̩˥ | tʰi˧-qɑ˩!}}} \textcolor{PineGreen}{\selectlanguage{french}le soir, on recouvre (le téléviseur d'un tissu)! / (on) (le) met le soir (sur le téléviseur)/ (on) (en) recouvre (le téléviseur) le soir!} \zh{晚上,(要)盖上! / 我们晚上盖电视机(为了防灰)!}  
 ¶ \textcolor{darkblue}{\textbf{\ipa{tso˧\textasciitilde{}tso˧ qɑ˥}}} \textcolor{PineGreen}{\selectlanguage{french}recouvrir quelque chose} \zh{覆盖东西}  
\ding{203} \textcolor{PineGreen}{\selectlanguage{french}Voiler, bloquer (la lumière), cacher au regard.} \zh{遮(云遮月)、遮挡。} 
\lhead{\firstmark}
\rhead{\botmark}

\subsection{\hspace{-0.5cm} {\Large \textcolor{darkblue}{\textbf{\ipa{‑qɑ˧˥}}}}\hspace{0.5cm}[\kern2pt{\textcolor{darkblue}{\textbf{\ipa{qɑ˧˥}}}}\kern2pt]} \hypertarget{‑qA\string_M\string_T1}{}
\markboth{\textcolor{darkblue}{\textbf{\ipa{‑qɑ˧˥}}}}{}
\textcolor{teal}{\mytextsc{postposition}} \hspace{4pt} Ton~: MH.
\textcolor{PineGreen}{\selectlanguage{french}À (datif); avec (comitatif).} \zh{给、对。} 
\lhead{\firstmark}
\rhead{\botmark}

\subsection{\hspace{-0.5cm} {\Large \textcolor{darkblue}{\textbf{\ipa{qɑ˧˥}}}}\hspace{0.5cm}[\kern2pt{\textcolor{darkblue}{\textbf{\ipa{qɑ˧˥}}}}\kern2pt]} \hypertarget{qA\string_M\string_T1}{}
\markboth{\textcolor{darkblue}{\textbf{\ipa{qɑ˧˥}}}}{}
\textcolor{teal}{\mytextsc{verbe}} \hspace{4pt} Ton~: MH.
\textcolor{PineGreen}{\selectlanguage{french}Aider.} \zh{帮助。}  ¶ \textcolor{darkblue}{\textbf{\ipa{tʰi˧-qɑ˧˥}}} \textcolor{PineGreen}{\selectlanguage{french}\mytextsc{dur}} \zh{\mytextsc{dur}}  
 ¶ \textcolor{darkblue}{\textbf{\ipa{qɑ˩\textasciitilde{}qɑ˧˥}}} \textcolor{PineGreen}{\selectlanguage{french}\mytextsc{red}} \zh{\mytextsc{重叠:帮帮忙}}  
 ¶ \textcolor{darkblue}{\textbf{\ipa{hĩ˧ qɑ˩\textasciitilde{}qɑ˩}}} \textcolor{PineGreen}{\selectlanguage{french}aider des gens; aller travailler chez autrui (par ex. pendant les récoltes)} \zh{帮人,到别人家去工作(例如收庄稼的时候)}  
 ¶ \textcolor{darkblue}{\textbf{\ipa{njɤ˧ no˧ qɑ˧\textasciitilde{}qɑ˥}}} \textcolor{PineGreen}{\selectlanguage{french}je t'aide} \zh{我帮你}  

\lhead{\firstmark}
\rhead{\botmark}

\subsection{\hspace{-0.5cm} {\Large \textcolor{darkblue}{\textbf{\ipa{qæ˥}}} \textsubscript{1}}\hspace{0.5cm}[\kern2pt{\textcolor{darkblue}{\textbf{\ipa{qæ˥}}}}\kern2pt]} \hypertarget{q\{\string_T1}{}
\markboth{\textcolor{darkblue}{\textbf{\ipa{qæ˥}}} \textsubscript{1}}{}
\textcolor{teal}{\mytextsc{verbe}} \hspace{4pt} Ton~: H.
\textcolor{PineGreen}{\selectlanguage{french}Déplacer, transporter (ex.: transporter de la terre, après avoir creusé).} \zh{搬。} Dialecte chinois local~:\zh{盘。} ¶ \textcolor{darkblue}{\textbf{\ipa{le˧-qæ˥}}} \textcolor{PineGreen}{\selectlanguage{french}\mytextsc{accomp}} \zh{\mytextsc{accomp}}  

\lhead{\firstmark}
\rhead{\botmark}

\subsection{\hspace{-0.5cm} {\Large \textcolor{darkblue}{\textbf{\ipa{qæ˥}}} \textsubscript{2}}\hspace{0.5cm}[\kern2pt{\textcolor{darkblue}{\textbf{\ipa{qæ˥}}}}\kern2pt]} \hypertarget{q\{\string_T2}{}
\markboth{\textcolor{darkblue}{\textbf{\ipa{qæ˥}}} \textsubscript{2}}{}
\textcolor{teal}{\mytextsc{verbe}} \hspace{4pt} Ton~: H.
\textcolor{PineGreen}{\selectlanguage{french}Changer.} \zh{换。}  ¶ \textcolor{darkblue}{\textbf{\ipa{le˧-qæ˥-ze˩}}} \textcolor{PineGreen}{\selectlanguage{french}\mytextsc{accomp} \string_ \mytextsc{pfv}} \zh{换了}  
 ¶ \textcolor{darkblue}{\textbf{\ipa{dʑi˧hṽ˧ qæ˧}}} \textcolor{PineGreen}{\selectlanguage{french}changer de vêtements} \zh{换衣服}  
 ¶ \textcolor{darkblue}{\textbf{\ipa{bɑ˩lɑ˩˥ | tʰi˧-qæ˥}}} \textcolor{PineGreen}{\selectlanguage{french}changer de vêtements} \zh{换衣服}  
 ¶ \textcolor{darkblue}{\textbf{\ipa{qæ˧\textasciitilde{}qæ˧}}} \textcolor{PineGreen}{\selectlanguage{french}\mytextsc{red}: échanger (un objet contre un autre)} \zh{\mytextsc{重叠:交换}}  
 ¶ \textcolor{darkblue}{\textbf{\ipa{qæ˧\textasciitilde{}qæ˧-ɻ̍˥}}} \textcolor{PineGreen}{\selectlanguage{french}\mytextsc{red} \mytextsc{inchoatif}} \zh{\mytextsc{red} \mytextsc{inceptive}}  
 ¶ \textcolor{darkblue}{\textbf{\ipa{qæ˧\textasciitilde{}qæ˧-ɻ̍˧-ze˥}}} \textcolor{PineGreen}{\selectlanguage{french}\mytextsc{red} \mytextsc{inchoatif} \mytextsc{pfv}} \zh{\mytextsc{red} \mytextsc{inceptive} \mytextsc{pfv}}  
 ¶ \textcolor{darkblue}{\textbf{\ipa{tso˧\textasciitilde{}tso˧ qæ˧\textasciitilde{}qæ˧}}} \textcolor{PineGreen}{\selectlanguage{french}échanger des choses} \zh{交换东西}  
 ¶ \textcolor{darkblue}{\textbf{\ipa{le˧-qæ˧\textasciitilde{}qæ˧(-ze˩)}}} \textcolor{PineGreen}{\selectlanguage{french}\mytextsc{accomp} \mytextsc{red} (\mytextsc{pfv})} \zh{\mytextsc{accomp} \mytextsc{red} (\mytextsc{pfv})}  

\lhead{\firstmark}
\rhead{\botmark}

\subsection{\hspace{-0.5cm} {\Large \textcolor{darkblue}{\textbf{\ipa{qæ˥}}} \textsubscript{3}}\hspace{0.5cm}[\kern2pt{\textcolor{darkblue}{\textbf{\ipa{qæ˥}}}}\kern2pt]} \hypertarget{q\{\string_T3}{}
\markboth{\textcolor{darkblue}{\textbf{\ipa{qæ˥}}} \textsubscript{3}}{}
\textcolor{teal}{\mytextsc{verbe}} \hspace{4pt} Ton~: H.
\textcolor{PineGreen}{\selectlanguage{french}Sculpter.} \zh{雕。}  ¶ \textcolor{darkblue}{\textbf{\ipa{le˧-qæ˥-ze˩}}} \textcolor{PineGreen}{\selectlanguage{french}\mytextsc{accomp} \string_ \mytextsc{pfv}} \zh{雕了}  
 ¶ \textcolor{darkblue}{\textbf{\ipa{bæ˩bæ˩ qæ˥}}} \textcolor{PineGreen}{\selectlanguage{french}sculpter une fleur} \zh{雕花}  

\lhead{\firstmark}
\rhead{\botmark}

\subsection{\hspace{-0.5cm} {\Large \textcolor{darkblue}{\textbf{\ipa{qæ˧do˧}}}}\hspace{0.5cm}[\kern2pt{\textcolor{darkblue}{\textbf{\ipa{qæ˩do˩˥}}}}\kern2pt]} \hypertarget{q\{\string_Mdo\string_M1}{}
\markboth{\textcolor{darkblue}{\textbf{\ipa{qæ˧do˧}}}}{}
\textcolor{teal}{\mytextsc{nom}} \hspace{4pt} Ton~: M.
\textcolor{PineGreen}{\selectlanguage{french}Bois de charpente, tronc coupé.} \zh{木材、木料。}  ¶ \textcolor{darkblue}{\textbf{\ipa{ʑi˧mi˧-qæ˩do˩}}} \textcolor{PineGreen}{\selectlanguage{french}bois de charpente utilisé pour le bâtiment principal} \zh{建主房的木材}  
 ¶ \textcolor{darkblue}{\textbf{\ipa{ʑi˧qʰwɤ˧-qæ˧do\#˥}}} \textcolor{PineGreen}{\selectlanguage{french}bois de charpente, bois pour la construction d'un bâtiment} \zh{建房子的木材}  
 \zh{量词}: \textcolor{darkblue}{\textbf{\ipa{kɤ˧˥}}}  \mytextsc{clf}~: \textcolor{darkblue}{\textbf{\ipa{kɤ˧˥}}} \textit{Syn~:} \hyperlink{}{\textcolor{darkblue}{\textbf{\ipa{qæ˧ɻ̍˧}}}}. 
\lhead{\firstmark}
\rhead{\botmark}

\subsection{\hspace{-0.5cm} {\Large \textcolor{darkblue}{\textbf{\ipa{qæ˧dzɯ˩}}}}\hspace{0.5cm}[\kern2pt{\textcolor{darkblue}{\textbf{\ipa{qæ˧dzɯ˩}}}}\kern2pt]} \hypertarget{q\{\string_MdzM\string_B1}{}
\markboth{\textcolor{darkblue}{\textbf{\ipa{qæ˧dzɯ˩}}}}{}
\textcolor{teal}{\mytextsc{nom}} \hspace{4pt} Ton~: L\#.
\textcolor{PineGreen}{\selectlanguage{french}Nom de clan/famille étendue. Deux familles portent ce nom à Yongning.} \zh{一个姓。这个姓,永宁有两家。}  ¶ \textcolor{darkblue}{\textbf{\ipa{qæ˧dzɯ˩-ɻ̍˩}}} \textcolor{PineGreen}{\selectlanguage{french}le clan \textcolor{darkblue}{\textbf{\ipa{/qæ˧dzɯ˩/}}}, la famille \textcolor{darkblue}{\textbf{\ipa{/qæ˧dzɯ˩/}}}} \zh{\textcolor{darkblue}{\textbf{\ipa{/qæ˧dzɯ˩/}}}家族}  
 ¶ \textcolor{darkblue}{\textbf{\ipa{qæ˧dzɯ˩ | -tsʰɯ˧ɻ̍˧}}} \textcolor{PineGreen}{\selectlanguage{french}nom d'une personne, comportant un nom de famille (\textcolor{darkblue}{\textbf{\ipa{/qæ˧dzɯ˩/}}}) et un prénom (\textcolor{darkblue}{\textbf{\ipa{/tsʰɯ˧ɻ\#˥/}}})} \zh{一个人的名字:姓为\textcolor{darkblue}{\textbf{\ipa{/qæ˧dzɯ˩/}}},名为\textcolor{darkblue}{\textbf{\ipa{/tsʰɯ˧ɻ\#˥/}}}}  

\lhead{\firstmark}
\rhead{\botmark}

\subsection{\hspace{-0.5cm} {\Large \textcolor{darkblue}{\textbf{\ipa{qæ˧ɻ̍˧}}}}\hspace{0.5cm}[\kern2pt{\textcolor{darkblue}{\textbf{\ipa{qæ˧ɻ̍˧}}}}\kern2pt]} \hypertarget{q\{\string_Mr£`̍\string_M1}{}
\markboth{\textcolor{darkblue}{\textbf{\ipa{qæ˧ɻ̍˧}}}}{}
\textcolor{teal}{\mytextsc{nom}} \hspace{4pt} Ton~: M.
\textcolor{PineGreen}{\selectlanguage{french}Bois de charpente, tronc coupé.} \zh{木材、木料。}  \zh{量词}: \textcolor{darkblue}{\textbf{\ipa{kɤ˧˥}}}  \mytextsc{clf}~: \textcolor{darkblue}{\textbf{\ipa{kɤ˧˥}}} \textit{Syn~:} \hyperlink{}{\textcolor{darkblue}{\textbf{\ipa{qæ˧do˧}}}}. 
\lhead{\firstmark}
\rhead{\botmark}

\subsection{\hspace{-0.5cm} {\Large \textcolor{darkblue}{\textbf{\ipa{qæ˩\textsubscript{a}}}}}\hspace{0.5cm}[\kern2pt{\textcolor{darkblue}{\textbf{\ipa{qæ˩˥}}}}\kern2pt]} \hypertarget{q\{\string_Ba1}{}
\markboth{\textcolor{darkblue}{\textbf{\ipa{qæ˩\textsubscript{a}}}}}{}
\textcolor{teal}{\mytextsc{verbe}} \hspace{4pt} Ton~: L\textsubscript{a}.
\textcolor{PineGreen}{\selectlanguage{french}Cajoler un enfant.} \zh{哄(孩子)。}  ¶ \textcolor{darkblue}{\textbf{\ipa{zo˧ qæ˥}}} \textcolor{PineGreen}{\selectlanguage{french}cajoler un enfant} \zh{哄孩子}  
 ¶ \textcolor{darkblue}{\textbf{\ipa{le˧-qæ˧\textasciitilde{}qæ˥ | le˧-ʑi˧-kʰɯ˥}}} \textcolor{PineGreen}{\selectlanguage{french}endormir (un enfant) en le cajolant} \zh{哄睡着}  

\lhead{\firstmark}
\rhead{\botmark}

\subsection{\hspace{-0.5cm} {\Large \textcolor{darkblue}{\textbf{\ipa{qæ˩\textsubscript{b}}}}}\hspace{0.5cm}[\kern2pt{\textcolor{darkblue}{\textbf{\ipa{qæ˩˥}}}}\kern2pt]} \hypertarget{q\{\string_Bb1}{}
\markboth{\textcolor{darkblue}{\textbf{\ipa{qæ˩\textsubscript{b}}}}}{}
\textcolor{teal}{\mytextsc{verbe}} \hspace{4pt} Ton~: L\textsubscript{b}.
\textcolor{PineGreen}{\selectlanguage{french}Tromper.} \zh{欺骗。}  ¶ \textcolor{darkblue}{\textbf{\ipa{hĩ˧ qæ˥-kv̩˩}}} \textcolor{PineGreen}{\selectlanguage{french}rusé, qui sait tromper son monde} \zh{狡猾、很能骗人的}  
 ¶ \textcolor{darkblue}{\textbf{\ipa{hĩ˧ qæ˥ | ʐwæ˩˥}}} \textcolor{PineGreen}{\selectlanguage{french}qui trompe magistralement son monde} \zh{狡猾、很能骗人的}  
 ¶ \textcolor{darkblue}{\textbf{\ipa{hĩ˧ qæ˥ mɤ˩-ɖo˩!}}} \textcolor{PineGreen}{\selectlanguage{french}il ne faut pas tromper (autrui)! (précepte inculqué à la locutrice par sa grand-mère)} \zh{不要骗人!(这个信条,是发音合作人的祖母教的)}  
 ¶ \textcolor{darkblue}{\textbf{\ipa{qæ˩-mɤ˩-ɖo˩˥!}}} \textcolor{PineGreen}{\selectlanguage{french}il ne faut pas tromper (autrui)! (précepte inculqué à la locutrice par sa grand-mère)} \zh{不要骗人!(这个信条,是发音合作人的祖母教的)}  
 ¶ \textcolor{darkblue}{\textbf{\ipa{mɤ˧-qæ˩}}} \textcolor{PineGreen}{\selectlanguage{french}\mytextsc{neg}} \zh{不骗}  
 ¶ \textcolor{darkblue}{\textbf{\ipa{hĩ˧ qæ˥-tso˩\textasciitilde{}tso˩!}}} \textcolor{PineGreen}{\selectlanguage{french}C'est de la camelote! / C'est un truc d'arnaqueurs! (au sujet d'une bobine de fil de mauvaise qualité, achetée dans le commerce)} \zh{骗人的东西!(关于买来的一团线,质量不好)}  

\lhead{\firstmark}
\rhead{\botmark}

\subsection{\hspace{-0.5cm} {\Large \textcolor{darkblue}{\textbf{\ipa{qæ˩di˩}}}}\hspace{0.5cm}[\kern2pt{\textcolor{darkblue}{\textbf{\ipa{qæ˩di˩˥}}}}\kern2pt]} \hypertarget{q\{\string_Bdi\string_B1}{}
\markboth{\textcolor{darkblue}{\textbf{\ipa{qæ˩di˩}}}}{}
\textcolor{teal}{\mytextsc{verbe}} \hspace{4pt} Ton~: L.
\textcolor{PineGreen}{\selectlanguage{french}Donner une chiquenaude.} \zh{弹(弹脸)。} 
\lhead{\firstmark}
\rhead{\botmark}

\subsection{\hspace{-0.5cm} {\Large \textcolor{darkblue}{\textbf{\ipa{qæ˧˥}}} \textsubscript{1}}\hspace{0.5cm}[\kern2pt{\textcolor{darkblue}{\textbf{\ipa{qæ˧˥}}}}\kern2pt]} \hypertarget{q\{\string_M\string_T1}{}
\markboth{\textcolor{darkblue}{\textbf{\ipa{qæ˧˥}}} \textsubscript{1}}{}
\textcolor{teal}{\mytextsc{verbe}} \hspace{4pt} Ton~: MH.
\textcolor{PineGreen}{\selectlanguage{french}Brûler quelque chose; par exemple: incinérer un corps.} \zh{燃烧,如:烧尸体(进行火葬时)。}  ¶ \textcolor{darkblue}{\textbf{\ipa{mv̩˧ qæ˩-ze˩}}} \textcolor{PineGreen}{\selectlanguage{french}le feu est parti, ça brûle, ça flambe; un incendie est parti} \zh{火烧着了 / 着火了}  
 ¶ \textcolor{darkblue}{\textbf{\ipa{mv̩˧ le˧-qæ˧˥ / mv̩˧ le˧-qæ˧-ze˥}}} \textcolor{PineGreen}{\selectlanguage{french}ça brûle; il y a un incendie} \zh{火在烧 / 着火了}  
 ¶ \textcolor{darkblue}{\textbf{\ipa{mv̩˧ qæ˥-ɻ̍˩}}} \textcolor{PineGreen}{\selectlanguage{french}ça brûle; il y a un incendie} \zh{火在烧 / 火烧着了}  
 ¶ \textcolor{darkblue}{\textbf{\ipa{mv̩˧ qæ˥-ɻ̍˩ kʰɯ˩}}} \textcolor{PineGreen}{\selectlanguage{french}lancer un incendie, déclencher un incendie, mettre le feu (acte criminel)} \zh{(有人)放火}  
 ¶ \textcolor{darkblue}{\textbf{\ipa{mv̩˧qæ˥-ɻ̍˩-hɯ˩}}} \textcolor{PineGreen}{\selectlanguage{french}un incendie est parti} \zh{(有人)放火了!}  

\lhead{\firstmark}
\rhead{\botmark}

\subsection{\hspace{-0.5cm} {\Large \textcolor{darkblue}{\textbf{\ipa{qæ˧˥}}} \textsubscript{2}}\hspace{0.5cm}[\kern2pt{\textcolor{darkblue}{\textbf{\ipa{qæ˧˥}}}}\kern2pt]} \hypertarget{q\{\string_M\string_T2}{}
\markboth{\textcolor{darkblue}{\textbf{\ipa{qæ˧˥}}} \textsubscript{2}}{}
\textcolor{teal}{\mytextsc{verbe}} \hspace{4pt} Ton~: MH.
\textcolor{PineGreen}{\selectlanguage{french}Souffrir, avoir mal.} \zh{疼。}  ¶ \textcolor{darkblue}{\textbf{\ipa{bi˧mi˧ qæ˧˥}}} \textcolor{PineGreen}{\selectlanguage{french}avoir mal au ventre} \zh{肚子疼}  
 ¶ \textcolor{darkblue}{\textbf{\ipa{ɬo˧kʰv̩˧ qæ˧˥}}} \textcolor{PineGreen}{\selectlanguage{french}avoir mal à la hanche} \zh{腰疼}  
 ¶ \textcolor{darkblue}{\textbf{\ipa{ʁo˧qʰwɤ˩ qæ˩}}} \textcolor{PineGreen}{\selectlanguage{french}avoir mal à la tête} \zh{头疼}  

\lhead{\firstmark}
\rhead{\botmark}

\subsection{\hspace{-0.5cm} {\Large \textcolor{darkblue}{\textbf{\ipa{qæ˩˥}}} \textsubscript{1}}\hspace{0.5cm}[\kern2pt{\textcolor{darkblue}{\textbf{\ipa{qæ˩˥}}}}\kern2pt]} \hypertarget{q\{\string_B\string_T1}{}
\markboth{\textcolor{darkblue}{\textbf{\ipa{qæ˩˥}}} \textsubscript{1}}{}
\textcolor{teal}{\mytextsc{nom}} \hspace{4pt} Ton~: LH.
\textcolor{PineGreen}{\selectlanguage{french}Huile (terme générique; huile de friture).} \zh{油,食用油。} 
\lhead{\firstmark}
\rhead{\botmark}

\subsection{\hspace{-0.5cm} {\Large \textcolor{darkblue}{\textbf{\ipa{qæ˩˥}}} \textsubscript{2}}\hspace{0.5cm}[\kern2pt{\textcolor{darkblue}{\textbf{\ipa{qæ˩˥}}}}\kern2pt]} \hypertarget{q\{\string_B\string_T2}{}
\markboth{\textcolor{darkblue}{\textbf{\ipa{qæ˩˥}}} \textsubscript{2}}{}
\textcolor{teal}{\mytextsc{nom}} \hspace{4pt} Ton~: LH.
\textcolor{PineGreen}{\selectlanguage{french}Colle.} \zh{胶。}  \zh{量词}: \textcolor{darkblue}{\textbf{\ipa{kʰwɤ˥}}}  \mytextsc{clf}~: \textcolor{darkblue}{\textbf{\ipa{kʰwɤ˥}}} 
\lhead{\firstmark}
\rhead{\botmark}

\subsection{\hspace{-0.5cm} {\Large \textcolor{darkblue}{\textbf{\ipa{qi˧qi˧}}}}\hspace{0.5cm}[\kern2pt{\textcolor{darkblue}{\textbf{\ipa{qi˧qi˧}}}}\kern2pt]} \hypertarget{qi\string_Mqi\string_M1}{}
\markboth{\textcolor{darkblue}{\textbf{\ipa{qi˧qi˧}}}}{}
\textcolor{teal}{\mytextsc{adverbe}} \hspace{4pt} Ton~: M.
\textcolor{PineGreen}{\selectlanguage{french}À l'origine.} \zh{原来、一开始。} 
\lhead{\firstmark}
\rhead{\botmark}

\subsection{\hspace{-0.5cm} {\Large \textcolor{darkblue}{\textbf{\ipa{qo˥}}} \textsubscript{1}}\hspace{0.5cm}[\kern2pt{\textcolor{darkblue}{\textbf{\ipa{qo˥}}}}\kern2pt]} \hypertarget{qo\string_T1}{}
\markboth{\textcolor{darkblue}{\textbf{\ipa{qo˥}}} \textsubscript{1}}{}
\textcolor{teal}{\mytextsc{verbe}} \hspace{4pt} Ton~: H.
\textcolor{PineGreen}{\selectlanguage{french}S'agenouiller (les mains au sol).} \zh{跪下。} 
\lhead{\firstmark}
\rhead{\botmark}

\subsection{\hspace{-0.5cm} {\Large \textcolor{darkblue}{\textbf{\ipa{qo˥}}} \textsubscript{2}}\hspace{0.5cm}[\kern2pt{\textcolor{darkblue}{\textbf{\ipa{qo˥}}}}\kern2pt]} \hypertarget{qo\string_T2}{}
\markboth{\textcolor{darkblue}{\textbf{\ipa{qo˥}}} \textsubscript{2}}{}
\textcolor{teal}{\mytextsc{verbe}} \hspace{4pt} Ton~: H.
\textcolor{PineGreen}{\selectlanguage{french}Aimer d'amour.} \zh{爱,关心。}  ¶ \textcolor{darkblue}{\textbf{\ipa{mɤ˧-qo˧}}} \textcolor{PineGreen}{\selectlanguage{french}\mytextsc{neg}} \zh{不爱}  
 ¶ \textcolor{darkblue}{\textbf{\ipa{zo˧mv̩˥zo˩ qo˩}}} \textcolor{PineGreen}{\selectlanguage{french}aimer (ses) enfants} \zh{爱孩子}  
 ¶ \textcolor{darkblue}{\textbf{\ipa{õ˧-hĩ˥ qo˩}}} \textcolor{PineGreen}{\selectlanguage{french}aimer sa famille} \zh{爱自己家人}  

\lhead{\firstmark}
\rhead{\botmark}

\subsection{\hspace{-0.5cm} {\Large \textcolor{darkblue}{\textbf{\ipa{-qo˧}}}}\hspace{0.5cm}[\kern2pt{\textcolor{darkblue}{\textbf{\ipa{qo˥}}}}\kern2pt]} \hypertarget{-qo\string_M1}{}
\markboth{\textcolor{darkblue}{\textbf{\ipa{-qo˧}}}}{}
\textcolor{teal}{\mytextsc{postposition}} \hspace{4pt} Ton~: M.
\textcolor{PineGreen}{\selectlanguage{french}Dans.} \zh{里。} \textit{Voir~:} \hyperlink{}{\textcolor{darkblue}{\textbf{\ipa{-qo˧lo˩}}}} 
\lhead{\firstmark}
\rhead{\botmark}

\subsection{\hspace{-0.5cm} {\Large \textcolor{darkblue}{\textbf{\ipa{-qo˧lo˩}}}}\hspace{0.5cm}[\kern2pt{\textcolor{darkblue}{\textbf{\ipa{qo˧lo˩}}}}\kern2pt]} \hypertarget{-qo\string_Mlo\string_B1}{}
\markboth{\textcolor{darkblue}{\textbf{\ipa{-qo˧lo˩}}}}{}
\textcolor{teal}{\mytextsc{postposition}} \hspace{4pt} Ton~: L\#.
\textcolor{PineGreen}{\selectlanguage{french}Dans.} \zh{里面。}  ¶ \textcolor{darkblue}{\textbf{\ipa{ʈʂʰɯ˧ | ɑ˩ʁo˧-qo˧lo˩ dʑo˩}}} \textcolor{PineGreen}{\selectlanguage{french}Il/elle est à la maison/dans la maison.} \zh{他在家里。}  
\textit{Voir~:} \hyperlink{}{\textcolor{darkblue}{\textbf{\ipa{qo˧lo˩}}}} 
\lhead{\firstmark}
\rhead{\botmark}

\subsection{\hspace{-0.5cm} {\Large \textcolor{darkblue}{\textbf{\ipa{qo˧lo˩}}}}\hspace{0.5cm}[\kern2pt{\textcolor{darkblue}{\textbf{\ipa{qo˧lo˩}}}}\kern2pt]} \hypertarget{qo\string_Mlo\string_B1}{}
\markboth{\textcolor{darkblue}{\textbf{\ipa{qo˧lo˩}}}}{}
\textcolor{teal}{\mytextsc{adverbe}} \hspace{4pt} Ton~: L\#.
\textcolor{PineGreen}{\selectlanguage{french}Dedans, à l'intérieur de, dans.} \zh{里面。} \textit{Voir~:} \hyperlink{}{\textcolor{darkblue}{\textbf{\ipa{-qo˧lo˩}}}} 
\lhead{\firstmark}
\rhead{\botmark}

\subsection{\hspace{-0.5cm} {\Large \textcolor{darkblue}{\textbf{\ipa{qo˧pv̩˩}}}}\hspace{0.5cm}[\kern2pt{\textcolor{darkblue}{\textbf{\ipa{qo˧pv̩˩}}}}\kern2pt]} \hypertarget{qo\string_Mpv\string_=\string_B1}{}
\markboth{\textcolor{darkblue}{\textbf{\ipa{qo˧pv̩˩}}}}{}
\textcolor{teal}{\mytextsc{nom}} \hspace{4pt} Ton~: L\#.
\textcolor{PineGreen}{\selectlanguage{french}Coucou.} \zh{布谷鸟。}  ¶ \textcolor{darkblue}{\textbf{\ipa{qo˧pv̩˩-ɻwæ˩ | ɖɯ˧-ɲi˥}}} \textcolor{PineGreen}{\selectlanguage{french}Le Jour des Ancêtres, au 1er jour du 5e mois. Littéralement: “le jour où chante le coucou”.} \zh{清明节。直译:“布谷鸟叫的那天”}  
 \zh{量词}: \textcolor{darkblue}{\textbf{\ipa{mi˩}}}  \mytextsc{clf}~: \textcolor{darkblue}{\textbf{\ipa{mi˩}}} 
\lhead{\firstmark}
\rhead{\botmark}

\subsection{\hspace{-0.5cm} {\Large \textcolor{darkblue}{\textbf{\ipa{qo˧pv̩˩-ʐwæ˩ɖʐæ˩}}}}\hspace{0.5cm}[\kern2pt{\textcolor{darkblue}{\textbf{\ipa{qo˩pv̩˧ʐwæ˧ɖʐæ˧}}}}\kern2pt]} \hypertarget{qo\string_Mpv\string_=\string_B-z`w\{\string_Bd`z`\{\string_B1}{}
\markboth{\textcolor{darkblue}{\textbf{\ipa{qo˧pv̩˩-ʐwæ˩ɖʐæ˩}}}}{}
\textcolor{teal}{\mytextsc{nom}} \hspace{4pt} Ton~: LM-.
\textcolor{PineGreen}{\selectlanguage{french}Geai, \textit{Garrulus glandarius sinensis}.} \zh{松鸦。} 
\lhead{\firstmark}
\rhead{\botmark}

\subsection{\hspace{-0.5cm} {\Large \textcolor{darkblue}{\textbf{\ipa{qo˧tv̩˩}}}}\hspace{0.5cm}[\kern2pt{\textcolor{darkblue}{\textbf{\ipa{qo˧tv̩˩}}}}\kern2pt]} \hypertarget{qo\string_Mtv\string_=\string_B1}{}
\markboth{\textcolor{darkblue}{\textbf{\ipa{qo˧tv̩˩}}}}{}
\textcolor{teal}{\mytextsc{nom}} \hspace{4pt} Ton~: L\#/LM.
\textcolor{PineGreen}{\selectlanguage{french}Noyau (aussi pour: graines de tournesol; et pour: bobines de fil).} \zh{果核。}  ¶ \textcolor{darkblue}{\textbf{\ipa{dʑi˧ʁo˩-qo˩tv̩˩}}} \textcolor{PineGreen}{\selectlanguage{french}noyau de pêche} \zh{桃子果核}  
 \zh{量词}: \textcolor{darkblue}{\textbf{\ipa{ɭɯ˧}}}  \mytextsc{clf}~: \textcolor{darkblue}{\textbf{\ipa{ɭɯ˧}}} 
\lhead{\firstmark}
\rhead{\botmark}

\subsection{\hspace{-0.5cm} {\Large \textcolor{darkblue}{\textbf{\ipa{qo˩\textsubscript{a}}}}}\hspace{0.5cm}[\kern2pt{\textcolor{darkblue}{\textbf{\ipa{qo˩˥}}}}\kern2pt]} \hypertarget{qo\string_Ba1}{}
\markboth{\textcolor{darkblue}{\textbf{\ipa{qo˩\textsubscript{a}}}}}{}
\textcolor{teal}{\mytextsc{verbe}} \hspace{4pt} Ton~: L\textsubscript{a}.
\textcolor{PineGreen}{\selectlanguage{french}Garder, serrer, ranger (de la nourriture dans un récipient à l'abri des mouches).} \zh{放、储存。} 
\lhead{\firstmark}
\rhead{\botmark}

\subsection{\hspace{-0.5cm} {\Large \textcolor{darkblue}{\textbf{\ipa{qo˩ho˧˥}}}}\hspace{0.5cm}[\kern2pt{\textcolor{darkblue}{\textbf{\ipa{qo˩ho˧˥}}}}\kern2pt]} \hypertarget{qo\string_Bho\string_M\string_T1}{}
\markboth{\textcolor{darkblue}{\textbf{\ipa{qo˩ho˧˥}}}}{}
\textcolor{teal}{\mytextsc{nom}} \hspace{4pt} Ton~: LM+MH\#.
\textcolor{PineGreen}{\selectlanguage{french}Boîte en vannerie ronde, dans laquelle on place les cadeaux qu’on vient offrir; est formée de deux parties qui s’emboîtent; on la porte lorsqu'on se rend chez quelqu'un dans le cadre d'un événement social important. Cf récit F4. Une photo de cet objet est présente dans le rapport d'enquête de terrain publié en 1986 en 3 volumes (永宁纳西族……调查).} \zh{礼盒。}  \zh{量词}: \textcolor{darkblue}{\textbf{\ipa{ɭɯ˧}}}  \mytextsc{clf}~: \textcolor{darkblue}{\textbf{\ipa{ɭɯ˧}}} 
\lhead{\firstmark}
\rhead{\botmark}

\subsection{\hspace{-0.5cm} {\Large \textcolor{darkblue}{\textbf{\ipa{qo˩qɑ˩}}}}\hspace{0.5cm}[\kern2pt{\textcolor{darkblue}{\textbf{\ipa{qo˩qɑ˩˥}}}}\kern2pt]} \hypertarget{qo\string_BqA\string_B1}{}
\markboth{\textcolor{darkblue}{\textbf{\ipa{qo˩qɑ˩}}}}{}
\textcolor{teal}{\mytextsc{nom}} \hspace{4pt} Ton~: L.
\textcolor{PineGreen}{\selectlanguage{french}Col (de montagne).} \zh{垭口。}  \zh{量词}: \textcolor{darkblue}{\textbf{\ipa{ɭɯ˧}}}  \mytextsc{clf}~: \textcolor{darkblue}{\textbf{\ipa{ɭɯ˧}}} 
\lhead{\firstmark}
\rhead{\botmark}

\subsection{\hspace{-0.5cm} {\Large \textcolor{darkblue}{\textbf{\ipa{qo˩tv̩˩-lv̩˥}}}}\hspace{0.5cm}[\kern2pt{\textcolor{darkblue}{\textbf{\ipa{xxxx non-correspondance entre le nombre de morphèmes et le nombre de tons de morphèmes}}}}\kern2pt]} \hypertarget{qo\string_Btv\string_=\string_B-lv\string_=\string_T1}{}
\markboth{\textcolor{darkblue}{\textbf{\ipa{qo˩tv̩˩-lv̩˥}}}}{}
\textcolor{teal}{\mytextsc{nom}} \hspace{4pt} Ton~: L+H\#.
\textcolor{PineGreen}{\selectlanguage{french}Boule.} \zh{团。}  ¶ \textcolor{darkblue}{\textbf{\ipa{li˩-qo˩tv̩˥-lv̩˩}}} \textcolor{PineGreen}{\selectlanguage{french}thé comprimé en boule} \zh{沱茶}  
 ¶ \textcolor{darkblue}{\textbf{\ipa{li˩-qo˩tv̩˥-lv̩˩ | ɖɯ˧-qʰwɤ˧ tɕɤ˥}}} \textcolor{PineGreen}{\selectlanguage{french}préparer un bol de thé avec du thé comprimé en boule} \zh{煮一碗沱茶}  
 \zh{量词}: \textcolor{darkblue}{\textbf{\ipa{ɭɯ˧}}}  \mytextsc{clf}~: \textcolor{darkblue}{\textbf{\ipa{ɭɯ˧}}} 
\lhead{\firstmark}
\rhead{\botmark}

\subsection{\hspace{-0.5cm} {\Large \textcolor{darkblue}{\textbf{\ipa{qv̩˩˥}}}}\hspace{0.5cm}[\kern2pt{\textcolor{darkblue}{\textbf{\ipa{qv̩˩˥}}}}\kern2pt]} \hypertarget{qv\string_=\string_B\string_T1}{}
\markboth{\textcolor{darkblue}{\textbf{\ipa{qv̩˩˥}}}}{}
\textcolor{teal}{\mytextsc{nom}} \hspace{4pt} Ton~: LH.
\textcolor{PineGreen}{\selectlanguage{french}Poignée, manche (d'une valise, d'une bouteille thermos, d'une louche...).} \zh{把手。}  \zh{量词}: \textcolor{darkblue}{\textbf{\ipa{kʰwɤ˥}}}  \mytextsc{clf}~: \textcolor{darkblue}{\textbf{\ipa{kʰwɤ˥}}} 
\lhead{\firstmark}
\rhead{\botmark}

\subsection{\hspace{-0.5cm} {\Large \textcolor{darkblue}{\textbf{\ipa{qv̩˧˥}}}}\hspace{0.5cm}[\kern2pt{\textcolor{darkblue}{\textbf{\ipa{qv̩˧˥}}}}\kern2pt]} \hypertarget{qv\string_=\string_M\string_T1}{}
\markboth{\textcolor{darkblue}{\textbf{\ipa{qv̩˧˥}}}}{}
\textcolor{teal}{\mytextsc{verbe}} \hspace{4pt} Ton~: MH.
\textcolor{PineGreen}{\selectlanguage{french}Faire peur, effrayer.} \zh{吓(吓唬)。}  ¶ \textcolor{darkblue}{\textbf{\ipa{hĩ˧ qv̩˩}}} \textcolor{PineGreen}{\selectlanguage{french}faire peur aux gens} \zh{吓人}  
 ¶ \textcolor{darkblue}{\textbf{\ipa{no˧ | hĩ˧ qv̩˩-zo˩! / ʈʂʰɯ˧-ɳɯ˧ | hĩ˧ qv̩˩-zo˩!}}} \textcolor{PineGreen}{\selectlanguage{french}tu fais peur aux gens! Il fait peur aux gens!} \zh{你吓人! / 他吓人!}  
 ¶ \textcolor{darkblue}{\textbf{\ipa{ʈʂʰɯ˧ | njæ˩ qv̩˩-tsʰɯ˩˥!}}} \textcolor{PineGreen}{\selectlanguage{french}il fait peur!} \zh{他吓人!}  
 ¶ \textcolor{darkblue}{\textbf{\ipa{njɤ˧ɳɯ˧ | ʈʂʰɯ˧ qv̩˩-bi˩!}}} \textcolor{PineGreen}{\selectlanguage{french}Je vais lui faire peur!} \zh{我要吓唬他一下!}  
 ¶ \textcolor{darkblue}{\textbf{\ipa{tʰɑ˧-qv̩˧˥!}}} \textcolor{PineGreen}{\selectlanguage{french}\mytextsc{prohib}} \zh{别吓唬(人家)!}  

\lhead{\firstmark}
\rhead{\botmark}

\subsection{\hspace{-0.5cm} {\Large \textcolor{darkblue}{\textbf{\ipa{qv̩˩\textsubscript{a}}}}}\hspace{0.5cm}[\kern2pt{\textcolor{darkblue}{\textbf{\ipa{qv̩˩˥}}}}\kern2pt]} \hypertarget{qv\string_=\string_Ba1}{}
\markboth{\textcolor{darkblue}{\textbf{\ipa{qv̩˩\textsubscript{a}}}}}{}
\textcolor{teal}{\mytextsc{verbe}} \hspace{4pt} Ton~: L\textsubscript{a}.
\textcolor{PineGreen}{\selectlanguage{french}Faire rouler, emporter, charrier (un objet lourd: par exemple, le courant emporte des cailloux, les charriant au loin; le verbe ne peut s'employer pour des objets légers, par exemple des feuilles).} \zh{冲走。}  ¶ \textcolor{darkblue}{\textbf{\ipa{le˧-qv̩˩ | le˧-po˧-tsʰɯ˧˥}}} \textcolor{PineGreen}{\selectlanguage{french}charrier, amener en faisant rouler: un torrent en crue charrie des cailloux jusque dans la plaine} \zh{冲到某个地方}  
 ¶ \textcolor{darkblue}{\textbf{\ipa{lv̩˧mi˧ | ɬi˧dʑɯ˩-ɳɯ˩ | qv̩˩˥.}}} \textcolor{PineGreen}{\selectlanguage{french}les pierres sont amenées par (le courant de) la rivière de Yongning} \zh{石头被永宁河水冲(到坝子)}  
 ¶ \textcolor{darkblue}{\textbf{\ipa{dʑɯ˧-ɳɯ˧ | le˧-qv̩˩ | le˧-po˧-tsʰɯ˧-hĩ˥ | lv̩˧mi˧}}} \textcolor{PineGreen}{\selectlanguage{french}pierres amenées par la rivière, pierres charriées (jusqu'ici) par la rivière} \zh{水流冲下来的石头}  

\lhead{\firstmark}
\rhead{\botmark}

\subsection{\hspace{-0.5cm} {\Large \textcolor{darkblue}{\textbf{\ipa{qv̩˧dzi˩}}}}\hspace{0.5cm}[\kern2pt{\textcolor{darkblue}{\textbf{\ipa{qv̩˧dzi˩}}}}\kern2pt]} \hypertarget{qv\string_=\string_Mdzi\string_B1}{}
\markboth{\textcolor{darkblue}{\textbf{\ipa{qv̩˧dzi˩}}}}{}
\textcolor{teal}{\mytextsc{nom}} \hspace{4pt} Ton~: L\#.
\textcolor{PineGreen}{\selectlanguage{french}\textit{Pinus massoniana D.Don in Lamb.}, conifère de la famille des \textit{Pinaceae}. Ses pignes ne sont pas comestibles: les poissons les mangent, mais pour les hommes elles sont vénéneuses.} \zh{马尾松。} Dialecte chinois local~:\zh{马松树。} ¶ \textcolor{darkblue}{\textbf{\ipa{qv̩˧dzi˩-lv̩˩\textasciitilde{}lv̩˩, | dzɯ˧ mɤ˧-ɖo˧!}}} \textcolor{PineGreen}{\selectlanguage{french}Il ne faut pas manger les pignes du pin de Masson! (Elles sont vénéneuses.)} \zh{马松树的果子,不要吃!(有毒)}  
 \zh{量词}: \textcolor{darkblue}{\textbf{\ipa{dzi˩, ʝi˧}}}  \mytextsc{clf}~: \textcolor{darkblue}{\textbf{\ipa{dzi˩, ʝi˧}}} 
\lhead{\firstmark}
\rhead{\botmark}

\subsection{\hspace{-0.5cm} {\Large \textcolor{darkblue}{\textbf{\ipa{qv̩˧ɻ\#˥}}}}\hspace{0.5cm}[\kern2pt{\textcolor{darkblue}{\textbf{\ipa{qv̩˧ɻ˧}}}}\kern2pt]} \hypertarget{qv\string_=\string_Mr£`\#\string_T1}{}
\markboth{\textcolor{darkblue}{\textbf{\ipa{qv̩˧ɻ\#˥}}}}{}
\textcolor{teal}{\mytextsc{nom}} \hspace{4pt} Ton~: \#H.
\textcolor{PineGreen}{\selectlanguage{french}Une montagne de Yongning.} \zh{永宁的一座山。}  ¶ \textcolor{darkblue}{\textbf{\ipa{qv̩˧ɻ̍˧-ʁo˧-qʰwɤ˥}}} \textcolor{PineGreen}{\selectlanguage{french}le sommet de la montagne \textcolor{darkblue}{\textbf{\ipa{/qv̩˧ɻ̍˧/}}}} \zh{\textcolor{darkblue}{\textbf{\ipa{/qv̩˧ɻ̍˧/}}}山的山顶}  

\lhead{\firstmark}
\rhead{\botmark}

\subsection{\hspace{-0.5cm} {\Large \textcolor{darkblue}{\textbf{\ipa{qv̩˧tɕi˥}}}}\hspace{0.5cm}[\kern2pt{\textcolor{darkblue}{\textbf{\ipa{qv̩˧tɕi˥}}}}\kern2pt]} \hypertarget{qv\string_=\string_Mts£i\string_T1}{}
\markboth{\textcolor{darkblue}{\textbf{\ipa{qv̩˧tɕi˥}}}}{}
\textcolor{teal}{\mytextsc{nom}} \hspace{4pt} Ton~: H\#.
\textcolor{PineGreen}{\selectlanguage{french}Crachat, mucus.} \zh{痰。} 
\lhead{\firstmark}
\rhead{\botmark}

\subsection{\hspace{-0.5cm} {\Large \textcolor{darkblue}{\textbf{\ipa{qv̩˧ʈʂæ˧˥}}}}\hspace{0.5cm}[\kern2pt{\textcolor{darkblue}{\textbf{\ipa{qv̩˧ʈʂæ˧˥}}}}\kern2pt]} \hypertarget{qv\string_=\string_Mt`s`\{\string_M\string_T1}{}
\markboth{\textcolor{darkblue}{\textbf{\ipa{qv̩˧ʈʂæ˧˥}}}}{}
\textcolor{teal}{\mytextsc{nom}} \hspace{4pt} Ton~: MH\#.
\ding{202} \textcolor{PineGreen}{\selectlanguage{french}Gorge.} \zh{喉咙。}  \zh{量词}: \textcolor{darkblue}{\textbf{\ipa{ɭɯ˧}}} \ding{203} \textcolor{PineGreen}{\selectlanguage{french}Voix.} \zh{声音。}  ¶ \textcolor{darkblue}{\textbf{\ipa{ʈʂʰɯ˧ | qv̩˧ʈʂæ˧ dʑɤ˥!}}} \textcolor{PineGreen}{\selectlanguage{french}Elle/il a une belle voix.} \zh{他嗓子好。}  
 ¶ \textcolor{darkblue}{\textbf{\ipa{ʈʂʰɯ˧ | qv̩˧ʈʂæ˧˥ | ɖwæ˧˥ | dʑɤ˩˥!}}} \textcolor{PineGreen}{\selectlanguage{french}Elle/il a une très belle voix.} \zh{他嗓子很好。}  
 \mytextsc{clf}~: \textcolor{darkblue}{\textbf{\ipa{ɭɯ˧}}} 
\lhead{\firstmark}
\rhead{\botmark}

\subsection{\hspace{-0.5cm} {\Large \textcolor{darkblue}{\textbf{\ipa{qwɑ˧mæ\#˥}}}}\hspace{0.5cm}[\kern2pt{\textcolor{darkblue}{\textbf{\ipa{qwɑ˧mæ˧}}}}\kern2pt]} \hypertarget{qwA\string_Mm\{\#\string_T1}{}
\markboth{\textcolor{darkblue}{\textbf{\ipa{qwɑ˧mæ\#˥}}}}{}
\textcolor{teal}{\mytextsc{nom}} \hspace{4pt} Ton~: \#H.
\textcolor{PineGreen}{\selectlanguage{french}Partie médiane du foyer: sur la partie surélevée, mais “côté cuisine”, pas la partie la plus noble de l'espace où on prend les repas.} \zh{主屋的中庭:在主屋上半部分与门之间的空间。}  \zh{量词}: \textcolor{darkblue}{\textbf{\ipa{kʰwɤ˥}}}  \mytextsc{clf}~: \textcolor{darkblue}{\textbf{\ipa{kʰwɤ˥}}} 
\lhead{\firstmark}
\rhead{\botmark}

\subsection{\hspace{-0.5cm} {\Large \textcolor{darkblue}{\textbf{\ipa{qwæ˧}}}}\hspace{0.5cm}[\kern2pt{\textcolor{darkblue}{\textbf{\ipa{qwæ˥}}}}\kern2pt]} \hypertarget{qw\{\string_M1}{}
\markboth{\textcolor{darkblue}{\textbf{\ipa{qwæ˧}}}}{}
\textcolor{teal}{\mytextsc{nom}} \hspace{4pt} Ton~: M.
\textcolor{PineGreen}{\selectlanguage{french}Sommier (de lit); banc large.} \zh{床垫子。}  ¶ \textcolor{darkblue}{\textbf{\ipa{qwæ˧mi\#˥}}} \textcolor{PineGreen}{\selectlanguage{french}grand sommier} \zh{大床垫子}  
 \zh{量词}: \textcolor{darkblue}{\textbf{\ipa{nɑ˧}}}  \mytextsc{clf}~: \textcolor{darkblue}{\textbf{\ipa{nɑ˧}}} 
\lhead{\firstmark}
\rhead{\botmark}

\subsection{\hspace{-0.5cm} {\Large \textcolor{darkblue}{\textbf{\ipa{qwæ˧lo˧˥}}}}\hspace{0.5cm}[\kern2pt{\textcolor{darkblue}{\textbf{\ipa{qwæ˧lo˧}}}}\kern2pt]} \hypertarget{qw\{\string_Mlo\string_M\string_T1}{}
\markboth{\textcolor{darkblue}{\textbf{\ipa{qwæ˧lo˧˥}}}}{}
\textcolor{teal}{\mytextsc{nom}} \hspace{4pt} Ton~: MH\#.
\textcolor{PineGreen}{\selectlanguage{french}Petit passage, petit sentier.} \zh{过道、小道。}  ¶ \textcolor{darkblue}{\textbf{\ipa{qwæ˧lo˧-qo˥ | gɤ˩tɕo˧ le˧-jo˩}}} \textcolor{PineGreen}{\selectlanguage{french}venir par le petit chemin/la sente (contexte: on demande à l'enquêteur si, pour se rendre de la maison de la consultante à son hameau natal, tout proche, il est passé par la rue principale de Yongning, ou a emprunté le petit chemin de derrière, parmi les champs)} \zh{抄小道}  
 \zh{量词}: \textcolor{darkblue}{\textbf{\ipa{kʰɯ˩}}}  \mytextsc{clf}~: \textcolor{darkblue}{\textbf{\ipa{kʰɯ˩}}} 
\lhead{\firstmark}
\rhead{\botmark}

\subsection{\hspace{-0.5cm} {\Large \textcolor{darkblue}{\textbf{\ipa{qwæ˧ʁo\#˥}}}}\hspace{0.5cm}[\kern2pt{\textcolor{darkblue}{\textbf{\ipa{qwæ˧ʁo˧}}}}\kern2pt]} \hypertarget{qw\{\string_MRo\#\string_T1}{}
\markboth{\textcolor{darkblue}{\textbf{\ipa{qwæ˧ʁo\#˥}}}}{}
\textcolor{teal}{\mytextsc{nom}} \hspace{4pt} Ton~: \#H.
\textcolor{PineGreen}{\selectlanguage{french}Banc de la pièce principale, proche du foyer, où s'asseyent les hôtes.} \zh{主屋里面的长凳:客人和老人坐的地方。}  \zh{量词}: \textcolor{darkblue}{\textbf{\ipa{ɭɯ˧}}}  \mytextsc{clf}~: \textcolor{darkblue}{\textbf{\ipa{ɭɯ˧}}} \textit{Voir~:} \hyperlink{}{\textcolor{darkblue}{\textbf{\ipa{qwæ˧˥}}} \textsubscript{3}} 
\lhead{\firstmark}
\rhead{\botmark}

\subsection{\hspace{-0.5cm} {\Large \textcolor{darkblue}{\textbf{\ipa{qwæ˧ʂe\#˥}}}}\hspace{0.5cm}[\kern2pt{\textcolor{darkblue}{\textbf{\ipa{qwæ˧ʂe˥}}}}\kern2pt]} \hypertarget{qw\{\string_Ms`e\#\string_T1}{}
\markboth{\textcolor{darkblue}{\textbf{\ipa{qwæ˧ʂe\#˥}}}}{}
\textcolor{teal}{\mytextsc{nom}} \hspace{4pt} Ton~: \#H.
\textcolor{PineGreen}{\selectlanguage{french}Punaise.} \zh{臭虫。}  \zh{量词}: \textcolor{darkblue}{\textbf{\ipa{mi˩}}}  \mytextsc{clf}~: \textcolor{darkblue}{\textbf{\ipa{mi˩}}} 
\lhead{\firstmark}
\rhead{\botmark}

\subsection{\hspace{-0.5cm} {\Large \textcolor{darkblue}{\textbf{\ipa{qwæ˧ʂe˧lɑ˧bv̩˥}}}}\hspace{0.5cm}[\kern2pt{\textcolor{darkblue}{\textbf{\ipa{qwæ˧ʂe˧lɑ˧bv̩˧}}}}\kern2pt]} \hypertarget{qw\{\string_Ms`e\string_MlA\string_Mbv\string_=\string_T1}{}
\markboth{\textcolor{darkblue}{\textbf{\ipa{qwæ˧ʂe˧lɑ˧bv̩˥}}}}{}
\textcolor{teal}{\mytextsc{nom}} \hspace{4pt} Ton~: H\#.
\textcolor{PineGreen}{\selectlanguage{french}Sorte de ver.} \zh{一种蠕虫。}  \zh{量词}: \textcolor{darkblue}{\textbf{\ipa{mi˩}}}  \mytextsc{clf}~: \textcolor{darkblue}{\textbf{\ipa{mi˩}}} 
\lhead{\firstmark}
\rhead{\botmark}

\subsection{\hspace{-0.5cm} {\Large \textcolor{darkblue}{\textbf{\ipa{qwæ˧zo˧zo˩}}}}\hspace{0.5cm}[\kern2pt{\textcolor{darkblue}{\textbf{\ipa{qwæ˧zo˧zo˥}}}}\kern2pt]} \hypertarget{qw\{\string_Mzo\string_Mzo\string_B1}{}
\markboth{\textcolor{darkblue}{\textbf{\ipa{qwæ˧zo˧zo˩}}}}{}
\textcolor{teal}{\mytextsc{nom}} \hspace{4pt} Ton~: L\#.
\textcolor{PineGreen}{\selectlanguage{french}Banc de la pièce principale, proche du foyer, où s'asseyent les hôtes.} \zh{主屋的长凳,离火塘近。这是客人的尊座。}  \zh{量词}: \textcolor{darkblue}{\textbf{\ipa{pɤ˩}}}  \mytextsc{clf}~: \textcolor{darkblue}{\textbf{\ipa{pɤ˩}}} 
\lhead{\firstmark}
\rhead{\botmark}

\subsection{\hspace{-0.5cm} {\Large \textcolor{darkblue}{\textbf{\ipa{qwæ˩ɖʐæ˩}}}}\hspace{0.5cm}[\kern2pt{\textcolor{darkblue}{\textbf{\ipa{qwæ˧ɖʐæ˧˥}}}}\kern2pt]} \hypertarget{qw\{\string_Bd`z`\{\string_B1}{}
\markboth{\textcolor{darkblue}{\textbf{\ipa{qwæ˩ɖʐæ˩}}}}{}
\textcolor{teal}{\mytextsc{nom}} \hspace{4pt} Ton~: L.
\textcolor{PineGreen}{\selectlanguage{french}Mâchoire; bouche.} \zh{颚、嘴、嘴巴、口。}  ¶ \textcolor{darkblue}{\textbf{\ipa{qwæ˩ɖʐæ˩-qo˥-ɳɯ˩ | ʈʰæ˧˥}}} \textcolor{PineGreen}{\selectlanguage{french}mastiquer, ronger} \zh{咬在嘴里}  
 \zh{量词}: \textcolor{darkblue}{\textbf{\ipa{ɭɯ˧}}}  \mytextsc{clf}~: \textcolor{darkblue}{\textbf{\ipa{ɭɯ˧}}} 
\lhead{\firstmark}
\rhead{\botmark}

\subsection{\hspace{-0.5cm} {\Large \textcolor{darkblue}{\textbf{\ipa{qwæ˩\textasciitilde{}qwæ˧˥}}}}\hspace{0.5cm}[\kern2pt{\textcolor{darkblue}{\textbf{\ipa{qwæ˧qwæ˩}}}}\kern2pt]} \hypertarget{qw\{\string_B~qw\{\string_M\string_T1}{}
\markboth{\textcolor{darkblue}{\textbf{\ipa{qwæ˩\textasciitilde{}qwæ˧˥}}}}{}
\textcolor{teal}{\mytextsc{verbe}} \hspace{4pt} Ton~: .
\textcolor{PineGreen}{\selectlanguage{french}Se gratter; gratter, gratouiller.} \zh{抠痒。}  ¶ \textcolor{darkblue}{\textbf{\ipa{le˧-qwæ˧\textasciitilde{}qwæ˩-ze˩}}} \textcolor{PineGreen}{\selectlanguage{french}\mytextsc{accomp} \string_ \mytextsc{red} \mytextsc{pfv}} \zh{\mytextsc{accomp} \string_ \mytextsc{red} \mytextsc{pfv}}  

\lhead{\firstmark}
\rhead{\botmark}

\subsection{\hspace{-0.5cm} {\Large \textcolor{darkblue}{\textbf{\ipa{qwæ˩ʂv̩˧˥}}}}\hspace{0.5cm}[\kern2pt{\textcolor{darkblue}{\textbf{\ipa{qwæ˧ʂv̩˧}}}}\kern2pt]} \hypertarget{qw\{\string_Bs`v\string_=\string_M\string_T1}{}
\markboth{\textcolor{darkblue}{\textbf{\ipa{qwæ˩ʂv̩˧˥}}}}{}
\textcolor{teal}{\mytextsc{nom}} \hspace{4pt} Ton~: LM+MH\#.
\textcolor{PineGreen}{\selectlanguage{french}Mors.} \zh{马嚼子。}  ¶ \textcolor{darkblue}{\textbf{\ipa{ʐwæ˧-qwæ˥ʂv̩˩}}} \textcolor{PineGreen}{\selectlanguage{french}mors de cheval} \zh{马嚼子}  
 \zh{量词}: \textcolor{darkblue}{\textbf{\ipa{nɑ˧}}}  \mytextsc{clf}~: \textcolor{darkblue}{\textbf{\ipa{nɑ˧}}} 
\lhead{\firstmark}
\rhead{\botmark}

\subsection{\hspace{-0.5cm} {\Large \textcolor{darkblue}{\textbf{\ipa{qwæ˧˥}}} \textsubscript{1}}\hspace{0.5cm}[\kern2pt{\textcolor{darkblue}{\textbf{\ipa{qwæ˧˥}}}}\kern2pt]} \hypertarget{qw\{\string_M\string_T1}{}
\markboth{\textcolor{darkblue}{\textbf{\ipa{qwæ˧˥}}} \textsubscript{1}}{}
\textcolor{teal}{\mytextsc{verbe}} \hspace{4pt} Ton~: MH.
\ding{202} \textcolor{PineGreen}{\selectlanguage{french}Creuser, piocher (dans la terre meuble).} \zh{挖(土)。}  ¶ \textcolor{darkblue}{\textbf{\ipa{tv̩˧qʰv̩˧ qwæ˧˥}}} \textcolor{PineGreen}{\selectlanguage{french}creuser un trou} \zh{挖洞}  
 ¶ \textcolor{darkblue}{\textbf{\ipa{ʈʂe˧ qwæ˩}}} \textcolor{PineGreen}{\selectlanguage{french}creuser la terre} \zh{挖土}  
 ¶ \textcolor{darkblue}{\textbf{\ipa{qʰæ˧lo˧ qwæ˥}}} \textcolor{PineGreen}{\selectlanguage{french}dégager une rigole} \zh{挖水沟}  
 ¶ \textcolor{darkblue}{\textbf{\ipa{jɤ˩jo˥ qwæ˩}}} \textcolor{PineGreen}{\selectlanguage{french}déterrer des pommes de terre, récolter des pommes de terre} \zh{挖洋芋}  
\ding{203} \textcolor{PineGreen}{\selectlanguage{french}Puiser (de l'eau).} \zh{舀(水)。}  ¶ \textcolor{darkblue}{\textbf{\ipa{dʑɯ˩ qwæ˩˥}}} \textcolor{PineGreen}{\selectlanguage{french}puiser de l'eau} \zh{舀水}  

\lhead{\firstmark}
\rhead{\botmark}

\subsection{\hspace{-0.5cm} {\Large \textcolor{darkblue}{\textbf{\ipa{qwæ˧˥}}} \textsubscript{2}}\hspace{0.5cm}[\kern2pt{\textcolor{darkblue}{\textbf{\ipa{qwæ˧˥}}}}\kern2pt]} \hypertarget{qw\{\string_M\string_T2}{}
\markboth{\textcolor{darkblue}{\textbf{\ipa{qwæ˧˥}}} \textsubscript{2}}{}
\textcolor{teal}{\mytextsc{verbe}} \hspace{4pt} Ton~: MH.
\textcolor{PineGreen}{\selectlanguage{french}Graver.} \zh{雕刻。}  ¶ \textcolor{darkblue}{\textbf{\ipa{bæ˩bæ˩ qwæ˥}}} \textcolor{PineGreen}{\selectlanguage{french}graver une fleur} \zh{刻花}  
 ¶ \textcolor{darkblue}{\textbf{\ipa{qwæ˩\textasciitilde{}qwæ˧˥}}} \textcolor{PineGreen}{\selectlanguage{french}\mytextsc{red}} \zh{\mytextsc{重叠}}  
 ¶ \textcolor{darkblue}{\textbf{\ipa{bæ˩bæ˩ qwæ˥\textasciitilde{}qwæ˩}}} \textcolor{PineGreen}{\selectlanguage{french}graver des fleurs} \zh{刻一朵花}  

\lhead{\firstmark}
\rhead{\botmark}

\subsection{\hspace{-0.5cm} {\Large \textcolor{darkblue}{\textbf{\ipa{qwæ˧˥}}} \textsubscript{3}}\hspace{0.5cm}[\kern2pt{\textcolor{darkblue}{\textbf{\ipa{qwæ˧˥}}}}\kern2pt]} \hypertarget{qw\{\string_M\string_T3}{}
\markboth{\textcolor{darkblue}{\textbf{\ipa{qwæ˧˥}}} \textsubscript{3}}{}
\textcolor{teal}{\mytextsc{nom}} \hspace{4pt} Ton~: \#H.
\textcolor{PineGreen}{\selectlanguage{french}Banc de la pièce principale, proche du foyer, où s'asseyent les hôtes.} \zh{主屋里面的长凳:客人和老人坐的地方。}  \zh{量词}: \textcolor{darkblue}{\textbf{\ipa{ɭɯ˧}}}  \mytextsc{clf}~: \textcolor{darkblue}{\textbf{\ipa{ɭɯ˧}}} \textit{Voir~:} \hyperlink{}{\textcolor{darkblue}{\textbf{\ipa{qwæ˧ʁo\#˥}}}} 
\lhead{\firstmark}
\rhead{\botmark}

\subsection{\hspace{-0.5cm} {\Large \textcolor{darkblue}{\textbf{\ipa{qwæ˩˥}}}}\hspace{0.5cm}[\kern2pt{\textcolor{darkblue}{\textbf{\ipa{qwæ˩˥}}}}\kern2pt]} \hypertarget{qw\{\string_B\string_T1}{}
\markboth{\textcolor{darkblue}{\textbf{\ipa{qwæ˩˥}}}}{}
\textcolor{teal}{\mytextsc{nom}} \hspace{4pt} Ton~: LH.
\textcolor{PineGreen}{\selectlanguage{french}Mâchoire (monosyllabe).} \zh{嘴巴(单音节)。}  \zh{量词}: \textcolor{darkblue}{\textbf{\ipa{ɭɯ˧}}}  \mytextsc{clf}~: \textcolor{darkblue}{\textbf{\ipa{ɭɯ˧}}} 
\lhead{\firstmark}
\rhead{\botmark}

\subsection{\hspace{-0.5cm} {\Large \textcolor{darkblue}{\textbf{\ipa{qwɤ˧}}}}\hspace{0.5cm}[\kern2pt{\textcolor{darkblue}{\textbf{\ipa{qwɤ˥}}}}\kern2pt]} \hypertarget{qw7\string_M1}{}
\markboth{\textcolor{darkblue}{\textbf{\ipa{qwɤ˧}}}}{}
\textcolor{teal}{\mytextsc{nom}} \hspace{4pt} Ton~: M.
\textcolor{PineGreen}{\selectlanguage{french}Foyer, âtre, lieu où on fait du feu dans la maison.} \zh{火塘。}  ¶ \textcolor{darkblue}{\textbf{\ipa{qwɤ˧, | mv̩˧ kʰɯ˩-di˩!}}} \textcolor{PineGreen}{\selectlanguage{french}Le foyer, c'est là où on allume le feu!} \zh{火塘,就是升火的地方!}  
 \zh{量词}: \textcolor{darkblue}{\textbf{\ipa{ɭɯ˧}}}  \mytextsc{clf}~: \textcolor{darkblue}{\textbf{\ipa{ɭɯ˧}}} 
\lhead{\firstmark}
\rhead{\botmark}

\subsection{\hspace{-0.5cm} {\Large \textcolor{darkblue}{\textbf{\ipa{qwɤ˧\textsubscript{a}}}}}\hspace{0.5cm}[\kern2pt{\textcolor{darkblue}{\textbf{\ipa{qwɤ˥}}}}\kern2pt]} \hypertarget{qw7\string_Ma1}{}
\markboth{\textcolor{darkblue}{\textbf{\ipa{qwɤ˧\textsubscript{a}}}}}{}
\textcolor{teal}{\mytextsc{verbe}} \hspace{4pt} Ton~: M\textsubscript{a}.
\textcolor{PineGreen}{\selectlanguage{french}Accuser.} \zh{告状。}  ¶ \textcolor{darkblue}{\textbf{\ipa{mɤ˧-qwɤ˧}}} \textcolor{PineGreen}{\selectlanguage{french}\mytextsc{neg}} \zh{不告状}  
 ¶ \textcolor{darkblue}{\textbf{\ipa{hĩ˧ qwɤ˩}}} \textcolor{PineGreen}{\selectlanguage{french}dénoncer quelqu'un} \zh{告一个人}  
 ¶ \textcolor{darkblue}{\textbf{\ipa{njɤ˧-ɳɯ˧ | qwɤ˧-bi˧!}}} \textcolor{PineGreen}{\selectlanguage{french}je vais (te) dénoncer!} \zh{我要告状!}  
 ¶ \textcolor{darkblue}{\textbf{\ipa{no˧ | le˧-qwɤ˧-hõ˧!}}} \textcolor{PineGreen}{\selectlanguage{french}va (le) dénoncer!} \zh{你去告状吧!}  
 ¶ \textcolor{darkblue}{\textbf{\ipa{qwɤ˧\textasciitilde{}qwɤ˩}}} \textcolor{PineGreen}{\selectlanguage{french}\mytextsc{red}} \zh{\mytextsc{重叠}}  

\lhead{\firstmark}
\rhead{\botmark}

\subsection{\hspace{-0.5cm} {\Large \textcolor{darkblue}{\textbf{\ipa{qwɤ˧ɭɯ\#˥}}}}\hspace{0.5cm}[\kern2pt{\textcolor{darkblue}{\textbf{\ipa{qwɤ˧ɭɯ˧}}}}\kern2pt]} \hypertarget{qw7\string_Ml\string_RM\#\string_T1}{}
\markboth{\textcolor{darkblue}{\textbf{\ipa{qwɤ˧ɭɯ\#˥}}}}{}
\textcolor{teal}{\mytextsc{nom}} \hspace{4pt} Ton~: \#H.
\textcolor{PineGreen}{\selectlanguage{french}Feu de camp: foyer bâti à l'extérieur, provisoirement, lorsqu'on campe sur la montagne.} \zh{营火、篝火。}  ¶ \textcolor{darkblue}{\textbf{\ipa{qwɤ˧ɭɯ˧-pʰɤ˧bɤ˥}}} \textcolor{PineGreen}{\selectlanguage{french}les cadeaux offerts aux ancêtres: même lorsqu'il ne s'agit que d'un foyer provisoire, bâti pour une seule journée dans un campement en montagne, on pratique l'offrande d'un peu de nourriture} \zh{敬给祖先的礼物:即使在山上升起篝火野餐,还是要像在家里一样,用餐前先敬给祖先一些饭。}  

\lhead{\firstmark}
\rhead{\botmark}

\subsection{\hspace{-0.5cm} {\Large \textcolor{darkblue}{\textbf{\ipa{qwɤ˩\textsubscript{a}}}}}\hspace{0.5cm}[\kern2pt{\textcolor{darkblue}{\textbf{\ipa{qwɤ˩˥}}}}\kern2pt]} \hypertarget{qw7\string_Ba1}{}
\markboth{\textcolor{darkblue}{\textbf{\ipa{qwɤ˩\textsubscript{a}}}}}{}
\textcolor{teal}{\mytextsc{verbe}} \hspace{4pt} Ton~: L\textsubscript{a}.
\textcolor{PineGreen}{\selectlanguage{french}Pousser, grandir.} \zh{生长、长。}  ¶ \textcolor{darkblue}{\textbf{\ipa{gɤ˩-qwɤ˥}}} \textcolor{PineGreen}{\selectlanguage{french}grandir, pousser} \zh{长大,生长}  
 ¶ \textcolor{darkblue}{\textbf{\ipa{ʈʂʰɯ˧ | gɤ˩-qwɤ˥-ze˩!}}} \textcolor{PineGreen}{\selectlanguage{french}Il/elle a grandi! (Au sujet d'un enfant qu'on revoit après un certain temps)} \zh{他长大了!(关于一个小孩)}  

\lhead{\firstmark}
\rhead{\botmark}

\subsection{\hspace{-0.5cm} {\Large \textcolor{darkblue}{\textbf{\ipa{qwɤ˩pi˩}}}}\hspace{0.5cm}[\kern2pt{\textcolor{darkblue}{\textbf{\ipa{qwɤ˩pi˩˥}}}}\kern2pt]} \hypertarget{qw7\string_Bpi\string_B1}{}
\markboth{\textcolor{darkblue}{\textbf{\ipa{qwɤ˩pi˩}}}}{}
\textcolor{teal}{\mytextsc{nom}} \hspace{4pt} Ton~: L.
\textcolor{PineGreen}{\selectlanguage{french}Bouche.} \zh{嘴巴。}  ¶ \textcolor{darkblue}{\textbf{\ipa{qwɤ˩pi˩-qo˩lo˥}}} \textcolor{PineGreen}{\selectlanguage{french}à l'intérieur de la bouche} \zh{嘴巴里}  
 ¶ \textcolor{darkblue}{\textbf{\ipa{[F5] ko˩pi˩-ko˩lo˧}}} \textcolor{PineGreen}{\selectlanguage{french}dans la bouche, à l'intérieur de la bouche} \zh{嘴巴里}  
 \zh{量词}: \textcolor{darkblue}{\textbf{\ipa{ɭɯ˧}}}  \mytextsc{clf}~: \textcolor{darkblue}{\textbf{\ipa{ɭɯ˧}}} 
\lhead{\firstmark}
\rhead{\botmark}

\newpage
\section*{\centering- \textcolor{darkblue}{\textbf{\ipa{qʰ}}} -}
\subsection{\hspace{-0.5cm} {\Large \textcolor{darkblue}{\textbf{\ipa{qʰɑ˥}}}}\hspace{0.5cm}[\kern2pt{\textcolor{darkblue}{\textbf{\ipa{qʰɑ˥}}}}\kern2pt]} \hypertarget{q\string_hA\string_T1}{}
\markboth{\textcolor{darkblue}{\textbf{\ipa{qʰɑ˥}}}}{}
\textcolor{teal}{\mytextsc{adjectif}} \hspace{4pt} Ton~: H.
\textcolor{PineGreen}{\selectlanguage{french}Amer.} \zh{苦。} 
\lhead{\firstmark}
\rhead{\botmark}

\subsection{\hspace{-0.5cm} {\Large \textcolor{darkblue}{\textbf{\ipa{qʰɑ˧-}}}}\hspace{0.5cm}[\kern2pt{\textcolor{darkblue}{\textbf{\ipa{qʰɑ˥}}}}\kern2pt]} \hypertarget{q\string_hA\string_M-1}{}
\markboth{\textcolor{darkblue}{\textbf{\ipa{qʰɑ˧-}}}}{}
\textcolor{teal}{\mytextsc{adverbe}} \hspace{4pt} Ton~: .
\textcolor{PineGreen}{\selectlanguage{french}Particulièrement, très.} \zh{多么、非常。}  ¶ \textcolor{darkblue}{\textbf{\ipa{qʰɑ˧-ɖɯ˧-hĩ˧}}} \textcolor{PineGreen}{\selectlanguage{french}extrêmement gros} \zh{非常大}  
 ¶ \textcolor{darkblue}{\textbf{\ipa{qʰɑ˧-ɖɯ˧-gv̩˧}}} \textcolor{PineGreen}{\selectlanguage{french}particulièrement grand} \zh{非常大}  
 ¶ \textcolor{darkblue}{\textbf{\ipa{qʰɑ˧-ʂwæ˧-gv̩˧}}} \textcolor{PineGreen}{\selectlanguage{french}particulièrement grand, de très grande taille} \zh{很高、非常高}  
 ¶ \textcolor{darkblue}{\textbf{\ipa{qʰɑ˧-ʂwæ˧-mi˧zo˥}}} \textcolor{PineGreen}{\selectlanguage{french}très grand, de très haute taille} \zh{很高}  
 ¶ \textcolor{darkblue}{\textbf{\ipa{qʰɑ˧-ɖɯ˧-mi˧zo˥}}} \textcolor{PineGreen}{\selectlanguage{french}très gros, de très grande envergure} \zh{很大}  
\textit{Voir~:} \hyperlink{}{\textcolor{darkblue}{\textbf{\ipa{qʰɑ˧}}} \textsubscript{1}} 
\lhead{\firstmark}
\rhead{\botmark}

\subsection{\hspace{-0.5cm} {\Large \textcolor{darkblue}{\textbf{\ipa{qʰɑ˧}}} \textsubscript{1}}\hspace{0.5cm}[\kern2pt{\textcolor{darkblue}{\textbf{\ipa{qʰɑ˥}}}}\kern2pt]} \hypertarget{q\string_hA\string_M1}{}
\markboth{\textcolor{darkblue}{\textbf{\ipa{qʰɑ˧}}} \textsubscript{1}}{}
\textcolor{teal}{\mytextsc{pronom}} \hspace{4pt} Ton~: M.
\textcolor{PineGreen}{\selectlanguage{french}Combien.} \zh{几、多少。}  ¶ \textcolor{darkblue}{\textbf{\ipa{hĩ˧ | qʰɑ˧-kv̩˧˥?}}} \textcolor{PineGreen}{\selectlanguage{french}combien de gens?} \zh{几个人?}  
 ¶ \textcolor{darkblue}{\textbf{\ipa{bæ˩bæ˩˥ | qʰɑ˧-bæ˩?}}} \textcolor{PineGreen}{\selectlanguage{french}combien de fleurs?} \zh{几朵花?}  
 ¶ \textcolor{darkblue}{\textbf{\ipa{qʰɑ˧-ʑi˩?}}} \textcolor{PineGreen}{\selectlanguage{french}combien de familles?} \zh{几家?}  
 ¶ \textcolor{darkblue}{\textbf{\ipa{hɑ˧ | qʰɑ˧-tɕʰi˩?}}} \textcolor{PineGreen}{\selectlanguage{french}combien de repas?} \zh{几顿饭?}  
 ¶ \textcolor{darkblue}{\textbf{\ipa{qʰɑ˧-ɲi˧?}}} \textcolor{PineGreen}{\selectlanguage{french}combien de jours?} \zh{几天?}  
 ¶ \textcolor{darkblue}{\textbf{\ipa{qʰɑ˧-kʰv̩˧ gv̩˧-ze˩?}}} \textcolor{PineGreen}{\selectlanguage{french}quel âge avez-(vous)?} \zh{几岁了?}  
 ¶ \textcolor{darkblue}{\textbf{\ipa{qʰɑ˧-kʰv̩˧˥?}}} \textcolor{PineGreen}{\selectlanguage{french}combien d'années?} \zh{几年?}  
 ¶ \textcolor{darkblue}{\textbf{\ipa{qʰɑ˧-kʰwɤ˧˥?}}} \textcolor{PineGreen}{\selectlanguage{french}combien de morceaux?} \zh{几块?}  
 ¶ \textcolor{darkblue}{\textbf{\ipa{qʰɑ˧-nɑ˧?}}} \textcolor{PineGreen}{\selectlanguage{french}combien (d'outils...)?} \zh{几把?}  
 ¶ \textcolor{darkblue}{\textbf{\ipa{sɯ˩tʰi˩˥ | qʰɑ˧-nɑ˧ dʑo˧?}}} \textcolor{PineGreen}{\selectlanguage{french}Combien y a-t-il de couteaux?} \zh{有几把刀?}  
 ¶ \textcolor{darkblue}{\textbf{\ipa{qʰɑ˧-kʰɯ˩}}} \textcolor{PineGreen}{\selectlanguage{french}combien (d'objets longs)} \zh{几条}  
 ¶ \textcolor{darkblue}{\textbf{\ipa{qʰɑ˧-kʰɯ˩ dʑo˩?}}} \textcolor{PineGreen}{\selectlanguage{french}Combien y a-t-il (d'objets longs)?} \zh{有几条?}  
 ¶ \textcolor{darkblue}{\textbf{\ipa{qʰɑ˧-mæ˩ dʑo˩?}}} \textcolor{PineGreen}{\selectlanguage{french}Combien (tu) as d'argent?} \zh{有几块(钱)?}  
 ¶ \textcolor{darkblue}{\textbf{\ipa{si˧dzi˩ | qʰɑ˧-dzi˩?}}} \textcolor{PineGreen}{\selectlanguage{french}combien d'arbres?} \zh{几棵树?}  
 ¶ \textcolor{darkblue}{\textbf{\ipa{si˧kɤ˧˥ | qʰɑ˧-kɤ˧˥?}}} \textcolor{PineGreen}{\selectlanguage{french}combien de branches?} \zh{几枝树枝?}  
 ¶ \textcolor{darkblue}{\textbf{\ipa{qʰɑ˧-kʰɤ˧˥?}}} \textcolor{PineGreen}{\selectlanguage{french}combien de cageots?} \zh{几筐?}  
\textit{Voir~:} \hyperlink{}{\textcolor{darkblue}{\textbf{\ipa{qʰɑ˧}}} \textsubscript{2}} 
\lhead{\firstmark}
\rhead{\botmark}

\subsection{\hspace{-0.5cm} {\Large \textcolor{darkblue}{\textbf{\ipa{qʰɑ˧}}} \textsubscript{2}}\hspace{0.5cm}[\kern2pt{\textcolor{darkblue}{\textbf{\ipa{qʰɑ˥}}}}\kern2pt]} \hypertarget{q\string_hA\string_M2}{}
\markboth{\textcolor{darkblue}{\textbf{\ipa{qʰɑ˧}}} \textsubscript{2}}{}
\textcolor{teal}{\mytextsc{adverbe}} \hspace{4pt} Ton~: M.
\textcolor{PineGreen}{\selectlanguage{french}Quelques, plusieurs.} \zh{几(如:十几个)。}  ¶ \textcolor{darkblue}{\textbf{\ipa{tsʰe˩-qʰɑ˩˥}}} \textcolor{PineGreen}{\selectlanguage{french}dix et plus (entre dix et vingt)} \zh{十几个、十来个}  
 ¶ \textcolor{darkblue}{\textbf{\ipa{tsʰe˩-qʰɑ˩-kv̩˩˥}}} \textcolor{PineGreen}{\selectlanguage{french}dix et plus (entre dix et vingt)} \zh{十几个、十来个}  
\textit{Voir~:} \hyperlink{}{\textcolor{darkblue}{\textbf{\ipa{qʰɑ˧}}} \textsubscript{1}} 
\lhead{\firstmark}
\rhead{\botmark}

\subsection{\hspace{-0.5cm} {\Large \textcolor{darkblue}{\textbf{\ipa{qʰɑ˧dze˧}}}}\hspace{0.5cm}[\kern2pt{\textcolor{darkblue}{\textbf{\ipa{qʰɑ˧dze˧}}}}\kern2pt]} \hypertarget{q\string_hA\string_Mdze\string_M1}{}
\markboth{\textcolor{darkblue}{\textbf{\ipa{qʰɑ˧dze˧}}}}{}
\textcolor{teal}{\mytextsc{nom}} \hspace{4pt} Ton~: M.
\textcolor{PineGreen}{\selectlanguage{french}Maïs.} \zh{玉米、包谷。}  ¶ \textcolor{darkblue}{\textbf{\ipa{qʰɑ˧dze˧-kʰɯ˩ʈɯ˩}}} \textcolor{PineGreen}{\selectlanguage{french}racines des plants de maïs} \zh{玉米的根}  
 ¶ \textcolor{darkblue}{\textbf{\ipa{qʰɑ˧dze˧ qʰæ˩}}} \textcolor{PineGreen}{\selectlanguage{french}cueillir le maïs: arracher les épis de maïs} \zh{采玉米:折断玉米棒子}  
 ¶ \textcolor{darkblue}{\textbf{\ipa{qʰɑ˧dze˧ ɖʐɤ˧˥}}} \textcolor{PineGreen}{\selectlanguage{french}cueillir le maïs} \zh{采玉米}  
 ¶ \textcolor{darkblue}{\textbf{\ipa{qʰɑ˧dze˧-tsɑ˩bɤ˩ | ɖɯ˧-mɤ˩}}} \textcolor{PineGreen}{\selectlanguage{french}un peu de farine de maïs} \zh{一点玉米粉}  
 ¶ \textcolor{darkblue}{\textbf{\ipa{qʰɑ˧dze˧-hɑ˧bɤ˥, | qʰɑ˧dze˧-hɑ˧ɭɯ\#˥, | qʰɑ˧dze˧-tsɑ˩bɤ˩}}} \textcolor{PineGreen}{\selectlanguage{french}le maïs sous trois formes: épis de maïs; maïs en grains; farine de maïs} \zh{玉米的三种形态:玉米棒子,玉米粒,玉米粉}  
 \zh{量词}: \textcolor{darkblue}{\textbf{\ipa{kɤ˧˥}}}  \mytextsc{clf}~: \textcolor{darkblue}{\textbf{\ipa{kɤ˧˥}}} 
\lhead{\firstmark}
\rhead{\botmark}

\subsection{\hspace{-0.5cm} {\Large \textcolor{darkblue}{\textbf{\ipa{qʰɑ˧dze˧-hwæ˩-di˩}}}}\hspace{0.5cm}[\kern2pt{\textcolor{darkblue}{\textbf{\ipa{xxxx non-correspondance entre le nombre de morphèmes et le nombre de tons de morphèmes}}}}\kern2pt]} \hypertarget{q\string_hA\string_Mdze\string_M-hw\{\string_B-di\string_B1}{}
\markboth{\textcolor{darkblue}{\textbf{\ipa{qʰɑ˧dze˧-hwæ˩-di˩}}}}{}
\textcolor{teal}{\mytextsc{nom}} \hspace{4pt} Ton~: \mytextsc{L}.
\textcolor{PineGreen}{\selectlanguage{french}Poutrelle d'espalier à sécher le maïs: partie horizontale de la structure en bois. Périphrase: 'endroit où on accroche le maïs'.} \zh{粮架的横梁。}  \zh{量词}: \textcolor{darkblue}{\textbf{\ipa{kɤ˧˥}}}  \mytextsc{clf}~: \textcolor{darkblue}{\textbf{\ipa{kɤ˧˥}}} 
\lhead{\firstmark}
\rhead{\botmark}

\subsection{\hspace{-0.5cm} {\Large \textcolor{darkblue}{\textbf{\ipa{qʰɑ˧dze˧-lv̩˧}}}}\hspace{0.5cm}[\kern2pt{\textcolor{darkblue}{\textbf{\ipa{xxxx non-correspondance entre le nombre de morphèmes et le nombre de tons de morphèmes}}}}\kern2pt]} \hypertarget{q\string_hA\string_Mdze\string_M-lv\string_=\string_M1}{}
\markboth{\textcolor{darkblue}{\textbf{\ipa{qʰɑ˧dze˧-lv̩˧}}}}{}
\textcolor{teal}{\mytextsc{nom}} \hspace{4pt} Ton~: M.
\textcolor{PineGreen}{\selectlanguage{french}Champ de maïs.} \zh{包谷田、玉米田。}  \zh{量词}: \textcolor{darkblue}{\textbf{\ipa{pʰv̩˩}}}  \mytextsc{clf}~: \textcolor{darkblue}{\textbf{\ipa{pʰv̩˩}}} 
\lhead{\firstmark}
\rhead{\botmark}

\subsection{\hspace{-0.5cm} {\Large \textcolor{darkblue}{\textbf{\ipa{qʰɑ˧tɑ˧}}}}\hspace{0.5cm}[\kern2pt{\textcolor{darkblue}{\textbf{\ipa{qʰɑ˧tɑ˧}}}}\kern2pt]} \hypertarget{q\string_hA\string_MtA\string_M1}{}
\markboth{\textcolor{darkblue}{\textbf{\ipa{qʰɑ˧tɑ˧}}}}{}
\textcolor{teal}{\mytextsc{pronom}} \hspace{4pt} Ton~: M.
\textcolor{PineGreen}{\selectlanguage{french}Quand.} \zh{什么时候。}  ¶ \textcolor{darkblue}{\textbf{\ipa{qʰɑ˧tɑ˧ bi˧?}}} \textcolor{PineGreen}{\selectlanguage{french}quand (y) vas(-tu)?} \zh{你什么时候去?}  

\lhead{\firstmark}
\rhead{\botmark}

\subsection{\hspace{-0.5cm} {\Large \textcolor{darkblue}{\textbf{\ipa{qʰɑ˩jɤ˩}}}}\hspace{0.5cm}[\kern2pt{\textcolor{darkblue}{\textbf{\ipa{qʰɑ˩jɤ˩˥}}}}\kern2pt]} \hypertarget{q\string_hA\string_Bj7\string_B1}{}
\markboth{\textcolor{darkblue}{\textbf{\ipa{qʰɑ˩jɤ˩}}}}{}
\textcolor{teal}{\mytextsc{pronom}} \hspace{4pt} Ton~: L.
\textcolor{PineGreen}{\selectlanguage{french}Combien.} \zh{多少。}  ¶ \textcolor{darkblue}{\textbf{\ipa{qʰɑ˩jɤ˩ tʰi˥-ki˩?}}} \textcolor{PineGreen}{\selectlanguage{french}combien donner? / c'est combien?} \zh{要给多少? = 多少钱?}  
 ¶ \textcolor{darkblue}{\textbf{\ipa{qʰɑ˩jɤ˩ ɲi˧?}}} \textcolor{PineGreen}{\selectlanguage{french}combien (t'en) faut-il?} \zh{要多少?}  
 ¶ \textcolor{darkblue}{\textbf{\ipa{ɖʐe˧ | qʰɑ˩jɤ˩ ɲi˧?}}} \textcolor{PineGreen}{\selectlanguage{french}Combien d'argent ça coûte?} \zh{要多少钱?}  

\lhead{\firstmark}
\rhead{\botmark}

\subsection{\hspace{-0.5cm} {\Large \textcolor{darkblue}{\textbf{\ipa{qʰɑ˩ne˩}}}}\hspace{0.5cm}[\kern2pt{\textcolor{darkblue}{\textbf{\ipa{qʰɑ˩ne˩˥}}}}\kern2pt]} \hypertarget{q\string_hA\string_Bne\string_B1}{}
\markboth{\textcolor{darkblue}{\textbf{\ipa{qʰɑ˩ne˩}}}}{}
\textcolor{teal}{\mytextsc{pronom}} \hspace{4pt} Ton~: L.
\textcolor{PineGreen}{\selectlanguage{french}Pronom interrogatif: comment?} \zh{怎么样。}  ¶ \textcolor{darkblue}{\textbf{\ipa{qʰɑ˩ne˩ ʝi˥?}}} \textcolor{PineGreen}{\selectlanguage{french}comment faire?} \zh{怎么做?}  
 ¶ \textcolor{darkblue}{\textbf{\ipa{qʰɑ˩ne˩ ʝi˥-tso˩-ɲi˩?}}} \textcolor{PineGreen}{\selectlanguage{french}comment faut-il faire?} \zh{要怎么做?}  
 ¶ \textcolor{darkblue}{\textbf{\ipa{qʰɑ˩ne˩ gv̩˩˥?}}} \textcolor{PineGreen}{\selectlanguage{french}comment faire?} \zh{怎么做?}  
 ¶ \textcolor{darkblue}{\textbf{\ipa{qʰɑ˩ne˩ gv̩˩-ho˥-ze˩?}}} \textcolor{PineGreen}{\selectlanguage{french}que s'est-il passé?} \zh{怎么样了?发展到什么程度?}  

\lhead{\firstmark}
\rhead{\botmark}

\subsection{\hspace{-0.5cm} {\Large \textcolor{darkblue}{\textbf{\ipa{qʰæ˥}}}}\hspace{0.5cm}[\kern2pt{\textcolor{darkblue}{\textbf{\ipa{qʰæ˥}}}}\kern2pt]} \hypertarget{q\string_h\{\string_T1}{}
\markboth{\textcolor{darkblue}{\textbf{\ipa{qʰæ˥}}}}{}
\textcolor{teal}{\mytextsc{nom}} \hspace{4pt} Ton~: \#H.
\ding{202} \textcolor{PineGreen}{\selectlanguage{french}Excréments, fèces.} \zh{屎、垃圾、 肥料。}  ¶ \textcolor{darkblue}{\textbf{\ipa{qʰæ˧ ɖɯ˧-pɤ˧ ʂe˧˥}}} \textcolor{PineGreen}{\selectlanguage{french}faire une crotte} \zh{拉一泡屎}  
 ¶ \textcolor{darkblue}{\textbf{\ipa{qʰæ˧ kv̩˥}}} \textcolor{PineGreen}{\selectlanguage{french}ramasser du crottin} \zh{捡(马……)屎}  
 ¶ \textcolor{darkblue}{\textbf{\ipa{qʰæ˧-pi˩ kv̩˩}}} \textcolor{PineGreen}{\selectlanguage{french}ramasser un peu de crottin} \zh{捡一点(马……)屎}  
 ¶ \textcolor{darkblue}{\textbf{\ipa{qʰæ˧ ɖɯ˧-pi˧ kv̩˥}}} \textcolor{PineGreen}{\selectlanguage{french}même sens: ramasser un peu de crottin} \zh{同上:捡一点(马……)屎}  
 \zh{量词}: \textcolor{darkblue}{\textbf{\ipa{pɤ˥}}} \ding{203} \textcolor{PineGreen}{\selectlanguage{french}Pet.} \zh{屁。}  ¶ \textcolor{darkblue}{\textbf{\ipa{qʰæ˧ kʰɯ˩}}} \textcolor{PineGreen}{\selectlanguage{french}péter} \zh{放屁}  
 ¶ \textcolor{darkblue}{\textbf{\ipa{qʰæ˧ | ɖɯ˧-pɤ˥ kʰɯ˩}}} \textcolor{PineGreen}{\selectlanguage{french}faire un pet} \zh{放一个屁}  
 \zh{量词}: \textcolor{darkblue}{\textbf{\ipa{pɤ˥}}} \ding{204} \textcolor{PineGreen}{\selectlanguage{french}Ordure, détritus.} \zh{垃圾。}  \mytextsc{clf}~: \textcolor{darkblue}{\textbf{\ipa{pɤ˥}}} \textcolor{darkblue}{\textbf{\ipa{pɤ˥}}} 
\lhead{\firstmark}
\rhead{\botmark}

\subsection{\hspace{-0.5cm} {\Large \textcolor{darkblue}{\textbf{\ipa{qʰæ˥}}} \textsubscript{1}}\hspace{0.5cm}[\kern2pt{\textcolor{darkblue}{\textbf{\ipa{qʰæ˥}}}}\kern2pt]} \hypertarget{q\string_h\{\string_T1}{}
\markboth{\textcolor{darkblue}{\textbf{\ipa{qʰæ˥}}} \textsubscript{1}}{}
\textcolor{teal}{\mytextsc{verbe}} \hspace{4pt} Ton~: H.
\textcolor{PineGreen}{\selectlanguage{french}Ronger.} \zh{啃(啃骨头)。} 
\lhead{\firstmark}
\rhead{\botmark}

\subsection{\hspace{-0.5cm} {\Large \textcolor{darkblue}{\textbf{\ipa{qʰæ˥}}} \textsubscript{2}}\hspace{0.5cm}[\kern2pt{\textcolor{darkblue}{\textbf{\ipa{qʰæ˥}}}}\kern2pt]} \hypertarget{q\string_h\{\string_T2}{}
\markboth{\textcolor{darkblue}{\textbf{\ipa{qʰæ˥}}} \textsubscript{2}}{}
\textcolor{teal}{\mytextsc{adjectif}} \hspace{4pt} Ton~: H.
\textcolor{PineGreen}{\selectlanguage{french}Froide (eau).} \zh{冷(水)。}  ¶ \textcolor{darkblue}{\textbf{\ipa{dʑɯ˩qʰæ˩}}} \textcolor{PineGreen}{\selectlanguage{french}eau froide} \zh{凉水}  
 ¶ \textcolor{darkblue}{\textbf{\ipa{qʰæ˧-ɕjæ˧-gv̩˧}}} \textcolor{PineGreen}{\selectlanguage{french}très froid} \zh{冷得很}  

\lhead{\firstmark}
\rhead{\botmark}

\subsection{\hspace{-0.5cm} {\Large \textcolor{darkblue}{\textbf{\ipa{qʰæ˥}}} \textsubscript{3}}\hspace{0.5cm}[\kern2pt{\textcolor{darkblue}{\textbf{\ipa{qʰæ˥}}}}\kern2pt]} \hypertarget{q\string_h\{\string_T3}{}
\markboth{\textcolor{darkblue}{\textbf{\ipa{qʰæ˥}}} \textsubscript{3}}{}
\textcolor{teal}{\mytextsc{nom}} \hspace{4pt} Ton~: \#H.
\textcolor{PineGreen}{\selectlanguage{french}Canal, rigole.} \zh{水沟(单音节)。}  \zh{量词}: \textcolor{darkblue}{\textbf{\ipa{kʰɯ˩}}}  \mytextsc{clf}~: \textcolor{darkblue}{\textbf{\ipa{kʰɯ˩}}} \textcolor{PineGreen}{\selectlanguage{french}objets longs}
\lhead{\firstmark}
\rhead{\botmark}

\subsection{\hspace{-0.5cm} {\Large \textcolor{darkblue}{\textbf{\ipa{qʰæ˧\textsubscript{b}}}}}\hspace{0.5cm}[\kern2pt{\textcolor{darkblue}{\textbf{\ipa{qʰæ˩˥}}}}\kern2pt]} \hypertarget{q\string_h\{\string_Mb1}{}
\markboth{\textcolor{darkblue}{\textbf{\ipa{qʰæ˧\textsubscript{b}}}}}{}
\textcolor{teal}{\mytextsc{verbe}} \hspace{4pt} Ton~: M\textsubscript{b}.
\textcolor{PineGreen}{\selectlanguage{french}(se) briser, (se) casser (ex.: un bâton).} \zh{断,破(棍子,竹竿)。}  ¶ \textcolor{darkblue}{\textbf{\ipa{le˧-qʰæ˧-ze˧}}} \textcolor{PineGreen}{\selectlanguage{french}\mytextsc{accomp} \string_ \mytextsc{pfv}} \zh{断了}  
 ¶ \textcolor{darkblue}{\textbf{\ipa{si˧ qʰæ˩}}} \textcolor{PineGreen}{\selectlanguage{french}briser du bois} \zh{砸木头}  

\lhead{\firstmark}
\rhead{\botmark}

\subsection{\hspace{-0.5cm} {\Large \textcolor{darkblue}{\textbf{\ipa{qʰæ˧kʰwɤ\#˥}}}}\hspace{0.5cm}[\kern2pt{\textcolor{darkblue}{\textbf{\ipa{qʰæ˧kʰwɤ˧}}}}\kern2pt]} \hypertarget{q\string_h\{\string_Mk\string_hw7\#\string_T1}{}
\markboth{\textcolor{darkblue}{\textbf{\ipa{qʰæ˧kʰwɤ\#˥}}}}{}
\textcolor{teal}{\mytextsc{nom}} \hspace{4pt} Ton~: \#H.
\textcolor{PineGreen}{\selectlanguage{french}Petit barrage pour bloquer un canal d'irrigation, fait de pierres et de terre. Pour irriguer, on l'ouvre à coups de houe.} \zh{小水坝,来堵塞田地里的小水渠。}  ¶ \textcolor{darkblue}{\textbf{\ipa{qʰæ˧kʰwɤ˧ ɖɯ˧-ɭɯ˧}}} \textcolor{PineGreen}{\selectlanguage{french}un petit canal} \zh{一个小水坝}  
 \zh{量词}: \textcolor{darkblue}{\textbf{\ipa{ɭɯ˧}}}  \mytextsc{clf}~: \textcolor{darkblue}{\textbf{\ipa{ɭɯ˧}}} 
\lhead{\firstmark}
\rhead{\botmark}

\subsection{\hspace{-0.5cm} {\Large \textcolor{darkblue}{\textbf{\ipa{qʰæ˧lo˧˥}}}}\hspace{0.5cm}[\kern2pt{\textcolor{darkblue}{\textbf{\ipa{qʰæ˧lo˧˥}}}}\kern2pt]} \hypertarget{q\string_h\{\string_Mlo\string_M\string_T1}{}
\markboth{\textcolor{darkblue}{\textbf{\ipa{qʰæ˧lo˧˥}}}}{}
\textcolor{teal}{\mytextsc{nom}} \hspace{4pt} Ton~: MH\#.
\textcolor{PineGreen}{\selectlanguage{french}Petite rigole, petit canal.} \zh{小水渠。}  \zh{量词}: \textcolor{darkblue}{\textbf{\ipa{kʰɯ˩}}}  \mytextsc{clf}~: \textcolor{darkblue}{\textbf{\ipa{kʰɯ˩}}} \textit{Voir~:} \hyperlink{}{\textcolor{darkblue}{\textbf{\ipa{qʰæ˧zo\#˥}}}} 
\lhead{\firstmark}
\rhead{\botmark}

\subsection{\hspace{-0.5cm} {\Large \textcolor{darkblue}{\textbf{\ipa{qʰæ˧mi˧}}}}\hspace{0.5cm}[\kern2pt{\textcolor{darkblue}{\textbf{\ipa{qʰæ˧mi˧}}}}\kern2pt]} \hypertarget{q\string_h\{\string_Mmi\string_M1}{}
\markboth{\textcolor{darkblue}{\textbf{\ipa{qʰæ˧mi˧}}}}{}
\textcolor{teal}{\mytextsc{nom}} \hspace{4pt} Ton~: M.
\textcolor{PineGreen}{\selectlanguage{french}Grand canal.} \zh{大水渠。}  \zh{量词}: \textcolor{darkblue}{\textbf{\ipa{kʰɯ˩}}}  \mytextsc{clf}~: \textcolor{darkblue}{\textbf{\ipa{kʰɯ˩}}} \textcolor{PineGreen}{\selectlanguage{french}objets longs}
\lhead{\firstmark}
\rhead{\botmark}

\subsection{\hspace{-0.5cm} {\Large \textcolor{darkblue}{\textbf{\ipa{qʰæ˧mo˩}}}}\hspace{0.5cm}[\kern2pt{\textcolor{darkblue}{\textbf{\ipa{qʰæ˧mo˩}}}}\kern2pt]} \hypertarget{q\string_h\{\string_Mmo\string_B1}{}
\markboth{\textcolor{darkblue}{\textbf{\ipa{qʰæ˧mo˩}}}}{}
\textcolor{teal}{\mytextsc{nom}} \hspace{4pt} Ton~: L\#.
\textcolor{PineGreen}{\selectlanguage{french}Un champignon vénéneux.} \zh{有毒的一种菌子。} 
\lhead{\firstmark}
\rhead{\botmark}

\subsection{\hspace{-0.5cm} {\Large \textcolor{darkblue}{\textbf{\ipa{qʰæ˧tv̩˧}}}}\hspace{0.5cm}[\kern2pt{\textcolor{darkblue}{\textbf{\ipa{qʰæ˧tv̩˧}}}}\kern2pt]} \hypertarget{q\string_h\{\string_Mtv\string_=\string_M1}{}
\markboth{\textcolor{darkblue}{\textbf{\ipa{qʰæ˧tv̩˧}}}}{}
\textcolor{teal}{\mytextsc{nom}} \hspace{4pt} Ton~: M.
\textcolor{PineGreen}{\selectlanguage{french}Anus.} \zh{肛门。}  \zh{量词}: \textcolor{darkblue}{\textbf{\ipa{ɭɯ˧}}}  \mytextsc{clf}~: \textcolor{darkblue}{\textbf{\ipa{ɭɯ˧}}} 
\lhead{\firstmark}
\rhead{\botmark}

\subsection{\hspace{-0.5cm} {\Large \textcolor{darkblue}{\textbf{\ipa{qʰæ˧tɕʰi˧}}}}\hspace{0.5cm}[\kern2pt{\textcolor{darkblue}{\textbf{\ipa{qʰæ˧tɕʰi˧}}}}\kern2pt]} \hypertarget{q\string_h\{\string_Mts£\string_hi\string_M1}{}
\markboth{\textcolor{darkblue}{\textbf{\ipa{qʰæ˧tɕʰi˧}}}}{}
\textcolor{teal}{\mytextsc{nom}} \hspace{4pt} Ton~: M.
\textcolor{PineGreen}{\selectlanguage{french}Un village de Yongning; nom chinois: Kaiji.} \zh{开基(永宁的一个村落)。}  ¶ \textcolor{darkblue}{\textbf{\ipa{ʈʂʰɯ˧ | qʰæ˧tɕʰi˧-hĩ˧ ɲi˥!}}} \textcolor{PineGreen}{\selectlanguage{french}c'est quelqu'un de Kaiji!} \zh{他是开基村人!}  
 ¶ \textcolor{darkblue}{\textbf{\ipa{dʑɤ˩bv̩˧kɤ˧-sɑ˥ʁwɤ˩, | hi˩ʁwɤ˩-lo˥, | æ˩mi˧-ʁwɤ\#˥, | lɑ˧lo˧-ʁwɤ˥, | lɑ˧ŋwɤ˧, | bɤ˧tsʰo˧gv̩˥, | ə˧lɑ˧-ʁwɤ\#˥, | gæ˧ɻæ˩, | qʰæ˧tɕʰi˧, | tʰo˧ʈɯ\#˥}}} \textcolor{PineGreen}{\selectlanguage{french}les dix villages comptant traditionnellement comme faisant partie de Yongning} \zh{摩梭传统地理概念中,属于永宁的十个村落}  

\lhead{\firstmark}
\rhead{\botmark}

\subsection{\hspace{-0.5cm} {\Large \textcolor{darkblue}{\textbf{\ipa{qʰæ˧ʈæ˧˥}}}}\hspace{0.5cm}[\kern2pt{\textcolor{darkblue}{\textbf{\ipa{qʰæ˧ʈæ˧˥}}}}\kern2pt]} \hypertarget{q\string_h\{\string_Mt`\{\string_M\string_T1}{}
\markboth{\textcolor{darkblue}{\textbf{\ipa{qʰæ˧ʈæ˧˥}}}}{}
\textcolor{teal}{\mytextsc{adjectif}} \hspace{4pt} Ton~: MH\#.
\textit{De:} \textbf{qʰæ˩a 1} \textcolor{PineGreen}{\selectlanguage{french}Paisible, tranquille (personnalité, trait de caractère).} \zh{安静。}  ¶ \textcolor{darkblue}{\textbf{\ipa{qʰæ˧ʈæ˧˥ | tʰi˧-dzi˩}}} \textcolor{PineGreen}{\selectlanguage{french}être assis tranquillement, être tranquille, avoir l'esprit libre} \zh{安静地坐着}  
 ¶ \textcolor{darkblue}{\textbf{\ipa{qʰæ˧ʈæ˧˥ | tʰi˧-ʝi˧}}} \textcolor{PineGreen}{\selectlanguage{french}travailler paisiblement} \zh{安静地工作}  

\lhead{\firstmark}
\rhead{\botmark}

\subsection{\hspace{-0.5cm} {\Large \textcolor{darkblue}{\textbf{\ipa{qʰæ˧zo\#˥}}}}\hspace{0.5cm}[\kern2pt{\textcolor{darkblue}{\textbf{\ipa{qʰæ˧zo˧}}}}\kern2pt]} \hypertarget{q\string_h\{\string_Mzo\#\string_T1}{}
\markboth{\textcolor{darkblue}{\textbf{\ipa{qʰæ˧zo\#˥}}}}{}
\textcolor{teal}{\mytextsc{nom}} \hspace{4pt} Ton~: \#H.
\textcolor{PineGreen}{\selectlanguage{french}Petit canal.} \zh{小水渠。}  \zh{量词}: \textcolor{darkblue}{\textbf{\ipa{kʰɯ˩}}}  \mytextsc{clf}~: \textcolor{darkblue}{\textbf{\ipa{kʰɯ˩}}} \textcolor{PineGreen}{\selectlanguage{french}objets longs}\textit{Voir~:} \hyperlink{}{\textcolor{darkblue}{\textbf{\ipa{qʰæ˧lo˧˥}}}} 
\lhead{\firstmark}
\rhead{\botmark}

\subsection{\hspace{-0.5cm} {\Large \textcolor{darkblue}{\textbf{\ipa{qʰæ˩}}}}\hspace{0.5cm}[\kern2pt{\textcolor{darkblue}{\textbf{\ipa{qʰæ˩˥}}}}\kern2pt]} \hypertarget{q\string_h\{\string_B1}{}
\markboth{\textcolor{darkblue}{\textbf{\ipa{qʰæ˩}}}}{}
\textcolor{teal}{\mytextsc{verbe}} \hspace{4pt} Ton~: L\textsubscript{a}.
\textcolor{PineGreen}{\selectlanguage{french}Casser; ex.: casser une branche, récolter un épi de maïs en le cassant du plant de maïs.} \zh{折断。}  ¶ \textcolor{darkblue}{\textbf{\ipa{qʰɑ˧dze˧ qʰæ˩}}} \textcolor{PineGreen}{\selectlanguage{french}cueillir du maïs} \zh{采玉米}  
 ¶ \textcolor{darkblue}{\textbf{\ipa{qʰɑ˧dze˧ | le˧-qʰæ˩-ze˩}}} \textcolor{PineGreen}{\selectlanguage{french}le maïs est cueilli} \zh{玉米收好了。}  
 ¶ \textcolor{darkblue}{\textbf{\ipa{qʰɑ˧dze˧ | ɖɯ˧-qʰæ˧\textasciitilde{}qʰæ˥-ɻ̍˩}}} \textcolor{PineGreen}{\selectlanguage{french}cueillir un peu de maïs} \zh{去采些玉米}  

\lhead{\firstmark}
\rhead{\botmark}

\subsection{\hspace{-0.5cm} {\Large \textcolor{darkblue}{\textbf{\ipa{qʰæ˩\textsubscript{a}}}} \textsubscript{1}}\hspace{0.5cm}[\kern2pt{\textcolor{darkblue}{\textbf{\ipa{qʰæ˩˥}}}}\kern2pt]} \hypertarget{q\string_h\{\string_Ba1}{}
\markboth{\textcolor{darkblue}{\textbf{\ipa{qʰæ˩\textsubscript{a}}}} \textsubscript{1}}{}
\textcolor{teal}{\mytextsc{adjectif}} \hspace{4pt} Ton~: L\textsubscript{a}.
\textcolor{PineGreen}{\selectlanguage{french}Faux, mensonger.} \zh{假。}  ¶ \textcolor{darkblue}{\textbf{\ipa{qʰæ˩-hĩ˩˥, | tʰɑ˧-ʐwɤ˩!}}} \textcolor{PineGreen}{\selectlanguage{french}ne dis pas de mensonges!} \zh{假话,不要说! =不要撒谎!}  
 ¶ \textcolor{darkblue}{\textbf{\ipa{qʰæ˧ ʐwɤ˧}}} \textcolor{PineGreen}{\selectlanguage{french}dire des mensonges} \zh{撒谎、说谎}  

\lhead{\firstmark}
\rhead{\botmark}

\subsection{\hspace{-0.5cm} {\Large \textcolor{darkblue}{\textbf{\ipa{qʰæ˩\textsubscript{a}}}} \textsubscript{2}}\hspace{0.5cm}[\kern2pt{\textcolor{darkblue}{\textbf{\ipa{qʰæ˩˥}}}}\kern2pt]} \hypertarget{q\string_h\{\string_Ba2}{}
\markboth{\textcolor{darkblue}{\textbf{\ipa{qʰæ˩\textsubscript{a}}}} \textsubscript{2}}{}
\textcolor{teal}{\mytextsc{adjectif}} \hspace{4pt} Ton~: L\textsubscript{a}.
\ding{202} \textcolor{PineGreen}{\selectlanguage{french}En paix, tranquille, paisible (époque); en bonne santé (corps).} \zh{平静、安静,安乐、(身体)健康。}  ¶ \textcolor{darkblue}{\textbf{\ipa{hĩ˧ | ə˩-qʰæ˩˥?}}} \textcolor{PineGreen}{\selectlanguage{french}est-ce que ça va bien?/tu vas bien? (Formule équivalente du \textcolor{darkblue}{\textbf{\ipa{/ə˥-lɑ˧~lɑ˩/}}} de Lijiang 'Est-ce que vous êtes en bonne santé?', qui à Yongning évoque malencontreusement le rédupliqué \textcolor{darkblue}{\textbf{\ipa{/ə˩-lɑ˩~lɑ˧˥/}}} 'est-ce que [vous] vous disputez?')} \zh{你好吗? / 一切好吗?}  
 ¶ \textcolor{darkblue}{\textbf{\ipa{njɤ˧ | mɤ˧-qʰæ˩.}}} \textcolor{PineGreen}{\selectlanguage{french}je ne me sens pas bien.} \zh{我不舒服。}  
\ding{203} \textcolor{PineGreen}{\selectlanguage{french}Léger, peu fatigant (travail).} \zh{轻松。}  ¶ \textcolor{darkblue}{\textbf{\ipa{qʰæ˩-hĩ˩˥}}} \textcolor{PineGreen}{\selectlanguage{french}\mytextsc{rel}} \zh{轻松的}  

\lhead{\firstmark}
\rhead{\botmark}

\subsection{\hspace{-0.5cm} {\Large \textcolor{darkblue}{\textbf{\ipa{qʰæ˩bæ˩}}}}\hspace{0.5cm}[\kern2pt{\textcolor{darkblue}{\textbf{\ipa{qʰæ˧bæ˧}}}}\kern2pt]} \hypertarget{q\string_h\{\string_Bb\{\string_B1}{}
\markboth{\textcolor{darkblue}{\textbf{\ipa{qʰæ˩bæ˩}}}}{}
\textcolor{teal}{\mytextsc{nom}} \hspace{4pt} Ton~: L.
\textcolor{PineGreen}{\selectlanguage{french}Cuillère de petite taille: pour le sel, le tsamba... Elle correspond aux cuillères à café et cuillères à soupe du paradigme européen.} \zh{调羹。}  \zh{量词}: \textcolor{darkblue}{\textbf{\ipa{nɑ˧}}}  \mytextsc{clf}~: \textcolor{darkblue}{\textbf{\ipa{nɑ˧}}} 
\lhead{\firstmark}
\rhead{\botmark}

\subsection{\hspace{-0.5cm} {\Large \textcolor{darkblue}{\textbf{\ipa{qʰæ˩ʈv̩˩ɻæ˥}}}}\hspace{0.5cm}[\kern2pt{\textcolor{darkblue}{\textbf{\ipa{qʰæ˩ʈv̩˩ɻæ˥}}}}\kern2pt]} \hypertarget{q\string_h\{\string_Bt`v\string_=\string_Br£`\{\string_T1}{}
\markboth{\textcolor{darkblue}{\textbf{\ipa{qʰæ˩ʈv̩˩ɻæ˥}}}}{}
\textcolor{teal}{\mytextsc{adjectif}} \hspace{4pt} Ton~: L+H\#.
\textcolor{PineGreen}{\selectlanguage{french}Serein.} \zh{安宁。}  ¶ \textcolor{darkblue}{\textbf{\ipa{qʰæ˩ʈv̩˩ɻæ˥ | ɖɯ˧-dzi˩-ɻ̍˩}}} \textcolor{PineGreen}{\selectlanguage{french}être assis tranquille, dans le calme} \zh{安静地坐一会}  
 ¶ \textcolor{darkblue}{\textbf{\ipa{qʰæ˩ʈv̩˩ɻæ˥-gv̩˩}}} \textcolor{PineGreen}{\selectlanguage{french}tranquillement} \zh{安宁地}  

\lhead{\firstmark}
\rhead{\botmark}

\subsection{\hspace{-0.5cm} {\Large \textcolor{darkblue}{\textbf{\ipa{qʰæ˧˥}}} \textsubscript{1}}\hspace{0.5cm}[\kern2pt{\textcolor{darkblue}{\textbf{\ipa{qʰæ˧˥}}}}\kern2pt]} \hypertarget{q\string_h\{\string_M\string_T1}{}
\markboth{\textcolor{darkblue}{\textbf{\ipa{qʰæ˧˥}}} \textsubscript{1}}{}
\textcolor{teal}{\mytextsc{verbe}} \hspace{4pt} Ton~: MH.
\textcolor{PineGreen}{\selectlanguage{french}Paraître, se lever (lune, soleil).} \zh{出来(月亮,太阳)。}  ¶ \textcolor{darkblue}{\textbf{\ipa{tʰi˧-qʰæ˧-ze˥}}} \textcolor{PineGreen}{\selectlanguage{french}\mytextsc{dur} \string_ \mytextsc{pfv}} \zh{\mytextsc{dur} \string_ \mytextsc{pfv}}  

\lhead{\firstmark}
\rhead{\botmark}

\subsection{\hspace{-0.5cm} {\Large \textcolor{darkblue}{\textbf{\ipa{qʰæ˧˥}}} \textsubscript{2}}\hspace{0.5cm}[\kern2pt{\textcolor{darkblue}{\textbf{\ipa{qʰæ˧˥}}}}\kern2pt]} \hypertarget{q\string_h\{\string_M\string_T2}{}
\markboth{\textcolor{darkblue}{\textbf{\ipa{qʰæ˧˥}}} \textsubscript{2}}{}
\textcolor{teal}{\mytextsc{verbe}} \hspace{4pt} Ton~: MH.
\textcolor{PineGreen}{\selectlanguage{french}Démolir.} \zh{拆。}  ¶ \textcolor{darkblue}{\textbf{\ipa{ʑi˧qʰwɤ˧ qʰæ˧˥}}} \textcolor{PineGreen}{\selectlanguage{french}démolir une maison} \zh{拆房子}  

\lhead{\firstmark}
\rhead{\botmark}

\subsection{\hspace{-0.5cm} {\Large \textcolor{darkblue}{\textbf{\ipa{qʰæ˧˥}}} \textsubscript{3}}\hspace{0.5cm}[\kern2pt{\textcolor{darkblue}{\textbf{\ipa{qʰæ˧˥}}}}\kern2pt]} \hypertarget{q\string_h\{\string_M\string_T3}{}
\markboth{\textcolor{darkblue}{\textbf{\ipa{qʰæ˧˥}}} \textsubscript{3}}{}
\textcolor{teal}{\mytextsc{verbe}} \hspace{4pt} Ton~: MH.
\textcolor{PineGreen}{\selectlanguage{french}Partager, répartir.} \zh{分东西、(大家)平分东西。} 
\lhead{\firstmark}
\rhead{\botmark}

\subsection{\hspace{-0.5cm} {\Large \textcolor{darkblue}{\textbf{\ipa{qʰæ˧˥}}} \textsubscript{4}}\hspace{0.5cm}[\kern2pt{\textcolor{darkblue}{\textbf{\ipa{qʰæ˧˥}}}}\kern2pt]} \hypertarget{q\string_h\{\string_M\string_T4}{}
\markboth{\textcolor{darkblue}{\textbf{\ipa{qʰæ˧˥}}} \textsubscript{4}}{}
\textcolor{teal}{\mytextsc{verbe}} \hspace{4pt} Ton~: MH.
\textcolor{PineGreen}{\selectlanguage{french}Tirer (avec une arme à feu, une arbalète; aussi avec un arc: tirer une flèche).} \zh{开枪。}  ¶ \textcolor{darkblue}{\textbf{\ipa{le˧-qʰæ˧-ze˥}}} \textcolor{PineGreen}{\selectlanguage{french}\mytextsc{accomp} \string_ \mytextsc{pfv}} \zh{开枪了}  
 ¶ \textcolor{darkblue}{\textbf{\ipa{mv̩˧ʐe˧ qʰæ˩(-ze˩)}}} \textcolor{PineGreen}{\selectlanguage{french}tirer avec une arme à feu} \zh{开枪}  

\lhead{\firstmark}
\rhead{\botmark}

\subsection{\hspace{-0.5cm} {\Large \textcolor{darkblue}{\textbf{\ipa{qʰæ˧˥}}} \textsubscript{5}}\hspace{0.5cm}[\kern2pt{\textcolor{darkblue}{\textbf{\ipa{qʰæ˧˥}}}}\kern2pt]} \hypertarget{q\string_h\{\string_M\string_T5}{}
\markboth{\textcolor{darkblue}{\textbf{\ipa{qʰæ˧˥}}} \textsubscript{5}}{}
\textcolor{teal}{\mytextsc{adjectif}} \hspace{4pt} Ton~: MH.
\textcolor{PineGreen}{\selectlanguage{french}Heureux.} \zh{幸福,安逸,平安。}  ¶ \textcolor{darkblue}{\textbf{\ipa{le˧-qʰæ˧-ze˥}}} \textcolor{PineGreen}{\selectlanguage{french}\mytextsc{accomp} \string_ \mytextsc{pfv}} \zh{\mytextsc{accomp} \string_ \mytextsc{pfv}}  
 ¶ \textcolor{darkblue}{\textbf{\ipa{lo˧ qʰæ˩}}} \textcolor{PineGreen}{\selectlanguage{french}travailler de façon tranquille, détendue, paisible} \zh{轻松工作}  

\lhead{\firstmark}
\rhead{\botmark}

\subsection{\hspace{-0.5cm} {\Large \textcolor{darkblue}{\textbf{\ipa{qʰæ˧˥}}} \textsubscript{6}}\hspace{0.5cm}[\kern2pt{\textcolor{darkblue}{\textbf{\ipa{qʰæ˧˥}}}}\kern2pt]} \hypertarget{q\string_h\{\string_M\string_T6}{}
\markboth{\textcolor{darkblue}{\textbf{\ipa{qʰæ˧˥}}} \textsubscript{6}}{}
\textcolor{teal}{\mytextsc{verbe}} \hspace{4pt} Ton~: MH.
\textcolor{PineGreen}{\selectlanguage{french}Brûler, griller, noircir: la graisse chauffée à feu vif noircit, fume et donne de l'acroléine; le riz devient sec, très/trop cuit. La nourriture reste comestible.} \zh{糊、变黑(高温让油、食物变黑,变糊了)。}  ¶ \textcolor{darkblue}{\textbf{\ipa{le˧-qʰæ˧-ze˥}}} \textcolor{PineGreen}{\selectlanguage{french}\mytextsc{accomp} \string_ \mytextsc{pfv}} \zh{\mytextsc{accomp} \string_ \mytextsc{pfv}}  
 ¶ \textcolor{darkblue}{\textbf{\ipa{mɤ˧ | le˧-qʰæ˧-ze˥}}} \textcolor{PineGreen}{\selectlanguage{french}L'huile a noirci / l'huile est parvenue à une très haute température.} \zh{油焦了!}  
 ¶ \textcolor{darkblue}{\textbf{\ipa{hɑ˧ | le˧-qʰæ˧-ze˥}}} \textcolor{PineGreen}{\selectlanguage{french}Le riz a brûlé. / Le riz est trop cuit.} \zh{饭糊了。}  
 ¶ \textcolor{darkblue}{\textbf{\ipa{v̩˩tsʰɤ˩˥ | hṽ˧\textasciitilde{}hṽ˧ F | le˧-qʰæ˧-ze˥!}}} \textcolor{PineGreen}{\selectlanguage{french}Les légumes, à force de frire, les voilà brûlés! / les voilà trop cuits!} \zh{菜都炒糊了!}  
 ¶ \textcolor{darkblue}{\textbf{\ipa{ʂe˧ | hṽ˧\textasciitilde{}hṽ˧ F | le˧-qʰæ˧-ze˥!}}} \textcolor{PineGreen}{\selectlanguage{french}La viande, à force de frire, la voilà brûlée! / la voilà trop cuite!} \zh{肉都炒焦了!}  

\lhead{\firstmark}
\rhead{\botmark}

\subsection{\hspace{-0.5cm} {\Large \textcolor{darkblue}{\textbf{\ipa{qʰo˧\textsubscript{a}}}}}\hspace{0.5cm}[\kern2pt{\textcolor{darkblue}{\textbf{\ipa{qʰo˥}}}}\kern2pt]} \hypertarget{q\string_ho\string_Ma1}{}
\markboth{\textcolor{darkblue}{\textbf{\ipa{qʰo˧\textsubscript{a}}}}}{}
\textcolor{teal}{\mytextsc{verbe}} \hspace{4pt} Ton~: M\textsubscript{a}.
\textcolor{PineGreen}{\selectlanguage{french}Empiler (par exemple des pierres).} \zh{堆起来。}  ¶ \textcolor{darkblue}{\textbf{\ipa{lv̩˧mi˧ tʰi˧-qʰo˧}}} \textcolor{PineGreen}{\selectlanguage{french}empiler des pierres} \zh{石头堆起来}  

\lhead{\firstmark}
\rhead{\botmark}

\subsection{\hspace{-0.5cm} {\Large \textcolor{darkblue}{\textbf{\ipa{qʰo˧lo˧}}}}\hspace{0.5cm}[\kern2pt{\textcolor{darkblue}{\textbf{\ipa{qʰo˧lo˧}}}}\kern2pt]} \hypertarget{q\string_ho\string_Mlo\string_M1}{}
\markboth{\textcolor{darkblue}{\textbf{\ipa{qʰo˧lo˧}}}}{}
\textcolor{teal}{\mytextsc{nom}} \hspace{4pt} Ton~: M.
\textcolor{PineGreen}{\selectlanguage{french}Roue.} \zh{轮子。}  \zh{量词}: \textcolor{darkblue}{\textbf{\ipa{ɭɯ˧}}}  \mytextsc{clf}~: \textcolor{darkblue}{\textbf{\ipa{ɭɯ˧}}} 
\lhead{\firstmark}
\rhead{\botmark}

\subsection{\hspace{-0.5cm} {\Large \textcolor{darkblue}{\textbf{\ipa{qʰo˧mo˥}}}}\hspace{0.5cm}[\kern2pt{\textcolor{darkblue}{\textbf{\ipa{qʰo˧mo˥}}}}\kern2pt]} \hypertarget{q\string_ho\string_Mmo\string_T1}{}
\markboth{\textcolor{darkblue}{\textbf{\ipa{qʰo˧mo˥}}}}{}
\textcolor{teal}{\mytextsc{nom}} \hspace{4pt} Ton~: H\#.
\textcolor{PineGreen}{\selectlanguage{french}Vieille vache (qui n'a plus de lait).} \zh{老牛(不产奶了)。}  \zh{量词}: \textcolor{darkblue}{\textbf{\ipa{pʰo˧˥}}}  \mytextsc{clf}~: \textcolor{darkblue}{\textbf{\ipa{pʰo˧˥}}} 
\lhead{\firstmark}
\rhead{\botmark}

\subsection{\hspace{-0.5cm} {\Large \textcolor{darkblue}{\textbf{\ipa{qʰo˩\textsubscript{b}}}}}\hspace{0.5cm}[\kern2pt{\textcolor{darkblue}{\textbf{\ipa{qʰo˩˥}}}}\kern2pt]} \hypertarget{q\string_ho\string_Bb1}{}
\markboth{\textcolor{darkblue}{\textbf{\ipa{qʰo˩\textsubscript{b}}}}}{}
\textcolor{teal}{\mytextsc{verbe}} \hspace{4pt} Ton~: L\textsubscript{b}.
\textcolor{PineGreen}{\selectlanguage{french}Inviter.} \zh{邀请、请。}  ¶ \textcolor{darkblue}{\textbf{\ipa{hĩ˧ qʰo˧˥}}} \textcolor{PineGreen}{\selectlanguage{french}inviter quelqu'un} \zh{邀请人}  
 ¶ \textcolor{darkblue}{\textbf{\ipa{hĩ˧bæ˧ qʰo˧˥}}} \textcolor{PineGreen}{\selectlanguage{french}inviter un hôte, convier un invité} \zh{邀请客人}  
 ¶ \textcolor{darkblue}{\textbf{\ipa{hĩ˧bæ˧ | qʰo˧-zo˧-ho˥}}} \textcolor{PineGreen}{\selectlanguage{french}Il va falloir inviter des hôtes!} \zh{需要请一下客人!}  
 ¶ \textcolor{darkblue}{\textbf{\ipa{hĩ˧bæ˧ qʰo˧-di˧˥}}} \textcolor{PineGreen}{\selectlanguage{french}Euphémisme pour désigner la mort-aux-rats. La croyance veut que si on expose clairement le projet, les rats vont se méfier et ne prendront pas cette nourriture empoisonnée.} \zh{‘待客的东西’(老鼠药的委婉语。如果说出来要买老鼠药,老鼠会知道,就不会吃的。)}  
 ¶ \textcolor{darkblue}{\textbf{\ipa{ɖɯ˧-qʰo˥\textasciitilde{}qʰo˩-ɻ̍˩}}} \textcolor{PineGreen}{\selectlanguage{french}\mytextsc{délimitative} \mytextsc{red} \mytextsc{inchoatif}} \zh{\mytextsc{delimitative} \mytextsc{red} \mytextsc{inceptive:请一下}}  
 ¶ \textcolor{darkblue}{\textbf{\ipa{qʰo˩-mɤ˥-qʰo˩}}} \textcolor{PineGreen}{\selectlanguage{french}inviter ou ne pas inviter} \zh{请不请}  
 ¶ \textcolor{darkblue}{\textbf{\ipa{qʰo˩-mɤ˩-ho˥}}} \textcolor{PineGreen}{\selectlanguage{french}... ne vais pas inviter} \zh{不请了 / 不要请了}  

\lhead{\firstmark}
\rhead{\botmark}

\subsection{\hspace{-0.5cm} {\Large \textcolor{darkblue}{\textbf{\ipa{qʰo˩dv̩˧˥}}}}\hspace{0.5cm}[\kern2pt{\textcolor{darkblue}{\textbf{\ipa{qʰo˩dv̩˧˥}}}}\kern2pt]} \hypertarget{q\string_ho\string_Bdv\string_=\string_M\string_T1}{}
\markboth{\textcolor{darkblue}{\textbf{\ipa{qʰo˩dv̩˧˥}}}}{}
\textcolor{teal}{\mytextsc{nom}} \hspace{4pt} Ton~: LM+MH\#.
\textcolor{PineGreen}{\selectlanguage{french}Masse, marteau de grande taille; typiquement: masse en bois utilisée pour défoncer les grosses mottes après les labours.} \zh{大锤子。}  ¶ \textcolor{darkblue}{\textbf{\ipa{ʂe˩-qʰo˩dv̩˧˥}}} \textcolor{PineGreen}{\selectlanguage{french}masse en fer} \zh{铁锤子}  
 \zh{量词}: \textcolor{darkblue}{\textbf{\ipa{ɭɯ˧}}}  \mytextsc{clf}~: \textcolor{darkblue}{\textbf{\ipa{ɭɯ˧}}} 
\lhead{\firstmark}
\rhead{\botmark}

\subsection{\hspace{-0.5cm} {\Large \textcolor{darkblue}{\textbf{\ipa{qʰo˩mv̩˩}}}}\hspace{0.5cm}[\kern2pt{\textcolor{darkblue}{\textbf{\ipa{qʰo˩mv̩˩˥}}}}\kern2pt]} \hypertarget{q\string_ho\string_Bmv\string_=\string_B1}{}
\markboth{\textcolor{darkblue}{\textbf{\ipa{qʰo˩mv̩˩}}}}{}
\textcolor{teal}{\mytextsc{nom}} \hspace{4pt} Ton~: L.
\textcolor{PineGreen}{\selectlanguage{french}Chapeau de paille.} \zh{斗笠。}  \zh{量词}: \textcolor{darkblue}{\textbf{\ipa{ɭɯ˧}}}  \mytextsc{clf}~: \textcolor{darkblue}{\textbf{\ipa{ɭɯ˧}}} 
\lhead{\firstmark}
\rhead{\botmark}

\subsection{\hspace{-0.5cm} {\Large \textcolor{darkblue}{\textbf{\ipa{qʰo˩tv̩˧˥}}}}\hspace{0.5cm}[\kern2pt{\textcolor{darkblue}{\textbf{\ipa{qʰo˩tv̩˧˥}}}}\kern2pt]} \hypertarget{q\string_ho\string_Btv\string_=\string_M\string_T1}{}
\markboth{\textcolor{darkblue}{\textbf{\ipa{qʰo˩tv̩˧˥}}}}{}
\textcolor{teal}{\mytextsc{nom}} \hspace{4pt} Ton~: LM+MH\#.
\textcolor{PineGreen}{\selectlanguage{french}Souche.} \zh{树墩、树桩。}  \zh{量词}: \textcolor{darkblue}{\textbf{\ipa{ɭɯ˧}}}  \mytextsc{clf}~: \textcolor{darkblue}{\textbf{\ipa{ɭɯ˧}}} 
\lhead{\firstmark}
\rhead{\botmark}

\subsection{\hspace{-0.5cm} {\Large \textcolor{darkblue}{\textbf{\ipa{qʰo˧˥}}} \textsubscript{1}}\hspace{0.5cm}[\kern2pt{\textcolor{darkblue}{\textbf{\ipa{qʰo˧˥}}}}\kern2pt]} \hypertarget{q\string_ho\string_M\string_T1}{}
\markboth{\textcolor{darkblue}{\textbf{\ipa{qʰo˧˥}}} \textsubscript{1}}{}
\textcolor{teal}{\mytextsc{verbe}} \hspace{4pt} Ton~: MH.
\textcolor{PineGreen}{\selectlanguage{french}Picorer.} \zh{啄。}  ¶ \textcolor{darkblue}{\textbf{\ipa{hɑ˧ qʰo˩(-ze˩)}}} \textcolor{PineGreen}{\selectlanguage{french}picorer des céréales} \zh{啄粮食}  
 ¶ \textcolor{darkblue}{\textbf{\ipa{hɑ˧ qʰo˥\textasciitilde{}qʰo˩ (-dʑo˩)}}} \textcolor{PineGreen}{\selectlanguage{french}picorer des céréales} \zh{啄粮食}  
 ¶ \textcolor{darkblue}{\textbf{\ipa{æ˩-ɳɯ˥ | hɑ˧ qʰo˩}}} \textcolor{PineGreen}{\selectlanguage{french}la poule picore} \zh{鸡在啄粮食}  

\lhead{\firstmark}
\rhead{\botmark}

\subsection{\hspace{-0.5cm} {\Large \textcolor{darkblue}{\textbf{\ipa{qʰo˧˥}}} \textsubscript{2}}\hspace{0.5cm}[\kern2pt{\textcolor{darkblue}{\textbf{\ipa{qʰo˧˥}}}}\kern2pt]} \hypertarget{q\string_ho\string_M\string_T2}{}
\markboth{\textcolor{darkblue}{\textbf{\ipa{qʰo˧˥}}} \textsubscript{2}}{}
\textcolor{teal}{\mytextsc{verbe}} \hspace{4pt} Ton~: MH.
\textcolor{PineGreen}{\selectlanguage{french}Tuer; abattre un animal.} \zh{杀,宰牲畜。}  ¶ \textcolor{darkblue}{\textbf{\ipa{bo˩ qʰo˧˥ / bo˩ qʰo˧-ze˥}}} \textcolor{PineGreen}{\selectlanguage{french}tuer le cochon} \zh{杀猪}  
 ¶ \textcolor{darkblue}{\textbf{\ipa{bo˩˥ | le˧-qʰo˧-ze˥}}} \textcolor{PineGreen}{\selectlanguage{french}le cochon a été abattu} \zh{杀了猪}  
 ¶ \textcolor{darkblue}{\textbf{\ipa{æ˩ qʰo˧˥}}} \textcolor{PineGreen}{\selectlanguage{french}tuer un poulet} \zh{杀鸡}  
 ¶ \textcolor{darkblue}{\textbf{\ipa{ʝi˧ qʰo˩}}} \textcolor{PineGreen}{\selectlanguage{french}tuer une vache} \zh{杀牛}  

\lhead{\firstmark}
\rhead{\botmark}

\subsection{\hspace{-0.5cm} {\Large \textcolor{darkblue}{\textbf{\ipa{qʰv̩˧}}} \textsubscript{1}}\hspace{0.5cm}[\kern2pt{\textcolor{darkblue}{\textbf{\ipa{qʰv̩˥}}}}\kern2pt]} \hypertarget{q\string_hv\string_=\string_M1}{}
\markboth{\textcolor{darkblue}{\textbf{\ipa{qʰv̩˧}}} \textsubscript{1}}{}
\textcolor{teal}{\mytextsc{nom}} \hspace{4pt} Ton~: M.
\ding{202} \textcolor{PineGreen}{\selectlanguage{french}Trou.} \zh{洞。}  \zh{量词}: \textcolor{darkblue}{\textbf{\ipa{ɭɯ˧}}} \ding{203} \textcolor{PineGreen}{\selectlanguage{french}Terrier.} \zh{野兽的洞穴、野兽的窝。}  ¶ \textcolor{darkblue}{\textbf{\ipa{ɖɤ˧-qʰv̩˧}}} \textcolor{PineGreen}{\selectlanguage{french}terrier de renard} \zh{狐狸的窝} \zh{tone: M}  
 ¶ \textcolor{darkblue}{\textbf{\ipa{ʂwæ˧ qʰv̩˧}}} \textcolor{PineGreen}{\selectlanguage{french}terrier de loutre} \zh{水獭的窝} \zh{tone: M}  
 \mytextsc{clf}~: \textcolor{darkblue}{\textbf{\ipa{ɭɯ˧}}} 
\lhead{\firstmark}
\rhead{\botmark}

\subsection{\hspace{-0.5cm} {\Large \textcolor{darkblue}{\textbf{\ipa{qʰv̩˧}}} \textsubscript{2}}\hspace{0.5cm}[\kern2pt{\textcolor{darkblue}{\textbf{\ipa{qʰv̩˥}}}}\kern2pt]} \hypertarget{q\string_hv\string_=\string_M2}{}
\markboth{\textcolor{darkblue}{\textbf{\ipa{qʰv̩˧}}} \textsubscript{2}}{}
\textcolor{teal}{\mytextsc{nom}} \hspace{4pt} Ton~: M.
\textcolor{PineGreen}{\selectlanguage{french}Corne.} \zh{犄角(锯下来的)。}  ¶ \textcolor{darkblue}{\textbf{\ipa{ʝi˧-qʰv̩\#˥}}} \textcolor{PineGreen}{\selectlanguage{french}Corne de boeuf. La corne de boeuf était autrefois utilisée comme récipient pour boissons.} \zh{牛角(过去,用牛角来当饮料容器)}  
 ¶ \textcolor{darkblue}{\textbf{\ipa{ʈʂʰæ˧-qʰv̩˥}}} \textcolor{PineGreen}{\selectlanguage{french}corne de cerf} \zh{鹿角}  
 \zh{量词}: \textcolor{darkblue}{\textbf{\ipa{ɭɯ˧}}} \textcolor{darkblue}{\textbf{\ipa{dze˩}}}  \mytextsc{clf}~: \textcolor{darkblue}{\textbf{\ipa{ɭɯ˧}}} \textcolor{darkblue}{\textbf{\ipa{dze˩}}} \textcolor{PineGreen}{\selectlanguage{french}paire}
\lhead{\firstmark}
\rhead{\botmark}

\subsection{\hspace{-0.5cm} {\Large \textcolor{darkblue}{\textbf{\ipa{qʰv̩˧˥}}} \textsubscript{1}}\hspace{0.5cm}[\kern2pt{\textcolor{darkblue}{\textbf{\ipa{qʰv̩˧˥}}}}\kern2pt]} \hypertarget{q\string_hv\string_=\string_M\string_T1}{}
\markboth{\textcolor{darkblue}{\textbf{\ipa{qʰv̩˧˥}}} \textsubscript{1}}{}
\textcolor{teal}{\mytextsc{verbe}} \hspace{4pt} Ton~: MH.
\textcolor{PineGreen}{\selectlanguage{french}Se recroqueviller.} \zh{蜷缩。}  ¶ \textcolor{darkblue}{\textbf{\ipa{ɲi˧-qʰv̩˧˥ | tʰi˧-dzi˩}}} \textcolor{PineGreen}{\selectlanguage{french}être assis penché en avant, le torse penché vers les cuisses} \zh{坐着身体缩成一团}  
 ¶ \textcolor{darkblue}{\textbf{\ipa{[M23] ɲi˧-qʰv̩˧-ʝi˥ | tʰi˧-dzi˩}}} \textcolor{PineGreen}{\selectlanguage{french}être assis penché en avant, le torse penché vers les cuisses} \zh{坐着身体缩成一团}  

\lhead{\firstmark}
\rhead{\botmark}

\subsection{\hspace{-0.5cm} {\Large \textcolor{darkblue}{\textbf{\ipa{qʰv̩˧˥}}} \textsubscript{2}}\hspace{0.5cm}[\kern2pt{\textcolor{darkblue}{\textbf{\ipa{qʰv̩˧˥}}}}\kern2pt]} \hypertarget{q\string_hv\string_=\string_M\string_T2}{}
\markboth{\textcolor{darkblue}{\textbf{\ipa{qʰv̩˧˥}}} \textsubscript{2}}{}
\textcolor{teal}{\mytextsc{nombre}} \hspace{4pt} Ton~: MH.
\textcolor{PineGreen}{\selectlanguage{french}6.} \zh{6。} 
\lhead{\firstmark}
\rhead{\botmark}

\subsection{\hspace{-0.5cm} {\Large \textcolor{darkblue}{\textbf{\ipa{qʰv̩˥}}}}\hspace{0.5cm}[\kern2pt{\textcolor{darkblue}{\textbf{\ipa{qʰv̩˥}}}}\kern2pt]} \hypertarget{q\string_hv\string_=\string_T1}{}
\markboth{\textcolor{darkblue}{\textbf{\ipa{qʰv̩˥}}}}{}
\textcolor{teal}{\mytextsc{nom}} \hspace{4pt} Ton~: \#H.
\textcolor{PineGreen}{\selectlanguage{french}Son, bruit.} \zh{声音。}  ¶ \textcolor{darkblue}{\textbf{\ipa{ʈʂʰɯ˧ | ə˧tso˧ qʰv̩˧ ɲi˥?}}} \textcolor{PineGreen}{\selectlanguage{french}c'est quoi ce bruit?} \zh{这是什么声音?}  
 \zh{量词}: \textcolor{darkblue}{\textbf{\ipa{kʰwɤ˥}}}  \mytextsc{clf}~: \textcolor{darkblue}{\textbf{\ipa{kʰwɤ˥}}} 
\lhead{\firstmark}
\rhead{\botmark}

\subsection{\hspace{-0.5cm} {\Large \textcolor{darkblue}{\textbf{\ipa{qʰv̩˥\textsubscript{a}}}}}\hspace{0.5cm}[\kern2pt{\textcolor{darkblue}{\textbf{\ipa{qʰv̩˥}}}}\kern2pt]} \hypertarget{q\string_hv\string_=\string_Ta1}{}
\markboth{\textcolor{darkblue}{\textbf{\ipa{qʰv̩˥\textsubscript{a}}}}}{}
\textcolor{teal}{\mytextsc{classificateur}} \hspace{4pt} Ton~: H\textsubscript{a}.
\textcolor{PineGreen}{\selectlanguage{french}Classificateur des hameaux.} \zh{量词:村落。}  ¶ \textcolor{darkblue}{\textbf{\ipa{ŋwɤ˧-qʰv̩˧, | tsʰe˧ɲi˧-ʑi˩}}} \textcolor{PineGreen}{\selectlanguage{french}Cinq hameaux, douze familles! (Formule résumant la statistique du village de /ə˧lɑ˧-ʁwɤ\#˥/)} \zh{五个村落,十二个家庭!(描写阿拉瓦村的情况)}  

\lhead{\firstmark}
\rhead{\botmark}

\subsection{\hspace{-0.5cm} {\Large \textcolor{darkblue}{\textbf{\ipa{qʰv̩˩ɖɯ˩}}}}\hspace{0.5cm}[\kern2pt{\textcolor{darkblue}{\textbf{\ipa{qʰv̩˩ɖɯ˩˥}}}}\kern2pt]} \hypertarget{q\string_hv\string_=\string_Bd`M\string_B1}{}
\markboth{\textcolor{darkblue}{\textbf{\ipa{qʰv̩˩ɖɯ˩}}}}{}
\textcolor{teal}{\mytextsc{nom}} \hspace{4pt} Ton~: L.
\textcolor{PineGreen}{\selectlanguage{french}Attachement envers quelqu'un; observé seulement dans l'expression “être attaché à, faire cas de, attacher du prix à”.} \zh{关心。}  ¶ \textcolor{darkblue}{\textbf{\ipa{qʰv̩˩ɖɯ˩ pʰv̩˥}}} \textcolor{PineGreen}{\selectlanguage{french}attacher de l'importance à, respecter, être attaché à (ex.: relation des enfants à leurs parents)} \zh{关心(一个人),重视(如:孩子重视父母)}  

\lhead{\firstmark}
\rhead{\botmark}

\subsection{\hspace{-0.5cm} {\Large \textcolor{darkblue}{\textbf{\ipa{qʰv̩˩ɖʐæ˩}}}}\hspace{0.5cm}[\kern2pt{\textcolor{darkblue}{\textbf{\ipa{qʰv̩˩ɖʐæ˩˥}}}}\kern2pt]} \hypertarget{q\string_hv\string_=\string_Bd`z`\{\string_B1}{}
\markboth{\textcolor{darkblue}{\textbf{\ipa{qʰv̩˩ɖʐæ˩}}}}{}
\textcolor{teal}{\mytextsc{nom}} \hspace{4pt} Ton~: L.
\textcolor{PineGreen}{\selectlanguage{french}Cordelette, ficelle.} \zh{小绳子,细的绳子。}  ¶ \textcolor{darkblue}{\textbf{\ipa{qʰv̩˩ɖʐæ˩ ʈʂʰɯ˩-kʰɯ˥}}} \textcolor{PineGreen}{\selectlanguage{french}\mytextsc{n}+\mytextsc{dem}+\mytextsc{clf}} \zh{一条细的绳子}  
 \zh{量词}: \textcolor{darkblue}{\textbf{\ipa{kʰɯ˩}}}  \mytextsc{clf}~: \textcolor{darkblue}{\textbf{\ipa{kʰɯ˩}}} 
\lhead{\firstmark}
\rhead{\botmark}

\subsection{\hspace{-0.5cm} {\Large \textcolor{darkblue}{\textbf{\ipa{qʰv̩˧dʑɯ˥\$}}}}\hspace{0.5cm}[\kern2pt{\textcolor{darkblue}{\textbf{\ipa{qʰv̩˧dʑɯ˥}}}}\kern2pt]} \hypertarget{q\string_hv\string_=\string_Mdz£M\string_T\$1}{}
\markboth{\textcolor{darkblue}{\textbf{\ipa{qʰv̩˧dʑɯ˥\$}}}}{}
\textcolor{teal}{\mytextsc{nom}} \hspace{4pt} Ton~: H\$.
\textcolor{PineGreen}{\selectlanguage{french}Cavité, trou (ex.: trou de souris; ou piège où on fait tomber les animaux sauvages).} \zh{窟窿。}  ¶ \textcolor{darkblue}{\textbf{\ipa{hwæ˧tsɯ˥-qʰv̩˩dʑi˩}}} \textcolor{PineGreen}{\selectlanguage{french}trou de souris} \zh{耗子洞}  
 ¶ \textcolor{darkblue}{\textbf{\ipa{qʰv̩˧dʑɯ˧ tsʰi˧ (-ze˩)}}} \textcolor{PineGreen}{\selectlanguage{french}percer un trou} \zh{挖一个洞}  
 \zh{量词}: \textcolor{darkblue}{\textbf{\ipa{ɭɯ˧}}}  \mytextsc{clf}~: \textcolor{darkblue}{\textbf{\ipa{ɭɯ˧}}} 
\lhead{\firstmark}
\rhead{\botmark}

\subsection{\hspace{-0.5cm} {\Large \textcolor{darkblue}{\textbf{\ipa{qʰv̩˧ɬi˧mi\#˥}}}}\hspace{0.5cm}[\kern2pt{\textcolor{darkblue}{\textbf{\ipa{qʰv̩˧ɬi˧mi˧}}}}\kern2pt]} \hypertarget{q\string_hv\string_=\string_MKi\string_Mmi\#\string_T1}{}
\markboth{\textcolor{darkblue}{\textbf{\ipa{qʰv̩˧ɬi˧mi\#˥}}}}{}
\textcolor{teal}{\mytextsc{nom}} \hspace{4pt} Ton~: \#H.
\textcolor{PineGreen}{\selectlanguage{french}6e mois.} \zh{六月。} 
\lhead{\firstmark}
\rhead{\botmark}

\subsection{\hspace{-0.5cm} {\Large \textcolor{darkblue}{\textbf{\ipa{qʰv̩˩tsʰi˧˥}}}}\hspace{0.5cm}[\kern2pt{\textcolor{darkblue}{\textbf{\ipa{qʰv̩˩tsʰi˧˥}}}}\kern2pt]} \hypertarget{q\string_hv\string_=\string_Bts\string_hi\string_M\string_T1}{}
\markboth{\textcolor{darkblue}{\textbf{\ipa{qʰv̩˩tsʰi˧˥}}}}{}
\textcolor{teal}{\mytextsc{nombre}} \hspace{4pt} Ton~: LM+MH\#.
\textcolor{PineGreen}{\selectlanguage{french}60.} \zh{60。} 
\lhead{\firstmark}
\rhead{\botmark}

\subsection{\hspace{-0.5cm} {\Large \textcolor{darkblue}{\textbf{\ipa{qʰv̩˧tʰv̩\#˥}}}}\hspace{0.5cm}[\kern2pt{\textcolor{darkblue}{\textbf{\ipa{qʰv̩˧tʰv̩˧}}}}\kern2pt]} \hypertarget{q\string_hv\string_=\string_Mt\string_hv\string_=\#\string_T1}{}
\markboth{\textcolor{darkblue}{\textbf{\ipa{qʰv̩˧tʰv̩\#˥}}}}{}
\textcolor{teal}{\mytextsc{classificateur}} \hspace{4pt} Ton~: \#H.
\textcolor{PineGreen}{\selectlanguage{french}Classificateur: quantité de liquide (ou de poudre) que tient une corne de vache.} \zh{量词:一个牛角的容量。}  ¶ \textcolor{darkblue}{\textbf{\ipa{ɖɯ˧-qʰv̩˧tʰv̩\#˥, | ɲi˧-qʰv̩˧tʰv̩\#˥, | so˩-qʰv̩˩tʰv̩˩˥, | ʐv̩˧-qʰv̩˧tʰv̩\#˥, | ŋwɤ˧-qʰv̩˧tʰv̩\#˥, | qʰv̩˧-qʰv̩˧tʰv̩\#˥, | ʂɯ˧-qʰv̩˧tʰv̩\#˥, | hõ˧-qʰv̩˧tʰv̩\#˥, | gv̩˧-qʰv̩˧tʰv̩\#˥, | tsʰe˩-qʰv̩˩tʰv̩˩˥}}} \textcolor{PineGreen}{\selectlanguage{french}association avec des numéraux, de 1 à 10} \zh{与数词结合,一至十}  
\textit{Voir~:} \hyperlink{}{\textcolor{darkblue}{\textbf{\ipa{qʰv̩˧tʰv˥\$}}}} 
\lhead{\firstmark}
\rhead{\botmark}

\subsection{\hspace{-0.5cm} {\Large \textcolor{darkblue}{\textbf{\ipa{qʰv̩˧tʰv˥\$}}}}\hspace{0.5cm}[\kern2pt{\textcolor{darkblue}{\textbf{\ipa{qʰv̩˧tʰv˥}}}}\kern2pt]} \hypertarget{q\string_hv\string_=\string_Mt\string_hv\string_T\$1}{}
\markboth{\textcolor{darkblue}{\textbf{\ipa{qʰv̩˧tʰv˥\$}}}}{}
\textcolor{teal}{\mytextsc{nom}} \hspace{4pt} Ton~: H\$.
\textcolor{PineGreen}{\selectlanguage{french}Corne (de vache).} \zh{(牛)角。}  ¶ \textcolor{darkblue}{\textbf{\ipa{qʰv˧tʰv˥ | ɖɯ˧-ɭɯ˧}}} \textcolor{PineGreen}{\selectlanguage{french}une corne} \zh{一个角}  
 ¶ \textcolor{darkblue}{\textbf{\ipa{qʰv˧tʰv˧ ɲi˥}}} \textcolor{PineGreen}{\selectlanguage{french}C'est une corne.} \zh{是(牛)角。}  
 \zh{量词}: \textcolor{darkblue}{\textbf{\ipa{ɭɯ˧}}}  \mytextsc{clf}~: \textcolor{darkblue}{\textbf{\ipa{ɭɯ˧}}} \textit{Voir~:} \hyperlink{}{\textcolor{darkblue}{\textbf{\ipa{qʰv̩˧tʰv̩\#˥}}}} 
\lhead{\firstmark}
\rhead{\botmark}

\subsection{\hspace{-0.5cm} {\Large \textcolor{darkblue}{\textbf{\ipa{qʰv̩˩\textasciitilde{}qʰv̩˧˥}}}}\hspace{0.5cm}[\kern2pt{\textcolor{darkblue}{\textbf{\ipa{qʰv̩˧qʰv̩˧˥}}}}\kern2pt]} \hypertarget{q\string_hv\string_=\string_B~q\string_hv\string_=\string_M\string_T1}{}
\markboth{\textcolor{darkblue}{\textbf{\ipa{qʰv̩˩\textasciitilde{}qʰv̩˧˥}}}}{}
\textcolor{teal}{\mytextsc{verbe}} \hspace{4pt} Ton~: MH.
\textcolor{PineGreen}{\selectlanguage{french}Plier (vêtements).} \zh{折叠、裹起来。}  ¶ \textcolor{darkblue}{\textbf{\ipa{qʰv̩˩\textasciitilde{}qʰv̩˧-ze˥}}} \textcolor{PineGreen}{\selectlanguage{french}\mytextsc{pfv}} \zh{折起来了}  
 ¶ \textcolor{darkblue}{\textbf{\ipa{le˧-qʰv̩˩\textasciitilde{}qʰv̩˩}}} \textcolor{PineGreen}{\selectlanguage{french}\mytextsc{accomp}} \zh{\mytextsc{accomp}}  

\lhead{\firstmark}
\rhead{\botmark}

\subsection{\hspace{-0.5cm} {\Large \textcolor{darkblue}{\textbf{\ipa{qʰwæ˧}}}}\hspace{0.5cm}[\kern2pt{\textcolor{darkblue}{\textbf{\ipa{qʰwæ˥}}}}\kern2pt]} \hypertarget{q\string_hw\{\string_M1}{}
\markboth{\textcolor{darkblue}{\textbf{\ipa{qʰwæ˧}}}}{}
\textcolor{teal}{\mytextsc{nom}} \hspace{4pt} Ton~: M.
\textcolor{PineGreen}{\selectlanguage{french}Lettre, message, parole/récit.} \zh{信息,信。}  ¶ \textcolor{darkblue}{\textbf{\ipa{qʰwæ˧ po˧˥}}} \textcolor{PineGreen}{\selectlanguage{french}envoyer une lettre} \zh{带信息、传信息,传一封信}  
 ¶ \textcolor{darkblue}{\textbf{\ipa{qʰwæ˧ kʰwɤ˧˥}}} \textcolor{PineGreen}{\selectlanguage{french}être en contact, être en correspondance; être en relation} \zh{互相通信息、有联系(两个人互相通信息)}  
 ¶ \textcolor{darkblue}{\textbf{\ipa{dɑ˧pɤ˧-qʰwæ\#˥}}} \textcolor{PineGreen}{\selectlanguage{french}les récits des prêtres \textcolor{darkblue}{\textbf{\ipa{/dɑ˧pɤ˧/}}}} \zh{达巴的故事}  
 \zh{量词}: \textcolor{darkblue}{\textbf{\ipa{kʰwɤ˥}}}  \mytextsc{clf}~: \textcolor{darkblue}{\textbf{\ipa{kʰwɤ˥}}} 
\lhead{\firstmark}
\rhead{\botmark}

\subsection{\hspace{-0.5cm} {\Large \textcolor{darkblue}{\textbf{\ipa{qʰwæ˧kʰwɤ\#˥}}}}\hspace{0.5cm}[\kern2pt{\textcolor{darkblue}{\textbf{\ipa{qʰwæ˩kʰwɤ˩˥}}}}\kern2pt]} \hypertarget{q\string_hw\{\string_Mk\string_hw7\#\string_T1}{}
\markboth{\textcolor{darkblue}{\textbf{\ipa{qʰwæ˧kʰwɤ\#˥}}}}{}
\textcolor{teal}{\mytextsc{nom}} \hspace{4pt} Ton~: \#H.
\textcolor{PineGreen}{\selectlanguage{french}Récit, racontar, ragot, histoire.} \zh{闲话、流言、蜚语、闲言碎语、八卦。}  ¶ \textcolor{darkblue}{\textbf{\ipa{ɖɯ˧-zɯ˧ qʰwæ˧kʰwɤ˧}}} \textcolor{PineGreen}{\selectlanguage{french}raconter un petit racontar, rapporter un petit ragot} \zh{讲一点八卦}  
 \zh{量词}: \textcolor{darkblue}{\textbf{\ipa{kʰwɤ˥}}}  \mytextsc{clf}~: \textcolor{darkblue}{\textbf{\ipa{kʰwɤ˥}}} 
\lhead{\firstmark}
\rhead{\botmark}

\subsection{\hspace{-0.5cm} {\Large \textcolor{darkblue}{\textbf{\ipa{qʰwæ˧ɭɯ˧}}}}\hspace{0.5cm}[\kern2pt{\textcolor{darkblue}{\textbf{\ipa{qʰwæ˧ɭɯ˧˥}}}}\kern2pt]} \hypertarget{q\string_hw\{\string_Ml\string_RM\string_M1}{}
\markboth{\textcolor{darkblue}{\textbf{\ipa{qʰwæ˧ɭɯ˧}}}}{}
\textcolor{teal}{\mytextsc{nom}} \hspace{4pt} Ton~: M.
\textcolor{PineGreen}{\selectlanguage{french}Potager.} \zh{菜园。}  \zh{量词}: \textcolor{darkblue}{\textbf{\ipa{kɤ˧˥}}}  \mytextsc{clf}~: \textcolor{darkblue}{\textbf{\ipa{kɤ˧˥}}} 
\lhead{\firstmark}
\rhead{\botmark}

\subsection{\hspace{-0.5cm} {\Large \textcolor{darkblue}{\textbf{\ipa{qʰwæ˧mi\#˥}}}}\hspace{0.5cm}[\kern2pt{\textcolor{darkblue}{\textbf{\ipa{qʰwæ˧mi˧}}}}\kern2pt]} \hypertarget{q\string_hw\{\string_Mmi\#\string_T1}{}
\markboth{\textcolor{darkblue}{\textbf{\ipa{qʰwæ˧mi\#˥}}}}{}
\textcolor{teal}{\mytextsc{nom}} \hspace{4pt} Ton~: \#H.
\textcolor{PineGreen}{\selectlanguage{french}Message, information (d'où: lettre).} \zh{口信, 信息。}  ¶ \textcolor{darkblue}{\textbf{\ipa{qʰwæ˧mi˧ ʝi˧}}} \textcolor{PineGreen}{\selectlanguage{french}porter un message} \zh{带一个口信}  
 \zh{量词}: \textcolor{darkblue}{\textbf{\ipa{kʰwɤ˥}}}  \mytextsc{clf}~: \textcolor{darkblue}{\textbf{\ipa{kʰwɤ˥}}} 
\lhead{\firstmark}
\rhead{\botmark}

\subsection{\hspace{-0.5cm} {\Large \textcolor{darkblue}{\textbf{\ipa{qʰwæ˧ʈɯ˥}}}}\hspace{0.5cm}[\kern2pt{\textcolor{darkblue}{\textbf{\ipa{qʰwæ˩ʈɯ˧˥}}}}\kern2pt]} \hypertarget{q\string_hw\{\string_Mt`M\string_T1}{}
\markboth{\textcolor{darkblue}{\textbf{\ipa{qʰwæ˧ʈɯ˥}}}}{}
\textcolor{teal}{\mytextsc{nom}} \hspace{4pt} Ton~: H\#.
\textcolor{PineGreen}{\selectlanguage{french}Fichu (tissu qu'on porte sur la tête).} \zh{头帕。}  \zh{量词}: \textcolor{darkblue}{\textbf{\ipa{bɤ˧˥}}}  \mytextsc{clf}~: \textcolor{darkblue}{\textbf{\ipa{bɤ˧˥}}} 
\lhead{\firstmark}
\rhead{\botmark}

\subsection{\hspace{-0.5cm} {\Large \textcolor{darkblue}{\textbf{\ipa{qʰwæ˩}}}}\hspace{0.5cm}[\kern2pt{\textcolor{darkblue}{\textbf{\ipa{qʰwæ˥}}}}\kern2pt]} \hypertarget{q\string_hw\{\string_B1}{}
\markboth{\textcolor{darkblue}{\textbf{\ipa{qʰwæ˩}}}}{}
\textcolor{teal}{\mytextsc{nom}} \hspace{4pt} Ton~: L.
\textcolor{PineGreen}{\selectlanguage{french}Haie, faite de bambou ou de broussailles épineuses.} \zh{篱笆。}  \zh{量词}: \textcolor{darkblue}{\textbf{\ipa{kɤ˧˥}}}  \mytextsc{clf}~: \textcolor{darkblue}{\textbf{\ipa{kɤ˧˥}}} 
\lhead{\firstmark}
\rhead{\botmark}

\subsection{\hspace{-0.5cm} {\Large \textcolor{darkblue}{\textbf{\ipa{qʰwæ˩\textsubscript{a}}}}}\hspace{0.5cm}[\kern2pt{\textcolor{darkblue}{\textbf{\ipa{qʰwæ˧˥}}}}\kern2pt]} \hypertarget{q\string_hw\{\string_Ba1}{}
\markboth{\textcolor{darkblue}{\textbf{\ipa{qʰwæ˩\textsubscript{a}}}}}{}
\textcolor{teal}{\mytextsc{verbe}} \hspace{4pt} Ton~: L\textsubscript{a}.
\textcolor{PineGreen}{\selectlanguage{french}Bloquer.} \zh{挡住。} 
\lhead{\firstmark}
\rhead{\botmark}

\subsection{\hspace{-0.5cm} {\Large \textcolor{darkblue}{\textbf{\ipa{qʰwæ˩kɤ˩}}}}\hspace{0.5cm}[\kern2pt{\textcolor{darkblue}{\textbf{\ipa{qʰwæ˩kɤ˩˥}}}}\kern2pt]} \hypertarget{q\string_hw\{\string_Bk7\string_B1}{}
\markboth{\textcolor{darkblue}{\textbf{\ipa{qʰwæ˩kɤ˩}}}}{}
\textcolor{teal}{\mytextsc{nom}} \hspace{4pt} Ton~: L.
\textcolor{PineGreen}{\selectlanguage{french}Une sorte d'arbuste d'environ 1 mètre 50 à 2 mètres de haut.} \zh{一种灌木,1.5至2米高,可以当篱笆用。}  ¶ \textcolor{darkblue}{\textbf{\ipa{qʰwæ˩kɤ˩-dzi˩˥}}} \textcolor{PineGreen}{\selectlanguage{french}même sens} \zh{同上}  
 \zh{量词}: \textcolor{darkblue}{\textbf{\ipa{dzi˩, ʝi˧}}}  \mytextsc{clf}~: \textcolor{darkblue}{\textbf{\ipa{dzi˩, ʝi˧}}} 
\lhead{\firstmark}
\rhead{\botmark}

\subsection{\hspace{-0.5cm} {\Large \textcolor{darkblue}{\textbf{\ipa{qʰwæ˧˥}}} \textsubscript{1}}\hspace{0.5cm}[\kern2pt{\textcolor{darkblue}{\textbf{\ipa{qʰwæ˥}}}}\kern2pt]} \hypertarget{q\string_hw\{\string_M\string_T1}{}
\markboth{\textcolor{darkblue}{\textbf{\ipa{qʰwæ˧˥}}} \textsubscript{1}}{}
\textcolor{teal}{\mytextsc{verbe}} \hspace{4pt} Ton~: MH.
\textcolor{PineGreen}{\selectlanguage{french}Briser (verre, vaisselle…), faire éclater; casser (des noix).} \zh{弄碎。}  ¶ \textcolor{darkblue}{\textbf{\ipa{ʁo˧do˧ qʰwæ˧˥}}} \textcolor{PineGreen}{\selectlanguage{french}casser des noix} \zh{敲开坚果(在永宁,不用夹子:用锤子敲开)}  

\lhead{\firstmark}
\rhead{\botmark}

\subsection{\hspace{-0.5cm} {\Large \textcolor{darkblue}{\textbf{\ipa{qʰwæ˧˥}}} \textsubscript{2}}\hspace{0.5cm}[\kern2pt{\textcolor{darkblue}{\textbf{\ipa{qʰwæ˧˥}}}}\kern2pt]} \hypertarget{q\string_hw\{\string_M\string_T2}{}
\markboth{\textcolor{darkblue}{\textbf{\ipa{qʰwæ˧˥}}} \textsubscript{2}}{}
\textcolor{teal}{\mytextsc{verbe}} \hspace{4pt} Ton~: MH.
\textcolor{PineGreen}{\selectlanguage{french}Gifler.} \zh{掴、打。}  ¶ \textcolor{darkblue}{\textbf{\ipa{le˧-qʰwæ˧-ze˥}}} \textcolor{PineGreen}{\selectlanguage{french}\mytextsc{accomp} \string_ \mytextsc{pfv}} \zh{掴了}  
 ¶ \textcolor{darkblue}{\textbf{\ipa{zɯ˧ɻ̍˧ qʰwæ˩}}} \textcolor{PineGreen}{\selectlanguage{french}gifler} \zh{打嘴巴}  
 ¶ \textcolor{darkblue}{\textbf{\ipa{zɯ˧ɻ̍˧ | ɖɯ˧-ɭɯ˧ | tʰi˧-qʰwæ˧-bi˥!}}} \textcolor{PineGreen}{\selectlanguage{french}Je vais te gifler! / Je vais te flanquer une gifle! (A un enfant)} \zh{我要打嘴巴了!(对孩子说)}  

\lhead{\firstmark}
\rhead{\botmark}

\subsection{\hspace{-0.5cm} {\Large \textcolor{darkblue}{\textbf{\ipa{qʰwæ˧˥\textsubscript{a}}}}}\hspace{0.5cm}[\kern2pt{\textcolor{darkblue}{\textbf{\ipa{qʰwæ˩˥}}}}\kern2pt]} \hypertarget{q\string_hw\{\string_M\string_Ta1}{}
\markboth{\textcolor{darkblue}{\textbf{\ipa{qʰwæ˧˥\textsubscript{a}}}}}{}
\textcolor{teal}{\mytextsc{classificateur}} \hspace{4pt} Ton~: MH\textsubscript{a}.
\textcolor{PineGreen}{\selectlanguage{french}Classificateur des filaments de chanvre avant filage.} \zh{量词:丝,如纺之前的麻丝(一根)。} 
\lhead{\firstmark}
\rhead{\botmark}

\subsection{\hspace{-0.5cm} {\Large \textcolor{darkblue}{\textbf{\ipa{qʰwɤ˧}}}}\hspace{0.5cm}[\kern2pt{\textcolor{darkblue}{\textbf{\ipa{qʰwɤ˥}}}}\kern2pt]} \hypertarget{q\string_hw7\string_M1}{}
\markboth{\textcolor{darkblue}{\textbf{\ipa{qʰwɤ˧}}}}{}
\textcolor{teal}{\mytextsc{nom}} \hspace{4pt} Ton~: M.
\textcolor{PineGreen}{\selectlanguage{french}Traces, piste (d'un animal; lorsqu'on chasse, on suit la piste d'un animal, on le suit à la trace).} \zh{痕迹。}  \zh{量词}: \textcolor{darkblue}{\textbf{\ipa{pʰo˧˥}}}  \mytextsc{clf}~: \textcolor{darkblue}{\textbf{\ipa{pʰo˧˥}}} 
\lhead{\firstmark}
\rhead{\botmark}

\subsection{\hspace{-0.5cm} {\Large \textcolor{darkblue}{\textbf{\ipa{qʰwɤ˧\textsubscript{a}}}}}\hspace{0.5cm}[\kern2pt{\textcolor{darkblue}{\textbf{\ipa{qʰwɤ˥}}}}\kern2pt]} \hypertarget{q\string_hw7\string_Ma1}{}
\markboth{\textcolor{darkblue}{\textbf{\ipa{qʰwɤ˧\textsubscript{a}}}}}{}
\textcolor{teal}{\mytextsc{verbe}} \hspace{4pt} Ton~: M\textsubscript{a}.
\textcolor{PineGreen}{\selectlanguage{french}Se guérir (blessure, maladie); se rétablir (une fracture).} \zh{治好(骨折、病)。}  ¶ \textcolor{darkblue}{\textbf{\ipa{le˧-qʰwɤ˧-ɲi˥!}}} \textcolor{PineGreen}{\selectlanguage{french}C'est guéri! / La fracture est rétablie!} \zh{治好了!}  

\lhead{\firstmark}
\rhead{\botmark}

\subsection{\hspace{-0.5cm} {\Large \textcolor{darkblue}{\textbf{\ipa{qʰwɤ˧bi˩}}}}\hspace{0.5cm}[\kern2pt{\textcolor{darkblue}{\textbf{\ipa{qʰwɤ˧bi˩}}}}\kern2pt]} \hypertarget{q\string_hw7\string_Mbi\string_B1}{}
\markboth{\textcolor{darkblue}{\textbf{\ipa{qʰwɤ˧bi˩}}}}{}
\textcolor{teal}{\mytextsc{nom}} \hspace{4pt} Ton~: L\#.
\ding{202} \textcolor{PineGreen}{\selectlanguage{french}Sabot, patte.} \zh{马蹄、马的脚。}  ¶ \textcolor{darkblue}{\textbf{\ipa{ʐwæ˧-qʰwɤ˧bi˥\#}}} \textcolor{PineGreen}{\selectlanguage{french}sabot de cheval} \zh{马蹄、(马、狗……的)脚}  
 ¶ \textcolor{darkblue}{\textbf{\ipa{kʰv̩˩-qʰwɤ˩bi˥\#}}} \textcolor{PineGreen}{\selectlanguage{french}patte de chien} \zh{狗脚}  
 \zh{量词}: \textcolor{darkblue}{\textbf{\ipa{bi˩}}} \textcolor{darkblue}{\textbf{\ipa{tʰv̩˧˥}}} \ding{203} \textcolor{PineGreen}{\selectlanguage{french}Traces, piste (d'un animal).} \zh{动物脚的痕迹、行径。}  \mytextsc{clf}~: \textcolor{darkblue}{\textbf{\ipa{bi˩}}} \textcolor{darkblue}{\textbf{\ipa{tʰv̩˧˥}}} 
\lhead{\firstmark}
\rhead{\botmark}

\subsection{\hspace{-0.5cm} {\Large \textcolor{darkblue}{\textbf{\ipa{qʰwɤ˧mi˥\$}}}}\hspace{0.5cm}[\kern2pt{\textcolor{darkblue}{\textbf{\ipa{qʰwɤ˧mi˥}}}}\kern2pt]} \hypertarget{q\string_hw7\string_Mmi\string_T\$1}{}
\markboth{\textcolor{darkblue}{\textbf{\ipa{qʰwɤ˧mi˥\$}}}}{}
\textcolor{teal}{\mytextsc{nom}} \hspace{4pt} Ton~: H\$.
\textcolor{PineGreen}{\selectlanguage{french}Grand bol (autrefois, les bols étaient en bois).} \zh{大碗(以前碗是用木头做的)。} \textit{Voir~:} \hyperlink{}{\textcolor{darkblue}{\textbf{\ipa{qʰwɤ˧pɤ˥\$}}}} 
\lhead{\firstmark}
\rhead{\botmark}

\subsection{\hspace{-0.5cm} {\Large \textcolor{darkblue}{\textbf{\ipa{qʰwɤ˧pɤ˥\$}}}}\hspace{0.5cm}[\kern2pt{\textcolor{darkblue}{\textbf{\ipa{qʰwɤ˧pɤ˥}}}}\kern2pt]} \hypertarget{q\string_hw7\string_Mp7\string_T\$1}{}
\markboth{\textcolor{darkblue}{\textbf{\ipa{qʰwɤ˧pɤ˥\$}}}}{}
\textcolor{teal}{\mytextsc{nom}} \hspace{4pt} Ton~: H\$.
\textcolor{PineGreen}{\selectlanguage{french}Grand bol.} \zh{大碗。} \textit{Voir~:} \hyperlink{}{\textcolor{darkblue}{\textbf{\ipa{qʰwɤ˧mi˥\$}}}} 
\lhead{\firstmark}
\rhead{\botmark}

\subsection{\hspace{-0.5cm} {\Large \textcolor{darkblue}{\textbf{\ipa{qʰwɤ˧ʂe˩}}}}\hspace{0.5cm}[\kern2pt{\textcolor{darkblue}{\textbf{\ipa{qʰwɤ˧ʂe˩}}}}\kern2pt]} \hypertarget{q\string_hw7\string_Ms`e\string_B1}{}
\markboth{\textcolor{darkblue}{\textbf{\ipa{qʰwɤ˧ʂe˩}}}}{}
\textcolor{teal}{\mytextsc{nom}} \hspace{4pt} Ton~: L\#.
\textcolor{PineGreen}{\selectlanguage{french}Fer à cheval.} \zh{马蹄铁。}  ¶ \textcolor{darkblue}{\textbf{\ipa{ʐwæ˧-qʰwɤ˧ʂe˥ (+ɲi˩)}}} \textcolor{PineGreen}{\selectlanguage{french}fer à cheval} \zh{马蹄铁}  
 \zh{量词}: \textcolor{darkblue}{\textbf{\ipa{nɑ˧}}} \textcolor{darkblue}{\textbf{\ipa{pʰo˧˥}}}  \mytextsc{clf}~: \textcolor{darkblue}{\textbf{\ipa{nɑ˧}}} \textcolor{darkblue}{\textbf{\ipa{pʰo˧˥}}} 
\lhead{\firstmark}
\rhead{\botmark}

\subsection{\hspace{-0.5cm} {\Large \textcolor{darkblue}{\textbf{\ipa{qʰwɤ˧to˩}}}}\hspace{0.5cm}[\kern2pt{\textcolor{darkblue}{\textbf{\ipa{qʰwɤ˧to˩}}}}\kern2pt]} \hypertarget{q\string_hw7\string_Mto\string_B1}{}
\markboth{\textcolor{darkblue}{\textbf{\ipa{qʰwɤ˧to˩}}}}{}
\textcolor{teal}{\mytextsc{nom}} \hspace{4pt} Ton~: L\#.
\textcolor{PineGreen}{\selectlanguage{french}Épaule (extrémité de l'épaule, bout de l'épaule).} \zh{肩膀的末端。}  ¶ \textcolor{darkblue}{\textbf{\ipa{hĩ˧ ʈʂʰɯ˧-v̩˧, | qʰwɤ˧to˩ | ɖɯ˧-pi˧˥ | ʂwæ˧-hṽ˩-di˩!}}} \textcolor{PineGreen}{\selectlanguage{french}Ce type, il a les épaules un peu de travers! / Il a une épaule plus haute que l'autre!} \zh{这个人的肩膀不正,一高一低!}  
 \zh{量词}: \textcolor{darkblue}{\textbf{\ipa{ɭɯ˧}}}  \mytextsc{clf}~: \textcolor{darkblue}{\textbf{\ipa{ɭɯ˧}}} 
\lhead{\firstmark}
\rhead{\botmark}

\subsection{\hspace{-0.5cm} {\Large \textcolor{darkblue}{\textbf{\ipa{qʰwɤ˧tʰv̩\#˥}}}}\hspace{0.5cm}[\kern2pt{\textcolor{darkblue}{\textbf{\ipa{qʰwɤ˧tʰv̩˧}}}}\kern2pt]} \hypertarget{q\string_hw7\string_Mt\string_hv\string_=\#\string_T1}{}
\markboth{\textcolor{darkblue}{\textbf{\ipa{qʰwɤ˧tʰv̩\#˥}}}}{}
\textcolor{teal}{\mytextsc{nom}} \hspace{4pt} Ton~: \#H.
\textcolor{PineGreen}{\selectlanguage{french}Hotte en bambou pour porter de l'eau.} \zh{竹篓。}  \zh{量词}: \textcolor{darkblue}{\textbf{\ipa{ɭɯ˧}}}  \mytextsc{clf}~: \textcolor{darkblue}{\textbf{\ipa{ɭɯ˧}}} 
\lhead{\firstmark}
\rhead{\botmark}

\subsection{\hspace{-0.5cm} {\Large \textcolor{darkblue}{\textbf{\ipa{qʰwɤ˧tsʰi˩}}}}\hspace{0.5cm}[\kern2pt{\textcolor{darkblue}{\textbf{\ipa{qʰwɤ˧tsʰi˩}}}}\kern2pt]} \hypertarget{q\string_hw7\string_Mts\string_hi\string_B1}{}
\markboth{\textcolor{darkblue}{\textbf{\ipa{qʰwɤ˧tsʰi˩}}}}{}
\textcolor{teal}{\mytextsc{nom}} \hspace{4pt} Ton~: L\#.
\textcolor{PineGreen}{\selectlanguage{french}Épaule.} \zh{肩膀。}  ¶ \textcolor{darkblue}{\textbf{\ipa{qʰwɤ˧tsʰi˩-ʁo˩ | hwæ˧pʰæ˩ | ɖɯ˧-nɑ˧-ʈʂʰɯ˧ gɤ˧˥}}} \textcolor{PineGreen}{\selectlanguage{french}porter une houe à l'épaule} \zh{肩上扛一把锄头}  
 \zh{量词}: \textcolor{darkblue}{\textbf{\ipa{ɭɯ˧}}}  \mytextsc{clf}~: \textcolor{darkblue}{\textbf{\ipa{ɭɯ˧}}} 
\lhead{\firstmark}
\rhead{\botmark}

\subsection{\hspace{-0.5cm} {\Large \textcolor{darkblue}{\textbf{\ipa{qʰwɤ˧zo˥\$}}}}\hspace{0.5cm}[\kern2pt{\textcolor{darkblue}{\textbf{\ipa{qʰwɤ˧zo˥}}}}\kern2pt]} \hypertarget{q\string_hw7\string_Mzo\string_T\$1}{}
\markboth{\textcolor{darkblue}{\textbf{\ipa{qʰwɤ˧zo˥\$}}}}{}
\textcolor{teal}{\mytextsc{nom}} \hspace{4pt} Ton~: H\$.
\textcolor{PineGreen}{\selectlanguage{french}Petit bol.} \zh{小碗。} 
\lhead{\firstmark}
\rhead{\botmark}

\subsection{\hspace{-0.5cm} {\Large \textcolor{darkblue}{\textbf{\ipa{qʰwɤ˩\textsubscript{a}}}} \textsubscript{1}}\hspace{0.5cm}[\kern2pt{\textcolor{darkblue}{\textbf{\ipa{qʰwɤ˩˥}}}}\kern2pt]} \hypertarget{q\string_hw7\string_Ba1}{}
\markboth{\textcolor{darkblue}{\textbf{\ipa{qʰwɤ˩\textsubscript{a}}}} \textsubscript{1}}{}
\textcolor{teal}{\mytextsc{adjectif}} \hspace{4pt} Ton~: L\textsubscript{a}.
\textcolor{PineGreen}{\selectlanguage{french}Intelligent.} \zh{聪明。}  ¶ \textcolor{darkblue}{\textbf{\ipa{qʰwɤ˩-hĩ˩˥}}} \textcolor{PineGreen}{\selectlanguage{french}\mytextsc{rel}/\mytextsc{nmlz}} \zh{聪明的}  
 ¶ \textcolor{darkblue}{\textbf{\ipa{qʰwɤ˩-le˥!}}} \textcolor{PineGreen}{\selectlanguage{french}(il est/tu es) intelligent! (Commentaire lorsque quelqu'un dit ou fait quelque chose d'astucieux)} \zh{很聪明! / 太聪明了!}  
 ¶ \textcolor{darkblue}{\textbf{\ipa{ɖwæ˧˥ | qʰwɤ˩˥!}}} \textcolor{PineGreen}{\selectlanguage{french}\mytextsc{intensif}.très \string_: très intelligent} \zh{很聪明!}  

\lhead{\firstmark}
\rhead{\botmark}

\subsection{\hspace{-0.5cm} {\Large \textcolor{darkblue}{\textbf{\ipa{qʰwɤ˩\textsubscript{a}}}} \textsubscript{2}}\hspace{0.5cm}[\kern2pt{\textcolor{darkblue}{\textbf{\ipa{qʰwɤ˩˥}}}}\kern2pt]} \hypertarget{q\string_hw7\string_Ba2}{}
\markboth{\textcolor{darkblue}{\textbf{\ipa{qʰwɤ˩\textsubscript{a}}}} \textsubscript{2}}{}
\textcolor{teal}{\mytextsc{adjectif}} \hspace{4pt} Ton~: L\textsubscript{a}.
\textcolor{PineGreen}{\selectlanguage{french}Mauvais.} \zh{坏。}  ¶ \textcolor{darkblue}{\textbf{\ipa{kʰv̩˧ | qʰwɤ˩-hĩ˩˥}}} \textcolor{PineGreen}{\selectlanguage{french}une mauvaise année} \zh{(收成)不好的一年}  
 ¶ \textcolor{darkblue}{\textbf{\ipa{kʰv̩˧ qʰwɤ˧˥}}} \textcolor{PineGreen}{\selectlanguage{french}une mauvaise année} \zh{(收成)不好的一年}  
 ¶ \textcolor{darkblue}{\textbf{\ipa{tsʰi˧-ʝi˧, | kʰv̩˧qʰwɤ˧ tʰv̩˧˥!}}} \textcolor{PineGreen}{\selectlanguage{french}cette année, c'est une mauvaise année (les récoltes sont mauvaises)!} \zh{今年,年景不好!(收成不好)}  
 ¶ \textcolor{darkblue}{\textbf{\ipa{ʈʂʰɯ˧ | nv̩˩mi˩˥ | ɖwæ˧˥ | qʰwɤ˩˥!}}} \textcolor{PineGreen}{\selectlanguage{french}Il a l'âme bien noire!} \zh{他心很坏!}  
 ¶ \textcolor{darkblue}{\textbf{\ipa{qʰwɤ˩-ʝi˩˥}}} \textcolor{PineGreen}{\selectlanguage{french}faire des bêtises, faire du mal: abîmer des choses, faire des misères aux gens...} \zh{干坏事:损坏东西,干扰人家……}  
 ¶ \textcolor{darkblue}{\textbf{\ipa{ʈʂʰɯ˧-ɳɯ˧ | njɤ˧-bv̩˧ tso˧\textasciitilde{}tso˧ | le˧-qʰwɤ˩-ʝi˩-ze˩!}}} \textcolor{PineGreen}{\selectlanguage{french}Il a abîmé mes affaires!} \zh{他弄坏了我的东西!}  
 ¶ \textcolor{darkblue}{\textbf{\ipa{hĩ˧ qʰwɤ˧-ʝi˥}}} \textcolor{PineGreen}{\selectlanguage{french}embêter les gens, faire des misères aux gens} \zh{干扰人家、麻烦人}  
 ¶ \textcolor{darkblue}{\textbf{\ipa{ʈʂʰɯ˧ | to˩to˧mi˥ hĩ˩ qʰwɤ˩-ʝi˩!}}} \textcolor{PineGreen}{\selectlanguage{french}Il/elle fait exprès d'embêter les gens!} \zh{他故意麻烦人!}  

\lhead{\firstmark}
\rhead{\botmark}

\subsection{\hspace{-0.5cm} {\Large \textcolor{darkblue}{\textbf{\ipa{qʰwɤ˩ɖɯ˩}}}}\hspace{0.5cm}[\kern2pt{\textcolor{darkblue}{\textbf{\ipa{qʰwɤ˩ɖɯ˩˥}}}}\kern2pt]} \hypertarget{q\string_hw7\string_Bd`M\string_B1}{}
\markboth{\textcolor{darkblue}{\textbf{\ipa{qʰwɤ˩ɖɯ˩}}}}{}
\textcolor{teal}{\mytextsc{nom}} \hspace{4pt} Ton~: L.
\textcolor{PineGreen}{\selectlanguage{french}Membres de la famille (étendue).} \zh{亲戚。}  ¶ \textcolor{darkblue}{\textbf{\ipa{ə˧zɯ˩ | qʰwɤ˩ɖɯ˩ ɲi˥}}} \textcolor{PineGreen}{\selectlanguage{french}Tous deux, on est de la même famille} \zh{咱们两个是一家人。}  
 ¶ \textcolor{darkblue}{\textbf{\ipa{qʰwɤ˩ɖɯ˩˥, | v̩˩dze˩˥}}} \textcolor{PineGreen}{\selectlanguage{french}famille et amis, cercle familial élargi} \zh{亲人:泛指亲戚与亲密朋友们}  
 ¶ \textcolor{darkblue}{\textbf{\ipa{qʰwɤ˩ɖɯ˩ to˥}}} \textcolor{PineGreen}{\selectlanguage{french}établir des liens familiaux, unir deux familles (par un mariage)} \zh{建立起两个家庭之间的联系(通过婚姻)}  
 \zh{量词}: \textcolor{darkblue}{\textbf{\ipa{v̩˧}}}  \mytextsc{clf}~: \textcolor{darkblue}{\textbf{\ipa{v̩˧}}} 
\lhead{\firstmark}
\rhead{\botmark}

\subsection{\hspace{-0.5cm} {\Large \textcolor{darkblue}{\textbf{\ipa{qʰwɤ˧˥}}} \textsubscript{1}}\hspace{0.5cm}[\kern2pt{\textcolor{darkblue}{\textbf{\ipa{qʰwɤ˧˥}}}}\kern2pt]} \hypertarget{q\string_hw7\string_M\string_T1}{}
\markboth{\textcolor{darkblue}{\textbf{\ipa{qʰwɤ˧˥}}} \textsubscript{1}}{}
\textcolor{teal}{\mytextsc{nom}} \hspace{4pt} Ton~: MH.
\textcolor{PineGreen}{\selectlanguage{french}Bol.} \zh{碗。}  \zh{量词}: \textcolor{darkblue}{\textbf{\ipa{ɭɯ˧}}}  \mytextsc{clf}~: \textcolor{darkblue}{\textbf{\ipa{ɭɯ˧}}} 
\lhead{\firstmark}
\rhead{\botmark}

\subsection{\hspace{-0.5cm} {\Large \textcolor{darkblue}{\textbf{\ipa{qʰwɤ˧˥}}} \textsubscript{2}}\hspace{0.5cm}[\kern2pt{\textcolor{darkblue}{\textbf{\ipa{qʰwɤ˧˥}}}}\kern2pt]} \hypertarget{q\string_hw7\string_M\string_T2}{}
\markboth{\textcolor{darkblue}{\textbf{\ipa{qʰwɤ˧˥}}} \textsubscript{2}}{}
\textcolor{teal}{\mytextsc{nom}} \hspace{4pt} Ton~: MH.
\textcolor{PineGreen}{\selectlanguage{french}Histoire, récit.} \zh{故事。}  ¶ \textcolor{darkblue}{\textbf{\ipa{æ˧ʂæ˧-qʰwɤ˧˥}}} \textcolor{PineGreen}{\selectlanguage{french}récit d'autrefois, conte} \zh{老故事}  
 \zh{量词}: \textcolor{darkblue}{\textbf{\ipa{kʰwɤ˥}}}  \mytextsc{clf}~: \textcolor{darkblue}{\textbf{\ipa{kʰwɤ˥}}} 
\lhead{\firstmark}
\rhead{\botmark}

\subsection{\hspace{-0.5cm} {\Large \textcolor{darkblue}{\textbf{\ipa{qʰwɤ˧˥\textsubscript{a}}}}}\hspace{0.5cm}[\kern2pt{\textcolor{darkblue}{\textbf{\ipa{qʰwɤ˧˥}}}}\kern2pt]} \hypertarget{q\string_hw7\string_M\string_Ta1}{}
\markboth{\textcolor{darkblue}{\textbf{\ipa{qʰwɤ˧˥\textsubscript{a}}}}}{}
\textcolor{teal}{\mytextsc{classificateur}} \hspace{4pt} Ton~: MH\textsubscript{a}.
\textcolor{PineGreen}{\selectlanguage{french}Classificateur des bols (utilisés comme quantité de mesure du non dénombrable).} \zh{量词:碗。} 
\lhead{\firstmark}
\rhead{\botmark}

\newpage
\section*{\centering- \textcolor{darkblue}{\textbf{\ipa{ɻ}}} -}
\subsection{\hspace{-0.5cm} {\Large \textcolor{darkblue}{\textbf{\ipa{ɻ̍˩}}}}\hspace{0.5cm}[\kern2pt{\textcolor{darkblue}{\textbf{\ipa{ɻ̍˥}}}}\kern2pt]} \hypertarget{r£`̍\string_B1}{}
\markboth{\textcolor{darkblue}{\textbf{\ipa{ɻ̍˩}}}}{}
\textcolor{teal}{\mytextsc{nom}} \hspace{4pt} Ton~: L.
\textcolor{PineGreen}{\selectlanguage{french}Direction, côté.} \zh{面(一个四方形物品的四面)。}  ¶ \textcolor{darkblue}{\textbf{\ipa{[“Housebuilding”] ʐv̩˧-ɻ̍˥}}} \textcolor{PineGreen}{\selectlanguage{french}les quatre directions, les quatre côtés (d'une maison)} \zh{四面}  

\lhead{\firstmark}
\rhead{\botmark}

\subsection{\hspace{-0.5cm} {\Large \textcolor{darkblue}{\textbf{\ipa{ɻ̍˩\textsubscript{b}}}}}\hspace{0.5cm}[\kern2pt{\textcolor{darkblue}{\textbf{\ipa{ɻ̍˩˥}}}}\kern2pt]} \hypertarget{r£`̍\string_Bb1}{}
\markboth{\textcolor{darkblue}{\textbf{\ipa{ɻ̍˩\textsubscript{b}}}}}{}
\textcolor{teal}{\mytextsc{verbe}} \hspace{4pt} Ton~: L\textsubscript{b}.
\textcolor{PineGreen}{\selectlanguage{french}Regarder, se tourner vers, faire face à.} \zh{对着。}  ¶ \textcolor{darkblue}{\textbf{\ipa{mɤ˧-ɻ̍˩}}} \textcolor{PineGreen}{\selectlanguage{french}\mytextsc{neg}} \zh{\mytextsc{neg}}  
 ¶ \textcolor{darkblue}{\textbf{\ipa{ɖɯ˧-ɻ̍˧\textasciitilde{}ɻ̍˩}}} \textcolor{PineGreen}{\selectlanguage{french}\mytextsc{délimitatif} \string_ \mytextsc{red}} \zh{\mytextsc{delimitative} \string_ \mytextsc{red}}  
 ¶ \textcolor{darkblue}{\textbf{\ipa{ze˩gi˧ ɻ̍˥?}}} \textcolor{PineGreen}{\selectlanguage{french}dans quelle direction regarder?} \zh{(我要)往哪边转?}  
 ¶ \textcolor{darkblue}{\textbf{\ipa{no˧ | ʈʂʰɯ˧tɕo˧ ɻ̍˩!}}} \textcolor{PineGreen}{\selectlanguage{french}Tourne-toi par ici! / Regarde par ici!} \zh{你往这里转/往这里看!}  
 ¶ \textcolor{darkblue}{\textbf{\ipa{gɤ˩-ɻ̍˥ mv̩˩-ɻ̍˩, | ə˧tso˧ li˧?}}} \textcolor{PineGreen}{\selectlanguage{french}Qu’as-tu à regarder de toutes parts ? / Tu regardes de toutes parts, que cherches-tu ?} \zh{你左转右转,(到底)在看什么?}  

\lhead{\firstmark}
\rhead{\botmark}

\subsection{\hspace{-0.5cm} {\Large \textcolor{darkblue}{\textbf{\ipa{ɻ̍˧bɤ˧}}}}\hspace{0.5cm}[\kern2pt{\textcolor{darkblue}{\textbf{\ipa{ɻ̍˧bɤ˧}}}}\kern2pt]} \hypertarget{r£`̍\string_Mb7\string_M1}{}
\markboth{\textcolor{darkblue}{\textbf{\ipa{ɻ̍˧bɤ˧}}}}{}
\textcolor{teal}{\mytextsc{nom}} \hspace{4pt} Ton~: M.
\textcolor{PineGreen}{\selectlanguage{french}La vérité; le vrai et le faux; les faits authentiques.} \zh{实情,真理。}  ¶ \textcolor{darkblue}{\textbf{\ipa{njɤ˧-ɳɯ˧ | ɻ̍˧bɤ˧ | ʐwɤ˩-bi˩˥!}}} \textcolor{PineGreen}{\selectlanguage{french}je vais dire toute la vérité/je vais faire la lumière!} \zh{我要把实情说出来!}  
 \zh{量词}: \textcolor{darkblue}{\textbf{\ipa{kʰwɤ˥}}}  \mytextsc{clf}~: \textcolor{darkblue}{\textbf{\ipa{kʰwɤ˥}}} 
\lhead{\firstmark}
\rhead{\botmark}

\subsection{\hspace{-0.5cm} {\Large \textcolor{darkblue}{\textbf{\ipa{ɻ̍˧qʰv̩˧}}}}\hspace{0.5cm}[\kern2pt{\textcolor{darkblue}{\textbf{\ipa{ɻ̍˧qʰv̩˧}}}}\kern2pt]} \hypertarget{r£`̍\string_Mq\string_hv\string_=\string_M1}{}
\markboth{\textcolor{darkblue}{\textbf{\ipa{ɻ̍˧qʰv̩˧}}}}{}
\textcolor{teal}{\mytextsc{nom}} \hspace{4pt} Ton~: M.
\textcolor{PineGreen}{\selectlanguage{french}Source chaude.} \zh{温泉。}  ¶ \textcolor{darkblue}{\textbf{\ipa{ɻ̍˧qʰv̩˧-dʑɯ˩}}} \textcolor{PineGreen}{\selectlanguage{french}eau de source chaude (non potable)} \zh{温泉水(不可饮用)}  
 \zh{量词}: \textcolor{darkblue}{\textbf{\ipa{ɭɯ˧}}}  \mytextsc{clf}~: \textcolor{darkblue}{\textbf{\ipa{ɭɯ˧}}} 
\lhead{\firstmark}
\rhead{\botmark}

\subsection{\hspace{-0.5cm} {\Large \textcolor{darkblue}{\textbf{\ipa{ɻ̍˩ɻ̍˧-lo˩}}}}\hspace{0.5cm}[\kern2pt{\textcolor{darkblue}{\textbf{\ipa{ɻ̍˩ɻ̍˧lo˧}}}}\kern2pt]} \hypertarget{r£`̍\string_Br£`̍\string_M-lo\string_B1}{}
\markboth{\textcolor{darkblue}{\textbf{\ipa{ɻ̍˩ɻ̍˧-lo˩}}}}{}
\textcolor{teal}{\mytextsc{nom}} \hspace{4pt} Ton~: LM-L.
\textcolor{PineGreen}{\selectlanguage{french}Le cheval qui marche en second (derrière le cheval de tête), dans une caravan.} \zh{马帮中的第二匹马。} 
\lhead{\firstmark}
\rhead{\botmark}

\subsection{\hspace{-0.5cm} {\Large \textcolor{darkblue}{\textbf{\ipa{ɻ̍˧tɑ˧}}}}\hspace{0.5cm}[\kern2pt{\textcolor{darkblue}{\textbf{\ipa{ɻ̍˧tɑ˧}}}}\kern2pt]} \hypertarget{r£`̍\string_MtA\string_M1}{}
\markboth{\textcolor{darkblue}{\textbf{\ipa{ɻ̍˧tɑ˧}}}}{}
\textcolor{teal}{\mytextsc{nom}} \hspace{4pt} Ton~: M.
\textcolor{PineGreen}{\selectlanguage{french}Ganglions.} \zh{淋巴结。}  \zh{量词}: \textcolor{darkblue}{\textbf{\ipa{ɭɯ˧}}}  \mytextsc{clf}~: \textcolor{darkblue}{\textbf{\ipa{ɭɯ˧}}} 
\lhead{\firstmark}
\rhead{\botmark}

\subsection{\hspace{-0.5cm} {\Large \textcolor{darkblue}{\textbf{\ipa{ɻ̍˩ʈʂʰe˧-ɖɯ˩mɑ˩}}}}\hspace{0.5cm}[\kern2pt{\textcolor{darkblue}{\textbf{\ipa{ɻ̍˩ʈʂʰe˧ɖɯ˩mɑ˩}}}}\kern2pt]} \hypertarget{r£`̍\string_Bt`s`\string_he\string_M-d`M\string_BmA\string_B1}{}
\markboth{\textcolor{darkblue}{\textbf{\ipa{ɻ̍˩ʈʂʰe˧-ɖɯ˩mɑ˩}}}}{}
\textcolor{teal}{\mytextsc{nom}} \hspace{4pt} Ton~: LM-L.
\textcolor{PineGreen}{\selectlanguage{french}Prénom féminin.} \zh{女性名字。} 
\lhead{\firstmark}
\rhead{\botmark}

\subsection{\hspace{-0.5cm} {\Large \textcolor{darkblue}{\textbf{\ipa{ɻ̍˩ʈʂʰe˧-tsʰɯ˩ɻ̍˩}}}}\hspace{0.5cm}[\kern2pt{\textcolor{darkblue}{\textbf{\ipa{ɻ̍˩ʈʂʰe˧tsʰɯ˩ɻ̍˩}}}}\kern2pt]} \hypertarget{r£`̍\string_Bt`s`\string_he\string_M-ts\string_hM\string_Br£`̍\string_B1}{}
\markboth{\textcolor{darkblue}{\textbf{\ipa{ɻ̍˩ʈʂʰe˧-tsʰɯ˩ɻ̍˩}}}}{}
\textcolor{teal}{\mytextsc{nom}} \hspace{4pt} Ton~: LM-L.
\textcolor{PineGreen}{\selectlanguage{french}Prénom masculin.} \zh{男性名字。} 
\lhead{\firstmark}
\rhead{\botmark}

\subsection{\hspace{-0.5cm} {\Large \textcolor{darkblue}{\textbf{\ipa{ɻ̍˩ʈʂʰe\#˥}}}}\hspace{0.5cm}[\kern2pt{\textcolor{darkblue}{\textbf{\ipa{ɻ̍˩ʈʂʰe˥}}}}\kern2pt]} \hypertarget{r£`̍\string_Bt`s`\string_he\#\string_T1}{}
\markboth{\textcolor{darkblue}{\textbf{\ipa{ɻ̍˩ʈʂʰe\#˥}}}}{}
\textcolor{teal}{\mytextsc{nom}} \hspace{4pt} Ton~: LM+\#H.
\textcolor{PineGreen}{\selectlanguage{french}Prénom masculin.} \zh{男性名字。} 
\lhead{\firstmark}
\rhead{\botmark}

\subsection{\hspace{-0.5cm} {\Large \textcolor{darkblue}{\textbf{\ipa{‑ɻ̍˩}}}}\hspace{0.5cm}[\kern2pt{\textcolor{darkblue}{\textbf{\ipa{ɻ̍˩˥}}}}\kern2pt]} \hypertarget{‑r£`̍\string_B1}{}
\markboth{\textcolor{darkblue}{\textbf{\ipa{‑ɻ̍˩}}}}{}
\textcolor{teal}{\mytextsc{suffixe}} \hspace{4pt} Ton~: L.
\textcolor{PineGreen}{\selectlanguage{french}\mytextsc{inchoatif}.} \zh{\mytextsc{发端。}} 
\lhead{\firstmark}
\rhead{\botmark}

\subsection{\hspace{-0.5cm} {\Large \textcolor{darkblue}{\textbf{\ipa{=ɻ̍˩}}}}\hspace{0.5cm}[\kern2pt{\textcolor{darkblue}{\textbf{\ipa{ɻ̍˩˥}}}}\kern2pt]} \hypertarget{=r£`̍\string_B1}{}
\markboth{\textcolor{darkblue}{\textbf{\ipa{=ɻ̍˩}}}}{}
\textcolor{teal}{\mytextsc{clitique}} \hspace{4pt} Ton~: L.
\textcolor{PineGreen}{\selectlanguage{french}Pluriel associatif, ou pluriel d'accompagnement, couramment utilisé avec les termes de parenté, les noms de clans...} \zh{联想复数:一家人、一族人、一辈人……。}  ¶ \textcolor{darkblue}{\textbf{\ipa{ʈʂʰɯ˧-ʑi˧=ɻ̍˥}}} \textcolor{PineGreen}{\selectlanguage{french}les gens de cette famille; cette maisonnée-ci} \zh{这家的人}  

\lhead{\firstmark}
\rhead{\botmark}

\subsection{\hspace{-0.5cm} {\Large \textcolor{darkblue}{\textbf{\ipa{=ɻæ˩}}}}\hspace{0.5cm}[\kern2pt{\textcolor{darkblue}{\textbf{\ipa{ɻæ˩˥}}}}\kern2pt]} \hypertarget{=r£`\{\string_B1}{}
\markboth{\textcolor{darkblue}{\textbf{\ipa{=ɻæ˩}}}}{}
\textcolor{teal}{\mytextsc{clitique}} \hspace{4pt} Ton~: L.
\textcolor{PineGreen}{\selectlanguage{french}Pluriel.} \zh{多数。}  ¶ \textcolor{darkblue}{\textbf{\ipa{ʈʂʰɯ˧-ɻæ˥\$}}} \textcolor{PineGreen}{\selectlanguage{french}ces choses-ci, cette sorte de choses} \zh{这类的东西,……之类}  

\lhead{\firstmark}
\rhead{\botmark}

\subsection{\hspace{-0.5cm} {\Large \textcolor{darkblue}{\textbf{\ipa{ɻæ˩\textsubscript{a}}}}}\hspace{0.5cm}[\kern2pt{\textcolor{darkblue}{\textbf{\ipa{ɻæ˩˥}}}}\kern2pt]} \hypertarget{r£`\{\string_Ba1}{}
\markboth{\textcolor{darkblue}{\textbf{\ipa{ɻæ˩\textsubscript{a}}}}}{}
\textcolor{teal}{\mytextsc{adjectif}} \hspace{4pt} Ton~: L\textsubscript{a}.
\textcolor{PineGreen}{\selectlanguage{french}Plat, de forme plate, aplati.} \zh{瘪。}  ¶ \textcolor{darkblue}{\textbf{\ipa{ɻæ˩-hĩ˩˥}}} \textcolor{PineGreen}{\selectlanguage{french}\mytextsc{rel}/\mytextsc{nmlz}} \zh{瘪的}  
 ¶ \textcolor{darkblue}{\textbf{\ipa{ɻæ˩ti˩ɻæ˥ (-gv̩˩)}}} \textcolor{PineGreen}{\selectlanguage{french}raplapla, ratatiné} \zh{瘪瘪的}  

\lhead{\firstmark}
\rhead{\botmark}

\subsection{\hspace{-0.5cm} {\Large \textcolor{darkblue}{\textbf{\ipa{ɻæ˩˥}}} \textsubscript{1}}\hspace{0.5cm}[\kern2pt{\textcolor{darkblue}{\textbf{\ipa{ɻæ˩˥}}}}\kern2pt]} \hypertarget{r£`\{\string_B\string_T1}{}
\markboth{\textcolor{darkblue}{\textbf{\ipa{ɻæ˩˥}}} \textsubscript{1}}{}
\textcolor{teal}{\mytextsc{nom}} \hspace{4pt} Ton~: LH.
\textcolor{PineGreen}{\selectlanguage{french}Graine.} \zh{种子。}  \zh{量词}: \textcolor{darkblue}{\textbf{\ipa{ɭɯ˧}}}  \mytextsc{clf}~: \textcolor{darkblue}{\textbf{\ipa{ɭɯ˧}}} \textcolor{PineGreen}{\selectlanguage{french}objets ronds}
\lhead{\firstmark}
\rhead{\botmark}

\subsection{\hspace{-0.5cm} {\Large \textcolor{darkblue}{\textbf{\ipa{ɻæ˩˥}}} \textsubscript{2}}\hspace{0.5cm}[\kern2pt{\textcolor{darkblue}{\textbf{\ipa{ɻæ˩˥}}}}\kern2pt]} \hypertarget{r£`\{\string_B\string_T2}{}
\markboth{\textcolor{darkblue}{\textbf{\ipa{ɻæ˩˥}}} \textsubscript{2}}{}
\textcolor{teal}{\mytextsc{nom}} \hspace{4pt} Ton~: LH.
\textcolor{PineGreen}{\selectlanguage{french}Joug (le terme est le même pour un ou deux animaux).} \zh{牛轭(单行或双行)。}  ¶ \textcolor{darkblue}{\textbf{\ipa{ʝi˧-ɻæ˥}}} \textcolor{PineGreen}{\selectlanguage{french}même sens que la forme monosyllabique: joug; littéralement 'joug de bœuf/buffle'} \zh{牛轭}  
 ¶ \textcolor{darkblue}{\textbf{\ipa{ɻæ˩ ʈʂʰɯ˩-ɭɯ˥ / ɻæ˩ ʈʂʰɯ˩-ɭɯ˩˥}}} \textcolor{PineGreen}{\selectlanguage{french}\mytextsc{n}+\mytextsc{dem}+\mytextsc{clf;} cette expression possède deux variantes tonales} \zh{这个牛轭}  
 \zh{量词}: \textcolor{darkblue}{\textbf{\ipa{ɭɯ˧}}}  \mytextsc{clf}~: \textcolor{darkblue}{\textbf{\ipa{ɭɯ˧}}} 
\lhead{\firstmark}
\rhead{\botmark}

\subsection{\hspace{-0.5cm} {\Large \textcolor{darkblue}{\textbf{\ipa{ɻwæ˥}}}}\hspace{0.5cm}[\kern2pt{\textcolor{darkblue}{\textbf{\ipa{ɻwæ˥}}}}\kern2pt]} \hypertarget{r£`w\{\string_T1}{}
\markboth{\textcolor{darkblue}{\textbf{\ipa{ɻwæ˥}}}}{}
\textcolor{teal}{\mytextsc{verbe}} \hspace{4pt} Ton~: H.
\ding{202} \textcolor{PineGreen}{\selectlanguage{french}Crier, hurler; miauler; braire; hennir; rugir (chat, bœuf, cochon, mouton, loup, lion).} \zh{喊、吼、叫(人、猫、牛、猪、羊、狼、驴、狮子、老虎、豺狼……)。}  ¶ \textcolor{darkblue}{\textbf{\ipa{mɤ˧-ɻwæ˥}}} \textcolor{PineGreen}{\selectlanguage{french}\mytextsc{neg}} \zh{不叫}  
 ¶ \textcolor{darkblue}{\textbf{\ipa{ɻwæ˧\textasciitilde{}ɻwæ˧}}} \textcolor{PineGreen}{\selectlanguage{french}\mytextsc{red}} \zh{\mytextsc{重叠}}  
 ¶ \textcolor{darkblue}{\textbf{\ipa{hĩ˧ ɻwæ˧-dʑo˩}}} \textcolor{PineGreen}{\selectlanguage{french}Il y a quelqu'un qui est est en train d'appeler/de crier} \zh{有人在叫。}  
 ¶ \textcolor{darkblue}{\textbf{\ipa{ɖɯ˧-ɻwæ˧-ɻ̍˥}}} \textcolor{PineGreen}{\selectlanguage{french}appeler, lancer un appel} \zh{叫一声}  
 ¶ \textcolor{darkblue}{\textbf{\ipa{hwɤ˧li˧ ɻwæ˥-dʑo˩}}} \textcolor{PineGreen}{\selectlanguage{french}le chat miaule} \zh{猫在叫}  
 ¶ \textcolor{darkblue}{\textbf{\ipa{æ̃˩ ɻwæ˥}}} \textcolor{PineGreen}{\selectlanguage{french}la poule caquette} \zh{鸡在叫}  
 ¶ \textcolor{darkblue}{\textbf{\ipa{ʐwæ˧pʰæ˧di˧˥ | tʰi˧-ɻwæ˥-dʑo˩}}} \textcolor{PineGreen}{\selectlanguage{french}l'âne brait} \zh{驴在叫}  
 ¶ \textcolor{darkblue}{\textbf{\ipa{[F5] ʐwæ˧ | tʰi˧-ɻwæ˥-dʑo˩}}} \textcolor{PineGreen}{\selectlanguage{french}le cheval est en train de hennir} \zh{马在嘶}  
 ¶ \textcolor{darkblue}{\textbf{\ipa{ʐwæ˧ ɻwæ˧-dʑo˩!}}} \textcolor{PineGreen}{\selectlanguage{french}le cheval est en train de hennir} \zh{马在嘶}  
\ding{203} \textcolor{PineGreen}{\selectlanguage{french}Inviter, faire venir.} \zh{请、叫(来)。}  ¶ \textcolor{darkblue}{\textbf{\ipa{ɖɯ˧-ɻwæ˧-ɻ̍˥}}} \textcolor{PineGreen}{\selectlanguage{french}\mytextsc{délimitatif} \string_ \mytextsc{inchoatif}} \zh{请来一下}  
 ¶ \textcolor{darkblue}{\textbf{\ipa{tʰɑ˧-ɻwæ˥!}}} \textcolor{PineGreen}{\selectlanguage{french}\mytextsc{prohib}} \zh{不要请!}  
 ¶ \textcolor{darkblue}{\textbf{\ipa{ɻwæ˧-mɤ˧-bi˧!}}} \textcolor{PineGreen}{\selectlanguage{french}(on) ne l'invite pas!} \zh{不请他!}  

\lhead{\firstmark}
\rhead{\botmark}

\subsection{\hspace{-0.5cm} {\Large \textcolor{darkblue}{\textbf{\ipa{ɻwæ˥\textsubscript{b}}}}}\hspace{0.5cm}[\kern2pt{\textcolor{darkblue}{\textbf{\ipa{ɻwæ˥}}}}\kern2pt]} \hypertarget{r£`w\{\string_Tb1}{}
\markboth{\textcolor{darkblue}{\textbf{\ipa{ɻwæ˥\textsubscript{b}}}}}{}
\textcolor{teal}{\mytextsc{classificateur}} \hspace{4pt} Ton~: H\textsubscript{b}.
\textcolor{PineGreen}{\selectlanguage{french}Classificateur des lieux, des endroits.} \zh{量词:地方(一处)。}  ¶ \textcolor{darkblue}{\textbf{\ipa{tʰv̩˧-ɻwæ˧-qo˥ | mɤ˧-tʰv̩˧-sɯ˩!}}} \textcolor{PineGreen}{\selectlanguage{french}...n'est jamais allé dans ces lieux-là} \zh{还没到这些地方}  

\lhead{\firstmark}
\rhead{\botmark}

\newpage
\section*{\centering- \textcolor{darkblue}{\textbf{\ipa{ɻ̃}}} -}
\subsection{\hspace{-0.5cm} {\Large \textcolor{darkblue}{\textbf{\ipa{ɻ̃˥}}}}\hspace{0.5cm}[\kern2pt{\textcolor{darkblue}{\textbf{\ipa{ɻ̃˥}}}}\kern2pt]} \hypertarget{r£`\string_~\string_T1}{}
\markboth{\textcolor{darkblue}{\textbf{\ipa{ɻ̃˥}}}}{}
\textcolor{teal}{\mytextsc{nom}} \hspace{4pt} Ton~: \#H.
\textcolor{PineGreen}{\selectlanguage{french}Os, ossement.} \zh{骨头。}  \zh{量词}: \textcolor{darkblue}{\textbf{\ipa{kɤ˧˥}}}  \mytextsc{clf}~: \textcolor{darkblue}{\textbf{\ipa{kɤ˧˥}}} 
\lhead{\firstmark}
\rhead{\botmark}

\subsection{\hspace{-0.5cm} {\Large \textcolor{darkblue}{\textbf{\ipa{ɻ̃˥}}}}\hspace{0.5cm}[\kern2pt{\textcolor{darkblue}{\textbf{\ipa{ɻ̃˥}}}}\kern2pt]} \hypertarget{r£`\string_~\string_T1}{}
\markboth{\textcolor{darkblue}{\textbf{\ipa{ɻ̃˥}}}}{}
\textcolor{teal}{\mytextsc{adjectif}} \hspace{4pt} Ton~: H.
\textcolor{PineGreen}{\selectlanguage{french}Démuni, en mauvaise passe.} \zh{困难、贫穷。}  ¶ \textcolor{darkblue}{\textbf{\ipa{le˧-ɻ̃˥-ze˩!}}} \textcolor{PineGreen}{\selectlanguage{french}(il) est en mauvaise passe!/il est à la rue!} \zh{(他)真的很穷苦!}  
 ¶ \textcolor{darkblue}{\textbf{\ipa{le˧-ɻ̃˧-bi˧}}} \textcolor{PineGreen}{\selectlanguage{french}\mytextsc{accomp} \string_ \mytextsc{fut}\string_imm} \zh{\mytextsc{accomp} \string_ \mytextsc{fut}\string_imm}  
 ¶ \textcolor{darkblue}{\textbf{\ipa{mɤ˧-ɻ̃˥}}} \textcolor{PineGreen}{\selectlanguage{french}\mytextsc{neg}} \zh{\mytextsc{neg}}  
 ¶ \textcolor{darkblue}{\textbf{\ipa{le˧-ɻ̃˧-zo˥, | ɻ̃˧-lɑ˩ bi˩-mɤ˩-dʑɯ˩!}}} \textcolor{PineGreen}{\selectlanguage{french}“Pour démuni/mal nourri/affamé qu'on soit, on n'en est pas encore maigre au point d'avoir les os à découvert!” Jeu de mots sur 'démuni' et 'ossement', qui sont homophones. Le proverbe sert à relativiser le malheur ressenti par quelqu'un.} \zh{很困难,也还没有到饿死的程度啊! / 再困难,也还没饿死!(直译:“再困难,也没有露出骨头!”这个成语,来安慰认为自己太可怜的人。)}  
 ¶ \textcolor{darkblue}{\textbf{\ipa{ɻ̃˧-ʐwɤ˧˥}}} \textcolor{PineGreen}{\selectlanguage{french}se plaindre} \zh{诉苦、抱怨}  
 ¶ \textcolor{darkblue}{\textbf{\ipa{ɻ̃˧-ʐwɤ˧ | dɑ˧-ʐwɤ˧-ɻ̍˥}}} \textcolor{PineGreen}{\selectlanguage{french}raconter ses malheurs; se plaindre} \zh{诉苦、讲自己的不幸}  
 ¶ \textcolor{darkblue}{\textbf{\ipa{ʈʂʰɯ˧ | mɑ˧dɑ˩-qʰwɤ˩, | ɻ̃˧-ʐwɤ˧ | dɑ˧-ʐwɤ˧-ɻ̍˥!}}} \textcolor{PineGreen}{\selectlanguage{french}Il est malheureux; il passe son temps à se plaindre!} \zh{他不幸福,一直在讲自己怎么可怜!}  

\lhead{\firstmark}
\rhead{\botmark}

\subsection{\hspace{-0.5cm} {\Large \textcolor{darkblue}{\textbf{\ipa{ɻ̃˧}}}}\hspace{0.5cm}[\kern2pt{\textcolor{darkblue}{\textbf{\ipa{ɻ̃˥}}}}\kern2pt]} \hypertarget{r£`\string_~\string_M1}{}
\markboth{\textcolor{darkblue}{\textbf{\ipa{ɻ̃˧}}}}{}
\textcolor{teal}{\mytextsc{nom}} \hspace{4pt} Ton~: M.
\textcolor{PineGreen}{\selectlanguage{french}Clan: ensemble de familles.} \zh{家族。}  ¶ \textcolor{darkblue}{\textbf{\ipa{ɻ̃˧ ɖɯ˧-ɻ̃˧}}} \textcolor{PineGreen}{\selectlanguage{french}un clan} \zh{一个家族}  
 \zh{量词}: \textcolor{darkblue}{\textbf{\ipa{ɻ̃˧}}}  \mytextsc{clf}~: \textcolor{darkblue}{\textbf{\ipa{ɻ̃˧}}} 
\lhead{\firstmark}
\rhead{\botmark}

\subsection{\hspace{-0.5cm} {\Large \textcolor{darkblue}{\textbf{\ipa{ɻ̃˧\textsubscript{b}}}}}\hspace{0.5cm}[\kern2pt{\textcolor{darkblue}{\textbf{\ipa{ɻ̃˥}}}}\kern2pt]} \hypertarget{r£`\string_~\string_Mb1}{}
\markboth{\textcolor{darkblue}{\textbf{\ipa{ɻ̃˧\textsubscript{b}}}}}{}
\textcolor{teal}{\mytextsc{classificateur}} \hspace{4pt} Ton~: M\textsubscript{b}.
\textcolor{PineGreen}{\selectlanguage{french}Classificateur des clans / groupes de familles: littéralement 'un os'. Echelon supérieur à celui de la 'communauté familiale' dans la terminologie de Fu Maoji (1983).} \zh{量词:家族。} 
\lhead{\firstmark}
\rhead{\botmark}

\subsection{\hspace{-0.5cm} {\Large \textcolor{darkblue}{\textbf{\ipa{ɻ̃˧hæ˩}}}}\hspace{0.5cm}[\kern2pt{\textcolor{darkblue}{\textbf{\ipa{ɻ̃˧hæ˩}}}}\kern2pt]} \hypertarget{r£`\string_~\string_Mh\{\string_B1}{}
\markboth{\textcolor{darkblue}{\textbf{\ipa{ɻ̃˧hæ˩}}}}{}
\textcolor{teal}{\mytextsc{nom}} \hspace{4pt} Ton~: L\#.
\textcolor{PineGreen}{\selectlanguage{french}Cartilage.} \zh{软骨。}  \zh{量词}: \textcolor{darkblue}{\textbf{\ipa{ɭɯ˧}}} \textcolor{darkblue}{\textbf{\ipa{kɤ˧˥}}}  \mytextsc{clf}~: \textcolor{darkblue}{\textbf{\ipa{ɭɯ˧}}} \textcolor{darkblue}{\textbf{\ipa{kɤ˧˥}}} 
\lhead{\firstmark}
\rhead{\botmark}

\subsection{\hspace{-0.5cm} {\Large \textcolor{darkblue}{\textbf{\ipa{ɻ̃˧kɤ˩}}}}\hspace{0.5cm}[\kern2pt{\textcolor{darkblue}{\textbf{\ipa{ɻ̃˧kɤ˩}}}}\kern2pt]} \hypertarget{r£`\string_~\string_Mk7\string_B1}{}
\markboth{\textcolor{darkblue}{\textbf{\ipa{ɻ̃˧kɤ˩}}}}{}
\textcolor{teal}{\mytextsc{nom}} \hspace{4pt} Ton~: L\#.
\textcolor{PineGreen}{\selectlanguage{french}Colonne vertébrale.} \zh{脊椎骨。}  \zh{量词}: \textcolor{darkblue}{\textbf{\ipa{kɤ˧˥}}}  \mytextsc{clf}~: \textcolor{darkblue}{\textbf{\ipa{kɤ˧˥}}} 
\lhead{\firstmark}
\rhead{\botmark}

\subsection{\hspace{-0.5cm} {\Large \textcolor{darkblue}{\textbf{\ipa{ɻ̃˧ko˩}}}}\hspace{0.5cm}[\kern2pt{\textcolor{darkblue}{\textbf{\ipa{ɻ̃˧ko˩}}}}\kern2pt]} \hypertarget{r£`\string_~\string_Mko\string_B1}{}
\markboth{\textcolor{darkblue}{\textbf{\ipa{ɻ̃˧ko˩}}}}{}
\textcolor{teal}{\mytextsc{nom}} \hspace{4pt} Ton~: L\#.
\textcolor{PineGreen}{\selectlanguage{french}Tibia.} \zh{胫骨。}  ¶ \textcolor{darkblue}{\textbf{\ipa{hĩ˧-dzɑ˧ | ɖʐe˧ tʰɑ˧-ʝi˥, | ɻ̃˧ko˩ mi˩ tʰɑ˩-tʰv̩˩. |}}} \textcolor{PineGreen}{\selectlanguage{french}“Le pauvre ne doit pas emprunter d'argent; le tibia ne doit pas recevoir de blessure!” (Ce proverbe enseigne qu'il ne faut pas toucher les points les plus sensibles, les plus fragiles.)} \zh{“穷人莫借钱,胫骨莫受伤!”}  
 \zh{量词}: \textcolor{darkblue}{\textbf{\ipa{kɤ˧˥}}}  \mytextsc{clf}~: \textcolor{darkblue}{\textbf{\ipa{kɤ˧˥}}} 
\lhead{\firstmark}
\rhead{\botmark}

\subsection{\hspace{-0.5cm} {\Large \textcolor{darkblue}{\textbf{\ipa{ɻ̃˧mi˧}}}}\hspace{0.5cm}[\kern2pt{\textcolor{darkblue}{\textbf{\ipa{ɻ̃˧mi˧}}}}\kern2pt]} \hypertarget{r£`\string_~\string_Mmi\string_M1}{}
\markboth{\textcolor{darkblue}{\textbf{\ipa{ɻ̃˧mi˧}}}}{}
\textcolor{teal}{\mytextsc{nom}} \hspace{4pt} Ton~: M.
\textcolor{PineGreen}{\selectlanguage{french}Tronc.} \zh{树干。}  ¶ \textcolor{darkblue}{\textbf{\ipa{si˧dzi˩-ɻ̃˩mi˩}}} \textcolor{PineGreen}{\selectlanguage{french}tronc d'arbre} \zh{树干}  
 \zh{量词}: \textcolor{darkblue}{\textbf{\ipa{kɤ˧˥}}}  \mytextsc{clf}~: \textcolor{darkblue}{\textbf{\ipa{kɤ˧˥}}} 
\lhead{\firstmark}
\rhead{\botmark}

\subsection{\hspace{-0.5cm} {\Large \textcolor{darkblue}{\textbf{\ipa{ɻ̃˧ʈʂæ˩}}}}\hspace{0.5cm}[\kern2pt{\textcolor{darkblue}{\textbf{\ipa{ɻ̃˧ʈʂæ˩}}}}\kern2pt]} \hypertarget{r£`\string_~\string_Mt`s`\{\string_B1}{}
\markboth{\textcolor{darkblue}{\textbf{\ipa{ɻ̃˧ʈʂæ˩}}}}{}
\textcolor{teal}{\mytextsc{nom}} \hspace{4pt} Ton~: L\#.
\textcolor{PineGreen}{\selectlanguage{french}Articulations (de la jambe: la cheville, le genou…; du bras: le poignet, le coude…).} \zh{关节部位,关节。}  \zh{量词}: \textcolor{darkblue}{\textbf{\ipa{ʈʂæ˧˥}}}  \mytextsc{clf}~: \textcolor{darkblue}{\textbf{\ipa{ʈʂæ˧˥}}} 
\lhead{\firstmark}
\rhead{\botmark}

\subsection{\hspace{-0.5cm} {\Large \textcolor{darkblue}{\textbf{\ipa{ɻ̃˧ʈʂwæ˩}}}}\hspace{0.5cm}[\kern2pt{\textcolor{darkblue}{\textbf{\ipa{ɻ̃˧ʈʂwæ˩}}}}\kern2pt]} \hypertarget{r£`\string_~\string_Mt`s`w\{\string_B1}{}
\markboth{\textcolor{darkblue}{\textbf{\ipa{ɻ̃˧ʈʂwæ˩}}}}{}
\textcolor{teal}{\mytextsc{nom}} \hspace{4pt} Ton~: L\#.
\textcolor{PineGreen}{\selectlanguage{french}\textit{Toricellia angulata Oliv.}.} \zh{接骨丹。}  ¶ \textcolor{darkblue}{\textbf{\ipa{ɻ̃˧ʈʂwæ˩-si˩}}} \textcolor{PineGreen}{\selectlanguage{french}même sens} \zh{同上}  

\lhead{\firstmark}
\rhead{\botmark}

\newpage
\section*{\centering- \textcolor{darkblue}{\textbf{\ipa{ʁ}}} -}
\subsection{\hspace{-0.5cm} {\Large \textcolor{darkblue}{\textbf{\ipa{ʁɑ˥}}}}\hspace{0.5cm}[\kern2pt{\textcolor{darkblue}{\textbf{\ipa{ʁɑ˥}}}}\kern2pt]} \hypertarget{RA\string_T1}{}
\markboth{\textcolor{darkblue}{\textbf{\ipa{ʁɑ˥}}}}{}
\textcolor{teal}{\mytextsc{nom}} \hspace{4pt} Ton~: \#H.
\textcolor{PineGreen}{\selectlanguage{french}Force.} \zh{力气。}  ¶ \textcolor{darkblue}{\textbf{\ipa{ʁɑ˧ ʑi˧}}} \textcolor{PineGreen}{\selectlanguage{french}avoir de la force} \zh{有力量}  
 ¶ \textcolor{darkblue}{\textbf{\ipa{no˧ɻ̍˩ | hĩ˧tɕʰi˧ ʁɑ˧ ʑi˧!}}} \textcolor{PineGreen}{\selectlanguage{french}votre famille/lignée/tribu est puissante!} \zh{你们家族很强大!}  
 ¶ \textcolor{darkblue}{\textbf{\ipa{ʁɑ˧ tʰv̩˧ (+ze˩)}}} \textcolor{PineGreen}{\selectlanguage{french}faire des efforts, donner toutes ses forces, s'impliquer (dans une tâche)} \zh{尽力}  

\lhead{\firstmark}
\rhead{\botmark}

\subsection{\hspace{-0.5cm} {\Large \textcolor{darkblue}{\textbf{\ipa{ʁɑ˥}}} \textsubscript{1}}\hspace{0.5cm}[\kern2pt{\textcolor{darkblue}{\textbf{\ipa{ʁɑ˥}}}}\kern2pt]} \hypertarget{RA\string_T1}{}
\markboth{\textcolor{darkblue}{\textbf{\ipa{ʁɑ˥}}} \textsubscript{1}}{}
\textcolor{teal}{\mytextsc{verbe}} \hspace{4pt} Ton~: H.
\textcolor{PineGreen}{\selectlanguage{french}Convier, faire venir, inviter.} \zh{请。} 
\lhead{\firstmark}
\rhead{\botmark}

\subsection{\hspace{-0.5cm} {\Large \textcolor{darkblue}{\textbf{\ipa{ʁɑ˥}}} \textsubscript{2}}\hspace{0.5cm}[\kern2pt{\textcolor{darkblue}{\textbf{\ipa{ʁɑ˥}}}}\kern2pt]} \hypertarget{RA\string_T2}{}
\markboth{\textcolor{darkblue}{\textbf{\ipa{ʁɑ˥}}} \textsubscript{2}}{}
\textcolor{teal}{\mytextsc{verbe}} \hspace{4pt} Ton~: H.
\textcolor{PineGreen}{\selectlanguage{french}Gagner.} \zh{赢。}  ¶ \textcolor{darkblue}{\textbf{\ipa{le˧-ʁɑ˥-ze˩}}} \textcolor{PineGreen}{\selectlanguage{french}\mytextsc{accomp} \string_ \mytextsc{pfv}} \zh{赢了}  

\lhead{\firstmark}
\rhead{\botmark}

\subsection{\hspace{-0.5cm} {\Large \textcolor{darkblue}{\textbf{\ipa{ʁɑ˧}}} \textsubscript{1}}\hspace{0.5cm}[\kern2pt{\textcolor{darkblue}{\textbf{\ipa{ʁɑ˥}}}}\kern2pt]} \hypertarget{RA\string_M1}{}
\markboth{\textcolor{darkblue}{\textbf{\ipa{ʁɑ˧}}} \textsubscript{1}}{}
\textcolor{teal}{\mytextsc{adjectif}} \hspace{4pt} Ton~: M.
\textcolor{PineGreen}{\selectlanguage{french}Bon, fiable: objet de bonne qualité; travail de bonne tenue; personne ayant bon caractère.} \zh{好(质量好,品质好,脾气好)。}  ¶ \textcolor{darkblue}{\textbf{\ipa{mɤ˧-ʁɑ˧-hĩ˧ ʂe˧}}} \textcolor{PineGreen}{\selectlanguage{french}de la mauvaise viande} \zh{不好的肉(质量不好)}  
 ¶ \textcolor{darkblue}{\textbf{\ipa{mɤ˧-ʁɑ˧-hĩ˧ ʂe˧-kʰwɤ˧ ki˩}}} \textcolor{PineGreen}{\selectlanguage{french}donner un morceau de mauvaise viande} \zh{给一块不好的肉}  
 ¶ \textcolor{darkblue}{\textbf{\ipa{pʰi˩ko˧ | mɤ˧-ʁɑ˧-ze˧!}}} \textcolor{PineGreen}{\selectlanguage{french}Les pommes ne sont plus bonnes! (Contexte: au mois de mars, les pommes de la récolte précédente ne sont plus bonnes, elles sont fripées ou pourries.)} \zh{苹果不好了! / 苹果不新鲜了!(三月份,上一季收获的苹果已经不好吃的了,或者烂了,或者变酸)}  
 ¶ \textcolor{darkblue}{\textbf{\ipa{hĩ˧ ɖɯ˧-v̩˧ | ʁɑ˧-mɤ˧-ʑi˧-hĩ˧ ʐwɤ˧˥!}}} \textcolor{PineGreen}{\selectlanguage{french}quelqu'un dit n'importe quoi} \zh{有人在乱说话!}  

\lhead{\firstmark}
\rhead{\botmark}

\subsection{\hspace{-0.5cm} {\Large \textcolor{darkblue}{\textbf{\ipa{ʁɑ˧}}} \textsubscript{2}}\hspace{0.5cm}[\kern2pt{\textcolor{darkblue}{\textbf{\ipa{ʁɑ˥}}}}\kern2pt]} \hypertarget{RA\string_M2}{}
\markboth{\textcolor{darkblue}{\textbf{\ipa{ʁɑ˧}}} \textsubscript{2}}{}
\textcolor{teal}{\mytextsc{verbe}} \hspace{4pt} Ton~: M.
\textit{\textcolor{PineGreen}{\selectlanguage{french}archaïque}} [\zh{古语}] \textcolor{PineGreen}{\selectlanguage{french}Présenter ses excuses, demander pardon.} \zh{道歉、请人家原谅。}  ¶ \textcolor{darkblue}{\textbf{\ipa{ʁɑ˧-ze˧!}}} \textcolor{PineGreen}{\selectlanguage{french}Pardon! (adressé à une personne de haut rang)} \zh{抱歉! / 请原谅!(对地位比自己高的人说)}  

\lhead{\firstmark}
\rhead{\botmark}

\subsection{\hspace{-0.5cm} {\Large \textcolor{darkblue}{\textbf{\ipa{ʁɑ˧\textsubscript{b}}}}}\hspace{0.5cm}[\kern2pt{\textcolor{darkblue}{\textbf{\ipa{ʁɑ˥}}}}\kern2pt]} \hypertarget{RA\string_Mb1}{}
\markboth{\textcolor{darkblue}{\textbf{\ipa{ʁɑ˧\textsubscript{b}}}}}{}
\textcolor{teal}{\mytextsc{verbe}} \hspace{4pt} Ton~: M\textsubscript{b}.
\textcolor{PineGreen}{\selectlanguage{french}Enjamber (un ruisseau...).} \zh{跨(跨过小沟)。}  ¶ \textcolor{darkblue}{\textbf{\ipa{le˧-ʁɑ˧-ze˧}}} \textcolor{PineGreen}{\selectlanguage{french}\mytextsc{accomp} \string_ \mytextsc{pfv}} \zh{跨过了}  

\lhead{\firstmark}
\rhead{\botmark}

\subsection{\hspace{-0.5cm} {\Large \textcolor{darkblue}{\textbf{\ipa{ʁɑ˧dzi˧}}}}\hspace{0.5cm}[\kern2pt{\textcolor{darkblue}{\textbf{\ipa{ʁɑ˧dzi˧}}}}\kern2pt]} \hypertarget{RA\string_Mdzi\string_M1}{}
\markboth{\textcolor{darkblue}{\textbf{\ipa{ʁɑ˧dzi˧}}}}{}
\textcolor{teal}{\mytextsc{nom}} \hspace{4pt} Ton~: M.
\textcolor{PineGreen}{\selectlanguage{french}Peuplier.} \zh{杨树。}  \zh{量词}: \textcolor{darkblue}{\textbf{\ipa{dzi˩}}}  \mytextsc{clf}~: \textcolor{darkblue}{\textbf{\ipa{dzi˩}}} 
\lhead{\firstmark}
\rhead{\botmark}

\subsection{\hspace{-0.5cm} {\Large \textcolor{darkblue}{\textbf{\ipa{ʁɑ˧ɭɯ\#˥}}}}\hspace{0.5cm}[\kern2pt{\textcolor{darkblue}{\textbf{\ipa{ʁɑ˧ɭɯ˧}}}}\kern2pt]} \hypertarget{RA\string_Ml\string_RM\#\string_T1}{}
\markboth{\textcolor{darkblue}{\textbf{\ipa{ʁɑ˧ɭɯ\#˥}}}}{}
\textcolor{teal}{\mytextsc{nom}} \hspace{4pt} Ton~: \#H.
\textcolor{PineGreen}{\selectlanguage{french}Cairn: tas de pierres qui aide à repérer un sentier.} \zh{石堆。}  ¶ \textcolor{darkblue}{\textbf{\ipa{qo˩qɑ˩-ʁɑ˥ɭɯ˩}}} \textcolor{PineGreen}{\selectlanguage{french}cairn situé à un col} \zh{垭口石堆:垭口上的石堆}  
 \zh{量词}: \textcolor{darkblue}{\textbf{\ipa{ɭɯ˧}}}  \mytextsc{clf}~: \textcolor{darkblue}{\textbf{\ipa{ɭɯ˧}}} 
\lhead{\firstmark}
\rhead{\botmark}

\subsection{\hspace{-0.5cm} {\Large \textcolor{darkblue}{\textbf{\ipa{ʁɑ˧pv̩˧}}}}\hspace{0.5cm}[\kern2pt{\textcolor{darkblue}{\textbf{\ipa{ʁɑ˧pv̩˧}}}}\kern2pt]} \hypertarget{RA\string_Mpv\string_=\string_M1}{}
\markboth{\textcolor{darkblue}{\textbf{\ipa{ʁɑ˧pv̩˧}}}}{}
\textcolor{teal}{\mytextsc{nom}} \hspace{4pt} Ton~: M.
\textcolor{PineGreen}{\selectlanguage{french}Poitrine.} \zh{胸脯、胸膛。}  \zh{量词}: \textcolor{darkblue}{\textbf{\ipa{ʈv̩˩}}}  \mytextsc{clf}~: \textcolor{darkblue}{\textbf{\ipa{ʈv̩˩}}} 
\lhead{\firstmark}
\rhead{\botmark}

\subsection{\hspace{-0.5cm} {\Large \textcolor{darkblue}{\textbf{\ipa{ʁɑ˧pv̩˧-ɻ̃\#˥}}}}\hspace{0.5cm}[\kern2pt{\textcolor{darkblue}{\textbf{\ipa{xxxx non-correspondance entre le nombre de morphèmes et le nombre de tons de morphèmes}}}}\kern2pt]} \hypertarget{RA\string_Mpv\string_=\string_M-r£`\string_~\#\string_T1}{}
\markboth{\textcolor{darkblue}{\textbf{\ipa{ʁɑ˧pv̩˧-ɻ̃\#˥}}}}{}
\textcolor{teal}{\mytextsc{nom}} \hspace{4pt} Ton~: \#H.
\textcolor{PineGreen}{\selectlanguage{french}Clavicule.} \zh{锁骨。}  \zh{量词}: \textcolor{darkblue}{\textbf{\ipa{pʰæ˧˥}}}  \mytextsc{clf}~: \textcolor{darkblue}{\textbf{\ipa{pʰæ˧˥}}} 
\lhead{\firstmark}
\rhead{\botmark}

\subsection{\hspace{-0.5cm} {\Large \textcolor{darkblue}{\textbf{\ipa{ʁɑ˧pʰv̩\#˥}}}}\hspace{0.5cm}[\kern2pt{\textcolor{darkblue}{\textbf{\ipa{ʁɑ˧pʰv̩˧}}}}\kern2pt]} \hypertarget{RA\string_Mp\string_hv\string_=\#\string_T1}{}
\markboth{\textcolor{darkblue}{\textbf{\ipa{ʁɑ˧pʰv̩\#˥}}}}{}
\textcolor{teal}{\mytextsc{nom}} \hspace{4pt} Ton~: \#H.
\textcolor{PineGreen}{\selectlanguage{french}Salaire, littéralement “prix du travail”.} \zh{工资, 工钱。}  \zh{量词}: \textcolor{darkblue}{\textbf{\ipa{kʰwɤ˥}}}  \mytextsc{clf}~: \textcolor{darkblue}{\textbf{\ipa{kʰwɤ˥}}} 
\lhead{\firstmark}
\rhead{\botmark}

\subsection{\hspace{-0.5cm} {\Large \textcolor{darkblue}{\textbf{\ipa{ʁɑ˧-ʐwɤ˧˥}}}}\hspace{0.5cm}[\kern2pt{\textcolor{darkblue}{\textbf{\ipa{xxxx non-correspondance entre le nombre de morphèmes et le nombre de tons de morphèmes}}}}\kern2pt]} \hypertarget{RA\string_M-z`w7\string_M\string_T1}{}
\markboth{\textcolor{darkblue}{\textbf{\ipa{ʁɑ˧-ʐwɤ˧˥}}}}{}
\textcolor{teal}{\mytextsc{verbe}} \hspace{4pt} Ton~: MH\#.
\textcolor{PineGreen}{\selectlanguage{french}Provoquer/humilier.} \zh{欺负。}  ¶ \textcolor{darkblue}{\textbf{\ipa{ʈʂʰɯ˧-v̩˧ | hĩ˧-ki˧ ʁɑ˧-ʐwɤ˧-ʝi˥!}}} \textcolor{PineGreen}{\selectlanguage{french}elle/il provoque/humilie quelqu'un d'autre} \zh{他欺负人、他对人发脾气}  
 ¶ \textcolor{darkblue}{\textbf{\ipa{ʁɑ˧ ʐwɤ˧-ɻ̍˥}}} \textcolor{PineGreen}{\selectlanguage{french}même sens} \zh{同上}  
 ¶ \textcolor{darkblue}{\textbf{\ipa{no˧ | ʁɑ˧ ʐwɤ˧-tʰɑ˧-ɻ̍˥!}}} \textcolor{PineGreen}{\selectlanguage{french}Ne provoque pas les gens!} \zh{你不要欺负人!}  

\lhead{\firstmark}
\rhead{\botmark}

\subsection{\hspace{-0.5cm} {\Large \textcolor{darkblue}{\textbf{\ipa{ʁɑ˩mi˥}}}}\hspace{0.5cm}[\kern2pt{\textcolor{darkblue}{\textbf{\ipa{ʁɑ˩mi˥}}}}\kern2pt]} \hypertarget{RA\string_Bmi\string_T1}{}
\markboth{\textcolor{darkblue}{\textbf{\ipa{ʁɑ˩mi˥}}}}{}
\textcolor{teal}{\mytextsc{verbe}} \hspace{4pt} Ton~: LH.
\textcolor{PineGreen}{\selectlanguage{french}Demander pardon; formule de requête, et de remerciement. Le spectre des significations rappelle l'étymologie de “merci”: de “crier merci” (implorer la vie sauve) à un emploi comme formule de politesse courante.} \zh{道歉。}  ¶ \textcolor{darkblue}{\textbf{\ipa{ʁɑ˩mi˥-ze˩!}}} \textcolor{PineGreen}{\selectlanguage{french}Merci!} \zh{谢谢!}  

\lhead{\firstmark}
\rhead{\botmark}

\subsection{\hspace{-0.5cm} {\Large \textcolor{darkblue}{\textbf{\ipa{ʁɑ˩ʂɯ˧}}}}\hspace{0.5cm}[\kern2pt{\textcolor{darkblue}{\textbf{\ipa{ʁɑ˩ʂɯ˥}}}}\kern2pt]} \hypertarget{RA\string_Bs`M\string_M1}{}
\markboth{\textcolor{darkblue}{\textbf{\ipa{ʁɑ˩ʂɯ˧}}}}{}
\textcolor{teal}{\mytextsc{adverbe}} \hspace{4pt} Ton~: LM.
\textcolor{PineGreen}{\selectlanguage{french}En fait, en réalité.} \zh{其实、事实上。} 
\lhead{\firstmark}
\rhead{\botmark}

\subsection{\hspace{-0.5cm} {\Large \textcolor{darkblue}{\textbf{\ipa{ʁæ˥}}}}\hspace{0.5cm}[\kern2pt{\textcolor{darkblue}{\textbf{\ipa{ʁæ˥}}}}\kern2pt]} \hypertarget{R\{\string_T1}{}
\markboth{\textcolor{darkblue}{\textbf{\ipa{ʁæ˥}}}}{}
\textcolor{teal}{\mytextsc{nom}} \hspace{4pt} Ton~: \#H.
\textcolor{PineGreen}{\selectlanguage{french}Cou (monosyllabe; moins usité que le disyllabe).} \zh{脖子(单音节)。}  \zh{量词}: \textcolor{darkblue}{\textbf{\ipa{ɭɯ˧}}}  \mytextsc{clf}~: \textcolor{darkblue}{\textbf{\ipa{ɭɯ˧}}} \textit{Voir~:} \hyperlink{}{\textcolor{darkblue}{\textbf{\ipa{ʁæ˧ŋv̩˥}}}} 
\lhead{\firstmark}
\rhead{\botmark}

\subsection{\hspace{-0.5cm} {\Large \textcolor{darkblue}{\textbf{\ipa{ʁæ˧}}}}\hspace{0.5cm}[\kern2pt{\textcolor{darkblue}{\textbf{\ipa{ʁæ˥}}}}\kern2pt]} \hypertarget{R\{\string_M1}{}
\markboth{\textcolor{darkblue}{\textbf{\ipa{ʁæ˧}}}}{}
\textcolor{teal}{\mytextsc{adjectif}} \hspace{4pt} Ton~: M.
\textcolor{PineGreen}{\selectlanguage{french}Riche.} \zh{富。} 
\lhead{\firstmark}
\rhead{\botmark}

\subsection{\hspace{-0.5cm} {\Large \textcolor{darkblue}{\textbf{\ipa{ʁæ˧bæ˧}}}}\hspace{0.5cm}[\kern2pt{\textcolor{darkblue}{\textbf{\ipa{ʁæ˧bæ˧}}}}\kern2pt]} \hypertarget{R\{\string_Mb\{\string_M1}{}
\markboth{\textcolor{darkblue}{\textbf{\ipa{ʁæ˧bæ˧}}}}{}
\textcolor{teal}{\mytextsc{nom}} \hspace{4pt} Ton~: M.
\textcolor{PineGreen}{\selectlanguage{french}Assiette.} \zh{盘子。}  \zh{量词}: \textcolor{darkblue}{\textbf{\ipa{ɭɯ˧}}}  \mytextsc{clf}~: \textcolor{darkblue}{\textbf{\ipa{ɭɯ˧}}} 
\lhead{\firstmark}
\rhead{\botmark}

\subsection{\hspace{-0.5cm} {\Large \textcolor{darkblue}{\textbf{\ipa{ʁæ˧ɭɯ˥}}}}\hspace{0.5cm}[\kern2pt{\textcolor{darkblue}{\textbf{\ipa{ʁæ˧ɭɯ˥}}}}\kern2pt]} \hypertarget{R\{\string_Ml\string_RM\string_T1}{}
\markboth{\textcolor{darkblue}{\textbf{\ipa{ʁæ˧ɭɯ˥}}}}{}
\textcolor{teal}{\mytextsc{nom}} \hspace{4pt} Ton~: H\#.
\textcolor{PineGreen}{\selectlanguage{french}Carcan (était en bois); littéralement “[objet dans lequel] on met le cou”.} \zh{枷锁(木头做的)。}  ¶ \textcolor{darkblue}{\textbf{\ipa{ʁæ˧ɭɯ˥ | ɖɯ˧-ɭɯ˧ kʰɯ˧˥}}} \textcolor{PineGreen}{\selectlanguage{french}mettre un joug (à quelqu'un)} \zh{套上一个枷锁(在一个人的脖子上)}  
 ¶ \textcolor{darkblue}{\textbf{\ipa{ʁæ˧ɭɯ˥ kʰɯ˩}}} \textcolor{PineGreen}{\selectlanguage{french}mettre le joug (à quelqu'un)} \zh{套上枷锁(在一个人的脖子上)}  

\lhead{\firstmark}
\rhead{\botmark}

\subsection{\hspace{-0.5cm} {\Large \textcolor{darkblue}{\textbf{\ipa{ʁæ˧mi˧}}}}\hspace{0.5cm}[\kern2pt{\textcolor{darkblue}{\textbf{\ipa{ʁæ˧mi˧}}}}\kern2pt]} \hypertarget{R\{\string_Mmi\string_M1}{}
\markboth{\textcolor{darkblue}{\textbf{\ipa{ʁæ˧mi˧}}}}{}
\textcolor{teal}{\mytextsc{nom}} \hspace{4pt} Ton~: M.
\textcolor{PineGreen}{\selectlanguage{french}Épée.} \zh{剑。}  \zh{量词}: \textcolor{darkblue}{\textbf{\ipa{nɑ˧}}}  \mytextsc{clf}~: \textcolor{darkblue}{\textbf{\ipa{nɑ˧}}} 
\lhead{\firstmark}
\rhead{\botmark}

\subsection{\hspace{-0.5cm} {\Large \textcolor{darkblue}{\textbf{\ipa{ʁæ˧ŋv̩˥}}}}\hspace{0.5cm}[\kern2pt{\textcolor{darkblue}{\textbf{\ipa{ʁæ˧ŋv̩˥}}}}\kern2pt]} \hypertarget{R\{\string_MNv\string_=\string_T1}{}
\markboth{\textcolor{darkblue}{\textbf{\ipa{ʁæ˧ŋv̩˥}}}}{}
\textcolor{teal}{\mytextsc{nom}} \hspace{4pt} Ton~: H\#.
\textcolor{PineGreen}{\selectlanguage{french}Col (précieux, avec des fils d'argent).} \zh{银衣领。}  \zh{量词}: \textcolor{darkblue}{\textbf{\ipa{ɭɯ˧}}}  \mytextsc{clf}~: \textcolor{darkblue}{\textbf{\ipa{ɭɯ˧}}} 
\lhead{\firstmark}
\rhead{\botmark}

\subsection{\hspace{-0.5cm} {\Large \textcolor{darkblue}{\textbf{\ipa{ʁæ˧ɻ̍˥}}}}\hspace{0.5cm}[\kern2pt{\textcolor{darkblue}{\textbf{\ipa{ʁæ˧ɻ̍˥}}}}\kern2pt]} \hypertarget{R\{\string_Mr£`̍\string_T1}{}
\markboth{\textcolor{darkblue}{\textbf{\ipa{ʁæ˧ɻ̍˥}}}}{}
\textcolor{teal}{\mytextsc{nom}} \hspace{4pt} Ton~: H\#.
\textcolor{PineGreen}{\selectlanguage{french}Cou.} \zh{脖子。}  \zh{量词}: \textcolor{darkblue}{\textbf{\ipa{ɭɯ˧}}}  \mytextsc{clf}~: \textcolor{darkblue}{\textbf{\ipa{ɭɯ˧}}} \textit{Voir~:} \hyperlink{}{\textcolor{darkblue}{\textbf{\ipa{ʁæ˧ʈv̩˥}}}} 
\lhead{\firstmark}
\rhead{\botmark}

\subsection{\hspace{-0.5cm} {\Large \textcolor{darkblue}{\textbf{\ipa{ʁæ˧tɑ˩}}}}\hspace{0.5cm}[\kern2pt{\textcolor{darkblue}{\textbf{\ipa{ʁæ˧tɑ˩}}}}\kern2pt]} \hypertarget{R\{\string_MtA\string_B1}{}
\markboth{\textcolor{darkblue}{\textbf{\ipa{ʁæ˧tɑ˩}}}}{}
\textcolor{teal}{\mytextsc{nom}} \hspace{4pt} Ton~: L\#.
\textcolor{PineGreen}{\selectlanguage{french}Garrot: partie du corps de l'animal sur lequel repose le joug.} \zh{肩隆。}  ¶ \textcolor{darkblue}{\textbf{\ipa{ʝi˧-ʁæ˧tɑ˥}}} \textcolor{PineGreen}{\selectlanguage{french}garrot de vache} \zh{牛肩隆}  
 ¶ \textcolor{darkblue}{\textbf{\ipa{ʁæ˧tɑ˩ tʰv̩˩-ɭɯ˩}}} \textcolor{PineGreen}{\selectlanguage{french}\mytextsc{n}+\mytextsc{dem}+\mytextsc{clf}} \zh{这只肩隆}  
 \zh{量词}: \textcolor{darkblue}{\textbf{\ipa{ɭɯ˧}}}  \mytextsc{clf}~: \textcolor{darkblue}{\textbf{\ipa{ɭɯ˧}}} 
\lhead{\firstmark}
\rhead{\botmark}

\subsection{\hspace{-0.5cm} {\Large \textcolor{darkblue}{\textbf{\ipa{ʁæ˧ʈv̩˥}}}}\hspace{0.5cm}[\kern2pt{\textcolor{darkblue}{\textbf{\ipa{ʁæ˧ʈv̩˥}}}}\kern2pt]} \hypertarget{R\{\string_Mt`v\string_=\string_T1}{}
\markboth{\textcolor{darkblue}{\textbf{\ipa{ʁæ˧ʈv̩˥}}}}{}
\textcolor{teal}{\mytextsc{nom}} \hspace{4pt} Ton~: H\#.
\textcolor{PineGreen}{\selectlanguage{french}Cou.} \zh{脖子。}  \zh{量词}: \textcolor{darkblue}{\textbf{\ipa{ɭɯ˧}}}  \mytextsc{clf}~: \textcolor{darkblue}{\textbf{\ipa{ɭɯ˧}}} \textit{Voir~:} \hyperlink{}{\textcolor{darkblue}{\textbf{\ipa{ʁæ˧ɻ̍˥}}}} 
\lhead{\firstmark}
\rhead{\botmark}

\subsection{\hspace{-0.5cm} {\Large \textcolor{darkblue}{\textbf{\ipa{ʁæ˧zo\#˥}}}}\hspace{0.5cm}[\kern2pt{\textcolor{darkblue}{\textbf{\ipa{ʁæ˧zo˧}}}}\kern2pt]} \hypertarget{R\{\string_Mzo\#\string_T1}{}
\markboth{\textcolor{darkblue}{\textbf{\ipa{ʁæ˧zo\#˥}}}}{}
\textcolor{teal}{\mytextsc{nom}} \hspace{4pt} Ton~: \#H.
\textcolor{PineGreen}{\selectlanguage{french}Petite épée.} \zh{短剑。} 
\lhead{\firstmark}
\rhead{\botmark}

\subsection{\hspace{-0.5cm} {\Large \textcolor{darkblue}{\textbf{\ipa{ʁæ˧ʑi˧}}}}\hspace{0.5cm}[\kern2pt{\textcolor{darkblue}{\textbf{\ipa{ʁæ˧ʑi˧}}}}\kern2pt]} \hypertarget{R\{\string_Mz£i\string_M1}{}
\markboth{\textcolor{darkblue}{\textbf{\ipa{ʁæ˧ʑi˧}}}}{}
\textcolor{teal}{\mytextsc{verbe}} \hspace{4pt} Ton~: M.
\textcolor{PineGreen}{\selectlanguage{french}S'occuper de; se mêler de; prendre en considération.} \zh{考虑。}  ¶ \textcolor{darkblue}{\textbf{\ipa{hĩ˧ | qʰɑ˧-kv̩˧ dʑo˧˥ | mɤ˧-ʁæ˧ʑi˧, | njɤ˧-ɳɯ˧ qʰæ˧˥! |}}} \textcolor{PineGreen}{\selectlanguage{french}Moi, j'aide, sans m'inquiéter de savoir combien il y a d'invités (littéralement “de gens”)! (contexte: F4 explique comment on se dévouait autrefois pour aider les amis, non membres de la famille, lors des grandes occasions, telles que les funérailles)} \zh{无论有多少个人,我都会去帮助!(情景:合作人描写她在永宁有大事时怎么去帮其它家庭的忙,不考虑活多么累,只考虑怎么能给予帮助)}  
 ¶ \textcolor{darkblue}{\textbf{\ipa{no˧ | mɤ˧-ʁæ˧ʑi˧!}}} \textcolor{PineGreen}{\selectlanguage{french}Fiche-moi la paix! / Laisse-moi tranquille! / Mêle-toi de tes affaires!} \zh{别管我了! / 请让我安静! / 请不要打扰我了!}  

\lhead{\firstmark}
\rhead{\botmark}

\subsection{\hspace{-0.5cm} {\Large \textcolor{darkblue}{\textbf{\ipa{ʁæ˩\textsubscript{a}}}} \textsubscript{1}}\hspace{0.5cm}[\kern2pt{\textcolor{darkblue}{\textbf{\ipa{ʁæ˩˥}}}}\kern2pt]} \hypertarget{R\{\string_Ba1}{}
\markboth{\textcolor{darkblue}{\textbf{\ipa{ʁæ˩\textsubscript{a}}}} \textsubscript{1}}{}
\textcolor{teal}{\mytextsc{verbe}} \hspace{4pt} Ton~: L\textsubscript{a}.
\textcolor{PineGreen}{\selectlanguage{french}Se défaire, fondre, se dissoudre: une motte de terre plongée dans l'eau se défait.} \zh{散、散开,化,溶化(一块土在水里面散开)。}  ¶ \textcolor{darkblue}{\textbf{\ipa{le˧-ʁæ˩-ze˩}}} \textcolor{PineGreen}{\selectlanguage{french}\mytextsc{accomp} \string_ \mytextsc{pfv}} \zh{\mytextsc{accomp} \string_ \mytextsc{pfv}}  
 ¶ \textcolor{darkblue}{\textbf{\ipa{le˧-ʁæ˧\textasciitilde{}ʁæ˥ (-ze˩ / -bi˩)}}} \textcolor{PineGreen}{\selectlanguage{french}\mytextsc{red}} \zh{\mytextsc{重叠}}  
 ¶ \textcolor{darkblue}{\textbf{\ipa{ɖɯ˧-kʰwɤ˧ ʁæ˥}}} \textcolor{PineGreen}{\selectlanguage{french}un morceau (de terre) se défait} \zh{一块(土)散开}  
 ¶ \textcolor{darkblue}{\textbf{\ipa{ʈʂe˧ʈv̩˥ | le˧-ʁæ˩-ze˩}}} \textcolor{PineGreen}{\selectlanguage{french}les mottes de terre se défond, se dissolvent (dans l'eau dont on inonde les champs après les labours)} \zh{土块散开在了(耕田后灌溉,土块散在水里)}  

\lhead{\firstmark}
\rhead{\botmark}

\subsection{\hspace{-0.5cm} {\Large \textcolor{darkblue}{\textbf{\ipa{ʁæ˩\textsubscript{a}}}} \textsubscript{2}}\hspace{0.5cm}[\kern2pt{\textcolor{darkblue}{\textbf{\ipa{ʁæ˩˥}}}}\kern2pt]} \hypertarget{R\{\string_Ba2}{}
\markboth{\textcolor{darkblue}{\textbf{\ipa{ʁæ˩\textsubscript{a}}}} \textsubscript{2}}{}
\textcolor{teal}{\mytextsc{adjectif}} \hspace{4pt} Ton~: L\textsubscript{a}.
\textcolor{PineGreen}{\selectlanguage{french}Ivre, saoul.} \zh{醉。}  ¶ \textcolor{darkblue}{\textbf{\ipa{ʐɯ˧ ʁæ˩(-ze˩)}}} \textcolor{PineGreen}{\selectlanguage{french}ivre d'alcool} \zh{醉酒}  

\lhead{\firstmark}
\rhead{\botmark}

\subsection{\hspace{-0.5cm} {\Large \textcolor{darkblue}{\textbf{\ipa{ʁæ˩\textsubscript{a}}}} \textsubscript{3}}\hspace{0.5cm}[\kern2pt{\textcolor{darkblue}{\textbf{\ipa{ʁæ˩˥}}}}\kern2pt]} \hypertarget{R\{\string_Ba3}{}
\markboth{\textcolor{darkblue}{\textbf{\ipa{ʁæ˩\textsubscript{a}}}} \textsubscript{3}}{}
\textcolor{teal}{\mytextsc{adjectif}} \hspace{4pt} Ton~: L\textsubscript{a}.
\textcolor{PineGreen}{\selectlanguage{french}Approprié; propice, favorable.} \zh{合适,吉利。}  ¶ \textcolor{darkblue}{\textbf{\ipa{ʁæ˧ mɤ˧-ʑi˧}}} \textcolor{PineGreen}{\selectlanguage{french}ce n'est pas propice/favorable} \zh{不吉利、不合适}  
 ¶ \textcolor{darkblue}{\textbf{\ipa{ʁæ˧ mɤ˧-ʑi˧, | ʝi˧ mɤ˧-tʰɑ˩! / ʝi˧-mɤ˧-ɖo˧!}}} \textcolor{PineGreen}{\selectlanguage{french}les circonstances ne sont pas propices / ce n'est pas une bonne idée, il ne faut pas le faire! (Mise en garde)} \zh{不吉利 / 不合适(的事情),不能做!/ 不要做!(警告)}  
 ¶ \textcolor{darkblue}{\textbf{\ipa{ʁæ˧ mɤ˧-ʑi˧, | ʐwɤ˩ mɤ˩-tʰɑ˥! / ʁæ˧ mɤ˧-ʑi˧, | ʐwɤ˩ mɤ˩-ɖo˩˥!}}} \textcolor{PineGreen}{\selectlanguage{french}Ca ne convient pas; on ne peut pas le dire / on ne doit pas le dire! / Il ne faut pas tenir de propos inappropriés! / Il faut faire attention à ce qu'on dit! (Mise en garde)} \zh{不合适(的话),不能说! / 不合适(的话),不要说!(警告)}  
 ¶ \textcolor{darkblue}{\textbf{\ipa{ʁæ˧-mɤ˧-ʑi˧, | tɕi˩-mɤ˩-ɖo˩˥!}}} \textcolor{PineGreen}{\selectlanguage{french}Il ne faut pas transcrire ceux qui sont pas bons! Il ne faut pas écrire n'importe quoi! (Contexte: cette phrase récapitule le principe qui préside au choix des récits à transcrire et traduire. J'expliquais de mon mieux mon souhait de choisir, parmi les récits enregistrés --relativement nombreux--, ceux qui sont les plus intéressants, et les plus réussis. Par cet énoncé, la locutrice apporte son assentiment, en même temps qu'elle indique qu'elle comprend l'idée: il faut transcrire les récits qui sont bons; il faut bien choisir, et écarter ceux qui ne seraient pas appropriés en quoi que ce soit. Par le même énoncé, la locutrice témoigne en outre de sa modestie: elle ne défend pas l'idée selon laquelle tous ses récits sont d'égale qualité, et accepte de bonne grâce l'idée que certains sont plus réussis que d'autres, ou plus adéquats pour le propos du linguiste.)} \zh{乱七八糟的,不要记录! / 不好的,不要记录!(情景:选择一个故事来做记音翻译等。合作人提出,要考虑好记录哪些、选择好的资料,不能什么都记录。)}  
 ¶ \textcolor{darkblue}{\textbf{\ipa{ʈʂʰɯ˧ | lo˧ | ʁæ˧-mɤ˧-ʑi˧ ʝi˧!}}} \textcolor{PineGreen}{\selectlanguage{french}Il ne fait pas attention dans son travail! il ne travaille pas avec soin! il fait n'importe quoi!} \zh{他工作做得乱七八糟!}  

\lhead{\firstmark}
\rhead{\botmark}

\subsection{\hspace{-0.5cm} {\Large \textcolor{darkblue}{\textbf{\ipa{ʁæ˩\textsubscript{a}}}} \textsubscript{4}}\hspace{0.5cm}[\kern2pt{\textcolor{darkblue}{\textbf{\ipa{ʁæ˩˥}}}}\kern2pt]} \hypertarget{R\{\string_Ba4}{}
\markboth{\textcolor{darkblue}{\textbf{\ipa{ʁæ˩\textsubscript{a}}}} \textsubscript{4}}{}
\textcolor{teal}{\mytextsc{adjectif}} \hspace{4pt} Ton~: L\textsubscript{a}.
\textcolor{PineGreen}{\selectlanguage{french}Laid, vilain.} \zh{丑陋。} \textit{Voir~:} \hyperlink{}{\textcolor{darkblue}{\textbf{\ipa{ɖʐv̩˩\textsubscript{a}}}} \textsubscript{1}} 
\lhead{\firstmark}
\rhead{\botmark}

\subsection{\hspace{-0.5cm} {\Large \textcolor{darkblue}{\textbf{\ipa{ʁæ˧˥}}}}\hspace{0.5cm}[\kern2pt{\textcolor{darkblue}{\textbf{\ipa{ʁæ˧˥}}}}\kern2pt]} \hypertarget{R\{\string_M\string_T1}{}
\markboth{\textcolor{darkblue}{\textbf{\ipa{ʁæ˧˥}}}}{}
\textcolor{teal}{\mytextsc{adjectif}} \hspace{4pt} Ton~: MH.
\textcolor{PineGreen}{\selectlanguage{french}Écoeurant, dégoûtant, pas bon au goût (pas forcément à cause d'un excès de graisse: par exemple, selon les critères gastronomiques locaux, mes flocons d'avoine entrent dans cette catégorie).} \zh{不好吃,恶心。} 
\lhead{\firstmark}
\rhead{\botmark}

\subsection{\hspace{-0.5cm} {\Large \textcolor{darkblue}{\textbf{\ipa{ʁæ˩˥}}}}\hspace{0.5cm}[\kern2pt{\textcolor{darkblue}{\textbf{\ipa{ʁæ˩˥}}}}\kern2pt]} \hypertarget{R\{\string_B\string_T1}{}
\markboth{\textcolor{darkblue}{\textbf{\ipa{ʁæ˩˥}}}}{}
\textcolor{teal}{\mytextsc{nom}} \hspace{4pt} Ton~: LH.
\textcolor{PineGreen}{\selectlanguage{french}Sève, résine.} \zh{树液。}  ¶ \textcolor{darkblue}{\textbf{\ipa{tʰo˩ʁæ˩˥}}} \textcolor{PineGreen}{\selectlanguage{french}même sens} \zh{同上}  

\lhead{\firstmark}
\rhead{\botmark}

\subsection{\hspace{-0.5cm} {\Large \textcolor{darkblue}{\textbf{\ipa{ʁo˥}}}}\hspace{0.5cm}[\kern2pt{\textcolor{darkblue}{\textbf{\ipa{ʁo˥}}}}\kern2pt]} \hypertarget{Ro\string_T1}{}
\markboth{\textcolor{darkblue}{\textbf{\ipa{ʁo˥}}}}{}
\textcolor{teal}{\mytextsc{nom}} \hspace{4pt} Ton~: \#H.
\ding{202} \textcolor{PineGreen}{\selectlanguage{french}Tête (monosyllabique).} \zh{头(单音节)。}  \zh{量词}: \textcolor{darkblue}{\textbf{\ipa{ɭɯ˧}}} \ding{203} \textcolor{PineGreen}{\selectlanguage{french}Début.} \zh{开头。}  ¶ \textcolor{darkblue}{\textbf{\ipa{ɬi˧-ʁo\#˥}}} \textcolor{PineGreen}{\selectlanguage{french}le début du mois} \zh{月初}  
 ¶ \textcolor{darkblue}{\textbf{\ipa{kʰv̩˧-ʁo˥\$}}} \textcolor{PineGreen}{\selectlanguage{french}le début de l'année} \zh{年初}  
 ¶ \textcolor{darkblue}{\textbf{\ipa{*ɲi˧-ʁo˩}}} \textcolor{PineGreen}{\selectlanguage{french}*le début de la journée} \zh{*天初}  
 \mytextsc{clf}~: \textcolor{darkblue}{\textbf{\ipa{ɭɯ˧}}} 
\lhead{\firstmark}
\rhead{\botmark}

\subsection{\hspace{-0.5cm} {\Large \textcolor{darkblue}{\textbf{\ipa{ʁo˥-ʐv̩˩}}}}\hspace{0.5cm}[\kern2pt{\textcolor{darkblue}{\textbf{\ipa{xxxx non-correspondance entre le nombre de morphèmes et le nombre de tons de morphèmes}}}}\kern2pt]} \hypertarget{Ro\string_T-z`v\string_=\string_B1}{}
\markboth{\textcolor{darkblue}{\textbf{\ipa{ʁo˥-ʐv̩˩}}}}{}
\textcolor{teal}{\mytextsc{verbe}} \hspace{4pt} Ton~: .
\textcolor{PineGreen}{\selectlanguage{french}Bénir et protéger.} \zh{保佑。}  ¶ \textcolor{darkblue}{\textbf{\ipa{mɤ˧-ʁo˥ʐv̩˩}}} \textcolor{PineGreen}{\selectlanguage{french}\mytextsc{neg}} \zh{\mytextsc{neg}}  
 ¶ \textcolor{darkblue}{\textbf{\ipa{gɤ˧lɑ˧ | ɖɯ˧-ʁo˥ʐv̩˩-ɻ̍˩!}}} \textcolor{PineGreen}{\selectlanguage{french}Que les esprits [te/nous] bénissent!} \zh{菩萨保佑!}  

\lhead{\firstmark}
\rhead{\botmark}

\subsection{\hspace{-0.5cm} {\Large \textcolor{darkblue}{\textbf{\ipa{ʁo˧}}} \textsubscript{1}}\hspace{0.5cm}[\kern2pt{\textcolor{darkblue}{\textbf{\ipa{ʁo˥}}}}\kern2pt]} \hypertarget{Ro\string_M1}{}
\markboth{\textcolor{darkblue}{\textbf{\ipa{ʁo˧}}} \textsubscript{1}}{}
\textcolor{teal}{\mytextsc{verbe}} \hspace{4pt} Ton~: M intrans.
\textcolor{PineGreen}{\selectlanguage{french}Pondre.} \zh{下蛋。}  ¶ \textcolor{darkblue}{\textbf{\ipa{æ˩ ʁo˥}}} \textcolor{PineGreen}{\selectlanguage{french}pondre des œufs} \zh{下蛋}  
 ¶ \textcolor{darkblue}{\textbf{\ipa{æ˩mi˧ tʰi˧-ʁo˧-dʑo˧!}}} \textcolor{PineGreen}{\selectlanguage{french}la poule est en train de pondre!} \zh{母鸡在下蛋!}  
 ¶ \textcolor{darkblue}{\textbf{\ipa{æ˩mi˧ | æ˩ ʁo˧-ze˩!}}} \textcolor{PineGreen}{\selectlanguage{french}la poule a pondu!} \zh{母鸡下蛋了!}  

\lhead{\firstmark}
\rhead{\botmark}

\subsection{\hspace{-0.5cm} {\Large \textcolor{darkblue}{\textbf{\ipa{ʁo˧}}} \textsubscript{2}}\hspace{0.5cm}[\kern2pt{\textcolor{darkblue}{\textbf{\ipa{ʁo˥}}}}\kern2pt]} \hypertarget{Ro\string_M2}{}
\markboth{\textcolor{darkblue}{\textbf{\ipa{ʁo˧}}} \textsubscript{2}}{}
\textcolor{teal}{\mytextsc{verbe}} \hspace{4pt} Ton~: M intrans.
\textcolor{PineGreen}{\selectlanguage{french}Arriver à, parvenir à.} \zh{能……、有能力做。}  ¶ \textcolor{darkblue}{\textbf{\ipa{njɤ˧ | tɕi˩-mɤ˩-ʁo˩˥!}}} \textcolor{PineGreen}{\selectlanguage{french}je ne parviens pas à écrire/je ne sais pas écrire!} \zh{我写不出来! / 我不会写!}  
 ¶ \textcolor{darkblue}{\textbf{\ipa{njɤ˧ | tɕi˩-ʁo˩˥!}}} \textcolor{PineGreen}{\selectlanguage{french}je parviens à écrire/je sais écrire!} \zh{我会写! / 我写得出来!}  

\lhead{\firstmark}
\rhead{\botmark}

\subsection{\hspace{-0.5cm} {\Large \textcolor{darkblue}{\textbf{\ipa{ʁo˧bv̩˧}}}}\hspace{0.5cm}[\kern2pt{\textcolor{darkblue}{\textbf{\ipa{ʁo˩bv̩˩˥}}}}\kern2pt]} \hypertarget{Ro\string_Mbv\string_=\string_M1}{}
\markboth{\textcolor{darkblue}{\textbf{\ipa{ʁo˧bv̩˧}}}}{}
\textcolor{teal}{\mytextsc{nom}} \hspace{4pt} Ton~: M.
\textcolor{PineGreen}{\selectlanguage{french}Pousse d'arbre.} \zh{树的萌芽、新发出来的叶子。}  ¶ \textcolor{darkblue}{\textbf{\ipa{tʰo˧-ʁo˧bv˥}}} \textcolor{PineGreen}{\selectlanguage{french}pousse de sapin} \zh{小松树尖}  
 ¶ \textcolor{darkblue}{\textbf{\ipa{tʰo˩ʂv˩-ʁo˥bv˩}}} \textcolor{PineGreen}{\selectlanguage{french}pousse de sapin; littéralement “pousses d'aiguilles de sapin”} \zh{小松树尖}  
 \zh{量词}: \textcolor{darkblue}{\textbf{\ipa{kʰwɤ˥}}}  \mytextsc{clf}~: \textcolor{darkblue}{\textbf{\ipa{kʰwɤ˥}}} 
\lhead{\firstmark}
\rhead{\botmark}

\subsection{\hspace{-0.5cm} {\Large \textcolor{darkblue}{\textbf{\ipa{ʁo˧dɑ˧}}}}\hspace{0.5cm}[\kern2pt{\textcolor{darkblue}{\textbf{\ipa{ʁo˧dɑ˧}}}}\kern2pt]} \hypertarget{Ro\string_MdA\string_M1}{}
\markboth{\textcolor{darkblue}{\textbf{\ipa{ʁo˧dɑ˧}}}}{}
\textcolor{teal}{\mytextsc{adverbe}} \hspace{4pt} Ton~: M.
\textcolor{PineGreen}{\selectlanguage{french}Devant, avant, auparavant.} \zh{前面,之前。}  ¶ \textcolor{darkblue}{\textbf{\ipa{ʂɯ˧-kʰv̩˧-ʁo˧dɑ˧}}} \textcolor{PineGreen}{\selectlanguage{french}il y a sept ans} \zh{七年前}  
 ¶ \textcolor{darkblue}{\textbf{\ipa{ʁo˧dɑ˧ ɖɯ˧-so˩ ɲi˩}}} \textcolor{PineGreen}{\selectlanguage{french}ces derniers jours, les quelques jours passés, il y a quelques jours} \zh{前几天}  

\lhead{\firstmark}
\rhead{\botmark}

\subsection{\hspace{-0.5cm} {\Large \textcolor{darkblue}{\textbf{\ipa{ʁo˧do˧}}} \textsubscript{1}}\hspace{0.5cm}[\kern2pt{\textcolor{darkblue}{\textbf{\ipa{ʁo˩do˥}}}}\kern2pt]} \hypertarget{Ro\string_Mdo\string_M1}{}
\markboth{\textcolor{darkblue}{\textbf{\ipa{ʁo˧do˧}}} \textsubscript{1}}{}
\textcolor{teal}{\mytextsc{nom}} \hspace{4pt} Ton~: M.
\textcolor{PineGreen}{\selectlanguage{french}Noix.} \zh{核桃。}  ¶ \textcolor{darkblue}{\textbf{\ipa{ʁo˧do˧ qʰwæ˧˥}}} \textcolor{PineGreen}{\selectlanguage{french}casser des noix} \zh{开核桃}  
 ¶ \textcolor{darkblue}{\textbf{\ipa{ʁo˧do˧ ʐwæ˧}}} \textcolor{PineGreen}{\selectlanguage{french}peser des noix} \zh{称核桃}  
 \zh{量词}: \textcolor{darkblue}{\textbf{\ipa{ɭɯ˧}}}  \mytextsc{clf}~: \textcolor{darkblue}{\textbf{\ipa{ɭɯ˧}}} 
\lhead{\firstmark}
\rhead{\botmark}

\subsection{\hspace{-0.5cm} {\Large \textcolor{darkblue}{\textbf{\ipa{ʁo˧do˧}}} \textsubscript{2}}\hspace{0.5cm}[\kern2pt{\textcolor{darkblue}{\textbf{\ipa{ʁo˧do˧}}}}\kern2pt]} \hypertarget{Ro\string_Mdo\string_M2}{}
\markboth{\textcolor{darkblue}{\textbf{\ipa{ʁo˧do˧}}} \textsubscript{2}}{}
\textcolor{teal}{\mytextsc{nom}} \hspace{4pt} Ton~: M.
\textcolor{PineGreen}{\selectlanguage{french}Intérêts.} \zh{利息。}  \zh{量词}: \textcolor{darkblue}{\textbf{\ipa{kʰwɤ˥}}}  \mytextsc{clf}~: \textcolor{darkblue}{\textbf{\ipa{kʰwɤ˥}}} 
\lhead{\firstmark}
\rhead{\botmark}

\subsection{\hspace{-0.5cm} {\Large \textcolor{darkblue}{\textbf{\ipa{ʁo˧dzi˥}}}}\hspace{0.5cm}[\kern2pt{\textcolor{darkblue}{\textbf{\ipa{ʁo˧dzi˥}}}}\kern2pt]} \hypertarget{Ro\string_Mdzi\string_T1}{}
\markboth{\textcolor{darkblue}{\textbf{\ipa{ʁo˧dzi˥}}}}{}
\textcolor{teal}{\mytextsc{verbe}} \hspace{4pt} Ton~: H\#.
\textcolor{PineGreen}{\selectlanguage{french}Heurter.} \zh{碰撞。}  ¶ \textcolor{darkblue}{\textbf{\ipa{le˧-ʁo˧dzi˥}}} \textcolor{PineGreen}{\selectlanguage{french}\mytextsc{accomp}} \zh{\mytextsc{accomp}}  
 ¶ \textcolor{darkblue}{\textbf{\ipa{hĩ˧ | tʰi˧-ʁo˧dzi˥ tsʰɯ˩(-ze˩)}}} \textcolor{PineGreen}{\selectlanguage{french}deux personnes se sont heurtées/sont entrées en collision} \zh{人们(互相)碰撞}  

\lhead{\firstmark}
\rhead{\botmark}

\subsection{\hspace{-0.5cm} {\Large \textcolor{darkblue}{\textbf{\ipa{ʁo˧dzi˩}}}}\hspace{0.5cm}[\kern2pt{\textcolor{darkblue}{\textbf{\ipa{ʁo˧dzi˩}}}}\kern2pt]} \hypertarget{Ro\string_Mdzi\string_B1}{}
\markboth{\textcolor{darkblue}{\textbf{\ipa{ʁo˧dzi˩}}}}{}
\textcolor{teal}{\mytextsc{nom}} \hspace{4pt} Ton~: L\#.
\textcolor{PineGreen}{\selectlanguage{french}Tibétain.} \zh{藏族。}  \zh{量词}: \textcolor{darkblue}{\textbf{\ipa{v̩˧}}}  \mytextsc{clf}~: \textcolor{darkblue}{\textbf{\ipa{v̩˧}}} 
\lhead{\firstmark}
\rhead{\botmark}

\subsection{\hspace{-0.5cm} {\Large \textcolor{darkblue}{\textbf{\ipa{ʁo˧dzi˩-di˩}}}}\hspace{0.5cm}[\kern2pt{\textcolor{darkblue}{\textbf{\ipa{ʁo˧dzi˩di˧}}}}\kern2pt]} \hypertarget{Ro\string_Mdzi\string_B-di\string_B1}{}
\markboth{\textcolor{darkblue}{\textbf{\ipa{ʁo˧dzi˩-di˩}}}}{}
\textcolor{teal}{\mytextsc{nom}} \hspace{4pt} Ton~: L\#-.
\textcolor{PineGreen}{\selectlanguage{french}Le Tibet (littéralement: 'la contrée des Tibétains').} \zh{西藏。} 
\lhead{\firstmark}
\rhead{\botmark}

\subsection{\hspace{-0.5cm} {\Large \textcolor{darkblue}{\textbf{\ipa{ʁo˧dzi˩-tʰæ˩ɻæ˩}}}}\hspace{0.5cm}[\kern2pt{\textcolor{darkblue}{\textbf{\ipa{ʁo˧dzi˩tʰæ˧ɻæ˧}}}}\kern2pt]} \hypertarget{Ro\string_Mdzi\string_B-t\string_h\{\string_Br£`\{\string_B1}{}
\markboth{\textcolor{darkblue}{\textbf{\ipa{ʁo˧dzi˩-tʰæ˩ɻæ˩}}}}{}
\textcolor{teal}{\mytextsc{nom}} \hspace{4pt} Ton~: L\#-.
\textcolor{PineGreen}{\selectlanguage{french}Drapeau, fanion (littéralement “écritures tibétaines”).} \zh{旗子。}  \zh{量词}: \textcolor{darkblue}{\textbf{\ipa{pʰæ˧˥}}}  \mytextsc{clf}~: \textcolor{darkblue}{\textbf{\ipa{pʰæ˧˥}}} 
\lhead{\firstmark}
\rhead{\botmark}

\subsection{\hspace{-0.5cm} {\Large \textcolor{darkblue}{\textbf{\ipa{ʁo˧dʑɯ˧}}}}\hspace{0.5cm}[\kern2pt{\textcolor{darkblue}{\textbf{\ipa{ʁo˧dʑɯ˧}}}}\kern2pt]} \hypertarget{Ro\string_Mdz£M\string_M1}{}
\markboth{\textcolor{darkblue}{\textbf{\ipa{ʁo˧dʑɯ˧}}}}{}
\textcolor{teal}{\mytextsc{nom}} \hspace{4pt} Ton~: M.
\textcolor{PineGreen}{\selectlanguage{french}Œillères.} \zh{马笼头。}  ¶ \textcolor{darkblue}{\textbf{\ipa{ʐwæ˧-ʁo˧dʑɯ˥ (ʈʂʰɯ˧ | ʐwæ˧-ʁo˧dʑɯ˥ ɲi˩)}}} \textcolor{PineGreen}{\selectlanguage{french}œillères de cheval} \zh{马笼头}  
 \zh{量词}: \textcolor{darkblue}{\textbf{\ipa{nɑ˧}}} \textcolor{darkblue}{\textbf{\ipa{pɤ˩}}}  \mytextsc{clf}~: \textcolor{darkblue}{\textbf{\ipa{nɑ˧}}} \textcolor{darkblue}{\textbf{\ipa{pɤ˩}}} 
\lhead{\firstmark}
\rhead{\botmark}

\subsection{\hspace{-0.5cm} {\Large \textcolor{darkblue}{\textbf{\ipa{ʁo˧ɖɯ˧˥}}}}\hspace{0.5cm}[\kern2pt{\textcolor{darkblue}{\textbf{\ipa{ʁo˧ɖɯ˧}}}}\kern2pt]} \hypertarget{Ro\string_Md`M\string_M\string_T1}{}
\markboth{\textcolor{darkblue}{\textbf{\ipa{ʁo˧ɖɯ˧˥}}}}{}
\textcolor{teal}{\mytextsc{nom}} \hspace{4pt} Ton~: MH\#.
\textcolor{PineGreen}{\selectlanguage{french}Têtard.} \zh{蝌蚪。} 
\lhead{\firstmark}
\rhead{\botmark}

\subsection{\hspace{-0.5cm} {\Large \textcolor{darkblue}{\textbf{\ipa{ʁo˧gv̩\#˥}}}}\hspace{0.5cm}[\kern2pt{\textcolor{darkblue}{\textbf{\ipa{ʁo˧gv̩˧}}}}\kern2pt]} \hypertarget{Ro\string_Mgv\string_=\#\string_T1}{}
\markboth{\textcolor{darkblue}{\textbf{\ipa{ʁo˧gv̩\#˥}}}}{}
\textcolor{teal}{\mytextsc{nom}} \hspace{4pt} Ton~: \#H.
\textcolor{PineGreen}{\selectlanguage{french}Oreiller.} \zh{枕头。}  \zh{量词}: \textcolor{darkblue}{\textbf{\ipa{ɭɯ˧}}}  \mytextsc{clf}~: \textcolor{darkblue}{\textbf{\ipa{ɭɯ˧}}} 
\lhead{\firstmark}
\rhead{\botmark}

\subsection{\hspace{-0.5cm} {\Large \textcolor{darkblue}{\textbf{\ipa{ʁo˧hṽ˧˥}}}}\hspace{0.5cm}[\kern2pt{\textcolor{darkblue}{\textbf{\ipa{ʁo˧hṽ˧˥}}}}\kern2pt]} \hypertarget{Ro\string_Mhv\string_~\string_M\string_T1}{}
\markboth{\textcolor{darkblue}{\textbf{\ipa{ʁo˧hṽ˧˥}}}}{}
\textcolor{teal}{\mytextsc{nom}} \hspace{4pt} Ton~: MH\#.
\textcolor{PineGreen}{\selectlanguage{french}Cheveux.} \zh{头发。}  \zh{量词}: \textcolor{darkblue}{\textbf{\ipa{kʰɯ˩}}}  \mytextsc{clf}~: \textcolor{darkblue}{\textbf{\ipa{kʰɯ˩}}} 
\lhead{\firstmark}
\rhead{\botmark}

\subsection{\hspace{-0.5cm} {\Large \textcolor{darkblue}{\textbf{\ipa{ʁo˧ʝi˧}}}}\hspace{0.5cm}[\kern2pt{\textcolor{darkblue}{\textbf{\ipa{ʁo˧ʝi˧}}}}\kern2pt]} \hypertarget{Ro\string_Mj££i\string_M1}{}
\markboth{\textcolor{darkblue}{\textbf{\ipa{ʁo˧ʝi˧}}}}{}
\textcolor{teal}{\mytextsc{adverbe}} \hspace{4pt} Ton~: M.
\textcolor{PineGreen}{\selectlanguage{french}Dans deux ans.} \zh{后年。}  ¶ \textcolor{darkblue}{\textbf{\ipa{ʁo˧ʝi˧ ɖɯ˧-kʰv̩˧˥}}} \textcolor{PineGreen}{\selectlanguage{french}l'année dans deux ans} \zh{后年}  

\lhead{\firstmark}
\rhead{\botmark}

\subsection{\hspace{-0.5cm} {\Large \textcolor{darkblue}{\textbf{\ipa{ʁo˧kɤ˩}}}}\hspace{0.5cm}[\kern2pt{\textcolor{darkblue}{\textbf{\ipa{ʁo˧kɤ˩}}}}\kern2pt]} \hypertarget{Ro\string_Mk7\string_B1}{}
\markboth{\textcolor{darkblue}{\textbf{\ipa{ʁo˧kɤ˩}}}}{}
\textcolor{teal}{\mytextsc{nom}} \hspace{4pt} Ton~: L\#.
\textcolor{PineGreen}{\selectlanguage{french}Coiffe en fils tressés; chez une jeune femme (qui a atteint ses 13 ans, mais n'a pas encore d'enfants), ce même objet est désigné comme \textcolor{darkblue}{\textbf{\ipa{/ʁo˧ni˥/}}}.} \zh{用来将长辫缠成盘头的黑色丝头饰(已经有孩子的女人戴的)。还没有孩子的青年女人,也戴这种头饰,但称作\textcolor{darkblue}{\textbf{\ipa{/ʁo˧ni˥/}}})。}  \zh{量词}: \textcolor{darkblue}{\textbf{\ipa{kɤ˧˥}}}  \mytextsc{clf}~: \textcolor{darkblue}{\textbf{\ipa{kɤ˧˥}}} 
\lhead{\firstmark}
\rhead{\botmark}

\subsection{\hspace{-0.5cm} {\Large \textcolor{darkblue}{\textbf{\ipa{ʁo˧lv̩˧}}}}\hspace{0.5cm}[\kern2pt{\textcolor{darkblue}{\textbf{\ipa{ʁo˧lv̩˧}}}}\kern2pt]} \hypertarget{Ro\string_Mlv\string_=\string_M1}{}
\markboth{\textcolor{darkblue}{\textbf{\ipa{ʁo˧lv̩˧}}}}{}
\textcolor{teal}{\mytextsc{verbe}} \hspace{4pt} Ton~: M.
\textcolor{PineGreen}{\selectlanguage{french}Se perdre, perdre son chemin.} \zh{迷路。}  ¶ \textcolor{darkblue}{\textbf{\ipa{le˧-ʁo˧lv̩˧}}} \textcolor{PineGreen}{\selectlanguage{french}\mytextsc{accomp}} \zh{\mytextsc{accomp}}  

\lhead{\firstmark}
\rhead{\botmark}

\subsection{\hspace{-0.5cm} {\Large \textcolor{darkblue}{\textbf{\ipa{ʁo˧ɬi˥}}}}\hspace{0.5cm}[\kern2pt{\textcolor{darkblue}{\textbf{\ipa{ʁo˧ɬi˥}}}}\kern2pt]} \hypertarget{Ro\string_MKi\string_T1}{}
\markboth{\textcolor{darkblue}{\textbf{\ipa{ʁo˧ɬi˥}}}}{}
\textcolor{teal}{\mytextsc{nom}} \hspace{4pt} Ton~: H\#.
\textcolor{PineGreen}{\selectlanguage{french}Grosse aiguille avec laquelle on coud le paquet de viande de cochon au salpêtre qui se conserve une décennie; contrairement à ce que dit M21, n'est pas utilisé pour coudre les peaux d'animaux (mouton, bœuf, yak…): pour cela, il faut utiliser un poinçon.} \zh{大粗针,用来缝琵琶肉。}  ¶ \textcolor{darkblue}{\textbf{\ipa{ʁo˧ɬi˥, | bo˩ʈʂʰæ˧ ʐv̩˩-di˩ ɲi˩.}}} \textcolor{PineGreen}{\selectlanguage{french}La grosse aiguille, ça sert à coudre le cochon-conservé-entier (viande séchée “pipa”).} \zh{大针,是用来缝琵琶肉的。}  
 \zh{量词}: \textcolor{darkblue}{\textbf{\ipa{ɭɯ˧}}}  \mytextsc{clf}~: \textcolor{darkblue}{\textbf{\ipa{ɭɯ˧}}} 
\lhead{\firstmark}
\rhead{\botmark}

\subsection{\hspace{-0.5cm} {\Large \textcolor{darkblue}{\textbf{\ipa{ʁo˧mi˥\$}}}}\hspace{0.5cm}[\kern2pt{\textcolor{darkblue}{\textbf{\ipa{ʁo˧mi˥}}}}\kern2pt]} \hypertarget{Ro\string_Mmi\string_T\$1}{}
\markboth{\textcolor{darkblue}{\textbf{\ipa{ʁo˧mi˥\$}}}}{}
\textcolor{teal}{\mytextsc{nom}} \hspace{4pt} Ton~: H\$.
\textcolor{PineGreen}{\selectlanguage{french}Grosse aiguille.} \zh{大针。}  \zh{量词}: \textcolor{darkblue}{\textbf{\ipa{ɭɯ˧}}}  \mytextsc{clf}~: \textcolor{darkblue}{\textbf{\ipa{ɭɯ˧}}} 
\lhead{\firstmark}
\rhead{\botmark}

\subsection{\hspace{-0.5cm} {\Large \textcolor{darkblue}{\textbf{\ipa{ʁo˧mi˧}}}}\hspace{0.5cm}[\kern2pt{\textcolor{darkblue}{\textbf{\ipa{ʁo˧mi˧}}}}\kern2pt]} \hypertarget{Ro\string_Mmi\string_M1}{}
\markboth{\textcolor{darkblue}{\textbf{\ipa{ʁo˧mi˧}}}}{}
\textcolor{teal}{\mytextsc{nom}} \hspace{4pt} Ton~: M.
\textcolor{PineGreen}{\selectlanguage{french}Roi; haut dignitaire, grand mandarin; chef.} \zh{国王、大臣、头领。}  ¶ \textcolor{darkblue}{\textbf{\ipa{ʁo˧mi˧ ʝi˧-hĩ˧ hĩ˧}}} \textcolor{PineGreen}{\selectlanguage{french}personne qui a fonction de dignitaire/chef} \zh{当国王、土司、大臣、头领……的人}  
 ¶ \textcolor{darkblue}{\textbf{\ipa{kʰv̩˧mæ˧-ʁo˧mi˧}}} \textcolor{PineGreen}{\selectlanguage{french}chef des brigands, capitaine d'une troupe de brigands} \zh{土匪的头领}  
 \zh{量词}: \textcolor{darkblue}{\textbf{\ipa{v̩˧}}}  \mytextsc{clf}~: \textcolor{darkblue}{\textbf{\ipa{v̩˧}}} 
\lhead{\firstmark}
\rhead{\botmark}

\subsection{\hspace{-0.5cm} {\Large \textcolor{darkblue}{\textbf{\ipa{ʁo˧ni˥}}}}\hspace{0.5cm}[\kern2pt{\textcolor{darkblue}{\textbf{\ipa{ʁo˧ni˥}}}}\kern2pt]} \hypertarget{Ro\string_Mni\string_T1}{}
\markboth{\textcolor{darkblue}{\textbf{\ipa{ʁo˧ni˥}}}}{}
\textcolor{teal}{\mytextsc{nom}} \hspace{4pt} Ton~: H\#.
\textcolor{PineGreen}{\selectlanguage{french}Coiffe en fils tressés des jeunes femmes qui n'ont pas encore d'enfant. Les femmes qui ont des enfants portent également cette pièce de costume, mais elle est alors désignée comme \textcolor{darkblue}{\textbf{\ipa{/ʁo˧kɤ˩/}}}.} \zh{用来将长辫缠成盘头的黑色丝头饰(还没有孩子的青年女人戴的)。已经有孩子的女人,也戴这种头饰,但称作\textcolor{darkblue}{\textbf{\ipa{/ʁo˧kɤ˩/}}})。}  \zh{量词}: \textcolor{darkblue}{\textbf{\ipa{bo˩}}}  \mytextsc{clf}~: \textcolor{darkblue}{\textbf{\ipa{bo˩}}} 
\lhead{\firstmark}
\rhead{\botmark}

\subsection{\hspace{-0.5cm} {\Large \textcolor{darkblue}{\textbf{\ipa{ʁo˧pʰɤ˩-ʁo˩dv̩˩lv̩˩}}}}\hspace{0.5cm}[\kern2pt{\textcolor{darkblue}{\textbf{\ipa{xxxx non-correspondance entre le nombre de morphèmes et le nombre de tons de morphèmes}}}}\kern2pt]} \hypertarget{Ro\string_Mp\string_h7\string_B-Ro\string_Bdv\string_=\string_Blv\string_=\string_B1}{}
\markboth{\textcolor{darkblue}{\textbf{\ipa{ʁo˧pʰɤ˩-ʁo˩dv̩˩lv̩˩}}}}{}
\textcolor{teal}{\mytextsc{nom}} \hspace{4pt} Ton~: L\#.
\textcolor{PineGreen}{\selectlanguage{french}\textit{Eugeron breviscapus} (une sorte de pâquerette).} \zh{短葶飞蓬。} 
\lhead{\firstmark}
\rhead{\botmark}

\subsection{\hspace{-0.5cm} {\Large \textcolor{darkblue}{\textbf{\ipa{ʁo˧qɑ˥}}}}\hspace{0.5cm}[\kern2pt{\textcolor{darkblue}{\textbf{\ipa{ʁo˧qɑ˥}}}}\kern2pt]} \hypertarget{Ro\string_MqA\string_T1}{}
\markboth{\textcolor{darkblue}{\textbf{\ipa{ʁo˧qɑ˥}}}}{}
\textcolor{teal}{\mytextsc{nom}} \hspace{4pt} Ton~: H\#.
\textcolor{PineGreen}{\selectlanguage{french}Couvercle.} \zh{锅盖、盖子。}  \zh{量词}: \textcolor{darkblue}{\textbf{\ipa{ɭɯ˧}}}  \mytextsc{clf}~: \textcolor{darkblue}{\textbf{\ipa{ɭɯ˧}}} 
\lhead{\firstmark}
\rhead{\botmark}

\subsection{\hspace{-0.5cm} {\Large \textcolor{darkblue}{\textbf{\ipa{ʁo˧qʰwɤ˩}}}}\hspace{0.5cm}[\kern2pt{\textcolor{darkblue}{\textbf{\ipa{ʁo˧qʰwɤ˩}}}}\kern2pt]} \hypertarget{Ro\string_Mq\string_hw7\string_B1}{}
\markboth{\textcolor{darkblue}{\textbf{\ipa{ʁo˧qʰwɤ˩}}}}{}
\textcolor{teal}{\mytextsc{nom}} \hspace{4pt} Ton~: L\#.
\ding{202} \textcolor{PineGreen}{\selectlanguage{french}Tête.} \zh{头,上面部分。}  ¶ \textcolor{darkblue}{\textbf{\ipa{ʁo˧qʰwɤ˩ dzi˩}}} \textcolor{PineGreen}{\selectlanguage{french}être assis à une place d'honneur} \zh{坐在贵宾的位置上}  
 ¶ \textcolor{darkblue}{\textbf{\ipa{õ˧-ʁo˥qʰwɤ˩}}} \textcolor{PineGreen}{\selectlanguage{french}sa propre tête} \zh{自己的头}  
 ¶ \textcolor{darkblue}{\textbf{\ipa{õ˧-ʁo˥qʰwɤ˩ lɑ˩}}} \textcolor{PineGreen}{\selectlanguage{french}se taper sur la tête (contexte: un enfant tape en rythme sur sa propre tête avec une baguette)} \zh{打自己的头(情景:一个小孩用小棍子敲打自己的头)}  
 \zh{量词}: \textcolor{darkblue}{\textbf{\ipa{ɭɯ˧}}} \ding{203} \textcolor{PineGreen}{\selectlanguage{french}Partie supérieure de.} \zh{上面部分。}  \mytextsc{clf}~: \textcolor{darkblue}{\textbf{\ipa{ɭɯ˧}}} 
\lhead{\firstmark}
\rhead{\botmark}

\subsection{\hspace{-0.5cm} {\Large \textcolor{darkblue}{\textbf{\ipa{ʁo˧so˩}}}}\hspace{0.5cm}[\kern2pt{\textcolor{darkblue}{\textbf{\ipa{ʁo˧so˩}}}}\kern2pt]} \hypertarget{Ro\string_Mso\string_B1}{}
\markboth{\textcolor{darkblue}{\textbf{\ipa{ʁo˧so˩}}}}{}
\textcolor{teal}{\mytextsc{adverbe}} \hspace{4pt} Ton~: L\#.
\textcolor{PineGreen}{\selectlanguage{french}Après-demain.} \zh{后天。}  ¶ \textcolor{darkblue}{\textbf{\ipa{ʁo˧so˩ | -ɖɯ˧ɲi˥}}} \textcolor{PineGreen}{\selectlanguage{french}la journée d'après-demain} \zh{后天}  

\lhead{\firstmark}
\rhead{\botmark}

\subsection{\hspace{-0.5cm} {\Large \textcolor{darkblue}{\textbf{\ipa{ʁo˧ʂv̩˧}}}}\hspace{0.5cm}[\kern2pt{\textcolor{darkblue}{\textbf{\ipa{ʁo˧ʂv̩˧}}}}\kern2pt]} \hypertarget{Ro\string_Ms`v\string_=\string_M1}{}
\markboth{\textcolor{darkblue}{\textbf{\ipa{ʁo˧ʂv̩˧}}}}{}
\textcolor{teal}{\mytextsc{verbe}} \hspace{4pt} Ton~: M.
\textcolor{PineGreen}{\selectlanguage{french}Conduire, guider.} \zh{带头、带路。}  ¶ \textcolor{darkblue}{\textbf{\ipa{ʐɤ˩mi˩ ʁo˩ʂv̩˩˥}}} \textcolor{PineGreen}{\selectlanguage{french}montrer le chemin} \zh{带路}  
 ¶ \textcolor{darkblue}{\textbf{\ipa{ɖɯ˧-ʑi˩-ɳɯ˩ | ʁo˧ʂv̩˧}}} \textcolor{PineGreen}{\selectlanguage{french}une famille montre l'exemple: par exemple, une famille commence à récolter le riz, et les autres suivent son exemple} \zh{有一家带头:例如收庄稼时,一个家先开始收割,于是其它家庭也跟着开始收割。}  
 ¶ \textcolor{darkblue}{\textbf{\ipa{ʁo˧ʂv̩˧-ze˧}}} \textcolor{PineGreen}{\selectlanguage{french}\mytextsc{pfv}} \zh{带了路}  
 ¶ \textcolor{darkblue}{\textbf{\ipa{njɤ˧=ɻ̍˩-ɳɯ˩ | ʁo˧ʂv̩˧!}}} \textcolor{PineGreen}{\selectlanguage{french}C'est nous qui lançons le mouvement!/ C'est nous qui donnons l'exemple aux autres! (explication: pour les travaux des champs, une maisonnée s'y attelait en premier, et les autres suivaient)} \zh{是我们带头的!(其他家庭是跟着我们来的!)(情景:农业活动,如:收庄稼,是一个家庭先开始的,然后其他家庭也跟着来。)}  

\lhead{\firstmark}
\rhead{\botmark}

\subsection{\hspace{-0.5cm} {\Large \textcolor{darkblue}{\textbf{\ipa{-ʁo˧to˩}}}}\hspace{0.5cm}[\kern2pt{\textcolor{darkblue}{\textbf{\ipa{ʁo˧to˩}}}}\kern2pt]} \hypertarget{-Ro\string_Mto\string_B1}{}
\markboth{\textcolor{darkblue}{\textbf{\ipa{-ʁo˧to˩}}}}{}
\textcolor{teal}{\mytextsc{postposition}} \hspace{4pt} Ton~: L\#.
\ding{202} \textcolor{PineGreen}{\selectlanguage{french}Sur.} \zh{……之上。}  ¶ \textcolor{darkblue}{\textbf{\ipa{qo˩qɑ˩-ʁo˩to˥}}} \textcolor{PineGreen}{\selectlanguage{french}en haut du col} \zh{垭口上}  
 ¶ \textcolor{darkblue}{\textbf{\ipa{ʁo˧qʰwɤ˩-ʁo˩to˩}}} \textcolor{PineGreen}{\selectlanguage{french}sur la tête, sur le sommet du crâne} \zh{头上}  
 ¶ \textcolor{darkblue}{\textbf{\ipa{ʑi˧qʰwɤ˧-ʁo˧to˩}}} \textcolor{PineGreen}{\selectlanguage{french}sur la maison; ex.: il y a un nid d’oiseaux sur la maison} \zh{房子上面:例如:有鸟窝在房顶上}  
\ding{203} \textcolor{PineGreen}{\selectlanguage{french}Pendant, au moment de.} \zh{……的时候。}  ¶ \textcolor{darkblue}{\textbf{\ipa{hɑ˧dzɯ˧-ʁo˧to˩, | ʈʂʰɯ˧-ɳɯ˧ | mɤ˧-fv̩˧-ʝi˧.}}} \textcolor{PineGreen}{\selectlanguage{french}Au cours du repas, il se mit en colère/ devint triste.} \zh{吃饭的时候,他不高兴了/生气了。}  
\ding{204} \textcolor{PineGreen}{\selectlanguage{french}À l'endroit de, à l'égard de, en direction de.} \zh{向、往。} \ding{205} \textcolor{PineGreen}{\selectlanguage{french}En comparaison de.} \zh{跟……相比。} 
\lhead{\firstmark}
\rhead{\botmark}

\subsection{\hspace{-0.5cm} {\Large \textcolor{darkblue}{\textbf{\ipa{-ʁo˧tʰo˩}}}}\hspace{0.5cm}[\kern2pt{\textcolor{darkblue}{\textbf{\ipa{ʁo˧tʰo˩}}}}\kern2pt]} \hypertarget{-Ro\string_Mt\string_ho\string_B1}{}
\markboth{\textcolor{darkblue}{\textbf{\ipa{-ʁo˧tʰo˩}}}}{}
\textcolor{teal}{\mytextsc{postposition}} \hspace{4pt} Ton~: L\#.
\textcolor{PineGreen}{\selectlanguage{french}Derrière; depuis.} \zh{后面,自从。}  ¶ \textcolor{darkblue}{\textbf{\ipa{ʑi˧-tʰo˩}}} \textcolor{PineGreen}{\selectlanguage{french}l'arrière de la maison (où il y a le potager)} \zh{家后院(=菜园的地方)}  
 ¶ \textcolor{darkblue}{\textbf{\ipa{ʑi˧-ʁo˥tʰo˩}}} \textcolor{PineGreen}{\selectlanguage{french}idem: derrière la maison, l'arrière de la maison} \zh{同上:家后院}  
 ¶ \textcolor{darkblue}{\textbf{\ipa{ʑi˧qʰwɤ˧-ʁo˧tʰo˩}}} \textcolor{PineGreen}{\selectlanguage{french}idem: derrière la maison, l'arrière de la maison} \zh{同上:家后院}  

\lhead{\firstmark}
\rhead{\botmark}

\subsection{\hspace{-0.5cm} {\Large \textcolor{darkblue}{\textbf{\ipa{ʁo˧tɕʰɤ\#˥}}}}\hspace{0.5cm}[\kern2pt{\textcolor{darkblue}{\textbf{\ipa{ʁo˧tɕʰɤ˧}}}}\kern2pt]} \hypertarget{Ro\string_Mts£\string_h7\#\string_T1}{}
\markboth{\textcolor{darkblue}{\textbf{\ipa{ʁo˧tɕʰɤ\#˥}}}}{}
\textcolor{teal}{\mytextsc{adjectif}} \hspace{4pt} Ton~: \#H.
\textcolor{PineGreen}{\selectlanguage{french}Pointu.} \zh{尖。}  ¶ \textcolor{darkblue}{\textbf{\ipa{[F5] ʁo˧tɕʰɤ˧\textasciitilde{}tɕʰɤ˧-gv̩˧}}} \textcolor{PineGreen}{\selectlanguage{french}pointu} \zh{尖}  

\lhead{\firstmark}
\rhead{\botmark}

\subsection{\hspace{-0.5cm} {\Large \textcolor{darkblue}{\textbf{\ipa{ʁo˧tsʰe˧ʁo\#˥}}}}\hspace{0.5cm}[\kern2pt{\textcolor{darkblue}{\textbf{\ipa{ʁo˧tsʰe˧ʁo˩}}}}\kern2pt]} \hypertarget{Ro\string_Mts\string_he\string_MRo\#\string_T1}{}
\markboth{\textcolor{darkblue}{\textbf{\ipa{ʁo˧tsʰe˧ʁo\#˥}}}}{}
\textcolor{teal}{\mytextsc{nom}} \hspace{4pt} Ton~: \#H.
\textcolor{PineGreen}{\selectlanguage{french}Sommet de, en haut de.} \zh{顶上,如:山顶。}  ¶ \textcolor{darkblue}{\textbf{\ipa{ʁwɤ˧-bv̩˧ | ʁo˧tsʰe˧ʁo˧}}} \textcolor{PineGreen}{\selectlanguage{french}le sommet de la montagne} \zh{山的顶,山顶}  
 ¶ \textcolor{darkblue}{\textbf{\ipa{ʁo˧qʰwɤ˩-ʁo˩tsʰe˩}}} \textcolor{PineGreen}{\selectlanguage{french}le sommet de la tête} \zh{头顶}  

\lhead{\firstmark}
\rhead{\botmark}

\subsection{\hspace{-0.5cm} {\Large \textcolor{darkblue}{\textbf{\ipa{ʁo˧ʈv̩˧ʈv̩˥}}}}\hspace{0.5cm}[\kern2pt{\textcolor{darkblue}{\textbf{\ipa{ʁo˧ʈv̩˧ʈv̩˧}}}}\kern2pt]} \hypertarget{Ro\string_Mt`v\string_=\string_Mt`v\string_=\string_T1}{}
\markboth{\textcolor{darkblue}{\textbf{\ipa{ʁo˧ʈv̩˧ʈv̩˥}}}}{}
\textcolor{teal}{\mytextsc{nom}} \hspace{4pt} Ton~: H\#.
\textcolor{PineGreen}{\selectlanguage{french}Yi (groupe ethnique): terme péjoratif: “les hirsutes”, “les ébouriffés”.} \zh{彝族(带偏见的说法:“乱糟糟的头发”)。}  \zh{量词}: \textcolor{darkblue}{\textbf{\ipa{v̩˧}}}  \mytextsc{clf}~: \textcolor{darkblue}{\textbf{\ipa{v̩˧}}} 
\lhead{\firstmark}
\rhead{\botmark}

\subsection{\hspace{-0.5cm} {\Large \textcolor{darkblue}{\textbf{\ipa{ʁo˧ʈʂe˩}}}}\hspace{0.5cm}[\kern2pt{\textcolor{darkblue}{\textbf{\ipa{ʁo˧ʈʂe˩}}}}\kern2pt]} \hypertarget{Ro\string_Mt`s`e\string_B1}{}
\markboth{\textcolor{darkblue}{\textbf{\ipa{ʁo˧ʈʂe˩}}}}{}
\textcolor{teal}{\mytextsc{nom}} \hspace{4pt} Ton~: L\#.
\textcolor{PineGreen}{\selectlanguage{french}Teigne.} \zh{癣。}  ¶ \textcolor{darkblue}{\textbf{\ipa{[M23] ʁu˧ʈʂɯ˩ ɖwæ˧˥ tʰi˧ di˩!}}} \textcolor{PineGreen}{\selectlanguage{french}(il) a vraiment la teigne à la tête!} \zh{他长了很多癣!}  
 \zh{量词}: \textcolor{darkblue}{\textbf{\ipa{pʰæ˧˥}}}  \mytextsc{clf}~: \textcolor{darkblue}{\textbf{\ipa{pʰæ˧˥}}} 
\lhead{\firstmark}
\rhead{\botmark}

\subsection{\hspace{-0.5cm} {\Large \textcolor{darkblue}{\textbf{\ipa{ʁo˧zo\#˥}}}}\hspace{0.5cm}[\kern2pt{\textcolor{darkblue}{\textbf{\ipa{ʁo˧zo˧˥}}}}\kern2pt]} \hypertarget{Ro\string_Mzo\#\string_T1}{}
\markboth{\textcolor{darkblue}{\textbf{\ipa{ʁo˧zo\#˥}}}}{}
\textcolor{teal}{\mytextsc{nom}} \hspace{4pt} Ton~: \#H.
\textcolor{PineGreen}{\selectlanguage{french}Petite aiguille.} \zh{小针。}  \zh{量词}: \textcolor{darkblue}{\textbf{\ipa{ɭɯ˧}}}  \mytextsc{clf}~: \textcolor{darkblue}{\textbf{\ipa{ɭɯ˧}}} 
\lhead{\firstmark}
\rhead{\botmark}

\subsection{\hspace{-0.5cm} {\Large \textcolor{darkblue}{\textbf{\ipa{ʁo˧ʑi˧˥}}}}\hspace{0.5cm}[\kern2pt{\textcolor{darkblue}{\textbf{\ipa{ʁo˧ʑi˥}}}}\kern2pt]} \hypertarget{Ro\string_Mz£i\string_M\string_T1}{}
\markboth{\textcolor{darkblue}{\textbf{\ipa{ʁo˧ʑi˧˥}}}}{}
\textcolor{teal}{\mytextsc{adverbe}} \hspace{4pt} Ton~: MH\#.
\textcolor{PineGreen}{\selectlanguage{french}À partir de.} \zh{从……开始。}  ¶ \textcolor{darkblue}{\textbf{\ipa{ɖɯ˧ɬi˧mi˧-ʁo˧ʑi˧˥}}} \textcolor{PineGreen}{\selectlanguage{french}à partir du premier mois} \zh{一月份开始}  
 ¶ \textcolor{darkblue}{\textbf{\ipa{tsʰi˧ɲi˧-ʁo˧ʑi˧˥}}} \textcolor{PineGreen}{\selectlanguage{french}à partir d'aujourd'hui} \zh{今天开始}  
 ¶ \textcolor{darkblue}{\textbf{\ipa{tsʰi˧ʝi˧ ɖɯ˧-kʰv̩˧˥-ʁo˧ʑi˧˥}}} \textcolor{PineGreen}{\selectlanguage{french}à partir de cette année} \zh{今年开始}  
 ¶ \textcolor{darkblue}{\textbf{\ipa{gv̩˩ɬi˩mi˩-ʁo˩ʑi˩˥}}} \textcolor{PineGreen}{\selectlanguage{french}à partir du 9e mois} \zh{九月份开始}  
 ¶ \textcolor{darkblue}{\textbf{\ipa{ʐe˧ʈæ˥ɬi˩-ʁo˩ʑi˩}}} \textcolor{PineGreen}{\selectlanguage{french}à partir du 11e mois} \zh{十一月份开始}  

\lhead{\firstmark}
\rhead{\botmark}

\subsection{\hspace{-0.5cm} {\Large \textcolor{darkblue}{\textbf{\ipa{ʁo˩}}}}\hspace{0.5cm}[\kern2pt{\textcolor{darkblue}{\textbf{\ipa{ʁo˩˥}}}}\kern2pt]} \hypertarget{Ro\string_B1}{}
\markboth{\textcolor{darkblue}{\textbf{\ipa{ʁo˩}}}}{}
\textcolor{teal}{\mytextsc{verbe}} \hspace{4pt} Ton~: L.
\textcolor{PineGreen}{\selectlanguage{french}Tomber, sombrer (ex.: quelqu'un coule dans l'eau; un bateau sombre peu à peu dans le lac).} \zh{掉入、沉下去。}  ¶ \textcolor{darkblue}{\textbf{\ipa{mv̩˩tɕo˥ ʁo˩}}} \textcolor{PineGreen}{\selectlanguage{french}s'enfoncer, être englouti (par l'eau…)} \zh{掉入}  

\lhead{\firstmark}
\rhead{\botmark}

\subsection{\hspace{-0.5cm} {\Large \textcolor{darkblue}{\textbf{\ipa{ʁo˩\textsubscript{b}}}}}\hspace{0.5cm}[\kern2pt{\textcolor{darkblue}{\textbf{\ipa{ʁo˩˥}}}}\kern2pt]} \hypertarget{Ro\string_Bb1}{}
\markboth{\textcolor{darkblue}{\textbf{\ipa{ʁo˩\textsubscript{b}}}}}{}
\textcolor{teal}{\mytextsc{classificateur}} \hspace{4pt} Ton~: L\textsubscript{b}.
\textcolor{PineGreen}{\selectlanguage{french}Classificateur des variétés/sortes de choses.} \zh{量词:种。}  ¶ \textcolor{darkblue}{\textbf{\ipa{ɖɯ˧-ʁo˩}}} \textcolor{PineGreen}{\selectlanguage{french}une sorte (de vêtement, de nourriture...)} \zh{一种(衣服、食物……)}  
 ¶ \textcolor{darkblue}{\textbf{\ipa{ʈʂʰɯ˧-ʁo˥}}} \textcolor{PineGreen}{\selectlanguage{french}cette sorte (de vêtement, de nourriture...)} \zh{这种(衣服、食物……)}  

\lhead{\firstmark}
\rhead{\botmark}

\subsection{\hspace{-0.5cm} {\Large \textcolor{darkblue}{\textbf{\ipa{ʁo˩\textsubscript{b}}}}}\hspace{0.5cm}[\kern2pt{\textcolor{darkblue}{\textbf{\ipa{ʁo˥}}}}\kern2pt]} \hypertarget{Ro\string_Bb1}{}
\markboth{\textcolor{darkblue}{\textbf{\ipa{ʁo˩\textsubscript{b}}}}}{}
\textcolor{teal}{\mytextsc{verbe}} \hspace{4pt} Ton~: L\textsubscript{b}.
\textcolor{PineGreen}{\selectlanguage{french}Se former, apparaître (un cor se forme, un durillon se forme). Ce verbe n'a été observé qu'en association avec le mot 'durillon, cor'.} \zh{出现、形成(如:出了茧子)。}  ¶ \textcolor{darkblue}{\textbf{\ipa{sɯ˧ʈv̩˥ ʁo˩-ze˩! |}}} \textcolor{PineGreen}{\selectlanguage{french}Un durillon s'est formé!} \zh{磨出了茧子!}  
 ¶ \textcolor{darkblue}{\textbf{\ipa{ʁo˩-mɤ˩-ho˥}}} \textcolor{PineGreen}{\selectlanguage{french}\string_ \mytextsc{neg} \mytextsc{désidératif}} \zh{不会出(茧子)}  

\lhead{\firstmark}
\rhead{\botmark}

\subsection{\hspace{-0.5cm} {\Large \textcolor{darkblue}{\textbf{\ipa{ʁo˩di˥}}}}\hspace{0.5cm}[\kern2pt{\textcolor{darkblue}{\textbf{\ipa{ʁo˧di˧}}}}\kern2pt]} \hypertarget{Ro\string_Bdi\string_T1}{}
\markboth{\textcolor{darkblue}{\textbf{\ipa{ʁo˩di˥}}}}{}
\textcolor{teal}{\mytextsc{nom}} \hspace{4pt} Ton~: LH.
\textcolor{PineGreen}{\selectlanguage{french}Fou, aliéné.} \zh{疯子。}  \zh{量词}: \textcolor{darkblue}{\textbf{\ipa{v̩˧}}}  \mytextsc{clf}~: \textcolor{darkblue}{\textbf{\ipa{v̩˧}}} 
\lhead{\firstmark}
\rhead{\botmark}

\subsection{\hspace{-0.5cm} {\Large \textcolor{darkblue}{\textbf{\ipa{ʁo˩ɖɯ˩so˧}}}}\hspace{0.5cm}[\kern2pt{\textcolor{darkblue}{\textbf{\ipa{ʁo˧ɖɯ˧so˧˥}}}}\kern2pt]} \hypertarget{Ro\string_Bd`M\string_Bso\string_M1}{}
\markboth{\textcolor{darkblue}{\textbf{\ipa{ʁo˩ɖɯ˩so˧}}}}{}
\textcolor{teal}{\mytextsc{adverbe}} \hspace{4pt} Ton~: .
\textcolor{PineGreen}{\selectlanguage{french}Dans trois jours.} \zh{大后天。} 
\lhead{\firstmark}
\rhead{\botmark}

\subsection{\hspace{-0.5cm} {\Large \textcolor{darkblue}{\textbf{\ipa{ʁo˩hi˩}}}}\hspace{0.5cm}[\kern2pt{\textcolor{darkblue}{\textbf{\ipa{ʁo˩hi˩˥}}}}\kern2pt]} \hypertarget{Ro\string_Bhi\string_B1}{}
\markboth{\textcolor{darkblue}{\textbf{\ipa{ʁo˩hi˩}}}}{}
\textcolor{teal}{\mytextsc{nom}} \hspace{4pt} Ton~: L.
\textcolor{PineGreen}{\selectlanguage{french}Molaires et prémolaires.} \zh{臼齿+后臼齿。}  \zh{量词}: \textcolor{darkblue}{\textbf{\ipa{ɭɯ˧}}}  \mytextsc{clf}~: \textcolor{darkblue}{\textbf{\ipa{ɭɯ˧}}} 
\lhead{\firstmark}
\rhead{\botmark}

\subsection{\hspace{-0.5cm} {\Large \textcolor{darkblue}{\textbf{\ipa{ʁo˩kʰv̩˩}}}}\hspace{0.5cm}[\kern2pt{\textcolor{darkblue}{\textbf{\ipa{ʁo˩kʰv̩˩˥}}}}\kern2pt]} \hypertarget{Ro\string_Bk\string_hv\string_=\string_B1}{}
\markboth{\textcolor{darkblue}{\textbf{\ipa{ʁo˩kʰv̩˩}}}}{}
\textcolor{teal}{\mytextsc{nom}} \hspace{4pt} Ton~: L.
\textcolor{PineGreen}{\selectlanguage{french}Arbre à épice, arbre à encens de petite taille, qui pousse en montagne, dans les espaces ombragés.} \zh{香木。} Dialecte chinois local~:\zh{柏香。} ¶ \textcolor{darkblue}{\textbf{\ipa{ʁo˩kʰv̩˩-si˩}}} \textcolor{PineGreen}{\selectlanguage{french}même sens} \zh{同上}  
\textit{Voir~:} \hyperlink{}{\textcolor{darkblue}{\textbf{\ipa{tsɤ˧di˧}}}} 
\lhead{\firstmark}
\rhead{\botmark}

\subsection{\hspace{-0.5cm} {\Large \textcolor{darkblue}{\textbf{\ipa{ʁo˧˥}}}}\hspace{0.5cm}[\kern2pt{\textcolor{darkblue}{\textbf{\ipa{ʁo˧˥}}}}\kern2pt]} \hypertarget{Ro\string_M\string_T1}{}
\markboth{\textcolor{darkblue}{\textbf{\ipa{ʁo˧˥}}}}{}
\textcolor{teal}{\mytextsc{nom}} \hspace{4pt} Ton~: MH.
\textcolor{PineGreen}{\selectlanguage{french}Aiguille.} \zh{针。}  \zh{量词}: \textcolor{darkblue}{\textbf{\ipa{ɭɯ˧}}}  \mytextsc{clf}~: \textcolor{darkblue}{\textbf{\ipa{ɭɯ˧}}} \textit{Voir~:} \textcolor{darkblue}{\textbf{\ipa{ʈʂe˥}}} 
\lhead{\firstmark}
\rhead{\botmark}

\subsection{\hspace{-0.5cm} {\Large \textcolor{darkblue}{\textbf{\ipa{ʁv̩˧˥}}}}\hspace{0.5cm}[\kern2pt{\textcolor{darkblue}{\textbf{\ipa{ʁv̩˧˥}}}}\kern2pt]} \hypertarget{Rv\string_=\string_M\string_T1}{}
\markboth{\textcolor{darkblue}{\textbf{\ipa{ʁv̩˧˥}}}}{}
\textcolor{teal}{\mytextsc{nom}} \hspace{4pt} Ton~: MH.
\textcolor{PineGreen}{\selectlanguage{french}Grue, Grus nigricollis Przew et autres espèces similaires. Il s'agit d'un oiseau migrateur.} \zh{黑颈鹤(候鸟)。}  ¶ \textcolor{darkblue}{\textbf{\ipa{ʁv̩˧nɑ˥mi˩}}} \textcolor{PineGreen}{\selectlanguage{french}même sens: grue} \zh{同上:黑颈鹤}  
 ¶ \textcolor{darkblue}{\textbf{\ipa{ʁv̩˧ dzɯ˥-ze˩}}} \textcolor{PineGreen}{\selectlanguage{french}...a mangé une grue} \zh{吃了黑颈鹤}  
 ¶ \textcolor{darkblue}{\textbf{\ipa{ʁv̩˧ hwæ˥-ze˩}}} \textcolor{PineGreen}{\selectlanguage{french}...a acheté une grue} \zh{买了黑颈鹤}  
 \zh{量词}: \textcolor{darkblue}{\textbf{\ipa{mi˩}}}  \mytextsc{clf}~: \textcolor{darkblue}{\textbf{\ipa{mi˩}}} 
\lhead{\firstmark}
\rhead{\botmark}

\subsection{\hspace{-0.5cm} {\Large \textcolor{darkblue}{\textbf{\ipa{ʁv̩˥}}}}\hspace{0.5cm}[\kern2pt{\textcolor{darkblue}{\textbf{\ipa{ʁv̩˥}}}}\kern2pt]} \hypertarget{Rv\string_=\string_T1}{}
\markboth{\textcolor{darkblue}{\textbf{\ipa{ʁv̩˥}}}}{}
\textcolor{teal}{\mytextsc{verbe}} \hspace{4pt} Ton~: H.
\textcolor{PineGreen}{\selectlanguage{french}Avaler, déglutir.} \zh{吞,咽。}  ¶ \textcolor{darkblue}{\textbf{\ipa{le˧-ʁv̩˥}}} \textcolor{PineGreen}{\selectlanguage{french}\mytextsc{accomp}} \zh{\mytextsc{accomp}}  

\lhead{\firstmark}
\rhead{\botmark}

\subsection{\hspace{-0.5cm} {\Large \textcolor{darkblue}{\textbf{\ipa{ʁv̩˧mi˥\$}}}}\hspace{0.5cm}[\kern2pt{\textcolor{darkblue}{\textbf{\ipa{ʁv̩˧mi˥}}}}\kern2pt]} \hypertarget{Rv\string_=\string_Mmi\string_T\$1}{}
\markboth{\textcolor{darkblue}{\textbf{\ipa{ʁv̩˧mi˥\$}}}}{}
\textcolor{teal}{\mytextsc{nom}} \hspace{4pt} Ton~: H\$.
\textcolor{PineGreen}{\selectlanguage{french}Grue femelle.} \zh{母鹤。}  \zh{量词}: \textcolor{darkblue}{\textbf{\ipa{mi˩}}}  \mytextsc{clf}~: \textcolor{darkblue}{\textbf{\ipa{mi˩}}} 
\lhead{\firstmark}
\rhead{\botmark}

\subsection{\hspace{-0.5cm} {\Large \textcolor{darkblue}{\textbf{\ipa{ʁv̩˧pʰv̩\#˥}}}}\hspace{0.5cm}[\kern2pt{\textcolor{darkblue}{\textbf{\ipa{ʁv̩˧pʰv̩˧}}}}\kern2pt]} \hypertarget{Rv\string_=\string_Mp\string_hv\string_=\#\string_T1}{}
\markboth{\textcolor{darkblue}{\textbf{\ipa{ʁv̩˧pʰv̩\#˥}}}}{}
\textcolor{teal}{\mytextsc{nom}} \hspace{4pt} Ton~: \#H.
\textcolor{PineGreen}{\selectlanguage{french}Grue mâle.} \zh{公鹤。}  ¶ \textcolor{darkblue}{\textbf{\ipa{ʁv̩˧pʰv̩˧-ʁv̩˧mi\#˥}}} \textcolor{PineGreen}{\selectlanguage{french}grue mâle et grue femelle} \zh{公鹤与母鹤}  
 \zh{量词}: \textcolor{darkblue}{\textbf{\ipa{mi˩}}}  \mytextsc{clf}~: \textcolor{darkblue}{\textbf{\ipa{mi˩}}} 
\lhead{\firstmark}
\rhead{\botmark}

\subsection{\hspace{-0.5cm} {\Large \textcolor{darkblue}{\textbf{\ipa{ʁv̩˧zo\#˥}}}}\hspace{0.5cm}[\kern2pt{\textcolor{darkblue}{\textbf{\ipa{ʁv̩˧zo˧}}}}\kern2pt]} \hypertarget{Rv\string_=\string_Mzo\#\string_T1}{}
\markboth{\textcolor{darkblue}{\textbf{\ipa{ʁv̩˧zo\#˥}}}}{}
\textcolor{teal}{\mytextsc{nom}} \hspace{4pt} Ton~: \#H.
\textcolor{PineGreen}{\selectlanguage{french}Enfant grue.} \zh{小鹤。}  \zh{量词}: \textcolor{darkblue}{\textbf{\ipa{ɭɯ˧}}}  \mytextsc{clf}~: \textcolor{darkblue}{\textbf{\ipa{ɭɯ˧}}} 
\lhead{\firstmark}
\rhead{\botmark}

\subsection{\hspace{-0.5cm} {\Large \textcolor{darkblue}{\textbf{\ipa{ʁwæ˥}}}}\hspace{0.5cm}[\kern2pt{\textcolor{darkblue}{\textbf{\ipa{ʁwæ˥}}}}\kern2pt]} \hypertarget{Rw\{\string_T1}{}
\markboth{\textcolor{darkblue}{\textbf{\ipa{ʁwæ˥}}}}{}
\textcolor{teal}{\mytextsc{nom}} \hspace{4pt} Ton~: \#H.
\textcolor{PineGreen}{\selectlanguage{french}Gauche (monosyllabe).} \zh{左边(单音节)。} 
\lhead{\firstmark}
\rhead{\botmark}

\subsection{\hspace{-0.5cm} {\Large \textcolor{darkblue}{\textbf{\ipa{ʁwæ˧gi\#˥}}}}\hspace{0.5cm}[\kern2pt{\textcolor{darkblue}{\textbf{\ipa{ʁwæ˧gi˧}}}}\kern2pt]} \hypertarget{Rw\{\string_Mgi\#\string_T1}{}
\markboth{\textcolor{darkblue}{\textbf{\ipa{ʁwæ˧gi\#˥}}}}{}
\textcolor{teal}{\mytextsc{nom}} \hspace{4pt} Ton~: \#H.
\textcolor{PineGreen}{\selectlanguage{french}Gauche, côté gauche.} \zh{左边。} 
\lhead{\firstmark}
\rhead{\botmark}

\subsection{\hspace{-0.5cm} {\Large \textcolor{darkblue}{\textbf{\ipa{ʁwæ˧gi˧dzɤ\#˥}}}}\hspace{0.5cm}[\kern2pt{\textcolor{darkblue}{\textbf{\ipa{ʁwæ˧gi˧dzɤ˧}}}}\kern2pt]} \hypertarget{Rw\{\string_Mgi\string_Mdz7\#\string_T1}{}
\markboth{\textcolor{darkblue}{\textbf{\ipa{ʁwæ˧gi˧dzɤ\#˥}}}}{}
\textcolor{teal}{\mytextsc{nom}} \hspace{4pt} Ton~: \#H.
\textcolor{PineGreen}{\selectlanguage{french}Gauche, côté gauche, direction de gauche.} \zh{左、左边。} 
\lhead{\firstmark}
\rhead{\botmark}

\subsection{\hspace{-0.5cm} {\Large \textcolor{darkblue}{\textbf{\ipa{ʁwæ˧lo˥}}}}\hspace{0.5cm}[\kern2pt{\textcolor{darkblue}{\textbf{\ipa{ʁwæ˧lo˥}}}}\kern2pt]} \hypertarget{Rw\{\string_Mlo\string_T1}{}
\markboth{\textcolor{darkblue}{\textbf{\ipa{ʁwæ˧lo˥}}}}{}
\textcolor{teal}{\mytextsc{nom}} \hspace{4pt} Ton~: H\#.
\textcolor{PineGreen}{\selectlanguage{french}Gauche; direction de gauche.} \zh{左边,左手。} 
\lhead{\firstmark}
\rhead{\botmark}

\subsection{\hspace{-0.5cm} {\Large \textcolor{darkblue}{\textbf{\ipa{ʁwæ˧tsɯ˥}}}}\hspace{0.5cm}[\kern2pt{\textcolor{darkblue}{\textbf{\ipa{ʁwæ˧tsɯ˥}}}}\kern2pt]} \hypertarget{Rw\{\string_MtsM\string_T1}{}
\markboth{\textcolor{darkblue}{\textbf{\ipa{ʁwæ˧tsɯ˥}}}}{}
\textcolor{teal}{\mytextsc{nom}} \hspace{4pt} Ton~: H\#.
\textcolor{PineGreen}{\selectlanguage{french}Chaussettes.} \zh{袜子。}  Emprunt~: chinois  \zh{袜子}

\lhead{\firstmark}
\rhead{\botmark}

\subsection{\hspace{-0.5cm} {\Large \textcolor{darkblue}{\textbf{\ipa{ʁwæ˧ʈʂʰe˩}}}}\hspace{0.5cm}[\kern2pt{\textcolor{darkblue}{\textbf{\ipa{ʁwæ˧ʈʂʰe˩}}}}\kern2pt]} \hypertarget{Rw\{\string_Mt`s`\string_he\string_B1}{}
\markboth{\textcolor{darkblue}{\textbf{\ipa{ʁwæ˧ʈʂʰe˩}}}}{}
\textcolor{teal}{\mytextsc{verbe}} \hspace{4pt} Ton~: L\#.
\textcolor{PineGreen}{\selectlanguage{french}Achever, mener à terme.} \zh{完成(汉语借词)。}  Emprunt~: chinois  \zh{完成}
 ¶ \textcolor{darkblue}{\textbf{\ipa{le˧-ʁwæ˧ʈʂʰe˩-ze˩!}}} \textcolor{PineGreen}{\selectlanguage{french}C'est achevé!} \zh{完成了!}  

\lhead{\firstmark}
\rhead{\botmark}

\subsection{\hspace{-0.5cm} {\Large \textcolor{darkblue}{\textbf{\ipa{ʁwɤ˧}}} \textsubscript{1}}\hspace{0.5cm}[\kern2pt{\textcolor{darkblue}{\textbf{\ipa{ʁwɤ˥}}}}\kern2pt]} \hypertarget{Rw7\string_M1}{}
\markboth{\textcolor{darkblue}{\textbf{\ipa{ʁwɤ˧}}} \textsubscript{1}}{}
\textcolor{teal}{\mytextsc{nom}} \hspace{4pt} Ton~: M.
\textcolor{PineGreen}{\selectlanguage{french}Montagne.} \zh{山。}  ¶ \textcolor{darkblue}{\textbf{\ipa{ʁo˧-ʂwæ˧}}} \textcolor{PineGreen}{\selectlanguage{french}haute montagne} \zh{高山}  
 \zh{量词}: \textcolor{darkblue}{\textbf{\ipa{ɭɯ˧}}}  \mytextsc{clf}~: \textcolor{darkblue}{\textbf{\ipa{ɭɯ˧}}} \textcolor{PineGreen}{\selectlanguage{french}objets ronds}
\lhead{\firstmark}
\rhead{\botmark}

\subsection{\hspace{-0.5cm} {\Large \textcolor{darkblue}{\textbf{\ipa{ʁwɤ˧}}} \textsubscript{2}}\hspace{0.5cm}[\kern2pt{\textcolor{darkblue}{\textbf{\ipa{ʁwɤ˥}}}}\kern2pt]} \hypertarget{Rw7\string_M2}{}
\markboth{\textcolor{darkblue}{\textbf{\ipa{ʁwɤ˧}}} \textsubscript{2}}{}
\textcolor{teal}{\mytextsc{nom}} \hspace{4pt} Ton~: M.
\textcolor{PineGreen}{\selectlanguage{french}Village, hameau.} \zh{村寨,村落。}  ¶ \textcolor{darkblue}{\textbf{\ipa{ʁwɤ˧-qo˧}}} \textcolor{PineGreen}{\selectlanguage{french}dans le village} \zh{村子里}  
 ¶ \textcolor{darkblue}{\textbf{\ipa{[M23] ɖɯ˧-ʁwɤ˧ mɤ˧-ɲi˩: | ʈʂʰɯ˧-ʁwɤ˧... | ʈʂʰɯ˧-ʁwɤ˧…}}} \textcolor{PineGreen}{\selectlanguage{french}Ce n'est pas le même endroit (littéralement: ce n'est pas le même village): ici, c'est (le village de)…; là, c'est (le village de)…} \zh{它们不属于一个村落:这边,是……村,而那边,是……村。}  
 \zh{量词}: \textcolor{darkblue}{\textbf{\ipa{ʁwɤ˧}}}  \mytextsc{clf}~: \textcolor{darkblue}{\textbf{\ipa{ʁwɤ˧}}} 
\lhead{\firstmark}
\rhead{\botmark}

\subsection{\hspace{-0.5cm} {\Large \textcolor{darkblue}{\textbf{\ipa{ʁwɤ˧}}} \textsubscript{3}}\hspace{0.5cm}[\kern2pt{\textcolor{darkblue}{\textbf{\ipa{ʁwɤ˥}}}}\kern2pt]} \hypertarget{Rw7\string_M3}{}
\markboth{\textcolor{darkblue}{\textbf{\ipa{ʁwɤ˧}}} \textsubscript{3}}{}
\textcolor{teal}{\mytextsc{nom}} \hspace{4pt} Ton~: M.
\textcolor{PineGreen}{\selectlanguage{french}Argent (non pas le métal, mais la monnaie).} \zh{钱。}  ¶ \textcolor{darkblue}{\textbf{\ipa{ɖʐe˧-ʁwɤ˧}}} \textcolor{PineGreen}{\selectlanguage{french}argent} \zh{钱}  

\lhead{\firstmark}
\rhead{\botmark}

\subsection{\hspace{-0.5cm} {\Large \textcolor{darkblue}{\textbf{\ipa{ʁwɤ˧\textsubscript{a}}}}}\hspace{0.5cm}[\kern2pt{\textcolor{darkblue}{\textbf{\ipa{ʁwɤ˥}}}}\kern2pt]} \hypertarget{Rw7\string_Ma1}{}
\markboth{\textcolor{darkblue}{\textbf{\ipa{ʁwɤ˧\textsubscript{a}}}}}{}
\textcolor{teal}{\mytextsc{classificateur}} \hspace{4pt} Ton~: M\textsubscript{a}.
\textcolor{PineGreen}{\selectlanguage{french}Classificateur des tas / amoncellements (de céréales, de bois coupé...); littéralement: 'une montagne de'.} \zh{量词:堆(一堆粮食、一堆柴……)。} 
\lhead{\firstmark}
\rhead{\botmark}

\subsection{\hspace{-0.5cm} {\Large \textcolor{darkblue}{\textbf{\ipa{ʁwɤ˧\textsubscript{a}}}}}\hspace{0.5cm}[\kern2pt{\textcolor{darkblue}{\textbf{\ipa{ʁwɤ˥}}}}\kern2pt]} \hypertarget{Rw7\string_Ma1}{}
\markboth{\textcolor{darkblue}{\textbf{\ipa{ʁwɤ˧\textsubscript{a}}}}}{}
\textcolor{teal}{\mytextsc{verbe}} \hspace{4pt} Ton~: M\textsubscript{a}.
\textcolor{PineGreen}{\selectlanguage{french}Amasser, entasser.} \zh{堆 (例如:堆积泥土)。}  ¶ \textcolor{darkblue}{\textbf{\ipa{ɖɯ˧-ʁwɤ˧ tʰi˧-ʁwɤ˧}}} \textcolor{PineGreen}{\selectlanguage{french}faire un tas, amasser en tas} \zh{堆在一起}  
 ¶ \textcolor{darkblue}{\textbf{\ipa{ɖɯ˧-ʁwɤ˧ tʰi˧-tɕɯ˥}}} \textcolor{PineGreen}{\selectlanguage{french}ranger en tas} \zh{收拾成一堆}  
 ¶ \textcolor{darkblue}{\textbf{\ipa{tso˧\textasciitilde{}tso˧ | gɤ˩-ʁwɤ˥ lv̩˩}}} \textcolor{PineGreen}{\selectlanguage{french}entasser des objets} \zh{东西堆起来}  
 ¶ \textcolor{darkblue}{\textbf{\ipa{ɖɯ˧-ʁwɤ˥-lv̩˩}}} \textcolor{PineGreen}{\selectlanguage{french}rassembler en un tas (ex.: des piments épars, des noix, des fruits…)} \zh{收拾成一堆(如:有果子散在地上,把它们堆在一起)}  
 ¶ \textcolor{darkblue}{\textbf{\ipa{tso˧\textasciitilde{}tso˧ ʁwɤ˩}}} \textcolor{PineGreen}{\selectlanguage{french}entasser des choses} \zh{东西堆在一起}  

\lhead{\firstmark}
\rhead{\botmark}

\subsection{\hspace{-0.5cm} {\Large \textcolor{darkblue}{\textbf{\ipa{ʁwɤ˧lɑ˩-bi˩}}}}\hspace{0.5cm}[\kern2pt{\textcolor{darkblue}{\textbf{\ipa{ʁwɤ˧lɑ˩bi˧}}}}\kern2pt]} \hypertarget{Rw7\string_MlA\string_B-bi\string_B1}{}
\markboth{\textcolor{darkblue}{\textbf{\ipa{ʁwɤ˧lɑ˩-bi˩}}}}{}
\textcolor{teal}{\mytextsc{nom}} \hspace{4pt} Ton~: L\#-.
\textcolor{PineGreen}{\selectlanguage{french}Walabie, un village de la plaine de Yongning. Il est peuplé de Na et de Pumi.} \zh{瓦拉别(永宁的一个村落)。}  ¶ \textcolor{darkblue}{\textbf{\ipa{ə˧go˧-ʁwɤ˧, | ʁwɤ˧lɑ˩-bi˩, | bæ˧ʁwɤ˧, | tʰo˧tsʰe\#˥, | pi˧tsʰe˩-di˩, | pɤ˧dʑɤ˩-di˩, | ʁwɤ˧tv̩˧}}} \textcolor{PineGreen}{\selectlanguage{french}Villages au sortir de la plaine de Yongning; les deux premiers comportent une population na; le troisième est un village na; les suivants sont essentiellement des villages pumi/prinmi.} \zh{永宁背向泸沽湖方向经过的村落。前两个村落拥有相当大的摩梭人口比例,第三个村落是摩梭村,最后一个是普米村。}  

\lhead{\firstmark}
\rhead{\botmark}

\subsection{\hspace{-0.5cm} {\Large \textcolor{darkblue}{\textbf{\ipa{ʁwɤ˧qʰv̩˧}}}}\hspace{0.5cm}[\kern2pt{\textcolor{darkblue}{\textbf{\ipa{ʁwɤ˧qʰv̩˧}}}}\kern2pt]} \hypertarget{Rw7\string_Mq\string_hv\string_=\string_M1}{}
\markboth{\textcolor{darkblue}{\textbf{\ipa{ʁwɤ˧qʰv̩˧}}}}{}
\textcolor{teal}{\mytextsc{nom}} \hspace{4pt} Ton~: M.
\textcolor{PineGreen}{\selectlanguage{french}Grotte, caverne.} \zh{山洞。}  \zh{量词}: \textcolor{darkblue}{\textbf{\ipa{ɭɯ˧}}}  \mytextsc{clf}~: \textcolor{darkblue}{\textbf{\ipa{ɭɯ˧}}} \textit{Voir~:} \hyperlink{}{\textcolor{darkblue}{\textbf{\ipa{ʁwɤ˧qʰv̩˧dʑɯ\#˥}}}} 
\lhead{\firstmark}
\rhead{\botmark}

\subsection{\hspace{-0.5cm} {\Large \textcolor{darkblue}{\textbf{\ipa{ʁwɤ˧qʰv̩˧dʑɯ\#˥}}}}\hspace{0.5cm}[\kern2pt{\textcolor{darkblue}{\textbf{\ipa{ʁwɤ˧qʰv̩˧dʑɯ˧}}}}\kern2pt]} \hypertarget{Rw7\string_Mq\string_hv\string_=\string_Mdz£M\#\string_T1}{}
\markboth{\textcolor{darkblue}{\textbf{\ipa{ʁwɤ˧qʰv̩˧dʑɯ\#˥}}}}{}
\textcolor{teal}{\mytextsc{nom}} \hspace{4pt} Ton~: \#H.
\textcolor{PineGreen}{\selectlanguage{french}Grotte, caverne (où il est facile d'entrer).} \zh{山洞。}  \zh{量词}: \textcolor{darkblue}{\textbf{\ipa{ɭɯ˧}}}  \mytextsc{clf}~: \textcolor{darkblue}{\textbf{\ipa{ɭɯ˧}}} \textit{Voir~:} \hyperlink{}{\textcolor{darkblue}{\textbf{\ipa{ʁwɤ˧qʰv̩˧}}}} 
\lhead{\firstmark}
\rhead{\botmark}

\subsection{\hspace{-0.5cm} {\Large \textcolor{darkblue}{\textbf{\ipa{ʁwɤ˧\textasciitilde{}ʁwɤ˥\textsubscript{a}}}}}\hspace{0.5cm}[\kern2pt{\textcolor{darkblue}{\textbf{\ipa{ʁwɤ˩ʁwɤ˩˥}}}}\kern2pt]} \hypertarget{Rw7\string_M~Rw7\string_Ta1}{}
\markboth{\textcolor{darkblue}{\textbf{\ipa{ʁwɤ˧\textasciitilde{}ʁwɤ˥\textsubscript{a}}}}}{}
\textcolor{teal}{\mytextsc{verbe}} \hspace{4pt} Ton~: L\textsubscript{a}.
\textcolor{PineGreen}{\selectlanguage{french}Discuter, négocier.} \zh{商量。}  ¶ \textcolor{darkblue}{\textbf{\ipa{ɖɯ˧-ʁwɤ˧\textasciitilde{}ʁwɤ˥-ɻ̍˩}}} \textcolor{PineGreen}{\selectlanguage{french}\mytextsc{délimitatif} \string_ \mytextsc{red} \mytextsc{inchoatif}} \zh{商量商量}  

\lhead{\firstmark}
\rhead{\botmark}

\subsection{\hspace{-0.5cm} {\Large \textcolor{darkblue}{\textbf{\ipa{ʁwɤ˧tv̩˧}}}}\hspace{0.5cm}[\kern2pt{\textcolor{darkblue}{\textbf{\ipa{ʁwɤ˧tv̩˧}}}}\kern2pt]} \hypertarget{Rw7\string_Mtv\string_=\string_M1}{}
\markboth{\textcolor{darkblue}{\textbf{\ipa{ʁwɤ˧tv̩˧}}}}{}
\textcolor{teal}{\mytextsc{nom}} \hspace{4pt} Ton~: M.
\textcolor{PineGreen}{\selectlanguage{french}Un village proche de Wenquan.} \zh{温泉乡的一个村落。}  ¶ \textcolor{darkblue}{\textbf{\ipa{ʁwɤ˧tv̩˧-ʁwɤ˧}}} \textcolor{PineGreen}{\selectlanguage{french}même sens} \zh{同上}  
 ¶ \textcolor{darkblue}{\textbf{\ipa{ə˧go˧-ʁwɤ˧, | ʁwɤ˧lɑ˩-bi˩, | bæ˧ʁwɤ˧, | tʰo˧tsʰe\#˥, | pi˧tsʰe˩-di˩, | pɤ˧dʑɤ˩-di˩, | ʁwɤ˧tv̩˧}}} \textcolor{PineGreen}{\selectlanguage{french}Villages au sortir de la plaine de Yongning; les deux premiers comportent une population na; le troisième est un village na; les suivants sont essentiellement des villages pumi/prinmi.} \zh{永宁背向泸沽湖方向经过的村落。前两个村落拥有相当大的摩梭人口比例,第三个村落是摩梭村,最后一个是普米村。}  
 ¶ \textcolor{darkblue}{\textbf{\ipa{ʁwɤ˧tv̩˧: | bɤ˧!}}} \textcolor{PineGreen}{\selectlanguage{french}\textcolor{darkblue}{\textbf{\ipa{/ʁwɤ˧tv̩˧/}}}, c'est un village pumi!} \zh{fv:/ʁwɤ˧tv̩˧/是一个普米族村落!}  

\lhead{\firstmark}
\rhead{\botmark}

\subsection{\hspace{-0.5cm} {\Large \textcolor{darkblue}{\textbf{\ipa{ʁwɤ˧ʐv̩\#˥}}}}\hspace{0.5cm}[\kern2pt{\textcolor{darkblue}{\textbf{\ipa{ʁwɤ˧ʐv̩˧}}}}\kern2pt]} \hypertarget{Rw7\string_Mz`v\string_=\#\string_T1}{}
\markboth{\textcolor{darkblue}{\textbf{\ipa{ʁwɤ˧ʐv̩\#˥}}}}{}
\textcolor{teal}{\mytextsc{nom}} \hspace{4pt} Ton~: \#H.
\textcolor{PineGreen}{\selectlanguage{french}Qiansuo (localité perçue par F4 comme comportant beaucoup de Yi, et des Chinois/Han, en plus des Na, d'où des contacts linguistiques/emprunts/mélanges).} \zh{前所。}  ¶ \textcolor{darkblue}{\textbf{\ipa{ʁwɤ˧ʐv̩˧-lo˩mæ˩}}} \textcolor{PineGreen}{\selectlanguage{french}même sens} \zh{同上}  
 ¶ \textcolor{darkblue}{\textbf{\ipa{ʁwɤ˧ʐv̩˧, | jɤ˧qʰɑ˧ dʑɤ˥; | hwɤ˧li˧-hɑ˧ mɤ˧-dʑo˧˥!}}} \textcolor{PineGreen}{\selectlanguage{french}dicton ancien: “A Qiansuo, le sarrasin amer pousse à merveille; les chats n'y ont rien à manger!” Explication: les chats ne mangent pas de sarrasin. / le sarrasin est bon, le chat n'aura rien à manger! (le chat passe en miaulant; mais il n'y a rien pour lui!) yyyy} \zh{俗语:“前所,苦荞(庄稼)很好。猫,没得吃!”(说明:猫不吃苦荞。)}  
 ¶ \textcolor{darkblue}{\textbf{\ipa{ʁwɤ˧ʐv˧, | jɤ˧qʰɑ˧ dʑɤ˥, | hwɤ˧li˧˥ | hɑ˧ mɤ˧-dʑo˧!}}} \textcolor{PineGreen}{\selectlanguage{french}comme ci-dessus} \zh{同上}  

\lhead{\firstmark}
\rhead{\botmark}

\subsection{\hspace{-0.5cm} {\Large \textcolor{darkblue}{\textbf{\ipa{*ʁwɤ˩\textsubscript{a}}}}}\hspace{0.5cm}[\kern2pt{\textcolor{darkblue}{\textbf{\ipa{ʁwɤ˩˥}}}}\kern2pt]} \hypertarget{*Rw7\string_Ba1}{}
\markboth{\textcolor{darkblue}{\textbf{\ipa{*ʁwɤ˩\textsubscript{a}}}}}{}
\textcolor{teal}{\mytextsc{verbe}} \hspace{4pt} Ton~: L\textsubscript{a}.
\textcolor{PineGreen}{\selectlanguage{french}Discuter, négocier (racine extraite de la forme rédupliquée).} \zh{商量(单音节)。} 
\lhead{\firstmark}
\rhead{\botmark}

\subsection{\hspace{-0.5cm} {\Large \textcolor{darkblue}{\textbf{\ipa{ʁwɤ˩ʁo˩}}}}\hspace{0.5cm}[\kern2pt{\textcolor{darkblue}{\textbf{\ipa{ʁwɤ˩ʁo˩˥}}}}\kern2pt]} \hypertarget{Rw7\string_BRo\string_B1}{}
\markboth{\textcolor{darkblue}{\textbf{\ipa{ʁwɤ˩ʁo˩}}}}{}
\textcolor{teal}{\mytextsc{nom}} \hspace{4pt} Ton~: L.
\textcolor{PineGreen}{\selectlanguage{french}Collines, versant des montagnes (pas forcément très escarpé, plutôt collines que très fortes pentes).} \zh{山坡。}  ¶ \textcolor{darkblue}{\textbf{\ipa{ʁwɤ˩ʁo˩ dʑɤ˥bv̩˩ ə˩bi˩?}}} \textcolor{PineGreen}{\selectlanguage{french}tu viens te détendre sur la montagne?} \zh{去山上玩,好吗?}  
 \zh{量词}: \textcolor{darkblue}{\textbf{\ipa{ʁwɤ˧}}}  \mytextsc{clf}~: \textcolor{darkblue}{\textbf{\ipa{ʁwɤ˧}}} 
\lhead{\firstmark}
\rhead{\botmark}

\newpage
\section*{\centering- \textcolor{darkblue}{\textbf{\ipa{s}}} -}
\subsection{\hspace{-0.5cm} {\Large \textcolor{darkblue}{\textbf{\ipa{sɑ˥}}} \textsubscript{1}}\hspace{0.5cm}[\kern2pt{\textcolor{darkblue}{\textbf{\ipa{sɑ˥}}}}\kern2pt]} \hypertarget{sA\string_T1}{}
\markboth{\textcolor{darkblue}{\textbf{\ipa{sɑ˥}}} \textsubscript{1}}{}
\textcolor{teal}{\mytextsc{nom}} \hspace{4pt} Ton~: \#H.
\ding{202} \textcolor{PineGreen}{\selectlanguage{french}Lin, \textit{Linum usitatissimum}, plante textile et oléagineuse.} \zh{亚麻。}  \zh{量词}: \textcolor{darkblue}{\textbf{\ipa{qʰwæ˧˥}}} \ding{203} \textcolor{PineGreen}{\selectlanguage{french}Chanvre, \textit{Cannabis sativa}, plante textile.} \zh{火麻、胡麻。}  \zh{量词}: \textcolor{darkblue}{\textbf{\ipa{qʰwæ˧˥}}}  \mytextsc{clf}~: \textcolor{darkblue}{\textbf{\ipa{qʰwæ˧˥}}} \textcolor{darkblue}{\textbf{\ipa{qʰwæ˧˥}}} \textcolor{PineGreen}{\selectlanguage{french}filaments avant le filage}
\lhead{\firstmark}
\rhead{\botmark}

\subsection{\hspace{-0.5cm} {\Large \textcolor{darkblue}{\textbf{\ipa{sɑ˥}}} \textsubscript{2}}\hspace{0.5cm}[\kern2pt{\textcolor{darkblue}{\textbf{\ipa{sɑ˥}}}}\kern2pt]} \hypertarget{sA\string_T2}{}
\markboth{\textcolor{darkblue}{\textbf{\ipa{sɑ˥}}} \textsubscript{2}}{}
\textcolor{teal}{\mytextsc{classificateur}} \hspace{4pt} Ton~: H*.
\textcolor{PineGreen}{\selectlanguage{french}Classificateur des choses/objets, utilisé seulement en tournure négative: 'quoi que ce soit'.} \zh{量词:样,如:‘一样都没有’。}  ¶ \textcolor{darkblue}{\textbf{\ipa{ɖɯ˧-sɑ˥ | mɤ˧-dʑo˧!}}} \textcolor{PineGreen}{\selectlanguage{french}Il n’y a rien du tout [à manger]! (phrase polie qd on invite quelqu'un à manger: on prie le convive d’excuser la pauvreté des mets proposés)} \zh{一样也没有! / 没什么东西!(请客时的礼貌、自我贬低说法:请客人原谅菜不够丰盛)}  
\textit{Voir~:} \hyperlink{}{\textcolor{darkblue}{\textbf{\ipa{so˥}}} \textsubscript{2}} 
\lhead{\firstmark}
\rhead{\botmark}

\subsection{\hspace{-0.5cm} {\Large \textcolor{darkblue}{\textbf{\ipa{sɑ˧bo\#˥}}}}\hspace{0.5cm}[\kern2pt{\textcolor{darkblue}{\textbf{\ipa{sɑ˧bo˧˥}}}}\kern2pt]} \hypertarget{sA\string_Mbo\#\string_T1}{}
\markboth{\textcolor{darkblue}{\textbf{\ipa{sɑ˧bo\#˥}}}}{}
\textcolor{teal}{\mytextsc{nom}} \hspace{4pt} Ton~: \#H.
\textcolor{PineGreen}{\selectlanguage{french}Quenouille: instrument en bois pour enrouler le fil, pour filer le chanvre.} \zh{卷线杆、拉线棒。}  ¶ \textcolor{darkblue}{\textbf{\ipa{sɑ˧bo˧-di˧˥}}} \textcolor{PineGreen}{\selectlanguage{french}même sens} \zh{同上}  
 \zh{量词}: \textcolor{darkblue}{\textbf{\ipa{nɑ˧}}}  \mytextsc{clf}~: \textcolor{darkblue}{\textbf{\ipa{nɑ˧}}} 
\lhead{\firstmark}
\rhead{\botmark}

\subsection{\hspace{-0.5cm} {\Large \textcolor{darkblue}{\textbf{\ipa{sɑ˧pʰv̩˧˥}}}}\hspace{0.5cm}[\kern2pt{\textcolor{darkblue}{\textbf{\ipa{sɑ˧pʰv̩˧}}}}\kern2pt]} \hypertarget{sA\string_Mp\string_hv\string_=\string_M\string_T1}{}
\markboth{\textcolor{darkblue}{\textbf{\ipa{sɑ˧pʰv̩˧˥}}}}{}
\textcolor{teal}{\mytextsc{nom}} \hspace{4pt} Ton~: MH\#.
\textcolor{PineGreen}{\selectlanguage{french}Fil de lin, \textit{Cannabis sativa}.} \zh{麻线。}  ¶ \textcolor{darkblue}{\textbf{\ipa{sɑ˧pʰv̩˧-sɑ˧jɤ˥}}} \textcolor{PineGreen}{\selectlanguage{french}fil de lin} \zh{麻线}  
 \zh{量词}: \textcolor{darkblue}{\textbf{\ipa{ɭɯ˧}}}  \mytextsc{clf}~: \textcolor{darkblue}{\textbf{\ipa{ɭɯ˧}}} 
\lhead{\firstmark}
\rhead{\botmark}

\subsection{\hspace{-0.5cm} {\Large \textcolor{darkblue}{\textbf{\ipa{sɑ˧tɕɯ˧}}}}\hspace{0.5cm}[\kern2pt{\textcolor{darkblue}{\textbf{\ipa{sɑ˧tɕɯ˩}}}}\kern2pt]} \hypertarget{sA\string_Mts£M\string_M1}{}
\markboth{\textcolor{darkblue}{\textbf{\ipa{sɑ˧tɕɯ˧}}}}{}
\textcolor{teal}{\mytextsc{nom}} \hspace{4pt} Ton~: M.
\textcolor{PineGreen}{\selectlanguage{french}Organe sexuel féminin, vagin (mot tabou).} \zh{女生殖器。}  \zh{量词}: \textcolor{darkblue}{\textbf{\ipa{ɭɯ˧}}}  \mytextsc{clf}~: \textcolor{darkblue}{\textbf{\ipa{ɭɯ˧}}} 
\lhead{\firstmark}
\rhead{\botmark}

\subsection{\hspace{-0.5cm} {\Large \textcolor{darkblue}{\textbf{\ipa{sɑ˧tsʰv̩˩}}}}\hspace{0.5cm}[\kern2pt{\textcolor{darkblue}{\textbf{\ipa{sɑ˧tsʰv̩˧}}}}\kern2pt]} \hypertarget{sA\string_Mts\string_hv\string_=\string_B1}{}
\markboth{\textcolor{darkblue}{\textbf{\ipa{sɑ˧tsʰv̩˩}}}}{}
\textcolor{teal}{\mytextsc{nom}} \hspace{4pt} Ton~: L\#.
\textcolor{PineGreen}{\selectlanguage{french}Vinaigre.} \zh{酸醋(汉语借词)。}  Emprunt~: chinois  \zh{酸醋}
\textit{Voir~:} \hyperlink{}{\textcolor{darkblue}{\textbf{\ipa{tsʰv̩˩˥}}}} 
\lhead{\firstmark}
\rhead{\botmark}

\subsection{\hspace{-0.5cm} {\Large \textcolor{darkblue}{\textbf{\ipa{sɑ˩mi˩}}}}\hspace{0.5cm}[\kern2pt{\textcolor{darkblue}{\textbf{\ipa{xxxx non-correspondance entre le nombre de morphèmes et le nombre de tons de morphèmes}}}}\kern2pt]} \hypertarget{sA\string_Bmi\string_B1}{}
\markboth{\textcolor{darkblue}{\textbf{\ipa{sɑ˩mi˩}}}}{}
\textcolor{teal}{\mytextsc{nom}} \hspace{4pt} Ton~: L.
\textcolor{PineGreen}{\selectlanguage{french}Chanvre indien, kanja, marijuana, \textit{Cannabis indica} (plante euphorisante/psychotrope, qui est également comestible: les Na en tiraient de l'huile).} \zh{大麻。}  ¶ \textcolor{darkblue}{\textbf{\ipa{sɑ˩mi˩-mæ˩ɻæ˥, | dzɯ˧-kv̩˩!}}} \textcolor{PineGreen}{\selectlanguage{french}L'huile de lin, c'est comestible/ça se mange!} \zh{大麻油,是可以吃的!}  
 \zh{量词}: \textcolor{darkblue}{\textbf{\ipa{kɤ˧˥}}}  \mytextsc{clf}~: \textcolor{darkblue}{\textbf{\ipa{kɤ˧˥}}} 
\lhead{\firstmark}
\rhead{\botmark}

\subsection{\hspace{-0.5cm} {\Large \textcolor{darkblue}{\textbf{\ipa{sɑ˩tsʰi˩}}} \textsubscript{1}}\hspace{0.5cm}[\kern2pt{\textcolor{darkblue}{\textbf{\ipa{sɑ˧tsʰi˧}}}}\kern2pt]} \hypertarget{sA\string_Bts\string_hi\string_B1}{}
\markboth{\textcolor{darkblue}{\textbf{\ipa{sɑ˩tsʰi˩}}} \textsubscript{1}}{}
\textcolor{teal}{\mytextsc{nom}} \hspace{4pt} Ton~: L.
\textcolor{PineGreen}{\selectlanguage{french}Rame.} \zh{桨。}  \zh{量词}: \textcolor{darkblue}{\textbf{\ipa{nɑ˧}}}  \mytextsc{clf}~: \textcolor{darkblue}{\textbf{\ipa{nɑ˧}}} \textit{Voir~:} \hyperlink{}{\textcolor{darkblue}{\textbf{\ipa{sɑ˩tsʰi˩}}} \textsubscript{2}} 
\lhead{\firstmark}
\rhead{\botmark}

\subsection{\hspace{-0.5cm} {\Large \textcolor{darkblue}{\textbf{\ipa{sɑ˩tsʰi˩}}} \textsubscript{2}}\hspace{0.5cm}[\kern2pt{\textcolor{darkblue}{\textbf{\ipa{sɑ˩tsʰi˩˥}}}}\kern2pt]} \hypertarget{sA\string_Bts\string_hi\string_B2}{}
\markboth{\textcolor{darkblue}{\textbf{\ipa{sɑ˩tsʰi˩}}} \textsubscript{2}}{}
\textcolor{teal}{\mytextsc{nom}} \hspace{4pt} Ton~: L.
\textcolor{PineGreen}{\selectlanguage{french}Pale en bois utilisée pour touiller la pâtée des cochons; ressemble à une rame: l'ustensile de cuisine est de taille nettement plus petite que la rame des bateaux, mais de forme similaire.} \zh{像桨的木头工具,来搅拌猪食。} \textit{Voir~:} \hyperlink{}{\textcolor{darkblue}{\textbf{\ipa{sɑ˩tsʰi˩}}} \textsubscript{1}} 
\lhead{\firstmark}
\rhead{\botmark}

\subsection{\hspace{-0.5cm} {\Large \textcolor{darkblue}{\textbf{\ipa{sɑ˧˥}}}}\hspace{0.5cm}[\kern2pt{\textcolor{darkblue}{\textbf{\ipa{sɑ˧˥}}}}\kern2pt]} \hypertarget{sA\string_M\string_T1}{}
\markboth{\textcolor{darkblue}{\textbf{\ipa{sɑ˧˥}}}}{}
\textcolor{teal}{\mytextsc{verbe}} \hspace{4pt} Ton~: MH.
\textcolor{PineGreen}{\selectlanguage{french}Livrer (à destination).} \zh{运送(货到目的地)。}  ¶ \textcolor{darkblue}{\textbf{\ipa{le˧-sɑ˧-tʰi˥-ki˩}}} \textcolor{PineGreen}{\selectlanguage{french}même sens: livrer (un objet, une marchandise)} \zh{送(东西到人家里)}  

\lhead{\firstmark}
\rhead{\botmark}

\subsection{\hspace{-0.5cm} {\Large \textcolor{darkblue}{\textbf{\ipa{sɑ˧˥\textsubscript{a}}}}}\hspace{0.5cm}[\kern2pt{\textcolor{darkblue}{\textbf{\ipa{sɑ˧˥}}}}\kern2pt]} \hypertarget{sA\string_M\string_Ta1}{}
\markboth{\textcolor{darkblue}{\textbf{\ipa{sɑ˧˥\textsubscript{a}}}}}{}
\textcolor{teal}{\mytextsc{classificateur}} \hspace{4pt} Ton~: MH\textsubscript{a}.
\textcolor{PineGreen}{\selectlanguage{french}Classificateur des pattes de cochon conservées.} \zh{量词:腊猪脚(烟熏腊猪蹄子)(一只)。}  ¶ \textcolor{darkblue}{\textbf{\ipa{ʂe˧sɑ˩ | ɖɯ˧-sɑ˧˥}}} \textcolor{PineGreen}{\selectlanguage{french}une patte de cochon conservée (viande des membres du cochon, conservée --séchée-- avec l'os)} \zh{一只腊猪脚}  

\lhead{\firstmark}
\rhead{\botmark}

\subsection{\hspace{-0.5cm} {\Large \textcolor{darkblue}{\textbf{\ipa{sæ˧tsʰɤ˩}}}}\hspace{0.5cm}[\kern2pt{\textcolor{darkblue}{\textbf{\ipa{sæ˩tsʰɤ˩˥}}}}\kern2pt]} \hypertarget{s\{\string_Mts\string_h7\string_B1}{}
\markboth{\textcolor{darkblue}{\textbf{\ipa{sæ˧tsʰɤ˩}}}}{}
\textcolor{teal}{\mytextsc{nom}} \hspace{4pt} Ton~: L\#.
\textcolor{PineGreen}{\selectlanguage{french}Légumes en saumure.} \zh{酸菜(汉语借词)、泡菜。}  Emprunt~: chinois  \zh{酸菜}

\lhead{\firstmark}
\rhead{\botmark}

\subsection{\hspace{-0.5cm} {\Large \textcolor{darkblue}{\textbf{\ipa{se˥}}}}\hspace{0.5cm}[\kern2pt{\textcolor{darkblue}{\textbf{\ipa{se˥}}}}\kern2pt]} \hypertarget{se\string_T1}{}
\markboth{\textcolor{darkblue}{\textbf{\ipa{se˥}}}}{}
\textcolor{teal}{\mytextsc{verbe}} \hspace{4pt} Ton~: H.
\textcolor{PineGreen}{\selectlanguage{french}Marcher.} \zh{走、走路。}  ¶ \textcolor{darkblue}{\textbf{\ipa{le˧-se˥-ze˩}}} \textcolor{PineGreen}{\selectlanguage{french}\mytextsc{accomp} \string_ \mytextsc{pfv}} \zh{走了}  
 ¶ \textcolor{darkblue}{\textbf{\ipa{se˧-ho˥-ze˩!}}} \textcolor{PineGreen}{\selectlanguage{french}[Le bébé] va bientôt marcher / va bientôt savoir marcher!} \zh{(婴儿)很快就学会走路了!}  
 ¶ \textcolor{darkblue}{\textbf{\ipa{ʐɤ˩mi˩-qo˥ | so˩-hɑ̃˩ se˩˥}}} \textcolor{PineGreen}{\selectlanguage{french}passer trois nuits en route / faire un voyage qui va durer trois jours (au sujet d'un trajet de trois nuits de Lijiang à Hanoi: un train de nuit; un car de nuit le lendemain; et un second train de nuit le troisième jour)} \zh{走在路上三天时间、走三天}  

\lhead{\firstmark}
\rhead{\botmark}

\subsection{\hspace{-0.5cm} {\Large \textcolor{darkblue}{\textbf{\ipa{se˧}}}}\hspace{0.5cm}[\kern2pt{\textcolor{darkblue}{\textbf{\ipa{se˩˥}}}}\kern2pt]} \hypertarget{se\string_M1}{}
\markboth{\textcolor{darkblue}{\textbf{\ipa{se˧}}}}{}
\textcolor{teal}{\mytextsc{nom}} \hspace{4pt} Ton~: M.
\textcolor{PineGreen}{\selectlanguage{french}\textit{Naemorhedus goral}. Le même terme est employé par la locutrice pour décrire des photos de \textit{Pseudois schaeferi}, sorte de bouquetin.} \zh{岩羊。}  \zh{量词}: \textcolor{darkblue}{\textbf{\ipa{pʰo˧˥}}}  \mytextsc{clf}~: \textcolor{darkblue}{\textbf{\ipa{pʰo˧˥}}} 
\lhead{\firstmark}
\rhead{\botmark}

\subsection{\hspace{-0.5cm} {\Large \textcolor{darkblue}{\textbf{\ipa{se˧gi\#˥}}}}\hspace{0.5cm}[\kern2pt{\textcolor{darkblue}{\textbf{\ipa{se˧gi˧}}}}\kern2pt]} \hypertarget{se\string_Mgi\#\string_T1}{}
\markboth{\textcolor{darkblue}{\textbf{\ipa{se˧gi\#˥}}}}{}
\textcolor{teal}{\mytextsc{nom}} \hspace{4pt} Ton~: \#H.
\textcolor{PineGreen}{\selectlanguage{french}Nom anciennement donné par les Tibétains à la montagne \textcolor{darkblue}{\textbf{\ipa{/kɤ˧mv̩˧˥/}}} (nom chinois: Gemu).} \zh{格姆山的藏语名字。}  ¶ \textcolor{darkblue}{\textbf{\ipa{se˧gi˧-kɤ˩mv̩˩}}} \textcolor{PineGreen}{\selectlanguage{french}même sens} \zh{同上}  

\lhead{\firstmark}
\rhead{\botmark}

\subsection{\hspace{-0.5cm} {\Large \textcolor{darkblue}{\textbf{\ipa{se˧kʰɯ˩}}}}\hspace{0.5cm}[\kern2pt{\textcolor{darkblue}{\textbf{\ipa{se˩kʰɯ˥}}}}\kern2pt]} \hypertarget{se\string_Mk\string_hM\string_B1}{}
\markboth{\textcolor{darkblue}{\textbf{\ipa{se˧kʰɯ˩}}}}{}
\textcolor{teal}{\mytextsc{nom}} \hspace{4pt} Ton~: L\#.
\textcolor{PineGreen}{\selectlanguage{french}Satin.} \zh{缎子。}  ¶ \textcolor{darkblue}{\textbf{\ipa{se˧kʰɯ˩-ʁo˩ni˩}}} \textcolor{PineGreen}{\selectlanguage{french}coiffe en satin} \zh{缎子发带}  
 \zh{量词}: \textcolor{darkblue}{\textbf{\ipa{kʰɯ˩}}}  \mytextsc{clf}~: \textcolor{darkblue}{\textbf{\ipa{kʰɯ˩}}} 
\lhead{\firstmark}
\rhead{\botmark}

\subsection{\hspace{-0.5cm} {\Large \textcolor{darkblue}{\textbf{\ipa{se˧mi\#˥}}}}\hspace{0.5cm}[\kern2pt{\textcolor{darkblue}{\textbf{\ipa{se˧mi˧}}}}\kern2pt]} \hypertarget{se\string_Mmi\#\string_T1}{}
\markboth{\textcolor{darkblue}{\textbf{\ipa{se˧mi\#˥}}}}{}
\textcolor{teal}{\mytextsc{nom}} \hspace{4pt} Ton~: \#H.
\textcolor{PineGreen}{\selectlanguage{french}\textit{Naemorhedus goral} femelle.} \zh{母岩羊。}  \zh{量词}: \textcolor{darkblue}{\textbf{\ipa{mi˩}}}  \mytextsc{clf}~: \textcolor{darkblue}{\textbf{\ipa{mi˩}}} 
\lhead{\firstmark}
\rhead{\botmark}

\subsection{\hspace{-0.5cm} {\Large \textcolor{darkblue}{\textbf{\ipa{se˧nɑ\#˥}}}}\hspace{0.5cm}[\kern2pt{\textcolor{darkblue}{\textbf{\ipa{se˧nɑ˩}}}}\kern2pt]} \hypertarget{se\string_MnA\#\string_T1}{}
\markboth{\textcolor{darkblue}{\textbf{\ipa{se˧nɑ\#˥}}}}{}
\textcolor{teal}{\mytextsc{adjectif}} \hspace{4pt} Ton~: \#H.
\textcolor{PineGreen}{\selectlanguage{french}Avare.} \zh{吝啬。}  ¶ \textcolor{darkblue}{\textbf{\ipa{ʈʂʰɯ˧ | se˧nɑ˧-hĩ˧ ɖɯ˧-v̩˧ ɲi˩!}}} \textcolor{PineGreen}{\selectlanguage{french}C'est quelqu'un d'avare!} \zh{他是一个吝啬的人!}  
 ¶ \textcolor{darkblue}{\textbf{\ipa{ʈʂʰɯ˧ | ə˧-se˧nɑ˧? - se˧nɑ˧ | ʐwæ˩˥!}}} \textcolor{PineGreen}{\selectlanguage{french}Est-il avare? - Oui, très avare!} \zh{他吝啬吗? - 非常吝啬!}  

\lhead{\firstmark}
\rhead{\botmark}

\subsection{\hspace{-0.5cm} {\Large \textcolor{darkblue}{\textbf{\ipa{se˧pʰɤ˧}}}}\hspace{0.5cm}[\kern2pt{\textcolor{darkblue}{\textbf{\ipa{se˧pʰɤ˩}}}}\kern2pt]} \hypertarget{se\string_Mp\string_h7\string_M1}{}
\markboth{\textcolor{darkblue}{\textbf{\ipa{se˧pʰɤ˧}}}}{}
\textcolor{teal}{\mytextsc{nom}} \hspace{4pt} Ton~: M.
\textcolor{PineGreen}{\selectlanguage{french}Complications.} \zh{大惊小怪,麻烦。}  ¶ \textcolor{darkblue}{\textbf{\ipa{se˧pʰɤ˧ ʝi˧}}} \textcolor{PineGreen}{\selectlanguage{french}se faire toute une affaire de quelque chose, s’en faire au point de porter comme une pierre dans le cœur} \zh{小事大作}  
 \zh{量词}: \textcolor{darkblue}{\textbf{\ipa{kʰwɤ˥}}}  \mytextsc{clf}~: \textcolor{darkblue}{\textbf{\ipa{kʰwɤ˥}}} 
\lhead{\firstmark}
\rhead{\botmark}

\subsection{\hspace{-0.5cm} {\Large \textcolor{darkblue}{\textbf{\ipa{se˧pʰv̩\#˥}}}}\hspace{0.5cm}[\kern2pt{\textcolor{darkblue}{\textbf{\ipa{se˧pʰv̩˩}}}}\kern2pt]} \hypertarget{se\string_Mp\string_hv\string_=\#\string_T1}{}
\markboth{\textcolor{darkblue}{\textbf{\ipa{se˧pʰv̩\#˥}}}}{}
\textcolor{teal}{\mytextsc{nom}} \hspace{4pt} Ton~: \#H.
\textcolor{PineGreen}{\selectlanguage{french}\textit{Naemorhedus goral} mâle.} \zh{公岩羊。}  \zh{量词}: \textcolor{darkblue}{\textbf{\ipa{mi˩}}}  \mytextsc{clf}~: \textcolor{darkblue}{\textbf{\ipa{mi˩}}} 
\lhead{\firstmark}
\rhead{\botmark}

\subsection{\hspace{-0.5cm} {\Large \textcolor{darkblue}{\textbf{\ipa{se˧ʂɯ˩}}}}\hspace{0.5cm}[\kern2pt{\textcolor{darkblue}{\textbf{\ipa{se˧ʂɯ˧}}}}\kern2pt]} \hypertarget{se\string_Ms`M\string_B1}{}
\markboth{\textcolor{darkblue}{\textbf{\ipa{se˧ʂɯ˩}}}}{}
\textcolor{teal}{\mytextsc{verbe}} \hspace{4pt} Ton~: L\#.
\textcolor{PineGreen}{\selectlanguage{french}Gaspiller.} \zh{浪费。}  ¶ \textcolor{darkblue}{\textbf{\ipa{ɖwæ˧˥ | se˧ʂɯ˩!}}} \textcolor{PineGreen}{\selectlanguage{french}C'est du gaspillage!} \zh{很浪费! / 太浪费了!}  

\lhead{\firstmark}
\rhead{\botmark}

\subsection{\hspace{-0.5cm} {\Large \textcolor{darkblue}{\textbf{\ipa{se˧tʰo˩}}}}\hspace{0.5cm}[\kern2pt{\textcolor{darkblue}{\textbf{\ipa{se˩tʰo˩˥}}}}\kern2pt]} \hypertarget{se\string_Mt\string_ho\string_B1}{}
\markboth{\textcolor{darkblue}{\textbf{\ipa{se˧tʰo˩}}}}{}
\textcolor{teal}{\mytextsc{nom}} \hspace{4pt} Ton~: L.
\textcolor{PineGreen}{\selectlanguage{french}Tenon.} \zh{榫头(汉语借词)。}  Emprunt~: chinois  \zh{榫头}
 ¶ \textcolor{darkblue}{\textbf{\ipa{se˧tʰo˩ | ɖɯ˧-ɭɯ˧}}} \textcolor{PineGreen}{\selectlanguage{french}un tenon} \zh{一个榫头}  
 \zh{量词}: \textcolor{darkblue}{\textbf{\ipa{ɭɯ˧}}}  \mytextsc{clf}~: \textcolor{darkblue}{\textbf{\ipa{ɭɯ˧}}} 
\lhead{\firstmark}
\rhead{\botmark}

\subsection{\hspace{-0.5cm} {\Large \textcolor{darkblue}{\textbf{\ipa{se˧zo\#˥}}}}\hspace{0.5cm}[\kern2pt{\textcolor{darkblue}{\textbf{\ipa{se˧zo˧}}}}\kern2pt]} \hypertarget{se\string_Mzo\#\string_T1}{}
\markboth{\textcolor{darkblue}{\textbf{\ipa{se˧zo\#˥}}}}{}
\textcolor{teal}{\mytextsc{nom}} \hspace{4pt} Ton~: \#H.
\textcolor{PineGreen}{\selectlanguage{french}Petit de \textit{Naemorhedus goral}.} \zh{小岩羊。}  \zh{量词}: \textcolor{darkblue}{\textbf{\ipa{ɭɯ˧}}}  \mytextsc{clf}~: \textcolor{darkblue}{\textbf{\ipa{ɭɯ˧}}} 
\lhead{\firstmark}
\rhead{\botmark}

\subsection{\hspace{-0.5cm} {\Large \textcolor{darkblue}{\textbf{\ipa{se˧ʐɯ˩}}}}\hspace{0.5cm}[\kern2pt{\textcolor{darkblue}{\textbf{\ipa{se˧ʐɯ˧}}}}\kern2pt]} \hypertarget{se\string_Mz`M\string_B1}{}
\markboth{\textcolor{darkblue}{\textbf{\ipa{se˧ʐɯ˩}}}}{}
\textcolor{teal}{\mytextsc{nom}} \hspace{4pt} Ton~: L\#.
\textcolor{PineGreen}{\selectlanguage{french}Anniversaire.} \zh{生日(汉语借词)。}  Emprunt~: chinois  \zh{生日}
 ¶ \textcolor{darkblue}{\textbf{\ipa{se˧ʐɯ˩ ko˩}}} \textcolor{PineGreen}{\selectlanguage{french}fêter un anniversaire} \zh{过生日}  

\lhead{\firstmark}
\rhead{\botmark}

\subsection{\hspace{-0.5cm} {\Large \textcolor{darkblue}{\textbf{\ipa{‑se˩}}}}\hspace{0.5cm}[\kern2pt{\textcolor{darkblue}{\textbf{\ipa{se˩˥}}}}\kern2pt]} \hypertarget{‑se\string_B1}{}
\markboth{\textcolor{darkblue}{\textbf{\ipa{‑se˩}}}}{}
\textcolor{teal}{\mytextsc{suffixe}} \hspace{4pt} Ton~: L.
\textcolor{PineGreen}{\selectlanguage{french}Suffixe indiquant l'achèvement d'une action: l'action est conduite à son terme.} \zh{\mytextsc{完成。}}  ¶ \textcolor{darkblue}{\textbf{\ipa{se˩-ze˥!}}} \textcolor{PineGreen}{\selectlanguage{french}C'est fini!} \zh{完了!}  
 ¶ \textcolor{darkblue}{\textbf{\ipa{no˧ | tʰi˧-dzi˩-kʰɯ˩-se˩-dʑo˩, | dʑɯ˩-tsʰi˧ ɖɯ˧-qʰwɤ˧ pʰv̩˥ | tʰi˧-ki˧!}}} \textcolor{PineGreen}{\selectlanguage{french}Après que tu te sois assis, je te verse un verre d’eau chaude.} \zh{让(你)坐下以后,(我)给你倒一杯开水。}  

\lhead{\firstmark}
\rhead{\botmark}

\subsection{\hspace{-0.5cm} {\Large \textcolor{darkblue}{\textbf{\ipa{se˩\textsubscript{a}}}}}\hspace{0.5cm}[\kern2pt{\textcolor{darkblue}{\textbf{\ipa{se˥}}}}\kern2pt]} \hypertarget{se\string_Ba1}{}
\markboth{\textcolor{darkblue}{\textbf{\ipa{se˩\textsubscript{a}}}}}{}
\textcolor{teal}{\mytextsc{verbe}} \hspace{4pt} Ton~: L\textsubscript{a}.
\textcolor{PineGreen}{\selectlanguage{french}Achever.} \zh{完成。}  ¶ \textcolor{darkblue}{\textbf{\ipa{le˧-ʝi˥ | le˧-se˩-ze˩!}}} \textcolor{PineGreen}{\selectlanguage{french}(je) l'ai fait, j'ai fini!} \zh{做完了! / 完成了!}  
 ¶ \textcolor{darkblue}{\textbf{\ipa{le˧-se˧\textasciitilde{}se˥-ze˩!}}} \textcolor{PineGreen}{\selectlanguage{french}C'est fini, c'est achevé!} \zh{完成了!}  
 ¶ \textcolor{darkblue}{\textbf{\ipa{mɤ˧-se˩}}} \textcolor{PineGreen}{\selectlanguage{french}Ce n'est pas fini!} \zh{没有完!}  
 ¶ \textcolor{darkblue}{\textbf{\ipa{se˩˥ | -dʑo˩, | se˩-mɤ˩-tʰɑ˩˥! | dʑɤ˩˥ | -dʑo˩, | dʑɤ˩-kʰɯ˧ tʰɑ˥!}}} \textcolor{PineGreen}{\selectlanguage{french}Parenthèse au sujet de la documentation linguistique: on ne peut pas en voir le bout (tout collecter de façon exhaustive); mais on peut réaliser de belles choses!} \zh{(想做)完,但是没办法做完!不过最后还是可以做得很好!(情景:谈及收集语言的工作)}  

\lhead{\firstmark}
\rhead{\botmark}

\subsection{\hspace{-0.5cm} {\Large \textcolor{darkblue}{\textbf{\ipa{se˩di˩}}}}\hspace{0.5cm}[\kern2pt{\textcolor{darkblue}{\textbf{\ipa{se˧di˧}}}}\kern2pt]} \hypertarget{se\string_Bdi\string_B1}{}
\markboth{\textcolor{darkblue}{\textbf{\ipa{se˩di˩}}}}{}
\textcolor{teal}{\mytextsc{nom}} \hspace{4pt} Ton~: L.
\textcolor{PineGreen}{\selectlanguage{french}Scie.} \zh{锯。}  ¶ \textcolor{darkblue}{\textbf{\ipa{se˩di˩˥ | ɖɯ˩-hĩ˩˥ | ɖɯ˧-nɑ˧}}} \textcolor{PineGreen}{\selectlanguage{french}une grande scie} \zh{一把大锯}  
 ¶ \textcolor{darkblue}{\textbf{\ipa{se˩di˩˥ | tɕi˩-hĩ˩˥ | ɖɯ˧-nɑ˧}}} \textcolor{PineGreen}{\selectlanguage{french}une petite scie} \zh{一把小锯}  
 ¶ \textcolor{darkblue}{\textbf{\ipa{se˩di˩˥ | ɬi˧-hĩ˧ | ɖɯ˧-nɑ˩}}} \textcolor{PineGreen}{\selectlanguage{french}une scie de taille moyenne} \zh{中间大小的锯子}  
 \zh{量词}: \textcolor{darkblue}{\textbf{\ipa{nɑ˧}}}  \mytextsc{clf}~: \textcolor{darkblue}{\textbf{\ipa{nɑ˧}}} 
\lhead{\firstmark}
\rhead{\botmark}

\subsection{\hspace{-0.5cm} {\Large \textcolor{darkblue}{\textbf{\ipa{se˩gwɤ˩mi˥}}}}\hspace{0.5cm}[\kern2pt{\textcolor{darkblue}{\textbf{\ipa{se˧gwɤ˧mi˧}}}}\kern2pt]} \hypertarget{se\string_Bgw7\string_Bmi\string_T1}{}
\markboth{\textcolor{darkblue}{\textbf{\ipa{se˩gwɤ˩mi˥}}}}{}
\textcolor{teal}{\mytextsc{nom}} \hspace{4pt} Ton~: L+H\#.
\textcolor{PineGreen}{\selectlanguage{french}Vautour. Le terme n'est pas restreint aux vautours femelles; dans l'état actuel de la langue, il ne fournit pas d'indication de sexe.} \zh{雕(不仅来指母雕)。}  ¶ \textcolor{darkblue}{\textbf{\ipa{se˩gwɤ˩mi˥-pʰv̩˩}}} \textcolor{PineGreen}{\selectlanguage{french}vautour mâle} \zh{公雕}  
 ¶ \textcolor{darkblue}{\textbf{\ipa{se˩gwɤ˩mi˥-zo˩}}} \textcolor{PineGreen}{\selectlanguage{french}petit vautour, bébé vautour} \zh{小雕}  
 ¶ \textcolor{darkblue}{\textbf{\ipa{se˩gwɤ˩mi˥-ʈʂʰɯ˩, | mi˩ ɲi˥!}}} \textcolor{PineGreen}{\selectlanguage{french}Ce vautour, c'est une femelle!} \zh{这只雕是母的!}  
 \zh{量词}: \textcolor{darkblue}{\textbf{\ipa{mi˩}}}  \mytextsc{clf}~: \textcolor{darkblue}{\textbf{\ipa{mi˩}}} 
\lhead{\firstmark}
\rhead{\botmark}

\subsection{\hspace{-0.5cm} {\Large \textcolor{darkblue}{\textbf{\ipa{sɤ˥}}}}\hspace{0.5cm}[\kern2pt{\textcolor{darkblue}{\textbf{\ipa{sɤ˥}}}}\kern2pt]} \hypertarget{s7\string_T1}{}
\markboth{\textcolor{darkblue}{\textbf{\ipa{sɤ˥}}}}{}
\textcolor{teal}{\mytextsc{nom}} \hspace{4pt} Ton~: \#H.
\textcolor{PineGreen}{\selectlanguage{french}Sang.} \zh{血。}  \zh{量词}: \textcolor{darkblue}{\textbf{\ipa{ʈʰɤ˥}}}  \mytextsc{clf}~: \textcolor{darkblue}{\textbf{\ipa{ʈʰɤ˥}}} 
\lhead{\firstmark}
\rhead{\botmark}

\subsection{\hspace{-0.5cm} {\Large \textcolor{darkblue}{\textbf{\ipa{sɤ˧ɭɯ˩}}}}\hspace{0.5cm}[\kern2pt{\textcolor{darkblue}{\textbf{\ipa{sɤ˧ɭɯ˧}}}}\kern2pt]} \hypertarget{s7\string_Ml\string_RM\string_B1}{}
\markboth{\textcolor{darkblue}{\textbf{\ipa{sɤ˧ɭɯ˩}}}}{}
\textcolor{teal}{\mytextsc{nom}} \hspace{4pt} Ton~: L\#.
\textcolor{PineGreen}{\selectlanguage{french}Poire.} \zh{梨子。}  \zh{量词}: \textcolor{darkblue}{\textbf{\ipa{kʰɤ˧˥}}} \textcolor{darkblue}{\textbf{\ipa{ɭɯ˧}}}  \mytextsc{clf}~: \textcolor{darkblue}{\textbf{\ipa{kʰɤ˧˥}}} \textcolor{darkblue}{\textbf{\ipa{ɭɯ˧}}} 
\lhead{\firstmark}
\rhead{\botmark}

\subsection{\hspace{-0.5cm} {\Large \textcolor{darkblue}{\textbf{\ipa{sɤ˧sɤ˧˥}}}}\hspace{0.5cm}[\kern2pt{\textcolor{darkblue}{\textbf{\ipa{sɤ˧sɤ˧˥}}}}\kern2pt]} \hypertarget{s7\string_Ms7\string_M\string_T1}{}
\markboth{\textcolor{darkblue}{\textbf{\ipa{sɤ˧sɤ˧˥}}}}{}
\textcolor{teal}{\mytextsc{adjectif}} \hspace{4pt} Ton~: MH\#.
\textcolor{PineGreen}{\selectlanguage{french}Agréable, plaisant (circonstances).} \zh{舒畅。}  ¶ \textcolor{darkblue}{\textbf{\ipa{si˧dzi˩-ʈʰæ˩qo˩dzi˩, | sɤ˧sɤ˧˥ | ʐwæ˩˥!}}} \textcolor{PineGreen}{\selectlanguage{french}assis sous cet arbre, c'est le bonheur!} \zh{在树下坐着,感到很舒畅!}  
 ¶ \textcolor{darkblue}{\textbf{\ipa{ʈʂʰɯ˧-ɳɯ˧ | ɖɯ˧-ɖʐɯ˩ gwɤ˩-dʑo˩, | sɤ˧sɤ˧˥ | ʐwæ˩˥!}}} \textcolor{PineGreen}{\selectlanguage{french}Il a chanté un moment; c'était vraiment plaisant!} \zh{他唱了一会,真舒畅!}  

\lhead{\firstmark}
\rhead{\botmark}

\subsection{\hspace{-0.5cm} {\Large \textcolor{darkblue}{\textbf{\ipa{sɤ˧tʰo˧˥}}}}\hspace{0.5cm}[\kern2pt{\textcolor{darkblue}{\textbf{\ipa{sɤ˧tʰo˧˥}}}}\kern2pt]} \hypertarget{s7\string_Mt\string_ho\string_M\string_T1}{}
\markboth{\textcolor{darkblue}{\textbf{\ipa{sɤ˧tʰo˧˥}}}}{}
\textcolor{teal}{\mytextsc{nom}} \hspace{4pt} Ton~: MH\#.
\textcolor{PineGreen}{\selectlanguage{french}Sorte de pin.} \zh{一种松树。} Dialecte chinois local~:\zh{阔松。} ¶ \textcolor{darkblue}{\textbf{\ipa{sɤ˧tʰo˧-dzi˧˥}}} \textcolor{PineGreen}{\selectlanguage{french}même sens (désigne une espèce de pin)} \zh{同上}  
 \zh{量词}: \textcolor{darkblue}{\textbf{\ipa{dzi˩}}}  \mytextsc{clf}~: \textcolor{darkblue}{\textbf{\ipa{dzi˩}}} 
\lhead{\firstmark}
\rhead{\botmark}

\subsection{\hspace{-0.5cm} {\Large \textcolor{darkblue}{\textbf{\ipa{sɤ˧tsi˥}}}}\hspace{0.5cm}[\kern2pt{\textcolor{darkblue}{\textbf{\ipa{sɤ˧tsi˥}}}}\kern2pt]} \hypertarget{s7\string_Mtsi\string_T1}{}
\markboth{\textcolor{darkblue}{\textbf{\ipa{sɤ˧tsi˥}}}}{}
\textcolor{teal}{\mytextsc{nom}} \hspace{4pt} Ton~: H\#.
\textcolor{PineGreen}{\selectlanguage{french}Veines.} \zh{血管。}  \zh{量词}: \textcolor{darkblue}{\textbf{\ipa{kʰɯ˩}}}  \mytextsc{clf}~: \textcolor{darkblue}{\textbf{\ipa{kʰɯ˩}}} 
\lhead{\firstmark}
\rhead{\botmark}

\subsection{\hspace{-0.5cm} {\Large \textcolor{darkblue}{\textbf{\ipa{sɤ˩˥}}}}\hspace{0.5cm}[\kern2pt{\textcolor{darkblue}{\textbf{\ipa{sɤ˩˥}}}}\kern2pt]} \hypertarget{s7\string_B\string_T1}{}
\markboth{\textcolor{darkblue}{\textbf{\ipa{sɤ˩˥}}}}{}
\textcolor{teal}{\mytextsc{nom}} \hspace{4pt} Ton~: LH.
\textcolor{PineGreen}{\selectlanguage{french}Grain de beauté.} \zh{黑痣。}  \zh{量词}: \textcolor{darkblue}{\textbf{\ipa{ɭɯ˧}}}  \mytextsc{clf}~: \textcolor{darkblue}{\textbf{\ipa{ɭɯ˧}}} 
\lhead{\firstmark}
\rhead{\botmark}

\subsection{\hspace{-0.5cm} {\Large \textcolor{darkblue}{\textbf{\ipa{si˥}}}}\hspace{0.5cm}[\kern2pt{\textcolor{darkblue}{\textbf{\ipa{si˧˥}}}}\kern2pt]} \hypertarget{si\string_T1}{}
\markboth{\textcolor{darkblue}{\textbf{\ipa{si˥}}}}{}
\textcolor{teal}{\mytextsc{nom}} \hspace{4pt} Ton~: \#H.
\textcolor{PineGreen}{\selectlanguage{french}Bois.} \zh{木头。}  ¶ \textcolor{darkblue}{\textbf{\ipa{si˧-mo˩}}} \textcolor{PineGreen}{\selectlanguage{french}bois mort} \zh{枯木}  
 \zh{量词}: \textcolor{darkblue}{\textbf{\ipa{kɤ˧˥}}}  \mytextsc{clf}~: \textcolor{darkblue}{\textbf{\ipa{kɤ˧˥}}} 
\lhead{\firstmark}
\rhead{\botmark}

\subsection{\hspace{-0.5cm} {\Large \textcolor{darkblue}{\textbf{\ipa{si˧\textsubscript{a}}}}}\hspace{0.5cm}[\kern2pt{\textcolor{darkblue}{\textbf{\ipa{si˥}}}}\kern2pt]} \hypertarget{si\string_Ma1}{}
\markboth{\textcolor{darkblue}{\textbf{\ipa{si˧\textsubscript{a}}}}}{}
\textcolor{teal}{\mytextsc{verbe}} \hspace{4pt} Ton~: M\textsubscript{a}.
\textcolor{PineGreen}{\selectlanguage{french}Choisir.} \zh{挑选。}  ¶ \textcolor{darkblue}{\textbf{\ipa{le˧-si˧-ze˧}}} \textcolor{PineGreen}{\selectlanguage{french}\mytextsc{accomp} \string_ \mytextsc{pfv}} \zh{选了}  
 ¶ \textcolor{darkblue}{\textbf{\ipa{no˧ si˧-bi˧!}}} \textcolor{PineGreen}{\selectlanguage{french}Tu choisis! / A toi le choix!} \zh{你要选!}  
 ¶ \textcolor{darkblue}{\textbf{\ipa{njɤ˧-ɳɯ˧ si˧-bi˧!}}} \textcolor{PineGreen}{\selectlanguage{french}C'est moi qui choisis!} \zh{是我来选!}  
 ¶ \textcolor{darkblue}{\textbf{\ipa{le˧-si˥\textasciitilde{}si˩}}} \textcolor{PineGreen}{\selectlanguage{french}\mytextsc{accomp} \string_ \mytextsc{red}} \zh{\mytextsc{accomp} \string_ \mytextsc{red}}  
 ¶ \textcolor{darkblue}{\textbf{\ipa{tso˧\textasciitilde{}tso˧ si˩}}} \textcolor{PineGreen}{\selectlanguage{french}choisir des choses} \zh{选东西}  
 ¶ \textcolor{darkblue}{\textbf{\ipa{tso˧\textasciitilde{}tso˧ si˧\textasciitilde{}si˥}}} \textcolor{PineGreen}{\selectlanguage{french}choisir des choses} \zh{选选东西}  
 ¶ \textcolor{darkblue}{\textbf{\ipa{dʑɤ˩-hĩ˥ | si˧}}} \textcolor{PineGreen}{\selectlanguage{french}choisir les plus beaux; en choisir de beaux (par exemple: sur la montagne, lorsqu'on choisit des arbres à abattre pour donner du bois du charpente)} \zh{挑好的}  

\lhead{\firstmark}
\rhead{\botmark}

\subsection{\hspace{-0.5cm} {\Large \textcolor{darkblue}{\textbf{\ipa{si˧bv̩˧}}}}\hspace{0.5cm}[\kern2pt{\textcolor{darkblue}{\textbf{\ipa{si˧bv̩˧}}}}\kern2pt]} \hypertarget{si\string_Mbv\string_=\string_M1}{}
\markboth{\textcolor{darkblue}{\textbf{\ipa{si˧bv̩˧}}}}{}
\textcolor{teal}{\mytextsc{nom}} \hspace{4pt} Ton~: M.
\textcolor{PineGreen}{\selectlanguage{french}Démon (forme obtenue par élicitation; nettement moins courante que la forme féminine).} \zh{鬼。}  \zh{量词}: \textcolor{darkblue}{\textbf{\ipa{v̩˧}}}  \mytextsc{clf}~: \textcolor{darkblue}{\textbf{\ipa{v̩˧}}} 
\lhead{\firstmark}
\rhead{\botmark}

\subsection{\hspace{-0.5cm} {\Large \textcolor{darkblue}{\textbf{\ipa{si˧bv̩˧-mi\#˥}}}}\hspace{0.5cm}[\kern2pt{\textcolor{darkblue}{\textbf{\ipa{xxxx non-correspondance entre le nombre de morphèmes et le nombre de tons de morphèmes}}}}\kern2pt]} \hypertarget{si\string_Mbv\string_=\string_M-mi\#\string_T1}{}
\markboth{\textcolor{darkblue}{\textbf{\ipa{si˧bv̩˧-mi\#˥}}}}{}
\textcolor{teal}{\mytextsc{nom}} \hspace{4pt} Ton~: \#H.
\textcolor{PineGreen}{\selectlanguage{french}Démone.} \zh{妖精。}  \zh{量词}: \textcolor{darkblue}{\textbf{\ipa{v̩˧}}}  \mytextsc{clf}~: \textcolor{darkblue}{\textbf{\ipa{v̩˧}}} 
\lhead{\firstmark}
\rhead{\botmark}

\subsection{\hspace{-0.5cm} {\Large \textcolor{darkblue}{\textbf{\ipa{si˧bv̩˧-zo\#˥}}}}\hspace{0.5cm}[\kern2pt{\textcolor{darkblue}{\textbf{\ipa{xxxx non-correspondance entre le nombre de morphèmes et le nombre de tons de morphèmes}}}}\kern2pt]} \hypertarget{si\string_Mbv\string_=\string_M-zo\#\string_T1}{}
\markboth{\textcolor{darkblue}{\textbf{\ipa{si˧bv̩˧-zo\#˥}}}}{}
\textcolor{teal}{\mytextsc{nom}} \hspace{4pt} Ton~: \#H.
\textcolor{PineGreen}{\selectlanguage{french}Démon masculin (forme élicitée, sur la base de la forme féminine; est un mot qui existe, mais peu courant).} \zh{鬼。}  \zh{量词}: \textcolor{darkblue}{\textbf{\ipa{v̩˧}}}  \mytextsc{clf}~: \textcolor{darkblue}{\textbf{\ipa{v̩˧}}} 
\lhead{\firstmark}
\rhead{\botmark}

\subsection{\hspace{-0.5cm} {\Large \textcolor{darkblue}{\textbf{\ipa{si˧ɕi˧˥}}}}\hspace{0.5cm}[\kern2pt{\textcolor{darkblue}{\textbf{\ipa{si˧ɕi˧}}}}\kern2pt]} \hypertarget{si\string_Ms£i\string_M\string_T1}{}
\markboth{\textcolor{darkblue}{\textbf{\ipa{si˧ɕi˧˥}}}}{}
\textcolor{teal}{\mytextsc{nom}} \hspace{4pt} Ton~: MH\#.
\textcolor{PineGreen}{\selectlanguage{french}Forêt (clairsemée).} \zh{森林。}  ¶ \textcolor{darkblue}{\textbf{\ipa{[F5] tʰo˧ɕi˧˥}}} \textcolor{PineGreen}{\selectlanguage{french}forêt de pins} \zh{松树森林}  
 \zh{量词}: \textcolor{darkblue}{\textbf{\ipa{pʰæ˧˥}}}  \mytextsc{clf}~: \textcolor{darkblue}{\textbf{\ipa{pʰæ˧˥}}} 
\lhead{\firstmark}
\rhead{\botmark}

\subsection{\hspace{-0.5cm} {\Large \textcolor{darkblue}{\textbf{\ipa{si˧dzi˩}}}}\hspace{0.5cm}[\kern2pt{\textcolor{darkblue}{\textbf{\ipa{si˧dzi˧˥}}}}\kern2pt]} \hypertarget{si\string_Mdzi\string_B1}{}
\markboth{\textcolor{darkblue}{\textbf{\ipa{si˧dzi˩}}}}{}
\textcolor{teal}{\mytextsc{nom}} \hspace{4pt} Ton~: L\#.
\textcolor{PineGreen}{\selectlanguage{french}Arbre.} \zh{树。}  \zh{量词}: \textcolor{darkblue}{\textbf{\ipa{dzi˩, ʝi˧}}}  \mytextsc{clf}~: \textcolor{darkblue}{\textbf{\ipa{dzi˩, ʝi˧}}} 
\lhead{\firstmark}
\rhead{\botmark}

\subsection{\hspace{-0.5cm} {\Large \textcolor{darkblue}{\textbf{\ipa{si˧dzi˩-mv̩˩tsɯ˩}}}}\hspace{0.5cm}[\kern2pt{\textcolor{darkblue}{\textbf{\ipa{xxxx non-correspondance entre le nombre de morphèmes et le nombre de tons de morphèmes}}}}\kern2pt]} \hypertarget{si\string_Mdzi\string_B-mv\string_=\string_BtsM\string_B1}{}
\markboth{\textcolor{darkblue}{\textbf{\ipa{si˧dzi˩-mv̩˩tsɯ˩}}}}{}
\textcolor{teal}{\mytextsc{nom}} \hspace{4pt} Ton~: \#L-L.
\textcolor{PineGreen}{\selectlanguage{french}Radicelle, petites racine.} \zh{胚根。} 
\lhead{\firstmark}
\rhead{\botmark}

\subsection{\hspace{-0.5cm} {\Large \textcolor{darkblue}{\textbf{\ipa{si˧dʑɯ˥}}}}\hspace{0.5cm}[\kern2pt{\textcolor{darkblue}{\textbf{\ipa{xxxx non-correspondance entre le nombre de morphèmes et le nombre de tons de morphèmes}}}}\kern2pt]} \hypertarget{si\string_Mdz£M\string_T1}{}
\markboth{\textcolor{darkblue}{\textbf{\ipa{si˧dʑɯ˥}}}}{}
\textcolor{teal}{\mytextsc{nom}} \hspace{4pt} Ton~: H\#.
\textcolor{PineGreen}{\selectlanguage{french}Petit bois, pour faire démarrer le feu; à Yongning, ce qu'on utilise: des morceaux de pin gorgés de résine, utilisés spécialement à cet effet.} \zh{火煤、火捻、火种、劈柴、引柴。}  \zh{量词}: \textcolor{darkblue}{\textbf{\ipa{kʰwɤ˥}}}  \mytextsc{clf}~: \textcolor{darkblue}{\textbf{\ipa{kʰwɤ˥}}} 
\lhead{\firstmark}
\rhead{\botmark}

\subsection{\hspace{-0.5cm} {\Large \textcolor{darkblue}{\textbf{\ipa{si˧gɯ˧}}}}\hspace{0.5cm}[\kern2pt{\textcolor{darkblue}{\textbf{\ipa{si˧gɯ˥}}}}\kern2pt]} \hypertarget{si\string_MgM\string_M1}{}
\markboth{\textcolor{darkblue}{\textbf{\ipa{si˧gɯ˧}}}}{}
\textcolor{teal}{\mytextsc{nom}} \hspace{4pt} Ton~: M.
\textcolor{PineGreen}{\selectlanguage{french}Lion.} \zh{狮子。}  Emprunt~: tibétain  seng ge སེང་གེ
 \zh{量词}: \textcolor{darkblue}{\textbf{\ipa{mi˩}}}  \mytextsc{clf}~: \textcolor{darkblue}{\textbf{\ipa{mi˩}}} 
\lhead{\firstmark}
\rhead{\botmark}

\subsection{\hspace{-0.5cm} {\Large \textcolor{darkblue}{\textbf{\ipa{si˧gɯ˧-mi˩}}}}\hspace{0.5cm}[\kern2pt{\textcolor{darkblue}{\textbf{\ipa{xxxx non-correspondance entre le nombre de morphèmes et le nombre de tons de morphèmes}}}}\kern2pt]} \hypertarget{si\string_MgM\string_M-mi\string_B1}{}
\markboth{\textcolor{darkblue}{\textbf{\ipa{si˧gɯ˧-mi˩}}}}{}
\textcolor{teal}{\mytextsc{nom}} \hspace{4pt} Ton~: \mytextsc{L}.
\textcolor{PineGreen}{\selectlanguage{french}Lionne.} \zh{母狮。}  \zh{量词}: \textcolor{darkblue}{\textbf{\ipa{mi˩}}}  \mytextsc{clf}~: \textcolor{darkblue}{\textbf{\ipa{mi˩}}} 
\lhead{\firstmark}
\rhead{\botmark}

\subsection{\hspace{-0.5cm} {\Large \textcolor{darkblue}{\textbf{\ipa{si˧gɯ˧-pʰv̩\#˥}}}}\hspace{0.5cm}[\kern2pt{\textcolor{darkblue}{\textbf{\ipa{xxxx non-correspondance entre le nombre de morphèmes et le nombre de tons de morphèmes}}}}\kern2pt]} \hypertarget{si\string_MgM\string_M-p\string_hv\string_=\#\string_T1}{}
\markboth{\textcolor{darkblue}{\textbf{\ipa{si˧gɯ˧-pʰv̩\#˥}}}}{}
\textcolor{teal}{\mytextsc{nom}} \hspace{4pt} Ton~: \#H.
\textcolor{PineGreen}{\selectlanguage{french}Lion (mâle).} \zh{公狮子。}  \zh{量词}: \textcolor{darkblue}{\textbf{\ipa{mi˩}}}  \mytextsc{clf}~: \textcolor{darkblue}{\textbf{\ipa{mi˩}}} 
\lhead{\firstmark}
\rhead{\botmark}

\subsection{\hspace{-0.5cm} {\Large \textcolor{darkblue}{\textbf{\ipa{si˧gɯ˧-tsʰo\#˥}}}}\hspace{0.5cm}[\kern2pt{\textcolor{darkblue}{\textbf{\ipa{xxxx non-correspondance entre le nombre de morphèmes et le nombre de tons de morphèmes}}}}\kern2pt]} \hypertarget{si\string_MgM\string_M-ts\string_ho\#\string_T1}{}
\markboth{\textcolor{darkblue}{\textbf{\ipa{si˧gɯ˧-tsʰo\#˥}}}}{}
\textcolor{teal}{\mytextsc{nom}} \hspace{4pt} Ton~: \#H.
\textcolor{PineGreen}{\selectlanguage{french}Danse du Lion: spectacle masqué, commandité par le seigneur féodal, qui participait lui-même à certaines des danses.} \zh{狮子舞:土司准备的礼仪性表演。土司也亲自参与舞蹈。} 
\lhead{\firstmark}
\rhead{\botmark}

\subsection{\hspace{-0.5cm} {\Large \textcolor{darkblue}{\textbf{\ipa{si˧gɯ˧-zo\#˥}}}}\hspace{0.5cm}[\kern2pt{\textcolor{darkblue}{\textbf{\ipa{xxxx non-correspondance entre le nombre de morphèmes et le nombre de tons de morphèmes}}}}\kern2pt]} \hypertarget{si\string_MgM\string_M-zo\#\string_T1}{}
\markboth{\textcolor{darkblue}{\textbf{\ipa{si˧gɯ˧-zo\#˥}}}}{}
\textcolor{teal}{\mytextsc{nom}} \hspace{4pt} Ton~: \#H.
\textcolor{PineGreen}{\selectlanguage{french}Lionceau, petit lion.} \zh{小狮子。}  \zh{量词}: \textcolor{darkblue}{\textbf{\ipa{ɭɯ˧}}} \textcolor{darkblue}{\textbf{\ipa{mi˩}}}  \mytextsc{clf}~: \textcolor{darkblue}{\textbf{\ipa{ɭɯ˧}}} \textcolor{darkblue}{\textbf{\ipa{mi˩}}} 
\lhead{\firstmark}
\rhead{\botmark}

\subsection{\hspace{-0.5cm} {\Large \textcolor{darkblue}{\textbf{\ipa{si˧kɤ˧˥}}}}\hspace{0.5cm}[\kern2pt{\textcolor{darkblue}{\textbf{\ipa{si˧kɤ˧˥}}}}\kern2pt]} \hypertarget{si\string_Mk7\string_M\string_T1}{}
\markboth{\textcolor{darkblue}{\textbf{\ipa{si˧kɤ˧˥}}}}{}
\textcolor{teal}{\mytextsc{nom}} \hspace{4pt} Ton~: MH\#.
\textcolor{PineGreen}{\selectlanguage{french}Branche; petite branche; bâton, gourdin, canne pour marcher.} \zh{树枝、小树枝,棍子。}  \zh{量词}: \textcolor{darkblue}{\textbf{\ipa{kɤ˧˥}}}  \mytextsc{clf}~: \textcolor{darkblue}{\textbf{\ipa{kɤ˧˥}}} 
\lhead{\firstmark}
\rhead{\botmark}

\subsection{\hspace{-0.5cm} {\Large \textcolor{darkblue}{\textbf{\ipa{si˧kwɤ˩}}}}\hspace{0.5cm}[\kern2pt{\textcolor{darkblue}{\textbf{\ipa{xxxx ton non trouvé, à faire manuellement...}}}}\kern2pt]} \hypertarget{si\string_Mkw7\string_B1}{}
\markboth{\textcolor{darkblue}{\textbf{\ipa{si˧kwɤ˩}}}}{}
\textcolor{teal}{\mytextsc{nom}} \hspace{4pt} Ton~: \#L.
\textcolor{PineGreen}{\selectlanguage{french}Structure, charpente, gros œuvre en bois (d'une maison).} \zh{木头框架,如:房子的木头框架。}  ¶ \textcolor{darkblue}{\textbf{\ipa{ʑi˧mi˧-si˧kwɤ˩}}} \textcolor{PineGreen}{\selectlanguage{french}la charpente d'une maison} \zh{房子的木头框架}  
 \zh{量词}: \textcolor{darkblue}{\textbf{\ipa{kwɤ˩}}}  \mytextsc{clf}~: \textcolor{darkblue}{\textbf{\ipa{kwɤ˩}}} 
\lhead{\firstmark}
\rhead{\botmark}

\subsection{\hspace{-0.5cm} {\Large \textcolor{darkblue}{\textbf{\ipa{si˧kʰɯ\#˥}}}}\hspace{0.5cm}[\kern2pt{\textcolor{darkblue}{\textbf{\ipa{si˧kʰɯ˥}}}}\kern2pt]} \hypertarget{si\string_Mk\string_hM\#\string_T1}{}
\markboth{\textcolor{darkblue}{\textbf{\ipa{si˧kʰɯ\#˥}}}}{}
\textcolor{teal}{\mytextsc{nom}} \hspace{4pt} Ton~: H\#.
\textcolor{PineGreen}{\selectlanguage{french}Yyyy.} \zh{色疙瘩。} Dialecte chinois local~:\zh{根三香。} ¶ \textcolor{darkblue}{\textbf{\ipa{si˧kʰɯ˧-bæ˥bæ˩}}} \textcolor{PineGreen}{\selectlanguage{french}fleur de...} \zh{色疙瘩花}  
\textit{Voir~:} \textcolor{darkblue}{\textbf{\ipa{si˧kʰɯ˧ɭɯ˧bv̩˥}}} 
\lhead{\firstmark}
\rhead{\botmark}

\subsection{\hspace{-0.5cm} {\Large \textcolor{darkblue}{\textbf{\ipa{si˧kʰɯ˧-ɭɯ˧bv̩˥}}}}\hspace{0.5cm}[\kern2pt{\textcolor{darkblue}{\textbf{\ipa{xxxx non-correspondance entre le nombre de morphèmes et le nombre de tons de morphèmes}}}}\kern2pt]} \hypertarget{si\string_Mk\string_hM\string_M-l\string_RM\string_Mbv\string_=\string_T1}{}
\markboth{\textcolor{darkblue}{\textbf{\ipa{si˧kʰɯ˧-ɭɯ˧bv̩˥}}}}{}
\textcolor{teal}{\mytextsc{nom}} \hspace{4pt} Ton~: H\#.
\textcolor{PineGreen}{\selectlanguage{french}Pivoine blanche de Chine, \textit{Paeonia lactiflora}.} \zh{白芍药。}  \zh{量词}: \textcolor{darkblue}{\textbf{\ipa{kɤ˧˥}}}  \mytextsc{clf}~: \textcolor{darkblue}{\textbf{\ipa{kɤ˧˥}}} \textit{Voir~:} \hyperlink{}{\textcolor{darkblue}{\textbf{\ipa{si˧kʰɯ\#˥}}}} 
\lhead{\firstmark}
\rhead{\botmark}

\subsection{\hspace{-0.5cm} {\Large \textcolor{darkblue}{\textbf{\ipa{si˧nɑ˥}}}}\hspace{0.5cm}[\kern2pt{\textcolor{darkblue}{\textbf{\ipa{si˧nɑ˥}}}}\kern2pt]} \hypertarget{si\string_MnA\string_T1}{}
\markboth{\textcolor{darkblue}{\textbf{\ipa{si˧nɑ˥}}}}{}
\textcolor{teal}{\mytextsc{nom}} \hspace{4pt} Ton~: H\#.
\textcolor{PineGreen}{\selectlanguage{french}Forêt épaisse.} \zh{森林深处(难走路)。}  \zh{量词}: \textcolor{darkblue}{\textbf{\ipa{pʰæ˧˥}}}  \mytextsc{clf}~: \textcolor{darkblue}{\textbf{\ipa{pʰæ˧˥}}} 
\lhead{\firstmark}
\rhead{\botmark}

\subsection{\hspace{-0.5cm} {\Large \textcolor{darkblue}{\textbf{\ipa{si˧-ʁæ˧bæ˥}}}}\hspace{0.5cm}[\kern2pt{\textcolor{darkblue}{\textbf{\ipa{xxxx non-correspondance entre le nombre de morphèmes et le nombre de tons de morphèmes}}}}\kern2pt]} \hypertarget{si\string_M-R\{\string_Mb\{\string_T1}{}
\markboth{\textcolor{darkblue}{\textbf{\ipa{si˧-ʁæ˧bæ˥}}}}{}
\textcolor{teal}{\mytextsc{nom}} \hspace{4pt} Ton~: H\#.
\textcolor{PineGreen}{\selectlanguage{french}Assiette en bois.} \zh{木盘子。}  \zh{量词}: \textcolor{darkblue}{\textbf{\ipa{ɭɯ˧}}}  \mytextsc{clf}~: \textcolor{darkblue}{\textbf{\ipa{ɭɯ˧}}} 
\lhead{\firstmark}
\rhead{\botmark}

\subsection{\hspace{-0.5cm} {\Large \textcolor{darkblue}{\textbf{\ipa{si˧ʁo\#˥}}}}\hspace{0.5cm}[\kern2pt{\textcolor{darkblue}{\textbf{\ipa{si˧ʁo˧}}}}\kern2pt]} \hypertarget{si\string_MRo\#\string_T1}{}
\markboth{\textcolor{darkblue}{\textbf{\ipa{si˧ʁo\#˥}}}}{}
\textcolor{teal}{\mytextsc{nom}} \hspace{4pt} Ton~: \#H.
\textcolor{PineGreen}{\selectlanguage{french}Arbre fruitier.} \zh{果树。}  \zh{量词}: \textcolor{darkblue}{\textbf{\ipa{dzi˩}}}  \mytextsc{clf}~: \textcolor{darkblue}{\textbf{\ipa{dzi˩}}} 
\lhead{\firstmark}
\rhead{\botmark}

\subsection{\hspace{-0.5cm} {\Large \textcolor{darkblue}{\textbf{\ipa{si˧ʁo˧si˧ɭɯ\#˥}}}}\hspace{0.5cm}[\kern2pt{\textcolor{darkblue}{\textbf{\ipa{si˧ʁo˧si˧ɭɯ˧}}}}\kern2pt]} \hypertarget{si\string_MRo\string_Msi\string_Ml\string_RM\#\string_T1}{}
\markboth{\textcolor{darkblue}{\textbf{\ipa{si˧ʁo˧si˧ɭɯ\#˥}}}}{}
\textcolor{teal}{\mytextsc{nom}} \hspace{4pt} Ton~: \#H.
\textcolor{PineGreen}{\selectlanguage{french}Fruit.} \zh{水果。}  ¶ \textcolor{darkblue}{\textbf{\ipa{si˧ʁo˧si˧ɭɯ˧ ɲi˥}}} \textcolor{PineGreen}{\selectlanguage{french}\mytextsc{cop}} \zh{是水果。}  
 \zh{量词}: \textcolor{darkblue}{\textbf{\ipa{ɭɯ˧}}}  \mytextsc{clf}~: \textcolor{darkblue}{\textbf{\ipa{ɭɯ˧}}} 
\lhead{\firstmark}
\rhead{\botmark}

\subsection{\hspace{-0.5cm} {\Large \textcolor{darkblue}{\textbf{\ipa{si˧-sæ˥qʰv̩˩}}}}\hspace{0.5cm}[\kern2pt{\textcolor{darkblue}{\textbf{\ipa{xxxx non-correspondance entre le nombre de morphèmes et le nombre de tons de morphèmes}}}}\kern2pt]} \hypertarget{si\string_M-s\{\string_Tq\string_hv\string_=\string_B1}{}
\markboth{\textcolor{darkblue}{\textbf{\ipa{si˧-sæ˥qʰv̩˩}}}}{}
\textcolor{teal}{\mytextsc{nom}} \hspace{4pt} Ton~: \#H.
\textcolor{PineGreen}{\selectlanguage{french}Bouleau, \textit{Betula szechuanica (Betula Pendula var. szechuanica)}.} \zh{四川桦树,白桦树。} 
\lhead{\firstmark}
\rhead{\botmark}

\subsection{\hspace{-0.5cm} {\Large \textcolor{darkblue}{\textbf{\ipa{si˧tʰv̩\#˥}}}}\hspace{0.5cm}[\kern2pt{\textcolor{darkblue}{\textbf{\ipa{si˧tʰv̩˧}}}}\kern2pt]} \hypertarget{si\string_Mt\string_hv\string_=\#\string_T1}{}
\markboth{\textcolor{darkblue}{\textbf{\ipa{si˧tʰv̩\#˥}}}}{}
\textcolor{teal}{\mytextsc{nom}} \hspace{4pt} Ton~: \#H.
\textcolor{PineGreen}{\selectlanguage{french}Meuble-autel des ancêtres, dans la pièce principale, qui constitue le lieu symbolique où résident les ancêtres; on y met des bougies au Nouvel An.} \zh{供桌:主屋里面的一个家具,是祖先的象征性住所。}  ¶ \textcolor{darkblue}{\textbf{\ipa{ʑi˧dv̩˧-nv̩˩mi˩, | si˧tʰv̩˧!}}} \textcolor{PineGreen}{\selectlanguage{french}le cœur de la maison, c'est le meuble-autel des ancêtres!} \zh{屋子的中心,就是祖先的供桌!}  

\lhead{\firstmark}
\rhead{\botmark}

\subsection{\hspace{-0.5cm} {\Large \textcolor{darkblue}{\textbf{\ipa{si˩qʰɑ˩}}}}\hspace{0.5cm}[\kern2pt{\textcolor{darkblue}{\textbf{\ipa{si˩qʰɑ˩˥}}}}\kern2pt]} \hypertarget{si\string_Bq\string_hA\string_B1}{}
\markboth{\textcolor{darkblue}{\textbf{\ipa{si˩qʰɑ˩}}}}{}
\textcolor{teal}{\mytextsc{nom}} \hspace{4pt} Ton~: L.
\textcolor{PineGreen}{\selectlanguage{french}Abricotier du Japon (essence bien représentée à Yongning).} \zh{梅子。}  ¶ \textcolor{darkblue}{\textbf{\ipa{si˩qʰɑ˩-dʑɯ˩}}} \textcolor{PineGreen}{\selectlanguage{french}un liquide préparé à base de prunelles d'abricotier du Japon, servant d'équivalent de vinaigre (le vinaigre a été introduit tardivement; il était acheté en pays chinois)} \zh{用梅子做的一种汁,用法类似于醋。过去,永宁没有醋,醋是从内地买来的。}  

\lhead{\firstmark}
\rhead{\botmark}

\subsection{\hspace{-0.5cm} {\Large \textcolor{darkblue}{\textbf{\ipa{si˩tsʰɤ˩}}}}\hspace{0.5cm}[\kern2pt{\textcolor{darkblue}{\textbf{\ipa{si˩tsʰɤ˩˥}}}}\kern2pt]} \hypertarget{si\string_Bts\string_h7\string_B1}{}
\markboth{\textcolor{darkblue}{\textbf{\ipa{si˩tsʰɤ˩}}}}{}
\textcolor{teal}{\mytextsc{nom}} \hspace{4pt} Ton~: L.
\ding{202} \textcolor{PineGreen}{\selectlanguage{french}Feuille.} \zh{叶子。}  ¶ \textcolor{darkblue}{\textbf{\ipa{si˧dzi˩-si˩tsʰɤ˩}}} \textcolor{PineGreen}{\selectlanguage{french}feuilles d'arbre} \zh{树叶}  
 \zh{量词}: \textcolor{darkblue}{\textbf{\ipa{tsʰɤ˧˥}}} \ding{203} \textcolor{PineGreen}{\selectlanguage{french}Crête (du coq, d'un oiseau).} \zh{鸡冠。}  ¶ \textcolor{darkblue}{\textbf{\ipa{æ̃˧ʂwæ˥-si˩tsʰɤ˩}}} \textcolor{PineGreen}{\selectlanguage{french}crête de coq} \zh{公鸡冠}  
 \mytextsc{clf}~: \textcolor{darkblue}{\textbf{\ipa{tsʰɤ˧˥}}} 
\lhead{\firstmark}
\rhead{\botmark}

\subsection{\hspace{-0.5cm} {\Large \textcolor{darkblue}{\textbf{\ipa{si˧˥}}} \textsubscript{1}}\hspace{0.5cm}[\kern2pt{\textcolor{darkblue}{\textbf{\ipa{si˩˥}}}}\kern2pt]} \hypertarget{si\string_M\string_T1}{}
\markboth{\textcolor{darkblue}{\textbf{\ipa{si˧˥}}} \textsubscript{1}}{}
\textcolor{teal}{\mytextsc{verbe}} \hspace{4pt} Ton~: MH.
\textcolor{PineGreen}{\selectlanguage{french}Raser (la barbe); gratter (la terre collée à un champignon).} \zh{剔,刮。}  ¶ \textcolor{darkblue}{\textbf{\ipa{mo˧ si˥}}} \textcolor{PineGreen}{\selectlanguage{french}gratter des champignons, pour en retirer la terre, les aiguilles de pin... Cela se fait souvent à sec.} \zh{刮菌子(刮掉污垢)}  
 ¶ \textcolor{darkblue}{\textbf{\ipa{mv̩˧tsɯ˧ si˥}}} \textcolor{PineGreen}{\selectlanguage{french}raser la barbe} \zh{刮胡子}  
 ¶ \textcolor{darkblue}{\textbf{\ipa{ʁo˧qʰwɤ˩ si˩}}} \textcolor{PineGreen}{\selectlanguage{french}raser le crâne, raser la tête} \zh{剃头}  
 ¶ \textcolor{darkblue}{\textbf{\ipa{ʁo˧qʰwɤ˩ si˩-di˩}}} \textcolor{PineGreen}{\selectlanguage{french}Rasoir, objet utilisé pour raser le crâne, mais aussi pour raser la barbe. Dans la jeunesse de F4, il existait quelques rasoirs; chaque famille n'en possédait pas. On faisait venir une personne sachant manier l'instrument. Ce sont les moines et les vieilles personnes qui faisaient le plus fréquemment appel à ces services.} \zh{理发刮刀}  

\lhead{\firstmark}
\rhead{\botmark}

\subsection{\hspace{-0.5cm} {\Large \textcolor{darkblue}{\textbf{\ipa{si˧˥}}} \textsubscript{2}}\hspace{0.5cm}[\kern2pt{\textcolor{darkblue}{\textbf{\ipa{si˧˥}}}}\kern2pt]} \hypertarget{si\string_M\string_T2}{}
\markboth{\textcolor{darkblue}{\textbf{\ipa{si˧˥}}} \textsubscript{2}}{}
\textcolor{teal}{\mytextsc{verbe}} \hspace{4pt} Ton~: MH.
\textcolor{PineGreen}{\selectlanguage{french}Assassiner, tuer (un homme).} \zh{杀(人)。}  ¶ \textcolor{darkblue}{\textbf{\ipa{hĩ˧ si˩}}} \textcolor{PineGreen}{\selectlanguage{french}tuer quelqu'un, assassiner quelqu'un} \zh{杀人}  

\lhead{\firstmark}
\rhead{\botmark}

\subsection{\hspace{-0.5cm} {\Large \textcolor{darkblue}{\textbf{\ipa{si˩˥}}}}\hspace{0.5cm}[\kern2pt{\textcolor{darkblue}{\textbf{\ipa{si˩˥}}}}\kern2pt]} \hypertarget{si\string_B\string_T1}{}
\markboth{\textcolor{darkblue}{\textbf{\ipa{si˩˥}}}}{}
\textcolor{teal}{\mytextsc{nom}} \hspace{4pt} Ton~: LH.
\textcolor{PineGreen}{\selectlanguage{french}Foie.} \zh{肝。}  \zh{量词}: \textcolor{darkblue}{\textbf{\ipa{ɭɯ˧}}}  \mytextsc{clf}~: \textcolor{darkblue}{\textbf{\ipa{ɭɯ˧}}} 
\lhead{\firstmark}
\rhead{\botmark}

\subsection{\hspace{-0.5cm} {\Large \textcolor{darkblue}{\textbf{\ipa{so˥}}} \textsubscript{1}}\hspace{0.5cm}[\kern2pt{\textcolor{darkblue}{\textbf{\ipa{so˥}}}}\kern2pt]} \hypertarget{so\string_T1}{}
\markboth{\textcolor{darkblue}{\textbf{\ipa{so˥}}} \textsubscript{1}}{}
\textcolor{teal}{\mytextsc{nom}} \hspace{4pt} Ton~: \#H.
\textcolor{PineGreen}{\selectlanguage{french}Offrande aux esprits: repas qu'on leur offre le matin; on y met du thé, du beurre, de la farine, et du miel (et éventuellement des fleurs: \textcolor{darkblue}{\textbf{\ipa{/so˧dze˧-bæ˩bæ˩/}}}); on le fait brûler sur un feu d'épines de pin.} \zh{早上献给神的食物(含茶、酥油、面、蜂蜜),扔进松针火里烧。}  ¶ \textcolor{darkblue}{\textbf{\ipa{so˧ dze˧ tʰi˧-qæ˩}}} \textcolor{PineGreen}{\selectlanguage{french}faire brûler du miel en offrande} \zh{烧蜂蜜献给神}  
 ¶ \textcolor{darkblue}{\textbf{\ipa{[M23] so˧ qæ˩}}} \textcolor{PineGreen}{\selectlanguage{french}brûler une offrande; traditionnellement, du pin gorgé de résine.} \zh{烧献给神(食物,……)}  

\lhead{\firstmark}
\rhead{\botmark}

\subsection{\hspace{-0.5cm} {\Large \textcolor{darkblue}{\textbf{\ipa{so˥}}} \textsubscript{2}}\hspace{0.5cm}[\kern2pt{\textcolor{darkblue}{\textbf{\ipa{so˥}}}}\kern2pt]} \hypertarget{so\string_T2}{}
\markboth{\textcolor{darkblue}{\textbf{\ipa{so˥}}} \textsubscript{2}}{}
\textcolor{teal}{\mytextsc{classificateur}} \hspace{4pt} Ton~: H*.
\textcolor{PineGreen}{\selectlanguage{french}Classificateur des choses/objets, utilisé seulement en tournure négative: 'quoi que ce soit'.} \zh{量词:样东西,如:‘一样东西都没有’。}  ¶ \textcolor{darkblue}{\textbf{\ipa{ɖɯ˧-so˥ | mɤ˧-dʑo˧!}}} \textcolor{PineGreen}{\selectlanguage{french}Il n’y a rien du tout [à manger]! (phrase polie qd on invite quelqu'un à manger: on prie le convive d’excuser la pauvreté des mets proposés)} \zh{一样也没有! / 没什么东西!(请客时的礼貌、自我贬低说法:请客人原谅菜不够丰盛)}  
\textit{Voir~:} \textcolor{darkblue}{\textbf{\ipa{sɑ˥}}} 
\lhead{\firstmark}
\rhead{\botmark}

\subsection{\hspace{-0.5cm} {\Large \textcolor{darkblue}{\textbf{\ipa{so˧\textsubscript{a}}}}}\hspace{0.5cm}[\kern2pt{\textcolor{darkblue}{\textbf{\ipa{so˩˥}}}}\kern2pt]} \hypertarget{so\string_Ma1}{}
\markboth{\textcolor{darkblue}{\textbf{\ipa{so˧\textsubscript{a}}}}}{}
\textcolor{teal}{\mytextsc{classificateur}} \hspace{4pt} Ton~: M\textsubscript{a}.
\textcolor{PineGreen}{\selectlanguage{french}Classificateur des matinées. Il existe trois expressions pour compter les journées: on peut dire: un jour; une matinée; ou une nuit.} \zh{量词:早晨(一个)。}  ¶ \textcolor{darkblue}{\textbf{\ipa{mv̩˩si˧-njɤ˧˥ | ɖɯ˧-so˧, | njɤ˧le˧gv̩˧ | ɖɯ˧-ɲi˥, | mv̩˧kʰv̩˥ | ɖɯ˧-hɑ̃˧˥!}}} \textcolor{PineGreen}{\selectlanguage{french}Une matinée; une journée; [ou] une nuit! (Expression didactique résumant les trois façons de compter les jours: on peut compter les matinées, les journées, ou les soirées.)} \zh{一个早晨,一个白天,(或者说)一个晚上!(这句话,总结数日子的三个方式:‘一天’,可以说成‘一个早晨’、‘一个白天’、或‘一个晚上’。)}  
 ¶ \textcolor{darkblue}{\textbf{\ipa{tʰv̩˧-so˩}}} \textcolor{PineGreen}{\selectlanguage{french}ce matin-là} \zh{那天早上}  

\lhead{\firstmark}
\rhead{\botmark}

\subsection{\hspace{-0.5cm} {\Large \textcolor{darkblue}{\textbf{\ipa{so˧dʑɯ\#˥}}}}\hspace{0.5cm}[\kern2pt{\textcolor{darkblue}{\textbf{\ipa{so˧dʑɯ˧˥}}}}\kern2pt]} \hypertarget{so\string_Mdz£M\#\string_T1}{}
\markboth{\textcolor{darkblue}{\textbf{\ipa{so˧dʑɯ\#˥}}}}{}
\textcolor{teal}{\mytextsc{nom}} \hspace{4pt} Ton~: \#H.
\textcolor{PineGreen}{\selectlanguage{french}Piège.} \zh{陷阱。}  ¶ \textcolor{darkblue}{\textbf{\ipa{so˧dʑɯ˧ | ɖɯ˧-ɭɯ˧ | qwæ˧˥}}} \textcolor{PineGreen}{\selectlanguage{french}creuser une fosse, un piège} \zh{挖一个陷阱}  

\lhead{\firstmark}
\rhead{\botmark}

\subsection{\hspace{-0.5cm} {\Large \textcolor{darkblue}{\textbf{\ipa{so˧hɑ̃˩}}}}\hspace{0.5cm}[\kern2pt{\textcolor{darkblue}{\textbf{\ipa{so˧hɑ̃˧}}}}\kern2pt]} \hypertarget{so\string_MhA\string_~\string_B1}{}
\markboth{\textcolor{darkblue}{\textbf{\ipa{so˧hɑ̃˩}}}}{}
\textcolor{teal}{\mytextsc{adverbe}} \hspace{4pt} Ton~: L\#.
\textcolor{PineGreen}{\selectlanguage{french}Demain soir.} \zh{明晚。}  ¶ \textcolor{darkblue}{\textbf{\ipa{so˧hɑ̃˩ | -ɖɯ˩hɑ̃˩˥}}} \textcolor{PineGreen}{\selectlanguage{french}demain soir} \zh{明天晚上}  

\lhead{\firstmark}
\rhead{\botmark}

\subsection{\hspace{-0.5cm} {\Large \textcolor{darkblue}{\textbf{\ipa{so˧hwɤ˩}}}}\hspace{0.5cm}[\kern2pt{\textcolor{darkblue}{\textbf{\ipa{so˧hwɤ˩}}}}\kern2pt]} \hypertarget{so\string_Mhw7\string_B1}{}
\markboth{\textcolor{darkblue}{\textbf{\ipa{so˧hwɤ˩}}}}{}
\textcolor{teal}{\mytextsc{adverbe}} \hspace{4pt} Ton~: L\#.
\textcolor{PineGreen}{\selectlanguage{french}Ensuite; par la suite; à partir de maintenant, désormais.} \zh{后来、以后,从此以后。} 
\lhead{\firstmark}
\rhead{\botmark}

\subsection{\hspace{-0.5cm} {\Large \textcolor{darkblue}{\textbf{\ipa{so˧ʝi˥\$}}}}\hspace{0.5cm}[\kern2pt{\textcolor{darkblue}{\textbf{\ipa{so˧ʝi˩}}}}\kern2pt]} \hypertarget{so\string_Mj££i\string_T\$1}{}
\markboth{\textcolor{darkblue}{\textbf{\ipa{so˧ʝi˥\$}}}}{}
\textcolor{teal}{\mytextsc{adverbe}} \hspace{4pt} Ton~: H\$.
\textcolor{PineGreen}{\selectlanguage{french}L'année prochaine, l'an prochain.} \zh{明年。} 
\lhead{\firstmark}
\rhead{\botmark}

\subsection{\hspace{-0.5cm} {\Large \textcolor{darkblue}{\textbf{\ipa{so˧lo˧}}}}\hspace{0.5cm}[\kern2pt{\textcolor{darkblue}{\textbf{\ipa{so˧lo˧}}}}\kern2pt]} \hypertarget{so\string_Mlo\string_M1}{}
\markboth{\textcolor{darkblue}{\textbf{\ipa{so˧lo˧}}}}{}
\textcolor{teal}{\mytextsc{nom}} \hspace{4pt} Ton~: M.
\textcolor{PineGreen}{\selectlanguage{french}Influence; exemple (dans l'éducation de quelqu'un).} \zh{影响,榜样。}  ¶ \textcolor{darkblue}{\textbf{\ipa{so˧lo˧ dzɑ˧! | mɤ˧-dʑɤ˩-hĩ˩ | ɖɯ˧-ʑi˩ ɲi˩!}}} \textcolor{PineGreen}{\selectlanguage{french}Il/elle exerce une mauvaise influence / il/elle donne un mauvais exemple! (Sa famille,) c'est une mauvaise famille!} \zh{他(对周围的人)有一个不好的影响!(他的家庭)是个不好的家庭!}  
 ¶ \textcolor{darkblue}{\textbf{\ipa{so˧lo˧ mɤ˧-dʑɤ˩!}}} \textcolor{PineGreen}{\selectlanguage{french}Même sens que l'exemple précédent: Son exemple n'est pas bon / son influence n'est pas bonne.} \zh{同上:(他对别人的)影响不好。}  
 ¶ \textcolor{darkblue}{\textbf{\ipa{so˧lo˧ dʑɤ˩}}} \textcolor{PineGreen}{\selectlanguage{french}bonne influence; bon exemple; bonne éducation} \zh{好榜样、好例子、好教育}  
 \zh{量词}: \textcolor{darkblue}{\textbf{\ipa{kʰwɤ˥}}}  \mytextsc{clf}~: \textcolor{darkblue}{\textbf{\ipa{kʰwɤ˥}}} 
\lhead{\firstmark}
\rhead{\botmark}

\subsection{\hspace{-0.5cm} {\Large \textcolor{darkblue}{\textbf{\ipa{so˧ɬi˧mi˧}}}}\hspace{0.5cm}[\kern2pt{\textcolor{darkblue}{\textbf{\ipa{so˧ɬi˧mi˥}}}}\kern2pt]} \hypertarget{so\string_MKi\string_Mmi\string_M1}{}
\markboth{\textcolor{darkblue}{\textbf{\ipa{so˧ɬi˧mi˧}}}}{}
\textcolor{teal}{\mytextsc{nom}} \hspace{4pt} Ton~: M.
\textcolor{PineGreen}{\selectlanguage{french}3e mois.} \zh{三月。} 
\lhead{\firstmark}
\rhead{\botmark}

\subsection{\hspace{-0.5cm} {\Large \textcolor{darkblue}{\textbf{\ipa{so˧ɲi˥}}}}\hspace{0.5cm}[\kern2pt{\textcolor{darkblue}{\textbf{\ipa{so˧ɲi˩}}}}\kern2pt]} \hypertarget{so\string_MJi\string_T1}{}
\markboth{\textcolor{darkblue}{\textbf{\ipa{so˧ɲi˥}}}}{}
\textcolor{teal}{\mytextsc{adverbe}} \hspace{4pt} Ton~: .
\textcolor{PineGreen}{\selectlanguage{french}Demain, le lendemain.} \zh{明天、第二天。} 
\lhead{\firstmark}
\rhead{\botmark}

\subsection{\hspace{-0.5cm} {\Large \textcolor{darkblue}{\textbf{\ipa{so˩}}}}\hspace{0.5cm}[\kern2pt{\textcolor{darkblue}{\textbf{\ipa{so˩˥}}}}\kern2pt]} \hypertarget{so\string_B1}{}
\markboth{\textcolor{darkblue}{\textbf{\ipa{so˩}}}}{}
\textcolor{teal}{\mytextsc{nombre}} \hspace{4pt} Ton~: L.
\textcolor{PineGreen}{\selectlanguage{french}3.} \zh{3。} 
\lhead{\firstmark}
\rhead{\botmark}

\subsection{\hspace{-0.5cm} {\Large \textcolor{darkblue}{\textbf{\ipa{so˩\textsubscript{a}}}} \textsubscript{1}}\hspace{0.5cm}[\kern2pt{\textcolor{darkblue}{\textbf{\ipa{so˩˥}}}}\kern2pt]} \hypertarget{so\string_Ba1}{}
\markboth{\textcolor{darkblue}{\textbf{\ipa{so˩\textsubscript{a}}}} \textsubscript{1}}{}
\textcolor{teal}{\mytextsc{adjectif}} \hspace{4pt} Ton~: L\textsubscript{a}.
\textcolor{PineGreen}{\selectlanguage{french}Agréable, bon (goût, odeur).} \zh{香(吃得香,气味香)。} 
\lhead{\firstmark}
\rhead{\botmark}

\subsection{\hspace{-0.5cm} {\Large \textcolor{darkblue}{\textbf{\ipa{so˩\textsubscript{a}}}} \textsubscript{2}}\hspace{0.5cm}[\kern2pt{\textcolor{darkblue}{\textbf{\ipa{so˩˥}}}}\kern2pt]} \hypertarget{so\string_Ba2}{}
\markboth{\textcolor{darkblue}{\textbf{\ipa{so˩\textsubscript{a}}}} \textsubscript{2}}{}
\textcolor{teal}{\mytextsc{verbe}} \hspace{4pt} Ton~: L\textsubscript{a}.
\ding{202} \textcolor{PineGreen}{\selectlanguage{french}Étudier.} \zh{学习。}  ¶ \textcolor{darkblue}{\textbf{\ipa{tʰæ˧ɻæ˩ so˩}}} \textcolor{PineGreen}{\selectlanguage{french}étudier (des livres)} \zh{读书、学习}  
 ¶ \textcolor{darkblue}{\textbf{\ipa{so˩ mɤ˩-se˥!}}} \textcolor{PineGreen}{\selectlanguage{french}c'est sans fin! / tu n'as jamais fini d'étudier! (au sujet du travail du linguiste, étudier une langue: à la différence des travaux manuels, ce n'est jamais fini, on n'en voit jamais le bout)} \zh{学不完!(关于语言学家的工作:做不完,不像做手工可以有一个明确的终点。)}  
 ¶ \textcolor{darkblue}{\textbf{\ipa{ɖɯ˧-so˧\textasciitilde{}so˥-ɻ̍˩}}} \textcolor{PineGreen}{\selectlanguage{french}étudier un peu} \zh{学一学}  
\ding{203} \textcolor{PineGreen}{\selectlanguage{french}Imiter.} \zh{学一个人、模仿一个人。}  ¶ \textcolor{darkblue}{\textbf{\ipa{tʰv̩˧ tʰɑ˧-so˧˥!}}} \textcolor{PineGreen}{\selectlanguage{french}Ne t'avise pas de suivre son exemple!/Ne va pas faire comme lui!/Ne va pas prendre exemple sur lui!} \zh{别学他! / 别做得像他一样!}  
\ding{204} \textcolor{PineGreen}{\selectlanguage{french}Enseigner.} \zh{教。}  ¶ \textcolor{darkblue}{\textbf{\ipa{tʰæ˧ɻæ˩ so˩}}} \textcolor{PineGreen}{\selectlanguage{french}enseigner} \zh{教书}  
 ¶ \textcolor{darkblue}{\textbf{\ipa{njɤ˧-ɳɯ˧ | no˧ so˧-bi˧!}}} \textcolor{PineGreen}{\selectlanguage{french}Je vais t'enseigner/t'apprendre!}  

\lhead{\firstmark}
\rhead{\botmark}

\subsection{\hspace{-0.5cm} {\Large \textcolor{darkblue}{\textbf{\ipa{so˩\textasciitilde{}so˧˥}}}}\hspace{0.5cm}[\kern2pt{\textcolor{darkblue}{\textbf{\ipa{so˧so˧˥}}}}\kern2pt]} \hypertarget{so\string_B~so\string_M\string_T1}{}
\markboth{\textcolor{darkblue}{\textbf{\ipa{so˩\textasciitilde{}so˧˥}}}}{}
\textcolor{teal}{\mytextsc{verbe}} \hspace{4pt} Ton~: MH.
\textcolor{PineGreen}{\selectlanguage{french}Frotter dans ses mains.} \zh{揉在手里。}  ¶ \textcolor{darkblue}{\textbf{\ipa{le˧-so˩\textasciitilde{}so˩}}} \textcolor{PineGreen}{\selectlanguage{french}\mytextsc{accomp} \string_ \mytextsc{red}} \zh{揉来揉去}  

\lhead{\firstmark}
\rhead{\botmark}

\subsection{\hspace{-0.5cm} {\Large \textcolor{darkblue}{\textbf{\ipa{so˧˥}}}}\hspace{0.5cm}[\kern2pt{\textcolor{darkblue}{\textbf{\ipa{so˧˥}}}}\kern2pt]} \hypertarget{so\string_M\string_T1}{}
\markboth{\textcolor{darkblue}{\textbf{\ipa{so˧˥}}}}{}
\textcolor{teal}{\mytextsc{nom}} \hspace{4pt} Ton~: MH.
\ding{202} \textcolor{PineGreen}{\selectlanguage{french}Souffle.} \zh{(一口)气。}  \zh{量词}: \textcolor{darkblue}{\textbf{\ipa{kʰɯ˩}}} \ding{203} \textcolor{PineGreen}{\selectlanguage{french}Vapeur.} \zh{蒸汽。}  ¶ \textcolor{darkblue}{\textbf{\ipa{so˧ tʰv̩˥-ze˩}}} \textcolor{PineGreen}{\selectlanguage{french}il y a de la vapeur qui sort, ça fait de la vapeur} \zh{热气冒出来了。}  
 \mytextsc{clf}~: \textcolor{darkblue}{\textbf{\ipa{kʰɯ˩}}} \textcolor{PineGreen}{\selectlanguage{french}objets longs}
\lhead{\firstmark}
\rhead{\botmark}

\subsection{\hspace{-0.5cm} {\Large \textcolor{darkblue}{\textbf{\ipa{sɯ˥}}} \textsubscript{1}}\hspace{0.5cm}[\kern2pt{\textcolor{darkblue}{\textbf{\ipa{sɯ˥}}}}\kern2pt]} \hypertarget{sM\string_T1}{}
\markboth{\textcolor{darkblue}{\textbf{\ipa{sɯ˥}}} \textsubscript{1}}{}
\textcolor{teal}{\mytextsc{verbe}} \hspace{4pt} Ton~: H.
\textcolor{PineGreen}{\selectlanguage{french}Aiguiser.} \zh{磨(刀)。}  ¶ \textcolor{darkblue}{\textbf{\ipa{ɖɯ˧-sɯ˧\textasciitilde{}sɯ˧-ɻ̍˥}}} \textcolor{PineGreen}{\selectlanguage{french}aiguiser un peu} \zh{磨一磨}  
 ¶ \textcolor{darkblue}{\textbf{\ipa{sɯ˩tʰi˩ sɯ˩˥}}} \textcolor{PineGreen}{\selectlanguage{french}aiguiser un couteau} \zh{磨刀}  

\lhead{\firstmark}
\rhead{\botmark}

\subsection{\hspace{-0.5cm} {\Large \textcolor{darkblue}{\textbf{\ipa{sɯ˥}}} \textsubscript{2}}\hspace{0.5cm}[\kern2pt{\textcolor{darkblue}{\textbf{\ipa{sɯ˥}}}}\kern2pt]} \hypertarget{sM\string_T2}{}
\markboth{\textcolor{darkblue}{\textbf{\ipa{sɯ˥}}} \textsubscript{2}}{}
\textcolor{teal}{\mytextsc{verbe}} \hspace{4pt} Ton~: H.
\textcolor{PineGreen}{\selectlanguage{french}Savoir.} \zh{知道。}  ¶ \textcolor{darkblue}{\textbf{\ipa{mɤ˧-sɯ˥}}} \textcolor{PineGreen}{\selectlanguage{french}\mytextsc{neg}} \zh{不知道}  

\lhead{\firstmark}
\rhead{\botmark}

\subsection{\hspace{-0.5cm} {\Large \textcolor{darkblue}{\textbf{\ipa{‑sɯ˧}}}}\hspace{0.5cm}[\kern2pt{\textcolor{darkblue}{\textbf{\ipa{sɯ˥}}}}\kern2pt]} \hypertarget{‑sM\string_M1}{}
\markboth{\textcolor{darkblue}{\textbf{\ipa{‑sɯ˧}}}}{}
\textcolor{teal}{\mytextsc{suffixe}} \hspace{4pt} Ton~: M.
\textcolor{PineGreen}{\selectlanguage{french}D'abord; encore (dans la tournure: pas encore).} \zh{首先、先。}  ¶ \textcolor{darkblue}{\textbf{\ipa{njɤ˧ ʈʂʰɯ˧-sɯ˩ | dzɯ˧-bi˧!}}} \textcolor{PineGreen}{\selectlanguage{french}je vais d'abord manger celui-ci!} \zh{我要先吃这个!}  
 ¶ \textcolor{darkblue}{\textbf{\ipa{njɤ˧ | ʈʂʰɯ˧-sɯ˩ | li˧-bi˧!}}} \textcolor{PineGreen}{\selectlanguage{french}je vais d'abord lire celui-ci! (au sujet de deux livres)} \zh{我要先读这本!}  
 ¶ \textcolor{darkblue}{\textbf{\ipa{ʈʂʰɯ˧ sɯ˩ | hwæ˧-bi˧!}}} \textcolor{PineGreen}{\selectlanguage{french}(je) vais d'abord acheter celui-ci!} \zh{先买这个吧!}  
 ¶ \textcolor{darkblue}{\textbf{\ipa{ʈʂʰɯ˧ sɯ˩ | tɕʰi˧-bi˧!}}} \textcolor{PineGreen}{\selectlanguage{french}(je) vais d'abord vendre celui-ci!} \zh{先卖这个吧!}  
 ¶ \textcolor{darkblue}{\textbf{\ipa{ʈʂʰɯ˧ sɯ˩ | dzɯ˧-bi˧!}}} \textcolor{PineGreen}{\selectlanguage{french}(je) vais d'abord manger celui-ci!} \zh{先吃这个吧!}  
 ¶ \textcolor{darkblue}{\textbf{\ipa{ʈʂʰɯ˧ sɯ˩ | ʑi˩-bi˩˥}}} \textcolor{PineGreen}{\selectlanguage{french}(je) vais d'abord prendre celui-ci!} \zh{先拿这个吧!}  
 ¶ \textcolor{darkblue}{\textbf{\ipa{ʈʂʰɯ˧ sɯ˩ | ʈʰɯ˩-bi˩˥}}} \textcolor{PineGreen}{\selectlanguage{french}(je) vais d'abord boire celui-ci!} \zh{先喝这个吧!}  
 ¶ \textcolor{darkblue}{\textbf{\ipa{ʈʂʰɯ˧ sɯ˩ | lɑ˧-bi˥}}} \textcolor{PineGreen}{\selectlanguage{french}(je) vais d'abord battre celui-ci!} \zh{先打这个吧!}  
 ¶ \textcolor{darkblue}{\textbf{\ipa{tv̩˧tv̩˥ sɯ˩ | tʰi˧-tsʰi˥!}}} \textcolor{PineGreen}{\selectlanguage{french}Mets d'abord ton chapeau! (injonction à un petit enfant, avant une sortie)} \zh{你先戴上帽子!(情景:出门前,让孩子戴上帽子)}  
 ¶ \textcolor{darkblue}{\textbf{\ipa{no˧ | le˧-sɯ˧ gv̩˧\textasciitilde{}gv̩˥!}}} \textcolor{PineGreen}{\selectlanguage{french}Commence par travailler tout seul! (Consigne de la locutrice quand j'arrive pour ma leçon du matin. Elle est occupée; et elle sait que j'ai de quoi m'occuper seul en l'attendant: toiletter des textes déjà transcrits, etc. Elle me dit “Commence par travailler tout seul! / Commence par les tâches que tu peux faire tout seul!”} \zh{你先自己工作(一会)吧!(情景:调查者早上到合作人的家,但她忙着,而她知道调查者有不同类型的工作要做,其中有一些可以自己做,比如重新核对记录过的长篇语料。她说:“你先忙自己的一会吧!”)}  

\lhead{\firstmark}
\rhead{\botmark}

\subsection{\hspace{-0.5cm} {\Large \textcolor{darkblue}{\textbf{\ipa{sɯ˧\textsubscript{a}}}}}\hspace{0.5cm}[\kern2pt{\textcolor{darkblue}{\textbf{\ipa{sɯ˥}}}}\kern2pt]} \hypertarget{sM\string_Ma1}{}
\markboth{\textcolor{darkblue}{\textbf{\ipa{sɯ˧\textsubscript{a}}}}}{}
\textcolor{teal}{\mytextsc{verbe}} \hspace{4pt} Ton~: M\textsubscript{a}.
\ding{202} \textcolor{PineGreen}{\selectlanguage{french}Enfiler (des perles).} \zh{串(珠)。}  ¶ \textcolor{darkblue}{\textbf{\ipa{sɯ˧ɻ̍˧ sɯ˧}}} \textcolor{PineGreen}{\selectlanguage{french}enfiler des perles} \zh{串珠}  
 ¶ \textcolor{darkblue}{\textbf{\ipa{le˧-sɯ˧-se˩-ze˩}}} \textcolor{PineGreen}{\selectlanguage{french}(j'ai) fini d'enfiler} \zh{串完了!}  
 ¶ \textcolor{darkblue}{\textbf{\ipa{tso˧\textasciitilde{}tso˧ sɯ˩}}} \textcolor{PineGreen}{\selectlanguage{french}enfiler des choses} \zh{串东西}  
\ding{203} \textcolor{PineGreen}{\selectlanguage{french}Enfiler (une jupe).} \zh{穿(裙子)。}  ¶ \textcolor{darkblue}{\textbf{\ipa{ʈʰæ˧qʰwɤ˧ sɯ˧}}} \textcolor{PineGreen}{\selectlanguage{french}enfiler une robe} \zh{穿裙子}  

\lhead{\firstmark}
\rhead{\botmark}

\subsection{\hspace{-0.5cm} {\Large \textcolor{darkblue}{\textbf{\ipa{sɯ˧gv̩\#˥}}}}\hspace{0.5cm}[\kern2pt{\textcolor{darkblue}{\textbf{\ipa{sɯ˧gv̩˧}}}}\kern2pt]} \hypertarget{sM\string_Mgv\string_=\#\string_T1}{}
\markboth{\textcolor{darkblue}{\textbf{\ipa{sɯ˧gv̩\#˥}}}}{}
\textcolor{teal}{\mytextsc{nom}} \hspace{4pt} Ton~: \#H.
\textcolor{PineGreen}{\selectlanguage{french}Caisse, coffre; par extension: armoire.} \zh{箱子,柜子。}  \zh{量词}: \textcolor{darkblue}{\textbf{\ipa{ɭɯ˧}}}  \mytextsc{clf}~: \textcolor{darkblue}{\textbf{\ipa{ɭɯ˧}}} 
\lhead{\firstmark}
\rhead{\botmark}

\subsection{\hspace{-0.5cm} {\Large \textcolor{darkblue}{\textbf{\ipa{sɯ˧kʰɯ˩}}}}\hspace{0.5cm}[\kern2pt{\textcolor{darkblue}{\textbf{\ipa{sɯ˩kʰɯ˥}}}}\kern2pt]} \hypertarget{sM\string_Mk\string_hM\string_B1}{}
\markboth{\textcolor{darkblue}{\textbf{\ipa{sɯ˧kʰɯ˩}}}}{}
\textcolor{teal}{\mytextsc{nom}} \hspace{4pt} Ton~: L\#.
\textcolor{PineGreen}{\selectlanguage{french}Rituel lors du décès d'une femme qui a quitté la maison de sa mère pour se marier.} \zh{斯克:嫁到外边的女人去世时进行的仪式。} 
\lhead{\firstmark}
\rhead{\botmark}

\subsection{\hspace{-0.5cm} {\Large \textcolor{darkblue}{\textbf{\ipa{sɯ˧ljɤ˧}}}}\hspace{0.5cm}[\kern2pt{\textcolor{darkblue}{\textbf{\ipa{sɯ˧ljɤ˧}}}}\kern2pt]} \hypertarget{sM\string_Mlj7\string_M1}{}
\markboth{\textcolor{darkblue}{\textbf{\ipa{sɯ˧ljɤ˧}}}}{}
\textcolor{teal}{\mytextsc{nom}} \hspace{4pt} Ton~: M.
\textcolor{PineGreen}{\selectlanguage{french}Plastique.} \zh{塑料(汉语借词)。}  Emprunt~: chinois  \zh{塑料}

\lhead{\firstmark}
\rhead{\botmark}

\subsection{\hspace{-0.5cm} {\Large \textcolor{darkblue}{\textbf{\ipa{sɯ˧ljɤ˧tʰo˧˥}}}}\hspace{0.5cm}[\kern2pt{\textcolor{darkblue}{\textbf{\ipa{sɯ˧ljɤ˧tʰo˧}}}}\kern2pt]} \hypertarget{sM\string_Mlj7\string_Mt\string_ho\string_M\string_T1}{}
\markboth{\textcolor{darkblue}{\textbf{\ipa{sɯ˧ljɤ˧tʰo˧˥}}}}{}
\textcolor{teal}{\mytextsc{nom}} \hspace{4pt} Ton~: MH\#.
\textcolor{PineGreen}{\selectlanguage{french}Container pour liquides, en matière plastique.} \zh{塑料桶(汉语借词)。}  \zh{量词}: \textcolor{darkblue}{\textbf{\ipa{ɭɯ˧}}}  \mytextsc{clf}~: \textcolor{darkblue}{\textbf{\ipa{ɭɯ˧}}} 
\lhead{\firstmark}
\rhead{\botmark}

\subsection{\hspace{-0.5cm} {\Large \textcolor{darkblue}{\textbf{\ipa{sɯ˧mɤ˩}}}}\hspace{0.5cm}[\kern2pt{\textcolor{darkblue}{\textbf{\ipa{sɯ˧mɤ˧˥}}}}\kern2pt]} \hypertarget{sM\string_Mm7\string_B1}{}
\markboth{\textcolor{darkblue}{\textbf{\ipa{sɯ˧mɤ˩}}}}{}
\textcolor{teal}{\mytextsc{nom}} \hspace{4pt} Ton~: L\#.
\textcolor{PineGreen}{\selectlanguage{french}Shiso, \textit{Perilla frutescens}, akajiso: plante alimentaire, aromatique, médicinale et ornementale de la famille des Lamiacées. Ses petites graines noires ressemblent au sésame. Elle a été introduite à Yongning récemment (années 1980?) On se sert des graines pour confectionner des confiseries; et on en extrait de l'huile, qui se mange.} \zh{紫苏。}  ¶ \textcolor{darkblue}{\textbf{\ipa{sɯ˧mɤ˩, | ɬi˧di˩ | tv̩˧-kv̩˧˥!}}} \textcolor{PineGreen}{\selectlanguage{french}Le shiso, on en cultive à Yongning!} \zh{在永宁,有紫苏!/ 有人种紫苏!}  
 ¶ \textcolor{darkblue}{\textbf{\ipa{sɯ˧mɤ˩-dze˩}}} \textcolor{PineGreen}{\selectlanguage{french}friandise contenant des graines de shiso} \zh{含紫苏的糖果}  
 ¶ \textcolor{darkblue}{\textbf{\ipa{sɯ˧mɤ˩-mæ˩ɻæ˩}}} \textcolor{PineGreen}{\selectlanguage{french}huile de shiso} \zh{紫苏油}  

\lhead{\firstmark}
\rhead{\botmark}

\subsection{\hspace{-0.5cm} {\Large \textcolor{darkblue}{\textbf{\ipa{sɯ˧pv̩˩}}}}\hspace{0.5cm}[\kern2pt{\textcolor{darkblue}{\textbf{\ipa{sɯ˧pv̩˩}}}}\kern2pt]} \hypertarget{sM\string_Mpv\string_=\string_B1}{}
\markboth{\textcolor{darkblue}{\textbf{\ipa{sɯ˧pv̩˩}}}}{}
\textcolor{teal}{\mytextsc{nom}} \hspace{4pt} Ton~: L\#.
\textcolor{PineGreen}{\selectlanguage{french}Vessie.} \zh{膀胱。}  \zh{量词}: \textcolor{darkblue}{\textbf{\ipa{kɤ˧˥}}}  \mytextsc{clf}~: \textcolor{darkblue}{\textbf{\ipa{kɤ˧˥}}} 
\lhead{\firstmark}
\rhead{\botmark}

\subsection{\hspace{-0.5cm} {\Large \textcolor{darkblue}{\textbf{\ipa{sɯ˧pv̩˩-ni˩gv̩˩}}}}\hspace{0.5cm}[\kern2pt{\textcolor{darkblue}{\textbf{\ipa{sɯ˧pv̩˩ni˧gv̩˧}}}}\kern2pt]} \hypertarget{sM\string_Mpv\string_=\string_B-ni\string_Bgv\string_=\string_B1}{}
\markboth{\textcolor{darkblue}{\textbf{\ipa{sɯ˧pv̩˩-ni˩gv̩˩}}}}{}
\textcolor{teal}{\mytextsc{adjectif}} \hspace{4pt} Ton~: L\#-.
\textcolor{PineGreen}{\selectlanguage{french}Gonflé, galbé, rond: littéralement “comme une vessie”. Une vessie de porc que l'on gonfle prend une forme toute ronde.} \zh{膀肿。} 
\lhead{\firstmark}
\rhead{\botmark}

\subsection{\hspace{-0.5cm} {\Large \textcolor{darkblue}{\textbf{\ipa{sɯ˧pv̩˧-sɯ˥nɑ˩}}}}\hspace{0.5cm}[\kern2pt{\textcolor{darkblue}{\textbf{\ipa{sɯ˧pv̩˧sɯ˥nɑ˩}}}}\kern2pt]} \hypertarget{sM\string_Mpv\string_=\string_M-sM\string_TnA\string_B1}{}
\markboth{\textcolor{darkblue}{\textbf{\ipa{sɯ˧pv̩˧-sɯ˥nɑ˩}}}}{}
\textcolor{teal}{\mytextsc{nom}} \hspace{4pt} Ton~: \#H-.
\textcolor{PineGreen}{\selectlanguage{french}Chenille.} \zh{毛虫。}  \zh{量词}: \textcolor{darkblue}{\textbf{\ipa{mi˩}}}  \mytextsc{clf}~: \textcolor{darkblue}{\textbf{\ipa{mi˩}}} 
\lhead{\firstmark}
\rhead{\botmark}

\subsection{\hspace{-0.5cm} {\Large \textcolor{darkblue}{\textbf{\ipa{sɯ˧pʰi˧}}}}\hspace{0.5cm}[\kern2pt{\textcolor{darkblue}{\textbf{\ipa{sɯ˧pʰi˥}}}}\kern2pt]} \hypertarget{sM\string_Mp\string_hi\string_M1}{}
\markboth{\textcolor{darkblue}{\textbf{\ipa{sɯ˧pʰi˧}}}}{}
\textcolor{teal}{\mytextsc{nom}} \hspace{4pt} Ton~: M.
\textcolor{PineGreen}{\selectlanguage{french}Noble, seigneur, chef: la plus haute des 3 castes de la société ancienne.} \zh{贵族,土司,奴隶主,官。音译:“司沛”。}  ¶ \textcolor{darkblue}{\textbf{\ipa{ɖæ˩mi˧-sɯ˩pʰi˩}}} \textcolor{PineGreen}{\selectlanguage{french}les nobles du monastère} \zh{大寺贵族}  
 ¶ \textcolor{darkblue}{\textbf{\ipa{sɯ˧pʰi˧ hĩ˩}}} \textcolor{PineGreen}{\selectlanguage{french}les sujets du seigneur, les gens du seigneur} \zh{贵族的臣子、贵族手下的人}  
 \zh{量词}: \textcolor{darkblue}{\textbf{\ipa{v̩˧}}}  \mytextsc{clf}~: \textcolor{darkblue}{\textbf{\ipa{v̩˧}}} 
\lhead{\firstmark}
\rhead{\botmark}

\subsection{\hspace{-0.5cm} {\Large \textcolor{darkblue}{\textbf{\ipa{sɯ˧pʰi˧-zo˧}}}}\hspace{0.5cm}[\kern2pt{\textcolor{darkblue}{\textbf{\ipa{xxxx non-correspondance entre le nombre de morphèmes et le nombre de tons de morphèmes}}}}\kern2pt]} \hypertarget{sM\string_Mp\string_hi\string_M-zo\string_M1}{}
\markboth{\textcolor{darkblue}{\textbf{\ipa{sɯ˧pʰi˧-zo˧}}}}{}
\textcolor{teal}{\mytextsc{nom}} \hspace{4pt} Ton~: H\#.
\textcolor{PineGreen}{\selectlanguage{french}Jeune homme de la noblesse, fils de mandarin.} \zh{少爷。}  \zh{量词}: \textcolor{darkblue}{\textbf{\ipa{v̩˧}}}  \mytextsc{clf}~: \textcolor{darkblue}{\textbf{\ipa{v̩˧}}} 
\lhead{\firstmark}
\rhead{\botmark}

\subsection{\hspace{-0.5cm} {\Large \textcolor{darkblue}{\textbf{\ipa{sɯ˧ɻ̍˧}}}}\hspace{0.5cm}[\kern2pt{\textcolor{darkblue}{\textbf{\ipa{sɯ˧ɻ̍˧}}}}\kern2pt]} \hypertarget{sM\string_Mr£`̍\string_M1}{}
\markboth{\textcolor{darkblue}{\textbf{\ipa{sɯ˧ɻ̍˧}}}}{}
\textcolor{teal}{\mytextsc{nom}} \hspace{4pt} Ton~: M.
\textcolor{PineGreen}{\selectlanguage{french}Perle.} \zh{珠,珠子,珍珠。}  \zh{量词}: \textcolor{darkblue}{\textbf{\ipa{ɭɯ˧}}}  \mytextsc{clf}~: \textcolor{darkblue}{\textbf{\ipa{ɭɯ˧}}} 
\lhead{\firstmark}
\rhead{\botmark}

\subsection{\hspace{-0.5cm} {\Large \textcolor{darkblue}{\textbf{\ipa{sɯ˧ɻæ˧}}}}\hspace{0.5cm}[\kern2pt{\textcolor{darkblue}{\textbf{\ipa{sɯ˧ɻæ˧}}}}\kern2pt]} \hypertarget{sM\string_Mr£`\{\string_M1}{}
\markboth{\textcolor{darkblue}{\textbf{\ipa{sɯ˧ɻæ˧}}}}{}
\textcolor{teal}{\mytextsc{nom}} \hspace{4pt} Ton~: M.
\textcolor{PineGreen}{\selectlanguage{french}Table (à Yongning, vers le milieu du XXe siècle, elles étaient basses et carrées).} \zh{桌子。}  \zh{量词}: \textcolor{darkblue}{\textbf{\ipa{pɤ˩}}}  \mytextsc{clf}~: \textcolor{darkblue}{\textbf{\ipa{pɤ˩}}} \textit{Voir~:} \hyperlink{}{\textcolor{darkblue}{\textbf{\ipa{ʈʂo˧tsɯ˥}}}} 
\lhead{\firstmark}
\rhead{\botmark}

\subsection{\hspace{-0.5cm} {\Large \textcolor{darkblue}{\textbf{\ipa{sɯ˧ɻ̃\#˥}}}}\hspace{0.5cm}[\kern2pt{\textcolor{darkblue}{\textbf{\ipa{sɯ˧ɻ̃˧}}}}\kern2pt]} \hypertarget{sM\string_Mr£`\string_~\#\string_T1}{}
\markboth{\textcolor{darkblue}{\textbf{\ipa{sɯ˧ɻ̃\#˥}}}}{}
\textcolor{teal}{\mytextsc{nom}} \hspace{4pt} Ton~: \#H.
\textcolor{PineGreen}{\selectlanguage{french}Tronc.} \zh{树干。}  ¶ \textcolor{darkblue}{\textbf{\ipa{si˧dzi˩ tʰv̩˩-dzi˩, | sɯ˧ɻ̃˧ dʑɤ˥!}}} \textcolor{PineGreen}{\selectlanguage{french}Cet arbre, il a un beau tronc! (c'est-à-dire qu'il est utilisable en menuiserie)} \zh{这是棵好树!(可以用做木料)}  
 \zh{量词}: \textcolor{darkblue}{\textbf{\ipa{lo˩}}}  \mytextsc{clf}~: \textcolor{darkblue}{\textbf{\ipa{lo˩}}} 
\lhead{\firstmark}
\rhead{\botmark}

\subsection{\hspace{-0.5cm} {\Large \textcolor{darkblue}{\textbf{\ipa{sɯ˧ɻ̃˧mi\#˥}}}}\hspace{0.5cm}[\kern2pt{\textcolor{darkblue}{\textbf{\ipa{sɯ˧ɻ̃˧mi˧}}}}\kern2pt]} \hypertarget{sM\string_Mr£`\string_~\string_Mmi\#\string_T1}{}
\markboth{\textcolor{darkblue}{\textbf{\ipa{sɯ˧ɻ̃˧mi\#˥}}}}{}
\textcolor{teal}{\mytextsc{nom}} \hspace{4pt} Ton~: \#H.
\textcolor{PineGreen}{\selectlanguage{french}Colonne vertébrale.} \zh{脊椎骨。}  \zh{量词}: \textcolor{darkblue}{\textbf{\ipa{dzi˩}}}  \mytextsc{clf}~: \textcolor{darkblue}{\textbf{\ipa{dzi˩}}} \textcolor{PineGreen}{\selectlanguage{french}cl des arbres}
\lhead{\firstmark}
\rhead{\botmark}

\subsection{\hspace{-0.5cm} {\Large \textcolor{darkblue}{\textbf{\ipa{sɯ˧sɯ˩}}}}\hspace{0.5cm}[\kern2pt{\textcolor{darkblue}{\textbf{\ipa{sɯ˧sɯ˩}}}}\kern2pt]} \hypertarget{sM\string_MsM\string_B1}{}
\markboth{\textcolor{darkblue}{\textbf{\ipa{sɯ˧sɯ˩}}}}{}
\textcolor{teal}{\mytextsc{adjectif}} \hspace{4pt} Ton~: L\#.
\textcolor{PineGreen}{\selectlanguage{french}Cru.} \zh{生(不熟)。}  ¶ \textcolor{darkblue}{\textbf{\ipa{ʂe˧ sɯ˧\textasciitilde{}sɯ˥}}} \textcolor{PineGreen}{\selectlanguage{french}viande crue} \zh{生肉}  
 ¶ \textcolor{darkblue}{\textbf{\ipa{ʈʂe˧ sɯ˧\textasciitilde{}sɯ˥}}} \textcolor{PineGreen}{\selectlanguage{french}'terre crue': terre qui n'a pas été préparée pour l'agriculture par l'ajout de fumier, etc} \zh{‘生土’:没有经过加工(加肥料等等)的土,还不适合种农作物}  

\lhead{\firstmark}
\rhead{\botmark}

\subsection{\hspace{-0.5cm} {\Large \textcolor{darkblue}{\textbf{\ipa{sɯ˧tsɯ˧}}}}\hspace{0.5cm}[\kern2pt{\textcolor{darkblue}{\textbf{\ipa{sɯ˧tsɯ˧}}}}\kern2pt]} \hypertarget{sM\string_MtsM\string_M1}{}
\markboth{\textcolor{darkblue}{\textbf{\ipa{sɯ˧tsɯ˧}}}}{}
\textcolor{teal}{\mytextsc{nom}} \hspace{4pt} Ton~: M.
\textcolor{PineGreen}{\selectlanguage{french}Lion.} \zh{狮子(汉语借词)。}  Emprunt~: chinois  \zh{狮子}
 \zh{量词}: \textcolor{darkblue}{\textbf{\ipa{mi˩}}}  \mytextsc{clf}~: \textcolor{darkblue}{\textbf{\ipa{mi˩}}} 
\lhead{\firstmark}
\rhead{\botmark}

\subsection{\hspace{-0.5cm} {\Large \textcolor{darkblue}{\textbf{\ipa{sɯ˧tsɯ˩}}}}\hspace{0.5cm}[\kern2pt{\textcolor{darkblue}{\textbf{\ipa{sɯ˧tsɯ˩}}}}\kern2pt]} \hypertarget{sM\string_MtsM\string_B1}{}
\markboth{\textcolor{darkblue}{\textbf{\ipa{sɯ˧tsɯ˩}}}}{}
\textcolor{teal}{\mytextsc{nom}} \hspace{4pt} Ton~: L\#.
\textcolor{PineGreen}{\selectlanguage{french}Camphre (arbre).} \zh{樟。}  ¶ \textcolor{darkblue}{\textbf{\ipa{sɯ˧tsɯ˩-dzi˩}}} \textcolor{PineGreen}{\selectlanguage{french}arbre à camphre} \zh{樟树}  

\lhead{\firstmark}
\rhead{\botmark}

\subsection{\hspace{-0.5cm} {\Large \textcolor{darkblue}{\textbf{\ipa{sɯ˧ʈv̩˥}}}}\hspace{0.5cm}[\kern2pt{\textcolor{darkblue}{\textbf{\ipa{sɯ˧ʈv̩˥}}}}\kern2pt]} \hypertarget{sM\string_Mt`v\string_=\string_T1}{}
\markboth{\textcolor{darkblue}{\textbf{\ipa{sɯ˧ʈv̩˥}}}}{}
\textcolor{teal}{\mytextsc{nom}} \hspace{4pt} Ton~: H\#.
\textcolor{PineGreen}{\selectlanguage{french}Cor, durillon.} \zh{茧子。}  ¶ \textcolor{darkblue}{\textbf{\ipa{hĩ˧ ʈʂʰɯ˧-v̩˧-bv̩˧ | mv̩˧ɲi˧, | sɯ˧ʈv̩˥ ʁo˩!}}} \textcolor{PineGreen}{\selectlanguage{french}le pouce de cette personne a un cor/un durillon!} \zh{这个人的拇指有茧子!}  
 ¶ \textcolor{darkblue}{\textbf{\ipa{sɯ˧ʈv̩˥ | mɤ˧-ʁo˩-ze˩!}}} \textcolor{PineGreen}{\selectlanguage{french}(il n'y a/je n'ai) plus de durillon!} \zh{没有茧子了!}  
 ¶ \textcolor{darkblue}{\textbf{\ipa{sɯ˧ʈv̩˥ ʁo˩-ze˩!}}} \textcolor{PineGreen}{\selectlanguage{french}Un durillon s'est formé!} \zh{磨出了茧子!}  
 \zh{量词}: \textcolor{darkblue}{\textbf{\ipa{ɭɯ˧}}}  \mytextsc{clf}~: \textcolor{darkblue}{\textbf{\ipa{ɭɯ˧}}} 
\lhead{\firstmark}
\rhead{\botmark}

\subsection{\hspace{-0.5cm} {\Large \textcolor{darkblue}{\textbf{\ipa{sɯ˧zɯ\#˥}}}}\hspace{0.5cm}[\kern2pt{\textcolor{darkblue}{\textbf{\ipa{sɯ˧zɯ˧}}}}\kern2pt]} \hypertarget{sM\string_MzM\#\string_T1}{}
\markboth{\textcolor{darkblue}{\textbf{\ipa{sɯ˧zɯ\#˥}}}}{}
\textcolor{teal}{\mytextsc{nom}} \hspace{4pt} Ton~: \#H.
\textcolor{PineGreen}{\selectlanguage{french}Communauté familiale; échelon inférieur à “os”.} \zh{家族、支系。}  ¶ \textcolor{darkblue}{\textbf{\ipa{sɯ˧zɯ˧ ɖɯ˧-lo˩}}} \textcolor{PineGreen}{\selectlanguage{french}une communauté familiale} \zh{一个支系,一条线}  
 ¶ \textcolor{darkblue}{\textbf{\ipa{sɯ˧zɯ˧ ɖɯ˧-ʁwɤ˧}}} \textcolor{PineGreen}{\selectlanguage{french}une communauté familiale} \zh{一个支系,一条线}  
 ¶ \textcolor{darkblue}{\textbf{\ipa{sɯ˧zɯ˧ ə˩-dʑo˩?}}} \textcolor{PineGreen}{\selectlanguage{french}est-ce qu'il a une grande famille/est-ce que sa famille est nombreuse? =est-ce que la mariée sera bien entourée, intégrée dans une grande famille? (Question que l'on pose lors des discussions préliminaires aux mariages: on s'inquiète des qualités de la maisonnée que la jeune femme va rejoindre)} \zh{家族齐全吗?/ 家族,人多吗?(谈婚姻前的题目之一:男方家族人多不多。以人多为好。)}  
 \zh{量词}: \textcolor{darkblue}{\textbf{\ipa{lo˩, ʁwɤ˧}}}  \mytextsc{clf}~: \textcolor{darkblue}{\textbf{\ipa{lo˩, ʁwɤ˧}}} 
\lhead{\firstmark}
\rhead{\botmark}

\subsection{\hspace{-0.5cm} {\Large \textcolor{darkblue}{\textbf{\ipa{sɯ˩\textsubscript{a}}}}}\hspace{0.5cm}[\kern2pt{\textcolor{darkblue}{\textbf{\ipa{sɯ˩˥}}}}\kern2pt]} \hypertarget{sM\string_Ba1}{}
\markboth{\textcolor{darkblue}{\textbf{\ipa{sɯ˩\textsubscript{a}}}}}{}
\textcolor{teal}{\mytextsc{verbe}} \hspace{4pt} Ton~: L\textsubscript{a}.
\textcolor{PineGreen}{\selectlanguage{french}Vivre, être vivant.} \zh{活。}  ¶ \textcolor{darkblue}{\textbf{\ipa{ʈʂʰɯ˧ tʰi˧-sɯ˩-dʑo˩!}}} \textcolor{PineGreen}{\selectlanguage{french}Elle/il est vivant(e)!} \zh{他活着!}  
 ¶ \textcolor{darkblue}{\textbf{\ipa{ʈʂʰɯ˧ | mɤ˧-ʂɯ˧! | tʰi˧-sɯ˩-dʑo˩!}}} \textcolor{PineGreen}{\selectlanguage{french}ce n'est pas mort! c'est encore vivant! (au sujet d'une plante/d'un animal qui paraissait mort(e))} \zh{它没死,还活着!(一个植物、动物)}  

\lhead{\firstmark}
\rhead{\botmark}

\subsection{\hspace{-0.5cm} {\Large \textcolor{darkblue}{\textbf{\ipa{sɯ˩pv̩˩}}}}\hspace{0.5cm}[\kern2pt{\textcolor{darkblue}{\textbf{\ipa{sɯ˧pv̩˥}}}}\kern2pt]} \hypertarget{sM\string_Bpv\string_=\string_B1}{}
\markboth{\textcolor{darkblue}{\textbf{\ipa{sɯ˩pv̩˩}}}}{}
\textcolor{teal}{\mytextsc{nom}} \hspace{4pt} Ton~: .
\textcolor{PineGreen}{\selectlanguage{french}Cloque (par exemple, après qu'on se soit ébouillanté).} \zh{水泡(例如:开水烫了手,会形成水泡)。}  ¶ \textcolor{darkblue}{\textbf{\ipa{sɯ˩pv̩˩ qʰwæ˥-ze˩!}}} \textcolor{PineGreen}{\selectlanguage{french}une cloque s'est formée!} \zh{起了水泡!}  
 \zh{量词}: \textcolor{darkblue}{\textbf{\ipa{ɭɯ˧}}}  \mytextsc{clf}~: \textcolor{darkblue}{\textbf{\ipa{ɭɯ˧}}} 
\lhead{\firstmark}
\rhead{\botmark}

\subsection{\hspace{-0.5cm} {\Large \textcolor{darkblue}{\textbf{\ipa{sɯ˩ɻ̍˩}}}}\hspace{0.5cm}[\kern2pt{\textcolor{darkblue}{\textbf{\ipa{sɯ˩ɻ̍˩˥}}}}\kern2pt]} \hypertarget{sM\string_Br£`̍\string_B1}{}
\markboth{\textcolor{darkblue}{\textbf{\ipa{sɯ˩ɻ̍˩}}}}{}
\textcolor{teal}{\mytextsc{nom}} \hspace{4pt} Ton~: L.
\textcolor{PineGreen}{\selectlanguage{french}Pierre à aiguiser, “fusil”.} \zh{磨刀石。}  \zh{量词}: \textcolor{darkblue}{\textbf{\ipa{ɭɯ˧}}}  \mytextsc{clf}~: \textcolor{darkblue}{\textbf{\ipa{ɭɯ˧}}} 
\lhead{\firstmark}
\rhead{\botmark}

\subsection{\hspace{-0.5cm} {\Large \textcolor{darkblue}{\textbf{\ipa{sɯ˩tʰi˩}}}}\hspace{0.5cm}[\kern2pt{\textcolor{darkblue}{\textbf{\ipa{sɯ˩tʰi˩˥}}}}\kern2pt]} \hypertarget{sM\string_Bt\string_hi\string_B1}{}
\markboth{\textcolor{darkblue}{\textbf{\ipa{sɯ˩tʰi˩}}}}{}
\textcolor{teal}{\mytextsc{nom}} \hspace{4pt} Ton~: L.
\textcolor{PineGreen}{\selectlanguage{french}Couteau.} \zh{刀。}  \zh{量词}: \textcolor{darkblue}{\textbf{\ipa{nɑ˧}}}  \mytextsc{clf}~: \textcolor{darkblue}{\textbf{\ipa{nɑ˧}}} 
\lhead{\firstmark}
\rhead{\botmark}

\subsection{\hspace{-0.5cm} {\Large \textcolor{darkblue}{\textbf{\ipa{sɯ˩tʰi˩-kʰɯ˥ʑi˩}}}}\hspace{0.5cm}[\kern2pt{\textcolor{darkblue}{\textbf{\ipa{xxxx ton non trouvé, à faire manuellement...}}}}\kern2pt]} \hypertarget{sM\string_Bt\string_hi\string_B-k\string_hM\string_Tz£i\string_B1}{}
\markboth{\textcolor{darkblue}{\textbf{\ipa{sɯ˩tʰi˩-kʰɯ˥ʑi˩}}}}{}
\textcolor{teal}{\mytextsc{nom}} \hspace{4pt} Ton~: L+\#H-.
\textcolor{PineGreen}{\selectlanguage{french}Fourreau du couteau, gaine du couteau.} \zh{刀鞘。}  \zh{量词}: \textcolor{darkblue}{\textbf{\ipa{ɭɯ˧}}}  \mytextsc{clf}~: \textcolor{darkblue}{\textbf{\ipa{ɭɯ˧}}} 
\lhead{\firstmark}
\rhead{\botmark}

\newpage
\section*{\centering- \textcolor{darkblue}{\textbf{\ipa{ʂ}}} -}
\subsection{\hspace{-0.5cm} {\Large \textcolor{darkblue}{\textbf{\ipa{ʂæ˥-ljɤ˩}}}}\hspace{0.5cm}[\kern2pt{\textcolor{darkblue}{\textbf{\ipa{xxxx non-correspondance entre le nombre de morphèmes et le nombre de tons de morphèmes}}}}\kern2pt]} \hypertarget{s`\{\string_T-lj7\string_B1}{}
\markboth{\textcolor{darkblue}{\textbf{\ipa{ʂæ˥-ljɤ˩}}}}{}
\textcolor{teal}{\mytextsc{verbe}} \hspace{4pt} Ton~: H-.
\textcolor{PineGreen}{\selectlanguage{french}Discuter.} \zh{商量(汉语借词。}  Emprunt~: chinois  \zh{商量}

\lhead{\firstmark}
\rhead{\botmark}

\subsection{\hspace{-0.5cm} {\Large \textcolor{darkblue}{\textbf{\ipa{ʂæ˧}}}}\hspace{0.5cm}[\kern2pt{\textcolor{darkblue}{\textbf{\ipa{ʂæ˥}}}}\kern2pt]} \hypertarget{s`\{\string_M1}{}
\markboth{\textcolor{darkblue}{\textbf{\ipa{ʂæ˧}}}}{}
\textcolor{teal}{\mytextsc{adjectif}} \hspace{4pt} Ton~: M.
\ding{202} \textcolor{PineGreen}{\selectlanguage{french}Long.} \zh{长。}  ¶ \textcolor{darkblue}{\textbf{\ipa{qʰɑ˧-ʂæ˧-gv̩˧}}} \textcolor{PineGreen}{\selectlanguage{french}très long} \zh{非常长}  
 ¶ \textcolor{darkblue}{\textbf{\ipa{le˧-ʈɤ˧-le˧-ʂæ˧ (+kʰɯ˧˥)}}} \textcolor{PineGreen}{\selectlanguage{french}allonger, étirer} \zh{拉长}  
\ding{203} \textcolor{PineGreen}{\selectlanguage{french}Lointain, distant, éloigné.} \zh{远。} 
\lhead{\firstmark}
\rhead{\botmark}

\subsection{\hspace{-0.5cm} {\Large \textcolor{darkblue}{\textbf{\ipa{ʂæ˧ɖæ\#˥}}}}\hspace{0.5cm}[\kern2pt{\textcolor{darkblue}{\textbf{\ipa{ʂæ˧ɖæ˧}}}}\kern2pt]} \hypertarget{s`\{\string_Md`\{\#\string_T1}{}
\markboth{\textcolor{darkblue}{\textbf{\ipa{ʂæ˧ɖæ\#˥}}}}{}
\textcolor{teal}{\mytextsc{nom}} \hspace{4pt} Ton~: \#H.
\textcolor{PineGreen}{\selectlanguage{french}Différence de longueur.} \zh{长度区别。}  ¶ \textcolor{darkblue}{\textbf{\ipa{ʂæ˧ɖæ˧ di˥, | mɤ˧-dʑɤ˩!}}} \textcolor{PineGreen}{\selectlanguage{french}S'il y a des différences de longueur, c'est vilain/ça ne convient pas! (Contexte: explication au sujet du choix d'arbres à abattre pour obtenir du bois de charpente.)} \zh{如果长短不一,不好!/不行!(情景:解释砍树时如何选择合适的树)}  
 ¶ \textcolor{darkblue}{\textbf{\ipa{ʂæ˧ɖæ˧ | mɤ˧-di˩!}}} \textcolor{PineGreen}{\selectlanguage{french}il n'y a pas de différences de longueur (=c'est très bien)! (Même contexte que ci-dessus: choix d'arbres à abattre pour obtenir du bois de charpente)} \zh{没有长度区别,都一样齐!(等于是好的建房木料)(情景:同上)}  
 \zh{量词}: \textcolor{darkblue}{\textbf{\ipa{kʰwɤ˥}}}  \mytextsc{clf}~: \textcolor{darkblue}{\textbf{\ipa{kʰwɤ˥}}} 
\lhead{\firstmark}
\rhead{\botmark}

\subsection{\hspace{-0.5cm} {\Large \textcolor{darkblue}{\textbf{\ipa{ʂæ˧-lo˩pv˩}}}}\hspace{0.5cm}[\kern2pt{\textcolor{darkblue}{\textbf{\ipa{ʂæ˧lo˧pv˧}}}}\kern2pt]} \hypertarget{s`\{\string_M-lo\string_Bpv\string_B1}{}
\markboth{\textcolor{darkblue}{\textbf{\ipa{ʂæ˧-lo˩pv˩}}}}{}
\textcolor{teal}{\mytextsc{nom}} \hspace{4pt} Ton~: -L.
\textcolor{PineGreen}{\selectlanguage{french}Yyyy.} \zh{山萝卜。}  Emprunt~: chinois  \zh{山萝卜}
\textit{Voir~:} \hyperlink{}{\textcolor{darkblue}{\textbf{\ipa{hwɤ˧li˧-hwæ˧qʰæ\#˥}}}} 
\lhead{\firstmark}
\rhead{\botmark}

\subsection{\hspace{-0.5cm} {\Large \textcolor{darkblue}{\textbf{\ipa{ʂæ˧pʰi˧}}}}\hspace{0.5cm}[\kern2pt{\textcolor{darkblue}{\textbf{\ipa{ʂæ˩pʰi˩˥}}}}\kern2pt]} \hypertarget{s`\{\string_Mp\string_hi\string_M1}{}
\markboth{\textcolor{darkblue}{\textbf{\ipa{ʂæ˧pʰi˧}}}}{}
\textcolor{teal}{\mytextsc{nom}} \hspace{4pt} Ton~: M.
\textcolor{PineGreen}{\selectlanguage{french}Marchandise, objet qui peut se vendre au marché.} \zh{商品。}  Emprunt~: chinois  \zh{商品}

\lhead{\firstmark}
\rhead{\botmark}

\subsection{\hspace{-0.5cm} {\Large \textcolor{darkblue}{\textbf{\ipa{ʂæ˧ʁwɤ˩}}}}\hspace{0.5cm}[\kern2pt{\textcolor{darkblue}{\textbf{\ipa{ʂæ˩ʁwɤ˩˥}}}}\kern2pt]} \hypertarget{s`\{\string_MRw7\string_B1}{}
\markboth{\textcolor{darkblue}{\textbf{\ipa{ʂæ˧ʁwɤ˩}}}}{}
\textcolor{teal}{\mytextsc{nom}} \hspace{4pt} Ton~: L\#.
\textcolor{PineGreen}{\selectlanguage{french}Shuhe: nom d'un village de la plaine de Lijiang (anciennement Longquan). Les terres de ce village étaient médiocres, et beaucoup de ses habitants se tournaient vers le commerce et voyageaient dans toute la région, d'où le fait que le nom de ce village soit connu à Yongning.} \zh{束河(旧称:龙泉):丽江坝子里的一个村落。由于束河商人多,经常有束河人到永宁等地,使得相当多的永宁人熟悉那个村落名。}  Emprunt~: naxi  \textcolor{darkblue}{\textbf{\ipa{/sɑ˥wɤ˧/}}}

\lhead{\firstmark}
\rhead{\botmark}

\subsection{\hspace{-0.5cm} {\Large \textcolor{darkblue}{\textbf{\ipa{ʂæ˧tsɯ˧}}}}\hspace{0.5cm}[\kern2pt{\textcolor{darkblue}{\textbf{\ipa{ʂæ˧tsɯ˩}}}}\kern2pt]} \hypertarget{s`\{\string_MtsM\string_M1}{}
\markboth{\textcolor{darkblue}{\textbf{\ipa{ʂæ˧tsɯ˧}}}}{}
\textcolor{teal}{\mytextsc{nom}} \hspace{4pt} Ton~: M.
\textcolor{PineGreen}{\selectlanguage{french}Caftan: vêtement que portaient les enfants avant leurs treize ans: robe ample (la même pour les filles et les garçons); anciennement, les hommes aussi portaient ce type de vêtement.} \zh{裋、卡夫坦长衣:成年前男女小孩均穿的裋,成年男人也穿。}  \zh{量词}: \textcolor{darkblue}{\textbf{\ipa{ɭɯ˧˥}}}  \mytextsc{clf}~: \textcolor{darkblue}{\textbf{\ipa{ɭɯ˧˥}}} 
\lhead{\firstmark}
\rhead{\botmark}

\subsection{\hspace{-0.5cm} {\Large \textcolor{darkblue}{\textbf{\ipa{ʂæ˩ɻ̃˩}}}}\hspace{0.5cm}[\kern2pt{\textcolor{darkblue}{\textbf{\ipa{ʂæ˧ɻ̃˧˥}}}}\kern2pt]} \hypertarget{s`\{\string_Br£`\string_~\string_B1}{}
\markboth{\textcolor{darkblue}{\textbf{\ipa{ʂæ˩ɻ̃˩}}}}{}
\textcolor{teal}{\mytextsc{nom}} \hspace{4pt} Ton~: L.
\textcolor{PineGreen}{\selectlanguage{french}Os, ossement.} \zh{骨头。}  \zh{量词}: \textcolor{darkblue}{\textbf{\ipa{kɤ˧˥}}}  \mytextsc{clf}~: \textcolor{darkblue}{\textbf{\ipa{kɤ˧˥}}} 
\lhead{\firstmark}
\rhead{\botmark}

\subsection{\hspace{-0.5cm} {\Large \textcolor{darkblue}{\textbf{\ipa{ʂæ˧˥}}} \textsubscript{1}}\hspace{0.5cm}[\kern2pt{\textcolor{darkblue}{\textbf{\ipa{ʂæ˥}}}}\kern2pt]} \hypertarget{s`\{\string_M\string_T1}{}
\markboth{\textcolor{darkblue}{\textbf{\ipa{ʂæ˧˥}}} \textsubscript{1}}{}
\textcolor{teal}{\mytextsc{verbe}} \hspace{4pt} Ton~: MH.
\textcolor{PineGreen}{\selectlanguage{french}Tenir un chien en laisse, mener un chien; mener, guider, amener (les vaches aux pâturages, etc).} \zh{牵(牵着牛)。}  ¶ \textcolor{darkblue}{\textbf{\ipa{kʰv̩˧ ʂæ˧˥}}} \textcolor{PineGreen}{\selectlanguage{french}mener un chien; chasser} \zh{遛狗,狩猎}  
 ¶ \textcolor{darkblue}{\textbf{\ipa{kʰv̩˧ʂæ˧ hɯ˧˥}}} \textcolor{PineGreen}{\selectlanguage{french}parti chasser, parti à la chasse} \zh{狩猎去了}  

\lhead{\firstmark}
\rhead{\botmark}

\subsection{\hspace{-0.5cm} {\Large \textcolor{darkblue}{\textbf{\ipa{ʂæ˧˥}}} \textsubscript{2}}\hspace{0.5cm}[\kern2pt{\textcolor{darkblue}{\textbf{\ipa{ʂæ˧˥}}}}\kern2pt]} \hypertarget{s`\{\string_M\string_T2}{}
\markboth{\textcolor{darkblue}{\textbf{\ipa{ʂæ˧˥}}} \textsubscript{2}}{}
\textcolor{teal}{\mytextsc{verbe}} \hspace{4pt} Ton~: MH.
\ding{202} \textcolor{PineGreen}{\selectlanguage{french}Attacher, nouer en bottes.} \zh{捆成一包。}  ¶ \textcolor{darkblue}{\textbf{\ipa{le˧-ʂæ˧˥}}} \textcolor{PineGreen}{\selectlanguage{french}\mytextsc{accomp}} \zh{\mytextsc{accomp}}  
 ¶ \textcolor{darkblue}{\textbf{\ipa{hɑ˧ ʂæ˩}}} \textcolor{PineGreen}{\selectlanguage{french}nouer le riz coupé en bottes} \zh{刚收割的稻子,捆成捆}  
\ding{203} \textcolor{PineGreen}{\selectlanguage{french}Envelopper, emballer (monosyllabique).} \zh{包。}  ¶ \textcolor{darkblue}{\textbf{\ipa{ʂæ˩\textasciitilde{}ʂæ˧˥}}} \textcolor{PineGreen}{\selectlanguage{french}\mytextsc{red}: emballer, envelopper} \zh{\mytextsc{重叠:包一包}}  
 ¶ \textcolor{darkblue}{\textbf{\ipa{ʂæ˩\textasciitilde{}ʂæ˧-ze˥}}} \textcolor{PineGreen}{\selectlanguage{french}\mytextsc{red} \mytextsc{accomp}} \zh{\mytextsc{red} \mytextsc{accomp}}  
 ¶ \textcolor{darkblue}{\textbf{\ipa{tso˧\textasciitilde{}tso˧ ʂæ˥\textasciitilde{}ʂæ˩}}} \textcolor{PineGreen}{\selectlanguage{french}emballer des choses} \zh{包一包东西}  

\lhead{\firstmark}
\rhead{\botmark}

\subsection{\hspace{-0.5cm} {\Large \textcolor{darkblue}{\textbf{\ipa{ʂæ˧˥\textsubscript{a}}}}}\hspace{0.5cm}[\kern2pt{\textcolor{darkblue}{\textbf{\ipa{ʂæ˧˥}}}}\kern2pt]} \hypertarget{s`\{\string_M\string_Ta1}{}
\markboth{\textcolor{darkblue}{\textbf{\ipa{ʂæ˧˥\textsubscript{a}}}}}{}
\textcolor{teal}{\mytextsc{classificateur}} \hspace{4pt} Ton~: MH\textsubscript{a}.
\textcolor{PineGreen}{\selectlanguage{french}Classificateur des gerbes: ce qu'on coupe en un coup de faucille et attache d'un brin.} \zh{量词:捆。}  ¶ \textcolor{darkblue}{\textbf{\ipa{zɯ˧ | ɖɯ˧-ʂæ˧˥}}} \textcolor{PineGreen}{\selectlanguage{french}une gerbe d'herbe (nouée ensemble par un brin)} \zh{一捆草}  
 ¶ \textcolor{darkblue}{\textbf{\ipa{ɕi˧ɭɯ˧ | ɖɯ˧-ʂæ˧˥}}} \textcolor{PineGreen}{\selectlanguage{french}une gerbe de riz (nouée par un brin)} \zh{一捆稻谷}  

\lhead{\firstmark}
\rhead{\botmark}

\subsection{\hspace{-0.5cm} {\Large \textcolor{darkblue}{\textbf{\ipa{ʂe˥}}} \textsubscript{1}}\hspace{0.5cm}[\kern2pt{\textcolor{darkblue}{\textbf{\ipa{ʂe˥}}}}\kern2pt]} \hypertarget{s`e\string_T1}{}
\markboth{\textcolor{darkblue}{\textbf{\ipa{ʂe˥}}} \textsubscript{1}}{}
\textcolor{teal}{\mytextsc{nom}} \hspace{4pt} Ton~: \#H.
\textcolor{PineGreen}{\selectlanguage{french}Viande, chair.} \zh{肉,肌肉。} 
\lhead{\firstmark}
\rhead{\botmark}

\subsection{\hspace{-0.5cm} {\Large \textcolor{darkblue}{\textbf{\ipa{ʂe˥}}} \textsubscript{2}}\hspace{0.5cm}[\kern2pt{\textcolor{darkblue}{\textbf{\ipa{ʂe˥}}}}\kern2pt]} \hypertarget{s`e\string_T2}{}
\markboth{\textcolor{darkblue}{\textbf{\ipa{ʂe˥}}} \textsubscript{2}}{}
\textcolor{teal}{\mytextsc{nom}} \hspace{4pt} Ton~: \#H.
\textcolor{PineGreen}{\selectlanguage{french}Céréales pas encore mûres: céréales en herbe, dont on voit déjà l'épi mais dont l'épi ne s'est pas encore incliné sous le poids du grain.} \zh{未熟粮食。}  ¶ \textcolor{darkblue}{\textbf{\ipa{ʂe˧do˧˥}}} \textcolor{PineGreen}{\selectlanguage{french}même sens} \zh{未熟粮食}  

\lhead{\firstmark}
\rhead{\botmark}

\subsection{\hspace{-0.5cm} {\Large \textcolor{darkblue}{\textbf{\ipa{ʂe˧\textsubscript{a}}}}}\hspace{0.5cm}[\kern2pt{\textcolor{darkblue}{\textbf{\ipa{ʂe˩˥}}}}\kern2pt]} \hypertarget{s`e\string_Ma1}{}
\markboth{\textcolor{darkblue}{\textbf{\ipa{ʂe˧\textsubscript{a}}}}}{}
\textcolor{teal}{\mytextsc{verbe}} \hspace{4pt} Ton~: M\textsubscript{a}.
\textcolor{PineGreen}{\selectlanguage{french}Chercher; se procurer.} \zh{寻找。}  ¶ \textcolor{darkblue}{\textbf{\ipa{le˧-ʂe˧ le˧-ɖɯ˧-ze˧!}}} \textcolor{PineGreen}{\selectlanguage{french}(j'ai) cherché et (j'ai) trouvé!} \zh{(我)找了……就找到了! / 找到了!}  
 ¶ \textcolor{darkblue}{\textbf{\ipa{hĩ˧ ɖɯ˧-v̩˧ ʂe˧}}} \textcolor{PineGreen}{\selectlanguage{french}littéralement 'chercher quelqu'un'; sens: fréquenter quelqu'un du sexe opposé, rendre visite à quelqu'un du sexe opposé (généralement: se dit d'un homme)} \zh{直译:‘找一个人’。实际含义:去访问异性的人(一般是男人去访问女人)}  
 ¶ \textcolor{darkblue}{\textbf{\ipa{hĩ˧ ʂe˩}}} \textcolor{PineGreen}{\selectlanguage{french}prendre femme, épouser une femme} \zh{娶媳妇}  
 ¶ \textcolor{darkblue}{\textbf{\ipa{tso˧\textasciitilde{}tso˧ ʂe˩}}} \textcolor{PineGreen}{\selectlanguage{french}chercher quelque chose} \zh{找东西}  
 ¶ \textcolor{darkblue}{\textbf{\ipa{lo˧ mɤ˧-dʑo˧, | lo˧ ʂe˧!}}} \textcolor{PineGreen}{\selectlanguage{french}[Il/elle] se crée des complications / se donner du travail!} \zh{没事找事!}  
 ¶ \textcolor{darkblue}{\textbf{\ipa{le˧-ʂe˧ tʰi˧-tɕɯ˥}}} \textcolor{PineGreen}{\selectlanguage{french}préparer (des ingrédients pour une recette, ses bagages...)} \zh{准备(做饭的材料、旅途用品……)}  

\lhead{\firstmark}
\rhead{\botmark}

\subsection{\hspace{-0.5cm} {\Large \textcolor{darkblue}{\textbf{\ipa{ʂe˧bæ˧}}}}\hspace{0.5cm}[\kern2pt{\textcolor{darkblue}{\textbf{\ipa{ʂe˩bæ˩˥}}}}\kern2pt]} \hypertarget{s`e\string_Mb\{\string_M1}{}
\markboth{\textcolor{darkblue}{\textbf{\ipa{ʂe˧bæ˧}}}}{}
\textcolor{teal}{\mytextsc{nom}} \hspace{4pt} Ton~: M.
\textcolor{PineGreen}{\selectlanguage{french}Collier; chaîne.} \zh{项圈、项链,锁链。}  ¶ \textcolor{darkblue}{\textbf{\ipa{ŋv̩˩-ʂe˩bæ˥}}} \textcolor{PineGreen}{\selectlanguage{french}collier en argent} \zh{银项链}  
 ¶ \textcolor{darkblue}{\textbf{\ipa{hæ̃˩-ʂe˩bæ˥}}} \textcolor{PineGreen}{\selectlanguage{french}collier en or} \zh{金项链}  
 ¶ \textcolor{darkblue}{\textbf{\ipa{ʂe˧mo˧ʂe˧bæ˧, | kʰv̩˩mi˩ pʰæ˩˥!}}} \textcolor{PineGreen}{\selectlanguage{french}Le collier de fer, c'est pour attacher le chien!} \zh{铁链,是来用拴狗的!}  
 ¶ \textcolor{darkblue}{\textbf{\ipa{kʰi˧-ʂe˧bæ˥, | ʂe˧mo˧ po˧-ɳɯ˧ | gv̩˩˥!}}} \zh{铁链,是来用拴狗的!}  
 \zh{量词}: \textcolor{darkblue}{\textbf{\ipa{kʰɯ˩}}}  \mytextsc{clf}~: \textcolor{darkblue}{\textbf{\ipa{kʰɯ˩}}} 
\lhead{\firstmark}
\rhead{\botmark}

\subsection{\hspace{-0.5cm} {\Large \textcolor{darkblue}{\textbf{\ipa{ʂe˧bv̩\#˥}}}}\hspace{0.5cm}[\kern2pt{\textcolor{darkblue}{\textbf{\ipa{ʂe˧bv̩˧}}}}\kern2pt]} \hypertarget{s`e\string_Mbv\string_=\#\string_T1}{}
\markboth{\textcolor{darkblue}{\textbf{\ipa{ʂe˧bv̩\#˥}}}}{}
\textcolor{teal}{\mytextsc{nom}} \hspace{4pt} Ton~: \#H.
\textcolor{PineGreen}{\selectlanguage{french}Saucisse; viande séchée conservée dans les intestins.} \zh{香肠,把瘦肉装在肠子里。} 
\lhead{\firstmark}
\rhead{\botmark}

\subsection{\hspace{-0.5cm} {\Large \textcolor{darkblue}{\textbf{\ipa{ʂe˧di˩}}}}\hspace{0.5cm}[\kern2pt{\textcolor{darkblue}{\textbf{\ipa{ʂe˩di˩˥}}}}\kern2pt]} \hypertarget{s`e\string_Mdi\string_B1}{}
\markboth{\textcolor{darkblue}{\textbf{\ipa{ʂe˧di˩}}}}{}
\textcolor{teal}{\mytextsc{adjectif}} \hspace{4pt} Ton~: L\#.
\textcolor{PineGreen}{\selectlanguage{french}Gros.} \zh{胖。}  ¶ \textcolor{darkblue}{\textbf{\ipa{ʂe˧ di˩-ze˩!}}} \textcolor{PineGreen}{\selectlanguage{french}(il/elle) a grossi!} \zh{胖了!}  

\lhead{\firstmark}
\rhead{\botmark}

\subsection{\hspace{-0.5cm} {\Large \textcolor{darkblue}{\textbf{\ipa{ʂe˧dzo\#˥}}}}\hspace{0.5cm}[\kern2pt{\textcolor{darkblue}{\textbf{\ipa{ʂe˧dzo˩}}}}\kern2pt]} \hypertarget{s`e\string_Mdzo\#\string_T1}{}
\markboth{\textcolor{darkblue}{\textbf{\ipa{ʂe˧dzo\#˥}}}}{}
\textcolor{teal}{\mytextsc{nom}} \hspace{4pt} Ton~: \#H.
\textcolor{PineGreen}{\selectlanguage{french}Meuble de cuisine: structure en bois sur laquelle on fait la cuisine: on y pose la planche à découper, les ustensiles….} \zh{放案板的家具。}  \zh{量词}: \textcolor{darkblue}{\textbf{\ipa{pɤ˩}}}  \mytextsc{clf}~: \textcolor{darkblue}{\textbf{\ipa{pɤ˩}}} 
\lhead{\firstmark}
\rhead{\botmark}

\subsection{\hspace{-0.5cm} {\Large \textcolor{darkblue}{\textbf{\ipa{ʂe˧kʰɯ˧}}}}\hspace{0.5cm}[\kern2pt{\textcolor{darkblue}{\textbf{\ipa{ʂe˧kʰɯ˩}}}}\kern2pt]} \hypertarget{s`e\string_Mk\string_hM\string_M1}{}
\markboth{\textcolor{darkblue}{\textbf{\ipa{ʂe˧kʰɯ˧}}}}{}
\textcolor{teal}{\mytextsc{nom}} \hspace{4pt} Ton~: M.
\textcolor{PineGreen}{\selectlanguage{french}Trépied de fer (dans le foyer, sur lequel on pose une casserole, une poële, une bouilloire…).} \zh{三脚架。}  \zh{量词}: \textcolor{darkblue}{\textbf{\ipa{nɑ˧}}}  \mytextsc{clf}~: \textcolor{darkblue}{\textbf{\ipa{nɑ˧}}} 
\lhead{\firstmark}
\rhead{\botmark}

\subsection{\hspace{-0.5cm} {\Large \textcolor{darkblue}{\textbf{\ipa{ʂe˧mi˧}}}}\hspace{0.5cm}[\kern2pt{\textcolor{darkblue}{\textbf{\ipa{ʂe˩mi˥}}}}\kern2pt]} \hypertarget{s`e\string_Mmi\string_M1}{}
\markboth{\textcolor{darkblue}{\textbf{\ipa{ʂe˧mi˧}}}}{}
\textcolor{teal}{\mytextsc{nom}} \hspace{4pt} Ton~: M.
\textcolor{PineGreen}{\selectlanguage{french}Pou.} \zh{虱子。}  \zh{量词}: \textcolor{darkblue}{\textbf{\ipa{mi˩}}}  \mytextsc{clf}~: \textcolor{darkblue}{\textbf{\ipa{mi˩}}} 
\lhead{\firstmark}
\rhead{\botmark}

\subsection{\hspace{-0.5cm} {\Large \textcolor{darkblue}{\textbf{\ipa{ʂe˧mo˧}}}}\hspace{0.5cm}[\kern2pt{\textcolor{darkblue}{\textbf{\ipa{ʂe˩mo˩˥}}}}\kern2pt]} \hypertarget{s`e\string_Mmo\string_M1}{}
\markboth{\textcolor{darkblue}{\textbf{\ipa{ʂe˧mo˧}}}}{}
\textcolor{teal}{\mytextsc{nom}} \hspace{4pt} Ton~: M.
\textcolor{PineGreen}{\selectlanguage{french}Fer (disyllabe).} \zh{铁(双音节)。} \textit{Voir~:} \hyperlink{}{\textcolor{darkblue}{\textbf{\ipa{ʂe˩}}}} 
\lhead{\firstmark}
\rhead{\botmark}

\subsection{\hspace{-0.5cm} {\Large \textcolor{darkblue}{\textbf{\ipa{ʂe˧nɑ˩}}}}\hspace{0.5cm}[\kern2pt{\textcolor{darkblue}{\textbf{\ipa{ʂe˧nɑ˧}}}}\kern2pt]} \hypertarget{s`e\string_MnA\string_B1}{}
\markboth{\textcolor{darkblue}{\textbf{\ipa{ʂe˧nɑ˩}}}}{}
\textcolor{teal}{\mytextsc{nom}} \hspace{4pt} Ton~: L\#.
\textcolor{PineGreen}{\selectlanguage{french}Viande maigre.} \zh{瘦肉。} 
\lhead{\firstmark}
\rhead{\botmark}

\subsection{\hspace{-0.5cm} {\Large \textcolor{darkblue}{\textbf{\ipa{ʂe˧ɲi˩}}}}\hspace{0.5cm}[\kern2pt{\textcolor{darkblue}{\textbf{\ipa{ʂe˧ɲi˧}}}}\kern2pt]} \hypertarget{s`e\string_MJi\string_B1}{}
\markboth{\textcolor{darkblue}{\textbf{\ipa{ʂe˧ɲi˩}}}}{}
\textcolor{teal}{\mytextsc{nom}} \hspace{4pt} Ton~: L\#.
\textcolor{PineGreen}{\selectlanguage{french}Pince à braises.} \zh{火钳。}  \zh{量词}: \textcolor{darkblue}{\textbf{\ipa{nɑ˧}}}  \mytextsc{clf}~: \textcolor{darkblue}{\textbf{\ipa{nɑ˧}}} 
\lhead{\firstmark}
\rhead{\botmark}

\subsection{\hspace{-0.5cm} {\Large \textcolor{darkblue}{\textbf{\ipa{ʂe˧pv̩˩}}}}\hspace{0.5cm}[\kern2pt{\textcolor{darkblue}{\textbf{\ipa{ʂe˧pv̩˧}}}}\kern2pt]} \hypertarget{s`e\string_Mpv\string_=\string_B1}{}
\markboth{\textcolor{darkblue}{\textbf{\ipa{ʂe˧pv̩˩}}}}{}
\textcolor{teal}{\mytextsc{nom}} \hspace{4pt} Ton~: L\#.
\textcolor{PineGreen}{\selectlanguage{french}Viande séchée.} \zh{腊肉。} 
\lhead{\firstmark}
\rhead{\botmark}

\subsection{\hspace{-0.5cm} {\Large \textcolor{darkblue}{\textbf{\ipa{ʂe˧qʰv̩˧}}}}\hspace{0.5cm}[\kern2pt{\textcolor{darkblue}{\textbf{\ipa{ʂe˧qʰv̩˧}}}}\kern2pt]} \hypertarget{s`e\string_Mq\string_hv\string_=\string_M1}{}
\markboth{\textcolor{darkblue}{\textbf{\ipa{ʂe˧qʰv̩˧}}}}{}
\textcolor{teal}{\mytextsc{nom}} \hspace{4pt} Ton~: M.
\textcolor{PineGreen}{\selectlanguage{french}Clou en fer.} \zh{铁钉,钉子。}  ¶ \textcolor{darkblue}{\textbf{\ipa{ʂe˧qʰv̩˧ lɑ˧˥}}} \textcolor{PineGreen}{\selectlanguage{french}enfoncer un clou, planter un clou} \zh{钉钉子}  
 \zh{量词}: \textcolor{darkblue}{\textbf{\ipa{ɭɯ˧}}}  \mytextsc{clf}~: \textcolor{darkblue}{\textbf{\ipa{ɭɯ˧}}} 
\lhead{\firstmark}
\rhead{\botmark}

\subsection{\hspace{-0.5cm} {\Large \textcolor{darkblue}{\textbf{\ipa{ʂe˧sɑ˩}}}}\hspace{0.5cm}[\kern2pt{\textcolor{darkblue}{\textbf{\ipa{ʂe˧sɑ˧}}}}\kern2pt]} \hypertarget{s`e\string_MsA\string_B1}{}
\markboth{\textcolor{darkblue}{\textbf{\ipa{ʂe˧sɑ˩}}}}{}
\textcolor{teal}{\mytextsc{nom}} \hspace{4pt} Ton~: L\#.
\textcolor{PineGreen}{\selectlanguage{french}Viande des membres du cochon: les membres postérieurs aussi bien que les membres antérieurs. Le terme s'emploie pour la pièce de boucherie: de la viande conservée (séchée) avec l'os; mais le même terme peut également s'employer pour désigner les membres de la bête vivante.} \zh{猪腿肉。}  \zh{量词}: \textcolor{darkblue}{\textbf{\ipa{sɑ˧˥}}}  \mytextsc{clf}~: \textcolor{darkblue}{\textbf{\ipa{sɑ˧˥}}} 
\lhead{\firstmark}
\rhead{\botmark}

\subsection{\hspace{-0.5cm} {\Large \textcolor{darkblue}{\textbf{\ipa{ʂe˧-sɯ˧sɯ˥}}}}\hspace{0.5cm}[\kern2pt{\textcolor{darkblue}{\textbf{\ipa{xxxx non-correspondance entre le nombre de morphèmes et le nombre de tons de morphèmes}}}}\kern2pt]} \hypertarget{s`e\string_M-sM\string_MsM\string_T1}{}
\markboth{\textcolor{darkblue}{\textbf{\ipa{ʂe˧-sɯ˧sɯ˥}}}}{}
\textcolor{teal}{\mytextsc{nom}} \hspace{4pt} Ton~: H\#.
\textcolor{PineGreen}{\selectlanguage{french}Viande crue.} \zh{生肉。} 
\lhead{\firstmark}
\rhead{\botmark}

\subsection{\hspace{-0.5cm} {\Large \textcolor{darkblue}{\textbf{\ipa{ʂe˧ʂe˧}}}}\hspace{0.5cm}[\kern2pt{\textcolor{darkblue}{\textbf{\ipa{ʂe˧ʂe˩}}}}\kern2pt]} \hypertarget{s`e\string_Ms`e\string_M1}{}
\markboth{\textcolor{darkblue}{\textbf{\ipa{ʂe˧ʂe˧}}}}{}
\textcolor{teal}{\mytextsc{verbe}} \hspace{4pt} Ton~: M.
\textcolor{PineGreen}{\selectlanguage{french}Prendre froid, attraper un rhume, attraper froid.} \zh{着凉。}  ¶ \textcolor{darkblue}{\textbf{\ipa{ʂe˧ʂe˧-ze˩}}} \textcolor{PineGreen}{\selectlanguage{french}\mytextsc{pfv}} \zh{着凉了}  

\lhead{\firstmark}
\rhead{\botmark}

\subsection{\hspace{-0.5cm} {\Large \textcolor{darkblue}{\textbf{\ipa{ʂe˧ʈʂe˩}}}}\hspace{0.5cm}[\kern2pt{\textcolor{darkblue}{\textbf{\ipa{ʂe˩ʈʂe˩˥}}}}\kern2pt]} \hypertarget{s`e\string_Mt`s`e\string_B1}{}
\markboth{\textcolor{darkblue}{\textbf{\ipa{ʂe˧ʈʂe˩}}}}{}
\textcolor{teal}{\mytextsc{nom}} \hspace{4pt} Ton~: L\#.
\textcolor{PineGreen}{\selectlanguage{french}Tissu de coton.} \zh{棉布,布料。}  \zh{量词}: \textcolor{darkblue}{\textbf{\ipa{pʰæ˧˥}}} \textcolor{darkblue}{\textbf{\ipa{kʰɤ˥}}}  \mytextsc{clf}~: \textcolor{darkblue}{\textbf{\ipa{pʰæ˧˥}}} \textcolor{darkblue}{\textbf{\ipa{kʰɤ˥}}} 
\lhead{\firstmark}
\rhead{\botmark}

\subsection{\hspace{-0.5cm} {\Large \textcolor{darkblue}{\textbf{\ipa{ʂe˧ʐe\#˥}}}}\hspace{0.5cm}[\kern2pt{\textcolor{darkblue}{\textbf{\ipa{ʂe˧ʐe˩}}}}\kern2pt]} \hypertarget{s`e\string_Mz`e\#\string_T1}{}
\markboth{\textcolor{darkblue}{\textbf{\ipa{ʂe˧ʐe\#˥}}}}{}
\textcolor{teal}{\mytextsc{nom}} \hspace{4pt} Ton~: \#H.
\textcolor{PineGreen}{\selectlanguage{french}Viande de cochon préservée. Le terme recouvre diverses pièces de boucherie, dont le jambon.} \zh{腊肉,包括不同几类的腊肉,如火腿等。}  \zh{量词}: \textcolor{darkblue}{\textbf{\ipa{ʐe˥}}}  \mytextsc{clf}~: \textcolor{darkblue}{\textbf{\ipa{ʐe˥}}} 
\lhead{\firstmark}
\rhead{\botmark}

\subsection{\hspace{-0.5cm} {\Large \textcolor{darkblue}{\textbf{\ipa{ʂe˩}}}}\hspace{0.5cm}[\kern2pt{\textcolor{darkblue}{\textbf{\ipa{ʂe˥}}}}\kern2pt]} \hypertarget{s`e\string_B1}{}
\markboth{\textcolor{darkblue}{\textbf{\ipa{ʂe˩}}}}{}
\textcolor{teal}{\mytextsc{nom}} \hspace{4pt} Ton~: L.
\textcolor{PineGreen}{\selectlanguage{french}Fer (monosyllabe).} \zh{铁(单音节)。} 
\lhead{\firstmark}
\rhead{\botmark}

\subsection{\hspace{-0.5cm} {\Large \textcolor{darkblue}{\textbf{\ipa{ʂe˩\textsubscript{b}}}}}\hspace{0.5cm}[\kern2pt{\textcolor{darkblue}{\textbf{\ipa{ʂe˥}}}}\kern2pt]} \hypertarget{s`e\string_Bb1}{}
\markboth{\textcolor{darkblue}{\textbf{\ipa{ʂe˩\textsubscript{b}}}}}{}
\textcolor{teal}{\mytextsc{verbe}} \hspace{4pt} Ton~: L\textsubscript{b}.
\textcolor{PineGreen}{\selectlanguage{french}Uriner, pisser, faire pipi; déféquer.} \zh{小便,尿; 屙尿; 解溲; 拉(屎)。}  ¶ \textcolor{darkblue}{\textbf{\ipa{dʑi˧ ʂe˧˥}}} \textcolor{PineGreen}{\selectlanguage{french}pisser} \zh{屙尿}  
 ¶ \textcolor{darkblue}{\textbf{\ipa{qʰæ˧ ʂe˧˥}}} \textcolor{PineGreen}{\selectlanguage{french}déféquer} \zh{拉屎}  
 ¶ \textcolor{darkblue}{\textbf{\ipa{le˧-ʂe˩-ze˩}}} \textcolor{PineGreen}{\selectlanguage{french}\mytextsc{accomp} \string_ \mytextsc{pfv}} \zh{尿了}  
 ¶ \textcolor{darkblue}{\textbf{\ipa{ɖɯ˧-ʈʰɤ˧ ʂe˧˥}}} \textcolor{PineGreen}{\selectlanguage{french}pisser une goutte} \zh{尿一滴尿}  

\lhead{\firstmark}
\rhead{\botmark}

\subsection{\hspace{-0.5cm} {\Large \textcolor{darkblue}{\textbf{\ipa{ʂe˩lɑ˩}}}}\hspace{0.5cm}[\kern2pt{\textcolor{darkblue}{\textbf{\ipa{ʂe˧lɑ˧}}}}\kern2pt]} \hypertarget{s`e\string_BlA\string_B1}{}
\markboth{\textcolor{darkblue}{\textbf{\ipa{ʂe˩lɑ˩}}}}{}
\textcolor{teal}{\mytextsc{nom}} \hspace{4pt} Ton~: L.
\textcolor{PineGreen}{\selectlanguage{french}Forger, battre le fer.} \zh{打铁。} 
\lhead{\firstmark}
\rhead{\botmark}

\subsection{\hspace{-0.5cm} {\Large \textcolor{darkblue}{\textbf{\ipa{ʂe˩-lɑ˩-hĩ˥}}}}\hspace{0.5cm}[\kern2pt{\textcolor{darkblue}{\textbf{\ipa{xxxx non-correspondance entre le nombre de morphèmes et le nombre de tons de morphèmes}}}}\kern2pt]} \hypertarget{s`e\string_B-lA\string_B-hi\string_~\string_T1}{}
\markboth{\textcolor{darkblue}{\textbf{\ipa{ʂe˩-lɑ˩-hĩ˥}}}}{}
\textcolor{teal}{\mytextsc{nom}} \hspace{4pt} Ton~: L+H\#.
\textcolor{PineGreen}{\selectlanguage{french}Forgeron.} \zh{铁匠。}  ¶ \textcolor{darkblue}{\textbf{\ipa{ʂe˩lɑ˩-hĩ˥ hĩ˩}}} \textcolor{PineGreen}{\selectlanguage{french}forgeron} \zh{铁匠}  
 \zh{量词}: \textcolor{darkblue}{\textbf{\ipa{v̩˧}}}  \mytextsc{clf}~: \textcolor{darkblue}{\textbf{\ipa{v̩˧}}} 
\lhead{\firstmark}
\rhead{\botmark}

\subsection{\hspace{-0.5cm} {\Large \textcolor{darkblue}{\textbf{\ipa{ʂe˩mɤ˩}}}}\hspace{0.5cm}[\kern2pt{\textcolor{darkblue}{\textbf{\ipa{ʂe˩mɤ˧˥}}}}\kern2pt]} \hypertarget{s`e\string_Bm7\string_B1}{}
\markboth{\textcolor{darkblue}{\textbf{\ipa{ʂe˩mɤ˩}}}}{}
\textcolor{teal}{\mytextsc{nom}} \hspace{4pt} Ton~: L.
\textcolor{PineGreen}{\selectlanguage{french}Viande grasse.} \zh{肥肉。} 
\lhead{\firstmark}
\rhead{\botmark}

\subsection{\hspace{-0.5cm} {\Large \textcolor{darkblue}{\textbf{\ipa{ʂe˩-mo˧˥}}}}\hspace{0.5cm}[\kern2pt{\textcolor{darkblue}{\textbf{\ipa{xxxx non-correspondance entre le nombre de morphèmes et le nombre de tons de morphèmes}}}}\kern2pt]} \hypertarget{s`e\string_B-mo\string_M\string_T1}{}
\markboth{\textcolor{darkblue}{\textbf{\ipa{ʂe˩-mo˧˥}}}}{}
\textcolor{teal}{\mytextsc{nom}} \hspace{4pt} Ton~: LM+MH\#.
\textcolor{PineGreen}{\selectlanguage{french}Champignon des pins, matsutake, \textit{Tricholoma matsutake}: un champignon comestible, rare et très apprécié.} \zh{松茸。} 
\lhead{\firstmark}
\rhead{\botmark}

\subsection{\hspace{-0.5cm} {\Large \textcolor{darkblue}{\textbf{\ipa{ʂe˩ʂv̩˩}}}}\hspace{0.5cm}[\kern2pt{\textcolor{darkblue}{\textbf{\ipa{ʂe˧ʂv̩˥}}}}\kern2pt]} \hypertarget{s`e\string_Bs`v\string_=\string_B1}{}
\markboth{\textcolor{darkblue}{\textbf{\ipa{ʂe˩ʂv̩˩}}}}{}
\textcolor{teal}{\mytextsc{nom}} \hspace{4pt} Ton~: L.
\textcolor{PineGreen}{\selectlanguage{french}Lente, oeuf de pou.} \zh{虮子。}  \zh{量词}: \textcolor{darkblue}{\textbf{\ipa{ɭɯ˧}}}  \mytextsc{clf}~: \textcolor{darkblue}{\textbf{\ipa{ɭɯ˧}}} 
\lhead{\firstmark}
\rhead{\botmark}

\subsection{\hspace{-0.5cm} {\Large \textcolor{darkblue}{\textbf{\ipa{ʂɤ˧do˧˥}}}}\hspace{0.5cm}[\kern2pt{\textcolor{darkblue}{\textbf{\ipa{ʂɤ˩do˩˥}}}}\kern2pt]} \hypertarget{s`7\string_Mdo\string_M\string_T1}{}
\markboth{\textcolor{darkblue}{\textbf{\ipa{ʂɤ˧do˧˥}}}}{}
\textcolor{teal}{\mytextsc{adjectif}} \hspace{4pt} Ton~: MH\#.
\ding{202} \textcolor{PineGreen}{\selectlanguage{french}Honteux.} \zh{害羞。}  ¶ \textcolor{darkblue}{\textbf{\ipa{ʂɤ˧do˧ mɤ˧-sɯ˥!}}} \textcolor{PineGreen}{\selectlanguage{french}(il/elle) est effronté(e), ne connaît pas la politesse, est malpoli} \zh{不知羞耻!}  
\ding{203} \textcolor{PineGreen}{\selectlanguage{french}Pudique; poli.} \zh{娴静、礼貌。}  ¶ \textcolor{darkblue}{\textbf{\ipa{ʈʂʰɯ˧ ʂɤ˧do˧-zo˥! / ʂɤ˧do˧ ʝi˥!}}} \textcolor{PineGreen}{\selectlanguage{french}Cette personne a de la pudeur!} \zh{他/她很娴静 / 很持重!}  

\lhead{\firstmark}
\rhead{\botmark}

\subsection{\hspace{-0.5cm} {\Large \textcolor{darkblue}{\textbf{\ipa{ʂɤ˧ɲi\#˥}}}}\hspace{0.5cm}[\kern2pt{\textcolor{darkblue}{\textbf{\ipa{ʂɤ˧ɲi˧}}}}\kern2pt]} \hypertarget{s`7\string_MJi\#\string_T1}{}
\markboth{\textcolor{darkblue}{\textbf{\ipa{ʂɤ˧ɲi\#˥}}}}{}
\textcolor{teal}{\mytextsc{nom}} \hspace{4pt} Ton~: \#H.
\textcolor{PineGreen}{\selectlanguage{french}Conseil, avis.} \zh{建议、意见。}  ¶ \textcolor{darkblue}{\textbf{\ipa{ʂɤ˧ɲi˧ ʑi˧˥}}} \textcolor{PineGreen}{\selectlanguage{french}demander un conseil / prendre le conseil (de quelqu'un)} \zh{请求意见,求教}  
 ¶ \textcolor{darkblue}{\textbf{\ipa{no˧ | hĩ˧-ki˧ | ʂɤ˧ɲi˧ mɤ˧-ʑi˧-zo˥!}}} \textcolor{PineGreen}{\selectlanguage{french}Tu n'as pas à prendre son conseil! / Tu n'as pas à prendre le conseil d'autrui [à ce sujet: à toi de décider]!} \zh{你不要问人家的意见!}  
 ¶ \textcolor{darkblue}{\textbf{\ipa{ə˧tse˧ʝi˧-zo˥ | ʂɤ˧ɲi˧ ʑi˧-tso˧-ɲi˥?}}} \textcolor{PineGreen}{\selectlanguage{french}Pourquoi donc lui demandes-tu conseil?} \zh{你为什么要问(他的)意见!}  
 \zh{量词}: \textcolor{darkblue}{\textbf{\ipa{kʰwɤ˥}}}  \mytextsc{clf}~: \textcolor{darkblue}{\textbf{\ipa{kʰwɤ˥}}} 
\lhead{\firstmark}
\rhead{\botmark}

\subsection{\hspace{-0.5cm} {\Large \textcolor{darkblue}{\textbf{\ipa{ʂɤ˩\textsubscript{a}}}}}\hspace{0.5cm}[\kern2pt{\textcolor{darkblue}{\textbf{\ipa{ʂɤ˩˥}}}}\kern2pt]} \hypertarget{s`7\string_Ba1}{}
\markboth{\textcolor{darkblue}{\textbf{\ipa{ʂɤ˩\textsubscript{a}}}}}{}
\textcolor{teal}{\mytextsc{verbe}} \hspace{4pt} Ton~: L\textsubscript{a}.
\textcolor{PineGreen}{\selectlanguage{french}Déchirer (ex.: du papier).} \zh{撕(纸……)。}  ¶ \textcolor{darkblue}{\textbf{\ipa{tso˧\textasciitilde{}tso˧ ʂɤ˥}}} \textcolor{PineGreen}{\selectlanguage{french}déchirer des choses} \zh{撕东西}  
 ¶ \textcolor{darkblue}{\textbf{\ipa{tso˧\textasciitilde{}tso˧ ʂɤ˧\textasciitilde{}ʂɤ˥ (+ze˩)}}} \textcolor{PineGreen}{\selectlanguage{french}déchirer des choses} \zh{撕东西}  
 ¶ \textcolor{darkblue}{\textbf{\ipa{le˧-ʂɤ˧\textasciitilde{}ʂɤ˥+ze˩}}} \textcolor{PineGreen}{\selectlanguage{french}\mytextsc{accomp} \string_ \mytextsc{red} \mytextsc{pfv}} \zh{撕了}  

\lhead{\firstmark}
\rhead{\botmark}

\subsection{\hspace{-0.5cm} {\Large \textcolor{darkblue}{\textbf{\ipa{ʂɤ˩ŋɤ\#˥}}}}\hspace{0.5cm}[\kern2pt{\textcolor{darkblue}{\textbf{\ipa{ʂɤ˩ŋɤ˥}}}}\kern2pt]} \hypertarget{s`7\string_BN7\#\string_T1}{}
\markboth{\textcolor{darkblue}{\textbf{\ipa{ʂɤ˩ŋɤ\#˥}}}}{}
\textcolor{teal}{\mytextsc{nom}} \hspace{4pt} Ton~: LM+\#H.
\textcolor{PineGreen}{\selectlanguage{french}Gong.} \zh{锣。}  ¶ \textcolor{darkblue}{\textbf{\ipa{ʂɤ˩ŋɤ˧ lɑ˩}}} \textcolor{PineGreen}{\selectlanguage{french}jouer du gong} \zh{打锣}  
 \zh{量词}: \textcolor{darkblue}{\textbf{\ipa{ɭɯ˧}}}  \mytextsc{clf}~: \textcolor{darkblue}{\textbf{\ipa{ɭɯ˧}}} 
\lhead{\firstmark}
\rhead{\botmark}

\subsection{\hspace{-0.5cm} {\Large \textcolor{darkblue}{\textbf{\ipa{ʂo˥}}}}\hspace{0.5cm}[\kern2pt{\textcolor{darkblue}{\textbf{\ipa{ʂo˥}}}}\kern2pt]} \hypertarget{s`o\string_T1}{}
\markboth{\textcolor{darkblue}{\textbf{\ipa{ʂo˥}}}}{}
\textcolor{teal}{\mytextsc{verbe}} \hspace{4pt} Ton~: H.
\textcolor{PineGreen}{\selectlanguage{french}Récolter.} \zh{收割。}  ¶ \textcolor{darkblue}{\textbf{\ipa{le˧-ʂo˥-ze˩}}} \textcolor{PineGreen}{\selectlanguage{french}\mytextsc{accomp} \string_ \mytextsc{pfv}} \zh{收割了}  
 ¶ \textcolor{darkblue}{\textbf{\ipa{ɖɯ˧-kʰv̩˥ ɖɯ˧-ʂɯ˩ | gɤ˩-ʂo˥-ze˩!}}} \textcolor{PineGreen}{\selectlanguage{french}on récolte (le riz) une fois par an!} \zh{每年收一次稻谷!}  

\lhead{\firstmark}
\rhead{\botmark}

\subsection{\hspace{-0.5cm} {\Large \textcolor{darkblue}{\textbf{\ipa{ʂo˧}}}}\hspace{0.5cm}[\kern2pt{\textcolor{darkblue}{\textbf{\ipa{ʂo˥}}}}\kern2pt]} \hypertarget{s`o\string_M1}{}
\markboth{\textcolor{darkblue}{\textbf{\ipa{ʂo˧}}}}{}
\textcolor{teal}{\mytextsc{verbe}} \hspace{4pt} Ton~: M.
\textcolor{PineGreen}{\selectlanguage{french}Rassembler, assembler, accumuler.} \zh{收集。}  ¶ \textcolor{darkblue}{\textbf{\ipa{le˧-ʂo˧\textasciitilde{}ʂo˧}}} \textcolor{PineGreen}{\selectlanguage{french}\mytextsc{accomp} \string_ \mytextsc{red}} \zh{\mytextsc{accomp} \string_ \mytextsc{red}}  
 ¶ \textcolor{darkblue}{\textbf{\ipa{ʂo˧\textasciitilde{}ʂo˧-zo˧-ho˩-ze˩}}} \textcolor{PineGreen}{\selectlanguage{french}Il va falloir rassembler/assembler.} \zh{该收集一些了。}  

\lhead{\firstmark}
\rhead{\botmark}

\subsection{\hspace{-0.5cm} {\Large \textcolor{darkblue}{\textbf{\ipa{ʂo˧}}}}\hspace{0.5cm}[\kern2pt{\textcolor{darkblue}{\textbf{\ipa{ʂo˥}}}}\kern2pt]} \hypertarget{s`o\string_M1}{}
\markboth{\textcolor{darkblue}{\textbf{\ipa{ʂo˧}}}}{}
\textcolor{teal}{\mytextsc{interjection}} \hspace{4pt} Ton~: M.
\textcolor{PineGreen}{\selectlanguage{french}Interjection employée pour faire avancer les cochons, lorsqu'on les guide sur le chemin du pâturage: “Zou! / Allez!”.} \zh{赶猪用的叹词:走!走!。}  ¶ \textcolor{darkblue}{\textbf{\ipa{ʂo˧! / ʂo˧bɤ˩!}}} \textcolor{PineGreen}{\selectlanguage{french}Interjection employée pour faire avancer les cochons, lorsqu'on les guide sur le chemin du pâturage: “Zou! / Allez!”} \zh{赶猪用的叹词}  

\lhead{\firstmark}
\rhead{\botmark}

\subsection{\hspace{-0.5cm} {\Large \textcolor{darkblue}{\textbf{\ipa{ʂo˩\textsubscript{a}}}}}\hspace{0.5cm}[\kern2pt{\textcolor{darkblue}{\textbf{\ipa{ʂo˥}}}}\kern2pt]} \hypertarget{s`o\string_Ba1}{}
\markboth{\textcolor{darkblue}{\textbf{\ipa{ʂo˩\textsubscript{a}}}}}{}
\textcolor{teal}{\mytextsc{adjectif}} \hspace{4pt} Ton~: L\textsubscript{a}.
\textcolor{PineGreen}{\selectlanguage{french}Propre (sens propre ou figuré); claire (eau).} \zh{干净、整洁,本质干净,清(水)。}  ¶ \textcolor{darkblue}{\textbf{\ipa{ʂo˩-hĩ˩˥}}} \textcolor{PineGreen}{\selectlanguage{french}\mytextsc{nmlz}} \zh{干净的}  
 ¶ \textcolor{darkblue}{\textbf{\ipa{mɤ˧-ʂo˩}}} \textcolor{PineGreen}{\selectlanguage{french}sale, malpropre} \zh{不干净、脏}  
 ¶ \textcolor{darkblue}{\textbf{\ipa{ʈʂʰɯ˧ | ʂo˩-hĩ˩ ɲi˥. |}}} \textcolor{PineGreen}{\selectlanguage{french}C'est propre.} \zh{这是干净的。}  
 ¶ \textcolor{darkblue}{\textbf{\ipa{dʑɯ˧ ʂo˧}}} \textcolor{PineGreen}{\selectlanguage{french}de l'eau claire, de l'eau propre} \zh{清水、干净的水}  

\lhead{\firstmark}
\rhead{\botmark}

\subsection{\hspace{-0.5cm} {\Large \textcolor{darkblue}{\textbf{\ipa{ʂo˩qæ˩}}}}\hspace{0.5cm}[\kern2pt{\textcolor{darkblue}{\textbf{\ipa{ʂo˩qæ˩˥}}}}\kern2pt]} \hypertarget{s`o\string_Bq\{\string_B1}{}
\markboth{\textcolor{darkblue}{\textbf{\ipa{ʂo˩qæ˩}}}}{}
\textcolor{teal}{\mytextsc{adjectif}} \hspace{4pt} Ton~: L.
\textcolor{PineGreen}{\selectlanguage{french}Tout propre.} \zh{很干净。}  ¶ \textcolor{darkblue}{\textbf{\ipa{ʂo˩qæ˩˥ | -gv̩˩}}} \textcolor{PineGreen}{\selectlanguage{french}tout propre} \zh{很干净}  
 ¶ \textcolor{darkblue}{\textbf{\ipa{ɑ˩ʁo˧ | le˧-gv̩˧\textasciitilde{}gv̩˥ | ʂo˩qæ˩˥ | -gv̩˩}}} \textcolor{PineGreen}{\selectlanguage{french}ranger la maison, qu'elle soit bien propre} \zh{家收拾得干干净净}  

\lhead{\firstmark}
\rhead{\botmark}

\subsection{\hspace{-0.5cm} {\Large \textcolor{darkblue}{\textbf{\ipa{ʂo˧˥}}} \textsubscript{1}}\hspace{0.5cm}[\kern2pt{\textcolor{darkblue}{\textbf{\ipa{ʂo˧˥}}}}\kern2pt]} \hypertarget{s`o\string_M\string_T1}{}
\markboth{\textcolor{darkblue}{\textbf{\ipa{ʂo˧˥}}} \textsubscript{1}}{}
\textcolor{teal}{\mytextsc{verbe}} \hspace{4pt} Ton~: MH.
\textcolor{PineGreen}{\selectlanguage{french}Glisser.} \zh{滑,光滑(路……)。}  ¶ \textcolor{darkblue}{\textbf{\ipa{mv̩˩tɕo˧ ʂo˧˥}}} \textcolor{PineGreen}{\selectlanguage{french}glisser vers le bas, glisser par terre} \zh{滑下、滑倒}  
 ¶ \textcolor{darkblue}{\textbf{\ipa{ʈʂʰɯ˧ | le˧-ʂo˧˥, | tʰi˧-ʈwæ˧-ze˥}}} \textcolor{PineGreen}{\selectlanguage{french}il a glissé et il est tombé} \zh{他滑了一跤}  
 ¶ \textcolor{darkblue}{\textbf{\ipa{ʂo˩\textasciitilde{}ʂo˧˥}}} \textcolor{PineGreen}{\selectlanguage{french}\mytextsc{red}} \zh{\mytextsc{重叠}}  
 ¶ \textcolor{darkblue}{\textbf{\ipa{ɖæ˩ʂo˩˥ / ɖæ˩ʂo˩-ze˥}}} \textcolor{PineGreen}{\selectlanguage{french}glisser; dévaler une pente en glissant} \zh{往下滑}  
 ¶ \textcolor{darkblue}{\textbf{\ipa{no˧ | ɖæ˩ʂo˩\textasciitilde{}ɖæ˥ʂo˩! |}}} \textcolor{PineGreen}{\selectlanguage{french}Tu es bien malhonnête! (Critique de quelqu'un qui n'est pas franc et direct, qui est faux jeton, qui n'a pas une bonne attitude, donnant une impression huileuse: qui s'esquive et se dérobe, comme un objet glissant qui se dérobe à la prise.)} \zh{你真滑头!}  
\textit{Voir~:} \hyperlink{}{\textcolor{darkblue}{\textbf{\ipa{ʂo˧˥}}} \textsubscript{2}} 
\lhead{\firstmark}
\rhead{\botmark}

\subsection{\hspace{-0.5cm} {\Large \textcolor{darkblue}{\textbf{\ipa{ʂo˧˥}}} \textsubscript{2}}\hspace{0.5cm}[\kern2pt{\textcolor{darkblue}{\textbf{\ipa{ʂo˧˥}}}}\kern2pt]} \hypertarget{s`o\string_M\string_T2}{}
\markboth{\textcolor{darkblue}{\textbf{\ipa{ʂo˧˥}}} \textsubscript{2}}{}
\textcolor{teal}{\mytextsc{adjectif}} \hspace{4pt} Ton~: MH.
\textcolor{PineGreen}{\selectlanguage{french}Lisse, glissant.} \zh{光滑(路……)。}  ¶ \textcolor{darkblue}{\textbf{\ipa{mɤ˩ ʂo˩-ʂo˥ |}}} \textcolor{PineGreen}{\selectlanguage{french}huileux, tout poisseux de graisse} \zh{油腻腻、滑腻}  
 ¶ \textcolor{darkblue}{\textbf{\ipa{ɲi˧to˧ ɖɯ˧-ɭɯ˧ | dze˧-ʂo˧\textasciitilde{}ʂo˥}}} \textcolor{PineGreen}{\selectlanguage{french}toute (sa) bouche est/était pleine de sucre / toute poisseuse à force de sucre} \zh{他嘴巴被糖粘得黏黏的}  
\textit{Voir~:} \hyperlink{}{\textcolor{darkblue}{\textbf{\ipa{ʂo˧˥}}} \textsubscript{1}} 
\lhead{\firstmark}
\rhead{\botmark}

\subsection{\hspace{-0.5cm} {\Large \textcolor{darkblue}{\textbf{\ipa{ʂɻ̍˧˥}}}}\hspace{0.5cm}[\kern2pt{\textcolor{darkblue}{\textbf{\ipa{ʂɻ̍˧˥}}}}\kern2pt]} \hypertarget{s`r£`̍\string_M\string_T1}{}
\markboth{\textcolor{darkblue}{\textbf{\ipa{ʂɻ̍˧˥}}}}{}
\textcolor{teal}{\mytextsc{adjectif}} \hspace{4pt} Ton~: MH.
\textcolor{PineGreen}{\selectlanguage{french}Rempli, plein.} \zh{满。}  ¶ \textcolor{darkblue}{\textbf{\ipa{le˧-ʂɻ̍˧-ze˥}}} \textcolor{PineGreen}{\selectlanguage{french}\mytextsc{accomp} \string_ \mytextsc{pfv}} \zh{满了}  

\lhead{\firstmark}
\rhead{\botmark}

\subsection{\hspace{-0.5cm} {\Large \textcolor{darkblue}{\textbf{\ipa{ʂɯ˧}}}}\hspace{0.5cm}[\kern2pt{\textcolor{darkblue}{\textbf{\ipa{ʂɯ˥}}}}\kern2pt]} \hypertarget{s`M\string_M1}{}
\markboth{\textcolor{darkblue}{\textbf{\ipa{ʂɯ˧}}}}{}
\textcolor{teal}{\mytextsc{nombre}} \hspace{4pt} Ton~: M? H\#? (pas L).
\textcolor{PineGreen}{\selectlanguage{french}7.} \zh{7。} 
\lhead{\firstmark}
\rhead{\botmark}

\subsection{\hspace{-0.5cm} {\Large \textcolor{darkblue}{\textbf{\ipa{ʂɯ˧\textsubscript{a}}}} \textsubscript{1}}\hspace{0.5cm}[\kern2pt{\textcolor{darkblue}{\textbf{\ipa{ʂɯ˥}}}}\kern2pt]} \hypertarget{s`M\string_Ma1}{}
\markboth{\textcolor{darkblue}{\textbf{\ipa{ʂɯ˧\textsubscript{a}}}} \textsubscript{1}}{}
\textcolor{teal}{\mytextsc{verbe}} \hspace{4pt} Ton~: M\textsubscript{a}.
\textcolor{PineGreen}{\selectlanguage{french}Fuir, s'écouler, se répandre, se vider.} \zh{漏。}  ¶ \textcolor{darkblue}{\textbf{\ipa{tʰi˧-ʂɯ˥\textasciitilde{}ʂɯ˩(-ze˩)}}} \textcolor{PineGreen}{\selectlanguage{french}Ca fuit / ça se vide!} \zh{漏了!}  
 ¶ \textcolor{darkblue}{\textbf{\ipa{mɤ˧-ʂɯ˥\textasciitilde{}ʂɯ˩! | mɤ˧-ʑi˧!}}} \textcolor{PineGreen}{\selectlanguage{french}Ca ne s'écoule pas, ça ne fuit pas! (ʑi˧: “couler”)} \zh{没漏,没流出去!}  

\lhead{\firstmark}
\rhead{\botmark}

\subsection{\hspace{-0.5cm} {\Large \textcolor{darkblue}{\textbf{\ipa{ʂɯ˧\textsubscript{a}}}} \textsubscript{2}}\hspace{0.5cm}[\kern2pt{\textcolor{darkblue}{\textbf{\ipa{ʂɯ˥}}}}\kern2pt]} \hypertarget{s`M\string_Ma2}{}
\markboth{\textcolor{darkblue}{\textbf{\ipa{ʂɯ˧\textsubscript{a}}}} \textsubscript{2}}{}
\textcolor{teal}{\mytextsc{verbe}} \hspace{4pt} Ton~: M\textsubscript{a}.
\textcolor{PineGreen}{\selectlanguage{french}Mourir, décéder.} \zh{死。}  ¶ \textcolor{darkblue}{\textbf{\ipa{le˧-ʂɯ˧-ho˩-ze˩}}} \textcolor{PineGreen}{\selectlanguage{french}ça va mourir! (au sujet d'une plante ou d'un animal malade)} \zh{快要死了!(病了的植物、动物)}  
 ¶ \textcolor{darkblue}{\textbf{\ipa{mɤ˧-ʂɯ˧-sɯ˩!}}} \textcolor{PineGreen}{\selectlanguage{french}(Il/elle/ce) n'est pas encore mort!} \zh{还没死!}  
 ¶ \textcolor{darkblue}{\textbf{\ipa{no˧ | le˧-ʂɯ˧-bi˧-tsæ˧-ɲi˧-ze˩!}}} \textcolor{PineGreen}{\selectlanguage{french}Crève donc! / Crève, charogne! (imprécation/malédiction, qu'on lance sous le coup de la colère)} \zh{你去死吧!}  

\lhead{\firstmark}
\rhead{\botmark}

\subsection{\hspace{-0.5cm} {\Large \textcolor{darkblue}{\textbf{\ipa{ʂɯ˧dʑi˧}}}}\hspace{0.5cm}[\kern2pt{\textcolor{darkblue}{\textbf{\ipa{ʂɯ˩dʑi˩˥}}}}\kern2pt]} \hypertarget{s`M\string_Mdz£i\string_M1}{}
\markboth{\textcolor{darkblue}{\textbf{\ipa{ʂɯ˧dʑi˧}}}}{}
\textcolor{teal}{\mytextsc{nom}} \hspace{4pt} Ton~: M.
\textcolor{PineGreen}{\selectlanguage{french}Linceul, suaire, vêtement mortuaire (de: “mourir” et “habit”).} \zh{寿衣。}  ¶ \textcolor{darkblue}{\textbf{\ipa{ʂɯ˧dʑi˧ ʐv̩˥}}} \textcolor{PineGreen}{\selectlanguage{french}coudre les vêtements mortuaires, coudre le linceul} \zh{缝寿衣}  

\lhead{\firstmark}
\rhead{\botmark}

\subsection{\hspace{-0.5cm} {\Large \textcolor{darkblue}{\textbf{\ipa{ʂɯ˧-ɬi˧mi˧}}}}\hspace{0.5cm}[\kern2pt{\textcolor{darkblue}{\textbf{\ipa{xxxx non-correspondance entre le nombre de morphèmes et le nombre de tons de morphèmes}}}}\kern2pt]} \hypertarget{s`M\string_M-Ki\string_Mmi\string_M1}{}
\markboth{\textcolor{darkblue}{\textbf{\ipa{ʂɯ˧-ɬi˧mi˧}}}}{}
\textcolor{teal}{\mytextsc{nom}} \hspace{4pt} Ton~: M.
\textcolor{PineGreen}{\selectlanguage{french}7e mois.} \zh{七月。} 
\lhead{\firstmark}
\rhead{\botmark}

\subsection{\hspace{-0.5cm} {\Large \textcolor{darkblue}{\textbf{\ipa{ʂɯ˧ɲi˥}}}}\hspace{0.5cm}[\kern2pt{\textcolor{darkblue}{\textbf{\ipa{ʂɯ˧ɲi˩}}}}\kern2pt]} \hypertarget{s`M\string_MJi\string_T1}{}
\markboth{\textcolor{darkblue}{\textbf{\ipa{ʂɯ˧ɲi˥}}}}{}
\textcolor{teal}{\mytextsc{adverbe}} \hspace{4pt} Ton~: H\#.
\textcolor{PineGreen}{\selectlanguage{french}Avant-hier.} \zh{前天。}  ¶ \textcolor{darkblue}{\textbf{\ipa{ʂɯ˧ɲi˥ | -ɖɯ˧ɲi˥}}} \textcolor{PineGreen}{\selectlanguage{french}avant-hier} \zh{前天}  

\lhead{\firstmark}
\rhead{\botmark}

\subsection{\hspace{-0.5cm} {\Large \textcolor{darkblue}{\textbf{\ipa{ʂɯ˧ʂɯ˧-dzi˩}}}}\hspace{0.5cm}[\kern2pt{\textcolor{darkblue}{\textbf{\ipa{ʂɯ˧ʂɯ˧dzi˧}}}}\kern2pt]} \hypertarget{s`M\string_Ms`M\string_M-dzi\string_B1}{}
\markboth{\textcolor{darkblue}{\textbf{\ipa{ʂɯ˧ʂɯ˧-dzi˩}}}}{}
\textcolor{teal}{\mytextsc{nom}} \hspace{4pt} Ton~: -L.
\textcolor{PineGreen}{\selectlanguage{french}Yyyy.} \zh{三颗针。}  \zh{量词}: \textcolor{darkblue}{\textbf{\ipa{dzi˩}}}  \mytextsc{clf}~: \textcolor{darkblue}{\textbf{\ipa{dzi˩}}} 
\lhead{\firstmark}
\rhead{\botmark}

\subsection{\hspace{-0.5cm} {\Large \textcolor{darkblue}{\textbf{\ipa{ʂɯ˧tɤ˧ɻ\#˥}}}}\hspace{0.5cm}[\kern2pt{\textcolor{darkblue}{\textbf{\ipa{ʂɯ˧tɤ˧ɻ˧}}}}\kern2pt]} \hypertarget{s`M\string_Mt7\string_Mr£`\#\string_T1}{}
\markboth{\textcolor{darkblue}{\textbf{\ipa{ʂɯ˧tɤ˧ɻ\#˥}}}}{}
\textcolor{teal}{\mytextsc{adjectif}} \hspace{4pt} Ton~: \#H.
\textcolor{PineGreen}{\selectlanguage{french}Lisse; par exemple: un pilier en bois, qui devient bien lisse par le travail du menuisier.} \zh{平滑。}  ¶ \textcolor{darkblue}{\textbf{\ipa{ʂɯ˧tɤ˧ɻ̍˧-zo˥}}} \textcolor{PineGreen}{\selectlanguage{french}bien lisse} \zh{很平滑}  
 ¶ \textcolor{darkblue}{\textbf{\ipa{ʂɯ˧tɤ˧ɻ̍˧ gv̩˧-ze˩}}} \textcolor{PineGreen}{\selectlanguage{french}on l'a bien lissé, on l'a rendu bien lisse} \zh{弄得平滑了}  

\lhead{\firstmark}
\rhead{\botmark}

\subsection{\hspace{-0.5cm} {\Large \textcolor{darkblue}{\textbf{\ipa{ʂɯ˧tsʰi˩}}}}\hspace{0.5cm}[\kern2pt{\textcolor{darkblue}{\textbf{\ipa{ʂɯ˧tsʰi˩}}}}\kern2pt]} \hypertarget{s`M\string_Mts\string_hi\string_B1}{}
\markboth{\textcolor{darkblue}{\textbf{\ipa{ʂɯ˧tsʰi˩}}}}{}
\textcolor{teal}{\mytextsc{nombre}} \hspace{4pt} Ton~: L\#.
\textcolor{PineGreen}{\selectlanguage{french}70.} \zh{70。} 
\lhead{\firstmark}
\rhead{\botmark}

\subsection{\hspace{-0.5cm} {\Large \textcolor{darkblue}{\textbf{\ipa{ʂɯ˩\textsubscript{b}}}}}\hspace{0.5cm}[\kern2pt{\textcolor{darkblue}{\textbf{\ipa{ʂɯ˥}}}}\kern2pt]} \hypertarget{s`M\string_Bb1}{}
\markboth{\textcolor{darkblue}{\textbf{\ipa{ʂɯ˩\textsubscript{b}}}}}{}
\textcolor{teal}{\mytextsc{classificateur}} \hspace{4pt} Ton~: L\textsubscript{b}.
\textcolor{PineGreen}{\selectlanguage{french}Classificateur des fois (répétitions d'une action).} \zh{量词:次数。} 
\lhead{\firstmark}
\rhead{\botmark}

\subsection{\hspace{-0.5cm} {\Large \textcolor{darkblue}{\textbf{\ipa{ʂɯ˩ʝi\#˥}}}}\hspace{0.5cm}[\kern2pt{\textcolor{darkblue}{\textbf{\ipa{ʂɯ˧ʝi˧}}}}\kern2pt]} \hypertarget{s`M\string_Bj££i\#\string_T1}{}
\markboth{\textcolor{darkblue}{\textbf{\ipa{ʂɯ˩ʝi\#˥}}}}{}
\textcolor{teal}{\mytextsc{adverbe}} \hspace{4pt} Ton~: LM+\#H.
\textcolor{PineGreen}{\selectlanguage{french}Il y a deux ans.} \zh{前年。}  ¶ \textcolor{darkblue}{\textbf{\ipa{ʂɯ˩ʝi˥ | ɖɯ˧-kʰv̩˧˥}}} \textcolor{PineGreen}{\selectlanguage{french}il y a deux ans, l'année il y a deux ans} \zh{前年}  

\lhead{\firstmark}
\rhead{\botmark}

\subsection{\hspace{-0.5cm} {\Large \textcolor{darkblue}{\textbf{\ipa{ʂɯ˩kwæ˩ɻæ˥}}}}\hspace{0.5cm}[\kern2pt{\textcolor{darkblue}{\textbf{\ipa{ʂɯ˧kwæ˧ɻæ˩}}}}\kern2pt]} \hypertarget{s`M\string_Bkw\{\string_Br£`\{\string_T1}{}
\markboth{\textcolor{darkblue}{\textbf{\ipa{ʂɯ˩kwæ˩ɻæ˥}}}}{}
\textcolor{teal}{\mytextsc{adjectif}} \hspace{4pt} Ton~: L+H\#.
\textcolor{PineGreen}{\selectlanguage{french}Jaune.} \zh{黄。}  ¶ \textcolor{darkblue}{\textbf{\ipa{ʂɯ˩kwæ˩ɻæ˥-hĩ˩ gv̩˩-ze˩}}} \textcolor{PineGreen}{\selectlanguage{french}[le livre] a jauni!} \zh{[书]变黄了!}  
 ¶ \textcolor{darkblue}{\textbf{\ipa{[F5] ʂɯ˩kwæ˩˥ | ʂɯ˩kwæ˩˥ | gv̩˩}}} \textcolor{PineGreen}{\selectlanguage{french}tout jaune} \zh{深黄}  

\lhead{\firstmark}
\rhead{\botmark}

\subsection{\hspace{-0.5cm} {\Large \textcolor{darkblue}{\textbf{\ipa{ʂɯ˩tsɯ˧}}}}\hspace{0.5cm}[\kern2pt{\textcolor{darkblue}{\textbf{\ipa{ʂɯ˩tsɯ˥}}}}\kern2pt]} \hypertarget{s`M\string_BtsM\string_M1}{}
\markboth{\textcolor{darkblue}{\textbf{\ipa{ʂɯ˩tsɯ˧}}}}{}
\textcolor{teal}{\mytextsc{nom}} \hspace{4pt} Ton~: LM.
\textcolor{PineGreen}{\selectlanguage{french}Kaki.} \zh{柿子(汉语借词)。}  Emprunt~: chinois  \zh{柿子}
 ¶ \textcolor{darkblue}{\textbf{\ipa{ʂɯ˩tsɯ˧ | ɖɯ˧-so˩-ɭɯ˩ hwæ˩-bi˩!}}} \textcolor{PineGreen}{\selectlanguage{french}(Je) vais acheter quelques kakis!} \zh{买一些柿子吧!}  

\lhead{\firstmark}
\rhead{\botmark}

\subsection{\hspace{-0.5cm} {\Large \textcolor{darkblue}{\textbf{\ipa{ʂɯ˩tsɯ˧}}}}\hspace{0.5cm}[\kern2pt{\textcolor{darkblue}{\textbf{\ipa{ʂɯ˩tsɯ˥}}}}\kern2pt]} \hypertarget{s`M\string_BtsM\string_M1}{}
\markboth{\textcolor{darkblue}{\textbf{\ipa{ʂɯ˩tsɯ˧}}}}{}
\textcolor{teal}{\mytextsc{nom}} \hspace{4pt} Ton~: LM.
\textcolor{PineGreen}{\selectlanguage{french}Pistolet.} \zh{手枪。}  ¶ \textcolor{darkblue}{\textbf{\ipa{ʂɯ˩tsɯ˧ | ɖɯ˧-nɑ˧ | tʰi˧-pɤ˥\textasciitilde{}pɤ˩}}} \textcolor{PineGreen}{\selectlanguage{french}porter un pistolet} \zh{带手枪}  

\lhead{\firstmark}
\rhead{\botmark}

\subsection{\hspace{-0.5cm} {\Large \textcolor{darkblue}{\textbf{\ipa{ʂɯ˧˥}}} \textsubscript{1}}\hspace{0.5cm}[\kern2pt{\textcolor{darkblue}{\textbf{\ipa{ʂɯ˧˥}}}}\kern2pt]} \hypertarget{s`M\string_M\string_T1}{}
\markboth{\textcolor{darkblue}{\textbf{\ipa{ʂɯ˧˥}}} \textsubscript{1}}{}
\textcolor{teal}{\mytextsc{verbe}} \hspace{4pt} Ton~: MH.
\textcolor{PineGreen}{\selectlanguage{french}Éplucher, peler, décortiquer (avec un instrument).} \zh{削(用刀)。}  ¶ \textcolor{darkblue}{\textbf{\ipa{ɣɯ˩ ʂɯ˧˥ / ɣɯ˩ʂɯ˧ ze˥}}} \textcolor{PineGreen}{\selectlanguage{french}éplucher la peau} \zh{削皮}  
 ¶ \textcolor{darkblue}{\textbf{\ipa{ɣɯ˩kɯ˧ ʂɯ˥}}} \textcolor{PineGreen}{\selectlanguage{french}éplucher la peau} \zh{削皮}  
 ¶ \textcolor{darkblue}{\textbf{\ipa{jɤ˩jo˧ ɣɯ˥ʂɯ˩}}} \textcolor{PineGreen}{\selectlanguage{french}peler des patates} \zh{削洋芋皮}  
 ¶ \textcolor{darkblue}{\textbf{\ipa{[F5] tso˧tso˧ ɣɯ˥ʂɯ˩}}} \textcolor{PineGreen}{\selectlanguage{french}éplucher des choses} \zh{削东西}  

\lhead{\firstmark}
\rhead{\botmark}

\subsection{\hspace{-0.5cm} {\Large \textcolor{darkblue}{\textbf{\ipa{ʂɯ˧˥}}} \textsubscript{2}}\hspace{0.5cm}[\kern2pt{\textcolor{darkblue}{\textbf{\ipa{ʂɯ˧˥}}}}\kern2pt]} \hypertarget{s`M\string_M\string_T2}{}
\markboth{\textcolor{darkblue}{\textbf{\ipa{ʂɯ˧˥}}} \textsubscript{2}}{}
\textcolor{teal}{\mytextsc{adjectif}} \hspace{4pt} Ton~: MH.
\textcolor{PineGreen}{\selectlanguage{french}Nouveau, neuf, frais.} \zh{新。}  ¶ \textcolor{darkblue}{\textbf{\ipa{ʂɯ˧-hĩ˧ ɲi˥!}}} \textcolor{PineGreen}{\selectlanguage{french}c'est neuf!} \zh{是新的!}  
 ¶ \textcolor{darkblue}{\textbf{\ipa{ʂe˧ ʂɯ˩}}} \textcolor{PineGreen}{\selectlanguage{french}de la viande fraîche} \zh{新鲜的肉}  

\lhead{\firstmark}
\rhead{\botmark}

\subsection{\hspace{-0.5cm} {\Large \textcolor{darkblue}{\textbf{\ipa{ʂv̩˧˥}}}}\hspace{0.5cm}[\kern2pt{\textcolor{darkblue}{\textbf{\ipa{ʂv̩˧˥}}}}\kern2pt]} \hypertarget{s`v\string_=\string_M\string_T1}{}
\markboth{\textcolor{darkblue}{\textbf{\ipa{ʂv̩˧˥}}}}{}
\textcolor{teal}{\mytextsc{verbe}} \hspace{4pt} Ton~: MH.
\textcolor{PineGreen}{\selectlanguage{french}Tordre, essorer (vêtement).} \zh{拧(拧毛巾)。}  ¶ \textcolor{darkblue}{\textbf{\ipa{le˧-ʂv̩˧-ze˥}}} \textcolor{PineGreen}{\selectlanguage{french}\mytextsc{accomp} \string_ \mytextsc{pfv}} \zh{拧了}  
 ¶ \textcolor{darkblue}{\textbf{\ipa{dʑi˧hṽ˧ ʂv̩˩}}} \textcolor{PineGreen}{\selectlanguage{french}essorer des vêtements} \zh{拧衣服}  

\lhead{\firstmark}
\rhead{\botmark}

\subsection{\hspace{-0.5cm} {\Large \textcolor{darkblue}{\textbf{\ipa{ʂv̩˥}}}}\hspace{0.5cm}[\kern2pt{\textcolor{darkblue}{\textbf{\ipa{ʂv̩˥}}}}\kern2pt]} \hypertarget{s`v\string_=\string_T1}{}
\markboth{\textcolor{darkblue}{\textbf{\ipa{ʂv̩˥}}}}{}
\textcolor{teal}{\mytextsc{nom}} \hspace{4pt} Ton~: H.
\textcolor{PineGreen}{\selectlanguage{french}Dé.} \zh{骰子。}  ¶ \textcolor{darkblue}{\textbf{\ipa{ʂv̩˧ | ʐv̩˩-ɭɯ˩˥}}} \textcolor{PineGreen}{\selectlanguage{french}quatre dés (les dés allaient par quatre)} \zh{四个骰子}  
 \zh{量词}: \textcolor{darkblue}{\textbf{\ipa{ɭɯ˧}}}  \mytextsc{clf}~: \textcolor{darkblue}{\textbf{\ipa{ɭɯ˧}}} 
\lhead{\firstmark}
\rhead{\botmark}

\subsection{\hspace{-0.5cm} {\Large \textcolor{darkblue}{\textbf{\ipa{ʂv̩˧\textsubscript{b}}}}}\hspace{0.5cm}[\kern2pt{\textcolor{darkblue}{\textbf{\ipa{ʂv̩˥}}}}\kern2pt]} \hypertarget{s`v\string_=\string_Mb1}{}
\markboth{\textcolor{darkblue}{\textbf{\ipa{ʂv̩˧\textsubscript{b}}}}}{}
\textcolor{teal}{\mytextsc{verbe}} \hspace{4pt} Ton~: M\textsubscript{b}.
\ding{202} \textcolor{PineGreen}{\selectlanguage{french}S'occuper de, surveiller.} \zh{带(孩子……)。}  ¶ \textcolor{darkblue}{\textbf{\ipa{zo˧mv̩˥ | ɖɯ˧-ɭɯ˧ ʂv̩˧}}} \textcolor{PineGreen}{\selectlanguage{french}garder un enfant, s'occuper d'un enfant} \zh{带个孩子}  
 ¶ \textcolor{darkblue}{\textbf{\ipa{zo˧mv̩˥ ʂv̩˩}}} \textcolor{PineGreen}{\selectlanguage{french}garder un enfant, s'occuper d'un enfant} \zh{带孩子}  
 ¶ \textcolor{darkblue}{\textbf{\ipa{le˧-ʂv̩˧ tʰi˧-kʰɯ˧˥ | tʰæ˧ɻ̍˩ so˩}}} \textcolor{PineGreen}{\selectlanguage{french}obliger à étudier (une mère oblige son enfant à faire ses devoirs)} \zh{让他学习、要求他学习(家长管孩子,让他学习)}  
\ding{203} \textcolor{PineGreen}{\selectlanguage{french}Mener, guider.} \zh{带(路)。}  ¶ \textcolor{darkblue}{\textbf{\ipa{ʈæ˧ʂɯ˧ | ɖʐv̩˧ ʂv̩˧-po˧-bi˧-ho˥!}}} \textcolor{PineGreen}{\selectlanguage{french}Dashi va accompagner ses amis [=les emmener en excursion à Yongning]!} \zh{达石要管朋友(带他们去永宁旅游)}  
 ¶ \textcolor{darkblue}{\textbf{\ipa{ʈæ˧ʂɯ˧ | ɖʐv̩˧ ʂv̩˧-bi˧-ho˩!}}} \textcolor{PineGreen}{\selectlanguage{french}Dashi va accompagner ses amis [=les emmener en excursion à Yongning]!} \zh{达石要管朋友(带他们去永宁旅游)}  

\lhead{\firstmark}
\rhead{\botmark}

\subsection{\hspace{-0.5cm} {\Large \textcolor{darkblue}{\textbf{\ipa{ʂv̩˧ɖv̩˧}}}}\hspace{0.5cm}[\kern2pt{\textcolor{darkblue}{\textbf{\ipa{ʂv̩˧ɖv̩˧}}}}\kern2pt]} \hypertarget{s`v\string_=\string_Md`v\string_=\string_M1}{}
\markboth{\textcolor{darkblue}{\textbf{\ipa{ʂv̩˧ɖv̩˧}}}}{}
\textcolor{teal}{\mytextsc{verbe}} \hspace{4pt} Ton~: M.
\ding{202} \textcolor{PineGreen}{\selectlanguage{french}Penser, réfléchir.} \zh{想。}  ¶ \textcolor{darkblue}{\textbf{\ipa{ə˧tso˧ ʂv̩˧ɖv̩˧?}}} \textcolor{PineGreen}{\selectlanguage{french}A quoi [tu] penses?} \zh{在想什么?}  
 ¶ \textcolor{darkblue}{\textbf{\ipa{njɤ˧ | ɖɯ˧ bæ˧ ʂv̩˧dv̩˧}}} \textcolor{PineGreen}{\selectlanguage{french}je pense à quelque chose} \zh{我在想一件事情。}  
 ¶ \textcolor{darkblue}{\textbf{\ipa{ʂv̩˧ɖv̩˧ tʰv̩˧}}} \textcolor{PineGreen}{\selectlanguage{french}comprendre; se souvenir} \zh{明白,想起}  
 ¶ \textcolor{darkblue}{\textbf{\ipa{njɤ˧ | ʂv̩˧ɖv̩˧ tʰv̩˧}}} \textcolor{PineGreen}{\selectlanguage{french}Je comprends.} \zh{我明白。}  
 ¶ \textcolor{darkblue}{\textbf{\ipa{ʈʂʰɯ˧ | le˧-ʂv̩˧ɖv̩˧-le˧-tʰv̩˧-ze˧}}} \textcolor{PineGreen}{\selectlanguage{french}Il a compris.} \zh{他明白了。}  
\ding{203} \textcolor{PineGreen}{\selectlanguage{french}Se souvenir.} \zh{想起、回忆。}  ¶ \textcolor{darkblue}{\textbf{\ipa{ʂv̩˧ɖv̩˧ tʰv̩˧}}} \textcolor{PineGreen}{\selectlanguage{french}se souvenir} \zh{想起}  
 ¶ \textcolor{darkblue}{\textbf{\ipa{njɤ˧ | ʂv̩˧ɖv̩˧ tʰv̩˧}}} \textcolor{PineGreen}{\selectlanguage{french}je me souviens} \zh{我想起}  
 ¶ \textcolor{darkblue}{\textbf{\ipa{ʈʂʰɯ˧ | le˧-ʂv̩˧ɖv̩˧-le˧-tʰv̩˧-ze˧}}} \textcolor{PineGreen}{\selectlanguage{french}il se rappelle, ça lui revient} \zh{他想起来了}  
\ding{204} \textcolor{PineGreen}{\selectlanguage{french}Avoir la nostalgie de.} \zh{想念、感到悲哀。}  ¶ \textcolor{darkblue}{\textbf{\ipa{ʂv̩˧ɖv̩˧ tʰv̩˧ | ʐwæ˩˥}}} \textcolor{PineGreen}{\selectlanguage{french}être plongé dans le chagrin} \zh{特别想念}  
 ¶ \textcolor{darkblue}{\textbf{\ipa{njɤ˧ | no˩ ʂv̩˩ɖv̩˩˥}}} \textcolor{PineGreen}{\selectlanguage{french}tu me manques!} \zh{我想你!}  
 ¶ \textcolor{darkblue}{\textbf{\ipa{ʂv̩˧ɖv̩˧ qʰwɤ˧ tʰv̩˧˥}}} \textcolor{PineGreen}{\selectlanguage{french}être nostalgique, avoir une crise de nostalgie} \zh{想念}  
 ¶ \textcolor{darkblue}{\textbf{\ipa{[F5] ʂv̩˧ɖv̩˧ mɤ˧-zo˧}}} \textcolor{PineGreen}{\selectlanguage{french}on n'a pas à se faire de souci, il n'y a pas lieu de se morfondre} \zh{不用发愁}  

\lhead{\firstmark}
\rhead{\botmark}

\subsection{\hspace{-0.5cm} {\Large \textcolor{darkblue}{\textbf{\ipa{ʂv̩˩gv̩˩}}}}\hspace{0.5cm}[\kern2pt{\textcolor{darkblue}{\textbf{\ipa{ʂv̩˧gv̩˧}}}}\kern2pt]} \hypertarget{s`v\string_=\string_Bgv\string_=\string_B1}{}
\markboth{\textcolor{darkblue}{\textbf{\ipa{ʂv̩˩gv̩˩}}}}{}
\textcolor{teal}{\mytextsc{nom}} \hspace{4pt} Ton~: L.
\textcolor{PineGreen}{\selectlanguage{french}Faucille.} \zh{镰刀。}  \zh{量词}: \textcolor{darkblue}{\textbf{\ipa{nɑ˧}}}  \mytextsc{clf}~: \textcolor{darkblue}{\textbf{\ipa{nɑ˧}}} 
\lhead{\firstmark}
\rhead{\botmark}

\subsection{\hspace{-0.5cm} {\Large \textcolor{darkblue}{\textbf{\ipa{ʂv̩˧kʰɯ˩}}}}\hspace{0.5cm}[\kern2pt{\textcolor{darkblue}{\textbf{\ipa{ʂv̩˩kʰɯ˩˥}}}}\kern2pt]} \hypertarget{s`v\string_=\string_Mk\string_hM\string_B1}{}
\markboth{\textcolor{darkblue}{\textbf{\ipa{ʂv̩˧kʰɯ˩}}}}{}
\textcolor{teal}{\mytextsc{verbe}} \hspace{4pt} Ton~: .
\textcolor{PineGreen}{\selectlanguage{french}Parier.} \zh{赌博。}  ¶ \textcolor{darkblue}{\textbf{\ipa{ʂv̩˧kʰɯ˩ | -jɤ˩po˧}}} \textcolor{PineGreen}{\selectlanguage{french}même sens} \zh{同上}  

\lhead{\firstmark}
\rhead{\botmark}

\subsection{\hspace{-0.5cm} {\Large \textcolor{darkblue}{\textbf{\ipa{ʂv̩˩njɤ˥}}}}\hspace{0.5cm}[\kern2pt{\textcolor{darkblue}{\textbf{\ipa{ʂv̩˩njɤ˥}}}}\kern2pt]} \hypertarget{s`v\string_=\string_Bnj7\string_T1}{}
\markboth{\textcolor{darkblue}{\textbf{\ipa{ʂv̩˩njɤ˥}}}}{}
\textcolor{teal}{\mytextsc{nom}} \hspace{4pt} Ton~: LH.
\textcolor{PineGreen}{\selectlanguage{french}Nœud (sur un arbre).} \zh{树瘤。}  ¶ \textcolor{darkblue}{\textbf{\ipa{ʂv̩˩njɤ˥ ɲi˩}}} \textcolor{PineGreen}{\selectlanguage{french}\mytextsc{cop}} \zh{是树瘤。}  
 \zh{量词}: \textcolor{darkblue}{\textbf{\ipa{ɭɯ˧}}}  \mytextsc{clf}~: \textcolor{darkblue}{\textbf{\ipa{ɭɯ˧}}} 
\lhead{\firstmark}
\rhead{\botmark}

\subsection{\hspace{-0.5cm} {\Large \textcolor{darkblue}{\textbf{\ipa{ʂv̩˧ʂv̩˧˥}}}}\hspace{0.5cm}[\kern2pt{\textcolor{darkblue}{\textbf{\ipa{ʂv̩˧ʂv̩˧˥}}}}\kern2pt]} \hypertarget{s`v\string_=\string_Ms`v\string_=\string_M\string_T1}{}
\markboth{\textcolor{darkblue}{\textbf{\ipa{ʂv̩˧ʂv̩˧˥}}}}{}
\textcolor{teal}{\mytextsc{nom}} \hspace{4pt} Ton~: MH\#.
\textcolor{PineGreen}{\selectlanguage{french}Papier.} \zh{纸。}  ¶ \textcolor{darkblue}{\textbf{\ipa{ʂv̩˧ʂv̩˧˥ | ɖɯ˧-pʰæ˧˥}}} \textcolor{PineGreen}{\selectlanguage{french}une feuille de papier} \zh{一张纸}  
 ¶ \textcolor{darkblue}{\textbf{\ipa{[F5] ʂv̩˧ʂv̩˧ ɖɯ˧ pʰæ˧˥}}} \textcolor{PineGreen}{\selectlanguage{french}une feuille de papier} \zh{一张纸}  
 \zh{量词}: \textcolor{darkblue}{\textbf{\ipa{pʰæ˧˥}}}  \mytextsc{clf}~: \textcolor{darkblue}{\textbf{\ipa{pʰæ˧˥}}} \textcolor{PineGreen}{\selectlanguage{french}objets plats}
\lhead{\firstmark}
\rhead{\botmark}

\subsection{\hspace{-0.5cm} {\Large \textcolor{darkblue}{\textbf{\ipa{ʂwæ˧}}}}\hspace{0.5cm}[\kern2pt{\textcolor{darkblue}{\textbf{\ipa{ʂwæ˥}}}}\kern2pt]} \hypertarget{s`w\{\string_M1}{}
\markboth{\textcolor{darkblue}{\textbf{\ipa{ʂwæ˧}}}}{}
\textcolor{teal}{\mytextsc{nom}} \hspace{4pt} Ton~: M.
\textcolor{PineGreen}{\selectlanguage{french}Loutre.} \zh{水獭。} Dialecte chinois local~:\zh{水潭猫。} ¶ \textcolor{darkblue}{\textbf{\ipa{ʂwæ˧-ɣɯ˩}}} \textcolor{PineGreen}{\selectlanguage{french}peau de loutre} \zh{水獭皮}  

\lhead{\firstmark}
\rhead{\botmark}

\subsection{\hspace{-0.5cm} {\Large \textcolor{darkblue}{\textbf{\ipa{ʂwæ˧}}}}\hspace{0.5cm}[\kern2pt{\textcolor{darkblue}{\textbf{\ipa{ʂwæ˥}}}}\kern2pt]} \hypertarget{s`w\{\string_M1}{}
\markboth{\textcolor{darkblue}{\textbf{\ipa{ʂwæ˧}}}}{}
\textcolor{teal}{\mytextsc{adjectif}} \hspace{4pt} Ton~: M.
\textcolor{PineGreen}{\selectlanguage{french}Haut, de haute taille, grand.} \zh{高。}  ¶ \textcolor{darkblue}{\textbf{\ipa{qʰɑ˧-ʂwæ˧-gv̩˧}}} \textcolor{PineGreen}{\selectlanguage{french}très grand} \zh{非常高}  
 ¶ \textcolor{darkblue}{\textbf{\ipa{ʈʂʰɯ˧ | ə˧pɤ˧ | -ʂwæ˩-gv̩˩˥!}}} \textcolor{PineGreen}{\selectlanguage{french}Elle/il est très grand(e)!} \zh{他非常高!}  
 ¶ \textcolor{darkblue}{\textbf{\ipa{gv̩˧mi˧ ʂwæ˧}}} \textcolor{PineGreen}{\selectlanguage{french}de grande taille; littéralement '(qui a un) grand corps'} \zh{高、身材高}  

\lhead{\firstmark}
\rhead{\botmark}

\subsection{\hspace{-0.5cm} {\Large \textcolor{darkblue}{\textbf{\ipa{ʂwæ˧\textsubscript{a}}}}}\hspace{0.5cm}[\kern2pt{\textcolor{darkblue}{\textbf{\ipa{ʂwæ˥}}}}\kern2pt]} \hypertarget{s`w\{\string_Ma1}{}
\markboth{\textcolor{darkblue}{\textbf{\ipa{ʂwæ˧\textsubscript{a}}}}}{}
\textcolor{teal}{\mytextsc{verbe}} \hspace{4pt} Ton~: M\textsubscript{a}.
\textcolor{PineGreen}{\selectlanguage{french}Remuer (monosyllabe).} \zh{搅拌。}  ¶ \textcolor{darkblue}{\textbf{\ipa{le˧-ʂwæ˧}}} \textcolor{PineGreen}{\selectlanguage{french}\mytextsc{accomp}} \zh{\mytextsc{accomp}}  
 ¶ \textcolor{darkblue}{\textbf{\ipa{mɤ˧-ʂwæ˧}}} \textcolor{PineGreen}{\selectlanguage{french}\mytextsc{neg}} \zh{不搅拌}  
 ¶ \textcolor{darkblue}{\textbf{\ipa{tso˧\textasciitilde{}tso˧ ʂwæ˩}}} \textcolor{PineGreen}{\selectlanguage{french}remuer des choses, touiller des choses} \zh{搅拌东西}  

\lhead{\firstmark}
\rhead{\botmark}

\subsection{\hspace{-0.5cm} {\Large \textcolor{darkblue}{\textbf{\ipa{ʂwæ˧bæ˩}}}}\hspace{0.5cm}[\kern2pt{\textcolor{darkblue}{\textbf{\ipa{ʂwæ˧bæ˩}}}}\kern2pt]} \hypertarget{s`w\{\string_Mb\{\string_B1}{}
\markboth{\textcolor{darkblue}{\textbf{\ipa{ʂwæ˧bæ˩}}}}{}
\textcolor{teal}{\mytextsc{nom}} \hspace{4pt} Ton~: L\#.
\textcolor{PineGreen}{\selectlanguage{french}Camélia.} \zh{映山红。} Dialecte chinois local~:\zh{山茶花。} ¶ \textcolor{darkblue}{\textbf{\ipa{ʂwæ˧bæ˩ bæ˩ |}}} \textcolor{PineGreen}{\selectlanguage{french}Les camélias sont en fleurs.} \zh{山茶花开了。}  
 ¶ \textcolor{darkblue}{\textbf{\ipa{so˧-ɬi˧mi˧, | ʂwæ˧bæ˩ bæ˩! |}}} \textcolor{PineGreen}{\selectlanguage{french}Les camélias fleurissent au troisième mois!} \zh{山茶花是在三月份开花的!}  
 ¶ \textcolor{darkblue}{\textbf{\ipa{ʂwæ˧bæ˩-si˩}}} \textcolor{PineGreen}{\selectlanguage{french}camélia (arbre); littéralement “arbre des fleurs de camélia”} \zh{山茶树}  

\lhead{\firstmark}
\rhead{\botmark}

\subsection{\hspace{-0.5cm} {\Large \textcolor{darkblue}{\textbf{\ipa{ʂwæ˧gv̩\#˥}}}}\hspace{0.5cm}[\kern2pt{\textcolor{darkblue}{\textbf{\ipa{ʂwæ˧gv̩˧}}}}\kern2pt]} \hypertarget{s`w\{\string_Mgv\string_=\#\string_T1}{}
\markboth{\textcolor{darkblue}{\textbf{\ipa{ʂwæ˧gv̩\#˥}}}}{}
\textcolor{teal}{\mytextsc{nom}} \hspace{4pt} Ton~: \#H.
\textcolor{PineGreen}{\selectlanguage{french}Une montagne au nord-ouest de Yongning.} \zh{加泽大山(位于永宁西北的一座山)。}  ¶ \textcolor{darkblue}{\textbf{\ipa{kɤ˧mv̩˧˥, | æ˧ʂæ˧, | ŋwɤ˧hɑ̃˩, | ʂwæ˧gv̩\#˥, | nɑ˩tsʰi˩˥ | -tɕʰɤ˧pɤ˧mi\#˥, | qv̩˧ɻ̍˧-ʈʂʰɑ˧nɑ˥ |}}} \textcolor{PineGreen}{\selectlanguage{french}Les six montagnes de Yongning qui portent un nom. Les autres sommets du voisinage n'ont pas une valeur symbolique comparable, et ne portent pas de nom communément utilisé.} \zh{永宁地区有固定名字的六座山。其它的山,因为没有重要的象征意义,因此没有取名。}  

\lhead{\firstmark}
\rhead{\botmark}

\subsection{\hspace{-0.5cm} {\Large \textcolor{darkblue}{\textbf{\ipa{ʂwæ˧si\#˥}}}}\hspace{0.5cm}[\kern2pt{\textcolor{darkblue}{\textbf{\ipa{ʂwæ˧si˧}}}}\kern2pt]} \hypertarget{s`w\{\string_Msi\#\string_T1}{}
\markboth{\textcolor{darkblue}{\textbf{\ipa{ʂwæ˧si\#˥}}}}{}
\textcolor{teal}{\mytextsc{nom}} \hspace{4pt} Ton~: \#H.
\textcolor{PineGreen}{\selectlanguage{french}Camélia (arbre).} \zh{山茶树。} 
\lhead{\firstmark}
\rhead{\botmark}

\subsection{\hspace{-0.5cm} {\Large \textcolor{darkblue}{\textbf{\ipa{ʂwæ˧tsɯ˥}}}}\hspace{0.5cm}[\kern2pt{\textcolor{darkblue}{\textbf{\ipa{ʂwæ˧tsɯ˥}}}}\kern2pt]} \hypertarget{s`w\{\string_MtsM\string_T1}{}
\markboth{\textcolor{darkblue}{\textbf{\ipa{ʂwæ˧tsɯ˥}}}}{}
\textcolor{teal}{\mytextsc{nom}} \hspace{4pt} Ton~: H\#.
\textcolor{PineGreen}{\selectlanguage{french}Brosse.} \zh{刷子。}  Emprunt~: chinois  \zh{刷子}
 \zh{量词}: \textcolor{darkblue}{\textbf{\ipa{nɑ˧}}}  \mytextsc{clf}~: \textcolor{darkblue}{\textbf{\ipa{nɑ˧}}} 
\lhead{\firstmark}
\rhead{\botmark}

\subsection{\hspace{-0.5cm} {\Large \textcolor{darkblue}{\textbf{\ipa{ʂwæ˩\textsubscript{a}}}}}\hspace{0.5cm}[\kern2pt{\textcolor{darkblue}{\textbf{\ipa{ʂwæ˩˥}}}}\kern2pt]} \hypertarget{s`w\{\string_Ba1}{}
\markboth{\textcolor{darkblue}{\textbf{\ipa{ʂwæ˩\textsubscript{a}}}}}{}
\textcolor{teal}{\mytextsc{verbe}} \hspace{4pt} Ton~: L\textsubscript{a}.
\textcolor{PineGreen}{\selectlanguage{french}Fumer (aliment).} \zh{熏。}  ¶ \textcolor{darkblue}{\textbf{\ipa{ʂe˧ ʂwæ˥}}} \textcolor{PineGreen}{\selectlanguage{french}fumer de la viande} \zh{熏肉}  

\lhead{\firstmark}
\rhead{\botmark}

\subsection{\hspace{-0.5cm} {\Large \textcolor{darkblue}{\textbf{\ipa{ʂwæ˩gv̩˩}}}}\hspace{0.5cm}[\kern2pt{\textcolor{darkblue}{\textbf{\ipa{ʂwæ˩gv̩˩˥}}}}\kern2pt]} \hypertarget{s`w\{\string_Bgv\string_=\string_B1}{}
\markboth{\textcolor{darkblue}{\textbf{\ipa{ʂwæ˩gv̩˩}}}}{}
\textcolor{teal}{\mytextsc{nom}} \hspace{4pt} Ton~: L.
\textcolor{PineGreen}{\selectlanguage{french}Buffet (où est rangée la vaisselle).} \zh{柜子。}  \zh{量词}: \textcolor{darkblue}{\textbf{\ipa{ɭɯ˧}}}  \mytextsc{clf}~: \textcolor{darkblue}{\textbf{\ipa{ɭɯ˧}}} 
\lhead{\firstmark}
\rhead{\botmark}

\subsection{\hspace{-0.5cm} {\Large \textcolor{darkblue}{\textbf{\ipa{ʂwæ˧˥}}}}\hspace{0.5cm}[\kern2pt{\textcolor{darkblue}{\textbf{\ipa{ʂwæ˧˥}}}}\kern2pt]} \hypertarget{s`w\{\string_M\string_T1}{}
\markboth{\textcolor{darkblue}{\textbf{\ipa{ʂwæ˧˥}}}}{}
\textcolor{teal}{\mytextsc{verbe}} \hspace{4pt} Ton~: MH.
\textcolor{PineGreen}{\selectlanguage{french}Déféquer, faire caca.} \zh{拉(屎)。}  ¶ \textcolor{darkblue}{\textbf{\ipa{qʰæ˧ ʂwæ˩}}} \textcolor{PineGreen}{\selectlanguage{french}déféquer} \zh{拉屎}  

\lhead{\firstmark}
\rhead{\botmark}

\subsection{\hspace{-0.5cm} {\Large \textcolor{darkblue}{\textbf{\ipa{ʂwæ˩˥}}}}\hspace{0.5cm}[\kern2pt{\textcolor{darkblue}{\textbf{\ipa{ʂwæ˩˥}}}}\kern2pt]} \hypertarget{s`w\{\string_B\string_T1}{}
\markboth{\textcolor{darkblue}{\textbf{\ipa{ʂwæ˩˥}}}}{}
\textcolor{teal}{\mytextsc{nom}} \hspace{4pt} Ton~: LH.
\textcolor{PineGreen}{\selectlanguage{french}Coin.} \zh{楔子。}  ¶ \textcolor{darkblue}{\textbf{\ipa{ʂwæ˩ lɑ˧˥ / ʂwæ˩ lɑ˧-ze˥}}} \textcolor{PineGreen}{\selectlanguage{french}frapper un coin} \zh{打一个楔子}  
 ¶ \textcolor{darkblue}{\textbf{\ipa{ʂwæ˩ hwæ˥-ze˩}}} \textcolor{PineGreen}{\selectlanguage{french}...a acheté un coin} \zh{买了楔子}  
 ¶ \textcolor{darkblue}{\textbf{\ipa{ʂwæ˩ tʰv̩˩-ɭɯ˩˥ / ʂwæ˩ tʰv̩˩-ɭɯ˥}}} \textcolor{PineGreen}{\selectlanguage{french}\string_ \mytextsc{dem}+\mytextsc{clf}.générique} \zh{这个楔子}  
 ¶ \textcolor{darkblue}{\textbf{\ipa{ʂwæ˩ tʰv̩˩-kʰwɤ˩˥}}} \textcolor{PineGreen}{\selectlanguage{french}\mytextsc{n}+\mytextsc{dem}+\mytextsc{clf}.morceaux} \zh{这个楔子}  
 ¶ \textcolor{darkblue}{\textbf{\ipa{[F5] ʂwæ˩ kʰɯ˥}}} \textcolor{PineGreen}{\selectlanguage{french}mettre un coin} \zh{放一个楔子}  
 \zh{量词}: \textcolor{darkblue}{\textbf{\ipa{kʰwɤ˥ / ɭɯ˧}}}  \mytextsc{clf}~: \textcolor{darkblue}{\textbf{\ipa{kʰwɤ˥ / ɭɯ˧}}} 
\lhead{\firstmark}
\rhead{\botmark}

\newpage
\section*{\centering- \textcolor{darkblue}{\textbf{\ipa{t}}} -}
\subsection{\hspace{-0.5cm} {\Large \textcolor{darkblue}{\textbf{\ipa{tɑ˥}}}}\hspace{0.5cm}[\kern2pt{\textcolor{darkblue}{\textbf{\ipa{tɑ˥}}}}\kern2pt]} \hypertarget{tA\string_T1}{}
\markboth{\textcolor{darkblue}{\textbf{\ipa{tɑ˥}}}}{}
\textcolor{teal}{\mytextsc{adjectif}} \hspace{4pt} Ton~: H.
\textcolor{PineGreen}{\selectlanguage{french}Fiable.} \zh{可靠。}  ¶ \textcolor{darkblue}{\textbf{\ipa{le˧-tɑ˥ (| ʐwæ˩˥)}}} \textcolor{PineGreen}{\selectlanguage{french}très fiable} \zh{很靠谱}  
 ¶ \textcolor{darkblue}{\textbf{\ipa{le˧ mɤ˧-tɑ˥ (| ʐwæ˩˥)}}} \textcolor{PineGreen}{\selectlanguage{french}pas fiable} \zh{不靠谱}  
 ¶ \textcolor{darkblue}{\textbf{\ipa{no˧ | le˧-mɤ˧-tɑ˥-hĩ˩ ɖɯ˧-v̩˧ ɲi˩!}}} \textcolor{PineGreen}{\selectlanguage{french}Tu es quelqu'un de pas fiable/pas responsable!} \zh{你是不靠谱的人!}  

\lhead{\firstmark}
\rhead{\botmark}

\subsection{\hspace{-0.5cm} {\Large \textcolor{darkblue}{\textbf{\ipa{tɑ˥mo˩}}}}\hspace{0.5cm}[\kern2pt{\textcolor{darkblue}{\textbf{\ipa{xxxx ton non trouvé, à faire manuellement...}}}}\kern2pt]} \hypertarget{tA\string_Tmo\string_B1}{}
\markboth{\textcolor{darkblue}{\textbf{\ipa{tɑ˥mo˩}}}}{}
\textcolor{teal}{\mytextsc{verbe}} \hspace{4pt} Ton~: HL.
\textcolor{PineGreen}{\selectlanguage{french}Faner.} \zh{萎、萎蔫。}  ¶ \textcolor{darkblue}{\textbf{\ipa{lə˧-tɑ˥mo˩-ze˩!}}} \textcolor{PineGreen}{\selectlanguage{french}Ca a fané! (Exemple: une fleur coupée, une fleur abîmée par le vent ou par un soleil trop ardent.)} \zh{萎蔫了!}  

\lhead{\firstmark}
\rhead{\botmark}

\subsection{\hspace{-0.5cm} {\Large \textcolor{darkblue}{\textbf{\ipa{tɑ˧\textsubscript{a}}}}}\hspace{0.5cm}[\kern2pt{\textcolor{darkblue}{\textbf{\ipa{tɑ˩˥}}}}\kern2pt]} \hypertarget{tA\string_Ma1}{}
\markboth{\textcolor{darkblue}{\textbf{\ipa{tɑ˧\textsubscript{a}}}}}{}
\textcolor{teal}{\mytextsc{verbe}} \hspace{4pt} Ton~: M\textsubscript{a}.
\textcolor{PineGreen}{\selectlanguage{french}Chauffer au feu.} \zh{烘干。}  ¶ \textcolor{darkblue}{\textbf{\ipa{kwɤ˧-kʰɯ˧ tʰi˧-tɑ˧}}} \textcolor{PineGreen}{\selectlanguage{french}réchauffer (la nourriture...) auprès du feu} \zh{放在火炉旁边热一下(饭)}  

\lhead{\firstmark}
\rhead{\botmark}

\subsection{\hspace{-0.5cm} {\Large \textcolor{darkblue}{\textbf{\ipa{tɑ˧dzi˩}}}}\hspace{0.5cm}[\kern2pt{\textcolor{darkblue}{\textbf{\ipa{tɑ˩dzi˧˥}}}}\kern2pt]} \hypertarget{tA\string_Mdzi\string_B1}{}
\markboth{\textcolor{darkblue}{\textbf{\ipa{tɑ˧dzi˩}}}}{}
\textcolor{teal}{\mytextsc{nom}} \hspace{4pt} Ton~: L\#.
\textcolor{PineGreen}{\selectlanguage{french}Village na en contrebas de Nhissei, en contrehaut de Lataddi (\textcolor{darkblue}{\textbf{\ipa{/lɑ˧tʰɑ˧-di˧˥/}}}).} \zh{村落名。}  ¶ \textcolor{darkblue}{\textbf{\ipa{ɬi˧ki˧, | ɲi˧se˩, | tɑ˧dzi˩, | mv̩˧qʰwæ˩, | lɑ˧tʰɑ˧-di˧˥}}} \textcolor{PineGreen}{\selectlanguage{french}Villages dans l'ordre, après la plaine de Yongning, ne comptant pas comme faisant partie de Yongning. Le dernier, \textcolor{darkblue}{\textbf{\ipa{/lɑ˧tʰɑ˧-di˧˥/}}}, désigne toute la région na au-delà du quatrième village.} \zh{永宁到泸沽湖所经过的村落,依次是:里格、尼赛、大祖、木垮,然后到拉塔地(拉塔地指的是泸沽湖周边的摩梭地区,包括左所、洛水村等)}  

\lhead{\firstmark}
\rhead{\botmark}

\subsection{\hspace{-0.5cm} {\Large \textcolor{darkblue}{\textbf{\ipa{tɑ˧gɤ˩}}}}\hspace{0.5cm}[\kern2pt{\textcolor{darkblue}{\textbf{\ipa{tɑ˩gɤ˥}}}}\kern2pt]} \hypertarget{tA\string_Mg7\string_B1}{}
\markboth{\textcolor{darkblue}{\textbf{\ipa{tɑ˧gɤ˩}}}}{}
\textcolor{teal}{\mytextsc{adjectif}} \hspace{4pt} Ton~: L\#.
\textcolor{PineGreen}{\selectlanguage{french}Maigre (personne maigre).} \zh{瘦弱、枯瘦。} 
\lhead{\firstmark}
\rhead{\botmark}

\subsection{\hspace{-0.5cm} {\Large \textcolor{darkblue}{\textbf{\ipa{tɑ˧ho˧}}}}\hspace{0.5cm}[\kern2pt{\textcolor{darkblue}{\textbf{\ipa{tɑ˧ho˩}}}}\kern2pt]} \hypertarget{tA\string_Mho\string_M1}{}
\markboth{\textcolor{darkblue}{\textbf{\ipa{tɑ˧ho˧}}}}{}
\textcolor{teal}{\mytextsc{adverbe}} \hspace{4pt} Ton~: M.
\textcolor{PineGreen}{\selectlanguage{french}Ensemble.} \zh{一起。}  ¶ \textcolor{darkblue}{\textbf{\ipa{ɖɯ˧-ʁwɤ˧ tɑ˧ho˧ kʰi˧˥}}} \textcolor{PineGreen}{\selectlanguage{french}Tout le village y est allé ensemble.} \zh{全村一起去了。}  
 ¶ \textcolor{darkblue}{\textbf{\ipa{tɑ˧ho˧ ʝi˧}}} \textcolor{PineGreen}{\selectlanguage{french}travailler ensemble} \zh{一起工作}  
 ¶ \textcolor{darkblue}{\textbf{\ipa{tɑ˧ho˧ tsʰo˧}}} \textcolor{PineGreen}{\selectlanguage{french}danser ensemble} \zh{一起跳舞}  

\lhead{\firstmark}
\rhead{\botmark}

\subsection{\hspace{-0.5cm} {\Large \textcolor{darkblue}{\textbf{\ipa{tɑ˧ko˧}}}}\hspace{0.5cm}[\kern2pt{\textcolor{darkblue}{\textbf{\ipa{tɑ˧ko˧}}}}\kern2pt]} \hypertarget{tA\string_Mko\string_M1}{}
\markboth{\textcolor{darkblue}{\textbf{\ipa{tɑ˧ko˧}}}}{}
\textcolor{teal}{\mytextsc{verbe}} \hspace{4pt} Ton~: M.
\textcolor{PineGreen}{\selectlanguage{french}Trouver du travail non qualifié (sur un chantier...), gagner de l'argent en faisant des petits boulots.} \zh{打工(汉语借词)。}  Emprunt~: chinois  \zh{打工}
 ¶ \textcolor{darkblue}{\textbf{\ipa{tɑ˧ko˧ hɯ˧-ze˩!}}} \textcolor{PineGreen}{\selectlanguage{french}(Elle/il) est parti(e) gagner de l'argent en faisant des petits boulots!} \zh{(他)打工去了!}  

\lhead{\firstmark}
\rhead{\botmark}

\subsection{\hspace{-0.5cm} {\Large \textcolor{darkblue}{\textbf{\ipa{tɑ˧ko˩}}}}\hspace{0.5cm}[\kern2pt{\textcolor{darkblue}{\textbf{\ipa{tɑ˧ko˩}}}}\kern2pt]} \hypertarget{tA\string_Mko\string_B1}{}
\markboth{\textcolor{darkblue}{\textbf{\ipa{tɑ˧ko˩}}}}{}
\textcolor{teal}{\mytextsc{verbe}} \hspace{4pt} Ton~: L\#.
\textcolor{PineGreen}{\selectlanguage{french}Retarder.} \zh{耽误。}  ¶ \textcolor{darkblue}{\textbf{\ipa{hĩ˧ tɑ˧ko˥}}} \textcolor{PineGreen}{\selectlanguage{french}retarder les gens} \zh{耽误人家}  
 ¶ \textcolor{darkblue}{\textbf{\ipa{ʈʂʰɯ˧ hĩ˧ tɑ˧ko˥ | ʐwæ˩˥!}}} \textcolor{PineGreen}{\selectlanguage{french}Il/elle retarde tout le monde!} \zh{他耽误大家很多!}  

\lhead{\firstmark}
\rhead{\botmark}

\subsection{\hspace{-0.5cm} {\Large \textcolor{darkblue}{\textbf{\ipa{tɑ˧nɑ˩}}}}\hspace{0.5cm}[\kern2pt{\textcolor{darkblue}{\textbf{\ipa{tɑ˧nɑ˩}}}}\kern2pt]} \hypertarget{tA\string_MnA\string_B1}{}
\markboth{\textcolor{darkblue}{\textbf{\ipa{tɑ˧nɑ˩}}}}{}
\textcolor{teal}{\mytextsc{nom}} \hspace{4pt} Ton~: L\#.
\textcolor{PineGreen}{\selectlanguage{french}Arbalète.} \zh{弩弓。}  \zh{量词}: \textcolor{darkblue}{\textbf{\ipa{pɤ˩}}} \textcolor{darkblue}{\textbf{\ipa{nɑ˧}}}  \mytextsc{clf}~: \textcolor{darkblue}{\textbf{\ipa{pɤ˩}}} \textcolor{darkblue}{\textbf{\ipa{nɑ˧}}} 
\lhead{\firstmark}
\rhead{\botmark}

\subsection{\hspace{-0.5cm} {\Large \textcolor{darkblue}{\textbf{\ipa{tɑ˧pi˧}}}}\hspace{0.5cm}[\kern2pt{\textcolor{darkblue}{\textbf{\ipa{tɑ˧pi˧}}}}\kern2pt]} \hypertarget{tA\string_Mpi\string_M1}{}
\markboth{\textcolor{darkblue}{\textbf{\ipa{tɑ˧pi˧}}}}{}
\textcolor{teal}{\mytextsc{adjectif}} \hspace{4pt} Ton~: M.
\textcolor{PineGreen}{\selectlanguage{french}Identique à, pareil à, semblable à, à l'exemple de.} \zh{如、像、像……那样。}  ¶ \textcolor{darkblue}{\textbf{\ipa{no˧-bi˧ tɑ˩pi˩, ...}}} \textcolor{PineGreen}{\selectlanguage{french}selon ton exemple; comme toi; identique à toi} \zh{像你}  
 ¶ \textcolor{darkblue}{\textbf{\ipa{njɤ˧-bi˧ tɑ˩pi˩…}}} \textcolor{PineGreen}{\selectlanguage{french}comme moi; à mon exemple} \zh{像我}  
 ¶ \textcolor{darkblue}{\textbf{\ipa{no˧=ɻ̍˩-bv̩˩, | njɤ˧=ɻ̍˩-bv̩˩, | tɑ˧pi˧!}}} \textcolor{PineGreen}{\selectlanguage{french}Le mien et le tien, ils sont faits sur le même exemple =ils sont pareils! (au sujet de bâtiments, par exemple: les maisons d'un même village sont bâties sur le même modèle)} \zh{你家的房子,我家的房子,都是一样的!(如:一个村子里的房子,都是按同一个模式建设的。)}  
 ¶ \textcolor{darkblue}{\textbf{\ipa{no˧-ɳɯ˧ gv̩˩, | njɤ˧-ɳɯ˧-gv̩˩, | tɑ˧pi˧!}}} \textcolor{PineGreen}{\selectlanguage{french}que ce soit toi ou moi qui construise [une maison], c'est pareil! / c'est sur le même modèle!} \zh{无论是谁来盖房,盖出来的都一样!}  
 ¶ \textcolor{darkblue}{\textbf{\ipa{ʈʂʰɯ˧-bi˩ | tɑ˧pi˧, | njɤ˧-ɳɯ˧ dɑ˧-bi˥-ze˩!}}} \textcolor{PineGreen}{\selectlanguage{french}je vais construire [une maison] comme celle-là/ je vais imiter cette maison-là! / Je vais construire une maison qui sera pareille à la sienne!} \zh{我要盖跟这一样的房子!}  

\lhead{\firstmark}
\rhead{\botmark}

\subsection{\hspace{-0.5cm} {\Large \textcolor{darkblue}{\textbf{\ipa{tɑ˧pi˧}}}}\hspace{0.5cm}[\kern2pt{\textcolor{darkblue}{\textbf{\ipa{tɑ˧pi˧}}}}\kern2pt]} \hypertarget{tA\string_Mpi\string_M1}{}
\markboth{\textcolor{darkblue}{\textbf{\ipa{tɑ˧pi˧}}}}{}
\textcolor{teal}{\mytextsc{verbe}} \hspace{4pt} Ton~: M.
\textcolor{PineGreen}{\selectlanguage{french}Prendre pour exemple.} \zh{打比方(汉语借词:当地汉语方言‘打比’)。} Dialecte chinois local~:\zh{打比。} Emprunt~: chinois  \zh{打比}
 ¶ \textcolor{darkblue}{\textbf{\ipa{tɑ˧pi˧-ze˩}}} \textcolor{PineGreen}{\selectlanguage{french}\mytextsc{pfv}} \zh{打比方}  

\lhead{\firstmark}
\rhead{\botmark}

\subsection{\hspace{-0.5cm} {\Large \textcolor{darkblue}{\textbf{\ipa{tɑ˧pv̩˩}}}}\hspace{0.5cm}[\kern2pt{\textcolor{darkblue}{\textbf{\ipa{tɑ˧pv̩˩}}}}\kern2pt]} \hypertarget{tA\string_Mpv\string_=\string_B1}{}
\markboth{\textcolor{darkblue}{\textbf{\ipa{tɑ˧pv̩˩}}}}{}
\textcolor{teal}{\mytextsc{adjectif}} \hspace{4pt} Ton~: L\#.
\textcolor{PineGreen}{\selectlanguage{french}Séché (au soleil) (ex.: légumes).} \zh{晒干的(水果、蔬菜……)。}  ¶ \textcolor{darkblue}{\textbf{\ipa{v̩˩tsʰɤ˧-tɑ˧pv̩˥}}} \textcolor{PineGreen}{\selectlanguage{french}légumes séchés au soleil} \zh{晒干的蔬菜}  

\lhead{\firstmark}
\rhead{\botmark}

\subsection{\hspace{-0.5cm} {\Large \textcolor{darkblue}{\textbf{\ipa{tɑ˧pʰi˩}}}}\hspace{0.5cm}[\kern2pt{\textcolor{darkblue}{\textbf{\ipa{tɑ˧pʰi˩}}}}\kern2pt]} \hypertarget{tA\string_Mp\string_hi\string_B1}{}
\markboth{\textcolor{darkblue}{\textbf{\ipa{tɑ˧pʰi˩}}}}{}
\textcolor{teal}{\mytextsc{nom}} \hspace{4pt} Ton~: L\#.
\textcolor{PineGreen}{\selectlanguage{french}Armoise de Chine, \textit{Artemisia argyi}.} \zh{艾、艾蒿。}  \zh{量词}: \textcolor{darkblue}{\textbf{\ipa{dzi˩}}}  \mytextsc{clf}~: \textcolor{darkblue}{\textbf{\ipa{dzi˩}}} 
\lhead{\firstmark}
\rhead{\botmark}

\subsection{\hspace{-0.5cm} {\Large \textcolor{darkblue}{\textbf{\ipa{tɑ˧\textasciitilde{}tɑ˧}}} \textsubscript{1}}\hspace{0.5cm}[\kern2pt{\textcolor{darkblue}{\textbf{\ipa{tɑ˧tɑ˧}}}}\kern2pt]} \hypertarget{tA\string_M~tA\string_M1}{}
\markboth{\textcolor{darkblue}{\textbf{\ipa{tɑ˧\textasciitilde{}tɑ˧}}} \textsubscript{1}}{}
\textcolor{teal}{\mytextsc{adjectif}} \hspace{4pt} Ton~: M.
\textcolor{PineGreen}{\selectlanguage{french}Sérieux, attentif, soigneux; exact, précis (une chaussure convient précisément à un pied; quelqu'un observe avec précision/exactitude).} \zh{严肃认真、细心、细致,(看得)清楚、清晰。}  ¶ \textcolor{darkblue}{\textbf{\ipa{mɤ˧-tɑ˧\textasciitilde{}tɑ˧}}} \textcolor{PineGreen}{\selectlanguage{french}pas sérieux, négligé (au sujet d'un travail)} \zh{邋遢、草率、潦草}  
 ¶ \textcolor{darkblue}{\textbf{\ipa{lo˧ ʝi˧ mɤ˧-tɑ˧\textasciitilde{}tɑ˧}}} \textcolor{PineGreen}{\selectlanguage{french}travailler sans soin, de façon négligée} \zh{工作草率}  
 ¶ \textcolor{darkblue}{\textbf{\ipa{hĩ˧ ʈʂʰɯ˧-v̩˧, | tɑ˧\textasciitilde{}tɑ˧!}}} \textcolor{PineGreen}{\selectlanguage{french}lui, il est soigneux!} \zh{他很认真!}  
\textit{Voir~:} \hyperlink{}{\textcolor{darkblue}{\textbf{\ipa{tɑ˧\textasciitilde{}tɑ˧}}} \textsubscript{2}} 
\lhead{\firstmark}
\rhead{\botmark}

\subsection{\hspace{-0.5cm} {\Large \textcolor{darkblue}{\textbf{\ipa{tɑ˧\textasciitilde{}tɑ˧}}} \textsubscript{2}}\hspace{0.5cm}[\kern2pt{\textcolor{darkblue}{\textbf{\ipa{tɑ˧tɑ˧}}}}\kern2pt]} \hypertarget{tA\string_M~tA\string_M2}{}
\markboth{\textcolor{darkblue}{\textbf{\ipa{tɑ˧\textasciitilde{}tɑ˧}}} \textsubscript{2}}{}
\textcolor{teal}{\mytextsc{adverbe}} \hspace{4pt} Ton~: M.
\textcolor{PineGreen}{\selectlanguage{french}Précisément, justement, juste à point nommé.} \zh{刚(好)、正(好)。}  ¶ \textcolor{darkblue}{\textbf{\ipa{tɑ˧\textasciitilde{}tɑ˧ | ho˩˥! |}}} \textcolor{PineGreen}{\selectlanguage{french}Ca convient exactement/ça convient précisément (ex.: au sujet d'une paire de chaussures qu'on vient de vous offrir)} \zh{刚刚好!(如:一双鞋刚好合适)}  
 ¶ \textcolor{darkblue}{\textbf{\ipa{le˧-li˧ tɑ˧\textasciitilde{}tɑ˧}}} \textcolor{PineGreen}{\selectlanguage{french}voir clairement} \zh{看清楚}  
\textit{Voir~:} \hyperlink{}{\textcolor{darkblue}{\textbf{\ipa{tɑ˧\textasciitilde{}tɑ˧}}} \textsubscript{1}} 
\lhead{\firstmark}
\rhead{\botmark}

\subsection{\hspace{-0.5cm} {\Large \textcolor{darkblue}{\textbf{\ipa{tɑ˩\textsubscript{a}}}}}\hspace{0.5cm}[\kern2pt{\textcolor{darkblue}{\textbf{\ipa{tɑ˧˥}}}}\kern2pt]} \hypertarget{tA\string_Ba1}{}
\markboth{\textcolor{darkblue}{\textbf{\ipa{tɑ˩\textsubscript{a}}}}}{}
\textcolor{teal}{\mytextsc{classificateur}} \hspace{4pt} Ton~: L\textsubscript{a}.
\textcolor{PineGreen}{\selectlanguage{french}Classificateur des fortes sommes d'argent.} \zh{量词:钱(一笔)。}  ¶ \textcolor{darkblue}{\textbf{\ipa{ɖʐe˧ | ɖɯ˧-tɑ˩}}} \textcolor{PineGreen}{\selectlanguage{french}un paquet d'argent, une liasse de billets…} \zh{一笔钱}  

\lhead{\firstmark}
\rhead{\botmark}

\subsection{\hspace{-0.5cm} {\Large \textcolor{darkblue}{\textbf{\ipa{tɑ˩dv̩˧˥}}}}\hspace{0.5cm}[\kern2pt{\textcolor{darkblue}{\textbf{\ipa{tɑ˩dv̩˥}}}}\kern2pt]} \hypertarget{tA\string_Bdv\string_=\string_M\string_T1}{}
\markboth{\textcolor{darkblue}{\textbf{\ipa{tɑ˩dv̩˧˥}}}}{}
\textcolor{teal}{\mytextsc{nom}} \hspace{4pt} Ton~: LM+MH\#.
\textcolor{PineGreen}{\selectlanguage{french}Poche.} \zh{口袋、衣袋、兜子。}  ¶ \textcolor{darkblue}{\textbf{\ipa{njɤ˧ | tɑ˩dv̩˧-qo˥ | tsʰe˩mæ˩-tɑ˥kɤ˩-lɑ˩ dʑo˩!}}} \textcolor{PineGreen}{\selectlanguage{french}Je n’ai que dix yuan en poche!} \zh{我兜子里只有十元钱!}  
 \zh{量词}: \textcolor{darkblue}{\textbf{\ipa{ɭɯ˧}}}  \mytextsc{clf}~: \textcolor{darkblue}{\textbf{\ipa{ɭɯ˧}}} 
\lhead{\firstmark}
\rhead{\botmark}

\subsection{\hspace{-0.5cm} {\Large \textcolor{darkblue}{\textbf{\ipa{tɑ˩dʑɤ\#˥}}}}\hspace{0.5cm}[\kern2pt{\textcolor{darkblue}{\textbf{\ipa{tɑ˩dʑɤ˧˥}}}}\kern2pt]} \hypertarget{tA\string_Bdz£7\#\string_T1}{}
\markboth{\textcolor{darkblue}{\textbf{\ipa{tɑ˩dʑɤ\#˥}}}}{}
\textcolor{teal}{\mytextsc{nom}} \hspace{4pt} Ton~: LM+\#H.
\textcolor{PineGreen}{\selectlanguage{french}Prénom masculin employé pour le second des jumeaux.} \zh{男性名字,双胞胎中老二的名字。} 
\lhead{\firstmark}
\rhead{\botmark}

\subsection{\hspace{-0.5cm} {\Large \textcolor{darkblue}{\textbf{\ipa{tɑ˩ɖʐo˧dzi˧˥}}}}\hspace{0.5cm}[\kern2pt{\textcolor{darkblue}{\textbf{\ipa{tɑ˧ɖʐo˧dzi˩}}}}\kern2pt]} \hypertarget{tA\string_Bd`z`o\string_Mdzi\string_M\string_T1}{}
\markboth{\textcolor{darkblue}{\textbf{\ipa{tɑ˩ɖʐo˧dzi˧˥}}}}{}
\textcolor{teal}{\mytextsc{nom}} \hspace{4pt} Ton~: LM+MH\#.
\textcolor{PineGreen}{\selectlanguage{french}Petit drapeau de prière.} \zh{小经幡。}  \zh{量词}: \textcolor{darkblue}{\textbf{\ipa{dzi˩}}}  \mytextsc{clf}~: \textcolor{darkblue}{\textbf{\ipa{dzi˩}}} 
\lhead{\firstmark}
\rhead{\botmark}

\subsection{\hspace{-0.5cm} {\Large \textcolor{darkblue}{\textbf{\ipa{tɑ˩hwɤ˩}}}}\hspace{0.5cm}[\kern2pt{\textcolor{darkblue}{\textbf{\ipa{tɑ˧hwɤ˧}}}}\kern2pt]} \hypertarget{tA\string_Bhw7\string_B1}{}
\markboth{\textcolor{darkblue}{\textbf{\ipa{tɑ˩hwɤ˩}}}}{}
\textcolor{teal}{\mytextsc{verbe}} \hspace{4pt} Ton~: L.
 Emprunt~: chinois  \zh{打发?}
\ding{202} \textcolor{PineGreen}{\selectlanguage{french}Offrir des présents à des gens extérieurs à la famille.} \zh{送礼(给家里以外的人)。}  ¶ \textcolor{darkblue}{\textbf{\ipa{hĩ˧-ki˧ | ɖɯ˧-kʰwɤ˧ tɑ˥hwɤ˩-zo˩-ʝi˩!}}} \textcolor{PineGreen}{\selectlanguage{french}Il va falloir faire un présent aux gens! / Ca va être l'occasion de faire un présent aux gens! (par exemple à l'occasion d'un rituel de passage à l'âge adulte)} \zh{应该给人家送礼了!(例如,人家为孩子进行成年礼时,要送礼。)}  
 ¶ \textcolor{darkblue}{\textbf{\ipa{zo˧mv̩˥-ki˩, | tɑ˩hwɤ˩ mɤ˥-kv̩˩!}}} \textcolor{PineGreen}{\selectlanguage{french}On ne fait pas de présents aux enfants! (Explication: le présent donné par une famille à une autre de façon ritualisée est offert aux aînés, pas aux enfants; c'est d'une nature différente des petits cadeaux qu'on peut leur faire au quotidien.)} \zh{不会专门给孩子送(大)礼的!(说明:送礼,是送给家里的主人)}  
 ¶ \textcolor{darkblue}{\textbf{\ipa{ʐɯ˧ tɑ˩hwɤ˩}}} \textcolor{PineGreen}{\selectlanguage{french}offrir du vin comme présent} \zh{送酒(作为礼物)}  
 ¶ \textcolor{darkblue}{\textbf{\ipa{li˩ tɑ˥hwɤ˩}}} \textcolor{PineGreen}{\selectlanguage{french}offrir du thé comme présent} \zh{送茶(作为礼物)}  
 ¶ \textcolor{darkblue}{\textbf{\ipa{dze˧ tɑ˥hwɤ˩}}} \textcolor{PineGreen}{\selectlanguage{french}offrir des sucreries/des bonbons comme présent} \zh{送糖(作为礼物)}  
\ding{203} \textcolor{PineGreen}{\selectlanguage{french}Fournir la dot: donner des biens à une jeune femme lorsqu'elle rejoint sa nouvelle famille lors du mariage. (Note: la dot est apportée à dos de cheval; elle est rangée dans deux caisses en bois: comme en d'autres circonstances, les présents doivent aller par deux.).} \zh{送陪嫁(嫁妆、陪奁)。}  ¶ \textcolor{darkblue}{\textbf{\ipa{ə˧tso˧ tɑ˩hwɤ˩-ʝi˩? | ə˧-sɯ˩kv̩˩ (-dʑo˩), | ɖɯ˧-li˧-ɻ̍˩-bi˩!}}} \textcolor{PineGreen}{\selectlanguage{french}Qu'est-ce qu'ils ont donné comme dot? Regardons un peu! / Allons voir! (Ce que disent les villageois invités à un mariage; les biens offerts en dot sont alors exposés, de façon à ce que chacun puisse apprécier la générosité des parents.)} \zh{给的是什么嫁妆?咱们去看一看吧!(结婚的时候,陪嫁展示在大家眼前,显示女方家的大方程度)}  
 ¶ \textcolor{darkblue}{\textbf{\ipa{ti˧tsɯ˥ | qʰɑ˧-ɭɯ˧ tɑ˩hwɤ˩? - ti˧tsɯ˥ | ɲi˧-ɭɯ˧ tɑ˩hwɤ˩!}}} \textcolor{PineGreen}{\selectlanguage{french}Combien de caisses sont offertes en dot/ de combien de caisses la dot se compose-t-elle? - De deux caisses!} \zh{陪嫁有几个木箱? - 陪嫁有两个(木箱)!}  

\lhead{\firstmark}
\rhead{\botmark}

\subsection{\hspace{-0.5cm} {\Large \textcolor{darkblue}{\textbf{\ipa{tɑ˩kɤ˧}}}}\hspace{0.5cm}[\kern2pt{\textcolor{darkblue}{\textbf{\ipa{tɑ˩kɤ˩˥}}}}\kern2pt]} \hypertarget{tA\string_Bk7\string_M1}{}
\markboth{\textcolor{darkblue}{\textbf{\ipa{tɑ˩kɤ˧}}}}{}
\textcolor{teal}{\mytextsc{verbe}} \hspace{4pt} Ton~: .
\textcolor{PineGreen}{\selectlanguage{french}Taquiner.} \zh{逗弄(动作)。} 
\lhead{\firstmark}
\rhead{\botmark}

\subsection{\hspace{-0.5cm} {\Large \textcolor{darkblue}{\textbf{\ipa{tɑ˩li˥}}}}\hspace{0.5cm}[\kern2pt{\textcolor{darkblue}{\textbf{\ipa{tɑ˩li˥}}}}\kern2pt]} \hypertarget{tA\string_Bli\string_T1}{}
\markboth{\textcolor{darkblue}{\textbf{\ipa{tɑ˩li˥}}}}{}
\textcolor{teal}{\mytextsc{nom}} \hspace{4pt} Ton~: LH.
\textcolor{PineGreen}{\selectlanguage{french}Dali (nom de ville).} \zh{大理(汉语借词)。}  Emprunt~: chinois  \zh{大理}

\lhead{\firstmark}
\rhead{\botmark}

\subsection{\hspace{-0.5cm} {\Large \textcolor{darkblue}{\textbf{\ipa{tɑ˩mv̩˩}}}}\hspace{0.5cm}[\kern2pt{\textcolor{darkblue}{\textbf{\ipa{tɑ˩mv̩˩˥}}}}\kern2pt]} \hypertarget{tA\string_Bmv\string_=\string_B1}{}
\markboth{\textcolor{darkblue}{\textbf{\ipa{tɑ˩mv̩˩}}}}{}
\textcolor{teal}{\mytextsc{verbe}} \hspace{4pt} Ton~: L.
\textcolor{PineGreen}{\selectlanguage{french}Proverbe.} \zh{谚语。}  ¶ \textcolor{darkblue}{\textbf{\ipa{æ˧ʂæ˧-tɑ˩mv̩˩}}} \textcolor{PineGreen}{\selectlanguage{french}même sens: proverbe (littéralement: 'proverbe ancien')} \zh{同上:谚语(直译:‘从前的老话’)}  
 ¶ \textcolor{darkblue}{\textbf{\ipa{[F5] æ˧ʂæ˧-tɑ˥mv̩˩}}} \textcolor{PineGreen}{\selectlanguage{french}proverbe; histoire ancienne} \zh{谚语、传统故事}  

\lhead{\firstmark}
\rhead{\botmark}

\subsection{\hspace{-0.5cm} {\Large \textcolor{darkblue}{\textbf{\ipa{tɑ˩so˩kʰo˥}}}}\hspace{0.5cm}[\kern2pt{\textcolor{darkblue}{\textbf{\ipa{tɑ˩so˩kʰo˥}}}}\kern2pt]} \hypertarget{tA\string_Bso\string_Bk\string_ho\string_T1}{}
\markboth{\textcolor{darkblue}{\textbf{\ipa{tɑ˩so˩kʰo˥}}}}{}
\textcolor{teal}{\mytextsc{adverbe}} \hspace{4pt} Ton~: L+H\#.
\textcolor{PineGreen}{\selectlanguage{french}En tailleur (posture assise).} \zh{盘腿(而坐)。}  ¶ \textcolor{darkblue}{\textbf{\ipa{tɑ˩so˩kʰo˥ | tʰi˧-dzi˩}}} \textcolor{PineGreen}{\selectlanguage{french}être assis en tailleur (posture assise des moines, et posture également courante chez les gens du commun)} \zh{打坐、盘腿而坐(和尚的坐姿)}  

\lhead{\firstmark}
\rhead{\botmark}

\subsection{\hspace{-0.5cm} {\Large \textcolor{darkblue}{\textbf{\ipa{tɑ˧˥}}}}\hspace{0.5cm}[\kern2pt{\textcolor{darkblue}{\textbf{\ipa{tɑ˧˥}}}}\kern2pt]} \hypertarget{tA\string_M\string_T1}{}
\markboth{\textcolor{darkblue}{\textbf{\ipa{tɑ˧˥}}}}{}
\textcolor{teal}{\mytextsc{verbe}} \hspace{4pt} Ton~: MH.
\textcolor{PineGreen}{\selectlanguage{french}Reculer, se retirer.} \zh{退后。}  ¶ \textcolor{darkblue}{\textbf{\ipa{ʁo˧tʰo˩ tɑ˩}}} \textcolor{PineGreen}{\selectlanguage{french}reculer, se retirer vers l'arrière} \zh{往后退}  

\lhead{\firstmark}
\rhead{\botmark}

\subsection{\hspace{-0.5cm} {\Large \textcolor{darkblue}{\textbf{\ipa{tɑ˧˥\textsubscript{a}}}}}\hspace{0.5cm}[\kern2pt{\textcolor{darkblue}{\textbf{\ipa{tɑ˥}}}}\kern2pt]} \hypertarget{tA\string_M\string_Ta1}{}
\markboth{\textcolor{darkblue}{\textbf{\ipa{tɑ˧˥\textsubscript{a}}}}}{}
\textcolor{teal}{\mytextsc{classificateur}} \hspace{4pt} Ton~: MH\textsubscript{a}.
\textcolor{PineGreen}{\selectlanguage{french}Entièrement, tout, tout le monde.} \zh{量词:全部、一切,大家。}  ¶ \textcolor{darkblue}{\textbf{\ipa{ɖɯ˧-tɑ˧˥}}} \textcolor{PineGreen}{\selectlanguage{french}entièrement, tout, tout le monde} \zh{全部、一切,大家}  
 ¶ \textcolor{darkblue}{\textbf{\ipa{ɖɯ˧-tɑ˧=ɻæ˥}}} \textcolor{PineGreen}{\selectlanguage{french}entièrement, tout, tout le monde (même sens que ci-dessus, avec le morphème de pluriel)} \zh{全部、一切,大家(同上,加上多数词素)}  

\lhead{\firstmark}
\rhead{\botmark}

\subsection{\hspace{-0.5cm} {\Large \textcolor{darkblue}{\textbf{\ipa{tɑ˩˥fv˩˥}}}}\hspace{0.5cm}[\kern2pt{\textcolor{darkblue}{\textbf{\ipa{tɑ˩fv˥}}}}\kern2pt]} \hypertarget{tA\string_B\string_Tfv\string_B\string_T1}{}
\markboth{\textcolor{darkblue}{\textbf{\ipa{tɑ˩˥fv˩˥}}}}{}
\textcolor{teal}{\mytextsc{nom}} \hspace{4pt} Ton~: LH.LH.
\textcolor{PineGreen}{\selectlanguage{french}Excrément.} \zh{大粪(汉语借词)。}  Emprunt~: chinois  \zh{大粪}

\lhead{\firstmark}
\rhead{\botmark}

\subsection{\hspace{-0.5cm} {\Large \textcolor{darkblue}{\textbf{\ipa{tæ˧pv̩˩}}}}\hspace{0.5cm}[\kern2pt{\textcolor{darkblue}{\textbf{\ipa{tæ˧pv̩˩}}}}\kern2pt]} \hypertarget{t\{\string_Mpv\string_=\string_B1}{}
\markboth{\textcolor{darkblue}{\textbf{\ipa{tæ˧pv̩˩}}}}{}
\textcolor{teal}{\mytextsc{adjectif}} \hspace{4pt} Ton~: L\#.
\textcolor{PineGreen}{\selectlanguage{french}Maigre, sec (personne maigre, au corps sec).} \zh{瘦(人很瘦)。}  ¶ \textcolor{darkblue}{\textbf{\ipa{v˩tsʰɤ˧˥ | le˧-tæ˥pv˩ kʰɯ˩}}} \textcolor{PineGreen}{\selectlanguage{french}faire sécher des légumes} \zh{将蔬菜弄干(晒干)}  
 ¶ \textcolor{darkblue}{\textbf{\ipa{v˩tsʰɤ˧˥ | tæ˧pv˩ gv˩}}} \textcolor{PineGreen}{\selectlanguage{french}faire sécher des légumes} \zh{将蔬菜弄干(晒干)}  
\textit{Voir~:} \hyperlink{}{\textcolor{darkblue}{\textbf{\ipa{tɑ˧gɤ˩}}}} 
\lhead{\firstmark}
\rhead{\botmark}

\subsection{\hspace{-0.5cm} {\Large \textcolor{darkblue}{\textbf{\ipa{tæ˧ɻæ˩}}}}\hspace{0.5cm}[\kern2pt{\textcolor{darkblue}{\textbf{\ipa{tæ˧ɻæ˩}}}}\kern2pt]} \hypertarget{t\{\string_Mr£`\{\string_B1}{}
\markboth{\textcolor{darkblue}{\textbf{\ipa{tæ˧ɻæ˩}}}}{}
\textcolor{teal}{\mytextsc{nom}} \hspace{4pt} Ton~: L\#.
\textcolor{PineGreen}{\selectlanguage{french}Pomme d'Adam; larynx, gorge, oesophage.} \zh{喉管、喉结。}  \zh{量词}: \textcolor{darkblue}{\textbf{\ipa{ɭɯ˧}}}  \mytextsc{clf}~: \textcolor{darkblue}{\textbf{\ipa{ɭɯ˧}}} 
\lhead{\firstmark}
\rhead{\botmark}

\subsection{\hspace{-0.5cm} {\Large \textcolor{darkblue}{\textbf{\ipa{ti˧}}}}\hspace{0.5cm}[\kern2pt{\textcolor{darkblue}{\textbf{\ipa{ti˥}}}}\kern2pt]} \hypertarget{ti\string_M1}{}
\markboth{\textcolor{darkblue}{\textbf{\ipa{ti˧}}}}{}
\textcolor{teal}{\mytextsc{verbe}} \hspace{4pt} Ton~: M.
\textcolor{PineGreen}{\selectlanguage{french}Mûrir, devenir adulte (d'une personne).} \zh{成熟(人成熟)。}  ¶ \textcolor{darkblue}{\textbf{\ipa{hĩ˧ tʰv̩˧-v̩˧, | gɤ˩-ti˧-ze˧!}}} \textcolor{PineGreen}{\selectlanguage{french}Cette personne a grandi / est devenue adulte / a mûri!} \zh{这个人,成熟了! / 是大人了!}  
 ¶ \textcolor{darkblue}{\textbf{\ipa{hĩ˧ tʰv̩˧-v̩˧, | mɤ˧-ti˧-sɯ˩!}}} \textcolor{PineGreen}{\selectlanguage{french}Cette personne n'est pas encore adulte!} \zh{这个人,还不成熟!}  
 ¶ \textcolor{darkblue}{\textbf{\ipa{zo˩mv̩˧ | gɤ˩-ti˧, | lo˧ hɑ˧!}}} \textcolor{PineGreen}{\selectlanguage{french}La croissance d'un enfant (jusqu'à l'âge adulte), c'est pas facile! (La difficulté est pour les parents, et aussi pour l'enfant)} \zh{孩子长成熟(的过程),还是挺难的!}  

\lhead{\firstmark}
\rhead{\botmark}

\subsection{\hspace{-0.5cm} {\Large \textcolor{darkblue}{\textbf{\ipa{ti˧ɖo˥}}}}\hspace{0.5cm}[\kern2pt{\textcolor{darkblue}{\textbf{\ipa{ti˩ɖo˩˥}}}}\kern2pt]} \hypertarget{ti\string_Md`o\string_T1}{}
\markboth{\textcolor{darkblue}{\textbf{\ipa{ti˧ɖo˥}}}}{}
\textcolor{teal}{\mytextsc{nom}} \hspace{4pt} Ton~: H\#.
\textcolor{PineGreen}{\selectlanguage{french}Prénom masculin.} \zh{男性名字。} 
\lhead{\firstmark}
\rhead{\botmark}

\subsection{\hspace{-0.5cm} {\Large \textcolor{darkblue}{\textbf{\ipa{ti˧pʰv̩\#˥}}}}\hspace{0.5cm}[\kern2pt{\textcolor{darkblue}{\textbf{\ipa{ti˩pʰv̩˩˥}}}}\kern2pt]} \hypertarget{ti\string_Mp\string_hv\string_=\#\string_T1}{}
\markboth{\textcolor{darkblue}{\textbf{\ipa{ti˧pʰv̩\#˥}}}}{}
\textcolor{teal}{\mytextsc{nom}} \hspace{4pt} Ton~: \#H.
\textcolor{PineGreen}{\selectlanguage{french}Coupe de cuivre pour les offrandes; elle est évasée, et de la taille d'un petit gobelet à alcool.} \zh{铜杯盏,做仪式用的。}  \zh{量词}: \textcolor{darkblue}{\textbf{\ipa{fv̩˩}}}  \mytextsc{clf}~: \textcolor{darkblue}{\textbf{\ipa{fv̩˩}}} 
\lhead{\firstmark}
\rhead{\botmark}

\subsection{\hspace{-0.5cm} {\Large \textcolor{darkblue}{\textbf{\ipa{ti˧tsɯ˥}}}}\hspace{0.5cm}[\kern2pt{\textcolor{darkblue}{\textbf{\ipa{ti˩tsɯ˥}}}}\kern2pt]} \hypertarget{ti\string_MtsM\string_T1}{}
\markboth{\textcolor{darkblue}{\textbf{\ipa{ti˧tsɯ˥}}}}{}
\textcolor{teal}{\mytextsc{nom}} \hspace{4pt} Ton~: H\#.
\textcolor{PineGreen}{\selectlanguage{french}Boîte en vannerie (objet qui n'est plus en usage aujourd'hui).} \zh{竹箱。}  \zh{量词}: \textcolor{darkblue}{\textbf{\ipa{ɭɯ˧}}}  \mytextsc{clf}~: \textcolor{darkblue}{\textbf{\ipa{ɭɯ˧}}} 
\lhead{\firstmark}
\rhead{\botmark}

\subsection{\hspace{-0.5cm} {\Large \textcolor{darkblue}{\textbf{\ipa{ti˧ʈʂʰɯ˩}}}}\hspace{0.5cm}[\kern2pt{\textcolor{darkblue}{\textbf{\ipa{ti˧ʈʂʰɯ˥}}}}\kern2pt]} \hypertarget{ti\string_Mt`s`\string_hM\string_B1}{}
\markboth{\textcolor{darkblue}{\textbf{\ipa{ti˧ʈʂʰɯ˩}}}}{}
\textcolor{teal}{\mytextsc{nom}} \hspace{4pt} Ton~: L\#.
\textcolor{PineGreen}{\selectlanguage{french}Marteau.} \zh{铁锤。}  \zh{量词}: \textcolor{darkblue}{\textbf{\ipa{ɭɯ˧}}}  \mytextsc{clf}~: \textcolor{darkblue}{\textbf{\ipa{ɭɯ˧}}} 
\lhead{\firstmark}
\rhead{\botmark}

\subsection{\hspace{-0.5cm} {\Large \textcolor{darkblue}{\textbf{\ipa{ti˩\textsubscript{a}}}}}\hspace{0.5cm}[\kern2pt{\textcolor{darkblue}{\textbf{\ipa{ti˩˥}}}}\kern2pt]} \hypertarget{ti\string_Ba1}{}
\markboth{\textcolor{darkblue}{\textbf{\ipa{ti˩\textsubscript{a}}}}}{}
\textcolor{teal}{\mytextsc{verbe}} \hspace{4pt} Ton~: L\textsubscript{a}.
\ding{202} \textcolor{PineGreen}{\selectlanguage{french}Piler: réduire quelque chose en poudre dans un mortier, par des coups répétés; comprimer de la terre pour former un mur de terre.} \zh{捣(花椒、大蒜……)。}  ¶ \textcolor{darkblue}{\textbf{\ipa{læ˧tsɯ˥ ti˩}}} \textcolor{PineGreen}{\selectlanguage{french}piler le piment, réduire le piment en poudre} \zh{捣辣椒}  
 ¶ \textcolor{darkblue}{\textbf{\ipa{tsʰo˧ko˧ ti˩}}} \textcolor{PineGreen}{\selectlanguage{french}piler la cardamome, réduire de cardamome en poudre} \zh{捣草果}  
 ¶ \textcolor{darkblue}{\textbf{\ipa{dze˩ ti˥}}} \textcolor{PineGreen}{\selectlanguage{french}piler le xanthoxyle, réduire le xanthoxyle en poudre} \zh{捣花椒}  
 ¶ \textcolor{darkblue}{\textbf{\ipa{ʈʂo˩bo˩ ti˥}}} \textcolor{PineGreen}{\selectlanguage{french}construire un mur en terre en comprimant la terre à coups de masse} \zh{垒土墙}  
\ding{203} \textcolor{PineGreen}{\selectlanguage{french}Donner une tape, tapoter, frapper quelqu'un légèrement.} \zh{拍打。}  ¶ \textcolor{darkblue}{\textbf{\ipa{hĩ˧ ti˥}}} \textcolor{PineGreen}{\selectlanguage{french}donner une tape à quelqu'un} \zh{拍打人}  
 ¶ \textcolor{darkblue}{\textbf{\ipa{hĩ˧ | ɖɯ˧-v̩˧ ti˩-ze˩}}} \textcolor{PineGreen}{\selectlanguage{french}(Elle/il) a donné une tape à quelqu'un.} \zh{(他)拍打了某人。}  

\lhead{\firstmark}
\rhead{\botmark}

\subsection{\hspace{-0.5cm} {\Large \textcolor{darkblue}{\textbf{\ipa{ti˩pʰo˩}}}}\hspace{0.5cm}[\kern2pt{\textcolor{darkblue}{\textbf{\ipa{ti˩pʰo˥}}}}\kern2pt]} \hypertarget{ti\string_Bp\string_ho\string_B1}{}
\markboth{\textcolor{darkblue}{\textbf{\ipa{ti˩pʰo˩}}}}{}
\textcolor{teal}{\mytextsc{nom}} \hspace{4pt} Ton~: L.
\textcolor{PineGreen}{\selectlanguage{french}Plafond.} \zh{天花板。}  \zh{量词}: \textcolor{darkblue}{\textbf{\ipa{nɑ˧}}}  \mytextsc{clf}~: \textcolor{darkblue}{\textbf{\ipa{nɑ˧}}} 
\lhead{\firstmark}
\rhead{\botmark}

\subsection{\hspace{-0.5cm} {\Large \textcolor{darkblue}{\textbf{\ipa{ti˩tje˧}}}}\hspace{0.5cm}[\kern2pt{\textcolor{darkblue}{\textbf{\ipa{ti˧tje˧}}}}\kern2pt]} \hypertarget{ti\string_Btje\string_M1}{}
\markboth{\textcolor{darkblue}{\textbf{\ipa{ti˩tje˧}}}}{}
\textcolor{teal}{\mytextsc{verbe}} \hspace{4pt} Ton~: LM.
\textcolor{PineGreen}{\selectlanguage{french}Traiter (quelqu'un).} \zh{对待(汉语借词)。}  Emprunt~: chinois  \zh{对待}

\lhead{\firstmark}
\rhead{\botmark}

\subsection{\hspace{-0.5cm} {\Large \textcolor{darkblue}{\textbf{\ipa{ti˧˥}}}}\hspace{0.5cm}[\kern2pt{\textcolor{darkblue}{\textbf{\ipa{ti˧˥}}}}\kern2pt]} \hypertarget{ti\string_M\string_T1}{}
\markboth{\textcolor{darkblue}{\textbf{\ipa{ti˧˥}}}}{}
\textcolor{teal}{\mytextsc{verbe}} \hspace{4pt} Ton~: MH.
\textcolor{PineGreen}{\selectlanguage{french}Décider, fixer, arrêter.} \zh{决定(汉语借词:定)。}  Emprunt~: chinois  \zh{定}

\lhead{\firstmark}
\rhead{\botmark}

\subsection{\hspace{-0.5cm} {\Large \textcolor{darkblue}{\textbf{\ipa{ti˧˥\textsubscript{a}}}}}\hspace{0.5cm}[\kern2pt{\textcolor{darkblue}{\textbf{\ipa{ti˩˥}}}}\kern2pt]} \hypertarget{ti\string_M\string_Ta1}{}
\markboth{\textcolor{darkblue}{\textbf{\ipa{ti˧˥\textsubscript{a}}}}}{}
\textcolor{teal}{\mytextsc{classificateur}} \hspace{4pt} Ton~: MH\textsubscript{a}.
\textcolor{PineGreen}{\selectlanguage{french}Couche (de poussière; de planches constituant un plancher; de tissu…).} \zh{量词:层(一层灰、一层木板……)。}  ¶ \textcolor{darkblue}{\textbf{\ipa{dʑɯ˩-nɑ˩mi˩˥ | gv̩˧-ti˩-qo˩ tʰv̩˩}}} \textcolor{PineGreen}{\selectlanguage{french}se retrouver au plus profond de la forêt: littéralement “dans la neuvième couche de forêt/d'alpage” (=au plus profond; on ne compte pas au-delà de la 9e “couche”; ce décompte est métaphorique, il ne correspond pas à un décompte en étapes à pied, par exemple.} \zh{到深山老林的最深处。直译:‘到深山老林的第九层’。这里的‘九’作为最高的数字,表示‘极深’的意思:不能说‘深山老林的第一层’、‘第二层’等。}  

\lhead{\firstmark}
\rhead{\botmark}

\subsection{\hspace{-0.5cm} {\Large \textcolor{darkblue}{\textbf{\ipa{tjɤ˧hwɑ˧˥}}}}\hspace{0.5cm}[\kern2pt{\textcolor{darkblue}{\textbf{\ipa{tjɤ˧hwɑ˧}}}}\kern2pt]} \hypertarget{tj7\string_MhwA\string_M\string_T1}{}
\markboth{\textcolor{darkblue}{\textbf{\ipa{tjɤ˧hwɑ˧˥}}}}{}
\textcolor{teal}{\mytextsc{nom}} \hspace{4pt} Ton~: MH\#.
\textcolor{PineGreen}{\selectlanguage{french}Téléphone.} \zh{电话(汉语借词)。}  Emprunt~: chinois  \zh{电话}
 \zh{量词}: \textcolor{darkblue}{\textbf{\ipa{ɭɯ˧}}}  \mytextsc{clf}~: \textcolor{darkblue}{\textbf{\ipa{ɭɯ˧}}} 
\lhead{\firstmark}
\rhead{\botmark}

\subsection{\hspace{-0.5cm} {\Large \textcolor{darkblue}{\textbf{\ipa{tjɤ˧po˧}}}}\hspace{0.5cm}[\kern2pt{\textcolor{darkblue}{\textbf{\ipa{tjɤ˧po˧˥}}}}\kern2pt]} \hypertarget{tj7\string_Mpo\string_M1}{}
\markboth{\textcolor{darkblue}{\textbf{\ipa{tjɤ˧po˧}}}}{}
\textcolor{teal}{\mytextsc{nom}} \hspace{4pt} Ton~: M.
\textcolor{PineGreen}{\selectlanguage{french}Fortin, forteresse.} \zh{碉堡(汉语借词)。}  Emprunt~: chinois  \zh{碉堡}
 \zh{量词}: \textcolor{darkblue}{\textbf{\ipa{ɭɯ˧}}}  \mytextsc{clf}~: \textcolor{darkblue}{\textbf{\ipa{ɭɯ˧}}} 
\lhead{\firstmark}
\rhead{\botmark}

\subsection{\hspace{-0.5cm} {\Large \textcolor{darkblue}{\textbf{\ipa{tjɤ˩˥ʂɯ˧}}}}\hspace{0.5cm}[\kern2pt{\textcolor{darkblue}{\textbf{\ipa{tjɤ˧ʂɯ˧}}}}\kern2pt]} \hypertarget{tj7\string_B\string_Ts`M\string_M1}{}
\markboth{\textcolor{darkblue}{\textbf{\ipa{tjɤ˩˥ʂɯ˧}}}}{}
\textcolor{teal}{\mytextsc{nom}} \hspace{4pt} Ton~: LH.M.
\textcolor{PineGreen}{\selectlanguage{french}Télévision.} \zh{电视(汉语借词)。}  Emprunt~: chinois  \zh{电视}
 ¶ \textcolor{darkblue}{\textbf{\ipa{tjɤ˩ʂɯ˧ li˥}}} \textcolor{PineGreen}{\selectlanguage{french}regarder la télévision} \zh{看电视}  
 ¶ \textcolor{darkblue}{\textbf{\ipa{tjɤ˩ʂɯ˧-qo˥}}} \textcolor{PineGreen}{\selectlanguage{french}à la télévision} \zh{电视上}  

\lhead{\firstmark}
\rhead{\botmark}

\subsection{\hspace{-0.5cm} {\Large \textcolor{darkblue}{\textbf{\ipa{to˥\textsubscript{a}}}}}\hspace{0.5cm}[\kern2pt{\textcolor{darkblue}{\textbf{\ipa{to˩˥}}}}\kern2pt]} \hypertarget{to\string_Ta1}{}
\markboth{\textcolor{darkblue}{\textbf{\ipa{to˥\textsubscript{a}}}}}{}
\textcolor{teal}{\mytextsc{classificateur}} \hspace{4pt} Ton~: H\textsubscript{a}.
\textcolor{PineGreen}{\selectlanguage{french}Une grande brassée, ce qu'on peut prendre dans les bras: par exemple lors de la récolte: une brassée de riz coupé.} \zh{量词:抱。}  ¶ \textcolor{darkblue}{\textbf{\ipa{qʰv̩˩ɖʐæ˩˥ | ɖɯ˧-to˥}}} \textcolor{PineGreen}{\selectlanguage{french}toute une brassée de ficelle/cordelette (voir le récit TraderAndHisSon)} \zh{一抱绳子}  

\lhead{\firstmark}
\rhead{\botmark}

\subsection{\hspace{-0.5cm} {\Large \textcolor{darkblue}{\textbf{\ipa{to˧bɤ\#˥}}}}\hspace{0.5cm}[\kern2pt{\textcolor{darkblue}{\textbf{\ipa{to˩bɤ˥}}}}\kern2pt]} \hypertarget{to\string_Mb7\#\string_T1}{}
\markboth{\textcolor{darkblue}{\textbf{\ipa{to˧bɤ\#˥}}}}{}
\textcolor{teal}{\mytextsc{adjectif}} \hspace{4pt} Ton~: \#H.
\textcolor{PineGreen}{\selectlanguage{french}Vide.} \zh{空。}  ¶ \textcolor{darkblue}{\textbf{\ipa{to˧bɤ˧-ze˩}}} \textcolor{PineGreen}{\selectlanguage{french}il n'y a plus rien (ex.: un bol est complètement vidé)} \zh{空了}  
 ¶ \textcolor{darkblue}{\textbf{\ipa{to˧bɤ˧ ɲi˥}}} \textcolor{PineGreen}{\selectlanguage{french}\string_ \mytextsc{cop}: c'est vide} \zh{是空的}  

\lhead{\firstmark}
\rhead{\botmark}

\subsection{\hspace{-0.5cm} {\Large \textcolor{darkblue}{\textbf{\ipa{to˧kɤ\#˥}}}}\hspace{0.5cm}[\kern2pt{\textcolor{darkblue}{\textbf{\ipa{to˧kɤ˧}}}}\kern2pt]} \hypertarget{to\string_Mk7\#\string_T1}{}
\markboth{\textcolor{darkblue}{\textbf{\ipa{to˧kɤ\#˥}}}}{}
\textcolor{teal}{\mytextsc{nom}} \hspace{4pt} Ton~: \#H.
\ding{202} \textcolor{PineGreen}{\selectlanguage{french}Front.} \zh{额头。}  \zh{量词}: \textcolor{darkblue}{\textbf{\ipa{ʈv̩˩}}} \ding{203} \textcolor{PineGreen}{\selectlanguage{french}Chance, bonne fortune.} \zh{运气。}  ¶ \textcolor{darkblue}{\textbf{\ipa{to˧kɤ˧ dʑɤ˥}}} \textcolor{PineGreen}{\selectlanguage{french}avoir de la chance; avoir un bon karma} \zh{好运气,运气好}  
 ¶ \textcolor{darkblue}{\textbf{\ipa{njɤ˧ | tsʰi˧ʝi˧ | to˧kɤ˧ dʑjɤ˥ (+ | ʐwæ˩˥)}}} \textcolor{PineGreen}{\selectlanguage{french}Cette année, j'ai de la chance!} \zh{我今年运气好!}  
 \mytextsc{clf}~: \textcolor{darkblue}{\textbf{\ipa{ʈv̩˩}}} 
\lhead{\firstmark}
\rhead{\botmark}

\subsection{\hspace{-0.5cm} {\Large \textcolor{darkblue}{\textbf{\ipa{to˧kɤ˧qʰæ˩di˩ | -bæ˩bæ˩˥}}}}\hspace{0.5cm}[\kern2pt{\textcolor{darkblue}{\textbf{\ipa{xxxx non-correspondance entre le nombre de groupes tonals et le nombre de tons}}}}\kern2pt]} \hypertarget{to\string_Mk7\string_Mq\string_h\{\string_Bdi\string_B | -b\{\string_Bb\{\string_B\string_T1}{}
\markboth{\textcolor{darkblue}{\textbf{\ipa{to˧kɤ˧qʰæ˩di˩ | -bæ˩bæ˩˥}}}}{}
\textcolor{teal}{\mytextsc{nom}} \hspace{4pt} Ton~: .
\textcolor{PineGreen}{\selectlanguage{french}Plante à longs filaments.} \zh{永宁的一种植物。}  \zh{量词}: \textcolor{darkblue}{\textbf{\ipa{bæ˩}}}  \mytextsc{clf}~: \textcolor{darkblue}{\textbf{\ipa{bæ˩}}} 
\lhead{\firstmark}
\rhead{\botmark}

\subsection{\hspace{-0.5cm} {\Large \textcolor{darkblue}{\textbf{\ipa{to˧qɑ˧}}}}\hspace{0.5cm}[\kern2pt{\textcolor{darkblue}{\textbf{\ipa{to˧qɑ˧}}}}\kern2pt]} \hypertarget{to\string_MqA\string_M1}{}
\markboth{\textcolor{darkblue}{\textbf{\ipa{to˧qɑ˧}}}}{}
\textcolor{teal}{\mytextsc{nom}} \hspace{4pt} Ton~: M.
\textcolor{PineGreen}{\selectlanguage{french}Chevreau.} \zh{羔羊。}  \zh{量词}: \textcolor{darkblue}{\textbf{\ipa{ɭɯ˧}}}  \mytextsc{clf}~: \textcolor{darkblue}{\textbf{\ipa{ɭɯ˧}}} 
\lhead{\firstmark}
\rhead{\botmark}

\subsection{\hspace{-0.5cm} {\Large \textcolor{darkblue}{\textbf{\ipa{to˧\textasciitilde{}to˧\textsubscript{b}}}}}\hspace{0.5cm}[\kern2pt{\textcolor{darkblue}{\textbf{\ipa{to˧to˥}}}}\kern2pt]} \hypertarget{to\string_M~to\string_Mb1}{}
\markboth{\textcolor{darkblue}{\textbf{\ipa{to˧\textasciitilde{}to˧\textsubscript{b}}}}}{}
\textcolor{teal}{\mytextsc{verbe}} \hspace{4pt} Ton~: M\textsubscript{b}.
\textcolor{PineGreen}{\selectlanguage{french}Prendre un enfant dans ses bras.} \zh{抱小孩子、搂,互相拥抱。}  ¶ \textcolor{darkblue}{\textbf{\ipa{zo˧mv̩˥ to˩\textasciitilde{}to˩}}} \textcolor{PineGreen}{\selectlanguage{french}porter un enfant dans ses bras} \zh{抱小孩子}  

\lhead{\firstmark}
\rhead{\botmark}

\subsection{\hspace{-0.5cm} {\Large \textcolor{darkblue}{\textbf{\ipa{to˩\textsubscript{a}}}} \textsubscript{1}}\hspace{0.5cm}[\kern2pt{\textcolor{darkblue}{\textbf{\ipa{to˧˥}}}}\kern2pt]} \hypertarget{to\string_Ba1}{}
\markboth{\textcolor{darkblue}{\textbf{\ipa{to˩\textsubscript{a}}}} \textsubscript{1}}{}
\textcolor{teal}{\mytextsc{verbe}} \hspace{4pt} Ton~: L\textsubscript{a}.
\textcolor{PineGreen}{\selectlanguage{french}Lutter, faire de la lutte.} \zh{摔交。}  ¶ \textcolor{darkblue}{\textbf{\ipa{le˧-to˩-ze˩}}} \textcolor{PineGreen}{\selectlanguage{french}\mytextsc{accomp} \string_ \mytextsc{pfv}} \zh{\mytextsc{accomp} \string_ \mytextsc{pfv}}  
 ¶ \textcolor{darkblue}{\textbf{\ipa{ɖʐæ˧\textasciitilde{}ɖʐæ˧ to˩}}} \textcolor{PineGreen}{\selectlanguage{french}lutter, faire de la lutte} \zh{摔交}  

\lhead{\firstmark}
\rhead{\botmark}

\subsection{\hspace{-0.5cm} {\Large \textcolor{darkblue}{\textbf{\ipa{to˩\textsubscript{a}}}} \textsubscript{2}}\hspace{0.5cm}[\kern2pt{\textcolor{darkblue}{\textbf{\ipa{to˩˥}}}}\kern2pt]} \hypertarget{to\string_Ba2}{}
\markboth{\textcolor{darkblue}{\textbf{\ipa{to˩\textsubscript{a}}}} \textsubscript{2}}{}
\textcolor{teal}{\mytextsc{verbe}} \hspace{4pt} Ton~: L\textsubscript{a}.
\textcolor{PineGreen}{\selectlanguage{french}S'allonger.} \zh{躺下。}  ¶ \textcolor{darkblue}{\textbf{\ipa{tʰi˧-to˩-ɕjɤ˩}}} \textcolor{PineGreen}{\selectlanguage{french}se reposer en position allongée, s'allonger pour prendre un peu de repos} \zh{躺着休息}  

\lhead{\firstmark}
\rhead{\botmark}

\subsection{\hspace{-0.5cm} {\Large \textcolor{darkblue}{\textbf{\ipa{to˩\textsubscript{a}}}} \textsubscript{3}}\hspace{0.5cm}[\kern2pt{\textcolor{darkblue}{\textbf{\ipa{to˩˥}}}}\kern2pt]} \hypertarget{to\string_Ba3}{}
\markboth{\textcolor{darkblue}{\textbf{\ipa{to˩\textsubscript{a}}}} \textsubscript{3}}{}
\textcolor{teal}{\mytextsc{verbe}} \hspace{4pt} Ton~: L\textsubscript{a}.
\textcolor{PineGreen}{\selectlanguage{french}Être en relation, entretenir un lien de parenté.} \zh{有亲属关系。}  ¶ \textcolor{darkblue}{\textbf{\ipa{le˧-to˧\textasciitilde{}to˥}}} \textcolor{PineGreen}{\selectlanguage{french}\mytextsc{accomp} \string_ \mytextsc{red}} \zh{\mytextsc{accomp} \string_ \mytextsc{red}}  
 ¶ \textcolor{darkblue}{\textbf{\ipa{qʰwɤ˩ɖɯ˩˥ | le˧-to˩-ze˩}}} \textcolor{PineGreen}{\selectlanguage{french}(nous) avons acquis un lien de parenté! (par adoption, mariage...)} \zh{我们成了亲戚!(通过领养、婚姻……)}  

\lhead{\firstmark}
\rhead{\botmark}

\subsection{\hspace{-0.5cm} {\Large \textcolor{darkblue}{\textbf{\ipa{to˩bi\#˥}}}}\hspace{0.5cm}[\kern2pt{\textcolor{darkblue}{\textbf{\ipa{to˩bi˩˥}}}}\kern2pt]} \hypertarget{to\string_Bbi\#\string_T1}{}
\markboth{\textcolor{darkblue}{\textbf{\ipa{to˩bi\#˥}}}}{}
\textcolor{teal}{\mytextsc{nom}} \hspace{4pt} Ton~: LM+\#H.
\textcolor{PineGreen}{\selectlanguage{french}Bouteille.} \zh{瓶子。}  \zh{量词}: \textcolor{darkblue}{\textbf{\ipa{ɭɯ˧}}}  \mytextsc{clf}~: \textcolor{darkblue}{\textbf{\ipa{ɭɯ˧}}} 
\lhead{\firstmark}
\rhead{\botmark}

\subsection{\hspace{-0.5cm} {\Large \textcolor{darkblue}{\textbf{\ipa{to˩bi˩}}}}\hspace{0.5cm}[\kern2pt{\textcolor{darkblue}{\textbf{\ipa{to˥}}}}\kern2pt]} \hypertarget{to\string_Bbi\string_B1}{}
\markboth{\textcolor{darkblue}{\textbf{\ipa{to˩bi˩}}}}{}
\textcolor{teal}{\mytextsc{classificateur}} \hspace{4pt} Ton~: L.
\textcolor{PineGreen}{\selectlanguage{french}Auto-classificateur des bouteilles.} \zh{量词:瓶。}  ¶ \textcolor{darkblue}{\textbf{\ipa{ɖɯ˧-to˩bi˩, so˩-to˩bi˩˥, ʐv̩˧-to˥bi˩, qʰv̩˧-to˥bi˩, ʂɯ˧-to˩bi˩, gv̩˧-to˥bi˩, tsʰe˩-to˩bi˩˥}}} \textcolor{PineGreen}{\selectlanguage{french}association avec des numéraux, de 1 à 10. Comportement tonal identique pour 1 et 2, 4 et 5, 6 et 8.} \zh{与数词结合,一至十}  

\lhead{\firstmark}
\rhead{\botmark}

\subsection{\hspace{-0.5cm} {\Large \textcolor{darkblue}{\textbf{\ipa{to˩kʰv̩˩mi˥}}}}\hspace{0.5cm}[\kern2pt{\textcolor{darkblue}{\textbf{\ipa{to˩kʰv̩˩mi˥}}}}\kern2pt]} \hypertarget{to\string_Bk\string_hv\string_=\string_Bmi\string_T1}{}
\markboth{\textcolor{darkblue}{\textbf{\ipa{to˩kʰv̩˩mi˥}}}}{}
\textcolor{teal}{\mytextsc{nom}} \hspace{4pt} Ton~: L+H\#.
\textcolor{PineGreen}{\selectlanguage{french}Chien (animal mâle).} \zh{公狗。}  \zh{量词}: \textcolor{darkblue}{\textbf{\ipa{mi˩}}} \textcolor{darkblue}{\textbf{\ipa{pʰo˧˥}}}  \mytextsc{clf}~: \textcolor{darkblue}{\textbf{\ipa{mi˩}}} \textcolor{darkblue}{\textbf{\ipa{pʰo˧˥}}} 
\lhead{\firstmark}
\rhead{\botmark}

\subsection{\hspace{-0.5cm} {\Large \textcolor{darkblue}{\textbf{\ipa{to˩mi˩}}} \textsubscript{1}}\hspace{0.5cm}[\kern2pt{\textcolor{darkblue}{\textbf{\ipa{to˩mi˩˥}}}}\kern2pt]} \hypertarget{to\string_Bmi\string_B1}{}
\markboth{\textcolor{darkblue}{\textbf{\ipa{to˩mi˩}}} \textsubscript{1}}{}
\textcolor{teal}{\mytextsc{nom}} \hspace{4pt} Ton~: L.
\textcolor{PineGreen}{\selectlanguage{french}Pilier.} \zh{柱子。}  ¶ \textcolor{darkblue}{\textbf{\ipa{hæ̃˧ʂɯ˩-to˩mi˩}}} \textcolor{PineGreen}{\selectlanguage{french}les Piliers d'Or, les Précieux Piliers: appellation solennelle pour les deux piliers de la maison} \zh{‘黄金柱’、‘宝贵柱’:对主屋两个柱子的庄严称呼}  
 \zh{量词}: \textcolor{darkblue}{\textbf{\ipa{ɭɯ˧}}}  \mytextsc{clf}~: \textcolor{darkblue}{\textbf{\ipa{ɭɯ˧}}} 
\lhead{\firstmark}
\rhead{\botmark}

\subsection{\hspace{-0.5cm} {\Large \textcolor{darkblue}{\textbf{\ipa{to˩mi˩}}} \textsubscript{2}}\hspace{0.5cm}[\kern2pt{\textcolor{darkblue}{\textbf{\ipa{to˩mi˩˥}}}}\kern2pt]} \hypertarget{to\string_Bmi\string_B2}{}
\markboth{\textcolor{darkblue}{\textbf{\ipa{to˩mi˩}}} \textsubscript{2}}{}
\textcolor{teal}{\mytextsc{nom}} \hspace{4pt} Ton~: L.
\textcolor{PineGreen}{\selectlanguage{french}Grande pente (mot attesté, pas combinaison artificielle).} \zh{大山坡。} 
\lhead{\firstmark}
\rhead{\botmark}

\subsection{\hspace{-0.5cm} {\Large \textcolor{darkblue}{\textbf{\ipa{to˩pi˩}}}}\hspace{0.5cm}[\kern2pt{\textcolor{darkblue}{\textbf{\ipa{to˩pi˩˥}}}}\kern2pt]} \hypertarget{to\string_Bpi\string_B1}{}
\markboth{\textcolor{darkblue}{\textbf{\ipa{to˩pi˩}}}}{}
\textcolor{teal}{\mytextsc{classificateur}} \hspace{4pt} Ton~: L.
\textcolor{PineGreen}{\selectlanguage{french}Fois, multiple de.} \zh{量词:倍(多几倍、少几倍等等)。}  ¶ \textcolor{darkblue}{\textbf{\ipa{ɖɯ˧-to˩pi˩, ɲi˧-to˩pi˩, so˩-to˩pi˩˥, ʐv̩˧-to˥pi˩, qʰv̩˧-to˥pi˩, ʂɯ˧-to˩pi˩, gv̩˧-to˥pi˩, tsʰe˩-to˩pi˩˥}}} \textcolor{PineGreen}{\selectlanguage{french}association avec des numéraux, de 1 à 10. Comportement tonal identique pour 1 et 2, 4 et 5, 6 et 8.} \zh{与数词结合,一至十}  

\lhead{\firstmark}
\rhead{\botmark}

\subsection{\hspace{-0.5cm} {\Large \textcolor{darkblue}{\textbf{\ipa{to˩pv̩˧}}}}\hspace{0.5cm}[\kern2pt{\textcolor{darkblue}{\textbf{\ipa{to˩pv̩˥}}}}\kern2pt]} \hypertarget{to\string_Bpv\string_=\string_M1}{}
\markboth{\textcolor{darkblue}{\textbf{\ipa{to˩pv̩˧}}}}{}
\textcolor{teal}{\mytextsc{adverbe}} \hspace{4pt} Ton~: LM.
\textcolor{PineGreen}{\selectlanguage{french}Au début, pour commencer.} \zh{最初。}  Emprunt~: tibétain?

\lhead{\firstmark}
\rhead{\botmark}

\subsection{\hspace{-0.5cm} {\Large \textcolor{darkblue}{\textbf{\ipa{to˩qo˩lv̩˥}}}}\hspace{0.5cm}[\kern2pt{\textcolor{darkblue}{\textbf{\ipa{to˩qo˩lv̩˥}}}}\kern2pt]} \hypertarget{to\string_Bqo\string_Blv\string_=\string_T1}{}
\markboth{\textcolor{darkblue}{\textbf{\ipa{to˩qo˩lv̩˥}}}}{}
\textcolor{teal}{\mytextsc{adjectif}} \hspace{4pt} Ton~: L+H\#.
\textcolor{PineGreen}{\selectlanguage{french}Rond.} \zh{圆形(球很圆)。}  ¶ \textcolor{darkblue}{\textbf{\ipa{to˩qo˩lv̩˥-gv̩˩}}} \textcolor{PineGreen}{\selectlanguage{french}rond} \zh{圆形}  

\lhead{\firstmark}
\rhead{\botmark}

\subsection{\hspace{-0.5cm} {\Large \textcolor{darkblue}{\textbf{\ipa{to˩qo˧˥}}}}\hspace{0.5cm}[\kern2pt{\textcolor{darkblue}{\textbf{\ipa{xxxx ton non trouvé, à faire manuellement...}}}}\kern2pt]} \hypertarget{to\string_Bqo\string_M\string_T1}{}
\markboth{\textcolor{darkblue}{\textbf{\ipa{to˩qo˧˥}}}}{}
\textcolor{teal}{\mytextsc{verbe}} \hspace{4pt} Ton~: L+MH\#.
\textcolor{PineGreen}{\selectlanguage{french}Renverser, verser; à l'envers (ex: renverser le contenu d'une boîte sur la table).} \zh{倒过来、倒放倒置。}  ¶ \textcolor{darkblue}{\textbf{\ipa{to˩qo˧˥ | tɕɯ˧}}} \textcolor{PineGreen}{\selectlanguage{french}mettre à l'envers, renverser} \zh{倒过来放}  
 ¶ \textcolor{darkblue}{\textbf{\ipa{njɤ˧-ɳɯ˧ | to˩qo˧-bi˧!}}} \textcolor{PineGreen}{\selectlanguage{french}je vais renverser (ce pot, cette assiette…)} \zh{我要(将这个东西)倒过来放!}  
 ¶ \textcolor{darkblue}{\textbf{\ipa{to˩qo˧-ze˥}}} \textcolor{PineGreen}{\selectlanguage{french}\mytextsc{pfv}} \zh{倒过来了}  

\lhead{\firstmark}
\rhead{\botmark}

\subsection{\hspace{-0.5cm} {\Large \textcolor{darkblue}{\textbf{\ipa{to˩to˧mi˥}}}}\hspace{0.5cm}[\kern2pt{\textcolor{darkblue}{\textbf{\ipa{to˩to˧mi˥}}}}\kern2pt]} \hypertarget{to\string_Bto\string_Mmi\string_T1}{}
\markboth{\textcolor{darkblue}{\textbf{\ipa{to˩to˧mi˥}}}}{}
\textcolor{teal}{\mytextsc{adverbe}} \hspace{4pt} Ton~: LM+H\#.
\ding{202} \textcolor{PineGreen}{\selectlanguage{french}Soigneusement, attentivement.} \zh{认真地。}  ¶ \textcolor{darkblue}{\textbf{\ipa{njɤ˧-ɳɯ˧ | to˩to˧ mi˥ | ʐwɤ˩-bi˩˥! |}}} \textcolor{PineGreen}{\selectlanguage{french}je vais parler soigneusement/je vais bien expliquer!} \zh{我要认真地讲!}  
 ¶ \textcolor{darkblue}{\textbf{\ipa{to˩to˧-mi˥ | so˩˥}}} \textcolor{PineGreen}{\selectlanguage{french}étudier attentivement} \zh{认真地学习}  
\ding{203} \textcolor{PineGreen}{\selectlanguage{french}Volontairement, délibérément, de propos délibéré: quelqu'un fait exprès de faire quelque chose.} \zh{故意地。} 
\lhead{\firstmark}
\rhead{\botmark}

\subsection{\hspace{-0.5cm} {\Large \textcolor{darkblue}{\textbf{\ipa{to˩ʈɯ˩}}}}\hspace{0.5cm}[\kern2pt{\textcolor{darkblue}{\textbf{\ipa{to˧ʈɯ˧}}}}\kern2pt]} \hypertarget{to\string_Bt`M\string_B1}{}
\markboth{\textcolor{darkblue}{\textbf{\ipa{to˩ʈɯ˩}}}}{}
\textcolor{teal}{\mytextsc{adjectif}} \hspace{4pt} Ton~: L.
\textcolor{PineGreen}{\selectlanguage{french}Petit (d'un homme).} \zh{矮。}  ¶ \textcolor{darkblue}{\textbf{\ipa{to˩ʈɯ˩\textasciitilde{}ʈɯ˥}}} \textcolor{PineGreen}{\selectlanguage{french}petit, de petite taille} \zh{矮}  

\lhead{\firstmark}
\rhead{\botmark}

\subsection{\hspace{-0.5cm} {\Large \textcolor{darkblue}{\textbf{\ipa{to˩zo˩}}}}\hspace{0.5cm}[\kern2pt{\textcolor{darkblue}{\textbf{\ipa{to˧zo˧}}}}\kern2pt]} \hypertarget{to\string_Bzo\string_B1}{}
\markboth{\textcolor{darkblue}{\textbf{\ipa{to˩zo˩}}}}{}
\textcolor{teal}{\mytextsc{nom}} \hspace{4pt} Ton~: L.
\textcolor{PineGreen}{\selectlanguage{french}Petite pente (mot attesté, pas combinaison artificielle).} \zh{小山坡。} 
\lhead{\firstmark}
\rhead{\botmark}

\subsection{\hspace{-0.5cm} {\Large \textcolor{darkblue}{\textbf{\ipa{to˩˥}}}}\hspace{0.5cm}[\kern2pt{\textcolor{darkblue}{\textbf{\ipa{to˩˥}}}}\kern2pt]} \hypertarget{to\string_B\string_T1}{}
\markboth{\textcolor{darkblue}{\textbf{\ipa{to˩˥}}}}{}
\textcolor{teal}{\mytextsc{nom}} \hspace{4pt} Ton~: LH.
\textcolor{PineGreen}{\selectlanguage{french}Pente, versant escarpé de montagne/colline.} \zh{山坡,岗。}  ¶ \textcolor{darkblue}{\textbf{\ipa{to˩ do˩˥}}} \textcolor{PineGreen}{\selectlanguage{french}grimper une pente} \zh{爬山坡}  
 ¶ \textcolor{darkblue}{\textbf{\ipa{ʁwɤ˧-to˩}}} \textcolor{PineGreen}{\selectlanguage{french}pente de montagne} \zh{山坡}  
 \zh{量词}: \textcolor{darkblue}{\textbf{\ipa{ɭɯ˧}}}  \mytextsc{clf}~: \textcolor{darkblue}{\textbf{\ipa{ɭɯ˧}}} \textcolor{PineGreen}{\selectlanguage{french}objets ronds}
\lhead{\firstmark}
\rhead{\botmark}

\subsection{\hspace{-0.5cm} {\Large \textcolor{darkblue}{\textbf{\ipa{tv̩˧˥}}} \textsubscript{1}}\hspace{0.5cm}[\kern2pt{\textcolor{darkblue}{\textbf{\ipa{tv̩˥}}}}\kern2pt]} \hypertarget{tv\string_=\string_M\string_T1}{}
\markboth{\textcolor{darkblue}{\textbf{\ipa{tv̩˧˥}}} \textsubscript{1}}{}
\textcolor{teal}{\mytextsc{verbe}} \hspace{4pt} Ton~: MH.
\textcolor{PineGreen}{\selectlanguage{french}Soutenir.} \zh{搀扶、撑住、稳住。} 
\lhead{\firstmark}
\rhead{\botmark}

\subsection{\hspace{-0.5cm} {\Large \textcolor{darkblue}{\textbf{\ipa{tv̩˧˥}}} \textsubscript{2}}\hspace{0.5cm}[\kern2pt{\textcolor{darkblue}{\textbf{\ipa{tv̩˧˥}}}}\kern2pt]} \hypertarget{tv\string_=\string_M\string_T2}{}
\markboth{\textcolor{darkblue}{\textbf{\ipa{tv̩˧˥}}} \textsubscript{2}}{}
\textcolor{teal}{\mytextsc{verbe}} \hspace{4pt} Ton~: MH.
\textcolor{PineGreen}{\selectlanguage{french}Verser (un liquide) dans la bouche de quelqu'un, faire boire à quelqu'un.} \zh{喂,喂到嘴里。}  ¶ \textcolor{darkblue}{\textbf{\ipa{ʈʂʰæ˧ɣɯ˧ | tʰi˧-tv̩˧˥}}} \textcolor{PineGreen}{\selectlanguage{french}verser un médicament dans la bouche de quelqu'un, faire boire un médicament à quelqu'un} \zh{喂药}  

\lhead{\firstmark}
\rhead{\botmark}

\subsection{\hspace{-0.5cm} {\Large \textcolor{darkblue}{\textbf{\ipa{tv̩˧\textsubscript{a}}}}}\hspace{0.5cm}[\kern2pt{\textcolor{darkblue}{\textbf{\ipa{tv̩˩˥}}}}\kern2pt]} \hypertarget{tv\string_=\string_Ma1}{}
\markboth{\textcolor{darkblue}{\textbf{\ipa{tv̩˧\textsubscript{a}}}}}{}
\textcolor{teal}{\mytextsc{verbe}} \hspace{4pt} Ton~: M\textsubscript{a}.
\textcolor{PineGreen}{\selectlanguage{french}Planter; aussi: repiquer (le riz).} \zh{耕种、插秧。}  ¶ \textcolor{darkblue}{\textbf{\ipa{ɕi˧ tv̩˧}}} \textcolor{PineGreen}{\selectlanguage{french}repiquer le riz} \zh{插秧}  
 ¶ \textcolor{darkblue}{\textbf{\ipa{le˧-tv̩˧-ze˧}}} \textcolor{PineGreen}{\selectlanguage{french}\mytextsc{accomp} \string_ \mytextsc{pfv}}  
 ¶ \textcolor{darkblue}{\textbf{\ipa{le˧-tv̩˥-tv̩˩-ze˩}}} \textcolor{PineGreen}{\selectlanguage{french}\mytextsc{red}}  

\lhead{\firstmark}
\rhead{\botmark}

\subsection{\hspace{-0.5cm} {\Large \textcolor{darkblue}{\textbf{\ipa{tv̩˧\textsubscript{a}}}} \textsubscript{1}}\hspace{0.5cm}[\kern2pt{\textcolor{darkblue}{\textbf{\ipa{tv̩˥}}}}\kern2pt]} \hypertarget{tv\string_=\string_Ma1}{}
\markboth{\textcolor{darkblue}{\textbf{\ipa{tv̩˧\textsubscript{a}}}} \textsubscript{1}}{}
\textcolor{teal}{\mytextsc{classificateur}} \hspace{4pt} Ton~: M\textsubscript{a}.
\textcolor{PineGreen}{\selectlanguage{french}1.000.} \zh{千(数词充当量词)。}  ¶ \textcolor{darkblue}{\textbf{\ipa{ɖɯ˧-tv̩˧}}} \textcolor{PineGreen}{\selectlanguage{french}mille} \zh{一千}  
 ¶ \textcolor{darkblue}{\textbf{\ipa{ɖɯ˧-tv̩˧ tv̩˧}}} \textcolor{PineGreen}{\selectlanguage{french}mille milliers = un million} \zh{一千千,等于一百万}  
 ¶ \textcolor{darkblue}{\textbf{\ipa{tsʰe˩-tv̩˩ mæ˥}}} \textcolor{PineGreen}{\selectlanguage{french}dix mille fois 10.000, soit cent millions} \zh{十千万,等于一亿}  

\lhead{\firstmark}
\rhead{\botmark}

\subsection{\hspace{-0.5cm} {\Large \textcolor{darkblue}{\textbf{\ipa{tv̩˧\textsubscript{a}}}} \textsubscript{2}}\hspace{0.5cm}[\kern2pt{\textcolor{darkblue}{\textbf{\ipa{tv̩˥}}}}\kern2pt]} \hypertarget{tv\string_=\string_Ma2}{}
\markboth{\textcolor{darkblue}{\textbf{\ipa{tv̩˧\textsubscript{a}}}} \textsubscript{2}}{}
\textcolor{teal}{\mytextsc{classificateur}} \hspace{4pt} Ton~: M\textsubscript{a}.
\textcolor{PineGreen}{\selectlanguage{french}Dixième d'unité monétaire.} \zh{量词:角(钱),一元的十分之一。} 
\lhead{\firstmark}
\rhead{\botmark}

\subsection{\hspace{-0.5cm} {\Large \textcolor{darkblue}{\textbf{\ipa{tv̩˧ɕi˩}}}}\hspace{0.5cm}[\kern2pt{\textcolor{darkblue}{\textbf{\ipa{tv̩˩ɕi˩˥}}}}\kern2pt]} \hypertarget{tv\string_=\string_Ms£i\string_B1}{}
\markboth{\textcolor{darkblue}{\textbf{\ipa{tv̩˧ɕi˩}}}}{}
\textcolor{teal}{\mytextsc{nom}} \hspace{4pt} Ton~: L\#.
\textcolor{PineGreen}{\selectlanguage{french}Millepattes.} \zh{蜈蚣。}  \zh{量词}: \textcolor{darkblue}{\textbf{\ipa{mi˩}}}  \mytextsc{clf}~: \textcolor{darkblue}{\textbf{\ipa{mi˩}}} 
\lhead{\firstmark}
\rhead{\botmark}

\subsection{\hspace{-0.5cm} {\Large \textcolor{darkblue}{\textbf{\ipa{tv̩˩ɭɯ˧˥}}}}\hspace{0.5cm}[\kern2pt{\textcolor{darkblue}{\textbf{\ipa{tv̩˩ɭɯ˧˥}}}}\kern2pt]} \hypertarget{tv\string_=\string_Bl\string_RM\string_M\string_T1}{}
\markboth{\textcolor{darkblue}{\textbf{\ipa{tv̩˩ɭɯ˧˥}}}}{}
\textcolor{teal}{\mytextsc{nom}} \hspace{4pt} Ton~: LM+MH\#.
\textcolor{PineGreen}{\selectlanguage{french}Hotte de grande qualité, dans laquelle on offrait des cadeaux; n'existe plus actuellement; était resserrée au milieu: de forme concave, pas convexe.} \zh{高级的背篓,过去用它放礼物。}  \zh{量词}: \textcolor{darkblue}{\textbf{\ipa{ɭɯ˧}}}  \mytextsc{clf}~: \textcolor{darkblue}{\textbf{\ipa{ɭɯ˧}}} 
\lhead{\firstmark}
\rhead{\botmark}

\subsection{\hspace{-0.5cm} {\Large \textcolor{darkblue}{\textbf{\ipa{tv̩˧po˩}}}}\hspace{0.5cm}[\kern2pt{\textcolor{darkblue}{\textbf{\ipa{tv̩˧po˩}}}}\kern2pt]} \hypertarget{tv\string_=\string_Mpo\string_B1}{}
\markboth{\textcolor{darkblue}{\textbf{\ipa{tv̩˧po˩}}}}{}
\textcolor{teal}{\mytextsc{verbe}} \hspace{4pt} Ton~: L\#.
\textcolor{PineGreen}{\selectlanguage{french}Parier, jouer à des jeux d'argent.} \zh{赌博(汉语借词)。}  Emprunt~: chinois  \zh{赌博}

\lhead{\firstmark}
\rhead{\botmark}

\subsection{\hspace{-0.5cm} {\Large \textcolor{darkblue}{\textbf{\ipa{tv̩˧qʰv̩˧}}}}\hspace{0.5cm}[\kern2pt{\textcolor{darkblue}{\textbf{\ipa{tv̩˧qʰv̩˧}}}}\kern2pt]} \hypertarget{tv\string_=\string_Mq\string_hv\string_=\string_M1}{}
\markboth{\textcolor{darkblue}{\textbf{\ipa{tv̩˧qʰv̩˧}}}}{}
\textcolor{teal}{\mytextsc{nom}} \hspace{4pt} Ton~: M.
\textcolor{PineGreen}{\selectlanguage{french}Tombe provisoire, où on place le corps du défunt avant la crémation.} \zh{临时坟墓。} 
\lhead{\firstmark}
\rhead{\botmark}

\subsection{\hspace{-0.5cm} {\Large \textcolor{darkblue}{\textbf{\ipa{tv̩˧tsʰɯ˧}}}}\hspace{0.5cm}[\kern2pt{\textcolor{darkblue}{\textbf{\ipa{tv̩˧tsʰɯ˧}}}}\kern2pt]} \hypertarget{tv\string_=\string_Mts\string_hM\string_M1}{}
\markboth{\textcolor{darkblue}{\textbf{\ipa{tv̩˧tsʰɯ˧}}}}{}
\textcolor{teal}{\mytextsc{nom}} \hspace{4pt} Ton~: M.
\ding{202} \textcolor{PineGreen}{\selectlanguage{french}Temps.} \zh{时间。}  ¶ \textcolor{darkblue}{\textbf{\ipa{[F5] njɤ˧ | tv̩˧tsʰɯ˧ mɤ˧-dʑo˧.}}} \textcolor{PineGreen}{\selectlanguage{french}Je n'ai pas le temps.} \zh{我没时间。}  
 ¶ \textcolor{darkblue}{\textbf{\ipa{[F5] njɤ˧ | tv̩˧tsʰɯ˧ dʑo˧.}}} \textcolor{PineGreen}{\selectlanguage{french}J'ai du temps libre. / J'ai le temps.} \zh{我有时间。}  
 \zh{量词}: \textcolor{darkblue}{\textbf{\ipa{ɭɯ˧}}} \ding{203} \textcolor{PineGreen}{\selectlanguage{french}Période de temps, heure.} \zh{时间段、小时。}  ¶ \textcolor{darkblue}{\textbf{\ipa{tv̩˧tsʰɯ˧ | ɖɯ˧-ɭɯ˧}}} \textcolor{PineGreen}{\selectlanguage{french}une heure} \zh{一个小时}  
 ¶ \textcolor{darkblue}{\textbf{\ipa{tv̩˧tsʰɯ˧ ɖɯ˧-ɭɯ˧ gv̩˧-ze˧!}}} \textcolor{PineGreen}{\selectlanguage{french}Une heure a passé.} \zh{一个小时过去了。}  
 ¶ \textcolor{darkblue}{\textbf{\ipa{tv̩˧tsʰɯ˧ ɖɯ˧-ɭɯ˧ le˧-hɯ˩-ze˩.}}} \textcolor{PineGreen}{\selectlanguage{french}Une heure s'est écoulée.} \zh{一个小时过去了。}  
 ¶ \textcolor{darkblue}{\textbf{\ipa{[F5] tv̩˧tsʰɯ˧ qʰɑ˧-ɭɯ˧?}}} \textcolor{PineGreen}{\selectlanguage{french}Quelle heure est-il?} \zh{几点了?}  
 \mytextsc{clf}~: \textcolor{darkblue}{\textbf{\ipa{ɭɯ˧}}} 
\lhead{\firstmark}
\rhead{\botmark}

\subsection{\hspace{-0.5cm} {\Large \textcolor{darkblue}{\textbf{\ipa{tv̩˩tv̩˩}}}}\hspace{0.5cm}[\kern2pt{\textcolor{darkblue}{\textbf{\ipa{tv̩˩tv̩˩˥}}}}\kern2pt]} \hypertarget{tv\string_=\string_Btv\string_=\string_B1}{}
\markboth{\textcolor{darkblue}{\textbf{\ipa{tv̩˩tv̩˩}}}}{}
\textcolor{teal}{\mytextsc{adjectif}} \hspace{4pt} Ton~: L.
\ding{202} \textcolor{PineGreen}{\selectlanguage{french}Droit, bien d'aplomb.} \zh{直,笔直的(如:站直)。} \ding{203} \textcolor{PineGreen}{\selectlanguage{french}Droit, intègre, honnête.} \zh{耿直。} 
\lhead{\firstmark}
\rhead{\botmark}

\subsection{\hspace{-0.5cm} {\Large \textcolor{darkblue}{\textbf{\ipa{tv̩˧tv̩˥}}}}\hspace{0.5cm}[\kern2pt{\textcolor{darkblue}{\textbf{\ipa{tv̩˩tv̩˩˥}}}}\kern2pt]} \hypertarget{tv\string_=\string_Mtv\string_=\string_T1}{}
\markboth{\textcolor{darkblue}{\textbf{\ipa{tv̩˧tv̩˥}}}}{}
\textcolor{teal}{\mytextsc{nom}} \hspace{4pt} Ton~: H\#.
\textcolor{PineGreen}{\selectlanguage{french}Chapeau.} \zh{帽子。}  \zh{量词}: \textcolor{darkblue}{\textbf{\ipa{ɭɯ˧}}}  \mytextsc{clf}~: \textcolor{darkblue}{\textbf{\ipa{ɭɯ˧}}} 
\lhead{\firstmark}
\rhead{\botmark}

\newpage
\section*{\centering- \textcolor{darkblue}{\textbf{\ipa{tʰ}}} -}
\subsection{\hspace{-0.5cm} {\Large \textcolor{darkblue}{\textbf{\ipa{tʰɑ˧‑}}}}\hspace{0.5cm}[\kern2pt{\textcolor{darkblue}{\textbf{\ipa{tʰɑ˥}}}}\kern2pt]} \hypertarget{t\string_hA\string_M‑1}{}
\markboth{\textcolor{darkblue}{\textbf{\ipa{tʰɑ˧‑}}}}{}
\textcolor{teal}{\mytextsc{préfixe}} \hspace{4pt} Ton~: M/0.
\textcolor{PineGreen}{\selectlanguage{french}Prohibitif.} \zh{不要、别\mytextsc{禁止式。}}  ¶ \textcolor{darkblue}{\textbf{\ipa{tʰɑ˧-lɑ˩\textasciitilde{}lɑ˩-ze˩!}}} \textcolor{PineGreen}{\selectlanguage{french}Arrêtez de vous disputer! / Ne vous disputez pas!} \zh{别吵架了!}  
 ¶ \textcolor{darkblue}{\textbf{\ipa{tʰɑ˧-dzo˧\textasciitilde{}dzo˥!}}} \textcolor{PineGreen}{\selectlanguage{french}Ne pas toucher! / Ne touchez pas!} \zh{不要动来动去! / 不要碰来碰去!}  

\lhead{\firstmark}
\rhead{\botmark}

\subsection{\hspace{-0.5cm} {\Large \textcolor{darkblue}{\textbf{\ipa{tʰɑ˧v̩˥}}}}\hspace{0.5cm}[\kern2pt{\textcolor{darkblue}{\textbf{\ipa{tʰɑ˧v̩˥}}}}\kern2pt]} \hypertarget{t\string_hA\string_Mv\string_=\string_T1}{}
\markboth{\textcolor{darkblue}{\textbf{\ipa{tʰɑ˧v̩˥}}}}{}
\textcolor{teal}{\mytextsc{nom}} \hspace{4pt} Ton~: H\#.
\textcolor{PineGreen}{\selectlanguage{french}Chambre des invités (emprunt au chinois local; sens en chinois standard: pièce centrale, salle de séjour). Il n'y a pas d'équivalent direct dans la maison na traditionnelle: c'est dans la resserre qu'on pouvait improviser une chambre supplémentaire.} \zh{堂屋(汉语借词),来指客房。}  \zh{量词}: \textcolor{darkblue}{\textbf{\ipa{ɭɯ˧}}}  \mytextsc{clf}~: \textcolor{darkblue}{\textbf{\ipa{ɭɯ˧}}} 
\lhead{\firstmark}
\rhead{\botmark}

\subsection{\hspace{-0.5cm} {\Large \textcolor{darkblue}{\textbf{\ipa{tʰɑ˩lo˧}}}}\hspace{0.5cm}[\kern2pt{\textcolor{darkblue}{\textbf{\ipa{tʰɑ˩lo˥}}}}\kern2pt]} \hypertarget{t\string_hA\string_Blo\string_M1}{}
\markboth{\textcolor{darkblue}{\textbf{\ipa{tʰɑ˩lo˧}}}}{}
\textcolor{teal}{\mytextsc{nom}} \hspace{4pt} Ton~: LM.
\textcolor{PineGreen}{\selectlanguage{french}Prononciation par les Na de thar lam, nom anciennement donné par les Tibétains à la plaine de Yongning.} \zh{永宁的藏语名称。}  Emprunt~: tibétain  thar lam
 ¶ \textcolor{darkblue}{\textbf{\ipa{tʰɑ˩lo˧-go˧bɤ˩}}} \textcolor{PineGreen}{\selectlanguage{french}le temple de Thar lam =le temple de Yongning, tel que l'appellent les Tibétains} \zh{永宁大寺}  
 ¶ \textcolor{darkblue}{\textbf{\ipa{tʰɑ˩lo˧ se˧gi˧ kɤ˩mv̩˩}}} \textcolor{PineGreen}{\selectlanguage{french}la montagne Gemu de Yongning} \zh{永宁格姆山}  

\lhead{\firstmark}
\rhead{\botmark}

\subsection{\hspace{-0.5cm} {\Large \textcolor{darkblue}{\textbf{\ipa{tʰɑ˩mi\#˥}}}}\hspace{0.5cm}[\kern2pt{\textcolor{darkblue}{\textbf{\ipa{tʰɑ˩mi˥}}}}\kern2pt]} \hypertarget{t\string_hA\string_Bmi\#\string_T1}{}
\markboth{\textcolor{darkblue}{\textbf{\ipa{tʰɑ˩mi\#˥}}}}{}
\textcolor{teal}{\mytextsc{nom}} \hspace{4pt} Ton~: LM+\#H.
\textcolor{PineGreen}{\selectlanguage{french}Buffle femelle.} \zh{母水牛。}  ¶ \textcolor{darkblue}{\textbf{\ipa{dʑi˧mi˧-tʰɑ˩mi˩}}} \textcolor{PineGreen}{\selectlanguage{french}même sens: buffle femelle} \zh{母水牛}  
 ¶ \textcolor{darkblue}{\textbf{\ipa{dʑi˧mi˧ ʈʂʰɯ˧-pʰo˩ dʑo˩, | tʰɑ˩mi˧ ɲi˥!}}} \textcolor{PineGreen}{\selectlanguage{french}ce buffle, c'est une femelle!} \zh{这头水牛是母的!}  
 \zh{量词}: \textcolor{darkblue}{\textbf{\ipa{pʰo˧˥}}}  \mytextsc{clf}~: \textcolor{darkblue}{\textbf{\ipa{pʰo˧˥}}} 
\lhead{\firstmark}
\rhead{\botmark}

\subsection{\hspace{-0.5cm} {\Large \textcolor{darkblue}{\textbf{\ipa{tʰɑ˩pʰv̩\#˥}}}}\hspace{0.5cm}[\kern2pt{\textcolor{darkblue}{\textbf{\ipa{tʰɑ˩pʰv̩˥}}}}\kern2pt]} \hypertarget{t\string_hA\string_Bp\string_hv\string_=\#\string_T1}{}
\markboth{\textcolor{darkblue}{\textbf{\ipa{tʰɑ˩pʰv̩\#˥}}}}{}
\textcolor{teal}{\mytextsc{nom}} \hspace{4pt} Ton~: LM+\#H.
\textcolor{PineGreen}{\selectlanguage{french}Buffle mâle.} \zh{公水牛。}  \zh{量词}: \textcolor{darkblue}{\textbf{\ipa{pʰo˧˥}}}  \mytextsc{clf}~: \textcolor{darkblue}{\textbf{\ipa{pʰo˧˥}}} 
\lhead{\firstmark}
\rhead{\botmark}

\subsection{\hspace{-0.5cm} {\Large \textcolor{darkblue}{\textbf{\ipa{tʰɑ˩tʰɑ˩}}}}\hspace{0.5cm}[\kern2pt{\textcolor{darkblue}{\textbf{\ipa{tʰɑ˩tʰɑ˩˥}}}}\kern2pt]} \hypertarget{t\string_hA\string_Bt\string_hA\string_B1}{}
\markboth{\textcolor{darkblue}{\textbf{\ipa{tʰɑ˩tʰɑ˩}}}}{}
\textcolor{teal}{\mytextsc{nom}} \hspace{4pt} Ton~: L.
\textcolor{PineGreen}{\selectlanguage{french}Bon coin pour la cueillette de champignons, de plantes sauvages...} \zh{采野生植物如菌子等的好地方。}  ¶ \textcolor{darkblue}{\textbf{\ipa{tʰɑ˩tʰɑ˩˥ | ɖɯ˧-kʰwɤ˥}}} \textcolor{PineGreen}{\selectlanguage{french}un bon coin (pour la cueillette)} \zh{一个好地方}  
 \zh{量词}: \textcolor{darkblue}{\textbf{\ipa{kʰwɤ˥}}}  \mytextsc{clf}~: \textcolor{darkblue}{\textbf{\ipa{kʰwɤ˥}}} 
\lhead{\firstmark}
\rhead{\botmark}

\subsection{\hspace{-0.5cm} {\Large \textcolor{darkblue}{\textbf{\ipa{tʰɑ˩zo\#˥}}}}\hspace{0.5cm}[\kern2pt{\textcolor{darkblue}{\textbf{\ipa{tʰɑ˩zo˥}}}}\kern2pt]} \hypertarget{t\string_hA\string_Bzo\#\string_T1}{}
\markboth{\textcolor{darkblue}{\textbf{\ipa{tʰɑ˩zo\#˥}}}}{}
\textcolor{teal}{\mytextsc{nom}} \hspace{4pt} Ton~: LM+\#H.
\textcolor{PineGreen}{\selectlanguage{french}Petit buffle, enfant du buffle.} \zh{小水牛。}  \zh{量词}: \textcolor{darkblue}{\textbf{\ipa{mi˩}}}  \mytextsc{clf}~: \textcolor{darkblue}{\textbf{\ipa{mi˩}}} 
\lhead{\firstmark}
\rhead{\botmark}

\subsection{\hspace{-0.5cm} {\Large \textcolor{darkblue}{\textbf{\ipa{tʰɑ˩-ʐwæ˧mi˧}}}}\hspace{0.5cm}[\kern2pt{\textcolor{darkblue}{\textbf{\ipa{tʰɑ˧ʐwæ˧mi˧}}}}\kern2pt]} \hypertarget{t\string_hA\string_B-z`w\{\string_Mmi\string_M1}{}
\markboth{\textcolor{darkblue}{\textbf{\ipa{tʰɑ˩-ʐwæ˧mi˧}}}}{}
\textcolor{teal}{\mytextsc{nom}} \hspace{4pt} Ton~: L-.
\textcolor{PineGreen}{\selectlanguage{french}Âne (mâle ou femelle).} \zh{驴子。}  \zh{量词}: \textcolor{darkblue}{\textbf{\ipa{pʰo˧˥}}}  \mytextsc{clf}~: \textcolor{darkblue}{\textbf{\ipa{pʰo˧˥}}} 
\lhead{\firstmark}
\rhead{\botmark}

\subsection{\hspace{-0.5cm} {\Large \textcolor{darkblue}{\textbf{\ipa{tʰɑ˧˥}}} \textsubscript{1}}\hspace{0.5cm}[\kern2pt{\textcolor{darkblue}{\textbf{\ipa{tʰɑ˧˥}}}}\kern2pt]} \hypertarget{t\string_hA\string_M\string_T1}{}
\markboth{\textcolor{darkblue}{\textbf{\ipa{tʰɑ˧˥}}} \textsubscript{1}}{}
\textcolor{teal}{\mytextsc{adjectif}} \hspace{4pt} Ton~: MH.
\textcolor{PineGreen}{\selectlanguage{french}Aiguisé, qui coupe bien, affûté.} \zh{锋利。} 
\lhead{\firstmark}
\rhead{\botmark}

\subsection{\hspace{-0.5cm} {\Large \textcolor{darkblue}{\textbf{\ipa{tʰɑ˧˥}}} \textsubscript{2}}\hspace{0.5cm}[\kern2pt{\textcolor{darkblue}{\textbf{\ipa{tʰɑ˧˥}}}}\kern2pt]} \hypertarget{t\string_hA\string_M\string_T2}{}
\markboth{\textcolor{darkblue}{\textbf{\ipa{tʰɑ˧˥}}} \textsubscript{2}}{}
\textcolor{teal}{\mytextsc{verbe}} \hspace{4pt} Ton~: MH.
\textcolor{PineGreen}{\selectlanguage{french}Être possible, être autorisé: \mytextsc{permissif}.} \zh{可以,允许。}  ¶ \textcolor{darkblue}{\textbf{\ipa{mɤ˧-tʰɑ˧˥ | mɤ˧-ʐv̩˩! | njɤ˧ | dzɯ˧-bi˧ni˧-mɤ˧-gv̩˧˥!}}} \textcolor{PineGreen}{\selectlanguage{french}Ce n'est pas vraiment bon! Je n'aime pas en manger!} \zh{不怎么好吃!我不喜欢吃!}  
 ¶ \textcolor{darkblue}{\textbf{\ipa{[Mushrooms] ə˩ljɤ˩hæ̃˩ʂɯ˥-mo˩-ʈʂʰɯ˩-dʑo˩, | hĩ˧ | mɤ˧-tʰɑ˧˥ | dv̩˩-mɤ˧-kv̩˧˥!}}} \textcolor{PineGreen}{\selectlanguage{french}Le Champignon Doré, il n'est pas si vénéneux que ça!} \zh{黄蜡伞不怎么会让人中毒!/ 毒性不太大!}  

\lhead{\firstmark}
\rhead{\botmark}

\subsection{\hspace{-0.5cm} {\Large \textcolor{darkblue}{\textbf{\ipa{tʰɑ˩˥}}}}\hspace{0.5cm}[\kern2pt{\textcolor{darkblue}{\textbf{\ipa{tʰɑ˩˥}}}}\kern2pt]} \hypertarget{t\string_hA\string_B\string_T1}{}
\markboth{\textcolor{darkblue}{\textbf{\ipa{tʰɑ˩˥}}}}{}
\textcolor{teal}{\mytextsc{nom}} \hspace{4pt} Ton~: LH.
\textcolor{PineGreen}{\selectlanguage{french}Buffle; forme monosyllabique, qui n'est pas utilisée telle quelle, seulement dans des formes telles que \zh{/tʰɑ˩mi\#˥/} 'buffle femelle'.} \zh{水牛。}  \zh{量词}: \textcolor{darkblue}{\textbf{\ipa{pʰo˧˥}}}  \mytextsc{clf}~: \textcolor{darkblue}{\textbf{\ipa{pʰo˧˥}}} 
\lhead{\firstmark}
\rhead{\botmark}

\subsection{\hspace{-0.5cm} {\Large \textcolor{darkblue}{\textbf{\ipa{tʰæ˧ɻæ˩}}}}\hspace{0.5cm}[\kern2pt{\textcolor{darkblue}{\textbf{\ipa{tʰæ˧ɻæ˩}}}}\kern2pt]} \hypertarget{t\string_h\{\string_Mr£`\{\string_B1}{}
\markboth{\textcolor{darkblue}{\textbf{\ipa{tʰæ˧ɻæ˩}}}}{}
\textcolor{teal}{\mytextsc{nom}} \hspace{4pt} Ton~: L\#.
\textcolor{PineGreen}{\selectlanguage{french}Livre.} \zh{书。}  \zh{量词}: \textcolor{darkblue}{\textbf{\ipa{pɤ˩}}}  \mytextsc{clf}~: \textcolor{darkblue}{\textbf{\ipa{pɤ˩}}} 
\lhead{\firstmark}
\rhead{\botmark}

\subsection{\hspace{-0.5cm} {\Large \textcolor{darkblue}{\textbf{\ipa{tʰæ˩tsɯ˧}}}}\hspace{0.5cm}[\kern2pt{\textcolor{darkblue}{\textbf{\ipa{tʰæ˩tsɯ˥}}}}\kern2pt]} \hypertarget{t\string_h\{\string_BtsM\string_M1}{}
\markboth{\textcolor{darkblue}{\textbf{\ipa{tʰæ˩tsɯ˧}}}}{}
\textcolor{teal}{\mytextsc{nom}} \hspace{4pt} Ton~: LM.
\textcolor{PineGreen}{\selectlanguage{french}Jarre.} \zh{坛子(汉语借词)。}  Emprunt~: chinois  \zh{坛子}

\lhead{\firstmark}
\rhead{\botmark}

\subsection{\hspace{-0.5cm} {\Large \textcolor{darkblue}{\textbf{\ipa{tʰi˧‑}}}}\hspace{0.5cm}[\kern2pt{\textcolor{darkblue}{\textbf{\ipa{tʰi˩˥}}}}\kern2pt]} \hypertarget{t\string_hi\string_M‑1}{}
\markboth{\textcolor{darkblue}{\textbf{\ipa{tʰi˧‑}}}}{}
\textcolor{teal}{\mytextsc{préfixe}} \hspace{4pt} Ton~: M/0.
\textcolor{PineGreen}{\selectlanguage{french}Duratif (\mytextsc{dur}).} \zh{持续体。}  ¶ \textcolor{darkblue}{\textbf{\ipa{tʰi˧-dzɯ˥-dʑo˩!}}} \textcolor{PineGreen}{\selectlanguage{french}(Elle) est en train de manger! / Elle mange! (Contexte: on constate avec joie qu'un enfant qui ne mangeait plus depuis deux jours est en train de ronger à belles dents un épi de maïs.)} \zh{她在吃东西!}  
 ¶ \textcolor{darkblue}{\textbf{\ipa{tʰi˧-mɤ˧-ɲi˥}}} \textcolor{PineGreen}{\selectlanguage{french}faute de quoi} \zh{否则、要不然}  

\lhead{\firstmark}
\rhead{\botmark}

\subsection{\hspace{-0.5cm} {\Large \textcolor{darkblue}{\textbf{\ipa{tʰi˧}}}}\hspace{0.5cm}[\kern2pt{\textcolor{darkblue}{\textbf{\ipa{tʰi˥}}}}\kern2pt]} \hypertarget{t\string_hi\string_M1}{}
\markboth{\textcolor{darkblue}{\textbf{\ipa{tʰi˧}}}}{}
\textcolor{teal}{\mytextsc{adjectif}} \hspace{4pt} Ton~: M.
\textcolor{PineGreen}{\selectlanguage{french}Compétent, habile.} \zh{能干。}  ¶ \textcolor{darkblue}{\textbf{\ipa{ɖwæ˧˥ | tʰi˧}}} \textcolor{PineGreen}{\selectlanguage{french}\mytextsc{intensif}.très: très habile} \zh{很能干}  
 ¶ \textcolor{darkblue}{\textbf{\ipa{mv̩˩tʰi˩ tʰv̩˩-v̩˩˥}}} \textcolor{PineGreen}{\selectlanguage{french}cette femme intelligente} \zh{那个聪明女人}  
 ¶ \textcolor{darkblue}{\textbf{\ipa{zo˧tʰi˧}}} \textcolor{PineGreen}{\selectlanguage{french}homme intelligent} \zh{聪明男人}  

\lhead{\firstmark}
\rhead{\botmark}

\subsection{\hspace{-0.5cm} {\Large \textcolor{darkblue}{\textbf{\ipa{tʰi˩\textsubscript{a}}}}}\hspace{0.5cm}[\kern2pt{\textcolor{darkblue}{\textbf{\ipa{tʰi˧˥}}}}\kern2pt]} \hypertarget{t\string_hi\string_Ba1}{}
\markboth{\textcolor{darkblue}{\textbf{\ipa{tʰi˩\textsubscript{a}}}}}{}
\textcolor{teal}{\mytextsc{verbe}} \hspace{4pt} Ton~: L\textsubscript{a}.
\textcolor{PineGreen}{\selectlanguage{french}Raboter.} \zh{刨。}  ¶ \textcolor{darkblue}{\textbf{\ipa{tso˧\textasciitilde{}tso˧ tʰi˥(-ze˩)}}} \textcolor{PineGreen}{\selectlanguage{french}raboter quelque chose} \zh{刨东西}  
 ¶ \textcolor{darkblue}{\textbf{\ipa{le˧-tʰi˩-ze˩}}} \textcolor{PineGreen}{\selectlanguage{french}\mytextsc{accomp} \string_ \mytextsc{pfv}} \zh{刨了}  
 ¶ \textcolor{darkblue}{\textbf{\ipa{tso˧\textasciitilde{}tso˧ | le˧-tʰi˩(-ze˩)}}} \textcolor{PineGreen}{\selectlanguage{french}raboter quelque chose} \zh{刨东西}  
 ¶ \textcolor{darkblue}{\textbf{\ipa{pæ˩pʰæ˧ tʰi˥}}} \textcolor{PineGreen}{\selectlanguage{french}raboter une planche} \zh{刨木板}  

\lhead{\firstmark}
\rhead{\botmark}

\subsection{\hspace{-0.5cm} {\Large \textcolor{darkblue}{\textbf{\ipa{tʰi˩mi\#˥}}}}\hspace{0.5cm}[\kern2pt{\textcolor{darkblue}{\textbf{\ipa{tʰi˧mi˥}}}}\kern2pt]} \hypertarget{t\string_hi\string_Bmi\#\string_T1}{}
\markboth{\textcolor{darkblue}{\textbf{\ipa{tʰi˩mi\#˥}}}}{}
\textcolor{teal}{\mytextsc{nom}} \hspace{4pt} Ton~: LM+\#H.
\textcolor{PineGreen}{\selectlanguage{french}Grand rabot.} \zh{大刨。}  \zh{量词}: \textcolor{darkblue}{\textbf{\ipa{nɑ˧}}}  \mytextsc{clf}~: \textcolor{darkblue}{\textbf{\ipa{nɑ˧}}} 
\lhead{\firstmark}
\rhead{\botmark}

\subsection{\hspace{-0.5cm} {\Large \textcolor{darkblue}{\textbf{\ipa{tʰi˩zo\#˥}}}}\hspace{0.5cm}[\kern2pt{\textcolor{darkblue}{\textbf{\ipa{tʰi˧zo˩}}}}\kern2pt]} \hypertarget{t\string_hi\string_Bzo\#\string_T1}{}
\markboth{\textcolor{darkblue}{\textbf{\ipa{tʰi˩zo\#˥}}}}{}
\textcolor{teal}{\mytextsc{nom}} \hspace{4pt} Ton~: LM+\#H.
\textcolor{PineGreen}{\selectlanguage{french}Petit rabot.} \zh{小刨。}  \zh{量词}: \textcolor{darkblue}{\textbf{\ipa{nɑ˧}}}  \mytextsc{clf}~: \textcolor{darkblue}{\textbf{\ipa{nɑ˧}}} 
\lhead{\firstmark}
\rhead{\botmark}

\subsection{\hspace{-0.5cm} {\Large \textcolor{darkblue}{\textbf{\ipa{tʰi˩˥}}} \textsubscript{1}}\hspace{0.5cm}[\kern2pt{\textcolor{darkblue}{\textbf{\ipa{tʰi˩˥}}}}\kern2pt]} \hypertarget{t\string_hi\string_B\string_T1}{}
\markboth{\textcolor{darkblue}{\textbf{\ipa{tʰi˩˥}}} \textsubscript{1}}{}
\textcolor{teal}{\mytextsc{nom}} \hspace{4pt} Ton~: LH.
\textcolor{PineGreen}{\selectlanguage{french}Rabot.} \zh{刨。}  \zh{量词}: \textcolor{darkblue}{\textbf{\ipa{nɑ˧}}}  \mytextsc{clf}~: \textcolor{darkblue}{\textbf{\ipa{nɑ˧}}} 
\lhead{\firstmark}
\rhead{\botmark}

\subsection{\hspace{-0.5cm} {\Large \textcolor{darkblue}{\textbf{\ipa{tʰi˩˥}}} \textsubscript{2}}\hspace{0.5cm}[\kern2pt{\textcolor{darkblue}{\textbf{\ipa{tʰi˩˥}}}}\kern2pt]} \hypertarget{t\string_hi\string_B\string_T2}{}
\markboth{\textcolor{darkblue}{\textbf{\ipa{tʰi˩˥}}} \textsubscript{2}}{}
\textcolor{teal}{\mytextsc{particule}} \textcolor{teal}{\mytextsc{de}} \textcolor{teal}{\mytextsc{discours}} \hspace{4pt} Ton~: LM? LH?.
\textcolor{PineGreen}{\selectlanguage{french}Particule de discours: alors, donc, après.} \zh{然后。} 
\lhead{\firstmark}
\rhead{\botmark}

\subsection{\hspace{-0.5cm} {\Large \textcolor{darkblue}{\textbf{\ipa{tʰo˥\textsubscript{a}}}}}\hspace{0.5cm}[\kern2pt{\textcolor{darkblue}{\textbf{\ipa{tʰo˥}}}}\kern2pt]} \hypertarget{t\string_ho\string_Ta1}{}
\markboth{\textcolor{darkblue}{\textbf{\ipa{tʰo˥\textsubscript{a}}}}}{}
\textcolor{teal}{\mytextsc{classificateur}} \hspace{4pt} Ton~: H\textsubscript{a}.
\textcolor{PineGreen}{\selectlanguage{french}Classificateur des solutions / issues heureuses.} \zh{量词:办法,解决的方法(一个)。}  ¶ \textcolor{darkblue}{\textbf{\ipa{ə˧tso˧ tʰo˧ dʑo˧-kv̩˩?}}} \textcolor{PineGreen}{\selectlanguage{french}Qu'est-ce qu'on y peut? / Qu'est-ce qu'on peut y faire?} \zh{有什么办法?}  

\lhead{\firstmark}
\rhead{\botmark}

\subsection{\hspace{-0.5cm} {\Large \textcolor{darkblue}{\textbf{\ipa{tʰo˥\textsubscript{a}}}}}\hspace{0.5cm}[\kern2pt{\textcolor{darkblue}{\textbf{\ipa{tʰo˩˥}}}}\kern2pt]} \hypertarget{t\string_ho\string_Ta1}{}
\markboth{\textcolor{darkblue}{\textbf{\ipa{tʰo˥\textsubscript{a}}}}}{}
\textcolor{teal}{\mytextsc{classificateur}} \hspace{4pt} Ton~: H\textsubscript{a}.
\textcolor{PineGreen}{\selectlanguage{french}Classificateur des ensembles, des lots.} \zh{量词:套(汉语借词)。}  Emprunt~: chinois  \zh{套}

\lhead{\firstmark}
\rhead{\botmark}

\subsection{\hspace{-0.5cm} {\Large \textcolor{darkblue}{\textbf{\ipa{tʰo˧ɕi˩}}}}\hspace{0.5cm}[\kern2pt{\textcolor{darkblue}{\textbf{\ipa{tʰo˧ɕi˧}}}}\kern2pt]} \hypertarget{t\string_ho\string_Ms£i\string_B1}{}
\markboth{\textcolor{darkblue}{\textbf{\ipa{tʰo˧ɕi˩}}}}{}
\textcolor{teal}{\mytextsc{nom}} \hspace{4pt} Ton~: L\#.
\textcolor{PineGreen}{\selectlanguage{french}Messager.} \zh{通信员(汉语借词)。}  Emprunt~: chinois  \zh{通信}
 \zh{量词}: \textcolor{darkblue}{\textbf{\ipa{v˧}}}  \mytextsc{clf}~: \textcolor{darkblue}{\textbf{\ipa{v˧}}} 
\lhead{\firstmark}
\rhead{\botmark}

\subsection{\hspace{-0.5cm} {\Large \textcolor{darkblue}{\textbf{\ipa{tʰo˧ɕi˧˥}}}}\hspace{0.5cm}[\kern2pt{\textcolor{darkblue}{\textbf{\ipa{tʰo˧ɕi˩}}}}\kern2pt]} \hypertarget{t\string_ho\string_Ms£i\string_M\string_T1}{}
\markboth{\textcolor{darkblue}{\textbf{\ipa{tʰo˧ɕi˧˥}}}}{}
\textcolor{teal}{\mytextsc{nom}} \hspace{4pt} Ton~: MH\#.
\textcolor{PineGreen}{\selectlanguage{french}Forêt de conifères.} \zh{松树林。}  \zh{量词}: \textcolor{darkblue}{\textbf{\ipa{pʰæ˧˥}}}  \mytextsc{clf}~: \textcolor{darkblue}{\textbf{\ipa{pʰæ˧˥}}} 
\lhead{\firstmark}
\rhead{\botmark}

\subsection{\hspace{-0.5cm} {\Large \textcolor{darkblue}{\textbf{\ipa{tʰo˧dzi˩}}}}\hspace{0.5cm}[\kern2pt{\textcolor{darkblue}{\textbf{\ipa{tʰo˧dzi˧˥}}}}\kern2pt]} \hypertarget{t\string_ho\string_Mdzi\string_B1}{}
\markboth{\textcolor{darkblue}{\textbf{\ipa{tʰo˧dzi˩}}}}{}
\textcolor{teal}{\mytextsc{nom}} \hspace{4pt} Ton~: L\#.
\textcolor{PineGreen}{\selectlanguage{french}Pin.} \zh{松树。}  \zh{量词}: \textcolor{darkblue}{\textbf{\ipa{dzi˩}}}  \mytextsc{clf}~: \textcolor{darkblue}{\textbf{\ipa{dzi˩}}} 
\lhead{\firstmark}
\rhead{\botmark}

\subsection{\hspace{-0.5cm} {\Large \textcolor{darkblue}{\textbf{\ipa{tʰo˧dzi˩-hwæ˩tsɯ˩}}}}\hspace{0.5cm}[\kern2pt{\textcolor{darkblue}{\textbf{\ipa{xxxx non-correspondance entre le nombre de morphèmes et le nombre de tons de morphèmes}}}}\kern2pt]} \hypertarget{t\string_ho\string_Mdzi\string_B-hw\{\string_BtsM\string_B1}{}
\markboth{\textcolor{darkblue}{\textbf{\ipa{tʰo˧dzi˩-hwæ˩tsɯ˩}}}}{}
\textcolor{teal}{\mytextsc{nom}} \hspace{4pt} Ton~: LM-.
\textcolor{PineGreen}{\selectlanguage{french}Hérisson; littéralement “souris des pins”.} \zh{刺猬。} 
\lhead{\firstmark}
\rhead{\botmark}

\subsection{\hspace{-0.5cm} {\Large \textcolor{darkblue}{\textbf{\ipa{tʰo˧fv̩˧}}}}\hspace{0.5cm}[\kern2pt{\textcolor{darkblue}{\textbf{\ipa{xxxx non-correspondance entre le nombre de morphèmes et le nombre de tons de morphèmes}}}}\kern2pt]} \hypertarget{t\string_ho\string_Mfv\string_=\string_M1}{}
\markboth{\textcolor{darkblue}{\textbf{\ipa{tʰo˧fv̩˧}}}}{}
\textcolor{teal}{\mytextsc{nom}} \hspace{4pt} Ton~: M.
\textcolor{PineGreen}{\selectlanguage{french}Bandit, maraudeur.} \zh{土匪(汉语借词)。}  Emprunt~: chinois  \zh{土匪}

\lhead{\firstmark}
\rhead{\botmark}

\subsection{\hspace{-0.5cm} {\Large \textcolor{darkblue}{\textbf{\ipa{tʰo˧lɑ˧tɕi˧}}}}\hspace{0.5cm}[\kern2pt{\textcolor{darkblue}{\textbf{\ipa{tʰo˧lɑ˧tɕi˧}}}}\kern2pt]} \hypertarget{t\string_ho\string_MlA\string_Mts£i\string_M1}{}
\markboth{\textcolor{darkblue}{\textbf{\ipa{tʰo˧lɑ˧tɕi˧}}}}{}
\textcolor{teal}{\mytextsc{nom}} \hspace{4pt} Ton~: M.
\textcolor{PineGreen}{\selectlanguage{french}Tracteur.} \zh{拖拉机(汉语借词)。}  Emprunt~: chinois  \zh{洋火}
 ¶ \textcolor{darkblue}{\textbf{\ipa{bo˩mi˧-tʰo˧lɑ˧tɕi˧}}} \textcolor{PineGreen}{\selectlanguage{french}'tracteur-truie': petit tracteur (le premier modèle introduit à Yongning)} \zh{‘母猪拖拉机’:小型拖拉机}  
 \zh{量词}: \textcolor{darkblue}{\textbf{\ipa{yyyy}}}  \mytextsc{clf}~: \textcolor{darkblue}{\textbf{\ipa{yyyy}}} 
\lhead{\firstmark}
\rhead{\botmark}

\subsection{\hspace{-0.5cm} {\Large \textcolor{darkblue}{\textbf{\ipa{tʰo˧li˧}}}}\hspace{0.5cm}[\kern2pt{\textcolor{darkblue}{\textbf{\ipa{tʰo˧li˧}}}}\kern2pt]} \hypertarget{t\string_ho\string_Mli\string_M1}{}
\markboth{\textcolor{darkblue}{\textbf{\ipa{tʰo˧li˧}}}}{}
\textcolor{teal}{\mytextsc{nom}} \hspace{4pt} Ton~: M.
\textcolor{PineGreen}{\selectlanguage{french}Lapin.} \zh{兔子。}  \zh{量词}: \textcolor{darkblue}{\textbf{\ipa{mi˩}}}  \mytextsc{clf}~: \textcolor{darkblue}{\textbf{\ipa{mi˩}}} 
\lhead{\firstmark}
\rhead{\botmark}

\subsection{\hspace{-0.5cm} {\Large \textcolor{darkblue}{\textbf{\ipa{tʰo˧li˧kʰv̩˧˥}}}}\hspace{0.5cm}[\kern2pt{\textcolor{darkblue}{\textbf{\ipa{tʰo˧li˧kʰv̩˧˥}}}}\kern2pt]} \hypertarget{t\string_ho\string_Mli\string_Mk\string_hv\string_=\string_M\string_T1}{}
\markboth{\textcolor{darkblue}{\textbf{\ipa{tʰo˧li˧kʰv̩˧˥}}}}{}
\textcolor{teal}{\mytextsc{nom}} \hspace{4pt} Ton~: MH\#.
\textcolor{PineGreen}{\selectlanguage{french}Année du Lapin.} \zh{兔年。} 
\lhead{\firstmark}
\rhead{\botmark}

\subsection{\hspace{-0.5cm} {\Large \textcolor{darkblue}{\textbf{\ipa{tʰo˧li˧-pʰv̩\#˥}}}}\hspace{0.5cm}[\kern2pt{\textcolor{darkblue}{\textbf{\ipa{xxxx non-correspondance entre le nombre de morphèmes et le nombre de tons de morphèmes}}}}\kern2pt]} \hypertarget{t\string_ho\string_Mli\string_M-p\string_hv\string_=\#\string_T1}{}
\markboth{\textcolor{darkblue}{\textbf{\ipa{tʰo˧li˧-pʰv̩\#˥}}}}{}
\textcolor{teal}{\mytextsc{nom}} \hspace{4pt} Ton~: \#H.
\textcolor{PineGreen}{\selectlanguage{french}Lapin mâle.} \zh{公兔。}  \zh{量词}: \textcolor{darkblue}{\textbf{\ipa{mi˩}}}  \mytextsc{clf}~: \textcolor{darkblue}{\textbf{\ipa{mi˩}}} 
\lhead{\firstmark}
\rhead{\botmark}

\subsection{\hspace{-0.5cm} {\Large \textcolor{darkblue}{\textbf{\ipa{tʰo˧li˧-zo\#˥}}}}\hspace{0.5cm}[\kern2pt{\textcolor{darkblue}{\textbf{\ipa{xxxx non-correspondance entre le nombre de morphèmes et le nombre de tons de morphèmes}}}}\kern2pt]} \hypertarget{t\string_ho\string_Mli\string_M-zo\#\string_T1}{}
\markboth{\textcolor{darkblue}{\textbf{\ipa{tʰo˧li˧-zo\#˥}}}}{}
\textcolor{teal}{\mytextsc{nom}} \hspace{4pt} Ton~: \#H.
\textcolor{PineGreen}{\selectlanguage{french}Petit lapin, bébé lapin.} \zh{小兔。}  \zh{量词}: \textcolor{darkblue}{\textbf{\ipa{ɭɯ˧}}}  \mytextsc{clf}~: \textcolor{darkblue}{\textbf{\ipa{ɭɯ˧}}} 
\lhead{\firstmark}
\rhead{\botmark}

\subsection{\hspace{-0.5cm} {\Large \textcolor{darkblue}{\textbf{\ipa{tʰo˧-mo˩}}}}\hspace{0.5cm}[\kern2pt{\textcolor{darkblue}{\textbf{\ipa{xxxx non-correspondance entre le nombre de morphèmes et le nombre de tons de morphèmes}}}}\kern2pt]} \hypertarget{t\string_ho\string_M-mo\string_B1}{}
\markboth{\textcolor{darkblue}{\textbf{\ipa{tʰo˧-mo˩}}}}{}
\textcolor{teal}{\mytextsc{nom}} \hspace{4pt} Ton~: L\#.
\textcolor{PineGreen}{\selectlanguage{french}“champignon des sapins”: champignon comestible, ainsi nommé parce qu'il pousse au pied des sapins.} \zh{“松树菌”:一种菌子。} 
\lhead{\firstmark}
\rhead{\botmark}

\subsection{\hspace{-0.5cm} {\Large \textcolor{darkblue}{\textbf{\ipa{tʰo˧ɻæ˥}}}}\hspace{0.5cm}[\kern2pt{\textcolor{darkblue}{\textbf{\ipa{tʰo˧ɻæ˥}}}}\kern2pt]} \hypertarget{t\string_ho\string_Mr£`\{\string_T1}{}
\markboth{\textcolor{darkblue}{\textbf{\ipa{tʰo˧ɻæ˥}}}}{}
\textcolor{teal}{\mytextsc{nom}} \hspace{4pt} Ton~: H\#.
\textcolor{PineGreen}{\selectlanguage{french}Pignon de pin (graine comestible).} \zh{松子。}  \zh{量词}: \textcolor{darkblue}{\textbf{\ipa{ʈʂwɤ˧}}}  \mytextsc{clf}~: \textcolor{darkblue}{\textbf{\ipa{ʈʂwɤ˧}}} \textcolor{PineGreen}{\selectlanguage{french}poignée}
\lhead{\firstmark}
\rhead{\botmark}

\subsection{\hspace{-0.5cm} {\Large \textcolor{darkblue}{\textbf{\ipa{tʰo˧tsʰe˧-ʁwɤ\#˥}}}}\hspace{0.5cm}[\kern2pt{\textcolor{darkblue}{\textbf{\ipa{xxxx non-correspondance entre le nombre de morphèmes et le nombre de tons de morphèmes}}}}\kern2pt]} \hypertarget{t\string_ho\string_Mts\string_he\string_M-Rw7\#\string_T1}{}
\markboth{\textcolor{darkblue}{\textbf{\ipa{tʰo˧tsʰe˧-ʁwɤ\#˥}}}}{}
\textcolor{teal}{\mytextsc{nom}} \hspace{4pt} Ton~: \#H.
\textcolor{PineGreen}{\selectlanguage{french}Un village proche des Sources Chaudes.} \zh{温泉乡的一个村落。}  ¶ \textcolor{darkblue}{\textbf{\ipa{tʰo˧tsʰe\#˥}}} \textcolor{PineGreen}{\selectlanguage{french}même sens} \zh{同上}  
 ¶ \textcolor{darkblue}{\textbf{\ipa{ə˧go˧-ʁwɤ˧, | ʁwɤ˧lɑ˩-bi˩, | bæ˧ʁwɤ˧, | tʰo˧tsʰe\#˥, | pi˧tsʰe˩-di˩, | pɤ˧dʑɤ˩-di˩, | ʁwɤ˧tv̩˧}}} \textcolor{PineGreen}{\selectlanguage{french}Villages au sortir de la plaine de Yongning; les deux premiers comportent une population na; le troisième est un village na; les suivants sont essentiellement des villages pumi/prinmi.} \zh{永宁背向泸沽湖方向经过的村落。前两个村落拥有相当大的摩梭人口比例,第三个村落是摩梭村,最后一个是普米村。}  
 ¶ \textcolor{darkblue}{\textbf{\ipa{tʰo˧tsʰe˧: | bɤ˧!}}} \textcolor{PineGreen}{\selectlanguage{french}\textcolor{darkblue}{\textbf{\ipa{/tʰo˧tsʰe˧/}}}, c'est un village pumi!} \zh{fv:/tʰo˧tsʰe˧/是一个普米族村落!}  

\lhead{\firstmark}
\rhead{\botmark}

\subsection{\hspace{-0.5cm} {\Large \textcolor{darkblue}{\textbf{\ipa{tʰo˧ʈɯ\#˥}}}}\hspace{0.5cm}[\kern2pt{\textcolor{darkblue}{\textbf{\ipa{tʰo˩ʈɯ˩˥}}}}\kern2pt]} \hypertarget{t\string_ho\string_Mt`M\#\string_T1}{}
\markboth{\textcolor{darkblue}{\textbf{\ipa{tʰo˧ʈɯ\#˥}}}}{}
\textcolor{teal}{\mytextsc{nom}} \hspace{4pt} Ton~: \#H.
\textcolor{PineGreen}{\selectlanguage{french}Un village de Yongning: Tuozhikaiji.} \zh{拖支开基村(永宁的一个村落)。}  ¶ \textcolor{darkblue}{\textbf{\ipa{dʑɤ˩bv̩˧kɤ˧-sɑ˥ʁwɤ˩, | hi˩ʁwɤ˩-lo˥, | æ˩mi˧-ʁwɤ\#˥, | lɑ˧lo˧-ʁwɤ˥, | lɑ˧ŋwɤ˧, | bɤ˧tsʰo˧gv̩˥, | ə˧lɑ˧-ʁwɤ\#˥, | gæ˧ɻæ˩, | qʰæ˧tɕʰi˧, | tʰo˧ʈɯ\#˥}}} \textcolor{PineGreen}{\selectlanguage{french}les dix villages comptant traditionnellement comme faisant partie de Yongning} \zh{摩梭传统地理概念中,属于永宁的十个村落}  

\lhead{\firstmark}
\rhead{\botmark}

\subsection{\hspace{-0.5cm} {\Large \textcolor{darkblue}{\textbf{\ipa{tʰo˧ʐv̩˥}}}}\hspace{0.5cm}[\kern2pt{\textcolor{darkblue}{\textbf{\ipa{tʰo˩ʐv̩˩˥}}}}\kern2pt]} \hypertarget{t\string_ho\string_Mz`v\string_=\string_T1}{}
\markboth{\textcolor{darkblue}{\textbf{\ipa{tʰo˧ʐv̩˥}}}}{}
\textcolor{teal}{\mytextsc{nom}} \hspace{4pt} Ton~: H\#.
\textcolor{PineGreen}{\selectlanguage{french}Pigeon.} \zh{鸽子。}  ¶ \textcolor{darkblue}{\textbf{\ipa{tʰo˧ʐv̩˥-mi˩}}} \textcolor{PineGreen}{\selectlanguage{french}pigeon femelle} \zh{母鸽子}  
 ¶ \textcolor{darkblue}{\textbf{\ipa{tʰo˧ʐv̩˥-pʰv̩˩}}} \textcolor{PineGreen}{\selectlanguage{french}pigeon mâle} \zh{公鸽子}  
 ¶ \textcolor{darkblue}{\textbf{\ipa{tʰo˧ʐv̩˥-zo˩}}} \textcolor{PineGreen}{\selectlanguage{french}petit pigeon} \zh{小鸽子}  
 \zh{量词}: \textcolor{darkblue}{\textbf{\ipa{mi˩}}}  \mytextsc{clf}~: \textcolor{darkblue}{\textbf{\ipa{mi˩}}} 
\lhead{\firstmark}
\rhead{\botmark}

\subsection{\hspace{-0.5cm} {\Large \textcolor{darkblue}{\textbf{\ipa{tʰo˩\textsubscript{a}}}}}\hspace{0.5cm}[\kern2pt{\textcolor{darkblue}{\textbf{\ipa{tʰo˥}}}}\kern2pt]} \hypertarget{t\string_ho\string_Ba1}{}
\markboth{\textcolor{darkblue}{\textbf{\ipa{tʰo˩\textsubscript{a}}}}}{}
\textcolor{teal}{\mytextsc{verbe}} \hspace{4pt} Ton~: L\textsubscript{a}.
\textcolor{PineGreen}{\selectlanguage{french}S’adosser à, s'appuyer.} \zh{靠。}  ¶ \textcolor{darkblue}{\textbf{\ipa{tʰi˧-tʰo˩}}} \textcolor{PineGreen}{\selectlanguage{french}\mytextsc{dur}} \zh{\mytextsc{dur}}  
 ¶ \textcolor{darkblue}{\textbf{\ipa{tʰi˧-tʰo˩-ɻ̍˩}}} \textcolor{PineGreen}{\selectlanguage{french}\mytextsc{dur} \string_ \mytextsc{inchoatif}} \zh{\mytextsc{dur} \string_ \mytextsc{inceptive}}  
 ¶ \textcolor{darkblue}{\textbf{\ipa{ɖɯ˧-tʰo˩-ɻ̍˩}}} \textcolor{PineGreen}{\selectlanguage{french}\mytextsc{délimitatif} \string_ \mytextsc{inchoatif}} \zh{\mytextsc{delimitative} \string_ \mytextsc{inceptive}}  

\lhead{\firstmark}
\rhead{\botmark}

\subsection{\hspace{-0.5cm} {\Large \textcolor{darkblue}{\textbf{\ipa{tʰo˩lo˧}}}}\hspace{0.5cm}[\kern2pt{\textcolor{darkblue}{\textbf{\ipa{tʰo˩lo˥}}}}\kern2pt]} \hypertarget{t\string_ho\string_Blo\string_M1}{}
\markboth{\textcolor{darkblue}{\textbf{\ipa{tʰo˩lo˧}}}}{}
\textcolor{teal}{\mytextsc{nom}} \hspace{4pt} Ton~: LM.
\textcolor{PineGreen}{\selectlanguage{french}Cheval de tête, dans une caravane.} \zh{头马:马帮里走在最前面的那匹马。} 
\lhead{\firstmark}
\rhead{\botmark}

\subsection{\hspace{-0.5cm} {\Large \textcolor{darkblue}{\textbf{\ipa{tʰo˩pʰv̩˧tɕʰɤ˧}}}}\hspace{0.5cm}[\kern2pt{\textcolor{darkblue}{\textbf{\ipa{tʰo˩pʰv̩˧tɕʰɤ˧}}}}\kern2pt]} \hypertarget{t\string_ho\string_Bp\string_hv\string_=\string_Mts£\string_h7\string_M1}{}
\markboth{\textcolor{darkblue}{\textbf{\ipa{tʰo˩pʰv̩˧tɕʰɤ˧}}}}{}
\textcolor{teal}{\mytextsc{nom}} \hspace{4pt} Ton~: LM.
\textcolor{PineGreen}{\selectlanguage{french}Arme à feu, fusil; arquebuse.} \zh{枪,明火枪。}  Emprunt~: chinois?
 \zh{量词}: \textcolor{darkblue}{\textbf{\ipa{kʰɯ˩}}}  \mytextsc{clf}~: \textcolor{darkblue}{\textbf{\ipa{kʰɯ˩}}} 
\lhead{\firstmark}
\rhead{\botmark}

\subsection{\hspace{-0.5cm} {\Large \textcolor{darkblue}{\textbf{\ipa{tʰo˩ʁæ˩}}}}\hspace{0.5cm}[\kern2pt{\textcolor{darkblue}{\textbf{\ipa{tʰo˩ʁæ˩˥}}}}\kern2pt]} \hypertarget{t\string_ho\string_BR\{\string_B1}{}
\markboth{\textcolor{darkblue}{\textbf{\ipa{tʰo˩ʁæ˩}}}}{}
\textcolor{teal}{\mytextsc{nom}} \hspace{4pt} Ton~: L.
\textcolor{PineGreen}{\selectlanguage{french}Résine de pin.} \zh{松香。}  \zh{量词}: \textcolor{darkblue}{\textbf{\ipa{ʈʰɤ˥}}}  \mytextsc{clf}~: \textcolor{darkblue}{\textbf{\ipa{ʈʰɤ˥}}} 
\lhead{\firstmark}
\rhead{\botmark}

\subsection{\hspace{-0.5cm} {\Large \textcolor{darkblue}{\textbf{\ipa{tʰo˩ʂv̩˩}}}}\hspace{0.5cm}[\kern2pt{\textcolor{darkblue}{\textbf{\ipa{tʰo˩ʂv̩˩˥}}}}\kern2pt]} \hypertarget{t\string_ho\string_Bs`v\string_=\string_B1}{}
\markboth{\textcolor{darkblue}{\textbf{\ipa{tʰo˩ʂv̩˩}}}}{}
\textcolor{teal}{\mytextsc{nom}} \hspace{4pt} Ton~: L.
\textcolor{PineGreen}{\selectlanguage{french}Aiguilles de pin.} \zh{树针。}  \zh{量词}: \textcolor{darkblue}{\textbf{\ipa{qɑ˩}}}  \mytextsc{clf}~: \textcolor{darkblue}{\textbf{\ipa{qɑ˩}}} \textcolor{PineGreen}{\selectlanguage{french}botte}
\lhead{\firstmark}
\rhead{\botmark}

\subsection{\hspace{-0.5cm} {\Large \textcolor{darkblue}{\textbf{\ipa{tʰo˩tɕi˧˥}}}}\hspace{0.5cm}[\kern2pt{\textcolor{darkblue}{\textbf{\ipa{tʰo˩tɕi˧˥}}}}\kern2pt]} \hypertarget{t\string_ho\string_Bts£i\string_M\string_T1}{}
\markboth{\textcolor{darkblue}{\textbf{\ipa{tʰo˩tɕi˧˥}}}}{}
\textcolor{teal}{\mytextsc{nom}} \hspace{4pt} Ton~: LM+MH\#.
\textcolor{PineGreen}{\selectlanguage{french}Brique à l'ancienne: brique crue.} \zh{砖。}  \zh{量词}: \textcolor{darkblue}{\textbf{\ipa{ɭɯ˧}}}  \mytextsc{clf}~: \textcolor{darkblue}{\textbf{\ipa{ɭɯ˧}}} 
\lhead{\firstmark}
\rhead{\botmark}

\subsection{\hspace{-0.5cm} {\Large \textcolor{darkblue}{\textbf{\ipa{tʰv̩˧˥}}} \textsubscript{1}}\hspace{0.5cm}[\kern2pt{\textcolor{darkblue}{\textbf{\ipa{tʰv̩˧˥}}}}\kern2pt]} \hypertarget{t\string_hv\string_=\string_M\string_T1}{}
\markboth{\textcolor{darkblue}{\textbf{\ipa{tʰv̩˧˥}}} \textsubscript{1}}{}
\textcolor{teal}{\mytextsc{verbe}} \hspace{4pt} Ton~: MH.
\textcolor{PineGreen}{\selectlanguage{french}Fouler du pied, marcher sur, écraser.} \zh{踩。}  ¶ \textcolor{darkblue}{\textbf{\ipa{tʰv̩˩\textasciitilde{}tʰv̩˧˥}}} \textcolor{PineGreen}{\selectlanguage{french}\mytextsc{red}} \zh{\mytextsc{重叠}}  
 ¶ \textcolor{darkblue}{\textbf{\ipa{ɖɯ˧-tʰv̩˧ tʰi˥-tʰv̩˩}}} \textcolor{PineGreen}{\selectlanguage{french}donner un coup de pied par terre, fouler le sol du pied} \zh{踢一脚}  
 ¶ \textcolor{darkblue}{\textbf{\ipa{kʰɯ˧tsʰɤ˧ tʰv̩˥-tsʰɯ˩}}} \textcolor{PineGreen}{\selectlanguage{french}donner un coup de pied par terre, fouler le sol du pied} \zh{踢一脚}  
 ¶ \textcolor{darkblue}{\textbf{\ipa{kʰɯ˧tsʰɤ˧ tʰv̩˥\textasciitilde{}tʰv̩˩}}} \textcolor{PineGreen}{\selectlanguage{french}donner un coup de pied par terre, fouler le sol du pied} \zh{踢一脚}  
 ¶ \textcolor{darkblue}{\textbf{\ipa{kʰɯ˧tsʰɤ˧ tʰɑ˧-tʰv̩˧˥!}}} \textcolor{PineGreen}{\selectlanguage{french}Ne donne pas de coup de pied!} \zh{别踢!}  

\lhead{\firstmark}
\rhead{\botmark}

\subsection{\hspace{-0.5cm} {\Large \textcolor{darkblue}{\textbf{\ipa{tʰv̩˧˥}}} \textsubscript{2}}\hspace{0.5cm}[\kern2pt{\textcolor{darkblue}{\textbf{\ipa{tʰv̩˧˥}}}}\kern2pt]} \hypertarget{t\string_hv\string_=\string_M\string_T2}{}
\markboth{\textcolor{darkblue}{\textbf{\ipa{tʰv̩˧˥}}} \textsubscript{2}}{}
\textcolor{teal}{\mytextsc{verbe}} \hspace{4pt} Ton~: MH.
\textcolor{PineGreen}{\selectlanguage{french}Se charger de, préparer, offrir (quelqu'un se charge d'offrir un repas aux gens du village; c'est lui qui paie, pas forcément qui fait la cuisine).} \zh{负担(某个活动的费用,如:请全村人吃饭)。} 
\lhead{\firstmark}
\rhead{\botmark}

\subsection{\hspace{-0.5cm} {\Large \textcolor{darkblue}{\textbf{\ipa{tʰv̩˥}}} \textsubscript{1}}\hspace{0.5cm}[\kern2pt{\textcolor{darkblue}{\textbf{\ipa{tʰv̩˧˥}}}}\kern2pt]} \hypertarget{t\string_hv\string_=\string_T1}{}
\markboth{\textcolor{darkblue}{\textbf{\ipa{tʰv̩˥}}} \textsubscript{1}}{}
\textcolor{teal}{\mytextsc{pronom}} \hspace{4pt} Ton~: \#H.
\textcolor{PineGreen}{\selectlanguage{french}Démonstratif distal, qui forme un couple avec le démonstratif proximal.} \zh{那\mytextsc{指示}.远指。}  ¶ \textcolor{darkblue}{\textbf{\ipa{tʰv̩˧ ɲi˥!}}} \textcolor{PineGreen}{\selectlanguage{french}c'est celui-là!} \zh{是那个!}  
 ¶ \textcolor{darkblue}{\textbf{\ipa{tʰv̩˧-v̩\#˥}}} \textcolor{PineGreen}{\selectlanguage{french}celui-là (\mytextsc{dem}.dist-\mytextsc{clf}.individu)} \zh{那个}  
\textit{Voir~:} \hyperlink{}{\textcolor{darkblue}{\textbf{\ipa{tʰv̩˥}}} \textsubscript{2}} \textit{Voir~:} \hyperlink{}{\textcolor{darkblue}{\textbf{\ipa{tʰv̩˥}}} \textsubscript{3}} 
\lhead{\firstmark}
\rhead{\botmark}

\subsection{\hspace{-0.5cm} {\Large \textcolor{darkblue}{\textbf{\ipa{tʰv̩˥}}} \textsubscript{2}}\hspace{0.5cm}[\kern2pt{\textcolor{darkblue}{\textbf{\ipa{tʰv̩˥}}}}\kern2pt]} \hypertarget{t\string_hv\string_=\string_T2}{}
\markboth{\textcolor{darkblue}{\textbf{\ipa{tʰv̩˥}}} \textsubscript{2}}{}
\textcolor{teal}{\mytextsc{pronom}} \hspace{4pt} Ton~: \#H.
\textcolor{PineGreen}{\selectlanguage{french}Pronom de troisième personne du singulier; provient du démonstratif distal.} \zh{他。}  ¶ \textcolor{darkblue}{\textbf{\ipa{tʰv̩˧=ɻ̍˩}}} \textcolor{PineGreen}{\selectlanguage{french}sa famille, sa maisonnée, les siens} \zh{他家、他家族、他的人}  
\textit{Voir~:} \hyperlink{}{\textcolor{darkblue}{\textbf{\ipa{tʰv̩˥}}} \textsubscript{1}} \textit{Voir~:} \hyperlink{}{\textcolor{darkblue}{\textbf{\ipa{tʰv̩˥}}} \textsubscript{3}} 
\lhead{\firstmark}
\rhead{\botmark}

\subsection{\hspace{-0.5cm} {\Large \textcolor{darkblue}{\textbf{\ipa{tʰv̩˥}}} \textsubscript{3}}\hspace{0.5cm}[\kern2pt{\textcolor{darkblue}{\textbf{\ipa{tʰv̩˥}}}}\kern2pt]} \hypertarget{t\string_hv\string_=\string_T3}{}
\markboth{\textcolor{darkblue}{\textbf{\ipa{tʰv̩˥}}} \textsubscript{3}}{}
\textcolor{teal}{\mytextsc{suffixe}} \hspace{4pt} Ton~: \#H.
\textcolor{PineGreen}{\selectlanguage{french}Focalisateur; grammaticalisé à partir du démonstratif distal.} \zh{\mytextsc{主题(°指示}.远指)。} \textit{Voir~:} \hyperlink{}{\textcolor{darkblue}{\textbf{\ipa{tʰv̩˥}}} \textsubscript{1}} \textit{Voir~:} \hyperlink{}{\textcolor{darkblue}{\textbf{\ipa{tʰv̩˥}}} \textsubscript{2}} 
\lhead{\firstmark}
\rhead{\botmark}

\subsection{\hspace{-0.5cm} {\Large \textcolor{darkblue}{\textbf{\ipa{tʰv̩˧\textsubscript{a}}}}}\hspace{0.5cm}[\kern2pt{\textcolor{darkblue}{\textbf{\ipa{tʰv̩˥}}}}\kern2pt]} \hypertarget{t\string_hv\string_=\string_Ma1}{}
\markboth{\textcolor{darkblue}{\textbf{\ipa{tʰv̩˧\textsubscript{a}}}}}{}
\textcolor{teal}{\mytextsc{verbe}} \hspace{4pt} Ton~: M\textsubscript{a}.
\ding{202} \textcolor{PineGreen}{\selectlanguage{french}Sortir.} \zh{出来。}  ¶ \textcolor{darkblue}{\textbf{\ipa{ɑ˩pʰo˩ tʰv̩˩˥}}} \textcolor{PineGreen}{\selectlanguage{french}sortir, ex.: un animal sort de son terrier} \zh{出来,如:动物从地洞里爬出来}  
 ¶ \textcolor{darkblue}{\textbf{\ipa{ɲi˧mi˧ tʰv̩˧}}} \textcolor{PineGreen}{\selectlanguage{french}le soleil paraît} \zh{太阳出来}  
\ding{203} \textcolor{PineGreen}{\selectlanguage{french}Souffler (vent).} \zh{刮(风)。} \ding{204} \textcolor{PineGreen}{\selectlanguage{french}Germer, bourgeonner, donner des bourgeons.} \zh{发芽、抽芽。}  ¶ \textcolor{darkblue}{\textbf{\ipa{si˧dzi˩ | ʁo˧bv̩˧ tʰv̩˧}}} \textcolor{PineGreen}{\selectlanguage{french}l'arbre fait des bourgeons} \zh{树抽芽}  
\ding{205} \textcolor{PineGreen}{\selectlanguage{french}Apparaître, se faire: une blessure apparaît, on reçoit une blessure.} \zh{出现。}  ¶ \textcolor{darkblue}{\textbf{\ipa{mi˧ tʰv̩˧}}} \textcolor{PineGreen}{\selectlanguage{french}se faire une blessure/avoir une blessure/se blesser} \zh{受伤}  
 ¶ \textcolor{darkblue}{\textbf{\ipa{ɖɯ˧-v̩˧ mi˧ tʰv̩˧-ze˧!}}} \textcolor{PineGreen}{\selectlanguage{french}quelqu'un s'est blessé!} \zh{有人受伤了!}  
\ding{206} \textcolor{PineGreen}{\selectlanguage{french}Créer, fonder; se trouver, se fabriquer.} \zh{建立、创造、制造出来。}  ¶ \textcolor{darkblue}{\textbf{\ipa{ʑi˩ tʰv̩˩}}} \textcolor{PineGreen}{\selectlanguage{french}créer une nouvelle maison, fonder une nouvelle maison; traduit en chinois par \zh{分家}, concept en fait assez différent dans la mesure où \textcolor{darkblue}{\textbf{\ipa{ʑi˩ tʰv̩˩}}} évoque un essaimage, plutôt qu'une séparation.} \zh{分家、建立新家}  
 ¶ \textcolor{darkblue}{\textbf{\ipa{ʈʂʰɯ˧ | ʑi˩ tʰv̩˩-ze˥!}}} \textcolor{PineGreen}{\selectlanguage{french}Il/elle a fondé sa propre maisonnée!} \zh{他建了新家!}  
 ¶ \textcolor{darkblue}{\textbf{\ipa{ʈʂʰɯ˧ | ʑi˩ tʰv̩˩-bi˩˥!}}} \textcolor{PineGreen}{\selectlanguage{french}Il/elle va fonder sa propre maisonnée!} \zh{他要建个新家!}  
 ¶ \textcolor{darkblue}{\textbf{\ipa{lo˧ mɤ˧-dʑo˧, | lo˧ tʰv̩˧˥! / no˧ | lo˧ mɤ˧-dʑo˧, | lo˧ tʰv̩˧-ɲi˥!}}} \textcolor{PineGreen}{\selectlanguage{french}Il n'a pas d'obligations, et pourtant il travaille! (Compliment à l'endroit d'un fonctionnaire qui pourrait se contenter de percevoir son salaire, mais qui se donne à lui-même des objectifs et des tâches à accomplir. La phrase peut également être employée de façon négative, pour critiquer quelqu'un qui déploie une activité inutile au lieu de se tenir tranquille.)} \zh{自找麻烦!(这句,除贬义用法,还能用来表扬,如表扬一位当官的人努力去做好事,给自己找有意义的事情干。)}  

\lhead{\firstmark}
\rhead{\botmark}

\subsection{\hspace{-0.5cm} {\Large \textcolor{darkblue}{\textbf{\ipa{tʰv̩˧˥\textsubscript{a}}}}}\hspace{0.5cm}[\kern2pt{\textcolor{darkblue}{\textbf{\ipa{tʰv̩˥}}}}\kern2pt]} \hypertarget{t\string_hv\string_=\string_M\string_Ta1}{}
\markboth{\textcolor{darkblue}{\textbf{\ipa{tʰv̩˧˥\textsubscript{a}}}}}{}
\textcolor{teal}{\mytextsc{classificateur}} \hspace{4pt} Ton~: MH\textsubscript{a}.
\textcolor{PineGreen}{\selectlanguage{french}Pas, enjambée.} \zh{量词:步。}  ¶ \textcolor{darkblue}{\textbf{\ipa{ɖɯ˧-tʰv̩˧\textasciitilde{}ɖɯ˥-tʰv̩˩}}} \textcolor{PineGreen}{\selectlanguage{french}pas à pas} \zh{一步一步}  
 ¶ \textcolor{darkblue}{\textbf{\ipa{ɖɯ˧-tʰv̩˧˥, | ɖɯ˧-tʰv̩˧˥}}} \textcolor{PineGreen}{\selectlanguage{french}idem, détachant les deux parties; cette forme est plus proche d'une répétition que d'une réduplication} \zh{一步又一步}  

\lhead{\firstmark}
\rhead{\botmark}

\subsection{\hspace{-0.5cm} {\Large \textcolor{darkblue}{\textbf{\ipa{tʰv̩˩\textsubscript{b}}}}}\hspace{0.5cm}[\kern2pt{\textcolor{darkblue}{\textbf{\ipa{tʰv̩˩˥}}}}\kern2pt]} \hypertarget{t\string_hv\string_=\string_Bb1}{}
\markboth{\textcolor{darkblue}{\textbf{\ipa{tʰv̩˩\textsubscript{b}}}}}{}
\textcolor{teal}{\mytextsc{classificateur}} \hspace{4pt} Ton~: L\textsubscript{b}.
\textcolor{PineGreen}{\selectlanguage{french}Classificateur des dizaines. Autrefois, le terme servait à compter par ensembles de 8. Le terme a demeuré, mais son sens s'est déplacé vers le sens de 'dizaine', suivant la généralisation du système de numération à base dix.} \zh{量词:一套(有十个)。更早的意思是八个。}  ¶ \textcolor{darkblue}{\textbf{\ipa{qʰwɤ˩˥ | ɖɯ˧-tʰv̩˩}}} \textcolor{PineGreen}{\selectlanguage{french}un lot de dix bols} \zh{一套十个碗}  
 ¶ \textcolor{darkblue}{\textbf{\ipa{ɖʐɯ˧ʂɯ˥ | ɖɯ˧-tʰv̩˩}}} \textcolor{PineGreen}{\selectlanguage{french}un paquet de dix (paires de) baguettes} \zh{一套十(双)筷子}  

\lhead{\firstmark}
\rhead{\botmark}

\subsection{\hspace{-0.5cm} {\Large \textcolor{darkblue}{\textbf{\ipa{tʰv̩˧\textsubscript{b}}}}}\hspace{0.5cm}[\kern2pt{\textcolor{darkblue}{\textbf{\ipa{tʰv̩˩˥}}}}\kern2pt]} \hypertarget{t\string_hv\string_=\string_Mb1}{}
\markboth{\textcolor{darkblue}{\textbf{\ipa{tʰv̩˧\textsubscript{b}}}}}{}
\textcolor{teal}{\mytextsc{verbe}} \hspace{4pt} Ton~: M\textsubscript{b}.
\textcolor{PineGreen}{\selectlanguage{french}Prêter (un objet).} \zh{借给人。}  ¶ \textcolor{darkblue}{\textbf{\ipa{tso˧\textasciitilde{}tso˧ tʰv̩˧}}} \textcolor{PineGreen}{\selectlanguage{french}prêter quelque chose} \zh{借东西(给人)}  

\lhead{\firstmark}
\rhead{\botmark}

\subsection{\hspace{-0.5cm} {\Large \textcolor{darkblue}{\textbf{\ipa{tʰv̩˧gi˧}}}}\hspace{0.5cm}[\kern2pt{\textcolor{darkblue}{\textbf{\ipa{tʰv̩˧gi˩}}}}\kern2pt]} \hypertarget{t\string_hv\string_=\string_Mgi\string_M1}{}
\markboth{\textcolor{darkblue}{\textbf{\ipa{tʰv̩˧gi˧}}}}{}
\textcolor{teal}{\mytextsc{adverbe}} \hspace{4pt} Ton~: .
\textcolor{PineGreen}{\selectlanguage{french}Là-bas, de ce côté-là.} \zh{那边。} 
\lhead{\firstmark}
\rhead{\botmark}

\subsection{\hspace{-0.5cm} {\Large \textcolor{darkblue}{\textbf{\ipa{tʰv̩˧ne-ʝi˥}}}}\hspace{0.5cm}[\kern2pt{\textcolor{darkblue}{\textbf{\ipa{xxxx non-correspondance entre le nombre de morphèmes et le nombre de tons de morphèmes}}}}\kern2pt]} \hypertarget{t\string_hv\string_=\string_Mne-j££i\string_T1}{}
\markboth{\textcolor{darkblue}{\textbf{\ipa{tʰv̩˧ne-ʝi˥}}}}{}
\textcolor{teal}{\mytextsc{adverbe}} \hspace{4pt} Ton~: MH\#.
\textcolor{PineGreen}{\selectlanguage{french}Ainsi, de cette façon (adverbe de manière), contenant le démonstratif distal.} \zh{那样。} 
\lhead{\firstmark}
\rhead{\botmark}

\subsection{\hspace{-0.5cm} {\Large \textcolor{darkblue}{\textbf{\ipa{tʰv̩˧ɲi\#˥}}}}\hspace{0.5cm}[\kern2pt{\textcolor{darkblue}{\textbf{\ipa{tʰv̩˧ɲi˧}}}}\kern2pt]} \hypertarget{t\string_hv\string_=\string_MJi\#\string_T1}{}
\markboth{\textcolor{darkblue}{\textbf{\ipa{tʰv̩˧ɲi\#˥}}}}{}
\textcolor{teal}{\mytextsc{adverbe}} \hspace{4pt} Ton~: \#H.
\textcolor{PineGreen}{\selectlanguage{french}Ce jour-là (déictique lointain).} \zh{那天。} 
\lhead{\firstmark}
\rhead{\botmark}

\subsection{\hspace{-0.5cm} {\Large \textcolor{darkblue}{\textbf{\ipa{tʰv̩˧qo˧}}}}\hspace{0.5cm}[\kern2pt{\textcolor{darkblue}{\textbf{\ipa{tʰv̩˧qo˧}}}}\kern2pt]} \hypertarget{t\string_hv\string_=\string_Mqo\string_M1}{}
\markboth{\textcolor{darkblue}{\textbf{\ipa{tʰv̩˧qo˧}}}}{}
\textcolor{teal}{\mytextsc{pronom}} \hspace{4pt} Ton~: M.
\textcolor{PineGreen}{\selectlanguage{french}Là-bas; cet endroit-là.} \zh{那里、那个地方。} 
\lhead{\firstmark}
\rhead{\botmark}

\subsection{\hspace{-0.5cm} {\Large \textcolor{darkblue}{\textbf{\ipa{tʰv̩˧-si˥}}}}\hspace{0.5cm}[\kern2pt{\textcolor{darkblue}{\textbf{\ipa{xxxx non-correspondance entre le nombre de morphèmes et le nombre de tons de morphèmes}}}}\kern2pt]} \hypertarget{t\string_hv\string_=\string_M-si\string_T1}{}
\markboth{\textcolor{darkblue}{\textbf{\ipa{tʰv̩˧-si˥}}}}{}
\textcolor{teal}{\mytextsc{adverbe}} \hspace{4pt} Ton~: H\#.
\textcolor{PineGreen}{\selectlanguage{french}Nombreux.} \zh{多。}  ¶ \textcolor{darkblue}{\textbf{\ipa{mv̩˧ʁo˧=ɻ̍˥-dʑo˩, | ɻæ˩˥ | tʰv̩˧-si˥ | tʰv̩˩-jɤ˩ dʑo˩˥!}}} \textcolor{PineGreen}{\selectlanguage{french}Les gens du Ciel, ils avaient des semences en abondance! (récit “Seeds”)} \zh{天上的人,有许多许多种子!(故事:“种子”)}  

\lhead{\firstmark}
\rhead{\botmark}

\subsection{\hspace{-0.5cm} {\Large \textcolor{darkblue}{\textbf{\ipa{‑tʰv̩˧}}} \textsubscript{1}}\hspace{0.5cm}[\kern2pt{\textcolor{darkblue}{\textbf{\ipa{tʰv̩˥}}}}\kern2pt]} \hypertarget{‑t\string_hv\string_=\string_M1}{}
\markboth{\textcolor{darkblue}{\textbf{\ipa{‑tʰv̩˧}}} \textsubscript{1}}{}
\textcolor{teal}{\mytextsc{postposition}} \hspace{4pt} Ton~: M.
\textcolor{PineGreen}{\selectlanguage{french}Postposition temporelle: jusqu'à.} \zh{到……为止。} 
\lhead{\firstmark}
\rhead{\botmark}

\subsection{\hspace{-0.5cm} {\Large \textcolor{darkblue}{\textbf{\ipa{‑tʰv̩˧}}} \textsubscript{2}}\hspace{0.5cm}[\kern2pt{\textcolor{darkblue}{\textbf{\ipa{tʰv̩˥}}}}\kern2pt]} \hypertarget{‑t\string_hv\string_=\string_M2}{}
\markboth{\textcolor{darkblue}{\textbf{\ipa{‑tʰv̩˧}}} \textsubscript{2}}{}
\textcolor{teal}{\mytextsc{suffixe}} \hspace{4pt} Ton~: M.
\textcolor{PineGreen}{\selectlanguage{french}Parvenir à, réussir à, réaliser avec succès; grammaticalisé à partir du verbe 'sortir'.} \zh{……成。}  ¶ \textcolor{darkblue}{\textbf{\ipa{lo˧ ʝi˧-mɤ˧-tʰv̩˧}}} \textcolor{PineGreen}{\selectlanguage{french}ne pas parvenir à venir à bien d'une tâche; ex.: une personne est constamment dérangée et ne parvient pas à travailler de façon concentrée} \zh{活做不出来、活做不成(比如:一个人经常被打扰,所以不能集中工作,没有效率,要做的事做不成)}  

\lhead{\firstmark}
\rhead{\botmark}

\newpage
\section*{\centering- \textcolor{darkblue}{\textbf{\ipa{tɕ}}} -}
\subsection{\hspace{-0.5cm} {\Large \textcolor{darkblue}{\textbf{\ipa{tɕæ˧hæ˩}}}}\hspace{0.5cm}[\kern2pt{\textcolor{darkblue}{\textbf{\ipa{tɕæ˧hæ˩}}}}\kern2pt]} \hypertarget{ts£\{\string_Mh\{\string_B1}{}
\markboth{\textcolor{darkblue}{\textbf{\ipa{tɕæ˧hæ˩}}}}{}
\textcolor{teal}{\mytextsc{nom}} \hspace{4pt} Ton~: L\#.
\textcolor{PineGreen}{\selectlanguage{french}Caoutchouc.} \zh{橡胶(汉语借词。第二个音节:未确定。)。}  Emprunt~: chinois  \zh{胶} +?
 ¶ \textcolor{darkblue}{\textbf{\ipa{tɕæ˧hæ˩-dzɑ˩qʰwɤ˩}}} \textcolor{PineGreen}{\selectlanguage{french}chaussures à semelle en gomme/en caoutchouc; baskets} \zh{橡胶鞋、橡胶底鞋}  
 \zh{量词}: \textcolor{darkblue}{\textbf{\ipa{dzi˧}}}  \mytextsc{clf}~: \textcolor{darkblue}{\textbf{\ipa{dzi˧}}} 
\lhead{\firstmark}
\rhead{\botmark}

\subsection{\hspace{-0.5cm} {\Large \textcolor{darkblue}{\textbf{\ipa{tɕæ˧pʰv̩˩}}}}\hspace{0.5cm}[\kern2pt{\textcolor{darkblue}{\textbf{\ipa{tɕæ˧pʰv̩˩}}}}\kern2pt]} \hypertarget{ts£\{\string_Mp\string_hv\string_=\string_B1}{}
\markboth{\textcolor{darkblue}{\textbf{\ipa{tɕæ˧pʰv̩˩}}}}{}
\textcolor{teal}{\mytextsc{adjectif}} \hspace{4pt} Ton~: L\#.
\textcolor{PineGreen}{\selectlanguage{french}Blanc (visage, habits, cheveux...).} \zh{白(脸、衣服)。}  ¶ \textcolor{darkblue}{\textbf{\ipa{tɕæ˧pʰv̩˩-bɑ˩lɑ˩}}} \textcolor{PineGreen}{\selectlanguage{french}vêtement blanc} \zh{白的衣服}  
 ¶ \textcolor{darkblue}{\textbf{\ipa{tɕæ˧pʰv̩˩-ʈæ˩qʰwɤ˩}}} \textcolor{PineGreen}{\selectlanguage{french}robe blanche} \zh{白色裙子}  

\lhead{\firstmark}
\rhead{\botmark}

\subsection{\hspace{-0.5cm} {\Large \textcolor{darkblue}{\textbf{\ipa{tɕæ˧ɻæ˩}}}}\hspace{0.5cm}[\kern2pt{\textcolor{darkblue}{\textbf{\ipa{tɕæ˧ɻæ˩}}}}\kern2pt]} \hypertarget{ts£\{\string_Mr£`\{\string_B1}{}
\markboth{\textcolor{darkblue}{\textbf{\ipa{tɕæ˧ɻæ˩}}}}{}
\textcolor{teal}{\mytextsc{nom}} \hspace{4pt} Ton~: L\#.
\textcolor{PineGreen}{\selectlanguage{french}Légumes en saumure. On en mangeait une sorte chaque jour pendant la saison d'hiver: un jour navet en saumure, etc.} \zh{酸菜、泡菜。}  ¶ \textcolor{darkblue}{\textbf{\ipa{wo˩-tɕæ˩ɻæ˥}}} \textcolor{PineGreen}{\selectlanguage{french}feuilles de navet conservées dans la saumure} \zh{圆根叶子酸菜}  
 ¶ \textcolor{darkblue}{\textbf{\ipa{tsʰɑ˧-tɕæ˧ɻæ˥}}}  
 ¶ \textcolor{darkblue}{\textbf{\ipa{ɬi˩bi˩-tɕæ˩ɻæ˥}}} \textcolor{PineGreen}{\selectlanguage{french}navet conservé dans la saumure} \zh{圆根酸菜}  
 ¶ \textcolor{darkblue}{\textbf{\ipa{pɤ˧pɤ˧tsʰɯ˧-tɕæ˧ɻæ˥}}} \textcolor{PineGreen}{\selectlanguage{french}chou chinois en saumure} \zh{圆白菜酸菜}  

\lhead{\firstmark}
\rhead{\botmark}

\subsection{\hspace{-0.5cm} {\Large \textcolor{darkblue}{\textbf{\ipa{tɕɤ}}}}\hspace{0.5cm}[\kern2pt{\textcolor{darkblue}{\textbf{\ipa{[]}}}}\kern2pt]} \hypertarget{ts£71}{}
\markboth{\textcolor{darkblue}{\textbf{\ipa{tɕɤ}}}}{}
\textcolor{teal}{\mytextsc{interjection}} \hspace{4pt} Ton~: 0.
\textcolor{PineGreen}{\selectlanguage{french}Interjection: tiens! eh!} \zh{感叹词:嘿!。} 
\lhead{\firstmark}
\rhead{\botmark}

\subsection{\hspace{-0.5cm} {\Large \textcolor{darkblue}{\textbf{\ipa{tɕɤ˥}}}}\hspace{0.5cm}[\kern2pt{\textcolor{darkblue}{\textbf{\ipa{tɕɤ˥}}}}\kern2pt]} \hypertarget{ts£7\string_T1}{}
\markboth{\textcolor{darkblue}{\textbf{\ipa{tɕɤ˥}}}}{}
\textcolor{teal}{\mytextsc{verbe}} \hspace{4pt} Ton~: H.
\textcolor{PineGreen}{\selectlanguage{french}S'effacer (couleur).} \zh{褪色。}  ¶ \textcolor{darkblue}{\textbf{\ipa{le˧-tɕɤ˥-ze˩}}} \textcolor{PineGreen}{\selectlanguage{french}\mytextsc{accomp} \string_ \mytextsc{pfv}} \zh{褪色了}  

\lhead{\firstmark}
\rhead{\botmark}

\subsection{\hspace{-0.5cm} {\Large \textcolor{darkblue}{\textbf{\ipa{tɕɤ˧fv̩˩}}}}\hspace{0.5cm}[\kern2pt{\textcolor{darkblue}{\textbf{\ipa{tɕɤ˧fv̩˩}}}}\kern2pt]} \hypertarget{ts£7\string_Mfv\string_=\string_B1}{}
\markboth{\textcolor{darkblue}{\textbf{\ipa{tɕɤ˧fv̩˩}}}}{}
\textcolor{teal}{\mytextsc{nom}} \hspace{4pt} Ton~: L\#.
\textcolor{PineGreen}{\selectlanguage{french}Container pour liquides; s'emploie pour désigner les containers en matière plastique.} \zh{塑料桶等存水用的容器。}  \zh{量词}: \textcolor{darkblue}{\textbf{\ipa{ɭɯ˧}}}  \mytextsc{clf}~: \textcolor{darkblue}{\textbf{\ipa{ɭɯ˧}}} 
\lhead{\firstmark}
\rhead{\botmark}

\subsection{\hspace{-0.5cm} {\Large \textcolor{darkblue}{\textbf{\ipa{tɕɤ˧ho˩pæ˧}}}}\hspace{0.5cm}[\kern2pt{\textcolor{darkblue}{\textbf{\ipa{xxxx ton non trouvé, à faire manuellement...}}}}\kern2pt]} \hypertarget{ts£7\string_Mho\string_Bp\{\string_M1}{}
\markboth{\textcolor{darkblue}{\textbf{\ipa{tɕɤ˧ho˩pæ˧}}}}{}
\textcolor{teal}{\mytextsc{nom}} \hspace{4pt} Ton~: MLM.
\textcolor{PineGreen}{\selectlanguage{french}Contreplaqué, panneau en contreplaqué.} \zh{胶合板(汉语借词)。}  Emprunt~: chinois  \zh{胶合板}

\lhead{\firstmark}
\rhead{\botmark}

\subsection{\hspace{-0.5cm} {\Large \textcolor{darkblue}{\textbf{\ipa{tɕɤ˧qʰɑ\#˥}}}}\hspace{0.5cm}[\kern2pt{\textcolor{darkblue}{\textbf{\ipa{tɕɤ˧qʰɑ˧}}}}\kern2pt]} \hypertarget{ts£7\string_Mq\string_hA\#\string_T1}{}
\markboth{\textcolor{darkblue}{\textbf{\ipa{tɕɤ˧qʰɑ\#˥}}}}{}
\textcolor{teal}{\mytextsc{nom}} \hspace{4pt} Ton~: \#H.
\textcolor{PineGreen}{\selectlanguage{french}Armoise, \textit{Artemisia vulgaris}.} \zh{蒿、青蒿。} Dialecte chinois local~:\zh{蒿草、蒿枝。} ¶ \textcolor{darkblue}{\textbf{\ipa{tɕɤ˧qʰɑ˧-mo˩}}} \textcolor{PineGreen}{\selectlanguage{french}un champignon comestible, nommé 'champignon de l'armoise' parce qu'il croît à proximité de l'armoise} \zh{一种可以吃的菌子,长在蒿附近}  
\textit{Voir~:} \hyperlink{}{\textcolor{darkblue}{\textbf{\ipa{ho˧ʈʂɯ˧}}}} 
\lhead{\firstmark}
\rhead{\botmark}

\subsection{\hspace{-0.5cm} {\Large \textcolor{darkblue}{\textbf{\ipa{tɕɤ˧tɑ˧}}}}\hspace{0.5cm}[\kern2pt{\textcolor{darkblue}{\textbf{\ipa{tɕɤ˧tɑ˧}}}}\kern2pt]} \hypertarget{ts£7\string_MtA\string_M1}{}
\markboth{\textcolor{darkblue}{\textbf{\ipa{tɕɤ˧tɑ˧}}}}{}
\textcolor{teal}{\mytextsc{nom}} \hspace{4pt} Ton~: M.
\textcolor{PineGreen}{\selectlanguage{french}Joug.} \zh{牛轭(单行)(汉语借词)。} Dialecte chinois local~:\zh{牛夹担、牛枷档、牛拴。} Emprunt~: chinois  \zh{夹担}
 ¶ \textcolor{darkblue}{\textbf{\ipa{tɕɤ˧tɑ˧ tʰv̩˧-ɭɯ˧}}} \textcolor{PineGreen}{\selectlanguage{french}\mytextsc{n}+\mytextsc{dem}+\mytextsc{clf}} \zh{这个牛轭}  
 \zh{量词}: \textcolor{darkblue}{\textbf{\ipa{ɭɯ˧}}}  \mytextsc{clf}~: \textcolor{darkblue}{\textbf{\ipa{ɭɯ˧}}} 
\lhead{\firstmark}
\rhead{\botmark}

\subsection{\hspace{-0.5cm} {\Large \textcolor{darkblue}{\textbf{\ipa{tɕɤ˧\textasciitilde{}tɕɤ˧}}}}\hspace{0.5cm}[\kern2pt{\textcolor{darkblue}{\textbf{\ipa{tɕɤ˧tɕɤ˧}}}}\kern2pt]} \hypertarget{ts£7\string_M~ts£7\string_M1}{}
\markboth{\textcolor{darkblue}{\textbf{\ipa{tɕɤ˧\textasciitilde{}tɕɤ˧}}}}{}
\textcolor{teal}{\mytextsc{adverbe}} \hspace{4pt} Ton~: M.
\textcolor{PineGreen}{\selectlanguage{french}Précisément, exactement (ex.: au moment précis où, juste au moment où).} \zh{将将(汉语借词)、刚刚。} Dialecte chinois local~:\zh{将将。} Emprunt~: chinois  \zh{将将}

\lhead{\firstmark}
\rhead{\botmark}

\subsection{\hspace{-0.5cm} {\Large \textcolor{darkblue}{\textbf{\ipa{tɕɤ˩}}}}\hspace{0.5cm}[\kern2pt{\textcolor{darkblue}{\textbf{\ipa{xxxx groupe tonal entier sans aucun ton}}}}\kern2pt]} \hypertarget{ts£7\string_B1}{}
\markboth{\textcolor{darkblue}{\textbf{\ipa{tɕɤ˩}}}}{}
\textcolor{teal}{\mytextsc{verbe}} \hspace{4pt} Ton~: .
\textcolor{PineGreen}{\selectlanguage{french}Attacher (ex.: un joug sur une vache; des troncs…).} \zh{打结、系上。}  Emprunt~: chinois  \zh{架?}
 ¶ \textcolor{darkblue}{\textbf{\ipa{ʁæ˧ɻ̍˥ | tʰi˧-tɕɤ˩}}} \textcolor{PineGreen}{\selectlanguage{french}fixer (un joug sur un buffle)} \zh{系上牛轭}  

\lhead{\firstmark}
\rhead{\botmark}

\subsection{\hspace{-0.5cm} {\Large \textcolor{darkblue}{\textbf{\ipa{tɕɤ˩ho˩tsɯ˥}}}}\hspace{0.5cm}[\kern2pt{\textcolor{darkblue}{\textbf{\ipa{tɕɤ˩ho˩tsɯ˥}}}}\kern2pt]} \hypertarget{ts£7\string_Bho\string_BtsM\string_T1}{}
\markboth{\textcolor{darkblue}{\textbf{\ipa{tɕɤ˩ho˩tsɯ˥}}}}{}
\textcolor{teal}{\mytextsc{nom}} \hspace{4pt} Ton~: L+H\#.
\textcolor{PineGreen}{\selectlanguage{french}Escroc.} \zh{骗子。}  ¶ \textcolor{darkblue}{\textbf{\ipa{ʈʂʰɯ˧ | hĩ˧ ʈʂʰɯ˧-v̩˧ | tɕɤ˩ho˩tsɯ˥ ɲi˩.}}} \textcolor{PineGreen}{\selectlanguage{french}Cet homme, c'est un escroc!} \zh{这个人是骗子!}  
 \zh{量词}: \textcolor{darkblue}{\textbf{\ipa{v̩˧}}}  \mytextsc{clf}~: \textcolor{darkblue}{\textbf{\ipa{v̩˧}}} 
\lhead{\firstmark}
\rhead{\botmark}

\subsection{\hspace{-0.5cm} {\Large \textcolor{darkblue}{\textbf{\ipa{tɕɤ˧˥}}}}\hspace{0.5cm}[\kern2pt{\textcolor{darkblue}{\textbf{\ipa{tɕɤ˧˥}}}}\kern2pt]} \hypertarget{ts£7\string_M\string_T1}{}
\markboth{\textcolor{darkblue}{\textbf{\ipa{tɕɤ˧˥}}}}{}
\textcolor{teal}{\mytextsc{verbe}} \hspace{4pt} Ton~: MH.
\textcolor{PineGreen}{\selectlanguage{french}Bouillir; cuire en faisant bouillir; cuire dans une casserole.} \zh{煮。}  ¶ \textcolor{darkblue}{\textbf{\ipa{ʂe˧ tɕɤ˩}}} \textcolor{PineGreen}{\selectlanguage{french}faire bouillir de la viande, faire cuire de la viande à l'eau} \zh{煮肉}  
 ¶ \textcolor{darkblue}{\textbf{\ipa{bo˩-hɑ˧ tɕɤ˩}}} \textcolor{PineGreen}{\selectlanguage{french}faire bouillir la pâtée des cochons} \zh{煮猪食}  
 ¶ \textcolor{darkblue}{\textbf{\ipa{ho˧ tɕɤ˩}}} \textcolor{PineGreen}{\selectlanguage{french}faire du ragoût} \zh{煮粥}  
 ¶ \textcolor{darkblue}{\textbf{\ipa{dʑɯ˩ʁo˩˥, | mo˧-no˥, | mo˧ tɕɤ˥-hĩ˩ lɑ˩-ɲi˩-mæ˩! |}}} \textcolor{PineGreen}{\selectlanguage{french}(Quand on se trouve sur) la montagne, les champignons, on les fait simplement cuire dans une casserole! (Littéralement: “on se contente de les faire bouillir”.) (On mettait simplement les champignons dans une casserole, sans eau, avec du sel et de la graisse; les champignons cuisaient alors dans leur propre eau.)} \zh{在山上,菌子,就是简单煮一下而已!(放在锅里,加油、加盐。用菌子自身的水分)}  

\lhead{\firstmark}
\rhead{\botmark}

\subsection{\hspace{-0.5cm} {\Large \textcolor{darkblue}{\textbf{\ipa{tɕi˥}}}}\hspace{0.5cm}[\kern2pt{\textcolor{darkblue}{\textbf{\ipa{tɕi˥}}}}\kern2pt]} \hypertarget{ts£i\string_T1}{}
\markboth{\textcolor{darkblue}{\textbf{\ipa{tɕi˥}}}}{}
\textcolor{teal}{\mytextsc{verbe}} \hspace{4pt} Ton~: H.
\textcolor{PineGreen}{\selectlanguage{french}Secouer (ex.: pour défroisser des vêtements après lavage; aussi: secouer la tête).} \zh{抖、抖动,摇动。}  ¶ \textcolor{darkblue}{\textbf{\ipa{le˧-tɕi˧\textasciitilde{}tɕi˧-ze˩}}} \textcolor{PineGreen}{\selectlanguage{french}\mytextsc{accomp} \string_ \mytextsc{pfv}} \zh{\mytextsc{accomp} \string_ \mytextsc{pfv}}  
 ¶ \textcolor{darkblue}{\textbf{\ipa{tʰi˧-tɕi˧\textasciitilde{}tɕi˧+ze˩}}} \textcolor{PineGreen}{\selectlanguage{french}\mytextsc{dur} \string_ \mytextsc{pfv}} \zh{\mytextsc{dur} \string_ \mytextsc{pfv}}  
 ¶ \textcolor{darkblue}{\textbf{\ipa{ʁo˧qʰwɤ˩ tɕi˩\textasciitilde{}tɕi˩}}} \textcolor{PineGreen}{\selectlanguage{french}agiter la tête, secouer la tête} \zh{摇头}  
 ¶ \textcolor{darkblue}{\textbf{\ipa{ɖɯ˧-tɕi˧\textasciitilde{}tɕi˧-ɻ̍˥}}} \textcolor{PineGreen}{\selectlanguage{french}\mytextsc{délimitatif} \string_ \mytextsc{red} \mytextsc{inchoatif}} \zh{摇一摇}  

\lhead{\firstmark}
\rhead{\botmark}

\subsection{\hspace{-0.5cm} {\Large \textcolor{darkblue}{\textbf{\ipa{tɕi˧}}} \textsubscript{1}}\hspace{0.5cm}[\kern2pt{\textcolor{darkblue}{\textbf{\ipa{tɕi˥}}}}\kern2pt]} \hypertarget{ts£i\string_M1}{}
\markboth{\textcolor{darkblue}{\textbf{\ipa{tɕi˧}}} \textsubscript{1}}{}
\textcolor{teal}{\mytextsc{adjectif}} \hspace{4pt} Ton~: M.
\ding{202} \textcolor{PineGreen}{\selectlanguage{french}Acide.} \zh{酸。}  ¶ \textcolor{darkblue}{\textbf{\ipa{tɕʰɯ˩-hĩ˩˥}}}  
 ¶ \textcolor{darkblue}{\textbf{\ipa{[M18] tɕi˧-hĩ˧ pʰi˩}}} \textcolor{PineGreen}{\selectlanguage{french}avoir des remontées acides} \zh{吐酸水}  
\ding{203} \textcolor{PineGreen}{\selectlanguage{french}Fermenté.} \zh{(通过发酵的)酸。} 
\lhead{\firstmark}
\rhead{\botmark}

\subsection{\hspace{-0.5cm} {\Large \textcolor{darkblue}{\textbf{\ipa{tɕi˧}}} \textsubscript{2}}\hspace{0.5cm}[\kern2pt{\textcolor{darkblue}{\textbf{\ipa{tɕi˥}}}}\kern2pt]} \hypertarget{ts£i\string_M2}{}
\markboth{\textcolor{darkblue}{\textbf{\ipa{tɕi˧}}} \textsubscript{2}}{}
\textcolor{teal}{\mytextsc{nom}} \hspace{4pt} Ton~: M.
\textcolor{PineGreen}{\selectlanguage{french}Piège.} \zh{圈套。}  ¶ \textcolor{darkblue}{\textbf{\ipa{tɕi˧ kʰɯ˧˥}}} \textcolor{PineGreen}{\selectlanguage{french}poser un piège} \zh{设下圈套}  
 \zh{量词}: \textcolor{darkblue}{\textbf{\ipa{ɭɯ˧}}}  \mytextsc{clf}~: \textcolor{darkblue}{\textbf{\ipa{ɭɯ˧}}} 
\lhead{\firstmark}
\rhead{\botmark}

\subsection{\hspace{-0.5cm} {\Large \textcolor{darkblue}{\textbf{\ipa{tɕi˧\textsubscript{a}}}}}\hspace{0.5cm}[\kern2pt{\textcolor{darkblue}{\textbf{\ipa{tɕi˥}}}}\kern2pt]} \hypertarget{ts£i\string_Ma1}{}
\markboth{\textcolor{darkblue}{\textbf{\ipa{tɕi˧\textsubscript{a}}}}}{}
\textcolor{teal}{\mytextsc{classificateur}} \hspace{4pt} Ton~: M\textsubscript{a}.
\textcolor{PineGreen}{\selectlanguage{french}Quelques-uns, certains, une partie.} \zh{量词:一些。}  ¶ \textcolor{darkblue}{\textbf{\ipa{ɖɯ˧-tɕi˧}}} \textcolor{PineGreen}{\selectlanguage{french}quelques-uns, certains} \zh{一些}  
 ¶ \textcolor{darkblue}{\textbf{\ipa{ʈʂʰɯ˧-tɕi˩}}} \textcolor{PineGreen}{\selectlanguage{french}ceux-ci} \zh{这些}  

\lhead{\firstmark}
\rhead{\botmark}

\subsection{\hspace{-0.5cm} {\Large \textcolor{darkblue}{\textbf{\ipa{tɕi˧do˩}}}}\hspace{0.5cm}[\kern2pt{\textcolor{darkblue}{\textbf{\ipa{tɕi˧do˩}}}}\kern2pt]} \hypertarget{ts£i\string_Mdo\string_B1}{}
\markboth{\textcolor{darkblue}{\textbf{\ipa{tɕi˧do˩}}}}{}
\textcolor{teal}{\mytextsc{nom}} \hspace{4pt} Ton~: L\#.
\textcolor{PineGreen}{\selectlanguage{french}Mandarine.} \zh{橘子。} Dialecte chinois local~:\zh{黄果。} \zh{量词}: \textcolor{darkblue}{\textbf{\ipa{ɭɯ˧}}}  \mytextsc{clf}~: \textcolor{darkblue}{\textbf{\ipa{ɭɯ˧}}} 
\lhead{\firstmark}
\rhead{\botmark}

\subsection{\hspace{-0.5cm} {\Large \textcolor{darkblue}{\textbf{\ipa{tɕi˧-dʑɯ˩}}}}\hspace{0.5cm}[\kern2pt{\textcolor{darkblue}{\textbf{\ipa{xxxx non-correspondance entre le nombre de morphèmes et le nombre de tons de morphèmes}}}}\kern2pt]} \hypertarget{ts£i\string_M-dz£M\string_B1}{}
\markboth{\textcolor{darkblue}{\textbf{\ipa{tɕi˧-dʑɯ˩}}}}{}
\textcolor{teal}{\mytextsc{nom}} \hspace{4pt} Ton~: L\#.
\ding{202} \textcolor{PineGreen}{\selectlanguage{french}Potion acide: une préparation à base de prunelles acides ou baies sauvages, utilisée pour faire vomir les personnes victimes d'un empoisonnement alimentaire (par exemple par des champignons vénéneux).} \zh{用梅子等野生果子做出来的一种药品(酸水),食物中毒的情况下给病人和这种酸水让他呕吐。} \ding{203} \textcolor{PineGreen}{\selectlanguage{french}Vinaigre.} \zh{醋。} 
\lhead{\firstmark}
\rhead{\botmark}

\subsection{\hspace{-0.5cm} {\Large \textcolor{darkblue}{\textbf{\ipa{tɕi˧kwɤ˧}}}}\hspace{0.5cm}[\kern2pt{\textcolor{darkblue}{\textbf{\ipa{tɕi˧kwɤ˩}}}}\kern2pt]} \hypertarget{ts£i\string_Mkw7\string_M1}{}
\markboth{\textcolor{darkblue}{\textbf{\ipa{tɕi˧kwɤ˧}}}}{}
\textcolor{teal}{\mytextsc{nom}} \hspace{4pt} Ton~: M.
\textcolor{PineGreen}{\selectlanguage{french}Courge (inclut les courgettes).} \zh{瓜。}  ¶ \textcolor{darkblue}{\textbf{\ipa{tɕi˧kwɤ˧ bv̩˧-ɻ̍˧ (+ɲi˩)}}} \textcolor{PineGreen}{\selectlanguage{french}petite courge} \zh{小瓜}  
 ¶ \textcolor{darkblue}{\textbf{\ipa{tɕi˧kwɤ˧ kwɤ˧mo˩}}} \textcolor{PineGreen}{\selectlanguage{french}grosse courge} \zh{大瓜}  
 \zh{量词}: \textcolor{darkblue}{\textbf{\ipa{ɭɯ˧}}}  \mytextsc{clf}~: \textcolor{darkblue}{\textbf{\ipa{ɭɯ˧}}} 
\lhead{\firstmark}
\rhead{\botmark}

\subsection{\hspace{-0.5cm} {\Large \textcolor{darkblue}{\textbf{\ipa{tɕi˧sɯ˧pɤ˧}}}}\hspace{0.5cm}[\kern2pt{\textcolor{darkblue}{\textbf{\ipa{tɕi˩sɯ˩pɤ˥}}}}\kern2pt]} \hypertarget{ts£i\string_MsM\string_Mp7\string_M1}{}
\markboth{\textcolor{darkblue}{\textbf{\ipa{tɕi˧sɯ˧pɤ˧}}}}{}
\textcolor{teal}{\mytextsc{nom}} \hspace{4pt} Ton~: M.
\textcolor{PineGreen}{\selectlanguage{french}Fromage au lait de yak. On commençait par écrémer le lait, puis on faisait bouillir, avec un additif pour le faire cailler; enfin la préparation se solidifiait. Cette préparation était utilisée dans l'alimentation: on en mettait dans les bouillies de céréales. Elle pouvait se conserver. Ce fromage, acide et dur, était recommandé aux personnes ayant des soucis digestifs (comme remède à la diarrhée), et aux personnes âgées.} \zh{牦牛奶酪。}  ¶ \textcolor{darkblue}{\textbf{\ipa{mv̩˧ɭɯ˩-pʰɤ˩bɤ˩, | tɕi˧sɯ˧pɤ˧!}}} \textcolor{PineGreen}{\selectlanguage{french}Le cadeau (qu'on ramène de Muli), c'est le fromage de yak! / La spécialité de Muli, c'est le fromage de yak! (Autrefois, c'était un des cadeaux que les jeunes gens offraient aux jeunes filles au retour de leurs voyages.)} \zh{木里的礼物:牦牛的奶酪! / 牦牛奶酪,是木里的特产!}  
 ¶ \textcolor{darkblue}{\textbf{\ipa{mv̩˧ɭɯ˩ pʰɤ˩bɤ˩, | tɕi˧sɯ˧pɤ˧! | ə˧ɖo˧ ʁo˧ ɖʐɯ˥\textasciitilde{}ɖʐɯ˩ ʝi˩-ze˩!}}} \textcolor{PineGreen}{\selectlanguage{french}Le cadeau (qu'on ramène de Muli), c'est le fromage de yak! Ma bien-aimée va secouer la tête (lorsqu'elle goûtera à ce fromage, très acide)! (Paroles d'une chanson qu'on chantait en chemin, en imaginant le retour.)} \zh{(从)木里(带回来)的礼物,就是牦牛奶酪!亲爱的(=收礼物的那个人),会摇头的!(吃、喝的时候会摇头,是因为牦牛奶奶酪比较酸)}  
 ¶ \textcolor{darkblue}{\textbf{\ipa{tɕi˧sɯ˧pɤ˧, | ɖɯ˧-tɑ˧˥ | gv̩˧-mɤ˧-kv̩˥! | ʝi˧-kʰv̩˥-lɑ˩ gv̩˩-kv̩˩!}}} \textcolor{PineGreen}{\selectlanguage{french}Ce n'est pas tout le monde qui savait faire du fromage de yak! Il n'y a que certaines (personnes/familles) qui savaient le faire!} \zh{不是每个人都会做牦牛奶酪!只有少数(人)才会做!}  
 ¶ \textcolor{darkblue}{\textbf{\ipa{tɕi˧sɯ˧pɤ˧-dʑɯ˩}}} \textcolor{PineGreen}{\selectlanguage{french}eau dans laquelle on a dilué du fromage de yak; elle a des propriétés médicinales} \zh{一种饮料:将牦牛奶酪溶化在水里}  
 ¶ \textcolor{darkblue}{\textbf{\ipa{tɕi˧sɯ˧pɤ˧ ʈʰɯ˩}}} \textcolor{PineGreen}{\selectlanguage{french}boire de l'eau dans laquelle on a dilué du fromage de yak; littéralement: 'boire du fromage de yak'} \zh{喝溶化在水里的牦牛奶酪(直译:喝牦牛奶酪)}  

\lhead{\firstmark}
\rhead{\botmark}

\subsection{\hspace{-0.5cm} {\Large \textcolor{darkblue}{\textbf{\ipa{tɕi˧tɕi˧ | læ˩sæ˧-dzi˩}}}}\hspace{0.5cm}[\kern2pt{\textcolor{darkblue}{\textbf{\ipa{xxxx non-correspondance entre le nombre de groupes tonals et le nombre de tons}}}}\kern2pt]} \hypertarget{ts£i\string_Mts£i\string_M | l\{\string_Bs\{\string_M-dzi\string_B1}{}
\markboth{\textcolor{darkblue}{\textbf{\ipa{tɕi˧tɕi˧ | læ˩sæ˧-dzi˩}}}}{}
\textcolor{teal}{\mytextsc{nom}} \hspace{4pt} Ton~: M-LH-L.
\textcolor{PineGreen}{\selectlanguage{french}Un arbre au bois très dur.} \zh{一种树,木质很硬。} 
\lhead{\firstmark}
\rhead{\botmark}

\subsection{\hspace{-0.5cm} {\Large \textcolor{darkblue}{\textbf{\ipa{tɕi˩\textsubscript{a}}}}}\hspace{0.5cm}[\kern2pt{\textcolor{darkblue}{\textbf{\ipa{tɕi˩˥}}}}\kern2pt]} \hypertarget{ts£i\string_Ba1}{}
\markboth{\textcolor{darkblue}{\textbf{\ipa{tɕi˩\textsubscript{a}}}}}{}
\textcolor{teal}{\mytextsc{adjectif}} \hspace{4pt} Ton~: L\textsubscript{a}.
\textcolor{PineGreen}{\selectlanguage{french}Petit.} \zh{矮,低,小。}  ¶ \textcolor{darkblue}{\textbf{\ipa{tɕi˩-hĩ˩˥}}} \textcolor{PineGreen}{\selectlanguage{french}(qui est) petit} \zh{矮的}  
 ¶ \textcolor{darkblue}{\textbf{\ipa{gv̩˧mi˧ tɕi˩}}} \textcolor{PineGreen}{\selectlanguage{french}de petite taille} \zh{矮}  

\lhead{\firstmark}
\rhead{\botmark}

\subsection{\hspace{-0.5cm} {\Large \textcolor{darkblue}{\textbf{\ipa{tɕi˩nv̩˧˥}}}}\hspace{0.5cm}[\kern2pt{\textcolor{darkblue}{\textbf{\ipa{tɕi˧nv̩˥}}}}\kern2pt]} \hypertarget{ts£i\string_Bnv\string_=\string_M\string_T1}{}
\markboth{\textcolor{darkblue}{\textbf{\ipa{tɕi˩nv̩˧˥}}}}{}
\textcolor{teal}{\mytextsc{nom}} \hspace{4pt} Ton~: LM+MH\#.
\textcolor{PineGreen}{\selectlanguage{french}Tapis de selle.} \zh{马鞍下面的毯子。}  ¶ \textcolor{darkblue}{\textbf{\ipa{ʐwæ˧-tɕi˥nv̩˩}}} \textcolor{PineGreen}{\selectlanguage{french}tapis de selle de cheval} \zh{马鞍毯子}  
 \zh{量词}: \textcolor{darkblue}{\textbf{\ipa{pɤ˩}}}  \mytextsc{clf}~: \textcolor{darkblue}{\textbf{\ipa{pɤ˩}}} 
\lhead{\firstmark}
\rhead{\botmark}

\subsection{\hspace{-0.5cm} {\Large \textcolor{darkblue}{\textbf{\ipa{tɕi˩qɑ˥}}}}\hspace{0.5cm}[\kern2pt{\textcolor{darkblue}{\textbf{\ipa{tɕi˩qɑ˧˥}}}}\kern2pt]} \hypertarget{ts£i\string_BqA\string_T1}{}
\markboth{\textcolor{darkblue}{\textbf{\ipa{tɕi˩qɑ˥}}}}{}
\textcolor{teal}{\mytextsc{nom}} \hspace{4pt} Ton~: LH.
\textcolor{PineGreen}{\selectlanguage{french}Tapis.} \zh{毯子。}  \zh{量词}: \textcolor{darkblue}{\textbf{\ipa{ɭɯ˧}}}  \mytextsc{clf}~: \textcolor{darkblue}{\textbf{\ipa{ɭɯ˧}}} 
\lhead{\firstmark}
\rhead{\botmark}

\subsection{\hspace{-0.5cm} {\Large \textcolor{darkblue}{\textbf{\ipa{tɕi˩˥}}}}\hspace{0.5cm}[\kern2pt{\textcolor{darkblue}{\textbf{\ipa{tɕi˩˥}}}}\kern2pt]} \hypertarget{ts£i\string_B\string_T1}{}
\markboth{\textcolor{darkblue}{\textbf{\ipa{tɕi˩˥}}}}{}
\textcolor{teal}{\mytextsc{nom}} \hspace{4pt} Ton~: LH.
\textcolor{PineGreen}{\selectlanguage{french}Selle.} \zh{马鞍。}  ¶ \textcolor{darkblue}{\textbf{\ipa{ʐwæ˧-tɕi˥}}} \textcolor{PineGreen}{\selectlanguage{french}selle de cheval} \zh{马鞍}  
 \zh{量词}: \textcolor{darkblue}{\textbf{\ipa{pɤ˩}}}  \mytextsc{clf}~: \textcolor{darkblue}{\textbf{\ipa{pɤ˩}}} 
\lhead{\firstmark}
\rhead{\botmark}

\subsection{\hspace{-0.5cm} {\Large \textcolor{darkblue}{\textbf{\ipa{tɕo˥}}}}\hspace{0.5cm}[\kern2pt{\textcolor{darkblue}{\textbf{\ipa{tɕo˩˥}}}}\kern2pt]} \hypertarget{ts£o\string_T1}{}
\markboth{\textcolor{darkblue}{\textbf{\ipa{tɕo˥}}}}{}
\textcolor{teal}{\mytextsc{nom}} \hspace{4pt} Ton~: H.
\textcolor{PineGreen}{\selectlanguage{french}Sens, direction.} \zh{方向。}  ¶ \textcolor{darkblue}{\textbf{\ipa{ʈʂʰɯ˧-tɕo˧}}} \textcolor{PineGreen}{\selectlanguage{french}dans cette direction-ci} \zh{这个方向,向这里}  
 ¶ \textcolor{darkblue}{\textbf{\ipa{ɖɯ˧-tɕo˥}}} \textcolor{PineGreen}{\selectlanguage{french}d'un côté, dans une direction} \zh{一边}  
 ¶ \textcolor{darkblue}{\textbf{\ipa{gɤ˩-tɕo˧}}} \textcolor{PineGreen}{\selectlanguage{french}vers le haut} \zh{向上,往上}  
 ¶ \textcolor{darkblue}{\textbf{\ipa{dv̩˩tɕo˧}}} \textcolor{PineGreen}{\selectlanguage{french}dans cette direction-là} \zh{那边}  

\lhead{\firstmark}
\rhead{\botmark}

\subsection{\hspace{-0.5cm} {\Large \textcolor{darkblue}{\textbf{\ipa{tɕo˩ɕjo˧}}}}\hspace{0.5cm}[\kern2pt{\textcolor{darkblue}{\textbf{\ipa{tɕo˩ɕjo˥}}}}\kern2pt]} \hypertarget{ts£o\string_Bs£jo\string_M1}{}
\markboth{\textcolor{darkblue}{\textbf{\ipa{tɕo˩ɕjo˧}}}}{}
\textcolor{teal}{\mytextsc{nom}} \hspace{4pt} Ton~: LM.
\textcolor{PineGreen}{\selectlanguage{french}Sifflement.} \zh{口哨。}  ¶ \textcolor{darkblue}{\textbf{\ipa{tɕo˩ɕjo˧ | ɖɯ˧-ɖʐo˩ kʰɯ˩}}} \textcolor{PineGreen}{\selectlanguage{french}siffler un air, siffler un coup} \zh{吹口哨、吹一声口哨}  

\lhead{\firstmark}
\rhead{\botmark}

\subsection{\hspace{-0.5cm} {\Large \textcolor{darkblue}{\textbf{\ipa{tɕo˩mv̩˧}}}}\hspace{0.5cm}[\kern2pt{\textcolor{darkblue}{\textbf{\ipa{tɕo˩mv̩˥}}}}\kern2pt]} \hypertarget{ts£o\string_Bmv\string_=\string_M1}{}
\markboth{\textcolor{darkblue}{\textbf{\ipa{tɕo˩mv̩˧}}}}{}
\textcolor{teal}{\mytextsc{nom}} \hspace{4pt} Ton~: LM.
\textcolor{PineGreen}{\selectlanguage{french}Femme de l'oncle maternel; constitué d'un emprunt chinois, \zh{舅} 'oncle maternel', et d'un mot na: 'femme'.} \zh{舅妈(舅:汉语借词,妈:摩梭话“女人”)。}  Emprunt~: chinois  \zh{舅}
 \zh{量词}: \textcolor{darkblue}{\textbf{\ipa{v̩˧}}}  \mytextsc{clf}~: \textcolor{darkblue}{\textbf{\ipa{v̩˧}}} 
\lhead{\firstmark}
\rhead{\botmark}

\subsection{\hspace{-0.5cm} {\Large \textcolor{darkblue}{\textbf{\ipa{tɕɯ˥}}}}\hspace{0.5cm}[\kern2pt{\textcolor{darkblue}{\textbf{\ipa{tɕɯ˥}}}}\kern2pt]} \hypertarget{ts£M\string_T1}{}
\markboth{\textcolor{darkblue}{\textbf{\ipa{tɕɯ˥}}}}{}
\textcolor{teal}{\mytextsc{verbe}} \hspace{4pt} Ton~: H.
\ding{202} \textcolor{PineGreen}{\selectlanguage{french}Poser, ranger, mettre, placer.} \zh{放置。}  ¶ \textcolor{darkblue}{\textbf{\ipa{tʰi˧-tɕɯ˥}}} \textcolor{PineGreen}{\selectlanguage{french}\mytextsc{dur}} \zh{\mytextsc{dur}}  
 ¶ \textcolor{darkblue}{\textbf{\ipa{[F5] ɖɯ˩hĩ˧ | ɖɯ˩˧ | tʰi˧-tɕɯ˥, | tɕi˩hĩ˧ | tɕi˩˧ | tʰi˧-tɕɯ˥}}} \textcolor{PineGreen}{\selectlanguage{french}mettre les grands avec les grands, les petits avec les petits} \zh{大小归类}  
\ding{203} \textcolor{PineGreen}{\selectlanguage{french}Fixer, décider (ex.: les puissances suprêmes fixent la durée de la vie humaine).} \zh{决定、定下来。}  ¶ \textcolor{darkblue}{\textbf{\ipa{le˧-ʐwɤ˩ | tʰi˧-tɕɯ˥}}} \textcolor{PineGreen}{\selectlanguage{french}fixer; arrêter; décider que} \zh{说好、决定}  
 ¶ \textcolor{darkblue}{\textbf{\ipa{le˧-ʐwɤ˩ | tʰi˧-tɕɯ˧-ɲi˥-tsɯ˩!}}} \textcolor{PineGreen}{\selectlanguage{french}C'est fixé/c'est décidé/c'est arrêté!} \zh{说好了! / 决定好了!}  

\lhead{\firstmark}
\rhead{\botmark}

\subsection{\hspace{-0.5cm} {\Large \textcolor{darkblue}{\textbf{\ipa{tɕɯ˧}}}}\hspace{0.5cm}[\kern2pt{\textcolor{darkblue}{\textbf{\ipa{tɕɯ˥}}}}\kern2pt]} \hypertarget{ts£M\string_M1}{}
\markboth{\textcolor{darkblue}{\textbf{\ipa{tɕɯ˧}}}}{}
\textcolor{teal}{\mytextsc{nom}} \hspace{4pt} Ton~: M.
\textcolor{PineGreen}{\selectlanguage{french}Nuage.} \zh{云。}  ¶ \textcolor{darkblue}{\textbf{\ipa{mv̩˧tɕɯ˥}}} \textcolor{PineGreen}{\selectlanguage{french}il y a des nuages, le temps est nuageux} \zh{天上多云}  
 ¶ \textcolor{darkblue}{\textbf{\ipa{mv̩˧ʁo˥, | tɕɯ˧!}}} \textcolor{PineGreen}{\selectlanguage{french}le ciel est nuageux!} \zh{天上有云!}  
 ¶ \textcolor{darkblue}{\textbf{\ipa{mv̩˧ʁo˥ tɕɯ˩ pʰv̩˩ |}}} \textcolor{PineGreen}{\selectlanguage{french}le ciel est nuageux!} \zh{天上有云!}  
 ¶ \textcolor{darkblue}{\textbf{\ipa{tɕɯ˧pʰv̩˩; tɕɯ˧ | pʰv̩˩tɕæ˩˥ | -gv̩˩}}} \textcolor{PineGreen}{\selectlanguage{french}nuage blanc} \zh{白云、白色的云}  
 ¶ \textcolor{darkblue}{\textbf{\ipa{mv̩˧nɑ˥-tɕɯ˩nɑ˩-ɻ̍˩!}}} \textcolor{PineGreen}{\selectlanguage{french}il fait sombre/ le ciel est très nuageux!} \zh{天很黑,有很多乌云}  
 \zh{量词}: \textcolor{darkblue}{\textbf{\ipa{kʰwɤ˥}}}  \mytextsc{clf}~: \textcolor{darkblue}{\textbf{\ipa{kʰwɤ˥}}} \textcolor{PineGreen}{\selectlanguage{french}morceaux}
\lhead{\firstmark}
\rhead{\botmark}

\subsection{\hspace{-0.5cm} {\Large \textcolor{darkblue}{\textbf{\ipa{tɕɯ˧\textsubscript{b}}}}}\hspace{0.5cm}[\kern2pt{\textcolor{darkblue}{\textbf{\ipa{tɕɯ˥}}}}\kern2pt]} \hypertarget{ts£M\string_Mb1}{}
\markboth{\textcolor{darkblue}{\textbf{\ipa{tɕɯ˧\textsubscript{b}}}}}{}
\textcolor{teal}{\mytextsc{verbe}} \hspace{4pt} Ton~: M\textsubscript{b}.
\textcolor{PineGreen}{\selectlanguage{french}Secouer (monosyllabe).} \zh{摇晃。}  ¶ \textcolor{darkblue}{\textbf{\ipa{tso˧\textasciitilde{}tso˧ tɕɯ˧}}} \textcolor{PineGreen}{\selectlanguage{french}secouer des choses} \zh{摇东西}  

\lhead{\firstmark}
\rhead{\botmark}

\subsection{\hspace{-0.5cm} {\Large \textcolor{darkblue}{\textbf{\ipa{tɕɯ˧ɭɯ˧}}}}\hspace{0.5cm}[\kern2pt{\textcolor{darkblue}{\textbf{\ipa{tɕɯ˩ɭɯ˩˥}}}}\kern2pt]} \hypertarget{ts£M\string_Ml\string_RM\string_M1}{}
\markboth{\textcolor{darkblue}{\textbf{\ipa{tɕɯ˧ɭɯ˧}}}}{}
\textcolor{teal}{\mytextsc{verbe}} \hspace{4pt} Ton~: M.
\textcolor{PineGreen}{\selectlanguage{french}Enrouler, embobiner.} \zh{缠绕。}  ¶ \textcolor{darkblue}{\textbf{\ipa{njɤ˧-ɳɯ˧ | tɕɯ˧ɭɯ˧-bi˧!}}} \textcolor{PineGreen}{\selectlanguage{french}Je me charge d'enrouler! / C'est moi qui vais enrouler!} \zh{让我来缠吧!}  

\lhead{\firstmark}
\rhead{\botmark}

\subsection{\hspace{-0.5cm} {\Large \textcolor{darkblue}{\textbf{\ipa{tɕɯ˧mi˥\$}}}}\hspace{0.5cm}[\kern2pt{\textcolor{darkblue}{\textbf{\ipa{tɕɯ˩mi˥}}}}\kern2pt]} \hypertarget{ts£M\string_Mmi\string_T\$1}{}
\markboth{\textcolor{darkblue}{\textbf{\ipa{tɕɯ˧mi˥\$}}}}{}
\textcolor{teal}{\mytextsc{nom}} \hspace{4pt} Ton~: H\$.
\textcolor{PineGreen}{\selectlanguage{french}Grande balance.} \zh{大称。} 
\lhead{\firstmark}
\rhead{\botmark}

\subsection{\hspace{-0.5cm} {\Large \textcolor{darkblue}{\textbf{\ipa{tɕɯ˧pv̩˧}}}}\hspace{0.5cm}[\kern2pt{\textcolor{darkblue}{\textbf{\ipa{tɕɯ˩pv̩˥}}}}\kern2pt]} \hypertarget{ts£M\string_Mpv\string_=\string_M1}{}
\markboth{\textcolor{darkblue}{\textbf{\ipa{tɕɯ˧pv̩˧}}}}{}
\textcolor{teal}{\mytextsc{adjectif}} \hspace{4pt} Ton~: .
\textcolor{PineGreen}{\selectlanguage{french}À l'aise, peinard.} \zh{轻松快乐、舒畅。}  ¶ \textcolor{darkblue}{\textbf{\ipa{ʈʂʰɯ˧qo˧ | tɕɯ˧pv̩˧-ʂe˧\textasciitilde{}ʂe˧ | ɖɯ˧-dzi˩-zo˩-ho˩!}}} \textcolor{PineGreen}{\selectlanguage{french}assieds-toi ici, bien peinard!} \zh{在这边舒畅地坐一会吧!}  
 ¶ \textcolor{darkblue}{\textbf{\ipa{ʈʂʰɯ˧qo˧ | tɕɯ˧pv̩˧-ʂe˧\textasciitilde{}ʂe˧-zo˥ | ɖɯ˧-dzi˩-bi˩-ɻ̍˩!}}} \textcolor{PineGreen}{\selectlanguage{french}asseyons-nous donc ici, bien peinards!} \zh{在这边舒畅地坐一会吧!}  

\lhead{\firstmark}
\rhead{\botmark}

\subsection{\hspace{-0.5cm} {\Large \textcolor{darkblue}{\textbf{\ipa{tɕɯ˧sɯ˧˥}}}}\hspace{0.5cm}[\kern2pt{\textcolor{darkblue}{\textbf{\ipa{tɕɯ˧sɯ˧˥}}}}\kern2pt]} \hypertarget{ts£M\string_MsM\string_M\string_T1}{}
\markboth{\textcolor{darkblue}{\textbf{\ipa{tɕɯ˧sɯ˧˥}}}}{}
\textcolor{teal}{\mytextsc{nom}} \hspace{4pt} Ton~: MH\#.
\textcolor{PineGreen}{\selectlanguage{french}Brume.} \zh{雾。}  ¶ \textcolor{darkblue}{\textbf{\ipa{tɕɯ˧sɯ˧mv̩˥}}} \textcolor{PineGreen}{\selectlanguage{french}il y a de la brume} \zh{有雾}  
 \zh{量词}: \textcolor{darkblue}{\textbf{\ipa{ti˧˥}}}  \mytextsc{clf}~: \textcolor{darkblue}{\textbf{\ipa{ti˧˥}}} \textcolor{PineGreen}{\selectlanguage{french}une couche de}
\lhead{\firstmark}
\rhead{\botmark}

\subsection{\hspace{-0.5cm} {\Large \textcolor{darkblue}{\textbf{\ipa{tɕɯ˧wɤ˧}}}}\hspace{0.5cm}[\kern2pt{\textcolor{darkblue}{\textbf{\ipa{tɕɯ˧wɤ˧}}}}\kern2pt]} \hypertarget{ts£M\string_Mw7\string_M1}{}
\markboth{\textcolor{darkblue}{\textbf{\ipa{tɕɯ˧wɤ˧}}}}{}
\textcolor{teal}{\mytextsc{verbe}} \hspace{4pt} Ton~: M.
\textcolor{PineGreen}{\selectlanguage{french}Se réincarner.} \zh{转生、转世。}  ¶ \textcolor{darkblue}{\textbf{\ipa{le˧-tɕɯ˧wɤ˧-ho˥!}}} \textcolor{PineGreen}{\selectlanguage{french}(La défunte / le défunt) va se réincarner!} \zh{他要转生了!}  

\lhead{\firstmark}
\rhead{\botmark}

\subsection{\hspace{-0.5cm} {\Large \textcolor{darkblue}{\textbf{\ipa{tɕɯ˧zo˥\$}}}}\hspace{0.5cm}[\kern2pt{\textcolor{darkblue}{\textbf{\ipa{tɕɯ˧zo˥}}}}\kern2pt]} \hypertarget{ts£M\string_Mzo\string_T\$1}{}
\markboth{\textcolor{darkblue}{\textbf{\ipa{tɕɯ˧zo˥\$}}}}{}
\textcolor{teal}{\mytextsc{nom}} \hspace{4pt} Ton~: H\$.
\textcolor{PineGreen}{\selectlanguage{french}Petite balance.} \zh{小称。} 
\lhead{\firstmark}
\rhead{\botmark}

\subsection{\hspace{-0.5cm} {\Large \textcolor{darkblue}{\textbf{\ipa{tɕɯ˩\textsubscript{a}}}}}\hspace{0.5cm}[\kern2pt{\textcolor{darkblue}{\textbf{\ipa{tɕɯ˥}}}}\kern2pt]} \hypertarget{ts£M\string_Ba1}{}
\markboth{\textcolor{darkblue}{\textbf{\ipa{tɕɯ˩\textsubscript{a}}}}}{}
\textcolor{teal}{\mytextsc{verbe}} \hspace{4pt} Ton~: L\textsubscript{a}.
\textcolor{PineGreen}{\selectlanguage{french}Écrire.} \zh{写。}  ¶ \textcolor{darkblue}{\textbf{\ipa{le˧-tɕɯ˩-ze˩}}} \textcolor{PineGreen}{\selectlanguage{french}\mytextsc{accomp}+\mytextsc{pfv}} \zh{写了}  
 ¶ \textcolor{darkblue}{\textbf{\ipa{tʰæ˧ɻæ˩ tɕɯ˩}}} \textcolor{PineGreen}{\selectlanguage{french}écrire quelque chose/écrire du texte/écrire un livre} \zh{写、写书}  
 ¶ \textcolor{darkblue}{\textbf{\ipa{ɖɯ˧-kʰv̩˥ | tsʰe˧-ɲi˧ ɬi˧, | njɤ˧ | tsʰe˧-ɲi˧ bæ˧ tɕɯ˩-bi˩-ʂv̩˩ɖv̩˩!}}} \textcolor{PineGreen}{\selectlanguage{french}Dans une année il y a douze mois; je voudrais transcrire douze histoires (au cours de l'année qui vient)! (contexte: en septembre 2011, Ama remarque que j'ai transcrit deux contes en deux mois; en m'offrant cet exemple, elle me souffle le projet de garder le rythme et de transcrire une histoire par mois soit douze pendant l'année qui vient)} \zh{一年有十二个月,我就想(一年之内)记十二个故事!(情景:我两个月内完成了两个故事的记录工作。发音合作人举这个例句,鼓励我坚持这种速度,一年内再记十二个故事。)}  
 ¶ \textcolor{darkblue}{\textbf{\ipa{ɖɯ˧-tɕɯ˧\textasciitilde{}tɕɯ˥-ɻ̍˩}}} \textcolor{PineGreen}{\selectlanguage{french}\mytextsc{délimitatif} \string_ \mytextsc{red} \mytextsc{inchoatif}} \zh{\mytextsc{delimitative} \string_ \mytextsc{red} \mytextsc{inceptive}}  
 ¶ \textcolor{darkblue}{\textbf{\ipa{tɕɯ˩-di˩˥}}} \textcolor{PineGreen}{\selectlanguage{french}pinceau; littéralement 'chose pour écrire'} \zh{笔。直译:‘(用来)书写的(东西)’}  
 ¶ \textcolor{darkblue}{\textbf{\ipa{tʰæ˧ɻæ˩-tɕɯ˩-di˩}}} \textcolor{PineGreen}{\selectlanguage{french}pinceau; littéralement 'chose pour écrire des livres'} \zh{笔。直译:‘(用来)写书的(东西)’}  
 ¶ \textcolor{darkblue}{\textbf{\ipa{ʈʂʰɯ˧ | tʰi˧-tɕɯ˧\textasciitilde{}tɕɯ˥ dʑo˩}}} \textcolor{PineGreen}{\selectlanguage{french}Elle/il est en train d'écrire} \zh{他正在写写东西。}  

\lhead{\firstmark}
\rhead{\botmark}

\subsection{\hspace{-0.5cm} {\Large \textcolor{darkblue}{\textbf{\ipa{tɕɯ˩lv̩˩ho˥}}}}\hspace{0.5cm}[\kern2pt{\textcolor{darkblue}{\textbf{\ipa{xxxx non-correspondance entre le nombre de morphèmes et le nombre de tons de morphèmes}}}}\kern2pt]} \hypertarget{ts£M\string_Blv\string_=\string_Bho\string_T1}{}
\markboth{\textcolor{darkblue}{\textbf{\ipa{tɕɯ˩lv̩˩ho˥}}}}{}
\textcolor{teal}{\mytextsc{nom}} \hspace{4pt} Ton~: L+H\#.
\textcolor{PineGreen}{\selectlanguage{french}Fronde (bande de tissu permettant de catapulter un objet).} \zh{弹弓。}  \zh{量词}: \textcolor{darkblue}{\textbf{\ipa{ɭɯ˧}}}  \mytextsc{clf}~: \textcolor{darkblue}{\textbf{\ipa{ɭɯ˧}}} 
\lhead{\firstmark}
\rhead{\botmark}

\subsection{\hspace{-0.5cm} {\Large \textcolor{darkblue}{\textbf{\ipa{tɕɯ˩ɭɯ˩}}}}\hspace{0.5cm}[\kern2pt{\textcolor{darkblue}{\textbf{\ipa{tɕɯ˧ɭɯ˧}}}}\kern2pt]} \hypertarget{ts£M\string_Bl\string_RM\string_B1}{}
\markboth{\textcolor{darkblue}{\textbf{\ipa{tɕɯ˩ɭɯ˩}}}}{}
\textcolor{teal}{\mytextsc{nom}} \hspace{4pt} Ton~: L.
\textcolor{PineGreen}{\selectlanguage{french}Pie grièche du Tibet, \textit{Lanius tephronotus}.} \zh{伯劳鸟。}  \zh{量词}: \textcolor{darkblue}{\textbf{\ipa{mi˩}}}  \mytextsc{clf}~: \textcolor{darkblue}{\textbf{\ipa{mi˩}}} 
\lhead{\firstmark}
\rhead{\botmark}

\subsection{\hspace{-0.5cm} {\Large \textcolor{darkblue}{\textbf{\ipa{tɕɯ˩ɭɯ˩-qʰæ˥bæ˩}}}}\hspace{0.5cm}[\kern2pt{\textcolor{darkblue}{\textbf{\ipa{xxxx non-correspondance entre le nombre de morphèmes et le nombre de tons de morphèmes}}}}\kern2pt]} \hypertarget{ts£M\string_Bl\string_RM\string_B-q\string_h\{\string_Tb\{\string_B1}{}
\markboth{\textcolor{darkblue}{\textbf{\ipa{tɕɯ˩ɭɯ˩-qʰæ˥bæ˩}}}}{}
\textcolor{teal}{\mytextsc{nom}} \hspace{4pt} Ton~: L+\#H-.
\textcolor{PineGreen}{\selectlanguage{french}Littéralement: “cuillère de l'oiseau yyyy shrike”; mettre des renvois.} \zh{终石藤。}  \zh{量词}: \textcolor{darkblue}{\textbf{\ipa{dzi˩}}}  \mytextsc{clf}~: \textcolor{darkblue}{\textbf{\ipa{dzi˩}}} 
\lhead{\firstmark}
\rhead{\botmark}

\subsection{\hspace{-0.5cm} {\Large \textcolor{darkblue}{\textbf{\ipa{tɕɯ˩mi˥}}}}\hspace{0.5cm}[\kern2pt{\textcolor{darkblue}{\textbf{\ipa{tɕɯ˩mi˥}}}}\kern2pt]} \hypertarget{ts£M\string_Bmi\string_T1}{}
\markboth{\textcolor{darkblue}{\textbf{\ipa{tɕɯ˩mi˥}}}}{}
\textcolor{teal}{\mytextsc{nom}} \hspace{4pt} Ton~: LH.
\textcolor{PineGreen}{\selectlanguage{french}Passereau de la famille des Leiothrichidae: (\textit{Leucodioptron canorum}). Le nom “hwamei” signifie «sourcils peints» en référence à la marque distinctive autour des yeux de l'oiseau.} \zh{画眉鸟。} Dialecte chinois local~:\zh{画眉鸟。} ¶ \textcolor{darkblue}{\textbf{\ipa{tɕɯ˩mi˥ | ə˧mi˧ ɲi˩!}}} \textcolor{PineGreen}{\selectlanguage{french}c'est une maman hwamei! (=une femelle)} \zh{是一个画眉鸟妈妈!(=是母的画眉鸟)}  
 ¶ \textcolor{darkblue}{\textbf{\ipa{tɕɯ˩mi˥ | zo˧ ɲi˥!}}} \textcolor{PineGreen}{\selectlanguage{french}c'est un petit hwamei! (=un enfant/bébé)} \zh{是一个小画眉鸟!}  
 ¶ \textcolor{darkblue}{\textbf{\ipa{tɕɯ˩mi˥ | pʰv̩˧ ɲi˩!}}} \textcolor{PineGreen}{\selectlanguage{french}C'est un hwamei mâle!} \zh{是公的画眉鸟!}  
 \zh{量词}: \textcolor{darkblue}{\textbf{\ipa{mi˩}}}  \mytextsc{clf}~: \textcolor{darkblue}{\textbf{\ipa{mi˩}}} 
\lhead{\firstmark}
\rhead{\botmark}

\subsection{\hspace{-0.5cm} {\Large \textcolor{darkblue}{\textbf{\ipa{tɕɯ˧˥}}} \textsubscript{1}}\hspace{0.5cm}[\kern2pt{\textcolor{darkblue}{\textbf{\ipa{tɕɯ˧˥}}}}\kern2pt]} \hypertarget{ts£M\string_M\string_T1}{}
\markboth{\textcolor{darkblue}{\textbf{\ipa{tɕɯ˧˥}}} \textsubscript{1}}{}
\textcolor{teal}{\mytextsc{verbe}} \hspace{4pt} Ton~: MH.
\textcolor{PineGreen}{\selectlanguage{french}Transporter à dos d'animaux.} \zh{驮运。}  ¶ \textcolor{darkblue}{\textbf{\ipa{ʐwæ˧ tɕɯ˩}}} \textcolor{PineGreen}{\selectlanguage{french}faire du commerce par caravanes, transporter en caravane, organiser une caravane} \zh{用马驮运、做马帮}  
 ¶ \textcolor{darkblue}{\textbf{\ipa{ʐwæ˧ʁo˧ tʰi˧-tɕɯ˧˥}}} \textcolor{PineGreen}{\selectlanguage{french}transporter à dos de cheval} \zh{用马驮运}  
 ¶ \textcolor{darkblue}{\textbf{\ipa{ʐwæ˧-tɕɯ˩-zo˩}}} \textcolor{PineGreen}{\selectlanguage{french}caravanier, personne qui va avec les caravanes} \zh{加入马帮的男人}  

\lhead{\firstmark}
\rhead{\botmark}

\subsection{\hspace{-0.5cm} {\Large \textcolor{darkblue}{\textbf{\ipa{tɕɯ˧˥}}} \textsubscript{2}}\hspace{0.5cm}[\kern2pt{\textcolor{darkblue}{\textbf{\ipa{tɕɯ˧˥}}}}\kern2pt]} \hypertarget{ts£M\string_M\string_T2}{}
\markboth{\textcolor{darkblue}{\textbf{\ipa{tɕɯ˧˥}}} \textsubscript{2}}{}
\textcolor{teal}{\mytextsc{nom}} \hspace{4pt} Ton~: MH.
\textcolor{PineGreen}{\selectlanguage{french}Sangsue.} \zh{水蛭、蚂蟥。} Dialecte chinois local~:\zh{蚂蟥。} \zh{量词}: \textcolor{darkblue}{\textbf{\ipa{mi˩}}}  \mytextsc{clf}~: \textcolor{darkblue}{\textbf{\ipa{mi˩}}} 
\lhead{\firstmark}
\rhead{\botmark}

\subsection{\hspace{-0.5cm} {\Large \textcolor{darkblue}{\textbf{\ipa{tɕɯ˧˥}}} \textsubscript{3}}\hspace{0.5cm}[\kern2pt{\textcolor{darkblue}{\textbf{\ipa{tɕɯ˧˥}}}}\kern2pt]} \hypertarget{ts£M\string_M\string_T3}{}
\markboth{\textcolor{darkblue}{\textbf{\ipa{tɕɯ˧˥}}} \textsubscript{3}}{}
\textcolor{teal}{\mytextsc{nom}} \hspace{4pt} Ton~: MH.
\textcolor{PineGreen}{\selectlanguage{french}Guêpe.} \zh{马蜂 (黄蜂)。}  ¶ \textcolor{darkblue}{\textbf{\ipa{tɕɯ˧mi˥\$}}} \textcolor{PineGreen}{\selectlanguage{french}guêpe femelle (élicité pour le propos de l'étude tonale)} \zh{母蚂蜂(人工的词)}  
 ¶ \textcolor{darkblue}{\textbf{\ipa{tɕɯ˧pʰv̩\#˥}}} \textcolor{PineGreen}{\selectlanguage{french}guêpe mâle (élicité pour le propos de l'étude tonale)} \zh{公马蜂}  
 ¶ \textcolor{darkblue}{\textbf{\ipa{tɕɯ˧zo\#˥}}} \textcolor{PineGreen}{\selectlanguage{french}bébé guêpe (élicité pour le propos de l'étude tonale)} \zh{小马蜂}  
 \zh{量词}: \textcolor{darkblue}{\textbf{\ipa{mi˩}}}  \mytextsc{clf}~: \textcolor{darkblue}{\textbf{\ipa{mi˩}}} 
\lhead{\firstmark}
\rhead{\botmark}

\subsection{\hspace{-0.5cm} {\Large \textcolor{darkblue}{\textbf{\ipa{tɕɯ˧˥}}} \textsubscript{4}}\hspace{0.5cm}[\kern2pt{\textcolor{darkblue}{\textbf{\ipa{tɕɯ˧˥}}}}\kern2pt]} \hypertarget{ts£M\string_M\string_T4}{}
\markboth{\textcolor{darkblue}{\textbf{\ipa{tɕɯ˧˥}}} \textsubscript{4}}{}
\textcolor{teal}{\mytextsc{nom}} \hspace{4pt} Ton~: MH.
\textcolor{PineGreen}{\selectlanguage{french}Balance.} \zh{称。}  \zh{量词}: \textcolor{darkblue}{\textbf{\ipa{nɑ˧}}}  \mytextsc{clf}~: \textcolor{darkblue}{\textbf{\ipa{nɑ˧}}} 
\lhead{\firstmark}
\rhead{\botmark}

\subsection{\hspace{-0.5cm} {\Large \textcolor{darkblue}{\textbf{\ipa{tɕɯ˧˥\textsubscript{a}}}} \textsubscript{1}}\hspace{0.5cm}[\kern2pt{\textcolor{darkblue}{\textbf{\ipa{tɕɯ˩˥}}}}\kern2pt]} \hypertarget{ts£M\string_M\string_Ta1}{}
\markboth{\textcolor{darkblue}{\textbf{\ipa{tɕɯ˧˥\textsubscript{a}}}} \textsubscript{1}}{}
\textcolor{teal}{\mytextsc{classificateur}} \hspace{4pt} Ton~: MH\textsubscript{a}.
\textcolor{PineGreen}{\selectlanguage{french}Classificateur des charges sur une bête de somme.} \zh{量词:驮子(一匹)。} 
\lhead{\firstmark}
\rhead{\botmark}

\subsection{\hspace{-0.5cm} {\Large \textcolor{darkblue}{\textbf{\ipa{tɕɯ˧˥\textsubscript{a}}}} \textsubscript{2}}\hspace{0.5cm}[\kern2pt{\textcolor{darkblue}{\textbf{\ipa{tɕɯ˧˥}}}}\kern2pt]} \hypertarget{ts£M\string_M\string_Ta2}{}
\markboth{\textcolor{darkblue}{\textbf{\ipa{tɕɯ˧˥\textsubscript{a}}}} \textsubscript{2}}{}
\textcolor{teal}{\mytextsc{classificateur}} \hspace{4pt} Ton~: MH\textsubscript{a}.
\textcolor{PineGreen}{\selectlanguage{french}Livre (aussi pour les liquides: pinte).} \zh{量词:斤(用于固体,也用于液体)(汉语借词)。}  Emprunt~: chinois  \zh{斤}
 ¶ \textcolor{darkblue}{\textbf{\ipa{ʐɯ˧ | ɖɯ˧-tɕɯ˧˥}}} \textcolor{PineGreen}{\selectlanguage{french}une pinte de vin} \zh{一斤酒}  

\lhead{\firstmark}
\rhead{\botmark}

\subsection{\hspace{-0.5cm} {\Large \textcolor{darkblue}{\textbf{\ipa{tɕɯ˩˥}}}}\hspace{0.5cm}[\kern2pt{\textcolor{darkblue}{\textbf{\ipa{tɕɯ˩˥}}}}\kern2pt]} \hypertarget{ts£M\string_B\string_T1}{}
\markboth{\textcolor{darkblue}{\textbf{\ipa{tɕɯ˩˥}}}}{}
\textcolor{teal}{\mytextsc{nom}} \hspace{4pt} Ton~: LH.
\textcolor{PineGreen}{\selectlanguage{french}Salive.} \zh{口水、唾、唾沫、唾液。} 
\lhead{\firstmark}
\rhead{\botmark}

\subsection{\hspace{-0.5cm} {\Large \textcolor{darkblue}{\textbf{\ipa{tɕʰɤ˧pɤ˧-mi\#˥}}}}\hspace{0.5cm}[\kern2pt{\textcolor{darkblue}{\textbf{\ipa{xxxx non-correspondance entre le nombre de morphèmes et le nombre de tons de morphèmes}}}}\kern2pt]} \hypertarget{ts£\string_h7\string_Mp7\string_M-mi\#\string_T1}{}
\markboth{\textcolor{darkblue}{\textbf{\ipa{tɕʰɤ˧pɤ˧-mi\#˥}}}}{}
\textcolor{teal}{\mytextsc{nom}} \hspace{4pt} Ton~: \#H.
\textcolor{PineGreen}{\selectlanguage{french}Nom d'une source sacrée, située dans une grotte, sur la montagne \textcolor{darkblue}{\textbf{\ipa{/nɑ˩tsʰi˩/}}}.} \zh{一处神泉。}  ¶ \textcolor{darkblue}{\textbf{\ipa{nɑ˩tsʰi˩˥ | tɕʰɤ˧pɤ˧-mi\#˥}}} \textcolor{PineGreen}{\selectlanguage{french}nom complet de la montagne où se trouve la source sacrée} \zh{神泉所在山的全称}  

\lhead{\firstmark}
\rhead{\botmark}

\subsection{\hspace{-0.5cm} {\Large \textcolor{darkblue}{\textbf{\ipa{tɕʰɤ˧ʂo\#˥}}}}\hspace{0.5cm}[\kern2pt{\textcolor{darkblue}{\textbf{\ipa{tɕʰɤ˧ʂo˧}}}}\kern2pt]} \hypertarget{ts£\string_h7\string_Ms`o\#\string_T1}{}
\markboth{\textcolor{darkblue}{\textbf{\ipa{tɕʰɤ˧ʂo\#˥}}}}{}
\textcolor{teal}{\mytextsc{nom}} \hspace{4pt} Ton~: \#H.
\textcolor{PineGreen}{\selectlanguage{french}Autel, lieu où on brûle de l'encens (dans la maison: l'autel principal est à l'étage, dans le bâtiment face à la porte de la ferme).} \zh{祭坛。}  \zh{量词}: \textcolor{darkblue}{\textbf{\ipa{nɑ˧}}}  \mytextsc{clf}~: \textcolor{darkblue}{\textbf{\ipa{nɑ˧}}} 
\lhead{\firstmark}
\rhead{\botmark}

\subsection{\hspace{-0.5cm} {\Large \textcolor{darkblue}{\textbf{\ipa{tɕʰɤ˧ti\#˥}}}}\hspace{0.5cm}[\kern2pt{\textcolor{darkblue}{\textbf{\ipa{tɕʰɤ˧ti˧}}}}\kern2pt]} \hypertarget{ts£\string_h7\string_Mti\#\string_T1}{}
\markboth{\textcolor{darkblue}{\textbf{\ipa{tɕʰɤ˧ti\#˥}}}}{}
\textcolor{teal}{\mytextsc{nom}} \hspace{4pt} Ton~: \#H.
\textcolor{PineGreen}{\selectlanguage{french}Stupa, tour.} \zh{塔。}  \zh{量词}: \textcolor{darkblue}{\textbf{\ipa{ɭɯ˧}}}  \mytextsc{clf}~: \textcolor{darkblue}{\textbf{\ipa{ɭɯ˧}}} 
\lhead{\firstmark}
\rhead{\botmark}

\subsection{\hspace{-0.5cm} {\Large \textcolor{darkblue}{\textbf{\ipa{tɕʰɤ˧tɕo˩}}}}\hspace{0.5cm}[\kern2pt{\textcolor{darkblue}{\textbf{\ipa{tɕʰɤ˧tɕo˩}}}}\kern2pt]} \hypertarget{ts£\string_h7\string_Mts£o\string_B1}{}
\markboth{\textcolor{darkblue}{\textbf{\ipa{tɕʰɤ˧tɕo˩}}}}{}
\textcolor{teal}{\mytextsc{nom}} \hspace{4pt} Ton~: L\#.
\textcolor{PineGreen}{\selectlanguage{french}Scie passe-partout: grande scie avec une poignée à chaque extrémité, maniée par deux bûcherons.} \zh{双人锯:以前用于把圆木截成板材的大的双人锯(汉语借词)。}  Emprunt~: chinois  \zh{??}

\lhead{\firstmark}
\rhead{\botmark}

\subsection{\hspace{-0.5cm} {\Large \textcolor{darkblue}{\textbf{\ipa{tɕʰɤ˧tɕʰɤ˧˥}}}}\hspace{0.5cm}[\kern2pt{\textcolor{darkblue}{\textbf{\ipa{tɕʰɤ˧tɕʰɤ˧˥}}}}\kern2pt]} \hypertarget{ts£\string_h7\string_Mts£\string_h7\string_M\string_T1}{}
\markboth{\textcolor{darkblue}{\textbf{\ipa{tɕʰɤ˧tɕʰɤ˧˥}}}}{}
\textcolor{teal}{\mytextsc{adverbe}} \hspace{4pt} Ton~: MH\#.
\textcolor{PineGreen}{\selectlanguage{french}Entièrement, tout à fait, complètement.} \zh{彻底。} 
\lhead{\firstmark}
\rhead{\botmark}

\subsection{\hspace{-0.5cm} {\Large \textcolor{darkblue}{\textbf{\ipa{tɕʰɤ˩lv̩˩}}}}\hspace{0.5cm}[\kern2pt{\textcolor{darkblue}{\textbf{\ipa{tɕʰɤ˩lv̩˩˥}}}}\kern2pt]} \hypertarget{ts£\string_h7\string_Blv\string_=\string_B1}{}
\markboth{\textcolor{darkblue}{\textbf{\ipa{tɕʰɤ˩lv̩˩}}}}{}
\textcolor{teal}{\mytextsc{nom}} \hspace{4pt} Ton~: L.
\textcolor{PineGreen}{\selectlanguage{french}Lis.} \zh{百合。}  ¶ \textcolor{darkblue}{\textbf{\ipa{tɕʰɤ˩lv̩˩-hṽ˩hṽ˩˥}}} \textcolor{PineGreen}{\selectlanguage{french}lis cuits au wok} \zh{炒百合}  
 ¶ \textcolor{darkblue}{\textbf{\ipa{tɕʰɤ˩lv̩˩˥, | kv̩˧-pʰæ˧di˥!}}} \textcolor{PineGreen}{\selectlanguage{french}Le lis, ça ressemble à de l'ail!} \zh{百合,像大蒜!}  
 ¶ \textcolor{darkblue}{\textbf{\ipa{tɕʰɤ˩lv̩˩˥, | dʑɯ˩-nɑ˩mi˩-ʁo˥ dʑɯ˩-nɑ˩mi˩-ʁo˥ di˩-kv̩˩!}}} \textcolor{PineGreen}{\selectlanguage{french}Le lis, ça pousse dans la montagne/en haute montagne!} \zh{百合长在高山上!}  
 ¶ \textcolor{darkblue}{\textbf{\ipa{tɕʰɤ˩lv̩˩˥, | dʑɯ˩-nɑ˩mi˩-ʁo˥ | di˩-kv̩˩˥! |}}} \textcolor{PineGreen}{\selectlanguage{french}Le lis, ça pousse dans la montagne/en haute montagne!} \zh{百合长在高山上!}  

\lhead{\firstmark}
\rhead{\botmark}

\subsection{\hspace{-0.5cm} {\Large \textcolor{darkblue}{\textbf{\ipa{tɕʰɤ˩ʈʂv̩˧}}}}\hspace{0.5cm}[\kern2pt{\textcolor{darkblue}{\textbf{\ipa{tɕʰɤ˩ʈʂv̩˥}}}}\kern2pt]} \hypertarget{ts£\string_h7\string_Bt`s`v\string_=\string_M1}{}
\markboth{\textcolor{darkblue}{\textbf{\ipa{tɕʰɤ˩ʈʂv̩˧}}}}{}
\textcolor{teal}{\mytextsc{nom}} \hspace{4pt} Ton~: LM.
\textcolor{PineGreen}{\selectlanguage{french}Verre pour le vin ou autres liquides (sans anse); en verre ou autre matériau.} \zh{酒杯。}  \zh{量词}: \textcolor{darkblue}{\textbf{\ipa{ɭɯ˧}}}  \mytextsc{clf}~: \textcolor{darkblue}{\textbf{\ipa{ɭɯ˧}}} 
\lhead{\firstmark}
\rhead{\botmark}

\subsection{\hspace{-0.5cm} {\Large \textcolor{darkblue}{\textbf{\ipa{tɕʰɤ˩ʈʂv˧}}}}\hspace{0.5cm}[\kern2pt{\textcolor{darkblue}{\textbf{\ipa{tɕʰɤ˩ʈʂv˥}}}}\kern2pt]} \hypertarget{ts£\string_h7\string_Bt`s`v\string_M1}{}
\markboth{\textcolor{darkblue}{\textbf{\ipa{tɕʰɤ˩ʈʂv˧}}}}{}
\textcolor{teal}{\mytextsc{nom}} \hspace{4pt} Ton~: LM.
\textcolor{PineGreen}{\selectlanguage{french}Verre, gobelet.} \zh{杯子。}  ¶ \textcolor{darkblue}{\textbf{\ipa{bo˧ʐæ˧-tɕʰɤ˩ʈʂv˩}}} \textcolor{PineGreen}{\selectlanguage{french}gobelet à thé en verre} \zh{玻璃茶杯}  

\lhead{\firstmark}
\rhead{\botmark}

\subsection{\hspace{-0.5cm} {\Large \textcolor{darkblue}{\textbf{\ipa{tɕʰɤ˧˥}}}}\hspace{0.5cm}[\kern2pt{\textcolor{darkblue}{\textbf{\ipa{tɕʰɤ˧˥}}}}\kern2pt]} \hypertarget{ts£\string_h7\string_M\string_T1}{}
\markboth{\textcolor{darkblue}{\textbf{\ipa{tɕʰɤ˧˥}}}}{}
\textcolor{teal}{\mytextsc{verbe}} \hspace{4pt} Ton~: MH.
\textcolor{PineGreen}{\selectlanguage{french}Tromper.} \zh{欺骗。}  ¶ \textcolor{darkblue}{\textbf{\ipa{le˧-tɕʰɤ˧-ze˥}}} \textcolor{PineGreen}{\selectlanguage{french}\mytextsc{accomp} \string_ \mytextsc{pfv}} \zh{欺骗了}  
 ¶ \textcolor{darkblue}{\textbf{\ipa{hĩ˧ tɕʰɤ˧(-ze˩)}}} \textcolor{PineGreen}{\selectlanguage{french}tromper les gens} \zh{骗人}  
 ¶ \textcolor{darkblue}{\textbf{\ipa{no˧ | hĩ˧ tɕʰɤ˧!}}} \textcolor{PineGreen}{\selectlanguage{french}vous trompez les gens!} \zh{你骗人!}  
 ¶ \textcolor{darkblue}{\textbf{\ipa{(hĩ˧ |) no˩ tɕʰɤ˩˥!}}} \textcolor{PineGreen}{\selectlanguage{french}les gens vous trompent!} \zh{人家骗你!}  
 ¶ \textcolor{darkblue}{\textbf{\ipa{(no˧ |) njɤ˩ tɕʰɤ˩˥!}}} \textcolor{PineGreen}{\selectlanguage{french}Vous me trompez!} \zh{你骗我!}  

\lhead{\firstmark}
\rhead{\botmark}

\subsection{\hspace{-0.5cm} {\Large \textcolor{darkblue}{\textbf{\ipa{tɕʰi˥}}}}\hspace{0.5cm}[\kern2pt{\textcolor{darkblue}{\textbf{\ipa{tɕʰi˥}}}}\kern2pt]} \hypertarget{ts£\string_hi\string_T1}{}
\markboth{\textcolor{darkblue}{\textbf{\ipa{tɕʰi˥}}}}{}
\textcolor{teal}{\mytextsc{nom}} \hspace{4pt} Ton~: \#H.
\textcolor{PineGreen}{\selectlanguage{french}Épine.} \zh{刺。}  \zh{量词}: \textcolor{darkblue}{\textbf{\ipa{kɤ˧˥}}}  \mytextsc{clf}~: \textcolor{darkblue}{\textbf{\ipa{kɤ˧˥}}} 
\lhead{\firstmark}
\rhead{\botmark}

\subsection{\hspace{-0.5cm} {\Large \textcolor{darkblue}{\textbf{\ipa{tɕʰi˧\textsubscript{b}}}} \textsubscript{1}}\hspace{0.5cm}[\kern2pt{\textcolor{darkblue}{\textbf{\ipa{tɕʰi˥}}}}\kern2pt]} \hypertarget{ts£\string_hi\string_Mb1}{}
\markboth{\textcolor{darkblue}{\textbf{\ipa{tɕʰi˧\textsubscript{b}}}} \textsubscript{1}}{}
\textcolor{teal}{\mytextsc{verbe}} \hspace{4pt} Ton~: M\textsubscript{b}.
\textcolor{PineGreen}{\selectlanguage{french}Garder, surveiller (ex.: garder la maison).} \zh{守卫。}  ¶ \textcolor{darkblue}{\textbf{\ipa{ɑ˩ʁo˧ tɕʰi˧}}} \textcolor{PineGreen}{\selectlanguage{french}garder la maison} \zh{守护家}  
 ¶ \textcolor{darkblue}{\textbf{\ipa{ɑ˩ʁo˧ tʰi˧-tɕʰi˧-dʑo˧}}} \textcolor{PineGreen}{\selectlanguage{french}en train de surveiller la maison} \zh{守着家}  
 ¶ \textcolor{darkblue}{\textbf{\ipa{tso˧\textasciitilde{}tso˧ tɕʰi˧}}} \textcolor{PineGreen}{\selectlanguage{french}surveiller des objets} \zh{守着东西}  

\lhead{\firstmark}
\rhead{\botmark}

\subsection{\hspace{-0.5cm} {\Large \textcolor{darkblue}{\textbf{\ipa{tɕʰi˧\textsubscript{b}}}} \textsubscript{2}}\hspace{0.5cm}[\kern2pt{\textcolor{darkblue}{\textbf{\ipa{tɕʰi˥}}}}\kern2pt]} \hypertarget{ts£\string_hi\string_Mb2}{}
\markboth{\textcolor{darkblue}{\textbf{\ipa{tɕʰi˧\textsubscript{b}}}} \textsubscript{2}}{}
\textcolor{teal}{\mytextsc{verbe}} \hspace{4pt} Ton~: M\textsubscript{b}.
\textcolor{PineGreen}{\selectlanguage{french}Vendre.} \zh{卖。} 
\lhead{\firstmark}
\rhead{\botmark}

\subsection{\hspace{-0.5cm} {\Large \textcolor{darkblue}{\textbf{\ipa{tɕʰi˧ɖv̩\#˥}}}}\hspace{0.5cm}[\kern2pt{\textcolor{darkblue}{\textbf{\ipa{tɕʰi˧ɖv̩˧}}}}\kern2pt]} \hypertarget{ts£\string_hi\string_Md`v\string_=\#\string_T1}{}
\markboth{\textcolor{darkblue}{\textbf{\ipa{tɕʰi˧ɖv̩\#˥}}}}{}
\textcolor{teal}{\mytextsc{nom}} \hspace{4pt} Ton~: \#H.
\textcolor{PineGreen}{\selectlanguage{french}Prénom féminin.} \zh{女性名字。} 
\lhead{\firstmark}
\rhead{\botmark}

\subsection{\hspace{-0.5cm} {\Large \textcolor{darkblue}{\textbf{\ipa{tɕʰi˧nɑ˥}}}}\hspace{0.5cm}[\kern2pt{\textcolor{darkblue}{\textbf{\ipa{tɕʰi˧nɑ˧}}}}\kern2pt]} \hypertarget{ts£\string_hi\string_MnA\string_T1}{}
\markboth{\textcolor{darkblue}{\textbf{\ipa{tɕʰi˧nɑ˥}}}}{}
\textcolor{teal}{\mytextsc{nom}} \hspace{4pt} Ton~: H\#.
\textcolor{PineGreen}{\selectlanguage{french}Prinsepia, \textit{Prinsepia utilis Royle}; végétal qui sert pour les haies, à grosses épines, petites fleurs jaunes, et tige verte vernissée. On tire de ses graines une huile de grand prix, utilisée dans des préparations alimentaires et comme cosmétique/huile de massage.} \zh{青刺果、青刺尖、阿娜斯果。}  ¶ \textcolor{darkblue}{\textbf{\ipa{tɕʰi˧nɑ˥-dzi˩}}} \textcolor{PineGreen}{\selectlanguage{french}prinsepia (la plante)} \zh{青刺尖}  
 ¶ \textcolor{darkblue}{\textbf{\ipa{tɕʰi˧nɑ˥-bæ˩bæ˩}}} \textcolor{PineGreen}{\selectlanguage{french}fleur de prinsepia} \zh{青刺果花}  

\lhead{\firstmark}
\rhead{\botmark}

\subsection{\hspace{-0.5cm} {\Large \textcolor{darkblue}{\textbf{\ipa{tɕʰi˧ʈʂʰɤ˥}}}}\hspace{0.5cm}[\kern2pt{\textcolor{darkblue}{\textbf{\ipa{xxxx non-correspondance entre le nombre de morphèmes et le nombre de tons de morphèmes}}}}\kern2pt]} \hypertarget{ts£\string_hi\string_Mt`s`\string_h7\string_T1}{}
\markboth{\textcolor{darkblue}{\textbf{\ipa{tɕʰi˧ʈʂʰɤ˥}}}}{}
\textcolor{teal}{\mytextsc{nom}} \hspace{4pt} Ton~: H\#.
\textcolor{PineGreen}{\selectlanguage{french}Voiture, automobile.} \zh{汽车(汉语借词)。}  Emprunt~: chinois  \zh{汽车}
 \zh{量词}: \textcolor{darkblue}{\textbf{\ipa{nɑ˧}}}  \mytextsc{clf}~: \textcolor{darkblue}{\textbf{\ipa{nɑ˧}}} 
\lhead{\firstmark}
\rhead{\botmark}

\subsection{\hspace{-0.5cm} {\Large \textcolor{darkblue}{\textbf{\ipa{tɕʰi˩\textsubscript{b}}}}}\hspace{0.5cm}[\kern2pt{\textcolor{darkblue}{\textbf{\ipa{tɕʰi˩˥}}}}\kern2pt]} \hypertarget{ts£\string_hi\string_Bb1}{}
\markboth{\textcolor{darkblue}{\textbf{\ipa{tɕʰi˩\textsubscript{b}}}}}{}
\textcolor{teal}{\mytextsc{classificateur}} \hspace{4pt} Ton~: L\textsubscript{b}.
\textcolor{PineGreen}{\selectlanguage{french}Classificateur des repas.} \zh{量词:饭(一顿)。}  ¶ \textcolor{darkblue}{\textbf{\ipa{ɖɯ˧-tɕʰi˩ dzɯ˩}}} \textcolor{PineGreen}{\selectlanguage{french}prendre un repas} \zh{吃一顿}  
 ¶ \textcolor{darkblue}{\textbf{\ipa{gv̩˧-tɕʰi˥}}} \textcolor{PineGreen}{\selectlanguage{french}neuf repas} \zh{就顿(饭)}  
 ¶ \textcolor{darkblue}{\textbf{\ipa{tɕʰi˩ tʰv̩˩˥}}} \textcolor{PineGreen}{\selectlanguage{french}apporter de la nourriture, apporter un repas: lorsqu'on est invité à participer à des cérémonies funéraires, on apporte à manger, pour contribuer aux repas collectifs} \zh{带饭,“出(一)顿(饭)”:被请参加守孝时,要给那家主人带上饭)}  
 ¶ \textcolor{darkblue}{\textbf{\ipa{tɕʰi˩tʰv̩˩-hĩ˥}}} \textcolor{PineGreen}{\selectlanguage{french}la personne qui se charge du repas / qui nourrit tous les participants (lors d'un repas de veillée funéraire)} \zh{给大家供饭的那个人(不一定是主人)}  

\lhead{\firstmark}
\rhead{\botmark}

\subsection{\hspace{-0.5cm} {\Large \textcolor{darkblue}{\textbf{\ipa{tɕʰi˩tsɯ˧}}}}\hspace{0.5cm}[\kern2pt{\textcolor{darkblue}{\textbf{\ipa{tɕʰi˧tsɯ˥}}}}\kern2pt]} \hypertarget{ts£\string_hi\string_BtsM\string_M1}{}
\markboth{\textcolor{darkblue}{\textbf{\ipa{tɕʰi˩tsɯ˧}}}}{}
\textcolor{teal}{\mytextsc{nom}} \hspace{4pt} Ton~: LM.
\textcolor{PineGreen}{\selectlanguage{french}Aubergine.} \zh{茄子。}  Emprunt~: chinois  \zh{茄子}

\lhead{\firstmark}
\rhead{\botmark}

\subsection{\hspace{-0.5cm} {\Large \textcolor{darkblue}{\textbf{\ipa{tɕʰo˩}}}}\hspace{0.5cm}[\kern2pt{\textcolor{darkblue}{\textbf{\ipa{tɕʰo˩˥}}}}\kern2pt]} \hypertarget{ts£\string_ho\string_B1}{}
\markboth{\textcolor{darkblue}{\textbf{\ipa{tɕʰo˩}}}}{}
\textcolor{teal}{\mytextsc{classificateur}} \hspace{4pt} Ton~: L *.
\textcolor{PineGreen}{\selectlanguage{french}Ensemble.} \zh{量词:一起。}  ¶ \textcolor{darkblue}{\textbf{\ipa{ɖɯ˧-tɕʰo˩}}} \textcolor{PineGreen}{\selectlanguage{french}ensemble} \zh{一起}  
 ¶ \textcolor{darkblue}{\textbf{\ipa{le˧-tɕʰo˥\textasciitilde{}tɕʰo˩}}} \textcolor{PineGreen}{\selectlanguage{french}même sens que ci-dessus: ensemble} \zh{同上:一起}  

\lhead{\firstmark}
\rhead{\botmark}

\subsection{\hspace{-0.5cm} {\Large \textcolor{darkblue}{\textbf{\ipa{tɕʰo˩\textsubscript{a}}}}}\hspace{0.5cm}[\kern2pt{\textcolor{darkblue}{\textbf{\ipa{tɕʰo˩˥}}}}\kern2pt]} \hypertarget{ts£\string_ho\string_Ba1}{}
\markboth{\textcolor{darkblue}{\textbf{\ipa{tɕʰo˩\textsubscript{a}}}}}{}
\textcolor{teal}{\mytextsc{verbe}} \hspace{4pt} Ton~: L\textsubscript{a}.
\textcolor{PineGreen}{\selectlanguage{french}Accompagner, suivre (quelqu'un lors d'un voyage, par exemple); aller avec.} \zh{陪伴、一起去、跟着。}  ¶ \textcolor{darkblue}{\textbf{\ipa{hĩ˧ tɕʰo˥}}} \textcolor{PineGreen}{\selectlanguage{french}suivre quelqu'un} \zh{陪伴某人}  
 ¶ \textcolor{darkblue}{\textbf{\ipa{ɖɯ˧-tɕʰo˩ tʰi˩-tɕʰo˩ |}}} \textcolor{PineGreen}{\selectlanguage{french}aller ensemble, former un ensemble: par exemple, dans la pièce principale de la maison, le thangka au-dessus du foyer et les peintures sur le buffet-autel des ancêtres forment un tout, elles vont ensemble} \zh{陪伴某人}  

\lhead{\firstmark}
\rhead{\botmark}

\subsection{\hspace{-0.5cm} {\Large \textcolor{darkblue}{\textbf{\ipa{tɕʰo˩mi\#˥}}}}\hspace{0.5cm}[\kern2pt{\textcolor{darkblue}{\textbf{\ipa{tɕʰo˩mi˥}}}}\kern2pt]} \hypertarget{ts£\string_ho\string_Bmi\#\string_T1}{}
\markboth{\textcolor{darkblue}{\textbf{\ipa{tɕʰo˩mi\#˥}}}}{}
\textcolor{teal}{\mytextsc{nom}} \hspace{4pt} Ton~: LM+\#H.
\textcolor{PineGreen}{\selectlanguage{french}Grande louche.} \zh{大瓢。}  \zh{量词}: \textcolor{darkblue}{\textbf{\ipa{nɑ˧}}}  \mytextsc{clf}~: \textcolor{darkblue}{\textbf{\ipa{nɑ˧}}} 
\lhead{\firstmark}
\rhead{\botmark}

\subsection{\hspace{-0.5cm} {\Large \textcolor{darkblue}{\textbf{\ipa{tɕʰo˩qʰwɤ˧}}}}\hspace{0.5cm}[\kern2pt{\textcolor{darkblue}{\textbf{\ipa{tɕʰo˩qʰwɤ˥}}}}\kern2pt]} \hypertarget{ts£\string_ho\string_Bq\string_hw7\string_M1}{}
\markboth{\textcolor{darkblue}{\textbf{\ipa{tɕʰo˩qʰwɤ˧}}}}{}
\textcolor{teal}{\mytextsc{nom}} \hspace{4pt} Ton~: LM.
\textcolor{PineGreen}{\selectlanguage{french}Louche utilisée pour les aliments des animaux; à la date de l'enquête, c'était un objet en aluminium, tandis que celui utilisé pour puiser l'eau, que l'on peut porter à la bouche, est en étain.} \zh{用来煮猪食的勺子。}  \zh{量词}: \textcolor{darkblue}{\textbf{\ipa{nɑ˧}}}  \mytextsc{clf}~: \textcolor{darkblue}{\textbf{\ipa{nɑ˧}}} 
\lhead{\firstmark}
\rhead{\botmark}

\subsection{\hspace{-0.5cm} {\Large \textcolor{darkblue}{\textbf{\ipa{tɕʰo˩zo\#˥}}}}\hspace{0.5cm}[\kern2pt{\textcolor{darkblue}{\textbf{\ipa{tɕʰo˩zo˥}}}}\kern2pt]} \hypertarget{ts£\string_ho\string_Bzo\#\string_T1}{}
\markboth{\textcolor{darkblue}{\textbf{\ipa{tɕʰo˩zo\#˥}}}}{}
\textcolor{teal}{\mytextsc{nom}} \hspace{4pt} Ton~: LM+\#H.
\textcolor{PineGreen}{\selectlanguage{french}Petite louche.} \zh{小瓢。}  \zh{量词}: \textcolor{darkblue}{\textbf{\ipa{nɑ˧}}}  \mytextsc{clf}~: \textcolor{darkblue}{\textbf{\ipa{nɑ˧}}} 
\lhead{\firstmark}
\rhead{\botmark}

\subsection{\hspace{-0.5cm} {\Large \textcolor{darkblue}{\textbf{\ipa{tɕʰo˧˥}}}}\hspace{0.5cm}[\kern2pt{\textcolor{darkblue}{\textbf{\ipa{tɕʰo˧˥}}}}\kern2pt]} \hypertarget{ts£\string_ho\string_M\string_T1}{}
\markboth{\textcolor{darkblue}{\textbf{\ipa{tɕʰo˧˥}}}}{}
\textcolor{teal}{\mytextsc{verbe}} \hspace{4pt} Ton~: MH.
\textcolor{PineGreen}{\selectlanguage{french}Équarrir (une grosse pièce de bois de charpente).} \zh{(将木料)砍成方形。}  ¶ \textcolor{darkblue}{\textbf{\ipa{bi˩mi˩-ɳɯ˥ | tɕʰo˧˥}}} \textcolor{PineGreen}{\selectlanguage{french}équarrir à la hache} \zh{用斧头砍成方形}  

\lhead{\firstmark}
\rhead{\botmark}

\subsection{\hspace{-0.5cm} {\Large \textcolor{darkblue}{\textbf{\ipa{tɕʰo˩˧}}}}\hspace{0.5cm}[\kern2pt{\textcolor{darkblue}{\textbf{\ipa{tɕʰo˩˥}}}}\kern2pt]} \hypertarget{ts£\string_ho\string_B\string_M1}{}
\markboth{\textcolor{darkblue}{\textbf{\ipa{tɕʰo˩˧}}}}{}
\textcolor{teal}{\mytextsc{nom}} \hspace{4pt} Ton~: LM.
\textcolor{PineGreen}{\selectlanguage{french}Louche (de grande taille: grosse louche pour puiser l'eau; tient plus d'un litre).} \zh{勺子、瓢。}  \zh{量词}: \textcolor{darkblue}{\textbf{\ipa{nɑ˧}}}  \mytextsc{clf}~: \textcolor{darkblue}{\textbf{\ipa{nɑ˧}}} 
\lhead{\firstmark}
\rhead{\botmark}

\subsection{\hspace{-0.5cm} {\Large \textcolor{darkblue}{\textbf{\ipa{tɕʰɯ˥}}}}\hspace{0.5cm}[\kern2pt{\textcolor{darkblue}{\textbf{\ipa{tɕʰɯ˧˥}}}}\kern2pt]} \hypertarget{ts£\string_hM\string_T1}{}
\markboth{\textcolor{darkblue}{\textbf{\ipa{tɕʰɯ˥}}}}{}
\textcolor{teal}{\mytextsc{verbe}} \hspace{4pt} Ton~: H.
\textcolor{PineGreen}{\selectlanguage{french}Percer, transpercer.} \zh{穿刺、 刺破。}  ¶ \textcolor{darkblue}{\textbf{\ipa{ʝi˧ ʈʂʰɯ˧-pʰo˩, | ɲi˧ tɕʰi˧-ze˩!}}} \textcolor{PineGreen}{\selectlanguage{french}Ce bœuf, on lui a percé le museau (pour y placer un anneau)!} \zh{这头牛的鼻子被穿刺(为了安一个牛鼻圈)}  

\lhead{\firstmark}
\rhead{\botmark}

\subsection{\hspace{-0.5cm} {\Large \textcolor{darkblue}{\textbf{\ipa{tɕʰɯ˧\textsubscript{a}}}} \textsubscript{1}}\hspace{0.5cm}[\kern2pt{\textcolor{darkblue}{\textbf{\ipa{tɕʰɯ˩˥}}}}\kern2pt]} \hypertarget{ts£\string_hM\string_Ma1}{}
\markboth{\textcolor{darkblue}{\textbf{\ipa{tɕʰɯ˧\textsubscript{a}}}} \textsubscript{1}}{}
\textcolor{teal}{\mytextsc{verbe}} \hspace{4pt} Ton~: M\textsubscript{a}.
\textcolor{PineGreen}{\selectlanguage{french}Lever (le bras…).} \zh{举、抬(胳膊)。}  ¶ \textcolor{darkblue}{\textbf{\ipa{lo˩qʰwɤ˥ | gɤ˩-tɕʰɯ˧}}} \textcolor{PineGreen}{\selectlanguage{french}lever le bras} \zh{举手、抬胳膊}  
 ¶ \textcolor{darkblue}{\textbf{\ipa{kʰɯ˧tsʰɤ˧˥ | gɤ˩-tɕʰɯ˧}}} \textcolor{PineGreen}{\selectlanguage{french}lever la jambe} \zh{抬脚}  
 ¶ \textcolor{darkblue}{\textbf{\ipa{gɤ˩-mɤ˧-tɕʰɯ˧}}} \textcolor{PineGreen}{\selectlanguage{french}ne pas lever} \zh{不抬起来}  

\lhead{\firstmark}
\rhead{\botmark}

\subsection{\hspace{-0.5cm} {\Large \textcolor{darkblue}{\textbf{\ipa{tɕʰɯ˧\textsubscript{a}}}} \textsubscript{2}}\hspace{0.5cm}[\kern2pt{\textcolor{darkblue}{\textbf{\ipa{tɕʰɯ˥}}}}\kern2pt]} \hypertarget{ts£\string_hM\string_Ma2}{}
\markboth{\textcolor{darkblue}{\textbf{\ipa{tɕʰɯ˧\textsubscript{a}}}} \textsubscript{2}}{}
\textcolor{teal}{\mytextsc{verbe}} \hspace{4pt} Ton~: M\textsubscript{a}.
\ding{202} \textcolor{PineGreen}{\selectlanguage{french}Garder, faire la garde.} \zh{守护。} \ding{203} \textcolor{PineGreen}{\selectlanguage{french}Veiller un défunt, lors d'une veillée funèbre.} \zh{居丧、守灵。}  ¶ \textcolor{darkblue}{\textbf{\ipa{hĩ˧ tɕʰɯ˧}}} \textcolor{PineGreen}{\selectlanguage{french}même sens: veiller un défunt} \zh{同上:守灵}  

\lhead{\firstmark}
\rhead{\botmark}

\subsection{\hspace{-0.5cm} {\Large \textcolor{darkblue}{\textbf{\ipa{tɕʰɯ˧bo˧˥}}}}\hspace{0.5cm}[\kern2pt{\textcolor{darkblue}{\textbf{\ipa{tɕʰɯ˧bo˧}}}}\kern2pt]} \hypertarget{ts£\string_hM\string_Mbo\string_M\string_T1}{}
\markboth{\textcolor{darkblue}{\textbf{\ipa{tɕʰɯ˧bo˧˥}}}}{}
\textcolor{teal}{\mytextsc{adjectif}} \hspace{4pt} Ton~: MH\#.
\textcolor{PineGreen}{\selectlanguage{french}Frais.} \zh{凉快。} 
\lhead{\firstmark}
\rhead{\botmark}

\subsection{\hspace{-0.5cm} {\Large \textcolor{darkblue}{\textbf{\ipa{tɕʰɯ˧lo\#˥}}}}\hspace{0.5cm}[\kern2pt{\textcolor{darkblue}{\textbf{\ipa{tɕʰɯ˩lo˩˥}}}}\kern2pt]} \hypertarget{ts£\string_hM\string_Mlo\#\string_T1}{}
\markboth{\textcolor{darkblue}{\textbf{\ipa{tɕʰɯ˧lo\#˥}}}}{}
\textcolor{teal}{\mytextsc{nom}} \hspace{4pt} Ton~: \#H.
\textcolor{PineGreen}{\selectlanguage{french}Grande assiette.} \zh{大盘子。}  \zh{量词}: \textcolor{darkblue}{\textbf{\ipa{ɭɯ˧}}}  \mytextsc{clf}~: \textcolor{darkblue}{\textbf{\ipa{ɭɯ˧}}} 
\lhead{\firstmark}
\rhead{\botmark}

\subsection{\hspace{-0.5cm} {\Large \textcolor{darkblue}{\textbf{\ipa{tɕʰɯ˧si˩-dʑɤ˩pv̩˩}}}}\hspace{0.5cm}[\kern2pt{\textcolor{darkblue}{\textbf{\ipa{tɕʰɯ˧si˩dʑɤ˧pv̩˧}}}}\kern2pt]} \hypertarget{ts£\string_hM\string_Msi\string_B-dz£7\string_Bpv\string_=\string_B1}{}
\markboth{\textcolor{darkblue}{\textbf{\ipa{tɕʰɯ˧si˩-dʑɤ˩pv̩˩}}}}{}
\textcolor{teal}{\mytextsc{nom}} \hspace{4pt} Ton~: L\#-.
\textcolor{PineGreen}{\selectlanguage{french}Monstre, revenant.} \zh{妖怪。}  ¶ \textcolor{darkblue}{\textbf{\ipa{no˧ | tɕʰɯ˧si˩-dʑɤ˩pv̩˩-ki˩ | le˧-hɯ˩-ɲi˩-ze˩!}}} \textcolor{PineGreen}{\selectlanguage{french}Tu es parti rejoindre les monstres! (propos tenus à un revenant qu'on enjoint de ne plus revenir hanter les vivants)} \zh{你已经到妖怪的世界那边(就恳求你不要回来了)!(对鬼说的话)}  

\lhead{\firstmark}
\rhead{\botmark}

\subsection{\hspace{-0.5cm} {\Large \textcolor{darkblue}{\textbf{\ipa{tɕʰɯ˧sɯ˥}}}}\hspace{0.5cm}[\kern2pt{\textcolor{darkblue}{\textbf{\ipa{tɕʰɯ˧sɯ˥}}}}\kern2pt]} \hypertarget{ts£\string_hM\string_MsM\string_T1}{}
\markboth{\textcolor{darkblue}{\textbf{\ipa{tɕʰɯ˧sɯ˥}}}}{}
\textcolor{teal}{\mytextsc{adjectif}} \hspace{4pt} Ton~: H\#.
\textcolor{PineGreen}{\selectlanguage{french}Triste, dans l'affliction, plongé dans le chagrin.} \zh{悲哀、伤心。} 
\lhead{\firstmark}
\rhead{\botmark}

\subsection{\hspace{-0.5cm} {\Large \textcolor{darkblue}{\textbf{\ipa{tɕʰɯ˩\textsubscript{a}}}}}\hspace{0.5cm}[\kern2pt{\textcolor{darkblue}{\textbf{\ipa{tɕʰɯ˧˥}}}}\kern2pt]} \hypertarget{ts£\string_hM\string_Ba1}{}
\markboth{\textcolor{darkblue}{\textbf{\ipa{tɕʰɯ˩\textsubscript{a}}}}}{}
\textcolor{teal}{\mytextsc{adjectif}} \hspace{4pt} Ton~: L\textsubscript{a}.
\textcolor{PineGreen}{\selectlanguage{french}Sucré.} \zh{甜。} 
\lhead{\firstmark}
\rhead{\botmark}

\subsection{\hspace{-0.5cm} {\Large \textcolor{darkblue}{\textbf{\ipa{tɕʰɯ˩di˩}}}}\hspace{0.5cm}[\kern2pt{\textcolor{darkblue}{\textbf{\ipa{tɕʰɯ˧di˧˥}}}}\kern2pt]} \hypertarget{ts£\string_hM\string_Bdi\string_B1}{}
\markboth{\textcolor{darkblue}{\textbf{\ipa{tɕʰɯ˩di˩}}}}{}
\textcolor{teal}{\mytextsc{verbe}} \hspace{4pt} Ton~: L.
\textcolor{PineGreen}{\selectlanguage{french}Chasser.} \zh{狩猎。} \textit{Voir~:} \textcolor{darkblue}{\textbf{\ipa{tɕʰɯ˩˥, di˧˥1}}} 
\lhead{\firstmark}
\rhead{\botmark}

\subsection{\hspace{-0.5cm} {\Large \textcolor{darkblue}{\textbf{\ipa{tɕʰɯ˩di˩kʰv̩˩}}}}\hspace{0.5cm}[\kern2pt{\textcolor{darkblue}{\textbf{\ipa{tɕʰɯ˩di˩kʰv̩˩˥}}}}\kern2pt]} \hypertarget{ts£\string_hM\string_Bdi\string_Bk\string_hv\string_=\string_B1}{}
\markboth{\textcolor{darkblue}{\textbf{\ipa{tɕʰɯ˩di˩kʰv̩˩}}}}{}
\textcolor{teal}{\mytextsc{nom}} \hspace{4pt} Ton~: L.
\textcolor{PineGreen}{\selectlanguage{french}Chien de chasse.} \zh{猎狗。}  ¶ \textcolor{darkblue}{\textbf{\ipa{tɕʰɯ˩di˩-kʰv̩˥mi˩}}} \textcolor{PineGreen}{\selectlanguage{french}même sens} \zh{猎狗}  
 \zh{量词}: \textcolor{darkblue}{\textbf{\ipa{mi˩}}}  \mytextsc{clf}~: \textcolor{darkblue}{\textbf{\ipa{mi˩}}} 
\lhead{\firstmark}
\rhead{\botmark}

\subsection{\hspace{-0.5cm} {\Large \textcolor{darkblue}{\textbf{\ipa{tɕʰɯ˩mi\#˥}}}}\hspace{0.5cm}[\kern2pt{\textcolor{darkblue}{\textbf{\ipa{tɕʰɯ˧mi˥}}}}\kern2pt]} \hypertarget{ts£\string_hM\string_Bmi\#\string_T1}{}
\markboth{\textcolor{darkblue}{\textbf{\ipa{tɕʰɯ˩mi\#˥}}}}{}
\textcolor{teal}{\mytextsc{nom}} \hspace{4pt} Ton~: LM+\#H / L.
\textcolor{PineGreen}{\selectlanguage{french}Muntjac femelle.} \zh{母麂子。}  \zh{量词}: \textcolor{darkblue}{\textbf{\ipa{mi˩}}}  \mytextsc{clf}~: \textcolor{darkblue}{\textbf{\ipa{mi˩}}} 
\lhead{\firstmark}
\rhead{\botmark}

\subsection{\hspace{-0.5cm} {\Large \textcolor{darkblue}{\textbf{\ipa{tɕʰɯ˩pʰv̩\#˥}}}}\hspace{0.5cm}[\kern2pt{\textcolor{darkblue}{\textbf{\ipa{tɕʰɯ˩pʰv̩˥}}}}\kern2pt]} \hypertarget{ts£\string_hM\string_Bp\string_hv\string_=\#\string_T1}{}
\markboth{\textcolor{darkblue}{\textbf{\ipa{tɕʰɯ˩pʰv̩\#˥}}}}{}
\textcolor{teal}{\mytextsc{nom}} \hspace{4pt} Ton~: LM+\#H / L.
\textcolor{PineGreen}{\selectlanguage{french}Muntjac mâle.} \zh{公麂子。}  \zh{量词}: \textcolor{darkblue}{\textbf{\ipa{mi˩}}}  \mytextsc{clf}~: \textcolor{darkblue}{\textbf{\ipa{mi˩}}} 
\lhead{\firstmark}
\rhead{\botmark}

\subsection{\hspace{-0.5cm} {\Large \textcolor{darkblue}{\textbf{\ipa{tɕʰɯ˩-ʁo˩-tɕʰɯ˥!}}}}\hspace{0.5cm}[\kern2pt{\textcolor{darkblue}{\textbf{\ipa{xxxx non-correspondance entre le nombre de morphèmes et le nombre de tons de morphèmes}}}}\kern2pt]} \hypertarget{ts£\string_hM\string_B-Ro\string_B-ts£\string_hM\string_T!1}{}
\markboth{\textcolor{darkblue}{\textbf{\ipa{tɕʰɯ˩-ʁo˩-tɕʰɯ˥!}}}}{}
\textcolor{teal}{\mytextsc{adverbe}} \hspace{4pt} Ton~: L+H\#.
\textcolor{PineGreen}{\selectlanguage{french}A vos souhaits! (formule que l'on dit lorsque quelqu'un éternue).} \zh{旁边的人打嚏喷时说的祝愿话。} 
\lhead{\firstmark}
\rhead{\botmark}

\subsection{\hspace{-0.5cm} {\Large \textcolor{darkblue}{\textbf{\ipa{tɕʰɯ˩\textasciitilde{}tɕʰɯ˧˥}}}}\hspace{0.5cm}[\kern2pt{\textcolor{darkblue}{\textbf{\ipa{tɕʰɯ˧tɕʰɯ˧˥}}}}\kern2pt]} \hypertarget{ts£\string_hM\string_B~ts£\string_hM\string_M\string_T1}{}
\markboth{\textcolor{darkblue}{\textbf{\ipa{tɕʰɯ˩\textasciitilde{}tɕʰɯ˧˥}}}}{}
\textcolor{teal}{\mytextsc{verbe}} \hspace{4pt} Ton~: MH.
\textcolor{PineGreen}{\selectlanguage{french}Sucer.} \zh{吸吮。}  ¶ \textcolor{darkblue}{\textbf{\ipa{lo˩mi˧ tɕʰi˩\textasciitilde{}tɕʰi˩}}} \textcolor{PineGreen}{\selectlanguage{french}sucer son pouce} \zh{吮拇指}  

\lhead{\firstmark}
\rhead{\botmark}

\subsection{\hspace{-0.5cm} {\Large \textcolor{darkblue}{\textbf{\ipa{tɕʰɯ˩zo\#˥}}}}\hspace{0.5cm}[\kern2pt{\textcolor{darkblue}{\textbf{\ipa{tɕʰɯ˩zo˥}}}}\kern2pt]} \hypertarget{ts£\string_hM\string_Bzo\#\string_T1}{}
\markboth{\textcolor{darkblue}{\textbf{\ipa{tɕʰɯ˩zo\#˥}}}}{}
\textcolor{teal}{\mytextsc{nom}} \hspace{4pt} Ton~: LM+\#H / L.
\textcolor{PineGreen}{\selectlanguage{french}Petit muntjac.} \zh{麂子崽子。}  \zh{量词}: \textcolor{darkblue}{\textbf{\ipa{ɭɯ˧}}}  \mytextsc{clf}~: \textcolor{darkblue}{\textbf{\ipa{ɭɯ˧}}} 
\lhead{\firstmark}
\rhead{\botmark}

\subsection{\hspace{-0.5cm} {\Large \textcolor{darkblue}{\textbf{\ipa{tɕʰɯ˧˥}}}}\hspace{0.5cm}[\kern2pt{\textcolor{darkblue}{\textbf{\ipa{tɕʰɯ˩˥}}}}\kern2pt]} \hypertarget{ts£\string_hM\string_M\string_T1}{}
\markboth{\textcolor{darkblue}{\textbf{\ipa{tɕʰɯ˧˥}}}}{}
\textcolor{teal}{\mytextsc{nom}} \hspace{4pt} Ton~: MH.
\textcolor{PineGreen}{\selectlanguage{french}Peinture, laque.} \zh{漆。}  Emprunt~: chinois  \zh{漆}
 ¶ \textcolor{darkblue}{\textbf{\ipa{tɕʰɯ˧ jɤ˥-zo˩-ho˩!}}} \textcolor{PineGreen}{\selectlanguage{french}Il va falloir (re)peindre} \zh{该刷漆了!}  

\lhead{\firstmark}
\rhead{\botmark}

\subsection{\hspace{-0.5cm} {\Large \textcolor{darkblue}{\textbf{\ipa{tɕʰɯ˧˥}}} \textsubscript{1}}\hspace{0.5cm}[\kern2pt{\textcolor{darkblue}{\textbf{\ipa{tɕʰɯ˧˥}}}}\kern2pt]} \hypertarget{ts£\string_hM\string_M\string_T1}{}
\markboth{\textcolor{darkblue}{\textbf{\ipa{tɕʰɯ˧˥}}} \textsubscript{1}}{}
\textcolor{teal}{\mytextsc{verbe}} \hspace{4pt} Ton~: MH.
\textcolor{PineGreen}{\selectlanguage{french}Jeter, se débarrasser de (poubelles, détritus…); abandonner.} \zh{扔(垃圾)。}  ¶ \textcolor{darkblue}{\textbf{\ipa{ɖæ˩˥ | tʰi˧-tɕʰɯ˧˥}}} \textcolor{PineGreen}{\selectlanguage{french}jeter des détritus} \zh{扔垃圾}  

\lhead{\firstmark}
\rhead{\botmark}

\subsection{\hspace{-0.5cm} {\Large \textcolor{darkblue}{\textbf{\ipa{tɕʰɯ˧˥}}} \textsubscript{2}}\hspace{0.5cm}[\kern2pt{\textcolor{darkblue}{\textbf{\ipa{tɕʰɯ˧˥}}}}\kern2pt]} \hypertarget{ts£\string_hM\string_M\string_T2}{}
\markboth{\textcolor{darkblue}{\textbf{\ipa{tɕʰɯ˧˥}}} \textsubscript{2}}{}
\textcolor{teal}{\mytextsc{verbe}} \hspace{4pt} Ton~: MH.
\textcolor{PineGreen}{\selectlanguage{french}Cracher.} \zh{吐(吐口水)。} 
\lhead{\firstmark}
\rhead{\botmark}

\subsection{\hspace{-0.5cm} {\Large \textcolor{darkblue}{\textbf{\ipa{tɕʰɯ˧˥}}} \textsubscript{3}}\hspace{0.5cm}[\kern2pt{\textcolor{darkblue}{\textbf{\ipa{tɕʰɯ˧˥}}}}\kern2pt]} \hypertarget{ts£\string_hM\string_M\string_T3}{}
\markboth{\textcolor{darkblue}{\textbf{\ipa{tɕʰɯ˧˥}}} \textsubscript{3}}{}
\textcolor{teal}{\mytextsc{verbe}} \hspace{4pt} Ton~: MH.
\textcolor{PineGreen}{\selectlanguage{french}Perdre.} \zh{丢失、弄丢。}  ¶ \textcolor{darkblue}{\textbf{\ipa{le˧-tɕʰɯ˧-ze˥}}} \textcolor{PineGreen}{\selectlanguage{french}\mytextsc{accomp} \string_ \mytextsc{pfv}} \zh{丢了}  
 ¶ \textcolor{darkblue}{\textbf{\ipa{le˧-tɕʰɯ˧-hɯ˥-ze˩!}}} \textcolor{PineGreen}{\selectlanguage{french}c'est perdu!} \zh{丢掉了!}  

\lhead{\firstmark}
\rhead{\botmark}

\subsection{\hspace{-0.5cm} {\Large \textcolor{darkblue}{\textbf{\ipa{tɕʰɯ˧˥}}} \textsubscript{4}}\hspace{0.5cm}[\kern2pt{\textcolor{darkblue}{\textbf{\ipa{tɕʰɯ˧˥}}}}\kern2pt]} \hypertarget{ts£\string_hM\string_M\string_T4}{}
\markboth{\textcolor{darkblue}{\textbf{\ipa{tɕʰɯ˧˥}}} \textsubscript{4}}{}
\textcolor{teal}{\mytextsc{adjectif}} \hspace{4pt} Ton~: MH.
\textcolor{PineGreen}{\selectlanguage{french}Inquiet, angoissé, tourmenté, oppressé.} \zh{担心。}  ¶ \textcolor{darkblue}{\textbf{\ipa{nv̩˩mi˩ tɕʰɯ˥}}} \textcolor{PineGreen}{\selectlanguage{french}inquiet} \zh{担心}  

\lhead{\firstmark}
\rhead{\botmark}

\subsection{\hspace{-0.5cm} {\Large \textcolor{darkblue}{\textbf{\ipa{tɕʰɯ˧˥}}} \textsubscript{5}}\hspace{0.5cm}[\kern2pt{\textcolor{darkblue}{\textbf{\ipa{tɕʰɯ˧˥}}}}\kern2pt]} \hypertarget{ts£\string_hM\string_M\string_T5}{}
\markboth{\textcolor{darkblue}{\textbf{\ipa{tɕʰɯ˧˥}}} \textsubscript{5}}{}
\textcolor{teal}{\mytextsc{adjectif}} \hspace{4pt} Ton~: MH.
\textit{\textcolor{PineGreen}{\selectlanguage{french}archaïque}} [\zh{古语}] \textcolor{PineGreen}{\selectlanguage{french}Observé seulement en tournure négative: ne pas avoir (de quoi vivre); être démuni. On peut imaginer comme sens ancien “bien doté, à l'aise”.} \zh{舒服。}  ¶ \textcolor{darkblue}{\textbf{\ipa{mɤ˧-tɕʰɯ˧-bi˥ / mɤ˧-tɕʰɯ˧˥ |-bi˩}}} \textcolor{PineGreen}{\selectlanguage{french}même si on est dans le besoin/quoi qu'on soit dans le besoin, ...} \zh{虽然很贫穷,……}  
 ¶ \textcolor{darkblue}{\textbf{\ipa{mɤ˧-dʑo˧ mɤ˧-tɕʰɯ˧-ɻ̍˧-bi˥, | ɖwæ˩ mɤ˧-zo˧!}}} \textcolor{PineGreen}{\selectlanguage{french}“Même si on est sans rien, dans le besoin, il ne faut pas s'inquiéter!” (car le Ciel vient en aide aux gens qui font de leur mieux)} \zh{虽然穷,莫担心!(因为菩萨会救好人)}  
 ¶ \textcolor{darkblue}{\textbf{\ipa{hĩ˧ ʈʂʰɯ˧-v̩˧-dʑo˩, | ɖwæ˧˥ | mɤ˧-tɕʰɯ˧˥!}}} \textcolor{PineGreen}{\selectlanguage{french}Il est vraiment dans le besoin/nécessiteux!} \zh{这个人,真的很穷!}  

\lhead{\firstmark}
\rhead{\botmark}

\subsection{\hspace{-0.5cm} {\Large \textcolor{darkblue}{\textbf{\ipa{tɕʰɯ˩˥}}}}\hspace{0.5cm}[\kern2pt{\textcolor{darkblue}{\textbf{\ipa{tɕʰɯ˥}}}}\kern2pt]} \hypertarget{ts£\string_hM\string_B\string_T1}{}
\markboth{\textcolor{darkblue}{\textbf{\ipa{tɕʰɯ˩˥}}}}{}
\textcolor{teal}{\mytextsc{nom}} \hspace{4pt} Ton~: LH.
\textcolor{PineGreen}{\selectlanguage{french}Muntjac.} \zh{麂子。}  ¶ \textcolor{darkblue}{\textbf{\ipa{tɕʰɯ˩ hwæ˧-ze˩}}} \textcolor{PineGreen}{\selectlanguage{french}...a acheté un muntjac} \zh{买麂子}  
 ¶ \textcolor{darkblue}{\textbf{\ipa{tɕʰɯ˩ dzɯ˩-ze˥}}} \textcolor{PineGreen}{\selectlanguage{french}...a mangé un muntjac} \zh{吃了麂子}  
 \zh{量词}: \textcolor{darkblue}{\textbf{\ipa{pʰo˧˥}}}  \mytextsc{clf}~: \textcolor{darkblue}{\textbf{\ipa{pʰo˧˥}}} 
\lhead{\firstmark}
\rhead{\botmark}

\newpage
\section*{\centering- \textcolor{darkblue}{\textbf{\ipa{ts}}} -}
\subsection{\hspace{-0.5cm} {\Large \textcolor{darkblue}{\textbf{\ipa{tsɑ˧}}}}\hspace{0.5cm}[\kern2pt{\textcolor{darkblue}{\textbf{\ipa{tsɑ˩˥}}}}\kern2pt]} \hypertarget{tsA\string_M1}{}
\markboth{\textcolor{darkblue}{\textbf{\ipa{tsɑ˧}}}}{}
\textcolor{teal}{\mytextsc{adjectif}} \hspace{4pt} Ton~: M.
\textcolor{PineGreen}{\selectlanguage{french}Occupé, affairé, pressé.} \zh{忙。}  ¶ \textcolor{darkblue}{\textbf{\ipa{ɖwæ˧˥ | tsɑ˧}}} \textcolor{PineGreen}{\selectlanguage{french}\mytextsc{intensif}.très: très occupé} \zh{很忙}  
 ¶ \textcolor{darkblue}{\textbf{\ipa{tsɑ˧ | ʐwæ˩˥}}} \textcolor{PineGreen}{\selectlanguage{french}extrêmement occupé} \zh{非常忙}  

\lhead{\firstmark}
\rhead{\botmark}

\subsection{\hspace{-0.5cm} {\Large \textcolor{darkblue}{\textbf{\ipa{tsɑ˧bɤ˧}}}}\hspace{0.5cm}[\kern2pt{\textcolor{darkblue}{\textbf{\ipa{tsɑ˧bɤ˧˥}}}}\kern2pt]} \hypertarget{tsA\string_Mb7\string_M1}{}
\markboth{\textcolor{darkblue}{\textbf{\ipa{tsɑ˧bɤ˧}}}}{}
\textcolor{teal}{\mytextsc{nom}} \hspace{4pt} Ton~: M.
\textcolor{PineGreen}{\selectlanguage{french}Poudre; farine.} \zh{糌粑、面粉、粉、粉末。}  Emprunt~: tibétain  rtsam pa
 ¶ \textcolor{darkblue}{\textbf{\ipa{qʰɑ˧dze˧-tsɑ˩bɤ˩}}} \textcolor{PineGreen}{\selectlanguage{french}farine de maïs} \zh{玉米粉}  
 ¶ \textcolor{darkblue}{\textbf{\ipa{dze˧ɭɯ˧-tsɑ˩bɤ˩}}} \textcolor{PineGreen}{\selectlanguage{french}farine de blé} \zh{小麦面}  
 ¶ \textcolor{darkblue}{\textbf{\ipa{lv̩˧mi˧-tsɑ˩bɤ˩}}} \textcolor{PineGreen}{\selectlanguage{french}poudre de pierre, pierre pulvérisée} \zh{石头粉、被磨成粉的石头}  
 ¶ \textcolor{darkblue}{\textbf{\ipa{tsʰi˧zi˧-tsɑ˧bɤ˥}}} \textcolor{PineGreen}{\selectlanguage{french}farine d'orge} \zh{青稞面粉}  
 ¶ \textcolor{darkblue}{\textbf{\ipa{mv̩˩zɯ˩-tsɑ˩bɤ˥}}} \textcolor{PineGreen}{\selectlanguage{french}farine d'avoine} \zh{燕麦面粉}  
 ¶ \textcolor{darkblue}{\textbf{\ipa{jɤ˩jo˧-tsɑ˧bɤ˥}}} \textcolor{PineGreen}{\selectlanguage{french}farine de pommes de terre} \zh{洋芋面粉}  
 ¶ \textcolor{darkblue}{\textbf{\ipa{nv̩˩ɭɯ˧-tsɑ˩bɤ˩}}} \textcolor{PineGreen}{\selectlanguage{french}farine de soja} \zh{黄豆面粉}  
 ¶ \textcolor{darkblue}{\textbf{\ipa{læ˧tsɯ˥-tsɑ˩bɤ˩}}} \textcolor{PineGreen}{\selectlanguage{french}piment en poudre} \zh{辣椒粉}  
 ¶ \textcolor{darkblue}{\textbf{\ipa{ʈʂʰæ˧ɣɯ˧-tsɑ˧bɤ˥}}} \textcolor{PineGreen}{\selectlanguage{french}Médicament en poudre. (Exemple: le désinfectant en poudre actuellement utilisé, de la marque \zh{云南白药}.)} \zh{药粉,粉状药品。如:“云南白药”消毒粉。}  
 ¶ \textcolor{darkblue}{\textbf{\ipa{ʂæ˩ɻ̃˩-tsɑ˩bɤ˥}}} \textcolor{PineGreen}{\selectlanguage{french}poudre d'os} \zh{骨头粉}  
 ¶ \textcolor{darkblue}{\textbf{\ipa{jɤ˧-tsɑ˧bɤ˧}}} \textcolor{PineGreen}{\selectlanguage{french}tabac en poudre} \zh{烟草粉}  
 ¶ \textcolor{darkblue}{\textbf{\ipa{jɤ˧ɻ̃˧-tsɑ˧bɤ˥}}} \textcolor{PineGreen}{\selectlanguage{french}poudre de tabac, tabac en poudre} \zh{烟草粉}  
 ¶ \textcolor{darkblue}{\textbf{\ipa{dze˩-tsɑ˩bɤ˥}}} \textcolor{PineGreen}{\selectlanguage{french}xanthoxyle en poudre} \zh{花椒粉}  
 ¶ \textcolor{darkblue}{\textbf{\ipa{dze˧-tsɑ˧bɤ˥}}} \textcolor{PineGreen}{\selectlanguage{french}sucre en poudre} \zh{砂糖}  
 ¶ \textcolor{darkblue}{\textbf{\ipa{tsɑ˧bɤ˧ mɤ˩}}} \textcolor{PineGreen}{\selectlanguage{french}manger du tsamba sec}  
 ¶ \textcolor{darkblue}{\textbf{\ipa{tsɑ˧bɤ˧ gv̩˩}}} \textcolor{PineGreen}{\selectlanguage{french}préparer du tsamba/de la farine grillée} \zh{炒糌粑,制作糌粑}  

\lhead{\firstmark}
\rhead{\botmark}

\subsection{\hspace{-0.5cm} {\Large \textcolor{darkblue}{\textbf{\ipa{tsɑ˧ʐo˩}}}}\hspace{0.5cm}[\kern2pt{\textcolor{darkblue}{\textbf{\ipa{tsɑ˧ʐo˩}}}}\kern2pt]} \hypertarget{tsA\string_Mz`o\string_B1}{}
\markboth{\textcolor{darkblue}{\textbf{\ipa{tsɑ˧ʐo˩}}}}{}
\textcolor{teal}{\mytextsc{adjectif}} \hspace{4pt} Ton~: L\#.
\textcolor{PineGreen}{\selectlanguage{french}Zélé, assidu.} \zh{勤快。} 
\lhead{\firstmark}
\rhead{\botmark}

\subsection{\hspace{-0.5cm} {\Large \textcolor{darkblue}{\textbf{\ipa{tsɑ˩}}}}\hspace{0.5cm}[\kern2pt{\textcolor{darkblue}{\textbf{\ipa{tsɑ˥}}}}\kern2pt]} \hypertarget{tsA\string_B1}{}
\markboth{\textcolor{darkblue}{\textbf{\ipa{tsɑ˩}}}}{}
\textcolor{teal}{\mytextsc{verbe}} \hspace{4pt} Ton~: L.
\textcolor{PineGreen}{\selectlanguage{french}Faire un clin d'oeil (discret signe d'intelligence).} \zh{眨眼。}  ¶ \textcolor{darkblue}{\textbf{\ipa{ʈʂʰɯ˧ | njɤ˩ɭɯ˧ tsɑ˩\textasciitilde{}tsɑ˩-dʑo˩!}}} \textcolor{PineGreen}{\selectlanguage{french}\mytextsc{red}: Elle/il est en train de faire un clin d'oeil!} \zh{\mytextsc{重叠:他在眨眨眼!}}  
 ¶ \textcolor{darkblue}{\textbf{\ipa{ʈʂʰɯ˧ | njɤ˩ɭɯ˧ tsɑ˩-dʑo˩!}}} \textcolor{PineGreen}{\selectlanguage{french}Elle/il est en train de faire un clin d'oeil!} \zh{他在眨眼!}  
 ¶ \textcolor{darkblue}{\textbf{\ipa{tsɑ˩\textasciitilde{}tsɑ˧˥}}} \textcolor{PineGreen}{\selectlanguage{french}\mytextsc{red}} \zh{\mytextsc{重叠}}  
 ¶ \textcolor{darkblue}{\textbf{\ipa{mɤ˧-tsɑ˩\textasciitilde{}tsɑ˩}}} \textcolor{PineGreen}{\selectlanguage{french}\mytextsc{neg} \mytextsc{red}} \zh{不眨眼}  
\textit{Voir~:} \hyperlink{}{\textcolor{darkblue}{\textbf{\ipa{tsɯ˩pʰɤ˩}}}} 
\lhead{\firstmark}
\rhead{\botmark}

\subsection{\hspace{-0.5cm} {\Large \textcolor{darkblue}{\textbf{\ipa{tsɑ˩tɕi˩}}}}\hspace{0.5cm}[\kern2pt{\textcolor{darkblue}{\textbf{\ipa{tsɑ˩tɕi˩˥}}}}\kern2pt]} \hypertarget{tsA\string_Bts£i\string_B1}{}
\markboth{\textcolor{darkblue}{\textbf{\ipa{tsɑ˩tɕi˩}}}}{}
\textcolor{teal}{\mytextsc{nom}} \hspace{4pt} Ton~: L.
\textcolor{PineGreen}{\selectlanguage{french}Champignons divers, champignons variés.} \zh{杂菌(汉语借词)。}  Emprunt~: chinois  \zh{杂菌}
 \zh{量词}: \textcolor{darkblue}{\textbf{\ipa{ɭɯ˧}}} \textcolor{darkblue}{\textbf{\ipa{mo˧˥}}}  \mytextsc{clf}~: \textcolor{darkblue}{\textbf{\ipa{ɭɯ˧}}} \textcolor{darkblue}{\textbf{\ipa{mo˧˥}}} 
\lhead{\firstmark}
\rhead{\botmark}

\subsection{\hspace{-0.5cm} {\Large \textcolor{darkblue}{\textbf{\ipa{tsɑ˧˥}}} \textsubscript{1}}\hspace{0.5cm}[\kern2pt{\textcolor{darkblue}{\textbf{\ipa{tsɑ˥}}}}\kern2pt]} \hypertarget{tsA\string_M\string_T1}{}
\markboth{\textcolor{darkblue}{\textbf{\ipa{tsɑ˧˥}}} \textsubscript{1}}{}
\textcolor{teal}{\mytextsc{verbe}} \hspace{4pt} Ton~: MH.
\ding{202} \textcolor{PineGreen}{\selectlanguage{french}Donner un coup de pied; briser (les mottes de terre, après le labour, avec une bêche, ou une masse en bois).} \zh{打碎(坷拉),踢(一脚)。}  ¶ \textcolor{darkblue}{\textbf{\ipa{le˧-tsɑ˧-ze˥}}} \textcolor{PineGreen}{\selectlanguage{french}\mytextsc{accomp}+\mytextsc{pfv}} \zh{\mytextsc{accomp}+\mytextsc{pfv}}  
 ¶ \textcolor{darkblue}{\textbf{\ipa{ʈʂe˧ tsɑ˩}}} \textcolor{PineGreen}{\selectlanguage{french}briser les mottes de terre après le labour (avec un instrument manuel: houe, bêche)} \zh{打碎土坷垃}  
 ¶ \textcolor{darkblue}{\textbf{\ipa{ɖɯ˧-tsɑ˧ tʰi˥-tsɑ˩}}} \textcolor{PineGreen}{\selectlanguage{french}donner une succession de coups de pied} \zh{踢了又踢}  
\ding{203} \textcolor{PineGreen}{\selectlanguage{french}Ramer (=geste comparable à celui de briser les mottes: geste répétitif, exerçant la force coup après coup, sur l'eau, comme on le ferait sur des mottes de terre).} \zh{划(船)。}  ¶ \textcolor{darkblue}{\textbf{\ipa{tsɑ˧-hɯ˥-tsɑ˩-ɻ̍˩}}} \textcolor{PineGreen}{\selectlanguage{french}ramer de façon soutenue} \zh{用力地划船、一直划船}  

\lhead{\firstmark}
\rhead{\botmark}

\subsection{\hspace{-0.5cm} {\Large \textcolor{darkblue}{\textbf{\ipa{tsɑ˧˥}}} \textsubscript{2}}\hspace{0.5cm}[\kern2pt{\textcolor{darkblue}{\textbf{\ipa{tsɑ˧˥}}}}\kern2pt]} \hypertarget{tsA\string_M\string_T2}{}
\markboth{\textcolor{darkblue}{\textbf{\ipa{tsɑ˧˥}}} \textsubscript{2}}{}
\textcolor{teal}{\mytextsc{verbe}} \hspace{4pt} Ton~: MH.
\textcolor{PineGreen}{\selectlanguage{french}Déposer, poser.} \zh{放置、放下。}  ¶ \textcolor{darkblue}{\textbf{\ipa{mv̩˩tɕo˧ tsɑ˧˥}}} \textcolor{PineGreen}{\selectlanguage{french}poser à terre} \zh{放下、放在地上}  

\lhead{\firstmark}
\rhead{\botmark}

\subsection{\hspace{-0.5cm} {\Large \textcolor{darkblue}{\textbf{\ipa{‑tsæ˧}}}}\hspace{0.5cm}[\kern2pt{\textcolor{darkblue}{\textbf{\ipa{tsæ˥}}}}\kern2pt]} \hypertarget{‑ts\{\string_M1}{}
\markboth{\textcolor{darkblue}{\textbf{\ipa{‑tsæ˧}}}}{}
\textcolor{teal}{\mytextsc{suffixe}} \hspace{4pt} Ton~: M.
\textcolor{PineGreen}{\selectlanguage{french}Causatif.} \zh{\mytextsc{使动:让。}}  ¶ \textcolor{darkblue}{\textbf{\ipa{tʰi˧-dzɯ˥-kʰɯ˩-tsæ˩-ɲi˩!}}} \textcolor{PineGreen}{\selectlanguage{french}il faut l'obliger à manger/il faut la faire manger! (Commentaire d'un membre de la famille au sujet d'une petite fille qui refuse un repas)} \zh{必须让她吃!(情景:一个小女孩拒绝吃饭,家人就说这句。)}  
 ¶ \textcolor{darkblue}{\textbf{\ipa{tʰi˧-ʐwɤ˩-kʰɯ˩-tsæ˩-ɲi˩!}}} \textcolor{PineGreen}{\selectlanguage{french}il faut le faire parler/il faut l'obliger à parler! (Variante créée par analogie avec l'exemple précédent)} \zh{必须让他说!(在以上例子的基础上编的句子)}  

\lhead{\firstmark}
\rhead{\botmark}

\subsection{\hspace{-0.5cm} {\Large \textcolor{darkblue}{\textbf{\ipa{tsæ˧qæ˥}}}}\hspace{0.5cm}[\kern2pt{\textcolor{darkblue}{\textbf{\ipa{tsæ˧qæ˥}}}}\kern2pt]} \hypertarget{ts\{\string_Mq\{\string_T1}{}
\markboth{\textcolor{darkblue}{\textbf{\ipa{tsæ˧qæ˥}}}}{}
\textcolor{teal}{\mytextsc{nom}} \hspace{4pt} Ton~: H\#.
\ding{202} \textcolor{PineGreen}{\selectlanguage{french}Crochet.} \zh{钩子。}  \zh{量词}: \textcolor{darkblue}{\textbf{\ipa{nɑ˧}}} \textcolor{darkblue}{\textbf{\ipa{ɭɯ˧}}} \ding{203} \textcolor{PineGreen}{\selectlanguage{french}Percuteur (de fusil).} \zh{撞针。}  \mytextsc{clf}~: \textcolor{darkblue}{\textbf{\ipa{nɑ˧}}} \textcolor{darkblue}{\textbf{\ipa{ɭɯ˧}}} 
\lhead{\firstmark}
\rhead{\botmark}

\subsection{\hspace{-0.5cm} {\Large \textcolor{darkblue}{\textbf{\ipa{tse˧bæ˥}}}}\hspace{0.5cm}[\kern2pt{\textcolor{darkblue}{\textbf{\ipa{tse˧bæ˥}}}}\kern2pt]} \hypertarget{tse\string_Mb\{\string_T1}{}
\markboth{\textcolor{darkblue}{\textbf{\ipa{tse˧bæ˥}}}}{}
\textcolor{teal}{\mytextsc{nom}} \hspace{4pt} Ton~: H\#.
\textcolor{PineGreen}{\selectlanguage{french}Amadou.} \zh{火绒。}  \zh{量词}: \textcolor{darkblue}{\textbf{\ipa{kʰɯ˩}}}  \mytextsc{clf}~: \textcolor{darkblue}{\textbf{\ipa{kʰɯ˩}}} 
\lhead{\firstmark}
\rhead{\botmark}

\subsection{\hspace{-0.5cm} {\Large \textcolor{darkblue}{\textbf{\ipa{tse˧bo\#˥}}}}\hspace{0.5cm}[\kern2pt{\textcolor{darkblue}{\textbf{\ipa{tse˧bo˧}}}}\kern2pt]} \hypertarget{tse\string_Mbo\#\string_T1}{}
\markboth{\textcolor{darkblue}{\textbf{\ipa{tse˧bo\#˥}}}}{}
\textcolor{teal}{\mytextsc{nom}} \hspace{4pt} Ton~: \#H.
\textcolor{PineGreen}{\selectlanguage{french}Clochette (portée par le bétail: chevaux, parfois chiens).} \zh{铃铛。}  \zh{量词}: \textcolor{darkblue}{\textbf{\ipa{ɭɯ˧}}}  \mytextsc{clf}~: \textcolor{darkblue}{\textbf{\ipa{ɭɯ˧}}} 
\lhead{\firstmark}
\rhead{\botmark}

\subsection{\hspace{-0.5cm} {\Large \textcolor{darkblue}{\textbf{\ipa{tse˧di\#˥}}}}\hspace{0.5cm}[\kern2pt{\textcolor{darkblue}{\textbf{\ipa{tse˧di˧}}}}\kern2pt]} \hypertarget{tse\string_Mdi\#\string_T1}{}
\markboth{\textcolor{darkblue}{\textbf{\ipa{tse˧di\#˥}}}}{}
\textcolor{teal}{\mytextsc{nom}} \hspace{4pt} Ton~: \#H.
\textcolor{PineGreen}{\selectlanguage{french}Bois de santal arbre à épice, arbre à encens.} \zh{檀香木、檀香、檀木。}  ¶ \textcolor{darkblue}{\textbf{\ipa{tse˧di˧-si\#˥}}} \textcolor{PineGreen}{\selectlanguage{french}même sens} \zh{同上}  

\lhead{\firstmark}
\rhead{\botmark}

\subsection{\hspace{-0.5cm} {\Large \textcolor{darkblue}{\textbf{\ipa{tse˧kʰo˩}}}}\hspace{0.5cm}[\kern2pt{\textcolor{darkblue}{\textbf{\ipa{tse˧kʰo˩}}}}\kern2pt]} \hypertarget{tse\string_Mk\string_ho\string_B1}{}
\markboth{\textcolor{darkblue}{\textbf{\ipa{tse˧kʰo˩}}}}{}
\textcolor{teal}{\mytextsc{nom}} \hspace{4pt} Ton~: L\#.
\textcolor{PineGreen}{\selectlanguage{french}Sanctuaire (petit sanctuaire sur la montagne; n'est pas habitable).} \zh{佛龛。}  \zh{量词}: \textcolor{darkblue}{\textbf{\ipa{ɭɯ˧}}}  \mytextsc{clf}~: \textcolor{darkblue}{\textbf{\ipa{ɭɯ˧}}} 
\lhead{\firstmark}
\rhead{\botmark}

\subsection{\hspace{-0.5cm} {\Large \textcolor{darkblue}{\textbf{\ipa{tse˧lv̩˥}}}}\hspace{0.5cm}[\kern2pt{\textcolor{darkblue}{\textbf{\ipa{tse˧lv̩˥}}}}\kern2pt]} \hypertarget{tse\string_Mlv\string_=\string_T1}{}
\markboth{\textcolor{darkblue}{\textbf{\ipa{tse˧lv̩˥}}}}{}
\textcolor{teal}{\mytextsc{nom}} \hspace{4pt} Ton~: H\#.
\textcolor{PineGreen}{\selectlanguage{french}Silex.} \zh{燧石。}  \zh{量词}: \textcolor{darkblue}{\textbf{\ipa{ɭɯ˧}}}  \mytextsc{clf}~: \textcolor{darkblue}{\textbf{\ipa{ɭɯ˧}}} 
\lhead{\firstmark}
\rhead{\botmark}

\subsection{\hspace{-0.5cm} {\Large \textcolor{darkblue}{\textbf{\ipa{tse˧mi˥}}}}\hspace{0.5cm}[\kern2pt{\textcolor{darkblue}{\textbf{\ipa{tse˧mi˥}}}}\kern2pt]} \hypertarget{tse\string_Mmi\string_T1}{}
\markboth{\textcolor{darkblue}{\textbf{\ipa{tse˧mi˥}}}}{}
\textcolor{teal}{\mytextsc{nom}} \hspace{4pt} Ton~: H\#.
\textcolor{PineGreen}{\selectlanguage{french}Briquet.} \zh{火镰。}  \zh{量词}: \textcolor{darkblue}{\textbf{\ipa{nɑ˧}}}  \mytextsc{clf}~: \textcolor{darkblue}{\textbf{\ipa{nɑ˧}}} 
\lhead{\firstmark}
\rhead{\botmark}

\subsection{\hspace{-0.5cm} {\Large \textcolor{darkblue}{\textbf{\ipa{tse˧mi˥-dʑɯ˩ʁo˩}}}}\hspace{0.5cm}[\kern2pt{\textcolor{darkblue}{\textbf{\ipa{tse˧mi˥dʑɯ˩ʁo˩}}}}\kern2pt]} \hypertarget{tse\string_Mmi\string_T-dz£M\string_BRo\string_B1}{}
\markboth{\textcolor{darkblue}{\textbf{\ipa{tse˧mi˥-dʑɯ˩ʁo˩}}}}{}
\textcolor{teal}{\mytextsc{nom}} \hspace{4pt} Ton~: H\#-L.
\textcolor{PineGreen}{\selectlanguage{french}Le village de Wenquan (possède des sources chaudes).} \zh{温泉乡的主要村落。} 
\lhead{\firstmark}
\rhead{\botmark}

\subsection{\hspace{-0.5cm} {\Large \textcolor{darkblue}{\textbf{\ipa{tse˩\textsubscript{a}}}} \textsubscript{1}}\hspace{0.5cm}[\kern2pt{\textcolor{darkblue}{\textbf{\ipa{tse˩˥}}}}\kern2pt]} \hypertarget{tse\string_Ba1}{}
\markboth{\textcolor{darkblue}{\textbf{\ipa{tse˩\textsubscript{a}}}} \textsubscript{1}}{}
\textcolor{teal}{\mytextsc{verbe}} \hspace{4pt} Ton~: L\textsubscript{a}.
\textcolor{PineGreen}{\selectlanguage{french}Suivre à la trace, poursuivre, pister.} \zh{追赶。}  ¶ \textcolor{darkblue}{\textbf{\ipa{hĩ˧ tse˥}}} \textcolor{PineGreen}{\selectlanguage{french}suivre quelqu'un à la trace} \zh{追赶某人}  

\lhead{\firstmark}
\rhead{\botmark}

\subsection{\hspace{-0.5cm} {\Large \textcolor{darkblue}{\textbf{\ipa{tse˩\textsubscript{a}}}} \textsubscript{2}}\hspace{0.5cm}[\kern2pt{\textcolor{darkblue}{\textbf{\ipa{tse˩˥}}}}\kern2pt]} \hypertarget{tse\string_Ba2}{}
\markboth{\textcolor{darkblue}{\textbf{\ipa{tse˩\textsubscript{a}}}} \textsubscript{2}}{}
\textcolor{teal}{\mytextsc{verbe}} \hspace{4pt} Ton~: L\textsubscript{a}.
\textcolor{PineGreen}{\selectlanguage{french}Flotter.} \zh{漂浮 (浮在水上)。}  ¶ \textcolor{darkblue}{\textbf{\ipa{gɤ˩tse˧}}} \textcolor{PineGreen}{\selectlanguage{french}même sens: flotter} \zh{同上:漂浮 (浮在水上)}  
 ¶ \textcolor{darkblue}{\textbf{\ipa{ɖɯ˧-tse˧\textasciitilde{}tse˥-ɻ̍˩}}} \textcolor{PineGreen}{\selectlanguage{french}\mytextsc{délimitatif} \string_ \mytextsc{red} \mytextsc{inchoatif}} \zh{\mytextsc{delimitative} \string_ \mytextsc{red} \mytextsc{inceptive}}  
 ¶ \textcolor{darkblue}{\textbf{\ipa{dʑɯ˩ʁo˩˥ | tʰi˧-tse˩ (-dʑo˩)}}} \textcolor{PineGreen}{\selectlanguage{french}faire flotter (dans un torrent), en montagne (ex.: des troncs qu'on ramène du lieu d'abattage jusqu'à la plaine)} \zh{让木头漂到下游}  
 ¶ \textcolor{darkblue}{\textbf{\ipa{dʑɯ˩ʁo˩ tse˧}}} \textcolor{PineGreen}{\selectlanguage{french}même sens: ramener de la montagne en faisant descendre la rivière} \zh{同上:让木头漂到下游}  

\lhead{\firstmark}
\rhead{\botmark}

\subsection{\hspace{-0.5cm} {\Large \textcolor{darkblue}{\textbf{\ipa{tse˩\textsubscript{a}}}} \textsubscript{3}}\hspace{0.5cm}[\kern2pt{\textcolor{darkblue}{\textbf{\ipa{tse˩˥}}}}\kern2pt]} \hypertarget{tse\string_Ba3}{}
\markboth{\textcolor{darkblue}{\textbf{\ipa{tse˩\textsubscript{a}}}} \textsubscript{3}}{}
\textcolor{teal}{\mytextsc{verbe}} \hspace{4pt} Ton~: L\textsubscript{a}.
\textcolor{PineGreen}{\selectlanguage{french}Fermer à clef.} \zh{锁门。}  ¶ \textcolor{darkblue}{\textbf{\ipa{kʰi˧ tse˥(-ze˩) / kʰi˧ tʰi˧-tse˩}}} \textcolor{PineGreen}{\selectlanguage{french}verrouiller la porte} \zh{锁门}  

\lhead{\firstmark}
\rhead{\botmark}

\subsection{\hspace{-0.5cm} {\Large \textcolor{darkblue}{\textbf{\ipa{tse˩pʰæ˧˥}}}}\hspace{0.5cm}[\kern2pt{\textcolor{darkblue}{\textbf{\ipa{tse˩pʰæ˧˥}}}}\kern2pt]} \hypertarget{tse\string_Bp\string_h\{\string_M\string_T1}{}
\markboth{\textcolor{darkblue}{\textbf{\ipa{tse˩pʰæ˧˥}}}}{}
\textcolor{teal}{\mytextsc{nom}} \hspace{4pt} Ton~: LM+MH\#.
\textcolor{PineGreen}{\selectlanguage{french}Pièces de l'époque impériale.} \zh{民国之前的货币。}  ¶ \textcolor{darkblue}{\textbf{\ipa{æ˧-tse˥pʰæ˩}}} \textcolor{PineGreen}{\selectlanguage{french}pièce en bronze de l'époque impériale} \zh{民国之前的铜币}  
 \zh{量词}: \textcolor{darkblue}{\textbf{\ipa{pʰæ˧˥}}}  \mytextsc{clf}~: \textcolor{darkblue}{\textbf{\ipa{pʰæ˧˥}}} 
\lhead{\firstmark}
\rhead{\botmark}

\subsection{\hspace{-0.5cm} {\Large \textcolor{darkblue}{\textbf{\ipa{tse˩qwæ˧˥}}}}\hspace{0.5cm}[\kern2pt{\textcolor{darkblue}{\textbf{\ipa{tse˩qwæ˧˥}}}}\kern2pt]} \hypertarget{tse\string_Bqw\{\string_M\string_T1}{}
\markboth{\textcolor{darkblue}{\textbf{\ipa{tse˩qwæ˧˥}}}}{}
\textcolor{teal}{\mytextsc{nom}} \hspace{4pt} Ton~: LM+MH\#.
\textcolor{PineGreen}{\selectlanguage{french}Clef.} \zh{钥匙。}  \zh{量词}: \textcolor{darkblue}{\textbf{\ipa{ɭɯ˧}}}  \mytextsc{clf}~: \textcolor{darkblue}{\textbf{\ipa{ɭɯ˧}}} 
\lhead{\firstmark}
\rhead{\botmark}

\subsection{\hspace{-0.5cm} {\Large \textcolor{darkblue}{\textbf{\ipa{tse˩tɑ˧˥}}}}\hspace{0.5cm}[\kern2pt{\textcolor{darkblue}{\textbf{\ipa{tse˩tɑ˧˥}}}}\kern2pt]} \hypertarget{tse\string_BtA\string_M\string_T1}{}
\markboth{\textcolor{darkblue}{\textbf{\ipa{tse˩tɑ˧˥}}}}{}
\textcolor{teal}{\mytextsc{nom}} \hspace{4pt} Ton~: LM+MH\#.
\textcolor{PineGreen}{\selectlanguage{french}Ciseaux.} \zh{剪刀。}  \zh{量词}: \textcolor{darkblue}{\textbf{\ipa{nɑ˧}}}  \mytextsc{clf}~: \textcolor{darkblue}{\textbf{\ipa{nɑ˧}}} 
\lhead{\firstmark}
\rhead{\botmark}

\subsection{\hspace{-0.5cm} {\Large \textcolor{darkblue}{\textbf{\ipa{tse˩ʈʂʰv̩˩}}}}\hspace{0.5cm}[\kern2pt{\textcolor{darkblue}{\textbf{\ipa{tse˩ʈʂʰv̩˩˥}}}}\kern2pt]} \hypertarget{tse\string_Bt`s`\string_hv\string_=\string_B1}{}
\markboth{\textcolor{darkblue}{\textbf{\ipa{tse˩ʈʂʰv̩˩}}}}{}
\textcolor{teal}{\mytextsc{nom}} \hspace{4pt} Ton~: L.
\textcolor{PineGreen}{\selectlanguage{french}Sac à puces: terme d'insulte pour un chien.} \zh{骂狗的话。} \textit{Voir~:} \hyperlink{}{\textcolor{darkblue}{\textbf{\ipa{tse˩ʈʂʰv̩˩-kʰv̩˥}}}} 
\lhead{\firstmark}
\rhead{\botmark}

\subsection{\hspace{-0.5cm} {\Large \textcolor{darkblue}{\textbf{\ipa{tse˩ʈʂʰv̩˩-kʰv̩˥}}}}\hspace{0.5cm}[\kern2pt{\textcolor{darkblue}{\textbf{\ipa{xxxx non-correspondance entre le nombre de morphèmes et le nombre de tons de morphèmes}}}}\kern2pt]} \hypertarget{tse\string_Bt`s`\string_hv\string_=\string_B-k\string_hv\string_=\string_T1}{}
\markboth{\textcolor{darkblue}{\textbf{\ipa{tse˩ʈʂʰv̩˩-kʰv̩˥}}}}{}
\textcolor{teal}{\mytextsc{nom}} \hspace{4pt} Ton~: L+H\#.
\textcolor{PineGreen}{\selectlanguage{french}Sac à puces: terme d'insulte pour un chien.} \zh{骂狗的话。}  ¶ \textcolor{darkblue}{\textbf{\ipa{tse˩ʈʂʰv̩˩-kʰv̩˧ ! | mv̩˩tɕo˧ se˥ !}}} \textcolor{PineGreen}{\selectlanguage{french}Descends, sac à puces! (Injonction adressée à un chien qui s'aventurait dans la partie haute de la salle à manger)} \zh{你这坏狗,下去!}  
\textit{Voir~:} \hyperlink{}{\textcolor{darkblue}{\textbf{\ipa{tse˩ʈʂʰv̩˩}}}} 
\lhead{\firstmark}
\rhead{\botmark}

\subsection{\hspace{-0.5cm} {\Large \textcolor{darkblue}{\textbf{\ipa{tse˩˥}}}}\hspace{0.5cm}[\kern2pt{\textcolor{darkblue}{\textbf{\ipa{tse˩˥}}}}\kern2pt]} \hypertarget{tse\string_B\string_T1}{}
\markboth{\textcolor{darkblue}{\textbf{\ipa{tse˩˥}}}}{}
\textcolor{teal}{\mytextsc{nom}} \hspace{4pt} Ton~: LH.
\textcolor{PineGreen}{\selectlanguage{french}Serrure, verrou.} \zh{锁。}  ¶ \textcolor{darkblue}{\textbf{\ipa{æ˧tse˥}}} \textcolor{PineGreen}{\selectlanguage{french}verrou en bronze} \zh{铜锁}  
 \zh{量词}: \textcolor{darkblue}{\textbf{\ipa{nɑ˧}}}  \mytextsc{clf}~: \textcolor{darkblue}{\textbf{\ipa{nɑ˧}}} 
\lhead{\firstmark}
\rhead{\botmark}

\subsection{\hspace{-0.5cm} {\Large \textcolor{darkblue}{\textbf{\ipa{tsɤ˧}}} \textsubscript{1}}\hspace{0.5cm}[\kern2pt{\textcolor{darkblue}{\textbf{\ipa{tsɤ˥}}}}\kern2pt]} \hypertarget{ts7\string_M1}{}
\markboth{\textcolor{darkblue}{\textbf{\ipa{tsɤ˧}}} \textsubscript{1}}{}
\textcolor{teal}{\mytextsc{verbe}} \hspace{4pt} Ton~: M intrans.
\textcolor{PineGreen}{\selectlanguage{french}Se transformer, créer, devenir; être.} \zh{形成,变成。}  ¶ \textcolor{darkblue}{\textbf{\ipa{sɯ˧pv̩˧-sɯ˥nɑ˩-ʈʂʰɯ˩ | ə˧dzɤ˧\textasciitilde{}dzɤ˥-zo˩ | pʰi˧li˩ tsɤ˩-ɲi˩-kv̩˩-tsɯ˩ | -mv̩˩!}}} \textcolor{PineGreen}{\selectlanguage{french}la chenille devient peu à peu papillon!} \zh{毛虫能慢慢变成蝴蝶,不是吗?}  
 ¶ \textcolor{darkblue}{\textbf{\ipa{ɖɯ˧-bæ˧ mɤ˧-tsɤ˧}}} \textcolor{PineGreen}{\selectlanguage{french}ce n'est pas la même chose, ce n'est pas pareil} \zh{有区别、不一样}  

\lhead{\firstmark}
\rhead{\botmark}

\subsection{\hspace{-0.5cm} {\Large \textcolor{darkblue}{\textbf{\ipa{tsɤ˧}}} \textsubscript{2}}\hspace{0.5cm}[\kern2pt{\textcolor{darkblue}{\textbf{\ipa{tsɤ˥}}}}\kern2pt]} \hypertarget{ts7\string_M2}{}
\markboth{\textcolor{darkblue}{\textbf{\ipa{tsɤ˧}}} \textsubscript{2}}{}
\textcolor{teal}{\mytextsc{adjectif}} \hspace{4pt} Ton~: M.
\textcolor{PineGreen}{\selectlanguage{french}Correct, qui va bien.} \zh{对,合适,成。}  ¶ \textcolor{darkblue}{\textbf{\ipa{(le˧-)tsɤ˧-ze˧!}}} \textcolor{PineGreen}{\selectlanguage{french}c'est bon, c'est arrangé!} \zh{好了! / 弄好了! / 成!}  
 ¶ \textcolor{darkblue}{\textbf{\ipa{tsɤ˧-ʝi˧!}}} \textcolor{PineGreen}{\selectlanguage{french}OK! C'est bon! (formule très courante, pour indiquer son acquiescement à une instruction reçue)} \zh{行! / 好的!(表示同意或接受命令)}  
 ¶ \textcolor{darkblue}{\textbf{\ipa{tsɤ˧ ɲi˥!}}} \textcolor{PineGreen}{\selectlanguage{french}c'est bon!} \zh{好的!}  
 ¶ \textcolor{darkblue}{\textbf{\ipa{no˧ | mɤ˧-bi˧ mɤ˧-tsɤ˧!}}} \textcolor{PineGreen}{\selectlanguage{french}Tu ne peux pas ne pas y aller!, littéralement “que tu n'y ailles pas, ça ne va pas!”} \zh{你如果不去,就不对! =你不能不去!}  
 ¶ \textcolor{darkblue}{\textbf{\ipa{ʈʂʰɯ˧ | ɖɯ˧-pi˧˥ | mɤ˧-tsɤ˧!}}} \textcolor{PineGreen}{\selectlanguage{french}Lui, il est pas très net! / Y'a quelque chose qui va pas chez lui!} \zh{他有一点不对劲吧!}  
 ¶ \textcolor{darkblue}{\textbf{\ipa{tsɤ˧ mɤ˧-ʝi˧-ze˧!}}} \textcolor{PineGreen}{\selectlanguage{french}Ca ne va plus!} \zh{不好了!/不行了!}  
 ¶ \textcolor{darkblue}{\textbf{\ipa{hĩ˧-ɳɯ˩ | le˧-so˩, | tsɤ˧!}}} \textcolor{PineGreen}{\selectlanguage{french}Quand on t'apprend quelque chose, c'est une chance à saisir ! / Quand il se trouve quelqu'un qui est disposé à t'apprendre quelque chose, c'est une chance à saisir! / Si tu écoutes les bons conseils, tout ira bien! (Contexte: on évoque quelqu'un qui n'est pas enclin à écouter les bons conseils: qui se braque quand on lui fournit d'utiles conseils.)} \zh{人家教,是好事! / 人家教,是要珍惜的! / 有人愿意教你,是件好事!}  
 ¶ \textcolor{darkblue}{\textbf{\ipa{hĩ˧-ɳɯ˩ | le˧-so˩, | tsɤ˧-kv˧˥!}}} \textcolor{PineGreen}{\selectlanguage{french}même sens} \zh{同上}  

\lhead{\firstmark}
\rhead{\botmark}

\subsection{\hspace{-0.5cm} {\Large \textcolor{darkblue}{\textbf{\ipa{tsɤ˧}}} \textsubscript{3}}\hspace{0.5cm}[\kern2pt{\textcolor{darkblue}{\textbf{\ipa{tsɤ˥}}}}\kern2pt]} \hypertarget{ts7\string_M3}{}
\markboth{\textcolor{darkblue}{\textbf{\ipa{tsɤ˧}}} \textsubscript{3}}{}
\textcolor{teal}{\mytextsc{adjectif}} \hspace{4pt} Ton~: M.
\textcolor{PineGreen}{\selectlanguage{french}Fine (poudre).} \zh{细(粉状)。}  ¶ \textcolor{darkblue}{\textbf{\ipa{tsɑ˧bɤ˧ tsɤ\#˥}}} \textcolor{PineGreen}{\selectlanguage{french}farine fine} \zh{细粮}  

\lhead{\firstmark}
\rhead{\botmark}

\subsection{\hspace{-0.5cm} {\Large \textcolor{darkblue}{\textbf{\ipa{tsɤ˧}}} \textsubscript{4}}\hspace{0.5cm}[\kern2pt{\textcolor{darkblue}{\textbf{\ipa{tsɤ˥}}}}\kern2pt]} \hypertarget{ts7\string_M4}{}
\markboth{\textcolor{darkblue}{\textbf{\ipa{tsɤ˧}}} \textsubscript{4}}{}
\textcolor{teal}{\mytextsc{adjectif}} \hspace{4pt} Ton~: M.
\textcolor{PineGreen}{\selectlanguage{french}Gourmand.} \zh{嘴馋。} \textit{Voir~:} \hyperlink{}{\textcolor{darkblue}{\textbf{\ipa{tsɤ˧ʁo˧-tsʰi˧ʁo˥}}}} 
\lhead{\firstmark}
\rhead{\botmark}

\subsection{\hspace{-0.5cm} {\Large \textcolor{darkblue}{\textbf{\ipa{tsɤ˧di˧}}}}\hspace{0.5cm}[\kern2pt{\textcolor{darkblue}{\textbf{\ipa{tsɤ˩di˩˥}}}}\kern2pt]} \hypertarget{ts7\string_Mdi\string_M1}{}
\markboth{\textcolor{darkblue}{\textbf{\ipa{tsɤ˧di˧}}}}{}
\textcolor{teal}{\mytextsc{nom}} \hspace{4pt} Ton~: M.
\textcolor{PineGreen}{\selectlanguage{french}Arbre à épice, arbre à encens de grande taille.} \zh{香木。} Dialecte chinois local~:\zh{柏香。} ¶ \textcolor{darkblue}{\textbf{\ipa{tsɤ˧di˧-dzi˩}}} \textcolor{PineGreen}{\selectlanguage{french}même sens} \zh{同上}  
\textit{Voir~:} \textcolor{darkblue}{\textbf{\ipa{ʁo˩kʰv˩}}} 
\lhead{\firstmark}
\rhead{\botmark}

\subsection{\hspace{-0.5cm} {\Large \textcolor{darkblue}{\textbf{\ipa{tsɤ˧ɖɯ˧}}}}\hspace{0.5cm}[\kern2pt{\textcolor{darkblue}{\textbf{\ipa{tsɤ˧ɖɯ˩}}}}\kern2pt]} \hypertarget{ts7\string_Md`M\string_M1}{}
\markboth{\textcolor{darkblue}{\textbf{\ipa{tsɤ˧ɖɯ˧}}}}{}
\textcolor{teal}{\mytextsc{verbe}} \hspace{4pt} Ton~: .
\textcolor{PineGreen}{\selectlanguage{french}Mettre bas (bovidé).} \zh{生崽子(牛类)。}  ¶ \textcolor{darkblue}{\textbf{\ipa{tsɤ˧ɖɯ˧-ze˩}}} \textcolor{PineGreen}{\selectlanguage{french}\mytextsc{pfv}} \zh{生崽子了}  
 ¶ \textcolor{darkblue}{\textbf{\ipa{(dʑi˧mi˧) tsɤ˧ɖɯ˧-ze˩}}} \textcolor{PineGreen}{\selectlanguage{french}(le buffle) a enfanté.} \zh{水牛生崽子了。}  

\lhead{\firstmark}
\rhead{\botmark}

\subsection{\hspace{-0.5cm} {\Large \textcolor{darkblue}{\textbf{\ipa{tsɤ˧ʁo˧-tsʰi˧ʁo˥}}}}\hspace{0.5cm}[\kern2pt{\textcolor{darkblue}{\textbf{\ipa{xxxx non-correspondance entre le nombre de morphèmes et le nombre de tons de morphèmes}}}}\kern2pt]} \hypertarget{ts7\string_MRo\string_M-ts\string_hi\string_MRo\string_T1}{}
\markboth{\textcolor{darkblue}{\textbf{\ipa{tsɤ˧ʁo˧-tsʰi˧ʁo˥}}}}{}
\textcolor{teal}{\mytextsc{adjectif}} \hspace{4pt} Ton~: H\#.
\textcolor{PineGreen}{\selectlanguage{french}Gourmand.} \zh{馋。}  ¶ \textcolor{darkblue}{\textbf{\ipa{tsɤ˧ʁo˧-tsʰi˧ʁo˥ tsʰi˩}}} \textcolor{PineGreen}{\selectlanguage{french}être gourmand} \zh{馋}  
\textit{Voir~:} \hyperlink{}{\textcolor{darkblue}{\textbf{\ipa{tsɤ˧}}} \textsubscript{4}} 
\lhead{\firstmark}
\rhead{\botmark}

\subsection{\hspace{-0.5cm} {\Large \textcolor{darkblue}{\textbf{\ipa{tsi˥}}}}\hspace{0.5cm}[\kern2pt{\textcolor{darkblue}{\textbf{\ipa{tsi˥}}}}\kern2pt]} \hypertarget{tsi\string_T1}{}
\markboth{\textcolor{darkblue}{\textbf{\ipa{tsi˥}}}}{}
\textcolor{teal}{\mytextsc{nom}} \hspace{4pt} Ton~: \#H.
\ding{202} \textcolor{PineGreen}{\selectlanguage{french}Fissure, interstice.} \zh{裂缝、缝隙。}  ¶ \textcolor{darkblue}{\textbf{\ipa{tsi˧ qʰwæ˧-ze˥!}}} \textcolor{PineGreen}{\selectlanguage{french}il y a une fissure qui s'est faite! / ça s'est fendu! to crack; to develop a chink/crack/fissure} \zh{有了裂缝!}  
 ¶ \textcolor{darkblue}{\textbf{\ipa{tsi˧ hɯ˧-ze˧!}}} \textcolor{PineGreen}{\selectlanguage{french}ça s'est fissuré!} \zh{有了裂缝!}  
 \zh{量词}: \textcolor{darkblue}{\textbf{\ipa{pʰæ˧˥}}} \ding{203} \textcolor{PineGreen}{\selectlanguage{french}Couture.} \zh{针脚。}  \mytextsc{clf}~: \textcolor{darkblue}{\textbf{\ipa{pʰæ˧˥}}} 
\lhead{\firstmark}
\rhead{\botmark}

\subsection{\hspace{-0.5cm} {\Large \textcolor{darkblue}{\textbf{\ipa{tsi˧\textsubscript{a}}}}}\hspace{0.5cm}[\kern2pt{\textcolor{darkblue}{\textbf{\ipa{tsi˥}}}}\kern2pt]} \hypertarget{tsi\string_Ma1}{}
\markboth{\textcolor{darkblue}{\textbf{\ipa{tsi˧\textsubscript{a}}}}}{}
\textcolor{teal}{\mytextsc{adjectif}} \hspace{4pt} Ton~: M\textsubscript{a}.
\textcolor{PineGreen}{\selectlanguage{french}Piquant, pimenté.} \zh{辣。}  ¶ \textcolor{darkblue}{\textbf{\ipa{ʈʂʰɯ˧ tsi˧-hĩ˧ ɲi˥!}}} \textcolor{PineGreen}{\selectlanguage{french}c'est piquant/c'est pimenté!} \zh{这是辣的!}  

\lhead{\firstmark}
\rhead{\botmark}

\subsection{\hspace{-0.5cm} {\Large \textcolor{darkblue}{\textbf{\ipa{tsi˧\textsubscript{b}}}}}\hspace{0.5cm}[\kern2pt{\textcolor{darkblue}{\textbf{\ipa{tsi˥}}}}\kern2pt]} \hypertarget{tsi\string_Mb1}{}
\markboth{\textcolor{darkblue}{\textbf{\ipa{tsi˧\textsubscript{b}}}}}{}
\textcolor{teal}{\mytextsc{verbe}} \hspace{4pt} Ton~: M\textsubscript{b}.
\textcolor{PineGreen}{\selectlanguage{french}Fixer, installer, mettre en place (ex.: un pilier, dans une maison en construction).} \zh{安装。}  ¶ \textcolor{darkblue}{\textbf{\ipa{tso˧\textasciitilde{}tso˧ tsi˧}}} \textcolor{PineGreen}{\selectlanguage{french}installer quelque chose} \zh{安装东西}  
 ¶ \textcolor{darkblue}{\textbf{\ipa{tso˧\textasciitilde{}tso˧ | gɤ˩-tsi˧-ɻ̍˥}}} \textcolor{PineGreen}{\selectlanguage{french}installer quelque chose, mettre quelque chose en place} \zh{安装东西}  
 ¶ \textcolor{darkblue}{\textbf{\ipa{gɤ˩-tsi˧ tʰi˧-tɕɯ˥}}} \textcolor{PineGreen}{\selectlanguage{french}(re)lever, (re)mettre d'aplomb} \zh{立起来}  

\lhead{\firstmark}
\rhead{\botmark}

\subsection{\hspace{-0.5cm} {\Large \textcolor{darkblue}{\textbf{\ipa{tsi˧gi˥\$}}}}\hspace{0.5cm}[\kern2pt{\textcolor{darkblue}{\textbf{\ipa{tsi˧gi˥}}}}\kern2pt]} \hypertarget{tsi\string_Mgi\string_T\$1}{}
\markboth{\textcolor{darkblue}{\textbf{\ipa{tsi˧gi˥\$}}}}{}
\textcolor{teal}{\mytextsc{nom}} \hspace{4pt} Ton~: H\$.
\textcolor{PineGreen}{\selectlanguage{french}Fissure.} \zh{缝隙,例如:墙上的。}  ¶ \textcolor{darkblue}{\textbf{\ipa{tsi˧gi˥ | ɖɯ˧-kʰwɤ˥}}} \textcolor{PineGreen}{\selectlanguage{french}une fissure} \zh{一个缝隙}  
 ¶ \textcolor{darkblue}{\textbf{\ipa{tsi˧gi˥ | ɖɯ˧-kʰwɤ˧ tʰi˧-di˥}}} \textcolor{PineGreen}{\selectlanguage{french}il y a une fissure} \zh{有一个缝隙}  
 \zh{量词}: \textcolor{darkblue}{\textbf{\ipa{ɭɯ˧, kʰwɤ˥}}}  \mytextsc{clf}~: \textcolor{darkblue}{\textbf{\ipa{ɭɯ˧, kʰwɤ˥}}} 
\lhead{\firstmark}
\rhead{\botmark}

\subsection{\hspace{-0.5cm} {\Large \textcolor{darkblue}{\textbf{\ipa{tsi˩\textsubscript{a}}}}}\hspace{0.5cm}[\kern2pt{\textcolor{darkblue}{\textbf{\ipa{tsi˩˥}}}}\kern2pt]} \hypertarget{tsi\string_Ba1}{}
\markboth{\textcolor{darkblue}{\textbf{\ipa{tsi˩\textsubscript{a}}}}}{}
\textcolor{teal}{\mytextsc{verbe}} \hspace{4pt} Ton~: L\textsubscript{a}.
\textcolor{PineGreen}{\selectlanguage{french}Faire bouillir.} \zh{烧开。}  ¶ \textcolor{darkblue}{\textbf{\ipa{dʑɯ˧ | le˧-tsi˩-tʰv̩˩-ze˩!}}} \textcolor{PineGreen}{\selectlanguage{french}L'eau bout!} \zh{水开了!}  
 ¶ \textcolor{darkblue}{\textbf{\ipa{dʑɯ˩ tsi˩-tʰv̩˩-ze˥!}}} \textcolor{PineGreen}{\selectlanguage{french}L'eau bout!} \zh{水开了!}  
 ¶ \textcolor{darkblue}{\textbf{\ipa{mɤ˧-tsi˩-tʰv̩˩-sɯ˩!}}} \textcolor{PineGreen}{\selectlanguage{french}Ca ne bout pas encore!} \zh{还没有烧开!}  
 ¶ \textcolor{darkblue}{\textbf{\ipa{ɖɯ˧-tsi˩-tʰv̩˩-ɻ̍˩-kʰɯ˩}}} \textcolor{PineGreen}{\selectlanguage{french}laisser bouillir un moment} \zh{煮一会儿}  
 ¶ \textcolor{darkblue}{\textbf{\ipa{ɖɯ˧-tsi˧\textasciitilde{}tsi˥-ɻ̍˩ kʰɯ˩}}} \textcolor{PineGreen}{\selectlanguage{french}faire bouillir un moment} \zh{煮一会儿}  

\lhead{\firstmark}
\rhead{\botmark}

\subsection{\hspace{-0.5cm} {\Large \textcolor{darkblue}{\textbf{\ipa{tsi˩ɭɯ˩}}}}\hspace{0.5cm}[\kern2pt{\textcolor{darkblue}{\textbf{\ipa{tsi˩ɭɯ˩˥}}}}\kern2pt]} \hypertarget{tsi\string_Bl\string_RM\string_B1}{}
\markboth{\textcolor{darkblue}{\textbf{\ipa{tsi˩ɭɯ˩}}}}{}
\textcolor{teal}{\mytextsc{nom}} \hspace{4pt} Ton~: L.
\textcolor{PineGreen}{\selectlanguage{french}Petit oiseau de couleur bleue/verte.} \zh{一种小鸟。}  \zh{量词}: \textcolor{darkblue}{\textbf{\ipa{mi˩}}}  \mytextsc{clf}~: \textcolor{darkblue}{\textbf{\ipa{mi˩}}} 
\lhead{\firstmark}
\rhead{\botmark}

\subsection{\hspace{-0.5cm} {\Large \textcolor{darkblue}{\textbf{\ipa{‑tso˧}}}}\hspace{0.5cm}[\kern2pt{\textcolor{darkblue}{\textbf{\ipa{tso˥}}}}\kern2pt]} \hypertarget{‑tso\string_M1}{}
\markboth{\textcolor{darkblue}{\textbf{\ipa{‑tso˧}}}}{}
\textcolor{teal}{\mytextsc{suffixe}} \hspace{4pt} Ton~: M.
\textcolor{PineGreen}{\selectlanguage{french}\mytextsc{volitif}.} \zh{\mytextsc{意志。}}  ¶ \textcolor{darkblue}{\textbf{\ipa{dʑɤ˩bv̩˥-tso˩-ɲi˩-mæ˩!}}} \textcolor{PineGreen}{\selectlanguage{french}On joue, d'accord? / Allez, on va jouer!} \zh{玩一玩吧!}  
 ¶ \textcolor{darkblue}{\textbf{\ipa{ə˧tso˧ tv̩˧-tso˧-ɲi˥ ?}}} \textcolor{PineGreen}{\selectlanguage{french}Qu'est-ce que vous comptez planter?} \zh{要种什么东西?}  
 ¶ \textcolor{darkblue}{\textbf{\ipa{ə˧tso˧ ʝi˧-bi˧-tso˧-ɲi˥?}}} \textcolor{PineGreen}{\selectlanguage{french}Qu'est-ce que vous comptez faire maintenant?} \zh{要做什么了?}  

\lhead{\firstmark}
\rhead{\botmark}

\subsection{\hspace{-0.5cm} {\Large \textcolor{darkblue}{\textbf{\ipa{tso˧kʰwɤ\#˥}}}}\hspace{0.5cm}[\kern2pt{\textcolor{darkblue}{\textbf{\ipa{tso˧kʰwɤ˧}}}}\kern2pt]} \hypertarget{tso\string_Mk\string_hw7\#\string_T1}{}
\markboth{\textcolor{darkblue}{\textbf{\ipa{tso˧kʰwɤ\#˥}}}}{}
\textcolor{teal}{\mytextsc{nom}} \hspace{4pt} Ton~: \#H.
\textcolor{PineGreen}{\selectlanguage{french}Sac (était fait en toile ou en cuir).} \zh{袋子。}  \zh{量词}: \textcolor{darkblue}{\textbf{\ipa{ɭɯ˧}}}  \mytextsc{clf}~: \textcolor{darkblue}{\textbf{\ipa{ɭɯ˧}}} 
\lhead{\firstmark}
\rhead{\botmark}

\subsection{\hspace{-0.5cm} {\Large \textcolor{darkblue}{\textbf{\ipa{tso˧lo˧-mv̩˥tso˩}}}}\hspace{0.5cm}[\kern2pt{\textcolor{darkblue}{\textbf{\ipa{tso˧lo˧mv̩˥tso˩}}}}\kern2pt]} \hypertarget{tso\string_Mlo\string_M-mv\string_=\string_Ttso\string_B1}{}
\markboth{\textcolor{darkblue}{\textbf{\ipa{tso˧lo˧-mv̩˥tso˩}}}}{}
\textcolor{teal}{\mytextsc{nom}} \hspace{4pt} Ton~: \#H-.
\textcolor{PineGreen}{\selectlanguage{french}Outil; chose, objet, truc.} \zh{东西,工具。}  \zh{量词}: \textcolor{darkblue}{\textbf{\ipa{nɑ˧, ɭɯ˧}}}  \mytextsc{clf}~: \textcolor{darkblue}{\textbf{\ipa{nɑ˧, ɭɯ˧}}} 
\lhead{\firstmark}
\rhead{\botmark}

\subsection{\hspace{-0.5cm} {\Large \textcolor{darkblue}{\textbf{\ipa{tso˧qwɤ˧}}}}\hspace{0.5cm}[\kern2pt{\textcolor{darkblue}{\textbf{\ipa{tso˧qwɤ˧}}}}\kern2pt]} \hypertarget{tso\string_Mqw7\string_M1}{}
\markboth{\textcolor{darkblue}{\textbf{\ipa{tso˧qwɤ˧}}}}{}
\textcolor{teal}{\mytextsc{nom}} \hspace{4pt} Ton~: M.
\textcolor{PineGreen}{\selectlanguage{french}Chambrette: partie de la pièce principale dans laquelle se trouve un couchage; on y place provisoirement les nouveaux-nés, et les défunts.} \zh{小床角:主屋里面的一个角落,有床垫子。用餐、招待客人的时候,会有人在上面坐。刚出生的婴儿也在此处睡觉。人去世后,尸体先放在那个地方。}  \zh{量词}: \textcolor{darkblue}{\textbf{\ipa{ɭɯ˧}}}  \mytextsc{clf}~: \textcolor{darkblue}{\textbf{\ipa{ɭɯ˧}}} 
\lhead{\firstmark}
\rhead{\botmark}

\subsection{\hspace{-0.5cm} {\Large \textcolor{darkblue}{\textbf{\ipa{tso˧tso\#˥}}}}\hspace{0.5cm}[\kern2pt{\textcolor{darkblue}{\textbf{\ipa{tso˧tso˧}}}}\kern2pt]} \hypertarget{tso\string_Mtso\#\string_T1}{}
\markboth{\textcolor{darkblue}{\textbf{\ipa{tso˧tso\#˥}}}}{}
\textcolor{teal}{\mytextsc{nom}} \hspace{4pt} Ton~: \#H.
\textcolor{PineGreen}{\selectlanguage{french}Chose, truc, bidule, objet, machin.} \zh{东西。}  ¶ \textcolor{darkblue}{\textbf{\ipa{tso˧\textasciitilde{}tso˧-zo˧\textasciitilde{}zo˧-mv̩˧\textasciitilde{}mv̩˥}}} \textcolor{PineGreen}{\selectlanguage{french}bidule} \zh{各种东西、各种各样乱七八糟东西}  
 ¶ \textcolor{darkblue}{\textbf{\ipa{tso˧\textasciitilde{}tso˧ hwæ˩}}} \textcolor{PineGreen}{\selectlanguage{french}acheter des choses} \zh{买东西}  
 ¶ \textcolor{darkblue}{\textbf{\ipa{tso˧\textasciitilde{}tso˧ tɕʰi˧(-ze˩)}}} \textcolor{PineGreen}{\selectlanguage{french}vendre des choses} \zh{卖东西}  
 ¶ \textcolor{darkblue}{\textbf{\ipa{tso˧\textasciitilde{}tso˧ dzɯ˧(-ze˩)}}} \textcolor{PineGreen}{\selectlanguage{french}manger des choses} \zh{吃东西}  
 ¶ \textcolor{darkblue}{\textbf{\ipa{tso˧\textasciitilde{}tso˧ dze˥}}} \textcolor{PineGreen}{\selectlanguage{french}couper des choses} \zh{切东西}  
 ¶ \textcolor{darkblue}{\textbf{\ipa{tso˧\textasciitilde{}tso˧ ʈʰɯ˩}}} \textcolor{PineGreen}{\selectlanguage{french}boire des choses} \zh{喝东西}  
 ¶ \textcolor{darkblue}{\textbf{\ipa{tso˧\textasciitilde{}tso˧ lɑ˩}}} \textcolor{PineGreen}{\selectlanguage{french}frapper des choses} \zh{打东西}  
 \zh{量词}: \textcolor{darkblue}{\textbf{\ipa{nɑ˧, ɭɯ˧}}}  \mytextsc{clf}~: \textcolor{darkblue}{\textbf{\ipa{nɑ˧, ɭɯ˧}}} 
\lhead{\firstmark}
\rhead{\botmark}

\subsection{\hspace{-0.5cm} {\Large \textcolor{darkblue}{\textbf{\ipa{tso˩\textsubscript{c}}}}}\hspace{0.5cm}[\kern2pt{\textcolor{darkblue}{\textbf{\ipa{tso˩˥}}}}\kern2pt]} \hypertarget{tso\string_Bc1}{}
\markboth{\textcolor{darkblue}{\textbf{\ipa{tso˩\textsubscript{c}}}}}{}
\textcolor{teal}{\mytextsc{classificateur}} \hspace{4pt} Ton~: L\textsubscript{c}.
\textcolor{PineGreen}{\selectlanguage{french}Classificateur des pièces (dans la maison), des compartiments (dans un grenier).} \zh{量词:间(房间,分隔间,包间)。}  ¶ \textcolor{darkblue}{\textbf{\ipa{ʈʂʰɯ˧-tso˥}}} \textcolor{PineGreen}{\selectlanguage{french}cette pièce} \zh{这间}  

\lhead{\firstmark}
\rhead{\botmark}

\subsection{\hspace{-0.5cm} {\Large \textcolor{darkblue}{\textbf{\ipa{tso˩\textsubscript{a}}}}}\hspace{0.5cm}[\kern2pt{\textcolor{darkblue}{\textbf{\ipa{tso˩˥}}}}\kern2pt]} \hypertarget{tso\string_Ba1}{}
\markboth{\textcolor{darkblue}{\textbf{\ipa{tso˩\textsubscript{a}}}}}{}
\textcolor{teal}{\mytextsc{verbe}} \hspace{4pt} Ton~: L\textsubscript{a}.
\textcolor{PineGreen}{\selectlanguage{french}Construire un mur, un pont….} \zh{砌(墙)、建(桥梁)。}  ¶ \textcolor{darkblue}{\textbf{\ipa{ɖʐɤ˩ tso˩}}} \textcolor{PineGreen}{\selectlanguage{french}construire un escalier} \zh{修一座楼梯}  
 ¶ \textcolor{darkblue}{\textbf{\ipa{dzo˩ tso˩}}} \textcolor{PineGreen}{\selectlanguage{french}construire un pont} \zh{修一座桥}  
 ¶ \textcolor{darkblue}{\textbf{\ipa{ɖɯ˧-tso˧\textasciitilde{}tso˥-ɻ̍˩}}} \textcolor{PineGreen}{\selectlanguage{french}construire quelque chose} \zh{修东西}  

\lhead{\firstmark}
\rhead{\botmark}

\subsection{\hspace{-0.5cm} {\Large \textcolor{darkblue}{\textbf{\ipa{tso˩qʰv̩˩ɻ̍˥}}}}\hspace{0.5cm}[\kern2pt{\textcolor{darkblue}{\textbf{\ipa{tso˧qʰv̩˧ɻ̍˥}}}}\kern2pt]} \hypertarget{tso\string_Bq\string_hv\string_=\string_Br£`̍\string_T1}{}
\markboth{\textcolor{darkblue}{\textbf{\ipa{tso˩qʰv̩˩ɻ̍˥}}}}{}
\textcolor{teal}{\mytextsc{nom}} \hspace{4pt} Ton~: H\#.
\textcolor{PineGreen}{\selectlanguage{french}Porche, vestibule: espace situé entre la cour et la pièce principale, c'est-à-dire l'espace, protégé de la pluie par la toiture, où l'on parvient lorsqu'on passe le seuil en sortant de la pièce principale. Dans certaines maisons, ce porche est séparé de la cour par une cloison de bois, percée d'une porte à peu près au milieu de sa longueur.} \zh{玄关、门厅。}  \zh{量词}: \textcolor{darkblue}{\textbf{\ipa{kʰwɤ˥}}}  \mytextsc{clf}~: \textcolor{darkblue}{\textbf{\ipa{kʰwɤ˥}}} 
\lhead{\firstmark}
\rhead{\botmark}

\subsection{\hspace{-0.5cm} {\Large \textcolor{darkblue}{\textbf{\ipa{tso˩\textasciitilde{}tso˧˥}}}}\hspace{0.5cm}[\kern2pt{\textcolor{darkblue}{\textbf{\ipa{tso˩tso˧˥}}}}\kern2pt]} \hypertarget{tso\string_B~tso\string_M\string_T1}{}
\markboth{\textcolor{darkblue}{\textbf{\ipa{tso˩\textasciitilde{}tso˧˥}}}}{}
\textcolor{teal}{\mytextsc{verbe}} \hspace{4pt} Ton~: LM+MH\#.
\textcolor{PineGreen}{\selectlanguage{french}Touiller, faire la pâtée (du chien…).} \zh{拌好狗食。} 
\lhead{\firstmark}
\rhead{\botmark}

\subsection{\hspace{-0.5cm} {\Large \textcolor{darkblue}{\textbf{\ipa{tsɯ˥}}} \textsubscript{1}}\hspace{0.5cm}[\kern2pt{\textcolor{darkblue}{\textbf{\ipa{tsɯ˥}}}}\kern2pt]} \hypertarget{tsM\string_T1}{}
\markboth{\textcolor{darkblue}{\textbf{\ipa{tsɯ˥}}} \textsubscript{1}}{}
\textcolor{teal}{\mytextsc{verbe}} \hspace{4pt} Ton~: H.
\ding{202} \textcolor{PineGreen}{\selectlanguage{french}Attacher.} \zh{绑、捆、栓。}  ¶ \textcolor{darkblue}{\textbf{\ipa{dʑi˧mi˧ tʰi˧-tsɯ˥}}} \textcolor{PineGreen}{\selectlanguage{french}attacher le buffle} \zh{栓水牛}  
 ¶ \textcolor{darkblue}{\textbf{\ipa{tsɯ˧\textasciitilde{}tsɯ˧}}} \textcolor{PineGreen}{\selectlanguage{french}\mytextsc{red}} \zh{\mytextsc{重叠}}  
 ¶ \textcolor{darkblue}{\textbf{\ipa{le˧-tsɯ˧\textasciitilde{}tsɯ˧}}} \textcolor{PineGreen}{\selectlanguage{french}\mytextsc{accomp} \mytextsc{red}} \zh{\mytextsc{accomp} \mytextsc{red}}  
 ¶ \textcolor{darkblue}{\textbf{\ipa{tʰi˧-tsɯ˧\textasciitilde{}tsɯ˧}}} \textcolor{PineGreen}{\selectlanguage{french}\mytextsc{dur} \mytextsc{red}} \zh{\mytextsc{dur} \mytextsc{red}}  
\ding{203} \textcolor{PineGreen}{\selectlanguage{french}Se pendre.} \zh{上吊自杀、缢。}  ¶ \textcolor{darkblue}{\textbf{\ipa{ʁæ˧tsɯ˧ le˧-ʂɯ˧ +ze˧}}} \textcolor{PineGreen}{\selectlanguage{french}se pendre} \zh{上吊自杀、缢}  

\lhead{\firstmark}
\rhead{\botmark}

\subsection{\hspace{-0.5cm} {\Large \textcolor{darkblue}{\textbf{\ipa{tsɯ˥}}} \textsubscript{2}}\hspace{0.5cm}[\kern2pt{\textcolor{darkblue}{\textbf{\ipa{tsɯ˥}}}}\kern2pt]} \hypertarget{tsM\string_T2}{}
\markboth{\textcolor{darkblue}{\textbf{\ipa{tsɯ˥}}} \textsubscript{2}}{}
\textcolor{teal}{\mytextsc{verbe}} \hspace{4pt} Ton~: H.
\textcolor{PineGreen}{\selectlanguage{french}Prendre avec une écumoire, récupérer dans l'eau.} \zh{打捞。} 
\lhead{\firstmark}
\rhead{\botmark}

\subsection{\hspace{-0.5cm} {\Large \textcolor{darkblue}{\textbf{\ipa{tsɯ˧}}}}\hspace{0.5cm}[\kern2pt{\textcolor{darkblue}{\textbf{\ipa{tsɯ˥}}}}\kern2pt]} \hypertarget{tsM\string_M1}{}
\markboth{\textcolor{darkblue}{\textbf{\ipa{tsɯ˧}}}}{}
\textcolor{teal}{\mytextsc{nom}} \hspace{4pt} Ton~: M.
\textcolor{PineGreen}{\selectlanguage{french}Lettre, caractère chinois.} \zh{字。}  Emprunt~: chinois  \zh{字}

\lhead{\firstmark}
\rhead{\botmark}

\subsection{\hspace{-0.5cm} {\Large \textcolor{darkblue}{\textbf{\ipa{tsɯ˩\textsubscript{a}}}}}\hspace{0.5cm}[\kern2pt{\textcolor{darkblue}{\textbf{\ipa{tsɯ˥}}}}\kern2pt]} \hypertarget{tsM\string_Ba1}{}
\markboth{\textcolor{darkblue}{\textbf{\ipa{tsɯ˩\textsubscript{a}}}}}{}
\textcolor{teal}{\mytextsc{verbe}} \hspace{4pt} Ton~: L\textsubscript{a}.
\textcolor{PineGreen}{\selectlanguage{french}Boucher/être bouché; obstruer (ex.: obstruer l'entrée d'un trou).} \zh{堵塞、塞住洞口。} 
\lhead{\firstmark}
\rhead{\botmark}

\subsection{\hspace{-0.5cm} {\Large \textcolor{darkblue}{\textbf{\ipa{tsɯ˩pʰɤ˩}}}}\hspace{0.5cm}[\kern2pt{\textcolor{darkblue}{\textbf{\ipa{tsɯ˧pʰɤ˧˥}}}}\kern2pt]} \hypertarget{tsM\string_Bp\string_h7\string_B1}{}
\markboth{\textcolor{darkblue}{\textbf{\ipa{tsɯ˩pʰɤ˩}}}}{}
\textcolor{teal}{\mytextsc{verbe}} \hspace{4pt} Ton~: L.
\ding{202} \textcolor{PineGreen}{\selectlanguage{french}Cligner des yeux.} \zh{眨眼。}  ¶ \textcolor{darkblue}{\textbf{\ipa{mɤ˧-tsɯ˩pʰɤ˩}}} \textcolor{PineGreen}{\selectlanguage{french}\mytextsc{neg}} \zh{不眨眼}  
 ¶ \textcolor{darkblue}{\textbf{\ipa{ɖɯ˧-tsɯ˧\textasciitilde{}tsɯ˥-ɻ̍˩}}} \textcolor{PineGreen}{\selectlanguage{french}\mytextsc{délimitatif} \mytextsc{red} \mytextsc{inchoatif}} \zh{\mytextsc{delimitative} \mytextsc{red} \mytextsc{inceptive}}  
 ¶ \textcolor{darkblue}{\textbf{\ipa{njɤ˩ɭɯ˧ tsɯ˩pʰɤ˩}}} \textcolor{PineGreen}{\selectlanguage{french}cligner des yeux} \zh{眨眼}  
\ding{203} \textcolor{PineGreen}{\selectlanguage{french}Faire un clin d'oeil (discret signe d'intelligence).} \zh{眨眼。}  ¶ \textcolor{darkblue}{\textbf{\ipa{ʈʂʰɯ˧ | njɤ˩ɭɯ˧ tsɯ˩pʰɤ˩-dʑo˩!}}} \textcolor{PineGreen}{\selectlanguage{french}Elle/il est en train de faire un clin d'oeil!} \zh{他在眨眼!}  

\lhead{\firstmark}
\rhead{\botmark}

\subsection{\hspace{-0.5cm} {\Large \textcolor{darkblue}{\textbf{\ipa{tsɯ˧˥}}}}\hspace{0.5cm}[\kern2pt{\textcolor{darkblue}{\textbf{\ipa{tsɯ˧˥}}}}\kern2pt]} \hypertarget{tsM\string_M\string_T1}{}
\markboth{\textcolor{darkblue}{\textbf{\ipa{tsɯ˧˥}}}}{}
\textcolor{teal}{\mytextsc{suffixe}} \hspace{4pt} Ton~: MH.
\textcolor{PineGreen}{\selectlanguage{french}Particule d'évidentialité rapportée: elle indique une connaissance indirecte, par ouï-dire, et non par connaissance directe.} \zh{据说\mytextsc{传闻据素。}} 
\lhead{\firstmark}
\rhead{\botmark}

\subsection{\hspace{-0.5cm} {\Large \textcolor{darkblue}{\textbf{\ipa{tsɯ˧˥}}}}\hspace{0.5cm}[\kern2pt{\textcolor{darkblue}{\textbf{\ipa{tsɯ˧˥}}}}\kern2pt]} \hypertarget{tsM\string_M\string_T1}{}
\markboth{\textcolor{darkblue}{\textbf{\ipa{tsɯ˧˥}}}}{}
\textcolor{teal}{\mytextsc{verbe}} \hspace{4pt} Ton~: MH.
\textcolor{PineGreen}{\selectlanguage{french}Appeler, nommer, désigner.} \zh{叫、叫做。}  ¶ \textcolor{darkblue}{\textbf{\ipa{ʈæ˧ʂɯ˧-ɳɯ˧ | no˧-ki˥ | jɤ˩-ʐe˧ ɲi˩-tsɯ˩-mɤ˩-tsɯ˩!}}} \textcolor{PineGreen}{\selectlanguage{french}\textcolor{darkblue}{\textbf{\ipa{/ʈæ˧ʂɯ˧/}}} t'appelle “l'étranger” / il te traite d'étranger!} \zh{达石把你叫作“老外”!}  

\lhead{\firstmark}
\rhead{\botmark}

\subsection{\hspace{-0.5cm} {\Large \textcolor{darkblue}{\textbf{\ipa{tsʰɑ˧bo\#˥}}}}\hspace{0.5cm}[\kern2pt{\textcolor{darkblue}{\textbf{\ipa{tsʰɑ˧bo˧}}}}\kern2pt]} \hypertarget{ts\string_hA\string_Mbo\#\string_T1}{}
\markboth{\textcolor{darkblue}{\textbf{\ipa{tsʰɑ˧bo\#˥}}}}{}
\textcolor{teal}{\mytextsc{nom}} \hspace{4pt} Ton~: \#H.
\textcolor{PineGreen}{\selectlanguage{french}Cuisinier.} \zh{厨师。}  ¶ \textcolor{darkblue}{\textbf{\ipa{tsʰɑ˧bo˧ lɑ˩}}} \textcolor{PineGreen}{\selectlanguage{french}être cuisinier, s'engager comme cuisinier, faire le travail de cuisinier} \zh{当厨师}  
 ¶ \textcolor{darkblue}{\textbf{\ipa{tsʰɑ˧bo˧ ʝi˧}}} \textcolor{PineGreen}{\selectlanguage{french}être cuisinier, s'engager comme cuisinier, faire le travail de cuisinier} \zh{当厨师}  

\lhead{\firstmark}
\rhead{\botmark}

\subsection{\hspace{-0.5cm} {\Large \textcolor{darkblue}{\textbf{\ipa{tsʰɑ˧kv̩˩}}}}\hspace{0.5cm}[\kern2pt{\textcolor{darkblue}{\textbf{\ipa{tsʰɑ˧kv̩˩}}}}\kern2pt]} \hypertarget{ts\string_hA\string_Mkv\string_=\string_B1}{}
\markboth{\textcolor{darkblue}{\textbf{\ipa{tsʰɑ˧kv̩˩}}}}{}
\textcolor{teal}{\mytextsc{nom}} \hspace{4pt} Ton~: L\#.
\textcolor{PineGreen}{\selectlanguage{french}Réserve, magasin.} \zh{仓库(汉语借词)。}  Emprunt~: chinois  \zh{仓库}

\lhead{\firstmark}
\rhead{\botmark}

\subsection{\hspace{-0.5cm} {\Large \textcolor{darkblue}{\textbf{\ipa{tsʰɑ˧tɕɤ˧˥}}}}\hspace{0.5cm}[\kern2pt{\textcolor{darkblue}{\textbf{\ipa{tsʰɑ˧tɕɤ˧˥}}}}\kern2pt]} \hypertarget{ts\string_hA\string_Mts£7\string_M\string_T1}{}
\markboth{\textcolor{darkblue}{\textbf{\ipa{tsʰɑ˧tɕɤ˧˥}}}}{}
\textcolor{teal}{\mytextsc{nom}} \hspace{4pt} Ton~: MH.
\textcolor{PineGreen}{\selectlanguage{french}Jeunes pousses, petites pousses qu'on récolte pour les manger.} \zh{青菜幼苗。} 
\lhead{\firstmark}
\rhead{\botmark}

\subsection{\hspace{-0.5cm} {\Large \textcolor{darkblue}{\textbf{\ipa{tsʰɑ˩pʰɑ˩lɑ˥}}}}\hspace{0.5cm}[\kern2pt{\textcolor{darkblue}{\textbf{\ipa{tsʰɑ˩pʰɑ˩lɑ˥}}}}\kern2pt]} \hypertarget{ts\string_hA\string_Bp\string_hA\string_BlA\string_T1}{}
\markboth{\textcolor{darkblue}{\textbf{\ipa{tsʰɑ˩pʰɑ˩lɑ˥}}}}{}
\textcolor{teal}{\mytextsc{nom}} \hspace{4pt} Ton~: L+H\#.
\textcolor{PineGreen}{\selectlanguage{french}Feuilles d'épi de maïs: les feuilles qui entourent l'épi de maïs.} \zh{苞谷叶(玉米穰子的叶子)。}  ¶ \textcolor{darkblue}{\textbf{\ipa{qʰɑ˧dze˧-tsʰɑ˩pʰɑ˩lɑ˩}}} \textcolor{PineGreen}{\selectlanguage{french}même sens} \zh{同上}  

\lhead{\firstmark}
\rhead{\botmark}

\subsection{\hspace{-0.5cm} {\Large \textcolor{darkblue}{\textbf{\ipa{tsʰæ˧pʰv˧˥}}}}\hspace{0.5cm}[\kern2pt{\textcolor{darkblue}{\textbf{\ipa{tsʰæ˧pʰv˧˥}}}}\kern2pt]} \hypertarget{ts\string_h\{\string_Mp\string_hv\string_M\string_T1}{}
\markboth{\textcolor{darkblue}{\textbf{\ipa{tsʰæ˧pʰv˧˥}}}}{}
\textcolor{teal}{\mytextsc{nom}} \hspace{4pt} Ton~: MH\#.
\textcolor{PineGreen}{\selectlanguage{french}Chou chinois. Il s'agit d'un calque du chinois 'légume blanc', employant le nom chinois pour 'légume' associé à l'adjectif na pour 'blanc'.} \zh{白菜(借汉语‘白菜’的第二个音节来充当这个名词的第一个音节:按摩梭话句法,形容词在名词后面,跟汉语相反)。}  Emprunt~: chinois  \zh{菜}
 \zh{量词}: \textcolor{darkblue}{\textbf{\ipa{po˧}}}  \mytextsc{clf}~: \textcolor{darkblue}{\textbf{\ipa{po˧}}} \textit{Voir~:} \hyperlink{}{\textcolor{darkblue}{\textbf{\ipa{v̩˩tsʰɤ˧-pʰv̩˥}}}} 
\lhead{\firstmark}
\rhead{\botmark}

\subsection{\hspace{-0.5cm} {\Large \textcolor{darkblue}{\textbf{\ipa{tsʰe\#˥}}}}\hspace{0.5cm}[\kern2pt{\textcolor{darkblue}{\textbf{\ipa{tsʰe˥}}}}\kern2pt]} \hypertarget{ts\string_he\#\string_T1}{}
\markboth{\textcolor{darkblue}{\textbf{\ipa{tsʰe\#˥}}}}{}
\textcolor{teal}{\mytextsc{nom}} \hspace{4pt} Ton~: \#H.
\textcolor{PineGreen}{\selectlanguage{french}Sel.} \zh{盐。} 
\lhead{\firstmark}
\rhead{\botmark}

\subsection{\hspace{-0.5cm} {\Large \textcolor{darkblue}{\textbf{\ipa{tsʰe˧}}}}\hspace{0.5cm}[\kern2pt{\textcolor{darkblue}{\textbf{\ipa{tsʰe˩˥}}}}\kern2pt]} \hypertarget{ts\string_he\string_M1}{}
\markboth{\textcolor{darkblue}{\textbf{\ipa{tsʰe˧}}}}{}
\textcolor{teal}{\mytextsc{nombre}} \hspace{4pt} Ton~: L.
\textcolor{PineGreen}{\selectlanguage{french}10.} \zh{10。} 
\lhead{\firstmark}
\rhead{\botmark}

\subsection{\hspace{-0.5cm} {\Large \textcolor{darkblue}{\textbf{\ipa{tsʰe˧do˧˥}}}}\hspace{0.5cm}[\kern2pt{\textcolor{darkblue}{\textbf{\ipa{tsʰe˧do˧˥}}}}\kern2pt]} \hypertarget{ts\string_he\string_Mdo\string_M\string_T1}{}
\markboth{\textcolor{darkblue}{\textbf{\ipa{tsʰe˧do˧˥}}}}{}
\textcolor{teal}{\mytextsc{adverbe}} \hspace{4pt} Ton~: MH\# | L.
\textcolor{PineGreen}{\selectlanguage{french}Début du mois.} \zh{月初。}  ¶ \textcolor{darkblue}{\textbf{\ipa{tsʰe˧do˧-ɖɯ˧ɲi\#˥ / tsʰe˧do˧-ɖɯ˧ɲi˥}}} \textcolor{PineGreen}{\selectlanguage{french}le 1er du mois} \zh{初一}  
 ¶ \textcolor{darkblue}{\textbf{\ipa{tsʰe˧do˧-ɲi˧ɲi\#˥ / tsʰe˧do˧-ɲi˧ɲi˥}}} \textcolor{PineGreen}{\selectlanguage{french}le deuxième jour du mois} \zh{初二}  
 ¶ \textcolor{darkblue}{\textbf{\ipa{tsʰe˧do˧˥ | -so˩ɲi˩˥}}} \textcolor{PineGreen}{\selectlanguage{french}le 3e du mois} \zh{初三}  
 ¶ \textcolor{darkblue}{\textbf{\ipa{tsʰe˧do˧-ŋwɤ˥ɲi˩}}} \textcolor{PineGreen}{\selectlanguage{french}le 5e jour du mois} \zh{初五}  
 ¶ \textcolor{darkblue}{\textbf{\ipa{tsʰe˧do˧-hõ˥ɲi˩}}} \textcolor{PineGreen}{\selectlanguage{french}le 8e jour du mois} \zh{初八}  
 ¶ \textcolor{darkblue}{\textbf{\ipa{tsʰe˧do˧˥ | -tsʰe˩ɲi˩˥}}} \textcolor{PineGreen}{\selectlanguage{french}le 10e jour du mois} \zh{初十}  
 ¶ \textcolor{darkblue}{\textbf{\ipa{tsʰe˧do˧˥ | -tsʰe˩ɖɯ˩ɲi˩˥}}} \textcolor{PineGreen}{\selectlanguage{french}le 11e jour du mois} \zh{十一日}  

\lhead{\firstmark}
\rhead{\botmark}

\subsection{\hspace{-0.5cm} {\Large \textcolor{darkblue}{\textbf{\ipa{tsʰe˧hṽ˧˥}}}}\hspace{0.5cm}[\kern2pt{\textcolor{darkblue}{\textbf{\ipa{tsʰe˧hṽ˧˥}}}}\kern2pt]} \hypertarget{ts\string_he\string_Mhv\string_~\string_M\string_T1}{}
\markboth{\textcolor{darkblue}{\textbf{\ipa{tsʰe˧hṽ˧˥}}}}{}
\textcolor{teal}{\mytextsc{nom}} \hspace{4pt} Ton~: MH\#.
\textcolor{PineGreen}{\selectlanguage{french}Aspidistra.} \zh{万年青。}  ¶ \textcolor{darkblue}{\textbf{\ipa{tsʰe˧hṽ˧-dzi˧˥}}} \textcolor{PineGreen}{\selectlanguage{french}arbre/plant d'aspidistra} \zh{万年青树}  
 ¶ \textcolor{darkblue}{\textbf{\ipa{tsʰe˧hṽ˧-bæ˥bæ˩}}} \textcolor{PineGreen}{\selectlanguage{french}fleurs d'aspidistra} \zh{万年青花}  
 \zh{量词}: \textcolor{darkblue}{\textbf{\ipa{dzi˩}}}  \mytextsc{clf}~: \textcolor{darkblue}{\textbf{\ipa{dzi˩}}} 
\lhead{\firstmark}
\rhead{\botmark}

\subsection{\hspace{-0.5cm} {\Large \textcolor{darkblue}{\textbf{\ipa{tsʰe˧jɤ˧mi˥}}}}\hspace{0.5cm}[\kern2pt{\textcolor{darkblue}{\textbf{\ipa{tsʰe˧jɤ˧mi˥}}}}\kern2pt]} \hypertarget{ts\string_he\string_Mj7\string_Mmi\string_T1}{}
\markboth{\textcolor{darkblue}{\textbf{\ipa{tsʰe˧jɤ˧mi˥}}}}{}
\textcolor{teal}{\mytextsc{nom}} \hspace{4pt} Ton~: H\#.
\textcolor{PineGreen}{\selectlanguage{french}Marécage.} \zh{沼泽。}  ¶ \textcolor{darkblue}{\textbf{\ipa{tsʰe˧jɤ˧mi˥-qo˩, | ʈʰæ˧tɕi˧ɭɯ˥ | tʰi˧-di˩!}}} \textcolor{PineGreen}{\selectlanguage{french}sur les terres marécageuses, il ne pousse que des petites touffes d'herbe!} \zh{沼泽里,只长野草!}  
 \zh{量词}: \textcolor{darkblue}{\textbf{\ipa{pʰæ˧˥}}}  \mytextsc{clf}~: \textcolor{darkblue}{\textbf{\ipa{pʰæ˧˥}}} 
\lhead{\firstmark}
\rhead{\botmark}

\subsection{\hspace{-0.5cm} {\Large \textcolor{darkblue}{\textbf{\ipa{tsʰe˧ɬi˧mi˧}}}}\hspace{0.5cm}[\kern2pt{\textcolor{darkblue}{\textbf{\ipa{tsʰe˧ɬi˧mi˧}}}}\kern2pt]} \hypertarget{ts\string_he\string_MKi\string_Mmi\string_M1}{}
\markboth{\textcolor{darkblue}{\textbf{\ipa{tsʰe˧ɬi˧mi˧}}}}{}
\textcolor{teal}{\mytextsc{nom}} \hspace{4pt} Ton~: M.
\textcolor{PineGreen}{\selectlanguage{french}10e mois.} \zh{十月。} 
\lhead{\firstmark}
\rhead{\botmark}

\subsection{\hspace{-0.5cm} {\Large \textcolor{darkblue}{\textbf{\ipa{tsʰe˧qʰɑ˩}}}}\hspace{0.5cm}[\kern2pt{\textcolor{darkblue}{\textbf{\ipa{tsʰe˧qʰɑ˩}}}}\kern2pt]} \hypertarget{ts\string_he\string_Mq\string_hA\string_B1}{}
\markboth{\textcolor{darkblue}{\textbf{\ipa{tsʰe˧qʰɑ˩}}}}{}
\textcolor{teal}{\mytextsc{adjectif}} \hspace{4pt} Ton~: L\#.
\textcolor{PineGreen}{\selectlanguage{french}Trop salé.} \zh{太咸。} \textit{Voir~:} \hyperlink{}{\textcolor{darkblue}{\textbf{\ipa{tsʰe˧so˧˥}}}} 
\lhead{\firstmark}
\rhead{\botmark}

\subsection{\hspace{-0.5cm} {\Large \textcolor{darkblue}{\textbf{\ipa{tsʰe˧so˧˥}}}}\hspace{0.5cm}[\kern2pt{\textcolor{darkblue}{\textbf{\ipa{tsʰe˧so˧˥}}}}\kern2pt]} \hypertarget{ts\string_he\string_Mso\string_M\string_T1}{}
\markboth{\textcolor{darkblue}{\textbf{\ipa{tsʰe˧so˧˥}}}}{}
\textcolor{teal}{\mytextsc{adjectif}} \hspace{4pt} Ton~: MH\#.
\textcolor{PineGreen}{\selectlanguage{french}Salé (agréablement: salé à point).} \zh{咸。} \textit{Voir~:} \hyperlink{}{\textcolor{darkblue}{\textbf{\ipa{tsʰe˧qʰɑ˩}}}} 
\lhead{\firstmark}
\rhead{\botmark}

\subsection{\hspace{-0.5cm} {\Large \textcolor{darkblue}{\textbf{\ipa{tsʰe˧tʰv̩\#˥}}}}\hspace{0.5cm}[\kern2pt{\textcolor{darkblue}{\textbf{\ipa{tsʰe˧tʰv̩˧}}}}\kern2pt]} \hypertarget{ts\string_he\string_Mt\string_hv\string_=\#\string_T1}{}
\markboth{\textcolor{darkblue}{\textbf{\ipa{tsʰe˧tʰv̩\#˥}}}}{}
\textcolor{teal}{\mytextsc{nom}} \hspace{4pt} Ton~: \#H.
\ding{202} \textcolor{PineGreen}{\selectlanguage{french}Variole, petite vérole.} \zh{天花。}  ¶ \textcolor{darkblue}{\textbf{\ipa{tsʰe˧tʰv̩˧ | bæ˩bæ˩ bæ˥-ze˩}}} \textcolor{PineGreen}{\selectlanguage{french}La variole s'est déclarée.} \zh{天花/麻疹犯了。}  
 \zh{量词}: \textcolor{darkblue}{\textbf{\ipa{ʂɯ˩}}} \ding{203} \textcolor{PineGreen}{\selectlanguage{french}Rougeole.} \zh{麻疹,疹子。}  ¶ \textcolor{darkblue}{\textbf{\ipa{gv̩˧-kʰv̩˩ mɤ˩-gv̩˩, | tsʰe˧ mɤ˧-tʰv̩˧, | hĩ˧ ʈʂɤ˧-mɤ˧-kv̩˩!}}} \textcolor{PineGreen}{\selectlanguage{french}Si, avant l'âge de neuf ans, on ne contracte pas la rougeole, on ne peut pas devenir adulte! / Attraper la rougeole, ça fait partie du processus de croissance vers l'âge adulte!} \zh{九岁前不得麻疹,不能成人! / 得麻疹,就是小孩生长过程中必须要的一件事情!}  
 \mytextsc{clf}~: \textcolor{darkblue}{\textbf{\ipa{ʂɯ˩}}} \textcolor{PineGreen}{\selectlanguage{french}fois}
\lhead{\firstmark}
\rhead{\botmark}

\subsection{\hspace{-0.5cm} {\Large \textcolor{darkblue}{\textbf{\ipa{tsʰe˧ʈʂæ˧}}}}\hspace{0.5cm}[\kern2pt{\textcolor{darkblue}{\textbf{\ipa{tsʰe˧ʈʂæ˧}}}}\kern2pt]} \hypertarget{ts\string_he\string_Mt`s`\{\string_M1}{}
\markboth{\textcolor{darkblue}{\textbf{\ipa{tsʰe˧ʈʂæ˧}}}}{}
\textcolor{teal}{\mytextsc{nom}} \hspace{4pt} Ton~: M.
\textcolor{PineGreen}{\selectlanguage{french}Chef de village, petit officiel.} \zh{村长。}  Emprunt~: chinois  \zh{村长}

\lhead{\firstmark}
\rhead{\botmark}

\subsection{\hspace{-0.5cm} {\Large \textcolor{darkblue}{\textbf{\ipa{tsʰe˩\textsubscript{b}}}}}\hspace{0.5cm}[\kern2pt{\textcolor{darkblue}{\textbf{\ipa{tsʰe˩˥}}}}\kern2pt]} \hypertarget{ts\string_he\string_Bb1}{}
\markboth{\textcolor{darkblue}{\textbf{\ipa{tsʰe˩\textsubscript{b}}}}}{}
\textcolor{teal}{\mytextsc{classificateur}} \hspace{4pt} Ton~: L\textsubscript{b}.
\textcolor{PineGreen}{\selectlanguage{french}Pouce (cette unité de mesure n'était pas en usage chez les Na avant son introduction par emprunt au chinois).} \zh{量词:寸(汉语借词)。}  Emprunt~: chinois  \zh{寸}

\lhead{\firstmark}
\rhead{\botmark}

\subsection{\hspace{-0.5cm} {\Large \textcolor{darkblue}{\textbf{\ipa{tsʰe˩\textsubscript{b}}}}}\hspace{0.5cm}[\kern2pt{\textcolor{darkblue}{\textbf{\ipa{tsʰe˩˥}}}}\kern2pt]} \hypertarget{ts\string_he\string_Bb1}{}
\markboth{\textcolor{darkblue}{\textbf{\ipa{tsʰe˩\textsubscript{b}}}}}{}
\textcolor{teal}{\mytextsc{classificateur}} \hspace{4pt} Ton~: L\textsubscript{b}.
\textcolor{PineGreen}{\selectlanguage{french}Classificateur des nœuds dans une tresse.} \zh{量词:数辫子的节(一节)。} 
\lhead{\firstmark}
\rhead{\botmark}

\subsection{\hspace{-0.5cm} {\Large \textcolor{darkblue}{\textbf{\ipa{tsʰe˩gv̩˩}}}}\hspace{0.5cm}[\kern2pt{\textcolor{darkblue}{\textbf{\ipa{tsʰe˩gv̩˩˥}}}}\kern2pt]} \hypertarget{ts\string_he\string_Bgv\string_=\string_B1}{}
\markboth{\textcolor{darkblue}{\textbf{\ipa{tsʰe˩gv̩˩}}}}{}
\textcolor{teal}{\mytextsc{nombre}} \hspace{4pt} Ton~: L.
\textcolor{PineGreen}{\selectlanguage{french}19.} \zh{19。} 
\lhead{\firstmark}
\rhead{\botmark}

\subsection{\hspace{-0.5cm} {\Large \textcolor{darkblue}{\textbf{\ipa{tsʰe˩hõ˩}}}}\hspace{0.5cm}[\kern2pt{\textcolor{darkblue}{\textbf{\ipa{tsʰe˩hõ˩˥}}}}\kern2pt]} \hypertarget{ts\string_he\string_Bho\string_~\string_B1}{}
\markboth{\textcolor{darkblue}{\textbf{\ipa{tsʰe˩hõ˩}}}}{}
\textcolor{teal}{\mytextsc{nombre}} \hspace{4pt} Ton~: L.
\textcolor{PineGreen}{\selectlanguage{french}18.} \zh{18。} 
\lhead{\firstmark}
\rhead{\botmark}

\subsection{\hspace{-0.5cm} {\Large \textcolor{darkblue}{\textbf{\ipa{tsʰe˩ŋwɤ˩}}}}\hspace{0.5cm}[\kern2pt{\textcolor{darkblue}{\textbf{\ipa{tsʰe˩ŋwɤ˩˥}}}}\kern2pt]} \hypertarget{ts\string_he\string_BNw7\string_B1}{}
\markboth{\textcolor{darkblue}{\textbf{\ipa{tsʰe˩ŋwɤ˩}}}}{}
\textcolor{teal}{\mytextsc{nombre}} \hspace{4pt} Ton~: L.
\textcolor{PineGreen}{\selectlanguage{french}15.} \zh{15。} 
\lhead{\firstmark}
\rhead{\botmark}

\subsection{\hspace{-0.5cm} {\Large \textcolor{darkblue}{\textbf{\ipa{tsʰe˩qʰv̩˩}}}}\hspace{0.5cm}[\kern2pt{\textcolor{darkblue}{\textbf{\ipa{tsʰe˩qʰv̩˩˥}}}}\kern2pt]} \hypertarget{ts\string_he\string_Bq\string_hv\string_=\string_B1}{}
\markboth{\textcolor{darkblue}{\textbf{\ipa{tsʰe˩qʰv̩˩}}}}{}
\textcolor{teal}{\mytextsc{nombre}} \hspace{4pt} Ton~: L.
\textcolor{PineGreen}{\selectlanguage{french}16.} \zh{16。} 
\lhead{\firstmark}
\rhead{\botmark}

\subsection{\hspace{-0.5cm} {\Large \textcolor{darkblue}{\textbf{\ipa{tsʰe˩ʐv̩˩}}}}\hspace{0.5cm}[\kern2pt{\textcolor{darkblue}{\textbf{\ipa{tsʰe˩ʐv̩˩˥}}}}\kern2pt]} \hypertarget{ts\string_he\string_Bz`v\string_=\string_B1}{}
\markboth{\textcolor{darkblue}{\textbf{\ipa{tsʰe˩ʐv̩˩}}}}{}
\textcolor{teal}{\mytextsc{nombre}} \hspace{4pt} Ton~: L.
\textcolor{PineGreen}{\selectlanguage{french}14.} \zh{14。} 
\lhead{\firstmark}
\rhead{\botmark}

\subsection{\hspace{-0.5cm} {\Large \textcolor{darkblue}{\textbf{\ipa{tsʰɤ˩\textsubscript{a}}}}}\hspace{0.5cm}[\kern2pt{\textcolor{darkblue}{\textbf{\ipa{tsʰɤ˩˥}}}}\kern2pt]} \hypertarget{ts\string_h7\string_Ba1}{}
\markboth{\textcolor{darkblue}{\textbf{\ipa{tsʰɤ˩\textsubscript{a}}}}}{}
\textcolor{teal}{\mytextsc{verbe}} \hspace{4pt} Ton~: L\textsubscript{a}.
\textcolor{PineGreen}{\selectlanguage{french}Tresser (les cheveux, fils).} \zh{编(头发,线)。}  ¶ \textcolor{darkblue}{\textbf{\ipa{ʁo˧qʰwɤ˩ tsʰɤ˩}}} \textcolor{PineGreen}{\selectlanguage{french}tresser les cheveux, littéralement “tresser la tête”} \zh{编辫子}  
 ¶ \textcolor{darkblue}{\textbf{\ipa{hæ̃˧pɤ˧ le˧-tsʰɤ˩}}} \textcolor{PineGreen}{\selectlanguage{french}faire une tresse} \zh{梳一条辫子}  
 ¶ \textcolor{darkblue}{\textbf{\ipa{ɖɯ˧-tsʰɤ˧\textasciitilde{}tsʰɤ˥-ɻ̍˩}}} \textcolor{PineGreen}{\selectlanguage{french}\mytextsc{délimitatif} \string_ \mytextsc{red} \mytextsc{inchoatif}} \zh{\mytextsc{delimitative} \string_ \mytextsc{red} \mytextsc{inceptive}}  

\lhead{\firstmark}
\rhead{\botmark}

\subsection{\hspace{-0.5cm} {\Large \textcolor{darkblue}{\textbf{\ipa{tsʰɤ˧˥}}} \textsubscript{1}}\hspace{0.5cm}[\kern2pt{\textcolor{darkblue}{\textbf{\ipa{tsʰɤ˧˥}}}}\kern2pt]} \hypertarget{ts\string_h7\string_M\string_T1}{}
\markboth{\textcolor{darkblue}{\textbf{\ipa{tsʰɤ˧˥}}} \textsubscript{1}}{}
\textcolor{teal}{\mytextsc{verbe}} \hspace{4pt} Ton~: MH.
\textcolor{PineGreen}{\selectlanguage{french}Traire (vache, brebis).} \zh{挤奶。}  ¶ \textcolor{darkblue}{\textbf{\ipa{tso˧\textasciitilde{}tso˧ tsʰɤ˩ + ze˩}}} \textcolor{PineGreen}{\selectlanguage{french}traire des choses} \zh{挤出东西}  
 ¶ \textcolor{darkblue}{\textbf{\ipa{ʝi˧-bv̩˧ | ɳæ˧ tsʰɤ˩}}} \textcolor{PineGreen}{\selectlanguage{french}traire (le lait de) la vache} \zh{挤牛奶}  

\lhead{\firstmark}
\rhead{\botmark}

\subsection{\hspace{-0.5cm} {\Large \textcolor{darkblue}{\textbf{\ipa{tsʰɤ˧˥}}} \textsubscript{2}}\hspace{0.5cm}[\kern2pt{\textcolor{darkblue}{\textbf{\ipa{tsʰɤ˧˥}}}}\kern2pt]} \hypertarget{ts\string_h7\string_M\string_T2}{}
\markboth{\textcolor{darkblue}{\textbf{\ipa{tsʰɤ˧˥}}} \textsubscript{2}}{}
\textcolor{teal}{\mytextsc{verbe}} \hspace{4pt} Ton~: MH.
\textcolor{PineGreen}{\selectlanguage{french}Frotter, faire une friction (ex.: un tissu grossier frotte sur la peau, et l'irrite).} \zh{摩擦。} 
\lhead{\firstmark}
\rhead{\botmark}

\subsection{\hspace{-0.5cm} {\Large \textcolor{darkblue}{\textbf{\ipa{tsʰɤ˧˥}}} \textsubscript{3}}\hspace{0.5cm}[\kern2pt{\textcolor{darkblue}{\textbf{\ipa{tsʰɤ˧˥}}}}\kern2pt]} \hypertarget{ts\string_h7\string_M\string_T3}{}
\markboth{\textcolor{darkblue}{\textbf{\ipa{tsʰɤ˧˥}}} \textsubscript{3}}{}
\textcolor{teal}{\mytextsc{verbe}} \hspace{4pt} Ton~: MH.
\textcolor{PineGreen}{\selectlanguage{french}Attaquer, piller, s'en prendre à (des brigands attaquent un convoi).} \zh{抢。} 
\lhead{\firstmark}
\rhead{\botmark}

\subsection{\hspace{-0.5cm} {\Large \textcolor{darkblue}{\textbf{\ipa{tsʰɤ˧˥}}} \textsubscript{4}}\hspace{0.5cm}[\kern2pt{\textcolor{darkblue}{\textbf{\ipa{tsʰɤ˧˥}}}}\kern2pt]} \hypertarget{ts\string_h7\string_M\string_T4}{}
\markboth{\textcolor{darkblue}{\textbf{\ipa{tsʰɤ˧˥}}} \textsubscript{4}}{}
\textcolor{teal}{\mytextsc{verbe}} \hspace{4pt} Ton~: MH.
\textcolor{PineGreen}{\selectlanguage{french}Rendre.} \zh{还(东西)。} 
\lhead{\firstmark}
\rhead{\botmark}

\subsection{\hspace{-0.5cm} {\Large \textcolor{darkblue}{\textbf{\ipa{tsʰɤ˧˥\textsubscript{a}}}}}\hspace{0.5cm}[\kern2pt{\textcolor{darkblue}{\textbf{\ipa{tsʰɤ˧˥}}}}\kern2pt]} \hypertarget{ts\string_h7\string_M\string_Ta1}{}
\markboth{\textcolor{darkblue}{\textbf{\ipa{tsʰɤ˧˥\textsubscript{a}}}}}{}
\textcolor{teal}{\mytextsc{classificateur}} \hspace{4pt} Ton~: MH\textsubscript{a}.
\textcolor{PineGreen}{\selectlanguage{french}Classificateur des objets bosselés: feuilles, crêtes de coq, têtes d'ail (une tête d'ail se divise en gousses un peu comme on effeuille une branche ou une fleur).} \zh{量词:凸凹的物品,如:鸡冠(一顶)、叶子(一片)、蒜(一头)。}  ¶ \textcolor{darkblue}{\textbf{\ipa{bæ˩bæ˩˥ | ɖɯ˧-tsʰɤ˧˥}}} \textcolor{PineGreen}{\selectlanguage{french}une fleur} \zh{一朵花}  

\lhead{\firstmark}
\rhead{\botmark}

\subsection{\hspace{-0.5cm} {\Large \textcolor{darkblue}{\textbf{\ipa{tsʰi\#˥}}}}\hspace{0.5cm}[\kern2pt{\textcolor{darkblue}{\textbf{\ipa{tsʰi˥}}}}\kern2pt]} \hypertarget{ts\string_hi\#\string_T1}{}
\markboth{\textcolor{darkblue}{\textbf{\ipa{tsʰi\#˥}}}}{}
\textcolor{teal}{\mytextsc{nom}} \hspace{4pt} Ton~: \#H.
\textcolor{PineGreen}{\selectlanguage{french}Saison sèche (hiver et printemps; du 9e mois au 2e mois du calendrier lunaire compris).} \zh{旱季(冬天至春天:农历九月到来年二月)。} 
\lhead{\firstmark}
\rhead{\botmark}

\subsection{\hspace{-0.5cm} {\Large \textcolor{darkblue}{\textbf{\ipa{tsʰi˥\textsubscript{a}}}}}\hspace{0.5cm}[\kern2pt{\textcolor{darkblue}{\textbf{\ipa{tsʰi˥}}}}\kern2pt]} \hypertarget{ts\string_hi\string_Ta1}{}
\markboth{\textcolor{darkblue}{\textbf{\ipa{tsʰi˥\textsubscript{a}}}}}{}
\textcolor{teal}{\mytextsc{classificateur}} \hspace{4pt} Ton~: H\textsubscript{a}.
\textcolor{PineGreen}{\selectlanguage{french}Classificateur des peaux d'animaux, et des pièces de tissu.} \zh{量词:动物皮(一张),布料(一块)。}  ¶ \textcolor{darkblue}{\textbf{\ipa{ɖɯ˧-tsʰi˥}}} \textcolor{PineGreen}{\selectlanguage{french}une peau} \zh{一张动物皮}  
 ¶ \textcolor{darkblue}{\textbf{\ipa{ɖɯ˧-tsʰi˧ ɲi˥}}} \textcolor{PineGreen}{\selectlanguage{french}c'est une peau} \zh{这是一张(动物皮)}  

\lhead{\firstmark}
\rhead{\botmark}

\subsection{\hspace{-0.5cm} {\Large \textcolor{darkblue}{\textbf{\ipa{tsʰi˧}}} \textsubscript{1}}\hspace{0.5cm}[\kern2pt{\textcolor{darkblue}{\textbf{\ipa{tsʰi˥}}}}\kern2pt]} \hypertarget{ts\string_hi\string_M1}{}
\markboth{\textcolor{darkblue}{\textbf{\ipa{tsʰi˧}}} \textsubscript{1}}{}
\textcolor{teal}{\mytextsc{adjectif}} \hspace{4pt} Ton~: M.
\textcolor{PineGreen}{\selectlanguage{french}Chaud.} \zh{热,烫。}  ¶ \textcolor{darkblue}{\textbf{\ipa{tsʰi˧-zo˧ mɤ˧-tʰɑ˧˥!}}} \textcolor{PineGreen}{\selectlanguage{french}il fait une chaleur insupportable!} \zh{热得受不了!}  
\textit{Voir~:} \hyperlink{}{\textcolor{darkblue}{\textbf{\ipa{tsʰi˧}}} \textsubscript{2}} 
\lhead{\firstmark}
\rhead{\botmark}

\subsection{\hspace{-0.5cm} {\Large \textcolor{darkblue}{\textbf{\ipa{tsʰi˧}}} \textsubscript{2}}\hspace{0.5cm}[\kern2pt{\textcolor{darkblue}{\textbf{\ipa{tsʰi˥}}}}\kern2pt]} \hypertarget{ts\string_hi\string_M2}{}
\markboth{\textcolor{darkblue}{\textbf{\ipa{tsʰi˧}}} \textsubscript{2}}{}
\textcolor{teal}{\mytextsc{adjectif}} \hspace{4pt} Ton~: M.
\textcolor{PineGreen}{\selectlanguage{french}Brillant, lumineux, ardent.} \zh{明亮。}  ¶ \textcolor{darkblue}{\textbf{\ipa{ɲi˧mi˧ tsʰi˧}}} \textcolor{PineGreen}{\selectlanguage{french}le soleil est très fort/le soleil tape dur} \zh{太阳很晒}  
 ¶ \textcolor{darkblue}{\textbf{\ipa{ɬi˧mi˧ tsʰi˧}}} \textcolor{PineGreen}{\selectlanguage{french}la lune brille, la lune luit, on y voit clair à la lumière de la lune} \zh{月亮很亮、月光很明亮}  
\textit{Voir~:} \hyperlink{}{\textcolor{darkblue}{\textbf{\ipa{tsʰi˧}}} \textsubscript{1}} 
\lhead{\firstmark}
\rhead{\botmark}

\subsection{\hspace{-0.5cm} {\Large \textcolor{darkblue}{\textbf{\ipa{tsʰi˧\textsubscript{b}}}}}\hspace{0.5cm}[\kern2pt{\textcolor{darkblue}{\textbf{\ipa{tsʰi˥}}}}\kern2pt]} \hypertarget{ts\string_hi\string_Mb1}{}
\markboth{\textcolor{darkblue}{\textbf{\ipa{tsʰi˧\textsubscript{b}}}}}{}
\textcolor{teal}{\mytextsc{verbe}} \hspace{4pt} Ton~: M\textsubscript{b}.
\textcolor{PineGreen}{\selectlanguage{french}Porter (un chapeau).} \zh{戴帽子。}  ¶ \textcolor{darkblue}{\textbf{\ipa{tv̩˧tv̩˥ tsʰi˩}}} \textcolor{PineGreen}{\selectlanguage{french}mettre un chapeau} \zh{戴上帽子}  

\lhead{\firstmark}
\rhead{\botmark}

\subsection{\hspace{-0.5cm} {\Large \textcolor{darkblue}{\textbf{\ipa{tsʰi˧bv̩˩}}}}\hspace{0.5cm}[\kern2pt{\textcolor{darkblue}{\textbf{\ipa{tsʰi˧bv̩˩}}}}\kern2pt]} \hypertarget{ts\string_hi\string_Mbv\string_=\string_B1}{}
\markboth{\textcolor{darkblue}{\textbf{\ipa{tsʰi˧bv̩˩}}}}{}
\textcolor{teal}{\mytextsc{adjectif}} \hspace{4pt} Ton~: L\#.
\textcolor{PineGreen}{\selectlanguage{french}Étouffant.} \zh{闷热。} 
\lhead{\firstmark}
\rhead{\botmark}

\subsection{\hspace{-0.5cm} {\Large \textcolor{darkblue}{\textbf{\ipa{tsʰi˧ʝi\#˥}}}}\hspace{0.5cm}[\kern2pt{\textcolor{darkblue}{\textbf{\ipa{tsʰi˧ʝi˧}}}}\kern2pt]} \hypertarget{ts\string_hi\string_Mj££i\#\string_T1}{}
\markboth{\textcolor{darkblue}{\textbf{\ipa{tsʰi˧ʝi\#˥}}}}{}
\textcolor{teal}{\mytextsc{adverbe}} \hspace{4pt} Ton~: \#H.
\textcolor{PineGreen}{\selectlanguage{french}Cette année.} \zh{今年。}  ¶ \textcolor{darkblue}{\textbf{\ipa{tsʰi˧ʝi˧-se˥, | …}}} \textcolor{PineGreen}{\selectlanguage{french}Jusqu'à cette année, ...} \zh{到了今年,……}  

\lhead{\firstmark}
\rhead{\botmark}

\subsection{\hspace{-0.5cm} {\Large \textcolor{darkblue}{\textbf{\ipa{tsʰi˧ɲi\#˥}}}}\hspace{0.5cm}[\kern2pt{\textcolor{darkblue}{\textbf{\ipa{tsʰi˧ɲi˧}}}}\kern2pt]} \hypertarget{ts\string_hi\string_MJi\#\string_T1}{}
\markboth{\textcolor{darkblue}{\textbf{\ipa{tsʰi˧ɲi\#˥}}}}{}
\textcolor{teal}{\mytextsc{adverbe}} \hspace{4pt} Ton~: \#H.
\textcolor{PineGreen}{\selectlanguage{french}Aujourd'hui.} \zh{今天。}  ¶ \textcolor{darkblue}{\textbf{\ipa{tsʰi˧ɲi˧-ʁo˧dɑ˧}}} \textcolor{PineGreen}{\selectlanguage{french}avant ce jour, avant aujourd'hui; précédemment} \zh{今天之前}  

\lhead{\firstmark}
\rhead{\botmark}

\subsection{\hspace{-0.5cm} {\Large \textcolor{darkblue}{\textbf{\ipa{tsʰi˧qʰæ˧˥}}}}\hspace{0.5cm}[\kern2pt{\textcolor{darkblue}{\textbf{\ipa{tsʰi˧qʰæ˧˥}}}}\kern2pt]} \hypertarget{ts\string_hi\string_Mq\string_h\{\string_M\string_T1}{}
\markboth{\textcolor{darkblue}{\textbf{\ipa{tsʰi˧qʰæ˧˥}}}}{}
\textcolor{teal}{\mytextsc{adverbe}} \hspace{4pt} Ton~: MH.
\textcolor{PineGreen}{\selectlanguage{french}En ce moment, actuellement, maintenant.} \zh{现在。} 
\lhead{\firstmark}
\rhead{\botmark}

\subsection{\hspace{-0.5cm} {\Large \textcolor{darkblue}{\textbf{\ipa{tsʰi˧si˩-dʑɤ˩pv̩˩}}}}\hspace{0.5cm}[\kern2pt{\textcolor{darkblue}{\textbf{\ipa{tsʰi˧si˩dʑɤ˧pv̩˧}}}}\kern2pt]} \hypertarget{ts\string_hi\string_Msi\string_B-dz£7\string_Bpv\string_=\string_B1}{}
\markboth{\textcolor{darkblue}{\textbf{\ipa{tsʰi˧si˩-dʑɤ˩pv̩˩}}}}{}
\textcolor{teal}{\mytextsc{nom}} \hspace{4pt} Ton~: L\#-.
\textcolor{PineGreen}{\selectlanguage{french}Le monde des esprits, le monde des morts.} \zh{神灵的世界、死人的世界。} 
\lhead{\firstmark}
\rhead{\botmark}

\subsection{\hspace{-0.5cm} {\Large \textcolor{darkblue}{\textbf{\ipa{tsʰi˧ti\#˥}}}}\hspace{0.5cm}[\kern2pt{\textcolor{darkblue}{\textbf{\ipa{tsʰi˧ti˧}}}}\kern2pt]} \hypertarget{ts\string_hi\string_Mti\#\string_T1}{}
\markboth{\textcolor{darkblue}{\textbf{\ipa{tsʰi˧ti\#˥}}}}{}
\textcolor{teal}{\mytextsc{nom}} \hspace{4pt} Ton~: \#H.
\textcolor{PineGreen}{\selectlanguage{french}Prénom masculin.} \zh{男性名字。} 
\lhead{\firstmark}
\rhead{\botmark}

\subsection{\hspace{-0.5cm} {\Large \textcolor{darkblue}{\textbf{\ipa{tsʰi˧zi\#˥}}}}\hspace{0.5cm}[\kern2pt{\textcolor{darkblue}{\textbf{\ipa{tsʰi˧zi˧}}}}\kern2pt]} \hypertarget{ts\string_hi\string_Mzi\#\string_T1}{}
\markboth{\textcolor{darkblue}{\textbf{\ipa{tsʰi˧zi\#˥}}}}{}
\textcolor{teal}{\mytextsc{nom}} \hspace{4pt} Ton~: \#H.
\textcolor{PineGreen}{\selectlanguage{french}Orge d'altitude, \textit{Hordeum vulgare var. nudum Hook. f.}.} \zh{青稞。}  ¶ \textcolor{darkblue}{\textbf{\ipa{tsʰi˧zi˧ | nɑ˩-hĩ˩˥}}} \textcolor{PineGreen}{\selectlanguage{french}orge noir} \zh{黑青稞}  
 ¶ \textcolor{darkblue}{\textbf{\ipa{tsʰi˧zi˧ | pʰv̩˩-hĩ˩˥}}} \textcolor{PineGreen}{\selectlanguage{french}orge blanc} \zh{白青稞}  
 \zh{量词}: \textcolor{darkblue}{\textbf{\ipa{kɤ˧˥}}}  \mytextsc{clf}~: \textcolor{darkblue}{\textbf{\ipa{kɤ˧˥}}} 
\lhead{\firstmark}
\rhead{\botmark}

\subsection{\hspace{-0.5cm} {\Large \textcolor{darkblue}{\textbf{\ipa{tsʰi˧zi˧-ɻ̃\#˥}}}}\hspace{0.5cm}[\kern2pt{\textcolor{darkblue}{\textbf{\ipa{xxxx non-correspondance entre le nombre de morphèmes et le nombre de tons de morphèmes}}}}\kern2pt]} \hypertarget{ts\string_hi\string_Mzi\string_M-r£`\string_~\#\string_T1}{}
\markboth{\textcolor{darkblue}{\textbf{\ipa{tsʰi˧zi˧-ɻ̃\#˥}}}}{}
\textcolor{teal}{\mytextsc{nom}} \hspace{4pt} Ton~: \#H.
\textcolor{PineGreen}{\selectlanguage{french}Paille d'orge.} \zh{青稞杆。}  \zh{量词}: \textcolor{darkblue}{\textbf{\ipa{kɤ˧˥}}}  \mytextsc{clf}~: \textcolor{darkblue}{\textbf{\ipa{kɤ˧˥}}} 
\lhead{\firstmark}
\rhead{\botmark}

\subsection{\hspace{-0.5cm} {\Large \textcolor{darkblue}{\textbf{\ipa{tsʰi˩\textsubscript{a}}}}}\hspace{0.5cm}[\kern2pt{\textcolor{darkblue}{\textbf{\ipa{tsʰi˩˥}}}}\kern2pt]} \hypertarget{ts\string_hi\string_Ba1}{}
\markboth{\textcolor{darkblue}{\textbf{\ipa{tsʰi˩\textsubscript{a}}}}}{}
\textcolor{teal}{\mytextsc{adjectif}} \hspace{4pt} Ton~: L\textsubscript{a}.
\textcolor{PineGreen}{\selectlanguage{french}Fin (objet).} \zh{细(树、体型细小)。}  ¶ \textcolor{darkblue}{\textbf{\ipa{tsʰi˩-hĩ˩˥}}} \textcolor{PineGreen}{\selectlanguage{french}\mytextsc{nmlz}} \zh{细的}  
 ¶ \textcolor{darkblue}{\textbf{\ipa{qʰɑ˧-tsʰi˧-gv̩˧}}} \textcolor{PineGreen}{\selectlanguage{french}très fin} \zh{非常细}  
 ¶ \textcolor{darkblue}{\textbf{\ipa{dʑɤ˧˥ | tsʰi˩-njæ˩˥ | -gv̩˩!}}} \textcolor{PineGreen}{\selectlanguage{french}C'est vraiment fin!} \zh{真细!}  

\lhead{\firstmark}
\rhead{\botmark}

\subsection{\hspace{-0.5cm} {\Large \textcolor{darkblue}{\textbf{\ipa{tsʰi˩mv̩˩tʰv̩˩}}}}\hspace{0.5cm}[\kern2pt{\textcolor{darkblue}{\textbf{\ipa{tsʰi˩mv̩˩tʰv̩˩˥}}}}\kern2pt]} \hypertarget{ts\string_hi\string_Bmv\string_=\string_Bt\string_hv\string_=\string_B1}{}
\markboth{\textcolor{darkblue}{\textbf{\ipa{tsʰi˩mv̩˩tʰv̩˩}}}}{}
\textcolor{teal}{\mytextsc{nom}} \hspace{4pt} Ton~: L.
\textcolor{PineGreen}{\selectlanguage{french}Démon qui danse.} \zh{跳着的鬼。} 
\lhead{\firstmark}
\rhead{\botmark}

\subsection{\hspace{-0.5cm} {\Large \textcolor{darkblue}{\textbf{\ipa{tsʰi˩tv̩˩}}}}\hspace{0.5cm}[\kern2pt{\textcolor{darkblue}{\textbf{\ipa{tsʰi˩tv̩˩˥}}}}\kern2pt]} \hypertarget{ts\string_hi\string_Btv\string_=\string_B1}{}
\markboth{\textcolor{darkblue}{\textbf{\ipa{tsʰi˩tv̩˩}}}}{}
\textcolor{teal}{\mytextsc{nom}} \hspace{4pt} Ton~: L.
\textcolor{PineGreen}{\selectlanguage{french}Moëlle.} \zh{骨髓。}  \zh{量词}: \textcolor{darkblue}{\textbf{\ipa{kʰwɤ˥}}}  \mytextsc{clf}~: \textcolor{darkblue}{\textbf{\ipa{kʰwɤ˥}}} 
\lhead{\firstmark}
\rhead{\botmark}

\subsection{\hspace{-0.5cm} {\Large \textcolor{darkblue}{\textbf{\ipa{tsʰi˩tsʰi˩}}}}\hspace{0.5cm}[\kern2pt{\textcolor{darkblue}{\textbf{\ipa{tsʰi˩tsʰi˩˥}}}}\kern2pt]} \hypertarget{ts\string_hi\string_Bts\string_hi\string_B1}{}
\markboth{\textcolor{darkblue}{\textbf{\ipa{tsʰi˩tsʰi˩}}}}{}
\textcolor{teal}{\mytextsc{nom}} \hspace{4pt} Ton~: L.
\textcolor{PineGreen}{\selectlanguage{french}Pois, petits pois.} \zh{豌豆。}  \zh{量词}: \textcolor{darkblue}{\textbf{\ipa{kɤ˧˥}}}  \mytextsc{clf}~: \textcolor{darkblue}{\textbf{\ipa{kɤ˧˥}}} 
\lhead{\firstmark}
\rhead{\botmark}

\subsection{\hspace{-0.5cm} {\Large \textcolor{darkblue}{\textbf{\ipa{tsʰi˧˥}}} \textsubscript{1}}\hspace{0.5cm}[\kern2pt{\textcolor{darkblue}{\textbf{\ipa{tsʰi˧˥}}}}\kern2pt]} \hypertarget{ts\string_hi\string_M\string_T1}{}
\markboth{\textcolor{darkblue}{\textbf{\ipa{tsʰi˧˥}}} \textsubscript{1}}{}
\textcolor{teal}{\mytextsc{verbe}} \hspace{4pt} Ton~: MH.
\textcolor{PineGreen}{\selectlanguage{french}Construire.} \zh{盖,建 (房子)。}  ¶ \textcolor{darkblue}{\textbf{\ipa{ʑi˧qʰwɤ˧ tsʰi˧˥}}} \textcolor{PineGreen}{\selectlanguage{french}construire un bâtiment} \zh{建 房子}  

\lhead{\firstmark}
\rhead{\botmark}

\subsection{\hspace{-0.5cm} {\Large \textcolor{darkblue}{\textbf{\ipa{tsʰi˧˥}}} \textsubscript{2}}\hspace{0.5cm}[\kern2pt{\textcolor{darkblue}{\textbf{\ipa{tsʰi˧˥}}}}\kern2pt]} \hypertarget{ts\string_hi\string_M\string_T2}{}
\markboth{\textcolor{darkblue}{\textbf{\ipa{tsʰi˧˥}}} \textsubscript{2}}{}
\textcolor{teal}{\mytextsc{verbe}} \hspace{4pt} Ton~: MH.
\textcolor{PineGreen}{\selectlanguage{french}Percer un trou, faire un trou (ex.: dans un tissu, un mur...).} \zh{穿一个洞。} 
\lhead{\firstmark}
\rhead{\botmark}

\subsection{\hspace{-0.5cm} {\Large \textcolor{darkblue}{\textbf{\ipa{tsʰi˧˥}}} \textsubscript{3}}\hspace{0.5cm}[\kern2pt{\textcolor{darkblue}{\textbf{\ipa{tsʰi˧˥}}}}\kern2pt]} \hypertarget{ts\string_hi\string_M\string_T3}{}
\markboth{\textcolor{darkblue}{\textbf{\ipa{tsʰi˧˥}}} \textsubscript{3}}{}
\textcolor{teal}{\mytextsc{verbe}} \hspace{4pt} Ton~: MH.
\textcolor{PineGreen}{\selectlanguage{french}Allumer (un feu).} \zh{点(火)。}  ¶ \textcolor{darkblue}{\textbf{\ipa{mv̩˧ tsʰi˧˥}}} \textcolor{PineGreen}{\selectlanguage{french}allumer un feu} \zh{点火}  
 ¶ \textcolor{darkblue}{\textbf{\ipa{njɤ˧-ɳɯ˧ | mv̩˧tsʰi˧-bi˥}}} \textcolor{PineGreen}{\selectlanguage{french}je vais allumer le feu} \zh{我要点个火}  
 ¶ \textcolor{darkblue}{\textbf{\ipa{mv̩˩tsʰo˩ tsʰi˧}}} \textcolor{PineGreen}{\selectlanguage{french}mettre le feu à un bout de bois plein de sève (pour faire partir le feu)} \zh{用含很多树脂的木头来引火}  

\lhead{\firstmark}
\rhead{\botmark}

\subsection{\hspace{-0.5cm} {\Large \textcolor{darkblue}{\textbf{\ipa{tsʰi˧˥}}} \textsubscript{4}}\hspace{0.5cm}[\kern2pt{\textcolor{darkblue}{\textbf{\ipa{tsʰi˧˥}}}}\kern2pt]} \hypertarget{ts\string_hi\string_M\string_T4}{}
\markboth{\textcolor{darkblue}{\textbf{\ipa{tsʰi˧˥}}} \textsubscript{4}}{}
\textcolor{teal}{\mytextsc{verbe}} \hspace{4pt} Ton~: MH.
\textcolor{PineGreen}{\selectlanguage{french}S'accroupir.} \zh{蹲。}  ¶ \textcolor{darkblue}{\textbf{\ipa{le˧-tsʰi˩\textasciitilde{}tsʰi˩ | tʰi˧-dzi˩}}} \textcolor{PineGreen}{\selectlanguage{french}être accroupi (être assis avec les genoux regroupés sur la poitrine)} \zh{盘腿坐}  
 ¶ \textcolor{darkblue}{\textbf{\ipa{gɤ˩-tsʰi˧\textasciitilde{}tsʰi˩ tʰi˧-dzi˩}}} \textcolor{PineGreen}{\selectlanguage{french}même sens} \zh{盘腿坐}  

\lhead{\firstmark}
\rhead{\botmark}

\subsection{\hspace{-0.5cm} {\Large \textcolor{darkblue}{\textbf{\ipa{tsʰi˧˥}}} \textsubscript{5}}\hspace{0.5cm}[\kern2pt{\textcolor{darkblue}{\textbf{\ipa{tsʰi˧˥}}}}\kern2pt]} \hypertarget{ts\string_hi\string_M\string_T5}{}
\markboth{\textcolor{darkblue}{\textbf{\ipa{tsʰi˧˥}}} \textsubscript{5}}{}
\textcolor{teal}{\mytextsc{adjectif}} \hspace{4pt} Ton~: MH.
\textit{\textcolor{PineGreen}{\selectlanguage{french}archaïque}} [\zh{古语}] \textcolor{PineGreen}{\selectlanguage{french}Malade, souffrant.} \zh{病。}  ¶ \textcolor{darkblue}{\textbf{\ipa{mɤ˧-go˩ mɤ˩-tsʰi˩-ɻ̍˩ |}}} \textcolor{PineGreen}{\selectlanguage{french}être bien portant, ne pas être malade} \zh{健康:不病、不痛}  

\lhead{\firstmark}
\rhead{\botmark}

\subsection{\hspace{-0.5cm} {\Large \textcolor{darkblue}{\textbf{\ipa{tsʰo˥}}}}\hspace{0.5cm}[\kern2pt{\textcolor{darkblue}{\textbf{\ipa{tsʰo˥}}}}\kern2pt]} \hypertarget{ts\string_ho\string_T1}{}
\markboth{\textcolor{darkblue}{\textbf{\ipa{tsʰo˥}}}}{}
\textcolor{teal}{\mytextsc{adjectif}} \hspace{4pt} Ton~: H.
\textcolor{PineGreen}{\selectlanguage{french}Complet, au grand complet.} \zh{齐全。}  ¶ \textcolor{darkblue}{\textbf{\ipa{ə˧tso˧-mɤ˧-ɲi˩, | tʰi˧-tsʰo˥-ze˩!}}} \textcolor{PineGreen}{\selectlanguage{french}Tout y est! Tout est prêt! (Au sujet de préparatifs pour une fête, un repas...)} \zh{什么都准备得很齐全!}  
 ¶ \textcolor{darkblue}{\textbf{\ipa{mɤ˧-tsʰo˧-sɯ˥! | wɤ˩˥ | ɲi˧-bæ˧ hwæ˧-zo˧-ho˩!}}} \textcolor{PineGreen}{\selectlanguage{french}On n'y est pas encore tout à fait / ce n'est pas encore tout à fait prêt! Il reste deux ou trois trucs à acheter! (Contexte: on achève la décoration d'un appartement à la ville; aux compliments des visiteurs, l'heureux propriétaire répond: 'Ce n'est pas encore terminé!')} \zh{还不算齐全! / 还没有装饰齐全!(情景:朋友们表扬新装修的丽江房子,主人谦虚回答:‘还不算齐全!’)}  
 ¶ \textcolor{darkblue}{\textbf{\ipa{tʰi˧-tsʰo˥-kʰɯ˩}}} \textcolor{PineGreen}{\selectlanguage{french}\mytextsc{dur} \string_ \mytextsc{caus}: porter à son point d'achèvement, porter au grand complet} \zh{\mytextsc{dur} \string_ \mytextsc{caus:完成、弄齐全}}  

\lhead{\firstmark}
\rhead{\botmark}

\subsection{\hspace{-0.5cm} {\Large \textcolor{darkblue}{\textbf{\ipa{tsʰo˧\textsubscript{b}}}}}\hspace{0.5cm}[\kern2pt{\textcolor{darkblue}{\textbf{\ipa{tsʰo˩˥}}}}\kern2pt]} \hypertarget{ts\string_ho\string_Mb1}{}
\markboth{\textcolor{darkblue}{\textbf{\ipa{tsʰo˧\textsubscript{b}}}}}{}
\textcolor{teal}{\mytextsc{verbe}} \hspace{4pt} Ton~: M\textsubscript{b}.
\textcolor{PineGreen}{\selectlanguage{french}Sauter.} \zh{跳。}  ¶ \textcolor{darkblue}{\textbf{\ipa{bæ˧ tsʰo˧}}} \textcolor{PineGreen}{\selectlanguage{french}sauter à la corde} \zh{跳绳}  
 ¶ \textcolor{darkblue}{\textbf{\ipa{tsʰo˧\textasciitilde{}tsʰo˧}}} \textcolor{PineGreen}{\selectlanguage{french}forme rédupliquée: trépigner, sautiller ici et là} \zh{\mytextsc{重叠}}  

\lhead{\firstmark}
\rhead{\botmark}

\subsection{\hspace{-0.5cm} {\Large \textcolor{darkblue}{\textbf{\ipa{tsʰo˧ɖɯ˩}}}}\hspace{0.5cm}[\kern2pt{\textcolor{darkblue}{\textbf{\ipa{tsʰo˧ɖɯ˩}}}}\kern2pt]} \hypertarget{ts\string_ho\string_Md`M\string_B1}{}
\markboth{\textcolor{darkblue}{\textbf{\ipa{tsʰo˧ɖɯ˩}}}}{}
\textcolor{teal}{\mytextsc{nom}} \hspace{4pt} Ton~: L\#.
\textcolor{PineGreen}{\selectlanguage{french}Danse en groupe: parfois dix personnes, parfois jusqu'à cent (un village entier).} \zh{集体舞。}  ¶ \textcolor{darkblue}{\textbf{\ipa{tsʰo˧ɖɯ˩ tsʰo˩}}} \textcolor{PineGreen}{\selectlanguage{french}faire une grande danse collective} \zh{跳一个集体舞}  

\lhead{\firstmark}
\rhead{\botmark}

\subsection{\hspace{-0.5cm} {\Large \textcolor{darkblue}{\textbf{\ipa{tsʰo˧ɖwæ\#˥}}}}\hspace{0.5cm}[\kern2pt{\textcolor{darkblue}{\textbf{\ipa{tsʰo˧ɖwæ˧}}}}\kern2pt]} \hypertarget{ts\string_ho\string_Md`w\{\#\string_T1}{}
\markboth{\textcolor{darkblue}{\textbf{\ipa{tsʰo˧ɖwæ\#˥}}}}{}
\textcolor{teal}{\mytextsc{nom}} \hspace{4pt} Ton~: \#H.
\textcolor{PineGreen}{\selectlanguage{french}Marche en pierre.} \zh{石头台阶。}  \zh{量词}: \textcolor{darkblue}{\textbf{\ipa{ɖwæ˥}}}  \mytextsc{clf}~: \textcolor{darkblue}{\textbf{\ipa{ɖwæ˥}}} 
\lhead{\firstmark}
\rhead{\botmark}

\subsection{\hspace{-0.5cm} {\Large \textcolor{darkblue}{\textbf{\ipa{tsʰo˧ko˧}}}}\hspace{0.5cm}[\kern2pt{\textcolor{darkblue}{\textbf{\ipa{tsʰo˧ko˧}}}}\kern2pt]} \hypertarget{ts\string_ho\string_Mko\string_M1}{}
\markboth{\textcolor{darkblue}{\textbf{\ipa{tsʰo˧ko˧}}}}{}
\textcolor{teal}{\mytextsc{nom}} \hspace{4pt} Ton~: M.
\textcolor{PineGreen}{\selectlanguage{french}Cardamome, \textit{Elletaria cardamomum}.} \zh{小豆蔻。} Dialecte chinois local~:\zh{草果。} Emprunt~: \zh{草果}
 \zh{量词}: \textcolor{darkblue}{\textbf{\ipa{ɭɯ˧}}}  \mytextsc{clf}~: \textcolor{darkblue}{\textbf{\ipa{ɭɯ˧}}} 
\lhead{\firstmark}
\rhead{\botmark}

\subsection{\hspace{-0.5cm} {\Large \textcolor{darkblue}{\textbf{\ipa{tsʰo˧pæ\#˥}}}}\hspace{0.5cm}[\kern2pt{\textcolor{darkblue}{\textbf{\ipa{tsʰo˧pæ˧}}}}\kern2pt]} \hypertarget{ts\string_ho\string_Mp\{\#\string_T1}{}
\markboth{\textcolor{darkblue}{\textbf{\ipa{tsʰo˧pæ\#˥}}}}{}
\textcolor{teal}{\mytextsc{nom}} \hspace{4pt} Ton~: \#H.
\textcolor{PineGreen}{\selectlanguage{french}Chef de caravane.} \zh{马帮头领。}  Emprunt~: tibétain  tshong.pa “merchant"

\lhead{\firstmark}
\rhead{\botmark}

\subsection{\hspace{-0.5cm} {\Large \textcolor{darkblue}{\textbf{\ipa{tsʰo˧pjɤ˧}}}}\hspace{0.5cm}[\kern2pt{\textcolor{darkblue}{\textbf{\ipa{tsʰo˧pjɤ˧}}}}\kern2pt]} \hypertarget{ts\string_ho\string_Mpj7\string_M1}{}
\markboth{\textcolor{darkblue}{\textbf{\ipa{tsʰo˧pjɤ˧}}}}{}
\textcolor{teal}{\mytextsc{nom}} \hspace{4pt} Ton~: M.
\textcolor{PineGreen}{\selectlanguage{french}Savon. Sans doute mot emprunté à une langue de birmanie: cp. nung: tshɑ³¹ pi⁵⁵ iɔ⁵⁵ [Dai 1992]; achang de Luxi et Lianghe: tshɑu⁵⁵ pjɑu⁵⁵ [Dai 1985]; achang de Longchuan: tshau³¹ piau³¹ [Dai 1992]. Culturellement, il est plausible que le savon ait été introduit par le contact/commerce avec des groupes ethniques de Birmanie.} \zh{肥皂。}  \zh{量词}: \textcolor{darkblue}{\textbf{\ipa{ɭɯ˧}}}  \mytextsc{clf}~: \textcolor{darkblue}{\textbf{\ipa{ɭɯ˧}}} 
\lhead{\firstmark}
\rhead{\botmark}

\subsection{\hspace{-0.5cm} {\Large \textcolor{darkblue}{\textbf{\ipa{tsʰo˧qʰwɤ˧mi\#˥}}}}\hspace{0.5cm}[\kern2pt{\textcolor{darkblue}{\textbf{\ipa{tsʰo˧qʰwɤ˧mi˧}}}}\kern2pt]} \hypertarget{ts\string_ho\string_Mq\string_hw7\string_Mmi\#\string_T1}{}
\markboth{\textcolor{darkblue}{\textbf{\ipa{tsʰo˧qʰwɤ˧mi\#˥}}}}{}
\textcolor{teal}{\mytextsc{nom}} \hspace{4pt} Ton~: \#H.
\textcolor{PineGreen}{\selectlanguage{french}Démon, fantôme.} \zh{鬼。}  \zh{量词}: \textcolor{darkblue}{\textbf{\ipa{v̩˧}}}  \mytextsc{clf}~: \textcolor{darkblue}{\textbf{\ipa{v̩˧}}} \textit{Voir~:} \hyperlink{}{\textcolor{darkblue}{\textbf{\ipa{tsʰo˧qʰwɤ˧zo\#˥}}}} 
\lhead{\firstmark}
\rhead{\botmark}

\subsection{\hspace{-0.5cm} {\Large \textcolor{darkblue}{\textbf{\ipa{tsʰo˧qʰwɤ˧mi˧-bæ˥bæ˩}}}}\hspace{0.5cm}[\kern2pt{\textcolor{darkblue}{\textbf{\ipa{tsʰo˧qʰwɤ˧mi˧bæ˥bæ˩}}}}\kern2pt]} \hypertarget{ts\string_ho\string_Mq\string_hw7\string_Mmi\string_M-b\{\string_Tb\{\string_B1}{}
\markboth{\textcolor{darkblue}{\textbf{\ipa{tsʰo˧qʰwɤ˧mi˧-bæ˥bæ˩}}}}{}
\textcolor{teal}{\mytextsc{nom}} \hspace{4pt} Ton~: \#H-.
\textcolor{PineGreen}{\selectlanguage{french}Une fleur bleue, \textit{Delphinium grandiflorum}.} \zh{翠雀花。} \textit{Voir~:} \hyperlink{}{\textcolor{darkblue}{\textbf{\ipa{tsʰo˧qʰwɤ˧mi\#˥}}}} 
\lhead{\firstmark}
\rhead{\botmark}

\subsection{\hspace{-0.5cm} {\Large \textcolor{darkblue}{\textbf{\ipa{tsʰo˧qʰwɤ˧zo\#˥}}}}\hspace{0.5cm}[\kern2pt{\textcolor{darkblue}{\textbf{\ipa{tsʰo˧qʰwɤ˧zo˧}}}}\kern2pt]} \hypertarget{ts\string_ho\string_Mq\string_hw7\string_Mzo\#\string_T1}{}
\markboth{\textcolor{darkblue}{\textbf{\ipa{tsʰo˧qʰwɤ˧zo\#˥}}}}{}
\textcolor{teal}{\mytextsc{nom}} \hspace{4pt} Ton~: \#H.
\textcolor{PineGreen}{\selectlanguage{french}Démon, fantôme (forme moins courante que celle comportant un suffixe féminin).} \zh{鬼。} \textit{Voir~:} \hyperlink{}{\textcolor{darkblue}{\textbf{\ipa{tsʰo˧qʰwɤ˧mi\#˥}}}} 
\lhead{\firstmark}
\rhead{\botmark}

\subsection{\hspace{-0.5cm} {\Large \textcolor{darkblue}{\textbf{\ipa{tsʰo˧qʰwɤ˩}}}}\hspace{0.5cm}[\kern2pt{\textcolor{darkblue}{\textbf{\ipa{tsʰo˧qʰwɤ˩}}}}\kern2pt]} \hypertarget{ts\string_ho\string_Mq\string_hw7\string_B1}{}
\markboth{\textcolor{darkblue}{\textbf{\ipa{tsʰo˧qʰwɤ˩}}}}{}
\textcolor{teal}{\mytextsc{adverbe}} \hspace{4pt} Ton~: L\#.
\textcolor{PineGreen}{\selectlanguage{french}Ce soir.} \zh{今晚。}  ¶ \textcolor{darkblue}{\textbf{\ipa{tsʰo˧qʰwɤ˩ | mv̩˩kʰv̩˧˥}}} \textcolor{PineGreen}{\selectlanguage{french}même sens: ce soir} \zh{同上:今晚}  

\lhead{\firstmark}
\rhead{\botmark}

\subsection{\hspace{-0.5cm} {\Large \textcolor{darkblue}{\textbf{\ipa{tsʰo˧ʁo\#˥}}}}\hspace{0.5cm}[\kern2pt{\textcolor{darkblue}{\textbf{\ipa{tsʰo˧ʁo˧}}}}\kern2pt]} \hypertarget{ts\string_ho\string_MRo\#\string_T1}{}
\markboth{\textcolor{darkblue}{\textbf{\ipa{tsʰo˧ʁo\#˥}}}}{}
\textcolor{teal}{\mytextsc{nom}} \hspace{4pt} Ton~: \#H.
\textcolor{PineGreen}{\selectlanguage{french}Étable: bâtiment à l'entrée de la ferme, que l'on traverse en entrant dans la ferme. Construit en bois. Au rez-de-chaussée se trouvent les étables des porcs; à l'étage un grenier à foin.} \zh{牲畜圈:家门口的那栋楼,下为畜厩,上存饲料或另辟为房间。}  \zh{量词}: \textcolor{darkblue}{\textbf{\ipa{ɭɯ˧}}}  \mytextsc{clf}~: \textcolor{darkblue}{\textbf{\ipa{ɭɯ˧}}} 
\lhead{\firstmark}
\rhead{\botmark}

\subsection{\hspace{-0.5cm} {\Large \textcolor{darkblue}{\textbf{\ipa{tsʰo˧tsɯ˧}}}}\hspace{0.5cm}[\kern2pt{\textcolor{darkblue}{\textbf{\ipa{tsʰo˧tsɯ˧}}}}\kern2pt]} \hypertarget{ts\string_ho\string_MtsM\string_M1}{}
\markboth{\textcolor{darkblue}{\textbf{\ipa{tsʰo˧tsɯ˧}}}}{}
\textcolor{teal}{\mytextsc{nom}} \hspace{4pt} Ton~: M.
\textcolor{PineGreen}{\selectlanguage{french}Poireau, oignon.} \zh{葱,韭葱。}  Emprunt~: chinois  \zh{葱子}
 \zh{量词}: \textcolor{darkblue}{\textbf{\ipa{po˧}}}  \mytextsc{clf}~: \textcolor{darkblue}{\textbf{\ipa{po˧}}} 
\lhead{\firstmark}
\rhead{\botmark}

\subsection{\hspace{-0.5cm} {\Large \textcolor{darkblue}{\textbf{\ipa{tsʰo˩}}}}\hspace{0.5cm}[\kern2pt{\textcolor{darkblue}{\textbf{\ipa{tsʰo˥}}}}\kern2pt]} \hypertarget{ts\string_ho\string_B1}{}
\markboth{\textcolor{darkblue}{\textbf{\ipa{tsʰo˩}}}}{}
\textcolor{teal}{\mytextsc{nom}} \hspace{4pt} Ton~: L.
\textcolor{PineGreen}{\selectlanguage{french}Espèce humaine, êtres humains; terme ancien apparaissant dans certains proverbes.} \zh{人类。} 
\lhead{\firstmark}
\rhead{\botmark}

\subsection{\hspace{-0.5cm} {\Large \textcolor{darkblue}{\textbf{\ipa{tsʰo˩mo˩}}}}\hspace{0.5cm}[\kern2pt{\textcolor{darkblue}{\textbf{\ipa{tsʰo˩mo˩˥}}}}\kern2pt]} \hypertarget{ts\string_ho\string_Bmo\string_B1}{}
\markboth{\textcolor{darkblue}{\textbf{\ipa{tsʰo˩mo˩}}}}{}
\textcolor{teal}{\mytextsc{nom}} \hspace{4pt} Ton~: L.
\textcolor{PineGreen}{\selectlanguage{french}Vieil homme, vieillard.} \zh{老头。} 
\lhead{\firstmark}
\rhead{\botmark}

\subsection{\hspace{-0.5cm} {\Large \textcolor{darkblue}{\textbf{\ipa{tsʰo˩tsɯ˧}}}}\hspace{0.5cm}[\kern2pt{\textcolor{darkblue}{\textbf{\ipa{tsʰo˩tsɯ˥}}}}\kern2pt]} \hypertarget{ts\string_ho\string_BtsM\string_M1}{}
\markboth{\textcolor{darkblue}{\textbf{\ipa{tsʰo˩tsɯ˧}}}}{}
\textcolor{teal}{\mytextsc{nom}} \hspace{4pt} Ton~: LM.
\textcolor{PineGreen}{\selectlanguage{french}Lime.} \zh{锉刀。}  Emprunt~: chinois  \zh{锉子}
 \zh{量词}: \textcolor{darkblue}{\textbf{\ipa{nɑ˧}}}  \mytextsc{clf}~: \textcolor{darkblue}{\textbf{\ipa{nɑ˧}}} 
\lhead{\firstmark}
\rhead{\botmark}

\subsection{\hspace{-0.5cm} {\Large \textcolor{darkblue}{\textbf{\ipa{tsʰo˧˥}}}}\hspace{0.5cm}[\kern2pt{\textcolor{darkblue}{\textbf{\ipa{tsʰo˧˥}}}}\kern2pt]} \hypertarget{ts\string_ho\string_M\string_T1}{}
\markboth{\textcolor{darkblue}{\textbf{\ipa{tsʰo˧˥}}}}{}
\textcolor{teal}{\mytextsc{nom}} \hspace{4pt} Ton~: MH.
\textcolor{PineGreen}{\selectlanguage{french}Respect, attention, estime.} \zh{重视、关心、恭敬。}  ¶ \textcolor{darkblue}{\textbf{\ipa{ʈʂʰɯ˧-ɳɯ˧ | njɤ˧-ki˧ | ɖwæ˧˥ | tsʰo˧ ʝi˥!}}} \textcolor{PineGreen}{\selectlanguage{french}Il/elle me traite avec les plus grands égards / est aux petits soins pour moi!} \zh{他很重视我 / 他对我很尊敬、很关心。}  

\lhead{\firstmark}
\rhead{\botmark}

\subsection{\hspace{-0.5cm} {\Large \textcolor{darkblue}{\textbf{\ipa{tsʰɯ˧hṽ˥\$}}}}\hspace{0.5cm}[\kern2pt{\textcolor{darkblue}{\textbf{\ipa{tsʰɯ˧hṽ˧}}}}\kern2pt]} \hypertarget{ts\string_hM\string_Mhv\string_~\string_T\$1}{}
\markboth{\textcolor{darkblue}{\textbf{\ipa{tsʰɯ˧hṽ˥\$}}}}{}
\textcolor{teal}{\mytextsc{nom}} \hspace{4pt} Ton~: H\$.
\textcolor{PineGreen}{\selectlanguage{french}Laine.} \zh{羊毛。}  \zh{量词}: \textcolor{darkblue}{\textbf{\ipa{kʰwɤ˥}}}  \mytextsc{clf}~: \textcolor{darkblue}{\textbf{\ipa{kʰwɤ˥}}} 
\lhead{\firstmark}
\rhead{\botmark}

\subsection{\hspace{-0.5cm} {\Large \textcolor{darkblue}{\textbf{\ipa{tsʰɯ˧mi˥\$}}}}\hspace{0.5cm}[\kern2pt{\textcolor{darkblue}{\textbf{\ipa{tsʰɯ˧mi˥}}}}\kern2pt]} \hypertarget{ts\string_hM\string_Mmi\string_T\$1}{}
\markboth{\textcolor{darkblue}{\textbf{\ipa{tsʰɯ˧mi˥\$}}}}{}
\textcolor{teal}{\mytextsc{nom}} \hspace{4pt} Ton~: H\$.
\textcolor{PineGreen}{\selectlanguage{french}Chèvre.} \zh{母山羊。}  ¶ \textcolor{darkblue}{\textbf{\ipa{tsʰɯ˧mi˧-po˧lo˥}}} \textcolor{PineGreen}{\selectlanguage{french}chèvre et bouc} \zh{母山羊与公山羊}  
 \zh{量词}: \textcolor{darkblue}{\textbf{\ipa{pʰo˧˥}}}  \mytextsc{clf}~: \textcolor{darkblue}{\textbf{\ipa{pʰo˧˥}}} 
\lhead{\firstmark}
\rhead{\botmark}

\subsection{\hspace{-0.5cm} {\Large \textcolor{darkblue}{\textbf{\ipa{tsʰɯ˧mi˧-to˧qɑ˥\$}}}}\hspace{0.5cm}[\kern2pt{\textcolor{darkblue}{\textbf{\ipa{xxxx non-correspondance entre le nombre de morphèmes et le nombre de tons de morphèmes}}}}\kern2pt]} \hypertarget{ts\string_hM\string_Mmi\string_M-to\string_MqA\string_T\$1}{}
\markboth{\textcolor{darkblue}{\textbf{\ipa{tsʰɯ˧mi˧-to˧qɑ˥\$}}}}{}
\textcolor{teal}{\mytextsc{nom}} \hspace{4pt} Ton~: H\$.
\textcolor{PineGreen}{\selectlanguage{french}Bouc; s'emploie aussi pour un chevreau (cabri), ou même pour toute l'espèce, y compris les chèvres.} \zh{公山羊(包括公山羊羔)(可以来指所有的山羊,包括母的和公的)。}  \zh{量词}: \textcolor{darkblue}{\textbf{\ipa{pʰo˧˥}}}  \mytextsc{clf}~: \textcolor{darkblue}{\textbf{\ipa{pʰo˧˥}}} 
\lhead{\firstmark}
\rhead{\botmark}

\subsection{\hspace{-0.5cm} {\Large \textcolor{darkblue}{\textbf{\ipa{tsʰɯ˧pʰv̩\#˥}}}}\hspace{0.5cm}[\kern2pt{\textcolor{darkblue}{\textbf{\ipa{tsʰɯ˩pʰv̩˩˥}}}}\kern2pt]} \hypertarget{ts\string_hM\string_Mp\string_hv\string_=\#\string_T1}{}
\markboth{\textcolor{darkblue}{\textbf{\ipa{tsʰɯ˧pʰv̩\#˥}}}}{}
\textcolor{teal}{\mytextsc{nom}} \hspace{4pt} Ton~: \#H.
\textcolor{PineGreen}{\selectlanguage{french}Bouc (terme élicité; plus courant: \textcolor{darkblue}{\textbf{\ipa{/po˧lo˧/}}}).} \zh{公山羊。}  \zh{量词}: \textcolor{darkblue}{\textbf{\ipa{pʰo˧˥}}}  \mytextsc{clf}~: \textcolor{darkblue}{\textbf{\ipa{pʰo˧˥}}} 
\lhead{\firstmark}
\rhead{\botmark}

\subsection{\hspace{-0.5cm} {\Large \textcolor{darkblue}{\textbf{\ipa{tsʰɯ˧ɻ̍\#˥}}}}\hspace{0.5cm}[\kern2pt{\textcolor{darkblue}{\textbf{\ipa{tsʰɯ˧ɻ̍˧}}}}\kern2pt]} \hypertarget{ts\string_hM\string_Mr£`̍\#\string_T1}{}
\markboth{\textcolor{darkblue}{\textbf{\ipa{tsʰɯ˧ɻ̍\#˥}}}}{}
\textcolor{teal}{\mytextsc{nom}} \hspace{4pt} Ton~: \#H.
\textcolor{PineGreen}{\selectlanguage{french}Prénom unisexe: prénom utilisé pour les deux sexes.} \zh{男女通用名。} 
\lhead{\firstmark}
\rhead{\botmark}

\subsection{\hspace{-0.5cm} {\Large \textcolor{darkblue}{\textbf{\ipa{tsʰɯ˧ʂwæ˥}}}}\hspace{0.5cm}[\kern2pt{\textcolor{darkblue}{\textbf{\ipa{tsʰɯ˧ʂwæ˥}}}}\kern2pt]} \hypertarget{ts\string_hM\string_Ms`w\{\string_T1}{}
\markboth{\textcolor{darkblue}{\textbf{\ipa{tsʰɯ˧ʂwæ˥}}}}{}
\textcolor{teal}{\mytextsc{nom}} \hspace{4pt} Ton~: H\#.
\textcolor{PineGreen}{\selectlanguage{french}Bouc castré.} \zh{阉山羊。}  \zh{量词}: \textcolor{darkblue}{\textbf{\ipa{pʰo˧˥}}}  \mytextsc{clf}~: \textcolor{darkblue}{\textbf{\ipa{pʰo˧˥}}} 
\lhead{\firstmark}
\rhead{\botmark}

\subsection{\hspace{-0.5cm} {\Large \textcolor{darkblue}{\textbf{\ipa{tsʰɯ˧-to˧qɑ˥}}}}\hspace{0.5cm}[\kern2pt{\textcolor{darkblue}{\textbf{\ipa{xxxx non-correspondance entre le nombre de morphèmes et le nombre de tons de morphèmes}}}}\kern2pt]} \hypertarget{ts\string_hM\string_M-to\string_MqA\string_T1}{}
\markboth{\textcolor{darkblue}{\textbf{\ipa{tsʰɯ˧-to˧qɑ˥}}}}{}
\textcolor{teal}{\mytextsc{nom}} \hspace{4pt} Ton~: H\#.
\textcolor{PineGreen}{\selectlanguage{french}Chevreau, cabri.} \zh{羔羊、羔子。}  \zh{量词}: \textcolor{darkblue}{\textbf{\ipa{pʰo˧˥}}}  \mytextsc{clf}~: \textcolor{darkblue}{\textbf{\ipa{pʰo˧˥}}} \textit{Voir~:} \hyperlink{}{\textcolor{darkblue}{\textbf{\ipa{tsʰɯ˧zo˥\$}}}} 
\lhead{\firstmark}
\rhead{\botmark}

\subsection{\hspace{-0.5cm} {\Large \textcolor{darkblue}{\textbf{\ipa{tsʰɯ˧zo\#˥}}}}\hspace{0.5cm}[\kern2pt{\textcolor{darkblue}{\textbf{\ipa{tsʰɯ˧zo˧}}}}\kern2pt]} \hypertarget{ts\string_hM\string_Mzo\#\string_T1}{}
\markboth{\textcolor{darkblue}{\textbf{\ipa{tsʰɯ˧zo\#˥}}}}{}
\textcolor{teal}{\mytextsc{nom}} \hspace{4pt} Ton~: \#H.
\textcolor{PineGreen}{\selectlanguage{french}Chevrette.} \zh{母山羊羔。}  \zh{量词}: \textcolor{darkblue}{\textbf{\ipa{ɭɯ˧}}}  \mytextsc{clf}~: \textcolor{darkblue}{\textbf{\ipa{ɭɯ˧}}} 
\lhead{\firstmark}
\rhead{\botmark}

\subsection{\hspace{-0.5cm} {\Large \textcolor{darkblue}{\textbf{\ipa{tsʰɯ˧zo˥\$}}}}\hspace{0.5cm}[\kern2pt{\textcolor{darkblue}{\textbf{\ipa{tsʰɯ˧zo˥}}}}\kern2pt]} \hypertarget{ts\string_hM\string_Mzo\string_T\$1}{}
\markboth{\textcolor{darkblue}{\textbf{\ipa{tsʰɯ˧zo˥\$}}}}{}
\textcolor{teal}{\mytextsc{nom}} \hspace{4pt} Ton~: H\$.
\textcolor{PineGreen}{\selectlanguage{french}Chevreau, cabri.} \zh{山羊羔。}  ¶ \textcolor{darkblue}{\textbf{\ipa{tsʰɯ˧zo˧-to˧qɑ˥}}} \textcolor{PineGreen}{\selectlanguage{french}chevrettes et chevreaux} \zh{母山羊羔与公山羊羔}  
 \zh{量词}: \textcolor{darkblue}{\textbf{\ipa{ɭɯ˧}}}  \mytextsc{clf}~: \textcolor{darkblue}{\textbf{\ipa{ɭɯ˧}}} \textit{Voir~:} \hyperlink{}{\textcolor{darkblue}{\textbf{\ipa{tsʰɯ˧-to˧qɑ˥}}}} 
\lhead{\firstmark}
\rhead{\botmark}

\subsection{\hspace{-0.5cm} {\Large \textcolor{darkblue}{\textbf{\ipa{tsʰɯ˩\textsubscript{a}}}}}\hspace{0.5cm}[\kern2pt{\textcolor{darkblue}{\textbf{\ipa{tsʰɯ˩˥}}}}\kern2pt]} \hypertarget{ts\string_hM\string_Ba1}{}
\markboth{\textcolor{darkblue}{\textbf{\ipa{tsʰɯ˩\textsubscript{a}}}}}{}
\textcolor{teal}{\mytextsc{verbe}} \hspace{4pt} Ton~: L\textsubscript{a}.
\textcolor{PineGreen}{\selectlanguage{french}Venir (\mytextsc{pst}).} \zh{来(过去式)。}  ¶ \textcolor{darkblue}{\textbf{\ipa{le˧-gwɤ˩\textasciitilde{}gwɤ˩ | le˧-tsʰɯ˩-ze˩}}} \textcolor{PineGreen}{\selectlanguage{french}revenir de promenade} \zh{散步回来}  
 ¶ \textcolor{darkblue}{\textbf{\ipa{le˧-tsʰɯ˩-ze˩}}} \textcolor{PineGreen}{\selectlanguage{french}être de retour} \zh{回来了}  
 ¶ \textcolor{darkblue}{\textbf{\ipa{ɖɯ˧-ʝi˧-ɳɯ˧ tsʰɯ˧˥, | ɖɯ˧-ki˧ tʰv̩˧!}}} \textcolor{PineGreen}{\selectlanguage{french}“Venus de différents endroits, nous voici réunis en ce lieu!” L'expression est obscure pour qui ne l'a pas apprise (par exemple pour des locuteurs du bord du Lac): son sens littéral n'est pas particulièrement parlant: “On vient d'un endroit; on arrive à un endroit!”} \zh{“我们都来自不同的地方,但现在在一起了!”}  

\lhead{\firstmark}
\rhead{\botmark}

\subsection{\hspace{-0.5cm} {\Large \textcolor{darkblue}{\textbf{\ipa{tsʰɯ˩tsʰɯ˩ɻ̃˩}}}}\hspace{0.5cm}[\kern2pt{\textcolor{darkblue}{\textbf{\ipa{tsʰɯ˩tsʰɯ˩ɻ̃˩˥}}}}\kern2pt]} \hypertarget{ts\string_hM\string_Bts\string_hM\string_Br£`\string_~\string_B1}{}
\markboth{\textcolor{darkblue}{\textbf{\ipa{tsʰɯ˩tsʰɯ˩ɻ̃˩}}}}{}
\textcolor{teal}{\mytextsc{nom}} \hspace{4pt} Ton~: L.
\textcolor{PineGreen}{\selectlanguage{french}Paille de petits pois.} \zh{豌豆干草。}  ¶ \textcolor{darkblue}{\textbf{\ipa{ʈʂʰɯ˧ | tsʰɯ˩tsʰɯ˩ɻ̃˩ ɲi˥.}}} \textcolor{PineGreen}{\selectlanguage{french}C'est de la paille de haricots.} \zh{这是豌豆干草。}  
 \zh{量词}: \textcolor{darkblue}{\textbf{\ipa{kɤ˧˥}}}  \mytextsc{clf}~: \textcolor{darkblue}{\textbf{\ipa{kɤ˧˥}}} 
\lhead{\firstmark}
\rhead{\botmark}

\subsection{\hspace{-0.5cm} {\Large \textcolor{darkblue}{\textbf{\ipa{tsʰɯ˧˥}}} \textsubscript{1}}\hspace{0.5cm}[\kern2pt{\textcolor{darkblue}{\textbf{\ipa{tsʰɯ˥}}}}\kern2pt]} \hypertarget{ts\string_hM\string_M\string_T1}{}
\markboth{\textcolor{darkblue}{\textbf{\ipa{tsʰɯ˧˥}}} \textsubscript{1}}{}
\textcolor{teal}{\mytextsc{verbe}} \hspace{4pt} Ton~: MH.
\textcolor{PineGreen}{\selectlanguage{french}Taillader (ex.: taillader un vêtement, le découper avec des ciseaux; n'est pas: tailler du tissu pour faire des vêtements).} \zh{剪成片。}  ¶ \textcolor{darkblue}{\textbf{\ipa{tʰɑ˧-tsʰɯ˧˥!}}} \textcolor{PineGreen}{\selectlanguage{french}\mytextsc{prohib}} \zh{\mytextsc{prohib}}  
 ¶ \textcolor{darkblue}{\textbf{\ipa{dʑi˧hṽ˧ tsʰɯ˩}}} \textcolor{PineGreen}{\selectlanguage{french}couper des vêtements en morceaux} \zh{把衣服剪成片}  

\lhead{\firstmark}
\rhead{\botmark}

\subsection{\hspace{-0.5cm} {\Large \textcolor{darkblue}{\textbf{\ipa{tsʰɯ˧˥}}} \textsubscript{2}}\hspace{0.5cm}[\kern2pt{\textcolor{darkblue}{\textbf{\ipa{tsʰɯ˧˥}}}}\kern2pt]} \hypertarget{ts\string_hM\string_M\string_T2}{}
\markboth{\textcolor{darkblue}{\textbf{\ipa{tsʰɯ˧˥}}} \textsubscript{2}}{}
\textcolor{teal}{\mytextsc{nom}} \hspace{4pt} Ton~: MH.
\textcolor{PineGreen}{\selectlanguage{french}Chèvre/bouc.} \zh{山羊。}  \zh{量词}: \textcolor{darkblue}{\textbf{\ipa{pʰo˧˥}}}  \mytextsc{clf}~: \textcolor{darkblue}{\textbf{\ipa{pʰo˧˥}}} 
\lhead{\firstmark}
\rhead{\botmark}

\subsection{\hspace{-0.5cm} {\Large \textcolor{darkblue}{\textbf{\ipa{tsʰv̩˩˥}}}}\hspace{0.5cm}[\kern2pt{\textcolor{darkblue}{\textbf{\ipa{tsʰv̩˩˥}}}}\kern2pt]} \hypertarget{ts\string_hv\string_=\string_B\string_T1}{}
\markboth{\textcolor{darkblue}{\textbf{\ipa{tsʰv̩˩˥}}}}{}
\textcolor{teal}{\mytextsc{nom}} \hspace{4pt} Ton~: LH.
\textcolor{PineGreen}{\selectlanguage{french}Vinaigre.} \zh{醋(汉语借词)。}  Emprunt~: chinois  \zh{醋}
\textit{Voir~:} \textcolor{darkblue}{\textbf{\ipa{sɑ˧tsʰv̩˩, tɕi˧-dʑɯ˩}}} 
\lhead{\firstmark}
\rhead{\botmark}

\newpage
\section*{\centering- \textcolor{darkblue}{\textbf{\ipa{ʈ}}} -}
\subsection{\hspace{-0.5cm} {\Large \textcolor{darkblue}{\textbf{\ipa{ʈæ˧bɤ˧}}}}\hspace{0.5cm}[\kern2pt{\textcolor{darkblue}{\textbf{\ipa{ʈæ˩bɤ˩˥}}}}\kern2pt]} \hypertarget{t`\{\string_Mb7\string_M1}{}
\markboth{\textcolor{darkblue}{\textbf{\ipa{ʈæ˧bɤ˧}}}}{}
\textcolor{teal}{\mytextsc{nom}} \hspace{4pt} Ton~: M.
\textcolor{PineGreen}{\selectlanguage{french}Moine, nonne.} \zh{和尚,尼姑。}  ¶ \textcolor{darkblue}{\textbf{\ipa{ʈæ˧bɤ˧ʈʂʰo˧}}} \textcolor{PineGreen}{\selectlanguage{french}même sens} \zh{同上}  
 ¶ \textcolor{darkblue}{\textbf{\ipa{ʈæ˧bɤ˧ ʝi˧-hĩ˧-hĩ˧}}} \textcolor{PineGreen}{\selectlanguage{french}même sens} \zh{当和尚的人}  
 ¶ \textcolor{darkblue}{\textbf{\ipa{hæ˧ʈæ˩bɤ˩}}} \textcolor{PineGreen}{\selectlanguage{french}moine chinois} \zh{汉人和尚}  
 \zh{量词}: \textcolor{darkblue}{\textbf{\ipa{v̩˧}}}  \mytextsc{clf}~: \textcolor{darkblue}{\textbf{\ipa{v̩˧}}} 
\lhead{\firstmark}
\rhead{\botmark}

\subsection{\hspace{-0.5cm} {\Large \textcolor{darkblue}{\textbf{\ipa{ʈæ˧kwæ˧˥}}}}\hspace{0.5cm}[\kern2pt{\textcolor{darkblue}{\textbf{\ipa{ʈæ˧kwæ˧˥}}}}\kern2pt]} \hypertarget{t`\{\string_Mkw\{\string_M\string_T1}{}
\markboth{\textcolor{darkblue}{\textbf{\ipa{ʈæ˧kwæ˧˥}}}}{}
\textcolor{teal}{\mytextsc{adjectif}} \hspace{4pt} Ton~: MH\#.
\textcolor{PineGreen}{\selectlanguage{french}Prodigue, qui dépense tout.} \zh{爱浪费。}  ¶ \textcolor{darkblue}{\textbf{\ipa{ʈʂʰɯ˧ | ʈæ˧kwæ˧-hĩ˥ | ɖɯ˧-v̩˧ ɲi˩.}}} \textcolor{PineGreen}{\selectlanguage{french}C'est un prodigue/quelqu'un qui dépense tout/qui mène la maison à la ruine.} \zh{他是爱浪费的人。}  

\lhead{\firstmark}
\rhead{\botmark}

\subsection{\hspace{-0.5cm} {\Large \textcolor{darkblue}{\textbf{\ipa{ʈæ˧qo˧}}}}\hspace{0.5cm}[\kern2pt{\textcolor{darkblue}{\textbf{\ipa{ʈæ˧qo˧}}}}\kern2pt]} \hypertarget{t`\{\string_Mqo\string_M1}{}
\markboth{\textcolor{darkblue}{\textbf{\ipa{ʈæ˧qo˧}}}}{}
\textcolor{teal}{\mytextsc{adverbe}} \hspace{4pt} Ton~: M.
\textcolor{PineGreen}{\selectlanguage{french}Au fond de.} \zh{底下。}  ¶ \textcolor{darkblue}{\textbf{\ipa{hi˩nɑ˧mi˧-ʈæ˧qo˥}}} \textcolor{PineGreen}{\selectlanguage{french}au fond du Lac} \zh{在湖底下}  
 ¶ \textcolor{darkblue}{\textbf{\ipa{ʈæ˧qo˧ tɕɯ˧}}} \textcolor{PineGreen}{\selectlanguage{french}mettre au fond de...} \zh{放在底下}  
 ¶ \textcolor{darkblue}{\textbf{\ipa{hi˩nɑ˧mi˧, | ʈæ˧ mɤ˧-do˩; | hĩ˧-nv̩˥mi˩, | ɳv̩˧ mɤ˧-tʰɑ˩.}}} \textcolor{PineGreen}{\selectlanguage{french}“On ne voit pas le fond du lac; on ne connaît pas le cœur des hommes!” (Proverbe qui apparaît dans les chansons que se chantaient les jeunes gens se faisant la cour.)} \zh{“人的心,湖底藏:看不清,摸不透!” 直译:“湖,(我们)看不到(它的)底下。人的心,是知道不了的!”(情歌里的一个谚语)}  

\lhead{\firstmark}
\rhead{\botmark}

\subsection{\hspace{-0.5cm} {\Large \textcolor{darkblue}{\textbf{\ipa{ʈæ˧ʂɯ˧}}}}\hspace{0.5cm}[\kern2pt{\textcolor{darkblue}{\textbf{\ipa{ʈæ˧ʂɯ˧}}}}\kern2pt]} \hypertarget{t`\{\string_Ms`M\string_M1}{}
\markboth{\textcolor{darkblue}{\textbf{\ipa{ʈæ˧ʂɯ˧}}}}{}
\textcolor{teal}{\mytextsc{nom}} \hspace{4pt} Ton~: M.
\textcolor{PineGreen}{\selectlanguage{french}Prénom masculin.} \zh{男性名字。} 
\lhead{\firstmark}
\rhead{\botmark}

\subsection{\hspace{-0.5cm} {\Large \textcolor{darkblue}{\textbf{\ipa{ʈæ˩\textsubscript{a}}}}}\hspace{0.5cm}[\kern2pt{\textcolor{darkblue}{\textbf{\ipa{ʈæ˧˥}}}}\kern2pt]} \hypertarget{t`\{\string_Ba1}{}
\markboth{\textcolor{darkblue}{\textbf{\ipa{ʈæ˩\textsubscript{a}}}}}{}
\textcolor{teal}{\mytextsc{verbe}} \hspace{4pt} Ton~: L\textsubscript{a}.
\ding{202} \textcolor{PineGreen}{\selectlanguage{french}Fermer, refermer; enfermer (ex.: des moutons); aussi: fermer une route.} \zh{关(门、羊)。}  ¶ \textcolor{darkblue}{\textbf{\ipa{bv̩˩qo˩ ʈæ˥}}} \textcolor{PineGreen}{\selectlanguage{french}fermer l'étable} \zh{关牛圈}  
 ¶ \textcolor{darkblue}{\textbf{\ipa{tʰi˧-ʈæ˩}}} \textcolor{PineGreen}{\selectlanguage{french}\mytextsc{dur}: fermer} \zh{关门}  
 ¶ \textcolor{darkblue}{\textbf{\ipa{kʰi˧ ʈæ˥}}} \textcolor{PineGreen}{\selectlanguage{french}fermer la porte} \zh{关门}  
\ding{203} \textcolor{PineGreen}{\selectlanguage{french}Nouer (un noeud).} \zh{扣(扣子)、系、结。} 
\lhead{\firstmark}
\rhead{\botmark}

\subsection{\hspace{-0.5cm} {\Large \textcolor{darkblue}{\textbf{\ipa{ʈæ˩ɖɯ˧}}}}\hspace{0.5cm}[\kern2pt{\textcolor{darkblue}{\textbf{\ipa{ʈæ˧ɖɯ˧}}}}\kern2pt]} \hypertarget{t`\{\string_Bd`M\string_M1}{}
\markboth{\textcolor{darkblue}{\textbf{\ipa{ʈæ˩ɖɯ˧}}}}{}
\textcolor{teal}{\mytextsc{adjectif}} \hspace{4pt} Ton~: LM.
\textcolor{PineGreen}{\selectlanguage{french}Satisfait, tranquille.} \zh{安乐。}  ¶ \textcolor{darkblue}{\textbf{\ipa{mɤ˧-ʈæ˩ɖɯ˩}}} \textcolor{PineGreen}{\selectlanguage{french}mécontent, furieux} \zh{不高兴、不安}  
 ¶ \textcolor{darkblue}{\textbf{\ipa{ə˧mɑ˧ | tsʰi˧-ɲi˧ | ʈæ˩ɖɯ˧ tʰi˧-dzi˩-dʑo˩!}}} \textcolor{PineGreen}{\selectlanguage{french}Aujourd'hui, Ama est assise bien tranquille!} \zh{今天,阿妈安乐地坐着。}  
 ¶ \textcolor{darkblue}{\textbf{\ipa{ʈʂʰɯ˧-ɳɯ˧ | njɤ˧-ki˧ | mɤ˧-ʈæ˩ɖɯ˩-hĩ˩ ʐwɤ˩!}}} \textcolor{PineGreen}{\selectlanguage{french}il m'a dit des choses qui fâchent! (=il m'a vexé, il m'a dit des choses désobligeantes)} \zh{他跟我说了一些让我不安的(事情)! / 他跟我说的,让我生气!}  

\lhead{\firstmark}
\rhead{\botmark}

\subsection{\hspace{-0.5cm} {\Large \textcolor{darkblue}{\textbf{\ipa{ʈæ˩tsʰo\#˥}}} \textsubscript{1}}\hspace{0.5cm}[\kern2pt{\textcolor{darkblue}{\textbf{\ipa{ʈæ˩tsʰo˥}}}}\kern2pt]} \hypertarget{t`\{\string_Bts\string_ho\#\string_T1}{}
\markboth{\textcolor{darkblue}{\textbf{\ipa{ʈæ˩tsʰo\#˥}}} \textsubscript{1}}{}
\textcolor{teal}{\mytextsc{nom}} \hspace{4pt} Ton~: LM+\#H.
\textcolor{PineGreen}{\selectlanguage{french}Classe, groupe, ensemble (de prêtres).} \zh{班、小组。}  ¶ \textcolor{darkblue}{\textbf{\ipa{ʈæ˩tsʰo˧ | ɖɯ˧-ɭɯ˧}}} \textcolor{PineGreen}{\selectlanguage{french}une classe, un groupe (de prêtres)} \zh{一个小组、一帮(和尚)}  
 \zh{量词}: \textcolor{darkblue}{\textbf{\ipa{ɭɯ˧}}}  \mytextsc{clf}~: \textcolor{darkblue}{\textbf{\ipa{ɭɯ˧}}} \textit{Voir~:} \hyperlink{}{\textcolor{darkblue}{\textbf{\ipa{ʈæ˩tsʰo\#˥}}} \textsubscript{2}} 
\lhead{\firstmark}
\rhead{\botmark}

\subsection{\hspace{-0.5cm} {\Large \textcolor{darkblue}{\textbf{\ipa{ʈæ˩tsʰo\#˥}}} \textsubscript{2}}\hspace{0.5cm}[\kern2pt{\textcolor{darkblue}{\textbf{\ipa{ʈæ˩tsʰo˩˥}}}}\kern2pt]} \hypertarget{t`\{\string_Bts\string_ho\#\string_T2}{}
\markboth{\textcolor{darkblue}{\textbf{\ipa{ʈæ˩tsʰo\#˥}}} \textsubscript{2}}{}
\textcolor{teal}{\mytextsc{classificateur}} \hspace{4pt} Ton~: L.
\textcolor{PineGreen}{\selectlanguage{french}Auto-classificateur des classes/groupes (de prêtres).} \zh{量词:和尚(一帮、一班)。}  ¶ \textcolor{darkblue}{\textbf{\ipa{ɖɯ˧-ʈæ˩tsʰo˩}}} \textcolor{PineGreen}{\selectlanguage{french}un groupe (de prêtres)} \zh{一班(和尚)}  
\textit{Voir~:} \hyperlink{}{\textcolor{darkblue}{\textbf{\ipa{ʈæ˩tsʰo\#˥}}} \textsubscript{1}} 
\lhead{\firstmark}
\rhead{\botmark}

\subsection{\hspace{-0.5cm} {\Large \textcolor{darkblue}{\textbf{\ipa{ʈæ˩ʈv̩\#˥}}}}\hspace{0.5cm}[\kern2pt{\textcolor{darkblue}{\textbf{\ipa{ʈæ˩ʈv̩˥}}}}\kern2pt]} \hypertarget{t`\{\string_Bt`v\string_=\#\string_T1}{}
\markboth{\textcolor{darkblue}{\textbf{\ipa{ʈæ˩ʈv̩\#˥}}}}{}
\textcolor{teal}{\mytextsc{nom}} \hspace{4pt} Ton~: LM+\#H.
\textcolor{PineGreen}{\selectlanguage{french}Prénom masculin.} \zh{男性名字。} 
\lhead{\firstmark}
\rhead{\botmark}

\subsection{\hspace{-0.5cm} {\Large \textcolor{darkblue}{\textbf{\ipa{ʈɤ˧\textsubscript{a}}}}}\hspace{0.5cm}[\kern2pt{\textcolor{darkblue}{\textbf{\ipa{ʈɤ˥}}}}\kern2pt]} \hypertarget{t`7\string_Ma1}{}
\markboth{\textcolor{darkblue}{\textbf{\ipa{ʈɤ˧\textsubscript{a}}}}}{}
\textcolor{teal}{\mytextsc{verbe}} \hspace{4pt} Ton~: M\textsubscript{a}.
\textcolor{PineGreen}{\selectlanguage{french}Tirer.} \zh{拉、拽。}  ¶ \textcolor{darkblue}{\textbf{\ipa{tso˧\textasciitilde{}tso˧ ʈɤ˩(-ze˩)}}} \textcolor{PineGreen}{\selectlanguage{french}tirer quelque chose} \zh{拉拽东西}  
 ¶ \textcolor{darkblue}{\textbf{\ipa{mv̩˧ʐe˧ qʰæ˩ | le˧-wo˧-ʈɤ˥-di˩}}} \textcolor{PineGreen}{\selectlanguage{french}périphrase pour désigner la gâchette d'un pistolet: ce qu'on tire vers soi pour faire feu} \zh{扳机}  

\lhead{\firstmark}
\rhead{\botmark}

\subsection{\hspace{-0.5cm} {\Large \textcolor{darkblue}{\textbf{\ipa{ʈi˥\textsubscript{a}}}}}\hspace{0.5cm}[\kern2pt{\textcolor{darkblue}{\textbf{\ipa{ʈi˩˥}}}}\kern2pt]} \hypertarget{t`i\string_Ta1}{}
\markboth{\textcolor{darkblue}{\textbf{\ipa{ʈi˥\textsubscript{a}}}}}{}
\textcolor{teal}{\mytextsc{classificateur}} \hspace{4pt} Ton~: H\textsubscript{a}.
\textcolor{PineGreen}{\selectlanguage{french}Empan: distance entre le pouce et l'index écartés. D'ordinaire, on n'emploie pas la distance entre pouce et majeur.} \zh{量词:拃(大拇指和食指之间的距离。一般不用大拇指和中指之间的距离。)。} 
\lhead{\firstmark}
\rhead{\botmark}

\subsection{\hspace{-0.5cm} {\Large \textcolor{darkblue}{\textbf{\ipa{ʈi˩\textsubscript{a}}}}}\hspace{0.5cm}[\kern2pt{\textcolor{darkblue}{\textbf{\ipa{ʈi˩˥}}}}\kern2pt]} \hypertarget{t`i\string_Ba1}{}
\markboth{\textcolor{darkblue}{\textbf{\ipa{ʈi˩\textsubscript{a}}}}}{}
\textcolor{teal}{\mytextsc{verbe}} \hspace{4pt} Ton~: L\textsubscript{a}.
\textcolor{PineGreen}{\selectlanguage{french}Se lever.} \zh{起(如:起来,起床)。}  ¶ \textcolor{darkblue}{\textbf{\ipa{gɤ˩-ʈi˧}}} \textcolor{PineGreen}{\selectlanguage{french}se lever} \zh{起来}  
 ¶ \textcolor{darkblue}{\textbf{\ipa{ʑi˧ ʈi˥}}} \textcolor{PineGreen}{\selectlanguage{french}se réveiller} \zh{醒来}  
 ¶ \textcolor{darkblue}{\textbf{\ipa{ʑi˧ gɤ˧-ʈi˩}}} \textcolor{PineGreen}{\selectlanguage{french}se réveiller} \zh{醒来}  
 ¶ \textcolor{darkblue}{\textbf{\ipa{gɤ˩ mɤ˥-ʈi˩}}} \textcolor{PineGreen}{\selectlanguage{french}ne pas se lever} \zh{不起床}  
 ¶ \textcolor{darkblue}{\textbf{\ipa{mɤ˧-ʈi˩-sɯ˩!}}} \textcolor{PineGreen}{\selectlanguage{french}(Il/elle) n'est pas encore levé(e)!} \zh{还没起床!}  
 ¶ \textcolor{darkblue}{\textbf{\ipa{le˧-ʈi˩-ze˩!}}} \textcolor{PineGreen}{\selectlanguage{french}(il/elle) s'est levé(e)!} \zh{起床了!}  
 ¶ \textcolor{darkblue}{\textbf{\ipa{ɖɯ˧-ʈi˧\textasciitilde{}ʈi˥-ɻ̍˩}}} \textcolor{PineGreen}{\selectlanguage{french}\mytextsc{délimitatif} \mytextsc{red} \mytextsc{inchoatif}} \zh{起来一下}  

\lhead{\firstmark}
\rhead{\botmark}

\subsection{\hspace{-0.5cm} {\Large \textcolor{darkblue}{\textbf{\ipa{ʈɯ˧\textsubscript{a}}}}}\hspace{0.5cm}[\kern2pt{\textcolor{darkblue}{\textbf{\ipa{ʈɯ˧˥}}}}\kern2pt]} \hypertarget{t`M\string_Ma1}{}
\markboth{\textcolor{darkblue}{\textbf{\ipa{ʈɯ˧\textsubscript{a}}}}}{}
\textcolor{teal}{\mytextsc{verbe}} \hspace{4pt} Ton~: M\textsubscript{a}.
\textcolor{PineGreen}{\selectlanguage{french}Mettre en place, installer à sa juste place.} \zh{安装、摆好。}  ¶ \textcolor{darkblue}{\textbf{\ipa{ʂe˧kʰɯ˧ tʰi˧-ʈɯ˧˥, | v̩˧ | tʰi˧-ʈɯ˧}}} \textcolor{PineGreen}{\selectlanguage{french}mettre en place le trépied de fer dans le foyer, mettre en place la grande casserole (sur le trépied) (Contexte: description de la “pendaison de crémaillère”, dans une nouvelle maison)} \zh{(建完新房后)安装三脚架、把锅摆好(在三脚架上)}  
 ¶ \textcolor{darkblue}{\textbf{\ipa{tsʰo˩-ɻ̃˩˥ | dʑɯ˩ mɤ˩-ʈɯ˩˥, | lɑ˧-ʂe˧ | kʰv̩˧ tʰɑ˩-ki˩!}}} \textcolor{PineGreen}{\selectlanguage{french}“Les ossements humains, on ne les met pas à l'eau! La chair du tigre, on ne la donne pas au chien!” Sens: on n'enterre pas les gens dans l'eau (à la différence de certaines coutumes tibétaines); on prenait soin de n'immerger dans l'eau, ni le corps, ni les cendres après la crémation.} \zh{“人骨头,莫碰水!老虎肉,莫给狗!”(这个谚语,来强调摩梭与藏族的一些不同习惯:摩梭禁止让尸体或骨灰沾水。)}  

\lhead{\firstmark}
\rhead{\botmark}

\subsection{\hspace{-0.5cm} {\Large \textcolor{darkblue}{\textbf{\ipa{ʈɯ˧ʈʰæ\#˥}}}}\hspace{0.5cm}[\kern2pt{\textcolor{darkblue}{\textbf{\ipa{ʈɯ˩ʈʰæ˩˥}}}}\kern2pt]} \hypertarget{t`M\string_Mt`\string_h\{\#\string_T1}{}
\markboth{\textcolor{darkblue}{\textbf{\ipa{ʈɯ˧ʈʰæ\#˥}}}}{}
\textcolor{teal}{\mytextsc{nom}} \hspace{4pt} Ton~: \#H.
\textcolor{PineGreen}{\selectlanguage{french}Patrimoine.} \zh{家底、财产(贵重物品)。}  ¶ \textcolor{darkblue}{\textbf{\ipa{ɑ˩ʁo˧ ʈɯ˧ʈʰæ˧!}}} \textcolor{PineGreen}{\selectlanguage{french}Prospérité à la famille!} \zh{祝你们家发财!}  
 ¶ \textcolor{darkblue}{\textbf{\ipa{ʈʂʰɯ˧ | ʈɯ˧ʈʰæ˧ | ɖwæ˧˥ | dʑo˧-ʝi˧!}}} \textcolor{PineGreen}{\selectlanguage{french}Il/elle est riche! / Sa famille est riche!} \zh{他家底很好! / 他家有钱!}  
 \zh{量词}: \textcolor{darkblue}{\textbf{\ipa{kʰwɤ˥}}}  \mytextsc{clf}~: \textcolor{darkblue}{\textbf{\ipa{kʰwɤ˥}}} 
\lhead{\firstmark}
\rhead{\botmark}

\subsection{\hspace{-0.5cm} {\Large \textcolor{darkblue}{\textbf{\ipa{ʈɯ˧˥}}}}\hspace{0.5cm}[\kern2pt{\textcolor{darkblue}{\textbf{\ipa{ʈɯ˥}}}}\kern2pt]} \hypertarget{t`M\string_M\string_T1}{}
\markboth{\textcolor{darkblue}{\textbf{\ipa{ʈɯ˧˥}}}}{}
\textcolor{teal}{\mytextsc{verbe}} \hspace{4pt} Ton~: MH.
\textcolor{PineGreen}{\selectlanguage{french}Blanchir à l'eau bouillante: du lin pour préparer du fil pour le tissage, des légumes séchés avant de les utiliser pour la cuisine...} \zh{以滚水将蔬菜或亚麻灼过。}  ¶ \textcolor{darkblue}{\textbf{\ipa{tʰi˧-ʈɯ˧˥}}} \textcolor{PineGreen}{\selectlanguage{french}\mytextsc{dur}} \zh{\mytextsc{dur}}  
 ¶ \textcolor{darkblue}{\textbf{\ipa{dʑɯ˩tsʰi˩-qo˥ | tʰi˧-ʈɯ˧˥ / dʑɯ˩tsʰi˩-qo˥ | ʈɯ˧˥}}} \textcolor{PineGreen}{\selectlanguage{french}blanchir à l'eau bouillante} \zh{以滚水灼过}  
 ¶ \textcolor{darkblue}{\textbf{\ipa{dʑɯ˧-qo˧ | ʈɯ˧˥}}} \textcolor{PineGreen}{\selectlanguage{french}blanchir à l'eau} \zh{以水灼过}  
 ¶ \textcolor{darkblue}{\textbf{\ipa{v˩tsʰɤ˧ ʈɯ˥}}} \textcolor{PineGreen}{\selectlanguage{french}blanchir des légumes} \zh{灼蔬菜}  
 ¶ \textcolor{darkblue}{\textbf{\ipa{sɑ˧, | ʈɯ˧-kv˥!}}} \textcolor{PineGreen}{\selectlanguage{french}Le chanvre, ça se blanchit! (Au cours de la préparation du chanvre pour en faire du fil, il faut le blanchir.)} \zh{亚麻,要灼过!}  

\lhead{\firstmark}
\rhead{\botmark}

\subsection{\hspace{-0.5cm} {\Large \textcolor{darkblue}{\textbf{\ipa{ʈv̩˩}}}}\hspace{0.5cm}[\kern2pt{\textcolor{darkblue}{\textbf{\ipa{ʈv̩˥}}}}\kern2pt]} \hypertarget{t`v\string_=\string_B1}{}
\markboth{\textcolor{darkblue}{\textbf{\ipa{ʈv̩˩}}}}{}
\textcolor{teal}{\mytextsc{nom}} \hspace{4pt} Ton~: L.
\textcolor{PineGreen}{\selectlanguage{french}Noeud.} \zh{死扣、死结。}  ¶ \textcolor{darkblue}{\textbf{\ipa{ɖɯ˧-ʈv̩˩}}} \textcolor{PineGreen}{\selectlanguage{french}un noeud} \zh{一个死结}  
 ¶ \textcolor{darkblue}{\textbf{\ipa{ɖɯ˧-ʈv̩˩ | tʰi˧-ʈv̩˩}}} \textcolor{PineGreen}{\selectlanguage{french}faire un noeud} \zh{打一个死结}  

\lhead{\firstmark}
\rhead{\botmark}

\subsection{\hspace{-0.5cm} {\Large \textcolor{darkblue}{\textbf{\ipa{ʈv̩˩\textsubscript{a}}}} \textsubscript{1}}\hspace{0.5cm}[\kern2pt{\textcolor{darkblue}{\textbf{\ipa{ʈv̩˧˥}}}}\kern2pt]} \hypertarget{t`v\string_=\string_Ba1}{}
\markboth{\textcolor{darkblue}{\textbf{\ipa{ʈv̩˩\textsubscript{a}}}} \textsubscript{1}}{}
\textcolor{teal}{\mytextsc{verbe}} \hspace{4pt} Ton~: L\textsubscript{a}.
\textcolor{PineGreen}{\selectlanguage{french}Tresser (vannerie).} \zh{编(竹子)。}  ¶ \textcolor{darkblue}{\textbf{\ipa{qʰwɤ˧tʰv̩˧ ʈv̩˥}}} \textcolor{PineGreen}{\selectlanguage{french}tresser une hotte dorsale (en bambou) pour porter l'eau} \zh{编背水的背篓}  
 ¶ \textcolor{darkblue}{\textbf{\ipa{mi˩ɬi˩ ʈv̩˥}}} \textcolor{PineGreen}{\selectlanguage{french}tresser du bambou} \zh{编竹子}  
 ¶ \textcolor{darkblue}{\textbf{\ipa{tso˧\textasciitilde{}tso˧ ʈv̩˥}}} \textcolor{PineGreen}{\selectlanguage{french}tresser des choses} \zh{编东西}  

\lhead{\firstmark}
\rhead{\botmark}

\subsection{\hspace{-0.5cm} {\Large \textcolor{darkblue}{\textbf{\ipa{ʈv̩˩\textsubscript{a}}}} \textsubscript{2}}\hspace{0.5cm}[\kern2pt{\textcolor{darkblue}{\textbf{\ipa{ʈv̩˩˥}}}}\kern2pt]} \hypertarget{t`v\string_=\string_Ba2}{}
\markboth{\textcolor{darkblue}{\textbf{\ipa{ʈv̩˩\textsubscript{a}}}} \textsubscript{2}}{}
\textcolor{teal}{\mytextsc{verbe}} \hspace{4pt} Ton~: L\textsubscript{a}.
\textcolor{PineGreen}{\selectlanguage{french}Lancer (une pierre sur quelqu'un).} \zh{掷(掷石头)。}  ¶ \textcolor{darkblue}{\textbf{\ipa{mɤ˧-ʈv̩˩}}} \textcolor{PineGreen}{\selectlanguage{french}\mytextsc{neg}} \zh{\mytextsc{neg}}  
 ¶ \textcolor{darkblue}{\textbf{\ipa{lv̩˧mi˧ ʈv̩˩}}} \textcolor{PineGreen}{\selectlanguage{french}lancer une pierre} \zh{掷石头}  
 ¶ \textcolor{darkblue}{\textbf{\ipa{tso˧\textasciitilde{}tso˧ ʈv̩˥}}} \textcolor{PineGreen}{\selectlanguage{french}jeter des choses} \zh{掷东西}  

\lhead{\firstmark}
\rhead{\botmark}

\subsection{\hspace{-0.5cm} {\Large \textcolor{darkblue}{\textbf{\ipa{ʈv̩˩\textsubscript{b}}}}}\hspace{0.5cm}[\kern2pt{\textcolor{darkblue}{\textbf{\ipa{ʈv̩˥}}}}\kern2pt]} \hypertarget{t`v\string_=\string_Bb1}{}
\markboth{\textcolor{darkblue}{\textbf{\ipa{ʈv̩˩\textsubscript{b}}}}}{}
\textcolor{teal}{\mytextsc{classificateur}} \hspace{4pt} Ton~: L\textsubscript{b}.
\textcolor{PineGreen}{\selectlanguage{french}Classificateur des quartiers/pièces: morceaux de taille supérieure à une bouchée. Il peut s'agir d'un morceau de viande qu'on donne à un convive, et qui se mange en plusieurs bouchées, mais aussi d'un gros quartier de viande (plusieurs kilos).} \zh{量词:大块,如:一块肉,从一个人的份到几公斤的重量。} 
\lhead{\firstmark}
\rhead{\botmark}

\subsection{\hspace{-0.5cm} {\Large \textcolor{darkblue}{\textbf{\ipa{ʈv̩˩qʰv̩˩}}}}\hspace{0.5cm}[\kern2pt{\textcolor{darkblue}{\textbf{\ipa{ʈv̩˩qʰv̩˩˥}}}}\kern2pt]} \hypertarget{t`v\string_=\string_Bq\string_hv\string_=\string_B1}{}
\markboth{\textcolor{darkblue}{\textbf{\ipa{ʈv̩˩qʰv̩˩}}}}{}
\textcolor{teal}{\mytextsc{nom}} \hspace{4pt} Ton~: L.
\textcolor{PineGreen}{\selectlanguage{french}Noeud coulant.} \zh{活扣。}  ¶ \textcolor{darkblue}{\textbf{\ipa{ʈv̩˩qʰv̩˩˥ | tʰi˧-ʈv̩˩}}} \textcolor{PineGreen}{\selectlanguage{french}faire un noeud coulant} \zh{打活扣}  
 ¶ \textcolor{darkblue}{\textbf{\ipa{ʈv̩˩qʰv̩˩˥ | ɖɯ˧-ɭɯ˧ | tʰi˧-ʈv̩˩}}} \textcolor{PineGreen}{\selectlanguage{french}faire un noeud coulant} \zh{打一个活扣}  

\lhead{\firstmark}
\rhead{\botmark}

\subsection{\hspace{-0.5cm} {\Large \textcolor{darkblue}{\textbf{\ipa{ʈwæ˩\textsubscript{a}}}}}\hspace{0.5cm}[\kern2pt{\textcolor{darkblue}{\textbf{\ipa{ʈwæ˧˥}}}}\kern2pt]} \hypertarget{t`w\{\string_Ba1}{}
\markboth{\textcolor{darkblue}{\textbf{\ipa{ʈwæ˩\textsubscript{a}}}}}{}
\textcolor{teal}{\mytextsc{verbe}} \hspace{4pt} Ton~: L\textsubscript{a}.
\textcolor{PineGreen}{\selectlanguage{french}Geler, se figer.} \zh{冻。}  ¶ \textcolor{darkblue}{\textbf{\ipa{dʑɯ˩ ʈwæ˩˥}}} \textcolor{PineGreen}{\selectlanguage{french}l'eau gèle, il gèle} \zh{水冻成冰}  
 ¶ \textcolor{darkblue}{\textbf{\ipa{dʑɯ˩pʰæ˩ ʈwæ˧-ze˩}}} \textcolor{PineGreen}{\selectlanguage{french}l'eau a gelé, de la glace s'est formée} \zh{水冻成冰了。}  
 ¶ \textcolor{darkblue}{\textbf{\ipa{ɖɯ˧-ʈwæ˧\textasciitilde{}ʈwæ˥-ɻ̍˩ kʰɯ˩}}} \textcolor{PineGreen}{\selectlanguage{french}faire geler, mettre à congeler} \zh{冷冻,放在冷箱}  

\lhead{\firstmark}
\rhead{\botmark}

\subsection{\hspace{-0.5cm} {\Large \textcolor{darkblue}{\textbf{\ipa{ʈwæ˧˥}}}}\hspace{0.5cm}[\kern2pt{\textcolor{darkblue}{\textbf{\ipa{ʈwæ˥}}}}\kern2pt]} \hypertarget{t`w\{\string_M\string_T1}{}
\markboth{\textcolor{darkblue}{\textbf{\ipa{ʈwæ˧˥}}}}{}
\textcolor{teal}{\mytextsc{verbe}} \hspace{4pt} Ton~: MH.
\textcolor{PineGreen}{\selectlanguage{french}Tomber (en glissant).} \zh{跌倒(路很滑)。} 
\lhead{\firstmark}
\rhead{\botmark}

\subsection{\hspace{-0.5cm} {\Large \textcolor{darkblue}{\textbf{\ipa{ʈwɤ˧\textsubscript{a}}}}}\hspace{0.5cm}[\kern2pt{\textcolor{darkblue}{\textbf{\ipa{ʈwɤ˩˥}}}}\kern2pt]} \hypertarget{t`w7\string_Ma1}{}
\markboth{\textcolor{darkblue}{\textbf{\ipa{ʈwɤ˧\textsubscript{a}}}}}{}
\textcolor{teal}{\mytextsc{verbe}} \hspace{4pt} Ton~: M\textsubscript{a}.
\textcolor{PineGreen}{\selectlanguage{french}Le coq chante/fait cocorico; un oiseau chante.} \zh{啼,鸡叫。}  ¶ \textcolor{darkblue}{\textbf{\ipa{æ̃˩ ʈwɤ˧ (+ze˧)}}} \textcolor{PineGreen}{\selectlanguage{french}Le coq chante/fait cocorico.} \zh{鸡叫。}  

\lhead{\firstmark}
\rhead{\botmark}

\newpage
\section*{\centering- \textcolor{darkblue}{\textbf{\ipa{ʈʰ}}} -}
\subsection{\hspace{-0.5cm} {\Large \textcolor{darkblue}{\textbf{\ipa{ʈʰæ˧-mɤ˧-ʝi\#˥}}}}\hspace{0.5cm}[\kern2pt{\textcolor{darkblue}{\textbf{\ipa{xxxx non-correspondance entre le nombre de morphèmes et le nombre de tons de morphèmes}}}}\kern2pt]} \hypertarget{t`\string_h\{\string_M-m7\string_M-j££i\#\string_T1}{}
\markboth{\textcolor{darkblue}{\textbf{\ipa{ʈʰæ˧-mɤ˧-ʝi\#˥}}}}{}
\textcolor{teal}{\mytextsc{adjectif}} \hspace{4pt} Ton~: \#H.
\textcolor{PineGreen}{\selectlanguage{french}Désordonné.} \zh{乱。}  ¶ \textcolor{darkblue}{\textbf{\ipa{ɑ˩ʁo˧ | ʈʰæ˧-mɤ˧-ʝi˧! |}}} \textcolor{PineGreen}{\selectlanguage{french}la maison est en grand désordre!} \zh{家很乱!}  
 ¶ \textcolor{darkblue}{\textbf{\ipa{ʈʰæ˧-mɤ˧-ʝi˧ ɲi˥! |}}} \textcolor{PineGreen}{\selectlanguage{french}C'est vraiment le désordre!} \zh{真乱!}  

\lhead{\firstmark}
\rhead{\botmark}

\subsection{\hspace{-0.5cm} {\Large \textcolor{darkblue}{\textbf{\ipa{ʈʰæ˧mi˧-ɳɯ˩}}}}\hspace{0.5cm}[\kern2pt{\textcolor{darkblue}{\textbf{\ipa{xxxx non-correspondance entre le nombre de morphèmes et le nombre de tons de morphèmes}}}}\kern2pt]} \hypertarget{t`\string_h\{\string_Mmi\string_M-n`M\string_B1}{}
\markboth{\textcolor{darkblue}{\textbf{\ipa{ʈʰæ˧mi˧-ɳɯ˩}}}}{}
\textcolor{teal}{\mytextsc{adverbe}} \hspace{4pt} Ton~: L\#.
\textcolor{PineGreen}{\selectlanguage{french}Vraiment, réellement.} \zh{真的。}  ¶ \textcolor{darkblue}{\textbf{\ipa{ʈʂʰɯ˧ | ʈʰæ˧mi˧-ɳɯ˩ | go˩˥!}}} \textcolor{PineGreen}{\selectlanguage{french}Il/elle est vraiment malade!} \zh{他真的病了!}  

\lhead{\firstmark}
\rhead{\botmark}

\subsection{\hspace{-0.5cm} {\Large \textcolor{darkblue}{\textbf{\ipa{-ʈʰæ˧qo˩}}}}\hspace{0.5cm}[\kern2pt{\textcolor{darkblue}{\textbf{\ipa{ʈʰæ˧qo˩}}}}\kern2pt]} \hypertarget{-t`\string_h\{\string_Mqo\string_B1}{}
\markboth{\textcolor{darkblue}{\textbf{\ipa{-ʈʰæ˧qo˩}}}}{}
\textcolor{teal}{\mytextsc{postposition}} \hspace{4pt} Ton~: L\#.
\textcolor{PineGreen}{\selectlanguage{french}Sous (sous le ciel; sous la tente); au pied (d'une montagne).} \zh{……之下、下面。} 
\lhead{\firstmark}
\rhead{\botmark}

\subsection{\hspace{-0.5cm} {\Large \textcolor{darkblue}{\textbf{\ipa{ʈʰæ˧qʰwɤ˧}}}}\hspace{0.5cm}[\kern2pt{\textcolor{darkblue}{\textbf{\ipa{ʈʰæ˧qʰwɤ˧}}}}\kern2pt]} \hypertarget{t`\string_h\{\string_Mq\string_hw7\string_M1}{}
\markboth{\textcolor{darkblue}{\textbf{\ipa{ʈʰæ˧qʰwɤ˧}}}}{}
\textcolor{teal}{\mytextsc{nom}} \hspace{4pt} Ton~: M.
\textcolor{PineGreen}{\selectlanguage{french}Jupe.} \zh{裙子。}  \zh{量词}: \textcolor{darkblue}{\textbf{\ipa{ɭɯ˧˥}}}  \mytextsc{clf}~: \textcolor{darkblue}{\textbf{\ipa{ɭɯ˧˥}}} 
\lhead{\firstmark}
\rhead{\botmark}

\subsection{\hspace{-0.5cm} {\Large \textcolor{darkblue}{\textbf{\ipa{*ʈʰæ˩}}}}\hspace{0.5cm}[\kern2pt{\textcolor{darkblue}{\textbf{\ipa{ʈʰæ˥}}}}\kern2pt]} \hypertarget{*t`\string_h\{\string_B1}{}
\markboth{\textcolor{darkblue}{\textbf{\ipa{*ʈʰæ˩}}}}{}
\textcolor{teal}{\mytextsc{nom}} \hspace{4pt} Ton~: L.
\textcolor{PineGreen}{\selectlanguage{french}Jupe; monosyllabe extrait d'après le comportement dans l'expression figée /ʈʰæ˩ ki˩˥/ 'enfiler la jupe', avec verbe au ton La (nom du rituel de passage à l'âge adulte). Le monosyllabe n'est pas usité hors de cette expression. Par exemple, */ʈʰæ˩ ɲi˩˥/ 'c'est une jupe' est catégoriquement refusé par F4.} \zh{裙子(单音节)。}  ¶ \textcolor{darkblue}{\textbf{\ipa{ʈʰæ˧ | le˧-ki˩}}} \textcolor{PineGreen}{\selectlanguage{french}enfiler la jupe (\mytextsc{accomp})} \zh{穿上裙子}  
 ¶ \textcolor{darkblue}{\textbf{\ipa{ʈʰæ˩ ki˩˥}}} \textcolor{PineGreen}{\selectlanguage{french}rituel de passage à l'âge adulte, pour les jeunes femmes: “enfiler la jupe”} \zh{女人成年的礼仪:“穿裙子”}  

\lhead{\firstmark}
\rhead{\botmark}

\subsection{\hspace{-0.5cm} {\Large \textcolor{darkblue}{\textbf{\ipa{ʈʰæ˩ki˩}}}}\hspace{0.5cm}[\kern2pt{\textcolor{darkblue}{\textbf{\ipa{ʈʰæ˩ki˩˥}}}}\kern2pt]} \hypertarget{t`\string_h\{\string_Bki\string_B1}{}
\markboth{\textcolor{darkblue}{\textbf{\ipa{ʈʰæ˩ki˩}}}}{}
\textcolor{teal}{\mytextsc{verbe}} \hspace{4pt} Ton~: L.
\textcolor{PineGreen}{\selectlanguage{french}Réaliser la cérémonie de passage à l'âge adulte des femmes.} \zh{举行女孩的成年礼。}  ¶ \textcolor{darkblue}{\textbf{\ipa{ʈʰæ˩ki˩-ze˥!}}} \textcolor{PineGreen}{\selectlanguage{french}Elle est adulte maintenant! / La cérémonie de passage à l'âge adulte a été réalisée!} \zh{穿裙了! / 行过穿裙礼了! / 她成年了!}  

\lhead{\firstmark}
\rhead{\botmark}

\subsection{\hspace{-0.5cm} {\Large \textcolor{darkblue}{\textbf{\ipa{ʈʰæ˩\textasciitilde{}ʈʰæ˧˥}}}}\hspace{0.5cm}[\kern2pt{\textcolor{darkblue}{\textbf{\ipa{ʈʰæ˧ʈʰæ˧˥}}}}\kern2pt]} \hypertarget{t`\string_h\{\string_B~t`\string_h\{\string_M\string_T1}{}
\markboth{\textcolor{darkblue}{\textbf{\ipa{ʈʰæ˩\textasciitilde{}ʈʰæ˧˥}}}}{}
\textcolor{teal}{\mytextsc{verbe}} \hspace{4pt} Ton~: MH.
\textcolor{PineGreen}{\selectlanguage{french}Démanger.} \zh{痒。}  ¶ \textcolor{darkblue}{\textbf{\ipa{le˧-ʈʰæ˩\textasciitilde{}ʈʰæ˩-ze˩}}} \textcolor{PineGreen}{\selectlanguage{french}\mytextsc{accomp} \mytextsc{red} \mytextsc{pfv}} \zh{\mytextsc{accomp} \mytextsc{red} \mytextsc{pfv}}  

\lhead{\firstmark}
\rhead{\botmark}

\subsection{\hspace{-0.5cm} {\Large \textcolor{darkblue}{\textbf{\ipa{ʈʰæ˧˥}}}}\hspace{0.5cm}[\kern2pt{\textcolor{darkblue}{\textbf{\ipa{ʈʰæ˧˥}}}}\kern2pt]} \hypertarget{t`\string_h\{\string_M\string_T1}{}
\markboth{\textcolor{darkblue}{\textbf{\ipa{ʈʰæ˧˥}}}}{}
\textcolor{teal}{\mytextsc{verbe}} \hspace{4pt} Ton~: MH.
\ding{202} \textcolor{PineGreen}{\selectlanguage{french}Mordre (mordre à belles dents dans quelque chose); piquer (une abeille pique quelqu'un).} \zh{咬、叮。}  ¶ \textcolor{darkblue}{\textbf{\ipa{tso˧\textasciitilde{}tso˧ ʈʰæ˩(-ze˩)}}} \textcolor{PineGreen}{\selectlanguage{french}mordre quelque chose} \zh{咬东西}  
 ¶ \textcolor{darkblue}{\textbf{\ipa{hĩ˧ ʈʰæ˩}}} \textcolor{PineGreen}{\selectlanguage{french}mordre quelqu'un (ex.: un chien mord un inconnu de passage)} \zh{咬人}  
\ding{203} \textcolor{PineGreen}{\selectlanguage{french}S'emboîter, s'ajuster (au sujet de pièces de charpenterie); emploi figuré de “mordre”: les pièces s'ajustent comme si elles mordaient les unes dans les autres.} \zh{对号、合适、相配:建房时,两块木材调剂地刚好合适,好像互相“咬紧”的样子。} 
\lhead{\firstmark}
\rhead{\botmark}

\subsection{\hspace{-0.5cm} {\Large \textcolor{darkblue}{\textbf{\ipa{ʈʰɤ˥\textsubscript{a}}}}}\hspace{0.5cm}[\kern2pt{\textcolor{darkblue}{\textbf{\ipa{ʈʰɤ˥}}}}\kern2pt]} \hypertarget{t`\string_h7\string_Ta1}{}
\markboth{\textcolor{darkblue}{\textbf{\ipa{ʈʰɤ˥\textsubscript{a}}}}}{}
\textcolor{teal}{\mytextsc{classificateur}} \hspace{4pt} Ton~: H\textsubscript{a}.
\textcolor{PineGreen}{\selectlanguage{french}Goutte (une goutte de liquide).} \zh{量词:滴。} 
\lhead{\firstmark}
\rhead{\botmark}

\subsection{\hspace{-0.5cm} {\Large \textcolor{darkblue}{\textbf{\ipa{ʈʰɤ˧˥}}}}\hspace{0.5cm}[\kern2pt{\textcolor{darkblue}{\textbf{\ipa{ʈʰɤ˩˥}}}}\kern2pt]} \hypertarget{t`\string_h7\string_M\string_T1}{}
\markboth{\textcolor{darkblue}{\textbf{\ipa{ʈʰɤ˧˥}}}}{}
\textcolor{teal}{\mytextsc{verbe}} \hspace{4pt} Ton~: MH.
\textcolor{PineGreen}{\selectlanguage{french}Goutter, dégouliner, couler goutte à goutte.} \zh{滴(水往下滴)。}  ¶ \textcolor{darkblue}{\textbf{\ipa{tʰi˧-ʈʰɤ˩\textasciitilde{}ʈʰɤ˩}}} \textcolor{PineGreen}{\selectlanguage{french}\mytextsc{dur} \mytextsc{red}} \zh{滴着滴着}  

\lhead{\firstmark}
\rhead{\botmark}

\subsection{\hspace{-0.5cm} {\Large \textcolor{darkblue}{\textbf{\ipa{ʈʰi˩\textsubscript{a}}}}}\hspace{0.5cm}[\kern2pt{\textcolor{darkblue}{\textbf{\ipa{ʈʰi˥}}}}\kern2pt]} \hypertarget{t`\string_hi\string_Ba1}{}
\markboth{\textcolor{darkblue}{\textbf{\ipa{ʈʰi˩\textsubscript{a}}}}}{}
\textcolor{teal}{\mytextsc{adjectif}} \hspace{4pt} Ton~: L\textsubscript{a}.
\textcolor{PineGreen}{\selectlanguage{french}Fatigué.} \zh{累、疲倦、精疲力竭。}  ¶ \textcolor{darkblue}{\textbf{\ipa{le˧-ʈʰi˩-ze˩}}} \textcolor{PineGreen}{\selectlanguage{french}\mytextsc{accomp} \string_ \mytextsc{pfv}} \zh{累了}  
 ¶ \textcolor{darkblue}{\textbf{\ipa{njɤ˧ | ʈʰi˩˥!}}} \textcolor{PineGreen}{\selectlanguage{french}je suis fatigué!} \zh{我累了!}  
 ¶ \textcolor{darkblue}{\textbf{\ipa{njɤ˧ | ʈʰi˩-ze˥!}}} \textcolor{PineGreen}{\selectlanguage{french}je suis fatigué!} \zh{我累了!}  

\lhead{\firstmark}
\rhead{\botmark}

\subsection{\hspace{-0.5cm} {\Large \textcolor{darkblue}{\textbf{\ipa{ʈʰɯ˩\textsubscript{a}}}}}\hspace{0.5cm}[\kern2pt{\textcolor{darkblue}{\textbf{\ipa{ʈʰɯ˥}}}}\kern2pt]} \hypertarget{t`\string_hM\string_Ba1}{}
\markboth{\textcolor{darkblue}{\textbf{\ipa{ʈʰɯ˩\textsubscript{a}}}}}{}
\textcolor{teal}{\mytextsc{verbe}} \hspace{4pt} Ton~: L\textsubscript{a}.
\textcolor{PineGreen}{\selectlanguage{french}Éternuer.} \zh{打喷嚏。}  ¶ \textcolor{darkblue}{\textbf{\ipa{ɖɯ˧-ʈʰɯ˧\textasciitilde{}ʈʰɯ˥}}} \textcolor{PineGreen}{\selectlanguage{french}\mytextsc{inchoatif} \mytextsc{red}}  

\lhead{\firstmark}
\rhead{\botmark}

\subsection{\hspace{-0.5cm} {\Large \textcolor{darkblue}{\textbf{\ipa{ʈʰɯ˩\textsubscript{b}}}}}\hspace{0.5cm}[\kern2pt{\textcolor{darkblue}{\textbf{\ipa{ʈʰɯ˩˥}}}}\kern2pt]} \hypertarget{t`\string_hM\string_Bb1}{}
\markboth{\textcolor{darkblue}{\textbf{\ipa{ʈʰɯ˩\textsubscript{b}}}}}{}
\textcolor{teal}{\mytextsc{verbe}} \hspace{4pt} Ton~: L\textsubscript{b}.
\textcolor{PineGreen}{\selectlanguage{french}Boire.} \zh{喝。}  ¶ \textcolor{darkblue}{\textbf{\ipa{njɤ˧ | mɤ˧-ʈʰɯ˩}}} \textcolor{PineGreen}{\selectlanguage{french}je ne bois pas} \zh{我不喝}  
 ¶ \textcolor{darkblue}{\textbf{\ipa{ʈʰɯ˩-ze˥}}} \textcolor{PineGreen}{\selectlanguage{french}\mytextsc{pfv}} \zh{喝了}  
 ¶ \textcolor{darkblue}{\textbf{\ipa{le˧-ʈʰɯ˩-ze˩}}} \textcolor{PineGreen}{\selectlanguage{french}\mytextsc{accomp} \string_ \mytextsc{pfv}} \zh{\mytextsc{accomp} \string_ \mytextsc{pfv}}  
 ¶ \textcolor{darkblue}{\textbf{\ipa{ʐɯ˧ ʈʰɯ˩}}} \textcolor{PineGreen}{\selectlanguage{french}boire du vin} \zh{喝酒}  
 ¶ \textcolor{darkblue}{\textbf{\ipa{jɤ˧ ʈʰɯ˩}}} \textcolor{PineGreen}{\selectlanguage{french}fumer (du tabac)} \zh{抽烟}  
 ¶ \textcolor{darkblue}{\textbf{\ipa{dʑɯ˩qʰæ˩ ʈʰɯ˩˥}}} \textcolor{PineGreen}{\selectlanguage{french}boire de l'eau froide} \zh{喝凉水}  
 ¶ \textcolor{darkblue}{\textbf{\ipa{dʑɯ˩tsʰi˩ ʈʰɯ˩˥}}} \textcolor{PineGreen}{\selectlanguage{french}boire de l'eau chaude} \zh{喝热水}  
 ¶ \textcolor{darkblue}{\textbf{\ipa{li˩ ʈʰɯ˩}}} \textcolor{PineGreen}{\selectlanguage{french}boire du thé} \zh{喝茶}  
 ¶ \textcolor{darkblue}{\textbf{\ipa{v̩˩dʑɯ˩ ʈʰɯ˩˥}}} \textcolor{PineGreen}{\selectlanguage{french}boire de la soupe} \zh{喝汤}  
 ¶ \textcolor{darkblue}{\textbf{\ipa{dʑɯ˧ ʈʰɯ˧}}} \textcolor{PineGreen}{\selectlanguage{french}boire de l'eau} \zh{喝水}  
 ¶ \textcolor{darkblue}{\textbf{\ipa{njɤ˧ | dʑɯ˧ ʈʰɯ˧-ze˧}}} \textcolor{PineGreen}{\selectlanguage{french}j'ai bu de l'eau} \zh{我喝了水}  
 ¶ \textcolor{darkblue}{\textbf{\ipa{njɤ˧ | dʑɯ˧ ʈʰɯ˧-zo˧-ho˩}}} \textcolor{PineGreen}{\selectlanguage{french}Il va falloir que je boive de l'eau.} \zh{我应该喝水了。}  

\lhead{\firstmark}
\rhead{\botmark}

\newpage
\section*{\centering- \textcolor{darkblue}{\textbf{\ipa{ʈʂ}}} -}
\subsection{\hspace{-0.5cm} {\Large \textcolor{darkblue}{\textbf{\ipa{ʈʂɑ˧tɑ˥}}}}\hspace{0.5cm}[\kern2pt{\textcolor{darkblue}{\textbf{\ipa{ʈʂɑ˧tɑ˥}}}}\kern2pt]} \hypertarget{t`s`A\string_MtA\string_T1}{}
\markboth{\textcolor{darkblue}{\textbf{\ipa{ʈʂɑ˧tɑ˥}}}}{}
\textcolor{teal}{\mytextsc{nom}} \hspace{4pt} Ton~: H\#.
\textcolor{PineGreen}{\selectlanguage{french}Signe.} \zh{记号。}  ¶ \textcolor{darkblue}{\textbf{\ipa{ʈʂɑ˧tɑ˥ ʝi˩}}} \textcolor{PineGreen}{\selectlanguage{french}faire une marque, inscrire un signe} \zh{写一个符号、画一个符号}  
 ¶ \textcolor{darkblue}{\textbf{\ipa{ʈʂɑ˧tɑ˥ tɕi˩}}} \textcolor{PineGreen}{\selectlanguage{french}écrire des signes, faire des marques (pas pour un unique signe: se rapproche de l'écriture d'un message/texte)} \zh{写符号、画符号}  
 \zh{量词}: \textcolor{darkblue}{\textbf{\ipa{kʰwɤ˥}}}  \mytextsc{clf}~: \textcolor{darkblue}{\textbf{\ipa{kʰwɤ˥}}} 
\lhead{\firstmark}
\rhead{\botmark}

\subsection{\hspace{-0.5cm} {\Large \textcolor{darkblue}{\textbf{\ipa{ʈʂæ˧mo\#˥}}}}\hspace{0.5cm}[\kern2pt{\textcolor{darkblue}{\textbf{\ipa{ʈʂæ˧mo˧}}}}\kern2pt]} \hypertarget{t`s`\{\string_Mmo\#\string_T1}{}
\markboth{\textcolor{darkblue}{\textbf{\ipa{ʈʂæ˧mo\#˥}}}}{}
\textcolor{teal}{\mytextsc{nom}} \hspace{4pt} Ton~: \#H.
\textcolor{PineGreen}{\selectlanguage{french}Un champignon vénéneux.} \zh{一种有毒的菌子。}  ¶ \textcolor{darkblue}{\textbf{\ipa{ʈʂæ˧mo˧-kʰi˧tɕʰɯ˩-mo˩ / kʰi˧tɕʰɯ˩-mo˩}}} \textcolor{PineGreen}{\selectlanguage{french}même sens} \zh{同上}  
\textit{Syn~:} \hyperlink{}{\textcolor{darkblue}{\textbf{\ipa{kʰi˧tɕʰɯ˩-mo˩}}}}. 
\lhead{\firstmark}
\rhead{\botmark}

\subsection{\hspace{-0.5cm} {\Large \textcolor{darkblue}{\textbf{\ipa{ʈʂæ˧ʈʂɯ˧}}}}\hspace{0.5cm}[\kern2pt{\textcolor{darkblue}{\textbf{\ipa{ʈʂæ˧ʈʂɯ˧}}}}\kern2pt]} \hypertarget{t`s`\{\string_Mt`s`M\string_M1}{}
\markboth{\textcolor{darkblue}{\textbf{\ipa{ʈʂæ˧ʈʂɯ˧}}}}{}
\textcolor{teal}{\mytextsc{adverbe}} \hspace{4pt} Ton~: M.
\textcolor{PineGreen}{\selectlanguage{french}Véritablement, vraiment, pour de vrai.} \zh{确切、真的。} 
\lhead{\firstmark}
\rhead{\botmark}

\subsection{\hspace{-0.5cm} {\Large \textcolor{darkblue}{\textbf{\ipa{ʈʂæ˧wɤ˩}}}}\hspace{0.5cm}[\kern2pt{\textcolor{darkblue}{\textbf{\ipa{ʈʂæ˧wɤ˩}}}}\kern2pt]} \hypertarget{t`s`\{\string_Mw7\string_B1}{}
\markboth{\textcolor{darkblue}{\textbf{\ipa{ʈʂæ˧wɤ˩}}}}{}
\textcolor{teal}{\mytextsc{nom}} \hspace{4pt} Ton~: L\#.
\textcolor{PineGreen}{\selectlanguage{french}Serviteur.} \zh{仆人,佣人。}  \zh{量词}: \textcolor{darkblue}{\textbf{\ipa{v̩˧}}}  \mytextsc{clf}~: \textcolor{darkblue}{\textbf{\ipa{v̩˧}}} 
\lhead{\firstmark}
\rhead{\botmark}

\subsection{\hspace{-0.5cm} {\Large \textcolor{darkblue}{\textbf{\ipa{ʈʂæ˩do\#˥}}}}\hspace{0.5cm}[\kern2pt{\textcolor{darkblue}{\textbf{\ipa{ʈʂæ˧do˧}}}}\kern2pt]} \hypertarget{t`s`\{\string_Bdo\#\string_T1}{}
\markboth{\textcolor{darkblue}{\textbf{\ipa{ʈʂæ˩do\#˥}}}}{}
\textcolor{teal}{\mytextsc{nom}} \hspace{4pt} Ton~: LM+\#H.
\textcolor{PineGreen}{\selectlanguage{french}Récipient dans lequel on bat le thé au beurre (tube-baratte en bois); aussi: grande baratte pour baratter le beurre.} \zh{打酥油茶的罐、酥油茶搅拌器,黄油搅乳器。}  \zh{量词}: \textcolor{darkblue}{\textbf{\ipa{ɭɯ˧}}}  \mytextsc{clf}~: \textcolor{darkblue}{\textbf{\ipa{ɭɯ˧}}} 
\lhead{\firstmark}
\rhead{\botmark}

\subsection{\hspace{-0.5cm} {\Large \textcolor{darkblue}{\textbf{\ipa{ʈʂæ˧˥}}} \textsubscript{1}}\hspace{0.5cm}[\kern2pt{\textcolor{darkblue}{\textbf{\ipa{ʈʂæ˧˥}}}}\kern2pt]} \hypertarget{t`s`\{\string_M\string_T1}{}
\markboth{\textcolor{darkblue}{\textbf{\ipa{ʈʂæ˧˥}}} \textsubscript{1}}{}
\textcolor{teal}{\mytextsc{verbe}} \hspace{4pt} Ton~: MH.
\textcolor{PineGreen}{\selectlanguage{french}Voler, s'emparer de, extorquer, arracher.} \zh{抢劫、抢。}  ¶ \textcolor{darkblue}{\textbf{\ipa{le˧-ʈʂæ˧-ze˥}}} \textcolor{PineGreen}{\selectlanguage{french}\mytextsc{accomp} \string_ \mytextsc{pfv}} \zh{抢了}  
 ¶ \textcolor{darkblue}{\textbf{\ipa{tso˧\textasciitilde{}tso˧ ʈʂæ˩}}} \textcolor{PineGreen}{\selectlanguage{french}voler des choses} \zh{抢东西}  
 ¶ \textcolor{darkblue}{\textbf{\ipa{le˧-ʈʂæ˧-po˥-hɯ˩(-ze˩)}}} \textcolor{PineGreen}{\selectlanguage{french}(il) a extorqué quelque chose et est parti avec} \zh{把东西抢走了}  
 ¶ \textcolor{darkblue}{\textbf{\ipa{hĩ˧ ʈʂæ˩}}} \textcolor{PineGreen}{\selectlanguage{french}voler les gens, extorquer des choses aux gens} \zh{抢劫}  

\lhead{\firstmark}
\rhead{\botmark}

\subsection{\hspace{-0.5cm} {\Large \textcolor{darkblue}{\textbf{\ipa{ʈʂæ˧˥}}} \textsubscript{2}}\hspace{0.5cm}[\kern2pt{\textcolor{darkblue}{\textbf{\ipa{ʈʂæ˧˥}}}}\kern2pt]} \hypertarget{t`s`\{\string_M\string_T2}{}
\markboth{\textcolor{darkblue}{\textbf{\ipa{ʈʂæ˧˥}}} \textsubscript{2}}{}
\textcolor{teal}{\mytextsc{verbe}} \hspace{4pt} Ton~: MH.
\textcolor{PineGreen}{\selectlanguage{french}Envoyer qqun.} \zh{派人。}  ¶ \textcolor{darkblue}{\textbf{\ipa{ɖɯ˧-v̩˧ ʈʂæ˧˥}}} \textcolor{PineGreen}{\selectlanguage{french}envoyer quelqu'un} \zh{派一个人}  
 ¶ \textcolor{darkblue}{\textbf{\ipa{hĩ˧ ʈʂæ˩}}} \textcolor{PineGreen}{\selectlanguage{french}idem} \zh{同上}  

\lhead{\firstmark}
\rhead{\botmark}

\subsection{\hspace{-0.5cm} {\Large \textcolor{darkblue}{\textbf{\ipa{ʈʂæ˧˥}}} \textsubscript{3}}\hspace{0.5cm}[\kern2pt{\textcolor{darkblue}{\textbf{\ipa{ʈʂæ˧˥}}}}\kern2pt]} \hypertarget{t`s`\{\string_M\string_T3}{}
\markboth{\textcolor{darkblue}{\textbf{\ipa{ʈʂæ˧˥}}} \textsubscript{3}}{}
\textcolor{teal}{\mytextsc{verbe}} \hspace{4pt} Ton~: MH.
\textcolor{PineGreen}{\selectlanguage{french}Fixer, accrocher (ex.: coudre un bouton sur un vêtement; attacher la selle sur un cheval).} \zh{安上(如:缝扣子、安上马鞍)。}  ¶ \textcolor{darkblue}{\textbf{\ipa{pv̩˩ɭɯ˥ ʈʂæ˩}}} \textcolor{PineGreen}{\selectlanguage{french}coudre un bouton (sur un vêtement)} \zh{缝扣子}  
 ¶ \textcolor{darkblue}{\textbf{\ipa{ʐwæ˧tɕi˥ ʈʂæ˩}}} \textcolor{PineGreen}{\selectlanguage{french}attacher la selle d'un cheval; seller un cheval} \zh{备鞍}  
 ¶ \textcolor{darkblue}{\textbf{\ipa{ɖɯ˧-ɲi˥, | so˧-ʂɯ˧ ʈʂæ˧˥!}}} \textcolor{PineGreen}{\selectlanguage{french}Au cours d'une journée, on selle trois fois (les chevaux, lorsqu'on est parti en caravane)!} \zh{(走马帮,)一天备鞍三次!}  

\lhead{\firstmark}
\rhead{\botmark}

\subsection{\hspace{-0.5cm} {\Large \textcolor{darkblue}{\textbf{\ipa{ʈʂæ˧˥}}} \textsubscript{4}}\hspace{0.5cm}[\kern2pt{\textcolor{darkblue}{\textbf{\ipa{ʈʂæ˧˥}}}}\kern2pt]} \hypertarget{t`s`\{\string_M\string_T4}{}
\markboth{\textcolor{darkblue}{\textbf{\ipa{ʈʂæ˧˥}}} \textsubscript{4}}{}
\textcolor{teal}{\mytextsc{nom}} \hspace{4pt} Ton~: MH.
\ding{202} \textcolor{PineGreen}{\selectlanguage{french}Articulation.} \zh{关节。}  \zh{量词}: \textcolor{darkblue}{\textbf{\ipa{ʈʂæ˧˥}}} \ding{203} \textcolor{PineGreen}{\selectlanguage{french}Période, époque; segment de temps.} \zh{段(时间)、时代。}  \mytextsc{clf}~: \textcolor{darkblue}{\textbf{\ipa{ʈʂæ˧˥}}} 
\lhead{\firstmark}
\rhead{\botmark}

\subsection{\hspace{-0.5cm} {\Large \textcolor{darkblue}{\textbf{\ipa{ʈʂæ˧˥\textsubscript{a}}}}}\hspace{0.5cm}[\kern2pt{\textcolor{darkblue}{\textbf{\ipa{ʈʂæ˥}}}}\kern2pt]} \hypertarget{t`s`\{\string_M\string_Ta1}{}
\markboth{\textcolor{darkblue}{\textbf{\ipa{ʈʂæ˧˥\textsubscript{a}}}}}{}
\textcolor{teal}{\mytextsc{classificateur}} \hspace{4pt} Ton~: MH\textsubscript{a}.
\textcolor{PineGreen}{\selectlanguage{french}Classificateur des épis de maïs (mûrs).} \zh{量词.玉米(一棒)。}  ¶ \textcolor{darkblue}{\textbf{\ipa{qʰɑ˧dze˧ | ɖɯ˧-ʈʂæ˧˥}}} \textcolor{PineGreen}{\selectlanguage{french}un épi de maïs} \zh{一棒玉米}  
 ¶ \textcolor{darkblue}{\textbf{\ipa{qʰɑ˧dze˧ | ɖɯ˧-ʈʂæ˧ ɖʐɤ˥}}} \textcolor{PineGreen}{\selectlanguage{french}arracher un épi de maïs, récolter un épi de maïs} \zh{掰一棒玉米}  

\lhead{\firstmark}
\rhead{\botmark}

\subsection{\hspace{-0.5cm} {\Large \textcolor{darkblue}{\textbf{\ipa{ʈʂe˥}}} \textsubscript{1}}\hspace{0.5cm}[\kern2pt{\textcolor{darkblue}{\textbf{\ipa{ʈʂe˥}}}}\kern2pt]} \hypertarget{t`s`e\string_T1}{}
\markboth{\textcolor{darkblue}{\textbf{\ipa{ʈʂe˥}}} \textsubscript{1}}{}
\textcolor{teal}{\mytextsc{nom}} \hspace{4pt} Ton~: \#H.
\textcolor{PineGreen}{\selectlanguage{french}Terre.} \zh{土壤。}  ¶ \textcolor{darkblue}{\textbf{\ipa{ʈʂe˧pv̩˩}}} \textcolor{PineGreen}{\selectlanguage{french}terre sèche} \zh{干土}  
 ¶ \textcolor{darkblue}{\textbf{\ipa{ʈʂe˧ sɯ˧\textasciitilde{}sɯ˥}}} \textcolor{PineGreen}{\selectlanguage{french}'terre crue': terre qui n'a pas été préparée pour l'agriculture par l'ajout de fumier, etc} \zh{‘生土’:没有经过加工(加肥料等等)的土,还不适合种农作物}  

\lhead{\firstmark}
\rhead{\botmark}

\subsection{\hspace{-0.5cm} {\Large \textcolor{darkblue}{\textbf{\ipa{ʈʂe˥}}} \textsubscript{2}}\hspace{0.5cm}[\kern2pt{\textcolor{darkblue}{\textbf{\ipa{ʈʂe˥}}}}\kern2pt]} \hypertarget{t`s`e\string_T2}{}
\markboth{\textcolor{darkblue}{\textbf{\ipa{ʈʂe˥}}} \textsubscript{2}}{}
\textcolor{teal}{\mytextsc{nom}} \hspace{4pt} Ton~: \#H.
\textcolor{PineGreen}{\selectlanguage{french}Aiguille.} \zh{针(汉语借词)。}  \zh{量词}: \textcolor{darkblue}{\textbf{\ipa{ɭɯ˧}}}  \mytextsc{clf}~: \textcolor{darkblue}{\textbf{\ipa{ɭɯ˧}}} \textit{Voir~:} \hyperlink{}{\textcolor{darkblue}{\textbf{\ipa{ʁo˧˥}}}} 
\lhead{\firstmark}
\rhead{\botmark}

\subsection{\hspace{-0.5cm} {\Large \textcolor{darkblue}{\textbf{\ipa{ʈʂe˧dɑ˥}}}}\hspace{0.5cm}[\kern2pt{\textcolor{darkblue}{\textbf{\ipa{ʈʂe˧dɑ˥}}}}\kern2pt]} \hypertarget{t`s`e\string_MdA\string_T1}{}
\markboth{\textcolor{darkblue}{\textbf{\ipa{ʈʂe˧dɑ˥}}}}{}
\textcolor{teal}{\mytextsc{nom}} \hspace{4pt} Ton~: H\#.
\textcolor{PineGreen}{\selectlanguage{french}Cloison.} \zh{隔板。}  \zh{量词}: \textcolor{darkblue}{\textbf{\ipa{do˥}}}  \mytextsc{clf}~: \textcolor{darkblue}{\textbf{\ipa{do˥}}} 
\lhead{\firstmark}
\rhead{\botmark}

\subsection{\hspace{-0.5cm} {\Large \textcolor{darkblue}{\textbf{\ipa{ʈʂe˧gi˥\$}}}}\hspace{0.5cm}[\kern2pt{\textcolor{darkblue}{\textbf{\ipa{ʈʂe˧gi˥}}}}\kern2pt]} \hypertarget{t`s`e\string_Mgi\string_T\$1}{}
\markboth{\textcolor{darkblue}{\textbf{\ipa{ʈʂe˧gi˥\$}}}}{}
\textcolor{teal}{\mytextsc{adverbe}} \hspace{4pt} Ton~: H\$.
\textcolor{PineGreen}{\selectlanguage{french}Entre, au milieu de.} \zh{中间、之间、间。}  ¶ \textcolor{darkblue}{\textbf{\ipa{ə˧-sɯ˩kv̩˩-ʈʂe˩gi˩}}} \textcolor{PineGreen}{\selectlanguage{french}entre nous, dans l'espace qui nous sépare} \zh{在咱们之间(的空间)}  

\lhead{\firstmark}
\rhead{\botmark}

\subsection{\hspace{-0.5cm} {\Large \textcolor{darkblue}{\textbf{\ipa{ʈʂe˩\textsubscript{a}}}}}\hspace{0.5cm}[\kern2pt{\textcolor{darkblue}{\textbf{\ipa{ʈʂe˩˥}}}}\kern2pt]} \hypertarget{t`s`e\string_Ba1}{}
\markboth{\textcolor{darkblue}{\textbf{\ipa{ʈʂe˩\textsubscript{a}}}}}{}
\textcolor{teal}{\mytextsc{verbe}} \hspace{4pt} Ton~: L\textsubscript{a}.
\textcolor{PineGreen}{\selectlanguage{french}Percer, transpercer (une écharde, un piquant de plante...).} \zh{刺痛。}  ¶ \textcolor{darkblue}{\textbf{\ipa{le˧-ʈʂe˩-ze˩}}} \textcolor{PineGreen}{\selectlanguage{french}\mytextsc{accomp} \string_ \mytextsc{pfv}} \zh{\mytextsc{accomp} \string_ \mytextsc{pfv}}  
 ¶ \textcolor{darkblue}{\textbf{\ipa{tɕʰi˧-ɳɯ˧ ʈʂe˩-ze˩}}} \textcolor{PineGreen}{\selectlanguage{french}être piqué par une épine, se prendre une épine} \zh{被刺所刺痛}  
 ¶ \textcolor{darkblue}{\textbf{\ipa{tso˧\textasciitilde{}tso˧ ʈʂe˥}}} \textcolor{PineGreen}{\selectlanguage{french}percer quelque chose} \zh{刺到一个东西}  

\lhead{\firstmark}
\rhead{\botmark}

\subsection{\hspace{-0.5cm} {\Large \textcolor{darkblue}{\textbf{\ipa{ʈʂe˩kʰɯ˩}}}}\hspace{0.5cm}[\kern2pt{\textcolor{darkblue}{\textbf{\ipa{ʈʂe˩kʰɯ˩˥}}}}\kern2pt]} \hypertarget{t`s`e\string_Bk\string_hM\string_B1}{}
\markboth{\textcolor{darkblue}{\textbf{\ipa{ʈʂe˩kʰɯ˩}}}}{}
\textcolor{teal}{\mytextsc{nom}} \hspace{4pt} Ton~: L.
\textcolor{PineGreen}{\selectlanguage{french}Couture (d'un vêtement).} \zh{缝。}  \zh{量词}: \textcolor{darkblue}{\textbf{\ipa{ɭɯ˧}}}  \mytextsc{clf}~: \textcolor{darkblue}{\textbf{\ipa{ɭɯ˧}}} 
\lhead{\firstmark}
\rhead{\botmark}

\subsection{\hspace{-0.5cm} {\Large \textcolor{darkblue}{\textbf{\ipa{ʈʂe˩ʂwæ˧˥}}}}\hspace{0.5cm}[\kern2pt{\textcolor{darkblue}{\textbf{\ipa{ʈʂe˩ʂwæ˧˥}}}}\kern2pt]} \hypertarget{t`s`e\string_Bs`w\{\string_M\string_T1}{}
\markboth{\textcolor{darkblue}{\textbf{\ipa{ʈʂe˩ʂwæ˧˥}}}}{}
\textcolor{teal}{\mytextsc{nom}} \hspace{4pt} Ton~: LM+MH\#.
\textcolor{PineGreen}{\selectlanguage{french}Gravier, sable grossier.} \zh{砾石。} 
\lhead{\firstmark}
\rhead{\botmark}

\subsection{\hspace{-0.5cm} {\Large \textcolor{darkblue}{\textbf{\ipa{ʈʂɤ˧\textsubscript{a}}}}}\hspace{0.5cm}[\kern2pt{\textcolor{darkblue}{\textbf{\ipa{ʈʂɤ˥}}}}\kern2pt]} \hypertarget{t`s`7\string_Ma1}{}
\markboth{\textcolor{darkblue}{\textbf{\ipa{ʈʂɤ˧\textsubscript{a}}}}}{}
\textcolor{teal}{\mytextsc{verbe}} \hspace{4pt} Ton~: M\textsubscript{a}.
\ding{202} \textcolor{PineGreen}{\selectlanguage{french}Compter; calculer.} \zh{数、算。}  ¶ \textcolor{darkblue}{\textbf{\ipa{ʈʂɤ˧\textasciitilde{}ʈʂɤ˩}}} \textcolor{PineGreen}{\selectlanguage{french}\mytextsc{red}} \zh{\mytextsc{重叠:算一算}}  
 ¶ \textcolor{darkblue}{\textbf{\ipa{ɖɯ˧-ʈʂɤ˥\textasciitilde{}ʈʂɤ˩-ɻ̍˩}}} \textcolor{PineGreen}{\selectlanguage{french}faire quelques calculs} \zh{算一下}  
 ¶ \textcolor{darkblue}{\textbf{\ipa{tso˧\textasciitilde{}tso˧ ʈʂɤ˩}}} \textcolor{PineGreen}{\selectlanguage{french}compter des choses} \zh{数东西}  
 ¶ \textcolor{darkblue}{\textbf{\ipa{hĩ˧ ʈʂɤ˩}}} \textcolor{PineGreen}{\selectlanguage{french}compter les gens} \zh{数人}  
 ¶ \textcolor{darkblue}{\textbf{\ipa{bo˩ ʈʂɤ˧}}} \textcolor{PineGreen}{\selectlanguage{french}compter les porcs} \zh{数猪}  
 ¶ \textcolor{darkblue}{\textbf{\ipa{le˧-ʈʂɤ˧-ze˧}}} \textcolor{PineGreen}{\selectlanguage{french}\mytextsc{accomp} \string_ \mytextsc{pfv}} \zh{数了}  
\ding{203} \textcolor{PineGreen}{\selectlanguage{french}Dire la bonne aventure, pratiquer la divination.} \zh{算命。}  ¶ \textcolor{darkblue}{\textbf{\ipa{le˧-ʈʂɤ˥\textasciitilde{}ʈʂɤ˩}}} \textcolor{PineGreen}{\selectlanguage{french}dire la bonne aventure, pratiquer la divination} \zh{算命}  
 ¶ \textcolor{darkblue}{\textbf{\ipa{ɖɯ˧-ʈʂɤ˥\textasciitilde{}ʈʂɤ˩-ɻ̍˩}}} \textcolor{PineGreen}{\selectlanguage{french}dire la bonne aventure} \zh{算一下命}  
 ¶ \textcolor{darkblue}{\textbf{\ipa{ɲi˧ŋwɤ˩hɑ̃˩tʰɑ˩ | ɖɯ˧-ɭɯ˧ | ʈʂɤ˧-bi˧!}}} \textcolor{PineGreen}{\selectlanguage{french}On va chercher un jour propice! (pour un événement tel qu'un mariage ou la construction d'une maison)} \zh{要掐算一下日子}  
 ¶ \textcolor{darkblue}{\textbf{\ipa{kɯ˧ ʈʂɤ˧, | hɑ̃˧ ʈʂɤ˧}}} \textcolor{PineGreen}{\selectlanguage{french}chercher un jour propice pour un événement, tel que le début de la construction d'une maison; littéralement: “compter les étoiles, compter les jours”} \zh{掐算一下。直译:“算星星,算日子”。}  
\ding{204} \textcolor{PineGreen}{\selectlanguage{french}Compter comme, être, avoir fonction de, avoir rôle de.} \zh{算是,当作。}  ¶ \textcolor{darkblue}{\textbf{\ipa{hĩ˧ ɖɯ˧-v̩˧ ʈʂɤ˧-ze˧!}}} \textcolor{PineGreen}{\selectlanguage{french}(Elle/il) compte maintenant comme une grande personne! / C'est un(e) adulte, maintenant! (Ce qu'on dit d'un enfant qui atteint l'âge adulte: 13 ans.)} \zh{变成大人了!(十三岁成年礼时常用的一句话)}  
 ¶ \textcolor{darkblue}{\textbf{\ipa{dʑɤ˩ ʈʂɤ˧}}} \textcolor{PineGreen}{\selectlanguage{french}être très bien} \zh{算是很好的}  
 ¶ \textcolor{darkblue}{\textbf{\ipa{ʈʂʰɯ˧ | õ˧-bv̩˥-õ˩ | dʑɤ˩ʈʂɤ˧ (+ | ʐwæ˩˥)}}} \textcolor{PineGreen}{\selectlanguage{french}Il a une haute idée de lui-même! / Il est orgueilleux!} \zh{他觉得自己很了不起!}  
 ¶ \textcolor{darkblue}{\textbf{\ipa{hɤ˩ ʈʂɤ˩˥}}} \textcolor{PineGreen}{\selectlanguage{french}être habile / admirable / remarquable, être considéré comme habile, compter comme (quelqu'un d')habile} \zh{算很了不起的,算很能干的}  
 ¶ \textcolor{darkblue}{\textbf{\ipa{ɖwæ˧˥ | hɤ˩ ʈʂɤ˩˥}}} \textcolor{PineGreen}{\selectlanguage{french}même sens} \zh{同上}  
 ¶ \textcolor{darkblue}{\textbf{\ipa{ʈʂʰɯ˧ | gi˧zɯ˧ ʈʂɤ˧-tso˧-ɲi˥.}}} \textcolor{PineGreen}{\selectlanguage{french}C'est le petit frère / il a le statut de petit frère! (Commentaire qui rappelle le statut familial de la personne concernée.)} \zh{他是做弟弟的!(强调该人的社会角色)}  
 ¶ \textcolor{darkblue}{\textbf{\ipa{ʈʂʰɯ˧ | gi˧zɯ˧ ʈʂɤ˧-ɲi˥!}}} \textcolor{PineGreen}{\selectlanguage{french}C'est le petit frère / il a le statut de petit frère! (Commentaire qui rappelle le statut familial de la personne concernée.)} \zh{他是做弟弟的!(强调该人的社会角色)}  
 ¶ \textcolor{darkblue}{\textbf{\ipa{ʈʂʰɯ˧ | bɤ˧ ʈʂɤ˧-tso˧-ɲi˥!}}} \textcolor{PineGreen}{\selectlanguage{french}Il/elle est pumi! (Commentaire qui rappelle un élément de l'identité de la personne concernée.)} \zh{他是普米族!(强调该人的民族)}  
 ¶ \textcolor{darkblue}{\textbf{\ipa{ʈʂʰɯ˧ | nɑ˩ ʈʂɤ˧-tso˧-ɲi˥!}}} \textcolor{PineGreen}{\selectlanguage{french}Il/elle est na! (Commentaire qui rappelle un élément de l'identité de la personne concernée)} \zh{他是摩梭人!(强调该人的民族身份)}  
 ¶ \textcolor{darkblue}{\textbf{\ipa{ʈʂʰɯ˧ | æ˧mv̩˩ ʈʂɤ˩-ɲi˩!}}} \textcolor{PineGreen}{\selectlanguage{french}C'est elle la grande soeur! / C'est lui le grand frère! (Commentaire qui rappelle le statut familial de la personne concernée.)} \zh{她是做姐姐的! / 他是做哥哥的!(强调该人的社会角色)}  
 ¶ \textcolor{darkblue}{\textbf{\ipa{ʈʂʰɯ˧ | gi˧zɯ˧-go˩mi˩ ʈʂɤ˩-ɲi˩!}}} \textcolor{PineGreen}{\selectlanguage{french}Ils sont frère et sœur!} \zh{他们是(兄)弟(姐)妹!}  
 ¶ \textcolor{darkblue}{\textbf{\ipa{ʈʂʰɯ˧ | æ˧mv̩˧-go˧mi˥ | ʈʂɤ˧-tso˧ mɤ˧-ɲi˥! | mɤ˧-ʈʂɤ˧!}}} \textcolor{PineGreen}{\selectlanguage{french}Ils ne sont pas frères et sœurs! / Ils n'ont pas cette relation-là!} \zh{他们不算是兄弟姐妹!}  
 ¶ \textcolor{darkblue}{\textbf{\ipa{ʐwæ˩ ʈʂɤ˩}}} \textcolor{PineGreen}{\selectlanguage{french}remarquable, extraordinaire, exceptionnel} \zh{了不起}  
 ¶ \textcolor{darkblue}{\textbf{\ipa{ʈʂʰɯ˧ | ə˧tso˧ ʐwæ˩ ʈʂɤ˩-tso˩ dʑo˩?}}} \textcolor{PineGreen}{\selectlanguage{french}Qu'est-ce qu'elle/il a de si remarquable? / Quelles qualités exceptionnelles a-t-il/elle (que je doive prendre son conseil/suivre son avis)?} \zh{他有什么了不起的?}  
 ¶ \textcolor{darkblue}{\textbf{\ipa{ʈʂʰɯ˧ | ʐwæ˩ ʈʂɤ˩˥ | ʐwæ˩˥!}}} \textcolor{PineGreen}{\selectlanguage{french}C'est quelqu'un de vraiment remarquable/extraordinaire!} \zh{他非常了不起!}  

\lhead{\firstmark}
\rhead{\botmark}

\subsection{\hspace{-0.5cm} {\Large \textcolor{darkblue}{\textbf{\ipa{ʈʂo˧kʰɯ˩}}}}\hspace{0.5cm}[\kern2pt{\textcolor{darkblue}{\textbf{\ipa{ʈʂo˧kʰɯ˩}}}}\kern2pt]} \hypertarget{t`s`o\string_Mk\string_hM\string_B1}{}
\markboth{\textcolor{darkblue}{\textbf{\ipa{ʈʂo˧kʰɯ˩}}}}{}
\textcolor{teal}{\mytextsc{nom}} \hspace{4pt} Ton~: L\#.
\textcolor{PineGreen}{\selectlanguage{french}Rituel pour la mort d'une personne de sa famille.} \zh{忠克:亲人去世时举行的仪式。} 
\lhead{\firstmark}
\rhead{\botmark}

\subsection{\hspace{-0.5cm} {\Large \textcolor{darkblue}{\textbf{\ipa{ʈʂo˧ɭɯ\#˥}}}}\hspace{0.5cm}[\kern2pt{\textcolor{darkblue}{\textbf{\ipa{ʈʂo˧ɭɯ˧}}}}\kern2pt]} \hypertarget{t`s`o\string_Ml\string_RM\#\string_T1}{}
\markboth{\textcolor{darkblue}{\textbf{\ipa{ʈʂo˧ɭɯ\#˥}}}}{}
\textcolor{teal}{\mytextsc{nom}} \hspace{4pt} Ton~: \#H.
\textcolor{PineGreen}{\selectlanguage{french}Moulin à main.} \zh{手推磨。}  ¶ \textcolor{darkblue}{\textbf{\ipa{ʈʂo˧ɭɯ˧-nv̩˥mi˩}}} \textcolor{PineGreen}{\selectlanguage{french}l'axe du moulin (littéralement: son cœur)} \zh{手推磨的轴心}  
 \zh{量词}: \textcolor{darkblue}{\textbf{\ipa{nɑ˧}}}  \mytextsc{clf}~: \textcolor{darkblue}{\textbf{\ipa{nɑ˧}}} 
\lhead{\firstmark}
\rhead{\botmark}

\subsection{\hspace{-0.5cm} {\Large \textcolor{darkblue}{\textbf{\ipa{ʈʂo˧ɭɯ˧ʈʂo˧˥}}}}\hspace{0.5cm}[\kern2pt{\textcolor{darkblue}{\textbf{\ipa{ʈʂo˧ɭɯ˧ʈʂo˧˥}}}}\kern2pt]} \hypertarget{t`s`o\string_Ml\string_RM\string_Mt`s`o\string_M\string_T1}{}
\markboth{\textcolor{darkblue}{\textbf{\ipa{ʈʂo˧ɭɯ˧ʈʂo˧˥}}}}{}
\textcolor{teal}{\mytextsc{nom}} \hspace{4pt} Ton~: MH\#.
\textcolor{PineGreen}{\selectlanguage{french}Insecte aquatique.} \zh{一种水虫。}  \zh{量词}: \textcolor{darkblue}{\textbf{\ipa{mi˩}}}  \mytextsc{clf}~: \textcolor{darkblue}{\textbf{\ipa{mi˩}}} 
\lhead{\firstmark}
\rhead{\botmark}

\subsection{\hspace{-0.5cm} {\Large \textcolor{darkblue}{\textbf{\ipa{ʈʂo˧ʂɯ\#˥}}}}\hspace{0.5cm}[\kern2pt{\textcolor{darkblue}{\textbf{\ipa{ʈʂo˧ʂɯ˧}}}}\kern2pt]} \hypertarget{t`s`o\string_Ms`M\#\string_T1}{}
\markboth{\textcolor{darkblue}{\textbf{\ipa{ʈʂo˧ʂɯ\#˥}}}}{}
\textcolor{teal}{\mytextsc{nom}} \hspace{4pt} Ton~: \#H.
\textcolor{PineGreen}{\selectlanguage{french}Village de Zhongshi.} \zh{忠实(永宁的一个村落)。}  ¶ \textcolor{darkblue}{\textbf{\ipa{ɖæ˩ʂɯ\#˥, | ʈʂo˧ʂɯ\#˥, | bɤ˩tɕʰɯ˩˥, | dɑ˧pʰo˥, | bɤ˧dzi˩, | dze˧bo˧}}} \textcolor{PineGreen}{\selectlanguage{french}les six villages de la plaine de Yongning, dans l'ordre, qui prend comme point d'origine le village le plus proche du Lac} \zh{永宁坝的六个村落,按传统排序:从距离泸沽湖最近的村落说起。}  

\lhead{\firstmark}
\rhead{\botmark}

\subsection{\hspace{-0.5cm} {\Large \textcolor{darkblue}{\textbf{\ipa{ʈʂo˧tsɯ˥}}}}\hspace{0.5cm}[\kern2pt{\textcolor{darkblue}{\textbf{\ipa{ʈʂo˧tsɯ˥}}}}\kern2pt]} \hypertarget{t`s`o\string_MtsM\string_T1}{}
\markboth{\textcolor{darkblue}{\textbf{\ipa{ʈʂo˧tsɯ˥}}}}{}
\textcolor{teal}{\mytextsc{nom}} \hspace{4pt} Ton~: H\#.
\textcolor{PineGreen}{\selectlanguage{french}Table.} \zh{桌子(汉语借词)。}  Emprunt~: chinois  \zh{桌子}
 \zh{量词}: \textcolor{darkblue}{\textbf{\ipa{pɤ˩}}}  \mytextsc{clf}~: \textcolor{darkblue}{\textbf{\ipa{pɤ˩}}} \textit{Voir~:} \hyperlink{}{\textcolor{darkblue}{\textbf{\ipa{sɯ˧ɻæ˧}}}} 
\lhead{\firstmark}
\rhead{\botmark}

\subsection{\hspace{-0.5cm} {\Large \textcolor{darkblue}{\textbf{\ipa{ʈʂo˩}}}}\hspace{0.5cm}[\kern2pt{\textcolor{darkblue}{\textbf{\ipa{ʈʂo˩˥}}}}\kern2pt]} \hypertarget{t`s`o\string_B1}{}
\markboth{\textcolor{darkblue}{\textbf{\ipa{ʈʂo˩}}}}{}
\textcolor{teal}{\mytextsc{classificateur}} \hspace{4pt} Ton~: L.
\textcolor{PineGreen}{\selectlanguage{french}Classificateur des repas.} \zh{量词:饭(一顿)。}  ¶ \textcolor{darkblue}{\textbf{\ipa{ɖɯ˧-ʈʂo˩ tʰi˩-pæ˩ |}}} \textcolor{PineGreen}{\selectlanguage{french}servir un repas} \zh{摆饭,摆饭桌}  
 ¶ \textcolor{darkblue}{\textbf{\ipa{ʐo˩˥, | njɤ˧ ɖɯ˧-ʈʂo˩ pæ˩-bi˩!}}} \textcolor{PineGreen}{\selectlanguage{french}au déjeuner, je vais (vous) servir (tout le) repas!} \zh{我来管午饭这一顿!}  
 ¶ \textcolor{darkblue}{\textbf{\ipa{hɑ˧ ɖɯ˧-ʈʂo˩}}} \textcolor{PineGreen}{\selectlanguage{french}un repas} \zh{一顿饭}  

\lhead{\firstmark}
\rhead{\botmark}

\subsection{\hspace{-0.5cm} {\Large \textcolor{darkblue}{\textbf{\ipa{ʈʂo˩bo˩}}}}\hspace{0.5cm}[\kern2pt{\textcolor{darkblue}{\textbf{\ipa{ʈʂo˧bo˧}}}}\kern2pt]} \hypertarget{t`s`o\string_Bbo\string_B1}{}
\markboth{\textcolor{darkblue}{\textbf{\ipa{ʈʂo˩bo˩}}}}{}
\textcolor{teal}{\mytextsc{nom}} \hspace{4pt} Ton~: L.
\textcolor{PineGreen}{\selectlanguage{french}Mur en terre.} \zh{土墙。}  ¶ \textcolor{darkblue}{\textbf{\ipa{ʈʂo˩bo˩ ti˥}}} \textcolor{PineGreen}{\selectlanguage{french}construire un mur en terre (en comprimant la terre)} \zh{垒墙}  
 \zh{量词}: \textcolor{darkblue}{\textbf{\ipa{do˥}}}  \mytextsc{clf}~: \textcolor{darkblue}{\textbf{\ipa{do˥}}} 
\lhead{\firstmark}
\rhead{\botmark}

\subsection{\hspace{-0.5cm} {\Large \textcolor{darkblue}{\textbf{\ipa{ʈʂo˩mv̩˩}}}}\hspace{0.5cm}[\kern2pt{\textcolor{darkblue}{\textbf{\ipa{ʈʂo˩mv̩˩˥}}}}\kern2pt]} \hypertarget{t`s`o\string_Bmv\string_=\string_B1}{}
\markboth{\textcolor{darkblue}{\textbf{\ipa{ʈʂo˩mv̩˩}}}}{}
\textcolor{teal}{\mytextsc{nom}} \hspace{4pt} Ton~: L.
\textcolor{PineGreen}{\selectlanguage{french}Sable fin.} \zh{沙子。} 
\lhead{\firstmark}
\rhead{\botmark}

\subsection{\hspace{-0.5cm} {\Large \textcolor{darkblue}{\textbf{\ipa{ʈʂɻ̍˥}}}}\hspace{0.5cm}[\kern2pt{\textcolor{darkblue}{\textbf{\ipa{ʈʂɻ̍˥}}}}\kern2pt]} \hypertarget{t`s`r£`̍\string_T1}{}
\markboth{\textcolor{darkblue}{\textbf{\ipa{ʈʂɻ̍˥}}}}{}
\textcolor{teal}{\mytextsc{verbe}} \hspace{4pt} Ton~: H.
\textcolor{PineGreen}{\selectlanguage{french}Tousser.} \zh{咳嗽。}  ¶ \textcolor{darkblue}{\textbf{\ipa{ʈʂʰɯ˧ | tʰi˧-ʈʂɻ̍˥-dʑo˩}}} \textcolor{PineGreen}{\selectlanguage{french}il tousse} \zh{他在咳嗽}  

\lhead{\firstmark}
\rhead{\botmark}

\subsection{\hspace{-0.5cm} {\Large \textcolor{darkblue}{\textbf{\ipa{ʈʂɯ˧}}}}\hspace{0.5cm}[\kern2pt{\textcolor{darkblue}{\textbf{\ipa{ʈʂɯ˧˥}}}}\kern2pt]} \hypertarget{t`s`M\string_M1}{}
\markboth{\textcolor{darkblue}{\textbf{\ipa{ʈʂɯ˧}}}}{}
\textcolor{teal}{\mytextsc{nom}} \hspace{4pt} Ton~: M.
\textcolor{PineGreen}{\selectlanguage{french}Griffes (d'un animal); serres (d'un oiseau).} \zh{爪子。}  \zh{量词}: \textcolor{darkblue}{\textbf{\ipa{ɭɯ˧}}}  \mytextsc{clf}~: \textcolor{darkblue}{\textbf{\ipa{ɭɯ˧}}} \textit{Voir~:} \hyperlink{}{\textcolor{darkblue}{\textbf{\ipa{kv̩˧ʈʂɯ˧˥}}}} 
\lhead{\firstmark}
\rhead{\botmark}

\subsection{\hspace{-0.5cm} {\Large \textcolor{darkblue}{\textbf{\ipa{ʈʂɯ˧dzi˩}}}}\hspace{0.5cm}[\kern2pt{\textcolor{darkblue}{\textbf{\ipa{ʈʂɯ˩dzi˩˥}}}}\kern2pt]} \hypertarget{t`s`M\string_Mdzi\string_B1}{}
\markboth{\textcolor{darkblue}{\textbf{\ipa{ʈʂɯ˧dzi˩}}}}{}
\textcolor{teal}{\mytextsc{nom}} \hspace{4pt} Ton~: L\#.
\textcolor{PineGreen}{\selectlanguage{french}Arbre à vernis.} \zh{漆树。} 
\lhead{\firstmark}
\rhead{\botmark}

\subsection{\hspace{-0.5cm} {\Large \textcolor{darkblue}{\textbf{\ipa{ʈʂɯ˧fv̩\#˥}}}}\hspace{0.5cm}[\kern2pt{\textcolor{darkblue}{\textbf{\ipa{ʈʂɯ˧fv̩˩}}}}\kern2pt]} \hypertarget{t`s`M\string_Mfv\string_=\#\string_T1}{}
\markboth{\textcolor{darkblue}{\textbf{\ipa{ʈʂɯ˧fv̩\#˥}}}}{}
\textcolor{teal}{\mytextsc{nom}} \hspace{4pt} Ton~: \#H.
\textcolor{PineGreen}{\selectlanguage{french}Gouvernement local, pouvoir local.} \zh{(土)知府,如:永宁知府(汉语借词)。}  Emprunt~: chinois  \zh{知府}
 ¶ \textcolor{darkblue}{\textbf{\ipa{no˧ | ɬi˧di˩-ʈʂɯ˩fv̩˩-ni˩-zo˩!}}} \textcolor{PineGreen}{\selectlanguage{french}Tu te prends pour le seigneur! (critique qu'on s'adressait aux gens qui se mêlaient de dicter leur conduite aux autres, comme s'ils étaient les maîtres des lieux)} \zh{你像永宁土知府! / 你是永宁土知府吧!(批评独断的人、一手包办的人)}  
 ¶ \textcolor{darkblue}{\textbf{\ipa{no˧ | ʈʂɯ˧fv̩˧-mi˧-ni˧˥ | -zo˩!}}} \textcolor{PineGreen}{\selectlanguage{french}Tu joues les princesses! (littéralement “les femmes du pouvoir”) Critique adressée à une femme qui prend de grands airs.} \zh{你好像是永宁大公主! / 你好像是永宁知府女主人!(批评一个独断的女人)}  

\lhead{\firstmark}
\rhead{\botmark}

\subsection{\hspace{-0.5cm} {\Large \textcolor{darkblue}{\textbf{\ipa{ʈʂɯ˧mɤ˩}}}}\hspace{0.5cm}[\kern2pt{\textcolor{darkblue}{\textbf{\ipa{ʈʂɯ˧mɤ˥}}}}\kern2pt]} \hypertarget{t`s`M\string_Mm7\string_B1}{}
\markboth{\textcolor{darkblue}{\textbf{\ipa{ʈʂɯ˧mɤ˩}}}}{}
\textcolor{teal}{\mytextsc{nom}} \hspace{4pt} Ton~: L\#.
\textcolor{PineGreen}{\selectlanguage{french}Sésame.} \zh{芝麻(汉语借词)。}  Emprunt~: chinois  \zh{芝麻}
 ¶ \textcolor{darkblue}{\textbf{\ipa{ʈʂɯ˧mɤ˩, | ɬi˧di˩ | mɤ˧-tʰv̩˧-ɲi˥!}}} \textcolor{PineGreen}{\selectlanguage{french}Le sésame ne pousse pas à Yongning!} \zh{永宁不产芝麻!}  

\lhead{\firstmark}
\rhead{\botmark}

\subsection{\hspace{-0.5cm} {\Large \textcolor{darkblue}{\textbf{\ipa{ʈʂɯ˧˥}}}}\hspace{0.5cm}[\kern2pt{\textcolor{darkblue}{\textbf{\ipa{ʈʂɯ˥}}}}\kern2pt]} \hypertarget{t`s`M\string_M\string_T1}{}
\markboth{\textcolor{darkblue}{\textbf{\ipa{ʈʂɯ˧˥}}}}{}
\textcolor{teal}{\mytextsc{verbe}} \hspace{4pt} Ton~: MH.
\textcolor{PineGreen}{\selectlanguage{french}Tamiser.} \zh{筛。}  ¶ \textcolor{darkblue}{\textbf{\ipa{le˧-ʈʂɯ˧-ze˥}}} \textcolor{PineGreen}{\selectlanguage{french}\mytextsc{accomp} \string_ \mytextsc{pfv}} \zh{\mytextsc{accomp} \string_ \mytextsc{pfv}}  
 ¶ \textcolor{darkblue}{\textbf{\ipa{ɖɯ˧-ʈʂɯ˧-ɻ̍˥}}} \textcolor{PineGreen}{\selectlanguage{french}\mytextsc{délimitatif} \string_ \mytextsc{inchoatif}} \zh{\mytextsc{delimitative} \string_ \mytextsc{inceptive}}  

\lhead{\firstmark}
\rhead{\botmark}

\subsection{\hspace{-0.5cm} {\Large \textcolor{darkblue}{\textbf{\ipa{ʈʂv̩˩}}}}\hspace{0.5cm}[\kern2pt{\textcolor{darkblue}{\textbf{\ipa{ʈʂv̩˩˥}}}}\kern2pt]} \hypertarget{t`s`v\string_=\string_B1}{}
\markboth{\textcolor{darkblue}{\textbf{\ipa{ʈʂv̩˩}}}}{}
\textcolor{teal}{\mytextsc{adjectif}} \hspace{4pt} Ton~: L.
\textcolor{PineGreen}{\selectlanguage{french}Paisible, aimable, pacifique, doux (signe astrologique); s'emploie au sujet des signes astrologiques: certains sont considérés comme 'paisibles', comme le Boeuf, le Lapin et la Chèvre, ce qui rend les personnes nées cette année-là appropriées pour certains rites/certaines tâches (ex.: lors du rite de passage à l'âge adulte), et au contraire non appropriées pour d'autres.} \zh{平和的(生肖)。}  ¶ \textcolor{darkblue}{\textbf{\ipa{kʰv̩˧ ʈʂv̩˧˥}}} \textcolor{PineGreen}{\selectlanguage{french}signe pacifique, calme, non belliqueux} \zh{平和的生肖,如牛、兔、羊}  

\lhead{\firstmark}
\rhead{\botmark}

\subsection{\hspace{-0.5cm} {\Large \textcolor{darkblue}{\textbf{\ipa{ʈʂv̩˩˥}}}}\hspace{0.5cm}[\kern2pt{\textcolor{darkblue}{\textbf{\ipa{ʈʂv̩˩˥}}}}\kern2pt]} \hypertarget{t`s`v\string_=\string_B\string_T1}{}
\markboth{\textcolor{darkblue}{\textbf{\ipa{ʈʂv̩˩˥}}}}{}
\textcolor{teal}{\mytextsc{nom}} \hspace{4pt} Ton~: LH.
\textcolor{PineGreen}{\selectlanguage{french}Sueur (monosyllabe).} \zh{汗(单音节)。}  ¶ \textcolor{darkblue}{\textbf{\ipa{ʈʂv̩˧ bv̩˧nv̩˩}}} \textcolor{PineGreen}{\selectlanguage{french}qui sent la sueur, malodorant} \zh{有汗(臭)的味道}  

\lhead{\firstmark}
\rhead{\botmark}

\subsection{\hspace{-0.5cm} {\Large \textcolor{darkblue}{\textbf{\ipa{ʈʂv̩˩\textsubscript{a}}}} \textsubscript{1}}\hspace{0.5cm}[\kern2pt{\textcolor{darkblue}{\textbf{\ipa{ʈʂv̩˩˥}}}}\kern2pt]} \hypertarget{t`s`v\string_=\string_Ba1}{}
\markboth{\textcolor{darkblue}{\textbf{\ipa{ʈʂv̩˩\textsubscript{a}}}} \textsubscript{1}}{}
\textcolor{teal}{\mytextsc{verbe}} \hspace{4pt} Ton~: L\textsubscript{a}.
\textcolor{PineGreen}{\selectlanguage{french}Contaminer, infecter.} \zh{传染。}  ¶ \textcolor{darkblue}{\textbf{\ipa{hĩ˧ ʈʂv̩˥-ho˩}}} \textcolor{PineGreen}{\selectlanguage{french}(la maladie) va contaminer quelqu'un} \zh{(病毒)会传染人的}  
 ¶ \textcolor{darkblue}{\textbf{\ipa{ʈʂv̩˧\textasciitilde{}ʈʂv̩˥}}} \textcolor{PineGreen}{\selectlanguage{french}\mytextsc{red}} \zh{\mytextsc{重叠}}  
 ¶ \textcolor{darkblue}{\textbf{\ipa{ʈʂv̩˧\textasciitilde{}ʈʂv̩˥-ɻ̍˩ ho˩}}} \textcolor{PineGreen}{\selectlanguage{french}(la maladie) va contaminer (des gens)} \zh{(病毒)会传染的。}  
 ¶ \textcolor{darkblue}{\textbf{\ipa{njɤ˧-ɳɯ˧ | no˧ ʈʂv̩˧-ʝi˥!}}} \textcolor{PineGreen}{\selectlanguage{french}(Attention,) je vais te contaminer/te passer (mon rhume)!} \zh{(要小心:)我会传染你的!}  

\lhead{\firstmark}
\rhead{\botmark}

\subsection{\hspace{-0.5cm} {\Large \textcolor{darkblue}{\textbf{\ipa{ʈʂv̩˩\textsubscript{a}}}} \textsubscript{2}}\hspace{0.5cm}[\kern2pt{\textcolor{darkblue}{\textbf{\ipa{ʈʂv̩˩˥}}}}\kern2pt]} \hypertarget{t`s`v\string_=\string_Ba2}{}
\markboth{\textcolor{darkblue}{\textbf{\ipa{ʈʂv̩˩\textsubscript{a}}}} \textsubscript{2}}{}
\textcolor{teal}{\mytextsc{verbe}} \hspace{4pt} Ton~: L\textsubscript{a}.
\textcolor{PineGreen}{\selectlanguage{french}Allumer (une bougie).} \zh{点(蜡烛……)。} 
\lhead{\firstmark}
\rhead{\botmark}

\subsection{\hspace{-0.5cm} {\Large \textcolor{darkblue}{\textbf{\ipa{ʈʂv̩˧di˧˥}}}}\hspace{0.5cm}[\kern2pt{\textcolor{darkblue}{\textbf{\ipa{ʈʂv̩˧di˧˥}}}}\kern2pt]} \hypertarget{t`s`v\string_=\string_Mdi\string_M\string_T1}{}
\markboth{\textcolor{darkblue}{\textbf{\ipa{ʈʂv̩˧di˧˥}}}}{}
\textcolor{teal}{\mytextsc{nom}} \hspace{4pt} Ton~: MH\#.
\textcolor{PineGreen}{\selectlanguage{french}Village na hors de la plaine de Yongning, vers le Lac, non loin de \textcolor{darkblue}{\textbf{\ipa{/lɑ˧tʰɑ˧-di˧˥/}}}.} \zh{村落名。} 
\lhead{\firstmark}
\rhead{\botmark}

\subsection{\hspace{-0.5cm} {\Large \textcolor{darkblue}{\textbf{\ipa{ʈʂv̩˩dʑɯ˥}}}}\hspace{0.5cm}[\kern2pt{\textcolor{darkblue}{\textbf{\ipa{ʈʂv̩˩dʑɯ˥}}}}\kern2pt]} \hypertarget{t`s`v\string_=\string_Bdz£M\string_T1}{}
\markboth{\textcolor{darkblue}{\textbf{\ipa{ʈʂv̩˩dʑɯ˥}}}}{}
\textcolor{teal}{\mytextsc{nom}} \hspace{4pt} Ton~: LH.
\textcolor{PineGreen}{\selectlanguage{french}Sueur.} \zh{汗。} 
\lhead{\firstmark}
\rhead{\botmark}

\subsection{\hspace{-0.5cm} {\Large \textcolor{darkblue}{\textbf{\ipa{ʈʂv̩˧pɤ˩}}}}\hspace{0.5cm}[\kern2pt{\textcolor{darkblue}{\textbf{\ipa{ʈʂv̩˧pɤ˩}}}}\kern2pt]} \hypertarget{t`s`v\string_=\string_Mp7\string_B1}{}
\markboth{\textcolor{darkblue}{\textbf{\ipa{ʈʂv̩˧pɤ˩}}}}{}
\textcolor{teal}{\mytextsc{nom}} \hspace{4pt} Ton~: L\#.
\textcolor{PineGreen}{\selectlanguage{french}Planche à découper.} \zh{菜板、俎。}  \zh{量词}: \textcolor{darkblue}{\textbf{\ipa{nɑ˧}}}  \mytextsc{clf}~: \textcolor{darkblue}{\textbf{\ipa{nɑ˧}}} 
\lhead{\firstmark}
\rhead{\botmark}

\subsection{\hspace{-0.5cm} {\Large \textcolor{darkblue}{\textbf{\ipa{ʈʂv̩˧tɕɯ˥}}}}\hspace{0.5cm}[\kern2pt{\textcolor{darkblue}{\textbf{\ipa{ʈʂv̩˧tɕɯ˥}}}}\kern2pt]} \hypertarget{t`s`v\string_=\string_Mts£M\string_T1}{}
\markboth{\textcolor{darkblue}{\textbf{\ipa{ʈʂv̩˧tɕɯ˥}}}}{}
\textcolor{teal}{\mytextsc{nom}} \hspace{4pt} Ton~: H\#.
\textcolor{PineGreen}{\selectlanguage{french}Crachat.} \zh{痰。} 
\lhead{\firstmark}
\rhead{\botmark}

\subsection{\hspace{-0.5cm} {\Large \textcolor{darkblue}{\textbf{\ipa{ʈʂwɑ˧\textasciitilde{}ʈʂwɑ˧-nɑ˧\textasciitilde{}nɑ\#˥}}}}\hspace{0.5cm}[\kern2pt{\textcolor{darkblue}{\textbf{\ipa{xxxx non-correspondance entre le nombre de morphèmes et le nombre de tons de morphèmes}}}}\kern2pt]} \hypertarget{t`s`wA\string_M~t`s`wA\string_M-nA\string_M~nA\#\string_T1}{}
\markboth{\textcolor{darkblue}{\textbf{\ipa{ʈʂwɑ˧\textasciitilde{}ʈʂwɑ˧-nɑ˧\textasciitilde{}nɑ\#˥}}}}{}
\textcolor{teal}{\mytextsc{adjectif}} \hspace{4pt} Ton~: \#H.
\textcolor{PineGreen}{\selectlanguage{french}Divers, varié, désordonné.} \zh{杂、混杂。}  ¶ \textcolor{darkblue}{\textbf{\ipa{ʈʂwɑ˧\textasciitilde{}ʈʂwɑ˧-nɑ˧\textasciitilde{}nɑ˧-hĩ˥}}} \textcolor{PineGreen}{\selectlanguage{french}\mytextsc{rel}/nmlz} \zh{混杂的}  
 ¶ \textcolor{darkblue}{\textbf{\ipa{ʈʂwɑ˧\textasciitilde{}ʈʂwɑ˧-nɑ˧\textasciitilde{}nɑ˧-ɻ̍˥}}} \textcolor{PineGreen}{\selectlanguage{french}désordonné} \zh{混杂}  

\lhead{\firstmark}
\rhead{\botmark}

\subsection{\hspace{-0.5cm} {\Large \textcolor{darkblue}{\textbf{\ipa{ʈʂwæ˥\textsubscript{a}}}}}\hspace{0.5cm}[\kern2pt{\textcolor{darkblue}{\textbf{\ipa{ʈʂwæ˧˥}}}}\kern2pt]} \hypertarget{t`s`w\{\string_Ta1}{}
\markboth{\textcolor{darkblue}{\textbf{\ipa{ʈʂwæ˥\textsubscript{a}}}}}{}
\textcolor{teal}{\mytextsc{classificateur}} \hspace{4pt} Ton~: H\textsubscript{a}.
\textcolor{PineGreen}{\selectlanguage{french}Classificateur des trajets.} \zh{量词:征途、路程、路途、征程,趟。}  ¶ \textcolor{darkblue}{\textbf{\ipa{ɖɯ˧-ɲi˥ | ɖɯ˧-ʈʂwæ˧ bi˧}}} \textcolor{PineGreen}{\selectlanguage{french}(y) aller une fois par jour} \zh{一天去一趟}  

\lhead{\firstmark}
\rhead{\botmark}

\subsection{\hspace{-0.5cm} {\Large \textcolor{darkblue}{\textbf{\ipa{ʈʂwæ˧tʰo˩}}}}\hspace{0.5cm}[\kern2pt{\textcolor{darkblue}{\textbf{\ipa{ʈʂwæ˧tʰo˧}}}}\kern2pt]} \hypertarget{t`s`w\{\string_Mt\string_ho\string_B1}{}
\markboth{\textcolor{darkblue}{\textbf{\ipa{ʈʂwæ˧tʰo˩}}}}{}
\textcolor{teal}{\mytextsc{nom}} \hspace{4pt} Ton~: L\#.
\textcolor{PineGreen}{\selectlanguage{french}Brique.} \zh{砖头(汉语借词)。}  Emprunt~: chinois  \zh{砖头}
\textit{Voir~:} \hyperlink{}{\textcolor{darkblue}{\textbf{\ipa{tʰo˩tɕi˧˥}}}} 
\lhead{\firstmark}
\rhead{\botmark}

\subsection{\hspace{-0.5cm} {\Large \textcolor{darkblue}{\textbf{\ipa{ʈʂwæ˧\textasciitilde{}ʈʂwæ˧}}}}\hspace{0.5cm}[\kern2pt{\textcolor{darkblue}{\textbf{\ipa{ʈʂwæ˧ʈʂwæ˧˥}}}}\kern2pt]} \hypertarget{t`s`w\{\string_M~t`s`w\{\string_M1}{}
\markboth{\textcolor{darkblue}{\textbf{\ipa{ʈʂwæ˧\textasciitilde{}ʈʂwæ˧}}}}{}
\textcolor{teal}{\mytextsc{verbe}} \hspace{4pt} Ton~: M.
\textcolor{PineGreen}{\selectlanguage{french}Mélanger.} \zh{搅拌、使混合。} 
\lhead{\firstmark}
\rhead{\botmark}

\subsection{\hspace{-0.5cm} {\Large \textcolor{darkblue}{\textbf{\ipa{ʈʂwæ˩ho˧ɻ̍˧}}}}\hspace{0.5cm}[\kern2pt{\textcolor{darkblue}{\textbf{\ipa{ʈʂwæ˧ho˧ɻ̍˧}}}}\kern2pt]} \hypertarget{t`s`w\{\string_Bho\string_Mr£`̍\string_M1}{}
\markboth{\textcolor{darkblue}{\textbf{\ipa{ʈʂwæ˩ho˧ɻ̍˧}}}}{}
\textcolor{teal}{\mytextsc{nom}} \hspace{4pt} Ton~: LM.
\textcolor{PineGreen}{\selectlanguage{french}Perceuse.} \zh{钻子。}  Emprunt~: chinois  \zh{钻}

\lhead{\firstmark}
\rhead{\botmark}

\subsection{\hspace{-0.5cm} {\Large \textcolor{darkblue}{\textbf{\ipa{ʈʂwæ˩\textasciitilde{}ʈʂwæ˧˥}}}}\hspace{0.5cm}[\kern2pt{\textcolor{darkblue}{\textbf{\ipa{ʈʂwæ˧ʈʂwæ˧}}}}\kern2pt]} \hypertarget{t`s`w\{\string_B~t`s`w\{\string_M\string_T1}{}
\markboth{\textcolor{darkblue}{\textbf{\ipa{ʈʂwæ˩\textasciitilde{}ʈʂwæ˧˥}}}}{}
\textcolor{teal}{\mytextsc{verbe}} \hspace{4pt} Ton~: MH.
\textcolor{PineGreen}{\selectlanguage{french}Se tenir par la main, se tenir la main.} \zh{手拉手。}  ¶ \textcolor{darkblue}{\textbf{\ipa{le˧-ʈʂwæ˧\textasciitilde{}ʈʂwæ˧-ze˧!}}} \textcolor{PineGreen}{\selectlanguage{french}\mytextsc{accomp} \mytextsc{red} \mytextsc{pfv}} \zh{\mytextsc{accomp} \mytextsc{red} \mytextsc{pfv}}  
 ¶ \textcolor{darkblue}{\textbf{\ipa{ʈʂwæ˩\textasciitilde{}ʈʂwæ˧-ɻ̍˥}}} \textcolor{PineGreen}{\selectlanguage{french}\mytextsc{red} \mytextsc{inchoatif;} même sens: se tenir par la main} \zh{\mytextsc{red} \mytextsc{inceptive}}  

\lhead{\firstmark}
\rhead{\botmark}

\subsection{\hspace{-0.5cm} {\Large \textcolor{darkblue}{\textbf{\ipa{ʈʂwæ˧˥}}} \textsubscript{1}}\hspace{0.5cm}[\kern2pt{\textcolor{darkblue}{\textbf{\ipa{ʈʂwæ˧˥}}}}\kern2pt]} \hypertarget{t`s`w\{\string_M\string_T1}{}
\markboth{\textcolor{darkblue}{\textbf{\ipa{ʈʂwæ˧˥}}} \textsubscript{1}}{}
\textcolor{teal}{\mytextsc{verbe}} \hspace{4pt} Ton~: MH.
 Emprunt~: chinois  \zh{装?}
\ding{202} \textcolor{PineGreen}{\selectlanguage{french}Installer.} \zh{安装。}  ¶ \textcolor{darkblue}{\textbf{\ipa{tjɤ˧hwɑ˧ ʈʂwæ˥}}} \textcolor{PineGreen}{\selectlanguage{french}installer le téléphone (dans une demeure qui n'y était pas reliée précédemment)} \zh{安装电话(座机)}  
 ¶ \textcolor{darkblue}{\textbf{\ipa{le˧-ʈʂwæ˧˥ le˧-tse˧-ze˧!}}} \textcolor{PineGreen}{\selectlanguage{french}C'est bien installé!} \zh{装好了!}  
\ding{203} \textcolor{PineGreen}{\selectlanguage{french}Soigner, réparer (une dent).} \zh{补(牙)、修好(坏牙)。}  ¶ \textcolor{darkblue}{\textbf{\ipa{hi˧ ʈʂwæ˩}}} \textcolor{PineGreen}{\selectlanguage{french}soigner une dent; littéralement “remettre une dent”} \zh{补牙、修好坏牙}  
 ¶ \textcolor{darkblue}{\textbf{\ipa{hi˧ | le˧-ʈʂwæ˧-ze˥!}}} \textcolor{PineGreen}{\selectlanguage{french}La dent est soignée! / La dent est réparée!} \zh{牙补好了!}  
\ding{204} \textcolor{PineGreen}{\selectlanguage{french}Nouer (des fils).} \zh{结线。}  ¶ \textcolor{darkblue}{\textbf{\ipa{kʰɯ˩ ʈʂwæ˩˥}}} \textcolor{PineGreen}{\selectlanguage{french}attacher des brins de fil ensemble, nouer des fils (par ex. lorsqu'on prépare le métier à tisser)} \zh{结线}  

\lhead{\firstmark}
\rhead{\botmark}

\subsection{\hspace{-0.5cm} {\Large \textcolor{darkblue}{\textbf{\ipa{ʈʂwæ˧˥}}} \textsubscript{2}}\hspace{0.5cm}[\kern2pt{\textcolor{darkblue}{\textbf{\ipa{ʈʂwæ˧˥}}}}\kern2pt]} \hypertarget{t`s`w\{\string_M\string_T2}{}
\markboth{\textcolor{darkblue}{\textbf{\ipa{ʈʂwæ˧˥}}} \textsubscript{2}}{}
\textcolor{teal}{\mytextsc{verbe}} \hspace{4pt} Ton~: MH.
\textcolor{PineGreen}{\selectlanguage{french}Savourer, déguster, siroter (nourriture ou boisson).} \zh{欣赏、品尝(饮食、音乐……)。}  ¶ \textcolor{darkblue}{\textbf{\ipa{no˧ | li˩ ʈʂwæ˧-ɻ̍˥! |}}} \textcolor{PineGreen}{\selectlanguage{french}Veuillez prendre un peu de thé! (Invitation polie)} \zh{请您品一点茶!(礼貌说法)}  
 ¶ \textcolor{darkblue}{\textbf{\ipa{ʐɯ˧ F | ʈʂwæ˧˥! | li˩˥ F | ʈʂwæ˧˥! hɑ˧ F | ʈʂwæ˧˥!}}} \textcolor{PineGreen}{\selectlanguage{french}L'alcool, ça se savoure; le thé, ça se savoure! (Explication au sujet des emplois du verbe)} \zh{酒,是可以品尝的!茶,是可以品尝的!饭,是可以品尝的!(关于‘品尝’这个动词的说明)}  
 ¶ \textcolor{darkblue}{\textbf{\ipa{hɑ˧ ʈʂwæ˩}}} \textcolor{PineGreen}{\selectlanguage{french}savourer de la nourriture} \zh{品尝食物}  
 ¶ \textcolor{darkblue}{\textbf{\ipa{li˩ ʈʂwæ˧˥}}} \textcolor{PineGreen}{\selectlanguage{french}savourer du thé} \zh{品茶}  
 ¶ \textcolor{darkblue}{\textbf{\ipa{ʐɯ˧ ʈʂwæ˧˥}}} \textcolor{PineGreen}{\selectlanguage{french}savourer de l'alcool} \zh{品酒}  
 ¶ \textcolor{darkblue}{\textbf{\ipa{ə˩kʰɯ˩ ʈʂwæ˥}}} \textcolor{PineGreen}{\selectlanguage{french}savourer des feuilles de navet (formulation ironique mais tout à fait acceptable)} \zh{尝尝圆根(玩笑话,因为圆根没有什么滋味)}  

\lhead{\firstmark}
\rhead{\botmark}

\subsection{\hspace{-0.5cm} {\Large \textcolor{darkblue}{\textbf{\ipa{ʈʂwɤ˧\textsubscript{a}}}}}\hspace{0.5cm}[\kern2pt{\textcolor{darkblue}{\textbf{\ipa{ʈʂwɤ˥}}}}\kern2pt]} \hypertarget{t`s`w7\string_Ma1}{}
\markboth{\textcolor{darkblue}{\textbf{\ipa{ʈʂwɤ˧\textsubscript{a}}}}}{}
\textcolor{teal}{\mytextsc{classificateur}} \hspace{4pt} Ton~: M\textsubscript{a}.
\textcolor{PineGreen}{\selectlanguage{french}Classificateur des poignées: ce que l'on peut prendre dans une seule main.} \zh{量词:捧。} 
\lhead{\firstmark}
\rhead{\botmark}

\subsection{\hspace{-0.5cm} {\Large \textcolor{darkblue}{\textbf{\ipa{ʈʂwɤ˧\textsubscript{a}}}}}\hspace{0.5cm}[\kern2pt{\textcolor{darkblue}{\textbf{\ipa{ʈʂwɤ˩˥}}}}\kern2pt]} \hypertarget{t`s`w7\string_Ma1}{}
\markboth{\textcolor{darkblue}{\textbf{\ipa{ʈʂwɤ˧\textsubscript{a}}}}}{}
\textcolor{teal}{\mytextsc{verbe}} \hspace{4pt} Ton~: M\textsubscript{a}.
\textcolor{PineGreen}{\selectlanguage{french}Griffer (ex.: un tigre griffe).} \zh{抓(用爪子抓)。}  ¶ \textcolor{darkblue}{\textbf{\ipa{tso˧\textasciitilde{}tso˧ ʈʂwɤ˩}}} \textcolor{PineGreen}{\selectlanguage{french}griffer des objets} \zh{抓东西}  

\lhead{\firstmark}
\rhead{\botmark}

\subsection{\hspace{-0.5cm} {\Large \textcolor{darkblue}{\textbf{\ipa{ʈʂwɤ˧\textasciitilde{}ʈʂwɤ˩}}}}\hspace{0.5cm}[\kern2pt{\textcolor{darkblue}{\textbf{\ipa{ʈʂwɤ˧ʈʂwɤ˧˥}}}}\kern2pt]} \hypertarget{t`s`w7\string_M~t`s`w7\string_B1}{}
\markboth{\textcolor{darkblue}{\textbf{\ipa{ʈʂwɤ˧\textasciitilde{}ʈʂwɤ˩}}}}{}
\textcolor{teal}{\mytextsc{verbe}} \hspace{4pt} Ton~: M.
\textcolor{PineGreen}{\selectlanguage{french}Toucher.} \zh{触碰。}  ¶ \textcolor{darkblue}{\textbf{\ipa{ə˧tso˧ mɤ˧-ɲi˩ ʈʂwɤ˧\textasciitilde{}ʈʂwɤ˩!}}} \textcolor{PineGreen}{\selectlanguage{french}(tu) touches vraiment à tout! (doux reproche adressé à un bébé qui se promène sur une table et s'empare de tout ce qui s'y trouve)} \zh{你什么都碰,是吗!(小孩爬在桌子上,试着拿每个东西)}  

\lhead{\firstmark}
\rhead{\botmark}

\subsection{\hspace{-0.5cm} {\Large \textcolor{darkblue}{\textbf{\ipa{ʈʂʰɑ˧lɑ˧}}}}\hspace{0.5cm}[\kern2pt{\textcolor{darkblue}{\textbf{\ipa{ʈʂʰɑ˧lɑ˧}}}}\kern2pt]} \hypertarget{t`s`\string_hA\string_MlA\string_M1}{}
\markboth{\textcolor{darkblue}{\textbf{\ipa{ʈʂʰɑ˧lɑ˧}}}}{}
\textcolor{teal}{\mytextsc{verbe}} \hspace{4pt} Ton~: M.
\textcolor{PineGreen}{\selectlanguage{french}Discuter, bavarder.} \zh{商量、交谈、谈天、聊天。}  ¶ \textcolor{darkblue}{\textbf{\ipa{hĩ˧-qɑ˩ ʈʂʰɑ˩lɑ˩}}} \textcolor{PineGreen}{\selectlanguage{french}bavarder avec quelqu'un} \zh{跟人聊天}  
 ¶ \textcolor{darkblue}{\textbf{\ipa{ɖɯ˧-kʰwɤ˧ ʈʂʰɑ˧lɑ˥}}} \textcolor{PineGreen}{\selectlanguage{french}bavarder un peu, avoir une causerie avec} \zh{聊聊天}  
 ¶ \textcolor{darkblue}{\textbf{\ipa{njɤ˧ | no˧-qɑ˧ ʈʂʰɑ˧lɑ˥}}} \textcolor{PineGreen}{\selectlanguage{french}je te raconte, je te dis} \zh{我给你讲、我跟你聊聊天}  

\lhead{\firstmark}
\rhead{\botmark}

\subsection{\hspace{-0.5cm} {\Large \textcolor{darkblue}{\textbf{\ipa{ʈʂʰɑ˧lɑ˧-mv̩˧lɑ˩}}}}\hspace{0.5cm}[\kern2pt{\textcolor{darkblue}{\textbf{\ipa{xxxx non-correspondance entre le nombre de morphèmes et le nombre de tons de morphèmes}}}}\kern2pt]} \hypertarget{t`s`\string_hA\string_MlA\string_M-mv\string_=\string_MlA\string_B1}{}
\markboth{\textcolor{darkblue}{\textbf{\ipa{ʈʂʰɑ˧lɑ˧-mv̩˧lɑ˩}}}}{}
\textcolor{teal}{\mytextsc{verbe}} \hspace{4pt} Ton~: M.
\textcolor{PineGreen}{\selectlanguage{french}Discuter, bavarder.} \zh{商量、交谈、谈天、聊天。}  ¶ \textcolor{darkblue}{\textbf{\ipa{ʈʂʰɑ˧lɑ˧-mv̩˧lɑ˩-ɻ̍˩}}} \textcolor{PineGreen}{\selectlanguage{french}bavarder un peu, avoir une causerie} \zh{聊聊天}  

\lhead{\firstmark}
\rhead{\botmark}

\subsection{\hspace{-0.5cm} {\Large \textcolor{darkblue}{\textbf{\ipa{ʈʂʰɑ˧nɑ˥}}}}\hspace{0.5cm}[\kern2pt{\textcolor{darkblue}{\textbf{\ipa{ʈʂʰɑ˧nɑ˥}}}}\kern2pt]} \hypertarget{t`s`\string_hA\string_MnA\string_T1}{}
\markboth{\textcolor{darkblue}{\textbf{\ipa{ʈʂʰɑ˧nɑ˥}}}}{}
\textcolor{teal}{\mytextsc{nom}} \hspace{4pt} Ton~: H\#.
\textcolor{PineGreen}{\selectlanguage{french}Nom d'une source sacrée, située au pied d'une falaise, sur la montagne \textcolor{darkblue}{\textbf{\ipa{/qv̩˧ɻ\#˥/;}}} on disait que son eau sortait du ventre de la montagne. Le récit DumbChildren raconte comment son eau était utilisée comme remède de fertilité.} \zh{一眼山泉的名字。}  ¶ \textcolor{darkblue}{\textbf{\ipa{qv̩˧ɻ̍˧-ʈʂʰɑ˧nɑ˥\#}}} \textcolor{PineGreen}{\selectlanguage{french}nom complet de la montagne} \zh{山的全名,包括水泉名}  
 ¶ \textcolor{darkblue}{\textbf{\ipa{kɤ˧mv̩˧˥, | æ˧ʂæ˧, | ŋwɤ˧hɑ̃˩, | ʂwæ˧gv̩\#˥, | nɑ˩tsʰi˩˥ | -tɕʰɤ˧pɤ˧mi\#˥, | qv̩˧ɻ̍˧-ʈʂʰɑ˧nɑ˥ |}}} \textcolor{PineGreen}{\selectlanguage{french}Les six montagnes de Yongning qui portent un nom. Les autres sommets du voisinage n'ont pas une valeur symbolique comparable, et ne portent pas de nom communément utilisé.} \zh{永宁地区有固定名字的六座山。其它的山,因为没有重要的象征意义,因此没有取名。}  

\lhead{\firstmark}
\rhead{\botmark}

\subsection{\hspace{-0.5cm} {\Large \textcolor{darkblue}{\textbf{\ipa{ʈʂʰæ˥}}}}\hspace{0.5cm}[\kern2pt{\textcolor{darkblue}{\textbf{\ipa{ʈʂʰæ˧˥}}}}\kern2pt]} \hypertarget{t`s`\string_h\{\string_T1}{}
\markboth{\textcolor{darkblue}{\textbf{\ipa{ʈʂʰæ˥}}}}{}
\textcolor{teal}{\mytextsc{verbe}} \hspace{4pt} Ton~: H.
\textcolor{PineGreen}{\selectlanguage{french}Laver (les habits, la vaisselle…), rincer (le riz…).} \zh{洗(洗衣服,洗澡……)。}  ¶ \textcolor{darkblue}{\textbf{\ipa{dʑi˧hṽ˧ ʈʂʰæ˧}}} \textcolor{PineGreen}{\selectlanguage{french}laver des vêtements} \zh{洗衣服}  
 ¶ \textcolor{darkblue}{\textbf{\ipa{bɑ˩lɑ˩ ʈʂʰæ˩˥}}} \textcolor{PineGreen}{\selectlanguage{french}laver des chemises} \zh{洗上衣}  
 ¶ \textcolor{darkblue}{\textbf{\ipa{ɬi˧qʰwɤ˩ ʈʂʰæ˩}}} \textcolor{PineGreen}{\selectlanguage{french}laver des pantalons} \zh{洗裤子}  
 ¶ \textcolor{darkblue}{\textbf{\ipa{gv̩˧mi˧ ʈʂʰæ˧}}} \textcolor{PineGreen}{\selectlanguage{french}se laver, prendre un bain/une douche} \zh{洗澡}  
 ¶ \textcolor{darkblue}{\textbf{\ipa{gv̩˧mi˧ ʈʂʰæ˧\textasciitilde{}ʈʂʰæ˧}}} \textcolor{PineGreen}{\selectlanguage{french}se laver un coup} \zh{洗一下身体}  
 ¶ \textcolor{darkblue}{\textbf{\ipa{hɑ˧ ʈʂʰæ˧}}} \textcolor{PineGreen}{\selectlanguage{french}rincer une céréale (avant de la cuire)} \zh{淘洗粮食}  
 ¶ \textcolor{darkblue}{\textbf{\ipa{ɕi˧ʈʂʰwæ˧ ʈʂʰæ˧(-ze˩)}}} \textcolor{PineGreen}{\selectlanguage{french}rincer le riz (avant de le cuire)} \zh{淘米}  

\lhead{\firstmark}
\rhead{\botmark}

\subsection{\hspace{-0.5cm} {\Large \textcolor{darkblue}{\textbf{\ipa{ʈʂʰæ˧ɣɯ\#˥}}}}\hspace{0.5cm}[\kern2pt{\textcolor{darkblue}{\textbf{\ipa{ʈʂʰæ˧ɣɯ˧}}}}\kern2pt]} \hypertarget{t`s`\string_h\{\string_MGM\#\string_T1}{}
\markboth{\textcolor{darkblue}{\textbf{\ipa{ʈʂʰæ˧ɣɯ\#˥}}}}{}
\textcolor{teal}{\mytextsc{nom}} \hspace{4pt} Ton~: \#H.
\textcolor{PineGreen}{\selectlanguage{french}Médicament.} \zh{药。}  ¶ \textcolor{darkblue}{\textbf{\ipa{ʈʂʰæ˧ɣɯ˧ ʈʰɯ˧˥}}} \textcolor{PineGreen}{\selectlanguage{french}Prendre un médicament. Littéralement: “boire un médicament” (collocation différente du chinois \zh{吃药} “manger un médicament”)} \zh{吃药(直译:“喝药”)}  
 ¶ \textcolor{darkblue}{\textbf{\ipa{ʈʂʰæ˧ɣɯ˧ lɑ˩}}} \textcolor{PineGreen}{\selectlanguage{french}répandre des pesticides, traiter (un verger, un potager, un champ…)} \zh{打农药}  

\lhead{\firstmark}
\rhead{\botmark}

\subsection{\hspace{-0.5cm} {\Large \textcolor{darkblue}{\textbf{\ipa{ʈʂʰæ˧ɣɯ˧-ki˩-hĩ˩-hĩ˩}}}}\hspace{0.5cm}[\kern2pt{\textcolor{darkblue}{\textbf{\ipa{xxxx non-correspondance entre le nombre de morphèmes et le nombre de tons de morphèmes}}}}\kern2pt]} \hypertarget{t`s`\string_h\{\string_MGM\string_M-ki\string_B-hi\string_~\string_B-hi\string_~\string_B1}{}
\markboth{\textcolor{darkblue}{\textbf{\ipa{ʈʂʰæ˧ɣɯ˧-ki˩-hĩ˩-hĩ˩}}}}{}
\textcolor{teal}{\mytextsc{nom}} \hspace{4pt} Ton~: \mytextsc{L}.
\textcolor{PineGreen}{\selectlanguage{french}Médecin, docteur; littéralement: “personne qui donne des médicaments”.} \zh{医生。}  \zh{量词}: \textcolor{darkblue}{\textbf{\ipa{v̩˧}}}  \mytextsc{clf}~: \textcolor{darkblue}{\textbf{\ipa{v̩˧}}} 
\lhead{\firstmark}
\rhead{\botmark}

\subsection{\hspace{-0.5cm} {\Large \textcolor{darkblue}{\textbf{\ipa{ʈʂʰæ˧mi˥\$}}}}\hspace{0.5cm}[\kern2pt{\textcolor{darkblue}{\textbf{\ipa{ʈʂʰæ˧mi˥}}}}\kern2pt]} \hypertarget{t`s`\string_h\{\string_Mmi\string_T\$1}{}
\markboth{\textcolor{darkblue}{\textbf{\ipa{ʈʂʰæ˧mi˥\$}}}}{}
\textcolor{teal}{\mytextsc{nom}} \hspace{4pt} Ton~: H\$.
\textcolor{PineGreen}{\selectlanguage{french}Biche.} \zh{母马鹿。}  \zh{量词}: \textcolor{darkblue}{\textbf{\ipa{pʰo˧˥}}}  \mytextsc{clf}~: \textcolor{darkblue}{\textbf{\ipa{pʰo˧˥}}} 
\lhead{\firstmark}
\rhead{\botmark}

\subsection{\hspace{-0.5cm} {\Large \textcolor{darkblue}{\textbf{\ipa{ʈʂʰæ˧nɑ˥}}}}\hspace{0.5cm}[\kern2pt{\textcolor{darkblue}{\textbf{\ipa{ʈʂʰæ˧nɑ˥}}}}\kern2pt]} \hypertarget{t`s`\string_h\{\string_MnA\string_T1}{}
\markboth{\textcolor{darkblue}{\textbf{\ipa{ʈʂʰæ˧nɑ˥}}}}{}
\textcolor{teal}{\mytextsc{nom}} \hspace{4pt} Ton~: H\#.
\textcolor{PineGreen}{\selectlanguage{french}Cerf noir: espèce légendaire, que seuls les esprits sont à même de chasser et abattre.} \zh{黑鹿。}  \zh{量词}: \textcolor{darkblue}{\textbf{\ipa{pʰo˧˥}}}  \mytextsc{clf}~: \textcolor{darkblue}{\textbf{\ipa{pʰo˧˥}}} 
\lhead{\firstmark}
\rhead{\botmark}

\subsection{\hspace{-0.5cm} {\Large \textcolor{darkblue}{\textbf{\ipa{ʈʂʰæ˧pʰv̩\#˥}}}}\hspace{0.5cm}[\kern2pt{\textcolor{darkblue}{\textbf{\ipa{ʈʂʰæ˧pʰv̩˧}}}}\kern2pt]} \hypertarget{t`s`\string_h\{\string_Mp\string_hv\string_=\#\string_T1}{}
\markboth{\textcolor{darkblue}{\textbf{\ipa{ʈʂʰæ˧pʰv̩\#˥}}}}{}
\textcolor{teal}{\mytextsc{nom}} \hspace{4pt} Ton~: \#H.
\textcolor{PineGreen}{\selectlanguage{french}Cerf (mâle).} \zh{公马鹿。}  \zh{量词}: \textcolor{darkblue}{\textbf{\ipa{pʰo˧˥}}}  \mytextsc{clf}~: \textcolor{darkblue}{\textbf{\ipa{pʰo˧˥}}} 
\lhead{\firstmark}
\rhead{\botmark}

\subsection{\hspace{-0.5cm} {\Large \textcolor{darkblue}{\textbf{\ipa{ʈʂʰæ˧qʰv̩˥\$}}}}\hspace{0.5cm}[\kern2pt{\textcolor{darkblue}{\textbf{\ipa{ʈʂʰæ˧qʰv̩˥}}}}\kern2pt]} \hypertarget{t`s`\string_h\{\string_Mq\string_hv\string_=\string_T\$1}{}
\markboth{\textcolor{darkblue}{\textbf{\ipa{ʈʂʰæ˧qʰv̩˥\$}}}}{}
\textcolor{teal}{\mytextsc{nom}} \hspace{4pt} Ton~: H\$.
\textcolor{PineGreen}{\selectlanguage{french}Bois d'un cerf (même mot pour les bois d'un jeune cerf, utilisés comme aphrodisiaque en médecine traditionnelle).} \zh{鹿角,鹿茸。}  \zh{量词}: \textcolor{darkblue}{\textbf{\ipa{ɭɯ˧}}}  \mytextsc{clf}~: \textcolor{darkblue}{\textbf{\ipa{ɭɯ˧}}} 
\lhead{\firstmark}
\rhead{\botmark}

\subsection{\hspace{-0.5cm} {\Large \textcolor{darkblue}{\textbf{\ipa{ʈʂʰæ˧\textasciitilde{}ʈʂʰæ˧}}}}\hspace{0.5cm}[\kern2pt{\textcolor{darkblue}{\textbf{\ipa{ʈʂʰæ˧ʈʂʰæ˧}}}}\kern2pt]} \hypertarget{t`s`\string_h\{\string_M~t`s`\string_h\{\string_M1}{}
\markboth{\textcolor{darkblue}{\textbf{\ipa{ʈʂʰæ˧\textasciitilde{}ʈʂʰæ˧}}}}{}
\textcolor{teal}{\mytextsc{adjectif}} \hspace{4pt} Ton~: M.
\textcolor{PineGreen}{\selectlanguage{french}Solide, de bonne qualité, résistant (vêtement, outil, objet...).} \zh{结实、质量好,(东西)耐用,(人)可靠。} 
\lhead{\firstmark}
\rhead{\botmark}

\subsection{\hspace{-0.5cm} {\Large \textcolor{darkblue}{\textbf{\ipa{ʈʂʰæ˧zo\#˥}}}}\hspace{0.5cm}[\kern2pt{\textcolor{darkblue}{\textbf{\ipa{ʈʂʰæ˧zo˧}}}}\kern2pt]} \hypertarget{t`s`\string_h\{\string_Mzo\#\string_T1}{}
\markboth{\textcolor{darkblue}{\textbf{\ipa{ʈʂʰæ˧zo\#˥}}}}{}
\textcolor{teal}{\mytextsc{nom}} \hspace{4pt} Ton~: \#H.
\textcolor{PineGreen}{\selectlanguage{french}Faon.} \zh{小鹿。}  \zh{量词}: \textcolor{darkblue}{\textbf{\ipa{ɭɯ˧}}}  \mytextsc{clf}~: \textcolor{darkblue}{\textbf{\ipa{ɭɯ˧}}} 
\lhead{\firstmark}
\rhead{\botmark}

\subsection{\hspace{-0.5cm} {\Large \textcolor{darkblue}{\textbf{\ipa{ʈʂʰæ˧˥}}} \textsubscript{1}}\hspace{0.5cm}[\kern2pt{\textcolor{darkblue}{\textbf{\ipa{ʈʂʰæ˧˥}}}}\kern2pt]} \hypertarget{t`s`\string_h\{\string_M\string_T1}{}
\markboth{\textcolor{darkblue}{\textbf{\ipa{ʈʂʰæ˧˥}}} \textsubscript{1}}{}
\textcolor{teal}{\mytextsc{nom}} \hspace{4pt} Ton~: MH.
\textcolor{PineGreen}{\selectlanguage{french}Cerf, \textit{Cervus elaphus kansuensis}.} \zh{马鹿。}  \zh{量词}: \textcolor{darkblue}{\textbf{\ipa{pʰo˧˥}}}  \mytextsc{clf}~: \textcolor{darkblue}{\textbf{\ipa{pʰo˧˥}}} 
\lhead{\firstmark}
\rhead{\botmark}

\subsection{\hspace{-0.5cm} {\Large \textcolor{darkblue}{\textbf{\ipa{ʈʂʰæ˧˥}}} \textsubscript{2}}\hspace{0.5cm}[\kern2pt{\textcolor{darkblue}{\textbf{\ipa{ʈʂʰæ˧˥}}}}\kern2pt]} \hypertarget{t`s`\string_h\{\string_M\string_T2}{}
\markboth{\textcolor{darkblue}{\textbf{\ipa{ʈʂʰæ˧˥}}} \textsubscript{2}}{}
\textcolor{teal}{\mytextsc{classificateur}} \hspace{4pt} Ton~: MH.
\textcolor{PineGreen}{\selectlanguage{french}Classificateur des générations.} \zh{量词:代、世、辈、世代。} 
\lhead{\firstmark}
\rhead{\botmark}

\subsection{\hspace{-0.5cm} {\Large \textcolor{darkblue}{\textbf{\ipa{ʈʂʰe˧\textsubscript{b}}}}}\hspace{0.5cm}[\kern2pt{\textcolor{darkblue}{\textbf{\ipa{ʈʂʰe˥}}}}\kern2pt]} \hypertarget{t`s`\string_he\string_Mb1}{}
\markboth{\textcolor{darkblue}{\textbf{\ipa{ʈʂʰe˧\textsubscript{b}}}}}{}
\textcolor{teal}{\mytextsc{verbe}} \hspace{4pt} Ton~: M\textsubscript{b}.
\textcolor{PineGreen}{\selectlanguage{french}Tendre, étendre (la main).} \zh{伸(伸手)。}  ¶ \textcolor{darkblue}{\textbf{\ipa{le˧-ʈʂʰe˧-ze˧}}} \textcolor{PineGreen}{\selectlanguage{french}\mytextsc{accomp} \string_ \mytextsc{pfv}} \zh{\mytextsc{accomp} \string_ \mytextsc{pfv}}  
 ¶ \textcolor{darkblue}{\textbf{\ipa{mv̩˩tɕo˧ ʈʂʰe˧}}} \textcolor{PineGreen}{\selectlanguage{french}étendre vers le bas} \zh{向下伸展}  
 ¶ \textcolor{darkblue}{\textbf{\ipa{lo˩qʰwɤ˧ | ə˩pʰo˩ ʈʂʰe˩˥}}} \textcolor{PineGreen}{\selectlanguage{french}étendre la main à l'extérieur (par une fenêtre)} \zh{手伸到外边}  
 ¶ \textcolor{darkblue}{\textbf{\ipa{tso˧\textasciitilde{}tso˧ ʈʂʰe˧}}} \textcolor{PineGreen}{\selectlanguage{french}étendre quelque chose: par exemple, faire sortir un bâton par une fenêtre} \zh{伸出一个东西,如:从车窗里伸出一个棍子}  

\lhead{\firstmark}
\rhead{\botmark}

\subsection{\hspace{-0.5cm} {\Large \textcolor{darkblue}{\textbf{\ipa{ʈʂʰe˧\textasciitilde{}ʈʂʰe˧}}}}\hspace{0.5cm}[\kern2pt{\textcolor{darkblue}{\textbf{\ipa{ʈʂʰe˧ʈʂʰe˧}}}}\kern2pt]} \hypertarget{t`s`\string_he\string_M~t`s`\string_he\string_M1}{}
\markboth{\textcolor{darkblue}{\textbf{\ipa{ʈʂʰe˧\textasciitilde{}ʈʂʰe˧}}}}{}
\textcolor{teal}{\mytextsc{classificateur}} \hspace{4pt} Ton~: M.
\textit{De:} \textbf{ʈʂʰe˧b} \textcolor{PineGreen}{\selectlanguage{french}Classificateur pour les murs, et donc pour la largeur de toute une pièce: un buffet/placard occupe toute la largeur d'une pièce, par exemple.} \zh{量词:一面(墙)。} 
\lhead{\firstmark}
\rhead{\botmark}

\subsection{\hspace{-0.5cm} {\Large \textcolor{darkblue}{\textbf{\ipa{ʈʂʰe˩ko˧}}}}\hspace{0.5cm}[\kern2pt{\textcolor{darkblue}{\textbf{\ipa{ʈʂʰe˩ko˥}}}}\kern2pt]} \hypertarget{t`s`\string_he\string_Bko\string_M1}{}
\markboth{\textcolor{darkblue}{\textbf{\ipa{ʈʂʰe˩ko˧}}}}{}
\textcolor{teal}{\mytextsc{verbe}} \hspace{4pt} Ton~: LM.
\textcolor{PineGreen}{\selectlanguage{french}Réussir.} \zh{成功(汉语借词)。}  Emprunt~: chinois  \zh{成功}

\lhead{\firstmark}
\rhead{\botmark}

\subsection{\hspace{-0.5cm} {\Large \textcolor{darkblue}{\textbf{\ipa{ʈʂʰɤ˧tsɯ˧}}}}\hspace{0.5cm}[\kern2pt{\textcolor{darkblue}{\textbf{\ipa{ʈʂʰɤ˧tsɯ˧}}}}\kern2pt]} \hypertarget{t`s`\string_h7\string_MtsM\string_M1}{}
\markboth{\textcolor{darkblue}{\textbf{\ipa{ʈʂʰɤ˧tsɯ˧}}}}{}
\textcolor{teal}{\mytextsc{nom}} \hspace{4pt} Ton~: M.
\textcolor{PineGreen}{\selectlanguage{french}Voiture, automobile, car.} \zh{车子(汉语借词)。}  Emprunt~: chinois  \zh{车子}

\lhead{\firstmark}
\rhead{\botmark}

\subsection{\hspace{-0.5cm} {\Large \textcolor{darkblue}{\textbf{\ipa{ʈʂʰɤ˧zo˥-ʈʂʰɤ˩mv̩˩}}}}\hspace{0.5cm}[\kern2pt{\textcolor{darkblue}{\textbf{\ipa{ʈʂʰɤ˧zo˥ʈʂʰɤ˩mv̩˩}}}}\kern2pt]} \hypertarget{t`s`\string_h7\string_Mzo\string_T-t`s`\string_h7\string_Bmv\string_=\string_B1}{}
\markboth{\textcolor{darkblue}{\textbf{\ipa{ʈʂʰɤ˧zo˥-ʈʂʰɤ˩mv̩˩}}}}{}
\textcolor{teal}{\mytextsc{nom}} \hspace{4pt} Ton~: H\#-L.
\textcolor{PineGreen}{\selectlanguage{french}Enfant naturel.} \zh{私生子:没有名分的孩子、不明来路。}  ¶ \textcolor{darkblue}{\textbf{\ipa{ə˧dɑ˥ | ɲi˩-ɲi˥ | mɤ˧-sɯ˥ | ʈʂʰɯ˧-v̩˧, | ʈʂʰɤ˧zo˥-ʈʂʰɤ˩mv̩˩ mv̩˩ʈʂæ˩.}}} \textcolor{PineGreen}{\selectlanguage{french}Celui qui ne sait pas qui est son père, on l'appelle “enfant naturel”.} \zh{一个人不知道他父亲是谁,就称作“私生子”。}  
 ¶ \textcolor{darkblue}{\textbf{\ipa{ə˧ʝi˧-ʂɯ˥ʝi˩, | ʈʂʰɤ˧zo˥-ʈʂʰɤ˩mv̩˩ ʐɤ˩-hĩ˩-lɑ˩ ɲi˩!}}} \textcolor{PineGreen}{\selectlanguage{french}Autrefois, les enfants naturels, on les élevait et voilà tout!/on les élevait tout simplement, sans faire de difficultés!} \zh{过去,大家会公开把“私生子”养大,不会大惊小怪的!}  

\lhead{\firstmark}
\rhead{\botmark}

\subsection{\hspace{-0.5cm} {\Large \textcolor{darkblue}{\textbf{\ipa{ʈʂʰɤ˩\textsubscript{a}}}}}\hspace{0.5cm}[\kern2pt{\textcolor{darkblue}{\textbf{\ipa{ʈʂʰɤ˥}}}}\kern2pt]} \hypertarget{t`s`\string_h7\string_Ba1}{}
\markboth{\textcolor{darkblue}{\textbf{\ipa{ʈʂʰɤ˩\textsubscript{a}}}}}{}
\textcolor{teal}{\mytextsc{verbe}} \hspace{4pt} Ton~: L\textsubscript{a}.
\textcolor{PineGreen}{\selectlanguage{french}Répartir, diviser.} \zh{分。}  ¶ \textcolor{darkblue}{\textbf{\ipa{ɖɯ˧-v̩˧ ɖɯ˧-kʰwɤ˥ | le˧-ʈʂʰɤ˧\textasciitilde{}ʈʂʰɤ˥}}} \textcolor{PineGreen}{\selectlanguage{french}répartir un morceau par personne} \zh{平分}  

\lhead{\firstmark}
\rhead{\botmark}

\subsection{\hspace{-0.5cm} {\Large \textcolor{darkblue}{\textbf{\ipa{ʈʂʰɤ˩ho˥}}}}\hspace{0.5cm}[\kern2pt{\textcolor{darkblue}{\textbf{\ipa{ʈʂʰɤ˩ho˥}}}}\kern2pt]} \hypertarget{t`s`\string_h7\string_Bho\string_T1}{}
\markboth{\textcolor{darkblue}{\textbf{\ipa{ʈʂʰɤ˩ho˥}}}}{}
\textcolor{teal}{\mytextsc{nom}} \hspace{4pt} Ton~: LH.
\textcolor{PineGreen}{\selectlanguage{french}Bouilloire.} \zh{水壶(汉语借词:茶壶)。}  Emprunt~: \zh{茶壶}
 \zh{量词}: \textcolor{darkblue}{\textbf{\ipa{ɭɯ˧}}}  \mytextsc{clf}~: \textcolor{darkblue}{\textbf{\ipa{ɭɯ˧}}} 
\lhead{\firstmark}
\rhead{\botmark}

\subsection{\hspace{-0.5cm} {\Large \textcolor{darkblue}{\textbf{\ipa{ʈʂʰɤ˩kɤ˧}}}}\hspace{0.5cm}[\kern2pt{\textcolor{darkblue}{\textbf{\ipa{ʈʂʰɤ˩kɤ˥}}}}\kern2pt]} \hypertarget{t`s`\string_h7\string_Bk7\string_M1}{}
\markboth{\textcolor{darkblue}{\textbf{\ipa{ʈʂʰɤ˩kɤ˧}}}}{}
\textcolor{teal}{\mytextsc{nom}} \hspace{4pt} Ton~: LM.
\textcolor{PineGreen}{\selectlanguage{french}Gobelet (avec anse); en métal ou poterie.} \zh{缸子,杯子。}  \zh{量词}: \textcolor{darkblue}{\textbf{\ipa{ɭɯ˧}}}  \mytextsc{clf}~: \textcolor{darkblue}{\textbf{\ipa{ɭɯ˧}}} 
\lhead{\firstmark}
\rhead{\botmark}

\subsection{\hspace{-0.5cm} {\Large \textcolor{darkblue}{\textbf{\ipa{ʈʂʰɤ˩qo˧}}}}\hspace{0.5cm}[\kern2pt{\textcolor{darkblue}{\textbf{\ipa{ʈʂʰɤ˩qo˥}}}}\kern2pt]} \hypertarget{t`s`\string_h7\string_Bqo\string_M1}{}
\markboth{\textcolor{darkblue}{\textbf{\ipa{ʈʂʰɤ˩qo˧}}}}{}
\textcolor{teal}{\mytextsc{nom}} \hspace{4pt} Ton~: LM.
\textcolor{PineGreen}{\selectlanguage{french}Attention, intérêt.} \zh{关注、关心。}  ¶ \textcolor{darkblue}{\textbf{\ipa{ʈʂʰɤ˩qo˧ kʰɯ˧˥}}} \textcolor{PineGreen}{\selectlanguage{french}se soucier de, prêter attention à} \zh{关心、关注}  
 ¶ \textcolor{darkblue}{\textbf{\ipa{ʈʂʰɤ˩qo˧ | ɖwæ˧˥ | tʰi˧-kʰɯ˧˥}}} \textcolor{PineGreen}{\selectlanguage{french}prêter une grande attention à, être très attentif à (ex.: une grand-mère très attentive à l'alimentation d'un petit enfant)} \zh{很关心、很关注}  
 ¶ \textcolor{darkblue}{\textbf{\ipa{ʈʂʰɤ˩qo˧ | mɤ˧-kʰɯ˧˥}}} \textcolor{PineGreen}{\selectlanguage{french}être insensible à, ne pas prêter attention à} \zh{不关心、不关注}  
 \zh{量词}: \textcolor{darkblue}{\textbf{\ipa{kʰwɤ˥}}}  \mytextsc{clf}~: \textcolor{darkblue}{\textbf{\ipa{kʰwɤ˥}}} 
\lhead{\firstmark}
\rhead{\botmark}

\subsection{\hspace{-0.5cm} {\Large \textcolor{darkblue}{\textbf{\ipa{ʈʂʰɤ˩tɕʰɯ˩}}}}\hspace{0.5cm}[\kern2pt{\textcolor{darkblue}{\textbf{\ipa{ʈʂʰɤ˩tɕʰɯ˩˥}}}}\kern2pt]} \hypertarget{t`s`\string_h7\string_Bts£\string_hM\string_B1}{}
\markboth{\textcolor{darkblue}{\textbf{\ipa{ʈʂʰɤ˩tɕʰɯ˩}}}}{}
\textcolor{teal}{\mytextsc{adjectif}} \hspace{4pt} Ton~: L.
\textcolor{PineGreen}{\selectlanguage{french}Très doué, très calé, possédant des qualités admirables (par ex.: personne très savante).} \zh{利害,值得崇拜。} 
\lhead{\firstmark}
\rhead{\botmark}

\subsection{\hspace{-0.5cm} {\Large \textcolor{darkblue}{\textbf{\ipa{ʈʂʰɤ˩\textasciitilde{}ʈʂʰɤ˧˥}}}}\hspace{0.5cm}[\kern2pt{\textcolor{darkblue}{\textbf{\ipa{ʈʂʰɤ˧ʈʂʰɤ˧˥}}}}\kern2pt]} \hypertarget{t`s`\string_h7\string_B~t`s`\string_h7\string_M\string_T1}{}
\markboth{\textcolor{darkblue}{\textbf{\ipa{ʈʂʰɤ˩\textasciitilde{}ʈʂʰɤ˧˥}}}}{}
\textcolor{teal}{\mytextsc{verbe}} \hspace{4pt} Ton~: MH.
\textcolor{PineGreen}{\selectlanguage{french}Toucher.} \zh{抚摸。}  ¶ \textcolor{darkblue}{\textbf{\ipa{ʈʂʰɤ˩ʈʂʰɤ˧ mɤ˥-tʰɑ˩!}}} \textcolor{PineGreen}{\selectlanguage{french}il ne faut pas toucher!} \zh{禁止触碰!}  
 ¶ \textcolor{darkblue}{\textbf{\ipa{tʰɑ˧-ʈʂʰɤ˩ʈʂʰɤ˩!}}} \textcolor{PineGreen}{\selectlanguage{french}ne touchez pas!} \zh{别碰!}  
 ¶ \textcolor{darkblue}{\textbf{\ipa{tso˧\textasciitilde{}tso˧ ʈʂʰɤ˥ʈʂʰɤ˩}}} \textcolor{PineGreen}{\selectlanguage{french}toucher quelque chose} \zh{抚摸东西}  

\lhead{\firstmark}
\rhead{\botmark}

\subsection{\hspace{-0.5cm} {\Large \textcolor{darkblue}{\textbf{\ipa{ʈʂʰo˥}}}}\hspace{0.5cm}[\kern2pt{\textcolor{darkblue}{\textbf{\ipa{ʈʂʰo˥}}}}\kern2pt]} \hypertarget{t`s`\string_ho\string_T1}{}
\markboth{\textcolor{darkblue}{\textbf{\ipa{ʈʂʰo˥}}}}{}
\textcolor{teal}{\mytextsc{verbe}} \hspace{4pt} Ton~: H.
\textcolor{PineGreen}{\selectlanguage{french}Prier (une divinité): réciter des prières, psalmodier des prières.} \zh{拜(神)。}  ¶ \textcolor{darkblue}{\textbf{\ipa{ʈʂʰo˧do˩ ʈʂʰo˩}}} \textcolor{PineGreen}{\selectlanguage{french}prier l'esprit du foyer} \zh{祭祀祖先}  
 ¶ \textcolor{darkblue}{\textbf{\ipa{hĩ˧-mo˥, | zo˩qo˧ ʂɯ˧, | zo˩qo˧-ɳɯ˧ ʈʂʰo˧-zo˧!}}} \textcolor{PineGreen}{\selectlanguage{french}Les membres décédés de la famille, c'est à l'endroit où ils sont morts qu'on leur rend hommage!} \zh{要在家人去世地点进行祭拜!}  

\lhead{\firstmark}
\rhead{\botmark}

\subsection{\hspace{-0.5cm} {\Large \textcolor{darkblue}{\textbf{\ipa{ʈʂʰo˧\textsubscript{b}}}}}\hspace{0.5cm}[\kern2pt{\textcolor{darkblue}{\textbf{\ipa{ʈʂʰo˥}}}}\kern2pt]} \hypertarget{t`s`\string_ho\string_Mb1}{}
\markboth{\textcolor{darkblue}{\textbf{\ipa{ʈʂʰo˧\textsubscript{b}}}}}{}
\textcolor{teal}{\mytextsc{verbe}} \hspace{4pt} Ton~: M\textsubscript{b}.
\textcolor{PineGreen}{\selectlanguage{french}Lire à haute voix.} \zh{朗读。}  ¶ \textcolor{darkblue}{\textbf{\ipa{le˧-ʈʂʰo˧-ze˧}}} \textcolor{PineGreen}{\selectlanguage{french}\mytextsc{accomp} \string_ \mytextsc{pfv}} \zh{朗读了}  
 ¶ \textcolor{darkblue}{\textbf{\ipa{le˧-ʈʂʰo˧-le˧-se˩}}} \textcolor{PineGreen}{\selectlanguage{french}(j'ai) fini de lire} \zh{朗读完了。}  
 ¶ \textcolor{darkblue}{\textbf{\ipa{tʰæ˧ɻæ˩ ʈʂʰo˩}}} \textcolor{PineGreen}{\selectlanguage{french}lire un livre} \zh{朗读一本书}  
 ¶ \textcolor{darkblue}{\textbf{\ipa{ʈʂʰo˧\textasciitilde{}ʈʂʰo˧}}} \textcolor{PineGreen}{\selectlanguage{french}\mytextsc{red}} \zh{\mytextsc{重叠}}  

\lhead{\firstmark}
\rhead{\botmark}

\subsection{\hspace{-0.5cm} {\Large \textcolor{darkblue}{\textbf{\ipa{ʈʂʰo˧bɤ\#˥}}}}\hspace{0.5cm}[\kern2pt{\textcolor{darkblue}{\textbf{\ipa{ʈʂʰo˩bɤ˩˥}}}}\kern2pt]} \hypertarget{t`s`\string_ho\string_Mb7\#\string_T1}{}
\markboth{\textcolor{darkblue}{\textbf{\ipa{ʈʂʰo˧bɤ\#˥}}}}{}
\textcolor{teal}{\mytextsc{nom}} \hspace{4pt} Ton~: \#H.
\textcolor{PineGreen}{\selectlanguage{french}Vêtement masculin, que les hommes portaient à partir de 13 ans: sorte de veste serrée à la ceinture, qu'on portait sur la chemise lors des grandes occasions: mariage, invitations….} \zh{男上衣。}  \zh{量词}: \textcolor{darkblue}{\textbf{\ipa{ɭɯ˧˥}}}  \mytextsc{clf}~: \textcolor{darkblue}{\textbf{\ipa{ɭɯ˧˥}}} 
\lhead{\firstmark}
\rhead{\botmark}

\subsection{\hspace{-0.5cm} {\Large \textcolor{darkblue}{\textbf{\ipa{ʈʂʰo˧bv̩˩}}}}\hspace{0.5cm}[\kern2pt{\textcolor{darkblue}{\textbf{\ipa{ʈʂʰo˧bv̩˧}}}}\kern2pt]} \hypertarget{t`s`\string_ho\string_Mbv\string_=\string_B1}{}
\markboth{\textcolor{darkblue}{\textbf{\ipa{ʈʂʰo˧bv̩˩}}}}{}
\textcolor{teal}{\mytextsc{nom}} \hspace{4pt} Ton~: L\#.
\textcolor{PineGreen}{\selectlanguage{french}Acore odorant, jonc odorant, \textit{Acorus calamus}: plante herbacée aquatique, pérenne, rhizomateuse.} \zh{菖蒲。}  \zh{量词}: \textcolor{darkblue}{\textbf{\ipa{dzi˩}}}  \mytextsc{clf}~: \textcolor{darkblue}{\textbf{\ipa{dzi˩}}} 
\lhead{\firstmark}
\rhead{\botmark}

\subsection{\hspace{-0.5cm} {\Large \textcolor{darkblue}{\textbf{\ipa{ʈʂʰo˧do˩}}}}\hspace{0.5cm}[\kern2pt{\textcolor{darkblue}{\textbf{\ipa{ʈʂʰo˧do˧}}}}\kern2pt]} \hypertarget{t`s`\string_ho\string_Mdo\string_B1}{}
\markboth{\textcolor{darkblue}{\textbf{\ipa{ʈʂʰo˧do˩}}}}{}
\textcolor{teal}{\mytextsc{nom}} \hspace{4pt} Ton~: L\#.
\textcolor{PineGreen}{\selectlanguage{french}Petite éminence juste à côté du foyer, en contrebas de l'autel où on offre des cadeaux aux ancêtres; c'est sur cette petite éminence qu'on dépose un peu de nourriture au début de chaque repas, en offrande aux ancêtres.} \zh{火塘上面祖先灵位。}  \zh{量词}: \textcolor{darkblue}{\textbf{\ipa{ɭɯ˧}}}  \mytextsc{clf}~: \textcolor{darkblue}{\textbf{\ipa{ɭɯ˧}}} 
\lhead{\firstmark}
\rhead{\botmark}

\subsection{\hspace{-0.5cm} {\Large \textcolor{darkblue}{\textbf{\ipa{ʈʂʰo˧lo\#˥}}}}\hspace{0.5cm}[\kern2pt{\textcolor{darkblue}{\textbf{\ipa{ʈʂʰo˧lo˧}}}}\kern2pt]} \hypertarget{t`s`\string_ho\string_Mlo\#\string_T1}{}
\markboth{\textcolor{darkblue}{\textbf{\ipa{ʈʂʰo˧lo\#˥}}}}{}
\textcolor{teal}{\mytextsc{nom}} \hspace{4pt} Ton~: \#H.
\textcolor{PineGreen}{\selectlanguage{french}Grande poêle à fond plat, diamètre un peu supérieur à 50 cm, pour frire des aliments (galettes de pomme de terre, fèves).} \zh{平底大锅(直径大概半米),用来煎洋芋饼等等。}  \zh{量词}: \textcolor{darkblue}{\textbf{\ipa{ɭɯ˧}}}  \mytextsc{clf}~: \textcolor{darkblue}{\textbf{\ipa{ɭɯ˧}}} 
\lhead{\firstmark}
\rhead{\botmark}

\subsection{\hspace{-0.5cm} {\Large \textcolor{darkblue}{\textbf{\ipa{ʈʂʰɻ̍˧}}}}\hspace{0.5cm}[\kern2pt{\textcolor{darkblue}{\textbf{\ipa{ʈʂʰɻ̍˥}}}}\kern2pt]} \hypertarget{t`s`\string_hr£`̍\string_M1}{}
\markboth{\textcolor{darkblue}{\textbf{\ipa{ʈʂʰɻ̍˧}}}}{}
\textcolor{teal}{\mytextsc{nom}} \hspace{4pt} Ton~: M.
\textcolor{PineGreen}{\selectlanguage{french}Soc de l'araire.} \zh{铧头,犁铧。}  ¶ \textcolor{darkblue}{\textbf{\ipa{ʈʂʰɻ̍˧ ʈʂʰɯ˧-ɭɯ˧}}} \textcolor{PineGreen}{\selectlanguage{french}\mytextsc{n}+\mytextsc{dem}+\mytextsc{clf}} \zh{这把铧头}  
 \zh{量词}: \textcolor{darkblue}{\textbf{\ipa{ɭɯ˧}}}  \mytextsc{clf}~: \textcolor{darkblue}{\textbf{\ipa{ɭɯ˧}}} 
\lhead{\firstmark}
\rhead{\botmark}

\subsection{\hspace{-0.5cm} {\Large \textcolor{darkblue}{\textbf{\ipa{ʈʂʰɻ̍˧˥}}} \textsubscript{1}}\hspace{0.5cm}[\kern2pt{\textcolor{darkblue}{\textbf{\ipa{ʈʂʰɻ̍˧˥}}}}\kern2pt]} \hypertarget{t`s`\string_hr£`̍\string_M\string_T1}{}
\markboth{\textcolor{darkblue}{\textbf{\ipa{ʈʂʰɻ̍˧˥}}} \textsubscript{1}}{}
\textcolor{teal}{\mytextsc{verbe}} \hspace{4pt} Ton~: MH.
\textcolor{PineGreen}{\selectlanguage{french}Empoigner, prendre en main, saisir, tenir fermement (ex.: couteau); serrer, crisper (le poing).} \zh{握 (握刀把)。}  ¶ \textcolor{darkblue}{\textbf{\ipa{sɯ˩tʰi˩˥ | (ɖɯ˧)-nɑ˧ | tʰi˧-ʈʂʰɻ̍˧˥ (+dʑo˩)}}} \textcolor{PineGreen}{\selectlanguage{french}empoigner un couteau} \zh{手里握刀}  
 ¶ \textcolor{darkblue}{\textbf{\ipa{ʈʂʰɻ̍˧ mɤ˧-bi˧!}}} \textcolor{PineGreen}{\selectlanguage{french}je ne veux pas empoigner/ pas question que j'empoigne (ce couteau,…)} \zh{我不要拿(刀)!}  
 ¶ \textcolor{darkblue}{\textbf{\ipa{lo˩qʰwɤ˧ ʈʂʰɻ̍˩\textasciitilde{}ʈʂʰɻ̍˩ |}}} \textcolor{PineGreen}{\selectlanguage{french}serrer le poing} \zh{攥紧拳头}  

\lhead{\firstmark}
\rhead{\botmark}

\subsection{\hspace{-0.5cm} {\Large \textcolor{darkblue}{\textbf{\ipa{ʈʂʰɻ̍˧˥}}} \textsubscript{2}}\hspace{0.5cm}[\kern2pt{\textcolor{darkblue}{\textbf{\ipa{ʈʂʰɻ̍˧˥}}}}\kern2pt]} \hypertarget{t`s`\string_hr£`̍\string_M\string_T2}{}
\markboth{\textcolor{darkblue}{\textbf{\ipa{ʈʂʰɻ̍˧˥}}} \textsubscript{2}}{}
\textcolor{teal}{\mytextsc{nom}} \hspace{4pt} Ton~: MH.
\textcolor{PineGreen}{\selectlanguage{french}Poumon.} \zh{肺。}  \zh{量词}: \textcolor{darkblue}{\textbf{\ipa{ɭɯ˧}}}  \mytextsc{clf}~: \textcolor{darkblue}{\textbf{\ipa{ɭɯ˧}}} 
\lhead{\firstmark}
\rhead{\botmark}

\subsection{\hspace{-0.5cm} {\Large \textcolor{darkblue}{\textbf{\ipa{ʈʂʰɻ̍˧˥\textsubscript{a}}}}}\hspace{0.5cm}[\kern2pt{\textcolor{darkblue}{\textbf{\ipa{ʈʂʰɻ̍˧˥}}}}\kern2pt]} \hypertarget{t`s`\string_hr£`̍\string_M\string_Ta1}{}
\markboth{\textcolor{darkblue}{\textbf{\ipa{ʈʂʰɻ̍˧˥\textsubscript{a}}}}}{}
\textcolor{teal}{\mytextsc{classificateur}} \hspace{4pt} Ton~: MH\textsubscript{a}.
\textcolor{PineGreen}{\selectlanguage{french}Classificateur des boules/poignées: la quantité que l'on compacte en la serrant dans une main, par exemple une poignée de céréale cuite qu'on compresse en boule.} \zh{量词:团,掐。指的是一只手里能拿的量,压成团,如:手里拿煮熟的粮食,压成饭团。} 
\lhead{\firstmark}
\rhead{\botmark}

\subsection{\hspace{-0.5cm} {\Large \textcolor{darkblue}{\textbf{\ipa{ʈʂʰɯ˥}}} \textsubscript{1}}\hspace{0.5cm}[\kern2pt{\textcolor{darkblue}{\textbf{\ipa{ʈʂʰɯ˧˥}}}}\kern2pt]} \hypertarget{t`s`\string_hM\string_T1}{}
\markboth{\textcolor{darkblue}{\textbf{\ipa{ʈʂʰɯ˥}}} \textsubscript{1}}{}
\textcolor{teal}{\mytextsc{pronom}} \hspace{4pt} Ton~: \#H.
\textcolor{PineGreen}{\selectlanguage{french}Démonstratif proximal, qui forme un couple avec le démonstratif distal.} \zh{这\mytextsc{指示}.近指。}  ¶ \textcolor{darkblue}{\textbf{\ipa{ʈʂʰɯ˧ ɲi˥!}}} \textcolor{PineGreen}{\selectlanguage{french}C'est ça!} \zh{是这个! / 对的!}  
 ¶ \textcolor{darkblue}{\textbf{\ipa{ʈʂʰɯ˧-v̩\#˥}}} \textcolor{PineGreen}{\selectlanguage{french}celui-ci (\mytextsc{dem}\string_prox-\mytextsc{clf}.individu)} \zh{这个}  
 ¶ \textcolor{darkblue}{\textbf{\ipa{ʈʂʰɯ˧=ɻæ˥}}} \textcolor{PineGreen}{\selectlanguage{french}\mytextsc{pl}: ces choses-là} \zh{这些}  
\textit{Voir~:} \hyperlink{}{\textcolor{darkblue}{\textbf{\ipa{ʈʂʰɯ˥}}} \textsubscript{2}} \textit{Voir~:} \textcolor{darkblue}{\textbf{\ipa{-ʈʂʰɯ˥}}} 
\lhead{\firstmark}
\rhead{\botmark}

\subsection{\hspace{-0.5cm} {\Large \textcolor{darkblue}{\textbf{\ipa{ʈʂʰɯ˥}}} \textsubscript{2}}\hspace{0.5cm}[\kern2pt{\textcolor{darkblue}{\textbf{\ipa{ʈʂʰɯ˥}}}}\kern2pt]} \hypertarget{t`s`\string_hM\string_T2}{}
\markboth{\textcolor{darkblue}{\textbf{\ipa{ʈʂʰɯ˥}}} \textsubscript{2}}{}
\textcolor{teal}{\mytextsc{pronom}} \hspace{4pt} Ton~: \#H.
\textcolor{PineGreen}{\selectlanguage{french}Pronom de troisième personne du singulier.} \zh{他。}  ¶ \textcolor{darkblue}{\textbf{\ipa{ʈʂʰɯ˧ ɲi˥!}}} \textcolor{PineGreen}{\selectlanguage{french}C'est elle/lui!} \zh{是他!}  
\textit{Voir~:} \hyperlink{}{\textcolor{darkblue}{\textbf{\ipa{ʈʂʰɯ˥}}} \textsubscript{1}} \textit{Voir~:} \textcolor{darkblue}{\textbf{\ipa{-ʈʂʰɯ˥}}} 
\lhead{\firstmark}
\rhead{\botmark}

\subsection{\hspace{-0.5cm} {\Large \textcolor{darkblue}{\textbf{\ipa{ʈʂʰɯ˧}}}}\hspace{0.5cm}[\kern2pt{\textcolor{darkblue}{\textbf{\ipa{ʈʂʰɯ˥}}}}\kern2pt]} \hypertarget{t`s`\string_hM\string_M1}{}
\markboth{\textcolor{darkblue}{\textbf{\ipa{ʈʂʰɯ˧}}}}{}
\textcolor{teal}{\mytextsc{suffixe}} \hspace{4pt} Ton~: M.
\textcolor{PineGreen}{\selectlanguage{french}Focalisateur; grammaticalisé à partir du démonstratif proximal.} \zh{\mytextsc{主题(°指示}.近指)。} \textit{Voir~:} \hyperlink{}{\textcolor{darkblue}{\textbf{\ipa{ʈʂʰɯ˥}}} \textsubscript{1}} \textit{Voir~:} \hyperlink{}{\textcolor{darkblue}{\textbf{\ipa{ʈʂʰɯ˥}}} \textsubscript{2}} 
\lhead{\firstmark}
\rhead{\botmark}

\subsection{\hspace{-0.5cm} {\Large \textcolor{darkblue}{\textbf{\ipa{ʈʂʰɯ˧-gɤ˧}}}}\hspace{0.5cm}[\kern2pt{\textcolor{darkblue}{\textbf{\ipa{xxxx non-correspondance entre le nombre de morphèmes et le nombre de tons de morphèmes}}}}\kern2pt]} \hypertarget{t`s`\string_hM\string_M-g7\string_M1}{}
\markboth{\textcolor{darkblue}{\textbf{\ipa{ʈʂʰɯ˧-gɤ˧}}}}{}
\textcolor{teal}{\mytextsc{adverbe}} \hspace{4pt} Ton~: M.
\textcolor{PineGreen}{\selectlanguage{french}Ici, à cet endroit-ci.} \zh{这里。} \textit{Voir~:} \hyperlink{}{\textcolor{darkblue}{\textbf{\ipa{ʈʂʰɯ˧gi\#˥}}}} 
\lhead{\firstmark}
\rhead{\botmark}

\subsection{\hspace{-0.5cm} {\Large \textcolor{darkblue}{\textbf{\ipa{ʈʂʰɯ˧gi\#˥}}}}\hspace{0.5cm}[\kern2pt{\textcolor{darkblue}{\textbf{\ipa{ʈʂʰɯ˧gi˧}}}}\kern2pt]} \hypertarget{t`s`\string_hM\string_Mgi\#\string_T1}{}
\markboth{\textcolor{darkblue}{\textbf{\ipa{ʈʂʰɯ˧gi\#˥}}}}{}
\textcolor{teal}{\mytextsc{adverbe}} \hspace{4pt} Ton~: \#H.
\textcolor{PineGreen}{\selectlanguage{french}Ici, à cet endroit-ci.} \zh{这边。} \textit{Voir~:} \hyperlink{}{\textcolor{darkblue}{\textbf{\ipa{ʈʂʰɯ˧-gɤ˧}}}} 
\lhead{\firstmark}
\rhead{\botmark}

\subsection{\hspace{-0.5cm} {\Large \textcolor{darkblue}{\textbf{\ipa{ʈʂʰɯ˧ne˧-ʝi˥}}}}\hspace{0.5cm}[\kern2pt{\textcolor{darkblue}{\textbf{\ipa{xxxx non-correspondance entre le nombre de morphèmes et le nombre de tons de morphèmes}}}}\kern2pt]} \hypertarget{t`s`\string_hM\string_Mne\string_M-j££i\string_T1}{}
\markboth{\textcolor{darkblue}{\textbf{\ipa{ʈʂʰɯ˧ne˧-ʝi˥}}}}{}
\textcolor{teal}{\mytextsc{adverbe}} \hspace{4pt} Ton~: MH\#.
\textcolor{PineGreen}{\selectlanguage{french}Ainsi, de cette façon (adverbe de manière).} \zh{这样,这么。}  ¶ \textcolor{darkblue}{\textbf{\ipa{ʈʂʰɯ˧ne˧-ʝi˥ | le˧-ʐwɤ˩!}}} \textcolor{PineGreen}{\selectlanguage{french}c'est comme ça qu'on dit!} \zh{是这样讲的!}  
 ¶ \textcolor{darkblue}{\textbf{\ipa{ʈʂʰɯ˧ne˧-ʝi˥ | le˧-pi˥!}}} \textcolor{PineGreen}{\selectlanguage{french}c'est comme ça qu'on parle!} \zh{是这样说的!}  
 ¶ \textcolor{darkblue}{\textbf{\ipa{ʈʂʰɯ˧ne˧-ʝi˥ | le˧-ʝi˥!}}} \textcolor{PineGreen}{\selectlanguage{french}c'est comme ça qu'on fait!} \zh{是这样做的!}  

\lhead{\firstmark}
\rhead{\botmark}

\subsection{\hspace{-0.5cm} {\Large \textcolor{darkblue}{\textbf{\ipa{ʈʂʰɯ˧qɑ˧}}}}\hspace{0.5cm}[\kern2pt{\textcolor{darkblue}{\textbf{\ipa{ʈʂʰɯ˧qɑ˧}}}}\kern2pt]} \hypertarget{t`s`\string_hM\string_MqA\string_M1}{}
\markboth{\textcolor{darkblue}{\textbf{\ipa{ʈʂʰɯ˧qɑ˧}}}}{}
\textcolor{teal}{\mytextsc{adverbe}} \hspace{4pt} Ton~: .
\textcolor{PineGreen}{\selectlanguage{french}Ensemble.} \zh{一起。} 
\lhead{\firstmark}
\rhead{\botmark}

\subsection{\hspace{-0.5cm} {\Large \textcolor{darkblue}{\textbf{\ipa{ʈʂʰɯ˧-qo˧}}}}\hspace{0.5cm}[\kern2pt{\textcolor{darkblue}{\textbf{\ipa{xxxx non-correspondance entre le nombre de morphèmes et le nombre de tons de morphèmes}}}}\kern2pt]} \hypertarget{t`s`\string_hM\string_M-qo\string_M1}{}
\markboth{\textcolor{darkblue}{\textbf{\ipa{ʈʂʰɯ˧-qo˧}}}}{}
\textcolor{teal}{\mytextsc{adverbe}} \hspace{4pt} Ton~: M.
\textcolor{PineGreen}{\selectlanguage{french}Ici, à cet endroit-ci.} \zh{这里。} 
\lhead{\firstmark}
\rhead{\botmark}

\subsection{\hspace{-0.5cm} {\Large \textcolor{darkblue}{\textbf{\ipa{ʈʂʰɯ˧tɕi˩}}}}\hspace{0.5cm}[\kern2pt{\textcolor{darkblue}{\textbf{\ipa{ʈʂʰɯ˧tɕi˩}}}}\kern2pt]} \hypertarget{t`s`\string_hM\string_Mts£i\string_B1}{}
\markboth{\textcolor{darkblue}{\textbf{\ipa{ʈʂʰɯ˧tɕi˩}}}}{}
\textcolor{teal}{\mytextsc{pronom}} \hspace{4pt} Ton~: L\#.
\textcolor{PineGreen}{\selectlanguage{french}Pronom de troisième personne pluriel.} \zh{他们。} 
\lhead{\firstmark}
\rhead{\botmark}

\subsection{\hspace{-0.5cm} {\Large \textcolor{darkblue}{\textbf{\ipa{ʈʂʰɯ˧=zɯ˩}}}}\hspace{0.5cm}[\kern2pt{\textcolor{darkblue}{\textbf{\ipa{ʈʂʰɯ˧zɯ˩}}}}\kern2pt]} \hypertarget{t`s`\string_hM\string_M=zM\string_B1}{}
\markboth{\textcolor{darkblue}{\textbf{\ipa{ʈʂʰɯ˧=zɯ˩}}}}{}
\textcolor{teal}{\mytextsc{pronom}} \hspace{4pt} Ton~: L\#.
\textcolor{PineGreen}{\selectlanguage{french}Pronom de troisième personne duel: eux deux.} \zh{他们两个。} 
\lhead{\firstmark}
\rhead{\botmark}

\subsection{\hspace{-0.5cm} {\Large \textcolor{darkblue}{\textbf{\ipa{ʈʂʰv̩˩}}} \textsubscript{1}}\hspace{0.5cm}[\kern2pt{\textcolor{darkblue}{\textbf{\ipa{ʈʂʰv̩˧˥}}}}\kern2pt]} \hypertarget{t`s`\string_hv\string_=\string_B1}{}
\markboth{\textcolor{darkblue}{\textbf{\ipa{ʈʂʰv̩˩}}} \textsubscript{1}}{}
\textcolor{teal}{\mytextsc{verbe}} \hspace{4pt} Ton~: MH.
\textcolor{PineGreen}{\selectlanguage{french}Achever, terminer.} \zh{完成。}  ¶ \textcolor{darkblue}{\textbf{\ipa{le˧-ʈʂʰv̩˩-se˩}}} \textcolor{PineGreen}{\selectlanguage{french}\mytextsc{accomp} \string_ \mytextsc{cmpl}} \zh{完成了}  
 ¶ \textcolor{darkblue}{\textbf{\ipa{tsʰi˧-ɲi˧-bv̩˧ | lo˧ | le˧-ʈʂʰv̩˩! | le˧-se˩-ze˩!}}} \textcolor{PineGreen}{\selectlanguage{french}Le travail d'aujourd'hui... on tourne la page! Il est fini!} \zh{今天的工作完成了!就算完工了吧!}  

\lhead{\firstmark}
\rhead{\botmark}

\subsection{\hspace{-0.5cm} {\Large \textcolor{darkblue}{\textbf{\ipa{ʈʂʰv̩˩}}} \textsubscript{2}}\hspace{0.5cm}[\kern2pt{\textcolor{darkblue}{\textbf{\ipa{ʈʂʰv̩˩˥}}}}\kern2pt]} \hypertarget{t`s`\string_hv\string_=\string_B2}{}
\markboth{\textcolor{darkblue}{\textbf{\ipa{ʈʂʰv̩˩}}} \textsubscript{2}}{}
\textcolor{teal}{\mytextsc{verbe}} \hspace{4pt} Ton~: L.
\textcolor{PineGreen}{\selectlanguage{french}Mettre à part.} \zh{除开。}  ¶ \textcolor{darkblue}{\textbf{\ipa{gɤ˩-ʈʂʰv̩˧, | mv̩˩-ʈʂʰv̩˧-tsæ˩-ɲi˩}}} \textcolor{PineGreen}{\selectlanguage{french}laisser à part, distinguer, ne pas mettre ensemble, ne pas fourrer dans le même sac} \zh{不算在里面、不算在一起}  
 ¶ \textcolor{darkblue}{\textbf{\ipa{no˧-bv̩˧ | gɤ˩-ʈʂʰv̩˧! | njɤ˧-bv̩˧, | mv̩˩-ʈʂʰv̩˧!}}} \textcolor{PineGreen}{\selectlanguage{french}Ce qui est à toi est à toi; ce qui est à moi est à moi!} \zh{你的算你的,我的算我的!}  

\lhead{\firstmark}
\rhead{\botmark}

\subsection{\hspace{-0.5cm} {\Large \textcolor{darkblue}{\textbf{\ipa{ʈʂʰv̩˧}}}}\hspace{0.5cm}[\kern2pt{\textcolor{darkblue}{\textbf{\ipa{ʈʂʰv̩˥}}}}\kern2pt]} \hypertarget{t`s`\string_hv\string_=\string_M1}{}
\markboth{\textcolor{darkblue}{\textbf{\ipa{ʈʂʰv̩˧}}}}{}
\textcolor{teal}{\mytextsc{nom}} \hspace{4pt} Ton~: M.
\textcolor{PineGreen}{\selectlanguage{french}Repas du matin/ petit déjeuner.} \zh{早饭。}  ¶ \textcolor{darkblue}{\textbf{\ipa{ʈʂʰv̩˧ dzɯ˧(-ze˩)}}} \textcolor{PineGreen}{\selectlanguage{french}prendre le petit déjeuner} \zh{吃早饭}  
 ¶ \textcolor{darkblue}{\textbf{\ipa{bæ˧qʰæ˧ ʈʂʰv̩\#˥}}} \textcolor{PineGreen}{\selectlanguage{french}le petit déjeuner qu'on prend au retour de la cérémonie de crémation: revenant du lieu où a eu lieu la crémation, les invités, en nombre relativement restreint, font une pause dans la maison du défunt, où on leur offre une collation avant qu'ils ne s'en retournent.} \zh{丧礼早餐:参加火葬仪式的人留在去世的人家,一起吃一点早饭再回家。}  

\lhead{\firstmark}
\rhead{\botmark}

\subsection{\hspace{-0.5cm} {\Large \textcolor{darkblue}{\textbf{\ipa{ʈʂʰv̩˧˥}}}}\hspace{0.5cm}[\kern2pt{\textcolor{darkblue}{\textbf{\ipa{ʈʂʰv̩˧˥}}}}\kern2pt]} \hypertarget{t`s`\string_hv\string_=\string_M\string_T1}{}
\markboth{\textcolor{darkblue}{\textbf{\ipa{ʈʂʰv̩˧˥}}}}{}
\textcolor{teal}{\mytextsc{verbe}} \hspace{4pt} Ton~: MH.
\textcolor{PineGreen}{\selectlanguage{french}Ajouter de l’eau, verser de l'eau.} \zh{掺和。}  ¶ \textcolor{darkblue}{\textbf{\ipa{le˧-ʈʂʰv̩˧-ze˥}}} \textcolor{PineGreen}{\selectlanguage{french}\mytextsc{accomp} \string_ \mytextsc{pfv}} \zh{\mytextsc{accomp} \string_ \mytextsc{pfv}}  
 ¶ \textcolor{darkblue}{\textbf{\ipa{dʑɯ˩ ʈʂʰv̩˩˥}}} \textcolor{PineGreen}{\selectlanguage{french}ajouter de l'eau (dans une marmite, ...)} \zh{加水(如:往锅里添加水)}  

\lhead{\firstmark}
\rhead{\botmark}

\subsection{\hspace{-0.5cm} {\Large \textcolor{darkblue}{\textbf{\ipa{ʈʂʰv̩˩\textsubscript{a}}}}}\hspace{0.5cm}[\kern2pt{\textcolor{darkblue}{\textbf{\ipa{ʈʂʰv̩˩˥}}}}\kern2pt]} \hypertarget{t`s`\string_hv\string_=\string_Ba1}{}
\markboth{\textcolor{darkblue}{\textbf{\ipa{ʈʂʰv̩˩\textsubscript{a}}}}}{}
\textcolor{teal}{\mytextsc{verbe}} \hspace{4pt} Ton~: L\textsubscript{a}.
\textcolor{PineGreen}{\selectlanguage{french}Teindre.} \zh{染。}  ¶ \textcolor{darkblue}{\textbf{\ipa{mɤ˧-ʈʂʰv̩˩}}} \textcolor{PineGreen}{\selectlanguage{french}\mytextsc{neg}} \zh{\mytextsc{neg}}  
 ¶ \textcolor{darkblue}{\textbf{\ipa{ʈʂʰv̩˩ mɤ˩-bi˩˥!}}} \textcolor{PineGreen}{\selectlanguage{french}\string_ \mytextsc{neg} \mytextsc{fut}\string_imm} \zh{\string_ \mytextsc{neg} \mytextsc{fut}\string_imm}  
 ¶ \textcolor{darkblue}{\textbf{\ipa{tso˧\textasciitilde{}tso˧ ʈʂʰv̩˥}}} \textcolor{PineGreen}{\selectlanguage{french}teindre des choses} \zh{染东西}  

\lhead{\firstmark}
\rhead{\botmark}

\subsection{\hspace{-0.5cm} {\Large \textcolor{darkblue}{\textbf{\ipa{ʈʂʰv̩˧dʑɯ˧}}}}\hspace{0.5cm}[\kern2pt{\textcolor{darkblue}{\textbf{\ipa{ʈʂʰv̩˧dʑɯ˧}}}}\kern2pt]} \hypertarget{t`s`\string_hv\string_=\string_Mdz£M\string_M1}{}
\markboth{\textcolor{darkblue}{\textbf{\ipa{ʈʂʰv̩˧dʑɯ˧}}}}{}
\textcolor{teal}{\mytextsc{nom}} \hspace{4pt} Ton~: M.
\textcolor{PineGreen}{\selectlanguage{french}Teinture.} \zh{染料。}  ¶ \textcolor{darkblue}{\textbf{\ipa{dʑi˧hṽ˧-ʈʂʰv̩˧dʑɯ˧}}} \textcolor{PineGreen}{\selectlanguage{french}teinture pour vêtements} \zh{衣服染料}  
 ¶ \textcolor{darkblue}{\textbf{\ipa{ʈʂʰv̩˧dʑɯ˧ | hṽ˩-hĩ˩˥}}} \textcolor{PineGreen}{\selectlanguage{french}teinture rouge} \zh{红色的染料}  
 \zh{量词}: \textcolor{darkblue}{\textbf{\ipa{kʰwɤ˥}}}  \mytextsc{clf}~: \textcolor{darkblue}{\textbf{\ipa{kʰwɤ˥}}} 
\lhead{\firstmark}
\rhead{\botmark}

\subsection{\hspace{-0.5cm} {\Large \textcolor{darkblue}{\textbf{\ipa{ʈʂʰv̩˧mi˧}}}}\hspace{0.5cm}[\kern2pt{\textcolor{darkblue}{\textbf{\ipa{ʈʂʰv̩˧mi˧}}}}\kern2pt]} \hypertarget{t`s`\string_hv\string_=\string_Mmi\string_M1}{}
\markboth{\textcolor{darkblue}{\textbf{\ipa{ʈʂʰv̩˧mi˧}}}}{}
\textcolor{teal}{\mytextsc{nom}} \hspace{4pt} Ton~: M.
\textcolor{PineGreen}{\selectlanguage{french}Épouse, femme.} \zh{太太、老婆、媳妇。}  \zh{量词}: \textcolor{darkblue}{\textbf{\ipa{v̩˧}}}  \mytextsc{clf}~: \textcolor{darkblue}{\textbf{\ipa{v̩˧}}} 
\lhead{\firstmark}
\rhead{\botmark}

\subsection{\hspace{-0.5cm} {\Large \textcolor{darkblue}{\textbf{\ipa{ʈʂʰv̩˧ɻ̍˧qʰv̩\#˥}}}}\hspace{0.5cm}[\kern2pt{\textcolor{darkblue}{\textbf{\ipa{ʈʂʰv̩˧ɻ̍˧qʰv̩˧}}}}\kern2pt]} \hypertarget{t`s`\string_hv\string_=\string_Mr£`̍\string_Mq\string_hv\string_=\#\string_T1}{}
\markboth{\textcolor{darkblue}{\textbf{\ipa{ʈʂʰv̩˧ɻ̍˧qʰv̩\#˥}}}}{}
\textcolor{teal}{\mytextsc{nom}} \hspace{4pt} Ton~: \#H.
\textcolor{PineGreen}{\selectlanguage{french}Fourmilière.} \zh{蚂蚁巢。}  ¶ \textcolor{darkblue}{\textbf{\ipa{ʈʂʰv̩˧ɻ̍˧qʰv̩˧ ɲi˥!}}} \textcolor{PineGreen}{\selectlanguage{french}c'est une fourmilière!} \zh{是蚂蚁巢!}  
 \zh{量词}: \textcolor{darkblue}{\textbf{\ipa{ɭɯ˧}}}  \mytextsc{clf}~: \textcolor{darkblue}{\textbf{\ipa{ɭɯ˧}}} \textit{Voir~:} \hyperlink{}{\textcolor{darkblue}{\textbf{\ipa{ʈʂʰv̩˧ɻ̍˥\$}}}} 
\lhead{\firstmark}
\rhead{\botmark}

\subsection{\hspace{-0.5cm} {\Large \textcolor{darkblue}{\textbf{\ipa{ʈʂʰv̩˧ɻ̍˥\$}}}}\hspace{0.5cm}[\kern2pt{\textcolor{darkblue}{\textbf{\ipa{ʈʂʰv̩˧ɻ̍˥}}}}\kern2pt]} \hypertarget{t`s`\string_hv\string_=\string_Mr£`̍\string_T\$1}{}
\markboth{\textcolor{darkblue}{\textbf{\ipa{ʈʂʰv̩˧ɻ̍˥\$}}}}{}
\textcolor{teal}{\mytextsc{nom}} \hspace{4pt} Ton~: H\$.
\textcolor{PineGreen}{\selectlanguage{french}Fourmi.} \zh{蚂蚁。}  ¶ \textcolor{darkblue}{\textbf{\ipa{ʈʂʰv̩˧ɻ̍˧ | tɕi˩-hĩ˩˥}}} \textcolor{PineGreen}{\selectlanguage{french}petite fourmi} \zh{小蚂蚁}  
 \zh{量词}: \textcolor{darkblue}{\textbf{\ipa{mi˩}}}  \mytextsc{clf}~: \textcolor{darkblue}{\textbf{\ipa{mi˩}}} 
\lhead{\firstmark}
\rhead{\botmark}

\subsection{\hspace{-0.5cm} {\Large \textcolor{darkblue}{\textbf{\ipa{ʈʂʰwæ˧\textsubscript{a}}}} \textsubscript{1}}\hspace{0.5cm}[\kern2pt{\textcolor{darkblue}{\textbf{\ipa{ʈʂʰwæ˩˥}}}}\kern2pt]} \hypertarget{t`s`\string_hw\{\string_Ma1}{}
\markboth{\textcolor{darkblue}{\textbf{\ipa{ʈʂʰwæ˧\textsubscript{a}}}} \textsubscript{1}}{}
\textcolor{teal}{\mytextsc{verbe}} \hspace{4pt} Ton~: M\textsubscript{a}.
\textcolor{PineGreen}{\selectlanguage{french}Pourrir.} \zh{腐烂。}  ¶ \textcolor{darkblue}{\textbf{\ipa{ʈʂʰwæ˧-ze˧}}} \textcolor{PineGreen}{\selectlanguage{french}\mytextsc{pfv}} \zh{烂了}  
 ¶ \textcolor{darkblue}{\textbf{\ipa{le˧-ʈʂʰwæ˧-ze˧}}} \textcolor{PineGreen}{\selectlanguage{french}\mytextsc{accomp} \string_ \mytextsc{pfv}} \zh{\mytextsc{accomp} \string_ \mytextsc{pfv}}  
 ¶ \textcolor{darkblue}{\textbf{\ipa{hĩ˧-ɳɯ˩ | mɤ˧-dzɯ˥, | le˧-ʈʂʰwæ˧-ze˧! |}}} \textcolor{PineGreen}{\selectlanguage{french}On a oublié de la manger, et maintenant c'est pourri! (au sujet d'une pastèque qui a traîné dans le garde-manger et est maintenant incomestible)} \zh{没人吃,就烂了!(一个西瓜被忘记在橱柜里,就腐烂了)}  

\lhead{\firstmark}
\rhead{\botmark}

\subsection{\hspace{-0.5cm} {\Large \textcolor{darkblue}{\textbf{\ipa{ʈʂʰwæ˧\textsubscript{a}}}} \textsubscript{2}}\hspace{0.5cm}[\kern2pt{\textcolor{darkblue}{\textbf{\ipa{ʈʂʰwæ˥}}}}\kern2pt]} \hypertarget{t`s`\string_hw\{\string_Ma2}{}
\markboth{\textcolor{darkblue}{\textbf{\ipa{ʈʂʰwæ˧\textsubscript{a}}}} \textsubscript{2}}{}
\textcolor{teal}{\mytextsc{verbe}} \hspace{4pt} Ton~: M\textsubscript{a}.
\textcolor{PineGreen}{\selectlanguage{french}Se réveiller.} \zh{醒来。}  ¶ \textcolor{darkblue}{\textbf{\ipa{le˧-ʈʂʰwæ˧-ze˧}}} \textcolor{PineGreen}{\selectlanguage{french}\mytextsc{accomp} \string_ \mytextsc{pfv}} \zh{\mytextsc{accomp} \string_ \mytextsc{pfv}}  
 ¶ \textcolor{darkblue}{\textbf{\ipa{gɤ˩ʈʂʰwæ˧}}} \textcolor{PineGreen}{\selectlanguage{french}se réveiller} \zh{醒来}  
 ¶ \textcolor{darkblue}{\textbf{\ipa{gɤ˩ʈʂʰwæ˧-ze˧!}}} \textcolor{PineGreen}{\selectlanguage{french}(il) s'est réveillé!} \zh{醒来了!}  

\lhead{\firstmark}
\rhead{\botmark}

\subsection{\hspace{-0.5cm} {\Large \textcolor{darkblue}{\textbf{\ipa{ʈʂʰwæ˧-bv̩˧nv̩\#˥}}}}\hspace{0.5cm}[\kern2pt{\textcolor{darkblue}{\textbf{\ipa{xxxx non-correspondance entre le nombre de morphèmes et le nombre de tons de morphèmes}}}}\kern2pt]} \hypertarget{t`s`\string_hw\{\string_M-bv\string_=\string_Mnv\string_=\#\string_T1}{}
\markboth{\textcolor{darkblue}{\textbf{\ipa{ʈʂʰwæ˧-bv̩˧nv̩\#˥}}}}{}
\textcolor{teal}{\mytextsc{adjectif}} \hspace{4pt} Ton~: \#H.
\textcolor{PineGreen}{\selectlanguage{french}Puant, à l'odeur de pourriture.} \zh{食物变味,有臭味道了。}  ¶ \textcolor{darkblue}{\textbf{\ipa{ʈʂʰwæ˧-bv̩˧nv̩˧ ɲi˥!}}} \textcolor{PineGreen}{\selectlanguage{french}Ca sent le pourri! / Ca pue la pourriture! / C'est vraiment malodorant!} \zh{臭了、有臭味道了}  

\lhead{\firstmark}
\rhead{\botmark}

\subsection{\hspace{-0.5cm} {\Large \textcolor{darkblue}{\textbf{\ipa{ʈʂʰwæ˧kɯ˧}}}}\hspace{0.5cm}[\kern2pt{\textcolor{darkblue}{\textbf{\ipa{ʈʂʰwæ˩kɯ˥}}}}\kern2pt]} \hypertarget{t`s`\string_hw\{\string_MkM\string_M1}{}
\markboth{\textcolor{darkblue}{\textbf{\ipa{ʈʂʰwæ˧kɯ˧}}}}{}
\textcolor{teal}{\mytextsc{nom}} \hspace{4pt} Ton~: M.
\textcolor{PineGreen}{\selectlanguage{french}Filet.} \zh{网。}  ¶ \textcolor{darkblue}{\textbf{\ipa{ʈʂʰwæ˧kɯ˧ tʰv̩˧-nɑ˩}}} \textcolor{PineGreen}{\selectlanguage{french}\mytextsc{n}+\mytextsc{dem}+\mytextsc{clf}} \zh{这个网}  
 \zh{量词}: \textcolor{darkblue}{\textbf{\ipa{nɑ˧}}}  \mytextsc{clf}~: \textcolor{darkblue}{\textbf{\ipa{nɑ˧}}} 
\lhead{\firstmark}
\rhead{\botmark}

\subsection{\hspace{-0.5cm} {\Large \textcolor{darkblue}{\textbf{\ipa{ʈʂʰwæ˧tsɯ˧}}}}\hspace{0.5cm}[\kern2pt{\textcolor{darkblue}{\textbf{\ipa{ʈʂʰwæ˩tsɯ˩˥}}}}\kern2pt]} \hypertarget{t`s`\string_hw\{\string_MtsM\string_M1}{}
\markboth{\textcolor{darkblue}{\textbf{\ipa{ʈʂʰwæ˧tsɯ˧}}}}{}
\textcolor{teal}{\mytextsc{nom}} \hspace{4pt} Ton~: M.
\textcolor{PineGreen}{\selectlanguage{french}Fenêtre.} \zh{窗户。} Dialecte chinois local~:\zh{窗子。} Emprunt~: chinois  \zh{窗子}
 \zh{量词}: \textcolor{darkblue}{\textbf{\ipa{nɑ˧}}}  \mytextsc{clf}~: \textcolor{darkblue}{\textbf{\ipa{nɑ˧}}} 
\lhead{\firstmark}
\rhead{\botmark}

\subsection{\hspace{-0.5cm} {\Large \textcolor{darkblue}{\textbf{\ipa{ʈʂʰwæ˧ʈʂʰwæ˧}}}}\hspace{0.5cm}[\kern2pt{\textcolor{darkblue}{\textbf{\ipa{ʈʂʰwæ˧ʈʂʰwæ˧}}}}\kern2pt]} \hypertarget{t`s`\string_hw\{\string_Mt`s`\string_hw\{\string_M1}{}
\markboth{\textcolor{darkblue}{\textbf{\ipa{ʈʂʰwæ˧ʈʂʰwæ˧}}}}{}
\textcolor{teal}{\mytextsc{nom}} \hspace{4pt} Ton~: M.
\textcolor{PineGreen}{\selectlanguage{french}Cymbales (mot emprunté au chinois).} \zh{钹。}  \zh{量词}: \textcolor{darkblue}{\textbf{\ipa{nɑ˧}}}  \mytextsc{clf}~: \textcolor{darkblue}{\textbf{\ipa{nɑ˧}}} 
\lhead{\firstmark}
\rhead{\botmark}

\subsection{\hspace{-0.5cm} {\Large \textcolor{darkblue}{\textbf{\ipa{ʈʂʰwæ˩\textsubscript{a}}}}}\hspace{0.5cm}[\kern2pt{\textcolor{darkblue}{\textbf{\ipa{ʈʂʰwæ˥}}}}\kern2pt]} \hypertarget{t`s`\string_hw\{\string_Ba1}{}
\markboth{\textcolor{darkblue}{\textbf{\ipa{ʈʂʰwæ˩\textsubscript{a}}}}}{}
\textcolor{teal}{\mytextsc{adjectif}} \hspace{4pt} Ton~: L\textsubscript{a}.
\textcolor{PineGreen}{\selectlanguage{french}Rapide.} \zh{快(动作快,跑得快)。}  ¶ \textcolor{darkblue}{\textbf{\ipa{ʈʂʰwæ˩-hĩ˩˥}}} \textcolor{PineGreen}{\selectlanguage{french}\mytextsc{rel}/\mytextsc{nmlz}} \zh{快的}  
 ¶ \textcolor{darkblue}{\textbf{\ipa{ɲi˧to˧ ʈʂʰwæ˩}}} \textcolor{PineGreen}{\selectlanguage{french}bavard, qui parle sans réfléchir suffisamment} \zh{嘴快}  

\lhead{\firstmark}
\rhead{\botmark}

\subsection{\hspace{-0.5cm} {\Large \textcolor{darkblue}{\textbf{\ipa{ʈʂʰwæ˩tsʰɯ˩}}}}\hspace{0.5cm}[\kern2pt{\textcolor{darkblue}{\textbf{\ipa{ʈʂʰwæ˧tsʰɯ˩}}}}\kern2pt]} \hypertarget{t`s`\string_hw\{\string_Bts\string_hM\string_B1}{}
\markboth{\textcolor{darkblue}{\textbf{\ipa{ʈʂʰwæ˩tsʰɯ˩}}}}{}
\textcolor{teal}{\mytextsc{verbe}} \hspace{4pt} Ton~: L.
\textcolor{PineGreen}{\selectlanguage{french}Créer.} \zh{创造(汉语借词)。}  Emprunt~: chinois  \zh{创造}

\lhead{\firstmark}
\rhead{\botmark}

\subsection{\hspace{-0.5cm} {\Large \textcolor{darkblue}{\textbf{\ipa{ʈʂʰwæ˧˥}}} \textsubscript{1}}\hspace{0.5cm}[\kern2pt{\textcolor{darkblue}{\textbf{\ipa{ʈʂʰwæ˩˥}}}}\kern2pt]} \hypertarget{t`s`\string_hw\{\string_M\string_T1}{}
\markboth{\textcolor{darkblue}{\textbf{\ipa{ʈʂʰwæ˧˥}}} \textsubscript{1}}{}
\textcolor{teal}{\mytextsc{verbe}} \hspace{4pt} Ton~: MH.
\textcolor{PineGreen}{\selectlanguage{french}Cacher: cacher un objet.} \zh{藏(东西)。} 
\lhead{\firstmark}
\rhead{\botmark}

\subsection{\hspace{-0.5cm} {\Large \textcolor{darkblue}{\textbf{\ipa{ʈʂʰwæ˧˥}}} \textsubscript{2}}\hspace{0.5cm}[\kern2pt{\textcolor{darkblue}{\textbf{\ipa{ʈʂʰwæ˧˥}}}}\kern2pt]} \hypertarget{t`s`\string_hw\{\string_M\string_T2}{}
\markboth{\textcolor{darkblue}{\textbf{\ipa{ʈʂʰwæ˧˥}}} \textsubscript{2}}{}
\textcolor{teal}{\mytextsc{verbe}} \hspace{4pt} Ton~: MH.
\textcolor{PineGreen}{\selectlanguage{french}Enfoncer (un couteau) d'un geste brutal: poignarder, de haut en bas, avec force; insérer, planter, ficher (ex.: un couteau, une aiguille...).} \zh{插、戳。} 
\lhead{\firstmark}
\rhead{\botmark}

\subsection{\hspace{-0.5cm} {\Large \textcolor{darkblue}{\textbf{\ipa{ʈʂʰwæ˩˧}}}}\hspace{0.5cm}[\kern2pt{\textcolor{darkblue}{\textbf{\ipa{ʈʂʰwæ˧˥}}}}\kern2pt]} \hypertarget{t`s`\string_hw\{\string_B\string_M1}{}
\markboth{\textcolor{darkblue}{\textbf{\ipa{ʈʂʰwæ˩˧}}}}{}
\textcolor{teal}{\mytextsc{nom}} \hspace{4pt} Ton~: LM.
\textcolor{PineGreen}{\selectlanguage{french}Bateau (emprunt chinois ancien); désormais, c'est le terme employé pour les grands bateaux, par ex. les barges pour passer le Yangtze; ce sont des Chinois et des Hmong qui auraient installé le bateau permettant de passer le Yangtze, d'où l'utilisation d'un mot chinois.} \zh{船(汉语借词)。}  Emprunt~: chinois  \zh{船}
 \zh{量词}: \textcolor{darkblue}{\textbf{\ipa{nɑ˧}}}  \mytextsc{clf}~: \textcolor{darkblue}{\textbf{\ipa{nɑ˧}}} 
\lhead{\firstmark}
\rhead{\botmark}

\subsection{\hspace{-0.5cm} {\Large \textcolor{darkblue}{\textbf{\ipa{ʈʂʰwɤ˧tsʰi˧˥}}}}\hspace{0.5cm}[\kern2pt{\textcolor{darkblue}{\textbf{\ipa{ʈʂʰwɤ˧tsʰi˧}}}}\kern2pt]} \hypertarget{t`s`\string_hw7\string_Mts\string_hi\string_M\string_T1}{}
\markboth{\textcolor{darkblue}{\textbf{\ipa{ʈʂʰwɤ˧tsʰi˧˥}}}}{}
\textcolor{teal}{\mytextsc{adjectif}} \hspace{4pt} Ton~: MH\#.
\textcolor{PineGreen}{\selectlanguage{french}Étroit.} \zh{窄。} 
\lhead{\firstmark}
\rhead{\botmark}

\subsection{\hspace{-0.5cm} {\Large \textcolor{darkblue}{\textbf{\ipa{ʈʂʰwɤ˩}}}}\hspace{0.5cm}[\kern2pt{\textcolor{darkblue}{\textbf{\ipa{ʈʂʰwɤ˥}}}}\kern2pt]} \hypertarget{t`s`\string_hw7\string_B1}{}
\markboth{\textcolor{darkblue}{\textbf{\ipa{ʈʂʰwɤ˩}}}}{}
\textcolor{teal}{\mytextsc{nom}} \hspace{4pt} Ton~: L.
\textcolor{PineGreen}{\selectlanguage{french}Repas du soir, dîner.} \zh{晚饭。}  ¶ \textcolor{darkblue}{\textbf{\ipa{ʈʂʰwɤ˩ gv̩˩˥}}} \textcolor{PineGreen}{\selectlanguage{french}cuisiner le dîner} \zh{做晚饭}  
 ¶ \textcolor{darkblue}{\textbf{\ipa{ʈʂʰwɤ˩ tʰv̩˩˥}}} \textcolor{PineGreen}{\selectlanguage{french}offrir à dîner, se charger du dîner (personne qui invite, pas nécessairement qui fait la cuisine elle-même)} \zh{请吃晚饭,提供晚餐(不一定自己做:意思是提供原料)}  
 ¶ \textcolor{darkblue}{\textbf{\ipa{ʈʂʰwɤ˩ dzɯ˩˥}}} \textcolor{PineGreen}{\selectlanguage{french}prendre le repas du soir} \zh{吃晚饭}  

\lhead{\firstmark}
\rhead{\botmark}

\newpage
\section*{\centering- \textcolor{darkblue}{\textbf{\ipa{u}}} -}
\subsection{\hspace{-0.5cm} {\Large \textcolor{darkblue}{\textbf{\ipa{u˧}}}}\hspace{0.5cm}[\kern2pt{\textcolor{darkblue}{\textbf{\ipa{u˥}}}}\kern2pt]} \hypertarget{u\string_M1}{}
\markboth{\textcolor{darkblue}{\textbf{\ipa{u˧}}}}{}
\textcolor{teal}{\mytextsc{pronom}} \hspace{4pt} Ton~: M.
\textcolor{PineGreen}{\selectlanguage{french}Pronom de 1e personne, associatif: les miens. Cette racine n'apparaît qu'en combinaison avec un clitique pluriel ou associatif.} \zh{我家人、我家族。}  ¶ \textcolor{darkblue}{\textbf{\ipa{u˧=ɻ˩, ʈʂʰɯ˧=ɻ˩}}} \textcolor{PineGreen}{\selectlanguage{french}mon clan, son clan : deux termes qui forment une opposition}  
 ¶ \textcolor{darkblue}{\textbf{\ipa{u˧ɻ̍˩ | ə˧si˧}}} \textcolor{PineGreen}{\selectlanguage{french}mon arrière-grand-mère} \zh{我家祖母}  
 ¶ \textcolor{darkblue}{\textbf{\ipa{u˧=ɻæ˩, ʈʂʰɯ˧=ɻæ˩}}} \textcolor{PineGreen}{\selectlanguage{french}Nous autres, eux : deux termes qui forment une opposition.}  

\lhead{\firstmark}
\rhead{\botmark}

\newpage
\section*{\centering- \textcolor{darkblue}{\textbf{\ipa{v}}} \textcolor{darkblue}{\textbf{\ipa{ṽ}}} -}
\subsection{\hspace{-0.5cm} {\Large \textcolor{darkblue}{\textbf{\ipa{v̩˩}}}}\hspace{0.5cm}[\kern2pt{\textcolor{darkblue}{\textbf{\ipa{v̩˥}}}}\kern2pt]} \hypertarget{v\string_=\string_B1}{}
\markboth{\textcolor{darkblue}{\textbf{\ipa{v̩˩}}}}{}
\textcolor{teal}{\mytextsc{verbe}} \hspace{4pt} Ton~: L\textsubscript{a}.
\textcolor{PineGreen}{\selectlanguage{french}Embrasser, prendre dans ses bras (monosyllabe extrait de la forme \textcolor{darkblue}{\textbf{\ipa{/le˧-v̩˧~v̩˥/}}}).} \zh{搂(人的脖子)。}  ¶ \textcolor{darkblue}{\textbf{\ipa{ʁæ˧ | le˧-v̩˧\textasciitilde{}v̩˥}}} \textcolor{PineGreen}{\selectlanguage{french}prendre le cou (de quelqu'un) dans son bras} \zh{搂人的脖子}  
 ¶ \textcolor{darkblue}{\textbf{\ipa{ʁæ˧ | le˧-v̩˩}}} \textcolor{PineGreen}{\selectlanguage{french}même sens} \zh{同上}  
 ¶ \textcolor{darkblue}{\textbf{\ipa{ʁæ˧ v̩˥ se˩}}} \textcolor{PineGreen}{\selectlanguage{french}marcher/se promener en tenant le cou de quelqu'un/enlacé avec quelqu'un} \zh{互相搂着走}  

\lhead{\firstmark}
\rhead{\botmark}

\subsection{\hspace{-0.5cm} {\Large \textcolor{darkblue}{\textbf{\ipa{v̩˧}}}}\hspace{0.5cm}[\kern2pt{\textcolor{darkblue}{\textbf{\ipa{v̩˩˥}}}}\kern2pt]} \hypertarget{v\string_=\string_M1}{}
\markboth{\textcolor{darkblue}{\textbf{\ipa{v̩˧}}}}{}
\textcolor{teal}{\mytextsc{classificateur}} \hspace{4pt} Ton~: M *.
\textcolor{PineGreen}{\selectlanguage{french}Classificateur pour un individu humain; ne peut s'employer que pour le chiffre 'un', autrement dit soit au singulier, soit avec des nombres dont le dernier chiffre est 'un'.} \zh{量词:人(一个人)。只能用于单数。}  ¶ \textcolor{darkblue}{\textbf{\ipa{ɖɯ˧-v̩\#˥; ɖɯ˧-v̩˧ ɲi˥}}} \textcolor{PineGreen}{\selectlanguage{french}1 personne; c'est 1 personne (élicité pour vérifier le ton)} \zh{一个人,是一个人(为了确认声调而问的短语)}  
 ¶ \textcolor{darkblue}{\textbf{\ipa{tsʰe˧ɖɯ˧-v̩˧}}} \textcolor{PineGreen}{\selectlanguage{french}11 personnes} \zh{十一个人}  
 ¶ \textcolor{darkblue}{\textbf{\ipa{ɲi˧tsi˧ɖɯ˧-v̩˧}}} \textcolor{PineGreen}{\selectlanguage{french}21 personnes} \zh{二十一个人}  
 ¶ \textcolor{darkblue}{\textbf{\ipa{so˧tsʰi˧ɖɯ˧-v̩˧}}} \textcolor{PineGreen}{\selectlanguage{french}31 personnes} \zh{三十一个人}  
 ¶ \textcolor{darkblue}{\textbf{\ipa{ʐv̩˧tsʰi˩ɖɯ˩-v̩˩}}} \textcolor{PineGreen}{\selectlanguage{french}41 personnes} \zh{四十一个人}  
 ¶ \textcolor{darkblue}{\textbf{\ipa{ŋwɤ˧tsʰi˩ɖɯ˩-v̩˩}}} \textcolor{PineGreen}{\selectlanguage{french}51 personnes} \zh{五十一个人}  
 ¶ \textcolor{darkblue}{\textbf{\ipa{qʰv̩˧tsʰi˧ɖɯ˧-v̩˥}}} \textcolor{PineGreen}{\selectlanguage{french}61 personnes} \zh{六十一个人}  
 ¶ \textcolor{darkblue}{\textbf{\ipa{ʂɯ˧tsʰi˩ɖɯ˩-v̩˩}}} \textcolor{PineGreen}{\selectlanguage{french}71 personnes} \zh{七十一个人}  
 ¶ \textcolor{darkblue}{\textbf{\ipa{hõ˧tsʰi˩ɖɯ˩-v̩˩}}} \textcolor{PineGreen}{\selectlanguage{french}81 personnes} \zh{八十一个人}  
 ¶ \textcolor{darkblue}{\textbf{\ipa{gv̩˧tsʰi˩ɖɯ˩-v̩˩}}} \textcolor{PineGreen}{\selectlanguage{french}91 personnes} \zh{九十一个人}  

\lhead{\firstmark}
\rhead{\botmark}

\subsection{\hspace{-0.5cm} {\Large \textcolor{darkblue}{\textbf{\ipa{v̩˥}}}}\hspace{0.5cm}[\kern2pt{\textcolor{darkblue}{\textbf{\ipa{v̩˥}}}}\kern2pt]} \hypertarget{v\string_=\string_T1}{}
\markboth{\textcolor{darkblue}{\textbf{\ipa{v̩˥}}}}{}
\textcolor{teal}{\mytextsc{nom}} \hspace{4pt} Ton~: \#H.
\textcolor{PineGreen}{\selectlanguage{french}Casserole (terme générique).} \zh{锅。}  \zh{量词}: \textcolor{darkblue}{\textbf{\ipa{ɭɯ˧}}}  \mytextsc{clf}~: \textcolor{darkblue}{\textbf{\ipa{ɭɯ˧}}} 
\lhead{\firstmark}
\rhead{\botmark}

\subsection{\hspace{-0.5cm} {\Large \textcolor{darkblue}{\textbf{\ipa{v̩˥\textsubscript{b}}}}}\hspace{0.5cm}[\kern2pt{\textcolor{darkblue}{\textbf{\ipa{v̩˥}}}}\kern2pt]} \hypertarget{v\string_=\string_Tb1}{}
\markboth{\textcolor{darkblue}{\textbf{\ipa{v̩˥\textsubscript{b}}}}}{}
\textcolor{teal}{\mytextsc{classificateur}} \hspace{4pt} Ton~: H\textsubscript{b}.
\textcolor{PineGreen}{\selectlanguage{french}Auto-classificateur des casseroles; et classificateur des casserolées (utilisant la casserole comme mesure de quantité de nourriture, liquide ou solide).} \zh{量词:锅(一口),或锅的容量。} 
\lhead{\firstmark}
\rhead{\botmark}

\subsection{\hspace{-0.5cm} {\Large \textcolor{darkblue}{\textbf{\ipa{v̩˩dze˩}}}}\hspace{0.5cm}[\kern2pt{\textcolor{darkblue}{\textbf{\ipa{v̩˥}}}}\kern2pt]} \hypertarget{v\string_=\string_Bdze\string_B1}{}
\markboth{\textcolor{darkblue}{\textbf{\ipa{v̩˩dze˩}}}}{}
\textcolor{teal}{\mytextsc{nom}} \hspace{4pt} Ton~: L.
\textcolor{PineGreen}{\selectlanguage{french}Oiseau.} \zh{鸟。}  ¶ \textcolor{darkblue}{\textbf{\ipa{v̩˩dze˩-bi˥ | hṽ˧ ʑi˥}}} \textcolor{PineGreen}{\selectlanguage{french}Sur l'oiseau, il y a des plumes.} \zh{鸟(身)上有(羽)毛。}  
 ¶ \textcolor{darkblue}{\textbf{\ipa{v̩˩dze˩-mi˩}}} \textcolor{PineGreen}{\selectlanguage{french}oiseau femelle} \zh{母鸟}  
 ¶ \textcolor{darkblue}{\textbf{\ipa{v̩˩dze˩-pʰv̩˩}}} \textcolor{PineGreen}{\selectlanguage{french}oiseau mâle} \zh{公鸟}  
 ¶ \textcolor{darkblue}{\textbf{\ipa{v̩˩dze˩-zo˩}}} \textcolor{PineGreen}{\selectlanguage{french}petit oiseau} \zh{小鸟}  
 \zh{量词}: \textcolor{darkblue}{\textbf{\ipa{mi˩}}}  \mytextsc{clf}~: \textcolor{darkblue}{\textbf{\ipa{mi˩}}} 
\lhead{\firstmark}
\rhead{\botmark}

\subsection{\hspace{-0.5cm} {\Large \textcolor{darkblue}{\textbf{\ipa{v̩˩dze˩-kʰv̩˩}}}}\hspace{0.5cm}[\kern2pt{\textcolor{darkblue}{\textbf{\ipa{xxxx non-correspondance entre le nombre de morphèmes et le nombre de tons de morphèmes}}}}\kern2pt]} \hypertarget{v\string_=\string_Bdze\string_B-k\string_hv\string_=\string_B1}{}
\markboth{\textcolor{darkblue}{\textbf{\ipa{v̩˩dze˩-kʰv̩˩}}}}{}
\textcolor{teal}{\mytextsc{nom}} \hspace{4pt} Ton~: L.
\textcolor{PineGreen}{\selectlanguage{french}Nid.} \zh{鸟窝,鸟巢。}  ¶ \textcolor{darkblue}{\textbf{\ipa{v̩˩dze˩kʰv̩˩ ɲi˥.}}} \textcolor{PineGreen}{\selectlanguage{french}\string_ \mytextsc{cop}} \zh{是鸟窝}  
 \zh{量词}: \textcolor{darkblue}{\textbf{\ipa{ɭɯ˧}}}  \mytextsc{clf}~: \textcolor{darkblue}{\textbf{\ipa{ɭɯ˧}}} 
\lhead{\firstmark}
\rhead{\botmark}

\subsection{\hspace{-0.5cm} {\Large \textcolor{darkblue}{\textbf{\ipa{v̩˧dʑo\#˥}}}}\hspace{0.5cm}[\kern2pt{\textcolor{darkblue}{\textbf{\ipa{v̩˩dʑo˩˥}}}}\kern2pt]} \hypertarget{v\string_=\string_Mdz£o\#\string_T1}{}
\markboth{\textcolor{darkblue}{\textbf{\ipa{v̩˧dʑo\#˥}}}}{}
\textcolor{teal}{\mytextsc{nom}} \hspace{4pt} Ton~: \#H.
\textcolor{PineGreen}{\selectlanguage{french}Wujiao (nom de village).} \zh{屋脚(村落名)。} 
\lhead{\firstmark}
\rhead{\botmark}

\subsection{\hspace{-0.5cm} {\Large \textcolor{darkblue}{\textbf{\ipa{v̩˩dʑɯ˩}}}}\hspace{0.5cm}[\kern2pt{\textcolor{darkblue}{\textbf{\ipa{v̩˧dʑɯ˧}}}}\kern2pt]} \hypertarget{v\string_=\string_Bdz£M\string_B1}{}
\markboth{\textcolor{darkblue}{\textbf{\ipa{v̩˩dʑɯ˩}}}}{}
\textcolor{teal}{\mytextsc{nom}} \hspace{4pt} Ton~: L.
\textcolor{PineGreen}{\selectlanguage{french}Soupe.} \zh{汤。}  ¶ \textcolor{darkblue}{\textbf{\ipa{æ˩ʂe˧-v̩˥dʑɯ˩}}} \textcolor{PineGreen}{\selectlanguage{french}soupe de poulet} \zh{鸡汤}  

\lhead{\firstmark}
\rhead{\botmark}

\subsection{\hspace{-0.5cm} {\Large \textcolor{darkblue}{\textbf{\ipa{v̩˧ko˧}}}}\hspace{0.5cm}[\kern2pt{\textcolor{darkblue}{\textbf{\ipa{v̩˩ko˩˥}}}}\kern2pt]} \hypertarget{v\string_=\string_Mko\string_M1}{}
\markboth{\textcolor{darkblue}{\textbf{\ipa{v̩˧ko˧}}}}{}
\textcolor{teal}{\mytextsc{nom}} \hspace{4pt} Ton~: M.
\textcolor{PineGreen}{\selectlanguage{french}Tortue.} \zh{乌龟(汉语借词)。} 
\lhead{\firstmark}
\rhead{\botmark}

\subsection{\hspace{-0.5cm} {\Large \textcolor{darkblue}{\textbf{\ipa{v̩˧lɑ˩-ʝi˩}}}}\hspace{0.5cm}[\kern2pt{\textcolor{darkblue}{\textbf{\ipa{xxxx non-correspondance entre le nombre de morphèmes et le nombre de tons de morphèmes}}}}\kern2pt]} \hypertarget{v\string_=\string_MlA\string_B-j££i\string_B1}{}
\markboth{\textcolor{darkblue}{\textbf{\ipa{v̩˧lɑ˩-ʝi˩}}}}{}
\textcolor{teal}{\mytextsc{verbe}} \hspace{4pt} Ton~: L\#-.
\textcolor{PineGreen}{\selectlanguage{french}Faire du commerce.} \zh{做生意。} 
\lhead{\firstmark}
\rhead{\botmark}

\subsection{\hspace{-0.5cm} {\Large \textcolor{darkblue}{\textbf{\ipa{v̩˧lɑ˩-ʝi˩-hĩ˩-hĩ˩}}}}\hspace{0.5cm}[\kern2pt{\textcolor{darkblue}{\textbf{\ipa{xxxx non-correspondance entre le nombre de morphèmes et le nombre de tons de morphèmes}}}}\kern2pt]} \hypertarget{v\string_=\string_MlA\string_B-j££i\string_B-hi\string_~\string_B-hi\string_~\string_B1}{}
\markboth{\textcolor{darkblue}{\textbf{\ipa{v̩˧lɑ˩-ʝi˩-hĩ˩-hĩ˩}}}}{}
\textcolor{teal}{\mytextsc{nom}} \hspace{4pt} Ton~: L\#-.
\textcolor{PineGreen}{\selectlanguage{french}Marchand.} \zh{商人。}  ¶ \textcolor{darkblue}{\textbf{\ipa{v̩˧lɑ˩-ʝi˩-hĩ˩}}} \textcolor{PineGreen}{\selectlanguage{french}marchand} \zh{商人}  
 \zh{量词}: \textcolor{darkblue}{\textbf{\ipa{v̩˧}}}  \mytextsc{clf}~: \textcolor{darkblue}{\textbf{\ipa{v̩˧}}} 
\lhead{\firstmark}
\rhead{\botmark}

\subsection{\hspace{-0.5cm} {\Large \textcolor{darkblue}{\textbf{\ipa{v̩˧mi\#˥}}}}\hspace{0.5cm}[\kern2pt{\textcolor{darkblue}{\textbf{\ipa{xxxx non-correspondance entre le nombre de morphèmes et le nombre de tons de morphèmes}}}}\kern2pt]} \hypertarget{v\string_=\string_Mmi\#\string_T1}{}
\markboth{\textcolor{darkblue}{\textbf{\ipa{v̩˧mi\#˥}}}}{}
\textcolor{teal}{\mytextsc{nom}} \hspace{4pt} Ton~: \#H.
\textcolor{PineGreen}{\selectlanguage{french}Grande casserole.} \zh{大锅。} 
\lhead{\firstmark}
\rhead{\botmark}

\subsection{\hspace{-0.5cm} {\Large \textcolor{darkblue}{\textbf{\ipa{v̩˩tsʰɤ˧˥}}}}\hspace{0.5cm}[\kern2pt{\textcolor{darkblue}{\textbf{\ipa{v̩˧tsʰɤ˧}}}}\kern2pt]} \hypertarget{v\string_=\string_Bts\string_h7\string_M\string_T1}{}
\markboth{\textcolor{darkblue}{\textbf{\ipa{v̩˩tsʰɤ˧˥}}}}{}
\textcolor{teal}{\mytextsc{nom}} \hspace{4pt} Ton~: LM+MH\#.
\textcolor{PineGreen}{\selectlanguage{french}Légumes.} \zh{蔬菜。}  ¶ \textcolor{darkblue}{\textbf{\ipa{[Housebuilding2] v˩tsʰɤ˧-tsʰɑ˧nɑ˥}}} \textcolor{PineGreen}{\selectlanguage{french}légumes frais. Littéralement 'légume de couleur sombre. L'expression ne renvoie pas à une espèce en particulier, mais désigne globalement les légumes verts, par opposition aux légumes conservés, qui perdaient de leur couleur au cours du processus de fermentation.} \zh{新鲜蔬菜。直译:‘绿油油的青菜’。指的不是某种具体的青菜,而是任何新鲜蔬菜,分别于酸菜。制造酸菜的过程中,蔬菜(萝卜等等)褪色:失去原来的深色。}  
 \zh{量词}: \textcolor{darkblue}{\textbf{\ipa{po˧}}}  \mytextsc{clf}~: \textcolor{darkblue}{\textbf{\ipa{po˧}}} 
\lhead{\firstmark}
\rhead{\botmark}

\subsection{\hspace{-0.5cm} {\Large \textcolor{darkblue}{\textbf{\ipa{v̩˩tsʰɤ˧-bv̩\#˥}}}}\hspace{0.5cm}[\kern2pt{\textcolor{darkblue}{\textbf{\ipa{xxxx non-correspondance entre le nombre de morphèmes et le nombre de tons de morphèmes}}}}\kern2pt]} \hypertarget{v\string_=\string_Bts\string_h7\string_M-bv\string_=\#\string_T1}{}
\markboth{\textcolor{darkblue}{\textbf{\ipa{v̩˩tsʰɤ˧-bv̩\#˥}}}}{}
\textcolor{teal}{\mytextsc{nom}} \hspace{4pt} Ton~: LM+\#H.
\textcolor{PineGreen}{\selectlanguage{french}Coccinelle.} \zh{瓢虫。}  \zh{量词}: \textcolor{darkblue}{\textbf{\ipa{mi˩}}}  \mytextsc{clf}~: \textcolor{darkblue}{\textbf{\ipa{mi˩}}} 
\lhead{\firstmark}
\rhead{\botmark}

\subsection{\hspace{-0.5cm} {\Large \textcolor{darkblue}{\textbf{\ipa{v̩˩tsʰɤ˧-pʰv̩˥}}}}\hspace{0.5cm}[\kern2pt{\textcolor{darkblue}{\textbf{\ipa{xxxx non-correspondance entre le nombre de morphèmes et le nombre de tons de morphèmes}}}}\kern2pt]} \hypertarget{v\string_=\string_Bts\string_h7\string_M-p\string_hv\string_=\string_T1}{}
\markboth{\textcolor{darkblue}{\textbf{\ipa{v̩˩tsʰɤ˧-pʰv̩˥}}}}{}
\textcolor{teal}{\mytextsc{nom}} \hspace{4pt} Ton~: LM+H\#.
\textcolor{PineGreen}{\selectlanguage{french}Chou chinois.} \zh{白菜。}  \zh{量词}: \textcolor{darkblue}{\textbf{\ipa{po˧}}}  \mytextsc{clf}~: \textcolor{darkblue}{\textbf{\ipa{po˧}}} \textit{Voir~:} \hyperlink{}{\textcolor{darkblue}{\textbf{\ipa{tsʰæ˧pʰv˧˥}}}} 
\lhead{\firstmark}
\rhead{\botmark}

\subsection{\hspace{-0.5cm} {\Large \textcolor{darkblue}{\textbf{\ipa{v̩˩tsʰɤ˧-v̩˥ɲi˩}}}}\hspace{0.5cm}[\kern2pt{\textcolor{darkblue}{\textbf{\ipa{xxxx non-correspondance entre le nombre de morphèmes et le nombre de tons de morphèmes}}}}\kern2pt]} \hypertarget{v\string_=\string_Bts\string_h7\string_M-v\string_=\string_TJi\string_B1}{}
\markboth{\textcolor{darkblue}{\textbf{\ipa{v̩˩tsʰɤ˧-v̩˥ɲi˩}}}}{}
\textcolor{teal}{\mytextsc{nom}} \hspace{4pt} Ton~: LM+\#H-.
\textcolor{PineGreen}{\selectlanguage{french}Légumes.} \zh{蔬菜。}  \zh{量词}: \textcolor{darkblue}{\textbf{\ipa{qɑ˩}}}  \mytextsc{clf}~: \textcolor{darkblue}{\textbf{\ipa{qɑ˩}}} \textcolor{PineGreen}{\selectlanguage{french}botte}
\lhead{\firstmark}
\rhead{\botmark}

\subsection{\hspace{-0.5cm} {\Large \textcolor{darkblue}{\textbf{\ipa{v̩˧zo\#˥}}}}\hspace{0.5cm}[\kern2pt{\textcolor{darkblue}{\textbf{\ipa{xxxx non-correspondance entre le nombre de morphèmes et le nombre de tons de morphèmes}}}}\kern2pt]} \hypertarget{v\string_=\string_Mzo\#\string_T1}{}
\markboth{\textcolor{darkblue}{\textbf{\ipa{v̩˧zo\#˥}}}}{}
\textcolor{teal}{\mytextsc{nom}} \hspace{4pt} Ton~: \#H.
\textcolor{PineGreen}{\selectlanguage{french}Petite casserole.} \zh{小锅。} 
\lhead{\firstmark}
\rhead{\botmark}

\subsection{\hspace{-0.5cm} {\Large \textcolor{darkblue}{\textbf{\ipa{v̩˧\textasciitilde{}v̩˧\textsubscript{a}}}}}\hspace{0.5cm}[\kern2pt{\textcolor{darkblue}{\textbf{\ipa{v̩˧v̩˧}}}}\kern2pt]} \hypertarget{v\string_=\string_M~v\string_=\string_Ma1}{}
\markboth{\textcolor{darkblue}{\textbf{\ipa{v̩˧\textasciitilde{}v̩˧\textsubscript{a}}}}}{}
\textcolor{teal}{\mytextsc{verbe}} \hspace{4pt} Ton~: M\textsubscript{a}.
\textcolor{PineGreen}{\selectlanguage{french}Mâcher.} \zh{嚼。}  ¶ \textcolor{darkblue}{\textbf{\ipa{le˧-v̩˧\textasciitilde{}v̩˧ +ze˩}}} \textcolor{PineGreen}{\selectlanguage{french}\mytextsc{accomp}} \zh{\mytextsc{accomp}}  
 ¶ \textcolor{darkblue}{\textbf{\ipa{le˧-wo˧ v̩˧\textasciitilde{}v̩˧}}} \textcolor{PineGreen}{\selectlanguage{french}ruminer (la vache rumine)} \zh{反刍}  

\lhead{\firstmark}
\rhead{\botmark}

\newpage
\section*{\centering- \textcolor{darkblue}{\textbf{\ipa{w}}} \textcolor{darkblue}{\textbf{\ipa{wɑ}}} \textcolor{darkblue}{\textbf{\ipa{wæ}}} \textcolor{darkblue}{\textbf{\ipa{wɤ}}} \textcolor{darkblue}{\textbf{\ipa{wo}}} \textcolor{darkblue}{\textbf{\ipa{wɤ̃}}} -}
\subsection{\hspace{-0.5cm} {\Large \textcolor{darkblue}{\textbf{\ipa{wɤ˧}}} \textsubscript{1}}\hspace{0.5cm}[\kern2pt{\textcolor{darkblue}{\textbf{\ipa{wɤ˩˥}}}}\kern2pt]} \hypertarget{w7\string_M1}{}
\markboth{\textcolor{darkblue}{\textbf{\ipa{wɤ˧}}} \textsubscript{1}}{}
\textcolor{teal}{\mytextsc{nom}} \hspace{4pt} Ton~: M.
\textcolor{PineGreen}{\selectlanguage{french}Serf, esclave (la plus basse caste de la société ancienne).} \zh{奴隶,农奴。音译:“俄”。}  \zh{量词}: \textcolor{darkblue}{\textbf{\ipa{v̩˧}}}  \mytextsc{clf}~: \textcolor{darkblue}{\textbf{\ipa{v̩˧}}} 
\lhead{\firstmark}
\rhead{\botmark}

\subsection{\hspace{-0.5cm} {\Large \textcolor{darkblue}{\textbf{\ipa{wɤ˧}}} \textsubscript{2}}\hspace{0.5cm}[\kern2pt{\textcolor{darkblue}{\textbf{\ipa{wɤ˥}}}}\kern2pt]} \hypertarget{w7\string_M2}{}
\markboth{\textcolor{darkblue}{\textbf{\ipa{wɤ˧}}} \textsubscript{2}}{}
\textcolor{teal}{\mytextsc{particule}} \textcolor{teal}{\mytextsc{de}} \textcolor{teal}{\mytextsc{discours}} \hspace{4pt} Ton~: M.
\textcolor{PineGreen}{\selectlanguage{french}Particule finale exclamative, avec une nuance d'évidence.} \zh{句尾助词:吧、呗。} 
\lhead{\firstmark}
\rhead{\botmark}

\subsection{\hspace{-0.5cm} {\Large \textcolor{darkblue}{\textbf{\ipa{wɤ˩\textsubscript{a}}}}}\hspace{0.5cm}[\kern2pt{\textcolor{darkblue}{\textbf{\ipa{wɤ˥}}}}\kern2pt]} \hypertarget{w7\string_Ba1}{}
\markboth{\textcolor{darkblue}{\textbf{\ipa{wɤ˩\textsubscript{a}}}}}{}
\textcolor{teal}{\mytextsc{verbe}} \hspace{4pt} Ton~: L\textsubscript{a}.
\textcolor{PineGreen}{\selectlanguage{french}Dépendre de, se reposer sur.} \zh{依赖。}  ¶ \textcolor{darkblue}{\textbf{\ipa{hĩ˧-bi˥ | wɤ˩-mɤ˩-bi˩˥!}}} \textcolor{PineGreen}{\selectlanguage{french}Il ne faut pas dépendre des autres!} \zh{不要依赖别人!}  
 ¶ \textcolor{darkblue}{\textbf{\ipa{hĩ˧-bi˥ | wɤ˩-v̩˩-tʰv̩˩˥!}}} \textcolor{PineGreen}{\selectlanguage{french}On se trouve dépendre des autres!} \zh{(无论如何)人都会依靠别人的!(意思是:人不能完全独立,人活在人间就会或多或少需要依靠别人。)}  

\lhead{\firstmark}
\rhead{\botmark}

\subsection{\hspace{-0.5cm} {\Large \textcolor{darkblue}{\textbf{\ipa{wɤ˩\textsubscript{b}}}}}\hspace{0.5cm}[\kern2pt{\textcolor{darkblue}{\textbf{\ipa{wɤ˩˥}}}}\kern2pt]} \hypertarget{w7\string_Bb1}{}
\markboth{\textcolor{darkblue}{\textbf{\ipa{wɤ˩\textsubscript{b}}}}}{}
\textcolor{teal}{\mytextsc{classificateur}} \hspace{4pt} Ton~: L\textsubscript{b}.
\textcolor{PineGreen}{\selectlanguage{french}Classificateur des charges/fardeaux qu'une personne peut porter.} \zh{量词:担,负荷。}  ¶ \textcolor{darkblue}{\textbf{\ipa{ɖɯ˧-wɤ˩ pɤ˩\textasciitilde{}pɤ˩ |}}} \textcolor{PineGreen}{\selectlanguage{french}porter une charge} \zh{背一担}  
 ¶ \textcolor{darkblue}{\textbf{\ipa{ɖɯ˧-wɤ˩, | ɖɯ˧-wɤ˩ | le˧-kʰɯ˩\textasciitilde{}kʰɯ˩ | tʰi˧-tɕɯ˥ |}}} \textcolor{PineGreen}{\selectlanguage{french}entasser les charges, l'une après l'autre} \zh{将驮的大包堆起来}  

\lhead{\firstmark}
\rhead{\botmark}

\subsection{\hspace{-0.5cm} {\Large \textcolor{darkblue}{\textbf{\ipa{wɤ˩\textasciitilde{}wɤ˩}}}}\hspace{0.5cm}[\kern2pt{\textcolor{darkblue}{\textbf{\ipa{wɤ˩wɤ˥}}}}\kern2pt]} \hypertarget{w7\string_B~w7\string_B1}{}
\markboth{\textcolor{darkblue}{\textbf{\ipa{wɤ˩\textasciitilde{}wɤ˩}}}}{}
\textcolor{teal}{\mytextsc{verbe}} \hspace{4pt} Ton~: L.
\textcolor{PineGreen}{\selectlanguage{french}Contourner.} \zh{绕过。}  ¶ \textcolor{darkblue}{\textbf{\ipa{le˧-wɤ˩-ze˩}}} \textcolor{PineGreen}{\selectlanguage{french}\mytextsc{accomp} \string_ pfv} \zh{绕了}  
 ¶ \textcolor{darkblue}{\textbf{\ipa{ɖɯ˧-wɤ˩\textasciitilde{}wɤ˩-ɻ̍˩}}} \textcolor{PineGreen}{\selectlanguage{french}\mytextsc{délimitatif} \string_ \mytextsc{red} \mytextsc{inchoatif}} \zh{绕一绕}  
 ¶ \textcolor{darkblue}{\textbf{\ipa{[PHONO] wɤ˩\textasciitilde{}wɤ˩ bi˩˥}}} \textcolor{PineGreen}{\selectlanguage{french}\mytextsc{fut}\string_imm} \zh{\mytextsc{imm}\string_fut}  
 ¶ \textcolor{darkblue}{\textbf{\ipa{[PHONO] wɤ˩\textasciitilde{}wɤ˩-ze˥}}} \textcolor{PineGreen}{\selectlanguage{french}\mytextsc{pfv}} \zh{绕了}  
 ¶ \textcolor{darkblue}{\textbf{\ipa{le˧-wɤ˩\textasciitilde{}wɤ˩ | le˧-se˥}}} \textcolor{PineGreen}{\selectlanguage{french}contourner à pied} \zh{走路绕过}  

\lhead{\firstmark}
\rhead{\botmark}

\subsection{\hspace{-0.5cm} {\Large \textcolor{darkblue}{\textbf{\ipa{wɤ˩˥}}}}\hspace{0.5cm}[\kern2pt{\textcolor{darkblue}{\textbf{\ipa{wɤ˥}}}}\kern2pt]} \hypertarget{w7\string_B\string_T1}{}
\markboth{\textcolor{darkblue}{\textbf{\ipa{wɤ˩˥}}}}{}
\textcolor{teal}{\mytextsc{adverbe}} \hspace{4pt} Ton~: LM? LH?.
\textcolor{PineGreen}{\selectlanguage{french}À nouveau, encore; aussi.} \zh{又,再。}  ¶ \textcolor{darkblue}{\textbf{\ipa{wɤ˩˥ | ɖɯ˧-ʂɯ˩}}} \textcolor{PineGreen}{\selectlanguage{french}une nouvelle fois, une fois de plus} \zh{再一次、又一次}  

\lhead{\firstmark}
\rhead{\botmark}

\subsection{\hspace{-0.5cm} {\Large \textcolor{darkblue}{\textbf{\ipa{wo˥}}}}\hspace{0.5cm}[\kern2pt{\textcolor{darkblue}{\textbf{\ipa{wo˧˥}}}}\kern2pt]} \hypertarget{wo\string_T1}{}
\markboth{\textcolor{darkblue}{\textbf{\ipa{wo˥}}}}{}
\textcolor{teal}{\mytextsc{adjectif}} \hspace{4pt} Ton~: H.
\textcolor{PineGreen}{\selectlanguage{french}Dur, solide, résistant.} \zh{硬,坚硬,结实。}  ¶ \textcolor{darkblue}{\textbf{\ipa{le˧-wo˥-ze˩}}} \textcolor{PineGreen}{\selectlanguage{french}\mytextsc{accomp} \string_ \mytextsc{pfv}: ça a durci, c'est devenu dur} \zh{硬了}  

\lhead{\firstmark}
\rhead{\botmark}

\subsection{\hspace{-0.5cm} {\Large \textcolor{darkblue}{\textbf{\ipa{wo˩\textsubscript{b}}}}}\hspace{0.5cm}[\kern2pt{\textcolor{darkblue}{\textbf{\ipa{wo˩˥}}}}\kern2pt]} \hypertarget{wo\string_Bb1}{}
\markboth{\textcolor{darkblue}{\textbf{\ipa{wo˩\textsubscript{b}}}}}{}
\textcolor{teal}{\mytextsc{classificateur}} \hspace{4pt} Ton~: L\textsubscript{b}.
\textcolor{PineGreen}{\selectlanguage{french}Classificateur des paires de bœufs; attelage de bœufs pour tirer l'araire. A Yongning, l'attelage comporte deux bœufs, ou deux petits buffles, ou un seul buffle vigoureux.} \zh{量词:牛(一架)。}  ¶ \textcolor{darkblue}{\textbf{\ipa{dʑi˧mi˧ | ɲi˧-pʰo˧˥, | ɖɯ˧-wo˩!}}} \textcolor{PineGreen}{\selectlanguage{french}Deux buffles, cela forme un attelage!} \zh{两头水牛,等于一架!}  

\lhead{\firstmark}
\rhead{\botmark}

\subsection{\hspace{-0.5cm} {\Large \textcolor{darkblue}{\textbf{\ipa{wo˩kɤ\#˥}}}}\hspace{0.5cm}[\kern2pt{\textcolor{darkblue}{\textbf{\ipa{wo˩kɤ˩˥}}}}\kern2pt]} \hypertarget{wo\string_Bk7\#\string_T1}{}
\markboth{\textcolor{darkblue}{\textbf{\ipa{wo˩kɤ\#˥}}}}{}
\textcolor{teal}{\mytextsc{nom}} \hspace{4pt} Ton~: LM+\#H.
\textcolor{PineGreen}{\selectlanguage{french}Balançoire.} \zh{秋千(鞦韆)。}  ¶ \textcolor{darkblue}{\textbf{\ipa{wo˩kɤ˧-tsɑ˧-di˧˥}}} \textcolor{PineGreen}{\selectlanguage{french}même sens: balançoire} \zh{同上:秋千}  
 ¶ \textcolor{darkblue}{\textbf{\ipa{wo˩kɤ˧ tsɑ˧˥}}} \textcolor{PineGreen}{\selectlanguage{french}même sens: balançoire} \zh{同上:秋千}  
 \zh{量词}: \textcolor{darkblue}{\textbf{\ipa{nɑ˧}}}  \mytextsc{clf}~: \textcolor{darkblue}{\textbf{\ipa{nɑ˧}}} 
\lhead{\firstmark}
\rhead{\botmark}

\subsection{\hspace{-0.5cm} {\Large \textcolor{darkblue}{\textbf{\ipa{wo˧˥}}}}\hspace{0.5cm}[\kern2pt{\textcolor{darkblue}{\textbf{\ipa{wo˩˥}}}}\kern2pt]} \hypertarget{wo\string_M\string_T1}{}
\markboth{\textcolor{darkblue}{\textbf{\ipa{wo˧˥}}}}{}
\textcolor{teal}{\mytextsc{verbe}} \hspace{4pt} Ton~: MH.
\textcolor{PineGreen}{\selectlanguage{french}Se retourner (quelqu'un est assis et se retourne: mouvement du torse).} \zh{重新做、再来做。}  ¶ \textcolor{darkblue}{\textbf{\ipa{le˧-wo˧ ʐwɤ˧˥}}} \textcolor{PineGreen}{\selectlanguage{french}répondre, donner une réponse} \zh{回答}  
 ¶ \textcolor{darkblue}{\textbf{\ipa{le˧-wo˧-ɻ̍˥}}} \textcolor{PineGreen}{\selectlanguage{french}se retourner} \zh{转身}  
 ¶ \textcolor{darkblue}{\textbf{\ipa{le˧-wo˧ li˥}}} \textcolor{PineGreen}{\selectlanguage{french}regarder derrière soi} \zh{往后看}  
 ¶ \textcolor{darkblue}{\textbf{\ipa{lə-˧wo˧ tʰo˥-tɕo˩}}} \textcolor{PineGreen}{\selectlanguage{french}se retourner} \zh{转身}  
 ¶ \textcolor{darkblue}{\textbf{\ipa{le˧-wo˧-tɕo˥!}}} \textcolor{PineGreen}{\selectlanguage{french}retourne-toi! (adressé à un bébé qui s'apprête à descendre d'un lit la tête la première)} \zh{转身!(婴儿爬下床,头朝下。奶奶告诉她:要先转身)}  
 ¶ \textcolor{darkblue}{\textbf{\ipa{le˧-wo˧˥ | le˧-hɯ˩}}} \textcolor{PineGreen}{\selectlanguage{french}...est reparti} \zh{回去了}  

\lhead{\firstmark}
\rhead{\botmark}

\subsection{\hspace{-0.5cm} {\Large \textcolor{darkblue}{\textbf{\ipa{wo˩˥}}}}\hspace{0.5cm}[\kern2pt{\textcolor{darkblue}{\textbf{\ipa{wo˥}}}}\kern2pt]} \hypertarget{wo\string_B\string_T1}{}
\markboth{\textcolor{darkblue}{\textbf{\ipa{wo˩˥}}}}{}
\textcolor{teal}{\mytextsc{nom}} \hspace{4pt} Ton~: LH.
\textcolor{PineGreen}{\selectlanguage{french}Feuilles du navet.} \zh{圆根的叶子。}  ¶ \textcolor{darkblue}{\textbf{\ipa{wo˩bɤ˧˥}}} \textcolor{PineGreen}{\selectlanguage{french}même sens: feuilles du navet} \zh{同上:圆根叶子}  
 ¶ \textcolor{darkblue}{\textbf{\ipa{[Housebuilding2] wo˩-v˥tsʰɤ˩}}} \textcolor{PineGreen}{\selectlanguage{french}même sens: feuilles du navet; littéralement 'légume-feuilles du navet'; l'expression souligne qu'il s'agit d'une variété de légume: d'un ingrédient pour la cuisine} \zh{同上:圆根叶子}  
 ¶ \textcolor{darkblue}{\textbf{\ipa{[Housebuilding2] wo˩-tɕæ˩ɻæ˥}}} \textcolor{PineGreen}{\selectlanguage{french}feuilles du navet conservées dans la saumure} \zh{圆根叶子酸菜}  

\lhead{\firstmark}
\rhead{\botmark}

\newpage
\section*{\centering- \textcolor{darkblue}{\textbf{\ipa{w̃}}} \textcolor{darkblue}{\textbf{\ipa{w̃æ}}} -}
\subsection{\hspace{-0.5cm} {\Large \textcolor{darkblue}{\textbf{\ipa{w̃æ˧}}}}\hspace{0.5cm}[\kern2pt{\textcolor{darkblue}{\textbf{\ipa{w̃æ˥}}}}\kern2pt]} \hypertarget{w\string_~\{\string_M1}{}
\markboth{\textcolor{darkblue}{\textbf{\ipa{w̃æ˧}}}}{}
\textcolor{teal}{\mytextsc{verbe}} \hspace{4pt} Ton~: M intrans.
\textcolor{PineGreen}{\selectlanguage{french}Se gonfler, enfler (ventre).} \zh{肿,膨胀,(肚子)胀。}  ¶ \textcolor{darkblue}{\textbf{\ipa{ɻ̍˧tɑ˧ w̃æ˧ (-ze˧)}}} \textcolor{PineGreen}{\selectlanguage{french}les ganglions sont enflés} \zh{淋巴结肿了}  
 ¶ \textcolor{darkblue}{\textbf{\ipa{tso˧\textasciitilde{}tso˧ w̃æ˩}}} \textcolor{PineGreen}{\selectlanguage{french}quelque chose a enflé} \zh{东西膨胀了}  

\lhead{\firstmark}
\rhead{\botmark}

\newpage
\section*{\centering- \textcolor{darkblue}{\textbf{\ipa{z}}} -}
\subsection{\hspace{-0.5cm} {\Large \textcolor{darkblue}{\textbf{\ipa{zɑ˥}}}}\hspace{0.5cm}[\kern2pt{\textcolor{darkblue}{\textbf{\ipa{xxxx groupe tonal entier sans aucun ton}}}}\kern2pt]} \hypertarget{zA\string_T1}{}
\markboth{\textcolor{darkblue}{\textbf{\ipa{zɑ˥}}}}{}
\textcolor{teal}{\mytextsc{adjectif}} \hspace{4pt} Ton~: H.
\textcolor{PineGreen}{\selectlanguage{french}Limité à, restreint à (en tournure négative).} \zh{仅仅。}  ¶ \textcolor{darkblue}{\textbf{\ipa{ʁwɤ˧-qo˧-ɳɯ˧-lɑ˧ mɤ˧-zɑ˥ (…)}}} \textcolor{PineGreen}{\selectlanguage{french}Il n'y avait pas que les gens du village (…) (récit “Funeral”)} \zh{不仅有村子里的人}  

\lhead{\firstmark}
\rhead{\botmark}

\subsection{\hspace{-0.5cm} {\Large \textcolor{darkblue}{\textbf{\ipa{zɑ˧ɭɯ˧}}}}\hspace{0.5cm}[\kern2pt{\textcolor{darkblue}{\textbf{\ipa{xxxx non-correspondance entre le nombre de morphèmes et le nombre de tons de morphèmes}}}}\kern2pt]} \hypertarget{zA\string_Ml\string_RM\string_M1}{}
\markboth{\textcolor{darkblue}{\textbf{\ipa{zɑ˧ɭɯ˧}}}}{}
\textcolor{teal}{\mytextsc{nom}} \hspace{4pt} Ton~: M.
\textcolor{PineGreen}{\selectlanguage{french}Porc castré.} \zh{阉猪。}  \zh{量词}: \textcolor{darkblue}{\textbf{\ipa{pʰo˧˥}}} \textcolor{darkblue}{\textbf{\ipa{v̩˧}}}  \mytextsc{clf}~: \textcolor{darkblue}{\textbf{\ipa{pʰo˧˥}}} \textcolor{darkblue}{\textbf{\ipa{v̩˧}}} 
\lhead{\firstmark}
\rhead{\botmark}

\subsection{\hspace{-0.5cm} {\Large \textcolor{darkblue}{\textbf{\ipa{zɑ˧zɑ˧}}}}\hspace{0.5cm}[\kern2pt{\textcolor{darkblue}{\textbf{\ipa{zɑ˧zɑ˧}}}}\kern2pt]} \hypertarget{zA\string_MzA\string_M1}{}
\markboth{\textcolor{darkblue}{\textbf{\ipa{zɑ˧zɑ˧}}}}{}
\textcolor{teal}{\mytextsc{adjectif}} \hspace{4pt} Ton~: M.
\textcolor{PineGreen}{\selectlanguage{french}Attentif, soigneux.} \zh{细心、细致。} 
\lhead{\firstmark}
\rhead{\botmark}

\subsection{\hspace{-0.5cm} {\Large \textcolor{darkblue}{\textbf{\ipa{zɑ˩\textsubscript{a}}}}}\hspace{0.5cm}[\kern2pt{\textcolor{darkblue}{\textbf{\ipa{zɑ˥}}}}\kern2pt]} \hypertarget{zA\string_Ba1}{}
\markboth{\textcolor{darkblue}{\textbf{\ipa{zɑ˩\textsubscript{a}}}}}{}
\textcolor{teal}{\mytextsc{verbe}} \hspace{4pt} Ton~: L\textsubscript{a}.
\textcolor{PineGreen}{\selectlanguage{french}Descendre (redescendre de la montagne).} \zh{下(山……)。}  ¶ \textcolor{darkblue}{\textbf{\ipa{ʁwɤ˩ zɑ˩˥}}} \textcolor{PineGreen}{\selectlanguage{french}descendre de la montagne} \zh{下山}  
 ¶ \textcolor{darkblue}{\textbf{\ipa{mɤ˧-zɑ˩-sɯ˩}}} \textcolor{PineGreen}{\selectlanguage{french}ne pas descendre encore} \zh{还没下来}  
 ¶ \textcolor{darkblue}{\textbf{\ipa{ɖɯ˧-zɑ˧\textasciitilde{}zɑ˥-ɻ̍˩}}} \textcolor{PineGreen}{\selectlanguage{french}\mytextsc{délimitatif} \string_ \mytextsc{red} \mytextsc{inchoatif}} \zh{下来一下}  

\lhead{\firstmark}
\rhead{\botmark}

\subsection{\hspace{-0.5cm} {\Large \textcolor{darkblue}{\textbf{\ipa{zɑ˩-bɑ˧lɑ˩}}}}\hspace{0.5cm}[\kern2pt{\textcolor{darkblue}{\textbf{\ipa{xxxx non-correspondance entre le nombre de morphèmes et le nombre de tons de morphèmes}}}}\kern2pt]} \hypertarget{zA\string_B-bA\string_MlA\string_B1}{}
\markboth{\textcolor{darkblue}{\textbf{\ipa{zɑ˩-bɑ˧lɑ˩}}}}{}
\textcolor{teal}{\mytextsc{nom}} \hspace{4pt} Ton~: L-L\#.
\ding{202} \textcolor{PineGreen}{\selectlanguage{french}Peinture religieuse (thangka) sur bois, dans une niche sur le mur, au-dessus du foyer.} \zh{火塘旁边墙上的壁画(唐卡:内容来自藏传佛教)。} \ding{203} \textcolor{PineGreen}{\selectlanguage{french}Divinité du feu, du foyer et de la maison.} \zh{火,火塘与家的神。} 
\lhead{\firstmark}
\rhead{\botmark}

\subsection{\hspace{-0.5cm} {\Large \textcolor{darkblue}{\textbf{\ipa{zɑ˩ɲi˥-ʂɤ˩}}}}\hspace{0.5cm}[\kern2pt{\textcolor{darkblue}{\textbf{\ipa{xxxx non-correspondance entre le nombre de morphèmes et le nombre de tons de morphèmes}}}}\kern2pt]} \hypertarget{zA\string_BJi\string_T-s`7\string_B1}{}
\markboth{\textcolor{darkblue}{\textbf{\ipa{zɑ˩ɲi˥-ʂɤ˩}}}}{}
\textcolor{teal}{\mytextsc{nom}} \hspace{4pt} Ton~: LH-.
\textcolor{PineGreen}{\selectlanguage{french}Vampire; démon malfaisant de forme humaine (de la taille d'un humain de grande taille), qui ne mange pas de viande, et se nourrit de sang.} \zh{吸血鬼。} 
\lhead{\firstmark}
\rhead{\botmark}

\subsection{\hspace{-0.5cm} {\Large \textcolor{darkblue}{\textbf{\ipa{‑ze˧}}}}\hspace{0.5cm}[\kern2pt{\textcolor{darkblue}{\textbf{\ipa{ze˥}}}}\kern2pt]} \hypertarget{‑ze\string_M1}{}
\markboth{\textcolor{darkblue}{\textbf{\ipa{‑ze˧}}}}{}
\textcolor{teal}{\mytextsc{suffixe}} \hspace{4pt} Ton~: M.
\textcolor{PineGreen}{\selectlanguage{french}Perfectif, \mytextsc{pfv}.} \zh{\mytextsc{整体体。}} 
\lhead{\firstmark}
\rhead{\botmark}

\subsection{\hspace{-0.5cm} {\Large \textcolor{darkblue}{\textbf{\ipa{ze˩}}}}\hspace{0.5cm}[\kern2pt{\textcolor{darkblue}{\textbf{\ipa{ze˩˥}}}}\kern2pt]} \hypertarget{ze\string_B1}{}
\markboth{\textcolor{darkblue}{\textbf{\ipa{ze˩}}}}{}
\textcolor{teal}{\mytextsc{pronom}} \hspace{4pt} Ton~: L.
\textcolor{PineGreen}{\selectlanguage{french}Quel, lequel.} \zh{哪。} 
\lhead{\firstmark}
\rhead{\botmark}

\subsection{\hspace{-0.5cm} {\Large \textcolor{darkblue}{\textbf{\ipa{ze˩bæ˧}}}}\hspace{0.5cm}[\kern2pt{\textcolor{darkblue}{\textbf{\ipa{ze˩bæ˥}}}}\kern2pt]} \hypertarget{ze\string_Bb\{\string_M1}{}
\markboth{\textcolor{darkblue}{\textbf{\ipa{ze˩bæ˧}}}}{}
\textcolor{teal}{\mytextsc{pronom}} \hspace{4pt} Ton~: LM.
\textcolor{PineGreen}{\selectlanguage{french}Quelle sorte de, lequel.} \zh{哪,哪个 (哪个碗),哪一种。}  ¶ \textcolor{darkblue}{\textbf{\ipa{ze˩bæ˧ ɲi˥?}}} \textcolor{PineGreen}{\selectlanguage{french}c'est lequel?/c'est de quelle sorte?} \zh{是哪个?是哪一样?}  

\lhead{\firstmark}
\rhead{\botmark}

\subsection{\hspace{-0.5cm} {\Large \textcolor{darkblue}{\textbf{\ipa{ze˩bæ˩}}}}\hspace{0.5cm}[\kern2pt{\textcolor{darkblue}{\textbf{\ipa{ze˩bæ˩˥}}}}\kern2pt]} \hypertarget{ze\string_Bb\{\string_B1}{}
\markboth{\textcolor{darkblue}{\textbf{\ipa{ze˩bæ˩}}}}{}
\textcolor{teal}{\mytextsc{nom}} \hspace{4pt} Ton~: L.
\textcolor{PineGreen}{\selectlanguage{french}Éclair.} \zh{闪电、打闪电、霹雷。}  ¶ \textcolor{darkblue}{\textbf{\ipa{ze˩bæ˩-ze˥!}}} \textcolor{PineGreen}{\selectlanguage{french}il y a eu un éclair!} \zh{打闪电了!}  
 ¶ \textcolor{darkblue}{\textbf{\ipa{ze˩bæ˩˥ | -dʑo˩!}}} \textcolor{PineGreen}{\selectlanguage{french}il y a des éclairs!} \zh{打着闪电!}  
 \zh{量词}: \textcolor{darkblue}{\textbf{\ipa{bæ˩}}}  \mytextsc{clf}~: \textcolor{darkblue}{\textbf{\ipa{bæ˩}}} 
\lhead{\firstmark}
\rhead{\botmark}

\subsection{\hspace{-0.5cm} {\Large \textcolor{darkblue}{\textbf{\ipa{ze˩gɤ˧}}}}\hspace{0.5cm}[\kern2pt{\textcolor{darkblue}{\textbf{\ipa{ze˩gɤ˥}}}}\kern2pt]} \hypertarget{ze\string_Bg7\string_M1}{}
\markboth{\textcolor{darkblue}{\textbf{\ipa{ze˩gɤ˧}}}}{}
\textcolor{teal}{\mytextsc{pronom}} \hspace{4pt} Ton~: LM.
\textcolor{PineGreen}{\selectlanguage{french}Où, à quel endroit.} \zh{哪里,什么地方。} 
\lhead{\firstmark}
\rhead{\botmark}

\subsection{\hspace{-0.5cm} {\Large \textcolor{darkblue}{\textbf{\ipa{ze˩mi˩}}}}\hspace{0.5cm}[\kern2pt{\textcolor{darkblue}{\textbf{\ipa{ze˩mi˩˥}}}}\kern2pt]} \hypertarget{ze\string_Bmi\string_B1}{}
\markboth{\textcolor{darkblue}{\textbf{\ipa{ze˩mi˩}}}}{}
\textcolor{teal}{\mytextsc{nom}} \hspace{4pt} Ton~: L.
\textcolor{PineGreen}{\selectlanguage{french}Nièce (enfant d'une soeur).} \zh{甥女(姐妹的女儿)。}  \zh{量词}: \textcolor{darkblue}{\textbf{\ipa{v̩˧}}}  \mytextsc{clf}~: \textcolor{darkblue}{\textbf{\ipa{v̩˧}}} 
\lhead{\firstmark}
\rhead{\botmark}

\subsection{\hspace{-0.5cm} {\Large \textcolor{darkblue}{\textbf{\ipa{ze˩v̩˩}}}}\hspace{0.5cm}[\kern2pt{\textcolor{darkblue}{\textbf{\ipa{ze˩v̩˩˥}}}}\kern2pt]} \hypertarget{ze\string_Bv\string_=\string_B1}{}
\markboth{\textcolor{darkblue}{\textbf{\ipa{ze˩v̩˩}}}}{}
\textcolor{teal}{\mytextsc{nom}} \hspace{4pt} Ton~: L.
\textcolor{PineGreen}{\selectlanguage{french}Neveu (fils d'une sœur).} \zh{外甥(姐妹的儿子)。}  \zh{量词}: \textcolor{darkblue}{\textbf{\ipa{v̩˧}}}  \mytextsc{clf}~: \textcolor{darkblue}{\textbf{\ipa{v̩˧}}} 
\lhead{\firstmark}
\rhead{\botmark}

\subsection{\hspace{-0.5cm} {\Large \textcolor{darkblue}{\textbf{\ipa{ze˩v̩˩-ze˧mi˩}}}}\hspace{0.5cm}[\kern2pt{\textcolor{darkblue}{\textbf{\ipa{ze˩v̩˩ze˧mi˩}}}}\kern2pt]} \hypertarget{ze\string_Bv\string_=\string_B-ze\string_Mmi\string_B1}{}
\markboth{\textcolor{darkblue}{\textbf{\ipa{ze˩v̩˩-ze˧mi˩}}}}{}
\textcolor{teal}{\mytextsc{nom}} \hspace{4pt} Ton~: L-L\#.
\textcolor{PineGreen}{\selectlanguage{french}Neveux et nièces (du côté des sœurs: enfants des sœurs).} \zh{外甥甥女(姐妹的儿女)。} 
\lhead{\firstmark}
\rhead{\botmark}

\subsection{\hspace{-0.5cm} {\Large \textcolor{darkblue}{\textbf{\ipa{zo˥}}}}\hspace{0.5cm}[\kern2pt{\textcolor{darkblue}{\textbf{\ipa{zo˩˥}}}}\kern2pt]} \hypertarget{zo\string_T1}{}
\markboth{\textcolor{darkblue}{\textbf{\ipa{zo˥}}}}{}
\textcolor{teal}{\mytextsc{nom}} \hspace{4pt} Ton~: \#H.
\ding{202} \textcolor{PineGreen}{\selectlanguage{french}Fils.} \zh{儿子。}  ¶ \textcolor{darkblue}{\textbf{\ipa{zo˧ ɲi˥-kv̩˩}}} \textcolor{PineGreen}{\selectlanguage{french}deux fils} \zh{两个儿子}  
 \zh{量词}: \textcolor{darkblue}{\textbf{\ipa{v̩˧}}} \ding{203} \textcolor{PineGreen}{\selectlanguage{french}Homme, \textit{Vir}.} \zh{男人。}  \mytextsc{clf}~: \textcolor{darkblue}{\textbf{\ipa{v̩˧}}} 
\lhead{\firstmark}
\rhead{\botmark}

\subsection{\hspace{-0.5cm} {\Large \textcolor{darkblue}{\textbf{\ipa{‑zo˧}}}}\hspace{0.5cm}[\kern2pt{\textcolor{darkblue}{\textbf{\ipa{zo˥}}}}\kern2pt]} \hypertarget{‑zo\string_M1}{}
\markboth{\textcolor{darkblue}{\textbf{\ipa{‑zo˧}}}}{}
\textcolor{teal}{\mytextsc{suffixe}} \hspace{4pt} Ton~: M.
\textcolor{PineGreen}{\selectlanguage{french}Obligatif.} \zh{应该、必须。} 
\lhead{\firstmark}
\rhead{\botmark}

\subsection{\hspace{-0.5cm} {\Large \textcolor{darkblue}{\textbf{\ipa{zo˧\textsubscript{a}}}}}\hspace{0.5cm}[\kern2pt{\textcolor{darkblue}{\textbf{\ipa{zo˩˥}}}}\kern2pt]} \hypertarget{zo\string_Ma1}{}
\markboth{\textcolor{darkblue}{\textbf{\ipa{zo˧\textsubscript{a}}}}}{}
\textcolor{teal}{\mytextsc{verbe}} \hspace{4pt} Ton~: M\textsubscript{a}.
\textcolor{PineGreen}{\selectlanguage{french}Devoir.} \zh{要,应该。}  ¶ \textcolor{darkblue}{\textbf{\ipa{mɤ˧-zo˧ (-ze˧)! | tʰi˧-kwɤ˩-kʰɯ˩!}}} \textcolor{PineGreen}{\selectlanguage{french}ce n'est pas la peine! laisse tomber!} \zh{不用了!算了吧!}  
 ¶ \textcolor{darkblue}{\textbf{\ipa{ʈʂʰɯ˧ne˧-ʝi˥ | ʝi˧-zo˧-ho˥-ɲi˩!}}} \textcolor{PineGreen}{\selectlanguage{french}Il faut faire comme ça!} \zh{是应该这样做的!}  

\lhead{\firstmark}
\rhead{\botmark}

\subsection{\hspace{-0.5cm} {\Large \textcolor{darkblue}{\textbf{\ipa{zo˧bæ˩}}}}\hspace{0.5cm}[\kern2pt{\textcolor{darkblue}{\textbf{\ipa{zo˧bæ˧}}}}\kern2pt]} \hypertarget{zo\string_Mb\{\string_B1}{}
\markboth{\textcolor{darkblue}{\textbf{\ipa{zo˧bæ˩}}}}{}
\textcolor{teal}{\mytextsc{nom}} \hspace{4pt} Ton~: L\#.
\textcolor{PineGreen}{\selectlanguage{french}Imbécile, idiot.} \zh{笨人、傻瓜。}  ¶ \textcolor{darkblue}{\textbf{\ipa{mɤ˧-zo˧bæ˩!}}} \textcolor{PineGreen}{\selectlanguage{french}(Non, tu n'es) pas idiot(e)! (Propos rassurant adressé à un interlocuteur accablé par ses propres maladresses.)} \zh{(你)不是笨蛋!(情景:一个人批评自己是笨蛋,人家安慰他。)}  
 ¶ \textcolor{darkblue}{\textbf{\ipa{zo˧bæ˩-mv̩˩bæ˩}}} \textcolor{PineGreen}{\selectlanguage{french}idiots, imbéciles (sans distinction de sexe)} \zh{傻瓜们(不分男女)}  
 \zh{量词}: \textcolor{darkblue}{\textbf{\ipa{v̩˧}}}  \mytextsc{clf}~: \textcolor{darkblue}{\textbf{\ipa{v̩˧}}} 
\lhead{\firstmark}
\rhead{\botmark}

\subsection{\hspace{-0.5cm} {\Large \textcolor{darkblue}{\textbf{\ipa{zo˧ɖɯ\#˥}}}}\hspace{0.5cm}[\kern2pt{\textcolor{darkblue}{\textbf{\ipa{zo˧ɖɯ˧}}}}\kern2pt]} \hypertarget{zo\string_Md`M\#\string_T1}{}
\markboth{\textcolor{darkblue}{\textbf{\ipa{zo˧ɖɯ\#˥}}}}{}
\textcolor{teal}{\mytextsc{nom}} \hspace{4pt} Ton~: \#H.
\textcolor{PineGreen}{\selectlanguage{french}Fils aîné.} \zh{大儿子。}  ¶ \textcolor{darkblue}{\textbf{\ipa{zo˧ɖɯ˧-mv̩˥ɖɯ˩}}} \textcolor{PineGreen}{\selectlanguage{french}fils et fille aînés} \zh{大儿子与大女儿}  

\lhead{\firstmark}
\rhead{\botmark}

\subsection{\hspace{-0.5cm} {\Large \textcolor{darkblue}{\textbf{\ipa{zo˧hṽ˧-mv̩˥zo˩}}}}\hspace{0.5cm}[\kern2pt{\textcolor{darkblue}{\textbf{\ipa{zo˧hṽ˧˥mv̩˧zo˧}}}}\kern2pt]} \hypertarget{zo\string_Mhv\string_~\string_M-mv\string_=\string_Tzo\string_B1}{}
\markboth{\textcolor{darkblue}{\textbf{\ipa{zo˧hṽ˧-mv̩˥zo˩}}}}{}
\textcolor{teal}{\mytextsc{nom}} \hspace{4pt} Ton~: MH\#-.
\textcolor{PineGreen}{\selectlanguage{french}Les descendants.} \zh{后代。}  ¶ \textcolor{darkblue}{\textbf{\ipa{zo˧hṽ˧mv̩˥zo˩=ɻæ˩}}} \textcolor{PineGreen}{\selectlanguage{french}\string_ \mytextsc{associatif}} \zh{\string_ \mytextsc{联想复数}}  

\lhead{\firstmark}
\rhead{\botmark}

\subsection{\hspace{-0.5cm} {\Large \textcolor{darkblue}{\textbf{\ipa{zo˧hṽ˧˥}}}}\hspace{0.5cm}[\kern2pt{\textcolor{darkblue}{\textbf{\ipa{zo˧hṽ˧˥}}}}\kern2pt]} \hypertarget{zo\string_Mhv\string_~\string_M\string_T1}{}
\markboth{\textcolor{darkblue}{\textbf{\ipa{zo˧hṽ˧˥}}}}{}
\textcolor{teal}{\mytextsc{nom}} \hspace{4pt} Ton~: MH\#.
\ding{202} \textcolor{PineGreen}{\selectlanguage{french}Fils.} \zh{儿子。}  ¶ \textcolor{darkblue}{\textbf{\ipa{zo˧hṽ˧=ɻæ˥}}} \textcolor{PineGreen}{\selectlanguage{french}les fils} \zh{儿子们}  
 \zh{量词}: \textcolor{darkblue}{\textbf{\ipa{v̩˧}}} \ding{203} \textcolor{PineGreen}{\selectlanguage{french}Jeune homme, petit gars.} \zh{小伙子、 青年男子。}  \mytextsc{clf}~: \textcolor{darkblue}{\textbf{\ipa{v̩˧}}} 
\lhead{\firstmark}
\rhead{\botmark}

\subsection{\hspace{-0.5cm} {\Large \textcolor{darkblue}{\textbf{\ipa{zo˧mv̩˥}}}}\hspace{0.5cm}[\kern2pt{\textcolor{darkblue}{\textbf{\ipa{zo˧mv̩˥}}}}\kern2pt]} \hypertarget{zo\string_Mmv\string_=\string_T1}{}
\markboth{\textcolor{darkblue}{\textbf{\ipa{zo˧mv̩˥}}}}{}
\textcolor{teal}{\mytextsc{nom}} \hspace{4pt} Ton~: H\#.
\textcolor{PineGreen}{\selectlanguage{french}Enfant.} \zh{孩子。}  ¶ \textcolor{darkblue}{\textbf{\ipa{zo˧mv̩˥ | æ˧mv̩˥tɕi˩-hĩ˩}}} \textcolor{PineGreen}{\selectlanguage{french}nouveau-né, nourrisson} \zh{新生婴儿}  
 \zh{量词}: \textcolor{darkblue}{\textbf{\ipa{ɭɯ˧}}}  \mytextsc{clf}~: \textcolor{darkblue}{\textbf{\ipa{ɭɯ˧}}} 
\lhead{\firstmark}
\rhead{\botmark}

\subsection{\hspace{-0.5cm} {\Large \textcolor{darkblue}{\textbf{\ipa{zo˧tv̩˧-mv̩˥tv̩˩}}}}\hspace{0.5cm}[\kern2pt{\textcolor{darkblue}{\textbf{\ipa{zo˧tv̩˧mv̩˥tv̩˩}}}}\kern2pt]} \hypertarget{zo\string_Mtv\string_=\string_M-mv\string_=\string_Ttv\string_=\string_B1}{}
\markboth{\textcolor{darkblue}{\textbf{\ipa{zo˧tv̩˧-mv̩˥tv̩˩}}}}{}
\textcolor{teal}{\mytextsc{nom}} \hspace{4pt} Ton~: \#H-.
\textcolor{PineGreen}{\selectlanguage{french}Enfant unique (fils unique ou fille unique).} \zh{独生子(男女通用)。} 
\lhead{\firstmark}
\rhead{\botmark}

\subsection{\hspace{-0.5cm} {\Large \textcolor{darkblue}{\textbf{\ipa{zo˧tv̩\#˥}}}}\hspace{0.5cm}[\kern2pt{\textcolor{darkblue}{\textbf{\ipa{zo˧tv̩˧}}}}\kern2pt]} \hypertarget{zo\string_Mtv\string_=\#\string_T1}{}
\markboth{\textcolor{darkblue}{\textbf{\ipa{zo˧tv̩\#˥}}}}{}
\textcolor{teal}{\mytextsc{nom}} \hspace{4pt} Ton~: \#H.
\textcolor{PineGreen}{\selectlanguage{french}Fils unique.} \zh{独生子,独生男孩。}  ¶ \textcolor{darkblue}{\textbf{\ipa{zo˧tv̩˧ ɖɯ˧-v̩˧-lɑ˧ dʑo˧˥!}}} \textcolor{PineGreen}{\selectlanguage{french}(elle) n'a qu'un fils unique!} \zh{(她)只有一个独生男孩子!}  
 ¶ \textcolor{darkblue}{\textbf{\ipa{ʂɯ˧-ɬi˧mi˧, | zo˧tv̩˧ ʐɤ˥-tʰɑ˩-se˩!}}} \textcolor{PineGreen}{\selectlanguage{french}“Au septième mois, un fils unique ne doit pas aller par les chemins!” (Le septième mois, au plus fort des grandes pluies, était considéré comme un mois défavorable pour voyager; les voyageurs risquaient d'y rester.)} \zh{“七月份,独生子不要上路!”(七月份是大雨季,摩梭人认为七月份的路最不安全:有生命危险)}  

\lhead{\firstmark}
\rhead{\botmark}

\subsection{\hspace{-0.5cm} {\Large \textcolor{darkblue}{\textbf{\ipa{zo˧tʰi˧}}}}\hspace{0.5cm}[\kern2pt{\textcolor{darkblue}{\textbf{\ipa{zo˧tʰi˧}}}}\kern2pt]} \hypertarget{zo\string_Mt\string_hi\string_M1}{}
\markboth{\textcolor{darkblue}{\textbf{\ipa{zo˧tʰi˧}}}}{}
\textcolor{teal}{\mytextsc{nom}} \hspace{4pt} Ton~: M.
\textcolor{PineGreen}{\selectlanguage{french}Personne intelligente.} \zh{聪明的人。}  ¶ \textcolor{darkblue}{\textbf{\ipa{zo˧tʰi˧ ɖɯ˧-v̩˧}}} \textcolor{PineGreen}{\selectlanguage{french}une personne intelligente} \zh{一个聪明的人}  

\lhead{\firstmark}
\rhead{\botmark}

\subsection{\hspace{-0.5cm} {\Large \textcolor{darkblue}{\textbf{\ipa{zo˧tʰi˧}}}}\hspace{0.5cm}[\kern2pt{\textcolor{darkblue}{\textbf{\ipa{zo˧tʰi˧}}}}\kern2pt]} \hypertarget{zo\string_Mt\string_hi\string_M1}{}
\markboth{\textcolor{darkblue}{\textbf{\ipa{zo˧tʰi˧}}}}{}
\textcolor{teal}{\mytextsc{adjectif}} \hspace{4pt} Ton~: M.
\textcolor{PineGreen}{\selectlanguage{french}Intelligent.} \zh{聪明。}  ¶ \textcolor{darkblue}{\textbf{\ipa{ʈʂʰɯ˧ | zo˧tʰi˧ | ʐwæ˩˥!}}} \textcolor{PineGreen}{\selectlanguage{french}il est très intelligent!} \zh{他很聪明!}  
 ¶ \textcolor{darkblue}{\textbf{\ipa{ʈʂʰɯ˧ | mɤ˧-tʰi˧!}}} \textcolor{PineGreen}{\selectlanguage{french}il n'est pas intelligent/il n'est pas bien malin! (on ne peut dire: /*mɤ˧-zo˧tʰi˧/)} \zh{他不聪明!}  

\lhead{\firstmark}
\rhead{\botmark}

\subsection{\hspace{-0.5cm} {\Large \textcolor{darkblue}{\textbf{\ipa{zo˧tɕi˥}}}}\hspace{0.5cm}[\kern2pt{\textcolor{darkblue}{\textbf{\ipa{zo˧tɕi˥}}}}\kern2pt]} \hypertarget{zo\string_Mts£i\string_T1}{}
\markboth{\textcolor{darkblue}{\textbf{\ipa{zo˧tɕi˥}}}}{}
\textcolor{teal}{\mytextsc{nom}} \hspace{4pt} Ton~: H\#.
\textcolor{PineGreen}{\selectlanguage{french}Fils dernier-né, benjamin.} \zh{最小的儿子。}  ¶ \textcolor{darkblue}{\textbf{\ipa{zo˧tɕi˥-mv̩˩tɕi˩}}} \textcolor{PineGreen}{\selectlanguage{french}le benjamin et la benjamine: les plus jeunes enfants} \zh{最小的儿子与女儿}  

\lhead{\firstmark}
\rhead{\botmark}

\subsection{\hspace{-0.5cm} {\Large \textcolor{darkblue}{\textbf{\ipa{zo˧zo˧-mv̩˧mv̩˥}}}}\hspace{0.5cm}[\kern2pt{\textcolor{darkblue}{\textbf{\ipa{xxxx non-correspondance entre le nombre de morphèmes et le nombre de tons de morphèmes}}}}\kern2pt]} \hypertarget{zo\string_Mzo\string_M-mv\string_=\string_Mmv\string_=\string_T1}{}
\markboth{\textcolor{darkblue}{\textbf{\ipa{zo˧zo˧-mv̩˧mv̩˥}}}}{}
\textcolor{teal}{\mytextsc{nom}} \hspace{4pt} Ton~: H\#.
\textcolor{PineGreen}{\selectlanguage{french}Truc, bidule.} \zh{东西。}  \zh{量词}: \textcolor{darkblue}{\textbf{\ipa{kʰwɤ˥}}}  \mytextsc{clf}~: \textcolor{darkblue}{\textbf{\ipa{kʰwɤ˥}}} 
\lhead{\firstmark}
\rhead{\botmark}

\subsection{\hspace{-0.5cm} {\Large \textcolor{darkblue}{\textbf{\ipa{zo˧ʐɤ\#˥}}}}\hspace{0.5cm}[\kern2pt{\textcolor{darkblue}{\textbf{\ipa{zo˧ʐɤ˧}}}}\kern2pt]} \hypertarget{zo\string_Mz`7\#\string_T1}{}
\markboth{\textcolor{darkblue}{\textbf{\ipa{zo˧ʐɤ\#˥}}}}{}
\textcolor{teal}{\mytextsc{nom}} \hspace{4pt} Ton~: \#H.
\textcolor{PineGreen}{\selectlanguage{french}Fils adoptif.} \zh{义子。} 
\lhead{\firstmark}
\rhead{\botmark}

\subsection{\hspace{-0.5cm} {\Large \textcolor{darkblue}{\textbf{\ipa{zo˩bv̩˥li˩}}}}\hspace{0.5cm}[\kern2pt{\textcolor{darkblue}{\textbf{\ipa{zo˧bv̩˧li˩}}}}\kern2pt]} \hypertarget{zo\string_Bbv\string_=\string_Tli\string_B1}{}
\markboth{\textcolor{darkblue}{\textbf{\ipa{zo˩bv̩˥li˩}}}}{}
\textcolor{teal}{\mytextsc{nom}} \hspace{4pt} Ton~: .
\textcolor{PineGreen}{\selectlanguage{french}Univers.} \zh{宇宙。}  Emprunt~: tibétain?  (Lidz 2010: 108)
 ¶ \textcolor{darkblue}{\textbf{\ipa{sɑ˧ | -zo˩bv̩˥-li˩}}} \textcolor{PineGreen}{\selectlanguage{french}l'univers} \zh{宇宙}  

\lhead{\firstmark}
\rhead{\botmark}

\subsection{\hspace{-0.5cm} {\Large \textcolor{darkblue}{\textbf{\ipa{zo˩no˧}}}}\hspace{0.5cm}[\kern2pt{\textcolor{darkblue}{\textbf{\ipa{zo˩no˥}}}}\kern2pt]} \hypertarget{zo\string_Bno\string_M1}{}
\markboth{\textcolor{darkblue}{\textbf{\ipa{zo˩no˧}}}}{}
\textcolor{teal}{\mytextsc{adverbe}} \hspace{4pt} Ton~: LM.
\textcolor{PineGreen}{\selectlanguage{french}Maintenant, actuellement: désigne le moment présent (heure de la journée), comme la période présente (époque contemporaine, par opposition à d'autres époques); également employé comme élément phatique, “gap-filler”: “alors…”; “eh bien…”.} \zh{现在。}  ¶ \textcolor{darkblue}{\textbf{\ipa{zo˩no˥ | gɤ˩-ʈi˧!}}} \textcolor{PineGreen}{\selectlanguage{french}Elle vient de se réveiller/de se lever! / Elle s'est réveillée à l'instant! (contexte: quelqu'un entre dans la maison, voit un petit enfant en train de jouer et constate: “Elle est réveillée!” Sa grand-mère répond: “Elle vient de se réveiller!”)} \zh{刚起床! / 刚才才起床!}  

\lhead{\firstmark}
\rhead{\botmark}

\subsection{\hspace{-0.5cm} {\Large \textcolor{darkblue}{\textbf{\ipa{zo˩qo˧}}}}\hspace{0.5cm}[\kern2pt{\textcolor{darkblue}{\textbf{\ipa{zo˩qo˥}}}}\kern2pt]} \hypertarget{zo\string_Bqo\string_M1}{}
\markboth{\textcolor{darkblue}{\textbf{\ipa{zo˩qo˧}}}}{}
\textcolor{teal}{\mytextsc{pronom}} \hspace{4pt} Ton~: LM.
\textcolor{PineGreen}{\selectlanguage{french}Où.} \zh{哪里。}  ¶ \textcolor{darkblue}{\textbf{\ipa{no˧ | zo˩qo˧ bi˧?}}} \textcolor{PineGreen}{\selectlanguage{french}Où tu vas?} \zh{你去哪里?}  
 ¶ \textcolor{darkblue}{\textbf{\ipa{zo˩qo˧-ɳɯ˧ | tsʰɯ˩˥?}}} \textcolor{PineGreen}{\selectlanguage{french}D'où (tu) viens?} \zh{从哪里来?}  
 ¶ \textcolor{darkblue}{\textbf{\ipa{no˧ | hɑ˧ | zo˩qo˧ dzɯ˧-bi˧-pi˧, | ɖɯ˧-bæ˧ lɑ˧ ɲi˥!}}} \textcolor{PineGreen}{\selectlanguage{french}Peu importe où tu vas manger, c'est partout pareil! (contexte: au sujet des restaurants récemment ouverts à Yongning, qui partagent les mêmes qualités et défauts dont des problèmes d'hygiène)} \zh{无论你到哪里去吃,都一样!(情景:新开的饭馆)}  
 ¶ \textcolor{darkblue}{\textbf{\ipa{zo˩qo˧ tʰv̩˧?}}} \textcolor{PineGreen}{\selectlanguage{french}Où tu es? (question typique quand on appelle quelqu'un sur son téléphone portable)} \zh{你到哪里了?(打手机)}  

\lhead{\firstmark}
\rhead{\botmark}

\subsection{\hspace{-0.5cm} {\Large \textcolor{darkblue}{\textbf{\ipa{zɯ˥}}}}\hspace{0.5cm}[\kern2pt{\textcolor{darkblue}{\textbf{\ipa{zɯ˥}}}}\kern2pt]} \hypertarget{zM\string_T1}{}
\markboth{\textcolor{darkblue}{\textbf{\ipa{zɯ˥}}}}{}
\textcolor{teal}{\mytextsc{nom}} \hspace{4pt} Ton~: \#H.
\textcolor{PineGreen}{\selectlanguage{french}Herbe.} \zh{草。}  \zh{量词}: \textcolor{darkblue}{\textbf{\ipa{kʰɤ˧˥}}}  \mytextsc{clf}~: \textcolor{darkblue}{\textbf{\ipa{kʰɤ˧˥}}} 
\lhead{\firstmark}
\rhead{\botmark}

\subsection{\hspace{-0.5cm} {\Large \textcolor{darkblue}{\textbf{\ipa{zɯ˧}}}}\hspace{0.5cm}[\kern2pt{\textcolor{darkblue}{\textbf{\ipa{zɯ˥}}}}\kern2pt]} \hypertarget{zM\string_M1}{}
\markboth{\textcolor{darkblue}{\textbf{\ipa{zɯ˧}}}}{}
\textcolor{teal}{\mytextsc{nom}} \hspace{4pt} Ton~: M.
\textcolor{PineGreen}{\selectlanguage{french}Vie, existence.} \zh{生命。}  ¶ \textcolor{darkblue}{\textbf{\ipa{zɯ˧ʂæ\#˥}}} \textcolor{PineGreen}{\selectlanguage{french}longue vie} \zh{长命、长的人生}  
 ¶ \textcolor{darkblue}{\textbf{\ipa{zɯ˧ ʂæ˧ | hɑ̃˧-ʝi˧-kʰɯ˩!}}} \textcolor{PineGreen}{\selectlanguage{french}Puisses-tu avoir longue vie! (bénédiction)} \zh{祝你长寿!}  
 ¶ \textcolor{darkblue}{\textbf{\ipa{zɯ˧ɖæ\#˥}}} \textcolor{PineGreen}{\selectlanguage{french}courte vie} \zh{短命}  
\textit{Voir~:} \hyperlink{}{\textcolor{darkblue}{\textbf{\ipa{zɯ˧\textsubscript{b}}}}} 
\lhead{\firstmark}
\rhead{\botmark}

\subsection{\hspace{-0.5cm} {\Large \textcolor{darkblue}{\textbf{\ipa{zɯ˧\textsubscript{b}}}}}\hspace{0.5cm}[\kern2pt{\textcolor{darkblue}{\textbf{\ipa{zɯ˥}}}}\kern2pt]} \hypertarget{zM\string_Mb1}{}
\markboth{\textcolor{darkblue}{\textbf{\ipa{zɯ˧\textsubscript{b}}}}}{}
\textcolor{teal}{\mytextsc{classificateur}} \hspace{4pt} Ton~: M\textsubscript{b}.
\textcolor{PineGreen}{\selectlanguage{french}Auto-classificateur de la vie, de l'existence entière.} \zh{量词:辈子。}  ¶ \textcolor{darkblue}{\textbf{\ipa{ɖɯ˧-zɯ˧}}} \textcolor{PineGreen}{\selectlanguage{french}toute la vie} \zh{一辈子(的时间)}  
\textit{Voir~:} \hyperlink{}{\textcolor{darkblue}{\textbf{\ipa{zɯ˧}}}} 
\lhead{\firstmark}
\rhead{\botmark}

\subsection{\hspace{-0.5cm} {\Large \textcolor{darkblue}{\textbf{\ipa{zɯ˧hṽ˩}}}}\hspace{0.5cm}[\kern2pt{\textcolor{darkblue}{\textbf{\ipa{zɯ˧hṽ˩}}}}\kern2pt]} \hypertarget{zM\string_Mhv\string_~\string_B1}{}
\markboth{\textcolor{darkblue}{\textbf{\ipa{zɯ˧hṽ˩}}}}{}
\textcolor{teal}{\mytextsc{adjectif}} \hspace{4pt} Ton~: L\#.
\textcolor{PineGreen}{\selectlanguage{french}Vert.} \zh{绿(布料、线)。}  ¶ \textcolor{darkblue}{\textbf{\ipa{zɯ˧hṽ˩-ni˩gv̩˩}}} \textcolor{PineGreen}{\selectlanguage{french}de couleur verte} \zh{绿}  
 ¶ \textcolor{darkblue}{\textbf{\ipa{[F5] zɯ˧hṽ˩ | \textasciitilde{}zɯ˧hṽ˩-ni˩gv̩˩}}} \textcolor{PineGreen}{\selectlanguage{french}tout vert} \zh{全绿}  

\lhead{\firstmark}
\rhead{\botmark}

\subsection{\hspace{-0.5cm} {\Large \textcolor{darkblue}{\textbf{\ipa{zɯ˧pv̩˩}}}}\hspace{0.5cm}[\kern2pt{\textcolor{darkblue}{\textbf{\ipa{zɯ˧pv̩˩}}}}\kern2pt]} \hypertarget{zM\string_Mpv\string_=\string_B1}{}
\markboth{\textcolor{darkblue}{\textbf{\ipa{zɯ˧pv̩˩}}}}{}
\textcolor{teal}{\mytextsc{nom}} \hspace{4pt} Ton~: L\#.
\textcolor{PineGreen}{\selectlanguage{french}Foin; s'emploie aussi parfois pour désigner la paille: dans la maison, on ne stocke que de la paille de riz, pas de foin; l'herbe cueillie verte puis séchée (foin) n'est pas entreposée, mais aussitôt donnée aux animaux.} \zh{干草。}  \zh{量词}: \textcolor{darkblue}{\textbf{\ipa{kʰɤ˧˥}}}  \mytextsc{clf}~: \textcolor{darkblue}{\textbf{\ipa{kʰɤ˧˥}}} 
\lhead{\firstmark}
\rhead{\botmark}

\subsection{\hspace{-0.5cm} {\Large \textcolor{darkblue}{\textbf{\ipa{zɯ˧-qʰɑ˧mi\#˥}}}}\hspace{0.5cm}[\kern2pt{\textcolor{darkblue}{\textbf{\ipa{xxxx non-correspondance entre le nombre de morphèmes et le nombre de tons de morphèmes}}}}\kern2pt]} \hypertarget{zM\string_M-q\string_hA\string_Mmi\#\string_T1}{}
\markboth{\textcolor{darkblue}{\textbf{\ipa{zɯ˧-qʰɑ˧mi\#˥}}}}{}
\textcolor{teal}{\mytextsc{nom}} \hspace{4pt} Ton~: \#H.
\textcolor{PineGreen}{\selectlanguage{french}\textit{Eulaliopsis binata (Retz.) C. E. Hubb.}, herbe sauvage. L'herbe n'est pas prisée du bétail, non plus que ses racines, et n'est jamais consommée par les humains. Les racines de cette herbe sont utilisées dans les rituels: elles ont une odeur forte à la combustion.} \zh{蓑草、山草、山草根、龙须草、山茅草、羊草、拟金茅。} Dialecte chinois local~:\zh{狗尾巴草。} \zh{量词}: \textcolor{darkblue}{\textbf{\ipa{po˧}}}  \mytextsc{clf}~: \textcolor{darkblue}{\textbf{\ipa{po˧}}} 
\lhead{\firstmark}
\rhead{\botmark}

\subsection{\hspace{-0.5cm} {\Large \textcolor{darkblue}{\textbf{\ipa{zɯ˧ɻ\#˥}}}}\hspace{0.5cm}[\kern2pt{\textcolor{darkblue}{\textbf{\ipa{zɯ˧ɻ˧}}}}\kern2pt]} \hypertarget{zM\string_Mr£`\#\string_T1}{}
\markboth{\textcolor{darkblue}{\textbf{\ipa{zɯ˧ɻ\#˥}}}}{}
\textcolor{teal}{\mytextsc{nom}} \hspace{4pt} Ton~: \#H.
\textcolor{PineGreen}{\selectlanguage{french}Joue (partie basse, en-dessous des pommettes; vers l'articulation des deux mâchoires).} \zh{腮、腮帮子。}  ¶ \textcolor{darkblue}{\textbf{\ipa{zɯ˧ɻ̍˧ qʰwæ˩}}} \textcolor{PineGreen}{\selectlanguage{french}gifler} \zh{掌掴、打嘴巴}  
 \zh{量词}: \textcolor{darkblue}{\textbf{\ipa{ɭɯ˧}}}  \mytextsc{clf}~: \textcolor{darkblue}{\textbf{\ipa{ɭɯ˧}}} 
\lhead{\firstmark}
\rhead{\botmark}

\subsection{\hspace{-0.5cm} {\Large \textcolor{darkblue}{\textbf{\ipa{zɯ˧\textasciitilde{}zɯ˧}}}}\hspace{0.5cm}[\kern2pt{\textcolor{darkblue}{\textbf{\ipa{zɯ˧zɯ˧}}}}\kern2pt]} \hypertarget{zM\string_M~zM\string_M1}{}
\markboth{\textcolor{darkblue}{\textbf{\ipa{zɯ˧\textasciitilde{}zɯ˧}}}}{}
\textcolor{teal}{\mytextsc{nom}} \hspace{4pt} Ton~: M.
\textcolor{PineGreen}{\selectlanguage{french}Vie, existence.} \zh{生命。}  ¶ \textcolor{darkblue}{\textbf{\ipa{hĩ˧-zɯ˧\textasciitilde{}zɯ˥\$}}} \textcolor{PineGreen}{\selectlanguage{french}la vie humaine} \zh{人生}  
 ¶ \textcolor{darkblue}{\textbf{\ipa{hĩ˧ zɯ˧ | ʂæ˧ | ʐwæ˩˥}}} \textcolor{PineGreen}{\selectlanguage{french}une très longue vie / la vie est longue} \zh{很长的人生 / 人生很长}  
 \zh{量词}: \textcolor{darkblue}{\textbf{\ipa{ljɤ˩}}}  \mytextsc{clf}~: \textcolor{darkblue}{\textbf{\ipa{ljɤ˩}}} 
\lhead{\firstmark}
\rhead{\botmark}

\subsection{\hspace{-0.5cm} {\Large \textcolor{darkblue}{\textbf{\ipa{zɯ˩\textasciitilde{}zɯ˩}}}}\hspace{0.5cm}[\kern2pt{\textcolor{darkblue}{\textbf{\ipa{zɯ˩zɯ˩˥}}}}\kern2pt]} \hypertarget{zM\string_B~zM\string_B1}{}
\markboth{\textcolor{darkblue}{\textbf{\ipa{zɯ˩\textasciitilde{}zɯ˩}}}}{}
\textcolor{teal}{\mytextsc{verbe}} \hspace{4pt} Ton~: L.
\textcolor{PineGreen}{\selectlanguage{french}Être engourdi.} \zh{麻木。}  ¶ \textcolor{darkblue}{\textbf{\ipa{gv̩˧dv̩˧gv̩˧mi˧ | zɯ˩\textasciitilde{}zɯ˩˥}}} \textcolor{PineGreen}{\selectlanguage{french}avoir le corps engourdi} \zh{身体麻木、全身麻木}  
 ¶ \textcolor{darkblue}{\textbf{\ipa{gv̩˧mi˧ | zɯ˩\textasciitilde{}zɯ˩˥}}} \textcolor{PineGreen}{\selectlanguage{french}avoir le corps engourdi} \zh{身体麻木、全身麻木}  
 ¶ \textcolor{darkblue}{\textbf{\ipa{tʰi˧-zɯ˩\textasciitilde{}zɯ˩}}} \textcolor{PineGreen}{\selectlanguage{french}\mytextsc{dur} \mytextsc{red}} \zh{\mytextsc{dur} \mytextsc{red}}  

\lhead{\firstmark}
\rhead{\botmark}

\newpage
\section*{\centering- \textcolor{darkblue}{\textbf{\ipa{ʐ}}} -}
\subsection{\hspace{-0.5cm} {\Large \textcolor{darkblue}{\textbf{\ipa{ʐ}}}}\hspace{0.5cm}[\kern2pt{\textcolor{darkblue}{\textbf{\ipa{[]}}}}\kern2pt]} \hypertarget{z`1}{}
\markboth{\textcolor{darkblue}{\textbf{\ipa{ʐ}}}}{}
\textcolor{teal}{\mytextsc{idéophone}} \hspace{4pt} Ton~: 0.
\textcolor{PineGreen}{\selectlanguage{french}Bruit de grondement des grosses charges qu'on traîne sur le sol, des moteurs de camions: Brrroum!} \zh{形声词:轰隆隆!(拉很重的物品在地板上的隆隆声,卡车的隆隆声)。} 
\lhead{\firstmark}
\rhead{\botmark}

\subsection{\hspace{-0.5cm} {\Large \textcolor{darkblue}{\textbf{\ipa{ʐæ˧}}}}\hspace{0.5cm}[\kern2pt{\textcolor{darkblue}{\textbf{\ipa{ʐæ˩˥}}}}\kern2pt]} \hypertarget{z`\{\string_M1}{}
\markboth{\textcolor{darkblue}{\textbf{\ipa{ʐæ˧}}}}{}
\textcolor{teal}{\mytextsc{adjectif}} \hspace{4pt} Ton~: M.
\textcolor{PineGreen}{\selectlanguage{french}Grand et fort, massif, baraqué.} \zh{高大。}  ¶ \textcolor{darkblue}{\textbf{\ipa{ʐæ˧-ni˩gv̩˩}}} \textcolor{PineGreen}{\selectlanguage{french}grand et fort} \zh{高大}  
 ¶ \textcolor{darkblue}{\textbf{\ipa{hĩ˧ | ʈʂʰɯ˧-v̩˧, | ʐæ˧-ni˩gv̩˩!}}} \textcolor{PineGreen}{\selectlanguage{french}Elle/il est grand(e) et fort(e) / impressionnant(e)!} \zh{这人很高大!}  
 ¶ \textcolor{darkblue}{\textbf{\ipa{ʐæ˧ni˩ | mɤ˧-gv̩˧}}} \textcolor{PineGreen}{\selectlanguage{french}pas bien grand (en taille), pas bien impressionnant} \zh{个子不高}  
 ¶ \textcolor{darkblue}{\textbf{\ipa{ʐæ˧ | ʐwæ˩˥}}} \textcolor{PineGreen}{\selectlanguage{french}très grand et fort} \zh{很高大}  

\lhead{\firstmark}
\rhead{\botmark}

\subsection{\hspace{-0.5cm} {\Large \textcolor{darkblue}{\textbf{\ipa{ʐæ˧\textsubscript{a}}}}}\hspace{0.5cm}[\kern2pt{\textcolor{darkblue}{\textbf{\ipa{ʐæ˩˥}}}}\kern2pt]} \hypertarget{z`\{\string_Ma1}{}
\markboth{\textcolor{darkblue}{\textbf{\ipa{ʐæ˧\textsubscript{a}}}}}{}
\textcolor{teal}{\mytextsc{verbe}} \hspace{4pt} Ton~: M\textsubscript{a}.
\ding{202} \textcolor{PineGreen}{\selectlanguage{french}Rire.} \zh{笑。}  ¶ \textcolor{darkblue}{\textbf{\ipa{zo˧hṽ˥ | hĩ˧ ʐæ˧\textasciitilde{}ʐæ˥-kʰɯ˩}}} \textcolor{PineGreen}{\selectlanguage{french}les enfants taquinent les gens, les font rire} \zh{孩子们把大家逗笑了。}  
 ¶ \textcolor{darkblue}{\textbf{\ipa{hĩ˧ | ʐæ˧\textasciitilde{}ʐæ˥ kʰɯ˩}}} \textcolor{PineGreen}{\selectlanguage{french}faire rire les gens, amuser les gens, faire rire le public} \zh{让大家笑一笑}  
 ¶ \textcolor{darkblue}{\textbf{\ipa{ʐæ˧\textasciitilde{}ʐæ˩-di˩}}} \textcolor{PineGreen}{\selectlanguage{french}plaisanteries, blagues} \zh{笑话,好笑的话}  
\ding{203} \textcolor{PineGreen}{\selectlanguage{french}Être impertinent, déranger, se moquer du monde.} \zh{嘲笑别人、出言不逊。}  ¶ \textcolor{darkblue}{\textbf{\ipa{le˧-ʐæ˧-ze˧}}} \textcolor{PineGreen}{\selectlanguage{french}\mytextsc{accomp} \string_ \mytextsc{pfv}} \zh{出言不逊了}  
 ¶ \textcolor{darkblue}{\textbf{\ipa{le˧-ʐæ˥\textasciitilde{}ʐæ˩}}} \textcolor{PineGreen}{\selectlanguage{french}\mytextsc{red}} \zh{笑一笑(别人)}  
 ¶ \textcolor{darkblue}{\textbf{\ipa{hĩ˧ ʐæ˩}}} \textcolor{PineGreen}{\selectlanguage{french}être impertinent avec les gens, déranger les gens} \zh{嘲笑人家}  
 ¶ \textcolor{darkblue}{\textbf{\ipa{le˧-ʐæ˥\textasciitilde{}ʐæ˩-ze˩}}} \textcolor{PineGreen}{\selectlanguage{french}\mytextsc{red} \mytextsc{pfv}} \zh{嘲笑了}  

\lhead{\firstmark}
\rhead{\botmark}

\subsection{\hspace{-0.5cm} {\Large \textcolor{darkblue}{\textbf{\ipa{ʐæ˧v̩˩-tʰv̩˩}}}}\hspace{0.5cm}[\kern2pt{\textcolor{darkblue}{\textbf{\ipa{ʐæ˧v̩˩tʰv̩˧}}}}\kern2pt]} \hypertarget{z`\{\string_Mv\string_=\string_B-t\string_hv\string_=\string_B1}{}
\markboth{\textcolor{darkblue}{\textbf{\ipa{ʐæ˧v̩˩-tʰv̩˩}}}}{}
\textcolor{teal}{\mytextsc{verbe}} \hspace{4pt} Ton~: L\#-.
\textcolor{PineGreen}{\selectlanguage{french}Blaguer, faire une blague, faire une plaisanterie.} \zh{开玩笑。}  ¶ \textcolor{darkblue}{\textbf{\ipa{ʐæ˧v̩˩-tʰv̩˩ | ʐwæ˩˥}}} \textcolor{PineGreen}{\selectlanguage{french}plaisanter follement, rire beaucoup} \zh{开很多玩笑、一直开玩笑}  
 ¶ \textcolor{darkblue}{\textbf{\ipa{ʐæ˧v̩˩-tʰv̩˩-hĩ˩ ʐwɤ˩}}} \textcolor{PineGreen}{\selectlanguage{french}lancer une blague, dire une plaisanterie} \zh{开个玩笑}  

\lhead{\firstmark}
\rhead{\botmark}

\subsection{\hspace{-0.5cm} {\Large \textcolor{darkblue}{\textbf{\ipa{ʐæ˩\textsubscript{b}}}}}\hspace{0.5cm}[\kern2pt{\textcolor{darkblue}{\textbf{\ipa{ʐæ˥}}}}\kern2pt]} \hypertarget{z`\{\string_Bb1}{}
\markboth{\textcolor{darkblue}{\textbf{\ipa{ʐæ˩\textsubscript{b}}}}}{}
\textcolor{teal}{\mytextsc{verbe}} \hspace{4pt} Ton~: L\textsubscript{b}.
\textcolor{PineGreen}{\selectlanguage{french}Mélanger, tourner (un mélange, une préparation).} \zh{搅拌合混。}  ¶ \textcolor{darkblue}{\textbf{\ipa{le˧-ʐæ˧\textasciitilde{}ʐæ˥}}} \textcolor{PineGreen}{\selectlanguage{french}\mytextsc{accomp} \string_ \mytextsc{red}} \zh{搅拌}  

\lhead{\firstmark}
\rhead{\botmark}

\subsection{\hspace{-0.5cm} {\Large \textcolor{darkblue}{\textbf{\ipa{ʐæ˩mi\#˥}}}}\hspace{0.5cm}[\kern2pt{\textcolor{darkblue}{\textbf{\ipa{ʐæ˧mi˧}}}}\kern2pt]} \hypertarget{z`\{\string_Bmi\#\string_T1}{}
\markboth{\textcolor{darkblue}{\textbf{\ipa{ʐæ˩mi\#˥}}}}{}
\textcolor{teal}{\mytextsc{nom}} \hspace{4pt} Ton~: LM+\#H.
\textcolor{PineGreen}{\selectlanguage{french}Léopard femelle.} \zh{母豹子。}  ¶ \textcolor{darkblue}{\textbf{\ipa{ʐæ˩mi˧-ʐæ˥zo˩}}} \textcolor{PineGreen}{\selectlanguage{french}léopard femelle et léopard mâle} \zh{母豹子与公豹子}  
 \zh{量词}: \textcolor{darkblue}{\textbf{\ipa{pʰo˧˥}}}  \mytextsc{clf}~: \textcolor{darkblue}{\textbf{\ipa{pʰo˧˥}}} 
\lhead{\firstmark}
\rhead{\botmark}

\subsection{\hspace{-0.5cm} {\Large \textcolor{darkblue}{\textbf{\ipa{ʐæ˩pʰv̩˧}}}}\hspace{0.5cm}[\kern2pt{\textcolor{darkblue}{\textbf{\ipa{xxxx non-correspondance entre le nombre de morphèmes et le nombre de tons de morphèmes}}}}\kern2pt]} \hypertarget{z`\{\string_Bp\string_hv\string_=\string_M1}{}
\markboth{\textcolor{darkblue}{\textbf{\ipa{ʐæ˩pʰv̩˧}}}}{}
\textcolor{teal}{\mytextsc{nom}} \hspace{4pt} Ton~: LM.
\textcolor{PineGreen}{\selectlanguage{french}Léopard mâle.} \zh{公豹子。}  ¶ \textcolor{darkblue}{\textbf{\ipa{ʐæ˩pʰv̩˧-ʐæ˩mi˩}}} \textcolor{PineGreen}{\selectlanguage{french}léopard mâle et léopard femelle} \zh{公豹子与母豹子}  
 \zh{量词}: \textcolor{darkblue}{\textbf{\ipa{pʰo˧˥}}}  \mytextsc{clf}~: \textcolor{darkblue}{\textbf{\ipa{pʰo˧˥}}} 
\lhead{\firstmark}
\rhead{\botmark}

\subsection{\hspace{-0.5cm} {\Large \textcolor{darkblue}{\textbf{\ipa{ʐæ˩sɯ˩}}}}\hspace{0.5cm}[\kern2pt{\textcolor{darkblue}{\textbf{\ipa{ʐæ˩sɯ˥}}}}\kern2pt]} \hypertarget{z`\{\string_BsM\string_B1}{}
\markboth{\textcolor{darkblue}{\textbf{\ipa{ʐæ˩sɯ˩}}}}{}
\textcolor{teal}{\mytextsc{nom}} \hspace{4pt} Ton~: L.
\textcolor{PineGreen}{\selectlanguage{french}Feutre grossier, fait uniquement de laine de mouton, dont on se drape en extérieur pour se protéger du froid.} \zh{披毡。}  \zh{量词}: \textcolor{darkblue}{\textbf{\ipa{ɭɯ˧˥}}}  \mytextsc{clf}~: \textcolor{darkblue}{\textbf{\ipa{ɭɯ˧˥}}} 
\lhead{\firstmark}
\rhead{\botmark}

\subsection{\hspace{-0.5cm} {\Large \textcolor{darkblue}{\textbf{\ipa{ʐæ˩sɯ˩-kʰwæ˩ɻæ˧}}}}\hspace{0.5cm}[\kern2pt{\textcolor{darkblue}{\textbf{\ipa{xxxx non-correspondance entre le nombre de morphèmes et le nombre de tons de morphèmes}}}}\kern2pt]} \hypertarget{z`\{\string_BsM\string_B-k\string_hw\{\string_Br£`\{\string_M1}{}
\markboth{\textcolor{darkblue}{\textbf{\ipa{ʐæ˩sɯ˩-kʰwæ˩ɻæ˧}}}}{}
\textcolor{teal}{\mytextsc{nom}} \hspace{4pt} Ton~: .
\textcolor{PineGreen}{\selectlanguage{french}Natte en feutre. Le terme désigne spécifiquement les nattes/matelas/tissus en feutre véritable, par opposition avec le sens étendu que peut avoir \textcolor{darkblue}{\textbf{\ipa{/kʰwæ˧ɻæ\#˥/}}}.} \zh{毡子(真正的毡子)做的垫子。}  \zh{量词}: \textcolor{darkblue}{\textbf{\ipa{tsʰi˥}}}  \mytextsc{clf}~: \textcolor{darkblue}{\textbf{\ipa{tsʰi˥}}} 
\lhead{\firstmark}
\rhead{\botmark}

\subsection{\hspace{-0.5cm} {\Large \textcolor{darkblue}{\textbf{\ipa{ʐæ˩ʂæ˧}}}}\hspace{0.5cm}[\kern2pt{\textcolor{darkblue}{\textbf{\ipa{ʐæ˩ʂæ˥}}}}\kern2pt]} \hypertarget{z`\{\string_Bs`\{\string_M1}{}
\markboth{\textcolor{darkblue}{\textbf{\ipa{ʐæ˩ʂæ˧}}}}{}
\textcolor{teal}{\mytextsc{adjectif}} \hspace{4pt} Ton~: LM.
\textcolor{PineGreen}{\selectlanguage{french}Loin, lointain.} \zh{远。}  ¶ \textcolor{darkblue}{\textbf{\ipa{no˧ | ʈʂʰɯ˧ | ə˩-ʐæ˥ʂæ˩? | dʑɤ˩kʰwɤ˧ ə˩-di˩? | - dʑɤ˩˥ | dʑɤ˩kʰwɤ˧ mɤ˧-di˥! | mɤ˧-ʐæ˩ʂæ˩!}}} \textcolor{PineGreen}{\selectlanguage{french}tu es loin de lui? Y a-t-il de la distance entre vous? (=vous êtes proches/intimes, ou pas?) - Non, il n'y a guère de distance! Nous ne somme pas éloignés!} \zh{你们很熟吗? - 不很熟!}  

\lhead{\firstmark}
\rhead{\botmark}

\subsection{\hspace{-0.5cm} {\Large \textcolor{darkblue}{\textbf{\ipa{ʐæ˩tsɯ˧˥}}}}\hspace{0.5cm}[\kern2pt{\textcolor{darkblue}{\textbf{\ipa{ʐæ˩tsɯ˧˥}}}}\kern2pt]} \hypertarget{z`\{\string_BtsM\string_M\string_T1}{}
\markboth{\textcolor{darkblue}{\textbf{\ipa{ʐæ˩tsɯ˧˥}}}}{}
\textcolor{teal}{\mytextsc{nom}} \hspace{4pt} Ton~: LM+MH\#.
\textcolor{PineGreen}{\selectlanguage{french}Sentier, petit chemin.} \zh{小路、径道。}  ¶ \textcolor{darkblue}{\textbf{\ipa{ʐæ˩tsɯ˧-ʐɤ˥mi˩}}} \textcolor{PineGreen}{\selectlanguage{french}chemin de traverse, raccourci} \zh{径道}  
 \zh{量词}: \textcolor{darkblue}{\textbf{\ipa{kʰɯ˩}}}  \mytextsc{clf}~: \textcolor{darkblue}{\textbf{\ipa{kʰɯ˩}}} 
\lhead{\firstmark}
\rhead{\botmark}

\subsection{\hspace{-0.5cm} {\Large \textcolor{darkblue}{\textbf{\ipa{ʐæ˩zo\#˥}}}}\hspace{0.5cm}[\kern2pt{\textcolor{darkblue}{\textbf{\ipa{ʐæ˩zo˥}}}}\kern2pt]} \hypertarget{z`\{\string_Bzo\#\string_T1}{}
\markboth{\textcolor{darkblue}{\textbf{\ipa{ʐæ˩zo\#˥}}}}{}
\textcolor{teal}{\mytextsc{nom}} \hspace{4pt} Ton~: LM+\#H.
\textcolor{PineGreen}{\selectlanguage{french}Bébé léopard, petit léopard.} \zh{小豹子。}  ¶ \textcolor{darkblue}{\textbf{\ipa{ʐæ˩zo˧-ʐæ˥mi˩}}} \textcolor{PineGreen}{\selectlanguage{french}petit léopard et léopard femelle} \zh{小豹子与母豹子}  
 \zh{量词}: \textcolor{darkblue}{\textbf{\ipa{ɭɯ˧}}}  \mytextsc{clf}~: \textcolor{darkblue}{\textbf{\ipa{ɭɯ˧}}} 
\lhead{\firstmark}
\rhead{\botmark}

\subsection{\hspace{-0.5cm} {\Large \textcolor{darkblue}{\textbf{\ipa{ʐæ˩˥}}}}\hspace{0.5cm}[\kern2pt{\textcolor{darkblue}{\textbf{\ipa{ʐæ˥}}}}\kern2pt]} \hypertarget{z`\{\string_B\string_T1}{}
\markboth{\textcolor{darkblue}{\textbf{\ipa{ʐæ˩˥}}}}{}
\textcolor{teal}{\mytextsc{nom}} \hspace{4pt} Ton~: LH.
\textcolor{PineGreen}{\selectlanguage{french}Léopard, panthère (note: ces deux termes sont homonymes en français).} \zh{豹子。}  ¶ \textcolor{darkblue}{\textbf{\ipa{ʐæ˩ dzɯ˧-ze˩}}} \textcolor{PineGreen}{\selectlanguage{french}...a mangé (un/le) léopard} \zh{吃了豹子}  
 ¶ \textcolor{darkblue}{\textbf{\ipa{ʐæ˩ hwæ˧-ze˩}}} \textcolor{PineGreen}{\selectlanguage{french}...a acheté (un/le) léopard} \zh{买了豹子}  
 \zh{量词}: \textcolor{darkblue}{\textbf{\ipa{pʰo˧˥}}}  \mytextsc{clf}~: \textcolor{darkblue}{\textbf{\ipa{pʰo˧˥}}} 
\lhead{\firstmark}
\rhead{\botmark}

\subsection{\hspace{-0.5cm} {\Large \textcolor{darkblue}{\textbf{\ipa{ʐe˥}}}}\hspace{0.5cm}[\kern2pt{\textcolor{darkblue}{\textbf{\ipa{ʐe˥}}}}\kern2pt]} \hypertarget{z`e\string_T1}{}
\markboth{\textcolor{darkblue}{\textbf{\ipa{ʐe˥}}}}{}
\textcolor{teal}{\mytextsc{classificateur}} \hspace{4pt} Ton~: H\textsubscript{a}.
\textcolor{PineGreen}{\selectlanguage{french}Classificateur des morceaux de viande conservée.} \zh{量词:熏肉(一块)。} 
\lhead{\firstmark}
\rhead{\botmark}

\subsection{\hspace{-0.5cm} {\Large \textcolor{darkblue}{\textbf{\ipa{ʐe˥}}} \textsubscript{1}}\hspace{0.5cm}[\kern2pt{\textcolor{darkblue}{\textbf{\ipa{ʐe˥}}}}\kern2pt]} \hypertarget{z`e\string_T1}{}
\markboth{\textcolor{darkblue}{\textbf{\ipa{ʐe˥}}} \textsubscript{1}}{}
\textcolor{teal}{\mytextsc{nom}} \hspace{4pt} Ton~: \#H.
\textcolor{PineGreen}{\selectlanguage{french}Flèche.} \zh{箭。}  ¶ \textcolor{darkblue}{\textbf{\ipa{ʐe˧ɻ̃˧ | ɖɯ˧-kʰɯ˩}}} \textcolor{PineGreen}{\selectlanguage{french}une flèche; désigne aussi, de façon métaphorique, une lignée/une famille} \zh{一枝箭。也来指一个家庭}  
 \zh{量词}: \textcolor{darkblue}{\textbf{\ipa{kʰɯ˩}}}  \mytextsc{clf}~: \textcolor{darkblue}{\textbf{\ipa{kʰɯ˩}}} 
\lhead{\firstmark}
\rhead{\botmark}

\subsection{\hspace{-0.5cm} {\Large \textcolor{darkblue}{\textbf{\ipa{ʐe˥}}} \textsubscript{2}}\hspace{0.5cm}[\kern2pt{\textcolor{darkblue}{\textbf{\ipa{ʐe˥}}}}\kern2pt]} \hypertarget{z`e\string_T2}{}
\markboth{\textcolor{darkblue}{\textbf{\ipa{ʐe˥}}} \textsubscript{2}}{}
\textcolor{teal}{\mytextsc{nom}} \hspace{4pt} Ton~: \#H.
\textcolor{PineGreen}{\selectlanguage{french}Saison des pluies (été et automne: du 3e au 8e mois du calendrier lunaire).} \zh{雨季(夏天至秋天:三月份至八月份)。} 
\lhead{\firstmark}
\rhead{\botmark}

\subsection{\hspace{-0.5cm} {\Large \textcolor{darkblue}{\textbf{\ipa{ʐe˧ʈæ˥-ɬi˩}}}}\hspace{0.5cm}[\kern2pt{\textcolor{darkblue}{\textbf{\ipa{ʐe˧ʈæ˥ɬi˩}}}}\kern2pt]} \hypertarget{z`e\string_Mt`\{\string_T-Ki\string_B1}{}
\markboth{\textcolor{darkblue}{\textbf{\ipa{ʐe˧ʈæ˥-ɬi˩}}}}{}
\textcolor{teal}{\mytextsc{nom}} \hspace{4pt} Ton~: H\#-L.
\textcolor{PineGreen}{\selectlanguage{french}11e mois.} \zh{十一月。} 
\lhead{\firstmark}
\rhead{\botmark}

\subsection{\hspace{-0.5cm} {\Large \textcolor{darkblue}{\textbf{\ipa{ʐe˧v̩\#˥}}}}\hspace{0.5cm}[\kern2pt{\textcolor{darkblue}{\textbf{\ipa{ʐe˧v̩˧}}}}\kern2pt]} \hypertarget{z`e\string_Mv\string_=\#\string_T1}{}
\markboth{\textcolor{darkblue}{\textbf{\ipa{ʐe˧v̩\#˥}}}}{}
\textcolor{teal}{\mytextsc{nom}} \hspace{4pt} Ton~: \#H.
\textcolor{PineGreen}{\selectlanguage{french}Taureau castré.} \zh{阉牛。}  \zh{量词}: \textcolor{darkblue}{\textbf{\ipa{pʰo˧˥}}}  \mytextsc{clf}~: \textcolor{darkblue}{\textbf{\ipa{pʰo˧˥}}} 
\lhead{\firstmark}
\rhead{\botmark}

\subsection{\hspace{-0.5cm} {\Large \textcolor{darkblue}{\textbf{\ipa{ʐe˧zo\#˥}}}}\hspace{0.5cm}[\kern2pt{\textcolor{darkblue}{\textbf{\ipa{ʐe˧zo˧}}}}\kern2pt]} \hypertarget{z`e\string_Mzo\#\string_T1}{}
\markboth{\textcolor{darkblue}{\textbf{\ipa{ʐe˧zo\#˥}}}}{}
\textcolor{teal}{\mytextsc{nom}} \hspace{4pt} Ton~: \#H.
\textcolor{PineGreen}{\selectlanguage{french}Flèche.} \zh{箭。}  ¶ \textcolor{darkblue}{\textbf{\ipa{ʐe˧zo˧ | ɖɯ˧-kʰɯ˩}}} \textcolor{PineGreen}{\selectlanguage{french}une flèche} \zh{一枝箭}  

\lhead{\firstmark}
\rhead{\botmark}

\subsection{\hspace{-0.5cm} {\Large \textcolor{darkblue}{\textbf{\ipa{ʐe˩ʐe˧-bæ˩bæ˩}}}}\hspace{0.5cm}[\kern2pt{\textcolor{darkblue}{\textbf{\ipa{ʐe˩ʐe˧bæ˩bæ˩}}}}\kern2pt]} \hypertarget{z`e\string_Bz`e\string_M-b\{\string_Bb\{\string_B1}{}
\markboth{\textcolor{darkblue}{\textbf{\ipa{ʐe˩ʐe˧-bæ˩bæ˩}}}}{}
\textcolor{teal}{\mytextsc{nom}} \hspace{4pt} Ton~: LM-L.
\textcolor{PineGreen}{\selectlanguage{french}Coton sauvage;; littéralement “la fleur des Occidentaux”.} \zh{野棉花(直译:‘洋人花’)。} \textit{Voir~:} \hyperlink{}{\textcolor{darkblue}{\textbf{\ipa{je˩ʐe˧}}}} 
\lhead{\firstmark}
\rhead{\botmark}

\subsection{\hspace{-0.5cm} {\Large \textcolor{darkblue}{\textbf{\ipa{ʐe˩ʐe˧-læ˧tsɯ˥}}}}\hspace{0.5cm}[\kern2pt{\textcolor{darkblue}{\textbf{\ipa{ʐe˩ʐe˧læ˧tsɯ˥}}}}\kern2pt]} \hypertarget{z`e\string_Bz`e\string_M-l\{\string_MtsM\string_T1}{}
\markboth{\textcolor{darkblue}{\textbf{\ipa{ʐe˩ʐe˧-læ˧tsɯ˥}}}}{}
\textcolor{teal}{\mytextsc{nom}} \hspace{4pt} Ton~: LM-H\#.
\textcolor{PineGreen}{\selectlanguage{french}Sorte de fourrage pour les cochons (il y en a trois en tout).} \zh{喂猪的牧草。} 
\lhead{\firstmark}
\rhead{\botmark}

\subsection{\hspace{-0.5cm} {\Large \textcolor{darkblue}{\textbf{\ipa{ʐɤ˧\textsubscript{b}}}}}\hspace{0.5cm}[\kern2pt{\textcolor{darkblue}{\textbf{\ipa{ʐɤ˩˥}}}}\kern2pt]} \hypertarget{z`7\string_Mb1}{}
\markboth{\textcolor{darkblue}{\textbf{\ipa{ʐɤ˧\textsubscript{b}}}}}{}
\textcolor{teal}{\mytextsc{verbe}} \hspace{4pt} Ton~: M\textsubscript{b}.
\textcolor{PineGreen}{\selectlanguage{french}Élever (des enfants ou des animaux); s'occuper de (personnes âgées).} \zh{饲养(动物)、养(孩子)、管(老人)。}  ¶ \textcolor{darkblue}{\textbf{\ipa{bo˩ ʐɤ˧}}} \textcolor{PineGreen}{\selectlanguage{french}élever des cochons} \zh{养猪}  
 ¶ \textcolor{darkblue}{\textbf{\ipa{ʐwæ˧zo˧ ʐɤ˧}}} \textcolor{PineGreen}{\selectlanguage{french}élever des poulains} \zh{养小马}  

\lhead{\firstmark}
\rhead{\botmark}

\subsection{\hspace{-0.5cm} {\Large \textcolor{darkblue}{\textbf{\ipa{ʐɤ˩\textsubscript{c}}}}}\hspace{0.5cm}[\kern2pt{\textcolor{darkblue}{\textbf{\ipa{ʐɤ˩˥}}}}\kern2pt]} \hypertarget{z`7\string_Bc1}{}
\markboth{\textcolor{darkblue}{\textbf{\ipa{ʐɤ˩\textsubscript{c}}}}}{}
\textcolor{teal}{\mytextsc{classificateur}} \hspace{4pt} Ton~: L\textsubscript{c}.
\textcolor{PineGreen}{\selectlanguage{french}Classificateur des motifs, tracés, lignes, dans les dessins, peintures et tissages.} \zh{量词:图案(画画或织布)(一个)。} 
\lhead{\firstmark}
\rhead{\botmark}

\subsection{\hspace{-0.5cm} {\Large \textcolor{darkblue}{\textbf{\ipa{ʐɤ˩\textsubscript{a}}}}}\hspace{0.5cm}[\kern2pt{\textcolor{darkblue}{\textbf{\ipa{ʐɤ˩˥}}}}\kern2pt]} \hypertarget{z`7\string_Ba1}{}
\markboth{\textcolor{darkblue}{\textbf{\ipa{ʐɤ˩\textsubscript{a}}}}}{}
\textcolor{teal}{\mytextsc{adjectif}} \hspace{4pt} Ton~: L\textsubscript{a}.
\textcolor{PineGreen}{\selectlanguage{french}Propre.} \zh{干净。}  ¶ \textcolor{darkblue}{\textbf{\ipa{ʈʂʰɯ˧ | ʐɤ˩-hĩ˩ ɲi˥}}} \textcolor{PineGreen}{\selectlanguage{french}c'est propre} \zh{这是干净的}  
 ¶ \textcolor{darkblue}{\textbf{\ipa{mɤ˧-ʐɤ˩}}} \textcolor{PineGreen}{\selectlanguage{french}crasseux, dégoûtant (vêtements, nourriture…)} \zh{不干净}  

\lhead{\firstmark}
\rhead{\botmark}

\subsection{\hspace{-0.5cm} {\Large \textcolor{darkblue}{\textbf{\ipa{ʐɤ˩mi˩}}}}\hspace{0.5cm}[\kern2pt{\textcolor{darkblue}{\textbf{\ipa{ʐɤ˩mi˩˥}}}}\kern2pt]} \hypertarget{z`7\string_Bmi\string_B1}{}
\markboth{\textcolor{darkblue}{\textbf{\ipa{ʐɤ˩mi˩}}}}{}
\textcolor{teal}{\mytextsc{nom}} \hspace{4pt} Ton~: L.
\textcolor{PineGreen}{\selectlanguage{french}Route.} \zh{路。}  ¶ \textcolor{darkblue}{\textbf{\ipa{hĩ˧ | ɖɯ˧-v̩˧\textasciitilde{}ɖɯ˧-v̩˧ | le˧-se˥, | ʐɤ˩mi˩ tsɤ˩˥!}}} \textcolor{PineGreen}{\selectlanguage{french}Contexte: on va couper du bois en montagne, à un endroit où il n'y a pas de chemin. Les gens se succèdent, et cela finit par ouvrir un chemin/former une sorte de chemin} \zh{路是人走出来的!}  
 \zh{量词}: \textcolor{darkblue}{\textbf{\ipa{kʰɯ˩}}}  \mytextsc{clf}~: \textcolor{darkblue}{\textbf{\ipa{kʰɯ˩}}} 
\lhead{\firstmark}
\rhead{\botmark}

\subsection{\hspace{-0.5cm} {\Large \textcolor{darkblue}{\textbf{\ipa{ʐɤ˩ni˩}}}}\hspace{0.5cm}[\kern2pt{\textcolor{darkblue}{\textbf{\ipa{ʐɤ˩ni˩˥}}}}\kern2pt]} \hypertarget{z`7\string_Bni\string_B1}{}
\markboth{\textcolor{darkblue}{\textbf{\ipa{ʐɤ˩ni˩}}}}{}
\textcolor{teal}{\mytextsc{adjectif}} \hspace{4pt} Ton~: L.
\textcolor{PineGreen}{\selectlanguage{french}Proche.} \zh{近。} 
\lhead{\firstmark}
\rhead{\botmark}

\subsection{\hspace{-0.5cm} {\Large \textcolor{darkblue}{\textbf{\ipa{ʐɤ˩qo˩}}}}\hspace{0.5cm}[\kern2pt{\textcolor{darkblue}{\textbf{\ipa{ʐɤ˩qo˩˥}}}}\kern2pt]} \hypertarget{z`7\string_Bqo\string_B1}{}
\markboth{\textcolor{darkblue}{\textbf{\ipa{ʐɤ˩qo˩}}}}{}
\textcolor{teal}{\mytextsc{nom}} \hspace{4pt} Ton~: L.
\ding{202} \textcolor{PineGreen}{\selectlanguage{french}Veau.} \zh{小牛。}  \zh{量词}: \textcolor{darkblue}{\textbf{\ipa{ɭɯ˧}}} \ding{203} \textcolor{PineGreen}{\selectlanguage{french}Pianniu, pienniu, dzo, zopiok (mâle).} \zh{公犏牛。}  \mytextsc{clf}~: \textcolor{darkblue}{\textbf{\ipa{ɭɯ˧}}} 
\lhead{\firstmark}
\rhead{\botmark}

\subsection{\hspace{-0.5cm} {\Large \textcolor{darkblue}{\textbf{\ipa{ʐɤ˩ʐɤ˧˥}}}}\hspace{0.5cm}[\kern2pt{\textcolor{darkblue}{\textbf{\ipa{ʐɤ˩ʐɤ˧˥}}}}\kern2pt]} \hypertarget{z`7\string_Bz`7\string_M\string_T1}{}
\markboth{\textcolor{darkblue}{\textbf{\ipa{ʐɤ˩ʐɤ˧˥}}}}{}
\textcolor{teal}{\mytextsc{nom}} \hspace{4pt} Ton~: LM+MH\#.
\textcolor{PineGreen}{\selectlanguage{french}Motif.} \zh{花纹、图案。}  ¶ \textcolor{darkblue}{\textbf{\ipa{ʐɤ˩ʐɤ˧ tʰi˧-di˥}}} \textcolor{PineGreen}{\selectlanguage{french}qui a des motifs, des dessins (ex.: un tissu)} \zh{有花纹}  
 ¶ \textcolor{darkblue}{\textbf{\ipa{[F5] bɑ˩lɑ˩˥ | ʈʰɯ˧-ɭɯ˥-bi˩ | ʐɤ˩ʐɤ˧ tʰi˧-di˥}}} \textcolor{PineGreen}{\selectlanguage{french}sur ce vêtement il y a un motif} \zh{这衣服上面有花纹。}  
 \zh{量词}: \textcolor{darkblue}{\textbf{\ipa{ʐɤ˩}}}  \mytextsc{clf}~: \textcolor{darkblue}{\textbf{\ipa{ʐɤ˩}}} 
\lhead{\firstmark}
\rhead{\botmark}

\subsection{\hspace{-0.5cm} {\Large \textcolor{darkblue}{\textbf{\ipa{ʐɤ˩˧}}}}\hspace{0.5cm}[\kern2pt{\textcolor{darkblue}{\textbf{\ipa{ʐɤ˥}}}}\kern2pt]} \hypertarget{z`7\string_B\string_M1}{}
\markboth{\textcolor{darkblue}{\textbf{\ipa{ʐɤ˩˧}}}}{}
\textcolor{teal}{\mytextsc{nom}} \hspace{4pt} Ton~: LM.
\textcolor{PineGreen}{\selectlanguage{french}Route (monosyllabe).} \zh{路(单音节)。}  ¶ \textcolor{darkblue}{\textbf{\ipa{ʐɤ˩mi˩-qo˥}}} \textcolor{PineGreen}{\selectlanguage{french}sur le chemin} \zh{路上}  
 ¶ \textcolor{darkblue}{\textbf{\ipa{ʐɤ˩mi˩-qo˥, | hĩ˧ se˧! |}}} \textcolor{PineGreen}{\selectlanguage{french}il y a des gens qui passent sur le chemin!} \zh{路上有人走!}  
 ¶ \textcolor{darkblue}{\textbf{\ipa{ʐɤ˩ se˩-zo˩˥}}} \textcolor{PineGreen}{\selectlanguage{french}voyageur, homme qui voyage; spécifiquement: personne partant faire du commerce en caravane} \zh{旅人,特别指走马帮的商人}  
 \zh{量词}: \textcolor{darkblue}{\textbf{\ipa{kʰɯ˩}}}  \mytextsc{clf}~: \textcolor{darkblue}{\textbf{\ipa{kʰɯ˩}}} \textcolor{PineGreen}{\selectlanguage{french}objets longs}
\lhead{\firstmark}
\rhead{\botmark}

\subsection{\hspace{-0.5cm} {\Large \textcolor{darkblue}{\textbf{\ipa{ʐo˩}}}}\hspace{0.5cm}[\kern2pt{\textcolor{darkblue}{\textbf{\ipa{ʐo˥}}}}\kern2pt]} \hypertarget{z`o\string_B1}{}
\markboth{\textcolor{darkblue}{\textbf{\ipa{ʐo˩}}}}{}
\textcolor{teal}{\mytextsc{nom}} \hspace{4pt} Ton~: L.
\textcolor{PineGreen}{\selectlanguage{french}Midi; repas de midi/déjeuner.} \zh{中午。}  ¶ \textcolor{darkblue}{\textbf{\ipa{ʐo˩ dzɯ˩˥}}} \textcolor{PineGreen}{\selectlanguage{french}prendre son déjeuner} \zh{吃午饭}  

\lhead{\firstmark}
\rhead{\botmark}

\subsection{\hspace{-0.5cm} {\Large \textcolor{darkblue}{\textbf{\ipa{ʐo˩\textsubscript{a}}}} \textsubscript{1}}\hspace{0.5cm}[\kern2pt{\textcolor{darkblue}{\textbf{\ipa{ʐo˥}}}}\kern2pt]} \hypertarget{z`o\string_Ba1}{}
\markboth{\textcolor{darkblue}{\textbf{\ipa{ʐo˩\textsubscript{a}}}} \textsubscript{1}}{}
\textcolor{teal}{\mytextsc{verbe}} \hspace{4pt} Ton~: L\textsubscript{a}.
\textcolor{PineGreen}{\selectlanguage{french}Se balancer.} \zh{甩来甩去。}  ¶ \textcolor{darkblue}{\textbf{\ipa{ɖɯ˧-ʐo˩-ɻ̍˩}}} \textcolor{PineGreen}{\selectlanguage{french}se balancer un peu} \zh{甩来甩去}  
 ¶ \textcolor{darkblue}{\textbf{\ipa{ʐo˩\textasciitilde{}ʐo˧-ze˥}}} \textcolor{PineGreen}{\selectlanguage{french}\mytextsc{red} \mytextsc{pfv}} \zh{\mytextsc{red} \mytextsc{pfv}}  
 ¶ \textcolor{darkblue}{\textbf{\ipa{[PHONO] le˧-ʐo˩\textasciitilde{}ʐo˩}}} \textcolor{PineGreen}{\selectlanguage{french}\mytextsc{accomp} \mytextsc{red}} \zh{\mytextsc{accomp} \mytextsc{red}}  

\lhead{\firstmark}
\rhead{\botmark}

\subsection{\hspace{-0.5cm} {\Large \textcolor{darkblue}{\textbf{\ipa{ʐo˩\textsubscript{a}}}} \textsubscript{2}}\hspace{0.5cm}[\kern2pt{\textcolor{darkblue}{\textbf{\ipa{ʐo˩˥}}}}\kern2pt]} \hypertarget{z`o\string_Ba2}{}
\markboth{\textcolor{darkblue}{\textbf{\ipa{ʐo˩\textsubscript{a}}}} \textsubscript{2}}{}
\textcolor{teal}{\mytextsc{adjectif}} \hspace{4pt} Ton~: L\textsubscript{a}.
\textcolor{PineGreen}{\selectlanguage{french}Léger.} \zh{轻。} 
\lhead{\firstmark}
\rhead{\botmark}

\subsection{\hspace{-0.5cm} {\Large \textcolor{darkblue}{\textbf{\ipa{ʐo˩dzɯ˩}}}}\hspace{0.5cm}[\kern2pt{\textcolor{darkblue}{\textbf{\ipa{ʐo˩dzɯ˩˥}}}}\kern2pt]} \hypertarget{z`o\string_BdzM\string_B1}{}
\markboth{\textcolor{darkblue}{\textbf{\ipa{ʐo˩dzɯ˩}}}}{}
\textcolor{teal}{\mytextsc{verbe}} \hspace{4pt} Ton~: L.
\textcolor{PineGreen}{\selectlanguage{french}Déjeuner, prendre le repas de midi.} \zh{吃午饭。}  ¶ \textcolor{darkblue}{\textbf{\ipa{ʐo˩ dzɯ˩˥}}} \textcolor{PineGreen}{\selectlanguage{french}déjeuner (verbe), prendre le déjeuner} \zh{吃午饭}  
 ¶ \textcolor{darkblue}{\textbf{\ipa{ʐo˩ dzɯ˩-se˥}}} \textcolor{PineGreen}{\selectlanguage{french}l'après-midi} \zh{下午}  

\lhead{\firstmark}
\rhead{\botmark}

\subsection{\hspace{-0.5cm} {\Large \textcolor{darkblue}{\textbf{\ipa{ʐo˩\textasciitilde{}ʐo˧˥}}}}\hspace{0.5cm}[\kern2pt{\textcolor{darkblue}{\textbf{\ipa{ʐo˧ʐo˧˥}}}}\kern2pt]} \hypertarget{z`o\string_B~z`o\string_M\string_T1}{}
\markboth{\textcolor{darkblue}{\textbf{\ipa{ʐo˩\textasciitilde{}ʐo˧˥}}}}{}
\textcolor{teal}{\mytextsc{verbe}} \hspace{4pt} Ton~: MH.
\textcolor{PineGreen}{\selectlanguage{french}Se balancer.} \zh{摔、摇摆。} 
\lhead{\firstmark}
\rhead{\botmark}

\subsection{\hspace{-0.5cm} {\Large \textcolor{darkblue}{\textbf{\ipa{ʐɯ˥}}}}\hspace{0.5cm}[\kern2pt{\textcolor{darkblue}{\textbf{\ipa{ʐɯ˥}}}}\kern2pt]} \hypertarget{z`M\string_T1}{}
\markboth{\textcolor{darkblue}{\textbf{\ipa{ʐɯ˥}}}}{}
\textcolor{teal}{\mytextsc{adjectif}} \hspace{4pt} Ton~: H.
\textcolor{PineGreen}{\selectlanguage{french}Lourd.} \zh{重。} 
\lhead{\firstmark}
\rhead{\botmark}

\subsection{\hspace{-0.5cm} {\Large \textcolor{darkblue}{\textbf{\ipa{ʐɯ˧}}}}\hspace{0.5cm}[\kern2pt{\textcolor{darkblue}{\textbf{\ipa{ʐɯ˥}}}}\kern2pt]} \hypertarget{z`M\string_M1}{}
\markboth{\textcolor{darkblue}{\textbf{\ipa{ʐɯ˧}}}}{}
\textcolor{teal}{\mytextsc{nom}} \hspace{4pt} Ton~: M.
\textcolor{PineGreen}{\selectlanguage{french}Alcool fermenté, chang, vin.} \zh{酒。}  ¶ \textcolor{darkblue}{\textbf{\ipa{ʐɯ˧ pʰv̩˧˥}}} \textcolor{PineGreen}{\selectlanguage{french}verser à boire} \zh{斟酒}  

\lhead{\firstmark}
\rhead{\botmark}

\subsection{\hspace{-0.5cm} {\Large \textcolor{darkblue}{\textbf{\ipa{ʐɯ˧ɭɯ˧}}}}\hspace{0.5cm}[\kern2pt{\textcolor{darkblue}{\textbf{\ipa{ʐɯ˧ɭɯ˧}}}}\kern2pt]} \hypertarget{z`M\string_Ml\string_RM\string_M1}{}
\markboth{\textcolor{darkblue}{\textbf{\ipa{ʐɯ˧ɭɯ˧}}}}{}
\textcolor{teal}{\mytextsc{verbe}} \hspace{4pt} Ton~: M.
\textcolor{PineGreen}{\selectlanguage{french}Tremblement de terre/la terre tremble.} \zh{地震。}  ¶ \textcolor{darkblue}{\textbf{\ipa{ʐɯ˧ɭɯ˧-ze˧!}}} \textcolor{PineGreen}{\selectlanguage{french}Il y a un tremblement de terre! / La terre tremble!} \zh{地震了!}  

\lhead{\firstmark}
\rhead{\botmark}

\subsection{\hspace{-0.5cm} {\Large \textcolor{darkblue}{\textbf{\ipa{ʐɯ˧nɑ˩}}}}\hspace{0.5cm}[\kern2pt{\textcolor{darkblue}{\textbf{\ipa{ʐɯ˧nɑ˩}}}}\kern2pt]} \hypertarget{z`M\string_MnA\string_B1}{}
\markboth{\textcolor{darkblue}{\textbf{\ipa{ʐɯ˧nɑ˩}}}}{}
\textcolor{teal}{\mytextsc{nom}} \hspace{4pt} Ton~: L\#.
\textcolor{PineGreen}{\selectlanguage{french}Alcool fort; alcool de qualité supérieure.} \zh{醇酒,好酒。} 
\lhead{\firstmark}
\rhead{\botmark}

\subsection{\hspace{-0.5cm} {\Large \textcolor{darkblue}{\textbf{\ipa{ʐɯ˩dzi˥}}}}\hspace{0.5cm}[\kern2pt{\textcolor{darkblue}{\textbf{\ipa{ʐɯ˩dzi˥}}}}\kern2pt]} \hypertarget{z`M\string_Bdzi\string_T1}{}
\markboth{\textcolor{darkblue}{\textbf{\ipa{ʐɯ˩dzi˥}}}}{}
\textcolor{teal}{\mytextsc{nom}} \hspace{4pt} Ton~: LH.
\textcolor{PineGreen}{\selectlanguage{french}Cèdre.} \zh{杉树。}  \zh{量词}: \textcolor{darkblue}{\textbf{\ipa{dzi˩}}}  \mytextsc{clf}~: \textcolor{darkblue}{\textbf{\ipa{dzi˩}}} 
\lhead{\firstmark}
\rhead{\botmark}

\subsection{\hspace{-0.5cm} {\Large \textcolor{darkblue}{\textbf{\ipa{ʐɯ˩gv̩˩}}}}\hspace{0.5cm}[\kern2pt{\textcolor{darkblue}{\textbf{\ipa{ʐɯ˩gv̩˩˥}}}}\kern2pt]} \hypertarget{z`M\string_Bgv\string_=\string_B1}{}
\markboth{\textcolor{darkblue}{\textbf{\ipa{ʐɯ˩gv̩˩}}}}{}
\textcolor{teal}{\mytextsc{nom}} \hspace{4pt} Ton~: L.
\textcolor{PineGreen}{\selectlanguage{french}Canot, bateau (utilisé uniquement pour les barques circulant sur le Lac, pas pour les autres bateaux).} \zh{船。}  ¶ \textcolor{darkblue}{\textbf{\ipa{ʐɯ˩gv̩˩ dzi˩˥}}} \textcolor{PineGreen}{\selectlanguage{french}être assis dans un bateau, être à bord d'un bateau} \zh{坐船}  
 \zh{量词}: \textcolor{darkblue}{\textbf{\ipa{ɭɯ˧}}} \textcolor{darkblue}{\textbf{\ipa{nɑ˧}}}  \mytextsc{clf}~: \textcolor{darkblue}{\textbf{\ipa{ɭɯ˧}}} \textcolor{darkblue}{\textbf{\ipa{nɑ˧}}} 
\lhead{\firstmark}
\rhead{\botmark}

\subsection{\hspace{-0.5cm} {\Large \textcolor{darkblue}{\textbf{\ipa{ʐɯ˩-mo˧˥}}}}\hspace{0.5cm}[\kern2pt{\textcolor{darkblue}{\textbf{\ipa{xxxx non-correspondance entre le nombre de morphèmes et le nombre de tons de morphèmes}}}}\kern2pt]} \hypertarget{z`M\string_B-mo\string_M\string_T1}{}
\markboth{\textcolor{darkblue}{\textbf{\ipa{ʐɯ˩-mo˧˥}}}}{}
\textcolor{teal}{\mytextsc{nom}} \hspace{4pt} Ton~: LM+MH\#.
\textcolor{PineGreen}{\selectlanguage{french}“champignon des cèdres”; champignon comestible, de la même famille que le “champignon des sapins”, \textcolor{darkblue}{\textbf{\ipa{/tʰo˧-mo˩/}}}.} \zh{“杉树菌”:一种菌子。}  ¶ \textcolor{darkblue}{\textbf{\ipa{tʰo˧mo˩-ʐɯ˩mo˩}}} \textcolor{PineGreen}{\selectlanguage{french}champignon des sapins et champignon des cèdres} \zh{松树菌与杉树菌}  
\textit{Voir~:} \hyperlink{}{\textcolor{darkblue}{\textbf{\ipa{ʐɯ˩dzi˥}}}} 
\lhead{\firstmark}
\rhead{\botmark}

\subsection{\hspace{-0.5cm} {\Large \textcolor{darkblue}{\textbf{\ipa{ʐɯ˩tse˧}}}}\hspace{0.5cm}[\kern2pt{\textcolor{darkblue}{\textbf{\ipa{ʐɯ˩tse˥}}}}\kern2pt]} \hypertarget{z`M\string_Btse\string_M1}{}
\markboth{\textcolor{darkblue}{\textbf{\ipa{ʐɯ˩tse˧}}}}{}
\textcolor{teal}{\mytextsc{nom}} \hspace{4pt} Ton~: LM.
\textcolor{PineGreen}{\selectlanguage{french}Esprit de la montagne.} \zh{山神。}  \zh{量词}: \textcolor{darkblue}{\textbf{\ipa{v̩˧}}}  \mytextsc{clf}~: \textcolor{darkblue}{\textbf{\ipa{v̩˧}}} 
\lhead{\firstmark}
\rhead{\botmark}

\subsection{\hspace{-0.5cm} {\Large \textcolor{darkblue}{\textbf{\ipa{ʐɯ˩tse˧-mæ˧ʂæ˩}}}}\hspace{0.5cm}[\kern2pt{\textcolor{darkblue}{\textbf{\ipa{ʐɯ˩tse˧mæ˧ʂæ˩}}}}\kern2pt]} \hypertarget{z`M\string_Btse\string_M-m\{\string_Ms`\{\string_B1}{}
\markboth{\textcolor{darkblue}{\textbf{\ipa{ʐɯ˩tse˧-mæ˧ʂæ˩}}}}{}
\textcolor{teal}{\mytextsc{nom}} \hspace{4pt} Ton~: LM-L\#.
\textcolor{PineGreen}{\selectlanguage{french}Faisan doré.} \zh{锦鸡。} Dialecte chinois local~:\zh{山扎拉。}\textit{Voir~:} \hyperlink{}{\textcolor{darkblue}{\textbf{\ipa{ʐɯ˩tse˧}}}} 
\lhead{\firstmark}
\rhead{\botmark}

\subsection{\hspace{-0.5cm} {\Large \textcolor{darkblue}{\textbf{\ipa{ʐɯ˩tsɯ˧}}}}\hspace{0.5cm}[\kern2pt{\textcolor{darkblue}{\textbf{\ipa{ʐɯ˩tsɯ˥}}}}\kern2pt]} \hypertarget{z`M\string_BtsM\string_M1}{}
\markboth{\textcolor{darkblue}{\textbf{\ipa{ʐɯ˩tsɯ˧}}}}{}
\textcolor{teal}{\mytextsc{nom}} \hspace{4pt} Ton~: LM.
\textcolor{PineGreen}{\selectlanguage{french}Jours, temps.} \zh{日子(汉语借词)。}  Emprunt~: chinois  \zh{日子}
 ¶ \textcolor{darkblue}{\textbf{\ipa{ʐɯ˩tsɯ˧ ʈʂɤ˧}}} \textcolor{PineGreen}{\selectlanguage{french}rechercher une date propice (pour la construction d'une maison ou autre projet important)} \zh{算日子(为了选择吉利的一天)}  

\lhead{\firstmark}
\rhead{\botmark}

\subsection{\hspace{-0.5cm} {\Large \textcolor{darkblue}{\textbf{\ipa{ʐɯ˩tsɯ˧mɤ˩ʈʂʰɤ˩}}}}\hspace{0.5cm}[\kern2pt{\textcolor{darkblue}{\textbf{\ipa{xxxx non-correspondance entre le nombre de morphèmes et le nombre de tons de morphèmes}}}}\kern2pt]} \hypertarget{z`M\string_BtsM\string_Mm7\string_Bt`s`\string_h7\string_B1}{}
\markboth{\textcolor{darkblue}{\textbf{\ipa{ʐɯ˩tsɯ˧mɤ˩ʈʂʰɤ˩}}}}{}
\textcolor{teal}{\mytextsc{nom}} \hspace{4pt} Ton~: LM-L.
\textcolor{PineGreen}{\selectlanguage{french}Matelas.} \zh{褥子。}  \zh{量词}: \textcolor{darkblue}{\textbf{\ipa{tsʰi˥}}}  \mytextsc{clf}~: \textcolor{darkblue}{\textbf{\ipa{tsʰi˥}}} 
\lhead{\firstmark}
\rhead{\botmark}

\subsection{\hspace{-0.5cm} {\Large \textcolor{darkblue}{\textbf{\ipa{ʐv̩˧}}}}\hspace{0.5cm}[\kern2pt{\textcolor{darkblue}{\textbf{\ipa{ʐv̩˥}}}}\kern2pt]} \hypertarget{z`v\string_=\string_M1}{}
\markboth{\textcolor{darkblue}{\textbf{\ipa{ʐv̩˧}}}}{}
\textcolor{teal}{\mytextsc{nombre}} \hspace{4pt} Ton~: M? H\#?.
\textcolor{PineGreen}{\selectlanguage{french}4.} \zh{4。} 
\lhead{\firstmark}
\rhead{\botmark}

\subsection{\hspace{-0.5cm} {\Large \textcolor{darkblue}{\textbf{\ipa{ʐv̩˧˥}}}}\hspace{0.5cm}[\kern2pt{\textcolor{darkblue}{\textbf{\ipa{ʐv̩˧˥}}}}\kern2pt]} \hypertarget{z`v\string_=\string_M\string_T1}{}
\markboth{\textcolor{darkblue}{\textbf{\ipa{ʐv̩˧˥}}}}{}
\textcolor{teal}{\mytextsc{verbe}} \hspace{4pt} Ton~: MH.
\textcolor{PineGreen}{\selectlanguage{french}Coudre.} \zh{缝。} 
\lhead{\firstmark}
\rhead{\botmark}

\subsection{\hspace{-0.5cm} {\Large \textcolor{darkblue}{\textbf{\ipa{ʐv̩˩\textsubscript{a}}}} \textsubscript{1}}\hspace{0.5cm}[\kern2pt{\textcolor{darkblue}{\textbf{\ipa{ʐv̩˩˥}}}}\kern2pt]} \hypertarget{z`v\string_=\string_Ba1}{}
\markboth{\textcolor{darkblue}{\textbf{\ipa{ʐv̩˩\textsubscript{a}}}} \textsubscript{1}}{}
\textcolor{teal}{\mytextsc{verbe}} \hspace{4pt} Ton~: L\textsubscript{a}.
\ding{202} \textcolor{PineGreen}{\selectlanguage{french}Pétrir (la pâte), malaxer.} \zh{揉(面)。}  ¶ \textcolor{darkblue}{\textbf{\ipa{pɤ˩jɤ˧ ʐv̩˥}}} \textcolor{PineGreen}{\selectlanguage{french}pétrir la pâte} \zh{揉面}  
 ¶ \textcolor{darkblue}{\textbf{\ipa{ʐv̩˧\textasciitilde{}ʐv̩˥}}} \textcolor{PineGreen}{\selectlanguage{french}\mytextsc{red}} \zh{\mytextsc{重叠}}  
 ¶ \textcolor{darkblue}{\textbf{\ipa{ɖɯ˧-kʰwɤ˧ ʐv̩˥}}} \textcolor{PineGreen}{\selectlanguage{french}pétrir un morceau} \zh{揉一块(面团)}  
\ding{203} \textcolor{PineGreen}{\selectlanguage{french}Froisser, plisser.} \zh{皱(衣服)。}  ¶ \textcolor{darkblue}{\textbf{\ipa{bɑ˩lɑ˩ ʐv̩˥(-ze˩)}}} \textcolor{PineGreen}{\selectlanguage{french}les vêtements sont froissés, les vêtements ont été froissés} \zh{衣服皱了}  
 ¶ \textcolor{darkblue}{\textbf{\ipa{ʐv̩˧\textasciitilde{}ʐv̩˥}}} \textcolor{PineGreen}{\selectlanguage{french}\mytextsc{red}} \zh{\mytextsc{重叠}}  
 ¶ \textcolor{darkblue}{\textbf{\ipa{le˧-ʐv̩˧\textasciitilde{}ʐv̩˥-ze˩}}} \textcolor{PineGreen}{\selectlanguage{french}\mytextsc{accomp} \string_ \mytextsc{red} \mytextsc{pfv}} \zh{\mytextsc{accomp} \string_ \mytextsc{red} \mytextsc{pfv}}  

\lhead{\firstmark}
\rhead{\botmark}

\subsection{\hspace{-0.5cm} {\Large \textcolor{darkblue}{\textbf{\ipa{ʐv̩˩\textsubscript{a}}}} \textsubscript{2}}\hspace{0.5cm}[\kern2pt{\textcolor{darkblue}{\textbf{\ipa{ʐv̩˩˥}}}}\kern2pt]} \hypertarget{z`v\string_=\string_Ba2}{}
\markboth{\textcolor{darkblue}{\textbf{\ipa{ʐv̩˩\textsubscript{a}}}} \textsubscript{2}}{}
\textcolor{teal}{\mytextsc{adjectif}} \hspace{4pt} Ton~: L\textsubscript{a}.
\textcolor{PineGreen}{\selectlanguage{french}Bon (au goût).} \zh{好吃。} 
\lhead{\firstmark}
\rhead{\botmark}

\subsection{\hspace{-0.5cm} {\Large \textcolor{darkblue}{\textbf{\ipa{ʐv̩˧bæ˧}}}}\hspace{0.5cm}[\kern2pt{\textcolor{darkblue}{\textbf{\ipa{ʐv̩˧bæ˧}}}}\kern2pt]} \hypertarget{z`v\string_=\string_Mb\{\string_M1}{}
\markboth{\textcolor{darkblue}{\textbf{\ipa{ʐv̩˧bæ˧}}}}{}
\textcolor{teal}{\mytextsc{nom}} \hspace{4pt} Ton~: M.
\textcolor{PineGreen}{\selectlanguage{french}Serpent.} \zh{蛇。}  ¶ \textcolor{darkblue}{\textbf{\ipa{ʐv̩˧bæ˧ ɣɯ˩ pʰv̩˩}}} \textcolor{PineGreen}{\selectlanguage{french}Le serpent mue} \zh{蛇蜕皮}  
 \zh{量词}: \textcolor{darkblue}{\textbf{\ipa{mi˩}}}  \mytextsc{clf}~: \textcolor{darkblue}{\textbf{\ipa{mi˩}}} 
\lhead{\firstmark}
\rhead{\botmark}

\subsection{\hspace{-0.5cm} {\Large \textcolor{darkblue}{\textbf{\ipa{ʐv̩˧bæ˧-bv̩˧-hɑ\#˥}}}}\hspace{0.5cm}[\kern2pt{\textcolor{darkblue}{\textbf{\ipa{xxxx non-correspondance entre le nombre de morphèmes et le nombre de tons de morphèmes}}}}\kern2pt]} \hypertarget{z`v\string_=\string_Mb\{\string_M-bv\string_=\string_M-hA\#\string_T1}{}
\markboth{\textcolor{darkblue}{\textbf{\ipa{ʐv̩˧bæ˧-bv̩˧-hɑ\#˥}}}}{}
\textcolor{teal}{\mytextsc{nom}} \hspace{4pt} Ton~: \#H.
\textcolor{PineGreen}{\selectlanguage{french}L'une des trois sortes de fourrage que l'on donne aux cochons; M18 propose comme étymologie populaire “serpent en colère”, du fait que cette plante a des feuilles grasses qui ressemblent à un serpent, et qui sont entortillées.} \zh{能喂给猪的三种草之一。}  \zh{量词}: \textcolor{darkblue}{\textbf{\ipa{qɑ˩}}}  \mytextsc{clf}~: \textcolor{darkblue}{\textbf{\ipa{qɑ˩}}} \textcolor{PineGreen}{\selectlanguage{french}botte}
\lhead{\firstmark}
\rhead{\botmark}

\subsection{\hspace{-0.5cm} {\Large \textcolor{darkblue}{\textbf{\ipa{ʐv̩˧bæ˧-pʰv̩\#˥}}}}\hspace{0.5cm}[\kern2pt{\textcolor{darkblue}{\textbf{\ipa{xxxx non-correspondance entre le nombre de morphèmes et le nombre de tons de morphèmes}}}}\kern2pt]} \hypertarget{z`v\string_=\string_Mb\{\string_M-p\string_hv\string_=\#\string_T1}{}
\markboth{\textcolor{darkblue}{\textbf{\ipa{ʐv̩˧bæ˧-pʰv̩\#˥}}}}{}
\textcolor{teal}{\mytextsc{nom}} \hspace{4pt} Ton~: \#H.
\textcolor{PineGreen}{\selectlanguage{french}Serpent mâle.} \zh{公蛇。}  \zh{量词}: \textcolor{darkblue}{\textbf{\ipa{mi˩}}}  \mytextsc{clf}~: \textcolor{darkblue}{\textbf{\ipa{mi˩}}} 
\lhead{\firstmark}
\rhead{\botmark}

\subsection{\hspace{-0.5cm} {\Large \textcolor{darkblue}{\textbf{\ipa{ʐv̩˧bæ˧-zo\#˥}}}}\hspace{0.5cm}[\kern2pt{\textcolor{darkblue}{\textbf{\ipa{xxxx non-correspondance entre le nombre de morphèmes et le nombre de tons de morphèmes}}}}\kern2pt]} \hypertarget{z`v\string_=\string_Mb\{\string_M-zo\#\string_T1}{}
\markboth{\textcolor{darkblue}{\textbf{\ipa{ʐv̩˧bæ˧-zo\#˥}}}}{}
\textcolor{teal}{\mytextsc{nom}} \hspace{4pt} Ton~: \#H.
\textcolor{PineGreen}{\selectlanguage{french}Petit serpent.} \zh{小蛇。}  \zh{量词}: \textcolor{darkblue}{\textbf{\ipa{ɭɯ˧}}}  \mytextsc{clf}~: \textcolor{darkblue}{\textbf{\ipa{ɭɯ˧}}} 
\lhead{\firstmark}
\rhead{\botmark}

\subsection{\hspace{-0.5cm} {\Large \textcolor{darkblue}{\textbf{\ipa{ʐv̩˧bɤ\#˥}}}}\hspace{0.5cm}[\kern2pt{\textcolor{darkblue}{\textbf{\ipa{ʐv̩˧bɤ˧}}}}\kern2pt]} \hypertarget{z`v\string_=\string_Mb7\#\string_T1}{}
\markboth{\textcolor{darkblue}{\textbf{\ipa{ʐv̩˧bɤ\#˥}}}}{}
\textcolor{teal}{\mytextsc{nom}} \hspace{4pt} Ton~: \#H.
\textcolor{PineGreen}{\selectlanguage{french}Les Pumi des montagnes, du côté de Muli et Jiaze.} \zh{高山普米族(永宁以北地区:木里等)。}  \zh{量词}: \textcolor{darkblue}{\textbf{\ipa{v̩˧}}}  \mytextsc{clf}~: \textcolor{darkblue}{\textbf{\ipa{v̩˧}}} 
\lhead{\firstmark}
\rhead{\botmark}

\subsection{\hspace{-0.5cm} {\Large \textcolor{darkblue}{\textbf{\ipa{ʐv̩˧di˧˥}}}}\hspace{0.5cm}[\kern2pt{\textcolor{darkblue}{\textbf{\ipa{ʐv̩˧di˧˥}}}}\kern2pt]} \hypertarget{z`v\string_=\string_Mdi\string_M\string_T1}{}
\markboth{\textcolor{darkblue}{\textbf{\ipa{ʐv̩˧di˧˥}}}}{}
\textcolor{teal}{\mytextsc{nom}} \hspace{4pt} Ton~: MH\#.
\textcolor{PineGreen}{\selectlanguage{french}Les rives du Yangtze; le climat y est chaud et humide. Ces régions sont perçus par les Na de Yongning comme peuplées de Pumi; ils imaginent que les habitants de Fengke et Labai seraient des descendants des Pumi. (Source: consultants F4, F5, M21.).} \zh{金沙江边的地方(气候热)。} 
\lhead{\firstmark}
\rhead{\botmark}

\subsection{\hspace{-0.5cm} {\Large \textcolor{darkblue}{\textbf{\ipa{ʐv̩˧dzi˩}}}}\hspace{0.5cm}[\kern2pt{\textcolor{darkblue}{\textbf{\ipa{ʐv̩˧dzi˩}}}}\kern2pt]} \hypertarget{z`v\string_=\string_Mdzi\string_B1}{}
\markboth{\textcolor{darkblue}{\textbf{\ipa{ʐv̩˧dzi˩}}}}{}
\textcolor{teal}{\mytextsc{nom}} \hspace{4pt} Ton~: L\#.
\textcolor{PineGreen}{\selectlanguage{french}Saule.} \zh{柳树,杨柳。}  \zh{量词}: \textcolor{darkblue}{\textbf{\ipa{dzi˩}}}  \mytextsc{clf}~: \textcolor{darkblue}{\textbf{\ipa{dzi˩}}} 
\lhead{\firstmark}
\rhead{\botmark}

\subsection{\hspace{-0.5cm} {\Large \textcolor{darkblue}{\textbf{\ipa{ʐv̩˧hĩ\#˥}}}}\hspace{0.5cm}[\kern2pt{\textcolor{darkblue}{\textbf{\ipa{ʐv̩˧hĩ˧}}}}\kern2pt]} \hypertarget{z`v\string_=\string_Mhi\string_~\#\string_T1}{}
\markboth{\textcolor{darkblue}{\textbf{\ipa{ʐv̩˧hĩ\#˥}}}}{}
\textcolor{teal}{\mytextsc{nom}} \hspace{4pt} Ton~: \#H.
\textcolor{PineGreen}{\selectlanguage{french}Désignation des Pumi.} \zh{普米族。}  \zh{量词}: \textcolor{darkblue}{\textbf{\ipa{v̩˧}}}  \mytextsc{clf}~: \textcolor{darkblue}{\textbf{\ipa{v̩˧}}} 
\lhead{\firstmark}
\rhead{\botmark}

\subsection{\hspace{-0.5cm} {\Large \textcolor{darkblue}{\textbf{\ipa{ʐv̩˩-ɬi˩mi˩}}}}\hspace{0.5cm}[\kern2pt{\textcolor{darkblue}{\textbf{\ipa{xxxx non-correspondance entre le nombre de morphèmes et le nombre de tons de morphèmes}}}}\kern2pt]} \hypertarget{z`v\string_=\string_B-Ki\string_Bmi\string_B1}{}
\markboth{\textcolor{darkblue}{\textbf{\ipa{ʐv̩˩-ɬi˩mi˩}}}}{}
\textcolor{teal}{\mytextsc{nom}} \hspace{4pt} Ton~: L.
\textcolor{PineGreen}{\selectlanguage{french}4e mois.} \zh{四月。} 
\lhead{\firstmark}
\rhead{\botmark}

\subsection{\hspace{-0.5cm} {\Large \textcolor{darkblue}{\textbf{\ipa{ʐv̩˩ɭɯ˥}}}}\hspace{0.5cm}[\kern2pt{\textcolor{darkblue}{\textbf{\ipa{ʐv̩˩ɭɯ˥}}}}\kern2pt]} \hypertarget{z`v\string_=\string_Bl\string_RM\string_T1}{}
\markboth{\textcolor{darkblue}{\textbf{\ipa{ʐv̩˩ɭɯ˥}}}}{}
\textcolor{teal}{\mytextsc{nom}} \hspace{4pt} Ton~: LH.
\textcolor{PineGreen}{\selectlanguage{french}Poutre soutenant la toiture, posée horizontalement, dans le sens de la longueur du bâtiment. Sur elle reposent les poutrelles courtes posées inclinées dans le sens de la largeur du bâtiment, /hæ̃˧kʰɤ˧˥/.} \zh{支撑顶板的梁。}  \zh{量词}: \textcolor{darkblue}{\textbf{\ipa{ɭɯ˧}}}  \mytextsc{clf}~: \textcolor{darkblue}{\textbf{\ipa{ɭɯ˧}}} 
\lhead{\firstmark}
\rhead{\botmark}

\subsection{\hspace{-0.5cm} {\Large \textcolor{darkblue}{\textbf{\ipa{ʐv̩˩mi˩}}}}\hspace{0.5cm}[\kern2pt{\textcolor{darkblue}{\textbf{\ipa{ʐv̩˩mi˩˥}}}}\kern2pt]} \hypertarget{z`v\string_=\string_Bmi\string_B1}{}
\markboth{\textcolor{darkblue}{\textbf{\ipa{ʐv̩˩mi˩}}}}{}
\textcolor{teal}{\mytextsc{nom}} \hspace{4pt} Ton~: L.
\textcolor{PineGreen}{\selectlanguage{french}Arc.} \zh{弓。}  \zh{量词}: \textcolor{darkblue}{\textbf{\ipa{nɑ˧}}}  \mytextsc{clf}~: \textcolor{darkblue}{\textbf{\ipa{nɑ˧}}} 
\lhead{\firstmark}
\rhead{\botmark}

\subsection{\hspace{-0.5cm} {\Large \textcolor{darkblue}{\textbf{\ipa{ʐv̩˧mi\#˥}}}}\hspace{0.5cm}[\kern2pt{\textcolor{darkblue}{\textbf{\ipa{ʐv̩˧mi˧}}}}\kern2pt]} \hypertarget{z`v\string_=\string_Mmi\#\string_T1}{}
\markboth{\textcolor{darkblue}{\textbf{\ipa{ʐv̩˧mi\#˥}}}}{}
\textcolor{teal}{\mytextsc{nom}} \hspace{4pt} Ton~: \#H.
\textcolor{PineGreen}{\selectlanguage{french}Petite-fille.} \zh{孙女。}  ¶ \textcolor{darkblue}{\textbf{\ipa{njɤ˧ | ʐv̩˧mi˧ | ɖɯ˧-ɭɯ˧ dʑo˧}}} \textcolor{PineGreen}{\selectlanguage{french}J'ai une petite-fille.} \zh{我有一个孙女。}  
 \zh{量词}: \textcolor{darkblue}{\textbf{\ipa{ɭɯ˧}}}  \mytextsc{clf}~: \textcolor{darkblue}{\textbf{\ipa{ɭɯ˧}}} 
\lhead{\firstmark}
\rhead{\botmark}

\subsection{\hspace{-0.5cm} {\Large \textcolor{darkblue}{\textbf{\ipa{ʐv̩˧mv̩˧lɑ˧di˧˥}}}}\hspace{0.5cm}[\kern2pt{\textcolor{darkblue}{\textbf{\ipa{ʐv̩˧mv̩˧lɑ˧di˧˥}}}}\kern2pt]} \hypertarget{z`v\string_=\string_Mmv\string_=\string_MlA\string_Mdi\string_M\string_T1}{}
\markboth{\textcolor{darkblue}{\textbf{\ipa{ʐv̩˧mv̩˧lɑ˧di˧˥}}}}{}
\textcolor{teal}{\mytextsc{nom}} \hspace{4pt} Ton~: MH\#.
\textcolor{PineGreen}{\selectlanguage{french}Les territoires des Pumi, au bord du fleuve Yangtze. Le terme est connoté péjorativement: cette région est perçue comme périphérique et moins plaisante que la plaine de Yongning.} \zh{江边普米族地区(带偏见的说法)。}  ¶ \textcolor{darkblue}{\textbf{\ipa{ʐv̩˧mv̩˧lɑ˧di˧-hĩ˥}}} \textcolor{PineGreen}{\selectlanguage{french}habitants des territoires pumi des bords du fleuve; personnes pumi} \zh{普米族地区的人们}  
 \zh{量词}: \textcolor{darkblue}{\textbf{\ipa{v̩˧}}}  \mytextsc{clf}~: \textcolor{darkblue}{\textbf{\ipa{v̩˧}}} 
\lhead{\firstmark}
\rhead{\botmark}

\subsection{\hspace{-0.5cm} {\Large \textcolor{darkblue}{\textbf{\ipa{ʐv̩˧-ɲi˧-ʁo˧tʰo˥}}}}\hspace{0.5cm}[\kern2pt{\textcolor{darkblue}{\textbf{\ipa{xxxx non-correspondance entre le nombre de morphèmes et le nombre de tons de morphèmes}}}}\kern2pt]} \hypertarget{z`v\string_=\string_M-Ji\string_M-Ro\string_Mt\string_ho\string_T1}{}
\markboth{\textcolor{darkblue}{\textbf{\ipa{ʐv̩˧-ɲi˧-ʁo˧tʰo˥}}}}{}
\textcolor{teal}{\mytextsc{adverbe}} \hspace{4pt} Ton~: H\#.
\textcolor{PineGreen}{\selectlanguage{french}Dans quatre jours.} \zh{四天以后。} 
\lhead{\firstmark}
\rhead{\botmark}

\subsection{\hspace{-0.5cm} {\Large \textcolor{darkblue}{\textbf{\ipa{ʐv̩˧ɻ̍˥}}}}\hspace{0.5cm}[\kern2pt{\textcolor{darkblue}{\textbf{\ipa{ʐv̩˧ɻ̍˥}}}}\kern2pt]} \hypertarget{z`v\string_=\string_Mr£`̍\string_T1}{}
\markboth{\textcolor{darkblue}{\textbf{\ipa{ʐv̩˧ɻ̍˥}}}}{}
\textcolor{teal}{\mytextsc{adjectif}} \hspace{4pt} Ton~: H\#.
\textcolor{PineGreen}{\selectlanguage{french}Carré.} \zh{正方形。}  ¶ \textcolor{darkblue}{\textbf{\ipa{ʐv̩˩-hĩ˩˥}}} \textcolor{PineGreen}{\selectlanguage{french}\mytextsc{nmlz}} \zh{方形的}  
 ¶ \textcolor{darkblue}{\textbf{\ipa{ʐv̩˧ɻ̍˥-gv̩˩}}} \textcolor{PineGreen}{\selectlanguage{french}carré} \zh{方形的}  

\lhead{\firstmark}
\rhead{\botmark}

\subsection{\hspace{-0.5cm} {\Large \textcolor{darkblue}{\textbf{\ipa{ʐv̩˧-tsʰi˩}}}}\hspace{0.5cm}[\kern2pt{\textcolor{darkblue}{\textbf{\ipa{xxxx non-correspondance entre le nombre de morphèmes et le nombre de tons de morphèmes}}}}\kern2pt]} \hypertarget{z`v\string_=\string_M-ts\string_hi\string_B1}{}
\markboth{\textcolor{darkblue}{\textbf{\ipa{ʐv̩˧-tsʰi˩}}}}{}
\textcolor{teal}{\mytextsc{nombre}} \hspace{4pt} Ton~: L\#.
\textcolor{PineGreen}{\selectlanguage{french}40.} \zh{40。} 
\lhead{\firstmark}
\rhead{\botmark}

\subsection{\hspace{-0.5cm} {\Large \textcolor{darkblue}{\textbf{\ipa{ʐv̩˧v̩˥-ʐv̩˩mi˩}}}}\hspace{0.5cm}[\kern2pt{\textcolor{darkblue}{\textbf{\ipa{ʐv̩˧v̩˥ʐv̩˩mi˩}}}}\kern2pt]} \hypertarget{z`v\string_=\string_Mv\string_=\string_T-z`v\string_=\string_Bmi\string_B1}{}
\markboth{\textcolor{darkblue}{\textbf{\ipa{ʐv̩˧v̩˥-ʐv̩˩mi˩}}}}{}
\textcolor{teal}{\mytextsc{nom}} \hspace{4pt} Ton~: H\#-.
\textcolor{PineGreen}{\selectlanguage{french}Petits-enfants.} \zh{孙子孙女。} 
\lhead{\firstmark}
\rhead{\botmark}

\subsection{\hspace{-0.5cm} {\Large \textcolor{darkblue}{\textbf{\ipa{ʐv̩˧v̩\#˥}}}}\hspace{0.5cm}[\kern2pt{\textcolor{darkblue}{\textbf{\ipa{ʐv̩˧v̩˧}}}}\kern2pt]} \hypertarget{z`v\string_=\string_Mv\string_=\#\string_T1}{}
\markboth{\textcolor{darkblue}{\textbf{\ipa{ʐv̩˧v̩\#˥}}}}{}
\textcolor{teal}{\mytextsc{nom}} \hspace{4pt} Ton~: \#H.
\textcolor{PineGreen}{\selectlanguage{french}Petit-fils.} \zh{孙子。}  ¶ \textcolor{darkblue}{\textbf{\ipa{njɤ˧ | ʐv̩˧v̩˧ ɖɯ˧-ɭɯ˧ dʑo˧.}}} \textcolor{PineGreen}{\selectlanguage{french}j'ai un petit-fils} \zh{我有一个孙子。}  
 \zh{量词}: \textcolor{darkblue}{\textbf{\ipa{ɭɯ˧}}}  \mytextsc{clf}~: \textcolor{darkblue}{\textbf{\ipa{ɭɯ˧}}} \textcolor{PineGreen}{\selectlanguage{french}objets ronds}
\lhead{\firstmark}
\rhead{\botmark}

\subsection{\hspace{-0.5cm} {\Large \textcolor{darkblue}{\textbf{\ipa{ʐv̩˧-zo\#˥}}}}\hspace{0.5cm}[\kern2pt{\textcolor{darkblue}{\textbf{\ipa{xxxx non-correspondance entre le nombre de morphèmes et le nombre de tons de morphèmes}}}}\kern2pt]} \hypertarget{z`v\string_=\string_M-zo\#\string_T1}{}
\markboth{\textcolor{darkblue}{\textbf{\ipa{ʐv̩˧-zo\#˥}}}}{}
\textcolor{teal}{\mytextsc{nom}} \hspace{4pt} Ton~: \#H.
\textcolor{PineGreen}{\selectlanguage{french}Petit arc.} \zh{小弓。}  \zh{量词}: \textcolor{darkblue}{\textbf{\ipa{nɑ˧}}}  \mytextsc{clf}~: \textcolor{darkblue}{\textbf{\ipa{nɑ˧}}} 
\lhead{\firstmark}
\rhead{\botmark}

\subsection{\hspace{-0.5cm} {\Large \textcolor{darkblue}{\textbf{\ipa{ʐwæ˥}}}}\hspace{0.5cm}[\kern2pt{\textcolor{darkblue}{\textbf{\ipa{ʐwæ˥}}}}\kern2pt]} \hypertarget{z`w\{\string_T1}{}
\markboth{\textcolor{darkblue}{\textbf{\ipa{ʐwæ˥}}}}{}
\textcolor{teal}{\mytextsc{nom}} \hspace{4pt} Ton~: \#H.
\textcolor{PineGreen}{\selectlanguage{french}Cheval.} \zh{马。}  ¶ \textcolor{darkblue}{\textbf{\ipa{dʑɯ˩ʁo˩-ʐwæ˩}}} \textcolor{PineGreen}{\selectlanguage{french}cheval sauvage, non domestiqué} \zh{野马}  
 ¶ \textcolor{darkblue}{\textbf{\ipa{o-ho-ho! ʐwæ˧-ɳɯ˩ | dzɯ˧-po˧-hɯ˥-ze˩!}}} \textcolor{PineGreen}{\selectlanguage{french}Houlàlà! Le cheval nous l'a mangé! (Contexte: on laisse dans la cour des céréales, ou du fourrage, et pendant qu'on a le dos tourné, le cheval chaparde cette nourriture.)} \zh{啊呀嚒!马把饲料都吃光了!}  
 \zh{量词}: \textcolor{darkblue}{\textbf{\ipa{v̩˧}}}  \mytextsc{clf}~: \textcolor{darkblue}{\textbf{\ipa{v̩˧}}} 
\lhead{\firstmark}
\rhead{\botmark}

\subsection{\hspace{-0.5cm} {\Large \textcolor{darkblue}{\textbf{\ipa{ʐwæ˧\textsubscript{a}}}}}\hspace{0.5cm}[\kern2pt{\textcolor{darkblue}{\textbf{\ipa{ʐwæ˥}}}}\kern2pt]} \hypertarget{z`w\{\string_Ma1}{}
\markboth{\textcolor{darkblue}{\textbf{\ipa{ʐwæ˧\textsubscript{a}}}}}{}
\textcolor{teal}{\mytextsc{verbe}} \hspace{4pt} Ton~: M\textsubscript{a}.
\textcolor{PineGreen}{\selectlanguage{french}Peser (à l'aide d'une balance).} \zh{称。}  ¶ \textcolor{darkblue}{\textbf{\ipa{mɤ˧-ʐwæ˧}}} \textcolor{PineGreen}{\selectlanguage{french}\mytextsc{neg}} \zh{不称}  
 ¶ \textcolor{darkblue}{\textbf{\ipa{le˧-ʐwæ˧-ze˧}}} \textcolor{PineGreen}{\selectlanguage{french}\mytextsc{accomp} \string_ \mytextsc{pfv}} \zh{称了}  
 ¶ \textcolor{darkblue}{\textbf{\ipa{tso˧\textasciitilde{}tso˧ ʐwæ˩}}} \textcolor{PineGreen}{\selectlanguage{french}peser des choses} \zh{称东西}  
 ¶ \textcolor{darkblue}{\textbf{\ipa{ʁo˧do˧ ʐwæ˧}}} \textcolor{PineGreen}{\selectlanguage{french}peser des noix} \zh{称核桃}  

\lhead{\firstmark}
\rhead{\botmark}

\subsection{\hspace{-0.5cm} {\Large \textcolor{darkblue}{\textbf{\ipa{ʐwæ˧bv̩˧˥}}}}\hspace{0.5cm}[\kern2pt{\textcolor{darkblue}{\textbf{\ipa{ʐwæ˧bv̩˧˥}}}}\kern2pt]} \hypertarget{z`w\{\string_Mbv\string_=\string_M\string_T1}{}
\markboth{\textcolor{darkblue}{\textbf{\ipa{ʐwæ˧bv̩˧˥}}}}{}
\textcolor{teal}{\mytextsc{nom}} \hspace{4pt} Ton~: MH\#.
\textcolor{PineGreen}{\selectlanguage{french}Enclos des chevaux.} \zh{马圈。}  \zh{量词}: \textcolor{darkblue}{\textbf{\ipa{ɭɯ˧}}}  \mytextsc{clf}~: \textcolor{darkblue}{\textbf{\ipa{ɭɯ˧}}} 
\lhead{\firstmark}
\rhead{\botmark}

\subsection{\hspace{-0.5cm} {\Large \textcolor{darkblue}{\textbf{\ipa{ʐwæ˧-hɑ\#˥}}}}\hspace{0.5cm}[\kern2pt{\textcolor{darkblue}{\textbf{\ipa{xxxx non-correspondance entre le nombre de morphèmes et le nombre de tons de morphèmes}}}}\kern2pt]} \hypertarget{z`w\{\string_M-hA\#\string_T1}{}
\markboth{\textcolor{darkblue}{\textbf{\ipa{ʐwæ˧-hɑ\#˥}}}}{}
\textcolor{teal}{\mytextsc{nom}} \hspace{4pt} Ton~: \#H.
\textcolor{PineGreen}{\selectlanguage{french}Nourriture pour cheval.} \zh{马料、马饲料。} 
\lhead{\firstmark}
\rhead{\botmark}

\subsection{\hspace{-0.5cm} {\Large \textcolor{darkblue}{\textbf{\ipa{ʐwæ˧-kʰv̩˩}}}}\hspace{0.5cm}[\kern2pt{\textcolor{darkblue}{\textbf{\ipa{xxxx non-correspondance entre le nombre de morphèmes et le nombre de tons de morphèmes}}}}\kern2pt]} \hypertarget{z`w\{\string_M-k\string_hv\string_=\string_B1}{}
\markboth{\textcolor{darkblue}{\textbf{\ipa{ʐwæ˧-kʰv̩˩}}}}{}
\textcolor{teal}{\mytextsc{nom}} \hspace{4pt} Ton~: L\#.
\textcolor{PineGreen}{\selectlanguage{french}Année du cheval.} \zh{马年。} 
\lhead{\firstmark}
\rhead{\botmark}

\subsection{\hspace{-0.5cm} {\Large \textcolor{darkblue}{\textbf{\ipa{ʐwæ˧-ɭɯ\#˥}}}}\hspace{0.5cm}[\kern2pt{\textcolor{darkblue}{\textbf{\ipa{xxxx non-correspondance entre le nombre de morphèmes et le nombre de tons de morphèmes}}}}\kern2pt]} \hypertarget{z`w\{\string_M-l\string_RM\#\string_T1}{}
\markboth{\textcolor{darkblue}{\textbf{\ipa{ʐwæ˧-ɭɯ\#˥}}}}{}
\textcolor{teal}{\mytextsc{nom}} \hspace{4pt} Ton~: \#H.
\textcolor{PineGreen}{\selectlanguage{french}Nourriture pour cheval (céréales).} \zh{马料(粮食)。} 
\lhead{\firstmark}
\rhead{\botmark}

\subsection{\hspace{-0.5cm} {\Large \textcolor{darkblue}{\textbf{\ipa{ʐwæ˧pʰæ˧di˧˥}}}}\hspace{0.5cm}[\kern2pt{\textcolor{darkblue}{\textbf{\ipa{ʐwæ˧pʰæ˧di˧˥}}}}\kern2pt]} \hypertarget{z`w\{\string_Mp\string_h\{\string_Mdi\string_M\string_T1}{}
\markboth{\textcolor{darkblue}{\textbf{\ipa{ʐwæ˧pʰæ˧di˧˥}}}}{}
\textcolor{teal}{\mytextsc{nom}} \hspace{4pt} Ton~: MH\#.
\textcolor{PineGreen}{\selectlanguage{french}Longe, objet pour accrocher le cheval.} \zh{拉马链子。}  \zh{量词}: \textcolor{darkblue}{\textbf{\ipa{ɭɯ˧}}}  \mytextsc{clf}~: \textcolor{darkblue}{\textbf{\ipa{ɭɯ˧}}} 
\lhead{\firstmark}
\rhead{\botmark}

\subsection{\hspace{-0.5cm} {\Large \textcolor{darkblue}{\textbf{\ipa{ʐwæ˧-qʰæ\#˥}}}}\hspace{0.5cm}[\kern2pt{\textcolor{darkblue}{\textbf{\ipa{xxxx non-correspondance entre le nombre de morphèmes et le nombre de tons de morphèmes}}}}\kern2pt]} \hypertarget{z`w\{\string_M-q\string_h\{\#\string_T1}{}
\markboth{\textcolor{darkblue}{\textbf{\ipa{ʐwæ˧-qʰæ\#˥}}}}{}
\textcolor{teal}{\mytextsc{nom}} \hspace{4pt} Ton~: \#H.
\textcolor{PineGreen}{\selectlanguage{french}Crottin de cheval.} \zh{马粪。}  \zh{量词}: \textcolor{darkblue}{\textbf{\ipa{ʁwɤ˧}}}  \mytextsc{clf}~: \textcolor{darkblue}{\textbf{\ipa{ʁwɤ˧}}} 
\lhead{\firstmark}
\rhead{\botmark}

\subsection{\hspace{-0.5cm} {\Large \textcolor{darkblue}{\textbf{\ipa{ʐwæ˧ʁo˩}}}}\hspace{0.5cm}[\kern2pt{\textcolor{darkblue}{\textbf{\ipa{ʐwæ˧ʁo˩}}}}\kern2pt]} \hypertarget{z`w\{\string_MRo\string_B1}{}
\markboth{\textcolor{darkblue}{\textbf{\ipa{ʐwæ˧ʁo˩}}}}{}
\textcolor{teal}{\mytextsc{nom}} \hspace{4pt} Ton~: L\#.
\textcolor{PineGreen}{\selectlanguage{french}Cheval castré.} \zh{骟马。}  \zh{量词}: \textcolor{darkblue}{\textbf{\ipa{v̩˧}}}  \mytextsc{clf}~: \textcolor{darkblue}{\textbf{\ipa{v̩˧}}} 
\lhead{\firstmark}
\rhead{\botmark}

\subsection{\hspace{-0.5cm} {\Large \textcolor{darkblue}{\textbf{\ipa{ʐwæ˧sɯ˩}}}}\hspace{0.5cm}[\kern2pt{\textcolor{darkblue}{\textbf{\ipa{ʐwæ˧sɯ˩}}}}\kern2pt]} \hypertarget{z`w\{\string_MsM\string_B1}{}
\markboth{\textcolor{darkblue}{\textbf{\ipa{ʐwæ˧sɯ˩}}}}{}
\textcolor{teal}{\mytextsc{nom}} \hspace{4pt} Ton~: L\#.
\textcolor{PineGreen}{\selectlanguage{french}Étalon.} \zh{公马。}  ¶ \textcolor{darkblue}{\textbf{\ipa{ʐwæ˧sɯ˩-ʐwæ˩mi˩}}} \textcolor{PineGreen}{\selectlanguage{french}étalon et jument} \zh{公马与母马}  
 ¶ \textcolor{darkblue}{\textbf{\ipa{ʐwæ˧sɯ˩-ʐwæ˩zo˩}}} \textcolor{PineGreen}{\selectlanguage{french}étalon et poulain} \zh{公马与小马}  
 \zh{量词}: \textcolor{darkblue}{\textbf{\ipa{mi˩}}}  \mytextsc{clf}~: \textcolor{darkblue}{\textbf{\ipa{mi˩}}} 
\lhead{\firstmark}
\rhead{\botmark}

\subsection{\hspace{-0.5cm} {\Large \textcolor{darkblue}{\textbf{\ipa{ʐwæ˧zo\#˥}}}}\hspace{0.5cm}[\kern2pt{\textcolor{darkblue}{\textbf{\ipa{ʐwæ˧zo˧}}}}\kern2pt]} \hypertarget{z`w\{\string_Mzo\#\string_T1}{}
\markboth{\textcolor{darkblue}{\textbf{\ipa{ʐwæ˧zo\#˥}}}}{}
\textcolor{teal}{\mytextsc{nom}} \hspace{4pt} Ton~: \#H.
\textcolor{PineGreen}{\selectlanguage{french}Poulain.} \zh{马驹子。}  ¶ \textcolor{darkblue}{\textbf{\ipa{ʐwæ˧zo˧-ʐwæ˥mi˩}}} \textcolor{PineGreen}{\selectlanguage{french}poulain et jument} \zh{马驹子与母马}  
 \zh{量词}: \textcolor{darkblue}{\textbf{\ipa{ɭɯ˧}}}  \mytextsc{clf}~: \textcolor{darkblue}{\textbf{\ipa{ɭɯ˧}}} 
\lhead{\firstmark}
\rhead{\botmark}

\subsection{\hspace{-0.5cm} {\Large \textcolor{darkblue}{\textbf{\ipa{ʐwæ˧-zɯ\#˥}}}}\hspace{0.5cm}[\kern2pt{\textcolor{darkblue}{\textbf{\ipa{xxxx non-correspondance entre le nombre de morphèmes et le nombre de tons de morphèmes}}}}\kern2pt]} \hypertarget{z`w\{\string_M-zM\#\string_T1}{}
\markboth{\textcolor{darkblue}{\textbf{\ipa{ʐwæ˧-zɯ\#˥}}}}{}
\textcolor{teal}{\mytextsc{nom}} \hspace{4pt} Ton~: \#H.
\textcolor{PineGreen}{\selectlanguage{french}Fourrage pour le cheval, foin pour cheval.} \zh{喂马的草。} 
\lhead{\firstmark}
\rhead{\botmark}

\subsection{\hspace{-0.5cm} {\Large \textcolor{darkblue}{\textbf{\ipa{ʐwæ˧\textasciitilde{}ʐwæ˧}}}}\hspace{0.5cm}[\kern2pt{\textcolor{darkblue}{\textbf{\ipa{ʐwæ˧ʐwæ˧}}}}\kern2pt]} \hypertarget{z`w\{\string_M~z`w\{\string_M1}{}
\markboth{\textcolor{darkblue}{\textbf{\ipa{ʐwæ˧\textasciitilde{}ʐwæ˧}}}}{}
\textcolor{teal}{\mytextsc{verbe}} \hspace{4pt} Ton~: M.
\ding{202} \textcolor{PineGreen}{\selectlanguage{french}Ranger (des objets).} \zh{收拾。}  ¶ \textcolor{darkblue}{\textbf{\ipa{tso˧\textasciitilde{}tso˧ ʐwæ˧\textasciitilde{}ʐwæ˧(-ze˩)}}} \textcolor{PineGreen}{\selectlanguage{french}ranger des choses} \zh{收拾东西}  
 ¶ \textcolor{darkblue}{\textbf{\ipa{le˧-ʐwæ˧\textasciitilde{}ʐwæ˧ ɖɯ˧-ʝi˧-tɕɯ˥}}} \textcolor{PineGreen}{\selectlanguage{french}ranger des choses et les mettre à leur place, ranger des choses ensemble} \zh{把东西收拾在一起}  
\ding{203} \textcolor{PineGreen}{\selectlanguage{french}Rassembler (des gens).} \zh{聚集。} 
\lhead{\firstmark}
\rhead{\botmark}

\subsection{\hspace{-0.5cm} {\Large \textcolor{darkblue}{\textbf{\ipa{ʐwæ˩}}}}\hspace{0.5cm}[\kern2pt{\textcolor{darkblue}{\textbf{\ipa{ʐwæ˩˥}}}}\kern2pt]} \hypertarget{z`w\{\string_B1}{}
\markboth{\textcolor{darkblue}{\textbf{\ipa{ʐwæ˩}}}}{}
\textcolor{teal}{\mytextsc{adverbe}} \hspace{4pt} Ton~: L.
\textcolor{PineGreen}{\selectlanguage{french}Extrêmement.} \zh{很、极。}  ¶ \textcolor{darkblue}{\textbf{\ipa{ʐwæ˩-ze˥!}}} \textcolor{PineGreen}{\selectlanguage{french}Il y en a trop! / ça fait trop!} \zh{太多了!}  

\lhead{\firstmark}
\rhead{\botmark}

\subsection{\hspace{-0.5cm} {\Large \textcolor{darkblue}{\textbf{\ipa{ʐwæ˩\textsubscript{a}}}} \textsubscript{1}}\hspace{0.5cm}[\kern2pt{\textcolor{darkblue}{\textbf{\ipa{ʐwæ˩˥}}}}\kern2pt]} \hypertarget{z`w\{\string_Ba1}{}
\markboth{\textcolor{darkblue}{\textbf{\ipa{ʐwæ˩\textsubscript{a}}}} \textsubscript{1}}{}
\textcolor{teal}{\mytextsc{verbe}} \hspace{4pt} Ton~: L\textsubscript{a}.
\textcolor{PineGreen}{\selectlanguage{french}S'évanouir.} \zh{昏,昏厥。}  ¶ \textcolor{darkblue}{\textbf{\ipa{le˧-ʈʰi˩ | le˧-ʐwæ˩-ze˩}}} \textcolor{PineGreen}{\selectlanguage{french}s'évanouir à force de fatigue, s'évanouir d'épuisement} \zh{累得都昏倒了}  

\lhead{\firstmark}
\rhead{\botmark}

\subsection{\hspace{-0.5cm} {\Large \textcolor{darkblue}{\textbf{\ipa{ʐwæ˩\textsubscript{a}}}} \textsubscript{2}}\hspace{0.5cm}[\kern2pt{\textcolor{darkblue}{\textbf{\ipa{ʐwæ˩˥}}}}\kern2pt]} \hypertarget{z`w\{\string_Ba2}{}
\markboth{\textcolor{darkblue}{\textbf{\ipa{ʐwæ˩\textsubscript{a}}}} \textsubscript{2}}{}
\textcolor{teal}{\mytextsc{adjectif}} \hspace{4pt} Ton~: L\textsubscript{a}.
\textcolor{PineGreen}{\selectlanguage{french}Habile, bon, capable.} \zh{好,能干。}  ¶ \textcolor{darkblue}{\textbf{\ipa{ʐwæ˩-hĩ˩˥}}} \textcolor{PineGreen}{\selectlanguage{french}\mytextsc{nmlz}} \zh{能干的}  
 ¶ \textcolor{darkblue}{\textbf{\ipa{ʈʂʰɯ˧ ɖwæ˧˥ | ʐwæ˩˥!}}} \textcolor{PineGreen}{\selectlanguage{french}Il est très habile! / Il est formidable!} \zh{他很能干!}  

\lhead{\firstmark}
\rhead{\botmark}

\subsection{\hspace{-0.5cm} {\Large \textcolor{darkblue}{\textbf{\ipa{ʐwæ˩mi˩}}}}\hspace{0.5cm}[\kern2pt{\textcolor{darkblue}{\textbf{\ipa{ʐwæ˩mi˩˥}}}}\kern2pt]} \hypertarget{z`w\{\string_Bmi\string_B1}{}
\markboth{\textcolor{darkblue}{\textbf{\ipa{ʐwæ˩mi˩}}}}{}
\textcolor{teal}{\mytextsc{nom}} \hspace{4pt} Ton~: L.
\textcolor{PineGreen}{\selectlanguage{french}Jument; également employé pour une jeune jument: pouliche, “bébé cheval” de sexe féminin.} \zh{母马。}  ¶ \textcolor{darkblue}{\textbf{\ipa{ʂe˩-ʐwæ˩mi˥}}} \textcolor{PineGreen}{\selectlanguage{french}vélo; néologisme introduit par F4 d'après le taïwanais tiěmǎ \zh{铁马} que j'essayais de traduire en na.} \zh{自行车(“铁马”)}  
 ¶ \textcolor{darkblue}{\textbf{\ipa{ʐwæ˩mi˩-ʐwæ˩zo˩}}} \textcolor{PineGreen}{\selectlanguage{french}jument et poulain} \zh{母马与马驹子}  
 \zh{量词}: \textcolor{darkblue}{\textbf{\ipa{v̩˧}}} \textcolor{darkblue}{\textbf{\ipa{jɤ˧˥}}}  \mytextsc{clf}~: \textcolor{darkblue}{\textbf{\ipa{v̩˧}}} \textcolor{darkblue}{\textbf{\ipa{jɤ˧˥}}} 
\lhead{\firstmark}
\rhead{\botmark}

\subsection{\hspace{-0.5cm} {\Large \textcolor{darkblue}{\textbf{\ipa{ʐwæ˧˥}}}}\hspace{0.5cm}[\kern2pt{\textcolor{darkblue}{\textbf{\ipa{ʐwæ˧˥}}}}\kern2pt]} \hypertarget{z`w\{\string_M\string_T1}{}
\markboth{\textcolor{darkblue}{\textbf{\ipa{ʐwæ˧˥}}}}{}
\textcolor{teal}{\mytextsc{verbe}} \hspace{4pt} Ton~: MH.
\textcolor{PineGreen}{\selectlanguage{french}Sarcler, biner.} \zh{薅锄、锄草。}  ¶ \textcolor{darkblue}{\textbf{\ipa{ʐwæ˩\textasciitilde{}ʐwæ˧˥}}} \textcolor{PineGreen}{\selectlanguage{french}\mytextsc{red}} \zh{\mytextsc{重叠}}  
 ¶ \textcolor{darkblue}{\textbf{\ipa{jɤ˩jo˥ ʐwæ˩}}} \textcolor{PineGreen}{\selectlanguage{french}sarcler des pommes de terre} \zh{洋芋地里锄草}  
 ¶ \textcolor{darkblue}{\textbf{\ipa{jɤ˩jo˧ ʐwæ˧\textasciitilde{}ʐwæ˥}}} \textcolor{PineGreen}{\selectlanguage{french}sarcler des pommes de terre} \zh{洋芋地里锄草}  
 ¶ \textcolor{darkblue}{\textbf{\ipa{qʰɑ˧dze˧ ʐwæ˧˥}}} \textcolor{PineGreen}{\selectlanguage{french}sarcler du maïs} \zh{苞谷地里锄草}  
 ¶ \textcolor{darkblue}{\textbf{\ipa{qʰɑ˧dze˧ ʐwæ˧\textasciitilde{}ʐwæ˥}}} \textcolor{PineGreen}{\selectlanguage{french}sarcler du maïs} \zh{苞谷地里锄草}  

\lhead{\firstmark}
\rhead{\botmark}

\subsection{\hspace{-0.5cm} {\Large \textcolor{darkblue}{\textbf{\ipa{ʐwɤ˧}}}}\hspace{0.5cm}[\kern2pt{\textcolor{darkblue}{\textbf{\ipa{ʐwɤ˥}}}}\kern2pt]} \hypertarget{z`w7\string_M1}{}
\markboth{\textcolor{darkblue}{\textbf{\ipa{ʐwɤ˧}}}}{}
\textcolor{teal}{\mytextsc{adjectif}} \hspace{4pt} Ton~: M.
\textcolor{PineGreen}{\selectlanguage{french}Qui a faim (forme monosyllabique). Se combine en disyllabe avec le mot “nourriture”.} \zh{饿。}  ¶ \textcolor{darkblue}{\textbf{\ipa{hɑ˧-ʐwɤ˩}}} \textcolor{PineGreen}{\selectlanguage{french}avoir faim} \zh{饿}  

\lhead{\firstmark}
\rhead{\botmark}

\subsection{\hspace{-0.5cm} {\Large \textcolor{darkblue}{\textbf{\ipa{ʐwɤ˧mv̩˧}}}}\hspace{0.5cm}[\kern2pt{\textcolor{darkblue}{\textbf{\ipa{ʐwɤ˩mv̩˩˥}}}}\kern2pt]} \hypertarget{z`w7\string_Mmv\string_=\string_M1}{}
\markboth{\textcolor{darkblue}{\textbf{\ipa{ʐwɤ˧mv̩˧}}}}{}
\textcolor{teal}{\mytextsc{nom}} \hspace{4pt} Ton~: M.
\textit{De:} \textbf{ʐwɤ˩b} \textcolor{PineGreen}{\selectlanguage{french}Formule toute faite, expression toute faite, expression idiomatique.} \zh{惯用语、习惯语、习语。}  ¶ \textcolor{darkblue}{\textbf{\ipa{ʐwɤ˧mv̩˧ dʑo˧-kv̩˧˥ !}}} \textcolor{PineGreen}{\selectlanguage{french}C'est comme ça qu'on dit! / Il y a une expression comme ça!} \zh{有这么一句老话! / 有这么一个说法!}  
 ¶ \textcolor{darkblue}{\textbf{\ipa{æ˧ʂæ˧-ʐwɤ˧mv̩˧ | ɖɯ˧-kʰwɤ˥}}} \textcolor{PineGreen}{\selectlanguage{french}une expression toute faite du temps jadis, un dicton} \zh{一句老话、一个传统的说法}  
 \zh{量词}: \textcolor{darkblue}{\textbf{\ipa{kʰwɤ˥}}}  \mytextsc{clf}~: \textcolor{darkblue}{\textbf{\ipa{kʰwɤ˥}}} 
\lhead{\firstmark}
\rhead{\botmark}

\subsection{\hspace{-0.5cm} {\Large \textcolor{darkblue}{\textbf{\ipa{ʐwɤ˩\textsubscript{b}}}}}\hspace{0.5cm}[\kern2pt{\textcolor{darkblue}{\textbf{\ipa{ʐwɤ˥}}}}\kern2pt]} \hypertarget{z`w7\string_Bb1}{}
\markboth{\textcolor{darkblue}{\textbf{\ipa{ʐwɤ˩\textsubscript{b}}}}}{}
\textcolor{teal}{\mytextsc{verbe}} \hspace{4pt} Ton~: L\textsubscript{b}.
\textcolor{PineGreen}{\selectlanguage{french}Parler.} \zh{讲话。}  ¶ \textcolor{darkblue}{\textbf{\ipa{ʐwɤ˧\textasciitilde{}ʐwɤ˩ mɤ˩-hĩ˩}}} \textcolor{PineGreen}{\selectlanguage{french}muet, personne muette, personne qui ne parle pas} \zh{哑巴、不会讲话的人}  
 ¶ \textcolor{darkblue}{\textbf{\ipa{ʐwɤ˧\textasciitilde{}ʐwɤ˩ mɤ˩-hĩ˩, | ʈʂʰɯ˧-v̩˧!}}} \textcolor{PineGreen}{\selectlanguage{french}Elle/il ne sait pas parler, elle/lui!} \zh{不会讲话,这个人! / 这个人,不会讲话!}  
 ¶ \textcolor{darkblue}{\textbf{\ipa{le˧-ʐwɤ˩-ze˩}}} \textcolor{PineGreen}{\selectlanguage{french}\mytextsc{accomp} \string_ \mytextsc{pfv}} \zh{讲了}  
 ¶ \textcolor{darkblue}{\textbf{\ipa{no˧ | ə˧tso˧ ʐwɤ˩-ɲi˩?}}} \textcolor{PineGreen}{\selectlanguage{french}Que dis-tu? / Qu'est-ce que tu veux dire?} \zh{你说什么?}  
 ¶ \textcolor{darkblue}{\textbf{\ipa{ʐwɤ˧\textasciitilde{}ʐwɤ˩}}} \textcolor{PineGreen}{\selectlanguage{french}\mytextsc{red}} \textcolor{PineGreen}{\selectlanguage{french}PHONO} \zh{\mytextsc{重叠}}  
 ¶ \textcolor{darkblue}{\textbf{\ipa{le˧-ʐwɤ˩}}} \textcolor{PineGreen}{\selectlanguage{french}répondre, donner une réponse} \zh{回答}  
 ¶ \textcolor{darkblue}{\textbf{\ipa{le˧-wo˧ ʐwɤ˧˥}}} \textcolor{PineGreen}{\selectlanguage{french}répondre, donner une réponse} \zh{回答}  
 ¶ \textcolor{darkblue}{\textbf{\ipa{ʈʂʰɯ˧ | le˧-ʐwɤ˩-bi˩-dʑo˩...}}} \textcolor{PineGreen}{\selectlanguage{french}A ce qu'elle/il dit...} \zh{依照他的说法……}  
 ¶ \textcolor{darkblue}{\textbf{\ipa{hĩ˧-qɑ˧ ʐwɤ˧\textasciitilde{}ʐwɤ˥}}} \textcolor{PineGreen}{\selectlanguage{french}parler aux gens} \zh{对人家讲}  
 ¶ \textcolor{darkblue}{\textbf{\ipa{le˧-ʐwɤ˧\textasciitilde{}ʐwɤ˥-ze˩}}} \textcolor{PineGreen}{\selectlanguage{french}\mytextsc{accomp} \mytextsc{red} \mytextsc{pfv}} \zh{\mytextsc{accomp} \mytextsc{red} \mytextsc{pfv}}  

\lhead{\firstmark}
\rhead{\botmark}

\newpage
\section*{\centering- \textcolor{darkblue}{\textbf{\ipa{ʑ}}} -}
\subsection{\hspace{-0.5cm} {\Large \textcolor{darkblue}{\textbf{\ipa{ʑi˥}}}}\hspace{0.5cm}[\kern2pt{\textcolor{darkblue}{\textbf{\ipa{ʑi˥}}}}\kern2pt]} \hypertarget{z£i\string_T1}{}
\markboth{\textcolor{darkblue}{\textbf{\ipa{ʑi˥}}}}{}
\textcolor{teal}{\mytextsc{verbe}} \hspace{4pt} Ton~: H.
\textcolor{PineGreen}{\selectlanguage{french}Être présent, y avoir; propriété du corps, de l'âme, d'un objet… Ex.: avoir de la force; avoir de la barbe; il y a une resserre dans la maison.} \zh{有,拥有(抽象:有力量,有勇气)。}  ¶ \textcolor{darkblue}{\textbf{\ipa{mɤ˧-ʑi˥}}} \textcolor{PineGreen}{\selectlanguage{french}\mytextsc{neg}} \zh{没有}  

\lhead{\firstmark}
\rhead{\botmark}

\subsection{\hspace{-0.5cm} {\Large \textcolor{darkblue}{\textbf{\ipa{ʑi˧\textsubscript{a}}}}}\hspace{0.5cm}[\kern2pt{\textcolor{darkblue}{\textbf{\ipa{ʑi˥}}}}\kern2pt]} \hypertarget{z£i\string_Ma1}{}
\markboth{\textcolor{darkblue}{\textbf{\ipa{ʑi˧\textsubscript{a}}}}}{}
\textcolor{teal}{\mytextsc{verbe}} \hspace{4pt} Ton~: M\textsubscript{a}.
\ding{202} \textcolor{PineGreen}{\selectlanguage{french}Couler, avoir une fuite; s'écouler (fleuve).} \zh{漏(水)。}  ¶ \textcolor{darkblue}{\textbf{\ipa{mv̩˩tɕo˧ ʑi˧}}} \textcolor{PineGreen}{\selectlanguage{french}fuir, avoir une fuite} \zh{(水)往下漏}  
\ding{203} \textcolor{PineGreen}{\selectlanguage{french}S'écouler, couler (rivière).} \zh{流(河水流着)。}  ¶ \textcolor{darkblue}{\textbf{\ipa{mv̩˩tɕo˧ ʑi˧}}} \textcolor{PineGreen}{\selectlanguage{french}couler (rivière)} \zh{(河)往下游流}  

\lhead{\firstmark}
\rhead{\botmark}

\subsection{\hspace{-0.5cm} {\Large \textcolor{darkblue}{\textbf{\ipa{ʑi˧dv̩˧}}}}\hspace{0.5cm}[\kern2pt{\textcolor{darkblue}{\textbf{\ipa{ʑi˩dv̩˩˥}}}}\kern2pt]} \hypertarget{z£i\string_Mdv\string_=\string_M1}{}
\markboth{\textcolor{darkblue}{\textbf{\ipa{ʑi˧dv̩˧}}}}{}
\textcolor{teal}{\mytextsc{nom}} \hspace{4pt} Ton~: M.
\ding{202} \textcolor{PineGreen}{\selectlanguage{french}Maisonnée.} \zh{家。}  ¶ \textcolor{darkblue}{\textbf{\ipa{ɖɯ˧-ʑi˩dv̩˩}}} \textcolor{PineGreen}{\selectlanguage{french}une maisonnée/ toute la maison} \zh{一家人,包括所有成员}  
 ¶ \textcolor{darkblue}{\textbf{\ipa{ɑ˩ʁo˧-ʑi˧dv̩˧ ʝi˧}}} \textcolor{PineGreen}{\selectlanguage{french}s'occuper de la maison, veiller au bon fonctionnement de la maison} \zh{管家}  
 \zh{量词}: \textcolor{darkblue}{\textbf{\ipa{ɭɯ˧}}} \ding{203} \textcolor{PineGreen}{\selectlanguage{french}L'ensemble de la ferme, comprenant plusieurs bâtiments, le bétail et les gens.} \zh{农舍,包括院子、人住的楼、动物住的楼等。}  \zh{量词}: \textcolor{darkblue}{\textbf{\ipa{ɭɯ˧}}}  \mytextsc{clf}~: \textcolor{darkblue}{\textbf{\ipa{ɭɯ˧}}} \textcolor{darkblue}{\textbf{\ipa{ɭɯ˧}}} 
\lhead{\firstmark}
\rhead{\botmark}

\subsection{\hspace{-0.5cm} {\Large \textcolor{darkblue}{\textbf{\ipa{ʑi˧dv̩˧ʝi˧-hĩ\#˥}}}}\hspace{0.5cm}[\kern2pt{\textcolor{darkblue}{\textbf{\ipa{xxxx non-correspondance entre le nombre de morphèmes et le nombre de tons de morphèmes}}}}\kern2pt]} \hypertarget{z£i\string_Mdv\string_=\string_Mj££i\string_M-hi\string_~\#\string_T1}{}
\markboth{\textcolor{darkblue}{\textbf{\ipa{ʑi˧dv̩˧ʝi˧-hĩ\#˥}}}}{}
\textcolor{teal}{\mytextsc{nom}} \hspace{4pt} Ton~: H\#.
\textcolor{PineGreen}{\selectlanguage{french}La personne qui s'occupe de la maison, le maître/la maîtresse de céans.} \zh{一家之主、家长。} 
\lhead{\firstmark}
\rhead{\botmark}

\subsection{\hspace{-0.5cm} {\Large \textcolor{darkblue}{\textbf{\ipa{ʑi˧kv̩˧wo˧}}}}\hspace{0.5cm}[\kern2pt{\textcolor{darkblue}{\textbf{\ipa{ʑi˩kv̩˧wo˧˥}}}}\kern2pt]} \hypertarget{z£i\string_Mkv\string_=\string_Mwo\string_M1}{}
\markboth{\textcolor{darkblue}{\textbf{\ipa{ʑi˧kv̩˧wo˧}}}}{}
\textcolor{teal}{\mytextsc{nom}} \hspace{4pt} Ton~: M.
\textcolor{PineGreen}{\selectlanguage{french}Toit.} \zh{房顶。}  \zh{量词}: \textcolor{darkblue}{\textbf{\ipa{tsʰi˩}}}  \mytextsc{clf}~: \textcolor{darkblue}{\textbf{\ipa{tsʰi˩}}} 
\lhead{\firstmark}
\rhead{\botmark}

\subsection{\hspace{-0.5cm} {\Large \textcolor{darkblue}{\textbf{\ipa{ʑi˧mi˧}}}}\hspace{0.5cm}[\kern2pt{\textcolor{darkblue}{\textbf{\ipa{ʑi˧mi˧}}}}\kern2pt]} \hypertarget{z£i\string_Mmi\string_M1}{}
\markboth{\textcolor{darkblue}{\textbf{\ipa{ʑi˧mi˧}}}}{}
\textcolor{teal}{\mytextsc{nom}} \hspace{4pt} Ton~: M.
\ding{202} \textcolor{PineGreen}{\selectlanguage{french}Le bâtiment de la maison où se trouve le foyer.} \zh{家里有火塘的那个房子(“祖母房”)。}  \zh{量词}: \textcolor{darkblue}{\textbf{\ipa{ɭɯ˧}}} \ding{203} \textcolor{PineGreen}{\selectlanguage{french}L'ensemble de la maison; l'ensemble de la ferme.} \zh{整个家园。}  \mytextsc{clf}~: \textcolor{darkblue}{\textbf{\ipa{ɭɯ˧}}} 
\lhead{\firstmark}
\rhead{\botmark}

\subsection{\hspace{-0.5cm} {\Large \textcolor{darkblue}{\textbf{\ipa{ʑi˧mv̩˧˥}}}}\hspace{0.5cm}[\kern2pt{\textcolor{darkblue}{\textbf{\ipa{ʑi˩mv̩˥}}}}\kern2pt]} \hypertarget{z£i\string_Mmv\string_=\string_M\string_T1}{}
\markboth{\textcolor{darkblue}{\textbf{\ipa{ʑi˧mv̩˧˥}}}}{}
\textcolor{teal}{\mytextsc{nom}} \hspace{4pt} Ton~: MH\#.
\textcolor{PineGreen}{\selectlanguage{french}Rêve.} \zh{梦。}  ¶ \textcolor{darkblue}{\textbf{\ipa{ʑi˧mv̩˧ qʰwɤ˧˥}}} \textcolor{PineGreen}{\selectlanguage{french}faire un rêve} \zh{做梦}  
 ¶ \textcolor{darkblue}{\textbf{\ipa{ʑi˧mv̩˧ sɯ˧}}} \textcolor{PineGreen}{\selectlanguage{french}être somnambule; parler dans son sommeil} \zh{梦游,梦呓}  
 ¶ \textcolor{darkblue}{\textbf{\ipa{njɤ˧ | ə˧hwɤ˧ | ʑi˧mv̩˥ | mɤ˧-dʑɤ˩!}}} \textcolor{PineGreen}{\selectlanguage{french}J'ai fait un cauchemar hier!} \zh{我昨天做了恶梦!}  
 \zh{量词}: \textcolor{darkblue}{\textbf{\ipa{kʰwɤ˥}}}  \mytextsc{clf}~: \textcolor{darkblue}{\textbf{\ipa{kʰwɤ˥}}} 
\lhead{\firstmark}
\rhead{\botmark}

\subsection{\hspace{-0.5cm} {\Large \textcolor{darkblue}{\textbf{\ipa{ʑi˧ŋɤ˥}}}}\hspace{0.5cm}[\kern2pt{\textcolor{darkblue}{\textbf{\ipa{ʑi˧ŋɤ˧˥}}}}\kern2pt]} \hypertarget{z£i\string_MN7\string_T1}{}
\markboth{\textcolor{darkblue}{\textbf{\ipa{ʑi˧ŋɤ˥}}}}{}
\textcolor{teal}{\mytextsc{verbe}} \hspace{4pt} Ton~: H\#.
\textcolor{PineGreen}{\selectlanguage{french}Somnoler.} \zh{打瞌睡。}  ¶ \textcolor{darkblue}{\textbf{\ipa{ʑi˧ŋɤ˥-ze˩}}} \textcolor{PineGreen}{\selectlanguage{french}\mytextsc{pfv}} \zh{\mytextsc{pfv}}  

\lhead{\firstmark}
\rhead{\botmark}

\subsection{\hspace{-0.5cm} {\Large \textcolor{darkblue}{\textbf{\ipa{ʑi˧ŋv̩˥}}}}\hspace{0.5cm}[\kern2pt{\textcolor{darkblue}{\textbf{\ipa{ʑi˧ŋv̩˥}}}}\kern2pt]} \hypertarget{z£i\string_MNv\string_=\string_T1}{}
\markboth{\textcolor{darkblue}{\textbf{\ipa{ʑi˧ŋv̩˥}}}}{}
\textcolor{teal}{\mytextsc{verbe}} \hspace{4pt} Ton~: H\#.
\textcolor{PineGreen}{\selectlanguage{french}Dormir.} \zh{睡觉。}  ¶ \textcolor{darkblue}{\textbf{\ipa{le˧-ʑi˧ŋv̩˥}}} \textcolor{PineGreen}{\selectlanguage{french}\mytextsc{accomp}} \zh{\mytextsc{accomp}}  
 ¶ \textcolor{darkblue}{\textbf{\ipa{ʑi˧ŋv̩˥-ho˩}}} \textcolor{PineGreen}{\selectlanguage{french}qui va s'endormir, qui est ensommeillé} \zh{要睡了}  

\lhead{\firstmark}
\rhead{\botmark}

\subsection{\hspace{-0.5cm} {\Large \textcolor{darkblue}{\textbf{\ipa{ʑi˧qʰwɤ˧}}}}\hspace{0.5cm}[\kern2pt{\textcolor{darkblue}{\textbf{\ipa{ʑi˩qʰwɤ˥}}}}\kern2pt]} \hypertarget{z£i\string_Mq\string_hw7\string_M1}{}
\markboth{\textcolor{darkblue}{\textbf{\ipa{ʑi˧qʰwɤ˧}}}}{}
\textcolor{teal}{\mytextsc{nom}} \hspace{4pt} Ton~: M.
\textcolor{PineGreen}{\selectlanguage{french}Bâtiment, bâtiment d'habitation, pièce d'habitation; en naxi, a aussi le sens de “maisonnée”.} \zh{房屋。}  ¶ \textcolor{darkblue}{\textbf{\ipa{ʑi˧qʰwɤ˧ gv̩˩}}} \textcolor{PineGreen}{\selectlanguage{french}bâtir une maison} \zh{建房}  
 ¶ \textcolor{darkblue}{\textbf{\ipa{ʑi˧qʰwɤ˧-lɑ˧ do˥!}}} \textcolor{PineGreen}{\selectlanguage{french}on ne voit que des maisons/des bâtiments! (commentaires au sujet de la ville de Lijiang, où on ne voit pas les champs, à la différence de la plaine de Yongning: campagne où il y avait peu de maisons et de grands espaces cultivés.} \zh{只看到房子! / 能看见的只有房子!(合作者说,丽江市区都是房子,看不到田。这一点,不像永宁坝:二十世纪的永宁,只有一些小村落分散在一大片田地中。)}  
 ¶ \textcolor{darkblue}{\textbf{\ipa{ɕjo˩ɕjɤ˩-ʑi˩qʰwɤ˥}}} \textcolor{PineGreen}{\selectlanguage{french}les bâtiments de l'école (du chinois \zh{学校} “école”)} \zh{学校的楼(‘学校’:汉语借词)}  
 ¶ \textcolor{darkblue}{\textbf{\ipa{ʑi˧qʰwɤ˧ ʈʂʰv̩˩}}} \textcolor{PineGreen}{\selectlanguage{french}peindre une maison; littéralement: “teindre une maison”} \zh{给房子刷颜色(直译:‘染房’)}  
 ¶ \textcolor{darkblue}{\textbf{\ipa{ʑi˧qʰwɤ˧ tɕʰi˧-hĩ˧ kʰv̩˥mi˩}}} \textcolor{PineGreen}{\selectlanguage{french}chien de garde, chien qui garde la maison} \zh{看门狗}  
 ¶ \textcolor{darkblue}{\textbf{\ipa{ʑi˧qʰwɤ˧ tɕʰi˧-hĩ˧ kʰv̩˥}}} \textcolor{PineGreen}{\selectlanguage{french}chien de garde, chien qui garde la maison} \zh{看门狗}  
 \zh{量词}: \textcolor{darkblue}{\textbf{\ipa{ɭɯ˧}}}  \mytextsc{clf}~: \textcolor{darkblue}{\textbf{\ipa{ɭɯ˧}}} 
\lhead{\firstmark}
\rhead{\botmark}

\subsection{\hspace{-0.5cm} {\Large \textcolor{darkblue}{\textbf{\ipa{ʑi˧ʁæ˥\$}}}}\hspace{0.5cm}[\kern2pt{\textcolor{darkblue}{\textbf{\ipa{ʑi˧ʁæ˧}}}}\kern2pt]} \hypertarget{z£i\string_MR\{\string_T\$1}{}
\markboth{\textcolor{darkblue}{\textbf{\ipa{ʑi˧ʁæ˥\$}}}}{}
\textcolor{teal}{\mytextsc{nom}} \hspace{4pt} Ton~: H\$.
\textcolor{PineGreen}{\selectlanguage{french}L'espace situé derrière la maison: entre le bâtiment et les murs de la ferme.} \zh{房屋的上后方。}  ¶ \textcolor{darkblue}{\textbf{\ipa{ɲi˧ʈʂæ˧-ʑi˧-ʁo˧tʰo˥, | ʑi˧ʁæ˧ ɲi˥ mæ˩!}}} \textcolor{PineGreen}{\selectlanguage{french}Derrière le bâtiment à deux étages, c'est 'ʑi˧ʁæ˥\$'! / Derrière le bâtiment à deux étages, il y a ce qu'on appelle 'l'espace derrière la maison'!} \zh{两层楼房后面(这块地方)叫做“房屋的上后方”! / 房屋背后(这块地方)叫做“房屋的上后方”!}  
 \zh{量词}: \textcolor{darkblue}{\textbf{\ipa{kʰwɤ˥}}}  \mytextsc{clf}~: \textcolor{darkblue}{\textbf{\ipa{kʰwɤ˥}}} 
\lhead{\firstmark}
\rhead{\botmark}

\subsection{\hspace{-0.5cm} {\Large \textcolor{darkblue}{\textbf{\ipa{ʑi˧ʁo˥\$}}}}\hspace{0.5cm}[\kern2pt{\textcolor{darkblue}{\textbf{\ipa{ʑi˧ʁo˥}}}}\kern2pt]} \hypertarget{z£i\string_MRo\string_T\$1}{}
\markboth{\textcolor{darkblue}{\textbf{\ipa{ʑi˧ʁo˥\$}}}}{}
\textcolor{teal}{\mytextsc{nom}} \hspace{4pt} Ton~: H\$.
\textcolor{PineGreen}{\selectlanguage{french}Lit (le couchage entier).} \zh{床。}  \zh{量词}: \textcolor{darkblue}{\textbf{\ipa{ɭɯ˧˥}}}  \mytextsc{clf}~: \textcolor{darkblue}{\textbf{\ipa{ɭɯ˧˥}}} 
\lhead{\firstmark}
\rhead{\botmark}

\subsection{\hspace{-0.5cm} {\Large \textcolor{darkblue}{\textbf{\ipa{ʑi˩}}} \textsubscript{2}}\hspace{0.5cm}[\kern2pt{\textcolor{darkblue}{\textbf{\ipa{xxxx ton non trouvé, à faire manuellement...}}}}\kern2pt]} \hypertarget{z£i\string_B2}{}
\markboth{\textcolor{darkblue}{\textbf{\ipa{ʑi˩}}} \textsubscript{2}}{}
\textcolor{teal}{\mytextsc{classificateur}} \hspace{4pt} Ton~: L\textsubscript{2}.
\textcolor{PineGreen}{\selectlanguage{french}Famille.} \zh{家庭(一户人)。}  ¶ \textcolor{darkblue}{\textbf{\ipa{hĩ˧ | ɖɯ˧-ʑi˩}}} \textcolor{PineGreen}{\selectlanguage{french}une famille} \zh{一家人}  
 ¶ \textcolor{darkblue}{\textbf{\ipa{ʈʂʰɯ˧-ʑi˥}}} \textcolor{PineGreen}{\selectlanguage{french}cette famille-ci} \zh{这家}  
 ¶ \textcolor{darkblue}{\textbf{\ipa{ŋwæ˧-qʰv̩˧, | tsʰe˧ɲi˧-ʑi˩}}} \textcolor{PineGreen}{\selectlanguage{french}Cinq hameaux, douze familles! (Formule résumant la statistique du village de \textcolor{darkblue}{\textbf{\ipa{/ə˧lɑ˧-ʁwɤ\#˥/}}}.)} \zh{五个村落,十二个家庭!(阿拉瓦村的人口简介)}  

\lhead{\firstmark}
\rhead{\botmark}

\subsection{\hspace{-0.5cm} {\Large \textcolor{darkblue}{\textbf{\ipa{ʑi˩\textsubscript{b}}}}}\hspace{0.5cm}[\kern2pt{\textcolor{darkblue}{\textbf{\ipa{ʑi˥}}}}\kern2pt]} \hypertarget{z£i\string_Bb1}{}
\markboth{\textcolor{darkblue}{\textbf{\ipa{ʑi˩\textsubscript{b}}}}}{}
\textcolor{teal}{\mytextsc{verbe}} \hspace{4pt} Ton~: L\textsubscript{b}.
\ding{202} \textcolor{PineGreen}{\selectlanguage{french}Attraper, saisir, prendre (ex.: un animal récalcitrant).} \zh{拿,捉 (捉鸡)。}  ¶ \textcolor{darkblue}{\textbf{\ipa{æ˩ ʑi˧}}} \textcolor{PineGreen}{\selectlanguage{french}attraper (un/le) poulet} \zh{捉鸡}  
 ¶ \textcolor{darkblue}{\textbf{\ipa{æ˩˥ | le˧-ʑi˩}}} \textcolor{PineGreen}{\selectlanguage{french}attraper (un/le) poulet} \zh{捉鸡}  
 ¶ \textcolor{darkblue}{\textbf{\ipa{hĩ˧ ʑi˧˥}}} \textcolor{PineGreen}{\selectlanguage{french}attraper quelqu'un} \zh{抓人}  
 ¶ \textcolor{darkblue}{\textbf{\ipa{ʁæ˧ ʑi˧}}} \textcolor{PineGreen}{\selectlanguage{french}passer le bras autour du cou de quelqu'un} \zh{搂(用胳膊搂脖子)}  
\ding{203} \textcolor{PineGreen}{\selectlanguage{french}Amener, prendre avec soi.} \zh{带、拿过来。}  ¶ \textcolor{darkblue}{\textbf{\ipa{tso˧\textasciitilde{}tso˧ ʑi˧˥}}} \textcolor{PineGreen}{\selectlanguage{french}attraper quelque chose} \zh{带东西过来}  

\lhead{\firstmark}
\rhead{\botmark}

\subsection{\hspace{-0.5cm} {\Large \textcolor{darkblue}{\textbf{\ipa{ʑi˩hṽ\#˥}}}}\hspace{0.5cm}[\kern2pt{\textcolor{darkblue}{\textbf{\ipa{ʑi˧hṽ˥}}}}\kern2pt]} \hypertarget{z£i\string_Bhv\string_~\#\string_T1}{}
\markboth{\textcolor{darkblue}{\textbf{\ipa{ʑi˩hṽ\#˥}}}}{}
\textcolor{teal}{\mytextsc{nom}} \hspace{4pt} Ton~: LM+\#H.
\textcolor{PineGreen}{\selectlanguage{french}Poils corporels.} \zh{人身上的毛。}  \zh{量词}: \textcolor{darkblue}{\textbf{\ipa{kʰɯ˩}}}  \mytextsc{clf}~: \textcolor{darkblue}{\textbf{\ipa{kʰɯ˩}}} 
\lhead{\firstmark}
\rhead{\botmark}

\subsection{\hspace{-0.5cm} {\Large \textcolor{darkblue}{\textbf{\ipa{ʑi˩-kʰv̩˧˥}}}}\hspace{0.5cm}[\kern2pt{\textcolor{darkblue}{\textbf{\ipa{xxxx non-correspondance entre le nombre de morphèmes et le nombre de tons de morphèmes}}}}\kern2pt]} \hypertarget{z£i\string_B-k\string_hv\string_=\string_M\string_T1}{}
\markboth{\textcolor{darkblue}{\textbf{\ipa{ʑi˩-kʰv̩˧˥}}}}{}
\textcolor{teal}{\mytextsc{nom}} \hspace{4pt} Ton~: LM+MH\#.
\textcolor{PineGreen}{\selectlanguage{french}Année du singe.} \zh{猴年。} 
\lhead{\firstmark}
\rhead{\botmark}

\subsection{\hspace{-0.5cm} {\Large \textcolor{darkblue}{\textbf{\ipa{ʑi˩mi\#˥}}}}\hspace{0.5cm}[\kern2pt{\textcolor{darkblue}{\textbf{\ipa{ʑi˧mi˧}}}}\kern2pt]} \hypertarget{z£i\string_Bmi\#\string_T1}{}
\markboth{\textcolor{darkblue}{\textbf{\ipa{ʑi˩mi\#˥}}}}{}
\textcolor{teal}{\mytextsc{nom}} \hspace{4pt} Ton~: LM+\#H.
\textcolor{PineGreen}{\selectlanguage{french}Singe femelle.} \zh{母猴。}  \zh{量词}: \textcolor{darkblue}{\textbf{\ipa{mi˩}}}  \mytextsc{clf}~: \textcolor{darkblue}{\textbf{\ipa{mi˩}}} 
\lhead{\firstmark}
\rhead{\botmark}

\subsection{\hspace{-0.5cm} {\Large \textcolor{darkblue}{\textbf{\ipa{ʑi˩pʰv̩\#˥}}}}\hspace{0.5cm}[\kern2pt{\textcolor{darkblue}{\textbf{\ipa{ʑi˧pʰv̩˥}}}}\kern2pt]} \hypertarget{z£i\string_Bp\string_hv\string_=\#\string_T1}{}
\markboth{\textcolor{darkblue}{\textbf{\ipa{ʑi˩pʰv̩\#˥}}}}{}
\textcolor{teal}{\mytextsc{nom}} \hspace{4pt} Ton~: LM+\#H.
\textcolor{PineGreen}{\selectlanguage{french}Singe mâle.} \zh{公猴。}  \zh{量词}: \textcolor{darkblue}{\textbf{\ipa{mi˩}}}  \mytextsc{clf}~: \textcolor{darkblue}{\textbf{\ipa{mi˩}}} 
\lhead{\firstmark}
\rhead{\botmark}

\subsection{\hspace{-0.5cm} {\Large \textcolor{darkblue}{\textbf{\ipa{ʑi˩zo\#˥}}}}\hspace{0.5cm}[\kern2pt{\textcolor{darkblue}{\textbf{\ipa{ʑi˧zo˥}}}}\kern2pt]} \hypertarget{z£i\string_Bzo\#\string_T1}{}
\markboth{\textcolor{darkblue}{\textbf{\ipa{ʑi˩zo\#˥}}}}{}
\textcolor{teal}{\mytextsc{nom}} \hspace{4pt} Ton~: LM+\#H.
\textcolor{PineGreen}{\selectlanguage{french}Petit singe.} \zh{小猴子。}  \zh{量词}: \textcolor{darkblue}{\textbf{\ipa{ɭɯ˧}}}  \mytextsc{clf}~: \textcolor{darkblue}{\textbf{\ipa{ɭɯ˧}}} 
\lhead{\firstmark}
\rhead{\botmark}

\subsection{\hspace{-0.5cm} {\Large \textcolor{darkblue}{\textbf{\ipa{ʑi˧˥}}} \textsubscript{1}}\hspace{0.5cm}[\kern2pt{\textcolor{darkblue}{\textbf{\ipa{ʑi˧˥}}}}\kern2pt]} \hypertarget{z£i\string_M\string_T1}{}
\markboth{\textcolor{darkblue}{\textbf{\ipa{ʑi˧˥}}} \textsubscript{1}}{}
\textcolor{teal}{\mytextsc{verbe}} \hspace{4pt} Ton~: MH.
\textcolor{PineGreen}{\selectlanguage{french}Dormir.} \zh{睡觉。}  ¶ \textcolor{darkblue}{\textbf{\ipa{le˧-ʑi˧˥}}} \textcolor{PineGreen}{\selectlanguage{french}\mytextsc{accomp}} \zh{\mytextsc{accomp}}  
 ¶ \textcolor{darkblue}{\textbf{\ipa{le˧-ʑi˧-ze˥}}} \textcolor{PineGreen}{\selectlanguage{french}\mytextsc{accomp}  \mytextsc{pfv}} \zh{\mytextsc{accomp}  \mytextsc{pfv}}  
 ¶ \textcolor{darkblue}{\textbf{\ipa{æ˩ ʑi˧-ze˥}}} \textcolor{PineGreen}{\selectlanguage{french}la poule s'est endormie} \zh{鸡睡觉了}  
 ¶ \textcolor{darkblue}{\textbf{\ipa{le˧-ʑi˧-bi˧-ze˩!}}} \textcolor{PineGreen}{\selectlanguage{french}( je) vais dormir!/(je) vais me coucher!} \zh{要睡觉了!}  
 ¶ \textcolor{darkblue}{\textbf{\ipa{le˧-ʑi˧˥, | ʑi˧-mɤ˥-tʰɑ˩!}}} \textcolor{PineGreen}{\selectlanguage{french}j'essaierais de dormir, que je n'y arriverais pas! / Dormir, il ne faut pas y penser/je n'y arriverais pas! (contexte: une personne âgée se plaint de maux de tête; on lui suggère d'aller se reposer/faire une sieste)} \zh{想睡,但睡不了!}  
 ¶ \textcolor{darkblue}{\textbf{\ipa{pʰæ˧tɕi˥-zo˩-ɳɯ˩ | mv̩˩zo˩-qɑ˥ ʑi˩}}} \textcolor{PineGreen}{\selectlanguage{french}Le jeune homme couche avec la jeune femme. (Formulation considérée comme vulgaire; ce thème est tabou.)} \zh{小伙子跟年轻女人睡!(庸俗说法)}  

\lhead{\firstmark}
\rhead{\botmark}

\subsection{\hspace{-0.5cm} {\Large \textcolor{darkblue}{\textbf{\ipa{ʑi˩˥}}}}\hspace{0.5cm}[\kern2pt{\textcolor{darkblue}{\textbf{\ipa{ʑi˩˥}}}}\kern2pt]} \hypertarget{z£i\string_B\string_T1}{}
\markboth{\textcolor{darkblue}{\textbf{\ipa{ʑi˩˥}}}}{}
\textcolor{teal}{\mytextsc{nom}} \hspace{4pt} Ton~: LH.
\textcolor{PineGreen}{\selectlanguage{french}Singe.} \zh{猴子。}  ¶ \textcolor{darkblue}{\textbf{\ipa{ʑi˩ dzɯ˧-ze˩}}} \textcolor{PineGreen}{\selectlanguage{french}(sujet non spécifié: un tigre...) a mangé (le) singe} \zh{吃了猴子}  
 ¶ \textcolor{darkblue}{\textbf{\ipa{ʑi˩ hwæ˧-ze˩}}} \textcolor{PineGreen}{\selectlanguage{french}(sujet non spécifié: quelqu'un) a acheté (le) singe} \zh{买了猴子}  
 \zh{量词}: \textcolor{darkblue}{\textbf{\ipa{mi˩}}}  \mytextsc{clf}~: \textcolor{darkblue}{\textbf{\ipa{mi˩}}} 
\lhead{\firstmark}
\rhead{\botmark}

\subsection{\hspace{-0.5cm} {\Large \textcolor{darkblue}{\textbf{\ipa{*ʑi˩˧}}}}\hspace{0.5cm}[\kern2pt{\textcolor{darkblue}{\textbf{\ipa{ʑi˩˥}}}}\kern2pt]} \hypertarget{*z£i\string_B\string_M1}{}
\markboth{\textcolor{darkblue}{\textbf{\ipa{*ʑi˩˧}}}}{}
\textcolor{teal}{\mytextsc{nom}} \hspace{4pt} Ton~: LM? LH?.
\textcolor{PineGreen}{\selectlanguage{french}Maison, bâtiment d'habitation; monosyllabe extrait de l'expression /ʑi˩ tsʰi˩˥, | æ̃˩ tsʰi˧/ 'bâtir une demeure': le schéma tonal, avec un verbe au ton MH, peut provenir d'un nom au ton LM ou LH.} \zh{房屋。}  ¶ \textcolor{darkblue}{\textbf{\ipa{ʑi˩ tsʰi˧˥, | æ̃˩ tsʰi˥}}} \textcolor{PineGreen}{\selectlanguage{french}Bâtir une demeure, bâtir un foyer (expression proverbiale)} \zh{建房立家(固定词语)}  
 ¶ \textcolor{darkblue}{\textbf{\ipa{ʑi˩ tʰv̩˩˥}}} \textcolor{PineGreen}{\selectlanguage{french}Fonder une nouvelle demeure: lorsque la famille est nombreuse, un de ses membres peut bâtir sa propre demeure} \zh{分家,建立自己的新房屋比如:孩子多,一个孩子建自己的房子)}  
 ¶ \textcolor{darkblue}{\textbf{\ipa{ʑi˩ qʰæ˧˥}}} \textcolor{PineGreen}{\selectlanguage{french}démolir une demeure} \zh{拆房子(这个例子是调查者构造的,F4确定是可以说的。造这个例子的目的有两个:看单音节词根“家”能不能跟其它动词结合,也试着确定它的调类。)}  
 ¶ \textcolor{darkblue}{\textbf{\ipa{*ʑi˩ hwæ˧}}} \textcolor{PineGreen}{\selectlanguage{french}*acheter une demeure; forme non correcte, proposée pour voir dans quelle mesure le monosyllabe peut se combiner avec divers verbes.} \zh{*买房(这个例子是调查者构造的,F4确定是不可以说的。造这个例子的目的有两个:看单音节词根“家”能不能跟其它动词结合,也试着确定它的调类。)}  

\lhead{\firstmark}
\rhead{\botmark}

\lhead{\firstmark}
\rhead{\botmark}

\lhead{\firstmark}
\rhead{\botmark}

\lhead{\firstmark}
\rhead{\botmark}

\lhead{\firstmark}
\rhead{\botmark}

\lhead{\firstmark}
\rhead{\botmark}

\lhead{\firstmark}
\rhead{\botmark}

\lhead{\firstmark}
\rhead{\botmark}

\lhead{\firstmark}
\rhead{\botmark}

\lhead{\firstmark}
\rhead{\botmark}

\lhead{\firstmark}
\rhead{\botmark}

\lhead{\firstmark}
\rhead{\botmark}

\lhead{\firstmark}
\rhead{\botmark}

\lhead{\firstmark}
\rhead{\botmark}

\lhead{\firstmark}
\rhead{\botmark}

\lhead{\firstmark}
\rhead{\botmark}

\lhead{\firstmark}
\rhead{\botmark}

\lhead{\firstmark}
\rhead{\botmark}

\lhead{\firstmark}
\rhead{\botmark}

\lhead{\firstmark}
\rhead{\botmark}

\lhead{\firstmark}
\rhead{\botmark}

\lhead{\firstmark}
\rhead{\botmark}

\end{multicols}

\newpage
\section*{\centering Mots dont aucun équivalent n'a été trouvé}
Cette liste regroupe les mots dont aucun équivalent n'a été trouvé. Même s'il ne s'agit que d'informations négatives, elles éclairent les limites du vocabulaire na des consultants.
\begin{center}
\begin{longtable}{r|l}
blaireau猪獾 & \textcolor{brown}{\zh{猪獾}} \\
lynx猞猁 & \textcolor{brown}{\zh{猞猁}} \\
pangolin穿山甲 & \textcolor{brown}{\zh{穿山甲}} \\
lobe\string_de\string_l'oreille耳垂 & \textcolor{brown}{\zh{耳垂}} \\
tempes太阳穴 & \textcolor{brown}{\zh{太阳穴}} \\
fromage\string_séché乳扇 & \textcolor{brown}{\zh{乳扇}} \\
brosse\string_à\string_dent\string_en\string_poil\string_de\string_cochon猪鬃毛牙刷 & \textcolor{brown}{\zh{猪鬃毛牙刷}} \\
catapulte抛石机 & \textcolor{brown}{\zh{抛石机}} \\
lance\string_pierre绷弓子 & \textcolor{brown}{\zh{绷弓子}} \\
Angelica当归 & \textcolor{brown}{\zh{当归}} \\
Anisodus\string_tanguticus山茛菪 & \textcolor{brown}{\zh{山茛菪}} \\
Benincasa\string_hispida冬瓜 & \textcolor{brown}{\zh{冬瓜}} \\
céleri芹菜 & \textcolor{brown}{\zh{芹菜}} \\
Ephedra\string_sinica草麻黄 & \textcolor{brown}{\zh{草麻黄}} \\
if红豆杉 & \textcolor{brown}{\zh{红豆杉}} \\
Ligularia\string_fischeria山紫菀 & \textcolor{brown}{\zh{山紫菀}} \\
Selaginella卷柏 & \textcolor{brown}{\zh{卷柏}} \\
yéti野人 & \textcolor{brown}{\zh{野人}} \\
jumeaux双胞胎 & \textcolor{brown}{\zh{双胞胎}} \\
boire\string_à\string_la\string_paille用吸管喝 & \textcolor{brown}{\zh{用吸管喝}} \\
regretter后悔 & \textcolor{brown}{\zh{后悔}} \\
\end{longtable}\end{center}
\end{document}
