\documentclass[twoside,11pt]{book}
\author{Alexis Michaud}
% ajouté en novembre 2016, en remplacement de vmargin
\usepackage[paperwidth=185mm,paperheight=260mm,top=16mm,bottom=16mm,left=15mm,right=20mm]{geometry}
\usepackage{pifont}
\usepackage{fontspec}
\usepackage{natbib}
\usepackage{booktabs}
\usepackage{xltxtra}
\usepackage{polyglossia}
\usepackage[dvipsnames,table]{xcolor}
\usepackage{longtable}
\definecolor{darkblue}{rgb}{0,0,0.75}
\usepackage{float}
\usepackage{lineno}
\usepackage{textcomp}
\usepackage{memhfixc}
\usepackage{lscape}
\usepackage{amssymb}
\usepackage{multicol}
\setlength{\columnseprule}{1pt}
\setlength{\columnsep}{1.5cm}
%\setmainfont[Mapping=tex-text,Numbers=OldStyle,Ligatures=Common]{Charis SIL} 
\setmainfont[Mapping=tex-text,Numbers=OldStyle,Ligatures=Common,ItalicFont=*,ItalicFeatures=FakeSlant,Scale=MatchLowercase]{DoulosSIL}
\newfontfamily\phon[Mapping=tex-text,Ligatures=Common,Scale=MatchLowercase]{Charis SIL} % ou DoulosSIL
\newcommand{\ipa}[1]{{\phon #1}} % API jamais en italique
\newcommand{\grise}[1]{\cellcolor{lightgray}\textbf{#1}}
\newcommand{\bleute}[1]{\cellcolor{green}\textbf{#1}}
\newcommand{\rouge}[1]{\cellcolor{red}\textbf{#1}}
\newfontfamily\cn[Mapping=tex-text,Ligatures=Common,Scale=MatchUppercase]{SimSun} % pour le chinois
\newcommand{\zh}[1]{{\cn #1}}
\newcommand{\topic}{\textsc{dem}}
\newcommand{\tete}{\textsuperscript{\textsc{head}}}
\newcommand{\rc}{\textsubscript{\textsc{rc}}}
\XeTeXlinebreaklocale "zh" % 使用中文换行
\XeTeXlinebreakskip = 0pt plus 1pt  % CIRCG
\usepackage{fancyhdr}
\pagestyle{fancy}
\fancyheadoffset{3.4em}
\fancyhead[LE,RO]{\thepage} % Numéros de page sur les côtés extérieurs
% \setlength{\oddsidemargin}{12mm}
% \setlength{\evensidemargin}{18mm}
\usepackage[dvipdfmx,xetex,bigfiles,final,activate=onclick,deactivate=onclick,transparent,passcontext]{media9}
\usepackage{graphicx}
\usepackage[bookmarks=true,colorlinks,linkcolor=blue]{hyperref}
\hypersetup{bookmarks=false,bookmarksnumbered,bookmarksopenlevel=5,bookmarksdepth=5,xetex,colorlinks=true,linkcolor=blue,citecolor=blue}
\usepackage[all]{hypcap}
%%% retirés en novembre 2016 : pas utiles. Remplacé par: geometry.
%%\usepackage{vmargin}
%%% {marge gauche}{marge en haut}{marge droite}{marge en bas}{hauteur de l'entête}{distance entre l'entête et le texte}{hauteur du pied de page}{distance entre le texte et le pied de page}
%%\setmarginsrb{2cm}{1cm}{1.5cm}{1cm}{0.5cm}{1cm}{0.5cm}{1cm}

\def\mytextsc{\bgroup\obeyspaces\mytextscaux}
\def\mytextscaux#1{\mytextscauxii #1\relax\relax\egroup}
\def\mytextscauxii#1{%
\ifx\relax#1\else \ifcat#1\@sptoken{} \expandafter\expandafter\expandafter\mytextscauxii\else
\ifnum`#1=\uccode`#1 {\normalsize #1}\else {\footnotesize \uppercase{#1}}\fi \expandafter\expandafter\expandafter\mytextscauxii\expandafter\fi\fi}
% Ne pas afficher la numérotation des sections, sous-sections etc
\setcounter{secnumdepth}{0}
\newfontfamily\englishfont{Linux Libertine O}
\newfontfamily\frenchfont{EB Garamond}



\title{Na-English-Chinese-French Dictionary}
\setdefaultlanguage{english}
\setotherlanguages{chinese}

\begin{document}
\maketitle
\newpage
\markboth{INTRODUCTION}{}

\pagenumbering{roman}
	{\LARGE \textbf{Introduction}}
	\section{About the language} \label{sec:language}

This dictionary documents the lexicon of the Na language (\ipa{nɑ˩-ʐwɤ˥}) as spoken in and around the plain of Yongning, located in Southwestern China, at the border between Yunnan and Sichuan, at a latitude of 27°50’ N and a longitude of 100°41’ E. This language is known locally as ‘Mosuo'. 

	\section{Chronology and method} \label{sec:method}

The author's fieldwork on Yongning Na began in October 2006, with tone as its main focus (lexical tone, and tonal morphology). This required examining as many lexical items as possible to ensure that no tone category was overlooked, but lexicographic work was not in itself a priority. A list of words was begun through elicitation, and gradually expanded and corrected as narratives were recorded and transcribed; addition of new words was therefore a slow process. An advantage of placing the emphasis on text collection is that a context is available to help clarify the meaning of newly encountered words, also offering a basis for further discussion of their usage. But systematic elicitation of large amounts of vocabulary was not carried out, hence the limited number of entries: currently slightly under 3,000. 

Unless otherwise stated, all the data are from one language consultant, Mrs. Latami Dashilame (\ipa{lɑ˧tʰɑ˧mi˥ ʈæ˧ʂɯ˧-lɑ˩mv˩}; Chinese: \zh{拉它米打史拉么}). She was born in 1950 in the hamlet called \ipa{ə˧lɑ˧-ʁwɤ\#˥} in Na, close to the monastery of Yongning. The administrative coordinates of this village are: Yúnnán province, Lìjiāng municipality, Nínglàng Yí autonomous county, Yǒngníng district, Ālāwǎ village (\zh{云南省丽江市宁蒗彝族自治县永宁乡阿拉瓦村}). The choice to work in one location only, and essentially with one consultant, is, again, based on the investigator's focus on the tone system. There is considerable dialectal diversity within the Na area (much more so than in the Naxi-speaking area); the tone systems of different villages are conspicuously different, and this geographical diversity combines with dramatic differences across social groups, and across generations. The obvious thing to do seemed to be an in-depth description and analysis of the language as spoken by one person (simultaneously making a few forays into other idiolects and dialects). Data from other speakers are indicated using their codes in the author's database of speakers of Naish languages. Table \ref{tab:consul} provides the speaker codes.

\begin{table}[H]
	\caption{Language consultant codes}
	\centering \label{tab:consul}
	\begin{tabular}{lllllll}
		\toprule
		speaker code &   name &  year of birth \\
		\midrule
F4 (main consultant) & \ipa{lɑ˧tʰɑ˧mi˥ ʈæ˧ʂɯ˧-lɑ˩mv˩} & 1950 \\ F5 &  \ipa{ki˧zo˧} & 1973  \\ F6 &  \ipa{tɕʰi˧ɖv\#˥} & 1987 \\ M18 &  \ipa{lɑ˧tʰɑ˧mi˥ ʈæ˧ʂɯ˧-ʈæ˩ʈv˩} & 1972 \\ M21 & \ipa{ho˧dʑɤ˧tsʰe˥} & 1942 \\ M23 & \ipa{ɖɯ˩ɖʐɯ˧} & 1974 \\
		\bottomrule
	\end{tabular}
\end{table}

The list of words as of 2011 was deposited in the STEDT database (http://stedt.berkeley.edu/). The same year, under the impetus of Guillaume Jacques and Aimée Lahaussois, plans were made to bring the word list closer to the standards of a full-fledged dictionary. A project was deposited with the Agence Nationale de la Recherche, accepted in 2012, and begun in 2013: the HimalCo project (ANR-12-CORP-0006). Céline Buret, a computing engineer, worked with the project team for two years (Nov. 2014-Oct. 2015). She converted the data to the format of the Field Linguist's Toolbox (MDF), then produced scripts for conversion to a XML format complying with the LMF standard, allowing for automatic conversion to an online format as well as to LaTeX files (with PDF as the final output for circulation). In 2015, version 1.0 of the online and PDF versions of the dictionary were produced and published online, along with the source document in MDF (Toolbox) format.

\section{Guide to using the dictionary} \label{sec:howto}

	\subsection{Formats: trilingual Na-Chinese-English or Na-Chinese French} \label{sec:versions}

Entries and examples have translations into English, Chinese and French. Two language settings are offered for the PDF and online dictionary: either Na-Chinese-English, or Na-Chinese-French. The English and the French are not typeset alongside each other in the same document because distinguishing them visually is not obvious, even with the help of typographic devices such as using different fonts and colours. In the author's own experience, it was found that the presence of four languages alongside one another made consultation more difficult; specifically, English translations tended to be a distraction slowing down access to the French translations, and English users may similarly find that French clutters the layout. On the other hand, Chinese characters are visually well-distinct from Latin-based scripts, and so it did not appear necessary to separate the Chinese and produce a Na-Chinese version. Moreover, Chinese translations are often a useful complement to the translation in English (or French), as there are often closer equivalents: for instance \ipa{gɤ˧˥} translates straightforwardly as Chinese \zh{扛} whereas the English translation is more roundabout: ‘to carry on the shoulder'. Users who wish to have access to all the information can download the original file in Toolbox (MDF) format. 

	\subsection{Format of entries} \label{sec:entries}

Each entry contains
\begin{itemize}
	\item \textit{phonological transcription:} the form of the word in phonetic alphabet; tone is indicated in terms of phonological categories
	\item \textit{part of speech:} an indication of the part of speech, using a simple set of labels
	\item \textit{tone:} the tone category of the word. This information is already present in the phonological transcription; having it repeated on its own facilitates searches
	\item \textit{definitions} in Chinese and English
	\item \textit{examples} with translations
	\item \textit{links} to related words, such as synonyms, or constituent parts of complex words 
	\item \textit{classifier:} for nouns, an indication on the more commonly associated classifiers
\end{itemize}

Among examples, those elicited to verify the output of certain combinations of tones are marked as ‘PHONO': examples elicited for the purpose of the phonological study. Proverbs and sayings are marked as ‘PROVERB'.

Some pieces of information are not shown in the PDF and online versions. These are:
\begin{itemize}
	\item An indication of \textit{semantic domain}: ‘society', ‘house', ‘body', ‘plant', ‘animal'... No attempt was made to use a fine-grained classification of the sort found in the WordNet database of English, where nouns, verbs, adjectives and adverbs are grouped into sets of cognitive synonym \citep{Fellbaum 2005}. This is simply a rough division into subsets for convenient sorting; the labelling relies partly on form, and partly on semantic contents. As for other aspects in the dictionary, choices made reflect the investigator's research priorities: for instance, the entries for ‘day’, ‘night’, ‘month’, ‘year’ were tagged as “classifiers", along with all other nouns that can appear immediately after a numeral. This allowed easy extraction of all classifiers for the purpose of a study of the tone patterns of classifiers \citep{Michaud2013}. These lexical items could just as well have been tagged as 'time', in view of their semantic field. The numbers ‘100’, ‘1,000’ and ‘10,000’ were likewise labelled as “classifiers" rather than numerals.
	\item \textit{Notes on past notations:} information tracing the history of notations, from the first fieldwork to the current version. For instance, the entry \ipa{ŋwɤ˧pʰæ˧˥} ‘tile' has a note that indicates that it was initially written with a M.H tone pattern, and with vowel \ipa{æ} in both syllables: *\ipa{ŋwæ˧pʰæ˥}. The note explains that the perception of \ipa{æ} in the first syllable is due to a phonetic tendency towards regressive vowel harmony. Verifications are also consigned in this field. About half the entries have information of this type.
	\item \textit{glosses:} glosses in English, Chinese and French, intended for the glossing of texts. The dictionary adopts the abbreviations recommended in the Leipzig Glossing Rules \citep{Comrie}; all other terms are provided in full. Glosses mostly follow the choices made by \citep{Lidz2010}.
\end{itemize}

	\subsection{Part-of-speech labelling} \label{sec:pos}
	
Dictionary entries carry a part-of-speech label. A rough-and-ready typology has been followed: see the table below. Needless to say, this system has limitations: a refined typology would require subcategories, e.g. defining classifiers as a subset among nouns; and categories such as ‘adverbs' raise greater difficulties, lacking a clear definition.
\begin{table}
	\caption{Parts of speech}
	\centering \label{tab:consul}
	\begin{tabular}{lll}
		\toprule
		label & meaning & Leipzig Glossing Rules? y/n \\
		\midrule
		adj & adjective & y \\
		clf & classifier & y \\
		clitic & (same) & n \\
		cnj & conjunction & n \\
		ideophone & (same) & n \\
		disc.PTCL & discourse particle & n \\
		intj & interjection & n \\
		lnk & linker & n \\
		n & noun & y \\
		num & numeral & n \\
		pref & prefix & n \\
		postp & postposition & n \\
		pro & pronoun & n \\
		suff & suffix & n \\
		v & verb & y \\
		\bottomrule
	\end{tabular}
\end{table}

No attempt was made at including expressive noises in the dictionary, such as the sound \ipa{ɬː}. The meaning of this sound in Na can be characterized in the same way as that of words in the dictionary: the full definition would be that  \textit{it expresses enjoyment of food or drink (‘Yummy!'), and is also used to express admiration of a beautiful object, scene, or prospect}. But a reason for leaving it out is that, unlike interjections,  \ipa{ɬː} is not pronounced on expiratory airflow, but on inspired airflow. The air flows through the sides of the mouth, which is where saliva flows when one's mouth waters. Observations about such sounds (including clicks), like that of gestures, appeared to fall outside the scope of the dictionary.

	\subsection{Loanwords} \label{sec:loan}

Borrowings from Chinese and Tibetan are indicated as such in cases where identification seems straightforward. No efforts at systematic elicitation of borrowings from either language were made, but all loanwords occurring in texts were added to the dictionary. The information provided includes: donor language; form in the donor language; and explanations. When the number of syllables in the borrowed word is the same as in the donor language, the glosses in English (and French) start by the original word followed by two colons and a translation: e.g. ‘\zh{办法}::solution' for \ipa{pæ˧˥hwɤ˧}. 

	\section{Planned improvements and mid-term perspectives} \label{sec:improv}
	
This dictionary is conceived of as work-in-progress: successive versions will be released, probably every two years or so, (i) as an online dictionary in HTML format, (ii) as PDF documents, and (iii) in database format (native Toolbox/MDF format, then, in due course, the successors of this format). 

Planned improvements for future versions include the addition of
\begin{itemize}
	\item \textit{a phonetic transcription of tone as it surfaces on the item pronounced in isolation:} a surface-phonological transcription of tone, in addition to the indication of the underlying tone category
	\item \textit{audio files for each head word:} this function has successfully been tested, but the editing of audio files still needs to be conducted
	\item \textit{links to the entire set of online recordings}: listing all textual occurrences in the lexicon entry, with links to the audio file and its aligned transcription. Textual occurrences ultimately constitute the best resource to document a word's usage. The examples currently presented in the dictionary are few in number, compared to the occurrences in texts; and their context of use may not be clear, despite efforts at clarifying their nature (singling out examples elicited for the purpose of morpho-phonological investigation by the mention ‘PHONO') and at providing contextual information for examples jotted down during fieldwork.
	\item \textit{more cross-references} between entries, pointing to synonyms, etc.
\end{itemize}

Collaborations are welcome for the following improvements :
\begin{itemize}
	\item \textit{the vocabulary of religion:} the field of religion remains mostly unexplored; the main consultant and I both lack the command of Tibetan that would be essential for this part of the investigation, and involvement of consultants from the Yongning monastery did not prove feasible in view of current restrictions on contacts with foreigners
	\item \textit{plants and animals:} as a dweller of the Yongning plain, the main consultant does not have extensive knowledge of wild plants and animals; the number of entries recorded so far remains small, and some definitions are currently limited to general indications such as ‘a type of pine'. To arrive at exact identification, and at more extensive lexicographic coverage, would require collaboration with other consultants, and with botanists.
\end{itemize}

Last but not least, Roselle Dobbs began adding a proposed orthographic representation for each head word, using a transcription that she developed with Na consultants, with a view to use within the Na community. Until this task is completed and orthography added to the online dictionary, requests for further information about orthographic developments should be sent directly to: rosellemay@hotmail.com 

	\section{Other resources about Yongning Na} \label{sec:resources}
	
	In the classical tradition of linguistic fieldwork \citep{Dixon2007}, a language description should include a dictionary, a grammar, and a collection of texts. 
	
	\begin{itemize}
		\item \textit{A set of Na recordings with time-aligned transcriptions} is available from the Pangloss Collection \citep{Michailovsky2014}; the current web address is lacito.vjf.cnrs.fr/pangloss/languages/Na\_en.htm 
		\item \textit{The grammar} is still in its early stages of preparation. A preliminary draft of a book-length study of Na morpho-tonology can be found online: https://halshs.archives-ouvertes.fr/halshs-01094049/document It also contains detailed information on the phonemic analysis.
	\end{itemize}
	
A review of the literature about Na and the other languages of the Naish  group is provided (in Chinese) by \citet{Li2015}. For an English-language introduction, see \citet{Michaud2015b}.

I would gratefully receive any comments or notifications of errors that the reader may wish to bring to my attention: please send e-mail to michaud.cnrs@gmail.com 


	\section{Acknowledgments} \label{sec:ackno}

Many thanks to Picus Ding for putting me in touch with the Mosuo scholar Latami Dashi. Special thanks to Latami Dashi for supporting and encouraging my work with his mother Latami Dashilame over the years. Many thanks to the main consultant, Latami Dashilame, and to all family members. 

Many thanks to Céline Buret and Séverine Guillaume for their much-appreciated computational expertise, and to Guillaume Jacques for suggestions all along the way. Many thanks to connoisseurs of the Na culture and language for useful exchanges: Lamu Gatusa \zh{拉木嘎吐萨} (Chinese pen-name: \zh{石高峰}), Liberty Lidz, Christine Mathieu, Pascale-Marie Milan and He Sana \zh{何梭娜}. Special thanks to Roselle Dobbs for extensive discussions and vigorous proof-reading over the years. Many thanks to A Hui \zh{阿慧} (to my knowledge the first speaker of Mosuo to read a M.A. degree in language and linguistics) for suggesting corrections. Remaining errors are my own responsibility. 

I am grateful for the opportunity allowed me by my home institution, Centre National de la Recherche Scientifique, of staying in China in 2011-2012 for extensive fieldwork, through a temporary affiliation with the CNRS’s research centre in China: CEFC (Centre d’Etudes Français sur la Chine contemporaine). From November 2012 to June 2016, I was based at the international research institute MICA, in Hanoi, in an exceptionally stimulating environment allowing for close collaboration with colleagues from Asia and elsewhere. Special thanks to the heads of the institute, Phạm Thị Ngọc Yến (succeeded in 2015 by Nguyễn Việt Sơn) and Eric Castelli, for their support and encouragement.

I am grateful to the Dongba Culture Research Institute (\zh{丽江市东巴文化研究院}) in Lijiang and the Horse-Tea Road Culture Research Centre (\zh{云南大学茶马古道文化研究所}) in Kunming for inviting me to become an Adjunct member (\zh{外聘研究员}), and for facilitating administrative and practical matters; special thanks to Li Dejing \zh{李德静} and to Mu Jihong\zh{木霁弘}. At Yunnan University, many thanks are due to Duan Bingchang \zh{段炳昌}, Wang Weidong \zh{王卫东}, Zhao Yanzhen \zh{赵燕珍} and Yang Liquan \zh{杨立权} for their careful and sensitive management of fieldwork-related administrative matters.
	
So many people have supported this project that I must apologize for those names that should be here but were inadvertently left off the list.

This work was supported financially by the ANR project HimalCo (ANR-12-CORP-0006), and constitutes a contribution to the LabEx “Empirical Foundations of Linguistics" project (ANR-10-LABX-0083).

\begin{thebibliography}{7}
	\providecommand{\natexlab}[1]{#1}
	\providecommand{\url}[1]{#1}
	\providecommand{\urlprefix}{}
	\expandafter\ifx\csname urlstyle\endcsname\relax
	\providecommand{\doi}[1]{doi:\discretionary{}{}{}#1}\else
	\providecommand{\doi}{doi:\discretionary{}{}{}\begingroup
		\urlstyle{rm}\Url}\fi
	
	\bibitem[{Li(2015)}]{Li2015}
	Li Zihe [\zh{李子鹤}]. 2015.
	\newblock
	\zh{纳西语言研究回顾------兼论语言在文化研究中的基础地位}
	[{A} review of {Naxi} language studies, with a discussion of the fundamental
	role of cultural studies for linguistic research].
	\newblock \zh{茶马古道研究期刊} 4. 125--131.
	
	\bibitem[{Comrie et~al.()Comrie, Haspelmath \& Bickel}]{Comrie}
	Comrie, Bernard, Martin Haspelmath \& Balthasar Bickel. 2008.
	\newblock Leipzig {Glossing Rules}.
	\newblock
	\urlprefix\url{http://www.eva.mpg.de/lingua/resources/glossing-rules.php}.
	
	\bibitem[{Dixon(2007)}]{Dixon2007}
	Dixon, Robert~M. 2007.
	\newblock Field linguistics: a minor manual.
	\newblock \emph{Sprachtypologie und Universalienforschung} 60(1). 12--31.
	
	\bibitem[{Lidz(2010)}]{Lidz2010}
	Lidz, Liberty. 2010.
	\newblock \emph{A descriptive grammar of {Yongning Na} ({Mosuo})}.
	\newblock Austin: University of Texas, Department of linguistics dissertation.
	\newblock
	\urlprefix\url{https://repositories.lib.utexas.edu/bitstream/handle/2152/ETD-UT-2010-12-2643/LIDZ-DISSERTATION.pdf}.
	\newblock Ph. D.
	
	\bibitem[{Michailovsky et~al.(2014)Michailovsky, Mazaudon, Michaud, Guillaume,
		Fran{\c{c}}ois \& Adamou}]{Michailovsky2014}
	Michailovsky, Boyd, Martine Mazaudon, Alexis Michaud, S{\'{e}}verine Guillaume,
	Alexandre Fran{\c{c}}ois \& Evangelia Adamou. 2014.
	\newblock Documenting and researching endangered languages: the {Pangloss
		Collection}.
	\newblock \emph{Language Documentation and Conservation} 8. 119--135.
	\newblock \urlprefix\url{http://hdl.handle.net/10125/4621}.
	
	\bibitem[{Michaud(2013)}]{Michaud2013}
	Michaud, Alexis. 2013.
	\newblock The tone patterns of numeral-plus-classifier phrases in {Yongning
		Na}: a synchronic description and analysis.
	\newblock In Nathan Hill \& Tom Owen-Smith (eds.), \emph{Transhimalayan
		{Linguistics}. {Historical} and {Descriptive} {Linguistics} of the
		{Himalayan} {Area}} (Trends in {Linguistics}. {Studies} and {Monographs}
	[{TiLSM}] 266), 275--311. Berlin: De Gruyter Mouton.
	
	\bibitem[{Michaud et~al.(2015)Michaud, Limin \& Yaoping}]{Michaud2015b}
	Michaud, Alexis, He~Limin \& Zhong Yaoping. 2015.
	\newblock Naxi / {Naish}.
	\newblock In Rint Sybesma, Wolfgang Behr, Zev Handel \& C.T.~James Huang
	(eds.), \emph{Encyclopedia of {Chinese} {Language} and {Linguistics}},
	Leiden: Brill.
	
\end{thebibliography}

\cleardoublepage


	{\LARGE \textbf{\zh{前言}}}
	\section{ \zh{缘起}} \label{sec:language}

 \zh{ 在北纬27.50度,东经100.41度的交界点上生活着一个知名的族群——摩梭人。它奇特的民族风俗使它名扬天下,但是,它的语言研究还远远跟不上它的名气。作为一名语音学家找到了摩梭语言就如同进入了阿拉丁的宝库:摩梭语音的丰富多彩,尤其是声调在整个语法系统里所起的核心作用,独特而又迷人。
 	本词典的全部信息来自于一位充满独特个人魅力的摩梭阿妈——拉它米打史拉么。1950年出生的这位阿妈,差不多与新中国同年龄,她身上的故事就是摩梭人今天与昨天的缩影。在云南省丽江市宁蒗县永宁乡平静村独自养育了四个儿女,其中一位就是摩梭著名学者拉他咪达石(拉他咪王勇)。有幸结识拉他咪达石先生与他亲爱的母亲是这本词典得以与世人见面的关键,在此向他们表示最深切的感谢。}

	\section{ \zh{词典}} \label{sec:method}

 \zh{2006年在纳西语语音研究告一段落之后,我开始了对摩梭语音的分析。一开始的初衷只是进行音系分析,但却惊奇地发现摩梭声调不仅是单纯的音系问题,同时与语法有着千丝万缕的联系。这一发现,促使我开始对摩梭话进行全方位的研究。研究方法以搜集来的长篇语料为蓝本,在搜集语料的同时进行音系实验(系统的机械问答),并以长篇语料中的词汇为基准,汇集成为本部词典。词典目前的条目数量不到三千,但我的初衷并不致力于单纯的词汇收集而是希望能在有限的条目中最大限度地呈现摩梭语音的独特性。

本词典以国际音标为注音基础。声调标注系统由于比较复杂,因此启用了一套专门符号,该符号的用法及说明见}Michaud 2015\zh{。

本词典有三种呈现格式,在线词典、PDF文本、与Toolbox资料库。内容将陆续进行更正及删补。同时欢迎读者的指正与批评。来信请寄}alexis.michaud@vjf.cnrs.fr \zh{。}

\section{ \zh{其它}} \label{sec:other}

 \zh{著名语言学家孙天心提出,语言学家有三件宝:长篇语料、词典与参考语法。以这个标准来看,长篇语料就是我囊中最大的宝贝,历时十年搜集而来的近二十个小时的资料及逐句标注翻译,已经全部在线公开。在线地址:“泛语资料库”(}Pangloss Collection \zh{),}lacito.vjf.cnrs.fr/pangloss/languages/Na\_en.htm \zh{。下一步的相关工作,是完成一本《摩梭话声调研究》,初稿也已上线,地址:}https://halshs.archives-ouvertes.fr/halshs-01094049/document \zh{。}

	\section{ \zh{致谢}} \label{sec:thks}
	
\zh{拉它米打史拉么阿妈及全家人、拉他咪达石(摩梭知名学者,出版有多部摩梭文化研究专著)、李德静(丽江市东巴文化研究院院长)、黄行(中央社会科学院民族学研究所所长)、段炳昌(云南大学人文学院院长)、木霁弘(云南大学茶马古道文化研究所所长)、王卫东(云南大学人文学院中文系系主任)、和学光(云南党校图书馆馆长)、赵燕珍(云南大学人文学院中文系教授)、杨立权(云南大学人文学院中文系教授)、丁思之教授(普米语研究专家)、拉木嘎吐萨(摩梭知名学者)、阿慧(摩梭人研究生)、}Céline Buret\zh{(工程师)、}Séverine Guillaume\zh{(工程师)、向柏霖(}Guillaume Jacques\zh{,语言学专家)、
李力(}Liberty Lidz\zh{(,摩梭话研究专家)、杜玫瑰(}Roselle Dobbs\zh{,摩梭研究者)、}Pascale-Marie Milan\zh{(人类学家,摩梭文化研究专家)、}Christine Mathieu\zh{(人类学家,摩梭文化研究专家):感谢你们的大力支持。

在此也向成书过程中曾给予帮助的许多朋友和专家们一并致谢!}




	\section{ \zh{参考书目}} \label{sec:refs}
	\begin{itemize}
		\item \zh{李子鹤. 纳西语言研究回顾——兼论语言在文化研究中的基础地位[J]. 茶马古道研究期刊, 2015, 4: 125–131.}
		\item \zh{孙天心. 藏缅语的调查[J]. 语言学论丛, 2007, 36: 98–107.}
		\item LIDZ L. A descriptive grammar of Yongning Na (Mosuo) [D]. Austin: University of Texas, Department of linguistics, 2010.
		\zh{(下载地址:}https://repositories.lib.utexas.edu/bitstream/handle/2152/ETD-UT-2010-12-2643/LIDZ-DISSERTATION.pdf\zh{)}
		\item MICHAUD A. Phrasing, prominence, and morphotonology: How utterances are divided into tone groups in Yongning Na [J]. Bulletin of Chinese Linguistics, 2015. \zh{(下载出版前版本的地址:}https://halshs.archives-ouvertes.fr/halshs-01162331\zh{)}
	\end{itemize}

\cleardoublepage
\pagenumbering{arabic}
\setcounter{page}{1}
 

	


\def\mytextsc{\bgroup\obeyspaces\mytextscaux}
\def\mytextscaux#1{\mytextscauxii #1\relax\relax\egroup}
\def\mytextscauxii#1{%
\ifx\relax#1\else \ifcat#1\@sptoken{} \expandafter\expandafter\expandafter\mytextscauxii\else
\ifnum`#1=\uccode`#1 {\normalsize #1}\else {\footnotesize \uppercase{#1}}\fi \expandafter\expandafter\expandafter\mytextscauxii\expandafter\fi\fi}

\setlength\parindent{0cm}

\addmediapath{/home/benjamin/Projets/pylmflib-1.1/examples/na/mp3}
\addmediapath{/home/benjamin/Projets/pylmflib-1.1/examples/na/mp3/mp3}
\addmediapath{/home/benjamin/Projets/pylmflib-1.1/examples/na/mp3/wav}
\graphicspath{{/home/benjamin/Projets/pylmflib-1.1/examples/na/pylmflib/output/img/}}

\newpage
\begin{multicols}{2}

\newpage
\section*{\centering- \textcolor{darkblue}{\textbf{\ipa{ɑ}}} -}
\subsection{\hspace{-0.5cm} {\Large \textcolor{darkblue}{\textbf{\ipa{ɑ˩mi\#˥}}}}\hspace{0.5cm}[\kern2pt{\textcolor{darkblue}{\textbf{\ipa{ɑ˩mi˥}}}}\kern2pt]} \hypertarget{A\string_Bmi\#\string_T1}{}
\markboth{\textcolor{darkblue}{\textbf{\ipa{ɑ˩mi\#˥}}}}{}
\textcolor{teal}{\mytextsc{noun}} \hspace{4pt} Tone: LM+\#H.
\textcolor{Sepia}{\selectlanguage{english}Female goose.} \zh{母鹅。}  ¶ \textcolor{darkblue}{\textbf{\ipa{ɑ˩mi˧-ɑ˥pʰv̩˩}}} \textcolor{Sepia}{\selectlanguage{english}goose and gander} \zh{公鹅与母鹅}  
 \zh{量词}: \textcolor{darkblue}{\textbf{\ipa{mi˩}}}  \mytextsc{clf}: \textcolor{darkblue}{\textbf{\ipa{mi˩}}} 
\lhead{\firstmark}
\rhead{\botmark}

\subsection{\hspace{-0.5cm} {\Large \textcolor{darkblue}{\textbf{\ipa{ɑ˩pʰo˩}}}}\hspace{0.5cm}[\kern2pt{\textcolor{darkblue}{\textbf{\ipa{ɑ˩pʰo˩˥}}}}\kern2pt]} \hypertarget{A\string_Bp\string_ho\string_B1}{}
\markboth{\textcolor{darkblue}{\textbf{\ipa{ɑ˩pʰo˩}}}}{}
\textcolor{teal}{\mytextsc{adverb(ial)}} \hspace{4pt} Tone: L.
\textcolor{Sepia}{\selectlanguage{english}Outside.} \zh{外面。}  ¶ \textcolor{darkblue}{\textbf{\ipa{ɑ˩pʰo˩ bi˩˥}}} \textcolor{Sepia}{\selectlanguage{english}to go outside; to attend to the call of nature} \zh{出去,解手}  
 ¶ \textcolor{darkblue}{\textbf{\ipa{ɑ˩pʰo˩ bi˩-ze˥!}}} \textcolor{Sepia}{\selectlanguage{english}Let's go out! / I must answer the call of nature!} \zh{出去了! / 出去解手!}  
 ¶ \textcolor{darkblue}{\textbf{\ipa{ɑ˩pʰo˩ bi˩-ʂv̩˥ɖv̩˩!}}} \textcolor{Sepia}{\selectlanguage{english}[She] wants to go out! (Context: on a sunny day, a family member senses that a toddler wants to go and play outside.)} \zh{(她)想出去!(情景:婴儿看外边,觉得她好像想出去。)}  
 ¶ \textcolor{darkblue}{\textbf{\ipa{ə˧dɑ˥ | ə˩pʰo˩ hɯ˩-ze˥!}}} \textcolor{Sepia}{\selectlanguage{english}Daddy went out! (Context: explanation provided to a little child who has just woken up and looks for her dad.)} \zh{爸爸出去了!}  
 ¶ \textcolor{darkblue}{\textbf{\ipa{ə˩pʰo˩-bv̩˥ | lo˧ ʝi˧}}} \textcolor{Sepia}{\selectlanguage{english}to work outside the household: to help other families (in particular for harvesting); to go and look for work in a city} \zh{在外边工作:去帮别人家的忙(特别是收庄稼的时候),或者到城市里面打工}  

\lhead{\firstmark}
\rhead{\botmark}

\subsection{\hspace{-0.5cm} {\Large \textcolor{darkblue}{\textbf{\ipa{ɑ˩pʰo˩-hĩ˩}}}}\hspace{0.5cm}[\kern2pt{\textcolor{darkblue}{\textbf{\ipa{xxxx non-correspondance entre le nombre de morphèmes et le nombre de tons de morphèmes}}}}\kern2pt]} \hypertarget{A\string_Bp\string_ho\string_B-hi\string_~\string_B1}{}
\markboth{\textcolor{darkblue}{\textbf{\ipa{ɑ˩pʰo˩-hĩ˩}}}}{}
\textcolor{teal}{\mytextsc{noun}} \hspace{4pt} Tone: L.
\textcolor{Sepia}{\selectlanguage{english}Outsiders; strangers; other people.} \zh{外人,别人。农村人称呼所有城里人为外面的人。} 
\lhead{\firstmark}
\rhead{\botmark}

\subsection{\hspace{-0.5cm} {\Large \textcolor{darkblue}{\textbf{\ipa{ɑ˩pʰv̩˧˥}}}}\hspace{0.5cm}[\kern2pt{\textcolor{darkblue}{\textbf{\ipa{ɑ˩pʰv̩˧˥}}}}\kern2pt]} \hypertarget{A\string_Bp\string_hv\string_=\string_M\string_T1}{}
\markboth{\textcolor{darkblue}{\textbf{\ipa{ɑ˩pʰv̩˧˥}}}}{}
\textcolor{teal}{\mytextsc{verb}} \hspace{4pt} Tone: LM+MH\#.
\textcolor{Sepia}{\selectlanguage{english}To belch, to burp.} \zh{打饱嗝儿。} 
\lhead{\firstmark}
\rhead{\botmark}

\subsection{\hspace{-0.5cm} {\Large \textcolor{darkblue}{\textbf{\ipa{ɑ˩pʰv̩\#˥}}}}\hspace{0.5cm}[\kern2pt{\textcolor{darkblue}{\textbf{\ipa{ɑ˩pʰv̩˥}}}}\kern2pt]} \hypertarget{A\string_Bp\string_hv\string_=\#\string_T1}{}
\markboth{\textcolor{darkblue}{\textbf{\ipa{ɑ˩pʰv̩\#˥}}}}{}
\textcolor{teal}{\mytextsc{noun}} \hspace{4pt} Tone: LM+\#H.
\textcolor{Sepia}{\selectlanguage{english}Gander; male goose.} \zh{公鹅。}  \zh{量词}: \textcolor{darkblue}{\textbf{\ipa{mi˩}}}  \mytextsc{clf}: \textcolor{darkblue}{\textbf{\ipa{mi˩}}} 
\lhead{\firstmark}
\rhead{\botmark}

\subsection{\hspace{-0.5cm} {\Large \textcolor{darkblue}{\textbf{\ipa{ɑ˩ʁo˧}}}}\hspace{0.5cm}[\kern2pt{\textcolor{darkblue}{\textbf{\ipa{ɑ˩ʁo˥}}}}\kern2pt]} \hypertarget{A\string_BRo\string_M1}{}
\markboth{\textcolor{darkblue}{\textbf{\ipa{ɑ˩ʁo˧}}}}{}
\textcolor{teal}{\mytextsc{noun}} \hspace{4pt} Tone: LM.
\textcolor{Sepia}{\selectlanguage{english}Home, central room in the house.} \zh{家、家里。}  ¶ \textcolor{darkblue}{\textbf{\ipa{ɑ˩ʁo˧-hĩ\#˥}}} \textcolor{Sepia}{\selectlanguage{english}members of the family, family members (who live under the same roof), lineage} \zh{家人(住在一起的家人)}  
 ¶ \textcolor{darkblue}{\textbf{\ipa{ɑ˩ʁo˧=ɻæ˩}}} \textcolor{Sepia}{\selectlanguage{english}the members of the family, the family group (living under the same roof)} \zh{家人、家族(住在一起的家人)}  
 ¶ \textcolor{darkblue}{\textbf{\ipa{njɤ˧ | ɑ˩ʁo˧}}} \textcolor{Sepia}{\selectlanguage{english}my home, my house} \zh{我家}  
 ¶ \textcolor{darkblue}{\textbf{\ipa{njɤ˧ | ɑ˩ʁo˧=ɻ̍˩}}} \textcolor{Sepia}{\selectlanguage{english}my family, my lineage, my kin} \zh{我的家族}  
 ¶ \textcolor{darkblue}{\textbf{\ipa{no˧ | ɑ˩ʁo˧}}} \textcolor{Sepia}{\selectlanguage{english}your home, your house} \zh{你家}  
 ¶ \textcolor{darkblue}{\textbf{\ipa{ɑ˩ʁo˧ ʝi˧}}} \textcolor{Sepia}{\selectlanguage{english}to take care of the household, to look after the affairs of the family (in particular: distributing work to the various members, and ensuring that the supplies are not running low)} \zh{管家里的事情(如:分配工作、家务等)}  
 \zh{量词}: \textcolor{darkblue}{\textbf{\ipa{ɭɯ˧}}}  \mytextsc{clf}: \textcolor{darkblue}{\textbf{\ipa{ɭɯ˧}}} 
\lhead{\firstmark}
\rhead{\botmark}

\subsection{\hspace{-0.5cm} {\Large \textcolor{darkblue}{\textbf{\ipa{ɑ˩zo\#˥}}}}\hspace{0.5cm}[\kern2pt{\textcolor{darkblue}{\textbf{\ipa{ɑ˩zo˥}}}}\kern2pt]} \hypertarget{A\string_Bzo\#\string_T1}{}
\markboth{\textcolor{darkblue}{\textbf{\ipa{ɑ˩zo\#˥}}}}{}
\textcolor{teal}{\mytextsc{noun}} \hspace{4pt} Tone: LM+\#H.
\textcolor{Sepia}{\selectlanguage{english}Baby goose.} \zh{小鹅。} 
\lhead{\firstmark}
\rhead{\botmark}

\subsection{\hspace{-0.5cm} {\Large \textcolor{darkblue}{\textbf{\ipa{ɑ˩˧}}}}\hspace{0.5cm}[\kern2pt{\textcolor{darkblue}{\textbf{\ipa{ɑ˩˥}}}}\kern2pt]} \hypertarget{A\string_B\string_M1}{}
\markboth{\textcolor{darkblue}{\textbf{\ipa{ɑ˩˧}}}}{}
\textcolor{teal}{\mytextsc{noun}} \hspace{4pt} Tone: LM.
\textcolor{Sepia}{\selectlanguage{english}Goose.} \zh{鹅。}  ¶ \textcolor{darkblue}{\textbf{\ipa{ɑ˩ dzɯ˥-ze˩}}} \textcolor{Sepia}{\selectlanguage{english}...has eaten (a/the) goose} \zh{吃了鹅}  
 ¶ \textcolor{darkblue}{\textbf{\ipa{ɑ˩ hwæ˧-ze˧}}} \textcolor{Sepia}{\selectlanguage{english}...has bought (a/the) goose} \zh{买了鹅}  
 \zh{量词}: \textcolor{darkblue}{\textbf{\ipa{mi˩}}}  \mytextsc{clf}: \textcolor{darkblue}{\textbf{\ipa{mi˩}}} 
\lhead{\firstmark}
\rhead{\botmark}

\newpage
\section*{\centering- \textcolor{darkblue}{\textbf{\ipa{æ}}} \textcolor{darkblue}{\textbf{\ipa{æ̃}}} -}
\subsection{\hspace{-0.5cm} {\Large \textcolor{darkblue}{\textbf{\ipa{æ˧bæ˧}}}}\hspace{0.5cm}[\kern2pt{\textcolor{darkblue}{\textbf{\ipa{æ˧bæ˧}}}}\kern2pt]} \hypertarget{\{\string_Mb\{\string_M1}{}
\markboth{\textcolor{darkblue}{\textbf{\ipa{æ˧bæ˧}}}}{}
\textcolor{teal}{\mytextsc{noun}} \hspace{4pt} Tone: M.
\textcolor{Sepia}{\selectlanguage{english}Goitre.} \zh{甲状腺肿瘤。}  \zh{量词}: \textcolor{darkblue}{\textbf{\ipa{ɭɯ˧}}}  \mytextsc{clf}: \textcolor{darkblue}{\textbf{\ipa{ɭɯ˧}}} 
\lhead{\firstmark}
\rhead{\botmark}

\subsection{\hspace{-0.5cm} {\Large \textcolor{darkblue}{\textbf{\ipa{æ˧bæ˧-ʈʂʰæ˧ɣɯ\#˥}}}}\hspace{0.5cm}[\kern2pt{\textcolor{darkblue}{\textbf{\ipa{xxxx non-correspondance entre le nombre de morphèmes et le nombre de tons de morphèmes}}}}\kern2pt]} \hypertarget{\{\string_Mb\{\string_M-t`s`\string_h\{\string_MGM\#\string_T1}{}
\markboth{\textcolor{darkblue}{\textbf{\ipa{æ˧bæ˧-ʈʂʰæ˧ɣɯ\#˥}}}}{}
\textcolor{teal}{\mytextsc{noun}} \hspace{4pt} Tone: \#H.
\textcolor{Sepia}{\selectlanguage{english}Kelp (literally “medicine against goiter”, because kelp was introduced in Yongning as a means to provide iodine as a diet supplement, to prevent goiters).} \zh{海带。} 
\lhead{\firstmark}
\rhead{\botmark}

\subsection{\hspace{-0.5cm} {\Large \textcolor{darkblue}{\textbf{\ipa{æ˧ʝi˩}}}}\hspace{0.5cm}[\kern2pt{\textcolor{darkblue}{\textbf{\ipa{æ˧ʝi˩}}}}\kern2pt]} \hypertarget{\{\string_Mj££i\string_B1}{}
\markboth{\textcolor{darkblue}{\textbf{\ipa{æ˧ʝi˩}}}}{}
\textcolor{teal}{\mytextsc{noun}} \hspace{4pt} Tone: L\#.
\textcolor{Sepia}{\selectlanguage{english}Cry.} \zh{叫声。}  ¶ \textcolor{darkblue}{\textbf{\ipa{æ˧ʝi˩ kʰɯ˩}}} \textcolor{Sepia}{\selectlanguage{english}to shout} \zh{叫}  
 ¶ \textcolor{darkblue}{\textbf{\ipa{no˧ | æ˧ʝi˩ tʰɑ˩-kʰɯ˩! | no˧ se˧dʑæ˩ɻæ˩-gv˩! |}}} \textcolor{Sepia}{\selectlanguage{english}Don't make noise!} \zh{别那么大声!}  

\lhead{\firstmark}
\rhead{\botmark}

\subsection{\hspace{-0.5cm} {\Large \textcolor{darkblue}{\textbf{\ipa{æ˧ɲi\#˥}}}}\hspace{0.5cm}[\kern2pt{\textcolor{darkblue}{\textbf{\ipa{æ˧ɲi˧}}}}\kern2pt]} \hypertarget{\{\string_MJi\#\string_T1}{}
\markboth{\textcolor{darkblue}{\textbf{\ipa{æ˧ɲi\#˥}}}}{}
\textcolor{teal}{\mytextsc{noun}} \hspace{4pt} Tone: \#H.
\textcolor{Sepia}{\selectlanguage{english}Suona, trumpet.} \zh{唢呐。}  \zh{量词}: \textcolor{darkblue}{\textbf{\ipa{ɭɯ˧}}}  \mytextsc{clf}: \textcolor{darkblue}{\textbf{\ipa{ɭɯ˧}}} 
\lhead{\firstmark}
\rhead{\botmark}

\subsection{\hspace{-0.5cm} {\Large \textcolor{darkblue}{\textbf{\ipa{æ˧ʁwæ˧}}}}\hspace{0.5cm}[\kern2pt{\textcolor{darkblue}{\textbf{\ipa{æ˧ʁwæ˧}}}}\kern2pt]} \hypertarget{\{\string_MRw\{\string_M1}{}
\markboth{\textcolor{darkblue}{\textbf{\ipa{æ˧ʁwæ˧}}}}{}
\textcolor{teal}{\mytextsc{noun}} \hspace{4pt} Tone: M.
\textcolor{Sepia}{\selectlanguage{english}Apricot.} \zh{杏。}  ¶ \textcolor{darkblue}{\textbf{\ipa{æ˧ʁwæ˧ | ɖɯ˧-ɭɯ˧}}} \textcolor{Sepia}{\selectlanguage{english}an apricot} \zh{一颗杏}  
 \zh{量词}: \textcolor{darkblue}{\textbf{\ipa{ɭɯ˧}}}  \mytextsc{clf}: \textcolor{darkblue}{\textbf{\ipa{ɭɯ˧}}} 
\lhead{\firstmark}
\rhead{\botmark}

\subsection{\hspace{-0.5cm} {\Large \textcolor{darkblue}{\textbf{\ipa{æ˧ʂæ\#˥}}}}\hspace{0.5cm}[\kern2pt{\textcolor{darkblue}{\textbf{\ipa{æ˧ʂæ˧}}}}\kern2pt]} \hypertarget{\{\string_Ms`\{\#\string_T1}{}
\markboth{\textcolor{darkblue}{\textbf{\ipa{æ˧ʂæ\#˥}}}}{}
\textcolor{teal}{\mytextsc{adverb(ial)}} \hspace{4pt} Tone: \#H.
\textcolor{Sepia}{\selectlanguage{english}Yore, long ago.} \zh{从前。}  ¶ \textcolor{darkblue}{\textbf{\ipa{æ˧ʂæ˧-kɤ˥ʈʂɯ˩}}} \textcolor{Sepia}{\selectlanguage{english}Sayings of the old times, oral traditions of the old times} \zh{过去的说法,过去的口传文化}  

\lhead{\firstmark}
\rhead{\botmark}

\subsection{\hspace{-0.5cm} {\Large \textcolor{darkblue}{\textbf{\ipa{æ˧ʂæ˧}}}}\hspace{0.5cm}[\kern2pt{\textcolor{darkblue}{\textbf{\ipa{æ˧ʂæ˧}}}}\kern2pt]} \hypertarget{\{\string_Ms`\{\string_M1}{}
\markboth{\textcolor{darkblue}{\textbf{\ipa{æ˧ʂæ˧}}}}{}
\textcolor{teal}{\mytextsc{noun}} \hspace{4pt} Tone: M.
\textcolor{Sepia}{\selectlanguage{english}Name of a mountain: one of the two main mountains in the vicinity of the Yongning plain. It is a masculine mountain (“the young man”: \textcolor{darkblue}{\textbf{\ipa{/pʰæ˧tɕi˥/}}}), the counterpart to the feminine mountain \textcolor{darkblue}{\textbf{\ipa{/kɤ˧mv̩˧˥/}}} (“the young woman”: \textcolor{darkblue}{\textbf{\ipa{mi˩zɯ˩˥/}}}).} \zh{一座山的名字。}  ¶ \textcolor{darkblue}{\textbf{\ipa{kɤ˧mv̩˧˥, | æ˧ʂæ˧, | ŋwɤ˧hɑ̃˩, | ʂwæ˧gv̩\#˥, | nɑ˩tsʰi˩˥ | -tɕʰɤ˧pɤ˧mi\#˥, | qv̩˧ɻ̍˧-ʈʂʰɑ˧nɑ˥ |}}} \textcolor{Sepia}{\selectlanguage{english}The six mountains of Yongning that carry a name and have a definite symbolic value. The other mountains do not have comparable symbolic value, and fewer people use specific names for them.} \zh{永宁地区有固定名字的六座山。其它的山,因为没有重要的象征意义,因此没有取名。}  

\lhead{\firstmark}
\rhead{\botmark}

\subsection{\hspace{-0.5cm} {\Large \textcolor{darkblue}{\textbf{\ipa{æ˧ʂæ˧-pi˧mv̩˧˥}}}}\hspace{0.5cm}[\kern2pt{\textcolor{darkblue}{\textbf{\ipa{xxxx non-correspondance entre le nombre de morphèmes et le nombre de tons de morphèmes}}}}\kern2pt]} \hypertarget{\{\string_Ms`\{\string_M-pi\string_Mmv\string_=\string_M\string_T1}{}
\markboth{\textcolor{darkblue}{\textbf{\ipa{æ˧ʂæ˧-pi˧mv̩˧˥}}}}{}
\textcolor{teal}{\mytextsc{noun}} \hspace{4pt} Tone: MH\#.
\textit{From:} \textbf{æ˧ʂæ\#˥ and pi˧mv̩˥\$} \textcolor{Sepia}{\selectlanguage{english}Folk tale, tradition; this term is more colloquial than \textcolor{darkblue}{\textbf{\ipa{/æ˧ʂæ˧-tɑ˩mv̩˩/}}}.} \zh{传统故事。}  \zh{量词}: \textcolor{darkblue}{\textbf{\ipa{kʰwɤ˥}}}  \mytextsc{clf}: \textcolor{darkblue}{\textbf{\ipa{kʰwɤ˥}}} 
\lhead{\firstmark}
\rhead{\botmark}

\subsection{\hspace{-0.5cm} {\Large \textcolor{darkblue}{\textbf{\ipa{æ˧ʂæ˧-qʰwæ\#˥}}}}\hspace{0.5cm}[\kern2pt{\textcolor{darkblue}{\textbf{\ipa{xxxx non-correspondance entre le nombre de morphèmes et le nombre de tons de morphèmes}}}}\kern2pt]} \hypertarget{\{\string_Ms`\{\string_M-q\string_hw\{\#\string_T1}{}
\markboth{\textcolor{darkblue}{\textbf{\ipa{æ˧ʂæ˧-qʰwæ\#˥}}}}{}
\textcolor{teal}{\mytextsc{noun}} \hspace{4pt} Tone: \#H.
\textit{From:} \textbf{æ˧ʂæ\#˥ and qʰwæ˧} \textcolor{Sepia}{\selectlanguage{english}Oral tradition; literally: “messages from old times”.} \zh{口传文化。直译:“(来自)古时候的寓意”。}  ¶ \textcolor{darkblue}{\textbf{\ipa{æ˧ʂæ˧-qʰwæ˧-ɳɯ˥ | dʑo˩-ɲi˥!}}} \textcolor{Sepia}{\selectlanguage{english}“I'm not making this up:) this is part of what the old folks have passed down to us! / This is what our traditions say!” (Context: the speaker cites a proverb or saying, and emphasizes that it is to be taken seriously.)} \zh{“(这些道理,不是我个人的意见:)传统中是这样讲的! / 咱们的口传文化中就是这么讲的!”(情景:一个人提到一个谚语,也强调这些不是空话,而是重要的一个道理。)}  

\lhead{\firstmark}
\rhead{\botmark}

\subsection{\hspace{-0.5cm} {\Large \textcolor{darkblue}{\textbf{\ipa{æ˧ʂæ˧-qʰwɤ˧˥}}}}\hspace{0.5cm}[\kern2pt{\textcolor{darkblue}{\textbf{\ipa{xxxx non-correspondance entre le nombre de morphèmes et le nombre de tons de morphèmes}}}}\kern2pt]} \hypertarget{\{\string_Ms`\{\string_M-q\string_hw7\string_M\string_T1}{}
\markboth{\textcolor{darkblue}{\textbf{\ipa{æ˧ʂæ˧-qʰwɤ˧˥}}}}{}
\textcolor{teal}{\mytextsc{noun}} \hspace{4pt} Tone: MH\#.
\textcolor{Sepia}{\selectlanguage{english}Story, folk tale.} \zh{故事。}  ¶ \textcolor{darkblue}{\textbf{\ipa{æ˧ʂæ˧qʰwɤ˧ ʐwɤ˧˥}}} \textcolor{Sepia}{\selectlanguage{english}to tell a story} \zh{讲故事}  
 ¶ \textcolor{darkblue}{\textbf{\ipa{ə˧ʝi˧-ʂɯ˥ʝi˩, | æ˧ʂæ˧qʰwɤ˧ ʐwɤ˧-kv̩˥!}}} \textcolor{Sepia}{\selectlanguage{english}In the old times, (people) used to tell stories!} \zh{在过去,大家经常讲故事!}  
 \zh{量词}: \textcolor{darkblue}{\textbf{\ipa{kʰwɤ˥}}}  \mytextsc{clf}: \textcolor{darkblue}{\textbf{\ipa{kʰwɤ˥}}} 
\lhead{\firstmark}
\rhead{\botmark}

\subsection{\hspace{-0.5cm} {\Large \textcolor{darkblue}{\textbf{\ipa{æ˧tse˥-pʰæ˩}}}}\hspace{0.5cm}[\kern2pt{\textcolor{darkblue}{\textbf{\ipa{æ˧tse˥pʰæ˩}}}}\kern2pt]} \hypertarget{\{\string_Mtse\string_T-p\string_h\{\string_B1}{}
\markboth{\textcolor{darkblue}{\textbf{\ipa{æ˧tse˥-pʰæ˩}}}}{}
\textcolor{teal}{\mytextsc{noun}} \hspace{4pt} Tone: H\#-L.
\textcolor{Sepia}{\selectlanguage{english}Kneebone.} \zh{膝盖骨。}  \zh{量词}: \textcolor{darkblue}{\textbf{\ipa{ɭɯ˧}}}  \mytextsc{clf}: \textcolor{darkblue}{\textbf{\ipa{ɭɯ˧}}} 
\lhead{\firstmark}
\rhead{\botmark}

\subsection{\hspace{-0.5cm} {\Large \textcolor{darkblue}{\textbf{\ipa{æ˧tsɯ˥-pɤ˩lv̩˩}}}}\hspace{0.5cm}[\kern2pt{\textcolor{darkblue}{\textbf{\ipa{æ˧tsɯ˥pɤ˩lv̩˩}}}}\kern2pt]} \hypertarget{\{\string_MtsM\string_T-p7\string_Blv\string_=\string_B1}{}
\markboth{\textcolor{darkblue}{\textbf{\ipa{æ˧tsɯ˥-pɤ˩lv̩˩}}}}{}
\textcolor{teal}{\mytextsc{noun}} \hspace{4pt} Tone: H\#-L.
\textcolor{Sepia}{\selectlanguage{english}Nape.} \zh{项背 、项、脖颈儿。}  \zh{量词}: \textcolor{darkblue}{\textbf{\ipa{ɭɯ˧}}}  \mytextsc{clf}: \textcolor{darkblue}{\textbf{\ipa{ɭɯ˧}}} 
\lhead{\firstmark}
\rhead{\botmark}

\subsection{\hspace{-0.5cm} {\Large \textcolor{darkblue}{\textbf{\ipa{æ˧ʈwɤ˩}}}}\hspace{0.5cm}[\kern2pt{\textcolor{darkblue}{\textbf{\ipa{æ˧ʈwɤ˩}}}}\kern2pt]} \hypertarget{\{\string_Mt`w7\string_B1}{}
\markboth{\textcolor{darkblue}{\textbf{\ipa{æ˧ʈwɤ˩}}}}{}
\textcolor{teal}{\mytextsc{noun}} \hspace{4pt} Tone: L\#.
\textcolor{Sepia}{\selectlanguage{english}The early morning; early in the morning.} \zh{清晨、一大早(鸡叫的时候)。} 
\lhead{\firstmark}
\rhead{\botmark}

\subsection{\hspace{-0.5cm} {\Large \textcolor{darkblue}{\textbf{\ipa{æ˩gv̩˩}}}}\hspace{0.5cm}[\kern2pt{\textcolor{darkblue}{\textbf{\ipa{æ˩gv̩˩˥}}}}\kern2pt]} \hypertarget{\{\string_Bgv\string_=\string_B1}{}
\markboth{\textcolor{darkblue}{\textbf{\ipa{æ˩gv̩˩}}}}{}
\textcolor{teal}{\mytextsc{noun}} \hspace{4pt} Tone: L.
\textcolor{Sepia}{\selectlanguage{english}Ard. There are no distinct words for 'ard' and 'plough'; only ards were in use at the time of fieldwork.} \zh{犁头。}  ¶ \textcolor{darkblue}{\textbf{\ipa{æ˩gv̩˩ tʰv̩˩-nɑ˥}}} \textcolor{Sepia}{\selectlanguage{english}\mytextsc{n}+\mytextsc{dem}+\mytextsc{clf}} \zh{这把犁头}  
 ¶ \textcolor{darkblue}{\textbf{\ipa{æ˩mo˥}}} \textcolor{Sepia}{\selectlanguage{english}used ard, plough which cannot be used anymore} \zh{陈旧的犁头(不能再用了)}  
 ¶ \textcolor{darkblue}{\textbf{\ipa{æ˩mo˥ tʰv̩˩-nɑ˩}}} \textcolor{Sepia}{\selectlanguage{english}\mytextsc{n}+\mytextsc{dem}+\mytextsc{clf}} \zh{这个旧犁杆}  
 ¶ \textcolor{darkblue}{\textbf{\ipa{æ˩-ʂɯ˩˥}}} \textcolor{Sepia}{\selectlanguage{english}new ard, brand new ard} \zh{新的犁头}  
 \zh{量词}: \textcolor{darkblue}{\textbf{\ipa{nɑ˧}}}  \mytextsc{clf}: \textcolor{darkblue}{\textbf{\ipa{nɑ˧}}} 
\lhead{\firstmark}
\rhead{\botmark}

\subsection{\hspace{-0.5cm} {\Large \textcolor{darkblue}{\textbf{\ipa{æ˩gv̩˩-mæ˩qo˥}}}}\hspace{0.5cm}[\kern2pt{\textcolor{darkblue}{\textbf{\ipa{xxxx non-correspondance entre le nombre de morphèmes et le nombre de tons de morphèmes}}}}\kern2pt]} \hypertarget{\{\string_Bgv\string_=\string_B-m\{\string_Bqo\string_T1}{}
\markboth{\textcolor{darkblue}{\textbf{\ipa{æ˩gv̩˩-mæ˩qo˥}}}}{}
\textcolor{teal}{\mytextsc{noun}} \hspace{4pt} Tone: L+H\#.
\textcolor{Sepia}{\selectlanguage{english}Handle (stilt) of the ard, used to control the ard's direction and the furrow's depth.} \zh{犁把。}  ¶ \textcolor{darkblue}{\textbf{\ipa{æ̃˩gv̩˩-mæ˩ ʑi˩-hĩ˥}}} \textcolor{Sepia}{\selectlanguage{english}the person holding the handle of the ard} \zh{抓着犁把的人}  
 ¶ \textcolor{darkblue}{\textbf{\ipa{æ̃˩gv̩˩-mæ˩qo˥ tʰv̩˩-nɑ˩}}} \textcolor{Sepia}{\selectlanguage{english}\mytextsc{n}+\mytextsc{dem}+\mytextsc{clf}} \zh{这个犁把}  
 \zh{量词}: \textcolor{darkblue}{\textbf{\ipa{nɑ˧}}}  \mytextsc{clf}: \textcolor{darkblue}{\textbf{\ipa{nɑ˧}}} 
\lhead{\firstmark}
\rhead{\botmark}

\subsection{\hspace{-0.5cm} {\Large \textcolor{darkblue}{\textbf{\ipa{æ˩mi˧-mv̩˧ʈv̩˥}}}}\hspace{0.5cm}[\kern2pt{\textcolor{darkblue}{\textbf{\ipa{xxxx non-correspondance entre le nombre de morphèmes et le nombre de tons de morphèmes}}}}\kern2pt]} \hypertarget{\{\string_Bmi\string_M-mv\string_=\string_Mt`v\string_=\string_T1}{}
\markboth{\textcolor{darkblue}{\textbf{\ipa{æ˩mi˧-mv̩˧ʈv̩˥}}}}{}
\textcolor{teal}{\mytextsc{noun}} \hspace{4pt} Tone: LM+H\#.
\textcolor{Sepia}{\selectlanguage{english}Anklebone, bone of the top of the foot.} \zh{踝骨。}  \zh{量词}: \textcolor{darkblue}{\textbf{\ipa{ɭɯ˧}}}  \mytextsc{clf}: \textcolor{darkblue}{\textbf{\ipa{ɭɯ˧}}} 
\lhead{\firstmark}
\rhead{\botmark}

\subsection{\hspace{-0.5cm} {\Large \textcolor{darkblue}{\textbf{\ipa{æ˩mi˧-ʁwɤ\#˥}}}}\hspace{0.5cm}[\kern2pt{\textcolor{darkblue}{\textbf{\ipa{xxxx non-correspondance entre le nombre de morphèmes et le nombre de tons de morphèmes}}}}\kern2pt]} \hypertarget{\{\string_Bmi\string_M-Rw7\#\string_T1}{}
\markboth{\textcolor{darkblue}{\textbf{\ipa{æ˩mi˧-ʁwɤ\#˥}}}}{}
\textcolor{teal}{\mytextsc{noun}} \hspace{4pt} Tone: LM+\#H.
\textcolor{Sepia}{\selectlanguage{english}Amiwa. This is the first village along the road from \textcolor{darkblue}{\textbf{\ipa{/qʰæ˧tɕʰi˧/}}} to \textcolor{darkblue}{\textbf{\ipa{/ʈʂo˧ʂɯ\#˥/}}}. In traditional Na geography, which takes Lugu lake as a point of origin, Amiwa is the third village of the plain.} \zh{阿咪瓦、阿米瓦(永宁的一个村落)。}  ¶ \textcolor{darkblue}{\textbf{\ipa{dʑɤ˩bv̩˧kɤ˧-sɑ˥ʁwɤ˩, | hi˩ʁwɤ˩-lo˥, | æ˩mi˧-ʁwɤ\#˥, | lɑ˧lo˧-ʁwɤ˥, | lɑ˧ŋwɤ˧, | bɤ˧tsʰo˧gv̩˥, | ə˧lɑ˧-ʁwɤ\#˥, | gæ˧ɻæ˩, | qʰæ˧tɕʰi˧, | tʰo˧ʈɯ\#˥}}} \textcolor{Sepia}{\selectlanguage{english}the ten villages traditionally considered as part of Yongning} \zh{摩梭传统地理概念中,属于永宁的十个村落}  

\lhead{\firstmark}
\rhead{\botmark}

\subsection{\hspace{-0.5cm} {\Large \textcolor{darkblue}{\textbf{\ipa{æ˩mo˧}}}}\hspace{0.5cm}[\kern2pt{\textcolor{darkblue}{\textbf{\ipa{æ˩mo˥}}}}\kern2pt]} \hypertarget{\{\string_Bmo\string_M1}{}
\markboth{\textcolor{darkblue}{\textbf{\ipa{æ˩mo˧}}}}{}
\textcolor{teal}{\mytextsc{noun}} \hspace{4pt} Tone: LM.
\textcolor{Sepia}{\selectlanguage{english}Beam of the ard, pole of the ard: a long piece of wood linking the yoke to the sole.} \zh{犁杆。}  ¶ \textcolor{darkblue}{\textbf{\ipa{æ˩gv̩˩-mo˥}}} \textcolor{Sepia}{\selectlanguage{english}same meaning} \zh{同上}  
 \zh{量词}: \textcolor{darkblue}{\textbf{\ipa{nɑ˧}}}  \mytextsc{clf}: \textcolor{darkblue}{\textbf{\ipa{nɑ˧}}} \textit{See:} \hyperlink{}{\textcolor{darkblue}{\textbf{\ipa{æ˩gv̩˩}}}} 
\lhead{\firstmark}
\rhead{\botmark}

\subsection{\hspace{-0.5cm} {\Large \textcolor{darkblue}{\textbf{\ipa{æ˩pʰæ˧˥}}}}\hspace{0.5cm}[\kern2pt{\textcolor{darkblue}{\textbf{\ipa{æ˩pʰæ˧˥}}}}\kern2pt]} \hypertarget{\{\string_Bp\string_h\{\string_M\string_T1}{}
\markboth{\textcolor{darkblue}{\textbf{\ipa{æ˩pʰæ˧˥}}}}{}
\textcolor{teal}{\mytextsc{noun}} \hspace{4pt} Tone: LM+MH\#.
\textcolor{Sepia}{\selectlanguage{english}Cliff, overhanging cliff. This term designates the top of the cliff: the relatively flat ground close to the precipice. To refer to the steep (vertical) side of the cliff, one adds \textcolor{darkblue}{\textbf{\ipa{/lɑ˧bi˧/}}} 'steep slope'.} \zh{悬崖、崖山、崖壁。}  ¶ \textcolor{darkblue}{\textbf{\ipa{æ˩pʰæ˧-lɑ˧bi˥}}} \textcolor{Sepia}{\selectlanguage{english}same meaning} \zh{同上}  
 \zh{量词}: \textcolor{darkblue}{\textbf{\ipa{pʰæ˧˥}}}  \mytextsc{clf}: \textcolor{darkblue}{\textbf{\ipa{pʰæ˧˥}}} 
\lhead{\firstmark}
\rhead{\botmark}

\subsection{\hspace{-0.5cm} {\Large \textcolor{darkblue}{\textbf{\ipa{æ˩qʰv̩˥}}}}\hspace{0.5cm}[\kern2pt{\textcolor{darkblue}{\textbf{\ipa{æ˩qʰv̩˥}}}}\kern2pt]} \hypertarget{\{\string_Bq\string_hv\string_=\string_T1}{}
\markboth{\textcolor{darkblue}{\textbf{\ipa{æ˩qʰv̩˥}}}}{}
\textcolor{teal}{\mytextsc{noun}} \hspace{4pt} Tone: LH.
\textcolor{Sepia}{\selectlanguage{english}Cave, cavern, crevice (difficult to enter, or too small for a person to enter).} \zh{小山洞(难进去,或者钻不进去的山洞)。}  \zh{量词}: \textcolor{darkblue}{\textbf{\ipa{ɭɯ˧}}}  \mytextsc{clf}: \textcolor{darkblue}{\textbf{\ipa{ɭɯ˧}}} 
\lhead{\firstmark}
\rhead{\botmark}

\subsection{\hspace{-0.5cm} {\Large \textcolor{darkblue}{\textbf{\ipa{æ˩ʈv̩˥}}}}\hspace{0.5cm}[\kern2pt{\textcolor{darkblue}{\textbf{\ipa{æ˩ʈv̩˥}}}}\kern2pt]} \hypertarget{\{\string_Bt`v\string_=\string_T1}{}
\markboth{\textcolor{darkblue}{\textbf{\ipa{æ˩ʈv̩˥}}}}{}
\textcolor{teal}{\mytextsc{noun}} \hspace{4pt} Tone: LH.
\textcolor{Sepia}{\selectlanguage{english}Large rock.} \zh{大岩石。}  \zh{量词}: \textcolor{darkblue}{\textbf{\ipa{ʈv̩˩}}}  \mytextsc{clf}: \textcolor{darkblue}{\textbf{\ipa{ʈv̩˩}}} 
\lhead{\firstmark}
\rhead{\botmark}

\subsection{\hspace{-0.5cm} {\Large \textcolor{darkblue}{\textbf{\ipa{æ̃˥}}}}\hspace{0.5cm}[\kern2pt{\textcolor{darkblue}{\textbf{\ipa{æ̃˥}}}}\kern2pt]} \hypertarget{\{\string_~\string_T1}{}
\markboth{\textcolor{darkblue}{\textbf{\ipa{æ̃˥}}}}{}
\textcolor{teal}{\mytextsc{noun}} \hspace{4pt} Tone: \#H.
\textcolor{Sepia}{\selectlanguage{english}Brass, copper, bronze.} \zh{铜,包括黄铜、红铜、青铜。}  ¶ \textcolor{darkblue}{\textbf{\ipa{æ̃˧tso˧-æ̃˧mo˩}}} \textcolor{Sepia}{\selectlanguage{english}instruments and objects made of brass} \zh{铜做的工具、物品}  
 ¶ \textcolor{darkblue}{\textbf{\ipa{æ̃˧ lɑ˩-zo˩-ɳɯ˩, | ʂe˧ mɤ˧-lɑ˧˥!}}} \textcolor{Sepia}{\selectlanguage{english}“The man who works copper does not work iron!” These two specialties require different skills: physical strength for working iron; and delicate gestures for working copper. This saying is used to point out that each person has her/his own area of expertise.} \zh{“打铜的人,不打铁!”这两种工作需要不同的能力:打铁需要体力,打铜需要细致。这个谚语意指:每个人有他的专业,不能随便跨越到其它领域去。}  

\lhead{\firstmark}
\rhead{\botmark}

\subsection{\hspace{-0.5cm} {\Large \textcolor{darkblue}{\textbf{\ipa{æ̃˧qæ˩}}} \textsubscript{1}}\hspace{0.5cm}[\kern2pt{\textcolor{darkblue}{\textbf{\ipa{æ̃˧qæ˩}}}}\kern2pt]} \hypertarget{\{\string_~\string_Mq\{\string_B1}{}
\markboth{\textcolor{darkblue}{\textbf{\ipa{æ̃˧qæ˩}}} \textsubscript{1}}{}
\textcolor{teal}{\mytextsc{noun}} \hspace{4pt} Tone: L\#.
\textcolor{Sepia}{\selectlanguage{english}Parrot.} \zh{鹦鹉。}  \zh{量词}: \textcolor{darkblue}{\textbf{\ipa{mi˩}}}  \mytextsc{clf}: \textcolor{darkblue}{\textbf{\ipa{mi˩}}} \textit{See:} \hyperlink{}{\textcolor{darkblue}{\textbf{\ipa{æ̃˧qæ˩}}} \textsubscript{2}} 
\lhead{\firstmark}
\rhead{\botmark}

\subsection{\hspace{-0.5cm} {\Large \textcolor{darkblue}{\textbf{\ipa{æ̃˧qæ˩}}} \textsubscript{2}}\hspace{0.5cm}[\kern2pt{\textcolor{darkblue}{\textbf{\ipa{æ̃˧qæ˩}}}}\kern2pt]} \hypertarget{\{\string_~\string_Mq\{\string_B2}{}
\markboth{\textcolor{darkblue}{\textbf{\ipa{æ̃˧qæ˩}}} \textsubscript{2}}{}
\textcolor{teal}{\mytextsc{adjective}} \hspace{4pt} Tone: L\#.
\textcolor{Sepia}{\selectlanguage{english}Blue-green; literally 'parrot[-coloured]'.} \zh{像鹦鹉的颜色:青色、蓝色、绿色。}  ¶ \textcolor{darkblue}{\textbf{\ipa{æ̃˧qæ˩-ni˩gv̩˩}}} \textcolor{Sepia}{\selectlanguage{english}vivid-coloured, blue-green; literally 'like a parrot', i.e. 'parrot-coloured'} \zh{像鹦鹉的颜色:青、蓝色、绿色}  
 ¶ \textcolor{darkblue}{\textbf{\ipa{[F5] æ̃˧qæ˩-bɑ˩lɑ˩}}} \textcolor{Sepia}{\selectlanguage{english}vivid-coloured, blue-green jacket: literally 'parrot(-coloured) jacket'} \zh{青、蓝色、绿色衣服}  
 ¶ \textcolor{darkblue}{\textbf{\ipa{[F5] æ̃˧qæ˩ni˩\textasciitilde{}æ̃˧qæ˩ni˩gv̩˩}}} \textcolor{Sepia}{\selectlanguage{english}\mytextsc{red;} same meaning: blue-green} \zh{\mytextsc{重叠。同上:青色}}  
\textit{See:} \hyperlink{}{\textcolor{darkblue}{\textbf{\ipa{æ̃˧qæ˩}}} \textsubscript{1}} 
\lhead{\firstmark}
\rhead{\botmark}

\subsection{\hspace{-0.5cm} {\Large \textcolor{darkblue}{\textbf{\ipa{æ̃˧ʂwæ˥}}}}\hspace{0.5cm}[\kern2pt{\textcolor{darkblue}{\textbf{\ipa{æ̃˧ʂwæ˥}}}}\kern2pt]} \hypertarget{\{\string_~\string_Ms`w\{\string_T1}{}
\markboth{\textcolor{darkblue}{\textbf{\ipa{æ̃˧ʂwæ˥}}}}{}
\textcolor{teal}{\mytextsc{noun}} \hspace{4pt} Tone: H\#.
\textcolor{Sepia}{\selectlanguage{english}Rooster.} \zh{公鸡。}  ¶ \textcolor{darkblue}{\textbf{\ipa{æ̃˧ʂwæ˥-æ̃˩mi˩}}} \textcolor{Sepia}{\selectlanguage{english}cock and hen} \zh{公鸡与母鸡}  
 \zh{量词}: \textcolor{darkblue}{\textbf{\ipa{mi˩}}}  \mytextsc{clf}: \textcolor{darkblue}{\textbf{\ipa{mi˩}}} 
\lhead{\firstmark}
\rhead{\botmark}

\subsection{\hspace{-0.5cm} {\Large \textcolor{darkblue}{\textbf{\ipa{æ̃˧tsɯ˥}}}}\hspace{0.5cm}[\kern2pt{\textcolor{darkblue}{\textbf{\ipa{æ̃˧tsɯ˥}}}}\kern2pt]} \hypertarget{\{\string_~\string_MtsM\string_T1}{}
\markboth{\textcolor{darkblue}{\textbf{\ipa{æ̃˧tsɯ˥}}}}{}
\textcolor{teal}{\mytextsc{noun}} \hspace{4pt} Tone: H\#.
\textcolor{Sepia}{\selectlanguage{english}Chick.} \zh{雏鸡、稚鸡。}  \zh{量词}: \textcolor{darkblue}{\textbf{\ipa{ɭɯ˧}}}  \mytextsc{clf}: \textcolor{darkblue}{\textbf{\ipa{ɭɯ˧}}} 
\lhead{\firstmark}
\rhead{\botmark}

\subsection{\hspace{-0.5cm} {\Large \textcolor{darkblue}{\textbf{\ipa{æ̃˧tsɯ˥-kʰɯ˩ʈʂɤ˩-mo˩}}}}\hspace{0.5cm}[\kern2pt{\textcolor{darkblue}{\textbf{\ipa{xxxx non-correspondance entre le nombre de morphèmes et le nombre de tons de morphèmes}}}}\kern2pt]} \hypertarget{\{\string_~\string_MtsM\string_T-k\string_hM\string_Bt`s`7\string_B-mo\string_B1}{}
\markboth{\textcolor{darkblue}{\textbf{\ipa{æ̃˧tsɯ˥-kʰɯ˩ʈʂɤ˩-mo˩}}}}{}
\textcolor{teal}{\mytextsc{noun}} \hspace{4pt} Tone: H\#-.
\textcolor{Sepia}{\selectlanguage{english}“chicken-claw mushroom”: an edible mushroom.} \zh{扫把菌,扫帚菌(一种菌子)。} 
\lhead{\firstmark}
\rhead{\botmark}

\subsection{\hspace{-0.5cm} {\Large \textcolor{darkblue}{\textbf{\ipa{æ̃˧ʈwɤ˩-mv̩˩kʰv̩˩}}}}\hspace{0.5cm}[\kern2pt{\textcolor{darkblue}{\textbf{\ipa{æ̃˧ʈwɤ˩mv̩˩kʰv̩˩}}}}\kern2pt]} \hypertarget{\{\string_~\string_Mt`w7\string_B-mv\string_=\string_Bk\string_hv\string_=\string_B1}{}
\markboth{\textcolor{darkblue}{\textbf{\ipa{æ̃˧ʈwɤ˩-mv̩˩kʰv̩˩}}}}{}
\textcolor{teal}{\mytextsc{adverb(ial)}} \hspace{4pt} Tone: L\#-L.
\textcolor{Sepia}{\selectlanguage{english}Constantly, all the time; literally: '[from] morning [till] evening'.} \zh{一直不停地,从早到晚。直译:‘(从)早上(到)晚上’。} 
\lhead{\firstmark}
\rhead{\botmark}

\subsection{\hspace{-0.5cm} {\Large \textcolor{darkblue}{\textbf{\ipa{æ̃˧-v̩\#˥}}}}\hspace{0.5cm}[\kern2pt{\textcolor{darkblue}{\textbf{\ipa{xxxx non-correspondance entre le nombre de morphèmes et le nombre de tons de morphèmes}}}}\kern2pt]} \hypertarget{\{\string_~\string_M-v\string_=\#\string_T1}{}
\markboth{\textcolor{darkblue}{\textbf{\ipa{æ̃˧-v̩\#˥}}}}{}
\textcolor{teal}{\mytextsc{noun}} \hspace{4pt} Tone: \#H.
\textcolor{Sepia}{\selectlanguage{english}Copper pot.} \zh{铜锅。}  \zh{量词}: \textcolor{darkblue}{\textbf{\ipa{ɭɯ˧}}}  \mytextsc{clf}: \textcolor{darkblue}{\textbf{\ipa{ɭɯ˧}}} 
\lhead{\firstmark}
\rhead{\botmark}

\subsection{\hspace{-0.5cm} {\Large \textcolor{darkblue}{\textbf{\ipa{æ̃˩\textsubscript{a}}}}}\hspace{0.5cm}[\kern2pt{\textcolor{darkblue}{\textbf{\ipa{æ̃˩˥}}}}\kern2pt]} \hypertarget{\{\string_~\string_Ba1}{}
\markboth{\textcolor{darkblue}{\textbf{\ipa{æ̃˩\textsubscript{a}}}}}{}
\textcolor{teal}{\mytextsc{classifier}} \hspace{4pt} Tone: L\textsubscript{a}.
\textcolor{Sepia}{\selectlanguage{english}Classifier for fires.} \zh{量词:火(一团)。}  ¶ \textcolor{darkblue}{\textbf{\ipa{mv̩˧ | ʈʂʰɯ˧-æ̃˥}}} \textcolor{Sepia}{\selectlanguage{english}this fire (tone: H\# / H\$)} \zh{这团火}  

\lhead{\firstmark}
\rhead{\botmark}

\subsection{\hspace{-0.5cm} {\Large \textcolor{darkblue}{\textbf{\ipa{æ̃˩\textsubscript{a}}}} \textsubscript{1}}\hspace{0.5cm}[\kern2pt{\textcolor{darkblue}{\textbf{\ipa{æ̃˩˥}}}}\kern2pt]} \hypertarget{\{\string_~\string_Ba1}{}
\markboth{\textcolor{darkblue}{\textbf{\ipa{æ̃˩\textsubscript{a}}}} \textsubscript{1}}{}
\textcolor{teal}{\mytextsc{verb}} \hspace{4pt} Tone: L\textsubscript{a}.
\textcolor{Sepia}{\selectlanguage{english}To reflect (a mirror reflects light).} \zh{反射、辉映。} 
\lhead{\firstmark}
\rhead{\botmark}

\subsection{\hspace{-0.5cm} {\Large \textcolor{darkblue}{\textbf{\ipa{æ̃˩\textsubscript{a}}}} \textsubscript{2}}\hspace{0.5cm}[\kern2pt{\textcolor{darkblue}{\textbf{\ipa{æ̃˩˥}}}}\kern2pt]} \hypertarget{\{\string_~\string_Ba2}{}
\markboth{\textcolor{darkblue}{\textbf{\ipa{æ̃˩\textsubscript{a}}}} \textsubscript{2}}{}
\textcolor{teal}{\mytextsc{verb}} \hspace{4pt} Tone: L\textsubscript{a}.
\textcolor{Sepia}{\selectlanguage{english}To get stuck.} \zh{堵塞、塞。}  ¶ \textcolor{darkblue}{\textbf{\ipa{ʝi˩mi˩˥ | ɖʐæ˩qʰæ˧-qo˩ æ̃˩!}}} \textcolor{Sepia}{\selectlanguage{english}The cow is stuck in the mud.} \zh{牛陷在泥巴里。}  

\lhead{\firstmark}
\rhead{\botmark}

\subsection{\hspace{-0.5cm} {\Large \textcolor{darkblue}{\textbf{\ipa{æ̃˩bi˩}}}}\hspace{0.5cm}[\kern2pt{\textcolor{darkblue}{\textbf{\ipa{æ̃˩bi˩˥}}}}\kern2pt]} \hypertarget{\{\string_~\string_Bbi\string_B1}{}
\markboth{\textcolor{darkblue}{\textbf{\ipa{æ̃˩bi˩}}}}{}
\textcolor{teal}{\mytextsc{noun}} \hspace{4pt} Tone: L.
\textcolor{Sepia}{\selectlanguage{english}A village just over the river on the Sichuan side of road to Qiansuo.} \zh{从阿拉瓦村到前所路上经过的一个村落。}  ¶ \textcolor{darkblue}{\textbf{\ipa{æ̃˩bi˩-ʁwɤ˥}}} \textcolor{Sepia}{\selectlanguage{english}same meaning: the village of \textcolor{darkblue}{\textbf{\ipa{/æ̃˩bi˩/}}}} \zh{同上}  
 ¶ \textcolor{darkblue}{\textbf{\ipa{æ̃˩bi˩-hĩ˥ ɲi˩!}}} \textcolor{Sepia}{\selectlanguage{english}[(S)he] is from the village of \textcolor{darkblue}{\textbf{\ipa{/æ̃˩bi˩/!}}}} \zh{是\textcolor{darkblue}{\textbf{\ipa{/æ̃˩bi˩/}}}村的人!}  

\lhead{\firstmark}
\rhead{\botmark}

\subsection{\hspace{-0.5cm} {\Large \textcolor{darkblue}{\textbf{\ipa{æ̃˩bv̩˥}}}}\hspace{0.5cm}[\kern2pt{\textcolor{darkblue}{\textbf{\ipa{æ̃˩bv̩˥}}}}\kern2pt]} \hypertarget{\{\string_~\string_Bbv\string_=\string_T1}{}
\markboth{\textcolor{darkblue}{\textbf{\ipa{æ̃˩bv̩˥}}}}{}
\textcolor{teal}{\mytextsc{noun}} \hspace{4pt} Tone: LH.
\textcolor{Sepia}{\selectlanguage{english}Poultry yard.} \zh{鸡圈。}  \zh{量词}: \textcolor{darkblue}{\textbf{\ipa{ɭɯ˧}}}  \mytextsc{clf}: \textcolor{darkblue}{\textbf{\ipa{ɭɯ˧}}} 
\lhead{\firstmark}
\rhead{\botmark}

\subsection{\hspace{-0.5cm} {\Large \textcolor{darkblue}{\textbf{\ipa{æ̃˩-kʰv̩˧˥}}}}\hspace{0.5cm}[\kern2pt{\textcolor{darkblue}{\textbf{\ipa{xxxx non-correspondance entre le nombre de morphèmes et le nombre de tons de morphèmes}}}}\kern2pt]} \hypertarget{\{\string_~\string_B-k\string_hv\string_=\string_M\string_T1}{}
\markboth{\textcolor{darkblue}{\textbf{\ipa{æ̃˩-kʰv̩˧˥}}}}{}
\textcolor{teal}{\mytextsc{noun}} \hspace{4pt} Tone: LM+MH\#.
\textcolor{Sepia}{\selectlanguage{english}Year of the cock.} \zh{鸡年。} 
\lhead{\firstmark}
\rhead{\botmark}

\subsection{\hspace{-0.5cm} {\Large \textcolor{darkblue}{\textbf{\ipa{æ̃˩li˧pʰæ˥}}}}\hspace{0.5cm}[\kern2pt{\textcolor{darkblue}{\textbf{\ipa{æ̃˩li˧pʰæ˥}}}}\kern2pt]} \hypertarget{\{\string_~\string_Bli\string_Mp\string_h\{\string_T1}{}
\markboth{\textcolor{darkblue}{\textbf{\ipa{æ̃˩li˧pʰæ˥}}}}{}
\textcolor{teal}{\mytextsc{noun}} \hspace{4pt} Tone: LM+H\#.
\textcolor{Sepia}{\selectlanguage{english}Mirror.} \zh{镜子。}  \zh{量词}: \textcolor{darkblue}{\textbf{\ipa{pʰæ˧˥}}}  \mytextsc{clf}: \textcolor{darkblue}{\textbf{\ipa{pʰæ˧˥}}} 
\lhead{\firstmark}
\rhead{\botmark}

\subsection{\hspace{-0.5cm} {\Large \textcolor{darkblue}{\textbf{\ipa{æ̃˩ɬi\#˥}}}}\hspace{0.5cm}[\kern2pt{\textcolor{darkblue}{\textbf{\ipa{æ̃˩ɬi˥}}}}\kern2pt]} \hypertarget{\{\string_~\string_BKi\#\string_T1}{}
\markboth{\textcolor{darkblue}{\textbf{\ipa{æ̃˩ɬi\#˥}}}}{}
\textcolor{teal}{\mytextsc{noun}} \hspace{4pt} Tone: LM+\#H.
\textcolor{Sepia}{\selectlanguage{english}Soul.} \zh{灵魂、魂魄。}  Borrowing: Tibetan  bla (older form: brla)
 \zh{量词}: \textcolor{darkblue}{\textbf{\ipa{v̩˧}}}  \mytextsc{clf}: \textcolor{darkblue}{\textbf{\ipa{v̩˧}}} 
\lhead{\firstmark}
\rhead{\botmark}

\subsection{\hspace{-0.5cm} {\Large \textcolor{darkblue}{\textbf{\ipa{æ̃˩mi˧}}}}\hspace{0.5cm}[\kern2pt{\textcolor{darkblue}{\textbf{\ipa{æ̃˩mi˥}}}}\kern2pt]} \hypertarget{\{\string_~\string_Bmi\string_M1}{}
\markboth{\textcolor{darkblue}{\textbf{\ipa{æ̃˩mi˧}}}}{}
\textcolor{teal}{\mytextsc{noun}} \hspace{4pt} Tone: LM.
\textcolor{Sepia}{\selectlanguage{english}Hen.} \zh{母鸡。}  ¶ \textcolor{darkblue}{\textbf{\ipa{æ̃˩mi˧-æ̃˧ʂwæ˥\#}}} \textcolor{Sepia}{\selectlanguage{english}hen and rooster} \zh{母鸡与公鸡}  
 ¶ \textcolor{darkblue}{\textbf{\ipa{æ̃˩mi˧-æ̃˧tsɯ˥\#}}} \textcolor{Sepia}{\selectlanguage{english}hen and chick} \zh{母鸡与稚鸡}  
 \zh{量词}: \textcolor{darkblue}{\textbf{\ipa{mi˩}}}  \mytextsc{clf}: \textcolor{darkblue}{\textbf{\ipa{mi˩}}} 
\lhead{\firstmark}
\rhead{\botmark}

\subsection{\hspace{-0.5cm} {\Large \textcolor{darkblue}{\textbf{\ipa{æ̃˩ʁv̩˩}}}}\hspace{0.5cm}[\kern2pt{\textcolor{darkblue}{\textbf{\ipa{æ̃˩ʁv̩˩˥}}}}\kern2pt]} \hypertarget{\{\string_~\string_BRv\string_=\string_B1}{}
\markboth{\textcolor{darkblue}{\textbf{\ipa{æ̃˩ʁv̩˩}}}}{}
\textcolor{teal}{\mytextsc{noun}} \hspace{4pt} Tone: L.
\textcolor{Sepia}{\selectlanguage{english}Egg.} \zh{蛋。}  ¶ \textcolor{darkblue}{\textbf{\ipa{bæ˧mi˧-æ̃˩ʁv̩˩}}} \textcolor{Sepia}{\selectlanguage{english}cane egg} \zh{鸭子蛋}  
 ¶ \textcolor{darkblue}{\textbf{\ipa{æ̃˩ʁv̩˩ dzɯ˩˥}}} \textcolor{Sepia}{\selectlanguage{english}to eat eggs} \zh{吃蛋}  
 \zh{量词}: \textcolor{darkblue}{\textbf{\ipa{ɭɯ˧}}}  \mytextsc{clf}: \textcolor{darkblue}{\textbf{\ipa{ɭɯ˧}}} 
\lhead{\firstmark}
\rhead{\botmark}

\subsection{\hspace{-0.5cm} {\Large \textcolor{darkblue}{\textbf{\ipa{æ̃˩ʂe˧li˥-mo˩}}}}\hspace{0.5cm}[\kern2pt{\textcolor{darkblue}{\textbf{\ipa{æ̃˩ʂe˧li˥mo˧}}}}\kern2pt]} \hypertarget{\{\string_~\string_Bs`e\string_Mli\string_T-mo\string_B1}{}
\markboth{\textcolor{darkblue}{\textbf{\ipa{æ̃˩ʂe˧li˥-mo˩}}}}{}
\textcolor{teal}{\mytextsc{noun}} \hspace{4pt} Tone: LM+H\#-.
\textcolor{Sepia}{\selectlanguage{english}“Chicken-meat mushroom”: an edible mushroom, \textit{Amanita spissa}.} \zh{麻母鸡菌:一种可以吃的菌子,块鳞灰毒鹅膏菌。} 
\lhead{\firstmark}
\rhead{\botmark}

\subsection{\hspace{-0.5cm} {\Large \textcolor{darkblue}{\textbf{\ipa{æ̃˩ʂe˩}}}}\hspace{0.5cm}[\kern2pt{\textcolor{darkblue}{\textbf{\ipa{æ̃˩ʂe˩˥}}}}\kern2pt]} \hypertarget{\{\string_~\string_Bs`e\string_B1}{}
\markboth{\textcolor{darkblue}{\textbf{\ipa{æ̃˩ʂe˩}}}}{}
\textcolor{teal}{\mytextsc{noun}} \hspace{4pt} Tone: L.
\textcolor{Sepia}{\selectlanguage{english}Muscle.} \zh{肌肉。}  ¶ \textcolor{darkblue}{\textbf{\ipa{æ̃˩ʂe˩ tsʰi˩˥}}} \textcolor{Sepia}{\selectlanguage{english}to run a temperature, to have a fever} \zh{发烧}  
 \zh{量词}: \textcolor{darkblue}{\textbf{\ipa{kʰwɤ˥}}}  \mytextsc{clf}: \textcolor{darkblue}{\textbf{\ipa{kʰwɤ˥}}} 
\lhead{\firstmark}
\rhead{\botmark}

\subsection{\hspace{-0.5cm} {\Large \textcolor{darkblue}{\textbf{\ipa{æ̃˩zɯ˩}}}}\hspace{0.5cm}[\kern2pt{\textcolor{darkblue}{\textbf{\ipa{æ̃˩zɯ˩˥}}}}\kern2pt]} \hypertarget{\{\string_~\string_BzM\string_B1}{}
\markboth{\textcolor{darkblue}{\textbf{\ipa{æ̃˩zɯ˩}}}}{}
\textcolor{teal}{\mytextsc{noun}} \hspace{4pt} Tone: L.
\textcolor{Sepia}{\selectlanguage{english}Agate. Agate of various colours is used in ornamentation. Beads range from the size of a quail egg to that of a chicken's egg.} \zh{玛瑙。}  ¶ \textcolor{darkblue}{\textbf{\ipa{sɯ˧ɻ̍˧-æ̃˩zɯ˩}}} \textcolor{Sepia}{\selectlanguage{english}pearl-shaped agate bead} \zh{珠子形状的玛瑙}  
 ¶ \textcolor{darkblue}{\textbf{\ipa{æ̃˩zɯ˩-ʂo˩\textasciitilde{}ʂo˥}}} \textcolor{Sepia}{\selectlanguage{english}with lots of agate on it (of a piece of clothing)} \zh{(衣服上)都镶嵌着玛瑙}  
 \zh{量词}: \textcolor{darkblue}{\textbf{\ipa{ɭɯ˧}}}  \mytextsc{clf}: \textcolor{darkblue}{\textbf{\ipa{ɭɯ˧}}} 
\lhead{\firstmark}
\rhead{\botmark}

\subsection{\hspace{-0.5cm} {\Large \textcolor{darkblue}{\textbf{\ipa{æ̃˩˥}}}}\hspace{0.5cm}[\kern2pt{\textcolor{darkblue}{\textbf{\ipa{æ̃˩˥}}}}\kern2pt]} \hypertarget{\{\string_~\string_B\string_T1}{}
\markboth{\textcolor{darkblue}{\textbf{\ipa{æ̃˩˥}}}}{}
\textcolor{teal}{\mytextsc{noun}} \hspace{4pt} Tone: LH.
\textcolor{Sepia}{\selectlanguage{english}Soul (monosyllable).} \zh{灵魂。}  \zh{量词}: \textcolor{darkblue}{\textbf{\ipa{v̩˧}}}  \mytextsc{clf}: \textcolor{darkblue}{\textbf{\ipa{v̩˧}}} 
\lhead{\firstmark}
\rhead{\botmark}

\subsection{\hspace{-0.5cm} {\Large \textcolor{darkblue}{\textbf{\ipa{æ̃˩˧}}}}\hspace{0.5cm}[\kern2pt{\textcolor{darkblue}{\textbf{\ipa{æ̃˩˥}}}}\kern2pt]} \hypertarget{\{\string_~\string_B\string_M1}{}
\markboth{\textcolor{darkblue}{\textbf{\ipa{æ̃˩˧}}}}{}
\textcolor{teal}{\mytextsc{noun}} \hspace{4pt} Tone: LM.
\textcolor{Sepia}{\selectlanguage{english}Chicken.} \zh{鸡。}  ¶ \textcolor{darkblue}{\textbf{\ipa{æ̃˩ dzɯ˥-ze˩}}} \textcolor{Sepia}{\selectlanguage{english}...has eaten (a/some) chicken} \zh{吃了鸡}  
 ¶ \textcolor{darkblue}{\textbf{\ipa{æ̃˩ hwæ˧-ze˧}}} \textcolor{Sepia}{\selectlanguage{english}...has bought (a) chicken} \zh{买了鸡}  
 ¶ \textcolor{darkblue}{\textbf{\ipa{æ̃˩˥, | kʰv̩˧, | bo˩˥, | hwɤ˧˥, | ʝi˧, | lɑ˧, | tʰo˧li˧, | mv̩˧gv̩˧, | bv̩˧ʐv̩˧, | ʐwæ˧, | jo˧, | ʑi˩˥}}} \textcolor{Sepia}{\selectlanguage{english}the twelve years of the duodenary cycle} \zh{十二个生肖}  
 ¶ \textcolor{darkblue}{\textbf{\ipa{æ̃˩-mɤ˥}}} \textcolor{Sepia}{\selectlanguage{english}chicken grease, chicken fat} \zh{鸡油}  
 ¶ \textcolor{darkblue}{\textbf{\ipa{æ̃˩-mɤ˥ dzɯ˩}}} \textcolor{Sepia}{\selectlanguage{english}to eat chicken fat} \zh{吃鸡油}  
 \zh{量词}: \textcolor{darkblue}{\textbf{\ipa{mi˩}}}  \mytextsc{clf}: \textcolor{darkblue}{\textbf{\ipa{mi˩}}} 
\lhead{\firstmark}
\rhead{\botmark}

\newpage
\section*{\centering- \textcolor{darkblue}{\textbf{\ipa{b}}} -}
\subsection{\hspace{-0.5cm} {\Large \textcolor{darkblue}{\textbf{\ipa{bɑ˧lɑ˧kʰɯ˧tsʰɤ˧}}}}\hspace{0.5cm}[\kern2pt{\textcolor{darkblue}{\textbf{\ipa{bɑ˩lɑ˩kʰɯ˩tsʰɤ˩˥}}}}\kern2pt]} \hypertarget{bA\string_MlA\string_Mk\string_hM\string_Mts\string_h7\string_M1}{}
\markboth{\textcolor{darkblue}{\textbf{\ipa{bɑ˧lɑ˧kʰɯ˧tsʰɤ˧}}}}{}
\textcolor{teal}{\mytextsc{noun}} \hspace{4pt} Tone: M.
\textcolor{Sepia}{\selectlanguage{english}Spider.} \zh{蜘蛛。}  \zh{量词}: \textcolor{darkblue}{\textbf{\ipa{kʰɯ˩}}}  \mytextsc{clf}: \textcolor{darkblue}{\textbf{\ipa{kʰɯ˩}}} 
\lhead{\firstmark}
\rhead{\botmark}

\subsection{\hspace{-0.5cm} {\Large \textcolor{darkblue}{\textbf{\ipa{bɑ˩lɑ˩}}}}\hspace{0.5cm}[\kern2pt{\textcolor{darkblue}{\textbf{\ipa{bɑ˩lɑ˩˥}}}}\kern2pt]} \hypertarget{bA\string_BlA\string_B1}{}
\markboth{\textcolor{darkblue}{\textbf{\ipa{bɑ˩lɑ˩}}}}{}
\textcolor{teal}{\mytextsc{noun}} \hspace{4pt} Tone: L.
\ding{202} \textcolor{Sepia}{\selectlanguage{english}Jacket, upper outer garment; clothes.} \zh{上衣,衣服。}  ¶ \textcolor{darkblue}{\textbf{\ipa{ɣɯ˩-bɑ˩lɑ˥ (+ɲi˩)}}} \textcolor{Sepia}{\selectlanguage{english}leather jacket} \zh{皮衣}  
 \zh{量词}: \textcolor{darkblue}{\textbf{\ipa{ɭɯ˧}}} \ding{203} \textcolor{Sepia}{\selectlanguage{english}Placenta.} \zh{胎盘、衣胞。}  \zh{量词}: \textcolor{darkblue}{\textbf{\ipa{ɭɯ˧}}}  \mytextsc{clf}: \textcolor{darkblue}{\textbf{\ipa{ɭɯ˧}}} \textcolor{darkblue}{\textbf{\ipa{ɭɯ˧}}} 
\lhead{\firstmark}
\rhead{\botmark}

\subsection{\hspace{-0.5cm} {\Large \textcolor{darkblue}{\textbf{\ipa{bɑ˩˥}}}}\hspace{0.5cm}[\kern2pt{\textcolor{darkblue}{\textbf{\ipa{bɑ˩˥}}}}\kern2pt]} \hypertarget{bA\string_B\string_T1}{}
\markboth{\textcolor{darkblue}{\textbf{\ipa{bɑ˩˥}}}}{}
\textcolor{teal}{\mytextsc{discourse}} \textcolor{teal}{\mytextsc{particle}} \hspace{4pt} Tone: L?.
\textcolor{Sepia}{\selectlanguage{english}Affirmative final particle; comparable to question-tag in English.} \zh{句尾助词,表示肯定:“……是吧。”。} 
\lhead{\firstmark}
\rhead{\botmark}

\subsection{\hspace{-0.5cm} {\Large \textcolor{darkblue}{\textbf{\ipa{bæ˧}}} \textsubscript{1}}\hspace{0.5cm}[\kern2pt{\textcolor{darkblue}{\textbf{\ipa{bæ˥}}}}\kern2pt]} \hypertarget{b\{\string_M1}{}
\markboth{\textcolor{darkblue}{\textbf{\ipa{bæ˧}}} \textsubscript{1}}{}
\textcolor{teal}{\mytextsc{adjective}} \hspace{4pt} Tone: M.
\textcolor{Sepia}{\selectlanguage{english}Stupid, idiot.} \zh{傻、笨、蠢。}  ¶ \textcolor{darkblue}{\textbf{\ipa{bæ˧-hĩ˧}}} \textcolor{Sepia}{\selectlanguage{english}\mytextsc{rel}} \zh{傻的}  

\lhead{\firstmark}
\rhead{\botmark}

\subsection{\hspace{-0.5cm} {\Large \textcolor{darkblue}{\textbf{\ipa{bæ˧}}} \textsubscript{2}}\hspace{0.5cm}[\kern2pt{\textcolor{darkblue}{\textbf{\ipa{bæ˥}}}}\kern2pt]} \hypertarget{b\{\string_M2}{}
\markboth{\textcolor{darkblue}{\textbf{\ipa{bæ˧}}} \textsubscript{2}}{}
\textcolor{teal}{\mytextsc{verb}} \hspace{4pt} Tone: M intrans.
\textcolor{Sepia}{\selectlanguage{english}To let go, to forget about something (as when providing consolation to someone who has failed, telling her/him not to feel desperate).} \zh{放弃。}  ¶ \textcolor{darkblue}{\textbf{\ipa{no˧ | bæ˧-ze˩!}}} \textcolor{Sepia}{\selectlanguage{english}Forget it! (Consolation to someone who has tried and failed.)} \zh{你算了吧!(感叹)}  
 ¶ \textcolor{darkblue}{\textbf{\ipa{bæ˧-ze˩ mæ˩!}}} \textcolor{Sepia}{\selectlanguage{english}Forget it! (Consolation to someone who has tried and failed.)} \zh{算了嘛!(感叹)}  

\lhead{\firstmark}
\rhead{\botmark}

\subsection{\hspace{-0.5cm} {\Large \textcolor{darkblue}{\textbf{\ipa{bæ˧\textsubscript{a}}}}}\hspace{0.5cm}[\kern2pt{\textcolor{darkblue}{\textbf{\ipa{bæ˥}}}}\kern2pt]} \hypertarget{b\{\string_Ma1}{}
\markboth{\textcolor{darkblue}{\textbf{\ipa{bæ˧\textsubscript{a}}}}}{}
\textcolor{teal}{\mytextsc{classifier}} \hspace{4pt} Tone: M\textsubscript{a}.
\textcolor{Sepia}{\selectlanguage{english}Classifier for sorts of things; used in statements of identity: “it is the same”.} \zh{量词:东西(一样)。}  ¶ \textcolor{darkblue}{\textbf{\ipa{ɖɯ˧-bæ˧-lɑ˧ ɲi˥!}}} \textcolor{Sepia}{\selectlanguage{english}It's the same!} \zh{是一样的!}  
 ¶ \textcolor{darkblue}{\textbf{\ipa{ʝi˧kʰv̩˥-dʑo˩, | ɲi˧-bæ˧ | ʐwɤ˩-tʰɑ˩˥! | ʝi˧kʰv̩˥-dʑo˩, | ɖɯ˧-bæ˧-lɑ˧ ʐwɤ˧-tʰɑ˥!}}} \textcolor{Sepia}{\selectlanguage{english}Some (phrases/combinations between words) can be said two different ways; whereas others can only be said in one way / only have one possible realization! (Context: the investigation bears on tonal variants for phrases, such as numeral-plus-classifier phrases; the consultant confirms that some combinations admit two variants, whereas others only have one possible tone pattern.)} \zh{有些(词组)有两种说法,有些只有一种说法!(情景:讨论的是一些有两种不同变调发音的词组,发音合作人确定:确实有些有两种不同的变调,而有些只有一种声调模型。)}  
 ¶ \textcolor{darkblue}{\textbf{\ipa{ɲi˧-bæ˧-ɳɯ˧ | ɖɯ˧-bæ˧ ʝi˧}}} \textcolor{Sepia}{\selectlanguage{english}to confuse two things, e.g. to confuse two sounds (phonemes), and to write them in the same way, missing their opposition} \zh{两者混淆,例如把两个音(两个不同的音位)写成一样,混淆两者}  

\lhead{\firstmark}
\rhead{\botmark}

\subsection{\hspace{-0.5cm} {\Large \textcolor{darkblue}{\textbf{\ipa{bæ˧bv̩˥}}}}\hspace{0.5cm}[\kern2pt{\textcolor{darkblue}{\textbf{\ipa{bæ˧bv̩˥}}}}\kern2pt]} \hypertarget{b\{\string_Mbv\string_=\string_T1}{}
\markboth{\textcolor{darkblue}{\textbf{\ipa{bæ˧bv̩˥}}}}{}
\textcolor{teal}{\mytextsc{noun}} \hspace{4pt} Tone: H\#.
\textcolor{Sepia}{\selectlanguage{english}Piglet.} \zh{猪崽。}  ¶ \textcolor{darkblue}{\textbf{\ipa{bæ˧bv̩˥-zo˩}}} \textcolor{Sepia}{\selectlanguage{english}same meaning: piglet} \zh{猪崽}  
 \zh{量词}: \textcolor{darkblue}{\textbf{\ipa{ɭɯ˧}}}  \mytextsc{clf}: \textcolor{darkblue}{\textbf{\ipa{ɭɯ˧}}} 
\lhead{\firstmark}
\rhead{\botmark}

\subsection{\hspace{-0.5cm} {\Large \textcolor{darkblue}{\textbf{\ipa{bæ˧ɖæ˧}}}}\hspace{0.5cm}[\kern2pt{\textcolor{darkblue}{\textbf{\ipa{bæ˧ɖæ˧}}}}\kern2pt]} \hypertarget{b\{\string_Md`\{\string_M1}{}
\markboth{\textcolor{darkblue}{\textbf{\ipa{bæ˧ɖæ˧}}}}{}
\textcolor{teal}{\mytextsc{noun}} \hspace{4pt} Tone: M.
\textcolor{Sepia}{\selectlanguage{english}Small rope, string.} \zh{细绳。}  \zh{量词}: \textcolor{darkblue}{\textbf{\ipa{kʰɯ˩}}}  \mytextsc{clf}: \textcolor{darkblue}{\textbf{\ipa{kʰɯ˩}}} 
\lhead{\firstmark}
\rhead{\botmark}

\subsection{\hspace{-0.5cm} {\Large \textcolor{darkblue}{\textbf{\ipa{bæ˧mi˧}}} \textsubscript{1}}\hspace{0.5cm}[\kern2pt{\textcolor{darkblue}{\textbf{\ipa{xxxx non-correspondance entre le nombre de morphèmes et le nombre de tons de morphèmes}}}}\kern2pt]} \hypertarget{b\{\string_Mmi\string_M1}{}
\markboth{\textcolor{darkblue}{\textbf{\ipa{bæ˧mi˧}}} \textsubscript{1}}{}
\textcolor{teal}{\mytextsc{noun}} \hspace{4pt} Tone: M.
\ding{202} \textcolor{Sepia}{\selectlanguage{english}Duck (without a specification of gender).} \zh{鸭子。}  \zh{量词}: \textcolor{darkblue}{\textbf{\ipa{mi˩}}} \ding{203} \textcolor{Sepia}{\selectlanguage{english}Female duck.} \zh{母鸭子。}  ¶ \textcolor{darkblue}{\textbf{\ipa{bæ˧mi˧-bæ˧pʰv̩\#˥}}} \textcolor{Sepia}{\selectlanguage{english}female duck and male duck} \zh{母鸭子与公鸭子}  
 ¶ \textcolor{darkblue}{\textbf{\ipa{bæ˧mi˧-bæ˧zo\#˥}}} \textcolor{Sepia}{\selectlanguage{english}female duck and duckling} \zh{母鸭与小鸭子}  
 \mytextsc{clf}: \textcolor{darkblue}{\textbf{\ipa{mi˩}}} 
\lhead{\firstmark}
\rhead{\botmark}

\subsection{\hspace{-0.5cm} {\Large \textcolor{darkblue}{\textbf{\ipa{bæ˧mi˧}}} \textsubscript{2}}\hspace{0.5cm}[\kern2pt{\textcolor{darkblue}{\textbf{\ipa{bæ˧mi˧}}}}\kern2pt]} \hypertarget{b\{\string_Mmi\string_M2}{}
\markboth{\textcolor{darkblue}{\textbf{\ipa{bæ˧mi˧}}} \textsubscript{2}}{}
\textcolor{teal}{\mytextsc{noun}} \hspace{4pt} Tone: M.
\textcolor{Sepia}{\selectlanguage{english}Thick rope.} \zh{粗绳索。}  \zh{量词}: \textcolor{darkblue}{\textbf{\ipa{kʰɯ˩}}}  \mytextsc{clf}: \textcolor{darkblue}{\textbf{\ipa{kʰɯ˩}}} 
\lhead{\firstmark}
\rhead{\botmark}

\subsection{\hspace{-0.5cm} {\Large \textcolor{darkblue}{\textbf{\ipa{bæ˧mi˧-pʰv̩\#˥}}}}\hspace{0.5cm}[\kern2pt{\textcolor{darkblue}{\textbf{\ipa{xxxx non-correspondance entre le nombre de morphèmes et le nombre de tons de morphèmes}}}}\kern2pt]} \hypertarget{b\{\string_Mmi\string_M-p\string_hv\string_=\#\string_T1}{}
\markboth{\textcolor{darkblue}{\textbf{\ipa{bæ˧mi˧-pʰv̩\#˥}}}}{}
\textcolor{teal}{\mytextsc{noun}} \hspace{4pt} Tone: \#H.
\textcolor{Sepia}{\selectlanguage{english}Male duck.} \zh{公鸭子。}  ¶ \textcolor{darkblue}{\textbf{\ipa{bæ˧mi˧-pʰv̩˧ tʰv̩˧-mi˧˥}}} \textcolor{Sepia}{\selectlanguage{english}\mytextsc{n}+\mytextsc{dem}+\mytextsc{clf}} \zh{这只公鸭子}  
 \zh{量词}: \textcolor{darkblue}{\textbf{\ipa{mi˩}}}  \mytextsc{clf}: \textcolor{darkblue}{\textbf{\ipa{mi˩}}} \textit{See:} \hyperlink{}{\textcolor{darkblue}{\textbf{\ipa{bæ˧pʰv̩\#˥}}}} 
\lhead{\firstmark}
\rhead{\botmark}

\subsection{\hspace{-0.5cm} {\Large \textcolor{darkblue}{\textbf{\ipa{bæ˧pʰv̩\#˥}}}}\hspace{0.5cm}[\kern2pt{\textcolor{darkblue}{\textbf{\ipa{bæ˩pʰv̩˥}}}}\kern2pt]} \hypertarget{b\{\string_Mp\string_hv\string_=\#\string_T1}{}
\markboth{\textcolor{darkblue}{\textbf{\ipa{bæ˧pʰv̩\#˥}}}}{}
\textcolor{teal}{\mytextsc{noun}} \hspace{4pt} Tone: \#H.
\textcolor{Sepia}{\selectlanguage{english}Male duck.} \zh{公鸭子。}  ¶ \textcolor{darkblue}{\textbf{\ipa{bæ˧pʰv̩˧ tʰv̩˧-mi˧˥ / bæ˧pʰv̩˧ tʰv̩˧-mi˥\#}}} \textcolor{Sepia}{\selectlanguage{english}\mytextsc{n}+\mytextsc{dem}+\mytextsc{clf}} \zh{这个公鸭子}  
 ¶ \textcolor{darkblue}{\textbf{\ipa{bæ˧pʰv̩˧-bæ˧mi\#˥}}} \textcolor{Sepia}{\selectlanguage{english}male duck and female duck} \zh{公鸭子与母鸭子}  
 \zh{量词}: \textcolor{darkblue}{\textbf{\ipa{mi˩}}}  \mytextsc{clf}: \textcolor{darkblue}{\textbf{\ipa{mi˩}}} \textit{See:} \hyperlink{}{\textcolor{darkblue}{\textbf{\ipa{bæ˧mi˧-pʰv̩\#˥}}}} 
\lhead{\firstmark}
\rhead{\botmark}

\subsection{\hspace{-0.5cm} {\Large \textcolor{darkblue}{\textbf{\ipa{bæ˧ʁwɤ˧}}}}\hspace{0.5cm}[\kern2pt{\textcolor{darkblue}{\textbf{\ipa{bæ˩ʁwɤ˩˥}}}}\kern2pt]} \hypertarget{b\{\string_MRw7\string_M1}{}
\markboth{\textcolor{darkblue}{\textbf{\ipa{bæ˧ʁwɤ˧}}}}{}
\textcolor{teal}{\mytextsc{noun}} \hspace{4pt} Tone: M.
\textcolor{Sepia}{\selectlanguage{english}A village close to the Hot Springs.} \zh{温泉乡的一个村落。}  ¶ \textcolor{darkblue}{\textbf{\ipa{bæ˧ʁwɤ˧-ʁwɤ˧}}} \textcolor{Sepia}{\selectlanguage{english}same meaning: the village of \textcolor{darkblue}{\textbf{\ipa{/bæ˧ʁwɤ˧/}}}} \zh{同上:\textcolor{darkblue}{\textbf{\ipa{/bæ˧ʁwɤ˧/}}}村}  
 ¶ \textcolor{darkblue}{\textbf{\ipa{ə˧go˧-ʁwɤ˧, | ʁwɤ˧lɑ˩-bi˩, | bæ˧ʁwɤ˧, | tʰo˧tsʰe\#˥, | pi˧tsʰe˩-di˩, | pɤ˧dʑɤ˩-di˩, | ʁwɤ˧tv̩˧}}} \textcolor{Sepia}{\selectlanguage{english}Villages that one encounters as one leaves the plain of Yongning (away from the Lake); the first two are perceived as villages with a high proportion of Na members, and the third as a mostly Na village, whereas the next ones are Pumi (Prinmi).} \zh{永宁背向泸沽湖方向经过的村落。前两个村落拥有相当大的摩梭人口比例,第三个村落是摩梭村,最后一个是普米村。}  
 ¶ \textcolor{darkblue}{\textbf{\ipa{bæ˧ʁwɤ˧: | nɑ˩˥!}}} \textcolor{Sepia}{\selectlanguage{english}\textcolor{darkblue}{\textbf{\ipa{/bæ˧ʁwɤ˧/}}} is a Na village!} \zh{\textcolor{darkblue}{\textbf{\ipa{/bæ˧ʁwɤ˧/}}}是一个摩梭人村落!}  

\lhead{\firstmark}
\rhead{\botmark}

\subsection{\hspace{-0.5cm} {\Large \textcolor{darkblue}{\textbf{\ipa{bæ˧zo\#˥}}}}\hspace{0.5cm}[\kern2pt{\textcolor{darkblue}{\textbf{\ipa{bæ˩zo˩˥}}}}\kern2pt]} \hypertarget{b\{\string_Mzo\#\string_T1}{}
\markboth{\textcolor{darkblue}{\textbf{\ipa{bæ˧zo\#˥}}}}{}
\textcolor{teal}{\mytextsc{noun}} \hspace{4pt} Tone: \#H.
\textcolor{Sepia}{\selectlanguage{english}Duckling.} \zh{小鸭子。}  ¶ \textcolor{darkblue}{\textbf{\ipa{bæ˧zo˧ tʰv̩˧-ɭɯ\#˥}}} \textcolor{Sepia}{\selectlanguage{english}\mytextsc{n}+\mytextsc{dem}+\mytextsc{clf}} \zh{这只小鸭子}  
 ¶ \textcolor{darkblue}{\textbf{\ipa{bæ˧zo˧-bæ˧mi\#˥}}} \textcolor{Sepia}{\selectlanguage{english}duckling and female duck} \zh{小鸭子与母鸭}  
 \zh{量词}: \textcolor{darkblue}{\textbf{\ipa{ɭɯ˧}}}  \mytextsc{clf}: \textcolor{darkblue}{\textbf{\ipa{ɭɯ˧}}} 
\lhead{\firstmark}
\rhead{\botmark}

\subsection{\hspace{-0.5cm} {\Large \textcolor{darkblue}{\textbf{\ipa{bæ˩}}} \textsubscript{1}}\hspace{0.5cm}[\kern2pt{\textcolor{darkblue}{\textbf{\ipa{bæ˥}}}}\kern2pt]} \hypertarget{b\{\string_B1}{}
\markboth{\textcolor{darkblue}{\textbf{\ipa{bæ˩}}} \textsubscript{1}}{}
\textcolor{teal}{\mytextsc{noun}} \hspace{4pt} Tone: L.
\textcolor{Sepia}{\selectlanguage{english}Rope.} \zh{绳子。}  ¶ \textcolor{darkblue}{\textbf{\ipa{bæ˩ ʈʂʰɯ˩-kʰɯ˥}}} \textcolor{Sepia}{\selectlanguage{english}\mytextsc{n}+\mytextsc{dem}+\mytextsc{clf}} \zh{这条绳子}  
 \zh{量词}: \textcolor{darkblue}{\textbf{\ipa{ʈʰɤ˥}}} \textcolor{darkblue}{\textbf{\ipa{ɖæ˩}}} \textcolor{darkblue}{\textbf{\ipa{kʰɯ˩}}}  \mytextsc{clf}: \textcolor{darkblue}{\textbf{\ipa{ʈʰɤ˥}}} \textcolor{darkblue}{\textbf{\ipa{ɖæ˩}}} \textcolor{darkblue}{\textbf{\ipa{kʰɯ˩}}} 
\lhead{\firstmark}
\rhead{\botmark}

\subsection{\hspace{-0.5cm} {\Large \textcolor{darkblue}{\textbf{\ipa{bæ˩}}} \textsubscript{2}}\hspace{0.5cm}[\kern2pt{\textcolor{darkblue}{\textbf{\ipa{bæ˩˥}}}}\kern2pt]} \hypertarget{b\{\string_B2}{}
\markboth{\textcolor{darkblue}{\textbf{\ipa{bæ˩}}} \textsubscript{2}}{}
\textcolor{teal}{\mytextsc{verb}} \hspace{4pt} Tone: L.
\textcolor{Sepia}{\selectlanguage{english}To fester (with pus), to suppurate, to be purulent.} \zh{脓。}  ¶ \textcolor{darkblue}{\textbf{\ipa{bæ˩ bæ˧-ze˩}}} \textcolor{Sepia}{\selectlanguage{english}the wound suppurates, the wound is pussy} \zh{伤口在化脓}  
 ¶ \textcolor{darkblue}{\textbf{\ipa{bæ˩˥ | le˧-bæ˩-ze˩}}} \textcolor{Sepia}{\selectlanguage{english}the wound suppurates, the wound is pussy} \zh{伤口在化脓}  
\textit{See:} \hyperlink{}{\textcolor{darkblue}{\textbf{\ipa{bæ˩˥}}}} 
\lhead{\firstmark}
\rhead{\botmark}

\subsection{\hspace{-0.5cm} {\Large \textcolor{darkblue}{\textbf{\ipa{bæ˩\textsubscript{a}}}} \textsubscript{1}}\hspace{0.5cm}[\kern2pt{\textcolor{darkblue}{\textbf{\ipa{bæ˩˥}}}}\kern2pt]} \hypertarget{b\{\string_Ba1}{}
\markboth{\textcolor{darkblue}{\textbf{\ipa{bæ˩\textsubscript{a}}}} \textsubscript{1}}{}
\textcolor{teal}{\mytextsc{verb}} \hspace{4pt} Tone: L\textsubscript{a}.
\textcolor{Sepia}{\selectlanguage{english}To sweep, to clean up.} \zh{扫。}  ¶ \textcolor{darkblue}{\textbf{\ipa{ɖæ˩ bæ˧}}} \textcolor{Sepia}{\selectlanguage{english}to sweep the dust, to sweep the floor} \zh{扫地}  
 ¶ \textcolor{darkblue}{\textbf{\ipa{le˧-bæ˧\textasciitilde{}bæ˥}}} \textcolor{Sepia}{\selectlanguage{english}\mytextsc{accomp} \mytextsc{red}} \zh{扫一扫}  
 ¶ \textcolor{darkblue}{\textbf{\ipa{ɖʐɤ˩ bæ˩˥}}} \textcolor{Sepia}{\selectlanguage{english}to sweep the stairs} \zh{扫楼梯}  
 ¶ \textcolor{darkblue}{\textbf{\ipa{njɤ˧ | ɖʐɤ˩ bæ˩-zo˩-ho˥.}}} \textcolor{Sepia}{\selectlanguage{english}I have to sweep the stairs!} \zh{我要扫楼梯了!}  
 ¶ \textcolor{darkblue}{\textbf{\ipa{gi˩ bæ˩˥}}} \textcolor{Sepia}{\selectlanguage{english}to sweep the granary} \zh{扫仓廪}  
 ¶ \textcolor{darkblue}{\textbf{\ipa{njɤ˧ | gi˩ bæ˩-zo˩-ho˥.}}} \textcolor{Sepia}{\selectlanguage{english}I have to sweep the granary!} \zh{我要扫仓廪了!}  

\lhead{\firstmark}
\rhead{\botmark}

\subsection{\hspace{-0.5cm} {\Large \textcolor{darkblue}{\textbf{\ipa{bæ˩\textsubscript{a}}}} \textsubscript{2}}\hspace{0.5cm}[\kern2pt{\textcolor{darkblue}{\textbf{\ipa{bæ˩˥}}}}\kern2pt]} \hypertarget{b\{\string_Ba2}{}
\markboth{\textcolor{darkblue}{\textbf{\ipa{bæ˩\textsubscript{a}}}} \textsubscript{2}}{}
\textcolor{teal}{\mytextsc{verb}} \hspace{4pt} Tone: L\textsubscript{a}.
\textcolor{Sepia}{\selectlanguage{english}To bloom.} \zh{开花。}  ¶ \textcolor{darkblue}{\textbf{\ipa{bæ˩bæ˩ bæ˥-ze˩}}} \textcolor{Sepia}{\selectlanguage{english}The flower has bloomed.} \zh{花开了。}  

\lhead{\firstmark}
\rhead{\botmark}

\subsection{\hspace{-0.5cm} {\Large \textcolor{darkblue}{\textbf{\ipa{bæ˩\textsubscript{a}}}} \textsubscript{3}}\hspace{0.5cm}[\kern2pt{\textcolor{darkblue}{\textbf{\ipa{bæ˩˥}}}}\kern2pt]} \hypertarget{b\{\string_Ba3}{}
\markboth{\textcolor{darkblue}{\textbf{\ipa{bæ˩\textsubscript{a}}}} \textsubscript{3}}{}
\textcolor{teal}{\mytextsc{classifier}} \hspace{4pt} Tone: L\textsubscript{a}.
\textcolor{Sepia}{\selectlanguage{english}Self-classifier for flowers.} \zh{量词:花(一朵)。}  ¶ \textcolor{darkblue}{\textbf{\ipa{tʰv̩˧-bæ˥}}} \textcolor{Sepia}{\selectlanguage{english}\mytextsc{dem} \string_ (tone: H\# / H\$)} \zh{\mytextsc{指示代词} \string_ :这朵(花)}  

\lhead{\firstmark}
\rhead{\botmark}

\subsection{\hspace{-0.5cm} {\Large \textcolor{darkblue}{\textbf{\ipa{bæ˩bæ˩}}} \textsubscript{1}}\hspace{0.5cm}[\kern2pt{\textcolor{darkblue}{\textbf{\ipa{bæ˩bæ˩˥}}}}\kern2pt]} \hypertarget{b\{\string_Bb\{\string_B1}{}
\markboth{\textcolor{darkblue}{\textbf{\ipa{bæ˩bæ˩}}} \textsubscript{1}}{}
\textcolor{teal}{\mytextsc{noun}} \hspace{4pt} Tone: L.
\textcolor{Sepia}{\selectlanguage{english}Flower.} \zh{花。}  \zh{量词}: \textcolor{darkblue}{\textbf{\ipa{bæ˩}}}  \mytextsc{clf}: \textcolor{darkblue}{\textbf{\ipa{bæ˩}}} \textit{See:} \hyperlink{}{\textcolor{darkblue}{\textbf{\ipa{bæ˩bæ˩}}} \textsubscript{2}} 
\lhead{\firstmark}
\rhead{\botmark}

\subsection{\hspace{-0.5cm} {\Large \textcolor{darkblue}{\textbf{\ipa{bæ˩bæ˩}}} \textsubscript{2}}\hspace{0.5cm}[\kern2pt{\textcolor{darkblue}{\textbf{\ipa{bæ˩bæ˩˥}}}}\kern2pt]} \hypertarget{b\{\string_Bb\{\string_B2}{}
\markboth{\textcolor{darkblue}{\textbf{\ipa{bæ˩bæ˩}}} \textsubscript{2}}{}
\textcolor{teal}{\mytextsc{adjective}} \hspace{4pt} Tone: L.
\textcolor{Sepia}{\selectlanguage{english}Spotted.} \zh{花的(蛋、石头、鸟)。}  ¶ \textcolor{darkblue}{\textbf{\ipa{bæ˩bæ˩ tʰi˩-di˥}}} \textcolor{Sepia}{\selectlanguage{english}same meaning: spotted (e.g. an egg, a bird, a stone)} \zh{花的,有花纹}  
\textit{See:} \hyperlink{}{\textcolor{darkblue}{\textbf{\ipa{bæ˩bæ˩}}} \textsubscript{1}} 
\lhead{\firstmark}
\rhead{\botmark}

\subsection{\hspace{-0.5cm} {\Large \textcolor{darkblue}{\textbf{\ipa{bæ˩dʑɯ˥}}}}\hspace{0.5cm}[\kern2pt{\textcolor{darkblue}{\textbf{\ipa{bæ˩dʑɯ˥}}}}\kern2pt]} \hypertarget{b\{\string_Bdz£M\string_T1}{}
\markboth{\textcolor{darkblue}{\textbf{\ipa{bæ˩dʑɯ˥}}}}{}
\textcolor{teal}{\mytextsc{noun}} \hspace{4pt} Tone: LH.
\textcolor{Sepia}{\selectlanguage{english}Crops, harvest.} \zh{庄稼。}  ¶ \textcolor{darkblue}{\textbf{\ipa{bæ˩dʑɯ˥ | mɤ˧-dʑɤ˩!}}} \textcolor{Sepia}{\selectlanguage{english}The harvest is not good!} \zh{收成不好!}  
 ¶ \textcolor{darkblue}{\textbf{\ipa{bæ˩dʑɯ˧ | tv̩˧-bæ˩ le˩-mv̩˩-kʰɯ˩!}}} \textcolor{Sepia}{\selectlanguage{english}May a thousand crops come to maturity! (A blessing to elders, used for instance during the rite of coming of age)} \zh{祝:一千棵庄稼成熟!(成年礼、过年等节庆时的祝福用语,晚辈对长辈的祝福)}  
 \zh{量词}: \textcolor{darkblue}{\textbf{\ipa{bæ˩}}}  \mytextsc{clf}: \textcolor{darkblue}{\textbf{\ipa{bæ˩}}} 
\lhead{\firstmark}
\rhead{\botmark}

\subsection{\hspace{-0.5cm} {\Large \textcolor{darkblue}{\textbf{\ipa{bæ˩-lɑ˩\textasciitilde{}lɑ˥}}}}\hspace{0.5cm}[\kern2pt{\textcolor{darkblue}{\textbf{\ipa{xxxx non-correspondance entre le nombre de morphèmes et le nombre de tons de morphèmes}}}}\kern2pt]} \hypertarget{b\{\string_B-lA\string_B~lA\string_T1}{}
\markboth{\textcolor{darkblue}{\textbf{\ipa{bæ˩-lɑ˩\textasciitilde{}lɑ˥}}}}{}
\textcolor{teal}{\mytextsc{adjective}} \hspace{4pt} Tone: L.
\textcolor{Sepia}{\selectlanguage{english}Soft, weak, pliant.} \zh{软,柔软、软塌塌。} 
\lhead{\firstmark}
\rhead{\botmark}

\subsection{\hspace{-0.5cm} {\Large \textcolor{darkblue}{\textbf{\ipa{bæ˩-ljɤ˧\textasciitilde{}ljɤ˧}}}}\hspace{0.5cm}[\kern2pt{\textcolor{darkblue}{\textbf{\ipa{xxxx non-correspondance entre le nombre de morphèmes et le nombre de tons de morphèmes}}}}\kern2pt]} \hypertarget{b\{\string_B-lj7\string_M~lj7\string_M1}{}
\markboth{\textcolor{darkblue}{\textbf{\ipa{bæ˩-ljɤ˧\textasciitilde{}ljɤ˧}}}}{}
\textcolor{teal}{\mytextsc{noun}} \hspace{4pt} Tone: L-.
\textcolor{Sepia}{\selectlanguage{english}China fir cone.} \zh{杉树果。}  \zh{量词}: \textcolor{darkblue}{\textbf{\ipa{ɭɯ˧}}}  \mytextsc{clf}: \textcolor{darkblue}{\textbf{\ipa{ɭɯ˧}}} 
\lhead{\firstmark}
\rhead{\botmark}

\subsection{\hspace{-0.5cm} {\Large \textcolor{darkblue}{\textbf{\ipa{bæ˩pʰv̩˥}}}}\hspace{0.5cm}[\kern2pt{\textcolor{darkblue}{\textbf{\ipa{bæ˧pʰv̩˧}}}}\kern2pt]} \hypertarget{b\{\string_Bp\string_hv\string_=\string_T1}{}
\markboth{\textcolor{darkblue}{\textbf{\ipa{bæ˩pʰv̩˥}}}}{}
\textcolor{teal}{\mytextsc{noun}} \hspace{4pt} Tone: L+H\#.
\textcolor{Sepia}{\selectlanguage{english}Crowndaisy chrysanthemum, \textit{Glebionis coronaria}.} \zh{茼蒿。}  ¶ \textcolor{darkblue}{\textbf{\ipa{bæ˩pʰv̩˥-bv̩˩ | bæ˩bæ˩˥}}} \textcolor{Sepia}{\selectlanguage{english}the flower of crowndaisy chrysanthemum} \zh{茼蒿的顶花}  
 ¶ \textcolor{darkblue}{\textbf{\ipa{bæ˩pʰv̩˥-bæ˩bæ˩}}} \textcolor{Sepia}{\selectlanguage{english}crowndaisy chrysanthemum flower} \zh{茼蒿顶花}  
 \zh{量词}: \textcolor{darkblue}{\textbf{\ipa{po˧}}}  \mytextsc{clf}: \textcolor{darkblue}{\textbf{\ipa{po˧}}} 
\lhead{\firstmark}
\rhead{\botmark}

\subsection{\hspace{-0.5cm} {\Large \textcolor{darkblue}{\textbf{\ipa{bæ˩-ʁwæ˩\textasciitilde{}ʁwæ˥}}}}\hspace{0.5cm}[\kern2pt{\textcolor{darkblue}{\textbf{\ipa{xxxx non-correspondance entre le nombre de morphèmes et le nombre de tons de morphèmes}}}}\kern2pt]} \hypertarget{b\{\string_B-Rw\{\string_B~Rw\{\string_T1}{}
\markboth{\textcolor{darkblue}{\textbf{\ipa{bæ˩-ʁwæ˩\textasciitilde{}ʁwæ˥}}}}{}
\textcolor{teal}{\mytextsc{adjective}} \hspace{4pt} Tone: L.
\textcolor{Sepia}{\selectlanguage{english}Loose, slack, lax.} \zh{松。}  ¶ \textcolor{darkblue}{\textbf{\ipa{ʈʂʰɯ˧ | ɖwæ˧˥ | bæ˩ʁwæ˩\textasciitilde{}ʁwæ˥-ʝi˩!}}} \textcolor{Sepia}{\selectlanguage{english}It's loose! / It's not well-fastened! (About a load on a mule's back)} \zh{(驮在马上面的货物没系好)松动了!}  

\lhead{\firstmark}
\rhead{\botmark}

\subsection{\hspace{-0.5cm} {\Large \textcolor{darkblue}{\textbf{\ipa{bæ˩ʈʂo˥}}}}\hspace{0.5cm}[\kern2pt{\textcolor{darkblue}{\textbf{\ipa{bæ˧ʈʂo˧}}}}\kern2pt]} \hypertarget{b\{\string_Bt`s`o\string_T1}{}
\markboth{\textcolor{darkblue}{\textbf{\ipa{bæ˩ʈʂo˥}}}}{}
\textcolor{teal}{\mytextsc{noun}} \hspace{4pt} Tone: LH.
\textcolor{Sepia}{\selectlanguage{english}Broom.} \zh{扫帚。}  \zh{量词}: \textcolor{darkblue}{\textbf{\ipa{nɑ˧}}}  \mytextsc{clf}: \textcolor{darkblue}{\textbf{\ipa{nɑ˧}}} 
\lhead{\firstmark}
\rhead{\botmark}

\subsection{\hspace{-0.5cm} {\Large \textcolor{darkblue}{\textbf{\ipa{bæ˩ʈʂwæ˩}}}}\hspace{0.5cm}[\kern2pt{\textcolor{darkblue}{\textbf{\ipa{bæ˩ʈʂwæ˥}}}}\kern2pt]} \hypertarget{b\{\string_Bt`s`w\{\string_B1}{}
\markboth{\textcolor{darkblue}{\textbf{\ipa{bæ˩ʈʂwæ˩}}}}{}
\textcolor{teal}{\mytextsc{noun}} \hspace{4pt} Tone: L.
\textcolor{Sepia}{\selectlanguage{english}Reins.} \zh{缰绳。}  ¶ \textcolor{darkblue}{\textbf{\ipa{ʐwæ˧-bæ˥ʈʂwæ˩}}} \textcolor{Sepia}{\selectlanguage{english}horse's reins} \zh{马缰绳}  
 \zh{量词}: \textcolor{darkblue}{\textbf{\ipa{kʰɯ˩}}}  \mytextsc{clf}: \textcolor{darkblue}{\textbf{\ipa{kʰɯ˩}}} 
\lhead{\firstmark}
\rhead{\botmark}

\subsection{\hspace{-0.5cm} {\Large \textcolor{darkblue}{\textbf{\ipa{bæ˧˥}}}}\hspace{0.5cm}[\kern2pt{\textcolor{darkblue}{\textbf{\ipa{bæ˧˥}}}}\kern2pt]} \hypertarget{b\{\string_M\string_T1}{}
\markboth{\textcolor{darkblue}{\textbf{\ipa{bæ˧˥}}}}{}
\textcolor{teal}{\mytextsc{verb}} \hspace{4pt} Tone: MH.
\textcolor{Sepia}{\selectlanguage{english}To run.} \zh{跑。}  ¶ \textcolor{darkblue}{\textbf{\ipa{le˧-bæ˧-ze˥}}} \textcolor{Sepia}{\selectlanguage{english}\mytextsc{accomp} \string_ \mytextsc{pfv}} \zh{跑了}  

\lhead{\firstmark}
\rhead{\botmark}

\subsection{\hspace{-0.5cm} {\Large \textcolor{darkblue}{\textbf{\ipa{bæ˩˥}}}}\hspace{0.5cm}[\kern2pt{\textcolor{darkblue}{\textbf{\ipa{bæ˩˥}}}}\kern2pt]} \hypertarget{b\{\string_B\string_T1}{}
\markboth{\textcolor{darkblue}{\textbf{\ipa{bæ˩˥}}}}{}
\textcolor{teal}{\mytextsc{noun}} \hspace{4pt} Tone: LH.
\textcolor{Sepia}{\selectlanguage{english}Pus.} \zh{脓。}  ¶ \textcolor{darkblue}{\textbf{\ipa{bæ˩ bæ˧-ze˩}}} \textcolor{Sepia}{\selectlanguage{english}the wound suppurates, the wound is pussy} \zh{伤口在化脓}  
 ¶ \textcolor{darkblue}{\textbf{\ipa{bæ˩˥ | le˧-bæ˩-ze˩}}} \textcolor{Sepia}{\selectlanguage{english}the wound suppurates, the wound is pussy} \zh{伤口在化脓}  
 ¶ \textcolor{darkblue}{\textbf{\ipa{bæ˩˥ | le˧-bæ˩-ze˩}}}  
 \zh{量词}: \textcolor{darkblue}{\textbf{\ipa{ʈʰɤ˥}}}  \mytextsc{clf}: \textcolor{darkblue}{\textbf{\ipa{ʈʰɤ˥}}} \textit{See:} \hyperlink{}{\textcolor{darkblue}{\textbf{\ipa{bæ˩}}} \textsubscript{2}} 
\lhead{\firstmark}
\rhead{\botmark}

\subsection{\hspace{-0.5cm} {\Large \textcolor{darkblue}{\textbf{\ipa{bæ˩˧}}}}\hspace{0.5cm}[\kern2pt{\textcolor{darkblue}{\textbf{\ipa{bæ˩˥}}}}\kern2pt]} \hypertarget{b\{\string_B\string_M1}{}
\markboth{\textcolor{darkblue}{\textbf{\ipa{bæ˩˧}}}}{}
\textcolor{teal}{\mytextsc{noun}} \hspace{4pt} Tone: LM.
\textcolor{Sepia}{\selectlanguage{english}Crops.} \zh{庄稼。}  ¶ \textcolor{darkblue}{\textbf{\ipa{bæ˩ ɲi˧}}} \textcolor{Sepia}{\selectlanguage{english}\mytextsc{cop}} \zh{是庄稼}  
 ¶ \textcolor{darkblue}{\textbf{\ipa{ɖɯ˧-kʰv̩˧ ʈv̩˧-bæ˥ mv̩˩, | ɕi˧-kʰv̩˧ | le˧-mɤ˧-dzɯ˧!}}} \textcolor{Sepia}{\selectlanguage{english}“This year, even if we had had a thousand harvests, it would not have lasted a hundred years!” This proverb is a consolation for years of bad harvests: “If the harvest had been excellent, it would not have lasted forever anyway! Everything begins anew every year, so let us look forward!”} \zh{“一年收千棵,不够吃百年!”(这个谚语,来慰藉收成不好的年份。)}  
 \zh{量词}: \textcolor{darkblue}{\textbf{\ipa{bæ˩}}}  \mytextsc{clf}: \textcolor{darkblue}{\textbf{\ipa{bæ˩}}} 
\lhead{\firstmark}
\rhead{\botmark}

\subsection{\hspace{-0.5cm} {\Large \textcolor{darkblue}{\textbf{\ipa{bɤ˥}}}}\hspace{0.5cm}[\kern2pt{\textcolor{darkblue}{\textbf{\ipa{bɤ˥}}}}\kern2pt]} \hypertarget{b7\string_T1}{}
\markboth{\textcolor{darkblue}{\textbf{\ipa{bɤ˥}}}}{}
\textcolor{teal}{\mytextsc{noun}} \hspace{4pt} Tone: \#H.
\textcolor{Sepia}{\selectlanguage{english}Pumi (Prinmi) (ethnic group).} \zh{普米族。}  ¶ \textcolor{darkblue}{\textbf{\ipa{bɤ˧-ʐwɤ˧ so˥}}} \textcolor{Sepia}{\selectlanguage{english}to learn the Pumi language} \zh{学普米语}  
 \zh{量词}: \textcolor{darkblue}{\textbf{\ipa{v̩˧}}}  \mytextsc{clf}: \textcolor{darkblue}{\textbf{\ipa{v̩˧}}} 
\lhead{\firstmark}
\rhead{\botmark}

\subsection{\hspace{-0.5cm} {\Large \textcolor{darkblue}{\textbf{\ipa{bɤ˧dzi˩}}}}\hspace{0.5cm}[\kern2pt{\textcolor{darkblue}{\textbf{\ipa{xxxx non-correspondance entre le nombre de morphèmes et le nombre de tons de morphèmes}}}}\kern2pt]} \hypertarget{b7\string_Mdzi\string_B1}{}
\markboth{\textcolor{darkblue}{\textbf{\ipa{bɤ˧dzi˩}}}}{}
\textcolor{teal}{\mytextsc{noun}} \hspace{4pt} Tone: L\#.
\textcolor{Sepia}{\selectlanguage{english}A village in Yongning.} \zh{八珠(永宁的一个村落)。}  ¶ \textcolor{darkblue}{\textbf{\ipa{ɖæ˩ʂɯ\#˥, | ʈʂo˧ʂɯ\#˥, | bɤ˩tɕʰɯ˩˥, | dɑ˧pʰo˥, | bɤ˧dzi˩, | dze˧bo˧}}} \textcolor{Sepia}{\selectlanguage{english}the six villages of the plain of Yongning, in traditional order: by order of increasing distance from the Lake} \zh{永宁坝的六个村落,按传统排序:从距离泸沽湖最近的村落说起。}  

\lhead{\firstmark}
\rhead{\botmark}

\subsection{\hspace{-0.5cm} {\Large \textcolor{darkblue}{\textbf{\ipa{bɤ˧kɯ˧}}}}\hspace{0.5cm}[\kern2pt{\textcolor{darkblue}{\textbf{\ipa{bɤ˩kɯ˥}}}}\kern2pt]} \hypertarget{b7\string_MkM\string_M1}{}
\markboth{\textcolor{darkblue}{\textbf{\ipa{bɤ˧kɯ˧}}}}{}
\textcolor{teal}{\mytextsc{noun}} \hspace{4pt} Tone: M.
\textcolor{Sepia}{\selectlanguage{english}Sifter, sieve.} \zh{筛子。}  \zh{量词}: \textcolor{darkblue}{\textbf{\ipa{nɑ˧}}}  \mytextsc{clf}: \textcolor{darkblue}{\textbf{\ipa{nɑ˧}}} 
\lhead{\firstmark}
\rhead{\botmark}

\subsection{\hspace{-0.5cm} {\Large \textcolor{darkblue}{\textbf{\ipa{bɤ˧mi\#˥}}}}\hspace{0.5cm}[\kern2pt{\textcolor{darkblue}{\textbf{\ipa{xxxx non-correspondance entre le nombre de morphèmes et le nombre de tons de morphèmes}}}}\kern2pt]} \hypertarget{b7\string_Mmi\#\string_T1}{}
\markboth{\textcolor{darkblue}{\textbf{\ipa{bɤ˧mi\#˥}}}}{}
\textcolor{teal}{\mytextsc{noun}} \hspace{4pt} Tone: \#H.
\textcolor{Sepia}{\selectlanguage{english}Pumi woman.} \zh{普米族女人。}  \zh{量词}: \textcolor{darkblue}{\textbf{\ipa{v̩˧}}}  \mytextsc{clf}: \textcolor{darkblue}{\textbf{\ipa{v̩˧}}} 
\lhead{\firstmark}
\rhead{\botmark}

\subsection{\hspace{-0.5cm} {\Large \textcolor{darkblue}{\textbf{\ipa{bɤ˧mi˥-ʂe˩}}}}\hspace{0.5cm}[\kern2pt{\textcolor{darkblue}{\textbf{\ipa{bɤ˩mi˧ʂe˧}}}}\kern2pt]} \hypertarget{b7\string_Mmi\string_T-s`e\string_B1}{}
\markboth{\textcolor{darkblue}{\textbf{\ipa{bɤ˧mi˥-ʂe˩}}}}{}
\textcolor{teal}{\mytextsc{noun}} \hspace{4pt} Tone: H\#-.
\textcolor{Sepia}{\selectlanguage{english}Copper-nickel alloy.} \zh{白铜。} 
\lhead{\firstmark}
\rhead{\botmark}

\subsection{\hspace{-0.5cm} {\Large \textcolor{darkblue}{\textbf{\ipa{bɤ˧ʂɯ˩}}}}\hspace{0.5cm}[\kern2pt{\textcolor{darkblue}{\textbf{\ipa{bɤ˩ʂɯ˥}}}}\kern2pt]} \hypertarget{b7\string_Ms`M\string_B1}{}
\markboth{\textcolor{darkblue}{\textbf{\ipa{bɤ˧ʂɯ˩}}}}{}
\textcolor{teal}{\mytextsc{noun}} \hspace{4pt} Tone: L\#.
\textcolor{Sepia}{\selectlanguage{english}Baisha: name of a village of the Lijiang plain. The village had a tradition of trading and peddling to faraway places, hence its familiarity to people in Yongning. Its Naxi name is \textcolor{darkblue}{\textbf{\ipa{/bɤ˧ʂɯ˩/}}}.} \zh{白沙(丽江坝子里的一个村落)。} 
\lhead{\firstmark}
\rhead{\botmark}

\subsection{\hspace{-0.5cm} {\Large \textcolor{darkblue}{\textbf{\ipa{bɤ˧tʰv̩˩}}}}\hspace{0.5cm}[\kern2pt{\textcolor{darkblue}{\textbf{\ipa{bɤ˧tʰv̩˥}}}}\kern2pt]} \hypertarget{b7\string_Mt\string_hv\string_=\string_B1}{}
\markboth{\textcolor{darkblue}{\textbf{\ipa{bɤ˧tʰv̩˩}}}}{}
\textcolor{teal}{\mytextsc{noun}} \hspace{4pt} Tone: L\#.
\textcolor{Sepia}{\selectlanguage{english}Footprints.} \zh{脚印。}  ¶ \textcolor{darkblue}{\textbf{\ipa{hĩ˧-bɤ˧tʰv̩˥}}} \textcolor{Sepia}{\selectlanguage{english}human footprints} \zh{人的脚印}  
 ¶ \textcolor{darkblue}{\textbf{\ipa{kʰv̩˩mi˩-bɤ˩tʰv̩˥}}} \textcolor{Sepia}{\selectlanguage{english}dog's footprints} \zh{狗爪印}  
 \zh{量词}: \textcolor{darkblue}{\textbf{\ipa{tʰv̩˧˥}}}  \mytextsc{clf}: \textcolor{darkblue}{\textbf{\ipa{tʰv̩˧˥}}} 
\lhead{\firstmark}
\rhead{\botmark}

\subsection{\hspace{-0.5cm} {\Large \textcolor{darkblue}{\textbf{\ipa{bɤ˧tsʰo˧gv̩˥}}}}\hspace{0.5cm}[\kern2pt{\textcolor{darkblue}{\textbf{\ipa{bɤ˧tsʰo˧gv̩˩}}}}\kern2pt]} \hypertarget{b7\string_Mts\string_ho\string_Mgv\string_=\string_T1}{}
\markboth{\textcolor{darkblue}{\textbf{\ipa{bɤ˧tsʰo˧gv̩˥}}}}{}
\textcolor{teal}{\mytextsc{noun}} \hspace{4pt} Tone: H\#.
\textcolor{Sepia}{\selectlanguage{english}A village of the Lijiang plain: the central village of the plain, where the marketplace was still located in the early 21st century.} \zh{巴搓古(永宁的一个村落)。}  ¶ \textcolor{darkblue}{\textbf{\ipa{bɤ˧tsʰo˧gv̩˥-hĩ˩}}} \textcolor{Sepia}{\selectlanguage{english}someone from Bacuogu} \zh{从巴搓古来的一个人}  
 ¶ \textcolor{darkblue}{\textbf{\ipa{dʑɤ˩bv̩˧kɤ˧-sɑ˥ʁwɤ˩, | hi˩ʁwɤ˩-lo˥, | æ˩mi˧-ʁwɤ\#˥, | lɑ˧lo˧-ʁwɤ˥, | lɑ˧ŋwɤ˧, | bɤ˧tsʰo˧gv̩˥, | ə˧lɑ˧-ʁwɤ\#˥, | gæ˧ɻæ˩, | qʰæ˧tɕʰi˧, | tʰo˧ʈɯ\#˥}}} \textcolor{Sepia}{\selectlanguage{english}the ten villages traditionally considered as part of Yongning} \zh{摩梭传统地理概念中,属于永宁的十个村落}  

\lhead{\firstmark}
\rhead{\botmark}

\subsection{\hspace{-0.5cm} {\Large \textcolor{darkblue}{\textbf{\ipa{bɤ˧zo\#˥}}}}\hspace{0.5cm}[\kern2pt{\textcolor{darkblue}{\textbf{\ipa{bɤ˧zo˧}}}}\kern2pt]} \hypertarget{b7\string_Mzo\#\string_T1}{}
\markboth{\textcolor{darkblue}{\textbf{\ipa{bɤ˧zo\#˥}}}}{}
\textcolor{teal}{\mytextsc{noun}} \hspace{4pt} Tone: \#H.
\textcolor{Sepia}{\selectlanguage{english}Pumi man.} \zh{普米族男人。}  \zh{量词}: \textcolor{darkblue}{\textbf{\ipa{v̩˧}}}  \mytextsc{clf}: \textcolor{darkblue}{\textbf{\ipa{v̩˧}}} 
\lhead{\firstmark}
\rhead{\botmark}

\subsection{\hspace{-0.5cm} {\Large \textcolor{darkblue}{\textbf{\ipa{bɤ˩\textsubscript{a}}}} \textsubscript{1}}\hspace{0.5cm}[\kern2pt{\textcolor{darkblue}{\textbf{\ipa{bɤ˩˥}}}}\kern2pt]} \hypertarget{b7\string_Ba1}{}
\markboth{\textcolor{darkblue}{\textbf{\ipa{bɤ˩\textsubscript{a}}}} \textsubscript{1}}{}
\textcolor{teal}{\mytextsc{classifier}} \hspace{4pt} Tone: L\textsubscript{a}.
\textcolor{Sepia}{\selectlanguage{english}Classifier for corncobs.} \zh{量词:玉米棒子(一根)。}  ¶ \textcolor{darkblue}{\textbf{\ipa{hɑ˧bɤ˥ | ɖɯ˧-bɤ˩}}} \textcolor{Sepia}{\selectlanguage{english}a corncob} \zh{一根玉米棒子}  
 ¶ \textcolor{darkblue}{\textbf{\ipa{tʰv̩˧-bɤ˥}}} \textcolor{Sepia}{\selectlanguage{english}\mytextsc{dem} \string_ (tone: H\# / H\$)} \zh{\mytextsc{指示代词} \string_:那根(玉米棒子)}  

\lhead{\firstmark}
\rhead{\botmark}

\subsection{\hspace{-0.5cm} {\Large \textcolor{darkblue}{\textbf{\ipa{bɤ˩\textsubscript{a}}}} \textsubscript{2}}\hspace{0.5cm}[\kern2pt{\textcolor{darkblue}{\textbf{\ipa{bɤ˩˥}}}}\kern2pt]} \hypertarget{b7\string_Ba2}{}
\markboth{\textcolor{darkblue}{\textbf{\ipa{bɤ˩\textsubscript{a}}}} \textsubscript{2}}{}
\textcolor{teal}{\mytextsc{classifier}} \hspace{4pt} Tone: L\textsubscript{a}.
\textcolor{Sepia}{\selectlanguage{english}Classifier for halves.} \zh{量词:半。}  ¶ \textcolor{darkblue}{\textbf{\ipa{ɖɯ˧-bɤ˩-lɑ˩ tʰv̩˩-sɯ˩! | ɖɯ˧-hu˧-ɻ̍˥!}}} \textcolor{Sepia}{\selectlanguage{english}I have only done one half as yet! Wait a bit! (Context: someone is sorting out clothes, and is midway through the task.)} \zh{我才干了一半!稍等!(情景:一个在收拾衣服,告诉对方:工作没完,还需要时间。)}  
 ¶ \textcolor{darkblue}{\textbf{\ipa{ɖɯ˧-bɤ˩-lɑ˩ tʰv̩˩-ze˩!}}} \textcolor{Sepia}{\selectlanguage{english}You are only half-way through! (An observation about the investigator's progress in studying the Na language. It emphasizes the road ahead; still, this is a more encouraging formulation than the previous example: this one uses the perfective, acknowledging that part of the path that has already been covered, whereas \textcolor{darkblue}{\textbf{\ipa{ɖɯ˧-bɤ˩-lɑ˩ tʰv̩˩-sɯ˩}}} would essentially emphasize all that remains ahead.)} \zh{你才到了一半!(合作人对调查者学摩梭话的评定)}  
 ¶ \textcolor{darkblue}{\textbf{\ipa{ʐæ˩ʂæ˥ | ʐwæ˩˥! | le˧-se˥, | ɖɯ˧-bɤ˩-qo˩-lɑ˩ tʰv̩˩-sɯ˩!}}} \textcolor{Sepia}{\selectlanguage{english}It's really far! We have walked (for quite some time), and only got mid-way!} \zh{真远!走啊走,才走了一半的路!}  
 ¶ \textcolor{darkblue}{\textbf{\ipa{ʐɤ˩mi˩˥ | ɖɯ˧-bɤ˩}}} \textcolor{Sepia}{\selectlanguage{english}half the way} \zh{半路}  
 ¶ \textcolor{darkblue}{\textbf{\ipa{ə˧mi˧! | wɤ˩˥ | ɖɯ˧-bɤ˩ dʑo˩-sɯ˩-wɤ˩!}}} \textcolor{Sepia}{\selectlanguage{english}Wow! (How far!) There is still half the way to go!} \zh{啊呀嚒!还剩一半的路啊!}  

\lhead{\firstmark}
\rhead{\botmark}

\subsection{\hspace{-0.5cm} {\Large \textcolor{darkblue}{\textbf{\ipa{bɤ˩tɕʰɯ˩}}}}\hspace{0.5cm}[\kern2pt{\textcolor{darkblue}{\textbf{\ipa{bɤ˧tɕʰɯ˩}}}}\kern2pt]} \hypertarget{b7\string_Bts£\string_hM\string_B1}{}
\markboth{\textcolor{darkblue}{\textbf{\ipa{bɤ˩tɕʰɯ˩}}}}{}
\textcolor{teal}{\mytextsc{noun}} \hspace{4pt} Tone: L.
\textcolor{Sepia}{\selectlanguage{english}A village in the plain of Lijiang.} \zh{八七(永宁的一个村落)。}  ¶ \textcolor{darkblue}{\textbf{\ipa{ɖæ˩ʂɯ\#˥, | ʈʂo˧ʂɯ\#˥, | bɤ˩tɕʰɯ˩˥, | dɑ˧pʰo˥, | bɤ˧dzi˩, | dze˧bo˧}}} \textcolor{Sepia}{\selectlanguage{english}the six villages of the plain of Yongning, in traditional order: by order of increasing distance from the Lake} \zh{永宁坝的六个村落,按传统排序:从距离泸沽湖最近的村落说起。}  

\lhead{\firstmark}
\rhead{\botmark}

\subsection{\hspace{-0.5cm} {\Large \textcolor{darkblue}{\textbf{\ipa{bɤ˧˥\textsubscript{a}}}}}\hspace{0.5cm}[\kern2pt{\textcolor{darkblue}{\textbf{\ipa{bɤ˩˥}}}}\kern2pt]} \hypertarget{b7\string_M\string_Ta1}{}
\markboth{\textcolor{darkblue}{\textbf{\ipa{bɤ˧˥\textsubscript{a}}}}}{}
\textcolor{teal}{\mytextsc{classifier}} \hspace{4pt} Tone: MH\textsubscript{a}.
\textcolor{Sepia}{\selectlanguage{english}Classifier for scarves.} \zh{量词:头帕(一条)。} 
\lhead{\firstmark}
\rhead{\botmark}

\subsection{\hspace{-0.5cm} {\Large \textcolor{darkblue}{\textbf{\ipa{‑bi}}}}\hspace{0.5cm}[\kern2pt{\textcolor{darkblue}{\textbf{\ipa{xxxx groupe tonal entier sans aucun ton}}}}\kern2pt]} \hypertarget{‑bi1}{}
\markboth{\textcolor{darkblue}{\textbf{\ipa{‑bi}}}}{}
\textcolor{teal}{\mytextsc{conjunction}} \hspace{4pt} Tone: 0.
\textcolor{Sepia}{\selectlanguage{english}\mytextsc{adversative}: no matter….} \zh{虽然……。}  ¶ \textcolor{darkblue}{\textbf{\ipa{ʈʂʰɯ˧ | nɑ˩ ɲi˥-pi˩-bi˩-bi˩, | nɑ˩-ʐwɤ˧ | mɤ˧-kv̩˧˥!}}} \textcolor{Sepia}{\selectlanguage{english}Although (s)he is Na, (s)he cannot speak the Na language!} \zh{他虽然是摩梭人但不会讲摩梭话。}  
 ¶ \textcolor{darkblue}{\textbf{\ipa{*ʈʂʰɯ˧ | nɑ˩ ɲi˥-bi˩, … / *ʈʂʰɯ˧ | nɑ˩ ɲi˥-bi˩-bi˩}}} \textcolor{Sepia}{\selectlanguage{english}an ungrammatical sentence; the intended meaning was 'although (s)he is Na...'} \zh{病句:不能这样说“他虽然是摩梭人……”}  

\lhead{\firstmark}
\rhead{\botmark}

\subsection{\hspace{-0.5cm} {\Large \textcolor{darkblue}{\textbf{\ipa{bi˥}}}}\hspace{0.5cm}[\kern2pt{\textcolor{darkblue}{\textbf{\ipa{bi˥}}}}\kern2pt]} \hypertarget{bi\string_T1}{}
\markboth{\textcolor{darkblue}{\textbf{\ipa{bi˥}}}}{}
\textcolor{teal}{\mytextsc{noun}} \hspace{4pt} Tone: \#H.
\textcolor{Sepia}{\selectlanguage{english}Snow.} \zh{雪。}  ¶ \textcolor{darkblue}{\textbf{\ipa{bi˧ gi˧-ze˩}}} \textcolor{Sepia}{\selectlanguage{english}it is snowing; it has snowed} \zh{下雪了}  
 \zh{量词}: \textcolor{darkblue}{\textbf{\ipa{ʁwɤ˧, etc}}}  \mytextsc{clf}: \textcolor{darkblue}{\textbf{\ipa{ʁwɤ˧, etc}}} 
\lhead{\firstmark}
\rhead{\botmark}

\subsection{\hspace{-0.5cm} {\Large \textcolor{darkblue}{\textbf{\ipa{bi˥}}}}\hspace{0.5cm}[\kern2pt{\textcolor{darkblue}{\textbf{\ipa{bi˥}}}}\kern2pt]} \hypertarget{bi\string_T1}{}
\markboth{\textcolor{darkblue}{\textbf{\ipa{bi˥}}}}{}
\textcolor{teal}{\mytextsc{adjective}} \hspace{4pt} Tone: H.
\textcolor{Sepia}{\selectlanguage{english}Thin; shallow.} \zh{薄,浅(水浅)。}  ¶ \textcolor{darkblue}{\textbf{\ipa{bi˧ | ʐwæ˩˥!}}} \textcolor{Sepia}{\selectlanguage{english}It's very shallow!} \zh{很浅!}  
 ¶ \textcolor{darkblue}{\textbf{\ipa{dʑɯ˧ | ɖɯ˧-pi˧ bi˧˥}}} \textcolor{Sepia}{\selectlanguage{english}The water is rather shallow.} \zh{水有点浅。}  
 ¶ \textcolor{darkblue}{\textbf{\ipa{dʑɯ˧ bi˧-hĩ˧, | mɤ˧-ɖwæ˩!}}} \textcolor{Sepia}{\selectlanguage{english}Shallow water is not frightening! / There is nothing frightening about shallow water! / Come on, don't be afraid: it's (just) shallow water!} \zh{水很浅,不用怕!}  

\lhead{\firstmark}
\rhead{\botmark}

\subsection{\hspace{-0.5cm} {\Large \textcolor{darkblue}{\textbf{\ipa{‑bi˧}}}}\hspace{0.5cm}[\kern2pt{\textcolor{darkblue}{\textbf{\ipa{bi˥}}}}\kern2pt]} \hypertarget{‑bi\string_M1}{}
\markboth{\textcolor{darkblue}{\textbf{\ipa{‑bi˧}}}}{}
\textcolor{teal}{\mytextsc{suffix}} \hspace{4pt} Tone: M.
\textcolor{Sepia}{\selectlanguage{english}Immediate future.} \zh{要\mytextsc{近将来。}} 
\lhead{\firstmark}
\rhead{\botmark}

\subsection{\hspace{-0.5cm} {\Large \textcolor{darkblue}{\textbf{\ipa{bi˧}}} \textsubscript{2}}\hspace{0.5cm}[\kern2pt{\textcolor{darkblue}{\textbf{\ipa{bi˥}}}}\kern2pt]} \hypertarget{bi\string_M2}{}
\markboth{\textcolor{darkblue}{\textbf{\ipa{bi˧}}} \textsubscript{2}}{}
\textcolor{teal}{\mytextsc{noun}} \hspace{4pt} Tone: M.
\textcolor{Sepia}{\selectlanguage{english}Village; neighbours, people in the village.} \zh{村落,邻居、村里的人们。} 
\lhead{\firstmark}
\rhead{\botmark}

\subsection{\hspace{-0.5cm} {\Large \textcolor{darkblue}{\textbf{\ipa{bi˧}}} \textsubscript{3}}\hspace{0.5cm}[\kern2pt{\textcolor{darkblue}{\textbf{\ipa{bi˥}}}}\kern2pt]} \hypertarget{bi\string_M3}{}
\markboth{\textcolor{darkblue}{\textbf{\ipa{bi˧}}} \textsubscript{3}}{}
\textcolor{teal}{\mytextsc{verb}} \hspace{4pt} Tone: M.
\textcolor{Sepia}{\selectlanguage{english}To dare.} \zh{敢。}  ¶ \textcolor{darkblue}{\textbf{\ipa{ʝi˧-mɤ˧-bi˧}}} \textcolor{Sepia}{\selectlanguage{english}not to dare to do} \zh{不敢做}  

\lhead{\firstmark}
\rhead{\botmark}

\subsection{\hspace{-0.5cm} {\Large \textcolor{darkblue}{\textbf{\ipa{bi˧\textsubscript{c}}}} \textsubscript{1}}\hspace{0.5cm}[\kern2pt{\textcolor{darkblue}{\textbf{\ipa{bi˩˥}}}}\kern2pt]} \hypertarget{bi\string_Mc1}{}
\markboth{\textcolor{darkblue}{\textbf{\ipa{bi˧\textsubscript{c}}}} \textsubscript{1}}{}
\textcolor{teal}{\mytextsc{verb}} \hspace{4pt} Tone: M\textsubscript{c}.
\textcolor{Sepia}{\selectlanguage{english}To go.} \zh{去。}  ¶ \textcolor{darkblue}{\textbf{\ipa{bi˧-tʰɑ˧!}}} \textcolor{Sepia}{\selectlanguage{english}\mytextsc{abilitive}} \zh{可以去!}  
 ¶ \textcolor{darkblue}{\textbf{\ipa{bi˧-tʰɑ˧-ze˥!}}} \textcolor{Sepia}{\selectlanguage{english}\mytextsc{abilitive}+\mytextsc{pfv}: We can go now!} \zh{可以去了!}  
 ¶ \textcolor{darkblue}{\textbf{\ipa{le˧-bi˩}}} \textcolor{Sepia}{\selectlanguage{english}to go back} \zh{回去,返回}  

\lhead{\firstmark}
\rhead{\botmark}

\subsection{\hspace{-0.5cm} {\Large \textcolor{darkblue}{\textbf{\ipa{bi˧bv̩˥}}}}\hspace{0.5cm}[\kern2pt{\textcolor{darkblue}{\textbf{\ipa{bi˩bv̩˥}}}}\kern2pt]} \hypertarget{bi\string_Mbv\string_=\string_T1}{}
\markboth{\textcolor{darkblue}{\textbf{\ipa{bi˧bv̩˥}}}}{}
\textcolor{teal}{\mytextsc{verb}} \hspace{4pt} Tone: H\#.
\textcolor{Sepia}{\selectlanguage{english}To flow profusely.} \zh{流淌,冲下去,下泻,很快地流。}  ¶ \textcolor{darkblue}{\textbf{\ipa{tʰi˧-ʈwæ˧˥, | sɤ˧ | bi˧bv̩˥-ze˩!}}} \textcolor{Sepia}{\selectlanguage{english}(She/he) fell down; blood is flowing profusely!} \zh{(他)摔倒了,流了很多血}  
 ¶ \textcolor{darkblue}{\textbf{\ipa{dʑɯ˧ | bi˧bv̩˥-ze˩!}}} \textcolor{Sepia}{\selectlanguage{english}The water is flowing profusely!} \zh{水流如注!}  
 ¶ \textcolor{darkblue}{\textbf{\ipa{dʑɯ˩nɑ˩mi˩ bi˩bv̩˥-ze˩-pʰæ˩di˩!}}} \textcolor{Sepia}{\selectlanguage{english}There seems to be a flood / landslide out on the mountain!} \zh{山上好像有了泥石流!}  

\lhead{\firstmark}
\rhead{\botmark}

\subsection{\hspace{-0.5cm} {\Large \textcolor{darkblue}{\textbf{\ipa{bi˧ɕi˧kv̩˥}}}}\hspace{0.5cm}[\kern2pt{\textcolor{darkblue}{\textbf{\ipa{bi˧ɕi˧kv̩˧}}}}\kern2pt]} \hypertarget{bi\string_Ms£i\string_Mkv\string_=\string_T1}{}
\markboth{\textcolor{darkblue}{\textbf{\ipa{bi˧ɕi˧kv̩˥}}}}{}
\textcolor{teal}{\mytextsc{noun}} \hspace{4pt} Tone: H\#.
\textcolor{Sepia}{\selectlanguage{english}Cheek.} \zh{腮,腮帮子。}  \zh{量词}: \textcolor{darkblue}{\textbf{\ipa{ɭɯ˧}}}  \mytextsc{clf}: \textcolor{darkblue}{\textbf{\ipa{ɭɯ˧}}} 
\lhead{\firstmark}
\rhead{\botmark}

\subsection{\hspace{-0.5cm} {\Large \textcolor{darkblue}{\textbf{\ipa{bi˧hæ˧˥}}}}\hspace{0.5cm}[\kern2pt{\textcolor{darkblue}{\textbf{\ipa{bi˧hæ˥}}}}\kern2pt]} \hypertarget{bi\string_Mh\{\string_M\string_T1}{}
\markboth{\textcolor{darkblue}{\textbf{\ipa{bi˧hæ˧˥}}}}{}
\textcolor{teal}{\mytextsc{noun}} \hspace{4pt} Tone: MH\#.
\textcolor{Sepia}{\selectlanguage{english}Girth (for horse).} \zh{马肚带。}  ¶ \textcolor{darkblue}{\textbf{\ipa{ʐwæ˧-bi˥hæ˩}}} \textcolor{Sepia}{\selectlanguage{english}horse's girth} \zh{马肚带}  
 \zh{量词}: \textcolor{darkblue}{\textbf{\ipa{kʰɯ˩}}}  \mytextsc{clf}: \textcolor{darkblue}{\textbf{\ipa{kʰɯ˩}}} 
\lhead{\firstmark}
\rhead{\botmark}

\subsection{\hspace{-0.5cm} {\Large \textcolor{darkblue}{\textbf{\ipa{bi˧-lv̩˧\textasciitilde{}lv̩˥}}}}\hspace{0.5cm}[\kern2pt{\textcolor{darkblue}{\textbf{\ipa{xxxx non-correspondance entre le nombre de morphèmes et le nombre de tons de morphèmes}}}}\kern2pt]} \hypertarget{bi\string_M-lv\string_=\string_M~lv\string_=\string_T1}{}
\markboth{\textcolor{darkblue}{\textbf{\ipa{bi˧-lv̩˧\textasciitilde{}lv̩˥}}}}{}
\textcolor{teal}{\mytextsc{noun}} \hspace{4pt} Tone: H\#.
\textcolor{Sepia}{\selectlanguage{english}Snow flakes.} \zh{雪花。}  \zh{量词}: \textcolor{darkblue}{\textbf{\ipa{ɭɯ˧}}}  \mytextsc{clf}: \textcolor{darkblue}{\textbf{\ipa{ɭɯ˧}}} 
\lhead{\firstmark}
\rhead{\botmark}

\subsection{\hspace{-0.5cm} {\Large \textcolor{darkblue}{\textbf{\ipa{bi˧mi˧}}}}\hspace{0.5cm}[\kern2pt{\textcolor{darkblue}{\textbf{\ipa{bi˩mi˩˥}}}}\kern2pt]} \hypertarget{bi\string_Mmi\string_M1}{}
\markboth{\textcolor{darkblue}{\textbf{\ipa{bi˧mi˧}}}}{}
\textcolor{teal}{\mytextsc{noun}} \hspace{4pt} Tone: M.
\textcolor{Sepia}{\selectlanguage{english}Belly, abdomen.} \zh{肚子。}  ¶ \textcolor{darkblue}{\textbf{\ipa{bi˧mi˧-ɖɯ˩}}} \textcolor{Sepia}{\selectlanguage{english}covetous, greedy} \zh{贪心不足,贪吃}  
 \zh{量词}: \textcolor{darkblue}{\textbf{\ipa{ɭɯ˧}}}  \mytextsc{clf}: \textcolor{darkblue}{\textbf{\ipa{ɭɯ˧}}} 
\lhead{\firstmark}
\rhead{\botmark}

\subsection{\hspace{-0.5cm} {\Large \textcolor{darkblue}{\textbf{\ipa{bi˧tɑ˧}}}}\hspace{0.5cm}[\kern2pt{\textcolor{darkblue}{\textbf{\ipa{bi˩tɑ˩˥}}}}\kern2pt]} \hypertarget{bi\string_MtA\string_M1}{}
\markboth{\textcolor{darkblue}{\textbf{\ipa{bi˧tɑ˧}}}}{}
\textcolor{teal}{\mytextsc{noun}} \hspace{4pt} Tone: M.
\textcolor{Sepia}{\selectlanguage{english}Broad piece of fabric worn as a belt, also strapped around the shoulders.} \zh{宽腰带。}  \zh{量词}: \textcolor{darkblue}{\textbf{\ipa{tsʰi˥}}}  \mytextsc{clf}: \textcolor{darkblue}{\textbf{\ipa{tsʰi˥}}} 
\lhead{\firstmark}
\rhead{\botmark}

\subsection{\hspace{-0.5cm} {\Large \textcolor{darkblue}{\textbf{\ipa{bi˧tɕɤ˩}}}}\hspace{0.5cm}[\kern2pt{\textcolor{darkblue}{\textbf{\ipa{bi˧tɕɤ˧}}}}\kern2pt]} \hypertarget{bi\string_Mts£7\string_B1}{}
\markboth{\textcolor{darkblue}{\textbf{\ipa{bi˧tɕɤ˩}}}}{}
\textcolor{teal}{\mytextsc{noun}} \hspace{4pt} Tone: L\#.
\textcolor{Sepia}{\selectlanguage{english}Navel.} \zh{肚脐。}  \zh{量词}: \textcolor{darkblue}{\textbf{\ipa{kʰwɤ˥}}}  \mytextsc{clf}: \textcolor{darkblue}{\textbf{\ipa{kʰwɤ˥}}} 
\lhead{\firstmark}
\rhead{\botmark}

\subsection{\hspace{-0.5cm} {\Large \textcolor{darkblue}{\textbf{\ipa{bi˧tɕo˧}}}}\hspace{0.5cm}[\kern2pt{\textcolor{darkblue}{\textbf{\ipa{bi˧tɕo˧}}}}\kern2pt]} \hypertarget{bi\string_Mts£o\string_M1}{}
\markboth{\textcolor{darkblue}{\textbf{\ipa{bi˧tɕo˧}}}}{}
\textcolor{teal}{\mytextsc{noun}} \hspace{4pt} Tone: M.
\textcolor{Sepia}{\selectlanguage{english}Neighbouring villages, neighbourhood.} \zh{周围的村落。} 
\lhead{\firstmark}
\rhead{\botmark}

\subsection{\hspace{-0.5cm} {\Large \textcolor{darkblue}{\textbf{\ipa{bi˧zɯ˧}}}}\hspace{0.5cm}[\kern2pt{\textcolor{darkblue}{\textbf{\ipa{bi˩zɯ˥}}}}\kern2pt]} \hypertarget{bi\string_MzM\string_M1}{}
\markboth{\textcolor{darkblue}{\textbf{\ipa{bi˧zɯ˧}}}}{}
\textcolor{teal}{\mytextsc{noun}} \hspace{4pt} Tone: M.
\textcolor{Sepia}{\selectlanguage{english}Lower abdomen.} \zh{小肚子。}  \zh{量词}: \textcolor{darkblue}{\textbf{\ipa{ɭɯ˧}}}  \mytextsc{clf}: \textcolor{darkblue}{\textbf{\ipa{ɭɯ˧}}} 
\lhead{\firstmark}
\rhead{\botmark}

\subsection{\hspace{-0.5cm} {\Large \textcolor{darkblue}{\textbf{\ipa{bi˩}}}}\hspace{0.5cm}[\kern2pt{\textcolor{darkblue}{\textbf{\ipa{bi˥}}}}\kern2pt]} \hypertarget{bi\string_B1}{}
\markboth{\textcolor{darkblue}{\textbf{\ipa{bi˩}}}}{}
\textcolor{teal}{\mytextsc{postposition}} \hspace{4pt} Tone: L.
\textcolor{Sepia}{\selectlanguage{english}On; at.} \zh{向、至、往。}  ¶ \textcolor{darkblue}{\textbf{\ipa{gv̩˧mi˧-bi˩}}} \textcolor{Sepia}{\selectlanguage{english}on the body} \zh{身上}  
 ¶ \textcolor{darkblue}{\textbf{\ipa{kʰɯ˧tsʰɤ˧-bi˥}}} \textcolor{Sepia}{\selectlanguage{english}on the feet} \zh{脚上}  
 ¶ \textcolor{darkblue}{\textbf{\ipa{ʐæ˩sɯ˩-bi˥ | tʰi˧-ʈʂʰv̩˧˥}}} \textcolor{Sepia}{\selectlanguage{english}to hold grip of the felt cape} \zh{抓住毡子}  
 ¶ \textcolor{darkblue}{\textbf{\ipa{lo˩qʰwɤ˧ bi˩}}} \textcolor{Sepia}{\selectlanguage{english}on the hand} \zh{手上}  
 ¶ \textcolor{darkblue}{\textbf{\ipa{pʰæ˧qʰwɤ˩ bi˩, | mɤ˩ tʰi˩-jɤ˩˥.}}} \textcolor{Sepia}{\selectlanguage{english}to put oil onto the face, to put on suncream} \zh{抹防晒霜}  

\lhead{\firstmark}
\rhead{\botmark}

\subsection{\hspace{-0.5cm} {\Large \textcolor{darkblue}{\textbf{\ipa{bi˩\textsubscript{c}}}}}\hspace{0.5cm}[\kern2pt{\textcolor{darkblue}{\textbf{\ipa{bi˥}}}}\kern2pt]} \hypertarget{bi\string_Bc1}{}
\markboth{\textcolor{darkblue}{\textbf{\ipa{bi˩\textsubscript{c}}}}}{}
\textcolor{teal}{\mytextsc{classifier}} \hspace{4pt} Tone: L\textsubscript{c}.
\textcolor{Sepia}{\selectlanguage{english}Self-classifier for animal hooves; also used for footprints.} \zh{量词:动物的脚或脚印(一只)。} 
\lhead{\firstmark}
\rhead{\botmark}

\subsection{\hspace{-0.5cm} {\Large \textcolor{darkblue}{\textbf{\ipa{bi˩bi˧}}}}\hspace{0.5cm}[\kern2pt{\textcolor{darkblue}{\textbf{\ipa{bi˧bi˧}}}}\kern2pt]} \hypertarget{bi\string_Bbi\string_M1}{}
\markboth{\textcolor{darkblue}{\textbf{\ipa{bi˩bi˧}}}}{}
\textcolor{teal}{\mytextsc{noun}} \hspace{4pt} Tone: LM.
\textcolor{Sepia}{\selectlanguage{english}Pod (of bean).} \zh{豆荚。}  ¶ \textcolor{darkblue}{\textbf{\ipa{bi˩bi˧ ɲi˩}}} \textcolor{Sepia}{\selectlanguage{english}\mytextsc{cop}} \zh{是豆荚}  
 ¶ \textcolor{darkblue}{\textbf{\ipa{nv̩˩ɭɯ˧-bi˩bi˩}}} \textcolor{Sepia}{\selectlanguage{english}soybean pods} \zh{黄豆荚}  
 \zh{量词}: \textcolor{darkblue}{\textbf{\ipa{kʰwɤ˥}}}  \mytextsc{clf}: \textcolor{darkblue}{\textbf{\ipa{kʰwɤ˥}}} 
\lhead{\firstmark}
\rhead{\botmark}

\subsection{\hspace{-0.5cm} {\Large \textcolor{darkblue}{\textbf{\ipa{bi˩mi˩}}}}\hspace{0.5cm}[\kern2pt{\textcolor{darkblue}{\textbf{\ipa{bi˧mi˥}}}}\kern2pt]} \hypertarget{bi\string_Bmi\string_B1}{}
\markboth{\textcolor{darkblue}{\textbf{\ipa{bi˩mi˩}}}}{}
\textcolor{teal}{\mytextsc{noun}} \hspace{4pt} Tone: L.
\textcolor{Sepia}{\selectlanguage{english}Axe.} \zh{斧头。}  \zh{量词}: \textcolor{darkblue}{\textbf{\ipa{nɑ˧}}}  \mytextsc{clf}: \textcolor{darkblue}{\textbf{\ipa{nɑ˧}}} 
\lhead{\firstmark}
\rhead{\botmark}

\subsection{\hspace{-0.5cm} {\Large \textcolor{darkblue}{\textbf{\ipa{bi˩pʰv̩˧˥}}}}\hspace{0.5cm}[\kern2pt{\textcolor{darkblue}{\textbf{\ipa{bi˧pʰv̩˧}}}}\kern2pt]} \hypertarget{bi\string_Bp\string_hv\string_=\string_M\string_T1}{}
\markboth{\textcolor{darkblue}{\textbf{\ipa{bi˩pʰv̩˧˥}}}}{}
\textcolor{teal}{\mytextsc{noun}} \hspace{4pt} Tone: LM+MH\#.
\textcolor{Sepia}{\selectlanguage{english}Bottle gourd; its fruit is the calabash, which becomes hard when dry.} \zh{葫芦。} 
\lhead{\firstmark}
\rhead{\botmark}

\subsection{\hspace{-0.5cm} {\Large \textcolor{darkblue}{\textbf{\ipa{bi˩pʰv̩˧-dʑɯ˧˥}}}}\hspace{0.5cm}[\kern2pt{\textcolor{darkblue}{\textbf{\ipa{xxxx non-correspondance entre le nombre de morphèmes et le nombre de tons de morphèmes}}}}\kern2pt]} \hypertarget{bi\string_Bp\string_hv\string_=\string_M-dz£M\string_M\string_T1}{}
\markboth{\textcolor{darkblue}{\textbf{\ipa{bi˩pʰv̩˧-dʑɯ˧˥}}}}{}
\textcolor{teal}{\mytextsc{noun}} \hspace{4pt} Tone: LM+MH\#.
\textcolor{Sepia}{\selectlanguage{english}Flood.} \zh{洪水。} 
\lhead{\firstmark}
\rhead{\botmark}

\subsection{\hspace{-0.5cm} {\Large \textcolor{darkblue}{\textbf{\ipa{bi˩ʁo˥}}}}\hspace{0.5cm}[\kern2pt{\textcolor{darkblue}{\textbf{\ipa{bi˩ʁo˧˥}}}}\kern2pt]} \hypertarget{bi\string_BRo\string_T1}{}
\markboth{\textcolor{darkblue}{\textbf{\ipa{bi˩ʁo˥}}}}{}
\textcolor{teal}{\mytextsc{noun}} \hspace{4pt} Tone: LH.
\textcolor{Sepia}{\selectlanguage{english}Purse (used to be attached to the belt, on the inside).} \zh{钱包(过去:是系在腰带上的)。}  ¶ \textcolor{darkblue}{\textbf{\ipa{bi˩ʁo˧ tʰi˧-ʂæ˧˥}}} \textcolor{Sepia}{\selectlanguage{english}to attach (one's) purse (to the belt)} \zh{系上钱包(在腰带上)}  
 \zh{量词}: \textcolor{darkblue}{\textbf{\ipa{ɭɯ˧}}}  \mytextsc{clf}: \textcolor{darkblue}{\textbf{\ipa{ɭɯ˧}}} 
\lhead{\firstmark}
\rhead{\botmark}

\subsection{\hspace{-0.5cm} {\Large \textcolor{darkblue}{\textbf{\ipa{bi˩tɑ˩}}}}\hspace{0.5cm}[\kern2pt{\textcolor{darkblue}{\textbf{\ipa{bi˩tɑ˥}}}}\kern2pt]} \hypertarget{bi\string_BtA\string_B1}{}
\markboth{\textcolor{darkblue}{\textbf{\ipa{bi˩tɑ˩}}}}{}
\textcolor{teal}{\mytextsc{verb}} \hspace{4pt} Tone: L.
\textcolor{Sepia}{\selectlanguage{english}To pull, to drag.} \zh{拖。}  ¶ \textcolor{darkblue}{\textbf{\ipa{bi˩tɑ˩-ze˥}}} \textcolor{Sepia}{\selectlanguage{english}\mytextsc{pfv}} \zh{拖了}  

\lhead{\firstmark}
\rhead{\botmark}

\subsection{\hspace{-0.5cm} {\Large \textcolor{darkblue}{\textbf{\ipa{bi˩-tsɯ˧tsɯ˧}}}}\hspace{0.5cm}[\kern2pt{\textcolor{darkblue}{\textbf{\ipa{xxxx non-correspondance entre le nombre de morphèmes et le nombre de tons de morphèmes}}}}\kern2pt]} \hypertarget{bi\string_B-tsM\string_MtsM\string_M1}{}
\markboth{\textcolor{darkblue}{\textbf{\ipa{bi˩-tsɯ˧tsɯ˧}}}}{}
\textcolor{teal}{\mytextsc{noun}} \hspace{4pt} Tone: L-.
\textcolor{Sepia}{\selectlanguage{english}Wild strawberry, \textit{Fragaria vesca}.} \zh{野草莓。} 
\lhead{\firstmark}
\rhead{\botmark}

\subsection{\hspace{-0.5cm} {\Large \textcolor{darkblue}{\textbf{\ipa{bi˩ʈʂʰɤ\#˥}}}}\hspace{0.5cm}[\kern2pt{\textcolor{darkblue}{\textbf{\ipa{bi˧ʈʂʰɤ˩}}}}\kern2pt]} \hypertarget{bi\string_Bt`s`\string_h7\#\string_T1}{}
\markboth{\textcolor{darkblue}{\textbf{\ipa{bi˩ʈʂʰɤ\#˥}}}}{}
\textcolor{teal}{\mytextsc{noun}} \hspace{4pt} Tone: LM+\#H.
\textcolor{Sepia}{\selectlanguage{english}Whiskers.} \zh{胡须。}  \zh{量词}: \textcolor{darkblue}{\textbf{\ipa{kʰwɤ˥}}}  \mytextsc{clf}: \textcolor{darkblue}{\textbf{\ipa{kʰwɤ˥}}} 
\lhead{\firstmark}
\rhead{\botmark}

\subsection{\hspace{-0.5cm} {\Large \textcolor{darkblue}{\textbf{\ipa{bi˩wɤ˧}}}}\hspace{0.5cm}[\kern2pt{\textcolor{darkblue}{\textbf{\ipa{xxxx non-correspondance entre le nombre de morphèmes et le nombre de tons de morphèmes}}}}\kern2pt]} \hypertarget{bi\string_Bw7\string_M1}{}
\markboth{\textcolor{darkblue}{\textbf{\ipa{bi˩wɤ˧}}}}{}
\textcolor{teal}{\mytextsc{noun}} \hspace{4pt} Tone: LM.
\textcolor{Sepia}{\selectlanguage{english}Services (or money) offered as a reward to a priest.} \zh{酬劳。}  \zh{量词}: \textcolor{darkblue}{\textbf{\ipa{ɭɯ˧}}}  \mytextsc{clf}: \textcolor{darkblue}{\textbf{\ipa{ɭɯ˧}}} 
\lhead{\firstmark}
\rhead{\botmark}

\subsection{\hspace{-0.5cm} {\Large \textcolor{darkblue}{\textbf{\ipa{bo}}}}\hspace{0.5cm}[\kern2pt{\textcolor{darkblue}{\textbf{\ipa{[]}}}}\kern2pt]} \hypertarget{bo1}{}
\markboth{\textcolor{darkblue}{\textbf{\ipa{bo}}}}{}
\textcolor{teal}{\mytextsc{discourse}} \textcolor{teal}{\mytextsc{particle}} \hspace{4pt} Tone: 0.
\textcolor{Sepia}{\selectlanguage{english}Final particle expressing vigorous affirmation.} \zh{句尾助词:啵(汉语借词)。}  Borrowing: Chinese  \zh{啵}

\lhead{\firstmark}
\rhead{\botmark}

\subsection{\hspace{-0.5cm} {\Large \textcolor{darkblue}{\textbf{\ipa{bo˧}}} \textsubscript{1}}\hspace{0.5cm}[\kern2pt{\textcolor{darkblue}{\textbf{\ipa{bo˩˥}}}}\kern2pt]} \hypertarget{bo\string_M1}{}
\markboth{\textcolor{darkblue}{\textbf{\ipa{bo˧}}} \textsubscript{1}}{}
\textcolor{teal}{\mytextsc{noun}} \hspace{4pt} Tone: M.
\textcolor{Sepia}{\selectlanguage{english}Small dike (as at the edge of a field; one can walk on it).} \zh{田埂、小堤坝。}  ¶ \textcolor{darkblue}{\textbf{\ipa{ʈʂʰɯ˧-qo˧ | bo˧ ɖɯ˧-ɭɯ˧ tʰi˧-di˩.}}} \textcolor{Sepia}{\selectlanguage{english}Here, there is a small dike.} \zh{这里有一个小堤坝。}  
 ¶ \textcolor{darkblue}{\textbf{\ipa{bo˧-kʰi˧}}} \textcolor{Sepia}{\selectlanguage{english}the edge of the small dike} \zh{小堤坝的边沿}  
 ¶ \textcolor{darkblue}{\textbf{\ipa{[F5] bo˧ ɖɯ˧-pʰæ˧˥}}} \textcolor{Sepia}{\selectlanguage{english}a small dike} \zh{一个小堤坝}  
 \zh{量词}: \textcolor{darkblue}{\textbf{\ipa{ɭɯ˧}}}  \mytextsc{clf}: \textcolor{darkblue}{\textbf{\ipa{ɭɯ˧}}} 
\lhead{\firstmark}
\rhead{\botmark}

\subsection{\hspace{-0.5cm} {\Large \textcolor{darkblue}{\textbf{\ipa{bo˧}}} \textsubscript{2}}\hspace{0.5cm}[\kern2pt{\textcolor{darkblue}{\textbf{\ipa{bo˥}}}}\kern2pt]} \hypertarget{bo\string_M2}{}
\markboth{\textcolor{darkblue}{\textbf{\ipa{bo˧}}} \textsubscript{2}}{}
\textcolor{teal}{\mytextsc{adjective}} \hspace{4pt} Tone: M.
\textcolor{Sepia}{\selectlanguage{english}Bright, shining.} \zh{亮,光明。}  ¶ \textcolor{darkblue}{\textbf{\ipa{tʰi˧-bo˧-dʑo˧}}} \textcolor{Sepia}{\selectlanguage{english}It casts light / it is bright. (Definition of a lamp.)} \zh{(它)在发光。(描写灯)}  
 ¶ \textcolor{darkblue}{\textbf{\ipa{bo˧-hĩ˧}}} \textcolor{Sepia}{\selectlanguage{english}\mytextsc{rel}} \zh{发亮的}  

\lhead{\firstmark}
\rhead{\botmark}

\subsection{\hspace{-0.5cm} {\Large \textcolor{darkblue}{\textbf{\ipa{bo˧\textsubscript{b}}}}}\hspace{0.5cm}[\kern2pt{\textcolor{darkblue}{\textbf{\ipa{bo˩˥}}}}\kern2pt]} \hypertarget{bo\string_Mb1}{}
\markboth{\textcolor{darkblue}{\textbf{\ipa{bo˧\textsubscript{b}}}}}{}
\textcolor{teal}{\mytextsc{verb}} \hspace{4pt} Tone: M\textsubscript{b}.
\textcolor{Sepia}{\selectlanguage{english}To spin.} \zh{纺(麻线),使旋转。}  ¶ \textcolor{darkblue}{\textbf{\ipa{tso˧\textasciitilde{}tso˧ bo˧}}} \textcolor{Sepia}{\selectlanguage{english}to spin something, to spin things} \zh{使东西旋转}  
\textit{See:} \hyperlink{}{\textcolor{darkblue}{\textbf{\ipa{sɑ˧bo\#˥}}}} 
\lhead{\firstmark}
\rhead{\botmark}

\subsection{\hspace{-0.5cm} {\Large \textcolor{darkblue}{\textbf{\ipa{bo˧tsi˩}}}}\hspace{0.5cm}[\kern2pt{\textcolor{darkblue}{\textbf{\ipa{bo˩tsi˥}}}}\kern2pt]} \hypertarget{bo\string_Mtsi\string_B1}{}
\markboth{\textcolor{darkblue}{\textbf{\ipa{bo˧tsi˩}}}}{}
\textcolor{teal}{\mytextsc{noun}} \hspace{4pt} Tone: L\#.
\textcolor{Sepia}{\selectlanguage{english}Mane.} \zh{(马)鬃。}  ¶ \textcolor{darkblue}{\textbf{\ipa{ʐwæ˧-bo˧tsi˥\#}}} \textcolor{Sepia}{\selectlanguage{english}horse mane} \zh{马鬃}  
 ¶ \textcolor{darkblue}{\textbf{\ipa{bo˩-bo˧tsi˩}}} \textcolor{Sepia}{\selectlanguage{english}hog bristle} \zh{猪鬃}  
 \zh{量词}: \textcolor{darkblue}{\textbf{\ipa{kʰwɤ˥}}} \textcolor{darkblue}{\textbf{\ipa{qɑ˩}}}  \mytextsc{clf}: \textcolor{darkblue}{\textbf{\ipa{kʰwɤ˥}}} \textcolor{darkblue}{\textbf{\ipa{qɑ˩}}} 
\lhead{\firstmark}
\rhead{\botmark}

\subsection{\hspace{-0.5cm} {\Large \textcolor{darkblue}{\textbf{\ipa{bo˧ʐæ˧}}}}\hspace{0.5cm}[\kern2pt{\textcolor{darkblue}{\textbf{\ipa{bo˩ʐæ˥}}}}\kern2pt]} \hypertarget{bo\string_Mz`\{\string_M1}{}
\markboth{\textcolor{darkblue}{\textbf{\ipa{bo˧ʐæ˧}}}}{}
\textcolor{teal}{\mytextsc{noun}} \hspace{4pt} Tone: M.
\textcolor{Sepia}{\selectlanguage{english}Glass (as a substance: window panes, drinking glasses... are made of glass).} \zh{玻璃。}  ¶ \textcolor{darkblue}{\textbf{\ipa{bo˧ʐæ˧-tɕʰɤ˩ʈʂv˩}}} \textcolor{Sepia}{\selectlanguage{english}goblet for drinking tea (made of glass)} \zh{玻璃茶杯}  

\lhead{\firstmark}
\rhead{\botmark}

\subsection{\hspace{-0.5cm} {\Large \textcolor{darkblue}{\textbf{\ipa{bo˧ʐæ˧ʈæ˧qʰv̩\#˥}}}}\hspace{0.5cm}[\kern2pt{\textcolor{darkblue}{\textbf{\ipa{bo˩ʐæ˧ʈæ˧qʰv̩˧}}}}\kern2pt]} \hypertarget{bo\string_Mz`\{\string_Mt`\{\string_Mq\string_hv\string_=\#\string_T1}{}
\markboth{\textcolor{darkblue}{\textbf{\ipa{bo˧ʐæ˧ʈæ˧qʰv̩\#˥}}}}{}
\textcolor{teal}{\mytextsc{noun}} \hspace{4pt} Tone: \#H.
\textcolor{Sepia}{\selectlanguage{english}Echo (in some places in the mountains, there is an echo).} \zh{回音。}  ¶ \textcolor{darkblue}{\textbf{\ipa{bo˧ʐæ˧ʈæ˧qʰv̩˧ | tʰi˧-ɖʐɯ˩\textasciitilde{}ɖʐɯ˩!}}} \textcolor{Sepia}{\selectlanguage{english}the echo resonates} \zh{有回音}  

\lhead{\firstmark}
\rhead{\botmark}

\subsection{\hspace{-0.5cm} {\Large \textcolor{darkblue}{\textbf{\ipa{bo˩\textsubscript{a}}}}}\hspace{0.5cm}[\kern2pt{\textcolor{darkblue}{\textbf{\ipa{bo˥}}}}\kern2pt]} \hypertarget{bo\string_Ba1}{}
\markboth{\textcolor{darkblue}{\textbf{\ipa{bo˩\textsubscript{a}}}}}{}
\textcolor{teal}{\mytextsc{verb}} \hspace{4pt} Tone: L\textsubscript{a}.
\textcolor{Sepia}{\selectlanguage{english}To kiss.} \zh{亲吻。}  ¶ \textcolor{darkblue}{\textbf{\ipa{ɖɯ˧-bo˩-ɻ̍˩}}} \textcolor{Sepia}{\selectlanguage{english}\mytextsc{delimitative} \string_ \mytextsc{inceptive}} \zh{吻一下}  
 ¶ \textcolor{darkblue}{\textbf{\ipa{ɖɯ˧-bo˧\textasciitilde{}bo˥-ɻ̍˩}}} \textcolor{Sepia}{\selectlanguage{english}\mytextsc{delimitative} \string_ \mytextsc{red} \mytextsc{inceptive}} \zh{吻一下}  

\lhead{\firstmark}
\rhead{\botmark}

\subsection{\hspace{-0.5cm} {\Large \textcolor{darkblue}{\textbf{\ipa{bo˩\textsubscript{b}}}}}\hspace{0.5cm}[\kern2pt{\textcolor{darkblue}{\textbf{\ipa{bo˧˥}}}}\kern2pt]} \hypertarget{bo\string_Bb1}{}
\markboth{\textcolor{darkblue}{\textbf{\ipa{bo˩\textsubscript{b}}}}}{}
\textcolor{teal}{\mytextsc{classifier}} \hspace{4pt} Tone: L\textsubscript{b}.
\textcolor{Sepia}{\selectlanguage{english}Classifier for women's traditional hair dresses / headdresses.} \zh{量词:缎子发带(一条)。} 
\lhead{\firstmark}
\rhead{\botmark}

\subsection{\hspace{-0.5cm} {\Large \textcolor{darkblue}{\textbf{\ipa{bo˩-bi˧mi˧}}}}\hspace{0.5cm}[\kern2pt{\textcolor{darkblue}{\textbf{\ipa{xxxx non-correspondance entre le nombre de morphèmes et le nombre de tons de morphèmes}}}}\kern2pt]} \hypertarget{bo\string_B-bi\string_Mmi\string_M1}{}
\markboth{\textcolor{darkblue}{\textbf{\ipa{bo˩-bi˧mi˧}}}}{}
\textcolor{teal}{\mytextsc{noun}} \hspace{4pt} Tone: L-.
\textcolor{Sepia}{\selectlanguage{english}Meat of the pig's belly.} \zh{猪肚肉。}  \zh{量词}: \textcolor{darkblue}{\textbf{\ipa{ɭɯ˧}}}  \mytextsc{clf}: \textcolor{darkblue}{\textbf{\ipa{ɭɯ˧}}} 
\lhead{\firstmark}
\rhead{\botmark}

\subsection{\hspace{-0.5cm} {\Large \textcolor{darkblue}{\textbf{\ipa{bo˩-bv̩˥}}}}\hspace{0.5cm}[\kern2pt{\textcolor{darkblue}{\textbf{\ipa{bo˧bv̩˧}}}}\kern2pt]} \hypertarget{bo\string_B-bv\string_=\string_T1}{}
\markboth{\textcolor{darkblue}{\textbf{\ipa{bo˩-bv̩˥}}}}{}
\textcolor{teal}{\mytextsc{noun}} \hspace{4pt} Tone: LH.
\textcolor{Sepia}{\selectlanguage{english}Pigsty, pigpen.} \zh{猪圈。}  \zh{量词}: \textcolor{darkblue}{\textbf{\ipa{ɭɯ˧}}}  \mytextsc{clf}: \textcolor{darkblue}{\textbf{\ipa{ɭɯ˧}}} 
\lhead{\firstmark}
\rhead{\botmark}

\subsection{\hspace{-0.5cm} {\Large \textcolor{darkblue}{\textbf{\ipa{bo˩dze˧}}}}\hspace{0.5cm}[\kern2pt{\textcolor{darkblue}{\textbf{\ipa{bo˩dze˥}}}}\kern2pt]} \hypertarget{bo\string_Bdze\string_M1}{}
\markboth{\textcolor{darkblue}{\textbf{\ipa{bo˩dze˧}}}}{}
\textcolor{teal}{\mytextsc{noun}} \hspace{4pt} Tone: LM.
\textcolor{Sepia}{\selectlanguage{english}Lark.} \zh{百灵鸟。} \textit{See:} \hyperlink{}{\textcolor{darkblue}{\textbf{\ipa{bo˩dze˧-ko˩dze˩}}}} 
\lhead{\firstmark}
\rhead{\botmark}

\subsection{\hspace{-0.5cm} {\Large \textcolor{darkblue}{\textbf{\ipa{bo˩dze˧-ko˩dze˩}}}}\hspace{0.5cm}[\kern2pt{\textcolor{darkblue}{\textbf{\ipa{xxxx non-correspondance entre le nombre de morphèmes et le nombre de tons de morphèmes}}}}\kern2pt]} \hypertarget{bo\string_Bdze\string_M-ko\string_Bdze\string_B1}{}
\markboth{\textcolor{darkblue}{\textbf{\ipa{bo˩dze˧-ko˩dze˩}}}}{}
\textcolor{teal}{\mytextsc{noun}} \hspace{4pt} Tone: LM-L.
\textcolor{Sepia}{\selectlanguage{english}Lark.} \zh{百灵鸟。} \textit{See:} \hyperlink{}{\textcolor{darkblue}{\textbf{\ipa{bo˩dze˧}}}} 
\lhead{\firstmark}
\rhead{\botmark}

\subsection{\hspace{-0.5cm} {\Large \textcolor{darkblue}{\textbf{\ipa{bo˩-ɣɯ˥}}}}\hspace{0.5cm}[\kern2pt{\textcolor{darkblue}{\textbf{\ipa{xxxx non-correspondance entre le nombre de morphèmes et le nombre de tons de morphèmes}}}}\kern2pt]} \hypertarget{bo\string_B-GM\string_T1}{}
\markboth{\textcolor{darkblue}{\textbf{\ipa{bo˩-ɣɯ˥}}}}{}
\textcolor{teal}{\mytextsc{noun}} \hspace{4pt} Tone: LH.
\textcolor{Sepia}{\selectlanguage{english}Pigskin, hogskin.} \zh{猪皮。}  ¶ \textcolor{darkblue}{\textbf{\ipa{bo˩-ɣɯ˧kɯ˩}}} \textcolor{Sepia}{\selectlanguage{english}same meaning: pigskin} \zh{同上:猪皮}  
 \zh{量词}: \textcolor{darkblue}{\textbf{\ipa{tsʰi˥}}}  \mytextsc{clf}: \textcolor{darkblue}{\textbf{\ipa{tsʰi˥}}} 
\lhead{\firstmark}
\rhead{\botmark}

\subsection{\hspace{-0.5cm} {\Large \textcolor{darkblue}{\textbf{\ipa{bo˩-hɑ\#˥}}}}\hspace{0.5cm}[\kern2pt{\textcolor{darkblue}{\textbf{\ipa{xxxx non-correspondance entre le nombre de morphèmes et le nombre de tons de morphèmes}}}}\kern2pt]} \hypertarget{bo\string_B-hA\#\string_T1}{}
\markboth{\textcolor{darkblue}{\textbf{\ipa{bo˩-hɑ\#˥}}}}{}
\textcolor{teal}{\mytextsc{noun}} \hspace{4pt} Tone: LM+\#H.
\textcolor{Sepia}{\selectlanguage{english}Pig feed, swill.} \zh{猪食。}  \zh{量词}: \textcolor{darkblue}{\textbf{\ipa{kʰɤ˧˥}}}  \mytextsc{clf}: \textcolor{darkblue}{\textbf{\ipa{kʰɤ˧˥}}} 
\lhead{\firstmark}
\rhead{\botmark}

\subsection{\hspace{-0.5cm} {\Large \textcolor{darkblue}{\textbf{\ipa{bo˩-kʰɯ˧}}}}\hspace{0.5cm}[\kern2pt{\textcolor{darkblue}{\textbf{\ipa{xxxx non-correspondance entre le nombre de morphèmes et le nombre de tons de morphèmes}}}}\kern2pt]} \hypertarget{bo\string_B-k\string_hM\string_M1}{}
\markboth{\textcolor{darkblue}{\textbf{\ipa{bo˩-kʰɯ˧}}}}{}
\textcolor{teal}{\mytextsc{noun}} \hspace{4pt} Tone: LM.
\textcolor{Sepia}{\selectlanguage{english}Pig's feet: dried meat preserved in the skin of the pig's foot.} \zh{猪脚腊肉:把猪腿的皮剥下来,缝成筒形,塞满瘦肉。}  \zh{量词}: \textcolor{darkblue}{\textbf{\ipa{ɭɯ˧}}}  \mytextsc{clf}: \textcolor{darkblue}{\textbf{\ipa{ɭɯ˧}}} 
\lhead{\firstmark}
\rhead{\botmark}

\subsection{\hspace{-0.5cm} {\Large \textcolor{darkblue}{\textbf{\ipa{bo˩-kʰv̩˧˥}}}}\hspace{0.5cm}[\kern2pt{\textcolor{darkblue}{\textbf{\ipa{xxxx non-correspondance entre le nombre de morphèmes et le nombre de tons de morphèmes}}}}\kern2pt]} \hypertarget{bo\string_B-k\string_hv\string_=\string_M\string_T1}{}
\markboth{\textcolor{darkblue}{\textbf{\ipa{bo˩-kʰv̩˧˥}}}}{}
\textcolor{teal}{\mytextsc{noun}} \hspace{4pt} Tone: LM+MH\#.
\textcolor{Sepia}{\selectlanguage{english}Year of the pig.} \zh{猪年。} 
\lhead{\firstmark}
\rhead{\botmark}

\subsection{\hspace{-0.5cm} {\Large \textcolor{darkblue}{\textbf{\ipa{bo˩lo˧}}}}\hspace{0.5cm}[\kern2pt{\textcolor{darkblue}{\textbf{\ipa{bo˩lo˥}}}}\kern2pt]} \hypertarget{bo\string_Blo\string_M1}{}
\markboth{\textcolor{darkblue}{\textbf{\ipa{bo˩lo˧}}}}{}
\textcolor{teal}{\mytextsc{noun}} \hspace{4pt} Tone: LM.
\textcolor{Sepia}{\selectlanguage{english}Mortise.} \zh{榫眼。}  ¶ \textcolor{darkblue}{\textbf{\ipa{bo˩lo˧ | ɖɯ˧-ɭɯ˧}}} \textcolor{Sepia}{\selectlanguage{english}a mortise} \zh{一个榫眼}  
 \zh{量词}: \textcolor{darkblue}{\textbf{\ipa{ɭɯ˧}}}  \mytextsc{clf}: \textcolor{darkblue}{\textbf{\ipa{ɭɯ˧}}} 
\lhead{\firstmark}
\rhead{\botmark}

\subsection{\hspace{-0.5cm} {\Large \textcolor{darkblue}{\textbf{\ipa{bo˩ɬɑ˥}}}}\hspace{0.5cm}[\kern2pt{\textcolor{darkblue}{\textbf{\ipa{bo˩ɬɑ˧˥}}}}\kern2pt]} \hypertarget{bo\string_BKA\string_T1}{}
\markboth{\textcolor{darkblue}{\textbf{\ipa{bo˩ɬɑ˥}}}}{}
\textcolor{teal}{\mytextsc{noun}} \hspace{4pt} Tone: LH.
\textcolor{Sepia}{\selectlanguage{english}Boar.} \zh{种公猪。}  \zh{量词}: \textcolor{darkblue}{\textbf{\ipa{v̩˧}}}  \mytextsc{clf}: \textcolor{darkblue}{\textbf{\ipa{v̩˧}}} 
\lhead{\firstmark}
\rhead{\botmark}

\subsection{\hspace{-0.5cm} {\Large \textcolor{darkblue}{\textbf{\ipa{bo˩-ɬo˥}}}}\hspace{0.5cm}[\kern2pt{\textcolor{darkblue}{\textbf{\ipa{xxxx non-correspondance entre le nombre de morphèmes et le nombre de tons de morphèmes}}}}\kern2pt]} \hypertarget{bo\string_B-Ko\string_T1}{}
\markboth{\textcolor{darkblue}{\textbf{\ipa{bo˩-ɬo˥}}}}{}
\textcolor{teal}{\mytextsc{noun}} \hspace{4pt} Tone: LH.
\textcolor{Sepia}{\selectlanguage{english}Pork ribs.} \zh{猪肋骨。}  ¶ \textcolor{darkblue}{\textbf{\ipa{bo˩ɬo˥ | ɖɯ˧-do˥}}} \textcolor{Sepia}{\selectlanguage{english}a large piece of pork ribs; this also designates (metaphorically) a family, to emphasize the strong bonds between all family members} \zh{一大块猪肋骨。也来比喻一个家庭,强调家所有成员之间的密切关系。}  
 \zh{量词}: \textcolor{darkblue}{\textbf{\ipa{ɭɯ˧}}}  \mytextsc{clf}: \textcolor{darkblue}{\textbf{\ipa{ɭɯ˧}}} 
\lhead{\firstmark}
\rhead{\botmark}

\subsection{\hspace{-0.5cm} {\Large \textcolor{darkblue}{\textbf{\ipa{bo˩-mæ˧qv̩˩}}}}\hspace{0.5cm}[\kern2pt{\textcolor{darkblue}{\textbf{\ipa{xxxx non-correspondance entre le nombre de morphèmes et le nombre de tons de morphèmes}}}}\kern2pt]} \hypertarget{bo\string_B-m\{\string_Mqv\string_=\string_B1}{}
\markboth{\textcolor{darkblue}{\textbf{\ipa{bo˩-mæ˧qv̩˩}}}}{}
\textcolor{teal}{\mytextsc{noun}} \hspace{4pt} Tone: L-L\#.
\textcolor{Sepia}{\selectlanguage{english}Pig's tail.} \zh{猪尾巴。}  \zh{量词}: \textcolor{darkblue}{\textbf{\ipa{ɭɯ˧}}}  \mytextsc{clf}: \textcolor{darkblue}{\textbf{\ipa{ɭɯ˧}}} 
\lhead{\firstmark}
\rhead{\botmark}

\subsection{\hspace{-0.5cm} {\Large \textcolor{darkblue}{\textbf{\ipa{bo˩-mɤ˥}}}}\hspace{0.5cm}[\kern2pt{\textcolor{darkblue}{\textbf{\ipa{xxxx non-correspondance entre le nombre de morphèmes et le nombre de tons de morphèmes}}}}\kern2pt]} \hypertarget{bo\string_B-m7\string_T1}{}
\markboth{\textcolor{darkblue}{\textbf{\ipa{bo˩-mɤ˥}}}}{}
\textcolor{teal}{\mytextsc{noun}} \hspace{4pt} Tone: LH.
\textcolor{Sepia}{\selectlanguage{english}Lard.} \zh{猪油。} 
\lhead{\firstmark}
\rhead{\botmark}

\subsection{\hspace{-0.5cm} {\Large \textcolor{darkblue}{\textbf{\ipa{bo˩mi˧}}}}\hspace{0.5cm}[\kern2pt{\textcolor{darkblue}{\textbf{\ipa{xxxx non-correspondance entre le nombre de morphèmes et le nombre de tons de morphèmes}}}}\kern2pt]} \hypertarget{bo\string_Bmi\string_M1}{}
\markboth{\textcolor{darkblue}{\textbf{\ipa{bo˩mi˧}}}}{}
\textcolor{teal}{\mytextsc{noun}} \hspace{4pt} Tone: LM.
\textcolor{Sepia}{\selectlanguage{english}Sow.} \zh{母猪。}  ¶ \textcolor{darkblue}{\textbf{\ipa{bo˩mi˧ ʑi˩}}} \textcolor{Sepia}{\selectlanguage{english}to catch (a/the) sow} \zh{抓母猪}  
 ¶ \textcolor{darkblue}{\textbf{\ipa{bo˩mi˧ do˧ (+ze˩)}}} \textcolor{Sepia}{\selectlanguage{english}...has seen (a/the) sow} \zh{见了母猪}  
 ¶ \textcolor{darkblue}{\textbf{\ipa{bo˩mi˧-bæ˧bv̩˥}}} \textcolor{Sepia}{\selectlanguage{english}sow and piglets} \zh{母猪与猪仔}  
 \zh{量词}: \textcolor{darkblue}{\textbf{\ipa{mi˩}}} \textcolor{darkblue}{\textbf{\ipa{v̩˧}}}  \mytextsc{clf}: \textcolor{darkblue}{\textbf{\ipa{mi˩}}} \textcolor{darkblue}{\textbf{\ipa{v̩˧}}} 
\lhead{\firstmark}
\rhead{\botmark}

\subsection{\hspace{-0.5cm} {\Large \textcolor{darkblue}{\textbf{\ipa{bo˩mi˧-dʑɯ˩pv̩˩}}}}\hspace{0.5cm}[\kern2pt{\textcolor{darkblue}{\textbf{\ipa{xxxx non-correspondance entre le nombre de morphèmes et le nombre de tons de morphèmes}}}}\kern2pt]} \hypertarget{bo\string_Bmi\string_M-dz£M\string_Bpv\string_=\string_B1}{}
\markboth{\textcolor{darkblue}{\textbf{\ipa{bo˩mi˧-dʑɯ˩pv̩˩}}}}{}
\textcolor{teal}{\mytextsc{noun}} \hspace{4pt} Tone: LM-L.
\textcolor{Sepia}{\selectlanguage{english}\textit{(Dytiscus}, a predaceous diving beetle.} \zh{龙虱。} 
\lhead{\firstmark}
\rhead{\botmark}

\subsection{\hspace{-0.5cm} {\Large \textcolor{darkblue}{\textbf{\ipa{bo˩mi˧-dʑɯ˩pʰv̩˩}}}}\hspace{0.5cm}[\kern2pt{\textcolor{darkblue}{\textbf{\ipa{bo˩mi˧dʑɯ˩pʰv̩˩}}}}\kern2pt]} \hypertarget{bo\string_Bmi\string_M-dz£M\string_Bp\string_hv\string_=\string_B1}{}
\markboth{\textcolor{darkblue}{\textbf{\ipa{bo˩mi˧-dʑɯ˩pʰv̩˩}}}}{}
\textcolor{teal}{\mytextsc{noun}} \hspace{4pt} Tone: LM-L.
\textcolor{Sepia}{\selectlanguage{english}Weevil, snout beetle.} \zh{象鼻虫,米象。} 
\lhead{\firstmark}
\rhead{\botmark}

\subsection{\hspace{-0.5cm} {\Large \textcolor{darkblue}{\textbf{\ipa{bo˩mi˧-ɳæ˧tɕʰɯ˩}}}}\hspace{0.5cm}[\kern2pt{\textcolor{darkblue}{\textbf{\ipa{bo˩mi˧ɳæ˩tɕʰɯ˩}}}}\kern2pt]} \hypertarget{bo\string_Bmi\string_M-n`\{\string_Mts£\string_hM\string_B1}{}
\markboth{\textcolor{darkblue}{\textbf{\ipa{bo˩mi˧-ɳæ˧tɕʰɯ˩}}}}{}
\textcolor{teal}{\mytextsc{noun}} \hspace{4pt} Tone: LM-L\#.
\textcolor{Sepia}{\selectlanguage{english}Dandelion.} \zh{蒲公英。}  \zh{量词}: \textcolor{darkblue}{\textbf{\ipa{po˧}}}  \mytextsc{clf}: \textcolor{darkblue}{\textbf{\ipa{po˧}}} 
\lhead{\firstmark}
\rhead{\botmark}

\subsection{\hspace{-0.5cm} {\Large \textcolor{darkblue}{\textbf{\ipa{bo˩mi˧-ʁo˩do˩}}}}\hspace{0.5cm}[\kern2pt{\textcolor{darkblue}{\textbf{\ipa{bo˩mi˧ʁo˧do˩}}}}\kern2pt]} \hypertarget{bo\string_Bmi\string_M-Ro\string_Bdo\string_B1}{}
\markboth{\textcolor{darkblue}{\textbf{\ipa{bo˩mi˧-ʁo˩do˩}}}}{}
\textcolor{teal}{\mytextsc{noun}} \hspace{4pt} Tone: LM-L.
\textcolor{Sepia}{\selectlanguage{english}Yyyy.} \zh{曼陀罗。}  \zh{量词}: \textcolor{darkblue}{\textbf{\ipa{dzi˩}}}  \mytextsc{clf}: \textcolor{darkblue}{\textbf{\ipa{dzi˩}}} 
\lhead{\firstmark}
\rhead{\botmark}

\subsection{\hspace{-0.5cm} {\Large \textcolor{darkblue}{\textbf{\ipa{bo˩pʰv̩˧}}}}\hspace{0.5cm}[\kern2pt{\textcolor{darkblue}{\textbf{\ipa{bo˩pʰv̩˥}}}}\kern2pt]} \hypertarget{bo\string_Bp\string_hv\string_=\string_M1}{}
\markboth{\textcolor{darkblue}{\textbf{\ipa{bo˩pʰv̩˧}}}}{}
\textcolor{teal}{\mytextsc{noun}} \hspace{4pt} Tone: LM.
\textcolor{Sepia}{\selectlanguage{english}Boar.} \zh{种公猪。}  \zh{量词}: \textcolor{darkblue}{\textbf{\ipa{v̩˧}}}  \mytextsc{clf}: \textcolor{darkblue}{\textbf{\ipa{v̩˧}}} 
\lhead{\firstmark}
\rhead{\botmark}

\subsection{\hspace{-0.5cm} {\Large \textcolor{darkblue}{\textbf{\ipa{bo˩qʰæ˧-pv̩˧ʈɤ˥-ɻ̍˩}}}}\hspace{0.5cm}[\kern2pt{\textcolor{darkblue}{\textbf{\ipa{xxxx non-correspondance entre le nombre de morphèmes et le nombre de tons de morphèmes}}}}\kern2pt]} \hypertarget{bo\string_Bq\string_h\{\string_M-pv\string_=\string_Mt`7\string_T-r£`̍\string_B1}{}
\markboth{\textcolor{darkblue}{\textbf{\ipa{bo˩qʰæ˧-pv̩˧ʈɤ˥-ɻ̍˩}}}}{}
\textcolor{teal}{\mytextsc{noun}} \hspace{4pt} Tone: LM - H\# -.
\textcolor{Sepia}{\selectlanguage{english}Dung beetle.} \zh{蜣螂。} 
\lhead{\firstmark}
\rhead{\botmark}

\subsection{\hspace{-0.5cm} {\Large \textcolor{darkblue}{\textbf{\ipa{bo˩tv̩\#˥}}}}\hspace{0.5cm}[\kern2pt{\textcolor{darkblue}{\textbf{\ipa{bo˧tv̩˩}}}}\kern2pt]} \hypertarget{bo\string_Btv\string_=\#\string_T1}{}
\markboth{\textcolor{darkblue}{\textbf{\ipa{bo˩tv̩\#˥}}}}{}
\textcolor{teal}{\mytextsc{noun}} \hspace{4pt} Tone: LM+\#H.
\textcolor{Sepia}{\selectlanguage{english}Wild boar.} \zh{野猪。}  ¶ \textcolor{darkblue}{\textbf{\ipa{bo˩tv̩˧ hwæ˥}}} \textcolor{Sepia}{\selectlanguage{english}to buy (a/the) wild boar} \zh{买野猪}  
 \zh{量词}: \textcolor{darkblue}{\textbf{\ipa{mi˩}}}  \mytextsc{clf}: \textcolor{darkblue}{\textbf{\ipa{mi˩}}} 
\lhead{\firstmark}
\rhead{\botmark}

\subsection{\hspace{-0.5cm} {\Large \textcolor{darkblue}{\textbf{\ipa{bo˩ʈʂʰæ˥}}}}\hspace{0.5cm}[\kern2pt{\textcolor{darkblue}{\textbf{\ipa{bo˩ʈʂʰæ˩˥}}}}\kern2pt]} \hypertarget{bo\string_Bt`s`\string_h\{\string_T1}{}
\markboth{\textcolor{darkblue}{\textbf{\ipa{bo˩ʈʂʰæ˥}}}}{}
\textcolor{teal}{\mytextsc{noun}} \hspace{4pt} Tone: LH.
\textcolor{Sepia}{\selectlanguage{english}Lard, fat meat of pig; also: boneless, fleshless preserved pork: cured pork made from a whole pig by removing all its internal organs from the opened stomach, seasoned with salt and spices and then the opening is stitched together. The whole sewn pig is then pressed with a slabstone.} \zh{猪膘,琵琶肉。} 
\lhead{\firstmark}
\rhead{\botmark}

\subsection{\hspace{-0.5cm} {\Large \textcolor{darkblue}{\textbf{\ipa{bo˩zɑ˧mi\#˥}}}}\hspace{0.5cm}[\kern2pt{\textcolor{darkblue}{\textbf{\ipa{bo˧zɑ˧mi˧}}}}\kern2pt]} \hypertarget{bo\string_BzA\string_Mmi\#\string_T1}{}
\markboth{\textcolor{darkblue}{\textbf{\ipa{bo˩zɑ˧mi\#˥}}}}{}
\textcolor{teal}{\mytextsc{noun}} \hspace{4pt} Tone: LM+\#H.
\textcolor{Sepia}{\selectlanguage{english}“Piggy-Sow”: a term used as a temporary name for little girls, during the first months of their life, before they are given a real name. This ugly term is intended to disgust evil spirits, which will therefore turn their attention away from the infant. (In the early 21st century, the registry office requires a name to be given at birth; but this name only begins to be used by the family after the first months of life have elapsed.).} \zh{猪崽子(给刚出生的女孩起的名字,让鬼对她不感兴趣,不会来害小孩)。} 
\lhead{\firstmark}
\rhead{\botmark}

\subsection{\hspace{-0.5cm} {\Large \textcolor{darkblue}{\textbf{\ipa{bo˩˧}}}}\hspace{0.5cm}[\kern2pt{\textcolor{darkblue}{\textbf{\ipa{xxxx groupe tonal entier sans aucun ton}}}}\kern2pt]} \hypertarget{bo\string_B\string_M1}{}
\markboth{\textcolor{darkblue}{\textbf{\ipa{bo˩˧}}}}{}
\textcolor{teal}{\mytextsc{noun}} \hspace{4pt} Tone: LM.
\textcolor{Sepia}{\selectlanguage{english}Pig.} \zh{猪。}  ¶ \textcolor{darkblue}{\textbf{\ipa{bo˩ hwæ˧-ze˧}}} \textcolor{Sepia}{\selectlanguage{english}...bought (some/a) pig} \zh{买了猪}  
 ¶ \textcolor{darkblue}{\textbf{\ipa{bo˩ dzɯ˥-ze˩}}} \textcolor{Sepia}{\selectlanguage{english}...ate (some/the) pig} \zh{吃了猪}  
 \zh{量词}: \textcolor{darkblue}{\textbf{\ipa{mi˩}}} \textcolor{darkblue}{\textbf{\ipa{v̩˧}}}  \mytextsc{clf}: \textcolor{darkblue}{\textbf{\ipa{mi˩}}} \textcolor{darkblue}{\textbf{\ipa{v̩˧}}} 
\lhead{\firstmark}
\rhead{\botmark}

\subsection{\hspace{-0.5cm} {\Large \textcolor{darkblue}{\textbf{\ipa{bõ}}}}\hspace{0.5cm}[\kern2pt{\textcolor{darkblue}{\textbf{\ipa{xxxx groupe tonal entier sans aucun ton}}}}\kern2pt]} \hypertarget{bo\string_~1}{}
\markboth{\textcolor{darkblue}{\textbf{\ipa{bõ}}}}{}
\textcolor{teal}{\mytextsc{ideophone}} \hspace{4pt} Tone: 0.
\textcolor{Sepia}{\selectlanguage{english}Noise of a shock between two hard objects, for instance the sound of an axe hitting a tree trunk: Bang!} \zh{形声词:斧头砍树。砰! / 啪!。} 
\lhead{\firstmark}
\rhead{\botmark}

\subsection{\hspace{-0.5cm} {\Large \textcolor{darkblue}{\textbf{\ipa{bv̩˩}}}}\hspace{0.5cm}[\kern2pt{\textcolor{darkblue}{\textbf{\ipa{bv̩˥}}}}\kern2pt]} \hypertarget{bv\string_=\string_B1}{}
\markboth{\textcolor{darkblue}{\textbf{\ipa{bv̩˩}}}}{}
\textcolor{teal}{\mytextsc{noun}} \hspace{4pt} Tone: L.
\textcolor{Sepia}{\selectlanguage{english}Pen, corral for cattle.} \zh{牲畜圈(单音节)。}  ¶ \textcolor{darkblue}{\textbf{\ipa{bv̩˩-qo˩}}} \textcolor{Sepia}{\selectlanguage{english}inside the corral} \zh{牲畜圈里面}  
 ¶ \textcolor{darkblue}{\textbf{\ipa{bv̩˩qo˩ ʈæ˧}}} \textcolor{Sepia}{\selectlanguage{english}to enclose (cattle) inside the pen} \zh{关在圈里}  
 \zh{量词}: \textcolor{darkblue}{\textbf{\ipa{ɭɯ˧}}}  \mytextsc{clf}: \textcolor{darkblue}{\textbf{\ipa{ɭɯ˧}}} 
\lhead{\firstmark}
\rhead{\botmark}

\subsection{\hspace{-0.5cm} {\Large \textcolor{darkblue}{\textbf{\ipa{bv̩˩˧}}} \textsubscript{1}}\hspace{0.5cm}[\kern2pt{\textcolor{darkblue}{\textbf{\ipa{bv̩˥}}}}\kern2pt]} \hypertarget{bv\string_=\string_B\string_M1}{}
\markboth{\textcolor{darkblue}{\textbf{\ipa{bv̩˩˧}}} \textsubscript{1}}{}
\textcolor{teal}{\mytextsc{noun}} \hspace{4pt} Tone: LM.
\textcolor{Sepia}{\selectlanguage{english}Yak, Bos grunniens. The same term is used for wild yaks and domesticated yaks.} \zh{牦牛/野牦牛。}  ¶ \textcolor{darkblue}{\textbf{\ipa{bv̩˩-hṽ˩˥}}} \textcolor{Sepia}{\selectlanguage{english}yak hair} \zh{牦牛毛}  
 ¶ \textcolor{darkblue}{\textbf{\ipa{bv̩˩ dzɯ˧-ze˩}}} \textcolor{Sepia}{\selectlanguage{english}...ate (some) yak} \zh{吃了牦牛}  
 ¶ \textcolor{darkblue}{\textbf{\ipa{bv̩˩ hwæ˧-ze˧}}} \textcolor{Sepia}{\selectlanguage{english}...bought (some) yak} \zh{买了牦牛}  
 \zh{量词}: \textcolor{darkblue}{\textbf{\ipa{pʰo˧˥}}}  \mytextsc{clf}: \textcolor{darkblue}{\textbf{\ipa{pʰo˧˥}}} 
\lhead{\firstmark}
\rhead{\botmark}

\subsection{\hspace{-0.5cm} {\Large \textcolor{darkblue}{\textbf{\ipa{bv̩˩˧}}} \textsubscript{2}}\hspace{0.5cm}[\kern2pt{\textcolor{darkblue}{\textbf{\ipa{bv̩˩˥}}}}\kern2pt]} \hypertarget{bv\string_=\string_B\string_M2}{}
\markboth{\textcolor{darkblue}{\textbf{\ipa{bv̩˩˧}}} \textsubscript{2}}{}
\textcolor{teal}{\mytextsc{noun}} \hspace{4pt} Tone: LM.
\textcolor{Sepia}{\selectlanguage{english}Food steamer.} \zh{蒸笼。}  \zh{量词}: \textcolor{darkblue}{\textbf{\ipa{mi˩}}}  \mytextsc{clf}: \textcolor{darkblue}{\textbf{\ipa{mi˩}}} \textit{See:} \hyperlink{}{\textcolor{darkblue}{\textbf{\ipa{bv̩˩di˩}}}} 
\lhead{\firstmark}
\rhead{\botmark}

\subsection{\hspace{-0.5cm} {\Large \textcolor{darkblue}{\textbf{\ipa{bv̩˧}}}}\hspace{0.5cm}[\kern2pt{\textcolor{darkblue}{\textbf{\ipa{bv̩˩˥}}}}\kern2pt]} \hypertarget{bv\string_=\string_M1}{}
\markboth{\textcolor{darkblue}{\textbf{\ipa{bv̩˧}}}}{}
\textcolor{teal}{\mytextsc{noun}} \hspace{4pt} Tone: M.
\textcolor{Sepia}{\selectlanguage{english}Intestine.} \zh{肠子。}  \zh{量词}: \textcolor{darkblue}{\textbf{\ipa{kʰɯ˩}}}  \mytextsc{clf}: \textcolor{darkblue}{\textbf{\ipa{kʰɯ˩}}} 
\lhead{\firstmark}
\rhead{\botmark}

\subsection{\hspace{-0.5cm} {\Large \textcolor{darkblue}{\textbf{\ipa{bv̩˥}}}}\hspace{0.5cm}[\kern2pt{\textcolor{darkblue}{\textbf{\ipa{bv̩˥}}}}\kern2pt]} \hypertarget{bv\string_=\string_T1}{}
\markboth{\textcolor{darkblue}{\textbf{\ipa{bv̩˥}}}}{}
\textcolor{teal}{\mytextsc{noun}} \hspace{4pt} Tone: \#H.
\textcolor{Sepia}{\selectlanguage{english}Worm; insect.} \zh{虫。}  ¶ \textcolor{darkblue}{\textbf{\ipa{bv̩˧ tʰv̩˧-mi˧˥ / bv̩˧ tʰv̩˧-mi˥\#}}} \textcolor{Sepia}{\selectlanguage{english}\mytextsc{n}+\mytextsc{dem}+\mytextsc{clf}} \zh{这只虫}  
 \zh{量词}: \textcolor{darkblue}{\textbf{\ipa{mi˩}}}  \mytextsc{clf}: \textcolor{darkblue}{\textbf{\ipa{mi˩}}} 
\lhead{\firstmark}
\rhead{\botmark}

\subsection{\hspace{-0.5cm} {\Large \textcolor{darkblue}{\textbf{\ipa{bv̩˥}}} \textsubscript{1}}\hspace{0.5cm}[\kern2pt{\textcolor{darkblue}{\textbf{\ipa{bv̩˥}}}}\kern2pt]} \hypertarget{bv\string_=\string_T1}{}
\markboth{\textcolor{darkblue}{\textbf{\ipa{bv̩˥}}} \textsubscript{1}}{}
\textcolor{teal}{\mytextsc{adjective}} \hspace{4pt} Tone: H.
\textcolor{Sepia}{\selectlanguage{english}Thick (tree trunk); coarse (flour, powder).} \zh{粗(树粗大,粉末不精细……)。}  ¶ \textcolor{darkblue}{\textbf{\ipa{qʰɑ˧-bv̩˧-gv̩˧}}} \textcolor{Sepia}{\selectlanguage{english}very thick} \zh{很粗}  
 ¶ \textcolor{darkblue}{\textbf{\ipa{qʰɑ˧bv̩˧\textasciitilde{}bv̩˧-gv̩˧}}} \textcolor{Sepia}{\selectlanguage{english}very thick (as above)} \zh{很粗(同上)}  

\lhead{\firstmark}
\rhead{\botmark}

\subsection{\hspace{-0.5cm} {\Large \textcolor{darkblue}{\textbf{\ipa{bv̩˥}}} \textsubscript{2}}\hspace{0.5cm}[\kern2pt{\textcolor{darkblue}{\textbf{\ipa{bv̩˥}}}}\kern2pt]} \hypertarget{bv\string_=\string_T2}{}
\markboth{\textcolor{darkblue}{\textbf{\ipa{bv̩˥}}} \textsubscript{2}}{}
\textcolor{teal}{\mytextsc{verb}} \hspace{4pt} Tone: H.
\textcolor{Sepia}{\selectlanguage{english}To distribute things, to allot things, to divide things between several persons.} \zh{分东西。}  ¶ \textcolor{darkblue}{\textbf{\ipa{ɖɯ˧-v̩˧ ɖɯ˧-kʰwɤ˥ | le˧-bv̩˧\textasciitilde{}bv̩˧}}} \textcolor{Sepia}{\selectlanguage{english}to share, giving each person a piece} \zh{分给一人一块}  
 ¶ \textcolor{darkblue}{\textbf{\ipa{le˧-bv̩˧\textasciitilde{}bv̩˧ tʰi˧-kwɤ˩}}} \textcolor{Sepia}{\selectlanguage{english}to scatter all over the place, to lay out in no good order} \zh{弄乱,散开}  

\lhead{\firstmark}
\rhead{\botmark}

\subsection{\hspace{-0.5cm} {\Large \textcolor{darkblue}{\textbf{\ipa{bv̩˥}}} \textsubscript{3}}\hspace{0.5cm}[\kern2pt{\textcolor{darkblue}{\textbf{\ipa{bv̩˥}}}}\kern2pt]} \hypertarget{bv\string_=\string_T3}{}
\markboth{\textcolor{darkblue}{\textbf{\ipa{bv̩˥}}} \textsubscript{3}}{}
\textcolor{teal}{\mytextsc{verb}} \hspace{4pt} Tone: H.
\textcolor{Sepia}{\selectlanguage{english}To roast, to grill.} \zh{烤,炙。}  ¶ \textcolor{darkblue}{\textbf{\ipa{hɑ˧ tʰi˧-bv̩˥}}} \textcolor{Sepia}{\selectlanguage{english}to roast food, to roast cereals} \zh{烤饭}  

\lhead{\firstmark}
\rhead{\botmark}

\subsection{\hspace{-0.5cm} {\Large \textcolor{darkblue}{\textbf{\ipa{bv̩˩\textsubscript{a}}}} \textsubscript{1}}\hspace{0.5cm}[\kern2pt{\textcolor{darkblue}{\textbf{\ipa{bv̩˥}}}}\kern2pt]} \hypertarget{bv\string_=\string_Ba1}{}
\markboth{\textcolor{darkblue}{\textbf{\ipa{bv̩˩\textsubscript{a}}}} \textsubscript{1}}{}
\textcolor{teal}{\mytextsc{verb}} \hspace{4pt} Tone: L\textsubscript{a}.
\textcolor{Sepia}{\selectlanguage{english}To hatch, to incubate.} \zh{孵。}  ¶ \textcolor{darkblue}{\textbf{\ipa{æ˩mi˧ bv̩˩}}} \textcolor{Sepia}{\selectlanguage{english}The hen is hatching eggs.} \zh{母鸡孵蛋}  

\lhead{\firstmark}
\rhead{\botmark}

\subsection{\hspace{-0.5cm} {\Large \textcolor{darkblue}{\textbf{\ipa{bv̩˩\textsubscript{a}}}} \textsubscript{2}}\hspace{0.5cm}[\kern2pt{\textcolor{darkblue}{\textbf{\ipa{bv̩˩˥}}}}\kern2pt]} \hypertarget{bv\string_=\string_Ba2}{}
\markboth{\textcolor{darkblue}{\textbf{\ipa{bv̩˩\textsubscript{a}}}} \textsubscript{2}}{}
\textcolor{teal}{\mytextsc{verb}} \hspace{4pt} Tone: L\textsubscript{a}.
\textcolor{Sepia}{\selectlanguage{english}To steam, to cook by steaming.} \zh{蒸。}  ¶ \textcolor{darkblue}{\textbf{\ipa{le˧-bv̩˩-ze˩}}} \textcolor{Sepia}{\selectlanguage{english}\mytextsc{accomp} \string_ \mytextsc{pfv}} \zh{蒸了}  
 ¶ \textcolor{darkblue}{\textbf{\ipa{pɤ˩jɤ˧ bv̩˥}}} \textcolor{Sepia}{\selectlanguage{english}to steam buns} \zh{蒸馒头}  
 ¶ \textcolor{darkblue}{\textbf{\ipa{hɑ˧ bv̩˥\textasciitilde{}bv̩˩}}} \textcolor{Sepia}{\selectlanguage{english}to steam rice} \zh{蒸米饭}  

\lhead{\firstmark}
\rhead{\botmark}

\subsection{\hspace{-0.5cm} {\Large \textcolor{darkblue}{\textbf{\ipa{bv̩˩\textsubscript{a}}}} \textsubscript{3}}\hspace{0.5cm}[\kern2pt{\textcolor{darkblue}{\textbf{\ipa{bv̩˩˥}}}}\kern2pt]} \hypertarget{bv\string_=\string_Ba3}{}
\markboth{\textcolor{darkblue}{\textbf{\ipa{bv̩˩\textsubscript{a}}}} \textsubscript{3}}{}
\textcolor{teal}{\mytextsc{verb}} \hspace{4pt} Tone: L\textsubscript{a}.
\textcolor{Sepia}{\selectlanguage{english}To live (one's life).} \zh{过(日子)。}  ¶ \textcolor{darkblue}{\textbf{\ipa{zɯ˧ bv̩˩}}} \textcolor{Sepia}{\selectlanguage{english}to live one's life} \zh{过日子}  
 ¶ \textcolor{darkblue}{\textbf{\ipa{hĩ˧-zɯ˧ bv̩˥, | lo˧hɑ˧!}}} \textcolor{Sepia}{\selectlanguage{english}Living one's life is hard! / Life is tough!} \zh{生活,是艰难的!}  
 ¶ \textcolor{darkblue}{\textbf{\ipa{hĩ˧-zɯ˧ | le˧-bv̩˩-ze˩}}} \textcolor{Sepia}{\selectlanguage{english}(Her/his) life has gone by! / (Her/his) life is over! (A reflection after someone's decease.)} \zh{他的日子,就结束了!(情景:一个人去世了,葬礼的时候,有人这样说。)}  
 ¶ \textcolor{darkblue}{\textbf{\ipa{qʰwɤ˧-ɭɯ˥, | ʈʂʰæ˧-mɤ˧-dʑɯ˧! | ʈʂʰɯ˧ ɖɯ˧-zɯ˧ bv̩˩-ze˩!}}} \textcolor{Sepia}{\selectlanguage{english}He never had to do the washing-up (in his entire life)! That's how he spent his lifetime (=without any practical concerns)! (About an office-holder whose every need in daily life was attended to by servants.)} \zh{他从来没有洗过碗!他这辈子就是这么过来的!(关于一个官员,完全不用管家务、日常生活中的活儿:有人来管一切。)}  
 ¶ \textcolor{darkblue}{\textbf{\ipa{ɖɯ˧-ɲi˥\textasciitilde{}ɖɯ˩-ɲi˩ | bv̩˩ lo˩ fv̩˩˥!}}} \textcolor{Sepia}{\selectlanguage{english}How easily days go by! / How time flies!} \zh{日子过得真快!}  

\lhead{\firstmark}
\rhead{\botmark}

\subsection{\hspace{-0.5cm} {\Large \textcolor{darkblue}{\textbf{\ipa{bv̩˩\textsubscript{a}}}} \textsubscript{4}}\hspace{0.5cm}[\kern2pt{\textcolor{darkblue}{\textbf{\ipa{bv̩˩˥}}}}\kern2pt]} \hypertarget{bv\string_=\string_Ba4}{}
\markboth{\textcolor{darkblue}{\textbf{\ipa{bv̩˩\textsubscript{a}}}} \textsubscript{4}}{}
\textcolor{teal}{\mytextsc{verb}} \hspace{4pt} Tone: L\textsubscript{a}.
\ding{202} \textcolor{Sepia}{\selectlanguage{english}To sprinkle water.} \zh{泼水,浇(浇菜)。}  ¶ \textcolor{darkblue}{\textbf{\ipa{le˧-bv̩˩-ze˩}}} \textcolor{Sepia}{\selectlanguage{english}\mytextsc{accomp} \string_ \mytextsc{pfv}} \zh{泼了}  
 ¶ \textcolor{darkblue}{\textbf{\ipa{dʑɯ˩ bv̩˩˥}}} \textcolor{Sepia}{\selectlanguage{english}to sprinkle water} \zh{泼水}  
 ¶ \textcolor{darkblue}{\textbf{\ipa{ɖɯ˧-bv̩˧\textasciitilde{}bv̩˥-ɻ̍˩}}} \textcolor{Sepia}{\selectlanguage{english}\mytextsc{delimitative} \mytextsc{red} \mytextsc{inceptive}} \zh{泼一泼}  
 ¶ \textcolor{darkblue}{\textbf{\ipa{le˧-bv̩˧\textasciitilde{}bv̩˥-ze˩}}} \textcolor{Sepia}{\selectlanguage{english}\mytextsc{accomp} \mytextsc{red} \mytextsc{pfv}} \zh{泼了一点}  
\ding{203} \textcolor{Sepia}{\selectlanguage{english}To sow (seeds).} \zh{撒(种子)。}  ¶ \textcolor{darkblue}{\textbf{\ipa{ɻæ˩ bv̩˥}}} \textcolor{Sepia}{\selectlanguage{english}to sow seeds} \zh{撒种子}  
 ¶ \textcolor{darkblue}{\textbf{\ipa{tʰi˧-bv̩˩-ɻ̍˩}}} \textcolor{Sepia}{\selectlanguage{english}Go ahead and sow!} \zh{撒吧!}  
 ¶ \textcolor{darkblue}{\textbf{\ipa{tʰi˧-bv̩˩-qɑ˩!}}} \textcolor{Sepia}{\selectlanguage{english}Sow!} \zh{撒吧!}  

\lhead{\firstmark}
\rhead{\botmark}

\subsection{\hspace{-0.5cm} {\Large \textcolor{darkblue}{\textbf{\ipa{bv̩˩\textsubscript{a}}}} \textsubscript{5}}\hspace{0.5cm}[\kern2pt{\textcolor{darkblue}{\textbf{\ipa{bv̩˩˥}}}}\kern2pt]} \hypertarget{bv\string_=\string_Ba5}{}
\markboth{\textcolor{darkblue}{\textbf{\ipa{bv̩˩\textsubscript{a}}}} \textsubscript{5}}{}
\textcolor{teal}{\mytextsc{adjective}} \hspace{4pt} Tone: L\textsubscript{a}.
\textcolor{Sepia}{\selectlanguage{english}Thin, scarce, sparse (e.g. hair).} \zh{细、薄。}  ¶ \textcolor{darkblue}{\textbf{\ipa{ʁo˧ bv̩˧˥}}} \textcolor{Sepia}{\selectlanguage{english}bald (literally “the head (has) scarce (hair)”)} \zh{头秃、头发很少}  
 ¶ \textcolor{darkblue}{\textbf{\ipa{ʁo˧-bv̩˧-hĩ˥}}} \textcolor{Sepia}{\selectlanguage{english}bald person} \zh{秃子}  
 ¶ \textcolor{darkblue}{\textbf{\ipa{ʁo˧-bv̩˧-zo˥}}} \textcolor{Sepia}{\selectlanguage{english}as above} \zh{同上}  
 ¶ \textcolor{darkblue}{\textbf{\ipa{ʈʂʰɯ˧ | ʁo˧ bv̩˧-ze˥}}} \textcolor{Sepia}{\selectlanguage{english}He lost his hair, he went bald} \zh{他秃头了,他头发掉了}  
 ¶ \textcolor{darkblue}{\textbf{\ipa{ʁo˧qʰwɤ˩ | le˧-bv̩˩-ze˩}}} \textcolor{Sepia}{\selectlanguage{english}(his) head has gone bald} \zh{(他)秃头了。}  
 ¶ \textcolor{darkblue}{\textbf{\ipa{bv̩˩-hĩ˩˥}}} \textcolor{Sepia}{\selectlanguage{english}\mytextsc{rel}} \zh{秃的}  

\lhead{\firstmark}
\rhead{\botmark}

\subsection{\hspace{-0.5cm} {\Large \textcolor{darkblue}{\textbf{\ipa{bv̩˧ɖæ˧}}}}\hspace{0.5cm}[\kern2pt{\textcolor{darkblue}{\textbf{\ipa{bv̩˩ɖæ˩˥}}}}\kern2pt]} \hypertarget{bv\string_=\string_Md`\{\string_M1}{}
\markboth{\textcolor{darkblue}{\textbf{\ipa{bv̩˧ɖæ˧}}}}{}
\textcolor{teal}{\mytextsc{adjective}} \hspace{4pt} Tone: M.
\textit{From:} \textbf{bv̩˧ and ɖæ˥} \textcolor{Sepia}{\selectlanguage{english}With a bad temper. Literally 'short-intestined': in folk representation, short intestines are associated with hasty emotional reactions, whereas long intestines allow their owner to digest vexations calmly.} \zh{脾气很坏。}  ¶ \textcolor{darkblue}{\textbf{\ipa{ʈʂʰɯ˧ | bv̩˧ɖæ˧-ze˩!}}} \textcolor{Sepia}{\selectlanguage{english}He is is a bad mood now.} \zh{他现在脾气很坏!/ 他生气了!}  

\lhead{\firstmark}
\rhead{\botmark}

\subsection{\hspace{-0.5cm} {\Large \textcolor{darkblue}{\textbf{\ipa{bv̩˩di˩}}}}\hspace{0.5cm}[\kern2pt{\textcolor{darkblue}{\textbf{\ipa{bv̩˧di˧}}}}\kern2pt]} \hypertarget{bv\string_=\string_Bdi\string_B1}{}
\markboth{\textcolor{darkblue}{\textbf{\ipa{bv̩˩di˩}}}}{}
\textcolor{teal}{\mytextsc{noun}} \hspace{4pt} Tone: L.
\textcolor{Sepia}{\selectlanguage{english}Food steamer.} \zh{蒸笼。}  \zh{量词}: \textcolor{darkblue}{\textbf{\ipa{ɭɯ˧}}}  \mytextsc{clf}: \textcolor{darkblue}{\textbf{\ipa{ɭɯ˧}}} \textit{See:} \hyperlink{}{\textcolor{darkblue}{\textbf{\ipa{bv̩˩˧}}} \textsubscript{2}} 
\lhead{\firstmark}
\rhead{\botmark}

\subsection{\hspace{-0.5cm} {\Large \textcolor{darkblue}{\textbf{\ipa{bv̩˩dze˩}}} \textsubscript{1}}\hspace{0.5cm}[\kern2pt{\textcolor{darkblue}{\textbf{\ipa{bv̩˩dze˩˥}}}}\kern2pt]} \hypertarget{bv\string_=\string_Bdze\string_B1}{}
\markboth{\textcolor{darkblue}{\textbf{\ipa{bv̩˩dze˩}}} \textsubscript{1}}{}
\textcolor{teal}{\mytextsc{noun}} \hspace{4pt} Tone: L.
\textcolor{Sepia}{\selectlanguage{english}Large spoon, used for rice.} \zh{大调羹。}  \zh{量词}: \textcolor{darkblue}{\textbf{\ipa{nɑ˧}}}  \mytextsc{clf}: \textcolor{darkblue}{\textbf{\ipa{nɑ˧}}} \textit{See:} \hyperlink{}{\textcolor{darkblue}{\textbf{\ipa{bv̩˩dze˩}}} \textsubscript{2}} 
\lhead{\firstmark}
\rhead{\botmark}

\subsection{\hspace{-0.5cm} {\Large \textcolor{darkblue}{\textbf{\ipa{bv̩˩dze˩}}} \textsubscript{2}}\hspace{0.5cm}[\kern2pt{\textcolor{darkblue}{\textbf{\ipa{bv̩˩dze˩˥}}}}\kern2pt]} \hypertarget{bv\string_=\string_Bdze\string_B2}{}
\markboth{\textcolor{darkblue}{\textbf{\ipa{bv̩˩dze˩}}} \textsubscript{2}}{}
\textcolor{teal}{\mytextsc{classifier}} \hspace{4pt} Tone: L.
\textcolor{Sepia}{\selectlanguage{english}Ladleful.} \zh{量词:勺。}  ¶ \textcolor{darkblue}{\textbf{\ipa{ɖɯ˧-bv̩˩dze˩}}} \textcolor{Sepia}{\selectlanguage{english}one ladleful} \zh{一勺}  
\textit{See:} \hyperlink{}{\textcolor{darkblue}{\textbf{\ipa{bv̩˩dze˩}}} \textsubscript{1}} 
\lhead{\firstmark}
\rhead{\botmark}

\subsection{\hspace{-0.5cm} {\Large \textcolor{darkblue}{\textbf{\ipa{bv̩˧hu˧˥}}}}\hspace{0.5cm}[\kern2pt{\textcolor{darkblue}{\textbf{\ipa{bv̩˩hu˩˥}}}}\kern2pt]} \hypertarget{bv\string_=\string_Mhu\string_M\string_T1}{}
\markboth{\textcolor{darkblue}{\textbf{\ipa{bv̩˧hu˧˥}}}}{}
\textcolor{teal}{\mytextsc{noun}} \hspace{4pt} Tone: MH\#.
\textcolor{Sepia}{\selectlanguage{english}Bowels: intestine+stomach.} \zh{胃与肠。}  \zh{量词}: \textcolor{darkblue}{\textbf{\ipa{kwɤ˩}}}  \mytextsc{clf}: \textcolor{darkblue}{\textbf{\ipa{kwɤ˩}}} 
\lhead{\firstmark}
\rhead{\botmark}

\subsection{\hspace{-0.5cm} {\Large \textcolor{darkblue}{\textbf{\ipa{bv̩˩hwɤ˩}}}}\hspace{0.5cm}[\kern2pt{\textcolor{darkblue}{\textbf{\ipa{bv̩˧hwɤ˧˥}}}}\kern2pt]} \hypertarget{bv\string_=\string_Bhw7\string_B1}{}
\markboth{\textcolor{darkblue}{\textbf{\ipa{bv̩˩hwɤ˩}}}}{}
\textcolor{teal}{\mytextsc{noun}} \hspace{4pt} Tone: L.
\textcolor{Sepia}{\selectlanguage{english}Wooden hut where shepherds stay while herding their flock on the mountain.} \zh{山上放牧的人暂时住的木头小房。}  \zh{量词}: \textcolor{darkblue}{\textbf{\ipa{ɭɯ˧}}}  \mytextsc{clf}: \textcolor{darkblue}{\textbf{\ipa{ɭɯ˧}}} 
\lhead{\firstmark}
\rhead{\botmark}

\subsection{\hspace{-0.5cm} {\Large \textcolor{darkblue}{\textbf{\ipa{bv̩˧kʰɯ˧˥}}}}\hspace{0.5cm}[\kern2pt{\textcolor{darkblue}{\textbf{\ipa{bv̩˩kʰɯ˩˥}}}}\kern2pt]} \hypertarget{bv\string_=\string_Mk\string_hM\string_M\string_T1}{}
\markboth{\textcolor{darkblue}{\textbf{\ipa{bv̩˧kʰɯ˧˥}}}}{}
\textcolor{teal}{\mytextsc{noun}} \hspace{4pt} Tone: MH\#.
\textcolor{Sepia}{\selectlanguage{english}Silkworm.} \zh{蚕。}  \zh{量词}: \textcolor{darkblue}{\textbf{\ipa{kʰɯ˩}}}  \mytextsc{clf}: \textcolor{darkblue}{\textbf{\ipa{kʰɯ˩}}} 
\lhead{\firstmark}
\rhead{\botmark}

\subsection{\hspace{-0.5cm} {\Large \textcolor{darkblue}{\textbf{\ipa{bv̩˩ɭɯ˩}}}}\hspace{0.5cm}[\kern2pt{\textcolor{darkblue}{\textbf{\ipa{bv̩˧ɭɯ˧˥}}}}\kern2pt]} \hypertarget{bv\string_=\string_Bl\string_RM\string_B1}{}
\markboth{\textcolor{darkblue}{\textbf{\ipa{bv̩˩ɭɯ˩}}}}{}
\textcolor{teal}{\mytextsc{noun}} \hspace{4pt} Tone: L.
\textcolor{Sepia}{\selectlanguage{english}Kidneys.} \zh{肾。}  \zh{量词}: \textcolor{darkblue}{\textbf{\ipa{ɭɯ˧}}}  \mytextsc{clf}: \textcolor{darkblue}{\textbf{\ipa{ɭɯ˧}}} 
\lhead{\firstmark}
\rhead{\botmark}

\subsection{\hspace{-0.5cm} {\Large \textcolor{darkblue}{\textbf{\ipa{bv̩˧mi˧}}} \textsubscript{1}}\hspace{0.5cm}[\kern2pt{\textcolor{darkblue}{\textbf{\ipa{bv̩˩mi˩˥}}}}\kern2pt]} \hypertarget{bv\string_=\string_Mmi\string_M1}{}
\markboth{\textcolor{darkblue}{\textbf{\ipa{bv̩˧mi˧}}} \textsubscript{1}}{}
\textcolor{teal}{\mytextsc{noun}} \hspace{4pt} Tone: M.
\textcolor{Sepia}{\selectlanguage{english}Female yak, dri, drimo, nak.} \zh{母牦牛。}  ¶ \textcolor{darkblue}{\textbf{\ipa{bv̩˧mi˧-bv̩˩ʂwæ˩}}} \textcolor{Sepia}{\selectlanguage{english}female yak and castrated yak} \zh{母牦牛与阉割牦牛}  
 ¶ \textcolor{darkblue}{\textbf{\ipa{bv̩˧mi˧-bv̩˧zo\#˥}}} \textcolor{Sepia}{\selectlanguage{english}female yak and yak calf (baby yak)} \zh{母牦牛与小牦牛}  
 \zh{量词}: \textcolor{darkblue}{\textbf{\ipa{mi˩}}}  \mytextsc{clf}: \textcolor{darkblue}{\textbf{\ipa{mi˩}}} 
\lhead{\firstmark}
\rhead{\botmark}

\subsection{\hspace{-0.5cm} {\Large \textcolor{darkblue}{\textbf{\ipa{bv̩˧mi˧}}} \textsubscript{2}}\hspace{0.5cm}[\kern2pt{\textcolor{darkblue}{\textbf{\ipa{bv̩˧mi˧}}}}\kern2pt]} \hypertarget{bv\string_=\string_Mmi\string_M2}{}
\markboth{\textcolor{darkblue}{\textbf{\ipa{bv̩˧mi˧}}} \textsubscript{2}}{}
\textcolor{teal}{\mytextsc{noun}} \hspace{4pt} Tone: M.
\textcolor{Sepia}{\selectlanguage{english}Large food steamer.} \zh{大蒸笼。}  \zh{量词}: \textcolor{darkblue}{\textbf{\ipa{mi˩}}}  \mytextsc{clf}: \textcolor{darkblue}{\textbf{\ipa{mi˩}}} 
\lhead{\firstmark}
\rhead{\botmark}

\subsection{\hspace{-0.5cm} {\Large \textcolor{darkblue}{\textbf{\ipa{bv̩˧-nɑ˥mi˩}}}}\hspace{0.5cm}[\kern2pt{\textcolor{darkblue}{\textbf{\ipa{xxxx non-correspondance entre le nombre de morphèmes et le nombre de tons de morphèmes}}}}\kern2pt]} \hypertarget{bv\string_=\string_M-nA\string_Tmi\string_B1}{}
\markboth{\textcolor{darkblue}{\textbf{\ipa{bv̩˧-nɑ˥mi˩}}}}{}
\textcolor{teal}{\mytextsc{noun}} \hspace{4pt} Tone: \#H-.
\textcolor{Sepia}{\selectlanguage{english}\textit{Mythimna separata (Walker)}.} \zh{玉米黏虫。} 
\lhead{\firstmark}
\rhead{\botmark}

\subsection{\hspace{-0.5cm} {\Large \textcolor{darkblue}{\textbf{\ipa{bv̩˧nv̩˧}}} \textsubscript{1}}\hspace{0.5cm}[\kern2pt{\textcolor{darkblue}{\textbf{\ipa{xxxx non-correspondance entre le nombre de morphèmes et le nombre de tons de morphèmes}}}}\kern2pt]} \hypertarget{bv\string_=\string_Mnv\string_=\string_M1}{}
\markboth{\textcolor{darkblue}{\textbf{\ipa{bv̩˧nv̩˧}}} \textsubscript{1}}{}
\textcolor{teal}{\mytextsc{verb}} \hspace{4pt} Tone: M intrans.
\textcolor{Sepia}{\selectlanguage{english}To smell, to perceive by smelling.} \zh{闻到(嗅觉)。}  ¶ \textcolor{darkblue}{\textbf{\ipa{no˧ | ɖɯ˧-bv̩˧nv̩˧-ɻ̍˩! | ɖwæ˩˥ | ɕjɤ˧!}}} \textcolor{Sepia}{\selectlanguage{english}Have a smell! It smells great!} \zh{你闻一闻吧!好香!}  
\textit{See:} \hyperlink{}{\textcolor{darkblue}{\textbf{\ipa{bv̩˧nv̩˧}}} \textsubscript{2}} 
\lhead{\firstmark}
\rhead{\botmark}

\subsection{\hspace{-0.5cm} {\Large \textcolor{darkblue}{\textbf{\ipa{bv̩˧nv̩˧}}} \textsubscript{2}}\hspace{0.5cm}[\kern2pt{\textcolor{darkblue}{\textbf{\ipa{bv̩˧nv̩˧}}}}\kern2pt]} \hypertarget{bv\string_=\string_Mnv\string_=\string_M2}{}
\markboth{\textcolor{darkblue}{\textbf{\ipa{bv̩˧nv̩˧}}} \textsubscript{2}}{}
\textcolor{teal}{\mytextsc{adjective}} \hspace{4pt} Tone: M intrans.
\textcolor{Sepia}{\selectlanguage{english}Stinking, smelly.} \zh{臭。}  ¶ \textcolor{darkblue}{\textbf{\ipa{bv̩˧nv̩˧-ze˧}}} \textcolor{Sepia}{\selectlanguage{english}\mytextsc{pfv}} \zh{臭了}  
\textit{See:} \hyperlink{}{\textcolor{darkblue}{\textbf{\ipa{bv̩˧nv̩˧}}} \textsubscript{1}} 
\lhead{\firstmark}
\rhead{\botmark}

\subsection{\hspace{-0.5cm} {\Large \textcolor{darkblue}{\textbf{\ipa{bv̩˧pʰv̩˧}}}}\hspace{0.5cm}[\kern2pt{\textcolor{darkblue}{\textbf{\ipa{bv̩˧pʰv̩˧}}}}\kern2pt]} \hypertarget{bv\string_=\string_Mp\string_hv\string_=\string_M1}{}
\markboth{\textcolor{darkblue}{\textbf{\ipa{bv̩˧pʰv̩˧}}}}{}
\textcolor{teal}{\mytextsc{noun}} \hspace{4pt} Tone: M.
\textcolor{Sepia}{\selectlanguage{english}Male yak (elicited form; the commonly used form is \textcolor{darkblue}{\textbf{\ipa{/bv̩˩ʂwæ˩/}}}).} \zh{公牦牛。}  \zh{量词}: \textcolor{darkblue}{\textbf{\ipa{mi˩}}}  \mytextsc{clf}: \textcolor{darkblue}{\textbf{\ipa{mi˩}}} 
\lhead{\firstmark}
\rhead{\botmark}

\subsection{\hspace{-0.5cm} {\Large \textcolor{darkblue}{\textbf{\ipa{bv̩˩-qʰæ˩}}}}\hspace{0.5cm}[\kern2pt{\textcolor{darkblue}{\textbf{\ipa{xxxx non-correspondance entre le nombre de morphèmes et le nombre de tons de morphèmes}}}}\kern2pt]} \hypertarget{bv\string_=\string_B-q\string_h\{\string_B1}{}
\markboth{\textcolor{darkblue}{\textbf{\ipa{bv̩˩-qʰæ˩}}}}{}
\textcolor{teal}{\mytextsc{noun}} \hspace{4pt} Tone: L.
\textcolor{Sepia}{\selectlanguage{english}Manure, dung.} \zh{肥料、粪。}  ¶ \textcolor{darkblue}{\textbf{\ipa{bv̩˩qʰæ˩ tʰv̩˩-ʁwɤ˥}}} \textcolor{Sepia}{\selectlanguage{english}\mytextsc{n}+\mytextsc{dem}+\mytextsc{clf}} \zh{这堆肥料}  
 \zh{量词}: \textcolor{darkblue}{\textbf{\ipa{ʁwɤ˧}}}  \mytextsc{clf}: \textcolor{darkblue}{\textbf{\ipa{ʁwɤ˧}}} 
\lhead{\firstmark}
\rhead{\botmark}

\subsection{\hspace{-0.5cm} {\Large \textcolor{darkblue}{\textbf{\ipa{bv̩˩qo˩-bv̩˧qʰæ˩}}}}\hspace{0.5cm}[\kern2pt{\textcolor{darkblue}{\textbf{\ipa{xxxx non-correspondance entre le nombre de morphèmes et le nombre de tons de morphèmes}}}}\kern2pt]} \hypertarget{bv\string_=\string_Bqo\string_B-bv\string_=\string_Mq\string_h\{\string_B1}{}
\markboth{\textcolor{darkblue}{\textbf{\ipa{bv̩˩qo˩-bv̩˧qʰæ˩}}}}{}
\textcolor{teal}{\mytextsc{noun}} \hspace{4pt} Tone: L-L\#.
\textcolor{Sepia}{\selectlanguage{english}Manure, dung.} \zh{肥料、粪。}  \zh{量词}: \textcolor{darkblue}{\textbf{\ipa{ʁwɤ˧}}}  \mytextsc{clf}: \textcolor{darkblue}{\textbf{\ipa{ʁwɤ˧}}} 
\lhead{\firstmark}
\rhead{\botmark}

\subsection{\hspace{-0.5cm} {\Large \textcolor{darkblue}{\textbf{\ipa{bv̩˩qo˩-qʰæ˩}}}}\hspace{0.5cm}[\kern2pt{\textcolor{darkblue}{\textbf{\ipa{bv̩˩qo˩qʰæ˩˥}}}}\kern2pt]} \hypertarget{bv\string_=\string_Bqo\string_B-q\string_h\{\string_B1}{}
\markboth{\textcolor{darkblue}{\textbf{\ipa{bv̩˩qo˩-qʰæ˩}}}}{}
\textcolor{teal}{\mytextsc{noun}} \hspace{4pt} Tone: L.
\textcolor{Sepia}{\selectlanguage{english}Manure, excrement.} \zh{肥料 , 粪。}  ¶ \textcolor{darkblue}{\textbf{\ipa{bv̩˩qo˩-qʰæ˩ tʰv̩˩-ʁwɤ˥}}} \textcolor{Sepia}{\selectlanguage{english}\mytextsc{n}+\mytextsc{dem}+\mytextsc{clf}} \zh{这堆肥料}  
 \zh{量词}: \textcolor{darkblue}{\textbf{\ipa{ʁwɤ˧}}}  \mytextsc{clf}: \textcolor{darkblue}{\textbf{\ipa{ʁwɤ˧}}} 
\lhead{\firstmark}
\rhead{\botmark}

\subsection{\hspace{-0.5cm} {\Large \textcolor{darkblue}{\textbf{\ipa{bv̩˩qʰv̩˩}}}}\hspace{0.5cm}[\kern2pt{\textcolor{darkblue}{\textbf{\ipa{bv̩˩qʰv̩˩˥}}}}\kern2pt]} \hypertarget{bv\string_=\string_Bq\string_hv\string_=\string_B1}{}
\markboth{\textcolor{darkblue}{\textbf{\ipa{bv̩˩qʰv̩˩}}}}{}
\textcolor{teal}{\mytextsc{noun}} \hspace{4pt} Tone: L.
\ding{202} \textcolor{Sepia}{\selectlanguage{english}Conch shell, \textit{Turbinella pyrum L.}. It is used in ceremonies. Each family has a pair of conchs.} \zh{法螺、海螺、螺号。}  \zh{量词}: \textcolor{darkblue}{\textbf{\ipa{dze˩}}} \ding{203} \textcolor{Sepia}{\selectlanguage{english}Lines of the hand.} \zh{掌纹。}  ¶ \textcolor{darkblue}{\textbf{\ipa{lo˩qʰwɤ˧-bv̩˧ | bv̩˩qʰv̩˩˥}}} \textcolor{Sepia}{\selectlanguage{english}the lines of the hand} \zh{掌纹}  
 \mytextsc{clf}: \textcolor{darkblue}{\textbf{\ipa{dze˩}}} 
\lhead{\firstmark}
\rhead{\botmark}

\subsection{\hspace{-0.5cm} {\Large \textcolor{darkblue}{\textbf{\ipa{bv̩˧qʰv̩˧ʑi˩-hĩ˩}}}}\hspace{0.5cm}[\kern2pt{\textcolor{darkblue}{\textbf{\ipa{xxxx non-correspondance entre le nombre de morphèmes et le nombre de tons de morphèmes}}}}\kern2pt]} \hypertarget{bv\string_=\string_Mq\string_hv\string_=\string_Mz£i\string_B-hi\string_~\string_B1}{}
\markboth{\textcolor{darkblue}{\textbf{\ipa{bv̩˧qʰv̩˧ʑi˩-hĩ˩}}}}{}
\textcolor{teal}{\mytextsc{noun}} \hspace{4pt} Tone: L\#.
\textcolor{Sepia}{\selectlanguage{english}Snail.} \zh{蜗牛,螺蛳。}  \zh{量词}: \textcolor{darkblue}{\textbf{\ipa{mi˩}}}  \mytextsc{clf}: \textcolor{darkblue}{\textbf{\ipa{mi˩}}} 
\lhead{\firstmark}
\rhead{\botmark}

\subsection{\hspace{-0.5cm} {\Large \textcolor{darkblue}{\textbf{\ipa{bv̩˧ɻ\#˥}}}}\hspace{0.5cm}[\kern2pt{\textcolor{darkblue}{\textbf{\ipa{bv̩˧ɻ˩}}}}\kern2pt]} \hypertarget{bv\string_=\string_Mr£`\#\string_T1}{}
\markboth{\textcolor{darkblue}{\textbf{\ipa{bv̩˧ɻ\#˥}}}}{}
\textcolor{teal}{\mytextsc{noun}} \hspace{4pt} Tone: \#H.
\textcolor{Sepia}{\selectlanguage{english}Fly.} \zh{苍蝇。}  ¶ \textcolor{darkblue}{\textbf{\ipa{bv̩˧ɻ̍˧ ʈʂʰɯ˧-mi˧˥ / bv̩˧ɻ̍˧ ʈʂʰɯ˧-mi˥\#}}} \textcolor{Sepia}{\selectlanguage{english}\mytextsc{n}+\mytextsc{dem}+\mytextsc{clf}} \zh{这只苍蝇}  
 \zh{量词}: \textcolor{darkblue}{\textbf{\ipa{mi˩}}}  \mytextsc{clf}: \textcolor{darkblue}{\textbf{\ipa{mi˩}}} 
\lhead{\firstmark}
\rhead{\botmark}

\subsection{\hspace{-0.5cm} {\Large \textcolor{darkblue}{\textbf{\ipa{bv̩˧ʂæ˧}}}}\hspace{0.5cm}[\kern2pt{\textcolor{darkblue}{\textbf{\ipa{bv̩˧ʂæ˧}}}}\kern2pt]} \hypertarget{bv\string_=\string_Ms`\{\string_M1}{}
\markboth{\textcolor{darkblue}{\textbf{\ipa{bv̩˧ʂæ˧}}}}{}
\textcolor{teal}{\mytextsc{adjective}} \hspace{4pt} Tone: M.
\textit{From:} \textbf{bv̩˧ and ʂæ˧} \textcolor{Sepia}{\selectlanguage{english}Good-tempered, with a good mood, good-humoured.} \zh{脾气好。}  ¶ \textcolor{darkblue}{\textbf{\ipa{ʈʂʰɯ˧ | bv̩˧ʂæ˧-ze˩}}} \textcolor{Sepia}{\selectlanguage{english}He is in a good mood now.} \zh{他现在脾气好。/ 他高兴了。}  
 ¶ \textcolor{darkblue}{\textbf{\ipa{bv̩˧ʂæ˧ | ʐwæ˩˥}}} \textcolor{Sepia}{\selectlanguage{english}in a very good mood} \zh{脾气很好}  

\lhead{\firstmark}
\rhead{\botmark}

\subsection{\hspace{-0.5cm} {\Large \textcolor{darkblue}{\textbf{\ipa{bv̩˩ʂwæ˩}}}}\hspace{0.5cm}[\kern2pt{\textcolor{darkblue}{\textbf{\ipa{bv̩˧ʂwæ˧}}}}\kern2pt]} \hypertarget{bv\string_=\string_Bs`w\{\string_B1}{}
\markboth{\textcolor{darkblue}{\textbf{\ipa{bv̩˩ʂwæ˩}}}}{}
\textcolor{teal}{\mytextsc{noun}} \hspace{4pt} Tone: L.
\textcolor{Sepia}{\selectlanguage{english}Castrated yak.} \zh{阉割过的牦牛。}  ¶ \textcolor{darkblue}{\textbf{\ipa{bv̩˩ʂwæ˩-bv̩˥mi˩}}} \textcolor{Sepia}{\selectlanguage{english}castrated yak and female yak} \zh{阉割过的公牦牛与母牦牛}  
 \zh{量词}: \textcolor{darkblue}{\textbf{\ipa{pʰo˧˥}}}  \mytextsc{clf}: \textcolor{darkblue}{\textbf{\ipa{pʰo˧˥}}} 
\lhead{\firstmark}
\rhead{\botmark}

\subsection{\hspace{-0.5cm} {\Large \textcolor{darkblue}{\textbf{\ipa{bv̩˧tɕi˧}}}}\hspace{0.5cm}[\kern2pt{\textcolor{darkblue}{\textbf{\ipa{bv̩˩tɕi˩˥}}}}\kern2pt]} \hypertarget{bv\string_=\string_Mts£i\string_M1}{}
\markboth{\textcolor{darkblue}{\textbf{\ipa{bv̩˧tɕi˧}}}}{}
\textcolor{teal}{\mytextsc{noun}} \hspace{4pt} Tone: M.
\textcolor{Sepia}{\selectlanguage{english}Wild peach.} \zh{毛桃。}  \zh{量词}: \textcolor{darkblue}{\textbf{\ipa{tɕi˧˥}}} \textcolor{darkblue}{\textbf{\ipa{ɭɯ˧}}}  \mytextsc{clf}: \textcolor{darkblue}{\textbf{\ipa{tɕi˧˥}}} \textcolor{darkblue}{\textbf{\ipa{ɭɯ˧}}} 
\lhead{\firstmark}
\rhead{\botmark}

\subsection{\hspace{-0.5cm} {\Large \textcolor{darkblue}{\textbf{\ipa{bv̩˧ʈʂɯ˥}}}}\hspace{0.5cm}[\kern2pt{\textcolor{darkblue}{\textbf{\ipa{bv̩˧ʈʂɯ˧}}}}\kern2pt]} \hypertarget{bv\string_=\string_Mt`s`M\string_T1}{}
\markboth{\textcolor{darkblue}{\textbf{\ipa{bv̩˧ʈʂɯ˥}}}}{}
\textcolor{teal}{\mytextsc{noun}} \hspace{4pt} Tone: H\#.
\textcolor{Sepia}{\selectlanguage{english}Sifter, sieve.} \zh{筛子。}  \zh{量词}: \textcolor{darkblue}{\textbf{\ipa{nɑ˧}}}  \mytextsc{clf}: \textcolor{darkblue}{\textbf{\ipa{nɑ˧}}} 
\lhead{\firstmark}
\rhead{\botmark}

\subsection{\hspace{-0.5cm} {\Large \textcolor{darkblue}{\textbf{\ipa{bv̩˧ʈʂʰv̩˧}}}}\hspace{0.5cm}[\kern2pt{\textcolor{darkblue}{\textbf{\ipa{bv̩˧ʈʂʰv̩˥}}}}\kern2pt]} \hypertarget{bv\string_=\string_Mt`s`\string_hv\string_=\string_M1}{}
\markboth{\textcolor{darkblue}{\textbf{\ipa{bv̩˧ʈʂʰv̩˧}}}}{}
\textcolor{teal}{\mytextsc{noun}} \hspace{4pt} Tone: M.
\textcolor{Sepia}{\selectlanguage{english}Cymbals.} \zh{钹。}  \zh{量词}: \textcolor{darkblue}{\textbf{\ipa{nɑ˧}}}  \mytextsc{clf}: \textcolor{darkblue}{\textbf{\ipa{nɑ˧}}} 
\lhead{\firstmark}
\rhead{\botmark}

\subsection{\hspace{-0.5cm} {\Large \textcolor{darkblue}{\textbf{\ipa{bv̩˩zo˩}}}}\hspace{0.5cm}[\kern2pt{\textcolor{darkblue}{\textbf{\ipa{bv̩˧zo˧}}}}\kern2pt]} \hypertarget{bv\string_=\string_Bzo\string_B1}{}
\markboth{\textcolor{darkblue}{\textbf{\ipa{bv̩˩zo˩}}}}{}
\textcolor{teal}{\mytextsc{noun}} \hspace{4pt} Tone: L.
\textcolor{Sepia}{\selectlanguage{english}Small food steamer.} \zh{小蒸笼。}  \zh{量词}: \textcolor{darkblue}{\textbf{\ipa{ɭɯ˧}}}  \mytextsc{clf}: \textcolor{darkblue}{\textbf{\ipa{ɭɯ˧}}} 
\lhead{\firstmark}
\rhead{\botmark}

\subsection{\hspace{-0.5cm} {\Large \textcolor{darkblue}{\textbf{\ipa{bv̩˧zo\#˥}}}}\hspace{0.5cm}[\kern2pt{\textcolor{darkblue}{\textbf{\ipa{bv̩˩zo˩˥}}}}\kern2pt]} \hypertarget{bv\string_=\string_Mzo\#\string_T1}{}
\markboth{\textcolor{darkblue}{\textbf{\ipa{bv̩˧zo\#˥}}}}{}
\textcolor{teal}{\mytextsc{noun}} \hspace{4pt} Tone: \#H.
\textcolor{Sepia}{\selectlanguage{english}Yak calf (baby yak).} \zh{小牦牛。}  ¶ \textcolor{darkblue}{\textbf{\ipa{bv̩˧zo˧ tʰv̩˧-mi˧˥ / bv̩˧zo˧ tʰv̩˧-mi˥\#}}} \textcolor{Sepia}{\selectlanguage{english}\mytextsc{n}+\mytextsc{dem}+\mytextsc{clf}} \zh{这头小牦牛}  
 \zh{量词}: \textcolor{darkblue}{\textbf{\ipa{mi˩}}}  \mytextsc{clf}: \textcolor{darkblue}{\textbf{\ipa{mi˩}}} 
\lhead{\firstmark}
\rhead{\botmark}

\subsection{\hspace{-0.5cm} {\Large \textcolor{darkblue}{\textbf{\ipa{bv̩˩ʐv̩˩-dzi˩}}}}\hspace{0.5cm}[\kern2pt{\textcolor{darkblue}{\textbf{\ipa{xxxx non-correspondance entre le nombre de morphèmes et le nombre de tons de morphèmes}}}}\kern2pt]} \hypertarget{bv\string_=\string_Bz`v\string_=\string_B-dzi\string_B1}{}
\markboth{\textcolor{darkblue}{\textbf{\ipa{bv̩˩ʐv̩˩-dzi˩}}}}{}
\textcolor{teal}{\mytextsc{noun}} \hspace{4pt} Tone: L.
\textcolor{Sepia}{\selectlanguage{english}Ivy.} \zh{常春藤。}  \zh{量词}: \textcolor{darkblue}{\textbf{\ipa{dzi˩}}}  \mytextsc{clf}: \textcolor{darkblue}{\textbf{\ipa{dzi˩}}} 
\lhead{\firstmark}
\rhead{\botmark}

\subsection{\hspace{-0.5cm} {\Large \textcolor{darkblue}{\textbf{\ipa{bv̩˧ʐv̩˧-kʰv̩˧˥}}}}\hspace{0.5cm}[\kern2pt{\textcolor{darkblue}{\textbf{\ipa{xxxx non-correspondance entre le nombre de morphèmes et le nombre de tons de morphèmes}}}}\kern2pt]} \hypertarget{bv\string_=\string_Mz`v\string_=\string_M-k\string_hv\string_=\string_M\string_T1}{}
\markboth{\textcolor{darkblue}{\textbf{\ipa{bv̩˧ʐv̩˧-kʰv̩˧˥}}}}{}
\textcolor{teal}{\mytextsc{noun}} \hspace{4pt} Tone: MH\#.
\textcolor{Sepia}{\selectlanguage{english}Year of the serpent.} \zh{蛇年。} \textit{See:} \hyperlink{}{\textcolor{darkblue}{\textbf{\ipa{ʐv̩˧bæ˧}}}} 
\lhead{\firstmark}
\rhead{\botmark}

\subsection{\hspace{-0.5cm} {\Large \textcolor{darkblue}{\textbf{\ipa{‑bv˧}}}}\hspace{0.5cm}[\kern2pt{\textcolor{darkblue}{\textbf{\ipa{xxxx groupe tonal entier sans aucun ton}}}}\kern2pt]} \hypertarget{‑bv\string_M1}{}
\markboth{\textcolor{darkblue}{\textbf{\ipa{‑bv˧}}}}{}
\textcolor{teal}{\mytextsc{suffix}} \hspace{4pt} Tone: 0.
\textcolor{Sepia}{\selectlanguage{english}Possessive.} \zh{属式:的。} 
\lhead{\firstmark}
\rhead{\botmark}

\newpage
\section*{\centering- \textcolor{darkblue}{\textbf{\ipa{ɕ}}} -}
\subsection{\hspace{-0.5cm} {\Large \textcolor{darkblue}{\textbf{\ipa{ɕi˥\textsubscript{a}}}}}\hspace{0.5cm}[\kern2pt{\textcolor{darkblue}{\textbf{\ipa{ɕi˥}}}}\kern2pt]} \hypertarget{s£i\string_Ta1}{}
\markboth{\textcolor{darkblue}{\textbf{\ipa{ɕi˥\textsubscript{a}}}}}{}
\textcolor{teal}{\mytextsc{classifier}} \hspace{4pt} Tone: H\textsubscript{a}.
\textcolor{Sepia}{\selectlanguage{english}100.} \zh{百。}  ¶ \textcolor{darkblue}{\textbf{\ipa{ɖɯ˧-ɕi˥}}} \textcolor{Sepia}{\selectlanguage{english}one hundred} \zh{一百}  
 ¶ \textcolor{darkblue}{\textbf{\ipa{ɖɯ˧-ɕi˧ kʰv̩˧˥}}} \textcolor{Sepia}{\selectlanguage{english}one hundred years} \zh{一百年}  
 ¶ \textcolor{darkblue}{\textbf{\ipa{ɖɯ˧-ɕi˧ kʰv̩˧\textasciitilde{}ɖɯ˥-ɕi˩ kʰv̩˩}}} \textcolor{Sepia}{\selectlanguage{english}century after century} \zh{一百年又一百年}  
 ¶ \textcolor{darkblue}{\textbf{\ipa{ɕi˧-kʰv̩˧˥}}} \textcolor{Sepia}{\selectlanguage{english}a century, one hundred years (abridged formulation)} \zh{百年(“一百年”的省略说法)}  

\lhead{\firstmark}
\rhead{\botmark}

\subsection{\hspace{-0.5cm} {\Large \textcolor{darkblue}{\textbf{\ipa{ɕi˥\textsubscript{b}}}}}\hspace{0.5cm}[\kern2pt{\textcolor{darkblue}{\textbf{\ipa{ɕi˥}}}}\kern2pt]} \hypertarget{s£i\string_Tb1}{}
\markboth{\textcolor{darkblue}{\textbf{\ipa{ɕi˥\textsubscript{b}}}}}{}
\textcolor{teal}{\mytextsc{classifier}} \hspace{4pt} Tone: H\textsubscript{b}.
\textcolor{Sepia}{\selectlanguage{english}One hundredth of a yuan, one penny.} \zh{量词:分(一分钱)。} 
\lhead{\firstmark}
\rhead{\botmark}

\subsection{\hspace{-0.5cm} {\Large \textcolor{darkblue}{\textbf{\ipa{ɕi˧}}}}\hspace{0.5cm}[\kern2pt{\textcolor{darkblue}{\textbf{\ipa{ɕi˩˥}}}}\kern2pt]} \hypertarget{s£i\string_M1}{}
\markboth{\textcolor{darkblue}{\textbf{\ipa{ɕi˧}}}}{}
\textcolor{teal}{\mytextsc{noun}} \hspace{4pt} Tone: M.
\textcolor{Sepia}{\selectlanguage{english}Rice (monosyllable).} \zh{米(单音节)。} 
\lhead{\firstmark}
\rhead{\botmark}

\subsection{\hspace{-0.5cm} {\Large \textcolor{darkblue}{\textbf{\ipa{ɕi˧ɕi˩-lo˩}}}}\hspace{0.5cm}[\kern2pt{\textcolor{darkblue}{\textbf{\ipa{xxxx non-correspondance entre le nombre de morphèmes et le nombre de tons de morphèmes}}}}\kern2pt]} \hypertarget{s£i\string_Ms£i\string_B-lo\string_B1}{}
\markboth{\textcolor{darkblue}{\textbf{\ipa{ɕi˧ɕi˩-lo˩}}}}{}
\textcolor{teal}{\mytextsc{noun}} \hspace{4pt} Tone: L\#-.
\textcolor{Sepia}{\selectlanguage{english}The smallest cutlets.} \zh{最细的肋骨。} 
\lhead{\firstmark}
\rhead{\botmark}

\subsection{\hspace{-0.5cm} {\Large \textcolor{darkblue}{\textbf{\ipa{ɕi˧-ho˩ʂɯ˩}}}}\hspace{0.5cm}[\kern2pt{\textcolor{darkblue}{\textbf{\ipa{xxxx non-correspondance entre le nombre de morphèmes et le nombre de tons de morphèmes}}}}\kern2pt]} \hypertarget{s£i\string_M-ho\string_Bs`M\string_B1}{}
\markboth{\textcolor{darkblue}{\textbf{\ipa{ɕi˧-ho˩ʂɯ˩}}}}{}
\textcolor{teal}{\mytextsc{noun}} \hspace{4pt} Tone: -L.
\textcolor{Sepia}{\selectlanguage{english}Tomato.} \zh{西红柿(汉语借词)。}  Borrowing: Chinese  \zh{西红柿}

\lhead{\firstmark}
\rhead{\botmark}

\subsection{\hspace{-0.5cm} {\Large \textcolor{darkblue}{\textbf{\ipa{ɕi˧lv̩˧}}}}\hspace{0.5cm}[\kern2pt{\textcolor{darkblue}{\textbf{\ipa{ɕi˧lv̩˧}}}}\kern2pt]} \hypertarget{s£i\string_Mlv\string_=\string_M1}{}
\markboth{\textcolor{darkblue}{\textbf{\ipa{ɕi˧lv̩˧}}}}{}
\textcolor{teal}{\mytextsc{noun}} \hspace{4pt} Tone: M.
\textcolor{Sepia}{\selectlanguage{english}Paddy field.} \zh{水田。}  \zh{量词}: \textcolor{darkblue}{\textbf{\ipa{pʰæ˧˥, kɤ˧˥}}}  \mytextsc{clf}: \textcolor{darkblue}{\textbf{\ipa{pʰæ˧˥, kɤ˧˥}}} 
\lhead{\firstmark}
\rhead{\botmark}

\subsection{\hspace{-0.5cm} {\Large \textcolor{darkblue}{\textbf{\ipa{ɕi˧ɭɯ˧}}}}\hspace{0.5cm}[\kern2pt{\textcolor{darkblue}{\textbf{\ipa{xxxx non-correspondance entre le nombre de morphèmes et le nombre de tons de morphèmes}}}}\kern2pt]} \hypertarget{s£i\string_Ml\string_RM\string_M1}{}
\markboth{\textcolor{darkblue}{\textbf{\ipa{ɕi˧ɭɯ˧}}}}{}
\textcolor{teal}{\mytextsc{noun}} \hspace{4pt} Tone: M.
\textcolor{Sepia}{\selectlanguage{english}Paddy rice; by extension: paddy field.} \zh{稻子,稻田。}  \zh{量词}: \textcolor{darkblue}{\textbf{\ipa{kɤ˧˥ (pour le grain)}}} \textcolor{darkblue}{\textbf{\ipa{pʰæ˧˥ (pour un champ)}}}  \mytextsc{clf}: \textcolor{darkblue}{\textbf{\ipa{kɤ˧˥ (pour le grain)}}} \textcolor{darkblue}{\textbf{\ipa{pʰæ˧˥ (pour un champ)}}} 
\lhead{\firstmark}
\rhead{\botmark}

\subsection{\hspace{-0.5cm} {\Large \textcolor{darkblue}{\textbf{\ipa{ɕi˧ɭɯ˧-lv̩˧pʰv̩˩}}}}\hspace{0.5cm}[\kern2pt{\textcolor{darkblue}{\textbf{\ipa{xxxx non-correspondance entre le nombre de morphèmes et le nombre de tons de morphèmes}}}}\kern2pt]} \hypertarget{s£i\string_Ml\string_RM\string_M-lv\string_=\string_Mp\string_hv\string_=\string_B1}{}
\markboth{\textcolor{darkblue}{\textbf{\ipa{ɕi˧ɭɯ˧-lv̩˧pʰv̩˩}}}}{}
\textcolor{teal}{\mytextsc{noun}} \hspace{4pt} Tone: \mytextsc{L}\#.
\textcolor{Sepia}{\selectlanguage{english}Paddy field.} \zh{水田。}  \zh{量词}: \textcolor{darkblue}{\textbf{\ipa{pʰæ˧˥, kɤ˧˥}}}  \mytextsc{clf}: \textcolor{darkblue}{\textbf{\ipa{pʰæ˧˥, kɤ˧˥}}} 
\lhead{\firstmark}
\rhead{\botmark}

\subsection{\hspace{-0.5cm} {\Large \textcolor{darkblue}{\textbf{\ipa{ɕi˧tɕʰi\#˥}}}}\hspace{0.5cm}[\kern2pt{\textcolor{darkblue}{\textbf{\ipa{ɕi˧tɕʰi˧}}}}\kern2pt]} \hypertarget{s£i\string_Mts£\string_hi\#\string_T1}{}
\markboth{\textcolor{darkblue}{\textbf{\ipa{ɕi˧tɕʰi\#˥}}}}{}
\textcolor{teal}{\mytextsc{noun}} \hspace{4pt} Tone: \#H.
\textcolor{Sepia}{\selectlanguage{english}Chaff; bran; husk (of rice).} \zh{米糠。}  \zh{量词}: \textcolor{darkblue}{\textbf{\ipa{mɤ˩}}}  \mytextsc{clf}: \textcolor{darkblue}{\textbf{\ipa{mɤ˩}}} 
\lhead{\firstmark}
\rhead{\botmark}

\subsection{\hspace{-0.5cm} {\Large \textcolor{darkblue}{\textbf{\ipa{ɕi˧ʈʂʰwæ˧}}}}\hspace{0.5cm}[\kern2pt{\textcolor{darkblue}{\textbf{\ipa{ɕi˧ʈʂʰwæ˧}}}}\kern2pt]} \hypertarget{s£i\string_Mt`s`\string_hw\{\string_M1}{}
\markboth{\textcolor{darkblue}{\textbf{\ipa{ɕi˧ʈʂʰwæ˧}}}}{}
\textcolor{teal}{\mytextsc{noun}} \hspace{4pt} Tone: M.
\textcolor{Sepia}{\selectlanguage{english}Husked rice.} \zh{米。}  ¶ \textcolor{darkblue}{\textbf{\ipa{ɕi˧ʈʂʰwæ˧-hɑ˧}}} \textcolor{Sepia}{\selectlanguage{english}cooked rice; literally “cooked-rice food”, specifying the term \textcolor{darkblue}{\textbf{\ipa{/hɑ˥/}}}, which refers to food in general.} \zh{米饭}  

\lhead{\firstmark}
\rhead{\botmark}

\subsection{\hspace{-0.5cm} {\Large \textcolor{darkblue}{\textbf{\ipa{ɕi˩}}}}\hspace{0.5cm}[\kern2pt{\textcolor{darkblue}{\textbf{\ipa{ɕi˧˥}}}}\kern2pt]} \hypertarget{s£i\string_B1}{}
\markboth{\textcolor{darkblue}{\textbf{\ipa{ɕi˩}}}}{}
\textcolor{teal}{\mytextsc{verb}} \hspace{4pt} Tone: L\textsubscript{a}?.
\textit{\textcolor{Sepia}{\selectlanguage{english}archaic}} [\zh{古语}] \textcolor{Sepia}{\selectlanguage{english}To be afraid of.} \zh{怕、害怕。}  ¶ \textcolor{darkblue}{\textbf{\ipa{njɤ˧ | no˩ ɕi˩ tʰɑ˥-mɤ˩-ʝi˩! | njɤ˧ | no˩ ɖwæ˩ tʰɑ˥-mɤ˩-ʝi˩!}}} \textcolor{Sepia}{\selectlanguage{english}Don't you fancy I am afraid of you! / Don't you imagine you frighten me!} \zh{不要以为我害怕你!(挑衅的话)}  
 ¶ \textcolor{darkblue}{\textbf{\ipa{njɤ˧ | no˩ ɕi˩-mɤ˩-ʝi˥!}}} \textcolor{Sepia}{\selectlanguage{english}You don't frighten me! / I'm not afraid of you!} \zh{你不让我害怕 / 我不害怕你!}  
 ¶ \textcolor{darkblue}{\textbf{\ipa{njɤ˧ | tʰv̩˧ ɕi˩-mɤ˩-ʝi˩!}}} \textcolor{Sepia}{\selectlanguage{english}I'm not afraid of him!} \zh{我不怕他!}  
 ¶ \textcolor{darkblue}{\textbf{\ipa{njɤ˧ | ʈʂʰɯ˧-v̩˧ do˧˥, | ʁo˧ ɕi˧˥ | ʐwæ˩˥!}}} \textcolor{Sepia}{\selectlanguage{english}When I see him, I'm terribly afraid! / When I see him, I get frightened!} \zh{我见他,非常害怕!}  

\lhead{\firstmark}
\rhead{\botmark}

\subsection{\hspace{-0.5cm} {\Large \textcolor{darkblue}{\textbf{\ipa{ɕi˩dv̩˥}}}}\hspace{0.5cm}[\kern2pt{\textcolor{darkblue}{\textbf{\ipa{xxxx non-correspondance entre le nombre de morphèmes et le nombre de tons de morphèmes}}}}\kern2pt]} \hypertarget{s£i\string_Bdv\string_=\string_T1}{}
\markboth{\textcolor{darkblue}{\textbf{\ipa{ɕi˩dv̩˥}}}}{}
\textcolor{teal}{\mytextsc{noun}} \hspace{4pt} Tone: LH.
\textcolor{Sepia}{\selectlanguage{english}Incense.} \zh{香,香火。}  ¶ \textcolor{darkblue}{\textbf{\ipa{ɕi˩dv̩˥ qæ˩}}} \textcolor{Sepia}{\selectlanguage{english}to burn incense} \zh{烧香}  

\lhead{\firstmark}
\rhead{\botmark}

\subsection{\hspace{-0.5cm} {\Large \textcolor{darkblue}{\textbf{\ipa{ɕi˩dzi˥}}}}\hspace{0.5cm}[\kern2pt{\textcolor{darkblue}{\textbf{\ipa{ɕi˩dzi˥}}}}\kern2pt]} \hypertarget{s£i\string_Bdzi\string_T1}{}
\markboth{\textcolor{darkblue}{\textbf{\ipa{ɕi˩dzi˥}}}}{}
\textcolor{teal}{\mytextsc{noun}} \hspace{4pt} Tone: LH.
\textcolor{Sepia}{\selectlanguage{english}Cypress.} \zh{柏树。}  \zh{量词}: \textcolor{darkblue}{\textbf{\ipa{dzi˩}}}  \mytextsc{clf}: \textcolor{darkblue}{\textbf{\ipa{dzi˩}}} 
\lhead{\firstmark}
\rhead{\botmark}

\subsection{\hspace{-0.5cm} {\Large \textcolor{darkblue}{\textbf{\ipa{ɕi˩ʈʰæ˧˥}}} \textsubscript{1}}\hspace{0.5cm}[\kern2pt{\textcolor{darkblue}{\textbf{\ipa{ɕi˩ʈʰæ˧˥}}}}\kern2pt]} \hypertarget{s£i\string_Bt`\string_h\{\string_M\string_T1}{}
\markboth{\textcolor{darkblue}{\textbf{\ipa{ɕi˩ʈʰæ˧˥}}} \textsubscript{1}}{}
\textcolor{teal}{\mytextsc{adjective}} \hspace{4pt} Tone: LM+MH\#.
\textcolor{Sepia}{\selectlanguage{english}To be a stammerer; to have a stammer.} \zh{结巴。}  ¶ \textcolor{darkblue}{\textbf{\ipa{ʈʂʰɯ˧ | ɖɯ˧-pi˧˥ | ɕi˩ʈʰæ˧˥}}} \textcolor{Sepia}{\selectlanguage{english}(S)he has a stammer.} \zh{他有一点结巴。}  
 ¶ \textcolor{darkblue}{\textbf{\ipa{ʈʂʰɯ˧ | ɕi˩ʈʰæ˧-zo˥.}}} \textcolor{Sepia}{\selectlanguage{english}(S)he stammers a lot.} \zh{他很结巴。}  
\textit{See:} \hyperlink{}{\textcolor{darkblue}{\textbf{\ipa{ɕi˩ʈʰæ˧˥}}} \textsubscript{2}} 
\lhead{\firstmark}
\rhead{\botmark}

\subsection{\hspace{-0.5cm} {\Large \textcolor{darkblue}{\textbf{\ipa{ɕi˩ʈʰæ˧˥}}} \textsubscript{2}}\hspace{0.5cm}[\kern2pt{\textcolor{darkblue}{\textbf{\ipa{ɕi˩ʈʰæ˧˥}}}}\kern2pt]} \hypertarget{s£i\string_Bt`\string_h\{\string_M\string_T2}{}
\markboth{\textcolor{darkblue}{\textbf{\ipa{ɕi˩ʈʰæ˧˥}}} \textsubscript{2}}{}
\textcolor{teal}{\mytextsc{noun}} \hspace{4pt} Tone: LM+MH\#.
\textcolor{Sepia}{\selectlanguage{english}Stammerer, stutterer.} \zh{结巴。}  ¶ \textcolor{darkblue}{\textbf{\ipa{ʈʂʰɯ˧ ɕi˩ʈʰæ˧ ɲi˥}}} \textcolor{Sepia}{\selectlanguage{english}(S)he is a stammerer.} \zh{他是结巴。}  
\textit{See:} \hyperlink{}{\textcolor{darkblue}{\textbf{\ipa{ɕi˩ʈʰæ˧˥}}} \textsubscript{1}} 
\lhead{\firstmark}
\rhead{\botmark}

\subsection{\hspace{-0.5cm} {\Large \textcolor{darkblue}{\textbf{\ipa{ɕi˩˥}}}}\hspace{0.5cm}[\kern2pt{\textcolor{darkblue}{\textbf{\ipa{ɕi˥}}}}\kern2pt]} \hypertarget{s£i\string_B\string_T1}{}
\markboth{\textcolor{darkblue}{\textbf{\ipa{ɕi˩˥}}}}{}
\textcolor{teal}{\mytextsc{noun}} \hspace{4pt} Tone: LH.
\textcolor{Sepia}{\selectlanguage{english}Incense (second syllable).} \zh{香(单音节)。}  ¶ \textcolor{darkblue}{\textbf{\ipa{ɕi˩ qæ˧˥}}} \textcolor{Sepia}{\selectlanguage{english}to burn incense} \zh{烧香}  

\lhead{\firstmark}
\rhead{\botmark}

\subsection{\hspace{-0.5cm} {\Large \textcolor{darkblue}{\textbf{\ipa{ɕjɤ˥}}}}\hspace{0.5cm}[\kern2pt{\textcolor{darkblue}{\textbf{\ipa{ɕjɤ˥}}}}\kern2pt]} \hypertarget{s£j7\string_T1}{}
\markboth{\textcolor{darkblue}{\textbf{\ipa{ɕjɤ˥}}}}{}
\textcolor{teal}{\mytextsc{verb}} \hspace{4pt} Tone: H.
\textcolor{Sepia}{\selectlanguage{english}To invent, to think out/up, to come up with (an idea, a solution).} \zh{发明、想出、找到(办法)。}  ¶ \textcolor{darkblue}{\textbf{\ipa{le˧-ɕjɤ˥}}} \textcolor{Sepia}{\selectlanguage{english}\mytextsc{accomp}} \zh{想出了}  
 ¶ \textcolor{darkblue}{\textbf{\ipa{ʈʂʰɯ˧ | pæ˧˥hwɤ˧ | ɕjɤ˧ ɣɯ˧!}}} \textcolor{Sepia}{\selectlanguage{english}(S)he knows to find solutions under all circumstances! / (S)he is good of finding solutions to all problems!} \zh{他很会想办法的!}  

\lhead{\firstmark}
\rhead{\botmark}

\subsection{\hspace{-0.5cm} {\Large \textcolor{darkblue}{\textbf{\ipa{ɕjɤ˧-bv̩˧nv̩˧}}}}\hspace{0.5cm}[\kern2pt{\textcolor{darkblue}{\textbf{\ipa{xxxx non-correspondance entre le nombre de morphèmes et le nombre de tons de morphèmes}}}}\kern2pt]} \hypertarget{s£j7\string_M-bv\string_=\string_Mnv\string_=\string_M1}{}
\markboth{\textcolor{darkblue}{\textbf{\ipa{ɕjɤ˧-bv̩˧nv̩˧}}}}{}
\textcolor{teal}{\mytextsc{adjective}} \hspace{4pt} Tone: M.
\textit{From:} \textbf{bv̩˧nv̩˧} \textcolor{Sepia}{\selectlanguage{english}Good (smell), fragrant.} \zh{香(气味)。}  ¶ \textcolor{darkblue}{\textbf{\ipa{ʈʂʰɯ˧ ɕjɤ˧-bv̩˧nv̩˧ ɲi˩.}}} \textcolor{Sepia}{\selectlanguage{english}It smells good.} \zh{这很香(气味香)。}  

\lhead{\firstmark}
\rhead{\botmark}

\subsection{\hspace{-0.5cm} {\Large \textcolor{darkblue}{\textbf{\ipa{ɕjɤ˩\textasciitilde{}ɕjɤ˩}}}}\hspace{0.5cm}[\kern2pt{\textcolor{darkblue}{\textbf{\ipa{ɕjɤ˩ɕjɤ˩˥}}}}\kern2pt]} \hypertarget{s£j7\string_B~s£j7\string_B1}{}
\markboth{\textcolor{darkblue}{\textbf{\ipa{ɕjɤ˩\textasciitilde{}ɕjɤ˩}}}}{}
\textcolor{teal}{\mytextsc{verb}} \hspace{4pt} Tone: L.
\textcolor{Sepia}{\selectlanguage{english}To browbeat, to ill-treat.} \zh{欺负。}  ¶ \textcolor{darkblue}{\textbf{\ipa{hĩ˧ ɕjɤ˥\textasciitilde{}ɕjɤ˩}}} \textcolor{Sepia}{\selectlanguage{english}to ill-treat someone, to ill-treat people} \zh{欺负人}  
 ¶ \textcolor{darkblue}{\textbf{\ipa{no˧ | njɤ˩ ɕjɤ˩\textasciitilde{}ɕjɤ˩-mv̩˩-zo˩˥! / no˧ | njɤ˩ ɕjɤ˩\textasciitilde{}ɕjɤ˩˥!}}} \textcolor{Sepia}{\selectlanguage{english}You are treating me badly! / You are bullying me!} \zh{你对我不好!你欺负我!}  
 ¶ \textcolor{darkblue}{\textbf{\ipa{no˧ | njɤ˩ ɕjɤ˩\textasciitilde{}ɕjɤ˩-ze˥!}}} \textcolor{Sepia}{\selectlanguage{english}You have treated me badly! / You have bullied me!} \zh{你欺负了我!}  

\lhead{\firstmark}
\rhead{\botmark}

\subsection{\hspace{-0.5cm} {\Large \textcolor{darkblue}{\textbf{\ipa{ɕjɤ˩jo˩}}}}\hspace{0.5cm}[\kern2pt{\textcolor{darkblue}{\textbf{\ipa{ɕjɤ˩jo˩˥}}}}\kern2pt]} \hypertarget{s£j7\string_Bjo\string_B1}{}
\markboth{\textcolor{darkblue}{\textbf{\ipa{ɕjɤ˩jo˩}}}}{}
\textcolor{teal}{\mytextsc{noun}} \hspace{4pt} Tone: L.
\textcolor{Sepia}{\selectlanguage{english}\textit{Fritillaria cirrhosa}.} \zh{贝母。}  \zh{量词}: \textcolor{darkblue}{\textbf{\ipa{ɭɯ˧}}}  \mytextsc{clf}: \textcolor{darkblue}{\textbf{\ipa{ɭɯ˧}}} 
\lhead{\firstmark}
\rhead{\botmark}

\subsection{\hspace{-0.5cm} {\Large \textcolor{darkblue}{\textbf{\ipa{ɕjɤ˩tʰv̩˧˥}}}}\hspace{0.5cm}[\kern2pt{\textcolor{darkblue}{\textbf{\ipa{ɕjɤ˩tʰv̩˧˥}}}}\kern2pt]} \hypertarget{s£j7\string_Bt\string_hv\string_=\string_M\string_T1}{}
\markboth{\textcolor{darkblue}{\textbf{\ipa{ɕjɤ˩tʰv̩˧˥}}}}{}
\textcolor{teal}{\mytextsc{verb}} \hspace{4pt} Tone: LM+MH\#.
\textcolor{Sepia}{\selectlanguage{english}To insult; to criticize.} \zh{骂,批评。}  ¶ \textcolor{darkblue}{\textbf{\ipa{hĩ˧ ɕjɤ˥tʰv̩˩}}} \textcolor{Sepia}{\selectlanguage{english}to insult people; to criticize people} \zh{骂人、批评人}  

\lhead{\firstmark}
\rhead{\botmark}

\subsection{\hspace{-0.5cm} {\Large \textcolor{darkblue}{\textbf{\ipa{ɕjɤ˧˥}}}}\hspace{0.5cm}[\kern2pt{\textcolor{darkblue}{\textbf{\ipa{ɕjɤ˧˥}}}}\kern2pt]} \hypertarget{s£j7\string_M\string_T1}{}
\markboth{\textcolor{darkblue}{\textbf{\ipa{ɕjɤ˧˥}}}}{}
\textcolor{teal}{\mytextsc{verb}} \hspace{4pt} Tone: MH.
\textcolor{Sepia}{\selectlanguage{english}To try; to taste.} \zh{尝试、体会、经过。}  ¶ \textcolor{darkblue}{\textbf{\ipa{le˧-ɕjɤ˧-ze˥}}} \textcolor{Sepia}{\selectlanguage{english}\mytextsc{accomp} \string_ \mytextsc{pfv}} \zh{试了}  
 ¶ \textcolor{darkblue}{\textbf{\ipa{tso˧\textasciitilde{}tso˧ ɕjɤ˩}}} \textcolor{Sepia}{\selectlanguage{english}to taste something} \zh{尝一个东西}  
 ¶ \textcolor{darkblue}{\textbf{\ipa{no˧ ɖɯ˧-kʰwɤ˥ ɕjɤ˩!}}} \textcolor{Sepia}{\selectlanguage{english}Have a taste! / Taste a bite!} \zh{你尝一口吧!}  
 ¶ \textcolor{darkblue}{\textbf{\ipa{ɖɯ˧-ɕjɤ˧-ɻ̍˥!}}} \textcolor{Sepia}{\selectlanguage{english}Have a try!} \zh{尝一尝吧! / 试一试吧!}  

\lhead{\firstmark}
\rhead{\botmark}

\subsection{\hspace{-0.5cm} {\Large \textcolor{darkblue}{\textbf{\ipa{ɕjo˩li\#˥}}}}\hspace{0.5cm}[\kern2pt{\textcolor{darkblue}{\textbf{\ipa{ɕjo˩li˥}}}}\kern2pt]} \hypertarget{s£jo\string_Bli\#\string_T1}{}
\markboth{\textcolor{darkblue}{\textbf{\ipa{ɕjo˩li\#˥}}}}{}
\textcolor{teal}{\mytextsc{noun}} \hspace{4pt} Tone: LM+\#H.
\textcolor{Sepia}{\selectlanguage{english}Flute.} \zh{笛子。}  \zh{量词}: \textcolor{darkblue}{\textbf{\ipa{ɭɯ˧}}}  \mytextsc{clf}: \textcolor{darkblue}{\textbf{\ipa{ɭɯ˧}}} 
\lhead{\firstmark}
\rhead{\botmark}

\subsection{\hspace{-0.5cm} {\Large \textcolor{darkblue}{\textbf{\ipa{ɕɯ˩\textsubscript{a}}}}}\hspace{0.5cm}[\kern2pt{\textcolor{darkblue}{\textbf{\ipa{ɕɯ˩˥}}}}\kern2pt]} \hypertarget{s£M\string_Ba1}{}
\markboth{\textcolor{darkblue}{\textbf{\ipa{ɕɯ˩\textsubscript{a}}}}}{}
\textcolor{teal}{\mytextsc{verb}} \hspace{4pt} Tone: L\textsubscript{a}.
\textcolor{Sepia}{\selectlanguage{english}To raise.} \zh{养。}  ¶ \textcolor{darkblue}{\textbf{\ipa{ɕɯ˩zo\#˥}}} \textcolor{Sepia}{\selectlanguage{english}adopted child} \zh{养儿}  
 ¶ \textcolor{darkblue}{\textbf{\ipa{ho˧zo˧-ɕɯ˧zo˥, | æ̃˩ mɤ˧-tsɤ˧! | hĩ˧-zo˧mv˥, | ʐɤ˧ tʰɑ˧-mɤ˧-ʝi˧!}}} \textcolor{Sepia}{\selectlanguage{english}The adopted baby pheasant does not become a chicken (=does not become domesticated)! One should not bring up other people's children! (Proverb which does not apply to the adoption of children who have lost ties with their biological family, but to the adoption of children who remain in touch with their relatives: no matter how much care one puts into bringing them up, they remain more attached to their lineage.)} \zh{养的小雉,不会变成鸡!人家的孩子,不要养!(指的不是领养孤儿,而是养别人的孩子:无论多么关心孩子,他还是会更爱自己原来的家人。)}  

\lhead{\firstmark}
\rhead{\botmark}

\newpage
\section*{\centering- \textcolor{darkblue}{\textbf{\ipa{d}}} -}
\subsection{\hspace{-0.5cm} {\Large \textcolor{darkblue}{\textbf{\ipa{dɑ˧ʝi˩}}}}\hspace{0.5cm}[\kern2pt{\textcolor{darkblue}{\textbf{\ipa{dɑ˧ʝi˩}}}}\kern2pt]} \hypertarget{dA\string_Mj££i\string_B1}{}
\markboth{\textcolor{darkblue}{\textbf{\ipa{dɑ˧ʝi˩}}}}{}
\textcolor{teal}{\mytextsc{noun}} \hspace{4pt} Tone: L\#.
\textcolor{Sepia}{\selectlanguage{english}Mule.} \zh{骡子。}  ¶ \textcolor{darkblue}{\textbf{\ipa{dɑ˧ʝi˩-dʑo˩, | ɖɯ˩mi˧ dʑo˧-kv̩˥-mæ˩! | ɖɯ˩zo˧ dʑo˧-kv̩˥-mæ˩!}}} \textcolor{Sepia}{\selectlanguage{english}As for mules, there exist female mules, (and) male mules! / Among mules, there is a distinction between females and males! (Explanation provided to a city dweller on a visit, who knew precious little about animal breeding.)} \zh{骡子呢,有母骡子!(也)有公骡子! / 骡子,分母的和公的!(这个说明是给一个不懂畜牧业的城里人听)}  
 \zh{量词}: \textcolor{darkblue}{\textbf{\ipa{v̩˧}}}  \mytextsc{clf}: \textcolor{darkblue}{\textbf{\ipa{v̩˧}}} 
\lhead{\firstmark}
\rhead{\botmark}

\subsection{\hspace{-0.5cm} {\Large \textcolor{darkblue}{\textbf{\ipa{dɑ˧pɤ˧}}}}\hspace{0.5cm}[\kern2pt{\textcolor{darkblue}{\textbf{\ipa{dɑ˧pɤ˧}}}}\kern2pt]} \hypertarget{dA\string_Mp7\string_M1}{}
\markboth{\textcolor{darkblue}{\textbf{\ipa{dɑ˧pɤ˧}}}}{}
\textcolor{teal}{\mytextsc{noun}} \hspace{4pt} Tone: M.
\textcolor{Sepia}{\selectlanguage{english}Priest of the local religion.} \zh{宗教礼师。音译:达巴。}  ¶ \textcolor{darkblue}{\textbf{\ipa{dɑ˧pɤ˧ ʝi˧-hĩ˧ hĩ˧}}} \textcolor{Sepia}{\selectlanguage{english}priest, person who performs the function of priest} \zh{当达巴的人}  
 \zh{量词}: \textcolor{darkblue}{\textbf{\ipa{v̩˧}}}  \mytextsc{clf}: \textcolor{darkblue}{\textbf{\ipa{v̩˧}}} 
\lhead{\firstmark}
\rhead{\botmark}

\subsection{\hspace{-0.5cm} {\Large \textcolor{darkblue}{\textbf{\ipa{dɑ˧pv̩\#˥}}}}\hspace{0.5cm}[\kern2pt{\textcolor{darkblue}{\textbf{\ipa{dɑ˧pv̩˧}}}}\kern2pt]} \hypertarget{dA\string_Mpv\string_=\#\string_T1}{}
\markboth{\textcolor{darkblue}{\textbf{\ipa{dɑ˧pv̩\#˥}}}}{}
\textcolor{teal}{\mytextsc{noun}} \hspace{4pt} Tone: \#H.
\textcolor{Sepia}{\selectlanguage{english}Host.} \zh{主人。}  ¶ \textcolor{darkblue}{\textbf{\ipa{ʑi˧dv̩˧ dɑ˧pv̩˧}}} \textcolor{Sepia}{\selectlanguage{english}the family host, the member of the family who has the role of host} \zh{家的主人}  
 ¶ \textcolor{darkblue}{\textbf{\ipa{ʑi˧dv̩˧-ʝi˧-hĩ˧ dɑ˧pv̩˧}}} \textcolor{Sepia}{\selectlanguage{english}ditto} \zh{同上}  
 \zh{量词}: \textcolor{darkblue}{\textbf{\ipa{v̩˧}}}  \mytextsc{clf}: \textcolor{darkblue}{\textbf{\ipa{v̩˧}}} 
\lhead{\firstmark}
\rhead{\botmark}

\subsection{\hspace{-0.5cm} {\Large \textcolor{darkblue}{\textbf{\ipa{dɑ˧pʰo˥}}}}\hspace{0.5cm}[\kern2pt{\textcolor{darkblue}{\textbf{\ipa{dɑ˩pʰo˥}}}}\kern2pt]} \hypertarget{dA\string_Mp\string_ho\string_T1}{}
\markboth{\textcolor{darkblue}{\textbf{\ipa{dɑ˧pʰo˥}}}}{}
\textcolor{teal}{\mytextsc{noun}} \hspace{4pt} Tone: LH.
\textcolor{Sepia}{\selectlanguage{english}Dapo.} \zh{达坡(永宁的一个村落)。}  ¶ \textcolor{darkblue}{\textbf{\ipa{ɖæ˩ʂɯ\#˥, | ʈʂo˧ʂɯ\#˥, | bɤ˩tɕʰɯ˩˥, | dɑ˧pʰo˥, | bɤ˧dzi˩, | dze˧bo˧}}} \textcolor{Sepia}{\selectlanguage{english}the six villages of the plain of Yongning, in traditional order: by order of increasing distance from the Lake} \zh{永宁坝的六个村落,按传统排序:从距离泸沽湖最近的村落说起。}  

\lhead{\firstmark}
\rhead{\botmark}

\subsection{\hspace{-0.5cm} {\Large \textcolor{darkblue}{\textbf{\ipa{dɑ˧ʁwɤ\#˥}}}}\hspace{0.5cm}[\kern2pt{\textcolor{darkblue}{\textbf{\ipa{dɑ˧ʁwɤ˧}}}}\kern2pt]} \hypertarget{dA\string_MRw7\#\string_T1}{}
\markboth{\textcolor{darkblue}{\textbf{\ipa{dɑ˧ʁwɤ\#˥}}}}{}
\textcolor{teal}{\mytextsc{noun}} \hspace{4pt} Tone: \#H.
\textcolor{Sepia}{\selectlanguage{english}A village downstream from Qiansuo; the language spoken there is reported to be close to that of the Yongning plain.} \zh{一个村落,在前所的下游。据说那边的方言跟丽江坝比较接近。} 
\lhead{\firstmark}
\rhead{\botmark}

\subsection{\hspace{-0.5cm} {\Large \textcolor{darkblue}{\textbf{\ipa{dɑ˩}}}}\hspace{0.5cm}[\kern2pt{\textcolor{darkblue}{\textbf{\ipa{dɑ˩˥}}}}\kern2pt]} \hypertarget{dA\string_B1}{}
\markboth{\textcolor{darkblue}{\textbf{\ipa{dɑ˩}}}}{}
\textcolor{teal}{\mytextsc{adjective}} \hspace{4pt} Tone: L.
\textit{\textcolor{Sepia}{\selectlanguage{english}archaic}} [\zh{古语}] \textcolor{Sepia}{\selectlanguage{english}Happy.} \zh{幸福、平安、安好。}  ¶ \textcolor{darkblue}{\textbf{\ipa{mɤ˧-dɑ˩-qʰwɤ˩}}} \textcolor{Sepia}{\selectlanguage{english}Melancholy song; song telling of one's unhappiness. This is one of the genres of singing, lamenting one's hardships.} \zh{悲情歌,讲述自己日子痛苦}  
 ¶ \textcolor{darkblue}{\textbf{\ipa{mɤ˧-dɑ˩!}}} \textcolor{Sepia}{\selectlanguage{english}Introductory formula for melancholy songs, and sometimes at the beginning of stories. (The same formula is also used in the Laze language.)} \zh{悲情歌的开头词}  
 ¶ \textcolor{darkblue}{\textbf{\ipa{mɤ˧-dɑ˩-mi˩}}} \textcolor{Sepia}{\selectlanguage{english}As above: same meaning as the form without a \zh{/-mi/} suffix.} \zh{同上}  
 ¶ \textcolor{darkblue}{\textbf{\ipa{ɖwæ˧˥ | hɤ˩-dɑ˥! | ɖwæ˧˥ | hɤ˩˥!}}} \textcolor{Sepia}{\selectlanguage{english}Good job! / Well done! (Context: compliment to a toddler who has climbed on a piece of furniture.)} \zh{很了不起啊!(情景:表扬一个小孩子成功地爬上了一个家具)}  
 ¶ \textcolor{darkblue}{\textbf{\ipa{ɖʐɯ˩dɑ˥-kʰɤ˩dɑ˩-ɻ̍˩}}} \textcolor{Sepia}{\selectlanguage{english}All is well. / All is for the best. (Used for instance to describe a period without food shortage, earthquake, epidemic, war or other catastrophe)} \zh{一切都安好。(如:来指一段时间没有饥荒、地震、流行病、战争等灾难)}  

\lhead{\firstmark}
\rhead{\botmark}

\subsection{\hspace{-0.5cm} {\Large \textcolor{darkblue}{\textbf{\ipa{dɑ˩\textsubscript{b}}}}}\hspace{0.5cm}[\kern2pt{\textcolor{darkblue}{\textbf{\ipa{dɑ˩˥}}}}\kern2pt]} \hypertarget{dA\string_Bb1}{}
\markboth{\textcolor{darkblue}{\textbf{\ipa{dɑ˩\textsubscript{b}}}}}{}
\textcolor{teal}{\mytextsc{verb}} \hspace{4pt} Tone: L\textsubscript{b}.
\textcolor{Sepia}{\selectlanguage{english}To weave.} \zh{织。}  ¶ \textcolor{darkblue}{\textbf{\ipa{ɣɯ˧ dɑ˩}}} \textcolor{Sepia}{\selectlanguage{english}to weave fabric} \zh{织布}  
 ¶ \textcolor{darkblue}{\textbf{\ipa{ɣɯ˧ | le˧-dɑ˩}}} \textcolor{Sepia}{\selectlanguage{english}to weave fabric} \zh{织布}  
 ¶ \textcolor{darkblue}{\textbf{\ipa{tso˧\textasciitilde{}tso˧ dɑ˩}}} \textcolor{Sepia}{\selectlanguage{english}to weave things} \zh{织东西}  
 ¶ \textcolor{darkblue}{\textbf{\ipa{ɖɯ˧-dɑ˧\textasciitilde{}dɑ˩-ɻ̍˩}}} \textcolor{Sepia}{\selectlanguage{english}\mytextsc{delimitative} \string_ \mytextsc{red} \mytextsc{inceptive}} \zh{织一下}  

\lhead{\firstmark}
\rhead{\botmark}

\subsection{\hspace{-0.5cm} {\Large \textcolor{darkblue}{\textbf{\ipa{dɑ˩kʰɤ˩}}}}\hspace{0.5cm}[\kern2pt{\textcolor{darkblue}{\textbf{\ipa{dɑ˩kʰɤ˩˥}}}}\kern2pt]} \hypertarget{dA\string_Bk\string_h7\string_B1}{}
\markboth{\textcolor{darkblue}{\textbf{\ipa{dɑ˩kʰɤ˩}}}}{}
\textcolor{teal}{\mytextsc{noun}} \hspace{4pt} Tone: L.
\textcolor{Sepia}{\selectlanguage{english}Drum.} \zh{鼓。}  ¶ \textcolor{darkblue}{\textbf{\ipa{dɑ˩kʰɤ˩ lɑ˥(-ze˩)}}} \textcolor{Sepia}{\selectlanguage{english}to play a drum} \zh{打鼓}  
 \zh{量词}: \textcolor{darkblue}{\textbf{\ipa{ɭɯ˧}}}  \mytextsc{clf}: \textcolor{darkblue}{\textbf{\ipa{ɭɯ˧}}} 
\lhead{\firstmark}
\rhead{\botmark}

\subsection{\hspace{-0.5cm} {\Large \textcolor{darkblue}{\textbf{\ipa{dɑ˩to\#˥}}}}\hspace{0.5cm}[\kern2pt{\textcolor{darkblue}{\textbf{\ipa{dɑ˩to˥}}}}\kern2pt]} \hypertarget{dA\string_Bto\#\string_T1}{}
\markboth{\textcolor{darkblue}{\textbf{\ipa{dɑ˩to\#˥}}}}{}
\textcolor{teal}{\mytextsc{adverb(ial)}} \hspace{4pt} Tone: LM+\#H.
\textcolor{Sepia}{\selectlanguage{english}Politely. This term was only observed in association with the verb 'to say', with the meaning 'to say polite words, polite small-talk'.} \zh{客气地。}  ¶ \textcolor{darkblue}{\textbf{\ipa{dɑ˩to˧ ʐwɤ˧˥}}} \textcolor{Sepia}{\selectlanguage{english}to say some polite things} \zh{说客气话}  
 ¶ \textcolor{darkblue}{\textbf{\ipa{dɑ˩to˧ ʐwɤ˧-hĩ˥-lɑ˩ ɲi˩!}}} \textcolor{Sepia}{\selectlanguage{english}It's just polite words! (Comment about an invitation by a neighbour, which was intended to be declined: it was not a true invitation.)} \zh{这只是客气话而已!}  

\lhead{\firstmark}
\rhead{\botmark}

\subsection{\hspace{-0.5cm} {\Large \textcolor{darkblue}{\textbf{\ipa{dɑ˩to˩}}}}\hspace{0.5cm}[\kern2pt{\textcolor{darkblue}{\textbf{\ipa{dɑ˩to˩˥}}}}\kern2pt]} \hypertarget{dA\string_Bto\string_B1}{}
\markboth{\textcolor{darkblue}{\textbf{\ipa{dɑ˩to˩}}}}{}
\textcolor{teal}{\mytextsc{adverb(ial)}} \hspace{4pt} Tone: L.
\textcolor{Sepia}{\selectlanguage{english}At bottom, in reality, when all is said and done.} \zh{说到底,根本上,归根结底。} 
\lhead{\firstmark}
\rhead{\botmark}

\subsection{\hspace{-0.5cm} {\Large \textcolor{darkblue}{\textbf{\ipa{dɑ˧˥}}}}\hspace{0.5cm}[\kern2pt{\textcolor{darkblue}{\textbf{\ipa{dɑ˧˥}}}}\kern2pt]} \hypertarget{dA\string_M\string_T1}{}
\markboth{\textcolor{darkblue}{\textbf{\ipa{dɑ˧˥}}}}{}
\textcolor{teal}{\mytextsc{noun}} \hspace{4pt} Tone: MH.
\textcolor{Sepia}{\selectlanguage{english}Misfortune, mishaps.} Local Chinese dialect:\zh{苦。} ¶ \textcolor{darkblue}{\textbf{\ipa{dɑ˧-ʐwɤ˧˥}}} \textcolor{Sepia}{\selectlanguage{english}to complain, to tell one's misfortunes} \zh{诉苦}  
 ¶ \textcolor{darkblue}{\textbf{\ipa{ɻ̃˧-ʐwɤ˧ | dɑ˧-ʐwɤ˧-ɻ̍˥}}} \textcolor{Sepia}{\selectlanguage{english}to bemoan one's misfortunes} \zh{讲自己的不幸}  
 ¶ \textcolor{darkblue}{\textbf{\ipa{ʈʂʰɯ˧ | mɑ˧dɑ˩-qʰwɤ˩, | ɻ̃˧-ʐwɤ˧ | dɑ˧-ʐwɤ˧-ɻ̍˥!}}} \textcolor{Sepia}{\selectlanguage{english}(S)he is unhappy; (s)he is constantly complaining!} \zh{他不幸福,他一直在诉苦!}  

\lhead{\firstmark}
\rhead{\botmark}

\subsection{\hspace{-0.5cm} {\Large \textcolor{darkblue}{\textbf{\ipa{dɑ˧˥}}} \textsubscript{1}}\hspace{0.5cm}[\kern2pt{\textcolor{darkblue}{\textbf{\ipa{dɑ˧˥}}}}\kern2pt]} \hypertarget{dA\string_M\string_T1}{}
\markboth{\textcolor{darkblue}{\textbf{\ipa{dɑ˧˥}}} \textsubscript{1}}{}
\textcolor{teal}{\mytextsc{verb}} \hspace{4pt} Tone: MH.
\textcolor{Sepia}{\selectlanguage{english}To build (a house...).} \zh{建(房子)。}  ¶ \textcolor{darkblue}{\textbf{\ipa{ʑi˧mi˧ dɑ˧˥}}} \textcolor{Sepia}{\selectlanguage{english}to build a house} \zh{建房}  

\lhead{\firstmark}
\rhead{\botmark}

\subsection{\hspace{-0.5cm} {\Large \textcolor{darkblue}{\textbf{\ipa{dɑ˧˥}}} \textsubscript{2}}\hspace{0.5cm}[\kern2pt{\textcolor{darkblue}{\textbf{\ipa{dɑ˧˥}}}}\kern2pt]} \hypertarget{dA\string_M\string_T2}{}
\markboth{\textcolor{darkblue}{\textbf{\ipa{dɑ˧˥}}} \textsubscript{2}}{}
\textcolor{teal}{\mytextsc{verb}} \hspace{4pt} Tone: MH.
\textcolor{Sepia}{\selectlanguage{english}To fell (a tree); to cut into pieces (a large piece of meat); to create a breach (in a dike).} \zh{砍(树),割(肉)。}  ¶ \textcolor{darkblue}{\textbf{\ipa{le˧-dɑ˧-ze˥}}} \textcolor{Sepia}{\selectlanguage{english}\mytextsc{accomp} \string_ \mytextsc{pfv}} \zh{砍了(树),割了(肉)}  
 ¶ \textcolor{darkblue}{\textbf{\ipa{ɖɯ˧-dɑ˧ tʰi˥-dɑ˩}}} \textcolor{Sepia}{\selectlanguage{english}to hit a blow} \zh{砍一下}  
 ¶ \textcolor{darkblue}{\textbf{\ipa{dɑ˩\textasciitilde{}dɑ˧˥}}} \textcolor{Sepia}{\selectlanguage{english}\mytextsc{red}} \zh{\mytextsc{重叠}}  
 ¶ \textcolor{darkblue}{\textbf{\ipa{le˧-dɑ˩\textasciitilde{}dɑ˩(-ze˩)}}} \textcolor{Sepia}{\selectlanguage{english}(I) have cut (e.g. a chicken) into pieces} \zh{(我把一只鸡)割成块了}  
 ¶ \textcolor{darkblue}{\textbf{\ipa{ʂe˧ dɑ˥\textasciitilde{}dɑ˩}}} \textcolor{Sepia}{\selectlanguage{english}to cut meat to pieces, to mince meat} \zh{把肉剁碎}  

\lhead{\firstmark}
\rhead{\botmark}

\subsection{\hspace{-0.5cm} {\Large \textcolor{darkblue}{\textbf{\ipa{dɑ˧˥\textsubscript{b}}}}}\hspace{0.5cm}[\kern2pt{\textcolor{darkblue}{\textbf{\ipa{dɑ˧˥}}}}\kern2pt]} \hypertarget{dA\string_M\string_Tb1}{}
\markboth{\textcolor{darkblue}{\textbf{\ipa{dɑ˧˥\textsubscript{b}}}}}{}
\textcolor{teal}{\mytextsc{classifier}} \hspace{4pt} Tone: MH\textsubscript{b}.
\textcolor{Sepia}{\selectlanguage{english}Self-classifier for blows.} \zh{量词:下(打一下)。}  ¶ \textcolor{darkblue}{\textbf{\ipa{ɖɯ˧-dɑ˧˥}}} \textcolor{Sepia}{\selectlanguage{english}a blow} \zh{当头一棒}  
 ¶ \textcolor{darkblue}{\textbf{\ipa{ɖɯ˧-dɑ˧ tʰi˥-dɑ˩}}} \textcolor{Sepia}{\selectlanguage{english}to strike a blow, to give a blow} \zh{打一下}  

\lhead{\firstmark}
\rhead{\botmark}

\subsection{\hspace{-0.5cm} {\Large \textcolor{darkblue}{\textbf{\ipa{dɤ˧-qo˧}}}}\hspace{0.5cm}[\kern2pt{\textcolor{darkblue}{\textbf{\ipa{xxxx non-correspondance entre le nombre de morphèmes et le nombre de tons de morphèmes}}}}\kern2pt]} \hypertarget{d7\string_M-qo\string_M1}{}
\markboth{\textcolor{darkblue}{\textbf{\ipa{dɤ˧-qo˧}}}}{}
\textcolor{teal}{\mytextsc{adverb(ial)}} \hspace{4pt} Tone: M.
\textcolor{Sepia}{\selectlanguage{english}Way over there.} \zh{那里(远指)。} 
\lhead{\firstmark}
\rhead{\botmark}

\subsection{\hspace{-0.5cm} {\Large \textcolor{darkblue}{\textbf{\ipa{dɤ˧-tʰv̩˧-gi\#˥}}}}\hspace{0.5cm}[\kern2pt{\textcolor{darkblue}{\textbf{\ipa{xxxx non-correspondance entre le nombre de morphèmes et le nombre de tons de morphèmes}}}}\kern2pt]} \hypertarget{d7\string_M-t\string_hv\string_=\string_M-gi\#\string_T1}{}
\markboth{\textcolor{darkblue}{\textbf{\ipa{dɤ˧-tʰv̩˧-gi\#˥}}}}{}
\textcolor{teal}{\mytextsc{adverb(ial)}} \hspace{4pt} Tone: \#H.
\textcolor{Sepia}{\selectlanguage{english}Way over there.} \zh{那边(远指)。} 
\lhead{\firstmark}
\rhead{\botmark}

\subsection{\hspace{-0.5cm} {\Large \textcolor{darkblue}{\textbf{\ipa{dɤ˧-tʰv̩˧qo˧}}}}\hspace{0.5cm}[\kern2pt{\textcolor{darkblue}{\textbf{\ipa{xxxx non-correspondance entre le nombre de morphèmes et le nombre de tons de morphèmes}}}}\kern2pt]} \hypertarget{d7\string_M-t\string_hv\string_=\string_Mqo\string_M1}{}
\markboth{\textcolor{darkblue}{\textbf{\ipa{dɤ˧-tʰv̩˧qo˧}}}}{}
\textcolor{teal}{\mytextsc{adverb(ial)}} \hspace{4pt} Tone: M.
\textcolor{Sepia}{\selectlanguage{english}Way over there.} \zh{那边(远指)。} \textit{See:} \hyperlink{}{\textcolor{darkblue}{\textbf{\ipa{dɤ˧-ʈʂʰɯ˧qo˧}}}} 
\lhead{\firstmark}
\rhead{\botmark}

\subsection{\hspace{-0.5cm} {\Large \textcolor{darkblue}{\textbf{\ipa{dɤ˧-ʈʂʰɯ˧qo˧}}}}\hspace{0.5cm}[\kern2pt{\textcolor{darkblue}{\textbf{\ipa{xxxx non-correspondance entre le nombre de morphèmes et le nombre de tons de morphèmes}}}}\kern2pt]} \hypertarget{d7\string_M-t`s`\string_hM\string_Mqo\string_M1}{}
\markboth{\textcolor{darkblue}{\textbf{\ipa{dɤ˧-ʈʂʰɯ˧qo˧}}}}{}
\textcolor{teal}{\mytextsc{adverb(ial)}} \hspace{4pt} Tone: M.
\textcolor{Sepia}{\selectlanguage{english}Way over there.} \zh{那边(远指)。} \textit{See:} \hyperlink{}{\textcolor{darkblue}{\textbf{\ipa{dɤ˧-tʰv̩˧qo˧}}}} 
\lhead{\firstmark}
\rhead{\botmark}

\subsection{\hspace{-0.5cm} {\Large \textcolor{darkblue}{\textbf{\ipa{di˧mi˧}}}}\hspace{0.5cm}[\kern2pt{\textcolor{darkblue}{\textbf{\ipa{di˧mi˧}}}}\kern2pt]} \hypertarget{di\string_Mmi\string_M1}{}
\markboth{\textcolor{darkblue}{\textbf{\ipa{di˧mi˧}}}}{}
\textcolor{teal}{\mytextsc{noun}} \hspace{4pt} Tone: M.
\textcolor{Sepia}{\selectlanguage{english}Large plain.} \zh{平坝。}  ¶ \textcolor{darkblue}{\textbf{\ipa{ɬi˧di˩-di˩mi˩}}} \textcolor{Sepia}{\selectlanguage{english}the plain of Yongning} \zh{永宁坝}  
 \zh{量词}: \textcolor{darkblue}{\textbf{\ipa{di˩}}}  \mytextsc{clf}: \textcolor{darkblue}{\textbf{\ipa{di˩}}} 
\lhead{\firstmark}
\rhead{\botmark}

\subsection{\hspace{-0.5cm} {\Large \textcolor{darkblue}{\textbf{\ipa{di˧qo˧}}}}\hspace{0.5cm}[\kern2pt{\textcolor{darkblue}{\textbf{\ipa{di˧qo˧}}}}\kern2pt]} \hypertarget{di\string_Mqo\string_M1}{}
\markboth{\textcolor{darkblue}{\textbf{\ipa{di˧qo˧}}}}{}
\textcolor{teal}{\mytextsc{noun}} \hspace{4pt} Tone: M.
\textcolor{Sepia}{\selectlanguage{english}Plain.} \zh{平坝。}  \zh{量词}: \textcolor{darkblue}{\textbf{\ipa{di˩}}}  \mytextsc{clf}: \textcolor{darkblue}{\textbf{\ipa{di˩}}} 
\lhead{\firstmark}
\rhead{\botmark}

\subsection{\hspace{-0.5cm} {\Large \textcolor{darkblue}{\textbf{\ipa{di˧ɻæ˧}}}}\hspace{0.5cm}[\kern2pt{\textcolor{darkblue}{\textbf{\ipa{di˧ɻæ˧}}}}\kern2pt]} \hypertarget{di\string_Mr£`\{\string_M1}{}
\markboth{\textcolor{darkblue}{\textbf{\ipa{di˧ɻæ˧}}}}{}
\textcolor{teal}{\mytextsc{noun}} \hspace{4pt} Tone: M.
\textcolor{Sepia}{\selectlanguage{english}Plain.} \zh{平坝。}  \zh{量词}: \textcolor{darkblue}{\textbf{\ipa{di˩}}}  \mytextsc{clf}: \textcolor{darkblue}{\textbf{\ipa{di˩}}} 
\lhead{\firstmark}
\rhead{\botmark}

\subsection{\hspace{-0.5cm} {\Large \textcolor{darkblue}{\textbf{\ipa{‑di˩}}}}\hspace{0.5cm}[\kern2pt{\textcolor{darkblue}{\textbf{\ipa{di˩˥}}}}\kern2pt]} \hypertarget{‑di\string_B1}{}
\markboth{\textcolor{darkblue}{\textbf{\ipa{‑di˩}}}}{}
\textcolor{teal}{\mytextsc{suffix}} \hspace{4pt} Tone: L.
\textcolor{Sepia}{\selectlanguage{english}Nominalizer; locative or purposive.} \zh{\mytextsc{名物化}/\mytextsc{处所格}/\mytextsc{目的格。}}  ¶ \textcolor{darkblue}{\textbf{\ipa{tso˧\textasciitilde{}tso˧-tɕɯ˧-di˧˥}}} \textcolor{Sepia}{\selectlanguage{english}(piece of furniture/object) on which one can put things} \zh{可以摆东西的(家具)}  

\lhead{\firstmark}
\rhead{\botmark}

\subsection{\hspace{-0.5cm} {\Large \textcolor{darkblue}{\textbf{\ipa{di˩\textsubscript{c}}}}}\hspace{0.5cm}[\kern2pt{\textcolor{darkblue}{\textbf{\ipa{di˩˥}}}}\kern2pt]} \hypertarget{di\string_Bc1}{}
\markboth{\textcolor{darkblue}{\textbf{\ipa{di˩\textsubscript{c}}}}}{}
\textcolor{teal}{\mytextsc{classifier}} \hspace{4pt} Tone: L\textsubscript{c}.
\textcolor{Sepia}{\selectlanguage{english}Self-classifier for plains, and places.} \zh{量词:坝子、地方(一个)。}  ¶ \textcolor{darkblue}{\textbf{\ipa{ɖɯ˧-v̩˧ | ɖɯ˧-di˩ hɯ˩}}} \textcolor{Sepia}{\selectlanguage{english}to separate, each going their separate ways} \zh{分开,每个人去不同的地方}  

\lhead{\firstmark}
\rhead{\botmark}

\subsection{\hspace{-0.5cm} {\Large \textcolor{darkblue}{\textbf{\ipa{di˩\textsubscript{a}}}}}\hspace{0.5cm}[\kern2pt{\textcolor{darkblue}{\textbf{\ipa{di˩˥}}}}\kern2pt]} \hypertarget{di\string_Ba1}{}
\markboth{\textcolor{darkblue}{\textbf{\ipa{di˩\textsubscript{a}}}}}{}
\textcolor{teal}{\mytextsc{verb}} \hspace{4pt} Tone: L\textsubscript{a}.
\textcolor{Sepia}{\selectlanguage{english}Existential verb: to have (a home); to have dirt on one's clothes; to have a different in length (two objects have a difference in length).} \zh{存在动词:有,拥有。例如:有家,有污垢在衣服上,有长短区别(两个物品有长短区别)。}  ¶ \textcolor{darkblue}{\textbf{\ipa{ʈʰɯ˧ | ɑ˩ʁo˧ mɤ˧-di˩-hĩ˩.}}} \textcolor{Sepia}{\selectlanguage{english}(S)he does not have a home. / (S)he is homeless.} \zh{他没有家。}  
 ¶ \textcolor{darkblue}{\textbf{\ipa{mɤ˧ tʰi˧-di˩}}} \textcolor{Sepia}{\selectlanguage{english}there is grease (eg there is grease around the mouth of someone who has been biting away at large slabs of meat)} \zh{有油(例如:一个人的嘴巴周围油乎乎,有油)}  
 ¶ \textcolor{darkblue}{\textbf{\ipa{ɖɯ˧-kʰwɤ˧ di˥}}} \textcolor{Sepia}{\selectlanguage{english}there is something} \zh{有一块东西}  

\lhead{\firstmark}
\rhead{\botmark}

\subsection{\hspace{-0.5cm} {\Large \textcolor{darkblue}{\textbf{\ipa{di˩-gɤ˩lɑ˥}}}}\hspace{0.5cm}[\kern2pt{\textcolor{darkblue}{\textbf{\ipa{xxxx non-correspondance entre le nombre de morphèmes et le nombre de tons de morphèmes}}}}\kern2pt]} \hypertarget{di\string_B-g7\string_BlA\string_T1}{}
\markboth{\textcolor{darkblue}{\textbf{\ipa{di˩-gɤ˩lɑ˥}}}}{}
\textcolor{teal}{\mytextsc{noun}} \hspace{4pt} Tone: L+H\#.
\textcolor{Sepia}{\selectlanguage{english}Earth spirit.} \zh{地菩萨。}  \zh{量词}: \textcolor{darkblue}{\textbf{\ipa{v̩˧}}}  \mytextsc{clf}: \textcolor{darkblue}{\textbf{\ipa{v̩˧}}} 
\lhead{\firstmark}
\rhead{\botmark}

\subsection{\hspace{-0.5cm} {\Large \textcolor{darkblue}{\textbf{\ipa{di˩li˩}}}}\hspace{0.5cm}[\kern2pt{\textcolor{darkblue}{\textbf{\ipa{di˩li˩˥}}}}\kern2pt]} \hypertarget{di\string_Bli\string_B1}{}
\markboth{\textcolor{darkblue}{\textbf{\ipa{di˩li˩}}}}{}
\textcolor{teal}{\mytextsc{noun}} \hspace{4pt} Tone: L.
\textcolor{Sepia}{\selectlanguage{english}Dandy: \textit{Ottelia Acuminata, Boottia acuminata, Ottelia yunnanensis}.} \zh{海菜花。} Local Chinese dialect:\zh{龙爪菜。} ¶ \textcolor{darkblue}{\textbf{\ipa{di˩li˩-ʁo˩bv̩˥ (ton: L+H\#)}}} \textcolor{Sepia}{\selectlanguage{english}dandy shoots} \zh{海菜花的萌芽}  
 ¶ \textcolor{darkblue}{\textbf{\ipa{di˩li˩-ʁo˩bv̩˥ hwæ˩}}} \textcolor{Sepia}{\selectlanguage{english}to buy dandy shoots} \zh{买海菜花萌芽}  
 ¶ \textcolor{darkblue}{\textbf{\ipa{di˩li˩-ʁo˩bv̩˥ tɕʰi˩}}} \textcolor{Sepia}{\selectlanguage{english}to sell dandy shoots} \zh{卖海菜花萌芽}  
 ¶ \textcolor{darkblue}{\textbf{\ipa{di˩li˩-ʁo˩bv̩˥ dzɯ˩}}} \textcolor{Sepia}{\selectlanguage{english}to eat dandy shoots} \zh{吃海菜花萌芽}  
 ¶ \textcolor{darkblue}{\textbf{\ipa{di˩li˩-ʁo˩bv̩˥ dze˩}}} \textcolor{Sepia}{\selectlanguage{english}to cut dandy shoots} \zh{割海菜花萌芽}  
 ¶ \textcolor{darkblue}{\textbf{\ipa{di˩li˩-ʁo˩bv̩˥ tɕɤ˩}}} \textcolor{Sepia}{\selectlanguage{english}to boil dandy shoots} \zh{煮海菜花萌芽}  
 \zh{量词}: \textcolor{darkblue}{\textbf{\ipa{qɑ˩}}}  \mytextsc{clf}: \textcolor{darkblue}{\textbf{\ipa{qɑ˩}}} 
\lhead{\firstmark}
\rhead{\botmark}

\subsection{\hspace{-0.5cm} {\Large \textcolor{darkblue}{\textbf{\ipa{di˧˥}}} \textsubscript{1}}\hspace{0.5cm}[\kern2pt{\textcolor{darkblue}{\textbf{\ipa{di˧˥}}}}\kern2pt]} \hypertarget{di\string_M\string_T1}{}
\markboth{\textcolor{darkblue}{\textbf{\ipa{di˧˥}}} \textsubscript{1}}{}
\textcolor{teal}{\mytextsc{verb}} \hspace{4pt} Tone: MH.
\textcolor{Sepia}{\selectlanguage{english}To hunt; to scatter, to drive out, to drive away.} \zh{打散,驱赶,撵,赶,打猎。}  ¶ \textcolor{darkblue}{\textbf{\ipa{tɕʰɯ˩di˩˥}}} \textcolor{Sepia}{\selectlanguage{english}to hunt the muntjac; to hunt} \zh{赶麂子,狩猎}  
 ¶ \textcolor{darkblue}{\textbf{\ipa{tɕʰɯ˩di˩-bi˩-ni˩gv̩˩˥}}} \textcolor{Sepia}{\selectlanguage{english}to have a habit of hunting, to have a fondness for hunting} \zh{有打猎的习惯、喜欢打猎}  
 ¶ \textcolor{darkblue}{\textbf{\ipa{di˩\textasciitilde{}di˧˥ / di˩\textasciitilde{}di˧-ze˥}}} \textcolor{Sepia}{\selectlanguage{english}\mytextsc{red}: to hunt, to track} \zh{\mytextsc{重叠:跟着、追着}}  
 ¶ \textcolor{darkblue}{\textbf{\ipa{tʰi˧-di˩\textasciitilde{}di˩}}} \textcolor{Sepia}{\selectlanguage{english}\mytextsc{dur} \mytextsc{red}} \zh{\mytextsc{dur} \mytextsc{red}}  

\lhead{\firstmark}
\rhead{\botmark}

\subsection{\hspace{-0.5cm} {\Large \textcolor{darkblue}{\textbf{\ipa{di˧˥}}} \textsubscript{2}}\hspace{0.5cm}[\kern2pt{\textcolor{darkblue}{\textbf{\ipa{di˧˥}}}}\kern2pt]} \hypertarget{di\string_M\string_T2}{}
\markboth{\textcolor{darkblue}{\textbf{\ipa{di˧˥}}} \textsubscript{2}}{}
\textcolor{teal}{\mytextsc{verb}} \hspace{4pt} Tone: MH.
\textcolor{Sepia}{\selectlanguage{english}To run; to have a runny belly = to have diarrhea.} \zh{拉(肚子)。}  ¶ \textcolor{darkblue}{\textbf{\ipa{bi˧mi˧ di˧˥}}} \textcolor{Sepia}{\selectlanguage{english}to have diarrhea} \zh{拉肚子}  

\lhead{\firstmark}
\rhead{\botmark}

\subsection{\hspace{-0.5cm} {\Large \textcolor{darkblue}{\textbf{\ipa{di˩˥}}}}\hspace{0.5cm}[\kern2pt{\textcolor{darkblue}{\textbf{\ipa{di˩˥}}}}\kern2pt]} \hypertarget{di\string_B\string_T1}{}
\markboth{\textcolor{darkblue}{\textbf{\ipa{di˩˥}}}}{}
\textcolor{teal}{\mytextsc{noun}} \hspace{4pt} Tone: LH.
\textcolor{Sepia}{\selectlanguage{english}Earth (as in: the sky and the earth).} \zh{地(天地的地)。}  ¶ \textcolor{darkblue}{\textbf{\ipa{di˩ dv̩˩-ze˥}}} \textcolor{Sepia}{\selectlanguage{english}to dig the earth} \zh{挖土}  
 ¶ \textcolor{darkblue}{\textbf{\ipa{di˩ hwæ˧-ze˩}}} \textcolor{Sepia}{\selectlanguage{english}bought some earth} \zh{买了土}  
 ¶ \textcolor{darkblue}{\textbf{\ipa{ɖɯ˧-di˩ ɖɯ˩-bæ˩!}}} \textcolor{Sepia}{\selectlanguage{english}Each place is different! (In particular, each place has its own pronunciation: its own dialect of the Na language)} \zh{一个地方,一个样! = 每个地方有自己的特色(如:每个村落有自己的摩梭方言/土语)}  
 \zh{量词}: \textcolor{darkblue}{\textbf{\ipa{di˩}}}  \mytextsc{clf}: \textcolor{darkblue}{\textbf{\ipa{di˩}}} 
\lhead{\firstmark}
\rhead{\botmark}

\subsection{\hspace{-0.5cm} {\Large \textcolor{darkblue}{\textbf{\ipa{do˥}}} \textsubscript{1}}\hspace{0.5cm}[\kern2pt{\textcolor{darkblue}{\textbf{\ipa{do˥}}}}\kern2pt]} \hypertarget{do\string_T1}{}
\markboth{\textcolor{darkblue}{\textbf{\ipa{do˥}}} \textsubscript{1}}{}
\textcolor{teal}{\mytextsc{verb}} \hspace{4pt} Tone: H.
\textcolor{Sepia}{\selectlanguage{english}To climb.} \zh{爬,上去,上山。}  ¶ \textcolor{darkblue}{\textbf{\ipa{ʈʂo˩bo˩ do˩˥}}} \textcolor{Sepia}{\selectlanguage{english}to climb a wall} \zh{爬墙}  
 ¶ \textcolor{darkblue}{\textbf{\ipa{gɤ˩-do˧}}} \textcolor{Sepia}{\selectlanguage{english}to escalate, to climb up} \zh{爬上}  
 ¶ \textcolor{darkblue}{\textbf{\ipa{ʁwɤ˩ do˩˥}}} \textcolor{Sepia}{\selectlanguage{english}to climb a mountain, to go hiking in the mountains} \zh{爬山}  
 ¶ \textcolor{darkblue}{\textbf{\ipa{to˩ do˩˥}}} \textcolor{Sepia}{\selectlanguage{english}to ascend a slope, to climb a slope} \zh{爬上一个山坡}  

\lhead{\firstmark}
\rhead{\botmark}

\subsection{\hspace{-0.5cm} {\Large \textcolor{darkblue}{\textbf{\ipa{do˥}}} \textsubscript{2}}\hspace{0.5cm}[\kern2pt{\textcolor{darkblue}{\textbf{\ipa{do˥}}}}\kern2pt]} \hypertarget{do\string_T2}{}
\markboth{\textcolor{darkblue}{\textbf{\ipa{do˥}}} \textsubscript{2}}{}
\textcolor{teal}{\mytextsc{verb}} \hspace{4pt} Tone: H.
\textcolor{Sepia}{\selectlanguage{english}To mate with; to pair (of animal).} \zh{交配、交尾。}  ¶ \textcolor{darkblue}{\textbf{\ipa{bo˩ɬɑ˥ | bo˩mi˧ do˧}}} \textcolor{Sepia}{\selectlanguage{english}The pig mates with the sow.} \zh{公猪与母猪交配。}  

\lhead{\firstmark}
\rhead{\botmark}

\subsection{\hspace{-0.5cm} {\Large \textcolor{darkblue}{\textbf{\ipa{do˥\textsubscript{a}}}}}\hspace{0.5cm}[\kern2pt{\textcolor{darkblue}{\textbf{\ipa{do˥}}}}\kern2pt]} \hypertarget{do\string_Ta1}{}
\markboth{\textcolor{darkblue}{\textbf{\ipa{do˥\textsubscript{a}}}}}{}
\textcolor{teal}{\mytextsc{classifier}} \hspace{4pt} Tone: H\textsubscript{a}.
\textcolor{Sepia}{\selectlanguage{english}Classifier for partitions/walls.} \zh{\mytextsc{量词}.墙壁(一堵)。}  ¶ \textcolor{darkblue}{\textbf{\ipa{ʈʂʰɯ˧-do\#˥}}} \textcolor{Sepia}{\selectlanguage{english}this partition/wall} \zh{这堵(墙壁)}  

\lhead{\firstmark}
\rhead{\botmark}

\subsection{\hspace{-0.5cm} {\Large \textcolor{darkblue}{\textbf{\ipa{do˧}}} \textsubscript{1}}\hspace{0.5cm}[\kern2pt{\textcolor{darkblue}{\textbf{\ipa{do˥}}}}\kern2pt]} \hypertarget{do\string_M1}{}
\markboth{\textcolor{darkblue}{\textbf{\ipa{do˧}}} \textsubscript{1}}{}
\textcolor{teal}{\mytextsc{adjective}} \hspace{4pt} Tone: M.
\textcolor{Sepia}{\selectlanguage{english}Stupid, silly, idiotic.} \zh{笨、愚蠢。}  ¶ \textcolor{darkblue}{\textbf{\ipa{zo˩ do˩˥}}} \textcolor{Sepia}{\selectlanguage{english}idiot, village idiot} \zh{傻瓜}  

\lhead{\firstmark}
\rhead{\botmark}

\subsection{\hspace{-0.5cm} {\Large \textcolor{darkblue}{\textbf{\ipa{do˧}}} \textsubscript{2}}\hspace{0.5cm}[\kern2pt{\textcolor{darkblue}{\textbf{\ipa{do˥}}}}\kern2pt]} \hypertarget{do\string_M2}{}
\markboth{\textcolor{darkblue}{\textbf{\ipa{do˧}}} \textsubscript{2}}{}
\textcolor{teal}{\mytextsc{adjective}} \hspace{4pt} Tone: M.
\textcolor{Sepia}{\selectlanguage{english}Sterile.} \zh{不能生育。} 
\lhead{\firstmark}
\rhead{\botmark}

\subsection{\hspace{-0.5cm} {\Large \textcolor{darkblue}{\textbf{\ipa{do˧bæ˧}}}}\hspace{0.5cm}[\kern2pt{\textcolor{darkblue}{\textbf{\ipa{do˧bæ˧}}}}\kern2pt]} \hypertarget{do\string_Mb\{\string_M1}{}
\markboth{\textcolor{darkblue}{\textbf{\ipa{do˧bæ˧}}}}{}
\textcolor{teal}{\mytextsc{noun}} \hspace{4pt} Tone: M.
\textcolor{Sepia}{\selectlanguage{english}Thigh.} \zh{大腿。}  ¶ \textcolor{darkblue}{\textbf{\ipa{do˧bæ˧ | ɖɯ˩-hĩ˩˥}}} \textcolor{Sepia}{\selectlanguage{english}thigh} \zh{大腿}  
 ¶ \textcolor{darkblue}{\textbf{\ipa{do˧bæ˧ | tɕi˩-hĩ˩˥}}} \textcolor{Sepia}{\selectlanguage{english}calf} \zh{小腿}  
 \zh{量词}: \textcolor{darkblue}{\textbf{\ipa{ʈv̩˩}}}  \mytextsc{clf}: \textcolor{darkblue}{\textbf{\ipa{ʈv̩˩}}} 
\lhead{\firstmark}
\rhead{\botmark}

\subsection{\hspace{-0.5cm} {\Large \textcolor{darkblue}{\textbf{\ipa{do˧bv̩˧}}}}\hspace{0.5cm}[\kern2pt{\textcolor{darkblue}{\textbf{\ipa{do˧bv̩˧}}}}\kern2pt]} \hypertarget{do\string_Mbv\string_=\string_M1}{}
\markboth{\textcolor{darkblue}{\textbf{\ipa{do˧bv̩˧}}}}{}
\textcolor{teal}{\mytextsc{noun}} \hspace{4pt} Tone: M.
\textcolor{Sepia}{\selectlanguage{english}Buttocks.} \zh{屁股。}  \zh{量词}: \textcolor{darkblue}{\textbf{\ipa{ɭɯ˧}}}  \mytextsc{clf}: \textcolor{darkblue}{\textbf{\ipa{ɭɯ˧}}} 
\lhead{\firstmark}
\rhead{\botmark}

\subsection{\hspace{-0.5cm} {\Large \textcolor{darkblue}{\textbf{\ipa{do˩}}}}\hspace{0.5cm}[\kern2pt{\textcolor{darkblue}{\textbf{\ipa{do˩˥}}}}\kern2pt]} \hypertarget{do\string_B1}{}
\markboth{\textcolor{darkblue}{\textbf{\ipa{do˩}}}}{}
\textcolor{teal}{\mytextsc{adjective}} \hspace{4pt} Tone: L.
\textit{\textcolor{Sepia}{\selectlanguage{english}archaic}} [\zh{古语}] \textcolor{Sepia}{\selectlanguage{english}Immature, lacking maturity.} \zh{不成熟、晚熟。}  ¶ \textcolor{darkblue}{\textbf{\ipa{ŋwɤ˩ɬi˩-mi˩˥, | ʂe˧ mɤ˧-mv̩˥, | ʂe˧ do˧˥! | tsʰe˩ŋwɤ˩ kʰv̩˥, | zo˧ mɤ˧-ti˩, | zo˧ do˧˥!}}} \textcolor{Sepia}{\selectlanguage{english}In the fifth month, if cereals are still green (=if they do not yet yield grain), the crop is immature (and may not yield any harvest). At age 15, if a boy does not become an adult (=if a boy does not visit girls), he is immature (he is not developing normally)!} \zh{五月份,谷物还是小草(还不出谷粒),算是晚熟!男人十五岁还不成熟(=还不见姑娘),算是晚熟!}  

\lhead{\firstmark}
\rhead{\botmark}

\subsection{\hspace{-0.5cm} {\Large \textcolor{darkblue}{\textbf{\ipa{do˩\textsubscript{b}}}}}\hspace{0.5cm}[\kern2pt{\textcolor{darkblue}{\textbf{\ipa{do˩˥}}}}\kern2pt]} \hypertarget{do\string_Bb1}{}
\markboth{\textcolor{darkblue}{\textbf{\ipa{do˩\textsubscript{b}}}}}{}
\textcolor{teal}{\mytextsc{verb}} \hspace{4pt} Tone: L\textsubscript{b}.
\textcolor{Sepia}{\selectlanguage{english}To see; to come across someone.} \zh{看见,遇见,见。}  ¶ \textcolor{darkblue}{\textbf{\ipa{ɖɯ˧-do˥\textasciitilde{}do˩-ɻ̍˩}}} \textcolor{Sepia}{\selectlanguage{english}\mytextsc{delimitative} \string_ \mytextsc{red} \mytextsc{inceptive}} \zh{见一见}  
 ¶ \textcolor{darkblue}{\textbf{\ipa{ɖɯ˧-kʰwɤ˧ do˧˥}}} \textcolor{Sepia}{\selectlanguage{english}to see a piece} \zh{看见一块(东西)}  
 ¶ \textcolor{darkblue}{\textbf{\ipa{tso˧\textasciitilde{}tso˧ do˧˥}}} \textcolor{Sepia}{\selectlanguage{english}to see things, to see something} \zh{看见东西}  
 ¶ \textcolor{darkblue}{\textbf{\ipa{do˩-mɤ˩-ho˥}}} \textcolor{Sepia}{\selectlanguage{english}\string_ \mytextsc{neg} \mytextsc{desiderative}} \zh{不许(看)见}  
 ¶ \textcolor{darkblue}{\textbf{\ipa{bo˩mi˧ do˩ (+ze˩)}}} \textcolor{Sepia}{\selectlanguage{english}...has seen (a/the) sow} \zh{看见了母猪}  

\lhead{\firstmark}
\rhead{\botmark}

\subsection{\hspace{-0.5cm} {\Large \textcolor{darkblue}{\textbf{\ipa{do˩bv̩\#˥}}}}\hspace{0.5cm}[\kern2pt{\textcolor{darkblue}{\textbf{\ipa{do˩bv̩˥}}}}\kern2pt]} \hypertarget{do\string_Bbv\string_=\#\string_T1}{}
\markboth{\textcolor{darkblue}{\textbf{\ipa{do˩bv̩\#˥}}}}{}
\textcolor{teal}{\mytextsc{noun}} \hspace{4pt} Tone: LM+\#H.
\textcolor{Sepia}{\selectlanguage{english}Mani wall, Mani pile: pile built of rubble and sand, with carved stone tablets, most with the inscription Om Mani Padme Hum. A Mani wall should be passed or circumvented from the left side, the clockwise direction in which the universe revolves, according to Buddhist doctrine.} \zh{嘛呢堆。}  \zh{量词}: \textcolor{darkblue}{\textbf{\ipa{ɭɯ˧}}}  \mytextsc{clf}: \textcolor{darkblue}{\textbf{\ipa{ɭɯ˧}}} \textit{See:} \textcolor{darkblue}{\textbf{\ipa{mɑ˩ɳɯ˧-do˥bv˩, mɑ˩ɳɯ\#˥}}} 
\lhead{\firstmark}
\rhead{\botmark}

\subsection{\hspace{-0.5cm} {\Large \textcolor{darkblue}{\textbf{\ipa{do˩kv̩\#˥}}}}\hspace{0.5cm}[\kern2pt{\textcolor{darkblue}{\textbf{\ipa{do˩kv̩˥}}}}\kern2pt]} \hypertarget{do\string_Bkv\string_=\#\string_T1}{}
\markboth{\textcolor{darkblue}{\textbf{\ipa{do˩kv̩\#˥}}}}{}
\textcolor{teal}{\mytextsc{noun}} \hspace{4pt} Tone: LM+\#H.
\textcolor{Sepia}{\selectlanguage{english}Small beams upholding the ceiling of the ground floor.} \zh{小梁子,作为楼上(第二层)木地板的底。}  \zh{量词}: \textcolor{darkblue}{\textbf{\ipa{ɭɯ˧}}}  \mytextsc{clf}: \textcolor{darkblue}{\textbf{\ipa{ɭɯ˧}}} 
\lhead{\firstmark}
\rhead{\botmark}

\subsection{\hspace{-0.5cm} {\Large \textcolor{darkblue}{\textbf{\ipa{dv̩˩}}} \textsubscript{1}}\hspace{0.5cm}[\kern2pt{\textcolor{darkblue}{\textbf{\ipa{dv̩˩˥}}}}\kern2pt]} \hypertarget{dv\string_=\string_B1}{}
\markboth{\textcolor{darkblue}{\textbf{\ipa{dv̩˩}}} \textsubscript{1}}{}
\textcolor{teal}{\mytextsc{classifier}} \hspace{4pt} Tone: L *.
\textcolor{Sepia}{\selectlanguage{english}Classifier for flocks of cattle; only used in the singular.} \zh{量词:人、牲畜(一群、一队)。} \textit{See:} \hyperlink{}{\textcolor{darkblue}{\textbf{\ipa{dɤ˧-tʰv̩˧-gi\#˥}}}} 
\lhead{\firstmark}
\rhead{\botmark}

\subsection{\hspace{-0.5cm} {\Large \textcolor{darkblue}{\textbf{\ipa{dv̩˩}}} \textsubscript{2}}\hspace{0.5cm}[\kern2pt{\textcolor{darkblue}{\textbf{\ipa{dv̩˩˥}}}}\kern2pt]} \hypertarget{dv\string_=\string_B2}{}
\markboth{\textcolor{darkblue}{\textbf{\ipa{dv̩˩}}} \textsubscript{2}}{}
\textcolor{teal}{\mytextsc{pronoun/pronominal}} \hspace{4pt} Tone: L?.
\textcolor{Sepia}{\selectlanguage{english}Distal demonstrative, appearing in the phrase “this way, in this direction”.} \zh{指示代词:那边(远指),从‘那个方向’这个短语提取出来的。}  ¶ \textcolor{darkblue}{\textbf{\ipa{dv̩˩-tɕo˧}}} \textcolor{Sepia}{\selectlanguage{english}that way} \zh{那个方向}  
 ¶ \textcolor{darkblue}{\textbf{\ipa{dv̩˩tɕo˧ fæ˧}}} \textcolor{Sepia}{\selectlanguage{english}that way} \zh{那个方向}  
\textit{See:} \hyperlink{}{\textcolor{darkblue}{\textbf{\ipa{dɤ˧-tʰv̩˧-gi\#˥}}}} 
\lhead{\firstmark}
\rhead{\botmark}

\subsection{\hspace{-0.5cm} {\Large \textcolor{darkblue}{\textbf{\ipa{dv̩˩˧}}} \textsubscript{1}}\hspace{0.5cm}[\kern2pt{\textcolor{darkblue}{\textbf{\ipa{dv̩˩˥}}}}\kern2pt]} \hypertarget{dv\string_=\string_B\string_M1}{}
\markboth{\textcolor{darkblue}{\textbf{\ipa{dv̩˩˧}}} \textsubscript{1}}{}
\textcolor{teal}{\mytextsc{noun}} \hspace{4pt} Tone: LM.
\textcolor{Sepia}{\selectlanguage{english}Weasel.} \zh{黄鼠狼,黄喉貂。}  ¶ \textcolor{darkblue}{\textbf{\ipa{dv̩˩ hwæ˧-ze˧}}} \textcolor{Sepia}{\selectlanguage{english}...bought (a) weasel} \zh{买了黄鼠狼}  
 ¶ \textcolor{darkblue}{\textbf{\ipa{dv̩˩ dzɯ˧-ze˩}}} \textcolor{Sepia}{\selectlanguage{english}...ate a weasel} \zh{吃了黄鼠狼}  
 \zh{量词}: \textcolor{darkblue}{\textbf{\ipa{mi˩}}}  \mytextsc{clf}: \textcolor{darkblue}{\textbf{\ipa{mi˩}}} 
\lhead{\firstmark}
\rhead{\botmark}

\subsection{\hspace{-0.5cm} {\Large \textcolor{darkblue}{\textbf{\ipa{dv̩˩˧}}} \textsubscript{2}}\hspace{0.5cm}[\kern2pt{\textcolor{darkblue}{\textbf{\ipa{dv̩˩˥}}}}\kern2pt]} \hypertarget{dv\string_=\string_B\string_M2}{}
\markboth{\textcolor{darkblue}{\textbf{\ipa{dv̩˩˧}}} \textsubscript{2}}{}
\textcolor{teal}{\mytextsc{noun}} \hspace{4pt} Tone: LM.
\textcolor{Sepia}{\selectlanguage{english}Poison.} \zh{毒。} \textit{See:} \hyperlink{}{\textcolor{darkblue}{\textbf{\ipa{dv̩˩\textsubscript{a}}}}} 
\lhead{\firstmark}
\rhead{\botmark}

\subsection{\hspace{-0.5cm} {\Large \textcolor{darkblue}{\textbf{\ipa{dv̩˥}}}}\hspace{0.5cm}[\kern2pt{\textcolor{darkblue}{\textbf{\ipa{dv̩˥}}}}\kern2pt]} \hypertarget{dv\string_=\string_T1}{}
\markboth{\textcolor{darkblue}{\textbf{\ipa{dv̩˥}}}}{}
\textcolor{teal}{\mytextsc{verb}} \hspace{4pt} Tone: H.
\textcolor{Sepia}{\selectlanguage{english}To dig.} \zh{挖。}  ¶ \textcolor{darkblue}{\textbf{\ipa{ʈʂe˧ dv̩˧(-ze˩)}}} \textcolor{Sepia}{\selectlanguage{english}to dig the earth} \zh{挖土}  

\lhead{\firstmark}
\rhead{\botmark}

\subsection{\hspace{-0.5cm} {\Large \textcolor{darkblue}{\textbf{\ipa{dv̩˩\textsubscript{a}}}}}\hspace{0.5cm}[\kern2pt{\textcolor{darkblue}{\textbf{\ipa{dv̩˩˥}}}}\kern2pt]} \hypertarget{dv\string_=\string_Ba1}{}
\markboth{\textcolor{darkblue}{\textbf{\ipa{dv̩˩\textsubscript{a}}}}}{}
\textcolor{teal}{\mytextsc{verb}} \hspace{4pt} Tone: L\textsubscript{a}.
\ding{202} \textcolor{Sepia}{\selectlanguage{english}To poison.} \zh{毒害、毒化。}  ¶ \textcolor{darkblue}{\textbf{\ipa{ʈʂʰɯ˧, | hĩ˧ dv̩˥-mɤ˩-kv̩˩! | ʈʂʰɯ˧, | hĩ˧ dv̩˥-kv̩˩!}}} \textcolor{Sepia}{\selectlanguage{english}This one is not poisonous / is edible (literally “this one does not poison people”)! That one [on the other hand] is poisonous / is not edible! (About different sorts of mushrooms.)} \zh{这个,不会让人中毒!那个(反倒)会让人中毒!(情景:谈不同菌子种类。)}  
 ¶ \textcolor{darkblue}{\textbf{\ipa{ʈʂʰɯ˧, | dv̩˩-mɤ˩-kv̩˥!}}} \textcolor{Sepia}{\selectlanguage{english}This one is not poisonous / is edible (literally “this one does not poison people”)! (About a mushroom species.)} \zh{这个,不会让人中毒!(情景:谈不同菌子种类。)}  
\ding{203} \textcolor{Sepia}{\selectlanguage{english}To hate, to detest.} \zh{讨厌、恨。}  ¶ \textcolor{darkblue}{\textbf{\ipa{le˧-dv̩˩-ze˩}}}  
 ¶ \textcolor{darkblue}{\textbf{\ipa{njɤ˧ | ʈʂʰɯ˧ dv̩˥ | ʐwæ˩˥!}}} \textcolor{Sepia}{\selectlanguage{english}I hate him/her!} \zh{我很讨厌他!}  
 ¶ \textcolor{darkblue}{\textbf{\ipa{dv̩˩-zo˧-mɤ˧-tʰɑ˧˥}}} \textcolor{Sepia}{\selectlanguage{english}to hate deeply} \zh{讨厌得不行}  
\textit{See:} \hyperlink{}{\textcolor{darkblue}{\textbf{\ipa{dv̩˩˧}}} \textsubscript{2}} 
\lhead{\firstmark}
\rhead{\botmark}

\subsection{\hspace{-0.5cm} {\Large \textcolor{darkblue}{\textbf{\ipa{dv̩˩\textsubscript{b}}}}}\hspace{0.5cm}[\kern2pt{\textcolor{darkblue}{\textbf{\ipa{dv̩˩˥}}}}\kern2pt]} \hypertarget{dv\string_=\string_Bb1}{}
\markboth{\textcolor{darkblue}{\textbf{\ipa{dv̩˩\textsubscript{b}}}}}{}
\textcolor{teal}{\mytextsc{classifier}} \hspace{4pt} Tone: L\textsubscript{b}.
\textcolor{Sepia}{\selectlanguage{english}Classifier for small groups of people: 3 or more.} \zh{量词:人(一些)。}  ¶ \textcolor{darkblue}{\textbf{\ipa{hĩ˧ ɖɯ˧-dv̩˩}}} \textcolor{Sepia}{\selectlanguage{english}a few people, a group of people} \zh{一些人}  
 ¶ \textcolor{darkblue}{\textbf{\ipa{hĩ˧ ʈʂʰɯ˧-dv̩˥}}} \textcolor{Sepia}{\selectlanguage{english}these people, this group of people} \zh{这些人}  

\lhead{\firstmark}
\rhead{\botmark}

\subsection{\hspace{-0.5cm} {\Large \textcolor{darkblue}{\textbf{\ipa{dv̩˩bi˩}}}}\hspace{0.5cm}[\kern2pt{\textcolor{darkblue}{\textbf{\ipa{dv̩˩bi˩˥}}}}\kern2pt]} \hypertarget{dv\string_=\string_Bbi\string_B1}{}
\markboth{\textcolor{darkblue}{\textbf{\ipa{dv̩˩bi˩}}}}{}
\textcolor{teal}{\mytextsc{adverb(ial)}} \hspace{4pt} Tone: L.
\textcolor{Sepia}{\selectlanguage{english}Opposite.} \zh{对面。} 
\lhead{\firstmark}
\rhead{\botmark}

\subsection{\hspace{-0.5cm} {\Large \textcolor{darkblue}{\textbf{\ipa{dv̩˩mi\#˥}}}}\hspace{0.5cm}[\kern2pt{\textcolor{darkblue}{\textbf{\ipa{dv̩˩mi˩˥}}}}\kern2pt]} \hypertarget{dv\string_=\string_Bmi\#\string_T1}{}
\markboth{\textcolor{darkblue}{\textbf{\ipa{dv̩˩mi\#˥}}}}{}
\textcolor{teal}{\mytextsc{noun}} \hspace{4pt} Tone: LM+\#H.
\textcolor{Sepia}{\selectlanguage{english}Female weasel.} \zh{母黄鼠狼。}  ¶ \textcolor{darkblue}{\textbf{\ipa{dv̩˩mi˧-dv̩˥pʰv̩˩}}} \textcolor{Sepia}{\selectlanguage{english}female weasel and male weasel} \zh{母黄鼠狼与公黄鼠狼}  
 \zh{量词}: \textcolor{darkblue}{\textbf{\ipa{mi˩}}}  \mytextsc{clf}: \textcolor{darkblue}{\textbf{\ipa{mi˩}}} 
\lhead{\firstmark}
\rhead{\botmark}

\subsection{\hspace{-0.5cm} {\Large \textcolor{darkblue}{\textbf{\ipa{dv̩˩pʰæ˧}}}}\hspace{0.5cm}[\kern2pt{\textcolor{darkblue}{\textbf{\ipa{dv̩˩pʰæ˥}}}}\kern2pt]} \hypertarget{dv\string_=\string_Bp\string_h\{\string_M1}{}
\markboth{\textcolor{darkblue}{\textbf{\ipa{dv̩˩pʰæ˧}}}}{}
\textcolor{teal}{\mytextsc{noun}} \hspace{4pt} Tone: LM.
\textcolor{Sepia}{\selectlanguage{english}The room in the main building of the farm where cereals were kept: the granary.} \zh{仓廪。摩梭话音译:‘独帕’。}  \zh{量词}: \textcolor{darkblue}{\textbf{\ipa{ɭɯ˧}}}  \mytextsc{clf}: \textcolor{darkblue}{\textbf{\ipa{ɭɯ˧}}} 
\lhead{\firstmark}
\rhead{\botmark}

\subsection{\hspace{-0.5cm} {\Large \textcolor{darkblue}{\textbf{\ipa{dv̩˩pʰv̩\#˥}}}}\hspace{0.5cm}[\kern2pt{\textcolor{darkblue}{\textbf{\ipa{dv̩˩pʰv̩˥}}}}\kern2pt]} \hypertarget{dv\string_=\string_Bp\string_hv\string_=\#\string_T1}{}
\markboth{\textcolor{darkblue}{\textbf{\ipa{dv̩˩pʰv̩\#˥}}}}{}
\textcolor{teal}{\mytextsc{noun}} \hspace{4pt} Tone: LM+\#H / LM.
\textcolor{Sepia}{\selectlanguage{english}Male weasel.} \zh{公黄鼠狼。}  \zh{量词}: \textcolor{darkblue}{\textbf{\ipa{mi˩}}}  \mytextsc{clf}: \textcolor{darkblue}{\textbf{\ipa{mi˩}}} 
\lhead{\firstmark}
\rhead{\botmark}

\subsection{\hspace{-0.5cm} {\Large \textcolor{darkblue}{\textbf{\ipa{dv̩˩zo\#˥}}}}\hspace{0.5cm}[\kern2pt{\textcolor{darkblue}{\textbf{\ipa{dv̩˧zo˧}}}}\kern2pt]} \hypertarget{dv\string_=\string_Bzo\#\string_T1}{}
\markboth{\textcolor{darkblue}{\textbf{\ipa{dv̩˩zo\#˥}}}}{}
\textcolor{teal}{\mytextsc{noun}} \hspace{4pt} Tone: LM+\#H / LM.
\textcolor{Sepia}{\selectlanguage{english}Baby weasel.} \zh{黄鼠狼的崽子。} 
\lhead{\firstmark}
\rhead{\botmark}

\newpage
\section*{\centering- \textcolor{darkblue}{\textbf{\ipa{dz}}} -}
\subsection{\hspace{-0.5cm} {\Large \textcolor{darkblue}{\textbf{\ipa{dzɑ˥}}}}\hspace{0.5cm}[\kern2pt{\textcolor{darkblue}{\textbf{\ipa{dzɑ˥}}}}\kern2pt]} \hypertarget{dzA\string_T1}{}
\markboth{\textcolor{darkblue}{\textbf{\ipa{dzɑ˥}}}}{}
\textcolor{teal}{\mytextsc{adjective}} \hspace{4pt} Tone: H.
\ding{202} \textcolor{Sepia}{\selectlanguage{english}Bad, mean (action), inferior.} \zh{坏、差、下级(行为……)。}  ¶ \textcolor{darkblue}{\textbf{\ipa{ʈʂʰɯ˧-ɳɯ˧ | njɤ˧-ki˧ | dzɑ˧-ʝi˧ | ʐwæ˩˥!}}} \textcolor{Sepia}{\selectlanguage{english}He really despises me!} \zh{他很瞧不起我!}  
 ¶ \textcolor{darkblue}{\textbf{\ipa{hĩ˧ ʈʂʰɯ˧-v̩˧ dʑo˩, | õ˧-ki˥ | dzɑ˧-ʝi˧-ze˩!}}} \textcolor{Sepia}{\selectlanguage{english}This person has no self-respect! (literally: This person is doing herself harm)} \zh{这个人,不尊重自己!}  
 ¶ \textcolor{darkblue}{\textbf{\ipa{mv̩˧ dzɑ˧.}}} \textcolor{Sepia}{\selectlanguage{english}The weather is bad.} \zh{天气很坏。}  
 ¶ \textcolor{darkblue}{\textbf{\ipa{mv̩˧ dzɑ˧-ze˩}}} \textcolor{Sepia}{\selectlanguage{english}The weather is getting bad.} \zh{天气变坏了。}  
 ¶ \textcolor{darkblue}{\textbf{\ipa{lo˧ dzɑ˧}}} \textcolor{Sepia}{\selectlanguage{english}poor (work), bad (job: e.g. someone has done a bad job)} \zh{(工作)差}  
\ding{203} \textcolor{Sepia}{\selectlanguage{english}Poor (person).} \zh{穷(人)。}  ¶ \textcolor{darkblue}{\textbf{\ipa{dzɑ˧ | -ʐwæ˩-ze˥!}}} \textcolor{Sepia}{\selectlanguage{english}(He/she) is really poor!} \zh{他很穷!}  
 ¶ \textcolor{darkblue}{\textbf{\ipa{ɑ˩ʁo˧ | bo˩ʈʂʰæ˧ mɤ˧-dʑo˧, | dzɑ˧ ʈʂɤ˧-kv̩˩!}}} \textcolor{Sepia}{\selectlanguage{english}If there is no fleshless preserved pork at home, it appears as if the family is really destitute!} \zh{如果家里没有猪膘,会显得很穷!}  
 ¶ \textcolor{darkblue}{\textbf{\ipa{dzɑ˧ ʈʂɤ˧ | ʐwæ˩˥!}}} \textcolor{Sepia}{\selectlanguage{english}It's really a shame / it's really something to be ashamed of! (Talking about a socially stigmatized situation, such as not having the required food items or pieces of clothing for important ceremonies.)} \zh{真羞耻啊!}  

\lhead{\firstmark}
\rhead{\botmark}

\subsection{\hspace{-0.5cm} {\Large \textcolor{darkblue}{\textbf{\ipa{dzɑ˩qʰwɤ˩}}}}\hspace{0.5cm}[\kern2pt{\textcolor{darkblue}{\textbf{\ipa{dzɑ˧qʰwɤ˥}}}}\kern2pt]} \hypertarget{dzA\string_Bq\string_hw7\string_B1}{}
\markboth{\textcolor{darkblue}{\textbf{\ipa{dzɑ˩qʰwɤ˩}}}}{}
\textcolor{teal}{\mytextsc{noun}} \hspace{4pt} Tone: L.
\textcolor{Sepia}{\selectlanguage{english}Shoe.} \zh{鞋、鞋子。}  \zh{量词}: \textcolor{darkblue}{\textbf{\ipa{dzi˧}}}  \mytextsc{clf}: \textcolor{darkblue}{\textbf{\ipa{dzi˧}}} 
\lhead{\firstmark}
\rhead{\botmark}

\subsection{\hspace{-0.5cm} {\Large \textcolor{darkblue}{\textbf{\ipa{dze˥}}}}\hspace{0.5cm}[\kern2pt{\textcolor{darkblue}{\textbf{\ipa{dze˩˥}}}}\kern2pt]} \hypertarget{dze\string_T1}{}
\markboth{\textcolor{darkblue}{\textbf{\ipa{dze˥}}}}{}
\textcolor{teal}{\mytextsc{noun}} \hspace{4pt} Tone: \#H.
\textcolor{Sepia}{\selectlanguage{english}Sugar.} \zh{糖。} 
\lhead{\firstmark}
\rhead{\botmark}

\subsection{\hspace{-0.5cm} {\Large \textcolor{darkblue}{\textbf{\ipa{dze˧bɤ˩}}}}\hspace{0.5cm}[\kern2pt{\textcolor{darkblue}{\textbf{\ipa{dze˧bɤ˧}}}}\kern2pt]} \hypertarget{dze\string_Mb7\string_B1}{}
\markboth{\textcolor{darkblue}{\textbf{\ipa{dze˧bɤ˩}}}}{}
\textcolor{teal}{\mytextsc{noun}} \hspace{4pt} Tone: L\#.
\textcolor{Sepia}{\selectlanguage{english}Bat; used for all species, including the flying squirrel.} \zh{蝙蝠、飞鼠。}  ¶ \textcolor{darkblue}{\textbf{\ipa{dze˧bɤ˩-zo˩ | ɖɯ˧-ɭɯ˧}}} \textcolor{Sepia}{\selectlanguage{english}a baby bat} \zh{一只小蝙蝠}  
 ¶ \textcolor{darkblue}{\textbf{\ipa{dze˧bɤ˩-pʰv̩˩ | ɖɯ˧-mi˩}}} \textcolor{Sepia}{\selectlanguage{english}a male bat} \zh{一只公蝙蝠}  
 ¶ \textcolor{darkblue}{\textbf{\ipa{dze˧bɤ˩-mi˩ | ɖɯ˧-mi˩}}} \textcolor{Sepia}{\selectlanguage{english}a female bat} \zh{一只母蝙蝠}  
 \zh{量词}: \textcolor{darkblue}{\textbf{\ipa{mi˩}}}  \mytextsc{clf}: \textcolor{darkblue}{\textbf{\ipa{mi˩}}} 
\lhead{\firstmark}
\rhead{\botmark}

\subsection{\hspace{-0.5cm} {\Large \textcolor{darkblue}{\textbf{\ipa{dze˧bo˧}}}}\hspace{0.5cm}[\kern2pt{\textcolor{darkblue}{\textbf{\ipa{dze˩bo˩˥}}}}\kern2pt]} \hypertarget{dze\string_Mbo\string_M1}{}
\markboth{\textcolor{darkblue}{\textbf{\ipa{dze˧bo˧}}}}{}
\textcolor{teal}{\mytextsc{noun}} \hspace{4pt} Tone: M.
\ding{202} \textcolor{Sepia}{\selectlanguage{english}A family name from Yongning.} \zh{者波(姓)。这个家族有三个家庭。}  ¶ \textcolor{darkblue}{\textbf{\ipa{dze˧bo˧=ɻ̍˩}}} \textcolor{Sepia}{\selectlanguage{english}the \textcolor{darkblue}{\textbf{\ipa{/dze˧bo˧/}}} clan, the \textcolor{darkblue}{\textbf{\ipa{/dze˧bo˧/}}} family} \zh{者波家族}  
\ding{203} \textcolor{Sepia}{\selectlanguage{english}A village in the Yongning plain. It consists of two parts, “upper” and "lower: \textcolor{darkblue}{\textbf{\ipa{/gɤ˩ʁwɤ˧/}}} and \textcolor{darkblue}{\textbf{\ipa{/mv̩˩ʁwɤ˧/}}}.} \zh{者波(永宁的一个村落)。村落有两个部分,\textcolor{darkblue}{\textbf{\ipa{/gɤ˩ʁwɤ˧/}}}‘上村’与\textcolor{darkblue}{\textbf{\ipa{/mv̩˩ʁwɤ˧/}}}‘下村’.}  ¶ \textcolor{darkblue}{\textbf{\ipa{ɖæ˩ʂɯ\#˥, | ʈʂo˧ʂɯ\#˥, | bɤ˩tɕʰɯ˩˥, | dɑ˧pʰo˥, | bɤ˧dzi˩, | dze˧bo˧}}} \textcolor{Sepia}{\selectlanguage{english}the six villages of the plain of Yongning, in traditional order: by order of increasing distance from the Lake} \zh{永宁坝的六个村落,按传统排序:从距离泸沽湖最近的村落说起。}  

\lhead{\firstmark}
\rhead{\botmark}

\subsection{\hspace{-0.5cm} {\Large \textcolor{darkblue}{\textbf{\ipa{dze˧dv̩˩}}}}\hspace{0.5cm}[\kern2pt{\textcolor{darkblue}{\textbf{\ipa{dze˧dv̩˩}}}}\kern2pt]} \hypertarget{dze\string_Mdv\string_=\string_B1}{}
\markboth{\textcolor{darkblue}{\textbf{\ipa{dze˧dv̩˩}}}}{}
\textcolor{teal}{\mytextsc{noun}} \hspace{4pt} Tone: L\#.
\textcolor{Sepia}{\selectlanguage{english}Cake, bread.} \zh{饼。}  ¶ \textcolor{darkblue}{\textbf{\ipa{dze˧dv̩˩-pɤ˩jɤ˩}}} \textcolor{Sepia}{\selectlanguage{english}cake of cereals} \zh{粮食饼}  

\lhead{\firstmark}
\rhead{\botmark}

\subsection{\hspace{-0.5cm} {\Large \textcolor{darkblue}{\textbf{\ipa{dze˧hi˧}}}}\hspace{0.5cm}[\kern2pt{\textcolor{darkblue}{\textbf{\ipa{dze˧hi˧}}}}\kern2pt]} \hypertarget{dze\string_Mhi\string_M1}{}
\markboth{\textcolor{darkblue}{\textbf{\ipa{dze˧hi˧}}}}{}
\textcolor{teal}{\mytextsc{noun}} \hspace{4pt} Tone: M.
\textcolor{Sepia}{\selectlanguage{english}In-laws.} \zh{丈人。}  ¶ \textcolor{darkblue}{\textbf{\ipa{njɤ˧ | dze˧hi˧-ki˩ bi˩!}}} \textcolor{Sepia}{\selectlanguage{english}I'm going to my in-laws' place! / I'm going to visit my in-laws!} \zh{我去我丈人(那边)!}  
 ¶ \textcolor{darkblue}{\textbf{\ipa{no˧ | dze˧hi˧ | ə˩-to˩-ze˥? - le˧-to˩-ze˩!}}} \textcolor{Sepia}{\selectlanguage{english}Do you have in-laws? / Do you stand in an 'in-law' relationship? (=Are you married?) - Yes, I have entered into such a relationship! (=Yes, I am married!)} \zh{你有丈人吗?(=你结婚了吗?)-有的!(=结婚了!)}  
 ¶ \textcolor{darkblue}{\textbf{\ipa{no˧ | dze˧hi˧ to˩ ə˩-bi˩?}}} \textcolor{Sepia}{\selectlanguage{english}Do you have plans to get married? (Literally: Are you going to enter an 'in-law' relationship?)} \zh{你打算结婚吗?}  

\lhead{\firstmark}
\rhead{\botmark}

\subsection{\hspace{-0.5cm} {\Large \textcolor{darkblue}{\textbf{\ipa{dze˧kʰɤ˧˥}}}}\hspace{0.5cm}[\kern2pt{\textcolor{darkblue}{\textbf{\ipa{dze˧kʰɤ˧}}}}\kern2pt]} \hypertarget{dze\string_Mk\string_h7\string_M\string_T1}{}
\markboth{\textcolor{darkblue}{\textbf{\ipa{dze˧kʰɤ˧˥}}}}{}
\textcolor{teal}{\mytextsc{noun}} \hspace{4pt} Tone: MH\#.
\textcolor{Sepia}{\selectlanguage{english}Commoner (second of the three ranks in feudal society).} \zh{百姓。音译:“责卡”。}  \zh{量词}: \textcolor{darkblue}{\textbf{\ipa{v̩˧}}}  \mytextsc{clf}: \textcolor{darkblue}{\textbf{\ipa{v̩˧}}} 
\lhead{\firstmark}
\rhead{\botmark}

\subsection{\hspace{-0.5cm} {\Large \textcolor{darkblue}{\textbf{\ipa{dze˧ɭɯ˧}}}}\hspace{0.5cm}[\kern2pt{\textcolor{darkblue}{\textbf{\ipa{dze˧ɭɯ˧˥}}}}\kern2pt]} \hypertarget{dze\string_Ml\string_RM\string_M1}{}
\markboth{\textcolor{darkblue}{\textbf{\ipa{dze˧ɭɯ˧}}}}{}
\textcolor{teal}{\mytextsc{noun}} \hspace{4pt} Tone: M.
\textcolor{Sepia}{\selectlanguage{english}Wheat.} \zh{小麦。} 
\lhead{\firstmark}
\rhead{\botmark}

\subsection{\hspace{-0.5cm} {\Large \textcolor{darkblue}{\textbf{\ipa{dze˧ɭɯ˧-ɻ̃\#˥}}}}\hspace{0.5cm}[\kern2pt{\textcolor{darkblue}{\textbf{\ipa{xxxx non-correspondance entre le nombre de morphèmes et le nombre de tons de morphèmes}}}}\kern2pt]} \hypertarget{dze\string_Ml\string_RM\string_M-r£`\string_~\#\string_T1}{}
\markboth{\textcolor{darkblue}{\textbf{\ipa{dze˧ɭɯ˧-ɻ̃\#˥}}}}{}
\textcolor{teal}{\mytextsc{noun}} \hspace{4pt} Tone: \#H.
\textcolor{Sepia}{\selectlanguage{english}Wheat straw.} \zh{麦杆。} 
\lhead{\firstmark}
\rhead{\botmark}

\subsection{\hspace{-0.5cm} {\Large \textcolor{darkblue}{\textbf{\ipa{dze˧-ɻ̃\#˥}}}}\hspace{0.5cm}[\kern2pt{\textcolor{darkblue}{\textbf{\ipa{xxxx non-correspondance entre le nombre de morphèmes et le nombre de tons de morphèmes}}}}\kern2pt]} \hypertarget{dze\string_M-r£`\string_~\#\string_T1}{}
\markboth{\textcolor{darkblue}{\textbf{\ipa{dze˧-ɻ̃\#˥}}}}{}
\textcolor{teal}{\mytextsc{noun}} \hspace{4pt} Tone: \#H.
\textcolor{Sepia}{\selectlanguage{english}Wheat straw.} \zh{小麦秆。} 
\lhead{\firstmark}
\rhead{\botmark}

\subsection{\hspace{-0.5cm} {\Large \textcolor{darkblue}{\textbf{\ipa{dze˧-tɕʰi\#˥}}}}\hspace{0.5cm}[\kern2pt{\textcolor{darkblue}{\textbf{\ipa{xxxx non-correspondance entre le nombre de morphèmes et le nombre de tons de morphèmes}}}}\kern2pt]} \hypertarget{dze\string_M-ts£\string_hi\#\string_T1}{}
\markboth{\textcolor{darkblue}{\textbf{\ipa{dze˧-tɕʰi\#˥}}}}{}
\textcolor{teal}{\mytextsc{noun}} \hspace{4pt} Tone: \#H.
\textcolor{Sepia}{\selectlanguage{english}Wheat beard.} \zh{麦芒。} 
\lhead{\firstmark}
\rhead{\botmark}

\subsection{\hspace{-0.5cm} {\Large \textcolor{darkblue}{\textbf{\ipa{dze˧-ʈʂæ˥}}}}\hspace{0.5cm}[\kern2pt{\textcolor{darkblue}{\textbf{\ipa{xxxx non-correspondance entre le nombre de morphèmes et le nombre de tons de morphèmes}}}}\kern2pt]} \hypertarget{dze\string_M-t`s`\{\string_T1}{}
\markboth{\textcolor{darkblue}{\textbf{\ipa{dze˧-ʈʂæ˥}}}}{}
\textcolor{teal}{\mytextsc{noun}} \hspace{4pt} Tone: H\#.
\textcolor{Sepia}{\selectlanguage{english}Sting organ.} \zh{蜜蜂的螫針。}  \zh{量词}: \textcolor{darkblue}{\textbf{\ipa{ɭɯ˧}}}  \mytextsc{clf}: \textcolor{darkblue}{\textbf{\ipa{ɭɯ˧}}} 
\lhead{\firstmark}
\rhead{\botmark}

\subsection{\hspace{-0.5cm} {\Large \textcolor{darkblue}{\textbf{\ipa{dze˧ʈʂɯ˧}}}}\hspace{0.5cm}[\kern2pt{\textcolor{darkblue}{\textbf{\ipa{dze˧ʈʂɯ˥}}}}\kern2pt]} \hypertarget{dze\string_Mt`s`M\string_M1}{}
\markboth{\textcolor{darkblue}{\textbf{\ipa{dze˧ʈʂɯ˧}}}}{}
\textcolor{teal}{\mytextsc{noun}} \hspace{4pt} Tone: M.
\textcolor{Sepia}{\selectlanguage{english}Sifter, sieve.} \zh{筛子。}  \zh{量词}: \textcolor{darkblue}{\textbf{\ipa{nɑ˧}}}  \mytextsc{clf}: \textcolor{darkblue}{\textbf{\ipa{nɑ˧}}} 
\lhead{\firstmark}
\rhead{\botmark}

\subsection{\hspace{-0.5cm} {\Large \textcolor{darkblue}{\textbf{\ipa{dze˧ʈʂʰɤ\$˥}}}}\hspace{0.5cm}[\kern2pt{\textcolor{darkblue}{\textbf{\ipa{dze˧ʈʂʰɤ˥}}}}\kern2pt]} \hypertarget{dze\string_Mt`s`\string_h7\$\string_T1}{}
\markboth{\textcolor{darkblue}{\textbf{\ipa{dze˧ʈʂʰɤ\$˥}}}}{}
\textcolor{teal}{\mytextsc{noun}} \hspace{4pt} Tone: H\$.
\textcolor{Sepia}{\selectlanguage{english}Cereals; the main cereal crop used to be barley, but the meaning of this word currently tends to become identified with the five main sorts of grains referred to in Chinese as 'the five cereals', \zh{五谷}, namely rice, two kinds of millet, wheat, and beans.} \zh{粮食。现在,这个词的含义受到汉语‘五谷’这个词的影响,用来指代‘五谷杂粮’,相当于所有粮食类,如:小米类、稻谷、麦子、玉米以及豆类与薯类。} 
\lhead{\firstmark}
\rhead{\botmark}

\subsection{\hspace{-0.5cm} {\Large \textcolor{darkblue}{\textbf{\ipa{dze˩}}}}\hspace{0.5cm}[\kern2pt{\textcolor{darkblue}{\textbf{\ipa{dze˩˥}}}}\kern2pt]} \hypertarget{dze\string_B1}{}
\markboth{\textcolor{darkblue}{\textbf{\ipa{dze˩}}}}{}
\textcolor{teal}{\mytextsc{verb}} \hspace{4pt} Tone: L.
\textcolor{Sepia}{\selectlanguage{english}To be left over (food or drink).} \zh{剩下(饭或饮料)。}  ¶ \textcolor{darkblue}{\textbf{\ipa{dzɯ˧-dze˥-ze˩!}}} \textcolor{Sepia}{\selectlanguage{english}There are some leftovers! / The food has not been eaten up!} \zh{剩了一些饭!/ 剩了一些吃的!}  
 ¶ \textcolor{darkblue}{\textbf{\ipa{gɤ˩-dze˥ +ze˩!}}} \textcolor{Sepia}{\selectlanguage{english}There are some leftovers!} \zh{有剩下的!}  
 ¶ \textcolor{darkblue}{\textbf{\ipa{ʈʰɯ˩ dze˩-ze˥}}} \textcolor{Sepia}{\selectlanguage{english}Some of the drink is left over! / (The drink) has not been drunk up!} \zh{喝剩了、没喝完}  
 ¶ \textcolor{darkblue}{\textbf{\ipa{le˧-se˩-ze˩! | gɤ˩-mɤ˧-dze˩!}}} \textcolor{Sepia}{\selectlanguage{english}It's completely finished (=eaten up / drunk up)! There are no leftovers!} \zh{完了!(=全部吃/喝完了!)没有剩!}  

\lhead{\firstmark}
\rhead{\botmark}

\subsection{\hspace{-0.5cm} {\Large \textcolor{darkblue}{\textbf{\ipa{dze˩\textsubscript{a}}}}}\hspace{0.5cm}[\kern2pt{\textcolor{darkblue}{\textbf{\ipa{dze˥}}}}\kern2pt]} \hypertarget{dze\string_Ba1}{}
\markboth{\textcolor{darkblue}{\textbf{\ipa{dze˩\textsubscript{a}}}}}{}
\textcolor{teal}{\mytextsc{classifier}} \hspace{4pt} Tone: L\textsubscript{a}.
\textcolor{Sepia}{\selectlanguage{english}Classifier for pairs of separable objects: a pair of pots, a pair of bottles….} \zh{量词:瓶子、锅(一对)。}  ¶ \textcolor{darkblue}{\textbf{\ipa{zo˧mv̩˥ | ɖɯ˧-dze˩}}} \textcolor{Sepia}{\selectlanguage{english}twins (literally: 'a pair of children') (F5)} \zh{双胞胎(直译:“一对孩子”)}  
 ¶ \textcolor{darkblue}{\textbf{\ipa{ʈʂʰɯ˧-dze˥}}} \textcolor{Sepia}{\selectlanguage{english}\mytextsc{dem} \string_ (tone: H\# / H\$)} \zh{\mytextsc{指示代词} \string_}  

\lhead{\firstmark}
\rhead{\botmark}

\subsection{\hspace{-0.5cm} {\Large \textcolor{darkblue}{\textbf{\ipa{dze˩\textsubscript{a}}}} \textsubscript{1}}\hspace{0.5cm}[\kern2pt{\textcolor{darkblue}{\textbf{\ipa{dze˩˥}}}}\kern2pt]} \hypertarget{dze\string_Ba1}{}
\markboth{\textcolor{darkblue}{\textbf{\ipa{dze˩\textsubscript{a}}}} \textsubscript{1}}{}
\textcolor{teal}{\mytextsc{verb}} \hspace{4pt} Tone: L\textsubscript{a}.
\textcolor{Sepia}{\selectlanguage{english}To fly.} \zh{飞。}  ¶ \textcolor{darkblue}{\textbf{\ipa{le˧-dze˩-hɯ˩-ze˩}}} \textcolor{Sepia}{\selectlanguage{english}(The bird) has flown away.} \zh{(鸟)飞走了。}  
 ¶ \textcolor{darkblue}{\textbf{\ipa{mv̩˧ʁo˧ dze˧˥}}} \textcolor{Sepia}{\selectlanguage{english}to fly in the sky} \zh{在天空中飞}  

\lhead{\firstmark}
\rhead{\botmark}

\subsection{\hspace{-0.5cm} {\Large \textcolor{darkblue}{\textbf{\ipa{dze˩\textsubscript{a}}}} \textsubscript{2}}\hspace{0.5cm}[\kern2pt{\textcolor{darkblue}{\textbf{\ipa{dze˩˥}}}}\kern2pt]} \hypertarget{dze\string_Ba2}{}
\markboth{\textcolor{darkblue}{\textbf{\ipa{dze˩\textsubscript{a}}}} \textsubscript{2}}{}
\textcolor{teal}{\mytextsc{verb}} \hspace{4pt} Tone: L\textsubscript{a}.
\textcolor{Sepia}{\selectlanguage{english}To cut (with a knife).} \zh{切(用刀)。}  ¶ \textcolor{darkblue}{\textbf{\ipa{le˧-dze˩}}} \textcolor{Sepia}{\selectlanguage{english}\mytextsc{accomp}} \zh{\mytextsc{accomp}}  
 ¶ \textcolor{darkblue}{\textbf{\ipa{dze˧\textasciitilde{}dze˥}}} \textcolor{Sepia}{\selectlanguage{english}\mytextsc{red}} \zh{\mytextsc{red}}  
 ¶ \textcolor{darkblue}{\textbf{\ipa{le˧-dze˧\textasciitilde{}dze˥}}} \textcolor{Sepia}{\selectlanguage{english}\mytextsc{accomp} \string_ \mytextsc{red}} \zh{\mytextsc{accomp} \string_ \mytextsc{red}}  
 ¶ \textcolor{darkblue}{\textbf{\ipa{v̩˩tsʰɤ˧ dze˧\textasciitilde{}dze˥}}} \textcolor{Sepia}{\selectlanguage{english}to cut vegetables} \zh{切菜}  
 ¶ \textcolor{darkblue}{\textbf{\ipa{nv̩˩dʑɯ˥ dze˩\textasciitilde{}dze˩}}} \textcolor{Sepia}{\selectlanguage{english}to cut tofu} \zh{切豆腐}  

\lhead{\firstmark}
\rhead{\botmark}

\subsection{\hspace{-0.5cm} {\Large \textcolor{darkblue}{\textbf{\ipa{dze˩dʑɯ˧˥}}}}\hspace{0.5cm}[\kern2pt{\textcolor{darkblue}{\textbf{\ipa{dze˧dʑɯ˩}}}}\kern2pt]} \hypertarget{dze\string_Bdz£M\string_M\string_T1}{}
\markboth{\textcolor{darkblue}{\textbf{\ipa{dze˩dʑɯ˧˥}}}}{}
\textcolor{teal}{\mytextsc{adjective}} \hspace{4pt} Tone: LM+MH\#.
\textcolor{Sepia}{\selectlanguage{english}Arrogant, conceited.} \zh{骄傲,自以为好。}  ¶ \textcolor{darkblue}{\textbf{\ipa{ʈʂʰɯ˧ | hĩ˧-bi˥ | mɤ˧-li˧! | dze˩dʑɯ˧˥ | ʐwæ˧˥!}}} \textcolor{Sepia}{\selectlanguage{english}He despises others! He is very arrogant!} \zh{他看不起别人!他很骄傲!}  

\lhead{\firstmark}
\rhead{\botmark}

\subsection{\hspace{-0.5cm} {\Large \textcolor{darkblue}{\textbf{\ipa{dze˩mi˧}}}}\hspace{0.5cm}[\kern2pt{\textcolor{darkblue}{\textbf{\ipa{dze˧mi˧}}}}\kern2pt]} \hypertarget{dze\string_Bmi\string_M1}{}
\markboth{\textcolor{darkblue}{\textbf{\ipa{dze˩mi˧}}}}{}
\textcolor{teal}{\mytextsc{noun}} \hspace{4pt} Tone: LM.
\textcolor{Sepia}{\selectlanguage{english}Bee.} \zh{蜜蜂。}  \zh{量词}: \textcolor{darkblue}{\textbf{\ipa{mi˩}}}  \mytextsc{clf}: \textcolor{darkblue}{\textbf{\ipa{mi˩}}} 
\lhead{\firstmark}
\rhead{\botmark}

\subsection{\hspace{-0.5cm} {\Large \textcolor{darkblue}{\textbf{\ipa{dze˩mi˧-bæ˩bæ˩}}}}\hspace{0.5cm}[\kern2pt{\textcolor{darkblue}{\textbf{\ipa{xxxx non-correspondance entre le nombre de morphèmes et le nombre de tons de morphèmes}}}}\kern2pt]} \hypertarget{dze\string_Bmi\string_M-b\{\string_Bb\{\string_B1}{}
\markboth{\textcolor{darkblue}{\textbf{\ipa{dze˩mi˧-bæ˩bæ˩}}}}{}
\textcolor{teal}{\mytextsc{noun}} \hspace{4pt} Tone: L\#.
\textcolor{Sepia}{\selectlanguage{english}\textit{Artemisia suboligata}.} \zh{茶绒蒿。}  \zh{量词}: \textcolor{darkblue}{\textbf{\ipa{bæ˩}}}  \mytextsc{clf}: \textcolor{darkblue}{\textbf{\ipa{bæ˩}}} 
\lhead{\firstmark}
\rhead{\botmark}

\subsection{\hspace{-0.5cm} {\Large \textcolor{darkblue}{\textbf{\ipa{dze˩mi˧-dze\#˥}}}}\hspace{0.5cm}[\kern2pt{\textcolor{darkblue}{\textbf{\ipa{xxxx non-correspondance entre le nombre de morphèmes et le nombre de tons de morphèmes}}}}\kern2pt]} \hypertarget{dze\string_Bmi\string_M-dze\#\string_T1}{}
\markboth{\textcolor{darkblue}{\textbf{\ipa{dze˩mi˧-dze\#˥}}}}{}
\textcolor{teal}{\mytextsc{noun}} \hspace{4pt} Tone: LM+\#H.
\textit{From:} \textbf{dze˩mi˧ and dze˥} \textcolor{Sepia}{\selectlanguage{english}Honey.} \zh{蜂蜜。}  ¶ \textcolor{darkblue}{\textbf{\ipa{dze˩mi˧dze˧ dzɯ˧}}} \textcolor{Sepia}{\selectlanguage{english}to eat honey} \zh{吃蜂蜜}  
 \zh{量词}: \textcolor{darkblue}{\textbf{\ipa{kʰwɤ˥}}}  \mytextsc{clf}: \textcolor{darkblue}{\textbf{\ipa{kʰwɤ˥}}} 
\lhead{\firstmark}
\rhead{\botmark}

\subsection{\hspace{-0.5cm} {\Large \textcolor{darkblue}{\textbf{\ipa{dze˩mi˧-kʰv̩˩}}}}\hspace{0.5cm}[\kern2pt{\textcolor{darkblue}{\textbf{\ipa{xxxx non-correspondance entre le nombre de morphèmes et le nombre de tons de morphèmes}}}}\kern2pt]} \hypertarget{dze\string_Bmi\string_M-k\string_hv\string_=\string_B1}{}
\markboth{\textcolor{darkblue}{\textbf{\ipa{dze˩mi˧-kʰv̩˩}}}}{}
\textcolor{teal}{\mytextsc{noun}} \hspace{4pt} Tone: LM-L.
\textcolor{Sepia}{\selectlanguage{english}Beehive, honeycomb.} \zh{蜂窝。}  \zh{量词}: \textcolor{darkblue}{\textbf{\ipa{ɭɯ˧}}}  \mytextsc{clf}: \textcolor{darkblue}{\textbf{\ipa{ɭɯ˧}}} 
\lhead{\firstmark}
\rhead{\botmark}

\subsection{\hspace{-0.5cm} {\Large \textcolor{darkblue}{\textbf{\ipa{dze˩mi˧-pv̩˥ɻ̍˩}}}}\hspace{0.5cm}[\kern2pt{\textcolor{darkblue}{\textbf{\ipa{dze˩mi˧pv̩˩ɻ̍˩}}}}\kern2pt]} \hypertarget{dze\string_Bmi\string_M-pv\string_=\string_Tr£`̍\string_B1}{}
\markboth{\textcolor{darkblue}{\textbf{\ipa{dze˩mi˧-pv̩˥ɻ̍˩}}}}{}
\textcolor{teal}{\mytextsc{noun}} \hspace{4pt} Tone: LM+\#H-.
\textcolor{Sepia}{\selectlanguage{english}Terme générique pour les fleurs qu'affectionnent les abeilles: diverses fleurs dont les vertus médicinales ne sont pas connues. Par exemple: \textit{Adenophora sp.}: root of straight ladybell (flower). yyyy.} \zh{沙参。} Local Chinese dialect:\zh{yyyy fusionner les 2 entrées; est un terme générique ; tʰi˧-pv˥ɻ˩ = tʰi˧-hɑ̃˧˥ se reposer qq part, se poser quelque part。} ¶ \textcolor{darkblue}{\textbf{\ipa{dze˩mi˧-pv̩˥ɻ̍˩-kʰɯ˩ʈɯ˩}}} \textcolor{Sepia}{\selectlanguage{english}root of straight ladybell} \zh{沙参根}  

\lhead{\firstmark}
\rhead{\botmark}

\subsection{\hspace{-0.5cm} {\Large \textcolor{darkblue}{\textbf{\ipa{dze˩mi˧-pv̩˥ɻ̍˩}}}}\hspace{0.5cm}[\kern2pt{\textcolor{darkblue}{\textbf{\ipa{dze˩mi˧pv̩˥ɻ̍˩}}}}\kern2pt]} \hypertarget{dze\string_Bmi\string_M-pv\string_=\string_Tr£`̍\string_B1}{}
\markboth{\textcolor{darkblue}{\textbf{\ipa{dze˩mi˧-pv̩˥ɻ̍˩}}}}{}
\textcolor{teal}{\mytextsc{noun}} \hspace{4pt} Tone: LM+\#H-.
\textcolor{Sepia}{\selectlanguage{english}Large-leaved gentian.} \zh{秦艽。}  ¶ \textcolor{darkblue}{\textbf{\ipa{dʑɯ˧qʰɑ˧-bæ˩bæ˩}}} \textcolor{Sepia}{\selectlanguage{english}gentian flowers} \zh{秦艽花}  
 \zh{量词}: \textcolor{darkblue}{\textbf{\ipa{qɑ˩}}}  \mytextsc{clf}: \textcolor{darkblue}{\textbf{\ipa{qɑ˩}}} 
\lhead{\firstmark}
\rhead{\botmark}

\subsection{\hspace{-0.5cm} {\Large \textcolor{darkblue}{\textbf{\ipa{*dze˩˧}}}}\hspace{0.5cm}[\kern2pt{\textcolor{darkblue}{\textbf{\ipa{dze˩˥}}}}\kern2pt]} \hypertarget{*dze\string_B\string_M1}{}
\markboth{\textcolor{darkblue}{\textbf{\ipa{*dze˩˧}}}}{}
\textcolor{teal}{\mytextsc{noun}} \hspace{4pt} Tone: LM.
\textcolor{Sepia}{\selectlanguage{english}Bee.} \zh{蜜蜂。} 
\lhead{\firstmark}
\rhead{\botmark}

\subsection{\hspace{-0.5cm} {\Large \textcolor{darkblue}{\textbf{\ipa{dze˩˧}}}}\hspace{0.5cm}[\kern2pt{\textcolor{darkblue}{\textbf{\ipa{dze˥}}}}\kern2pt]} \hypertarget{dze\string_B\string_M1}{}
\markboth{\textcolor{darkblue}{\textbf{\ipa{dze˩˧}}}}{}
\textcolor{teal}{\mytextsc{noun}} \hspace{4pt} Tone: LM.
\textcolor{Sepia}{\selectlanguage{english}Wild pepper, Szechuan pepper.} \zh{花椒。}  \zh{量词}: \textcolor{darkblue}{\textbf{\ipa{mɤ˩}}}  \mytextsc{clf}: \textcolor{darkblue}{\textbf{\ipa{mɤ˩}}} 
\lhead{\firstmark}
\rhead{\botmark}

\subsection{\hspace{-0.5cm} {\Large \textcolor{darkblue}{\textbf{\ipa{dzɤ˥\textsubscript{b}}}}}\hspace{0.5cm}[\kern2pt{\textcolor{darkblue}{\textbf{\ipa{dzɤ˥}}}}\kern2pt]} \hypertarget{dz7\string_Tb1}{}
\markboth{\textcolor{darkblue}{\textbf{\ipa{dzɤ˥\textsubscript{b}}}}}{}
\textcolor{teal}{\mytextsc{classifier}} \hspace{4pt} Tone: H\textsubscript{b}.
\textcolor{Sepia}{\selectlanguage{english}Classifier for sides.} \zh{量词:面。}  ¶ \textcolor{darkblue}{\textbf{\ipa{ʈʂʰɯ˧-dzɤ˧}}} \textcolor{Sepia}{\selectlanguage{english}this side} \zh{这面}  
 ¶ \textcolor{darkblue}{\textbf{\ipa{ɖɯ˧-dzɤ˥}}} \textcolor{Sepia}{\selectlanguage{english}one side} \zh{一面}  

\lhead{\firstmark}
\rhead{\botmark}

\subsection{\hspace{-0.5cm} {\Large \textcolor{darkblue}{\textbf{\ipa{dzɤ˩\textsubscript{a}}}}}\hspace{0.5cm}[\kern2pt{\textcolor{darkblue}{\textbf{\ipa{dzɤ˧˥}}}}\kern2pt]} \hypertarget{dz7\string_Ba1}{}
\markboth{\textcolor{darkblue}{\textbf{\ipa{dzɤ˩\textsubscript{a}}}}}{}
\textcolor{teal}{\mytextsc{verb}} \hspace{4pt} Tone: L\textsubscript{a}.
\textcolor{Sepia}{\selectlanguage{english}To collapse, to topple over, to fall into ruin.} \zh{塌毁,倒塌 ,倒。}  ¶ \textcolor{darkblue}{\textbf{\ipa{mv̩˩tɕo˧ dzɤ˩}}} \textcolor{Sepia}{\selectlanguage{english}same meaning: to collapse} \zh{同上:塌毁}  
 ¶ \textcolor{darkblue}{\textbf{\ipa{le˧-dzɤ˩-ze˩}}} \textcolor{Sepia}{\selectlanguage{english}\mytextsc{accomp} \string_ \mytextsc{pfv}} \zh{塌毁了}  

\lhead{\firstmark}
\rhead{\botmark}

\subsection{\hspace{-0.5cm} {\Large \textcolor{darkblue}{\textbf{\ipa{dzi˥}}}}\hspace{0.5cm}[\kern2pt{\textcolor{darkblue}{\textbf{\ipa{dzi˧˥}}}}\kern2pt]} \hypertarget{dzi\string_T1}{}
\markboth{\textcolor{darkblue}{\textbf{\ipa{dzi˥}}}}{}
\textcolor{teal}{\mytextsc{noun}} \hspace{4pt} Tone: \#H.
\textcolor{Sepia}{\selectlanguage{english}Chisel.} \zh{凿子。}  \zh{量词}: \textcolor{darkblue}{\textbf{\ipa{ɭɯ˧ (*nɑ˧)}}}  \mytextsc{clf}: \textcolor{darkblue}{\textbf{\ipa{ɭɯ˧ (*nɑ˧)}}} 
\lhead{\firstmark}
\rhead{\botmark}

\subsection{\hspace{-0.5cm} {\Large \textcolor{darkblue}{\textbf{\ipa{dzi˧\textsubscript{b}}}}}\hspace{0.5cm}[\kern2pt{\textcolor{darkblue}{\textbf{\ipa{dzi˩˥}}}}\kern2pt]} \hypertarget{dzi\string_Mb1}{}
\markboth{\textcolor{darkblue}{\textbf{\ipa{dzi˧\textsubscript{b}}}}}{}
\textcolor{teal}{\mytextsc{classifier}} \hspace{4pt} Tone: M\textsubscript{b}.
\textcolor{Sepia}{\selectlanguage{english}Classifier for pairs of objects, when the pair makes up a unit: e.g. a pair of shoes.} \zh{量词:鞋(一双)。}  ¶ \textcolor{darkblue}{\textbf{\ipa{ɣɯ˩-dzɑ˩qʰwɤ˥ | ɖɯ˧-dzi˧}}} \textcolor{Sepia}{\selectlanguage{english}a pair of leather shoes} \zh{一双皮鞋}  

\lhead{\firstmark}
\rhead{\botmark}

\subsection{\hspace{-0.5cm} {\Large \textcolor{darkblue}{\textbf{\ipa{dzi˧dzi˧}}}}\hspace{0.5cm}[\kern2pt{\textcolor{darkblue}{\textbf{\ipa{dzi˧dzi˧}}}}\kern2pt]} \hypertarget{dzi\string_Mdzi\string_M1}{}
\markboth{\textcolor{darkblue}{\textbf{\ipa{dzi˧dzi˧}}}}{}
\textcolor{teal}{\mytextsc{noun}} \hspace{4pt} Tone: M.
\textcolor{Sepia}{\selectlanguage{english}Oriental white oak.} \zh{青冈树、槲栎。}  ¶ \textcolor{darkblue}{\textbf{\ipa{dzi˧dzi˧, | si˧dzi˩-mv̩˩!}}} \textcolor{Sepia}{\selectlanguage{english}\textcolor{darkblue}{\textbf{\ipa{/dzi˧dzi˧/}}} is the name of a tree!} \zh{\textcolor{darkblue}{\textbf{\ipa{dzi˧dzi˧}}}是一种树的名字!}  
\textit{Syn:} \hyperlink{}{\textcolor{darkblue}{\textbf{\ipa{dʑɯ˩si˩}}}}. 
\lhead{\firstmark}
\rhead{\botmark}

\subsection{\hspace{-0.5cm} {\Large \textcolor{darkblue}{\textbf{\ipa{dzi˧dzi˧-mo˧˥}}}}\hspace{0.5cm}[\kern2pt{\textcolor{darkblue}{\textbf{\ipa{xxxx non-correspondance entre le nombre de morphèmes et le nombre de tons de morphèmes}}}}\kern2pt]} \hypertarget{dzi\string_Mdzi\string_M-mo\string_M\string_T1}{}
\markboth{\textcolor{darkblue}{\textbf{\ipa{dzi˧dzi˧-mo˧˥}}}}{}
\textcolor{teal}{\mytextsc{noun}} \hspace{4pt} Tone: MH\#.
\textcolor{Sepia}{\selectlanguage{english}A species of edible mushroom that grows on fallen trees; it is used as a medicine against stomach-ache.} \zh{一种可以吃的菌子,长在枯木上。} 
\lhead{\firstmark}
\rhead{\botmark}

\subsection{\hspace{-0.5cm} {\Large \textcolor{darkblue}{\textbf{\ipa{dzi˧ɖæ˧}}}}\hspace{0.5cm}[\kern2pt{\textcolor{darkblue}{\textbf{\ipa{dzi˧ɖæ˧}}}}\kern2pt]} \hypertarget{dzi\string_Md`\{\string_M1}{}
\markboth{\textcolor{darkblue}{\textbf{\ipa{dzi˧ɖæ˧}}}}{}
\textcolor{teal}{\mytextsc{noun}} \hspace{4pt} Tone: M.
\textcolor{Sepia}{\selectlanguage{english}Location.} \zh{位置、所在地。}  \zh{量词}: \textcolor{darkblue}{\textbf{\ipa{kʰwɤ˥}}}  \mytextsc{clf}: \textcolor{darkblue}{\textbf{\ipa{kʰwɤ˥}}} 
\lhead{\firstmark}
\rhead{\botmark}

\subsection{\hspace{-0.5cm} {\Large \textcolor{darkblue}{\textbf{\ipa{dzi˩}}} \textsubscript{1}}\hspace{0.5cm}[\kern2pt{\textcolor{darkblue}{\textbf{\ipa{dzi˥}}}}\kern2pt]} \hypertarget{dzi\string_B1}{}
\markboth{\textcolor{darkblue}{\textbf{\ipa{dzi˩}}} \textsubscript{1}}{}
\textcolor{teal}{\mytextsc{verb}} \hspace{4pt} Tone: L.
\textcolor{Sepia}{\selectlanguage{english}To fall, to come (of night); to get (dark).} \zh{来(晚上来了)。}  ¶ \textcolor{darkblue}{\textbf{\ipa{nɑ˩˥ | le˧-dzi˩-ze˩!}}} \textcolor{Sepia}{\selectlanguage{english}The night has fallen! / It has got dark!} \zh{天黑了!}  
 ¶ \textcolor{darkblue}{\textbf{\ipa{nɑ˩˥ | le˧-dzi˩ | le˧-se˩-ze˩!}}} \textcolor{Sepia}{\selectlanguage{english}It has got completely dark!} \zh{天完全黑了!}  

\lhead{\firstmark}
\rhead{\botmark}

\subsection{\hspace{-0.5cm} {\Large \textcolor{darkblue}{\textbf{\ipa{dzi˩}}} \textsubscript{2}}\hspace{0.5cm}[\kern2pt{\textcolor{darkblue}{\textbf{\ipa{dzi˩˥}}}}\kern2pt]} \hypertarget{dzi\string_B2}{}
\markboth{\textcolor{darkblue}{\textbf{\ipa{dzi˩}}} \textsubscript{2}}{}
\textcolor{teal}{\mytextsc{classifier}} \hspace{4pt} Tone: L\textsubscript{c}.
\textcolor{Sepia}{\selectlanguage{english}Classifier for entire dresses.} \zh{量词:衣服(一套)。}  ¶ \textcolor{darkblue}{\textbf{\ipa{dʑi˧hṽ˥ | ɖɯ˧-dzi˩}}} \textcolor{Sepia}{\selectlanguage{english}a full set of dress, a complete dress} \zh{一套衣服}  
 ¶ \textcolor{darkblue}{\textbf{\ipa{dʑi˧hṽ˧ ɖɯ˧-dzi˩}}} \textcolor{Sepia}{\selectlanguage{english}a full set of dress, a complete dress (same as preceding example, integrated into a single tone group)} \zh{一套衣服(同上,但将短语合在一起,构成一个单一的声调短语)}  

\lhead{\firstmark}
\rhead{\botmark}

\subsection{\hspace{-0.5cm} {\Large \textcolor{darkblue}{\textbf{\ipa{dzi˩\textsubscript{a}}}} \textsubscript{1}}\hspace{0.5cm}[\kern2pt{\textcolor{darkblue}{\textbf{\ipa{dzi˥}}}}\kern2pt]} \hypertarget{dzi\string_Ba1}{}
\markboth{\textcolor{darkblue}{\textbf{\ipa{dzi˩\textsubscript{a}}}} \textsubscript{1}}{}
\textcolor{teal}{\mytextsc{verb}} \hspace{4pt} Tone: L\textsubscript{a}.
\ding{202} \textcolor{Sepia}{\selectlanguage{english}To sit.} \zh{坐。}  ¶ \textcolor{darkblue}{\textbf{\ipa{tʰi˧-dzi˩!}}} \textcolor{Sepia}{\selectlanguage{english}Sit down!} \zh{坐下!}  
 ¶ \textcolor{darkblue}{\textbf{\ipa{hĩ˧bæ˧ ʈʂʰɯ˧-qo˧ dzi˩.}}} \textcolor{Sepia}{\selectlanguage{english}The guest sits here.} \zh{客人是坐在这边的。}  
 ¶ \textcolor{darkblue}{\textbf{\ipa{(ʈʂʰɯ˧ | ) tʰi˧-dzi˩-kʰɯ˩-se˩.}}} \textcolor{Sepia}{\selectlanguage{english}(S)he has got seated.} \zh{他坐下了。}  
 ¶ \textcolor{darkblue}{\textbf{\ipa{le˧-dzi˧\textasciitilde{}dzi˥}}} \textcolor{Sepia}{\selectlanguage{english}to remain seated; used as a euphemism to mean: to sit with others at a funeral wake, to keep a deathwatch} \zh{坐一坐。来指:居丧、守灵(委婉语)}  
\ding{203} \textcolor{Sepia}{\selectlanguage{english}To dwell, to live at a place.} \zh{住。}  ¶ \textcolor{darkblue}{\textbf{\ipa{dzi˩-bi˩-ni˩gv̩˩}}} \textcolor{Sepia}{\selectlanguage{english}to be accustomed to; to get accustomed to, to feel at ease, to adapt (to an environment)} \zh{习惯(一个新的环境、一个地方的饮食……)}  
 ¶ \textcolor{darkblue}{\textbf{\ipa{dzi˩-bi˩-ni˩-mɤ˩-gv̩˩˥}}} \textcolor{Sepia}{\selectlanguage{english}not to get accustomed; to feel awkward} \zh{不习惯}  
 ¶ \textcolor{darkblue}{\textbf{\ipa{njɤ˧ | ʈʂʰɯ˧-qo˧ | dzi˩-bi˩-ni˩-mɤ˩-gv̩˩˥}}} \textcolor{Sepia}{\selectlanguage{english}I can't get accustomed (to this place)! / (I) can't make myself at ease here! / (I) don't like it here!} \zh{我不习惯这里! / 我不喜欢这里! / 我不想待了!}  

\lhead{\firstmark}
\rhead{\botmark}

\subsection{\hspace{-0.5cm} {\Large \textcolor{darkblue}{\textbf{\ipa{dzi˩\textsubscript{a}}}} \textsubscript{2}}\hspace{0.5cm}[\kern2pt{\textcolor{darkblue}{\textbf{\ipa{dzi˩˥}}}}\kern2pt]} \hypertarget{dzi\string_Ba2}{}
\markboth{\textcolor{darkblue}{\textbf{\ipa{dzi˩\textsubscript{a}}}} \textsubscript{2}}{}
\textcolor{teal}{\mytextsc{verb}} \hspace{4pt} Tone: L\textsubscript{a}.
\textcolor{Sepia}{\selectlanguage{english}To gather, to assemble (people gather together).} \zh{聚集。}  ¶ \textcolor{darkblue}{\textbf{\ipa{ɖɯ˧-ʁwɤ˧ | le˧-dzi˧\textasciitilde{}dzi˥}}} \textcolor{Sepia}{\selectlanguage{english}the whole village gathered together} \zh{全村都聚集在一起}  
 ¶ \textcolor{darkblue}{\textbf{\ipa{hĩ˧ ɖɯ˧-v̩˧ | le˧-ʂɯ˧-ze˧! le˧-dzi˧\textasciitilde{}dzi˥ jo˩!}}} \textcolor{Sepia}{\selectlanguage{english}Someone has passed away! Come and join the gathering!} \zh{一个人去世了!来参加丧礼吧!}  

\lhead{\firstmark}
\rhead{\botmark}

\subsection{\hspace{-0.5cm} {\Large \textcolor{darkblue}{\textbf{\ipa{dzi˩\textsubscript{a}}}} \textsubscript{3}}\hspace{0.5cm}[\kern2pt{\textcolor{darkblue}{\textbf{\ipa{dzi˩˥}}}}\kern2pt]} \hypertarget{dzi\string_Ba3}{}
\markboth{\textcolor{darkblue}{\textbf{\ipa{dzi˩\textsubscript{a}}}} \textsubscript{3}}{}
\textcolor{teal}{\mytextsc{verb}} \hspace{4pt} Tone: L\textsubscript{a}.
\textcolor{Sepia}{\selectlanguage{english}To drop, to fall, to sink (e.g. boat slowly sinking down into a lake).} \zh{掉入、沉下去。}  ¶ \textcolor{darkblue}{\textbf{\ipa{mv̩˩tɕo˧ dzi˩}}} \textcolor{Sepia}{\selectlanguage{english}to sink down} \zh{往下掉入、沉下去}  

\lhead{\firstmark}
\rhead{\botmark}

\subsection{\hspace{-0.5cm} {\Large \textcolor{darkblue}{\textbf{\ipa{dzi˩\textsubscript{b}}}}}\hspace{0.5cm}[\kern2pt{\textcolor{darkblue}{\textbf{\ipa{dzi˩˥}}}}\kern2pt]} \hypertarget{dzi\string_Bb1}{}
\markboth{\textcolor{darkblue}{\textbf{\ipa{dzi˩\textsubscript{b}}}}}{}
\textcolor{teal}{\mytextsc{classifier}} \hspace{4pt} Tone: L\textsubscript{b}.
\textcolor{Sepia}{\selectlanguage{english}Classifier for trees, bamboos….} \zh{量词:树(一棵),竹子(一根)。}  ¶ \textcolor{darkblue}{\textbf{\ipa{si˧dzi˩ | ɖɯ˧-dzi˩}}} \textcolor{Sepia}{\selectlanguage{english}a tree} \zh{一棵树}  
 ¶ \textcolor{darkblue}{\textbf{\ipa{tʰv̩˧-dzi˧˥}}} \textcolor{Sepia}{\selectlanguage{english}that tree} \zh{那棵树}  

\lhead{\firstmark}
\rhead{\botmark}

\subsection{\hspace{-0.5cm} {\Large \textcolor{darkblue}{\textbf{\ipa{dzi˩ʁo˩}}}}\hspace{0.5cm}[\kern2pt{\textcolor{darkblue}{\textbf{\ipa{dzi˧ʁo˧}}}}\kern2pt]} \hypertarget{dzi\string_BRo\string_B1}{}
\markboth{\textcolor{darkblue}{\textbf{\ipa{dzi˩ʁo˩}}}}{}
\textcolor{teal}{\mytextsc{noun}} \hspace{4pt} Tone: L.
\textcolor{Sepia}{\selectlanguage{english}Seat, place.} \zh{座位。}  \zh{量词}: \textcolor{darkblue}{\textbf{\ipa{kʰwɤ˥}}}  \mytextsc{clf}: \textcolor{darkblue}{\textbf{\ipa{kʰwɤ˥}}} 
\lhead{\firstmark}
\rhead{\botmark}

\subsection{\hspace{-0.5cm} {\Large \textcolor{darkblue}{\textbf{\ipa{dzi˩tsʰɤ˩}}}}\hspace{0.5cm}[\kern2pt{\textcolor{darkblue}{\textbf{\ipa{dzi˩tsʰɤ˩˥}}}}\kern2pt]} \hypertarget{dzi\string_Bts\string_h7\string_B1}{}
\markboth{\textcolor{darkblue}{\textbf{\ipa{dzi˩tsʰɤ˩}}}}{}
\textcolor{teal}{\mytextsc{noun}} \hspace{4pt} Tone: L.
\textcolor{Sepia}{\selectlanguage{english}A shrub with sharp thorns.} \zh{永宁的一种灌木。}  \zh{量词}: \textcolor{darkblue}{\textbf{\ipa{dzi˩}}}  \mytextsc{clf}: \textcolor{darkblue}{\textbf{\ipa{dzi˩}}} 
\lhead{\firstmark}
\rhead{\botmark}

\subsection{\hspace{-0.5cm} {\Large \textcolor{darkblue}{\textbf{\ipa{dzi˧˥}}}}\hspace{0.5cm}[\kern2pt{\textcolor{darkblue}{\textbf{\ipa{dzi˩˥}}}}\kern2pt]} \hypertarget{dzi\string_M\string_T1}{}
\markboth{\textcolor{darkblue}{\textbf{\ipa{dzi˧˥}}}}{}
\textcolor{teal}{\mytextsc{verb}} \hspace{4pt} Tone: MH.
\textcolor{Sepia}{\selectlanguage{english}To tremble, to shake.} \zh{颤抖、抖动。}  ¶ \textcolor{darkblue}{\textbf{\ipa{njɤ˩ dzi˧˥}}} \textcolor{Sepia}{\selectlanguage{english}the eyelid trembles (literally 'the eye trembles')} \zh{眼皮跳}  
 ¶ \textcolor{darkblue}{\textbf{\ipa{njɤ˩ dzi˧-ze˥}}} \textcolor{Sepia}{\selectlanguage{english}the eyelid trembles} \zh{眼皮跳}  
 ¶ \textcolor{darkblue}{\textbf{\ipa{njɤ˩ dzi˧˥ | ʐwæ˩˥}}} \textcolor{Sepia}{\selectlanguage{english}the eyelid trembles terribly} \zh{眼皮跳得很厉害}  
 ¶ \textcolor{darkblue}{\textbf{\ipa{njɤ˩˥ | le˧-dzi˧-ze˥}}} \textcolor{Sepia}{\selectlanguage{english}the eyelid trembles} \zh{眼皮跳}  
 ¶ \textcolor{darkblue}{\textbf{\ipa{njɤ˩˥ | mɤ˧-dzi˧˥}}} \textcolor{Sepia}{\selectlanguage{english}the eyelid does not tremble} \zh{眼皮不跳}  

\lhead{\firstmark}
\rhead{\botmark}

\subsection{\hspace{-0.5cm} {\Large \textcolor{darkblue}{\textbf{\ipa{dzo˥}}}}\hspace{0.5cm}[\kern2pt{\textcolor{darkblue}{\textbf{\ipa{dzo˩˥}}}}\kern2pt]} \hypertarget{dzo\string_T1}{}
\markboth{\textcolor{darkblue}{\textbf{\ipa{dzo˥}}}}{}
\textcolor{teal}{\mytextsc{noun}} \hspace{4pt} Tone: \#H.
\textcolor{Sepia}{\selectlanguage{english}Hail.} \zh{冰雹。}  ¶ \textcolor{darkblue}{\textbf{\ipa{dzo˧ lɑ˩}}} \textcolor{Sepia}{\selectlanguage{english}there is some hail} \zh{下冰雹}  
 ¶ \textcolor{darkblue}{\textbf{\ipa{dzo˧ gi˧-ze˩}}} \textcolor{Sepia}{\selectlanguage{english}there is some hail} \zh{下冰雹了}  

\lhead{\firstmark}
\rhead{\botmark}

\subsection{\hspace{-0.5cm} {\Large \textcolor{darkblue}{\textbf{\ipa{dzo˧-lv̩˧\textasciitilde{}lv̩˥}}}}\hspace{0.5cm}[\kern2pt{\textcolor{darkblue}{\textbf{\ipa{xxxx non-correspondance entre le nombre de morphèmes et le nombre de tons de morphèmes}}}}\kern2pt]} \hypertarget{dzo\string_M-lv\string_=\string_M~lv\string_=\string_T1}{}
\markboth{\textcolor{darkblue}{\textbf{\ipa{dzo˧-lv̩˧\textasciitilde{}lv̩˥}}}}{}
\textcolor{teal}{\mytextsc{noun}} \hspace{4pt} Tone: H\#.
\textcolor{Sepia}{\selectlanguage{english}Hailstone.} \zh{冰雹。}  \zh{量词}: \textcolor{darkblue}{\textbf{\ipa{ɭɯ˧}}}  \mytextsc{clf}: \textcolor{darkblue}{\textbf{\ipa{ɭɯ˧}}} 
\lhead{\firstmark}
\rhead{\botmark}

\subsection{\hspace{-0.5cm} {\Large \textcolor{darkblue}{\textbf{\ipa{dzo˧mi˧}}}}\hspace{0.5cm}[\kern2pt{\textcolor{darkblue}{\textbf{\ipa{dzo˧mi˥}}}}\kern2pt]} \hypertarget{dzo\string_Mmi\string_M1}{}
\markboth{\textcolor{darkblue}{\textbf{\ipa{dzo˧mi˧}}}}{}
\textcolor{teal}{\mytextsc{noun}} \hspace{4pt} Tone: M.
\textcolor{Sepia}{\selectlanguage{english}Large vat.} \zh{大桶。}  \zh{量词}: \textcolor{darkblue}{\textbf{\ipa{ɭɯ˧}}}  \mytextsc{clf}: \textcolor{darkblue}{\textbf{\ipa{ɭɯ˧}}} 
\lhead{\firstmark}
\rhead{\botmark}

\subsection{\hspace{-0.5cm} {\Large \textcolor{darkblue}{\textbf{\ipa{dzo˧zo\#˥}}}}\hspace{0.5cm}[\kern2pt{\textcolor{darkblue}{\textbf{\ipa{dzo˩zo˥}}}}\kern2pt]} \hypertarget{dzo\string_Mzo\#\string_T1}{}
\markboth{\textcolor{darkblue}{\textbf{\ipa{dzo˧zo\#˥}}}}{}
\textcolor{teal}{\mytextsc{noun}} \hspace{4pt} Tone: \#H.
\textcolor{Sepia}{\selectlanguage{english}Small vat.} \zh{小桶。}  \zh{量词}: \textcolor{darkblue}{\textbf{\ipa{ɭɯ˧}}}  \mytextsc{clf}: \textcolor{darkblue}{\textbf{\ipa{ɭɯ˧}}} 
\lhead{\firstmark}
\rhead{\botmark}

\subsection{\hspace{-0.5cm} {\Large \textcolor{darkblue}{\textbf{\ipa{dzo˩}}}}\hspace{0.5cm}[\kern2pt{\textcolor{darkblue}{\textbf{\ipa{dzo˥}}}}\kern2pt]} \hypertarget{dzo\string_B1}{}
\markboth{\textcolor{darkblue}{\textbf{\ipa{dzo˩}}}}{}
\textcolor{teal}{\mytextsc{noun}} \hspace{4pt} Tone: L.
\textcolor{Sepia}{\selectlanguage{english}Bridge.} \zh{桥。}  ¶ \textcolor{darkblue}{\textbf{\ipa{dzo˧ | ɖɯ˧ pɤ˩}}} \textcolor{Sepia}{\selectlanguage{english}a bridge} \zh{一辆桥}  
 ¶ \textcolor{darkblue}{\textbf{\ipa{dzo˩ bæ˩˥}}} \textcolor{Sepia}{\selectlanguage{english}to sweep (a/the) bridge} \zh{扫桥}  
 ¶ \textcolor{darkblue}{\textbf{\ipa{njɤ˧ | dzo˩ bæ˩-zo˩-ho˥.}}} \textcolor{Sepia}{\selectlanguage{english}I have to sweep the bridge.} \zh{我要扫桥了。}  
 ¶ \textcolor{darkblue}{\textbf{\ipa{dzo˩ gv̩˩˥}}} \textcolor{Sepia}{\selectlanguage{english}to build (/repair) a bridge} \zh{建一辆桥}  
 ¶ \textcolor{darkblue}{\textbf{\ipa{njɤ˧ | dzo˩ gv̩˩-zo˩-ho˥.}}} \textcolor{Sepia}{\selectlanguage{english}I have to build (/repair) a bridge.}  
 \zh{量词}: \textcolor{darkblue}{\textbf{\ipa{pɤ˩}}} \textcolor{darkblue}{\textbf{\ipa{nɑ˧}}}  \mytextsc{clf}: \textcolor{darkblue}{\textbf{\ipa{pɤ˩}}} \textcolor{darkblue}{\textbf{\ipa{nɑ˧}}} 
\lhead{\firstmark}
\rhead{\botmark}

\subsection{\hspace{-0.5cm} {\Large \textcolor{darkblue}{\textbf{\ipa{dzo˩\textasciitilde{}dzo˧˥}}}}\hspace{0.5cm}[\kern2pt{\textcolor{darkblue}{\textbf{\ipa{dzo˧dzo˧}}}}\kern2pt]} \hypertarget{dzo\string_B~dzo\string_M\string_T1}{}
\markboth{\textcolor{darkblue}{\textbf{\ipa{dzo˩\textasciitilde{}dzo˧˥}}}}{}
\textcolor{teal}{\mytextsc{verb}} \hspace{4pt} Tone: MH.
\textcolor{Sepia}{\selectlanguage{english}To laugh at, to poke fun at, to mock, to ridicule.} \zh{嘲笑、取笑。}  ¶ \textcolor{darkblue}{\textbf{\ipa{hĩ˧ dzo˧-dzo˥-ho˩-ni˩zo˩!}}} \textcolor{Sepia}{\selectlanguage{english}It looks like (he/she) is going to poke fun (at...)} \zh{他好像要开始取笑人家了!}  
 ¶ \textcolor{darkblue}{\textbf{\ipa{tʰɑ˧ dzo˩\textasciitilde{}dzo˩!}}} \textcolor{Sepia}{\selectlanguage{english}Don't laugh at people!} \zh{别嘲笑(人家)!}  

\lhead{\firstmark}
\rhead{\botmark}

\subsection{\hspace{-0.5cm} {\Large \textcolor{darkblue}{\textbf{\ipa{dzo˩mi\#˥}}}}\hspace{0.5cm}[\kern2pt{\textcolor{darkblue}{\textbf{\ipa{dzo˧mi˧}}}}\kern2pt]} \hypertarget{dzo\string_Bmi\#\string_T1}{}
\markboth{\textcolor{darkblue}{\textbf{\ipa{dzo˩mi\#˥}}}}{}
\textcolor{teal}{\mytextsc{noun}} \hspace{4pt} Tone: LM+\#H.
\textcolor{Sepia}{\selectlanguage{english}Female lizard.} \zh{母壁虎。}  ¶ \textcolor{darkblue}{\textbf{\ipa{dzo˧mi˧-dzo˩pʰv̩˩}}} \textcolor{Sepia}{\selectlanguage{english}female lizard and male lizard} \zh{母壁虎与公壁虎}  
 \zh{量词}: \textcolor{darkblue}{\textbf{\ipa{mi˩}}}  \mytextsc{clf}: \textcolor{darkblue}{\textbf{\ipa{mi˩}}} 
\lhead{\firstmark}
\rhead{\botmark}

\subsection{\hspace{-0.5cm} {\Large \textcolor{darkblue}{\textbf{\ipa{dzo˩pʰv̩˩}}}}\hspace{0.5cm}[\kern2pt{\textcolor{darkblue}{\textbf{\ipa{dzo˩pʰv̩˥}}}}\kern2pt]} \hypertarget{dzo\string_Bp\string_hv\string_=\string_B1}{}
\markboth{\textcolor{darkblue}{\textbf{\ipa{dzo˩pʰv̩˩}}}}{}
\textcolor{teal}{\mytextsc{noun}} \hspace{4pt} Tone: L.
\textcolor{Sepia}{\selectlanguage{english}Male lizard.} \zh{公壁虎。}  ¶ \textcolor{darkblue}{\textbf{\ipa{dzo˩pʰv̩˩-dzo˩mi˥}}} \textcolor{Sepia}{\selectlanguage{english}male lizard and female lizard} \zh{公壁虎与母壁虎}  
 \zh{量词}: \textcolor{darkblue}{\textbf{\ipa{mi˩}}} \textcolor{darkblue}{\textbf{\ipa{ɭɯ˧}}}  \mytextsc{clf}: \textcolor{darkblue}{\textbf{\ipa{mi˩}}} \textcolor{darkblue}{\textbf{\ipa{ɭɯ˧}}} 
\lhead{\firstmark}
\rhead{\botmark}

\subsection{\hspace{-0.5cm} {\Large \textcolor{darkblue}{\textbf{\ipa{dzo˩zo\#˥}}}}\hspace{0.5cm}[\kern2pt{\textcolor{darkblue}{\textbf{\ipa{dzo˩zo˥}}}}\kern2pt]} \hypertarget{dzo\string_Bzo\#\string_T1}{}
\markboth{\textcolor{darkblue}{\textbf{\ipa{dzo˩zo\#˥}}}}{}
\textcolor{teal}{\mytextsc{noun}} \hspace{4pt} Tone: LM+\#H.
\textcolor{Sepia}{\selectlanguage{english}Baby lizard.} \zh{小壁虎。}  \zh{量词}: \textcolor{darkblue}{\textbf{\ipa{mi˩}}} \textcolor{darkblue}{\textbf{\ipa{ɭɯ˧}}}  \mytextsc{clf}: \textcolor{darkblue}{\textbf{\ipa{mi˩}}} \textcolor{darkblue}{\textbf{\ipa{ɭɯ˧}}} 
\lhead{\firstmark}
\rhead{\botmark}

\subsection{\hspace{-0.5cm} {\Large \textcolor{darkblue}{\textbf{\ipa{dzo˩˧}}}}\hspace{0.5cm}[\kern2pt{\textcolor{darkblue}{\textbf{\ipa{dzo˥}}}}\kern2pt]} \hypertarget{dzo\string_B\string_M1}{}
\markboth{\textcolor{darkblue}{\textbf{\ipa{dzo˩˧}}}}{}
\textcolor{teal}{\mytextsc{noun}} \hspace{4pt} Tone: LM.
\textcolor{Sepia}{\selectlanguage{english}Lizard.} \zh{壁虎,蜥蜴,四脚蛇。}  ¶ \textcolor{darkblue}{\textbf{\ipa{dzo˩ hwæ˧-ze˧}}} \textcolor{Sepia}{\selectlanguage{english}...bought (a) lizard} \zh{买了壁虎}  
 ¶ \textcolor{darkblue}{\textbf{\ipa{dzo˩ dzɯ˧-ze˩}}} \textcolor{Sepia}{\selectlanguage{english}...ate (a) lizard} \zh{吃了壁虎}  
 \zh{量词}: \textcolor{darkblue}{\textbf{\ipa{mi˩}}}  \mytextsc{clf}: \textcolor{darkblue}{\textbf{\ipa{mi˩}}} 
\lhead{\firstmark}
\rhead{\botmark}

\subsection{\hspace{-0.5cm} {\Large \textcolor{darkblue}{\textbf{\ipa{dzɯ˥}}}}\hspace{0.5cm}[\kern2pt{\textcolor{darkblue}{\textbf{\ipa{dzɯ˧˥}}}}\kern2pt]} \hypertarget{dzM\string_T1}{}
\markboth{\textcolor{darkblue}{\textbf{\ipa{dzɯ˥}}}}{}
\textcolor{teal}{\mytextsc{verb}} \hspace{4pt} Tone: H.
\textcolor{Sepia}{\selectlanguage{english}To eat.} \zh{吃。}  ¶ \textcolor{darkblue}{\textbf{\ipa{le˧-dzɯ˥}}} \textcolor{Sepia}{\selectlanguage{english}\mytextsc{accomp}} \zh{\mytextsc{实施}}  
 ¶ \textcolor{darkblue}{\textbf{\ipa{hɑ˧ dzɯ˧}}} \textcolor{Sepia}{\selectlanguage{english}to have a meal; to take some food} \zh{吃饭}  
 ¶ \textcolor{darkblue}{\textbf{\ipa{njɤ˧ | hɑ˧ le˧-dzɯ˥-ze˩}}} \textcolor{Sepia}{\selectlanguage{english}I have eaten. / I have had a meal.} \zh{我吃过饭了。}  
 ¶ \textcolor{darkblue}{\textbf{\ipa{dzɯ˧-di˧˥}}} \textcolor{Sepia}{\selectlanguage{english}food, thing to eat} \zh{吃的(东西)}  
 ¶ \textcolor{darkblue}{\textbf{\ipa{dzɯ˧-bi˩-ze˩!}}} \textcolor{Sepia}{\selectlanguage{english}Let's eat! / It's time to eat!} \zh{要吃饭了!}  

\lhead{\firstmark}
\rhead{\botmark}

\subsection{\hspace{-0.5cm} {\Large \textcolor{darkblue}{\textbf{\ipa{dzɯ˧tsɯ˧˥}}}}\hspace{0.5cm}[\kern2pt{\textcolor{darkblue}{\textbf{\ipa{dzɯ˧tsɯ˧˥}}}}\kern2pt]} \hypertarget{dzM\string_MtsM\string_M\string_T1}{}
\markboth{\textcolor{darkblue}{\textbf{\ipa{dzɯ˧tsɯ˧˥}}}}{}
\textcolor{teal}{\mytextsc{noun}} \hspace{4pt} Tone: MH\#.
\textcolor{Sepia}{\selectlanguage{english}A shrub that grows in Yongning.} \zh{永宁的一种灌木。}  \zh{量词}: \textcolor{darkblue}{\textbf{\ipa{pʰæ˧˥}}}  \mytextsc{clf}: \textcolor{darkblue}{\textbf{\ipa{pʰæ˧˥}}} 
\lhead{\firstmark}
\rhead{\botmark}

\newpage
\section*{\centering- \textcolor{darkblue}{\textbf{\ipa{dʑ}}} -}
\subsection{\hspace{-0.5cm} {\Large \textcolor{darkblue}{\textbf{\ipa{dʑɤ˧bo˩}}}}\hspace{0.5cm}[\kern2pt{\textcolor{darkblue}{\textbf{\ipa{dʑɤ˩bo˩˥}}}}\kern2pt]} \hypertarget{dz£7\string_Mbo\string_B1}{}
\markboth{\textcolor{darkblue}{\textbf{\ipa{dʑɤ˧bo˩}}}}{}
\textcolor{teal}{\mytextsc{noun}} \hspace{4pt} Tone: L\#.
\textcolor{Sepia}{\selectlanguage{english}Granary (a special building).} \zh{粮仓。}  \zh{量词}: \textcolor{darkblue}{\textbf{\ipa{ɭɯ˧}}}  \mytextsc{clf}: \textcolor{darkblue}{\textbf{\ipa{ɭɯ˧}}} 
\lhead{\firstmark}
\rhead{\botmark}

\subsection{\hspace{-0.5cm} {\Large \textcolor{darkblue}{\textbf{\ipa{dʑɤ˧do˩}}}}\hspace{0.5cm}[\kern2pt{\textcolor{darkblue}{\textbf{\ipa{dʑɤ˩do˩˥}}}}\kern2pt]} \hypertarget{dz£7\string_Mdo\string_B1}{}
\markboth{\textcolor{darkblue}{\textbf{\ipa{dʑɤ˧do˩}}}}{}
\textcolor{teal}{\mytextsc{noun}} \hspace{4pt} Tone: L\#.
\textcolor{Sepia}{\selectlanguage{english}Zhongdian (place name).} \zh{中甸。}  ¶ \textcolor{darkblue}{\textbf{\ipa{dʑɤ˧do˩-bɤ˩}}} \textcolor{Sepia}{\selectlanguage{english}the Pumi people of Zhongdian} \zh{中甸普米族}  

\lhead{\firstmark}
\rhead{\botmark}

\subsection{\hspace{-0.5cm} {\Large \textcolor{darkblue}{\textbf{\ipa{dʑɤ˩\textsubscript{b}}}}}\hspace{0.5cm}[\kern2pt{\textcolor{darkblue}{\textbf{\ipa{dʑɤ˥}}}}\kern2pt]} \hypertarget{dz£7\string_Bb1}{}
\markboth{\textcolor{darkblue}{\textbf{\ipa{dʑɤ˩\textsubscript{b}}}}}{}
\textcolor{teal}{\mytextsc{adjective}} \hspace{4pt} Tone: L\textsubscript{b}.
\textcolor{Sepia}{\selectlanguage{english}Good (good decision).} \zh{好。}  ¶ \textcolor{darkblue}{\textbf{\ipa{mɤ˧-dʑɤ˩}}} \textcolor{Sepia}{\selectlanguage{english}bad} \zh{坏}  
 ¶ \textcolor{darkblue}{\textbf{\ipa{dʑɤ˩-hĩ˥}}} \textcolor{Sepia}{\selectlanguage{english}\mytextsc{rel}} \zh{\mytextsc{rel}}  
 ¶ \textcolor{darkblue}{\textbf{\ipa{(no˧) ɖwæ˧˥ | dʑɤ˩˥!}}} \textcolor{Sepia}{\selectlanguage{english}You're great!} \zh{你很好!}  
 ¶ \textcolor{darkblue}{\textbf{\ipa{dʑɤ˩-kʰɯ˥!}}} \textcolor{Sepia}{\selectlanguage{english}A benediction used on the New Year: “Let there be good (things)!”, i.e. “Prosperity!”, “All the best for the New Year!”} \zh{新年祝福:“祝一切好! / 万事如意!”}  
 ¶ \textcolor{darkblue}{\textbf{\ipa{no˧ | le˧-ʝi˥ | dʑɤ˩˥, | hĩ˧-ɳɯ˩ | do˩˥! | ʈʂʰɯ˧ | le˧-ʝi˥ | mɤ˧-dʑɤ˩, | hĩ˧-ɳɯ˩ | ʐwɤ˩˥!}}} \textcolor{Sepia}{\selectlanguage{english}“If you do well, people see it / people realize so! If (s)he does badly, people say so!” = “A job well done earns recognition; a job badly done earns criticism!” (Context: talking about a bad book. In the Na world view as remembered by the consultant, there is no question that it is better to do good, that good deeds and good attitudes eventually get rewarded, and bad deeds and bad attitudes eventually get punished.} \zh{你做得好,人家(会)发现!他做的不好,人家(会)说(他)!}  

\lhead{\firstmark}
\rhead{\botmark}

\subsection{\hspace{-0.5cm} {\Large \textcolor{darkblue}{\textbf{\ipa{dʑɤ˩bv̩˥}}}}\hspace{0.5cm}[\kern2pt{\textcolor{darkblue}{\textbf{\ipa{dʑɤ˧bv̩˩}}}}\kern2pt]} \hypertarget{dz£7\string_Bbv\string_=\string_T1}{}
\markboth{\textcolor{darkblue}{\textbf{\ipa{dʑɤ˩bv̩˥}}}}{}
\textcolor{teal}{\mytextsc{verb}} \hspace{4pt} Tone: LH.
\textcolor{Sepia}{\selectlanguage{english}To play.} \zh{玩,玩耍。}  ¶ \textcolor{darkblue}{\textbf{\ipa{dʑɤ˩bv̩˥ -bi˩/-ze˩}}} \textcolor{Sepia}{\selectlanguage{english}\string_ \mytextsc{fut}.imm/\mytextsc{pfv}} \zh{要玩耍 / 玩耍了}  
 ¶ \textcolor{darkblue}{\textbf{\ipa{le˧-dʑɤ˩bv̩˩ +ze˩}}} \textcolor{Sepia}{\selectlanguage{english}\mytextsc{accomp} \string_ \mytextsc{pfv}} \zh{玩耍了}  
 ¶ \textcolor{darkblue}{\textbf{\ipa{tʰi˧-dʑɤ˩bv̩˩, | tʰi˧-dʑɤ˩bv̩˩, | le˧-fv̩˧!}}} \textcolor{Sepia}{\selectlanguage{english}They play, they play... they're happy! / (By) playing on and on, they get really happy! (About children playing together)} \zh{他们玩着玩着,很高兴!(情景:几个小孩子一起玩)}  

\lhead{\firstmark}
\rhead{\botmark}

\subsection{\hspace{-0.5cm} {\Large \textcolor{darkblue}{\textbf{\ipa{dʑɤ˩bv̩˥-di˩}}}}\hspace{0.5cm}[\kern2pt{\textcolor{darkblue}{\textbf{\ipa{xxxx non-correspondance entre le nombre de morphèmes et le nombre de tons de morphèmes}}}}\kern2pt]} \hypertarget{dz£7\string_Bbv\string_=\string_T-di\string_B1}{}
\markboth{\textcolor{darkblue}{\textbf{\ipa{dʑɤ˩bv̩˥-di˩}}}}{}
\textcolor{teal}{\mytextsc{noun}} \hspace{4pt} Tone: LH-.
\textcolor{Sepia}{\selectlanguage{english}Toy.} \zh{玩具。} 
\lhead{\firstmark}
\rhead{\botmark}

\subsection{\hspace{-0.5cm} {\Large \textcolor{darkblue}{\textbf{\ipa{dʑɤ˩bv̩˧kɤ˧-sɑ˥ʁwɤ˩}}}}\hspace{0.5cm}[\kern2pt{\textcolor{darkblue}{\textbf{\ipa{dʑɤ˩bv̩˩kɤ˥sɑ˧ʁwɤ˧}}}}\kern2pt]} \hypertarget{dz£7\string_Bbv\string_=\string_Mk7\string_M-sA\string_TRw7\string_B1}{}
\markboth{\textcolor{darkblue}{\textbf{\ipa{dʑɤ˩bv̩˧kɤ˧-sɑ˥ʁwɤ˩}}}}{}
\textcolor{teal}{\mytextsc{noun}} \hspace{4pt} Tone: LM+MH\#-.
\textcolor{Sepia}{\selectlanguage{english}Gaoming, a village north-east of Yongning).} \zh{高明 (摩梭话名称的汉译:嘎撒瓦)(永宁的一个村落)。}  ¶ \textcolor{darkblue}{\textbf{\ipa{dʑɤ˩bv̩˧kɤ˧-sɑ˥ʁwɤ˩, | hi˩ʁwɤ˩-lo˥, | æ˩mi˧-ʁwɤ\#˥, | lɑ˧lo˧-ʁwɤ˥, | lɑ˧ŋwɤ˧, | bɤ˧tsʰo˧gv̩˥, | ə˧lɑ˧-ʁwɤ\#˥, | gæ˧ɻæ˩, | qʰæ˧tɕʰi˧, | tʰo˧ʈɯ\#˥}}} \textcolor{Sepia}{\selectlanguage{english}the ten villages traditionally considered as part of Yongning} \zh{摩梭传统地理概念中,属于永宁的十个村落}  

\lhead{\firstmark}
\rhead{\botmark}

\subsection{\hspace{-0.5cm} {\Large \textcolor{darkblue}{\textbf{\ipa{dʑɤ˩bv̩˥-ʁwɤ˩}}}}\hspace{0.5cm}[\kern2pt{\textcolor{darkblue}{\textbf{\ipa{dʑɤ˩bv̩˧˥ʁwɤ˧}}}}\kern2pt]} \hypertarget{dz£7\string_Bbv\string_=\string_T-Rw7\string_B1}{}
\markboth{\textcolor{darkblue}{\textbf{\ipa{dʑɤ˩bv̩˥-ʁwɤ˩}}}}{}
\textcolor{teal}{\mytextsc{noun}} \hspace{4pt} Tone: LH-.
\textcolor{Sepia}{\selectlanguage{english}Jiabowa (name of a village).} \zh{甲波瓦(永宁的一个村落)。} 
\lhead{\firstmark}
\rhead{\botmark}

\subsection{\hspace{-0.5cm} {\Large \textcolor{darkblue}{\textbf{\ipa{dʑɤ˩ɕjɤ˩}}}}\hspace{0.5cm}[\kern2pt{\textcolor{darkblue}{\textbf{\ipa{xxxx non-correspondance entre le nombre de morphèmes et le nombre de tons de morphèmes}}}}\kern2pt]} \hypertarget{dz£7\string_Bs£j7\string_B1}{}
\markboth{\textcolor{darkblue}{\textbf{\ipa{dʑɤ˩ɕjɤ˩}}}}{}
\textcolor{teal}{\mytextsc{noun}} \hspace{4pt} Tone: L.
\textcolor{Sepia}{\selectlanguage{english}Shoe-pad; insole.} \zh{鞋垫。} 
\lhead{\firstmark}
\rhead{\botmark}

\subsection{\hspace{-0.5cm} {\Large \textcolor{darkblue}{\textbf{\ipa{dʑɤ˩kʰwɤ˧}}}}\hspace{0.5cm}[\kern2pt{\textcolor{darkblue}{\textbf{\ipa{xxxx ton non trouvé, à faire manuellement...}}}}\kern2pt]} \hypertarget{dz£7\string_Bk\string_hw7\string_M1}{}
\markboth{\textcolor{darkblue}{\textbf{\ipa{dʑɤ˩kʰwɤ˧}}}}{}
\textcolor{teal}{\mytextsc{noun}} \hspace{4pt} Tone: LM.
\textcolor{Sepia}{\selectlanguage{english}Distance.} \zh{距离。}  ¶ \textcolor{darkblue}{\textbf{\ipa{no˧ | ʈʂʰɯ˧ | ə˩-ʐæ˥ʂæ˩? | dʑɤ˩kʰwɤ˧ ə˩-di˩? | - dʑɤ˩˥ | dʑɤ˩kʰwɤ˧ mɤ˧-di˥! | mɤ˧-ʐæ˩ʂæ˩!}}} \textcolor{Sepia}{\selectlanguage{english}Are you distant from him? Is there distance (between you)? - There is not much distance to speak of! We are not distant! (=we are close friends)} \zh{你们很熟吗?/ 你们很亲吗? - 不很熟!/ 不很亲!}  

\lhead{\firstmark}
\rhead{\botmark}

\subsection{\hspace{-0.5cm} {\Large \textcolor{darkblue}{\textbf{\ipa{dʑɤ˩pi\#˥}}}}\hspace{0.5cm}[\kern2pt{\textcolor{darkblue}{\textbf{\ipa{dʑɤ˩pi˥}}}}\kern2pt]} \hypertarget{dz£7\string_Bpi\#\string_T1}{}
\markboth{\textcolor{darkblue}{\textbf{\ipa{dʑɤ˩pi\#˥}}}}{}
\textcolor{teal}{\mytextsc{adjective}} \hspace{4pt} Tone: LM+\#H.
\textcolor{Sepia}{\selectlanguage{english}Plenty of.} \zh{多。}  ¶ \textcolor{darkblue}{\textbf{\ipa{dʑɤ˩pi˧ ʝi˧}}} \textcolor{Sepia}{\selectlanguage{english}It's very useful!} \zh{很有用、有很多用处}  
 ¶ \textcolor{darkblue}{\textbf{\ipa{dʑɤ˩pi˧ dʑo˧!}}} \textcolor{Sepia}{\selectlanguage{english}(I) have plenty! / (I) have a lot! (possession)} \zh{我有很多!}  
 ¶ \textcolor{darkblue}{\textbf{\ipa{dʑɤ˩pi˧ dʑo˧˥}}} \textcolor{Sepia}{\selectlanguage{english}There is plenty / there is a lot. (Note: existence, and not possession)} \zh{有很多。}  

\lhead{\firstmark}
\rhead{\botmark}

\subsection{\hspace{-0.5cm} {\Large \textcolor{darkblue}{\textbf{\ipa{dʑɤ˩qʰɑ˥}}}}\hspace{0.5cm}[\kern2pt{\textcolor{darkblue}{\textbf{\ipa{dʑɤ˩qʰɑ˩˥}}}}\kern2pt]} \hypertarget{dz£7\string_Bq\string_hA\string_T1}{}
\markboth{\textcolor{darkblue}{\textbf{\ipa{dʑɤ˩qʰɑ˥}}}}{}
\textcolor{teal}{\mytextsc{adverb(ial)}} \hspace{4pt} Tone: LH.
\textcolor{Sepia}{\selectlanguage{english}Continuously, with full might.} \zh{一直、一个劲地。}  ¶ \textcolor{darkblue}{\textbf{\ipa{dʑɤ˩qʰɑ˥ ʈɤ˩}}} \textcolor{Sepia}{\selectlanguage{english}to pull with full might} \zh{一个劲地拉}  
 ¶ \textcolor{darkblue}{\textbf{\ipa{dʑɤ˩qʰɑ˥ mi˩}}} \textcolor{Sepia}{\selectlanguage{english}to push with full might} \zh{一个劲地推}  
 ¶ \textcolor{darkblue}{\textbf{\ipa{dʑɤ˩qʰɑ˥ lɑ˩}}} \textcolor{Sepia}{\selectlanguage{english}to beat with full might} \zh{一个劲地打}  

\lhead{\firstmark}
\rhead{\botmark}

\subsection{\hspace{-0.5cm} {\Large \textcolor{darkblue}{\textbf{\ipa{dʑɤ˩so˧}}}}\hspace{0.5cm}[\kern2pt{\textcolor{darkblue}{\textbf{\ipa{dʑɤ˧so˧}}}}\kern2pt]} \hypertarget{dz£7\string_Bso\string_M1}{}
\markboth{\textcolor{darkblue}{\textbf{\ipa{dʑɤ˩so˧}}}}{}
\textcolor{teal}{\mytextsc{adjective}} \hspace{4pt} Tone: LM.
\textcolor{Sepia}{\selectlanguage{english}Many, a great number of.} \zh{好几(个)。}  ¶ \textcolor{darkblue}{\textbf{\ipa{dʑɤ˩-so˧ ɲi˧}}} \textcolor{Sepia}{\selectlanguage{english}many days; a long time} \zh{好几天}  

\lhead{\firstmark}
\rhead{\botmark}

\subsection{\hspace{-0.5cm} {\Large \textcolor{darkblue}{\textbf{\ipa{dʑɤ˩tsʰi\#˥}}}}\hspace{0.5cm}[\kern2pt{\textcolor{darkblue}{\textbf{\ipa{xxxx non-correspondance entre le nombre de morphèmes et le nombre de tons de morphèmes}}}}\kern2pt]} \hypertarget{dz£7\string_Bts\string_hi\#\string_T1}{}
\markboth{\textcolor{darkblue}{\textbf{\ipa{dʑɤ˩tsʰi\#˥}}}}{}
\textcolor{teal}{\mytextsc{noun}} \hspace{4pt} Tone: LM+\#H.
\textcolor{Sepia}{\selectlanguage{english}Masculine given name.} \zh{男性名字。} 
\lhead{\firstmark}
\rhead{\botmark}

\subsection{\hspace{-0.5cm} {\Large \textcolor{darkblue}{\textbf{\ipa{dʑɤ˩tsʰi˧-ɖɯ˩mɑ˩}}}}\hspace{0.5cm}[\kern2pt{\textcolor{darkblue}{\textbf{\ipa{xxxx non-correspondance entre le nombre de morphèmes et le nombre de tons de morphèmes}}}}\kern2pt]} \hypertarget{dz£7\string_Bts\string_hi\string_M-d`M\string_BmA\string_B1}{}
\markboth{\textcolor{darkblue}{\textbf{\ipa{dʑɤ˩tsʰi˧-ɖɯ˩mɑ˩}}}}{}
\textcolor{teal}{\mytextsc{noun}} \hspace{4pt} Tone: LM-L.
\textcolor{Sepia}{\selectlanguage{english}Feminine given name.} \zh{女性名字。} 
\lhead{\firstmark}
\rhead{\botmark}

\subsection{\hspace{-0.5cm} {\Large \textcolor{darkblue}{\textbf{\ipa{dʑɤ˩tsʰi˧-tsi˩mv̩˩}}}}\hspace{0.5cm}[\kern2pt{\textcolor{darkblue}{\textbf{\ipa{dʑɤ˩tsʰi˧tsi˩mv̩˩}}}}\kern2pt]} \hypertarget{dz£7\string_Bts\string_hi\string_M-tsi\string_Bmv\string_=\string_B1}{}
\markboth{\textcolor{darkblue}{\textbf{\ipa{dʑɤ˩tsʰi˧-tsi˩mv̩˩}}}}{}
\textcolor{teal}{\mytextsc{noun}} \hspace{4pt} Tone: LM-L.
\textcolor{Sepia}{\selectlanguage{english}Prayer flag.} \zh{经幡、风马旗(挂在家旁边的树上,或房顶上)。}  \zh{量词}: \textcolor{darkblue}{\textbf{\ipa{dzi˩}}}  \mytextsc{clf}: \textcolor{darkblue}{\textbf{\ipa{dzi˩}}} 
\lhead{\firstmark}
\rhead{\botmark}

\subsection{\hspace{-0.5cm} {\Large \textcolor{darkblue}{\textbf{\ipa{dʑɤ˩tsʰo˧}}}}\hspace{0.5cm}[\kern2pt{\textcolor{darkblue}{\textbf{\ipa{dʑɤ˩tsʰo˥}}}}\kern2pt]} \hypertarget{dz£7\string_Bts\string_ho\string_M1}{}
\markboth{\textcolor{darkblue}{\textbf{\ipa{dʑɤ˩tsʰo˧}}}}{}
\textcolor{teal}{\mytextsc{verb}} \hspace{4pt} Tone: LM.
\textcolor{Sepia}{\selectlanguage{english}To dance.} \zh{跳舞。}  ¶ \textcolor{darkblue}{\textbf{\ipa{ʈʂʰɯ˧ | dʑɤ˩tsʰo˧ mɤ˧-dʑɤ˩!}}} \textcolor{Sepia}{\selectlanguage{english}(S)he does not dance well!} \zh{他舞跳得不好!}  
 ¶ \textcolor{darkblue}{\textbf{\ipa{dʑɤ˩tsʰo˧ | ɖɯ˧-hɑ̃˧ tsʰo˧}}} \textcolor{Sepia}{\selectlanguage{english}to dance all evening, to dance a whole night} \zh{跳一整夜舞}  
 ¶ \textcolor{darkblue}{\textbf{\ipa{ʑi˧qʰwɤ˧-ʂɯ˧-qo˧ | dʑɤ˩tsʰo˧ ʁɑ˧ ʂe˩}}} \textcolor{Sepia}{\selectlanguage{english}to throw a housewarming party in a newly built house} \zh{在新房子举办乔迁宴会}  

\lhead{\firstmark}
\rhead{\botmark}

\subsection{\hspace{-0.5cm} {\Large \textcolor{darkblue}{\textbf{\ipa{dʑi˥}}} \textsubscript{1}}\hspace{0.5cm}[\kern2pt{\textcolor{darkblue}{\textbf{\ipa{dʑi˥}}}}\kern2pt]} \hypertarget{dz£i\string_T1}{}
\markboth{\textcolor{darkblue}{\textbf{\ipa{dʑi˥}}} \textsubscript{1}}{}
\textcolor{teal}{\mytextsc{noun}} \hspace{4pt} Tone: \#H.
\textcolor{Sepia}{\selectlanguage{english}Urine.} \zh{尿。}  ¶ \textcolor{darkblue}{\textbf{\ipa{dʑi˧ bæ˥}}} \textcolor{Sepia}{\selectlanguage{english}to sweep urine} \zh{扫尿}  
 ¶ \textcolor{darkblue}{\textbf{\ipa{dʑi˧-lɑ˩ | qʰæ˧}}} \textcolor{Sepia}{\selectlanguage{english}excrements: urine and faeces} \zh{大小便的统称}  

\lhead{\firstmark}
\rhead{\botmark}

\subsection{\hspace{-0.5cm} {\Large \textcolor{darkblue}{\textbf{\ipa{dʑi˥}}} \textsubscript{2}}\hspace{0.5cm}[\kern2pt{\textcolor{darkblue}{\textbf{\ipa{dʑi˥}}}}\kern2pt]} \hypertarget{dz£i\string_T2}{}
\markboth{\textcolor{darkblue}{\textbf{\ipa{dʑi˥}}} \textsubscript{2}}{}
\textcolor{teal}{\mytextsc{noun}} \hspace{4pt} Tone: H.
\textcolor{Sepia}{\selectlanguage{english}Clothes, clothing (monosyllabic form).} \zh{衣服。}  ¶ \textcolor{darkblue}{\textbf{\ipa{kʰv̩˧ʂɯ˥, | dʑi˧ qæ˧!}}} \textcolor{Sepia}{\selectlanguage{english}On New Year's Eve, one changes one's clothing / one wears new clothes!} \zh{过年,换衣服! / 过年,要穿新衣服!}  
 ¶ \textcolor{darkblue}{\textbf{\ipa{dʑi˧ qæ˧-ze˩!}}} \textcolor{Sepia}{\selectlanguage{english}(He/she) has changed clothes!} \zh{换衣服了!}  
 ¶ \textcolor{darkblue}{\textbf{\ipa{nɑ˩-dʑi\#˥}}} \textcolor{Sepia}{\selectlanguage{english}Na clothing} \zh{摩梭服装}  
 ¶ \textcolor{darkblue}{\textbf{\ipa{hæ˧-dʑi\#˥}}} \textcolor{Sepia}{\selectlanguage{english}Chinese (Han) clothing} \zh{汉族服装}  
 \zh{量词}: \textcolor{darkblue}{\textbf{\ipa{ɭɯ˧}}}  \mytextsc{clf}: \textcolor{darkblue}{\textbf{\ipa{ɭɯ˧}}} 
\lhead{\firstmark}
\rhead{\botmark}

\subsection{\hspace{-0.5cm} {\Large \textcolor{darkblue}{\textbf{\ipa{dʑi˧hṽ˥\$}}}}\hspace{0.5cm}[\kern2pt{\textcolor{darkblue}{\textbf{\ipa{dʑi˧hṽ˧˥}}}}\kern2pt]} \hypertarget{dz£i\string_Mhv\string_~\string_T\$1}{}
\markboth{\textcolor{darkblue}{\textbf{\ipa{dʑi˧hṽ˥\$}}}}{}
\textcolor{teal}{\mytextsc{noun}} \hspace{4pt} Tone: H\$.
\textcolor{Sepia}{\selectlanguage{english}Clothes, clothing.} \zh{衣服。}  \zh{量词}: \textcolor{darkblue}{\textbf{\ipa{ɭɯ˧}}}  \mytextsc{clf}: \textcolor{darkblue}{\textbf{\ipa{ɭɯ˧}}} 
\lhead{\firstmark}
\rhead{\botmark}

\subsection{\hspace{-0.5cm} {\Large \textcolor{darkblue}{\textbf{\ipa{dʑi˧mi\#˥}}}}\hspace{0.5cm}[\kern2pt{\textcolor{darkblue}{\textbf{\ipa{dʑi˧mi˥}}}}\kern2pt]} \hypertarget{dz£i\string_Mmi\#\string_T1}{}
\markboth{\textcolor{darkblue}{\textbf{\ipa{dʑi˧mi\#˥}}}}{}
\textcolor{teal}{\mytextsc{noun}} \hspace{4pt} Tone: \#H.
\textcolor{Sepia}{\selectlanguage{english}Female water buffalo.} \zh{母水牛。}  ¶ \textcolor{darkblue}{\textbf{\ipa{dʑi˧mi˧ tʰv̩˧-pʰo˩}}} \textcolor{Sepia}{\selectlanguage{english}\mytextsc{n}+\mytextsc{dem}+\mytextsc{clf}} \zh{这头母水牛}  
 ¶ \textcolor{darkblue}{\textbf{\ipa{dʑi˧mi˧-dʑi˧zo\#˥ / dʑi˧mi˧-dʑi˥zo˩}}} \textcolor{Sepia}{\selectlanguage{english}female water buffalo and male water buffalo} \zh{母水牛与公水牛}  
 \zh{量词}: \textcolor{darkblue}{\textbf{\ipa{pʰo˧˥}}}  \mytextsc{clf}: \textcolor{darkblue}{\textbf{\ipa{pʰo˧˥}}} 
\lhead{\firstmark}
\rhead{\botmark}

\subsection{\hspace{-0.5cm} {\Large \textcolor{darkblue}{\textbf{\ipa{dʑi˧zo\#˥}}}}\hspace{0.5cm}[\kern2pt{\textcolor{darkblue}{\textbf{\ipa{dʑi˩zo˩˥}}}}\kern2pt]} \hypertarget{dz£i\string_Mzo\#\string_T1}{}
\markboth{\textcolor{darkblue}{\textbf{\ipa{dʑi˧zo\#˥}}}}{}
\textcolor{teal}{\mytextsc{noun}} \hspace{4pt} Tone: \#H.
\textcolor{Sepia}{\selectlanguage{english}Male baby buffalo.} \zh{小水牛(水牛崽子),一般指公的小水牛。}  ¶ \textcolor{darkblue}{\textbf{\ipa{dʑi˧zo˧ tʰv̩˧-ɭɯ\#˥}}} \textcolor{Sepia}{\selectlanguage{english}\mytextsc{n}+\mytextsc{dem}+\mytextsc{clf}} \zh{这只水牛崽子}  
 \zh{量词}: \textcolor{darkblue}{\textbf{\ipa{ɭɯ˧}}}  \mytextsc{clf}: \textcolor{darkblue}{\textbf{\ipa{ɭɯ˧}}} 
\lhead{\firstmark}
\rhead{\botmark}

\subsection{\hspace{-0.5cm} {\Large \textcolor{darkblue}{\textbf{\ipa{dʑi˩wɤ˩}}}}\hspace{0.5cm}[\kern2pt{\textcolor{darkblue}{\textbf{\ipa{dʑi˩wɤ˩˥}}}}\kern2pt]} \hypertarget{dz£i\string_Bw7\string_B1}{}
\markboth{\textcolor{darkblue}{\textbf{\ipa{dʑi˩wɤ˩}}}}{}
\textcolor{teal}{\mytextsc{noun}} \hspace{4pt} Tone: L.
\textcolor{Sepia}{\selectlanguage{english}Stirrup.} \zh{马镫。}  \zh{量词}: \textcolor{darkblue}{\textbf{\ipa{dze˩}}}  \mytextsc{clf}: \textcolor{darkblue}{\textbf{\ipa{dze˩}}} 
\lhead{\firstmark}
\rhead{\botmark}

\subsection{\hspace{-0.5cm} {\Large \textcolor{darkblue}{\textbf{\ipa{‑dʑo˥}}}}\hspace{0.5cm}[\kern2pt{\textcolor{darkblue}{\textbf{\ipa{dʑo˥}}}}\kern2pt]} \hypertarget{‑dz£o\string_T1}{}
\markboth{\textcolor{darkblue}{\textbf{\ipa{‑dʑo˥}}}}{}
\textcolor{teal}{\mytextsc{suffix}} \hspace{4pt} Tone: H.
\textcolor{Sepia}{\selectlanguage{english}Topic marker.} \zh{\mytextsc{主题。}} 
\lhead{\firstmark}
\rhead{\botmark}

\subsection{\hspace{-0.5cm} {\Large \textcolor{darkblue}{\textbf{\ipa{‑dʑo˧}}}}\hspace{0.5cm}[\kern2pt{\textcolor{darkblue}{\textbf{\ipa{dʑo˥}}}}\kern2pt]} \hypertarget{‑dz£o\string_M1}{}
\markboth{\textcolor{darkblue}{\textbf{\ipa{‑dʑo˧}}}}{}
\textcolor{teal}{\mytextsc{suffix}} \hspace{4pt} Tone: M.
\textcolor{Sepia}{\selectlanguage{english}Progressive aspect.} \zh{\mytextsc{进行式。}} 
\lhead{\firstmark}
\rhead{\botmark}

\subsection{\hspace{-0.5cm} {\Large \textcolor{darkblue}{\textbf{\ipa{dʑo˧\textsubscript{b}}}}}\hspace{0.5cm}[\kern2pt{\textcolor{darkblue}{\textbf{\ipa{dʑo˩˥}}}}\kern2pt]} \hypertarget{dz£o\string_Mb1}{}
\markboth{\textcolor{darkblue}{\textbf{\ipa{dʑo˧\textsubscript{b}}}}}{}
\textcolor{teal}{\mytextsc{verb}} \hspace{4pt} Tone: M\textsubscript{b}.
\textcolor{Sepia}{\selectlanguage{english}To possess.} \zh{有,拥有。}  ¶ \textcolor{darkblue}{\textbf{\ipa{mɤ˧-dʑo˧-ze˧!}}} \textcolor{Sepia}{\selectlanguage{english}There isn't any left!} \zh{没有了!}  
 ¶ \textcolor{darkblue}{\textbf{\ipa{le˧-dʑo˧-ze˧!}}} \textcolor{Sepia}{\selectlanguage{english}There is some, now!} \zh{有了!}  
 ¶ \textcolor{darkblue}{\textbf{\ipa{ʈʂʰɯ˧ | ɑ˩ʁo˧ | ɖɯ˧-sɑ˥ | mɤ˧-dʑo˧!}}} \textcolor{Sepia}{\selectlanguage{english}At his home, they have nothing at all = he is needy} \zh{他家什么也没有 = 他家贫穷}  
 ¶ \textcolor{darkblue}{\textbf{\ipa{njɤ˧ | mv̩˩zɯ˩-ni˥mi˩ | ŋi˧-kv̩˧ dʑo˧˥!}}} \textcolor{Sepia}{\selectlanguage{english}I have two siblings!} \zh{我有两个兄弟姐妹!}  
 ¶ \textcolor{darkblue}{\textbf{\ipa{dʑo˧-tʰɑ˧˥!}}} \textcolor{Sepia}{\selectlanguage{english}There could be some! / It's possible that there will be some!} \zh{会有的!}  
 ¶ \textcolor{darkblue}{\textbf{\ipa{tso˧\textasciitilde{}tso˧ dʑo˧}}} \textcolor{Sepia}{\selectlanguage{english}he has some things} \zh{他有东西}  
 ¶ \textcolor{darkblue}{\textbf{\ipa{njɤ˧-ɻ̍˩, | ɖɯ˧-ɭɯ˧-lɑ˧ dʑo˥!}}} \textcolor{Sepia}{\selectlanguage{english}We only have one (child)!} \zh{我们只有一个(孩子)!}  
 ¶ \textcolor{darkblue}{\textbf{\ipa{ɖwæ˧˥ | dʑo˧-ɲi˥!}}} \textcolor{Sepia}{\selectlanguage{english}There are lots! (For instance, when preparing to build a house, one needs large quantities of lumber; someone may comment: “There are lots!”)} \zh{有很多!(如:准备建房,积累的木材有很多)}  

\lhead{\firstmark}
\rhead{\botmark}

\subsection{\hspace{-0.5cm} {\Large \textcolor{darkblue}{\textbf{\ipa{dʑo˩\textsubscript{b}}}}}\hspace{0.5cm}[\kern2pt{\textcolor{darkblue}{\textbf{\ipa{dʑo˥}}}}\kern2pt]} \hypertarget{dz£o\string_Bb1}{}
\markboth{\textcolor{darkblue}{\textbf{\ipa{dʑo˩\textsubscript{b}}}}}{}
\textcolor{teal}{\mytextsc{verb}} \hspace{4pt} Tone: L\textsubscript{b}.
\textcolor{Sepia}{\selectlanguage{english}Existential: there is someone in the toilets/at home; there are n people in the family.} \zh{存在动词:有,存在着。如:某人在家或家里有几口人。}  ¶ \textcolor{darkblue}{\textbf{\ipa{mɤ˧-dʑo˩}}} \textcolor{Sepia}{\selectlanguage{english}\mytextsc{neg}} \zh{没有、不在}  
 ¶ \textcolor{darkblue}{\textbf{\ipa{ʈʂʰɯ˧ | ɑ˩ʁo˧ mɤ˧-dʑo˩!}}} \textcolor{Sepia}{\selectlanguage{english}(S)he is not at home!} \zh{他不在家!}  

\lhead{\firstmark}
\rhead{\botmark}

\subsection{\hspace{-0.5cm} {\Large \textcolor{darkblue}{\textbf{\ipa{‑dʑɯ˧}}}}\hspace{0.5cm}[\kern2pt{\textcolor{darkblue}{\textbf{\ipa{dʑɯ˥}}}}\kern2pt]} \hypertarget{‑dz£M\string_M1}{}
\markboth{\textcolor{darkblue}{\textbf{\ipa{‑dʑɯ˧}}}}{}
\textcolor{teal}{\mytextsc{suffix}} \hspace{4pt} Tone: M.
\textcolor{Sepia}{\selectlanguage{english}\mytextsc{experiential}.} \zh{……过。} 
\lhead{\firstmark}
\rhead{\botmark}

\subsection{\hspace{-0.5cm} {\Large \textcolor{darkblue}{\textbf{\ipa{dʑɯ˧}}}}\hspace{0.5cm}[\kern2pt{\textcolor{darkblue}{\textbf{\ipa{dʑɯ˥}}}}\kern2pt]} \hypertarget{dz£M\string_M1}{}
\markboth{\textcolor{darkblue}{\textbf{\ipa{dʑɯ˧}}}}{}
\textcolor{teal}{\mytextsc{noun}} \hspace{4pt} Tone: M.
\textcolor{Sepia}{\selectlanguage{english}Moment, time (of a certain event).} \zh{……的时间。}  ¶ \textcolor{darkblue}{\textbf{\ipa{ʈʂʰwɤ˩ dzɯ˩-bi˩-dʑɯ˩˥}}} \textcolor{Sepia}{\selectlanguage{english}dinner-time} \zh{吃晚餐的时间}  
 ¶ \textcolor{darkblue}{\textbf{\ipa{ɑ˩pʰo˩ bi˩-dʑɯ˩˥}}} \textcolor{Sepia}{\selectlanguage{english}time to go out; the right time to go outside} \zh{出去的(合适)时间}  
 ¶ \textcolor{darkblue}{\textbf{\ipa{le˧-ʑi˧-bi˧-dʑɯ˧ tʰv̩˧-ze˩!}}} \textcolor{Sepia}{\selectlanguage{english}It is time to go to sleep!} \zh{睡觉的时间到了!}  
 ¶ \textcolor{darkblue}{\textbf{\ipa{ʐo˩ dzɯ˩-bi˩-dʑɯ˩˥}}} \textcolor{Sepia}{\selectlanguage{english}lunch-time} \zh{午饭的时间}  

\lhead{\firstmark}
\rhead{\botmark}

\subsection{\hspace{-0.5cm} {\Large \textcolor{darkblue}{\textbf{\ipa{dʑɯ˧dv̩˧}}}}\hspace{0.5cm}[\kern2pt{\textcolor{darkblue}{\textbf{\ipa{xxxx non-correspondance entre le nombre de morphèmes et le nombre de tons de morphèmes}}}}\kern2pt]} \hypertarget{dz£M\string_Mdv\string_=\string_M1}{}
\markboth{\textcolor{darkblue}{\textbf{\ipa{dʑɯ˧dv̩˧}}}}{}
\textcolor{teal}{\mytextsc{noun}} \hspace{4pt} Tone: M.
\textcolor{Sepia}{\selectlanguage{english}Earthworm.} \zh{蚯蚓。}  ¶ \textcolor{darkblue}{\textbf{\ipa{dʑɯ˧dv̩˧-mi˩, | ə˩-dʑo˩˥?}}} \textcolor{Sepia}{\selectlanguage{english}Do female earthworms exist? (An artificially designed question, so as to elicit a form of 'earthworm' with the 'female' suffix, with a view to understanding the synchronically productive tone assignment rules for the gender suffixes.)} \zh{有母蚯蚓吗?}  
 ¶ \textcolor{darkblue}{\textbf{\ipa{dʑɯ˧dv̩˧-pʰv̩˩, | ə˩-dʑo˩˥?}}} \textcolor{Sepia}{\selectlanguage{english}Do male earthworms exist? (An artificially designed question, so as to elicit a form of 'earthworm' with the 'male' suffix, with a view to understanding the synchronically productive tone assignment rules for the gender suffixes.)} \zh{有公蚯蚓吗?}  
 \zh{量词}: \textcolor{darkblue}{\textbf{\ipa{kʰɯ˩}}}  \mytextsc{clf}: \textcolor{darkblue}{\textbf{\ipa{kʰɯ˩}}} 
\lhead{\firstmark}
\rhead{\botmark}

\subsection{\hspace{-0.5cm} {\Large \textcolor{darkblue}{\textbf{\ipa{dʑɯ˧dze˧mi\#˥}}}}\hspace{0.5cm}[\kern2pt{\textcolor{darkblue}{\textbf{\ipa{dʑɯ˩dze˩mi˩˥}}}}\kern2pt]} \hypertarget{dz£M\string_Mdze\string_Mmi\#\string_T1}{}
\markboth{\textcolor{darkblue}{\textbf{\ipa{dʑɯ˧dze˧mi\#˥}}}}{}
\textcolor{teal}{\mytextsc{noun}} \hspace{4pt} Tone: \#H.
\textcolor{Sepia}{\selectlanguage{english}Cicada.} \zh{蝉。}  ¶ \textcolor{darkblue}{\textbf{\ipa{dʑɯ˧dze˧-mi˧ tʰv̩˧-mi˧˥ / dʑɯ˧dze˧-mi˧ tʰv̩˧-mi˥\#}}} \textcolor{Sepia}{\selectlanguage{english}\mytextsc{n}+\mytextsc{dem}+\mytextsc{clf}} \zh{这只蝉}  
 \zh{量词}: \textcolor{darkblue}{\textbf{\ipa{mi˩}}}  \mytextsc{clf}: \textcolor{darkblue}{\textbf{\ipa{mi˩}}} 
\lhead{\firstmark}
\rhead{\botmark}

\subsection{\hspace{-0.5cm} {\Large \textcolor{darkblue}{\textbf{\ipa{dʑɯ˧ki˥}}}}\hspace{0.5cm}[\kern2pt{\textcolor{darkblue}{\textbf{\ipa{dʑɯ˩ki˩˥}}}}\kern2pt]} \hypertarget{dz£M\string_Mki\string_T1}{}
\markboth{\textcolor{darkblue}{\textbf{\ipa{dʑɯ˧ki˥}}}}{}
\textcolor{teal}{\mytextsc{noun}} \hspace{4pt} Tone: H\#.
\textcolor{Sepia}{\selectlanguage{english}Girdle, waistband (a large piece of fabric worn at the waist; can be used to carry a child); belt.} \zh{布带子,用来背小孩的带子,腰带。}  \zh{量词}: \textcolor{darkblue}{\textbf{\ipa{kʰɯ˩}}}  \mytextsc{clf}: \textcolor{darkblue}{\textbf{\ipa{kʰɯ˩}}} 
\lhead{\firstmark}
\rhead{\botmark}

\subsection{\hspace{-0.5cm} {\Large \textcolor{darkblue}{\textbf{\ipa{dʑɯ˧-li˧}}}}\hspace{0.5cm}[\kern2pt{\textcolor{darkblue}{\textbf{\ipa{xxxx non-correspondance entre le nombre de morphèmes et le nombre de tons de morphèmes}}}}\kern2pt]} \hypertarget{dz£M\string_M-li\string_M1}{}
\markboth{\textcolor{darkblue}{\textbf{\ipa{dʑɯ˧-li˧}}}}{}
\textcolor{teal}{\mytextsc{verb}} \hspace{4pt} Tone: M.
\textcolor{Sepia}{\selectlanguage{english}To irrigate.} \zh{灌溉。}  ¶ \textcolor{darkblue}{\textbf{\ipa{dʑɯ˧-li˧-ze˧}}} \textcolor{Sepia}{\selectlanguage{english}\mytextsc{pfv}} \zh{灌溉了}  
 ¶ \textcolor{darkblue}{\textbf{\ipa{dʑɯ˧-mɤ˧-li˧-hĩ˧ lv̩˧}}} \textcolor{Sepia}{\selectlanguage{english}dry farmland, dry field: a field that is not irrigated} \zh{旱田:不灌溉的田}  

\lhead{\firstmark}
\rhead{\botmark}

\subsection{\hspace{-0.5cm} {\Large \textcolor{darkblue}{\textbf{\ipa{dʑɯ˧ɭɯ˧}}}}\hspace{0.5cm}[\kern2pt{\textcolor{darkblue}{\textbf{\ipa{dʑɯ˧ɭɯ˧}}}}\kern2pt]} \hypertarget{dz£M\string_Ml\string_RM\string_M1}{}
\markboth{\textcolor{darkblue}{\textbf{\ipa{dʑɯ˧ɭɯ˧}}}}{}
\textcolor{teal}{\mytextsc{noun}} \hspace{4pt} Tone: M.
\textcolor{Sepia}{\selectlanguage{english}Broomcorn millet, \textit{Panicum miliaceum}.} \zh{黍,小米。}  ¶ \textcolor{darkblue}{\textbf{\ipa{dʑɯ˧ɭɯ˧-ho\#˥}}} \textcolor{Sepia}{\selectlanguage{english}millet gruel} \zh{小米粥}  
\textit{See:} \textcolor{darkblue}{\textbf{\ipa{dʑɯ˧njɤ˧, dʑɯ˧ʈʂʰwæ\#˥}}} 
\lhead{\firstmark}
\rhead{\botmark}

\subsection{\hspace{-0.5cm} {\Large \textcolor{darkblue}{\textbf{\ipa{dʑɯ˧mi˧}}}}\hspace{0.5cm}[\kern2pt{\textcolor{darkblue}{\textbf{\ipa{dʑɯ˧mi˧}}}}\kern2pt]} \hypertarget{dz£M\string_Mmi\string_M1}{}
\markboth{\textcolor{darkblue}{\textbf{\ipa{dʑɯ˧mi˧}}}}{}
\textcolor{teal}{\mytextsc{noun}} \hspace{4pt} Tone: M.
\textcolor{Sepia}{\selectlanguage{english}Large river.} \zh{大河。}  \zh{量词}: \textcolor{darkblue}{\textbf{\ipa{kʰɯ˩}}}  \mytextsc{clf}: \textcolor{darkblue}{\textbf{\ipa{kʰɯ˩}}} 
\lhead{\firstmark}
\rhead{\botmark}

\subsection{\hspace{-0.5cm} {\Large \textcolor{darkblue}{\textbf{\ipa{dʑɯ˧njɤ˧}}}}\hspace{0.5cm}[\kern2pt{\textcolor{darkblue}{\textbf{\ipa{dʑɯ˧njɤ˧}}}}\kern2pt]} \hypertarget{dz£M\string_Mnj7\string_M1}{}
\markboth{\textcolor{darkblue}{\textbf{\ipa{dʑɯ˧njɤ˧}}}}{}
\textcolor{teal}{\mytextsc{noun}} \hspace{4pt} Tone: M.
\textcolor{Sepia}{\selectlanguage{english}Broomcorn millet, \textit{Panicum miliaceum}.} \zh{黍,小米。}  ¶ \textcolor{darkblue}{\textbf{\ipa{dʑɯ˧njɤ˧, | ʐɯ˧ tɕɤ˧˥!}}} \textcolor{Sepia}{\selectlanguage{english}Millet is used to make wine!} \zh{小米,用来酿酒!}  
 ¶ \textcolor{darkblue}{\textbf{\ipa{dʑɯ˧njɤ˧-hɑ\#˥}}} \textcolor{Sepia}{\selectlanguage{english}cooked millet} \zh{小米饭}  
\textit{See:} \textcolor{darkblue}{\textbf{\ipa{dʑɯ˧ɭɯ˧, dʑɯ˧ʈʂʰwæ\#˥}}} 
\lhead{\firstmark}
\rhead{\botmark}

\subsection{\hspace{-0.5cm} {\Large \textcolor{darkblue}{\textbf{\ipa{dʑɯ˧qʰɑ˧}}}}\hspace{0.5cm}[\kern2pt{\textcolor{darkblue}{\textbf{\ipa{dʑɯ˧qʰɑ˧}}}}\kern2pt]} \hypertarget{dz£M\string_Mq\string_hA\string_M1}{}
\markboth{\textcolor{darkblue}{\textbf{\ipa{dʑɯ˧qʰɑ˧}}}}{}
\textcolor{teal}{\mytextsc{noun}} \hspace{4pt} Tone: M.
\textcolor{Sepia}{\selectlanguage{english}Selfheal (a plant used in Chinese medicine).} \zh{夏枯草。}  ¶ \textcolor{darkblue}{\textbf{\ipa{dʑɯ˧qʰɑ˧-bæ˩bæ˩}}} \textcolor{Sepia}{\selectlanguage{english}selfheal flowers} \zh{夏枯草花}  
 \zh{量词}: \textcolor{darkblue}{\textbf{\ipa{qɑ˩}}}  \mytextsc{clf}: \textcolor{darkblue}{\textbf{\ipa{qɑ˩}}} 
\lhead{\firstmark}
\rhead{\botmark}

\subsection{\hspace{-0.5cm} {\Large \textcolor{darkblue}{\textbf{\ipa{dʑɯ˧qʰv̩˩}}}}\hspace{0.5cm}[\kern2pt{\textcolor{darkblue}{\textbf{\ipa{dʑɯ˧qʰv̩˩}}}}\kern2pt]} \hypertarget{dz£M\string_Mq\string_hv\string_=\string_B1}{}
\markboth{\textcolor{darkblue}{\textbf{\ipa{dʑɯ˧qʰv̩˩}}}}{}
\textcolor{teal}{\mytextsc{noun}} \hspace{4pt} Tone: L\#.
\textcolor{Sepia}{\selectlanguage{english}A wild plant of Yongning.} \zh{永宁的一种植物。}  ¶ \textcolor{darkblue}{\textbf{\ipa{dʑɯ˧qʰv̩˩-lv̩˩lv̩˩}}} \textcolor{Sepia}{\selectlanguage{english}the grains of this plant} \zh{这种植物的种子}  
 \zh{量词}: \textcolor{darkblue}{\textbf{\ipa{ɭɯ˧}}}  \mytextsc{clf}: \textcolor{darkblue}{\textbf{\ipa{ɭɯ˧}}} 
\lhead{\firstmark}
\rhead{\botmark}

\subsection{\hspace{-0.5cm} {\Large \textcolor{darkblue}{\textbf{\ipa{dʑɯ˧qʰv̩˧}}}}\hspace{0.5cm}[\kern2pt{\textcolor{darkblue}{\textbf{\ipa{dʑɯ˧qʰv̩˧}}}}\kern2pt]} \hypertarget{dz£M\string_Mq\string_hv\string_=\string_M1}{}
\markboth{\textcolor{darkblue}{\textbf{\ipa{dʑɯ˧qʰv̩˧}}}}{}
\textcolor{teal}{\mytextsc{noun}} \hspace{4pt} Tone: M.
\textcolor{Sepia}{\selectlanguage{english}Well.} \zh{井、水井。}  ¶ \textcolor{darkblue}{\textbf{\ipa{ɑ˩ʁo˥ | dʑɯ˧qʰv̩˧ tʰi˧-di˩.}}} \textcolor{Sepia}{\selectlanguage{english}There is a well at home / in the farm.} \zh{家里有水井。}  
 \zh{量词}: \textcolor{darkblue}{\textbf{\ipa{ɭɯ˧}}}  \mytextsc{clf}: \textcolor{darkblue}{\textbf{\ipa{ɭɯ˧}}} 
\lhead{\firstmark}
\rhead{\botmark}

\subsection{\hspace{-0.5cm} {\Large \textcolor{darkblue}{\textbf{\ipa{dʑɯ˧ʁo˩}}}}\hspace{0.5cm}[\kern2pt{\textcolor{darkblue}{\textbf{\ipa{dʑɯ˧ʁo˩}}}}\kern2pt]} \hypertarget{dz£M\string_MRo\string_B1}{}
\markboth{\textcolor{darkblue}{\textbf{\ipa{dʑɯ˧ʁo˩}}}}{}
\textcolor{teal}{\mytextsc{noun}} \hspace{4pt} Tone: L\#.
\textcolor{Sepia}{\selectlanguage{english}Peach.} \zh{桃子。} 
\lhead{\firstmark}
\rhead{\botmark}

\subsection{\hspace{-0.5cm} {\Large \textcolor{darkblue}{\textbf{\ipa{dʑɯ˧ʈʂʰwæ\#˥}}}}\hspace{0.5cm}[\kern2pt{\textcolor{darkblue}{\textbf{\ipa{dʑɯ˧ʈʂʰwæ˧}}}}\kern2pt]} \hypertarget{dz£M\string_Mt`s`\string_hw\{\#\string_T1}{}
\markboth{\textcolor{darkblue}{\textbf{\ipa{dʑɯ˧ʈʂʰwæ\#˥}}}}{}
\textcolor{teal}{\mytextsc{noun}} \hspace{4pt} Tone: \#H.
\textcolor{Sepia}{\selectlanguage{english}Husked broomcorn millet, \textit{Panicum miliaceum}.} \zh{已碾的小米。} \textit{See:} \textcolor{darkblue}{\textbf{\ipa{dʑɯ˧ɭɯ˧, dʑɯ˧njɤ˧}}} 
\lhead{\firstmark}
\rhead{\botmark}

\subsection{\hspace{-0.5cm} {\Large \textcolor{darkblue}{\textbf{\ipa{dʑɯ˩}}}}\hspace{0.5cm}[\kern2pt{\textcolor{darkblue}{\textbf{\ipa{dʑɯ˥}}}}\kern2pt]} \hypertarget{dz£M\string_B1}{}
\markboth{\textcolor{darkblue}{\textbf{\ipa{dʑɯ˩}}}}{}
\textcolor{teal}{\mytextsc{noun}} \hspace{4pt} Tone: L.
\ding{202} \textcolor{Sepia}{\selectlanguage{english}Water.} \zh{水。}  ¶ \textcolor{darkblue}{\textbf{\ipa{dʑɯ˧ ʈʰɯ˧}}} \textcolor{Sepia}{\selectlanguage{english}to drink water} \zh{喝水}  
 ¶ \textcolor{darkblue}{\textbf{\ipa{ʈʂʰɯ˧ dʑɯ˧ ʈʰɯ˧-dʑo˧!}}} \textcolor{Sepia}{\selectlanguage{english}(S)he is drinking water} \zh{他在喝水}  
 ¶ \textcolor{darkblue}{\textbf{\ipa{dʑɯ˧ | ɖɯ˧-ʈʰɤ˧ ʈʰɯ˧˥}}} \textcolor{Sepia}{\selectlanguage{english}to drink a little water (literally 'a drop of water')} \zh{喝一点水(直译:‘一滴水’)}  
 ¶ \textcolor{darkblue}{\textbf{\ipa{dʑɯ˩ kʰɯ˩˥}}} \textcolor{Sepia}{\selectlanguage{english}to put water} \zh{放水}  
 ¶ \textcolor{darkblue}{\textbf{\ipa{dʑɯ˩ mæ˩˥}}} \textcolor{Sepia}{\selectlanguage{english}to irrigate, to water} \zh{浇灌、灌溉}  
 ¶ \textcolor{darkblue}{\textbf{\ipa{dʑɯ˩ qæ˩, | hɑ˩ qæ˩˥ |}}} \textcolor{Sepia}{\selectlanguage{english}a description of the traveller's changes in environment: 'to change water, to change food'. This requires using strategies to avoid ailments: in particular, it was customary to boil in water a little earth of the place where one had arrived, and to drink this preparation.} \zh{‘换水换土’:这个短语描述旅人到他人乡的情况,带来水土不服的危险。为了预防这类不良反应,摩梭旅人习惯水煮一点当地的土,喝下去,为了适应当地的水土。}  
 ¶ \textcolor{darkblue}{\textbf{\ipa{[F5] dʑɯ˧ | mv̩˩tɕo˧ dɑ˧˥}}} \textcolor{Sepia}{\selectlanguage{english}the water flows downwards} \zh{水往下流}  
 \zh{量词}: \textcolor{darkblue}{\textbf{\ipa{kʰɯ˩}}} \ding{203} \textcolor{Sepia}{\selectlanguage{english}River, waterway.} \zh{河流。}  \mytextsc{clf}: \textcolor{darkblue}{\textbf{\ipa{kʰɯ˩}}} 
\lhead{\firstmark}
\rhead{\botmark}

\subsection{\hspace{-0.5cm} {\Large \textcolor{darkblue}{\textbf{\ipa{dʑɯ˩\textsubscript{a}}}}}\hspace{0.5cm}[\kern2pt{\textcolor{darkblue}{\textbf{\ipa{dʑɯ˥}}}}\kern2pt]} \hypertarget{dz£M\string_Ba1}{}
\markboth{\textcolor{darkblue}{\textbf{\ipa{dʑɯ˩\textsubscript{a}}}}}{}
\textcolor{teal}{\mytextsc{verb}} \hspace{4pt} Tone: L\textsubscript{a}.
\textcolor{Sepia}{\selectlanguage{english}To twist (strings) together (to make a rope).} \zh{搓(搓绳子)。}  ¶ \textcolor{darkblue}{\textbf{\ipa{le˧-dʑɯ˩-ze˩}}} \textcolor{Sepia}{\selectlanguage{english}\mytextsc{accomp} \string_ \mytextsc{pfv}} \zh{搓了}  
 ¶ \textcolor{darkblue}{\textbf{\ipa{bæ˩ dʑɯ˩˥}}} \textcolor{Sepia}{\selectlanguage{english}to twist (strings into) a rope} \zh{搓绳子}  
 ¶ \textcolor{darkblue}{\textbf{\ipa{qʰv̩˩ɖʐæ˩ dʑɯ˥}}} \textcolor{Sepia}{\selectlanguage{english}to twist a string, a small rope} \zh{搓一根小绳子}  
 ¶ \textcolor{darkblue}{\textbf{\ipa{ɖɯ˧-kʰwɤ˧ dʑɯ˥}}} \textcolor{Sepia}{\selectlanguage{english}to twist a little} \zh{搓一下}  

\lhead{\firstmark}
\rhead{\botmark}

\subsection{\hspace{-0.5cm} {\Large \textcolor{darkblue}{\textbf{\ipa{dʑɯ˩-æ̃˩tsɯ˧}}}}\hspace{0.5cm}[\kern2pt{\textcolor{darkblue}{\textbf{\ipa{xxxx non-correspondance entre le nombre de morphèmes et le nombre de tons de morphèmes}}}}\kern2pt]} \hypertarget{dz£M\string_B-\{\string_~\string_BtsM\string_M1}{}
\markboth{\textcolor{darkblue}{\textbf{\ipa{dʑɯ˩-æ̃˩tsɯ˧}}}}{}
\textcolor{teal}{\mytextsc{noun}} \hspace{4pt} Tone: L-LM.
\textcolor{Sepia}{\selectlanguage{english}Water fowl: used as a cover term for a variety of birds including sandpiper (\textit{Calidris}), avocet, Baillon's crake, and blue-breasted banded rail.} \zh{水禽,包括几种不同的小鸟,如:鹬。}  \zh{量词}: \textcolor{darkblue}{\textbf{\ipa{ɭɯ˧}}}  \mytextsc{clf}: \textcolor{darkblue}{\textbf{\ipa{ɭɯ˧}}} 
\lhead{\firstmark}
\rhead{\botmark}

\subsection{\hspace{-0.5cm} {\Large \textcolor{darkblue}{\textbf{\ipa{dʑɯ˩dze˩}}}}\hspace{0.5cm}[\kern2pt{\textcolor{darkblue}{\textbf{\ipa{dʑɯ˧dze˧}}}}\kern2pt]} \hypertarget{dz£M\string_Bdze\string_B1}{}
\markboth{\textcolor{darkblue}{\textbf{\ipa{dʑɯ˩dze˩}}}}{}
\textcolor{teal}{\mytextsc{noun}} \hspace{4pt} Tone: L.
\textcolor{Sepia}{\selectlanguage{english}Ladle used for people's food.} \zh{舀汤的勺子。}  \zh{量词}: \textcolor{darkblue}{\textbf{\ipa{nɑ˧}}}  \mytextsc{clf}: \textcolor{darkblue}{\textbf{\ipa{nɑ˧}}} 
\lhead{\firstmark}
\rhead{\botmark}

\subsection{\hspace{-0.5cm} {\Large \textcolor{darkblue}{\textbf{\ipa{dʑɯ˩gɤ˩di˩}}}}\hspace{0.5cm}[\kern2pt{\textcolor{darkblue}{\textbf{\ipa{dʑɯ˧gɤ˧di˧}}}}\kern2pt]} \hypertarget{dz£M\string_Bg7\string_Bdi\string_B1}{}
\markboth{\textcolor{darkblue}{\textbf{\ipa{dʑɯ˩gɤ˩di˩}}}}{}
\textcolor{teal}{\mytextsc{noun}} \hspace{4pt} Tone: L.
\textcolor{Sepia}{\selectlanguage{english}Carrying/shoulder pole.} \zh{扁担。}  \zh{量词}: \textcolor{darkblue}{\textbf{\ipa{nɑ˧}}}  \mytextsc{clf}: \textcolor{darkblue}{\textbf{\ipa{nɑ˧}}} 
\lhead{\firstmark}
\rhead{\botmark}

\subsection{\hspace{-0.5cm} {\Large \textcolor{darkblue}{\textbf{\ipa{dʑɯ˩gv̩˩}}}}\hspace{0.5cm}[\kern2pt{\textcolor{darkblue}{\textbf{\ipa{dʑɯ˩gv̩˩˥}}}}\kern2pt]} \hypertarget{dz£M\string_Bgv\string_=\string_B1}{}
\markboth{\textcolor{darkblue}{\textbf{\ipa{dʑɯ˩gv̩˩}}}}{}
\textcolor{teal}{\mytextsc{noun}} \hspace{4pt} Tone: L.
\textcolor{Sepia}{\selectlanguage{english}Large barrel where drinking water is kept; water trough.} \zh{大水桶,水槽。}  ¶ \textcolor{darkblue}{\textbf{\ipa{[F5] pv̩˩-dʑɯ˩gv̩˩˥}}}  
 \zh{量词}: \textcolor{darkblue}{\textbf{\ipa{ɭɯ˧}}}  \mytextsc{clf}: \textcolor{darkblue}{\textbf{\ipa{ɭɯ˧}}} 
\lhead{\firstmark}
\rhead{\botmark}

\subsection{\hspace{-0.5cm} {\Large \textcolor{darkblue}{\textbf{\ipa{dʑɯ˩gv̩˥}}}}\hspace{0.5cm}[\kern2pt{\textcolor{darkblue}{\textbf{\ipa{dʑɯ˩gv̩˩˥}}}}\kern2pt]} \hypertarget{dz£M\string_Bgv\string_=\string_T1}{}
\markboth{\textcolor{darkblue}{\textbf{\ipa{dʑɯ˩gv̩˥}}}}{}
\textcolor{teal}{\mytextsc{adjective}} \hspace{4pt} Tone: LH.
\textcolor{Sepia}{\selectlanguage{english}Round-shouldered, stooping.} \zh{驼背。} 
\lhead{\firstmark}
\rhead{\botmark}

\subsection{\hspace{-0.5cm} {\Large \textcolor{darkblue}{\textbf{\ipa{dʑɯ˩hṽ˧˥}}}}\hspace{0.5cm}[\kern2pt{\textcolor{darkblue}{\textbf{\ipa{dʑɯ˩hṽ˥}}}}\kern2pt]} \hypertarget{dz£M\string_Bhv\string_~\string_M\string_T1}{}
\markboth{\textcolor{darkblue}{\textbf{\ipa{dʑɯ˩hṽ˧˥}}}}{}
\textcolor{teal}{\mytextsc{noun}} \hspace{4pt} Tone: LM+MH\#.
\textcolor{Sepia}{\selectlanguage{english}Dough made of flour and water.} \zh{面和水和成的浆糊。} 
\lhead{\firstmark}
\rhead{\botmark}

\subsection{\hspace{-0.5cm} {\Large \textcolor{darkblue}{\textbf{\ipa{dʑɯ˩-hwæ˩tsɯ˥}}}}\hspace{0.5cm}[\kern2pt{\textcolor{darkblue}{\textbf{\ipa{xxxx non-correspondance entre le nombre de morphèmes et le nombre de tons de morphèmes}}}}\kern2pt]} \hypertarget{dz£M\string_B-hw\{\string_BtsM\string_T1}{}
\markboth{\textcolor{darkblue}{\textbf{\ipa{dʑɯ˩-hwæ˩tsɯ˥}}}}{}
\textcolor{teal}{\mytextsc{noun}} \hspace{4pt} Tone: L+H\#.
\textcolor{Sepia}{\selectlanguage{english}Shrew: the consultant uses a periphrasis: “wild mouse”.} \zh{尖鼠、鼩鼱。} 
\lhead{\firstmark}
\rhead{\botmark}

\subsection{\hspace{-0.5cm} {\Large \textcolor{darkblue}{\textbf{\ipa{dʑɯ˩kʰi˩}}}}\hspace{0.5cm}[\kern2pt{\textcolor{darkblue}{\textbf{\ipa{dʑɯ˩kʰi˥}}}}\kern2pt]} \hypertarget{dz£M\string_Bk\string_hi\string_B1}{}
\markboth{\textcolor{darkblue}{\textbf{\ipa{dʑɯ˩kʰi˩}}}}{}
\textcolor{teal}{\mytextsc{noun}} \hspace{4pt} Tone: L.
\textcolor{Sepia}{\selectlanguage{english}Water's edge.} \zh{水边。} 
\lhead{\firstmark}
\rhead{\botmark}

\subsection{\hspace{-0.5cm} {\Large \textcolor{darkblue}{\textbf{\ipa{dʑɯ˩kʰv̩˩}}}}\hspace{0.5cm}[\kern2pt{\textcolor{darkblue}{\textbf{\ipa{dʑɯ˩kʰv̩˩˥}}}}\kern2pt]} \hypertarget{dz£M\string_Bk\string_hv\string_=\string_B1}{}
\markboth{\textcolor{darkblue}{\textbf{\ipa{dʑɯ˩kʰv̩˩}}}}{}
\textcolor{teal}{\mytextsc{noun}} \hspace{4pt} Tone: L.
\textcolor{Sepia}{\selectlanguage{english}Moss.} \zh{青苔。} 
\lhead{\firstmark}
\rhead{\botmark}

\subsection{\hspace{-0.5cm} {\Large \textcolor{darkblue}{\textbf{\ipa{dʑɯ˩nɑ˩hæ̃˩tʰɑ˩}}}}\hspace{0.5cm}[\kern2pt{\textcolor{darkblue}{\textbf{\ipa{dʑɯ˩nɑ˩hæ̃˩tʰɑ˩˥}}}}\kern2pt]} \hypertarget{dz£M\string_BnA\string_Bh\{\string_~\string_Bt\string_hA\string_B1}{}
\markboth{\textcolor{darkblue}{\textbf{\ipa{dʑɯ˩nɑ˩hæ̃˩tʰɑ˩}}}}{}
\textcolor{teal}{\mytextsc{noun}} \hspace{4pt} Tone: L.
\textcolor{Sepia}{\selectlanguage{english}Water-mill.} \zh{水磨。}  \zh{量词}: \textcolor{darkblue}{\textbf{\ipa{pɤ˩}}}  \mytextsc{clf}: \textcolor{darkblue}{\textbf{\ipa{pɤ˩}}} 
\lhead{\firstmark}
\rhead{\botmark}

\subsection{\hspace{-0.5cm} {\Large \textcolor{darkblue}{\textbf{\ipa{dʑɯ˩nɑ˩mi˩}}}}\hspace{0.5cm}[\kern2pt{\textcolor{darkblue}{\textbf{\ipa{dʑɯ˩nɑ˩mi˩˥}}}}\kern2pt]} \hypertarget{dz£M\string_BnA\string_Bmi\string_B1}{}
\markboth{\textcolor{darkblue}{\textbf{\ipa{dʑɯ˩nɑ˩mi˩}}}}{}
\textcolor{teal}{\mytextsc{noun}} \hspace{4pt} Tone: L.
\textcolor{Sepia}{\selectlanguage{english}Mountain areas (uncultivated), mountain forest, wilderness.} \zh{深山老林、高山上的地方。} \textit{See:} \textcolor{darkblue}{\textbf{\ipa{dʑɯ˩nɑ˩mi˩-ʁo˩, dʑɯ˩ʁo˩}}} 
\lhead{\firstmark}
\rhead{\botmark}

\subsection{\hspace{-0.5cm} {\Large \textcolor{darkblue}{\textbf{\ipa{dʑɯ˩nɑ˩mi˩-ʁo˩}}}}\hspace{0.5cm}[\kern2pt{\textcolor{darkblue}{\textbf{\ipa{xxxx non-correspondance entre le nombre de morphèmes et le nombre de tons de morphèmes}}}}\kern2pt]} \hypertarget{dz£M\string_BnA\string_Bmi\string_B-Ro\string_B1}{}
\markboth{\textcolor{darkblue}{\textbf{\ipa{dʑɯ˩nɑ˩mi˩-ʁo˩}}}}{}
\textcolor{teal}{\mytextsc{noun}} \hspace{4pt} Tone: L.
\textcolor{Sepia}{\selectlanguage{english}Mountain areas (uncultivated), mountain forest, wilderness.} \zh{深山老林、高山上的地方。} \textit{See:} \textcolor{darkblue}{\textbf{\ipa{dʑɯ˩nɑ˩mi˩, dʑɯ˩ʁo˩}}} 
\lhead{\firstmark}
\rhead{\botmark}

\subsection{\hspace{-0.5cm} {\Large \textcolor{darkblue}{\textbf{\ipa{dʑɯ˩pɤ˩-kv̩˧hĩ˩}}}}\hspace{0.5cm}[\kern2pt{\textcolor{darkblue}{\textbf{\ipa{dʑɯ˩pɤ˩kv̩˧hĩ˩}}}}\kern2pt]} \hypertarget{dz£M\string_Bp7\string_B-kv\string_=\string_Mhi\string_~\string_B1}{}
\markboth{\textcolor{darkblue}{\textbf{\ipa{dʑɯ˩pɤ˩-kv̩˧hĩ˩}}}}{}
\textcolor{teal}{\mytextsc{noun}} \hspace{4pt} Tone: L-L\#.
\textcolor{Sepia}{\selectlanguage{english}Spring.} \zh{水泉、山泉。}  ¶ \textcolor{darkblue}{\textbf{\ipa{dʑɯ˩pɤ˩-kv̩˧hĩ˩ | tʰi˧-di˩}}} \textcolor{Sepia}{\selectlanguage{english}there is a spring} \zh{有水泉}  
\textit{See:} \textcolor{darkblue}{\textbf{\ipa{dʑɯ˩pɤ˩qʰv̩˩, dʑɯ˩pɤ˩tv̩˩qʰv̩˥}}} 
\lhead{\firstmark}
\rhead{\botmark}

\subsection{\hspace{-0.5cm} {\Large \textcolor{darkblue}{\textbf{\ipa{dʑɯ˩pɤ˩qʰv̩˩}}}}\hspace{0.5cm}[\kern2pt{\textcolor{darkblue}{\textbf{\ipa{dʑɯ˩pɤ˩qʰv̩˩˥}}}}\kern2pt]} \hypertarget{dz£M\string_Bp7\string_Bq\string_hv\string_=\string_B1}{}
\markboth{\textcolor{darkblue}{\textbf{\ipa{dʑɯ˩pɤ˩qʰv̩˩}}}}{}
\textcolor{teal}{\mytextsc{noun}} \hspace{4pt} Tone: L.
\textcolor{Sepia}{\selectlanguage{english}Spring.} \zh{水泉、山泉。} \textit{See:} \textcolor{darkblue}{\textbf{\ipa{dʑɯ˩pɤ˩-kv̩˧hĩ˩, dʑɯ˩pɤ˩tv̩˩qʰv̩˥}}} 
\lhead{\firstmark}
\rhead{\botmark}

\subsection{\hspace{-0.5cm} {\Large \textcolor{darkblue}{\textbf{\ipa{dʑɯ˩pɤ˩tv̩˩qʰv̩˥}}}}\hspace{0.5cm}[\kern2pt{\textcolor{darkblue}{\textbf{\ipa{dʑɯ˩pɤ˩tv̩˩qʰv̩˥}}}}\kern2pt]} \hypertarget{dz£M\string_Bp7\string_Btv\string_=\string_Bq\string_hv\string_=\string_T1}{}
\markboth{\textcolor{darkblue}{\textbf{\ipa{dʑɯ˩pɤ˩tv̩˩qʰv̩˥}}}}{}
\textcolor{teal}{\mytextsc{noun}} \hspace{4pt} Tone: L+H\#.
\textcolor{Sepia}{\selectlanguage{english}Spring.} \zh{水泉、山泉。} \textit{See:} \textcolor{darkblue}{\textbf{\ipa{dʑɯ˩pɤ˩qʰv̩˩, dʑɯ˩pɤ˩-kv̩˧hĩ˩}}} 
\lhead{\firstmark}
\rhead{\botmark}

\subsection{\hspace{-0.5cm} {\Large \textcolor{darkblue}{\textbf{\ipa{dʑɯ˩pʰæ˩}}}}\hspace{0.5cm}[\kern2pt{\textcolor{darkblue}{\textbf{\ipa{dʑɯ˩pʰæ˩˥}}}}\kern2pt]} \hypertarget{dz£M\string_Bp\string_h\{\string_B1}{}
\markboth{\textcolor{darkblue}{\textbf{\ipa{dʑɯ˩pʰæ˩}}}}{}
\textcolor{teal}{\mytextsc{noun}} \hspace{4pt} Tone: L.
\textcolor{Sepia}{\selectlanguage{english}Ice.} \zh{冰。}  \zh{量词}: \textcolor{darkblue}{\textbf{\ipa{pʰæ˧˥}}}  \mytextsc{clf}: \textcolor{darkblue}{\textbf{\ipa{pʰæ˧˥}}} 
\lhead{\firstmark}
\rhead{\botmark}

\subsection{\hspace{-0.5cm} {\Large \textcolor{darkblue}{\textbf{\ipa{dʑɯ˩qʰæ˩}}}}\hspace{0.5cm}[\kern2pt{\textcolor{darkblue}{\textbf{\ipa{dʑɯ˩qʰæ˩˥}}}}\kern2pt]} \hypertarget{dz£M\string_Bq\string_h\{\string_B1}{}
\markboth{\textcolor{darkblue}{\textbf{\ipa{dʑɯ˩qʰæ˩}}}}{}
\textcolor{teal}{\mytextsc{noun}} \hspace{4pt} Tone: L.
\textcolor{Sepia}{\selectlanguage{english}Cold water.} \zh{凉水。} 
\lhead{\firstmark}
\rhead{\botmark}

\subsection{\hspace{-0.5cm} {\Large \textcolor{darkblue}{\textbf{\ipa{dʑɯ˩qʰwɤ˩-lv̩˩}}}}\hspace{0.5cm}[\kern2pt{\textcolor{darkblue}{\textbf{\ipa{xxxx non-correspondance entre le nombre de morphèmes et le nombre de tons de morphèmes}}}}\kern2pt]} \hypertarget{dz£M\string_Bq\string_hw7\string_B-lv\string_=\string_B1}{}
\markboth{\textcolor{darkblue}{\textbf{\ipa{dʑɯ˩qʰwɤ˩-lv̩˩}}}}{}
\textcolor{teal}{\mytextsc{noun}} \hspace{4pt} Tone: L.
\textcolor{Sepia}{\selectlanguage{english}Marsh, bog, swamp (unsuitable for agriculture).} \zh{沼泽、湿地。} Local Chinese dialect:\zh{潮地。} \zh{量词}: \textcolor{darkblue}{\textbf{\ipa{kɤ˧˥}}}  \mytextsc{clf}: \textcolor{darkblue}{\textbf{\ipa{kɤ˧˥}}} 
\lhead{\firstmark}
\rhead{\botmark}

\subsection{\hspace{-0.5cm} {\Large \textcolor{darkblue}{\textbf{\ipa{dʑɯ˩ʁo˩}}}}\hspace{0.5cm}[\kern2pt{\textcolor{darkblue}{\textbf{\ipa{dʑɯ˩ʁo˩˥}}}}\kern2pt]} \hypertarget{dz£M\string_BRo\string_B1}{}
\markboth{\textcolor{darkblue}{\textbf{\ipa{dʑɯ˩ʁo˩}}}}{}
\textcolor{teal}{\mytextsc{noun}} \hspace{4pt} Tone: L.
\textcolor{Sepia}{\selectlanguage{english}Mountain areas (uncultivated), mountain forest, wilderness.} \zh{深山老林、高山上的地方。} \textit{See:} \textcolor{darkblue}{\textbf{\ipa{dʑɯ˩nɑ˩mi˩, dʑɯ˩nɑ˩mi˩-ʁo˩}}} 
\lhead{\firstmark}
\rhead{\botmark}

\subsection{\hspace{-0.5cm} {\Large \textcolor{darkblue}{\textbf{\ipa{dʑɯ˩ʁo˩-æ̃˧}}}}\hspace{0.5cm}[\kern2pt{\textcolor{darkblue}{\textbf{\ipa{dʑɯ˩ʁo˩æ̃˥}}}}\kern2pt]} \hypertarget{dz£M\string_BRo\string_B-\{\string_~\string_M1}{}
\markboth{\textcolor{darkblue}{\textbf{\ipa{dʑɯ˩ʁo˩-æ̃˧}}}}{}
\textcolor{teal}{\mytextsc{noun}} \hspace{4pt} Tone: L-M.
\textcolor{Sepia}{\selectlanguage{english}Quail, rail, \textit{Coturnix}; used when identifying pictures of various species of \textit{Turnix}, \textit{Coturnix}, and \textit{Crex}.} \zh{鹌鹑。}  \zh{量词}: \textcolor{darkblue}{\textbf{\ipa{mi˩}}}  \mytextsc{clf}: \textcolor{darkblue}{\textbf{\ipa{mi˩}}} 
\lhead{\firstmark}
\rhead{\botmark}

\subsection{\hspace{-0.5cm} {\Large \textcolor{darkblue}{\textbf{\ipa{dʑɯ˩ʁo˩-bo˧}}}}\hspace{0.5cm}[\kern2pt{\textcolor{darkblue}{\textbf{\ipa{dʑɯ˩ʁo˩bo˥}}}}\kern2pt]} \hypertarget{dz£M\string_BRo\string_B-bo\string_M1}{}
\markboth{\textcolor{darkblue}{\textbf{\ipa{dʑɯ˩ʁo˩-bo˧}}}}{}
\textcolor{teal}{\mytextsc{noun}} \hspace{4pt} Tone: L-M.
\textcolor{Sepia}{\selectlanguage{english}Wild boar.} \zh{野猪。}  \zh{量词}: \textcolor{darkblue}{\textbf{\ipa{mi˩}}}  \mytextsc{clf}: \textcolor{darkblue}{\textbf{\ipa{mi˩}}} \textit{Syn:} \hyperlink{}{\textcolor{darkblue}{\textbf{\ipa{bo˩tv̩\#˥}}}}. 
\lhead{\firstmark}
\rhead{\botmark}

\subsection{\hspace{-0.5cm} {\Large \textcolor{darkblue}{\textbf{\ipa{dʑɯ˩ʁo˩-dze˧}}}}\hspace{0.5cm}[\kern2pt{\textcolor{darkblue}{\textbf{\ipa{dʑɯ˩ʁo˩dze˥}}}}\kern2pt]} \hypertarget{dz£M\string_BRo\string_B-dze\string_M1}{}
\markboth{\textcolor{darkblue}{\textbf{\ipa{dʑɯ˩ʁo˩-dze˧}}}}{}
\textcolor{teal}{\mytextsc{noun}} \hspace{4pt} Tone: L-M.
\textcolor{Sepia}{\selectlanguage{english}Wild pepper.} \zh{野花椒。} 
\lhead{\firstmark}
\rhead{\botmark}

\subsection{\hspace{-0.5cm} {\Large \textcolor{darkblue}{\textbf{\ipa{dʑɯ˩ʁo˩-hwɤ˩li˧}}}}\hspace{0.5cm}[\kern2pt{\textcolor{darkblue}{\textbf{\ipa{dʑɯ˩ʁo˩hwɤ˩li˥}}}}\kern2pt]} \hypertarget{dz£M\string_BRo\string_B-hw7\string_Bli\string_M1}{}
\markboth{\textcolor{darkblue}{\textbf{\ipa{dʑɯ˩ʁo˩-hwɤ˩li˧}}}}{}
\textcolor{teal}{\mytextsc{noun}} \hspace{4pt} Tone: L-LM.
\textcolor{Sepia}{\selectlanguage{english}Yunnan wild cat, \textit{Felis temincki}.} \zh{野猫。}  \zh{量词}: \textcolor{darkblue}{\textbf{\ipa{mi˩}}}  \mytextsc{clf}: \textcolor{darkblue}{\textbf{\ipa{mi˩}}} 
\lhead{\firstmark}
\rhead{\botmark}

\subsection{\hspace{-0.5cm} {\Large \textcolor{darkblue}{\textbf{\ipa{dʑɯ˩ʁo˩-ɬi˩bi˧}}}}\hspace{0.5cm}[\kern2pt{\textcolor{darkblue}{\textbf{\ipa{dʑɯ˩ʁo˩ɬi˩bi˥}}}}\kern2pt]} \hypertarget{dz£M\string_BRo\string_B-Ki\string_Bbi\string_M1}{}
\markboth{\textcolor{darkblue}{\textbf{\ipa{dʑɯ˩ʁo˩-ɬi˩bi˧}}}}{}
\textcolor{teal}{\mytextsc{noun}} \hspace{4pt} Tone: L-LM.
\textcolor{Sepia}{\selectlanguage{english}Chinese yam (shan-yao).} \zh{山药。}  \zh{量词}: \textcolor{darkblue}{\textbf{\ipa{ɭɯ˧}}}  \mytextsc{clf}: \textcolor{darkblue}{\textbf{\ipa{ɭɯ˧}}} 
\lhead{\firstmark}
\rhead{\botmark}

\subsection{\hspace{-0.5cm} {\Large \textcolor{darkblue}{\textbf{\ipa{dʑɯ˩ʁo˩-zɯ˩}}}}\hspace{0.5cm}[\kern2pt{\textcolor{darkblue}{\textbf{\ipa{xxxx non-correspondance entre le nombre de morphèmes et le nombre de tons de morphèmes}}}}\kern2pt]} \hypertarget{dz£M\string_BRo\string_B-zM\string_B1}{}
\markboth{\textcolor{darkblue}{\textbf{\ipa{dʑɯ˩ʁo˩-zɯ˩}}}}{}
\textcolor{teal}{\mytextsc{noun}} \hspace{4pt} Tone: L.
\textcolor{Sepia}{\selectlanguage{english}Wild herbs.} \zh{野草。}  ¶ \textcolor{darkblue}{\textbf{\ipa{ʈʂʰɯ˧ | dʑɯ˩ʁo˩-zɯ˩ ɲi˥.}}} \textcolor{Sepia}{\selectlanguage{english}\mytextsc{dem} \string_ \mytextsc{cop}} \zh{\mytextsc{指示代词} \string_ \mytextsc{系词}}  
 \zh{量词}: \textcolor{darkblue}{\textbf{\ipa{qɑ˩}}} \textcolor{darkblue}{\textbf{\ipa{po˧}}}  \mytextsc{clf}: \textcolor{darkblue}{\textbf{\ipa{qɑ˩}}} \textcolor{darkblue}{\textbf{\ipa{po˧}}} 
\lhead{\firstmark}
\rhead{\botmark}

\subsection{\hspace{-0.5cm} {\Large \textcolor{darkblue}{\textbf{\ipa{dʑɯ˩si˩}}}}\hspace{0.5cm}[\kern2pt{\textcolor{darkblue}{\textbf{\ipa{dʑɯ˩si˩˥}}}}\kern2pt]} \hypertarget{dz£M\string_Bsi\string_B1}{}
\markboth{\textcolor{darkblue}{\textbf{\ipa{dʑɯ˩si˩}}}}{}
\textcolor{teal}{\mytextsc{noun}} \hspace{4pt} Tone: L.
\textcolor{Sepia}{\selectlanguage{english}Oriental white oak.} \zh{青冈树、槲栎。} \textit{Syn:} \hyperlink{}{\textcolor{darkblue}{\textbf{\ipa{dzi˧dzi˧}}}}. 
\lhead{\firstmark}
\rhead{\botmark}

\subsection{\hspace{-0.5cm} {\Large \textcolor{darkblue}{\textbf{\ipa{dʑɯ˩so˩}}}}\hspace{0.5cm}[\kern2pt{\textcolor{darkblue}{\textbf{\ipa{dʑɯ˩so˩˥}}}}\kern2pt]} \hypertarget{dz£M\string_Bso\string_B1}{}
\markboth{\textcolor{darkblue}{\textbf{\ipa{dʑɯ˩so˩}}}}{}
\textcolor{teal}{\mytextsc{noun}} \hspace{4pt} Tone: L.
\textcolor{Sepia}{\selectlanguage{english}Wave.} \zh{波浪。}  ¶ \textcolor{darkblue}{\textbf{\ipa{dʑɯ˩so˩ pʰv̩˩˥}}} \textcolor{Sepia}{\selectlanguage{english}there is a wave, a wave breaks} \zh{有波浪}  
 \zh{量词}: \textcolor{darkblue}{\textbf{\ipa{pʰæ˧˥}}}  \mytextsc{clf}: \textcolor{darkblue}{\textbf{\ipa{pʰæ˧˥}}} 
\lhead{\firstmark}
\rhead{\botmark}

\subsection{\hspace{-0.5cm} {\Large \textcolor{darkblue}{\textbf{\ipa{dʑɯ˩ʂo˥}}}}\hspace{0.5cm}[\kern2pt{\textcolor{darkblue}{\textbf{\ipa{dʑɯ˩ʂo˥}}}}\kern2pt]} \hypertarget{dz£M\string_Bs`o\string_T1}{}
\markboth{\textcolor{darkblue}{\textbf{\ipa{dʑɯ˩ʂo˥}}}}{}
\textcolor{teal}{\mytextsc{noun}} \hspace{4pt} Tone: L+H\#.
\textcolor{Sepia}{\selectlanguage{english}Name of a ritual.} \zh{一项仪式。}  ¶ \textcolor{darkblue}{\textbf{\ipa{dʑɯ˩ʂo˥-tsɑ˩bɤ˩}}} \textcolor{Sepia}{\selectlanguage{english}flour used for ceremonies; it must not contain oatmeal. After the ceremony, it is thrown away (not eaten).} \zh{做仪式时所使用的面粉。这种面粉里不要含有燕麦。仪式结束后,面粉被扔掉。}  

\lhead{\firstmark}
\rhead{\botmark}

\subsection{\hspace{-0.5cm} {\Large \textcolor{darkblue}{\textbf{\ipa{dʑɯ˩ʂwæ˩}}}}\hspace{0.5cm}[\kern2pt{\textcolor{darkblue}{\textbf{\ipa{dʑɯ˩ʂwæ˩˥}}}}\kern2pt]} \hypertarget{dz£M\string_Bs`w\{\string_B1}{}
\markboth{\textcolor{darkblue}{\textbf{\ipa{dʑɯ˩ʂwæ˩}}}}{}
\textcolor{teal}{\mytextsc{noun}} \hspace{4pt} Tone: L.
\textcolor{Sepia}{\selectlanguage{english}\textit{Lysimachia christinae Hance} yyyy translation into English.} \zh{过路黄。}  ¶ \textcolor{darkblue}{\textbf{\ipa{dʑɯ˩ʂwæ˩-bæ˥bæ˩}}} \textcolor{Sepia}{\selectlanguage{english}flower of yyyy} \zh{过路黄花}  

\lhead{\firstmark}
\rhead{\botmark}

\subsection{\hspace{-0.5cm} {\Large \textcolor{darkblue}{\textbf{\ipa{dʑɯ˩tɤ˩ɻ̍˥}}}}\hspace{0.5cm}[\kern2pt{\textcolor{darkblue}{\textbf{\ipa{dʑɯ˩tɤ˩ɻ̍˥}}}}\kern2pt]} \hypertarget{dz£M\string_Bt7\string_Br£`̍\string_T1}{}
\markboth{\textcolor{darkblue}{\textbf{\ipa{dʑɯ˩tɤ˩ɻ̍˥}}}}{}
\textcolor{teal}{\mytextsc{adjective}} \hspace{4pt} Tone: L+H\#.
\textcolor{Sepia}{\selectlanguage{english}Humid, moist.} \zh{湿。}  ¶ \textcolor{darkblue}{\textbf{\ipa{dʑɯ˩tɤ˩ɻ̍˥ gv̩˩-ze˩}}} \textcolor{Sepia}{\selectlanguage{english}It got wet.} \zh{湿了!}  

\lhead{\firstmark}
\rhead{\botmark}

\subsection{\hspace{-0.5cm} {\Large \textcolor{darkblue}{\textbf{\ipa{dʑɯ˩tɕʰɯ˩lɑ˩qʰɑ˥}}}}\hspace{0.5cm}[\kern2pt{\textcolor{darkblue}{\textbf{\ipa{dʑɯ˩tɕʰɯ˩lɑ˩qʰɑ˥}}}}\kern2pt]} \hypertarget{dz£M\string_Bts£\string_hM\string_BlA\string_Bq\string_hA\string_T1}{}
\markboth{\textcolor{darkblue}{\textbf{\ipa{dʑɯ˩tɕʰɯ˩lɑ˩qʰɑ˥}}}}{}
\textcolor{teal}{\mytextsc{noun}} \hspace{4pt} Tone: L+H\#.
\textcolor{Sepia}{\selectlanguage{english}Plum.} \zh{一种梅子。} 
\lhead{\firstmark}
\rhead{\botmark}

\subsection{\hspace{-0.5cm} {\Large \textcolor{darkblue}{\textbf{\ipa{dʑɯ˩tɕʰɯ˩lɑ˩qʰæ˥}}}}\hspace{0.5cm}[\kern2pt{\textcolor{darkblue}{\textbf{\ipa{dʑɯ˩tɕʰɯ˩lɑ˩qʰæ˥}}}}\kern2pt]} \hypertarget{dz£M\string_Bts£\string_hM\string_BlA\string_Bq\string_h\{\string_T1}{}
\markboth{\textcolor{darkblue}{\textbf{\ipa{dʑɯ˩tɕʰɯ˩lɑ˩qʰæ˥}}}}{}
\textcolor{teal}{\mytextsc{noun}} \hspace{4pt} Tone: L+H\#.
\textcolor{Sepia}{\selectlanguage{english}Buckthorn, \textit{Hippophae rhamnoides Linn.}.} \zh{沙棘。} 
\lhead{\firstmark}
\rhead{\botmark}

\subsection{\hspace{-0.5cm} {\Large \textcolor{darkblue}{\textbf{\ipa{dʑɯ˩tsʰi˩}}}}\hspace{0.5cm}[\kern2pt{\textcolor{darkblue}{\textbf{\ipa{dʑɯ˩tsʰi˩˥}}}}\kern2pt]} \hypertarget{dz£M\string_Bts\string_hi\string_B1}{}
\markboth{\textcolor{darkblue}{\textbf{\ipa{dʑɯ˩tsʰi˩}}}}{}
\textcolor{teal}{\mytextsc{noun}} \hspace{4pt} Tone: L.
\textcolor{Sepia}{\selectlanguage{english}Boiled water, hot water.} \zh{开水,热水。} 
\lhead{\firstmark}
\rhead{\botmark}

\subsection{\hspace{-0.5cm} {\Large \textcolor{darkblue}{\textbf{\ipa{dʑɯ˩tsʰi˩ʈʰɯ˩di˩}}}}\hspace{0.5cm}[\kern2pt{\textcolor{darkblue}{\textbf{\ipa{dʑɯ˩tsʰi˩ʈʰɯ˩di˩˥}}}}\kern2pt]} \hypertarget{dz£M\string_Bts\string_hi\string_Bt`\string_hM\string_Bdi\string_B1}{}
\markboth{\textcolor{darkblue}{\textbf{\ipa{dʑɯ˩tsʰi˩ʈʰɯ˩di˩}}}}{}
\textcolor{teal}{\mytextsc{noun}} \hspace{4pt} Tone: L.
\textcolor{Sepia}{\selectlanguage{english}Small container for hot water (for 1 person).} \zh{口杯。}  \zh{量词}: \textcolor{darkblue}{\textbf{\ipa{ɭɯ˧}}}  \mytextsc{clf}: \textcolor{darkblue}{\textbf{\ipa{ɭɯ˧}}} 
\lhead{\firstmark}
\rhead{\botmark}

\subsection{\hspace{-0.5cm} {\Large \textcolor{darkblue}{\textbf{\ipa{dʑɯ˩ʈv̩˧}}}}\hspace{0.5cm}[\kern2pt{\textcolor{darkblue}{\textbf{\ipa{dʑɯ˩ʈv̩˥}}}}\kern2pt]} \hypertarget{dz£M\string_Bt`v\string_=\string_M1}{}
\markboth{\textcolor{darkblue}{\textbf{\ipa{dʑɯ˩ʈv̩˧}}}}{}
\textcolor{teal}{\mytextsc{adjective}} \hspace{4pt} Tone: LM.
\textcolor{Sepia}{\selectlanguage{english}To be a hunchback/humpback.} \zh{驼背(严重的病)。}  ¶ \textcolor{darkblue}{\textbf{\ipa{dʑɯ˩ʈv̩˧-ze˩}}} \textcolor{Sepia}{\selectlanguage{english}\mytextsc{pfv}} \zh{驼背了}  

\lhead{\firstmark}
\rhead{\botmark}

\subsection{\hspace{-0.5cm} {\Large \textcolor{darkblue}{\textbf{\ipa{dʑɯ˩zo˩}}}}\hspace{0.5cm}[\kern2pt{\textcolor{darkblue}{\textbf{\ipa{dʑɯ˩zo˩˥}}}}\kern2pt]} \hypertarget{dz£M\string_Bzo\string_B1}{}
\markboth{\textcolor{darkblue}{\textbf{\ipa{dʑɯ˩zo˩}}}}{}
\textcolor{teal}{\mytextsc{noun}} \hspace{4pt} Tone: L.
\textcolor{Sepia}{\selectlanguage{english}Brook, rivulet.} \zh{溪流。}  \zh{量词}: \textcolor{darkblue}{\textbf{\ipa{kʰɯ˩}}}  \mytextsc{clf}: \textcolor{darkblue}{\textbf{\ipa{kʰɯ˩}}} 
\lhead{\firstmark}
\rhead{\botmark}

\subsection{\hspace{-0.5cm} {\Large \textcolor{darkblue}{\textbf{\ipa{dʑɯ˩ʐv̩˩}}}}\hspace{0.5cm}[\kern2pt{\textcolor{darkblue}{\textbf{\ipa{dʑɯ˩ʐv̩˩˥}}}}\kern2pt]} \hypertarget{dz£M\string_Bz`v\string_=\string_B1}{}
\markboth{\textcolor{darkblue}{\textbf{\ipa{dʑɯ˩ʐv̩˩}}}}{}
\textcolor{teal}{\mytextsc{verb}} \hspace{4pt} Tone: L.
\textcolor{Sepia}{\selectlanguage{english}To swim.} \zh{游泳。} 
\lhead{\firstmark}
\rhead{\botmark}

\subsection{\hspace{-0.5cm} {\Large \textcolor{darkblue}{\textbf{\ipa{dʑɯ˧˥}}}}\hspace{0.5cm}[\kern2pt{\textcolor{darkblue}{\textbf{\ipa{dʑɯ˥}}}}\kern2pt]} \hypertarget{dz£M\string_M\string_T1}{}
\markboth{\textcolor{darkblue}{\textbf{\ipa{dʑɯ˧˥}}}}{}
\textcolor{teal}{\mytextsc{adjective}} \hspace{4pt} Tone: MH.
\textcolor{Sepia}{\selectlanguage{english}Many, much.} \zh{多。}  ¶ \textcolor{darkblue}{\textbf{\ipa{hĩ˧ dʑɯ˩}}} \textcolor{Sepia}{\selectlanguage{english}There are many people.} \zh{人多。}  

\lhead{\firstmark}
\rhead{\botmark}

\newpage
\section*{\centering- \textcolor{darkblue}{\textbf{\ipa{ɖ}}} -}
\subsection{\hspace{-0.5cm} {\Large \textcolor{darkblue}{\textbf{\ipa{ɖæ˥}}}}\hspace{0.5cm}[\kern2pt{\textcolor{darkblue}{\textbf{\ipa{ɖæ˥}}}}\kern2pt]} \hypertarget{d`\{\string_T1}{}
\markboth{\textcolor{darkblue}{\textbf{\ipa{ɖæ˥}}}}{}
\textcolor{teal}{\mytextsc{adjective}} \hspace{4pt} Tone: H.
\textcolor{Sepia}{\selectlanguage{english}Short.} \zh{短。} 
\lhead{\firstmark}
\rhead{\botmark}

\subsection{\hspace{-0.5cm} {\Large \textcolor{darkblue}{\textbf{\ipa{ɖæ˧\textasciitilde{}ɖæ˩}}}}\hspace{0.5cm}[\kern2pt{\textcolor{darkblue}{\textbf{\ipa{ɖæ˧ɖæ˩}}}}\kern2pt]} \hypertarget{d`\{\string_M~d`\{\string_B1}{}
\markboth{\textcolor{darkblue}{\textbf{\ipa{ɖæ˧\textasciitilde{}ɖæ˩}}}}{}
\textcolor{teal}{\mytextsc{adjective}} \hspace{4pt} Tone: L\#.
\textcolor{Sepia}{\selectlanguage{english}Horizontal.} \zh{横着(横躺在路上)。}  ¶ \textcolor{darkblue}{\textbf{\ipa{ɖæ˧\textasciitilde{}ɖæ˩ | tʰi˧-tɕɯ˥}}} \textcolor{Sepia}{\selectlanguage{english}to lay flat} \zh{横着放}  

\lhead{\firstmark}
\rhead{\botmark}

\subsection{\hspace{-0.5cm} {\Large \textcolor{darkblue}{\textbf{\ipa{ɖæ˩\textsubscript{a}}}}}\hspace{0.5cm}[\kern2pt{\textcolor{darkblue}{\textbf{\ipa{ɖæ˩˥}}}}\kern2pt]} \hypertarget{d`\{\string_Ba1}{}
\markboth{\textcolor{darkblue}{\textbf{\ipa{ɖæ˩\textsubscript{a}}}}}{}
\textcolor{teal}{\mytextsc{classifier}} \hspace{4pt} Tone: L\textsubscript{a}.
\textcolor{Sepia}{\selectlanguage{english}A section of (road); a bolt of cloth.} \zh{量词:路(段)/布(匹)。}  ¶ \textcolor{darkblue}{\textbf{\ipa{ʐɤ˩mi˩˥ | ɖɯ˧-ɖæ˩}}} \textcolor{Sepia}{\selectlanguage{english}a section of road, a stretch of road} \zh{一段路}  
 ¶ \textcolor{darkblue}{\textbf{\ipa{ɲi˧, ɲi˩-ɖæ˩! |}}} \textcolor{Sepia}{\selectlanguage{english}Two stretches a day! (Set phrase: in one day, one can cover a distance of two 'stretches'. If one can get somewhere before lunch, the distance counts as one stretch/section; if one can only arrive there in the afternoon, it counts as two stretches/sections.)} \zh{一天两段路!(说明:早上出发,如果午饭前能到目的地,距离算是一段,如果下午晚上才到,算两段。)}  
 ¶ \textcolor{darkblue}{\textbf{\ipa{ʈʂʰɯ˧-ɖæ˥}}} \textcolor{Sepia}{\selectlanguage{english}\mytextsc{dem} \string_ (tone: H\# / H\$)} \zh{\mytextsc{指示代词} \string_}  

\lhead{\firstmark}
\rhead{\botmark}

\subsection{\hspace{-0.5cm} {\Large \textcolor{darkblue}{\textbf{\ipa{ɖæ˩\textsubscript{a}}}}}\hspace{0.5cm}[\kern2pt{\textcolor{darkblue}{\textbf{\ipa{ɖæ˩˥}}}}\kern2pt]} \hypertarget{d`\{\string_Ba1}{}
\markboth{\textcolor{darkblue}{\textbf{\ipa{ɖæ˩\textsubscript{a}}}}}{}
\textcolor{teal}{\mytextsc{verb}} \hspace{4pt} Tone: L\textsubscript{a}.
\textcolor{Sepia}{\selectlanguage{english}To pass over, to cross (a river on a boat, a mountain…).} \zh{渡(坐船渡河……)。}  ¶ \textcolor{darkblue}{\textbf{\ipa{dʑɯ˩ ɖæ˩˥ / dʑɯ˩ ɖæ˩-ze˥}}} \textcolor{Sepia}{\selectlanguage{english}to cross a river} \zh{渡河}  
 ¶ \textcolor{darkblue}{\textbf{\ipa{dʑɯ˧ | ɖɯ˧-kʰɯ˩ ɖæ˩}}} \textcolor{Sepia}{\selectlanguage{english}as above} \zh{同上}  

\lhead{\firstmark}
\rhead{\botmark}

\subsection{\hspace{-0.5cm} {\Large \textcolor{darkblue}{\textbf{\ipa{ɖæ˩dʑɯ˥}}}}\hspace{0.5cm}[\kern2pt{\textcolor{darkblue}{\textbf{\ipa{ɖæ˩dʑɯ˥}}}}\kern2pt]} \hypertarget{d`\{\string_Bdz£M\string_T1}{}
\markboth{\textcolor{darkblue}{\textbf{\ipa{ɖæ˩dʑɯ˥}}}}{}
\textcolor{teal}{\mytextsc{noun}} \hspace{4pt} Tone: LH.
\textcolor{Sepia}{\selectlanguage{english}Dirt, filth.} \zh{污垢。}  \zh{量词}: \textcolor{darkblue}{\textbf{\ipa{ʁwɤ˧, etc}}}  \mytextsc{clf}: \textcolor{darkblue}{\textbf{\ipa{ʁwɤ˧, etc}}} 
\lhead{\firstmark}
\rhead{\botmark}

\subsection{\hspace{-0.5cm} {\Large \textcolor{darkblue}{\textbf{\ipa{ɖæ˩-lɑ˧so˧}}}}\hspace{0.5cm}[\kern2pt{\textcolor{darkblue}{\textbf{\ipa{ɖæ˧lɑ˧so˧}}}}\kern2pt]} \hypertarget{d`\{\string_B-lA\string_Mso\string_M1}{}
\markboth{\textcolor{darkblue}{\textbf{\ipa{ɖæ˩-lɑ˧so˧}}}}{}
\textcolor{teal}{\mytextsc{noun}} \hspace{4pt} Tone: L-.
\textcolor{Sepia}{\selectlanguage{english}Name of a ceremony conducted at home once a year, during the first two weeks of the year, by one or two monks invited to the farm: offering grain (or fruit) to the gods. The aim is to ensure prosperity for the household.} \zh{一种祈福仪式,和尚在过年时主持行礼。}  Borrowing: Tibetan  'bras lha gsol

\lhead{\firstmark}
\rhead{\botmark}

\subsection{\hspace{-0.5cm} {\Large \textcolor{darkblue}{\textbf{\ipa{ɖæ˩mi˧}}}}\hspace{0.5cm}[\kern2pt{\textcolor{darkblue}{\textbf{\ipa{ɖæ˩mi˥}}}}\kern2pt]} \hypertarget{d`\{\string_Bmi\string_M1}{}
\markboth{\textcolor{darkblue}{\textbf{\ipa{ɖæ˩mi˧}}}}{}
\textcolor{teal}{\mytextsc{noun}} \hspace{4pt} Tone: LM.
\textcolor{Sepia}{\selectlanguage{english}The Yongning monastery.} \zh{永宁大寺。}  Borrowing: Tibetan  dgra med
 ¶ \textcolor{darkblue}{\textbf{\ipa{ɖæ˩mi˧-ʈæ˩bɤ˩}}} \textcolor{Sepia}{\selectlanguage{english}a priest from the monastery} \zh{永宁大寺的和尚}  

\lhead{\firstmark}
\rhead{\botmark}

\subsection{\hspace{-0.5cm} {\Large \textcolor{darkblue}{\textbf{\ipa{ɖæ˩mi˧-go˧bɤ˩}}}}\hspace{0.5cm}[\kern2pt{\textcolor{darkblue}{\textbf{\ipa{ɖæ˩mi˧go˧bɤ˩}}}}\kern2pt]} \hypertarget{d`\{\string_Bmi\string_M-go\string_Mb7\string_B1}{}
\markboth{\textcolor{darkblue}{\textbf{\ipa{ɖæ˩mi˧-go˧bɤ˩}}}}{}
\textcolor{teal}{\mytextsc{noun}} \hspace{4pt} Tone: LM-L\#.
\textcolor{Sepia}{\selectlanguage{english}Yongning temple.} \zh{永宁大寺。}  Borrowing: Tibetan  dgra med dgon pa

\lhead{\firstmark}
\rhead{\botmark}

\subsection{\hspace{-0.5cm} {\Large \textcolor{darkblue}{\textbf{\ipa{ɖæ˩pʰv̩˥}}}}\hspace{0.5cm}[\kern2pt{\textcolor{darkblue}{\textbf{\ipa{ɖæ˩pʰv̩˥}}}}\kern2pt]} \hypertarget{d`\{\string_Bp\string_hv\string_=\string_T1}{}
\markboth{\textcolor{darkblue}{\textbf{\ipa{ɖæ˩pʰv̩˥}}}}{}
\textcolor{teal}{\mytextsc{noun}} \hspace{4pt} Tone: LH.
\textcolor{Sepia}{\selectlanguage{english}Dust, dirt.} \zh{灰尘。}  \zh{量词}: \textcolor{darkblue}{\textbf{\ipa{ti˧˥}}}  \mytextsc{clf}: \textcolor{darkblue}{\textbf{\ipa{ti˧˥}}} 
\lhead{\firstmark}
\rhead{\botmark}

\subsection{\hspace{-0.5cm} {\Large \textcolor{darkblue}{\textbf{\ipa{ɖæ˩ʂɯ\#˥}}}}\hspace{0.5cm}[\kern2pt{\textcolor{darkblue}{\textbf{\ipa{ɖæ˩ʂɯ˥}}}}\kern2pt]} \hypertarget{d`\{\string_Bs`M\#\string_T1}{}
\markboth{\textcolor{darkblue}{\textbf{\ipa{ɖæ˩ʂɯ\#˥}}}}{}
\textcolor{teal}{\mytextsc{noun}} \hspace{4pt} Tone: LM+\#H.
\textcolor{Sepia}{\selectlanguage{english}A village name.} \zh{扎实(永宁的一个村落)。}  ¶ \textcolor{darkblue}{\textbf{\ipa{ɖæ˩ʂɯ˧-ʁwɤ\#˥}}} \textcolor{Sepia}{\selectlanguage{english}same meaning} \zh{同上:扎实村}  
 ¶ \textcolor{darkblue}{\textbf{\ipa{ɖæ˩ʂɯ\#˥, | ʈʂo˧ʂɯ\#˥, | bɤ˩tɕʰɯ˩˥, | dɑ˧pʰo˥, | bɤ˧dzi˩, | dze˧bo˧}}} \textcolor{Sepia}{\selectlanguage{english}the six villages of the plain of Yongning, in traditional order: by order of increasing distance from the Lake} \zh{永宁坝的六个村落,按传统排序:从距离泸沽湖最近的村落说起。}  

\lhead{\firstmark}
\rhead{\botmark}

\subsection{\hspace{-0.5cm} {\Large \textcolor{darkblue}{\textbf{\ipa{ɖæ˩˧}}}}\hspace{0.5cm}[\kern2pt{\textcolor{darkblue}{\textbf{\ipa{ɖæ˩˥}}}}\kern2pt]} \hypertarget{d`\{\string_B\string_M1}{}
\markboth{\textcolor{darkblue}{\textbf{\ipa{ɖæ˩˧}}}}{}
\textcolor{teal}{\mytextsc{noun}} \hspace{4pt} Tone: LM.
\ding{202} \textcolor{Sepia}{\selectlanguage{english}Dust.} \zh{尘土。}  ¶ \textcolor{darkblue}{\textbf{\ipa{ɖæ˩˥ | ɖɯ˧-ti˧ tʰi˧-di˥}}} \textcolor{Sepia}{\selectlanguage{english}there is a layer of dust} \zh{有一层灰}  
 ¶ \textcolor{darkblue}{\textbf{\ipa{ɖæ˩ bæ˧}}} \textcolor{Sepia}{\selectlanguage{english}to sweep the dust} \zh{扫灰}  
 \zh{量词}: \textcolor{darkblue}{\textbf{\ipa{ti˧˥}}} \ding{203} \textcolor{Sepia}{\selectlanguage{english}Dirt, filth.} \zh{污垢。}  \mytextsc{clf}: \textcolor{darkblue}{\textbf{\ipa{ti˧˥}}} 
\lhead{\firstmark}
\rhead{\botmark}

\subsection{\hspace{-0.5cm} {\Large \textcolor{darkblue}{\textbf{\ipa{ɖɤ˥}}}}\hspace{0.5cm}[\kern2pt{\textcolor{darkblue}{\textbf{\ipa{ɖɤ˥}}}}\kern2pt]} \hypertarget{d`7\string_T1}{}
\markboth{\textcolor{darkblue}{\textbf{\ipa{ɖɤ˥}}}}{}
\textcolor{teal}{\mytextsc{verb}} \hspace{4pt} Tone: H.
\textcolor{Sepia}{\selectlanguage{english}To crawl, to creep.} \zh{爬行,匍匐。}  ¶ \textcolor{darkblue}{\textbf{\ipa{ɖɤ˧\textasciitilde{}ɖɤ˧ (-ze˩)}}} \textcolor{Sepia}{\selectlanguage{english}\mytextsc{red}} \zh{\mytextsc{重叠:爬一爬}}  
 ¶ \textcolor{darkblue}{\textbf{\ipa{ʈʂʰɯ˧ | ɖɤ˧\textasciitilde{}ɖɤ˧-ʁo˧-ze˩!}}} \textcolor{Sepia}{\selectlanguage{english}She can crawl! / She knows how to crawl! (About a baby that crawls around.)} \zh{她会爬了!}  

\lhead{\firstmark}
\rhead{\botmark}

\subsection{\hspace{-0.5cm} {\Large \textcolor{darkblue}{\textbf{\ipa{ɖɤ˧mi˧}}}}\hspace{0.5cm}[\kern2pt{\textcolor{darkblue}{\textbf{\ipa{ɖɤ˧mi˧}}}}\kern2pt]} \hypertarget{d`7\string_Mmi\string_M1}{}
\markboth{\textcolor{darkblue}{\textbf{\ipa{ɖɤ˧mi˧}}}}{}
\textcolor{teal}{\mytextsc{noun}} \hspace{4pt} Tone: M.
\textcolor{Sepia}{\selectlanguage{english}Fox.} \zh{狐狸。}  ¶ \textcolor{darkblue}{\textbf{\ipa{ɖɤ˧mi˧-zo\#˥}}} \textcolor{Sepia}{\selectlanguage{english}little fox, baby fox} \zh{小狐狸}  
 ¶ \textcolor{darkblue}{\textbf{\ipa{ɖɤ˧mi˧-pʰv̩\#˥}}} \textcolor{Sepia}{\selectlanguage{english}male fox} \zh{公狐狸}  
 ¶ \textcolor{darkblue}{\textbf{\ipa{ɖɤ˧mi˧, | mi˩ ɲi˥!}}} \textcolor{Sepia}{\selectlanguage{english}This fox is a female!} \zh{这只狐狸是母的!}  
 \zh{量词}: \textcolor{darkblue}{\textbf{\ipa{pʰo˧˥}}}  \mytextsc{clf}: \textcolor{darkblue}{\textbf{\ipa{pʰo˧˥}}} 
\lhead{\firstmark}
\rhead{\botmark}

\subsection{\hspace{-0.5cm} {\Large \textcolor{darkblue}{\textbf{\ipa{ɖɤ˩\textsubscript{a}}}}}\hspace{0.5cm}[\kern2pt{\textcolor{darkblue}{\textbf{\ipa{ɖɤ˩˥}}}}\kern2pt]} \hypertarget{d`7\string_Ba1}{}
\markboth{\textcolor{darkblue}{\textbf{\ipa{ɖɤ˩\textsubscript{a}}}}}{}
\textcolor{teal}{\mytextsc{adjective}} \hspace{4pt} Tone: L\textsubscript{a}.
\textcolor{Sepia}{\selectlanguage{english}Hot (weather).} \zh{很热(天气),阳光强烈。}  ¶ \textcolor{darkblue}{\textbf{\ipa{ɲi˧mi˧ | ɖɤ˩-ze˥!}}} \textcolor{Sepia}{\selectlanguage{english}The sun is burning hot, scalding} \zh{太阳很大、很强烈}  
 ¶ \textcolor{darkblue}{\textbf{\ipa{ɖɤ˩-hĩ˩˥}}} \textcolor{Sepia}{\selectlanguage{english}\mytextsc{rel}} \zh{热的}  

\lhead{\firstmark}
\rhead{\botmark}

\subsection{\hspace{-0.5cm} {\Large \textcolor{darkblue}{\textbf{\ipa{ɖo˧}}}}\hspace{0.5cm}[\kern2pt{\textcolor{darkblue}{\textbf{\ipa{ɖo˥}}}}\kern2pt]} \hypertarget{d`o\string_M1}{}
\markboth{\textcolor{darkblue}{\textbf{\ipa{ɖo˧}}}}{}
\textcolor{teal}{\mytextsc{verb}} \hspace{4pt} Tone: M intrans.
\textcolor{Sepia}{\selectlanguage{english}To allow, to permit; also: to order about; to run errands for.} \zh{让,指使、使唤。}  ¶ \textcolor{darkblue}{\textbf{\ipa{po˧ mɤ˧-ɖo˧!}}} \textcolor{Sepia}{\selectlanguage{english}(You) are not allowed to take it! / You must not take it! (eg telling a child that (s)he is not allowed to take a knife)} \zh{不许拿!}  
 ¶ \textcolor{darkblue}{\textbf{\ipa{ʈʂʰɯ˧, | po˧ ɖo˧!}}} \textcolor{Sepia}{\selectlanguage{english}That one, you can have it / you can take it / you can play with it! (Context: as above: telling a child what (s)he can and cannot toy with.)} \zh{那个,是可以拿的! / 那个,是可以玩的!(情景同上:告诉一个小孩子什么东西可以拿,什么不可以拿。)}  
 ¶ \textcolor{darkblue}{\textbf{\ipa{gɤ˩ do˧ mɤ˧-ɖo˧!}}} \textcolor{Sepia}{\selectlanguage{english}(You) are not allowed to climb (on the table,...)} \zh{不许爬上(桌子……)}  
 ¶ \textcolor{darkblue}{\textbf{\ipa{lɑ˧-kʰv̩˧˥, | ʑi˧qʰwɤ˧ tsʰi˧-mɤ˧-ɖo˧! | ʑi˩-kʰv̩˩˥, | ʑi˧qʰwɤ˧ tsʰi˧-mɤ˧-ɖo˧! |}}} \textcolor{Sepia}{\selectlanguage{english}(During) the year of the Tiger, one should not build a house! (During) the year of the Monkey, one should not build a house! (These years are considered too “hard”, \textcolor{darkblue}{\textbf{\ipa{/wu˧/}}}, by astrology.)} \zh{虎年,不要建房!猴年,不要建房!(这样的年,被认为是太‘硬’的。)}  
 ¶ \textcolor{darkblue}{\textbf{\ipa{ʝi˧ mɤ˧-ɖo˧!}}} \textcolor{Sepia}{\selectlanguage{english}(One) must not do (that)!} \zh{不要做!}  

\lhead{\firstmark}
\rhead{\botmark}

\subsection{\hspace{-0.5cm} {\Large \textcolor{darkblue}{\textbf{\ipa{ɖɯ˧-}}}}\hspace{0.5cm}[\kern2pt{\textcolor{darkblue}{\textbf{\ipa{ɖɯ˥}}}}\kern2pt]} \hypertarget{d`M\string_M-1}{}
\markboth{\textcolor{darkblue}{\textbf{\ipa{ɖɯ˧-}}}}{}
\textcolor{teal}{\mytextsc{preposition}} \hspace{4pt} Tone: M.
\textcolor{Sepia}{\selectlanguage{english}\mytextsc{delimitative}.} \zh{\mytextsc{进行时态。}} 
\lhead{\firstmark}
\rhead{\botmark}

\subsection{\hspace{-0.5cm} {\Large \textcolor{darkblue}{\textbf{\ipa{ɖɯ˧\textsubscript{b}}}}}\hspace{0.5cm}[\kern2pt{\textcolor{darkblue}{\textbf{\ipa{ɖɯ˥}}}}\kern2pt]} \hypertarget{d`M\string_Mb1}{}
\markboth{\textcolor{darkblue}{\textbf{\ipa{ɖɯ˧\textsubscript{b}}}}}{}
\textcolor{teal}{\mytextsc{verb}} \hspace{4pt} Tone: M\textsubscript{b}.
\textcolor{Sepia}{\selectlanguage{english}To obtain, to get.} \zh{得到。}  ¶ \textcolor{darkblue}{\textbf{\ipa{le˧-ʂe˧ le˧-ɖɯ˧-ze˧!}}} \textcolor{Sepia}{\selectlanguage{english}(I) have looked for something and found it! / I have found (something by looking around for it)!} \zh{找到了!}  
 ¶ \textcolor{darkblue}{\textbf{\ipa{ɖɯ˧-tʰɑ˧˥!}}} \textcolor{Sepia}{\selectlanguage{english}It is possible to obtain it! / It can be obtained!} \zh{可以得到的!}  
 ¶ \textcolor{darkblue}{\textbf{\ipa{ɖɯ˧-tʰɑ˧-ze˥!}}} \textcolor{Sepia}{\selectlanguage{english}We have managed to obtain it! / We found it possible to obtain it!} \zh{(我们)成功地得到了!}  
 ¶ \textcolor{darkblue}{\textbf{\ipa{tso˧\textasciitilde{}tso˧ ɖɯ˧ (+ze˧)}}} \textcolor{Sepia}{\selectlanguage{english}to obtain something} \zh{得到东西}  

\lhead{\firstmark}
\rhead{\botmark}

\subsection{\hspace{-0.5cm} {\Large \textcolor{darkblue}{\textbf{\ipa{ɖɯ˧-ɬi˧mi˧}}}}\hspace{0.5cm}[\kern2pt{\textcolor{darkblue}{\textbf{\ipa{xxxx non-correspondance entre le nombre de morphèmes et le nombre de tons de morphèmes}}}}\kern2pt]} \hypertarget{d`M\string_M-Ki\string_Mmi\string_M1}{}
\markboth{\textcolor{darkblue}{\textbf{\ipa{ɖɯ˧-ɬi˧mi˧}}}}{}
\textcolor{teal}{\mytextsc{noun}} \hspace{4pt} Tone: M.
\textcolor{Sepia}{\selectlanguage{english}1st month.} \zh{正月。} 
\lhead{\firstmark}
\rhead{\botmark}

\subsection{\hspace{-0.5cm} {\Large \textcolor{darkblue}{\textbf{\ipa{ɖɯ˧-njɤ˧}}}}\hspace{0.5cm}[\kern2pt{\textcolor{darkblue}{\textbf{\ipa{xxxx non-correspondance entre le nombre de morphèmes et le nombre de tons de morphèmes}}}}\kern2pt]} \hypertarget{d`M\string_M-nj7\string_M1}{}
\markboth{\textcolor{darkblue}{\textbf{\ipa{ɖɯ˧-njɤ˧}}}}{}
\textcolor{teal}{\mytextsc{adverb(ial)}} \hspace{4pt} Tone: M.
\textcolor{Sepia}{\selectlanguage{english}Continuously, ceaselessly.} \zh{一直、一直不停。}  ¶ \textcolor{darkblue}{\textbf{\ipa{ɖɯ˧-njɤ˧ | so˩˥}}} \textcolor{Sepia}{\selectlanguage{english}to study ceaselessly} \zh{一直不停地学习}  
 ¶ \textcolor{darkblue}{\textbf{\ipa{ɖɯ˧-njɤ˧ | lo˧ ʝi˧}}} \textcolor{Sepia}{\selectlanguage{english}to work ceaselessly} \zh{一直不停地工作}  
 ¶ \textcolor{darkblue}{\textbf{\ipa{ɖɯ˧-njɤ˧-zo˥}}} \textcolor{Sepia}{\selectlanguage{english}often} \zh{经常、常}  
 ¶ \textcolor{darkblue}{\textbf{\ipa{ɖɯ˧-njɤ˧ hwæ˩; ɖɯ˧-njɤ˧ tɕʰi˧; ɖɯ˧-njɤ˧ dzɯ˧; ɖɯ˧-njɤ˧ dze˧˥; ɖɯ˧-njɤ˧ ʐwɤ˧˥; ɖɯ˧-njɤ˧ lɑ˧˥}}} \textcolor{Sepia}{\selectlanguage{english}combinations with verbs in the six tones: to buy, to sell, to eat, to cut, to speak, to strike} \zh{跟六个调类的动词结合:买,卖,吃,切,说,打}  
 ¶ \textcolor{darkblue}{\textbf{\ipa{ɖɯ˧-njɤ˧ | hwæ˧; ɖɯ˧-njɤ˧ | tɕʰi˧; ɖɯ˧-njɤ˧ | dzɯ˧; ɖɯ˧-njɤ˧ | dze˩˥; ɖɯ˧-njɤ˧ | ʐwɤ˩˥; ɖɯ˧-njɤ˧ | lɑ˧˥}}} \textcolor{Sepia}{\selectlanguage{english}combinations with verbs in the six tones: to buy, to sell, to eat, to cut, to speak, to strike (separating into two tone groups)} \zh{跟六个调类的动词结合:买,卖,吃,切,说,打}  

\lhead{\firstmark}
\rhead{\botmark}

\subsection{\hspace{-0.5cm} {\Large \textcolor{darkblue}{\textbf{\ipa{ɖɯ˧-ɲi˧-ɖɯ˥-hɑ̃˩}}}}\hspace{0.5cm}[\kern2pt{\textcolor{darkblue}{\textbf{\ipa{xxxx non-correspondance entre le nombre de morphèmes et le nombre de tons de morphèmes}}}}\kern2pt]} \hypertarget{d`M\string_M-Ji\string_M-d`M\string_T-hA\string_~\string_B1}{}
\markboth{\textcolor{darkblue}{\textbf{\ipa{ɖɯ˧-ɲi˧-ɖɯ˥-hɑ̃˩}}}}{}
\textcolor{teal}{\mytextsc{noun}} \hspace{4pt} Tone: \#H-.
\textcolor{Sepia}{\selectlanguage{english}One day and one night.} \zh{一天一夜。} 
\lhead{\firstmark}
\rhead{\botmark}

\subsection{\hspace{-0.5cm} {\Large \textcolor{darkblue}{\textbf{\ipa{ɖɯ˧-so˩}}}}\hspace{0.5cm}[\kern2pt{\textcolor{darkblue}{\textbf{\ipa{xxxx non-correspondance entre le nombre de morphèmes et le nombre de tons de morphèmes}}}}\kern2pt]} \hypertarget{d`M\string_M-so\string_B1}{}
\markboth{\textcolor{darkblue}{\textbf{\ipa{ɖɯ˧-so˩}}}}{}
\textcolor{teal}{\mytextsc{noun}} \hspace{4pt} Tone: L\textsubscript{a}.
\textcolor{Sepia}{\selectlanguage{english}Some, a few. Made up of 'one' and 'three'.} \zh{一些、两三个(直译:‘一三(个)’。}  ¶ \textcolor{darkblue}{\textbf{\ipa{hĩ˧ | ɖɯ˧-so˩ kv̩˩}}} \textcolor{Sepia}{\selectlanguage{english}a few people} \zh{几个人}  
 ¶ \textcolor{darkblue}{\textbf{\ipa{ɖɯ˧-so˩ ɲi˩}}} \textcolor{Sepia}{\selectlanguage{english}a few days} \zh{几天}  

\lhead{\firstmark}
\rhead{\botmark}

\subsection{\hspace{-0.5cm} {\Large \textcolor{darkblue}{\textbf{\ipa{ɖɯ˧ʈæ˩}}}}\hspace{0.5cm}[\kern2pt{\textcolor{darkblue}{\textbf{\ipa{ɖɯ˧ʈæ˩}}}}\kern2pt]} \hypertarget{d`M\string_Mt`\{\string_B1}{}
\markboth{\textcolor{darkblue}{\textbf{\ipa{ɖɯ˧ʈæ˩}}}}{}
\textcolor{teal}{\mytextsc{noun}} \hspace{4pt} Tone: L\#.
\textcolor{Sepia}{\selectlanguage{english}A ritual performed by monks after someone's decease.} \zh{一个葬礼仪式,由和尚主持。} 
\lhead{\firstmark}
\rhead{\botmark}

\subsection{\hspace{-0.5cm} {\Large \textcolor{darkblue}{\textbf{\ipa{ɖɯ˩\textsubscript{a}}}}}\hspace{0.5cm}[\kern2pt{\textcolor{darkblue}{\textbf{\ipa{ɖɯ˩˥}}}}\kern2pt]} \hypertarget{d`M\string_Ba1}{}
\markboth{\textcolor{darkblue}{\textbf{\ipa{ɖɯ˩\textsubscript{a}}}}}{}
\textcolor{teal}{\mytextsc{adjective}} \hspace{4pt} Tone: L\textsubscript{a}.
\textcolor{Sepia}{\selectlanguage{english}Big, large.} \zh{大。}  ¶ \textcolor{darkblue}{\textbf{\ipa{mɤ˧-ɖɯ˩}}} \textcolor{Sepia}{\selectlanguage{english}\mytextsc{neg}} \zh{\mytextsc{neg}}  
 ¶ \textcolor{darkblue}{\textbf{\ipa{ɖɯ˩-hĩ˩˥}}} \textcolor{Sepia}{\selectlanguage{english}\mytextsc{rel}} \zh{\mytextsc{rel}}  
 ¶ \textcolor{darkblue}{\textbf{\ipa{le˧-ɖɯ˩(-ze˩)}}} \textcolor{Sepia}{\selectlanguage{english}\mytextsc{accomp}+\mytextsc{pfv}} \zh{\mytextsc{accomp}+\mytextsc{pfv}}  
 ¶ \textcolor{darkblue}{\textbf{\ipa{ə˧pɤ˥ɖɯ˩-gv̩˩}}} \textcolor{Sepia}{\selectlanguage{english}very big} \zh{好大、大得很}  

\lhead{\firstmark}
\rhead{\botmark}

\subsection{\hspace{-0.5cm} {\Large \textcolor{darkblue}{\textbf{\ipa{ɖɯ˩ɖʐɯ˧}}}}\hspace{0.5cm}[\kern2pt{\textcolor{darkblue}{\textbf{\ipa{ɖɯ˩ɖʐɯ˥}}}}\kern2pt]} \hypertarget{d`M\string_Bd`z`M\string_M1}{}
\markboth{\textcolor{darkblue}{\textbf{\ipa{ɖɯ˩ɖʐɯ˧}}}}{}
\textcolor{teal}{\mytextsc{noun}} \hspace{4pt} Tone: LM.
\textcolor{Sepia}{\selectlanguage{english}Masculine given name.} \zh{男性名字:独知。}  Borrowing: Tibetan

\lhead{\firstmark}
\rhead{\botmark}

\subsection{\hspace{-0.5cm} {\Large \textcolor{darkblue}{\textbf{\ipa{ɖɯ˩ɖʐɯ˧-tsʰɯ˩ɻ̍˩}}}}\hspace{0.5cm}[\kern2pt{\textcolor{darkblue}{\textbf{\ipa{ɖɯ˩ɖʐɯ˧tsʰɯ˩ɻ̍˩}}}}\kern2pt]} \hypertarget{d`M\string_Bd`z`M\string_M-ts\string_hM\string_Br£`̍\string_B1}{}
\markboth{\textcolor{darkblue}{\textbf{\ipa{ɖɯ˩ɖʐɯ˧-tsʰɯ˩ɻ̍˩}}}}{}
\textcolor{teal}{\mytextsc{noun}} \hspace{4pt} Tone: LM-L.
\textcolor{Sepia}{\selectlanguage{english}Masculine given name.} \zh{男性名字。} 
\lhead{\firstmark}
\rhead{\botmark}

\subsection{\hspace{-0.5cm} {\Large \textcolor{darkblue}{\textbf{\ipa{ɖɯ˩hĩ˩}}}}\hspace{0.5cm}[\kern2pt{\textcolor{darkblue}{\textbf{\ipa{ɖɯ˩hĩ˩˥}}}}\kern2pt]} \hypertarget{d`M\string_Bhi\string_~\string_B1}{}
\markboth{\textcolor{darkblue}{\textbf{\ipa{ɖɯ˩hĩ˩}}}}{}
\textcolor{teal}{\mytextsc{noun}} \hspace{4pt} Tone: L.
\textcolor{Sepia}{\selectlanguage{english}Important people (including elders).} \zh{大人、重要的人(包括长辈)。}  \zh{量词}: \textcolor{darkblue}{\textbf{\ipa{v̩˧}}}  \mytextsc{clf}: \textcolor{darkblue}{\textbf{\ipa{v̩˧}}} 
\lhead{\firstmark}
\rhead{\botmark}

\subsection{\hspace{-0.5cm} {\Large \textcolor{darkblue}{\textbf{\ipa{ɖɯ˩lo\#˥}}}}\hspace{0.5cm}[\kern2pt{\textcolor{darkblue}{\textbf{\ipa{ɖɯ˩lo˥}}}}\kern2pt]} \hypertarget{d`M\string_Blo\#\string_T1}{}
\markboth{\textcolor{darkblue}{\textbf{\ipa{ɖɯ˩lo\#˥}}}}{}
\textcolor{teal}{\mytextsc{noun}} \hspace{4pt} Tone: LM+\#H.
\ding{202} \textcolor{Sepia}{\selectlanguage{english}Tradition.} \zh{传统。}  ¶ \textcolor{darkblue}{\textbf{\ipa{ɖɯ˩lo˧ ɖɯ˧-kʰwɤ˥ | tʰi˧-so˥-ɻ̍˩}}} \textcolor{Sepia}{\selectlanguage{english}to teach a custom} \zh{教授一个传统、一个习俗}  
 \zh{量词}: \textcolor{darkblue}{\textbf{\ipa{kʰwɤ˥}}} \ding{203} \textcolor{Sepia}{\selectlanguage{english}Good manners.} \zh{礼仪、礼貌。}  ¶ \textcolor{darkblue}{\textbf{\ipa{ʈʂʰɯ˧ | ɖɯ˩lo˧ dʑɤ˥!}}} \textcolor{Sepia}{\selectlanguage{english}(S)he knows the customs / (s)he has good manners} \zh{他懂礼貌、他会做人}  
\ding{204} \textcolor{Sepia}{\selectlanguage{english}The order of things.} \zh{道理。}  \mytextsc{clf}: \textcolor{darkblue}{\textbf{\ipa{kʰwɤ˥}}} 
\lhead{\firstmark}
\rhead{\botmark}

\subsection{\hspace{-0.5cm} {\Large \textcolor{darkblue}{\textbf{\ipa{ɖɯ˩mɑ\#˥}}}}\hspace{0.5cm}[\kern2pt{\textcolor{darkblue}{\textbf{\ipa{ɖɯ˩mɑ˥}}}}\kern2pt]} \hypertarget{d`M\string_BmA\#\string_T1}{}
\markboth{\textcolor{darkblue}{\textbf{\ipa{ɖɯ˩mɑ\#˥}}}}{}
\textcolor{teal}{\mytextsc{noun}} \hspace{4pt} Tone: LM+\#H.
\textcolor{Sepia}{\selectlanguage{english}Feminine given name.} \zh{女性名字。} 
\lhead{\firstmark}
\rhead{\botmark}

\subsection{\hspace{-0.5cm} {\Large \textcolor{darkblue}{\textbf{\ipa{ɖɯ˩mɑ˧-ɬɑ˩tsʰo˩}}}}\hspace{0.5cm}[\kern2pt{\textcolor{darkblue}{\textbf{\ipa{ɖɯ˩mɑ˧ɬɑ˩tsʰo˩}}}}\kern2pt]} \hypertarget{d`M\string_BmA\string_M-KA\string_Bts\string_ho\string_B1}{}
\markboth{\textcolor{darkblue}{\textbf{\ipa{ɖɯ˩mɑ˧-ɬɑ˩tsʰo˩}}}}{}
\textcolor{teal}{\mytextsc{noun}} \hspace{4pt} Tone: LM-L.
\textcolor{Sepia}{\selectlanguage{english}Feminine given name.} \zh{女性名字。} 
\lhead{\firstmark}
\rhead{\botmark}

\subsection{\hspace{-0.5cm} {\Large \textcolor{darkblue}{\textbf{\ipa{ɖɯ˩mɑ˧-pv̩˩ʈʰɯ˩}}}}\hspace{0.5cm}[\kern2pt{\textcolor{darkblue}{\textbf{\ipa{ɖɯ˩mɑ˧pv̩˩ʈʰɯ˩}}}}\kern2pt]} \hypertarget{d`M\string_BmA\string_M-pv\string_=\string_Bt`\string_hM\string_B1}{}
\markboth{\textcolor{darkblue}{\textbf{\ipa{ɖɯ˩mɑ˧-pv̩˩ʈʰɯ˩}}}}{}
\textcolor{teal}{\mytextsc{noun}} \hspace{4pt} Tone: LM-L.
\textcolor{Sepia}{\selectlanguage{english}Feminine given name.} \zh{女性名字。} 
\lhead{\firstmark}
\rhead{\botmark}

\subsection{\hspace{-0.5cm} {\Large \textcolor{darkblue}{\textbf{\ipa{ɖɯ˩mi\#˥}}}}\hspace{0.5cm}[\kern2pt{\textcolor{darkblue}{\textbf{\ipa{ɖɯ˩mi˥}}}}\kern2pt]} \hypertarget{d`M\string_Bmi\#\string_T1}{}
\markboth{\textcolor{darkblue}{\textbf{\ipa{ɖɯ˩mi\#˥}}}}{}
\textcolor{teal}{\mytextsc{noun}} \hspace{4pt} Tone: LM+\#H.
\textcolor{Sepia}{\selectlanguage{english}Female mule. This is a sterile animal. It is docile, and suitable for tasks such as leading a caravan. It is therefore a highly valued animal.} \zh{母骡子、母马骡。}  ¶ \textcolor{darkblue}{\textbf{\ipa{ɖɯ˩mi˧-ɖɯ˥zo˩}}} \textcolor{Sepia}{\selectlanguage{english}Female mule and male mule} \zh{母骡子与公骡子}  
 \zh{量词}: \textcolor{darkblue}{\textbf{\ipa{mi˩}}}  \mytextsc{clf}: \textcolor{darkblue}{\textbf{\ipa{mi˩}}} 
\lhead{\firstmark}
\rhead{\botmark}

\subsection{\hspace{-0.5cm} {\Large \textcolor{darkblue}{\textbf{\ipa{ɖɯ˩zo\#˥}}}}\hspace{0.5cm}[\kern2pt{\textcolor{darkblue}{\textbf{\ipa{ɖɯ˩zo˥}}}}\kern2pt]} \hypertarget{d`M\string_Bzo\#\string_T1}{}
\markboth{\textcolor{darkblue}{\textbf{\ipa{ɖɯ˩zo\#˥}}}}{}
\textcolor{teal}{\mytextsc{noun}} \hspace{4pt} Tone: LM+\#H.
\textcolor{Sepia}{\selectlanguage{english}Male mule.} \zh{公骡子。}  ¶ \textcolor{darkblue}{\textbf{\ipa{ɖɯ˩zo˧-ɖɯ˥mi˩}}} \textcolor{Sepia}{\selectlanguage{english}male mule and female mule} \zh{公骡子与母骡子}  
 \zh{量词}: \textcolor{darkblue}{\textbf{\ipa{ɭɯ˧}}}  \mytextsc{clf}: \textcolor{darkblue}{\textbf{\ipa{ɭɯ˧}}} 
\lhead{\firstmark}
\rhead{\botmark}

\subsection{\hspace{-0.5cm} {\Large \textcolor{darkblue}{\textbf{\ipa{ɖɯ˧˥}}}}\hspace{0.5cm}[\kern2pt{\textcolor{darkblue}{\textbf{\ipa{ɖɯ˧˥}}}}\kern2pt]} \hypertarget{d`M\string_M\string_T1}{}
\markboth{\textcolor{darkblue}{\textbf{\ipa{ɖɯ˧˥}}}}{}
\textcolor{teal}{\mytextsc{number}} \hspace{4pt} Tone: MH.
\textcolor{Sepia}{\selectlanguage{english}1.} \zh{1。} 
\lhead{\firstmark}
\rhead{\botmark}

\subsection{\hspace{-0.5cm} {\Large \textcolor{darkblue}{\textbf{\ipa{ɖv̩˩}}}}\hspace{0.5cm}[\kern2pt{\textcolor{darkblue}{\textbf{\ipa{ɖv̩˥}}}}\kern2pt]} \hypertarget{d`v\string_=\string_B1}{}
\markboth{\textcolor{darkblue}{\textbf{\ipa{ɖv̩˩}}}}{}
\textcolor{teal}{\mytextsc{noun}} \hspace{4pt} Tone: L.
\textcolor{Sepia}{\selectlanguage{english}Wing (monosyllabic form; the disyllabic form is preferred).} \zh{翅膀。}  \zh{量词}: \textcolor{darkblue}{\textbf{\ipa{dze˩}}}  \mytextsc{clf}: \textcolor{darkblue}{\textbf{\ipa{dze˩}}} 
\lhead{\firstmark}
\rhead{\botmark}

\subsection{\hspace{-0.5cm} {\Large \textcolor{darkblue}{\textbf{\ipa{ɖv̩˧qæ˧}}}}\hspace{0.5cm}[\kern2pt{\textcolor{darkblue}{\textbf{\ipa{ɖv̩˩qæ˥}}}}\kern2pt]} \hypertarget{d`v\string_=\string_Mq\{\string_M1}{}
\markboth{\textcolor{darkblue}{\textbf{\ipa{ɖv̩˧qæ˧}}}}{}
\textcolor{teal}{\mytextsc{noun}} \hspace{4pt} Tone: M.
\textcolor{Sepia}{\selectlanguage{english}Wing.} \zh{翅膀。}  ¶ \textcolor{darkblue}{\textbf{\ipa{kɤ˩nɑ˧mi˧-ɖv̩˧qæ˥}}} \textcolor{Sepia}{\selectlanguage{english}eagle wings} \zh{老鹰翅膀}  
 \zh{量词}: \textcolor{darkblue}{\textbf{\ipa{dze˩}}}  \mytextsc{clf}: \textcolor{darkblue}{\textbf{\ipa{dze˩}}} 
\lhead{\firstmark}
\rhead{\botmark}

\subsection{\hspace{-0.5cm} {\Large \textcolor{darkblue}{\textbf{\ipa{ɖwæ˥}}}}\hspace{0.5cm}[\kern2pt{\textcolor{darkblue}{\textbf{\ipa{ɖwæ˥}}}}\kern2pt]} \hypertarget{d`w\{\string_T1}{}
\markboth{\textcolor{darkblue}{\textbf{\ipa{ɖwæ˥}}}}{}
\textcolor{teal}{\mytextsc{noun}} \hspace{4pt} Tone: \#H.
\ding{202} \textcolor{Sepia}{\selectlanguage{english}Pond.} \zh{池塘。}  ¶ \textcolor{darkblue}{\textbf{\ipa{[F5] ɖwæ˩ɬo˩mi˧}}} \textcolor{Sepia}{\selectlanguage{english}large pond} \zh{大池塘}  
 \zh{量词}: \textcolor{darkblue}{\textbf{\ipa{ɭɯ˧}}} \ding{203} \textcolor{Sepia}{\selectlanguage{english}Pool (artificial).} \zh{水坑。}  \mytextsc{clf}: \textcolor{darkblue}{\textbf{\ipa{ɭɯ˧}}} 
\lhead{\firstmark}
\rhead{\botmark}

\subsection{\hspace{-0.5cm} {\Large \textcolor{darkblue}{\textbf{\ipa{ɖwæ˥}}}}\hspace{0.5cm}[\kern2pt{\textcolor{darkblue}{\textbf{\ipa{ɖwæ˧˥}}}}\kern2pt]} \hypertarget{d`w\{\string_T1}{}
\markboth{\textcolor{darkblue}{\textbf{\ipa{ɖwæ˥}}}}{}
\textcolor{teal}{\mytextsc{adjective}} \hspace{4pt} Tone: H.
\textcolor{Sepia}{\selectlanguage{english}Muddy, turbid.} \zh{浑浊 (水)。}  ¶ \textcolor{darkblue}{\textbf{\ipa{dʑɯ˧ ɖwæ\#˥}}} \textcolor{Sepia}{\selectlanguage{english}turbid water} \zh{浑浊的水}  
 ¶ \textcolor{darkblue}{\textbf{\ipa{dʑɯ˧ | ɖwæ˧-ze˩!}}} \textcolor{Sepia}{\selectlanguage{english}The water has become turbid.} \zh{水浑浊了。}  

\lhead{\firstmark}
\rhead{\botmark}

\subsection{\hspace{-0.5cm} {\Large \textcolor{darkblue}{\textbf{\ipa{ɖwæ˥\textsubscript{a}}}}}\hspace{0.5cm}[\kern2pt{\textcolor{darkblue}{\textbf{\ipa{ɖwæ˩˥}}}}\kern2pt]} \hypertarget{d`w\{\string_Ta1}{}
\markboth{\textcolor{darkblue}{\textbf{\ipa{ɖwæ˥\textsubscript{a}}}}}{}
\textcolor{teal}{\mytextsc{classifier}} \hspace{4pt} Tone: H\textsubscript{a}.
\textcolor{Sepia}{\selectlanguage{english}Classifier for steps (of stairs).} \zh{量词:梯级、楼梯(一节)。}  ¶ \textcolor{darkblue}{\textbf{\ipa{ɖɯ˧-ɖwæ˧ ɲi˥}}} \textcolor{Sepia}{\selectlanguage{english}It's a step (of stairs). (Elicited to investigate the word's tonal behaviour)} \zh{是一节/一节阶梯。(引出这句是为了了解这个词在不同语境的声调变化。)}  
 ¶ \textcolor{darkblue}{\textbf{\ipa{ʈʂʰɯ˧-ɖwæ\#˥}}} \textcolor{Sepia}{\selectlanguage{english}this step} \zh{这节阶梯}  

\lhead{\firstmark}
\rhead{\botmark}

\subsection{\hspace{-0.5cm} {\Large \textcolor{darkblue}{\textbf{\ipa{ɖwæ˧-pɤ˧ɭɯ˥}}}}\hspace{0.5cm}[\kern2pt{\textcolor{darkblue}{\textbf{\ipa{xxxx non-correspondance entre le nombre de morphèmes et le nombre de tons de morphèmes}}}}\kern2pt]} \hypertarget{d`w\{\string_M-p7\string_Ml\string_RM\string_T1}{}
\markboth{\textcolor{darkblue}{\textbf{\ipa{ɖwæ˧-pɤ˧ɭɯ˥}}}}{}
\textcolor{teal}{\mytextsc{noun}} \hspace{4pt} Tone: H\#.
\textcolor{Sepia}{\selectlanguage{english}Puddle, pool (natural).} \zh{水潭。}  ¶ \textcolor{darkblue}{\textbf{\ipa{ɖwæ˧ tʰi˧-pɤ˥ɭɯ˩}}} \textcolor{Sepia}{\selectlanguage{english}there is water in the pool; a puddle has formed} \zh{有水潭}  
 \zh{量词}: \textcolor{darkblue}{\textbf{\ipa{ɭɯ˧}}}  \mytextsc{clf}: \textcolor{darkblue}{\textbf{\ipa{ɭɯ˧}}} 
\lhead{\firstmark}
\rhead{\botmark}

\subsection{\hspace{-0.5cm} {\Large \textcolor{darkblue}{\textbf{\ipa{ɖwæ˩\textsubscript{a}}}}}\hspace{0.5cm}[\kern2pt{\textcolor{darkblue}{\textbf{\ipa{ɖwæ˥}}}}\kern2pt]} \hypertarget{d`w\{\string_Ba1}{}
\markboth{\textcolor{darkblue}{\textbf{\ipa{ɖwæ˩\textsubscript{a}}}}}{}
\textcolor{teal}{\mytextsc{verb}} \hspace{4pt} Tone: L\textsubscript{a}.
\textcolor{Sepia}{\selectlanguage{english}To be afraid.} \zh{害怕。}  ¶ \textcolor{darkblue}{\textbf{\ipa{njɤ˧ | ɖwæ˩˥!}}} \textcolor{Sepia}{\selectlanguage{english}I am afraid!} \zh{我害怕!}  
 ¶ \textcolor{darkblue}{\textbf{\ipa{njɤ˧ | ʈʂʰɯ˧-v̩˧ | ɖwæ˩˥ | ʐwæ˩˥!}}} \textcolor{Sepia}{\selectlanguage{english}I am really afraid of this person!} \zh{我很害怕那个人!}  

\lhead{\firstmark}
\rhead{\botmark}

\subsection{\hspace{-0.5cm} {\Large \textcolor{darkblue}{\textbf{\ipa{ɖwæ˧˥}}} \textsubscript{1}}\hspace{0.5cm}[\kern2pt{\textcolor{darkblue}{\textbf{\ipa{ɖwæ˩˥}}}}\kern2pt]} \hypertarget{d`w\{\string_M\string_T1}{}
\markboth{\textcolor{darkblue}{\textbf{\ipa{ɖwæ˧˥}}} \textsubscript{1}}{}
\textcolor{teal}{\mytextsc{verb}} \hspace{4pt} Tone: MH.
\textcolor{Sepia}{\selectlanguage{english}To whip.} \zh{鞭打、抽打、加鞭。}  ¶ \textcolor{darkblue}{\textbf{\ipa{mæ˧qv̩˩-po˩-ɳɯ˩ | ɖwæ˧˥}}} \textcolor{Sepia}{\selectlanguage{english}to whip with the tail (e.g. a tiger whips the ground with its tail)} \zh{用尾巴来抽打(如:老虎用尾巴来抽打地面)}  

\lhead{\firstmark}
\rhead{\botmark}

\subsection{\hspace{-0.5cm} {\Large \textcolor{darkblue}{\textbf{\ipa{ɖwæ˧˥}}} \textsubscript{2}}\hspace{0.5cm}[\kern2pt{\textcolor{darkblue}{\textbf{\ipa{ɖwæ˧˥}}}}\kern2pt]} \hypertarget{d`w\{\string_M\string_T2}{}
\markboth{\textcolor{darkblue}{\textbf{\ipa{ɖwæ˧˥}}} \textsubscript{2}}{}
\textcolor{teal}{\mytextsc{adverb(ial)}} \hspace{4pt} Tone: MH.
\textcolor{Sepia}{\selectlanguage{english}Intensive: very.} \zh{很、极。}  ¶ \textcolor{darkblue}{\textbf{\ipa{ʈʂʰɯ˧ | ɖwæ˧˥ | æ˧mv̩˩ fv̩˩!}}} \textcolor{Sepia}{\selectlanguage{english}She likes her elder sister very much!} \zh{她很喜欢她姐姐!}  

\lhead{\firstmark}
\rhead{\botmark}

\newpage
\section*{\centering- \textcolor{darkblue}{\textbf{\ipa{ɖʐ}}} -}
\subsection{\hspace{-0.5cm} {\Large \textcolor{darkblue}{\textbf{\ipa{ɖʐæ˧\textsubscript{b}}}}}\hspace{0.5cm}[\kern2pt{\textcolor{darkblue}{\textbf{\ipa{ɖʐæ˩˥}}}}\kern2pt]} \hypertarget{d`z`\{\string_Mb1}{}
\markboth{\textcolor{darkblue}{\textbf{\ipa{ɖʐæ˧\textsubscript{b}}}}}{}
\textcolor{teal}{\mytextsc{verb}} \hspace{4pt} Tone: M\textsubscript{b}.
\textcolor{Sepia}{\selectlanguage{english}To ride (a horse).} \zh{骑马。}  ¶ \textcolor{darkblue}{\textbf{\ipa{le˧-ɖʐæ˧-ze˧}}} \textcolor{Sepia}{\selectlanguage{english}\mytextsc{accomp} \string_ \mytextsc{pfv}} \zh{\mytextsc{accomp} \string_ \mytextsc{pfv}}  
 ¶ \textcolor{darkblue}{\textbf{\ipa{ʐwæ˧ ɖʐæ˧}}} \textcolor{Sepia}{\selectlanguage{english}to ride a horse} \zh{骑马}  
 ¶ \textcolor{darkblue}{\textbf{\ipa{ɖʐæ˧-tʰɑ˧˥!}}} \textcolor{Sepia}{\selectlanguage{english}It's possible to ride (it)! / It's OK to ride (it)!} \zh{可以骑的!}  

\lhead{\firstmark}
\rhead{\botmark}

\subsection{\hspace{-0.5cm} {\Large \textcolor{darkblue}{\textbf{\ipa{ɖʐæ˧qʰæ˥\$}}}}\hspace{0.5cm}[\kern2pt{\textcolor{darkblue}{\textbf{\ipa{ɖʐæ˩qʰæ˩˥}}}}\kern2pt]} \hypertarget{d`z`\{\string_Mq\string_h\{\string_T\$1}{}
\markboth{\textcolor{darkblue}{\textbf{\ipa{ɖʐæ˧qʰæ˥\$}}}}{}
\textcolor{teal}{\mytextsc{noun}} \hspace{4pt} Tone: H\$.
\textcolor{Sepia}{\selectlanguage{english}Mud.} \zh{泥巴。}  ¶ \textcolor{darkblue}{\textbf{\ipa{ɖʐæ˧qʰæ˧ ʐæ˥(-ze˩)}}} \textcolor{Sepia}{\selectlanguage{english}There is mud; mud has formed.} \zh{有泥巴了。}  
 ¶ \textcolor{darkblue}{\textbf{\ipa{[F5] ɖʐæ˧qʰæ˧ ʐæ˧\textasciitilde{}ʐæ˥}}} \textcolor{Sepia}{\selectlanguage{english}There is mud; mud has formed.} \zh{有泥巴了}  

\lhead{\firstmark}
\rhead{\botmark}

\subsection{\hspace{-0.5cm} {\Large \textcolor{darkblue}{\textbf{\ipa{ɖʐæ˩\textsubscript{a}}}}}\hspace{0.5cm}[\kern2pt{\textcolor{darkblue}{\textbf{\ipa{ɖʐæ˥}}}}\kern2pt]} \hypertarget{d`z`\{\string_Ba1}{}
\markboth{\textcolor{darkblue}{\textbf{\ipa{ɖʐæ˩\textsubscript{a}}}}}{}
\textcolor{teal}{\mytextsc{verb}} \hspace{4pt} Tone: L\textsubscript{a}.
\textcolor{Sepia}{\selectlanguage{english}To melt; to thaw.} \zh{融化。}  ¶ \textcolor{darkblue}{\textbf{\ipa{mɤ˧ | le˧-ɖʐæ˩-ze˩}}} \textcolor{Sepia}{\selectlanguage{english}The grease has melted.} \zh{油融化了。}  
 ¶ \textcolor{darkblue}{\textbf{\ipa{dʑi˩pʰæ˩˥ | le˧-ɖʐæ˩-ze˩}}} \textcolor{Sepia}{\selectlanguage{english}The ice has melted.} \zh{冰融化了。}  

\lhead{\firstmark}
\rhead{\botmark}

\subsection{\hspace{-0.5cm} {\Large \textcolor{darkblue}{\textbf{\ipa{ɖʐæ˩bv˩}}}}\hspace{0.5cm}[\kern2pt{\textcolor{darkblue}{\textbf{\ipa{ɖʐæ˧bv˧}}}}\kern2pt]} \hypertarget{d`z`\{\string_Bbv\string_B1}{}
\markboth{\textcolor{darkblue}{\textbf{\ipa{ɖʐæ˩bv˩}}}}{}
\textcolor{teal}{\mytextsc{noun}} \hspace{4pt} Tone: L.
\textcolor{Sepia}{\selectlanguage{english}Sorcerer.} \zh{法师。}  ¶ \textcolor{darkblue}{\textbf{\ipa{ə˧pʰv˧-ɖʐæ˩bv˩}}} \textcolor{Sepia}{\selectlanguage{english}'Grandfather sorcerer': a respectful term of address for a sorcerer who is advanced in years or considered to have great powers} \zh{‘法师爷爷’:对年龄高(或被认为本事很大)的法师的尊重称呼}  
 ¶ \textcolor{darkblue}{\textbf{\ipa{ə˧v˧-ɖʐæ˥bv˩}}} \textcolor{Sepia}{\selectlanguage{english}'Uncle sorcerer': a respectful term of address for a sorcerer} \zh{‘法师舅舅’:对法师的尊重称呼}  
 \zh{量词}: \textcolor{darkblue}{\textbf{\ipa{v̩˧}}}  \mytextsc{clf}: \textcolor{darkblue}{\textbf{\ipa{v̩˧}}} 
\lhead{\firstmark}
\rhead{\botmark}

\subsection{\hspace{-0.5cm} {\Large \textcolor{darkblue}{\textbf{\ipa{ɖʐe˧}}}}\hspace{0.5cm}[\kern2pt{\textcolor{darkblue}{\textbf{\ipa{ɖʐe˥}}}}\kern2pt]} \hypertarget{d`z`e\string_M1}{}
\markboth{\textcolor{darkblue}{\textbf{\ipa{ɖʐe˧}}}}{}
\textcolor{teal}{\mytextsc{noun}} \hspace{4pt} Tone: M.
\textcolor{Sepia}{\selectlanguage{english}Money.} \zh{钱。} 
\lhead{\firstmark}
\rhead{\botmark}

\subsection{\hspace{-0.5cm} {\Large \textcolor{darkblue}{\textbf{\ipa{ɖʐe˧gɯ˧}}}}\hspace{0.5cm}[\kern2pt{\textcolor{darkblue}{\textbf{\ipa{ɖʐe˩gɯ˧˥}}}}\kern2pt]} \hypertarget{d`z`e\string_MgM\string_M1}{}
\markboth{\textcolor{darkblue}{\textbf{\ipa{ɖʐe˧gɯ˧}}}}{}
\textcolor{teal}{\mytextsc{noun}} \hspace{4pt} Tone: M.
\textcolor{Sepia}{\selectlanguage{english}Yongsheng (place name).} \zh{永胜(地名)。}  ¶ \textcolor{darkblue}{\textbf{\ipa{ɖʐe˧gɯ˧-to˩mi˩}}} \textcolor{Sepia}{\selectlanguage{english}a high mountain located in Yongsheng} \zh{永胜的一座高山}  
 ¶ \textcolor{darkblue}{\textbf{\ipa{ɖʐe˧gɯ˧-hæ˧}}} \textcolor{Sepia}{\selectlanguage{english}Yongsheng Chinese (Han) (note: Yongsheng is mainly populated by Han Chinese)} \zh{永胜汉族}  
 ¶ \textcolor{darkblue}{\textbf{\ipa{ɖʐe˧gɯ˧-dʑo˧, | hæ˧-ʂo˧\textasciitilde{}ʂo˩!}}} \textcolor{Sepia}{\selectlanguage{english}In Yongsheng, there are lots of Chinese (Han) people!} \zh{永胜,汉族群多!}  

\lhead{\firstmark}
\rhead{\botmark}

\subsection{\hspace{-0.5cm} {\Large \textcolor{darkblue}{\textbf{\ipa{ɖʐe˧ʁwɤ˧}}}}\hspace{0.5cm}[\kern2pt{\textcolor{darkblue}{\textbf{\ipa{xxxx non-correspondance entre le nombre de morphèmes et le nombre de tons de morphèmes}}}}\kern2pt]} \hypertarget{d`z`e\string_MRw7\string_M1}{}
\markboth{\textcolor{darkblue}{\textbf{\ipa{ɖʐe˧ʁwɤ˧}}}}{}
\textcolor{teal}{\mytextsc{noun}} \hspace{4pt} Tone: M.
\textcolor{Sepia}{\selectlanguage{english}Money, wealth.} \zh{钱。} 
\lhead{\firstmark}
\rhead{\botmark}

\subsection{\hspace{-0.5cm} {\Large \textcolor{darkblue}{\textbf{\ipa{ɖʐɤ˧qʰwɤ˧}}}}\hspace{0.5cm}[\kern2pt{\textcolor{darkblue}{\textbf{\ipa{ɖʐɤ˩qʰwɤ˥}}}}\kern2pt]} \hypertarget{d`z`7\string_Mq\string_hw7\string_M1}{}
\markboth{\textcolor{darkblue}{\textbf{\ipa{ɖʐɤ˧qʰwɤ˧}}}}{}
\textcolor{teal}{\mytextsc{noun}} \hspace{4pt} Tone: M.
\textcolor{Sepia}{\selectlanguage{english}Cold, flu.} \zh{感冒。}  ¶ \textcolor{darkblue}{\textbf{\ipa{ɖʐɤ˧qʰwɤ˧ go˩}}} \textcolor{Sepia}{\selectlanguage{english}to have a cold; to have a flu} \zh{感冒}  
 ¶ \textcolor{darkblue}{\textbf{\ipa{ɖʐɤ˧qʰwɤ˧ mɤ˧-go˩}}} \textcolor{Sepia}{\selectlanguage{english}...has no cold / does not have a cold} \zh{没感冒}  
 \zh{量词}: \textcolor{darkblue}{\textbf{\ipa{ʂɯ˩}}}  \mytextsc{clf}: \textcolor{darkblue}{\textbf{\ipa{ʂɯ˩}}} 
\lhead{\firstmark}
\rhead{\botmark}

\subsection{\hspace{-0.5cm} {\Large \textcolor{darkblue}{\textbf{\ipa{ɖʐɤ˧qʰwɤ˧ʈʂe\#˥}}}}\hspace{0.5cm}[\kern2pt{\textcolor{darkblue}{\textbf{\ipa{ɖʐɤ˧qʰwɤ˧ʈʂe˧}}}}\kern2pt]} \hypertarget{d`z`7\string_Mq\string_hw7\string_Mt`s`e\#\string_T1}{}
\markboth{\textcolor{darkblue}{\textbf{\ipa{ɖʐɤ˧qʰwɤ˧ʈʂe\#˥}}}}{}
\textcolor{teal}{\mytextsc{noun}} \hspace{4pt} Tone: \#H.
\textcolor{Sepia}{\selectlanguage{english}Awl.} \zh{锥、锥子。}  \zh{量词}: \textcolor{darkblue}{\textbf{\ipa{ɭɯ˧}}}  \mytextsc{clf}: \textcolor{darkblue}{\textbf{\ipa{ɭɯ˧}}} 
\lhead{\firstmark}
\rhead{\botmark}

\subsection{\hspace{-0.5cm} {\Large \textcolor{darkblue}{\textbf{\ipa{ɖʐɤ˩}}}}\hspace{0.5cm}[\kern2pt{\textcolor{darkblue}{\textbf{\ipa{ɖʐɤ˥}}}}\kern2pt]} \hypertarget{d`z`7\string_B1}{}
\markboth{\textcolor{darkblue}{\textbf{\ipa{ɖʐɤ˩}}}}{}
\textcolor{teal}{\mytextsc{noun}} \hspace{4pt} Tone: L.
\ding{202} \textcolor{Sepia}{\selectlanguage{english}Ladder.} \zh{梯子。}  ¶ \textcolor{darkblue}{\textbf{\ipa{ɖʐɤ˩ do˧}}} \textcolor{Sepia}{\selectlanguage{english}to climb a ladder} \zh{爬上一个梯子}  
 ¶ \textcolor{darkblue}{\textbf{\ipa{ɖʐɤ˧ | gɤ˩-do˧}}} \textcolor{Sepia}{\selectlanguage{english}to climb up a ladder} \zh{爬上一个梯子}  
 \zh{量词}: \textcolor{darkblue}{\textbf{\ipa{pɤ˩}}} \ding{203} \textcolor{Sepia}{\selectlanguage{english}Stairs.} \zh{楼梯。}  ¶ \textcolor{darkblue}{\textbf{\ipa{lv̩˧mi˧-ɖʐɤ˩ (+ɲi˩)}}} \textcolor{Sepia}{\selectlanguage{english}stone stairs} \zh{石头楼梯}  
 \mytextsc{clf}: \textcolor{darkblue}{\textbf{\ipa{pɤ˩}}} 
\lhead{\firstmark}
\rhead{\botmark}

\subsection{\hspace{-0.5cm} {\Large \textcolor{darkblue}{\textbf{\ipa{ɖʐɤ˩ɖwæ˩}}}}\hspace{0.5cm}[\kern2pt{\textcolor{darkblue}{\textbf{\ipa{ɖʐɤ˧ɖwæ˩}}}}\kern2pt]} \hypertarget{d`z`7\string_Bd`w\{\string_B1}{}
\markboth{\textcolor{darkblue}{\textbf{\ipa{ɖʐɤ˩ɖwæ˩}}}}{}
\textcolor{teal}{\mytextsc{noun}} \hspace{4pt} Tone: L.
\textcolor{Sepia}{\selectlanguage{english}Step of stairs.} \zh{台阶。}  ¶ \textcolor{darkblue}{\textbf{\ipa{lv̩˧mi˧-ɖʐɤ˩ɖwæ˩}}} \textcolor{Sepia}{\selectlanguage{english}stone step} \zh{石头台阶}  
 \zh{量词}: \textcolor{darkblue}{\textbf{\ipa{ɖwæ˥}}}  \mytextsc{clf}: \textcolor{darkblue}{\textbf{\ipa{ɖwæ˥}}} 
\lhead{\firstmark}
\rhead{\botmark}

\subsection{\hspace{-0.5cm} {\Large \textcolor{darkblue}{\textbf{\ipa{ɖʐɤ˩kɤ˥\$}}}}\hspace{0.5cm}[\kern2pt{\textcolor{darkblue}{\textbf{\ipa{ɖʐɤ˩kɤ˩˥}}}}\kern2pt]} \hypertarget{d`z`7\string_Bk7\string_T\$1}{}
\markboth{\textcolor{darkblue}{\textbf{\ipa{ɖʐɤ˩kɤ˥\$}}}}{}
\textcolor{teal}{\mytextsc{noun}} \hspace{4pt} Tone: LM+H\$.
\textcolor{Sepia}{\selectlanguage{english}A family name from Yongning. There are two families in Yongning that carry this name.} \zh{一个姓。这个姓,永宁有两家。}  ¶ \textcolor{darkblue}{\textbf{\ipa{ɖʐɤ˩kɤ˧=ɻ̍˥\$}}} \textcolor{Sepia}{\selectlanguage{english}the \textcolor{darkblue}{\textbf{\ipa{/ɖʐɤ˩kɤ˥\$/}}} clan} \zh{\textcolor{darkblue}{\textbf{\ipa{/ɖʐɤ˩kɤ˥\$/}}}家族}  

\lhead{\firstmark}
\rhead{\botmark}

\subsection{\hspace{-0.5cm} {\Large \textcolor{darkblue}{\textbf{\ipa{ɖʐɤ˧˥}}} \textsubscript{1}}\hspace{0.5cm}[\kern2pt{\textcolor{darkblue}{\textbf{\ipa{ɖʐɤ˩˥}}}}\kern2pt]} \hypertarget{d`z`7\string_M\string_T1}{}
\markboth{\textcolor{darkblue}{\textbf{\ipa{ɖʐɤ˧˥}}} \textsubscript{1}}{}
\textcolor{teal}{\mytextsc{verb}} \hspace{4pt} Tone: MH.
\ding{202} \textcolor{Sepia}{\selectlanguage{english}To pluck (fruit, weeds), to pick (vegetables).} \zh{摘(果子、蔬菜)、扯(草)。}  ¶ \textcolor{darkblue}{\textbf{\ipa{le˧-ɖʐɤ˧-po˥-jo˩!}}} \textcolor{Sepia}{\selectlanguage{english}Pluck some (fruit) and pass them over (to us)!} \zh{(你)去给摘(一些)过来吧!}  
 ¶ \textcolor{darkblue}{\textbf{\ipa{v̩˩tsʰɤ˧ ɖʐɤ˥}}} \textcolor{Sepia}{\selectlanguage{english}to pick vegetables} \zh{摘蔬菜}  
 ¶ \textcolor{darkblue}{\textbf{\ipa{le˧-ɖʐɤ˧˥, | mv̩˩-tɕo˧ kwɤ˩}}} \textcolor{Sepia}{\selectlanguage{english}to pluck and throw away (weeds)} \zh{扯(荒草),扔掉}  
\ding{203} \textcolor{Sepia}{\selectlanguage{english}To snap, to cut (thread); to smash; to destroy (a building).} \zh{拆(线),拔,捣碎。}  ¶ \textcolor{darkblue}{\textbf{\ipa{le˧-ɖʐɤ˩\textasciitilde{}ɖʐɤ˩}}} \textcolor{Sepia}{\selectlanguage{english}\mytextsc{red}} \zh{\mytextsc{重叠:拆拆}}  
 ¶ \textcolor{darkblue}{\textbf{\ipa{ʑi˧qʰwɤ˧ ɖʐɤ˧˥}}} \textcolor{Sepia}{\selectlanguage{english}to destroy a house} \zh{拆房子}  
 ¶ \textcolor{darkblue}{\textbf{\ipa{le˧-ɖʐɤ˧˥ | ɲi˧-gi˧ gv̩˧}}} \textcolor{Sepia}{\selectlanguage{english}to tear into two pieces} \zh{拆成两半}  

\lhead{\firstmark}
\rhead{\botmark}

\subsection{\hspace{-0.5cm} {\Large \textcolor{darkblue}{\textbf{\ipa{ɖʐɤ˧˥}}} \textsubscript{2}}\hspace{0.5cm}[\kern2pt{\textcolor{darkblue}{\textbf{\ipa{ɖʐɤ˧˥}}}}\kern2pt]} \hypertarget{d`z`7\string_M\string_T2}{}
\markboth{\textcolor{darkblue}{\textbf{\ipa{ɖʐɤ˧˥}}} \textsubscript{2}}{}
\textcolor{teal}{\mytextsc{verb}} \hspace{4pt} Tone: MH.
\textcolor{Sepia}{\selectlanguage{english}To prop open (a tent).} \zh{撑开(帐篷)。}  ¶ \textcolor{darkblue}{\textbf{\ipa{le˧-ɖʐɤ˩\textasciitilde{}ɖʐɤ˩}}} \textcolor{Sepia}{\selectlanguage{english}\mytextsc{red}} \zh{\mytextsc{重叠}}  

\lhead{\firstmark}
\rhead{\botmark}

\subsection{\hspace{-0.5cm} {\Large \textcolor{darkblue}{\textbf{\ipa{ɖʐo˥}}}}\hspace{0.5cm}[\kern2pt{\textcolor{darkblue}{\textbf{\ipa{ɖʐo˥}}}}\kern2pt]} \hypertarget{d`z`o\string_T1}{}
\markboth{\textcolor{darkblue}{\textbf{\ipa{ɖʐo˥}}}}{}
\textcolor{teal}{\mytextsc{noun}} \hspace{4pt} Tone: \#H.
\textcolor{Sepia}{\selectlanguage{english}Major roof beam.} \zh{大梁。}  \zh{量词}: \textcolor{darkblue}{\textbf{\ipa{ɖʐo˥}}}  \mytextsc{clf}: \textcolor{darkblue}{\textbf{\ipa{ɖʐo˥}}} 
\lhead{\firstmark}
\rhead{\botmark}

\subsection{\hspace{-0.5cm} {\Large \textcolor{darkblue}{\textbf{\ipa{ɖʐo˥}}}}\hspace{0.5cm}[\kern2pt{\textcolor{darkblue}{\textbf{\ipa{ɖʐo˥}}}}\kern2pt]} \hypertarget{d`z`o\string_T1}{}
\markboth{\textcolor{darkblue}{\textbf{\ipa{ɖʐo˥}}}}{}
\textcolor{teal}{\mytextsc{adjective}} \hspace{4pt} Tone: H.
\textcolor{Sepia}{\selectlanguage{english}Cold (weather).} \zh{冷(天气……)。} 
\lhead{\firstmark}
\rhead{\botmark}

\subsection{\hspace{-0.5cm} {\Large \textcolor{darkblue}{\textbf{\ipa{ɖʐo˥\textsubscript{a}}}}}\hspace{0.5cm}[\kern2pt{\textcolor{darkblue}{\textbf{\ipa{ɖʐo˩˥}}}}\kern2pt]} \hypertarget{d`z`o\string_Ta1}{}
\markboth{\textcolor{darkblue}{\textbf{\ipa{ɖʐo˥\textsubscript{a}}}}}{}
\textcolor{teal}{\mytextsc{classifier}} \hspace{4pt} Tone: H\textsubscript{a}.
\textcolor{Sepia}{\selectlanguage{english}Classifier for beams (in carpentry).} \zh{量词:梁(一根)。} 
\lhead{\firstmark}
\rhead{\botmark}

\subsection{\hspace{-0.5cm} {\Large \textcolor{darkblue}{\textbf{\ipa{ɖʐo˩\textsubscript{b}}}}}\hspace{0.5cm}[\kern2pt{\textcolor{darkblue}{\textbf{\ipa{ɖʐo˩˥}}}}\kern2pt]} \hypertarget{d`z`o\string_Bb1}{}
\markboth{\textcolor{darkblue}{\textbf{\ipa{ɖʐo˩\textsubscript{b}}}}}{}
\textcolor{teal}{\mytextsc{verb}} \hspace{4pt} Tone: L\textsubscript{b}.
\textcolor{Sepia}{\selectlanguage{english}To crush, to crumble (with the teeth or with a grindstone).} \zh{弄碎(用牙齿、手磨)。}  ¶ \textcolor{darkblue}{\textbf{\ipa{ʈʂo˧ɭɯ˧ ɖʐo˧˥}}} \textcolor{Sepia}{\selectlanguage{english}to crush with a grindstone} \zh{用手磨弄碎}  
 ¶ \textcolor{darkblue}{\textbf{\ipa{ɖɯ˧-kʰwɤ˧ ɖʐo˧˥}}} \textcolor{Sepia}{\selectlanguage{english}to crush a piece (of something)} \zh{弄碎一块}  
 ¶ \textcolor{darkblue}{\textbf{\ipa{ɖɯ˧-mɤ˩ ɖʐo˩}}} \textcolor{Sepia}{\selectlanguage{english}to crush a little (of something)} \zh{弄碎一点(东西)}  

\lhead{\firstmark}
\rhead{\botmark}

\subsection{\hspace{-0.5cm} {\Large \textcolor{darkblue}{\textbf{\ipa{ɖʐɯ˥}}}}\hspace{0.5cm}[\kern2pt{\textcolor{darkblue}{\textbf{\ipa{ɖʐɯ˧˥}}}}\kern2pt]} \hypertarget{d`z`M\string_T1}{}
\markboth{\textcolor{darkblue}{\textbf{\ipa{ɖʐɯ˥}}}}{}
\textcolor{teal}{\mytextsc{noun}} \hspace{4pt} Tone: \#H.
\textcolor{Sepia}{\selectlanguage{english}Market.} \zh{集市(圩场,街子)。}  \zh{量词}: \textcolor{darkblue}{\textbf{\ipa{ɖʐɯ˩}}}  \mytextsc{clf}: \textcolor{darkblue}{\textbf{\ipa{ɖʐɯ˩}}} 
\lhead{\firstmark}
\rhead{\botmark}

\subsection{\hspace{-0.5cm} {\Large \textcolor{darkblue}{\textbf{\ipa{ɖʐɯ˥\textsubscript{a}}}}}\hspace{0.5cm}[\kern2pt{\textcolor{darkblue}{\textbf{\ipa{ɖʐɯ˩˥}}}}\kern2pt]} \hypertarget{d`z`M\string_Ta1}{}
\markboth{\textcolor{darkblue}{\textbf{\ipa{ɖʐɯ˥\textsubscript{a}}}}}{}
\textcolor{teal}{\mytextsc{classifier}} \hspace{4pt} Tone: H\textsubscript{a}.
\textcolor{Sepia}{\selectlanguage{english}Self-classifier for marketplaces/towns.} \zh{量词:市场(一个),城市(一个)。}  ¶ \textcolor{darkblue}{\textbf{\ipa{ɖʐɯ˧ | ɖɯ˧-ɖʐɯ˥}}} \textcolor{Sepia}{\selectlanguage{english}a marketplace, a town} \zh{一个市场}  

\lhead{\firstmark}
\rhead{\botmark}

\subsection{\hspace{-0.5cm} {\Large \textcolor{darkblue}{\textbf{\ipa{ɖʐɯ˥kʰɤ˩}}}}\hspace{0.5cm}[\kern2pt{\textcolor{darkblue}{\textbf{\ipa{ɖʐɯ˩kʰɤ˥}}}}\kern2pt]} \hypertarget{d`z`M\string_Tk\string_h7\string_B1}{}
\markboth{\textcolor{darkblue}{\textbf{\ipa{ɖʐɯ˥kʰɤ˩}}}}{}
\textcolor{teal}{\mytextsc{noun}} \hspace{4pt} Tone: .
\textcolor{Sepia}{\selectlanguage{english}Moment.} \zh{(一)会儿。}  ¶ \textcolor{darkblue}{\textbf{\ipa{ɖɯ˧-ɖʐɯ˥kʰɤ˩}}} \textcolor{Sepia}{\selectlanguage{english}a moment} \zh{一会儿}  

\lhead{\firstmark}
\rhead{\botmark}

\subsection{\hspace{-0.5cm} {\Large \textcolor{darkblue}{\textbf{\ipa{ɖʐɯ˧qo˩}}}}\hspace{0.5cm}[\kern2pt{\textcolor{darkblue}{\textbf{\ipa{ɖʐɯ˧qo˩}}}}\kern2pt]} \hypertarget{d`z`M\string_Mqo\string_B1}{}
\markboth{\textcolor{darkblue}{\textbf{\ipa{ɖʐɯ˧qo˩}}}}{}
\textcolor{teal}{\mytextsc{adverb(ial)}} \hspace{4pt} Tone: L\#.
\textcolor{Sepia}{\selectlanguage{english}In town, in the street.} \zh{在城里、在市里。}  ¶ \textcolor{darkblue}{\textbf{\ipa{ɖʐɯ˧qo˩ kʰi˩}}} \textcolor{Sepia}{\selectlanguage{english}to go into town} \zh{上街}  

\lhead{\firstmark}
\rhead{\botmark}

\subsection{\hspace{-0.5cm} {\Large \textcolor{darkblue}{\textbf{\ipa{ɖʐɯ˧ʂɯ˥}}}}\hspace{0.5cm}[\kern2pt{\textcolor{darkblue}{\textbf{\ipa{ɖʐɯ˧ʂɯ˥}}}}\kern2pt]} \hypertarget{d`z`M\string_Ms`M\string_T1}{}
\markboth{\textcolor{darkblue}{\textbf{\ipa{ɖʐɯ˧ʂɯ˥}}}}{}
\textcolor{teal}{\mytextsc{noun}} \hspace{4pt} Tone: H\#.
\textcolor{Sepia}{\selectlanguage{english}Chopsticks.} \zh{筷子。}  \zh{量词}: \textcolor{darkblue}{\textbf{\ipa{dzi˧}}}  \mytextsc{clf}: \textcolor{darkblue}{\textbf{\ipa{dzi˧}}} 
\lhead{\firstmark}
\rhead{\botmark}

\subsection{\hspace{-0.5cm} {\Large \textcolor{darkblue}{\textbf{\ipa{ɖʐɯ˧ʈʂɯ˥}}}}\hspace{0.5cm}[\kern2pt{\textcolor{darkblue}{\textbf{\ipa{ɖʐɯ˧ʈʂɯ˥}}}}\kern2pt]} \hypertarget{d`z`M\string_Mt`s`M\string_T1}{}
\markboth{\textcolor{darkblue}{\textbf{\ipa{ɖʐɯ˧ʈʂɯ˥}}}}{}
\textcolor{teal}{\mytextsc{noun}} \hspace{4pt} Tone: H\#.
\textcolor{Sepia}{\selectlanguage{english}Sifter, sieve.} \zh{筛子。}  \zh{量词}: \textcolor{darkblue}{\textbf{\ipa{nɑ˧}}}  \mytextsc{clf}: \textcolor{darkblue}{\textbf{\ipa{nɑ˧}}} 
\lhead{\firstmark}
\rhead{\botmark}

\subsection{\hspace{-0.5cm} {\Large \textcolor{darkblue}{\textbf{\ipa{ɖʐɯ˩\textasciitilde{}ɖʐɯ˧˥}}}}\hspace{0.5cm}[\kern2pt{\textcolor{darkblue}{\textbf{\ipa{ɖʐɯ˧ɖʐɯ˧˥}}}}\kern2pt]} \hypertarget{d`z`M\string_B~d`z`M\string_M\string_T1}{}
\markboth{\textcolor{darkblue}{\textbf{\ipa{ɖʐɯ˩\textasciitilde{}ɖʐɯ˧˥}}}}{}
\textcolor{teal}{\mytextsc{verb}} \hspace{4pt} Tone: MH.
\textcolor{Sepia}{\selectlanguage{english}To shake (one's head).} \zh{摇(头)。}  ¶ \textcolor{darkblue}{\textbf{\ipa{ʁo˧ ɖʐɯ˥\textasciitilde{}ɖʐɯ˩}}} \textcolor{Sepia}{\selectlanguage{english}to shake one's head} \zh{摇头}  
 ¶ \textcolor{darkblue}{\textbf{\ipa{ʁo˧ | le˧-ɖʐɯ˩\textasciitilde{}ɖʐɯ˩-ze˩}}} \textcolor{Sepia}{\selectlanguage{english}shook (her/his) head} \zh{摇了头}  

\lhead{\firstmark}
\rhead{\botmark}

\subsection{\hspace{-0.5cm} {\Large \textcolor{darkblue}{\textbf{\ipa{ɖʐɯ˩kʰɤ˥}}}}\hspace{0.5cm}[\kern2pt{\textcolor{darkblue}{\textbf{\ipa{ɖʐɯ˧kʰɤ˥}}}}\kern2pt]} \hypertarget{d`z`M\string_Bk\string_h7\string_T1}{}
\markboth{\textcolor{darkblue}{\textbf{\ipa{ɖʐɯ˩kʰɤ˥}}}}{}
\textcolor{teal}{\mytextsc{noun}} \hspace{4pt} Tone: LH.
\textit{From:} \textbf{ɖʐɯ˩a} \textcolor{Sepia}{\selectlanguage{english}Period of time, era.} \zh{时代。}  \zh{量词}: \textcolor{darkblue}{\textbf{\ipa{ɖʐɯ˩}}}  \mytextsc{clf}: \textcolor{darkblue}{\textbf{\ipa{ɖʐɯ˩}}} 
\lhead{\firstmark}
\rhead{\botmark}

\subsection{\hspace{-0.5cm} {\Large \textcolor{darkblue}{\textbf{\ipa{ɖʐɯ˩tso\#˥}}}}\hspace{0.5cm}[\kern2pt{\textcolor{darkblue}{\textbf{\ipa{ɖʐɯ˩tso˥}}}}\kern2pt]} \hypertarget{d`z`M\string_Btso\#\string_T1}{}
\markboth{\textcolor{darkblue}{\textbf{\ipa{ɖʐɯ˩tso\#˥}}}}{}
\textcolor{teal}{\mytextsc{noun}} \hspace{4pt} Tone: LM+\#H.
\textcolor{Sepia}{\selectlanguage{english}Rules of society.} \zh{社会规矩。}  ¶ \textcolor{darkblue}{\textbf{\ipa{ɖʐɯ˩tso˥ | hĩ˧-qo˩-ɳɯ˩ | le˧-tsʰɯ˩-ɲi˩-tsɯ˩-mæ˩!}}} \textcolor{Sepia}{\selectlanguage{english}The rules of society, the moral teachings (including proverbs, tales...) come from people / are human creations / are the fruit of human experience!} \zh{社会规矩,是通过人类的经验形成的! / 社会规矩,是人(按一代代的经验)创造的!}  
 \zh{量词}: \textcolor{darkblue}{\textbf{\ipa{kʰwɤ˥}}}  \mytextsc{clf}: \textcolor{darkblue}{\textbf{\ipa{kʰwɤ˥}}} 
\lhead{\firstmark}
\rhead{\botmark}

\subsection{\hspace{-0.5cm} {\Large \textcolor{darkblue}{\textbf{\ipa{ɖʐv̩˧}}} \textsubscript{1}}\hspace{0.5cm}[\kern2pt{\textcolor{darkblue}{\textbf{\ipa{ɖʐv̩˥}}}}\kern2pt]} \hypertarget{d`z`v\string_=\string_M1}{}
\markboth{\textcolor{darkblue}{\textbf{\ipa{ɖʐv̩˧}}} \textsubscript{1}}{}
\textcolor{teal}{\mytextsc{verb}} \hspace{4pt} Tone: M intrans.
\textcolor{Sepia}{\selectlanguage{english}To burn, to catch fire.} \zh{燃烧。}  ¶ \textcolor{darkblue}{\textbf{\ipa{mv̩˧ ɖʐv̩˧-ze˩!}}} \textcolor{Sepia}{\selectlanguage{english}It has caught fire!} \zh{着火了!}  
 ¶ \textcolor{darkblue}{\textbf{\ipa{mv̩˧ le˧-ɖʐv̩˧-ze˧!}}} \textcolor{Sepia}{\selectlanguage{english}The fire has caught!} \zh{开始着火了!}  
 ¶ \textcolor{darkblue}{\textbf{\ipa{tʰi˧-ɖʐv̩˧-dʑo˧!}}} \textcolor{Sepia}{\selectlanguage{english}The fire is burning!} \zh{火在燃烧!}  

\lhead{\firstmark}
\rhead{\botmark}

\subsection{\hspace{-0.5cm} {\Large \textcolor{darkblue}{\textbf{\ipa{ɖʐv̩˧}}} \textsubscript{2}}\hspace{0.5cm}[\kern2pt{\textcolor{darkblue}{\textbf{\ipa{ɖʐv̩˥}}}}\kern2pt]} \hypertarget{d`z`v\string_=\string_M2}{}
\markboth{\textcolor{darkblue}{\textbf{\ipa{ɖʐv̩˧}}} \textsubscript{2}}{}
\textcolor{teal}{\mytextsc{noun}} \hspace{4pt} Tone: M.
\textcolor{Sepia}{\selectlanguage{english}Friend, companion, partner.} \zh{朋友、伙伴、伴侣。}  ¶ \textcolor{darkblue}{\textbf{\ipa{njɤ˧ | ɖʐv̩˧ ɲi˩.}}} \textcolor{Sepia}{\selectlanguage{english}[(S)he] is my friend.} \zh{是我朋友。}  
 ¶ \textcolor{darkblue}{\textbf{\ipa{õ˧ ɖʐv̩˥, õ˩ li˩! |}}} \textcolor{Sepia}{\selectlanguage{english}'One is easily influenced by one's friends!' (Literally: 'One's friends, one observes'.) The proverb refers to influence from friends, good or bad: good friends exert good influences; bad friends exert bad influences.} \zh{“大家都容易受朋友的影响!”(直译:“自己的朋友,自己看(=自己爱学他们的习惯)”)}  
 \zh{量词}: \textcolor{darkblue}{\textbf{\ipa{v̩˧}}}  \mytextsc{clf}: \textcolor{darkblue}{\textbf{\ipa{v̩˧}}} 
\lhead{\firstmark}
\rhead{\botmark}

\subsection{\hspace{-0.5cm} {\Large \textcolor{darkblue}{\textbf{\ipa{ɖʐv̩˧}}} \textsubscript{3}}\hspace{0.5cm}[\kern2pt{\textcolor{darkblue}{\textbf{\ipa{ɖʐv̩˥}}}}\kern2pt]} \hypertarget{d`z`v\string_=\string_M3}{}
\markboth{\textcolor{darkblue}{\textbf{\ipa{ɖʐv̩˧}}} \textsubscript{3}}{}
\textcolor{teal}{\mytextsc{noun}} \hspace{4pt} Tone: M.
\textcolor{Sepia}{\selectlanguage{english}An important and unfortunate event, such as a serious accident.} \zh{事故,(不幸的)大事。}  ¶ \textcolor{darkblue}{\textbf{\ipa{ɖʐv̩˧ kʰɯ˧˥}}} \textcolor{Sepia}{\selectlanguage{english}to cause an accident; to commit a fault; something serious happens} \zh{犯错误,出大事}  
 ¶ \textcolor{darkblue}{\textbf{\ipa{ɖʐv̩˧ kʰɯ˧-ze˥}}} \textcolor{Sepia}{\selectlanguage{english}As above, with the \mytextsc{pfv} morpheme} \zh{同上,加上\mytextsc{pfv语素}}  
 ¶ \textcolor{darkblue}{\textbf{\ipa{ɖʐv̩˧ ɖɯ˧-ɖʐv̩˧ | kʰɯ˧-ze˥!}}} \textcolor{Sepia}{\selectlanguage{english}An accident has happened! / There's been an accident!} \zh{出大事了!}  
 \zh{量词}: \textcolor{darkblue}{\textbf{\ipa{ɖʐv̩˧}}}  \mytextsc{clf}: \textcolor{darkblue}{\textbf{\ipa{ɖʐv̩˧}}} 
\lhead{\firstmark}
\rhead{\botmark}

\subsection{\hspace{-0.5cm} {\Large \textcolor{darkblue}{\textbf{\ipa{ɖʐv̩˧}}} \textsubscript{4}}\hspace{0.5cm}[\kern2pt{\textcolor{darkblue}{\textbf{\ipa{ɖʐv̩˥}}}}\kern2pt]} \hypertarget{d`z`v\string_=\string_M4}{}
\markboth{\textcolor{darkblue}{\textbf{\ipa{ɖʐv̩˧}}} \textsubscript{4}}{}
\textcolor{teal}{\mytextsc{noun}} \hspace{4pt} Tone: M.
\textcolor{Sepia}{\selectlanguage{english}Dew.} \zh{露水。} \textit{See:} \hyperlink{}{\textcolor{darkblue}{\textbf{\ipa{ɖʐv̩˧qʰɑ˧}}}} 
\lhead{\firstmark}
\rhead{\botmark}

\subsection{\hspace{-0.5cm} {\Large \textcolor{darkblue}{\textbf{\ipa{ɖʐv̩˥}}}}\hspace{0.5cm}[\kern2pt{\textcolor{darkblue}{\textbf{\ipa{ɖʐv̩˥}}}}\kern2pt]} \hypertarget{d`z`v\string_=\string_T1}{}
\markboth{\textcolor{darkblue}{\textbf{\ipa{ɖʐv̩˥}}}}{}
\textcolor{teal}{\mytextsc{noun}} \hspace{4pt} Tone: \#H.
\textcolor{Sepia}{\selectlanguage{english}Large vein, artery.} \zh{动脉。}  \zh{量词}: \textcolor{darkblue}{\textbf{\ipa{kʰɯ˩}}}  \mytextsc{clf}: \textcolor{darkblue}{\textbf{\ipa{kʰɯ˩}}} \textit{See:} \hyperlink{}{\textcolor{darkblue}{\textbf{\ipa{ɖʐv̩˧tsi˥}}}} 
\lhead{\firstmark}
\rhead{\botmark}

\subsection{\hspace{-0.5cm} {\Large \textcolor{darkblue}{\textbf{\ipa{ɖʐv̩˥}}}}\hspace{0.5cm}[\kern2pt{\textcolor{darkblue}{\textbf{\ipa{ɖʐv̩˥}}}}\kern2pt]} \hypertarget{d`z`v\string_=\string_T1}{}
\markboth{\textcolor{darkblue}{\textbf{\ipa{ɖʐv̩˥}}}}{}
\textcolor{teal}{\mytextsc{adjective}} \hspace{4pt} Tone: H.
\textcolor{Sepia}{\selectlanguage{english}Moist, wet, damp, humid.} \zh{湿。}  ¶ \textcolor{darkblue}{\textbf{\ipa{le˧-ɖʐv̩˥-ze˩}}} \textcolor{Sepia}{\selectlanguage{english}\mytextsc{accomp} \string_ \mytextsc{pfv}} \zh{\mytextsc{accomp} \string_ \mytextsc{pfv}}  
 ¶ \textcolor{darkblue}{\textbf{\ipa{ɖʐv̩˧\textasciitilde{}ɖʐv̩˧}}} \textcolor{Sepia}{\selectlanguage{english}\mytextsc{red}} \zh{\mytextsc{red}}  
 ¶ \textcolor{darkblue}{\textbf{\ipa{ʈʂe˧ ɖʐv̩˧-ze˩!}}} \textcolor{Sepia}{\selectlanguage{english}The earth is damp!} \zh{土湿了。}  

\lhead{\firstmark}
\rhead{\botmark}

\subsection{\hspace{-0.5cm} {\Large \textcolor{darkblue}{\textbf{\ipa{ɖʐv̩˥}}}}\hspace{0.5cm}[\kern2pt{\textcolor{darkblue}{\textbf{\ipa{ɖʐv̩˥}}}}\kern2pt]} \hypertarget{d`z`v\string_=\string_T1}{}
\markboth{\textcolor{darkblue}{\textbf{\ipa{ɖʐv̩˥}}}}{}
\textcolor{teal}{\mytextsc{verb}} \hspace{4pt} Tone: H.
\textcolor{Sepia}{\selectlanguage{english}To rise, to go up, to increase.} \zh{涨。}  ¶ \textcolor{darkblue}{\textbf{\ipa{hĩ˧ ɖʐv̩˧}}} \textcolor{Sepia}{\selectlanguage{english}people become numerous} \zh{人变多}  
 ¶ \textcolor{darkblue}{\textbf{\ipa{mo˧ ɖʐv̩˥}}} \textcolor{Sepia}{\selectlanguage{english}mushrooms multiply, become numerous} \zh{菌子长得多}  

\lhead{\firstmark}
\rhead{\botmark}

\subsection{\hspace{-0.5cm} {\Large \textcolor{darkblue}{\textbf{\ipa{ɖʐv̩˩\textsubscript{a}}}} \textsubscript{1}}\hspace{0.5cm}[\kern2pt{\textcolor{darkblue}{\textbf{\ipa{ɖʐv̩˩˥}}}}\kern2pt]} \hypertarget{d`z`v\string_=\string_Ba1}{}
\markboth{\textcolor{darkblue}{\textbf{\ipa{ɖʐv̩˩\textsubscript{a}}}} \textsubscript{1}}{}
\textcolor{teal}{\mytextsc{adjective}} \hspace{4pt} Tone: L\textsubscript{a}.
\textcolor{Sepia}{\selectlanguage{english}Ugly.} \zh{丑陋。}  ¶ \textcolor{darkblue}{\textbf{\ipa{ɖʐv̩˩-hĩ˩˥}}} \textcolor{Sepia}{\selectlanguage{english}\mytextsc{rel}/\mytextsc{nmlz}} \zh{丑的}  
 ¶ \textcolor{darkblue}{\textbf{\ipa{ʈʂʰɯ˧-v̩˧ | ɖwæ˧˥ | ɖʐv̩˩˥!}}} \textcolor{Sepia}{\selectlanguage{english}This one is really ugly!} \zh{这个好丑!}  

\lhead{\firstmark}
\rhead{\botmark}

\subsection{\hspace{-0.5cm} {\Large \textcolor{darkblue}{\textbf{\ipa{ɖʐv̩˩\textsubscript{a}}}} \textsubscript{2}}\hspace{0.5cm}[\kern2pt{\textcolor{darkblue}{\textbf{\ipa{ɖʐv̩˩˥}}}}\kern2pt]} \hypertarget{d`z`v\string_=\string_Ba2}{}
\markboth{\textcolor{darkblue}{\textbf{\ipa{ɖʐv̩˩\textsubscript{a}}}} \textsubscript{2}}{}
\textcolor{teal}{\mytextsc{verb}} \hspace{4pt} Tone: L\textsubscript{a}.
\textcolor{Sepia}{\selectlanguage{english}To decide, to make a decision.} \zh{决定、选择、拿主意。}  ¶ \textcolor{darkblue}{\textbf{\ipa{njɤ˧-ɳɯ˧ | ɖʐv̩˧ ʝi˧-bi˧!}}} \textcolor{Sepia}{\selectlanguage{english}I'm going to decide!} \zh{我来决定吧!}  

\lhead{\firstmark}
\rhead{\botmark}

\subsection{\hspace{-0.5cm} {\Large \textcolor{darkblue}{\textbf{\ipa{ɖʐv̩˧\textsubscript{b}}}}}\hspace{0.5cm}[\kern2pt{\textcolor{darkblue}{\textbf{\ipa{ɖʐv̩˥}}}}\kern2pt]} \hypertarget{d`z`v\string_=\string_Mb1}{}
\markboth{\textcolor{darkblue}{\textbf{\ipa{ɖʐv̩˧\textsubscript{b}}}}}{}
\textcolor{teal}{\mytextsc{classifier}} \hspace{4pt} Tone: M\textsubscript{b}.
\textcolor{Sepia}{\selectlanguage{english}Self-classifier for accidents.} \zh{量词:事故(一场)。} 
\lhead{\firstmark}
\rhead{\botmark}

\subsection{\hspace{-0.5cm} {\Large \textcolor{darkblue}{\textbf{\ipa{ɖʐv̩˧-nɑ˥mi˩}}}}\hspace{0.5cm}[\kern2pt{\textcolor{darkblue}{\textbf{\ipa{ɖʐv̩˧nɑ˥mi˩}}}}\kern2pt]} \hypertarget{d`z`v\string_=\string_M-nA\string_Tmi\string_B1}{}
\markboth{\textcolor{darkblue}{\textbf{\ipa{ɖʐv̩˧-nɑ˥mi˩}}}}{}
\textcolor{teal}{\mytextsc{noun}} \hspace{4pt} Tone: \#H-.
\textcolor{Sepia}{\selectlanguage{english}Heron.} \zh{鹳。}  \zh{量词}: \textcolor{darkblue}{\textbf{\ipa{mi˩}}}  \mytextsc{clf}: \textcolor{darkblue}{\textbf{\ipa{mi˩}}} 
\lhead{\firstmark}
\rhead{\botmark}

\subsection{\hspace{-0.5cm} {\Large \textcolor{darkblue}{\textbf{\ipa{ɖʐv̩˧qʰɑ˧}}}}\hspace{0.5cm}[\kern2pt{\textcolor{darkblue}{\textbf{\ipa{ɖʐv̩˧qʰɑ˧}}}}\kern2pt]} \hypertarget{d`z`v\string_=\string_Mq\string_hA\string_M1}{}
\markboth{\textcolor{darkblue}{\textbf{\ipa{ɖʐv̩˧qʰɑ˧}}}}{}
\textcolor{teal}{\mytextsc{noun}} \hspace{4pt} Tone: M.
\textcolor{Sepia}{\selectlanguage{english}Dew.} \zh{露水。} \textit{See:} \hyperlink{}{\textcolor{darkblue}{\textbf{\ipa{ɖʐv̩˧}}} \textsubscript{4}} 
\lhead{\firstmark}
\rhead{\botmark}

\subsection{\hspace{-0.5cm} {\Large \textcolor{darkblue}{\textbf{\ipa{ɖʐv̩˩ti\#˥}}}}\hspace{0.5cm}[\kern2pt{\textcolor{darkblue}{\textbf{\ipa{ɖʐv̩˩ti˥}}}}\kern2pt]} \hypertarget{d`z`v\string_=\string_Bti\#\string_T1}{}
\markboth{\textcolor{darkblue}{\textbf{\ipa{ɖʐv̩˩ti\#˥}}}}{}
\textcolor{teal}{\mytextsc{noun}} \hspace{4pt} Tone: LM+\#H.
\textcolor{Sepia}{\selectlanguage{english}Spear.} \zh{矛。} 
\lhead{\firstmark}
\rhead{\botmark}

\subsection{\hspace{-0.5cm} {\Large \textcolor{darkblue}{\textbf{\ipa{ɖʐv̩˧tsi˥}}}}\hspace{0.5cm}[\kern2pt{\textcolor{darkblue}{\textbf{\ipa{ɖʐv̩˧tsi˥}}}}\kern2pt]} \hypertarget{d`z`v\string_=\string_Mtsi\string_T1}{}
\markboth{\textcolor{darkblue}{\textbf{\ipa{ɖʐv̩˧tsi˥}}}}{}
\textcolor{teal}{\mytextsc{noun}} \hspace{4pt} Tone: H\#.
\ding{202} \textcolor{Sepia}{\selectlanguage{english}Artery.} \zh{动脉。}  \zh{量词}: \textcolor{darkblue}{\textbf{\ipa{kʰɯ˩}}} \ding{203} \textcolor{Sepia}{\selectlanguage{english}Stem, stalk.} \zh{茎。}  \mytextsc{clf}: \textcolor{darkblue}{\textbf{\ipa{kʰɯ˩}}} \textit{See:} \textcolor{darkblue}{\textbf{\ipa{ɖʐv̩˥}}} 
\lhead{\firstmark}
\rhead{\botmark}

\subsection{\hspace{-0.5cm} {\Large \textcolor{darkblue}{\textbf{\ipa{ɖʐv̩˧ʐv̩˧-ɖʐv̩˧mi\#˥}}}}\hspace{0.5cm}[\kern2pt{\textcolor{darkblue}{\textbf{\ipa{xxxx non-correspondance entre le nombre de morphèmes et le nombre de tons de morphèmes}}}}\kern2pt]} \hypertarget{d`z`v\string_=\string_Mz`v\string_=\string_M-d`z`v\string_=\string_Mmi\#\string_T1}{}
\markboth{\textcolor{darkblue}{\textbf{\ipa{ɖʐv̩˧ʐv̩˧-ɖʐv̩˧mi\#˥}}}}{}
\textcolor{teal}{\mytextsc{noun}} \hspace{4pt} Tone: \#H.
\textcolor{Sepia}{\selectlanguage{english}Friend, companion, partner.} \zh{朋友、伙伴、伴侣。} 
\lhead{\firstmark}
\rhead{\botmark}

\subsection{\hspace{-0.5cm} {\Large \textcolor{darkblue}{\textbf{\ipa{ɖʐwæ˥}}}}\hspace{0.5cm}[\kern2pt{\textcolor{darkblue}{\textbf{\ipa{ɖʐwæ˥}}}}\kern2pt]} \hypertarget{d`z`w\{\string_T1}{}
\markboth{\textcolor{darkblue}{\textbf{\ipa{ɖʐwæ˥}}}}{}
\textcolor{teal}{\mytextsc{noun}} \hspace{4pt} Tone: \#H.
\textcolor{Sepia}{\selectlanguage{english}Small hoe (smaller than \textcolor{darkblue}{\textbf{\ipa{/hwæ˧pʰæ˩/}}}).} \zh{锄头。}  \zh{量词}: \textcolor{darkblue}{\textbf{\ipa{nɑ˧}}}  \mytextsc{clf}: \textcolor{darkblue}{\textbf{\ipa{nɑ˧}}} 
\lhead{\firstmark}
\rhead{\botmark}

\subsection{\hspace{-0.5cm} {\Large \textcolor{darkblue}{\textbf{\ipa{ɖʐwæ˧lɑ˧-ʁo˧ɖɯ˧˥}}}}\hspace{0.5cm}[\kern2pt{\textcolor{darkblue}{\textbf{\ipa{xxxx non-correspondance entre le nombre de morphèmes et le nombre de tons de morphèmes}}}}\kern2pt]} \hypertarget{d`z`w\{\string_MlA\string_M-Ro\string_Md`M\string_M\string_T1}{}
\markboth{\textcolor{darkblue}{\textbf{\ipa{ɖʐwæ˧lɑ˧-ʁo˧ɖɯ˧˥}}}}{}
\textcolor{teal}{\mytextsc{noun}} \hspace{4pt} Tone: MH\#.
\textcolor{Sepia}{\selectlanguage{english}A type of sparrow.} \zh{雀。}  \zh{量词}: \textcolor{darkblue}{\textbf{\ipa{mi˩}}}  \mytextsc{clf}: \textcolor{darkblue}{\textbf{\ipa{mi˩}}} 
\lhead{\firstmark}
\rhead{\botmark}

\subsection{\hspace{-0.5cm} {\Large \textcolor{darkblue}{\textbf{\ipa{ɖʐwæ˧mi˧}}}}\hspace{0.5cm}[\kern2pt{\textcolor{darkblue}{\textbf{\ipa{ɖʐwæ˧mi˧}}}}\kern2pt]} \hypertarget{d`z`w\{\string_Mmi\string_M1}{}
\markboth{\textcolor{darkblue}{\textbf{\ipa{ɖʐwæ˧mi˧}}}}{}
\textcolor{teal}{\mytextsc{noun}} \hspace{4pt} Tone: M.
\textcolor{Sepia}{\selectlanguage{english}Sparrow.} \zh{麻雀。}  \zh{量词}: \textcolor{darkblue}{\textbf{\ipa{mi˩}}}  \mytextsc{clf}: \textcolor{darkblue}{\textbf{\ipa{mi˩}}} 
\lhead{\firstmark}
\rhead{\botmark}

\subsection{\hspace{-0.5cm} {\Large \textcolor{darkblue}{\textbf{\ipa{ɖʐwæ˧pʰv̩\#˥}}}}\hspace{0.5cm}[\kern2pt{\textcolor{darkblue}{\textbf{\ipa{ɖʐwæ˧pʰv̩˧}}}}\kern2pt]} \hypertarget{d`z`w\{\string_Mp\string_hv\string_=\#\string_T1}{}
\markboth{\textcolor{darkblue}{\textbf{\ipa{ɖʐwæ˧pʰv̩\#˥}}}}{}
\textcolor{teal}{\mytextsc{noun}} \hspace{4pt} Tone: \#H.
\textcolor{Sepia}{\selectlanguage{english}Male sparrow.} \zh{公麻雀。}  ¶ \textcolor{darkblue}{\textbf{\ipa{ɖʐwæ˧pʰv̩˧ tʰv̩˧-mi˧˥ / ɖʐwæ˧pʰv̩˧ tʰv̩˧-mi˥\#}}} \textcolor{Sepia}{\selectlanguage{english}\mytextsc{n}+\mytextsc{dem}+\mytextsc{clf}} \zh{这只公麻雀}  
 \zh{量词}: \textcolor{darkblue}{\textbf{\ipa{mi˩}}}  \mytextsc{clf}: \textcolor{darkblue}{\textbf{\ipa{mi˩}}} 
\lhead{\firstmark}
\rhead{\botmark}

\subsection{\hspace{-0.5cm} {\Large \textcolor{darkblue}{\textbf{\ipa{ɖʐwæ˧zo\#˥}}}}\hspace{0.5cm}[\kern2pt{\textcolor{darkblue}{\textbf{\ipa{ɖʐwæ˧zo˧}}}}\kern2pt]} \hypertarget{d`z`w\{\string_Mzo\#\string_T1}{}
\markboth{\textcolor{darkblue}{\textbf{\ipa{ɖʐwæ˧zo\#˥}}}}{}
\textcolor{teal}{\mytextsc{noun}} \hspace{4pt} Tone: \#H.
\textcolor{Sepia}{\selectlanguage{english}Baby sparrow, little sparrow.} \zh{小麻雀。}  \zh{量词}: \textcolor{darkblue}{\textbf{\ipa{v̩˧, mi˩}}}  \mytextsc{clf}: \textcolor{darkblue}{\textbf{\ipa{v̩˧, mi˩}}} 
\lhead{\firstmark}
\rhead{\botmark}

\subsection{\hspace{-0.5cm} {\Large \textcolor{darkblue}{\textbf{\ipa{ɖʐwæ˩\textsubscript{a}}}}}\hspace{0.5cm}[\kern2pt{\textcolor{darkblue}{\textbf{\ipa{ɖʐwæ˩˥}}}}\kern2pt]} \hypertarget{d`z`w\{\string_Ba1}{}
\markboth{\textcolor{darkblue}{\textbf{\ipa{ɖʐwæ˩\textsubscript{a}}}}}{}
\textcolor{teal}{\mytextsc{verb}} \hspace{4pt} Tone: L\textsubscript{a}.
\textcolor{Sepia}{\selectlanguage{english}To quarrel.} \zh{吵架。}  ¶ \textcolor{darkblue}{\textbf{\ipa{ɖʐwæ˧\textasciitilde{}ɖʐwæ˥}}} \textcolor{Sepia}{\selectlanguage{english}\mytextsc{red}} \zh{\mytextsc{重叠}}  

\lhead{\firstmark}
\rhead{\botmark}

\subsection{\hspace{-0.5cm} {\Large \textcolor{darkblue}{\textbf{\ipa{ɖʐwæ˩hi˩}}}}\hspace{0.5cm}[\kern2pt{\textcolor{darkblue}{\textbf{\ipa{ɖʐwæ˩hi˩˥}}}}\kern2pt]} \hypertarget{d`z`w\{\string_Bhi\string_B1}{}
\markboth{\textcolor{darkblue}{\textbf{\ipa{ɖʐwæ˩hi˩}}}}{}
\textcolor{teal}{\mytextsc{noun}} \hspace{4pt} Tone: L.
\ding{202} \textcolor{Sepia}{\selectlanguage{english}Canine tooth, fang.} \zh{獠牙。}  \zh{量词}: \textcolor{darkblue}{\textbf{\ipa{ɭɯ˧}}} \ding{203} \textcolor{Sepia}{\selectlanguage{english}Fang.} \zh{动物的牙(犬牙)。}  \mytextsc{clf}: \textcolor{darkblue}{\textbf{\ipa{ɭɯ˧}}} 
\lhead{\firstmark}
\rhead{\botmark}

\subsection{\hspace{-0.5cm} {\Large \textcolor{darkblue}{\textbf{\ipa{ɖʐwæ˧˥}}}}\hspace{0.5cm}[\kern2pt{\textcolor{darkblue}{\textbf{\ipa{ɖʐwæ˧˥}}}}\kern2pt]} \hypertarget{d`z`w\{\string_M\string_T1}{}
\markboth{\textcolor{darkblue}{\textbf{\ipa{ɖʐwæ˧˥}}}}{}
\textcolor{teal}{\mytextsc{verb}} \hspace{4pt} Tone: MH.
\textcolor{Sepia}{\selectlanguage{english}To fall down; to release, to drop.} \zh{掉下。}  ¶ \textcolor{darkblue}{\textbf{\ipa{mv̩˩tɕo˧ ɖʐwæ˧˥ / mv̩˩tɕo˧ ɖʐwæ˧-ze˥}}} \textcolor{Sepia}{\selectlanguage{english}to fall down} \zh{掉下去(+了)}  

\lhead{\firstmark}
\rhead{\botmark}

\subsection{\hspace{-0.5cm} {\Large \textcolor{darkblue}{\textbf{\ipa{ɖʐwæ˩˧}}}}\hspace{0.5cm}[\kern2pt{\textcolor{darkblue}{\textbf{\ipa{ɖʐwæ˩˥}}}}\kern2pt]} \hypertarget{d`z`w\{\string_B\string_M1}{}
\markboth{\textcolor{darkblue}{\textbf{\ipa{ɖʐwæ˩˧}}}}{}
\textcolor{teal}{\mytextsc{noun}} \hspace{4pt} Tone: LM.
\textcolor{Sepia}{\selectlanguage{english}Sparrow (monosyllabic form; not in common use).} \zh{麻雀。}  \zh{量词}: \textcolor{darkblue}{\textbf{\ipa{mi˩}}}  \mytextsc{clf}: \textcolor{darkblue}{\textbf{\ipa{mi˩}}} 
\lhead{\firstmark}
\rhead{\botmark}

\newpage
\section*{\centering- \textcolor{darkblue}{\textbf{\ipa{ə}}} -}
\subsection{\hspace{-0.5cm} {\Large \textcolor{darkblue}{\textbf{\ipa{ə˧bɑ˩-lɑ˩bɑ˩}}}}\hspace{0.5cm}[\kern2pt{\textcolor{darkblue}{\textbf{\ipa{ə˧bɑ˩lɑ˧bɑ˧}}}}\kern2pt]} \hypertarget{@\string_MbA\string_B-lA\string_BbA\string_B1}{}
\markboth{\textcolor{darkblue}{\textbf{\ipa{ə˧bɑ˩-lɑ˩bɑ˩}}}}{}
\textcolor{teal}{\mytextsc{noun}} \hspace{4pt} Tone: L\#-.
\textcolor{Sepia}{\selectlanguage{english}Cactus.} \zh{仙人掌。}  ¶ \textcolor{darkblue}{\textbf{\ipa{ə˧bɑ˩-lɑ˩bɑ˩ | ɖɯ˧-dzi˩}}} \textcolor{Sepia}{\selectlanguage{english}a cactus plant} \zh{一棵仙人掌}  

\lhead{\firstmark}
\rhead{\botmark}

\subsection{\hspace{-0.5cm} {\Large \textcolor{darkblue}{\textbf{\ipa{ə˧bo˥\$}}}}\hspace{0.5cm}[\kern2pt{\textcolor{darkblue}{\textbf{\ipa{ə˧bo˥}}}}\kern2pt]} \hypertarget{@\string_Mbo\string_T\$1}{}
\markboth{\textcolor{darkblue}{\textbf{\ipa{ə˧bo˥\$}}}}{}
\textcolor{teal}{\mytextsc{noun}} \hspace{4pt} Tone: H\$.
\textcolor{Sepia}{\selectlanguage{english}Paternal uncle.} \zh{父亲的兄弟。}  ¶ \textcolor{darkblue}{\textbf{\ipa{ə˧bo˧-ɖɯ˧˥}}} \textcolor{Sepia}{\selectlanguage{english}paternal uncle, father's elder brother} \zh{伯父:父亲的哥哥}  
 ¶ \textcolor{darkblue}{\textbf{\ipa{ə˧bo˧-tɕi˥ (+ɲi˩)}}} \textcolor{Sepia}{\selectlanguage{english}paternal uncle, father's younger brother} \zh{叔叔:父亲的弟弟}  
 \zh{量词}: \textcolor{darkblue}{\textbf{\ipa{v̩˧}}}  \mytextsc{clf}: \textcolor{darkblue}{\textbf{\ipa{v̩˧}}} 
\lhead{\firstmark}
\rhead{\botmark}

\subsection{\hspace{-0.5cm} {\Large \textcolor{darkblue}{\textbf{\ipa{ə˧bo˧tɕo˧li˧}}}}\hspace{0.5cm}[\kern2pt{\textcolor{darkblue}{\textbf{\ipa{ə˧bo˧tɕo˧li˧}}}}\kern2pt]} \hypertarget{@\string_Mbo\string_Mts£o\string_Mli\string_M1}{}
\markboth{\textcolor{darkblue}{\textbf{\ipa{ə˧bo˧tɕo˧li˧}}}}{}
\textcolor{teal}{\mytextsc{noun}} \hspace{4pt} Tone: M.
\textcolor{Sepia}{\selectlanguage{english}Cricket.} \zh{蟋蟀。}  \zh{量词}: \textcolor{darkblue}{\textbf{\ipa{mi˩}}}  \mytextsc{clf}: \textcolor{darkblue}{\textbf{\ipa{mi˩}}} 
\lhead{\firstmark}
\rhead{\botmark}

\subsection{\hspace{-0.5cm} {\Large \textcolor{darkblue}{\textbf{\ipa{ə˧bv̩˩}}}}\hspace{0.5cm}[\kern2pt{\textcolor{darkblue}{\textbf{\ipa{ə˧bv̩˩}}}}\kern2pt]} \hypertarget{@\string_Mbv\string_=\string_B1}{}
\markboth{\textcolor{darkblue}{\textbf{\ipa{ə˧bv̩˩}}}}{}
\textcolor{teal}{\mytextsc{noun}} \hspace{4pt} Tone: L\#.
\textcolor{Sepia}{\selectlanguage{english}Oven to make bricks, ceramics...} \zh{烤砖、陶器等用的烤炉。}  \zh{量词}: \textcolor{darkblue}{\textbf{\ipa{ɭɯ˧}}}  \mytextsc{clf}: \textcolor{darkblue}{\textbf{\ipa{ɭɯ˧}}} 
\lhead{\firstmark}
\rhead{\botmark}

\subsection{\hspace{-0.5cm} {\Large \textcolor{darkblue}{\textbf{\ipa{ə˧bv̩˧-ʁwɤ˧}}}}\hspace{0.5cm}[\kern2pt{\textcolor{darkblue}{\textbf{\ipa{xxxx non-correspondance entre le nombre de morphèmes et le nombre de tons de morphèmes}}}}\kern2pt]} \hypertarget{@\string_Mbv\string_=\string_M-Rw7\string_M1}{}
\markboth{\textcolor{darkblue}{\textbf{\ipa{ə˧bv̩˧-ʁwɤ˧}}}}{}
\textcolor{teal}{\mytextsc{noun}} \hspace{4pt} Tone: M.
\textcolor{Sepia}{\selectlanguage{english}Name of a village.} \zh{阿布瓦村。} 
\lhead{\firstmark}
\rhead{\botmark}

\subsection{\hspace{-0.5cm} {\Large \textcolor{darkblue}{\textbf{\ipa{ə˧ɕjɤ˩}}}}\hspace{0.5cm}[\kern2pt{\textcolor{darkblue}{\textbf{\ipa{ə˧ɕjɤ˩}}}}\kern2pt]} \hypertarget{@\string_Ms£j7\string_B1}{}
\markboth{\textcolor{darkblue}{\textbf{\ipa{ə˧ɕjɤ˩}}}}{}
\textcolor{teal}{\mytextsc{noun}} \hspace{4pt} Tone: L\#.
\textcolor{Sepia}{\selectlanguage{english}Lover, boy/girl-friend.} \zh{情人。}  \zh{量词}: \textcolor{darkblue}{\textbf{\ipa{v̩˧}}}  \mytextsc{clf}: \textcolor{darkblue}{\textbf{\ipa{v̩˧}}} \textit{See:} \hyperlink{}{\textcolor{darkblue}{\textbf{\ipa{ə˧ɖo˧}}}} 
\lhead{\firstmark}
\rhead{\botmark}

\subsection{\hspace{-0.5cm} {\Large \textcolor{darkblue}{\textbf{\ipa{ə˧ɕjo˩}}}}\hspace{0.5cm}[\kern2pt{\textcolor{darkblue}{\textbf{\ipa{ə˧ɕjo˩}}}}\kern2pt]} \hypertarget{@\string_Ms£jo\string_B1}{}
\markboth{\textcolor{darkblue}{\textbf{\ipa{ə˧ɕjo˩}}}}{}
\textcolor{teal}{\mytextsc{noun}} \hspace{4pt} Tone: L\#.
\textcolor{Sepia}{\selectlanguage{english}A family name from Yongning. There are two families in Yongning that carry this name.} \zh{一个姓。这个姓,永宁有两家。}  ¶ \textcolor{darkblue}{\textbf{\ipa{ə˧ɕjo˩=ɻ̍˩}}} \textcolor{Sepia}{\selectlanguage{english}the \textcolor{darkblue}{\textbf{\ipa{/ə˧ɕjo˩/}}} clan} \zh{\textcolor{darkblue}{\textbf{\ipa{/ə˧ɕjo˩/}}}家族}  

\lhead{\firstmark}
\rhead{\botmark}

\subsection{\hspace{-0.5cm} {\Large \textcolor{darkblue}{\textbf{\ipa{ə˧dɑ˥\$}}}}\hspace{0.5cm}[\kern2pt{\textcolor{darkblue}{\textbf{\ipa{ə˧dɑ˥}}}}\kern2pt]} \hypertarget{@\string_MdA\string_T\$1}{}
\markboth{\textcolor{darkblue}{\textbf{\ipa{ə˧dɑ˥\$}}}}{}
\textcolor{teal}{\mytextsc{noun}} \hspace{4pt} Tone: H\$.
\textcolor{Sepia}{\selectlanguage{english}Father.} \zh{父亲。}  \zh{量词}: \textcolor{darkblue}{\textbf{\ipa{v̩˧}}}  \mytextsc{clf}: \textcolor{darkblue}{\textbf{\ipa{v̩˧}}} 
\lhead{\firstmark}
\rhead{\botmark}

\subsection{\hspace{-0.5cm} {\Large \textcolor{darkblue}{\textbf{\ipa{ə˧dɑ˧-ə˧mi\#˥}}}}\hspace{0.5cm}[\kern2pt{\textcolor{darkblue}{\textbf{\ipa{xxxx non-correspondance entre le nombre de morphèmes et le nombre de tons de morphèmes}}}}\kern2pt]} \hypertarget{@\string_MdA\string_M-@\string_Mmi\#\string_T1}{}
\markboth{\textcolor{darkblue}{\textbf{\ipa{ə˧dɑ˧-ə˧mi\#˥}}}}{}
\textcolor{teal}{\mytextsc{noun}} \hspace{4pt} Tone: \#H.
\textcolor{Sepia}{\selectlanguage{english}Father and mother.} \zh{父母。}  ¶ \textcolor{darkblue}{\textbf{\ipa{ə˧dɑ˧-ə˧mi˧ ɲi˥-kv̩˩}}} \textcolor{Sepia}{\selectlanguage{english}the father and mother, as a pair} \zh{父母亲}  

\lhead{\firstmark}
\rhead{\botmark}

\subsection{\hspace{-0.5cm} {\Large \textcolor{darkblue}{\textbf{\ipa{ə˧dɑ˧-zo\#˥}}}}\hspace{0.5cm}[\kern2pt{\textcolor{darkblue}{\textbf{\ipa{xxxx non-correspondance entre le nombre de morphèmes et le nombre de tons de morphèmes}}}}\kern2pt]} \hypertarget{@\string_MdA\string_M-zo\#\string_T1}{}
\markboth{\textcolor{darkblue}{\textbf{\ipa{ə˧dɑ˧-zo\#˥}}}}{}
\textcolor{teal}{\mytextsc{noun}} \hspace{4pt} Tone: \#H.
\textcolor{Sepia}{\selectlanguage{english}Father and son.} \zh{父子。} 
\lhead{\firstmark}
\rhead{\botmark}

\subsection{\hspace{-0.5cm} {\Large \textcolor{darkblue}{\textbf{\ipa{ə˧dze˧}}}}\hspace{0.5cm}[\kern2pt{\textcolor{darkblue}{\textbf{\ipa{ə˧dze˧}}}}\kern2pt]} \hypertarget{@\string_Mdze\string_M1}{}
\markboth{\textcolor{darkblue}{\textbf{\ipa{ə˧dze˧}}}}{}
\textcolor{teal}{\mytextsc{noun}} \hspace{4pt} Tone: M.
\textcolor{Sepia}{\selectlanguage{english}Purple gromwell, red-root gromwell, \textit{Lithospermum erythrorhizon Sieb. et Zucc.}.} \zh{紫草。} Local Chinese dialect:\zh{紫红草。} ¶ \textcolor{darkblue}{\textbf{\ipa{ə˧dze˧-njɤ˩hṽ˩}}} \textcolor{Sepia}{\selectlanguage{english}same meaning: purple gromwell} \zh{紫草}  
 ¶ \textcolor{darkblue}{\textbf{\ipa{ə˧dze˧-bæ˩bæ˩}}} \textcolor{Sepia}{\selectlanguage{english}gromwell flowers} \zh{紫草花}  

\lhead{\firstmark}
\rhead{\botmark}

\subsection{\hspace{-0.5cm} {\Large \textcolor{darkblue}{\textbf{\ipa{ə˧-dzɤ˥\$}}}}\hspace{0.5cm}[\kern2pt{\textcolor{darkblue}{\textbf{\ipa{xxxx non-correspondance entre le nombre de morphèmes et le nombre de tons de morphèmes}}}}\kern2pt]} \hypertarget{@\string_M-dz7\string_T\$1}{}
\markboth{\textcolor{darkblue}{\textbf{\ipa{ə˧-dzɤ˥\$}}}}{}
\textcolor{teal}{\mytextsc{adverb(ial)}} \hspace{4pt} Tone: H\$.
\textcolor{Sepia}{\selectlanguage{english}Slowly.} \zh{慢。}  ¶ \textcolor{darkblue}{\textbf{\ipa{ə˧-dzɤ˧ ʝi˧}}} \textcolor{Sepia}{\selectlanguage{english}to work slowly, to do slowly} \zh{慢慢做}  
 ¶ \textcolor{darkblue}{\textbf{\ipa{ə˧dzɤ˧ le˧-hõ˩! |}}} \textcolor{Sepia}{\selectlanguage{english}Goodbye! (Said by the host to their guest. Literally: “Walk slowly!” = “Take your time on the way!”)} \zh{慢走!}  
 ¶ \textcolor{darkblue}{\textbf{\ipa{ə˧dzɤ˥ | le˧-hõ˩! |}}} \textcolor{Sepia}{\selectlanguage{english}Goodbye!} \zh{慢走!}  
 ¶ \textcolor{darkblue}{\textbf{\ipa{ə˧dzɤ˧ le˧-dzi˩! |}}} \textcolor{Sepia}{\selectlanguage{english}Goodbye! (Said by the guest to their host. Literally: “Sit quietly!” = “Take it easy!”)} \zh{慢慢坐!}  
\textit{See:} \textcolor{darkblue}{\textbf{\ipa{ə˧ze˧, ə˧-dzɤ˧\textasciitilde{}dzɤ˥}}} 
\lhead{\firstmark}
\rhead{\botmark}

\subsection{\hspace{-0.5cm} {\Large \textcolor{darkblue}{\textbf{\ipa{ə˧-dzɤ˧\textasciitilde{}dzɤ˥}}}}\hspace{0.5cm}[\kern2pt{\textcolor{darkblue}{\textbf{\ipa{xxxx non-correspondance entre le nombre de morphèmes et le nombre de tons de morphèmes}}}}\kern2pt]} \hypertarget{@\string_M-dz7\string_M~dz7\string_T1}{}
\markboth{\textcolor{darkblue}{\textbf{\ipa{ə˧-dzɤ˧\textasciitilde{}dzɤ˥}}}}{}
\textcolor{teal}{\mytextsc{adverb(ial)}} \hspace{4pt} Tone: H\#.
\textit{From:} \textbf{ə˧-dzɤ˥\$, ə˧ze˧} \textcolor{Sepia}{\selectlanguage{english}Slowly.} \zh{慢慢地。}  ¶ \textcolor{darkblue}{\textbf{\ipa{ʈʂʰɯ˧ | ɖwæ˧˥ | ə˧-dzɤ˧\textasciitilde{}dzɤ˥ ʝi˩-kv̩˩!}}} \textcolor{Sepia}{\selectlanguage{english}(S)he works very carefully. (The literal meaning is 'very slowly'; this is not a criticism, however: it means that they know to take their time in order to do a good job.)} \zh{他工作很细致。(直译:‘他工作很慢’,但不是批评:意味着那个人懂得慢慢来做,做得更仔细。)}  
 ¶ \textcolor{darkblue}{\textbf{\ipa{[F5] ə˧-dzɤ˧\textasciitilde{}dzɤ˥ ʝi˩}}} \textcolor{Sepia}{\selectlanguage{english}to do (something) slowly} \zh{慢慢地做}  
 ¶ \textcolor{darkblue}{\textbf{\ipa{[M21] ə˧-zɤ˧\textasciitilde{}zɤ˥ ʝi˩}}} \textcolor{Sepia}{\selectlanguage{english}to do (something) slowly} \zh{慢慢地做}  
\textit{See:} \hyperlink{}{\textcolor{darkblue}{\textbf{\ipa{ə˧-dzɤ˥\$}}}} 
\lhead{\firstmark}
\rhead{\botmark}

\subsection{\hspace{-0.5cm} {\Large \textcolor{darkblue}{\textbf{\ipa{ə˧ɖo˧}}}}\hspace{0.5cm}[\kern2pt{\textcolor{darkblue}{\textbf{\ipa{ə˧ɖo˧}}}}\kern2pt]} \hypertarget{@\string_Md`o\string_M1}{}
\markboth{\textcolor{darkblue}{\textbf{\ipa{ə˧ɖo˧}}}}{}
\textcolor{teal}{\mytextsc{noun}} \hspace{4pt} Tone: M.
\textcolor{Sepia}{\selectlanguage{english}Lover, boy/girl-friend.} \zh{情人(音译:阿注)。}  \zh{量词}: \textcolor{darkblue}{\textbf{\ipa{v̩˧}}}  \mytextsc{clf}: \textcolor{darkblue}{\textbf{\ipa{v̩˧}}} \textit{See:} \hyperlink{}{\textcolor{darkblue}{\textbf{\ipa{ə˧ɕjɤ˩}}}} 
\lhead{\firstmark}
\rhead{\botmark}

\subsection{\hspace{-0.5cm} {\Large \textcolor{darkblue}{\textbf{\ipa{ə˧go˧}}}}\hspace{0.5cm}[\kern2pt{\textcolor{darkblue}{\textbf{\ipa{ə˧go˧}}}}\kern2pt]} \hypertarget{@\string_Mgo\string_M1}{}
\markboth{\textcolor{darkblue}{\textbf{\ipa{ə˧go˧}}}}{}
\textcolor{teal}{\mytextsc{noun}} \hspace{4pt} Tone: M.
\textcolor{Sepia}{\selectlanguage{english}A family name from Yongning. There are three families in Yongning that carry this name.} \zh{一个姓。这个姓,永宁有三家。}  ¶ \textcolor{darkblue}{\textbf{\ipa{ə˧go˧=ɻ̍˩}}} \textcolor{Sepia}{\selectlanguage{english}the \textcolor{darkblue}{\textbf{\ipa{/ə˧go˧/}}} clan} \zh{\textcolor{darkblue}{\textbf{\ipa{/ə˧go˧/}}}家族}  
 ¶ \textcolor{darkblue}{\textbf{\ipa{ə˧go˧ | dʑɤ˩tsʰi˧}}} \textcolor{Sepia}{\selectlanguage{english}The person called \textcolor{darkblue}{\textbf{\ipa{/dʑɤ˩tsʰi\#˥/}}}, of the \textcolor{darkblue}{\textbf{\ipa{/ə˧go˧/}}} clan} \zh{\textcolor{darkblue}{\textbf{\ipa{/ə˧go˧/}}} 家族名叫\textcolor{darkblue}{\textbf{\ipa{/dʑɤ˩tsʰi\#˥/}}}那个人}  

\lhead{\firstmark}
\rhead{\botmark}

\subsection{\hspace{-0.5cm} {\Large \textcolor{darkblue}{\textbf{\ipa{ə˧go˧-ʁwɤ˧}}}}\hspace{0.5cm}[\kern2pt{\textcolor{darkblue}{\textbf{\ipa{xxxx non-correspondance entre le nombre de morphèmes et le nombre de tons de morphèmes}}}}\kern2pt]} \hypertarget{@\string_Mgo\string_M-Rw7\string_M1}{}
\markboth{\textcolor{darkblue}{\textbf{\ipa{ə˧go˧-ʁwɤ˧}}}}{}
\textcolor{teal}{\mytextsc{noun}} \hspace{4pt} Tone: M.
\textcolor{Sepia}{\selectlanguage{english}Name of a village of the Hot Springs area.} \zh{温泉乡的一个村落。}  ¶ \textcolor{darkblue}{\textbf{\ipa{ə˧go˧-ʁwɤ˧, | ʁwɤ˧lɑ˩-bi˩, | bæ˧ʁwɤ˧, | tʰo˧tsʰe\#˥, | pi˧tsʰe˩-di˩, | pɤ˧dʑɤ˩-di˩, | ʁwɤ˧tv̩˧}}} \textcolor{Sepia}{\selectlanguage{english}Villages that one encounters as one leaves the plain of Yongning (away from the Lake); the first two are perceived as villages with a high proportion of Na members, and the third as a mostly Na village, whereas the next ones are Pumi (Prinmi).} \zh{永宁背向泸沽湖方向经过的村落。前两个村落拥有相当大的摩梭人口比例,第三个村落是摩梭村,最后一个是普米村。}  

\lhead{\firstmark}
\rhead{\botmark}

\subsection{\hspace{-0.5cm} {\Large \textcolor{darkblue}{\textbf{\ipa{ə˧gɯ˩}}}}\hspace{0.5cm}[\kern2pt{\textcolor{darkblue}{\textbf{\ipa{ə˧gɯ˩}}}}\kern2pt]} \hypertarget{@\string_MgM\string_B1}{}
\markboth{\textcolor{darkblue}{\textbf{\ipa{ə˧gɯ˩}}}}{}
\textcolor{teal}{\mytextsc{noun}} \hspace{4pt} Tone: L\#.
\textcolor{Sepia}{\selectlanguage{english}Peppermint.} \zh{薄荷。}  \zh{量词}: \textcolor{darkblue}{\textbf{\ipa{po˧}}}  \mytextsc{clf}: \textcolor{darkblue}{\textbf{\ipa{po˧}}} 
\lhead{\firstmark}
\rhead{\botmark}

\subsection{\hspace{-0.5cm} {\Large \textcolor{darkblue}{\textbf{\ipa{ə˧hɑ˩-bɑ˩lɑ˩}}}}\hspace{0.5cm}[\kern2pt{\textcolor{darkblue}{\textbf{\ipa{ə˧hɑ˩bɑ˧lɑ˧}}}}\kern2pt]} \hypertarget{@\string_MhA\string_B-bA\string_BlA\string_B1}{}
\markboth{\textcolor{darkblue}{\textbf{\ipa{ə˧hɑ˩-bɑ˩lɑ˩}}}}{}
\textcolor{teal}{\mytextsc{noun}} \hspace{4pt} Tone: L\#-.
\textcolor{Sepia}{\selectlanguage{english}Traditional song.} \zh{民歌。}  ¶ \textcolor{darkblue}{\textbf{\ipa{ə˧hɑ˩bɑ˩lɑ˩ | ɖɯ˧-ɖʐo˩ gwɤ˩}}} \textcolor{Sepia}{\selectlanguage{english}to sing a song} \zh{唱一首摩梭歌}  
 \zh{量词}: \textcolor{darkblue}{\textbf{\ipa{ɖʐo˩}}}  \mytextsc{clf}: \textcolor{darkblue}{\textbf{\ipa{ɖʐo˩}}} 
\lhead{\firstmark}
\rhead{\botmark}

\subsection{\hspace{-0.5cm} {\Large \textcolor{darkblue}{\textbf{\ipa{ə˧hĩ˥}}}}\hspace{0.5cm}[\kern2pt{\textcolor{darkblue}{\textbf{\ipa{ə˧hĩ˥}}}}\kern2pt]} \hypertarget{@\string_Mhi\string_~\string_T1}{}
\markboth{\textcolor{darkblue}{\textbf{\ipa{ə˧hĩ˥}}}}{}
\textcolor{teal}{\mytextsc{adverb(ial)}} \hspace{4pt} Tone: H\#.
\textcolor{Sepia}{\selectlanguage{english}In a moment.} \zh{一会儿、待会儿、等一下。}  ¶ \textcolor{darkblue}{\textbf{\ipa{ə˧hĩ˥-ɳɯ˩, | li˧-kʰɯ˧-bi˥!}}} \textcolor{Sepia}{\selectlanguage{english}I will show you in a moment!} \zh{待会儿,我给你看吧!}  

\lhead{\firstmark}
\rhead{\botmark}

\subsection{\hspace{-0.5cm} {\Large \textcolor{darkblue}{\textbf{\ipa{ə˧hwɤ˧}}}}\hspace{0.5cm}[\kern2pt{\textcolor{darkblue}{\textbf{\ipa{ə˧hwɤ˧}}}}\kern2pt]} \hypertarget{@\string_Mhw7\string_M1}{}
\markboth{\textcolor{darkblue}{\textbf{\ipa{ə˧hwɤ˧}}}}{}
\textcolor{teal}{\mytextsc{adverb(ial)}} \hspace{4pt} Tone: M.
\textcolor{Sepia}{\selectlanguage{english}Yesterday evening.} \zh{昨晚。}  ¶ \textcolor{darkblue}{\textbf{\ipa{ə˧hwɤ˧ | mv̩˩kʰv̩˧˥}}} \textcolor{Sepia}{\selectlanguage{english}yesterday evening, during the night} \zh{昨晚,夜里}  

\lhead{\firstmark}
\rhead{\botmark}

\subsection{\hspace{-0.5cm} {\Large \textcolor{darkblue}{\textbf{\ipa{ə˧jɤ˩}}}}\hspace{0.5cm}[\kern2pt{\textcolor{darkblue}{\textbf{\ipa{ə˧jɤ˩}}}}\kern2pt]} \hypertarget{@\string_Mj7\string_B1}{}
\markboth{\textcolor{darkblue}{\textbf{\ipa{ə˧jɤ˩}}}}{}
\textcolor{teal}{\mytextsc{noun}} \hspace{4pt} Tone: L\#.
\textcolor{Sepia}{\selectlanguage{english}Maternal aunt (mother's elder sister).} \zh{姨母 (比母亲大)。}  \zh{量词}: \textcolor{darkblue}{\textbf{\ipa{v̩˧}}}  \mytextsc{clf}: \textcolor{darkblue}{\textbf{\ipa{v̩˧}}} 
\lhead{\firstmark}
\rhead{\botmark}

\subsection{\hspace{-0.5cm} {\Large \textcolor{darkblue}{\textbf{\ipa{ə˧ʝi˥\$}}}}\hspace{0.5cm}[\kern2pt{\textcolor{darkblue}{\textbf{\ipa{ə˧ʝi˥}}}}\kern2pt]} \hypertarget{@\string_Mj££i\string_T\$1}{}
\markboth{\textcolor{darkblue}{\textbf{\ipa{ə˧ʝi˥\$}}}}{}
\textcolor{teal}{\mytextsc{adverb(ial)}} \hspace{4pt} Tone: H\$.
\textcolor{Sepia}{\selectlanguage{english}Last year.} \zh{去年。} 
\lhead{\firstmark}
\rhead{\botmark}

\subsection{\hspace{-0.5cm} {\Large \textcolor{darkblue}{\textbf{\ipa{ə˧ʝi˧-ʂɯ˥ʝi˩}}}}\hspace{0.5cm}[\kern2pt{\textcolor{darkblue}{\textbf{\ipa{ə˧ʝi˧ʂɯ˥ʝi˩}}}}\kern2pt]} \hypertarget{@\string_Mj££i\string_M-s`M\string_Tj££i\string_B1}{}
\markboth{\textcolor{darkblue}{\textbf{\ipa{ə˧ʝi˧-ʂɯ˥ʝi˩}}}}{}
\textcolor{teal}{\mytextsc{adverb(ial)}} \hspace{4pt} Tone: \#H-.
\textcolor{Sepia}{\selectlanguage{english}Long ago; in the past; once upon a time.} \zh{很久以前,古时候,传说古代。} 
\lhead{\firstmark}
\rhead{\botmark}

\subsection{\hspace{-0.5cm} {\Large \textcolor{darkblue}{\textbf{\ipa{ə˧ʝi˧-tsʰi˧ʝi\#˥}}}}\hspace{0.5cm}[\kern2pt{\textcolor{darkblue}{\textbf{\ipa{xxxx non-correspondance entre le nombre de morphèmes et le nombre de tons de morphèmes}}}}\kern2pt]} \hypertarget{@\string_Mj££i\string_M-ts\string_hi\string_Mj££i\#\string_T1}{}
\markboth{\textcolor{darkblue}{\textbf{\ipa{ə˧ʝi˧-tsʰi˧ʝi\#˥}}}}{}
\textcolor{teal}{\mytextsc{adverb(ial)}} \hspace{4pt} Tone: \#H.
\textcolor{Sepia}{\selectlanguage{english}These years, currently.} \zh{这几年、现在这个时代。} 
\lhead{\firstmark}
\rhead{\botmark}

\subsection{\hspace{-0.5cm} {\Large \textcolor{darkblue}{\textbf{\ipa{ə˧lɑ˧}}}}\hspace{0.5cm}[\kern2pt{\textcolor{darkblue}{\textbf{\ipa{ə˧lɑ˧}}}}\kern2pt]} \hypertarget{@\string_MlA\string_M1}{}
\markboth{\textcolor{darkblue}{\textbf{\ipa{ə˧lɑ˧}}}}{}
\textcolor{teal}{\mytextsc{noun}} \hspace{4pt} Tone: M.
\textcolor{Sepia}{\selectlanguage{english}A family name from Yongning. There are three families in Yongning that carry this name. This is one of the three first Na clans who settled in the vicinity of the monastery, the other two being \textcolor{darkblue}{\textbf{\ipa{/kɤ˧˥tʰɑ˩/}}} and \textcolor{darkblue}{\textbf{\ipa{/lɑ˧tʰɑ˧mi˥\$/}}}.} \zh{一个姓。这个姓,永宁有三家。}  ¶ \textcolor{darkblue}{\textbf{\ipa{ə˧lɑ˧=ɻ̍˩}}} \textcolor{Sepia}{\selectlanguage{english}the \textcolor{darkblue}{\textbf{\ipa{/ə˧lɑ˧/}}} clan} \zh{\textcolor{darkblue}{\textbf{\ipa{/ə˧lɑ˧/}}}家族}  

\lhead{\firstmark}
\rhead{\botmark}

\subsection{\hspace{-0.5cm} {\Large \textcolor{darkblue}{\textbf{\ipa{ə˧lɑ˧-ʁwɤ\#˥}}}}\hspace{0.5cm}[\kern2pt{\textcolor{darkblue}{\textbf{\ipa{xxxx non-correspondance entre le nombre de morphèmes et le nombre de tons de morphèmes}}}}\kern2pt]} \hypertarget{@\string_MlA\string_M-Rw7\#\string_T1}{}
\markboth{\textcolor{darkblue}{\textbf{\ipa{ə˧lɑ˧-ʁwɤ\#˥}}}}{}
\textcolor{teal}{\mytextsc{noun}} \hspace{4pt} Tone: \#H.
\textcolor{Sepia}{\selectlanguage{english}A hamlet of Yongning, close to the monastery.} \zh{永宁寺旁边的村落(主合作人住的地方)。(音译:阿拉瓦,旧名:七家村,因为村落在1960年左右有七个家庭)。}  ¶ \textcolor{darkblue}{\textbf{\ipa{dʑɤ˩bv̩˧kɤ˧-sɑ˥ʁwɤ˩, | hi˩ʁwɤ˩-lo˥, | æ˩mi˧-ʁwɤ\#˥, | lɑ˧lo˧-ʁwɤ˥, | lɑ˧ŋwɤ˧, | bɤ˧tsʰo˧gv̩˥, | ə˧lɑ˧-ʁwɤ\#˥, | gæ˧ɻæ˩, | qʰæ˧tɕʰi˧, | tʰo˧ʈɯ\#˥}}} \textcolor{Sepia}{\selectlanguage{english}the ten villages traditionally considered as part of Yongning} \zh{摩梭传统地理概念中,属于永宁的十个村落}  

\lhead{\firstmark}
\rhead{\botmark}

\subsection{\hspace{-0.5cm} {\Large \textcolor{darkblue}{\textbf{\ipa{ə˧mɑ˧}}}}\hspace{0.5cm}[\kern2pt{\textcolor{darkblue}{\textbf{\ipa{ə˧mɑ˧}}}}\kern2pt]} \hypertarget{@\string_MmA\string_M1}{}
\markboth{\textcolor{darkblue}{\textbf{\ipa{ə˧mɑ˧}}}}{}
\textcolor{teal}{\mytextsc{noun}} \hspace{4pt} Tone: M.
\textcolor{Sepia}{\selectlanguage{english}Mother (term of address used by children).} \zh{阿妈(孩子对母亲的称呼)。}  \zh{量词}: \textcolor{darkblue}{\textbf{\ipa{v̩˧}}}  \mytextsc{clf}: \textcolor{darkblue}{\textbf{\ipa{v̩˧}}} 
\lhead{\firstmark}
\rhead{\botmark}

\subsection{\hspace{-0.5cm} {\Large \textcolor{darkblue}{\textbf{\ipa{ə˧mi˧}}}}\hspace{0.5cm}[\kern2pt{\textcolor{darkblue}{\textbf{\ipa{ə˧mi˧}}}}\kern2pt]} \hypertarget{@\string_Mmi\string_M1}{}
\markboth{\textcolor{darkblue}{\textbf{\ipa{ə˧mi˧}}}}{}
\textcolor{teal}{\mytextsc{noun}} \hspace{4pt} Tone: M.
\textcolor{Sepia}{\selectlanguage{english}Mother; aunt.} \zh{母亲、姑母、姨母、伯母、叔母、大娘、婶、大妈、姨、伯母、舅母、大婶、大姨、阿姨、妗母、妗子、舅妈、婶母、婶娘、婶子、叔母、姨妈、姨母、姨娘。}  ¶ \textcolor{darkblue}{\textbf{\ipa{ə˧mi˧=ɻæ˩}}} \textcolor{Sepia}{\selectlanguage{english}\string_ \mytextsc{associative}} \zh{母亲们 =长辈女性}  
 \zh{量词}: \textcolor{darkblue}{\textbf{\ipa{v̩˧}}} \textcolor{darkblue}{\textbf{\ipa{jɤ˧˥}}}  \mytextsc{clf}: \textcolor{darkblue}{\textbf{\ipa{v̩˧}}} \textcolor{darkblue}{\textbf{\ipa{jɤ˧˥}}} 
\lhead{\firstmark}
\rhead{\botmark}

\subsection{\hspace{-0.5cm} {\Large \textcolor{darkblue}{\textbf{\ipa{ə˧mi˧-ɖɯ˩}}}}\hspace{0.5cm}[\kern2pt{\textcolor{darkblue}{\textbf{\ipa{xxxx non-correspondance entre le nombre de morphèmes et le nombre de tons de morphèmes}}}}\kern2pt]} \hypertarget{@\string_Mmi\string_M-d`M\string_B1}{}
\markboth{\textcolor{darkblue}{\textbf{\ipa{ə˧mi˧-ɖɯ˩}}}}{}
\textcolor{teal}{\mytextsc{noun}} \hspace{4pt} Tone: L\#.
\textcolor{Sepia}{\selectlanguage{english}Maternal aunt (mother's elder sister).} \zh{姨母 (比母亲大)。}  \zh{量词}: \textcolor{darkblue}{\textbf{\ipa{v̩˧}}}  \mytextsc{clf}: \textcolor{darkblue}{\textbf{\ipa{v̩˧}}} 
\lhead{\firstmark}
\rhead{\botmark}

\subsection{\hspace{-0.5cm} {\Large \textcolor{darkblue}{\textbf{\ipa{ə˧mi˧-tɕi˩}}}}\hspace{0.5cm}[\kern2pt{\textcolor{darkblue}{\textbf{\ipa{xxxx non-correspondance entre le nombre de morphèmes et le nombre de tons de morphèmes}}}}\kern2pt]} \hypertarget{@\string_Mmi\string_M-ts£i\string_B1}{}
\markboth{\textcolor{darkblue}{\textbf{\ipa{ə˧mi˧-tɕi˩}}}}{}
\textcolor{teal}{\mytextsc{noun}} \hspace{4pt} Tone: L\#.
\textcolor{Sepia}{\selectlanguage{english}Maternal aunt (mother's younger sister).} \zh{姨母 (比母亲小)。}  \zh{量词}: \textcolor{darkblue}{\textbf{\ipa{v̩˧}}}  \mytextsc{clf}: \textcolor{darkblue}{\textbf{\ipa{v̩˧}}} 
\lhead{\firstmark}
\rhead{\botmark}

\subsection{\hspace{-0.5cm} {\Large \textcolor{darkblue}{\textbf{\ipa{ə˧mi˧-zo\#˥}}}}\hspace{0.5cm}[\kern2pt{\textcolor{darkblue}{\textbf{\ipa{xxxx non-correspondance entre le nombre de morphèmes et le nombre de tons de morphèmes}}}}\kern2pt]} \hypertarget{@\string_Mmi\string_M-zo\#\string_T1}{}
\markboth{\textcolor{darkblue}{\textbf{\ipa{ə˧mi˧-zo\#˥}}}}{}
\textcolor{teal}{\mytextsc{noun}} \hspace{4pt} Tone: \#H.
\textcolor{Sepia}{\selectlanguage{english}Mother and son.} \zh{母子。} 
\lhead{\firstmark}
\rhead{\botmark}

\subsection{\hspace{-0.5cm} {\Large \textcolor{darkblue}{\textbf{\ipa{ə˧mv̩˩}}}}\hspace{0.5cm}[\kern2pt{\textcolor{darkblue}{\textbf{\ipa{ə˧mv̩˩}}}}\kern2pt]} \hypertarget{@\string_Mmv\string_=\string_B1}{}
\markboth{\textcolor{darkblue}{\textbf{\ipa{ə˧mv̩˩}}}}{}
\textcolor{teal}{\mytextsc{noun}} \hspace{4pt} Tone: L\#.
\textcolor{Sepia}{\selectlanguage{english}Elder sibling (brother or sister).} \zh{哥哥,姐姐(也指堂哥堂姐)。}  ¶ \textcolor{darkblue}{\textbf{\ipa{æ˧mv̩˩=ɻæ˩}}} \textcolor{Sepia}{\selectlanguage{english}\mytextsc{associative}: elder siblings} \zh{联想复数:哥哥们、姐姐们}  
 \zh{量词}: \textcolor{darkblue}{\textbf{\ipa{v̩˧}}}  \mytextsc{clf}: \textcolor{darkblue}{\textbf{\ipa{v̩˧}}} 
\lhead{\firstmark}
\rhead{\botmark}

\subsection{\hspace{-0.5cm} {\Large \textcolor{darkblue}{\textbf{\ipa{ə˧mv̩˧-gi˥zɯ˩}}}}\hspace{0.5cm}[\kern2pt{\textcolor{darkblue}{\textbf{\ipa{ə˧mv̩˧gi˥zɯ˩}}}}\kern2pt]} \hypertarget{@\string_Mmv\string_=\string_M-gi\string_TzM\string_B1}{}
\markboth{\textcolor{darkblue}{\textbf{\ipa{ə˧mv̩˧-gi˥zɯ˩}}}}{}
\textcolor{teal}{\mytextsc{noun}} \hspace{4pt} Tone: \#H-.
\textcolor{Sepia}{\selectlanguage{english}Brothers, irrespective of age (elder or younger).} \zh{兄弟(哥哥们与弟弟们)。} 
\lhead{\firstmark}
\rhead{\botmark}

\subsection{\hspace{-0.5cm} {\Large \textcolor{darkblue}{\textbf{\ipa{ə˧mv̩˥-tɕi˩}}}}\hspace{0.5cm}[\kern2pt{\textcolor{darkblue}{\textbf{\ipa{xxxx non-correspondance entre le nombre de morphèmes et le nombre de tons de morphèmes}}}}\kern2pt]} \hypertarget{@\string_Mmv\string_=\string_T-ts£i\string_B1}{}
\markboth{\textcolor{darkblue}{\textbf{\ipa{ə˧mv̩˥-tɕi˩}}}}{}
\textcolor{teal}{\mytextsc{adjective}} \hspace{4pt} Tone: H\#.
\textcolor{Sepia}{\selectlanguage{english}Very small, diminutive.} \zh{小、细小。}  ¶ \textcolor{darkblue}{\textbf{\ipa{ə˧mv̩˥-tɕi˩-gv̩˩}}} \textcolor{Sepia}{\selectlanguage{english}very small, diminutive} \zh{小、细小}  
 ¶ \textcolor{darkblue}{\textbf{\ipa{ə˧mv̩˥-tɕi˩-hĩ˩}}} \textcolor{Sepia}{\selectlanguage{english}very small} \zh{细小的}  
 ¶ \textcolor{darkblue}{\textbf{\ipa{ə˧mv̩˥tɕi˩ | ɖɯ˧-kʰwɤ˥}}} \textcolor{Sepia}{\selectlanguage{english}a little piece, a little bit} \zh{一小块}  

\lhead{\firstmark}
\rhead{\botmark}

\subsection{\hspace{-0.5cm} {\Large \textcolor{darkblue}{\textbf{\ipa{ə˧ɲi˥\$}}}}\hspace{0.5cm}[\kern2pt{\textcolor{darkblue}{\textbf{\ipa{ə˧ɲi˥}}}}\kern2pt]} \hypertarget{@\string_MJi\string_T\$1}{}
\markboth{\textcolor{darkblue}{\textbf{\ipa{ə˧ɲi˥\$}}}}{}
\textcolor{teal}{\mytextsc{adverb(ial)}} \hspace{4pt} Tone: H\$.
\textcolor{Sepia}{\selectlanguage{english}Yesterday.} \zh{昨天。} 
\lhead{\firstmark}
\rhead{\botmark}

\subsection{\hspace{-0.5cm} {\Large \textcolor{darkblue}{\textbf{\ipa{ə˧ɲi˥-tsæ˩qæ˩}}}}\hspace{0.5cm}[\kern2pt{\textcolor{darkblue}{\textbf{\ipa{ə˧ɲi˥tsæ˩qæ˩}}}}\kern2pt]} \hypertarget{@\string_MJi\string_T-ts\{\string_Bq\{\string_B1}{}
\markboth{\textcolor{darkblue}{\textbf{\ipa{ə˧ɲi˥-tsæ˩qæ˩}}}}{}
\textcolor{teal}{\mytextsc{noun}} \hspace{4pt} Tone: H\#-L.
\textcolor{Sepia}{\selectlanguage{english}Little finger.} \zh{小指。}  \zh{量词}: \textcolor{darkblue}{\textbf{\ipa{ɭɯ˧}}}  \mytextsc{clf}: \textcolor{darkblue}{\textbf{\ipa{ɭɯ˧}}} 
\lhead{\firstmark}
\rhead{\botmark}

\subsection{\hspace{-0.5cm} {\Large \textcolor{darkblue}{\textbf{\ipa{ə˧ɲi˧-tsʰi˧ɲi\#˥}}}}\hspace{0.5cm}[\kern2pt{\textcolor{darkblue}{\textbf{\ipa{xxxx non-correspondance entre le nombre de morphèmes et le nombre de tons de morphèmes}}}}\kern2pt]} \hypertarget{@\string_MJi\string_M-ts\string_hi\string_MJi\#\string_T1}{}
\markboth{\textcolor{darkblue}{\textbf{\ipa{ə˧ɲi˧-tsʰi˧ɲi\#˥}}}}{}
\textcolor{teal}{\mytextsc{adverb(ial)}} \hspace{4pt} Tone: \#H.
\textcolor{Sepia}{\selectlanguage{english}These days.} \zh{近来。} 
\lhead{\firstmark}
\rhead{\botmark}

\subsection{\hspace{-0.5cm} {\Large \textcolor{darkblue}{\textbf{\ipa{ə˧pʰv̩˧}}}}\hspace{0.5cm}[\kern2pt{\textcolor{darkblue}{\textbf{\ipa{ə˧pʰv̩˧}}}}\kern2pt]} \hypertarget{@\string_Mp\string_hv\string_=\string_M1}{}
\markboth{\textcolor{darkblue}{\textbf{\ipa{ə˧pʰv̩˧}}}}{}
\textcolor{teal}{\mytextsc{noun}} \hspace{4pt} Tone: M.
\textcolor{Sepia}{\selectlanguage{english}Grandmother's brother=mother's uncle (on her mother's side); extended meaning: male elder two generations above oneself.} \zh{舅姥爷:姥姥的哥哥或弟弟(也就是母亲的舅舅)。泛指:“祖父”。}  \zh{量词}: \textcolor{darkblue}{\textbf{\ipa{v̩˧}}}  \mytextsc{clf}: \textcolor{darkblue}{\textbf{\ipa{v̩˧}}} 
\lhead{\firstmark}
\rhead{\botmark}

\subsection{\hspace{-0.5cm} {\Large \textcolor{darkblue}{\textbf{\ipa{ə˧si˧}}}}\hspace{0.5cm}[\kern2pt{\textcolor{darkblue}{\textbf{\ipa{ə˧si˧}}}}\kern2pt]} \hypertarget{@\string_Msi\string_M1}{}
\markboth{\textcolor{darkblue}{\textbf{\ipa{ə˧si˧}}}}{}
\textcolor{teal}{\mytextsc{noun}} \hspace{4pt} Tone: M.
\textcolor{Sepia}{\selectlanguage{english}Great-grandmother (on the mother's side); extended meaning: great-grandmother and her brothers and sisters: third-generation elders.} \zh{祖母。泛指:祖母与其兄弟姐妹。}  \zh{量词}: \textcolor{darkblue}{\textbf{\ipa{v̩˧}}}  \mytextsc{clf}: \textcolor{darkblue}{\textbf{\ipa{v̩˧}}} 
\lhead{\firstmark}
\rhead{\botmark}

\subsection{\hspace{-0.5cm} {\Large \textcolor{darkblue}{\textbf{\ipa{ə˧si˧-ə˧pʰv̩\#˥}}}}\hspace{0.5cm}[\kern2pt{\textcolor{darkblue}{\textbf{\ipa{xxxx non-correspondance entre le nombre de morphèmes et le nombre de tons de morphèmes}}}}\kern2pt]} \hypertarget{@\string_Msi\string_M-@\string_Mp\string_hv\string_=\#\string_T1}{}
\markboth{\textcolor{darkblue}{\textbf{\ipa{ə˧si˧-ə˧pʰv̩\#˥}}}}{}
\textcolor{teal}{\mytextsc{noun}} \hspace{4pt} Tone: \#H.
\textcolor{Sepia}{\selectlanguage{english}Ancestors of the third and fourth generations.} \zh{祖宗(三、四代以前)。} 
\lhead{\firstmark}
\rhead{\botmark}

\subsection{\hspace{-0.5cm} {\Large \textcolor{darkblue}{\textbf{\ipa{ə˧so˧}}}}\hspace{0.5cm}[\kern2pt{\textcolor{darkblue}{\textbf{\ipa{ə˧so˧}}}}\kern2pt]} \hypertarget{@\string_Mso\string_M1}{}
\markboth{\textcolor{darkblue}{\textbf{\ipa{ə˧so˧}}}}{}
\textcolor{teal}{\mytextsc{adverb(ial)}} \hspace{4pt} Tone: M.
\textcolor{Sepia}{\selectlanguage{english}A short time ago, a moment ago.} \zh{刚才。} 
\lhead{\firstmark}
\rhead{\botmark}

\subsection{\hspace{-0.5cm} {\Large \textcolor{darkblue}{\textbf{\ipa{ə˧ʂɯ˧ɲi˥-ɖɯ˧ɲi˥}}}}\hspace{0.5cm}[\kern2pt{\textcolor{darkblue}{\textbf{\ipa{ə˧ʂɯ˧ɲi˥ɖɯ˩ɲi˩}}}}\kern2pt]} \hypertarget{@\string_Ms`M\string_MJi\string_T-d`M\string_MJi\string_T1}{}
\markboth{\textcolor{darkblue}{\textbf{\ipa{ə˧ʂɯ˧ɲi˥-ɖɯ˧ɲi˥}}}}{}
\textcolor{teal}{\mytextsc{adverb(ial)}} \hspace{4pt} Tone: H\#-H\$.
\textcolor{Sepia}{\selectlanguage{english}The past few days.} \zh{前几天。} 
\lhead{\firstmark}
\rhead{\botmark}

\subsection{\hspace{-0.5cm} {\Large \textcolor{darkblue}{\textbf{\ipa{ə˧ti˥-dzi˩}}}}\hspace{0.5cm}[\kern2pt{\textcolor{darkblue}{\textbf{\ipa{ə˧ti˥dzi˩}}}}\kern2pt]} \hypertarget{@\string_Mti\string_T-dzi\string_B1}{}
\markboth{\textcolor{darkblue}{\textbf{\ipa{ə˧ti˥-dzi˩}}}}{}
\textcolor{teal}{\mytextsc{noun}} \hspace{4pt} Tone: H\#-.
\textcolor{Sepia}{\selectlanguage{english}The city of Weixi, in Yunnan.} \zh{维西。} 
\lhead{\firstmark}
\rhead{\botmark}

\subsection{\hspace{-0.5cm} {\Large \textcolor{darkblue}{\textbf{\ipa{ə˧tɕi˩}}}}\hspace{0.5cm}[\kern2pt{\textcolor{darkblue}{\textbf{\ipa{ə˧tɕi˩}}}}\kern2pt]} \hypertarget{@\string_Mts£i\string_B1}{}
\markboth{\textcolor{darkblue}{\textbf{\ipa{ə˧tɕi˩}}}}{}
\textcolor{teal}{\mytextsc{noun}} \hspace{4pt} Tone: L\#.
\textcolor{Sepia}{\selectlanguage{english}Maternal aunt (mother's younger sister).} \zh{姨母 (比母亲小)。}  ¶ \textcolor{darkblue}{\textbf{\ipa{ə˧tɕi˩=ɻæ˩}}} \textcolor{Sepia}{\selectlanguage{english}\mytextsc{associative}: aunts} \zh{姨母们}  
 \zh{量词}: \textcolor{darkblue}{\textbf{\ipa{v̩˧}}}  \mytextsc{clf}: \textcolor{darkblue}{\textbf{\ipa{v̩˧}}} 
\lhead{\firstmark}
\rhead{\botmark}

\subsection{\hspace{-0.5cm} {\Large \textcolor{darkblue}{\textbf{\ipa{ə˧tse˥\$}}}}\hspace{0.5cm}[\kern2pt{\textcolor{darkblue}{\textbf{\ipa{ə˧tse˥}}}}\kern2pt]} \hypertarget{@\string_Mtse\string_T\$1}{}
\markboth{\textcolor{darkblue}{\textbf{\ipa{ə˧tse˥\$}}}}{}
\textcolor{teal}{\mytextsc{adverb(ial)}} \hspace{4pt} Tone: H\$.
\textcolor{Sepia}{\selectlanguage{english}Why.} \zh{为什么。}  ¶ \textcolor{darkblue}{\textbf{\ipa{ə˧tse˧-ʝi˥ / ə˧tse˧-ʝi˧}}} \textcolor{Sepia}{\selectlanguage{english}Why? (Two variants; same meaning)} \zh{为什么?(有两个变体,意思一致)}  
 ¶ \textcolor{darkblue}{\textbf{\ipa{no˧ | ə˧tse˧-ʝi˥ | mɤ˧-tsʰɯ˩ ɲi˩? / no˧ | ə˧tse˧-ʝi˧-zo˥ | mɤ˧-tsʰɯ˩ ɲi˩?}}} \textcolor{Sepia}{\selectlanguage{english}Why didn't you come?} \zh{你为什么没有来?}  
 ¶ \textcolor{darkblue}{\textbf{\ipa{no˧ | ə˧tse˧-ʝi˥ | mɤ˧-dzɯ˥? = no˧ | ə˧tse˧-ʝi˥ | mɤ˧-dzɯ˧-ɲi˥?}}} \textcolor{Sepia}{\selectlanguage{english}Why don't you eat?} \zh{你为什么不吃?}  
 ¶ \textcolor{darkblue}{\textbf{\ipa{ʈʂʰɯ˧ | ə˧tse˧-ɲi˥-hɯ˩?}}} \textcolor{Sepia}{\selectlanguage{english}What does that mean?} \zh{这是怎么一回事?}  

\lhead{\firstmark}
\rhead{\botmark}

\subsection{\hspace{-0.5cm} {\Large \textcolor{darkblue}{\textbf{\ipa{ə˧tso˧}}}}\hspace{0.5cm}[\kern2pt{\textcolor{darkblue}{\textbf{\ipa{ə˧tso˧}}}}\kern2pt]} \hypertarget{@\string_Mtso\string_M1}{}
\markboth{\textcolor{darkblue}{\textbf{\ipa{ə˧tso˧}}}}{}
\textcolor{teal}{\mytextsc{pronoun/pronominal}} \hspace{4pt} Tone: M.
\textcolor{Sepia}{\selectlanguage{english}What.} \zh{什么。}  ¶ \textcolor{darkblue}{\textbf{\ipa{ə˧tso˧ ɲi˩?}}} \textcolor{Sepia}{\selectlanguage{english}What is it?} \zh{是什么?}  
 ¶ \textcolor{darkblue}{\textbf{\ipa{no˧ | ə˧tso˧ ʝi˧ bi˧?}}} \textcolor{Sepia}{\selectlanguage{english}What are you going to do?} \zh{你要做什么?}  
 ¶ \textcolor{darkblue}{\textbf{\ipa{no˧ | ə˧tse˧ bi˧?}}} \textcolor{Sepia}{\selectlanguage{english}contracted form of the previous example: \textcolor{darkblue}{\textbf{\ipa{/tso˧/}}} and the following \textcolor{darkblue}{\textbf{\ipa{/ʝi˧/}}} fuse into a single syllable, \textcolor{darkblue}{\textbf{\ipa{[tse˧]}}}.} \zh{上面例子的缩短格式:\textcolor{darkblue}{\textbf{\ipa{/tso˧/}}}与\textcolor{darkblue}{\textbf{\ipa{/ʝi˧/}}}合成一个音节,\textcolor{darkblue}{\textbf{\ipa{[tse˧]}}}。}  

\lhead{\firstmark}
\rhead{\botmark}

\subsection{\hspace{-0.5cm} {\Large \textcolor{darkblue}{\textbf{\ipa{ə˧tso˧\textasciitilde{}ə˧tso˥}}}}\hspace{0.5cm}[\kern2pt{\textcolor{darkblue}{\textbf{\ipa{ə˧tso˧ə˧tso˥}}}}\kern2pt]} \hypertarget{@\string_Mtso\string_M~@\string_Mtso\string_T1}{}
\markboth{\textcolor{darkblue}{\textbf{\ipa{ə˧tso˧\textasciitilde{}ə˧tso˥}}}}{}
\textcolor{teal}{\mytextsc{pronoun/pronominal}} \hspace{4pt} Tone: H\#.
\textcolor{Sepia}{\selectlanguage{english}What (reduplicated).} \zh{什么(重叠)。} 
\lhead{\firstmark}
\rhead{\botmark}

\subsection{\hspace{-0.5cm} {\Large \textcolor{darkblue}{\textbf{\ipa{ə˧tso˧-mɤ˧-ɲi˩}}}}\hspace{0.5cm}[\kern2pt{\textcolor{darkblue}{\textbf{\ipa{xxxx non-correspondance entre le nombre de morphèmes et le nombre de tons de morphèmes}}}}\kern2pt]} \hypertarget{@\string_Mtso\string_M-m7\string_M-Ji\string_B1}{}
\markboth{\textcolor{darkblue}{\textbf{\ipa{ə˧tso˧-mɤ˧-ɲi˩}}}}{}
\textcolor{teal}{\mytextsc{pronoun/pronominal}} \hspace{4pt} Tone: L\#.
\textcolor{Sepia}{\selectlanguage{english}All, all sorts of.} \zh{各种。} 
\lhead{\firstmark}
\rhead{\botmark}

\subsection{\hspace{-0.5cm} {\Large \textcolor{darkblue}{\textbf{\ipa{ə˧v̩˧˥}}} \textsubscript{1}}\hspace{0.5cm}[\kern2pt{\textcolor{darkblue}{\textbf{\ipa{ə˧v̩˧˥}}}}\kern2pt]} \hypertarget{@\string_Mv\string_=\string_M\string_T1}{}
\markboth{\textcolor{darkblue}{\textbf{\ipa{ə˧v̩˧˥}}} \textsubscript{1}}{}
\textcolor{teal}{\mytextsc{adjective}} \hspace{4pt} Tone: MH\#.
\textcolor{Sepia}{\selectlanguage{english}Beautiful, pretty.} \zh{美,好看,美丽。}  ¶ \textcolor{darkblue}{\textbf{\ipa{ɖwæ˧˥ | ə˧v̩˧˥}}} \textcolor{Sepia}{\selectlanguage{english}\mytextsc{intensive}.very} \zh{很好看!}  
 ¶ \textcolor{darkblue}{\textbf{\ipa{ə˧-mɤ˧-v̩˧˥}}} \textcolor{Sepia}{\selectlanguage{english}\mytextsc{neg}: ugly} \zh{丑陋}  

\lhead{\firstmark}
\rhead{\botmark}

\subsection{\hspace{-0.5cm} {\Large \textcolor{darkblue}{\textbf{\ipa{ə˧v̩˧˥}}} \textsubscript{2}}\hspace{0.5cm}[\kern2pt{\textcolor{darkblue}{\textbf{\ipa{ə˧v̩˧˥}}}}\kern2pt]} \hypertarget{@\string_Mv\string_=\string_M\string_T2}{}
\markboth{\textcolor{darkblue}{\textbf{\ipa{ə˧v̩˧˥}}} \textsubscript{2}}{}
\textcolor{teal}{\mytextsc{noun}} \hspace{4pt} Tone: MH\#.
\textcolor{Sepia}{\selectlanguage{english}Maternal uncle (mother's brother: same term for older brother and younger brother).} \zh{舅舅、舅父 (比母亲大或比母亲小不区分)。}  ¶ \textcolor{darkblue}{\textbf{\ipa{ə˧v̩˧-ɖɯ˧˥}}} \textcolor{Sepia}{\selectlanguage{english}mother's elder brother} \zh{比母亲大的舅舅}  
 ¶ \textcolor{darkblue}{\textbf{\ipa{ə˧v̩˧-tɕi˥}}} \textcolor{Sepia}{\selectlanguage{english}mother's younger brother} \zh{比母亲小的舅舅}  
 ¶ \textcolor{darkblue}{\textbf{\ipa{mv˧ʁo˥ | tʰi˧-dze˩, | kɤ˩-nɑ˧mi˧ ɖɯ˧˥ ! | di˧qo˧ ʈʰɯ˧-dʑo˩, | ə˧v˧ ɖɯ˧˥!}}} \textcolor{Sepia}{\selectlanguage{english}“As the Eagle is greatest of all that fly in the sky, so the Uncle is greatest of all that walk the earth.”} \zh{“天上飞的,是老鹰最大。天下走的,是舅舅最大。”}  
 ¶ \textcolor{darkblue}{\textbf{\ipa{mv˧ʁo˥ dze˩hĩ˩-dʑo˥, | kɤ˩-nɑ˧mi˧; | di˧qo˧ se˧-dʑo˩, | ə˧v˧˥!}}} \textcolor{Sepia}{\selectlanguage{english}“As the Eagle is greatest of all that fly in the sky, so the Uncle is greatest of all that walk the earth.”} \zh{“天上飞的,是老鹰最大。天下走的,是舅舅最大。”}  
 ¶ \textcolor{darkblue}{\textbf{\ipa{mv˧ʁo˥ dze˩hĩ˩˥ | -dʑo˥, | kɤ˩-nɑ˧mi˧; | di˧qo˧ se˧-dʑo˩, | ə˧v˧˥!}}} \textcolor{Sepia}{\selectlanguage{english}“As the Eagle is greatest of all that fly in the sky, so the Uncle is greatest of all that walk the earth.”} \zh{“天上飞的,是老鹰最大。天下走的,是舅舅最大。”}  
 ¶ \textcolor{darkblue}{\textbf{\ipa{mv˧ʁo˥ dze˩hĩ˩-dʑo˥, | kɤ˩-nɑ˧mi˧; | di˧qo˧-dʑo˧, | ə˧v˧˥!}}} \textcolor{Sepia}{\selectlanguage{english}“As the Eagle is greatest of all that fly in the sky, so the Uncle is greatest of all that walk the earth.”} \zh{“天上飞的,是老鹰最大。天下走的,是舅舅最大。”}  
 ¶ \textcolor{darkblue}{\textbf{\ipa{mv˧ʁo˥ | dze˩-hĩ˩-dʑo˥, | ɖɯ˩-hĩ˩-dʑo˥, | kɤ˩-nɑ˧mi˧! | mv˧di˧-qo˥ | ɖɯ˩-hĩ˩-dʑo˥, | ə˧v˧˥!}}} \textcolor{Sepia}{\selectlanguage{english}“As the Eagle is greatest of all that fly in the sky, so the Uncle is greatest of all that walk the earth.”} \zh{“天上飞的,是老鹰最大。天下走的,是舅舅最大。”}  
 \zh{量词}: \textcolor{darkblue}{\textbf{\ipa{v̩˧}}}  \mytextsc{clf}: \textcolor{darkblue}{\textbf{\ipa{v̩˧}}} 
\lhead{\firstmark}
\rhead{\botmark}

\subsection{\hspace{-0.5cm} {\Large \textcolor{darkblue}{\textbf{\ipa{ə˧v̩˧-ze˥v̩˩}}}}\hspace{0.5cm}[\kern2pt{\textcolor{darkblue}{\textbf{\ipa{ə˧v̩˥ze˩v̩˩}}}}\kern2pt]} \hypertarget{@\string_Mv\string_=\string_M-ze\string_Tv\string_=\string_B1}{}
\markboth{\textcolor{darkblue}{\textbf{\ipa{ə˧v̩˧-ze˥v̩˩}}}}{}
\textcolor{teal}{\mytextsc{noun}} \hspace{4pt} Tone: H\#-.
\textcolor{Sepia}{\selectlanguage{english}Uncle and nephew.} \zh{叔叔侄子。} 
\lhead{\firstmark}
\rhead{\botmark}

\subsection{\hspace{-0.5cm} {\Large \textcolor{darkblue}{\textbf{\ipa{ə˧ze˧}}}}\hspace{0.5cm}[\kern2pt{\textcolor{darkblue}{\textbf{\ipa{ə˧ze˥}}}}\kern2pt]} \hypertarget{@\string_Mze\string_M1}{}
\markboth{\textcolor{darkblue}{\textbf{\ipa{ə˧ze˧}}}}{}
\textcolor{teal}{\mytextsc{adverb(ial)}} \hspace{4pt} Tone: H\#.
\textcolor{Sepia}{\selectlanguage{english}Slowly.} \zh{慢慢地。}  ¶ \textcolor{darkblue}{\textbf{\ipa{ə˧ze˧ le˧-hõ˩!}}} \textcolor{Sepia}{\selectlanguage{english}Walk slowly! / Take your time on the road! / Have a quiet and pleasant journey! (Polite salutation to someone who is leaving.)} \zh{慢走!}  
 ¶ \textcolor{darkblue}{\textbf{\ipa{ə˧ze˧ le˧-dzi˩!}}} \textcolor{Sepia}{\selectlanguage{english}Just stay seated! (Polite salutation when leaving someone.)} \zh{慢慢坐!}  
\textit{See:} \hyperlink{}{\textcolor{darkblue}{\textbf{\ipa{ə˧-dzɤ˥\$}}}} 
\lhead{\firstmark}
\rhead{\botmark}

\subsection{\hspace{-0.5cm} {\Large \textcolor{darkblue}{\textbf{\ipa{ə˧zo˩-ʁwɤ˩}}}}\hspace{0.5cm}[\kern2pt{\textcolor{darkblue}{\textbf{\ipa{ə˧zo˩ʁwɤ˧}}}}\kern2pt]} \hypertarget{@\string_Mzo\string_B-Rw7\string_B1}{}
\markboth{\textcolor{darkblue}{\textbf{\ipa{ə˧zo˩-ʁwɤ˩}}}}{}
\textcolor{teal}{\mytextsc{noun}} \hspace{4pt} Tone: L\#-.
\textcolor{Sepia}{\selectlanguage{english}A village close to the Hot Springs.} \zh{温泉乡的一个村落。} 
\lhead{\firstmark}
\rhead{\botmark}

\subsection{\hspace{-0.5cm} {\Large \textcolor{darkblue}{\textbf{\ipa{ə˧=zɯ˩}}}}\hspace{0.5cm}[\kern2pt{\textcolor{darkblue}{\textbf{\ipa{ə˧zɯ˩}}}}\kern2pt]} \hypertarget{@\string_M=zM\string_B1}{}
\markboth{\textcolor{darkblue}{\textbf{\ipa{ə˧=zɯ˩}}}}{}
\textcolor{teal}{\mytextsc{pronoun/pronominal}} \hspace{4pt} Tone: L\# / L.
\textcolor{Sepia}{\selectlanguage{english}Dual inclusive first person pronoun: us two, the two of us (the speaker and the addressee).} \zh{咱们两个。} 
\lhead{\firstmark}
\rhead{\botmark}

\subsection{\hspace{-0.5cm} {\Large \textcolor{darkblue}{\textbf{\ipa{ə˧ʐv̩˩}}}}\hspace{0.5cm}[\kern2pt{\textcolor{darkblue}{\textbf{\ipa{ə˧ʐv̩˩}}}}\kern2pt]} \hypertarget{@\string_Mz`v\string_=\string_B1}{}
\markboth{\textcolor{darkblue}{\textbf{\ipa{ə˧ʐv̩˩}}}}{}
\textcolor{teal}{\mytextsc{adjective}} \hspace{4pt} Tone: L\#.
\textcolor{Sepia}{\selectlanguage{english}Old, used.} \zh{陈旧。}  ¶ \textcolor{darkblue}{\textbf{\ipa{ʂe˧ ʐv̩˥}}} \textcolor{Sepia}{\selectlanguage{english}old meat, meat that is not fresh} \zh{陈肉、不新鲜的肉}  

\lhead{\firstmark}
\rhead{\botmark}

\subsection{\hspace{-0.5cm} {\Large \textcolor{darkblue}{\textbf{\ipa{ə˧ʑi˧-ə˧pʰv̩˧˥}}}}\hspace{0.5cm}[\kern2pt{\textcolor{darkblue}{\textbf{\ipa{xxxx non-correspondance entre le nombre de morphèmes et le nombre de tons de morphèmes}}}}\kern2pt]} \hypertarget{@\string_Mz£i\string_M-@\string_Mp\string_hv\string_=\string_M\string_T1}{}
\markboth{\textcolor{darkblue}{\textbf{\ipa{ə˧ʑi˧-ə˧pʰv̩˧˥}}}}{}
\textcolor{teal}{\mytextsc{noun}} \hspace{4pt} Tone: MH\#.
\textcolor{Sepia}{\selectlanguage{english}Elders by two generations: the grandmother and her brothers.} \zh{奶奶与她的兄弟。} 
\lhead{\firstmark}
\rhead{\botmark}

\subsection{\hspace{-0.5cm} {\Large \textcolor{darkblue}{\textbf{\ipa{ə˧ʑi˧-ʐv̩˥mi˩}}}}\hspace{0.5cm}[\kern2pt{\textcolor{darkblue}{\textbf{\ipa{ə˧ʑi˥ʐv̩˩mi˩}}}}\kern2pt]} \hypertarget{@\string_Mz£i\string_M-z`v\string_=\string_Tmi\string_B1}{}
\markboth{\textcolor{darkblue}{\textbf{\ipa{ə˧ʑi˧-ʐv̩˥mi˩}}}}{}
\textcolor{teal}{\mytextsc{noun}} \hspace{4pt} Tone: H\#-.
\textcolor{Sepia}{\selectlanguage{english}Grandmother and granddaughter.} \zh{奶奶与孙女。} 
\lhead{\firstmark}
\rhead{\botmark}

\subsection{\hspace{-0.5cm} {\Large \textcolor{darkblue}{\textbf{\ipa{ə˧ʑi˧˥}}}}\hspace{0.5cm}[\kern2pt{\textcolor{darkblue}{\textbf{\ipa{ə˧ʑi˧˥}}}}\kern2pt]} \hypertarget{@\string_Mz£i\string_M\string_T1}{}
\markboth{\textcolor{darkblue}{\textbf{\ipa{ə˧ʑi˧˥}}}}{}
\textcolor{teal}{\mytextsc{noun}} \hspace{4pt} Tone: MH\#.
\textcolor{Sepia}{\selectlanguage{english}Grandmother (on mother's side); elderly woman.} \zh{祖母,姥姥,老妪。}  ¶ \textcolor{darkblue}{\textbf{\ipa{ə˧ʑi˧ ʝi˧ so˥-zo˩-ho˩-ze˩!}}} \textcolor{Sepia}{\selectlanguage{english}I shall have to learn to be a grandmother! / I shall have to learn to behave as a grandmother! (Humorous remark by the main consultant, after a doctor has advised her to avoid low, soft seats such as sofas and to adopt a taller wooden chair. Paraphrase: “I guess I have entered the category of elderly persons!”)} \zh{我要学习当老太太了!(情景:一位医生建议合作人不要坐在小凳子或者软沙发上了,而要坐更高的木头椅子。她幽默地说:“看来我是老年人了!”)}  
 \zh{量词}: \textcolor{darkblue}{\textbf{\ipa{v̩˧}}}  \mytextsc{clf}: \textcolor{darkblue}{\textbf{\ipa{v̩˧}}} 
\lhead{\firstmark}
\rhead{\botmark}

\subsection{\hspace{-0.5cm} {\Large \textcolor{darkblue}{\textbf{\ipa{ə˩‑}}}}\hspace{0.5cm}[\kern2pt{\textcolor{darkblue}{\textbf{\ipa{ə˩˥}}}}\kern2pt]} \hypertarget{@\string_B‑1}{}
\markboth{\textcolor{darkblue}{\textbf{\ipa{ə˩‑}}}}{}
\textcolor{teal}{\mytextsc{pronoun/pronominal}} \hspace{4pt} Tone: L.
\textcolor{Sepia}{\selectlanguage{english}Total interrogation.} \zh{……吗?。}  ¶ \textcolor{darkblue}{\textbf{\ipa{dʑɯ˧ | ə˩-dʑo˧?}}} \textcolor{Sepia}{\selectlanguage{english}Is there any water?} \zh{有谁吗?}  
 ¶ \textcolor{darkblue}{\textbf{\ipa{ə˩-ŋi˩˥ ?}}} \textcolor{Sepia}{\selectlanguage{english}Is that right? / Is that correct? / ... isn't it?} \zh{对吗? / 对吧?}  

\lhead{\firstmark}
\rhead{\botmark}

\subsection{\hspace{-0.5cm} {\Large \textcolor{darkblue}{\textbf{\ipa{ə˩kʰɯ˩}}}}\hspace{0.5cm}[\kern2pt{\textcolor{darkblue}{\textbf{\ipa{ə˩kʰɯ˩˥}}}}\kern2pt]} \hypertarget{@\string_Bk\string_hM\string_B1}{}
\markboth{\textcolor{darkblue}{\textbf{\ipa{ə˩kʰɯ˩}}}}{}
\textcolor{teal}{\mytextsc{noun}} \hspace{4pt} Tone: L.
\textcolor{Sepia}{\selectlanguage{english}Turnip, wild cabbage, \textit{Brassica rapa}.} \zh{芜菁 、扁萝卜、大头菜、蔓菁。}  ¶ \textcolor{darkblue}{\textbf{\ipa{ə˩kʰɯ˩-bv̩˧ | kʰɯ˩ʈɯ˩˥}}} \textcolor{Sepia}{\selectlanguage{english}the root of wild cabbage} \zh{芜菁的根}  
 \zh{量词}: \textcolor{darkblue}{\textbf{\ipa{ɭɯ˧}}}  \mytextsc{clf}: \textcolor{darkblue}{\textbf{\ipa{ɭɯ˧}}} 
\lhead{\firstmark}
\rhead{\botmark}

\subsection{\hspace{-0.5cm} {\Large \textcolor{darkblue}{\textbf{\ipa{ə˩ljɤ˩hæ̃˩ʂɯ˥-mo˩}}}}\hspace{0.5cm}[\kern2pt{\textcolor{darkblue}{\textbf{\ipa{ə˩ljɤ˩hæ̃˩ʂɯ˥mo˧}}}}\kern2pt]} \hypertarget{@\string_Blj7\string_Bh\{\string_~\string_Bs`M\string_T-mo\string_B1}{}
\markboth{\textcolor{darkblue}{\textbf{\ipa{ə˩ljɤ˩hæ̃˩ʂɯ˥-mo˩}}}}{}
\textcolor{teal}{\mytextsc{noun}} \hspace{4pt} Tone: L+H\#-.
\textcolor{Sepia}{\selectlanguage{english}A sort of mushroom: \textit{Hygrophorus lucorum Kalc hbr.}.} \zh{柠檬黄蜡伞(一种菌子)。} Local Chinese dialect:\zh{黄蜡伞。}
\lhead{\firstmark}
\rhead{\botmark}

\subsection{\hspace{-0.5cm} {\Large \textcolor{darkblue}{\textbf{\ipa{ə˩qo˥}}}}\hspace{0.5cm}[\kern2pt{\textcolor{darkblue}{\textbf{\ipa{ə˩qo˥}}}}\kern2pt]} \hypertarget{@\string_Bqo\string_T1}{}
\markboth{\textcolor{darkblue}{\textbf{\ipa{ə˩qo˥}}}}{}
\textcolor{teal}{\mytextsc{adverb(ial)}} \hspace{4pt} Tone: LH.
\textcolor{Sepia}{\selectlanguage{english}Inward.} \zh{往里。} 
\lhead{\firstmark}
\rhead{\botmark}

\newpage
\section*{\centering- \textcolor{darkblue}{\textbf{\ipa{f}}} -}
\subsection{\hspace{-0.5cm} {\Large \textcolor{darkblue}{\textbf{\ipa{fɑ˧tɑ˧˥}}}}\hspace{0.5cm}[\kern2pt{\textcolor{darkblue}{\textbf{\ipa{fɑ˧tɑ˧˥}}}}\kern2pt]} \hypertarget{fA\string_MtA\string_M\string_T1}{}
\markboth{\textcolor{darkblue}{\textbf{\ipa{fɑ˧tɑ˧˥}}}}{}
\textcolor{teal}{\mytextsc{adjective}} \hspace{4pt} Tone: MH.
\textcolor{Sepia}{\selectlanguage{english}Developed, flourishing.} \zh{发达。}  Borrowing: Chinese  \zh{发达}
 ¶ \textcolor{darkblue}{\textbf{\ipa{fɑ˧tɑ˧-ze˥}}} \textcolor{Sepia}{\selectlanguage{english}\mytextsc{pfv}} \zh{很发达的了}  

\lhead{\firstmark}
\rhead{\botmark}

\subsection{\hspace{-0.5cm} {\Large \textcolor{darkblue}{\textbf{\ipa{fɑ˩\textsubscript{a}}}}}\hspace{0.5cm}[\kern2pt{\textcolor{darkblue}{\textbf{\ipa{fɑ˩˥}}}}\kern2pt]} \hypertarget{fA\string_Ba1}{}
\markboth{\textcolor{darkblue}{\textbf{\ipa{fɑ˩\textsubscript{a}}}}}{}
\textcolor{teal}{\mytextsc{verb}} \hspace{4pt} Tone: L\textsubscript{a}.
\textcolor{Sepia}{\selectlanguage{english}To ferment.} \zh{发酵(汉语借词:发)。}  Borrowing: Chinese  \zh{发(酵)}
 ¶ \textcolor{darkblue}{\textbf{\ipa{tsɑ˧bɤ˧ ɖɯ˧-mɤ˩ | tʰi˧-fɑ˩}}} \textcolor{Sepia}{\selectlanguage{english}to make a little flour ferment, to prepare a little bread dough} \zh{发一点面}  
 ¶ \textcolor{darkblue}{\textbf{\ipa{tsɑ˧bɤ˧ tʰi˧-fɑ˩! | pɤ˩jɤ˧ gv̩˥-bi˩!}}} \textcolor{Sepia}{\selectlanguage{english}Make some flour to ferment! We're going to prepare buns!} \zh{你发一点面吧!要做馒头!}  

\lhead{\firstmark}
\rhead{\botmark}

\subsection{\hspace{-0.5cm} {\Large \textcolor{darkblue}{\textbf{\ipa{fæ˧}}}}\hspace{0.5cm}[\kern2pt{\textcolor{darkblue}{\textbf{\ipa{fæ˥}}}}\kern2pt]} \hypertarget{f\{\string_M1}{}
\markboth{\textcolor{darkblue}{\textbf{\ipa{fæ˧}}}}{}
\textcolor{teal}{\mytextsc{noun}} \hspace{4pt} Tone: M.
\textcolor{Sepia}{\selectlanguage{english}Direction.} \zh{方(方向的方)(汉语借词)。}  Borrowing: Chinese  \zh{方}
 ¶ \textcolor{darkblue}{\textbf{\ipa{dv̩˩tɕo˧ fæ˧}}} \textcolor{Sepia}{\selectlanguage{english}that way} \zh{那个方向}  
\textit{See:} \hyperlink{}{\textcolor{darkblue}{\textbf{\ipa{dɤ˧-tʰv̩˧-gi\#˥}}}} 
\lhead{\firstmark}
\rhead{\botmark}

\subsection{\hspace{-0.5cm} {\Large \textcolor{darkblue}{\textbf{\ipa{fv̩˩˧}}}}\hspace{0.5cm}[\kern2pt{\textcolor{darkblue}{\textbf{\ipa{fv̩˩˥}}}}\kern2pt]} \hypertarget{fv\string_=\string_B\string_M1}{}
\markboth{\textcolor{darkblue}{\textbf{\ipa{fv̩˩˧}}}}{}
\textcolor{teal}{\mytextsc{noun}} \hspace{4pt} Tone: LM.
\textcolor{Sepia}{\selectlanguage{english}Neighbours.} \zh{邻居,村里的人们。} 
\lhead{\firstmark}
\rhead{\botmark}

\subsection{\hspace{-0.5cm} {\Large \textcolor{darkblue}{\textbf{\ipa{fv̩˧}}}}\hspace{0.5cm}[\kern2pt{\textcolor{darkblue}{\textbf{\ipa{fv̩˥}}}}\kern2pt]} \hypertarget{fv\string_=\string_M1}{}
\markboth{\textcolor{darkblue}{\textbf{\ipa{fv̩˧}}}}{}
\textcolor{teal}{\mytextsc{adjective}} \hspace{4pt} Tone: M.
\textcolor{Sepia}{\selectlanguage{english}Glad, pleased, happy, delighted; to like.} \zh{高兴、起劲,喜欢、爱、愿意。}  ¶ \textcolor{darkblue}{\textbf{\ipa{ɖwæ˧˥ | fv̩˧}}} \textcolor{Sepia}{\selectlanguage{english}\mytextsc{intensive}.very: really glad, very happy} \zh{很高兴}  
 ¶ \textcolor{darkblue}{\textbf{\ipa{dʑɤ˩˥ | fv̩˧}}} \textcolor{Sepia}{\selectlanguage{english}really glad, very happy} \zh{很高兴}  
 ¶ \textcolor{darkblue}{\textbf{\ipa{mɤ˧-fv̩˧ ʝi˧}}} \textcolor{Sepia}{\selectlanguage{english}to get angry, to lose one's temper, to air one's anger} \zh{生气}  
 ¶ \textcolor{darkblue}{\textbf{\ipa{ʈʂʰɯ˧ mɤ˧-fv̩˧ ʝi˧!}}} \textcolor{Sepia}{\selectlanguage{english}He/she is angry.} \zh{他在生气。}  
 ¶ \textcolor{darkblue}{\textbf{\ipa{[F5] ɖwæ˧˥ | fv̩˧hĩ˧ ɖɯ˧-v̩˧ ɲi˩}}} \textcolor{Sepia}{\selectlanguage{english}It's a very agreeable person.} \zh{他是很善良的人。}  

\lhead{\firstmark}
\rhead{\botmark}

\subsection{\hspace{-0.5cm} {\Large \textcolor{darkblue}{\textbf{\ipa{fv̩˩bi˩}}}}\hspace{0.5cm}[\kern2pt{\textcolor{darkblue}{\textbf{\ipa{fv̩˩bi˩˥}}}}\kern2pt]} \hypertarget{fv\string_=\string_Bbi\string_B1}{}
\markboth{\textcolor{darkblue}{\textbf{\ipa{fv̩˩bi˩}}}}{}
\textcolor{teal}{\mytextsc{noun}} \hspace{4pt} Tone: L.
\textcolor{Sepia}{\selectlanguage{english}Neighbourhood (in the extended sense: encompasses several small villages).} \zh{邻里、邻村:大家族居住的那片地方,包括几个小村落。} 
\lhead{\firstmark}
\rhead{\botmark}

\subsection{\hspace{-0.5cm} {\Large \textcolor{darkblue}{\textbf{\ipa{fv̩˧kʰo˥}}}}\hspace{0.5cm}[\kern2pt{\textcolor{darkblue}{\textbf{\ipa{fv̩˧kʰo˥}}}}\kern2pt]} \hypertarget{fv\string_=\string_Mk\string_ho\string_T1}{}
\markboth{\textcolor{darkblue}{\textbf{\ipa{fv̩˧kʰo˥}}}}{}
\textcolor{teal}{\mytextsc{noun}} \hspace{4pt} Tone: H\#.
\textcolor{Sepia}{\selectlanguage{english}Fengke: a village located close to the Yangtze river, on the right bank.} \zh{奉科(金沙江边的一个地区)。} 
\lhead{\firstmark}
\rhead{\botmark}

\subsection{\hspace{-0.5cm} {\Large \textcolor{darkblue}{\textbf{\ipa{fv̩˧ʂɯ˩}}}}\hspace{0.5cm}[\kern2pt{\textcolor{darkblue}{\textbf{\ipa{fv̩˧ʂɯ˩}}}}\kern2pt]} \hypertarget{fv\string_=\string_Ms`M\string_B1}{}
\markboth{\textcolor{darkblue}{\textbf{\ipa{fv̩˧ʂɯ˩}}}}{}
\textcolor{teal}{\mytextsc{noun}} \hspace{4pt} Tone: L\#.
\textcolor{Sepia}{\selectlanguage{english}Rhumatism.} \zh{风湿(汉语借词)。}  Borrowing: Chinese  \zh{风湿}
 ¶ \textcolor{darkblue}{\textbf{\ipa{fv̩˧ʂɯ˩ go˩}}} \textcolor{Sepia}{\selectlanguage{english}to suffer from rhumatism, to have rhumatism} \zh{有风湿、得风湿}  

\lhead{\firstmark}
\rhead{\botmark}

\newpage
\section*{\centering- \textcolor{darkblue}{\textbf{\ipa{g}}} -}
\subsection{\hspace{-0.5cm} {\Large \textcolor{darkblue}{\textbf{\ipa{gæ˧ɻæ˩}}}}\hspace{0.5cm}[\kern2pt{\textcolor{darkblue}{\textbf{\ipa{gæ˧ɻæ˩}}}}\kern2pt]} \hypertarget{g\{\string_Mr£`\{\string_B1}{}
\markboth{\textcolor{darkblue}{\textbf{\ipa{gæ˧ɻæ˩}}}}{}
\textcolor{teal}{\mytextsc{noun}} \hspace{4pt} Tone: L\#.
\textcolor{Sepia}{\selectlanguage{english}The name of a village located about 1,500 meters West of \textcolor{darkblue}{\textbf{\ipa{/ə˧lɑ˧-ʁwɤ\#˥/:}}} to the left when leaving the plain of Yongning towards Eya; Chinese: Gaer.} \zh{嘎尔村。}  ¶ \textcolor{darkblue}{\textbf{\ipa{dʑɤ˩bv̩˧kɤ˧-sɑ˥ʁwɤ˩, | hi˩ʁwɤ˩-lo˥, | æ˩mi˧-ʁwɤ\#˥, | lɑ˧lo˧-ʁwɤ˥, | lɑ˧ŋwɤ˧, | bɤ˧tsʰo˧gv̩˥, | ə˧lɑ˧-ʁwɤ\#˥, | gæ˧ɻæ˩, | qʰæ˧tɕʰi˧, | tʰo˧ʈɯ\#˥}}} \textcolor{Sepia}{\selectlanguage{english}the ten villages traditionally considered as part of Yongning} \zh{摩梭传统地理概念中,属于永宁的十个村落}  

\lhead{\firstmark}
\rhead{\botmark}

\subsection{\hspace{-0.5cm} {\Large \textcolor{darkblue}{\textbf{\ipa{gæ˩ɖæ˧}}}}\hspace{0.5cm}[\kern2pt{\textcolor{darkblue}{\textbf{\ipa{gæ˩ɖæ˥}}}}\kern2pt]} \hypertarget{g\{\string_Bd`\{\string_M1}{}
\markboth{\textcolor{darkblue}{\textbf{\ipa{gæ˩ɖæ˧}}}}{}
\textcolor{teal}{\mytextsc{noun}} \hspace{4pt} Tone: LM.
\textcolor{Sepia}{\selectlanguage{english}Top part of body.} \zh{上半身。} 
\lhead{\firstmark}
\rhead{\botmark}

\subsection{\hspace{-0.5cm} {\Large \textcolor{darkblue}{\textbf{\ipa{gæ˩pʰæ˧}}}}\hspace{0.5cm}[\kern2pt{\textcolor{darkblue}{\textbf{\ipa{gæ˩pʰæ˥}}}}\kern2pt]} \hypertarget{g\{\string_Bp\string_h\{\string_M1}{}
\markboth{\textcolor{darkblue}{\textbf{\ipa{gæ˩pʰæ˧}}}}{}
\textcolor{teal}{\mytextsc{noun}} \hspace{4pt} Tone: LM.
\textcolor{Sepia}{\selectlanguage{english}Storeroom, larder: a room where food is kept.} \zh{储藏室、库房:存粮食、火腿的房间。}  \zh{量词}: \textcolor{darkblue}{\textbf{\ipa{tso˩}}}  \mytextsc{clf}: \textcolor{darkblue}{\textbf{\ipa{tso˩}}} 
\lhead{\firstmark}
\rhead{\botmark}

\subsection{\hspace{-0.5cm} {\Large \textcolor{darkblue}{\textbf{\ipa{-gɤ˧}}}}\hspace{0.5cm}[\kern2pt{\textcolor{darkblue}{\textbf{\ipa{gɤ˥}}}}\kern2pt]} \hypertarget{-g7\string_M1}{}
\markboth{\textcolor{darkblue}{\textbf{\ipa{-gɤ˧}}}}{}
\textcolor{teal}{\mytextsc{noun}} \hspace{4pt} Tone: M.
\ding{202} \textcolor{Sepia}{\selectlanguage{english}Place.} \zh{地方。}  ¶ \textcolor{darkblue}{\textbf{\ipa{njɤ˧ | ɖɯ˧-ʝi˧ (-gɤ˧) bi˧-zo˧-ho˩!}}} \textcolor{Sepia}{\selectlanguage{english}I have to go somewhere! / I have to make a trip! / I'm off!} \zh{我要去一个别的地方! / 我要换一个地方了! / 我要走了!}  
 ¶ \textcolor{darkblue}{\textbf{\ipa{ze˩ gɤ˧}}} \textcolor{Sepia}{\selectlanguage{english}which place} \zh{什么地方}  
 ¶ \textcolor{darkblue}{\textbf{\ipa{ʈʂʰɯ˧-gɤ˧}}} \textcolor{Sepia}{\selectlanguage{english}this place} \zh{这个地方}  
 ¶ \textcolor{darkblue}{\textbf{\ipa{tʰv̩˧-gɤ˧}}} \textcolor{Sepia}{\selectlanguage{english}that place} \zh{那个地方}  
\ding{203} \textcolor{Sepia}{\selectlanguage{english}Moment.} \zh{时候。}  ¶ \textcolor{darkblue}{\textbf{\ipa{ʂɯ˧-ɬi˧mi˧-qo˧-gɤ˧ tʰv̩˧}}} \textcolor{Sepia}{\selectlanguage{english}when the seventh month has come, when one is in the seventh month} \zh{七月到了的时候}  
 ¶ \textcolor{darkblue}{\textbf{\ipa{ʂɯ˧-ɬi˧mi˧-qo˧-gɤ˧-dʑo˥}}} \textcolor{Sepia}{\selectlanguage{english}in the seventh month, during the seventh month} \zh{七月的时候}  

\lhead{\firstmark}
\rhead{\botmark}

\subsection{\hspace{-0.5cm} {\Large \textcolor{darkblue}{\textbf{\ipa{gɤ˧\textsubscript{b}}}}}\hspace{0.5cm}[\kern2pt{\textcolor{darkblue}{\textbf{\ipa{gɤ˩˥}}}}\kern2pt]} \hypertarget{g7\string_Mb1}{}
\markboth{\textcolor{darkblue}{\textbf{\ipa{gɤ˧\textsubscript{b}}}}}{}
\textcolor{teal}{\mytextsc{verb}} \hspace{4pt} Tone: M\textsubscript{b}.
\textcolor{Sepia}{\selectlanguage{english}To lack something (someone lacks a certain ability).} \zh{缺乏。}  ¶ \textcolor{darkblue}{\textbf{\ipa{mɤ˧-gɤ˧}}} \textcolor{Sepia}{\selectlanguage{english}\mytextsc{neg}: not to lack} \zh{不缺乏}  

\lhead{\firstmark}
\rhead{\botmark}

\subsection{\hspace{-0.5cm} {\Large \textcolor{darkblue}{\textbf{\ipa{gɤ˧bɤ˧}}}}\hspace{0.5cm}[\kern2pt{\textcolor{darkblue}{\textbf{\ipa{gɤ˧bɤ˩}}}}\kern2pt]} \hypertarget{g7\string_Mb7\string_M1}{}
\markboth{\textcolor{darkblue}{\textbf{\ipa{gɤ˧bɤ˧}}}}{}
\textcolor{teal}{\mytextsc{noun}} \hspace{4pt} Tone: M.
\textcolor{Sepia}{\selectlanguage{english}Shadow.} \zh{影子。}  ¶ \textcolor{darkblue}{\textbf{\ipa{gɤ˧bɤ˧ li˧}}} \textcolor{Sepia}{\selectlanguage{english}to watch television (coinage to avoid the loanword 'television')} \zh{看电视}  
 \zh{量词}: \textcolor{darkblue}{\textbf{\ipa{v̩˧}}}  \mytextsc{clf}: \textcolor{darkblue}{\textbf{\ipa{v̩˧}}} 
\lhead{\firstmark}
\rhead{\botmark}

\subsection{\hspace{-0.5cm} {\Large \textcolor{darkblue}{\textbf{\ipa{-gɤ˧bi\#˥}}}}\hspace{0.5cm}[\kern2pt{\textcolor{darkblue}{\textbf{\ipa{gɤ˩bi˥}}}}\kern2pt]} \hypertarget{-g7\string_Mbi\#\string_T1}{}
\markboth{\textcolor{darkblue}{\textbf{\ipa{-gɤ˧bi\#˥}}}}{}
\textcolor{teal}{\mytextsc{postposition}} \hspace{4pt} Tone: \#H.
\textcolor{Sepia}{\selectlanguage{english}On top.} \zh{上面。}  ¶ \textcolor{darkblue}{\textbf{\ipa{ʑi˧qʰwɤ˧-gɤ˧bi˧}}} \textcolor{Sepia}{\selectlanguage{english}on the (roof of) the house = on the roof} \zh{在房頂上}  

\lhead{\firstmark}
\rhead{\botmark}

\subsection{\hspace{-0.5cm} {\Large \textcolor{darkblue}{\textbf{\ipa{gɤ˧lɑ˧}}}}\hspace{0.5cm}[\kern2pt{\textcolor{darkblue}{\textbf{\ipa{gɤ˩lɑ˥}}}}\kern2pt]} \hypertarget{g7\string_MlA\string_M1}{}
\markboth{\textcolor{darkblue}{\textbf{\ipa{gɤ˧lɑ˧}}}}{}
\textcolor{teal}{\mytextsc{noun}} \hspace{4pt} Tone: M.
\textcolor{Sepia}{\selectlanguage{english}God, Pusa, Buddha, Bodhisattva.} \zh{神,菩萨,佛。}  Borrowing: Tibetan  lha
 \zh{量词}: \textcolor{darkblue}{\textbf{\ipa{v̩˧}}}  \mytextsc{clf}: \textcolor{darkblue}{\textbf{\ipa{v̩˧}}} 
\lhead{\firstmark}
\rhead{\botmark}

\subsection{\hspace{-0.5cm} {\Large \textcolor{darkblue}{\textbf{\ipa{gɤ˧lɑ˧-pɤ\#˥}}}}\hspace{0.5cm}[\kern2pt{\textcolor{darkblue}{\textbf{\ipa{xxxx non-correspondance entre le nombre de morphèmes et le nombre de tons de morphèmes}}}}\kern2pt]} \hypertarget{g7\string_MlA\string_M-p7\#\string_T1}{}
\markboth{\textcolor{darkblue}{\textbf{\ipa{gɤ˧lɑ˧-pɤ\#˥}}}}{}
\textcolor{teal}{\mytextsc{noun}} \hspace{4pt} Tone: \#H.
\textcolor{Sepia}{\selectlanguage{english}Image of Buddha.} \zh{佛像。}  \zh{量词}: \textcolor{darkblue}{\textbf{\ipa{pɤ˥}}}  \mytextsc{clf}: \textcolor{darkblue}{\textbf{\ipa{pɤ˥}}} 
\lhead{\firstmark}
\rhead{\botmark}

\subsection{\hspace{-0.5cm} {\Large \textcolor{darkblue}{\textbf{\ipa{gɤ˧lɑ˧-ʑi˩}}}}\hspace{0.5cm}[\kern2pt{\textcolor{darkblue}{\textbf{\ipa{xxxx non-correspondance entre le nombre de morphèmes et le nombre de tons de morphèmes}}}}\kern2pt]} \hypertarget{g7\string_MlA\string_M-z£i\string_B1}{}
\markboth{\textcolor{darkblue}{\textbf{\ipa{gɤ˧lɑ˧-ʑi˩}}}}{}
\textcolor{teal}{\mytextsc{noun}} \hspace{4pt} Tone: \mytextsc{L}.
\textcolor{Sepia}{\selectlanguage{english}Room where the ancestors are worshipped.} \zh{经堂(拜佛、拜祖先的房间)。}  ¶ \textcolor{darkblue}{\textbf{\ipa{gɤ˧lɑ˧-ʑi˩-di˩}}} \textcolor{Sepia}{\selectlanguage{english}same meaning} \zh{同上}  
 \zh{量词}: \textcolor{darkblue}{\textbf{\ipa{ɭɯ˧}}}  \mytextsc{clf}: \textcolor{darkblue}{\textbf{\ipa{ɭɯ˧}}} 
\lhead{\firstmark}
\rhead{\botmark}

\subsection{\hspace{-0.5cm} {\Large \textcolor{darkblue}{\textbf{\ipa{gɤ˧qo˥}}}}\hspace{0.5cm}[\kern2pt{\textcolor{darkblue}{\textbf{\ipa{gɤ˧qo˧˥}}}}\kern2pt]} \hypertarget{g7\string_Mqo\string_T1}{}
\markboth{\textcolor{darkblue}{\textbf{\ipa{gɤ˧qo˥}}}}{}
\textcolor{teal}{\mytextsc{noun}} \hspace{4pt} Tone: MH.
\textcolor{Sepia}{\selectlanguage{english}Higher part of the main room.} \zh{主屋的高处:人吃饭的地方。}  \zh{量词}: \textcolor{darkblue}{\textbf{\ipa{ɭɯ˧}}}  \mytextsc{clf}: \textcolor{darkblue}{\textbf{\ipa{ɭɯ˧}}} 
\lhead{\firstmark}
\rhead{\botmark}

\subsection{\hspace{-0.5cm} {\Large \textcolor{darkblue}{\textbf{\ipa{gɤ˩‑}}}}\hspace{0.5cm}[\kern2pt{\textcolor{darkblue}{\textbf{\ipa{gɤ˩˥}}}}\kern2pt]} \hypertarget{g7\string_B‑1}{}
\markboth{\textcolor{darkblue}{\textbf{\ipa{gɤ˩‑}}}}{}
\textcolor{teal}{\mytextsc{adverb(ial)}} \hspace{4pt} Tone: L.
\textcolor{Sepia}{\selectlanguage{english}Directional prefix: upward.} \zh{向上、往上。} \textit{See:} \textcolor{darkblue}{\textbf{\ipa{gɤ˩-qo˧, gɤ˩-tʰv̩˧qo˧, gɤ˩-ʈʂʰɯ˧qo˧}}} 
\lhead{\firstmark}
\rhead{\botmark}

\subsection{\hspace{-0.5cm} {\Large \textcolor{darkblue}{\textbf{\ipa{gɤ˩}}}}\hspace{0.5cm}[\kern2pt{\textcolor{darkblue}{\textbf{\ipa{gɤ˥}}}}\kern2pt]} \hypertarget{g7\string_B1}{}
\markboth{\textcolor{darkblue}{\textbf{\ipa{gɤ˩}}}}{}
\textcolor{teal}{\mytextsc{adjective}} \hspace{4pt} Tone: L.
\textcolor{Sepia}{\selectlanguage{english}Quarrelsome. This term is used to describe the personality associated with certain astrological signs: some, such as the Tiger and the Monkey, are considered as quarrelsome, making the people born during the corresponding years less suitable for participating in certain rites (e.g. the Coming of Age rite), and more suitable for certain other rites and occasions.} \zh{爱吵架。}  ¶ \textcolor{darkblue}{\textbf{\ipa{kʰv̩˧ gɤ˧˥}}} \textcolor{Sepia}{\selectlanguage{english}“quarrelsome year”: a year whose astrological sign is a quarrelsome animal. Astrological signs such as the Tiger and the Monkey are considered as quarrelsome; people born during one of these years are said to be tough and quarrelsome.} \zh{爱打架的年份/生肖:十二个生肖中,虎、猴……被认为是爱打架的。}  
 ¶ \textcolor{darkblue}{\textbf{\ipa{kʰv̩˧ gɤ˧-hĩ˥}}} \textcolor{Sepia}{\selectlanguage{english}person whose astrological sign is a quarrelsome animal. Astrological signs such as the Tiger and the Monkey are considered as quarrelsome.} \zh{属一个爱打架的年份/生肖的人。十二个生肖中,虎、猴……被认为是爱打架的。}  
 ¶ \textcolor{darkblue}{\textbf{\ipa{ʑi˩hṽ˥, | lɑ˧ : | kʰv̩˧ gɤ˧˥!}}} \textcolor{Sepia}{\selectlanguage{english}The Monkey and the Ape are quarrelsome birth signs!} \zh{属猴和属虎的人很爱吵架!}  
 ¶ \textcolor{darkblue}{\textbf{\ipa{ʑi˩˥, | lɑ˧, | kʰv̩˧ gɤ˧˥!}}} \textcolor{Sepia}{\selectlanguage{english}Same meaning as above; the investigator substituted the colloquial term for 'ape, monkey'.} \zh{同上}  

\lhead{\firstmark}
\rhead{\botmark}

\subsection{\hspace{-0.5cm} {\Large \textcolor{darkblue}{\textbf{\ipa{gɤ˩\textsubscript{a}}}} \textsubscript{1}}\hspace{0.5cm}[\kern2pt{\textcolor{darkblue}{\textbf{\ipa{gɤ˩˥}}}}\kern2pt]} \hypertarget{g7\string_Ba1}{}
\markboth{\textcolor{darkblue}{\textbf{\ipa{gɤ˩\textsubscript{a}}}} \textsubscript{1}}{}
\textcolor{teal}{\mytextsc{verb}} \hspace{4pt} Tone: L\textsubscript{a}.
\textcolor{Sepia}{\selectlanguage{english}To go out (fire).} \zh{灭,熄。}  ¶ \textcolor{darkblue}{\textbf{\ipa{mv̩˧ | le˧-gɤ˩(-ze˩)}}} \textcolor{Sepia}{\selectlanguage{english}The fire has gone out. / The fire went out.} \zh{火灭了。}  

\lhead{\firstmark}
\rhead{\botmark}

\subsection{\hspace{-0.5cm} {\Large \textcolor{darkblue}{\textbf{\ipa{gɤ˩\textsubscript{a}}}} \textsubscript{2}}\hspace{0.5cm}[\kern2pt{\textcolor{darkblue}{\textbf{\ipa{gɤ˩˥}}}}\kern2pt]} \hypertarget{g7\string_Ba2}{}
\markboth{\textcolor{darkblue}{\textbf{\ipa{gɤ˩\textsubscript{a}}}} \textsubscript{2}}{}
\textcolor{teal}{\mytextsc{verb}} \hspace{4pt} Tone: L\textsubscript{a}.
\textcolor{Sepia}{\selectlanguage{english}To be satisfied/happy; to feel that things are fair.} \zh{满意,幸福,甘心,服气。}  ¶ \textcolor{darkblue}{\textbf{\ipa{hɤ˩-zo˥, | le˧-gɤ˩-ze˩!}}} \textcolor{Sepia}{\selectlanguage{english}(S)he has made a good job of it; (s)he is satisfied/happy!} \zh{很成功,真高兴! / 他成功了,很满意!}  
 ¶ \textcolor{darkblue}{\textbf{\ipa{ʈʂʰɯ˧ | ɖwæ˧˥ | le˧-gɤ˩-ze˩!}}} \textcolor{Sepia}{\selectlanguage{english}(S)he is very satisfied/happy!} \zh{他很满意!}  
 ¶ \textcolor{darkblue}{\textbf{\ipa{no˩-se˥, | ɖwæ˧˥ | le˧-gɤ˩-ze˩: | zo˧mv̩˥ hɤ˩-zo˩!}}} \textcolor{Sepia}{\selectlanguage{english}You have grounds for satisfaction: your children are really bright!} \zh{你呢,(应该)很满意:(你的)孩子很成功!}  
 ¶ \textcolor{darkblue}{\textbf{\ipa{mɤ˧-gɤ˩}}} \textcolor{Sepia}{\selectlanguage{english}to be dissatisfied, not resigned, recalcitrant} \zh{不满意、不甘心、不服气}  

\lhead{\firstmark}
\rhead{\botmark}

\subsection{\hspace{-0.5cm} {\Large \textcolor{darkblue}{\textbf{\ipa{gɤ˩\textsubscript{a}}}} \textsubscript{3}}\hspace{0.5cm}[\kern2pt{\textcolor{darkblue}{\textbf{\ipa{gɤ˩˥}}}}\kern2pt]} \hypertarget{g7\string_Ba3}{}
\markboth{\textcolor{darkblue}{\textbf{\ipa{gɤ˩\textsubscript{a}}}} \textsubscript{3}}{}
\textcolor{teal}{\mytextsc{adjective}} \hspace{4pt} Tone: L\textsubscript{a}.
\textcolor{Sepia}{\selectlanguage{english}Startled, amazed, shocked, awestruck; terrified.} \zh{震惊。}  ¶ \textcolor{darkblue}{\textbf{\ipa{le˧-gɤ˩-ze˩}}} \textcolor{Sepia}{\selectlanguage{english}\mytextsc{accomp} \string_ \mytextsc{pfv}} \zh{震惊了}  
 ¶ \textcolor{darkblue}{\textbf{\ipa{no˧ | hĩ˧ gɤ˧-kʰɯ˥!}}} \textcolor{Sepia}{\selectlanguage{english}You frighten people! / People are afraid of you!} \zh{你让人害怕!}  

\lhead{\firstmark}
\rhead{\botmark}

\subsection{\hspace{-0.5cm} {\Large \textcolor{darkblue}{\textbf{\ipa{gɤ˩bv̩˧}}}}\hspace{0.5cm}[\kern2pt{\textcolor{darkblue}{\textbf{\ipa{gɤ˧bv̩˧}}}}\kern2pt]} \hypertarget{g7\string_Bbv\string_=\string_M1}{}
\markboth{\textcolor{darkblue}{\textbf{\ipa{gɤ˩bv̩˧}}}}{}
\textcolor{teal}{\mytextsc{verb}} \hspace{4pt} Tone: LM.
\textcolor{Sepia}{\selectlanguage{english}To overflow.} \zh{溢出来。}  ¶ \textcolor{darkblue}{\textbf{\ipa{gɤ˩bv̩˧-ze˩}}} \textcolor{Sepia}{\selectlanguage{english}\mytextsc{pfv}} \zh{溢出来了}  

\lhead{\firstmark}
\rhead{\botmark}

\subsection{\hspace{-0.5cm} {\Large \textcolor{darkblue}{\textbf{\ipa{gɤ˩dzɤ˧}}}}\hspace{0.5cm}[\kern2pt{\textcolor{darkblue}{\textbf{\ipa{gɤ˩dzɤ˥}}}}\kern2pt]} \hypertarget{g7\string_Bdz7\string_M1}{}
\markboth{\textcolor{darkblue}{\textbf{\ipa{gɤ˩dzɤ˧}}}}{}
\textcolor{teal}{\mytextsc{adverb(ial)}} \hspace{4pt} Tone: LM.
\textcolor{Sepia}{\selectlanguage{english}At the top part: inside a room, at a table..., this is the place of honour.} \zh{在上部分,上座。}  ¶ \textcolor{darkblue}{\textbf{\ipa{gɤ˩dzɤ˧ dzi˧˥}}} \textcolor{Sepia}{\selectlanguage{english}to sit at a place of honour, to sit at the superior part (of a table, a room...)} \zh{坐上座}  
 ¶ \textcolor{darkblue}{\textbf{\ipa{no˧ | gɤ˩dzɤ˧ dzi˧˥!}}} \textcolor{Sepia}{\selectlanguage{english}Please be seated at the place of honour!} \zh{请您坐在上座!}  
\textit{See:} \textcolor{darkblue}{\textbf{\ipa{gɤ˩-}}} 
\lhead{\firstmark}
\rhead{\botmark}

\subsection{\hspace{-0.5cm} {\Large \textcolor{darkblue}{\textbf{\ipa{gɤ˩-qo˧}}}}\hspace{0.5cm}[\kern2pt{\textcolor{darkblue}{\textbf{\ipa{xxxx non-correspondance entre le nombre de morphèmes et le nombre de tons de morphèmes}}}}\kern2pt]} \hypertarget{g7\string_B-qo\string_M1}{}
\markboth{\textcolor{darkblue}{\textbf{\ipa{gɤ˩-qo˧}}}}{}
\textcolor{teal}{\mytextsc{adverb(ial)}} \hspace{4pt} Tone: M.
\textcolor{Sepia}{\selectlanguage{english}Way up there.} \zh{那上面(指高处)。} \textit{See:} \textcolor{darkblue}{\textbf{\ipa{gɤ˩-, gɤ˩-ʈʂʰɯ˧qo˧, gɤ˩-tʰv̩˧qo˧,}}} 
\lhead{\firstmark}
\rhead{\botmark}

\subsection{\hspace{-0.5cm} {\Large \textcolor{darkblue}{\textbf{\ipa{gɤ˩qwɤ˧}}}}\hspace{0.5cm}[\kern2pt{\textcolor{darkblue}{\textbf{\ipa{gɤ˩qwɤ˥}}}}\kern2pt]} \hypertarget{g7\string_Bqw7\string_M1}{}
\markboth{\textcolor{darkblue}{\textbf{\ipa{gɤ˩qwɤ˧}}}}{}
\textcolor{teal}{\mytextsc{noun}} \hspace{4pt} Tone: LM.
\textcolor{Sepia}{\selectlanguage{english}Altar above the hearth (where gifts made to the family are displayed).} \zh{火炉旁边的祭坛(上面摆礼物等)。}  \zh{量词}: \textcolor{darkblue}{\textbf{\ipa{ɭɯ˧}}}  \mytextsc{clf}: \textcolor{darkblue}{\textbf{\ipa{ɭɯ˧}}} 
\lhead{\firstmark}
\rhead{\botmark}

\subsection{\hspace{-0.5cm} {\Large \textcolor{darkblue}{\textbf{\ipa{gɤ˩ʁwɤ\#˥}}}}\hspace{0.5cm}[\kern2pt{\textcolor{darkblue}{\textbf{\ipa{gɤ˩ʁwɤ˥}}}}\kern2pt]} \hypertarget{g7\string_BRw7\#\string_T1}{}
\markboth{\textcolor{darkblue}{\textbf{\ipa{gɤ˩ʁwɤ\#˥}}}}{}
\textcolor{teal}{\mytextsc{noun}} \hspace{4pt} Tone: LM+\#H.
\textcolor{Sepia}{\selectlanguage{english}The village of Gewa.} \zh{格瓦村:永宁的一个村落。直译:上村。音译:格瓦。} 
\lhead{\firstmark}
\rhead{\botmark}

\subsection{\hspace{-0.5cm} {\Large \textcolor{darkblue}{\textbf{\ipa{gɤ˩ʁwɤ˧}}}}\hspace{0.5cm}[\kern2pt{\textcolor{darkblue}{\textbf{\ipa{gɤ˩ʁwɤ˥}}}}\kern2pt]} \hypertarget{g7\string_BRw7\string_M1}{}
\markboth{\textcolor{darkblue}{\textbf{\ipa{gɤ˩ʁwɤ˧}}}}{}
\textcolor{teal}{\mytextsc{noun}} \hspace{4pt} Tone: LM.
\textcolor{Sepia}{\selectlanguage{english}Upper reaches of a river; upstream.} \zh{上游。} 
\lhead{\firstmark}
\rhead{\botmark}

\subsection{\hspace{-0.5cm} {\Large \textcolor{darkblue}{\textbf{\ipa{gɤ˩-tʰv̩˧-gi\#˥}}}}\hspace{0.5cm}[\kern2pt{\textcolor{darkblue}{\textbf{\ipa{xxxx non-correspondance entre le nombre de morphèmes et le nombre de tons de morphèmes}}}}\kern2pt]} \hypertarget{g7\string_B-t\string_hv\string_=\string_M-gi\#\string_T1}{}
\markboth{\textcolor{darkblue}{\textbf{\ipa{gɤ˩-tʰv̩˧-gi\#˥}}}}{}
\textcolor{teal}{\mytextsc{adverb(ial)}} \hspace{4pt} Tone: L-\#H.
\textcolor{Sepia}{\selectlanguage{english}Way up there.} \zh{那里(指高处)。} \textit{See:} \textcolor{darkblue}{\textbf{\ipa{gɤ˩‑, gɤ˩-qo˧, gɤ˩-tʰv̩˧qo˧, gɤ˩-ʈʂʰɯ˧qo˧}}} 
\lhead{\firstmark}
\rhead{\botmark}

\subsection{\hspace{-0.5cm} {\Large \textcolor{darkblue}{\textbf{\ipa{gɤ˩-tʰv̩˧qo˧}}}}\hspace{0.5cm}[\kern2pt{\textcolor{darkblue}{\textbf{\ipa{gɤ˩tʰv̩˧qo˧}}}}\kern2pt]} \hypertarget{g7\string_B-t\string_hv\string_=\string_Mqo\string_M1}{}
\markboth{\textcolor{darkblue}{\textbf{\ipa{gɤ˩-tʰv̩˧qo˧}}}}{}
\textcolor{teal}{\mytextsc{adverb(ial)}} \hspace{4pt} Tone: L-\#H.
\textcolor{Sepia}{\selectlanguage{english}Way up there.} \zh{那里(指高处)。} \textit{See:} \textcolor{darkblue}{\textbf{\ipa{gɤ˩‑, gɤ˩-qo˧, gɤ˩-ʈʂʰɯ˧qo˧}}} 
\lhead{\firstmark}
\rhead{\botmark}

\subsection{\hspace{-0.5cm} {\Large \textcolor{darkblue}{\textbf{\ipa{gɤ˩ʈʂæ˧˥}}}}\hspace{0.5cm}[\kern2pt{\textcolor{darkblue}{\textbf{\ipa{gɤ˩ʈʂæ˧˥}}}}\kern2pt]} \hypertarget{g7\string_Bt`s`\{\string_M\string_T1}{}
\markboth{\textcolor{darkblue}{\textbf{\ipa{gɤ˩ʈʂæ˧˥}}}}{}
\textcolor{teal}{\mytextsc{noun}} \hspace{4pt} Tone: LM+MH\#.
\textcolor{Sepia}{\selectlanguage{english}Top part (of the body=above the waist).} \zh{上半(身)。} 
\lhead{\firstmark}
\rhead{\botmark}

\subsection{\hspace{-0.5cm} {\Large \textcolor{darkblue}{\textbf{\ipa{gɤ˩ʈʂʰæ˧-hĩ˧˥}}}}\hspace{0.5cm}[\kern2pt{\textcolor{darkblue}{\textbf{\ipa{xxxx non-correspondance entre le nombre de morphèmes et le nombre de tons de morphèmes}}}}\kern2pt]} \hypertarget{g7\string_Bt`s`\string_h\{\string_M-hi\string_~\string_M\string_T1}{}
\markboth{\textcolor{darkblue}{\textbf{\ipa{gɤ˩ʈʂʰæ˧-hĩ˧˥}}}}{}
\textcolor{teal}{\mytextsc{noun}} \hspace{4pt} Tone: LM+MH\#.
\textcolor{Sepia}{\selectlanguage{english}Ancestors, past generations.} \zh{祖先。} 
\lhead{\firstmark}
\rhead{\botmark}

\subsection{\hspace{-0.5cm} {\Large \textcolor{darkblue}{\textbf{\ipa{gɤ˩-ʈʂʰɯ˧-gi\#˥}}}}\hspace{0.5cm}[\kern2pt{\textcolor{darkblue}{\textbf{\ipa{xxxx non-correspondance entre le nombre de morphèmes et le nombre de tons de morphèmes}}}}\kern2pt]} \hypertarget{g7\string_B-t`s`\string_hM\string_M-gi\#\string_T1}{}
\markboth{\textcolor{darkblue}{\textbf{\ipa{gɤ˩-ʈʂʰɯ˧-gi\#˥}}}}{}
\textcolor{teal}{\mytextsc{adverb(ial)}} \hspace{4pt} Tone: L-\#H.
\textcolor{Sepia}{\selectlanguage{english}Way up there.} \zh{那里(指高处)。} \textit{See:} \textcolor{darkblue}{\textbf{\ipa{gɤ˩-, gɤ˩-qo˧, gɤ˩-tʰv̩˧qo˧, gɤ˩-ʈʂʰɯ˧qo˧}}} 
\lhead{\firstmark}
\rhead{\botmark}

\subsection{\hspace{-0.5cm} {\Large \textcolor{darkblue}{\textbf{\ipa{gɤ˩-ʈʂʰɯ˧qo˧}}}}\hspace{0.5cm}[\kern2pt{\textcolor{darkblue}{\textbf{\ipa{gɤ˩ʈʂʰɯ˧qo˧}}}}\kern2pt]} \hypertarget{g7\string_B-t`s`\string_hM\string_Mqo\string_M1}{}
\markboth{\textcolor{darkblue}{\textbf{\ipa{gɤ˩-ʈʂʰɯ˧qo˧}}}}{}
\textcolor{teal}{\mytextsc{adverb(ial)}} \hspace{4pt} Tone: L-\#H.
\textcolor{Sepia}{\selectlanguage{english}Way up there.} \zh{那里(指高处)。} \textit{See:} \textcolor{darkblue}{\textbf{\ipa{gɤ˩-, gɤ˩-qo˧, gɤ˩-tʰv̩˧qo˧}}} 
\lhead{\firstmark}
\rhead{\botmark}

\subsection{\hspace{-0.5cm} {\Large \textcolor{darkblue}{\textbf{\ipa{gɤ˧˥}}}}\hspace{0.5cm}[\kern2pt{\textcolor{darkblue}{\textbf{\ipa{gɤ˥}}}}\kern2pt]} \hypertarget{g7\string_M\string_T1}{}
\markboth{\textcolor{darkblue}{\textbf{\ipa{gɤ˧˥}}}}{}
\textcolor{teal}{\mytextsc{verb}} \hspace{4pt} Tone: MH.
\textcolor{Sepia}{\selectlanguage{english}To carry on the shoulder; to carry on a shoulder pole.} \zh{扛,担。}  ¶ \textcolor{darkblue}{\textbf{\ipa{tʰi˧-gɤ˧˥}}} \textcolor{Sepia}{\selectlanguage{english}\mytextsc{dur}} \zh{\mytextsc{dur}}  
 ¶ \textcolor{darkblue}{\textbf{\ipa{tʰi˧-gɤ˧-ze˥}}} \textcolor{Sepia}{\selectlanguage{english}\mytextsc{dur} \string_ \mytextsc{pfv}} \zh{\mytextsc{dur} \string_ \mytextsc{pfv}}  
 ¶ \textcolor{darkblue}{\textbf{\ipa{le˧-gɤ˧-ze˥}}} \textcolor{Sepia}{\selectlanguage{english}\mytextsc{accomp} \string_ \mytextsc{pfv}} \zh{扛了}  
 ¶ \textcolor{darkblue}{\textbf{\ipa{tso˧\textasciitilde{}tso˧ gɤ˩}}} \textcolor{Sepia}{\selectlanguage{english}to carry something on the shoulder} \zh{扛东西}  
 ¶ \textcolor{darkblue}{\textbf{\ipa{njɤ˧(-ɳɯ˧) | gɤ˧-bi˥!}}} \textcolor{Sepia}{\selectlanguage{english}Let me carry it!} \zh{我来扛吧!}  

\lhead{\firstmark}
\rhead{\botmark}

\subsection{\hspace{-0.5cm} {\Large \textcolor{darkblue}{\textbf{\ipa{gi˥}}} \textsubscript{1}}\hspace{0.5cm}[\kern2pt{\textcolor{darkblue}{\textbf{\ipa{gi˥}}}}\kern2pt]} \hypertarget{gi\string_T1}{}
\markboth{\textcolor{darkblue}{\textbf{\ipa{gi˥}}} \textsubscript{1}}{}
\textcolor{teal}{\mytextsc{verb}} \hspace{4pt} Tone: H.
\textcolor{Sepia}{\selectlanguage{english}To fall (snow, rain), to snow/to rain.} \zh{下(雨,雪)。}  ¶ \textcolor{darkblue}{\textbf{\ipa{bi˧ gi˧. / bi˧ gi˧-ze˩.}}} \textcolor{Sepia}{\selectlanguage{english}It snows. / It has snowed.} \zh{下雪。 / 下雪了。}  
 ¶ \textcolor{darkblue}{\textbf{\ipa{hi˩ gi˩˥. / hi˩ gi˩-ze˥.}}} \textcolor{Sepia}{\selectlanguage{english}It rains. / It has rained.} \zh{下雨。 / 下雨了。}  
 ¶ \textcolor{darkblue}{\textbf{\ipa{tsʰi˧-ɲi˧-dʑo˩, | hi˩ gi˩-ze˥, | le˧-gɤ˩-ze˩!}}} \textcolor{Sepia}{\selectlanguage{english}Today, it is raining; that's good! / it's a good thing! (A comment made at the beginning of the rainy season, after a long drought.)} \zh{今天,下雨了,真好!(情景:大旱灾过后,雨季终于来了,这对庄稼很好。)}  

\lhead{\firstmark}
\rhead{\botmark}

\subsection{\hspace{-0.5cm} {\Large \textcolor{darkblue}{\textbf{\ipa{gi˥}}} \textsubscript{2}}\hspace{0.5cm}[\kern2pt{\textcolor{darkblue}{\textbf{\ipa{gi˥}}}}\kern2pt]} \hypertarget{gi\string_T2}{}
\markboth{\textcolor{darkblue}{\textbf{\ipa{gi˥}}} \textsubscript{2}}{}
\textcolor{teal}{\mytextsc{verb}} \hspace{4pt} Tone: H.
\textcolor{Sepia}{\selectlanguage{english}To owe money.} \zh{欠(钱)。}  ¶ \textcolor{darkblue}{\textbf{\ipa{ɖʐe˧ | tʰi˧-gi˥}}} \textcolor{Sepia}{\selectlanguage{english}to owe money} \zh{欠钱}  

\lhead{\firstmark}
\rhead{\botmark}

\subsection{\hspace{-0.5cm} {\Large \textcolor{darkblue}{\textbf{\ipa{gi˥\textsubscript{a}}}}}\hspace{0.5cm}[\kern2pt{\textcolor{darkblue}{\textbf{\ipa{gi˥}}}}\kern2pt]} \hypertarget{gi\string_Ta1}{}
\markboth{\textcolor{darkblue}{\textbf{\ipa{gi˥\textsubscript{a}}}}}{}
\textcolor{teal}{\mytextsc{classifier}} \hspace{4pt} Tone: H\textsubscript{a}.
\ding{202} \textcolor{Sepia}{\selectlanguage{english}A half.} \zh{量词:一半。}  ¶ \textcolor{darkblue}{\textbf{\ipa{ɖɯ˧-gi˥}}} \textcolor{Sepia}{\selectlanguage{english}one half} \zh{一半}  
 ¶ \textcolor{darkblue}{\textbf{\ipa{tsʰe˩ʐv̩˩-gi˥}}} \textcolor{Sepia}{\selectlanguage{english}fourteen halves (combination elicited to determine the tonal category of the classifier)} \zh{十四个半(注:这是为了确定调类而问的短语)}  
 ¶ \textcolor{darkblue}{\textbf{\ipa{tv̩˧tsʰɯ˧ | ɖɯ˧-gi˥}}} \textcolor{Sepia}{\selectlanguage{english}half the time, half the duration} \zh{一半的时间}  
\ding{203} \textcolor{Sepia}{\selectlanguage{english}A side; a direction.} \zh{量词:一面(房屋的一面)。}  ¶ \textcolor{darkblue}{\textbf{\ipa{ɖɯ˧-gi˧ hõ˧}}} \textcolor{Sepia}{\selectlanguage{english}to go in a certain direction, to go one's way} \zh{往一个方向走、走自己的方向}  
 ¶ \textcolor{darkblue}{\textbf{\ipa{ɖɯ˧-v̩˧ | ɖɯ˧-gi˧ hɯ˧}}} \textcolor{Sepia}{\selectlanguage{english}to go each one's separate way; to go each in a different direction} \zh{分开:每个人去自己的方向}  

\lhead{\firstmark}
\rhead{\botmark}

\subsection{\hspace{-0.5cm} {\Large \textcolor{darkblue}{\textbf{\ipa{gi˧dʑɯ˧}}}}\hspace{0.5cm}[\kern2pt{\textcolor{darkblue}{\textbf{\ipa{gi˥}}}}\kern2pt]} \hypertarget{gi\string_Mdz£M\string_M1}{}
\markboth{\textcolor{darkblue}{\textbf{\ipa{gi˧dʑɯ˧}}}}{}
\textcolor{teal}{\mytextsc{noun}} \hspace{4pt} Tone: M.
\textcolor{Sepia}{\selectlanguage{english}The Yangtze river (Yellow Sands river).} \zh{金沙江。}  ¶ \textcolor{darkblue}{\textbf{\ipa{gi˧dʑɯ˧-kʰi\#˥}}} \textcolor{Sepia}{\selectlanguage{english}the banks of the Yangtze river: Fengke, Labai...} \zh{金沙江边:奉科,拉伯……}  
 ¶ \textcolor{darkblue}{\textbf{\ipa{gi˧dʑɯ˧-kʰi˧-hĩ\#˥}}} \textcolor{Sepia}{\selectlanguage{english}inhabitants of the banks of the Yangtze: people of Labai, Fengke...} \zh{金沙江边的人:奉科人,拉伯人……}  

\lhead{\firstmark}
\rhead{\botmark}

\subsection{\hspace{-0.5cm} {\Large \textcolor{darkblue}{\textbf{\ipa{gi˧-nɑ˧mi\#˥}}}}\hspace{0.5cm}[\kern2pt{\textcolor{darkblue}{\textbf{\ipa{xxxx non-correspondance entre le nombre de morphèmes et le nombre de tons de morphèmes}}}}\kern2pt]} \hypertarget{gi\string_M-nA\string_Mmi\#\string_T1}{}
\markboth{\textcolor{darkblue}{\textbf{\ipa{gi˧-nɑ˧mi\#˥}}}}{}
\textcolor{teal}{\mytextsc{noun}} \hspace{4pt} Tone: \#H.
\textcolor{Sepia}{\selectlanguage{english}Bear; she-bear. There is no way to refer unambiguously to a female bear, as the same term is used for bears irrespective of sex.} \zh{熊,母熊。}  ¶ \textcolor{darkblue}{\textbf{\ipa{gi˧-nɑ˧mi˧ tʰv̩˧-pʰo˩}}} \textcolor{Sepia}{\selectlanguage{english}\mytextsc{n}+\mytextsc{dem}+\mytextsc{clf}} \zh{这只熊}  
 \zh{量词}: \textcolor{darkblue}{\textbf{\ipa{pʰo˧˥}}}  \mytextsc{clf}: \textcolor{darkblue}{\textbf{\ipa{pʰo˧˥}}} 
\lhead{\firstmark}
\rhead{\botmark}

\subsection{\hspace{-0.5cm} {\Large \textcolor{darkblue}{\textbf{\ipa{gi˧-nɑ˧mi˧-pʰv̩\#˥}}}}\hspace{0.5cm}[\kern2pt{\textcolor{darkblue}{\textbf{\ipa{xxxx non-correspondance entre le nombre de morphèmes et le nombre de tons de morphèmes}}}}\kern2pt]} \hypertarget{gi\string_M-nA\string_Mmi\string_M-p\string_hv\string_=\#\string_T1}{}
\markboth{\textcolor{darkblue}{\textbf{\ipa{gi˧-nɑ˧mi˧-pʰv̩\#˥}}}}{}
\textcolor{teal}{\mytextsc{noun}} \hspace{4pt} Tone: \#H.
\textcolor{Sepia}{\selectlanguage{english}He-bear, male bear.} \zh{公熊。}  ¶ \textcolor{darkblue}{\textbf{\ipa{gi˧-nɑ˧mi˧-pʰv̩˧ tʰv̩˧-pʰo˩}}} \textcolor{Sepia}{\selectlanguage{english}\mytextsc{n}+\mytextsc{dem}+\mytextsc{clf}} \zh{这只公熊}  
 \zh{量词}: \textcolor{darkblue}{\textbf{\ipa{pʰo˧˥}}}  \mytextsc{clf}: \textcolor{darkblue}{\textbf{\ipa{pʰo˧˥}}} 
\lhead{\firstmark}
\rhead{\botmark}

\subsection{\hspace{-0.5cm} {\Large \textcolor{darkblue}{\textbf{\ipa{gi˧-nɑ˧mi˧-zo\#˥}}}}\hspace{0.5cm}[\kern2pt{\textcolor{darkblue}{\textbf{\ipa{xxxx non-correspondance entre le nombre de morphèmes et le nombre de tons de morphèmes}}}}\kern2pt]} \hypertarget{gi\string_M-nA\string_Mmi\string_M-zo\#\string_T1}{}
\markboth{\textcolor{darkblue}{\textbf{\ipa{gi˧-nɑ˧mi˧-zo\#˥}}}}{}
\textcolor{teal}{\mytextsc{noun}} \hspace{4pt} Tone: \#H.
\textcolor{Sepia}{\selectlanguage{english}Little bear, bear cub.} \zh{小熊。}  ¶ \textcolor{darkblue}{\textbf{\ipa{gi˧-nɑ˧mi˧-zo˧ tʰv̩˧-ɭɯ\#˥}}} \textcolor{Sepia}{\selectlanguage{english}\mytextsc{n}+\mytextsc{dem}+\mytextsc{clf}} \zh{这只小熊}  
 \zh{量词}: \textcolor{darkblue}{\textbf{\ipa{ɭɯ˧}}}  \mytextsc{clf}: \textcolor{darkblue}{\textbf{\ipa{ɭɯ˧}}} 
\lhead{\firstmark}
\rhead{\botmark}

\subsection{\hspace{-0.5cm} {\Large \textcolor{darkblue}{\textbf{\ipa{gi˧zɯ\#˥}}}}\hspace{0.5cm}[\kern2pt{\textcolor{darkblue}{\textbf{\ipa{gi˧zɯ˧}}}}\kern2pt]} \hypertarget{gi\string_MzM\#\string_T1}{}
\markboth{\textcolor{darkblue}{\textbf{\ipa{gi˧zɯ\#˥}}}}{}
\textcolor{teal}{\mytextsc{noun}} \hspace{4pt} Tone: \#H.
\textcolor{Sepia}{\selectlanguage{english}Little brother, younger brother; the term is also used to refer to younger cousins.} \zh{弟弟(也可指更年轻的表弟)。}  ¶ \textcolor{darkblue}{\textbf{\ipa{gi˧zɯ˧=ɻæ˥}}} \textcolor{Sepia}{\selectlanguage{english}\mytextsc{associative}: younger brothers, cousins...} \zh{联想复数:弟弟们,表弟们}  
 \zh{量词}: \textcolor{darkblue}{\textbf{\ipa{v̩˧}}}  \mytextsc{clf}: \textcolor{darkblue}{\textbf{\ipa{v̩˧}}} 
\lhead{\firstmark}
\rhead{\botmark}

\subsection{\hspace{-0.5cm} {\Large \textcolor{darkblue}{\textbf{\ipa{gi˧zɯ˧-go˧mi\#˥}}}}\hspace{0.5cm}[\kern2pt{\textcolor{darkblue}{\textbf{\ipa{xxxx non-correspondance entre le nombre de morphèmes et le nombre de tons de morphèmes}}}}\kern2pt]} \hypertarget{gi\string_MzM\string_M-go\string_Mmi\#\string_T1}{}
\markboth{\textcolor{darkblue}{\textbf{\ipa{gi˧zɯ˧-go˧mi\#˥}}}}{}
\textcolor{teal}{\mytextsc{noun}} \hspace{4pt} Tone: \#H.
\textcolor{Sepia}{\selectlanguage{english}Younger siblings (brothers and sisters).} \zh{弟弟妹妹。} 
\lhead{\firstmark}
\rhead{\botmark}

\subsection{\hspace{-0.5cm} {\Large \textcolor{darkblue}{\textbf{\ipa{gi˩}}}}\hspace{0.5cm}[\kern2pt{\textcolor{darkblue}{\textbf{\ipa{gi˥}}}}\kern2pt]} \hypertarget{gi\string_B1}{}
\markboth{\textcolor{darkblue}{\textbf{\ipa{gi˩}}}}{}
\textcolor{teal}{\mytextsc{noun}} \hspace{4pt} Tone: L.
\textcolor{Sepia}{\selectlanguage{english}Bear.} \zh{大熊。} 
\lhead{\firstmark}
\rhead{\botmark}

\subsection{\hspace{-0.5cm} {\Large \textcolor{darkblue}{\textbf{\ipa{gi˩}}}}\hspace{0.5cm}[\kern2pt{\textcolor{darkblue}{\textbf{\ipa{gi˥}}}}\kern2pt]} \hypertarget{gi\string_B1}{}
\markboth{\textcolor{darkblue}{\textbf{\ipa{gi˩}}}}{}
\textcolor{teal}{\mytextsc{noun}} \hspace{4pt} Tone: L.
\textcolor{Sepia}{\selectlanguage{english}Granary (room within the house where grain is stored).} \zh{粮仓。}  ¶ \textcolor{darkblue}{\textbf{\ipa{gi˧mi˧}}} \textcolor{Sepia}{\selectlanguage{english}large granary} \zh{大粮仓}  
 ¶ \textcolor{darkblue}{\textbf{\ipa{gi˩zo˩˥}}} \textcolor{Sepia}{\selectlanguage{english}small granary} \zh{小粮仓}  
 ¶ \textcolor{darkblue}{\textbf{\ipa{njɤ˧ | gi˩ gv̩˩-zo˩-ho˥}}} \textcolor{Sepia}{\selectlanguage{english}I shall have to repair the granary!} \zh{我应该修粮仓!}  
 \zh{量词}: \textcolor{darkblue}{\textbf{\ipa{ɭɯ˧}}}  \mytextsc{clf}: \textcolor{darkblue}{\textbf{\ipa{ɭɯ˧}}} 
\lhead{\firstmark}
\rhead{\botmark}

\subsection{\hspace{-0.5cm} {\Large \textcolor{darkblue}{\textbf{\ipa{gi˩\textsubscript{a}}}}}\hspace{0.5cm}[\kern2pt{\textcolor{darkblue}{\textbf{\ipa{gi˩˥}}}}\kern2pt]} \hypertarget{gi\string_Ba1}{}
\markboth{\textcolor{darkblue}{\textbf{\ipa{gi˩\textsubscript{a}}}}}{}
\textcolor{teal}{\mytextsc{adjective}} \hspace{4pt} Tone: L\textsubscript{a}.
\textit{From:} \textbf{/gɯ˩a 2/ and /ʝi˥/} \textcolor{Sepia}{\selectlanguage{english}True, real; really, truly.} \zh{真,真的。}  ¶ \textcolor{darkblue}{\textbf{\ipa{mɤ˧-gi˩!}}} \textcolor{Sepia}{\selectlanguage{english}(It) is not true!} \zh{不是的! / 不是真的!}  
 ¶ \textcolor{darkblue}{\textbf{\ipa{ə˩-gi˩˥?}}} \textcolor{Sepia}{\selectlanguage{english}Right? / Is that true? / It is true, isn't it?} \zh{对吧? / 对吗?}  
 ¶ \textcolor{darkblue}{\textbf{\ipa{ə˩-gi˩˥ ? – gi˩˥!}}} \textcolor{Sepia}{\selectlanguage{english}Is that right? - Yes, it is! (One speaker asks for confirmation; the other provides confirmation.)} \zh{对吧? -对的!}  
 ¶ \textcolor{darkblue}{\textbf{\ipa{gi˩-hĩ˩ ʐwɤ˥}}} \textcolor{Sepia}{\selectlanguage{english}to speak the truth, to tell the truth} \zh{说实话,老实说}  
 ¶ \textcolor{darkblue}{\textbf{\ipa{gi˩˥ | -gɯ˩˥}}} \textcolor{Sepia}{\selectlanguage{english}truly, veritably} \zh{真的,真正的}  

\lhead{\firstmark}
\rhead{\botmark}

\subsection{\hspace{-0.5cm} {\Large \textcolor{darkblue}{\textbf{\ipa{gi˩kɯ˩}}}}\hspace{0.5cm}[\kern2pt{\textcolor{darkblue}{\textbf{\ipa{gi˧kɯ˧}}}}\kern2pt]} \hypertarget{gi\string_BkM\string_B1}{}
\markboth{\textcolor{darkblue}{\textbf{\ipa{gi˩kɯ˩}}}}{}
\textcolor{teal}{\mytextsc{noun}} \hspace{4pt} Tone: L.
\textcolor{Sepia}{\selectlanguage{english}Musk (literally: 'bear's gall').} \zh{麝香(直译:大熊胆)。} 
\lhead{\firstmark}
\rhead{\botmark}

\subsection{\hspace{-0.5cm} {\Large \textcolor{darkblue}{\textbf{\ipa{‑gi˧˥}}}}\hspace{0.5cm}[\kern2pt{\textcolor{darkblue}{\textbf{\ipa{gi˧˥}}}}\kern2pt]} \hypertarget{‑gi\string_M\string_T1}{}
\markboth{\textcolor{darkblue}{\textbf{\ipa{‑gi˧˥}}}}{}
\textcolor{teal}{\mytextsc{postposition}} \hspace{4pt} Tone: MH.
\textcolor{Sepia}{\selectlanguage{english}Behind.} \zh{后面,(最)后。}  ¶ \textcolor{darkblue}{\textbf{\ipa{ə˧mɑ˧-gi˧˥}}} \textcolor{Sepia}{\selectlanguage{english}behind mummy} \zh{妈妈后面}  
 ¶ \textcolor{darkblue}{\textbf{\ipa{lɑ˧-gi˧˥}}} \textcolor{Sepia}{\selectlanguage{english}behind the tiger} \zh{老虎后面}  
 ¶ \textcolor{darkblue}{\textbf{\ipa{bo˩-gi˥}}} \textcolor{Sepia}{\selectlanguage{english}behind the pig} \zh{猪后面}  
 ¶ \textcolor{darkblue}{\textbf{\ipa{mv̩˩-gi˥}}} \textcolor{Sepia}{\selectlanguage{english}behind the daughter} \zh{女儿后面}  
 ¶ \textcolor{darkblue}{\textbf{\ipa{ʐwæ˧-gi˥}}} \textcolor{Sepia}{\selectlanguage{english}behind the horse} \zh{马后面}  
 ¶ \textcolor{darkblue}{\textbf{\ipa{ʈʂʰɯ˧-gi˥ | tʰi˧-tɕʰo˩}}} \textcolor{Sepia}{\selectlanguage{english}to hide in there (literally 'behind there')} \zh{藏那后面}  
 ¶ \textcolor{darkblue}{\textbf{\ipa{no˧-gi˧ njɤ˥ ʈʂwæ˩!}}} \textcolor{Sepia}{\selectlanguage{english}I follow in your footsteps! / I follow you! / I imitate you!} \zh{我跟你走! / 我都按你说的来做吧!}  
 ¶ \textcolor{darkblue}{\textbf{\ipa{ɖɯ˧-v̩˧-gi˧˥, | ɖɯ˧-v̩˧ hwæ˧!}}} \textcolor{Sepia}{\selectlanguage{english}to buy one after the other (context: someone buys one horse after the other, to put together a complete caravan of his own)} \zh{一个接着一个地买(情景:一个人接二连三地买马,最后组成自己的马帮队)}  
 ¶ \textcolor{darkblue}{\textbf{\ipa{[F5] gi˧˥ | ɖɯ˧-qɑ˩ gv̩˩-bi˩!}}} \textcolor{Sepia}{\selectlanguage{english}Let's do one last bundle! (Context: women are extracting flax fiber, processing bundle after bundle; towards the end of a long work session, someone says: “Let's do one last bundle! / One last bundle and we shall call it a day!”)} \zh{再做一捆吧!(情景:女人们在纺麻线,工作了很久,一个人就说:“再做最后一捆(就收工吧)!”)}  
 ¶ \textcolor{darkblue}{\textbf{\ipa{gi˧-se˧}}} \textcolor{Sepia}{\selectlanguage{english}to walk after, to follow after} \zh{在后面走,在后面跟着}  

\lhead{\firstmark}
\rhead{\botmark}

\subsection{\hspace{-0.5cm} {\Large \textcolor{darkblue}{\textbf{\ipa{go˧bɤ˩}}}}\hspace{0.5cm}[\kern2pt{\textcolor{darkblue}{\textbf{\ipa{go˩bɤ˥}}}}\kern2pt]} \hypertarget{go\string_Mb7\string_B1}{}
\markboth{\textcolor{darkblue}{\textbf{\ipa{go˧bɤ˩}}}}{}
\textcolor{teal}{\mytextsc{noun}} \hspace{4pt} Tone: L\#.
\textcolor{Sepia}{\selectlanguage{english}Temple, monastery.} \zh{庙,寺。}  Borrowing: Tibetan  dgon pa
 \zh{量词}: \textcolor{darkblue}{\textbf{\ipa{ɭɯ˧}}}  \mytextsc{clf}: \textcolor{darkblue}{\textbf{\ipa{ɭɯ˧}}} 
\lhead{\firstmark}
\rhead{\botmark}

\subsection{\hspace{-0.5cm} {\Large \textcolor{darkblue}{\textbf{\ipa{go˧mi˧}}}}\hspace{0.5cm}[\kern2pt{\textcolor{darkblue}{\textbf{\ipa{go˧mi˧}}}}\kern2pt]} \hypertarget{go\string_Mmi\string_M1}{}
\markboth{\textcolor{darkblue}{\textbf{\ipa{go˧mi˧}}}}{}
\textcolor{teal}{\mytextsc{noun}} \hspace{4pt} Tone: M.
\textcolor{Sepia}{\selectlanguage{english}Younger sister.} \zh{妹妹。}  ¶ \textcolor{darkblue}{\textbf{\ipa{go˧mi˧=ɻæ˩}}} \textcolor{Sepia}{\selectlanguage{english}\mytextsc{associative}: younger sisters, younger cousins} \zh{联想复数:妹妹们,表妹们}  
 \zh{量词}: \textcolor{darkblue}{\textbf{\ipa{v̩˧}}}  \mytextsc{clf}: \textcolor{darkblue}{\textbf{\ipa{v̩˧}}} 
\lhead{\firstmark}
\rhead{\botmark}

\subsection{\hspace{-0.5cm} {\Large \textcolor{darkblue}{\textbf{\ipa{go˩\textsubscript{a}}}}}\hspace{0.5cm}[\kern2pt{\textcolor{darkblue}{\textbf{\ipa{go˧˥}}}}\kern2pt]} \hypertarget{go\string_Ba1}{}
\markboth{\textcolor{darkblue}{\textbf{\ipa{go˩\textsubscript{a}}}}}{}
\textcolor{teal}{\mytextsc{verb}} \hspace{4pt} Tone: L\textsubscript{a}.
\textcolor{Sepia}{\selectlanguage{english}To suffer; to be sick, to be ill.} \zh{痛,病 (生病)。}  ¶ \textcolor{darkblue}{\textbf{\ipa{njɤ˧ | go˩˥!}}} \textcolor{Sepia}{\selectlanguage{english}I am suffering! / It hurts!} \zh{我痛!}  
 ¶ \textcolor{darkblue}{\textbf{\ipa{njɤ˧ | go˩˥ | ʐwæ˩˥!}}} \textcolor{Sepia}{\selectlanguage{english}I am suffering a lot! / It hurts a lot!} \zh{我好疼!}  
 ¶ \textcolor{darkblue}{\textbf{\ipa{go˩-hĩ˩˥}}} \textcolor{Sepia}{\selectlanguage{english}\mytextsc{nmlz}: patient, sick person} \zh{病人,病的(人)}  
 ¶ \textcolor{darkblue}{\textbf{\ipa{hĩ˧ | go˩-hĩ˩˥}}} \textcolor{Sepia}{\selectlanguage{english}sick person, person who is ill, patient} \zh{病人}  
 ¶ \textcolor{darkblue}{\textbf{\ipa{bi˧mi˧ go˩}}} \textcolor{Sepia}{\selectlanguage{english}to have stomach-ache} \zh{肚子疼}  

\lhead{\firstmark}
\rhead{\botmark}

\subsection{\hspace{-0.5cm} {\Large \textcolor{darkblue}{\textbf{\ipa{go˩bi˧}}}}\hspace{0.5cm}[\kern2pt{\textcolor{darkblue}{\textbf{\ipa{go˧bi˧}}}}\kern2pt]} \hypertarget{go\string_Bbi\string_M1}{}
\markboth{\textcolor{darkblue}{\textbf{\ipa{go˩bi˧}}}}{}
\textcolor{teal}{\mytextsc{noun}} \hspace{4pt} Tone: LM.
\textcolor{Sepia}{\selectlanguage{english}The city of Lijiang.} \zh{丽江城。}  ¶ \textcolor{darkblue}{\textbf{\ipa{go˩bi˧-ɖʐɯ˧qo˩}}} \textcolor{Sepia}{\selectlanguage{english}the city of Lijiang} \zh{丽江城}  

\lhead{\firstmark}
\rhead{\botmark}

\subsection{\hspace{-0.5cm} {\Large \textcolor{darkblue}{\textbf{\ipa{go˩bo˥}}}}\hspace{0.5cm}[\kern2pt{\textcolor{darkblue}{\textbf{\ipa{go˧bo˧}}}}\kern2pt]} \hypertarget{go\string_Bbo\string_T1}{}
\markboth{\textcolor{darkblue}{\textbf{\ipa{go˩bo˥}}}}{}
\textcolor{teal}{\mytextsc{noun}} \hspace{4pt} Tone: LH.
\textcolor{Sepia}{\selectlanguage{english}Livestock.} \zh{牲畜。}  \zh{量词}: \textcolor{darkblue}{\textbf{\ipa{pʰo˧˥}}}  \mytextsc{clf}: \textcolor{darkblue}{\textbf{\ipa{pʰo˧˥}}} 
\lhead{\firstmark}
\rhead{\botmark}

\subsection{\hspace{-0.5cm} {\Large \textcolor{darkblue}{\textbf{\ipa{gɯ˩\textsubscript{a}}}} \textsubscript{1}}\hspace{0.5cm}[\kern2pt{\textcolor{darkblue}{\textbf{\ipa{gɯ˩˥}}}}\kern2pt]} \hypertarget{gM\string_Ba1}{}
\markboth{\textcolor{darkblue}{\textbf{\ipa{gɯ˩\textsubscript{a}}}} \textsubscript{1}}{}
\textcolor{teal}{\mytextsc{verb}} \hspace{4pt} Tone: L\textsubscript{a}.
\textcolor{Sepia}{\selectlanguage{english}To believe.} \zh{相信。}  ¶ \textcolor{darkblue}{\textbf{\ipa{ʈʂʰɯ˧-ɳɯ˧ ʐwɤ˩-hĩ˩, | njɤ˧ | mɤ˧-gɯ˩!}}} \textcolor{Sepia}{\selectlanguage{english}I do not believe what (s)he says!} \zh{他说的话,我不相信!}  

\lhead{\firstmark}
\rhead{\botmark}

\subsection{\hspace{-0.5cm} {\Large \textcolor{darkblue}{\textbf{\ipa{gɯ˩\textsubscript{a}}}} \textsubscript{2}}\hspace{0.5cm}[\kern2pt{\textcolor{darkblue}{\textbf{\ipa{gɯ˩˥}}}}\kern2pt]} \hypertarget{gM\string_Ba2}{}
\markboth{\textcolor{darkblue}{\textbf{\ipa{gɯ˩\textsubscript{a}}}} \textsubscript{2}}{}
\textcolor{teal}{\mytextsc{adjective}} \hspace{4pt} Tone: L\textsubscript{a}.
\textcolor{Sepia}{\selectlanguage{english}True, authentic, veritable.} \zh{真,真的。}  ¶ \textcolor{darkblue}{\textbf{\ipa{mɤ˧-gɯ˩}}} \textcolor{Sepia}{\selectlanguage{english}not true} \zh{不是真的}  
 ¶ \textcolor{darkblue}{\textbf{\ipa{gɯ˩-hĩ˩˥}}} \textcolor{Sepia}{\selectlanguage{english}\mytextsc{nmlz}} \zh{真的}  
 ¶ \textcolor{darkblue}{\textbf{\ipa{ə˩-gɯ˩˥?}}} \textcolor{Sepia}{\selectlanguage{english}Is that true?} \zh{真的吗?}  
 ¶ \textcolor{darkblue}{\textbf{\ipa{gɯ˩ wɤ˩-ɻ̍˥!}}} \textcolor{Sepia}{\selectlanguage{english}It's really like that! / Yes, it is indeed true!} \zh{就是真的啊! / 的确是这样啊!}  
 ¶ \textcolor{darkblue}{\textbf{\ipa{gɯ˩-ʝi˥?}}} \textcolor{Sepia}{\selectlanguage{english}Really?} \zh{原来是这样吗?}  
 ¶ \textcolor{darkblue}{\textbf{\ipa{gɯ˩ ʂv̩˩ɖv̩˩˥}}} \textcolor{Sepia}{\selectlanguage{english}to believe in (something); literally: 'to think (that something is) true'} \zh{相信}  
 ¶ \textcolor{darkblue}{\textbf{\ipa{gɯ˧ ʐwɤ˧}}} \textcolor{Sepia}{\selectlanguage{english}to say the truth} \zh{说实话}  

\lhead{\firstmark}
\rhead{\botmark}

\subsection{\hspace{-0.5cm} {\Large \textcolor{darkblue}{\textbf{\ipa{gɯ˩ɭɯ˧˥}}}}\hspace{0.5cm}[\kern2pt{\textcolor{darkblue}{\textbf{\ipa{gɯ˩ɭɯ˧˥}}}}\kern2pt]} \hypertarget{gM\string_Bl\string_RM\string_M\string_T1}{}
\markboth{\textcolor{darkblue}{\textbf{\ipa{gɯ˩ɭɯ˧˥}}}}{}
\textcolor{teal}{\mytextsc{verb}} \hspace{4pt} Tone: LM+MH\#.
\textcolor{Sepia}{\selectlanguage{english}To rub, to knead (e.g. rub one's hands).} \zh{揉。}  ¶ \textcolor{darkblue}{\textbf{\ipa{gɯ˩ɭɯ˧-ze˥}}} \textcolor{Sepia}{\selectlanguage{english}\mytextsc{pfv}} \zh{揉了}  
 ¶ \textcolor{darkblue}{\textbf{\ipa{le˧-gɯ˩ɭɯ˩+ze˩}}} \textcolor{Sepia}{\selectlanguage{english}\mytextsc{accomp} \string_ \mytextsc{pfv}} \zh{揉了}  
 ¶ \textcolor{darkblue}{\textbf{\ipa{le˧-gɯ˩ɭɯ˩\textasciitilde{}le˧-gɯ˩ɭɯ˩}}} \textcolor{Sepia}{\selectlanguage{english}\mytextsc{accomp} \mytextsc{red} \mytextsc{pfv}} \zh{揉一揉}  

\lhead{\firstmark}
\rhead{\botmark}

\subsection{\hspace{-0.5cm} {\Large \textcolor{darkblue}{\textbf{\ipa{gv̩˧}}} \textsubscript{1}}\hspace{0.5cm}[\kern2pt{\textcolor{darkblue}{\textbf{\ipa{gv̩˥}}}}\kern2pt]} \hypertarget{gv\string_=\string_M1}{}
\markboth{\textcolor{darkblue}{\textbf{\ipa{gv̩˧}}} \textsubscript{1}}{}
\textcolor{teal}{\mytextsc{verb}} \hspace{4pt} Tone: M\textsubscript{c}.
\textcolor{Sepia}{\selectlanguage{english}To flow, to go by, to elapse (time); to take place, to occur (event).} \zh{过去 (时间)、过,发生。}  ¶ \textcolor{darkblue}{\textbf{\ipa{le˧-gv̩˩-ze˩}}} \textcolor{Sepia}{\selectlanguage{english}\mytextsc{accomp} \string_ \mytextsc{pfv}} \zh{已经过去了}  
 ¶ \textcolor{darkblue}{\textbf{\ipa{ɖɯ˧-ɭɯ˧ gv̩˧}}} \textcolor{Sepia}{\selectlanguage{english}an hour goes by} \zh{一个小时过去了}  
 ¶ \textcolor{darkblue}{\textbf{\ipa{tsʰe˩-ɲi˩ gv̩˩-ze˥!}}} \textcolor{Sepia}{\selectlanguage{english}Ten days have gone by/ten days have elapsed} \zh{十天过去了}  
 ¶ \textcolor{darkblue}{\textbf{\ipa{mɤ˧-gv̩˧-ze˧!}}} \textcolor{Sepia}{\selectlanguage{english}It won't do! / It won't work! / It's no good!} \zh{不好了! / 不行了!}  
 ¶ \textcolor{darkblue}{\textbf{\ipa{ʈʂʰɯ˧ne˧-ʝi˥ | gv̩˧, -tsɯ˩-mv̩˩!}}} \textcolor{Sepia}{\selectlanguage{english}They say that's how it happened!} \zh{据说是这样发生的!}  

\lhead{\firstmark}
\rhead{\botmark}

\subsection{\hspace{-0.5cm} {\Large \textcolor{darkblue}{\textbf{\ipa{gv̩˧}}} \textsubscript{2}}\hspace{0.5cm}[\kern2pt{\textcolor{darkblue}{\textbf{\ipa{gv̩˥}}}}\kern2pt]} \hypertarget{gv\string_=\string_M2}{}
\markboth{\textcolor{darkblue}{\textbf{\ipa{gv̩˧}}} \textsubscript{2}}{}
\textcolor{teal}{\mytextsc{adjective}} \hspace{4pt} Tone: M.
\textcolor{Sepia}{\selectlanguage{english}Good (good heart).} \zh{好(心好)。}  ¶ \textcolor{darkblue}{\textbf{\ipa{ɖwæ˧˥ | gv̩˧!}}} \textcolor{Sepia}{\selectlanguage{english}\mytextsc{intensive}.very} \zh{很好!}  
 ¶ \textcolor{darkblue}{\textbf{\ipa{mɤ˧-gv̩˧!}}} \textcolor{Sepia}{\selectlanguage{english}\mytextsc{neg}} \zh{不好}  

\lhead{\firstmark}
\rhead{\botmark}

\subsection{\hspace{-0.5cm} {\Large \textcolor{darkblue}{\textbf{\ipa{gv̩˧}}} \textsubscript{3}}\hspace{0.5cm}[\kern2pt{\textcolor{darkblue}{\textbf{\ipa{gv̩˥}}}}\kern2pt]} \hypertarget{gv\string_=\string_M3}{}
\markboth{\textcolor{darkblue}{\textbf{\ipa{gv̩˧}}} \textsubscript{3}}{}
\textcolor{teal}{\mytextsc{number}} \hspace{4pt} Tone: M? H\#? (pas L).
\textcolor{Sepia}{\selectlanguage{english}9.} \zh{9。} 
\lhead{\firstmark}
\rhead{\botmark}

\subsection{\hspace{-0.5cm} {\Large \textcolor{darkblue}{\textbf{\ipa{gv̩˧}}} \textsubscript{4}}\hspace{0.5cm}[\kern2pt{\textcolor{darkblue}{\textbf{\ipa{gv̩˥}}}}\kern2pt]} \hypertarget{gv\string_=\string_M4}{}
\markboth{\textcolor{darkblue}{\textbf{\ipa{gv̩˧}}} \textsubscript{4}}{}
\textcolor{teal}{\mytextsc{verb}} \hspace{4pt} Tone: M.
\textcolor{Sepia}{\selectlanguage{english}To be; to become.} \zh{系词。}  ¶ \textcolor{darkblue}{\textbf{\ipa{ʈʂʰɯ˧ | no˧ | ɲi˧gɤ˧ | ʂwæ˧-mɤ˧-gv̩˧!}}} \textcolor{Sepia}{\selectlanguage{english}Her nose is less straight than yours! (About a little girl whose nose does not resemble her father's straight nose)} \zh{她的鼻子没有你的直!(关于一个鼻子比较扁的小女孩)}  
 ¶ \textcolor{darkblue}{\textbf{\ipa{ʐæ˧ni˩ | mɤ˧-gv̩˧}}} \textcolor{Sepia}{\selectlanguage{english}not tall, not impressive, not great-looking} \zh{个子不高}  

\lhead{\firstmark}
\rhead{\botmark}

\subsection{\hspace{-0.5cm} {\Large \textcolor{darkblue}{\textbf{\ipa{gv̩˥}}}}\hspace{0.5cm}[\kern2pt{\textcolor{darkblue}{\textbf{\ipa{gv̩˥}}}}\kern2pt]} \hypertarget{gv\string_=\string_T1}{}
\markboth{\textcolor{darkblue}{\textbf{\ipa{gv̩˥}}}}{}
\textcolor{teal}{\mytextsc{verb}} \hspace{4pt} Tone: H.
\textcolor{Sepia}{\selectlanguage{english}To cross, to get over (a river, a lake…).} \zh{过(一条河、一个湖……)。}  ¶ \textcolor{darkblue}{\textbf{\ipa{dʑɯ˩ gv̩˩˥}}} \textcolor{Sepia}{\selectlanguage{english}to cross a river} \zh{过河}  

\lhead{\firstmark}
\rhead{\botmark}

\subsection{\hspace{-0.5cm} {\Large \textcolor{darkblue}{\textbf{\ipa{gv̩˥}}}}\hspace{0.5cm}[\kern2pt{\textcolor{darkblue}{\textbf{\ipa{gv̩˥}}}}\kern2pt]} \hypertarget{gv\string_=\string_T1}{}
\markboth{\textcolor{darkblue}{\textbf{\ipa{gv̩˥}}}}{}
\textcolor{teal}{\mytextsc{noun}} \hspace{4pt} Tone: \#H.
\textcolor{Sepia}{\selectlanguage{english}Manger.} \zh{马槽。}  ¶ \textcolor{darkblue}{\textbf{\ipa{ʐwæ˧gv̩\#˥}}} \textcolor{Sepia}{\selectlanguage{english}horse's manger} \zh{马槽}  
 \zh{量词}: \textcolor{darkblue}{\textbf{\ipa{ɭɯ˧}}}  \mytextsc{clf}: \textcolor{darkblue}{\textbf{\ipa{ɭɯ˧}}} 
\lhead{\firstmark}
\rhead{\botmark}

\subsection{\hspace{-0.5cm} {\Large \textcolor{darkblue}{\textbf{\ipa{gv̩˩\textsubscript{a}}}} \textsubscript{1}}\hspace{0.5cm}[\kern2pt{\textcolor{darkblue}{\textbf{\ipa{gv̩˩˥}}}}\kern2pt]} \hypertarget{gv\string_=\string_Ba1}{}
\markboth{\textcolor{darkblue}{\textbf{\ipa{gv̩˩\textsubscript{a}}}} \textsubscript{1}}{}
\textcolor{teal}{\mytextsc{verb}} \hspace{4pt} Tone: L\textsubscript{a}.
\ding{202} \textcolor{Sepia}{\selectlanguage{english}To prepare (a meal), to cook.} \zh{做(饭)。}  ¶ \textcolor{darkblue}{\textbf{\ipa{hɑ˧ gv̩˥}}} \textcolor{Sepia}{\selectlanguage{english}to cook, to prepare a meal} \zh{做饭}  
 ¶ \textcolor{darkblue}{\textbf{\ipa{le˧-gv̩˩-ze˩}}} \textcolor{Sepia}{\selectlanguage{english}\mytextsc{accomp} \string_ \mytextsc{pfv}} \zh{做(饭)了}  
 ¶ \textcolor{darkblue}{\textbf{\ipa{njɤ˧ | hɑ˧ gv̩˥-bi˩!}}} \textcolor{Sepia}{\selectlanguage{english}Let me do the cooking! / I'm doing the cooking!} \zh{我来做饭吧!}  
\ding{203} \textcolor{Sepia}{\selectlanguage{english}To construct, to build (a house).} \zh{盖、建 (房子)。}  ¶ \textcolor{darkblue}{\textbf{\ipa{ʑi˧qʰwɤ˧ gv̩˩}}} \textcolor{Sepia}{\selectlanguage{english}to build a house} \zh{建房}  
\ding{204} \textcolor{Sepia}{\selectlanguage{english}To repair; to make (a tool, a machine…).} \zh{修理、做出来(工具)。}  ¶ \textcolor{darkblue}{\textbf{\ipa{le˧-gv̩˧\textasciitilde{}gv̩˥}}} \textcolor{Sepia}{\selectlanguage{english}\mytextsc{red}: to repair} \zh{\mytextsc{重叠:修理}}  
 ¶ \textcolor{darkblue}{\textbf{\ipa{le˧-gv̩˩ | le˧-tʰv̩˧-ze˧!}}} \textcolor{Sepia}{\selectlanguage{english}It's repaired! / It's done! / I have finished doing it!} \zh{修理好了!/ 修理出来了!}  

\lhead{\firstmark}
\rhead{\botmark}

\subsection{\hspace{-0.5cm} {\Large \textcolor{darkblue}{\textbf{\ipa{gv̩˩\textsubscript{a}}}} \textsubscript{2}}\hspace{0.5cm}[\kern2pt{\textcolor{darkblue}{\textbf{\ipa{gv̩˩˥}}}}\kern2pt]} \hypertarget{gv\string_=\string_Ba2}{}
\markboth{\textcolor{darkblue}{\textbf{\ipa{gv̩˩\textsubscript{a}}}} \textsubscript{2}}{}
\textcolor{teal}{\mytextsc{verb}} \hspace{4pt} Tone: L\textsubscript{a}.
\textcolor{Sepia}{\selectlanguage{english}To tidy up, to sort out.} \zh{收拾。}  ¶ \textcolor{darkblue}{\textbf{\ipa{tʰi˧-gv̩˧\textasciitilde{}gv̩˥}}} \textcolor{Sepia}{\selectlanguage{english}\mytextsc{dur}} \zh{\mytextsc{dur}}  
 ¶ \textcolor{darkblue}{\textbf{\ipa{ɖɯ˧-gv̩˧\textasciitilde{}gv̩˥-ɻ̍˩}}} \textcolor{Sepia}{\selectlanguage{english}to clear up a little} \zh{收拾一下}  
 ¶ \textcolor{darkblue}{\textbf{\ipa{le˧-gv̩˧\textasciitilde{}gv̩˥ | tʰi˧-tɕɯ˥}}} \textcolor{Sepia}{\selectlanguage{english}to tidy up and put (everything) into place} \zh{收拾,摆好}  

\lhead{\firstmark}
\rhead{\botmark}

\subsection{\hspace{-0.5cm} {\Large \textcolor{darkblue}{\textbf{\ipa{gv̩˩\textsubscript{a}}}} \textsubscript{3}}\hspace{0.5cm}[\kern2pt{\textcolor{darkblue}{\textbf{\ipa{gv̩˩˥}}}}\kern2pt]} \hypertarget{gv\string_=\string_Ba3}{}
\markboth{\textcolor{darkblue}{\textbf{\ipa{gv̩˩\textsubscript{a}}}} \textsubscript{3}}{}
\textcolor{teal}{\mytextsc{verb}} \hspace{4pt} Tone: L\textsubscript{a}.
\textcolor{Sepia}{\selectlanguage{english}To set (the sun sets), to decline.} \zh{落下(太阳落山)。}  ¶ \textcolor{darkblue}{\textbf{\ipa{ɲi˧mi˧ gv̩˩-se˩}}} \textcolor{Sepia}{\selectlanguage{english}after the sun has set, after sunset} \zh{在太阳落山之后,在太阳落山了以后}  
 ¶ \textcolor{darkblue}{\textbf{\ipa{ɲi˧mi˧ | le˧-gv̩˩-ze˩.}}} \textcolor{Sepia}{\selectlanguage{english}The sun has set.} \zh{太阳落山了。}  
 ¶ \textcolor{darkblue}{\textbf{\ipa{ɲi˧mi˧ | mɤ˧-gv̩˩-sɯ˩.}}} \textcolor{Sepia}{\selectlanguage{english}The sun has not set yet.} \zh{太阳还没有落。}  

\lhead{\firstmark}
\rhead{\botmark}

\subsection{\hspace{-0.5cm} {\Large \textcolor{darkblue}{\textbf{\ipa{gv̩˧dv̩˧}}}}\hspace{0.5cm}[\kern2pt{\textcolor{darkblue}{\textbf{\ipa{gv̩˩dv̩˩˥}}}}\kern2pt]} \hypertarget{gv\string_=\string_Mdv\string_=\string_M1}{}
\markboth{\textcolor{darkblue}{\textbf{\ipa{gv̩˧dv̩˧}}}}{}
\textcolor{teal}{\mytextsc{noun}} \hspace{4pt} Tone: M.
\textcolor{Sepia}{\selectlanguage{english}Back.} \zh{脊背。}  \zh{量词}: \textcolor{darkblue}{\textbf{\ipa{ʈv̩˩}}}  \mytextsc{clf}: \textcolor{darkblue}{\textbf{\ipa{ʈv̩˩}}} 
\lhead{\firstmark}
\rhead{\botmark}

\subsection{\hspace{-0.5cm} {\Large \textcolor{darkblue}{\textbf{\ipa{gv̩˧dv̩˧-gv̩˧mi˧}}}}\hspace{0.5cm}[\kern2pt{\textcolor{darkblue}{\textbf{\ipa{xxxx non-correspondance entre le nombre de morphèmes et le nombre de tons de morphèmes}}}}\kern2pt]} \hypertarget{gv\string_=\string_Mdv\string_=\string_M-gv\string_=\string_Mmi\string_M1}{}
\markboth{\textcolor{darkblue}{\textbf{\ipa{gv̩˧dv̩˧-gv̩˧mi˧}}}}{}
\textcolor{teal}{\mytextsc{noun}} \hspace{4pt} Tone: M.
\textit{From:} \textbf{gv̩˧dv̩˧ and gv̩˧mi˧} \textcolor{Sepia}{\selectlanguage{english}Body.} \zh{身体。}  \zh{量词}: \textcolor{darkblue}{\textbf{\ipa{ɭɯ˧}}}  \mytextsc{clf}: \textcolor{darkblue}{\textbf{\ipa{ɭɯ˧}}} 
\lhead{\firstmark}
\rhead{\botmark}

\subsection{\hspace{-0.5cm} {\Large \textcolor{darkblue}{\textbf{\ipa{gv̩˩dʑɯ˩}}}}\hspace{0.5cm}[\kern2pt{\textcolor{darkblue}{\textbf{\ipa{gv̩˧dʑɯ˧}}}}\kern2pt]} \hypertarget{gv\string_=\string_Bdz£M\string_B1}{}
\markboth{\textcolor{darkblue}{\textbf{\ipa{gv̩˩dʑɯ˩}}}}{}
\textcolor{teal}{\mytextsc{adjective}} \hspace{4pt} Tone: L.
\textcolor{Sepia}{\selectlanguage{english}Angry; afflicted.} \zh{生气。} 
\lhead{\firstmark}
\rhead{\botmark}

\subsection{\hspace{-0.5cm} {\Large \textcolor{darkblue}{\textbf{\ipa{gv̩˧kv̩˩}}}}\hspace{0.5cm}[\kern2pt{\textcolor{darkblue}{\textbf{\ipa{gv̩˩kv̩˩˥}}}}\kern2pt]} \hypertarget{gv\string_=\string_Mkv\string_=\string_B1}{}
\markboth{\textcolor{darkblue}{\textbf{\ipa{gv̩˧kv̩˩}}}}{}
\textcolor{teal}{\mytextsc{noun}} \hspace{4pt} Tone: L\#.
\textcolor{Sepia}{\selectlanguage{english}Intonation, way of speaking; can be used, by extension, to refer to tones.} \zh{语调,声调。}  ¶ \textcolor{darkblue}{\textbf{\ipa{gv̩˧kv̩˩-gv̩˩li˩ | ʐwɤ˩˥}}} \textcolor{Sepia}{\selectlanguage{english}to speak with a pleasant style, to deliver one's speech with elegance} \zh{说话说得好听、有口才、口若悬河、能言善辩}  

\lhead{\firstmark}
\rhead{\botmark}

\subsection{\hspace{-0.5cm} {\Large \textcolor{darkblue}{\textbf{\ipa{gv̩˩ɬi˩mi˩}}}}\hspace{0.5cm}[\kern2pt{\textcolor{darkblue}{\textbf{\ipa{gv̩˧ɬi˧mi˩}}}}\kern2pt]} \hypertarget{gv\string_=\string_BKi\string_Bmi\string_B1}{}
\markboth{\textcolor{darkblue}{\textbf{\ipa{gv̩˩ɬi˩mi˩}}}}{}
\textcolor{teal}{\mytextsc{noun}} \hspace{4pt} Tone: L.
\textcolor{Sepia}{\selectlanguage{english}9th month.} \zh{九月。} 
\lhead{\firstmark}
\rhead{\botmark}

\subsection{\hspace{-0.5cm} {\Large \textcolor{darkblue}{\textbf{\ipa{gv̩˧mɑ˧}}}}\hspace{0.5cm}[\kern2pt{\textcolor{darkblue}{\textbf{\ipa{gv̩˩mɑ˩˥}}}}\kern2pt]} \hypertarget{gv\string_=\string_MmA\string_M1}{}
\markboth{\textcolor{darkblue}{\textbf{\ipa{gv̩˧mɑ˧}}}}{}
\textcolor{teal}{\mytextsc{noun}} \hspace{4pt} Tone: M.
\textcolor{Sepia}{\selectlanguage{english}Masculine given name.} \zh{男性名字。}  ¶ \textcolor{darkblue}{\textbf{\ipa{hĩ˧ | ʈʂʰɯ˧-v̩˧, | gv̩˧mɑ˧ mv̩˧ʈʂæ˧˥!}}} \textcolor{Sepia}{\selectlanguage{english}This person is called \textcolor{darkblue}{\textbf{\ipa{/gv̩˧mɑ˧/!}}}} \zh{这个人,名叫\textcolor{darkblue}{\textbf{\ipa{/gv̩˧mɑ˧/}}}!}  

\lhead{\firstmark}
\rhead{\botmark}

\subsection{\hspace{-0.5cm} {\Large \textcolor{darkblue}{\textbf{\ipa{gv̩˧mi˧}}}}\hspace{0.5cm}[\kern2pt{\textcolor{darkblue}{\textbf{\ipa{gv̩˧mi˧}}}}\kern2pt]} \hypertarget{gv\string_=\string_Mmi\string_M1}{}
\markboth{\textcolor{darkblue}{\textbf{\ipa{gv̩˧mi˧}}}}{}
\textcolor{teal}{\mytextsc{noun}} \hspace{4pt} Tone: M.
\textcolor{Sepia}{\selectlanguage{english}Body.} \zh{身体。}  \zh{量词}: \textcolor{darkblue}{\textbf{\ipa{ɭɯ˧}}}  \mytextsc{clf}: \textcolor{darkblue}{\textbf{\ipa{ɭɯ˧}}} 
\lhead{\firstmark}
\rhead{\botmark}

\subsection{\hspace{-0.5cm} {\Large \textcolor{darkblue}{\textbf{\ipa{gv̩˩pʰæ˩}}}}\hspace{0.5cm}[\kern2pt{\textcolor{darkblue}{\textbf{\ipa{gv̩˧pʰæ˧}}}}\kern2pt]} \hypertarget{gv\string_=\string_Bp\string_h\{\string_B1}{}
\markboth{\textcolor{darkblue}{\textbf{\ipa{gv̩˩pʰæ˩}}}}{}
\textcolor{teal}{\mytextsc{noun}} \hspace{4pt} Tone: L.
\textcolor{Sepia}{\selectlanguage{english}Thin plank.} \zh{相当薄的木板。}  \zh{量词}: \textcolor{darkblue}{\textbf{\ipa{pʰæ˧˥}}}  \mytextsc{clf}: \textcolor{darkblue}{\textbf{\ipa{pʰæ˧˥}}} 
\lhead{\firstmark}
\rhead{\botmark}

\subsection{\hspace{-0.5cm} {\Large \textcolor{darkblue}{\textbf{\ipa{gv̩˧sɯ˥-pv̩˩}}}}\hspace{0.5cm}[\kern2pt{\textcolor{darkblue}{\textbf{\ipa{xxxx non-correspondance entre le nombre de morphèmes et le nombre de tons de morphèmes}}}}\kern2pt]} \hypertarget{gv\string_=\string_MsM\string_T-pv\string_=\string_B1}{}
\markboth{\textcolor{darkblue}{\textbf{\ipa{gv̩˧sɯ˥-pv̩˩}}}}{}
\textcolor{teal}{\mytextsc{noun}} \hspace{4pt} Tone: H\#-L.
\textcolor{Sepia}{\selectlanguage{english}Shoulderblade, scapula.} \zh{肩胛骨。}  \zh{量词}: \textcolor{darkblue}{\textbf{\ipa{kʰwɤ˥}}}  \mytextsc{clf}: \textcolor{darkblue}{\textbf{\ipa{kʰwɤ˥}}} 
\lhead{\firstmark}
\rhead{\botmark}

\subsection{\hspace{-0.5cm} {\Large \textcolor{darkblue}{\textbf{\ipa{gv̩˧tɕʰɯ˧˥}}}}\hspace{0.5cm}[\kern2pt{\textcolor{darkblue}{\textbf{\ipa{xxxx non-correspondance entre le nombre de morphèmes et le nombre de tons de morphèmes}}}}\kern2pt]} \hypertarget{gv\string_=\string_Mts£\string_hM\string_M\string_T1}{}
\markboth{\textcolor{darkblue}{\textbf{\ipa{gv̩˧tɕʰɯ˧˥}}}}{}
\textcolor{teal}{\mytextsc{verb}} \hspace{4pt} Tone: MH\#.
\textcolor{Sepia}{\selectlanguage{english}To catch a cold.} \zh{着凉。} 
\lhead{\firstmark}
\rhead{\botmark}

\subsection{\hspace{-0.5cm} {\Large \textcolor{darkblue}{\textbf{\ipa{gv̩˧tsʰi˩}}}}\hspace{0.5cm}[\kern2pt{\textcolor{darkblue}{\textbf{\ipa{gv̩˧tsʰi˧˥}}}}\kern2pt]} \hypertarget{gv\string_=\string_Mts\string_hi\string_B1}{}
\markboth{\textcolor{darkblue}{\textbf{\ipa{gv̩˧tsʰi˩}}}}{}
\textcolor{teal}{\mytextsc{number}} \hspace{4pt} Tone: L\#.
\textcolor{Sepia}{\selectlanguage{english}90.} \zh{90。} 
\lhead{\firstmark}
\rhead{\botmark}

\subsection{\hspace{-0.5cm} {\Large \textcolor{darkblue}{\textbf{\ipa{gwɤ˩\textsubscript{a}}}}}\hspace{0.5cm}[\kern2pt{\textcolor{darkblue}{\textbf{\ipa{gwɤ˩˥}}}}\kern2pt]} \hypertarget{gw7\string_Ba1}{}
\markboth{\textcolor{darkblue}{\textbf{\ipa{gwɤ˩\textsubscript{a}}}}}{}
\textcolor{teal}{\mytextsc{verb}} \hspace{4pt} Tone: L\textsubscript{a}.
\textcolor{Sepia}{\selectlanguage{english}To sing.} \zh{唱、唱歌。}  ¶ \textcolor{darkblue}{\textbf{\ipa{njɤ˧ | ɖɯ˧-ɖʐo˩ | gwɤ˩-ze˥!}}} \textcolor{Sepia}{\selectlanguage{english}I have sung a song!} \zh{我唱了一首歌!}  
 ¶ \textcolor{darkblue}{\textbf{\ipa{no˧ | ɖɯ˧-ɖʐo˩ gwɤ˩!}}} \textcolor{Sepia}{\selectlanguage{english}Please sing a song! / Go ahead and sing us a song!} \zh{你唱一首吧!}  
 ¶ \textcolor{darkblue}{\textbf{\ipa{ɖɯ˧-kʰwɤ˧ gwɤ˥}}} \textcolor{Sepia}{\selectlanguage{english}to sing a song} \zh{唱一下}  
 ¶ \textcolor{darkblue}{\textbf{\ipa{ɖɯ˧-kʰwɤ˧ gwɤ˥-ɻ̍˩}}} \textcolor{Sepia}{\selectlanguage{english}to sing a song} \zh{唱一下}  
 ¶ \textcolor{darkblue}{\textbf{\ipa{nɑ˩-gwɤ˥}}} \textcolor{Sepia}{\selectlanguage{english}Na songs} \zh{摩梭民歌}  
 ¶ \textcolor{darkblue}{\textbf{\ipa{ʈʂʰɯ˧ | nɑ˩-gwɤ˥ F | kv̩˧˥! | hæ˧-gwɤ˩ F | kv̩˧˥! | ʁo˧dzi˩-gwɤ˩ F | kv̩˧-ʝi˥! |}}} \textcolor{Sepia}{\selectlanguage{english}He can sing (lots of different styles:) Na songs! and also Chinese (Han) songs! and also Tibetan songs!} \zh{他会唱很多种风格的歌曲:摩梭的,会唱!汉族的,会唱!藏族的,会唱!}  

\lhead{\firstmark}
\rhead{\botmark}

\subsection{\hspace{-0.5cm} {\Large \textcolor{darkblue}{\textbf{\ipa{gwɤ˩\textasciitilde{}gwɤ˧˥}}}}\hspace{0.5cm}[\kern2pt{\textcolor{darkblue}{\textbf{\ipa{gwɤ˩gwɤ˥}}}}\kern2pt]} \hypertarget{gw7\string_B~gw7\string_M\string_T1}{}
\markboth{\textcolor{darkblue}{\textbf{\ipa{gwɤ˩\textasciitilde{}gwɤ˧˥}}}}{}
\textcolor{teal}{\mytextsc{verb}} \hspace{4pt} Tone: L.
\textcolor{Sepia}{\selectlanguage{english}To stroll, to ramble, to roam.} \zh{逛,玩,游。}  ¶ \textcolor{darkblue}{\textbf{\ipa{le˧-gwɤ˩\textasciitilde{}gwɤ˩ | le˧-tsʰɯ˩-ze˩!}}} \textcolor{Sepia}{\selectlanguage{english}So you are back from a stroll! / You are back from your little walk, eh?} \zh{你已经散步回来了!}  
 ¶ \textcolor{darkblue}{\textbf{\ipa{ʈʂʰɯ˧ | gwɤ˩\textasciitilde{}gwɤ˩-hɯ˩-ze˥!}}} \textcolor{Sepia}{\selectlanguage{english}(S)he has gone out for a walk!} \zh{他散步去了!}  
 ¶ \textcolor{darkblue}{\textbf{\ipa{æ˧ʂæ˧ gwɤ˩; | qv̩˧ɻ̍˧ gwɤ˥; | nɑ˩tsʰi˩ gwɤ˥}}} \textcolor{Sepia}{\selectlanguage{english}“to walk around Mount \textcolor{darkblue}{\textbf{\ipa{/æ˧ʂæ˧/;}}} to walk around Mount \textcolor{darkblue}{\textbf{\ipa{/qv̩˧ɻ\#˥/;}}} to walk around Mount \textcolor{darkblue}{\textbf{\ipa{/nɑ˩tsʰi˩/”:}}} i.e. to do rituals on these mountains, in particular to obtain fertility, or to obtain a cure for a child who did not learn to speak.} \zh{绕\textcolor{darkblue}{\textbf{\ipa{æ˧ʂæ˧}}}山,绕\textcolor{darkblue}{\textbf{\ipa{qv̩˧ɻ\#˥}}}山,绕\textcolor{darkblue}{\textbf{\ipa{nɑ˩tsʰi˩}}}山(做“绕山”仪式,为了求生子等)}  

\lhead{\firstmark}
\rhead{\botmark}

\subsection{\hspace{-0.5cm} {\Large \textcolor{darkblue}{\textbf{\ipa{gwɤ˩ʝi˧}}}}\hspace{0.5cm}[\kern2pt{\textcolor{darkblue}{\textbf{\ipa{gwɤ˩ʝi˩˥}}}}\kern2pt]} \hypertarget{gw7\string_Bj££i\string_M1}{}
\markboth{\textcolor{darkblue}{\textbf{\ipa{gwɤ˩ʝi˧}}}}{}
\textcolor{teal}{\mytextsc{adverb(ial)}} \hspace{4pt} Tone: LM.
\textcolor{Sepia}{\selectlanguage{english}In good order.} \zh{整齐。}  ¶ \textcolor{darkblue}{\textbf{\ipa{tso˧\textasciitilde{}tso˧ | gwɤ˩ʝi˧ tʰi˧-tɕɯ˥ |}}} \textcolor{Sepia}{\selectlanguage{english}to put things in good order} \zh{把东西摆整齐}  

\lhead{\firstmark}
\rhead{\botmark}

\newpage
\section*{\centering- \textcolor{darkblue}{\textbf{\ipa{ɣ}}} -}
\subsection{\hspace{-0.5cm} {\Large \textcolor{darkblue}{\textbf{\ipa{ɣɯ˥}}}}\hspace{0.5cm}[\kern2pt{\textcolor{darkblue}{\textbf{\ipa{ɣɯ˥}}}}\kern2pt]} \hypertarget{GM\string_T1}{}
\markboth{\textcolor{darkblue}{\textbf{\ipa{ɣɯ˥}}}}{}
\textcolor{teal}{\mytextsc{adjective}} \hspace{4pt} Tone: H.
\textcolor{Sepia}{\selectlanguage{english}Competent, able.} \zh{能干、好(做事情做得好)。}  ¶ \textcolor{darkblue}{\textbf{\ipa{mɤ˧-ɣɯ˥}}} \textcolor{Sepia}{\selectlanguage{english}\mytextsc{neg}} \zh{不能干}  
 ¶ \textcolor{darkblue}{\textbf{\ipa{ʈʂʰɯ˧-ɳɯ˧, | bɑ˩lɑ˩ hwæ˧ | ɣɯ˧!}}} \textcolor{Sepia}{\selectlanguage{english}He/she is very good at buying clothes! / He/she has talent for choosing clothes!} \zh{他很会买衣服!}  

\lhead{\firstmark}
\rhead{\botmark}

\subsection{\hspace{-0.5cm} {\Large \textcolor{darkblue}{\textbf{\ipa{ɣɯ˧}}}}\hspace{0.5cm}[\kern2pt{\textcolor{darkblue}{\textbf{\ipa{ɣɯ˥}}}}\kern2pt]} \hypertarget{GM\string_M1}{}
\markboth{\textcolor{darkblue}{\textbf{\ipa{ɣɯ˧}}}}{}
\textcolor{teal}{\mytextsc{noun}} \hspace{4pt} Tone: \#H.
\textcolor{Sepia}{\selectlanguage{english}Cloth.} \zh{布料。}  ¶ \textcolor{darkblue}{\textbf{\ipa{ɣɯ˧dzo˩, | ɣɯ˧ni˧˥, | ɣɯ˧, | ɖɯ˧-ʑi˩ ɲi˩-ze˩!}}} \textcolor{Sepia}{\selectlanguage{english}The weaving-machine, the bamboo structure keeping the threads together, and fabric: these belong to the same family! / these are all part of the same sphere!} \zh{织布机、竹子的框(让线不乱混)、布料,属于同一类!(直译:“都是一家的!”)}  
 \zh{量词}: \textcolor{darkblue}{\textbf{\ipa{bo˩}}}  \mytextsc{clf}: \textcolor{darkblue}{\textbf{\ipa{bo˩}}} 
\lhead{\firstmark}
\rhead{\botmark}

\subsection{\hspace{-0.5cm} {\Large \textcolor{darkblue}{\textbf{\ipa{ɣɯ˧bo˩}}}}\hspace{0.5cm}[\kern2pt{\textcolor{darkblue}{\textbf{\ipa{ɣɯ˧bo˧˥}}}}\kern2pt]} \hypertarget{GM\string_Mbo\string_B1}{}
\markboth{\textcolor{darkblue}{\textbf{\ipa{ɣɯ˧bo˩}}}}{}
\textcolor{teal}{\mytextsc{noun}} \hspace{4pt} Tone: MH\#.
\textcolor{Sepia}{\selectlanguage{english}Weft, weft thread, pick.} \zh{纬线、纬纱。}  \zh{量词}: \textcolor{darkblue}{\textbf{\ipa{bo˩}}}  \mytextsc{clf}: \textcolor{darkblue}{\textbf{\ipa{bo˩}}} 
\lhead{\firstmark}
\rhead{\botmark}

\subsection{\hspace{-0.5cm} {\Large \textcolor{darkblue}{\textbf{\ipa{ɣɯ˧dzo˩}}}}\hspace{0.5cm}[\kern2pt{\textcolor{darkblue}{\textbf{\ipa{ɣɯ˧dzo˩}}}}\kern2pt]} \hypertarget{GM\string_Mdzo\string_B1}{}
\markboth{\textcolor{darkblue}{\textbf{\ipa{ɣɯ˧dzo˩}}}}{}
\textcolor{teal}{\mytextsc{noun}} \hspace{4pt} Tone: L\#.
\textcolor{Sepia}{\selectlanguage{english}Loom.} \zh{织布机。}  \zh{量词}: \textcolor{darkblue}{\textbf{\ipa{nɑ˧}}}  \mytextsc{clf}: \textcolor{darkblue}{\textbf{\ipa{nɑ˧}}} 
\lhead{\firstmark}
\rhead{\botmark}

\subsection{\hspace{-0.5cm} {\Large \textcolor{darkblue}{\textbf{\ipa{ɣɯ˧ni˧˥}}}}\hspace{0.5cm}[\kern2pt{\textcolor{darkblue}{\textbf{\ipa{ɣɯ˧ni˧˥}}}}\kern2pt]} \hypertarget{GM\string_Mni\string_M\string_T1}{}
\markboth{\textcolor{darkblue}{\textbf{\ipa{ɣɯ˧ni˧˥}}}}{}
\textcolor{teal}{\mytextsc{noun}} \hspace{4pt} Tone: MH\#.
\textcolor{Sepia}{\selectlanguage{english}A part of the loom: a small bamboo structure hanging from the top of the loom, keeping the threads together.} \zh{织布机的一部分:竹子的框,让线不乱混。}  \zh{量词}: \textcolor{darkblue}{\textbf{\ipa{dze˩}}}  \mytextsc{clf}: \textcolor{darkblue}{\textbf{\ipa{dze˩}}} 
\lhead{\firstmark}
\rhead{\botmark}

\subsection{\hspace{-0.5cm} {\Large \textcolor{darkblue}{\textbf{\ipa{ɣɯ˩kɯ˧˥}}}}\hspace{0.5cm}[\kern2pt{\textcolor{darkblue}{\textbf{\ipa{ɣɯ˩kɯ˧˥}}}}\kern2pt]} \hypertarget{GM\string_BkM\string_M\string_T1}{}
\markboth{\textcolor{darkblue}{\textbf{\ipa{ɣɯ˩kɯ˧˥}}}}{}
\textcolor{teal}{\mytextsc{noun}} \hspace{4pt} Tone: LM+MH\#.
\ding{202} \textcolor{Sepia}{\selectlanguage{english}Peel, rind.} \zh{皮、鸡蛋壳、麦麸。}  ¶ \textcolor{darkblue}{\textbf{\ipa{pʰi˩ko˧-ɣɯ˩kɯ˩}}} \textcolor{Sepia}{\selectlanguage{english}peel of an apple} \zh{苹果皮}  
 ¶ \textcolor{darkblue}{\textbf{\ipa{jɤ˩jo˧-ɣɯ˥kɯ˩}}} \textcolor{Sepia}{\selectlanguage{english}potato peel} \zh{洋芋皮}  
 \zh{量词}: \textcolor{darkblue}{\textbf{\ipa{kʰwɤ˥}}} \ding{203} \textcolor{Sepia}{\selectlanguage{english}Fur, pelt, skin (of animal).} \zh{皮。}  ¶ \textcolor{darkblue}{\textbf{\ipa{ʂe˧-ɣɯ˥kɯ˩}}} \textcolor{Sepia}{\selectlanguage{english}skin of meat, i.e. skin on a piece of meat} \zh{肉皮:鸡皮、猪肉的皮……}  
\ding{204} \textcolor{Sepia}{\selectlanguage{english}Eggshell.} \zh{蛋壳。} \ding{205} \textcolor{Sepia}{\selectlanguage{english}Bran.} \zh{麸。}  ¶ \textcolor{darkblue}{\textbf{\ipa{dze˧ɭɯ˧-ɣɯ˩kɯ˩}}} \textcolor{Sepia}{\selectlanguage{english}wheat bran} \zh{小麦麸}  
 \mytextsc{clf}: \textcolor{darkblue}{\textbf{\ipa{kʰwɤ˥}}} 
\lhead{\firstmark}
\rhead{\botmark}

\subsection{\hspace{-0.5cm} {\Large \textcolor{darkblue}{\textbf{\ipa{ɣɯ˩-nɑ˥mi˩}}}}\hspace{0.5cm}[\kern2pt{\textcolor{darkblue}{\textbf{\ipa{ɣɯ˩˥nɑ˧mi˧}}}}\kern2pt]} \hypertarget{GM\string_B-nA\string_Tmi\string_B1}{}
\markboth{\textcolor{darkblue}{\textbf{\ipa{ɣɯ˩-nɑ˥mi˩}}}}{}
\textcolor{teal}{\mytextsc{noun}} \hspace{4pt} Tone: LH-.
\textcolor{Sepia}{\selectlanguage{english}Yi (derogatory term).} \zh{彝族(带偏见的说法)。}  ¶ \textcolor{darkblue}{\textbf{\ipa{ɣɯ˩-nɑ˥mi˩-zo˩}}} \textcolor{Sepia}{\selectlanguage{english}Yi man} \zh{彝族男人}  
 ¶ \textcolor{darkblue}{\textbf{\ipa{ɣɯ˩-nɑ˥mi˩-mv̩˩}}} \textcolor{Sepia}{\selectlanguage{english}Yi woman} \zh{彝族女人}  
 \zh{量词}: \textcolor{darkblue}{\textbf{\ipa{v̩˧}}}  \mytextsc{clf}: \textcolor{darkblue}{\textbf{\ipa{v̩˧}}} 
\lhead{\firstmark}
\rhead{\botmark}

\subsection{\hspace{-0.5cm} {\Large \textcolor{darkblue}{\textbf{\ipa{ɣɯ˩˥}}}}\hspace{0.5cm}[\kern2pt{\textcolor{darkblue}{\textbf{\ipa{ɣɯ˩˥}}}}\kern2pt]} \hypertarget{GM\string_B\string_T1}{}
\markboth{\textcolor{darkblue}{\textbf{\ipa{ɣɯ˩˥}}}}{}
\textcolor{teal}{\mytextsc{noun}} \hspace{4pt} Tone: LH.
\textcolor{Sepia}{\selectlanguage{english}Skin.} \zh{皮肤。}  ¶ \textcolor{darkblue}{\textbf{\ipa{ɣɯ˩ dzɯ˩˥}}} \textcolor{Sepia}{\selectlanguage{english}to eat skin} \zh{吃皮}  
 ¶ \textcolor{darkblue}{\textbf{\ipa{ɣɯ˩˥ | ɖɯ˧-ʂɯ˩ pʰv˩}}} \textcolor{Sepia}{\selectlanguage{english}literally 'to shed one's skin once'; meaning: to be worn out and physically hurt (by an exhausting task, such as felling trees high up on the mountains and carrying lumber back to the plain)} \zh{直译:‘脱皮一次’。意思:疲劳而受伤(因为做了很辛苦的工作,如:在深山老林砍树、扛树干回到坝子)}  
 \zh{量词}: \textcolor{darkblue}{\textbf{\ipa{tsʰi˥}}}  \mytextsc{clf}: \textcolor{darkblue}{\textbf{\ipa{tsʰi˥}}} 
\lhead{\firstmark}
\rhead{\botmark}

\newpage
\section*{\centering- \textcolor{darkblue}{\textbf{\ipa{h}}} -}
\subsection{\hspace{-0.5cm} {\Large \textcolor{darkblue}{\textbf{\ipa{hɑ˥}}}}\hspace{0.5cm}[\kern2pt{\textcolor{darkblue}{\textbf{\ipa{hɑ˧˥}}}}\kern2pt]} \hypertarget{hA\string_T1}{}
\markboth{\textcolor{darkblue}{\textbf{\ipa{hɑ˥}}}}{}
\textcolor{teal}{\mytextsc{noun}} \hspace{4pt} Tone: \#H.
\textcolor{Sepia}{\selectlanguage{english}Food.} \zh{饭,米饭。}  ¶ \textcolor{darkblue}{\textbf{\ipa{hɑ˧-ʈv̩˧\textasciitilde{}ʈv̩˥}}} \textcolor{Sepia}{\selectlanguage{english}ball of cereals} \zh{饭坨坨、饭团}  
 ¶ \textcolor{darkblue}{\textbf{\ipa{hɑ˧ dzɯ˧}}} \textcolor{Sepia}{\selectlanguage{english}to eat} \zh{吃饭}  
 ¶ \textcolor{darkblue}{\textbf{\ipa{ʈʂʰɯ˧ | hɑ˧ dzɯ˧-dʑo˩!}}} \textcolor{Sepia}{\selectlanguage{english}(S)he is eating!} \zh{他在吃饭!}  
 ¶ \textcolor{darkblue}{\textbf{\ipa{hɑ˧ʂɯ˩}}} \textcolor{Sepia}{\selectlanguage{english}fresh cereals (freshly reaped; they yield especially good-tasting cakes)} \zh{新鲜的粮食(可以用来烤很香的饼)}  

\lhead{\firstmark}
\rhead{\botmark}

\subsection{\hspace{-0.5cm} {\Large \textcolor{darkblue}{\textbf{\ipa{hɑ˧bɤ˥}}}}\hspace{0.5cm}[\kern2pt{\textcolor{darkblue}{\textbf{\ipa{xxxx non-correspondance entre le nombre de morphèmes et le nombre de tons de morphèmes}}}}\kern2pt]} \hypertarget{hA\string_Mb7\string_T1}{}
\markboth{\textcolor{darkblue}{\textbf{\ipa{hɑ˧bɤ˥}}}}{}
\textcolor{teal}{\mytextsc{noun}} \hspace{4pt} Tone: H\#.
\textcolor{Sepia}{\selectlanguage{english}Corncob.} \zh{玉米棒子。}  ¶ \textcolor{darkblue}{\textbf{\ipa{qʰɑ˧dze˧-hɑ˧bɤ˥}}} \textcolor{Sepia}{\selectlanguage{english}sweetcorn ear} \zh{玉米棒子}  
 \zh{量词}: \textcolor{darkblue}{\textbf{\ipa{bɤ˩}}}  \mytextsc{clf}: \textcolor{darkblue}{\textbf{\ipa{bɤ˩}}} 
\lhead{\firstmark}
\rhead{\botmark}

\subsection{\hspace{-0.5cm} {\Large \textcolor{darkblue}{\textbf{\ipa{hɑ˧-bv̩˥\textasciitilde{}bv̩˩-di˩}}}}\hspace{0.5cm}[\kern2pt{\textcolor{darkblue}{\textbf{\ipa{xxxx non-correspondance entre le nombre de morphèmes et le nombre de tons de morphèmes}}}}\kern2pt]} \hypertarget{hA\string_M-bv\string_=\string_T~bv\string_=\string_B-di\string_B1}{}
\markboth{\textcolor{darkblue}{\textbf{\ipa{hɑ˧-bv̩˥\textasciitilde{}bv̩˩-di˩}}}}{}
\textcolor{teal}{\mytextsc{noun}} \hspace{4pt} Tone: \#H--.
\textcolor{Sepia}{\selectlanguage{english}Rice steamer.} \zh{甑。}  \zh{量词}: \textcolor{darkblue}{\textbf{\ipa{ɭɯ˧}}}  \mytextsc{clf}: \textcolor{darkblue}{\textbf{\ipa{ɭɯ˧}}} 
\lhead{\firstmark}
\rhead{\botmark}

\subsection{\hspace{-0.5cm} {\Large \textcolor{darkblue}{\textbf{\ipa{hɑ˧-gv̩˥-di˩}}}}\hspace{0.5cm}[\kern2pt{\textcolor{darkblue}{\textbf{\ipa{xxxx non-correspondance entre le nombre de morphèmes et le nombre de tons de morphèmes}}}}\kern2pt]} \hypertarget{hA\string_M-gv\string_=\string_T-di\string_B1}{}
\markboth{\textcolor{darkblue}{\textbf{\ipa{hɑ˧-gv̩˥-di˩}}}}{}
\textcolor{teal}{\mytextsc{noun}} \hspace{4pt} Tone: H\#-.
\textcolor{Sepia}{\selectlanguage{english}Stove.} \zh{炉子、灶头。} 
\lhead{\firstmark}
\rhead{\botmark}

\subsection{\hspace{-0.5cm} {\Large \textcolor{darkblue}{\textbf{\ipa{hɑ˧ɭɯ\#˥}}}}\hspace{0.5cm}[\kern2pt{\textcolor{darkblue}{\textbf{\ipa{hɑ˧ɭɯ˧}}}}\kern2pt]} \hypertarget{hA\string_Ml\string_RM\#\string_T1}{}
\markboth{\textcolor{darkblue}{\textbf{\ipa{hɑ˧ɭɯ\#˥}}}}{}
\textcolor{teal}{\mytextsc{noun}} \hspace{4pt} Tone: \#H.
\textcolor{Sepia}{\selectlanguage{english}Cereals.} \zh{粮食。} 
\lhead{\firstmark}
\rhead{\botmark}

\subsection{\hspace{-0.5cm} {\Large \textcolor{darkblue}{\textbf{\ipa{hɑ˧mi˥}}}}\hspace{0.5cm}[\kern2pt{\textcolor{darkblue}{\textbf{\ipa{hɑ˧mi˥}}}}\kern2pt]} \hypertarget{hA\string_Mmi\string_T1}{}
\markboth{\textcolor{darkblue}{\textbf{\ipa{hɑ˧mi˥}}}}{}
\textcolor{teal}{\mytextsc{verb}} \hspace{4pt} Tone: H\#.
\textcolor{Sepia}{\selectlanguage{english}To beg.} \zh{讨饭。}  ¶ \textcolor{darkblue}{\textbf{\ipa{hɑ˧mi˥-hĩ˩}}} \textcolor{Sepia}{\selectlanguage{english}\string_ \mytextsc{rel}: beggar, [person] who begs} \zh{要饭的、乞丐}  
\textit{See:} \hyperlink{}{\textcolor{darkblue}{\textbf{\ipa{mi˩\textsubscript{a}}}}} 
\lhead{\firstmark}
\rhead{\botmark}

\subsection{\hspace{-0.5cm} {\Large \textcolor{darkblue}{\textbf{\ipa{hɑ˧pv̩˩}}}}\hspace{0.5cm}[\kern2pt{\textcolor{darkblue}{\textbf{\ipa{hɑ˧pv̩˩}}}}\kern2pt]} \hypertarget{hA\string_Mpv\string_=\string_B1}{}
\markboth{\textcolor{darkblue}{\textbf{\ipa{hɑ˧pv̩˩}}}}{}
\textcolor{teal}{\mytextsc{noun}} \hspace{4pt} Tone: L\#.
\textcolor{Sepia}{\selectlanguage{english}'dry' cooked rice: the type of rice usually served at meals, as distinct from watery rice gruel.} \zh{干的米饭(与稀饭不同)。} 
\lhead{\firstmark}
\rhead{\botmark}

\subsection{\hspace{-0.5cm} {\Large \textcolor{darkblue}{\textbf{\ipa{hɑ˧ʂɯ˥}}}}\hspace{0.5cm}[\kern2pt{\textcolor{darkblue}{\textbf{\ipa{hɑ˧ʂɯ˥}}}}\kern2pt]} \hypertarget{hA\string_Ms`M\string_T1}{}
\markboth{\textcolor{darkblue}{\textbf{\ipa{hɑ˧ʂɯ˥}}}}{}
\textcolor{teal}{\mytextsc{linker}} \hspace{4pt} Tone: H\#.
\textcolor{Sepia}{\selectlanguage{english}Gap-filler, borrowed from the Chinese: “still/also...”.} \zh{还是(汉语借词)。}  Borrowing: Chinese  \zh{还是}

\lhead{\firstmark}
\rhead{\botmark}

\subsection{\hspace{-0.5cm} {\Large \textcolor{darkblue}{\textbf{\ipa{hɑ˧-ʐwɤ˩}}}}\hspace{0.5cm}[\kern2pt{\textcolor{darkblue}{\textbf{\ipa{xxxx non-correspondance entre le nombre de morphèmes et le nombre de tons de morphèmes}}}}\kern2pt]} \hypertarget{hA\string_M-z`w7\string_B1}{}
\markboth{\textcolor{darkblue}{\textbf{\ipa{hɑ˧-ʐwɤ˩}}}}{}
\textcolor{teal}{\mytextsc{adjective}} \hspace{4pt} Tone: L\#.
\textcolor{Sepia}{\selectlanguage{english}Hungry.} \zh{饿(饭)。} \textit{See:} \hyperlink{}{\textcolor{darkblue}{\textbf{\ipa{ʐwɤ˧}}}} 
\lhead{\firstmark}
\rhead{\botmark}

\subsection{\hspace{-0.5cm} {\Large \textcolor{darkblue}{\textbf{\ipa{hɑ˩\textsubscript{a}}}}}\hspace{0.5cm}[\kern2pt{\textcolor{darkblue}{\textbf{\ipa{hɑ˧˥}}}}\kern2pt]} \hypertarget{hA\string_Ba1}{}
\markboth{\textcolor{darkblue}{\textbf{\ipa{hɑ˩\textsubscript{a}}}}}{}
\textcolor{teal}{\mytextsc{verb}} \hspace{4pt} Tone: L\textsubscript{a}.
\textcolor{Sepia}{\selectlanguage{english}To open (one's eyes).} \zh{睁开(眼睛)。}  ¶ \textcolor{darkblue}{\textbf{\ipa{tʰi˧-hɑ˩}}} \textcolor{Sepia}{\selectlanguage{english}\mytextsc{dur}} \zh{\mytextsc{dur}}  
 ¶ \textcolor{darkblue}{\textbf{\ipa{njɤ˩ɭɯ˥ | gɤ˩-hɑ˥ |}}} \textcolor{Sepia}{\selectlanguage{english}to open one's eyes} \zh{睁开眼睛}  
 ¶ \textcolor{darkblue}{\textbf{\ipa{njɤ˩ɭɯ˧ hɑ˩}}} \textcolor{Sepia}{\selectlanguage{english}to open one's eyes} \zh{睁开眼睛}  

\lhead{\firstmark}
\rhead{\botmark}

\subsection{\hspace{-0.5cm} {\Large \textcolor{darkblue}{\textbf{\ipa{hɑ̃˧mo˥}}}}\hspace{0.5cm}[\kern2pt{\textcolor{darkblue}{\textbf{\ipa{hɑ̃˧mo˥}}}}\kern2pt]} \hypertarget{hA\string_~\string_Mmo\string_T1}{}
\markboth{\textcolor{darkblue}{\textbf{\ipa{hɑ̃˧mo˥}}}}{}
\textcolor{teal}{\mytextsc{adjective}} \hspace{4pt} Tone: H\#.
\textcolor{Sepia}{\selectlanguage{english}Old (person).} \zh{年老。}  ¶ \textcolor{darkblue}{\textbf{\ipa{hĩ˧ ʈʂʰɯ˧-v̩˧ | hɑ̃˧mo˥ | ʐwæ˩˥!}}} \textcolor{Sepia}{\selectlanguage{english}This person is extremely old/extremely advanced in years!} \zh{这个人,年纪非常大!}  

\lhead{\firstmark}
\rhead{\botmark}

\subsection{\hspace{-0.5cm} {\Large \textcolor{darkblue}{\textbf{\ipa{hɑ̃˧˥}}} \textsubscript{1}}\hspace{0.5cm}[\kern2pt{\textcolor{darkblue}{\textbf{\ipa{hɑ̃˧˥}}}}\kern2pt]} \hypertarget{hA\string_~\string_M\string_T1}{}
\markboth{\textcolor{darkblue}{\textbf{\ipa{hɑ̃˧˥}}} \textsubscript{1}}{}
\textcolor{teal}{\mytextsc{verb}} \hspace{4pt} Tone: MH.
\textcolor{Sepia}{\selectlanguage{english}To put out, to extinguish (e.g. the fire of the stove).} \zh{把火炉灭了。}  ¶ \textcolor{darkblue}{\textbf{\ipa{mv̩˧ le˧-hɑ̃˧˥}}} \textcolor{Sepia}{\selectlanguage{english}to put out the fire} \zh{灭火}  

\lhead{\firstmark}
\rhead{\botmark}

\subsection{\hspace{-0.5cm} {\Large \textcolor{darkblue}{\textbf{\ipa{hɑ̃˧˥}}} \textsubscript{2}}\hspace{0.5cm}[\kern2pt{\textcolor{darkblue}{\textbf{\ipa{hɑ̃˧˥}}}}\kern2pt]} \hypertarget{hA\string_~\string_M\string_T2}{}
\markboth{\textcolor{darkblue}{\textbf{\ipa{hɑ̃˧˥}}} \textsubscript{2}}{}
\textcolor{teal}{\mytextsc{verb}} \hspace{4pt} Tone: MH.
\textcolor{Sepia}{\selectlanguage{english}To spend the night (at a certain place).} \zh{过夜。}  ¶ \textcolor{darkblue}{\textbf{\ipa{ɖɯ˧-hɑ̃˧ tʰi˥-hɑ̃˩ |}}} \textcolor{Sepia}{\selectlanguage{english}to spend a night (somewhere), to stay for the night} \zh{过夜}  
 ¶ \textcolor{darkblue}{\textbf{\ipa{ʑi˧qʰwɤ˧ ɖɯ˧-ɭɯ˧-qo˧ hɑ̃˧˥}}} \textcolor{Sepia}{\selectlanguage{english}to spend the night in a house} \zh{在一个人家过夜}  

\lhead{\firstmark}
\rhead{\botmark}

\subsection{\hspace{-0.5cm} {\Large \textcolor{darkblue}{\textbf{\ipa{hɑ̃˧˥\textsubscript{a}}}}}\hspace{0.5cm}[\kern2pt{\textcolor{darkblue}{\textbf{\ipa{hɑ̃˩˥}}}}\kern2pt]} \hypertarget{hA\string_~\string_M\string_Ta1}{}
\markboth{\textcolor{darkblue}{\textbf{\ipa{hɑ̃˧˥\textsubscript{a}}}}}{}
\textcolor{teal}{\mytextsc{classifier}} \hspace{4pt} Tone: MH\textsubscript{a}.
\textcolor{Sepia}{\selectlanguage{english}Classifier for nights.} \zh{量词:夜。}  ¶ \textcolor{darkblue}{\textbf{\ipa{ɖɯ˧-hɑ̃˧˥}}} \textcolor{Sepia}{\selectlanguage{english}one night} \zh{一夜}  
 ¶ \textcolor{darkblue}{\textbf{\ipa{tsʰe˩-hɑ̃˩˥}}} \textcolor{Sepia}{\selectlanguage{english}ten nights = ten days} \zh{十夜(等于十天)}  
 ¶ \textcolor{darkblue}{\textbf{\ipa{ɖɯ˧-hɑ̃˧ lɑ˥-dʑo˩!}}} \textcolor{Sepia}{\selectlanguage{english}There is only one day left!} \zh{只有一个晚上了!}  
 ¶ \textcolor{darkblue}{\textbf{\ipa{[F5] ɖɯ˧-hɑ̃˧-ɳɯ˥ | le˧-li˧-le˧-se˩-ze˩!}}} \textcolor{Sepia}{\selectlanguage{english}He has entirely read it in one night! / He has read the whole (book) in just one night! (Imagined context: someone is given a book; he finishes reading it within a day)} \zh{一个晚上就读完了!/一天之内都读完了!(情景:送一个人一本书,他马上全部读完)}  

\lhead{\firstmark}
\rhead{\botmark}

\subsection{\hspace{-0.5cm} {\Large \textcolor{darkblue}{\textbf{\ipa{hæ˧}}}}\hspace{0.5cm}[\kern2pt{\textcolor{darkblue}{\textbf{\ipa{hæ˧˥}}}}\kern2pt]} \hypertarget{h\{\string_M1}{}
\markboth{\textcolor{darkblue}{\textbf{\ipa{hæ˧}}}}{}
\textcolor{teal}{\mytextsc{noun}} \hspace{4pt} Tone: M.
\textcolor{Sepia}{\selectlanguage{english}Chinese (Han).} \zh{汉人。}  ¶ \textcolor{darkblue}{\textbf{\ipa{hæ˧-mi\#˥}}} \textcolor{Sepia}{\selectlanguage{english}a Chinese woman, a Han Chinese woman} \zh{汉族女人}  
 ¶ \textcolor{darkblue}{\textbf{\ipa{hæ˧-mv̩˧ hæ˧-di˧˥}}} \textcolor{Sepia}{\selectlanguage{english}(Han) Chinese territory: Chengdu, Kunming...} \zh{汉族地区,包括成都、昆明等等}  
 ¶ \textcolor{darkblue}{\textbf{\ipa{hæ˧-di˩}}} \textcolor{Sepia}{\selectlanguage{english}(Han) Chinese territory: Chengdu, Kunming...; used to mean 'the south'} \zh{汉族地区,包括成都、昆明等等,来代指南方}  
 ¶ \textcolor{darkblue}{\textbf{\ipa{hæ˧-zo˧bæ˩}}} \textcolor{Sepia}{\selectlanguage{english}Han Chinese man (derogatory: literally 'Chinese idiot')} \zh{汉男人(带偏见的称呼)}  
 \zh{量词}: \textcolor{darkblue}{\textbf{\ipa{v̩˧}}}  \mytextsc{clf}: \textcolor{darkblue}{\textbf{\ipa{v̩˧}}} 
\lhead{\firstmark}
\rhead{\botmark}

\subsection{\hspace{-0.5cm} {\Large \textcolor{darkblue}{\textbf{\ipa{hæ˧di˩-ʈæ˩bɤ˩}}}}\hspace{0.5cm}[\kern2pt{\textcolor{darkblue}{\textbf{\ipa{xxxx non-correspondance entre le nombre de morphèmes et le nombre de tons de morphèmes}}}}\kern2pt]} \hypertarget{h\{\string_Mdi\string_B-t`\{\string_Bb7\string_B1}{}
\markboth{\textcolor{darkblue}{\textbf{\ipa{hæ˧di˩-ʈæ˩bɤ˩}}}}{}
\textcolor{teal}{\mytextsc{noun}} \hspace{4pt} Tone: \mytextsc{L}.
\textcolor{Sepia}{\selectlanguage{english}Beggar-monk (of the Buddhist religion).} \zh{比丘、游僧。} 
\lhead{\firstmark}
\rhead{\botmark}

\subsection{\hspace{-0.5cm} {\Large \textcolor{darkblue}{\textbf{\ipa{hæ˧ɭɯ\#˥}}}}\hspace{0.5cm}[\kern2pt{\textcolor{darkblue}{\textbf{\ipa{hæ˧ɭɯ˧}}}}\kern2pt]} \hypertarget{h\{\string_Ml\string_RM\#\string_T1}{}
\markboth{\textcolor{darkblue}{\textbf{\ipa{hæ˧ɭɯ\#˥}}}}{}
\textcolor{teal}{\mytextsc{noun}} \hspace{4pt} Tone: \#H.
\textcolor{Sepia}{\selectlanguage{english}Chinese sorghum.} \zh{高粱。} \textit{See:} \hyperlink{}{\textcolor{darkblue}{\textbf{\ipa{kɤ˧ljɤ˩}}}} 
\lhead{\firstmark}
\rhead{\botmark}

\subsection{\hspace{-0.5cm} {\Large \textcolor{darkblue}{\textbf{\ipa{hæ˧se˧}}}}\hspace{0.5cm}[\kern2pt{\textcolor{darkblue}{\textbf{\ipa{hæ˧se˧}}}}\kern2pt]} \hypertarget{h\{\string_Mse\string_M1}{}
\markboth{\textcolor{darkblue}{\textbf{\ipa{hæ˧se˧}}}}{}
\textcolor{teal}{\mytextsc{noun}} \hspace{4pt} Tone: M.
\textcolor{Sepia}{\selectlanguage{english}Agar.} \zh{石花菜、海参。}  Borrowing: Chinese  dialect \zh{海参}

\lhead{\firstmark}
\rhead{\botmark}

\subsection{\hspace{-0.5cm} {\Large \textcolor{darkblue}{\textbf{\ipa{hæ˧ʐwɤ˩}}}}\hspace{0.5cm}[\kern2pt{\textcolor{darkblue}{\textbf{\ipa{hæ˧ʐwɤ˩}}}}\kern2pt]} \hypertarget{h\{\string_Mz`w7\string_B1}{}
\markboth{\textcolor{darkblue}{\textbf{\ipa{hæ˧ʐwɤ˩}}}}{}
\textcolor{teal}{\mytextsc{noun}} \hspace{4pt} Tone: L\#.
\textcolor{Sepia}{\selectlanguage{english}The Chinese language.} \zh{汉语。} 
\lhead{\firstmark}
\rhead{\botmark}

\subsection{\hspace{-0.5cm} {\Large \textcolor{darkblue}{\textbf{\ipa{hæ˩\textsubscript{a}}}}}\hspace{0.5cm}[\kern2pt{\textcolor{darkblue}{\textbf{\ipa{hæ˥}}}}\kern2pt]} \hypertarget{h\{\string_Ba1}{}
\markboth{\textcolor{darkblue}{\textbf{\ipa{hæ˩\textsubscript{a}}}}}{}
\textcolor{teal}{\mytextsc{verb}} \hspace{4pt} Tone: L\textsubscript{a}.
\textcolor{Sepia}{\selectlanguage{english}To harm, to cause trouble.} \zh{祸害、害。}  ¶ \textcolor{darkblue}{\textbf{\ipa{hĩ˧ hæ˥}}} \textcolor{Sepia}{\selectlanguage{english}to harm people} \zh{害人}  
 ¶ \textcolor{darkblue}{\textbf{\ipa{hĩ˧ hæ˥-kv̩˩}}} \textcolor{Sepia}{\selectlanguage{english}who can harm people; cruel} \zh{会害人的、残忍、凶狠}  
 ¶ \textcolor{darkblue}{\textbf{\ipa{hĩ˧ hæ˥-zo˩}}} \textcolor{Sepia}{\selectlanguage{english}terrifying person, frightening person, creepy person} \zh{可怕的人}  

\lhead{\firstmark}
\rhead{\botmark}

\subsection{\hspace{-0.5cm} {\Large \textcolor{darkblue}{\textbf{\ipa{hæ˧˥}}} \textsubscript{1}}\hspace{0.5cm}[\kern2pt{\textcolor{darkblue}{\textbf{\ipa{hæ˥}}}}\kern2pt]} \hypertarget{h\{\string_M\string_T1}{}
\markboth{\textcolor{darkblue}{\textbf{\ipa{hæ˧˥}}} \textsubscript{1}}{}
\textcolor{teal}{\mytextsc{adjective}} \hspace{4pt} Tone: MH.
\ding{202} \textcolor{Sepia}{\selectlanguage{english}Supple, lithe.} \zh{软、柔软(树枝……)。}  ¶ \textcolor{darkblue}{\textbf{\ipa{hæ˧njæ˧˥ | -gv̩˩}}} \textcolor{Sepia}{\selectlanguage{english}soft, lithe, supple} \zh{软、柔软(树枝……)}  
\ding{203} \textcolor{Sepia}{\selectlanguage{english}Thin, watery (soup, gruel).} \zh{稀(粥、汤)。} 
\lhead{\firstmark}
\rhead{\botmark}

\subsection{\hspace{-0.5cm} {\Large \textcolor{darkblue}{\textbf{\ipa{hæ˧˥}}} \textsubscript{2}}\hspace{0.5cm}[\kern2pt{\textcolor{darkblue}{\textbf{\ipa{hæ˧˥}}}}\kern2pt]} \hypertarget{h\{\string_M\string_T2}{}
\markboth{\textcolor{darkblue}{\textbf{\ipa{hæ˧˥}}} \textsubscript{2}}{}
\textcolor{teal}{\mytextsc{noun}} \hspace{4pt} Tone: MH.
\textcolor{Sepia}{\selectlanguage{english}Lime.} \zh{石灰。}  ¶ \textcolor{darkblue}{\textbf{\ipa{hæ˧ hwæ˥}}} \textcolor{Sepia}{\selectlanguage{english}to buy lime} \zh{买石灰}  
 ¶ \textcolor{darkblue}{\textbf{\ipa{hæ˧ tɕʰi˥}}} \textcolor{Sepia}{\selectlanguage{english}to sell lime} \zh{卖石灰}  
 ¶ \textcolor{darkblue}{\textbf{\ipa{hæ˧ ki˥}}} \textcolor{Sepia}{\selectlanguage{english}to give lime} \zh{给石灰}  
 ¶ \textcolor{darkblue}{\textbf{\ipa{hæ˧ dv̩˥}}} \textcolor{Sepia}{\selectlanguage{english}to dig lime} \zh{挖石灰}  
 ¶ \textcolor{darkblue}{\textbf{\ipa{hæ˧ bæ˥}}} \textcolor{Sepia}{\selectlanguage{english}to sweep lime} \zh{扫石灰}  
 ¶ \textcolor{darkblue}{\textbf{\ipa{hæ˧ gɤ˥}}} \textcolor{Sepia}{\selectlanguage{english}to carry lime} \zh{扛石灰}  

\lhead{\firstmark}
\rhead{\botmark}

\subsection{\hspace{-0.5cm} {\Large \textcolor{darkblue}{\textbf{\ipa{hæ̃˧}}}}\hspace{0.5cm}[\kern2pt{\textcolor{darkblue}{\textbf{\ipa{hæ̃˥}}}}\kern2pt]} \hypertarget{h\{\string_~\string_M1}{}
\markboth{\textcolor{darkblue}{\textbf{\ipa{hæ̃˧}}}}{}
\textcolor{teal}{\mytextsc{noun}} \hspace{4pt} Tone: M.
\textcolor{Sepia}{\selectlanguage{english}Wind.} \zh{风。}  ¶ \textcolor{darkblue}{\textbf{\ipa{hæ̃˧ tʰv̩˧ / hæ̃˧ tʰv̩˧-ze˧}}} \textcolor{Sepia}{\selectlanguage{english}the wind has risen, the wind is blowing} \zh{刮风了}  
 ¶ \textcolor{darkblue}{\textbf{\ipa{wɤ˩˥ | hæ̃˧ tʰv̩˧-ho˩-ze˩!}}} \textcolor{Sepia}{\selectlanguage{english}There's going to be some wind again!} \zh{风又要刮起来了!}  
 \zh{量词}: \textcolor{darkblue}{\textbf{\ipa{kʰwɤ˥}}}  \mytextsc{clf}: \textcolor{darkblue}{\textbf{\ipa{kʰwɤ˥}}} 
\lhead{\firstmark}
\rhead{\botmark}

\subsection{\hspace{-0.5cm} {\Large \textcolor{darkblue}{\textbf{\ipa{hæ̃˧\textsubscript{a}}}}}\hspace{0.5cm}[\kern2pt{\textcolor{darkblue}{\textbf{\ipa{hæ̃˥}}}}\kern2pt]} \hypertarget{h\{\string_~\string_Ma1}{}
\markboth{\textcolor{darkblue}{\textbf{\ipa{hæ̃˧\textsubscript{a}}}}}{}
\textcolor{teal}{\mytextsc{verb}} \hspace{4pt} Tone: M\textsubscript{a}.
\textcolor{Sepia}{\selectlanguage{english}To use the wind to winnow.} \zh{扬(粮食)。}  ¶ \textcolor{darkblue}{\textbf{\ipa{hɑ˧ hæ̃˩}}} \textcolor{Sepia}{\selectlanguage{english}to winnow cereals} \zh{扬粮食}  
 ¶ \textcolor{darkblue}{\textbf{\ipa{tso˧\textasciitilde{}tso˧ hæ̃˩}}} \textcolor{Sepia}{\selectlanguage{english}to winnow things} \zh{扬东西}  

\lhead{\firstmark}
\rhead{\botmark}

\subsection{\hspace{-0.5cm} {\Large \textcolor{darkblue}{\textbf{\ipa{hæ̃˧do˧}}}}\hspace{0.5cm}[\kern2pt{\textcolor{darkblue}{\textbf{\ipa{hæ̃˧do˧}}}}\kern2pt]} \hypertarget{h\{\string_~\string_Mdo\string_M1}{}
\markboth{\textcolor{darkblue}{\textbf{\ipa{hæ̃˧do˧}}}}{}
\textcolor{teal}{\mytextsc{noun}} \hspace{4pt} Tone: M.
\textcolor{Sepia}{\selectlanguage{english}Threshing ground.} \zh{打场。}  ¶ \textcolor{darkblue}{\textbf{\ipa{hæ̃˧do˧ bæ˩}}} \textcolor{Sepia}{\selectlanguage{english}to sweep the threshing ground} \zh{清扫打场}  
 \zh{量词}: \textcolor{darkblue}{\textbf{\ipa{ɭɯ˧}}}  \mytextsc{clf}: \textcolor{darkblue}{\textbf{\ipa{ɭɯ˧}}} 
\lhead{\firstmark}
\rhead{\botmark}

\subsection{\hspace{-0.5cm} {\Large \textcolor{darkblue}{\textbf{\ipa{hæ̃˧kʰɤ˧˥}}}}\hspace{0.5cm}[\kern2pt{\textcolor{darkblue}{\textbf{\ipa{hæ̃˧kʰɤ˧˥}}}}\kern2pt]} \hypertarget{h\{\string_~\string_Mk\string_h7\string_M\string_T1}{}
\markboth{\textcolor{darkblue}{\textbf{\ipa{hæ̃˧kʰɤ˧˥}}}}{}
\textcolor{teal}{\mytextsc{noun}} \hspace{4pt} Tone: MH\#.
\textcolor{Sepia}{\selectlanguage{english}Rafter; beam.} \zh{椽子。}  \zh{量词}: \textcolor{darkblue}{\textbf{\ipa{ɭɯ˧}}}  \mytextsc{clf}: \textcolor{darkblue}{\textbf{\ipa{ɭɯ˧}}} 
\lhead{\firstmark}
\rhead{\botmark}

\subsection{\hspace{-0.5cm} {\Large \textcolor{darkblue}{\textbf{\ipa{hæ̃˧kʰo˧}}}}\hspace{0.5cm}[\kern2pt{\textcolor{darkblue}{\textbf{\ipa{hæ̃˧kʰo˧}}}}\kern2pt]} \hypertarget{h\{\string_~\string_Mk\string_ho\string_M1}{}
\markboth{\textcolor{darkblue}{\textbf{\ipa{hæ̃˧kʰo˧}}}}{}
\textcolor{teal}{\mytextsc{noun}} \hspace{4pt} Tone: M.
\textcolor{Sepia}{\selectlanguage{english}Princess, young lady of the nobility.} \zh{小姐、公主。}  ¶ \textcolor{darkblue}{\textbf{\ipa{hæ̃˧kʰo˧-mi˧}}} \textcolor{Sepia}{\selectlanguage{english}same meaning: young lady} \zh{同上:小姐、公主}  
 \zh{量词}: \textcolor{darkblue}{\textbf{\ipa{v̩˧}}}  \mytextsc{clf}: \textcolor{darkblue}{\textbf{\ipa{v̩˧}}} 
\lhead{\firstmark}
\rhead{\botmark}

\subsection{\hspace{-0.5cm} {\Large \textcolor{darkblue}{\textbf{\ipa{hæ̃˧pɤ˧}}}}\hspace{0.5cm}[\kern2pt{\textcolor{darkblue}{\textbf{\ipa{hæ̃˧pɤ˧}}}}\kern2pt]} \hypertarget{h\{\string_~\string_Mp7\string_M1}{}
\markboth{\textcolor{darkblue}{\textbf{\ipa{hæ̃˧pɤ˧}}}}{}
\textcolor{teal}{\mytextsc{noun}} \hspace{4pt} Tone: M.
\textcolor{Sepia}{\selectlanguage{english}Plait; braid.} \zh{辫子。}  \zh{量词}: \textcolor{darkblue}{\textbf{\ipa{kʰɯ˩}}}  \mytextsc{clf}: \textcolor{darkblue}{\textbf{\ipa{kʰɯ˩}}} 
\lhead{\firstmark}
\rhead{\botmark}

\subsection{\hspace{-0.5cm} {\Large \textcolor{darkblue}{\textbf{\ipa{hæ̃˧qʰv̩˥\$}}}}\hspace{0.5cm}[\kern2pt{\textcolor{darkblue}{\textbf{\ipa{hæ̃˧qʰv̩˥}}}}\kern2pt]} \hypertarget{h\{\string_~\string_Mq\string_hv\string_=\string_T\$1}{}
\markboth{\textcolor{darkblue}{\textbf{\ipa{hæ̃˧qʰv̩˥\$}}}}{}
\textcolor{teal}{\mytextsc{adverb(ial)}} \hspace{4pt} Tone: H\$.
\textcolor{Sepia}{\selectlanguage{english}Late at night, in the middle of night.} \zh{半夜。} 
\lhead{\firstmark}
\rhead{\botmark}

\subsection{\hspace{-0.5cm} {\Large \textcolor{darkblue}{\textbf{\ipa{hæ̃˧ʂɯ˩-}}}}\hspace{0.5cm}[\kern2pt{\textcolor{darkblue}{\textbf{\ipa{hæ̃˧ʂɯ˩}}}}\kern2pt]} \hypertarget{h\{\string_~\string_Ms`M\string_B-1}{}
\markboth{\textcolor{darkblue}{\textbf{\ipa{hæ̃˧ʂɯ˩-}}}}{}
\textcolor{teal}{\mytextsc{noun}} \hspace{4pt} Tone: L\#.
\textcolor{Sepia}{\selectlanguage{english}'precious': a prefix added to certain nouns to coin a prestige term. This prefix is not currently productive: it cannot be added to terms such as 'mother', 'house'...}  ¶ \textcolor{darkblue}{\textbf{\ipa{hæ̃˧ʂɯ˩-to˩mi˩}}} \textcolor{Sepia}{\selectlanguage{english}the Precious Pillars, the Golden Pillars: a solemn designation for the two pillars of the main building} \zh{‘黄金柱’、‘宝贵柱’:对主屋两个柱子的庄严称呼}  
 \zh{量词}: \textcolor{darkblue}{\textbf{\ipa{nɑ˧}}}  \mytextsc{clf}: \textcolor{darkblue}{\textbf{\ipa{nɑ˧}}} 
\lhead{\firstmark}
\rhead{\botmark}

\subsection{\hspace{-0.5cm} {\Large \textcolor{darkblue}{\textbf{\ipa{hæ̃˧ʂɯ˩-pæ˩pʰæ˩}}}}\hspace{0.5cm}[\kern2pt{\textcolor{darkblue}{\textbf{\ipa{hæ̃˧ʂɯ˩pæ˧pʰæ˧}}}}\kern2pt]} \hypertarget{h\{\string_~\string_Ms`M\string_B-p\{\string_Bp\string_h\{\string_B1}{}
\markboth{\textcolor{darkblue}{\textbf{\ipa{hæ̃˧ʂɯ˩-pæ˩pʰæ˩}}}}{}
\textcolor{teal}{\mytextsc{noun}} \hspace{4pt} Tone: L\#-.
\textcolor{Sepia}{\selectlanguage{english}Rack for drying grain.} \zh{粮架。}  \zh{量词}: \textcolor{darkblue}{\textbf{\ipa{pʰæ˧˥}}}  \mytextsc{clf}: \textcolor{darkblue}{\textbf{\ipa{pʰæ˧˥}}} 
\lhead{\firstmark}
\rhead{\botmark}

\subsection{\hspace{-0.5cm} {\Large \textcolor{darkblue}{\textbf{\ipa{hæ̃˧ʂv̩˧pɤ˥}}}}\hspace{0.5cm}[\kern2pt{\textcolor{darkblue}{\textbf{\ipa{hæ̃˧ʂv̩˧pɤ˥}}}}\kern2pt]} \hypertarget{h\{\string_~\string_Ms`v\string_=\string_Mp7\string_T1}{}
\markboth{\textcolor{darkblue}{\textbf{\ipa{hæ̃˧ʂv̩˧pɤ˥}}}}{}
\textcolor{teal}{\mytextsc{noun}} \hspace{4pt} Tone: H\#.
\textcolor{Sepia}{\selectlanguage{english}Husband.} \zh{丈夫。} 
\lhead{\firstmark}
\rhead{\botmark}

\subsection{\hspace{-0.5cm} {\Large \textcolor{darkblue}{\textbf{\ipa{hæ̃˧ʐɤ˥}}}}\hspace{0.5cm}[\kern2pt{\textcolor{darkblue}{\textbf{\ipa{hæ̃˧ʐɤ˥}}}}\kern2pt]} \hypertarget{h\{\string_~\string_Mz`7\string_T1}{}
\markboth{\textcolor{darkblue}{\textbf{\ipa{hæ̃˧ʐɤ˥}}}}{}
\textcolor{teal}{\mytextsc{noun}} \hspace{4pt} Tone: H\#.
\textcolor{Sepia}{\selectlanguage{english}Trace of cutting.} \zh{切割的痕迹。}  ¶ \textcolor{darkblue}{\textbf{\ipa{hæ̃˧ʐɤ˥ tʰv̩˩-kʰwɤ˩}}} \textcolor{Sepia}{\selectlanguage{english}\mytextsc{n}+\mytextsc{dem}+\mytextsc{clf}: this trace of cutting} \zh{这道割痕}  
 \zh{量词}: \textcolor{darkblue}{\textbf{\ipa{kʰwɤ˥}}}  \mytextsc{clf}: \textcolor{darkblue}{\textbf{\ipa{kʰwɤ˥}}} 
\lhead{\firstmark}
\rhead{\botmark}

\subsection{\hspace{-0.5cm} {\Large \textcolor{darkblue}{\textbf{\ipa{hæ̃˩}}}}\hspace{0.5cm}[\kern2pt{\textcolor{darkblue}{\textbf{\ipa{hæ̃˥}}}}\kern2pt]} \hypertarget{h\{\string_~\string_B1}{}
\markboth{\textcolor{darkblue}{\textbf{\ipa{hæ̃˩}}}}{}
\textcolor{teal}{\mytextsc{noun}} \hspace{4pt} Tone: L.
\textcolor{Sepia}{\selectlanguage{english}Gold.} \zh{金子。}  \zh{量词}: \textcolor{darkblue}{\textbf{\ipa{ʈv̩˩}}}  \mytextsc{clf}: \textcolor{darkblue}{\textbf{\ipa{ʈv̩˩}}} 
\lhead{\firstmark}
\rhead{\botmark}

\subsection{\hspace{-0.5cm} {\Large \textcolor{darkblue}{\textbf{\ipa{hæ̃˩-bɑ˧lɑ˩}}}}\hspace{0.5cm}[\kern2pt{\textcolor{darkblue}{\textbf{\ipa{xxxx non-correspondance entre le nombre de morphèmes et le nombre de tons de morphèmes}}}}\kern2pt]} \hypertarget{h\{\string_~\string_B-bA\string_MlA\string_B1}{}
\markboth{\textcolor{darkblue}{\textbf{\ipa{hæ̃˩-bɑ˧lɑ˩}}}}{}
\textcolor{teal}{\mytextsc{noun}} \hspace{4pt} Tone: L-L\#.
\textcolor{Sepia}{\selectlanguage{english}Silk.} \zh{丝绸。}  \zh{量词}: \textcolor{darkblue}{\textbf{\ipa{ɭɯ˧}}}  \mytextsc{clf}: \textcolor{darkblue}{\textbf{\ipa{ɭɯ˧}}} 
\lhead{\firstmark}
\rhead{\botmark}

\subsection{\hspace{-0.5cm} {\Large \textcolor{darkblue}{\textbf{\ipa{hæ̃˩bæ˩}}}}\hspace{0.5cm}[\kern2pt{\textcolor{darkblue}{\textbf{\ipa{hæ̃˩bæ˩˥}}}}\kern2pt]} \hypertarget{h\{\string_~\string_Bb\{\string_B1}{}
\markboth{\textcolor{darkblue}{\textbf{\ipa{hæ̃˩bæ˩}}}}{}
\textcolor{teal}{\mytextsc{verb}} \hspace{4pt} Tone: L.
\textcolor{Sepia}{\selectlanguage{english}To dance a ritual dance.} \zh{跳大神。} 
\lhead{\firstmark}
\rhead{\botmark}

\subsection{\hspace{-0.5cm} {\Large \textcolor{darkblue}{\textbf{\ipa{hæ̃˩di˩}}}}\hspace{0.5cm}[\kern2pt{\textcolor{darkblue}{\textbf{\ipa{xxxx non-correspondance entre le nombre de morphèmes et le nombre de tons de morphèmes}}}}\kern2pt]} \hypertarget{h\{\string_~\string_Bdi\string_B1}{}
\markboth{\textcolor{darkblue}{\textbf{\ipa{hæ̃˩di˩}}}}{}
\textcolor{teal}{\mytextsc{noun}} \hspace{4pt} Tone: L.
\textcolor{Sepia}{\selectlanguage{english}Ruler.} \zh{尺。}  \zh{量词}: \textcolor{darkblue}{\textbf{\ipa{nɑ˧}}}  \mytextsc{clf}: \textcolor{darkblue}{\textbf{\ipa{nɑ˧}}} 
\lhead{\firstmark}
\rhead{\botmark}

\subsection{\hspace{-0.5cm} {\Large \textcolor{darkblue}{\textbf{\ipa{hæ̃˩qʰwɤ˩}}}}\hspace{0.5cm}[\kern2pt{\textcolor{darkblue}{\textbf{\ipa{hæ̃˩qʰwɤ˩˥}}}}\kern2pt]} \hypertarget{h\{\string_~\string_Bq\string_hw7\string_B1}{}
\markboth{\textcolor{darkblue}{\textbf{\ipa{hæ̃˩qʰwɤ˩}}}}{}
\textcolor{teal}{\mytextsc{noun}} \hspace{4pt} Tone: L.
\textcolor{Sepia}{\selectlanguage{english}Lavender.} \zh{薰衣草(永宁的一种植物)。}  \zh{量词}: \textcolor{darkblue}{\textbf{\ipa{ɭɯ˧}}}  \mytextsc{clf}: \textcolor{darkblue}{\textbf{\ipa{ɭɯ˧}}} 
\lhead{\firstmark}
\rhead{\botmark}

\subsection{\hspace{-0.5cm} {\Large \textcolor{darkblue}{\textbf{\ipa{hæ̃˩sɤ˩}}}}\hspace{0.5cm}[\kern2pt{\textcolor{darkblue}{\textbf{\ipa{hæ̃˩sɤ˩˥}}}}\kern2pt]} \hypertarget{h\{\string_~\string_Bs7\string_B1}{}
\markboth{\textcolor{darkblue}{\textbf{\ipa{hæ̃˩sɤ˩}}}}{}
\textcolor{teal}{\mytextsc{noun}} \hspace{4pt} Tone: L.
\textcolor{Sepia}{\selectlanguage{english}Magpie.} \zh{喜鹊。}  \zh{量词}: \textcolor{darkblue}{\textbf{\ipa{mi˩}}}  \mytextsc{clf}: \textcolor{darkblue}{\textbf{\ipa{mi˩}}} 
\lhead{\firstmark}
\rhead{\botmark}

\subsection{\hspace{-0.5cm} {\Large \textcolor{darkblue}{\textbf{\ipa{hæ̃˧˥}}}}\hspace{0.5cm}[\kern2pt{\textcolor{darkblue}{\textbf{\ipa{hæ̃˥}}}}\kern2pt]} \hypertarget{h\{\string_~\string_M\string_T1}{}
\markboth{\textcolor{darkblue}{\textbf{\ipa{hæ̃˧˥}}}}{}
\textcolor{teal}{\mytextsc{verb}} \hspace{4pt} Tone: MH.
\ding{202} \textcolor{Sepia}{\selectlanguage{english}To cut (with a blade: sword…), e.g. to cut cloth (to make clothes).} \zh{切,裁。}  ¶ \textcolor{darkblue}{\textbf{\ipa{le˧-hæ̃˧-ze˥}}} \textcolor{Sepia}{\selectlanguage{english}\mytextsc{accomp} \string_ \mytextsc{pfv}} \zh{切了}  
 ¶ \textcolor{darkblue}{\textbf{\ipa{tʰɑ˧-hæ̃˧˥!}}} \textcolor{Sepia}{\selectlanguage{english}\mytextsc{prohib}} \zh{别切!}  
 ¶ \textcolor{darkblue}{\textbf{\ipa{bɑ˩lɑ˩˥ | le˧-hæ̃˧˥, | le˧-ʐv̩˧˥}}} \textcolor{Sepia}{\selectlanguage{english}to cut cloth to make clothes, and to sew clothes} \zh{裁(布料来做)衣服,又缝(衣服) / 先裁布料,再缝衣服}  
\ding{203} \textcolor{Sepia}{\selectlanguage{english}To castrate.} \zh{阉割。} 
\lhead{\firstmark}
\rhead{\botmark}

\subsection{\hspace{-0.5cm} {\Large \textcolor{darkblue}{\textbf{\ipa{hɤ˧}}}}\hspace{0.5cm}[\kern2pt{\textcolor{darkblue}{\textbf{\ipa{hɤ˥}}}}\kern2pt]} \hypertarget{h7\string_M1}{}
\markboth{\textcolor{darkblue}{\textbf{\ipa{hɤ˧}}}}{}
\textcolor{teal}{\mytextsc{noun}} \hspace{4pt} Tone: M.
\textcolor{Sepia}{\selectlanguage{english}All.} \zh{全部。}  ¶ \textcolor{darkblue}{\textbf{\ipa{ɖɯ˧-hɤ˧ | mɤ˧-go˩}}} \textcolor{Sepia}{\selectlanguage{english}to have no ailment at all, to be free from any pain} \zh{一点也没病、没有任何痛苦}  
 ¶ \textcolor{darkblue}{\textbf{\ipa{ɖɯ˧-hɤ˧ | mɤ˧-sɯ˥}}} \textcolor{Sepia}{\selectlanguage{english}to be ignorant of everything (literally: not to know a thing)} \zh{什么也不知道}  
 ¶ \textcolor{darkblue}{\textbf{\ipa{ʈʂʰɯ˧ | ɖɯ˧-hɤ˧ hwæ˧}}} \textcolor{Sepia}{\selectlanguage{english}(S)he buys everything / buys the lot} \zh{他全部都买。/他什么都买。}  

\lhead{\firstmark}
\rhead{\botmark}

\subsection{\hspace{-0.5cm} {\Large \textcolor{darkblue}{\textbf{\ipa{hɤ˩\textsubscript{a}}}} \textsubscript{1}}\hspace{0.5cm}[\kern2pt{\textcolor{darkblue}{\textbf{\ipa{hɤ˩˥}}}}\kern2pt]} \hypertarget{h7\string_Ba1}{}
\markboth{\textcolor{darkblue}{\textbf{\ipa{hɤ˩\textsubscript{a}}}} \textsubscript{1}}{}
\textcolor{teal}{\mytextsc{verb}} \hspace{4pt} Tone: L\textsubscript{a}.
\textcolor{Sepia}{\selectlanguage{english}To dry beside or over a fire.} \zh{烘干。}  ¶ \textcolor{darkblue}{\textbf{\ipa{tʰi˧-hɤ˩}}} \textcolor{Sepia}{\selectlanguage{english}\mytextsc{dur}} \zh{\mytextsc{dur}}  
 ¶ \textcolor{darkblue}{\textbf{\ipa{le˧-hɤ˩}}} \textcolor{Sepia}{\selectlanguage{english}\mytextsc{accomp}} \zh{\mytextsc{accomp}}  
 ¶ \textcolor{darkblue}{\textbf{\ipa{ɖɯ˧-hɤ˩-ɻ̍˩}}} \textcolor{Sepia}{\selectlanguage{english}\mytextsc{delimitative} \string_ \mytextsc{inceptive}} \zh{烘干一下}  
 ¶ \textcolor{darkblue}{\textbf{\ipa{le˧-hɤ˩-ze˩, | le˧-pv̩˧-ze˧!}}} \textcolor{Sepia}{\selectlanguage{english}It was dried beside the fire, and it got dry / and it is now dry!} \zh{烘干了,(现在)干了!}  
 ¶ \textcolor{darkblue}{\textbf{\ipa{ɖɯ˧-kʰwɤ˧ hɤ˥}}} \textcolor{Sepia}{\selectlanguage{english}to dry something beside the fire} \zh{烘干一个东西}  

\lhead{\firstmark}
\rhead{\botmark}

\subsection{\hspace{-0.5cm} {\Large \textcolor{darkblue}{\textbf{\ipa{hɤ˩\textsubscript{a}}}} \textsubscript{2}}\hspace{0.5cm}[\kern2pt{\textcolor{darkblue}{\textbf{\ipa{hɤ˩˥}}}}\kern2pt]} \hypertarget{h7\string_Ba2}{}
\markboth{\textcolor{darkblue}{\textbf{\ipa{hɤ˩\textsubscript{a}}}} \textsubscript{2}}{}
\textcolor{teal}{\mytextsc{verb}} \hspace{4pt} Tone: L\textsubscript{a}.
\textcolor{Sepia}{\selectlanguage{english}To go, past perfective form: has gone.} \zh{去,\mytextsc{过去式,°整体体。}} 
\lhead{\firstmark}
\rhead{\botmark}

\subsection{\hspace{-0.5cm} {\Large \textcolor{darkblue}{\textbf{\ipa{hɤ˩\textsubscript{a}}}} \textsubscript{3}}\hspace{0.5cm}[\kern2pt{\textcolor{darkblue}{\textbf{\ipa{hɤ˩˥}}}}\kern2pt]} \hypertarget{h7\string_Ba3}{}
\markboth{\textcolor{darkblue}{\textbf{\ipa{hɤ˩\textsubscript{a}}}} \textsubscript{3}}{}
\textcolor{teal}{\mytextsc{adjective}} \hspace{4pt} Tone: L\textsubscript{a}.
\textcolor{Sepia}{\selectlanguage{english}Appropriate, good; of person: able, good at a certain technique.} \zh{好(技巧好),好(表扬一个人的行为)。}  ¶ \textcolor{darkblue}{\textbf{\ipa{ɖwæ˧˥ | hɤ˩˥!}}} \textcolor{Sepia}{\selectlanguage{english}\mytextsc{intensive}.very} \zh{很好!}  
 ¶ \textcolor{darkblue}{\textbf{\ipa{mɤ˧-hɤ˩}}} \textcolor{Sepia}{\selectlanguage{english}\mytextsc{neg}} \zh{不好}  
 ¶ \textcolor{darkblue}{\textbf{\ipa{hɤ˩-hĩ˩˥}}} \textcolor{Sepia}{\selectlanguage{english}\mytextsc{rel}/\mytextsc{nmlz}} \zh{好的}  

\lhead{\firstmark}
\rhead{\botmark}

\subsection{\hspace{-0.5cm} {\Large \textcolor{darkblue}{\textbf{\ipa{hi˥}}}}\hspace{0.5cm}[\kern2pt{\textcolor{darkblue}{\textbf{\ipa{hi˥}}}}\kern2pt]} \hypertarget{hi\string_T1}{}
\markboth{\textcolor{darkblue}{\textbf{\ipa{hi˥}}}}{}
\textcolor{teal}{\mytextsc{noun}} \hspace{4pt} Tone: \#H.
\textcolor{Sepia}{\selectlanguage{english}Tooth.} \zh{牙齿。}  ¶ \textcolor{darkblue}{\textbf{\ipa{hi˧ go˧˥}}} \textcolor{Sepia}{\selectlanguage{english}(a) tooth aches; to have a tooth-ache} \zh{牙疼}  
 \zh{量词}: \textcolor{darkblue}{\textbf{\ipa{ɭɯ˧}}}  \mytextsc{clf}: \textcolor{darkblue}{\textbf{\ipa{ɭɯ˧}}} 
\lhead{\firstmark}
\rhead{\botmark}

\subsection{\hspace{-0.5cm} {\Large \textcolor{darkblue}{\textbf{\ipa{*hi˧}}}}\hspace{0.5cm}[\kern2pt{\textcolor{darkblue}{\textbf{\ipa{hi˥}}}}\kern2pt]} \hypertarget{*hi\string_M1}{}
\markboth{\textcolor{darkblue}{\textbf{\ipa{*hi˧}}}}{}
\textcolor{teal}{\mytextsc{adjective}} \hspace{4pt} Tone: H?.
\textit{\textcolor{Sepia}{\selectlanguage{english}archaic}} [\zh{古语}] \textcolor{Sepia}{\selectlanguage{english}Fast.} \zh{快。}  ¶ \textcolor{darkblue}{\textbf{\ipa{hi˧le˩ ʝi˩}}} \textcolor{Sepia}{\selectlanguage{english}to do quickly} \zh{快速做}  
 ¶ \textcolor{darkblue}{\textbf{\ipa{hi˧le˩ | le˧-jo˩!}}} \textcolor{Sepia}{\selectlanguage{english}Come quickly!} \zh{快来!}  
 ¶ \textcolor{darkblue}{\textbf{\ipa{ʈʂʰɯ˧ | ɖwæ˧˥ | hi˧le˩ | ʝi˧-kv̩˩!}}} \textcolor{Sepia}{\selectlanguage{english}(S)he knows how to work really fast!} \zh{他做事很麻利!}  

\lhead{\firstmark}
\rhead{\botmark}

\subsection{\hspace{-0.5cm} {\Large \textcolor{darkblue}{\textbf{\ipa{hi˧dʑi˧}}}}\hspace{0.5cm}[\kern2pt{\textcolor{darkblue}{\textbf{\ipa{hi˧dʑi˧}}}}\kern2pt]} \hypertarget{hi\string_Mdz£i\string_M1}{}
\markboth{\textcolor{darkblue}{\textbf{\ipa{hi˧dʑi˧}}}}{}
\textcolor{teal}{\mytextsc{noun}} \hspace{4pt} Tone: M.
\textcolor{Sepia}{\selectlanguage{english}Rain cape, rainware made from straw, rush….} \zh{蓑衣。}  ¶ \textcolor{darkblue}{\textbf{\ipa{hi˩ gi˩-ze˥, | hi˧dʑi˧ tʰi˧-mv̩˧.}}} \textcolor{Sepia}{\selectlanguage{english}It has begun to rain / it's raining; put on a rain cape.} \zh{下雨了,披蓑衣(雨衣)吧。}  
 \zh{量词}: \textcolor{darkblue}{\textbf{\ipa{ɭɯ˧}}}  \mytextsc{clf}: \textcolor{darkblue}{\textbf{\ipa{ɭɯ˧}}} 
\lhead{\firstmark}
\rhead{\botmark}

\subsection{\hspace{-0.5cm} {\Large \textcolor{darkblue}{\textbf{\ipa{hi˧kʰɯ\#˥}}}}\hspace{0.5cm}[\kern2pt{\textcolor{darkblue}{\textbf{\ipa{hi˧kʰɯ˧}}}}\kern2pt]} \hypertarget{hi\string_Mk\string_hM\#\string_T1}{}
\markboth{\textcolor{darkblue}{\textbf{\ipa{hi˧kʰɯ\#˥}}}}{}
\textcolor{teal}{\mytextsc{noun}} \hspace{4pt} Tone: \#H.
\textcolor{Sepia}{\selectlanguage{english}Gum; gingiva.} \zh{牙龈。}  ¶ \textcolor{darkblue}{\textbf{\ipa{hi˧kʰɯ˧ ʈʂʰæ˧}}} \textcolor{Sepia}{\selectlanguage{english}to brush one's teeth} \zh{刷牙}  
 ¶ \textcolor{darkblue}{\textbf{\ipa{hi˧kʰɯ˧-ʈv̩˥}}} \textcolor{Sepia}{\selectlanguage{english}root of the teeth} \zh{牙根}  

\lhead{\firstmark}
\rhead{\botmark}

\subsection{\hspace{-0.5cm} {\Large \textcolor{darkblue}{\textbf{\ipa{hi˧qʰwɤ˩}}}}\hspace{0.5cm}[\kern2pt{\textcolor{darkblue}{\textbf{\ipa{hi˧qʰwɤ˩}}}}\kern2pt]} \hypertarget{hi\string_Mq\string_hw7\string_B1}{}
\markboth{\textcolor{darkblue}{\textbf{\ipa{hi˧qʰwɤ˩}}}}{}
\textcolor{teal}{\mytextsc{noun}} \hspace{4pt} Tone: L\#.
\textcolor{Sepia}{\selectlanguage{english}Decayed teeth; dental caries.} \zh{蛀牙。}  \zh{量词}: \textcolor{darkblue}{\textbf{\ipa{ɭɯ˧}}}  \mytextsc{clf}: \textcolor{darkblue}{\textbf{\ipa{ɭɯ˧}}} 
\lhead{\firstmark}
\rhead{\botmark}

\subsection{\hspace{-0.5cm} {\Large \textcolor{darkblue}{\textbf{\ipa{hi˧tʰɑ˩}}}}\hspace{0.5cm}[\kern2pt{\textcolor{darkblue}{\textbf{\ipa{hi˧tʰɑ˩}}}}\kern2pt]} \hypertarget{hi\string_Mt\string_hA\string_B1}{}
\markboth{\textcolor{darkblue}{\textbf{\ipa{hi˧tʰɑ˩}}}}{}
\textcolor{teal}{\mytextsc{adjective}} \hspace{4pt} Tone: L\#.
\textcolor{Sepia}{\selectlanguage{english}Sharp, keen (blade).} \zh{锋利。} 
\lhead{\firstmark}
\rhead{\botmark}

\subsection{\hspace{-0.5cm} {\Large \textcolor{darkblue}{\textbf{\ipa{hi˧tʰo˧˥}}}}\hspace{0.5cm}[\kern2pt{\textcolor{darkblue}{\textbf{\ipa{hi˧tʰo˧˥}}}}\kern2pt]} \hypertarget{hi\string_Mt\string_ho\string_M\string_T1}{}
\markboth{\textcolor{darkblue}{\textbf{\ipa{hi˧tʰo˧˥}}}}{}
\textcolor{teal}{\mytextsc{noun}} \hspace{4pt} Tone: MH\#.
\textcolor{Sepia}{\selectlanguage{english}Tooth.} \zh{牙齿。}  \zh{量词}: \textcolor{darkblue}{\textbf{\ipa{ɭɯ˧}}}  \mytextsc{clf}: \textcolor{darkblue}{\textbf{\ipa{ɭɯ˧}}} 
\lhead{\firstmark}
\rhead{\botmark}

\subsection{\hspace{-0.5cm} {\Large \textcolor{darkblue}{\textbf{\ipa{hi˧tsɯ˩}}}}\hspace{0.5cm}[\kern2pt{\textcolor{darkblue}{\textbf{\ipa{hi˧tsɯ˩}}}}\kern2pt]} \hypertarget{hi\string_MtsM\string_B1}{}
\markboth{\textcolor{darkblue}{\textbf{\ipa{hi˧tsɯ˩}}}}{}
\textcolor{teal}{\mytextsc{noun}} \hspace{4pt} Tone: L\#.
\textcolor{Sepia}{\selectlanguage{english}Incisors, front teeth.} \zh{门牙。}  \zh{量词}: \textcolor{darkblue}{\textbf{\ipa{ɭɯ˧}}}  \mytextsc{clf}: \textcolor{darkblue}{\textbf{\ipa{ɭɯ˧}}} 
\lhead{\firstmark}
\rhead{\botmark}

\subsection{\hspace{-0.5cm} {\Large \textcolor{darkblue}{\textbf{\ipa{hi˩}}} \textsubscript{1}}\hspace{0.5cm}[\kern2pt{\textcolor{darkblue}{\textbf{\ipa{hi˥}}}}\kern2pt]} \hypertarget{hi\string_B1}{}
\markboth{\textcolor{darkblue}{\textbf{\ipa{hi˩}}} \textsubscript{1}}{}
\textcolor{teal}{\mytextsc{noun}} \hspace{4pt} Tone: L.
\textcolor{Sepia}{\selectlanguage{english}Lake (monosyllabic word).} \zh{湖、海(单音节)。}  \zh{量词}: \textcolor{darkblue}{\textbf{\ipa{ɭɯ˧}}}  \mytextsc{clf}: \textcolor{darkblue}{\textbf{\ipa{ɭɯ˧}}} 
\lhead{\firstmark}
\rhead{\botmark}

\subsection{\hspace{-0.5cm} {\Large \textcolor{darkblue}{\textbf{\ipa{hi˩}}} \textsubscript{2}}\hspace{0.5cm}[\kern2pt{\textcolor{darkblue}{\textbf{\ipa{hi˩˥}}}}\kern2pt]} \hypertarget{hi\string_B2}{}
\markboth{\textcolor{darkblue}{\textbf{\ipa{hi˩}}} \textsubscript{2}}{}
\textcolor{teal}{\mytextsc{verb}} \hspace{4pt} Tone: L.
\textcolor{Sepia}{\selectlanguage{english}Existential verb, for unmovable objects: e.g.the Lake exists/is at a certain place.} \zh{存在动词:固定不动的物体,如:泸沽湖。}  ¶ \textcolor{darkblue}{\textbf{\ipa{hi˩nɑ˧mi˧ | tʰi˧-hi˩}}} \textcolor{Sepia}{\selectlanguage{english}the Lake exists/is there} \zh{有(泸沽)湖(在那儿)}  

\lhead{\firstmark}
\rhead{\botmark}

\subsection{\hspace{-0.5cm} {\Large \textcolor{darkblue}{\textbf{\ipa{hi˩dʑɯ˩}}}}\hspace{0.5cm}[\kern2pt{\textcolor{darkblue}{\textbf{\ipa{hi˩dʑɯ˩˥}}}}\kern2pt]} \hypertarget{hi\string_Bdz£M\string_B1}{}
\markboth{\textcolor{darkblue}{\textbf{\ipa{hi˩dʑɯ˩}}}}{}
\textcolor{teal}{\mytextsc{noun}} \hspace{4pt} Tone: L.
\textcolor{Sepia}{\selectlanguage{english}Charcoal.} \zh{炭。}  \zh{量词}: \textcolor{darkblue}{\textbf{\ipa{kʰɤ˧˥}}}  \mytextsc{clf}: \textcolor{darkblue}{\textbf{\ipa{kʰɤ˧˥}}} 
\lhead{\firstmark}
\rhead{\botmark}

\subsection{\hspace{-0.5cm} {\Large \textcolor{darkblue}{\textbf{\ipa{hi˩mi˩}}}}\hspace{0.5cm}[\kern2pt{\textcolor{darkblue}{\textbf{\ipa{hi˩mi˩˥}}}}\kern2pt]} \hypertarget{hi\string_Bmi\string_B1}{}
\markboth{\textcolor{darkblue}{\textbf{\ipa{hi˩mi˩}}}}{}
\textcolor{teal}{\mytextsc{noun}} \hspace{4pt} Tone: L.
\textcolor{Sepia}{\selectlanguage{english}Tongue.} \zh{舌头。}  ¶ \textcolor{darkblue}{\textbf{\ipa{hi˩mi˩˥, | ɻ̃˧ mɤ˧-ʑi˧! | ə˧tso˧ ʐwɤ˩-bi˩, | õ˧-lɑ˥ ɖʐv̩˩!}}} \textcolor{Sepia}{\selectlanguage{english}The tongue has no bone! Only oneself knows what one is going to say! (Proverb meaning that one is responsible for one's speech: only oneself knows whether one is telling the truth or not.)} \zh{“舌头没有骨头。讲的是什么(=是否真的),只有自己才知道!”(谚语)}  
 \zh{量词}: \textcolor{darkblue}{\textbf{\ipa{ɭɯ˧}}}  \mytextsc{clf}: \textcolor{darkblue}{\textbf{\ipa{ɭɯ˧}}} 
\lhead{\firstmark}
\rhead{\botmark}

\subsection{\hspace{-0.5cm} {\Large \textcolor{darkblue}{\textbf{\ipa{hi˩nɑ˧mi\#˥}}}}\hspace{0.5cm}[\kern2pt{\textcolor{darkblue}{\textbf{\ipa{hi˩nɑ˧mi˧}}}}\kern2pt]} \hypertarget{hi\string_BnA\string_Mmi\#\string_T1}{}
\markboth{\textcolor{darkblue}{\textbf{\ipa{hi˩nɑ˧mi\#˥}}}}{}
\textcolor{teal}{\mytextsc{noun}} \hspace{4pt} Tone: LM+\#H.
\textcolor{Sepia}{\selectlanguage{english}Lake.} \zh{湖。}  \zh{量词}: \textcolor{darkblue}{\textbf{\ipa{ɭɯ˧}}}  \mytextsc{clf}: \textcolor{darkblue}{\textbf{\ipa{ɭɯ˧}}} 
\lhead{\firstmark}
\rhead{\botmark}

\subsection{\hspace{-0.5cm} {\Large \textcolor{darkblue}{\textbf{\ipa{hi˩ɲi˩zo˩}}}}\hspace{0.5cm}[\kern2pt{\textcolor{darkblue}{\textbf{\ipa{hi˩ɲi˩zo˩˥}}}}\kern2pt]} \hypertarget{hi\string_BJi\string_Bzo\string_B1}{}
\markboth{\textcolor{darkblue}{\textbf{\ipa{hi˩ɲi˩zo˩}}}}{}
\textcolor{teal}{\mytextsc{noun}} \hspace{4pt} Tone: L.
\textcolor{Sepia}{\selectlanguage{english}Salamander.} \zh{娃娃鱼。}  \zh{量词}: \textcolor{darkblue}{\textbf{\ipa{mi˩}}}  \mytextsc{clf}: \textcolor{darkblue}{\textbf{\ipa{mi˩}}} 
\lhead{\firstmark}
\rhead{\botmark}

\subsection{\hspace{-0.5cm} {\Large \textcolor{darkblue}{\textbf{\ipa{hi˩qʰɑ˩}}}}\hspace{0.5cm}[\kern2pt{\textcolor{darkblue}{\textbf{\ipa{hi˩qʰɑ˩˥}}}}\kern2pt]} \hypertarget{hi\string_Bq\string_hA\string_B1}{}
\markboth{\textcolor{darkblue}{\textbf{\ipa{hi˩qʰɑ˩}}}}{}
\textcolor{teal}{\mytextsc{noun}} \hspace{4pt} Tone: L.
\textcolor{Sepia}{\selectlanguage{english}Torrential rain, cloudburst.} \zh{暴雨。}  ¶ \textcolor{darkblue}{\textbf{\ipa{hi˩qʰɑ˩ lɑ˥(-ze˩)}}} \textcolor{Sepia}{\selectlanguage{english}torrential rain is falling} \zh{下暴雨了}  
 \zh{量词}: \textcolor{darkblue}{\textbf{\ipa{ʂɯ˩}}}  \mytextsc{clf}: \textcolor{darkblue}{\textbf{\ipa{ʂɯ˩}}} 
\lhead{\firstmark}
\rhead{\botmark}

\subsection{\hspace{-0.5cm} {\Large \textcolor{darkblue}{\textbf{\ipa{hi˩ʁwɤ˩-lo˧}}}}\hspace{0.5cm}[\kern2pt{\textcolor{darkblue}{\textbf{\ipa{hi˩ʁwɤ˩lo˥}}}}\kern2pt]} \hypertarget{hi\string_BRw7\string_B-lo\string_M1}{}
\markboth{\textcolor{darkblue}{\textbf{\ipa{hi˩ʁwɤ˩-lo˧}}}}{}
\textcolor{teal}{\mytextsc{noun}} \hspace{4pt} Tone: L-.
\textcolor{Sepia}{\selectlanguage{english}The name of a village in the plain of Yongning.} \zh{永宁的一个村落。}  ¶ \textcolor{darkblue}{\textbf{\ipa{dʑɤ˩bv̩˧kɤ˧-sɑ˥ʁwɤ˩, | hi˩ʁwɤ˩-lo˥, | æ˩mi˧-ʁwɤ\#˥, | lɑ˧lo˧-ʁwɤ˥, | lɑ˧ŋwɤ˧, | bɤ˧tsʰo˧gv̩˥, | ə˧lɑ˧-ʁwɤ\#˥, | gæ˧ɻæ˩, | qʰæ˧tɕʰi˧, | tʰo˧ʈɯ\#˥}}} \textcolor{Sepia}{\selectlanguage{english}the ten villages traditionally considered as part of Yongning} \zh{摩梭传统地理概念中,属于永宁的十个村落}  

\lhead{\firstmark}
\rhead{\botmark}

\subsection{\hspace{-0.5cm} {\Large \textcolor{darkblue}{\textbf{\ipa{hi˩ʐæ˥}}}}\hspace{0.5cm}[\kern2pt{\textcolor{darkblue}{\textbf{\ipa{hi˩ʐæ˥}}}}\kern2pt]} \hypertarget{hi\string_Bz`\{\string_T1}{}
\markboth{\textcolor{darkblue}{\textbf{\ipa{hi˩ʐæ˥}}}}{}
\textcolor{teal}{\mytextsc{noun}} \hspace{4pt} Tone: LH.
\ding{202} \textcolor{Sepia}{\selectlanguage{english}Uvula.} \zh{小舌。}  ¶ \textcolor{darkblue}{\textbf{\ipa{qv̩˧ʈʂæ˧-bv̩˥ | hi˩ʐæ˧}}} \textcolor{Sepia}{\selectlanguage{english}the uvula; specifying 'the throat's...' disambiguates between the uvula and the tendon of the tongue, which are referred to by the same term, \textcolor{darkblue}{\textbf{\ipa{/hi˩ʐæ˥/}}}.} \zh{小舌}  
 \zh{量词}: \textcolor{darkblue}{\textbf{\ipa{ɭɯ˧}}} \ding{203} \textcolor{Sepia}{\selectlanguage{english}Tendon of the tongue.} \zh{舌头的筋。}  ¶ \textcolor{darkblue}{\textbf{\ipa{hi˩mi˩-bv̩˧ | hi˩ʐæ˧}}} \textcolor{Sepia}{\selectlanguage{english}the tendon of the tongue; specifying 'the tongue's...' disambiguates between the uvula and the tendon of the tongue, which are referred to by the same term, \textcolor{darkblue}{\textbf{\ipa{/hi˩ʐæ˥/}}}.} \zh{舌头的筋}  
 \mytextsc{clf}: \textcolor{darkblue}{\textbf{\ipa{ɭɯ˧}}} 
\lhead{\firstmark}
\rhead{\botmark}

\subsection{\hspace{-0.5cm} {\Large \textcolor{darkblue}{\textbf{\ipa{hi˩˥}}}}\hspace{0.5cm}[\kern2pt{\textcolor{darkblue}{\textbf{\ipa{hi˩˥}}}}\kern2pt]} \hypertarget{hi\string_B\string_T1}{}
\markboth{\textcolor{darkblue}{\textbf{\ipa{hi˩˥}}}}{}
\textcolor{teal}{\mytextsc{noun}} \hspace{4pt} Tone: LH.
\textcolor{Sepia}{\selectlanguage{english}Rain.} \zh{雨。}  ¶ \textcolor{darkblue}{\textbf{\ipa{hi˩ gi˩˥ / hi˩ gi˩-ze˥}}} \textcolor{Sepia}{\selectlanguage{english}it's raining} \zh{下雨了}  
 \zh{量词}: \textcolor{darkblue}{\textbf{\ipa{ʂɯ˩}}}  \mytextsc{clf}: \textcolor{darkblue}{\textbf{\ipa{ʂɯ˩}}} 
\lhead{\firstmark}
\rhead{\botmark}

\subsection{\hspace{-0.5cm} {\Large \textcolor{darkblue}{\textbf{\ipa{‑hĩ˥}}}}\hspace{0.5cm}[\kern2pt{\textcolor{darkblue}{\textbf{\ipa{xxxx groupe tonal entier sans aucun ton}}}}\kern2pt]} \hypertarget{‑hi\string_~\string_T1}{}
\markboth{\textcolor{darkblue}{\textbf{\ipa{‑hĩ˥}}}}{}
\textcolor{teal}{\mytextsc{conjunction}} \hspace{4pt} Tone: 0.
\textcolor{Sepia}{\selectlanguage{english}Relativizer and nominalizer.} \zh{关系从句/名词化。} 
\lhead{\firstmark}
\rhead{\botmark}

\subsection{\hspace{-0.5cm} {\Large \textcolor{darkblue}{\textbf{\ipa{hĩ˥}}}}\hspace{0.5cm}[\kern2pt{\textcolor{darkblue}{\textbf{\ipa{hĩ˥}}}}\kern2pt]} \hypertarget{hi\string_~\string_T1}{}
\markboth{\textcolor{darkblue}{\textbf{\ipa{hĩ˥}}}}{}
\textcolor{teal}{\mytextsc{noun}} \hspace{4pt} Tone: \#H.
\textcolor{Sepia}{\selectlanguage{english}Person, human being, man (without any indication of gender).} \zh{人。}  ¶ \textcolor{darkblue}{\textbf{\ipa{hĩ˧ | ɖɯ˧-v̩˧}}} \textcolor{Sepia}{\selectlanguage{english}one person, an individual} \zh{一个人}  
 ¶ \textcolor{darkblue}{\textbf{\ipa{hĩ˧-ɻ̃˧ | ɖɯ˧-lo˩}}} \textcolor{Sepia}{\selectlanguage{english}a lineage, a family} \zh{一个家族}  
 ¶ \textcolor{darkblue}{\textbf{\ipa{hĩ˧-mv˥ hĩ˩-di˩}}} \textcolor{Sepia}{\selectlanguage{english}other people's home, other people's place (as opposed to one's home place)} \zh{人家的地方,人家的故乡(不是自己的地方)}  
 ¶ \textcolor{darkblue}{\textbf{\ipa{hĩ˧-mv˥ hĩ˩-di˩ | qʰɑ˧-dʑɤ˥\textasciitilde{}dʑɤ˩, | õ˧-mv˥ õ˩-di˩ tsʰe˩ mɤ˩-gv˩!}}} \textcolor{Sepia}{\selectlanguage{english}No matter how beautiful other people's places are, they can never be equal to one's own homeland!} \zh{其他人的地方怎么好,也比不过自己的地方!}  
 \zh{量词}: \textcolor{darkblue}{\textbf{\ipa{v̩˧}}}  \mytextsc{clf}: \textcolor{darkblue}{\textbf{\ipa{v̩˧}}} 
\lhead{\firstmark}
\rhead{\botmark}

\subsection{\hspace{-0.5cm} {\Large \textcolor{darkblue}{\textbf{\ipa{hĩ˧bæ\#˥}}}}\hspace{0.5cm}[\kern2pt{\textcolor{darkblue}{\textbf{\ipa{hĩ˧bæ˧}}}}\kern2pt]} \hypertarget{hi\string_~\string_Mb\{\#\string_T1}{}
\markboth{\textcolor{darkblue}{\textbf{\ipa{hĩ˧bæ\#˥}}}}{}
\textcolor{teal}{\mytextsc{noun}} \hspace{4pt} Tone: \#H.
\textcolor{Sepia}{\selectlanguage{english}Guest, visitor.} \zh{客人。}  ¶ \textcolor{darkblue}{\textbf{\ipa{hĩ˧bæ˧ ʝi˧}}} \textcolor{Sepia}{\selectlanguage{english}to be a guest, to be invited, to attend a party} \zh{做客}  
 ¶ \textcolor{darkblue}{\textbf{\ipa{hĩ˧bæ˧ tsʰɯ˧-ze˥ ! |}}} \textcolor{Sepia}{\selectlanguage{english}A guest has arrived!} \zh{客人来了!}  
 \zh{量词}: \textcolor{darkblue}{\textbf{\ipa{v̩˧}}}  \mytextsc{clf}: \textcolor{darkblue}{\textbf{\ipa{v̩˧}}} 
\lhead{\firstmark}
\rhead{\botmark}

\subsection{\hspace{-0.5cm} {\Large \textcolor{darkblue}{\textbf{\ipa{hĩ˧hĩ\#˥}}}}\hspace{0.5cm}[\kern2pt{\textcolor{darkblue}{\textbf{\ipa{hĩ˧hĩ˧}}}}\kern2pt]} \hypertarget{hi\string_~\string_Mhi\string_~\#\string_T1}{}
\markboth{\textcolor{darkblue}{\textbf{\ipa{hĩ˧hĩ\#˥}}}}{}
\textcolor{teal}{\mytextsc{noun}} \hspace{4pt} Tone: \#H.
\textcolor{Sepia}{\selectlanguage{english}Strangers, people outside the family.} \zh{外人。}  \zh{量词}: \textcolor{darkblue}{\textbf{\ipa{v̩˧}}}  \mytextsc{clf}: \textcolor{darkblue}{\textbf{\ipa{v̩˧}}} 
\lhead{\firstmark}
\rhead{\botmark}

\subsection{\hspace{-0.5cm} {\Large \textcolor{darkblue}{\textbf{\ipa{hĩ˧-lɑ˩-kv̩˩-hĩ˩}}}}\hspace{0.5cm}[\kern2pt{\textcolor{darkblue}{\textbf{\ipa{hĩ˧lɑ˧kv̩˧hĩ˧}}}}\kern2pt]} \hypertarget{hi\string_~\string_M-lA\string_B-kv\string_=\string_B-hi\string_~\string_B1}{}
\markboth{\textcolor{darkblue}{\textbf{\ipa{hĩ˧-lɑ˩-kv̩˩-hĩ˩}}}}{}
\textcolor{teal}{\mytextsc{noun}} \hspace{4pt} Tone: -L--.
\textcolor{Sepia}{\selectlanguage{english}Dangerous person; enemy.} \zh{危险的人,仇人,敌人。} 
\lhead{\firstmark}
\rhead{\botmark}

\subsection{\hspace{-0.5cm} {\Large \textcolor{darkblue}{\textbf{\ipa{hĩ˧mo˥}}}}\hspace{0.5cm}[\kern2pt{\textcolor{darkblue}{\textbf{\ipa{hĩ˧mo˥}}}}\kern2pt]} \hypertarget{hi\string_~\string_Mmo\string_T1}{}
\markboth{\textcolor{darkblue}{\textbf{\ipa{hĩ˧mo˥}}}}{}
\textcolor{teal}{\mytextsc{noun}} \hspace{4pt} Tone: H\#.
\textcolor{Sepia}{\selectlanguage{english}Elderly person.} \zh{老人。}  ¶ \textcolor{darkblue}{\textbf{\ipa{hĩ˧mo˥-hĩ˩}}} \textcolor{Sepia}{\selectlanguage{english}\string_ \mytextsc{rel;} same meaning} \zh{老人、老的人}  
 \zh{量词}: \textcolor{darkblue}{\textbf{\ipa{v̩˧}}}  \mytextsc{clf}: \textcolor{darkblue}{\textbf{\ipa{v̩˧}}} 
\lhead{\firstmark}
\rhead{\botmark}

\subsection{\hspace{-0.5cm} {\Large \textcolor{darkblue}{\textbf{\ipa{hĩ˧mo˩}}}}\hspace{0.5cm}[\kern2pt{\textcolor{darkblue}{\textbf{\ipa{hĩ˧mo˩}}}}\kern2pt]} \hypertarget{hi\string_~\string_Mmo\string_B1}{}
\markboth{\textcolor{darkblue}{\textbf{\ipa{hĩ˧mo˩}}}}{}
\textcolor{teal}{\mytextsc{noun}} \hspace{4pt} Tone: L\#.
\ding{202} \textcolor{Sepia}{\selectlanguage{english}Corpse.} \zh{尸体。}  ¶ \textcolor{darkblue}{\textbf{\ipa{hĩ˧mo˩-kʰɯ˩-di˩}}} \textcolor{Sepia}{\selectlanguage{english}coffin; literally 'thing (in which) to put a corpse'} \zh{棺材}  
 \zh{量词}: \textcolor{darkblue}{\textbf{\ipa{mo˧}}} \ding{203} \textcolor{Sepia}{\selectlanguage{english}Tomb.} \zh{坟墓。}  \mytextsc{clf}: \textcolor{darkblue}{\textbf{\ipa{mo˧}}} 
\lhead{\firstmark}
\rhead{\botmark}

\subsection{\hspace{-0.5cm} {\Large \textcolor{darkblue}{\textbf{\ipa{hĩ˧-tɕʰɯ\#˥}}}}\hspace{0.5cm}[\kern2pt{\textcolor{darkblue}{\textbf{\ipa{xxxx non-correspondance entre le nombre de morphèmes et le nombre de tons de morphèmes}}}}\kern2pt]} \hypertarget{hi\string_~\string_M-ts£\string_hM\#\string_T1}{}
\markboth{\textcolor{darkblue}{\textbf{\ipa{hĩ˧-tɕʰɯ\#˥}}}}{}
\textcolor{teal}{\mytextsc{noun}} \hspace{4pt} Tone: \#H.
\textcolor{Sepia}{\selectlanguage{english}Family member belonging to the same generation: brother, sister, or cousin (on the mother's side).} \zh{同一辈的亲戚:兄弟姐妹、堂兄弟姐妹。}  ¶ \textcolor{darkblue}{\textbf{\ipa{hĩ˧-tɕʰɯ˧ - hĩ˧-ʈʂɤ\#˥}}} \textcolor{Sepia}{\selectlanguage{english}same meaning: the family members belonging to the same generation} \zh{同一辈的亲戚:兄弟姐妹、堂兄弟姐妹}  
 ¶ \textcolor{darkblue}{\textbf{\ipa{ʈʂʰɯ˧ | njɤ˧ | hĩ˧ tɕʰɯ˧ ɲi˥!}}} \textcolor{Sepia}{\selectlanguage{english}(S)he is someone of my generation! (=my cousin, my brother/sister...)} \zh{他是跟我同一辈的亲戚!(=堂兄弟姐妹)}  
 ¶ \textcolor{darkblue}{\textbf{\ipa{hĩ˧-tɕʰɯ˧ mɤ˧-ɲi˥ F | hĩ˧-tɕʰɯ˧ ʝi˧ tʰɑ˩-kv̩˩!}}} \textcolor{Sepia}{\selectlanguage{english}“Even if one is not (born) a family member, it is possible to become one!” A saying that refers to quasi-family links between friends, which amount to a form of adoption into the family circle.} \zh{“不是亲戚,也可以变成亲戚!”这个俗语来形容朋友之间的深情,变成像家人之间的感情。}  
\textit{Syn:} \hyperlink{}{\textcolor{darkblue}{\textbf{\ipa{hĩ˧-ʈʂɤ\#˥}}}}. 
\lhead{\firstmark}
\rhead{\botmark}

\subsection{\hspace{-0.5cm} {\Large \textcolor{darkblue}{\textbf{\ipa{hĩ˧-ʈʂɤ\#˥}}}}\hspace{0.5cm}[\kern2pt{\textcolor{darkblue}{\textbf{\ipa{xxxx non-correspondance entre le nombre de morphèmes et le nombre de tons de morphèmes}}}}\kern2pt]} \hypertarget{hi\string_~\string_M-t`s`7\#\string_T1}{}
\markboth{\textcolor{darkblue}{\textbf{\ipa{hĩ˧-ʈʂɤ\#˥}}}}{}
\textcolor{teal}{\mytextsc{noun}} \hspace{4pt} Tone: \#H.
\textcolor{Sepia}{\selectlanguage{english}Family member belonging to the same generation: brother, sister, or cousin (on the mother's side).} \zh{同一辈的亲戚:兄弟姐妹、堂兄弟姐妹。}  ¶ \textcolor{darkblue}{\textbf{\ipa{hĩ˧-tɕʰɯ˧ - hĩ˧-ʈʂɤ\#˥}}} \textcolor{Sepia}{\selectlanguage{english}same meaning: the family members belonging to the same generation} \zh{同一辈的亲戚:兄弟姐妹、堂兄弟姐妹}  
\textit{Syn:} \hyperlink{}{\textcolor{darkblue}{\textbf{\ipa{hĩ˧-tɕʰɯ\#˥}}}}. 
\lhead{\firstmark}
\rhead{\botmark}

\subsection{\hspace{-0.5cm} {\Large \textcolor{darkblue}{\textbf{\ipa{hĩ˧˥}}} \textsubscript{1}}\hspace{0.5cm}[\kern2pt{\textcolor{darkblue}{\textbf{\ipa{hĩ˧˥}}}}\kern2pt]} \hypertarget{hi\string_~\string_M\string_T1}{}
\markboth{\textcolor{darkblue}{\textbf{\ipa{hĩ˧˥}}} \textsubscript{1}}{}
\textcolor{teal}{\mytextsc{verb}} \hspace{4pt} Tone: MH.
\textcolor{Sepia}{\selectlanguage{english}To stand, to stand upright.} \zh{站(站立)。} 
\lhead{\firstmark}
\rhead{\botmark}

\subsection{\hspace{-0.5cm} {\Large \textcolor{darkblue}{\textbf{\ipa{hĩ˧˥}}} \textsubscript{2}}\hspace{0.5cm}[\kern2pt{\textcolor{darkblue}{\textbf{\ipa{hĩ˧˥}}}}\kern2pt]} \hypertarget{hi\string_~\string_M\string_T2}{}
\markboth{\textcolor{darkblue}{\textbf{\ipa{hĩ˧˥}}} \textsubscript{2}}{}
\textcolor{teal}{\mytextsc{verb}} \hspace{4pt} Tone: MH.
\textcolor{Sepia}{\selectlanguage{english}To have to, to be necessary.} \zh{应该。}  ¶ \textcolor{darkblue}{\textbf{\ipa{mɤ˧-hĩ˧}}} \textcolor{Sepia}{\selectlanguage{english}\mytextsc{neg}} \zh{\mytextsc{否定}}  
 ¶ \textcolor{darkblue}{\textbf{\ipa{no˧ | ʝi˧-hĩ˧˥!}}} \textcolor{Sepia}{\selectlanguage{english}You have to do it!} \zh{你应该做!}  
 ¶ \textcolor{darkblue}{\textbf{\ipa{njɤ˧ | ʝi˧-mɤ˧-hĩ˧-hĩ˥ | (ɖɯ˧-pi˧) ʝi˧-ze˩! |}}} \textcolor{Sepia}{\selectlanguage{english}I have done something I shouldn't have!} \zh{我做了一件不应该做的事!}  
 ¶ \textcolor{darkblue}{\textbf{\ipa{no˧ | lo˧ ʝi˧-hĩ˧!}}} \textcolor{Sepia}{\selectlanguage{english}You have to work! / You must work!} \zh{你应该工作啊!}  

\lhead{\firstmark}
\rhead{\botmark}

\subsection{\hspace{-0.5cm} {\Large \textcolor{darkblue}{\textbf{\ipa{ho˥}}} \textsubscript{1}}\hspace{0.5cm}[\kern2pt{\textcolor{darkblue}{\textbf{\ipa{ho˧˥}}}}\kern2pt]} \hypertarget{ho\string_T1}{}
\markboth{\textcolor{darkblue}{\textbf{\ipa{ho˥}}} \textsubscript{1}}{}
\textcolor{teal}{\mytextsc{noun}} \hspace{4pt} Tone: \#H.
\textcolor{Sepia}{\selectlanguage{english}Partridge.} \zh{雉。}  ¶ \textcolor{darkblue}{\textbf{\ipa{ho˧ tʰv̩˧-mi˧˥ / ho˧ tʰv̩˧-mi˥\#}}} \textcolor{Sepia}{\selectlanguage{english}\mytextsc{n}+\mytextsc{dem}+\mytextsc{clf}} \zh{这只雉}  
 \zh{量词}: \textcolor{darkblue}{\textbf{\ipa{mi˩}}}  \mytextsc{clf}: \textcolor{darkblue}{\textbf{\ipa{mi˩}}} 
\lhead{\firstmark}
\rhead{\botmark}

\subsection{\hspace{-0.5cm} {\Large \textcolor{darkblue}{\textbf{\ipa{ho˥}}} \textsubscript{2}}\hspace{0.5cm}[\kern2pt{\textcolor{darkblue}{\textbf{\ipa{ho˥}}}}\kern2pt]} \hypertarget{ho\string_T2}{}
\markboth{\textcolor{darkblue}{\textbf{\ipa{ho˥}}} \textsubscript{2}}{}
\textcolor{teal}{\mytextsc{noun}} \hspace{4pt} Tone: \#H.
\textcolor{Sepia}{\selectlanguage{english}Porridge, gruel, congee.} \zh{粥。}  ¶ \textcolor{darkblue}{\textbf{\ipa{ho˧ ʈʰɯ˧˥}}} \textcolor{Sepia}{\selectlanguage{english}to drink gruel} \zh{喝粥}  

\lhead{\firstmark}
\rhead{\botmark}

\subsection{\hspace{-0.5cm} {\Large \textcolor{darkblue}{\textbf{\ipa{ho˧ɕjæ˩}}}}\hspace{0.5cm}[\kern2pt{\textcolor{darkblue}{\textbf{\ipa{ho˩ɕjæ˥}}}}\kern2pt]} \hypertarget{ho\string_Ms£j\{\string_B1}{}
\markboth{\textcolor{darkblue}{\textbf{\ipa{ho˧ɕjæ˩}}}}{}
\textcolor{teal}{\mytextsc{noun}} \hspace{4pt} Tone: LM.
\textcolor{Sepia}{\selectlanguage{english}Cord to which fire is put in order to shoot.} \zh{火绳,导火索。} Local Chinese dialect:\zh{火线。} Borrowing: Chinese  \zh{火线}

\lhead{\firstmark}
\rhead{\botmark}

\subsection{\hspace{-0.5cm} {\Large \textcolor{darkblue}{\textbf{\ipa{ho˧di˧}}}}\hspace{0.5cm}[\kern2pt{\textcolor{darkblue}{\textbf{\ipa{ho˩di˥}}}}\kern2pt]} \hypertarget{ho\string_Mdi\string_M1}{}
\markboth{\textcolor{darkblue}{\textbf{\ipa{ho˧di˧}}}}{}
\textcolor{teal}{\mytextsc{noun}} \hspace{4pt} Tone: M.
\textcolor{Sepia}{\selectlanguage{english}Chinese (Han) areas of Sichuan: Yanyuan, Yanbian, Xichang...} \zh{四川(盐源、盐边、西昌……)。} 
\lhead{\firstmark}
\rhead{\botmark}

\subsection{\hspace{-0.5cm} {\Large \textcolor{darkblue}{\textbf{\ipa{ho˧dʑɯ˧tɤ˥ɻ̍˩}}}}\hspace{0.5cm}[\kern2pt{\textcolor{darkblue}{\textbf{\ipa{ho˧dʑɯ˧tɤ˧ɻ̍˧˥}}}}\kern2pt]} \hypertarget{ho\string_Mdz£M\string_Mt7\string_Tr£`̍\string_B1}{}
\markboth{\textcolor{darkblue}{\textbf{\ipa{ho˧dʑɯ˧tɤ˥ɻ̍˩}}}}{}
\textcolor{teal}{\mytextsc{noun}} \hspace{4pt} Tone: \#H-.
\textcolor{Sepia}{\selectlanguage{english}Paste; starch.} \zh{浆糊,浆子。} \textit{See:} \hyperlink{}{\textcolor{darkblue}{\textbf{\ipa{ho˧dʑɯ˧˥}}}} 
\lhead{\firstmark}
\rhead{\botmark}

\subsection{\hspace{-0.5cm} {\Large \textcolor{darkblue}{\textbf{\ipa{ho˧dʑɯ˧˥}}}}\hspace{0.5cm}[\kern2pt{\textcolor{darkblue}{\textbf{\ipa{ho˩dʑɯ˩˥}}}}\kern2pt]} \hypertarget{ho\string_Mdz£M\string_M\string_T1}{}
\markboth{\textcolor{darkblue}{\textbf{\ipa{ho˧dʑɯ˧˥}}}}{}
\textcolor{teal}{\mytextsc{noun}} \hspace{4pt} Tone: MH\#.
\textcolor{Sepia}{\selectlanguage{english}Paste; starch.} \zh{浆糊,浆子。} \textit{See:} \hyperlink{}{\textcolor{darkblue}{\textbf{\ipa{ho˧dʑɯ˧tɤ˥ɻ̍˩}}}} 
\lhead{\firstmark}
\rhead{\botmark}

\subsection{\hspace{-0.5cm} {\Large \textcolor{darkblue}{\textbf{\ipa{ho˧ko˧}}}}\hspace{0.5cm}[\kern2pt{\textcolor{darkblue}{\textbf{\ipa{xxxx non-correspondance entre le nombre de morphèmes et le nombre de tons de morphèmes}}}}\kern2pt]} \hypertarget{ho\string_Mko\string_M1}{}
\markboth{\textcolor{darkblue}{\textbf{\ipa{ho˧ko˧}}}}{}
\textcolor{teal}{\mytextsc{noun}} \hspace{4pt} Tone: M.
\textcolor{Sepia}{\selectlanguage{english}Cooking pot for making hotpot; traditionally made of copper, with a hole in the centre.} \zh{火锅(汉语借词)。}  Borrowing: Chinese  \zh{火锅}
 ¶ \textcolor{darkblue}{\textbf{\ipa{æ̃˧-ho˧ko˥}}} \textcolor{Sepia}{\selectlanguage{english}copper pot for hotpot} \zh{铜火锅}  

\lhead{\firstmark}
\rhead{\botmark}

\subsection{\hspace{-0.5cm} {\Large \textcolor{darkblue}{\textbf{\ipa{ho˧mi\#˥}}}}\hspace{0.5cm}[\kern2pt{\textcolor{darkblue}{\textbf{\ipa{ho˩mi˥}}}}\kern2pt]} \hypertarget{ho\string_Mmi\#\string_T1}{}
\markboth{\textcolor{darkblue}{\textbf{\ipa{ho˧mi\#˥}}}}{}
\textcolor{teal}{\mytextsc{noun}} \hspace{4pt} Tone: \#H.
\textcolor{Sepia}{\selectlanguage{english}Female pheasant.} \zh{母雉。}  ¶ \textcolor{darkblue}{\textbf{\ipa{ho˧mi˧ tʰv̩˧-mi˧˥ / ho˧mi˧ tʰv̩˧-mi˥\#}}} \textcolor{Sepia}{\selectlanguage{english}\mytextsc{n}+\mytextsc{dem}+\mytextsc{clf}} \zh{这个母雉}  
 ¶ \textcolor{darkblue}{\textbf{\ipa{ho˧mi˧-ho˧pʰv̩˥ / ho˧mi˧-ho˥pʰv̩˩}}} \textcolor{Sepia}{\selectlanguage{english}female and male pheasant} \zh{母雉与公雉}  
 \zh{量词}: \textcolor{darkblue}{\textbf{\ipa{mi˩}}}  \mytextsc{clf}: \textcolor{darkblue}{\textbf{\ipa{mi˩}}} 
\lhead{\firstmark}
\rhead{\botmark}

\subsection{\hspace{-0.5cm} {\Large \textcolor{darkblue}{\textbf{\ipa{ho˧pʰv̩\#˥}}}}\hspace{0.5cm}[\kern2pt{\textcolor{darkblue}{\textbf{\ipa{ho˧pʰv̩˧}}}}\kern2pt]} \hypertarget{ho\string_Mp\string_hv\string_=\#\string_T1}{}
\markboth{\textcolor{darkblue}{\textbf{\ipa{ho˧pʰv̩\#˥}}}}{}
\textcolor{teal}{\mytextsc{noun}} \hspace{4pt} Tone: \#H.
\textcolor{Sepia}{\selectlanguage{english}Male pheasant.} \zh{公雉。}  ¶ \textcolor{darkblue}{\textbf{\ipa{ho˧pʰv̩˧ tʰv̩˧-mi˧˥ / ho˧pʰv̩˧ tʰv̩˧-mi˥\#}}} \textcolor{Sepia}{\selectlanguage{english}\mytextsc{n}+\mytextsc{dem}+\mytextsc{clf}} \zh{这只公雉}  
 \zh{量词}: \textcolor{darkblue}{\textbf{\ipa{mi˩}}}  \mytextsc{clf}: \textcolor{darkblue}{\textbf{\ipa{mi˩}}} 
\lhead{\firstmark}
\rhead{\botmark}

\subsection{\hspace{-0.5cm} {\Large \textcolor{darkblue}{\textbf{\ipa{ho˧ʈʂɯ˧}}}}\hspace{0.5cm}[\kern2pt{\textcolor{darkblue}{\textbf{\ipa{ho˩ʈʂɯ˧˥}}}}\kern2pt]} \hypertarget{ho\string_Mt`s`M\string_M1}{}
\markboth{\textcolor{darkblue}{\textbf{\ipa{ho˧ʈʂɯ˧}}}}{}
\textcolor{teal}{\mytextsc{noun}} \hspace{4pt} Tone: M.
\textcolor{Sepia}{\selectlanguage{english}Mugwort, wormwood, \textit{Artemisia vulgaris}.} \zh{蒿(汉语借词:蒿枝)。} Local Chinese dialect:\zh{蒿草、蒿枝。} Borrowing: Chinese  \zh{蒿枝}
\textit{See:} \hyperlink{}{\textcolor{darkblue}{\textbf{\ipa{tɕɤ˧qʰɑ\#˥}}}} 
\lhead{\firstmark}
\rhead{\botmark}

\subsection{\hspace{-0.5cm} {\Large \textcolor{darkblue}{\textbf{\ipa{ho˧zo\#˥}}}}\hspace{0.5cm}[\kern2pt{\textcolor{darkblue}{\textbf{\ipa{ho˧zo˧}}}}\kern2pt]} \hypertarget{ho\string_Mzo\#\string_T1}{}
\markboth{\textcolor{darkblue}{\textbf{\ipa{ho˧zo\#˥}}}}{}
\textcolor{teal}{\mytextsc{noun}} \hspace{4pt} Tone: \#H.
\textcolor{Sepia}{\selectlanguage{english}Baby pheasant.} \zh{小雉。}  ¶ \textcolor{darkblue}{\textbf{\ipa{ho˧zo˧ tʰv̩˧-ɭɯ\#˥}}} \textcolor{Sepia}{\selectlanguage{english}\mytextsc{n}+\mytextsc{dem}+\mytextsc{clf}} \zh{这只小雉}  
 \zh{量词}: \textcolor{darkblue}{\textbf{\ipa{ɭɯ˧}}}  \mytextsc{clf}: \textcolor{darkblue}{\textbf{\ipa{ɭɯ˧}}} 
\lhead{\firstmark}
\rhead{\botmark}

\subsection{\hspace{-0.5cm} {\Large \textcolor{darkblue}{\textbf{\ipa{‑ho˩}}}}\hspace{0.5cm}[\kern2pt{\textcolor{darkblue}{\textbf{\ipa{ho˩˥}}}}\kern2pt]} \hypertarget{‑ho\string_B1}{}
\markboth{\textcolor{darkblue}{\textbf{\ipa{‑ho˩}}}}{}
\textcolor{teal}{\mytextsc{suffix}} \hspace{4pt} Tone: L.
\textcolor{Sepia}{\selectlanguage{english}Future / desiderative / conjecture.} \zh{\mytextsc{未来}\string_愿望。}  ¶ \textcolor{darkblue}{\textbf{\ipa{hi˩gi˩ ə˥-ho˩? - hi˩ gi˩ ho˥!}}} \textcolor{Sepia}{\selectlanguage{english}Is it going to rain? - Yes, it is going to rain!} \zh{要下雨了吗? - 是的,要下雨了!}  
 ¶ \textcolor{darkblue}{\textbf{\ipa{ʈʂʰɯ˧ | so˧ɲi˥ | le˧-jo˩ ho˩-hĩ˩ | ə˩-ɲi˩˥ ? - ʈʂʰɯ˧ | so˧ɲi˥ | le˧-jo˩-ho˩-ɲi˩-mæ˩.}}} \textcolor{Sepia}{\selectlanguage{english}Is he going to come tomorrow? - (Yes,) he will come tomorrow.} \zh{他明天要来买? - (是的,)他明天会来的。(回答表示:比较肯定。)}  
 ¶ \textcolor{darkblue}{\textbf{\ipa{so˧ɲi˥ | le˧-ɬi˥ | mɤ˧-ho˥!}}} \textcolor{Sepia}{\selectlanguage{english}Tomorrow, they won't be on holiday anymore! (Context: on a Sunday, talking about a kindergarten that has been on holiday during the previous week.)} \zh{明天就不休息了!}  
 ¶ \textcolor{darkblue}{\textbf{\ipa{tɕʰi˧ ə˧-ho˩?}}} \textcolor{Sepia}{\selectlanguage{english}Is (she/he) going to sell?} \zh{要卖吗?/ 会卖吗?}  
 ¶ \textcolor{darkblue}{\textbf{\ipa{hwæ˧ ə˧-ho˥?}}} \textcolor{Sepia}{\selectlanguage{english}Is (she/he) going to buy?} \zh{要买吗?/ 会买马?}  

\lhead{\firstmark}
\rhead{\botmark}

\subsection{\hspace{-0.5cm} {\Large \textcolor{darkblue}{\textbf{\ipa{ho˩\textsubscript{a}}}}}\hspace{0.5cm}[\kern2pt{\textcolor{darkblue}{\textbf{\ipa{ho˥}}}}\kern2pt]} \hypertarget{ho\string_Ba1}{}
\markboth{\textcolor{darkblue}{\textbf{\ipa{ho˩\textsubscript{a}}}}}{}
\textcolor{teal}{\mytextsc{adjective}} \hspace{4pt} Tone: L\textsubscript{a}.
\textcolor{Sepia}{\selectlanguage{english}Correct; suitable, appropriate.} \zh{准确,合适。}  Borrowing: chinois  ancien: \zh{合} ?
 ¶ \textcolor{darkblue}{\textbf{\ipa{mɤ˧-ho˩}}} \textcolor{Sepia}{\selectlanguage{english}\mytextsc{neg}} \zh{不合适,不准,不对}  
 ¶ \textcolor{darkblue}{\textbf{\ipa{ho˩-ze˧!}}} \textcolor{Sepia}{\selectlanguage{english}\mytextsc{pfv}} \zh{对了! / 准确!}  
 ¶ \textcolor{darkblue}{\textbf{\ipa{ho˩-hĩ˩˥}}} \textcolor{Sepia}{\selectlanguage{english}\mytextsc{rel}/\mytextsc{nmlz}} \zh{准确的}  

\lhead{\firstmark}
\rhead{\botmark}

\subsection{\hspace{-0.5cm} {\Large \textcolor{darkblue}{\textbf{\ipa{ho˩ɕjæ˧}}}}\hspace{0.5cm}[\kern2pt{\textcolor{darkblue}{\textbf{\ipa{ho˩ɕjæ˩˥}}}}\kern2pt]} \hypertarget{ho\string_Bs£j\{\string_M1}{}
\markboth{\textcolor{darkblue}{\textbf{\ipa{ho˩ɕjæ˧}}}}{}
\textcolor{teal}{\mytextsc{noun}} \hspace{4pt} Tone: LM.
\textcolor{Sepia}{\selectlanguage{english}Wrinkled giant hyssop, \textit{Elshotzia sp.}.} \zh{藿香。}  Borrowing: Chinese  \zh{藿香}

\lhead{\firstmark}
\rhead{\botmark}

\subsection{\hspace{-0.5cm} {\Large \textcolor{darkblue}{\textbf{\ipa{ho˩dʑɯ˩}}}}\hspace{0.5cm}[\kern2pt{\textcolor{darkblue}{\textbf{\ipa{ho˧dʑɯ˧}}}}\kern2pt]} \hypertarget{ho\string_Bdz£M\string_B1}{}
\markboth{\textcolor{darkblue}{\textbf{\ipa{ho˩dʑɯ˩}}}}{}
\textcolor{teal}{\mytextsc{adjective}} \hspace{4pt} Tone: L.
\textcolor{Sepia}{\selectlanguage{english}Destitute, impoverished, down and out.} \zh{穷苦、凋敝、寒苦、竭蹶、穷乏。}  ¶ \textcolor{darkblue}{\textbf{\ipa{ho˩dʑɯ˩-ze˥}}} \textcolor{Sepia}{\selectlanguage{english}\mytextsc{pfv}} \zh{变穷苦了}  

\lhead{\firstmark}
\rhead{\botmark}

\subsection{\hspace{-0.5cm} {\Large \textcolor{darkblue}{\textbf{\ipa{ho˩lo˧pv̩˥}}}}\hspace{0.5cm}[\kern2pt{\textcolor{darkblue}{\textbf{\ipa{ho˧lo˧pv̩˧}}}}\kern2pt]} \hypertarget{ho\string_Blo\string_Mpv\string_=\string_T1}{}
\markboth{\textcolor{darkblue}{\textbf{\ipa{ho˩lo˧pv̩˥}}}}{}
\textcolor{teal}{\mytextsc{noun}} \hspace{4pt} Tone: LM+H\#.
\textcolor{Sepia}{\selectlanguage{english}Carrot.} \zh{胡萝卜。}  Borrowing: Chinese  \zh{胡萝卜}
 \zh{量词}: \textcolor{darkblue}{\textbf{\ipa{ɭɯ˧}}}  \mytextsc{clf}: \textcolor{darkblue}{\textbf{\ipa{ɭɯ˧}}} 
\lhead{\firstmark}
\rhead{\botmark}

\subsection{\hspace{-0.5cm} {\Large \textcolor{darkblue}{\textbf{\ipa{ho˧˥}}}}\hspace{0.5cm}[\kern2pt{\textcolor{darkblue}{\textbf{\ipa{ho˧˥}}}}\kern2pt]} \hypertarget{ho\string_M\string_T1}{}
\markboth{\textcolor{darkblue}{\textbf{\ipa{ho˧˥}}}}{}
\textcolor{teal}{\mytextsc{verb}} \hspace{4pt} Tone: MH.
\textcolor{Sepia}{\selectlanguage{english}To sip: to drink by small mouthfuls.} \zh{小口地喝。}  ¶ \textcolor{darkblue}{\textbf{\ipa{ʐɯ˧ ho˧˥}}} \textcolor{Sepia}{\selectlanguage{english}to sip wine} \zh{小口地喝酒}  
 ¶ \textcolor{darkblue}{\textbf{\ipa{ʐɯ˧ ho˧\textasciitilde{}ho˥}}} \textcolor{Sepia}{\selectlanguage{english}to sip wine} \zh{小口地喝酒}  
 ¶ \textcolor{darkblue}{\textbf{\ipa{ʐɯ˧ | ɖɯ˧-ho˧\textasciitilde{}ho˥}}} \textcolor{Sepia}{\selectlanguage{english}to sip wine} \zh{喝一小口酒}  

\lhead{\firstmark}
\rhead{\botmark}

\subsection{\hspace{-0.5cm} {\Large \textcolor{darkblue}{\textbf{\ipa{hõ˧}}}}\hspace{0.5cm}[\kern2pt{\textcolor{darkblue}{\textbf{\ipa{hõ˥}}}}\kern2pt]} \hypertarget{ho\string_~\string_M1}{}
\markboth{\textcolor{darkblue}{\textbf{\ipa{hõ˧}}}}{}
\textcolor{teal}{\mytextsc{verb}} \hspace{4pt} Tone: M intrans.
\ding{202} \textcolor{Sepia}{\selectlanguage{english}To go away (imperative).} \zh{走(离开),\mytextsc{命令式。}}  ¶ \textcolor{darkblue}{\textbf{\ipa{no˧ hõ˧!}}} \textcolor{Sepia}{\selectlanguage{english}Go!} \zh{你走吧!}  
 ¶ \textcolor{darkblue}{\textbf{\ipa{no˧ | le˧-hõ˧!}}} \textcolor{Sepia}{\selectlanguage{english}Go!} \zh{你走吧!}  
 ¶ \textcolor{darkblue}{\textbf{\ipa{ə˧-ze˧\textasciitilde{}ze˥ hõ˩! / ə˧-dzɤ˥ | le˧-hõ˧!}}} \textcolor{Sepia}{\selectlanguage{english}Walk slowly! / Take your time on the road! / Have a quiet and pleasant journey! (Polite salutation to someone who is leaving.)} \zh{慢走!}  
 ¶ \textcolor{darkblue}{\textbf{\ipa{ɑ˩pʰo˩ hõ˩˥!}}} \textcolor{Sepia}{\selectlanguage{english}Get out!} \zh{出去!走开!滚出去!}  
\ding{203} \textcolor{Sepia}{\selectlanguage{english}Imperative.} \zh{\mytextsc{命令式。}}  ¶ \textcolor{darkblue}{\textbf{\ipa{no˧ | dzɯ˧-hõ˧!}}} \textcolor{Sepia}{\selectlanguage{english}Eat!} \zh{你吃吧!}  

\lhead{\firstmark}
\rhead{\botmark}

\subsection{\hspace{-0.5cm} {\Large \textcolor{darkblue}{\textbf{\ipa{hõ˧-ɬi˧mi\#˥}}}}\hspace{0.5cm}[\kern2pt{\textcolor{darkblue}{\textbf{\ipa{xxxx non-correspondance entre le nombre de morphèmes et le nombre de tons de morphèmes}}}}\kern2pt]} \hypertarget{ho\string_~\string_M-Ki\string_Mmi\#\string_T1}{}
\markboth{\textcolor{darkblue}{\textbf{\ipa{hõ˧-ɬi˧mi\#˥}}}}{}
\textcolor{teal}{\mytextsc{noun}} \hspace{4pt} Tone: \#H.
\textcolor{Sepia}{\selectlanguage{english}8th month.} \zh{八月。} 
\lhead{\firstmark}
\rhead{\botmark}

\subsection{\hspace{-0.5cm} {\Large \textcolor{darkblue}{\textbf{\ipa{hõ˩tsʰi˧˥}}}}\hspace{0.5cm}[\kern2pt{\textcolor{darkblue}{\textbf{\ipa{hõ˧tsʰi˧}}}}\kern2pt]} \hypertarget{ho\string_~\string_Bts\string_hi\string_M\string_T1}{}
\markboth{\textcolor{darkblue}{\textbf{\ipa{hõ˩tsʰi˧˥}}}}{}
\textcolor{teal}{\mytextsc{number}} \hspace{4pt} Tone: LM+MH\#.
\textcolor{Sepia}{\selectlanguage{english}80.} \zh{80。} 
\lhead{\firstmark}
\rhead{\botmark}

\subsection{\hspace{-0.5cm} {\Large \textcolor{darkblue}{\textbf{\ipa{hõ˧˥}}}}\hspace{0.5cm}[\kern2pt{\textcolor{darkblue}{\textbf{\ipa{hõ˥}}}}\kern2pt]} \hypertarget{ho\string_~\string_M\string_T1}{}
\markboth{\textcolor{darkblue}{\textbf{\ipa{hõ˧˥}}}}{}
\textcolor{teal}{\mytextsc{number}} \hspace{4pt} Tone: MH.
\textcolor{Sepia}{\selectlanguage{english}8.} \zh{8。} 
\lhead{\firstmark}
\rhead{\botmark}

\subsection{\hspace{-0.5cm} {\Large \textcolor{darkblue}{\textbf{\ipa{hu˥}}}}\hspace{0.5cm}[\kern2pt{\textcolor{darkblue}{\textbf{\ipa{hu˧˥}}}}\kern2pt]} \hypertarget{hu\string_T1}{}
\markboth{\textcolor{darkblue}{\textbf{\ipa{hu˥}}}}{}
\textcolor{teal}{\mytextsc{verb}} \hspace{4pt} Tone: H.
\textcolor{Sepia}{\selectlanguage{english}To wait.} \zh{等候。}  ¶ \textcolor{darkblue}{\textbf{\ipa{le˧-hu˥-ze˩}}} \textcolor{Sepia}{\selectlanguage{english}\mytextsc{accomp} \string_ \mytextsc{pfv}} \zh{等了}  
 ¶ \textcolor{darkblue}{\textbf{\ipa{ɖɯ˧-hu˥ / ɖɯ˧-hu˧-ɻ̍˥}}} \textcolor{Sepia}{\selectlanguage{english}to wait a while / Wait a while!} \zh{等一下 / 请等一下!}  
 ¶ \textcolor{darkblue}{\textbf{\ipa{hĩ˧ hu˧}}} \textcolor{Sepia}{\selectlanguage{english}to wait for someone} \zh{等人}  

\lhead{\firstmark}
\rhead{\botmark}

\subsection{\hspace{-0.5cm} {\Large \textcolor{darkblue}{\textbf{\ipa{hu˧mi˥\$}}}}\hspace{0.5cm}[\kern2pt{\textcolor{darkblue}{\textbf{\ipa{hu˧mi˧}}}}\kern2pt]} \hypertarget{hu\string_Mmi\string_T\$1}{}
\markboth{\textcolor{darkblue}{\textbf{\ipa{hu˧mi˥\$}}}}{}
\textcolor{teal}{\mytextsc{noun}} \hspace{4pt} Tone: H\$.
\textcolor{Sepia}{\selectlanguage{english}Stomach.} \zh{胃。}  \zh{量词}: \textcolor{darkblue}{\textbf{\ipa{ɭɯ˧}}}  \mytextsc{clf}: \textcolor{darkblue}{\textbf{\ipa{ɭɯ˧}}} 
\lhead{\firstmark}
\rhead{\botmark}

\subsection{\hspace{-0.5cm} {\Large \textcolor{darkblue}{\textbf{\ipa{hu˧˥}}} \textsubscript{1}}\hspace{0.5cm}[\kern2pt{\textcolor{darkblue}{\textbf{\ipa{hu˥}}}}\kern2pt]} \hypertarget{hu\string_M\string_T1}{}
\markboth{\textcolor{darkblue}{\textbf{\ipa{hu˧˥}}} \textsubscript{1}}{}
\textcolor{teal}{\mytextsc{verb}} \hspace{4pt} Tone: MH.
\textcolor{Sepia}{\selectlanguage{english}To miss, to long for, to have the nostalgia of.} \zh{想念。}  ¶ \textcolor{darkblue}{\textbf{\ipa{ə˧mi˧ hu˧˥}}} \textcolor{Sepia}{\selectlanguage{english}to miss (one's) mother} \zh{想念母亲}  

\lhead{\firstmark}
\rhead{\botmark}

\subsection{\hspace{-0.5cm} {\Large \textcolor{darkblue}{\textbf{\ipa{hu˧˥}}} \textsubscript{2}}\hspace{0.5cm}[\kern2pt{\textcolor{darkblue}{\textbf{\ipa{hu˧˥}}}}\kern2pt]} \hypertarget{hu\string_M\string_T2}{}
\markboth{\textcolor{darkblue}{\textbf{\ipa{hu˧˥}}} \textsubscript{2}}{}
\textcolor{teal}{\mytextsc{noun}} \hspace{4pt} Tone: MH.
\textcolor{Sepia}{\selectlanguage{english}Entrails.} \zh{内脏:胃、肠子等。}  \zh{量词}: \textcolor{darkblue}{\textbf{\ipa{ɭɯ˧}}}  \mytextsc{clf}: \textcolor{darkblue}{\textbf{\ipa{ɭɯ˧}}} 
\lhead{\firstmark}
\rhead{\botmark}

\subsection{\hspace{-0.5cm} {\Large \textcolor{darkblue}{\textbf{\ipa{hɯ˧}}}}\hspace{0.5cm}[\kern2pt{\textcolor{darkblue}{\textbf{\ipa{hɯ˥}}}}\kern2pt]} \hypertarget{hM\string_M1}{}
\markboth{\textcolor{darkblue}{\textbf{\ipa{hɯ˧}}}}{}
\textcolor{teal}{\mytextsc{verb}} \hspace{4pt} Tone: M\textsubscript{c}.
\textcolor{Sepia}{\selectlanguage{english}To go, past form.} \zh{走(过去式)。}  ¶ \textcolor{darkblue}{\textbf{\ipa{(ki˧zo˧) | lo˧ ʝi˧-hɯ˧(-ze˩)!}}} \textcolor{Sepia}{\selectlanguage{english}Kizo has gone to work!} \zh{给若(人名)干活去了!}  
 ¶ \textcolor{darkblue}{\textbf{\ipa{le˧-hɯ˩-hĩ˩ hĩ˩}}} \textcolor{Sepia}{\selectlanguage{english}euphemism for 'deceased person': literally 'person who has gone'} \zh{委婉语:‘走了的人’=去世了的人}  

\lhead{\firstmark}
\rhead{\botmark}

\subsection{\hspace{-0.5cm} {\Large \textcolor{darkblue}{\textbf{\ipa{hɯ˧\textsubscript{b}}}}}\hspace{0.5cm}[\kern2pt{\textcolor{darkblue}{\textbf{\ipa{hɯ˥}}}}\kern2pt]} \hypertarget{hM\string_Mb1}{}
\markboth{\textcolor{darkblue}{\textbf{\ipa{hɯ˧\textsubscript{b}}}}}{}
\textcolor{teal}{\mytextsc{classifier}} \hspace{4pt} Tone: M\textsubscript{b}.
\textcolor{Sepia}{\selectlanguage{english}A few, some, a little.} \zh{量词:一点。}  ¶ \textcolor{darkblue}{\textbf{\ipa{ʈʂʰæ˧ɣɯ˧ ɖɯ˧-hɯ˧}}} \textcolor{Sepia}{\selectlanguage{english}some medicines, a few medicines} \zh{一些药}  

\lhead{\firstmark}
\rhead{\botmark}

\subsection{\hspace{-0.5cm} {\Large \textcolor{darkblue}{\textbf{\ipa{hṽ˥}}}}\hspace{0.5cm}[\kern2pt{\textcolor{darkblue}{\textbf{\ipa{hṽ˥}}}}\kern2pt]} \hypertarget{hv\string_~\string_T1}{}
\markboth{\textcolor{darkblue}{\textbf{\ipa{hṽ˥}}}}{}
\textcolor{teal}{\mytextsc{verb}} \hspace{4pt} Tone: H.
\textcolor{Sepia}{\selectlanguage{english}To stir-fry.} \zh{炒(肉、菜)。}  ¶ \textcolor{darkblue}{\textbf{\ipa{hṽ˧\textasciitilde{}hṽ˧}}} \textcolor{Sepia}{\selectlanguage{english}\mytextsc{red}} \zh{重叠}  
 ¶ \textcolor{darkblue}{\textbf{\ipa{le˧-hṽ˧\textasciitilde{}hṽ˧}}} \textcolor{Sepia}{\selectlanguage{english}\mytextsc{accomp} \mytextsc{red}} \zh{\mytextsc{accomp} \mytextsc{red}}  
 ¶ \textcolor{darkblue}{\textbf{\ipa{hṽ˧\textasciitilde{}hṽ˧-ze˩}}} \textcolor{Sepia}{\selectlanguage{english}\mytextsc{red} \mytextsc{pfv}} \zh{炒了}  
 ¶ \textcolor{darkblue}{\textbf{\ipa{ʂe˧ hṽ˧\textasciitilde{}hṽ˧}}} \textcolor{Sepia}{\selectlanguage{english}to stir-fry some meat} \zh{炒肉}  
 ¶ \textcolor{darkblue}{\textbf{\ipa{v̩˩tsʰɤ˧ hṽ˧\textasciitilde{}hṽ˧}}} \textcolor{Sepia}{\selectlanguage{english}to stir-fry some vegetables} \zh{炒蔬菜}  
 ¶ \textcolor{darkblue}{\textbf{\ipa{læ˧tsɯ˥ hṽ˩\textasciitilde{}hṽ˩}}} \textcolor{Sepia}{\selectlanguage{english}to stir-fry chili peppers} \zh{炒辣椒}  
 ¶ \textcolor{darkblue}{\textbf{\ipa{hɑ˧ hṽ˧\textasciitilde{}hṽ˧}}} \textcolor{Sepia}{\selectlanguage{english}to stir-fry some rice, to stir-fry some food} \zh{炒饭}  

\lhead{\firstmark}
\rhead{\botmark}

\subsection{\hspace{-0.5cm} {\Large \textcolor{darkblue}{\textbf{\ipa{hṽ˥}}}}\hspace{0.5cm}[\kern2pt{\textcolor{darkblue}{\textbf{\ipa{hṽ˥}}}}\kern2pt]} \hypertarget{hv\string_~\string_T1}{}
\markboth{\textcolor{darkblue}{\textbf{\ipa{hṽ˥}}}}{}
\textcolor{teal}{\mytextsc{noun}} \hspace{4pt} Tone: \#H.
\textcolor{Sepia}{\selectlanguage{english}Body hair; animal's hair; porcupine's quills.} \zh{毛、豪猪的翎。}  \zh{量词}: \textcolor{darkblue}{\textbf{\ipa{kʰɯ˩}}}  \mytextsc{clf}: \textcolor{darkblue}{\textbf{\ipa{kʰɯ˩}}} 
\lhead{\firstmark}
\rhead{\botmark}

\subsection{\hspace{-0.5cm} {\Large \textcolor{darkblue}{\textbf{\ipa{hṽ˧dɤ˧ɻ\#˥}}}}\hspace{0.5cm}[\kern2pt{\textcolor{darkblue}{\textbf{\ipa{hṽ˩dɤ˩ɻ˩˥}}}}\kern2pt]} \hypertarget{hv\string_~\string_Md7\string_Mr£`\#\string_T1}{}
\markboth{\textcolor{darkblue}{\textbf{\ipa{hṽ˧dɤ˧ɻ\#˥}}}}{}
\textcolor{teal}{\mytextsc{adjective}} \hspace{4pt} Tone: \#H.
\textcolor{Sepia}{\selectlanguage{english}Clumsy, incapable.} \zh{笨拙,经常损坏东西。}  ¶ \textcolor{darkblue}{\textbf{\ipa{hṽ˩-hĩ˩˥}}} \textcolor{Sepia}{\selectlanguage{english}\mytextsc{rel}} \zh{笨拙的}  
 ¶ \textcolor{darkblue}{\textbf{\ipa{hṽ˧dɤ˧ɻ̍˧\textasciitilde{}hṽ˧dɤ˧ɻ̍˧-zo˥}}} \textcolor{Sepia}{\selectlanguage{english}clumsy person / clumsy boy} \zh{笨手笨脚的(男)人}  
 ¶ \textcolor{darkblue}{\textbf{\ipa{[F5] hṽ˩ɖɻ̍˩\textasciitilde{}hṽ˧ɖɻ̍˧-gv̩˧}}} \textcolor{Sepia}{\selectlanguage{english}\mytextsc{red}} \zh{重叠:笨笨的}  

\lhead{\firstmark}
\rhead{\botmark}

\subsection{\hspace{-0.5cm} {\Large \textcolor{darkblue}{\textbf{\ipa{hṽ˧\textasciitilde{}hṽ˩-ɖʐæ˩\textasciitilde{}ɖʐæ˩}}}}\hspace{0.5cm}[\kern2pt{\textcolor{darkblue}{\textbf{\ipa{xxxx non-correspondance entre le nombre de morphèmes et le nombre de tons de morphèmes}}}}\kern2pt]} \hypertarget{hv\string_~\string_M~hv\string_~\string_B-d`z`\{\string_B~d`z`\{\string_B1}{}
\markboth{\textcolor{darkblue}{\textbf{\ipa{hṽ˧\textasciitilde{}hṽ˩-ɖʐæ˩\textasciitilde{}ɖʐæ˩}}}}{}
\textcolor{teal}{\mytextsc{adjective}} \hspace{4pt} Tone: L\#-.
\textit{From:} \textbf{hṽ˩a 1} \textcolor{Sepia}{\selectlanguage{english}Intensely red, red all over.} \zh{红红的。} 
\lhead{\firstmark}
\rhead{\botmark}

\subsection{\hspace{-0.5cm} {\Large \textcolor{darkblue}{\textbf{\ipa{hṽ˧nɑ˩}}}}\hspace{0.5cm}[\kern2pt{\textcolor{darkblue}{\textbf{\ipa{hṽ˩nɑ˥}}}}\kern2pt]} \hypertarget{hv\string_~\string_MnA\string_B1}{}
\markboth{\textcolor{darkblue}{\textbf{\ipa{hṽ˧nɑ˩}}}}{}
\textcolor{teal}{\mytextsc{noun}} \hspace{4pt} Tone: L\#.
\textcolor{Sepia}{\selectlanguage{english}Wild animal.} \zh{野兽。}  \zh{量词}: \textcolor{darkblue}{\textbf{\ipa{mi˩}}}  \mytextsc{clf}: \textcolor{darkblue}{\textbf{\ipa{mi˩}}} 
\lhead{\firstmark}
\rhead{\botmark}

\subsection{\hspace{-0.5cm} {\Large \textcolor{darkblue}{\textbf{\ipa{hṽ˩\textsubscript{a}}}} \textsubscript{1}}\hspace{0.5cm}[\kern2pt{\textcolor{darkblue}{\textbf{\ipa{hṽ˥}}}}\kern2pt]} \hypertarget{hv\string_~\string_Ba1}{}
\markboth{\textcolor{darkblue}{\textbf{\ipa{hṽ˩\textsubscript{a}}}} \textsubscript{1}}{}
\textcolor{teal}{\mytextsc{adjective}} \hspace{4pt} Tone: L\textsubscript{a}.
\textcolor{Sepia}{\selectlanguage{english}Red.} \zh{红色的。} 
\lhead{\firstmark}
\rhead{\botmark}

\subsection{\hspace{-0.5cm} {\Large \textcolor{darkblue}{\textbf{\ipa{hṽ˩\textsubscript{a}}}} \textsubscript{2}}\hspace{0.5cm}[\kern2pt{\textcolor{darkblue}{\textbf{\ipa{hṽ˩˥}}}}\kern2pt]} \hypertarget{hv\string_~\string_Ba2}{}
\markboth{\textcolor{darkblue}{\textbf{\ipa{hṽ˩\textsubscript{a}}}} \textsubscript{2}}{}
\textcolor{teal}{\mytextsc{adjective}} \hspace{4pt} Tone: L\textsubscript{a}.
\textcolor{Sepia}{\selectlanguage{english}Of small stature, small in stature, short, not tall; low.} \zh{矮、低。} 
\lhead{\firstmark}
\rhead{\botmark}

\subsection{\hspace{-0.5cm} {\Large \textcolor{darkblue}{\textbf{\ipa{hṽ˩-ɖʐæ˩ɻæ˥}}}}\hspace{0.5cm}[\kern2pt{\textcolor{darkblue}{\textbf{\ipa{xxxx non-correspondance entre le nombre de morphèmes et le nombre de tons de morphèmes}}}}\kern2pt]} \hypertarget{hv\string_~\string_B-d`z`\{\string_Br£`\{\string_T1}{}
\markboth{\textcolor{darkblue}{\textbf{\ipa{hṽ˩-ɖʐæ˩ɻæ˥}}}}{}
\textcolor{teal}{\mytextsc{adjective}} \hspace{4pt} Tone: L+H\#.
\textit{From:} \textbf{hṽ˩a 1} \textcolor{Sepia}{\selectlanguage{english}Intensely red, red all over.} \zh{红红的。}  ¶ \textcolor{darkblue}{\textbf{\ipa{hṽ˩ɖʐæ˩ɻæ˥-gv̩˩}}} \textcolor{Sepia}{\selectlanguage{english}intensely red, red all over} \zh{红红的}  
 ¶ \textcolor{darkblue}{\textbf{\ipa{[F5] hṽ˩ɖʐæ˩˥ | hṽ˩ɖʐæ˩˥ gv̩˩}}} \textcolor{Sepia}{\selectlanguage{english}\mytextsc{red;} the first two syllables are higher-pitched than the following two} \zh{重叠}  

\lhead{\firstmark}
\rhead{\botmark}

\subsection{\hspace{-0.5cm} {\Large \textcolor{darkblue}{\textbf{\ipa{hṽ˩\textasciitilde{}hṽ˩}}}}\hspace{0.5cm}[\kern2pt{\textcolor{darkblue}{\textbf{\ipa{hṽ˧hṽ˩}}}}\kern2pt]} \hypertarget{hv\string_~\string_B~hv\string_~\string_B1}{}
\markboth{\textcolor{darkblue}{\textbf{\ipa{hṽ˩\textasciitilde{}hṽ˩}}}}{}
\textcolor{teal}{\mytextsc{verb}} \hspace{4pt} Tone: L.
\textcolor{Sepia}{\selectlanguage{english}To disturb, to interfere, to hinder, to obstruct, to impede.} \zh{干扰、防碍。}  ¶ \textcolor{darkblue}{\textbf{\ipa{hĩ˧ hṽ˥\textasciitilde{}hṽ˩}}} \textcolor{Sepia}{\selectlanguage{english}to annoy people} \zh{干扰人家}  

\lhead{\firstmark}
\rhead{\botmark}

\subsection{\hspace{-0.5cm} {\Large \textcolor{darkblue}{\textbf{\ipa{hwɑ˩kwɤ˧}}}}\hspace{0.5cm}[\kern2pt{\textcolor{darkblue}{\textbf{\ipa{hwɑ˧kwɤ˥}}}}\kern2pt]} \hypertarget{hwA\string_Bkw7\string_M1}{}
\markboth{\textcolor{darkblue}{\textbf{\ipa{hwɑ˩kwɤ˧}}}}{}
\textcolor{teal}{\mytextsc{noun}} \hspace{4pt} Tone: LM.
\textcolor{Sepia}{\selectlanguage{english}Cucumber.} \zh{黄瓜(汉语借词)。}  Borrowing: Chinese  \zh{黄瓜}
 \zh{量词}: \textcolor{darkblue}{\textbf{\ipa{ɭɯ˧}}}  \mytextsc{clf}: \textcolor{darkblue}{\textbf{\ipa{ɭɯ˧}}} 
\lhead{\firstmark}
\rhead{\botmark}

\subsection{\hspace{-0.5cm} {\Large \textcolor{darkblue}{\textbf{\ipa{hwæ˧\textsubscript{a}}}}}\hspace{0.5cm}[\kern2pt{\textcolor{darkblue}{\textbf{\ipa{hwæ˩˥}}}}\kern2pt]} \hypertarget{hw\{\string_Ma1}{}
\markboth{\textcolor{darkblue}{\textbf{\ipa{hwæ˧\textsubscript{a}}}}}{}
\textcolor{teal}{\mytextsc{verb}} \hspace{4pt} Tone: M\textsubscript{a}.
\textcolor{Sepia}{\selectlanguage{english}To buy.} \zh{买。}  ¶ \textcolor{darkblue}{\textbf{\ipa{le˧-hwæ˧}}} \textcolor{Sepia}{\selectlanguage{english}\mytextsc{accomp}} \zh{\mytextsc{accomp}}  
 ¶ \textcolor{darkblue}{\textbf{\ipa{tso˧\textasciitilde{}tso˧ hwæ˩}}} \textcolor{Sepia}{\selectlanguage{english}to buy things} \zh{买东西}  
 ¶ \textcolor{darkblue}{\textbf{\ipa{ɖɯ˧-kʰwɤ˥ hwæ˩}}} \textcolor{Sepia}{\selectlanguage{english}to buy a piece (of something)} \zh{买一块}  
 ¶ \textcolor{darkblue}{\textbf{\ipa{hwæ˧\textasciitilde{}hwæ˩}}} \textcolor{Sepia}{\selectlanguage{english}\mytextsc{red}} \zh{\mytextsc{重叠}}  

\lhead{\firstmark}
\rhead{\botmark}

\subsection{\hspace{-0.5cm} {\Large \textcolor{darkblue}{\textbf{\ipa{hwæ˧ɖʐæ˥}}} \textsubscript{1}}\hspace{0.5cm}[\kern2pt{\textcolor{darkblue}{\textbf{\ipa{hwæ˧ɖʐæ˧}}}}\kern2pt]} \hypertarget{hw\{\string_Md`z`\{\string_T1}{}
\markboth{\textcolor{darkblue}{\textbf{\ipa{hwæ˧ɖʐæ˥}}} \textsubscript{1}}{}
\textcolor{teal}{\mytextsc{noun}} \hspace{4pt} Tone: H\#.
\textcolor{Sepia}{\selectlanguage{english}Squirrel.} \zh{松鼠,灰鼠。}  ¶ \textcolor{darkblue}{\textbf{\ipa{hwæ˧ɖʐæ˥-pʰv̩˩}}} \textcolor{Sepia}{\selectlanguage{english}male squirrel} \zh{公松鼠}  
 ¶ \textcolor{darkblue}{\textbf{\ipa{hwæ˧ɖʐæ˥-mi˩}}} \textcolor{Sepia}{\selectlanguage{english}female squirrel} \zh{母松鼠}  
 \zh{量词}: \textcolor{darkblue}{\textbf{\ipa{mi˩}}}  \mytextsc{clf}: \textcolor{darkblue}{\textbf{\ipa{mi˩}}} 
\lhead{\firstmark}
\rhead{\botmark}

\subsection{\hspace{-0.5cm} {\Large \textcolor{darkblue}{\textbf{\ipa{hwæ˧ɖʐæ˥}}} \textsubscript{2}}\hspace{0.5cm}[\kern2pt{\textcolor{darkblue}{\textbf{\ipa{hwæ˧ɖʐæ˥}}}}\kern2pt]} \hypertarget{hw\{\string_Md`z`\{\string_T2}{}
\markboth{\textcolor{darkblue}{\textbf{\ipa{hwæ˧ɖʐæ˥}}} \textsubscript{2}}{}
\textcolor{teal}{\mytextsc{noun}} \hspace{4pt} Tone: H\#.
\textcolor{Sepia}{\selectlanguage{english}Wart.} \zh{瘊子、肉赘。}  ¶ \textcolor{darkblue}{\textbf{\ipa{hwæ˧ʈʂæ˥ tʰv̩˩}}} \textcolor{Sepia}{\selectlanguage{english}a wart forms} \zh{长瘊子}  
 ¶ \textcolor{darkblue}{\textbf{\ipa{hwæ˧ʈʂæ˥ | le˧-tʰv̩˧-ze˧!}}} \textcolor{Sepia}{\selectlanguage{english}A wart has formed!} \zh{长瘊子了!}  
 \zh{量词}: \textcolor{darkblue}{\textbf{\ipa{mi˩}}}  \mytextsc{clf}: \textcolor{darkblue}{\textbf{\ipa{mi˩}}} 
\lhead{\firstmark}
\rhead{\botmark}

\subsection{\hspace{-0.5cm} {\Large \textcolor{darkblue}{\textbf{\ipa{hwæ˧pʰæ˥}}}}\hspace{0.5cm}[\kern2pt{\textcolor{darkblue}{\textbf{\ipa{hwæ˧pʰæ˩}}}}\kern2pt]} \hypertarget{hw\{\string_Mp\string_h\{\string_T1}{}
\markboth{\textcolor{darkblue}{\textbf{\ipa{hwæ˧pʰæ˥}}}}{}
\textcolor{teal}{\mytextsc{noun}} \hspace{4pt} Tone: H\#.
\textcolor{Sepia}{\selectlanguage{english}A piece of cloth.} \zh{一块布。}  \zh{量词}: \textcolor{darkblue}{\textbf{\ipa{pʰæ˧˥}}}  \mytextsc{clf}: \textcolor{darkblue}{\textbf{\ipa{pʰæ˧˥}}} 
\lhead{\firstmark}
\rhead{\botmark}

\subsection{\hspace{-0.5cm} {\Large \textcolor{darkblue}{\textbf{\ipa{hwæ˧pʰæ˩}}}}\hspace{0.5cm}[\kern2pt{\textcolor{darkblue}{\textbf{\ipa{hwæ˩pʰæ˥}}}}\kern2pt]} \hypertarget{hw\{\string_Mp\string_h\{\string_B1}{}
\markboth{\textcolor{darkblue}{\textbf{\ipa{hwæ˧pʰæ˩}}}}{}
\textcolor{teal}{\mytextsc{noun}} \hspace{4pt} Tone: L\#.
\textcolor{Sepia}{\selectlanguage{english}Large hoe.} \zh{大锄。} Local Chinese dialect:\zh{挖锄。} ¶ \textcolor{darkblue}{\textbf{\ipa{hwæ˧pʰæ˩ tʰv̩˩-nɑ˩}}} \textcolor{Sepia}{\selectlanguage{english}\mytextsc{n}+\mytextsc{dem}+\mytextsc{clf}} \zh{这把大锄}  
 \zh{量词}: \textcolor{darkblue}{\textbf{\ipa{nɑ˧}}}  \mytextsc{clf}: \textcolor{darkblue}{\textbf{\ipa{nɑ˧}}} 
\lhead{\firstmark}
\rhead{\botmark}

\subsection{\hspace{-0.5cm} {\Large \textcolor{darkblue}{\textbf{\ipa{hwæ˧pʰæ˩-gv̩˩-di˩}}}}\hspace{0.5cm}[\kern2pt{\textcolor{darkblue}{\textbf{\ipa{xxxx non-correspondance entre le nombre de morphèmes et le nombre de tons de morphèmes}}}}\kern2pt]} \hypertarget{hw\{\string_Mp\string_h\{\string_B-gv\string_=\string_B-di\string_B1}{}
\markboth{\textcolor{darkblue}{\textbf{\ipa{hwæ˧pʰæ˩-gv̩˩-di˩}}}}{}
\textcolor{teal}{\mytextsc{noun}} \hspace{4pt} Tone: L\#--.
\textcolor{Sepia}{\selectlanguage{english}Loom.} \zh{织布机。}  ¶ \textcolor{darkblue}{\textbf{\ipa{hwæ˧pʰæ˩gv̩˩di˩-tɕi˩tɕʰi˧}}} \textcolor{Sepia}{\selectlanguage{english}industrial sewing machine (formed of the Na word plus the Chinese word for 'engine')} \zh{工业织布机。直译:“织布机器”(在摩梭词后面加上汉语的“机器”)}  
 \zh{量词}: \textcolor{darkblue}{\textbf{\ipa{nɑ˧}}}  \mytextsc{clf}: \textcolor{darkblue}{\textbf{\ipa{nɑ˧}}} 
\lhead{\firstmark}
\rhead{\botmark}

\subsection{\hspace{-0.5cm} {\Large \textcolor{darkblue}{\textbf{\ipa{hwæ˧tsɯ˥}}}}\hspace{0.5cm}[\kern2pt{\textcolor{darkblue}{\textbf{\ipa{xxxx non-correspondance entre le nombre de morphèmes et le nombre de tons de morphèmes}}}}\kern2pt]} \hypertarget{hw\{\string_MtsM\string_T1}{}
\markboth{\textcolor{darkblue}{\textbf{\ipa{hwæ˧tsɯ˥}}}}{}
\textcolor{teal}{\mytextsc{noun}} \hspace{4pt} Tone: H\#.
\textcolor{Sepia}{\selectlanguage{english}Rat.} \zh{老鼠(汉语借词)。} Local Chinese dialect:\zh{耗子。} Borrowing: Chinese  \zh{耗子}
 ¶ \textcolor{darkblue}{\textbf{\ipa{hwæ˧tsɯ˥-pʰv̩˩}}} \textcolor{Sepia}{\selectlanguage{english}male rat} \zh{公老鼠}  
 ¶ \textcolor{darkblue}{\textbf{\ipa{hwæ˧tsɯ˥-mi˩}}} \textcolor{Sepia}{\selectlanguage{english}female rat} \zh{母老鼠}  
 \zh{量词}: \textcolor{darkblue}{\textbf{\ipa{mi˩}}}  \mytextsc{clf}: \textcolor{darkblue}{\textbf{\ipa{mi˩}}} 
\lhead{\firstmark}
\rhead{\botmark}

\subsection{\hspace{-0.5cm} {\Large \textcolor{darkblue}{\textbf{\ipa{hwæ˧tsɯ˥-njɤ˩di˩}}}}\hspace{0.5cm}[\kern2pt{\textcolor{darkblue}{\textbf{\ipa{xxxx non-correspondance entre le nombre de morphèmes et le nombre de tons de morphèmes}}}}\kern2pt]} \hypertarget{hw\{\string_MtsM\string_T-nj7\string_Bdi\string_B1}{}
\markboth{\textcolor{darkblue}{\textbf{\ipa{hwæ˧tsɯ˥-njɤ˩di˩}}}}{}
\textcolor{teal}{\mytextsc{noun}} \hspace{4pt} Tone: H\#-.
\textcolor{Sepia}{\selectlanguage{english}Setose thisle.} \zh{大蓟。}  \zh{量词}: \textcolor{darkblue}{\textbf{\ipa{dzi˩}}}  \mytextsc{clf}: \textcolor{darkblue}{\textbf{\ipa{dzi˩}}} 
\lhead{\firstmark}
\rhead{\botmark}

\subsection{\hspace{-0.5cm} {\Large \textcolor{darkblue}{\textbf{\ipa{hwæ˧tsɯ˥-njɤ˩di˩-si˩dzi˩}}}}\hspace{0.5cm}[\kern2pt{\textcolor{darkblue}{\textbf{\ipa{xxxx non-correspondance entre le nombre de morphèmes et le nombre de tons de morphèmes}}}}\kern2pt]} \hypertarget{hw\{\string_MtsM\string_T-nj7\string_Bdi\string_B-si\string_Bdzi\string_B1}{}
\markboth{\textcolor{darkblue}{\textbf{\ipa{hwæ˧tsɯ˥-njɤ˩di˩-si˩dzi˩}}}}{}
\textcolor{teal}{\mytextsc{noun}} \hspace{4pt} Tone: H\#--.
\textcolor{Sepia}{\selectlanguage{english}Burdock.} \zh{牛蒡。} Local Chinese dialect:\zh{牛蒡子。}
\lhead{\firstmark}
\rhead{\botmark}

\subsection{\hspace{-0.5cm} {\Large \textcolor{darkblue}{\textbf{\ipa{hwæ˩\textsubscript{a}}}} \textsubscript{1}}\hspace{0.5cm}[\kern2pt{\textcolor{darkblue}{\textbf{\ipa{xxxx non-correspondance entre le nombre de morphèmes et le nombre de tons de morphèmes}}}}\kern2pt]} \hypertarget{hw\{\string_Ba1}{}
\markboth{\textcolor{darkblue}{\textbf{\ipa{hwæ˩\textsubscript{a}}}} \textsubscript{1}}{}
\textcolor{teal}{\mytextsc{verb}} \hspace{4pt} Tone: L\textsubscript{a}.
\textcolor{Sepia}{\selectlanguage{english}To close the door (from outside).} \zh{关(出门,就关门)。}  ¶ \textcolor{darkblue}{\textbf{\ipa{kʰi˧ | tʰi˧-hwæ˩!}}} \textcolor{Sepia}{\selectlanguage{english}Close the door!} \zh{关门吧!}  
 ¶ \textcolor{darkblue}{\textbf{\ipa{ʂe˧bæ˧ | le˧-wo˧-hwæ˥}}}  
 ¶ \textcolor{darkblue}{\textbf{\ipa{kʰi˧-bi˥ di˩-hĩ˩ ʂe˩bæ˩}}}  
 ¶  
\textit{See:} \hyperlink{}{\textcolor{darkblue}{\textbf{\ipa{ʈæ˩\textsubscript{a}}}}} 
\lhead{\firstmark}
\rhead{\botmark}

\subsection{\hspace{-0.5cm} {\Large \textcolor{darkblue}{\textbf{\ipa{hwæ˩\textsubscript{a}}}} \textsubscript{2}}\hspace{0.5cm}[\kern2pt{\textcolor{darkblue}{\textbf{\ipa{hwæ˩˥}}}}\kern2pt]} \hypertarget{hw\{\string_Ba2}{}
\markboth{\textcolor{darkblue}{\textbf{\ipa{hwæ˩\textsubscript{a}}}} \textsubscript{2}}{}
\textcolor{teal}{\mytextsc{verb}} \hspace{4pt} Tone: L\textsubscript{a}.
\textcolor{Sepia}{\selectlanguage{english}To suspend, to hang (in a place).} \zh{悬挂、挂在墙上。}  ¶ \textcolor{darkblue}{\textbf{\ipa{tso˧\textasciitilde{}tso˧ | gɤ˧bi˧ hwæ˥}}} \textcolor{Sepia}{\selectlanguage{english}to suspend things up high (e.g. on a hook)} \zh{挂东西在上面}  
 ¶ \textcolor{darkblue}{\textbf{\ipa{tso˧\textasciitilde{}tso˧ hwæ˥}}} \textcolor{Sepia}{\selectlanguage{english}to suspend things} \zh{挂东西}  
 ¶ \textcolor{darkblue}{\textbf{\ipa{ʂe˧ | tʰi˧-hwæ˩}}} \textcolor{Sepia}{\selectlanguage{english}to hang meat (above the hearth, to smoke it)} \zh{挂肉(在火塘上,为了熏肉)}  
 ¶ \textcolor{darkblue}{\textbf{\ipa{tso˧\textasciitilde{}tso˧ | tʰi˧-hwæ˩}}} \textcolor{Sepia}{\selectlanguage{english}to suspend things} \zh{挂东西}  
 ¶ \textcolor{darkblue}{\textbf{\ipa{ʂe˧-hwæ˥-di˩}}} \textcolor{Sepia}{\selectlanguage{english}A small beam in the main room (right under the main beams) where meat is suspended to smoke it. The word literally means “thing to hang meat”.} \zh{主屋里面的小梁(大梁下面),用来挂肉,熏肉。直译:“挂肉的东西”。}  
 ¶ \textcolor{darkblue}{\textbf{\ipa{tso˧\textasciitilde{}tso˧-hwæ˥-di˩}}} \textcolor{Sepia}{\selectlanguage{english}object used to suspend things; this can refer to any object from a hook to a beam on which things are hung} \zh{挂(东西)用的(东西),如:钩子、用来挂肉的小梁……}  

\lhead{\firstmark}
\rhead{\botmark}

\subsection{\hspace{-0.5cm} {\Large \textcolor{darkblue}{\textbf{\ipa{hwɤ̃˩\textsubscript{a}}}}}\hspace{0.5cm}[\kern2pt{\textcolor{darkblue}{\textbf{\ipa{hwɤ̃˥}}}}\kern2pt]} \hypertarget{hw7\string_~\string_Ba1}{}
\markboth{\textcolor{darkblue}{\textbf{\ipa{hwɤ̃˩\textsubscript{a}}}}}{}
\textcolor{teal}{\mytextsc{adjective}} \hspace{4pt} Tone: L\textsubscript{a}.
\textcolor{Sepia}{\selectlanguage{english}Late.} \zh{迟,晚。}  ¶ \textcolor{darkblue}{\textbf{\ipa{hwɤ̃˩-hĩ˩˥}}} \textcolor{Sepia}{\selectlanguage{english}\mytextsc{rel}/\mytextsc{nmlz}} \zh{迟的}  
 ¶ \textcolor{darkblue}{\textbf{\ipa{ʈʂʰɯ˧ ʑi˧-ʈi˥ hwɤ̃˩!}}} \textcolor{Sepia}{\selectlanguage{english}He/she gets up late!} \zh{他起床起得晚!}  

\lhead{\firstmark}
\rhead{\botmark}

\subsection{\hspace{-0.5cm} {\Large \textcolor{darkblue}{\textbf{\ipa{hwɤ˥}}}}\hspace{0.5cm}[\kern2pt{\textcolor{darkblue}{\textbf{\ipa{hwɤ˥}}}}\kern2pt]} \hypertarget{hw7\string_T1}{}
\markboth{\textcolor{darkblue}{\textbf{\ipa{hwɤ˥}}}}{}
\textcolor{teal}{\mytextsc{noun}} \hspace{4pt} Tone: \#H.
\textcolor{Sepia}{\selectlanguage{english}Gift made on important occasions (weddings, etc).} \zh{在大事发生的时候送的礼物(近期一般给钱):婚礼、葬礼。}  ¶ \textcolor{darkblue}{\textbf{\ipa{hwɤ˧ | ɖɯ˧-kʰwɤ˥}}} \textcolor{Sepia}{\selectlanguage{english}a gift, a donation} \zh{一份大礼物}  
 \zh{量词}: \textcolor{darkblue}{\textbf{\ipa{kʰwɤ˥}}}  \mytextsc{clf}: \textcolor{darkblue}{\textbf{\ipa{kʰwɤ˥}}} 
\lhead{\firstmark}
\rhead{\botmark}

\subsection{\hspace{-0.5cm} {\Large \textcolor{darkblue}{\textbf{\ipa{hwɤ˥}}}}\hspace{0.5cm}[\kern2pt{\textcolor{darkblue}{\textbf{\ipa{hwɤ˧˥}}}}\kern2pt]} \hypertarget{hw7\string_T1}{}
\markboth{\textcolor{darkblue}{\textbf{\ipa{hwɤ˥}}}}{}
\textcolor{teal}{\mytextsc{verb}} \hspace{4pt} Tone: H.
\textcolor{Sepia}{\selectlanguage{english}To participate in a funeral ceremony (literally 'to see [the deceased] out').} \zh{执绋送丧。} 
\lhead{\firstmark}
\rhead{\botmark}

\subsection{\hspace{-0.5cm} {\Large \textcolor{darkblue}{\textbf{\ipa{hwɤ˧}}}}\hspace{0.5cm}[\kern2pt{\textcolor{darkblue}{\textbf{\ipa{hwɤ˩˥}}}}\kern2pt]} \hypertarget{hw7\string_M1}{}
\markboth{\textcolor{darkblue}{\textbf{\ipa{hwɤ˧}}}}{}
\textcolor{teal}{\mytextsc{adjective}} \hspace{4pt} Tone: M.
\textcolor{Sepia}{\selectlanguage{english}Broad, vast, extensive; big (plain; piece of cloth, vegetable…).} \zh{宽,辽阔,宽敞。}  ¶ \textcolor{darkblue}{\textbf{\ipa{qʰɑ˧-hwɤ˧-gv̩˧}}} \textcolor{Sepia}{\selectlanguage{english}extremely vast} \zh{非常宽敞}  

\lhead{\firstmark}
\rhead{\botmark}

\subsection{\hspace{-0.5cm} {\Large \textcolor{darkblue}{\textbf{\ipa{hwɤ˧kʰv̩˥}}}}\hspace{0.5cm}[\kern2pt{\textcolor{darkblue}{\textbf{\ipa{hwɤ˩kʰv̩˥}}}}\kern2pt]} \hypertarget{hw7\string_Mk\string_hv\string_=\string_T1}{}
\markboth{\textcolor{darkblue}{\textbf{\ipa{hwɤ˧kʰv̩˥}}}}{}
\textcolor{teal}{\mytextsc{noun}} \hspace{4pt} Tone: H\#.
\textcolor{Sepia}{\selectlanguage{english}Year of the Cat (corresponding to the Chinese year of the Rat).} \zh{鼠年(摩梭话称作“猫年”)。} 
\lhead{\firstmark}
\rhead{\botmark}

\subsection{\hspace{-0.5cm} {\Large \textcolor{darkblue}{\textbf{\ipa{hwɤ˧li˧-bv̩˥}}}}\hspace{0.5cm}[\kern2pt{\textcolor{darkblue}{\textbf{\ipa{xxxx non-correspondance entre le nombre de morphèmes et le nombre de tons de morphèmes}}}}\kern2pt]} \hypertarget{hw7\string_Mli\string_M-bv\string_=\string_T1}{}
\markboth{\textcolor{darkblue}{\textbf{\ipa{hwɤ˧li˧-bv̩˥}}}}{}
\textcolor{teal}{\mytextsc{noun}} \hspace{4pt} Tone: H\#.
\textcolor{Sepia}{\selectlanguage{english}Lower balcony, mezzanine.} \zh{夹层:主屋的夹层。因为烟多,所以人不能将这个空间当卧室。只有一层薄的木地板。}  \zh{量词}: \textcolor{darkblue}{\textbf{\ipa{kɤ˧˥}}}  \mytextsc{clf}: \textcolor{darkblue}{\textbf{\ipa{kɤ˧˥}}} \textit{Syn:} \hyperlink{}{\textcolor{darkblue}{\textbf{\ipa{hwɤ˧li˧-se˧-di˧˥}}}}. 
\lhead{\firstmark}
\rhead{\botmark}

\subsection{\hspace{-0.5cm} {\Large \textcolor{darkblue}{\textbf{\ipa{hwɤ˧li˧-hwæ˧qʰæ\#˥}}}}\hspace{0.5cm}[\kern2pt{\textcolor{darkblue}{\textbf{\ipa{xxxx non-correspondance entre le nombre de morphèmes et le nombre de tons de morphèmes}}}}\kern2pt]} \hypertarget{hw7\string_Mli\string_M-hw\{\string_Mq\string_h\{\#\string_T1}{}
\markboth{\textcolor{darkblue}{\textbf{\ipa{hwɤ˧li˧-hwæ˧qʰæ\#˥}}}}{}
\textcolor{teal}{\mytextsc{noun}} \hspace{4pt} Tone: \#H.
\textcolor{Sepia}{\selectlanguage{english}Scabious.} \zh{山萝卜。} 
\lhead{\firstmark}
\rhead{\botmark}

\subsection{\hspace{-0.5cm} {\Large \textcolor{darkblue}{\textbf{\ipa{hwɤ˧li˧-se˧-di˧˥}}}}\hspace{0.5cm}[\kern2pt{\textcolor{darkblue}{\textbf{\ipa{xxxx non-correspondance entre le nombre de morphèmes et le nombre de tons de morphèmes}}}}\kern2pt]} \hypertarget{hw7\string_Mli\string_M-se\string_M-di\string_M\string_T1}{}
\markboth{\textcolor{darkblue}{\textbf{\ipa{hwɤ˧li˧-se˧-di˧˥}}}}{}
\textcolor{teal}{\mytextsc{noun}} \hspace{4pt} Tone: MH\#.
\textcolor{Sepia}{\selectlanguage{english}Lower balcony, mezzanine.} \zh{夹层:主屋的夹层。因为烟多,所以人不能将这个空间当卧室。只有一层薄的木地板。}  \zh{量词}: \textcolor{darkblue}{\textbf{\ipa{kɤ˧˥}}}  \mytextsc{clf}: \textcolor{darkblue}{\textbf{\ipa{kɤ˧˥}}} \textit{Syn:} \hyperlink{}{\textcolor{darkblue}{\textbf{\ipa{hwɤ˧li˧-bv̩˥}}}}. 
\lhead{\firstmark}
\rhead{\botmark}

\subsection{\hspace{-0.5cm} {\Large \textcolor{darkblue}{\textbf{\ipa{hwɤ˧li˧-ʂɯ˧mo˥}}}}\hspace{0.5cm}[\kern2pt{\textcolor{darkblue}{\textbf{\ipa{xxxx non-correspondance entre le nombre de morphèmes et le nombre de tons de morphèmes}}}}\kern2pt]} \hypertarget{hw7\string_Mli\string_M-s`M\string_Mmo\string_T1}{}
\markboth{\textcolor{darkblue}{\textbf{\ipa{hwɤ˧li˧-ʂɯ˧mo˥}}}}{}
\textcolor{teal}{\mytextsc{noun}} \hspace{4pt} Tone: H\#.
\textcolor{Sepia}{\selectlanguage{english}Old cat (male or female).} \zh{老猫(不分公、母)。}  \zh{量词}: \textcolor{darkblue}{\textbf{\ipa{mi˩}}}  \mytextsc{clf}: \textcolor{darkblue}{\textbf{\ipa{mi˩}}} 
\lhead{\firstmark}
\rhead{\botmark}

\subsection{\hspace{-0.5cm} {\Large \textcolor{darkblue}{\textbf{\ipa{hwɤ˧li˧-zo˧˥}}}}\hspace{0.5cm}[\kern2pt{\textcolor{darkblue}{\textbf{\ipa{xxxx non-correspondance entre le nombre de morphèmes et le nombre de tons de morphèmes}}}}\kern2pt]} \hypertarget{hw7\string_Mli\string_M-zo\string_M\string_T1}{}
\markboth{\textcolor{darkblue}{\textbf{\ipa{hwɤ˧li˧-zo˧˥}}}}{}
\textcolor{teal}{\mytextsc{noun}} \hspace{4pt} Tone: MH\#.
\textcolor{Sepia}{\selectlanguage{english}Kitten, cub.} \zh{小猫。}  \zh{量词}: \textcolor{darkblue}{\textbf{\ipa{ɭɯ˧}}}  \mytextsc{clf}: \textcolor{darkblue}{\textbf{\ipa{ɭɯ˧}}} 
\lhead{\firstmark}
\rhead{\botmark}

\subsection{\hspace{-0.5cm} {\Large \textcolor{darkblue}{\textbf{\ipa{hwɤ˧li˧˥}}}}\hspace{0.5cm}[\kern2pt{\textcolor{darkblue}{\textbf{\ipa{hwɤ˧li˥}}}}\kern2pt]} \hypertarget{hw7\string_Mli\string_M\string_T1}{}
\markboth{\textcolor{darkblue}{\textbf{\ipa{hwɤ˧li˧˥}}}}{}
\textcolor{teal}{\mytextsc{noun}} \hspace{4pt} Tone: MH\#.
\textcolor{Sepia}{\selectlanguage{english}Cat.} \zh{猫。}  \zh{量词}: \textcolor{darkblue}{\textbf{\ipa{mi˩}}}  \mytextsc{clf}: \textcolor{darkblue}{\textbf{\ipa{mi˩}}} 
\lhead{\firstmark}
\rhead{\botmark}

\subsection{\hspace{-0.5cm} {\Large \textcolor{darkblue}{\textbf{\ipa{hwɤ˧mi˥\$}}}}\hspace{0.5cm}[\kern2pt{\textcolor{darkblue}{\textbf{\ipa{hwɤ˧mi˧˥}}}}\kern2pt]} \hypertarget{hw7\string_Mmi\string_T\$1}{}
\markboth{\textcolor{darkblue}{\textbf{\ipa{hwɤ˧mi˥\$}}}}{}
\textcolor{teal}{\mytextsc{noun}} \hspace{4pt} Tone: H\$.
\textcolor{Sepia}{\selectlanguage{english}She-cat, queen.} \zh{母猫。}  ¶ \textcolor{darkblue}{\textbf{\ipa{hwɤ˧mi˧-hwɤ˥pʰv̩˩ / hwɤ˧mi˧-hwɤ˧pʰv̩˥\#}}} \textcolor{Sepia}{\selectlanguage{english}she-cat and tom-cat} \zh{母猫与公猫}  
 \zh{量词}: \textcolor{darkblue}{\textbf{\ipa{mi˩}}}  \mytextsc{clf}: \textcolor{darkblue}{\textbf{\ipa{mi˩}}} 
\lhead{\firstmark}
\rhead{\botmark}

\subsection{\hspace{-0.5cm} {\Large \textcolor{darkblue}{\textbf{\ipa{hwɤ˧pʰv̩\#˥}}}}\hspace{0.5cm}[\kern2pt{\textcolor{darkblue}{\textbf{\ipa{hwɤ˧pʰv̩˥}}}}\kern2pt]} \hypertarget{hw7\string_Mp\string_hv\string_=\#\string_T1}{}
\markboth{\textcolor{darkblue}{\textbf{\ipa{hwɤ˧pʰv̩\#˥}}}}{}
\textcolor{teal}{\mytextsc{noun}} \hspace{4pt} Tone: \#H.
\textcolor{Sepia}{\selectlanguage{english}Tom-cat, tom.} \zh{公猫。}  ¶ \textcolor{darkblue}{\textbf{\ipa{hwɤ˧pʰv̩˧ tʰv̩˧-mi˥\#}}} \textcolor{Sepia}{\selectlanguage{english}\mytextsc{n}+\mytextsc{dem}+\mytextsc{clf}} \zh{那个公猫}  
 ¶ \textcolor{darkblue}{\textbf{\ipa{hwɤ˧pʰv̩˧-hwɤ˧mi˥}}} \textcolor{Sepia}{\selectlanguage{english}tom-cat and she-cat} \zh{公猫与母猫}  
 \zh{量词}: \textcolor{darkblue}{\textbf{\ipa{mi˩}}}  \mytextsc{clf}: \textcolor{darkblue}{\textbf{\ipa{mi˩}}} 
\lhead{\firstmark}
\rhead{\botmark}

\subsection{\hspace{-0.5cm} {\Large \textcolor{darkblue}{\textbf{\ipa{hwɤ˧se˧}}}}\hspace{0.5cm}[\kern2pt{\textcolor{darkblue}{\textbf{\ipa{hwɤ˧se˧}}}}\kern2pt]} \hypertarget{hw7\string_Mse\string_M1}{}
\markboth{\textcolor{darkblue}{\textbf{\ipa{hwɤ˧se˧}}}}{}
\textcolor{teal}{\mytextsc{noun}} \hspace{4pt} Tone: M.
\textcolor{Sepia}{\selectlanguage{english}Peanuts.} \zh{花生。}  Borrowing: Chinese  \zh{花生}
 ¶ \textcolor{darkblue}{\textbf{\ipa{hwɤ˧se˧-qo˧tv̩˩}}} \textcolor{Sepia}{\selectlanguage{english}peanuts} \zh{花生米}  

\lhead{\firstmark}
\rhead{\botmark}

\subsection{\hspace{-0.5cm} {\Large \textcolor{darkblue}{\textbf{\ipa{hwɤ˧tɕi˥}}}}\hspace{0.5cm}[\kern2pt{\textcolor{darkblue}{\textbf{\ipa{hwɤ˧tɕi˧}}}}\kern2pt]} \hypertarget{hw7\string_Mts£i\string_T1}{}
\markboth{\textcolor{darkblue}{\textbf{\ipa{hwɤ˧tɕi˥}}}}{}
\textcolor{teal}{\mytextsc{noun}} \hspace{4pt} Tone: H\#.
\textcolor{Sepia}{\selectlanguage{english}Curly dock, \textit{Rumex crispus}. It is one of three sorts of plants used as pig fodder; it is also used as food for humans.} \zh{土大黄(学名:皱叶酸模)(喂猪的牧草)。}  ¶ \textcolor{darkblue}{\textbf{\ipa{hwɤ˧tɕi˥-bæ˩bæ˩}}} \textcolor{Sepia}{\selectlanguage{english}same meaning} \zh{同上}  
 ¶ \textcolor{darkblue}{\textbf{\ipa{hwɤ˧tɕʰi˥-ʁo˩bv̩˩}}} \textcolor{Sepia}{\selectlanguage{english}sprouts of curly dock} \zh{土大黄的嫩芽}  
 \zh{量词}: \textcolor{darkblue}{\textbf{\ipa{po˧}}}  \mytextsc{clf}: \textcolor{darkblue}{\textbf{\ipa{po˧}}} 
\lhead{\firstmark}
\rhead{\botmark}

\subsection{\hspace{-0.5cm} {\Large \textcolor{darkblue}{\textbf{\ipa{hwɤ˧zo\#˥}}}}\hspace{0.5cm}[\kern2pt{\textcolor{darkblue}{\textbf{\ipa{hwɤ˩zo˥}}}}\kern2pt]} \hypertarget{hw7\string_Mzo\#\string_T1}{}
\markboth{\textcolor{darkblue}{\textbf{\ipa{hwɤ˧zo\#˥}}}}{}
\textcolor{teal}{\mytextsc{noun}} \hspace{4pt} Tone: \#H.
\textcolor{Sepia}{\selectlanguage{english}Kitten.} \zh{小猫。}  ¶ \textcolor{darkblue}{\textbf{\ipa{hwɤ˧zo˧ tʰv̩˧-ɭɯ\#˥}}} \textcolor{Sepia}{\selectlanguage{english}\mytextsc{n}+\mytextsc{dem}+\mytextsc{clf}} \zh{那个小猫}  
 ¶ \textcolor{darkblue}{\textbf{\ipa{hwɤ˧zo˧-hwɤ˧mi˥}}} \textcolor{Sepia}{\selectlanguage{english}cats, the cat family: kitten and parents} \zh{猫,包括小猫、母猫和公猫}  
 \zh{量词}: \textcolor{darkblue}{\textbf{\ipa{ɭɯ˧}}}  \mytextsc{clf}: \textcolor{darkblue}{\textbf{\ipa{ɭɯ˧}}} 
\lhead{\firstmark}
\rhead{\botmark}

\subsection{\hspace{-0.5cm} {\Large \textcolor{darkblue}{\textbf{\ipa{hwɤ˩}}}}\hspace{0.5cm}[\kern2pt{\textcolor{darkblue}{\textbf{\ipa{xxxx non-correspondance entre le nombre de morphèmes et le nombre de tons de morphèmes}}}}\kern2pt]} \hypertarget{hw7\string_B1}{}
\markboth{\textcolor{darkblue}{\textbf{\ipa{hwɤ˩}}}}{}
\textcolor{teal}{\mytextsc{verb}} \hspace{4pt} Tone: L\textsubscript{a}.
\ding{202} \textcolor{Sepia}{\selectlanguage{english}To pack, to tie together into a bundle.} \zh{捆(捆成捆儿)。}  ¶ \textcolor{darkblue}{\textbf{\ipa{zɯ˧-wɤ˧ hwɤ˥}}} \textcolor{Sepia}{\selectlanguage{english}to make a bundle of hay} \zh{将草捆成一垛、捆一垛草}  
 ¶ \textcolor{darkblue}{\textbf{\ipa{si˧-wɤ˧ hwɤ˥}}} \textcolor{Sepia}{\selectlanguage{english}to make a bundle of wood} \zh{将木头捆成一堆、捆一堆木头}  
 ¶ \textcolor{darkblue}{\textbf{\ipa{hɑ˧-wɤ˧ hwɤ˥}}} \textcolor{Sepia}{\selectlanguage{english}to make a bundle of cut cereals} \zh{将粮食捆成一包、捆一包粮食}  
 ¶ \textcolor{darkblue}{\textbf{\ipa{wɤ˩ hwɤ˩˥}}} \textcolor{Sepia}{\selectlanguage{english}to tie together into a bundle, to make a bundle} \zh{捆成一包}  
 ¶ \textcolor{darkblue}{\textbf{\ipa{wɤ˩˥ | tʰi˧-hwɤ˩}}} \textcolor{Sepia}{\selectlanguage{english}and then, (we) tie (it) into a bundle!} \zh{然后,捆成一包!}  
 ¶ \textcolor{darkblue}{\textbf{\ipa{wɤ˩˥ | ɖɯ˧-wɤ˩ hwɤ˩}}} \textcolor{Sepia}{\selectlanguage{english}to tie another bundle, to make (yet) another bundle} \zh{又捆一包}  
\ding{203} \textcolor{Sepia}{\selectlanguage{english}To be in trouble, to put oneself into trouble (figurative sense: as if one were all tied up with a rope, unable to move, to live one's life normally).} \zh{有困难、像把自己捆起来一样。}  ¶ \textcolor{darkblue}{\textbf{\ipa{hĩ˧, | wɤ˩ hwɤ˧ ʝi˧-ni˥gv̩˩!}}} \textcolor{Sepia}{\selectlanguage{english}The people seem to be unhappy / under strain! (Figuratively: they look all tied up, as if they were tied with a rope, unable to move = to live a normal life.)} \zh{人家难受,像被捆一样}  
 ¶ \textcolor{darkblue}{\textbf{\ipa{wɤ˩hwɤ˧ ʝi˧-ni˥gv̩˩-ɲi˩-ze˩!}}} \textcolor{Sepia}{\selectlanguage{english}I have put myself in trouble! / I have made a lot of trouble for myself! / I have put my knickers in a twist!} \zh{我给自己找麻烦了!}  

\lhead{\firstmark}
\rhead{\botmark}

\subsection{\hspace{-0.5cm} {\Large \textcolor{darkblue}{\textbf{\ipa{hwɤ˩\textsubscript{a}}}}}\hspace{0.5cm}[\kern2pt{\textcolor{darkblue}{\textbf{\ipa{hwɤ˩˥}}}}\kern2pt]} \hypertarget{hw7\string_Ba1}{}
\markboth{\textcolor{darkblue}{\textbf{\ipa{hwɤ˩\textsubscript{a}}}}}{}
\textcolor{teal}{\mytextsc{verb}} \hspace{4pt} Tone: L\textsubscript{a}.
\textcolor{Sepia}{\selectlanguage{english}To hand over, to pass over, to send.} \zh{递过去。}  ¶ \textcolor{darkblue}{\textbf{\ipa{hĩ˧-ki˧ | tso˧\textasciitilde{}tso˧ hwɤ˥}}} \textcolor{Sepia}{\selectlanguage{english}to send some stuff to someone} \zh{给人家寄东西}  

\lhead{\firstmark}
\rhead{\botmark}

\subsection{\hspace{-0.5cm} {\Large \textcolor{darkblue}{\textbf{\ipa{hwɤ˩dʑɯ˩}}}}\hspace{0.5cm}[\kern2pt{\textcolor{darkblue}{\textbf{\ipa{hwɤ˩dʑɯ˩˥}}}}\kern2pt]} \hypertarget{hw7\string_Bdz£M\string_B1}{}
\markboth{\textcolor{darkblue}{\textbf{\ipa{hwɤ˩dʑɯ˩}}}}{}
\textcolor{teal}{\mytextsc{noun}} \hspace{4pt} Tone: L.
\textcolor{Sepia}{\selectlanguage{english}Cabin, hut.} \zh{山上过夜的小木房。}  \zh{量词}: \textcolor{darkblue}{\textbf{\ipa{ɭɯ˧}}}  \mytextsc{clf}: \textcolor{darkblue}{\textbf{\ipa{ɭɯ˧}}} 
\lhead{\firstmark}
\rhead{\botmark}

\subsection{\hspace{-0.5cm} {\Large \textcolor{darkblue}{\textbf{\ipa{hwɤ˩kæ˧}}}}\hspace{0.5cm}[\kern2pt{\textcolor{darkblue}{\textbf{\ipa{hwɤ˩kæ˩˥}}}}\kern2pt]} \hypertarget{hw7\string_Bk\{\string_M1}{}
\markboth{\textcolor{darkblue}{\textbf{\ipa{hwɤ˩kæ˧}}}}{}
\textcolor{teal}{\mytextsc{noun}} \hspace{4pt} Tone: LM.
\textcolor{Sepia}{\selectlanguage{english}Red birch; its wood is good: it is used to make ards.} \zh{红桦树。}  ¶ \textcolor{darkblue}{\textbf{\ipa{hwɤ˩kæ˧-si˧dzi˩}}} \textcolor{Sepia}{\selectlanguage{english}same meaning: red birch} \zh{同上:红桦树}  

\lhead{\firstmark}
\rhead{\botmark}

\subsection{\hspace{-0.5cm} {\Large \textcolor{darkblue}{\textbf{\ipa{hwɤ˩ʈi˥}}}}\hspace{0.5cm}[\kern2pt{\textcolor{darkblue}{\textbf{\ipa{hwɤ˧ʈi˥}}}}\kern2pt]} \hypertarget{hw7\string_Bt`i\string_T1}{}
\markboth{\textcolor{darkblue}{\textbf{\ipa{hwɤ˩ʈi˥}}}}{}
\textcolor{teal}{\mytextsc{verb}} \hspace{4pt} Tone: LH.
\textcolor{Sepia}{\selectlanguage{english}To become rusty, to get rusty, to rust.} \zh{生锈。}  ¶ \textcolor{darkblue}{\textbf{\ipa{hwɤ˩ʈi˥-ze˩}}} \textcolor{Sepia}{\selectlanguage{english}\mytextsc{pfv}: it has become rusty} \zh{生锈了}  

\lhead{\firstmark}
\rhead{\botmark}

\subsection{\hspace{-0.5cm} {\Large \textcolor{darkblue}{\textbf{\ipa{hwɤ˧˥}}} \textsubscript{1}}\hspace{0.5cm}[\kern2pt{\textcolor{darkblue}{\textbf{\ipa{hwɤ˥}}}}\kern2pt]} \hypertarget{hw7\string_M\string_T1}{}
\markboth{\textcolor{darkblue}{\textbf{\ipa{hwɤ˧˥}}} \textsubscript{1}}{}
\textcolor{teal}{\mytextsc{verb}} \hspace{4pt} Tone: MH.
\textcolor{Sepia}{\selectlanguage{english}To mend, to patch.} \zh{补。}  ¶ \textcolor{darkblue}{\textbf{\ipa{le˧-hwɤ˧˥}}} \textcolor{Sepia}{\selectlanguage{english}\mytextsc{accomp}} \zh{\mytextsc{accomp}}  
 ¶ \textcolor{darkblue}{\textbf{\ipa{bɑ˩lɑ˩ hwɤ˥}}} \textcolor{Sepia}{\selectlanguage{english}to mend clothes} \zh{补衣服}  

\lhead{\firstmark}
\rhead{\botmark}

\subsection{\hspace{-0.5cm} {\Large \textcolor{darkblue}{\textbf{\ipa{hwɤ˧˥}}} \textsubscript{2}}\hspace{0.5cm}[\kern2pt{\textcolor{darkblue}{\textbf{\ipa{hwɤ˧˥}}}}\kern2pt]} \hypertarget{hw7\string_M\string_T2}{}
\markboth{\textcolor{darkblue}{\textbf{\ipa{hwɤ˧˥}}} \textsubscript{2}}{}
\textcolor{teal}{\mytextsc{noun}} \hspace{4pt} Tone: MH.
\textcolor{Sepia}{\selectlanguage{english}Cat (monosyllable).} \zh{猫(单音节)。}  \zh{量词}: \textcolor{darkblue}{\textbf{\ipa{mi˩}}}  \mytextsc{clf}: \textcolor{darkblue}{\textbf{\ipa{mi˩}}} 
\lhead{\firstmark}
\rhead{\botmark}

\subsection{\hspace{-0.5cm} {\Large \textcolor{darkblue}{\textbf{\ipa{hwɤ˧˥}}} \textsubscript{3}}\hspace{0.5cm}[\kern2pt{\textcolor{darkblue}{\textbf{\ipa{hwɤ˧˥}}}}\kern2pt]} \hypertarget{hw7\string_M\string_T3}{}
\markboth{\textcolor{darkblue}{\textbf{\ipa{hwɤ˧˥}}} \textsubscript{3}}{}
\textcolor{teal}{\mytextsc{noun}} \hspace{4pt} Tone: MH.
\textcolor{Sepia}{\selectlanguage{english}Rust (monosyllable).} \zh{锈(单音节)。} 
\lhead{\firstmark}
\rhead{\botmark}

\newpage
\section*{\centering- \textcolor{darkblue}{\textbf{\ipa{ĩ}}} -}
\subsection{\hspace{-0.5cm} {\Large \textcolor{darkblue}{\textbf{\ipa{ĩ˧}}}}\hspace{0.5cm}[\kern2pt{\textcolor{darkblue}{\textbf{\ipa{ĩ˥}}}}\kern2pt]} \hypertarget{i\string_~\string_M1}{}
\markboth{\textcolor{darkblue}{\textbf{\ipa{ĩ˧}}}}{}
\textcolor{teal}{\mytextsc{interjection}} \hspace{4pt} Tone: M.
\textcolor{Sepia}{\selectlanguage{english}Yes, OK.} \zh{是的,好的。}  ¶ \textcolor{darkblue}{\textbf{\ipa{ʈʂʰɯ˧-ɳɯ˧ | “ĩ˧! ĩ˧!” | pi˧. |}}} \textcolor{Sepia}{\selectlanguage{english}(S)he said “Yes! yes!”} \zh{他说:“是的,是的!”}  

\lhead{\firstmark}
\rhead{\botmark}

\newpage
\section*{\centering- \textcolor{darkblue}{\textbf{\ipa{j}}} \textcolor{darkblue}{\textbf{\ipa{jæ}}} \textcolor{darkblue}{\textbf{\ipa{jɤ}}} \textcolor{darkblue}{\textbf{\ipa{jo}}} \textcolor{darkblue}{\textbf{\ipa{je}}} -}
\subsection{\hspace{-0.5cm} {\Large \textcolor{darkblue}{\textbf{\ipa{jɤ˧}}} \textsubscript{1}}\hspace{0.5cm}[\kern2pt{\textcolor{darkblue}{\textbf{\ipa{jɤ˥}}}}\kern2pt]} \hypertarget{j7\string_M1}{}
\markboth{\textcolor{darkblue}{\textbf{\ipa{jɤ˧}}} \textsubscript{1}}{}
\textcolor{teal}{\mytextsc{adjective}} \hspace{4pt} Tone: M.
\textcolor{Sepia}{\selectlanguage{english}Good (only appears in negative construction).} \zh{好(只出现在否定词后面)。}  ¶ \textcolor{darkblue}{\textbf{\ipa{mɤ˧-jɤ˧}}} \textcolor{Sepia}{\selectlanguage{english}\mytextsc{neg}: it's not good! It's not right! (About someone's behaviour)} \zh{不好(形容一个人的行为)}  

\lhead{\firstmark}
\rhead{\botmark}

\subsection{\hspace{-0.5cm} {\Large \textcolor{darkblue}{\textbf{\ipa{jɤ˧}}} \textsubscript{2}}\hspace{0.5cm}[\kern2pt{\textcolor{darkblue}{\textbf{\ipa{jɤ˥}}}}\kern2pt]} \hypertarget{j7\string_M2}{}
\markboth{\textcolor{darkblue}{\textbf{\ipa{jɤ˧}}} \textsubscript{2}}{}
\textcolor{teal}{\mytextsc{adjective}} \hspace{4pt} Tone: M.
\textcolor{Sepia}{\selectlanguage{english}Flat.} \zh{平(土地)。}  ¶ \textcolor{darkblue}{\textbf{\ipa{mɤ˧-jɤ˧}}} \textcolor{Sepia}{\selectlanguage{english}\mytextsc{neg}: not flat; uneven} \zh{不平}  

\lhead{\firstmark}
\rhead{\botmark}

\subsection{\hspace{-0.5cm} {\Large \textcolor{darkblue}{\textbf{\ipa{jɤ˧}}} \textsubscript{3}}\hspace{0.5cm}[\kern2pt{\textcolor{darkblue}{\textbf{\ipa{jɤ˥}}}}\kern2pt]} \hypertarget{j7\string_M3}{}
\markboth{\textcolor{darkblue}{\textbf{\ipa{jɤ˧}}} \textsubscript{3}}{}
\textcolor{teal}{\mytextsc{noun}} \hspace{4pt} Tone: M.
\textcolor{Sepia}{\selectlanguage{english}Tobacco, cigarettes.} \zh{烟。}  Borrowing: Chinese  \zh{烟?}
 ¶ \textcolor{darkblue}{\textbf{\ipa{jɤ˧ ʈʰɯ˩}}} \textcolor{Sepia}{\selectlanguage{english}to smoke tobacco} \zh{抽烟}  
 \zh{量词}: \textcolor{darkblue}{\textbf{\ipa{ko˧}}}  \mytextsc{clf}: \textcolor{darkblue}{\textbf{\ipa{ko˧}}} 
\lhead{\firstmark}
\rhead{\botmark}

\subsection{\hspace{-0.5cm} {\Large \textcolor{darkblue}{\textbf{\ipa{jɤ˧gɯ˩}}}}\hspace{0.5cm}[\kern2pt{\textcolor{darkblue}{\textbf{\ipa{jɤ˧gɯ˩}}}}\kern2pt]} \hypertarget{j7\string_MgM\string_B1}{}
\markboth{\textcolor{darkblue}{\textbf{\ipa{jɤ˧gɯ˩}}}}{}
\textcolor{teal}{\mytextsc{noun}} \hspace{4pt} Tone: L\#.
\textcolor{Sepia}{\selectlanguage{english}Buckwheat, \textit{Fagopyrum esculentum}.} \zh{甜荞/荞麦/花荞。}  \zh{量词}: \textcolor{darkblue}{\textbf{\ipa{kɤ˧˥}}}  \mytextsc{clf}: \textcolor{darkblue}{\textbf{\ipa{kɤ˧˥}}} \textit{See:} \hyperlink{}{\textcolor{darkblue}{\textbf{\ipa{jɤ˧qʰɑ\#˥}}}} 
\lhead{\firstmark}
\rhead{\botmark}

\subsection{\hspace{-0.5cm} {\Large \textcolor{darkblue}{\textbf{\ipa{jɤ˧ŋɤ˧}}}}\hspace{0.5cm}[\kern2pt{\textcolor{darkblue}{\textbf{\ipa{jɤ˧ŋɤ˧}}}}\kern2pt]} \hypertarget{j7\string_MN7\string_M1}{}
\markboth{\textcolor{darkblue}{\textbf{\ipa{jɤ˧ŋɤ˧}}}}{}
\textcolor{teal}{\mytextsc{noun}} \hspace{4pt} Tone: M.
\textcolor{Sepia}{\selectlanguage{english}The city of Chengdu, in Sichuan.} \zh{成都。}  ¶ \textcolor{darkblue}{\textbf{\ipa{ho˧di˧-jɤ˧ŋɤ˧}}} \textcolor{Sepia}{\selectlanguage{english}same meaning} \zh{同上}  

\lhead{\firstmark}
\rhead{\botmark}

\subsection{\hspace{-0.5cm} {\Large \textcolor{darkblue}{\textbf{\ipa{jɤ˧qʰɑ\#˥}}}}\hspace{0.5cm}[\kern2pt{\textcolor{darkblue}{\textbf{\ipa{jɤ˧qʰɑ˧}}}}\kern2pt]} \hypertarget{j7\string_Mq\string_hA\#\string_T1}{}
\markboth{\textcolor{darkblue}{\textbf{\ipa{jɤ˧qʰɑ\#˥}}}}{}
\textcolor{teal}{\mytextsc{noun}} \hspace{4pt} Tone: \#H.
\textcolor{Sepia}{\selectlanguage{english}Bitter buckwheat, \textit{Fagopyrum tataricum Gaertn}.} \zh{苦荞。}  \zh{量词}: \textcolor{darkblue}{\textbf{\ipa{kɤ˧˥}}}  \mytextsc{clf}: \textcolor{darkblue}{\textbf{\ipa{kɤ˧˥}}} \textit{See:} \hyperlink{}{\textcolor{darkblue}{\textbf{\ipa{jɤ˧gɯ˩}}}} 
\lhead{\firstmark}
\rhead{\botmark}

\subsection{\hspace{-0.5cm} {\Large \textcolor{darkblue}{\textbf{\ipa{jɤ˧qʰɑ˧-pɤ˥jɤ˩-mo˩}}}}\hspace{0.5cm}[\kern2pt{\textcolor{darkblue}{\textbf{\ipa{xxxx non-correspondance entre le nombre de morphèmes et le nombre de tons de morphèmes}}}}\kern2pt]} \hypertarget{j7\string_Mq\string_hA\string_M-p7\string_Tj7\string_B-mo\string_B1}{}
\markboth{\textcolor{darkblue}{\textbf{\ipa{jɤ˧qʰɑ˧-pɤ˥jɤ˩-mo˩}}}}{}
\textcolor{teal}{\mytextsc{noun}} \hspace{4pt} Tone: \#H\mytextsc{}.
\textcolor{Sepia}{\selectlanguage{english}Cep, penny bun, porcino, \textit{Boletus edulis} (a type of edible fungus); literally “buckwheat bun mushroom”, due to its texture.} \zh{牛肝菌。} \textit{See:} \hyperlink{}{\textcolor{darkblue}{\textbf{\ipa{njo˩kæ˧tɕi˩˥}}}} 
\lhead{\firstmark}
\rhead{\botmark}

\subsection{\hspace{-0.5cm} {\Large \textcolor{darkblue}{\textbf{\ipa{jɤ˧wo˧˥}}}}\hspace{0.5cm}[\kern2pt{\textcolor{darkblue}{\textbf{\ipa{jɤ˧wo˧˥}}}}\kern2pt]} \hypertarget{j7\string_Mwo\string_M\string_T1}{}
\markboth{\textcolor{darkblue}{\textbf{\ipa{jɤ˧wo˧˥}}}}{}
\textcolor{teal}{\mytextsc{verb}} \hspace{4pt} Tone: MH\#.
\textcolor{Sepia}{\selectlanguage{english}To regress.} \zh{倒退、退步。}  ¶ \textcolor{darkblue}{\textbf{\ipa{no˧ | jɤ˧wo˧˥ | sɯ˧ɖʐæ˧! / no˧ | le˧-wo˥ | sɯ˧ɖʐæ˧!}}} \textcolor{Sepia}{\selectlanguage{english}You are regressing! (Said to a child who had already developed a habit of going to the loo in the previous weeks, but who, that day, pooed in her trousers.)} \zh{你这是在退步!(情景:一个小孩已经几个礼拜有了上厕所的习惯,那天又把屎拉在裤头里)}  

\lhead{\firstmark}
\rhead{\botmark}

\subsection{\hspace{-0.5cm} {\Large \textcolor{darkblue}{\textbf{\ipa{jɤ˩\textsubscript{a}}}} \textsubscript{1}}\hspace{0.5cm}[\kern2pt{\textcolor{darkblue}{\textbf{\ipa{jɤ˩˥}}}}\kern2pt]} \hypertarget{j7\string_Ba1}{}
\markboth{\textcolor{darkblue}{\textbf{\ipa{jɤ˩\textsubscript{a}}}} \textsubscript{1}}{}
\textcolor{teal}{\mytextsc{verb}} \hspace{4pt} Tone: L\textsubscript{a}.
\textcolor{Sepia}{\selectlanguage{english}To coil.} \zh{盘、盘绕(线)。}  ¶ \textcolor{darkblue}{\textbf{\ipa{sɑ˧ jɤ˥}}} \textcolor{Sepia}{\selectlanguage{english}to coil linen thread} \zh{盘麻线}  
 ¶ \textcolor{darkblue}{\textbf{\ipa{sɑ˧ | le˧-jɤ˩}}} \textcolor{Sepia}{\selectlanguage{english}to coil linen thread} \zh{盘麻线}  
\textit{See:} \hyperlink{}{\textcolor{darkblue}{\textbf{\ipa{tɕɯ˧ɭɯ˧}}}} 
\lhead{\firstmark}
\rhead{\botmark}

\subsection{\hspace{-0.5cm} {\Large \textcolor{darkblue}{\textbf{\ipa{jɤ˩\textsubscript{a}}}} \textsubscript{2}}\hspace{0.5cm}[\kern2pt{\textcolor{darkblue}{\textbf{\ipa{jɤ˩˥}}}}\kern2pt]} \hypertarget{j7\string_Ba2}{}
\markboth{\textcolor{darkblue}{\textbf{\ipa{jɤ˩\textsubscript{a}}}} \textsubscript{2}}{}
\textcolor{teal}{\mytextsc{adjective}} \hspace{4pt} Tone: L\textsubscript{a}.
\textcolor{Sepia}{\selectlanguage{english}Overcooked, overdone, mushy, sodden, mushed.} \zh{(煮)烂。}  ¶ \textcolor{darkblue}{\textbf{\ipa{le˧-tɕɤ˧˥ | le˧-jɤ˩-ze˩!}}} \textcolor{Sepia}{\selectlanguage{english}It got sodden after boiling! / After boiling, it got all mushy/overdone!} \zh{煮烂了!}  
 ¶ \textcolor{darkblue}{\textbf{\ipa{jɤ˩-hĩ˩˥}}} \textcolor{Sepia}{\selectlanguage{english}\mytextsc{rel}/\mytextsc{nmlz}} \zh{烂的}  

\lhead{\firstmark}
\rhead{\botmark}

\subsection{\hspace{-0.5cm} {\Large \textcolor{darkblue}{\textbf{\ipa{jɤ˩\textsubscript{b}}}} \textsubscript{1}}\hspace{0.5cm}[\kern2pt{\textcolor{darkblue}{\textbf{\ipa{jɤ˩˥}}}}\kern2pt]} \hypertarget{j7\string_Bb1}{}
\markboth{\textcolor{darkblue}{\textbf{\ipa{jɤ˩\textsubscript{b}}}} \textsubscript{1}}{}
\textcolor{teal}{\mytextsc{verb}} \hspace{4pt} Tone: L\textsubscript{b}.
\textcolor{Sepia}{\selectlanguage{english}To be listless, to be dejected.} \zh{没精神。}  ¶ \textcolor{darkblue}{\textbf{\ipa{tʰi˧-jɤ˩-ho˩-ze˩!}}} \textcolor{Sepia}{\selectlanguage{english}(S)he is getting listless/dispirited!} \zh{他没精神了!}  
 ¶ \textcolor{darkblue}{\textbf{\ipa{ɑ˩ʁo˧ ʂv̩˧ɖv̩˧ | tʰi˧-jɤ˩-ho˩-tsɯ˩!}}} \textcolor{Sepia}{\selectlanguage{english}When one misses home, one gets listless/dispirited!} \zh{想家的时候,没精神!}  
 ¶ \textcolor{darkblue}{\textbf{\ipa{ɑ˩ʁo˧ ʂv̩˧ɖv̩˧-zo˥, | tʰi˧-jɤ˩-ho˩!}}} \textcolor{Sepia}{\selectlanguage{english}When one misses home, one gets listless/dispirited!} \zh{想家的时候,没精神!}  
 ¶ \textcolor{darkblue}{\textbf{\ipa{ɲi˧mi˧ tsʰi˧-zo˩, | tʰi˧-jɤ˩-ho˩!}}} \textcolor{Sepia}{\selectlanguage{english}When the weather is hot, one gets listless/dispirited!} \zh{天气很热,没精神!}  
 ¶ \textcolor{darkblue}{\textbf{\ipa{jɤ˩-mɤ˥-jɤ˩}}} \textcolor{Sepia}{\selectlanguage{english}\string_ \mytextsc{neg} \string_} \zh{\string_ \mytextsc{neg} \string_}  

\lhead{\firstmark}
\rhead{\botmark}

\subsection{\hspace{-0.5cm} {\Large \textcolor{darkblue}{\textbf{\ipa{jɤ˩\textsubscript{b}}}} \textsubscript{2}}\hspace{0.5cm}[\kern2pt{\textcolor{darkblue}{\textbf{\ipa{jɤ˩˥}}}}\kern2pt]} \hypertarget{j7\string_Bb2}{}
\markboth{\textcolor{darkblue}{\textbf{\ipa{jɤ˩\textsubscript{b}}}} \textsubscript{2}}{}
\textcolor{teal}{\mytextsc{classifier}} \hspace{4pt} Tone: L\textsubscript{b}.
\textcolor{Sepia}{\selectlanguage{english}Row: classifier for rows of vegetables.} \zh{量词:排(一排菜)。}  ¶ \textcolor{darkblue}{\textbf{\ipa{v˩tsʰɤ˧˥ | ɖɯ˧-jɤ˩ tʰi˩-pʰo˩}}} \textcolor{Sepia}{\selectlanguage{english}to plant a row of vegetables} \zh{种一排菜}  

\lhead{\firstmark}
\rhead{\botmark}

\subsection{\hspace{-0.5cm} {\Large \textcolor{darkblue}{\textbf{\ipa{jɤ˩ɕjo˧-dzɑ˧qʰwɤ˩}}}}\hspace{0.5cm}[\kern2pt{\textcolor{darkblue}{\textbf{\ipa{jɤ˩ɕjo˧dzɑ˧qʰwɤ˩}}}}\kern2pt]} \hypertarget{j7\string_Bs£jo\string_M-dzA\string_Mq\string_hw7\string_B1}{}
\markboth{\textcolor{darkblue}{\textbf{\ipa{jɤ˩ɕjo˧-dzɑ˧qʰwɤ˩}}}}{}
\textcolor{teal}{\mytextsc{noun}} \hspace{4pt} Tone: LM-L\#.
\textcolor{Sepia}{\selectlanguage{english}Sandal.} \zh{凉鞋。}  \zh{量词}: \textcolor{darkblue}{\textbf{\ipa{dzi˧}}}  \mytextsc{clf}: \textcolor{darkblue}{\textbf{\ipa{dzi˧}}} 
\lhead{\firstmark}
\rhead{\botmark}

\subsection{\hspace{-0.5cm} {\Large \textcolor{darkblue}{\textbf{\ipa{jɤ˩ho˧}}}}\hspace{0.5cm}[\kern2pt{\textcolor{darkblue}{\textbf{\ipa{jɤ˩ho˥}}}}\kern2pt]} \hypertarget{j7\string_Bho\string_M1}{}
\markboth{\textcolor{darkblue}{\textbf{\ipa{jɤ˩ho˧}}}}{}
\textcolor{teal}{\mytextsc{noun}} \hspace{4pt} Tone: LM.
\textcolor{Sepia}{\selectlanguage{english}Matches.} \zh{火柴(洋火)。}  Borrowing: Chinese  \zh{洋火}
 \zh{量词}: \textcolor{darkblue}{\textbf{\ipa{po˩}}}  \mytextsc{clf}: \textcolor{darkblue}{\textbf{\ipa{po˩}}} 
\lhead{\firstmark}
\rhead{\botmark}

\subsection{\hspace{-0.5cm} {\Large \textcolor{darkblue}{\textbf{\ipa{jɤ˩jo\#˥}}}}\hspace{0.5cm}[\kern2pt{\textcolor{darkblue}{\textbf{\ipa{jɤ˩jo˥}}}}\kern2pt]} \hypertarget{j7\string_Bjo\#\string_T1}{}
\markboth{\textcolor{darkblue}{\textbf{\ipa{jɤ˩jo\#˥}}}}{}
\textcolor{teal}{\mytextsc{noun}} \hspace{4pt} Tone: LM+\#H.
\textcolor{Sepia}{\selectlanguage{english}Potato.} \zh{洋芋、土豆 、马铃薯(汉语借词)。}  Borrowing: Chinese  \zh{洋芋}
 \zh{量词}: \textcolor{darkblue}{\textbf{\ipa{kɤ˧˥}}}  \mytextsc{clf}: \textcolor{darkblue}{\textbf{\ipa{kɤ˧˥}}} 
\lhead{\firstmark}
\rhead{\botmark}

\subsection{\hspace{-0.5cm} {\Large \textcolor{darkblue}{\textbf{\ipa{jɤ˩jo˧-bv̩\#˥}}}}\hspace{0.5cm}[\kern2pt{\textcolor{darkblue}{\textbf{\ipa{xxxx non-correspondance entre le nombre de morphèmes et le nombre de tons de morphèmes}}}}\kern2pt]} \hypertarget{j7\string_Bjo\string_M-bv\string_=\#\string_T1}{}
\markboth{\textcolor{darkblue}{\textbf{\ipa{jɤ˩jo˧-bv̩\#˥}}}}{}
\textcolor{teal}{\mytextsc{noun}} \hspace{4pt} Tone: LM+\#H.
\textit{From:} \textbf{jɤ˩jo\#˥ and bv̩˥} \textcolor{Sepia}{\selectlanguage{english}Potato grub, \textit{Agriotes lineatus}.} \zh{蛴螬。}  \zh{量词}: \textcolor{darkblue}{\textbf{\ipa{mi˩}}}  \mytextsc{clf}: \textcolor{darkblue}{\textbf{\ipa{mi˩}}} 
\lhead{\firstmark}
\rhead{\botmark}

\subsection{\hspace{-0.5cm} {\Large \textcolor{darkblue}{\textbf{\ipa{jɤ˩pæ˧sɯ˥\$}}}}\hspace{0.5cm}[\kern2pt{\textcolor{darkblue}{\textbf{\ipa{xxxx ton non trouvé, à faire manuellement...}}}}\kern2pt]} \hypertarget{j7\string_Bp\{\string_MsM\string_T\$1}{}
\markboth{\textcolor{darkblue}{\textbf{\ipa{jɤ˩pæ˧sɯ˥\$}}}}{}
\textcolor{teal}{\mytextsc{noun}} \hspace{4pt} Tone: LM+H\$.
\textcolor{Sepia}{\selectlanguage{english}Yang Chieftain: a family name from Yongning, containing a name borrowed from Chinese (Yang \zh{杨}) plus a term referring to the lowest degree in the hierarchy of feudal leaders: the hamlet chieftain, \zh{把事}. Only one family in Yongning carries this name.} \zh{杨把事。这个姓,由两部分组成的:‘杨’姓(汉语借词)与封建社会最小领导层次:‘把事’。}  ¶ \textcolor{darkblue}{\textbf{\ipa{jɤ˩pæ˧sɯ˧=ɻ̍˥\$}}} \textcolor{Sepia}{\selectlanguage{english}\string_ \mytextsc{associative}: the people of the Yang Chieftain family} \zh{杨把事家族}  
 ¶ \textcolor{darkblue}{\textbf{\ipa{jɤ˩pɑ˧sɯ˥ | ʈæ˧ʂɯ˧}}} \textcolor{Sepia}{\selectlanguage{english}the proper name of a person of the Yang Chieftain family (given name: Dashi): 'Dashi of the Yang Chieftain family'.} \zh{杨把事家的一个人的名字:杨把事•达石}  
\textit{See:} \hyperlink{}{\textcolor{darkblue}{\textbf{\ipa{pæ˧sɯ˧}}}} 
\lhead{\firstmark}
\rhead{\botmark}

\subsection{\hspace{-0.5cm} {\Large \textcolor{darkblue}{\textbf{\ipa{jɤ˩po˧}}}}\hspace{0.5cm}[\kern2pt{\textcolor{darkblue}{\textbf{\ipa{jɤ˩po˥}}}}\kern2pt]} \hypertarget{j7\string_Bpo\string_M1}{}
\markboth{\textcolor{darkblue}{\textbf{\ipa{jɤ˩po˧}}}}{}
\textcolor{teal}{\mytextsc{verb}} \hspace{4pt} Tone: LM.
\textcolor{Sepia}{\selectlanguage{english}To gamble, to bet, to wager.} \zh{赌博、打赌。} 
\lhead{\firstmark}
\rhead{\botmark}

\subsection{\hspace{-0.5cm} {\Large \textcolor{darkblue}{\textbf{\ipa{jɤ˩tʰi˧-ʁæ˩bæ˩}}}}\hspace{0.5cm}[\kern2pt{\textcolor{darkblue}{\textbf{\ipa{jɤ˩tʰi˧ʁæ˩bæ˩}}}}\kern2pt]} \hypertarget{j7\string_Bt\string_hi\string_M-R\{\string_Bb\{\string_B1}{}
\markboth{\textcolor{darkblue}{\textbf{\ipa{jɤ˩tʰi˧-ʁæ˩bæ˩}}}}{}
\textcolor{teal}{\mytextsc{noun}} \hspace{4pt} Tone: LM-L.
\textcolor{Sepia}{\selectlanguage{english}Porcelain plate.} \zh{瓷盘。}  \zh{量词}: \textcolor{darkblue}{\textbf{\ipa{ɭɯ˧}}}  \mytextsc{clf}: \textcolor{darkblue}{\textbf{\ipa{ɭɯ˧}}} 
\lhead{\firstmark}
\rhead{\botmark}

\subsection{\hspace{-0.5cm} {\Large \textcolor{darkblue}{\textbf{\ipa{jɤ˧˥}}} \textsubscript{1}}\hspace{0.5cm}[\kern2pt{\textcolor{darkblue}{\textbf{\ipa{jɤ˧˥}}}}\kern2pt]} \hypertarget{j7\string_M\string_T1}{}
\markboth{\textcolor{darkblue}{\textbf{\ipa{jɤ˧˥}}} \textsubscript{1}}{}
\textcolor{teal}{\mytextsc{verb}} \hspace{4pt} Tone: MH.
\textcolor{Sepia}{\selectlanguage{english}To lick.} \zh{舔。}  ¶ \textcolor{darkblue}{\textbf{\ipa{tso˧\textasciitilde{}tso˧ jɤ˩}}} \textcolor{Sepia}{\selectlanguage{english}to lick something} \zh{舔东西}  
 ¶ \textcolor{darkblue}{\textbf{\ipa{dzɯ˧-di˧ jɤ˥}}} \textcolor{Sepia}{\selectlanguage{english}to lick food} \zh{舔食品}  
 ¶ \textcolor{darkblue}{\textbf{\ipa{[F5] tso˧tso˧ ɖɯ˧-kʰwɤ˥ jɤ˩-ze˩}}} \textcolor{Sepia}{\selectlanguage{english}(S)he has licked something.} \zh{他舔了一个东西。}  

\lhead{\firstmark}
\rhead{\botmark}

\subsection{\hspace{-0.5cm} {\Large \textcolor{darkblue}{\textbf{\ipa{jɤ˧˥}}} \textsubscript{2}}\hspace{0.5cm}[\kern2pt{\textcolor{darkblue}{\textbf{\ipa{jɤ˧˥}}}}\kern2pt]} \hypertarget{j7\string_M\string_T2}{}
\markboth{\textcolor{darkblue}{\textbf{\ipa{jɤ˧˥}}} \textsubscript{2}}{}
\textcolor{teal}{\mytextsc{noun}} \hspace{4pt} Tone: MH.
\textcolor{Sepia}{\selectlanguage{english}A wild radish that grows on the mountains; it is edible; it is picked and eaten in the Spring, when vegetables are not ripe yet. Yi people harvest it and sell it in the plain.} \zh{红萝卜菜:一种山上的野菜。春天的时候,菜园的蔬菜还没有成熟的时候,永宁的人吃红萝卜菜。彝族在高山上采下来,在永宁卖。} Local Chinese dialect:\zh{野山菜,\textcolor{darkblue}{\textbf{\ipa{/ʝi˧ʂæ˧tsʰɤ˩/}}}。} ¶ \textcolor{darkblue}{\textbf{\ipa{jɤ˧ dzɯ˧ | qʰɑ˧-sɯ˥\textasciitilde{}sɯ˩, | jɤ˧ ʈʂɤ˥ ŋv̩˩-ɭɯ˩\textasciitilde{}ɭɯ˩!}}} \textcolor{Sepia}{\selectlanguage{english}“The wild radish is bitter; and its harvest costs tears! / The wild radish tastes bitter; and its harvest is bitter, too! / The wild radish is all bitterness inside, and all bitterness at the harvest!” This proverb evokes the difficulty of the harvest, which requires long wanderings up high on the mountain.} \zh{“红萝卜菜,味道苦,去摘也要流眼泪! / 红萝卜菜,吃起来苦,摘起来也苦!”摘红萝卜菜,需要爬高山,寻找时间长,永宁坝子的农民觉得这比较苦。}  
\textit{See:} \hyperlink{}{\textcolor{darkblue}{\textbf{\ipa{ʝi˧ʂæ˧tsʰɤ˩}}}} 
\lhead{\firstmark}
\rhead{\botmark}

\subsection{\hspace{-0.5cm} {\Large \textcolor{darkblue}{\textbf{\ipa{jɤ˧˥}}} \textsubscript{3}}\hspace{0.5cm}[\kern2pt{\textcolor{darkblue}{\textbf{\ipa{jɤ˧˥}}}}\kern2pt]} \hypertarget{j7\string_M\string_T3}{}
\markboth{\textcolor{darkblue}{\textbf{\ipa{jɤ˧˥}}} \textsubscript{3}}{}
\textcolor{teal}{\mytextsc{verb}} \hspace{4pt} Tone: MH.
\textcolor{Sepia}{\selectlanguage{english}To spread, to put on, to smear.} \zh{抹、涂抹。}  ¶ \textcolor{darkblue}{\textbf{\ipa{pʰv˧ʂɯ˧ jɤ˧˥}}} \textcolor{Sepia}{\selectlanguage{english}to put on beauty cream or sunscreen} \zh{抹防晒霜}  
 ¶ \textcolor{darkblue}{\textbf{\ipa{mɤ˩ jɤ˩˥}}} \textcolor{Sepia}{\selectlanguage{english}to apply grease (e.g. to the skin)} \zh{涂抹油}  
 ¶ \textcolor{darkblue}{\textbf{\ipa{tʰi˧-jɤ˧˥}}}  

\lhead{\firstmark}
\rhead{\botmark}

\subsection{\hspace{-0.5cm} {\Large \textcolor{darkblue}{\textbf{\ipa{jɤ˧˥\textsubscript{a}}}} \textsubscript{1}}\hspace{0.5cm}[\kern2pt{\textcolor{darkblue}{\textbf{\ipa{jɤ˧˥}}}}\kern2pt]} \hypertarget{j7\string_M\string_Ta1}{}
\markboth{\textcolor{darkblue}{\textbf{\ipa{jɤ˧˥\textsubscript{a}}}} \textsubscript{1}}{}
\textcolor{teal}{\mytextsc{classifier}} \hspace{4pt} Tone: MH\textsubscript{a}.
\textcolor{Sepia}{\selectlanguage{english}Classifier used for women, and for some female domestic animals; it does not carry any hint of deprecation, nor does it convey any hint of respect by itself.} \zh{量词:母性、雌性(人或动物)(一个/一只)。} 
\lhead{\firstmark}
\rhead{\botmark}

\subsection{\hspace{-0.5cm} {\Large \textcolor{darkblue}{\textbf{\ipa{jɤ˧˥\textsubscript{a}}}} \textsubscript{2}}\hspace{0.5cm}[\kern2pt{\textcolor{darkblue}{\textbf{\ipa{jɤ˧˥}}}}\kern2pt]} \hypertarget{j7\string_M\string_Ta2}{}
\markboth{\textcolor{darkblue}{\textbf{\ipa{jɤ˧˥\textsubscript{a}}}} \textsubscript{2}}{}
\textcolor{teal}{\mytextsc{classifier}} \hspace{4pt} Tone: MH\textsubscript{a}.
\textcolor{Sepia}{\selectlanguage{english}Classifier for dough balls and teacakes. One dough ball is the quantity of dough that can be prepared with one egg. Tea consumed in the Yongning area in the first half of the 20th century was green tea from a large leaf variety of Camellia sinensis (C. sinensis assamica) found in the mountains of southern Yunnan; it used to be pressed into 'teacake' shape.} \zh{量词:面(一团),茶饼(一个)等。(一团面,是和了一个鸡蛋的面团的量。)。}  ¶ \textcolor{darkblue}{\textbf{\ipa{æ˩ʁv̩˩-pɤ˥jɤ˩ | ɖɯ˧-jɤ˧˥}}} \textcolor{Sepia}{\selectlanguage{english}a ball of egg dough} \zh{一个鸡蛋面团}  
 ¶ \textcolor{darkblue}{\textbf{\ipa{ʝi˧-jɤ˧˥}}} \textcolor{Sepia}{\selectlanguage{english}one ball/cake} \zh{一个团/并}  

\lhead{\firstmark}
\rhead{\botmark}

\subsection{\hspace{-0.5cm} {\Large \textcolor{darkblue}{\textbf{\ipa{jo˥}}}}\hspace{0.5cm}[\kern2pt{\textcolor{darkblue}{\textbf{\ipa{jo˥}}}}\kern2pt]} \hypertarget{jo\string_T1}{}
\markboth{\textcolor{darkblue}{\textbf{\ipa{jo˥}}}}{}
\textcolor{teal}{\mytextsc{noun}} \hspace{4pt} Tone: \#H.
\textcolor{Sepia}{\selectlanguage{english}Jade.} \zh{玉石。}  Borrowing: Chinese  \zh{玉}
 \zh{量词}: \textcolor{darkblue}{\textbf{\ipa{pʰo˧˥}}}  \mytextsc{clf}: \textcolor{darkblue}{\textbf{\ipa{pʰo˧˥}}} 
\lhead{\firstmark}
\rhead{\botmark}

\subsection{\hspace{-0.5cm} {\Large \textcolor{darkblue}{\textbf{\ipa{jo˧}}}}\hspace{0.5cm}[\kern2pt{\textcolor{darkblue}{\textbf{\ipa{jo˥}}}}\kern2pt]} \hypertarget{jo\string_M1}{}
\markboth{\textcolor{darkblue}{\textbf{\ipa{jo˧}}}}{}
\textcolor{teal}{\mytextsc{verb}} \hspace{4pt} Tone: M intrans.
\textcolor{Sepia}{\selectlanguage{english}To come; to come in.} \zh{来。}  ¶ \textcolor{darkblue}{\textbf{\ipa{le˧-jo˧-ze˧!}}} \textcolor{Sepia}{\selectlanguage{english}\mytextsc{accomp} \string_ \mytextsc{pfv}: (s)he has come} \zh{来了!}  

\lhead{\firstmark}
\rhead{\botmark}

\subsection{\hspace{-0.5cm} {\Large \textcolor{darkblue}{\textbf{\ipa{jo˧gv̩˧}}}}\hspace{0.5cm}[\kern2pt{\textcolor{darkblue}{\textbf{\ipa{jo˧gv̩˧}}}}\kern2pt]} \hypertarget{jo\string_Mgv\string_=\string_M1}{}
\markboth{\textcolor{darkblue}{\textbf{\ipa{jo˧gv̩˧}}}}{}
\textcolor{teal}{\mytextsc{noun}} \hspace{4pt} Tone: M.
\textcolor{Sepia}{\selectlanguage{english}Lijiang.} \zh{丽江(包括丽江坝子)。} 
\lhead{\firstmark}
\rhead{\botmark}

\subsection{\hspace{-0.5cm} {\Large \textcolor{darkblue}{\textbf{\ipa{jo˧gv̩˧-ŋv̩˧lv̩˧}}}}\hspace{0.5cm}[\kern2pt{\textcolor{darkblue}{\textbf{\ipa{xxxx non-correspondance entre le nombre de morphèmes et le nombre de tons de morphèmes}}}}\kern2pt]} \hypertarget{jo\string_Mgv\string_=\string_M-Nv\string_=\string_Mlv\string_=\string_M1}{}
\markboth{\textcolor{darkblue}{\textbf{\ipa{jo˧gv̩˧-ŋv̩˧lv̩˧}}}}{}
\textcolor{teal}{\mytextsc{noun}} \hspace{4pt} Tone: M.
\textcolor{Sepia}{\selectlanguage{english}Yulong snow mountain; literally 'Lijiang's snow mountain'.} \zh{玉龙雪山。} 
\lhead{\firstmark}
\rhead{\botmark}

\subsection{\hspace{-0.5cm} {\Large \textcolor{darkblue}{\textbf{\ipa{jo˧mi˧}}}}\hspace{0.5cm}[\kern2pt{\textcolor{darkblue}{\textbf{\ipa{jo˧mi˧}}}}\kern2pt]} \hypertarget{jo\string_Mmi\string_M1}{}
\markboth{\textcolor{darkblue}{\textbf{\ipa{jo˧mi˧}}}}{}
\textcolor{teal}{\mytextsc{noun}} \hspace{4pt} Tone: M.
\textcolor{Sepia}{\selectlanguage{english}Ewe.} \zh{母绵羊。}  ¶ \textcolor{darkblue}{\textbf{\ipa{jo˧mi˧-po˧lo˧}}} \textcolor{Sepia}{\selectlanguage{english}ewe and ram} \zh{母绵羊与公羊}  
 \zh{量词}: \textcolor{darkblue}{\textbf{\ipa{pʰo˧˥}}}  \mytextsc{clf}: \textcolor{darkblue}{\textbf{\ipa{pʰo˧˥}}} 
\lhead{\firstmark}
\rhead{\botmark}

\subsection{\hspace{-0.5cm} {\Large \textcolor{darkblue}{\textbf{\ipa{jo˧mi˧-ʁwɤ˧}}}}\hspace{0.5cm}[\kern2pt{\textcolor{darkblue}{\textbf{\ipa{xxxx non-correspondance entre le nombre de morphèmes et le nombre de tons de morphèmes}}}}\kern2pt]} \hypertarget{jo\string_Mmi\string_M-Rw7\string_M1}{}
\markboth{\textcolor{darkblue}{\textbf{\ipa{jo˧mi˧-ʁwɤ˧}}}}{}
\textcolor{teal}{\mytextsc{noun}} \hspace{4pt} Tone: M.
\textcolor{Sepia}{\selectlanguage{english}The second village that one crosses when going from \textcolor{darkblue}{\textbf{\ipa{/qʰæ˧tɕʰi˧/}}} to \textcolor{darkblue}{\textbf{\ipa{/ʈʂo˧ʂɯ\#˥/}}}.} \zh{有米瓦村。} 
\lhead{\firstmark}
\rhead{\botmark}

\subsection{\hspace{-0.5cm} {\Large \textcolor{darkblue}{\textbf{\ipa{jo˩}}}}\hspace{0.5cm}[\kern2pt{\textcolor{darkblue}{\textbf{\ipa{jo˥}}}}\kern2pt]} \hypertarget{jo\string_B1}{}
\markboth{\textcolor{darkblue}{\textbf{\ipa{jo˩}}}}{}
\textcolor{teal}{\mytextsc{noun}} \hspace{4pt} Tone: L.
\textcolor{Sepia}{\selectlanguage{english}Sheep.} \zh{绵羊。}  ¶ \textcolor{darkblue}{\textbf{\ipa{jo˩-ɣɯ˩˥}}} \textcolor{Sepia}{\selectlanguage{english}sheep skin} \zh{羊皮}  
 \zh{量词}: \textcolor{darkblue}{\textbf{\ipa{pʰo˧˥}}}  \mytextsc{clf}: \textcolor{darkblue}{\textbf{\ipa{pʰo˧˥}}} 
\lhead{\firstmark}
\rhead{\botmark}

\subsection{\hspace{-0.5cm} {\Large \textcolor{darkblue}{\textbf{\ipa{jo˩\textsubscript{b}}}}}\hspace{0.5cm}[\kern2pt{\textcolor{darkblue}{\textbf{\ipa{jo˩˥}}}}\kern2pt]} \hypertarget{jo\string_Bb1}{}
\markboth{\textcolor{darkblue}{\textbf{\ipa{jo˩\textsubscript{b}}}}}{}
\textcolor{teal}{\mytextsc{classifier}} \hspace{4pt} Tone: L\textsubscript{b}.
\textcolor{Sepia}{\selectlanguage{english}An ounce.} \zh{量词:两(一两)。} 
\lhead{\firstmark}
\rhead{\botmark}

\subsection{\hspace{-0.5cm} {\Large \textcolor{darkblue}{\textbf{\ipa{jo˩gi˩}}}}\hspace{0.5cm}[\kern2pt{\textcolor{darkblue}{\textbf{\ipa{jo˩gi˩˥}}}}\kern2pt]} \hypertarget{jo\string_Bgi\string_B1}{}
\markboth{\textcolor{darkblue}{\textbf{\ipa{jo˩gi˩}}}}{}
\textcolor{teal}{\mytextsc{noun}} \hspace{4pt} Tone: L.
\textcolor{Sepia}{\selectlanguage{english}Right (opposite of left).} \zh{右边。}  ¶ \textcolor{darkblue}{\textbf{\ipa{jo˩gi˩dzɤ˩}}} \textcolor{Sepia}{\selectlanguage{english}the side to the right, the right} \zh{右边}  
\textit{See:} \hyperlink{}{\textcolor{darkblue}{\textbf{\ipa{jo˩˧}}}} 
\lhead{\firstmark}
\rhead{\botmark}

\subsection{\hspace{-0.5cm} {\Large \textcolor{darkblue}{\textbf{\ipa{jo˩kʰv̩˩}}}}\hspace{0.5cm}[\kern2pt{\textcolor{darkblue}{\textbf{\ipa{jo˩kʰv̩˩˥}}}}\kern2pt]} \hypertarget{jo\string_Bk\string_hv\string_=\string_B1}{}
\markboth{\textcolor{darkblue}{\textbf{\ipa{jo˩kʰv̩˩}}}}{}
\textcolor{teal}{\mytextsc{noun}} \hspace{4pt} Tone: L.
\textcolor{Sepia}{\selectlanguage{english}Year of the goat.} \zh{羊年。} 
\lhead{\firstmark}
\rhead{\botmark}

\subsection{\hspace{-0.5cm} {\Large \textcolor{darkblue}{\textbf{\ipa{jo˩lo˩}}}}\hspace{0.5cm}[\kern2pt{\textcolor{darkblue}{\textbf{\ipa{jo˩lo˩˥}}}}\kern2pt]} \hypertarget{jo\string_Blo\string_B1}{}
\markboth{\textcolor{darkblue}{\textbf{\ipa{jo˩lo˩}}}}{}
\textcolor{teal}{\mytextsc{noun}} \hspace{4pt} Tone: L.
\textcolor{Sepia}{\selectlanguage{english}Right (opposite of left).} \zh{右边。} \textit{See:} \hyperlink{}{\textcolor{darkblue}{\textbf{\ipa{jo˩˧}}}} 
\lhead{\firstmark}
\rhead{\botmark}

\subsection{\hspace{-0.5cm} {\Large \textcolor{darkblue}{\textbf{\ipa{jo˩pv̩˧}}}}\hspace{0.5cm}[\kern2pt{\textcolor{darkblue}{\textbf{\ipa{jo˩pv̩˥}}}}\kern2pt]} \hypertarget{jo\string_Bpv\string_=\string_M1}{}
\markboth{\textcolor{darkblue}{\textbf{\ipa{jo˩pv̩˧}}}}{}
\textcolor{teal}{\mytextsc{noun}} \hspace{4pt} Tone: LM / LM+MH\#.
\textcolor{Sepia}{\selectlanguage{english}Oilcloth; tarpaulin.} \zh{油布。}  Borrowing: Chinese  \zh{油布}
 ¶ \textcolor{darkblue}{\textbf{\ipa{jo˩pv̩˧˥}}} \textcolor{Sepia}{\selectlanguage{english}oilcloth (tonal variant)} \zh{油布(声调变体)}  
 \zh{量词}: \textcolor{darkblue}{\textbf{\ipa{tsʰi˥}}}  \mytextsc{clf}: \textcolor{darkblue}{\textbf{\ipa{tsʰi˥}}} 
\lhead{\firstmark}
\rhead{\botmark}

\subsection{\hspace{-0.5cm} {\Large \textcolor{darkblue}{\textbf{\ipa{jo˩pʰv̩˩}}}}\hspace{0.5cm}[\kern2pt{\textcolor{darkblue}{\textbf{\ipa{jo˩pʰv̩˩˥}}}}\kern2pt]} \hypertarget{jo\string_Bp\string_hv\string_=\string_B1}{}
\markboth{\textcolor{darkblue}{\textbf{\ipa{jo˩pʰv̩˩}}}}{}
\textcolor{teal}{\mytextsc{noun}} \hspace{4pt} Tone: L.
\textcolor{Sepia}{\selectlanguage{english}Male sheep.} \zh{公绵羊。}  ¶ \textcolor{darkblue}{\textbf{\ipa{jo˧pʰv̩˧ tʰv̩˧-mi˥\#}}} \textcolor{Sepia}{\selectlanguage{english}\string_ \mytextsc{dem} \mytextsc{clf}: that ram} \zh{这头公羊}  
 \zh{量词}: \textcolor{darkblue}{\textbf{\ipa{pʰo˧˥}}} \textcolor{darkblue}{\textbf{\ipa{mi˩}}}  \mytextsc{clf}: \textcolor{darkblue}{\textbf{\ipa{pʰo˧˥}}} \textcolor{darkblue}{\textbf{\ipa{mi˩}}} \textit{See:} \hyperlink{}{\textcolor{darkblue}{\textbf{\ipa{po˧lo˧}}}} 
\lhead{\firstmark}
\rhead{\botmark}

\subsection{\hspace{-0.5cm} {\Large \textcolor{darkblue}{\textbf{\ipa{jo˩ʂwæ˩}}}}\hspace{0.5cm}[\kern2pt{\textcolor{darkblue}{\textbf{\ipa{jo˩ʂwæ˩˥}}}}\kern2pt]} \hypertarget{jo\string_Bs`w\{\string_B1}{}
\markboth{\textcolor{darkblue}{\textbf{\ipa{jo˩ʂwæ˩}}}}{}
\textcolor{teal}{\mytextsc{noun}} \hspace{4pt} Tone: L.
\textcolor{Sepia}{\selectlanguage{english}Wether (castrated ram, neutered ram).} \zh{阉羊。}  \zh{量词}: \textcolor{darkblue}{\textbf{\ipa{pʰo˧˥}}}  \mytextsc{clf}: \textcolor{darkblue}{\textbf{\ipa{pʰo˧˥}}} 
\lhead{\firstmark}
\rhead{\botmark}

\subsection{\hspace{-0.5cm} {\Large \textcolor{darkblue}{\textbf{\ipa{jo˩zo˩}}}}\hspace{0.5cm}[\kern2pt{\textcolor{darkblue}{\textbf{\ipa{jo˩zo˩˥}}}}\kern2pt]} \hypertarget{jo\string_Bzo\string_B1}{}
\markboth{\textcolor{darkblue}{\textbf{\ipa{jo˩zo˩}}}}{}
\textcolor{teal}{\mytextsc{noun}} \hspace{4pt} Tone: L.
\textcolor{Sepia}{\selectlanguage{english}Lamb.} \zh{绵羊羔。}  \zh{量词}: \textcolor{darkblue}{\textbf{\ipa{ɭɯ˧}}}  \mytextsc{clf}: \textcolor{darkblue}{\textbf{\ipa{ɭɯ˧}}} 
\lhead{\firstmark}
\rhead{\botmark}

\subsection{\hspace{-0.5cm} {\Large \textcolor{darkblue}{\textbf{\ipa{jo˧˥}}}}\hspace{0.5cm}[\kern2pt{\textcolor{darkblue}{\textbf{\ipa{jo˧˥}}}}\kern2pt]} \hypertarget{jo\string_M\string_T1}{}
\markboth{\textcolor{darkblue}{\textbf{\ipa{jo˧˥}}}}{}
\textcolor{teal}{\mytextsc{verb}} \hspace{4pt} Tone: MH.
\textcolor{Sepia}{\selectlanguage{english}To offer.} \zh{赠给。} 
\lhead{\firstmark}
\rhead{\botmark}

\subsection{\hspace{-0.5cm} {\Large \textcolor{darkblue}{\textbf{\ipa{jo˩˧}}}}\hspace{0.5cm}[\kern2pt{\textcolor{darkblue}{\textbf{\ipa{jo˩˥}}}}\kern2pt]} \hypertarget{jo\string_B\string_M1}{}
\markboth{\textcolor{darkblue}{\textbf{\ipa{jo˩˧}}}}{}
\textcolor{teal}{\mytextsc{noun}} \hspace{4pt} Tone: LM.
\textcolor{Sepia}{\selectlanguage{english}Right (opposite of: left).} \zh{右边。} \textit{See:} \hyperlink{}{\textcolor{darkblue}{\textbf{\ipa{jo˩gi˩}}}} 
\lhead{\firstmark}
\rhead{\botmark}

\subsection{\hspace{-0.5cm} {\Large \textcolor{darkblue}{\textbf{\ipa{je˧pʰi˧-jɤ\#˥}}}}\hspace{0.5cm}[\kern2pt{\textcolor{darkblue}{\textbf{\ipa{xxxx non-correspondance entre le nombre de morphèmes et le nombre de tons de morphèmes}}}}\kern2pt]} \hypertarget{je\string_Mp\string_hi\string_M-j7\#\string_T1}{}
\markboth{\textcolor{darkblue}{\textbf{\ipa{je˧pʰi˧-jɤ\#˥}}}}{}
\textcolor{teal}{\mytextsc{noun}} \hspace{4pt} Tone: \#H.
\textit{From:} \textbf{\zh{鸦片} and jɤ˧} \textcolor{Sepia}{\selectlanguage{english}Opium.} \zh{鸦片(汉语借词)。}  Borrowing: Chinese  \zh{鸦片}

\lhead{\firstmark}
\rhead{\botmark}

\subsection{\hspace{-0.5cm} {\Large \textcolor{darkblue}{\textbf{\ipa{je˩ʐe˧}}}}\hspace{0.5cm}[\kern2pt{\textcolor{darkblue}{\textbf{\ipa{je˧ʐe˧}}}}\kern2pt]} \hypertarget{je\string_Bz`e\string_M1}{}
\markboth{\textcolor{darkblue}{\textbf{\ipa{je˩ʐe˧}}}}{}
\textcolor{teal}{\mytextsc{noun}} \hspace{4pt} Tone: LM.
\textcolor{Sepia}{\selectlanguage{english}Westerner.} \zh{西方人(“洋人”)(汉语借词)。} Local Chinese dialect:\zh{洋人。} Borrowing: Chinese  \zh{洋人}
 \zh{量词}: \textcolor{darkblue}{\textbf{\ipa{v̩˧}}}  \mytextsc{clf}: \textcolor{darkblue}{\textbf{\ipa{v̩˧}}} 
\lhead{\firstmark}
\rhead{\botmark}

\newpage
\section*{\centering- \textcolor{darkblue}{\textbf{\ipa{ʝ}}} -}
\subsection{\hspace{-0.5cm} {\Large \textcolor{darkblue}{\textbf{\ipa{ʝi˥}}} \textsubscript{1}}\hspace{0.5cm}[\kern2pt{\textcolor{darkblue}{\textbf{\ipa{ʝi˥}}}}\kern2pt]} \hypertarget{j££i\string_T1}{}
\markboth{\textcolor{darkblue}{\textbf{\ipa{ʝi˥}}} \textsubscript{1}}{}
\textcolor{teal}{\mytextsc{noun}} \hspace{4pt} Tone: \#H.
\textcolor{Sepia}{\selectlanguage{english}Ox.} \zh{牛。}  ¶ \textcolor{darkblue}{\textbf{\ipa{ʝi˧-ɣɯ˥}}} \textcolor{Sepia}{\selectlanguage{english}ox skin} \zh{牛皮}  
 ¶ \textcolor{darkblue}{\textbf{\ipa{ʝi˧ tʰv̩˧-pʰo˩}}} \textcolor{Sepia}{\selectlanguage{english}\mytextsc{n}+\mytextsc{dem}+\mytextsc{clf}} \zh{那头牛}  
 \zh{量词}: \textcolor{darkblue}{\textbf{\ipa{pʰo˧˥}}}  \mytextsc{clf}: \textcolor{darkblue}{\textbf{\ipa{pʰo˧˥}}} 
\lhead{\firstmark}
\rhead{\botmark}

\subsection{\hspace{-0.5cm} {\Large \textcolor{darkblue}{\textbf{\ipa{ʝi˥}}} \textsubscript{2}}\hspace{0.5cm}[\kern2pt{\textcolor{darkblue}{\textbf{\ipa{ʝi˥}}}}\kern2pt]} \hypertarget{j££i\string_T2}{}
\markboth{\textcolor{darkblue}{\textbf{\ipa{ʝi˥}}} \textsubscript{2}}{}
\textcolor{teal}{\mytextsc{verb}} \hspace{4pt} Tone: H.
\textcolor{Sepia}{\selectlanguage{english}To do, to work.} \zh{做,工作。}  ¶ \textcolor{darkblue}{\textbf{\ipa{ɖwæ˧˥ | lo˧ ʝi˧}}} \textcolor{Sepia}{\selectlanguage{english}hard-working, who works hard} \zh{勤劳、努力}  
 ¶ \textcolor{darkblue}{\textbf{\ipa{ɖɯ˧-sɑ˥ | mɤ˧-ʝi˥}}} \textcolor{Sepia}{\selectlanguage{english}to do nothing at all} \zh{什么也不干}  
 ¶ \textcolor{darkblue}{\textbf{\ipa{ə˧tso˧-mɤ˧-ɲi˩ | ʝi˧-bi˧-zo˧-ho˥!}}} \textcolor{Sepia}{\selectlanguage{english}[I/we] will have to take charge of everything / [I/we] will have to do all the work!} \zh{什么都要做! / 我什么都要干(/管)!}  
 ¶ \textcolor{darkblue}{\textbf{\ipa{ʈʂʰɯ˧ne-ʝi˥ | ʝi˧-zo˧-ho˥-ɲi˩!}}} \textcolor{Sepia}{\selectlanguage{english}This is how it must be done! / This is how it is done!} \zh{应该这样做的!}  
 ¶ \textcolor{darkblue}{\textbf{\ipa{ɑ˩ʁo˧ ʝi˧}}} \textcolor{Sepia}{\selectlanguage{english}to take care of the household, to look after the affairs of the family; in particular: distributing work to the various members, and ensuring that the supplies are not running low} \zh{管理家里的大小事情(如:分配工作、家务等)}  

\lhead{\firstmark}
\rhead{\botmark}

\subsection{\hspace{-0.5cm} {\Large \textcolor{darkblue}{\textbf{\ipa{ʝi˥}}} \textsubscript{3}}\hspace{0.5cm}[\kern2pt{\textcolor{darkblue}{\textbf{\ipa{ʝi˥}}}}\kern2pt]} \hypertarget{j££i\string_T3}{}
\markboth{\textcolor{darkblue}{\textbf{\ipa{ʝi˥}}} \textsubscript{3}}{}
\textcolor{teal}{\mytextsc{verb}} \hspace{4pt} Tone: H.
\textcolor{Sepia}{\selectlanguage{english}To draw.} \zh{画。}  ¶ \textcolor{darkblue}{\textbf{\ipa{mɤ˧-ʝi˥}}} \textcolor{Sepia}{\selectlanguage{english}\mytextsc{neg}} \zh{不画}  
 ¶ \textcolor{darkblue}{\textbf{\ipa{tʰɑ˧-ʝi˥!}}} \textcolor{Sepia}{\selectlanguage{english}\mytextsc{prohib}} \zh{别画!}  
 ¶ \textcolor{darkblue}{\textbf{\ipa{ʈʂɑ˧tɑ˥ ʝi˩}}} \textcolor{Sepia}{\selectlanguage{english}to draw a sign} \zh{画一个符号}  

\lhead{\firstmark}
\rhead{\botmark}

\subsection{\hspace{-0.5cm} {\Large \textcolor{darkblue}{\textbf{\ipa{ʝi˥}}} \textsubscript{4}}\hspace{0.5cm}[\kern2pt{\textcolor{darkblue}{\textbf{\ipa{ʝi˥}}}}\kern2pt]} \hypertarget{j££i\string_T4}{}
\markboth{\textcolor{darkblue}{\textbf{\ipa{ʝi˥}}} \textsubscript{4}}{}
\textcolor{teal}{\mytextsc{noun}} \hspace{4pt} Tone: \#H.
\textcolor{Sepia}{\selectlanguage{english}Earthen jar.} \zh{坛子,罐子 (陶器)。}  \zh{量词}: \textcolor{darkblue}{\textbf{\ipa{ɭɯ˧}}}  \mytextsc{clf}: \textcolor{darkblue}{\textbf{\ipa{ɭɯ˧}}} 
\lhead{\firstmark}
\rhead{\botmark}

\subsection{\hspace{-0.5cm} {\Large \textcolor{darkblue}{\textbf{\ipa{ʝi˥}}} \textsubscript{5}}\hspace{0.5cm}[\kern2pt{\textcolor{darkblue}{\textbf{\ipa{ʝi˥}}}}\kern2pt]} \hypertarget{j££i\string_T5}{}
\markboth{\textcolor{darkblue}{\textbf{\ipa{ʝi˥}}} \textsubscript{5}}{}
\textcolor{teal}{\mytextsc{verb}} \hspace{4pt} Tone: H.
\textcolor{Sepia}{\selectlanguage{english}To inform, to tell.} \zh{通知、告诉。}  ¶ \textcolor{darkblue}{\textbf{\ipa{le˧-ʝi˥-ze˩}}} \textcolor{Sepia}{\selectlanguage{english}\mytextsc{accomp} \string_ \mytextsc{pfv}} \zh{通知了}  
 ¶ \textcolor{darkblue}{\textbf{\ipa{qʰwæ˧ mi˧ ʝi˧}}} \textcolor{Sepia}{\selectlanguage{english}to provide a piece of news, to provide some information} \zh{告诉(一个)消息}  
 ¶ \textcolor{darkblue}{\textbf{\ipa{njɤ˧ | hĩ˧-ki˧ | qʰwæ˧mi˧ ʝi˧-ze˩}}} \textcolor{Sepia}{\selectlanguage{english}I have told people the news.} \zh{我告诉了人家(那个消息)。}  

\lhead{\firstmark}
\rhead{\botmark}

\subsection{\hspace{-0.5cm} {\Large \textcolor{darkblue}{\textbf{\ipa{ʝi˥}}} \textsubscript{6}}\hspace{0.5cm}[\kern2pt{\textcolor{darkblue}{\textbf{\ipa{ʝi˥}}}}\kern2pt]} \hypertarget{j££i\string_T6}{}
\markboth{\textcolor{darkblue}{\textbf{\ipa{ʝi˥}}} \textsubscript{6}}{}
\textcolor{teal}{\mytextsc{noun}} \hspace{4pt} Tone: \#H.
\textit{\textcolor{Sepia}{\selectlanguage{english}archaic}} [\zh{古语}] \textcolor{Sepia}{\selectlanguage{english}Man, male person.} \zh{男人。} 
\lhead{\firstmark}
\rhead{\botmark}

\subsection{\hspace{-0.5cm} {\Large \textcolor{darkblue}{\textbf{\ipa{ʝi˥}}} \textsubscript{7}}\hspace{0.5cm}[\kern2pt{\textcolor{darkblue}{\textbf{\ipa{ʝi˥}}}}\kern2pt]} \hypertarget{j££i\string_T7}{}
\markboth{\textcolor{darkblue}{\textbf{\ipa{ʝi˥}}} \textsubscript{7}}{}
\textcolor{teal}{\mytextsc{verb}} \hspace{4pt} Tone: H.
\textcolor{Sepia}{\selectlanguage{english}Verb of existence, for movable things.} \zh{存在动词:有(可移动物品)。}  ¶ \textcolor{darkblue}{\textbf{\ipa{ə˧tso˧-mɤ˧-ɲi˩, | le˧-ʂe˧, | le˧-ʝi˥!}}} \textcolor{Sepia}{\selectlanguage{english}We get all sorts of things (all the necessary paraphernalia for a ritual, a feast...) and we have it (at hand for when we need it) / We get all sorts of things ready (for the ritual / the feast)!} \zh{所有(的东西都)找,(就)有了 = 所有的东西都备好了}  

\lhead{\firstmark}
\rhead{\botmark}

\subsection{\hspace{-0.5cm} {\Large \textcolor{darkblue}{\textbf{\ipa{ʝi˧}}} \textsubscript{1}}\hspace{0.5cm}[\kern2pt{\textcolor{darkblue}{\textbf{\ipa{ʝi˩˥}}}}\kern2pt]} \hypertarget{j££i\string_M1}{}
\markboth{\textcolor{darkblue}{\textbf{\ipa{ʝi˧}}} \textsubscript{1}}{}
\textcolor{teal}{\mytextsc{verb}} \hspace{4pt} Tone: M\textsubscript{c}.
\textcolor{Sepia}{\selectlanguage{english}To come.} \zh{来。}  ¶ \textcolor{darkblue}{\textbf{\ipa{lɑ˧ ʝi˧-ze˧!}}} \textcolor{Sepia}{\selectlanguage{english}A tiger is coming! / A tiger has come round!} \zh{老虎来了!}  
 ¶ \textcolor{darkblue}{\textbf{\ipa{lɑ˧ le˧-ʝi˩-ze˩!}}} \textcolor{Sepia}{\selectlanguage{english}The tiger is coming back! / The tiger is coming again!} \zh{老虎又来了!}  
 ¶ \textcolor{darkblue}{\textbf{\ipa{mɤ˧-ʝi˧-ze˧!}}} \textcolor{Sepia}{\selectlanguage{english}It's going wrong! / Something is going wrong! / We're in for trouble!} \zh{不好了!不行了!}  

\lhead{\firstmark}
\rhead{\botmark}

\subsection{\hspace{-0.5cm} {\Large \textcolor{darkblue}{\textbf{\ipa{ʝi˧}}} \textsubscript{2}}\hspace{0.5cm}[\kern2pt{\textcolor{darkblue}{\textbf{\ipa{ʝi˥}}}}\kern2pt]} \hypertarget{j££i\string_M2}{}
\markboth{\textcolor{darkblue}{\textbf{\ipa{ʝi˧}}} \textsubscript{2}}{}
\textcolor{teal}{\mytextsc{noun}} \hspace{4pt} Tone: M.
\textcolor{Sepia}{\selectlanguage{english}One (restricted use: only in association with /ɭɯ˧/).} \zh{一。}  ¶ \textcolor{darkblue}{\textbf{\ipa{zo˧mv̩˥ | ʝi˧-ɭɯ˧ ʂv̩˧}}} \textcolor{Sepia}{\selectlanguage{english}to take care of a child} \zh{管一个孩子}  

\lhead{\firstmark}
\rhead{\botmark}

\subsection{\hspace{-0.5cm} {\Large \textcolor{darkblue}{\textbf{\ipa{ʝi˧\textsubscript{b}}}}}\hspace{0.5cm}[\kern2pt{\textcolor{darkblue}{\textbf{\ipa{ʝi˥}}}}\kern2pt]} \hypertarget{j££i\string_Mb1}{}
\markboth{\textcolor{darkblue}{\textbf{\ipa{ʝi˧\textsubscript{b}}}}}{}
\textcolor{teal}{\mytextsc{classifier}} \hspace{4pt} Tone: M\textsubscript{b}.
\textcolor{Sepia}{\selectlanguage{english}Classifier for places.} \zh{量词:地方(一个)。}  ¶ \textcolor{darkblue}{\textbf{\ipa{ɖɯ˧-ʝi˧}}} \textcolor{Sepia}{\selectlanguage{english}a place, somewhere} \zh{一个地方}  
 ¶ \textcolor{darkblue}{\textbf{\ipa{ɖɯ˧-ʝi˧ dzi˩}}} \textcolor{Sepia}{\selectlanguage{english}to live somewhere; to move to somewhere} \zh{住在一个地方,搬家到一个地方}  
 ¶ \textcolor{darkblue}{\textbf{\ipa{ɖɯ˧-v˧ | ɖɯ˧-ʝi˧ hɯ˧}}} \textcolor{Sepia}{\selectlanguage{english}each goes her/his own way (context: explaining that, in many families, people go to live in different cities for professional reasons)} \zh{个去个的地方!/ 每个人去不同的地方!(情景:由于工作原因,一家的成员经常需要去不同的城市工作。)}  

\lhead{\firstmark}
\rhead{\botmark}

\subsection{\hspace{-0.5cm} {\Large \textcolor{darkblue}{\textbf{\ipa{ʝi˧-bv̩˧˥}}}}\hspace{0.5cm}[\kern2pt{\textcolor{darkblue}{\textbf{\ipa{xxxx non-correspondance entre le nombre de morphèmes et le nombre de tons de morphèmes}}}}\kern2pt]} \hypertarget{j££i\string_M-bv\string_=\string_M\string_T1}{}
\markboth{\textcolor{darkblue}{\textbf{\ipa{ʝi˧-bv̩˧˥}}}}{}
\textcolor{teal}{\mytextsc{noun}} \hspace{4pt} Tone: MH\#.
\textcolor{Sepia}{\selectlanguage{english}Cow pen.} \zh{牛圈。}  \zh{量词}: \textcolor{darkblue}{\textbf{\ipa{ɭɯ˧}}}  \mytextsc{clf}: \textcolor{darkblue}{\textbf{\ipa{ɭɯ˧}}} 
\lhead{\firstmark}
\rhead{\botmark}

\subsection{\hspace{-0.5cm} {\Large \textcolor{darkblue}{\textbf{\ipa{ʝi˧kʰv̩˩}}}}\hspace{0.5cm}[\kern2pt{\textcolor{darkblue}{\textbf{\ipa{ʝi˩kʰv̩˥}}}}\kern2pt]} \hypertarget{j££i\string_Mk\string_hv\string_=\string_B1}{}
\markboth{\textcolor{darkblue}{\textbf{\ipa{ʝi˧kʰv̩˩}}}}{}
\textcolor{teal}{\mytextsc{noun}} \hspace{4pt} Tone: L\#.
\textcolor{Sepia}{\selectlanguage{english}Year of the ox.} \zh{牛年。} 
\lhead{\firstmark}
\rhead{\botmark}

\subsection{\hspace{-0.5cm} {\Large \textcolor{darkblue}{\textbf{\ipa{ʝi˧kʰv̩˥}}}}\hspace{0.5cm}[\kern2pt{\textcolor{darkblue}{\textbf{\ipa{ʝi˧kʰv̩˩}}}}\kern2pt]} \hypertarget{j££i\string_Mk\string_hv\string_=\string_T1}{}
\markboth{\textcolor{darkblue}{\textbf{\ipa{ʝi˧kʰv̩˥}}}}{}
\textcolor{teal}{\mytextsc{pronoun/pronominal}} \hspace{4pt} Tone: H\#.
\textcolor{Sepia}{\selectlanguage{english}Some, a few.} \zh{一些。}  ¶ \textcolor{darkblue}{\textbf{\ipa{hĩ˧ ʝi˧kʰv̩˥}}} \textcolor{Sepia}{\selectlanguage{english}some people, part of the people} \zh{一些人}  

\lhead{\firstmark}
\rhead{\botmark}

\subsection{\hspace{-0.5cm} {\Large \textcolor{darkblue}{\textbf{\ipa{ʝi˧kʰwɤ˥\$}}}}\hspace{0.5cm}[\kern2pt{\textcolor{darkblue}{\textbf{\ipa{ʝi˧kʰwɤ˥}}}}\kern2pt]} \hypertarget{j££i\string_Mk\string_hw7\string_T\$1}{}
\markboth{\textcolor{darkblue}{\textbf{\ipa{ʝi˧kʰwɤ˥\$}}}}{}
\textcolor{teal}{\mytextsc{pronoun/pronominal}} \hspace{4pt} Tone: H\$.
\textcolor{Sepia}{\selectlanguage{english}A little, some.} \zh{一点。} 
\lhead{\firstmark}
\rhead{\botmark}

\subsection{\hspace{-0.5cm} {\Large \textcolor{darkblue}{\textbf{\ipa{ʝi˧lo\#˥}}}}\hspace{0.5cm}[\kern2pt{\textcolor{darkblue}{\textbf{\ipa{ʝi˧lo˥}}}}\kern2pt]} \hypertarget{j££i\string_Mlo\#\string_T1}{}
\markboth{\textcolor{darkblue}{\textbf{\ipa{ʝi˧lo\#˥}}}}{}
\textcolor{teal}{\mytextsc{noun}} \hspace{4pt} Tone: \#H.
\textcolor{Sepia}{\selectlanguage{english}Attitude towards others.} \zh{态度、对待的态度。}  ¶ \textcolor{darkblue}{\textbf{\ipa{ʝi˧lo˧ dʑɤ˥!}}} \textcolor{Sepia}{\selectlanguage{english}(He/she) has a good attitude!} \zh{态度积极}  
 ¶ \textcolor{darkblue}{\textbf{\ipa{ʈʂʰɯ˧ | ʝi˧lo˧ | dʑɤ˩˥! | hĩ˧-ki˧ | dʑɤ˩-ʝi˥!}}} \textcolor{Sepia}{\selectlanguage{english}He/she has a good attitude towards people! He/she is kind to people / does some good around him/her!} \zh{他(对人)态度好!对人好/做好事!}  
 ¶ \textcolor{darkblue}{\textbf{\ipa{ʝi˧lo˧ dzɑ˧}}} \textcolor{Sepia}{\selectlanguage{english}(to have) a bad attitude: to be lazy, dissipated...} \zh{态度不好}  
 ¶ \textcolor{darkblue}{\textbf{\ipa{njɤ˧-ɳɯ˧ hɑ˧ gv̩˥, | ʝi˧lo˧ dzɑ˧!}}} \textcolor{Sepia}{\selectlanguage{english}When I cook, I don't make a good job of it / I don't (manage to) put any heart into it / I make a mess of it!} \zh{我做饭,集中不了精神 / 做的乱七八糟!}  
 ¶ \textcolor{darkblue}{\textbf{\ipa{ʈʂʰɯ˧ | ə˧tso˧ ʝi˧lo˧ ɲi˥?}}} \textcolor{Sepia}{\selectlanguage{english}What sort of an attitude is this? (Criticism of someone who does not have a proper attitude)} \zh{这是什么态度啊?(批评一个人的态度)}  

\lhead{\firstmark}
\rhead{\botmark}

\subsection{\hspace{-0.5cm} {\Large \textcolor{darkblue}{\textbf{\ipa{ʝi˧mi˧}}}}\hspace{0.5cm}[\kern2pt{\textcolor{darkblue}{\textbf{\ipa{ʝi˩mi˩˥}}}}\kern2pt]} \hypertarget{j££i\string_Mmi\string_M1}{}
\markboth{\textcolor{darkblue}{\textbf{\ipa{ʝi˧mi˧}}}}{}
\textcolor{teal}{\mytextsc{noun}} \hspace{4pt} Tone: M.
\textcolor{Sepia}{\selectlanguage{english}Jar.} \zh{坛子,罐子 (陶器)。}  \zh{量词}: \textcolor{darkblue}{\textbf{\ipa{ɭɯ˧}}}  \mytextsc{clf}: \textcolor{darkblue}{\textbf{\ipa{ɭɯ˧}}} 
\lhead{\firstmark}
\rhead{\botmark}

\subsection{\hspace{-0.5cm} {\Large \textcolor{darkblue}{\textbf{\ipa{ʝi˧pʰv̩\#˥}}}}\hspace{0.5cm}[\kern2pt{\textcolor{darkblue}{\textbf{\ipa{ʝi˩pʰv̩˧˥}}}}\kern2pt]} \hypertarget{j££i\string_Mp\string_hv\string_=\#\string_T1}{}
\markboth{\textcolor{darkblue}{\textbf{\ipa{ʝi˧pʰv̩\#˥}}}}{}
\textcolor{teal}{\mytextsc{noun}} \hspace{4pt} Tone: \#H.
\textcolor{Sepia}{\selectlanguage{english}Male ox, bull.} \zh{公牛。}  ¶ \textcolor{darkblue}{\textbf{\ipa{ʝi˧pʰv̩˧ tʰv̩˧-mi˥\#}}} \textcolor{Sepia}{\selectlanguage{english}\mytextsc{n}+\mytextsc{dem}+\mytextsc{clf}} \zh{那头公牛}  
 ¶ \textcolor{darkblue}{\textbf{\ipa{ʝi˧pʰv̩˧ tʰv̩˧-ɭɯ\#˥}}} \textcolor{Sepia}{\selectlanguage{english}\mytextsc{n}+\mytextsc{dem}+\mytextsc{clf}.animaux} \zh{那头公牛}  
 \zh{量词}: \textcolor{darkblue}{\textbf{\ipa{ɭɯ˧ / mi˩}}}  \mytextsc{clf}: \textcolor{darkblue}{\textbf{\ipa{ɭɯ˧ / mi˩}}} 
\lhead{\firstmark}
\rhead{\botmark}

\subsection{\hspace{-0.5cm} {\Large \textcolor{darkblue}{\textbf{\ipa{ʝi˧qv̩˥}}}}\hspace{0.5cm}[\kern2pt{\textcolor{darkblue}{\textbf{\ipa{ʝi˩qv̩˩˥}}}}\kern2pt]} \hypertarget{j££i\string_Mqv\string_=\string_T1}{}
\markboth{\textcolor{darkblue}{\textbf{\ipa{ʝi˧qv̩˥}}}}{}
\textcolor{teal}{\mytextsc{noun}} \hspace{4pt} Tone: H\#.
\textcolor{Sepia}{\selectlanguage{english}Neck strap: a part of the buffalo's harness for ploughing: a strap that fastens the yoke.} \zh{轭的一个部分,将牛轭安在牛的脖子上。}  \zh{量词}: \textcolor{darkblue}{\textbf{\ipa{ɭɯ˧}}}  \mytextsc{clf}: \textcolor{darkblue}{\textbf{\ipa{ɭɯ˧}}} 
\lhead{\firstmark}
\rhead{\botmark}

\subsection{\hspace{-0.5cm} {\Large \textcolor{darkblue}{\textbf{\ipa{ʝi˧ʁæ˥}}}}\hspace{0.5cm}[\kern2pt{\textcolor{darkblue}{\textbf{\ipa{ʝi˧ʁæ˥}}}}\kern2pt]} \hypertarget{j££i\string_MR\{\string_T1}{}
\markboth{\textcolor{darkblue}{\textbf{\ipa{ʝi˧ʁæ˥}}}}{}
\textcolor{teal}{\mytextsc{noun}} \hspace{4pt} Tone: H\#.
\textcolor{Sepia}{\selectlanguage{english}Cow, beef.} \zh{黄牛。}  \zh{量词}: \textcolor{darkblue}{\textbf{\ipa{pʰo˧˥}}}  \mytextsc{clf}: \textcolor{darkblue}{\textbf{\ipa{pʰo˧˥}}} 
\lhead{\firstmark}
\rhead{\botmark}

\subsection{\hspace{-0.5cm} {\Large \textcolor{darkblue}{\textbf{\ipa{ʝi˧ʁo\#˥}}}}\hspace{0.5cm}[\kern2pt{\textcolor{darkblue}{\textbf{\ipa{ʝi˧ʁo˥}}}}\kern2pt]} \hypertarget{j££i\string_MRo\#\string_T1}{}
\markboth{\textcolor{darkblue}{\textbf{\ipa{ʝi˧ʁo\#˥}}}}{}
\textcolor{teal}{\mytextsc{adjective}} \hspace{4pt} Tone: \#H.
\textit{From:} \textbf{ʝi˥ 1 and ʁo˧ 2} \textcolor{Sepia}{\selectlanguage{english}Able, capable, able-bodied.} \zh{能干、不缺劳力。}  ¶ \textcolor{darkblue}{\textbf{\ipa{ʈʂʰɯ˧ | ʝi˧ʁo˧-hĩ˧ | ɖɯ˧-v̩˧ ɲi˩}}} \textcolor{Sepia}{\selectlanguage{english}It's an able/competent person.} \zh{他是一个能干/称职的人。}  
 ¶ \textcolor{darkblue}{\textbf{\ipa{ʝi˧ʁo˧-zo˥}}} \textcolor{Sepia}{\selectlanguage{english}a competent lad, a capable fellow} \zh{一个能干的男人}  
 ¶ \textcolor{darkblue}{\textbf{\ipa{ʝi˧ʁo˧ ɲi˥}}} \textcolor{Sepia}{\selectlanguage{english}\mytextsc{cop}} \zh{\mytextsc{cop}}  

\lhead{\firstmark}
\rhead{\botmark}

\subsection{\hspace{-0.5cm} {\Large \textcolor{darkblue}{\textbf{\ipa{ʝi˧sɑ˧}}}}\hspace{0.5cm}[\kern2pt{\textcolor{darkblue}{\textbf{\ipa{ʝi˧sɑ˧}}}}\kern2pt]} \hypertarget{j££i\string_MsA\string_M1}{}
\markboth{\textcolor{darkblue}{\textbf{\ipa{ʝi˧sɑ˧}}}}{}
\textcolor{teal}{\mytextsc{noun}} \hspace{4pt} Tone: M.
\textcolor{Sepia}{\selectlanguage{english}Umbrella.} \zh{雨伞。}  Borrowing: Chinese  \zh{雨伞}
 \zh{量词}: \textcolor{darkblue}{\textbf{\ipa{nɑ˧}}}  \mytextsc{clf}: \textcolor{darkblue}{\textbf{\ipa{nɑ˧}}} 
\lhead{\firstmark}
\rhead{\botmark}

\subsection{\hspace{-0.5cm} {\Large \textcolor{darkblue}{\textbf{\ipa{ʝi˧se˧}}} \textsubscript{1}}\hspace{0.5cm}[\kern2pt{\textcolor{darkblue}{\textbf{\ipa{ʝi˧se˩}}}}\kern2pt]} \hypertarget{j££i\string_Mse\string_M1}{}
\markboth{\textcolor{darkblue}{\textbf{\ipa{ʝi˧se˧}}} \textsubscript{1}}{}
\textcolor{teal}{\mytextsc{noun}} \hspace{4pt} Tone: M.
\textcolor{Sepia}{\selectlanguage{english}Doctor.} \zh{医生(汉语借词)。}  Borrowing: Chinese  \zh{医生}

\lhead{\firstmark}
\rhead{\botmark}

\subsection{\hspace{-0.5cm} {\Large \textcolor{darkblue}{\textbf{\ipa{ʝi˧se˧}}} \textsubscript{2}}\hspace{0.5cm}[\kern2pt{\textcolor{darkblue}{\textbf{\ipa{ʝi˧se˧}}}}\kern2pt]} \hypertarget{j££i\string_Mse\string_M2}{}
\markboth{\textcolor{darkblue}{\textbf{\ipa{ʝi˧se˧}}} \textsubscript{2}}{}
\textcolor{teal}{\mytextsc{adjective}} \hspace{4pt} Tone: M.
\textcolor{Sepia}{\selectlanguage{english}Wild (as opposed to: cultivated; e.g. wild plants, wild animals).} \zh{野生(汉语借词)。}  Borrowing: Chinese  \zh{野生}
 ¶ \textcolor{darkblue}{\textbf{\ipa{ʝi˧se˧-hĩ˧}}} \textcolor{Sepia}{\selectlanguage{english}\string_ \mytextsc{rel}/\mytextsc{nmlz}} \zh{野生的}  

\lhead{\firstmark}
\rhead{\botmark}

\subsection{\hspace{-0.5cm} {\Large \textcolor{darkblue}{\textbf{\ipa{ʝi˧sɯ˥}}}}\hspace{0.5cm}[\kern2pt{\textcolor{darkblue}{\textbf{\ipa{ʝi˧sɯ˧}}}}\kern2pt]} \hypertarget{j££i\string_MsM\string_T1}{}
\markboth{\textcolor{darkblue}{\textbf{\ipa{ʝi˧sɯ˥}}}}{}
\textcolor{teal}{\mytextsc{noun}} \hspace{4pt} Tone: H\#.
\textcolor{Sepia}{\selectlanguage{english}Meaning, sense.} \zh{意思(汉语借词)。}  Borrowing: Chinese  \zh{意思}

\lhead{\firstmark}
\rhead{\botmark}

\subsection{\hspace{-0.5cm} {\Large \textcolor{darkblue}{\textbf{\ipa{ʝi˧ʂæ˧tsʰɤ˩}}}}\hspace{0.5cm}[\kern2pt{\textcolor{darkblue}{\textbf{\ipa{ʝi˧ʂæ˧tsʰɤ˧}}}}\kern2pt]} \hypertarget{j££i\string_Ms`\{\string_Mts\string_h7\string_B1}{}
\markboth{\textcolor{darkblue}{\textbf{\ipa{ʝi˧ʂæ˧tsʰɤ˩}}}}{}
\textcolor{teal}{\mytextsc{noun}} \hspace{4pt} Tone: L\#.
\textcolor{Sepia}{\selectlanguage{english}A wild radish that grows on the mountains; it is edible; it is picked and eaten in the Spring, when vegetables are not ripe yet. Yi people harvest it and sell it in the plain.} \zh{红萝卜菜(汉语借词:野山菜):一种山上的野菜。春天的时候,菜园的蔬菜还没有成熟的时候,永宁的人吃红萝卜菜。彝族人从高山上采下来,在永宁卖。} Local Chinese dialect:\zh{野山菜。} Borrowing: Chinese  \zh{野山菜}
\textit{See:} \hyperlink{}{\textcolor{darkblue}{\textbf{\ipa{jɤ˧˥}}} \textsubscript{2}} 
\lhead{\firstmark}
\rhead{\botmark}

\subsection{\hspace{-0.5cm} {\Large \textcolor{darkblue}{\textbf{\ipa{ʝi˧ʂɯ˥}}}}\hspace{0.5cm}[\kern2pt{\textcolor{darkblue}{\textbf{\ipa{ʝi˧ʂɯ˥}}}}\kern2pt]} \hypertarget{j££i\string_Ms`M\string_T1}{}
\markboth{\textcolor{darkblue}{\textbf{\ipa{ʝi˧ʂɯ˥}}}}{}
\textcolor{teal}{\mytextsc{noun}} \hspace{4pt} Tone: H\#.
\textcolor{Sepia}{\selectlanguage{english}Masculine given name.} \zh{男性名字。} 
\lhead{\firstmark}
\rhead{\botmark}

\subsection{\hspace{-0.5cm} {\Large \textcolor{darkblue}{\textbf{\ipa{ʝi˧tɕi˧}}}}\hspace{0.5cm}[\kern2pt{\textcolor{darkblue}{\textbf{\ipa{ʝi˧tɕi˥}}}}\kern2pt]} \hypertarget{j££i\string_Mts£i\string_M1}{}
\markboth{\textcolor{darkblue}{\textbf{\ipa{ʝi˧tɕi˧}}}}{}
\textcolor{teal}{\mytextsc{noun}} \hspace{4pt} Tone: M.
\textcolor{Sepia}{\selectlanguage{english}Feminine given name.} \zh{女性名字。} 
\lhead{\firstmark}
\rhead{\botmark}

\subsection{\hspace{-0.5cm} {\Large \textcolor{darkblue}{\textbf{\ipa{ʝi˧tɕi˧-ɖɯ˩mɑ˩}}}}\hspace{0.5cm}[\kern2pt{\textcolor{darkblue}{\textbf{\ipa{xxxx non-correspondance entre le nombre de morphèmes et le nombre de tons de morphèmes}}}}\kern2pt]} \hypertarget{j££i\string_Mts£i\string_M-d`M\string_BmA\string_B1}{}
\markboth{\textcolor{darkblue}{\textbf{\ipa{ʝi˧tɕi˧-ɖɯ˩mɑ˩}}}}{}
\textcolor{teal}{\mytextsc{noun}} \hspace{4pt} Tone: \mytextsc{L}.
\textcolor{Sepia}{\selectlanguage{english}Feminine given name.} \zh{女性名字。} 
\lhead{\firstmark}
\rhead{\botmark}

\subsection{\hspace{-0.5cm} {\Large \textcolor{darkblue}{\textbf{\ipa{ʝi˧tsɯ˧}}}}\hspace{0.5cm}[\kern2pt{\textcolor{darkblue}{\textbf{\ipa{ʝi˧tsɯ˧}}}}\kern2pt]} \hypertarget{j££i\string_MtsM\string_M1}{}
\markboth{\textcolor{darkblue}{\textbf{\ipa{ʝi˧tsɯ˧}}}}{}
\textcolor{teal}{\mytextsc{noun}} \hspace{4pt} Tone: M.
\textcolor{Sepia}{\selectlanguage{english}Chair (borrowing).} \zh{椅子。}  Borrowing: Chinese  \zh{椅子}
 \zh{量词}: \textcolor{darkblue}{\textbf{\ipa{nɑ˧}}}  \mytextsc{clf}: \textcolor{darkblue}{\textbf{\ipa{nɑ˧}}} 
\lhead{\firstmark}
\rhead{\botmark}

\subsection{\hspace{-0.5cm} {\Large \textcolor{darkblue}{\textbf{\ipa{ʝi˧ʈʂʰe˥-mi˩}}}}\hspace{0.5cm}[\kern2pt{\textcolor{darkblue}{\textbf{\ipa{ʝi˧ʈʂʰe˥mi˩}}}}\kern2pt]} \hypertarget{j££i\string_Mt`s`\string_he\string_T-mi\string_B1}{}
\markboth{\textcolor{darkblue}{\textbf{\ipa{ʝi˧ʈʂʰe˥-mi˩}}}}{}
\textcolor{teal}{\mytextsc{noun}} \hspace{4pt} Tone: H\#-.
\textcolor{Sepia}{\selectlanguage{english}South.} \zh{南方。}  ¶ \textcolor{darkblue}{\textbf{\ipa{ʝi˧ʈʂʰe˥mi˩-gi˩dzɤ˩ se˩}}} \textcolor{Sepia}{\selectlanguage{english}to walk towards the south} \zh{往南方走}  

\lhead{\firstmark}
\rhead{\botmark}

\subsection{\hspace{-0.5cm} {\Large \textcolor{darkblue}{\textbf{\ipa{ʝi˧zo\#˥}}}}\hspace{0.5cm}[\kern2pt{\textcolor{darkblue}{\textbf{\ipa{ʝi˧zo˧}}}}\kern2pt]} \hypertarget{j££i\string_Mzo\#\string_T1}{}
\markboth{\textcolor{darkblue}{\textbf{\ipa{ʝi˧zo\#˥}}}}{}
\textcolor{teal}{\mytextsc{noun}} \hspace{4pt} Tone: \#H.
\textcolor{Sepia}{\selectlanguage{english}Calf.} \zh{小牛。}  ¶ \textcolor{darkblue}{\textbf{\ipa{ʝi˧zo˧ tʰv̩˧-ɭɯ\#˥}}} \textcolor{Sepia}{\selectlanguage{english}\mytextsc{n}+\mytextsc{dem}+\mytextsc{clf}} \zh{那头小牛}  
 \zh{量词}: \textcolor{darkblue}{\textbf{\ipa{pʰo˧˥ / ɭɯ˧}}}  \mytextsc{clf}: \textcolor{darkblue}{\textbf{\ipa{pʰo˧˥ / ɭɯ˧}}} 
\lhead{\firstmark}
\rhead{\botmark}

\subsection{\hspace{-0.5cm} {\Large \textcolor{darkblue}{\textbf{\ipa{ʝi˩bv̩˩}}} \textsubscript{1}}\hspace{0.5cm}[\kern2pt{\textcolor{darkblue}{\textbf{\ipa{ʝi˧bv̩˧}}}}\kern2pt]} \hypertarget{j££i\string_Bbv\string_=\string_B1}{}
\markboth{\textcolor{darkblue}{\textbf{\ipa{ʝi˩bv̩˩}}} \textsubscript{1}}{}
\textcolor{teal}{\mytextsc{noun}} \hspace{4pt} Tone: L.
\textcolor{Sepia}{\selectlanguage{english}Pockmarked person.} \zh{麻子。}  ¶ \textcolor{darkblue}{\textbf{\ipa{ʝi˩bv̩˩-ʝi˧ʈv̩˩ʈv̩˩}}} \textcolor{Sepia}{\selectlanguage{english}same meaning} \zh{同上}  
 \zh{量词}: \textcolor{darkblue}{\textbf{\ipa{v̩˧}}}  \mytextsc{clf}: \textcolor{darkblue}{\textbf{\ipa{v̩˧}}} 
\lhead{\firstmark}
\rhead{\botmark}

\subsection{\hspace{-0.5cm} {\Large \textcolor{darkblue}{\textbf{\ipa{ʝi˩bv̩˩}}} \textsubscript{2}}\hspace{0.5cm}[\kern2pt{\textcolor{darkblue}{\textbf{\ipa{ʝi˩bv̩˩˥}}}}\kern2pt]} \hypertarget{j££i\string_Bbv\string_=\string_B2}{}
\markboth{\textcolor{darkblue}{\textbf{\ipa{ʝi˩bv̩˩}}} \textsubscript{2}}{}
\textcolor{teal}{\mytextsc{noun}} \hspace{4pt} Tone: L.
\textcolor{Sepia}{\selectlanguage{english}Bull (male).} \zh{公牛。}  \zh{量词}: \textcolor{darkblue}{\textbf{\ipa{pʰo˧˥}}}  \mytextsc{clf}: \textcolor{darkblue}{\textbf{\ipa{pʰo˧˥}}} 
\lhead{\firstmark}
\rhead{\botmark}

\subsection{\hspace{-0.5cm} {\Large \textcolor{darkblue}{\textbf{\ipa{ʝi˩di˩-mi˥}}}}\hspace{0.5cm}[\kern2pt{\textcolor{darkblue}{\textbf{\ipa{xxxx non-correspondance entre le nombre de morphèmes et le nombre de tons de morphèmes}}}}\kern2pt]} \hypertarget{j££i\string_Bdi\string_B-mi\string_T1}{}
\markboth{\textcolor{darkblue}{\textbf{\ipa{ʝi˩di˩-mi˥}}}}{}
\textcolor{teal}{\mytextsc{noun}} \hspace{4pt} Tone: L+H\#.
\textcolor{Sepia}{\selectlanguage{english}Heifer; also used for a female pianniu: hybrid of yak and cattle.} \zh{小牝牛(包括黄牛和小母犏牛)。}  \zh{量词}: \textcolor{darkblue}{\textbf{\ipa{ɭɯ˧}}} \textcolor{darkblue}{\textbf{\ipa{pʰo˧˥}}}  \mytextsc{clf}: \textcolor{darkblue}{\textbf{\ipa{ɭɯ˧}}} \textcolor{darkblue}{\textbf{\ipa{pʰo˧˥}}} 
\lhead{\firstmark}
\rhead{\botmark}

\subsection{\hspace{-0.5cm} {\Large \textcolor{darkblue}{\textbf{\ipa{ʝi˩mi˩}}}}\hspace{0.5cm}[\kern2pt{\textcolor{darkblue}{\textbf{\ipa{ʝi˧mi˧}}}}\kern2pt]} \hypertarget{j££i\string_Bmi\string_B1}{}
\markboth{\textcolor{darkblue}{\textbf{\ipa{ʝi˩mi˩}}}}{}
\textcolor{teal}{\mytextsc{noun}} \hspace{4pt} Tone: L.
\textcolor{Sepia}{\selectlanguage{english}Cow (female).} \zh{母牛。}  ¶ \textcolor{darkblue}{\textbf{\ipa{ʝi˩mi˩-ʐɤ˥qo˩}}} \textcolor{Sepia}{\selectlanguage{english}cow and calf} \zh{母牛与小牛}  
 \zh{量词}: \textcolor{darkblue}{\textbf{\ipa{pʰo˧˥}}}  \mytextsc{clf}: \textcolor{darkblue}{\textbf{\ipa{pʰo˧˥}}} 
\lhead{\firstmark}
\rhead{\botmark}

\subsection{\hspace{-0.5cm} {\Large \textcolor{darkblue}{\textbf{\ipa{ʝi˩næ˩-se˧}}}}\hspace{0.5cm}[\kern2pt{\textcolor{darkblue}{\textbf{\ipa{xxxx non-correspondance entre le nombre de morphèmes et le nombre de tons de morphèmes}}}}\kern2pt]} \hypertarget{j££i\string_Bn\{\string_B-se\string_M1}{}
\markboth{\textcolor{darkblue}{\textbf{\ipa{ʝi˩næ˩-se˧}}}}{}
\textcolor{teal}{\mytextsc{noun}} \hspace{4pt} Tone: L-M.
\textcolor{Sepia}{\selectlanguage{english}Kunming, and the Eastern part of the province of Yunnan.} \zh{云南,昆明……。}  Borrowing: Chinese  \zh{云南省}
 ¶ \textcolor{darkblue}{\textbf{\ipa{sɯ˧pʰi˧ | ʝi˩næ˩-se˧-qo˧ hɯ˧-ɲi˥!}}} \textcolor{Sepia}{\selectlanguage{english}The (feudal) lord has gone to Kunming!} \zh{土司到昆明去了!}  

\lhead{\firstmark}
\rhead{\botmark}

\subsection{\hspace{-0.5cm} {\Large \textcolor{darkblue}{\textbf{\ipa{ʝi˩ŋɤ˧˥}}}}\hspace{0.5cm}[\kern2pt{\textcolor{darkblue}{\textbf{\ipa{xxxx non-correspondance entre le nombre de morphèmes et le nombre de tons de morphèmes}}}}\kern2pt]} \hypertarget{j££i\string_BN7\string_M\string_T1}{}
\markboth{\textcolor{darkblue}{\textbf{\ipa{ʝi˩ŋɤ˧˥}}}}{}
\textcolor{teal}{\mytextsc{verb}} \hspace{4pt} Tone: LM+MH\#.
\textcolor{Sepia}{\selectlanguage{english}To bend (one's body) backwards.} \zh{往后仰。}  ¶ \textcolor{darkblue}{\textbf{\ipa{ʝi˩ŋɤ˧-ze˥}}} \textcolor{Sepia}{\selectlanguage{english}\mytextsc{pfv}} \zh{往后仰了}  
 ¶ \textcolor{darkblue}{\textbf{\ipa{ʝi˩ŋɤ˧˥ | tʰi˧-dzi˩}}} \textcolor{Sepia}{\selectlanguage{english}to be seated leaning backwards, to lean against the back of one's seat} \zh{坐着往后仰}  

\lhead{\firstmark}
\rhead{\botmark}

\subsection{\hspace{-0.5cm} {\Large \textcolor{darkblue}{\textbf{\ipa{ʝi˩qʰv̩˩}}}}\hspace{0.5cm}[\kern2pt{\textcolor{darkblue}{\textbf{\ipa{ʝi˧qʰv̩˧}}}}\kern2pt]} \hypertarget{j££i\string_Bq\string_hv\string_=\string_B1}{}
\markboth{\textcolor{darkblue}{\textbf{\ipa{ʝi˩qʰv̩˩}}}}{}
\textcolor{teal}{\mytextsc{noun}} \hspace{4pt} Tone: L.
\textcolor{Sepia}{\selectlanguage{english}Sleeve.} \zh{袖子。}  \zh{量词}: \textcolor{darkblue}{\textbf{\ipa{ɭɯ˧}}}  \mytextsc{clf}: \textcolor{darkblue}{\textbf{\ipa{ɭɯ˧}}} 
\lhead{\firstmark}
\rhead{\botmark}

\subsection{\hspace{-0.5cm} {\Large \textcolor{darkblue}{\textbf{\ipa{ʝi˩ʈʂæ˧˥}}}}\hspace{0.5cm}[\kern2pt{\textcolor{darkblue}{\textbf{\ipa{ʝi˩ʈʂæ˧˥}}}}\kern2pt]} \hypertarget{j££i\string_Bt`s`\{\string_M\string_T1}{}
\markboth{\textcolor{darkblue}{\textbf{\ipa{ʝi˩ʈʂæ˧˥}}}}{}
\textcolor{teal}{\mytextsc{noun}} \hspace{4pt} Tone: LM+MH\#.
\textcolor{Sepia}{\selectlanguage{english}Waist.} \zh{腰。}  \zh{量词}: \textcolor{darkblue}{\textbf{\ipa{ɭɯ˧}}}  \mytextsc{clf}: \textcolor{darkblue}{\textbf{\ipa{ɭɯ˧}}} 
\lhead{\firstmark}
\rhead{\botmark}

\subsection{\hspace{-0.5cm} {\Large \textcolor{darkblue}{\textbf{\ipa{ʝi˩˥}}}}\hspace{0.5cm}[\kern2pt{\textcolor{darkblue}{\textbf{\ipa{ʝi˩˥}}}}\kern2pt]} \hypertarget{j££i\string_B\string_T1}{}
\markboth{\textcolor{darkblue}{\textbf{\ipa{ʝi˩˥}}}}{}
\textcolor{teal}{\mytextsc{noun}} \hspace{4pt} Tone: LH.
\textcolor{Sepia}{\selectlanguage{english}Spot, pimple.} \zh{痘痘。}  ¶ \textcolor{darkblue}{\textbf{\ipa{ʝi˩ tʰv̩˩˥}}} \textcolor{Sepia}{\selectlanguage{english}to have spots, to get pimples} \zh{出痘痘}  
 \zh{量词}: \textcolor{darkblue}{\textbf{\ipa{ɭɯ˧}}}  \mytextsc{clf}: \textcolor{darkblue}{\textbf{\ipa{ɭɯ˧}}} 
\lhead{\firstmark}
\rhead{\botmark}

\newpage
\section*{\centering- \textcolor{darkblue}{\textbf{\ipa{k}}} -}
\subsection{\hspace{-0.5cm} {\Large \textcolor{darkblue}{\textbf{\ipa{kæ˧ʈʂe˧}}}}\hspace{0.5cm}[\kern2pt{\textcolor{darkblue}{\textbf{\ipa{kæ˧ʈʂe˧}}}}\kern2pt]} \hypertarget{k\{\string_Mt`s`e\string_M1}{}
\markboth{\textcolor{darkblue}{\textbf{\ipa{kæ˧ʈʂe˧}}}}{}
\textcolor{teal}{\mytextsc{noun}} \hspace{4pt} Tone: M.
\textcolor{Sepia}{\selectlanguage{english}Acupuncture needles; acupuncture.} \zh{针灸(汉语借词:干针)。} Local Chinese dialect:\zh{干针。} Borrowing: Chinese  \zh{干针}
 ¶ \textcolor{darkblue}{\textbf{\ipa{kæ˧ʈʂe˧ lɑ˧˥}}} \textcolor{Sepia}{\selectlanguage{english}to do an acupuncture session, to use acupuncture needles} \zh{扎针灸}  

\lhead{\firstmark}
\rhead{\botmark}

\subsection{\hspace{-0.5cm} {\Large \textcolor{darkblue}{\textbf{\ipa{kæ˧ʈʂɯ˧}}}}\hspace{0.5cm}[\kern2pt{\textcolor{darkblue}{\textbf{\ipa{kæ˧ʈʂɯ˧}}}}\kern2pt]} \hypertarget{k\{\string_Mt`s`M\string_M1}{}
\markboth{\textcolor{darkblue}{\textbf{\ipa{kæ˧ʈʂɯ˧}}}}{}
\textcolor{teal}{\mytextsc{noun}} \hspace{4pt} Tone: M.
\textcolor{Sepia}{\selectlanguage{english}Sugar cane.} \zh{甘蔗。}  Borrowing: Chinese  \zh{甘蔗}

\lhead{\firstmark}
\rhead{\botmark}

\subsection{\hspace{-0.5cm} {\Large \textcolor{darkblue}{\textbf{\ipa{kɤ˧dzi˧}}}}\hspace{0.5cm}[\kern2pt{\textcolor{darkblue}{\textbf{\ipa{kɤ˧dzi˧}}}}\kern2pt]} \hypertarget{k7\string_Mdzi\string_M1}{}
\markboth{\textcolor{darkblue}{\textbf{\ipa{kɤ˧dzi˧}}}}{}
\textcolor{teal}{\mytextsc{verb}} \hspace{4pt} Tone: M.
\textcolor{Sepia}{\selectlanguage{english}To take a seat, to get seated.} \zh{坐下(在饭桌)。}  ¶ \textcolor{darkblue}{\textbf{\ipa{ɑ˩ʁo˧-hĩ˧ | kɤ˧dzi˧-ze˧.}}} \textcolor{Sepia}{\selectlanguage{english}The members of the family took their seats / got seated.} \zh{家人入座了。}  

\lhead{\firstmark}
\rhead{\botmark}

\subsection{\hspace{-0.5cm} {\Large \textcolor{darkblue}{\textbf{\ipa{kɤ˧kɤ˩}}}}\hspace{0.5cm}[\kern2pt{\textcolor{darkblue}{\textbf{\ipa{kɤ˧kɤ˩}}}}\kern2pt]} \hypertarget{k7\string_Mk7\string_B1}{}
\markboth{\textcolor{darkblue}{\textbf{\ipa{kɤ˧kɤ˩}}}}{}
\textcolor{teal}{\mytextsc{adverb(ial)}} \hspace{4pt} Tone: L\#.
\textcolor{Sepia}{\selectlanguage{english}Next to, close to.} \zh{挨着(坐……)。}  ¶ \textcolor{darkblue}{\textbf{\ipa{(tso˧\textasciitilde{}tso˧) kɤ˧kɤ˩ | tʰi˧-tɕɯ˥}}} \textcolor{Sepia}{\selectlanguage{english}to arrange, to put in good order} \zh{摆整齐、使均匀,如:一排排挨着}  
 ¶ \textcolor{darkblue}{\textbf{\ipa{[Tiger2] kɤ˧kɤ˩ | tʰi˧-se˥}}} \textcolor{Sepia}{\selectlanguage{english}to walk in a line, one behind the other} \zh{并排走}  
 ¶ \textcolor{darkblue}{\textbf{\ipa{kɤ˧kɤ˩ | tʰi˧-dzi˩}}} \textcolor{Sepia}{\selectlanguage{english}to sit close to one another} \zh{挨着坐}  

\lhead{\firstmark}
\rhead{\botmark}

\subsection{\hspace{-0.5cm} {\Large \textcolor{darkblue}{\textbf{\ipa{kɤ˧ljɤ˩}}}}\hspace{0.5cm}[\kern2pt{\textcolor{darkblue}{\textbf{\ipa{kɤ˧ljɤ˩}}}}\kern2pt]} \hypertarget{k7\string_Mlj7\string_B1}{}
\markboth{\textcolor{darkblue}{\textbf{\ipa{kɤ˧ljɤ˩}}}}{}
\textcolor{teal}{\mytextsc{noun}} \hspace{4pt} Tone: L\#.
\textcolor{Sepia}{\selectlanguage{english}Chinese sorghum.} \zh{高粱(汉语借词)。}  Borrowing: Chinese  \zh{高粱}
\textit{See:} \hyperlink{}{\textcolor{darkblue}{\textbf{\ipa{hæ˧ɭɯ\#˥}}}} 
\lhead{\firstmark}
\rhead{\botmark}

\subsection{\hspace{-0.5cm} {\Large \textcolor{darkblue}{\textbf{\ipa{kɤ˧mi˧}}} \textsubscript{1}}\hspace{0.5cm}[\kern2pt{\textcolor{darkblue}{\textbf{\ipa{kɤ˧mi˧}}}}\kern2pt]} \hypertarget{k7\string_Mmi\string_M1}{}
\markboth{\textcolor{darkblue}{\textbf{\ipa{kɤ˧mi˧}}} \textsubscript{1}}{}
\textcolor{teal}{\mytextsc{noun}} \hspace{4pt} Tone: M.
\textcolor{Sepia}{\selectlanguage{english}Female falcon.} \zh{母隼。}  ¶ \textcolor{darkblue}{\textbf{\ipa{kɤ˩mi˩-kɤ˩pʰv̩˥}}} \textcolor{Sepia}{\selectlanguage{english}female falcon and male falcon} \zh{母隼与公隼}  
 \zh{量词}: \textcolor{darkblue}{\textbf{\ipa{mi˩}}}  \mytextsc{clf}: \textcolor{darkblue}{\textbf{\ipa{mi˩}}} 
\lhead{\firstmark}
\rhead{\botmark}

\subsection{\hspace{-0.5cm} {\Large \textcolor{darkblue}{\textbf{\ipa{kɤ˧mi˧}}} \textsubscript{2}}\hspace{0.5cm}[\kern2pt{\textcolor{darkblue}{\textbf{\ipa{kɤ˧mi˧}}}}\kern2pt]} \hypertarget{k7\string_Mmi\string_M2}{}
\markboth{\textcolor{darkblue}{\textbf{\ipa{kɤ˧mi˧}}} \textsubscript{2}}{}
\textcolor{teal}{\mytextsc{noun}} \hspace{4pt} Tone: M.
\textcolor{Sepia}{\selectlanguage{english}A large jar; a large bottle.} \zh{大坛子,大瓶。}  \zh{量词}: \textcolor{darkblue}{\textbf{\ipa{ɭɯ˧}}}  \mytextsc{clf}: \textcolor{darkblue}{\textbf{\ipa{ɭɯ˧}}} 
\lhead{\firstmark}
\rhead{\botmark}

\subsection{\hspace{-0.5cm} {\Large \textcolor{darkblue}{\textbf{\ipa{kɤ˧mv̩˧˥}}}}\hspace{0.5cm}[\kern2pt{\textcolor{darkblue}{\textbf{\ipa{kɤ˧mv̩˧˥}}}}\kern2pt]} \hypertarget{k7\string_Mmv\string_=\string_M\string_T1}{}
\markboth{\textcolor{darkblue}{\textbf{\ipa{kɤ˧mv̩˧˥}}}}{}
\textcolor{teal}{\mytextsc{noun}} \hspace{4pt} Tone: MH\#.
\textcolor{Sepia}{\selectlanguage{english}The Gemu mountain (Yongning).} \zh{格母山。}  ¶ \textcolor{darkblue}{\textbf{\ipa{ɬi˧di˩-kɤ˩mv̩˩}}} \textcolor{Sepia}{\selectlanguage{english}Mount Gemu, in Yongning} \zh{永宁格姆山}  
 ¶ \textcolor{darkblue}{\textbf{\ipa{kɤ˧mv̩˧-hæ̃˧kʰo˥}}} \textcolor{Sepia}{\selectlanguage{english}the Gemu princess: another name for Mount Gemu, considered as a female deity} \zh{格姆公主:格姆山别名(格姆山被看作女神)}  
 ¶ \textcolor{darkblue}{\textbf{\ipa{kɤ˧mv̩˧˥, | æ˧ʂæ˧, | ŋwɤ˧hɑ̃˩, | ʂwæ˧gv̩\#˥, | nɑ˩tsʰi˩˥ | -tɕʰɤ˧pɤ˧mi\#˥, | qv̩˧ɻ̍˧-ʈʂʰɑ˧nɑ˥ |}}} \textcolor{Sepia}{\selectlanguage{english}The six mountains of Yongning that carry a name and have a definite symbolic value. The other mountains do not have comparable symbolic value, and fewer people use specific names for them.} \zh{永宁地区有固定名字的六座山。其它山,没有重要的象征意义,因此也没有固定名称。}  

\lhead{\firstmark}
\rhead{\botmark}

\subsection{\hspace{-0.5cm} {\Large \textcolor{darkblue}{\textbf{\ipa{kɤ˧ʈʂɯ˩}}} \textsubscript{1}}\hspace{0.5cm}[\kern2pt{\textcolor{darkblue}{\textbf{\ipa{kɤ˧ʈʂɯ˩}}}}\kern2pt]} \hypertarget{k7\string_Mt`s`M\string_B1}{}
\markboth{\textcolor{darkblue}{\textbf{\ipa{kɤ˧ʈʂɯ˩}}} \textsubscript{1}}{}
\textcolor{teal}{\mytextsc{verb}} \hspace{4pt} Tone: L\#.
\textcolor{Sepia}{\selectlanguage{english}To tell.} \zh{讲。}  ¶ \textcolor{darkblue}{\textbf{\ipa{hĩ˧-ki˧ | tʰɑ˧-kɤ˧ʈʂɯ˩!}}} \textcolor{Sepia}{\selectlanguage{english}Don't tell it! / Don't tell anyone!} \zh{不要告诉人家!}  
 ¶ \textcolor{darkblue}{\textbf{\ipa{kɤ˧-tʰɑ˥-ʈʂɯ˩!}}} \textcolor{Sepia}{\selectlanguage{english}Don't tell it! / Don't tell anyone!} \zh{不要告诉人家!}  
 ¶ \textcolor{darkblue}{\textbf{\ipa{njɤ˧-ɳɯ˧ | kɤ˧ʈʂɯ˩-bi˩!}}} \textcolor{Sepia}{\selectlanguage{english}I'm going to (jump in and) say something!} \zh{我要说一点事情!}  
 ¶ \textcolor{darkblue}{\textbf{\ipa{no˧ | kɤ˧ʈʂɯ˩ dʑo˩-ɲi˩!}}} \textcolor{Sepia}{\selectlanguage{english}You have to say something!} \zh{你得说话啊!}  
 ¶ \textcolor{darkblue}{\textbf{\ipa{ʈʂʰɯ˧ | kɤ˧ʈʂɯ˩ | dʑɤ˩˥ | mɤ˧-mv̩˧-sɯ˥! / ʈʂʰɯ˧ | kɤ˧ʈʂɯ˩ dʑɤ˩˥ | mɤ˧-mv̩˧\textasciitilde{}mv̩˧-sɯ˥!}}} \textcolor{Sepia}{\selectlanguage{english}She does not really understand yet! (About a toddler aged 2 who does not yet speak distinctly or follow conversations)} \zh{她还不怎么听得懂话!(关于一个不会说话的两岁小孩)}  
 ¶ \textcolor{darkblue}{\textbf{\ipa{tʰɑ˧-kɤ˧ʈʂɯ˩! | hĩ˧ ɳv̩˧ tʰɑ˧-kʰɯ˩!}}} \textcolor{Sepia}{\selectlanguage{english}Don't talk about it! Don't let people know!} \zh{不要告诉(人家)!别让人家知道!}  
 ¶ \textcolor{darkblue}{\textbf{\ipa{hĩ˧-ki˧ | kɤ˧-mɤ˧-ʈʂɯ˩}}} \textcolor{Sepia}{\selectlanguage{english}not to tell people; (to do something secretly) without telling anyone} \zh{不跟人家说(自己做什么事)}  
 ¶ \textcolor{darkblue}{\textbf{\ipa{kɤ˧ʈʂɯ˩ ɲi˩}}} \textcolor{Sepia}{\selectlanguage{english}well-behaved, obedient (child) (literally: who listens to what (s)he is told)} \zh{听话,乖(来形容一个孩子)}  
\textit{See:} \hyperlink{}{\textcolor{darkblue}{\textbf{\ipa{kɤ˧ʈʂɯ˩}}} \textsubscript{2}} 
\lhead{\firstmark}
\rhead{\botmark}

\subsection{\hspace{-0.5cm} {\Large \textcolor{darkblue}{\textbf{\ipa{kɤ˧ʈʂɯ˩}}} \textsubscript{2}}\hspace{0.5cm}[\kern2pt{\textcolor{darkblue}{\textbf{\ipa{kɤ˧ʈʂɯ˩}}}}\kern2pt]} \hypertarget{k7\string_Mt`s`M\string_B2}{}
\markboth{\textcolor{darkblue}{\textbf{\ipa{kɤ˧ʈʂɯ˩}}} \textsubscript{2}}{}
\textcolor{teal}{\mytextsc{noun}} \hspace{4pt} Tone: L\#.
\textcolor{Sepia}{\selectlanguage{english}Speech.} \zh{话。}  ¶ \textcolor{darkblue}{\textbf{\ipa{kɤ˧ʈʂɯ˩ ʝi˩}}} \textcolor{Sepia}{\selectlanguage{english}to promise; to make an oath; also: to swear before the gods: when people had a disagreement that they were unable to settle, they would go to the monastery and present their point of view before the gods, swearing that they were telling the truth; the gods would then punish the guilty one (through plagues and misfortunes).} \zh{答应,誓、发誓。两个人发生矛盾的时候,如果无法协调,他们会去大寺,在神像前讲述他们各自的观点,发誓他们自己讲的是真的。神会惩罚说谎的人(他家会有祸害)。}  
 ¶ \textcolor{darkblue}{\textbf{\ipa{ʈʂʰɯ˧ | kɤ˧ʈʂɯ˩-ʝi˩}}} \textcolor{Sepia}{\selectlanguage{english}(s)he promises} \zh{他答应}  
 ¶ \textcolor{darkblue}{\textbf{\ipa{ʈʂʰɯ˧ | kɤ˧ʈʂɯ˩ | mɤ˧-ʝi˥!}}} \textcolor{Sepia}{\selectlanguage{english}(s)he has not promised!} \zh{他没有答应!}  
 ¶ \textcolor{darkblue}{\textbf{\ipa{hĩ˧-kɤ˧ʈʂɯ˥ ɲi˩}}} \textcolor{Sepia}{\selectlanguage{english}to listen to people's advice, to pay attention to what other people say (a good attitude in the consultant's view)} \zh{听别人的建议、把别人的话当回事}  
 ¶ \textcolor{darkblue}{\textbf{\ipa{hĩ˧-kɤ˧ʈʂɯ˥ | le˧-ɲi˥}}} \textcolor{Sepia}{\selectlanguage{english}as above} \zh{同上}  
 ¶ \textcolor{darkblue}{\textbf{\ipa{hĩ˧-kɤ˧ʈʂɯ˥ | mɤ˧-ɲi˥}}} \textcolor{Sepia}{\selectlanguage{english}to fail to listen to people's advice} \zh{听不进去别人的意见与建议}  
 \zh{量词}: \textcolor{darkblue}{\textbf{\ipa{kʰwɤ˥}}}  \mytextsc{clf}: \textcolor{darkblue}{\textbf{\ipa{kʰwɤ˥}}} \textit{See:} \hyperlink{}{\textcolor{darkblue}{\textbf{\ipa{kɤ˧ʈʂɯ˩}}} \textsubscript{1}} 
\lhead{\firstmark}
\rhead{\botmark}

\subsection{\hspace{-0.5cm} {\Large \textcolor{darkblue}{\textbf{\ipa{kɤ˧v̩\#˥}}}}\hspace{0.5cm}[\kern2pt{\textcolor{darkblue}{\textbf{\ipa{kɤ˧v̩˧}}}}\kern2pt]} \hypertarget{k7\string_Mv\string_=\#\string_T1}{}
\markboth{\textcolor{darkblue}{\textbf{\ipa{kɤ˧v̩\#˥}}}}{}
\textcolor{teal}{\mytextsc{noun}} \hspace{4pt} Tone: \#H.
\textcolor{Sepia}{\selectlanguage{english}Amulet.} \zh{护符,护身符。}  \zh{量词}: \textcolor{darkblue}{\textbf{\ipa{ɭɯ˧}}}  \mytextsc{clf}: \textcolor{darkblue}{\textbf{\ipa{ɭɯ˧}}} 
\lhead{\firstmark}
\rhead{\botmark}

\subsection{\hspace{-0.5cm} {\Large \textcolor{darkblue}{\textbf{\ipa{kɤ˧wɤ\#˥}}}}\hspace{0.5cm}[\kern2pt{\textcolor{darkblue}{\textbf{\ipa{kɤ˧wɤ˧}}}}\kern2pt]} \hypertarget{k7\string_Mw7\#\string_T1}{}
\markboth{\textcolor{darkblue}{\textbf{\ipa{kɤ˧wɤ\#˥}}}}{}
\textcolor{teal}{\mytextsc{noun}} \hspace{4pt} Tone: \#H.
\textcolor{Sepia}{\selectlanguage{english}Predestination, predestined affinity.} \zh{缘分、共同命运。}  ¶ \textcolor{darkblue}{\textbf{\ipa{kɤ˧wɤ˧-ljɤ˧˥}}} \textcolor{Sepia}{\selectlanguage{english}to have a predestined affinity; to have a common destiny} \zh{有缘分、有共同命运}  

\lhead{\firstmark}
\rhead{\botmark}

\subsection{\hspace{-0.5cm} {\Large \textcolor{darkblue}{\textbf{\ipa{kɤ˧zo\#˥}}}}\hspace{0.5cm}[\kern2pt{\textcolor{darkblue}{\textbf{\ipa{kɤ˧zo˧}}}}\kern2pt]} \hypertarget{k7\string_Mzo\#\string_T1}{}
\markboth{\textcolor{darkblue}{\textbf{\ipa{kɤ˧zo\#˥}}}}{}
\textcolor{teal}{\mytextsc{noun}} \hspace{4pt} Tone: \#H.
\textcolor{Sepia}{\selectlanguage{english}Masculine given name.} \zh{男性名字。} 
\lhead{\firstmark}
\rhead{\botmark}

\subsection{\hspace{-0.5cm} {\Large \textcolor{darkblue}{\textbf{\ipa{kɤ˩}}}}\hspace{0.5cm}[\kern2pt{\textcolor{darkblue}{\textbf{\ipa{kɤ˥}}}}\kern2pt]} \hypertarget{k7\string_B1}{}
\markboth{\textcolor{darkblue}{\textbf{\ipa{kɤ˩}}}}{}
\textcolor{teal}{\mytextsc{noun}} \hspace{4pt} Tone: L.
\textcolor{Sepia}{\selectlanguage{english}Bottle.} \zh{瓶子。}  ¶ \textcolor{darkblue}{\textbf{\ipa{ʐɯ˧-kɤ˩}}} \textcolor{Sepia}{\selectlanguage{english}wine bottle} \zh{酒瓶}  
 ¶ \textcolor{darkblue}{\textbf{\ipa{ʐɯ˧ ɖɯ˧-kɤ˩}}} \textcolor{Sepia}{\selectlanguage{english}one bottle of wine} \zh{一瓶酒}  
 \zh{量词}: \textcolor{darkblue}{\textbf{\ipa{ɭɯ˧}}}  \mytextsc{clf}: \textcolor{darkblue}{\textbf{\ipa{ɭɯ˧}}} 
\lhead{\firstmark}
\rhead{\botmark}

\subsection{\hspace{-0.5cm} {\Large \textcolor{darkblue}{\textbf{\ipa{kɤ˩\textsubscript{a}}}}}\hspace{0.5cm}[\kern2pt{\textcolor{darkblue}{\textbf{\ipa{kɤ˩˥}}}}\kern2pt]} \hypertarget{k7\string_Ba1}{}
\markboth{\textcolor{darkblue}{\textbf{\ipa{kɤ˩\textsubscript{a}}}}}{}
\textcolor{teal}{\mytextsc{classifier}} \hspace{4pt} Tone: L\textsubscript{a}.
\textcolor{Sepia}{\selectlanguage{english}A bottle of.} \zh{量词:瓶。}  ¶ \textcolor{darkblue}{\textbf{\ipa{kɤ˩zo˩˥}}} \textcolor{Sepia}{\selectlanguage{english}small bottle} \zh{一小瓶}  
 ¶ \textcolor{darkblue}{\textbf{\ipa{ʈʂʰɯ˧-kɤ˥}}} \textcolor{Sepia}{\selectlanguage{english}\mytextsc{dem} \string_ (tone: H\# / H\$)} \zh{\mytextsc{指示代词} \string_}  

\lhead{\firstmark}
\rhead{\botmark}

\subsection{\hspace{-0.5cm} {\Large \textcolor{darkblue}{\textbf{\ipa{kɤ˩\textasciitilde{}kɤ˧˥}}}}\hspace{0.5cm}[\kern2pt{\textcolor{darkblue}{\textbf{\ipa{kɤ˧kɤ˧˥}}}}\kern2pt]} \hypertarget{k7\string_B~k7\string_M\string_T1}{}
\markboth{\textcolor{darkblue}{\textbf{\ipa{kɤ˩\textasciitilde{}kɤ˧˥}}}}{}
\textcolor{teal}{\mytextsc{verb}} \hspace{4pt} Tone: MH.
\textcolor{Sepia}{\selectlanguage{english}To knock, to tap, to poke.} \zh{敲、拍。}  ¶ \textcolor{darkblue}{\textbf{\ipa{kʰi˧ kɤ˥\textasciitilde{}kɤ˩}}} \textcolor{Sepia}{\selectlanguage{english}to knock at the door} \zh{敲门}  
 ¶ \textcolor{darkblue}{\textbf{\ipa{njɤ˧-ɳɯ˧ | no˧ | kɤ˩\textasciitilde{}kɤ˧-bi˥!}}} \textcolor{Sepia}{\selectlanguage{english}I am going to slap your buttocks! (An adult threatens a child.)} \zh{我要打你屁股了!(大人对孩子说)}  
 ¶ \textcolor{darkblue}{\textbf{\ipa{ʈʂo˧tsɯ˥ kɤ˩\textasciitilde{}kɤ˩ (-ze˩/-bi˩)}}} \textcolor{Sepia}{\selectlanguage{english}to tap the table, to rap on the table} \zh{拍拍桌子}  
 ¶ \textcolor{darkblue}{\textbf{\ipa{gv̩˧dv̩˧ kɤ˧\textasciitilde{}kɤ˩}}} \textcolor{Sepia}{\selectlanguage{english}to tap someone's back (to relieve back pain)} \zh{敲敲背}  
\textit{See:} \hyperlink{}{\textcolor{darkblue}{\textbf{\ipa{kɤ˧˥}}}} 
\lhead{\firstmark}
\rhead{\botmark}

\subsection{\hspace{-0.5cm} {\Large \textcolor{darkblue}{\textbf{\ipa{kɤ˩lo˧˥}}}}\hspace{0.5cm}[\kern2pt{\textcolor{darkblue}{\textbf{\ipa{kɤ˩lo˧˥}}}}\kern2pt]} \hypertarget{k7\string_Blo\string_M\string_T1}{}
\markboth{\textcolor{darkblue}{\textbf{\ipa{kɤ˩lo˧˥}}}}{}
\textcolor{teal}{\mytextsc{noun}} \hspace{4pt} Tone: LM+MH\#.
\textcolor{Sepia}{\selectlanguage{english}Branch.} \zh{树枝。}  ¶ \textcolor{darkblue}{\textbf{\ipa{si˧dzi˩-kɤ˩lo˩}}} \textcolor{Sepia}{\selectlanguage{english}branch of tree} \zh{树枝}  
 ¶ \textcolor{darkblue}{\textbf{\ipa{si˧-kɤ˥lo˩}}} \textcolor{Sepia}{\selectlanguage{english}as above} \zh{同上}  
 \zh{量词}: \textcolor{darkblue}{\textbf{\ipa{kɤ˧˥}}}  \mytextsc{clf}: \textcolor{darkblue}{\textbf{\ipa{kɤ˧˥}}} 
\lhead{\firstmark}
\rhead{\botmark}

\subsection{\hspace{-0.5cm} {\Large \textcolor{darkblue}{\textbf{\ipa{kɤ˩-nɑ˧mi˧}}}}\hspace{0.5cm}[\kern2pt{\textcolor{darkblue}{\textbf{\ipa{kɤ˧nɑ˧mi˧}}}}\kern2pt]} \hypertarget{k7\string_B-nA\string_Mmi\string_M1}{}
\markboth{\textcolor{darkblue}{\textbf{\ipa{kɤ˩-nɑ˧mi˧}}}}{}
\textcolor{teal}{\mytextsc{noun}} \hspace{4pt} Tone: L-.
\textcolor{Sepia}{\selectlanguage{english}Eagle.} \zh{老鹰。}  \zh{量词}: \textcolor{darkblue}{\textbf{\ipa{mi˩}}}  \mytextsc{clf}: \textcolor{darkblue}{\textbf{\ipa{mi˩}}} 
\lhead{\firstmark}
\rhead{\botmark}

\subsection{\hspace{-0.5cm} {\Large \textcolor{darkblue}{\textbf{\ipa{kɤ˩pʰv̩˩}}}}\hspace{0.5cm}[\kern2pt{\textcolor{darkblue}{\textbf{\ipa{kɤ˩pʰv̩˩˥}}}}\kern2pt]} \hypertarget{k7\string_Bp\string_hv\string_=\string_B1}{}
\markboth{\textcolor{darkblue}{\textbf{\ipa{kɤ˩pʰv̩˩}}}}{}
\textcolor{teal}{\mytextsc{noun}} \hspace{4pt} Tone: L.
\textcolor{Sepia}{\selectlanguage{english}Male falcon.} \zh{公隼。}  ¶ \textcolor{darkblue}{\textbf{\ipa{kɤ˩pʰv̩˩-kɤ˩mi˥}}} \textcolor{Sepia}{\selectlanguage{english}male falcon and female falcon} \zh{公隼与母隼}  
 \zh{量词}: \textcolor{darkblue}{\textbf{\ipa{mi˩}}}  \mytextsc{clf}: \textcolor{darkblue}{\textbf{\ipa{mi˩}}} 
\lhead{\firstmark}
\rhead{\botmark}

\subsection{\hspace{-0.5cm} {\Large \textcolor{darkblue}{\textbf{\ipa{kɤ˩-tjɤ˧ljɤ\#˥}}}}\hspace{0.5cm}[\kern2pt{\textcolor{darkblue}{\textbf{\ipa{kɤ˧tjɤ˧ljɤ˧}}}}\kern2pt]} \hypertarget{k7\string_B-tj7\string_Mlj7\#\string_T1}{}
\markboth{\textcolor{darkblue}{\textbf{\ipa{kɤ˩-tjɤ˧ljɤ\#˥}}}}{}
\textcolor{teal}{\mytextsc{noun}} \hspace{4pt} Tone: L-\#H.
\textcolor{Sepia}{\selectlanguage{english}Small bell hung to an animal's neck (e.g. horse's bell).} \zh{铃铛。}  \zh{量词}: \textcolor{darkblue}{\textbf{\ipa{ɭɯ˧}}}  \mytextsc{clf}: \textcolor{darkblue}{\textbf{\ipa{ɭɯ˧}}} 
\lhead{\firstmark}
\rhead{\botmark}

\subsection{\hspace{-0.5cm} {\Large \textcolor{darkblue}{\textbf{\ipa{kɤ˩zo˩}}} \textsubscript{1}}\hspace{0.5cm}[\kern2pt{\textcolor{darkblue}{\textbf{\ipa{kɤ˩zo˩˥}}}}\kern2pt]} \hypertarget{k7\string_Bzo\string_B1}{}
\markboth{\textcolor{darkblue}{\textbf{\ipa{kɤ˩zo˩}}} \textsubscript{1}}{}
\textcolor{teal}{\mytextsc{noun}} \hspace{4pt} Tone: L.
\textcolor{Sepia}{\selectlanguage{english}Baby falcon.} \zh{小隼。} 
\lhead{\firstmark}
\rhead{\botmark}

\subsection{\hspace{-0.5cm} {\Large \textcolor{darkblue}{\textbf{\ipa{kɤ˩zo˩}}} \textsubscript{2}}\hspace{0.5cm}[\kern2pt{\textcolor{darkblue}{\textbf{\ipa{kɤ˩zo˩˥}}}}\kern2pt]} \hypertarget{k7\string_Bzo\string_B2}{}
\markboth{\textcolor{darkblue}{\textbf{\ipa{kɤ˩zo˩}}} \textsubscript{2}}{}
\textcolor{teal}{\mytextsc{noun}} \hspace{4pt} Tone: L.
\textcolor{Sepia}{\selectlanguage{english}Small bottle.} \zh{小瓶子。} 
\lhead{\firstmark}
\rhead{\botmark}

\subsection{\hspace{-0.5cm} {\Large \textcolor{darkblue}{\textbf{\ipa{kɤ˩xxxx}}}}\hspace{0.5cm}[\kern2pt{\textcolor{darkblue}{\textbf{\ipa{xxxx groupe tonal entier sans aucun ton}}}}\kern2pt]} \hypertarget{k7\string_Bxxxx1}{}
\markboth{\textcolor{darkblue}{\textbf{\ipa{kɤ˩xxxx}}}}{}
\textcolor{teal}{\mytextsc{verb}} \hspace{4pt} Tone: xxxx vérifier : L\textsubscript{a}? L\textsubscript{b}?.
\textcolor{Sepia}{\selectlanguage{english}To saw.} \zh{锯(木头)。} 
\lhead{\firstmark}
\rhead{\botmark}

\subsection{\hspace{-0.5cm} {\Large \textcolor{darkblue}{\textbf{\ipa{kɤ˧˥}}}}\hspace{0.5cm}[\kern2pt{\textcolor{darkblue}{\textbf{\ipa{kɤ˧˥}}}}\kern2pt]} \hypertarget{k7\string_M\string_T1}{}
\markboth{\textcolor{darkblue}{\textbf{\ipa{kɤ˧˥}}}}{}
\textcolor{teal}{\mytextsc{verb}} \hspace{4pt} Tone: MH.
\textcolor{Sepia}{\selectlanguage{english}To knock on the door.} \zh{敲门。}  ¶ \textcolor{darkblue}{\textbf{\ipa{tʰi˧-kɤ˧˥}}} \textcolor{Sepia}{\selectlanguage{english}\mytextsc{dur}} \zh{\mytextsc{dur}}  
\textit{See:} \textcolor{darkblue}{\textbf{\ipa{kɤ˩kɤ˧˥}}} 
\lhead{\firstmark}
\rhead{\botmark}

\subsection{\hspace{-0.5cm} {\Large \textcolor{darkblue}{\textbf{\ipa{kɤ˧˥\textsubscript{a}}}} \textsubscript{1}}\hspace{0.5cm}[\kern2pt{\textcolor{darkblue}{\textbf{\ipa{kɤ˧˥}}}}\kern2pt]} \hypertarget{k7\string_M\string_Ta1}{}
\markboth{\textcolor{darkblue}{\textbf{\ipa{kɤ˧˥\textsubscript{a}}}} \textsubscript{1}}{}
\textcolor{teal}{\mytextsc{classifier}} \hspace{4pt} Tone: MH\textsubscript{a}.
\textcolor{Sepia}{\selectlanguage{english}Classifier for sticks/rods.} \zh{量词:棍子、树枝(一根)。}  ¶ \textcolor{darkblue}{\textbf{\ipa{si˧-kɤ˧˥ | ɖɯ˧-kɤ˧˥}}} \textcolor{Sepia}{\selectlanguage{english}a branch (of a tree)} \zh{一根树枝}  

\lhead{\firstmark}
\rhead{\botmark}

\subsection{\hspace{-0.5cm} {\Large \textcolor{darkblue}{\textbf{\ipa{kɤ˧˥\textsubscript{a}}}} \textsubscript{2}}\hspace{0.5cm}[\kern2pt{\textcolor{darkblue}{\textbf{\ipa{kɤ˧˥}}}}\kern2pt]} \hypertarget{k7\string_M\string_Ta2}{}
\markboth{\textcolor{darkblue}{\textbf{\ipa{kɤ˧˥\textsubscript{a}}}} \textsubscript{2}}{}
\textcolor{teal}{\mytextsc{classifier}} \hspace{4pt} Tone: MH\textsubscript{a}.
\textcolor{Sepia}{\selectlanguage{english}A tract of land.} \zh{量词:地(一片)。} 
\lhead{\firstmark}
\rhead{\botmark}

\subsection{\hspace{-0.5cm} {\Large \textcolor{darkblue}{\textbf{\ipa{kɤ˧˥tʰɑ˩}}}}\hspace{0.5cm}[\kern2pt{\textcolor{darkblue}{\textbf{\ipa{kɤ˧tʰɑ˧˥}}}}\kern2pt]} \hypertarget{k7\string_M\string_Tt\string_hA\string_B1}{}
\markboth{\textcolor{darkblue}{\textbf{\ipa{kɤ˧˥tʰɑ˩}}}}{}
\textcolor{teal}{\mytextsc{noun}} \hspace{4pt} Tone: MH+L.
\textcolor{Sepia}{\selectlanguage{english}A family name from Yongning. There are two families in Yongning that carry this name. This is one of the first three clans who settled in the vicinity of the Yongning monastery, the other two being \textcolor{darkblue}{\textbf{\ipa{/ə˧lɑ˧/}}} and \textcolor{darkblue}{\textbf{\ipa{/lɑ˧tʰɑ˧mi˥\$/}}}.} \zh{一个姓。这个姓,永宁有两家。}  ¶ \textcolor{darkblue}{\textbf{\ipa{kɤ˧˥tʰɑ˩=ɻ̍˩}}} \textcolor{Sepia}{\selectlanguage{english}the \textcolor{darkblue}{\textbf{\ipa{/kɤ˧˥tʰɑ˩/}}} clan} \zh{\textcolor{darkblue}{\textbf{\ipa{/kɤ˧˥tʰɑ˩/}}}家族}  

\lhead{\firstmark}
\rhead{\botmark}

\subsection{\hspace{-0.5cm} {\Large \textcolor{darkblue}{\textbf{\ipa{kɤ˩˧}}}}\hspace{0.5cm}[\kern2pt{\textcolor{darkblue}{\textbf{\ipa{kɤ˩˥}}}}\kern2pt]} \hypertarget{k7\string_B\string_M1}{}
\markboth{\textcolor{darkblue}{\textbf{\ipa{kɤ˩˧}}}}{}
\textcolor{teal}{\mytextsc{noun}} \hspace{4pt} Tone: LM.
\textcolor{Sepia}{\selectlanguage{english}Falcon.} \zh{隼、“小鹰”。}  ¶ \textcolor{darkblue}{\textbf{\ipa{kɤ˩ hwæ˧-ze˧}}} \textcolor{Sepia}{\selectlanguage{english}...bought (a) falcon} \zh{买了隼}  
 ¶ \textcolor{darkblue}{\textbf{\ipa{kɤ˩ dzɯ˧-ze˩}}} \textcolor{Sepia}{\selectlanguage{english}...ate (a) falcon} \zh{吃了隼}  
 \zh{量词}: \textcolor{darkblue}{\textbf{\ipa{mi˩}}}  \mytextsc{clf}: \textcolor{darkblue}{\textbf{\ipa{mi˩}}} 
\lhead{\firstmark}
\rhead{\botmark}

\subsection{\hspace{-0.5cm} {\Large \textcolor{darkblue}{\textbf{\ipa{ki˥\textsubscript{a}}}}}\hspace{0.5cm}[\kern2pt{\textcolor{darkblue}{\textbf{\ipa{ki˥}}}}\kern2pt]} \hypertarget{ki\string_Ta1}{}
\markboth{\textcolor{darkblue}{\textbf{\ipa{ki˥\textsubscript{a}}}}}{}
\textcolor{teal}{\mytextsc{classifier}} \hspace{4pt} Tone: H\textsubscript{a}.
\textcolor{Sepia}{\selectlanguage{english}In association with the numeral 'one', this classifier means 'together'.} \zh{量词:加上数词‘一’,这个量词表示‘一起’。}  ¶ \textcolor{darkblue}{\textbf{\ipa{ɖɯ˧-ki˥}}} \textcolor{Sepia}{\selectlanguage{english}together} \zh{一起(共事)}  
 ¶ \textcolor{darkblue}{\textbf{\ipa{ɖɯ˧-ki˧ tʰv̩˧}}} \textcolor{Sepia}{\selectlanguage{english}to arrive together/at the same time} \zh{同时到达}  
 ¶ \textcolor{darkblue}{\textbf{\ipa{ɖɯ˧-ʝi˧-ɳɯ˧ tsʰɯ˧˥, | ɖɯ˧-ki˧ tʰv̩˧!}}} \textcolor{Sepia}{\selectlanguage{english}Coming from the same place, we arrive together!} \zh{从一个地方,一起到!}  
 ¶ \textcolor{darkblue}{\textbf{\ipa{ɖɯ˧-ki˧ dzi˧˥}}} \textcolor{Sepia}{\selectlanguage{english}to live together} \zh{住在一起}  

\lhead{\firstmark}
\rhead{\botmark}

\subsection{\hspace{-0.5cm} {\Large \textcolor{darkblue}{\textbf{\ipa{‑ki˧}}}}\hspace{0.5cm}[\kern2pt{\textcolor{darkblue}{\textbf{\ipa{ki˥}}}}\kern2pt]} \hypertarget{‑ki\string_M1}{}
\markboth{\textcolor{darkblue}{\textbf{\ipa{‑ki˧}}}}{}
\textcolor{teal}{\mytextsc{suffix}} \hspace{4pt} Tone: M.
\textcolor{Sepia}{\selectlanguage{english}Dative (particle indicating the recipient) / allative (indicating a direction).} \zh{对\mytextsc{与格}/向格。}  ¶ \textcolor{darkblue}{\textbf{\ipa{ʈʂʰɯ˧-ki˧ ʐwɤ˧˥}}} \textcolor{Sepia}{\selectlanguage{english}to speak to her/him} \zh{给他讲}  
 ¶ \textcolor{darkblue}{\textbf{\ipa{ʈʂʰɯ˧-ki˧ ʐwɤ˧-bi˥}}} \textcolor{Sepia}{\selectlanguage{english}as above, with immediate future} \zh{要给他讲}  
 ¶ \textcolor{darkblue}{\textbf{\ipa{ə˧mɑ˧-ɳɯ˧ | njɤ˧-ki˧ | nɑ˩-ʐwɤ˧ so˩!}}} \textcolor{Sepia}{\selectlanguage{english}Ama teaches me the Na language!} \zh{阿妈教我摩梭话!}  

\lhead{\firstmark}
\rhead{\botmark}

\subsection{\hspace{-0.5cm} {\Large \textcolor{darkblue}{\textbf{\ipa{ki˧\textsubscript{a}}}}}\hspace{0.5cm}[\kern2pt{\textcolor{darkblue}{\textbf{\ipa{ki˥}}}}\kern2pt]} \hypertarget{ki\string_Ma1}{}
\markboth{\textcolor{darkblue}{\textbf{\ipa{ki˧\textsubscript{a}}}}}{}
\textcolor{teal}{\mytextsc{verb}} \hspace{4pt} Tone: M\textsubscript{a}.
\textcolor{Sepia}{\selectlanguage{english}To give, to pass on, to transmit, to offer.} \zh{给、传、献给、发(工资),嫁给。}  ¶ \textcolor{darkblue}{\textbf{\ipa{ki˧\textasciitilde{}ki˩}}} \textcolor{Sepia}{\selectlanguage{english}\mytextsc{red}} \zh{重叠}  
 ¶ \textcolor{darkblue}{\textbf{\ipa{tso˧\textasciitilde{}tso˧-ki˩}}} \textcolor{Sepia}{\selectlanguage{english}to give things} \zh{给东西}  
 ¶ \textcolor{darkblue}{\textbf{\ipa{tso˧\textasciitilde{}tso˧ ki˧\textasciitilde{}ki˥}}} \textcolor{Sepia}{\selectlanguage{english}to give things} \zh{给东西}  
 ¶ \textcolor{darkblue}{\textbf{\ipa{hĩ˧-ki˧ ki˩}}} \textcolor{Sepia}{\selectlanguage{english}1. to give to someone. 2. to give oneself (in marriage) to someone, to marry someone (for a woman)} \zh{1.许配给人家。2.嫁给人}  
 ¶ \textcolor{darkblue}{\textbf{\ipa{hĩ˧-ki˧ | ɖwæ˧˥ | tʰi˧-ki˧}}} \textcolor{Sepia}{\selectlanguage{english}to be generous, to be open-handed} \zh{大方}  
 ¶ \textcolor{darkblue}{\textbf{\ipa{hĩ˧-ki˧ ki˩ fv̩˩}}} \textcolor{Sepia}{\selectlanguage{english}to like to make gifts, to like to give things to people} \zh{爱送礼,爱给别人送东西}  
 ¶ \textcolor{darkblue}{\textbf{\ipa{pʰɤ˧bɤ˧ ki˧ (-bi˧)}}} \textcolor{Sepia}{\selectlanguage{english}to offer gifts} \zh{送礼物}  
 ¶ \textcolor{darkblue}{\textbf{\ipa{hɑ˧ ki˩}}} \textcolor{Sepia}{\selectlanguage{english}to feed, to give food} \zh{喂饭}  

\lhead{\firstmark}
\rhead{\botmark}

\subsection{\hspace{-0.5cm} {\Large \textcolor{darkblue}{\textbf{\ipa{ki˧li˥}}}}\hspace{0.5cm}[\kern2pt{\textcolor{darkblue}{\textbf{\ipa{ki˧li˥}}}}\kern2pt]} \hypertarget{ki\string_Mli\string_T1}{}
\markboth{\textcolor{darkblue}{\textbf{\ipa{ki˧li˥}}}}{}
\textcolor{teal}{\mytextsc{adverb(ial)}} \hspace{4pt} Tone: H\#.
\textcolor{Sepia}{\selectlanguage{english}In a mess.} \zh{乱七八糟。} 
\lhead{\firstmark}
\rhead{\botmark}

\subsection{\hspace{-0.5cm} {\Large \textcolor{darkblue}{\textbf{\ipa{ki˧zo\#˥}}}}\hspace{0.5cm}[\kern2pt{\textcolor{darkblue}{\textbf{\ipa{ki˧zo˧}}}}\kern2pt]} \hypertarget{ki\string_Mzo\#\string_T1}{}
\markboth{\textcolor{darkblue}{\textbf{\ipa{ki˧zo\#˥}}}}{}
\textcolor{teal}{\mytextsc{noun}} \hspace{4pt} Tone: \#H.
\textcolor{Sepia}{\selectlanguage{english}A unixex given name: a given name used for both men and women.} \zh{男女通用名。} 
\lhead{\firstmark}
\rhead{\botmark}

\subsection{\hspace{-0.5cm} {\Large \textcolor{darkblue}{\textbf{\ipa{ki˩\textsubscript{a}}}}}\hspace{0.5cm}[\kern2pt{\textcolor{darkblue}{\textbf{\ipa{ki˩˥}}}}\kern2pt]} \hypertarget{ki\string_Ba1}{}
\markboth{\textcolor{darkblue}{\textbf{\ipa{ki˩\textsubscript{a}}}}}{}
\textcolor{teal}{\mytextsc{verb}} \hspace{4pt} Tone: L\textsubscript{a}.
\textcolor{Sepia}{\selectlanguage{english}To put on (a skirt, trousers...).} \zh{穿上(裤子、袜子、鞋子)。}  ¶ \textcolor{darkblue}{\textbf{\ipa{ɬi˧qʰwɤ˩ | ɖɯ˧-ɭɯ˧ ki˩}}} \textcolor{Sepia}{\selectlanguage{english}to put on trousers} \zh{穿上裤子}  
 ¶ \textcolor{darkblue}{\textbf{\ipa{dzɑ˩qʰwɤ˩˥ | ɖɯ˧-dzi˧ ki˩}}} \textcolor{Sepia}{\selectlanguage{english}to put on a pair of shoes} \zh{穿上一双鞋}  
 ¶ \textcolor{darkblue}{\textbf{\ipa{ʈʰæ˩ ki˩˥}}} \textcolor{Sepia}{\selectlanguage{english}'to put on a skirt'; this is the name of the ritual of entry into adulthood, after a girl has reached age 13} \zh{“穿裙”:这是成年礼的名称(穿裙礼:女孩满13岁,即为成人)}  
 ¶ \textcolor{darkblue}{\textbf{\ipa{ɬi˧ ki˥}}} \textcolor{Sepia}{\selectlanguage{english}'to put on trousers'; this is the name of the ritual of entry into adulthood, after a boy has reached age 13} \zh{“穿裤”:这是成年礼的名称(穿裤礼:男孩满了13岁,即为成人)}  

\lhead{\firstmark}
\rhead{\botmark}

\subsection{\hspace{-0.5cm} {\Large \textcolor{darkblue}{\textbf{\ipa{ki˩mi˧}}}}\hspace{0.5cm}[\kern2pt{\textcolor{darkblue}{\textbf{\ipa{ki˩mi˥}}}}\kern2pt]} \hypertarget{ki\string_Bmi\string_M1}{}
\markboth{\textcolor{darkblue}{\textbf{\ipa{ki˩mi˧}}}}{}
\textcolor{teal}{\mytextsc{noun}} \hspace{4pt} Tone: LM.
\textcolor{Sepia}{\selectlanguage{english}A large fly with a green head; its larvas are particularly harmful.} \zh{绿头苍蝇。}  \zh{量词}: \textcolor{darkblue}{\textbf{\ipa{mi˩}}}  \mytextsc{clf}: \textcolor{darkblue}{\textbf{\ipa{mi˩}}} 
\lhead{\firstmark}
\rhead{\botmark}

\subsection{\hspace{-0.5cm} {\Large \textcolor{darkblue}{\textbf{\ipa{ki˩tɑ\#˥}}}}\hspace{0.5cm}[\kern2pt{\textcolor{darkblue}{\textbf{\ipa{ki˩tɑ˥}}}}\kern2pt]} \hypertarget{ki\string_BtA\#\string_T1}{}
\markboth{\textcolor{darkblue}{\textbf{\ipa{ki˩tɑ\#˥}}}}{}
\textcolor{teal}{\mytextsc{noun}} \hspace{4pt} Tone: LM+\#H.
\textcolor{Sepia}{\selectlanguage{english}Bag made of leather and linen, in which silver coins used to be kept, buried somewhere in the house to hide it from robbers.} \zh{皮袋,来装家里的财物:金币、银币……这个皮袋,埋在房子里的一个保密地方,防贼。为了让它很结实,袋子有四、五层麻布内衬。可以保存很长时间。}  ¶ \textcolor{darkblue}{\textbf{\ipa{æ˧-tse˥pʰæ˩ | ɖɯ˧-ki˩tɑ˩}}} \textcolor{Sepia}{\selectlanguage{english}a bag of bronze coins} \zh{一袋铜币}  

\lhead{\firstmark}
\rhead{\botmark}

\subsection{\hspace{-0.5cm} {\Large \textcolor{darkblue}{\textbf{\ipa{ki˩ti\#˥}}}}\hspace{0.5cm}[\kern2pt{\textcolor{darkblue}{\textbf{\ipa{ki˩ti˥}}}}\kern2pt]} \hypertarget{ki\string_Bti\#\string_T1}{}
\markboth{\textcolor{darkblue}{\textbf{\ipa{ki˩ti\#˥}}}}{}
\textcolor{teal}{\mytextsc{noun}} \hspace{4pt} Tone: LM+\#H.
\textcolor{Sepia}{\selectlanguage{english}Leather belt.} \zh{皮腰带。}  \zh{量词}: \textcolor{darkblue}{\textbf{\ipa{kʰɯ˩}}}  \mytextsc{clf}: \textcolor{darkblue}{\textbf{\ipa{kʰɯ˩}}} 
\lhead{\firstmark}
\rhead{\botmark}

\subsection{\hspace{-0.5cm} {\Large \textcolor{darkblue}{\textbf{\ipa{ko˥}}}}\hspace{0.5cm}[\kern2pt{\textcolor{darkblue}{\textbf{\ipa{ko˥}}}}\kern2pt]} \hypertarget{ko\string_T1}{}
\markboth{\textcolor{darkblue}{\textbf{\ipa{ko˥}}}}{}
\textcolor{teal}{\mytextsc{noun}} \hspace{4pt} Tone: \#H.
\textcolor{Sepia}{\selectlanguage{english}Hill, small mountain.} \zh{小山。}  \zh{量词}: \textcolor{darkblue}{\textbf{\ipa{ɭɯ˧}}}  \mytextsc{clf}: \textcolor{darkblue}{\textbf{\ipa{ɭɯ˧}}} 
\lhead{\firstmark}
\rhead{\botmark}

\subsection{\hspace{-0.5cm} {\Large \textcolor{darkblue}{\textbf{\ipa{ko˧\textsubscript{a}}}}}\hspace{0.5cm}[\kern2pt{\textcolor{darkblue}{\textbf{\ipa{ko˥}}}}\kern2pt]} \hypertarget{ko\string_Ma1}{}
\markboth{\textcolor{darkblue}{\textbf{\ipa{ko˧\textsubscript{a}}}}}{}
\textcolor{teal}{\mytextsc{classifier}} \hspace{4pt} Tone: M\textsubscript{a}.
\textcolor{Sepia}{\selectlanguage{english}Classifier for small objects, e.g. cigarettes.} \zh{量词:小东西,例如烟(一只)。} 
\lhead{\firstmark}
\rhead{\botmark}

\subsection{\hspace{-0.5cm} {\Large \textcolor{darkblue}{\textbf{\ipa{ko˧ɖæ\#˥}}}}\hspace{0.5cm}[\kern2pt{\textcolor{darkblue}{\textbf{\ipa{ko˧ɖæ˧}}}}\kern2pt]} \hypertarget{ko\string_Md`\{\#\string_T1}{}
\markboth{\textcolor{darkblue}{\textbf{\ipa{ko˧ɖæ\#˥}}}}{}
\textcolor{teal}{\mytextsc{noun}} \hspace{4pt} Tone: \#H.
\textcolor{Sepia}{\selectlanguage{english}Sculpture of Buddha (Tibetan borrowing).} \zh{佛像。}  Borrowing: Tibetan  sku-vdra (sku-'dra)
 ¶ \textcolor{darkblue}{\textbf{\ipa{ko˧ɖæ˧-zo˧}}} \textcolor{Sepia}{\selectlanguage{english}small statue of Buddha} \zh{小佛像}  
 \zh{量词}: \textcolor{darkblue}{\textbf{\ipa{ɭɯ˧}}}  \mytextsc{clf}: \textcolor{darkblue}{\textbf{\ipa{ɭɯ˧}}} 
\lhead{\firstmark}
\rhead{\botmark}

\subsection{\hspace{-0.5cm} {\Large \textcolor{darkblue}{\textbf{\ipa{ko˧li\#˥}}}}\hspace{0.5cm}[\kern2pt{\textcolor{darkblue}{\textbf{\ipa{ko˧li˧}}}}\kern2pt]} \hypertarget{ko\string_Mli\#\string_T1}{}
\markboth{\textcolor{darkblue}{\textbf{\ipa{ko˧li\#˥}}}}{}
\textcolor{teal}{\mytextsc{noun}} \hspace{4pt} Tone: \#H.
\textcolor{Sepia}{\selectlanguage{english}Blow tube: tube to blow on a fire.} \zh{吹火筒,用来吹火的小管子。}  \zh{量词}: \textcolor{darkblue}{\textbf{\ipa{ɭɯ˧}}}  \mytextsc{clf}: \textcolor{darkblue}{\textbf{\ipa{ɭɯ˧}}} 
\lhead{\firstmark}
\rhead{\botmark}

\subsection{\hspace{-0.5cm} {\Large \textcolor{darkblue}{\textbf{\ipa{ko˧no˧-ʁo\#˥}}}}\hspace{0.5cm}[\kern2pt{\textcolor{darkblue}{\textbf{\ipa{xxxx non-correspondance entre le nombre de morphèmes et le nombre de tons de morphèmes}}}}\kern2pt]} \hypertarget{ko\string_Mno\string_M-Ro\#\string_T1}{}
\markboth{\textcolor{darkblue}{\textbf{\ipa{ko˧no˧-ʁo\#˥}}}}{}
\textcolor{teal}{\mytextsc{noun}} \hspace{4pt} Tone: \#H.
\textcolor{Sepia}{\selectlanguage{english}Mountain ridge; bridge in the mountains.} \zh{山梁。}  \zh{量词}: \textcolor{darkblue}{\textbf{\ipa{kʰwɤ˥}}}  \mytextsc{clf}: \textcolor{darkblue}{\textbf{\ipa{kʰwɤ˥}}} 
\lhead{\firstmark}
\rhead{\botmark}

\subsection{\hspace{-0.5cm} {\Large \textcolor{darkblue}{\textbf{\ipa{ko˧sɯ\#˥}}}}\hspace{0.5cm}[\kern2pt{\textcolor{darkblue}{\textbf{\ipa{ko˧sɯ˧}}}}\kern2pt]} \hypertarget{ko\string_MsM\#\string_T1}{}
\markboth{\textcolor{darkblue}{\textbf{\ipa{ko˧sɯ\#˥}}}}{}
\textcolor{teal}{\mytextsc{noun}} \hspace{4pt} Tone: \#H.
\textcolor{Sepia}{\selectlanguage{english}Shop.} \zh{商店、小卖部(汉语借词:公司)。}  Borrowing: Chinese  \zh{公司}

\lhead{\firstmark}
\rhead{\botmark}

\subsection{\hspace{-0.5cm} {\Large \textcolor{darkblue}{\textbf{\ipa{ko˩\textsubscript{a}}}}}\hspace{0.5cm}[\kern2pt{\textcolor{darkblue}{\textbf{\ipa{ko˩˥}}}}\kern2pt]} \hypertarget{ko\string_Ba1}{}
\markboth{\textcolor{darkblue}{\textbf{\ipa{ko˩\textsubscript{a}}}}}{}
\textcolor{teal}{\mytextsc{verb}} \hspace{4pt} Tone: L\textsubscript{a}.
\textcolor{Sepia}{\selectlanguage{english}To warm oneself at a fire; to bask in the sun.} \zh{烤火取暖,晒太阳。}  ¶ \textcolor{darkblue}{\textbf{\ipa{mv̩˧ ko˥}}} \textcolor{Sepia}{\selectlanguage{english}to warm oneself at a fire} \zh{烤火取暖}  
 ¶ \textcolor{darkblue}{\textbf{\ipa{le˧-ko˩-ze˩}}} \textcolor{Sepia}{\selectlanguage{english}\mytextsc{accomp} \string_ \mytextsc{pfv}} \zh{烤火了}  
 ¶ \textcolor{darkblue}{\textbf{\ipa{ɲi˧mi˧ ko˩}}} \textcolor{Sepia}{\selectlanguage{english}to bask in the sun} \zh{晒太阳}  
 ¶ \textcolor{darkblue}{\textbf{\ipa{ɲi˧mi˧ ɖɯ˧-ko˩-ɻ̍˩}}} \textcolor{Sepia}{\selectlanguage{english}to bask in the sun for a while} \zh{晒晒太阳}  
 ¶ \textcolor{darkblue}{\textbf{\ipa{ɲi˧mi˧ ɖɯ˧-ko˧\textasciitilde{}ko˥-ɻ̍˩}}} \textcolor{Sepia}{\selectlanguage{english}to bask in the sun for a while} \zh{晒晒太阳}  

\lhead{\firstmark}
\rhead{\botmark}

\subsection{\hspace{-0.5cm} {\Large \textcolor{darkblue}{\textbf{\ipa{ko˩dze˧}}}}\hspace{0.5cm}[\kern2pt{\textcolor{darkblue}{\textbf{\ipa{ko˩dze˥}}}}\kern2pt]} \hypertarget{ko\string_Bdze\string_M1}{}
\markboth{\textcolor{darkblue}{\textbf{\ipa{ko˩dze˧}}}}{}
\textcolor{teal}{\mytextsc{noun}} \hspace{4pt} Tone: LM.
\textcolor{Sepia}{\selectlanguage{english}A sort of dove.} \zh{一种鸽子。}  \zh{量词}: \textcolor{darkblue}{\textbf{\ipa{mi˩}}}  \mytextsc{clf}: \textcolor{darkblue}{\textbf{\ipa{mi˩}}} 
\lhead{\firstmark}
\rhead{\botmark}

\subsection{\hspace{-0.5cm} {\Large \textcolor{darkblue}{\textbf{\ipa{ko˩ɖʐo˩}}}}\hspace{0.5cm}[\kern2pt{\textcolor{darkblue}{\textbf{\ipa{ko˩ɖʐo˩˥}}}}\kern2pt]} \hypertarget{ko\string_Bd`z`o\string_B1}{}
\markboth{\textcolor{darkblue}{\textbf{\ipa{ko˩ɖʐo˩}}}}{}
\textcolor{teal}{\mytextsc{noun}} \hspace{4pt} Tone: L.
\textcolor{Sepia}{\selectlanguage{english}Flail.} \zh{连枷。}  \zh{量词}: \textcolor{darkblue}{\textbf{\ipa{nɑ˧}}}  \mytextsc{clf}: \textcolor{darkblue}{\textbf{\ipa{nɑ˧}}} 
\lhead{\firstmark}
\rhead{\botmark}

\subsection{\hspace{-0.5cm} {\Large \textcolor{darkblue}{\textbf{\ipa{ko˩qʰɑ˧-dʑɯ\#˥}}}}\hspace{0.5cm}[\kern2pt{\textcolor{darkblue}{\textbf{\ipa{xxxx non-correspondance entre le nombre de morphèmes et le nombre de tons de morphèmes}}}}\kern2pt]} \hypertarget{ko\string_Bq\string_hA\string_M-dz£M\#\string_T1}{}
\markboth{\textcolor{darkblue}{\textbf{\ipa{ko˩qʰɑ˧-dʑɯ\#˥}}}}{}
\textcolor{teal}{\mytextsc{noun}} \hspace{4pt} Tone: LM+\#H.
\textcolor{Sepia}{\selectlanguage{english}Yyyy.} \zh{金梅花。}  ¶ \textcolor{darkblue}{\textbf{\ipa{ko˩qʰɑ˧-dʑɯ˧-bæ˥bæ˩}}} \textcolor{Sepia}{\selectlanguage{english}flower of...} \zh{金梅花的花}  

\lhead{\firstmark}
\rhead{\botmark}

\subsection{\hspace{-0.5cm} {\Large \textcolor{darkblue}{\textbf{\ipa{ko˧˥}}} \textsubscript{1}}\hspace{0.5cm}[\kern2pt{\textcolor{darkblue}{\textbf{\ipa{ko˧˥}}}}\kern2pt]} \hypertarget{ko\string_M\string_T1}{}
\markboth{\textcolor{darkblue}{\textbf{\ipa{ko˧˥}}} \textsubscript{1}}{}
\textcolor{teal}{\mytextsc{adverb(ial)}} \hspace{4pt} Tone: MH.
\textcolor{Sepia}{\selectlanguage{english}Too much, excessively.} \zh{过于,太(汉语借词)。}  Borrowing: Chinese  \zh{过}

\lhead{\firstmark}
\rhead{\botmark}

\subsection{\hspace{-0.5cm} {\Large \textcolor{darkblue}{\textbf{\ipa{ko˧˥}}} \textsubscript{2}}\hspace{0.5cm}[\kern2pt{\textcolor{darkblue}{\textbf{\ipa{ko˧˥}}}}\kern2pt]} \hypertarget{ko\string_M\string_T2}{}
\markboth{\textcolor{darkblue}{\textbf{\ipa{ko˧˥}}} \textsubscript{2}}{}
\textcolor{teal}{\mytextsc{verb}} \hspace{4pt} Tone: MH.
\textcolor{Sepia}{\selectlanguage{english}To happen, to take place, to pass, to go by (days, existence).} \zh{过(汉语借词)。}  Borrowing: Chinese  \zh{过}
 ¶ \textcolor{darkblue}{\textbf{\ipa{se˧ʐɯ˩ ko˩}}} \textcolor{Sepia}{\selectlanguage{english}to celebrate a birthday} \zh{过生日}  

\lhead{\firstmark}
\rhead{\botmark}

\subsection{\hspace{-0.5cm} {\Large \textcolor{darkblue}{\textbf{\ipa{kɯ˥}}}}\hspace{0.5cm}[\kern2pt{\textcolor{darkblue}{\textbf{\ipa{kɯ˥}}}}\kern2pt]} \hypertarget{kM\string_T1}{}
\markboth{\textcolor{darkblue}{\textbf{\ipa{kɯ˥}}}}{}
\textcolor{teal}{\mytextsc{noun}} \hspace{4pt} Tone: \#H.
\ding{202} \textcolor{Sepia}{\selectlanguage{english}Gallbladder.} \zh{胆。}  \zh{量词}: \textcolor{darkblue}{\textbf{\ipa{ɭɯ˧}}} \ding{203} \textcolor{Sepia}{\selectlanguage{english}Gall.} \zh{胆汁。}  \mytextsc{clf}: \textcolor{darkblue}{\textbf{\ipa{ɭɯ˧}}} 
\lhead{\firstmark}
\rhead{\botmark}

\subsection{\hspace{-0.5cm} {\Large \textcolor{darkblue}{\textbf{\ipa{kɯ˥}}}}\hspace{0.5cm}[\kern2pt{\textcolor{darkblue}{\textbf{\ipa{kɯ˥}}}}\kern2pt]} \hypertarget{kM\string_T1}{}
\markboth{\textcolor{darkblue}{\textbf{\ipa{kɯ˥}}}}{}
\textcolor{teal}{\mytextsc{adjective}} \hspace{4pt} Tone: H.
\textcolor{Sepia}{\selectlanguage{english}Tight, tense.} \zh{紧。}  ¶ \textcolor{darkblue}{\textbf{\ipa{le˧-tsɯ˥ | le˧-kɯ˥-kʰɯ˩}}} \textcolor{Sepia}{\selectlanguage{english}to attach tightly, to attach so that it will be quite tight} \zh{绑紧}  
 ¶ \textcolor{darkblue}{\textbf{\ipa{le˧-kɯ˥-se˩}}} \textcolor{Sepia}{\selectlanguage{english}\mytextsc{accomp} \string_ \mytextsc{pfv}} \zh{紧了}  

\lhead{\firstmark}
\rhead{\botmark}

\subsection{\hspace{-0.5cm} {\Large \textcolor{darkblue}{\textbf{\ipa{kɯ˧}}}}\hspace{0.5cm}[\kern2pt{\textcolor{darkblue}{\textbf{\ipa{kɯ˥}}}}\kern2pt]} \hypertarget{kM\string_M1}{}
\markboth{\textcolor{darkblue}{\textbf{\ipa{kɯ˧}}}}{}
\textcolor{teal}{\mytextsc{noun}} \hspace{4pt} Tone: M.
\textcolor{Sepia}{\selectlanguage{english}Star.} \zh{星星。}  ¶ \textcolor{darkblue}{\textbf{\ipa{mv̩˧ʁo˥ | kɯ˧}}} \textcolor{Sepia}{\selectlanguage{english}there are stars in the sky, one can see stars} \zh{天上有星星、天上看得见星星}  
 ¶ \textcolor{darkblue}{\textbf{\ipa{nɑ˩-ʈʂʰɯ˥, | kɯ˧ mɤ˧-li˧! | di˧mi˧-lɑ˧ li˥!}}} \textcolor{Sepia}{\selectlanguage{english}The Na do not look at the stars! They only look at the plain (=at the plain of Yongning)! (A comment by the consultant about her lack of knowledge of the names of stars and constellations.)} \zh{摩梭呢,不看星星,只看平坝(=永宁坝子)!(合作人说明为什么她不知道星星、星座的名字:摩梭人本来对天文不太感兴趣。)}  
 \zh{量词}: \textcolor{darkblue}{\textbf{\ipa{ɭɯ˧, kɯ˧}}}  \mytextsc{clf}: \textcolor{darkblue}{\textbf{\ipa{ɭɯ˧, kɯ˧}}} 
\lhead{\firstmark}
\rhead{\botmark}

\subsection{\hspace{-0.5cm} {\Large \textcolor{darkblue}{\textbf{\ipa{kɯ˧\textsubscript{b}}}}}\hspace{0.5cm}[\kern2pt{\textcolor{darkblue}{\textbf{\ipa{kɯ˩˥}}}}\kern2pt]} \hypertarget{kM\string_Mb1}{}
\markboth{\textcolor{darkblue}{\textbf{\ipa{kɯ˧\textsubscript{b}}}}}{}
\textcolor{teal}{\mytextsc{classifier}} \hspace{4pt} Tone: M\textsubscript{b}.
\textcolor{Sepia}{\selectlanguage{english}Self-classifier for stars.} \zh{量词:星星(一个)。} 
\lhead{\firstmark}
\rhead{\botmark}

\subsection{\hspace{-0.5cm} {\Large \textcolor{darkblue}{\textbf{\ipa{kɯ˧ɭɯ˧}}}}\hspace{0.5cm}[\kern2pt{\textcolor{darkblue}{\textbf{\ipa{kɯ˧ɭɯ˧˥}}}}\kern2pt]} \hypertarget{kM\string_Ml\string_RM\string_M1}{}
\markboth{\textcolor{darkblue}{\textbf{\ipa{kɯ˧ɭɯ˧}}}}{}
\textcolor{teal}{\mytextsc{noun}} \hspace{4pt} Tone: M.
\textcolor{Sepia}{\selectlanguage{english}Spirit.} \zh{神。}  ¶ \textcolor{darkblue}{\textbf{\ipa{kɯ˧ɭɯ˧ | ɖɯ˧-dze˩}}} \textcolor{Sepia}{\selectlanguage{english}a pair of (good) spirits, two (good) spirits} \zh{两个(好)神}  
 \zh{量词}: \textcolor{darkblue}{\textbf{\ipa{dze˩}}}  \mytextsc{clf}: \textcolor{darkblue}{\textbf{\ipa{dze˩}}} 
\lhead{\firstmark}
\rhead{\botmark}

\subsection{\hspace{-0.5cm} {\Large \textcolor{darkblue}{\textbf{\ipa{kɯ˧qʰæ˧ʂe˧˥}}}}\hspace{0.5cm}[\kern2pt{\textcolor{darkblue}{\textbf{\ipa{kɯ˧qʰæ˧ʂe˧˥}}}}\kern2pt]} \hypertarget{kM\string_Mq\string_h\{\string_Ms`e\string_M\string_T1}{}
\markboth{\textcolor{darkblue}{\textbf{\ipa{kɯ˧qʰæ˧ʂe˧˥}}}}{}
\textcolor{teal}{\mytextsc{noun}} \hspace{4pt} Tone: MH\#.
\textcolor{Sepia}{\selectlanguage{english}Comet.} \zh{流星。}  \zh{量词}: \textcolor{darkblue}{\textbf{\ipa{ʂɯ˩}}}  \mytextsc{clf}: \textcolor{darkblue}{\textbf{\ipa{ʂɯ˩}}} 
\lhead{\firstmark}
\rhead{\botmark}

\subsection{\hspace{-0.5cm} {\Large \textcolor{darkblue}{\textbf{\ipa{kɯ˩\textsubscript{a}}}}}\hspace{0.5cm}[\kern2pt{\textcolor{darkblue}{\textbf{\ipa{kɯ˧˥}}}}\kern2pt]} \hypertarget{kM\string_Ba1}{}
\markboth{\textcolor{darkblue}{\textbf{\ipa{kɯ˩\textsubscript{a}}}}}{}
\textcolor{teal}{\mytextsc{verb}} \hspace{4pt} Tone: L\textsubscript{a}.
\textcolor{Sepia}{\selectlanguage{english}To ignore someone who would need help, to leave someone alone with difficulties one could help with. This is a term for which no straightforward Chinese equivalent has been found; it refers to a situation where lack of real attachment to someone shows up in the lack of impulse to go out of one's way and help them.} \zh{不理需要帮忙的人:知道一个人需要帮助,自己也有能力帮忙,但假装没看见、什么事没有。}  ¶ \textcolor{darkblue}{\textbf{\ipa{hĩ˧ kɯ˥}}} \textcolor{Sepia}{\selectlanguage{english}same meaning} \zh{同上}  
 ¶ \textcolor{darkblue}{\textbf{\ipa{hĩ˧-ɳɯ˩ | kɯ˩-kv̩˥!}}} \textcolor{Sepia}{\selectlanguage{english}People will sometimes ignore you when you are in need! / (You will realize that, in cases where you need help) people will sometimes ignore you and leave you alone with your difficulties!} \zh{人家在你需要帮忙的时候就会不理你的!(如果处不好关系,人家对你没有什么好感,到时候你需要帮忙他们就不理你了。)}  
 ¶ \textcolor{darkblue}{\textbf{\ipa{kɯ˩-mɤ˩-kv̩˥!}}} \textcolor{Sepia}{\selectlanguage{english}(They) are not going to help you! / You're not going to get any help (from them)!} \zh{人家在你需要帮忙的时候就会不理你的!}  
 ¶ \textcolor{darkblue}{\textbf{\ipa{hĩ˧-ɳɯ˩ | kɯ˩-tʰɑ˩-kʰɯ˥!}}} \textcolor{Sepia}{\selectlanguage{english}Don't (behave in such a way as to) let people ignore you when you are in need! (Explanation: one should build trust for oneself, making others feel real trust and gratitude, so that they will help as a matter of course when the need for it arises; otherwise they will ignore us when we are in need of help.)} \zh{别让人家(在你需要帮忙的时候)不理你!}  
 ¶ \textcolor{darkblue}{\textbf{\ipa{njɤ˧ | no˩ kɯ˩-hĩ˥ mɤ˩-ɲi˩! | njɤ˧ | no˧-ki˧ | dʑɤ˩-so˥-ɲi˩!}}} \textcolor{Sepia}{\selectlanguage{english}I am not neglecting you at all! (On the contrary) I am teaching you good things / I am doing my best to teach you! (Context: a student considers himself neglected by a teacher; the teacher realizes that the student is dissatisfied, and provides a clarification.)} \zh{我不是不重视你!(刚好相反:)我是用心教你的 / 我努力教你最好的!(情景:一名学生认为老师忽视他,老师发现学生不高兴,就说明。)}  

\lhead{\firstmark}
\rhead{\botmark}

\subsection{\hspace{-0.5cm} {\Large \textcolor{darkblue}{\textbf{\ipa{kɯ˩ɻ̍˧}}}}\hspace{0.5cm}[\kern2pt{\textcolor{darkblue}{\textbf{\ipa{kɯ˩ɻ̍˥}}}}\kern2pt]} \hypertarget{kM\string_Br£`̍\string_M1}{}
\markboth{\textcolor{darkblue}{\textbf{\ipa{kɯ˩ɻ̍˧}}}}{}
\textcolor{teal}{\mytextsc{noun}} \hspace{4pt} Tone: LM.
\textcolor{Sepia}{\selectlanguage{english}Two-string violin, erhu.} \zh{胡琴,二胡。}  ¶ \textcolor{darkblue}{\textbf{\ipa{kɯ˩ɻ̍˧ ʈɤ˧}}} \textcolor{Sepia}{\selectlanguage{english}to play erhu} \zh{拉二胡}  
 \zh{量词}: \textcolor{darkblue}{\textbf{\ipa{nɑ˧}}}  \mytextsc{clf}: \textcolor{darkblue}{\textbf{\ipa{nɑ˧}}} 
\lhead{\firstmark}
\rhead{\botmark}

\subsection{\hspace{-0.5cm} {\Large \textcolor{darkblue}{\textbf{\ipa{kv̩˧˥}}}}\hspace{0.5cm}[\kern2pt{\textcolor{darkblue}{\textbf{\ipa{kv̩˧˥}}}}\kern2pt]} \hypertarget{kv\string_=\string_M\string_T1}{}
\markboth{\textcolor{darkblue}{\textbf{\ipa{kv̩˧˥}}}}{}
\textcolor{teal}{\mytextsc{verb}} \hspace{4pt} Tone: MH.
\textcolor{Sepia}{\selectlanguage{english}To be able to.} \zh{会、有能力做。} 
\lhead{\firstmark}
\rhead{\botmark}

\subsection{\hspace{-0.5cm} {\Large \textcolor{darkblue}{\textbf{\ipa{kv̩˥}}}}\hspace{0.5cm}[\kern2pt{\textcolor{darkblue}{\textbf{\ipa{kv̩˥}}}}\kern2pt]} \hypertarget{kv\string_=\string_T1}{}
\markboth{\textcolor{darkblue}{\textbf{\ipa{kv̩˥}}}}{}
\textcolor{teal}{\mytextsc{noun}} \hspace{4pt} Tone: \#H.
\textcolor{Sepia}{\selectlanguage{english}Garlic, \textit{Allium sativum}.} \zh{大蒜。}  \zh{量词}: \textcolor{darkblue}{\textbf{\ipa{ɭɯ˧}}} \textcolor{darkblue}{\textbf{\ipa{tsʰɤ˧˥}}}  \mytextsc{clf}: \textcolor{darkblue}{\textbf{\ipa{ɭɯ˧}}} \textcolor{darkblue}{\textbf{\ipa{tsʰɤ˧˥}}} 
\lhead{\firstmark}
\rhead{\botmark}

\subsection{\hspace{-0.5cm} {\Large \textcolor{darkblue}{\textbf{\ipa{kv̩˩\textsubscript{a}}}} \textsubscript{1}}\hspace{0.5cm}[\kern2pt{\textcolor{darkblue}{\textbf{\ipa{kv̩˥}}}}\kern2pt]} \hypertarget{kv\string_=\string_Ba1}{}
\markboth{\textcolor{darkblue}{\textbf{\ipa{kv̩˩\textsubscript{a}}}} \textsubscript{1}}{}
\textcolor{teal}{\mytextsc{verb}} \hspace{4pt} Tone: L\textsubscript{a}.
\ding{202} \textcolor{Sepia}{\selectlanguage{english}To pick up (from the ground), to gather.} \zh{捡起来,拾。}  ¶ \textcolor{darkblue}{\textbf{\ipa{kv̩˧\textasciitilde{}kv̩˥}}} \textcolor{Sepia}{\selectlanguage{english}\mytextsc{red}} \zh{\mytextsc{重叠}}  
 ¶ \textcolor{darkblue}{\textbf{\ipa{gɤ˩-kv̩˧\textasciitilde{}kv̩˥}}} \textcolor{Sepia}{\selectlanguage{english}to pick up (something that was on the ground)} \zh{捡起来(地上的东西)}  
 ¶ \textcolor{darkblue}{\textbf{\ipa{le˧-kv̩˧\textasciitilde{}kv̩˥}}} \textcolor{Sepia}{\selectlanguage{english}to pick up (something that was on the ground)} \zh{捡起来(地上的东西)}  
\ding{203} \textcolor{Sepia}{\selectlanguage{english}To fish.} \zh{钓鱼。}  ¶ \textcolor{darkblue}{\textbf{\ipa{ɲi˧zo˧ kv̩˥}}} \textcolor{Sepia}{\selectlanguage{english}to fish} \zh{钓鱼}  

\lhead{\firstmark}
\rhead{\botmark}

\subsection{\hspace{-0.5cm} {\Large \textcolor{darkblue}{\textbf{\ipa{kv̩˩\textsubscript{a}}}} \textsubscript{2}}\hspace{0.5cm}[\kern2pt{\textcolor{darkblue}{\textbf{\ipa{kv̩˩˥}}}}\kern2pt]} \hypertarget{kv\string_=\string_Ba2}{}
\markboth{\textcolor{darkblue}{\textbf{\ipa{kv̩˩\textsubscript{a}}}} \textsubscript{2}}{}
\textcolor{teal}{\mytextsc{verb}} \hspace{4pt} Tone: L\textsubscript{a}.
\textcolor{Sepia}{\selectlanguage{english}To cross.} \zh{过。}  ¶ \textcolor{darkblue}{\textbf{\ipa{ʈʂʰwæ˩ kv̩˥}}} \textcolor{Sepia}{\selectlanguage{english}to cross (a river) in a boat} \zh{坐船过(河)}  

\lhead{\firstmark}
\rhead{\botmark}

\subsection{\hspace{-0.5cm} {\Large \textcolor{darkblue}{\textbf{\ipa{kv̩˧dʑɯ˧˥}}}}\hspace{0.5cm}[\kern2pt{\textcolor{darkblue}{\textbf{\ipa{kv̩˧dʑɯ˧˥}}}}\kern2pt]} \hypertarget{kv\string_=\string_Mdz£M\string_M\string_T1}{}
\markboth{\textcolor{darkblue}{\textbf{\ipa{kv̩˧dʑɯ˧˥}}}}{}
\textcolor{teal}{\mytextsc{noun}} \hspace{4pt} Tone: MH.
\textcolor{Sepia}{\selectlanguage{english}Tent.} \zh{帐篷。}  ¶ \textcolor{darkblue}{\textbf{\ipa{kv̩˧dʑɯ˧ lɑ˥}}} \textcolor{Sepia}{\selectlanguage{english}to put up a tent, to set up a tent} \zh{安装帐篷、搭建帐篷}  
 \zh{量词}: \textcolor{darkblue}{\textbf{\ipa{nɑ˧}}}  \mytextsc{clf}: \textcolor{darkblue}{\textbf{\ipa{nɑ˧}}} 
\lhead{\firstmark}
\rhead{\botmark}

\subsection{\hspace{-0.5cm} {\Large \textcolor{darkblue}{\textbf{\ipa{kv̩˧ʝi˥}}}}\hspace{0.5cm}[\kern2pt{\textcolor{darkblue}{\textbf{\ipa{kv̩˧ʝi˧˥}}}}\kern2pt]} \hypertarget{kv\string_=\string_Mj££i\string_T1}{}
\markboth{\textcolor{darkblue}{\textbf{\ipa{kv̩˧ʝi˥}}}}{}
\textcolor{teal}{\mytextsc{adverb(ial)}} \hspace{4pt} Tone: H\#.
\textcolor{Sepia}{\selectlanguage{english}Truly, really, for good.} \zh{真的、的确、确实。} 
\lhead{\firstmark}
\rhead{\botmark}

\subsection{\hspace{-0.5cm} {\Large \textcolor{darkblue}{\textbf{\ipa{kv̩˧ʝi˥\$}}}}\hspace{0.5cm}[\kern2pt{\textcolor{darkblue}{\textbf{\ipa{kv̩˧ʝi˥}}}}\kern2pt]} \hypertarget{kv\string_=\string_Mj££i\string_T\$1}{}
\markboth{\textcolor{darkblue}{\textbf{\ipa{kv̩˧ʝi˥\$}}}}{}
\textcolor{teal}{\mytextsc{noun}} \hspace{4pt} Tone: H\$.
\textcolor{Sepia}{\selectlanguage{english}Life, existence, lifetime.} \zh{生命。} 
\lhead{\firstmark}
\rhead{\botmark}

\subsection{\hspace{-0.5cm} {\Large \textcolor{darkblue}{\textbf{\ipa{kv̩˩kv̩˩}}}}\hspace{0.5cm}[\kern2pt{\textcolor{darkblue}{\textbf{\ipa{kv̩˩kv̩˩˥}}}}\kern2pt]} \hypertarget{kv\string_=\string_Bkv\string_=\string_B1}{}
\markboth{\textcolor{darkblue}{\textbf{\ipa{kv̩˩kv̩˩}}}}{}
\textcolor{teal}{\mytextsc{noun}} \hspace{4pt} Tone: L.
\textcolor{Sepia}{\selectlanguage{english}Cheekbone.} \zh{颧骨。}  \zh{量词}: \textcolor{darkblue}{\textbf{\ipa{ɭɯ˧}}}  \mytextsc{clf}: \textcolor{darkblue}{\textbf{\ipa{ɭɯ˧}}} \textit{See:} \hyperlink{}{\textcolor{darkblue}{\textbf{\ipa{njɤ˧kv̩˩}}}} 
\lhead{\firstmark}
\rhead{\botmark}

\subsection{\hspace{-0.5cm} {\Large \textcolor{darkblue}{\textbf{\ipa{kv̩˧lv̩˧lv̩˥}}}}\hspace{0.5cm}[\kern2pt{\textcolor{darkblue}{\textbf{\ipa{kv̩˧lv̩˧lv̩˧}}}}\kern2pt]} \hypertarget{kv\string_=\string_Mlv\string_=\string_Mlv\string_=\string_T1}{}
\markboth{\textcolor{darkblue}{\textbf{\ipa{kv̩˧lv̩˧lv̩˥}}}}{}
\textcolor{teal}{\mytextsc{noun}} \hspace{4pt} Tone: H\#.
\textcolor{Sepia}{\selectlanguage{english}Garlic braid: garlic bulbs with long leaves, braided into a large clump.} \zh{蒜瓣。}  \zh{量词}: \textcolor{darkblue}{\textbf{\ipa{ɭɯ˧}}}  \mytextsc{clf}: \textcolor{darkblue}{\textbf{\ipa{ɭɯ˧}}} 
\lhead{\firstmark}
\rhead{\botmark}

\subsection{\hspace{-0.5cm} {\Large \textcolor{darkblue}{\textbf{\ipa{kv̩˩nɑ˧˥}}}}\hspace{0.5cm}[\kern2pt{\textcolor{darkblue}{\textbf{\ipa{kv̩˧nɑ˥}}}}\kern2pt]} \hypertarget{kv\string_=\string_BnA\string_M\string_T1}{}
\markboth{\textcolor{darkblue}{\textbf{\ipa{kv̩˩nɑ˧˥}}}}{}
\textcolor{teal}{\mytextsc{noun}} \hspace{4pt} Tone: LM+MH\#.
\textcolor{Sepia}{\selectlanguage{english}Silk.} \zh{丝绸。}  ¶ \textcolor{darkblue}{\textbf{\ipa{kv̩˩nɑ˧-bɑ˧lɑ˥}}} \textcolor{Sepia}{\selectlanguage{english}silk garment} \zh{丝绸衣服}  
 \zh{量词}: \textcolor{darkblue}{\textbf{\ipa{tsʰi˥}}}  \mytextsc{clf}: \textcolor{darkblue}{\textbf{\ipa{tsʰi˥}}} 
\lhead{\firstmark}
\rhead{\botmark}

\subsection{\hspace{-0.5cm} {\Large \textcolor{darkblue}{\textbf{\ipa{kv̩˧ɲi˥}}}}\hspace{0.5cm}[\kern2pt{\textcolor{darkblue}{\textbf{\ipa{kv̩˧ɲi˥}}}}\kern2pt]} \hypertarget{kv\string_=\string_MJi\string_T1}{}
\markboth{\textcolor{darkblue}{\textbf{\ipa{kv̩˧ɲi˥}}}}{}
\textcolor{teal}{\mytextsc{adjective}} \hspace{4pt} Tone: H\#.
\textcolor{Sepia}{\selectlanguage{english}Empty.} \zh{空手,空。}  ¶ \textcolor{darkblue}{\textbf{\ipa{bi˩ʁo˧ | kv̩˧ɲi˥-kʰɯ˩}}} \textcolor{Sepia}{\selectlanguage{english}to empty (someone's) purse, i.e. to take someone's money} \zh{(把一个人的)钱包弄空}  
 ¶ \textcolor{darkblue}{\textbf{\ipa{tɕʰɯ˩ di˩-hɯ˩˥, | mɤ˧-ɖɯ˧, | kv̩˧ɲi˥ | le˧-tsʰɯ˩!}}} \textcolor{Sepia}{\selectlanguage{english}He went to hunt the muntjac, but did not kill any, and came back empty-handed!} \zh{他去狩猎,没得(任何猎物),空手回来!}  

\lhead{\firstmark}
\rhead{\botmark}

\subsection{\hspace{-0.5cm} {\Large \textcolor{darkblue}{\textbf{\ipa{kv̩˧ʁo˧bv̩˥}}}}\hspace{0.5cm}[\kern2pt{\textcolor{darkblue}{\textbf{\ipa{kv̩˧ʁo˧bv̩˧˥}}}}\kern2pt]} \hypertarget{kv\string_=\string_MRo\string_Mbv\string_=\string_T1}{}
\markboth{\textcolor{darkblue}{\textbf{\ipa{kv̩˧ʁo˧bv̩˥}}}}{}
\textcolor{teal}{\mytextsc{noun}} \hspace{4pt} Tone: H\#.
\textcolor{Sepia}{\selectlanguage{english}Garlic sprouts (consumed as a vegetable).} \zh{蒜苗。}  \zh{量词}: \textcolor{darkblue}{\textbf{\ipa{kɤ˧˥}}}  \mytextsc{clf}: \textcolor{darkblue}{\textbf{\ipa{kɤ˧˥}}} 
\lhead{\firstmark}
\rhead{\botmark}

\subsection{\hspace{-0.5cm} {\Large \textcolor{darkblue}{\textbf{\ipa{kv̩˧ʂe˥\$}}}}\hspace{0.5cm}[\kern2pt{\textcolor{darkblue}{\textbf{\ipa{kv̩˧ʂe˧˥}}}}\kern2pt]} \hypertarget{kv\string_=\string_Ms`e\string_T\$1}{}
\markboth{\textcolor{darkblue}{\textbf{\ipa{kv̩˧ʂe˥\$}}}}{}
\textcolor{teal}{\mytextsc{noun}} \hspace{4pt} Tone: H\$.
\textcolor{Sepia}{\selectlanguage{english}Flea.} \zh{跳蚤。}  \zh{量词}: \textcolor{darkblue}{\textbf{\ipa{mi˩}}}  \mytextsc{clf}: \textcolor{darkblue}{\textbf{\ipa{mi˩}}} 
\lhead{\firstmark}
\rhead{\botmark}

\subsection{\hspace{-0.5cm} {\Large \textcolor{darkblue}{\textbf{\ipa{kv̩˩tɑ˩}}}}\hspace{0.5cm}[\kern2pt{\textcolor{darkblue}{\textbf{\ipa{kv̩˩tɑ˩˥}}}}\kern2pt]} \hypertarget{kv\string_=\string_BtA\string_B1}{}
\markboth{\textcolor{darkblue}{\textbf{\ipa{kv̩˩tɑ˩}}}}{}
\textcolor{teal}{\mytextsc{verb}} \hspace{4pt} Tone: L.
\textcolor{Sepia}{\selectlanguage{english}To assemble, to group, to bring together (e.g. after felling trees, putting pieces of timber together).} \zh{集中在一起(如:砍木材后,把木材堆在一起)。} 
\lhead{\firstmark}
\rhead{\botmark}

\subsection{\hspace{-0.5cm} {\Large \textcolor{darkblue}{\textbf{\ipa{kv̩˧tsʰɑ˥\$}}}}\hspace{0.5cm}[\kern2pt{\textcolor{darkblue}{\textbf{\ipa{kv̩˧tsʰɑ˥}}}}\kern2pt]} \hypertarget{kv\string_=\string_Mts\string_hA\string_T\$1}{}
\markboth{\textcolor{darkblue}{\textbf{\ipa{kv̩˧tsʰɑ˥\$}}}}{}
\textcolor{teal}{\mytextsc{noun}} \hspace{4pt} Tone: H\$.
\textcolor{Sepia}{\selectlanguage{english}Family name (that of the Muli feudal lord, belonging to the Pumi/Prinmi ethnic group).} \zh{一个姓(木里土司,普米族,的姓)。}  ¶ \textcolor{darkblue}{\textbf{\ipa{kv̩˧tsʰɑ˧=ɻ̍˥\$}}} \textcolor{Sepia}{\selectlanguage{english}the \textcolor{darkblue}{\textbf{\ipa{/kv̩˧tsʰɑ˥\$/}}} clan, the \textcolor{darkblue}{\textbf{\ipa{/kv̩˧tsʰɑ˥\$/}}} family} \zh{\textcolor{darkblue}{\textbf{\ipa{/kv̩˧tsʰɑ˥\$/}}}家族}  
 ¶ \textcolor{darkblue}{\textbf{\ipa{kv̩˧tsʰɑ˧=ɻ̍˧ pi˥-zo˩!}}} \textcolor{Sepia}{\selectlanguage{english}They were called “the \textcolor{darkblue}{\textbf{\ipa{/kv̩˧tsʰɑ˥\$/}}} family”!} \zh{人家把他们称作“\textcolor{darkblue}{\textbf{\ipa{/kv̩˧tsʰɑ˥\$/}}}家族”!}  

\lhead{\firstmark}
\rhead{\botmark}

\subsection{\hspace{-0.5cm} {\Large \textcolor{darkblue}{\textbf{\ipa{kv̩˧tsʰɤ˩}}}}\hspace{0.5cm}[\kern2pt{\textcolor{darkblue}{\textbf{\ipa{kv̩˧tsʰɤ˩}}}}\kern2pt]} \hypertarget{kv\string_=\string_Mts\string_h7\string_B1}{}
\markboth{\textcolor{darkblue}{\textbf{\ipa{kv̩˧tsʰɤ˩}}}}{}
\textcolor{teal}{\mytextsc{noun}} \hspace{4pt} Tone: L\#.
\textcolor{Sepia}{\selectlanguage{english}Head of garlic.} \zh{蒜头。}  \zh{量词}: \textcolor{darkblue}{\textbf{\ipa{tsʰɤ˧˥}}}  \mytextsc{clf}: \textcolor{darkblue}{\textbf{\ipa{tsʰɤ˧˥}}} 
\lhead{\firstmark}
\rhead{\botmark}

\subsection{\hspace{-0.5cm} {\Large \textcolor{darkblue}{\textbf{\ipa{kv̩˧ʈʂɯ˧˥}}}}\hspace{0.5cm}[\kern2pt{\textcolor{darkblue}{\textbf{\ipa{kv̩˧ʈʂɯ˧˥}}}}\kern2pt]} \hypertarget{kv\string_=\string_Mt`s`M\string_M\string_T1}{}
\markboth{\textcolor{darkblue}{\textbf{\ipa{kv̩˧ʈʂɯ˧˥}}}}{}
\textcolor{teal}{\mytextsc{noun}} \hspace{4pt} Tone: MH\#.
\textcolor{Sepia}{\selectlanguage{english}(finger)nail, (toe)nail.} \zh{指甲。}  \zh{量词}: \textcolor{darkblue}{\textbf{\ipa{ɭɯ˧}}}  \mytextsc{clf}: \textcolor{darkblue}{\textbf{\ipa{ɭɯ˧}}} 
\lhead{\firstmark}
\rhead{\botmark}

\subsection{\hspace{-0.5cm} {\Large \textcolor{darkblue}{\textbf{\ipa{‑kv̩˧˥}}}}\hspace{0.5cm}[\kern2pt{\textcolor{darkblue}{\textbf{\ipa{kv̩˧˥}}}}\kern2pt]} \hypertarget{‑kv\string_=\string_M\string_T1}{}
\markboth{\textcolor{darkblue}{\textbf{\ipa{‑kv̩˧˥}}}}{}
\textcolor{teal}{\mytextsc{suffix}} \hspace{4pt} Tone: MH.
\textcolor{Sepia}{\selectlanguage{english}\mytextsc{abilitive;} also indicates future.} \zh{能。} 
\lhead{\firstmark}
\rhead{\botmark}

\subsection{\hspace{-0.5cm} {\Large \textcolor{darkblue}{\textbf{\ipa{kwɑ˧fæ˩}}}}\hspace{0.5cm}[\kern2pt{\textcolor{darkblue}{\textbf{\ipa{kwɑ˧fæ˩}}}}\kern2pt]} \hypertarget{kwA\string_Mf\{\string_B1}{}
\markboth{\textcolor{darkblue}{\textbf{\ipa{kwɑ˧fæ˩}}}}{}
\textcolor{teal}{\mytextsc{noun}} \hspace{4pt} Tone: L\#.
\textcolor{Sepia}{\selectlanguage{english}Name of a hotel.} \zh{官房(汉语借词),酒店名称。}  Borrowing: Chinese  \zh{官房}
 ¶ \textcolor{darkblue}{\textbf{\ipa{kwɑ˧fæ˩}}} \textcolor{Sepia}{\selectlanguage{english}the abridged name of a five-star hotel where one of the main consultant's daughters works} \zh{丽江官房大酒店的简称。注:发音合作人的女儿在丽江官房大酒店工作。}  

\lhead{\firstmark}
\rhead{\botmark}

\subsection{\hspace{-0.5cm} {\Large \textcolor{darkblue}{\textbf{\ipa{kwɑ˧tsʰɑ˧}}}}\hspace{0.5cm}[\kern2pt{\textcolor{darkblue}{\textbf{\ipa{kwɑ˧tsʰɑ˧}}}}\kern2pt]} \hypertarget{kwA\string_Mts\string_hA\string_M1}{}
\markboth{\textcolor{darkblue}{\textbf{\ipa{kwɑ˧tsʰɑ˧}}}}{}
\textcolor{teal}{\mytextsc{noun}} \hspace{4pt} Tone: M.
\textcolor{Sepia}{\selectlanguage{english}Coffin.} \zh{棺材(汉语借词)。}  Borrowing: Chinese  \zh{棺材}
 ¶ \textcolor{darkblue}{\textbf{\ipa{kwɑ˧tsʰɑ˧, | hĩ˧-mo˩-kʰɯ˩-di˩ ɲi˩!}}} \textcolor{Sepia}{\selectlanguage{english}The coffin is the thing in which the corpse is put! / The coffin is the thing to put the corpse!} \zh{棺材,是装尸体的! / 棺材,是用来装尸体的!}  
 \zh{量词}: \textcolor{darkblue}{\textbf{\ipa{ɭɯ˧}}}  \mytextsc{clf}: \textcolor{darkblue}{\textbf{\ipa{ɭɯ˧}}} 
\lhead{\firstmark}
\rhead{\botmark}

\subsection{\hspace{-0.5cm} {\Large \textcolor{darkblue}{\textbf{\ipa{kwæ˧}}}}\hspace{0.5cm}[\kern2pt{\textcolor{darkblue}{\textbf{\ipa{kwæ˥}}}}\kern2pt]} \hypertarget{kw\{\string_M1}{}
\markboth{\textcolor{darkblue}{\textbf{\ipa{kwæ˧}}}}{}
\textcolor{teal}{\mytextsc{verb}} \hspace{4pt} Tone: M.
\textcolor{Sepia}{\selectlanguage{english}To take care of, to take charge of.} \zh{管(汉语借词)。}  Borrowing: Chinese  \zh{管}

\lhead{\firstmark}
\rhead{\botmark}

\subsection{\hspace{-0.5cm} {\Large \textcolor{darkblue}{\textbf{\ipa{kwæ˧fæ˥}}}}\hspace{0.5cm}[\kern2pt{\textcolor{darkblue}{\textbf{\ipa{kwæ˧fæ˥}}}}\kern2pt]} \hypertarget{kw\{\string_Mf\{\string_T1}{}
\markboth{\textcolor{darkblue}{\textbf{\ipa{kwæ˧fæ˥}}}}{}
\textcolor{teal}{\mytextsc{noun}} \hspace{4pt} Tone: H\#.
\textcolor{Sepia}{\selectlanguage{english}Medium-sized beam.} \zh{中等大小的梁。}  \zh{量词}: \textcolor{darkblue}{\textbf{\ipa{pʰæ˧˥}}}  \mytextsc{clf}: \textcolor{darkblue}{\textbf{\ipa{pʰæ˧˥}}} 
\lhead{\firstmark}
\rhead{\botmark}

\subsection{\hspace{-0.5cm} {\Large \textcolor{darkblue}{\textbf{\ipa{kwæ˧pæ˥}}}}\hspace{0.5cm}[\kern2pt{\textcolor{darkblue}{\textbf{\ipa{kwæ˧pæ˥}}}}\kern2pt]} \hypertarget{kw\{\string_Mp\{\string_T1}{}
\markboth{\textcolor{darkblue}{\textbf{\ipa{kwæ˧pæ˥}}}}{}
\textcolor{teal}{\mytextsc{noun}} \hspace{4pt} Tone: H\#.
\textcolor{Sepia}{\selectlanguage{english}Harrow (made of wood).} \zh{耙(可能是汉语借词。原来借来的词:刮板?? 刮耙??)。}  \zh{量词}: \textcolor{darkblue}{\textbf{\ipa{nɑ˧}}}  \mytextsc{clf}: \textcolor{darkblue}{\textbf{\ipa{nɑ˧}}} 
\lhead{\firstmark}
\rhead{\botmark}

\subsection{\hspace{-0.5cm} {\Large \textcolor{darkblue}{\textbf{\ipa{kwæ˧tsɯ˧}}}}\hspace{0.5cm}[\kern2pt{\textcolor{darkblue}{\textbf{\ipa{kwæ˧tsɯ˧}}}}\kern2pt]} \hypertarget{kw\{\string_MtsM\string_M1}{}
\markboth{\textcolor{darkblue}{\textbf{\ipa{kwæ˧tsɯ˧}}}}{}
\textcolor{teal}{\mytextsc{noun}} \hspace{4pt} Tone: M.
\textcolor{Sepia}{\selectlanguage{english}Sunflower seed.} \zh{葵花瓜籽(汉语借词)。}  Borrowing: Chinese  \zh{瓜子}

\lhead{\firstmark}
\rhead{\botmark}

\subsection{\hspace{-0.5cm} {\Large \textcolor{darkblue}{\textbf{\ipa{‑kwɤ}}}}\hspace{0.5cm}[\kern2pt{\textcolor{darkblue}{\textbf{\ipa{xxxx groupe tonal entier sans aucun ton}}}}\kern2pt]} \hypertarget{‑kw71}{}
\markboth{\textcolor{darkblue}{\textbf{\ipa{‑kwɤ}}}}{}
\textcolor{teal}{\mytextsc{conjunction}} \hspace{4pt} Tone: 0.
\textcolor{Sepia}{\selectlanguage{english}When.} \zh{……的时候。} 
\lhead{\firstmark}
\rhead{\botmark}

\subsection{\hspace{-0.5cm} {\Large \textcolor{darkblue}{\textbf{\ipa{kwɤ˧ɭɯ˩}}}}\hspace{0.5cm}[\kern2pt{\textcolor{darkblue}{\textbf{\ipa{kwɤ˧ɭɯ˩}}}}\kern2pt]} \hypertarget{kw7\string_Ml\string_RM\string_B1}{}
\markboth{\textcolor{darkblue}{\textbf{\ipa{kwɤ˧ɭɯ˩}}}}{}
\textcolor{teal}{\mytextsc{noun}} \hspace{4pt} Tone: L\#.
\textcolor{Sepia}{\selectlanguage{english}Jug; jar; pitcher; also: treasure, valuable possession.} \zh{坛子、罐子 (陶器),宝贝。}  ¶ \textcolor{darkblue}{\textbf{\ipa{ʈʂʰɯ˧ | njɤ˧ kwɤ˧ɭɯ˩ ɲi˩!}}} \textcolor{Sepia}{\selectlanguage{english}(S)he is my treasure! (About a child)} \zh{他是我宝贝!(母亲说孩子是她的宝贝)}  
 \zh{量词}: \textcolor{darkblue}{\textbf{\ipa{ɭɯ˧}}}  \mytextsc{clf}: \textcolor{darkblue}{\textbf{\ipa{ɭɯ˧}}} 
\lhead{\firstmark}
\rhead{\botmark}

\subsection{\hspace{-0.5cm} {\Large \textcolor{darkblue}{\textbf{\ipa{kwɤ˧pɤ˧}}}}\hspace{0.5cm}[\kern2pt{\textcolor{darkblue}{\textbf{\ipa{kwɤ˧pɤ˧}}}}\kern2pt]} \hypertarget{kw7\string_Mp7\string_M1}{}
\markboth{\textcolor{darkblue}{\textbf{\ipa{kwɤ˧pɤ˧}}}}{}
\textcolor{teal}{\mytextsc{noun}} \hspace{4pt} Tone: M.
\textcolor{Sepia}{\selectlanguage{english}Teaching, explanation.} \zh{解释,教导、教诲。}  ¶ \textcolor{darkblue}{\textbf{\ipa{kwɤ˧pɤ˧ ɖɯ˧-kʰwɤ˥ lɑ˩}}} \textcolor{Sepia}{\selectlanguage{english}to provide an explanation, to teach something} \zh{解释一个道理、教一件事}  
 ¶ \textcolor{darkblue}{\textbf{\ipa{kwɤ˧pɤ˧ ɖɯ˧-kʰwɤ˥ | tʰi˧-lɑ˩-ɻ̍˩}}} \textcolor{Sepia}{\selectlanguage{english}As above: to provide an explanation, to teach something} \zh{同上:解释一个道理、教一件事}  
 ¶ \textcolor{darkblue}{\textbf{\ipa{[M23] kwɤ˧pɤ˧ lɑ˧˥}}} \textcolor{Sepia}{\selectlanguage{english}to teach} \zh{教、解释}  
 \zh{量词}: \textcolor{darkblue}{\textbf{\ipa{kʰwɤ˥}}}  \mytextsc{clf}: \textcolor{darkblue}{\textbf{\ipa{kʰwɤ˥}}} 
\lhead{\firstmark}
\rhead{\botmark}

\subsection{\hspace{-0.5cm} {\Large \textcolor{darkblue}{\textbf{\ipa{‑kwɤ˧tɕɯ˥}}}}\hspace{0.5cm}[\kern2pt{\textcolor{darkblue}{\textbf{\ipa{kwɤ˧tɕɯ˥}}}}\kern2pt]} \hypertarget{‑kw7\string_Mts£M\string_T1}{}
\markboth{\textcolor{darkblue}{\textbf{\ipa{‑kwɤ˧tɕɯ˥}}}}{}
\textcolor{teal}{\mytextsc{conjunction}} \hspace{4pt} Tone: H\#.
\textcolor{Sepia}{\selectlanguage{english}After; because, since, as.} \zh{因为,由于,既然。}  ¶ \textcolor{darkblue}{\textbf{\ipa{-kwɤ˧tɕɯ˥-lɑ˩}}} \textcolor{Sepia}{\selectlanguage{english}same meaning} \zh{同上}  
 ¶ \textcolor{darkblue}{\textbf{\ipa{ʈʂʰɯ˧ | go˩-kwɤ˩tɕɯ˥-lɑ˩, | hɑ˧ mɤ˧-dzɯ˥.}}} \textcolor{Sepia}{\selectlanguage{english}Because (s)he is ill, (s)he does not eat.} \zh{他病了,吃不下饭。}  
 ¶ \textcolor{darkblue}{\textbf{\ipa{[M18] ʈʂʰɯ˧ne˧-ʝi˥ | pi˧-kwɤ˩tɕɯ˩-lɑ˩, | wɤ˩˥ | lɑ˧hɑ˥ | ɖɯ˧-kʰwɤ˧ ʐwɤ˧˥.}}} \textcolor{Sepia}{\selectlanguage{english}After he said that, he went on to say something different / he changed his mind and said something quite different.} \zh{他这样说完以后,又讲了些其它的。}  
 ¶ \textcolor{darkblue}{\textbf{\ipa{[M18] ʈʂʰɯ˧ | tʰi˧-dzi˩-kwɤ˩-tɕɯ˩, | ɖɯ˧-kʰwɤ˧ ʐwɤ˧-ɻ̍˥: | “sɤ˧sɤ˧˥ | ʐwæ˧˥!”}}} \textcolor{Sepia}{\selectlanguage{english}After he got seated, he said the following: “How comfortable!”} \zh{他坐下后,说了这么一句:“真舒服!”}  

\lhead{\firstmark}
\rhead{\botmark}

\subsection{\hspace{-0.5cm} {\Large \textcolor{darkblue}{\textbf{\ipa{kwɤ˩\textsubscript{a}}}}}\hspace{0.5cm}[\kern2pt{\textcolor{darkblue}{\textbf{\ipa{kwɤ˩˥}}}}\kern2pt]} \hypertarget{kw7\string_Ba1}{}
\markboth{\textcolor{darkblue}{\textbf{\ipa{kwɤ˩\textsubscript{a}}}}}{}
\textcolor{teal}{\mytextsc{classifier}} \hspace{4pt} Tone: L\textsubscript{a}.
\textcolor{Sepia}{\selectlanguage{english}A string, a cluster of.} \zh{量词:串。}  ¶ \textcolor{darkblue}{\textbf{\ipa{kv̩˧ | ɖɯ˧-kwɤ˩}}} \textcolor{Sepia}{\selectlanguage{english}a braid of garlic} \zh{一辫大蒜}  
 ¶ \textcolor{darkblue}{\textbf{\ipa{lɑ˧tsɯ˥ | ɖɯ˧-kwɤ˩}}} \textcolor{Sepia}{\selectlanguage{english}a braid of hot peppers} \zh{一辫辣椒}  
 ¶ \textcolor{darkblue}{\textbf{\ipa{ʈʂʰɯ˧-kwɤ˥}}} \textcolor{Sepia}{\selectlanguage{english}\mytextsc{dem} \string_ (tone: H\# / H\$)} \zh{\mytextsc{指示代词} \string_}  

\lhead{\firstmark}
\rhead{\botmark}

\subsection{\hspace{-0.5cm} {\Large \textcolor{darkblue}{\textbf{\ipa{kwɤ˩\textsubscript{a}}}} \textsubscript{1}}\hspace{0.5cm}[\kern2pt{\textcolor{darkblue}{\textbf{\ipa{kwɤ˩˥}}}}\kern2pt]} \hypertarget{kw7\string_Ba1}{}
\markboth{\textcolor{darkblue}{\textbf{\ipa{kwɤ˩\textsubscript{a}}}} \textsubscript{1}}{}
\textcolor{teal}{\mytextsc{verb}} \hspace{4pt} Tone: L\textsubscript{a}.
\textcolor{Sepia}{\selectlanguage{english}To throw away (rubbish).} \zh{扔掉。}  ¶ \textcolor{darkblue}{\textbf{\ipa{mv̩˩tɕo˧ kwɤ˩}}} \textcolor{Sepia}{\selectlanguage{english}to throw away (rubbish)} \zh{扔掉(垃圾)}  
 ¶ \textcolor{darkblue}{\textbf{\ipa{tso˧\textasciitilde{}tso˧ kwɤ˥}}} \textcolor{Sepia}{\selectlanguage{english}to throw stuff away} \zh{扔东西}  

\lhead{\firstmark}
\rhead{\botmark}

\subsection{\hspace{-0.5cm} {\Large \textcolor{darkblue}{\textbf{\ipa{kwɤ˩\textsubscript{a}}}} \textsubscript{2}}\hspace{0.5cm}[\kern2pt{\textcolor{darkblue}{\textbf{\ipa{kwɤ˩˥}}}}\kern2pt]} \hypertarget{kw7\string_Ba2}{}
\markboth{\textcolor{darkblue}{\textbf{\ipa{kwɤ˩\textsubscript{a}}}} \textsubscript{2}}{}
\textcolor{teal}{\mytextsc{verb}} \hspace{4pt} Tone: L\textsubscript{a}.
\textcolor{Sepia}{\selectlanguage{english}To manage, to be in charge of, to take care of.} \zh{管(汉语借词)。}  ¶ \textcolor{darkblue}{\textbf{\ipa{ɖɯ˧-kʰwɤ˧ kwɤ˥}}} \textcolor{Sepia}{\selectlanguage{english}to supervise a bit} \zh{管一些}  

\lhead{\firstmark}
\rhead{\botmark}

\subsection{\hspace{-0.5cm} {\Large \textcolor{darkblue}{\textbf{\ipa{kwɤ˩-tjɤ˧ljɤ\#˥}}}}\hspace{0.5cm}[\kern2pt{\textcolor{darkblue}{\textbf{\ipa{kwɤ˧tjɤ˧ljɤ˧}}}}\kern2pt]} \hypertarget{kw7\string_B-tj7\string_Mlj7\#\string_T1}{}
\markboth{\textcolor{darkblue}{\textbf{\ipa{kwɤ˩-tjɤ˧ljɤ\#˥}}}}{}
\textcolor{teal}{\mytextsc{noun}} \hspace{4pt} Tone: L-\#H.
\textcolor{Sepia}{\selectlanguage{english}Small bell.} \zh{铃铛。}  \zh{量词}: \textcolor{darkblue}{\textbf{\ipa{ɭɯ˧}}}  \mytextsc{clf}: \textcolor{darkblue}{\textbf{\ipa{ɭɯ˧}}} 
\lhead{\firstmark}
\rhead{\botmark}

\subsection{\hspace{-0.5cm} {\Large \textcolor{darkblue}{\textbf{\ipa{kwɤ˩to˥}}}}\hspace{0.5cm}[\kern2pt{\textcolor{darkblue}{\textbf{\ipa{kwɤ˩to˥}}}}\kern2pt]} \hypertarget{kw7\string_Bto\string_T1}{}
\markboth{\textcolor{darkblue}{\textbf{\ipa{kwɤ˩to˥}}}}{}
\textcolor{teal}{\mytextsc{noun}} \hspace{4pt} Tone: LH.
\textcolor{Sepia}{\selectlanguage{english}Jawbone, mandible, lower jaw.} \zh{颌骨。}  \zh{量词}: \textcolor{darkblue}{\textbf{\ipa{ɭɯ˧}}}  \mytextsc{clf}: \textcolor{darkblue}{\textbf{\ipa{ɭɯ˧}}} 
\lhead{\firstmark}
\rhead{\botmark}

\newpage
\section*{\centering- \textcolor{darkblue}{\textbf{\ipa{kʰ}}} -}
\subsection{\hspace{-0.5cm} {\Large \textcolor{darkblue}{\textbf{\ipa{kʰɤ˧mi˥\$}}}}\hspace{0.5cm}[\kern2pt{\textcolor{darkblue}{\textbf{\ipa{xxxx ton non trouvé, à faire manuellement...}}}}\kern2pt]} \hypertarget{k\string_h7\string_Mmi\string_T\$1}{}
\markboth{\textcolor{darkblue}{\textbf{\ipa{kʰɤ˧mi˥\$}}}}{}
\textcolor{teal}{\mytextsc{noun}} \hspace{4pt} Tone: \$H.
\textcolor{Sepia}{\selectlanguage{english}Large basket carried on the back.} \zh{大背篓。}  \zh{量词}: \textcolor{darkblue}{\textbf{\ipa{kʰɤ˧˥}}}  \mytextsc{clf}: \textcolor{darkblue}{\textbf{\ipa{kʰɤ˧˥}}} 
\lhead{\firstmark}
\rhead{\botmark}

\subsection{\hspace{-0.5cm} {\Large \textcolor{darkblue}{\textbf{\ipa{kʰɤ˧ʂɯ˧}}}}\hspace{0.5cm}[\kern2pt{\textcolor{darkblue}{\textbf{\ipa{kʰɤ˧ʂɯ˧}}}}\kern2pt]} \hypertarget{k\string_h7\string_Ms`M\string_M1}{}
\markboth{\textcolor{darkblue}{\textbf{\ipa{kʰɤ˧ʂɯ˧}}}}{}
\textcolor{teal}{\mytextsc{verb}} \hspace{4pt} Tone: M.
\textcolor{Sepia}{\selectlanguage{english}To begin.} \zh{开始(汉语借词)。}  Borrowing: Chinese  \zh{开始}

\lhead{\firstmark}
\rhead{\botmark}

\subsection{\hspace{-0.5cm} {\Large \textcolor{darkblue}{\textbf{\ipa{kʰɤ˧zo˥\$}}}}\hspace{0.5cm}[\kern2pt{\textcolor{darkblue}{\textbf{\ipa{xxxx ton non trouvé, à faire manuellement...}}}}\kern2pt]} \hypertarget{k\string_h7\string_Mzo\string_T\$1}{}
\markboth{\textcolor{darkblue}{\textbf{\ipa{kʰɤ˧zo˥\$}}}}{}
\textcolor{teal}{\mytextsc{noun}} \hspace{4pt} Tone: \$H.
\textcolor{Sepia}{\selectlanguage{english}Small basket carried on the back.} \zh{小背篓。}  \zh{量词}: \textcolor{darkblue}{\textbf{\ipa{kʰɤ˧˥}}}  \mytextsc{clf}: \textcolor{darkblue}{\textbf{\ipa{kʰɤ˧˥}}} 
\lhead{\firstmark}
\rhead{\botmark}

\subsection{\hspace{-0.5cm} {\Large \textcolor{darkblue}{\textbf{\ipa{kʰɤ˩njɤ˩\textasciitilde{}kʰɤ˧njɤ˧}}}}\hspace{0.5cm}[\kern2pt{\textcolor{darkblue}{\textbf{\ipa{xxxx non-correspondance entre le nombre de morphèmes et le nombre de tons de morphèmes}}}}\kern2pt]} \hypertarget{k\string_h7\string_Bnj7\string_B~k\string_h7\string_Mnj7\string_M1}{}
\markboth{\textcolor{darkblue}{\textbf{\ipa{kʰɤ˩njɤ˩\textasciitilde{}kʰɤ˧njɤ˧}}}}{}
\textcolor{teal}{\mytextsc{adjective}} \hspace{4pt} Tone: L-.
\textcolor{Sepia}{\selectlanguage{english}Supple (movement).} \zh{柔软(动作)。} 
\lhead{\firstmark}
\rhead{\botmark}

\subsection{\hspace{-0.5cm} {\Large \textcolor{darkblue}{\textbf{\ipa{kʰɤ˧˥}}} \textsubscript{1}}\hspace{0.5cm}[\kern2pt{\textcolor{darkblue}{\textbf{\ipa{kʰɤ˧˥}}}}\kern2pt]} \hypertarget{k\string_h7\string_M\string_T1}{}
\markboth{\textcolor{darkblue}{\textbf{\ipa{kʰɤ˧˥}}} \textsubscript{1}}{}
\textcolor{teal}{\mytextsc{verb}} \hspace{4pt} Tone: MH.
\textcolor{Sepia}{\selectlanguage{english}To put out (a fire).} \zh{灭(火)。} \textit{See:} \hyperlink{}{\textcolor{darkblue}{\textbf{\ipa{hɑ̃˧˥}}} \textsubscript{1}} 
\lhead{\firstmark}
\rhead{\botmark}

\subsection{\hspace{-0.5cm} {\Large \textcolor{darkblue}{\textbf{\ipa{kʰɤ˧˥}}} \textsubscript{2}}\hspace{0.5cm}[\kern2pt{\textcolor{darkblue}{\textbf{\ipa{kʰɤ˧˥}}}}\kern2pt]} \hypertarget{k\string_h7\string_M\string_T2}{}
\markboth{\textcolor{darkblue}{\textbf{\ipa{kʰɤ˧˥}}} \textsubscript{2}}{}
\textcolor{teal}{\mytextsc{noun}} \hspace{4pt} Tone: MH.
\textcolor{Sepia}{\selectlanguage{english}Basket carried on the back.} \zh{背篓。}  \zh{量词}: \textcolor{darkblue}{\textbf{\ipa{kʰɤ˧˥}}}  \mytextsc{clf}: \textcolor{darkblue}{\textbf{\ipa{kʰɤ˧˥}}} 
\lhead{\firstmark}
\rhead{\botmark}

\subsection{\hspace{-0.5cm} {\Large \textcolor{darkblue}{\textbf{\ipa{kʰɤ˧˥\textsubscript{a}}}}}\hspace{0.5cm}[\kern2pt{\textcolor{darkblue}{\textbf{\ipa{kʰɤ˧˥}}}}\kern2pt]} \hypertarget{k\string_h7\string_M\string_Ta1}{}
\markboth{\textcolor{darkblue}{\textbf{\ipa{kʰɤ˧˥\textsubscript{a}}}}}{}
\textcolor{teal}{\mytextsc{classifier}} \hspace{4pt} Tone: MH\textsubscript{a}.
\textcolor{Sepia}{\selectlanguage{english}A basket of.} \zh{量词:筐。} 
\lhead{\firstmark}
\rhead{\botmark}

\subsection{\hspace{-0.5cm} {\Large \textcolor{darkblue}{\textbf{\ipa{kʰi˥}}}}\hspace{0.5cm}[\kern2pt{\textcolor{darkblue}{\textbf{\ipa{kʰi˥}}}}\kern2pt]} \hypertarget{k\string_hi\string_T1}{}
\markboth{\textcolor{darkblue}{\textbf{\ipa{kʰi˥}}}}{}
\textcolor{teal}{\mytextsc{verb}} \hspace{4pt} Tone: H.
\textcolor{Sepia}{\selectlanguage{english}To separate, to take apart (e.g. to separate fibers of linen to make thread).} \zh{拆开、分离(几根线)。}  ¶ \textcolor{darkblue}{\textbf{\ipa{sɑ˧ | le˧-kʰi˥}}} \textcolor{Sepia}{\selectlanguage{english}to separate linen fibres (to make thread)} \zh{拆开粗麻(为了纺出细麻线)}  

\lhead{\firstmark}
\rhead{\botmark}

\subsection{\hspace{-0.5cm} {\Large \textcolor{darkblue}{\textbf{\ipa{kʰi˥}}} \textsubscript{1}}\hspace{0.5cm}[\kern2pt{\textcolor{darkblue}{\textbf{\ipa{kʰi˥}}}}\kern2pt]} \hypertarget{k\string_hi\string_T1}{}
\markboth{\textcolor{darkblue}{\textbf{\ipa{kʰi˥}}} \textsubscript{1}}{}
\textcolor{teal}{\mytextsc{noun}} \hspace{4pt} Tone: \#H.
\textcolor{Sepia}{\selectlanguage{english}Door.} \zh{门。}  ¶ \textcolor{darkblue}{\textbf{\ipa{kʰi˧-zo\#˥}}} \textcolor{Sepia}{\selectlanguage{english}small door} \zh{小门}  
 \zh{量词}: \textcolor{darkblue}{\textbf{\ipa{̩pɤ˩}}}  \mytextsc{clf}: \textcolor{darkblue}{\textbf{\ipa{̩pɤ˩}}} 
\lhead{\firstmark}
\rhead{\botmark}

\subsection{\hspace{-0.5cm} {\Large \textcolor{darkblue}{\textbf{\ipa{kʰi˥}}} \textsubscript{2}}\hspace{0.5cm}[\kern2pt{\textcolor{darkblue}{\textbf{\ipa{kʰi˥}}}}\kern2pt]} \hypertarget{k\string_hi\string_T2}{}
\markboth{\textcolor{darkblue}{\textbf{\ipa{kʰi˥}}} \textsubscript{2}}{}
\textcolor{teal}{\mytextsc{noun}} \hspace{4pt} Tone: \#H.
\textcolor{Sepia}{\selectlanguage{english}Edge (monosyllable).} \zh{边(单音节)。} 
\lhead{\firstmark}
\rhead{\botmark}

\subsection{\hspace{-0.5cm} {\Large \textcolor{darkblue}{\textbf{\ipa{kʰi˧bɤ\#˥}}}}\hspace{0.5cm}[\kern2pt{\textcolor{darkblue}{\textbf{\ipa{kʰi˧bɤ˧}}}}\kern2pt]} \hypertarget{k\string_hi\string_Mb7\#\string_T1}{}
\markboth{\textcolor{darkblue}{\textbf{\ipa{kʰi˧bɤ\#˥}}}}{}
\textcolor{teal}{\mytextsc{noun}} \hspace{4pt} Tone: \#H.
\textcolor{Sepia}{\selectlanguage{english}Threshold.} \zh{门槛。}  \zh{量词}: \textcolor{darkblue}{\textbf{\ipa{ɭɯ˧}}}  \mytextsc{clf}: \textcolor{darkblue}{\textbf{\ipa{ɭɯ˧}}} 
\lhead{\firstmark}
\rhead{\botmark}

\subsection{\hspace{-0.5cm} {\Large \textcolor{darkblue}{\textbf{\ipa{-kʰi˧\textasciitilde{}kʰi˧}}}}\hspace{0.5cm}[\kern2pt{\textcolor{darkblue}{\textbf{\ipa{kʰi˧kʰi˧}}}}\kern2pt]} \hypertarget{-k\string_hi\string_M~k\string_hi\string_M1}{}
\markboth{\textcolor{darkblue}{\textbf{\ipa{-kʰi˧\textasciitilde{}kʰi˧}}}}{}
\textcolor{teal}{\mytextsc{postposition}} \hspace{4pt} Tone: \#H.
\textcolor{Sepia}{\selectlanguage{english}Around, close to, near, nearby.} \zh{周围、左右、旁边。}  ¶ \textcolor{darkblue}{\textbf{\ipa{ʑi˧qʰwɤ˧-kʰi˧\textasciitilde{}kʰi˧}}} \textcolor{Sepia}{\selectlanguage{english}near the house, in the vicinity of the house} \zh{房子周围}  
 ¶ \textcolor{darkblue}{\textbf{\ipa{[F5] njɤ˧-bv̩˧ | kʰi˧\textasciitilde{}kʰi˧}}} \textcolor{Sepia}{\selectlanguage{english}near me, around me} \zh{我的周围}  
 ¶ \textcolor{darkblue}{\textbf{\ipa{[M21] ʐɤ˩mi˩-kʰi˩\textasciitilde{}kʰi˩ se˩˥}}} \textcolor{Sepia}{\selectlanguage{english}to walk on the roadside, to walk by the side of the road} \zh{走在马路边}  

\lhead{\firstmark}
\rhead{\botmark}

\subsection{\hspace{-0.5cm} {\Large \textcolor{darkblue}{\textbf{\ipa{kʰi˧mi˧}}}}\hspace{0.5cm}[\kern2pt{\textcolor{darkblue}{\textbf{\ipa{kʰi˧mi˧}}}}\kern2pt]} \hypertarget{k\string_hi\string_Mmi\string_M1}{}
\markboth{\textcolor{darkblue}{\textbf{\ipa{kʰi˧mi˧}}}}{}
\textcolor{teal}{\mytextsc{noun}} \hspace{4pt} Tone: M.
\textcolor{Sepia}{\selectlanguage{english}Main entrance, main door.} \zh{大门。}  \zh{量词}: \textcolor{darkblue}{\textbf{\ipa{pɤ˩}}}  \mytextsc{clf}: \textcolor{darkblue}{\textbf{\ipa{pɤ˩}}} 
\lhead{\firstmark}
\rhead{\botmark}

\subsection{\hspace{-0.5cm} {\Large \textcolor{darkblue}{\textbf{\ipa{kʰi˧qʰv̩\#˥}}}}\hspace{0.5cm}[\kern2pt{\textcolor{darkblue}{\textbf{\ipa{kʰi˧qʰv̩˧}}}}\kern2pt]} \hypertarget{k\string_hi\string_Mq\string_hv\string_=\#\string_T1}{}
\markboth{\textcolor{darkblue}{\textbf{\ipa{kʰi˧qʰv̩\#˥}}}}{}
\textcolor{teal}{\mytextsc{noun}} \hspace{4pt} Tone: \#H.
\textcolor{Sepia}{\selectlanguage{english}Door.} \zh{门。}  ¶ \textcolor{darkblue}{\textbf{\ipa{kʰi˧qʰv˧ tʰv˧-ɲi˥}}} \textcolor{Sepia}{\selectlanguage{english}to reach the door, to get to the door (context: reaching the door of one's home, getting back home from a long trip)} \zh{到达(家)门(情景:从远方回家,到达家门)}  
 ¶ \textcolor{darkblue}{\textbf{\ipa{ɑ˩ʁo˧ kʰi˧qʰv˧ tʰv˧}}} \textcolor{Sepia}{\selectlanguage{english}to reach the door of the house} \zh{到达家门}  

\lhead{\firstmark}
\rhead{\botmark}

\subsection{\hspace{-0.5cm} {\Large \textcolor{darkblue}{\textbf{\ipa{kʰi˧-qʰwɤ˩}}}}\hspace{0.5cm}[\kern2pt{\textcolor{darkblue}{\textbf{\ipa{xxxx non-correspondance entre le nombre de morphèmes et le nombre de tons de morphèmes}}}}\kern2pt]} \hypertarget{k\string_hi\string_M-q\string_hw7\string_B1}{}
\markboth{\textcolor{darkblue}{\textbf{\ipa{kʰi˧-qʰwɤ˩}}}}{}
\textcolor{teal}{\mytextsc{noun}} \hspace{4pt} Tone: L\#.
\textcolor{Sepia}{\selectlanguage{english}Hinge.} \zh{门的合页。}  \zh{量词}: \textcolor{darkblue}{\textbf{\ipa{ɭɯ˧}}}  \mytextsc{clf}: \textcolor{darkblue}{\textbf{\ipa{ɭɯ˧}}} \textit{Syn:} \hyperlink{}{\textcolor{darkblue}{\textbf{\ipa{kʰi˧-bv̩˧lv̩˩}}}}. 
\lhead{\firstmark}
\rhead{\botmark}

\subsection{\hspace{-0.5cm} {\Large \textcolor{darkblue}{\textbf{\ipa{kʰi˧tɕʰɯ˩-mo˩}}}}\hspace{0.5cm}[\kern2pt{\textcolor{darkblue}{\textbf{\ipa{kʰi˧tɕʰɯ˩mo˧}}}}\kern2pt]} \hypertarget{k\string_hi\string_Mts£\string_hM\string_B-mo\string_B1}{}
\markboth{\textcolor{darkblue}{\textbf{\ipa{kʰi˧tɕʰɯ˩-mo˩}}}}{}
\textcolor{teal}{\mytextsc{noun}} \hspace{4pt} Tone: L\#-.
\textcolor{Sepia}{\selectlanguage{english}A poisonous mushroom.} \zh{一种有毒的菌子。}  ¶ \textcolor{darkblue}{\textbf{\ipa{ʈʂæ˧mo˧-kʰi˧tɕʰɯ˩-mo˩}}} \textcolor{Sepia}{\selectlanguage{english}same meaning} \zh{同上}  
\textit{Syn:} \hyperlink{}{\textcolor{darkblue}{\textbf{\ipa{ʈʂæ˧mo\#˥}}}}. 
\lhead{\firstmark}
\rhead{\botmark}

\subsection{\hspace{-0.5cm} {\Large \textcolor{darkblue}{\textbf{\ipa{kʰi˧˥}}}}\hspace{0.5cm}[\kern2pt{\textcolor{darkblue}{\textbf{\ipa{kʰi˧˥}}}}\kern2pt]} \hypertarget{k\string_hi\string_M\string_T1}{}
\markboth{\textcolor{darkblue}{\textbf{\ipa{kʰi˧˥}}}}{}
\textcolor{teal}{\mytextsc{verb}} \hspace{4pt} Tone: MH.
\textcolor{Sepia}{\selectlanguage{english}Past form of verb 'to leave'.} \zh{走(过去式)。}  ¶ \textcolor{darkblue}{\textbf{\ipa{ʈʂʰɯ˧ | zo˩qo˧ kʰi˧?}}} \textcolor{Sepia}{\selectlanguage{english}Where has (s)he gone to? / Where has (s)he left for?} \zh{他到哪里去了?}  
 ¶ \textcolor{darkblue}{\textbf{\ipa{[M23] no˧ | tsʰi˧ɲi˧ | ɑ˩pʰo˩˥ | ə˩-kʰi˩˥?}}} \textcolor{Sepia}{\selectlanguage{english}Did you go outside today? / Did you take a stroll today?} \zh{你今天出去了吗?}  

\lhead{\firstmark}
\rhead{\botmark}

\subsection{\hspace{-0.5cm} {\Large \textcolor{darkblue}{\textbf{\ipa{kʰo˥}}}}\hspace{0.5cm}[\kern2pt{\textcolor{darkblue}{\textbf{\ipa{kʰo˥}}}}\kern2pt]} \hypertarget{k\string_ho\string_T1}{}
\markboth{\textcolor{darkblue}{\textbf{\ipa{kʰo˥}}}}{}
\textcolor{teal}{\mytextsc{verb}} \hspace{4pt} Tone: H.
\textcolor{Sepia}{\selectlanguage{english}To spread (e.g. to do a bed; to spread/scatter objects all over the floor).} \zh{铺(床……)、铺得满地(果子、工具……)。}  ¶ \textcolor{darkblue}{\textbf{\ipa{kʰwæ˧ɻæ˧ kʰo˧}}} \textcolor{Sepia}{\selectlanguage{english}to spread a mat} \zh{铺垫子}  

\lhead{\firstmark}
\rhead{\botmark}

\subsection{\hspace{-0.5cm} {\Large \textcolor{darkblue}{\textbf{\ipa{kʰo˧bɤ˧}}}}\hspace{0.5cm}[\kern2pt{\textcolor{darkblue}{\textbf{\ipa{kʰo˧bɤ˧}}}}\kern2pt]} \hypertarget{k\string_ho\string_Mb7\string_M1}{}
\markboth{\textcolor{darkblue}{\textbf{\ipa{kʰo˧bɤ˧}}}}{}
\textcolor{teal}{\mytextsc{noun}} \hspace{4pt} Tone: M.
\textcolor{Sepia}{\selectlanguage{english}Home (solemn, formal word).} \zh{家(文言):母亲生活的空间:有家人,有火塘,有母亲在那里生活的那个空间。}  ¶ \textcolor{darkblue}{\textbf{\ipa{dʑi˧kʰi˧ le˧-gwɤ˩ | qo˩ tɑ˧-ze˥, | njɤ˧-ɕi˩ ə˩mɑ˩ kʰo˩bɤ˩ dʑɤ˩. |}}}  
 ¶ \textcolor{darkblue}{\textbf{\ipa{dʑi˧kʰi˧ le˧-gwɤ˩ | qo˩ tɑ˧-ze˥, | njɤ˧-ɕi˩ ə˩mɑ˩ kʰo˩bɤ˩-qo˩. |}}}  
 ¶ \textcolor{darkblue}{\textbf{\ipa{dʑi˧kʰi˧ le˧-gwɤ˩ qo˩ tɑ˩-ze˩, | njɤ˧-ɕi˩ ə˩mɑ˩ kʰo˩bɤ˩ dʑɤ˩. |}}}  
 \zh{量词}: \textcolor{darkblue}{\textbf{\ipa{ɭɯ˧}}}  \mytextsc{clf}: \textcolor{darkblue}{\textbf{\ipa{ɭɯ˧}}} 
\lhead{\firstmark}
\rhead{\botmark}

\subsection{\hspace{-0.5cm} {\Large \textcolor{darkblue}{\textbf{\ipa{kʰo˧lo˧}}}}\hspace{0.5cm}[\kern2pt{\textcolor{darkblue}{\textbf{\ipa{kʰo˧lo˧}}}}\kern2pt]} \hypertarget{k\string_ho\string_Mlo\string_M1}{}
\markboth{\textcolor{darkblue}{\textbf{\ipa{kʰo˧lo˧}}}}{}
\textcolor{teal}{\mytextsc{noun}} \hspace{4pt} Tone: M.
\textcolor{Sepia}{\selectlanguage{english}Prayer wheel.} \zh{转经筒。} Local Chinese dialect:\zh{祈祷轱辘。} \zh{量词}: \textcolor{darkblue}{\textbf{\ipa{ɭɯ˧}}}  \mytextsc{clf}: \textcolor{darkblue}{\textbf{\ipa{ɭɯ˧}}} 
\lhead{\firstmark}
\rhead{\botmark}

\subsection{\hspace{-0.5cm} {\Large \textcolor{darkblue}{\textbf{\ipa{kʰɯ˧di˧˥}}}}\hspace{0.5cm}[\kern2pt{\textcolor{darkblue}{\textbf{\ipa{kʰɯ˩di˩˥}}}}\kern2pt]} \hypertarget{k\string_hM\string_Mdi\string_M\string_T1}{}
\markboth{\textcolor{darkblue}{\textbf{\ipa{kʰɯ˧di˧˥}}}}{}
\textcolor{teal}{\mytextsc{noun}} \hspace{4pt} Tone: MH\#.
\textcolor{Sepia}{\selectlanguage{english}Container (general term).} \zh{容器。}  \zh{量词}: \textcolor{darkblue}{\textbf{\ipa{ɭɯ˧}}}  \mytextsc{clf}: \textcolor{darkblue}{\textbf{\ipa{ɭɯ˧}}} 
\lhead{\firstmark}
\rhead{\botmark}

\subsection{\hspace{-0.5cm} {\Large \textcolor{darkblue}{\textbf{\ipa{kʰɯ˧dv̩\#˥}}}}\hspace{0.5cm}[\kern2pt{\textcolor{darkblue}{\textbf{\ipa{kʰɯ˧dv̩˧˥}}}}\kern2pt]} \hypertarget{k\string_hM\string_Mdv\string_=\#\string_T1}{}
\markboth{\textcolor{darkblue}{\textbf{\ipa{kʰɯ˧dv̩\#˥}}}}{}
\textcolor{teal}{\mytextsc{noun}} \hspace{4pt} Tone: \#H.
\textcolor{Sepia}{\selectlanguage{english}Cripple, lame person.} \zh{跛。}  ¶ \textcolor{darkblue}{\textbf{\ipa{kʰɯ˧dv̩˧-hĩ˧}}} \textcolor{Sepia}{\selectlanguage{english}cripple} \zh{跛}  
 ¶ \textcolor{darkblue}{\textbf{\ipa{kʰɯ˧dv̩˧-tsʰo˧qʰwɤ˧}}} \textcolor{Sepia}{\selectlanguage{english}lame demon} \zh{跛鬼}  
 \zh{量词}: \textcolor{darkblue}{\textbf{\ipa{v̩˧}}}  \mytextsc{clf}: \textcolor{darkblue}{\textbf{\ipa{v̩˧}}} 
\lhead{\firstmark}
\rhead{\botmark}

\subsection{\hspace{-0.5cm} {\Large \textcolor{darkblue}{\textbf{\ipa{kʰɯ˧dʑɯ˧˥}}}}\hspace{0.5cm}[\kern2pt{\textcolor{darkblue}{\textbf{\ipa{kʰɯ˧dʑɯ˧}}}}\kern2pt]} \hypertarget{k\string_hM\string_Mdz£M\string_M\string_T1}{}
\markboth{\textcolor{darkblue}{\textbf{\ipa{kʰɯ˧dʑɯ˧˥}}}}{}
\textcolor{teal}{\mytextsc{noun}} \hspace{4pt} Tone: MH\#.
\textcolor{Sepia}{\selectlanguage{english}Leggings, puttee.} \zh{裹腿。}  \zh{量词}: \textcolor{darkblue}{\textbf{\ipa{dzi˧}}}  \mytextsc{clf}: \textcolor{darkblue}{\textbf{\ipa{dzi˧}}} 
\lhead{\firstmark}
\rhead{\botmark}

\subsection{\hspace{-0.5cm} {\Large \textcolor{darkblue}{\textbf{\ipa{kʰɯ˧pi˧}}}}\hspace{0.5cm}[\kern2pt{\textcolor{darkblue}{\textbf{\ipa{kʰɯ˧pi˧}}}}\kern2pt]} \hypertarget{k\string_hM\string_Mpi\string_M1}{}
\markboth{\textcolor{darkblue}{\textbf{\ipa{kʰɯ˧pi˧}}}}{}
\textcolor{teal}{\mytextsc{verb}} \hspace{4pt} Tone: .
\textcolor{Sepia}{\selectlanguage{english}To stumble, to trip.} \zh{绊。}  ¶ \textcolor{darkblue}{\textbf{\ipa{njɤ˧ kʰɯ˧pi˧-ze˧!}}} \textcolor{Sepia}{\selectlanguage{english}I have stumbled!} \zh{我绊了一跤!}  

\lhead{\firstmark}
\rhead{\botmark}

\subsection{\hspace{-0.5cm} {\Large \textcolor{darkblue}{\textbf{\ipa{kʰɯ˧pʰv̩˩}}}}\hspace{0.5cm}[\kern2pt{\textcolor{darkblue}{\textbf{\ipa{kʰɯ˧pʰv̩˩}}}}\kern2pt]} \hypertarget{k\string_hM\string_Mp\string_hv\string_=\string_B1}{}
\markboth{\textcolor{darkblue}{\textbf{\ipa{kʰɯ˧pʰv̩˩}}}}{}
\textcolor{teal}{\mytextsc{noun}} \hspace{4pt} Tone: L\#.
\textit{\textcolor{Sepia}{\selectlanguage{english}archaic}} [\zh{古语}] \textcolor{Sepia}{\selectlanguage{english}Chinese, Han.} \zh{汉族。}  \zh{量词}: \textcolor{darkblue}{\textbf{\ipa{v̩˧}}}  \mytextsc{clf}: \textcolor{darkblue}{\textbf{\ipa{v̩˧}}} 
\lhead{\firstmark}
\rhead{\botmark}

\subsection{\hspace{-0.5cm} {\Large \textcolor{darkblue}{\textbf{\ipa{kʰɯ˧tʰo˧˥}}}}\hspace{0.5cm}[\kern2pt{\textcolor{darkblue}{\textbf{\ipa{kʰɯ˧tʰo˥}}}}\kern2pt]} \hypertarget{k\string_hM\string_Mt\string_ho\string_M\string_T1}{}
\markboth{\textcolor{darkblue}{\textbf{\ipa{kʰɯ˧tʰo˧˥}}}}{}
\textcolor{teal}{\mytextsc{noun}} \hspace{4pt} Tone: H\#.
\textcolor{Sepia}{\selectlanguage{english}Chains (to tie a criminal's feet), made of iron.} \zh{脚链。}  ¶ \textcolor{darkblue}{\textbf{\ipa{kʰɯ˧tʰo˧ kʰɯ˥}}} \textcolor{Sepia}{\selectlanguage{english}to put chains (on someone's feet)} \zh{戴上脚链(在一个人的脚上)}  

\lhead{\firstmark}
\rhead{\botmark}

\subsection{\hspace{-0.5cm} {\Large \textcolor{darkblue}{\textbf{\ipa{kʰɯ˧tʰv̩\#˥}}}}\hspace{0.5cm}[\kern2pt{\textcolor{darkblue}{\textbf{\ipa{kʰɯ˧tʰv̩˧}}}}\kern2pt]} \hypertarget{k\string_hM\string_Mt\string_hv\string_=\#\string_T1}{}
\markboth{\textcolor{darkblue}{\textbf{\ipa{kʰɯ˧tʰv̩\#˥}}}}{}
\textcolor{teal}{\mytextsc{noun}} \hspace{4pt} Tone: \#H.
\textcolor{Sepia}{\selectlanguage{english}Pedal of the loom (to invert the vertical position of the threads).} \zh{织布机的脚蹬子=踏板。}  \zh{量词}: \textcolor{darkblue}{\textbf{\ipa{dze˩}}}  \mytextsc{clf}: \textcolor{darkblue}{\textbf{\ipa{dze˩}}} 
\lhead{\firstmark}
\rhead{\botmark}

\subsection{\hspace{-0.5cm} {\Large \textcolor{darkblue}{\textbf{\ipa{kʰɯ˧tsɯ˧bæ˥}}}}\hspace{0.5cm}[\kern2pt{\textcolor{darkblue}{\textbf{\ipa{kʰɯ˧tsɯ˧bæ˥}}}}\kern2pt]} \hypertarget{k\string_hM\string_MtsM\string_Mb\{\string_T1}{}
\markboth{\textcolor{darkblue}{\textbf{\ipa{kʰɯ˧tsɯ˧bæ˥}}}}{}
\textcolor{teal}{\mytextsc{noun}} \hspace{4pt} Tone: H\#.
\textcolor{Sepia}{\selectlanguage{english}Strip of fabric used to tie the bottom part of trousers, which were wide; in addition to this function, this piece of clothing was also decorative; it came from Tibetan regions.} \zh{绑腿布:用来绑裤腿的一块缠布,也有装饰功能(从藏族地区传过来的)。} 
\lhead{\firstmark}
\rhead{\botmark}

\subsection{\hspace{-0.5cm} {\Large \textcolor{darkblue}{\textbf{\ipa{kʰɯ˧tsʰɤ˧˥}}}}\hspace{0.5cm}[\kern2pt{\textcolor{darkblue}{\textbf{\ipa{kʰɯ˧tsʰɤ˧˥}}}}\kern2pt]} \hypertarget{k\string_hM\string_Mts\string_h7\string_M\string_T1}{}
\markboth{\textcolor{darkblue}{\textbf{\ipa{kʰɯ˧tsʰɤ˧˥}}}}{}
\textcolor{teal}{\mytextsc{noun}} \hspace{4pt} Tone: MH\#.
\textcolor{Sepia}{\selectlanguage{english}Leg.} \zh{腿,脚。}  \zh{量词}: \textcolor{darkblue}{\textbf{\ipa{pʰo˧˥}}}  \mytextsc{clf}: \textcolor{darkblue}{\textbf{\ipa{pʰo˧˥}}} 
\lhead{\firstmark}
\rhead{\botmark}

\subsection{\hspace{-0.5cm} {\Large \textcolor{darkblue}{\textbf{\ipa{kʰɯ˧ʈʂæ˧˥}}}}\hspace{0.5cm}[\kern2pt{\textcolor{darkblue}{\textbf{\ipa{kʰɯ˧ʈʂæ˧˥}}}}\kern2pt]} \hypertarget{k\string_hM\string_Mt`s`\{\string_M\string_T1}{}
\markboth{\textcolor{darkblue}{\textbf{\ipa{kʰɯ˧ʈʂæ˧˥}}}}{}
\textcolor{teal}{\mytextsc{noun}} \hspace{4pt} Tone: MH\#.
\textcolor{Sepia}{\selectlanguage{english}Ankle.} \zh{踝关节。}  \zh{量词}: \textcolor{darkblue}{\textbf{\ipa{ʈʂæ˧˥}}}  \mytextsc{clf}: \textcolor{darkblue}{\textbf{\ipa{ʈʂæ˧˥}}} 
\lhead{\firstmark}
\rhead{\botmark}

\subsection{\hspace{-0.5cm} {\Large \textcolor{darkblue}{\textbf{\ipa{kʰɯ˧ʈʂɤ\#˥}}}}\hspace{0.5cm}[\kern2pt{\textcolor{darkblue}{\textbf{\ipa{kʰɯ˧ʈʂɤ˧}}}}\kern2pt]} \hypertarget{k\string_hM\string_Mt`s`7\#\string_T1}{}
\markboth{\textcolor{darkblue}{\textbf{\ipa{kʰɯ˧ʈʂɤ\#˥}}}}{}
\textcolor{teal}{\mytextsc{noun}} \hspace{4pt} Tone: \#H.
\textcolor{Sepia}{\selectlanguage{english}Chicken feet.} \zh{鸡爪。}  ¶ \textcolor{darkblue}{\textbf{\ipa{kʰɯ˧ʈʂɤ˧ tʰv̩˧-ɭɯ\#˥}}} \textcolor{Sepia}{\selectlanguage{english}\mytextsc{n}+\mytextsc{dem}+\mytextsc{clf}} \zh{这只鸡爪}  
 ¶ \textcolor{darkblue}{\textbf{\ipa{kʰɯ˧ʈʂɤ˧ tʰv̩˧-ʈv̩˥\#}}} \textcolor{Sepia}{\selectlanguage{english}\mytextsc{n}+\mytextsc{dem}+\mytextsc{clf}} \zh{这只鸡爪}  
 \zh{量词}: \textcolor{darkblue}{\textbf{\ipa{ʈv̩˩ / ɭɯ˧}}}  \mytextsc{clf}: \textcolor{darkblue}{\textbf{\ipa{ʈv̩˩ / ɭɯ˧}}} 
\lhead{\firstmark}
\rhead{\botmark}

\subsection{\hspace{-0.5cm} {\Large \textcolor{darkblue}{\textbf{\ipa{kʰɯ˧ʐɯ˥\$}}}}\hspace{0.5cm}[\kern2pt{\textcolor{darkblue}{\textbf{\ipa{kʰɯ˧ʐɯ˥}}}}\kern2pt]} \hypertarget{k\string_hM\string_Mz`M\string_T\$1}{}
\markboth{\textcolor{darkblue}{\textbf{\ipa{kʰɯ˧ʐɯ˥\$}}}}{}
\textcolor{teal}{\mytextsc{noun}} \hspace{4pt} Tone: H\$.
\textcolor{Sepia}{\selectlanguage{english}Rice wine (low alcohol).} \zh{黄酒。}  \zh{量词}: \textcolor{darkblue}{\textbf{\ipa{qʰwɤ˧˥}}}  \mytextsc{clf}: \textcolor{darkblue}{\textbf{\ipa{qʰwɤ˧˥}}} 
\lhead{\firstmark}
\rhead{\botmark}

\subsection{\hspace{-0.5cm} {\Large \textcolor{darkblue}{\textbf{\ipa{kʰɯ˩}}}}\hspace{0.5cm}[\kern2pt{\textcolor{darkblue}{\textbf{\ipa{kʰɯ˥}}}}\kern2pt]} \hypertarget{k\string_hM\string_B1}{}
\markboth{\textcolor{darkblue}{\textbf{\ipa{kʰɯ˩}}}}{}
\textcolor{teal}{\mytextsc{noun}} \hspace{4pt} Tone: L.
\textcolor{Sepia}{\selectlanguage{english}Thread.} \zh{线。}  \zh{量词}: \textcolor{darkblue}{\textbf{\ipa{kʰɯ˩}}}  \mytextsc{clf}: \textcolor{darkblue}{\textbf{\ipa{kʰɯ˩}}} 
\lhead{\firstmark}
\rhead{\botmark}

\subsection{\hspace{-0.5cm} {\Large \textcolor{darkblue}{\textbf{\ipa{kʰɯ˩\textsubscript{b}}}}}\hspace{0.5cm}[\kern2pt{\textcolor{darkblue}{\textbf{\ipa{kʰɯ˥}}}}\kern2pt]} \hypertarget{k\string_hM\string_Bb1}{}
\markboth{\textcolor{darkblue}{\textbf{\ipa{kʰɯ˩\textsubscript{b}}}}}{}
\textcolor{teal}{\mytextsc{classifier}} \hspace{4pt} Tone: L\textsubscript{b}.
\textcolor{Sepia}{\selectlanguage{english}Classifier for threads.} \zh{量词:线(一根、一条)。}  ¶ \textcolor{darkblue}{\textbf{\ipa{kʰɯ˧ | ɖɯ˧-kʰɯ˩}}} \textcolor{Sepia}{\selectlanguage{english}a thread of string} \zh{一根线}  
 ¶ \textcolor{darkblue}{\textbf{\ipa{zɯ˧ | ɖɯ˧-kʰɯ˩}}} \textcolor{Sepia}{\selectlanguage{english}a blade of grass} \zh{一根草}  
 ¶ \textcolor{darkblue}{\textbf{\ipa{bæ˩ ɖɯ˥-kʰɯ˩}}} \textcolor{Sepia}{\selectlanguage{english}a thread of rope} \zh{一条绳子}  
 ¶ \textcolor{darkblue}{\textbf{\ipa{kʰɯ˧ | ʈʂʰɯ˧-kʰɯ˧˥}}} \textcolor{Sepia}{\selectlanguage{english}this thread (note: irregular tone pattern)} \zh{这根线}  

\lhead{\firstmark}
\rhead{\botmark}

\subsection{\hspace{-0.5cm} {\Large \textcolor{darkblue}{\textbf{\ipa{kʰɯ˩pv̩˩}}}}\hspace{0.5cm}[\kern2pt{\textcolor{darkblue}{\textbf{\ipa{kʰɯ˩pv̩˩˥}}}}\kern2pt]} \hypertarget{k\string_hM\string_Bpv\string_=\string_B1}{}
\markboth{\textcolor{darkblue}{\textbf{\ipa{kʰɯ˩pv̩˩}}}}{}
\textcolor{teal}{\mytextsc{noun}} \hspace{4pt} Tone: L.
\textcolor{Sepia}{\selectlanguage{english}Shuttle.} \zh{梭,梭子。}  \zh{量词}: \textcolor{darkblue}{\textbf{\ipa{ɭɯ˧}}}  \mytextsc{clf}: \textcolor{darkblue}{\textbf{\ipa{ɭɯ˧}}} \textit{See:} \hyperlink{}{\textcolor{darkblue}{\textbf{\ipa{pv̩˧qʰwɤ˥}}}} 
\lhead{\firstmark}
\rhead{\botmark}

\subsection{\hspace{-0.5cm} {\Large \textcolor{darkblue}{\textbf{\ipa{kʰɯ˩ʈɯ˩}}}}\hspace{0.5cm}[\kern2pt{\textcolor{darkblue}{\textbf{\ipa{kʰɯ˩ʈɯ˩˥}}}}\kern2pt]} \hypertarget{k\string_hM\string_Bt`M\string_B1}{}
\markboth{\textcolor{darkblue}{\textbf{\ipa{kʰɯ˩ʈɯ˩}}}}{}
\textcolor{teal}{\mytextsc{noun}} \hspace{4pt} Tone: L.
\textcolor{Sepia}{\selectlanguage{english}Root.} \zh{根。}  ¶ \textcolor{darkblue}{\textbf{\ipa{si˧dzi˩-kʰɯ˩ʈɯ˩}}} \textcolor{Sepia}{\selectlanguage{english}tree root} \zh{树根}  
 \zh{量词}: \textcolor{darkblue}{\textbf{\ipa{ʈv̩˩}}}  \mytextsc{clf}: \textcolor{darkblue}{\textbf{\ipa{ʈv̩˩}}} 
\lhead{\firstmark}
\rhead{\botmark}

\subsection{\hspace{-0.5cm} {\Large \textcolor{darkblue}{\textbf{\ipa{kʰɯ˧˥}}} \textsubscript{1}}\hspace{0.5cm}[\kern2pt{\textcolor{darkblue}{\textbf{\ipa{kʰɯ˧˥}}}}\kern2pt]} \hypertarget{k\string_hM\string_M\string_T1}{}
\markboth{\textcolor{darkblue}{\textbf{\ipa{kʰɯ˧˥}}} \textsubscript{1}}{}
\textcolor{teal}{\mytextsc{verb}} \hspace{4pt} Tone: MH.
\ding{202} \textcolor{Sepia}{\selectlanguage{english}To put into (e.g. to put into a bag); to dibble in seeds.} \zh{放,装(如:装进袋里),点种,收下。}  ¶ \textcolor{darkblue}{\textbf{\ipa{kʰɯ˩\textasciitilde{}kʰɯ˧˥}}} \textcolor{Sepia}{\selectlanguage{english}\mytextsc{red}} \zh{\mytextsc{red}}  
 ¶ \textcolor{darkblue}{\textbf{\ipa{qwɤ˧-qo˧ | si˧ tʰi˧-kʰɯ˧˥}}} \textcolor{Sepia}{\selectlanguage{english}to add wood into the fire} \zh{放木头在火中}  
\ding{203} \textcolor{Sepia}{\selectlanguage{english}To allow; to let; to cause (causative value).} \zh{让,\mytextsc{使动。}}  ¶ \textcolor{darkblue}{\textbf{\ipa{kʰv̩˩mi˩ zɯ˩\textasciitilde{}zɯ˩˥, | le˧-ɖæ˥-kʰɯ˩! | hĩ˧-zɯ˧\textasciitilde{}zɯ˥, | le˧-ʂæ˧-kʰɯ˥!}}} \textcolor{Sepia}{\selectlanguage{english}Dog's lifespan was made shorter, and man's lifespan was made longer! (A summary of the legend “How dog and man exchanged their lifespans”.)} \zh{够的寿命,变短了/使得变短!(而)人的寿命,变长了/使得变长!(《狗和人交换寿命》故事的一个提要)}  
 ¶ \textcolor{darkblue}{\textbf{\ipa{hwæ˧ kʰɯ˧ ə˥-bi˩? | - hwæ˧ kʰɯ˧-bi˥!}}} \textcolor{Sepia}{\selectlanguage{english}Do you agree to buy? - Yes!} \zh{(你)让买吗? - 让买!}  
 ¶ \textcolor{darkblue}{\textbf{\ipa{tɕʰi˧ kʰɯ˧ ə˥-bi˩?}}} \textcolor{Sepia}{\selectlanguage{english}Do you agree to sell?} \zh{(你)让卖吗?}  
 ¶ \textcolor{darkblue}{\textbf{\ipa{dzɯ˧ kʰɯ˩ ə˩-bi˩?}}} \textcolor{Sepia}{\selectlanguage{english}Do you agree to eat?} \zh{(你)让吃吗?}  
 ¶ \textcolor{darkblue}{\textbf{\ipa{tɕi˩ kʰɯ˥ ə˩-bi˩?}}} \textcolor{Sepia}{\selectlanguage{english}Do you agree to write?} \zh{(你)让写吗?}  
 ¶ \textcolor{darkblue}{\textbf{\ipa{ʈʰɯ˩ kʰɯ˩ ə˥-bi˩?}}} \textcolor{Sepia}{\selectlanguage{english}Do you agree to drink?} \zh{(你)让喝吗?}  
 ¶ \textcolor{darkblue}{\textbf{\ipa{ʐv̩˧ kʰɯ˥ ə˩-bi˩?}}} \textcolor{Sepia}{\selectlanguage{english}Do you agree to sew?} \zh{(你)让缝吗?}  

\lhead{\firstmark}
\rhead{\botmark}

\subsection{\hspace{-0.5cm} {\Large \textcolor{darkblue}{\textbf{\ipa{kʰɯ˧˥}}} \textsubscript{2}}\hspace{0.5cm}[\kern2pt{\textcolor{darkblue}{\textbf{\ipa{kʰɯ˧˥}}}}\kern2pt]} \hypertarget{k\string_hM\string_M\string_T2}{}
\markboth{\textcolor{darkblue}{\textbf{\ipa{kʰɯ˧˥}}} \textsubscript{2}}{}
\textcolor{teal}{\mytextsc{verb}} \hspace{4pt} Tone: MH.
\textcolor{Sepia}{\selectlanguage{english}To throw.} \zh{甩、扔(石头)。}  ¶ \textcolor{darkblue}{\textbf{\ipa{le˧-kʰɯ˧-ze˥}}} \textcolor{Sepia}{\selectlanguage{english}\mytextsc{accomp} \string_ \mytextsc{pfv}} \zh{甩了}  
 ¶ \textcolor{darkblue}{\textbf{\ipa{lv̩˧mi˧ kʰɯ˧˥}}} \textcolor{Sepia}{\selectlanguage{english}to throw a stone} \zh{扔石头}  

\lhead{\firstmark}
\rhead{\botmark}

\subsection{\hspace{-0.5cm} {\Large \textcolor{darkblue}{\textbf{\ipa{kʰɯ˧˥}}} \textsubscript{3}}\hspace{0.5cm}[\kern2pt{\textcolor{darkblue}{\textbf{\ipa{kʰɯ˧˥}}}}\kern2pt]} \hypertarget{k\string_hM\string_M\string_T3}{}
\markboth{\textcolor{darkblue}{\textbf{\ipa{kʰɯ˧˥}}} \textsubscript{3}}{}
\textcolor{teal}{\mytextsc{verb}} \hspace{4pt} Tone: MH.
\textcolor{Sepia}{\selectlanguage{english}To wear (a bracelet).} \zh{戴(手镯)。}  ¶ \textcolor{darkblue}{\textbf{\ipa{le˧-kʰɯ˧-ze˥}}} \textcolor{Sepia}{\selectlanguage{english}\mytextsc{accomp} \string_ \mytextsc{pfv}} \zh{戴了}  
 ¶ \textcolor{darkblue}{\textbf{\ipa{lo˩dʑo˧ kʰɯ˩}}} \textcolor{Sepia}{\selectlanguage{english}to wear a bracelet} \zh{戴手镯}  

\lhead{\firstmark}
\rhead{\botmark}

\subsection{\hspace{-0.5cm} {\Large \textcolor{darkblue}{\textbf{\ipa{kʰv̩˧˥}}}}\hspace{0.5cm}[\kern2pt{\textcolor{darkblue}{\textbf{\ipa{kʰv̩˧˥}}}}\kern2pt]} \hypertarget{k\string_hv\string_=\string_M\string_T1}{}
\markboth{\textcolor{darkblue}{\textbf{\ipa{kʰv̩˧˥}}}}{}
\textcolor{teal}{\mytextsc{noun}} \hspace{4pt} Tone: MH.
\ding{202} \textcolor{Sepia}{\selectlanguage{english}Year; year of age.} \zh{年、岁。}  ¶ \textcolor{darkblue}{\textbf{\ipa{kʰv̩˧-mæ˥}}} \textcolor{Sepia}{\selectlanguage{english}end of the year} \zh{年尾}  
 ¶ \textcolor{darkblue}{\textbf{\ipa{kʰv̩˧-mæ˥ ʂæ˩}}} \textcolor{Sepia}{\selectlanguage{english}intercalary year: a year with 13 months; this happens every 4 years or so} \zh{闰年(有13个月)}  
 ¶ \textcolor{darkblue}{\textbf{\ipa{kʰv̩˧-mæ˥ ɖæ˩}}} \textcolor{Sepia}{\selectlanguage{english}normal year, usual year: a year that has 12 months} \zh{正常的年份,普通年:一年十二个月}  
\ding{203} \textcolor{Sepia}{\selectlanguage{english}Astrological sign.} \zh{生肖。}  ¶ \textcolor{darkblue}{\textbf{\ipa{no˧ | ə˧tso˧ kʰv̩˧ ɲi˥?}}} \textcolor{Sepia}{\selectlanguage{english}What is your astrological sign?} \zh{你是属什么的?}  

\lhead{\firstmark}
\rhead{\botmark}

\subsection{\hspace{-0.5cm} {\Large \textcolor{darkblue}{\textbf{\ipa{kʰv̩˥}}} \textsubscript{1}}\hspace{0.5cm}[\kern2pt{\textcolor{darkblue}{\textbf{\ipa{kʰv̩˧˥}}}}\kern2pt]} \hypertarget{k\string_hv\string_=\string_T1}{}
\markboth{\textcolor{darkblue}{\textbf{\ipa{kʰv̩˥}}} \textsubscript{1}}{}
\textcolor{teal}{\mytextsc{noun}} \hspace{4pt} Tone: \#H.
\textcolor{Sepia}{\selectlanguage{english}Nest (monosyllable).} \zh{(鸟)巢。}  ¶ \textcolor{darkblue}{\textbf{\ipa{kʰv̩˧ ʈʂʰɯ˧-ɭɯ\#˥}}} \textcolor{Sepia}{\selectlanguage{english}\mytextsc{n}+\mytextsc{dem}+\mytextsc{clf}} \zh{这只鸟巢}  
 \zh{量词}: \textcolor{darkblue}{\textbf{\ipa{ɭɯ˧}}}  \mytextsc{clf}: \textcolor{darkblue}{\textbf{\ipa{ɭɯ˧}}} 
\lhead{\firstmark}
\rhead{\botmark}

\subsection{\hspace{-0.5cm} {\Large \textcolor{darkblue}{\textbf{\ipa{kʰv̩˥}}} \textsubscript{2}}\hspace{0.5cm}[\kern2pt{\textcolor{darkblue}{\textbf{\ipa{kʰv̩˥}}}}\kern2pt]} \hypertarget{k\string_hv\string_=\string_T2}{}
\markboth{\textcolor{darkblue}{\textbf{\ipa{kʰv̩˥}}} \textsubscript{2}}{}
\textcolor{teal}{\mytextsc{verb}} \hspace{4pt} Tone: H.
\textcolor{Sepia}{\selectlanguage{english}To harvest grass, to cut grass.} \zh{割(草)。}  ¶ \textcolor{darkblue}{\textbf{\ipa{le˧-kʰv̩˥-ze˩}}} \textcolor{Sepia}{\selectlanguage{english}\mytextsc{accomp} \string_ \mytextsc{pfv}} \zh{割了}  
 ¶ \textcolor{darkblue}{\textbf{\ipa{zɯ˧-kʰv̩˧}}} \textcolor{Sepia}{\selectlanguage{english}to cut grass} \zh{割草}  

\lhead{\firstmark}
\rhead{\botmark}

\subsection{\hspace{-0.5cm} {\Large \textcolor{darkblue}{\textbf{\ipa{kʰv̩˥}}} \textsubscript{3}}\hspace{0.5cm}[\kern2pt{\textcolor{darkblue}{\textbf{\ipa{kʰv̩˥}}}}\kern2pt]} \hypertarget{k\string_hv\string_=\string_T3}{}
\markboth{\textcolor{darkblue}{\textbf{\ipa{kʰv̩˥}}} \textsubscript{3}}{}
\textcolor{teal}{\mytextsc{noun}} \hspace{4pt} Tone: \#H.
\textcolor{Sepia}{\selectlanguage{english}Dog (monosyllable).} \zh{狗。}  ¶ \textcolor{darkblue}{\textbf{\ipa{kʰv̩˧-ʂe˧ dzɯ˧}}} \textcolor{Sepia}{\selectlanguage{english}to eat dog meat (a practice which is strongly antagonistic to Na culture, which considers dog as man's benefactor)} \zh{吃狗肉}  
 ¶ \textcolor{darkblue}{\textbf{\ipa{kʰv̩˧-zɯ˧\textasciitilde{}zɯ˥}}} \textcolor{Sepia}{\selectlanguage{english}dog's existence, dog's life (which dog exchanged with man, according to the legend)} \zh{狗的生命(传说狗与人交换了生命)}  
 ¶ \textcolor{darkblue}{\textbf{\ipa{kʰv̩˧ tʰv̩˧-mi˥\#}}} \textcolor{Sepia}{\selectlanguage{english}\mytextsc{n}+\mytextsc{dem}+\mytextsc{clf}} \zh{那条狗}  
 ¶ \textcolor{darkblue}{\textbf{\ipa{kʰv̩˧-gɤ˥ljɤ˩}}} \textcolor{Sepia}{\selectlanguage{english}roving dog} \zh{流浪狗}  
 \zh{量词}: \textcolor{darkblue}{\textbf{\ipa{mi˩}}} \textcolor{darkblue}{\textbf{\ipa{v̩˧}}} \textcolor{darkblue}{\textbf{\ipa{jɤ˧˥}}}  \mytextsc{clf}: \textcolor{darkblue}{\textbf{\ipa{mi˩}}} \textcolor{darkblue}{\textbf{\ipa{v̩˧}}} \textcolor{darkblue}{\textbf{\ipa{jɤ˧˥}}} 
\lhead{\firstmark}
\rhead{\botmark}

\subsection{\hspace{-0.5cm} {\Large \textcolor{darkblue}{\textbf{\ipa{kʰv̩˥}}} \textsubscript{4}}\hspace{0.5cm}[\kern2pt{\textcolor{darkblue}{\textbf{\ipa{kʰv̩˥}}}}\kern2pt]} \hypertarget{k\string_hv\string_=\string_T4}{}
\markboth{\textcolor{darkblue}{\textbf{\ipa{kʰv̩˥}}} \textsubscript{4}}{}
\textcolor{teal}{\mytextsc{verb}} \hspace{4pt} Tone: H.
\textcolor{Sepia}{\selectlanguage{english}To steal.} \zh{偷。}  ¶ \textcolor{darkblue}{\textbf{\ipa{hĩ˧-bv̩˧ tso˧\textasciitilde{}tso˧ kʰv̩˧}}} \textcolor{Sepia}{\selectlanguage{english}to steal someone's stuff, to steal someone else's property} \zh{偷别人的东西}  

\lhead{\firstmark}
\rhead{\botmark}

\subsection{\hspace{-0.5cm} {\Large \textcolor{darkblue}{\textbf{\ipa{kʰv̩˧˥\textsubscript{a}}}}}\hspace{0.5cm}[\kern2pt{\textcolor{darkblue}{\textbf{\ipa{kʰv̩˩˥}}}}\kern2pt]} \hypertarget{k\string_hv\string_=\string_M\string_Ta1}{}
\markboth{\textcolor{darkblue}{\textbf{\ipa{kʰv̩˧˥\textsubscript{a}}}}}{}
\textcolor{teal}{\mytextsc{classifier}} \hspace{4pt} Tone: MH\textsubscript{a}.
\textcolor{Sepia}{\selectlanguage{english}Year; year of age.} \zh{量词:年、岁。}  ¶ \textcolor{darkblue}{\textbf{\ipa{ɖɯ˧-kʰv̩˧˥}}} \textcolor{Sepia}{\selectlanguage{english}one year} \zh{一年}  

\lhead{\firstmark}
\rhead{\botmark}

\subsection{\hspace{-0.5cm} {\Large \textcolor{darkblue}{\textbf{\ipa{kʰv̩˧bv̩˧˥}}}}\hspace{0.5cm}[\kern2pt{\textcolor{darkblue}{\textbf{\ipa{kʰv̩˧bv̩˧˥}}}}\kern2pt]} \hypertarget{k\string_hv\string_=\string_Mbv\string_=\string_M\string_T1}{}
\markboth{\textcolor{darkblue}{\textbf{\ipa{kʰv̩˧bv̩˧˥}}}}{}
\textcolor{teal}{\mytextsc{noun}} \hspace{4pt} Tone: MH\#.
\textcolor{Sepia}{\selectlanguage{english}Kennel, doghouse.} \zh{狗窝。}  \zh{量词}: \textcolor{darkblue}{\textbf{\ipa{ɭɯ˧}}}  \mytextsc{clf}: \textcolor{darkblue}{\textbf{\ipa{ɭɯ˧}}} 
\lhead{\firstmark}
\rhead{\botmark}

\subsection{\hspace{-0.5cm} {\Large \textcolor{darkblue}{\textbf{\ipa{kʰv̩˩-kʰɤ˩}}}}\hspace{0.5cm}[\kern2pt{\textcolor{darkblue}{\textbf{\ipa{xxxx non-correspondance entre le nombre de morphèmes et le nombre de tons de morphèmes}}}}\kern2pt]} \hypertarget{k\string_hv\string_=\string_B-k\string_h7\string_B1}{}
\markboth{\textcolor{darkblue}{\textbf{\ipa{kʰv̩˩-kʰɤ˩}}}}{}
\textcolor{teal}{\mytextsc{noun}} \hspace{4pt} Tone: L.
\textcolor{Sepia}{\selectlanguage{english}Chicken nest.} \zh{鸡窝。}  \zh{量词}: \textcolor{darkblue}{\textbf{\ipa{ɭɯ˧}}}  \mytextsc{clf}: \textcolor{darkblue}{\textbf{\ipa{ɭɯ˧}}} 
\lhead{\firstmark}
\rhead{\botmark}

\subsection{\hspace{-0.5cm} {\Large \textcolor{darkblue}{\textbf{\ipa{kʰv̩˧kʰv̩˩}}}}\hspace{0.5cm}[\kern2pt{\textcolor{darkblue}{\textbf{\ipa{kʰv̩˩kʰv̩˩˥}}}}\kern2pt]} \hypertarget{k\string_hv\string_=\string_Mk\string_hv\string_=\string_B1}{}
\markboth{\textcolor{darkblue}{\textbf{\ipa{kʰv̩˧kʰv̩˩}}}}{}
\textcolor{teal}{\mytextsc{noun}} \hspace{4pt} Tone: L\#.
\textcolor{Sepia}{\selectlanguage{english}Year of the dog.} \zh{狗年。} 
\lhead{\firstmark}
\rhead{\botmark}

\subsection{\hspace{-0.5cm} {\Large \textcolor{darkblue}{\textbf{\ipa{kʰv̩˧kwæ˧}}}}\hspace{0.5cm}[\kern2pt{\textcolor{darkblue}{\textbf{\ipa{kʰv̩˧kwæ˩}}}}\kern2pt]} \hypertarget{k\string_hv\string_=\string_Mkw\{\string_M1}{}
\markboth{\textcolor{darkblue}{\textbf{\ipa{kʰv̩˧kwæ˧}}}}{}
\textcolor{teal}{\mytextsc{noun}} \hspace{4pt} Tone: M.
\textcolor{Sepia}{\selectlanguage{english}Bitter melon.} \zh{苦瓜。}  Borrowing: Chinese  \zh{苦瓜}
 \zh{量词}: \textcolor{darkblue}{\textbf{\ipa{ɭɯ˧}}}  \mytextsc{clf}: \textcolor{darkblue}{\textbf{\ipa{ɭɯ˧}}} 
\lhead{\firstmark}
\rhead{\botmark}

\subsection{\hspace{-0.5cm} {\Large \textcolor{darkblue}{\textbf{\ipa{kʰv̩˧mæ˧}}}}\hspace{0.5cm}[\kern2pt{\textcolor{darkblue}{\textbf{\ipa{kʰv̩˧mæ˥}}}}\kern2pt]} \hypertarget{k\string_hv\string_=\string_Mm\{\string_M1}{}
\markboth{\textcolor{darkblue}{\textbf{\ipa{kʰv̩˧mæ˧}}}}{}
\textcolor{teal}{\mytextsc{noun}} \hspace{4pt} Tone: M.
\textcolor{Sepia}{\selectlanguage{english}Robber, bandit.} \zh{强盗。}  ¶ \textcolor{darkblue}{\textbf{\ipa{kʰv̩˧mæ˧ ʝi˧-hĩ˧-hĩ˧}}} \textcolor{Sepia}{\selectlanguage{english}person who robs, robber} \zh{当强盗的人=强盗}  
 ¶ \textcolor{darkblue}{\textbf{\ipa{kʰv̩˧mæ˧-ni˩-zo˩! | hĩ˧ lɑ˩-ho˩!}}} \textcolor{Sepia}{\selectlanguage{english}He's like a bandit! He may hit people!} \zh{他像强盗似的!会打人的!}  
 ¶ \textcolor{darkblue}{\textbf{\ipa{kʰv̩˧mæ˧-ʑi˩}}} \textcolor{Sepia}{\selectlanguage{english}prison: literally “house for thieves”} \zh{监狱。直译:“贼家”}  
 ¶ \textcolor{darkblue}{\textbf{\ipa{kʰv̩˧mæ˧-ʝi˧-hĩ˧, | lo˧ʑi˥bv̩˩-qo˩ ʈæ˩!}}} \textcolor{Sepia}{\selectlanguage{english}Thieves are tied up in prisons / are sent to prison!} \zh{贼,被关在监狱!}  
 ¶ \textcolor{darkblue}{\textbf{\ipa{no˧ | kʰv̩˧mæ˧-pʰæ˧qʰwɤ˩-ne˩-ʝi˩-zo˩!}}} \textcolor{Sepia}{\selectlanguage{english}You have the face of a thief! / You really look like a thief! (An accusation about someone one thinks is a thief)} \zh{你有一张贼脸!(控告一个人)}  
 \zh{量词}: \textcolor{darkblue}{\textbf{\ipa{v̩˧}}}  \mytextsc{clf}: \textcolor{darkblue}{\textbf{\ipa{v̩˧}}} 
\lhead{\firstmark}
\rhead{\botmark}

\subsection{\hspace{-0.5cm} {\Large \textcolor{darkblue}{\textbf{\ipa{kʰv̩˩mi˩}}}}\hspace{0.5cm}[\kern2pt{\textcolor{darkblue}{\textbf{\ipa{kʰv̩˧mi˧}}}}\kern2pt]} \hypertarget{k\string_hv\string_=\string_Bmi\string_B1}{}
\markboth{\textcolor{darkblue}{\textbf{\ipa{kʰv̩˩mi˩}}}}{}
\textcolor{teal}{\mytextsc{noun}} \hspace{4pt} Tone: L.
\textcolor{Sepia}{\selectlanguage{english}Dog (either he-dog or she-dog).} \zh{狗。}  ¶ \textcolor{darkblue}{\textbf{\ipa{kʰv̩˩mi˩ ʈʂʰɯ˩-jɤ˧}}} \textcolor{Sepia}{\selectlanguage{english}\mytextsc{n}+\mytextsc{dem}+\mytextsc{clf}} \zh{这条狗}  
 ¶ \textcolor{darkblue}{\textbf{\ipa{di˧qo˧-kʰv̩˩mi˩}}} \textcolor{Sepia}{\selectlanguage{english}the dogs of the plain (which, unlike dogs in mountain hamlets, get to see lots of passers-by, and are less likely to bite strangers)} \zh{平坝的狗}  
 ¶ \textcolor{darkblue}{\textbf{\ipa{kʰv̩˩mi˩-gɤ˥ljɤ˩}}} \textcolor{Sepia}{\selectlanguage{english}roving dog} \zh{流浪的狗}  
 \zh{量词}: \textcolor{darkblue}{\textbf{\ipa{v̩˧, terme respectueux (le même que pour les humains)}}} \textcolor{darkblue}{\textbf{\ipa{on peut aussi dire: jɤ˧˥}}}  \mytextsc{clf}: \textcolor{darkblue}{\textbf{\ipa{v̩˧, terme respectueux (le même que pour les humains)}}} \textcolor{darkblue}{\textbf{\ipa{on peut aussi dire: jɤ˧˥}}} 
\lhead{\firstmark}
\rhead{\botmark}

\subsection{\hspace{-0.5cm} {\Large \textcolor{darkblue}{\textbf{\ipa{kʰv̩˧mv̩˥}}}}\hspace{0.5cm}[\kern2pt{\textcolor{darkblue}{\textbf{\ipa{kʰv̩˩mv̩˩˥}}}}\kern2pt]} \hypertarget{k\string_hv\string_=\string_Mmv\string_=\string_T1}{}
\markboth{\textcolor{darkblue}{\textbf{\ipa{kʰv̩˧mv̩˥}}}}{}
\textcolor{teal}{\mytextsc{noun}} \hspace{4pt} Tone: H\#.
\textcolor{Sepia}{\selectlanguage{english}Female puppy. The term is also used as a temporary name for little girls, during the first months of their life, before they are given a real name. This ugly term is intended to disgust evil spirits, which will therefore turn their attention away from the infant. (In the early 21st century, the registry office requires a name to be given at birth; but this name only begins to be used by the family after the first months of life have elapsed.).} \zh{小母狗(给刚出生的女孩起的名字,让鬼对她不感兴趣,不会来害小孩)。}  \zh{量词}: \textcolor{darkblue}{\textbf{\ipa{v̩˧}}}  \mytextsc{clf}: \textcolor{darkblue}{\textbf{\ipa{v̩˧}}} 
\lhead{\firstmark}
\rhead{\botmark}

\subsection{\hspace{-0.5cm} {\Large \textcolor{darkblue}{\textbf{\ipa{kʰv̩˧nɑ˥}}}}\hspace{0.5cm}[\kern2pt{\textcolor{darkblue}{\textbf{\ipa{kʰv̩˩nɑ˧˥}}}}\kern2pt]} \hypertarget{k\string_hv\string_=\string_MnA\string_T1}{}
\markboth{\textcolor{darkblue}{\textbf{\ipa{kʰv̩˧nɑ˥}}}}{}
\textcolor{teal}{\mytextsc{noun}} \hspace{4pt} Tone: H\#.
\textcolor{Sepia}{\selectlanguage{english}Dog (formal word, used in elevated speech).} \zh{狗。}  \zh{量词}: \textcolor{darkblue}{\textbf{\ipa{mi˩}}}  \mytextsc{clf}: \textcolor{darkblue}{\textbf{\ipa{mi˩}}} 
\lhead{\firstmark}
\rhead{\botmark}

\subsection{\hspace{-0.5cm} {\Large \textcolor{darkblue}{\textbf{\ipa{kʰv̩˧pʰæ˧}}}}\hspace{0.5cm}[\kern2pt{\textcolor{darkblue}{\textbf{\ipa{kʰv̩˧pʰæ˥}}}}\kern2pt]} \hypertarget{k\string_hv\string_=\string_Mp\string_h\{\string_M1}{}
\markboth{\textcolor{darkblue}{\textbf{\ipa{kʰv̩˧pʰæ˧}}}}{}
\textcolor{teal}{\mytextsc{noun}} \hspace{4pt} Tone: M.
\textcolor{Sepia}{\selectlanguage{english}Age.} \zh{年龄。}  ¶ \textcolor{darkblue}{\textbf{\ipa{kʰv̩˧pʰæ˧ tɕi˩}}} \textcolor{Sepia}{\selectlanguage{english}young} \zh{年轻}  
 ¶ \textcolor{darkblue}{\textbf{\ipa{kʰv̩˧pʰæ˧ | tɕi˩-hĩ˩˥}}} \textcolor{Sepia}{\selectlanguage{english}young} \zh{年轻的}  

\lhead{\firstmark}
\rhead{\botmark}

\subsection{\hspace{-0.5cm} {\Large \textcolor{darkblue}{\textbf{\ipa{kʰv̩˧-pʰo˥}}}}\hspace{0.5cm}[\kern2pt{\textcolor{darkblue}{\textbf{\ipa{xxxx non-correspondance entre le nombre de morphèmes et le nombre de tons de morphèmes}}}}\kern2pt]} \hypertarget{k\string_hv\string_=\string_M-p\string_ho\string_T1}{}
\markboth{\textcolor{darkblue}{\textbf{\ipa{kʰv̩˧-pʰo˥}}}}{}
\textcolor{teal}{\mytextsc{noun}} \hspace{4pt} Tone: H\#.
\textcolor{Sepia}{\selectlanguage{english}Half a year.} \zh{半年。}  ¶ \textcolor{darkblue}{\textbf{\ipa{ɖɯ˧-kʰv̩˧-kʰv̩˥-pʰo˩}}} \textcolor{Sepia}{\selectlanguage{english}one year and a half} \zh{一年半}  

\lhead{\firstmark}
\rhead{\botmark}

\subsection{\hspace{-0.5cm} {\Large \textcolor{darkblue}{\textbf{\ipa{kʰv̩˧pʰv̩\#˥}}}}\hspace{0.5cm}[\kern2pt{\textcolor{darkblue}{\textbf{\ipa{kʰv̩˧pʰv̩˥}}}}\kern2pt]} \hypertarget{k\string_hv\string_=\string_Mp\string_hv\string_=\#\string_T1}{}
\markboth{\textcolor{darkblue}{\textbf{\ipa{kʰv̩˧pʰv̩\#˥}}}}{}
\textcolor{teal}{\mytextsc{noun}} \hspace{4pt} Tone: \#H.
\textcolor{Sepia}{\selectlanguage{english}He-dog.} \zh{公狗。}  ¶ \textcolor{darkblue}{\textbf{\ipa{kʰv̩˧pʰv̩˧ ʈʂʰɯ˧-ɭɯ\#˥}}} \textcolor{Sepia}{\selectlanguage{english}\mytextsc{n}+\mytextsc{dem}+\mytextsc{clf}} \zh{这只公狗}  
 ¶ \textcolor{darkblue}{\textbf{\ipa{kʰv̩˧pʰv̩˧ tʰv̩˧-mi˧˥}}} \textcolor{Sepia}{\selectlanguage{english}\mytextsc{n}+\mytextsc{dem}+\mytextsc{clf}} \zh{这只公狗}  
 ¶ \textcolor{darkblue}{\textbf{\ipa{kʰv̩˧pʰv̩˧ tʰv̩˧-v̩\#˥}}} \textcolor{Sepia}{\selectlanguage{english}\mytextsc{n}+\mytextsc{dem}+\mytextsc{clf}} \zh{这个公狗}  
 \zh{量词}: \textcolor{darkblue}{\textbf{\ipa{v̩˧ / mi˩ / ɭɯ˧}}}  \mytextsc{clf}: \textcolor{darkblue}{\textbf{\ipa{v̩˧ / mi˩ / ɭɯ˧}}} 
\lhead{\firstmark}
\rhead{\botmark}

\subsection{\hspace{-0.5cm} {\Large \textcolor{darkblue}{\textbf{\ipa{kʰv̩˧qʰwɤ˧˥}}}}\hspace{0.5cm}[\kern2pt{\textcolor{darkblue}{\textbf{\ipa{kʰv̩˧qʰwɤ˧}}}}\kern2pt]} \hypertarget{k\string_hv\string_=\string_Mq\string_hw7\string_M\string_T1}{}
\markboth{\textcolor{darkblue}{\textbf{\ipa{kʰv̩˧qʰwɤ˧˥}}}}{}
\textcolor{teal}{\mytextsc{noun}} \hspace{4pt} Tone: MH\#.
\textcolor{Sepia}{\selectlanguage{english}Bad year, year when the crops are bad.} \zh{庄稼收成不好的(一)年。}  ¶ \textcolor{darkblue}{\textbf{\ipa{kʰv̩˧qʰwɤ˧ tʰv̩˧˥}}} \textcolor{Sepia}{\selectlanguage{english}the year is bad; crops are bad this year; a bad year has come} \zh{今年,收成不好。}  

\lhead{\firstmark}
\rhead{\botmark}

\subsection{\hspace{-0.5cm} {\Large \textcolor{darkblue}{\textbf{\ipa{kʰv̩˧ʂæ˧˥}}}}\hspace{0.5cm}[\kern2pt{\textcolor{darkblue}{\textbf{\ipa{kʰv̩˧ʂæ˥}}}}\kern2pt]} \hypertarget{k\string_hv\string_=\string_Ms`\{\string_M\string_T1}{}
\markboth{\textcolor{darkblue}{\textbf{\ipa{kʰv̩˧ʂæ˧˥}}}}{}
\textcolor{teal}{\mytextsc{verb}} \hspace{4pt} Tone: MH.
\textcolor{Sepia}{\selectlanguage{english}To hunt (leading a dog).} \zh{打猎、赶走、驱逐。}  ¶ \textcolor{darkblue}{\textbf{\ipa{kʰv̩˧ʂæ˧ hɯ˧˥}}} \textcolor{Sepia}{\selectlanguage{english}(He/she) has gone hunting} \zh{狩猎去了}  

\lhead{\firstmark}
\rhead{\botmark}

\subsection{\hspace{-0.5cm} {\Large \textcolor{darkblue}{\textbf{\ipa{kʰv̩˧ʂɯ˥}}}}\hspace{0.5cm}[\kern2pt{\textcolor{darkblue}{\textbf{\ipa{kʰv̩˧ʂɯ˥}}}}\kern2pt]} \hypertarget{k\string_hv\string_=\string_Ms`M\string_T1}{}
\markboth{\textcolor{darkblue}{\textbf{\ipa{kʰv̩˧ʂɯ˥}}}}{}
\textcolor{teal}{\mytextsc{verb}} \hspace{4pt} Tone: .
\textcolor{Sepia}{\selectlanguage{english}To celebrate the New Year.} \zh{过年。} 
\lhead{\firstmark}
\rhead{\botmark}

\subsection{\hspace{-0.5cm} {\Large \textcolor{darkblue}{\textbf{\ipa{kʰv̩˧sɯ˧sɯ˩}}}}\hspace{0.5cm}[\kern2pt{\textcolor{darkblue}{\textbf{\ipa{kʰv̩˧sɯ˧sɯ˩}}}}\kern2pt]} \hypertarget{k\string_hv\string_=\string_MsM\string_MsM\string_B1}{}
\markboth{\textcolor{darkblue}{\textbf{\ipa{kʰv̩˧sɯ˧sɯ˩}}}}{}
\textcolor{teal}{\mytextsc{noun}} \hspace{4pt} Tone: L\#.
\textcolor{Sepia}{\selectlanguage{english}A flowering plant in the legume family: \textit{Flemingia strobilifera}, also known as \textit{Moghania fruticulosa}.} \zh{球穗千斤拔、半灌木千斤拔、大苞千斤拔。} Local Chinese dialect:\zh{耗子耳朵。} \zh{量词}: \textcolor{darkblue}{\textbf{\ipa{kɤ˧˥}}}  \mytextsc{clf}: \textcolor{darkblue}{\textbf{\ipa{kɤ˧˥}}} 
\lhead{\firstmark}
\rhead{\botmark}

\subsection{\hspace{-0.5cm} {\Large \textcolor{darkblue}{\textbf{\ipa{kʰv̩˧tɕʰi˥\$}}}}\hspace{0.5cm}[\kern2pt{\textcolor{darkblue}{\textbf{\ipa{kʰv̩˧tɕʰi˥}}}}\kern2pt]} \hypertarget{k\string_hv\string_=\string_Mts£\string_hi\string_T\$1}{}
\markboth{\textcolor{darkblue}{\textbf{\ipa{kʰv̩˧tɕʰi˥\$}}}}{}
\textcolor{teal}{\mytextsc{noun}} \hspace{4pt} Tone: H\$.
\textcolor{Sepia}{\selectlanguage{english}Solution, way out.} \zh{办法。}  ¶ \textcolor{darkblue}{\textbf{\ipa{kʰv̩˧tɕʰi˥ | mɤ˧-dʑo˧-ze˧! | ɻ̃˧-ɻ̍˧ tʰo˩!}}} \textcolor{Sepia}{\selectlanguage{english}There is nothing we can do anymore! It's a catastrophe!} \zh{没有办法了!糟糕了!}  

\lhead{\firstmark}
\rhead{\botmark}

\subsection{\hspace{-0.5cm} {\Large \textcolor{darkblue}{\textbf{\ipa{kʰv̩˧tsʰi˧-bo˥tsʰi˩}}}}\hspace{0.5cm}[\kern2pt{\textcolor{darkblue}{\textbf{\ipa{kʰv̩˧tsʰi˧bo˥tsʰi˩}}}}\kern2pt]} \hypertarget{k\string_hv\string_=\string_Mts\string_hi\string_M-bo\string_Tts\string_hi\string_B1}{}
\markboth{\textcolor{darkblue}{\textbf{\ipa{kʰv̩˧tsʰi˧-bo˥tsʰi˩}}}}{}
\textcolor{teal}{\mytextsc{noun}} \hspace{4pt} Tone: \#H-.
\textcolor{Sepia}{\selectlanguage{english}Mole shrew.} \zh{鼹鼠。}  \zh{量词}: \textcolor{darkblue}{\textbf{\ipa{pʰo˧˥}}} \textcolor{darkblue}{\textbf{\ipa{v̩˧}}}  \mytextsc{clf}: \textcolor{darkblue}{\textbf{\ipa{pʰo˧˥}}} \textcolor{darkblue}{\textbf{\ipa{v̩˧}}} 
\lhead{\firstmark}
\rhead{\botmark}

\subsection{\hspace{-0.5cm} {\Large \textcolor{darkblue}{\textbf{\ipa{kʰv̩˩tsɤ˩mi˥}}}}\hspace{0.5cm}[\kern2pt{\textcolor{darkblue}{\textbf{\ipa{kʰv̩˩tsɤ˩mi˥}}}}\kern2pt]} \hypertarget{k\string_hv\string_=\string_Bts7\string_Bmi\string_T1}{}
\markboth{\textcolor{darkblue}{\textbf{\ipa{kʰv̩˩tsɤ˩mi˥}}}}{}
\textcolor{teal}{\mytextsc{noun}} \hspace{4pt} Tone: L+H\#.
\textcolor{Sepia}{\selectlanguage{english}She-dog.} \zh{母狗。}  \zh{量词}: \textcolor{darkblue}{\textbf{\ipa{v̩˧}}}  \mytextsc{clf}: \textcolor{darkblue}{\textbf{\ipa{v̩˧}}} 
\lhead{\firstmark}
\rhead{\botmark}

\subsection{\hspace{-0.5cm} {\Large \textcolor{darkblue}{\textbf{\ipa{kʰv̩˧zo˥\$}}}}\hspace{0.5cm}[\kern2pt{\textcolor{darkblue}{\textbf{\ipa{kʰv̩˧zo˥}}}}\kern2pt]} \hypertarget{k\string_hv\string_=\string_Mzo\string_T\$1}{}
\markboth{\textcolor{darkblue}{\textbf{\ipa{kʰv̩˧zo˥\$}}}}{}
\textcolor{teal}{\mytextsc{noun}} \hspace{4pt} Tone: H\$.
\textcolor{Sepia}{\selectlanguage{english}A family name from Yongning. There are two families in Yongning that carry this name.} \zh{一个姓。这个姓,永宁有两家。}  ¶ \textcolor{darkblue}{\textbf{\ipa{kʰv̩˧zo˧=ɻ̍˥\$}}} \textcolor{Sepia}{\selectlanguage{english}the \textcolor{darkblue}{\textbf{\ipa{/kʰv̩˧zo˥\$/}}} clan, the \textcolor{darkblue}{\textbf{\ipa{/kʰv̩˧zo˥\$/}}} family} \zh{\textcolor{darkblue}{\textbf{\ipa{/kʰv̩˧zo˥\$/}}}家族}  
 ¶ \textcolor{darkblue}{\textbf{\ipa{kʰv̩˧zo˥-tsʰɯ˩ɻ̍˩}}} \textcolor{Sepia}{\selectlanguage{english}the name of a person, containing both a family name: \textcolor{darkblue}{\textbf{\ipa{/kʰv̩˧zo˥\$/}}}, and a given name: \textcolor{darkblue}{\textbf{\ipa{/tsʰɯ˧ɻ\#˥/}}}} \zh{一个人的名字:姓为\textcolor{darkblue}{\textbf{\ipa{/kʰv̩˧zo˥\$/}}},名为\textcolor{darkblue}{\textbf{\ipa{/tsʰɯ˧ɻ\#˥/}}}}  

\lhead{\firstmark}
\rhead{\botmark}

\subsection{\hspace{-0.5cm} {\Large \textcolor{darkblue}{\textbf{\ipa{kʰv̩˧zo\#˥}}}}\hspace{0.5cm}[\kern2pt{\textcolor{darkblue}{\textbf{\ipa{kʰv̩˧zo˧}}}}\kern2pt]} \hypertarget{k\string_hv\string_=\string_Mzo\#\string_T1}{}
\markboth{\textcolor{darkblue}{\textbf{\ipa{kʰv̩˧zo\#˥}}}}{}
\textcolor{teal}{\mytextsc{noun}} \hspace{4pt} Tone: \#H.
\textcolor{Sepia}{\selectlanguage{english}Male dog. The term is also used as a temporary name for little boys, during the first months of their life, before they are given a real name. This ugly term is intended to disgust evil spirits, which will therefore turn their attention away from the infant. (In the early 21st century, the registry office requires a name to be given at birth; but this name only begins to be used by the family after the first months of life have elapsed.).} \zh{公狗(给刚出生的男孩子的名字,让鬼对他不感兴趣,不过来害小孩)。}  ¶ \textcolor{darkblue}{\textbf{\ipa{kʰv̩˧zo˧ ʈʂʰɯ˧-ɭɯ\#˥}}} \textcolor{Sepia}{\selectlanguage{english}\mytextsc{n}+\mytextsc{dem}+\mytextsc{clf}} \zh{这只公狗}  
 ¶ \textcolor{darkblue}{\textbf{\ipa{kʰv̩˧zo˧ tʰv̩˧-mi˧˥}}} \textcolor{Sepia}{\selectlanguage{english}\mytextsc{n}+\mytextsc{dem}+\mytextsc{clf}} \zh{这只公狗}  
 ¶ \textcolor{darkblue}{\textbf{\ipa{kʰv̩˧zo˧ tʰv̩˧-v̩\#˥}}} \textcolor{Sepia}{\selectlanguage{english}\mytextsc{n}+\mytextsc{dem}+\mytextsc{clf}} \zh{这只公狗}  
 ¶ \textcolor{darkblue}{\textbf{\ipa{kʰv̩˧zo˥-kʰv̩˩mv̩˩}}} \textcolor{Sepia}{\selectlanguage{english}puppy and she-dog} \zh{小狗与母狗}  
 \zh{量词}: \textcolor{darkblue}{\textbf{\ipa{v̩˧ / mi˩ / ɭɯ˧}}}  \mytextsc{clf}: \textcolor{darkblue}{\textbf{\ipa{v̩˧ / mi˩ / ɭɯ˧}}} 
\lhead{\firstmark}
\rhead{\botmark}

\subsection{\hspace{-0.5cm} {\Large \textcolor{darkblue}{\textbf{\ipa{kʰwæ˧ɻæ\#˥}}}}\hspace{0.5cm}[\kern2pt{\textcolor{darkblue}{\textbf{\ipa{kʰwæ˧ɻæ˧}}}}\kern2pt]} \hypertarget{k\string_hw\{\string_Mr£`\{\#\string_T1}{}
\markboth{\textcolor{darkblue}{\textbf{\ipa{kʰwæ˧ɻæ\#˥}}}}{}
\textcolor{teal}{\mytextsc{noun}} \hspace{4pt} Tone: \#H.
\textcolor{Sepia}{\selectlanguage{english}Felt; extended use: mat (even if not made of felt), cushion….} \zh{毡子。也用来指席子,垫子等。}  ¶ \textcolor{darkblue}{\textbf{\ipa{kʰwæ˧ɻæ˧ tʰi˧-kʰo˥}}} \textcolor{Sepia}{\selectlanguage{english}to spread a mat} \zh{铺席子}  
 \zh{量词}: \textcolor{darkblue}{\textbf{\ipa{tsʰi˥}}}  \mytextsc{clf}: \textcolor{darkblue}{\textbf{\ipa{tsʰi˥}}} 
\lhead{\firstmark}
\rhead{\botmark}

\subsection{\hspace{-0.5cm} {\Large \textcolor{darkblue}{\textbf{\ipa{kʰwɤ˥\textsubscript{a}}}}}\hspace{0.5cm}[\kern2pt{\textcolor{darkblue}{\textbf{\ipa{kʰwɤ˥}}}}\kern2pt]} \hypertarget{k\string_hw7\string_Ta1}{}
\markboth{\textcolor{darkblue}{\textbf{\ipa{kʰwɤ˥\textsubscript{a}}}}}{}
\textcolor{teal}{\mytextsc{classifier}} \hspace{4pt} Tone: H\textsubscript{a}.
\textcolor{Sepia}{\selectlanguage{english}A piece of, a chunk of; a mouthful of.} \zh{量词:块。一块肉、一口饭。}  ¶ \textcolor{darkblue}{\textbf{\ipa{ɖɯ˧-kʰwɤ˥\textasciitilde{}ɖɯ˩-kʰwɤ˩}}} \textcolor{Sepia}{\selectlanguage{english}chunk by chunk, one chunk after the other} \zh{一块一块地}  
 ¶ \textcolor{darkblue}{\textbf{\ipa{kʰwɤ˧ | ɖɯ˧-ʂe˧-ɻ̍˩!}}} \textcolor{Sepia}{\selectlanguage{english}Go ahead and decide! / Please make a decision!} \zh{你们得要做出决定!}  
 ¶ \textcolor{darkblue}{\textbf{\ipa{ɖɯ˧-kʰwɤ˧ so˧˥, | ɖɯ˧-kʰwɤ˥ fv̩˩!}}} \textcolor{Sepia}{\selectlanguage{english}Each new word is a new joy! (A comment by the consultant about the investigator's enjoyment of fieldwork. She takes a look at a draft dictionary, and comments that it represents a great deal of work, and that what matters is that the investigator should feel an interest in it, considering each new 'piece' – each addition to the dictionary – as a source of joy.)} \zh{学一点,就高兴一点!(评说语言调查工作:合作人看着本词典的初稿,说:这是一项很大的工程,关键的是调查者要有兴趣,欣赏每个新学的语言信息。)}  

\lhead{\firstmark}
\rhead{\botmark}

\subsection{\hspace{-0.5cm} {\Large \textcolor{darkblue}{\textbf{\ipa{kʰwɤ˧pʰv̩˧}}}}\hspace{0.5cm}[\kern2pt{\textcolor{darkblue}{\textbf{\ipa{kʰwɤ˧pʰv̩˧}}}}\kern2pt]} \hypertarget{k\string_hw7\string_Mp\string_hv\string_=\string_M1}{}
\markboth{\textcolor{darkblue}{\textbf{\ipa{kʰwɤ˧pʰv̩˧}}}}{}
\textcolor{teal}{\mytextsc{noun}} \hspace{4pt} Tone: M.
\textcolor{Sepia}{\selectlanguage{english}Meadow.} \zh{草坪、草地。} 
\lhead{\firstmark}
\rhead{\botmark}

\subsection{\hspace{-0.5cm} {\Large \textcolor{darkblue}{\textbf{\ipa{kʰwɤ˧pʰv̩˧-mo˧˥}}}}\hspace{0.5cm}[\kern2pt{\textcolor{darkblue}{\textbf{\ipa{xxxx non-correspondance entre le nombre de morphèmes et le nombre de tons de morphèmes}}}}\kern2pt]} \hypertarget{k\string_hw7\string_Mp\string_hv\string_=\string_M-mo\string_M\string_T1}{}
\markboth{\textcolor{darkblue}{\textbf{\ipa{kʰwɤ˧pʰv̩˧-mo˧˥}}}}{}
\textcolor{teal}{\mytextsc{noun}} \hspace{4pt} Tone: MH\#.
\textcolor{Sepia}{\selectlanguage{english}Meadow mushroom: a sort of edible mushroom that grows on meadows (not yet identified; perhaps \textit{Agaricus campestris}).} \zh{可以吃的一种菌子:可能是四孢蘑菇。直译:“草坪菌”。} 
\lhead{\firstmark}
\rhead{\botmark}

\newpage
\section*{\centering- \textcolor{darkblue}{\textbf{\ipa{l}}} \textcolor{darkblue}{\textbf{\ipa{ɭ}}} -}
\subsection{\hspace{-0.5cm} {\Large \textcolor{darkblue}{\textbf{\ipa{‑lɑ˧}}} \textsubscript{1}}\hspace{0.5cm}[\kern2pt{\textcolor{darkblue}{\textbf{\ipa{xxxx groupe tonal entier sans aucun ton}}}}\kern2pt]} \hypertarget{‑lA\string_M1}{}
\markboth{\textcolor{darkblue}{\textbf{\ipa{‑lɑ˧}}} \textsubscript{1}}{}
\textcolor{teal}{\mytextsc{adverb(ial)}} \hspace{4pt} Tone: 0.
\textcolor{Sepia}{\selectlanguage{english}Only.} \zh{只,才。}  ¶ \textcolor{darkblue}{\textbf{\ipa{ʈʂʰɯ˧-lɑ˩ ɲi˩-ze˩-mæ˩!}}} \textcolor{Sepia}{\selectlanguage{english}That's all!} \zh{就这些了! / 就这些而已! / 就这样!}  

\lhead{\firstmark}
\rhead{\botmark}

\subsection{\hspace{-0.5cm} {\Large \textcolor{darkblue}{\textbf{\ipa{‑lɑ˧}}} \textsubscript{2}}\hspace{0.5cm}[\kern2pt{\textcolor{darkblue}{\textbf{\ipa{xxxx groupe tonal entier sans aucun ton}}}}\kern2pt]} \hypertarget{‑lA\string_M2}{}
\markboth{\textcolor{darkblue}{\textbf{\ipa{‑lɑ˧}}} \textsubscript{2}}{}
\textcolor{teal}{\mytextsc{adverb(ial)}} \hspace{4pt} Tone: 0.
\textcolor{Sepia}{\selectlanguage{english}Too, also, and.} \zh{和、与、跟。}  ¶ \textcolor{darkblue}{\textbf{\ipa{ɖɯ˧-kʰv̩˧-lɑ˥ | so˩-ɬi˩˥}}} \textcolor{Sepia}{\selectlanguage{english}one year and three months (context: indicating the age of an infant)} \zh{一岁三个月}  
 ¶ \textcolor{darkblue}{\textbf{\ipa{ʈʂʰɯ˧-lɑ˧ | mɤ˧-bi˧, | njɤ˧-lɑ˧ mɤ˧-bi˧!}}} \textcolor{Sepia}{\selectlanguage{english}(If) (s)he does not go, I'm not going either!} \zh{他不去(的话),我也不去!}  
 ¶ \textcolor{darkblue}{\textbf{\ipa{hĩ˧-lɑ˩ | dʑɤ˧˥, | mv̩˧di˧-lɑ˥ | dʑɤ˧˥! / hĩ˧-lɑ˩ | dʑɤ˧˥, | lv̩˧-lɑ˧ | dʑɤ˧˥!}}} \textcolor{Sepia}{\selectlanguage{english}The people are good; and the land is good! / The people are good; and the fields are good! (A set phrase to recommend a family which a young woman is considering joining through marriage: the people are good, and their land is good.)} \zh{人也好,田也好!(习语:将女孩嫁出去前,一家人打听对方家如何,推荐的人保证:“他们家,人也好,田也好!”)}  
 ¶ \textcolor{darkblue}{\textbf{\ipa{hĩ˧ F | dʑɤ˧˥, | mv̩˧di˧˥ F | dʑɤ˧˥!}}} \textcolor{Sepia}{\selectlanguage{english}as above} \zh{同上}  
 ¶ \textcolor{darkblue}{\textbf{\ipa{mɤ˧-lɑ˧ dʑɤ˧˥!}}} \textcolor{Sepia}{\selectlanguage{english}The grease too is good! (Elicited variant on the preceding examples)} \zh{猪油也好!(按照上面例子的变体)}  
 ¶ \textcolor{darkblue}{\textbf{\ipa{qæ˩-lɑ˥ | dʑɤ˧˥!}}} \textcolor{Sepia}{\selectlanguage{english}The oil too is good! (Elicited variant on the preceding examples)} \zh{油也好!}  
 ¶ \textcolor{darkblue}{\textbf{\ipa{ʈʂʰɯ˧-lɑ˧ | mɤ˧-bi˧, | njɤ˧ | mɤ˧-bi˧-ze˧! / ʈʰɯ˧ mɤ˧-bi˧-ze˧-dʑo˧, | njɤ˧-lɑ˧ | mɤ˧-bi˧-ze˧!}}} \textcolor{Sepia}{\selectlanguage{english}If he doesn't go, I'm not going either!} \zh{他如果不去,我也不去!}  

\lhead{\firstmark}
\rhead{\botmark}

\subsection{\hspace{-0.5cm} {\Large \textcolor{darkblue}{\textbf{\ipa{lɑ˧}}}}\hspace{0.5cm}[\kern2pt{\textcolor{darkblue}{\textbf{\ipa{lɑ˥}}}}\kern2pt]} \hypertarget{lA\string_M1}{}
\markboth{\textcolor{darkblue}{\textbf{\ipa{lɑ˧}}}}{}
\textcolor{teal}{\mytextsc{noun}} \hspace{4pt} Tone: M.
\textcolor{Sepia}{\selectlanguage{english}Tiger.} \zh{老虎。}  \zh{量词}: \textcolor{darkblue}{\textbf{\ipa{pʰo˧˥}}}  \mytextsc{clf}: \textcolor{darkblue}{\textbf{\ipa{pʰo˧˥}}} 
\lhead{\firstmark}
\rhead{\botmark}

\subsection{\hspace{-0.5cm} {\Large \textcolor{darkblue}{\textbf{\ipa{lɑ˧bi\#˥}}}}\hspace{0.5cm}[\kern2pt{\textcolor{darkblue}{\textbf{\ipa{lɑ˧bi˧}}}}\kern2pt]} \hypertarget{lA\string_Mbi\#\string_T1}{}
\markboth{\textcolor{darkblue}{\textbf{\ipa{lɑ˧bi\#˥}}}}{}
\textcolor{teal}{\mytextsc{noun}} \hspace{4pt} Tone: \#H.
\textcolor{Sepia}{\selectlanguage{english}Steep slope.} \zh{陡坡、土坡、斜坡。}  ¶ \textcolor{darkblue}{\textbf{\ipa{lɑ˧bi˧-tsɤ˧}}} \textcolor{Sepia}{\selectlanguage{english}steep (literally 'like a steep slope')} \zh{‘像陡坡’,等于:很陡}  
 ¶ \textcolor{darkblue}{\textbf{\ipa{lɑ˧bi˧-tsɤ˧ | ʐwæ˩˥!}}} \textcolor{Sepia}{\selectlanguage{english}It is really steep!} \zh{陡得很!}  

\lhead{\firstmark}
\rhead{\botmark}

\subsection{\hspace{-0.5cm} {\Large \textcolor{darkblue}{\textbf{\ipa{lɑ˧do\#˥}}}}\hspace{0.5cm}[\kern2pt{\textcolor{darkblue}{\textbf{\ipa{lɑ˧do˧}}}}\kern2pt]} \hypertarget{lA\string_Mdo\#\string_T1}{}
\markboth{\textcolor{darkblue}{\textbf{\ipa{lɑ˧do\#˥}}}}{}
\textcolor{teal}{\mytextsc{noun}} \hspace{4pt} Tone: \#H.
\textcolor{Sepia}{\selectlanguage{english}Horse groom.} \zh{马夫(参加马帮)。} 
\lhead{\firstmark}
\rhead{\botmark}

\subsection{\hspace{-0.5cm} {\Large \textcolor{darkblue}{\textbf{\ipa{lɑ˧hwɤ˩}}}}\hspace{0.5cm}[\kern2pt{\textcolor{darkblue}{\textbf{\ipa{lɑ˧hwɤ˩}}}}\kern2pt]} \hypertarget{lA\string_Mhw7\string_B1}{}
\markboth{\textcolor{darkblue}{\textbf{\ipa{lɑ˧hwɤ˩}}}}{}
\textcolor{teal}{\mytextsc{noun}} \hspace{4pt} Tone: L\#.
\textcolor{Sepia}{\selectlanguage{english}A Na village outside the Yongning plain, close to the Lake, not far from \textcolor{darkblue}{\textbf{\ipa{/lɑ˧tʰɑ˧-di˧˥/}}}.} \zh{村落名。} 
\lhead{\firstmark}
\rhead{\botmark}

\subsection{\hspace{-0.5cm} {\Large \textcolor{darkblue}{\textbf{\ipa{lɑ˧kɤ˩}}}}\hspace{0.5cm}[\kern2pt{\textcolor{darkblue}{\textbf{\ipa{lɑ˧kɤ˩}}}}\kern2pt]} \hypertarget{lA\string_Mk7\string_B1}{}
\markboth{\textcolor{darkblue}{\textbf{\ipa{lɑ˧kɤ˩}}}}{}
\textcolor{teal}{\mytextsc{noun}} \hspace{4pt} Tone: L\#.
\textcolor{Sepia}{\selectlanguage{english}Small jar used to preserve wine.} \zh{小坛子,用来存酒。}  \zh{量词}: \textcolor{darkblue}{\textbf{\ipa{ɭɯ˧}}}  \mytextsc{clf}: \textcolor{darkblue}{\textbf{\ipa{ɭɯ˧}}} 
\lhead{\firstmark}
\rhead{\botmark}

\subsection{\hspace{-0.5cm} {\Large \textcolor{darkblue}{\textbf{\ipa{lɑ˧kʰv̩˧˥}}}}\hspace{0.5cm}[\kern2pt{\textcolor{darkblue}{\textbf{\ipa{lɑ˧kʰv̩˧˥}}}}\kern2pt]} \hypertarget{lA\string_Mk\string_hv\string_=\string_M\string_T1}{}
\markboth{\textcolor{darkblue}{\textbf{\ipa{lɑ˧kʰv̩˧˥}}}}{}
\textcolor{teal}{\mytextsc{noun}} \hspace{4pt} Tone: MH\#.
\textcolor{Sepia}{\selectlanguage{english}Year of the Tiger.} \zh{虎年。} 
\lhead{\firstmark}
\rhead{\botmark}

\subsection{\hspace{-0.5cm} {\Large \textcolor{darkblue}{\textbf{\ipa{lɑ˧\textasciitilde{}lɑ˧}}}}\hspace{0.5cm}[\kern2pt{\textcolor{darkblue}{\textbf{\ipa{lɑ˧lɑ˧}}}}\kern2pt]} \hypertarget{lA\string_M~lA\string_M1}{}
\markboth{\textcolor{darkblue}{\textbf{\ipa{lɑ˧\textasciitilde{}lɑ˧}}}}{}
\textcolor{teal}{\mytextsc{adjective}} \hspace{4pt} Tone: M.
\textcolor{Sepia}{\selectlanguage{english}Flaccid, flabby.} \zh{松弛。} 
\lhead{\firstmark}
\rhead{\botmark}

\subsection{\hspace{-0.5cm} {\Large \textcolor{darkblue}{\textbf{\ipa{lɑ˧\textasciitilde{}lɑ˧\textsubscript{b}}}}}\hspace{0.5cm}[\kern2pt{\textcolor{darkblue}{\textbf{\ipa{lɑ˧lɑ˧}}}}\kern2pt]} \hypertarget{lA\string_M~lA\string_Mb1}{}
\markboth{\textcolor{darkblue}{\textbf{\ipa{lɑ˧\textasciitilde{}lɑ˧\textsubscript{b}}}}}{}
\textcolor{teal}{\mytextsc{verb}} \hspace{4pt} Tone: M\textsubscript{b}.
\textcolor{Sepia}{\selectlanguage{english}To dilute in water.} \zh{掺水。}  ¶ \textcolor{darkblue}{\textbf{\ipa{(dʑɯ˧-qo˧) le˧-lɑ˧\textasciitilde{}lɑ˧}}} \textcolor{Sepia}{\selectlanguage{english}to dilute in water} \zh{掺水}  

\lhead{\firstmark}
\rhead{\botmark}

\subsection{\hspace{-0.5cm} {\Large \textcolor{darkblue}{\textbf{\ipa{lɑ˧lo˧-ʁwɤ˥}}}}\hspace{0.5cm}[\kern2pt{\textcolor{darkblue}{\textbf{\ipa{xxxx non-correspondance entre le nombre de morphèmes et le nombre de tons de morphèmes}}}}\kern2pt]} \hypertarget{lA\string_Mlo\string_M-Rw7\string_T1}{}
\markboth{\textcolor{darkblue}{\textbf{\ipa{lɑ˧lo˧-ʁwɤ˥}}}}{}
\textcolor{teal}{\mytextsc{noun}} \hspace{4pt} Tone: H\#.
\textcolor{Sepia}{\selectlanguage{english}A village of Yongning; Chinese name: Laluowa.} \zh{拉洛瓦村(永宁的一个村落)。}  ¶ \textcolor{darkblue}{\textbf{\ipa{dʑɤ˩bv̩˧kɤ˧-sɑ˥ʁwɤ˩, | hi˩ʁwɤ˩-lo˥, | æ˩mi˧-ʁwɤ\#˥, | lɑ˧lo˧-ʁwɤ˥, | lɑ˧ŋwɤ˧, | bɤ˧tsʰo˧gv̩˥, | ə˧lɑ˧-ʁwɤ\#˥, | gæ˧ɻæ˩, | qʰæ˧tɕʰi˧, | tʰo˧ʈɯ\#˥}}} \textcolor{Sepia}{\selectlanguage{english}the ten villages traditionally considered as part of Yongning} \zh{摩梭传统地理概念中,属于永宁的十个村落}  

\lhead{\firstmark}
\rhead{\botmark}

\subsection{\hspace{-0.5cm} {\Large \textcolor{darkblue}{\textbf{\ipa{lɑ˧ɬɑ˧˥}}}}\hspace{0.5cm}[\kern2pt{\textcolor{darkblue}{\textbf{\ipa{lɑ˧ɬɑ˧˥}}}}\kern2pt]} \hypertarget{lA\string_MKA\string_M\string_T1}{}
\markboth{\textcolor{darkblue}{\textbf{\ipa{lɑ˧ɬɑ˧˥}}}}{}
\textcolor{teal}{\mytextsc{conjunction}} \hspace{4pt} Tone: MH\#.
\textcolor{Sepia}{\selectlanguage{english}Apart from, aside of, other than.} \zh{这以外。}  ¶ \textcolor{darkblue}{\textbf{\ipa{tsɑ˧bɤ˧ mɤ˧-pʰv̩˧ɖɯ˧! | lɑ˧ɬɑ˧˥, | ə˧tso˧-mɤ˧-ɲi˩ | pʰv̩˩ɖɯ˩˥!}}} \textcolor{Sepia}{\selectlanguage{english}Flour is not expensive; apart from it, everything is expensive! / Flour is cheap; but everything else is expensive! (An observation about the cost of living in early 21st-century Yongning)} \zh{面粉不贵。其它的呢,什么都贵!(题目:讲今日永宁食品物价)}  

\lhead{\firstmark}
\rhead{\botmark}

\subsection{\hspace{-0.5cm} {\Large \textcolor{darkblue}{\textbf{\ipa{lɑ˧ɬɑ˧˥}}} \textsubscript{1}}\hspace{0.5cm}[\kern2pt{\textcolor{darkblue}{\textbf{\ipa{lɑ˧ɬɑ˧˥}}}}\kern2pt]} \hypertarget{lA\string_MKA\string_M\string_T1}{}
\markboth{\textcolor{darkblue}{\textbf{\ipa{lɑ˧ɬɑ˧˥}}} \textsubscript{1}}{}
\textcolor{teal}{\mytextsc{pronoun/pronominal}} \hspace{4pt} Tone: MH\#.
\textcolor{Sepia}{\selectlanguage{english}Other.} \zh{别的。}  ¶ \textcolor{darkblue}{\textbf{\ipa{lɑ˧ɬɑ˧˥ | ɖɯ˧-tɕi˥}}} \textcolor{Sepia}{\selectlanguage{english}some others, a few others} \zh{其它一些}  
 ¶ \textcolor{darkblue}{\textbf{\ipa{lɑ˧ɬɑ˧˥ | ʈʂʰɯ˧-tɕi˩}}} \textcolor{Sepia}{\selectlanguage{english}those other, those few others, the few that remained} \zh{其它的那些}  
 ¶ \textcolor{darkblue}{\textbf{\ipa{lɑ˧ɬɑ˧˥ | ɖɯ˧-ʁo˩ ɲi˩!}}} \textcolor{Sepia}{\selectlanguage{english}It's something different! / That's a different matter!} \zh{是另一回事! / 是另一码事!}  
\textit{See:} \hyperlink{}{\textcolor{darkblue}{\textbf{\ipa{lɑ˧ɬɑ˧˥}}} \textsubscript{2}} 
\lhead{\firstmark}
\rhead{\botmark}

\subsection{\hspace{-0.5cm} {\Large \textcolor{darkblue}{\textbf{\ipa{lɑ˧ɬɑ˧˥}}} \textsubscript{2}}\hspace{0.5cm}[\kern2pt{\textcolor{darkblue}{\textbf{\ipa{lɑ˧ɬɑ˧˥}}}}\kern2pt]} \hypertarget{lA\string_MKA\string_M\string_T2}{}
\markboth{\textcolor{darkblue}{\textbf{\ipa{lɑ˧ɬɑ˧˥}}} \textsubscript{2}}{}
\textcolor{teal}{\mytextsc{adjective}} \hspace{4pt} Tone: MH\#.
\textcolor{Sepia}{\selectlanguage{english}Other.} \zh{别的。}  ¶ \textcolor{darkblue}{\textbf{\ipa{lɑ˧ɬɑ˧ hĩ˥}}} \textcolor{Sepia}{\selectlanguage{english}other people} \zh{其它人}  
 ¶ \textcolor{darkblue}{\textbf{\ipa{ɖɯ˧-bæ˧ | le˧-se˩, | ɖɯ˧-bæ˧ ʝi˧! / ɖɯ˧-bæ˧ | le˧-se˩, | wɤ˩˥ | lɑ˧ɬɑ˧˥ | ɖɯ˧-bæ˧ ʝi˧! |}}} \textcolor{Sepia}{\selectlanguage{english}When one has finished one task, one moves on to another!} \zh{做完一件事情,就轮到另一个!}  
\textit{See:} \hyperlink{}{\textcolor{darkblue}{\textbf{\ipa{lɑ˧ɬɑ˧˥}}} \textsubscript{1}} 
\lhead{\firstmark}
\rhead{\botmark}

\subsection{\hspace{-0.5cm} {\Large \textcolor{darkblue}{\textbf{\ipa{lɑ˧mɑ˧}}}}\hspace{0.5cm}[\kern2pt{\textcolor{darkblue}{\textbf{\ipa{lɑ˧mɑ˧}}}}\kern2pt]} \hypertarget{lA\string_MmA\string_M1}{}
\markboth{\textcolor{darkblue}{\textbf{\ipa{lɑ˧mɑ˧}}}}{}
\textcolor{teal}{\mytextsc{noun}} \hspace{4pt} Tone: M.
\textcolor{Sepia}{\selectlanguage{english}Lama.} \zh{喇嘛。}  Borrowing: Tibetan  bla-ma
 ¶ \textcolor{darkblue}{\textbf{\ipa{hæ˧-lɑ˩mɑ˩}}} \textcolor{Sepia}{\selectlanguage{english}Chinese lama} \zh{汉族喇嘛}  
 \zh{量词}: \textcolor{darkblue}{\textbf{\ipa{v̩˧}}}  \mytextsc{clf}: \textcolor{darkblue}{\textbf{\ipa{v̩˧}}} 
\lhead{\firstmark}
\rhead{\botmark}

\subsection{\hspace{-0.5cm} {\Large \textcolor{darkblue}{\textbf{\ipa{lɑ˧mi\#˥}}}}\hspace{0.5cm}[\kern2pt{\textcolor{darkblue}{\textbf{\ipa{lɑ˧mi˧}}}}\kern2pt]} \hypertarget{lA\string_Mmi\#\string_T1}{}
\markboth{\textcolor{darkblue}{\textbf{\ipa{lɑ˧mi\#˥}}}}{}
\textcolor{teal}{\mytextsc{noun}} \hspace{4pt} Tone: \#H.
\textcolor{Sepia}{\selectlanguage{english}Female tiger.} \zh{母老虎。}  ¶ \textcolor{darkblue}{\textbf{\ipa{lɑ˧mi˧ tʰv̩˧-mi˧˥ / lɑ˧mi˧ tʰv̩˧-mi˥\#}}} \textcolor{Sepia}{\selectlanguage{english}\mytextsc{n}+\mytextsc{dem}+\mytextsc{clf}} \zh{那只老虎}  
 \zh{量词}: \textcolor{darkblue}{\textbf{\ipa{pʰo˧˥ / mi˩}}}  \mytextsc{clf}: \textcolor{darkblue}{\textbf{\ipa{pʰo˧˥ / mi˩}}} 
\lhead{\firstmark}
\rhead{\botmark}

\subsection{\hspace{-0.5cm} {\Large \textcolor{darkblue}{\textbf{\ipa{lɑ˧ŋwɤ˧}}}}\hspace{0.5cm}[\kern2pt{\textcolor{darkblue}{\textbf{\ipa{lɑ˧ŋwɤ˧}}}}\kern2pt]} \hypertarget{lA\string_MNw7\string_M1}{}
\markboth{\textcolor{darkblue}{\textbf{\ipa{lɑ˧ŋwɤ˧}}}}{}
\textcolor{teal}{\mytextsc{noun}} \hspace{4pt} Tone: M.
\textcolor{Sepia}{\selectlanguage{english}The name of a mountain on the way from Yongning to Wujiao; this name is also used to refer to the hamlets on the slope of this mountain.} \zh{一座山的名字。}  ¶ \textcolor{darkblue}{\textbf{\ipa{dʑɤ˩bv̩˧kɤ˧-sɑ˥ʁwɤ˩, | hi˩ʁwɤ˩-lo˥, | æ˩mi˧-ʁwɤ\#˥, | lɑ˧lo˧-ʁwɤ˥, | lɑ˧ŋwɤ˧, | bɤ˧tsʰo˧gv̩˥, | ə˧lɑ˧-ʁwɤ\#˥, | gæ˧ɻæ˩, | qʰæ˧tɕʰi˧, | tʰo˧ʈɯ\#˥}}} \textcolor{Sepia}{\selectlanguage{english}the ten villages traditionally considered as part of Yongning} \zh{摩梭传统地理概念中,属于永宁的十个村落}  

\lhead{\firstmark}
\rhead{\botmark}

\subsection{\hspace{-0.5cm} {\Large \textcolor{darkblue}{\textbf{\ipa{lɑ˧pʰv̩\#˥}}}}\hspace{0.5cm}[\kern2pt{\textcolor{darkblue}{\textbf{\ipa{lɑ˧pʰv̩˧}}}}\kern2pt]} \hypertarget{lA\string_Mp\string_hv\string_=\#\string_T1}{}
\markboth{\textcolor{darkblue}{\textbf{\ipa{lɑ˧pʰv̩\#˥}}}}{}
\textcolor{teal}{\mytextsc{noun}} \hspace{4pt} Tone: \#H.
\textcolor{Sepia}{\selectlanguage{english}Male tiger.} \zh{公老虎。}  ¶ \textcolor{darkblue}{\textbf{\ipa{lɑ˧pʰv̩˧ tʰv̩˧-ɭɯ\#˥}}} \textcolor{Sepia}{\selectlanguage{english}\mytextsc{n}+\mytextsc{dem}+\mytextsc{clf}} \zh{那只老虎}  
 \zh{量词}: \textcolor{darkblue}{\textbf{\ipa{pʰo˧˥ / ɭɯ˧}}}  \mytextsc{clf}: \textcolor{darkblue}{\textbf{\ipa{pʰo˧˥ / ɭɯ˧}}} 
\lhead{\firstmark}
\rhead{\botmark}

\subsection{\hspace{-0.5cm} {\Large \textcolor{darkblue}{\textbf{\ipa{lɑ˧tʰɑ˧-di˧˥}}}}\hspace{0.5cm}[\kern2pt{\textcolor{darkblue}{\textbf{\ipa{xxxx non-correspondance entre le nombre de morphèmes et le nombre de tons de morphèmes}}}}\kern2pt]} \hypertarget{lA\string_Mt\string_hA\string_M-di\string_M\string_T1}{}
\markboth{\textcolor{darkblue}{\textbf{\ipa{lɑ˧tʰɑ˧-di˧˥}}}}{}
\textcolor{teal}{\mytextsc{noun}} \hspace{4pt} Tone: MH\#.
\textcolor{Sepia}{\selectlanguage{english}The entire Na area around Lugu lake, including Zuosuo (currently Luguhu Zhen) and the village of Luoshui.} \zh{泸沽湖周边的摩梭地区,包括左所(今为泸沽湖镇)、洛水村等。}  ¶ \textcolor{darkblue}{\textbf{\ipa{ɬi˧ki˧, | ɲi˧se˩, | tɑ˧dzi˩, | mv̩˧qʰwæ˩, | lɑ˧tʰɑ˧-di˧˥}}} \textcolor{Sepia}{\selectlanguage{english}Villages that one passes when moving away from the Yongning plain, towards Lugu lake. These villages do not count as part of Yongning proper. The last, \textcolor{darkblue}{\textbf{\ipa{/lɑ˧tʰɑ˧-di˧˥/}}}, is not a village name like the preceding four: it refers to the entire Na area beyond the fourth village.} \zh{从永宁往泸沽湖所经过的村落,依次是:里格、尼赛、大祖、木垮,然后到拉塔地(拉塔地指的是泸沽湖周边的摩梭地区,包括左所、洛水村等)}  

\lhead{\firstmark}
\rhead{\botmark}

\subsection{\hspace{-0.5cm} {\Large \textcolor{darkblue}{\textbf{\ipa{lɑ˧tʰɑ˧mi˥\$}}}}\hspace{0.5cm}[\kern2pt{\textcolor{darkblue}{\textbf{\ipa{lɑ˧tʰɑ˧mi˥}}}}\kern2pt]} \hypertarget{lA\string_Mt\string_hA\string_Mmi\string_T\$1}{}
\markboth{\textcolor{darkblue}{\textbf{\ipa{lɑ˧tʰɑ˧mi˥\$}}}}{}
\textcolor{teal}{\mytextsc{noun}} \hspace{4pt} Tone: H\$.
\textcolor{Sepia}{\selectlanguage{english}A family name from Yongning. There are five families in Yongning that carry this name. This is one of the first three clans who settled in the vicinity of the Yongning monastery, the other two being \textcolor{darkblue}{\textbf{\ipa{/kɤ˧˥tʰɑ˩/}}} and \textcolor{darkblue}{\textbf{\ipa{/ə˧lɑ˧/}}}.} \zh{一个姓。这个姓,永宁有五个家。音译:拉他咪。}  ¶ \textcolor{darkblue}{\textbf{\ipa{lɑ˧tʰɑ˧mi˧=ɻ̍˥\$}}} \textcolor{Sepia}{\selectlanguage{english}the \textcolor{darkblue}{\textbf{\ipa{/lɑ˧tʰɑ˧mi˥\$/}}} clan, the \textcolor{darkblue}{\textbf{\ipa{/lɑ˧tʰɑ˧mi˥\$/}}} family} \zh{\textcolor{darkblue}{\textbf{\ipa{/lɑ˧tʰɑ˧mi˥\$/}}}家族}  

\lhead{\firstmark}
\rhead{\botmark}

\subsection{\hspace{-0.5cm} {\Large \textcolor{darkblue}{\textbf{\ipa{lɑ˧tʰɑ˧mi˥-ʈæ˧ʂɯ˧-lɑ˩mv̩˩}}}}\hspace{0.5cm}[\kern2pt{\textcolor{darkblue}{\textbf{\ipa{lɑ˧tʰɑ˧mi˥ʈæ˩ʂɯ˩lɑ˩mv̩˩}}}}\kern2pt]} \hypertarget{lA\string_Mt\string_hA\string_Mmi\string_T-t`\{\string_Ms`M\string_M-lA\string_Bmv\string_=\string_B1}{}
\markboth{\textcolor{darkblue}{\textbf{\ipa{lɑ˧tʰɑ˧mi˥-ʈæ˧ʂɯ˧-lɑ˩mv̩˩}}}}{}
\textcolor{teal}{\mytextsc{noun}} \hspace{4pt} Tone: H\#-M-L.
\textcolor{Sepia}{\selectlanguage{english}Proper name of the main consultant (reference speaker) for this volume (speaker code: F4).} \zh{拉他咪•达石拉么:本著作的标准发音合作人。} 
\lhead{\firstmark}
\rhead{\botmark}

\subsection{\hspace{-0.5cm} {\Large \textcolor{darkblue}{\textbf{\ipa{lɑ˧zi˥}}}}\hspace{0.5cm}[\kern2pt{\textcolor{darkblue}{\textbf{\ipa{lɑ˧zi˥}}}}\kern2pt]} \hypertarget{lA\string_Mzi\string_T1}{}
\markboth{\textcolor{darkblue}{\textbf{\ipa{lɑ˧zi˥}}}}{}
\textcolor{teal}{\mytextsc{noun}} \hspace{4pt} Tone: H\#.
\textcolor{Sepia}{\selectlanguage{english}Painter.} \zh{画家。}  ¶ \textcolor{darkblue}{\textbf{\ipa{ʈʂʰɯ˧-v̩˧, | lɑ˧zi˥ ɲi˩!}}} \textcolor{Sepia}{\selectlanguage{english}(S)he is a painter! / (S)he can paint!} \zh{他是画家!}  

\lhead{\firstmark}
\rhead{\botmark}

\subsection{\hspace{-0.5cm} {\Large \textcolor{darkblue}{\textbf{\ipa{lɑ˧zo\#˥}}}}\hspace{0.5cm}[\kern2pt{\textcolor{darkblue}{\textbf{\ipa{lɑ˧zo˧}}}}\kern2pt]} \hypertarget{lA\string_Mzo\#\string_T1}{}
\markboth{\textcolor{darkblue}{\textbf{\ipa{lɑ˧zo\#˥}}}}{}
\textcolor{teal}{\mytextsc{noun}} \hspace{4pt} Tone: \#H.
\textcolor{Sepia}{\selectlanguage{english}Baby tiger, child of tiger.} \zh{小老虎。}  ¶ \textcolor{darkblue}{\textbf{\ipa{lɑ˧zo˧ tʰv̩˧-ɭɯ\#˥}}} \textcolor{Sepia}{\selectlanguage{english}\mytextsc{n}+\mytextsc{dem}+\mytextsc{clf}} \zh{那只小老虎}  
 \zh{量词}: \textcolor{darkblue}{\textbf{\ipa{ɭɯ˧}}}  \mytextsc{clf}: \textcolor{darkblue}{\textbf{\ipa{ɭɯ˧}}} 
\lhead{\firstmark}
\rhead{\botmark}

\subsection{\hspace{-0.5cm} {\Large \textcolor{darkblue}{\textbf{\ipa{lɑ˩gv̩˧}}}}\hspace{0.5cm}[\kern2pt{\textcolor{darkblue}{\textbf{\ipa{lɑ˩gv̩˥}}}}\kern2pt]} \hypertarget{lA\string_Bgv\string_=\string_M1}{}
\markboth{\textcolor{darkblue}{\textbf{\ipa{lɑ˩gv̩˧}}}}{}
\textcolor{teal}{\mytextsc{adjective}} \hspace{4pt} Tone: LM.
\textcolor{Sepia}{\selectlanguage{english}Curved, crooked, bent (e.g. tree).} \zh{弯(树...)。}  ¶ \textcolor{darkblue}{\textbf{\ipa{si˧dzi˩ | lɑ˩-gv̩˧-ze˩}}} \textcolor{Sepia}{\selectlanguage{english}The tree got crooked.} \zh{树弯了。}  

\lhead{\firstmark}
\rhead{\botmark}

\subsection{\hspace{-0.5cm} {\Large \textcolor{darkblue}{\textbf{\ipa{lɑ˩gv̩˧-lɑ˩ɲi˩}}}}\hspace{0.5cm}[\kern2pt{\textcolor{darkblue}{\textbf{\ipa{lɑ˩gv̩˧lɑ˩ɲi˩}}}}\kern2pt]} \hypertarget{lA\string_Bgv\string_=\string_M-lA\string_BJi\string_B1}{}
\markboth{\textcolor{darkblue}{\textbf{\ipa{lɑ˩gv̩˧-lɑ˩ɲi˩}}}}{}
\textcolor{teal}{\mytextsc{adjective}} \hspace{4pt} Tone: LM-L.
\textit{From:} \textbf{lɑ˩gv̩˧} \textcolor{Sepia}{\selectlanguage{english}Crooked, curved, bent (e.g. road, person's limbs).} \zh{弯(路,植物,人的四肢)。} 
\lhead{\firstmark}
\rhead{\botmark}

\subsection{\hspace{-0.5cm} {\Large \textcolor{darkblue}{\textbf{\ipa{lɑ˩jɤ˧-ɬi˧}}}}\hspace{0.5cm}[\kern2pt{\textcolor{darkblue}{\textbf{\ipa{lɑ˩jɤ˧ɬi˧}}}}\kern2pt]} \hypertarget{lA\string_Bj7\string_M-Ki\string_M1}{}
\markboth{\textcolor{darkblue}{\textbf{\ipa{lɑ˩jɤ˧-ɬi˧}}}}{}
\textcolor{teal}{\mytextsc{noun}} \hspace{4pt} Tone: LM-.
\textcolor{Sepia}{\selectlanguage{english}12th month.} \zh{十二月。} 
\lhead{\firstmark}
\rhead{\botmark}

\subsection{\hspace{-0.5cm} {\Large \textcolor{darkblue}{\textbf{\ipa{lɑ˩\textasciitilde{}lɑ˧˥}}}}\hspace{0.5cm}[\kern2pt{\textcolor{darkblue}{\textbf{\ipa{lɑ˧lɑ˧˥}}}}\kern2pt]} \hypertarget{lA\string_B~lA\string_M\string_T1}{}
\markboth{\textcolor{darkblue}{\textbf{\ipa{lɑ˩\textasciitilde{}lɑ˧˥}}}}{}
\textcolor{teal}{\mytextsc{verb}} \hspace{4pt} Tone: MH.
\textcolor{Sepia}{\selectlanguage{english}To fight, to scuffle, to come to blows.} \zh{打架、吵架。}  ¶ \textcolor{darkblue}{\textbf{\ipa{lɑ˩lɑ˧-hĩ˥ | ʈʂʰɯ˧-tɕi˩}}} \textcolor{Sepia}{\selectlanguage{english}those people who are fighting} \zh{打架的这些(人)}  
\textit{See:} \textcolor{darkblue}{\textbf{\ipa{lɑ˧˥}}} 
\lhead{\firstmark}
\rhead{\botmark}

\subsection{\hspace{-0.5cm} {\Large \textcolor{darkblue}{\textbf{\ipa{lɑ˩mɑ˩}}}}\hspace{0.5cm}[\kern2pt{\textcolor{darkblue}{\textbf{\ipa{lɑ˩mɑ˩˥}}}}\kern2pt]} \hypertarget{lA\string_BmA\string_B1}{}
\markboth{\textcolor{darkblue}{\textbf{\ipa{lɑ˩mɑ˩}}}}{}
\textcolor{teal}{\mytextsc{noun}} \hspace{4pt} Tone: L.
\textcolor{Sepia}{\selectlanguage{english}A family name from Yongning. There are four families in Yongning that carry this name.} \zh{一个姓。这个姓,永宁有四个家。}  ¶ \textcolor{darkblue}{\textbf{\ipa{lɑ˩mɑ˩=ɻ̍˥\$}}} \textcolor{Sepia}{\selectlanguage{english}the \textcolor{darkblue}{\textbf{\ipa{/lɑ˩mɑ˩/}}} clan, the \textcolor{darkblue}{\textbf{\ipa{/lɑ˩mɑ˩/}}} family} \zh{\textcolor{darkblue}{\textbf{\ipa{/lɑ˩mɑ˩/}}}家族}  
 ¶ \textcolor{darkblue}{\textbf{\ipa{lɑ˩mɑ˩-gv̩˥mɑ˩}}} \textcolor{Sepia}{\selectlanguage{english}the name of a person, containing both a family name: \textcolor{darkblue}{\textbf{\ipa{/lɑ˩mɑ˩/}}}, and a given name: \textcolor{darkblue}{\textbf{\ipa{/gv̩˧mɑ˧/}}}} \zh{一个人的名字:姓为\textcolor{darkblue}{\textbf{\ipa{/lɑ˩mɑ˩/}}},名为\textcolor{darkblue}{\textbf{\ipa{/gv̩˧mɑ˧/}}}}  

\lhead{\firstmark}
\rhead{\botmark}

\subsection{\hspace{-0.5cm} {\Large \textcolor{darkblue}{\textbf{\ipa{lɑ˩tɑ˧}}}}\hspace{0.5cm}[\kern2pt{\textcolor{darkblue}{\textbf{\ipa{lɑ˩tɑ˥}}}}\kern2pt]} \hypertarget{lA\string_BtA\string_M1}{}
\markboth{\textcolor{darkblue}{\textbf{\ipa{lɑ˩tɑ˧}}}}{}
\textcolor{teal}{\mytextsc{adjective}} \hspace{4pt} Tone: LM.
\textcolor{Sepia}{\selectlanguage{english}Askew, slanting (e.g. hat).} \zh{歪,偏 (帽子戴得歪)。} 
\lhead{\firstmark}
\rhead{\botmark}

\subsection{\hspace{-0.5cm} {\Large \textcolor{darkblue}{\textbf{\ipa{-lɑ˩tɑ˩}}}}\hspace{0.5cm}[\kern2pt{\textcolor{darkblue}{\textbf{\ipa{lɑ˩tɑ˩˥}}}}\kern2pt]} \hypertarget{-lA\string_BtA\string_B1}{}
\markboth{\textcolor{darkblue}{\textbf{\ipa{-lɑ˩tɑ˩}}}}{}
\textcolor{teal}{\mytextsc{postposition}} \hspace{4pt} Tone: L.
\textcolor{Sepia}{\selectlanguage{english}Close to.}  ¶ \textcolor{darkblue}{\textbf{\ipa{ɑ˩ʁo˧ | -lɑ˩tɑ˩˥}}} \textcolor{Sepia}{\selectlanguage{english}the perimeter of the house: the surface on which the house (farm) extends} \zh{家的面积}  

\lhead{\firstmark}
\rhead{\botmark}

\subsection{\hspace{-0.5cm} {\Large \textcolor{darkblue}{\textbf{\ipa{lɑ˩ʈʂv̩˩}}}}\hspace{0.5cm}[\kern2pt{\textcolor{darkblue}{\textbf{\ipa{lɑ˩ʈʂv̩˩˥}}}}\kern2pt]} \hypertarget{lA\string_Bt`s`v\string_=\string_B1}{}
\markboth{\textcolor{darkblue}{\textbf{\ipa{lɑ˩ʈʂv̩˩}}}}{}
\textcolor{teal}{\mytextsc{noun}} \hspace{4pt} Tone: L.
\textcolor{Sepia}{\selectlanguage{english}Candle.} \zh{蜡烛。}  Borrowing: Chinese  \zh{蜡烛}
 \zh{量词}: \textcolor{darkblue}{\textbf{\ipa{kɤ˧˥}}}  \mytextsc{clf}: \textcolor{darkblue}{\textbf{\ipa{kɤ˧˥}}} 
\lhead{\firstmark}
\rhead{\botmark}

\subsection{\hspace{-0.5cm} {\Large \textcolor{darkblue}{\textbf{\ipa{lɑ˧˥}}} \textsubscript{1}}\hspace{0.5cm}[\kern2pt{\textcolor{darkblue}{\textbf{\ipa{lɑ˧˥}}}}\kern2pt]} \hypertarget{lA\string_M\string_T1}{}
\markboth{\textcolor{darkblue}{\textbf{\ipa{lɑ˧˥}}} \textsubscript{1}}{}
\textcolor{teal}{\mytextsc{verb}} \hspace{4pt} Tone: MH.
\textcolor{Sepia}{\selectlanguage{english}To strike someone, to beat someone.} \zh{打(打人,钉钉子……)。}  ¶ \textcolor{darkblue}{\textbf{\ipa{hĩ˧ lɑ˩}}} \textcolor{Sepia}{\selectlanguage{english}to strike someone} \zh{打人}  
 ¶ \textcolor{darkblue}{\textbf{\ipa{hɑ˧ lɑ˩}}} \textcolor{Sepia}{\selectlanguage{english}to beat the grain} \zh{打粮食}  
 ¶ \textcolor{darkblue}{\textbf{\ipa{nv̩˩ɭɯ˧ lɑ˧}}} \textcolor{Sepia}{\selectlanguage{english}to beat soy} \zh{打大豆}  
 ¶ \textcolor{darkblue}{\textbf{\ipa{sɯ˩tʰi˩-po˥-ɳɯ˩ | lɑ˧˥}}} \textcolor{Sepia}{\selectlanguage{english}to break with a knife (brick tea: compressed tea leaves)} \zh{用刀子来砍(沱茶、茶饼)}  
 ¶ \textcolor{darkblue}{\textbf{\ipa{ə˧ʝi˧-ʂɯ˥ʝi˩, | ɬi˧di˩-dʑo˩, | æ˧ lɑ˩-hĩ˩ F | dʑo˩˥! | ʂe˧ lɑ˧-hĩ˥ F | dʑo˩˥! | hæ̃˩ lɑ˩-hĩ˥ F | dʑo˩˥! | ŋv̩˩ lɑ˩-hĩ˥ F | dʑo˩˥!}}} \textcolor{Sepia}{\selectlanguage{english}In the past, in Yongning, there were craftsmen who forged copper! craftsmen who forged iron! craftsmen who forged gold! and craftsmen who forged silver!} \zh{过去,在永宁,有铜匠、铁匠、金匠、银匠。}  
 ¶ \textcolor{darkblue}{\textbf{\ipa{ə˧ʝi˧-ʂɯ˥ʝi˩, | ɬi˧di˩-dʑo˩, | æ˧ lɑ˩-hĩ˩ dʑo˩, | ʂe˧ lɑ˧-hĩ˥ dʑo˩, | hæ̃˩ lɑ˩-hĩ˥ dʑo˩, | ŋv̩˩ lɑ˩-hĩ˥ dʑo˩.}}} \textcolor{Sepia}{\selectlanguage{english}In the past, in Yongning, there were craftsmen who forged copper; craftsmen who forged iron; craftsmen who forged gold; and craftsmen who forged silver.} \zh{过去,在永宁,有铜匠、铁匠、金匠、银匠。}  
\textit{See:} \hyperlink{}{\textcolor{darkblue}{\textbf{\ipa{lɑ˩\textasciitilde{}lɑ˧˥}}}} 
\lhead{\firstmark}
\rhead{\botmark}

\subsection{\hspace{-0.5cm} {\Large \textcolor{darkblue}{\textbf{\ipa{lɑ˧˥}}} \textsubscript{2}}\hspace{0.5cm}[\kern2pt{\textcolor{darkblue}{\textbf{\ipa{lɑ˧˥}}}}\kern2pt]} \hypertarget{lA\string_M\string_T2}{}
\markboth{\textcolor{darkblue}{\textbf{\ipa{lɑ˧˥}}} \textsubscript{2}}{}
\textcolor{teal}{\mytextsc{verb}} \hspace{4pt} Tone: MH.
\textcolor{Sepia}{\selectlanguage{english}To form, to be there, to have appeared (dew).} \zh{有,结(露水)。}  ¶ \textcolor{darkblue}{\textbf{\ipa{ɖʐv̩˧ lɑ˧˥}}} \textcolor{Sepia}{\selectlanguage{english}Some dew has appeared; there is some dew} \zh{结露水了。}  
 ¶ \textcolor{darkblue}{\textbf{\ipa{ɖʐv̩˧qʰɑ˧ lɑ˧˥}}} \textcolor{Sepia}{\selectlanguage{english}Some dew has appeared; there is some dew} \zh{结露水了。}  

\lhead{\firstmark}
\rhead{\botmark}

\subsection{\hspace{-0.5cm} {\Large \textcolor{darkblue}{\textbf{\ipa{‑læ˧}}}}\hspace{0.5cm}[\kern2pt{\textcolor{darkblue}{\textbf{\ipa{læ˥}}}}\kern2pt]} \hypertarget{‑l\{\string_M1}{}
\markboth{\textcolor{darkblue}{\textbf{\ipa{‑læ˧}}}}{}
\textcolor{teal}{\mytextsc{suffix}} \hspace{4pt} Tone: M.
\textcolor{Sepia}{\selectlanguage{english}This \mytextsc{top} marker introduces a new element, without necessarily contrasting it with others. Possible gloss: concerning… .} \zh{\mytextsc{主题:……的话、关于……。}}  ¶ \textcolor{darkblue}{\textbf{\ipa{ɖɯ˩mɑ˧ | -læ˧…}}} \textcolor{Sepia}{\selectlanguage{english}Concerning (my granddaughter) \textcolor{darkblue}{\textbf{\ipa{ɖɯ˩mɑ˧}}}, …} \zh{关于独妈呢,……}  
 ¶ \textcolor{darkblue}{\textbf{\ipa{lɑ˩mv̩˩˥ | -læ˧...}}} \textcolor{Sepia}{\selectlanguage{english}Concerning \textcolor{darkblue}{\textbf{\ipa{lɑ˩mv̩˩˥}}} [a given name], ...} \zh{关于拉姆呢,……}  
 ¶ \textcolor{darkblue}{\textbf{\ipa{ti˧ɖo˥ | -læ˧…}}} \textcolor{Sepia}{\selectlanguage{english}Concerning \textcolor{darkblue}{\textbf{\ipa{ti˧ɖo˥}}} [a given name], …} \zh{关于\textcolor{darkblue}{\textbf{\ipa{ti˧ɖo˥}}}[人的名字]呢,……}  

\lhead{\firstmark}
\rhead{\botmark}

\subsection{\hspace{-0.5cm} {\Large \textcolor{darkblue}{\textbf{\ipa{læ˧dæ˧qæ˥}}}}\hspace{0.5cm}[\kern2pt{\textcolor{darkblue}{\textbf{\ipa{læ˧dæ˧qæ˥}}}}\kern2pt]} \hypertarget{l\{\string_Md\{\string_Mq\{\string_T1}{}
\markboth{\textcolor{darkblue}{\textbf{\ipa{læ˧dæ˧qæ˥}}}}{}
\textcolor{teal}{\mytextsc{noun}} \hspace{4pt} Tone: H\#.
\textcolor{Sepia}{\selectlanguage{english}Armpit.} \zh{腋下。} Local Chinese dialect:\zh{膈肢窝。} \zh{量词}: \textcolor{darkblue}{\textbf{\ipa{ɭɯ˧}}}  \mytextsc{clf}: \textcolor{darkblue}{\textbf{\ipa{ɭɯ˧}}} 
\lhead{\firstmark}
\rhead{\botmark}

\subsection{\hspace{-0.5cm} {\Large \textcolor{darkblue}{\textbf{\ipa{læ˧ʁæ˥\$}}}}\hspace{0.5cm}[\kern2pt{\textcolor{darkblue}{\textbf{\ipa{læ˧ʁæ˥}}}}\kern2pt]} \hypertarget{l\{\string_MR\{\string_T\$1}{}
\markboth{\textcolor{darkblue}{\textbf{\ipa{læ˧ʁæ˥\$}}}}{}
\textcolor{teal}{\mytextsc{noun}} \hspace{4pt} Tone: H\$.
\textcolor{Sepia}{\selectlanguage{english}Raven.} \zh{乌鸦。}  \zh{量词}: \textcolor{darkblue}{\textbf{\ipa{mi˩}}}  \mytextsc{clf}: \textcolor{darkblue}{\textbf{\ipa{mi˩}}} 
\lhead{\firstmark}
\rhead{\botmark}

\subsection{\hspace{-0.5cm} {\Large \textcolor{darkblue}{\textbf{\ipa{læ˧ʁæ˧mi˥\$}}}}\hspace{0.5cm}[\kern2pt{\textcolor{darkblue}{\textbf{\ipa{læ˧ʁæ˧mi˥}}}}\kern2pt]} \hypertarget{l\{\string_MR\{\string_Mmi\string_T\$1}{}
\markboth{\textcolor{darkblue}{\textbf{\ipa{læ˧ʁæ˧mi˥\$}}}}{}
\textcolor{teal}{\mytextsc{noun}} \hspace{4pt} Tone: H\$.
\textcolor{Sepia}{\selectlanguage{english}Female raven.} \zh{母乌鸦。}  \zh{量词}: \textcolor{darkblue}{\textbf{\ipa{mi˩}}}  \mytextsc{clf}: \textcolor{darkblue}{\textbf{\ipa{mi˩}}} 
\lhead{\firstmark}
\rhead{\botmark}

\subsection{\hspace{-0.5cm} {\Large \textcolor{darkblue}{\textbf{\ipa{læ˧ʁæ˧-pʰv̩\#˥}}}}\hspace{0.5cm}[\kern2pt{\textcolor{darkblue}{\textbf{\ipa{xxxx non-correspondance entre le nombre de morphèmes et le nombre de tons de morphèmes}}}}\kern2pt]} \hypertarget{l\{\string_MR\{\string_M-p\string_hv\string_=\#\string_T1}{}
\markboth{\textcolor{darkblue}{\textbf{\ipa{læ˧ʁæ˧-pʰv̩\#˥}}}}{}
\textcolor{teal}{\mytextsc{noun}} \hspace{4pt} Tone: \#H.
\textcolor{Sepia}{\selectlanguage{english}Male raven.} \zh{公乌鸦。}  ¶ \textcolor{darkblue}{\textbf{\ipa{læ˧ʁæ˧-pʰv̩˧ tʰv̩˧-mi˥\$}}} \textcolor{Sepia}{\selectlanguage{english}\mytextsc{n}+\mytextsc{dem}+\mytextsc{clf}} \zh{那只公乌鸦}  
 \zh{量词}: \textcolor{darkblue}{\textbf{\ipa{mi˩}}}  \mytextsc{clf}: \textcolor{darkblue}{\textbf{\ipa{mi˩}}} 
\lhead{\firstmark}
\rhead{\botmark}

\subsection{\hspace{-0.5cm} {\Large \textcolor{darkblue}{\textbf{\ipa{læ˧tsɯ˥}}}}\hspace{0.5cm}[\kern2pt{\textcolor{darkblue}{\textbf{\ipa{læ˧tsɯ˥}}}}\kern2pt]} \hypertarget{l\{\string_MtsM\string_T1}{}
\markboth{\textcolor{darkblue}{\textbf{\ipa{læ˧tsɯ˥}}}}{}
\textcolor{teal}{\mytextsc{noun}} \hspace{4pt} Tone: H\#.
\textcolor{Sepia}{\selectlanguage{english}Chilly peppers.} \zh{辣椒(汉语借词:辣子)。} Local Chinese dialect:\zh{辣子。} Borrowing: Chinese  \zh{辣子}
 ¶ \textcolor{darkblue}{\textbf{\ipa{læ˧tsɯ˥ hṽ˩\textasciitilde{}hṽ˩}}} \textcolor{Sepia}{\selectlanguage{english}to fry chilly peppers} \zh{炒辣椒}  
 \zh{量词}: \textcolor{darkblue}{\textbf{\ipa{ɭɯ˧}}}  \mytextsc{clf}: \textcolor{darkblue}{\textbf{\ipa{ɭɯ˧}}} 
\lhead{\firstmark}
\rhead{\botmark}

\subsection{\hspace{-0.5cm} {\Large \textcolor{darkblue}{\textbf{\ipa{le˧‑}}}}\hspace{0.5cm}[\kern2pt{\textcolor{darkblue}{\textbf{\ipa{le˥}}}}\kern2pt]} \hypertarget{le\string_M‑1}{}
\markboth{\textcolor{darkblue}{\textbf{\ipa{le˧‑}}}}{}
\textcolor{teal}{\mytextsc{prefix}} \hspace{4pt} Tone: M/0.
\textcolor{Sepia}{\selectlanguage{english}\mytextsc{accomplished} aspect.} \zh{\mytextsc{实施。}} 
\lhead{\firstmark}
\rhead{\botmark}

\subsection{\hspace{-0.5cm} {\Large \textcolor{darkblue}{\textbf{\ipa{le˧-tɑ˧˥}}}}\hspace{0.5cm}[\kern2pt{\textcolor{darkblue}{\textbf{\ipa{xxxx non-correspondance entre le nombre de morphèmes et le nombre de tons de morphèmes}}}}\kern2pt]} \hypertarget{le\string_M-tA\string_M\string_T1}{}
\markboth{\textcolor{darkblue}{\textbf{\ipa{le˧-tɑ˧˥}}}}{}
\textcolor{teal}{\mytextsc{conjunction}} \hspace{4pt} Tone: MH.
\textcolor{Sepia}{\selectlanguage{english}Up to, all the way to; even.} \zh{到……为止、一直到……、连……。} 
\lhead{\firstmark}
\rhead{\botmark}

\subsection{\hspace{-0.5cm} {\Large \textcolor{darkblue}{\textbf{\ipa{le˧-wo˥}}}}\hspace{0.5cm}[\kern2pt{\textcolor{darkblue}{\textbf{\ipa{xxxx non-correspondance entre le nombre de morphèmes et le nombre de tons de morphèmes}}}}\kern2pt]} \hypertarget{le\string_M-wo\string_T1}{}
\markboth{\textcolor{darkblue}{\textbf{\ipa{le˧-wo˥}}}}{}
\textcolor{teal}{\mytextsc{adverb(ial)}} \hspace{4pt} Tone: H\#.
\textcolor{Sepia}{\selectlanguage{english}Over again, once over again; back.} \zh{再、又、重新。} 
\lhead{\firstmark}
\rhead{\botmark}

\subsection{\hspace{-0.5cm} {\Large \textcolor{darkblue}{\textbf{\ipa{le˧-wo˧}}}}\hspace{0.5cm}[\kern2pt{\textcolor{darkblue}{\textbf{\ipa{xxxx non-correspondance entre le nombre de morphèmes et le nombre de tons de morphèmes}}}}\kern2pt]} \hypertarget{le\string_M-wo\string_M1}{}
\markboth{\textcolor{darkblue}{\textbf{\ipa{le˧-wo˧}}}}{}
\textcolor{teal}{\mytextsc{adverb(ial)}} \hspace{4pt} Tone: M.
\textcolor{Sepia}{\selectlanguage{english}Again; back.} \zh{又,……回去。}  ¶ \textcolor{darkblue}{\textbf{\ipa{le˧-wo˧ jo˧}}} \textcolor{Sepia}{\selectlanguage{english}to come back} \zh{回}  
 ¶ \textcolor{darkblue}{\textbf{\ipa{le˧-wo˧ le˧-gv̩˩}}} \textcolor{Sepia}{\selectlanguage{english}to do over again} \zh{从头开始}  
 ¶ \textcolor{darkblue}{\textbf{\ipa{le˧-wo˧ le˧-gv̩˧\textasciitilde{}gv̩˥}}} \textcolor{Sepia}{\selectlanguage{english}to build anew, to make anew, to rebuild} \zh{重新做、重新建}  
 ¶ \textcolor{darkblue}{\textbf{\ipa{le˧-wo˧ le˥-tɕo˩ ʐwɤ˩}}} \textcolor{Sepia}{\selectlanguage{english}to speak over and over again, to rant, to repeat ceaselessly} \zh{重复讲说过的话}  

\lhead{\firstmark}
\rhead{\botmark}

\subsection{\hspace{-0.5cm} {\Large \textcolor{darkblue}{\textbf{\ipa{le˩}}}}\hspace{0.5cm}[\kern2pt{\textcolor{darkblue}{\textbf{\ipa{le˩˥}}}}\kern2pt]} \hypertarget{le\string_B1}{}
\markboth{\textcolor{darkblue}{\textbf{\ipa{le˩}}}}{}
\textcolor{teal}{\mytextsc{discourse}} \textcolor{teal}{\mytextsc{particle}} \hspace{4pt} Tone: L?.
\textcolor{Sepia}{\selectlanguage{english}Exclamative final particle.} \zh{句尾助词:感叹。}  ¶ \textcolor{darkblue}{\textbf{\ipa{dʑɤ˩ le˥!}}} \textcolor{Sepia}{\selectlanguage{english}Well done! / Great!} \zh{好了!/ 太好了!}  

\lhead{\firstmark}
\rhead{\botmark}

\subsection{\hspace{-0.5cm} {\Large \textcolor{darkblue}{\textbf{\ipa{li˧\textsubscript{a}}}}}\hspace{0.5cm}[\kern2pt{\textcolor{darkblue}{\textbf{\ipa{li˥}}}}\kern2pt]} \hypertarget{li\string_Ma1}{}
\markboth{\textcolor{darkblue}{\textbf{\ipa{li˧\textsubscript{a}}}}}{}
\textcolor{teal}{\mytextsc{verb}} \hspace{4pt} Tone: M\textsubscript{a}.
\ding{202} \textcolor{Sepia}{\selectlanguage{english}To look at.} \zh{看。}  ¶ \textcolor{darkblue}{\textbf{\ipa{tʰi˧-li˧-dʑo˧}}} \textcolor{Sepia}{\selectlanguage{english}\mytextsc{dur} \string_ \mytextsc{prog}} \zh{正在看}  
 ¶ \textcolor{darkblue}{\textbf{\ipa{tso˧\textasciitilde{}tso˧ li˩}}} \textcolor{Sepia}{\selectlanguage{english}to look at things} \zh{看东西}  
\ding{203} \textcolor{Sepia}{\selectlanguage{english}To manage, to be in charge of.} \zh{管理。}  ¶ \textcolor{darkblue}{\textbf{\ipa{ɑ˩ʁo˧ li˧}}} \textcolor{Sepia}{\selectlanguage{english}to manage the household, to look after the house; to keep watch over the house} \zh{管家、管家里的事情,看守家}  
\ding{204} \textcolor{Sepia}{\selectlanguage{english}To visit, to go and see (someone).} \zh{访问。}  ¶ \textcolor{darkblue}{\textbf{\ipa{pʰæ˧tɕi˥-zo˩-ɳɯ˩ | mv̩˩zo˩ li˥}}} \textcolor{Sepia}{\selectlanguage{english}The young man sees the young woman. (Euphemistic phrasing, meaning “the young man has sexual intercourse with the young woman”.)} \zh{小伙子去拜访年轻女人(委婉语,指性交)}  

\lhead{\firstmark}
\rhead{\botmark}

\subsection{\hspace{-0.5cm} {\Large \textcolor{darkblue}{\textbf{\ipa{li˧ʐv̩˩}}}}\hspace{0.5cm}[\kern2pt{\textcolor{darkblue}{\textbf{\ipa{li˩ʐv̩˩˥}}}}\kern2pt]} \hypertarget{li\string_Mz`v\string_=\string_B1}{}
\markboth{\textcolor{darkblue}{\textbf{\ipa{li˧ʐv̩˩}}}}{}
\textcolor{teal}{\mytextsc{noun}} \hspace{4pt} Tone: L\#.
\textcolor{Sepia}{\selectlanguage{english}Tenderloins.} \zh{里脊肉。} 
\lhead{\firstmark}
\rhead{\botmark}

\subsection{\hspace{-0.5cm} {\Large \textcolor{darkblue}{\textbf{\ipa{li˩pi˥}}}}\hspace{0.5cm}[\kern2pt{\textcolor{darkblue}{\textbf{\ipa{li˧pi˥}}}}\kern2pt]} \hypertarget{li\string_Bpi\string_T1}{}
\markboth{\textcolor{darkblue}{\textbf{\ipa{li˩pi˥}}}}{}
\textcolor{teal}{\mytextsc{noun}} \hspace{4pt} Tone: LH.
\textcolor{Sepia}{\selectlanguage{english}Tea that has infused for too long, tea dregs.} \zh{已经泡了太久的茶叶。}  \zh{量词}: \textcolor{darkblue}{\textbf{\ipa{kʰwɤ˥}}}  \mytextsc{clf}: \textcolor{darkblue}{\textbf{\ipa{kʰwɤ˥}}} 
\lhead{\firstmark}
\rhead{\botmark}

\subsection{\hspace{-0.5cm} {\Large \textcolor{darkblue}{\textbf{\ipa{li˩pʰv̩˩}}}}\hspace{0.5cm}[\kern2pt{\textcolor{darkblue}{\textbf{\ipa{li˧pʰv̩˩}}}}\kern2pt]} \hypertarget{li\string_Bp\string_hv\string_=\string_B1}{}
\markboth{\textcolor{darkblue}{\textbf{\ipa{li˩pʰv̩˩}}}}{}
\textcolor{teal}{\mytextsc{noun}} \hspace{4pt} Tone: L.
\textcolor{Sepia}{\selectlanguage{english}Whiteworm Lichen, \textit{Thamnolia vermicularis}; it used to be gathered on the seventh lunar month. It was used as a herbal tea.} \zh{雪茶。}  ¶ \textcolor{darkblue}{\textbf{\ipa{ŋwɤ˧hɑ̃˩-li˩pʰv˩}}} \textcolor{Sepia}{\selectlanguage{english}lichen tea from the mountain \textcolor{darkblue}{\textbf{\ipa{ŋwɤ˧hɑ̃˩}}} (this type of lichen grows abundantly on that mountain, and was generally harvested there)} \zh{\textcolor{darkblue}{\textbf{\ipa{ŋwɤ˧hɑ̃˩}}} 山的雪茶(说明:这种苔藓在那座山上多,七月份人家去采)}  

\lhead{\firstmark}
\rhead{\botmark}

\subsection{\hspace{-0.5cm} {\Large \textcolor{darkblue}{\textbf{\ipa{li˩˥}}}}\hspace{0.5cm}[\kern2pt{\textcolor{darkblue}{\textbf{\ipa{li˩˥}}}}\kern2pt]} \hypertarget{li\string_B\string_T1}{}
\markboth{\textcolor{darkblue}{\textbf{\ipa{li˩˥}}}}{}
\textcolor{teal}{\mytextsc{noun}} \hspace{4pt} Tone: LH.
\textcolor{Sepia}{\selectlanguage{english}Tea.} \zh{茶。}  ¶ \textcolor{darkblue}{\textbf{\ipa{li˩qʰɑ˩}}} \textcolor{Sepia}{\selectlanguage{english}'bitter tea': herbal tea prepared with leaves of Chinese peony, when there was no tea available; it had medicinal properties} \zh{‘苦茶’:用白芍药来泡的饮料,没有茶的时候就喝这种‘苦茶’。它有医疗作用,帮助消化。}  
 \zh{量词}: \textcolor{darkblue}{\textbf{\ipa{qʰwɤ˧˥}}}  \mytextsc{clf}: \textcolor{darkblue}{\textbf{\ipa{qʰwɤ˧˥}}} 
\lhead{\firstmark}
\rhead{\botmark}

\subsection{\hspace{-0.5cm} {\Large \textcolor{darkblue}{\textbf{\ipa{ljɤ˩\textsubscript{a}}}}}\hspace{0.5cm}[\kern2pt{\textcolor{darkblue}{\textbf{\ipa{ljɤ˩˥}}}}\kern2pt]} \hypertarget{lj7\string_Ba1}{}
\markboth{\textcolor{darkblue}{\textbf{\ipa{ljɤ˩\textsubscript{a}}}}}{}
\textcolor{teal}{\mytextsc{classifier}} \hspace{4pt} Tone: L\textsubscript{a}.
\textcolor{Sepia}{\selectlanguage{english}Self-classifier for lives/destinies.} \zh{量词:命、命运。}  ¶ \textcolor{darkblue}{\textbf{\ipa{ʈʂʰɯ˧-ljɤ˥}}} \textcolor{Sepia}{\selectlanguage{english}\mytextsc{dem} \string_ (tone: H\# / H\$)} \zh{\mytextsc{指示代词} \string_}  

\lhead{\firstmark}
\rhead{\botmark}

\subsection{\hspace{-0.5cm} {\Large \textcolor{darkblue}{\textbf{\ipa{ljɤ˩mi˥}}}}\hspace{0.5cm}[\kern2pt{\textcolor{darkblue}{\textbf{\ipa{ljɤ˩mi˩˥}}}}\kern2pt]} \hypertarget{lj7\string_Bmi\string_T1}{}
\markboth{\textcolor{darkblue}{\textbf{\ipa{ljɤ˩mi˥}}}}{}
\textcolor{teal}{\mytextsc{noun}} \hspace{4pt} Tone: LH.
\textcolor{Sepia}{\selectlanguage{english}Major (supporting) beam.} \zh{大梁。}  \zh{量词}: \textcolor{darkblue}{\textbf{\ipa{pʰæ˧˥}}}  \mytextsc{clf}: \textcolor{darkblue}{\textbf{\ipa{pʰæ˧˥}}} 
\lhead{\firstmark}
\rhead{\botmark}

\subsection{\hspace{-0.5cm} {\Large \textcolor{darkblue}{\textbf{\ipa{ljɤ˩mi˥-ʈæ˩qo˩}}}}\hspace{0.5cm}[\kern2pt{\textcolor{darkblue}{\textbf{\ipa{xxxx non-correspondance entre le nombre de morphèmes et le nombre de tons de morphèmes}}}}\kern2pt]} \hypertarget{lj7\string_Bmi\string_T-t`\{\string_Bqo\string_B1}{}
\markboth{\textcolor{darkblue}{\textbf{\ipa{ljɤ˩mi˥-ʈæ˩qo˩}}}}{}
\textcolor{teal}{\mytextsc{noun}} \hspace{4pt} Tone: LH-.
\textcolor{Sepia}{\selectlanguage{english}Decoration of major (supporting) beam: symbolically, this is the beam's 'ear'.} \zh{大梁的装饰:大梁的‘耳朵’。}  \zh{量词}: \textcolor{darkblue}{\textbf{\ipa{pʰæ˧˥}}}  \mytextsc{clf}: \textcolor{darkblue}{\textbf{\ipa{pʰæ˧˥}}} 
\lhead{\firstmark}
\rhead{\botmark}

\subsection{\hspace{-0.5cm} {\Large \textcolor{darkblue}{\textbf{\ipa{ljɤ˩ʂɯ˩}}}}\hspace{0.5cm}[\kern2pt{\textcolor{darkblue}{\textbf{\ipa{xxxx non-correspondance entre le nombre de morphèmes et le nombre de tons de morphèmes}}}}\kern2pt]} \hypertarget{lj7\string_Bs`M\string_B1}{}
\markboth{\textcolor{darkblue}{\textbf{\ipa{ljɤ˩ʂɯ˩}}}}{}
\textcolor{teal}{\mytextsc{noun}} \hspace{4pt} Tone: L.
\textcolor{Sepia}{\selectlanguage{english}Cereals.} \zh{粮食(汉语借词)。}  Borrowing: Chinese  \zh{粮食}

\lhead{\firstmark}
\rhead{\botmark}

\subsection{\hspace{-0.5cm} {\Large \textcolor{darkblue}{\textbf{\ipa{ljɤ˩˥}}} \textsubscript{1}}\hspace{0.5cm}[\kern2pt{\textcolor{darkblue}{\textbf{\ipa{ljɤ˩˥}}}}\kern2pt]} \hypertarget{lj7\string_B\string_T1}{}
\markboth{\textcolor{darkblue}{\textbf{\ipa{ljɤ˩˥}}} \textsubscript{1}}{}
\textcolor{teal}{\mytextsc{noun}} \hspace{4pt} Tone: LH.
\textcolor{Sepia}{\selectlanguage{english}Beam.} \zh{梁。}  \zh{量词}: \textcolor{darkblue}{\textbf{\ipa{pʰæ˧˥}}}  \mytextsc{clf}: \textcolor{darkblue}{\textbf{\ipa{pʰæ˧˥}}} 
\lhead{\firstmark}
\rhead{\botmark}

\subsection{\hspace{-0.5cm} {\Large \textcolor{darkblue}{\textbf{\ipa{ljɤ˩˥}}} \textsubscript{2}}\hspace{0.5cm}[\kern2pt{\textcolor{darkblue}{\textbf{\ipa{ljɤ˩˥}}}}\kern2pt]} \hypertarget{lj7\string_B\string_T2}{}
\markboth{\textcolor{darkblue}{\textbf{\ipa{ljɤ˩˥}}} \textsubscript{2}}{}
\textcolor{teal}{\mytextsc{noun}} \hspace{4pt} Tone: LM?LH?.
\textcolor{Sepia}{\selectlanguage{english}Life, existence, destiny, fate.} \zh{命、生命、命运。}  ¶ \textcolor{darkblue}{\textbf{\ipa{no˧ | ljɤ˩ ʈʂʰɯ˧-ljɤ˩-dʑo˩, | qʰæ˩˥ | ʐwæ˩˥!}}} \textcolor{Sepia}{\selectlanguage{english}You really have a happy lot!} \zh{你命好!}  
 ¶ \textcolor{darkblue}{\textbf{\ipa{hĩ˧-ljɤ˥}}} \textcolor{Sepia}{\selectlanguage{english}human existence, the human lot} \zh{人命、人类的命运}  
 \zh{量词}: \textcolor{darkblue}{\textbf{\ipa{ljɤ˩}}}  \mytextsc{clf}: \textcolor{darkblue}{\textbf{\ipa{ljɤ˩}}} 
\lhead{\firstmark}
\rhead{\botmark}

\subsection{\hspace{-0.5cm} {\Large \textcolor{darkblue}{\textbf{\ipa{lje˩fe˧}}}}\hspace{0.5cm}[\kern2pt{\textcolor{darkblue}{\textbf{\ipa{lje˧fe˩}}}}\kern2pt]} \hypertarget{lje\string_Bfe\string_M1}{}
\markboth{\textcolor{darkblue}{\textbf{\ipa{lje˩fe˧}}}}{}
\textcolor{teal}{\mytextsc{noun}} \hspace{4pt} Tone: LM.
\textcolor{Sepia}{\selectlanguage{english}Mungo bean jelly.} \zh{凉粉。}  Borrowing: Chinese  \zh{凉粉}

\lhead{\firstmark}
\rhead{\botmark}

\subsection{\hspace{-0.5cm} {\Large \textcolor{darkblue}{\textbf{\ipa{lo˧}}}}\hspace{0.5cm}[\kern2pt{\textcolor{darkblue}{\textbf{\ipa{lo˥}}}}\kern2pt]} \hypertarget{lo\string_M1}{}
\markboth{\textcolor{darkblue}{\textbf{\ipa{lo˧}}}}{}
\textcolor{teal}{\mytextsc{noun}} \hspace{4pt} Tone: M.
\ding{202} \textcolor{Sepia}{\selectlanguage{english}Work, occupation, task.} \zh{事情。}  ¶ \textcolor{darkblue}{\textbf{\ipa{lo˧ dʑo˧}}} \textcolor{Sepia}{\selectlanguage{english}to be busy, to have work to do} \zh{忙,有活要干}  
 ¶ \textcolor{darkblue}{\textbf{\ipa{njɤ˧ | lo˧ mɤ˧-dʑo˧.}}} \textcolor{Sepia}{\selectlanguage{english}I am not busy. / I have some free time.} \zh{我不忙。}  
 \zh{量词}: \textcolor{darkblue}{\textbf{\ipa{lo˧}}} \ding{203} \textcolor{Sepia}{\selectlanguage{english}Usefulness.} \zh{用处。}  ¶ \textcolor{darkblue}{\textbf{\ipa{lo˧ mɤ˧-dʑo˧}}} \textcolor{Sepia}{\selectlanguage{english}It's no use / it does not have any usefulness. (Context: talking about ivy, which cannot be fed to cattle and is not used for medical purposes, or for firewood, or for making ropes, tools...)} \zh{没有用!(情景:谈到常春藤,说它是没有用处的植物)}  
 \mytextsc{clf}: \textcolor{darkblue}{\textbf{\ipa{lo˧}}} 
\lhead{\firstmark}
\rhead{\botmark}

\subsection{\hspace{-0.5cm} {\Large \textcolor{darkblue}{\textbf{\ipa{lo˧\textsubscript{b}}}}}\hspace{0.5cm}[\kern2pt{\textcolor{darkblue}{\textbf{\ipa{lo˩˥}}}}\kern2pt]} \hypertarget{lo\string_Mb1}{}
\markboth{\textcolor{darkblue}{\textbf{\ipa{lo˧\textsubscript{b}}}}}{}
\textcolor{teal}{\mytextsc{classifier}} \hspace{4pt} Tone: M\textsubscript{b}.
\textcolor{Sepia}{\selectlanguage{english}Self-classifier for tasks/occupations.} \zh{量词:事情(一件)、活(一个)。} 
\lhead{\firstmark}
\rhead{\botmark}

\subsection{\hspace{-0.5cm} {\Large \textcolor{darkblue}{\textbf{\ipa{lo˧bæ˧˥}}}}\hspace{0.5cm}[\kern2pt{\textcolor{darkblue}{\textbf{\ipa{lo˧bæ˧}}}}\kern2pt]} \hypertarget{lo\string_Mb\{\string_M\string_T1}{}
\markboth{\textcolor{darkblue}{\textbf{\ipa{lo˧bæ˧˥}}}}{}
\textcolor{teal}{\mytextsc{noun}} \hspace{4pt} Tone: MH\#.
\textcolor{Sepia}{\selectlanguage{english}Suspended bridge; zip line, flying fox.} \zh{索桥,溜索。} 
\lhead{\firstmark}
\rhead{\botmark}

\subsection{\hspace{-0.5cm} {\Large \textcolor{darkblue}{\textbf{\ipa{lo˧bv̩˩-ʈʂʰɯ˩}}}}\hspace{0.5cm}[\kern2pt{\textcolor{darkblue}{\textbf{\ipa{lo˩bv̩˧ʈʂʰɯ˧}}}}\kern2pt]} \hypertarget{lo\string_Mbv\string_=\string_B-t`s`\string_hM\string_B1}{}
\markboth{\textcolor{darkblue}{\textbf{\ipa{lo˧bv̩˩-ʈʂʰɯ˩}}}}{}
\textcolor{teal}{\mytextsc{noun}} \hspace{4pt} Tone: L\#-.
\textcolor{Sepia}{\selectlanguage{english}Elephant.} \zh{象、大象。}  Borrowing: Tibetan
 \zh{量词}: \textcolor{darkblue}{\textbf{\ipa{pʰo˧˥}}} \textcolor{darkblue}{\textbf{\ipa{v̩˧}}}  \mytextsc{clf}: \textcolor{darkblue}{\textbf{\ipa{pʰo˧˥}}} \textcolor{darkblue}{\textbf{\ipa{v̩˧}}} 
\lhead{\firstmark}
\rhead{\botmark}

\subsection{\hspace{-0.5cm} {\Large \textcolor{darkblue}{\textbf{\ipa{lo˧ɖʐɤ˩}}}}\hspace{0.5cm}[\kern2pt{\textcolor{darkblue}{\textbf{\ipa{lo˩ɖʐɤ˥}}}}\kern2pt]} \hypertarget{lo\string_Md`z`7\string_B1}{}
\markboth{\textcolor{darkblue}{\textbf{\ipa{lo˧ɖʐɤ˩}}}}{}
\textcolor{teal}{\mytextsc{noun}} \hspace{4pt} Tone: L\#.
\textcolor{Sepia}{\selectlanguage{english}Weeding hoe: hand instrument with three spikes perpendicular to the handle, to loosen the soil. At the time of fieldwork, this tool had a metal head.} \zh{三齿耙。}  ¶ \textcolor{darkblue}{\textbf{\ipa{lo˧ɖʐɤ˩ ʈʂʰɯ˩-nɑ˩}}} \textcolor{Sepia}{\selectlanguage{english}\mytextsc{n}+\mytextsc{dem}+\mytextsc{clf}} \zh{这把三齿耙}  
 \zh{量词}: \textcolor{darkblue}{\textbf{\ipa{nɑ˧}}}  \mytextsc{clf}: \textcolor{darkblue}{\textbf{\ipa{nɑ˧}}} 
\lhead{\firstmark}
\rhead{\botmark}

\subsection{\hspace{-0.5cm} {\Large \textcolor{darkblue}{\textbf{\ipa{lo˧fv̩˧}}}}\hspace{0.5cm}[\kern2pt{\textcolor{darkblue}{\textbf{\ipa{lo˧fv̩˩}}}}\kern2pt]} \hypertarget{lo\string_Mfv\string_=\string_M1}{}
\markboth{\textcolor{darkblue}{\textbf{\ipa{lo˧fv̩˧}}}}{}
\textcolor{teal}{\mytextsc{adjective}} \hspace{4pt} Tone: .
\textcolor{Sepia}{\selectlanguage{english}Easy.} \zh{容易,容易做。}  ¶ \textcolor{darkblue}{\textbf{\ipa{lo˧fv̩˧ | ʐwæ˩˥}}} \textcolor{Sepia}{\selectlanguage{english}very easy} \zh{很容易}  

\lhead{\firstmark}
\rhead{\botmark}

\subsection{\hspace{-0.5cm} {\Large \textcolor{darkblue}{\textbf{\ipa{lo˧gv̩˩}}}}\hspace{0.5cm}[\kern2pt{\textcolor{darkblue}{\textbf{\ipa{lo˧gv̩˩}}}}\kern2pt]} \hypertarget{lo\string_Mgv\string_=\string_B1}{}
\markboth{\textcolor{darkblue}{\textbf{\ipa{lo˧gv̩˩}}}}{}
\textcolor{teal}{\mytextsc{noun}} \hspace{4pt} Tone: L\#.
\textcolor{Sepia}{\selectlanguage{english}Ninglang.} \zh{宁蒗。}  ¶ \textcolor{darkblue}{\textbf{\ipa{lo˧gv̩˩-di˩mi˩}}} \textcolor{Sepia}{\selectlanguage{english}the Ninglang plain} \zh{宁蒗坝子}  

\lhead{\firstmark}
\rhead{\botmark}

\subsection{\hspace{-0.5cm} {\Large \textcolor{darkblue}{\textbf{\ipa{lo˧ʝi˧-hĩ˧-hĩ\#˥}}}}\hspace{0.5cm}[\kern2pt{\textcolor{darkblue}{\textbf{\ipa{xxxx non-correspondance entre le nombre de morphèmes et le nombre de tons de morphèmes}}}}\kern2pt]} \hypertarget{lo\string_Mj££i\string_M-hi\string_~\string_M-hi\string_~\#\string_T1}{}
\markboth{\textcolor{darkblue}{\textbf{\ipa{lo˧ʝi˧-hĩ˧-hĩ\#˥}}}}{}
\textcolor{teal}{\mytextsc{noun}} \hspace{4pt} Tone: \#H.
\textcolor{Sepia}{\selectlanguage{english}Worker (in the fields or elsewhere).} \zh{劳动人民,农民。}  \zh{量词}: \textcolor{darkblue}{\textbf{\ipa{v̩˧}}}  \mytextsc{clf}: \textcolor{darkblue}{\textbf{\ipa{v̩˧}}} 
\lhead{\firstmark}
\rhead{\botmark}

\subsection{\hspace{-0.5cm} {\Large \textcolor{darkblue}{\textbf{\ipa{lo˧lo˧}}}}\hspace{0.5cm}[\kern2pt{\textcolor{darkblue}{\textbf{\ipa{lo˧lo˧}}}}\kern2pt]} \hypertarget{lo\string_Mlo\string_M1}{}
\markboth{\textcolor{darkblue}{\textbf{\ipa{lo˧lo˧}}}}{}
\textcolor{teal}{\mytextsc{noun}} \hspace{4pt} Tone: M.
\textcolor{Sepia}{\selectlanguage{english}Yi (ethnic group).} \zh{彝族。}  \zh{量词}: \textcolor{darkblue}{\textbf{\ipa{v̩˧}}}  \mytextsc{clf}: \textcolor{darkblue}{\textbf{\ipa{v̩˧}}} 
\lhead{\firstmark}
\rhead{\botmark}

\subsection{\hspace{-0.5cm} {\Large \textcolor{darkblue}{\textbf{\ipa{lo˧ɲi˥}}}}\hspace{0.5cm}[\kern2pt{\textcolor{darkblue}{\textbf{\ipa{lo˧ɲi˥}}}}\kern2pt]} \hypertarget{lo\string_MJi\string_T1}{}
\markboth{\textcolor{darkblue}{\textbf{\ipa{lo˧ɲi˥}}}}{}
\textcolor{teal}{\mytextsc{noun}} \hspace{4pt} Tone: H\#.
\textcolor{Sepia}{\selectlanguage{english}Finger.} \zh{手指。}  \zh{量词}: \textcolor{darkblue}{\textbf{\ipa{ɭɯ˧}}}  \mytextsc{clf}: \textcolor{darkblue}{\textbf{\ipa{ɭɯ˧}}} 
\lhead{\firstmark}
\rhead{\botmark}

\subsection{\hspace{-0.5cm} {\Large \textcolor{darkblue}{\textbf{\ipa{lo˧ɲi˥ | ɖɯ˧-ɭɯ˧}}}}\hspace{0.5cm}[\kern2pt{\textcolor{darkblue}{\textbf{\ipa{xxxx non-correspondance entre le nombre de groupes tonals et le nombre de tons}}}}\kern2pt]} \hypertarget{lo\string_MJi\string_T | d`M\string_M-l\string_RM\string_M1}{}
\markboth{\textcolor{darkblue}{\textbf{\ipa{lo˧ɲi˥ | ɖɯ˧-ɭɯ˧}}}}{}
\textcolor{teal}{\mytextsc{noun}} \hspace{4pt} Tone: H\# | M.
\textcolor{Sepia}{\selectlanguage{english}Index.} \zh{食指。} 
\lhead{\firstmark}
\rhead{\botmark}

\subsection{\hspace{-0.5cm} {\Large \textcolor{darkblue}{\textbf{\ipa{lo˧ɲi˥ | ɲi˧-ɭɯ˧}}}}\hspace{0.5cm}[\kern2pt{\textcolor{darkblue}{\textbf{\ipa{xxxx non-correspondance entre le nombre de groupes tonals et le nombre de tons}}}}\kern2pt]} \hypertarget{lo\string_MJi\string_T | Ji\string_M-l\string_RM\string_M1}{}
\markboth{\textcolor{darkblue}{\textbf{\ipa{lo˧ɲi˥ | ɲi˧-ɭɯ˧}}}}{}
\textcolor{teal}{\mytextsc{noun}} \hspace{4pt} Tone: H\# | M.
\textcolor{Sepia}{\selectlanguage{english}Middle finger.} \zh{中指。} 
\lhead{\firstmark}
\rhead{\botmark}

\subsection{\hspace{-0.5cm} {\Large \textcolor{darkblue}{\textbf{\ipa{lo˧ʂv̩˩}}}}\hspace{0.5cm}[\kern2pt{\textcolor{darkblue}{\textbf{\ipa{lo˧ʂv̩˩}}}}\kern2pt]} \hypertarget{lo\string_Ms`v\string_=\string_B1}{}
\markboth{\textcolor{darkblue}{\textbf{\ipa{lo˧ʂv̩˩}}}}{}
\textcolor{teal}{\mytextsc{noun}} \hspace{4pt} Tone: L\#.
\textcolor{Sepia}{\selectlanguage{english}The village of Luoshui.} \zh{洛水村。} 
\lhead{\firstmark}
\rhead{\botmark}

\subsection{\hspace{-0.5cm} {\Large \textcolor{darkblue}{\textbf{\ipa{lo˧ʂv̩˩ | -hi˩-nɑ˧mi˧}}}}\hspace{0.5cm}[\kern2pt{\textcolor{darkblue}{\textbf{\ipa{xxxx non-correspondance entre le nombre de groupes tonals et le nombre de tons}}}}\kern2pt]} \hypertarget{lo\string_Ms`v\string_=\string_B | -hi\string_B-nA\string_Mmi\string_M1}{}
\markboth{\textcolor{darkblue}{\textbf{\ipa{lo˧ʂv̩˩ | -hi˩-nɑ˧mi˧}}}}{}
\textcolor{teal}{\mytextsc{noun}} \hspace{4pt} Tone: L\# | L-.
\textcolor{Sepia}{\selectlanguage{english}Lugu lake.} \zh{泸沽湖。} 
\lhead{\firstmark}
\rhead{\botmark}

\subsection{\hspace{-0.5cm} {\Large \textcolor{darkblue}{\textbf{\ipa{lo˧tɑ˧-lo˧tɕi\#˥}}}}\hspace{0.5cm}[\kern2pt{\textcolor{darkblue}{\textbf{\ipa{xxxx non-correspondance entre le nombre de morphèmes et le nombre de tons de morphèmes}}}}\kern2pt]} \hypertarget{lo\string_MtA\string_M-lo\string_Mts£i\#\string_T1}{}
\markboth{\textcolor{darkblue}{\textbf{\ipa{lo˧tɑ˧-lo˧tɕi\#˥}}}}{}
\textcolor{teal}{\mytextsc{noun}} \hspace{4pt} Tone: \#H.
\textcolor{Sepia}{\selectlanguage{english}Streamer of scriptures.} \zh{经幡、风马旗(挂在山上)。}  Borrowing: Tibetan  rlung rta
 ¶ \textcolor{darkblue}{\textbf{\ipa{lo˧tɑ˧-lo˧tɕi˧ | le˧-lɑ˧˥}}} \textcolor{Sepia}{\selectlanguage{english}to print a streamer of scriptures; more generally: to string together a streamer of scriptures} \zh{直译:印出一个经幡。也来指准备经幡的工作(到山上去挂之前)}  
 \zh{量词}: \textcolor{darkblue}{\textbf{\ipa{pɤ˥}}} \textcolor{darkblue}{\textbf{\ipa{pʰæ˧˥}}}  \mytextsc{clf}: \textcolor{darkblue}{\textbf{\ipa{pɤ˥}}} \textcolor{darkblue}{\textbf{\ipa{pʰæ˧˥}}} 
\lhead{\firstmark}
\rhead{\botmark}

\subsection{\hspace{-0.5cm} {\Large \textcolor{darkblue}{\textbf{\ipa{lo˩}}} \textsubscript{1}}\hspace{0.5cm}[\kern2pt{\textcolor{darkblue}{\textbf{\ipa{lo˩˥}}}}\kern2pt]} \hypertarget{lo\string_B1}{}
\markboth{\textcolor{darkblue}{\textbf{\ipa{lo˩}}} \textsubscript{1}}{}
\textcolor{teal}{\mytextsc{verb}} \hspace{4pt} Tone: L.
\textcolor{Sepia}{\selectlanguage{english}To cross (a mountain pass).} \zh{过(垭口)。}  ¶ \textcolor{darkblue}{\textbf{\ipa{mv̩˩tɕo˧-lo˩}}} \textcolor{Sepia}{\selectlanguage{english}to go down (after crossing a mountain pass)} \zh{往下过去(过了垭口以后)}  

\lhead{\firstmark}
\rhead{\botmark}

\subsection{\hspace{-0.5cm} {\Large \textcolor{darkblue}{\textbf{\ipa{lo˩}}} \textsubscript{2}}\hspace{0.5cm}[\kern2pt{\textcolor{darkblue}{\textbf{\ipa{lo˥}}}}\kern2pt]} \hypertarget{lo\string_B2}{}
\markboth{\textcolor{darkblue}{\textbf{\ipa{lo˩}}} \textsubscript{2}}{}
\textcolor{teal}{\mytextsc{noun}} \hspace{4pt} Tone: L.
\textcolor{Sepia}{\selectlanguage{english}Mountain valley.} \zh{山谷。}  ¶ \textcolor{darkblue}{\textbf{\ipa{lo˧-qo˧}}} \textcolor{Sepia}{\selectlanguage{english}in the valley} \zh{山谷里}  
 \zh{量词}: \textcolor{darkblue}{\textbf{\ipa{lo˩}}}  \mytextsc{clf}: \textcolor{darkblue}{\textbf{\ipa{lo˩}}} 
\lhead{\firstmark}
\rhead{\botmark}

\subsection{\hspace{-0.5cm} {\Large \textcolor{darkblue}{\textbf{\ipa{lo˩\textsubscript{b}}}}}\hspace{0.5cm}[\kern2pt{\textcolor{darkblue}{\textbf{\ipa{lo˥}}}}\kern2pt]} \hypertarget{lo\string_Bb1}{}
\markboth{\textcolor{darkblue}{\textbf{\ipa{lo˩\textsubscript{b}}}}}{}
\textcolor{teal}{\mytextsc{classifier}} \hspace{4pt} Tone: L\textsubscript{b}.
\textcolor{Sepia}{\selectlanguage{english}Self-classifier for valleys.} \zh{量词:谷。}  ¶ \textcolor{darkblue}{\textbf{\ipa{hĩ˧-ɻ̃˧ | ɖɯ˧-lo˩}}} \textcolor{Sepia}{\selectlanguage{english}literally 'a valley of people', to mean: all the population of that valley} \zh{住在一座山谷里的所有人(直译:‘一山谷的人’)}  
 ¶ \textcolor{darkblue}{\textbf{\ipa{si˧dzi˩ | ɖɯ˧-lo˩}}} \textcolor{Sepia}{\selectlanguage{english}'a valley [of/covered with] trees', i.e. a large tract of woodland} \zh{一山谷的树,一片森林(直译:‘一山谷的树’)}  

\lhead{\firstmark}
\rhead{\botmark}

\subsection{\hspace{-0.5cm} {\Large \textcolor{darkblue}{\textbf{\ipa{lo˩bɤ˩}}}}\hspace{0.5cm}[\kern2pt{\textcolor{darkblue}{\textbf{\ipa{lo˧bɤ˧˥}}}}\kern2pt]} \hypertarget{lo\string_Bb7\string_B1}{}
\markboth{\textcolor{darkblue}{\textbf{\ipa{lo˩bɤ˩}}}}{}
\textcolor{teal}{\mytextsc{noun}} \hspace{4pt} Tone: L.
\textcolor{Sepia}{\selectlanguage{english}Palm of the hand.} \zh{手掌。}  \zh{量词}: \textcolor{darkblue}{\textbf{\ipa{ɭɯ˧}}}  \mytextsc{clf}: \textcolor{darkblue}{\textbf{\ipa{ɭɯ˧}}} 
\lhead{\firstmark}
\rhead{\botmark}

\subsection{\hspace{-0.5cm} {\Large \textcolor{darkblue}{\textbf{\ipa{lo˩bv̩˧-ɭɯ˩}}}}\hspace{0.5cm}[\kern2pt{\textcolor{darkblue}{\textbf{\ipa{xxxx non-correspondance entre le nombre de morphèmes et le nombre de tons de morphèmes}}}}\kern2pt]} \hypertarget{lo\string_Bbv\string_=\string_M-l\string_RM\string_B1}{}
\markboth{\textcolor{darkblue}{\textbf{\ipa{lo˩bv̩˧-ɭɯ˩}}}}{}
\textcolor{teal}{\mytextsc{noun}} \hspace{4pt} Tone: LM-L.
\textcolor{Sepia}{\selectlanguage{english}Elbow.} \zh{肘。}  \zh{量词}: \textcolor{darkblue}{\textbf{\ipa{ɭɯ˧}}}  \mytextsc{clf}: \textcolor{darkblue}{\textbf{\ipa{ɭɯ˧}}} 
\lhead{\firstmark}
\rhead{\botmark}

\subsection{\hspace{-0.5cm} {\Large \textcolor{darkblue}{\textbf{\ipa{lo˩dv̩\#˥}}}}\hspace{0.5cm}[\kern2pt{\textcolor{darkblue}{\textbf{\ipa{lo˩dv̩˥}}}}\kern2pt]} \hypertarget{lo\string_Bdv\string_=\#\string_T1}{}
\markboth{\textcolor{darkblue}{\textbf{\ipa{lo˩dv̩\#˥}}}}{}
\textcolor{teal}{\mytextsc{noun}} \hspace{4pt} Tone: LM+\#H.
\textcolor{Sepia}{\selectlanguage{english}Person with a single arm or hand, one-armed (or one-handed) person.} \zh{独臂人:缺一只胳膊(手)的人。}  \zh{量词}: \textcolor{darkblue}{\textbf{\ipa{v̩˧}}}  \mytextsc{clf}: \textcolor{darkblue}{\textbf{\ipa{v̩˧}}} 
\lhead{\firstmark}
\rhead{\botmark}

\subsection{\hspace{-0.5cm} {\Large \textcolor{darkblue}{\textbf{\ipa{lo˩dzi˩}}}}\hspace{0.5cm}[\kern2pt{\textcolor{darkblue}{\textbf{\ipa{lo˩dzi˥}}}}\kern2pt]} \hypertarget{lo\string_Bdzi\string_B1}{}
\markboth{\textcolor{darkblue}{\textbf{\ipa{lo˩dzi˩}}}}{}
\textcolor{teal}{\mytextsc{classifier}} \hspace{4pt} Tone: L.
\textcolor{Sepia}{\selectlanguage{english}A handful (using both hands).} \zh{量词:捧(用两只手)。} 
\lhead{\firstmark}
\rhead{\botmark}

\subsection{\hspace{-0.5cm} {\Large \textcolor{darkblue}{\textbf{\ipa{lo˩dʑo˥}}}}\hspace{0.5cm}[\kern2pt{\textcolor{darkblue}{\textbf{\ipa{lo˩dʑo˩˥}}}}\kern2pt]} \hypertarget{lo\string_Bdz£o\string_T1}{}
\markboth{\textcolor{darkblue}{\textbf{\ipa{lo˩dʑo˥}}}}{}
\textcolor{teal}{\mytextsc{noun}} \hspace{4pt} Tone: LH.
\textcolor{Sepia}{\selectlanguage{english}Bracelet.} \zh{手镯。}  ¶ \textcolor{darkblue}{\textbf{\ipa{ŋv̩˩-lo˩dʑo˧ (+ɲi˩)}}} \textcolor{Sepia}{\selectlanguage{english}silver bracelet} \zh{银手镯}  
 ¶ \textcolor{darkblue}{\textbf{\ipa{hæ̃˩-lo˩dʑo˥ (+ɲi˩)}}} \textcolor{Sepia}{\selectlanguage{english}gold bracelet} \zh{金手镯}  
 ¶ \textcolor{darkblue}{\textbf{\ipa{jo˧-lo˥dʑo˩}}} \textcolor{Sepia}{\selectlanguage{english}jade bracelet} \zh{玉手镯}  
 ¶ \textcolor{darkblue}{\textbf{\ipa{lo˩dʑo˥ kʰɯ˩}}} \textcolor{Sepia}{\selectlanguage{english}to put on a bracelet} \zh{戴上手镯}  
 \zh{量词}: \textcolor{darkblue}{\textbf{\ipa{pʰo˧˥}}}  \mytextsc{clf}: \textcolor{darkblue}{\textbf{\ipa{pʰo˧˥}}} 
\lhead{\firstmark}
\rhead{\botmark}

\subsection{\hspace{-0.5cm} {\Large \textcolor{darkblue}{\textbf{\ipa{lo˩ɖɯ˧}}}}\hspace{0.5cm}[\kern2pt{\textcolor{darkblue}{\textbf{\ipa{xxxx non-correspondance entre le nombre de morphèmes et le nombre de tons de morphèmes}}}}\kern2pt]} \hypertarget{lo\string_Bd`M\string_M1}{}
\markboth{\textcolor{darkblue}{\textbf{\ipa{lo˩ɖɯ˧}}}}{}
\textcolor{teal}{\mytextsc{adjective}} \hspace{4pt} Tone: LM.
\textcolor{Sepia}{\selectlanguage{english}Generous.} \zh{大方。} 
\lhead{\firstmark}
\rhead{\botmark}

\subsection{\hspace{-0.5cm} {\Large \textcolor{darkblue}{\textbf{\ipa{lo˩-gv̩˧dv̩˧}}}}\hspace{0.5cm}[\kern2pt{\textcolor{darkblue}{\textbf{\ipa{lo˧gv̩˧dv̩˧}}}}\kern2pt]} \hypertarget{lo\string_B-gv\string_=\string_Mdv\string_=\string_M1}{}
\markboth{\textcolor{darkblue}{\textbf{\ipa{lo˩-gv̩˧dv̩˧}}}}{}
\textcolor{teal}{\mytextsc{noun}} \hspace{4pt} Tone: L-.
\textcolor{Sepia}{\selectlanguage{english}Back of the hand.} \zh{手背。}  \zh{量词}: \textcolor{darkblue}{\textbf{\ipa{kʰwɤ˥}}}  \mytextsc{clf}: \textcolor{darkblue}{\textbf{\ipa{kʰwɤ˥}}} 
\lhead{\firstmark}
\rhead{\botmark}

\subsection{\hspace{-0.5cm} {\Large \textcolor{darkblue}{\textbf{\ipa{lo˩jɤ˧}}}}\hspace{0.5cm}[\kern2pt{\textcolor{darkblue}{\textbf{\ipa{lo˩jɤ˥}}}}\kern2pt]} \hypertarget{lo\string_Bj7\string_M1}{}
\markboth{\textcolor{darkblue}{\textbf{\ipa{lo˩jɤ˧}}}}{}
\textcolor{teal}{\mytextsc{noun}} \hspace{4pt} Tone: LM.
\textcolor{Sepia}{\selectlanguage{english}Silver coin, silver yuan.} \zh{银元。}  ¶ \textcolor{darkblue}{\textbf{\ipa{lo˩jɤ˧ | ɖɯ˧-pʰæ˧˥}}} \textcolor{Sepia}{\selectlanguage{english}one silver coin} \zh{一块银元}  

\lhead{\firstmark}
\rhead{\botmark}

\subsection{\hspace{-0.5cm} {\Large \textcolor{darkblue}{\textbf{\ipa{lo˩ko˧}}}}\hspace{0.5cm}[\kern2pt{\textcolor{darkblue}{\textbf{\ipa{lo˩ko˥}}}}\kern2pt]} \hypertarget{lo\string_Bko\string_M1}{}
\markboth{\textcolor{darkblue}{\textbf{\ipa{lo˩ko˧}}}}{}
\textcolor{teal}{\mytextsc{noun}} \hspace{4pt} Tone: LM.
\textcolor{Sepia}{\selectlanguage{english}Pot for cooking rice, soup...; used to be made of copper.} \zh{煮饭或煮汤的锣锅。在过去,锣锅一般是铜做的。}  Borrowing: Chinese  \zh{锣锅}
 ¶ \textcolor{darkblue}{\textbf{\ipa{lo˩ko˧: | hɑ˧ tɕɤ˩-di˩! | æ˧-v̩˧, | dʑɯ˩-kʰɯ˩-di˩˥! | ʈʂʰɤ˧ho˥, | dʑɯ˩ tɕɯ˩-di˩˥! |}}} \textcolor{Sepia}{\selectlanguage{english}The cooking pot is for cooking cereals! The copper pot is for putting water! The boiler is for boiling water! (A summary of the respective uses of the three types of pots in use in Yongning about the middle of the 20th century.)} \zh{锣锅,是用来煮饭的!铜锅,是放水用的!水壶,是来煮水的!(描写永宁二十世纪中使用的三种锅)}  
 \zh{量词}: \textcolor{darkblue}{\textbf{\ipa{ɭɯ˧}}}  \mytextsc{clf}: \textcolor{darkblue}{\textbf{\ipa{ɭɯ˧}}} 
\lhead{\firstmark}
\rhead{\botmark}

\subsection{\hspace{-0.5cm} {\Large \textcolor{darkblue}{\textbf{\ipa{lo˩mi˧}}}}\hspace{0.5cm}[\kern2pt{\textcolor{darkblue}{\textbf{\ipa{lo˩mi˥}}}}\kern2pt]} \hypertarget{lo\string_Bmi\string_M1}{}
\markboth{\textcolor{darkblue}{\textbf{\ipa{lo˩mi˧}}}}{}
\textcolor{teal}{\mytextsc{noun}} \hspace{4pt} Tone: LM.
\textcolor{Sepia}{\selectlanguage{english}Thumb.} \zh{大拇指。}  \zh{量词}: \textcolor{darkblue}{\textbf{\ipa{ɭɯ˧}}}  \mytextsc{clf}: \textcolor{darkblue}{\textbf{\ipa{ɭɯ˧}}} 
\lhead{\firstmark}
\rhead{\botmark}

\subsection{\hspace{-0.5cm} {\Large \textcolor{darkblue}{\textbf{\ipa{lo˩mi˧-qɑ˩}}}}\hspace{0.5cm}[\kern2pt{\textcolor{darkblue}{\textbf{\ipa{lo˩mi˧qɑ˧}}}}\kern2pt]} \hypertarget{lo\string_Bmi\string_M-qA\string_B1}{}
\markboth{\textcolor{darkblue}{\textbf{\ipa{lo˩mi˧-qɑ˩}}}}{}
\textcolor{teal}{\mytextsc{noun}} \hspace{4pt} Tone: LM-L.
\textcolor{Sepia}{\selectlanguage{english}Space between thumb and index finger.} \zh{虎口。}  \zh{量词}: \textcolor{darkblue}{\textbf{\ipa{ɭɯ˧}}}  \mytextsc{clf}: \textcolor{darkblue}{\textbf{\ipa{ɭɯ˧}}} 
\lhead{\firstmark}
\rhead{\botmark}

\subsection{\hspace{-0.5cm} {\Large \textcolor{darkblue}{\textbf{\ipa{lo˩pv̩˧˥}}}}\hspace{0.5cm}[\kern2pt{\textcolor{darkblue}{\textbf{\ipa{lo˩pv̩˧˥}}}}\kern2pt]} \hypertarget{lo\string_Bpv\string_=\string_M\string_T1}{}
\markboth{\textcolor{darkblue}{\textbf{\ipa{lo˩pv̩˧˥}}}}{}
\textcolor{teal}{\mytextsc{noun}} \hspace{4pt} Tone: LM+MH\#.
\textcolor{Sepia}{\selectlanguage{english}Ring.} \zh{戒指。}  ¶ \textcolor{darkblue}{\textbf{\ipa{ŋv̩˩-lo˩pv̩˩}}} \textcolor{Sepia}{\selectlanguage{english}silver ring} \zh{银戒指}  
 ¶ \textcolor{darkblue}{\textbf{\ipa{hæ̃˩-lo˩pv̩˩}}} \textcolor{Sepia}{\selectlanguage{english}gold ring} \zh{金戒指}  
 \zh{量词}: \textcolor{darkblue}{\textbf{\ipa{ɭɯ˧}}}  \mytextsc{clf}: \textcolor{darkblue}{\textbf{\ipa{ɭɯ˧}}} 
\lhead{\firstmark}
\rhead{\botmark}

\subsection{\hspace{-0.5cm} {\Large \textcolor{darkblue}{\textbf{\ipa{lo˩qʰv̩˩}}}}\hspace{0.5cm}[\kern2pt{\textcolor{darkblue}{\textbf{\ipa{lo˩qʰv̩˩˥}}}}\kern2pt]} \hypertarget{lo\string_Bq\string_hv\string_=\string_B1}{}
\markboth{\textcolor{darkblue}{\textbf{\ipa{lo˩qʰv̩˩}}}}{}
\textcolor{teal}{\mytextsc{noun}} \hspace{4pt} Tone: L.
\textcolor{Sepia}{\selectlanguage{english}Gully; ravine; valley.} \zh{山沟。}  \zh{量词}: \textcolor{darkblue}{\textbf{\ipa{lo˩}}}  \mytextsc{clf}: \textcolor{darkblue}{\textbf{\ipa{lo˩}}} 
\lhead{\firstmark}
\rhead{\botmark}

\subsection{\hspace{-0.5cm} {\Large \textcolor{darkblue}{\textbf{\ipa{lo˩qʰwɤ˧}}}}\hspace{0.5cm}[\kern2pt{\textcolor{darkblue}{\textbf{\ipa{lo˩qʰwɤ˥}}}}\kern2pt]} \hypertarget{lo\string_Bq\string_hw7\string_M1}{}
\markboth{\textcolor{darkblue}{\textbf{\ipa{lo˩qʰwɤ˧}}}}{}
\textcolor{teal}{\mytextsc{noun}} \hspace{4pt} Tone: LM.
\ding{202} \textcolor{Sepia}{\selectlanguage{english}Arm.} \zh{胳膊。}  ¶ \textcolor{darkblue}{\textbf{\ipa{lo˩qʰwɤ˧ li˧}}} \textcolor{Sepia}{\selectlanguage{english}to look at (the) arm} \zh{看胳膊}  
 \zh{量词}: \textcolor{darkblue}{\textbf{\ipa{pʰo˧˥}}} \ding{203} \textcolor{Sepia}{\selectlanguage{english}Hand.} \zh{手。}  ¶ \textcolor{darkblue}{\textbf{\ipa{lo˩qʰwɤ˧ ʈʂʰæ˧}}} \textcolor{Sepia}{\selectlanguage{english}to wash one's hands} \zh{洗手}  
 \mytextsc{clf}: \textcolor{darkblue}{\textbf{\ipa{pʰo˧˥}}} 
\lhead{\firstmark}
\rhead{\botmark}

\subsection{\hspace{-0.5cm} {\Large \textcolor{darkblue}{\textbf{\ipa{lo˩qʰwɤ˧-kʰɯ˧ʑi˧˥}}}}\hspace{0.5cm}[\kern2pt{\textcolor{darkblue}{\textbf{\ipa{xxxx non-correspondance entre le nombre de morphèmes et le nombre de tons de morphèmes}}}}\kern2pt]} \hypertarget{lo\string_Bq\string_hw7\string_M-k\string_hM\string_Mz£i\string_M\string_T1}{}
\markboth{\textcolor{darkblue}{\textbf{\ipa{lo˩qʰwɤ˧-kʰɯ˧ʑi˧˥}}}}{}
\textcolor{teal}{\mytextsc{noun}} \hspace{4pt} Tone: LM+MH\#.
\textcolor{Sepia}{\selectlanguage{english}Glove.} \zh{手套。} 
\lhead{\firstmark}
\rhead{\botmark}

\subsection{\hspace{-0.5cm} {\Large \textcolor{darkblue}{\textbf{\ipa{lo˩ʁwæ\#˥}}}}\hspace{0.5cm}[\kern2pt{\textcolor{darkblue}{\textbf{\ipa{lo˩ʁwæ˥}}}}\kern2pt]} \hypertarget{lo\string_BRw\{\#\string_T1}{}
\markboth{\textcolor{darkblue}{\textbf{\ipa{lo˩ʁwæ\#˥}}}}{}
\textcolor{teal}{\mytextsc{noun}} \hspace{4pt} Tone: LM+\#H.
\textcolor{Sepia}{\selectlanguage{english}Left-handed person.} \zh{左撇子。} 
\lhead{\firstmark}
\rhead{\botmark}

\subsection{\hspace{-0.5cm} {\Large \textcolor{darkblue}{\textbf{\ipa{lo˩tʰo˧}}}}\hspace{0.5cm}[\kern2pt{\textcolor{darkblue}{\textbf{\ipa{lo˩tʰo˥}}}}\kern2pt]} \hypertarget{lo\string_Bt\string_ho\string_M1}{}
\markboth{\textcolor{darkblue}{\textbf{\ipa{lo˩tʰo˧}}}}{}
\textcolor{teal}{\mytextsc{noun}} \hspace{4pt} Tone: LM.
\textcolor{Sepia}{\selectlanguage{english}Handcuffs, chains to tie a criminal's hands.} \zh{手铐。}  ¶ \textcolor{darkblue}{\textbf{\ipa{lo˩tʰo˧ kʰɯ˧˥}}} \textcolor{Sepia}{\selectlanguage{english}to put handcuffs, to put on chains (on someone's hands)} \zh{戴上手铐}  

\lhead{\firstmark}
\rhead{\botmark}

\subsection{\hspace{-0.5cm} {\Large \textcolor{darkblue}{\textbf{\ipa{lo˩tsʰɯ˥-sɑ˩}}}}\hspace{0.5cm}[\kern2pt{\textcolor{darkblue}{\textbf{\ipa{lo˩tsʰɯ˥sɑ˧}}}}\kern2pt]} \hypertarget{lo\string_Bts\string_hM\string_T-sA\string_B1}{}
\markboth{\textcolor{darkblue}{\textbf{\ipa{lo˩tsʰɯ˥-sɑ˩}}}}{}
\textcolor{teal}{\mytextsc{noun}} \hspace{4pt} Tone: LH-.
\textcolor{Sepia}{\selectlanguage{english}Meat of the anterior limbs of cattle.} \zh{牲畜前腿的肉。} 
\lhead{\firstmark}
\rhead{\botmark}

\subsection{\hspace{-0.5cm} {\Large \textcolor{darkblue}{\textbf{\ipa{lo˩ʈv̩˧}}}}\hspace{0.5cm}[\kern2pt{\textcolor{darkblue}{\textbf{\ipa{lo˩ʈv̩˥}}}}\kern2pt]} \hypertarget{lo\string_Bt`v\string_=\string_M1}{}
\markboth{\textcolor{darkblue}{\textbf{\ipa{lo˩ʈv̩˧}}}}{}
\textcolor{teal}{\mytextsc{noun}} \hspace{4pt} Tone: LM.
\textcolor{Sepia}{\selectlanguage{english}Fist.} \zh{拳。}  \zh{量词}: \textcolor{darkblue}{\textbf{\ipa{ʈv̩˩}}}  \mytextsc{clf}: \textcolor{darkblue}{\textbf{\ipa{ʈv̩˩}}} 
\lhead{\firstmark}
\rhead{\botmark}

\subsection{\hspace{-0.5cm} {\Large \textcolor{darkblue}{\textbf{\ipa{lo˩ʈʰɯ˧}}}}\hspace{0.5cm}[\kern2pt{\textcolor{darkblue}{\textbf{\ipa{lo˩ʈʰɯ˥}}}}\kern2pt]} \hypertarget{lo\string_Bt`\string_hM\string_M1}{}
\markboth{\textcolor{darkblue}{\textbf{\ipa{lo˩ʈʰɯ˧}}}}{}
\textcolor{teal}{\mytextsc{noun}} \hspace{4pt} Tone: LM.
\textcolor{Sepia}{\selectlanguage{english}Elbow.} \zh{肘。}  \zh{量词}: \textcolor{darkblue}{\textbf{\ipa{ʈv̩˩}}}  \mytextsc{clf}: \textcolor{darkblue}{\textbf{\ipa{ʈv̩˩}}} 
\lhead{\firstmark}
\rhead{\botmark}

\subsection{\hspace{-0.5cm} {\Large \textcolor{darkblue}{\textbf{\ipa{lo˩ʈʂæ˧˥}}}}\hspace{0.5cm}[\kern2pt{\textcolor{darkblue}{\textbf{\ipa{lo˩ʈʂæ˧˥}}}}\kern2pt]} \hypertarget{lo\string_Bt`s`\{\string_M\string_T1}{}
\markboth{\textcolor{darkblue}{\textbf{\ipa{lo˩ʈʂæ˧˥}}}}{}
\textcolor{teal}{\mytextsc{noun}} \hspace{4pt} Tone: LM+MH\#.
\textcolor{Sepia}{\selectlanguage{english}Joints of the arm: wrist, elbow.} \zh{手臂的关节(手腕、肘弯)。}  \zh{量词}: \textcolor{darkblue}{\textbf{\ipa{ʈʂæ˧˥}}}  \mytextsc{clf}: \textcolor{darkblue}{\textbf{\ipa{ʈʂæ˧˥}}} 
\lhead{\firstmark}
\rhead{\botmark}

\subsection{\hspace{-0.5cm} {\Large \textcolor{darkblue}{\textbf{\ipa{lo˧˥}}} \textsubscript{1}}\hspace{0.5cm}[\kern2pt{\textcolor{darkblue}{\textbf{\ipa{lo˥}}}}\kern2pt]} \hypertarget{lo\string_M\string_T1}{}
\markboth{\textcolor{darkblue}{\textbf{\ipa{lo˧˥}}} \textsubscript{1}}{}
\textcolor{teal}{\mytextsc{adjective}} \hspace{4pt} Tone: MH.
\textcolor{Sepia}{\selectlanguage{english}Thick.} \zh{厚。}  ¶ \textcolor{darkblue}{\textbf{\ipa{ʈʂʰɯ˧ | lo˧-pæ˧-ɻæ˥-gv̩˩!}}} \textcolor{Sepia}{\selectlanguage{english}It's really thick!} \zh{很厚啊!}  

\lhead{\firstmark}
\rhead{\botmark}

\subsection{\hspace{-0.5cm} {\Large \textcolor{darkblue}{\textbf{\ipa{lo˧˥}}} \textsubscript{2}}\hspace{0.5cm}[\kern2pt{\textcolor{darkblue}{\textbf{\ipa{lo˧˥}}}}\kern2pt]} \hypertarget{lo\string_M\string_T2}{}
\markboth{\textcolor{darkblue}{\textbf{\ipa{lo˧˥}}} \textsubscript{2}}{}
\textcolor{teal}{\mytextsc{verb}} \hspace{4pt} Tone: MH.
\textcolor{Sepia}{\selectlanguage{english}To join hands in an indication of submission.} \zh{拱手作揖。}  ¶ \textcolor{darkblue}{\textbf{\ipa{tsʰɤ˧tsʰɤ˧ lo˧˥}}} \textcolor{Sepia}{\selectlanguage{english}to join hands in an indication of submission} \zh{拱手作揖}  
 ¶ \textcolor{darkblue}{\textbf{\ipa{tsʰɤ˧tsʰɤ˧ | le˧-lo˧-ze˥}}} \textcolor{Sepia}{\selectlanguage{english}\mytextsc{accomp} \string_ \mytextsc{pfv}} \zh{\mytextsc{accomp} \string_ \mytextsc{pfv}}  

\lhead{\firstmark}
\rhead{\botmark}

\subsection{\hspace{-0.5cm} {\Large \textcolor{darkblue}{\textbf{\ipa{*lo˩˧}}}}\hspace{0.5cm}[\kern2pt{\textcolor{darkblue}{\textbf{\ipa{lo˩˥}}}}\kern2pt]} \hypertarget{*lo\string_B\string_M1}{}
\markboth{\textcolor{darkblue}{\textbf{\ipa{*lo˩˧}}}}{}
\textcolor{teal}{\mytextsc{noun}} \hspace{4pt} Tone: LM.
\textcolor{Sepia}{\selectlanguage{english}Thumb.} \zh{大拇指(单音节,按照双音节词构拟出来的)。} 
\lhead{\firstmark}
\rhead{\botmark}

\subsection{\hspace{-0.5cm} {\Large \textcolor{darkblue}{\textbf{\ipa{lv̩˧}}} \textsubscript{1}}\hspace{0.5cm}[\kern2pt{\textcolor{darkblue}{\textbf{\ipa{lv̩˥}}}}\kern2pt]} \hypertarget{lv\string_=\string_M1}{}
\markboth{\textcolor{darkblue}{\textbf{\ipa{lv̩˧}}} \textsubscript{1}}{}
\textcolor{teal}{\mytextsc{noun}} \hspace{4pt} Tone: M.
\textcolor{Sepia}{\selectlanguage{english}Field.} \zh{田地。}  \zh{量词}: \textcolor{darkblue}{\textbf{\ipa{kɤ˧˥}}}  \mytextsc{clf}: \textcolor{darkblue}{\textbf{\ipa{kɤ˧˥}}} 
\lhead{\firstmark}
\rhead{\botmark}

\subsection{\hspace{-0.5cm} {\Large \textcolor{darkblue}{\textbf{\ipa{lv̩˧}}} \textsubscript{2}}\hspace{0.5cm}[\kern2pt{\textcolor{darkblue}{\textbf{\ipa{lv̩˥}}}}\kern2pt]} \hypertarget{lv\string_=\string_M2}{}
\markboth{\textcolor{darkblue}{\textbf{\ipa{lv̩˧}}} \textsubscript{2}}{}
\textcolor{teal}{\mytextsc{noun}} \hspace{4pt} Tone: M.
\textcolor{Sepia}{\selectlanguage{english}Cereals for horses or cattle.} \zh{喂给马或牛的粮食。}  ¶ \textcolor{darkblue}{\textbf{\ipa{ʐwæ˧-lv̩˧}}} \textcolor{Sepia}{\selectlanguage{english}cereals fed to horses; same meaning as \textcolor{darkblue}{\textbf{\ipa{/ʐwæ˧-ɭɯ\#˥/}}}} \zh{喂给马的粮食}  
\textit{See:} \hyperlink{}{\textcolor{darkblue}{\textbf{\ipa{ʐwæ˧-ɭɯ\#˥}}}} 
\lhead{\firstmark}
\rhead{\botmark}

\subsection{\hspace{-0.5cm} {\Large \textcolor{darkblue}{\textbf{\ipa{lv̩˧˥}}}}\hspace{0.5cm}[\kern2pt{\textcolor{darkblue}{\textbf{\ipa{lv̩˧˥}}}}\kern2pt]} \hypertarget{lv\string_=\string_M\string_T1}{}
\markboth{\textcolor{darkblue}{\textbf{\ipa{lv̩˧˥}}}}{}
\textcolor{teal}{\mytextsc{noun}} \hspace{4pt} Tone: MH.
\textcolor{Sepia}{\selectlanguage{english}Maggot.} \zh{蛆。} 
\lhead{\firstmark}
\rhead{\botmark}

\subsection{\hspace{-0.5cm} {\Large \textcolor{darkblue}{\textbf{\ipa{lv̩˧˥}}} \textsubscript{1}}\hspace{0.5cm}[\kern2pt{\textcolor{darkblue}{\textbf{\ipa{lv̩˧˥}}}}\kern2pt]} \hypertarget{lv\string_=\string_M\string_T1}{}
\markboth{\textcolor{darkblue}{\textbf{\ipa{lv̩˧˥}}} \textsubscript{1}}{}
\textcolor{teal}{\mytextsc{verb}} \hspace{4pt} Tone: MH.
\textcolor{Sepia}{\selectlanguage{english}To herd.} \zh{放牧。}  ¶ \textcolor{darkblue}{\textbf{\ipa{go˩bo˥ lv̩˩}}} \textcolor{Sepia}{\selectlanguage{english}to graze cattle, to herd cattle} \zh{放牧牲畜}  
 ¶ \textcolor{darkblue}{\textbf{\ipa{ʐwæ˧ lv̩˩}}} \textcolor{Sepia}{\selectlanguage{english}to graze horses, to herd horses} \zh{放马}  
 ¶ \textcolor{darkblue}{\textbf{\ipa{ʝi˧ lv̩˩}}} \textcolor{Sepia}{\selectlanguage{english}to graze cows, to herd cows} \zh{放牛}  
 ¶ \textcolor{darkblue}{\textbf{\ipa{bo˩ lv̩˩˥}}} \textcolor{Sepia}{\selectlanguage{english}to herd pigs} \zh{放猪}  
 ¶ \textcolor{darkblue}{\textbf{\ipa{tsʰɯ˧ lv̩˥}}} \textcolor{Sepia}{\selectlanguage{english}to graze goats, to herd goats} \zh{放山羊}  
 ¶ \textcolor{darkblue}{\textbf{\ipa{ɖɯ˧-hɤ˧ mɤ˧-lv̩˩\textasciitilde{}lv̩˩}}} \textcolor{Sepia}{\selectlanguage{english}lazy, who does not take care of anything} \zh{懒,什么也不管}  

\lhead{\firstmark}
\rhead{\botmark}

\subsection{\hspace{-0.5cm} {\Large \textcolor{darkblue}{\textbf{\ipa{lv̩˧˥}}} \textsubscript{2}}\hspace{0.5cm}[\kern2pt{\textcolor{darkblue}{\textbf{\ipa{lv̩˧˥}}}}\kern2pt]} \hypertarget{lv\string_=\string_M\string_T2}{}
\markboth{\textcolor{darkblue}{\textbf{\ipa{lv̩˧˥}}} \textsubscript{2}}{}
\textcolor{teal}{\mytextsc{verb}} \hspace{4pt} Tone: MH.
\textcolor{Sepia}{\selectlanguage{english}To escape, to flee.} \zh{逃跑,逃掉。} 
\lhead{\firstmark}
\rhead{\botmark}

\subsection{\hspace{-0.5cm} {\Large \textcolor{darkblue}{\textbf{\ipa{lv̩˥}}}}\hspace{0.5cm}[\kern2pt{\textcolor{darkblue}{\textbf{\ipa{lv̩˥}}}}\kern2pt]} \hypertarget{lv\string_=\string_T1}{}
\markboth{\textcolor{darkblue}{\textbf{\ipa{lv̩˥}}}}{}
\textcolor{teal}{\mytextsc{verb}} \hspace{4pt} Tone: H.
\textcolor{Sepia}{\selectlanguage{english}To wind, to coil, to wrap.} \zh{缠(线……)、裹(毡子……)。}  ¶ \textcolor{darkblue}{\textbf{\ipa{le˧-qo˥-lv̩˩}}} \textcolor{Sepia}{\selectlanguage{english}to wrap, to coil} \zh{裹起来}  
 ¶ \textcolor{darkblue}{\textbf{\ipa{kʰɯ˧ qo˧-lv̩˥}}} \textcolor{Sepia}{\selectlanguage{english}to wind a thread} \zh{缠线}  
 ¶ \textcolor{darkblue}{\textbf{\ipa{qo˧-lv̩˩}}} \textcolor{Sepia}{\selectlanguage{english}to wrap, to coil} \zh{裹}  

\lhead{\firstmark}
\rhead{\botmark}

\subsection{\hspace{-0.5cm} {\Large \textcolor{darkblue}{\textbf{\ipa{lv̩˩\textsubscript{a}}}} \textsubscript{1}}\hspace{0.5cm}[\kern2pt{\textcolor{darkblue}{\textbf{\ipa{lv̩˧˥}}}}\kern2pt]} \hypertarget{lv\string_=\string_Ba1}{}
\markboth{\textcolor{darkblue}{\textbf{\ipa{lv̩˩\textsubscript{a}}}} \textsubscript{1}}{}
\textcolor{teal}{\mytextsc{verb}} \hspace{4pt} Tone: L\textsubscript{a}.
\textcolor{Sepia}{\selectlanguage{english}To bark (a dog barks).} \zh{狗吠。}  ¶ \textcolor{darkblue}{\textbf{\ipa{kʰv̩˩mi˩ lv̩˥ |}}} \textcolor{Sepia}{\selectlanguage{english}the dog barks} \zh{狗吠}  
 ¶ \textcolor{darkblue}{\textbf{\ipa{kʰv̩˩ lv̩˥-dʑo˩ |}}} \textcolor{Sepia}{\selectlanguage{english}the dog is barking} \zh{狗在叫}  
 ¶ \textcolor{darkblue}{\textbf{\ipa{ɖɯ˧-lv̩˧\textasciitilde{}lv̩˥-ɻ̍˩}}} \textcolor{Sepia}{\selectlanguage{english}\mytextsc{delimitative} \string_ \mytextsc{red} \mytextsc{inceptive}} \zh{叫一叫}  

\lhead{\firstmark}
\rhead{\botmark}

\subsection{\hspace{-0.5cm} {\Large \textcolor{darkblue}{\textbf{\ipa{lv̩˩\textsubscript{a}}}} \textsubscript{2}}\hspace{0.5cm}[\kern2pt{\textcolor{darkblue}{\textbf{\ipa{lv̩˩˥}}}}\kern2pt]} \hypertarget{lv\string_=\string_Ba2}{}
\markboth{\textcolor{darkblue}{\textbf{\ipa{lv̩˩\textsubscript{a}}}} \textsubscript{2}}{}
\textcolor{teal}{\mytextsc{verb}} \hspace{4pt} Tone: L\textsubscript{a}.
\textcolor{Sepia}{\selectlanguage{english}To roll, to coil (fabric).} \zh{把布卷起来。}  ¶ \textcolor{darkblue}{\textbf{\ipa{le˧-qæ˥-lv̩˩}}} \textcolor{Sepia}{\selectlanguage{english}to coil} \zh{卷起来}  
 ¶ \textcolor{darkblue}{\textbf{\ipa{le˧-lv̩˧\textasciitilde{}lv̩˧}}}  
 ¶ \textcolor{darkblue}{\textbf{\ipa{tso˧\textasciitilde{}tso˧ lv̩˧\textasciitilde{}lv̩˧}}} \textcolor{Sepia}{\selectlanguage{english}to coil things} \zh{卷东西}  
 ¶ \textcolor{darkblue}{\textbf{\ipa{ɖɯ˧-kʰwɤ˧ lv̩˥}}} \textcolor{Sepia}{\selectlanguage{english}to coil something} \zh{卷一块(东西)}  

\lhead{\firstmark}
\rhead{\botmark}

\subsection{\hspace{-0.5cm} {\Large \textcolor{darkblue}{\textbf{\ipa{lv̩˩\textsubscript{a}}}} \textsubscript{3}}\hspace{0.5cm}[\kern2pt{\textcolor{darkblue}{\textbf{\ipa{lv̩˩˥}}}}\kern2pt]} \hypertarget{lv\string_=\string_Ba3}{}
\markboth{\textcolor{darkblue}{\textbf{\ipa{lv̩˩\textsubscript{a}}}} \textsubscript{3}}{}
\textcolor{teal}{\mytextsc{verb}} \hspace{4pt} Tone: L\textsubscript{a}.
\textcolor{Sepia}{\selectlanguage{english}To plough, to till.} \zh{耕种。}  ¶ \textcolor{darkblue}{\textbf{\ipa{le˧-lv̩˩-ze˩}}} \textcolor{Sepia}{\selectlanguage{english}\mytextsc{accomp} \string_ \mytextsc{pfv}} \zh{耕种了}  
 ¶ \textcolor{darkblue}{\textbf{\ipa{ʝi˧-lv̩˧˥}}} \textcolor{Sepia}{\selectlanguage{english}to plough} \zh{耕种}  
 ¶ \textcolor{darkblue}{\textbf{\ipa{dʑi˧mi˧ lv̩˧˥ / dʑi˧mi˧ lv̩˧-ze˥}}} \textcolor{Sepia}{\selectlanguage{english}to plough with a water buffalo} \zh{用水牛耕田}  
 ¶ \textcolor{darkblue}{\textbf{\ipa{ʝi˧ ɖɯ˧-lv̩˧\textasciitilde{}lv̩˥-ɻ̍˩}}} \textcolor{Sepia}{\selectlanguage{english}to plough a little} \zh{耕一耕}  

\lhead{\firstmark}
\rhead{\botmark}

\subsection{\hspace{-0.5cm} {\Large \textcolor{darkblue}{\textbf{\ipa{lv̩˩\textsubscript{a}}}} \textsubscript{4}}\hspace{0.5cm}[\kern2pt{\textcolor{darkblue}{\textbf{\ipa{lv̩˩˥}}}}\kern2pt]} \hypertarget{lv\string_=\string_Ba4}{}
\markboth{\textcolor{darkblue}{\textbf{\ipa{lv̩˩\textsubscript{a}}}} \textsubscript{4}}{}
\textcolor{teal}{\mytextsc{verb}} \hspace{4pt} Tone: L\textsubscript{a}.
\textcolor{Sepia}{\selectlanguage{english}To suffice, to be enough.} \zh{足够。}  ¶ \textcolor{darkblue}{\textbf{\ipa{ə˩-lv̩˩˥? / ə˩-lv̩˩-ze˥?}}} \textcolor{Sepia}{\selectlanguage{english}Is it enough? Is it sufficient?} \zh{够了吗?}  

\lhead{\firstmark}
\rhead{\botmark}

\subsection{\hspace{-0.5cm} {\Large \textcolor{darkblue}{\textbf{\ipa{lv̩˧dʑɯ˥}}}}\hspace{0.5cm}[\kern2pt{\textcolor{darkblue}{\textbf{\ipa{lv̩˧dʑɯ˧}}}}\kern2pt]} \hypertarget{lv\string_=\string_Mdz£M\string_T1}{}
\markboth{\textcolor{darkblue}{\textbf{\ipa{lv̩˧dʑɯ˥}}}}{}
\textcolor{teal}{\mytextsc{noun}} \hspace{4pt} Tone: H\#.
\textcolor{Sepia}{\selectlanguage{english}Stone chips, little slabs of stone.} \zh{零碎的石头块。}  \zh{量词}: \textcolor{darkblue}{\textbf{\ipa{ʈʂwɤ˧}}}  \mytextsc{clf}: \textcolor{darkblue}{\textbf{\ipa{ʈʂwɤ˧}}} 
\lhead{\firstmark}
\rhead{\botmark}

\subsection{\hspace{-0.5cm} {\Large \textcolor{darkblue}{\textbf{\ipa{lv̩˩ʝi˧}}}}\hspace{0.5cm}[\kern2pt{\textcolor{darkblue}{\textbf{\ipa{lv̩˧ʝi˧}}}}\kern2pt]} \hypertarget{lv\string_=\string_Bj££i\string_M1}{}
\markboth{\textcolor{darkblue}{\textbf{\ipa{lv̩˩ʝi˧}}}}{}
\textcolor{teal}{\mytextsc{verb}} \hspace{4pt} Tone: LM.
\textcolor{Sepia}{\selectlanguage{english}To record sound.} \zh{录音(汉语借词)。}  Borrowing: Chinese  \zh{录音}
 ¶ \textcolor{darkblue}{\textbf{\ipa{hɑ˧ le˧-dzɯ˧-se˥, | lv̩˩ ʝi˧-bi˧ !}}} \textcolor{Sepia}{\selectlanguage{english}After the meal, we'll do a recording!} \zh{吃完饭,就录音吧! / 吃完饭就可以录音了!}  

\lhead{\firstmark}
\rhead{\botmark}

\subsection{\hspace{-0.5cm} {\Large \textcolor{darkblue}{\textbf{\ipa{lv̩˧mi˧}}}}\hspace{0.5cm}[\kern2pt{\textcolor{darkblue}{\textbf{\ipa{lv̩˩mi˥}}}}\kern2pt]} \hypertarget{lv\string_=\string_Mmi\string_M1}{}
\markboth{\textcolor{darkblue}{\textbf{\ipa{lv̩˧mi˧}}}}{}
\textcolor{teal}{\mytextsc{noun}} \hspace{4pt} Tone: M.
\textcolor{Sepia}{\selectlanguage{english}Stone.} \zh{石头。}  ¶ \textcolor{darkblue}{\textbf{\ipa{kʰv̩˧pʰæ˧tɕi˩, | lv̩˧mi˧ dzɯ˧-bi˧-ʁo˧-ho˩!}}} \textcolor{Sepia}{\selectlanguage{english}'When one is young, one could eat stones!' (Meaning: when one is young, one can eat anything, one has an excellent digestion; as one gets old, one is less tolerant of food that is not easy to digest.)} \zh{‘年轻人,石头都能吃!’(意思:年轻人消化好,吃什么都行,而人变老就不那么容易消化了,要注意吃什么。)}  
 \zh{量词}: \textcolor{darkblue}{\textbf{\ipa{ɭɯ˧}}}  \mytextsc{clf}: \textcolor{darkblue}{\textbf{\ipa{ɭɯ˧}}} 
\lhead{\firstmark}
\rhead{\botmark}

\subsection{\hspace{-0.5cm} {\Large \textcolor{darkblue}{\textbf{\ipa{lv̩˧mi˧-bo\#˥}}}}\hspace{0.5cm}[\kern2pt{\textcolor{darkblue}{\textbf{\ipa{xxxx non-correspondance entre le nombre de morphèmes et le nombre de tons de morphèmes}}}}\kern2pt]} \hypertarget{lv\string_=\string_Mmi\string_M-bo\#\string_T1}{}
\markboth{\textcolor{darkblue}{\textbf{\ipa{lv̩˧mi˧-bo\#˥}}}}{}
\textcolor{teal}{\mytextsc{noun}} \hspace{4pt} Tone: \#H.
\textcolor{Sepia}{\selectlanguage{english}Stone wall.} \zh{石墙。}  \zh{量词}: \textcolor{darkblue}{\textbf{\ipa{ɭɯ˧}}}  \mytextsc{clf}: \textcolor{darkblue}{\textbf{\ipa{ɭɯ˧}}} 
\lhead{\firstmark}
\rhead{\botmark}

\subsection{\hspace{-0.5cm} {\Large \textcolor{darkblue}{\textbf{\ipa{lv̩˧mi˧-dʑɯ˧dʑɯ˩}}}}\hspace{0.5cm}[\kern2pt{\textcolor{darkblue}{\textbf{\ipa{xxxx non-correspondance entre le nombre de morphèmes et le nombre de tons de morphèmes}}}}\kern2pt]} \hypertarget{lv\string_=\string_Mmi\string_M-dz£M\string_Mdz£M\string_B1}{}
\markboth{\textcolor{darkblue}{\textbf{\ipa{lv̩˧mi˧-dʑɯ˧dʑɯ˩}}}}{}
\textcolor{teal}{\mytextsc{noun}} \hspace{4pt} Tone: \mytextsc{L}\#.
\textcolor{Sepia}{\selectlanguage{english}Little slabs of stone, stone chips.} \zh{零碎的石块。}  \zh{量词}: \textcolor{darkblue}{\textbf{\ipa{kʰwɤ˥}}}  \mytextsc{clf}: \textcolor{darkblue}{\textbf{\ipa{kʰwɤ˥}}} 
\lhead{\firstmark}
\rhead{\botmark}

\subsection{\hspace{-0.5cm} {\Large \textcolor{darkblue}{\textbf{\ipa{lv̩˧pʰv̩˩}}} \textsubscript{2}}\hspace{0.5cm}[\kern2pt{\textcolor{darkblue}{\textbf{\ipa{lv̩˧pʰv̩˩}}}}\kern2pt]} \hypertarget{lv\string_=\string_Mp\string_hv\string_=\string_B2}{}
\markboth{\textcolor{darkblue}{\textbf{\ipa{lv̩˧pʰv̩˩}}} \textsubscript{2}}{}
\textcolor{teal}{\mytextsc{noun}} \hspace{4pt} Tone: L\#.
\textcolor{Sepia}{\selectlanguage{english}Paddy field.} \zh{水田。}  \zh{量词}: \textcolor{darkblue}{\textbf{\ipa{pʰv̩˩}}}  \mytextsc{clf}: \textcolor{darkblue}{\textbf{\ipa{pʰv̩˩}}} 
\lhead{\firstmark}
\rhead{\botmark}

\subsection{\hspace{-0.5cm} {\Large \textcolor{darkblue}{\textbf{\ipa{lv̩˧qæ\#˥}}}}\hspace{0.5cm}[\kern2pt{\textcolor{darkblue}{\textbf{\ipa{xxxx groupe tonal entier sans aucun ton}}}}\kern2pt]} \hypertarget{lv\string_=\string_Mq\{\#\string_T1}{}
\markboth{\textcolor{darkblue}{\textbf{\ipa{lv̩˧qæ\#˥}}}}{}
\textcolor{teal}{\mytextsc{noun}} \hspace{4pt} Tone: \#˥.
\textcolor{Sepia}{\selectlanguage{english}Limit, boundary between fields belonging to different families. It is typically materialized by a small dike.} \zh{地界:不同家庭田地之间的界限。} 
\lhead{\firstmark}
\rhead{\botmark}

\subsection{\hspace{-0.5cm} {\Large \textcolor{darkblue}{\textbf{\ipa{lv̩˧sɯ˥}}}}\hspace{0.5cm}[\kern2pt{\textcolor{darkblue}{\textbf{\ipa{lv̩˧sɯ˥}}}}\kern2pt]} \hypertarget{lv\string_=\string_MsM\string_T1}{}
\markboth{\textcolor{darkblue}{\textbf{\ipa{lv̩˧sɯ˥}}}}{}
\textcolor{teal}{\mytextsc{noun}} \hspace{4pt} Tone: H\#.
\textcolor{Sepia}{\selectlanguage{english}Lisu (ethnic group).} \zh{傈僳族。}  \zh{量词}: \textcolor{darkblue}{\textbf{\ipa{v̩˧}}}  \mytextsc{clf}: \textcolor{darkblue}{\textbf{\ipa{v̩˧}}} 
\lhead{\firstmark}
\rhead{\botmark}

\subsection{\hspace{-0.5cm} {\Large \textcolor{darkblue}{\textbf{\ipa{lv̩˩tɕʰɯ˧}}}}\hspace{0.5cm}[\kern2pt{\textcolor{darkblue}{\textbf{\ipa{lv̩˩tɕʰɯ˥}}}}\kern2pt]} \hypertarget{lv\string_=\string_Bts£\string_hM\string_M1}{}
\markboth{\textcolor{darkblue}{\textbf{\ipa{lv̩˩tɕʰɯ˧}}}}{}
\textcolor{teal}{\mytextsc{noun}} \hspace{4pt} Tone: LM.
\textcolor{Sepia}{\selectlanguage{english}The village of Fengke (close to the Yangtze river): this is the former name of the area in Chinese.} \zh{六区,今奉科乡(汉语借词)。}  Borrowing: Chinese  \zh{六区}

\lhead{\firstmark}
\rhead{\botmark}

\subsection{\hspace{-0.5cm} {\Large \textcolor{darkblue}{\textbf{\ipa{lv̩˩tɕʰɯ˧-hĩ\#˥}}}}\hspace{0.5cm}[\kern2pt{\textcolor{darkblue}{\textbf{\ipa{xxxx non-correspondance entre le nombre de morphèmes et le nombre de tons de morphèmes}}}}\kern2pt]} \hypertarget{lv\string_=\string_Bts£\string_hM\string_M-hi\string_~\#\string_T1}{}
\markboth{\textcolor{darkblue}{\textbf{\ipa{lv̩˩tɕʰɯ˧-hĩ\#˥}}}}{}
\textcolor{teal}{\mytextsc{noun}} \hspace{4pt} Tone: LM+\#H.
\textcolor{Sepia}{\selectlanguage{english}The inhabitants of the village of Fengke (Fv-kho).} \zh{奉科的人。}  Borrowing: Chinese  \zh{六区}

\lhead{\firstmark}
\rhead{\botmark}

\subsection{\hspace{-0.5cm} {\Large \textcolor{darkblue}{\textbf{\ipa{lv̩˧tsɯ˥}}}}\hspace{0.5cm}[\kern2pt{\textcolor{darkblue}{\textbf{\ipa{lv̩˧tsɯ˥}}}}\kern2pt]} \hypertarget{lv\string_=\string_MtsM\string_T1}{}
\markboth{\textcolor{darkblue}{\textbf{\ipa{lv̩˧tsɯ˥}}}}{}
\textcolor{teal}{\mytextsc{noun}} \hspace{4pt} Tone: H\#.
\textcolor{Sepia}{\selectlanguage{english}Oven.} \zh{炉子(汉语借词)。}  Borrowing: Chinese  \zh{炉子}
 \zh{量词}: \textcolor{darkblue}{\textbf{\ipa{nɑ˧}}}  \mytextsc{clf}: \textcolor{darkblue}{\textbf{\ipa{nɑ˧}}} 
\lhead{\firstmark}
\rhead{\botmark}

\subsection{\hspace{-0.5cm} {\Large \textcolor{darkblue}{\textbf{\ipa{lv̩˩\textasciitilde{}lv̩˧˥}}}}\hspace{0.5cm}[\kern2pt{\textcolor{darkblue}{\textbf{\ipa{lv̩˧lv̩˧˥}}}}\kern2pt]} \hypertarget{lv\string_=\string_B~lv\string_=\string_M\string_T1}{}
\markboth{\textcolor{darkblue}{\textbf{\ipa{lv̩˩\textasciitilde{}lv̩˧˥}}}}{}
\textcolor{teal}{\mytextsc{verb}} \hspace{4pt} Tone: MH.
\textcolor{Sepia}{\selectlanguage{english}To move.} \zh{动(虫、桌子、小孩子动)。}  ¶ \textcolor{darkblue}{\textbf{\ipa{lv̩˩\textasciitilde{}lv̩˧-ze˥}}} \textcolor{Sepia}{\selectlanguage{english}\mytextsc{pfv}} \zh{动了}  
 ¶ \textcolor{darkblue}{\textbf{\ipa{tʰi˧-lv̩˩\textasciitilde{}lv̩˩(-ze˩)}}} \textcolor{Sepia}{\selectlanguage{english}\mytextsc{dur} \mytextsc{red}} \zh{\mytextsc{dur} \mytextsc{red}}  
 ¶ \textcolor{darkblue}{\textbf{\ipa{tʰi˧-lv̩˩\textasciitilde{}lv̩˩ | se˧}}} \textcolor{Sepia}{\selectlanguage{english}to walk askance, to walk askew: e.g. a lame person walks with difficulty} \zh{歪着走、扭着走、例如:残疾人走路有困难}  
 ¶ \textcolor{darkblue}{\textbf{\ipa{kʰɯ˧tsʰɤ˧ lv̩˥\textasciitilde{}lv̩˩}}} \textcolor{Sepia}{\selectlanguage{english}to move one's leg around} \zh{活动一下(自己的)腿}  

\lhead{\firstmark}
\rhead{\botmark}

\subsection{\hspace{-0.5cm} {\Large \textcolor{darkblue}{\textbf{\ipa{lv˧bv˧}}}}\hspace{0.5cm}[\kern2pt{\textcolor{darkblue}{\textbf{\ipa{lv˩bv˩˥}}}}\kern2pt]} \hypertarget{lv\string_Mbv\string_M1}{}
\markboth{\textcolor{darkblue}{\textbf{\ipa{lv˧bv˧}}}}{}
\textcolor{teal}{\mytextsc{noun}} \hspace{4pt} Tone: M.
\textcolor{Sepia}{\selectlanguage{english}Vegetable bed.} \zh{菜畦。}  ¶ \textcolor{darkblue}{\textbf{\ipa{v˩tsʰɤ˧-lv˧bv̩\#˥}}} \textcolor{Sepia}{\selectlanguage{english}same meaning: vegetable bed (in a vegetable garden)} \zh{同上:菜畦}  
 ¶ \textcolor{darkblue}{\textbf{\ipa{qʰwæ˧ɭɯ˧-qo˧ | v˩tsʰɤ˧-lv˧bv˧ | le˧-gv˩, v˩tsʰɤ˧˥ | ɖɯ˧-jɤ˩ tʰi˩-pʰo˩}}} \textcolor{Sepia}{\selectlanguage{english}to make a vegetable bed in the vegetable garden, and to sow a row of vegetables} \zh{菜园里建菜畦,种一排菜}  
 \zh{量词}: \textcolor{darkblue}{\textbf{\ipa{kɤ˧˥}}}  \mytextsc{clf}: \textcolor{darkblue}{\textbf{\ipa{kɤ˧˥}}} 
\lhead{\firstmark}
\rhead{\botmark}

\subsection{\hspace{-0.5cm} {\Large \textcolor{darkblue}{\textbf{\ipa{lwæ˩pʰv̩˩}}}}\hspace{0.5cm}[\kern2pt{\textcolor{darkblue}{\textbf{\ipa{lwæ˩pʰv̩˩˥}}}}\kern2pt]} \hypertarget{lw\{\string_Bp\string_hv\string_=\string_B1}{}
\markboth{\textcolor{darkblue}{\textbf{\ipa{lwæ˩pʰv̩˩}}}}{}
\textcolor{teal}{\mytextsc{noun}} \hspace{4pt} Tone: L.
\textcolor{Sepia}{\selectlanguage{english}Ashes.} \zh{灰。}  ¶ \textcolor{darkblue}{\textbf{\ipa{[F5] lwæ˩pʰv̩˩-ni˥gv̩˩}}} \textcolor{Sepia}{\selectlanguage{english}grey; literally: “like ashes”} \zh{灰色的(直译:“像白灰”)}  

\lhead{\firstmark}
\rhead{\botmark}

\subsection{\hspace{-0.5cm} {\Large \textcolor{darkblue}{\textbf{\ipa{lwɤ˩˥}}}}\hspace{0.5cm}[\kern2pt{\textcolor{darkblue}{\textbf{\ipa{lwɤ˩˥}}}}\kern2pt]} \hypertarget{lw7\string_B\string_T1}{}
\markboth{\textcolor{darkblue}{\textbf{\ipa{lwɤ˩˥}}}}{}
\textcolor{teal}{\mytextsc{noun}} \hspace{4pt} Tone: LH.
\textcolor{Sepia}{\selectlanguage{english}Ashes (of plants, charcoal...), cinders.} \zh{灰,灰烬(包括草木灰等等)。}  ¶ \textcolor{darkblue}{\textbf{\ipa{lwɤ˩-pʰæ˧di˩}}} \textcolor{Sepia}{\selectlanguage{english}like ashes; gray-coloured} \zh{像灰烬,灰色}  
 \zh{量词}: \textcolor{darkblue}{\textbf{\ipa{ʈʂwɤ˧}}}  \mytextsc{clf}: \textcolor{darkblue}{\textbf{\ipa{ʈʂwɤ˧}}} 
\lhead{\firstmark}
\rhead{\botmark}

\subsection{\hspace{-0.5cm} {\Large \textcolor{darkblue}{\textbf{\ipa{ɭɯ˧\textsubscript{b}}}}}\hspace{0.5cm}[\kern2pt{\textcolor{darkblue}{\textbf{\ipa{ɭɯ˥}}}}\kern2pt]} \hypertarget{l\string_RM\string_Mb1}{}
\markboth{\textcolor{darkblue}{\textbf{\ipa{ɭɯ˧\textsubscript{b}}}}}{}
\textcolor{teal}{\mytextsc{classifier}} \hspace{4pt} Tone: M\textsubscript{b}.
\textcolor{Sepia}{\selectlanguage{english}Originally a classifier for round objects: grains, bowls… Now a generic classifier, used e.g. for pieces of clothing.} \zh{最常用的量词,相当于汉语中‘个’的用法。本意是圆形颗粒。一粒(米……),一个(碗……),件(衣服……)。}  ¶ \textcolor{darkblue}{\textbf{\ipa{ɕi˧ ɖɯ˧-ɭɯ˧ |}}} \textcolor{Sepia}{\selectlanguage{english}a grain of rice} \zh{一粒米}  
 ¶ \textcolor{darkblue}{\textbf{\ipa{hõ˧-ɭɯ˥}}} \textcolor{Sepia}{\selectlanguage{english}eight grains} \zh{八粒}  
 ¶ \textcolor{darkblue}{\textbf{\ipa{ɖɯ˧-ɭɯ˧ hwæ˧-mɤ˧-ɖo˧! | le˧-qʰwæ˧-kv̩˥!}}} \textcolor{Sepia}{\selectlanguage{english}Don't buy just one / Don't buy a single one: it would break! (Explanation: objects must be bought by pairs: 2, 4, 6, 8, 10..., not by sets of odd numbers (1, 3, 5, 7, 9...), otherwise it bears ill luck and the objects get broken or lost)} \zh{不要(只)买一个!会碎的!(东西要一对一对买:2、4、6、8、10……,单数不吉利,东西会碎的。)}  
 ¶ \textcolor{darkblue}{\textbf{\ipa{ʈʂʰɯ˧ | zo˧hṽ˥ | dʑɤ˩-ɭɯ˥ dʑo˩!}}} \textcolor{Sepia}{\selectlanguage{english}She has a really pretty child! (Context: the main consultant had a polite conversation with a neighbour who had a lovely grandson; later on, she told me: “She has a really pretty child!”)} \zh{她有个很漂亮的孩子!}  

\lhead{\firstmark}
\rhead{\botmark}

\subsection{\hspace{-0.5cm} {\Large \textcolor{darkblue}{\textbf{\ipa{ɭɯ˧˥\textsubscript{a}}}}}\hspace{0.5cm}[\kern2pt{\textcolor{darkblue}{\textbf{\ipa{ɭɯ˧˥}}}}\kern2pt]} \hypertarget{l\string_RM\string_M\string_Ta1}{}
\markboth{\textcolor{darkblue}{\textbf{\ipa{ɭɯ˧˥\textsubscript{a}}}}}{}
\textcolor{teal}{\mytextsc{classifier}} \hspace{4pt} Tone: MH\textsubscript{a}.
\zh{量词:衣服(一件)。}  ¶ \textcolor{darkblue}{\textbf{\ipa{ʈʰæ˧qʰwɤ˧ ɖɯ˧-ɭɯ˧˥}}} \textcolor{Sepia}{\selectlanguage{english}a skirt} \zh{一件裙子}  
 ¶ \textcolor{darkblue}{\textbf{\ipa{bɑ˩lɑ˩˥ | ɖɯ˧-ɭɯ˧˥ |}}} \textcolor{Sepia}{\selectlanguage{english}a piece of clothing; a shirt} \zh{一件衣服}  
 ¶ \textcolor{darkblue}{\textbf{\ipa{*dʑi˧hṽ˥\$+ɖɯ˧-ɭɯ˧˥}}} \textcolor{Sepia}{\selectlanguage{english}This classifier cannot combine with /dʑi˧hṽ˥\$/, which takes /ɖɯ˧-dzi˩/ as its classifier.} \zh{(这个量词不能与/dʑi˧hṽ˥\$/结合。)}  

\lhead{\firstmark}
\rhead{\botmark}

\newpage
\section*{\centering- \textcolor{darkblue}{\textbf{\ipa{ɬ}}} -}
\subsection{\hspace{-0.5cm} {\Large \textcolor{darkblue}{\textbf{\ipa{ɬɑ˧mv̩˥\$}}}}\hspace{0.5cm}[\kern2pt{\textcolor{darkblue}{\textbf{\ipa{ɬɑ˧mv̩˥}}}}\kern2pt]} \hypertarget{KA\string_Mmv\string_=\string_T\$1}{}
\markboth{\textcolor{darkblue}{\textbf{\ipa{ɬɑ˧mv̩˥\$}}}}{}
\textcolor{teal}{\mytextsc{noun}} \hspace{4pt} Tone: H\$.
\textcolor{Sepia}{\selectlanguage{english}Feminine given name.} \zh{女性名字。} 
\lhead{\firstmark}
\rhead{\botmark}

\subsection{\hspace{-0.5cm} {\Large \textcolor{darkblue}{\textbf{\ipa{ɬɑ˧pɤ˩}}}}\hspace{0.5cm}[\kern2pt{\textcolor{darkblue}{\textbf{\ipa{ɬɑ˧pɤ˩}}}}\kern2pt]} \hypertarget{KA\string_Mp7\string_B1}{}
\markboth{\textcolor{darkblue}{\textbf{\ipa{ɬɑ˧pɤ˩}}}}{}
\textcolor{teal}{\mytextsc{adverb(ial)}} \hspace{4pt} Tone: L\#.
\textcolor{Sepia}{\selectlanguage{english}A lot, hard.} \zh{多、使劲。}  ¶ \textcolor{darkblue}{\textbf{\ipa{ɬɑ˧pɤ˩ ʝi˩}}} \textcolor{Sepia}{\selectlanguage{english}to work hard, to work in a concentrated manner} \zh{使劲工作、使劲干}  
 ¶ \textcolor{darkblue}{\textbf{\ipa{ɬɑ˧pɤ˩ | ɖɯ˧-kʰwɤ˧ ʝi˧}}} \textcolor{Sepia}{\selectlanguage{english}to work hard for a while, to get some solid work done} \zh{使劲工作一下}  
 ¶ \textcolor{darkblue}{\textbf{\ipa{ɬɑ˧pɤ˩ | ɖɯ˧-kʰwɤ˧ so˥}}} \textcolor{Sepia}{\selectlanguage{english}to study hard, to make headway in one's studies} \zh{努力学习一下}  

\lhead{\firstmark}
\rhead{\botmark}

\subsection{\hspace{-0.5cm} {\Large \textcolor{darkblue}{\textbf{\ipa{ɬɑ˧sɑ˧}}}}\hspace{0.5cm}[\kern2pt{\textcolor{darkblue}{\textbf{\ipa{ɬɑ˧sɑ˧}}}}\kern2pt]} \hypertarget{KA\string_MsA\string_M1}{}
\markboth{\textcolor{darkblue}{\textbf{\ipa{ɬɑ˧sɑ˧}}}}{}
\textcolor{teal}{\mytextsc{noun}} \hspace{4pt} Tone: M.
\textcolor{Sepia}{\selectlanguage{english}Lhasa.} \zh{拉萨。} 
\lhead{\firstmark}
\rhead{\botmark}

\subsection{\hspace{-0.5cm} {\Large \textcolor{darkblue}{\textbf{\ipa{ɬɑ˧tɑ˥}}}}\hspace{0.5cm}[\kern2pt{\textcolor{darkblue}{\textbf{\ipa{ɬɑ˧tɑ˥}}}}\kern2pt]} \hypertarget{KA\string_MtA\string_T1}{}
\markboth{\textcolor{darkblue}{\textbf{\ipa{ɬɑ˧tɑ˥}}}}{}
\textcolor{teal}{\mytextsc{noun}} \hspace{4pt} Tone: H\#.
\textit{\textcolor{Sepia}{\selectlanguage{english}archaic}} [\zh{古语}] \textcolor{Sepia}{\selectlanguage{english}Jerkin, leather vest.} \zh{皮革背心。}  \zh{量词}: \textcolor{darkblue}{\textbf{\ipa{ɭɯ˧}}}  \mytextsc{clf}: \textcolor{darkblue}{\textbf{\ipa{ɭɯ˧}}} 
\lhead{\firstmark}
\rhead{\botmark}

\subsection{\hspace{-0.5cm} {\Large \textcolor{darkblue}{\textbf{\ipa{ɬɑ˧tsʰo\#˥}}}}\hspace{0.5cm}[\kern2pt{\textcolor{darkblue}{\textbf{\ipa{ɬɑ˧tsʰo˧}}}}\kern2pt]} \hypertarget{KA\string_Mts\string_ho\#\string_T1}{}
\markboth{\textcolor{darkblue}{\textbf{\ipa{ɬɑ˧tsʰo\#˥}}}}{}
\textcolor{teal}{\mytextsc{noun}} \hspace{4pt} Tone: \#H.
\textcolor{Sepia}{\selectlanguage{english}Feminine given name.} \zh{女性名字。} 
\lhead{\firstmark}
\rhead{\botmark}

\subsection{\hspace{-0.5cm} {\Large \textcolor{darkblue}{\textbf{\ipa{ɬɑ˧˥}}}}\hspace{0.5cm}[\kern2pt{\textcolor{darkblue}{\textbf{\ipa{ɬɑ˧˥}}}}\kern2pt]} \hypertarget{KA\string_M\string_T1}{}
\markboth{\textcolor{darkblue}{\textbf{\ipa{ɬɑ˧˥}}}}{}
\textcolor{teal}{\mytextsc{adjective}} \hspace{4pt} Tone: MH.
\textcolor{Sepia}{\selectlanguage{english}Numerous, abundant, plentiful.} \zh{多、丰富、充分。}  ¶ \textcolor{darkblue}{\textbf{\ipa{dʑɤ˩-hĩ˩˥, | le˧-ɳɯ˥! | mɤ˧-dʑɤ˩-hĩ˩, | le˧-ɬɑ˧˥!}}} \textcolor{Sepia}{\selectlanguage{english}Good ones are few! Not-so-good ones are numerous! (Context: discussing universities, among which high-school graduates choose.)} \zh{好的,不多!不好的,就很多了!(情景:谈高中学生想入大学)}  

\lhead{\firstmark}
\rhead{\botmark}

\subsection{\hspace{-0.5cm} {\Large \textcolor{darkblue}{\textbf{\ipa{ɬi˥}}}}\hspace{0.5cm}[\kern2pt{\textcolor{darkblue}{\textbf{\ipa{ɬi˥}}}}\kern2pt]} \hypertarget{Ki\string_T1}{}
\markboth{\textcolor{darkblue}{\textbf{\ipa{ɬi˥}}}}{}
\textcolor{teal}{\mytextsc{verb}} \hspace{4pt} Tone: H.
\textcolor{Sepia}{\selectlanguage{english}To rest, to relax.} \zh{休息,松懈。}  ¶ \textcolor{darkblue}{\textbf{\ipa{le˧-ɬi˥}}} \textcolor{Sepia}{\selectlanguage{english}\mytextsc{accomp} \string_} \zh{\mytextsc{accomp} \string_}  

\lhead{\firstmark}
\rhead{\botmark}

\subsection{\hspace{-0.5cm} {\Large \textcolor{darkblue}{\textbf{\ipa{ɬi˧\textsubscript{b}}}}}\hspace{0.5cm}[\kern2pt{\textcolor{darkblue}{\textbf{\ipa{ɬi˩˥}}}}\kern2pt]} \hypertarget{Ki\string_Mb1}{}
\markboth{\textcolor{darkblue}{\textbf{\ipa{ɬi˧\textsubscript{b}}}}}{}
\textcolor{teal}{\mytextsc{classifier}} \hspace{4pt} Tone: M\textsubscript{b}.
\textcolor{Sepia}{\selectlanguage{english}Month.} \zh{量词:月。} 
\lhead{\firstmark}
\rhead{\botmark}

\subsection{\hspace{-0.5cm} {\Large \textcolor{darkblue}{\textbf{\ipa{ɬi˧bo\#˥}}}}\hspace{0.5cm}[\kern2pt{\textcolor{darkblue}{\textbf{\ipa{ɬi˩bo˩˥}}}}\kern2pt]} \hypertarget{Ki\string_Mbo\#\string_T1}{}
\markboth{\textcolor{darkblue}{\textbf{\ipa{ɬi˧bo\#˥}}}}{}
\textcolor{teal}{\mytextsc{noun}} \hspace{4pt} Tone: \#H.
\textcolor{Sepia}{\selectlanguage{english}Deaf person.} \zh{聋子。}  ¶ \textcolor{darkblue}{\textbf{\ipa{ɬi˧bo˧-hĩ˧}}} \textcolor{Sepia}{\selectlanguage{english}deaf person} \zh{耳朵聋的人}  
 \zh{量词}: \textcolor{darkblue}{\textbf{\ipa{v̩˧}}}  \mytextsc{clf}: \textcolor{darkblue}{\textbf{\ipa{v̩˧}}} 
\lhead{\firstmark}
\rhead{\botmark}

\subsection{\hspace{-0.5cm} {\Large \textcolor{darkblue}{\textbf{\ipa{ɬi˧bv̩˧}}}}\hspace{0.5cm}[\kern2pt{\textcolor{darkblue}{\textbf{\ipa{ɬi˧bv̩˧}}}}\kern2pt]} \hypertarget{Ki\string_Mbv\string_=\string_M1}{}
\markboth{\textcolor{darkblue}{\textbf{\ipa{ɬi˧bv̩˧}}}}{}
\textcolor{teal}{\mytextsc{noun}} \hspace{4pt} Tone: M.
\textcolor{Sepia}{\selectlanguage{english}Bai (ethnic group).} \zh{白族。}  \zh{量词}: \textcolor{darkblue}{\textbf{\ipa{v̩˧}}}  \mytextsc{clf}: \textcolor{darkblue}{\textbf{\ipa{v̩˧}}} 
\lhead{\firstmark}
\rhead{\botmark}

\subsection{\hspace{-0.5cm} {\Large \textcolor{darkblue}{\textbf{\ipa{ɬi˧bv̩˩ | dʑɤ˩tsʰi˧-si\#˥}}}}\hspace{0.5cm}[\kern2pt{\textcolor{darkblue}{\textbf{\ipa{xxxx non-correspondance entre le nombre de groupes tonals et le nombre de tons}}}}\kern2pt]} \hypertarget{Ki\string_Mbv\string_=\string_B | dz£7\string_Bts\string_hi\string_M-si\#\string_T1}{}
\markboth{\textcolor{darkblue}{\textbf{\ipa{ɬi˧bv̩˩ | dʑɤ˩tsʰi˧-si\#˥}}}}{}
\textcolor{teal}{\mytextsc{noun}} \hspace{4pt} Tone: L\# | LM+\#H.
\textcolor{Sepia}{\selectlanguage{english}Chinese toon, fragrant cedar, \textit{Ailanthus chinensis}.} \zh{香椿、香椿树。} 
\lhead{\firstmark}
\rhead{\botmark}

\subsection{\hspace{-0.5cm} {\Large \textcolor{darkblue}{\textbf{\ipa{ɬi˧bv̩˧-mi\#˥}}}}\hspace{0.5cm}[\kern2pt{\textcolor{darkblue}{\textbf{\ipa{xxxx non-correspondance entre le nombre de morphèmes et le nombre de tons de morphèmes}}}}\kern2pt]} \hypertarget{Ki\string_Mbv\string_=\string_M-mi\#\string_T1}{}
\markboth{\textcolor{darkblue}{\textbf{\ipa{ɬi˧bv̩˧-mi\#˥}}}}{}
\textcolor{teal}{\mytextsc{noun}} \hspace{4pt} Tone: \#H.
\textcolor{Sepia}{\selectlanguage{english}Bai woman.} \zh{白族女人。}  \zh{量词}: \textcolor{darkblue}{\textbf{\ipa{v̩˧}}}  \mytextsc{clf}: \textcolor{darkblue}{\textbf{\ipa{v̩˧}}} 
\lhead{\firstmark}
\rhead{\botmark}

\subsection{\hspace{-0.5cm} {\Large \textcolor{darkblue}{\textbf{\ipa{ɬi˧bv̩˧-zo\#˥}}}}\hspace{0.5cm}[\kern2pt{\textcolor{darkblue}{\textbf{\ipa{xxxx non-correspondance entre le nombre de morphèmes et le nombre de tons de morphèmes}}}}\kern2pt]} \hypertarget{Ki\string_Mbv\string_=\string_M-zo\#\string_T1}{}
\markboth{\textcolor{darkblue}{\textbf{\ipa{ɬi˧bv̩˧-zo\#˥}}}}{}
\textcolor{teal}{\mytextsc{noun}} \hspace{4pt} Tone: \#H.
\textcolor{Sepia}{\selectlanguage{english}Bai man.} Local Chinese dialect:\zh{白族男人。} \zh{量词}: \textcolor{darkblue}{\textbf{\ipa{v̩˧}}}  \mytextsc{clf}: \textcolor{darkblue}{\textbf{\ipa{v̩˧}}} 
\lhead{\firstmark}
\rhead{\botmark}

\subsection{\hspace{-0.5cm} {\Large \textcolor{darkblue}{\textbf{\ipa{ɬi˧di˩}}}}\hspace{0.5cm}[\kern2pt{\textcolor{darkblue}{\textbf{\ipa{ɬi˧di˧}}}}\kern2pt]} \hypertarget{Ki\string_Mdi\string_B1}{}
\markboth{\textcolor{darkblue}{\textbf{\ipa{ɬi˧di˩}}}}{}
\textcolor{teal}{\mytextsc{noun}} \hspace{4pt} Tone: L\#.
\textcolor{Sepia}{\selectlanguage{english}Yongning (place name).} \zh{永宁(地名)。} 
\lhead{\firstmark}
\rhead{\botmark}

\subsection{\hspace{-0.5cm} {\Large \textcolor{darkblue}{\textbf{\ipa{ɬi˧di˩-hĩ˩}}}}\hspace{0.5cm}[\kern2pt{\textcolor{darkblue}{\textbf{\ipa{xxxx non-correspondance entre le nombre de morphèmes et le nombre de tons de morphèmes}}}}\kern2pt]} \hypertarget{Ki\string_Mdi\string_B-hi\string_~\string_B1}{}
\markboth{\textcolor{darkblue}{\textbf{\ipa{ɬi˧di˩-hĩ˩}}}}{}
\textcolor{teal}{\mytextsc{noun}} \hspace{4pt} Tone: L\#-.
\textcolor{Sepia}{\selectlanguage{english}People of Yongning. Unless otherwise specified, this is mainly understood as referring to the Na (Mosuo).} \zh{永宁人(纳人)。}  \zh{量词}: \textcolor{darkblue}{\textbf{\ipa{v̩˧}}}  \mytextsc{clf}: \textcolor{darkblue}{\textbf{\ipa{v̩˧}}} 
\lhead{\firstmark}
\rhead{\botmark}

\subsection{\hspace{-0.5cm} {\Large \textcolor{darkblue}{\textbf{\ipa{ɬi˧dʑɯ˩}}}}\hspace{0.5cm}[\kern2pt{\textcolor{darkblue}{\textbf{\ipa{xxxx non-correspondance entre le nombre de morphèmes et le nombre de tons de morphèmes}}}}\kern2pt]} \hypertarget{Ki\string_Mdz£M\string_B1}{}
\markboth{\textcolor{darkblue}{\textbf{\ipa{ɬi˧dʑɯ˩}}}}{}
\textcolor{teal}{\mytextsc{noun}} \hspace{4pt} Tone: L\#.
\textcolor{Sepia}{\selectlanguage{english}The river that flows through the plain of Yongning.} \zh{永宁坝子的河流。}  \zh{量词}: \textcolor{darkblue}{\textbf{\ipa{kʰɯ˩}}}  \mytextsc{clf}: \textcolor{darkblue}{\textbf{\ipa{kʰɯ˩}}} 
\lhead{\firstmark}
\rhead{\botmark}

\subsection{\hspace{-0.5cm} {\Large \textcolor{darkblue}{\textbf{\ipa{ɬi˧gv̩\#˥}}}}\hspace{0.5cm}[\kern2pt{\textcolor{darkblue}{\textbf{\ipa{ɬi˧gv̩˩}}}}\kern2pt]} \hypertarget{Ki\string_Mgv\string_=\#\string_T1}{}
\markboth{\textcolor{darkblue}{\textbf{\ipa{ɬi˧gv̩\#˥}}}}{}
\textcolor{teal}{\mytextsc{noun}} \hspace{4pt} Tone: \#H.
\textcolor{Sepia}{\selectlanguage{english}Middle part; (in) the centre.} \zh{中部,中间。}  ¶ \textcolor{darkblue}{\textbf{\ipa{ɬi˧gv̩˧ dzi˥}}} \textcolor{Sepia}{\selectlanguage{english}to sit in the centre} \zh{坐在中间}  

\lhead{\firstmark}
\rhead{\botmark}

\subsection{\hspace{-0.5cm} {\Large \textcolor{darkblue}{\textbf{\ipa{ɬi˧hĩ\#˥}}} \textsubscript{1}}\hspace{0.5cm}[\kern2pt{\textcolor{darkblue}{\textbf{\ipa{ɬi˧hĩ˧}}}}\kern2pt]} \hypertarget{Ki\string_Mhi\string_~\#\string_T1}{}
\markboth{\textcolor{darkblue}{\textbf{\ipa{ɬi˧hĩ\#˥}}} \textsubscript{1}}{}
\textcolor{teal}{\mytextsc{noun}} \hspace{4pt} Tone: \#H.
\textcolor{Sepia}{\selectlanguage{english}Man in middle position among siblings: neither eldest brother nor youngest brother; literal translation: “person in the middle”.} \zh{兄弟里面夹中的男孩(上有哥哥下有弟弟的孩子)。} 
\lhead{\firstmark}
\rhead{\botmark}

\subsection{\hspace{-0.5cm} {\Large \textcolor{darkblue}{\textbf{\ipa{ɬi˧hĩ\#˥}}} \textsubscript{2}}\hspace{0.5cm}[\kern2pt{\textcolor{darkblue}{\textbf{\ipa{ɬi˧hĩ˧}}}}\kern2pt]} \hypertarget{Ki\string_Mhi\string_~\#\string_T2}{}
\markboth{\textcolor{darkblue}{\textbf{\ipa{ɬi˧hĩ\#˥}}} \textsubscript{2}}{}
\textcolor{teal}{\mytextsc{noun}} \hspace{4pt} Tone: \#H.
\textcolor{Sepia}{\selectlanguage{english}Inhabitant of Yongning; as used by the main consultant, the term includes Pumi (Prinmi) people along with Na people.} \zh{永宁的人。} 
\lhead{\firstmark}
\rhead{\botmark}

\subsection{\hspace{-0.5cm} {\Large \textcolor{darkblue}{\textbf{\ipa{ɬi˧ki\#˥}}}}\hspace{0.5cm}[\kern2pt{\textcolor{darkblue}{\textbf{\ipa{ɬi˧ki˥}}}}\kern2pt]} \hypertarget{Ki\string_Mki\#\string_T1}{}
\markboth{\textcolor{darkblue}{\textbf{\ipa{ɬi˧ki\#˥}}}}{}
\textcolor{teal}{\mytextsc{noun}} \hspace{4pt} Tone: \#H.
\textcolor{Sepia}{\selectlanguage{english}The name of a Na village, outside the plain of Yongning, close to the Lake.} \zh{泸沽湖附近的一个村落。}  ¶ \textcolor{darkblue}{\textbf{\ipa{ɬi˧ki˧, | ɲi˧se˩, | tɑ˧dzi˩, | mv̩˧qʰwæ˩, | lɑ˧tʰɑ˧-di˧˥}}} \textcolor{Sepia}{\selectlanguage{english}Villages that one passes when moving away from the Yongning plain, towards Lugu lake. These villages do not count as part of Yongning proper. The last, \textcolor{darkblue}{\textbf{\ipa{/lɑ˧tʰɑ˧-di˧˥/}}}, is not a village name like the preceding four: it refers to the entire Na area beyond the fourth village.} \zh{永宁到泸沽湖所经过的村落,依次是:里格、尼赛、大祖、木垮,然后到拉塔地(拉塔地指的是泸沽湖周边的摩梭地区,包括左所、洛水村等)}  

\lhead{\firstmark}
\rhead{\botmark}

\subsection{\hspace{-0.5cm} {\Large \textcolor{darkblue}{\textbf{\ipa{ɬi˧ki˥}}}}\hspace{0.5cm}[\kern2pt{\textcolor{darkblue}{\textbf{\ipa{ɬi˧ki˧}}}}\kern2pt]} \hypertarget{Ki\string_Mki\string_T1}{}
\markboth{\textcolor{darkblue}{\textbf{\ipa{ɬi˧ki˥}}}}{}
\textcolor{teal}{\mytextsc{noun}} \hspace{4pt} Tone: H\#.
\textcolor{Sepia}{\selectlanguage{english}Ritual for boys coming of age, i.e. reaching the age of 13 years: “wearing trousers”; at that age adolescents begin to wear trousers instead of children's robes.} \zh{男性成年礼:直译“穿裤”。} 
\lhead{\firstmark}
\rhead{\botmark}

\subsection{\hspace{-0.5cm} {\Large \textcolor{darkblue}{\textbf{\ipa{ɬi˧mi˧}}} \textsubscript{1}}\hspace{0.5cm}[\kern2pt{\textcolor{darkblue}{\textbf{\ipa{ɬi˧mi˧}}}}\kern2pt]} \hypertarget{Ki\string_Mmi\string_M1}{}
\markboth{\textcolor{darkblue}{\textbf{\ipa{ɬi˧mi˧}}} \textsubscript{1}}{}
\textcolor{teal}{\mytextsc{noun}} \hspace{4pt} Tone: M.
\ding{202} \textcolor{Sepia}{\selectlanguage{english}Moon (disyllable).} \zh{月亮(双音节)。}  \zh{量词}: \textcolor{darkblue}{\textbf{\ipa{ɭɯ˧}}} \ding{203} \textcolor{Sepia}{\selectlanguage{english}Month (disyllable).} \zh{月(双音节)。}  ¶ \textcolor{darkblue}{\textbf{\ipa{ɬi˧mi˧ ɖɯ˧-gi˥}}} \textcolor{Sepia}{\selectlanguage{english}half a month} \zh{半个月}  
 ¶ \textcolor{darkblue}{\textbf{\ipa{ɬi˧mi˧ le˧-gv̩˩}}} \textcolor{Sepia}{\selectlanguage{english}the latter half of the month; literally 'the declining half of the month'} \zh{下半月份}  
 \mytextsc{clf}: \textcolor{darkblue}{\textbf{\ipa{ɭɯ˧}}} 
\lhead{\firstmark}
\rhead{\botmark}

\subsection{\hspace{-0.5cm} {\Large \textcolor{darkblue}{\textbf{\ipa{ɬi˧mi˧}}} \textsubscript{2}}\hspace{0.5cm}[\kern2pt{\textcolor{darkblue}{\textbf{\ipa{ɬi˧mi˧}}}}\kern2pt]} \hypertarget{Ki\string_Mmi\string_M2}{}
\markboth{\textcolor{darkblue}{\textbf{\ipa{ɬi˧mi˧}}} \textsubscript{2}}{}
\textcolor{teal}{\mytextsc{noun}} \hspace{4pt} Tone: M.
\textcolor{Sepia}{\selectlanguage{english}Female roebuck.} \zh{母獐子。}  \zh{量词}: \textcolor{darkblue}{\textbf{\ipa{v̩˧}}}  \mytextsc{clf}: \textcolor{darkblue}{\textbf{\ipa{v̩˧}}} 
\lhead{\firstmark}
\rhead{\botmark}

\subsection{\hspace{-0.5cm} {\Large \textcolor{darkblue}{\textbf{\ipa{ɬi˧mi˧dɑ˧dzɯ\#˥}}}}\hspace{0.5cm}[\kern2pt{\textcolor{darkblue}{\textbf{\ipa{ɬi˧mi˧dɑ˧dzɯ˧}}}}\kern2pt]} \hypertarget{Ki\string_Mmi\string_MdA\string_MdzM\#\string_T1}{}
\markboth{\textcolor{darkblue}{\textbf{\ipa{ɬi˧mi˧dɑ˧dzɯ\#˥}}}}{}
\textcolor{teal}{\mytextsc{noun}} \hspace{4pt} Tone: \#H.
\textcolor{Sepia}{\selectlanguage{english}Lunar eclipse.} \zh{月蚀。}  ¶ \textcolor{darkblue}{\textbf{\ipa{ɬi˧mi˧dɑ˧dzɯ˧ tʰv̩˧}}} \textcolor{Sepia}{\selectlanguage{english}there is a lunar eclipse} \zh{有月蚀}  
 ¶ \textcolor{darkblue}{\textbf{\ipa{ʈʂʰɯ˧ | ɬi˧mi˧dɑ˧dzɯ˧ ɲi˥!}}} \textcolor{Sepia}{\selectlanguage{english}This is a lunar eclipse! (Answer to the question 'What is happening? / What is that supposed to mean?')} \zh{这是月蚀!(一个人问:‘这是怎么回事?’,另一个回答:‘这是月蚀!’)}  
 \zh{量词}: \textcolor{darkblue}{\textbf{\ipa{ʂɯ˩}}}  \mytextsc{clf}: \textcolor{darkblue}{\textbf{\ipa{ʂɯ˩}}} 
\lhead{\firstmark}
\rhead{\botmark}

\subsection{\hspace{-0.5cm} {\Large \textcolor{darkblue}{\textbf{\ipa{ɬi˧ɳæ˩}}}}\hspace{0.5cm}[\kern2pt{\textcolor{darkblue}{\textbf{\ipa{ɬi˧ɳæ˧}}}}\kern2pt]} \hypertarget{Ki\string_Mn`\{\string_B1}{}
\markboth{\textcolor{darkblue}{\textbf{\ipa{ɬi˧ɳæ˩}}}}{}
\textcolor{teal}{\mytextsc{noun}} \hspace{4pt} Tone: L\#.
\textcolor{Sepia}{\selectlanguage{english}Menses; period.} \zh{月经。}  ¶ \textcolor{darkblue}{\textbf{\ipa{ʈʂʰɯ˧ | ɬi˧ɳæ˩-ze˩}}} \textcolor{Sepia}{\selectlanguage{english}She is having her menses.} \zh{她来了月经。}  
 ¶ \textcolor{darkblue}{\textbf{\ipa{ɬi˧ɳæ˩ go˩}}} \textcolor{Sepia}{\selectlanguage{english}to have painful menses} \zh{来了月经,疼}  
 \zh{量词}: \textcolor{darkblue}{\textbf{\ipa{ɬi˧}}}  \mytextsc{clf}: \textcolor{darkblue}{\textbf{\ipa{ɬi˧}}} 
\lhead{\firstmark}
\rhead{\botmark}

\subsection{\hspace{-0.5cm} {\Large \textcolor{darkblue}{\textbf{\ipa{ɬi˧pæ˥}}}}\hspace{0.5cm}[\kern2pt{\textcolor{darkblue}{\textbf{\ipa{ɬi˧pæ˩}}}}\kern2pt]} \hypertarget{Ki\string_Mp\{\string_T1}{}
\markboth{\textcolor{darkblue}{\textbf{\ipa{ɬi˧pæ˥}}}}{}
\textcolor{teal}{\mytextsc{noun}} \hspace{4pt} Tone: H\#.
\textcolor{Sepia}{\selectlanguage{english}Earring.} \zh{耳环。}  ¶ \textcolor{darkblue}{\textbf{\ipa{ŋv̩˩-ɬi˩pæ˥ (+ɲi˩)}}} \textcolor{Sepia}{\selectlanguage{english}silver earring} \zh{银耳环}  
 ¶ \textcolor{darkblue}{\textbf{\ipa{hæ̃˩-ɬi˩pæ˥ (+ɲi˩)}}} \textcolor{Sepia}{\selectlanguage{english}gold earring} \zh{金耳环}  
 \zh{量词}: \textcolor{darkblue}{\textbf{\ipa{dze˩}}}  \mytextsc{clf}: \textcolor{darkblue}{\textbf{\ipa{dze˩}}} 
\lhead{\firstmark}
\rhead{\botmark}

\subsection{\hspace{-0.5cm} {\Large \textcolor{darkblue}{\textbf{\ipa{ɬi˧pi˩}}}}\hspace{0.5cm}[\kern2pt{\textcolor{darkblue}{\textbf{\ipa{ɬi˩pi˥}}}}\kern2pt]} \hypertarget{Ki\string_Mpi\string_B1}{}
\markboth{\textcolor{darkblue}{\textbf{\ipa{ɬi˧pi˩}}}}{}
\textcolor{teal}{\mytextsc{noun}} \hspace{4pt} Tone: L\#.
\textcolor{Sepia}{\selectlanguage{english}Ear.} \zh{耳朵。}  \zh{量词}: \textcolor{darkblue}{\textbf{\ipa{pʰo˧˥}}}  \mytextsc{clf}: \textcolor{darkblue}{\textbf{\ipa{pʰo˧˥}}} 
\lhead{\firstmark}
\rhead{\botmark}

\subsection{\hspace{-0.5cm} {\Large \textcolor{darkblue}{\textbf{\ipa{ɬi˧pv̩˧lv̩˥}}}}\hspace{0.5cm}[\kern2pt{\textcolor{darkblue}{\textbf{\ipa{ɬi˩pv̩˩lv̩˩˥}}}}\kern2pt]} \hypertarget{Ki\string_Mpv\string_=\string_Mlv\string_=\string_T1}{}
\markboth{\textcolor{darkblue}{\textbf{\ipa{ɬi˧pv̩˧lv̩˥}}}}{}
\textcolor{teal}{\mytextsc{noun}} \hspace{4pt} Tone: H\#.
\textcolor{Sepia}{\selectlanguage{english}Ear tumour, pathological excrescence of the ear.} \zh{耳朵瘤。}  \zh{量词}: \textcolor{darkblue}{\textbf{\ipa{ɭɯ˧}}}  \mytextsc{clf}: \textcolor{darkblue}{\textbf{\ipa{ɭɯ˧}}} 
\lhead{\firstmark}
\rhead{\botmark}

\subsection{\hspace{-0.5cm} {\Large \textcolor{darkblue}{\textbf{\ipa{ɬi˧pʰv̩\#˥}}}}\hspace{0.5cm}[\kern2pt{\textcolor{darkblue}{\textbf{\ipa{ɬi˧pʰv̩˥}}}}\kern2pt]} \hypertarget{Ki\string_Mp\string_hv\string_=\#\string_T1}{}
\markboth{\textcolor{darkblue}{\textbf{\ipa{ɬi˧pʰv̩\#˥}}}}{}
\textcolor{teal}{\mytextsc{noun}} \hspace{4pt} Tone: \#H.
\textcolor{Sepia}{\selectlanguage{english}Male roebuck, male hornless river deer.} \zh{公獐子。}  ¶ \textcolor{darkblue}{\textbf{\ipa{ɬi˧pʰv̩˧ tʰv̩˧-mi˥\# / ɬi˧pʰv̩˧ tʰv̩˧-mi˧˥}}} \textcolor{Sepia}{\selectlanguage{english}\mytextsc{n}+\mytextsc{dem}+\mytextsc{clf}} \zh{那只公獐子}  
 \zh{量词}: \textcolor{darkblue}{\textbf{\ipa{v̩˧}}} \textcolor{darkblue}{\textbf{\ipa{ɭɯ˧}}} \textcolor{darkblue}{\textbf{\ipa{mi˩}}}  \mytextsc{clf}: \textcolor{darkblue}{\textbf{\ipa{v̩˧}}} \textcolor{darkblue}{\textbf{\ipa{ɭɯ˧}}} \textcolor{darkblue}{\textbf{\ipa{mi˩}}} 
\lhead{\firstmark}
\rhead{\botmark}

\subsection{\hspace{-0.5cm} {\Large \textcolor{darkblue}{\textbf{\ipa{ɬi˧qʰæ\#˥}}}}\hspace{0.5cm}[\kern2pt{\textcolor{darkblue}{\textbf{\ipa{ɬi˧qʰæ˧}}}}\kern2pt]} \hypertarget{Ki\string_Mq\string_h\{\#\string_T1}{}
\markboth{\textcolor{darkblue}{\textbf{\ipa{ɬi˧qʰæ\#˥}}}}{}
\textcolor{teal}{\mytextsc{noun}} \hspace{4pt} Tone: \#H.
\textcolor{Sepia}{\selectlanguage{english}Earwax.} \zh{耳垢。}  \zh{量词}: \textcolor{darkblue}{\textbf{\ipa{kʰwɤ˥}}}  \mytextsc{clf}: \textcolor{darkblue}{\textbf{\ipa{kʰwɤ˥}}} 
\lhead{\firstmark}
\rhead{\botmark}

\subsection{\hspace{-0.5cm} {\Large \textcolor{darkblue}{\textbf{\ipa{ɬi˧qʰv̩\#˥}}}}\hspace{0.5cm}[\kern2pt{\textcolor{darkblue}{\textbf{\ipa{ɬi˧qʰv̩˧}}}}\kern2pt]} \hypertarget{Ki\string_Mq\string_hv\string_=\#\string_T1}{}
\markboth{\textcolor{darkblue}{\textbf{\ipa{ɬi˧qʰv̩\#˥}}}}{}
\textcolor{teal}{\mytextsc{noun}} \hspace{4pt} Tone: \#H.
\textcolor{Sepia}{\selectlanguage{english}Auditory canal.} \zh{耳孔。}  ¶ \textcolor{darkblue}{\textbf{\ipa{ʈʂʰɯ˧ | ɬi˧qʰv̩˧ | ɖɯ˧-pi˧˥ | tʰɑ˩˥!}}} \textcolor{Sepia}{\selectlanguage{english}She has a sensitive ear! (Context: about a 2-year old girl who wakes up from her siesta as soon as guests come in.)} \zh{她耳朵很好使! / 她耳朵很尖!(情景:一有客人到家的声音,睡午觉的两岁女孩子立即醒来。)}  
 \zh{量词}: \textcolor{darkblue}{\textbf{\ipa{ɭɯ˧}}}  \mytextsc{clf}: \textcolor{darkblue}{\textbf{\ipa{ɭɯ˧}}} 
\lhead{\firstmark}
\rhead{\botmark}

\subsection{\hspace{-0.5cm} {\Large \textcolor{darkblue}{\textbf{\ipa{ɬi˧ʈv̩˥}}}}\hspace{0.5cm}[\kern2pt{\textcolor{darkblue}{\textbf{\ipa{ɬi˩ʈv̩˥}}}}\kern2pt]} \hypertarget{Ki\string_Mt`v\string_=\string_T1}{}
\markboth{\textcolor{darkblue}{\textbf{\ipa{ɬi˧ʈv̩˥}}}}{}
\textcolor{teal}{\mytextsc{noun}} \hspace{4pt} Tone: H\#.
\textcolor{Sepia}{\selectlanguage{english}Asiatic plantain.} \zh{车前草。}  \zh{量词}: \textcolor{darkblue}{\textbf{\ipa{po˧}}}  \mytextsc{clf}: \textcolor{darkblue}{\textbf{\ipa{po˧}}} 
\lhead{\firstmark}
\rhead{\botmark}

\subsection{\hspace{-0.5cm} {\Large \textcolor{darkblue}{\textbf{\ipa{ɬi˩}}} \textsubscript{1}}\hspace{0.5cm}[\kern2pt{\textcolor{darkblue}{\textbf{\ipa{ɬi˩˥}}}}\kern2pt]} \hypertarget{Ki\string_B1}{}
\markboth{\textcolor{darkblue}{\textbf{\ipa{ɬi˩}}} \textsubscript{1}}{}
\textcolor{teal}{\mytextsc{verb}} \hspace{4pt} Tone: L\textsubscript{a}.
\textcolor{Sepia}{\selectlanguage{english}To measure (e.g. a piece of fabric) to find its length, in armspans.} \zh{量(一块布料……)有多长:有多少庹。}  ¶ \textcolor{darkblue}{\textbf{\ipa{ɬi˩-se˥ (-ze˩)}}} \textcolor{Sepia}{\selectlanguage{english}\string_ \mytextsc{achev} (\mytextsc{pfv})} \zh{量完(了)}  

\lhead{\firstmark}
\rhead{\botmark}

\subsection{\hspace{-0.5cm} {\Large \textcolor{darkblue}{\textbf{\ipa{ɬi˩}}} \textsubscript{2}}\hspace{0.5cm}[\kern2pt{\textcolor{darkblue}{\textbf{\ipa{ɬi˥}}}}\kern2pt]} \hypertarget{Ki\string_B2}{}
\markboth{\textcolor{darkblue}{\textbf{\ipa{ɬi˩}}} \textsubscript{2}}{}
\textcolor{teal}{\mytextsc{noun}} \hspace{4pt} Tone: L.
\textcolor{Sepia}{\selectlanguage{english}Roebuck, hornless river deer.} \zh{獐子。}  \zh{量词}: \textcolor{darkblue}{\textbf{\ipa{pʰo˧˥}}} \textcolor{darkblue}{\textbf{\ipa{mi˩}}}  \mytextsc{clf}: \textcolor{darkblue}{\textbf{\ipa{pʰo˧˥}}} \textcolor{darkblue}{\textbf{\ipa{mi˩}}} 
\lhead{\firstmark}
\rhead{\botmark}

\subsection{\hspace{-0.5cm} {\Large \textcolor{darkblue}{\textbf{\ipa{ɬi˩\textsubscript{b}}}}}\hspace{0.5cm}[\kern2pt{\textcolor{darkblue}{\textbf{\ipa{ɬi˥}}}}\kern2pt]} \hypertarget{Ki\string_Bb1}{}
\markboth{\textcolor{darkblue}{\textbf{\ipa{ɬi˩\textsubscript{b}}}}}{}
\textcolor{teal}{\mytextsc{classifier}} \hspace{4pt} Tone: L\textsubscript{b}.
\textcolor{Sepia}{\selectlanguage{english}A span, an armspread.} \zh{量词:庹。}  ¶ \textcolor{darkblue}{\textbf{\ipa{tsʰe˧-ɬi˧}}} \textcolor{Sepia}{\selectlanguage{english}10 spans, 10 armspreads} \zh{十庹}  

\lhead{\firstmark}
\rhead{\botmark}

\subsection{\hspace{-0.5cm} {\Large \textcolor{darkblue}{\textbf{\ipa{ɬi˩bi˩}}}}\hspace{0.5cm}[\kern2pt{\textcolor{darkblue}{\textbf{\ipa{ɬi˧bi˧}}}}\kern2pt]} \hypertarget{Ki\string_Bbi\string_B1}{}
\markboth{\textcolor{darkblue}{\textbf{\ipa{ɬi˩bi˩}}}}{}
\textcolor{teal}{\mytextsc{noun}} \hspace{4pt} Tone: L.
\textcolor{Sepia}{\selectlanguage{english}Turnip; radish.} \zh{萝卜。}  \zh{量词}: \textcolor{darkblue}{\textbf{\ipa{ɭɯ˧}}}  \mytextsc{clf}: \textcolor{darkblue}{\textbf{\ipa{ɭɯ˧}}} 
\lhead{\firstmark}
\rhead{\botmark}

\subsection{\hspace{-0.5cm} {\Large \textcolor{darkblue}{\textbf{\ipa{ɬi˩qʰwɤ˩}}}}\hspace{0.5cm}[\kern2pt{\textcolor{darkblue}{\textbf{\ipa{ɬi˧qʰwɤ˧}}}}\kern2pt]} \hypertarget{Ki\string_Bq\string_hw7\string_B1}{}
\markboth{\textcolor{darkblue}{\textbf{\ipa{ɬi˩qʰwɤ˩}}}}{}
\textcolor{teal}{\mytextsc{noun}} \hspace{4pt} Tone: L.
\textcolor{Sepia}{\selectlanguage{english}Trousers.} \zh{裤子。}  \zh{量词}: \textcolor{darkblue}{\textbf{\ipa{ɭɯ˧}}}  \mytextsc{clf}: \textcolor{darkblue}{\textbf{\ipa{ɭɯ˧}}} 
\lhead{\firstmark}
\rhead{\botmark}

\subsection{\hspace{-0.5cm} {\Large \textcolor{darkblue}{\textbf{\ipa{ɬi˩ʁɑ˩}}}}\hspace{0.5cm}[\kern2pt{\textcolor{darkblue}{\textbf{\ipa{ɬi˩ʁɑ˩˥}}}}\kern2pt]} \hypertarget{Ki\string_BRA\string_B1}{}
\markboth{\textcolor{darkblue}{\textbf{\ipa{ɬi˩ʁɑ˩}}}}{}
\textcolor{teal}{\mytextsc{adjective}} \hspace{4pt} Tone: L.
\textcolor{Sepia}{\selectlanguage{english}Infuriated, in a rage (connotation: attitude of a violent and overbearing person).} \zh{大发雷霆。}  ¶ \textcolor{darkblue}{\textbf{\ipa{ɬi˩ʁɑ˩ ʝi˧}}} \textcolor{Sepia}{\selectlanguage{english}to abandon oneself to one's rage} \zh{大发雷霆}  

\lhead{\firstmark}
\rhead{\botmark}

\subsection{\hspace{-0.5cm} {\Large \textcolor{darkblue}{\textbf{\ipa{ɬi˩ʈɯ˩mæ˥}}}}\hspace{0.5cm}[\kern2pt{\textcolor{darkblue}{\textbf{\ipa{ɬi˩ʈɯ˩mæ˩˥}}}}\kern2pt]} \hypertarget{Ki\string_Bt`M\string_Bm\{\string_T1}{}
\markboth{\textcolor{darkblue}{\textbf{\ipa{ɬi˩ʈɯ˩mæ˥}}}}{}
\textcolor{teal}{\mytextsc{noun}} \hspace{4pt} Tone: L+H\#.
\textcolor{Sepia}{\selectlanguage{english}Lower part of the main room.} \zh{主屋里面没有火铺的地方:没有木地板、小狗可以偶尔进来的地方(家人就给它扔骨头)。}  ¶ \textcolor{darkblue}{\textbf{\ipa{u˧=ɻ̍˩, | kʰv̩˩mi˩ ʈʂʰɯ˩-jɤ˥ | ɖɯ˧-njɤ˧-zo˥ | ɬi˩ʈɯ˩mæ˥ hĩ˩ dʑo˩.}}} \textcolor{Sepia}{\selectlanguage{english}Us (=in our family), this dog is often seated in the lower part of the room.} \zh{咱们家这只狗经常呆在主屋火塘下面的地方。}  
 \zh{量词}: \textcolor{darkblue}{\textbf{\ipa{kʰwɤ˥}}}  \mytextsc{clf}: \textcolor{darkblue}{\textbf{\ipa{kʰwɤ˥}}} 
\lhead{\firstmark}
\rhead{\botmark}

\subsection{\hspace{-0.5cm} {\Large \textcolor{darkblue}{\textbf{\ipa{ɬi˩zo˩}}}}\hspace{0.5cm}[\kern2pt{\textcolor{darkblue}{\textbf{\ipa{ɬi˧zo˥}}}}\kern2pt]} \hypertarget{Ki\string_Bzo\string_B1}{}
\markboth{\textcolor{darkblue}{\textbf{\ipa{ɬi˩zo˩}}}}{}
\textcolor{teal}{\mytextsc{noun}} \hspace{4pt} Tone: L.
\textcolor{Sepia}{\selectlanguage{english}Baby roebuck.} \zh{小獐子。} 
\lhead{\firstmark}
\rhead{\botmark}

\subsection{\hspace{-0.5cm} {\Large \textcolor{darkblue}{\textbf{\ipa{ɬi˧˥}}}}\hspace{0.5cm}[\kern2pt{\textcolor{darkblue}{\textbf{\ipa{ɬi˧˥}}}}\kern2pt]} \hypertarget{Ki\string_M\string_T1}{}
\markboth{\textcolor{darkblue}{\textbf{\ipa{ɬi˧˥}}}}{}
\textcolor{teal}{\mytextsc{verb}} \hspace{4pt} Tone: MH.
\textcolor{Sepia}{\selectlanguage{english}To dry in the sun.} \zh{晒干。}  ¶ \textcolor{darkblue}{\textbf{\ipa{le˧-pv̩˧ tʰi˧-ɬi˧˥}}} \textcolor{Sepia}{\selectlanguage{english}to put in the sun to dry} \zh{晒干}  

\lhead{\firstmark}
\rhead{\botmark}

\subsection{\hspace{-0.5cm} {\Large \textcolor{darkblue}{\textbf{\ipa{ɬo˥}}}}\hspace{0.5cm}[\kern2pt{\textcolor{darkblue}{\textbf{\ipa{ɬo˧˥}}}}\kern2pt]} \hypertarget{Ko\string_T1}{}
\markboth{\textcolor{darkblue}{\textbf{\ipa{ɬo˥}}}}{}
\textcolor{teal}{\mytextsc{noun}} \hspace{4pt} Tone: \#H.
\textcolor{Sepia}{\selectlanguage{english}Rib.} \zh{肋骨。}  ¶ \textcolor{darkblue}{\textbf{\ipa{bo˩ɬo˧}}} \textcolor{Sepia}{\selectlanguage{english}pork rib} \zh{猪肋骨}  
 \zh{量词}: \textcolor{darkblue}{\textbf{\ipa{ɭɯ˧}}}  \mytextsc{clf}: \textcolor{darkblue}{\textbf{\ipa{ɭɯ˧}}} 
\lhead{\firstmark}
\rhead{\botmark}

\subsection{\hspace{-0.5cm} {\Large \textcolor{darkblue}{\textbf{\ipa{ɬo˧kʰv̩˧}}}}\hspace{0.5cm}[\kern2pt{\textcolor{darkblue}{\textbf{\ipa{ɬo˧kʰv̩˧}}}}\kern2pt]} \hypertarget{Ko\string_Mk\string_hv\string_=\string_M1}{}
\markboth{\textcolor{darkblue}{\textbf{\ipa{ɬo˧kʰv̩˧}}}}{}
\textcolor{teal}{\mytextsc{noun}} \hspace{4pt} Tone: M.
\textcolor{Sepia}{\selectlanguage{english}Hip.} \zh{胯。}  \zh{量词}: \textcolor{darkblue}{\textbf{\ipa{ɭɯ˧}}}  \mytextsc{clf}: \textcolor{darkblue}{\textbf{\ipa{ɭɯ˧}}} \textit{Syn:} \hyperlink{}{\textcolor{darkblue}{\textbf{\ipa{ɬo˩tsʰe˩mæ˥}}}}. 
\lhead{\firstmark}
\rhead{\botmark}

\subsection{\hspace{-0.5cm} {\Large \textcolor{darkblue}{\textbf{\ipa{ɬo˧pɤ˥}}}}\hspace{0.5cm}[\kern2pt{\textcolor{darkblue}{\textbf{\ipa{ɬo˧pɤ˥}}}}\kern2pt]} \hypertarget{Ko\string_Mp7\string_T1}{}
\markboth{\textcolor{darkblue}{\textbf{\ipa{ɬo˧pɤ˥}}}}{}
\textcolor{teal}{\mytextsc{noun}} \hspace{4pt} Tone: H\#.
\textcolor{Sepia}{\selectlanguage{english}Blister (on the hands or feet).} \zh{水泡。}  ¶ \textcolor{darkblue}{\textbf{\ipa{ɬo˧pɤ˥ qʰwæ˩-ze˩!}}} \textcolor{Sepia}{\selectlanguage{english}(I/you/(s)he) got a blister!} \zh{起了水泡!}  
 ¶ \textcolor{darkblue}{\textbf{\ipa{ɬo˧pɤ˥ | ɖɯ˧-ɭɯ˧ | qʰwæ˧-ze˥!}}} \textcolor{Sepia}{\selectlanguage{english}(I/you/(s)he) got a blister!} \zh{起了一个水泡!}  
 ¶ \textcolor{darkblue}{\textbf{\ipa{ɬo˧pɤ˥ | ʁo˩-po˥-ɳɯ˩ | ʈʂe˩˥}}} \textcolor{Sepia}{\selectlanguage{english}to pierce a blister with a needle} \zh{用针来扎水泡}  
 \zh{量词}: \textcolor{darkblue}{\textbf{\ipa{ɭɯ˧}}}  \mytextsc{clf}: \textcolor{darkblue}{\textbf{\ipa{ɭɯ˧}}} 
\lhead{\firstmark}
\rhead{\botmark}

\subsection{\hspace{-0.5cm} {\Large \textcolor{darkblue}{\textbf{\ipa{ɬo˧pv̩˥}}}}\hspace{0.5cm}[\kern2pt{\textcolor{darkblue}{\textbf{\ipa{ɬo˧pv̩˥}}}}\kern2pt]} \hypertarget{Ko\string_Mpv\string_=\string_T1}{}
\markboth{\textcolor{darkblue}{\textbf{\ipa{ɬo˧pv̩˥}}}}{}
\textcolor{teal}{\mytextsc{noun}} \hspace{4pt} Tone: H\#.
\textcolor{Sepia}{\selectlanguage{english}Kow-tow.} \zh{跪下磕头 (叩头)。}  ¶ \textcolor{darkblue}{\textbf{\ipa{ɬo˧pv̩˥ ti˩}}} \textcolor{Sepia}{\selectlanguage{english}to kow-tow} \zh{跪下磕头}  
 ¶ \textcolor{darkblue}{\textbf{\ipa{ɬo˧pv̩˥ | le˧-ti˩}}} \textcolor{Sepia}{\selectlanguage{english}to kow-tow} \zh{跪下磕头}  
\textit{See:} \hyperlink{}{\textcolor{darkblue}{\textbf{\ipa{ɬo˧˥}}}} 
\lhead{\firstmark}
\rhead{\botmark}

\subsection{\hspace{-0.5cm} {\Large \textcolor{darkblue}{\textbf{\ipa{ɬo˧tɑ˧}}}}\hspace{0.5cm}[\kern2pt{\textcolor{darkblue}{\textbf{\ipa{ɬo˧tɑ˧}}}}\kern2pt]} \hypertarget{Ko\string_MtA\string_M1}{}
\markboth{\textcolor{darkblue}{\textbf{\ipa{ɬo˧tɑ˧}}}}{}
\textcolor{teal}{\mytextsc{preposition}} \hspace{4pt} Tone: M.
\textcolor{Sepia}{\selectlanguage{english}On the side of, beside.} \zh{旁边。}  ¶ \textcolor{darkblue}{\textbf{\ipa{ɬo˧tɑ˧ ɻ̍˩}}} \textcolor{Sepia}{\selectlanguage{english}to turn to the side} \zh{向侧面转}  
 ¶ \textcolor{darkblue}{\textbf{\ipa{ʁo˧qʰwɤ˩ | ɬo˧tɑ˧ | go˩˥}}} \textcolor{Sepia}{\selectlanguage{english}to have a headache; one's temples are throbbing (literally: 'to hurt on the sides of the head')} \zh{头疼,太阳穴阵痛}  

\lhead{\firstmark}
\rhead{\botmark}

\subsection{\hspace{-0.5cm} {\Large \textcolor{darkblue}{\textbf{\ipa{ɬo˩kɤ˩}}}}\hspace{0.5cm}[\kern2pt{\textcolor{darkblue}{\textbf{\ipa{ɬo˩kɤ˩˥}}}}\kern2pt]} \hypertarget{Ko\string_Bk7\string_B1}{}
\markboth{\textcolor{darkblue}{\textbf{\ipa{ɬo˩kɤ˩}}}}{}
\textcolor{teal}{\mytextsc{noun}} \hspace{4pt} Tone: L.
\textcolor{Sepia}{\selectlanguage{english}Rib.} \zh{肋骨。}  \zh{量词}: \textcolor{darkblue}{\textbf{\ipa{kɤ˧˥}}}  \mytextsc{clf}: \textcolor{darkblue}{\textbf{\ipa{kɤ˧˥}}} 
\lhead{\firstmark}
\rhead{\botmark}

\subsection{\hspace{-0.5cm} {\Large \textcolor{darkblue}{\textbf{\ipa{ɬo˩tsʰe˩mæ˥}}}}\hspace{0.5cm}[\kern2pt{\textcolor{darkblue}{\textbf{\ipa{ɬo˩tsʰe˩mæ˥}}}}\kern2pt]} \hypertarget{Ko\string_Bts\string_he\string_Bm\{\string_T1}{}
\markboth{\textcolor{darkblue}{\textbf{\ipa{ɬo˩tsʰe˩mæ˥}}}}{}
\textcolor{teal}{\mytextsc{noun}} \hspace{4pt} Tone: L+H\#.
\textit{\textcolor{Sepia}{\selectlanguage{english}archaic}} [\zh{古语}] \textcolor{Sepia}{\selectlanguage{english}Hip.} \zh{胯。}  \zh{量词}: \textcolor{darkblue}{\textbf{\ipa{ɭɯ˧}}}  \mytextsc{clf}: \textcolor{darkblue}{\textbf{\ipa{ɭɯ˧}}} \textit{Syn:} \hyperlink{}{\textcolor{darkblue}{\textbf{\ipa{ɬo˧kʰv̩˧}}}}. 
\lhead{\firstmark}
\rhead{\botmark}

\subsection{\hspace{-0.5cm} {\Large \textcolor{darkblue}{\textbf{\ipa{ɬo˧˥}}}}\hspace{0.5cm}[\kern2pt{\textcolor{darkblue}{\textbf{\ipa{ɬo˧˥}}}}\kern2pt]} \hypertarget{Ko\string_M\string_T1}{}
\markboth{\textcolor{darkblue}{\textbf{\ipa{ɬo˧˥}}}}{}
\textcolor{teal}{\mytextsc{adjective}} \hspace{4pt} Tone: MH.
\textcolor{Sepia}{\selectlanguage{english}Deep (water).} \zh{深(水深)。} 
\lhead{\firstmark}
\rhead{\botmark}

\subsection{\hspace{-0.5cm} {\Large \textcolor{darkblue}{\textbf{\ipa{ɬv̩˧˥}}}}\hspace{0.5cm}[\kern2pt{\textcolor{darkblue}{\textbf{\ipa{ɬv̩˧˥}}}}\kern2pt]} \hypertarget{Kv\string_=\string_M\string_T1}{}
\markboth{\textcolor{darkblue}{\textbf{\ipa{ɬv̩˧˥}}}}{}
\textcolor{teal}{\mytextsc{noun}} \hspace{4pt} Tone: MH.
\ding{202} \textcolor{Sepia}{\selectlanguage{english}Brains.} \zh{脑子、脑髓。}  \zh{量词}: \textcolor{darkblue}{\textbf{\ipa{ʈv̩˩}}} \ding{203} \textcolor{Sepia}{\selectlanguage{english}Marrow.} \zh{骨髓。}  \mytextsc{clf}: \textcolor{darkblue}{\textbf{\ipa{ʈv̩˩}}} 
\lhead{\firstmark}
\rhead{\botmark}

\subsection{\hspace{-0.5cm} {\Large \textcolor{darkblue}{\textbf{\ipa{ɬv̩˩\textsubscript{a}}}} \textsubscript{1}}\hspace{0.5cm}[\kern2pt{\textcolor{darkblue}{\textbf{\ipa{ɬv̩˩˥}}}}\kern2pt]} \hypertarget{Kv\string_=\string_Ba1}{}
\markboth{\textcolor{darkblue}{\textbf{\ipa{ɬv̩˩\textsubscript{a}}}} \textsubscript{1}}{}
\textcolor{teal}{\mytextsc{verb}} \hspace{4pt} Tone: L\textsubscript{a}.
\textcolor{Sepia}{\selectlanguage{english}To hold in the mouth; to let melt in the mouth.} \zh{含在嘴里、在嘴巴里溶化。}  ¶ \textcolor{darkblue}{\textbf{\ipa{tso˧\textasciitilde{}tso˧ ɬv̩˥}}} \textcolor{Sepia}{\selectlanguage{english}to hold something in the mouth, to have something in the mouth (context: a small child who does not yet know to distinguish between food and non-edible stuff puts things in its mouth)} \zh{含在嘴里(情景:一个小孩把不能吃的东西含在嘴巴里)}  

\lhead{\firstmark}
\rhead{\botmark}

\subsection{\hspace{-0.5cm} {\Large \textcolor{darkblue}{\textbf{\ipa{ɬv̩˩\textsubscript{a}}}} \textsubscript{2}}\hspace{0.5cm}[\kern2pt{\textcolor{darkblue}{\textbf{\ipa{ɬv̩˩˥}}}}\kern2pt]} \hypertarget{Kv\string_=\string_Ba2}{}
\markboth{\textcolor{darkblue}{\textbf{\ipa{ɬv̩˩\textsubscript{a}}}} \textsubscript{2}}{}
\textcolor{teal}{\mytextsc{adjective}} \hspace{4pt} Tone: L\textsubscript{a}.
\textcolor{Sepia}{\selectlanguage{english}Warm.} \zh{温暖,暖和。}  ¶ \textcolor{darkblue}{\textbf{\ipa{dʑɤ˩˥ | ɬv̩˩˥}}} \textcolor{Sepia}{\selectlanguage{english}nice and warm} \zh{温暖}  
 ¶ \textcolor{darkblue}{\textbf{\ipa{ɖwæ˧˥ | ɬv̩˩˥}}} \textcolor{Sepia}{\selectlanguage{english}\mytextsc{intensive}.very} \zh{很暖和}  
 ¶ \textcolor{darkblue}{\textbf{\ipa{ɬv̩˩-hĩ˩˥}}} \textcolor{Sepia}{\selectlanguage{english}\mytextsc{rel}/\mytextsc{nmlz}} \zh{温暖的}  

\lhead{\firstmark}
\rhead{\botmark}

\subsection{\hspace{-0.5cm} {\Large \textcolor{darkblue}{\textbf{\ipa{ɬv̩˩\textsubscript{a}}}} \textsubscript{3}}\hspace{0.5cm}[\kern2pt{\textcolor{darkblue}{\textbf{\ipa{ɬv̩˩˥}}}}\kern2pt]} \hypertarget{Kv\string_=\string_Ba3}{}
\markboth{\textcolor{darkblue}{\textbf{\ipa{ɬv̩˩\textsubscript{a}}}} \textsubscript{3}}{}
\textcolor{teal}{\mytextsc{verb}} \hspace{4pt} Tone: L\textsubscript{a}.
\textit{From:} \textbf{ɬv̩˩a 2} \textcolor{Sepia}{\selectlanguage{english}To warm up (food).} \zh{热饭。}  ¶ \textcolor{darkblue}{\textbf{\ipa{hɑ˧ ɬv̩˧˥}}} \textcolor{Sepia}{\selectlanguage{english}to warm up rice / food} \zh{热饭}  
 ¶ \textcolor{darkblue}{\textbf{\ipa{hɑ˧ | le˧-ɬv̩˩}}} \textcolor{Sepia}{\selectlanguage{english}to warm up rice / food} \zh{热饭}  
 ¶ \textcolor{darkblue}{\textbf{\ipa{hɑ˧ | ɖɯ˧-ɬv̩˧\textasciitilde{}ɬv̩˥-ɻ̍˩}}} \textcolor{Sepia}{\selectlanguage{english}to warm up food a little} \zh{饭热一热}  

\lhead{\firstmark}
\rhead{\botmark}

\subsection{\hspace{-0.5cm} {\Large \textcolor{darkblue}{\textbf{\ipa{ɬv̩˧gv̩\#˥}}}}\hspace{0.5cm}[\kern2pt{\textcolor{darkblue}{\textbf{\ipa{ɬv̩˧gv̩˥}}}}\kern2pt]} \hypertarget{Kv\string_=\string_Mgv\string_=\#\string_T1}{}
\markboth{\textcolor{darkblue}{\textbf{\ipa{ɬv̩˧gv̩\#˥}}}}{}
\textcolor{teal}{\mytextsc{noun}} \hspace{4pt} Tone: \#H.
\textcolor{Sepia}{\selectlanguage{english}Ritual offering of food to the deceased, seven days after cremation.} \zh{火葬后第七天的送食物仪式。} 
\lhead{\firstmark}
\rhead{\botmark}

\subsection{\hspace{-0.5cm} {\Large \textcolor{darkblue}{\textbf{\ipa{ɬv̩˧mi˧mæ˧dv̩˧mi\#˥}}}}\hspace{0.5cm}[\kern2pt{\textcolor{darkblue}{\textbf{\ipa{ɬv̩˧mi˧mæ˧dv̩˧mi˧}}}}\kern2pt]} \hypertarget{Kv\string_=\string_Mmi\string_Mm\{\string_Mdv\string_=\string_Mmi\#\string_T1}{}
\markboth{\textcolor{darkblue}{\textbf{\ipa{ɬv̩˧mi˧mæ˧dv̩˧mi\#˥}}}}{}
\textcolor{teal}{\mytextsc{noun}} \hspace{4pt} Tone: \#H.
\textcolor{Sepia}{\selectlanguage{english}Praying mantis.} \zh{螳螂。}  ¶ \textcolor{darkblue}{\textbf{\ipa{ɬv̩˧mi˧mæ˧dv̩˧mi˧ tʰv̩˧-mi˧˥ / ɬv̩˧mi˧mæ˧dv̩˧mi˧ tʰv̩˧-mi˥\#}}} \textcolor{Sepia}{\selectlanguage{english}\mytextsc{n}+\mytextsc{dem}+\mytextsc{clf}} \zh{那只螳螂}  
 \zh{量词}: \textcolor{darkblue}{\textbf{\ipa{mi˩}}}  \mytextsc{clf}: \textcolor{darkblue}{\textbf{\ipa{mi˩}}} 
\lhead{\firstmark}
\rhead{\botmark}

\subsection{\hspace{-0.5cm} {\Large \textcolor{darkblue}{\textbf{\ipa{ɬv̩˧ʁwɤ\#˥}}}}\hspace{0.5cm}[\kern2pt{\textcolor{darkblue}{\textbf{\ipa{ɬv̩˧ʁwɤ˧}}}}\kern2pt]} \hypertarget{Kv\string_=\string_MRw7\#\string_T1}{}
\markboth{\textcolor{darkblue}{\textbf{\ipa{ɬv̩˧ʁwɤ\#˥}}}}{}
\textcolor{teal}{\mytextsc{noun}} \hspace{4pt} Tone: \#H.
\textcolor{Sepia}{\selectlanguage{english}Village name.} \zh{村落名。} 
\lhead{\firstmark}
\rhead{\botmark}

\newpage
\section*{\centering- \textcolor{darkblue}{\textbf{\ipa{m}}} -}
\subsection{\hspace{-0.5cm} {\Large \textcolor{darkblue}{\textbf{\ipa{mɑ˧pʰv̩˧}}}}\hspace{0.5cm}[\kern2pt{\textcolor{darkblue}{\textbf{\ipa{mɑ˧pʰv̩˧}}}}\kern2pt]} \hypertarget{mA\string_Mp\string_hv\string_=\string_M1}{}
\markboth{\textcolor{darkblue}{\textbf{\ipa{mɑ˧pʰv̩˧}}}}{}
\textcolor{teal}{\mytextsc{noun}} \hspace{4pt} Tone: M.
\textcolor{Sepia}{\selectlanguage{english}Butter.} \zh{酥油。} 
\lhead{\firstmark}
\rhead{\botmark}

\subsection{\hspace{-0.5cm} {\Large \textcolor{darkblue}{\textbf{\ipa{mɑ˧tsɑ˥}}}}\hspace{0.5cm}[\kern2pt{\textcolor{darkblue}{\textbf{\ipa{mɑ˧tsɑ˧}}}}\kern2pt]} \hypertarget{mA\string_MtsA\string_T1}{}
\markboth{\textcolor{darkblue}{\textbf{\ipa{mɑ˧tsɑ˥}}}}{}
\textcolor{teal}{\mytextsc{noun}} \hspace{4pt} Tone: H\#.
\textcolor{Sepia}{\selectlanguage{english}Origin, distant cause, remote cause.} \zh{来历、发源地、深层原因/来源、来龙去脉、脉络。}  ¶ \textcolor{darkblue}{\textbf{\ipa{mɑ˧tsɑ˥ | ʈʂʰɯ˧-qo˧ le˧-tsʰɯ˩-ɲi˩! |}}} \textcolor{Sepia}{\selectlanguage{english}(Of an event:) It comes from afar! / It does not take place simply by chance: there is a long story behind it!} \zh{这(件事情)出处很远! / 有它的来龙去脉(=不是突然一下子出现的)!}  
 ¶ \textcolor{darkblue}{\textbf{\ipa{mɑ˧tsɑ˥ ʈʂʰɯ˩-kʰwɤ˩ |}}} \textcolor{Sepia}{\selectlanguage{english}\mytextsc{n}+\mytextsc{dem}+\mytextsc{clf}: this cause, this origin} \zh{这个来历}  
 \zh{量词}: \textcolor{darkblue}{\textbf{\ipa{kʰwɤ˥}}}  \mytextsc{clf}: \textcolor{darkblue}{\textbf{\ipa{kʰwɤ˥}}} 
\lhead{\firstmark}
\rhead{\botmark}

\subsection{\hspace{-0.5cm} {\Large \textcolor{darkblue}{\textbf{\ipa{mɑ˩dzɑ˩}}}}\hspace{0.5cm}[\kern2pt{\textcolor{darkblue}{\textbf{\ipa{mɑ˧dzɑ˧}}}}\kern2pt]} \hypertarget{mA\string_BdzA\string_B1}{}
\markboth{\textcolor{darkblue}{\textbf{\ipa{mɑ˩dzɑ˩}}}}{}
\textcolor{teal}{\mytextsc{noun}} \hspace{4pt} Tone: L.
\textcolor{Sepia}{\selectlanguage{english}Ink (solid).} \zh{墨。}  \zh{量词}: \textcolor{darkblue}{\textbf{\ipa{qʰwɤ˧˥}}}  \mytextsc{clf}: \textcolor{darkblue}{\textbf{\ipa{qʰwɤ˧˥}}} 
\lhead{\firstmark}
\rhead{\botmark}

\subsection{\hspace{-0.5cm} {\Large \textcolor{darkblue}{\textbf{\ipa{mɑ˩ɳɯ\#˥}}}}\hspace{0.5cm}[\kern2pt{\textcolor{darkblue}{\textbf{\ipa{xxxx non-correspondance entre le nombre de morphèmes et le nombre de tons de morphèmes}}}}\kern2pt]} \hypertarget{mA\string_Bn`M\#\string_T1}{}
\markboth{\textcolor{darkblue}{\textbf{\ipa{mɑ˩ɳɯ\#˥}}}}{}
\textcolor{teal}{\mytextsc{noun}} \hspace{4pt} Tone: LM+\#H.
\textcolor{Sepia}{\selectlanguage{english}Mani wall, Mani pile: pile built of rubble and sand, with carved stone tablets, most with the inscription Om Mani Padme Hum. A Mani wall should be passed or circumvented from the left side, the clockwise direction in which the universe revolves, according to Buddhist doctrine.} \zh{嘛呢堆。}  \zh{量词}: \textcolor{darkblue}{\textbf{\ipa{ɭɯ˧}}}  \mytextsc{clf}: \textcolor{darkblue}{\textbf{\ipa{ɭɯ˧}}} \textit{See:} \textcolor{darkblue}{\textbf{\ipa{mɑ˩ɳɯ˧-do˥bv˩, do˩bv̩\#˥}}} 
\lhead{\firstmark}
\rhead{\botmark}

\subsection{\hspace{-0.5cm} {\Large \textcolor{darkblue}{\textbf{\ipa{mɑ˩ɳɯ˧-do˥bv̩˩}}}}\hspace{0.5cm}[\kern2pt{\textcolor{darkblue}{\textbf{\ipa{xxxx non-correspondance entre le nombre de morphèmes et le nombre de tons de morphèmes}}}}\kern2pt]} \hypertarget{mA\string_Bn`M\string_M-do\string_Tbv\string_=\string_B1}{}
\markboth{\textcolor{darkblue}{\textbf{\ipa{mɑ˩ɳɯ˧-do˥bv̩˩}}}}{}
\textcolor{teal}{\mytextsc{noun}} \hspace{4pt} Tone: LM+\#H-.
\textcolor{Sepia}{\selectlanguage{english}Mani wall, Mani pile: pile built of rubble and sand, with carved stone tablets, most with the inscription Om Mani Padme Hum. A Mani wall should be passed or circumvented from the left side, the clockwise direction in which the universe revolves, according to Buddhist doctrine.} \zh{嘛呢堆。}  \zh{量词}: \textcolor{darkblue}{\textbf{\ipa{ɭɯ˧}}}  \mytextsc{clf}: \textcolor{darkblue}{\textbf{\ipa{ɭɯ˧}}} \textit{See:} \textcolor{darkblue}{\textbf{\ipa{mɑ˩ɳɯ\#˥, do˩bv̩\#˥}}} 
\lhead{\firstmark}
\rhead{\botmark}

\subsection{\hspace{-0.5cm} {\Large \textcolor{darkblue}{\textbf{\ipa{mæ˧}}}}\hspace{0.5cm}[\kern2pt{\textcolor{darkblue}{\textbf{\ipa{mæ˥}}}}\kern2pt]} \hypertarget{m\{\string_M1}{}
\markboth{\textcolor{darkblue}{\textbf{\ipa{mæ˧}}}}{}
\textcolor{teal}{\mytextsc{verb}} \hspace{4pt} Tone: M.
\textcolor{Sepia}{\selectlanguage{english}To achieve, to succeed in, to complete (a task).} \zh{……成、……成功。}  ¶ \textcolor{darkblue}{\textbf{\ipa{njɤ˧ ɖʐɤ˧˥ | tʰi˧-mɤ˧-mæ˧!}}} \textcolor{Sepia}{\selectlanguage{english}I can't fetch it!} \zh{我够不着!(例如:够不着树枝上的果子)}  

\lhead{\firstmark}
\rhead{\botmark}

\subsection{\hspace{-0.5cm} {\Large \textcolor{darkblue}{\textbf{\ipa{mæ˧}}}}\hspace{0.5cm}[\kern2pt{\textcolor{darkblue}{\textbf{\ipa{mæ˥}}}}\kern2pt]} \hypertarget{m\{\string_M1}{}
\markboth{\textcolor{darkblue}{\textbf{\ipa{mæ˧}}}}{}
\textcolor{teal}{\mytextsc{discourse}} \textcolor{teal}{\mytextsc{particle}} \hspace{4pt} Tone: M.
\textcolor{Sepia}{\selectlanguage{english}Final particle conveying obviousness.} \zh{句尾助词,表示显然、理所当然:“……呗!”。}  ¶ \textcolor{darkblue}{\textbf{\ipa{[Healing.66] hu˧mi˧-ʈʂʰæ˧ɣɯ˧ | le˧-ʈʰɯ˩, | le˧-qʰwɤ˧-ze˧ mæ˧! |}}} \textcolor{Sepia}{\selectlanguage{english}[Nowadays] one simply takes medicines for the stomach, and one is healed! [unlike in the old times, when there were no hospitals]} \zh{吃了胃药,就好了呗!}  

\lhead{\firstmark}
\rhead{\botmark}

\subsection{\hspace{-0.5cm} {\Large \textcolor{darkblue}{\textbf{\ipa{mæ˧}}} \textsubscript{1}}\hspace{0.5cm}[\kern2pt{\textcolor{darkblue}{\textbf{\ipa{mæ˥}}}}\kern2pt]} \hypertarget{m\{\string_M1}{}
\markboth{\textcolor{darkblue}{\textbf{\ipa{mæ˧}}} \textsubscript{1}}{}
\textcolor{teal}{\mytextsc{verb}} \hspace{4pt} Tone: M.
\textcolor{Sepia}{\selectlanguage{english}To be free to, to have the time to.} \zh{(有)空。}  ¶ \textcolor{darkblue}{\textbf{\ipa{njɤ˧ | mɤ˧-mæ˧.}}} \textcolor{Sepia}{\selectlanguage{english}I do not have the time; I am busy} \zh{我忙、我没有空}  
 ¶ \textcolor{darkblue}{\textbf{\ipa{njɤ˧ | mæ˧-mɤ˧-ho˩.}}} \textcolor{Sepia}{\selectlanguage{english}I won't have the time.} \zh{我不会有时间。}  

\lhead{\firstmark}
\rhead{\botmark}

\subsection{\hspace{-0.5cm} {\Large \textcolor{darkblue}{\textbf{\ipa{mæ˧}}} \textsubscript{2}}\hspace{0.5cm}[\kern2pt{\textcolor{darkblue}{\textbf{\ipa{mæ˥}}}}\kern2pt]} \hypertarget{m\{\string_M2}{}
\markboth{\textcolor{darkblue}{\textbf{\ipa{mæ˧}}} \textsubscript{2}}{}
\textcolor{teal}{\mytextsc{verb}} \hspace{4pt} Tone: M.
\textcolor{Sepia}{\selectlanguage{english}To manage (to do something).} \zh{能够(做)。}  ¶ \textcolor{darkblue}{\textbf{\ipa{ɖɯ˩-hĩ˩ qʰɑ˥ mæ˩\textasciitilde{}mæ˩! | tɕi˩-hĩ˩ lə˥-mɤ˩-mæ˩! / ɖɯ˩-hĩ˩˥, | qʰɑ˧ mæ˥\textasciitilde{}mæ˩! | tɕi˩-hĩ˩˥, | le˧-mɤ˧-mæ˧!}}} \textcolor{Sepia}{\selectlanguage{english}“Adults can manage all sorts of things, (whereas) children can't manage (that much) yet!” This saying is used when someone puts high demands on children or adolescents: Let the children play! To each age its occupations: children should play, not work. Adults' tasks are not their business!} \zh{“大人管干活,小孩管玩耍!”这个谚语的意思是:不要让孩子干活,每个年龄有自己的事,孩子的事就是玩。成年人的活儿,不是他们的事!}  

\lhead{\firstmark}
\rhead{\botmark}

\subsection{\hspace{-0.5cm} {\Large \textcolor{darkblue}{\textbf{\ipa{mæ˧\textsubscript{a}}}}}\hspace{0.5cm}[\kern2pt{\textcolor{darkblue}{\textbf{\ipa{mæ˥}}}}\kern2pt]} \hypertarget{m\{\string_Ma1}{}
\markboth{\textcolor{darkblue}{\textbf{\ipa{mæ˧\textsubscript{a}}}}}{}
\textcolor{teal}{\mytextsc{verb}} \hspace{4pt} Tone: M\textsubscript{a}.
\ding{202} \textcolor{Sepia}{\selectlanguage{english}To clutch, to catch hold of.} \zh{钩住(东西)。}  ¶ \textcolor{darkblue}{\textbf{\ipa{tʰi˧-mæ˧-ze˧}}} \textcolor{Sepia}{\selectlanguage{english}\mytextsc{dur} \string_ \mytextsc{pfv}} \zh{钩住了}  
\ding{203} \textcolor{Sepia}{\selectlanguage{english}To catch up with (someone).} \zh{跟上。} 
\lhead{\firstmark}
\rhead{\botmark}

\subsection{\hspace{-0.5cm} {\Large \textcolor{darkblue}{\textbf{\ipa{mæ˧pæ˧}}}}\hspace{0.5cm}[\kern2pt{\textcolor{darkblue}{\textbf{\ipa{mæ˩pæ˥}}}}\kern2pt]} \hypertarget{m\{\string_Mp\{\string_M1}{}
\markboth{\textcolor{darkblue}{\textbf{\ipa{mæ˧pæ˧}}}}{}
\textcolor{teal}{\mytextsc{noun}} \hspace{4pt} Tone: M.
\textcolor{Sepia}{\selectlanguage{english}Large sifter.} \zh{大筛子。}  \zh{量词}: \textcolor{darkblue}{\textbf{\ipa{nɑ˧}}}  \mytextsc{clf}: \textcolor{darkblue}{\textbf{\ipa{nɑ˧}}} 
\lhead{\firstmark}
\rhead{\botmark}

\subsection{\hspace{-0.5cm} {\Large \textcolor{darkblue}{\textbf{\ipa{-mæ˧qo˩}}}}\hspace{0.5cm}[\kern2pt{\textcolor{darkblue}{\textbf{\ipa{mæ˧qo˩}}}}\kern2pt]} \hypertarget{-m\{\string_Mqo\string_B1}{}
\markboth{\textcolor{darkblue}{\textbf{\ipa{-mæ˧qo˩}}}}{}
\textcolor{teal}{\mytextsc{postposition}} \hspace{4pt} Tone: L\#.
\textcolor{Sepia}{\selectlanguage{english}Below, behind.} \zh{下面,后面。} \textit{See:} \hyperlink{}{\textcolor{darkblue}{\textbf{\ipa{mæ˧qo˩}}}} 
\lhead{\firstmark}
\rhead{\botmark}

\subsection{\hspace{-0.5cm} {\Large \textcolor{darkblue}{\textbf{\ipa{mæ˧qo˩}}}}\hspace{0.5cm}[\kern2pt{\textcolor{darkblue}{\textbf{\ipa{mæ˧qo˧}}}}\kern2pt]} \hypertarget{m\{\string_Mqo\string_B1}{}
\markboth{\textcolor{darkblue}{\textbf{\ipa{mæ˧qo˩}}}}{}
\textcolor{teal}{\mytextsc{adverb(ial)}} \hspace{4pt} Tone: L\#.
\textcolor{Sepia}{\selectlanguage{english}At the extremity, at the end; at the bottom, in the lower part.} \zh{在尽头、在极点,在下面、在后面。} \textit{See:} \hyperlink{}{\textcolor{darkblue}{\textbf{\ipa{-mæ˧qo˩}}}} 
\lhead{\firstmark}
\rhead{\botmark}

\subsection{\hspace{-0.5cm} {\Large \textcolor{darkblue}{\textbf{\ipa{mæ˧qv̩˩}}}}\hspace{0.5cm}[\kern2pt{\textcolor{darkblue}{\textbf{\ipa{mæ˧qv̩˩}}}}\kern2pt]} \hypertarget{m\{\string_Mqv\string_=\string_B1}{}
\markboth{\textcolor{darkblue}{\textbf{\ipa{mæ˧qv̩˩}}}}{}
\textcolor{teal}{\mytextsc{noun}} \hspace{4pt} Tone: L\#.
\textcolor{Sepia}{\selectlanguage{english}Tail.} \zh{尾巴。}  ¶ \textcolor{darkblue}{\textbf{\ipa{ʝi˧-mæ˧qv̩˥}}} \textcolor{Sepia}{\selectlanguage{english}cow's tail} \zh{牛尾巴}  
 \zh{量词}: \textcolor{darkblue}{\textbf{\ipa{ɭɯ˧}}}  \mytextsc{clf}: \textcolor{darkblue}{\textbf{\ipa{ɭɯ˧}}} 
\lhead{\firstmark}
\rhead{\botmark}

\subsection{\hspace{-0.5cm} {\Large \textcolor{darkblue}{\textbf{\ipa{mæ˧ɻæ˩}}}}\hspace{0.5cm}[\kern2pt{\textcolor{darkblue}{\textbf{\ipa{mæ˧ɻæ˩}}}}\kern2pt]} \hypertarget{m\{\string_Mr£`\{\string_B1}{}
\markboth{\textcolor{darkblue}{\textbf{\ipa{mæ˧ɻæ˩}}}}{}
\textcolor{teal}{\mytextsc{noun}} \hspace{4pt} Tone: L\#.
\textcolor{Sepia}{\selectlanguage{english}Vegetable oil.} \zh{植物油。} 
\lhead{\firstmark}
\rhead{\botmark}

\subsection{\hspace{-0.5cm} {\Large \textcolor{darkblue}{\textbf{\ipa{mæ˧ɻ̃\#˥}}}}\hspace{0.5cm}[\kern2pt{\textcolor{darkblue}{\textbf{\ipa{mæ˧ɻ̃˩}}}}\kern2pt]} \hypertarget{m\{\string_Mr£`\string_~\#\string_T1}{}
\markboth{\textcolor{darkblue}{\textbf{\ipa{mæ˧ɻ̃\#˥}}}}{}
\textcolor{teal}{\mytextsc{noun}} \hspace{4pt} Tone: \#H.
\textcolor{Sepia}{\selectlanguage{english}Coccyx.} \zh{尾椎骨。} Local Chinese dialect:\zh{尾结骨。} \zh{量词}: \textcolor{darkblue}{\textbf{\ipa{ɭɯ˧}}}  \mytextsc{clf}: \textcolor{darkblue}{\textbf{\ipa{ɭɯ˧}}} 
\lhead{\firstmark}
\rhead{\botmark}

\subsection{\hspace{-0.5cm} {\Large \textcolor{darkblue}{\textbf{\ipa{mæ˩}}}}\hspace{0.5cm}[\kern2pt{\textcolor{darkblue}{\textbf{\ipa{mæ˩˥}}}}\kern2pt]} \hypertarget{m\{\string_B1}{}
\markboth{\textcolor{darkblue}{\textbf{\ipa{mæ˩}}}}{}
\textcolor{teal}{\mytextsc{verb}} \hspace{4pt} Tone: L.
\textcolor{Sepia}{\selectlanguage{english}To water, to irrigate (making small trenches and pouring water into them).} \zh{灌溉。}  ¶ \textcolor{darkblue}{\textbf{\ipa{dʑɯ˩ mæ˩˥}}} \textcolor{Sepia}{\selectlanguage{english}to irrigate, to water} \zh{浇灌}  
 ¶ \textcolor{darkblue}{\textbf{\ipa{dʑɯ˧ | le˧-mæ˩}}} \textcolor{Sepia}{\selectlanguage{english}\mytextsc{accomp}: to water, to irrigate} \zh{浇灌}  

\lhead{\firstmark}
\rhead{\botmark}

\subsection{\hspace{-0.5cm} {\Large \textcolor{darkblue}{\textbf{\ipa{mæ˩\textsubscript{a}}}}}\hspace{0.5cm}[\kern2pt{\textcolor{darkblue}{\textbf{\ipa{mæ˩˥}}}}\kern2pt]} \hypertarget{m\{\string_Ba1}{}
\markboth{\textcolor{darkblue}{\textbf{\ipa{mæ˩\textsubscript{a}}}}}{}
\textcolor{teal}{\mytextsc{verb}} \hspace{4pt} Tone: L\textsubscript{a}.
\textcolor{Sepia}{\selectlanguage{english}To aim at; to point at.} \zh{瞄准,指。}  ¶ \textcolor{darkblue}{\textbf{\ipa{tʰi˧-mæ˩-ze˩, | qʰæ˧-bi˥-ze˩.}}} \textcolor{Sepia}{\selectlanguage{english}[(S)he] has aimed; [(s)he] will now shoot.} \zh{瞄准了,要开枪了。}  
 ¶ \textcolor{darkblue}{\textbf{\ipa{lo˧ɲi˥ mæ˩}}} \textcolor{Sepia}{\selectlanguage{english}to point at with the finger} \zh{用手指出}  
 ¶ \textcolor{darkblue}{\textbf{\ipa{tso˧\textasciitilde{}tso˧ mæ˥}}} \textcolor{Sepia}{\selectlanguage{english}to point at things} \zh{指东西}  

\lhead{\firstmark}
\rhead{\botmark}

\subsection{\hspace{-0.5cm} {\Large \textcolor{darkblue}{\textbf{\ipa{mæ˩\textsubscript{a}}}} \textsubscript{1}}\hspace{0.5cm}[\kern2pt{\textcolor{darkblue}{\textbf{\ipa{mæ˩˥}}}}\kern2pt]} \hypertarget{m\{\string_Ba1}{}
\markboth{\textcolor{darkblue}{\textbf{\ipa{mæ˩\textsubscript{a}}}} \textsubscript{1}}{}
\textcolor{teal}{\mytextsc{classifier}} \hspace{4pt} Tone: L\textsubscript{a}.
\textcolor{Sepia}{\selectlanguage{english}Monetary unit: yuan.} \zh{量词:钱(一元)。}  ¶ \textcolor{darkblue}{\textbf{\ipa{ʈʂʰɯ˧-mæ˥}}} \textcolor{Sepia}{\selectlanguage{english}\mytextsc{dem} \string_ (tone: H\# / H\$)} \zh{\mytextsc{指示代词} \string_}  

\lhead{\firstmark}
\rhead{\botmark}

\subsection{\hspace{-0.5cm} {\Large \textcolor{darkblue}{\textbf{\ipa{mæ˩\textsubscript{a}}}} \textsubscript{2}}\hspace{0.5cm}[\kern2pt{\textcolor{darkblue}{\textbf{\ipa{mæ˩˥}}}}\kern2pt]} \hypertarget{m\{\string_Ba2}{}
\markboth{\textcolor{darkblue}{\textbf{\ipa{mæ˩\textsubscript{a}}}} \textsubscript{2}}{}
\textcolor{teal}{\mytextsc{classifier}} \hspace{4pt} Tone: L\textsubscript{a}.
\textcolor{Sepia}{\selectlanguage{english}10,000.} \zh{万(数词充当量词)。}  Borrowing: Chinese  \zh{元}, MC *mjonH (Baxter 2000)
 ¶ \textcolor{darkblue}{\textbf{\ipa{ɖɯ˧-mæ˩}}} \textcolor{Sepia}{\selectlanguage{english}10,000} \zh{一万}  
 ¶ \textcolor{darkblue}{\textbf{\ipa{tsʰe˩-tv̩˩ mæ˥}}} \textcolor{Sepia}{\selectlanguage{english}ten thousand times 10,000, i.e. one hundred million} \zh{十千万,等于一亿}  
 ¶ \textcolor{darkblue}{\textbf{\ipa{ɖɯ˧-ɕi˧ mæ˩}}} \textcolor{Sepia}{\selectlanguage{english}one hundred times 10,000, i.e. one million} \zh{一百万}  

\lhead{\firstmark}
\rhead{\botmark}

\subsection{\hspace{-0.5cm} {\Large \textcolor{darkblue}{\textbf{\ipa{mæ˩ɖʐo˥}}}}\hspace{0.5cm}[\kern2pt{\textcolor{darkblue}{\textbf{\ipa{mæ˩ɖʐo˩˥}}}}\kern2pt]} \hypertarget{m\{\string_Bd`z`o\string_T1}{}
\markboth{\textcolor{darkblue}{\textbf{\ipa{mæ˩ɖʐo˥}}}}{}
\textcolor{teal}{\mytextsc{noun}} \hspace{4pt} Tone: LH.
\textcolor{Sepia}{\selectlanguage{english}Whip.} \zh{鞭子。}  ¶ \textcolor{darkblue}{\textbf{\ipa{ʐwæ˧-mæ˥ɖʐo˩}}} \textcolor{Sepia}{\selectlanguage{english}horse whip} \zh{马鞭}  
 \zh{量词}: \textcolor{darkblue}{\textbf{\ipa{kʰɯ˩}}}  \mytextsc{clf}: \textcolor{darkblue}{\textbf{\ipa{kʰɯ˩}}} 
\lhead{\firstmark}
\rhead{\botmark}

\subsection{\hspace{-0.5cm} {\Large \textcolor{darkblue}{\textbf{\ipa{mæ˩ko˥}}}}\hspace{0.5cm}[\kern2pt{\textcolor{darkblue}{\textbf{\ipa{mæ˩ko˥}}}}\kern2pt]} \hypertarget{m\{\string_Bko\string_T1}{}
\markboth{\textcolor{darkblue}{\textbf{\ipa{mæ˩ko˥}}}}{}
\textcolor{teal}{\mytextsc{noun}} \hspace{4pt} Tone: LH.
\textcolor{Sepia}{\selectlanguage{english}Harness.} \zh{挽具,后鞧。}  ¶ \textcolor{darkblue}{\textbf{\ipa{ʐwæ˧-mæ˥ko˩}}} \textcolor{Sepia}{\selectlanguage{english}horse harness} \zh{马后鞧}  
 \zh{量词}: \textcolor{darkblue}{\textbf{\ipa{ɭɯ˧}}}  \mytextsc{clf}: \textcolor{darkblue}{\textbf{\ipa{ɭɯ˧}}} 
\lhead{\firstmark}
\rhead{\botmark}

\subsection{\hspace{-0.5cm} {\Large \textcolor{darkblue}{\textbf{\ipa{mæ˧˥}}}}\hspace{0.5cm}[\kern2pt{\textcolor{darkblue}{\textbf{\ipa{mæ˧˥}}}}\kern2pt]} \hypertarget{m\{\string_M\string_T1}{}
\markboth{\textcolor{darkblue}{\textbf{\ipa{mæ˧˥}}}}{}
\textcolor{teal}{\mytextsc{verb}} \hspace{4pt} Tone: MH.
\ding{202} \textcolor{Sepia}{\selectlanguage{english}To close (the mouth).} \zh{闭(嘴)。}  ¶ \textcolor{darkblue}{\textbf{\ipa{ɲi˧to˧ | tʰi˧-mæ˧˥}}} \textcolor{Sepia}{\selectlanguage{english}to close the mouth} \zh{闭嘴}  
 ¶ \textcolor{darkblue}{\textbf{\ipa{mæ˩\textasciitilde{}mæ˧˥}}} \textcolor{Sepia}{\selectlanguage{english}\mytextsc{red}} \zh{\mytextsc{重叠}}  
 ¶ \textcolor{darkblue}{\textbf{\ipa{ɲi˧to˧ | tʰi˧-mæ˩\textasciitilde{}mæ˩}}} \textcolor{Sepia}{\selectlanguage{english}to close the mouth} \zh{闭嘴}  
\ding{203} \textcolor{Sepia}{\selectlanguage{english}To purse (one's lips).} \zh{抿(嘴巴)。} 
\lhead{\firstmark}
\rhead{\botmark}

\subsection{\hspace{-0.5cm} {\Large \textcolor{darkblue}{\textbf{\ipa{mɤ˧‑}}}}\hspace{0.5cm}[\kern2pt{\textcolor{darkblue}{\textbf{\ipa{mɤ˥}}}}\kern2pt]} \hypertarget{m7\string_M‑1}{}
\markboth{\textcolor{darkblue}{\textbf{\ipa{mɤ˧‑}}}}{}
\textcolor{teal}{\mytextsc{prefix}} \hspace{4pt} Tone: M.
\textcolor{Sepia}{\selectlanguage{english}Negation.} \zh{否定:不,没。} 
\lhead{\firstmark}
\rhead{\botmark}

\subsection{\hspace{-0.5cm} {\Large \textcolor{darkblue}{\textbf{\ipa{mɤ˧ʈʂʰɤ˧}}}}\hspace{0.5cm}[\kern2pt{\textcolor{darkblue}{\textbf{\ipa{mɤ˧ʈʂʰɤ˧}}}}\kern2pt]} \hypertarget{m7\string_Mt`s`\string_h7\string_M1}{}
\markboth{\textcolor{darkblue}{\textbf{\ipa{mɤ˧ʈʂʰɤ˧}}}}{}
\textcolor{teal}{\mytextsc{noun}} \hspace{4pt} Tone: M.
\textcolor{Sepia}{\selectlanguage{english}Cart.} \zh{马车(汉语借词)。}  Borrowing: Chinese  \zh{马车}

\lhead{\firstmark}
\rhead{\botmark}

\subsection{\hspace{-0.5cm} {\Large \textcolor{darkblue}{\textbf{\ipa{mɤ˩}}}}\hspace{0.5cm}[\kern2pt{\textcolor{darkblue}{\textbf{\ipa{mɤ˥}}}}\kern2pt]} \hypertarget{m7\string_B1}{}
\markboth{\textcolor{darkblue}{\textbf{\ipa{mɤ˩}}}}{}
\textcolor{teal}{\mytextsc{noun}} \hspace{4pt} Tone: L.
\textcolor{Sepia}{\selectlanguage{english}Animal fat.} \zh{动物油。}  ¶ \textcolor{darkblue}{\textbf{\ipa{njɤ˧ | mɤ˩ mɤ˩ dzɯ˩˥!}}} \textcolor{Sepia}{\selectlanguage{english}I don't eat animal fat! (One of the investigator's peculiarities)} \zh{我不吃猪油!(这是调查者的特点之一)}  

\lhead{\firstmark}
\rhead{\botmark}

\subsection{\hspace{-0.5cm} {\Large \textcolor{darkblue}{\textbf{\ipa{mɤ˩\textsubscript{a}}}}}\hspace{0.5cm}[\kern2pt{\textcolor{darkblue}{\textbf{\ipa{mɤ˩˥}}}}\kern2pt]} \hypertarget{m7\string_Ba1}{}
\markboth{\textcolor{darkblue}{\textbf{\ipa{mɤ˩\textsubscript{a}}}}}{}
\textcolor{teal}{\mytextsc{classifier}} \hspace{4pt} Tone: L\textsubscript{a}.
\textcolor{Sepia}{\selectlanguage{english}A few, a little.} \zh{量词:一些、一点。}  ¶ \textcolor{darkblue}{\textbf{\ipa{ɕi˧ɭɯ˧-ɻæ˩ | ɖɯ˧-mɤ˩}}} \textcolor{Sepia}{\selectlanguage{english}a few seeds of rice} \zh{一些稻谷种子}  
 ¶ \textcolor{darkblue}{\textbf{\ipa{ɻæ˩˥ | ɖɯ˧-mɤ˩}}} \textcolor{Sepia}{\selectlanguage{english}a few seeds} \zh{一些种子}  
 ¶ \textcolor{darkblue}{\textbf{\ipa{tsɑ˧bɤ˧ | ɖɯ˧-mɤ˩, | tsɑ˧bɤ˧ | ɲi˧-mɤ˩}}} \textcolor{Sepia}{\selectlanguage{english}a small quantity of flour; two small quantities of flour; etc} \zh{一小捧面粉、两小捧面粉……}  
 ¶ \textcolor{darkblue}{\textbf{\ipa{ʈʂʰɯ˧-mɤ˥}}} \textcolor{Sepia}{\selectlanguage{english}\mytextsc{dem} \string_ (tone: H\# / H\$)} \zh{\mytextsc{指示代词} \string_}  

\lhead{\firstmark}
\rhead{\botmark}

\subsection{\hspace{-0.5cm} {\Large \textcolor{darkblue}{\textbf{\ipa{mɤ˩\textsubscript{b}}}}}\hspace{0.5cm}[\kern2pt{\textcolor{darkblue}{\textbf{\ipa{mɤ˩˥}}}}\kern2pt]} \hypertarget{m7\string_Bb1}{}
\markboth{\textcolor{darkblue}{\textbf{\ipa{mɤ˩\textsubscript{b}}}}}{}
\textcolor{teal}{\mytextsc{verb}} \hspace{4pt} Tone: L\textsubscript{b}.
\textcolor{Sepia}{\selectlanguage{english}To eat food in powder form, typically tsamba.} \zh{将粉状的食品放在嘴里(如:干糌粑)。}  ¶ \textcolor{darkblue}{\textbf{\ipa{tsɑ˧bɤ˧ mɤ˩}}} \textcolor{Sepia}{\selectlanguage{english}to eat dry tsamba: one takes a spoonful, pours it into the mouth, and lets it get impregnated with saliva} \zh{吃干糌粑}  
 ¶ \textcolor{darkblue}{\textbf{\ipa{tsɑ˧bɤ˧ | ɖɯ˧-mɤ˧\textasciitilde{}mɤ˩-ɻ̍˩}}} \textcolor{Sepia}{\selectlanguage{english}to eat a little dry tsamba, to take the time to appreciate some dry tsamba} \zh{品干糌粑、慢慢享受一点干糌粑}  

\lhead{\firstmark}
\rhead{\botmark}

\subsection{\hspace{-0.5cm} {\Large \textcolor{darkblue}{\textbf{\ipa{mɤ˩ɬi˩}}}}\hspace{0.5cm}[\kern2pt{\textcolor{darkblue}{\textbf{\ipa{mɤ˩ɬi˩˥}}}}\kern2pt]} \hypertarget{m7\string_BKi\string_B1}{}
\markboth{\textcolor{darkblue}{\textbf{\ipa{mɤ˩ɬi˩}}}}{}
\textcolor{teal}{\mytextsc{noun}} \hspace{4pt} Tone: L.
\textcolor{Sepia}{\selectlanguage{english}Butter tea.} \zh{酥油茶。}  \zh{量词}: \textcolor{darkblue}{\textbf{\ipa{qʰwɤ˧˥}}}  \mytextsc{clf}: \textcolor{darkblue}{\textbf{\ipa{qʰwɤ˧˥}}} 
\lhead{\firstmark}
\rhead{\botmark}

\subsection{\hspace{-0.5cm} {\Large \textcolor{darkblue}{\textbf{\ipa{mɤ˩mv̩˩}}}}\hspace{0.5cm}[\kern2pt{\textcolor{darkblue}{\textbf{\ipa{mɤ˩mv̩˩˥}}}}\kern2pt]} \hypertarget{m7\string_Bmv\string_=\string_B1}{}
\markboth{\textcolor{darkblue}{\textbf{\ipa{mɤ˩mv̩˩}}}}{}
\textcolor{teal}{\mytextsc{noun}} \hspace{4pt} Tone: L.
\textcolor{Sepia}{\selectlanguage{english}Candle holder.} \zh{烛台。}  \zh{量词}: \textcolor{darkblue}{\textbf{\ipa{qʰwɤ˧˥}}}  \mytextsc{clf}: \textcolor{darkblue}{\textbf{\ipa{qʰwɤ˧˥}}} 
\lhead{\firstmark}
\rhead{\botmark}

\subsection{\hspace{-0.5cm} {\Large \textcolor{darkblue}{\textbf{\ipa{mɤ˩tʰɑ˧}}}}\hspace{0.5cm}[\kern2pt{\textcolor{darkblue}{\textbf{\ipa{mɤ˩tʰɑ˥}}}}\kern2pt]} \hypertarget{m7\string_Bt\string_hA\string_M1}{}
\markboth{\textcolor{darkblue}{\textbf{\ipa{mɤ˩tʰɑ˧}}}}{}
\textcolor{teal}{\mytextsc{noun}} \hspace{4pt} Tone: LM.
\textcolor{Sepia}{\selectlanguage{english}Sesame candy.} \zh{麻糖(汉语借词)。}  Borrowing: Chinese  \zh{麻糖}

\lhead{\firstmark}
\rhead{\botmark}

\subsection{\hspace{-0.5cm} {\Large \textcolor{darkblue}{\textbf{\ipa{mi˧}}}}\hspace{0.5cm}[\kern2pt{\textcolor{darkblue}{\textbf{\ipa{mi˩˥}}}}\kern2pt]} \hypertarget{mi\string_M1}{}
\markboth{\textcolor{darkblue}{\textbf{\ipa{mi˧}}}}{}
\textcolor{teal}{\mytextsc{noun}} \hspace{4pt} Tone: M.
\textcolor{Sepia}{\selectlanguage{english}Wound.} \zh{伤口。}  \zh{量词}: \textcolor{darkblue}{\textbf{\ipa{kʰwɤ˥}}}  \mytextsc{clf}: \textcolor{darkblue}{\textbf{\ipa{kʰwɤ˥}}} 
\lhead{\firstmark}
\rhead{\botmark}

\subsection{\hspace{-0.5cm} {\Large \textcolor{darkblue}{\textbf{\ipa{mi˧kʰwɤ\#˥}}}}\hspace{0.5cm}[\kern2pt{\textcolor{darkblue}{\textbf{\ipa{mi˩kʰwɤ˥}}}}\kern2pt]} \hypertarget{mi\string_Mk\string_hw7\#\string_T1}{}
\markboth{\textcolor{darkblue}{\textbf{\ipa{mi˧kʰwɤ\#˥}}}}{}
\textcolor{teal}{\mytextsc{noun}} \hspace{4pt} Tone: \#H.
\ding{202} \textcolor{Sepia}{\selectlanguage{english}Wound.} \zh{伤口。}  \zh{量词}: \textcolor{darkblue}{\textbf{\ipa{kʰwɤ˥}}} \ding{203} \textcolor{Sepia}{\selectlanguage{english}Ulcer.} \zh{疮。}  \mytextsc{clf}: \textcolor{darkblue}{\textbf{\ipa{kʰwɤ˥}}} 
\lhead{\firstmark}
\rhead{\botmark}

\subsection{\hspace{-0.5cm} {\Large \textcolor{darkblue}{\textbf{\ipa{mi˧ɬo\#˥}}}}\hspace{0.5cm}[\kern2pt{\textcolor{darkblue}{\textbf{\ipa{mi˩ɬo˥}}}}\kern2pt]} \hypertarget{mi\string_MKo\#\string_T1}{}
\markboth{\textcolor{darkblue}{\textbf{\ipa{mi˧ɬo\#˥}}}}{}
\textcolor{teal}{\mytextsc{noun}} \hspace{4pt} Tone: \#H.
\textcolor{Sepia}{\selectlanguage{english}Prayer.} \zh{祈祷。}  ¶ \textcolor{darkblue}{\textbf{\ipa{mi˧ɬo˧ lɑ˩}}} \textcolor{Sepia}{\selectlanguage{english}to pray} \zh{祈祷}  
 \zh{量词}: \textcolor{darkblue}{\textbf{\ipa{kʰwɤ˥}}}  \mytextsc{clf}: \textcolor{darkblue}{\textbf{\ipa{kʰwɤ˥}}} \textit{See:} \hyperlink{}{\textcolor{darkblue}{\textbf{\ipa{ɬo˧˥}}}} 
\lhead{\firstmark}
\rhead{\botmark}

\subsection{\hspace{-0.5cm} {\Large \textcolor{darkblue}{\textbf{\ipa{mi˧mi˧}}}}\hspace{0.5cm}[\kern2pt{\textcolor{darkblue}{\textbf{\ipa{mi˧mi˧}}}}\kern2pt]} \hypertarget{mi\string_Mmi\string_M1}{}
\markboth{\textcolor{darkblue}{\textbf{\ipa{mi˧mi˧}}}}{}
\textcolor{teal}{\mytextsc{noun}} \hspace{4pt} Tone: M.
\textcolor{Sepia}{\selectlanguage{english}Kernel (of a seed).} \zh{核,仁。}  Borrowing: Chinese  (dialectal) \zh{米米}

\lhead{\firstmark}
\rhead{\botmark}

\subsection{\hspace{-0.5cm} {\Large \textcolor{darkblue}{\textbf{\ipa{mi˧pɤ\#˥}}}}\hspace{0.5cm}[\kern2pt{\textcolor{darkblue}{\textbf{\ipa{mi˩pɤ˩˥}}}}\kern2pt]} \hypertarget{mi\string_Mp7\#\string_T1}{}
\markboth{\textcolor{darkblue}{\textbf{\ipa{mi˧pɤ\#˥}}}}{}
\textcolor{teal}{\mytextsc{noun}} \hspace{4pt} Tone: \#H.
\textcolor{Sepia}{\selectlanguage{english}Scar.} \zh{疤。}  \zh{量词}: \textcolor{darkblue}{\textbf{\ipa{kʰwɤ˥}}}  \mytextsc{clf}: \textcolor{darkblue}{\textbf{\ipa{kʰwɤ˥}}} 
\lhead{\firstmark}
\rhead{\botmark}

\subsection{\hspace{-0.5cm} {\Large \textcolor{darkblue}{\textbf{\ipa{mi˧tʰv̩\#˥}}}}\hspace{0.5cm}[\kern2pt{\textcolor{darkblue}{\textbf{\ipa{mi˩tʰv̩˩˥}}}}\kern2pt]} \hypertarget{mi\string_Mt\string_hv\string_=\#\string_T1}{}
\markboth{\textcolor{darkblue}{\textbf{\ipa{mi˧tʰv̩\#˥}}}}{}
\textcolor{teal}{\mytextsc{noun}} \hspace{4pt} Tone: \#H.
\textcolor{Sepia}{\selectlanguage{english}Walking-stick.} \zh{拐棍。}  \zh{量词}: \textcolor{darkblue}{\textbf{\ipa{kɤ˧˥}}}  \mytextsc{clf}: \textcolor{darkblue}{\textbf{\ipa{kɤ˧˥}}} 
\lhead{\firstmark}
\rhead{\botmark}

\subsection{\hspace{-0.5cm} {\Large \textcolor{darkblue}{\textbf{\ipa{mi˩\textsubscript{a}}}}}\hspace{0.5cm}[\kern2pt{\textcolor{darkblue}{\textbf{\ipa{mi˧˥}}}}\kern2pt]} \hypertarget{mi\string_Ba1}{}
\markboth{\textcolor{darkblue}{\textbf{\ipa{mi˩\textsubscript{a}}}}}{}
\textcolor{teal}{\mytextsc{verb}} \hspace{4pt} Tone: L\textsubscript{a}.
\textcolor{Sepia}{\selectlanguage{english}To ask for.} \zh{请求、要,讨饭。}  ¶ \textcolor{darkblue}{\textbf{\ipa{hɑ˧ mi˥}}} \textcolor{Sepia}{\selectlanguage{english}to beg (literally: 'to ask for food')} \zh{讨饭}  
 ¶ \textcolor{darkblue}{\textbf{\ipa{hɑ˧ | ɖɯ˧-mi˧\textasciitilde{}mi˥-ɻ̍˩}}} \textcolor{Sepia}{\selectlanguage{english}to beg a little, to ask around for some food} \zh{讨点饭}  

\lhead{\firstmark}
\rhead{\botmark}

\subsection{\hspace{-0.5cm} {\Large \textcolor{darkblue}{\textbf{\ipa{mi˩\textsubscript{b}}}}}\hspace{0.5cm}[\kern2pt{\textcolor{darkblue}{\textbf{\ipa{mi˩˥}}}}\kern2pt]} \hypertarget{mi\string_Bb1}{}
\markboth{\textcolor{darkblue}{\textbf{\ipa{mi˩\textsubscript{b}}}}}{}
\textcolor{teal}{\mytextsc{classifier}} \hspace{4pt} Tone: L\textsubscript{b}.
\textcolor{Sepia}{\selectlanguage{english}Classifier for small animals.} \zh{量词:小动物(一只)。}  ¶ \textcolor{darkblue}{\textbf{\ipa{ʈʂʰɯ˧-mi˧˥}}} \textcolor{Sepia}{\selectlanguage{english}this animal} \zh{这只}  

\lhead{\firstmark}
\rhead{\botmark}

\subsection{\hspace{-0.5cm} {\Large \textcolor{darkblue}{\textbf{\ipa{mi˩hwɑ˧}}}}\hspace{0.5cm}[\kern2pt{\textcolor{darkblue}{\textbf{\ipa{mi˩hwɑ˩˥}}}}\kern2pt]} \hypertarget{mi\string_BhwA\string_M1}{}
\markboth{\textcolor{darkblue}{\textbf{\ipa{mi˩hwɑ˧}}}}{}
\textcolor{teal}{\mytextsc{noun}} \hspace{4pt} Tone: LM.
\textcolor{Sepia}{\selectlanguage{english}Cotton.} \zh{棉花(汉语借词)。}  Borrowing: \zh{棉花}
 ¶ \textcolor{darkblue}{\textbf{\ipa{mi˩hwɑ˧-bɑ˩lɑ˩}}} \textcolor{Sepia}{\selectlanguage{english}cotton clothes} \zh{棉布衣服}  

\lhead{\firstmark}
\rhead{\botmark}

\subsection{\hspace{-0.5cm} {\Large \textcolor{darkblue}{\textbf{\ipa{mi˩ɬi˩}}}}\hspace{0.5cm}[\kern2pt{\textcolor{darkblue}{\textbf{\ipa{mi˧ɬi˧}}}}\kern2pt]} \hypertarget{mi\string_BKi\string_B1}{}
\markboth{\textcolor{darkblue}{\textbf{\ipa{mi˩ɬi˩}}}}{}
\textcolor{teal}{\mytextsc{noun}} \hspace{4pt} Tone: L.
\textcolor{Sepia}{\selectlanguage{english}Large bamboo.} \zh{大竹子。}  ¶ \textcolor{darkblue}{\textbf{\ipa{mi˩ɬi˩-bæ˩ʈʂo˥}}} \textcolor{Sepia}{\selectlanguage{english}bamboo broom} \zh{竹扫帚}  
 ¶ \textcolor{darkblue}{\textbf{\ipa{mi˩ɬi˩-ʈʂæ˥do˩}}} \textcolor{Sepia}{\selectlanguage{english}bamboo bucket to carry water (on one's back)} \zh{竹桶,用来背水}  
 \zh{量词}: \textcolor{darkblue}{\textbf{\ipa{dzi˩}}}  \mytextsc{clf}: \textcolor{darkblue}{\textbf{\ipa{dzi˩}}} 
\lhead{\firstmark}
\rhead{\botmark}

\subsection{\hspace{-0.5cm} {\Large \textcolor{darkblue}{\textbf{\ipa{mi˩ɬi˩-ʁo˩bv̩˥}}}}\hspace{0.5cm}[\kern2pt{\textcolor{darkblue}{\textbf{\ipa{xxxx non-correspondance entre le nombre de morphèmes et le nombre de tons de morphèmes}}}}\kern2pt]} \hypertarget{mi\string_BKi\string_B-Ro\string_Bbv\string_=\string_T1}{}
\markboth{\textcolor{darkblue}{\textbf{\ipa{mi˩ɬi˩-ʁo˩bv̩˥}}}}{}
\textcolor{teal}{\mytextsc{noun}} \hspace{4pt} Tone: L+H\#.
\textcolor{Sepia}{\selectlanguage{english}Bamboo shoot.} \zh{竹笋。}  \zh{量词}: \textcolor{darkblue}{\textbf{\ipa{kɤ˧˥}}}  \mytextsc{clf}: \textcolor{darkblue}{\textbf{\ipa{kɤ˧˥}}} 
\lhead{\firstmark}
\rhead{\botmark}

\subsection{\hspace{-0.5cm} {\Large \textcolor{darkblue}{\textbf{\ipa{mi˩mo˩}}}}\hspace{0.5cm}[\kern2pt{\textcolor{darkblue}{\textbf{\ipa{mi˧mo˧}}}}\kern2pt]} \hypertarget{mi\string_Bmo\string_B1}{}
\markboth{\textcolor{darkblue}{\textbf{\ipa{mi˩mo˩}}}}{}
\textcolor{teal}{\mytextsc{noun}} \hspace{4pt} Tone: L.
\textcolor{Sepia}{\selectlanguage{english}Small sifter.} \zh{小筛子。}  \zh{量词}: \textcolor{darkblue}{\textbf{\ipa{nɑ˧}}}  \mytextsc{clf}: \textcolor{darkblue}{\textbf{\ipa{nɑ˧}}} 
\lhead{\firstmark}
\rhead{\botmark}

\subsection{\hspace{-0.5cm} {\Large \textcolor{darkblue}{\textbf{\ipa{mi˩pʰv̩˩}}}}\hspace{0.5cm}[\kern2pt{\textcolor{darkblue}{\textbf{\ipa{mi˧pʰv̩˧}}}}\kern2pt]} \hypertarget{mi\string_Bp\string_hv\string_=\string_B1}{}
\markboth{\textcolor{darkblue}{\textbf{\ipa{mi˩pʰv̩˩}}}}{}
\textcolor{teal}{\mytextsc{noun}} \hspace{4pt} Tone: L.
\textcolor{Sepia}{\selectlanguage{english}Nettle.} \zh{荨麻。}  \zh{量词}: \textcolor{darkblue}{\textbf{\ipa{dzi˩}}}  \mytextsc{clf}: \textcolor{darkblue}{\textbf{\ipa{dzi˩}}} 
\lhead{\firstmark}
\rhead{\botmark}

\subsection{\hspace{-0.5cm} {\Large \textcolor{darkblue}{\textbf{\ipa{mi˩zɯ˩}}}}\hspace{0.5cm}[\kern2pt{\textcolor{darkblue}{\textbf{\ipa{mi˧zɯ˧}}}}\kern2pt]} \hypertarget{mi\string_BzM\string_B1}{}
\markboth{\textcolor{darkblue}{\textbf{\ipa{mi˩zɯ˩}}}}{}
\textcolor{teal}{\mytextsc{noun}} \hspace{4pt} Tone: L.
\textcolor{Sepia}{\selectlanguage{english}Woman; also the name of the second pillar in the main room (the feminine pillar).} \zh{女人。主屋的第二个柱子(代表女性的那个柱子)也是用这个名字。}  \zh{量词}: \textcolor{darkblue}{\textbf{\ipa{v̩˧}}}  \mytextsc{clf}: \textcolor{darkblue}{\textbf{\ipa{v̩˧}}} 
\lhead{\firstmark}
\rhead{\botmark}

\subsection{\hspace{-0.5cm} {\Large \textcolor{darkblue}{\textbf{\ipa{mi˧˥}}}}\hspace{0.5cm}[\kern2pt{\textcolor{darkblue}{\textbf{\ipa{mi˥}}}}\kern2pt]} \hypertarget{mi\string_M\string_T1}{}
\markboth{\textcolor{darkblue}{\textbf{\ipa{mi˧˥}}}}{}
\textcolor{teal}{\mytextsc{verb}} \hspace{4pt} Tone: MH.
\textcolor{Sepia}{\selectlanguage{english}To push.} \zh{推、拥挤。}  ¶ \textcolor{darkblue}{\textbf{\ipa{le˧-mi˧-ze˥}}} \textcolor{Sepia}{\selectlanguage{english}\mytextsc{accomp} \string_ \mytextsc{pfv}} \zh{推开了}  
 ¶ \textcolor{darkblue}{\textbf{\ipa{le˧-mi˧˥}}} \textcolor{Sepia}{\selectlanguage{english}\mytextsc{accomp}} \zh{推}  
 ¶ \textcolor{darkblue}{\textbf{\ipa{tʰi˧-mi˧˥}}} \textcolor{Sepia}{\selectlanguage{english}\mytextsc{dur}} \zh{推}  
 ¶ \textcolor{darkblue}{\textbf{\ipa{tso˧\textasciitilde{}tso˧ mi˩}}} \textcolor{Sepia}{\selectlanguage{english}to push something} \zh{推开一个东西}  
 ¶ \textcolor{darkblue}{\textbf{\ipa{mi˩\textasciitilde{}mi˧˥}}} \textcolor{Sepia}{\selectlanguage{english}\mytextsc{red}: to push and squeeze} \zh{\mytextsc{重叠:推、拥挤}}  
 ¶ \textcolor{darkblue}{\textbf{\ipa{mi˩\textasciitilde{}mi˧-ɻ̍˥}}} \textcolor{Sepia}{\selectlanguage{english}\mytextsc{red} \mytextsc{inceptive}} \zh{\mytextsc{重叠:推、拥挤}}  

\lhead{\firstmark}
\rhead{\botmark}

\subsection{\hspace{-0.5cm} {\Large \textcolor{darkblue}{\textbf{\ipa{-mi˩˧}}}}\hspace{0.5cm}[\kern2pt{\textcolor{darkblue}{\textbf{\ipa{mi˩˥}}}}\kern2pt]} \hypertarget{-mi\string_B\string_M1}{}
\markboth{\textcolor{darkblue}{\textbf{\ipa{-mi˩˧}}}}{}
\textcolor{teal}{\mytextsc{suffix}} \hspace{4pt} Tone: LM.
\ding{202} \textcolor{Sepia}{\selectlanguage{english}Feminine suffix.} \zh{阴性后缀。}  \zh{量词}: \textcolor{darkblue}{\textbf{\ipa{v̩˧}}} \ding{203} \textcolor{Sepia}{\selectlanguage{english}Augmentative suffix.} \zh{指大词。}  \zh{量词}: \textcolor{darkblue}{\textbf{\ipa{v̩˧}}}  \mytextsc{clf}: \textcolor{darkblue}{\textbf{\ipa{v̩˧}}} \textcolor{darkblue}{\textbf{\ipa{v̩˧}}} 
\lhead{\firstmark}
\rhead{\botmark}

\subsection{\hspace{-0.5cm} {\Large \textcolor{darkblue}{\textbf{\ipa{mi˩˧}}}}\hspace{0.5cm}[\kern2pt{\textcolor{darkblue}{\textbf{\ipa{mi˥}}}}\kern2pt]} \hypertarget{mi\string_B\string_M1}{}
\markboth{\textcolor{darkblue}{\textbf{\ipa{mi˩˧}}}}{}
\textcolor{teal}{\mytextsc{noun}} \hspace{4pt} Tone: LM.
\textcolor{Sepia}{\selectlanguage{english}Female (animal).} \zh{母的(动物)。}  ¶ \textcolor{darkblue}{\textbf{\ipa{ʈʂʰɯ˧, | mi˩˥! / ʈʂʰɯ˧, | mi˩ ɲi˥!}}} \textcolor{Sepia}{\selectlanguage{english}It's a female!} \zh{是母的!}  
 \zh{量词}: \textcolor{darkblue}{\textbf{\ipa{v̩˧}}}  \mytextsc{clf}: \textcolor{darkblue}{\textbf{\ipa{v̩˧}}} 
\lhead{\firstmark}
\rhead{\botmark}

\subsection{\hspace{-0.5cm} {\Large \textcolor{darkblue}{\textbf{\ipa{mje˧˥}}}}\hspace{0.5cm}[\kern2pt{\textcolor{darkblue}{\textbf{\ipa{mje˥}}}}\kern2pt]} \hypertarget{mje\string_M\string_T1}{}
\markboth{\textcolor{darkblue}{\textbf{\ipa{mje˧˥}}}}{}
\textcolor{teal}{\mytextsc{noun}} \hspace{4pt} Tone: .
\textcolor{Sepia}{\selectlanguage{english}Noodles.} \zh{面条。}  Borrowing: Chinese  \zh{面}
 ¶ \textcolor{darkblue}{\textbf{\ipa{mjæ˧˥ | dzɯ˧-bi˧! \textasciitilde{} mjæ˧ dzɯ˧-bi˧! \textasciitilde{} mjæ˧ dzɯ˥-bi˩!}}} \textcolor{Sepia}{\selectlanguage{english}Let's eat noodles!} \zh{吃面吧!}  
 ¶ \textcolor{darkblue}{\textbf{\ipa{mjæ˧˥ | ɖɯ˧-qʰwɤ˧ tɕɤ˥}}} \textcolor{Sepia}{\selectlanguage{english}to boil a bowl of noodles, to cook a bowl of noodles} \zh{煮一碗面}  
 ¶ \textcolor{darkblue}{\textbf{\ipa{mjæ˧ hwæ˥-bi˩}}} \textcolor{Sepia}{\selectlanguage{english}(we) will buy noodles} \zh{要买面}  

\lhead{\firstmark}
\rhead{\botmark}

\subsection{\hspace{-0.5cm} {\Large \textcolor{darkblue}{\textbf{\ipa{mo˥\textsubscript{a}}}}}\hspace{0.5cm}[\kern2pt{\textcolor{darkblue}{\textbf{\ipa{mo˥}}}}\kern2pt]} \hypertarget{mo\string_Ta1}{}
\markboth{\textcolor{darkblue}{\textbf{\ipa{mo˥\textsubscript{a}}}}}{}
\textcolor{teal}{\mytextsc{classifier}} \hspace{4pt} Tone: H\textsubscript{a}.
\textcolor{Sepia}{\selectlanguage{english}One Chinese acre, amounting to one-sixth of an acre.} \zh{量词:地(一亩地)(汉语借词)。}  Borrowing: Chinese  \zh{亩}

\lhead{\firstmark}
\rhead{\botmark}

\subsection{\hspace{-0.5cm} {\Large \textcolor{darkblue}{\textbf{\ipa{mo˧\textsubscript{a}}}}}\hspace{0.5cm}[\kern2pt{\textcolor{darkblue}{\textbf{\ipa{mo˥}}}}\kern2pt]} \hypertarget{mo\string_Ma1}{}
\markboth{\textcolor{darkblue}{\textbf{\ipa{mo˧\textsubscript{a}}}}}{}
\textcolor{teal}{\mytextsc{classifier}} \hspace{4pt} Tone: M\textsubscript{a}.
\textcolor{Sepia}{\selectlanguage{english}Classifier for corpses.} \zh{量词:尸体。} 
\lhead{\firstmark}
\rhead{\botmark}

\subsection{\hspace{-0.5cm} {\Large \textcolor{darkblue}{\textbf{\ipa{mo˧ɖʐv̩˥}}}}\hspace{0.5cm}[\kern2pt{\textcolor{darkblue}{\textbf{\ipa{mo˧ɖʐv̩˥}}}}\kern2pt]} \hypertarget{mo\string_Md`z`v\string_=\string_T1}{}
\markboth{\textcolor{darkblue}{\textbf{\ipa{mo˧ɖʐv̩˥}}}}{}
\textcolor{teal}{\mytextsc{noun}} \hspace{4pt} Tone: H\#.
\textcolor{Sepia}{\selectlanguage{english}Morel, hickory chick: an edible mushroom.} \zh{羊肚菌。} Local Chinese dialect:\zh{羊菌。} ¶ \textcolor{darkblue}{\textbf{\ipa{ʂɯ˧-ɬi˧mi˧, | mo˧ɖʐv̩˥!}}} \textcolor{Sepia}{\selectlanguage{english}The seventh month is the season of morels!} \zh{七月份,是羊肚菌的季节!}  
 ¶ \textcolor{darkblue}{\textbf{\ipa{ʂɯ˧-ɬi˧mi˧ | mo˧ɖʐv̩˥-ne˩-ʝi˩-zo˩!}}} \textcolor{Sepia}{\selectlanguage{english}'[They have kids] like (=as numerous as) morels in the seventh month!', i.e. they have children in great abundance. This is a humorous comment made about people who had one child after the other. The abundance of morels in the seventh month is spectacular and proverbial.} \zh{(你们家孩子)生得像七月份的羊肚菌一样!(来形容一家有很多孩子出生,一个又一个。在永宁地区,七月份羊肚菌很多。)}  

\lhead{\firstmark}
\rhead{\botmark}

\subsection{\hspace{-0.5cm} {\Large \textcolor{darkblue}{\textbf{\ipa{mo˧jo˩-mi˩}}}}\hspace{0.5cm}[\kern2pt{\textcolor{darkblue}{\textbf{\ipa{mo˧jo˩mi˧}}}}\kern2pt]} \hypertarget{mo\string_Mjo\string_B-mi\string_B1}{}
\markboth{\textcolor{darkblue}{\textbf{\ipa{mo˧jo˩-mi˩}}}}{}
\textcolor{teal}{\mytextsc{noun}} \hspace{4pt} Tone: L\#-.
\textcolor{Sepia}{\selectlanguage{english}Owl.} \zh{猫头鹰。}  \zh{量词}: \textcolor{darkblue}{\textbf{\ipa{mi˩}}}  \mytextsc{clf}: \textcolor{darkblue}{\textbf{\ipa{mi˩}}} 
\lhead{\firstmark}
\rhead{\botmark}

\subsection{\hspace{-0.5cm} {\Large \textcolor{darkblue}{\textbf{\ipa{mo˧jo˩mi˩-pʰv̩˩}}}}\hspace{0.5cm}[\kern2pt{\textcolor{darkblue}{\textbf{\ipa{mo˧jo˧mi˩pʰv̩˧}}}}\kern2pt]} \hypertarget{mo\string_Mjo\string_Bmi\string_B-p\string_hv\string_=\string_B1}{}
\markboth{\textcolor{darkblue}{\textbf{\ipa{mo˧jo˩mi˩-pʰv̩˩}}}}{}
\textcolor{teal}{\mytextsc{noun}} \hspace{4pt} Tone: L\#-.
\textcolor{Sepia}{\selectlanguage{english}Male owl.} \zh{公猫头鹰。}  \zh{量词}: \textcolor{darkblue}{\textbf{\ipa{mi˩}}}  \mytextsc{clf}: \textcolor{darkblue}{\textbf{\ipa{mi˩}}} 
\lhead{\firstmark}
\rhead{\botmark}

\subsection{\hspace{-0.5cm} {\Large \textcolor{darkblue}{\textbf{\ipa{mo˧jo˩mi˩-zo˩}}}}\hspace{0.5cm}[\kern2pt{\textcolor{darkblue}{\textbf{\ipa{mo˧jo˧mi˩zo˧}}}}\kern2pt]} \hypertarget{mo\string_Mjo\string_Bmi\string_B-zo\string_B1}{}
\markboth{\textcolor{darkblue}{\textbf{\ipa{mo˧jo˩mi˩-zo˩}}}}{}
\textcolor{teal}{\mytextsc{noun}} \hspace{4pt} Tone: L\#-.
\textcolor{Sepia}{\selectlanguage{english}Baby owl.} \zh{小的猫头鹰。}  \zh{量词}: \textcolor{darkblue}{\textbf{\ipa{ɭɯ˧}}}  \mytextsc{clf}: \textcolor{darkblue}{\textbf{\ipa{ɭɯ˧}}} 
\lhead{\firstmark}
\rhead{\botmark}

\subsection{\hspace{-0.5cm} {\Large \textcolor{darkblue}{\textbf{\ipa{mo˧kɤ˩}}}}\hspace{0.5cm}[\kern2pt{\textcolor{darkblue}{\textbf{\ipa{mo˧kɤ˩}}}}\kern2pt]} \hypertarget{mo\string_Mk7\string_B1}{}
\markboth{\textcolor{darkblue}{\textbf{\ipa{mo˧kɤ˩}}}}{}
\textcolor{teal}{\mytextsc{noun}} \hspace{4pt} Tone: L\#.
\textcolor{Sepia}{\selectlanguage{english}Azalea.} \zh{杜鹃花、踯躅、山石榴、照山红、唐杜鹃。} Local Chinese dialect:\zh{杨花木。} ¶ \textcolor{darkblue}{\textbf{\ipa{mo˧kɤ˩-bæ˩bæ˩}}} \textcolor{Sepia}{\selectlanguage{english}azalea flowers} \zh{杜鹃花}  

\lhead{\firstmark}
\rhead{\botmark}

\subsection{\hspace{-0.5cm} {\Large \textcolor{darkblue}{\textbf{\ipa{mo˧ɬɑ˥}}}}\hspace{0.5cm}[\kern2pt{\textcolor{darkblue}{\textbf{\ipa{mo˧ɬɑ˥}}}}\kern2pt]} \hypertarget{mo\string_MKA\string_T1}{}
\markboth{\textcolor{darkblue}{\textbf{\ipa{mo˧ɬɑ˥}}}}{}
\textcolor{teal}{\mytextsc{noun}} \hspace{4pt} Tone: H\#.
\textcolor{Sepia}{\selectlanguage{english}Slander.} \zh{诬蔑、坏话。}  ¶ \textcolor{darkblue}{\textbf{\ipa{mo˧ɬɑ˥ ʐwɤ˩}}} \textcolor{Sepia}{\selectlanguage{english}to slander, to speak ill of others} \zh{说人的坏话}  

\lhead{\firstmark}
\rhead{\botmark}

\subsection{\hspace{-0.5cm} {\Large \textcolor{darkblue}{\textbf{\ipa{mo˧mo˥}}}}\hspace{0.5cm}[\kern2pt{\textcolor{darkblue}{\textbf{\ipa{mo˧mo˥}}}}\kern2pt]} \hypertarget{mo\string_Mmo\string_T1}{}
\markboth{\textcolor{darkblue}{\textbf{\ipa{mo˧mo˥}}}}{}
\textcolor{teal}{\mytextsc{noun}} \hspace{4pt} Tone: H\#.
\textcolor{Sepia}{\selectlanguage{english}Steamed bun.} \zh{馒头、包子。}  \zh{量词}: \textcolor{darkblue}{\textbf{\ipa{ɭɯ˧}}}  \mytextsc{clf}: \textcolor{darkblue}{\textbf{\ipa{ɭɯ˧}}} 
\lhead{\firstmark}
\rhead{\botmark}

\subsection{\hspace{-0.5cm} {\Large \textcolor{darkblue}{\textbf{\ipa{mo˧nɑ˥}}} \textsubscript{1}}\hspace{0.5cm}[\kern2pt{\textcolor{darkblue}{\textbf{\ipa{mo˧nɑ˥}}}}\kern2pt]} \hypertarget{mo\string_MnA\string_T1}{}
\markboth{\textcolor{darkblue}{\textbf{\ipa{mo˧nɑ˥}}} \textsubscript{1}}{}
\textcolor{teal}{\mytextsc{noun}} \hspace{4pt} Tone: H\#.
\textcolor{Sepia}{\selectlanguage{english}Gossip.} \zh{闲话。}  ¶ \textcolor{darkblue}{\textbf{\ipa{mo˧nɑ˥ ʐwɤ˩}}} \textcolor{Sepia}{\selectlanguage{english}to indulge in gossip, to speak badly of others} \zh{八卦、讲别人的坏话}  
 ¶ \textcolor{darkblue}{\textbf{\ipa{ʈʂʰɯ˧ | ɖɯ˧-ɲi˥ | mo˧nɑ˥ ʐwɤ˩-dʑo˩!}}} \textcolor{Sepia}{\selectlanguage{english}(S)he gossips all day!} \zh{他一天到晚都在八卦!}  
 ¶ \textcolor{darkblue}{\textbf{\ipa{mo˧nɑ˥-ɕi˩mi˩}}} \textcolor{Sepia}{\selectlanguage{english}same meaning: gossip} \zh{同上:八卦、坏话}  
 ¶ \textcolor{darkblue}{\textbf{\ipa{mo˧nɑ˥-ɕi˩mi˩ ʐwɤ˩}}} \textcolor{Sepia}{\selectlanguage{english}to indulge in gossip, to speak badly of others} \zh{八卦、讲别人的坏话}  
 ¶ \textcolor{darkblue}{\textbf{\ipa{hĩ˧ | ʈʂʰɯ˧-v̩˧, | mo˧nɑ˥-ɕi˩mi˩ | ɖɯ˧-v̩˧ ɲi˩!}}} \textcolor{Sepia}{\selectlanguage{english}He's a gossiper, he talks badly of other people} \zh{他爱八卦、爱说别人坏话}  

\lhead{\firstmark}
\rhead{\botmark}

\subsection{\hspace{-0.5cm} {\Large \textcolor{darkblue}{\textbf{\ipa{mo˧nɑ˥}}} \textsubscript{2}}\hspace{0.5cm}[\kern2pt{\textcolor{darkblue}{\textbf{\ipa{mo˧nɑ˥}}}}\kern2pt]} \hypertarget{mo\string_MnA\string_T2}{}
\markboth{\textcolor{darkblue}{\textbf{\ipa{mo˧nɑ˥}}} \textsubscript{2}}{}
\textcolor{teal}{\mytextsc{noun}} \hspace{4pt} Tone: H\#.
\textcolor{Sepia}{\selectlanguage{english}Chopped straw, used when preparing pickled vegetables.} \zh{剁成粉的秸杆。}  ¶ \textcolor{darkblue}{\textbf{\ipa{mv˩zɯ˩-mo˩nɑ˥}}} \textcolor{Sepia}{\selectlanguage{english}chopped oat straw} \zh{剁成粉的燕麦秸杆}  

\lhead{\firstmark}
\rhead{\botmark}

\subsection{\hspace{-0.5cm} {\Large \textcolor{darkblue}{\textbf{\ipa{mo˧qʰwɤ˥}}}}\hspace{0.5cm}[\kern2pt{\textcolor{darkblue}{\textbf{\ipa{mo˧qʰwɤ˥}}}}\kern2pt]} \hypertarget{mo\string_Mq\string_hw7\string_T1}{}
\markboth{\textcolor{darkblue}{\textbf{\ipa{mo˧qʰwɤ˥}}}}{}
\textcolor{teal}{\mytextsc{noun}} \hspace{4pt} Tone: H\#.
\textcolor{Sepia}{\selectlanguage{english}Wooden shuttle of zip line (flying fox): it glides along the rope; the passenger, horse, or load of goods is tied to the shuttle.} \zh{溜索上往返移动的木头梭。}  \zh{量词}: \textcolor{darkblue}{\textbf{\ipa{ɭɯ˧}}}  \mytextsc{clf}: \textcolor{darkblue}{\textbf{\ipa{ɭɯ˧}}} 
\lhead{\firstmark}
\rhead{\botmark}

\subsection{\hspace{-0.5cm} {\Large \textcolor{darkblue}{\textbf{\ipa{mo˧qʰwɤ˧˥}}}}\hspace{0.5cm}[\kern2pt{\textcolor{darkblue}{\textbf{\ipa{mo˧qʰwɤ˧˥}}}}\kern2pt]} \hypertarget{mo\string_Mq\string_hw7\string_M\string_T1}{}
\markboth{\textcolor{darkblue}{\textbf{\ipa{mo˧qʰwɤ˧˥}}}}{}
\textcolor{teal}{\mytextsc{adjective}} \hspace{4pt} Tone: MH.
\textcolor{Sepia}{\selectlanguage{english}Fond of food; voracious (can range from neutral uses to clearly negative uses).} \zh{胃口好,或:贪吃。} 
\lhead{\firstmark}
\rhead{\botmark}

\subsection{\hspace{-0.5cm} {\Large \textcolor{darkblue}{\textbf{\ipa{mo˩}}}}\hspace{0.5cm}[\kern2pt{\textcolor{darkblue}{\textbf{\ipa{mo˩˥}}}}\kern2pt]} \hypertarget{mo\string_B1}{}
\markboth{\textcolor{darkblue}{\textbf{\ipa{mo˩}}}}{}
\textcolor{teal}{\mytextsc{discourse}} \textcolor{teal}{\mytextsc{particle}} \hspace{4pt} Tone: L.
\textcolor{Sepia}{\selectlanguage{english}Final particle indicating invitation/suggestion to do something.} \zh{句尾助词:请……。}  ¶ \textcolor{darkblue}{\textbf{\ipa{no˧ | ɖɯ˧-ʈʰɯ˩-ɻ̍˩ mo˩!}}} \textcolor{Sepia}{\selectlanguage{english}Please drink a little! / Do have a sip!} \zh{请你喝一点!}  

\lhead{\firstmark}
\rhead{\botmark}

\subsection{\hspace{-0.5cm} {\Large \textcolor{darkblue}{\textbf{\ipa{mo˩\textsubscript{a}}}} \textsubscript{1}}\hspace{0.5cm}[\kern2pt{\textcolor{darkblue}{\textbf{\ipa{mo˩˥}}}}\kern2pt]} \hypertarget{mo\string_Ba1}{}
\markboth{\textcolor{darkblue}{\textbf{\ipa{mo˩\textsubscript{a}}}} \textsubscript{1}}{}
\textcolor{teal}{\mytextsc{adjective}} \hspace{4pt} Tone: L\textsubscript{a}.
\textcolor{Sepia}{\selectlanguage{english}Old.} \zh{年老。}  ¶ \textcolor{darkblue}{\textbf{\ipa{mo˩ hĩ˩˥}}} \textcolor{Sepia}{\selectlanguage{english}old person} \zh{老人}  
 ¶ \textcolor{darkblue}{\textbf{\ipa{si˧ mo˥}}} \textcolor{Sepia}{\selectlanguage{english}old wood; old tree} \zh{老树、老木头}  
 ¶ \textcolor{darkblue}{\textbf{\ipa{le˧-mo˩-ze˩}}} \textcolor{Sepia}{\selectlanguage{english}\mytextsc{accomp} \string_ \mytextsc{pfv}: (he/she) has become old / has aged.} \zh{\mytextsc{accomp} \string_ \mytextsc{pfv}}  
 ¶ \textcolor{darkblue}{\textbf{\ipa{le˧-mo˩-hĩ˩}}} \textcolor{Sepia}{\selectlanguage{english}Old person, person who has become old} \zh{老了的人}  
 ¶ \textcolor{darkblue}{\textbf{\ipa{hĩ˧ mo˥, | õ˧-di˧ fv̩˥! | ʐwæ˧ mo˥, | to˩ do˩ ɖwæ˥!}}} \textcolor{Sepia}{\selectlanguage{english}Old folk like their own place (=their own home); old horses are afraid to climb slopes! (Proverb.)} \zh{老人爱自家,老马怕山坡!(谚语,描写不爱到处跑的老年人)}  
 ¶ \textcolor{darkblue}{\textbf{\ipa{lv̩˧ mo˥ F | dʑɯ˧ | le˧-qv̩˩; | si˧ mo˥ F | le˧-dze˩-kv̩˩! | no˧ F | ə˧tse˧ | le˧-ʂɯ˧-mɤ˧-tʰɑ˧˥ | di˩!}}} \textcolor{Sepia}{\selectlanguage{english}Old stones are carried away by the stream; and old wood gets chopped down! And you, why can't you die? (Mocking an elderly person. Na tradition assigns man a lifespan of sixty years; people getting past seventy are considered to have reached the end of their lifespan.)} \zh{老石头要被河流冲走,老木头要被砍掉。你呢,怎么还不死? (嘲笑一个年龄很高的人。摩梭传统中,人的寿命是六十岁:过了七十岁的人,被认为是已经到达了命的尽头。)}  

\lhead{\firstmark}
\rhead{\botmark}

\subsection{\hspace{-0.5cm} {\Large \textcolor{darkblue}{\textbf{\ipa{mo˩\textsubscript{a}}}} \textsubscript{2}}\hspace{0.5cm}[\kern2pt{\textcolor{darkblue}{\textbf{\ipa{mo˩˥}}}}\kern2pt]} \hypertarget{mo\string_Ba2}{}
\markboth{\textcolor{darkblue}{\textbf{\ipa{mo˩\textsubscript{a}}}} \textsubscript{2}}{}
\textcolor{teal}{\mytextsc{verb}} \hspace{4pt} Tone: L\textsubscript{a}.
\textcolor{Sepia}{\selectlanguage{english}To die.} \zh{死、去世。}  ¶ \textcolor{darkblue}{\textbf{\ipa{mɤ˧-mo˩-sɯ˩!}}} \textcolor{Sepia}{\selectlanguage{english}(She/he/it) is not dead yet!} \zh{还没死!}  
 ¶ \textcolor{darkblue}{\textbf{\ipa{si˧ mo˩}}} \textcolor{Sepia}{\selectlanguage{english}dead wood} \zh{老干柴(直译:死了的木头)}  

\lhead{\firstmark}
\rhead{\botmark}

\subsection{\hspace{-0.5cm} {\Large \textcolor{darkblue}{\textbf{\ipa{mo˩kv̩\#˥}}}}\hspace{0.5cm}[\kern2pt{\textcolor{darkblue}{\textbf{\ipa{mo˩kv̩˥}}}}\kern2pt]} \hypertarget{mo\string_Bkv\string_=\#\string_T1}{}
\markboth{\textcolor{darkblue}{\textbf{\ipa{mo˩kv̩\#˥}}}}{}
\textcolor{teal}{\mytextsc{noun}} \hspace{4pt} Tone: LM+\#H.
\textcolor{Sepia}{\selectlanguage{english}Mushrooms that grows on fallen trunks, e.g. oaks.} \zh{蘑菇:长在倒在地上的树(如青冈等树木)上的菌子(汉语借词)。}  Borrowing: Chinese  \zh{蘑菇}
 ¶ \textcolor{darkblue}{\textbf{\ipa{mo˩kv̩˥, | si˧dzi˩-mo˩!}}} \textcolor{Sepia}{\selectlanguage{english}\textcolor{darkblue}{\textbf{\ipa{/mo˩kv̩\#˥/}}} refers to mushrooms that grow on trees!} \zh{\textcolor{darkblue}{\textbf{\ipa{/mo˩kv̩\#˥/}}},指的是长在(倒在地上的)树上的菌子!}  

\lhead{\firstmark}
\rhead{\botmark}

\subsection{\hspace{-0.5cm} {\Large \textcolor{darkblue}{\textbf{\ipa{mo˩ɻ\#˥}}}}\hspace{0.5cm}[\kern2pt{\textcolor{darkblue}{\textbf{\ipa{mo˩ɻ˥}}}}\kern2pt]} \hypertarget{mo\string_Br£`\#\string_T1}{}
\markboth{\textcolor{darkblue}{\textbf{\ipa{mo˩ɻ\#˥}}}}{}
\textcolor{teal}{\mytextsc{noun}} \hspace{4pt} Tone: LM+\#H.
\textcolor{Sepia}{\selectlanguage{english}Black mushroom, 'wood ear' (an edible fungus).} \zh{木耳(汉语借词)。}  Borrowing: Chinese  \zh{木耳}

\lhead{\firstmark}
\rhead{\botmark}

\subsection{\hspace{-0.5cm} {\Large \textcolor{darkblue}{\textbf{\ipa{mo˩zo\#˥}}}}\hspace{0.5cm}[\kern2pt{\textcolor{darkblue}{\textbf{\ipa{mo˩zo˥}}}}\kern2pt]} \hypertarget{mo\string_Bzo\#\string_T1}{}
\markboth{\textcolor{darkblue}{\textbf{\ipa{mo˩zo\#˥}}}}{}
\textcolor{teal}{\mytextsc{noun}} \hspace{4pt} Tone: LM+\#H.
\textcolor{Sepia}{\selectlanguage{english}Soldier.} \zh{士兵。}  ¶ \textcolor{darkblue}{\textbf{\ipa{mo˩zo˧ ʝi˧-hɯ˧ ɲi˥!}}} \textcolor{Sepia}{\selectlanguage{english}He went to the army! / He joined the army! / He became a soldier!} \zh{当兵去了!}  
 \zh{量词}: \textcolor{darkblue}{\textbf{\ipa{v̩˧}}}  \mytextsc{clf}: \textcolor{darkblue}{\textbf{\ipa{v̩˧}}} 
\lhead{\firstmark}
\rhead{\botmark}

\subsection{\hspace{-0.5cm} {\Large \textcolor{darkblue}{\textbf{\ipa{mo˧˥}}}}\hspace{0.5cm}[\kern2pt{\textcolor{darkblue}{\textbf{\ipa{mo˧˥}}}}\kern2pt]} \hypertarget{mo\string_M\string_T1}{}
\markboth{\textcolor{darkblue}{\textbf{\ipa{mo˧˥}}}}{}
\textcolor{teal}{\mytextsc{noun}} \hspace{4pt} Tone: MH.
\textcolor{Sepia}{\selectlanguage{english}Mushroom.} \zh{菌子、蘑菇。}  \zh{量词}: \textcolor{darkblue}{\textbf{\ipa{ɭɯ˧}}} \textcolor{darkblue}{\textbf{\ipa{mo˧˥}}}  \mytextsc{clf}: \textcolor{darkblue}{\textbf{\ipa{ɭɯ˧}}} \textcolor{darkblue}{\textbf{\ipa{mo˧˥}}} \textit{See:} \hyperlink{}{\textcolor{darkblue}{\textbf{\ipa{mo˧˥\textsubscript{a}}}}} 
\lhead{\firstmark}
\rhead{\botmark}

\subsection{\hspace{-0.5cm} {\Large \textcolor{darkblue}{\textbf{\ipa{mo˧˥\textsubscript{a}}}}}\hspace{0.5cm}[\kern2pt{\textcolor{darkblue}{\textbf{\ipa{mo˧˥}}}}\kern2pt]} \hypertarget{mo\string_M\string_Ta1}{}
\markboth{\textcolor{darkblue}{\textbf{\ipa{mo˧˥\textsubscript{a}}}}}{}
\textcolor{teal}{\mytextsc{classifier}} \hspace{4pt} Tone: MH\textsubscript{a}.
\textcolor{Sepia}{\selectlanguage{english}Self-classifier for mushrooms.} \zh{量词:蘑菇(一只)。} \textit{See:} \hyperlink{}{\textcolor{darkblue}{\textbf{\ipa{mo˧˥}}}} 
\lhead{\firstmark}
\rhead{\botmark}

\subsection{\hspace{-0.5cm} {\Large \textcolor{darkblue}{\textbf{\ipa{mv̩˩˥}}}}\hspace{0.5cm}[\kern2pt{\textcolor{darkblue}{\textbf{\ipa{mv̩˩˥}}}}\kern2pt]} \hypertarget{mv\string_=\string_B\string_T1}{}
\markboth{\textcolor{darkblue}{\textbf{\ipa{mv̩˩˥}}}}{}
\textcolor{teal}{\mytextsc{noun}} \hspace{4pt} Tone: LH.
\textcolor{Sepia}{\selectlanguage{english}Daughter.} \zh{女儿。}  \zh{量词}: \textcolor{darkblue}{\textbf{\ipa{v̩˧}}}  \mytextsc{clf}: \textcolor{darkblue}{\textbf{\ipa{v̩˧}}} 
\lhead{\firstmark}
\rhead{\botmark}

\subsection{\hspace{-0.5cm} {\Large \textcolor{darkblue}{\textbf{\ipa{mv̩˧}}}}\hspace{0.5cm}[\kern2pt{\textcolor{darkblue}{\textbf{\ipa{mv̩˥}}}}\kern2pt]} \hypertarget{mv\string_=\string_M1}{}
\markboth{\textcolor{darkblue}{\textbf{\ipa{mv̩˧}}}}{}
\textcolor{teal}{\mytextsc{noun}} \hspace{4pt} Tone: M.
\textcolor{Sepia}{\selectlanguage{english}Name (given name or family name).} \zh{姓名。}  ¶ \textcolor{darkblue}{\textbf{\ipa{ɑ˩ʁo˧-bv̩˧ | mv̩˧ (+ɲi˩)}}} \textcolor{Sepia}{\selectlanguage{english}This is the family name! / This is my family name!} \zh{这是家里的姓! / 这是我家的姓!}  
 ¶ \textcolor{darkblue}{\textbf{\ipa{njɤ˧ | mv̩˧ ɖɯ˧-kʰwɤ˥ | ʂe˧-zo˧-ho˩!}}} \textcolor{Sepia}{\selectlanguage{english}I have to go and get a name (from the monks at the monastery) (for a newborn child)} \zh{我得去(向大寺里的和尚)求一个名字(给刚出生的孩子起名)}  

\lhead{\firstmark}
\rhead{\botmark}

\subsection{\hspace{-0.5cm} {\Large \textcolor{darkblue}{\textbf{\ipa{mv̩˧}}}}\hspace{0.5cm}[\kern2pt{\textcolor{darkblue}{\textbf{\ipa{mv̩˥}}}}\kern2pt]} \hypertarget{mv\string_=\string_M1}{}
\markboth{\textcolor{darkblue}{\textbf{\ipa{mv̩˧}}}}{}
\textcolor{teal}{\mytextsc{discourse}} \textcolor{teal}{\mytextsc{particle}} \hspace{4pt} Tone: M.
\textcolor{Sepia}{\selectlanguage{english}Affirmative final particle.} \zh{句尾助词,表示肯定:“嘛”。} 
\lhead{\firstmark}
\rhead{\botmark}

\subsection{\hspace{-0.5cm} {\Large \textcolor{darkblue}{\textbf{\ipa{mv̩˥}}}}\hspace{0.5cm}[\kern2pt{\textcolor{darkblue}{\textbf{\ipa{mv̩˥}}}}\kern2pt]} \hypertarget{mv\string_=\string_T1}{}
\markboth{\textcolor{darkblue}{\textbf{\ipa{mv̩˥}}}}{}
\textcolor{teal}{\mytextsc{verb}} \hspace{4pt} Tone: H.
\ding{202} \textcolor{Sepia}{\selectlanguage{english}To hear.} \zh{懂,听见。}  ¶ \textcolor{darkblue}{\textbf{\ipa{njɤ˧ | le˧-mv̩˥-ze˩}}} \textcolor{Sepia}{\selectlanguage{english}I have heard} \zh{我听见了}  
\ding{203} \textcolor{Sepia}{\selectlanguage{english}To understand.} \zh{懂。}  ¶ \textcolor{darkblue}{\textbf{\ipa{njɤ˧ | le˧-mv̩˥-ze˩}}} \textcolor{Sepia}{\selectlanguage{english}I have understood} \zh{我懂了}  

\lhead{\firstmark}
\rhead{\botmark}

\subsection{\hspace{-0.5cm} {\Large \textcolor{darkblue}{\textbf{\ipa{mv̩˥}}} \textsubscript{1}}\hspace{0.5cm}[\kern2pt{\textcolor{darkblue}{\textbf{\ipa{mv̩˥}}}}\kern2pt]} \hypertarget{mv\string_=\string_T1}{}
\markboth{\textcolor{darkblue}{\textbf{\ipa{mv̩˥}}} \textsubscript{1}}{}
\textcolor{teal}{\mytextsc{noun}} \hspace{4pt} Tone: \#H.
\textcolor{Sepia}{\selectlanguage{english}Sky.} \zh{天。}  ¶ \textcolor{darkblue}{\textbf{\ipa{mv̩˧tʰv̩˧(-ze˩)}}} \textcolor{Sepia}{\selectlanguage{english}the day is bright, the sky is clear} \zh{天晴,天色亮}  
 ¶ \textcolor{darkblue}{\textbf{\ipa{hĩ˧-ɳɯ˩ mɤ˩-do˩, | mv̩˧-ɳɯ˩ | do˩˥!}}} \textcolor{Sepia}{\selectlanguage{english}“What humans do not see, the Heavens see it!” (Meaning: a good deed earns one happiness in future; and a count of bad deeds, even those that go unseen by humans, is also kept in the Heavens.)} \zh{“人看不见的,老天能看见!”}  
 \zh{量词}: \textcolor{darkblue}{\textbf{\ipa{ɭɯ˧}}}  \mytextsc{clf}: \textcolor{darkblue}{\textbf{\ipa{ɭɯ˧}}} 
\lhead{\firstmark}
\rhead{\botmark}

\subsection{\hspace{-0.5cm} {\Large \textcolor{darkblue}{\textbf{\ipa{mv̩˥}}} \textsubscript{2}}\hspace{0.5cm}[\kern2pt{\textcolor{darkblue}{\textbf{\ipa{mv̩˥}}}}\kern2pt]} \hypertarget{mv\string_=\string_T2}{}
\markboth{\textcolor{darkblue}{\textbf{\ipa{mv̩˥}}} \textsubscript{2}}{}
\textcolor{teal}{\mytextsc{noun}} \hspace{4pt} Tone: \#H.
\textcolor{Sepia}{\selectlanguage{english}Fire.} \zh{火。}  ¶ \textcolor{darkblue}{\textbf{\ipa{mv̩˧ kʰɯ˩}}} \textcolor{Sepia}{\selectlanguage{english}to light a fire, to do a fire} \zh{点火}  
 \zh{量词}: \textcolor{darkblue}{\textbf{\ipa{æ̃˩}}}  \mytextsc{clf}: \textcolor{darkblue}{\textbf{\ipa{æ̃˩}}} 
\lhead{\firstmark}
\rhead{\botmark}

\subsection{\hspace{-0.5cm} {\Large \textcolor{darkblue}{\textbf{\ipa{mv̩˩\textsubscript{a}}}} \textsubscript{1}}\hspace{0.5cm}[\kern2pt{\textcolor{darkblue}{\textbf{\ipa{mv̩˥}}}}\kern2pt]} \hypertarget{mv\string_=\string_Ba1}{}
\markboth{\textcolor{darkblue}{\textbf{\ipa{mv̩˩\textsubscript{a}}}} \textsubscript{1}}{}
\textcolor{teal}{\mytextsc{verb}} \hspace{4pt} Tone: L\textsubscript{a}.
\textcolor{Sepia}{\selectlanguage{english}To blow (e.g. to blow the fire, to blow a horn).} \zh{吹(灰,乐器)。}  ¶ \textcolor{darkblue}{\textbf{\ipa{mv̩˧\textasciitilde{}mv̩˥(-ze˩)}}} \textcolor{Sepia}{\selectlanguage{english}\mytextsc{red}} \zh{\mytextsc{重叠:吹吹}}  
 ¶ \textcolor{darkblue}{\textbf{\ipa{ʝi˧qʰv̩˧ mv̩˥}}} \textcolor{Sepia}{\selectlanguage{english}to blow a horn} \zh{吹号角}  

\lhead{\firstmark}
\rhead{\botmark}

\subsection{\hspace{-0.5cm} {\Large \textcolor{darkblue}{\textbf{\ipa{mv̩˩\textsubscript{a}}}} \textsubscript{2}}\hspace{0.5cm}[\kern2pt{\textcolor{darkblue}{\textbf{\ipa{mv̩˩˥}}}}\kern2pt]} \hypertarget{mv\string_=\string_Ba2}{}
\markboth{\textcolor{darkblue}{\textbf{\ipa{mv̩˩\textsubscript{a}}}} \textsubscript{2}}{}
\textcolor{teal}{\mytextsc{verb}} \hspace{4pt} Tone: L\textsubscript{a}.
\textcolor{Sepia}{\selectlanguage{english}To flush away, to carry away: a strong current carries a swimmer away.} \zh{冲(走)。} 
\lhead{\firstmark}
\rhead{\botmark}

\subsection{\hspace{-0.5cm} {\Large \textcolor{darkblue}{\textbf{\ipa{mv̩˩\textsubscript{a}}}} \textsubscript{3}}\hspace{0.5cm}[\kern2pt{\textcolor{darkblue}{\textbf{\ipa{mv̩˩˥}}}}\kern2pt]} \hypertarget{mv\string_=\string_Ba3}{}
\markboth{\textcolor{darkblue}{\textbf{\ipa{mv̩˩\textsubscript{a}}}} \textsubscript{3}}{}
\textcolor{teal}{\mytextsc{adjective}} \hspace{4pt} Tone: L\textsubscript{a}.
\ding{202} \textcolor{Sepia}{\selectlanguage{english}Ripe.} \zh{熟、成熟(植物、水果)。}  ¶ \textcolor{darkblue}{\textbf{\ipa{mv̩˩-hĩ˩˥}}} \textcolor{Sepia}{\selectlanguage{english}\mytextsc{rel}} \zh{熟的}  
\ding{203} \textcolor{Sepia}{\selectlanguage{english}Cooked, well-cooked, done.} \zh{熟(食物)。} 
\lhead{\firstmark}
\rhead{\botmark}

\subsection{\hspace{-0.5cm} {\Large \textcolor{darkblue}{\textbf{\ipa{mv̩˩\textsubscript{a}}}} \textsubscript{4}}\hspace{0.5cm}[\kern2pt{\textcolor{darkblue}{\textbf{\ipa{mv̩˩˥}}}}\kern2pt]} \hypertarget{mv\string_=\string_Ba4}{}
\markboth{\textcolor{darkblue}{\textbf{\ipa{mv̩˩\textsubscript{a}}}} \textsubscript{4}}{}
\textcolor{teal}{\mytextsc{verb}} \hspace{4pt} Tone: L\textsubscript{a}.
\textcolor{Sepia}{\selectlanguage{english}To burn, to become consumed (e.g. a body on the funeral pyre becomes consumed).} \zh{燃烧。} 
\lhead{\firstmark}
\rhead{\botmark}

\subsection{\hspace{-0.5cm} {\Large \textcolor{darkblue}{\textbf{\ipa{mv̩˧\textsubscript{a}}}}}\hspace{0.5cm}[\kern2pt{\textcolor{darkblue}{\textbf{\ipa{mv̩˩˥}}}}\kern2pt]} \hypertarget{mv\string_=\string_Ma1}{}
\markboth{\textcolor{darkblue}{\textbf{\ipa{mv̩˧\textsubscript{a}}}}}{}
\textcolor{teal}{\mytextsc{verb}} \hspace{4pt} Tone: M\textsubscript{a}.
\textcolor{Sepia}{\selectlanguage{english}To put on (a shirt, a jacket).} \zh{穿(衣服、上衣)。}  ¶ \textcolor{darkblue}{\textbf{\ipa{bɑ˩lɑ˩ mv̩˥}}} \textcolor{Sepia}{\selectlanguage{english}to put on a shirt/jacket} \zh{穿衣服}  
 ¶ \textcolor{darkblue}{\textbf{\ipa{bɑ˩lɑ˩˥ | tʰi˧-mv̩˧}}} \textcolor{Sepia}{\selectlanguage{english}to put on a shirt/jacket} \zh{穿衣服}  
 ¶ \textcolor{darkblue}{\textbf{\ipa{dʑi˧hṽ˧ mv̩˩}}} \textcolor{Sepia}{\selectlanguage{english}to put on clothes} \zh{穿衣服}  
\textit{See:} \hyperlink{}{\textcolor{darkblue}{\textbf{\ipa{ki˩\textsubscript{a}}}}} 
\lhead{\firstmark}
\rhead{\botmark}

\subsection{\hspace{-0.5cm} {\Large \textcolor{darkblue}{\textbf{\ipa{mv̩˩-bæ˧mi˩}}}}\hspace{0.5cm}[\kern2pt{\textcolor{darkblue}{\textbf{\ipa{xxxx non-correspondance entre le nombre de morphèmes et le nombre de tons de morphèmes}}}}\kern2pt]} \hypertarget{mv\string_=\string_B-b\{\string_Mmi\string_B1}{}
\markboth{\textcolor{darkblue}{\textbf{\ipa{mv̩˩-bæ˧mi˩}}}}{}
\textcolor{teal}{\mytextsc{noun}} \hspace{4pt} Tone: L-L\#.
\textcolor{Sepia}{\selectlanguage{english}Fool, idiot (female).} \zh{傻女人、笨女人。}  \zh{量词}: \textcolor{darkblue}{\textbf{\ipa{v̩˧}}}  \mytextsc{clf}: \textcolor{darkblue}{\textbf{\ipa{v̩˧}}} 
\lhead{\firstmark}
\rhead{\botmark}

\subsection{\hspace{-0.5cm} {\Large \textcolor{darkblue}{\textbf{\ipa{mv̩˧bɤ\#˥}}}}\hspace{0.5cm}[\kern2pt{\textcolor{darkblue}{\textbf{\ipa{xxxx non-correspondance entre le nombre de morphèmes et le nombre de tons de morphèmes}}}}\kern2pt]} \hypertarget{mv\string_=\string_Mb7\#\string_T1}{}
\markboth{\textcolor{darkblue}{\textbf{\ipa{mv̩˧bɤ\#˥}}}}{}
\textcolor{teal}{\mytextsc{noun}} \hspace{4pt} Tone: \#H.
\textcolor{Sepia}{\selectlanguage{english}Sole of the foot.} \zh{脚底。}  \zh{量词}: \textcolor{darkblue}{\textbf{\ipa{kʰwɤ˥}}}  \mytextsc{clf}: \textcolor{darkblue}{\textbf{\ipa{kʰwɤ˥}}} 
\lhead{\firstmark}
\rhead{\botmark}

\subsection{\hspace{-0.5cm} {\Large \textcolor{darkblue}{\textbf{\ipa{mv̩˧bv̩˧ʐv̩˥}}}}\hspace{0.5cm}[\kern2pt{\textcolor{darkblue}{\textbf{\ipa{mv̩˧bv̩˧ʐv̩˧}}}}\kern2pt]} \hypertarget{mv\string_=\string_Mbv\string_=\string_Mz`v\string_=\string_T1}{}
\markboth{\textcolor{darkblue}{\textbf{\ipa{mv̩˧bv̩˧ʐv̩˥}}}}{}
\textcolor{teal}{\mytextsc{noun}} \hspace{4pt} Tone: H\#.
\textcolor{Sepia}{\selectlanguage{english}Dragon.} \zh{龙。}  \zh{量词}: \textcolor{darkblue}{\textbf{\ipa{mi˩}}}  \mytextsc{clf}: \textcolor{darkblue}{\textbf{\ipa{mi˩}}} 
\lhead{\firstmark}
\rhead{\botmark}

\subsection{\hspace{-0.5cm} {\Large \textcolor{darkblue}{\textbf{\ipa{mv̩˧ɕi˥}}}}\hspace{0.5cm}[\kern2pt{\textcolor{darkblue}{\textbf{\ipa{mv̩˧ɕi˥}}}}\kern2pt]} \hypertarget{mv\string_=\string_Ms£i\string_T1}{}
\markboth{\textcolor{darkblue}{\textbf{\ipa{mv̩˧ɕi˥}}}}{}
\textcolor{teal}{\mytextsc{noun}} \hspace{4pt} Tone: H\#.
\textcolor{Sepia}{\selectlanguage{english}Spark.} \zh{火花。}  \zh{量词}: \textcolor{darkblue}{\textbf{\ipa{æ̃˩}}}  \mytextsc{clf}: \textcolor{darkblue}{\textbf{\ipa{æ̃˩}}} 
\lhead{\firstmark}
\rhead{\botmark}

\subsection{\hspace{-0.5cm} {\Large \textcolor{darkblue}{\textbf{\ipa{mv̩˧ɕi˥dʑɯ˩ʈʰɯ˩}}}}\hspace{0.5cm}[\kern2pt{\textcolor{darkblue}{\textbf{\ipa{mv̩˧ɕi˧dʑɯ˧ʈʰɯ˥}}}}\kern2pt]} \hypertarget{mv\string_=\string_Ms£i\string_Tdz£M\string_Bt`\string_hM\string_B1}{}
\markboth{\textcolor{darkblue}{\textbf{\ipa{mv̩˧ɕi˥dʑɯ˩ʈʰɯ˩}}}}{}
\textcolor{teal}{\mytextsc{noun}} \hspace{4pt} Tone: H\#-L.
\textcolor{Sepia}{\selectlanguage{english}Rainbow.} \zh{彩虹。}  \zh{量词}: \textcolor{darkblue}{\textbf{\ipa{kʰɯ˩}}}  \mytextsc{clf}: \textcolor{darkblue}{\textbf{\ipa{kʰɯ˩}}} 
\lhead{\firstmark}
\rhead{\botmark}

\subsection{\hspace{-0.5cm} {\Large \textcolor{darkblue}{\textbf{\ipa{mv̩˩ɖæ˧}}}}\hspace{0.5cm}[\kern2pt{\textcolor{darkblue}{\textbf{\ipa{xxxx non-correspondance entre le nombre de morphèmes et le nombre de tons de morphèmes}}}}\kern2pt]} \hypertarget{mv\string_=\string_Bd`\{\string_M1}{}
\markboth{\textcolor{darkblue}{\textbf{\ipa{mv̩˩ɖæ˧}}}}{}
\textcolor{teal}{\mytextsc{noun}} \hspace{4pt} Tone: LM.
\textcolor{Sepia}{\selectlanguage{english}Bottom part of body.} \zh{下半身。} 
\lhead{\firstmark}
\rhead{\botmark}

\subsection{\hspace{-0.5cm} {\Large \textcolor{darkblue}{\textbf{\ipa{mv̩˧di˧˥}}}}\hspace{0.5cm}[\kern2pt{\textcolor{darkblue}{\textbf{\ipa{mv̩˩di˥}}}}\kern2pt]} \hypertarget{mv\string_=\string_Mdi\string_M\string_T1}{}
\markboth{\textcolor{darkblue}{\textbf{\ipa{mv̩˧di˧˥}}}}{}
\textcolor{teal}{\mytextsc{noun}} \hspace{4pt} Tone: MH\#.
\ding{202} \textcolor{Sepia}{\selectlanguage{english}Field.} \zh{田地。}  \zh{量词}: \textcolor{darkblue}{\textbf{\ipa{kɤ˧˥}}} \ding{203} \textcolor{Sepia}{\selectlanguage{english}The Earth, the place where mankind lives (as opposed to the Heavens).} \zh{天下。}  \mytextsc{clf}: \textcolor{darkblue}{\textbf{\ipa{kɤ˧˥}}} 
\lhead{\firstmark}
\rhead{\botmark}

\subsection{\hspace{-0.5cm} {\Large \textcolor{darkblue}{\textbf{\ipa{mv̩˩do˩}}}}\hspace{0.5cm}[\kern2pt{\textcolor{darkblue}{\textbf{\ipa{mv̩˧do˧˥}}}}\kern2pt]} \hypertarget{mv\string_=\string_Bdo\string_B1}{}
\markboth{\textcolor{darkblue}{\textbf{\ipa{mv̩˩do˩}}}}{}
\textcolor{teal}{\mytextsc{verb}} \hspace{4pt} Tone: L.
\textcolor{Sepia}{\selectlanguage{english}To ask.} \zh{问。}  ¶ \textcolor{darkblue}{\textbf{\ipa{le˧-mv̩˩do˩}}} \textcolor{Sepia}{\selectlanguage{english}\mytextsc{accomp}} \zh{\mytextsc{accomp}}  
 ¶ \textcolor{darkblue}{\textbf{\ipa{mv̩˩do˩-ze˥}}} \textcolor{Sepia}{\selectlanguage{english}\mytextsc{pfv}} \zh{问了}  
 ¶ \textcolor{darkblue}{\textbf{\ipa{ə˧tso˧ mv̩˩do˩-bi˩? |}}} \textcolor{Sepia}{\selectlanguage{english}What would [you] like to ask? / What is your question?} \zh{要问什么呢?}  

\lhead{\firstmark}
\rhead{\botmark}

\subsection{\hspace{-0.5cm} {\Large \textcolor{darkblue}{\textbf{\ipa{mv̩˩ɖɯ˩}}}}\hspace{0.5cm}[\kern2pt{\textcolor{darkblue}{\textbf{\ipa{mv̩˩ɖɯ˩˥}}}}\kern2pt]} \hypertarget{mv\string_=\string_Bd`M\string_B1}{}
\markboth{\textcolor{darkblue}{\textbf{\ipa{mv̩˩ɖɯ˩}}}}{}
\textcolor{teal}{\mytextsc{noun}} \hspace{4pt} Tone: L.
\textcolor{Sepia}{\selectlanguage{english}Eldest daughter.} \zh{大女儿。}  ¶ \textcolor{darkblue}{\textbf{\ipa{zo˧ɖɯ˧-mv̩˥ɖɯ˩}}} \textcolor{Sepia}{\selectlanguage{english}eldest son and eldest daughter (i.e. eldest male and female siblings)} \zh{大儿子与大女儿}  

\lhead{\firstmark}
\rhead{\botmark}

\subsection{\hspace{-0.5cm} {\Large \textcolor{darkblue}{\textbf{\ipa{mv̩˧dze˧}}}}\hspace{0.5cm}[\kern2pt{\textcolor{darkblue}{\textbf{\ipa{mv̩˩dze˩˥}}}}\kern2pt]} \hypertarget{mv\string_=\string_Mdze\string_M1}{}
\markboth{\textcolor{darkblue}{\textbf{\ipa{mv̩˧dze˧}}}}{}
\textcolor{teal}{\mytextsc{noun}} \hspace{4pt} Tone: M.
\textcolor{Sepia}{\selectlanguage{english}Barley, \textit{Hordeum vulgare L}.} \zh{大麦。}  \zh{量词}: \textcolor{darkblue}{\textbf{\ipa{kɤ˧˥}}}  \mytextsc{clf}: \textcolor{darkblue}{\textbf{\ipa{kɤ˧˥}}} 
\lhead{\firstmark}
\rhead{\botmark}

\subsection{\hspace{-0.5cm} {\Large \textcolor{darkblue}{\textbf{\ipa{mv̩˧dze˧-tɕʰi\#˥}}}}\hspace{0.5cm}[\kern2pt{\textcolor{darkblue}{\textbf{\ipa{xxxx non-correspondance entre le nombre de morphèmes et le nombre de tons de morphèmes}}}}\kern2pt]} \hypertarget{mv\string_=\string_Mdze\string_M-ts£\string_hi\#\string_T1}{}
\markboth{\textcolor{darkblue}{\textbf{\ipa{mv̩˧dze˧-tɕʰi\#˥}}}}{}
\textcolor{teal}{\mytextsc{noun}} \hspace{4pt} Tone: \#H.
\textcolor{Sepia}{\selectlanguage{english}Highland barley beard.} \zh{青稞芒。} 
\lhead{\firstmark}
\rhead{\botmark}

\subsection{\hspace{-0.5cm} {\Large \textcolor{darkblue}{\textbf{\ipa{mv̩˩dzɤ˧}}}}\hspace{0.5cm}[\kern2pt{\textcolor{darkblue}{\textbf{\ipa{mv̩˧dzɤ˧}}}}\kern2pt]} \hypertarget{mv\string_=\string_Bdz7\string_M1}{}
\markboth{\textcolor{darkblue}{\textbf{\ipa{mv̩˩dzɤ˧}}}}{}
\textcolor{teal}{\mytextsc{noun}} \hspace{4pt} Tone: LM.
\textcolor{Sepia}{\selectlanguage{english}Bottom part (symbolically: “the tail”).} \zh{下面部分。}  ¶ \textcolor{darkblue}{\textbf{\ipa{mv̩˩dzɤ˧ dzi˧˥}}} \textcolor{Sepia}{\selectlanguage{english}to be seated in the bottom part (of the room)} \zh{坐在(房间的)下面部分}  
 ¶ \textcolor{darkblue}{\textbf{\ipa{no˧ | mv̩˩dzɤ˧ dzi˧˥!}}} \textcolor{Sepia}{\selectlanguage{english}Go and get seated in the bottom part (of the room)!} \zh{你到下面去坐!}  

\lhead{\firstmark}
\rhead{\botmark}

\subsection{\hspace{-0.5cm} {\Large \textcolor{darkblue}{\textbf{\ipa{mv̩˧gɤ˥gɤ˩}}}}\hspace{0.5cm}[\kern2pt{\textcolor{darkblue}{\textbf{\ipa{mv̩˩gɤ˧gɤ˧}}}}\kern2pt]} \hypertarget{mv\string_=\string_Mg7\string_Tg7\string_B1}{}
\markboth{\textcolor{darkblue}{\textbf{\ipa{mv̩˧gɤ˥gɤ˩}}}}{}
\textcolor{teal}{\mytextsc{noun}} \hspace{4pt} Tone: .
\textcolor{Sepia}{\selectlanguage{english}Descendants.} \zh{下一代、后裔、后人。} 
\lhead{\firstmark}
\rhead{\botmark}

\subsection{\hspace{-0.5cm} {\Large \textcolor{darkblue}{\textbf{\ipa{mv̩˧-gɤ˧lɑ˥}}}}\hspace{0.5cm}[\kern2pt{\textcolor{darkblue}{\textbf{\ipa{xxxx non-correspondance entre le nombre de morphèmes et le nombre de tons de morphèmes}}}}\kern2pt]} \hypertarget{mv\string_=\string_M-g7\string_MlA\string_T1}{}
\markboth{\textcolor{darkblue}{\textbf{\ipa{mv̩˧-gɤ˧lɑ˥}}}}{}
\textcolor{teal}{\mytextsc{noun}} \hspace{4pt} Tone: H\#.
\textcolor{Sepia}{\selectlanguage{english}Sky spirit.} \zh{天宫菩萨。}  \zh{量词}: \textcolor{darkblue}{\textbf{\ipa{v̩˧}}}  \mytextsc{clf}: \textcolor{darkblue}{\textbf{\ipa{v̩˧}}} 
\lhead{\firstmark}
\rhead{\botmark}

\subsection{\hspace{-0.5cm} {\Large \textcolor{darkblue}{\textbf{\ipa{mv̩˧-gv̩˧dv̩˧}}}}\hspace{0.5cm}[\kern2pt{\textcolor{darkblue}{\textbf{\ipa{xxxx non-correspondance entre le nombre de morphèmes et le nombre de tons de morphèmes}}}}\kern2pt]} \hypertarget{mv\string_=\string_M-gv\string_=\string_Mdv\string_=\string_M1}{}
\markboth{\textcolor{darkblue}{\textbf{\ipa{mv̩˧-gv̩˧dv̩˧}}}}{}
\textcolor{teal}{\mytextsc{noun}} \hspace{4pt} Tone: M.
\textcolor{Sepia}{\selectlanguage{english}Instep, top part of the foot.} \zh{脚背。}  \zh{量词}: \textcolor{darkblue}{\textbf{\ipa{ɭɯ˧}}}  \mytextsc{clf}: \textcolor{darkblue}{\textbf{\ipa{ɭɯ˧}}} 
\lhead{\firstmark}
\rhead{\botmark}

\subsection{\hspace{-0.5cm} {\Large \textcolor{darkblue}{\textbf{\ipa{mv̩˧gv̩˧-kʰv̩˩}}}}\hspace{0.5cm}[\kern2pt{\textcolor{darkblue}{\textbf{\ipa{xxxx non-correspondance entre le nombre de morphèmes et le nombre de tons de morphèmes}}}}\kern2pt]} \hypertarget{mv\string_=\string_Mgv\string_=\string_M-k\string_hv\string_=\string_B1}{}
\markboth{\textcolor{darkblue}{\textbf{\ipa{mv̩˧gv̩˧-kʰv̩˩}}}}{}
\textcolor{teal}{\mytextsc{noun}} \hspace{4pt} Tone: L\#.
\textcolor{Sepia}{\selectlanguage{english}Year of the dragon.} \zh{龙年。} 
\lhead{\firstmark}
\rhead{\botmark}

\subsection{\hspace{-0.5cm} {\Large \textcolor{darkblue}{\textbf{\ipa{mv̩˧gv̩\#˥}}}}\hspace{0.5cm}[\kern2pt{\textcolor{darkblue}{\textbf{\ipa{mv̩˧gv̩˧}}}}\kern2pt]} \hypertarget{mv\string_=\string_Mgv\string_=\#\string_T1}{}
\markboth{\textcolor{darkblue}{\textbf{\ipa{mv̩˧gv̩\#˥}}}}{}
\textcolor{teal}{\mytextsc{noun}} \hspace{4pt} Tone: \#H.
\textcolor{Sepia}{\selectlanguage{english}Clap of thunder.} \zh{雷、雷声。}  ¶ \textcolor{darkblue}{\textbf{\ipa{mv̩˧gv̩˧ | gv̩˧-ze˩}}} \textcolor{Sepia}{\selectlanguage{english}there is a clap of thunder} \zh{打雷了}  
 ¶ \textcolor{darkblue}{\textbf{\ipa{mv̩˧gv̩˧ lɑ˩}}} \textcolor{Sepia}{\selectlanguage{english}there is a clap of thunder} \zh{打雷了}  
 \zh{量词}: \textcolor{darkblue}{\textbf{\ipa{ɭɯ˧}}}  \mytextsc{clf}: \textcolor{darkblue}{\textbf{\ipa{ɭɯ˧}}} 
\lhead{\firstmark}
\rhead{\botmark}

\subsection{\hspace{-0.5cm} {\Large \textcolor{darkblue}{\textbf{\ipa{mv̩˩kʰv̩˧˥}}}}\hspace{0.5cm}[\kern2pt{\textcolor{darkblue}{\textbf{\ipa{mv̩˩kʰv̩˧˥}}}}\kern2pt]} \hypertarget{mv\string_=\string_Bk\string_hv\string_=\string_M\string_T1}{}
\markboth{\textcolor{darkblue}{\textbf{\ipa{mv̩˩kʰv̩˧˥}}}}{}
\textcolor{teal}{\mytextsc{noun}} \hspace{4pt} Tone: LM+MH\#.
\textcolor{Sepia}{\selectlanguage{english}Evening (starting when it begins to get dark).} \zh{晚上。} 
\lhead{\firstmark}
\rhead{\botmark}

\subsection{\hspace{-0.5cm} {\Large \textcolor{darkblue}{\textbf{\ipa{mv̩˧kʰv̩˧˥}}}}\hspace{0.5cm}[\kern2pt{\textcolor{darkblue}{\textbf{\ipa{mv̩˧kʰv̩˧˥}}}}\kern2pt]} \hypertarget{mv\string_=\string_Mk\string_hv\string_=\string_M\string_T1}{}
\markboth{\textcolor{darkblue}{\textbf{\ipa{mv̩˧kʰv̩˧˥}}}}{}
\textcolor{teal}{\mytextsc{noun}} \hspace{4pt} Tone: MH\#.
\textcolor{Sepia}{\selectlanguage{english}Smoke.} \zh{烟。}  ¶ \textcolor{darkblue}{\textbf{\ipa{mv̩˧kʰv̩˧ lv̩˥}}} \textcolor{Sepia}{\selectlanguage{english}there is a lot of smoke} \zh{烟很多}  
 \zh{量词}: \textcolor{darkblue}{\textbf{\ipa{æ̃˩}}}  \mytextsc{clf}: \textcolor{darkblue}{\textbf{\ipa{æ̃˩}}} 
\lhead{\firstmark}
\rhead{\botmark}

\subsection{\hspace{-0.5cm} {\Large \textcolor{darkblue}{\textbf{\ipa{mv̩˩ɬi˥}}}}\hspace{0.5cm}[\kern2pt{\textcolor{darkblue}{\textbf{\ipa{mv̩˩ɬi˥}}}}\kern2pt]} \hypertarget{mv\string_=\string_BKi\string_T1}{}
\markboth{\textcolor{darkblue}{\textbf{\ipa{mv̩˩ɬi˥}}}}{}
\textcolor{teal}{\mytextsc{noun}} \hspace{4pt} Tone: LH.
\textcolor{Sepia}{\selectlanguage{english}Second daughter; literally “middle daughter”.} \zh{二女儿。} 
\lhead{\firstmark}
\rhead{\botmark}

\subsection{\hspace{-0.5cm} {\Large \textcolor{darkblue}{\textbf{\ipa{mv̩˧ɭɯ˩}}}}\hspace{0.5cm}[\kern2pt{\textcolor{darkblue}{\textbf{\ipa{mv̩˧ɭɯ˩}}}}\kern2pt]} \hypertarget{mv\string_=\string_Ml\string_RM\string_B1}{}
\markboth{\textcolor{darkblue}{\textbf{\ipa{mv̩˧ɭɯ˩}}}}{}
\textcolor{teal}{\mytextsc{noun}} \hspace{4pt} Tone: L\#.
\textcolor{Sepia}{\selectlanguage{english}Muli county.} \zh{木里。} 
\lhead{\firstmark}
\rhead{\botmark}

\subsection{\hspace{-0.5cm} {\Large \textcolor{darkblue}{\textbf{\ipa{mv̩˧mi˧}}}}\hspace{0.5cm}[\kern2pt{\textcolor{darkblue}{\textbf{\ipa{mv̩˧mi˧}}}}\kern2pt]} \hypertarget{mv\string_=\string_Mmi\string_M1}{}
\markboth{\textcolor{darkblue}{\textbf{\ipa{mv̩˧mi˧}}}}{}
\textcolor{teal}{\mytextsc{noun}} \hspace{4pt} Tone: M.
\textcolor{Sepia}{\selectlanguage{english}Woman.} \zh{女人。}  ¶ \textcolor{darkblue}{\textbf{\ipa{mv̩˧mi˧ so˩tsʰi˩-kʰv̩˩, | qʰo˧mo˥ gi˩ le˩-ʈɤ˩! | ʝi˧=ɻæ˧ qʰv̩˧tsʰi˧-kʰv̩˩, | bɤ˧di˩ lɑ˩ hṽ˩ ɖʐæ˩!}}} \textcolor{Sepia}{\selectlanguage{english}“A woman of thirty must be pulled along like an old cow; a man of sixty stills rides tigers bareback in the land of the Pumi!” This proverb is about ageing in both sexes, with special emphasis on the appeal that they exert on the opposite sex: at thirty, a woman is old; at sixty, a man is still ready for the greatest exploits. The proverb is reported to be used by women, as an ironic (covertly mocking) comment about an ageing man.} \zh{“女人,到三十岁就算是得拉着的老牛。男人,到六十岁还能在普米山上骑老虎!”这个谚语讲男人与女人老化过程,特别描写相互吸引的程度:三十岁女人算是老了,六十岁男人还认为自己有伟大的威力。女人可以用这个谚语隐蔽地嘲弄一个老男人。}  
 \zh{量词}: \textcolor{darkblue}{\textbf{\ipa{v̩˧}}}  \mytextsc{clf}: \textcolor{darkblue}{\textbf{\ipa{v̩˧}}} 
\lhead{\firstmark}
\rhead{\botmark}

\subsection{\hspace{-0.5cm} {\Large \textcolor{darkblue}{\textbf{\ipa{mv̩˧-mv̩˥-di˩}}}}\hspace{0.5cm}[\kern2pt{\textcolor{darkblue}{\textbf{\ipa{xxxx non-correspondance entre le nombre de morphèmes et le nombre de tons de morphèmes}}}}\kern2pt]} \hypertarget{mv\string_=\string_M-mv\string_=\string_T-di\string_B1}{}
\markboth{\textcolor{darkblue}{\textbf{\ipa{mv̩˧-mv̩˥-di˩}}}}{}
\textcolor{teal}{\mytextsc{noun}} \hspace{4pt} Tone: H\#-.
\textcolor{Sepia}{\selectlanguage{english}Bellows.} \zh{风箱。}  \zh{量词}: \textcolor{darkblue}{\textbf{\ipa{ɭɯ˧}}}  \mytextsc{clf}: \textcolor{darkblue}{\textbf{\ipa{ɭɯ˧}}} 
\lhead{\firstmark}
\rhead{\botmark}

\subsection{\hspace{-0.5cm} {\Large \textcolor{darkblue}{\textbf{\ipa{mv̩˧ɲi˧}}}}\hspace{0.5cm}[\kern2pt{\textcolor{darkblue}{\textbf{\ipa{mv̩˧ɲi˧}}}}\kern2pt]} \hypertarget{mv\string_=\string_MJi\string_M1}{}
\markboth{\textcolor{darkblue}{\textbf{\ipa{mv̩˧ɲi˧}}}}{}
\textcolor{teal}{\mytextsc{noun}} \hspace{4pt} Tone: M.
\textcolor{Sepia}{\selectlanguage{english}Toe.} \zh{脚趾。}  \zh{量词}: \textcolor{darkblue}{\textbf{\ipa{ɭɯ˧}}}  \mytextsc{clf}: \textcolor{darkblue}{\textbf{\ipa{ɭɯ˧}}} 
\lhead{\firstmark}
\rhead{\botmark}

\subsection{\hspace{-0.5cm} {\Large \textcolor{darkblue}{\textbf{\ipa{mv̩˩pʰæ˧}}}}\hspace{0.5cm}[\kern2pt{\textcolor{darkblue}{\textbf{\ipa{mv̩˩pʰæ˥}}}}\kern2pt]} \hypertarget{mv\string_=\string_Bp\string_h\{\string_M1}{}
\markboth{\textcolor{darkblue}{\textbf{\ipa{mv̩˩pʰæ˧}}}}{}
\textcolor{teal}{\mytextsc{noun}} \hspace{4pt} Tone: LM.
\textcolor{Sepia}{\selectlanguage{english}Kitchen: the room where pig swill is cooked, where wine is distilled, and where some of the dishes of people are prepared too.} \zh{备料房:煮猪食、煮酒的地方,有时候也在那边准备人的饭。}  \zh{量词}: \textcolor{darkblue}{\textbf{\ipa{ɭɯ˧}}}  \mytextsc{clf}: \textcolor{darkblue}{\textbf{\ipa{ɭɯ˧}}} 
\lhead{\firstmark}
\rhead{\botmark}

\subsection{\hspace{-0.5cm} {\Large \textcolor{darkblue}{\textbf{\ipa{mv̩˧qo˩}}}}\hspace{0.5cm}[\kern2pt{\textcolor{darkblue}{\textbf{\ipa{mv̩˧qo˩}}}}\kern2pt]} \hypertarget{mv\string_=\string_Mqo\string_B1}{}
\markboth{\textcolor{darkblue}{\textbf{\ipa{mv̩˧qo˩}}}}{}
\textcolor{teal}{\mytextsc{noun}} \hspace{4pt} Tone: L\#.
\textcolor{Sepia}{\selectlanguage{english}Papaya.} \zh{木瓜。}  ¶ \textcolor{darkblue}{\textbf{\ipa{mv̩˧qo˩-dʑɯ˩}}} \textcolor{Sepia}{\selectlanguage{english}a liquid prepared from the papaya, which served as an equivalent of vinegar (vinegar was introduced late: it was bought in Chinese areas)} \zh{用木瓜做的一种汁,用法类似于醋。过去,永宁没有醋,醋是从内地(汉族地区)买来的。}  
 \zh{量词}: \textcolor{darkblue}{\textbf{\ipa{ɭɯ˧}}}  \mytextsc{clf}: \textcolor{darkblue}{\textbf{\ipa{ɭɯ˧}}} 
\lhead{\firstmark}
\rhead{\botmark}

\subsection{\hspace{-0.5cm} {\Large \textcolor{darkblue}{\textbf{\ipa{mv̩˧qʰwæ˩}}}}\hspace{0.5cm}[\kern2pt{\textcolor{darkblue}{\textbf{\ipa{mv̩˧qʰwæ˩}}}}\kern2pt]} \hypertarget{mv\string_=\string_Mq\string_hw\{\string_B1}{}
\markboth{\textcolor{darkblue}{\textbf{\ipa{mv̩˧qʰwæ˩}}}}{}
\textcolor{teal}{\mytextsc{noun}} \hspace{4pt} Tone: L\#.
\textcolor{Sepia}{\selectlanguage{english}The name of a village outside the plain of Yongning, close to the Lake.} \zh{木垮:村落名。}  ¶ \textcolor{darkblue}{\textbf{\ipa{ɬi˧ki˧, | ɲi˧se˩, | tɑ˧dzi˩, | mv̩˧qʰwæ˩, | lɑ˧tʰɑ˧-di˧˥}}} \textcolor{Sepia}{\selectlanguage{english}Villages that one passes when moving away from the Yongning plain, towards Lugu lake. These villages do not count as part of Yongning proper. The last, \textcolor{darkblue}{\textbf{\ipa{/lɑ˧tʰɑ˧-di˧˥/}}}, is not a village name like the preceding four: it refers to the entire Na area beyond the fourth village.} \zh{永宁到泸沽湖所经过的村落,依次是:里格、尼赛、大祖、木垮,然后到拉塔地(拉塔地指的是泸沽湖周边的摩梭地区,包括左所、洛水村等)}  

\lhead{\firstmark}
\rhead{\botmark}

\subsection{\hspace{-0.5cm} {\Large \textcolor{darkblue}{\textbf{\ipa{mv̩˧ʁo˥\$}}}}\hspace{0.5cm}[\kern2pt{\textcolor{darkblue}{\textbf{\ipa{mv̩˧ʁo˥}}}}\kern2pt]} \hypertarget{mv\string_=\string_MRo\string_T\$1}{}
\markboth{\textcolor{darkblue}{\textbf{\ipa{mv̩˧ʁo˥\$}}}}{}
\textcolor{teal}{\mytextsc{noun}} \hspace{4pt} Tone: H\$.
\textcolor{Sepia}{\selectlanguage{english}Heavens, sky.} \zh{天空。}  \zh{量词}: \textcolor{darkblue}{\textbf{\ipa{ɭɯ˧}}}  \mytextsc{clf}: \textcolor{darkblue}{\textbf{\ipa{ɭɯ˧}}} 
\lhead{\firstmark}
\rhead{\botmark}

\subsection{\hspace{-0.5cm} {\Large \textcolor{darkblue}{\textbf{\ipa{mv̩˩ʁwɤ˧}}} \textsubscript{1}}\hspace{0.5cm}[\kern2pt{\textcolor{darkblue}{\textbf{\ipa{mv̩˩ʁwɤ˥}}}}\kern2pt]} \hypertarget{mv\string_=\string_BRw7\string_M1}{}
\markboth{\textcolor{darkblue}{\textbf{\ipa{mv̩˩ʁwɤ˧}}} \textsubscript{1}}{}
\textcolor{teal}{\mytextsc{noun}} \hspace{4pt} Tone: LM.
\textcolor{Sepia}{\selectlanguage{english}Lower reaches of a river; downstream.} \zh{下游。} 
\lhead{\firstmark}
\rhead{\botmark}

\subsection{\hspace{-0.5cm} {\Large \textcolor{darkblue}{\textbf{\ipa{mv̩˩ʁwɤ˧}}} \textsubscript{2}}\hspace{0.5cm}[\kern2pt{\textcolor{darkblue}{\textbf{\ipa{mv̩˩ʁwɤ˥}}}}\kern2pt]} \hypertarget{mv\string_=\string_BRw7\string_M2}{}
\markboth{\textcolor{darkblue}{\textbf{\ipa{mv̩˩ʁwɤ˧}}} \textsubscript{2}}{}
\textcolor{teal}{\mytextsc{noun}} \hspace{4pt} Tone: LM.
\textcolor{Sepia}{\selectlanguage{english}The name of a village.} \zh{下村,比如者波下村(永宁的一个村落)。} 
\lhead{\firstmark}
\rhead{\botmark}

\subsection{\hspace{-0.5cm} {\Large \textcolor{darkblue}{\textbf{\ipa{mv̩˩si˧˥}}}}\hspace{0.5cm}[\kern2pt{\textcolor{darkblue}{\textbf{\ipa{mv̩˩si˧˥}}}}\kern2pt]} \hypertarget{mv\string_=\string_Bsi\string_M\string_T1}{}
\markboth{\textcolor{darkblue}{\textbf{\ipa{mv̩˩si˧˥}}}}{}
\textcolor{teal}{\mytextsc{noun}} \hspace{4pt} Tone: LM+MH\#.
\textcolor{Sepia}{\selectlanguage{english}Morning.} \zh{早晨。} 
\lhead{\firstmark}
\rhead{\botmark}

\subsection{\hspace{-0.5cm} {\Large \textcolor{darkblue}{\textbf{\ipa{mv̩˩si˧-njɤ˧˥}}}}\hspace{0.5cm}[\kern2pt{\textcolor{darkblue}{\textbf{\ipa{xxxx non-correspondance entre le nombre de morphèmes et le nombre de tons de morphèmes}}}}\kern2pt]} \hypertarget{mv\string_=\string_Bsi\string_M-nj7\string_M\string_T1}{}
\markboth{\textcolor{darkblue}{\textbf{\ipa{mv̩˩si˧-njɤ˧˥}}}}{}
\textcolor{teal}{\mytextsc{adverb(ial)}} \hspace{4pt} Tone: LM+MH\#.
\textcolor{Sepia}{\selectlanguage{english}Early in the morning.} \zh{一大早。} 
\lhead{\firstmark}
\rhead{\botmark}

\subsection{\hspace{-0.5cm} {\Large \textcolor{darkblue}{\textbf{\ipa{mv̩˩tɑ\#˥}}}}\hspace{0.5cm}[\kern2pt{\textcolor{darkblue}{\textbf{\ipa{mv̩˩tɑ˥}}}}\kern2pt]} \hypertarget{mv\string_=\string_BtA\#\string_T1}{}
\markboth{\textcolor{darkblue}{\textbf{\ipa{mv̩˩tɑ\#˥}}}}{}
\textcolor{teal}{\mytextsc{verb}} \hspace{4pt} Tone: LM+\#H.
\textcolor{Sepia}{\selectlanguage{english}To praise, to commend.} \zh{表扬。}  ¶ \textcolor{darkblue}{\textbf{\ipa{mv̩˩tɑ˧ ʝi˧}}} \textcolor{Sepia}{\selectlanguage{english}to praise} \zh{表扬}  
 ¶ \textcolor{darkblue}{\textbf{\ipa{hĩ˧-ɳɯ˩ | mv̩˩tɑ˥ F | ʝi˧ le˧-hɯ˩-ze˩.}}} \textcolor{Sepia}{\selectlanguage{english}(She/he did some good things, and) people praised him.} \zh{(他做了好事情,于是)人家大大地表扬他了。}  

\lhead{\firstmark}
\rhead{\botmark}

\subsection{\hspace{-0.5cm} {\Large \textcolor{darkblue}{\textbf{\ipa{mv̩˧ʈʰæ\#˥}}}}\hspace{0.5cm}[\kern2pt{\textcolor{darkblue}{\textbf{\ipa{mv̩˧ʈʰæ˧}}}}\kern2pt]} \hypertarget{mv\string_=\string_Mt`\string_h\{\#\string_T1}{}
\markboth{\textcolor{darkblue}{\textbf{\ipa{mv̩˧ʈʰæ\#˥}}}}{}
\textcolor{teal}{\mytextsc{adverb(ial)}} \hspace{4pt} Tone: \#H.
\textcolor{Sepia}{\selectlanguage{english}Under.} \zh{下面。}  ¶ \textcolor{darkblue}{\textbf{\ipa{ʈʂʰɯ˧ | mv̩˧ʈʰæ˧-lɑ˩ li˩! | gɤ˧bi˧ mɤ˧-li˩!}}} \textcolor{Sepia}{\selectlanguage{english}He only looks down, he never glances up! (About someone who constantly sits at his desk, and complains about headaches and a sore neck: the speaker points out that it may be due to a bad posture while at work.)} \zh{他老低头是往下看,不往上看!(情景:有人经常脖子疼、头疼,阿妈提出,这应该跟工作姿势不对有关:那个人一直坐在办公桌前,低着头)}  

\lhead{\firstmark}
\rhead{\botmark}

\subsection{\hspace{-0.5cm} {\Large \textcolor{darkblue}{\textbf{\ipa{mv̩˩tɕi˥}}}}\hspace{0.5cm}[\kern2pt{\textcolor{darkblue}{\textbf{\ipa{mv̩˩tɕi˥}}}}\kern2pt]} \hypertarget{mv\string_=\string_Bts£i\string_T1}{}
\markboth{\textcolor{darkblue}{\textbf{\ipa{mv̩˩tɕi˥}}}}{}
\textcolor{teal}{\mytextsc{noun}} \hspace{4pt} Tone: LH.
\textcolor{Sepia}{\selectlanguage{english}Youngest daughter.} \zh{最小的女儿。} 
\lhead{\firstmark}
\rhead{\botmark}

\subsection{\hspace{-0.5cm} {\Large \textcolor{darkblue}{\textbf{\ipa{mv̩˩tɕo˧}}}}\hspace{0.5cm}[\kern2pt{\textcolor{darkblue}{\textbf{\ipa{mv̩˩tɕo˥}}}}\kern2pt]} \hypertarget{mv\string_=\string_Bts£o\string_M1}{}
\markboth{\textcolor{darkblue}{\textbf{\ipa{mv̩˩tɕo˧}}}}{}
\textcolor{teal}{\mytextsc{adverb(ial)}} \hspace{4pt} Tone: LM.
\textcolor{Sepia}{\selectlanguage{english}Downward.} \zh{往下。}  ¶ \textcolor{darkblue}{\textbf{\ipa{mv̩˩tɕo˧ kwɤ˩}}} \textcolor{Sepia}{\selectlanguage{english}to throw down} \zh{往下扔}  
 ¶ \textcolor{darkblue}{\textbf{\ipa{mv̩˩tɕo˧ se˧!}}} \textcolor{Sepia}{\selectlanguage{english}Get down! Go down! (Command to the dog if it climbs onto the floorboard of the house, contrary to the rule)} \zh{下去!(命令狗从主屋的地板下去:狗不准来上面)}  

\lhead{\firstmark}
\rhead{\botmark}

\subsection{\hspace{-0.5cm} {\Large \textcolor{darkblue}{\textbf{\ipa{mv̩˩tʰi˩}}}}\hspace{0.5cm}[\kern2pt{\textcolor{darkblue}{\textbf{\ipa{mv̩˩tʰi˩˥}}}}\kern2pt]} \hypertarget{mv\string_=\string_Bt\string_hi\string_B1}{}
\markboth{\textcolor{darkblue}{\textbf{\ipa{mv̩˩tʰi˩}}}}{}
\textcolor{teal}{\mytextsc{adjective}} \hspace{4pt} Tone: L.
\textit{From:} \textbf{mv̩˩˥ and tʰi˧} \textcolor{Sepia}{\selectlanguage{english}Intelligent.} \zh{聪明。}  ¶ \textcolor{darkblue}{\textbf{\ipa{ʈʂʰɯ˧ | mv̩˩tʰi˩˥ | ʐwæ˩˥!}}} \textcolor{Sepia}{\selectlanguage{english}She is very intelligent!} \zh{她很聪明!}  

\lhead{\firstmark}
\rhead{\botmark}

\subsection{\hspace{-0.5cm} {\Large \textcolor{darkblue}{\textbf{\ipa{mv̩˩ʈʂæ˧˥}}}}\hspace{0.5cm}[\kern2pt{\textcolor{darkblue}{\textbf{\ipa{mv̩˩ʈʂæ˧˥}}}}\kern2pt]} \hypertarget{mv\string_=\string_Bt`s`\{\string_M\string_T1}{}
\markboth{\textcolor{darkblue}{\textbf{\ipa{mv̩˩ʈʂæ˧˥}}}}{}
\textcolor{teal}{\mytextsc{noun}} \hspace{4pt} Tone: LM+MH\#.
\textcolor{Sepia}{\selectlanguage{english}Lower part (of the body=below the waist).} \zh{下半(身)。} 
\lhead{\firstmark}
\rhead{\botmark}

\subsection{\hspace{-0.5cm} {\Large \textcolor{darkblue}{\textbf{\ipa{mv̩˧ʈʂæ˧˥}}}}\hspace{0.5cm}[\kern2pt{\textcolor{darkblue}{\textbf{\ipa{mv̩˧ʈʂæ˧˥}}}}\kern2pt]} \hypertarget{mv\string_=\string_Mt`s`\{\string_M\string_T1}{}
\markboth{\textcolor{darkblue}{\textbf{\ipa{mv̩˧ʈʂæ˧˥}}}}{}
\textcolor{teal}{\mytextsc{verb}} \hspace{4pt} Tone: MH\#.
\textcolor{Sepia}{\selectlanguage{english}To call, to give the name... to, to refer to... as...} \zh{叫做、称作、名叫。}  ¶ \textcolor{darkblue}{\textbf{\ipa{(ʈʂʰɯ˧ | ) ə˧tso˧ mv̩˧ʈʂæ˧˥?}}} \textcolor{Sepia}{\selectlanguage{english}What's her/his name? / What is (it/he/she) called?} \zh{他叫什么名字?}  
 ¶ \textcolor{darkblue}{\textbf{\ipa{njɤ˧ | ... mv̩˧ʈʂæ˧˥}}} \textcolor{Sepia}{\selectlanguage{english}My name is...} \zh{我名字叫……}  

\lhead{\firstmark}
\rhead{\botmark}

\subsection{\hspace{-0.5cm} {\Large \textcolor{darkblue}{\textbf{\ipa{mv̩˧tsʰi\#˥}}}}\hspace{0.5cm}[\kern2pt{\textcolor{darkblue}{\textbf{\ipa{mv̩˧tsʰi˧}}}}\kern2pt]} \hypertarget{mv\string_=\string_Mts\string_hi\#\string_T1}{}
\markboth{\textcolor{darkblue}{\textbf{\ipa{mv̩˧tsʰi\#˥}}}}{}
\textcolor{teal}{\mytextsc{noun}} \hspace{4pt} Tone: \#H.
\textcolor{Sepia}{\selectlanguage{english}Dry season (winter and spring: from the 9th lunar month to the 2nd lunar month).} \zh{旱季(冬天与春天:农历九月到二月)。}  ¶ \textcolor{darkblue}{\textbf{\ipa{mv̩˧tsʰi˧-qo˩}}} \textcolor{Sepia}{\selectlanguage{english}during the dry season} \zh{旱季的时候}  

\lhead{\firstmark}
\rhead{\botmark}

\subsection{\hspace{-0.5cm} {\Large \textcolor{darkblue}{\textbf{\ipa{mv̩˩tsʰo˩}}}}\hspace{0.5cm}[\kern2pt{\textcolor{darkblue}{\textbf{\ipa{mv̩˩tsʰo˩˥}}}}\kern2pt]} \hypertarget{mv\string_=\string_Bts\string_ho\string_B1}{}
\markboth{\textcolor{darkblue}{\textbf{\ipa{mv̩˩tsʰo˩}}}}{}
\textcolor{teal}{\mytextsc{noun}} \hspace{4pt} Tone: L.
\textcolor{Sepia}{\selectlanguage{english}Firewood full of resin, used to start a fire.} \zh{含很多树脂的木头,用来引火。}  \zh{量词}: \textcolor{darkblue}{\textbf{\ipa{kɤ˧˥}}}  \mytextsc{clf}: \textcolor{darkblue}{\textbf{\ipa{kɤ˧˥}}} 
\lhead{\firstmark}
\rhead{\botmark}

\subsection{\hspace{-0.5cm} {\Large \textcolor{darkblue}{\textbf{\ipa{mv̩˧ʈʂʰɤ˩}}}}\hspace{0.5cm}[\kern2pt{\textcolor{darkblue}{\textbf{\ipa{mv̩˧ʈʂʰɤ˩}}}}\kern2pt]} \hypertarget{mv\string_=\string_Mt`s`\string_h7\string_B1}{}
\markboth{\textcolor{darkblue}{\textbf{\ipa{mv̩˧ʈʂʰɤ˩}}}}{}
\textcolor{teal}{\mytextsc{noun}} \hspace{4pt} Tone: L\#.
\textcolor{Sepia}{\selectlanguage{english}Chin.} \zh{下巴。}  \zh{量词}: \textcolor{darkblue}{\textbf{\ipa{kʰwɤ˥}}}  \mytextsc{clf}: \textcolor{darkblue}{\textbf{\ipa{kʰwɤ˥}}} 
\lhead{\firstmark}
\rhead{\botmark}

\subsection{\hspace{-0.5cm} {\Large \textcolor{darkblue}{\textbf{\ipa{mv̩˧tsɯ˧˥}}}}\hspace{0.5cm}[\kern2pt{\textcolor{darkblue}{\textbf{\ipa{mv̩˧tsɯ˧˥}}}}\kern2pt]} \hypertarget{mv\string_=\string_MtsM\string_M\string_T1}{}
\markboth{\textcolor{darkblue}{\textbf{\ipa{mv̩˧tsɯ˧˥}}}}{}
\textcolor{teal}{\mytextsc{noun}} \hspace{4pt} Tone: MH\#.
\textcolor{Sepia}{\selectlanguage{english}Beard.} \zh{胡子。}  ¶ \textcolor{darkblue}{\textbf{\ipa{mv̩˧tsɯ˧ ʑi˥}}} \textcolor{Sepia}{\selectlanguage{english}to have a beard} \zh{有胡子}  
 \zh{量词}: \textcolor{darkblue}{\textbf{\ipa{kʰwɤ˥}}}  \mytextsc{clf}: \textcolor{darkblue}{\textbf{\ipa{kʰwɤ˥}}} 
\lhead{\firstmark}
\rhead{\botmark}

\subsection{\hspace{-0.5cm} {\Large \textcolor{darkblue}{\textbf{\ipa{mv̩˧ʈʂv̩˩}}}}\hspace{0.5cm}[\kern2pt{\textcolor{darkblue}{\textbf{\ipa{mv̩˧ʈʂv̩˩}}}}\kern2pt]} \hypertarget{mv\string_=\string_Mt`s`v\string_=\string_B1}{}
\markboth{\textcolor{darkblue}{\textbf{\ipa{mv̩˧ʈʂv̩˩}}}}{}
\textcolor{teal}{\mytextsc{noun}} \hspace{4pt} Tone: L\#.
\textcolor{Sepia}{\selectlanguage{english}Mortar.} \zh{臼。}  \zh{量词}: \textcolor{darkblue}{\textbf{\ipa{nɑ˧}}}  \mytextsc{clf}: \textcolor{darkblue}{\textbf{\ipa{nɑ˧}}} 
\lhead{\firstmark}
\rhead{\botmark}

\subsection{\hspace{-0.5cm} {\Large \textcolor{darkblue}{\textbf{\ipa{mv̩˧ʈʂv̩˥}}} \textsubscript{1}}\hspace{0.5cm}[\kern2pt{\textcolor{darkblue}{\textbf{\ipa{mv̩˧ʈʂv̩˥}}}}\kern2pt]} \hypertarget{mv\string_=\string_Mt`s`v\string_=\string_T1}{}
\markboth{\textcolor{darkblue}{\textbf{\ipa{mv̩˧ʈʂv̩˥}}} \textsubscript{1}}{}
\textcolor{teal}{\mytextsc{adjective}} \hspace{4pt} Tone: H\#.
\ding{202} \textcolor{Sepia}{\selectlanguage{english}Creased.} \zh{皱(衣服)。} \ding{203} \textcolor{Sepia}{\selectlanguage{english}Wrinkled.} \zh{(脸)有皱纹。}  ¶ \textcolor{darkblue}{\textbf{\ipa{to˧kɤ˧ | mv̩˧ʈʂv̩˥ ze˩.}}} \textcolor{Sepia}{\selectlanguage{english}(His/her) forehead became wrinkled.} \zh{他的前额有了皱纹。}  
 ¶ \textcolor{darkblue}{\textbf{\ipa{to˧kɤ˧ | le˧-mv̩˧ʈʂv̩˥}}} \textcolor{Sepia}{\selectlanguage{english}(His/her) forehead is wrinkled.} \zh{他的前额有皱纹。}  
 ¶ \textcolor{darkblue}{\textbf{\ipa{[F5] æ˩ʂe˩˥ | le˧-mv̩˧ʈʂv̩˥}}} \textcolor{Sepia}{\selectlanguage{english}The skin is wrinkled (literally “the flesh is wrinkled”)} \zh{皮肤有皱纹(直译:“肉有皱纹”)}  
\ding{204} \textcolor{Sepia}{\selectlanguage{english}Withered.} \zh{谢(花谢了)。}  ¶ \textcolor{darkblue}{\textbf{\ipa{bæ˩bæ˩˥ | le˧-mv̩˧ʈʂv̩˥-ze˩}}} \textcolor{Sepia}{\selectlanguage{english}The flower has withered.} \zh{花谢了。}  
\textit{See:} \hyperlink{}{\textcolor{darkblue}{\textbf{\ipa{mv̩˧ʈʂv̩˥}}} \textsubscript{2}} 
\lhead{\firstmark}
\rhead{\botmark}

\subsection{\hspace{-0.5cm} {\Large \textcolor{darkblue}{\textbf{\ipa{mv̩˧ʈʂv̩˥}}} \textsubscript{2}}\hspace{0.5cm}[\kern2pt{\textcolor{darkblue}{\textbf{\ipa{mv̩˧ʈʂv̩˥}}}}\kern2pt]} \hypertarget{mv\string_=\string_Mt`s`v\string_=\string_T2}{}
\markboth{\textcolor{darkblue}{\textbf{\ipa{mv̩˧ʈʂv̩˥}}} \textsubscript{2}}{}
\textcolor{teal}{\mytextsc{noun}} \hspace{4pt} Tone: H\#.
\textcolor{Sepia}{\selectlanguage{english}Wrinkle.} \zh{皱纹。}  \zh{量词}: \textcolor{darkblue}{\textbf{\ipa{kʰɯ˩}}}  \mytextsc{clf}: \textcolor{darkblue}{\textbf{\ipa{kʰɯ˩}}} \textit{See:} \hyperlink{}{\textcolor{darkblue}{\textbf{\ipa{mv̩˧ʈʂv̩˥}}} \textsubscript{1}} 
\lhead{\firstmark}
\rhead{\botmark}

\subsection{\hspace{-0.5cm} {\Large \textcolor{darkblue}{\textbf{\ipa{mv̩˧ʈʂv̩˩-nv̩˩mi˩}}}}\hspace{0.5cm}[\kern2pt{\textcolor{darkblue}{\textbf{\ipa{xxxx non-correspondance entre le nombre de morphèmes et le nombre de tons de morphèmes}}}}\kern2pt]} \hypertarget{mv\string_=\string_Mt`s`v\string_=\string_B-nv\string_=\string_Bmi\string_B1}{}
\markboth{\textcolor{darkblue}{\textbf{\ipa{mv̩˧ʈʂv̩˩-nv̩˩mi˩}}}}{}
\textcolor{teal}{\mytextsc{noun}} \hspace{4pt} Tone: L\#-.
\textit{From:} \textbf{mv̩˧ʈʂv̩˩ and nv̩˩mi˩} \textcolor{Sepia}{\selectlanguage{english}Small pestle.} \zh{杵。}  \zh{量词}: \textcolor{darkblue}{\textbf{\ipa{nɑ˧}}}  \mytextsc{clf}: \textcolor{darkblue}{\textbf{\ipa{nɑ˧}}} 
\lhead{\firstmark}
\rhead{\botmark}

\subsection{\hspace{-0.5cm} {\Large \textcolor{darkblue}{\textbf{\ipa{mv̩˧ʈʰɯ˧}}}}\hspace{0.5cm}[\kern2pt{\textcolor{darkblue}{\textbf{\ipa{xxxx non-correspondance entre le nombre de morphèmes et le nombre de tons de morphèmes}}}}\kern2pt]} \hypertarget{mv\string_=\string_Mt`\string_hM\string_M1}{}
\markboth{\textcolor{darkblue}{\textbf{\ipa{mv̩˧ʈʰɯ˧}}}}{}
\textcolor{teal}{\mytextsc{noun}} \hspace{4pt} Tone: M.
\textcolor{Sepia}{\selectlanguage{english}Heel.} \zh{脚跟。}  \zh{量词}: \textcolor{darkblue}{\textbf{\ipa{kʰwɤ˥}}}  \mytextsc{clf}: \textcolor{darkblue}{\textbf{\ipa{kʰwɤ˥}}} 
\lhead{\firstmark}
\rhead{\botmark}

\subsection{\hspace{-0.5cm} {\Large \textcolor{darkblue}{\textbf{\ipa{mv̩˩tv̩˩}}}}\hspace{0.5cm}[\kern2pt{\textcolor{darkblue}{\textbf{\ipa{mv̩˧tv̩˧}}}}\kern2pt]} \hypertarget{mv\string_=\string_Btv\string_=\string_B1}{}
\markboth{\textcolor{darkblue}{\textbf{\ipa{mv̩˩tv̩˩}}}}{}
\textcolor{teal}{\mytextsc{noun}} \hspace{4pt} Tone: L.
\textcolor{Sepia}{\selectlanguage{english}Only daughter.} \zh{独生女。}  ¶ \textcolor{darkblue}{\textbf{\ipa{mv̩˩tv̩˩˥ | ɖɯ˧-v̩˧-lɑ˧ dʑo˧˥!}}} \textcolor{Sepia}{\selectlanguage{english}(She) just has an only daughter!} \zh{她只有一个独生女!}  

\lhead{\firstmark}
\rhead{\botmark}

\subsection{\hspace{-0.5cm} {\Large \textcolor{darkblue}{\textbf{\ipa{mv̩˧tʰv̩˧˥}}}}\hspace{0.5cm}[\kern2pt{\textcolor{darkblue}{\textbf{\ipa{mv̩˩tʰv̩˩˥}}}}\kern2pt]} \hypertarget{mv\string_=\string_Mt\string_hv\string_=\string_M\string_T1}{}
\markboth{\textcolor{darkblue}{\textbf{\ipa{mv̩˧tʰv̩˧˥}}}}{}
\textcolor{teal}{\mytextsc{noun}} \hspace{4pt} Tone: MH\#.
\textcolor{Sepia}{\selectlanguage{english}Torch.} \zh{火把。}  \zh{量词}: \textcolor{darkblue}{\textbf{\ipa{qɑ˩}}}  \mytextsc{clf}: \textcolor{darkblue}{\textbf{\ipa{qɑ˩}}} 
\lhead{\firstmark}
\rhead{\botmark}

\subsection{\hspace{-0.5cm} {\Large \textcolor{darkblue}{\textbf{\ipa{mv̩˧ʐe˧˥}}} \textsubscript{1}}\hspace{0.5cm}[\kern2pt{\textcolor{darkblue}{\textbf{\ipa{mv̩˧ʐe˧˥}}}}\kern2pt]} \hypertarget{mv\string_=\string_Mz`e\string_M\string_T1}{}
\markboth{\textcolor{darkblue}{\textbf{\ipa{mv̩˧ʐe˧˥}}} \textsubscript{1}}{}
\textcolor{teal}{\mytextsc{noun}} \hspace{4pt} Tone: MH\#.
\textcolor{Sepia}{\selectlanguage{english}Rainy season (summer and autumn: from the 3rd to the 8th month of the lunar calendar).} \zh{雨季(夏天与秋天:三月份至八月份)。}  ¶ \textcolor{darkblue}{\textbf{\ipa{mv̩˧ʐe˧-qo˥}}} \textcolor{Sepia}{\selectlanguage{english}during the rainy season} \zh{雨季的时候}  

\lhead{\firstmark}
\rhead{\botmark}

\subsection{\hspace{-0.5cm} {\Large \textcolor{darkblue}{\textbf{\ipa{mv̩˧ʐe˧-ʈʂʰæ˧ɣɯ\#˥}}}}\hspace{0.5cm}[\kern2pt{\textcolor{darkblue}{\textbf{\ipa{xxxx non-correspondance entre le nombre de morphèmes et le nombre de tons de morphèmes}}}}\kern2pt]} \hypertarget{mv\string_=\string_Mz`e\string_M-t`s`\string_h\{\string_MGM\#\string_T1}{}
\markboth{\textcolor{darkblue}{\textbf{\ipa{mv̩˧ʐe˧-ʈʂʰæ˧ɣɯ\#˥}}}}{}
\textcolor{teal}{\mytextsc{noun}} \hspace{4pt} Tone: \#H.
\textcolor{Sepia}{\selectlanguage{english}Gunpowder.} \zh{火药。}  \zh{量词}: \textcolor{darkblue}{\textbf{\ipa{po˩}}}  \mytextsc{clf}: \textcolor{darkblue}{\textbf{\ipa{po˩}}} 
\lhead{\firstmark}
\rhead{\botmark}

\subsection{\hspace{-0.5cm} {\Large \textcolor{darkblue}{\textbf{\ipa{mv̩˧ʐe\#˥}}} \textsubscript{2}}\hspace{0.5cm}[\kern2pt{\textcolor{darkblue}{\textbf{\ipa{mv̩˧ʐe˧}}}}\kern2pt]} \hypertarget{mv\string_=\string_Mz`e\#\string_T2}{}
\markboth{\textcolor{darkblue}{\textbf{\ipa{mv̩˧ʐe\#˥}}} \textsubscript{2}}{}
\textcolor{teal}{\mytextsc{noun}} \hspace{4pt} Tone: \#H.
\textcolor{Sepia}{\selectlanguage{english}Gun; firelock.} \zh{枪,明火枪。}  \zh{量词}: \textcolor{darkblue}{\textbf{\ipa{kʰɯ˩}}}  \mytextsc{clf}: \textcolor{darkblue}{\textbf{\ipa{kʰɯ˩}}} 
\lhead{\firstmark}
\rhead{\botmark}

\subsection{\hspace{-0.5cm} {\Large \textcolor{darkblue}{\textbf{\ipa{mv̩˧ʑi˩}}}}\hspace{0.5cm}[\kern2pt{\textcolor{darkblue}{\textbf{\ipa{mv̩˧ʑi˧}}}}\kern2pt]} \hypertarget{mv\string_=\string_Mz£i\string_B1}{}
\markboth{\textcolor{darkblue}{\textbf{\ipa{mv̩˧ʑi˩}}}}{}
\textcolor{teal}{\mytextsc{noun}} \hspace{4pt} Tone: L.
\textcolor{Sepia}{\selectlanguage{english}News, gossip.} \zh{消息、闲话、八卦。}  ¶ \textcolor{darkblue}{\textbf{\ipa{mv̩˧ʑi˩ | ɖɯ˧-kʰwɤ˥}}} \textcolor{Sepia}{\selectlanguage{english}a piece of gossip} \zh{一个八卦}  
 \zh{量词}: \textcolor{darkblue}{\textbf{\ipa{kʰwɤ˥}}}  \mytextsc{clf}: \textcolor{darkblue}{\textbf{\ipa{kʰwɤ˥}}} 
\lhead{\firstmark}
\rhead{\botmark}

\subsection{\hspace{-0.5cm} {\Large \textcolor{darkblue}{\textbf{\ipa{mv̩˩zo˩}}}}\hspace{0.5cm}[\kern2pt{\textcolor{darkblue}{\textbf{\ipa{mv̩˩zo˩˥}}}}\kern2pt]} \hypertarget{mv\string_=\string_Bzo\string_B1}{}
\markboth{\textcolor{darkblue}{\textbf{\ipa{mv̩˩zo˩}}}}{}
\textcolor{teal}{\mytextsc{noun}} \hspace{4pt} Tone: L.
\textcolor{Sepia}{\selectlanguage{english}Young lady.} \zh{姑娘。}  ¶ \textcolor{darkblue}{\textbf{\ipa{mv̩˩zo˩=ɻæ˧}}} \textcolor{Sepia}{\selectlanguage{english}young ladies} \zh{姑娘们}  
 \zh{量词}: \textcolor{darkblue}{\textbf{\ipa{ɭɯ˧}}} \textcolor{darkblue}{\textbf{\ipa{v̩˧}}}  \mytextsc{clf}: \textcolor{darkblue}{\textbf{\ipa{ɭɯ˧}}} \textcolor{darkblue}{\textbf{\ipa{v̩˧}}} 
\lhead{\firstmark}
\rhead{\botmark}

\subsection{\hspace{-0.5cm} {\Large \textcolor{darkblue}{\textbf{\ipa{mv̩˩ʐɤ˩}}}}\hspace{0.5cm}[\kern2pt{\textcolor{darkblue}{\textbf{\ipa{mv̩˩ʐɤ˩˥}}}}\kern2pt]} \hypertarget{mv\string_=\string_Bz`7\string_B1}{}
\markboth{\textcolor{darkblue}{\textbf{\ipa{mv̩˩ʐɤ˩}}}}{}
\textcolor{teal}{\mytextsc{noun}} \hspace{4pt} Tone: L.
\textcolor{Sepia}{\selectlanguage{english}Adopted daughter.} \zh{义女。} 
\lhead{\firstmark}
\rhead{\botmark}

\subsection{\hspace{-0.5cm} {\Large \textcolor{darkblue}{\textbf{\ipa{mv̩˩zo˩-ə˩mi˥}}}}\hspace{0.5cm}[\kern2pt{\textcolor{darkblue}{\textbf{\ipa{xxxx non-correspondance entre le nombre de morphèmes et le nombre de tons de morphèmes}}}}\kern2pt]} \hypertarget{mv\string_=\string_Bzo\string_B-@\string_Bmi\string_T1}{}
\markboth{\textcolor{darkblue}{\textbf{\ipa{mv̩˩zo˩-ə˩mi˥}}}}{}
\textcolor{teal}{\mytextsc{noun}} \hspace{4pt} Tone: L+H\#.
\textcolor{Sepia}{\selectlanguage{english}A young lady and her mother.} \zh{姑娘与母亲。} 
\lhead{\firstmark}
\rhead{\botmark}

\subsection{\hspace{-0.5cm} {\Large \textcolor{darkblue}{\textbf{\ipa{mv̩˩zɯ˩}}} \textsubscript{1}}\hspace{0.5cm}[\kern2pt{\textcolor{darkblue}{\textbf{\ipa{mv̩˩zɯ˥}}}}\kern2pt]} \hypertarget{mv\string_=\string_BzM\string_B1}{}
\markboth{\textcolor{darkblue}{\textbf{\ipa{mv̩˩zɯ˩}}} \textsubscript{1}}{}
\textcolor{teal}{\mytextsc{noun}} \hspace{4pt} Tone: L.
\textcolor{Sepia}{\selectlanguage{english}Brothers.} \zh{兄弟(哥哥们与弟弟们)。}  ¶ \textcolor{darkblue}{\textbf{\ipa{ʈʂʰɯ˧ | nɑ˧dʑi˧-bv̩˧ | mv̩˩zɯ˩-ʝi˥-hĩ˩ ɲi˩!}}} \textcolor{Sepia}{\selectlanguage{english}He is \textcolor{darkblue}{\textbf{\ipa{nɑ˧dʑi˧/'s}}} brother!} \zh{他是\textcolor{darkblue}{\textbf{\ipa{nɑ˧dʑi˧/}}}的兄弟!}  
 \zh{量词}: \textcolor{darkblue}{\textbf{\ipa{v̩˧}}}  \mytextsc{clf}: \textcolor{darkblue}{\textbf{\ipa{v̩˧}}} 
\lhead{\firstmark}
\rhead{\botmark}

\subsection{\hspace{-0.5cm} {\Large \textcolor{darkblue}{\textbf{\ipa{mv̩˩zɯ˩}}} \textsubscript{2}}\hspace{0.5cm}[\kern2pt{\textcolor{darkblue}{\textbf{\ipa{mv̩˩zɯ˩˥}}}}\kern2pt]} \hypertarget{mv\string_=\string_BzM\string_B2}{}
\markboth{\textcolor{darkblue}{\textbf{\ipa{mv̩˩zɯ˩}}} \textsubscript{2}}{}
\textcolor{teal}{\mytextsc{noun}} \hspace{4pt} Tone: L.
\textcolor{Sepia}{\selectlanguage{english}Oats.} \zh{燕麦。}  \zh{量词}: \textcolor{darkblue}{\textbf{\ipa{kɤ˧˥}}}  \mytextsc{clf}: \textcolor{darkblue}{\textbf{\ipa{kɤ˧˥}}} 
\lhead{\firstmark}
\rhead{\botmark}

\subsection{\hspace{-0.5cm} {\Large \textcolor{darkblue}{\textbf{\ipa{mv̩˩zɯ˩-ni˥mi˩}}}}\hspace{0.5cm}[\kern2pt{\textcolor{darkblue}{\textbf{\ipa{xxxx non-correspondance entre le nombre de morphèmes et le nombre de tons de morphèmes}}}}\kern2pt]} \hypertarget{mv\string_=\string_BzM\string_B-ni\string_Tmi\string_B1}{}
\markboth{\textcolor{darkblue}{\textbf{\ipa{mv̩˩zɯ˩-ni˥mi˩}}}}{}
\textcolor{teal}{\mytextsc{noun}} \hspace{4pt} Tone: L+\#H-.
\textcolor{Sepia}{\selectlanguage{english}Brothers and sisters, siblings.} \zh{兄弟姐妹,堂兄弟姐妹。} 
\lhead{\firstmark}
\rhead{\botmark}

\subsection{\hspace{-0.5cm} {\Large \textcolor{darkblue}{\textbf{\ipa{mv̩˧‑}}}}\hspace{0.5cm}[\kern2pt{\textcolor{darkblue}{\textbf{\ipa{xxxx non-correspondance entre le nombre de morphèmes et le nombre de tons de morphèmes}}}}\kern2pt]} \hypertarget{mv\string_=\string_M‑1}{}
\markboth{\textcolor{darkblue}{\textbf{\ipa{mv̩˧‑}}}}{}
\textcolor{teal}{\mytextsc{prefix}} \hspace{4pt} Tone: M.
\textcolor{Sepia}{\selectlanguage{english}Aspect/mood: the event is about to take place: the event is imminent.} \zh{即将、快要、马上会、立即。}  ¶ \textcolor{darkblue}{\textbf{\ipa{ʈʂʰɯ˧ | mv̩˧-dzɯ˧-kwɤ˩-tɕɯ˩!}}} \textcolor{Sepia}{\selectlanguage{english}Come on, eat it up! / Come on, finish your bowl!} \zh{你吃完吧!}  
 ¶ \textcolor{darkblue}{\textbf{\ipa{tʰi˧-mv̩˧-dzɯ˧-kwɤ˩-tɕɯ˩!}}} \textcolor{Sepia}{\selectlanguage{english}Same as previous example, with the \mytextsc{durative}} \zh{同上}  
 ¶ \textcolor{darkblue}{\textbf{\ipa{[M18] ʈʂʰɯ˧ mv̩˧-ʂɯ˧ bi˩-ni˩gv̩˩! njɤ˧ | gv̩˩dʑɯ˩˥ | ʐwæ˩˥! |}}} \textcolor{Sepia}{\selectlanguage{english}(S)he is going to die! I am devastated!} \zh{他要死了!我很伤心!}  
 ¶ \textcolor{darkblue}{\textbf{\ipa{hĩ˧ ʈʂʰɯ˧-v̩˧ tʰv̩˧ mv̩˧-ʂɯ˧-kwɤ˧tɕɯ˥-lɑ˩...}}} \textcolor{Sepia}{\selectlanguage{english}as this person is going to die soon...} \zh{因为这个人快要去世……}  
 ¶ \textcolor{darkblue}{\textbf{\ipa{mv̩˧-dzɯ˧-bi˩-ze˩!}}} \textcolor{Sepia}{\selectlanguage{english}[We are] about to eat! / We are going to eat right now!} \zh{马上要吃了!}  
 ¶ \textcolor{darkblue}{\textbf{\ipa{mv̩˧-hwæ˧}}} \textcolor{Sepia}{\selectlanguage{english}about to buy} \zh{即将买}  
 ¶ \textcolor{darkblue}{\textbf{\ipa{mv̩˧-tɕʰi˧}}} \textcolor{Sepia}{\selectlanguage{english}about to sell} \zh{即将卖}  
 ¶ \textcolor{darkblue}{\textbf{\ipa{mv̩˧-dzɯ˧-kwɤ˧tɕɯ˥-lɑ˩...}}} \textcolor{Sepia}{\selectlanguage{english}since (she/he) is about to eat...} \zh{因为马上要吃……}  
 ¶ \textcolor{darkblue}{\textbf{\ipa{mv̩˧-lɑ˩-kwɤ˩tɕɯ˩-lɑ˩...}}} \textcolor{Sepia}{\selectlanguage{english}since (she/he) is about to strike...} \zh{因为要打……}  

\lhead{\firstmark}
\rhead{\botmark}

\subsection{\hspace{-0.5cm} {\Large \textcolor{darkblue}{\textbf{\ipa{mv̩˧\textasciitilde{}mv̩\#˥}}}}\hspace{0.5cm}[\kern2pt{\textcolor{darkblue}{\textbf{\ipa{mv̩˧mv̩˧}}}}\kern2pt]} \hypertarget{mv\string_=\string_M~mv\string_=\#\string_T1}{}
\markboth{\textcolor{darkblue}{\textbf{\ipa{mv̩˧\textasciitilde{}mv̩\#˥}}}}{}
\textcolor{teal}{\mytextsc{adjective}} \hspace{4pt} Tone: .
\textit{From:} \textbf{mv̩˥} \textcolor{Sepia}{\selectlanguage{english}Clear (speech).} \zh{清楚(话、事情)。}  ¶ \textcolor{darkblue}{\textbf{\ipa{ʐwɤ˧ mv̩˧\textasciitilde{}mv̩˧}}} \textcolor{Sepia}{\selectlanguage{english}to speak clearly; clear speech} \zh{讲清楚}  
 ¶ \textcolor{darkblue}{\textbf{\ipa{le˧-mv̩˧\textasciitilde{}mv̩˧-kʰɯ˩}}} \textcolor{Sepia}{\selectlanguage{english}to clarify, to explain} \zh{弄明白、讲清楚}  

\lhead{\firstmark}
\rhead{\botmark}

\newpage
\section*{\centering- \textcolor{darkblue}{\textbf{\ipa{n}}} -}
\subsection{\hspace{-0.5cm} {\Large \textcolor{darkblue}{\textbf{\ipa{nɑ˥}}}}\hspace{0.5cm}[\kern2pt{\textcolor{darkblue}{\textbf{\ipa{nɑ˥}}}}\kern2pt]} \hypertarget{nA\string_T1}{}
\markboth{\textcolor{darkblue}{\textbf{\ipa{nɑ˥}}}}{}
\textcolor{teal}{\mytextsc{adjective}} \hspace{4pt} Tone: H.
\textcolor{Sepia}{\selectlanguage{english}Important, serious (e.g. a wound).} \zh{严重,重要。}  ¶ \textcolor{darkblue}{\textbf{\ipa{mɤ˧-nɑ˥}}} \textcolor{Sepia}{\selectlanguage{english}not serious} \zh{不严重}  

\lhead{\firstmark}
\rhead{\botmark}

\subsection{\hspace{-0.5cm} {\Large \textcolor{darkblue}{\textbf{\ipa{nɑ˧\textsubscript{a}}}}}\hspace{0.5cm}[\kern2pt{\textcolor{darkblue}{\textbf{\ipa{nɑ˥}}}}\kern2pt]} \hypertarget{nA\string_Ma1}{}
\markboth{\textcolor{darkblue}{\textbf{\ipa{nɑ˧\textsubscript{a}}}}}{}
\textcolor{teal}{\mytextsc{classifier}} \hspace{4pt} Tone: M\textsubscript{a}.
\textcolor{Sepia}{\selectlanguage{english}Classifier for tools.} \zh{量词:工具(一把)。}  ¶ \textcolor{darkblue}{\textbf{\ipa{ɖɯ˧-nɑ˧ dʑo˧}}} \textcolor{Sepia}{\selectlanguage{english}there is one (tool)} \zh{有一把(工具)}  

\lhead{\firstmark}
\rhead{\botmark}

\subsection{\hspace{-0.5cm} {\Large \textcolor{darkblue}{\textbf{\ipa{nɑ˧dʑi\#˥}}}}\hspace{0.5cm}[\kern2pt{\textcolor{darkblue}{\textbf{\ipa{nɑ˩dʑi˥}}}}\kern2pt]} \hypertarget{nA\string_Mdz£i\#\string_T1}{}
\markboth{\textcolor{darkblue}{\textbf{\ipa{nɑ˧dʑi\#˥}}}}{}
\textcolor{teal}{\mytextsc{noun}} \hspace{4pt} Tone: \#H.
\textcolor{Sepia}{\selectlanguage{english}Feminine given name.} \zh{女性名字。} 
\lhead{\firstmark}
\rhead{\botmark}

\subsection{\hspace{-0.5cm} {\Large \textcolor{darkblue}{\textbf{\ipa{nɑ˧mi\#˥}}}}\hspace{0.5cm}[\kern2pt{\textcolor{darkblue}{\textbf{\ipa{nɑ˩mi˥}}}}\kern2pt]} \hypertarget{nA\string_Mmi\#\string_T1}{}
\markboth{\textcolor{darkblue}{\textbf{\ipa{nɑ˧mi\#˥}}}}{}
\textcolor{teal}{\mytextsc{noun}} \hspace{4pt} Tone: \#H.
\textcolor{Sepia}{\selectlanguage{english}Difficulties, complications, hardship, overwork, great fatigue.} \zh{受累、劳累、辛苦、困难、艰难、艰苦。}  ¶ \textcolor{darkblue}{\textbf{\ipa{nɑ˧mi˧ tʰv̩˧!}}} \textcolor{Sepia}{\selectlanguage{english}Hardship has come!} \zh{现在是艰苦的时候! / 现在很贫困!}  
 \zh{量词}: \textcolor{darkblue}{\textbf{\ipa{kʰwɤ˥}}}  \mytextsc{clf}: \textcolor{darkblue}{\textbf{\ipa{kʰwɤ˥}}} 
\lhead{\firstmark}
\rhead{\botmark}

\subsection{\hspace{-0.5cm} {\Large \textcolor{darkblue}{\textbf{\ipa{nɑ˩\textsubscript{b}}}}}\hspace{0.5cm}[\kern2pt{\textcolor{darkblue}{\textbf{\ipa{nɑ˥}}}}\kern2pt]} \hypertarget{nA\string_Bb1}{}
\markboth{\textcolor{darkblue}{\textbf{\ipa{nɑ˩\textsubscript{b}}}}}{}
\textcolor{teal}{\mytextsc{adjective}} \hspace{4pt} Tone: L\textsubscript{b}.
\textcolor{Sepia}{\selectlanguage{english}Black.} \zh{黑,暗(颜色,天色)。}  ¶ \textcolor{darkblue}{\textbf{\ipa{nɑ˩-hĩ˥}}} \textcolor{Sepia}{\selectlanguage{english}\mytextsc{rel}} \zh{黑的}  
 ¶ \textcolor{darkblue}{\textbf{\ipa{mɤ˧-nɑ˩}}} \textcolor{Sepia}{\selectlanguage{english}\mytextsc{neg}} \zh{不黑}  

\lhead{\firstmark}
\rhead{\botmark}

\subsection{\hspace{-0.5cm} {\Large \textcolor{darkblue}{\textbf{\ipa{nɑ˩bɑ˧-ʁɑ˧ɭɯ\#˥}}}}\hspace{0.5cm}[\kern2pt{\textcolor{darkblue}{\textbf{\ipa{xxxx non-correspondance entre le nombre de morphèmes et le nombre de tons de morphèmes}}}}\kern2pt]} \hypertarget{nA\string_BbA\string_M-RA\string_Ml\string_RM\#\string_T1}{}
\markboth{\textcolor{darkblue}{\textbf{\ipa{nɑ˩bɑ˧-ʁɑ˧ɭɯ\#˥}}}}{}
\textcolor{teal}{\mytextsc{noun}} \hspace{4pt} Tone: LM+\#H.
\textcolor{Sepia}{\selectlanguage{english}Name of a mountain.} \zh{一座山的名字。} 
\lhead{\firstmark}
\rhead{\botmark}

\subsection{\hspace{-0.5cm} {\Large \textcolor{darkblue}{\textbf{\ipa{nɑ˩dzi˧}}}}\hspace{0.5cm}[\kern2pt{\textcolor{darkblue}{\textbf{\ipa{nɑ˩dzi˥}}}}\kern2pt]} \hypertarget{nA\string_Bdzi\string_M1}{}
\markboth{\textcolor{darkblue}{\textbf{\ipa{nɑ˩dzi˧}}}}{}
\textcolor{teal}{\mytextsc{adverb(ial)}} \hspace{4pt} Tone: LM.
\textcolor{Sepia}{\selectlanguage{english}Dark (at twilight, dusk).} \zh{暗(黄昏/暮的时候,天变暗)。}  ¶ \textcolor{darkblue}{\textbf{\ipa{nɑ˩dzi˧-ze˩!}}} \textcolor{Sepia}{\selectlanguage{english}It has got dark! Twilight has come!} \zh{天变暗了! / 黄昏到了!}  
 ¶ \textcolor{darkblue}{\textbf{\ipa{nɑ˩dzi˧-ho˩-ze˩!}}} \textcolor{Sepia}{\selectlanguage{english}It's going to get dark!} \zh{(天)要变暗了!}  

\lhead{\firstmark}
\rhead{\botmark}

\subsection{\hspace{-0.5cm} {\Large \textcolor{darkblue}{\textbf{\ipa{nɑ˩hĩ\#˥}}}}\hspace{0.5cm}[\kern2pt{\textcolor{darkblue}{\textbf{\ipa{nɑ˧hĩ˧}}}}\kern2pt]} \hypertarget{nA\string_Bhi\string_~\#\string_T1}{}
\markboth{\textcolor{darkblue}{\textbf{\ipa{nɑ˩hĩ\#˥}}}}{}
\textcolor{teal}{\mytextsc{noun}} \hspace{4pt} Tone: LM+\#H.
\textcolor{Sepia}{\selectlanguage{english}Naxi (ethnic group).} \zh{纳西族。}  ¶ \textcolor{darkblue}{\textbf{\ipa{nɑ˩hĩ˧-mi˧ ɲi˥!}}} \textcolor{Sepia}{\selectlanguage{english}She is a Naxi women! / It's a Naxi woman!} \zh{她是纳西族女人!}  
 ¶ \textcolor{darkblue}{\textbf{\ipa{nɑ˩hĩ˧-bɑ˧lɑ˥}}} \textcolor{Sepia}{\selectlanguage{english}the Naxi costume, Naxi garments} \zh{纳西族服装}  
 ¶ \textcolor{darkblue}{\textbf{\ipa{nɑ˩hĩ˧-ʐwɤ˧ so˥}}} \textcolor{Sepia}{\selectlanguage{english}to study the Naxi language} \zh{学纳西语}  
 ¶ \textcolor{darkblue}{\textbf{\ipa{nɑ˩hĩ˧-tʰæ˧ɻæ˥}}} \textcolor{Sepia}{\selectlanguage{english}Naxi books} \zh{纳西族的书}  
 \zh{量词}: \textcolor{darkblue}{\textbf{\ipa{v̩˧}}}  \mytextsc{clf}: \textcolor{darkblue}{\textbf{\ipa{v̩˧}}} 
\lhead{\firstmark}
\rhead{\botmark}

\subsection{\hspace{-0.5cm} {\Large \textcolor{darkblue}{\textbf{\ipa{nɑ˩kwɤ˧}}}}\hspace{0.5cm}[\kern2pt{\textcolor{darkblue}{\textbf{\ipa{nɑ˧kwɤ˩}}}}\kern2pt]} \hypertarget{nA\string_Bkw7\string_M1}{}
\markboth{\textcolor{darkblue}{\textbf{\ipa{nɑ˩kwɤ˧}}}}{}
\textcolor{teal}{\mytextsc{noun}} \hspace{4pt} Tone: LM.
\textcolor{Sepia}{\selectlanguage{english}Pumpkin; cushaw.} \zh{南瓜。}  \zh{量词}: \textcolor{darkblue}{\textbf{\ipa{ɭɯ˧}}}  \mytextsc{clf}: \textcolor{darkblue}{\textbf{\ipa{ɭɯ˧}}} 
\lhead{\firstmark}
\rhead{\botmark}

\subsection{\hspace{-0.5cm} {\Large \textcolor{darkblue}{\textbf{\ipa{nɑ˩mi\#˥}}}}\hspace{0.5cm}[\kern2pt{\textcolor{darkblue}{\textbf{\ipa{nɑ˩mi˥}}}}\kern2pt]} \hypertarget{nA\string_Bmi\#\string_T1}{}
\markboth{\textcolor{darkblue}{\textbf{\ipa{nɑ˩mi\#˥}}}}{}
\textcolor{teal}{\mytextsc{noun}} \hspace{4pt} Tone: LM+\#H.
\textcolor{Sepia}{\selectlanguage{english}Na woman.} \zh{摩梭女人。} 
\lhead{\firstmark}
\rhead{\botmark}

\subsection{\hspace{-0.5cm} {\Large \textcolor{darkblue}{\textbf{\ipa{nɑ˩mv̩˥-nɑ˩dzi˩dzi˩}}}}\hspace{0.5cm}[\kern2pt{\textcolor{darkblue}{\textbf{\ipa{xxxx non-correspondance entre le nombre de morphèmes et le nombre de tons de morphèmes}}}}\kern2pt]} \hypertarget{nA\string_Bmv\string_=\string_T-nA\string_Bdzi\string_Bdzi\string_B1}{}
\markboth{\textcolor{darkblue}{\textbf{\ipa{nɑ˩mv̩˥-nɑ˩dzi˩dzi˩}}}}{}
\textcolor{teal}{\mytextsc{adjective}} \hspace{4pt} Tone: .
\textcolor{Sepia}{\selectlanguage{english}All dark, quite dark (at twilight, dusk).} \zh{很暗(天变得很暗)。} 
\lhead{\firstmark}
\rhead{\botmark}

\subsection{\hspace{-0.5cm} {\Large \textcolor{darkblue}{\textbf{\ipa{nɑ˩pv̩˧-qʰwɤ˧}}}}\hspace{0.5cm}[\kern2pt{\textcolor{darkblue}{\textbf{\ipa{nɑ˩pv̩˧qʰwɤ˧}}}}\kern2pt]} \hypertarget{nA\string_Bpv\string_=\string_M-q\string_hw7\string_M1}{}
\markboth{\textcolor{darkblue}{\textbf{\ipa{nɑ˩pv̩˧-qʰwɤ˧}}}}{}
\textcolor{teal}{\mytextsc{noun}} \hspace{4pt} Tone: LM-.
\textcolor{Sepia}{\selectlanguage{english}Emperor (borrowed from the Mongolian?).} \zh{皇帝。}  ¶ \textcolor{darkblue}{\textbf{\ipa{ʈʂʰɯ˧ | nɑ˩pʰv̩˧-qʰwɤ˧-ni˩gv̩˩!}}} \textcolor{Sepia}{\selectlanguage{english}He's got an empereror's looks! / He thinks he's the emperor! (Mocking someone who thinks he or she can impose his/her decisions to everyone, who thinks (s)he is a great leader.)} \zh{他摆出做皇帝的样子! / 他以为他是皇帝吧!(嘲笑一个自以为是的人)}  

\lhead{\firstmark}
\rhead{\botmark}

\subsection{\hspace{-0.5cm} {\Large \textcolor{darkblue}{\textbf{\ipa{nɑ˩tsʰi˩}}}}\hspace{0.5cm}[\kern2pt{\textcolor{darkblue}{\textbf{\ipa{nɑ˩tsʰi˩˥}}}}\kern2pt]} \hypertarget{nA\string_Bts\string_hi\string_B1}{}
\markboth{\textcolor{darkblue}{\textbf{\ipa{nɑ˩tsʰi˩}}}}{}
\textcolor{teal}{\mytextsc{noun}} \hspace{4pt} Tone: L.
\textcolor{Sepia}{\selectlanguage{english}Name of a mountain.} \zh{一座山的名字。}  ¶ \textcolor{darkblue}{\textbf{\ipa{kɤ˧mv̩˧˥, | æ˧ʂæ˧, | ŋwɤ˧hɑ̃˩, | ʂwæ˧gv̩\#˥, | nɑ˩tsʰi˩˥ | -tɕʰɤ˧pɤ˧mi\#˥, | qv̩˧ɻ̍˧-ʈʂʰɑ˧nɑ˥ |}}} \textcolor{Sepia}{\selectlanguage{english}The six mountains of Yongning that carry a name and have a definite symbolic value. The other mountains do not have comparable symbolic value, and fewer people use specific names for them.} \zh{永宁地区有固定名字的六座山。其它的山,因为没有重要的象征意义,因此没有取名。}  

\lhead{\firstmark}
\rhead{\botmark}

\subsection{\hspace{-0.5cm} {\Large \textcolor{darkblue}{\textbf{\ipa{nɑ˩zo\#˥}}}}\hspace{0.5cm}[\kern2pt{\textcolor{darkblue}{\textbf{\ipa{nɑ˩zo˥}}}}\kern2pt]} \hypertarget{nA\string_Bzo\#\string_T1}{}
\markboth{\textcolor{darkblue}{\textbf{\ipa{nɑ˩zo\#˥}}}}{}
\textcolor{teal}{\mytextsc{noun}} \hspace{4pt} Tone: LM+\#H.
\textcolor{Sepia}{\selectlanguage{english}Na man.} \zh{摩梭男人。} 
\lhead{\firstmark}
\rhead{\botmark}

\subsection{\hspace{-0.5cm} {\Large \textcolor{darkblue}{\textbf{\ipa{nɑ˩-ʐwɤ˥}}}}\hspace{0.5cm}[\kern2pt{\textcolor{darkblue}{\textbf{\ipa{xxxx non-correspondance entre le nombre de morphèmes et le nombre de tons de morphèmes}}}}\kern2pt]} \hypertarget{nA\string_B-z`w7\string_T1}{}
\markboth{\textcolor{darkblue}{\textbf{\ipa{nɑ˩-ʐwɤ˥}}}}{}
\textcolor{teal}{\mytextsc{noun}} \hspace{4pt} Tone: LH.
\textcolor{Sepia}{\selectlanguage{english}Autonym of the language: the Na language.} \zh{本语言:摩梭话(纳语)。} 
\lhead{\firstmark}
\rhead{\botmark}

\subsection{\hspace{-0.5cm} {\Large \textcolor{darkblue}{\textbf{\ipa{nɑ˧˥}}}}\hspace{0.5cm}[\kern2pt{\textcolor{darkblue}{\textbf{\ipa{nɑ˧˥}}}}\kern2pt]} \hypertarget{nA\string_M\string_T1}{}
\markboth{\textcolor{darkblue}{\textbf{\ipa{nɑ˧˥}}}}{}
\textcolor{teal}{\mytextsc{verb}} \hspace{4pt} Tone: MH.
\textcolor{Sepia}{\selectlanguage{english}To tremble.} \zh{发抖,颤抖。}  ¶ \textcolor{darkblue}{\textbf{\ipa{nɑ˩\textasciitilde{}nɑ˧-ze˥}}} \textcolor{Sepia}{\selectlanguage{english}\mytextsc{red} \mytextsc{pfv}} \zh{发抖了}  
 ¶ \textcolor{darkblue}{\textbf{\ipa{le˧-nɑ˩\textasciitilde{}nɑ˩}}} \textcolor{Sepia}{\selectlanguage{english}\mytextsc{accomp} \mytextsc{red}} \zh{\mytextsc{accomp} \mytextsc{red}}  
 ¶ \textcolor{darkblue}{\textbf{\ipa{lo˩qʰwɤ˥ | nɑ˩\textasciitilde{}nɑ˧˥}}} \textcolor{Sepia}{\selectlanguage{english}the hand trembles} \zh{手抖}  

\lhead{\firstmark}
\rhead{\botmark}

\subsection{\hspace{-0.5cm} {\Large \textcolor{darkblue}{\textbf{\ipa{nɑ˩˧}}}}\hspace{0.5cm}[\kern2pt{\textcolor{darkblue}{\textbf{\ipa{nɑ˩˥}}}}\kern2pt]} \hypertarget{nA\string_B\string_M1}{}
\markboth{\textcolor{darkblue}{\textbf{\ipa{nɑ˩˧}}}}{}
\textcolor{teal}{\mytextsc{noun}} \hspace{4pt} Tone: LM.
\textcolor{Sepia}{\selectlanguage{english}Endonym: Na.} \zh{自称:摩梭族。}  ¶ \textcolor{darkblue}{\textbf{\ipa{nɑ˩-mv̩˧ nɑ˥-di˩ |}}} \textcolor{Sepia}{\selectlanguage{english}Na territory} \zh{摩梭人地区}  
 ¶ \textcolor{darkblue}{\textbf{\ipa{ə˧ʝi˧-ʂɯ˥ʝi˩, | nɑ˩zo˧-tɑ˥mv̩˩-ɳɯ˩ | dʑo˧-ɲi˥-tsɯ˩!}}} \textcolor{Sepia}{\selectlanguage{english}Na traditions used to mention this! / There used to be Na traditions about this! (Context: when reference is made to local customs, to explain what is allowed and what is not.)} \zh{过去,摩梭人的传统(里)有(关于这些问题的说法)嘛!}  
 \zh{量词}: \textcolor{darkblue}{\textbf{\ipa{v̩˧}}}  \mytextsc{clf}: \textcolor{darkblue}{\textbf{\ipa{v̩˧}}} 
\lhead{\firstmark}
\rhead{\botmark}

\subsection{\hspace{-0.5cm} {\Large \textcolor{darkblue}{\textbf{\ipa{ni˥}}}}\hspace{0.5cm}[\kern2pt{\textcolor{darkblue}{\textbf{\ipa{ni˥}}}}\kern2pt]} \hypertarget{ni\string_T1}{}
\markboth{\textcolor{darkblue}{\textbf{\ipa{ni˥}}}}{}
\textcolor{teal}{\mytextsc{noun}} \hspace{4pt} Tone: \#H.
\textcolor{Sepia}{\selectlanguage{english}Amaranth.} \zh{苋米。}  \zh{量词}: \textcolor{darkblue}{\textbf{\ipa{po˧}}}  \mytextsc{clf}: \textcolor{darkblue}{\textbf{\ipa{po˧}}} 
\lhead{\firstmark}
\rhead{\botmark}

\subsection{\hspace{-0.5cm} {\Large \textcolor{darkblue}{\textbf{\ipa{ni˧fv̩˥}}}}\hspace{0.5cm}[\kern2pt{\textcolor{darkblue}{\textbf{\ipa{ni˧fv̩˥}}}}\kern2pt]} \hypertarget{ni\string_Mfv\string_=\string_T1}{}
\markboth{\textcolor{darkblue}{\textbf{\ipa{ni˧fv̩˥}}}}{}
\textcolor{teal}{\mytextsc{noun}} \hspace{4pt} Tone: H\#.
\textcolor{Sepia}{\selectlanguage{english}A very large bag, either made of leather (to carry products over long distances by caravan) or of linen (to wrap up a corpse for temporary inhumation).} \zh{大包:用来包装物品的皮包(马帮用的),或者来装尸体的麻布包(为了在火葬前暂时存放尸体)。}  ¶ \textcolor{darkblue}{\textbf{\ipa{jɤ˧ŋɤ˧-ni˧fv̩˥}}} \textcolor{Sepia}{\selectlanguage{english}Chengdu bag (note: this kind of large, solid bag was often purchased in the area of Chengdu, hence their association with this place name.)} \zh{成都大包。(据说这类的包一般是成都地区生产的。)}  
 \zh{量词}: \textcolor{darkblue}{\textbf{\ipa{ɭɯ˧}}}  \mytextsc{clf}: \textcolor{darkblue}{\textbf{\ipa{ɭɯ˧}}} 
\lhead{\firstmark}
\rhead{\botmark}

\subsection{\hspace{-0.5cm} {\Large \textcolor{darkblue}{\textbf{\ipa{‑ni˧gv̩˧˥}}}}\hspace{0.5cm}[\kern2pt{\textcolor{darkblue}{\textbf{\ipa{ni˧gv̩˧˥}}}}\kern2pt]} \hypertarget{‑ni\string_Mgv\string_=\string_M\string_T1}{}
\markboth{\textcolor{darkblue}{\textbf{\ipa{‑ni˧gv̩˧˥}}}}{}
\textcolor{teal}{\mytextsc{adverb(ial)}} \hspace{4pt} Tone: MH\#.
\textcolor{Sepia}{\selectlanguage{english}To be like, to seem like.} \zh{如、像。}  ¶ \textcolor{darkblue}{\textbf{\ipa{zɯ˧hṽ˩-ni˩gv̩˩}}} \textcolor{Sepia}{\selectlanguage{english}like grass, i.e. vivid green} \zh{像草,等于绿色}  
 ¶ \textcolor{darkblue}{\textbf{\ipa{æ̃˧qæ˩-ni˩gv̩˩}}} \textcolor{Sepia}{\selectlanguage{english}like a parrot (i.e. blue/green-coloured)} \zh{像鹦鹉,等于青色}  
 ¶ \textcolor{darkblue}{\textbf{\ipa{lwæ˩pʰv̩˩-ni˥gv̩˩}}} \textcolor{Sepia}{\selectlanguage{english}like ashes, i.e. grey-coloured} \zh{灰色}  
 ¶ \textcolor{darkblue}{\textbf{\ipa{sɯ˧pv̩˩-ni˩gv̩˩}}} \textcolor{Sepia}{\selectlanguage{english}like a urinary bladder, in the shape of a bladder} \zh{像膀胱}  
 ¶ \textcolor{darkblue}{\textbf{\ipa{(nv̩˩mi˩˥ | ) ɖɯ˧-v̩˧-ni˩gv̩˩}}} \textcolor{Sepia}{\selectlanguage{english}(their heart is) like one, as one} \zh{一条心,想得一致}  
 ¶ \textcolor{darkblue}{\textbf{\ipa{dzi˩bi˩-ni˩gv̩˩˥}}} \textcolor{Sepia}{\selectlanguage{english}to be accustomed to, to get accustomed to} \zh{习惯(一个环境)}  
 ¶ \textcolor{darkblue}{\textbf{\ipa{[élicitation lors de la transcription de Agriculture.70] li˩ ʈʰɯ˩-bi˩-ni˩-gv̩˩˥}}} \textcolor{Sepia}{\selectlanguage{english}to have a habit of drinking tea, to be a tea-drinker} \zh{习惯喝茶、有喝茶的习惯}  
 ¶ \textcolor{darkblue}{\textbf{\ipa{ʈʂʰɯ˧ | ʂɯ˧-bi˧-ni˩gv̩˩!}}} \textcolor{Sepia}{\selectlanguage{english}It looks like it's going to die! / It looks as if it were going to die!} \zh{他好像要死了!}  

\lhead{\firstmark}
\rhead{\botmark}

\subsection{\hspace{-0.5cm} {\Large \textcolor{darkblue}{\textbf{\ipa{ni˧mi\#˥}}}}\hspace{0.5cm}[\kern2pt{\textcolor{darkblue}{\textbf{\ipa{ni˧mi˧}}}}\kern2pt]} \hypertarget{ni\string_Mmi\#\string_T1}{}
\markboth{\textcolor{darkblue}{\textbf{\ipa{ni˧mi\#˥}}}}{}
\textcolor{teal}{\mytextsc{noun}} \hspace{4pt} Tone: \#H.
\textcolor{Sepia}{\selectlanguage{english}Sisters.} \zh{姐妹。}  ¶ \textcolor{darkblue}{\textbf{\ipa{ʈʂʰɯ˧ | ʈæ˧ʂɯ˧-bv̩˧ | ni˧mi˧ ɲi˥.}}} \textcolor{Sepia}{\selectlanguage{english}She is \textcolor{darkblue}{\textbf{\ipa{/ʈæ˧ʂɯ˧/’s}}} sister.} \zh{她是达石的姐姐(或妹妹)}  
 \zh{量词}: \textcolor{darkblue}{\textbf{\ipa{v̩˧}}}  \mytextsc{clf}: \textcolor{darkblue}{\textbf{\ipa{v̩˧}}} 
\lhead{\firstmark}
\rhead{\botmark}

\subsection{\hspace{-0.5cm} {\Large \textcolor{darkblue}{\textbf{\ipa{njæ˥-qv̩˩}}}}\hspace{0.5cm}[\kern2pt{\textcolor{darkblue}{\textbf{\ipa{njæ˥qv̩˩}}}}\kern2pt]} \hypertarget{nj\{\string_T-qv\string_=\string_B1}{}
\markboth{\textcolor{darkblue}{\textbf{\ipa{njæ˥-qv̩˩}}}}{}
\textcolor{teal}{\mytextsc{verb}} \hspace{4pt} Tone: H\#-.
\textcolor{Sepia}{\selectlanguage{english}To look away from.} \zh{看别的方向(蔑视态度)。}  ¶ \textcolor{darkblue}{\textbf{\ipa{hĩ˧ njæ˧qv̩˥}}} \textcolor{Sepia}{\selectlanguage{english}to turn away the head from, to look away from (someone that one despises, hates...)} \zh{看别的方向,不直接看(蔑视态度)}  
 ¶ \textcolor{darkblue}{\textbf{\ipa{mɤ˧-njæ˥qv̩˩}}} \textcolor{Sepia}{\selectlanguage{english}\mytextsc{neg}} \zh{\mytextsc{neg}}  

\lhead{\firstmark}
\rhead{\botmark}

\subsection{\hspace{-0.5cm} {\Large \textcolor{darkblue}{\textbf{\ipa{njæ˧bæ˥}}}}\hspace{0.5cm}[\kern2pt{\textcolor{darkblue}{\textbf{\ipa{njæ˧bæ˥}}}}\kern2pt]} \hypertarget{nj\{\string_Mb\{\string_T1}{}
\markboth{\textcolor{darkblue}{\textbf{\ipa{njæ˧bæ˥}}}}{}
\textcolor{teal}{\mytextsc{noun}} \hspace{4pt} Tone: H\#.
\textcolor{Sepia}{\selectlanguage{english}Tear.} \zh{眼泪。}  \zh{量词}: \textcolor{darkblue}{\textbf{\ipa{ʈʰɤ˥}}}  \mytextsc{clf}: \textcolor{darkblue}{\textbf{\ipa{ʈʰɤ˥}}} 
\lhead{\firstmark}
\rhead{\botmark}

\subsection{\hspace{-0.5cm} {\Large \textcolor{darkblue}{\textbf{\ipa{njæ˧tsɯ˩}}}}\hspace{0.5cm}[\kern2pt{\textcolor{darkblue}{\textbf{\ipa{njæ˧tsɯ˩}}}}\kern2pt]} \hypertarget{nj\{\string_MtsM\string_B1}{}
\markboth{\textcolor{darkblue}{\textbf{\ipa{njæ˧tsɯ˩}}}}{}
\textcolor{teal}{\mytextsc{noun}} \hspace{4pt} Tone: L\#.
\ding{202} \textcolor{Sepia}{\selectlanguage{english}Eyebrow.} \zh{眉毛。}  ¶ \textcolor{darkblue}{\textbf{\ipa{njæ˧tsɯ˩-ɖæ˩}}} \textcolor{Sepia}{\selectlanguage{english}eyebrow (this formulation avoids ambiguity between 'eyebrow' and 'eyelashes')} \zh{眉毛}  
 ¶ \textcolor{darkblue}{\textbf{\ipa{njæ˧tsɯ˩ | mv̩˩tɕo˧ kʰɯ˧˥}}} \textcolor{Sepia}{\selectlanguage{english}to knit the brows (literally “to lower the brows”)} \zh{皱眉毛}  
 \zh{量词}: \textcolor{darkblue}{\textbf{\ipa{kʰwɤ˥}}} \ding{203} \textcolor{Sepia}{\selectlanguage{english}Eyelashes.} \zh{睫毛、眼睫毛、眼毛。}  ¶ \textcolor{darkblue}{\textbf{\ipa{njæ˧tsɯ˩-ʂæ˩}}}  
 \mytextsc{clf}: \textcolor{darkblue}{\textbf{\ipa{kʰwɤ˥}}} 
\lhead{\firstmark}
\rhead{\botmark}

\subsection{\hspace{-0.5cm} {\Large \textcolor{darkblue}{\textbf{\ipa{njæ˧=zɯ˩}}}}\hspace{0.5cm}[\kern2pt{\textcolor{darkblue}{\textbf{\ipa{njæ˧zɯ˩}}}}\kern2pt]} \hypertarget{nj\{\string_M=zM\string_B1}{}
\markboth{\textcolor{darkblue}{\textbf{\ipa{njæ˧=zɯ˩}}}}{}
\textcolor{teal}{\mytextsc{pronoun/pronominal}} \hspace{4pt} Tone: L\#.
\textcolor{Sepia}{\selectlanguage{english}Dual exclusive first person pronoun: us two, the two of us (the speaker plus another person who is not the addressee).} \zh{我们两个。}  ¶ \textcolor{darkblue}{\textbf{\ipa{ɑ˩ʁo˧(-hĩ˧) | njæ˧zɯ˩ ho˩-dʑo˩!}}} \textcolor{Sepia}{\selectlanguage{english}(We cannot stay any longer because) our family is waiting for us!} \zh{(我们两个不能再呆在这里了,)家里在等我们!}  

\lhead{\firstmark}
\rhead{\botmark}

\subsection{\hspace{-0.5cm} {\Large \textcolor{darkblue}{\textbf{\ipa{njæ˩pʰv̩˧}}}}\hspace{0.5cm}[\kern2pt{\textcolor{darkblue}{\textbf{\ipa{njæ˩pʰv̩˥}}}}\kern2pt]} \hypertarget{nj\{\string_Bp\string_hv\string_=\string_M1}{}
\markboth{\textcolor{darkblue}{\textbf{\ipa{njæ˩pʰv̩˧}}}}{}
\textcolor{teal}{\mytextsc{noun}} \hspace{4pt} Tone: LM.
\textcolor{Sepia}{\selectlanguage{english}White of the eye.} \zh{白眼球。}  \zh{量词}: \textcolor{darkblue}{\textbf{\ipa{ɭɯ˧}}}  \mytextsc{clf}: \textcolor{darkblue}{\textbf{\ipa{ɭɯ˧}}} 
\lhead{\firstmark}
\rhead{\botmark}

\subsection{\hspace{-0.5cm} {\Large \textcolor{darkblue}{\textbf{\ipa{njæ˩qwæ˧˥}}}}\hspace{0.5cm}[\kern2pt{\textcolor{darkblue}{\textbf{\ipa{njæ˩qwæ˧˥}}}}\kern2pt]} \hypertarget{nj\{\string_Bqw\{\string_M\string_T1}{}
\markboth{\textcolor{darkblue}{\textbf{\ipa{njæ˩qwæ˧˥}}}}{}
\textcolor{teal}{\mytextsc{adjective}} \hspace{4pt} Tone: LM+MH\#.
\textcolor{Sepia}{\selectlanguage{english}Blind.} \zh{眼睛瞎了。}  ¶ \textcolor{darkblue}{\textbf{\ipa{ʈʂʰɯ˧ | njæ˩qwæ˧-ze˥}}} \textcolor{Sepia}{\selectlanguage{english}(S)he went blind.} \zh{他眼睛瞎了。}  
 ¶ \textcolor{darkblue}{\textbf{\ipa{ʈʂʰɯ˧ | njæ˩qwæ˧ ɲi˥.}}} \textcolor{Sepia}{\selectlanguage{english}(S)he is blind.} \zh{他是瞎子。}  
 ¶ \textcolor{darkblue}{\textbf{\ipa{njæ˩qwæ˧-mi\#˥}}} \textcolor{Sepia}{\selectlanguage{english}blind woman} \zh{眼睛瞎了的女人}  
 ¶ \textcolor{darkblue}{\textbf{\ipa{njæ˩qwæ˧-zo\#˥}}} \textcolor{Sepia}{\selectlanguage{english}blind man} \zh{眼睛瞎了的男人}  
 ¶ \textcolor{darkblue}{\textbf{\ipa{njæ˩qwæ˧-hĩ\#˥}}} \textcolor{Sepia}{\selectlanguage{english}blind person} \zh{瞎子}  
 \zh{量词}: \textcolor{darkblue}{\textbf{\ipa{v̩˧}}}  \mytextsc{clf}: \textcolor{darkblue}{\textbf{\ipa{v̩˧}}} 
\lhead{\firstmark}
\rhead{\botmark}

\subsection{\hspace{-0.5cm} {\Large \textcolor{darkblue}{\textbf{\ipa{njæ˩qʰæ\#˥}}}}\hspace{0.5cm}[\kern2pt{\textcolor{darkblue}{\textbf{\ipa{njæ˩qʰæ˥}}}}\kern2pt]} \hypertarget{nj\{\string_Bq\string_h\{\#\string_T1}{}
\markboth{\textcolor{darkblue}{\textbf{\ipa{njæ˩qʰæ\#˥}}}}{}
\textcolor{teal}{\mytextsc{noun}} \hspace{4pt} Tone: LM+\#H.
\textcolor{Sepia}{\selectlanguage{english}Eye sand, gum in the eyes, rheum.} \zh{眼屎。}  \zh{量词}: \textcolor{darkblue}{\textbf{\ipa{kʰwɤ˥}}}  \mytextsc{clf}: \textcolor{darkblue}{\textbf{\ipa{kʰwɤ˥}}} 
\lhead{\firstmark}
\rhead{\botmark}

\subsection{\hspace{-0.5cm} {\Large \textcolor{darkblue}{\textbf{\ipa{njɤ˧di˧˥}}}}\hspace{0.5cm}[\kern2pt{\textcolor{darkblue}{\textbf{\ipa{njɤ˧di˧˥}}}}\kern2pt]} \hypertarget{nj7\string_Mdi\string_M\string_T1}{}
\markboth{\textcolor{darkblue}{\textbf{\ipa{njɤ˧di˧˥}}}}{}
\textcolor{teal}{\mytextsc{noun}} \hspace{4pt} Tone: MH\#.
\textcolor{Sepia}{\selectlanguage{english}Glue.} \zh{胶。}  \zh{量词}: \textcolor{darkblue}{\textbf{\ipa{kʰwɤ˥}}}  \mytextsc{clf}: \textcolor{darkblue}{\textbf{\ipa{kʰwɤ˥}}} 
\lhead{\firstmark}
\rhead{\botmark}

\subsection{\hspace{-0.5cm} {\Large \textcolor{darkblue}{\textbf{\ipa{njɤ˧kv̩˩}}}}\hspace{0.5cm}[\kern2pt{\textcolor{darkblue}{\textbf{\ipa{njɤ˧kv̩˩}}}}\kern2pt]} \hypertarget{nj7\string_Mkv\string_=\string_B1}{}
\markboth{\textcolor{darkblue}{\textbf{\ipa{njɤ˧kv̩˩}}}}{}
\textcolor{teal}{\mytextsc{noun}} \hspace{4pt} Tone: L\#.
\textcolor{Sepia}{\selectlanguage{english}Cheekbone.} \zh{颧骨。}  \zh{量词}: \textcolor{darkblue}{\textbf{\ipa{ɭɯ˧}}}  \mytextsc{clf}: \textcolor{darkblue}{\textbf{\ipa{ɭɯ˧}}} \textit{See:} \hyperlink{}{\textcolor{darkblue}{\textbf{\ipa{kv̩˩kv̩˩}}}} 
\lhead{\firstmark}
\rhead{\botmark}

\subsection{\hspace{-0.5cm} {\Large \textcolor{darkblue}{\textbf{\ipa{njɤ˧kv̩˩-njɤ˩tsʰɤ˩}}}}\hspace{0.5cm}[\kern2pt{\textcolor{darkblue}{\textbf{\ipa{njɤ˧kv̩˩njɤ˧tsʰɤ˧}}}}\kern2pt]} \hypertarget{nj7\string_Mkv\string_=\string_B-nj7\string_Bts\string_h7\string_B1}{}
\markboth{\textcolor{darkblue}{\textbf{\ipa{njɤ˧kv̩˩-njɤ˩tsʰɤ˩}}}}{}
\textcolor{teal}{\mytextsc{adjective}} \hspace{4pt} Tone: L\#-.
\textcolor{Sepia}{\selectlanguage{english}Beautiful; with a pretty face (of a woman).} \zh{美丽、面貌美。}  ¶ \textcolor{darkblue}{\textbf{\ipa{ə˧mi˧! | mv̩˩zo˩ ʈʂʰɯ˩-ɭɯ˥ | njɤ˧kv̩˩-njɤ˩tsʰɤ˩! | ɖwæ˧˥ | ə˧v̩˧˥!}}} \textcolor{Sepia}{\selectlanguage{english}Wow! This young lady is really beautiful! Very pretty!} \zh{啊呀,这个少女真美丽!很漂亮!}  
 ¶ \textcolor{darkblue}{\textbf{\ipa{njɤ˧kv̩˩njɤ˩tsʰɤ˩ | ʐwæ˩˥}}} \textcolor{Sepia}{\selectlanguage{english}extremely beautiful} \zh{非常美}  
\textit{See:} \textcolor{darkblue}{\textbf{\ipa{njɤ˧kv̩˩, tsʰɤ˧˥a}}} 
\lhead{\firstmark}
\rhead{\botmark}

\subsection{\hspace{-0.5cm} {\Large \textcolor{darkblue}{\textbf{\ipa{njɤ˧le˧gv̩\#˥}}}}\hspace{0.5cm}[\kern2pt{\textcolor{darkblue}{\textbf{\ipa{njɤ˧le˧gv̩˧}}}}\kern2pt]} \hypertarget{nj7\string_Mle\string_Mgv\string_=\#\string_T1}{}
\markboth{\textcolor{darkblue}{\textbf{\ipa{njɤ˧le˧gv̩\#˥}}}}{}
\textcolor{teal}{\mytextsc{noun}} \hspace{4pt} Tone: \#H.
\textcolor{Sepia}{\selectlanguage{english}Daytime.} \zh{白天、大白天。}  ¶ \textcolor{darkblue}{\textbf{\ipa{ɲi˧mi˧-njɤ˩le˩gv̩˩}}} \textcolor{Sepia}{\selectlanguage{english}daytime} \zh{白天}  

\lhead{\firstmark}
\rhead{\botmark}

\subsection{\hspace{-0.5cm} {\Large \textcolor{darkblue}{\textbf{\ipa{njɤ˧mv̩˥}}}}\hspace{0.5cm}[\kern2pt{\textcolor{darkblue}{\textbf{\ipa{njɤ˧mv̩˥}}}}\kern2pt]} \hypertarget{nj7\string_Mmv\string_=\string_T1}{}
\markboth{\textcolor{darkblue}{\textbf{\ipa{njɤ˧mv̩˥}}}}{}
\textcolor{teal}{\mytextsc{noun}} \hspace{4pt} Tone: H\#.
\textcolor{Sepia}{\selectlanguage{english}A plant used as fodder for the pigs, \textit{Chenopodium album}.} \zh{灰条菜、灰灰菜:喂猪的牧草。} Local Chinese dialect:\zh{灰凋。} \zh{量词}: \textcolor{darkblue}{\textbf{\ipa{qɑ˩}}}  \mytextsc{clf}: \textcolor{darkblue}{\textbf{\ipa{qɑ˩}}} 
\lhead{\firstmark}
\rhead{\botmark}

\subsection{\hspace{-0.5cm} {\Large \textcolor{darkblue}{\textbf{\ipa{njɤ˧mv̩˥-mi˩}}}}\hspace{0.5cm}[\kern2pt{\textcolor{darkblue}{\textbf{\ipa{njɤ˧mv̩˥mi˩}}}}\kern2pt]} \hypertarget{nj7\string_Mmv\string_=\string_T-mi\string_B1}{}
\markboth{\textcolor{darkblue}{\textbf{\ipa{njɤ˧mv̩˥-mi˩}}}}{}
\textcolor{teal}{\mytextsc{noun}} \hspace{4pt} Tone: H\#-L.
\textcolor{Sepia}{\selectlanguage{english}Camel.} \zh{骆驼。}  ¶ \textcolor{darkblue}{\textbf{\ipa{njɤ˧mv̩˥mi˩-zo˩}}} \textcolor{Sepia}{\selectlanguage{english}baby camel} \zh{小骆驼}  
 ¶ \textcolor{darkblue}{\textbf{\ipa{njɤ˧mv̩˥mi˩-pʰv̩˩}}} \textcolor{Sepia}{\selectlanguage{english}male camel} \zh{公骆驼}  
 \zh{量词}: \textcolor{darkblue}{\textbf{\ipa{mi˩}}}  \mytextsc{clf}: \textcolor{darkblue}{\textbf{\ipa{mi˩}}} 
\lhead{\firstmark}
\rhead{\botmark}

\subsection{\hspace{-0.5cm} {\Large \textcolor{darkblue}{\textbf{\ipa{njɤ˧nɑ˩}}}}\hspace{0.5cm}[\kern2pt{\textcolor{darkblue}{\textbf{\ipa{njɤ˧nɑ˩}}}}\kern2pt]} \hypertarget{nj7\string_MnA\string_B1}{}
\markboth{\textcolor{darkblue}{\textbf{\ipa{njɤ˧nɑ˩}}}}{}
\textcolor{teal}{\mytextsc{noun}} \hspace{4pt} Tone: L\#.
\textcolor{Sepia}{\selectlanguage{english}Eyeball.} \zh{眼珠。}  \zh{量词}: \textcolor{darkblue}{\textbf{\ipa{ɭɯ˧}}}  \mytextsc{clf}: \textcolor{darkblue}{\textbf{\ipa{ɭɯ˧}}} 
\lhead{\firstmark}
\rhead{\botmark}

\subsection{\hspace{-0.5cm} {\Large \textcolor{darkblue}{\textbf{\ipa{njɤ˧ʈʂɤ˥}}}}\hspace{0.5cm}[\kern2pt{\textcolor{darkblue}{\textbf{\ipa{njɤ˧ʈʂɤ˥}}}}\kern2pt]} \hypertarget{nj7\string_Mt`s`7\string_T1}{}
\markboth{\textcolor{darkblue}{\textbf{\ipa{njɤ˧ʈʂɤ˥}}}}{}
\textcolor{teal}{\mytextsc{noun}} \hspace{4pt} Tone: H\#.
\textcolor{Sepia}{\selectlanguage{english}Black-jack, beggar-ticks, cobbler's pegs, Spanish needle, \textit{Bidens pilosa L.}, a species of flowering plant in the aster family. The barbed awns of the seeds catch onto fur or clothing, and can injure flesh.} \zh{鬼针草。} 
\lhead{\firstmark}
\rhead{\botmark}

\subsection{\hspace{-0.5cm} {\Large \textcolor{darkblue}{\textbf{\ipa{njɤ˩}}}}\hspace{0.5cm}[\kern2pt{\textcolor{darkblue}{\textbf{\ipa{njɤ˩˥}}}}\kern2pt]} \hypertarget{nj7\string_B1}{}
\markboth{\textcolor{darkblue}{\textbf{\ipa{njɤ˩}}}}{}
\textcolor{teal}{\mytextsc{pronoun/pronominal}} \hspace{4pt} Tone: L.
\textcolor{Sepia}{\selectlanguage{english}1st singular pronoun, \mytextsc{1sg}.} \zh{我。}  ¶ \textcolor{darkblue}{\textbf{\ipa{njɤ˩ ɲi˩˥!}}} \textcolor{Sepia}{\selectlanguage{english}It's me! (Typical answer at the door)} \zh{是我!(情景:一个人敲门,里面的人问是谁,人家回答:“是我!”)}  
 ¶ \textcolor{darkblue}{\textbf{\ipa{njɤ˧ no˧ lɑ˧˥}}} \textcolor{Sepia}{\selectlanguage{english}I strike you} \zh{我打你}  

\lhead{\firstmark}
\rhead{\botmark}

\subsection{\hspace{-0.5cm} {\Large \textcolor{darkblue}{\textbf{\ipa{njɤ˩\textsubscript{b}}}}}\hspace{0.5cm}[\kern2pt{\textcolor{darkblue}{\textbf{\ipa{njɤ˩˥}}}}\kern2pt]} \hypertarget{nj7\string_Bb1}{}
\markboth{\textcolor{darkblue}{\textbf{\ipa{njɤ˩\textsubscript{b}}}}}{}
\textcolor{teal}{\mytextsc{verb}} \hspace{4pt} Tone: L\textsubscript{b}.
\textcolor{Sepia}{\selectlanguage{english}To husk.} \zh{舂米。}  ¶ \textcolor{darkblue}{\textbf{\ipa{le˧-njɤ˩-ze˩}}} \textcolor{Sepia}{\selectlanguage{english}\mytextsc{accomp} \string_ \mytextsc{pfv}} \zh{舂了}  
 ¶ \textcolor{darkblue}{\textbf{\ipa{hɑ˧ njɤ˧˥}}} \textcolor{Sepia}{\selectlanguage{english}to husk rice} \zh{舂米}  
 ¶ \textcolor{darkblue}{\textbf{\ipa{hɑ˧ | le˧-njɤ˩}}} \textcolor{Sepia}{\selectlanguage{english}to husk rice} \zh{舂米}  
 ¶ \textcolor{darkblue}{\textbf{\ipa{hɑ˧ | ɖɯ˧-njɤ˧\textasciitilde{}njɤ˩-ɻ̍˩}}} \textcolor{Sepia}{\selectlanguage{english}rice - \mytextsc{delimitative} \mytextsc{red} \mytextsc{inceptive}} \zh{把米舂一舂}  

\lhead{\firstmark}
\rhead{\botmark}

\subsection{\hspace{-0.5cm} {\Large \textcolor{darkblue}{\textbf{\ipa{njɤ˩bi˥}}}}\hspace{0.5cm}[\kern2pt{\textcolor{darkblue}{\textbf{\ipa{njɤ˩bi˥}}}}\kern2pt]} \hypertarget{nj7\string_Bbi\string_T1}{}
\markboth{\textcolor{darkblue}{\textbf{\ipa{njɤ˩bi˥}}}}{}
\textcolor{teal}{\mytextsc{noun}} \hspace{4pt} Tone: LH.
\textcolor{Sepia}{\selectlanguage{english}Eyelid (top eyelid).} \zh{上眼皮。}  \zh{量词}: \textcolor{darkblue}{\textbf{\ipa{ɭɯ˧}}}  \mytextsc{clf}: \textcolor{darkblue}{\textbf{\ipa{ɭɯ˧}}} 
\lhead{\firstmark}
\rhead{\botmark}

\subsection{\hspace{-0.5cm} {\Large \textcolor{darkblue}{\textbf{\ipa{njɤ˩-gɤ˧lɑ˩}}}}\hspace{0.5cm}[\kern2pt{\textcolor{darkblue}{\textbf{\ipa{njɤ˧gɤ˧lɑ˩}}}}\kern2pt]} \hypertarget{nj7\string_B-g7\string_MlA\string_B1}{}
\markboth{\textcolor{darkblue}{\textbf{\ipa{njɤ˩-gɤ˧lɑ˩}}}}{}
\textcolor{teal}{\mytextsc{noun}} \hspace{4pt} Tone: L-L\#.
\textcolor{Sepia}{\selectlanguage{english}Eyeball.} \zh{眼珠。}  \zh{量词}: \textcolor{darkblue}{\textbf{\ipa{ɭɯ˧}}}  \mytextsc{clf}: \textcolor{darkblue}{\textbf{\ipa{ɭɯ˧}}} 
\lhead{\firstmark}
\rhead{\botmark}

\subsection{\hspace{-0.5cm} {\Large \textcolor{darkblue}{\textbf{\ipa{njɤ˩kʰi\#˥}}}}\hspace{0.5cm}[\kern2pt{\textcolor{darkblue}{\textbf{\ipa{njɤ˩kʰi˥}}}}\kern2pt]} \hypertarget{nj7\string_Bk\string_hi\#\string_T1}{}
\markboth{\textcolor{darkblue}{\textbf{\ipa{njɤ˩kʰi\#˥}}}}{}
\textcolor{teal}{\mytextsc{noun}} \hspace{4pt} Tone: LM+\#H.
\textcolor{Sepia}{\selectlanguage{english}Bottom eyelid.} \zh{下眼皮。}  \zh{量词}: \textcolor{darkblue}{\textbf{\ipa{kʰwɤ˥}}}  \mytextsc{clf}: \textcolor{darkblue}{\textbf{\ipa{kʰwɤ˥}}} 
\lhead{\firstmark}
\rhead{\botmark}

\subsection{\hspace{-0.5cm} {\Large \textcolor{darkblue}{\textbf{\ipa{njɤ˩ɭɯ˧}}}}\hspace{0.5cm}[\kern2pt{\textcolor{darkblue}{\textbf{\ipa{njɤ˩ɭɯ˥}}}}\kern2pt]} \hypertarget{nj7\string_Bl\string_RM\string_M1}{}
\markboth{\textcolor{darkblue}{\textbf{\ipa{njɤ˩ɭɯ˧}}}}{}
\textcolor{teal}{\mytextsc{noun}} \hspace{4pt} Tone: LM.
\textcolor{Sepia}{\selectlanguage{english}Eye.} \zh{眼睛。}  \zh{量词}: \textcolor{darkblue}{\textbf{\ipa{ɭɯ˧}}}  \mytextsc{clf}: \textcolor{darkblue}{\textbf{\ipa{ɭɯ˧}}} 
\lhead{\firstmark}
\rhead{\botmark}

\subsection{\hspace{-0.5cm} {\Large \textcolor{darkblue}{\textbf{\ipa{njɤ˩qʰwɤ˧˥}}}}\hspace{0.5cm}[\kern2pt{\textcolor{darkblue}{\textbf{\ipa{njɤ˩qʰwɤ˧˥}}}}\kern2pt]} \hypertarget{nj7\string_Bq\string_hw7\string_M\string_T1}{}
\markboth{\textcolor{darkblue}{\textbf{\ipa{njɤ˩qʰwɤ˧˥}}}}{}
\textcolor{teal}{\mytextsc{noun}} \hspace{4pt} Tone: LM+MH\#.
\textcolor{Sepia}{\selectlanguage{english}Orbit; eye socket.} \zh{眼眶。}  \zh{量词}: \textcolor{darkblue}{\textbf{\ipa{ɭɯ˧}}}  \mytextsc{clf}: \textcolor{darkblue}{\textbf{\ipa{ɭɯ˧}}} 
\lhead{\firstmark}
\rhead{\botmark}

\subsection{\hspace{-0.5cm} {\Large \textcolor{darkblue}{\textbf{\ipa{njɤ˩-tse˧\textasciitilde{}tse˩}}}}\hspace{0.5cm}[\kern2pt{\textcolor{darkblue}{\textbf{\ipa{njɤ˧tse˧tse˩}}}}\kern2pt]} \hypertarget{nj7\string_B-tse\string_M~tse\string_B1}{}
\markboth{\textcolor{darkblue}{\textbf{\ipa{njɤ˩-tse˧\textasciitilde{}tse˩}}}}{}
\textcolor{teal}{\mytextsc{noun}} \hspace{4pt} Tone: L-L\#.
\textcolor{Sepia}{\selectlanguage{english}Branched horsetail, \textit{Equisetum ramosissimum Desf.} This is a wild herb used in traditional medicine; its stem consists of small segments; when pulled/plucked, the stem breaks at one of these articulations.} \zh{节节草。} Local Chinese dialect:\zh{节节高。} \zh{量词}: \textcolor{darkblue}{\textbf{\ipa{po˧}}}  \mytextsc{clf}: \textcolor{darkblue}{\textbf{\ipa{po˧}}} 
\lhead{\firstmark}
\rhead{\botmark}

\subsection{\hspace{-0.5cm} {\Large \textcolor{darkblue}{\textbf{\ipa{njɤ˩ʈʂv̩˧˥}}}}\hspace{0.5cm}[\kern2pt{\textcolor{darkblue}{\textbf{\ipa{njɤ˩ʈʂv̩˧˥}}}}\kern2pt]} \hypertarget{nj7\string_Bt`s`v\string_=\string_M\string_T1}{}
\markboth{\textcolor{darkblue}{\textbf{\ipa{njɤ˩ʈʂv̩˧˥}}}}{}
\textcolor{teal}{\mytextsc{noun}} \hspace{4pt} Tone: LM+MH\#.
\textcolor{Sepia}{\selectlanguage{english}Loach (a kind of fish).} \zh{泥鳅。}  \zh{量词}: \textcolor{darkblue}{\textbf{\ipa{mi˩}}}  \mytextsc{clf}: \textcolor{darkblue}{\textbf{\ipa{mi˩}}} 
\lhead{\firstmark}
\rhead{\botmark}

\subsection{\hspace{-0.5cm} {\Large \textcolor{darkblue}{\textbf{\ipa{njɤ˧˥}}} \textsubscript{1}}\hspace{0.5cm}[\kern2pt{\textcolor{darkblue}{\textbf{\ipa{njɤ˧˥}}}}\kern2pt]} \hypertarget{nj7\string_M\string_T1}{}
\markboth{\textcolor{darkblue}{\textbf{\ipa{njɤ˧˥}}} \textsubscript{1}}{}
\textcolor{teal}{\mytextsc{verb}} \hspace{4pt} Tone: MH.
\textcolor{Sepia}{\selectlanguage{english}To glue (two objects together).} \zh{贴。}  ¶ \textcolor{darkblue}{\textbf{\ipa{le˧-njɤ˧-ze˥!}}} \textcolor{Sepia}{\selectlanguage{english}\mytextsc{accomp} \string_ \mytextsc{pfv}: It's glued!} \zh{粘在一起了!}  
 ¶ \textcolor{darkblue}{\textbf{\ipa{tso˧\textasciitilde{}tso˧ le˧-ɖʐɤ˧, | le˧-njɤ˧˥!}}} \textcolor{Sepia}{\selectlanguage{english}When something (e.g. a book) is torn, (we) glue it together!} \zh{东西撕破了,粘在一起(就好了)}  
\textit{See:} \hyperlink{}{\textcolor{darkblue}{\textbf{\ipa{njɤ˧˥}}} \textsubscript{2}} 
\lhead{\firstmark}
\rhead{\botmark}

\subsection{\hspace{-0.5cm} {\Large \textcolor{darkblue}{\textbf{\ipa{njɤ˧˥}}} \textsubscript{2}}\hspace{0.5cm}[\kern2pt{\textcolor{darkblue}{\textbf{\ipa{njɤ˧˥}}}}\kern2pt]} \hypertarget{nj7\string_M\string_T2}{}
\markboth{\textcolor{darkblue}{\textbf{\ipa{njɤ˧˥}}} \textsubscript{2}}{}
\textcolor{teal}{\mytextsc{adjective}} \hspace{4pt} Tone: MH.
\textcolor{Sepia}{\selectlanguage{english}Sticky.} \zh{黏(胶,树脂)。} \textit{See:} \hyperlink{}{\textcolor{darkblue}{\textbf{\ipa{njɤ˧˥}}} \textsubscript{1}} 
\lhead{\firstmark}
\rhead{\botmark}

\subsection{\hspace{-0.5cm} {\Large \textcolor{darkblue}{\textbf{\ipa{njɤ˧˥}}} \textsubscript{3}}\hspace{0.5cm}[\kern2pt{\textcolor{darkblue}{\textbf{\ipa{njɤ˧˥}}}}\kern2pt]} \hypertarget{nj7\string_M\string_T3}{}
\markboth{\textcolor{darkblue}{\textbf{\ipa{njɤ˧˥}}} \textsubscript{3}}{}
\textcolor{teal}{\mytextsc{adjective}} \hspace{4pt} Tone: MH.
\textcolor{Sepia}{\selectlanguage{english}Early (to rise early).} \zh{早。} 
\lhead{\firstmark}
\rhead{\botmark}

\subsection{\hspace{-0.5cm} {\Large \textcolor{darkblue}{\textbf{\ipa{njɤ˩˥}}}}\hspace{0.5cm}[\kern2pt{\textcolor{darkblue}{\textbf{\ipa{njɤ˩˥}}}}\kern2pt]} \hypertarget{nj7\string_B\string_T1}{}
\markboth{\textcolor{darkblue}{\textbf{\ipa{njɤ˩˥}}}}{}
\textcolor{teal}{\mytextsc{noun}} \hspace{4pt} Tone: LH.
\textcolor{Sepia}{\selectlanguage{english}Eye (monosyllable).} \zh{眼睛(单音节)。}  \zh{量词}: \textcolor{darkblue}{\textbf{\ipa{ɭɯ˧}}}  \mytextsc{clf}: \textcolor{darkblue}{\textbf{\ipa{ɭɯ˧}}} 
\lhead{\firstmark}
\rhead{\botmark}

\subsection{\hspace{-0.5cm} {\Large \textcolor{darkblue}{\textbf{\ipa{njo˥}}}}\hspace{0.5cm}[\kern2pt{\textcolor{darkblue}{\textbf{\ipa{njo˥}}}}\kern2pt]} \hypertarget{njo\string_T1}{}
\markboth{\textcolor{darkblue}{\textbf{\ipa{njo˥}}}}{}
\textcolor{teal}{\mytextsc{noun}} \hspace{4pt} Tone: \#H.
\textcolor{Sepia}{\selectlanguage{english}Cucurbit flute, hulusi: a free reed wind instrument.} \zh{葫芦丝、葫芦箫。}  ¶ \textcolor{darkblue}{\textbf{\ipa{njo˧ mv̩˥}}} \textcolor{Sepia}{\selectlanguage{english}to play the cucurbit flute} \zh{吹响葫芦丝}  

\lhead{\firstmark}
\rhead{\botmark}

\subsection{\hspace{-0.5cm} {\Large \textcolor{darkblue}{\textbf{\ipa{njo˧}}}}\hspace{0.5cm}[\kern2pt{\textcolor{darkblue}{\textbf{\ipa{njo˥}}}}\kern2pt]} \hypertarget{njo\string_M1}{}
\markboth{\textcolor{darkblue}{\textbf{\ipa{njo˧}}}}{}
\textcolor{teal}{\mytextsc{noun}} \hspace{4pt} Tone: M.
\textcolor{Sepia}{\selectlanguage{english}Ear (of wheat, barley).} \zh{谷穗。}  ¶ \textcolor{darkblue}{\textbf{\ipa{hɑ˧-njo˩}}} \textcolor{Sepia}{\selectlanguage{english}ear of cereals} \zh{谷穗}  
 ¶ \textcolor{darkblue}{\textbf{\ipa{mv̩˧dze˧-njo˧ (+ɲi˩)}}} \textcolor{Sepia}{\selectlanguage{english}ear of barley} \zh{大麦穗}  
 ¶ \textcolor{darkblue}{\textbf{\ipa{tsʰi˧zi˧-hɑ˧njo˥ (+ɲi˩)}}} \textcolor{Sepia}{\selectlanguage{english}ear of highland barley} \zh{青稞穗}  

\lhead{\firstmark}
\rhead{\botmark}

\subsection{\hspace{-0.5cm} {\Large \textcolor{darkblue}{\textbf{\ipa{njo˧bi˧li˥}}}}\hspace{0.5cm}[\kern2pt{\textcolor{darkblue}{\textbf{\ipa{njo˧bi˧li˥}}}}\kern2pt]} \hypertarget{njo\string_Mbi\string_Mli\string_T1}{}
\markboth{\textcolor{darkblue}{\textbf{\ipa{njo˧bi˧li˥}}}}{}
\textcolor{teal}{\mytextsc{noun}} \hspace{4pt} Tone: H\#.
\textcolor{Sepia}{\selectlanguage{english}Lips.} \zh{嘴唇。}  \zh{量词}: \textcolor{darkblue}{\textbf{\ipa{ɭɯ˧}}}  \mytextsc{clf}: \textcolor{darkblue}{\textbf{\ipa{ɭɯ˧}}} 
\lhead{\firstmark}
\rhead{\botmark}

\subsection{\hspace{-0.5cm} {\Large \textcolor{darkblue}{\textbf{\ipa{njo˩bi˥}}}}\hspace{0.5cm}[\kern2pt{\textcolor{darkblue}{\textbf{\ipa{njo˩bi˥}}}}\kern2pt]} \hypertarget{njo\string_Bbi\string_T1}{}
\markboth{\textcolor{darkblue}{\textbf{\ipa{njo˩bi˥}}}}{}
\textcolor{teal}{\mytextsc{noun}} \hspace{4pt} Tone: LH.
\textcolor{Sepia}{\selectlanguage{english}Breast.} \zh{乳房。}  ¶ \textcolor{darkblue}{\textbf{\ipa{ʝi˧-njo˥bi˩}}} \textcolor{Sepia}{\selectlanguage{english}cow's breast} \zh{牛的奶头}  
 ¶ \textcolor{darkblue}{\textbf{\ipa{[F5] njo˩bi˧-ʁo˧qʰwɤ˩}}} \textcolor{Sepia}{\selectlanguage{english}nipple, teat} \zh{乳头}  
 \zh{量词}: \textcolor{darkblue}{\textbf{\ipa{ɭɯ˧}}}  \mytextsc{clf}: \textcolor{darkblue}{\textbf{\ipa{ɭɯ˧}}} 
\lhead{\firstmark}
\rhead{\botmark}

\subsection{\hspace{-0.5cm} {\Large \textcolor{darkblue}{\textbf{\ipa{njo˩kæ˧tɕi˩˥}}}}\hspace{0.5cm}[\kern2pt{\textcolor{darkblue}{\textbf{\ipa{xxxx ton non trouvé, à faire manuellement...}}}}\kern2pt]} \hypertarget{njo\string_Bk\{\string_Mts£i\string_B\string_T1}{}
\markboth{\textcolor{darkblue}{\textbf{\ipa{njo˩kæ˧tɕi˩˥}}}}{}
\textcolor{teal}{\mytextsc{noun}} \hspace{4pt} Tone: LM+LH.
\textcolor{Sepia}{\selectlanguage{english}Cep, \textit{Boletus edulis}.} \zh{牛肝菌(汉语借词)。}  Borrowing: Chinese  \zh{牛肝菌}
\textit{See:} \hyperlink{}{\textcolor{darkblue}{\textbf{\ipa{jɤ˧qʰɑ˧-pɤ˥jɤ˩-mo˩}}}} 
\lhead{\firstmark}
\rhead{\botmark}

\subsection{\hspace{-0.5cm} {\Large \textcolor{darkblue}{\textbf{\ipa{njo˩pɤ˩lv̩˥}}}}\hspace{0.5cm}[\kern2pt{\textcolor{darkblue}{\textbf{\ipa{njo˩pɤ˩lv̩˥}}}}\kern2pt]} \hypertarget{njo\string_Bp7\string_Blv\string_=\string_T1}{}
\markboth{\textcolor{darkblue}{\textbf{\ipa{njo˩pɤ˩lv̩˥}}}}{}
\textcolor{teal}{\mytextsc{noun}} \hspace{4pt} Tone: L+H\#.
\textcolor{Sepia}{\selectlanguage{english}Udder.} \zh{牛的奶头。}  \zh{量词}: \textcolor{darkblue}{\textbf{\ipa{ɭɯ˧}}}  \mytextsc{clf}: \textcolor{darkblue}{\textbf{\ipa{ɭɯ˧}}} 
\lhead{\firstmark}
\rhead{\botmark}

\subsection{\hspace{-0.5cm} {\Large \textcolor{darkblue}{\textbf{\ipa{njo˩˥}}}}\hspace{0.5cm}[\kern2pt{\textcolor{darkblue}{\textbf{\ipa{njo˩˥}}}}\kern2pt]} \hypertarget{njo\string_B\string_T1}{}
\markboth{\textcolor{darkblue}{\textbf{\ipa{njo˩˥}}}}{}
\textcolor{teal}{\mytextsc{noun}} \hspace{4pt} Tone: LH.
\textcolor{Sepia}{\selectlanguage{english}Milk.} \zh{奶汁。}  ¶ \textcolor{darkblue}{\textbf{\ipa{njo˩ ki˧}}} \textcolor{Sepia}{\selectlanguage{english}to breast-feed (literally 'to give milk')} \zh{给(喂)奶}  
 ¶ \textcolor{darkblue}{\textbf{\ipa{njo˩ ʈʰɯ˩˥}}} \textcolor{Sepia}{\selectlanguage{english}to drink milk} \zh{喝奶}  

\lhead{\firstmark}
\rhead{\botmark}

\subsection{\hspace{-0.5cm} {\Large \textcolor{darkblue}{\textbf{\ipa{no˧no˧}}}}\hspace{0.5cm}[\kern2pt{\textcolor{darkblue}{\textbf{\ipa{no˧no˧}}}}\kern2pt]} \hypertarget{no\string_Mno\string_M1}{}
\markboth{\textcolor{darkblue}{\textbf{\ipa{no˧no˧}}}}{}
\textcolor{teal}{\mytextsc{noun}} \hspace{4pt} Tone: M.
\textcolor{Sepia}{\selectlanguage{english}Feminine given name.} \zh{女性名字。} 
\lhead{\firstmark}
\rhead{\botmark}

\subsection{\hspace{-0.5cm} {\Large \textcolor{darkblue}{\textbf{\ipa{no˧=ɻ̍˩}}}}\hspace{0.5cm}[\kern2pt{\textcolor{darkblue}{\textbf{\ipa{no˧ɻ̍˩}}}}\kern2pt]} \hypertarget{no\string_M=r£`̍\string_B1}{}
\markboth{\textcolor{darkblue}{\textbf{\ipa{no˧=ɻ̍˩}}}}{}
\textcolor{teal}{\mytextsc{pronoun/pronominal}} \hspace{4pt} Tone: L\#.
\textcolor{Sepia}{\selectlanguage{english}Second person plural.} \zh{你们。} \textit{See:} \hyperlink{}{\textcolor{darkblue}{\textbf{\ipa{no˧-sɯ˩kv̩˩}}}} 
\lhead{\firstmark}
\rhead{\botmark}

\subsection{\hspace{-0.5cm} {\Large \textcolor{darkblue}{\textbf{\ipa{no˧=zɯ˩}}}}\hspace{0.5cm}[\kern2pt{\textcolor{darkblue}{\textbf{\ipa{no˧zɯ˩}}}}\kern2pt]} \hypertarget{no\string_M=zM\string_B1}{}
\markboth{\textcolor{darkblue}{\textbf{\ipa{no˧=zɯ˩}}}}{}
\textcolor{teal}{\mytextsc{pronoun/pronominal}} \hspace{4pt} Tone: L\#.
\textcolor{Sepia}{\selectlanguage{english}Dual second person pronoun: you two.} \zh{你们俩。} \textit{See:} \hyperlink{}{\textcolor{darkblue}{\textbf{\ipa{no˩zɯ˧˥}}}} 
\lhead{\firstmark}
\rhead{\botmark}

\subsection{\hspace{-0.5cm} {\Large \textcolor{darkblue}{\textbf{\ipa{no˩}}}}\hspace{0.5cm}[\kern2pt{\textcolor{darkblue}{\textbf{\ipa{no˩˥}}}}\kern2pt]} \hypertarget{no\string_B1}{}
\markboth{\textcolor{darkblue}{\textbf{\ipa{no˩}}}}{}
\textcolor{teal}{\mytextsc{pronoun/pronominal}} \hspace{4pt} Tone: L.
\textcolor{Sepia}{\selectlanguage{english}Second person singular pronoun.} \zh{你。}  ¶ \textcolor{darkblue}{\textbf{\ipa{no˩ ɲi˩˥}}} \textcolor{Sepia}{\selectlanguage{english}It's you!} \zh{是你!}  

\lhead{\firstmark}
\rhead{\botmark}

\subsection{\hspace{-0.5cm} {\Large \textcolor{darkblue}{\textbf{\ipa{no˩bv̩˧}}}}\hspace{0.5cm}[\kern2pt{\textcolor{darkblue}{\textbf{\ipa{no˩bv̩˥}}}}\kern2pt]} \hypertarget{no\string_Bbv\string_=\string_M1}{}
\markboth{\textcolor{darkblue}{\textbf{\ipa{no˩bv̩˧}}}}{}
\textcolor{teal}{\mytextsc{noun}} \hspace{4pt} Tone: LM.
\textcolor{Sepia}{\selectlanguage{english}Masculine given name.} \zh{男性名字。} 
\lhead{\firstmark}
\rhead{\botmark}

\subsection{\hspace{-0.5cm} {\Large \textcolor{darkblue}{\textbf{\ipa{no˩qo˥}}}}\hspace{0.5cm}[\kern2pt{\textcolor{darkblue}{\textbf{\ipa{no˩qo˥}}}}\kern2pt]} \hypertarget{no\string_Bqo\string_T1}{}
\markboth{\textcolor{darkblue}{\textbf{\ipa{no˩qo˥}}}}{}
\textcolor{teal}{\mytextsc{adverb(ial)}} \hspace{4pt} Tone: LH.
\textcolor{Sepia}{\selectlanguage{english}Close to, next to.} \zh{……附近。} 
\lhead{\firstmark}
\rhead{\botmark}

\subsection{\hspace{-0.5cm} {\Large \textcolor{darkblue}{\textbf{\ipa{no˩zɯ˧˥}}}}\hspace{0.5cm}[\kern2pt{\textcolor{darkblue}{\textbf{\ipa{no˩zɯ˧˥}}}}\kern2pt]} \hypertarget{no\string_BzM\string_M\string_T1}{}
\markboth{\textcolor{darkblue}{\textbf{\ipa{no˩zɯ˧˥}}}}{}
\textcolor{teal}{\mytextsc{pronoun/pronominal}} \hspace{4pt} Tone: LM+MH\#.
\textcolor{Sepia}{\selectlanguage{english}Dual second person pronoun: you two.} \zh{你们俩。} \textit{See:} \hyperlink{}{\textcolor{darkblue}{\textbf{\ipa{no˧=zɯ˩}}}} 
\lhead{\firstmark}
\rhead{\botmark}

\subsection{\hspace{-0.5cm} {\Large \textcolor{darkblue}{\textbf{\ipa{no˩xx}}}}\hspace{0.5cm}[\kern2pt{\textcolor{darkblue}{\textbf{\ipa{xxxx ton non trouvé, à faire manuellement...}}}}\kern2pt]} \hypertarget{no\string_Bxx1}{}
\markboth{\textcolor{darkblue}{\textbf{\ipa{no˩xx}}}}{}
\textcolor{teal}{\mytextsc{verb}} \hspace{4pt} Tone: Lxx.
\textcolor{Sepia}{\selectlanguage{english}To add, to blend in, to mix.} \zh{搀。}  ¶ \textcolor{darkblue}{\textbf{\ipa{ɳæ˧ | tsɑ˧bɤ˧-qo˧ tʰi˧-no˩}}} \textcolor{Sepia}{\selectlanguage{english}to mix grilled flour with milk, to add milk to grilled flour} \zh{在糌粑里搀奶、糌粑里搀奶}  
 ¶ \textcolor{darkblue}{\textbf{\ipa{le˧-no˩}}} \textcolor{Sepia}{\selectlanguage{english}\mytextsc{accomp} \string_} \zh{\mytextsc{accomp} \string_}  
 ¶ \textcolor{darkblue}{\textbf{\ipa{hɑ˧-qo˩ | tɕæ˧ɻæ˩ tʰi˩-no˩: hɑ˧-qo˩ | tɕæ˧ɻæ˩ tʰi˩-kʰɯ˩}}} \textcolor{Sepia}{\selectlanguage{english}a paraphrase to explain the verb's meaning: 'to blend pickled vegetables into the food/rice, that means: to add picked vegetables to the food.'} \zh{关于这个动词的说明:‘饭里面搀泡菜,就是说:在饭里面放泡菜。’}  
 ¶ \textcolor{darkblue}{\textbf{\ipa{hɑ˧-qo˩ | tɕæ˧ɻæ˩ tʰi˩-no˩: tɕæ˧ɻæ˩-lɑ˩ | hɑ˧ | ɖɯ˧-tɕʰo˩ dzɯ˩}}} \textcolor{Sepia}{\selectlanguage{english}a paraphrase to explain the verb's meaning: 'to blend pickled vegetables into the food/rice, that means: to eat picked vegetables and food/rice together.'} \zh{关于这个动词的说明:‘饭里面搀泡菜,就是说:把泡菜和饭一起吃。’}  

\lhead{\firstmark}
\rhead{\botmark}

\subsection{\hspace{-0.5cm} {\Large \textcolor{darkblue}{\textbf{\ipa{‑no˧˥}}}}\hspace{0.5cm}[\kern2pt{\textcolor{darkblue}{\textbf{\ipa{no˧˥}}}}\kern2pt]} \hypertarget{‑no\string_M\string_T1}{}
\markboth{\textcolor{darkblue}{\textbf{\ipa{‑no˧˥}}}}{}
\textcolor{teal}{\mytextsc{discourse}} \textcolor{teal}{\mytextsc{particle}} \hspace{4pt} Tone: MH.
\textcolor{Sepia}{\selectlanguage{english}Contrastive topic.} \zh{\mytextsc{主题:对于、关于。}}  ¶ \textcolor{darkblue}{\textbf{\ipa{qæ˧do˧ | -no˧˥}}} \textcolor{Sepia}{\selectlanguage{english}As for lumber, ...} \zh{关于木材,……}  
 ¶ \textcolor{darkblue}{\textbf{\ipa{ʑi˧qʰwɤ˧ | -no˧˥}}} \textcolor{Sepia}{\selectlanguage{english}As for the main room, ...} \zh{关于主屋……}  

\lhead{\firstmark}
\rhead{\botmark}

\subsection{\hspace{-0.5cm} {\Large \textcolor{darkblue}{\textbf{\ipa{nv̩˥}}} \textsubscript{1}}\hspace{0.5cm}[\kern2pt{\textcolor{darkblue}{\textbf{\ipa{nv̩˥}}}}\kern2pt]} \hypertarget{nv\string_=\string_T1}{}
\markboth{\textcolor{darkblue}{\textbf{\ipa{nv̩˥}}} \textsubscript{1}}{}
\textcolor{teal}{\mytextsc{verb}} \hspace{4pt} Tone: H.
\textcolor{Sepia}{\selectlanguage{english}To chase after, to pursue.} \zh{追赶。}  ¶ \textcolor{darkblue}{\textbf{\ipa{le˧-nv̩˥}}} \textcolor{Sepia}{\selectlanguage{english}\mytextsc{accomp}} \zh{\mytextsc{accomp}}  
 ¶ \textcolor{darkblue}{\textbf{\ipa{ʈʂɤ˩nv̩˩}}} \textcolor{Sepia}{\selectlanguage{english}to chase after, to pursue} \zh{追赶}  
 ¶ \textcolor{darkblue}{\textbf{\ipa{le˧-ʈʂɤ˩nv̩˩}}} \textcolor{Sepia}{\selectlanguage{english}to chase after, to pursue} \zh{追赶}  
 ¶ \textcolor{darkblue}{\textbf{\ipa{le˧-ʈʂɤ˩nv̩˩ | le˧-hɯ˩}}} \textcolor{Sepia}{\selectlanguage{english}He went to chase after} \zh{追赶去了}  

\lhead{\firstmark}
\rhead{\botmark}

\subsection{\hspace{-0.5cm} {\Large \textcolor{darkblue}{\textbf{\ipa{nv̩˥}}} \textsubscript{2}}\hspace{0.5cm}[\kern2pt{\textcolor{darkblue}{\textbf{\ipa{nv̩˥}}}}\kern2pt]} \hypertarget{nv\string_=\string_T2}{}
\markboth{\textcolor{darkblue}{\textbf{\ipa{nv̩˥}}} \textsubscript{2}}{}
\textcolor{teal}{\mytextsc{verb}} \hspace{4pt} Tone: H.
\textcolor{Sepia}{\selectlanguage{english}To bury.} \zh{埋。} 
\lhead{\firstmark}
\rhead{\botmark}

\subsection{\hspace{-0.5cm} {\Large \textcolor{darkblue}{\textbf{\ipa{nv̩˩dʑɯ˥}}}}\hspace{0.5cm}[\kern2pt{\textcolor{darkblue}{\textbf{\ipa{nv̩˩dʑɯ˩˥}}}}\kern2pt]} \hypertarget{nv\string_=\string_Bdz£M\string_T1}{}
\markboth{\textcolor{darkblue}{\textbf{\ipa{nv̩˩dʑɯ˥}}}}{}
\textcolor{teal}{\mytextsc{noun}} \hspace{4pt} Tone: LH.
\textcolor{Sepia}{\selectlanguage{english}Tofu, bean curd.} \zh{豆腐。}  \zh{量词}: \textcolor{darkblue}{\textbf{\ipa{v̩˥}}}  \mytextsc{clf}: \textcolor{darkblue}{\textbf{\ipa{v̩˥}}} 
\lhead{\firstmark}
\rhead{\botmark}

\subsection{\hspace{-0.5cm} {\Large \textcolor{darkblue}{\textbf{\ipa{nv̩˩ho\#˥}}}}\hspace{0.5cm}[\kern2pt{\textcolor{darkblue}{\textbf{\ipa{nv̩˩ho˥}}}}\kern2pt]} \hypertarget{nv\string_=\string_Bho\#\string_T1}{}
\markboth{\textcolor{darkblue}{\textbf{\ipa{nv̩˩ho\#˥}}}}{}
\textcolor{teal}{\mytextsc{noun}} \hspace{4pt} Tone: LM+\#H.
\textcolor{Sepia}{\selectlanguage{english}Long-boiled soft beancurd.} \zh{豆花。}  \zh{量词}: \textcolor{darkblue}{\textbf{\ipa{v̩˥}}}  \mytextsc{clf}: \textcolor{darkblue}{\textbf{\ipa{v̩˥}}} 
\lhead{\firstmark}
\rhead{\botmark}

\subsection{\hspace{-0.5cm} {\Large \textcolor{darkblue}{\textbf{\ipa{nv̩˧hṽ˩}}}}\hspace{0.5cm}[\kern2pt{\textcolor{darkblue}{\textbf{\ipa{nv̩˩hṽ˥}}}}\kern2pt]} \hypertarget{nv\string_=\string_Mhv\string_~\string_B1}{}
\markboth{\textcolor{darkblue}{\textbf{\ipa{nv̩˧hṽ˩}}}}{}
\textcolor{teal}{\mytextsc{noun}} \hspace{4pt} Tone: L\#.
\textcolor{Sepia}{\selectlanguage{english}Bean; string bean, kidney bean.} \zh{豆子,四季豆,花腰豆。}  \zh{量词}: \textcolor{darkblue}{\textbf{\ipa{v̩˥}}}  \mytextsc{clf}: \textcolor{darkblue}{\textbf{\ipa{v̩˥}}} 
\lhead{\firstmark}
\rhead{\botmark}

\subsection{\hspace{-0.5cm} {\Large \textcolor{darkblue}{\textbf{\ipa{nv̩˧hṽ˩-bi˩bi˩}}}}\hspace{0.5cm}[\kern2pt{\textcolor{darkblue}{\textbf{\ipa{xxxx non-correspondance entre le nombre de morphèmes et le nombre de tons de morphèmes}}}}\kern2pt]} \hypertarget{nv\string_=\string_Mhv\string_~\string_B-bi\string_Bbi\string_B1}{}
\markboth{\textcolor{darkblue}{\textbf{\ipa{nv̩˧hṽ˩-bi˩bi˩}}}}{}
\textcolor{teal}{\mytextsc{noun}} \hspace{4pt} Tone: L\#-.
\textcolor{Sepia}{\selectlanguage{english}Green bean, snap bean, string bean; one consumes the pod with the seed inside.} \zh{四季豆、玉豆、帶莢豌豆、菜豆、刀豆、豆角、敏豆仔、敏豆。}  \zh{量词}: \textcolor{darkblue}{\textbf{\ipa{kʰɤ˧˥}}}  \mytextsc{clf}: \textcolor{darkblue}{\textbf{\ipa{kʰɤ˧˥}}} 
\lhead{\firstmark}
\rhead{\botmark}

\subsection{\hspace{-0.5cm} {\Large \textcolor{darkblue}{\textbf{\ipa{nv̩˩ɭɯ˧}}}}\hspace{0.5cm}[\kern2pt{\textcolor{darkblue}{\textbf{\ipa{xxxx non-correspondance entre le nombre de morphèmes et le nombre de tons de morphèmes}}}}\kern2pt]} \hypertarget{nv\string_=\string_Bl\string_RM\string_M1}{}
\markboth{\textcolor{darkblue}{\textbf{\ipa{nv̩˩ɭɯ˧}}}}{}
\textcolor{teal}{\mytextsc{noun}} \hspace{4pt} Tone: LM.
\textcolor{Sepia}{\selectlanguage{english}Soy beans, soya beans.} \zh{黄豆。}  \zh{量词}: \textcolor{darkblue}{\textbf{\ipa{wɤ˩}}}  \mytextsc{clf}: \textcolor{darkblue}{\textbf{\ipa{wɤ˩}}} 
\lhead{\firstmark}
\rhead{\botmark}

\subsection{\hspace{-0.5cm} {\Large \textcolor{darkblue}{\textbf{\ipa{nv̩˩mi˩}}}}\hspace{0.5cm}[\kern2pt{\textcolor{darkblue}{\textbf{\ipa{nv̩˩mi˥}}}}\kern2pt]} \hypertarget{nv\string_=\string_Bmi\string_B1}{}
\markboth{\textcolor{darkblue}{\textbf{\ipa{nv̩˩mi˩}}}}{}
\textcolor{teal}{\mytextsc{noun}} \hspace{4pt} Tone: L.
\ding{202} \textcolor{Sepia}{\selectlanguage{english}Heart.} \zh{心脏。}  ¶ \textcolor{darkblue}{\textbf{\ipa{hĩ˧ ʈʂʰɯ˧-v̩˧, | nv̩˩mi˩ tɕi˥! |}}} \textcolor{Sepia}{\selectlanguage{english}This person lacks courage! (literally “...(his/her) heart is small”)} \zh{这个人,胆小!(直译:“心小”)}  
 ¶ \textcolor{darkblue}{\textbf{\ipa{nv̩˩mi˩˥ | ɖɯ˧-ɭɯ˧ tsɤ˧ |}}} \textcolor{Sepia}{\selectlanguage{english}in sympathy, in unison} \zh{情投意合}  
 ¶ \textcolor{darkblue}{\textbf{\ipa{nv̩˩mi˩˥ | tʰi˧-tɕɯ˥ | so˩˥}}} \textcolor{Sepia}{\selectlanguage{english}to study patiently / to teach patiently} \zh{耐心地学习 / 耐心地教}  
 ¶ \textcolor{darkblue}{\textbf{\ipa{nv̩˩mi˩-qo˥ | tʰi˧-ʑi˥}}} \textcolor{Sepia}{\selectlanguage{english}to remember, to keep in mind} \zh{记住、记得(直译:‘心里存着’、‘心里有’)}  
 ¶ \textcolor{darkblue}{\textbf{\ipa{nv̩˩mi˩-qo˥ | tʰi˧-kʰɯ˧˥}}} \textcolor{Sepia}{\selectlanguage{english}to make an effort to remember, to carry in mind} \zh{记住(直译:‘放在心里’)}  
 \zh{量词}: \textcolor{darkblue}{\textbf{\ipa{ɭɯ˧}}} \ding{203} \textcolor{Sepia}{\selectlanguage{english}State of mind.} \zh{心情。}  ¶ \textcolor{darkblue}{\textbf{\ipa{nv̩˩mi˩ dzɯ˩\textasciitilde{}dzɯ˩-ɻ̍˥}}} \textcolor{Sepia}{\selectlanguage{english}not to get along well; to quarrel all the time; to poison each other's life} \zh{经常吵架、过不到一起去}  
 \mytextsc{clf}: \textcolor{darkblue}{\textbf{\ipa{ɭɯ˧}}} 
\lhead{\firstmark}
\rhead{\botmark}

\subsection{\hspace{-0.5cm} {\Large \textcolor{darkblue}{\textbf{\ipa{nv̩˩mi˩-ɖɯ˩}}}}\hspace{0.5cm}[\kern2pt{\textcolor{darkblue}{\textbf{\ipa{xxxx non-correspondance entre le nombre de morphèmes et le nombre de tons de morphèmes}}}}\kern2pt]} \hypertarget{nv\string_=\string_Bmi\string_B-d`M\string_B1}{}
\markboth{\textcolor{darkblue}{\textbf{\ipa{nv̩˩mi˩-ɖɯ˩}}}}{}
\textcolor{teal}{\mytextsc{adjective}} \hspace{4pt} Tone: L.
\textit{From:} \textbf{nv̩˩mi˩ and ɖɯ˩a} \textcolor{Sepia}{\selectlanguage{english}Courageous, brave.} \zh{勇敢、有勇气的。}  ¶ \textcolor{darkblue}{\textbf{\ipa{ʈʂʰɯ˧ | nv̩˩mi˩˥ | ɖwæ˧˥ | ɖɯ˩˥! | hĩ˧ | mɤ˧-ɖwæ˥!}}} \textcolor{Sepia}{\selectlanguage{english}(S)he is very brave! (S)he is not afraid of others / (s)he fears no one!} \zh{他很勇敢!谁也不怕!}  
 ¶ \textcolor{darkblue}{\textbf{\ipa{pɤ˩mi˩˥ | nv̩˩mi˩ ɖɯ˩˥, | ʝi˧-ɳɯ˧ tʰv̩˧˥; | mi˩zɯ˩ nv̩˥mi˩ ɖɯ˩ (-dʑo˩), | hĩ˧-ɳɯ˧ lɑ˧˥!}}} \textcolor{Sepia}{\selectlanguage{english}“If the frog is brave, it gets stamped on by the ox; if the woman is brave, she gets beaten!” (Explanation: weaker creatures must not be too brave: if a frog fears nothing and ventures onto the roads, it can easily get crushed to death; if a woman behaves with masculine self-assurance and courage, she gets into situations where people come to hands, and she eventually has the lower hand.)} \zh{“勇敢的青蛙,被牛压死。勇敢的女人,被人家打!”(说明:青蛙太勇敢,上马路,就容易被压死,而女人太勇敢,容易跟别人发生矛盾,最后就打不过男人。)}  

\lhead{\firstmark}
\rhead{\botmark}

\subsection{\hspace{-0.5cm} {\Large \textcolor{darkblue}{\textbf{\ipa{nv̩˩mi˩-ki˧ki˩}}}}\hspace{0.5cm}[\kern2pt{\textcolor{darkblue}{\textbf{\ipa{xxxx non-correspondance entre le nombre de morphèmes et le nombre de tons de morphèmes}}}}\kern2pt]} \hypertarget{nv\string_=\string_Bmi\string_B-ki\string_Mki\string_B1}{}
\markboth{\textcolor{darkblue}{\textbf{\ipa{nv̩˩mi˩-ki˧ki˩}}}}{}
\textcolor{teal}{\mytextsc{adjective}} \hspace{4pt} Tone: .
\textit{From:} \textbf{nv̩˩mi˩ and ki˧a} \textcolor{Sepia}{\selectlanguage{english}With similar mood/frame of mind.} \zh{心意相通。} 
\lhead{\firstmark}
\rhead{\botmark}

\subsection{\hspace{-0.5cm} {\Large \textcolor{darkblue}{\textbf{\ipa{nv̩˩mi˩-ʈʰi˩}}}}\hspace{0.5cm}[\kern2pt{\textcolor{darkblue}{\textbf{\ipa{xxxx non-correspondance entre le nombre de morphèmes et le nombre de tons de morphèmes}}}}\kern2pt]} \hypertarget{nv\string_=\string_Bmi\string_B-t`\string_hi\string_B1}{}
\markboth{\textcolor{darkblue}{\textbf{\ipa{nv̩˩mi˩-ʈʰi˩}}}}{}
\textcolor{teal}{\mytextsc{adjective}} \hspace{4pt} Tone: L.
\textit{From:} \textbf{nv̩˩mi˩ and ʈʰi˩a} \textcolor{Sepia}{\selectlanguage{english}Weak, worn out.} \zh{累得没精神了。} 
\lhead{\firstmark}
\rhead{\botmark}

\subsection{\hspace{-0.5cm} {\Large \textcolor{darkblue}{\textbf{\ipa{nv̩˩pi˧}}}}\hspace{0.5cm}[\kern2pt{\textcolor{darkblue}{\textbf{\ipa{nv̩˩pi˥}}}}\kern2pt]} \hypertarget{nv\string_=\string_Bpi\string_M1}{}
\markboth{\textcolor{darkblue}{\textbf{\ipa{nv̩˩pi˧}}}}{}
\textcolor{teal}{\mytextsc{noun}} \hspace{4pt} Tone: LM.
\textcolor{Sepia}{\selectlanguage{english}Soybean dregs.} \zh{豆粕。} 
\lhead{\firstmark}
\rhead{\botmark}

\subsection{\hspace{-0.5cm} {\Large \textcolor{darkblue}{\textbf{\ipa{nv̩˧pɤ˩}}}}\hspace{0.5cm}[\kern2pt{\textcolor{darkblue}{\textbf{\ipa{nv̩˧pɤ˩}}}}\kern2pt]} \hypertarget{nv\string_=\string_Mp7\string_B1}{}
\markboth{\textcolor{darkblue}{\textbf{\ipa{nv̩˧pɤ˩}}}}{}
\textcolor{teal}{\mytextsc{noun}} \hspace{4pt} Tone: L\#.
\textcolor{Sepia}{\selectlanguage{english}Broad beans; lima beans.} \zh{蚕豆。} 
\lhead{\firstmark}
\rhead{\botmark}

\subsection{\hspace{-0.5cm} {\Large \textcolor{darkblue}{\textbf{\ipa{nv̩˩tɕʰi\#˥}}}}\hspace{0.5cm}[\kern2pt{\textcolor{darkblue}{\textbf{\ipa{nv̩˩tɕʰi˥}}}}\kern2pt]} \hypertarget{nv\string_=\string_Bts£\string_hi\#\string_T1}{}
\markboth{\textcolor{darkblue}{\textbf{\ipa{nv̩˩tɕʰi\#˥}}}}{}
\textcolor{teal}{\mytextsc{noun}} \hspace{4pt} Tone: LM+\#H.
\textcolor{Sepia}{\selectlanguage{english}Fine chaff of beans (used to feed cows).} \zh{豆类的细糠秕,来喂牛。}  \zh{量词}: \textcolor{darkblue}{\textbf{\ipa{kʰɤ˧˥}}}  \mytextsc{clf}: \textcolor{darkblue}{\textbf{\ipa{kʰɤ˧˥}}} 
\lhead{\firstmark}
\rhead{\botmark}

\subsection{\hspace{-0.5cm} {\Large \textcolor{darkblue}{\textbf{\ipa{nv̩˩tsɑ˧˥}}}}\hspace{0.5cm}[\kern2pt{\textcolor{darkblue}{\textbf{\ipa{nv̩˩tsɑ˧˥}}}}\kern2pt]} \hypertarget{nv\string_=\string_BtsA\string_M\string_T1}{}
\markboth{\textcolor{darkblue}{\textbf{\ipa{nv̩˩tsɑ˧˥}}}}{}
\textcolor{teal}{\mytextsc{noun}} \hspace{4pt} Tone: LM+MH\#.
\textcolor{Sepia}{\selectlanguage{english}Coarse chaff of beans.} \zh{粗的豆糠。}  \zh{量词}: \textcolor{darkblue}{\textbf{\ipa{mɤ˩, etc}}}  \mytextsc{clf}: \textcolor{darkblue}{\textbf{\ipa{mɤ˩, etc}}} 
\lhead{\firstmark}
\rhead{\botmark}

\subsection{\hspace{-0.5cm} {\Large \textcolor{darkblue}{\textbf{\ipa{nv̩˧tv̩˥}}}}\hspace{0.5cm}[\kern2pt{\textcolor{darkblue}{\textbf{\ipa{nv̩˧tv̩˥}}}}\kern2pt]} \hypertarget{nv\string_=\string_Mtv\string_=\string_T1}{}
\markboth{\textcolor{darkblue}{\textbf{\ipa{nv̩˧tv̩˥}}}}{}
\textcolor{teal}{\mytextsc{noun}} \hspace{4pt} Tone: H\#.
\textcolor{Sepia}{\selectlanguage{english}Nosebag.} \zh{(挂在马脖子下面的)饲料袋子、马粮袋子。}  \zh{量词}: \textcolor{darkblue}{\textbf{\ipa{ɭɯ˧}}}  \mytextsc{clf}: \textcolor{darkblue}{\textbf{\ipa{ɭɯ˧}}} 
\lhead{\firstmark}
\rhead{\botmark}

\subsection{\hspace{-0.5cm} {\Large \textcolor{darkblue}{\textbf{\ipa{nv̩˩ze˧}}}}\hspace{0.5cm}[\kern2pt{\textcolor{darkblue}{\textbf{\ipa{nv̩˩ze˥}}}}\kern2pt]} \hypertarget{nv\string_=\string_Bze\string_M1}{}
\markboth{\textcolor{darkblue}{\textbf{\ipa{nv̩˩ze˧}}}}{}
\textcolor{teal}{\mytextsc{noun}} \hspace{4pt} Tone: LM.
\textcolor{Sepia}{\selectlanguage{english}Chickpea, \textit{Cicer arietinum}, black-coloured; the dish \zh{黑色凉粉} is made out of this pea.} \zh{鹰嘴豆、桃尔豆、鸡豆、鸡心豆。} Local Chinese dialect:\zh{鸡豆。}
\lhead{\firstmark}
\rhead{\botmark}

\newpage
\section*{\centering- \textcolor{darkblue}{\textbf{\ipa{ɳ}}} -}
\subsection{\hspace{-0.5cm} {\Large \textcolor{darkblue}{\textbf{\ipa{ɳæ˥}}}}\hspace{0.5cm}[\kern2pt{\textcolor{darkblue}{\textbf{\ipa{ɳæ˧˥}}}}\kern2pt]} \hypertarget{n`\{\string_T1}{}
\markboth{\textcolor{darkblue}{\textbf{\ipa{ɳæ˥}}}}{}
\textcolor{teal}{\mytextsc{verb}} \hspace{4pt} Tone: H.
\textcolor{Sepia}{\selectlanguage{english}To hide, to conceal oneself.} \zh{躲藏。}  ¶ \textcolor{darkblue}{\textbf{\ipa{tʰi˧-ɳæ˥}}} \textcolor{Sepia}{\selectlanguage{english}\mytextsc{dur} \string_} \zh{\mytextsc{dur} \string_}  

\lhead{\firstmark}
\rhead{\botmark}

\subsection{\hspace{-0.5cm} {\Large \textcolor{darkblue}{\textbf{\ipa{ɳæ˧=ɻ̍˩}}}}\hspace{0.5cm}[\kern2pt{\textcolor{darkblue}{\textbf{\ipa{ɳæ˧ɻ̍˩}}}}\kern2pt]} \hypertarget{n`\{\string_M=r£`̍\string_B1}{}
\markboth{\textcolor{darkblue}{\textbf{\ipa{ɳæ˧=ɻ̍˩}}}}{}
\textcolor{teal}{\mytextsc{pronoun/pronominal}} \hspace{4pt} Tone: L\#.
\textcolor{Sepia}{\selectlanguage{english}Second person plural. This is a variant of \textcolor{darkblue}{\textbf{\ipa{/no˧=ɻ̍˩/}}}; the form \textcolor{darkblue}{\textbf{\ipa{/no˧=ɻ̍˩/}}} is considered more correct.} \zh{你们。这是\textcolor{darkblue}{\textbf{\ipa{/no˧=ɻ̍˩/}}}的一个变体。\textcolor{darkblue}{\textbf{\ipa{/no˧=ɻ̍˩/}}}被认为是更标准的。} \textit{See:} \textcolor{darkblue}{\textbf{\ipa{ɳæ˩=ɻæ˧, no˧=ɻ̍˩}}} 
\lhead{\firstmark}
\rhead{\botmark}

\subsection{\hspace{-0.5cm} {\Large \textcolor{darkblue}{\textbf{\ipa{ɳæ˩=ɻæ˧}}}}\hspace{0.5cm}[\kern2pt{\textcolor{darkblue}{\textbf{\ipa{ɳæ˩ɻæ˥}}}}\kern2pt]} \hypertarget{n`\{\string_B=r£`\{\string_M1}{}
\markboth{\textcolor{darkblue}{\textbf{\ipa{ɳæ˩=ɻæ˧}}}}{}
\textcolor{teal}{\mytextsc{pronoun/pronominal}} \hspace{4pt} Tone: LM.
\textcolor{Sepia}{\selectlanguage{english}Second person associative plural.} \zh{你们家、你们家族。} \textit{See:} \textcolor{darkblue}{\textbf{\ipa{ɳæ˧=ɻ̍˩, no˧=ɻ̍˩}}} 
\lhead{\firstmark}
\rhead{\botmark}

\subsection{\hspace{-0.5cm} {\Large \textcolor{darkblue}{\textbf{\ipa{ɳæ˧˥}}}}\hspace{0.5cm}[\kern2pt{\textcolor{darkblue}{\textbf{\ipa{ɳæ˧˥}}}}\kern2pt]} \hypertarget{n`\{\string_M\string_T1}{}
\markboth{\textcolor{darkblue}{\textbf{\ipa{ɳæ˧˥}}}}{}
\textcolor{teal}{\mytextsc{verb}} \hspace{4pt} Tone: MH.
\ding{202} \textcolor{Sepia}{\selectlanguage{english}To press, to push down (with the hand); to press flat, to flatten; to squeeze.} \zh{按(用手)、压扁、挤压。}  ¶ \textcolor{darkblue}{\textbf{\ipa{mv̩˩tɕo˧ ɳæ˧˥}}} \textcolor{Sepia}{\selectlanguage{english}to push down, to press down} \zh{往下按}  
 ¶ \textcolor{darkblue}{\textbf{\ipa{le˧-ɳæ˩\textasciitilde{}ɳæ˩}}}  
\ding{203} \textcolor{Sepia}{\selectlanguage{english}To oppress.} \zh{压迫。}  ¶ \textcolor{darkblue}{\textbf{\ipa{hĩ˧ kʰv̩˧, | hĩ˧ ɳæ˩}}} \textcolor{Sepia}{\selectlanguage{english}to steal and oppress (description of a despot's behaviour)} \zh{偷和迫(描述专制统治者的行为)}  

\lhead{\firstmark}
\rhead{\botmark}

\subsection{\hspace{-0.5cm} {\Large \textcolor{darkblue}{\textbf{\ipa{ɳɯ˥}}}}\hspace{0.5cm}[\kern2pt{\textcolor{darkblue}{\textbf{\ipa{ɳɯ˥}}}}\kern2pt]} \hypertarget{n`M\string_T1}{}
\markboth{\textcolor{darkblue}{\textbf{\ipa{ɳɯ˥}}}}{}
\textcolor{teal}{\mytextsc{adjective}} \hspace{4pt} Tone: H.
\textcolor{Sepia}{\selectlanguage{english}Few.} \zh{少。}  ¶ \textcolor{darkblue}{\textbf{\ipa{hĩ˧ ɳɯ˧}}} \textcolor{Sepia}{\selectlanguage{english}people are few / there are few people} \zh{好的,不多!不好的,就很多了!}  
 ¶ \textcolor{darkblue}{\textbf{\ipa{tso˧\textasciitilde{}tso˧ | ɳɯ˧-ze˩}}} \textcolor{Sepia}{\selectlanguage{english}there are fewer things, the amount of things has decreased} \zh{东西(变)少了}  
 ¶ \textcolor{darkblue}{\textbf{\ipa{dʑɤ˩-hĩ˩˥, | le˧-ɳɯ˥! | mɤ˧-dʑɤ˩-hĩ˩, | le˧-dʑɯ˧˥!}}} \textcolor{Sepia}{\selectlanguage{english}Good one are few; bad ones are many! / There are few good ones, but many bad ones! (A comment about higher education institutions, among which laureates of the national University entrance examination are given a choice.)} \zh{好的少,不好的多!(关于大学:高考后,学生要报志愿)}  

\lhead{\firstmark}
\rhead{\botmark}

\subsection{\hspace{-0.5cm} {\Large \textcolor{darkblue}{\textbf{\ipa{‑ɳɯ˧}}} \textsubscript{1}}\hspace{0.5cm}[\kern2pt{\textcolor{darkblue}{\textbf{\ipa{ɳɯ˥}}}}\kern2pt]} \hypertarget{‑n`M\string_M1}{}
\markboth{\textcolor{darkblue}{\textbf{\ipa{‑ɳɯ˧}}} \textsubscript{1}}{}
\textcolor{teal}{\mytextsc{postposition}} \hspace{4pt} Tone: M.
\textcolor{Sepia}{\selectlanguage{english}Ablative, agent, and topic marker.} \zh{离格,施动者,主题。接近汉语的‘由’。} 
\lhead{\firstmark}
\rhead{\botmark}

\subsection{\hspace{-0.5cm} {\Large \textcolor{darkblue}{\textbf{\ipa{ɳɯ˧˥}}}}\hspace{0.5cm}[\kern2pt{\textcolor{darkblue}{\textbf{\ipa{ɳɯ˧˥}}}}\kern2pt]} \hypertarget{n`M\string_M\string_T1}{}
\markboth{\textcolor{darkblue}{\textbf{\ipa{ɳɯ˧˥}}}}{}
\textcolor{teal}{\mytextsc{verb}} \hspace{4pt} Tone: MH.
\textcolor{Sepia}{\selectlanguage{english}To wring, to tighten, to clamp.} \zh{拧。}  ¶ \textcolor{darkblue}{\textbf{\ipa{le˧-ɳɯ˧-ze˥}}} \textcolor{Sepia}{\selectlanguage{english}\mytextsc{accomp} \string_ \mytextsc{pfv}} \zh{拧了}  
 ¶ \textcolor{darkblue}{\textbf{\ipa{ʁo˧qɑ˥ | ʈʰɯ˧-ɭɯ˧ | le˧-ɳɯ˧-qɑ˥-jo˩!}}} \textcolor{Sepia}{\selectlanguage{english}Tighten the lid! (of a glass jar, used as drinking glass)} \zh{(你)拧一下盖子吧!}  

\lhead{\firstmark}
\rhead{\botmark}

\subsection{\hspace{-0.5cm} {\Large \textcolor{darkblue}{\textbf{\ipa{ɳv̩˩˧}}}}\hspace{0.5cm}[\kern2pt{\textcolor{darkblue}{\textbf{\ipa{ɳv̩˩˥}}}}\kern2pt]} \hypertarget{n`v\string_=\string_B\string_M1}{}
\markboth{\textcolor{darkblue}{\textbf{\ipa{ɳv̩˩˧}}}}{}
\textcolor{teal}{\mytextsc{noun}} \hspace{4pt} Tone: LM.
\textcolor{Sepia}{\selectlanguage{english}Moth; insect that eats into wood, books, clothes etc.} \zh{蛀虫。}  \zh{量词}: \textcolor{darkblue}{\textbf{\ipa{mi˩}}}  \mytextsc{clf}: \textcolor{darkblue}{\textbf{\ipa{mi˩}}} 
\lhead{\firstmark}
\rhead{\botmark}

\subsection{\hspace{-0.5cm} {\Large \textcolor{darkblue}{\textbf{\ipa{ɳv̩˥}}}}\hspace{0.5cm}[\kern2pt{\textcolor{darkblue}{\textbf{\ipa{ɳv̩˥}}}}\kern2pt]} \hypertarget{n`v\string_=\string_T1}{}
\markboth{\textcolor{darkblue}{\textbf{\ipa{ɳv̩˥}}}}{}
\textcolor{teal}{\mytextsc{verb}} \hspace{4pt} Tone: H.
\ding{202} \textcolor{Sepia}{\selectlanguage{english}To sniff.} \zh{闻嗅。} \ding{203} \textcolor{Sepia}{\selectlanguage{english}To hear, to get to know (good news…).} \zh{听到(消息)、风闻。}  ¶ \textcolor{darkblue}{\textbf{\ipa{mɤ˧-ɳv̩˥}}} \textcolor{Sepia}{\selectlanguage{english}I am not aware of this piece of news! / I didn't know about that!} \zh{(我)不知道这个消息!}  
 ¶ \textcolor{darkblue}{\textbf{\ipa{no˧ ə˧tso˧ ɳv̩˥?}}} \textcolor{Sepia}{\selectlanguage{english}Which piece of news did you get? / What did you get to know?} \zh{你听到了什么消息呢?}  

\lhead{\firstmark}
\rhead{\botmark}

\newpage
\section*{\centering- \textcolor{darkblue}{\textbf{\ipa{ɲ}}} -}
\subsection{\hspace{-0.5cm} {\Large \textcolor{darkblue}{\textbf{\ipa{ɲi˥}}} \textsubscript{1}}\hspace{0.5cm}[\kern2pt{\textcolor{darkblue}{\textbf{\ipa{ɲi˥}}}}\kern2pt]} \hypertarget{Ji\string_T1}{}
\markboth{\textcolor{darkblue}{\textbf{\ipa{ɲi˥}}} \textsubscript{1}}{}
\textcolor{teal}{\mytextsc{verb}} \hspace{4pt} Tone: H.
\textcolor{Sepia}{\selectlanguage{english}To listen.} \zh{听。}  ¶ \textcolor{darkblue}{\textbf{\ipa{tʰi˧-ɲi˥}}} \textcolor{Sepia}{\selectlanguage{english}\mytextsc{dur}} \zh{\mytextsc{dur}}  
 ¶ \textcolor{darkblue}{\textbf{\ipa{tso˧\textasciitilde{}tso˧ ɲi˧}}} \textcolor{Sepia}{\selectlanguage{english}to listen to things} \zh{听东西}  
 ¶ \textcolor{darkblue}{\textbf{\ipa{le˧-ɲi˥-ze˩}}} \textcolor{Sepia}{\selectlanguage{english}\mytextsc{accomp} \string_ \mytextsc{pfv}} \zh{听了}  

\lhead{\firstmark}
\rhead{\botmark}

\subsection{\hspace{-0.5cm} {\Large \textcolor{darkblue}{\textbf{\ipa{ɲi˥}}} \textsubscript{2}}\hspace{0.5cm}[\kern2pt{\textcolor{darkblue}{\textbf{\ipa{ɲi˥}}}}\kern2pt]} \hypertarget{Ji\string_T2}{}
\markboth{\textcolor{darkblue}{\textbf{\ipa{ɲi˥}}} \textsubscript{2}}{}
\textcolor{teal}{\mytextsc{verb}} \hspace{4pt} Tone: H.
\textcolor{Sepia}{\selectlanguage{english}To borrow from someone.} \zh{向别人借。}  ¶ \textcolor{darkblue}{\textbf{\ipa{hĩ˧-ki˧ | tso˧\textasciitilde{}tso˧ ɲi˧ |}}} \textcolor{Sepia}{\selectlanguage{english}to borrow things from someone} \zh{向别人借东西}  

\lhead{\firstmark}
\rhead{\botmark}

\subsection{\hspace{-0.5cm} {\Large \textcolor{darkblue}{\textbf{\ipa{ɲi˥}}} \textsubscript{3}}\hspace{0.5cm}[\kern2pt{\textcolor{darkblue}{\textbf{\ipa{ɲi˥}}}}\kern2pt]} \hypertarget{Ji\string_T3}{}
\markboth{\textcolor{darkblue}{\textbf{\ipa{ɲi˥}}} \textsubscript{3}}{}
\textcolor{teal}{\mytextsc{verb}} \hspace{4pt} Tone: H.
\textcolor{Sepia}{\selectlanguage{english}To lose, to be defeated.} \zh{败、输。} 
\lhead{\firstmark}
\rhead{\botmark}

\subsection{\hspace{-0.5cm} {\Large \textcolor{darkblue}{\textbf{\ipa{ɲi˥\textsubscript{b}}}}}\hspace{0.5cm}[\kern2pt{\textcolor{darkblue}{\textbf{\ipa{ɲi˥}}}}\kern2pt]} \hypertarget{Ji\string_Tb1}{}
\markboth{\textcolor{darkblue}{\textbf{\ipa{ɲi˥\textsubscript{b}}}}}{}
\textcolor{teal}{\mytextsc{classifier}} \hspace{4pt} Tone: H\textsubscript{b}.
\textcolor{Sepia}{\selectlanguage{english}Day.} \zh{日、天。}  ¶ \textcolor{darkblue}{\textbf{\ipa{ɖɯ˧-ɲi˥}}} \textcolor{Sepia}{\selectlanguage{english}one day} \zh{一天}  

\lhead{\firstmark}
\rhead{\botmark}

\subsection{\hspace{-0.5cm} {\Large \textcolor{darkblue}{\textbf{\ipa{ɲi˧}}}}\hspace{0.5cm}[\kern2pt{\textcolor{darkblue}{\textbf{\ipa{ɲi˥}}}}\kern2pt]} \hypertarget{Ji\string_M1}{}
\markboth{\textcolor{darkblue}{\textbf{\ipa{ɲi˧}}}}{}
\textcolor{teal}{\mytextsc{adjective}} \hspace{4pt} Tone: M.
\textcolor{Sepia}{\selectlanguage{english}Full, satiated.} \zh{饱。}  ¶ \textcolor{darkblue}{\textbf{\ipa{le˧-ɲi˧-ze˧}}} \textcolor{Sepia}{\selectlanguage{english}\mytextsc{accomp} \string_ \mytextsc{pfv}} \zh{饱了}  
 ¶ \textcolor{darkblue}{\textbf{\ipa{hɑ˧-ɲi˧(-ze˩)}}} \textcolor{Sepia}{\selectlanguage{english}I am full. / I am satiated.} \zh{吃饱了。 / 吃饱饭了。}  
 ¶ \textcolor{darkblue}{\textbf{\ipa{njɤ˧ | le˧-ɲi˧-ze˧!}}} \textcolor{Sepia}{\selectlanguage{english}I am full. / I am satiated.} \zh{我饱了。}  

\lhead{\firstmark}
\rhead{\botmark}

\subsection{\hspace{-0.5cm} {\Large \textcolor{darkblue}{\textbf{\ipa{ɲi˧\textsubscript{a}}}}}\hspace{0.5cm}[\kern2pt{\textcolor{darkblue}{\textbf{\ipa{ɲi˥}}}}\kern2pt]} \hypertarget{Ji\string_Ma1}{}
\markboth{\textcolor{darkblue}{\textbf{\ipa{ɲi˧\textsubscript{a}}}}}{}
\textcolor{teal}{\mytextsc{verb}} \hspace{4pt} Tone: M\textsubscript{a}.
\textcolor{Sepia}{\selectlanguage{english}To need.} \zh{需要。}  ¶ \textcolor{darkblue}{\textbf{\ipa{no˧ | ə˩-ɲi˧? | mɤ˧-ɲi˧!}}} \textcolor{Sepia}{\selectlanguage{english}Do you want (some)? - No!} \zh{你要吗?- 不要!}  

\lhead{\firstmark}
\rhead{\botmark}

\subsection{\hspace{-0.5cm} {\Large \textcolor{darkblue}{\textbf{\ipa{ɲi˧dʑɯ˧}}}}\hspace{0.5cm}[\kern2pt{\textcolor{darkblue}{\textbf{\ipa{ɲi˧dʑɯ˥}}}}\kern2pt]} \hypertarget{Ji\string_Mdz£M\string_M1}{}
\markboth{\textcolor{darkblue}{\textbf{\ipa{ɲi˧dʑɯ˧}}}}{}
\textcolor{teal}{\mytextsc{noun}} \hspace{4pt} Tone: H\#.
\textcolor{Sepia}{\selectlanguage{english}Penis.} \zh{男生殖器。}  \zh{量词}: \textcolor{darkblue}{\textbf{\ipa{ɭɯ˧}}}  \mytextsc{clf}: \textcolor{darkblue}{\textbf{\ipa{ɭɯ˧}}} 
\lhead{\firstmark}
\rhead{\botmark}

\subsection{\hspace{-0.5cm} {\Large \textcolor{darkblue}{\textbf{\ipa{ɲi˧gɤ\#˥}}}}\hspace{0.5cm}[\kern2pt{\textcolor{darkblue}{\textbf{\ipa{ɲi˧gɤ˧}}}}\kern2pt]} \hypertarget{Ji\string_Mg7\#\string_T1}{}
\markboth{\textcolor{darkblue}{\textbf{\ipa{ɲi˧gɤ\#˥}}}}{}
\textcolor{teal}{\mytextsc{noun}} \hspace{4pt} Tone: \#H.
\textcolor{Sepia}{\selectlanguage{english}Nose.} \zh{鼻子。}  \zh{量词}: \textcolor{darkblue}{\textbf{\ipa{ɭɯ˧}}}  \mytextsc{clf}: \textcolor{darkblue}{\textbf{\ipa{ɭɯ˧}}} 
\lhead{\firstmark}
\rhead{\botmark}

\subsection{\hspace{-0.5cm} {\Large \textcolor{darkblue}{\textbf{\ipa{ɲi˧gɤ˧-bæ˧˥}}}}\hspace{0.5cm}[\kern2pt{\textcolor{darkblue}{\textbf{\ipa{xxxx non-correspondance entre le nombre de morphèmes et le nombre de tons de morphèmes}}}}\kern2pt]} \hypertarget{Ji\string_Mg7\string_M-b\{\string_M\string_T1}{}
\markboth{\textcolor{darkblue}{\textbf{\ipa{ɲi˧gɤ˧-bæ˧˥}}}}{}
\textcolor{teal}{\mytextsc{noun}} \hspace{4pt} Tone: MH\#.
\textcolor{Sepia}{\selectlanguage{english}Rope attached to a cow's nasal ring.} \zh{牛鼻绳。也可以来指牛鼻圈。}  \zh{量词}: \textcolor{darkblue}{\textbf{\ipa{kʰɯ˩}}}  \mytextsc{clf}: \textcolor{darkblue}{\textbf{\ipa{kʰɯ˩}}} 
\lhead{\firstmark}
\rhead{\botmark}

\subsection{\hspace{-0.5cm} {\Large \textcolor{darkblue}{\textbf{\ipa{ɲi˧ɬi˧mi˧}}}}\hspace{0.5cm}[\kern2pt{\textcolor{darkblue}{\textbf{\ipa{ɲi˧ɬi˧mi˧}}}}\kern2pt]} \hypertarget{Ji\string_MKi\string_Mmi\string_M1}{}
\markboth{\textcolor{darkblue}{\textbf{\ipa{ɲi˧ɬi˧mi˧}}}}{}
\textcolor{teal}{\mytextsc{noun}} \hspace{4pt} Tone: M.
\textcolor{Sepia}{\selectlanguage{english}Second month.} \zh{二月。} 
\lhead{\firstmark}
\rhead{\botmark}

\subsection{\hspace{-0.5cm} {\Large \textcolor{darkblue}{\textbf{\ipa{ɲi˧mi˧}}}}\hspace{0.5cm}[\kern2pt{\textcolor{darkblue}{\textbf{\ipa{ɲi˧mi˧}}}}\kern2pt]} \hypertarget{Ji\string_Mmi\string_M1}{}
\markboth{\textcolor{darkblue}{\textbf{\ipa{ɲi˧mi˧}}}}{}
\textcolor{teal}{\mytextsc{noun}} \hspace{4pt} Tone: M.
\ding{202} \textcolor{Sepia}{\selectlanguage{english}Sun.} \zh{太阳。}  ¶ \textcolor{darkblue}{\textbf{\ipa{ɲi˧mi˧ tʰv̩˧}}} \textcolor{Sepia}{\selectlanguage{english}the sun rises} \zh{太阳出来、日出}  
 \zh{量词}: \textcolor{darkblue}{\textbf{\ipa{ɭɯ˧}}} \ding{203} \textcolor{Sepia}{\selectlanguage{english}Day; daytime; time.} \zh{日、时间。}  \mytextsc{clf}: \textcolor{darkblue}{\textbf{\ipa{ɭɯ˧}}} 
\lhead{\firstmark}
\rhead{\botmark}

\subsection{\hspace{-0.5cm} {\Large \textcolor{darkblue}{\textbf{\ipa{ɲi˧mi˧dɑ˧dzɯ\#˥}}}}\hspace{0.5cm}[\kern2pt{\textcolor{darkblue}{\textbf{\ipa{ɲi˧mi˧dɑ˧dzɯ˧}}}}\kern2pt]} \hypertarget{Ji\string_Mmi\string_MdA\string_MdzM\#\string_T1}{}
\markboth{\textcolor{darkblue}{\textbf{\ipa{ɲi˧mi˧dɑ˧dzɯ\#˥}}}}{}
\textcolor{teal}{\mytextsc{noun}} \hspace{4pt} Tone: \#H.
\textcolor{Sepia}{\selectlanguage{english}Solar eclipse.} \zh{日蚀。}  ¶ \textcolor{darkblue}{\textbf{\ipa{ɲi˧mi˧dɑ˧dzɯ˧ tʰv̩˧}}} \textcolor{Sepia}{\selectlanguage{english}there is a solar eclipse} \zh{有日蚀}  
 ¶ \textcolor{darkblue}{\textbf{\ipa{ɲi˧mi˧dɑ˧dzɯ˧ ɲi˥!}}} \textcolor{Sepia}{\selectlanguage{english}Yes, it's a solar eclipse!} \zh{是的,是日蚀!}  
 \zh{量词}: \textcolor{darkblue}{\textbf{\ipa{ʂɯ˩}}}  \mytextsc{clf}: \textcolor{darkblue}{\textbf{\ipa{ʂɯ˩}}} 
\lhead{\firstmark}
\rhead{\botmark}

\subsection{\hspace{-0.5cm} {\Large \textcolor{darkblue}{\textbf{\ipa{ɲi˧mi˧-gv̩˩}}}}\hspace{0.5cm}[\kern2pt{\textcolor{darkblue}{\textbf{\ipa{xxxx non-correspondance entre le nombre de morphèmes et le nombre de tons de morphèmes}}}}\kern2pt]} \hypertarget{Ji\string_Mmi\string_M-gv\string_=\string_B1}{}
\markboth{\textcolor{darkblue}{\textbf{\ipa{ɲi˧mi˧-gv̩˩}}}}{}
\textcolor{teal}{\mytextsc{noun}} \hspace{4pt} Tone: \mytextsc{L}.
\textcolor{Sepia}{\selectlanguage{english}West: “[the direction where] the sun sets”.} \zh{西方。}  ¶ \textcolor{darkblue}{\textbf{\ipa{ɲi˧mi˧gv̩˩-gi˩-dzɤ˩ se˩}}} \textcolor{Sepia}{\selectlanguage{english}to walk towards the west} \zh{往西边走}  

\lhead{\firstmark}
\rhead{\botmark}

\subsection{\hspace{-0.5cm} {\Large \textcolor{darkblue}{\textbf{\ipa{ɲi˧mi˧-kʰɯ˧ʂɯ˧}}}}\hspace{0.5cm}[\kern2pt{\textcolor{darkblue}{\textbf{\ipa{xxxx non-correspondance entre le nombre de morphèmes et le nombre de tons de morphèmes}}}}\kern2pt]} \hypertarget{Ji\string_Mmi\string_M-k\string_hM\string_Ms`M\string_M1}{}
\markboth{\textcolor{darkblue}{\textbf{\ipa{ɲi˧mi˧-kʰɯ˧ʂɯ˧}}}}{}
\textcolor{teal}{\mytextsc{noun}} \hspace{4pt} Tone: M.
\textcolor{Sepia}{\selectlanguage{english}Rays (of sunshine).} \zh{太阳的光线。}  \zh{量词}: \textcolor{darkblue}{\textbf{\ipa{kʰɯ˩}}}  \mytextsc{clf}: \textcolor{darkblue}{\textbf{\ipa{kʰɯ˩}}} 
\lhead{\firstmark}
\rhead{\botmark}

\subsection{\hspace{-0.5cm} {\Large \textcolor{darkblue}{\textbf{\ipa{ɲi˧mi˧tʰv̩\#˥}}}}\hspace{0.5cm}[\kern2pt{\textcolor{darkblue}{\textbf{\ipa{ɲi˧mi˧tʰv̩˧}}}}\kern2pt]} \hypertarget{Ji\string_Mmi\string_Mt\string_hv\string_=\#\string_T1}{}
\markboth{\textcolor{darkblue}{\textbf{\ipa{ɲi˧mi˧tʰv̩\#˥}}}}{}
\textcolor{teal}{\mytextsc{noun}} \hspace{4pt} Tone: \#H.
\textcolor{Sepia}{\selectlanguage{english}East, orient.} \zh{东方。}  ¶ \textcolor{darkblue}{\textbf{\ipa{ɲi˧mi˧tʰv̩˧-gi˧}}} \textcolor{Sepia}{\selectlanguage{english}the direction of the east} \zh{东方方向}  
 ¶ \textcolor{darkblue}{\textbf{\ipa{ɲi˧mi˧tʰv̩˧-gi˧ | se˧}}} \textcolor{Sepia}{\selectlanguage{english}to walk towards the east} \zh{向东面走}  
 ¶ \textcolor{darkblue}{\textbf{\ipa{ɲi˧mi˧tʰv̩˧-gi˧ | dʑo˩˥}}} \textcolor{Sepia}{\selectlanguage{english}to live in the East, to live in the Orient. (Context: the consultant imagines that I am in Europe, thinking of her, saying: 'She lives in the Orient'.)} \zh{住在东方(合作人想象我在欧洲,想着她说:‘她住在东方’。)}  

\lhead{\firstmark}
\rhead{\botmark}

\subsection{\hspace{-0.5cm} {\Large \textcolor{darkblue}{\textbf{\ipa{ɲi˧mi˧-ʈæ˧ʂɯ˧}}}}\hspace{0.5cm}[\kern2pt{\textcolor{darkblue}{\textbf{\ipa{xxxx non-correspondance entre le nombre de morphèmes et le nombre de tons de morphèmes}}}}\kern2pt]} \hypertarget{Ji\string_Mmi\string_M-t`\{\string_Ms`M\string_M1}{}
\markboth{\textcolor{darkblue}{\textbf{\ipa{ɲi˧mi˧-ʈæ˧ʂɯ˧}}}}{}
\textcolor{teal}{\mytextsc{noun}} \hspace{4pt} Tone: M.
\textcolor{Sepia}{\selectlanguage{english}Sunflower.} \zh{葵花。}  \zh{量词}: \textcolor{darkblue}{\textbf{\ipa{dzi˩}}}  \mytextsc{clf}: \textcolor{darkblue}{\textbf{\ipa{dzi˩}}} 
\lhead{\firstmark}
\rhead{\botmark}

\subsection{\hspace{-0.5cm} {\Large \textcolor{darkblue}{\textbf{\ipa{ɲi˧nɑ˩}}}}\hspace{0.5cm}[\kern2pt{\textcolor{darkblue}{\textbf{\ipa{ɲi˧nɑ˩}}}}\kern2pt]} \hypertarget{Ji\string_MnA\string_B1}{}
\markboth{\textcolor{darkblue}{\textbf{\ipa{ɲi˧nɑ˩}}}}{}
\textcolor{teal}{\mytextsc{noun}} \hspace{4pt} Tone: L\#.
\textcolor{Sepia}{\selectlanguage{english}Cane; rattan.} \zh{藤子。} 
\lhead{\firstmark}
\rhead{\botmark}

\subsection{\hspace{-0.5cm} {\Large \textcolor{darkblue}{\textbf{\ipa{ɲi˧pʰv̩˩}}}}\hspace{0.5cm}[\kern2pt{\textcolor{darkblue}{\textbf{\ipa{ɲi˧pʰv̩˩}}}}\kern2pt]} \hypertarget{Ji\string_Mp\string_hv\string_=\string_B1}{}
\markboth{\textcolor{darkblue}{\textbf{\ipa{ɲi˧pʰv̩˩}}}}{}
\textcolor{teal}{\mytextsc{noun}} \hspace{4pt} Tone: L\#.
\textcolor{Sepia}{\selectlanguage{english}Frost.} \zh{霜。}  ¶ \textcolor{darkblue}{\textbf{\ipa{ɲi˧pʰv̩˩ lɑ˩-ze˩}}} \textcolor{Sepia}{\selectlanguage{english}there is some frost} \zh{有霜}  

\lhead{\firstmark}
\rhead{\botmark}

\subsection{\hspace{-0.5cm} {\Large \textcolor{darkblue}{\textbf{\ipa{ɲi˧qʰv̩˧}}}}\hspace{0.5cm}[\kern2pt{\textcolor{darkblue}{\textbf{\ipa{ɲi˧qʰv̩˧}}}}\kern2pt]} \hypertarget{Ji\string_Mq\string_hv\string_=\string_M1}{}
\markboth{\textcolor{darkblue}{\textbf{\ipa{ɲi˧qʰv̩˧}}}}{}
\textcolor{teal}{\mytextsc{noun}} \hspace{4pt} Tone: M.
\ding{202} \textcolor{Sepia}{\selectlanguage{english}Nostril.} \zh{鼻孔。}  \zh{量词}: \textcolor{darkblue}{\textbf{\ipa{ɭɯ˧}}} \ding{203} \textcolor{Sepia}{\selectlanguage{english}Snivel, nasal mucus.} \zh{鼻涕。}  \mytextsc{clf}: \textcolor{darkblue}{\textbf{\ipa{ɭɯ˧}}} 
\lhead{\firstmark}
\rhead{\botmark}

\subsection{\hspace{-0.5cm} {\Large \textcolor{darkblue}{\textbf{\ipa{ɲi˧se˩}}}}\hspace{0.5cm}[\kern2pt{\textcolor{darkblue}{\textbf{\ipa{ɲi˧se˩}}}}\kern2pt]} \hypertarget{Ji\string_Mse\string_B1}{}
\markboth{\textcolor{darkblue}{\textbf{\ipa{ɲi˧se˩}}}}{}
\textcolor{teal}{\mytextsc{noun}} \hspace{4pt} Tone: L\#.
\textcolor{Sepia}{\selectlanguage{english}The name of a village.} \zh{小落水(村落名)。}  ¶ \textcolor{darkblue}{\textbf{\ipa{ɲi˧se˩, | nɑ˩-lɑ˧ ɲi˥!}}} \textcolor{Sepia}{\selectlanguage{english}Nhissei is a thoroughly Na village! / Na is populated entirely by Na people!} \zh{小落水,是纯摩梭的一个村落!}  
 ¶ \textcolor{darkblue}{\textbf{\ipa{ɬi˧ki˧, | ɲi˧se˩, | tɑ˧dzi˩, | mv̩˧qʰwæ˩, | lɑ˧tʰɑ˧-di˧˥}}} \textcolor{Sepia}{\selectlanguage{english}Villages that one passes when moving away from the Yongning plain, towards Lugu lake. These villages do not count as part of Yongning proper. The last, \textcolor{darkblue}{\textbf{\ipa{/lɑ˧tʰɑ˧-di˧˥/}}}, is not a village name like the preceding four: it refers to the entire Na area beyond the fourth village.} \zh{永宁到泸沽湖所经过的村落,依次是:里格、尼赛、大祖、木垮,然后到拉塔地(拉塔地指的是泸沽湖周边的摩梭地区,包括左所、洛水村等)}  

\lhead{\firstmark}
\rhead{\botmark}

\subsection{\hspace{-0.5cm} {\Large \textcolor{darkblue}{\textbf{\ipa{ɲi˧to˧}}}}\hspace{0.5cm}[\kern2pt{\textcolor{darkblue}{\textbf{\ipa{ɲi˧to˧}}}}\kern2pt]} \hypertarget{Ji\string_Mto\string_M1}{}
\markboth{\textcolor{darkblue}{\textbf{\ipa{ɲi˧to˧}}}}{}
\textcolor{teal}{\mytextsc{noun}} \hspace{4pt} Tone: M.
\textcolor{Sepia}{\selectlanguage{english}The mouth, seen as including the part of the face surrounding the mouth, in particular the jaw.} \zh{嘴巴,包括嘴周围的部位:颚等。}  ¶ \textcolor{darkblue}{\textbf{\ipa{ɲi˧to˧ ʈʂʰwæ˩}}} \textcolor{Sepia}{\selectlanguage{english}talkative} \zh{多嘴、拉不断扯不断(直译:“嘴快”)}  
 \zh{量词}: \textcolor{darkblue}{\textbf{\ipa{kʰwɤ˥}}}  \mytextsc{clf}: \textcolor{darkblue}{\textbf{\ipa{kʰwɤ˥}}} 
\lhead{\firstmark}
\rhead{\botmark}

\subsection{\hspace{-0.5cm} {\Large \textcolor{darkblue}{\textbf{\ipa{ɲi˧-ʈʂæ˧-ʑi˧˥}}}}\hspace{0.5cm}[\kern2pt{\textcolor{darkblue}{\textbf{\ipa{xxxx non-correspondance entre le nombre de morphèmes et le nombre de tons de morphèmes}}}}\kern2pt]} \hypertarget{Ji\string_M-t`s`\{\string_M-z£i\string_M\string_T1}{}
\markboth{\textcolor{darkblue}{\textbf{\ipa{ɲi˧-ʈʂæ˧-ʑi˧˥}}}}{}
\textcolor{teal}{\mytextsc{noun}} \hspace{4pt} Tone: MH\#.
\textcolor{Sepia}{\selectlanguage{english}The building inside the farm where the bedrooms are located, and a living-room (downstairs in the centre). Literally 'the two-floor building', as this is the only building that has rooms on two floors.} \zh{二层房:农场里面的一栋楼,正对着农场大门。}  ¶ \textcolor{darkblue}{\textbf{\ipa{ɲi˧-ʈʂæ˧-ʑi˧-di˥}}} \textcolor{Sepia}{\selectlanguage{english}same meaning} \zh{同上}  
 \zh{量词}: \textcolor{darkblue}{\textbf{\ipa{ɭɯ˧}}}  \mytextsc{clf}: \textcolor{darkblue}{\textbf{\ipa{ɭɯ˧}}} 
\lhead{\firstmark}
\rhead{\botmark}

\subsection{\hspace{-0.5cm} {\Large \textcolor{darkblue}{\textbf{\ipa{ɲi˧zo\#˥}}}}\hspace{0.5cm}[\kern2pt{\textcolor{darkblue}{\textbf{\ipa{ɲi˧zo˧}}}}\kern2pt]} \hypertarget{Ji\string_Mzo\#\string_T1}{}
\markboth{\textcolor{darkblue}{\textbf{\ipa{ɲi˧zo\#˥}}}}{}
\textcolor{teal}{\mytextsc{noun}} \hspace{4pt} Tone: \#H.
\textcolor{Sepia}{\selectlanguage{english}Fish.} \zh{鱼。}  ¶ \textcolor{darkblue}{\textbf{\ipa{ɲi˧zo˧ tʰv̩˧-mi˥\# / ɲi˧zo˧ tʰv̩˧-mi˧˥}}} \textcolor{Sepia}{\selectlanguage{english}\mytextsc{n}+\mytextsc{dem}+\mytextsc{clf}} \zh{那条鱼}  
 ¶ \textcolor{darkblue}{\textbf{\ipa{ɲi˧zo˧-tɑ˧pv̩˥}}} \textcolor{Sepia}{\selectlanguage{english}dried fish} \zh{干鱼}  
 \zh{量词}: \textcolor{darkblue}{\textbf{\ipa{mi˩}}}  \mytextsc{clf}: \textcolor{darkblue}{\textbf{\ipa{mi˩}}} 
\lhead{\firstmark}
\rhead{\botmark}

\subsection{\hspace{-0.5cm} {\Large \textcolor{darkblue}{\textbf{\ipa{ɲi˩}}} \textsubscript{1}}\hspace{0.5cm}[\kern2pt{\textcolor{darkblue}{\textbf{\ipa{ɲi˩˥}}}}\kern2pt]} \hypertarget{Ji\string_B1}{}
\markboth{\textcolor{darkblue}{\textbf{\ipa{ɲi˩}}} \textsubscript{1}}{}
\textcolor{teal}{\mytextsc{verb}} \hspace{4pt} Tone: L\textsubscript{a}.
\textcolor{Sepia}{\selectlanguage{english}To press, to hold (clamped under the arm, between the legs...).} \zh{夹、夹持。}  ¶ \textcolor{darkblue}{\textbf{\ipa{ɖɯ˧-ɲi˧\textasciitilde{}ɲi˥-ɻ̍˩}}} \textcolor{Sepia}{\selectlanguage{english}to press  a little} \zh{夹一点}  

\lhead{\firstmark}
\rhead{\botmark}

\subsection{\hspace{-0.5cm} {\Large \textcolor{darkblue}{\textbf{\ipa{ɲi˩}}} \textsubscript{2}}\hspace{0.5cm}[\kern2pt{\textcolor{darkblue}{\textbf{\ipa{ɲi˩˥}}}}\kern2pt]} \hypertarget{Ji\string_B2}{}
\markboth{\textcolor{darkblue}{\textbf{\ipa{ɲi˩}}} \textsubscript{2}}{}
\textcolor{teal}{\mytextsc{pronoun/pronominal}} \hspace{4pt} Tone: L.
\textcolor{Sepia}{\selectlanguage{english}Who.} \zh{谁。}  ¶ \textcolor{darkblue}{\textbf{\ipa{ɲi˩ ɲi˧?}}} \textcolor{Sepia}{\selectlanguage{english}Who is that?} \zh{是谁?}  
 ¶ \textcolor{darkblue}{\textbf{\ipa{no˧ | ɲi˩ ɲi˧?}}} \textcolor{Sepia}{\selectlanguage{english}Who are you?} \zh{你是谁?}  
 ¶ \textcolor{darkblue}{\textbf{\ipa{ʈʂʰɯ˧ | ɲi˩ ɲi˧?}}} \textcolor{Sepia}{\selectlanguage{english}Who is this person?} \zh{他是谁?}  
 ¶ \textcolor{darkblue}{\textbf{\ipa{no˧ | ɲi˩˥ | -ki˩ bi˩-pi˩, | ɖɯ˧-bæ˧ lɑ˧ ɲi˥!}}} \textcolor{Sepia}{\selectlanguage{english}No matter where you go, it's the same everywhere!} \zh{无论你去谁(家),都一样!}  
 ¶ \textcolor{darkblue}{\textbf{\ipa{no˧ | ɲi˩-ki˥ bi˩-pi˩, | ɖɯ˧-bæ˧ lɑ˧ ɲi˥!}}} \textcolor{Sepia}{\selectlanguage{english}As previous example, with a different division into tone groups} \zh{同上,声调段界不同}  

\lhead{\firstmark}
\rhead{\botmark}

\subsection{\hspace{-0.5cm} {\Large \textcolor{darkblue}{\textbf{\ipa{-ɲi˩}}}}\hspace{0.5cm}[\kern2pt{\textcolor{darkblue}{\textbf{\ipa{ɲi˩˥}}}}\kern2pt]} \hypertarget{-Ji\string_B1}{}
\markboth{\textcolor{darkblue}{\textbf{\ipa{-ɲi˩}}}}{}
\textcolor{teal}{\mytextsc{discourse}} \textcolor{teal}{\mytextsc{particle}} \hspace{4pt} Tone: L.
\textcolor{Sepia}{\selectlanguage{english}A particle derived from the copula, described by L. Lidz (2010:497) as conveying “an epistemic strategy that marks a high degree of certitude”.} \zh{\mytextsc{肯定(°系词)。}} 
\lhead{\firstmark}
\rhead{\botmark}

\subsection{\hspace{-0.5cm} {\Large \textcolor{darkblue}{\textbf{\ipa{ɲi˩\textsubscript{a}}}} \textsubscript{1}}\hspace{0.5cm}[\kern2pt{\textcolor{darkblue}{\textbf{\ipa{ɲi˩˥}}}}\kern2pt]} \hypertarget{Ji\string_Ba1}{}
\markboth{\textcolor{darkblue}{\textbf{\ipa{ɲi˩\textsubscript{a}}}} \textsubscript{1}}{}
\textcolor{teal}{\mytextsc{verb}} \hspace{4pt} Tone: L\textsubscript{a}.
\textcolor{Sepia}{\selectlanguage{english}To twine, to wind; twist with the fingers (e.g. linen, to make thread).} \zh{捻,缠线。}  ¶ \textcolor{darkblue}{\textbf{\ipa{le˧-ɲi˩}}} \textcolor{Sepia}{\selectlanguage{english}\mytextsc{accomp}} \zh{\mytextsc{accomp}}  
 ¶ \textcolor{darkblue}{\textbf{\ipa{sɑ˧ ɲi˥}}} \textcolor{Sepia}{\selectlanguage{english}to twine hemp (to make thread)} \zh{捻麻}  
 ¶ \textcolor{darkblue}{\textbf{\ipa{ɖɯ˧-ɲi˧\textasciitilde{}ɲi˥-ɻ̍˩}}} \textcolor{Sepia}{\selectlanguage{english}\mytextsc{delimitative} \mytextsc{red} \mytextsc{inceptive}} \zh{捻一捻}  

\lhead{\firstmark}
\rhead{\botmark}

\subsection{\hspace{-0.5cm} {\Large \textcolor{darkblue}{\textbf{\ipa{ɲi˩\textsubscript{a}}}} \textsubscript{2}}\hspace{0.5cm}[\kern2pt{\textcolor{darkblue}{\textbf{\ipa{ɲi˩˥}}}}\kern2pt]} \hypertarget{Ji\string_Ba2}{}
\markboth{\textcolor{darkblue}{\textbf{\ipa{ɲi˩\textsubscript{a}}}} \textsubscript{2}}{}
\textcolor{teal}{\mytextsc{verb}} \hspace{4pt} Tone: L\textsubscript{a}.
\textcolor{Sepia}{\selectlanguage{english}To break (tool), to be broken.} \zh{设备坏了。}  ¶ \textcolor{darkblue}{\textbf{\ipa{le˧-ɲi˩-ze˩}}} \textcolor{Sepia}{\selectlanguage{english}\mytextsc{accomp} \string_ \mytextsc{pfv}: it's broken!} \zh{坏了!/破了!}  
 ¶ \textcolor{darkblue}{\textbf{\ipa{tso˧\textasciitilde{}tso˧ ɲi˥}}} \textcolor{Sepia}{\selectlanguage{english}to break things} \zh{东西坏了}  

\lhead{\firstmark}
\rhead{\botmark}

\subsection{\hspace{-0.5cm} {\Large \textcolor{darkblue}{\textbf{\ipa{ɲi˩\textsubscript{a}}}} \textsubscript{3}}\hspace{0.5cm}[\kern2pt{\textcolor{darkblue}{\textbf{\ipa{ɲi˩˥}}}}\kern2pt]} \hypertarget{Ji\string_Ba3}{}
\markboth{\textcolor{darkblue}{\textbf{\ipa{ɲi˩\textsubscript{a}}}} \textsubscript{3}}{}
\textcolor{teal}{\mytextsc{verb}} \hspace{4pt} Tone: L\textsubscript{a}.
\textcolor{Sepia}{\selectlanguage{english}Copula.} \zh{是\mytextsc{系词。}} 
\lhead{\firstmark}
\rhead{\botmark}

\subsection{\hspace{-0.5cm} {\Large \textcolor{darkblue}{\textbf{\ipa{ɲi˩bv̩˩}}}}\hspace{0.5cm}[\kern2pt{\textcolor{darkblue}{\textbf{\ipa{ɲi˩bv̩˩˥}}}}\kern2pt]} \hypertarget{Ji\string_Bbv\string_=\string_B1}{}
\markboth{\textcolor{darkblue}{\textbf{\ipa{ɲi˩bv̩˩}}}}{}
\textcolor{teal}{\mytextsc{noun}} \hspace{4pt} Tone: L.
\textcolor{Sepia}{\selectlanguage{english}Grasshopper, cricket.} \zh{蟋蟀。}  \zh{量词}: \textcolor{darkblue}{\textbf{\ipa{mi˩}}}  \mytextsc{clf}: \textcolor{darkblue}{\textbf{\ipa{mi˩}}} 
\lhead{\firstmark}
\rhead{\botmark}

\subsection{\hspace{-0.5cm} {\Large \textcolor{darkblue}{\textbf{\ipa{ɲi˩bv̩˩-ʂe˩sɑ˧}}}}\hspace{0.5cm}[\kern2pt{\textcolor{darkblue}{\textbf{\ipa{xxxx non-correspondance entre le nombre de morphèmes et le nombre de tons de morphèmes}}}}\kern2pt]} \hypertarget{Ji\string_Bbv\string_=\string_B-s`e\string_BsA\string_M1}{}
\markboth{\textcolor{darkblue}{\textbf{\ipa{ɲi˩bv̩˩-ʂe˩sɑ˧}}}}{}
\textcolor{teal}{\mytextsc{noun}} \hspace{4pt} Tone: LM.
\textcolor{Sepia}{\selectlanguage{english}Dragonfly.} \zh{蜻蜓。}  \zh{量词}: \textcolor{darkblue}{\textbf{\ipa{mi˩}}}  \mytextsc{clf}: \textcolor{darkblue}{\textbf{\ipa{mi˩}}} 
\lhead{\firstmark}
\rhead{\botmark}

\subsection{\hspace{-0.5cm} {\Large \textcolor{darkblue}{\textbf{\ipa{ɲi˩mɑ\#˥}}}}\hspace{0.5cm}[\kern2pt{\textcolor{darkblue}{\textbf{\ipa{ɲi˩mɑ˥}}}}\kern2pt]} \hypertarget{Ji\string_BmA\#\string_T1}{}
\markboth{\textcolor{darkblue}{\textbf{\ipa{ɲi˩mɑ\#˥}}}}{}
\textcolor{teal}{\mytextsc{noun}} \hspace{4pt} Tone: LM+\#H.
\textcolor{Sepia}{\selectlanguage{english}Masculine given name used for the elder of two twins (the child who is born first).} \zh{男性名字,起给双胞胎中的老大。} 
\lhead{\firstmark}
\rhead{\botmark}

\subsection{\hspace{-0.5cm} {\Large \textcolor{darkblue}{\textbf{\ipa{ɲi˩pʰv̩˩}}}}\hspace{0.5cm}[\kern2pt{\textcolor{darkblue}{\textbf{\ipa{ɲi˩pʰv̩˩˥}}}}\kern2pt]} \hypertarget{Ji\string_Bp\string_hv\string_=\string_B1}{}
\markboth{\textcolor{darkblue}{\textbf{\ipa{ɲi˩pʰv̩˩}}}}{}
\textcolor{teal}{\mytextsc{noun}} \hspace{4pt} Tone: L.
\textcolor{Sepia}{\selectlanguage{english}A mountain plant; the consultant proposes this term for water mint, \textit{Mentha aquatica, Mentha hirsuta Huds.} but this is unlikely to be the correct identification.} \zh{一种植物。合作人看水薄荷的图片就觉得像这种植物,但很可能不是。李达珠等(2015:98)翻译为“野牡丹”但这好像也不准确。}  ¶ \textcolor{darkblue}{\textbf{\ipa{ɲi˩pʰv̩˩-bæ˥bæ˩}}} \textcolor{Sepia}{\selectlanguage{english}the flower of this plant} \zh{这种植物的花}  

\lhead{\firstmark}
\rhead{\botmark}

\subsection{\hspace{-0.5cm} {\Large \textcolor{darkblue}{\textbf{\ipa{ɲi˩=ɻ̍˥}}}}\hspace{0.5cm}[\kern2pt{\textcolor{darkblue}{\textbf{\ipa{ɲi˩ɻ̍˥}}}}\kern2pt]} \hypertarget{Ji\string_B=r£`̍\string_T1}{}
\markboth{\textcolor{darkblue}{\textbf{\ipa{ɲi˩=ɻ̍˥}}}}{}
\textcolor{teal}{\mytextsc{pronoun/pronominal}} \hspace{4pt} Tone: LM+H\#.
\textcolor{Sepia}{\selectlanguage{english}Second person associative pronoun: you and your clan/family/friends.} \zh{第二人称,联想复数:你与周边的人(家人、家族、亲戚、朋友们……)。} 
\lhead{\firstmark}
\rhead{\botmark}

\subsection{\hspace{-0.5cm} {\Large \textcolor{darkblue}{\textbf{\ipa{ɲi˩tsɯ\#˥}}}}\hspace{0.5cm}[\kern2pt{\textcolor{darkblue}{\textbf{\ipa{ɲi˩tsɯ˥}}}}\kern2pt]} \hypertarget{Ji\string_BtsM\#\string_T1}{}
\markboth{\textcolor{darkblue}{\textbf{\ipa{ɲi˩tsɯ\#˥}}}}{}
\textcolor{teal}{\mytextsc{noun}} \hspace{4pt} Tone: LM+\#H.
\textcolor{Sepia}{\selectlanguage{english}Hmong (ethnic group).} \zh{苗族。}  \zh{量词}: \textcolor{darkblue}{\textbf{\ipa{v̩˧}}}  \mytextsc{clf}: \textcolor{darkblue}{\textbf{\ipa{v̩˧}}} 
\lhead{\firstmark}
\rhead{\botmark}

\subsection{\hspace{-0.5cm} {\Large \textcolor{darkblue}{\textbf{\ipa{ɲi˩ʈʂe˩}}}}\hspace{0.5cm}[\kern2pt{\textcolor{darkblue}{\textbf{\ipa{ɲi˩ʈʂe˩˥}}}}\kern2pt]} \hypertarget{Ji\string_Bt`s`e\string_B1}{}
\markboth{\textcolor{darkblue}{\textbf{\ipa{ɲi˩ʈʂe˩}}}}{}
\textcolor{teal}{\mytextsc{noun}} \hspace{4pt} Tone: L.
\textcolor{Sepia}{\selectlanguage{english}Door bar.} \zh{门闩。}  ¶ \textcolor{darkblue}{\textbf{\ipa{ɲi˩ʈʂe˩ tʰi˥-kʰɯ˩, | tʰi˧-ʈæ˩!}}} \textcolor{Sepia}{\selectlanguage{english}Put on the door bar, to lock (the main door)!} \zh{放门闩,(好好)锁(门)!}  

\lhead{\firstmark}
\rhead{\botmark}

\subsection{\hspace{-0.5cm} {\Large \textcolor{darkblue}{\textbf{\ipa{ɲi˧˥}}}}\hspace{0.5cm}[\kern2pt{\textcolor{darkblue}{\textbf{\ipa{ɲi˧˥}}}}\kern2pt]} \hypertarget{Ji\string_M\string_T1}{}
\markboth{\textcolor{darkblue}{\textbf{\ipa{ɲi˧˥}}}}{}
\textcolor{teal}{\mytextsc{number}} \hspace{4pt} Tone: MH.
\textcolor{Sepia}{\selectlanguage{english}2.} \zh{2。} 
\lhead{\firstmark}
\rhead{\botmark}

\newpage
\section*{\centering- \textcolor{darkblue}{\textbf{\ipa{ŋ}}} -}
\subsection{\hspace{-0.5cm} {\Large \textcolor{darkblue}{\textbf{\ipa{ŋæ˧ʝi˩}}}}\hspace{0.5cm}[\kern2pt{\textcolor{darkblue}{\textbf{\ipa{ŋæ˩ʝi˥}}}}\kern2pt]} \hypertarget{N\{\string_Mj££i\string_B1}{}
\markboth{\textcolor{darkblue}{\textbf{\ipa{ŋæ˧ʝi˩}}}}{}
\textcolor{teal}{\mytextsc{adjective}} \hspace{4pt} Tone: L\#.
\textcolor{Sepia}{\selectlanguage{english}Easy and comfortable, at ease.} \zh{安逸(汉语借词)。}  Borrowing: Chinese  \zh{安逸}

\lhead{\firstmark}
\rhead{\botmark}

\subsection{\hspace{-0.5cm} {\Large \textcolor{darkblue}{\textbf{\ipa{ŋɤ˩ŋɤ˩}}}}\hspace{0.5cm}[\kern2pt{\textcolor{darkblue}{\textbf{\ipa{ŋɤ˩ŋɤ˩˥}}}}\kern2pt]} \hypertarget{N7\string_BN7\string_B1}{}
\markboth{\textcolor{darkblue}{\textbf{\ipa{ŋɤ˩ŋɤ˩}}}}{}
\textcolor{teal}{\mytextsc{noun}} \hspace{4pt} Tone: L.
\textcolor{Sepia}{\selectlanguage{english}Palate.} \zh{上腭。}  \zh{量词}: \textcolor{darkblue}{\textbf{\ipa{kʰwɤ˥}}}  \mytextsc{clf}: \textcolor{darkblue}{\textbf{\ipa{kʰwɤ˥}}} 
\lhead{\firstmark}
\rhead{\botmark}

\subsection{\hspace{-0.5cm} {\Large \textcolor{darkblue}{\textbf{\ipa{ŋv̩˩}}}}\hspace{0.5cm}[\kern2pt{\textcolor{darkblue}{\textbf{\ipa{ŋv̩˥}}}}\kern2pt]} \hypertarget{Nv\string_=\string_B1}{}
\markboth{\textcolor{darkblue}{\textbf{\ipa{ŋv̩˩}}}}{}
\textcolor{teal}{\mytextsc{noun}} \hspace{4pt} Tone: L.
\ding{202} \textcolor{Sepia}{\selectlanguage{english}Silver; money.} \zh{银子。}  ¶ \textcolor{darkblue}{\textbf{\ipa{ŋv˧hæ̃˩/ or et ærgent càd ærgent, pætrimoine}}} \textcolor{Sepia}{\selectlanguage{english}money, wealth; literally 'silver and gold'} \zh{金钱、钱财、财富。直译:‘银子与金子’}  
\ding{203} \textcolor{Sepia}{\selectlanguage{english}Money.} \zh{钱。} 
\lhead{\firstmark}
\rhead{\botmark}

\subsection{\hspace{-0.5cm} {\Large \textcolor{darkblue}{\textbf{\ipa{ŋv̩˩\textsubscript{a}}}}}\hspace{0.5cm}[\kern2pt{\textcolor{darkblue}{\textbf{\ipa{ŋv̩˩˥}}}}\kern2pt]} \hypertarget{Nv\string_=\string_Ba1}{}
\markboth{\textcolor{darkblue}{\textbf{\ipa{ŋv̩˩\textsubscript{a}}}}}{}
\textcolor{teal}{\mytextsc{verb}} \hspace{4pt} Tone: L\textsubscript{a}.
\textcolor{Sepia}{\selectlanguage{english}To cry, to weep.} \zh{哭。}  ¶ \textcolor{darkblue}{\textbf{\ipa{(tʰi˧-)ŋv̩˧\textasciitilde{}ŋv̩˥}}} \textcolor{Sepia}{\selectlanguage{english}\mytextsc{dur} \mytextsc{red}} \zh{哭一场}  

\lhead{\firstmark}
\rhead{\botmark}

\subsection{\hspace{-0.5cm} {\Large \textcolor{darkblue}{\textbf{\ipa{ŋwæ˧qʰv̩˧}}}}\hspace{0.5cm}[\kern2pt{\textcolor{darkblue}{\textbf{\ipa{ŋwæ˧qʰv̩˧}}}}\kern2pt]} \hypertarget{Nw\{\string_Mq\string_hv\string_=\string_M1}{}
\markboth{\textcolor{darkblue}{\textbf{\ipa{ŋwæ˧qʰv̩˧}}}}{}
\textcolor{teal}{\mytextsc{noun}} \hspace{4pt} Tone: M.
\textcolor{Sepia}{\selectlanguage{english}Oven to make bricks.} \zh{烧瓦的烤炉。}  ¶ \textcolor{darkblue}{\textbf{\ipa{ŋwæ˧qʰv̩˧ ʂɯ˧-ʑi˩}}} \textcolor{Sepia}{\selectlanguage{english}'the seven families of the Brick Oven': an expression formerly used to designate the people from Alawa village, at a time when there were only seven families living there.} \zh{‘瓦炉七家’:过去来指阿拉瓦村的人,当时那里只有七家住}  
 ¶ \textcolor{darkblue}{\textbf{\ipa{ə˧lɑ˧-ʁwɤ˧ | ŋwæ˧qʰv̩˧ | tsʰe˧ɲi˧ ʑi˩}}} \textcolor{Sepia}{\selectlanguage{english}'the twelve families of Alawa and the Brick Oven': an expression formerly used to designate the people from Alawa village, at a time when the number of families had increased from seven to twelve through migration.} \zh{‘阿拉瓦瓦炉十二家’:过去来指阿拉瓦村的人,当时那里住的人家,从七家已经增加到十二家}  
 \zh{量词}: \textcolor{darkblue}{\textbf{\ipa{ɭɯ˧}}}  \mytextsc{clf}: \textcolor{darkblue}{\textbf{\ipa{ɭɯ˧}}} 
\lhead{\firstmark}
\rhead{\botmark}

\subsection{\hspace{-0.5cm} {\Large \textcolor{darkblue}{\textbf{\ipa{ŋwɤ˧}}}}\hspace{0.5cm}[\kern2pt{\textcolor{darkblue}{\textbf{\ipa{ŋwɤ˥}}}}\kern2pt]} \hypertarget{Nw7\string_M1}{}
\markboth{\textcolor{darkblue}{\textbf{\ipa{ŋwɤ˧}}}}{}
\textcolor{teal}{\mytextsc{number}} \hspace{4pt} Tone: M? H\#? (pas L).
\textcolor{Sepia}{\selectlanguage{english}5.} \zh{5。} 
\lhead{\firstmark}
\rhead{\botmark}

\subsection{\hspace{-0.5cm} {\Large \textcolor{darkblue}{\textbf{\ipa{ŋwɤ˧hɑ̃˩}}}}\hspace{0.5cm}[\kern2pt{\textcolor{darkblue}{\textbf{\ipa{ŋwɤ˧hɑ̃˩}}}}\kern2pt]} \hypertarget{Nw7\string_MhA\string_~\string_B1}{}
\markboth{\textcolor{darkblue}{\textbf{\ipa{ŋwɤ˧hɑ̃˩}}}}{}
\textcolor{teal}{\mytextsc{noun}} \hspace{4pt} Tone: L\#.
\textcolor{Sepia}{\selectlanguage{english}A mountain to the South-West of Yongning.} \zh{位于永宁西南的一座山。}  ¶ \textcolor{darkblue}{\textbf{\ipa{kɤ˧mv̩˧˥, | æ˧ʂæ˧, | ŋwɤ˧hɑ̃˩, | ʂwæ˧gv̩\#˥, | nɑ˩tsʰi˩˥ | -tɕʰɤ˧pɤ˧mi\#˥, | qv̩˧ɻ̍˧-ʈʂʰɑ˧nɑ˥ |}}} \textcolor{Sepia}{\selectlanguage{english}The six mountains of Yongning that carry a name and have a definite symbolic value. The other mountains do not have comparable symbolic value, and fewer people use specific names for them.} \zh{永宁地区有固定名字的六座山。其它的山,因为没有重要的象征意义,因此没有取名。}  

\lhead{\firstmark}
\rhead{\botmark}

\subsection{\hspace{-0.5cm} {\Large \textcolor{darkblue}{\textbf{\ipa{ŋwɤ˧pʰæ˧˥}}}}\hspace{0.5cm}[\kern2pt{\textcolor{darkblue}{\textbf{\ipa{ŋwɤ˧pʰæ˧˥}}}}\kern2pt]} \hypertarget{Nw7\string_Mp\string_h\{\string_M\string_T1}{}
\markboth{\textcolor{darkblue}{\textbf{\ipa{ŋwɤ˧pʰæ˧˥}}}}{}
\textcolor{teal}{\mytextsc{noun}} \hspace{4pt} Tone: MH\#.
\textcolor{Sepia}{\selectlanguage{english}Tile.} \zh{瓦(汉语借词)。}  Borrowing: Chinese  \zh{瓦}
 \zh{量词}: \textcolor{darkblue}{\textbf{\ipa{pʰæ˧˥}}}  \mytextsc{clf}: \textcolor{darkblue}{\textbf{\ipa{pʰæ˧˥}}} 
\lhead{\firstmark}
\rhead{\botmark}

\subsection{\hspace{-0.5cm} {\Large \textcolor{darkblue}{\textbf{\ipa{ŋwɤ˧qo˥}}}}\hspace{0.5cm}[\kern2pt{\textcolor{darkblue}{\textbf{\ipa{ŋwɤ˧qo˥}}}}\kern2pt]} \hypertarget{Nw7\string_Mqo\string_T1}{}
\markboth{\textcolor{darkblue}{\textbf{\ipa{ŋwɤ˧qo˥}}}}{}
\textcolor{teal}{\mytextsc{noun}} \hspace{4pt} Tone: H\#.
\textcolor{Sepia}{\selectlanguage{english}Knee.} \zh{膝盖。}  \zh{量词}: \textcolor{darkblue}{\textbf{\ipa{ɭɯ˧}}}  \mytextsc{clf}: \textcolor{darkblue}{\textbf{\ipa{ɭɯ˧}}} \textit{See:} \hyperlink{}{\textcolor{darkblue}{\textbf{\ipa{ŋwɤ˩ɬv̩˧˥}}}} 
\lhead{\firstmark}
\rhead{\botmark}

\subsection{\hspace{-0.5cm} {\Large \textcolor{darkblue}{\textbf{\ipa{ŋwɤ˧tsʰi˩}}}}\hspace{0.5cm}[\kern2pt{\textcolor{darkblue}{\textbf{\ipa{ŋwɤ˧tsʰi˩}}}}\kern2pt]} \hypertarget{Nw7\string_Mts\string_hi\string_B1}{}
\markboth{\textcolor{darkblue}{\textbf{\ipa{ŋwɤ˧tsʰi˩}}}}{}
\textcolor{teal}{\mytextsc{number}} \hspace{4pt} Tone: L\#.
\textcolor{Sepia}{\selectlanguage{english}50.} \zh{50。} 
\lhead{\firstmark}
\rhead{\botmark}

\subsection{\hspace{-0.5cm} {\Large \textcolor{darkblue}{\textbf{\ipa{ŋwɤ˩ɭɯ˧-tse˥pʰæ˩}}}}\hspace{0.5cm}[\kern2pt{\textcolor{darkblue}{\textbf{\ipa{ŋwɤ˩ɭɯ˧tse˥pʰæ˩}}}}\kern2pt]} \hypertarget{Nw7\string_Bl\string_RM\string_M-tse\string_Tp\string_h\{\string_B1}{}
\markboth{\textcolor{darkblue}{\textbf{\ipa{ŋwɤ˩ɭɯ˧-tse˥pʰæ˩}}}}{}
\textcolor{teal}{\mytextsc{noun}} \hspace{4pt} Tone: LM+\#H-.
\textcolor{Sepia}{\selectlanguage{english}Kneebone.} \zh{膝盖骨。}  \zh{量词}: \textcolor{darkblue}{\textbf{\ipa{ɭɯ˧}}}  \mytextsc{clf}: \textcolor{darkblue}{\textbf{\ipa{ɭɯ˧}}} 
\lhead{\firstmark}
\rhead{\botmark}

\subsection{\hspace{-0.5cm} {\Large \textcolor{darkblue}{\textbf{\ipa{ŋwɤ˩ɬi˩mi˩}}}}\hspace{0.5cm}[\kern2pt{\textcolor{darkblue}{\textbf{\ipa{ŋwɤ˩ɬi˩mi˩˥}}}}\kern2pt]} \hypertarget{Nw7\string_BKi\string_Bmi\string_B1}{}
\markboth{\textcolor{darkblue}{\textbf{\ipa{ŋwɤ˩ɬi˩mi˩}}}}{}
\textcolor{teal}{\mytextsc{noun}} \hspace{4pt} Tone: L.
\textcolor{Sepia}{\selectlanguage{english}5th month.} \zh{五月。} 
\lhead{\firstmark}
\rhead{\botmark}

\subsection{\hspace{-0.5cm} {\Large \textcolor{darkblue}{\textbf{\ipa{ŋwɤ˩ɬv̩˧˥}}}}\hspace{0.5cm}[\kern2pt{\textcolor{darkblue}{\textbf{\ipa{ŋwɤ˩ɬv̩˧˥}}}}\kern2pt]} \hypertarget{Nw7\string_BKv\string_=\string_M\string_T1}{}
\markboth{\textcolor{darkblue}{\textbf{\ipa{ŋwɤ˩ɬv̩˧˥}}}}{}
\textcolor{teal}{\mytextsc{noun}} \hspace{4pt} Tone: LM+MH\#.
\textcolor{Sepia}{\selectlanguage{english}Cartilages of the knee; literally “marrow of the knee”. This expression emphasizes the fragility of this articulation.} \zh{膝盖(直译:“膝盖髓”)。这个说法强调膝盖的脆弱。}  ¶ \textcolor{darkblue}{\textbf{\ipa{[M23] ŋwɤ˩ɬv̩˧-ko˧lo˥ go˩.}}} \textcolor{Sepia}{\selectlanguage{english}to feel pain inside the knee} \zh{膝盖疼。}  
 \zh{量词}: \textcolor{darkblue}{\textbf{\ipa{ɭɯ˧}}}  \mytextsc{clf}: \textcolor{darkblue}{\textbf{\ipa{ɭɯ˧}}} \textit{See:} \hyperlink{}{\textcolor{darkblue}{\textbf{\ipa{ŋwɤ˧qo˥}}}} 
\lhead{\firstmark}
\rhead{\botmark}

\subsection{\hspace{-0.5cm} {\Large \textcolor{darkblue}{\textbf{\ipa{ŋwɤ˧˥}}}}\hspace{0.5cm}[\kern2pt{\textcolor{darkblue}{\textbf{\ipa{ŋwɤ˧˥}}}}\kern2pt]} \hypertarget{Nw7\string_M\string_T1}{}
\markboth{\textcolor{darkblue}{\textbf{\ipa{ŋwɤ˧˥}}}}{}
\textcolor{teal}{\mytextsc{verb}} \hspace{4pt} Tone: MH.
\textcolor{Sepia}{\selectlanguage{english}To sting, to pierce.} \zh{刺痛。}  ¶ \textcolor{darkblue}{\textbf{\ipa{tɕʰi˧ ŋwɤ˩-ze˩}}} \textcolor{Sepia}{\selectlanguage{english}(He/she) was stung by a thorn} \zh{(他)被刺扎疼了。}  

\lhead{\firstmark}
\rhead{\botmark}

\newpage
\section*{\centering- \textcolor{darkblue}{\textbf{\ipa{õ}}} -}
\subsection{\hspace{-0.5cm} {\Large \textcolor{darkblue}{\textbf{\ipa{õ˧dɤ˧ɻ̍˧}}}}\hspace{0.5cm}[\kern2pt{\textcolor{darkblue}{\textbf{\ipa{õ˧dɤ˧ɻ̍˧}}}}\kern2pt]} \hypertarget{o\string_~\string_Md7\string_Mr£`̍\string_M1}{}
\markboth{\textcolor{darkblue}{\textbf{\ipa{õ˧dɤ˧ɻ̍˧}}}}{}
\textcolor{teal}{\mytextsc{noun}} \hspace{4pt} Tone: M.
\textcolor{Sepia}{\selectlanguage{english}Foundation, fundamentals.} \zh{根本。}  ¶ \textcolor{darkblue}{\textbf{\ipa{õ˧dɤ˧ɻ̍˧-ɳɯ˧, | hĩ˧ ʈʂʰɯ˧-v̩˧ | ʈʂʰɯ˧ne˧ gv̩˧˥ | -ɲi˩!}}} \textcolor{Sepia}{\selectlanguage{english}So that is how he really behaves / does! (Comment on someone whose behaviour is not respectful of good manners)} \zh{他原来是这样做事情的! / 他原来这么不懂事!}  

\lhead{\firstmark}
\rhead{\botmark}

\subsection{\hspace{-0.5cm} {\Large \textcolor{darkblue}{\textbf{\ipa{õ˧ʈʂwɤ˧}}}}\hspace{0.5cm}[\kern2pt{\textcolor{darkblue}{\textbf{\ipa{õ˧ʈʂwɤ˧}}}}\kern2pt]} \hypertarget{o\string_~\string_Mt`s`w7\string_M1}{}
\markboth{\textcolor{darkblue}{\textbf{\ipa{õ˧ʈʂwɤ˧}}}}{}
\textcolor{teal}{\mytextsc{noun}} \hspace{4pt} Tone: M.
\textcolor{Sepia}{\selectlanguage{english}Mosquito.} \zh{蚊子。}  ¶ \textcolor{darkblue}{\textbf{\ipa{õ˧ʈʂwɤ˧ le˧-tʰv̩˧-ze˧!}}} \textcolor{Sepia}{\selectlanguage{english}Here comes a mosquito! / A mosquito has come in! (=into the room, into the mosquito net...)} \zh{有一只蚊子!}  
 ¶ \textcolor{darkblue}{\textbf{\ipa{ʂɯ˧-ɬi˧mi˧, | õ˧ʈʂwɤ˧! |}}} \textcolor{Sepia}{\selectlanguage{english}In the seventh month, there are lots of mosquitoes!} \zh{七月份,蚊子多! / 七月份,是蚊子多的一个月!}  
 \zh{量词}: \textcolor{darkblue}{\textbf{\ipa{mi˩}}}  \mytextsc{clf}: \textcolor{darkblue}{\textbf{\ipa{mi˩}}} 
\lhead{\firstmark}
\rhead{\botmark}

\subsection{\hspace{-0.5cm} {\Large \textcolor{darkblue}{\textbf{\ipa{õ˧ʈʂʰɯ˧ne˧-ʝi˥}}}}\hspace{0.5cm}[\kern2pt{\textcolor{darkblue}{\textbf{\ipa{xxxx non-correspondance entre le nombre de morphèmes et le nombre de tons de morphèmes}}}}\kern2pt]} \hypertarget{o\string_~\string_Mt`s`\string_hM\string_Mne\string_M-j££i\string_T1}{}
\markboth{\textcolor{darkblue}{\textbf{\ipa{õ˧ʈʂʰɯ˧ne˧-ʝi˥}}}}{}
\textcolor{teal}{\mytextsc{adverb(ial)}} \hspace{4pt} Tone: MH\#.
\textcolor{Sepia}{\selectlanguage{english}In that way.} \zh{那样。} 
\lhead{\firstmark}
\rhead{\botmark}

\subsection{\hspace{-0.5cm} {\Large \textcolor{darkblue}{\textbf{\ipa{õ˩dv̩˧˥}}}}\hspace{0.5cm}[\kern2pt{\textcolor{darkblue}{\textbf{\ipa{õ˩dv̩˧˥}}}}\kern2pt]} \hypertarget{o\string_~\string_Bdv\string_=\string_M\string_T1}{}
\markboth{\textcolor{darkblue}{\textbf{\ipa{õ˩dv̩˧˥}}}}{}
\textcolor{teal}{\mytextsc{noun}} \hspace{4pt} Tone: LM+MH\#.
\textcolor{Sepia}{\selectlanguage{english}Wolf.} \zh{狼。}  \zh{量词}: \textcolor{darkblue}{\textbf{\ipa{mi˩}}}  \mytextsc{clf}: \textcolor{darkblue}{\textbf{\ipa{mi˩}}} 
\lhead{\firstmark}
\rhead{\botmark}

\subsection{\hspace{-0.5cm} {\Large \textcolor{darkblue}{\textbf{\ipa{õ˩dv̩˧-kʰv̩˥mi˩}}}}\hspace{0.5cm}[\kern2pt{\textcolor{darkblue}{\textbf{\ipa{õ˩dv̩˧˥kʰv̩˧mi˧}}}}\kern2pt]} \hypertarget{o\string_~\string_Bdv\string_=\string_M-k\string_hv\string_=\string_Tmi\string_B1}{}
\markboth{\textcolor{darkblue}{\textbf{\ipa{õ˩dv̩˧-kʰv̩˥mi˩}}}}{}
\textcolor{teal}{\mytextsc{noun}} \hspace{4pt} Tone: LM+MH\#-.
\textcolor{Sepia}{\selectlanguage{english}Wolfhound.} \zh{狼狗。}  \zh{量词}: \textcolor{darkblue}{\textbf{\ipa{mi˩}}}  \mytextsc{clf}: \textcolor{darkblue}{\textbf{\ipa{mi˩}}} 
\lhead{\firstmark}
\rhead{\botmark}

\subsection{\hspace{-0.5cm} {\Large \textcolor{darkblue}{\textbf{\ipa{õ˩dv̩˧-mi˥}}}}\hspace{0.5cm}[\kern2pt{\textcolor{darkblue}{\textbf{\ipa{xxxx non-correspondance entre le nombre de morphèmes et le nombre de tons de morphèmes}}}}\kern2pt]} \hypertarget{o\string_~\string_Bdv\string_=\string_M-mi\string_T1}{}
\markboth{\textcolor{darkblue}{\textbf{\ipa{õ˩dv̩˧-mi˥}}}}{}
\textcolor{teal}{\mytextsc{noun}} \hspace{4pt} Tone: LM+H\#.
\textcolor{Sepia}{\selectlanguage{english}Female wolf.} \zh{母狼。}  \zh{量词}: \textcolor{darkblue}{\textbf{\ipa{mi˩}}}  \mytextsc{clf}: \textcolor{darkblue}{\textbf{\ipa{mi˩}}} 
\lhead{\firstmark}
\rhead{\botmark}

\subsection{\hspace{-0.5cm} {\Large \textcolor{darkblue}{\textbf{\ipa{õ˩dv̩˧-pʰv̩\#˥}}}}\hspace{0.5cm}[\kern2pt{\textcolor{darkblue}{\textbf{\ipa{xxxx non-correspondance entre le nombre de morphèmes et le nombre de tons de morphèmes}}}}\kern2pt]} \hypertarget{o\string_~\string_Bdv\string_=\string_M-p\string_hv\string_=\#\string_T1}{}
\markboth{\textcolor{darkblue}{\textbf{\ipa{õ˩dv̩˧-pʰv̩\#˥}}}}{}
\textcolor{teal}{\mytextsc{noun}} \hspace{4pt} Tone: LM+\#H.
\textcolor{Sepia}{\selectlanguage{english}Male wolf.} \zh{公狼。}  \zh{量词}: \textcolor{darkblue}{\textbf{\ipa{mi˩}}}  \mytextsc{clf}: \textcolor{darkblue}{\textbf{\ipa{mi˩}}} 
\lhead{\firstmark}
\rhead{\botmark}

\subsection{\hspace{-0.5cm} {\Large \textcolor{darkblue}{\textbf{\ipa{õ˩dv̩˧-zo\#˥}}}}\hspace{0.5cm}[\kern2pt{\textcolor{darkblue}{\textbf{\ipa{õ˩dv̩˧zo˥}}}}\kern2pt]} \hypertarget{o\string_~\string_Bdv\string_=\string_M-zo\#\string_T1}{}
\markboth{\textcolor{darkblue}{\textbf{\ipa{õ˩dv̩˧-zo\#˥}}}}{}
\textcolor{teal}{\mytextsc{noun}} \hspace{4pt} Tone: LM+\#H-.
\textcolor{Sepia}{\selectlanguage{english}Little wolf.} \zh{小狼。} 
\lhead{\firstmark}
\rhead{\botmark}

\subsection{\hspace{-0.5cm} {\Large \textcolor{darkblue}{\textbf{\ipa{õ˧˥}}}}\hspace{0.5cm}[\kern2pt{\textcolor{darkblue}{\textbf{\ipa{õ˧˥}}}}\kern2pt]} \hypertarget{o\string_~\string_M\string_T1}{}
\markboth{\textcolor{darkblue}{\textbf{\ipa{õ˧˥}}}}{}
\textcolor{teal}{\mytextsc{pronoun/pronominal}} \hspace{4pt} Tone: MH.
\textcolor{Sepia}{\selectlanguage{english}(one)self.} \zh{自己。}  ¶ \textcolor{darkblue}{\textbf{\ipa{õ˧-ɑ˥ʁo˩}}} \textcolor{Sepia}{\selectlanguage{english}one's house} \zh{自己家}  
 ¶ \textcolor{darkblue}{\textbf{\ipa{õ˧-dʑɯ˥, õ˩ ʈʰɯ˩! |}}} \textcolor{Sepia}{\selectlanguage{english}Each drinks from her own bottle! (Context: a toddler has grabbed another's milk bottle; parents prevent her from drinking from it.)} \zh{自己喝自己的!(情景:一个婴儿抓另一个婴儿的奶瓶。)}  
 ¶ \textcolor{darkblue}{\textbf{\ipa{õ˧-ʂe˥, õ˩ ʈʰæ˩! |}}} \textcolor{Sepia}{\selectlanguage{english}Each person eats their own slab of meat! (Describing table manners: each person used to receive one slice of meat and eat it up, unlike Chinese custom, in which each guest picks food mouthful by mouthful, with chopsticks, from the dishes placed on the table.)} \zh{自己吃自己的(那块)肉!(关于饮食习惯:吃饭的时候,每人分得一块肉,自己吃完。当地人认为,汉族没有这种分吃的习惯。)}  
 ¶ \textcolor{darkblue}{\textbf{\ipa{õ˧-bv̩˥-õ˩ ʝi˩-ɳɯ˩ | sɯ˧-kv̩˩!}}} \textcolor{Sepia}{\selectlanguage{english}One learns by practising oneself! / It's by practising oneself that one really masters a skill!} \zh{自己做,就能学会!/ 要学会,就得自己熟练!}  
 ¶ \textcolor{darkblue}{\textbf{\ipa{õ˧-bv̩˥-õ˩ +N |}}} \textcolor{Sepia}{\selectlanguage{english}one's own N} \zh{自己的(+名词)}  
 ¶ \textcolor{darkblue}{\textbf{\ipa{õ˧-bv̩˥-õ˩ ʐwæ˩}}} \textcolor{Sepia}{\selectlanguage{english}one's own horse} \zh{自己的马}  
 ¶ \textcolor{darkblue}{\textbf{\ipa{õ˧-bv̩˥-õ˩ ʝi˩}}} \textcolor{Sepia}{\selectlanguage{english}one's own cow} \zh{自己的牛}  
 ¶ \textcolor{darkblue}{\textbf{\ipa{õ˧-bv̩˥-õ˩ lv̩˩}}} \textcolor{Sepia}{\selectlanguage{english}one's own field} \zh{自己的田地}  
 ¶ \textcolor{darkblue}{\textbf{\ipa{õ˧-bv̩˥-õ˩ ɖʐe˩}}} \textcolor{Sepia}{\selectlanguage{english}one's own money} \zh{自己的钱}  
 ¶ \textcolor{darkblue}{\textbf{\ipa{õ˧mv̩˥-õ˩di˩}}} \textcolor{Sepia}{\selectlanguage{english}birth place} \zh{出生的地方、老家、故乡}  
 ¶ \textcolor{darkblue}{\textbf{\ipa{hĩ˧-mv˥ hĩ˩-di˩ | qʰɑ˧-dʑɤ˥\textasciitilde{}dʑɤ˩, | õ˧-mv˥ õ˩-di˩ tsʰe˩ mɤ˩-gv˩!}}} \textcolor{Sepia}{\selectlanguage{english}No matter how beautiful other people's places are, they can never be equal to one's own homeland!} \zh{其他人的地方怎么好,也比不过自己的地方!}  
 ¶ \textcolor{darkblue}{\textbf{\ipa{õ˧-ə˧mv̩˥ / õ˧-ə˥mv̩˩ / õ˧-ə˧mv̩˧˥}}} \textcolor{Sepia}{\selectlanguage{english}one's own elder (brother or sister)} \zh{自家姐姐(或哥哥)}  
 ¶ \textcolor{darkblue}{\textbf{\ipa{õ˧-ə˧v̩˥ / õ˧-ə˥v̩˩}}} \textcolor{Sepia}{\selectlanguage{english}one's own maternal uncle} \zh{自家舅舅(母亲的兄弟)}  
 ¶ \textcolor{darkblue}{\textbf{\ipa{õ˧-ʐɤ˥mi˩, õ˩ ɲi˩! |}}} \textcolor{Sepia}{\selectlanguage{english}One's path, that is one's identity / one's destiny! / The path you choose is your destiny!} \zh{自己的道路,就是自己!/ 每个人有自己的命运!}  

\lhead{\firstmark}
\rhead{\botmark}

\newpage
\section*{\centering- \textcolor{darkblue}{\textbf{\ipa{p}}} -}
\subsection{\hspace{-0.5cm} {\Large \textcolor{darkblue}{\textbf{\ipa{pɑ˧tɕɤ˧}}}}\hspace{0.5cm}[\kern2pt{\textcolor{darkblue}{\textbf{\ipa{pɑ˧tɕɤ˥}}}}\kern2pt]} \hypertarget{pA\string_Mts£7\string_M1}{}
\markboth{\textcolor{darkblue}{\textbf{\ipa{pɑ˧tɕɤ˧}}}}{}
\textcolor{teal}{\mytextsc{noun}} \hspace{4pt} Tone: M.
\textcolor{Sepia}{\selectlanguage{english}Plantain.} \zh{芭蕉(汉语借词)。}  Borrowing: Chinese  \zh{芭蕉}

\lhead{\firstmark}
\rhead{\botmark}

\subsection{\hspace{-0.5cm} {\Large \textcolor{darkblue}{\textbf{\ipa{pæ˥}}}}\hspace{0.5cm}[\kern2pt{\textcolor{darkblue}{\textbf{\ipa{pæ˥}}}}\kern2pt]} \hypertarget{p\{\string_T1}{}
\markboth{\textcolor{darkblue}{\textbf{\ipa{pæ˥}}}}{}
\textcolor{teal}{\mytextsc{verb}} \hspace{4pt} Tone: H.
\textcolor{Sepia}{\selectlanguage{english}To move house.} \zh{搬(家)。}  Borrowing: Chinese  \zh{搬?}

\lhead{\firstmark}
\rhead{\botmark}

\subsection{\hspace{-0.5cm} {\Large \textcolor{darkblue}{\textbf{\ipa{pæ˥\textsubscript{a}}}}}\hspace{0.5cm}[\kern2pt{\textcolor{darkblue}{\textbf{\ipa{pæ˥}}}}\kern2pt]} \hypertarget{p\{\string_Ta1}{}
\markboth{\textcolor{darkblue}{\textbf{\ipa{pæ˥\textsubscript{a}}}}}{}
\textcolor{teal}{\mytextsc{classifier}} \hspace{4pt} Tone: H\textsubscript{a}.
\textcolor{Sepia}{\selectlanguage{english}Classifier for packs/herds (of horses...), troops (of soldiers)...} \zh{量词:马、军人……(一队)。} 
\lhead{\firstmark}
\rhead{\botmark}

\subsection{\hspace{-0.5cm} {\Large \textcolor{darkblue}{\textbf{\ipa{pæ˧kʰwɤ\#˥}}}}\hspace{0.5cm}[\kern2pt{\textcolor{darkblue}{\textbf{\ipa{pæ˧kʰwɤ˧˥}}}}\kern2pt]} \hypertarget{p\{\string_Mk\string_hw7\#\string_T1}{}
\markboth{\textcolor{darkblue}{\textbf{\ipa{pæ˧kʰwɤ\#˥}}}}{}
\textcolor{teal}{\mytextsc{noun}} \hspace{4pt} Tone: \#H.
\textcolor{Sepia}{\selectlanguage{english}Silver coin of the imperial times.} \zh{民国之前的银币。}  ¶ \textcolor{darkblue}{\textbf{\ipa{ə˧mi˧! | pæ˧kʰwɤ˧ so˧-ɭɯ˥ ki˩-mæ˩!}}} \textcolor{Sepia}{\selectlanguage{english}Wow! [(S)he] is giving you three silver coins!! (According to the main consultant's memories, this is the type of comment that uncles and aunts would make when a child who turned 13 received significant amounts of money on the occasion of their coming of age. The equivalent today would be about half a month's salary. To give only one coin would not be right, because gifts have to come in pairs. To give two coins is fully sufficient: a beautiful gift. To give three coins is an impressive gift, beyond expectations.)} \zh{哇!(他)给三块银币!(在一个孩子成年时,亲戚会给银币。给一块,不合适,因为礼物不能只给一个,要给两个。给两块银币,是合适的,也是够的。给三块银币,超出期望,是大礼物了。按现在的标准/说法,三个银币等于半个月的工资左右。)}  
 ¶ \textcolor{darkblue}{\textbf{\ipa{pæ˧kʰwɤ˧ ɖɯ˧-ɭɯ˥\# ; pæ˧kʰwɤ˧ ɲi˧-ɭɯ˥\# ; pæ˧kʰwɤ˧ so˧-ɭɯ˥\#}}} \textcolor{Sepia}{\selectlanguage{english}one silver coin, two silver coins, three silver coins} \zh{一块银币,两块银币,三块银币}  
 ¶ \textcolor{darkblue}{\textbf{\ipa{pæ˧kʰwɤ˧ ɖɯ˧-ki˩tɑ˩}}} \textcolor{Sepia}{\selectlanguage{english}a bag of silver coins (to be interred in a secret place)} \zh{一包银币(埋在地里,为了藏)}  
 \zh{量词}: \textcolor{darkblue}{\textbf{\ipa{ɭɯ˧}}}  \mytextsc{clf}: \textcolor{darkblue}{\textbf{\ipa{ɭɯ˧}}} 
\lhead{\firstmark}
\rhead{\botmark}

\subsection{\hspace{-0.5cm} {\Large \textcolor{darkblue}{\textbf{\ipa{pæ˧li˩}}}}\hspace{0.5cm}[\kern2pt{\textcolor{darkblue}{\textbf{\ipa{pæ˧li˧}}}}\kern2pt]} \hypertarget{p\{\string_Mli\string_B1}{}
\markboth{\textcolor{darkblue}{\textbf{\ipa{pæ˧li˩}}}}{}
\textcolor{teal}{\mytextsc{noun}} \hspace{4pt} Tone: L\#.
\textcolor{Sepia}{\selectlanguage{english}Chinese chestnut.} \zh{板栗。}  Borrowing: Chinese  \zh{板栗}
 ¶ \textcolor{darkblue}{\textbf{\ipa{pæ˧li˩-si˩dzi˩}}} \textcolor{Sepia}{\selectlanguage{english}chestnut tree} \zh{板栗树}  
 ¶ \textcolor{darkblue}{\textbf{\ipa{pæ˧li˩-dzi˩}}} \textcolor{Sepia}{\selectlanguage{english}chestnut tree} \zh{板栗树}  

\lhead{\firstmark}
\rhead{\botmark}

\subsection{\hspace{-0.5cm} {\Large \textcolor{darkblue}{\textbf{\ipa{pæ˧ɻæ˩-ʈʂʰo˩}}}}\hspace{0.5cm}[\kern2pt{\textcolor{darkblue}{\textbf{\ipa{xxxx non-correspondance entre le nombre de morphèmes et le nombre de tons de morphèmes}}}}\kern2pt]} \hypertarget{p\{\string_Mr£`\{\string_B-t`s`\string_ho\string_B1}{}
\markboth{\textcolor{darkblue}{\textbf{\ipa{pæ˧ɻæ˩-ʈʂʰo˩}}}}{}
\textcolor{teal}{\mytextsc{noun}} \hspace{4pt} Tone: L\#-.
\textcolor{Sepia}{\selectlanguage{english}Hongqiao, a (mostly Han Chinese) village on the road from Ninglang to Yongning.} \zh{红桥。}  ¶ \textcolor{darkblue}{\textbf{\ipa{no˧ | pæ˧ɻæ˩ʈʂʰo˩-hĩ˩-ni˩-zo˩!}}} \textcolor{Sepia}{\selectlanguage{english}“You look like someone from Hongqiao!” This is an insult, meaning “You are ugly”. Popular Na geography had it that the people of Hongqiao (a village which the caravans crossed) had coarse, unlovely physical features, such as big snub noses.} \zh{解放前用的侮辱语句:“你像红桥人!”=“你很丑!”摩梭民间文化中,红桥(马帮路过的一个乡)的人被认为难看,面貌不“眉清目秀”,比如有扁鼻子。}  

\lhead{\firstmark}
\rhead{\botmark}

\subsection{\hspace{-0.5cm} {\Large \textcolor{darkblue}{\textbf{\ipa{pæ˧sɯ˧}}}}\hspace{0.5cm}[\kern2pt{\textcolor{darkblue}{\textbf{\ipa{xxxx non-correspondance entre le nombre de morphèmes et le nombre de tons de morphèmes}}}}\kern2pt]} \hypertarget{p\{\string_MsM\string_M1}{}
\markboth{\textcolor{darkblue}{\textbf{\ipa{pæ˧sɯ˧}}}}{}
\textcolor{teal}{\mytextsc{noun}} \hspace{4pt} Tone: M.
\textcolor{Sepia}{\selectlanguage{english}The lowest rank in the hierarchy of feudal officials.} \zh{把事(封建官员系统中的最低等级)(汉语借词)。}  Borrowing: Chinese  \zh{把事}

\lhead{\firstmark}
\rhead{\botmark}

\subsection{\hspace{-0.5cm} {\Large \textcolor{darkblue}{\textbf{\ipa{pæ˧te˩}}}}\hspace{0.5cm}[\kern2pt{\textcolor{darkblue}{\textbf{\ipa{pæ˧te˧}}}}\kern2pt]} \hypertarget{p\{\string_Mte\string_B1}{}
\markboth{\textcolor{darkblue}{\textbf{\ipa{pæ˧te˩}}}}{}
\textcolor{teal}{\mytextsc{noun}} \hspace{4pt} Tone: L\#.
\textcolor{Sepia}{\selectlanguage{english}Bench, stool.} \zh{板凳。}  Borrowing: Chinese  \zh{板凳}
 \zh{量词}: \textcolor{darkblue}{\textbf{\ipa{ɭɯ˧}}}  \mytextsc{clf}: \textcolor{darkblue}{\textbf{\ipa{ɭɯ˧}}} 
\lhead{\firstmark}
\rhead{\botmark}

\subsection{\hspace{-0.5cm} {\Large \textcolor{darkblue}{\textbf{\ipa{pæ˩\textsubscript{a}}}}}\hspace{0.5cm}[\kern2pt{\textcolor{darkblue}{\textbf{\ipa{pæ˥}}}}\kern2pt]} \hypertarget{p\{\string_Ba1}{}
\markboth{\textcolor{darkblue}{\textbf{\ipa{pæ˩\textsubscript{a}}}}}{}
\textcolor{teal}{\mytextsc{verb}} \hspace{4pt} Tone: M\textsubscript{a}.
\textcolor{Sepia}{\selectlanguage{english}To lay (the table).} \zh{摆桌子、供应饭菜。}  Borrowing: Chinese  \zh{摆?}
 ¶ \textcolor{darkblue}{\textbf{\ipa{hɑ˧ tʰi˧-pæ˩ tsæ˩-ɲi˩-ze˩! | hɑ˧ dzɯ˧-bi˧-ze˩!}}} \textcolor{Sepia}{\selectlanguage{english}The table is set / everything is ready! Let's eat!} \zh{饭摆好了!吃饭了!}  

\lhead{\firstmark}
\rhead{\botmark}

\subsection{\hspace{-0.5cm} {\Large \textcolor{darkblue}{\textbf{\ipa{pæ˩pʰæ˧˥}}} \textsubscript{1}}\hspace{0.5cm}[\kern2pt{\textcolor{darkblue}{\textbf{\ipa{pæ˧pʰæ˩}}}}\kern2pt]} \hypertarget{p\{\string_Bp\string_h\{\string_M\string_T1}{}
\markboth{\textcolor{darkblue}{\textbf{\ipa{pæ˩pʰæ˧˥}}} \textsubscript{1}}{}
\textcolor{teal}{\mytextsc{noun}} \hspace{4pt} Tone: LM+MH\#.
\ding{202} \textcolor{Sepia}{\selectlanguage{english}Thick wood plank. A well-prepared plank, used in construction, could last a hundred years.} \zh{厚的木板、 木板子。}  \zh{量词}: \textcolor{darkblue}{\textbf{\ipa{pʰæ˧˥}}} \ding{203} \textcolor{Sepia}{\selectlanguage{english}Harrow; the term is the same as that for 'plank', as the harrow essentially consisted in a large, squared piece of lumber, without teeth.} \zh{耙。}  \mytextsc{clf}: \textcolor{darkblue}{\textbf{\ipa{pʰæ˧˥}}} 
\lhead{\firstmark}
\rhead{\botmark}

\subsection{\hspace{-0.5cm} {\Large \textcolor{darkblue}{\textbf{\ipa{pæ˩pʰæ˧˥}}} \textsubscript{2}}\hspace{0.5cm}[\kern2pt{\textcolor{darkblue}{\textbf{\ipa{pæ˩pʰæ˧˥}}}}\kern2pt]} \hypertarget{p\{\string_Bp\string_h\{\string_M\string_T2}{}
\markboth{\textcolor{darkblue}{\textbf{\ipa{pæ˩pʰæ˧˥}}} \textsubscript{2}}{}
\textcolor{teal}{\mytextsc{noun}} \hspace{4pt} Tone: LM+MH\#.
\textcolor{Sepia}{\selectlanguage{english}Masculine given name.} \zh{男性名字。} 
\lhead{\firstmark}
\rhead{\botmark}

\subsection{\hspace{-0.5cm} {\Large \textcolor{darkblue}{\textbf{\ipa{pæ˧˥}}} \textsubscript{1}}\hspace{0.5cm}[\kern2pt{\textcolor{darkblue}{\textbf{\ipa{pæ˧˥}}}}\kern2pt]} \hypertarget{p\{\string_M\string_T1}{}
\markboth{\textcolor{darkblue}{\textbf{\ipa{pæ˧˥}}} \textsubscript{1}}{}
\textcolor{teal}{\mytextsc{verb}} \hspace{4pt} Tone: MH.
\textcolor{Sepia}{\selectlanguage{english}To cultivate land.} \zh{种(地)。} 
\lhead{\firstmark}
\rhead{\botmark}

\subsection{\hspace{-0.5cm} {\Large \textcolor{darkblue}{\textbf{\ipa{pæ˧˥}}} \textsubscript{2}}\hspace{0.5cm}[\kern2pt{\textcolor{darkblue}{\textbf{\ipa{pæ˧˥}}}}\kern2pt]} \hypertarget{p\{\string_M\string_T2}{}
\markboth{\textcolor{darkblue}{\textbf{\ipa{pæ˧˥}}} \textsubscript{2}}{}
\textcolor{teal}{\mytextsc{verb}} \hspace{4pt} Tone: MH.
\textcolor{Sepia}{\selectlanguage{english}To exceed; to let slip.} \zh{超过,错过。}  ¶ \textcolor{darkblue}{\textbf{\ipa{pæ˧˥ | -kʰɯ˩-pi˩, | mɤ˧-tsɤ˧! |}}} \textcolor{Sepia}{\selectlanguage{english}It's not good to let (an auspicious day) slip by! / It's not good to miss the opportunity (of an auspicious days; for the building of a house, for instance)} \zh{错过(一个吉日),不好!}  
 ¶ \textcolor{darkblue}{\textbf{\ipa{pæ˧˥ | -tʰɑ˧-kʰɯ˩}}} \textcolor{Sepia}{\selectlanguage{english}Don't let (this opportunity) slip by! / (You/we) mustn't miss this opportunity!} \zh{不要错过(机会)!}  
 ¶ \textcolor{darkblue}{\textbf{\ipa{le˧-pæ˧-ze˥!}}} \textcolor{Sepia}{\selectlanguage{english}It's too late! / We have let the opportunity slip by!} \zh{错过了!}  

\lhead{\firstmark}
\rhead{\botmark}

\subsection{\hspace{-0.5cm} {\Large \textcolor{darkblue}{\textbf{\ipa{pæ˧˥hwɤ˧}}}}\hspace{0.5cm}[\kern2pt{\textcolor{darkblue}{\textbf{\ipa{pæ˧hwɤ˧}}}}\kern2pt]} \hypertarget{p\{\string_M\string_Thw7\string_M1}{}
\markboth{\textcolor{darkblue}{\textbf{\ipa{pæ˧˥hwɤ˧}}}}{}
\textcolor{teal}{\mytextsc{noun}} \hspace{4pt} Tone: MH.M.
\textcolor{Sepia}{\selectlanguage{english}Solution, method (early borrowing from Chinese).} \zh{办法(早期汉语借词)。}  Borrowing: Chinese  \zh{办法}
 ¶ \textcolor{darkblue}{\textbf{\ipa{ʈʂʰɯ˧ | pæ˧˥hwɤ˧ | ɕjɤ˩ ɣɯ˧ (+ | ʐwæ˩˥)!}}} \textcolor{Sepia}{\selectlanguage{english}He/she is great at finding solutions / at handling all sorts of difficult situations!} \zh{他很会想办法的!}  
 \zh{量词}: \textcolor{darkblue}{\textbf{\ipa{kʰwɤ˥}}}  \mytextsc{clf}: \textcolor{darkblue}{\textbf{\ipa{kʰwɤ˥}}} 
\lhead{\firstmark}
\rhead{\botmark}

\subsection{\hspace{-0.5cm} {\Large \textcolor{darkblue}{\textbf{\ipa{pe˧ʂe˧}}}}\hspace{0.5cm}[\kern2pt{\textcolor{darkblue}{\textbf{\ipa{pe˧ʂe˧}}}}\kern2pt]} \hypertarget{pe\string_Ms`e\string_M1}{}
\markboth{\textcolor{darkblue}{\textbf{\ipa{pe˧ʂe˧}}}}{}
\textcolor{teal}{\mytextsc{adverb(ial)}} \hspace{4pt} Tone: M.
\textcolor{Sepia}{\selectlanguage{english}Itself, per se.} \zh{本身(汉语借词)。}  Borrowing: Chinese  \zh{本身}

\lhead{\firstmark}
\rhead{\botmark}

\subsection{\hspace{-0.5cm} {\Large \textcolor{darkblue}{\textbf{\ipa{pɤ˥}}}}\hspace{0.5cm}[\kern2pt{\textcolor{darkblue}{\textbf{\ipa{pɤ˥}}}}\kern2pt]} \hypertarget{p7\string_T1}{}
\markboth{\textcolor{darkblue}{\textbf{\ipa{pɤ˥}}}}{}
\textcolor{teal}{\mytextsc{noun}} \hspace{4pt} Tone: \#H.
\textcolor{Sepia}{\selectlanguage{english}Drawing, painting.} \zh{画。}  \zh{量词}: \textcolor{darkblue}{\textbf{\ipa{pɤ˥}}} \textcolor{darkblue}{\textbf{\ipa{pʰæ˧˥}}}  \mytextsc{clf}: \textcolor{darkblue}{\textbf{\ipa{pɤ˥}}} \textcolor{darkblue}{\textbf{\ipa{pʰæ˧˥}}} 
\lhead{\firstmark}
\rhead{\botmark}

\subsection{\hspace{-0.5cm} {\Large \textcolor{darkblue}{\textbf{\ipa{pɤ˥}}}}\hspace{0.5cm}[\kern2pt{\textcolor{darkblue}{\textbf{\ipa{pɤ˥}}}}\kern2pt]} \hypertarget{p7\string_T1}{}
\markboth{\textcolor{darkblue}{\textbf{\ipa{pɤ˥}}}}{}
\textcolor{teal}{\mytextsc{verb}} \hspace{4pt} Tone: H.
\textcolor{Sepia}{\selectlanguage{english}To curl up; to hunch, to huddle up.} \zh{蜷曲、蜷缩。}  ¶ \textcolor{darkblue}{\textbf{\ipa{æ˩ ʈʂʰɯ˧-mi˥ | si˧dzi˩-ʈʰæ˩qo˩ | tʰi˧-pɤ˥-dʑo˩!}}} \textcolor{Sepia}{\selectlanguage{english}The hen has huddled up under a tree!} \zh{那只鸡,在树下蜷缩着!}  
 ¶ \textcolor{darkblue}{\textbf{\ipa{ʈʂʰɯ˧-qo˧ ɖɯ˧-pɤ˥ ɕjɤ˩-ɻ̍˩!}}} \textcolor{Sepia}{\selectlanguage{english}Come and lay here (for a rest)!} \zh{过来这边躺一下!}  

\lhead{\firstmark}
\rhead{\botmark}

\subsection{\hspace{-0.5cm} {\Large \textcolor{darkblue}{\textbf{\ipa{pɤ˥\textsubscript{b}}}}}\hspace{0.5cm}[\kern2pt{\textcolor{darkblue}{\textbf{\ipa{pɤ˥}}}}\kern2pt]} \hypertarget{p7\string_Tb1}{}
\markboth{\textcolor{darkblue}{\textbf{\ipa{pɤ˥\textsubscript{b}}}}}{}
\textcolor{teal}{\mytextsc{classifier}} \hspace{4pt} Tone: H\textsubscript{b}.
\textcolor{Sepia}{\selectlanguage{english}Classifier for statues, paintings...} \zh{量词:雕像,如:佛像(一尊)。} 
\lhead{\firstmark}
\rhead{\botmark}

\subsection{\hspace{-0.5cm} {\Large \textcolor{darkblue}{\textbf{\ipa{pɤ˧\textsubscript{a}}}}}\hspace{0.5cm}[\kern2pt{\textcolor{darkblue}{\textbf{\ipa{pɤ˥}}}}\kern2pt]} \hypertarget{p7\string_Ma1}{}
\markboth{\textcolor{darkblue}{\textbf{\ipa{pɤ˧\textsubscript{a}}}}}{}
\textcolor{teal}{\mytextsc{verb}} \hspace{4pt} Tone: M\textsubscript{a}.
\textcolor{Sepia}{\selectlanguage{english}To carry on one's back.} \zh{背(水、柴、孩子……)。}  ¶ \textcolor{darkblue}{\textbf{\ipa{pɤ˧\textasciitilde{}pɤ˧}}} \textcolor{Sepia}{\selectlanguage{english}\mytextsc{red}} \zh{\mytextsc{重叠:背一背}}  
 ¶ \textcolor{darkblue}{\textbf{\ipa{tʰi˧-pɤ˥\textasciitilde{}pɤ˩}}} \textcolor{Sepia}{\selectlanguage{english}\mytextsc{dur} \mytextsc{red}} \zh{背一背}  
 ¶ \textcolor{darkblue}{\textbf{\ipa{qʰæ˧ pɤ˧\textasciitilde{}pɤ˥}}} \textcolor{Sepia}{\selectlanguage{english}to carry manure} \zh{背肥料}  
 ¶ \textcolor{darkblue}{\textbf{\ipa{kʰɤ˧ pɤ˧\textasciitilde{}pɤ˥}}} \textcolor{Sepia}{\selectlanguage{english}to carry a dorsal basket} \zh{背背篓}  
 ¶ \textcolor{darkblue}{\textbf{\ipa{zɯ˧ pɤ˧\textasciitilde{}pɤ˥}}} \textcolor{Sepia}{\selectlanguage{english}to carry grass} \zh{背草}  
 ¶ \textcolor{darkblue}{\textbf{\ipa{tso˧\textasciitilde{}tso˧ pɤ˧\textasciitilde{}pɤ˥}}} \textcolor{Sepia}{\selectlanguage{english}to carry things} \zh{背东西}  
 ¶ \textcolor{darkblue}{\textbf{\ipa{*tso˧\textasciitilde{}tso˧ pɤ˩}}} \textcolor{Sepia}{\selectlanguage{english}to carry things (this expression is well-formed syntactically, but apparently not in use)} \zh{背东西(语法上,这个短语没有问题,但发音合作人不那么说。)}  
 ¶ \textcolor{darkblue}{\textbf{\ipa{njɤ˧-ɳɯ˧ pɤ˧\textasciitilde{}pɤ˩ (+bi˩)!}}} \textcolor{Sepia}{\selectlanguage{english}I'll do the carrying! / Let me carry (it)!} \zh{我来背!}  
 ¶ \textcolor{darkblue}{\textbf{\ipa{dʑɯ˩ pɤ˩\textasciitilde{}pɤ˥}}} \textcolor{Sepia}{\selectlanguage{english}to carry water} \zh{背水}  
 ¶ \textcolor{darkblue}{\textbf{\ipa{zo˧mv̩˥ pɤ˩\textasciitilde{}pɤ˩}}} \textcolor{Sepia}{\selectlanguage{english}to carry a child on the back} \zh{背孩子}  

\lhead{\firstmark}
\rhead{\botmark}

\subsection{\hspace{-0.5cm} {\Large \textcolor{darkblue}{\textbf{\ipa{pɤ˧dʑɤ˩-di˩}}}}\hspace{0.5cm}[\kern2pt{\textcolor{darkblue}{\textbf{\ipa{pɤ˧dʑɤ˩di˧}}}}\kern2pt]} \hypertarget{p7\string_Mdz£7\string_B-di\string_B1}{}
\markboth{\textcolor{darkblue}{\textbf{\ipa{pɤ˧dʑɤ˩-di˩}}}}{}
\textcolor{teal}{\mytextsc{noun}} \hspace{4pt} Tone: L\#-.
\textcolor{Sepia}{\selectlanguage{english}A village close to the Hot Springs.} \zh{温泉乡的一个村落。}  ¶ \textcolor{darkblue}{\textbf{\ipa{ə˧go˧-ʁwɤ˧, | ʁwɤ˧lɑ˩-bi˩, | bæ˧ʁwɤ˧, | tʰo˧tsʰe\#˥, | pi˧tsʰe˩-di˩, | pɤ˧dʑɤ˩-di˩, | ʁwɤ˧tv̩˧}}} \textcolor{Sepia}{\selectlanguage{english}Villages that one encounters as one leaves the plain of Yongning (away from the Lake); the first two are perceived as villages with a high proportion of Na members, and the third as a mostly Na village, whereas the next ones are Pumi (Prinmi).} \zh{永宁背向泸沽湖方向经过的村落。前两个村落拥有相当大的摩梭人口比例,第三个村落是摩梭村,最后一个是普米村。}  

\lhead{\firstmark}
\rhead{\botmark}

\subsection{\hspace{-0.5cm} {\Large \textcolor{darkblue}{\textbf{\ipa{pɤ˧lɑ˩}}}}\hspace{0.5cm}[\kern2pt{\textcolor{darkblue}{\textbf{\ipa{xxxx non-correspondance entre le nombre de morphèmes et le nombre de tons de morphèmes}}}}\kern2pt]} \hypertarget{p7\string_MlA\string_B1}{}
\markboth{\textcolor{darkblue}{\textbf{\ipa{pɤ˧lɑ˩}}}}{}
\textcolor{teal}{\mytextsc{noun}} \hspace{4pt} Tone: L\#.
\textcolor{Sepia}{\selectlanguage{english}Photo, photography (newly coined word).} \zh{相片,照片。}  \zh{量词}: \textcolor{darkblue}{\textbf{\ipa{pʰæ˧˥}}}  \mytextsc{clf}: \textcolor{darkblue}{\textbf{\ipa{pʰæ˧˥}}} 
\lhead{\firstmark}
\rhead{\botmark}

\subsection{\hspace{-0.5cm} {\Large \textcolor{darkblue}{\textbf{\ipa{pɤ˧ʁɑ˧}}}}\hspace{0.5cm}[\kern2pt{\textcolor{darkblue}{\textbf{\ipa{pɤ˧ʁɑ˧}}}}\kern2pt]} \hypertarget{p7\string_MRA\string_M1}{}
\markboth{\textcolor{darkblue}{\textbf{\ipa{pɤ˧ʁɑ˧}}}}{}
\textcolor{teal}{\mytextsc{classifier}} \hspace{4pt} Tone: M.
\textcolor{Sepia}{\selectlanguage{english}A big step.} \zh{量词:一大步。}  ¶ \textcolor{darkblue}{\textbf{\ipa{ɖɯ˧-pɤ˧ʁɑ˧\textasciitilde{}ɖɯ˧-pɤ˧ʁɑ˧}}} \textcolor{Sepia}{\selectlanguage{english}with great strides} \zh{大步流星地}  
 ¶ \textcolor{darkblue}{\textbf{\ipa{ɲi˧-pɤ˧ʁɑ˧}}} \textcolor{Sepia}{\selectlanguage{english}two great strides} \zh{两大步}  

\lhead{\firstmark}
\rhead{\botmark}

\subsection{\hspace{-0.5cm} {\Large \textcolor{darkblue}{\textbf{\ipa{‑pɤ˧to˩}}}}\hspace{0.5cm}[\kern2pt{\textcolor{darkblue}{\textbf{\ipa{pɤ˧to˩}}}}\kern2pt]} \hypertarget{‑p7\string_Mto\string_B1}{}
\markboth{\textcolor{darkblue}{\textbf{\ipa{‑pɤ˧to˩}}}}{}
\textcolor{teal}{\mytextsc{conjunction}} \hspace{4pt} Tone: L\#.
\textcolor{Sepia}{\selectlanguage{english}Even.} \zh{连。}  ¶ \textcolor{darkblue}{\textbf{\ipa{ʈʂʰɯ˧ | li˩-pɤ˥to˩ | ʈʰɯ˩-ɲi˥!}}} \textcolor{Sepia}{\selectlanguage{english}She even drinks tea! (About the eating and drinking habits of a one-year-old child)} \zh{她连茶都喝!(关于一个一岁孩子的饮食习惯)}  
 ¶ \textcolor{darkblue}{\textbf{\ipa{ʈʂʰɯ˧ | pɤ˩jɤ˧-pɤ˥to˩ | dzɯ˩-ɲi˥!}}} \textcolor{Sepia}{\selectlanguage{english}She even eats bread! (About the eating and drinking habits of a one-year-old child)} \zh{她连面包都吃!}  
 ¶ \textcolor{darkblue}{\textbf{\ipa{hæ˧, | kʰv̩˩mi˩-ʂe˩-pɤ˥to˩ dzɯ˩-kv̩˩!}}} \textcolor{Sepia}{\selectlanguage{english}The (Han) Chinese even eat dog meat! (Note: consumption of dog meat is forbidden in Na culture)} \zh{汉族连狗肉都吃!(注:摩梭人不吃狗肉)}  
 ¶ \textcolor{darkblue}{\textbf{\ipa{hæ˧, | kʰv̩˩mi˩-ʂe˩˥ F dzɯ˩-kv̩˩!}}} \textcolor{Sepia}{\selectlanguage{english}as above} \zh{同上}  
 ¶ \textcolor{darkblue}{\textbf{\ipa{bo˩-pɤ˥to˩; lɑ˧-pɤ˧to˩; mv̩˩-pɤ˥to˩; ʐwæ˧-pɤ˧to˩; ɬi˧mi˧-pɤ˧to˩; ɲi˧mi˧-pɤ˧to˩; hwɤ˧li˧-pɤ˥-to˩; hwɤ˧mi˧-pɤ˥to˩; kʰv̩˩mi˩-pɤ˥-to˩; ʁo˧dzi˩-pɤ˩to˩; ʝi˩ʈʂæ˧-pɤ˥to˩; nɑ˩hĩ˧-pɤ˧to˩; bo˩mi˧-pɤ˧to˩; bo˩ɬɑ˧-pɤ˩to˩; ʁæ˧ʈv̩˥-pɤ˩to˩}}} \textcolor{Sepia}{\selectlanguage{english}combinations with nouns of the various tone categories} \zh{与不同声调类的名词结合}  

\lhead{\firstmark}
\rhead{\botmark}

\subsection{\hspace{-0.5cm} {\Large \textcolor{darkblue}{\textbf{\ipa{pɤ˧tv̩˥}}}}\hspace{0.5cm}[\kern2pt{\textcolor{darkblue}{\textbf{\ipa{pɤ˧tv̩˥}}}}\kern2pt]} \hypertarget{p7\string_Mtv\string_=\string_T1}{}
\markboth{\textcolor{darkblue}{\textbf{\ipa{pɤ˧tv̩˥}}}}{}
\textcolor{teal}{\mytextsc{noun}} \hspace{4pt} Tone: H\#.
\textcolor{Sepia}{\selectlanguage{english}Wickerwork basket.} \zh{篮子、竹篮。}  ¶ \textcolor{darkblue}{\textbf{\ipa{ɖʐɯ˧ʂɯ˥-pɤ˩tv̩˩}}} \textcolor{Sepia}{\selectlanguage{english}small basket where chopsticks are kept} \zh{筷子篮}  

\lhead{\firstmark}
\rhead{\botmark}

\subsection{\hspace{-0.5cm} {\Large \textcolor{darkblue}{\textbf{\ipa{pɤ˧tʰi˩}}}}\hspace{0.5cm}[\kern2pt{\textcolor{darkblue}{\textbf{\ipa{pɤ˧tʰi˩}}}}\kern2pt]} \hypertarget{p7\string_Mt\string_hi\string_B1}{}
\markboth{\textcolor{darkblue}{\textbf{\ipa{pɤ˧tʰi˩}}}}{}
\textcolor{teal}{\mytextsc{noun}} \hspace{4pt} Tone: L\#.
\textcolor{Sepia}{\selectlanguage{english}A family name from Yongning. There are two families in Yongning that carry this name.} \zh{一个姓。这个姓,永宁有两家。}  ¶ \textcolor{darkblue}{\textbf{\ipa{pɤ˧tʰi˩=ɻ̍˩}}} \textcolor{Sepia}{\selectlanguage{english}the \textcolor{darkblue}{\textbf{\ipa{/pɤ˧tʰi˩/}}} clan, the \textcolor{darkblue}{\textbf{\ipa{/pɤ˧tʰi˩/}}} family} \zh{\textcolor{darkblue}{\textbf{\ipa{/pɤ˧tʰi˩/}}}家族}  

\lhead{\firstmark}
\rhead{\botmark}

\subsection{\hspace{-0.5cm} {\Large \textcolor{darkblue}{\textbf{\ipa{pɤ˩\textsubscript{a}}}}}\hspace{0.5cm}[\kern2pt{\textcolor{darkblue}{\textbf{\ipa{pɤ˩˥}}}}\kern2pt]} \hypertarget{p7\string_Ba1}{}
\markboth{\textcolor{darkblue}{\textbf{\ipa{pɤ˩\textsubscript{a}}}}}{}
\textcolor{teal}{\mytextsc{verb}} \hspace{4pt} Tone: L\textsubscript{a}.
\textcolor{Sepia}{\selectlanguage{english}To come out, to emerge, to appear.} \zh{出现、出来、浮现。}  ¶ \textcolor{darkblue}{\textbf{\ipa{dʑɯ˩ pɤ˩˥}}} \textcolor{Sepia}{\selectlanguage{english}some water comes out} \zh{涌出水来}  
 ¶ \textcolor{darkblue}{\textbf{\ipa{dʑɯ˧qʰv̩˧-qo˧ | dʑɯ˩ pɤ˩-ze˥}}} \textcolor{Sepia}{\selectlanguage{english}Water emerges at the source.} \zh{水泉里面,涌出水来。}  
 ¶ \textcolor{darkblue}{\textbf{\ipa{tʰi˧-pɤ˩-dʑo˩}}} \textcolor{Sepia}{\selectlanguage{english}\mytextsc{dur} \string_ \mytextsc{prog}: it is emerging} \zh{正在涌出水来}  
 ¶ \textcolor{darkblue}{\textbf{\ipa{gɤ˩-pɤ˥}}} \textcolor{Sepia}{\selectlanguage{english}to emerge, to come up, to appear (e.g. the sun comes out)} \zh{出现、上来:太阳出来}  

\lhead{\firstmark}
\rhead{\botmark}

\subsection{\hspace{-0.5cm} {\Large \textcolor{darkblue}{\textbf{\ipa{pɤ˩\textsubscript{b}}}}}\hspace{0.5cm}[\kern2pt{\textcolor{darkblue}{\textbf{\ipa{pɤ˩˥}}}}\kern2pt]} \hypertarget{p7\string_Bb1}{}
\markboth{\textcolor{darkblue}{\textbf{\ipa{pɤ˩\textsubscript{b}}}}}{}
\textcolor{teal}{\mytextsc{classifier}} \hspace{4pt} Tone: L\textsubscript{b}.
\textcolor{Sepia}{\selectlanguage{english}Classifier for ladders, doors….} \zh{量词:木工件,如梯子、门等等(一扇门,一把梯子)。} 
\lhead{\firstmark}
\rhead{\botmark}

\subsection{\hspace{-0.5cm} {\Large \textcolor{darkblue}{\textbf{\ipa{pɤ˩dʑɯ˩}}}}\hspace{0.5cm}[\kern2pt{\textcolor{darkblue}{\textbf{\ipa{pɤ˩dʑɯ˩˥}}}}\kern2pt]} \hypertarget{p7\string_Bdz£M\string_B1}{}
\markboth{\textcolor{darkblue}{\textbf{\ipa{pɤ˩dʑɯ˩}}}}{}
\textcolor{teal}{\mytextsc{noun}} \hspace{4pt} Tone: L.
\textcolor{Sepia}{\selectlanguage{english}Spring water.} \zh{泉水。} 
\lhead{\firstmark}
\rhead{\botmark}

\subsection{\hspace{-0.5cm} {\Large \textcolor{darkblue}{\textbf{\ipa{pɤ˩-ho˩\textasciitilde{}ho˥}}}}\hspace{0.5cm}[\kern2pt{\textcolor{darkblue}{\textbf{\ipa{xxxx non-correspondance entre le nombre de morphèmes et le nombre de tons de morphèmes}}}}\kern2pt]} \hypertarget{p7\string_B-ho\string_B~ho\string_T1}{}
\markboth{\textcolor{darkblue}{\textbf{\ipa{pɤ˩-ho˩\textasciitilde{}ho˥}}}}{}
\textcolor{teal}{\mytextsc{adjective}} \hspace{4pt} Tone: L+H\#.
\textcolor{Sepia}{\selectlanguage{english}Soft.} \zh{柔软。}  ¶ \textcolor{darkblue}{\textbf{\ipa{pɤ˩-ho˩\textasciitilde{}ho˥-gv̩˩}}} \textcolor{Sepia}{\selectlanguage{english}soft} \zh{柔软}  
 ¶ \textcolor{darkblue}{\textbf{\ipa{ʁo˧qʰwɤ˩ | pɤ˩-ho˩\textasciitilde{}ho˥-gv̩˩-hĩ˩ | tʰv̩˧-kʰwɤ˥}}} \textcolor{Sepia}{\selectlanguage{english}the place where the head is soft =the fontanel} \zh{头上软软的那块 =囟门}  

\lhead{\firstmark}
\rhead{\botmark}

\subsection{\hspace{-0.5cm} {\Large \textcolor{darkblue}{\textbf{\ipa{pɤ˩jɤ˧bv̩˥-di˩}}}}\hspace{0.5cm}[\kern2pt{\textcolor{darkblue}{\textbf{\ipa{xxxx non-correspondance entre le nombre de morphèmes et le nombre de tons de morphèmes}}}}\kern2pt]} \hypertarget{p7\string_Bj7\string_Mbv\string_=\string_T-di\string_B1}{}
\markboth{\textcolor{darkblue}{\textbf{\ipa{pɤ˩jɤ˧bv̩˥-di˩}}}}{}
\textcolor{teal}{\mytextsc{noun}} \hspace{4pt} Tone: LM+\#H-.
\textcolor{Sepia}{\selectlanguage{english}Steamer used for bread (buns).} \zh{用来蒸面团(馒头等等)的蒸笼。}  \zh{量词}: \textcolor{darkblue}{\textbf{\ipa{ɭɯ˧}}}  \mytextsc{clf}: \textcolor{darkblue}{\textbf{\ipa{ɭɯ˧}}} 
\lhead{\firstmark}
\rhead{\botmark}

\subsection{\hspace{-0.5cm} {\Large \textcolor{darkblue}{\textbf{\ipa{pɤ˩jɤ˧˥}}}}\hspace{0.5cm}[\kern2pt{\textcolor{darkblue}{\textbf{\ipa{pɤ˩jɤ˧˥}}}}\kern2pt]} \hypertarget{p7\string_Bj7\string_M\string_T1}{}
\markboth{\textcolor{darkblue}{\textbf{\ipa{pɤ˩jɤ˧˥}}}}{}
\textcolor{teal}{\mytextsc{noun}} \hspace{4pt} Tone: LM+MH\#.
\ding{202} \textcolor{Sepia}{\selectlanguage{english}Dough for making steamed bread.} \zh{做面包的面团(可以蒸成馒头)。}  \zh{量词}: \textcolor{darkblue}{\textbf{\ipa{jɤ˧˥}}} \ding{203} \textcolor{Sepia}{\selectlanguage{english}Round flat cake.} \zh{饼。}  ¶ \textcolor{darkblue}{\textbf{\ipa{li˩-pɤ˥jɤ˩ | ɖɯ˧-ɭɯ˧}}} \textcolor{Sepia}{\selectlanguage{english}a piece of brick tea, a brick of tea (tea leaves pressed into the shape of a round flat cake)} \zh{一块茶饼}  
 ¶ \textcolor{darkblue}{\textbf{\ipa{ɕi˧ʈʂʰwæ˧-pɤ˩jɤ˩}}} \textcolor{Sepia}{\selectlanguage{english}rice cake} \zh{米饼}  
 ¶ \textcolor{darkblue}{\textbf{\ipa{dze˧ɭɯ˧-pɤ˩jɤ˩}}} \textcolor{Sepia}{\selectlanguage{english}wheat cake} \zh{小麦饼}  
 ¶ \textcolor{darkblue}{\textbf{\ipa{qʰɑ˧dze˧-pɤ˩jɤ˩}}} \textcolor{Sepia}{\selectlanguage{english}sweetcorn cake} \zh{玉米饼}  
 ¶ \textcolor{darkblue}{\textbf{\ipa{tsʰi˧zi˧-pɤ˥jɤ˩}}} \textcolor{Sepia}{\selectlanguage{english}highland barley cake} \zh{青稞饼}  
 ¶ \textcolor{darkblue}{\textbf{\ipa{jɤ˧gɯ˩-pɤ˩jɤ˩}}} \textcolor{Sepia}{\selectlanguage{english}buckwheat cake} \zh{甜荞饼}  
 ¶ \textcolor{darkblue}{\textbf{\ipa{jɤ˧qʰɑ˧-pɤ˥jɤ˩}}} \textcolor{Sepia}{\selectlanguage{english}bitter buckwheat cake} \zh{苦荞饼}  
 \zh{量词}: \textcolor{darkblue}{\textbf{\ipa{jɤ˧˥}}}  \mytextsc{clf}: \textcolor{darkblue}{\textbf{\ipa{jɤ˧˥}}} \textcolor{darkblue}{\textbf{\ipa{jɤ˧˥}}} 
\lhead{\firstmark}
\rhead{\botmark}

\subsection{\hspace{-0.5cm} {\Large \textcolor{darkblue}{\textbf{\ipa{pɤ˩lv̩˩}}}}\hspace{0.5cm}[\kern2pt{\textcolor{darkblue}{\textbf{\ipa{pɤ˧lv̩˧}}}}\kern2pt]} \hypertarget{p7\string_Blv\string_=\string_B1}{}
\markboth{\textcolor{darkblue}{\textbf{\ipa{pɤ˩lv̩˩}}}}{}
\textcolor{teal}{\mytextsc{noun}} \hspace{4pt} Tone: L.
\textcolor{Sepia}{\selectlanguage{english}Nape.} \zh{项背 、项、脖颈儿。}  \zh{量词}: \textcolor{darkblue}{\textbf{\ipa{ɭɯ˧}}}  \mytextsc{clf}: \textcolor{darkblue}{\textbf{\ipa{ɭɯ˧}}} 
\lhead{\firstmark}
\rhead{\botmark}

\subsection{\hspace{-0.5cm} {\Large \textcolor{darkblue}{\textbf{\ipa{pɤ˩lv̩˧}}}}\hspace{0.5cm}[\kern2pt{\textcolor{darkblue}{\textbf{\ipa{pɤ˩lv̩˩˥}}}}\kern2pt]} \hypertarget{p7\string_Blv\string_=\string_M1}{}
\markboth{\textcolor{darkblue}{\textbf{\ipa{pɤ˩lv̩˧}}}}{}
\textcolor{teal}{\mytextsc{noun}} \hspace{4pt} Tone: LM.
\textcolor{Sepia}{\selectlanguage{english}Warehouse, storehouse: a one-floor building, opposite the main building (\textcolor{darkblue}{\textbf{\ipa{/ʑi˧mi˧/}}}); it is used for storing objects, such as the ard, and preserved meat.} \zh{仓库:主屋对面的房子,只有一层。用来收藏大工具,例如犁,或者腊肉。} 
\lhead{\firstmark}
\rhead{\botmark}

\subsection{\hspace{-0.5cm} {\Large \textcolor{darkblue}{\textbf{\ipa{pɤ˩mi˩}}}}\hspace{0.5cm}[\kern2pt{\textcolor{darkblue}{\textbf{\ipa{pɤ˩mi˥}}}}\kern2pt]} \hypertarget{p7\string_Bmi\string_B1}{}
\markboth{\textcolor{darkblue}{\textbf{\ipa{pɤ˩mi˩}}}}{}
\textcolor{teal}{\mytextsc{noun}} \hspace{4pt} Tone: L.
\textcolor{Sepia}{\selectlanguage{english}Frog.} \zh{青蛙。}  ¶ \textcolor{darkblue}{\textbf{\ipa{pɤ˩mi˩-pɤ˥pʰv̩˩}}} \textcolor{Sepia}{\selectlanguage{english}female frog and male frog} \zh{母青蛙与公青蛙}  
 ¶ \textcolor{darkblue}{\textbf{\ipa{pɤ˩mi˩-ʝi˥pʰv̩˩}}} \textcolor{Sepia}{\selectlanguage{english}A species of large frog or toad, which is abundant in the Yongning plain. This is one of three species distinguished by the consultant. It is not eaten by the Naxi (nor by the Na, who do not eat any sort of frog). This is the term used for \textit{Kaloula verrucosa} and \textit{Rana chaochiaoensis}.} \zh{一种大青蛙,在永宁坝子很常见。这是发音合作人认识的三种蛙之一。纳西族人不吃这种动物(摩梭人不吃任何蛙类动物)。}  
 ¶ \textcolor{darkblue}{\textbf{\ipa{pɤ˩mi˩-ʝi˥pʰv̩˩-mi˩}}} \textcolor{Sepia}{\selectlanguage{english}same meaning} \zh{同上}  
 ¶ \textcolor{darkblue}{\textbf{\ipa{hæ̃˧ʂɯ˩-pɤ˩mi˩}}} \textcolor{Sepia}{\selectlanguage{english}A beautiful species of frog, with a long body. It is only found in the forest, on the mountain. This is the second of three species distinguished by the consultant.} \zh{一种很美的青蛙,身体很长。只出现在山上森林里。这是发音合作人认识的第二种青蛙。}  
 ¶ \textcolor{darkblue}{\textbf{\ipa{dʑɯ˩-pɤ˩mi˩˥}}} \textcolor{Sepia}{\selectlanguage{english}A species of frog with a small head and large eyes, considered by the consultant as spending most of the time in the water. This is the third of three species of frogs distinguished by the consultant. The Naxi hunt it, especially in the fifth month.} \zh{一种青蛙,头小、眼睛大。这是发音合作人认识的第三种青蛙。纳西族吃这种青蛙。}  
 ¶ \textcolor{darkblue}{\textbf{\ipa{nɑ˩hĩ˥ | pɤ˧-ʂe˧ dzɯ˧; | pɤ˧-ɣɯ˧ | ɬɑ˧tɑ˥ mv̩˩! | pɤ˧-mæ˧, | bæ˧ʈʂo˥ ʝi˩!}}} \textcolor{Sepia}{\selectlanguage{english}Proverb: “The Naxi eat frog meat; they wear vests made of frog skin; and they make brooms with frog tails!”} \zh{谚语:“纳西人吃青蛙,披青蛙皮衣,蛙尾巴当扫帚!”}  
 \zh{量词}: \textcolor{darkblue}{\textbf{\ipa{mi˩}}}  \mytextsc{clf}: \textcolor{darkblue}{\textbf{\ipa{mi˩}}} 
\lhead{\firstmark}
\rhead{\botmark}

\subsection{\hspace{-0.5cm} {\Large \textcolor{darkblue}{\textbf{\ipa{pɤ˩pʰv̩˩}}}}\hspace{0.5cm}[\kern2pt{\textcolor{darkblue}{\textbf{\ipa{pɤ˧pʰv̩˧˥}}}}\kern2pt]} \hypertarget{p7\string_Bp\string_hv\string_=\string_B1}{}
\markboth{\textcolor{darkblue}{\textbf{\ipa{pɤ˩pʰv̩˩}}}}{}
\textcolor{teal}{\mytextsc{noun}} \hspace{4pt} Tone: L.
\textcolor{Sepia}{\selectlanguage{english}Male frog.} \zh{公青蛙。}  \zh{量词}: \textcolor{darkblue}{\textbf{\ipa{mi˩}}}  \mytextsc{clf}: \textcolor{darkblue}{\textbf{\ipa{mi˩}}} 
\lhead{\firstmark}
\rhead{\botmark}

\subsection{\hspace{-0.5cm} {\Large \textcolor{darkblue}{\textbf{\ipa{pɤ˩ti\#˥}}}}\hspace{0.5cm}[\kern2pt{\textcolor{darkblue}{\textbf{\ipa{pɤ˩ti˥}}}}\kern2pt]} \hypertarget{p7\string_Bti\#\string_T1}{}
\markboth{\textcolor{darkblue}{\textbf{\ipa{pɤ˩ti\#˥}}}}{}
\textcolor{teal}{\mytextsc{noun}} \hspace{4pt} Tone: LM+\#H.
\textcolor{Sepia}{\selectlanguage{english}Stool, small bench.} \zh{凳子。}  \zh{量词}: \textcolor{darkblue}{\textbf{\ipa{ɭɯ˧}}}  \mytextsc{clf}: \textcolor{darkblue}{\textbf{\ipa{ɭɯ˧}}} 
\lhead{\firstmark}
\rhead{\botmark}

\subsection{\hspace{-0.5cm} {\Large \textcolor{darkblue}{\textbf{\ipa{pɤ˩tɕɯ˧-pɤ˥mi˩}}}}\hspace{0.5cm}[\kern2pt{\textcolor{darkblue}{\textbf{\ipa{pɤ˩tɕɯ˧pɤ˥mi˩}}}}\kern2pt]} \hypertarget{p7\string_Bts£M\string_M-p7\string_Tmi\string_B1}{}
\markboth{\textcolor{darkblue}{\textbf{\ipa{pɤ˩tɕɯ˧-pɤ˥mi˩}}}}{}
\textcolor{teal}{\mytextsc{noun}} \hspace{4pt} Tone: LM+\#H-.
\textcolor{Sepia}{\selectlanguage{english}Tadpole.} \zh{蝌蚪。}  \zh{量词}: \textcolor{darkblue}{\textbf{\ipa{mi˩}}}  \mytextsc{clf}: \textcolor{darkblue}{\textbf{\ipa{mi˩}}} \textit{See:} \textcolor{darkblue}{\textbf{\ipa{pɤ˩tɕɯ˧˥, pɤ˩tɕɯ˧-ʁo˧ɖɯ˧˥}}} 
\lhead{\firstmark}
\rhead{\botmark}

\subsection{\hspace{-0.5cm} {\Large \textcolor{darkblue}{\textbf{\ipa{pɤ˩tɕɯ˧-ʁo˧ɖɯ˧˥}}}}\hspace{0.5cm}[\kern2pt{\textcolor{darkblue}{\textbf{\ipa{xxxx non-correspondance entre le nombre de morphèmes et le nombre de tons de morphèmes}}}}\kern2pt]} \hypertarget{p7\string_Bts£M\string_M-Ro\string_Md`M\string_M\string_T1}{}
\markboth{\textcolor{darkblue}{\textbf{\ipa{pɤ˩tɕɯ˧-ʁo˧ɖɯ˧˥}}}}{}
\textcolor{teal}{\mytextsc{noun}} \hspace{4pt} Tone: LM+MH\#.
\textcolor{Sepia}{\selectlanguage{english}Tadpole.} \zh{蝌蚪。}  \zh{量词}: \textcolor{darkblue}{\textbf{\ipa{mi˩}}}  \mytextsc{clf}: \textcolor{darkblue}{\textbf{\ipa{mi˩}}} \textit{See:} \textcolor{darkblue}{\textbf{\ipa{pɤ˩tɕɯ˧˥, pɤ˩tɕɯ˧-pɤ˥mi˩}}} 
\lhead{\firstmark}
\rhead{\botmark}

\subsection{\hspace{-0.5cm} {\Large \textcolor{darkblue}{\textbf{\ipa{pɤ˩tɕɯ˧˥}}}}\hspace{0.5cm}[\kern2pt{\textcolor{darkblue}{\textbf{\ipa{pɤ˩tɕɯ˧˥}}}}\kern2pt]} \hypertarget{p7\string_Bts£M\string_M\string_T1}{}
\markboth{\textcolor{darkblue}{\textbf{\ipa{pɤ˩tɕɯ˧˥}}}}{}
\textcolor{teal}{\mytextsc{noun}} \hspace{4pt} Tone: LM+MH\#.
\textcolor{Sepia}{\selectlanguage{english}Tadpole.} \zh{蝌蚪。}  \zh{量词}: \textcolor{darkblue}{\textbf{\ipa{mi˩}}}  \mytextsc{clf}: \textcolor{darkblue}{\textbf{\ipa{mi˩}}} \textit{See:} \textcolor{darkblue}{\textbf{\ipa{pɤ˩tɕɯ˧-ʁo˧ɖɯ˧˥, pɤ˩tɕɯ˧-pɤ˥mi˩}}} 
\lhead{\firstmark}
\rhead{\botmark}

\subsection{\hspace{-0.5cm} {\Large \textcolor{darkblue}{\textbf{\ipa{pɤ˧˥}}}}\hspace{0.5cm}[\kern2pt{\textcolor{darkblue}{\textbf{\ipa{pɤ˧˥}}}}\kern2pt]} \hypertarget{p7\string_M\string_T1}{}
\markboth{\textcolor{darkblue}{\textbf{\ipa{pɤ˧˥}}}}{}
\textcolor{teal}{\mytextsc{verb}} \hspace{4pt} Tone: MH.
\textcolor{Sepia}{\selectlanguage{english}To harrow.} \zh{耙地。}  ¶ \textcolor{darkblue}{\textbf{\ipa{ʝi˧ pɤ˥}}} \textcolor{Sepia}{\selectlanguage{english}to harrow} \zh{耙地}  
 ¶ \textcolor{darkblue}{\textbf{\ipa{ɕi˧ tv̩˧-dʑo˧, | ʝi˧ le˧-pɤ˩!}}} \textcolor{Sepia}{\selectlanguage{english}When one plants rice, one must harrow the field (first)!} \zh{种稻谷,要(先)耙地!}  

\lhead{\firstmark}
\rhead{\botmark}

\subsection{\hspace{-0.5cm} {\Large \textcolor{darkblue}{\textbf{\ipa{pɤ˩˧ʐv̩˩}}}}\hspace{0.5cm}[\kern2pt{\textcolor{darkblue}{\textbf{\ipa{xxxx non-correspondance entre le nombre de morphèmes et le nombre de tons de morphèmes}}}}\kern2pt]} \hypertarget{p7\string_B\string_Mz`v\string_=\string_B1}{}
\markboth{\textcolor{darkblue}{\textbf{\ipa{pɤ˩˧ʐv̩˩}}}}{}
\textcolor{teal}{\mytextsc{noun}} \hspace{4pt} Tone: LM-L.
\textcolor{Sepia}{\selectlanguage{english}Mattress.} \zh{褥子(汉语借词:被褥)。} Local Chinese dialect:\zh{被褥。} Borrowing: Chinese  \zh{被褥}
 \zh{量词}: \textcolor{darkblue}{\textbf{\ipa{tsʰi˥}}}  \mytextsc{clf}: \textcolor{darkblue}{\textbf{\ipa{tsʰi˥}}} 
\lhead{\firstmark}
\rhead{\botmark}

\subsection{\hspace{-0.5cm} {\Large \textcolor{darkblue}{\textbf{\ipa{pi˥}}}}\hspace{0.5cm}[\kern2pt{\textcolor{darkblue}{\textbf{\ipa{pi˩˥}}}}\kern2pt]} \hypertarget{pi\string_T1}{}
\markboth{\textcolor{darkblue}{\textbf{\ipa{pi˥}}}}{}
\textcolor{teal}{\mytextsc{verb}} \hspace{4pt} Tone: H.
\textcolor{Sepia}{\selectlanguage{english}To say.} \zh{说。}  ¶ \textcolor{darkblue}{\textbf{\ipa{tʰɑ˧-pi˥!}}} \textcolor{Sepia}{\selectlanguage{english}Don't say it! / Don't speak about it!} \zh{别说!}  
 ¶ \textcolor{darkblue}{\textbf{\ipa{ə˧tso˧ pi˧?}}} \textcolor{Sepia}{\selectlanguage{english}What did you say? (Call for repetition)} \zh{(你刚才)说什么?(请人家重新说一遍)}  
 ¶ \textcolor{darkblue}{\textbf{\ipa{ə˧tso˧ pi˧-ɲi˥?}}} \textcolor{Sepia}{\selectlanguage{english}What did you say? (Call for repetition)} \zh{(你刚才)说什么?(请人家重新说一遍)}  

\lhead{\firstmark}
\rhead{\botmark}

\subsection{\hspace{-0.5cm} {\Large \textcolor{darkblue}{\textbf{\ipa{pi˧lv̩\#˥}}}}\hspace{0.5cm}[\kern2pt{\textcolor{darkblue}{\textbf{\ipa{pi˧lv̩˩}}}}\kern2pt]} \hypertarget{pi\string_Mlv\string_=\#\string_T1}{}
\markboth{\textcolor{darkblue}{\textbf{\ipa{pi˧lv̩\#˥}}}}{}
\textcolor{teal}{\mytextsc{noun}} \hspace{4pt} Tone: \#H.
\textcolor{Sepia}{\selectlanguage{english}Residue left by the production of alcohol, distiller's grains: grains that are fed to the pigs.} \zh{酒糟:煮酒剩下的渣滓(一般给猪吃)。} Local Chinese dialect:\zh{酒糟。} ¶ \textcolor{darkblue}{\textbf{\ipa{pi˧lv̩˧, | hĩ˧ | dzɯ˧-mɤ˧-kv̩˩!}}} \textcolor{Sepia}{\selectlanguage{english}Distiller's grains are not suitable for human consumption! / People don't eat distiller's grains!} \zh{酒糟,人不能吃!}  

\lhead{\firstmark}
\rhead{\botmark}

\subsection{\hspace{-0.5cm} {\Large \textcolor{darkblue}{\textbf{\ipa{pi˧mɑ˧}}}}\hspace{0.5cm}[\kern2pt{\textcolor{darkblue}{\textbf{\ipa{pi˧mɑ˧}}}}\kern2pt]} \hypertarget{pi\string_MmA\string_M1}{}
\markboth{\textcolor{darkblue}{\textbf{\ipa{pi˧mɑ˧}}}}{}
\textcolor{teal}{\mytextsc{noun}} \hspace{4pt} Tone: M.
\textcolor{Sepia}{\selectlanguage{english}A unixex given name: a given name used for both men and women.} \zh{男女通用名。} 
\lhead{\firstmark}
\rhead{\botmark}

\subsection{\hspace{-0.5cm} {\Large \textcolor{darkblue}{\textbf{\ipa{pi˧mɑ˧-ɬɑ˩mv̩˩}}}}\hspace{0.5cm}[\kern2pt{\textcolor{darkblue}{\textbf{\ipa{xxxx non-correspondance entre le nombre de morphèmes et le nombre de tons de morphèmes}}}}\kern2pt]} \hypertarget{pi\string_MmA\string_M-KA\string_Bmv\string_=\string_B1}{}
\markboth{\textcolor{darkblue}{\textbf{\ipa{pi˧mɑ˧-ɬɑ˩mv̩˩}}}}{}
\textcolor{teal}{\mytextsc{noun}} \hspace{4pt} Tone: \mytextsc{L}.
\textcolor{Sepia}{\selectlanguage{english}Feminine given name.} \zh{女性名字。} 
\lhead{\firstmark}
\rhead{\botmark}

\subsection{\hspace{-0.5cm} {\Large \textcolor{darkblue}{\textbf{\ipa{pi˧mv̩˥\$}}}}\hspace{0.5cm}[\kern2pt{\textcolor{darkblue}{\textbf{\ipa{pi˧mv̩˥}}}}\kern2pt]} \hypertarget{pi\string_Mmv\string_=\string_T\$1}{}
\markboth{\textcolor{darkblue}{\textbf{\ipa{pi˧mv̩˥\$}}}}{}
\textcolor{teal}{\mytextsc{noun}} \hspace{4pt} Tone: H\$.
\textcolor{Sepia}{\selectlanguage{english}Set phrase, idiom, adage.} \zh{成语、俗语。} 
\lhead{\firstmark}
\rhead{\botmark}

\subsection{\hspace{-0.5cm} {\Large \textcolor{darkblue}{\textbf{\ipa{pi˧tsʰe˩-di˩}}}}\hspace{0.5cm}[\kern2pt{\textcolor{darkblue}{\textbf{\ipa{pi˧tsʰe˩di˧}}}}\kern2pt]} \hypertarget{pi\string_Mts\string_he\string_B-di\string_B1}{}
\markboth{\textcolor{darkblue}{\textbf{\ipa{pi˧tsʰe˩-di˩}}}}{}
\textcolor{teal}{\mytextsc{noun}} \hspace{4pt} Tone: L\#-.
\textcolor{Sepia}{\selectlanguage{english}A village close to the Hot Springs.} \zh{温泉乡的一个村落。}  ¶ \textcolor{darkblue}{\textbf{\ipa{ə˧go˧-ʁwɤ˧, | ʁwɤ˧lɑ˩-bi˩, | bæ˧ʁwɤ˧, | tʰo˧tsʰe\#˥, | pi˧tsʰe˩-di˩, | pɤ˧dʑɤ˩-di˩, | ʁwɤ˧tv̩˧}}} \textcolor{Sepia}{\selectlanguage{english}Villages that one encounters as one leaves the plain of Yongning (away from the Lake); the first two are perceived as villages with a high proportion of Na members, and the third as a mostly Na village, whereas the next ones are Pumi (Prinmi).} \zh{永宁背向泸沽湖方向经过的村落。前两个村落拥有相当大的摩梭人口比例,第三个村落是摩梭村,最后一个是普米村。}  
 ¶ \textcolor{darkblue}{\textbf{\ipa{pi˧tsʰe˩: bɤ˩! |}}} \textcolor{Sepia}{\selectlanguage{english}\textcolor{darkblue}{\textbf{\ipa{/pi˧tsʰe˧/}}} is a Pumi village!} \zh{fv:/pi˧tsʰe˩/是一个普米族村落!}  

\lhead{\firstmark}
\rhead{\botmark}

\subsection{\hspace{-0.5cm} {\Large \textcolor{darkblue}{\textbf{\ipa{pi˩ɻ̍˥}}}}\hspace{0.5cm}[\kern2pt{\textcolor{darkblue}{\textbf{\ipa{pi˩ɻ̍˥}}}}\kern2pt]} \hypertarget{pi\string_Br£`̍\string_T1}{}
\markboth{\textcolor{darkblue}{\textbf{\ipa{pi˩ɻ̍˥}}}}{}
\textcolor{teal}{\mytextsc{noun}} \hspace{4pt} Tone: LH.
\textcolor{Sepia}{\selectlanguage{english}Double chin; flesh under the chin.} \zh{双下巴。}  \zh{量词}: \textcolor{darkblue}{\textbf{\ipa{ɭɯ˧}}}  \mytextsc{clf}: \textcolor{darkblue}{\textbf{\ipa{ɭɯ˧}}} 
\lhead{\firstmark}
\rhead{\botmark}

\subsection{\hspace{-0.5cm} {\Large \textcolor{darkblue}{\textbf{\ipa{pi˩ti\#˥}}}}\hspace{0.5cm}[\kern2pt{\textcolor{darkblue}{\textbf{\ipa{pi˩ti˥}}}}\kern2pt]} \hypertarget{pi\string_Bti\#\string_T1}{}
\markboth{\textcolor{darkblue}{\textbf{\ipa{pi˩ti\#˥}}}}{}
\textcolor{teal}{\mytextsc{noun}} \hspace{4pt} Tone: LM+\#H.
\textcolor{Sepia}{\selectlanguage{english}Silver nugget, piece of raw silver.} \zh{银块。}  \zh{量词}: \textcolor{darkblue}{\textbf{\ipa{ɭɯ˧}}}  \mytextsc{clf}: \textcolor{darkblue}{\textbf{\ipa{ɭɯ˧}}} 
\lhead{\firstmark}
\rhead{\botmark}

\subsection{\hspace{-0.5cm} {\Large \textcolor{darkblue}{\textbf{\ipa{pi˧˥\textsubscript{a}}}}}\hspace{0.5cm}[\kern2pt{\textcolor{darkblue}{\textbf{\ipa{pi˧˥}}}}\kern2pt]} \hypertarget{pi\string_M\string_Ta1}{}
\markboth{\textcolor{darkblue}{\textbf{\ipa{pi˧˥\textsubscript{a}}}}}{}
\textcolor{teal}{\mytextsc{classifier}} \hspace{4pt} Tone: MH\textsubscript{a}.
\textcolor{Sepia}{\selectlanguage{english}A little (noncount); mostly appears in combination with the numeral 'one'.} \zh{量词:少。}  ¶ \textcolor{darkblue}{\textbf{\ipa{ɖɯ˧-pi˧˥}}} \textcolor{Sepia}{\selectlanguage{english}a little} \zh{一点}  
 ¶ \textcolor{darkblue}{\textbf{\ipa{qʰæ˧-pi˩}}} \textcolor{Sepia}{\selectlanguage{english}a little manure} \zh{一点粪肥}  
 ¶ \textcolor{darkblue}{\textbf{\ipa{ŋv̩˧-pi˧}}} \textcolor{Sepia}{\selectlanguage{english}a little money} \zh{一点钱}  
 ¶ \textcolor{darkblue}{\textbf{\ipa{ŋv̩˧ | ɖɯ˧-pi˧˥}}} \textcolor{Sepia}{\selectlanguage{english}a little money} \zh{一点钱}  
 ¶ \textcolor{darkblue}{\textbf{\ipa{lwɤ˧˥ | ɖɯ˧ pi˧˥}}} \textcolor{Sepia}{\selectlanguage{english}a little ashes} \zh{一点灰}  
 ¶ \textcolor{darkblue}{\textbf{\ipa{ʈʂʰɯ˧ | ɖʐe˧ ɖɯ˧-pi˧ dʑo˧!}}} \textcolor{Sepia}{\selectlanguage{english}He has a little money! / He is rather affluent!} \zh{他有一些钱!}  

\lhead{\firstmark}
\rhead{\botmark}

\subsection{\hspace{-0.5cm} {\Large \textcolor{darkblue}{\textbf{\ipa{pi˩˥}}}}\hspace{0.5cm}[\kern2pt{\textcolor{darkblue}{\textbf{\ipa{pi˥}}}}\kern2pt]} \hypertarget{pi\string_B\string_T1}{}
\markboth{\textcolor{darkblue}{\textbf{\ipa{pi˩˥}}}}{}
\textcolor{teal}{\mytextsc{noun}} \hspace{4pt} Tone: LH.
\textcolor{Sepia}{\selectlanguage{english}Brush (Chinese borrowing).} \zh{笔。}  Borrowing: Chinese  \zh{笔}
 ¶ \textcolor{darkblue}{\textbf{\ipa{tʰæ˧ɻæ˩ tɕɯ˩-di˩, | pi˩˥!}}} \textcolor{Sepia}{\selectlanguage{english}The thing used to write is called “pen”!} \zh{用来写字的那个东西,(叫做)“笔”!}  

\lhead{\firstmark}
\rhead{\botmark}

\subsection{\hspace{-0.5cm} {\Large \textcolor{darkblue}{\textbf{\ipa{pjɤ˥}}}}\hspace{0.5cm}[\kern2pt{\textcolor{darkblue}{\textbf{\ipa{pjɤ˥}}}}\kern2pt]} \hypertarget{pj7\string_T1}{}
\markboth{\textcolor{darkblue}{\textbf{\ipa{pjɤ˥}}}}{}
\textcolor{teal}{\mytextsc{adjective}} \hspace{4pt} Tone: H.
\textcolor{Sepia}{\selectlanguage{english}Square.} \zh{方形的。}  ¶ \textcolor{darkblue}{\textbf{\ipa{tɑ˧-pjɤ˧\textasciitilde{}pjɤ˥ (-zo˩)}}} \textcolor{Sepia}{\selectlanguage{english}(of a face or object) unpleasantly squarish, lacking smoothness} \zh{(脸、物品)太方,不圆滑}  

\lhead{\firstmark}
\rhead{\botmark}

\subsection{\hspace{-0.5cm} {\Large \textcolor{darkblue}{\textbf{\ipa{po˥}}}}\hspace{0.5cm}[\kern2pt{\textcolor{darkblue}{\textbf{\ipa{po˧˥}}}}\kern2pt]} \hypertarget{po\string_T1}{}
\markboth{\textcolor{darkblue}{\textbf{\ipa{po˥}}}}{}
\textcolor{teal}{\mytextsc{verb}} \hspace{4pt} Tone: .
\textcolor{Sepia}{\selectlanguage{english}To pack.} \zh{包(量词)(汉语借词)。}  Borrowing: Chinese  \zh{包}
 ¶ \textcolor{darkblue}{\textbf{\ipa{le˧-po˥}}} \textcolor{Sepia}{\selectlanguage{english}\mytextsc{accomp}} \zh{\mytextsc{accomp}}  

\lhead{\firstmark}
\rhead{\botmark}

\subsection{\hspace{-0.5cm} {\Large \textcolor{darkblue}{\textbf{\ipa{po˧\textsubscript{a}}}}}\hspace{0.5cm}[\kern2pt{\textcolor{darkblue}{\textbf{\ipa{po˥}}}}\kern2pt]} \hypertarget{po\string_Ma1}{}
\markboth{\textcolor{darkblue}{\textbf{\ipa{po˧\textsubscript{a}}}}}{}
\textcolor{teal}{\mytextsc{classifier}} \hspace{4pt} Tone: M\textsubscript{a}.
\textcolor{Sepia}{\selectlanguage{english}Classifier for plants with a stalk; also used for pieces of clothing.} \zh{量词:有根的植物,衣服(一棵,一件)。} 
\lhead{\firstmark}
\rhead{\botmark}

\subsection{\hspace{-0.5cm} {\Large \textcolor{darkblue}{\textbf{\ipa{po˧ɖʐɯ\#˥}}}}\hspace{0.5cm}[\kern2pt{\textcolor{darkblue}{\textbf{\ipa{po˧ɖʐɯ˧}}}}\kern2pt]} \hypertarget{po\string_Md`z`M\#\string_T1}{}
\markboth{\textcolor{darkblue}{\textbf{\ipa{po˧ɖʐɯ\#˥}}}}{}
\textcolor{teal}{\mytextsc{noun}} \hspace{4pt} Tone: \#H.
\textcolor{Sepia}{\selectlanguage{english}Craftsman.} \zh{工匠。}  ¶ \textcolor{darkblue}{\textbf{\ipa{po˧ɖʐɯ˧ ʝi˧-hĩ˧-hĩ˧}}} \textcolor{Sepia}{\selectlanguage{english}person who works as a craftsman} \zh{当工匠的人}  
 \zh{量词}: \textcolor{darkblue}{\textbf{\ipa{v̩˧}}}  \mytextsc{clf}: \textcolor{darkblue}{\textbf{\ipa{v̩˧}}} 
\lhead{\firstmark}
\rhead{\botmark}

\subsection{\hspace{-0.5cm} {\Large \textcolor{darkblue}{\textbf{\ipa{po˧lo˧}}}}\hspace{0.5cm}[\kern2pt{\textcolor{darkblue}{\textbf{\ipa{po˩lo˧˥}}}}\kern2pt]} \hypertarget{po\string_Mlo\string_M1}{}
\markboth{\textcolor{darkblue}{\textbf{\ipa{po˧lo˧}}}}{}
\textcolor{teal}{\mytextsc{noun}} \hspace{4pt} Tone: M.
\textcolor{Sepia}{\selectlanguage{english}Ram.} \zh{公绵羊。}  ¶ \textcolor{darkblue}{\textbf{\ipa{po˧lo˧ lɑ˧˥}}} \textcolor{Sepia}{\selectlanguage{english}to strike a ram} \zh{打公绵羊}  
 \zh{量词}: \textcolor{darkblue}{\textbf{\ipa{pʰo˧˥}}}  \mytextsc{clf}: \textcolor{darkblue}{\textbf{\ipa{pʰo˧˥}}} 
\lhead{\firstmark}
\rhead{\botmark}

\subsection{\hspace{-0.5cm} {\Large \textcolor{darkblue}{\textbf{\ipa{po˧po˧}}}}\hspace{0.5cm}[\kern2pt{\textcolor{darkblue}{\textbf{\ipa{po˩po˩˥}}}}\kern2pt]} \hypertarget{po\string_Mpo\string_M1}{}
\markboth{\textcolor{darkblue}{\textbf{\ipa{po˧po˧}}}}{}
\textcolor{teal}{\mytextsc{noun}} \hspace{4pt} Tone: M.
\textcolor{Sepia}{\selectlanguage{english}Ball.} \zh{球。}  ¶ \textcolor{darkblue}{\textbf{\ipa{[F5] po˧po˧ lɑ˧˥}}} \textcolor{Sepia}{\selectlanguage{english}to play (foot)ball} \zh{打球}  
 \zh{量词}: \textcolor{darkblue}{\textbf{\ipa{ɭɯ˧}}}  \mytextsc{clf}: \textcolor{darkblue}{\textbf{\ipa{ɭɯ˧}}} 
\lhead{\firstmark}
\rhead{\botmark}

\subsection{\hspace{-0.5cm} {\Large \textcolor{darkblue}{\textbf{\ipa{po˧po˧tsʰɤ˧˥}}}}\hspace{0.5cm}[\kern2pt{\textcolor{darkblue}{\textbf{\ipa{po˧po˧tsʰɤ˧}}}}\kern2pt]} \hypertarget{po\string_Mpo\string_Mts\string_h7\string_M\string_T1}{}
\markboth{\textcolor{darkblue}{\textbf{\ipa{po˧po˧tsʰɤ˧˥}}}}{}
\textcolor{teal}{\mytextsc{noun}} \hspace{4pt} Tone: MH\#.
\textcolor{Sepia}{\selectlanguage{english}Cabbage.} \zh{圆白菜。} Local Chinese dialect:\zh{包包菜。} Borrowing: Chinese  \zh{包包菜}
 \zh{量词}: \textcolor{darkblue}{\textbf{\ipa{ɭɯ˧}}}  \mytextsc{clf}: \textcolor{darkblue}{\textbf{\ipa{ɭɯ˧}}} 
\lhead{\firstmark}
\rhead{\botmark}

\subsection{\hspace{-0.5cm} {\Large \textcolor{darkblue}{\textbf{\ipa{po˩\textsubscript{b}}}}}\hspace{0.5cm}[\kern2pt{\textcolor{darkblue}{\textbf{\ipa{po˩˥}}}}\kern2pt]} \hypertarget{po\string_Bb1}{}
\markboth{\textcolor{darkblue}{\textbf{\ipa{po˩\textsubscript{b}}}}}{}
\textcolor{teal}{\mytextsc{classifier}} \hspace{4pt} Tone: L\textsubscript{b}.
\textcolor{Sepia}{\selectlanguage{english}Classifier for packs (e.g. a pack of cigarettes).} \zh{量词:包(汉语借词)。}  Borrowing: Chinese  \zh{包}

\lhead{\firstmark}
\rhead{\botmark}

\subsection{\hspace{-0.5cm} {\Large \textcolor{darkblue}{\textbf{\ipa{po˧˥}}}}\hspace{0.5cm}[\kern2pt{\textcolor{darkblue}{\textbf{\ipa{po˧˥}}}}\kern2pt]} \hypertarget{po\string_M\string_T1}{}
\markboth{\textcolor{darkblue}{\textbf{\ipa{po˧˥}}}}{}
\textcolor{teal}{\mytextsc{verb}} \hspace{4pt} Tone: MH.
\ding{202} \textcolor{Sepia}{\selectlanguage{english}To bring; to send (a letter), to deliver (a message).} \zh{寄信、服送、带过来、拿、送。}  ¶ \textcolor{darkblue}{\textbf{\ipa{qʰwæ˧ po˧˥}}} \textcolor{Sepia}{\selectlanguage{english}to bring a letter/a message} \zh{带来一封信/一个消息}  
 ¶ \textcolor{darkblue}{\textbf{\ipa{tso˧\textasciitilde{}tso˧ tʰi˧-po˧˥}}} \textcolor{Sepia}{\selectlanguage{english}to bring something} \zh{带来一个东西}  
 ¶ \textcolor{darkblue}{\textbf{\ipa{hĩ˧ ɖɯ˧-v̩˧ | tso˧\textasciitilde{}tso˧ ɖɯ˧-kʰwɤ˥ | tʰi˧-po˧˥}}} \textcolor{Sepia}{\selectlanguage{english}someone brings something} \zh{有人带东西过来}  
 ¶ \textcolor{darkblue}{\textbf{\ipa{ʈʂʰwæ˧˥ | po˧-jo˥!}}} \textcolor{Sepia}{\selectlanguage{english}Bring it over, quick!} \zh{快拿过来吧!/ 快带过来吧!}  
\ding{203} \textcolor{Sepia}{\selectlanguage{english}To carry (a child), i.e. to be pregnant.} \zh{怀孕。}  ¶ \textcolor{darkblue}{\textbf{\ipa{ʈʂʰɯ˧ | zo˧mv̩˥ po˩.}}} \textcolor{Sepia}{\selectlanguage{english}She is pregnant.} \zh{她怀孕了。}  
 ¶ \textcolor{darkblue}{\textbf{\ipa{zo˧ po˩ (+ze˩)}}} \textcolor{Sepia}{\selectlanguage{english}to carry a child, i.e. to be pregnant} \zh{怀孕}  

\lhead{\firstmark}
\rhead{\botmark}

\subsection{\hspace{-0.5cm} {\Large \textcolor{darkblue}{\textbf{\ipa{pv̩˩}}}}\hspace{0.5cm}[\kern2pt{\textcolor{darkblue}{\textbf{\ipa{pv̩˩˥}}}}\kern2pt]} \hypertarget{pv\string_=\string_B1}{}
\markboth{\textcolor{darkblue}{\textbf{\ipa{pv̩˩}}}}{}
\textcolor{teal}{\mytextsc{verb}} \hspace{4pt} Tone: L.
\textcolor{Sepia}{\selectlanguage{english}To go by, to flow (of time).} \zh{过、过去(时间过去、日子过去)。}  ¶ \textcolor{darkblue}{\textbf{\ipa{ɲi˧mi˧ pv̩˩}}} \textcolor{Sepia}{\selectlanguage{english}time goes by; literally: the day goes by} \zh{时间过去。直译:(一)天(慢慢)过(去)}  
 ¶ \textcolor{darkblue}{\textbf{\ipa{ɲi˧mi˧ | le˧-pv̩˩-ze˩}}} \textcolor{Sepia}{\selectlanguage{english}time has gone by, the day has gone by} \zh{时间过去了,(一)天过去了}  
 ¶ \textcolor{darkblue}{\textbf{\ipa{dʑɤ˩-dzɯ˧ qʰwɤ˧-dzɯ˥, | bi˧mi˧ ʂv̩˧-qʰwɤ˧-ɻ̍˥; | dʑɤ˩-ʐwɤ˥ qʰwɤ˩-ʐwɤ˩, | ɲi˧mi˧ ʂæ˧ pv̩˩-di˩!}}} \textcolor{Sepia}{\selectlanguage{english}Whether one eats good stuff or bad stuff, that fills the stomach / that does the trick of filling your belly! Whether one tells good stories or bad ones, that helps make the long day go by / that does the trick of chipping a long (and tedious) day away/of filling a day pleasantly! (A laid-back proverb in praise of small talk and gossip.)} \zh{“吃好吃坏,(都)能填满肚子/(都)能吃饱!说好说坏,(都)能让一天(轻松)过去!”(这个谚语,说闲聊的好。)}  
 ¶ \textcolor{darkblue}{\textbf{\ipa{dʑɤ˩-dzɯ˧ qʰwɤ˧-dzɯ˥, | bi˧mi˧ ʂv̩˧˥; | dʑɤ˩-ʐwɤ˥ qʰwɤ˩-ʐwɤ˩, | ɲi˧mi˧ ʂæ˧-pv̩˩-di˩!}}} \textcolor{Sepia}{\selectlanguage{english}Variant of the above proverb.} \zh{上述谚语的变体}  

\lhead{\firstmark}
\rhead{\botmark}

\subsection{\hspace{-0.5cm} {\Large \textcolor{darkblue}{\textbf{\ipa{pv̩˧}}} \textsubscript{1}}\hspace{0.5cm}[\kern2pt{\textcolor{darkblue}{\textbf{\ipa{pv̩˥}}}}\kern2pt]} \hypertarget{pv\string_=\string_M1}{}
\markboth{\textcolor{darkblue}{\textbf{\ipa{pv̩˧}}} \textsubscript{1}}{}
\textcolor{teal}{\mytextsc{verb}} \hspace{4pt} Tone: M\textsubscript{c}.
\textcolor{Sepia}{\selectlanguage{english}To perform (a sacrifice, a ritual), to celebrate (a festival), to chant (a ritual).} \zh{祭。}  ¶ \textcolor{darkblue}{\textbf{\ipa{kʰv̩˧ pv̩˥}}} \textcolor{Sepia}{\selectlanguage{english}to do the New Year ceremony, to celebrate the New Year} \zh{做过年的祭礼}  
 ¶ \textcolor{darkblue}{\textbf{\ipa{tsʰi˧ɲi˧, | kʰv̩˧ pv̩˥-tso˩-ɲi˩!}}} \textcolor{Sepia}{\selectlanguage{english}Tonight, we are going to celebrate the New Year!} \zh{今天就要过年了!}  

\lhead{\firstmark}
\rhead{\botmark}

\subsection{\hspace{-0.5cm} {\Large \textcolor{darkblue}{\textbf{\ipa{pv̩˧}}} \textsubscript{2}}\hspace{0.5cm}[\kern2pt{\textcolor{darkblue}{\textbf{\ipa{pv̩˥}}}}\kern2pt]} \hypertarget{pv\string_=\string_M2}{}
\markboth{\textcolor{darkblue}{\textbf{\ipa{pv̩˧}}} \textsubscript{2}}{}
\textcolor{teal}{\mytextsc{adjective}} \hspace{4pt} Tone: M.
\textcolor{Sepia}{\selectlanguage{english}Dry.} \zh{干燥。}  ¶ \textcolor{darkblue}{\textbf{\ipa{le˧-pv̩˧-ze˧}}} \textcolor{Sepia}{\selectlanguage{english}\mytextsc{accomp} \string_ \mytextsc{pfv}} \zh{干了}  
 ¶ \textcolor{darkblue}{\textbf{\ipa{le˧-pv̩˧ le˧-ʐwæ˩-ze˩}}} \textcolor{Sepia}{\selectlanguage{english}It has dried up / it has completely dried / it is now completely dry} \zh{干透了}  
 ¶ \textcolor{darkblue}{\textbf{\ipa{pv̩˧-kæ˧-ɻæ˩-gv̩˩}}} \textcolor{Sepia}{\selectlanguage{english}all dry, completely dry} \zh{全干、完全干}  

\lhead{\firstmark}
\rhead{\botmark}

\subsection{\hspace{-0.5cm} {\Large \textcolor{darkblue}{\textbf{\ipa{pv̩˧˥}}} \textsubscript{1}}\hspace{0.5cm}[\kern2pt{\textcolor{darkblue}{\textbf{\ipa{pv̩˧˥}}}}\kern2pt]} \hypertarget{pv\string_=\string_M\string_T1}{}
\markboth{\textcolor{darkblue}{\textbf{\ipa{pv̩˧˥}}} \textsubscript{1}}{}
\textcolor{teal}{\mytextsc{verb}} \hspace{4pt} Tone: MH.
\textcolor{Sepia}{\selectlanguage{english}To pull out (weeds), to weed.} \zh{拔、扯(草)。}  ¶ \textcolor{darkblue}{\textbf{\ipa{zɯ˧ pv̩˩}}} \textcolor{Sepia}{\selectlanguage{english}to pull out (weeds), to weed; to cut grass for cattle} \zh{拔草}  
 ¶ \textcolor{darkblue}{\textbf{\ipa{zɯ˧ | le˧-pv̩˧˥}}} \textcolor{Sepia}{\selectlanguage{english}to pull out (weeds), to weed; to cut grass for cattle} \zh{拔草}  

\lhead{\firstmark}
\rhead{\botmark}

\subsection{\hspace{-0.5cm} {\Large \textcolor{darkblue}{\textbf{\ipa{pv̩˧˥}}} \textsubscript{2}}\hspace{0.5cm}[\kern2pt{\textcolor{darkblue}{\textbf{\ipa{pv̩˧˥}}}}\kern2pt]} \hypertarget{pv\string_=\string_M\string_T2}{}
\markboth{\textcolor{darkblue}{\textbf{\ipa{pv̩˧˥}}} \textsubscript{2}}{}
\textcolor{teal}{\mytextsc{verb}} \hspace{4pt} Tone: MH.
\textcolor{Sepia}{\selectlanguage{english}To draw (a weapon), to take out of its sheath.} \zh{拉出(剑……)。}  ¶ \textcolor{darkblue}{\textbf{\ipa{ʁæ˧mi˧ | tʰi˧-pv̩˧˥}}} \textcolor{Sepia}{\selectlanguage{english}to draw a sword} \zh{拉出剑}  
 ¶ \textcolor{darkblue}{\textbf{\ipa{gæ˩-pv̩˧˥}}} \textcolor{Sepia}{\selectlanguage{english}to draw (a weapon), to take out of its sheath} \zh{拉出(剑……)}  
 ¶ \textcolor{darkblue}{\textbf{\ipa{ʁæ˧mi˧ | gæ˩-pv̩˧˥}}} \textcolor{Sepia}{\selectlanguage{english}to draw a sword} \zh{拉出剑}  

\lhead{\firstmark}
\rhead{\botmark}

\subsection{\hspace{-0.5cm} {\Large \textcolor{darkblue}{\textbf{\ipa{pv̩˧˥}}} \textsubscript{3}}\hspace{0.5cm}[\kern2pt{\textcolor{darkblue}{\textbf{\ipa{pv̩˧˥}}}}\kern2pt]} \hypertarget{pv\string_=\string_M\string_T3}{}
\markboth{\textcolor{darkblue}{\textbf{\ipa{pv̩˧˥}}} \textsubscript{3}}{}
\textcolor{teal}{\mytextsc{classifier}} \hspace{4pt} Tone: MH\textsubscript{a}.
\textcolor{Sepia}{\selectlanguage{english}Classifier for steps / strides.} \zh{量词:步。}  Borrowing: Chinese  \zh{步}

\lhead{\firstmark}
\rhead{\botmark}

\subsection{\hspace{-0.5cm} {\Large \textcolor{darkblue}{\textbf{\ipa{pv̩˩\textsubscript{a}}}} \textsubscript{1}}\hspace{0.5cm}[\kern2pt{\textcolor{darkblue}{\textbf{\ipa{pv̩˩˥}}}}\kern2pt]} \hypertarget{pv\string_=\string_Ba1}{}
\markboth{\textcolor{darkblue}{\textbf{\ipa{pv̩˩\textsubscript{a}}}} \textsubscript{1}}{}
\textcolor{teal}{\mytextsc{verb}} \hspace{4pt} Tone: L\textsubscript{a}.
\textcolor{Sepia}{\selectlanguage{english}To see off.} \zh{送行。}  ¶ \textcolor{darkblue}{\textbf{\ipa{hĩ˧bæ˧ pv̩˥}}} \textcolor{Sepia}{\selectlanguage{english}to see a guest off} \zh{送客}  

\lhead{\firstmark}
\rhead{\botmark}

\subsection{\hspace{-0.5cm} {\Large \textcolor{darkblue}{\textbf{\ipa{pv̩˩\textsubscript{a}}}} \textsubscript{2}}\hspace{0.5cm}[\kern2pt{\textcolor{darkblue}{\textbf{\ipa{pv̩˩˥}}}}\kern2pt]} \hypertarget{pv\string_=\string_Ba2}{}
\markboth{\textcolor{darkblue}{\textbf{\ipa{pv̩˩\textsubscript{a}}}} \textsubscript{2}}{}
\textcolor{teal}{\mytextsc{verb}} \hspace{4pt} Tone: L\textsubscript{a}.
\textcolor{Sepia}{\selectlanguage{english}To allow, to authorize (someone to do something, e.g. to marry); to finance (i.e. to invest money in a caravan); to require (someone to do something).} \zh{让,安排,投资,要求。}  ¶ \textcolor{darkblue}{\textbf{\ipa{sɯ˧pʰi˧-ɳɯ˧ | pv̩˩-kʰɯ˥-ɲi˩!}}} \textcolor{Sepia}{\selectlanguage{english}It was the feudal lord who financed (the caravan)!} \zh{(马帮)是土司来投资的!}  
 ¶ \textcolor{darkblue}{\textbf{\ipa{ʈʂʰɯ˧ | ɖʐe˧ ʂe˧ pv̩˩-kʰɯ˩-tso˩-ɲi˩!}}} \textcolor{Sepia}{\selectlanguage{english}(S)he is bringing the capital! / (S)he is financing the whole thing! (e.g. a caravan)} \zh{是他来投资的!(如:马帮)}  
 ¶ \textcolor{darkblue}{\textbf{\ipa{hĩ˧-ɳɯ˩ | pv̩˩-mɤ˩-kʰɯ˥!}}} \textcolor{Sepia}{\selectlanguage{english}People do not allow access! / Access is not allowed! (Context: a discussion about difficulties for the investigator to be allowed to stay in an area of Sichuan where Naish languages are spoken. The consultant summarizes as: “Access is not allowed!”)} \zh{人家不让去!}  

\lhead{\firstmark}
\rhead{\botmark}

\subsection{\hspace{-0.5cm} {\Large \textcolor{darkblue}{\textbf{\ipa{pv̩˩\textsubscript{a}}}} \textsubscript{3}}\hspace{0.5cm}[\kern2pt{\textcolor{darkblue}{\textbf{\ipa{pv̩˩˥}}}}\kern2pt]} \hypertarget{pv\string_=\string_Ba3}{}
\markboth{\textcolor{darkblue}{\textbf{\ipa{pv̩˩\textsubscript{a}}}} \textsubscript{3}}{}
\textcolor{teal}{\mytextsc{verb}} \hspace{4pt} Tone: L\textsubscript{a}.
\textcolor{Sepia}{\selectlanguage{english}To comb.} \zh{梳。}  ¶ \textcolor{darkblue}{\textbf{\ipa{ʁo˧qʰwɤ˩ pv̩˩}}} \textcolor{Sepia}{\selectlanguage{english}to comb one's head} \zh{梳头}  
 ¶ \textcolor{darkblue}{\textbf{\ipa{ʁo˧ pv̩˥}}} \textcolor{Sepia}{\selectlanguage{english}to comb one's head} \zh{梳头}  

\lhead{\firstmark}
\rhead{\botmark}

\subsection{\hspace{-0.5cm} {\Large \textcolor{darkblue}{\textbf{\ipa{pv̩˩ɭɯ˥}}}}\hspace{0.5cm}[\kern2pt{\textcolor{darkblue}{\textbf{\ipa{pv̩˩ɭɯ˥}}}}\kern2pt]} \hypertarget{pv\string_=\string_Bl\string_RM\string_T1}{}
\markboth{\textcolor{darkblue}{\textbf{\ipa{pv̩˩ɭɯ˥}}}}{}
\textcolor{teal}{\mytextsc{noun}} \hspace{4pt} Tone: LH.
\textcolor{Sepia}{\selectlanguage{english}Button.} \zh{扣子。}  \zh{量词}: \textcolor{darkblue}{\textbf{\ipa{ɭɯ˧}}}  \mytextsc{clf}: \textcolor{darkblue}{\textbf{\ipa{ɭɯ˧}}} 
\lhead{\firstmark}
\rhead{\botmark}

\subsection{\hspace{-0.5cm} {\Large \textcolor{darkblue}{\textbf{\ipa{pv̩˧lv̩˧}}}}\hspace{0.5cm}[\kern2pt{\textcolor{darkblue}{\textbf{\ipa{pv̩˧lv̩˧}}}}\kern2pt]} \hypertarget{pv\string_=\string_Mlv\string_=\string_M1}{}
\markboth{\textcolor{darkblue}{\textbf{\ipa{pv̩˧lv̩˧}}}}{}
\textcolor{teal}{\mytextsc{noun}} \hspace{4pt} Tone: M.
\textcolor{Sepia}{\selectlanguage{english}Nonirrigated farmland; dry land.} \zh{旱地。}  \zh{量词}: \textcolor{darkblue}{\textbf{\ipa{pʰv̩˩}}}  \mytextsc{clf}: \textcolor{darkblue}{\textbf{\ipa{pʰv̩˩}}} 
\lhead{\firstmark}
\rhead{\botmark}

\subsection{\hspace{-0.5cm} {\Large \textcolor{darkblue}{\textbf{\ipa{pv̩˩mi˩}}}}\hspace{0.5cm}[\kern2pt{\textcolor{darkblue}{\textbf{\ipa{pv̩˩mi˩˥}}}}\kern2pt]} \hypertarget{pv\string_=\string_Bmi\string_B1}{}
\markboth{\textcolor{darkblue}{\textbf{\ipa{pv̩˩mi˩}}}}{}
\textcolor{teal}{\mytextsc{noun}} \hspace{4pt} Tone: L.
\textcolor{Sepia}{\selectlanguage{english}Comb (coarse).} \zh{粗齿梳子。}  \zh{量词}: \textcolor{darkblue}{\textbf{\ipa{nɑ˧}}}  \mytextsc{clf}: \textcolor{darkblue}{\textbf{\ipa{nɑ˧}}} 
\lhead{\firstmark}
\rhead{\botmark}

\subsection{\hspace{-0.5cm} {\Large \textcolor{darkblue}{\textbf{\ipa{pv̩˩pv̩˧}}}}\hspace{0.5cm}[\kern2pt{\textcolor{darkblue}{\textbf{\ipa{pv̩˩pv̩˥}}}}\kern2pt]} \hypertarget{pv\string_=\string_Bpv\string_=\string_M1}{}
\markboth{\textcolor{darkblue}{\textbf{\ipa{pv̩˩pv̩˧}}}}{}
\textcolor{teal}{\mytextsc{noun}} \hspace{4pt} Tone: LM.
\textcolor{Sepia}{\selectlanguage{english}Pocket.} \zh{衣兜。}  ¶ \textcolor{darkblue}{\textbf{\ipa{bɑ˩lɑ˩-pv̩˥pv̩˩}}} \textcolor{Sepia}{\selectlanguage{english}pocket of the shirt; it can contain small objects such as tobacco and coins.} \zh{上衣兜子}  
\textit{Syn:} \hyperlink{}{\textcolor{darkblue}{\textbf{\ipa{tɑ˩dv̩˧˥}}}}. 
\lhead{\firstmark}
\rhead{\botmark}

\subsection{\hspace{-0.5cm} {\Large \textcolor{darkblue}{\textbf{\ipa{pv̩˧qʰwɤ˥}}}}\hspace{0.5cm}[\kern2pt{\textcolor{darkblue}{\textbf{\ipa{pv̩˧qʰwɤ˥}}}}\kern2pt]} \hypertarget{pv\string_=\string_Mq\string_hw7\string_T1}{}
\markboth{\textcolor{darkblue}{\textbf{\ipa{pv̩˧qʰwɤ˥}}}}{}
\textcolor{teal}{\mytextsc{noun}} \hspace{4pt} Tone: H\#.
\textcolor{Sepia}{\selectlanguage{english}Shuttle (traditional shuttle made of wood).} \zh{梭,梭子(传统的,木头做的)。}  ¶ \textcolor{darkblue}{\textbf{\ipa{ɣɯ˧dzo˩-bv̩˩ | pv̩˧qʰwɤ˥}}} \textcolor{Sepia}{\selectlanguage{english}the shuttle of the loom} \zh{织布机的梭子}  
 \zh{量词}: \textcolor{darkblue}{\textbf{\ipa{ɭɯ˧}}}  \mytextsc{clf}: \textcolor{darkblue}{\textbf{\ipa{ɭɯ˧}}} \textit{See:} \hyperlink{}{\textcolor{darkblue}{\textbf{\ipa{kʰɯ˩pv̩˩}}}} 
\lhead{\firstmark}
\rhead{\botmark}

\subsection{\hspace{-0.5cm} {\Large \textcolor{darkblue}{\textbf{\ipa{pv̩˧ɻ\#˥}}}}\hspace{0.5cm}[\kern2pt{\textcolor{darkblue}{\textbf{\ipa{pv̩˧ɻ˧}}}}\kern2pt]} \hypertarget{pv\string_=\string_Mr£`\#\string_T1}{}
\markboth{\textcolor{darkblue}{\textbf{\ipa{pv̩˧ɻ\#˥}}}}{}
\textcolor{teal}{\mytextsc{noun}} \hspace{4pt} Tone: \#H.
\textcolor{Sepia}{\selectlanguage{english}Tibetan wool fabric.} \zh{氆氇。}  \zh{量词}: \textcolor{darkblue}{\textbf{\ipa{tsʰi˥}}}  \mytextsc{clf}: \textcolor{darkblue}{\textbf{\ipa{tsʰi˥}}} 
\lhead{\firstmark}
\rhead{\botmark}

\subsection{\hspace{-0.5cm} {\Large \textcolor{darkblue}{\textbf{\ipa{pv̩˧ʂɯ˩}}}}\hspace{0.5cm}[\kern2pt{\textcolor{darkblue}{\textbf{\ipa{pv̩˧ʂɯ˩}}}}\kern2pt]} \hypertarget{pv\string_=\string_Ms`M\string_B1}{}
\markboth{\textcolor{darkblue}{\textbf{\ipa{pv̩˧ʂɯ˩}}}}{}
\textcolor{teal}{\mytextsc{noun}} \hspace{4pt} Tone: L\#.
\textcolor{Sepia}{\selectlanguage{english}Amber.} \zh{琥珀。}  \zh{量词}: \textcolor{darkblue}{\textbf{\ipa{ɭɯ˧}}}  \mytextsc{clf}: \textcolor{darkblue}{\textbf{\ipa{ɭɯ˧}}} 
\lhead{\firstmark}
\rhead{\botmark}

\subsection{\hspace{-0.5cm} {\Large \textcolor{darkblue}{\textbf{\ipa{pv̩˩tɑ˩}}}}\hspace{0.5cm}[\kern2pt{\textcolor{darkblue}{\textbf{\ipa{pv̩˩tɑ˩˥}}}}\kern2pt]} \hypertarget{pv\string_=\string_BtA\string_B1}{}
\markboth{\textcolor{darkblue}{\textbf{\ipa{pv̩˩tɑ˩}}}}{}
\textcolor{teal}{\mytextsc{noun}} \hspace{4pt} Tone: L.
\textcolor{Sepia}{\selectlanguage{english}Bucket, pail.} \zh{桶。}  \zh{量词}: \textcolor{darkblue}{\textbf{\ipa{ɭɯ˧}}}  \mytextsc{clf}: \textcolor{darkblue}{\textbf{\ipa{ɭɯ˧}}} 
\lhead{\firstmark}
\rhead{\botmark}

\subsection{\hspace{-0.5cm} {\Large \textcolor{darkblue}{\textbf{\ipa{pv̩˩tsɯ˧˥}}}}\hspace{0.5cm}[\kern2pt{\textcolor{darkblue}{\textbf{\ipa{pv̩˩tsɯ˧˥}}}}\kern2pt]} \hypertarget{pv\string_=\string_BtsM\string_M\string_T1}{}
\markboth{\textcolor{darkblue}{\textbf{\ipa{pv̩˩tsɯ˧˥}}}}{}
\textcolor{teal}{\mytextsc{noun}} \hspace{4pt} Tone: LM+MH\#.
\ding{202} \textcolor{Sepia}{\selectlanguage{english}Fine comb (used to comb out lice).} \zh{篦子(用来梳虱子)。}  \zh{量词}: \textcolor{darkblue}{\textbf{\ipa{nɑ˧}}} \ding{203} \textcolor{Sepia}{\selectlanguage{english}Iron threads in a wooden frame (like a comb in which the weft is caught), used to tamp down the threads while weaving.} \zh{用来夯实布料的木头架子,里面有铁丝。}  \mytextsc{clf}: \textcolor{darkblue}{\textbf{\ipa{nɑ˧}}} 
\lhead{\firstmark}
\rhead{\botmark}

\subsection{\hspace{-0.5cm} {\Large \textcolor{darkblue}{\textbf{\ipa{pv̩˧ʈʂɯ˩}}}}\hspace{0.5cm}[\kern2pt{\textcolor{darkblue}{\textbf{\ipa{pv̩˧ʈʂɯ˩}}}}\kern2pt]} \hypertarget{pv\string_=\string_Mt`s`M\string_B1}{}
\markboth{\textcolor{darkblue}{\textbf{\ipa{pv̩˧ʈʂɯ˩}}}}{}
\textcolor{teal}{\mytextsc{verb}} \hspace{4pt} Tone: L\#.
\textcolor{Sepia}{\selectlanguage{english}To press, to squeeze.} \zh{挤、挤压。}  ¶ \textcolor{darkblue}{\textbf{\ipa{njɤ˧-ɳɯ˧ | pv̩˧ʈʂɯ˩-bi˩!}}} \textcolor{Sepia}{\selectlanguage{english}I'm going to press (it)! / Let me press it!} \zh{我来压吧!}  
 ¶ \textcolor{darkblue}{\textbf{\ipa{le˧-pv̩˥ʈʂɯ˩}}} \textcolor{Sepia}{\selectlanguage{english}\mytextsc{accomp}} \zh{\mytextsc{accomp}}  

\lhead{\firstmark}
\rhead{\botmark}

\subsection{\hspace{-0.5cm} {\Large \textcolor{darkblue}{\textbf{\ipa{pv̩˩tsɯ˧-pv̩˥mi˩}}}}\hspace{0.5cm}[\kern2pt{\textcolor{darkblue}{\textbf{\ipa{pv̩˩tsɯ˧pv̩˥mi˩}}}}\kern2pt]} \hypertarget{pv\string_=\string_BtsM\string_M-pv\string_=\string_Tmi\string_B1}{}
\markboth{\textcolor{darkblue}{\textbf{\ipa{pv̩˩tsɯ˧-pv̩˥mi˩}}}}{}
\textcolor{teal}{\mytextsc{noun}} \hspace{4pt} Tone: LM+\#H-.
\textcolor{Sepia}{\selectlanguage{english}Combs.} \zh{梳子(总称)。} 
\lhead{\firstmark}
\rhead{\botmark}

\subsection{\hspace{-0.5cm} {\Large \textcolor{darkblue}{\textbf{\ipa{pv̩˩ʈʰɯ˧}}}}\hspace{0.5cm}[\kern2pt{\textcolor{darkblue}{\textbf{\ipa{pv̩˩ʈʰɯ˥}}}}\kern2pt]} \hypertarget{pv\string_=\string_Bt`\string_hM\string_M1}{}
\markboth{\textcolor{darkblue}{\textbf{\ipa{pv̩˩ʈʰɯ˧}}}}{}
\textcolor{teal}{\mytextsc{noun}} \hspace{4pt} Tone: LM.
\textcolor{Sepia}{\selectlanguage{english}Feminine given name.} \zh{女性名字。} 
\lhead{\firstmark}
\rhead{\botmark}

\subsection{\hspace{-0.5cm} {\Large \textcolor{darkblue}{\textbf{\ipa{pv˧tsɤ\#˥}}}}\hspace{0.5cm}[\kern2pt{\textcolor{darkblue}{\textbf{\ipa{pv˧tsɤ˧}}}}\kern2pt]} \hypertarget{pv\string_Mts7\#\string_T1}{}
\markboth{\textcolor{darkblue}{\textbf{\ipa{pv˧tsɤ\#˥}}}}{}
\textcolor{teal}{\mytextsc{noun}} \hspace{4pt} Tone: \#H.
\textcolor{Sepia}{\selectlanguage{english}Mortise.} \zh{榫眼。}  ¶ \textcolor{darkblue}{\textbf{\ipa{pv˧tsɤ˧ | ɖɯ˧-ɭɯ˧}}} \textcolor{Sepia}{\selectlanguage{english}a mortise} \zh{一个榫}  
 \zh{量词}: \textcolor{darkblue}{\textbf{\ipa{ɭɯ˧}}}  \mytextsc{clf}: \textcolor{darkblue}{\textbf{\ipa{ɭɯ˧}}} 
\lhead{\firstmark}
\rhead{\botmark}

\newpage
\section*{\centering- \textcolor{darkblue}{\textbf{\ipa{pʰ}}} -}
\subsection{\hspace{-0.5cm} {\Large \textcolor{darkblue}{\textbf{\ipa{pʰæ˧\textsubscript{b}}}}}\hspace{0.5cm}[\kern2pt{\textcolor{darkblue}{\textbf{\ipa{pʰæ˥}}}}\kern2pt]} \hypertarget{p\string_h\{\string_Mb1}{}
\markboth{\textcolor{darkblue}{\textbf{\ipa{pʰæ˧\textsubscript{b}}}}}{}
\textcolor{teal}{\mytextsc{verb}} \hspace{4pt} Tone: M\textsubscript{b}.
\ding{202} \textcolor{Sepia}{\selectlanguage{english}To tie, to fasten (an animal).} \zh{拴(牛……)。}  ¶ \textcolor{darkblue}{\textbf{\ipa{tʰi˧-pʰæ˧+ze˧}}} \textcolor{Sepia}{\selectlanguage{english}\mytextsc{dur} \string_ \mytextsc{pfv}} \zh{\mytextsc{dur} \string_ \mytextsc{pfv}}  
 ¶ \textcolor{darkblue}{\textbf{\ipa{pʰæ˧\textasciitilde{}pʰæ˧}}} \textcolor{Sepia}{\selectlanguage{english}\mytextsc{red}} \zh{\mytextsc{重叠}}  
\ding{203} \textcolor{Sepia}{\selectlanguage{english}To be linked, to have ties: e.g. the members of a family have ties.} \zh{有联系,有缘分,有深的关系。}  ¶ \textcolor{darkblue}{\textbf{\ipa{pʰæ˧\textasciitilde{}pʰæ˧=ɻæ˩ ɲi˩!}}} \textcolor{Sepia}{\selectlanguage{english}They are united / they make up a couple / they are united into a couple (about two young people)} \zh{他们有联系了/他们成了一俩了!(关于两个年轻人)}  

\lhead{\firstmark}
\rhead{\botmark}

\subsection{\hspace{-0.5cm} {\Large \textcolor{darkblue}{\textbf{\ipa{pʰæ˧qʰwɤ˩}}}}\hspace{0.5cm}[\kern2pt{\textcolor{darkblue}{\textbf{\ipa{pʰæ˩qʰwɤ˧˥}}}}\kern2pt]} \hypertarget{p\string_h\{\string_Mq\string_hw7\string_B1}{}
\markboth{\textcolor{darkblue}{\textbf{\ipa{pʰæ˧qʰwɤ˩}}}}{}
\textcolor{teal}{\mytextsc{noun}} \hspace{4pt} Tone: L\#.
\textcolor{Sepia}{\selectlanguage{english}Face.} \zh{脸。}  \zh{量词}: \textcolor{darkblue}{\textbf{\ipa{ɭɯ˧}}}  \mytextsc{clf}: \textcolor{darkblue}{\textbf{\ipa{ɭɯ˧}}} 
\lhead{\firstmark}
\rhead{\botmark}

\subsection{\hspace{-0.5cm} {\Large \textcolor{darkblue}{\textbf{\ipa{pʰæ˧ʂv̩˧-di˧˥}}}}\hspace{0.5cm}[\kern2pt{\textcolor{darkblue}{\textbf{\ipa{xxxx non-correspondance entre le nombre de morphèmes et le nombre de tons de morphèmes}}}}\kern2pt]} \hypertarget{p\string_h\{\string_Ms`v\string_=\string_M-di\string_M\string_T1}{}
\markboth{\textcolor{darkblue}{\textbf{\ipa{pʰæ˧ʂv̩˧-di˧˥}}}}{}
\textcolor{teal}{\mytextsc{noun}} \hspace{4pt} Tone: .
\textcolor{Sepia}{\selectlanguage{english}Scarf.} \zh{围巾。} 
\lhead{\firstmark}
\rhead{\botmark}

\subsection{\hspace{-0.5cm} {\Large \textcolor{darkblue}{\textbf{\ipa{pʰæ˧tɕi˥}}}}\hspace{0.5cm}[\kern2pt{\textcolor{darkblue}{\textbf{\ipa{pʰæ˧tɕi˥}}}}\kern2pt]} \hypertarget{p\string_h\{\string_Mts£i\string_T1}{}
\markboth{\textcolor{darkblue}{\textbf{\ipa{pʰæ˧tɕi˥}}}}{}
\textcolor{teal}{\mytextsc{noun}} \hspace{4pt} Tone: H\#.
\ding{202} \textcolor{Sepia}{\selectlanguage{english}Young man, young chap, young lad.} \zh{小伙子、 青年男子。}  ¶ \textcolor{darkblue}{\textbf{\ipa{pʰæ˧tɕi˥-zo˩}}} \textcolor{Sepia}{\selectlanguage{english}young man} \zh{小伙子}  
 ¶ \textcolor{darkblue}{\textbf{\ipa{pʰæ˧tɕi˥=ɻæ˩}}} \textcolor{Sepia}{\selectlanguage{english}young men} \zh{小伙子们}  
 \zh{量词}: \textcolor{darkblue}{\textbf{\ipa{v̩˧}}} \ding{203} \textcolor{Sepia}{\selectlanguage{english}Name of the first pillar in the main room, the one closest to the door (masculine pillar, the other one being feminine).} \zh{第一根柱子的名称(代表男人、男性的那根柱子)。}  \zh{量词}: \textcolor{darkblue}{\textbf{\ipa{v̩˧}}}  \mytextsc{clf}: \textcolor{darkblue}{\textbf{\ipa{v̩˧}}} \textcolor{darkblue}{\textbf{\ipa{v̩˧}}} 
\lhead{\firstmark}
\rhead{\botmark}

\subsection{\hspace{-0.5cm} {\Large \textcolor{darkblue}{\textbf{\ipa{pʰæ˧ʈʂʰæ˧lo\#˥}}}}\hspace{0.5cm}[\kern2pt{\textcolor{darkblue}{\textbf{\ipa{pʰæ˧ʈʂʰæ˧lo˩}}}}\kern2pt]} \hypertarget{p\string_h\{\string_Mt`s`\string_h\{\string_Mlo\#\string_T1}{}
\markboth{\textcolor{darkblue}{\textbf{\ipa{pʰæ˧ʈʂʰæ˧lo\#˥}}}}{}
\textcolor{teal}{\mytextsc{noun}} \hspace{4pt} Tone: \#H.
\textcolor{Sepia}{\selectlanguage{english}Washbasin, basin to wash one's face.} \zh{脸盆,木盆。}  \zh{量词}: \textcolor{darkblue}{\textbf{\ipa{ɭɯ˧}}}  \mytextsc{clf}: \textcolor{darkblue}{\textbf{\ipa{ɭɯ˧}}} 
\lhead{\firstmark}
\rhead{\botmark}

\subsection{\hspace{-0.5cm} {\Large \textcolor{darkblue}{\textbf{\ipa{pʰæ˧˥}}}}\hspace{0.5cm}[\kern2pt{\textcolor{darkblue}{\textbf{\ipa{pʰæ˧˥}}}}\kern2pt]} \hypertarget{p\string_h\{\string_M\string_T1}{}
\markboth{\textcolor{darkblue}{\textbf{\ipa{pʰæ˧˥}}}}{}
\textcolor{teal}{\mytextsc{verb}} \hspace{4pt} Tone: MH.
\textcolor{Sepia}{\selectlanguage{english}To shove, to push away.} \zh{推搡。}  ¶ \textcolor{darkblue}{\textbf{\ipa{ɖɯ˩-tɕo˧ pʰæ˧˥, | ʈʂʰɯ˧-tɕo˧ pʰæ˧˥}}} \textcolor{Sepia}{\selectlanguage{english}to shove on this side, to shove on that side (e.g. at a station, when lots of people are shoving their way towards the ticket counter)} \zh{东推西挤}  
 ¶ \textcolor{darkblue}{\textbf{\ipa{[Housebuilding2] ʈʂe˧ | le˧-pʰæ˩\textasciitilde{}pʰæ˩}}} \textcolor{Sepia}{\selectlanguage{english}to throw earth here and there: a chicken is scratching the soil to find food, and sends spurts of earth here and there} \zh{将土扔这里扔那里:一只鸡在抓地找吃的,让土飞这里飞那里}  

\lhead{\firstmark}
\rhead{\botmark}

\subsection{\hspace{-0.5cm} {\Large \textcolor{darkblue}{\textbf{\ipa{pʰæ˧˥\textsubscript{a}}}}}\hspace{0.5cm}[\kern2pt{\textcolor{darkblue}{\textbf{\ipa{pʰæ˧˥}}}}\kern2pt]} \hypertarget{p\string_h\{\string_M\string_Ta1}{}
\markboth{\textcolor{darkblue}{\textbf{\ipa{pʰæ˧˥\textsubscript{a}}}}}{}
\textcolor{teal}{\mytextsc{classifier}} \hspace{4pt} Tone: MH\textsubscript{a}.
\textcolor{Sepia}{\selectlanguage{english}Classifier for flat objects, e.g. a sheet (of paper).} \zh{量词:平面的东西,如:纸(一张、一片)。} 
\lhead{\firstmark}
\rhead{\botmark}

\subsection{\hspace{-0.5cm} {\Large \textcolor{darkblue}{\textbf{\ipa{pʰe˧}}}}\hspace{0.5cm}[\kern2pt{\textcolor{darkblue}{\textbf{\ipa{pʰe˥}}}}\kern2pt]} \hypertarget{p\string_he\string_M1}{}
\markboth{\textcolor{darkblue}{\textbf{\ipa{pʰe˧}}}}{}
\textcolor{teal}{\mytextsc{interjection}} \hspace{4pt} Tone: M.
\textcolor{Sepia}{\selectlanguage{english}Interjection: No way! The speaker signals that the interlocutor is making wrong statements, and that (s)he (the speaker) will now put forward different views.} \zh{呸!(表示唾弃的感叹词)。} 
\lhead{\firstmark}
\rhead{\botmark}

\subsection{\hspace{-0.5cm} {\Large \textcolor{darkblue}{\textbf{\ipa{pʰɤ˧bɤ˧}}}}\hspace{0.5cm}[\kern2pt{\textcolor{darkblue}{\textbf{\ipa{pʰɤ˧bɤ˧}}}}\kern2pt]} \hypertarget{p\string_h7\string_Mb7\string_M1}{}
\markboth{\textcolor{darkblue}{\textbf{\ipa{pʰɤ˧bɤ˧}}}}{}
\textcolor{teal}{\mytextsc{noun}} \hspace{4pt} Tone: M.
\textcolor{Sepia}{\selectlanguage{english}Gift, present (typical gifts are tobacco, tea leaf, candies, and wine; one does not usually offer clothes, apart from specific ritual occasions).} \zh{礼物。}  ¶ \textcolor{darkblue}{\textbf{\ipa{pʰɤ˧bɤ˧ po˧-tsʰɯ˧˥}}} \textcolor{Sepia}{\selectlanguage{english}to bring gifts} \zh{带礼物}  
 ¶ \textcolor{darkblue}{\textbf{\ipa{ʈʂʰɯ˧ | ʈæ˧ʂɯ˧ ki˧-hĩ˧ pʰɤ˧bɤ˧ ŋi˩.}}} \textcolor{Sepia}{\selectlanguage{english}This is a gift from Dashi!} \zh{这是达石给的礼物!}  
 ¶ \textcolor{darkblue}{\textbf{\ipa{ʈʂʰɯ˧ | ʈæ˧ʂɯ˧ tʰi˧-ki˧-hĩ˧ pʰɤ˧bɤ˧ ŋi˩.}}} \textcolor{Sepia}{\selectlanguage{english}This is a gift from Dashi! / Here is a gift for you from Dashi!} \zh{这是达石送你的礼物!}  
 \zh{量词}: \textcolor{darkblue}{\textbf{\ipa{kʰwɤ˥}}}  \mytextsc{clf}: \textcolor{darkblue}{\textbf{\ipa{kʰwɤ˥}}} 
\lhead{\firstmark}
\rhead{\botmark}

\subsection{\hspace{-0.5cm} {\Large \textcolor{darkblue}{\textbf{\ipa{pʰɤ˧fv̩˩}}}}\hspace{0.5cm}[\kern2pt{\textcolor{darkblue}{\textbf{\ipa{pʰɤ˧fv̩˩}}}}\kern2pt]} \hypertarget{p\string_h7\string_Mfv\string_=\string_B1}{}
\markboth{\textcolor{darkblue}{\textbf{\ipa{pʰɤ˧fv̩˩}}}}{}
\textcolor{teal}{\mytextsc{noun}} \hspace{4pt} Tone: L\#.
\textcolor{Sepia}{\selectlanguage{english}Teapot.} \zh{茶壶。}  Borrowing: Chinese  \zh{壶?}

\lhead{\firstmark}
\rhead{\botmark}

\subsection{\hspace{-0.5cm} {\Large \textcolor{darkblue}{\textbf{\ipa{pʰɤ˧pʰv̩\#˥}}}}\hspace{0.5cm}[\kern2pt{\textcolor{darkblue}{\textbf{\ipa{pʰɤ˩pʰv̩˩˥}}}}\kern2pt]} \hypertarget{p\string_h7\string_Mp\string_hv\string_=\#\string_T1}{}
\markboth{\textcolor{darkblue}{\textbf{\ipa{pʰɤ˧pʰv̩\#˥}}}}{}
\textcolor{teal}{\mytextsc{noun}} \hspace{4pt} Tone: \#H.
\textcolor{Sepia}{\selectlanguage{english}Male jackal.} \zh{公豺。}  \zh{量词}: \textcolor{darkblue}{\textbf{\ipa{mi˩}}}  \mytextsc{clf}: \textcolor{darkblue}{\textbf{\ipa{mi˩}}} 
\lhead{\firstmark}
\rhead{\botmark}

\subsection{\hspace{-0.5cm} {\Large \textcolor{darkblue}{\textbf{\ipa{pʰɤ˩mi˩}}}}\hspace{0.5cm}[\kern2pt{\textcolor{darkblue}{\textbf{\ipa{pʰɤ˩mi˩˥}}}}\kern2pt]} \hypertarget{p\string_h7\string_Bmi\string_B1}{}
\markboth{\textcolor{darkblue}{\textbf{\ipa{pʰɤ˩mi˩}}}}{}
\textcolor{teal}{\mytextsc{noun}} \hspace{4pt} Tone: L.
\textcolor{Sepia}{\selectlanguage{english}Female jackal.} \zh{母豺。}  \zh{量词}: \textcolor{darkblue}{\textbf{\ipa{mi˩}}}  \mytextsc{clf}: \textcolor{darkblue}{\textbf{\ipa{mi˩}}} 
\lhead{\firstmark}
\rhead{\botmark}

\subsection{\hspace{-0.5cm} {\Large \textcolor{darkblue}{\textbf{\ipa{pʰɤ˩-so˩\textasciitilde{}so˥}}}}\hspace{0.5cm}[\kern2pt{\textcolor{darkblue}{\textbf{\ipa{xxxx non-correspondance entre le nombre de morphèmes et le nombre de tons de morphèmes}}}}\kern2pt]} \hypertarget{p\string_h7\string_B-so\string_B~so\string_T1}{}
\markboth{\textcolor{darkblue}{\textbf{\ipa{pʰɤ˩-so˩\textasciitilde{}so˥}}}}{}
\textcolor{teal}{\mytextsc{adjective}} \hspace{4pt} Tone: .
\textcolor{Sepia}{\selectlanguage{english}Loose (the soil is loose after being forked over).} \zh{松(土)。}  ¶ \textcolor{darkblue}{\textbf{\ipa{ʈʂe˧ | pʰɤ˩-so˩\textasciitilde{}so˥-gv̩˩}}} \textcolor{Sepia}{\selectlanguage{english}the soil is loose, the soil has been loosened} \zh{土是松的}  

\lhead{\firstmark}
\rhead{\botmark}

\subsection{\hspace{-0.5cm} {\Large \textcolor{darkblue}{\textbf{\ipa{pʰɤ˩zo˩}}}}\hspace{0.5cm}[\kern2pt{\textcolor{darkblue}{\textbf{\ipa{pʰɤ˩zo˩˥}}}}\kern2pt]} \hypertarget{p\string_h7\string_Bzo\string_B1}{}
\markboth{\textcolor{darkblue}{\textbf{\ipa{pʰɤ˩zo˩}}}}{}
\textcolor{teal}{\mytextsc{noun}} \hspace{4pt} Tone: L.
\textcolor{Sepia}{\selectlanguage{english}Baby jackal.} \zh{豺崽子。}  \zh{量词}: \textcolor{darkblue}{\textbf{\ipa{ɭɯ˧}}}  \mytextsc{clf}: \textcolor{darkblue}{\textbf{\ipa{ɭɯ˧}}} 
\lhead{\firstmark}
\rhead{\botmark}

\subsection{\hspace{-0.5cm} {\Large \textcolor{darkblue}{\textbf{\ipa{pʰɤ˩˧}}}}\hspace{0.5cm}[\kern2pt{\textcolor{darkblue}{\textbf{\ipa{pʰɤ˩˥}}}}\kern2pt]} \hypertarget{p\string_h7\string_B\string_M1}{}
\markboth{\textcolor{darkblue}{\textbf{\ipa{pʰɤ˩˧}}}}{}
\textcolor{teal}{\mytextsc{noun}} \hspace{4pt} Tone: LM.
\textcolor{Sepia}{\selectlanguage{english}Jackal, hyena.} \zh{豺。}  ¶ \textcolor{darkblue}{\textbf{\ipa{pʰɤ˩ hwæ˧-ze˧}}} \textcolor{Sepia}{\selectlanguage{english}...bought (a/the) jackal} \zh{买了豺}  
 ¶ \textcolor{darkblue}{\textbf{\ipa{pʰɤ˩ dzɯ˧-ze˩}}} \textcolor{Sepia}{\selectlanguage{english}...ate jackal} \zh{吃了豺}  
 \zh{量词}: \textcolor{darkblue}{\textbf{\ipa{mi˩}}}  \mytextsc{clf}: \textcolor{darkblue}{\textbf{\ipa{mi˩}}} 
\lhead{\firstmark}
\rhead{\botmark}

\subsection{\hspace{-0.5cm} {\Large \textcolor{darkblue}{\textbf{\ipa{pʰi˧}}}}\hspace{0.5cm}[\kern2pt{\textcolor{darkblue}{\textbf{\ipa{pʰi˥}}}}\kern2pt]} \hypertarget{p\string_hi\string_M1}{}
\markboth{\textcolor{darkblue}{\textbf{\ipa{pʰi˧}}}}{}
\textcolor{teal}{\mytextsc{noun}} \hspace{4pt} Tone: M.
\textcolor{Sepia}{\selectlanguage{english}Linen cloth.} \zh{麻布,亚麻布。}  ¶ \textcolor{darkblue}{\textbf{\ipa{pʰi˩ dɑ˩˥}}} \textcolor{Sepia}{\selectlanguage{english}to weave linen} \zh{织麻布}  
 \zh{量词}: \textcolor{darkblue}{\textbf{\ipa{kʰwɤ˥}}}  \mytextsc{clf}: \textcolor{darkblue}{\textbf{\ipa{kʰwɤ˥}}} 
\lhead{\firstmark}
\rhead{\botmark}

\subsection{\hspace{-0.5cm} {\Large \textcolor{darkblue}{\textbf{\ipa{pʰi˧\textsubscript{b}}}}}\hspace{0.5cm}[\kern2pt{\textcolor{darkblue}{\textbf{\ipa{pʰi˧˥}}}}\kern2pt]} \hypertarget{p\string_hi\string_Mb1}{}
\markboth{\textcolor{darkblue}{\textbf{\ipa{pʰi˧\textsubscript{b}}}}}{}
\textcolor{teal}{\mytextsc{verb}} \hspace{4pt} Tone: M\textsubscript{b}.
\textcolor{Sepia}{\selectlanguage{english}To winnow with a fan.} \zh{簸。}  ¶ \textcolor{darkblue}{\textbf{\ipa{hɑ˧ pʰi˧}}} \textcolor{Sepia}{\selectlanguage{english}to winnow cereals} \zh{簸粮食}  
 ¶ \textcolor{darkblue}{\textbf{\ipa{pʰi˧\textasciitilde{}pʰi˧}}} \textcolor{Sepia}{\selectlanguage{english}\mytextsc{red}} \zh{\mytextsc{重叠:簸一簸}}  
 ¶ \textcolor{darkblue}{\textbf{\ipa{le˧-pʰi˧(-ze˧)}}} \textcolor{Sepia}{\selectlanguage{english}\mytextsc{accomp} \string_ (\mytextsc{pfv})} \zh{簸了}  

\lhead{\firstmark}
\rhead{\botmark}

\subsection{\hspace{-0.5cm} {\Large \textcolor{darkblue}{\textbf{\ipa{pʰi˧kʰv̩˧}}}}\hspace{0.5cm}[\kern2pt{\textcolor{darkblue}{\textbf{\ipa{pʰi˩kʰv̩˥}}}}\kern2pt]} \hypertarget{p\string_hi\string_Mk\string_hv\string_=\string_M1}{}
\markboth{\textcolor{darkblue}{\textbf{\ipa{pʰi˧kʰv̩˧}}}}{}
\textcolor{teal}{\mytextsc{noun}} \hspace{4pt} Tone: M.
\textcolor{Sepia}{\selectlanguage{english}Dustpan, wicker scoop, dirt-scooping implement.} \zh{畚箕。}  \zh{量词}: \textcolor{darkblue}{\textbf{\ipa{nɑ˧}}}  \mytextsc{clf}: \textcolor{darkblue}{\textbf{\ipa{nɑ˧}}} 
\lhead{\firstmark}
\rhead{\botmark}

\subsection{\hspace{-0.5cm} {\Large \textcolor{darkblue}{\textbf{\ipa{pʰi˧kʰv̩˧}}}}\hspace{0.5cm}[\kern2pt{\textcolor{darkblue}{\textbf{\ipa{pʰi˧kʰv̩˧}}}}\kern2pt]} \hypertarget{p\string_hi\string_Mk\string_hv\string_=\string_M1}{}
\markboth{\textcolor{darkblue}{\textbf{\ipa{pʰi˧kʰv̩˧}}}}{}
\textcolor{teal}{\mytextsc{noun}} \hspace{4pt} Tone: M.
\textcolor{Sepia}{\selectlanguage{english}Clamshell.} \zh{贝壳。} 
\lhead{\firstmark}
\rhead{\botmark}

\subsection{\hspace{-0.5cm} {\Large \textcolor{darkblue}{\textbf{\ipa{pʰi˧li˩}}}}\hspace{0.5cm}[\kern2pt{\textcolor{darkblue}{\textbf{\ipa{pʰi˧li˧}}}}\kern2pt]} \hypertarget{p\string_hi\string_Mli\string_B1}{}
\markboth{\textcolor{darkblue}{\textbf{\ipa{pʰi˧li˩}}}}{}
\textcolor{teal}{\mytextsc{noun}} \hspace{4pt} Tone: L\#.
\textcolor{Sepia}{\selectlanguage{english}Butterfly.} \zh{蝴蝶。}  \zh{量词}: \textcolor{darkblue}{\textbf{\ipa{mi˩}}}  \mytextsc{clf}: \textcolor{darkblue}{\textbf{\ipa{mi˩}}} 
\lhead{\firstmark}
\rhead{\botmark}

\subsection{\hspace{-0.5cm} {\Large \textcolor{darkblue}{\textbf{\ipa{pʰi˧mo˩}}} \textsubscript{1}}\hspace{0.5cm}[\kern2pt{\textcolor{darkblue}{\textbf{\ipa{pʰi˧mo˩}}}}\kern2pt]} \hypertarget{p\string_hi\string_Mmo\string_B1}{}
\markboth{\textcolor{darkblue}{\textbf{\ipa{pʰi˧mo˩}}} \textsubscript{1}}{}
\textcolor{teal}{\mytextsc{noun}} \hspace{4pt} Tone: L\#.
\textcolor{Sepia}{\selectlanguage{english}Winnowing fan.} \zh{簸箕(用来簸粮食等)。}  \zh{量词}: \textcolor{darkblue}{\textbf{\ipa{nɑ˧}}}  \mytextsc{clf}: \textcolor{darkblue}{\textbf{\ipa{nɑ˧}}} \textit{See:} \hyperlink{}{\textcolor{darkblue}{\textbf{\ipa{pʰi˧mo˩}}} \textsubscript{2}} 
\lhead{\firstmark}
\rhead{\botmark}

\subsection{\hspace{-0.5cm} {\Large \textcolor{darkblue}{\textbf{\ipa{pʰi˧mo˩}}} \textsubscript{2}}\hspace{0.5cm}[\kern2pt{\textcolor{darkblue}{\textbf{\ipa{pʰi˧mo˩}}}}\kern2pt]} \hypertarget{p\string_hi\string_Mmo\string_B2}{}
\markboth{\textcolor{darkblue}{\textbf{\ipa{pʰi˧mo˩}}} \textsubscript{2}}{}
\textcolor{teal}{\mytextsc{noun}} \hspace{4pt} Tone: L\#.
\textcolor{Sepia}{\selectlanguage{english}Snare to catch birds.} \zh{抓鸟的圈套。}  ¶ \textcolor{darkblue}{\textbf{\ipa{v̩˩dze˩ qo˥-di˩, | pʰi˧mo˩!}}} \textcolor{Sepia}{\selectlanguage{english}The thing to catch birds is called “snare”!} \zh{抓鸟的东西,(叫做)圈套!}  
\textit{See:} \hyperlink{}{\textcolor{darkblue}{\textbf{\ipa{pʰi˧mo˩}}} \textsubscript{1}} 
\lhead{\firstmark}
\rhead{\botmark}

\subsection{\hspace{-0.5cm} {\Large \textcolor{darkblue}{\textbf{\ipa{pʰi˧tsʰo\#˥}}}}\hspace{0.5cm}[\kern2pt{\textcolor{darkblue}{\textbf{\ipa{pʰi˧tsʰo˧}}}}\kern2pt]} \hypertarget{p\string_hi\string_Mts\string_ho\#\string_T1}{}
\markboth{\textcolor{darkblue}{\textbf{\ipa{pʰi˧tsʰo\#˥}}}}{}
\textcolor{teal}{\mytextsc{noun}} \hspace{4pt} Tone: \#H.
\textcolor{Sepia}{\selectlanguage{english}Masculine given name.} \zh{男性名字。} 
\lhead{\firstmark}
\rhead{\botmark}

\subsection{\hspace{-0.5cm} {\Large \textcolor{darkblue}{\textbf{\ipa{pʰi˧ʈʂæ˧}}}}\hspace{0.5cm}[\kern2pt{\textcolor{darkblue}{\textbf{\ipa{pʰi˧ʈʂæ˧}}}}\kern2pt]} \hypertarget{p\string_hi\string_Mt`s`\{\string_M1}{}
\markboth{\textcolor{darkblue}{\textbf{\ipa{pʰi˧ʈʂæ˧}}}}{}
\textcolor{teal}{\mytextsc{verb}} \hspace{4pt} Tone: M.
\textcolor{Sepia}{\selectlanguage{english}To drape oneself in a felt cape, to drape a piece of felt over one's shoulders.} \zh{披毡(汉语借词)。}  Borrowing: Chinese  \zh{披毡}

\lhead{\firstmark}
\rhead{\botmark}

\subsection{\hspace{-0.5cm} {\Large \textcolor{darkblue}{\textbf{\ipa{pʰi˩}}}}\hspace{0.5cm}[\kern2pt{\textcolor{darkblue}{\textbf{\ipa{pʰi˥}}}}\kern2pt]} \hypertarget{p\string_hi\string_B1}{}
\markboth{\textcolor{darkblue}{\textbf{\ipa{pʰi˩}}}}{}
\textcolor{teal}{\mytextsc{adjective}} \hspace{4pt} Tone: L.
\textcolor{Sepia}{\selectlanguage{english}Flat.} \zh{平(汉语借词)。}  Borrowing: Chinese  \zh{平}

\lhead{\firstmark}
\rhead{\botmark}

\subsection{\hspace{-0.5cm} {\Large \textcolor{darkblue}{\textbf{\ipa{pʰi˩hæ˩}}}}\hspace{0.5cm}[\kern2pt{\textcolor{darkblue}{\textbf{\ipa{pʰi˧hæ˧}}}}\kern2pt]} \hypertarget{p\string_hi\string_Bh\{\string_B1}{}
\markboth{\textcolor{darkblue}{\textbf{\ipa{pʰi˩hæ˩}}}}{}
\textcolor{teal}{\mytextsc{noun}} \hspace{4pt} Tone: L.
\textcolor{Sepia}{\selectlanguage{english}Sandal.} \zh{凉鞋。汉语借词:第一个音节:皮,第二个音节:不明确,同\textcolor{darkblue}{\textbf{\ipa{/tɕæ˧hæ˩/}}}。}  Borrowing: Chinese  \zh{皮} (second syllable: not identified)
 \zh{量词}: \textcolor{darkblue}{\textbf{\ipa{dzi˧}}}  \mytextsc{clf}: \textcolor{darkblue}{\textbf{\ipa{dzi˧}}} 
\lhead{\firstmark}
\rhead{\botmark}

\subsection{\hspace{-0.5cm} {\Large \textcolor{darkblue}{\textbf{\ipa{pʰi˩ko˧}}}}\hspace{0.5cm}[\kern2pt{\textcolor{darkblue}{\textbf{\ipa{pʰi˩ko˩˥}}}}\kern2pt]} \hypertarget{p\string_hi\string_Bko\string_M1}{}
\markboth{\textcolor{darkblue}{\textbf{\ipa{pʰi˩ko˧}}}}{}
\textcolor{teal}{\mytextsc{noun}} \hspace{4pt} Tone: LM.
\textcolor{Sepia}{\selectlanguage{english}Apple.} \zh{苹果。}  Borrowing: Chinese  \zh{苹果}
 \zh{量词}: \textcolor{darkblue}{\textbf{\ipa{ɭɯ˧}}}  \mytextsc{clf}: \textcolor{darkblue}{\textbf{\ipa{ɭɯ˧}}} 
\lhead{\firstmark}
\rhead{\botmark}

\subsection{\hspace{-0.5cm} {\Large \textcolor{darkblue}{\textbf{\ipa{pʰi˩tʰɑ˩}}}}\hspace{0.5cm}[\kern2pt{\textcolor{darkblue}{\textbf{\ipa{pʰi˩tʰɑ˩˥}}}}\kern2pt]} \hypertarget{p\string_hi\string_Bt\string_hA\string_B1}{}
\markboth{\textcolor{darkblue}{\textbf{\ipa{pʰi˩tʰɑ˩}}}}{}
\textcolor{teal}{\mytextsc{noun}} \hspace{4pt} Tone: L.
\textcolor{Sepia}{\selectlanguage{english}Tinder, touchwood.} \zh{火草。}  \zh{量词}: \textcolor{darkblue}{\textbf{\ipa{po˧}}}  \mytextsc{clf}: \textcolor{darkblue}{\textbf{\ipa{po˧}}} 
\lhead{\firstmark}
\rhead{\botmark}

\subsection{\hspace{-0.5cm} {\Large \textcolor{darkblue}{\textbf{\ipa{pʰi˧˥}}}}\hspace{0.5cm}[\kern2pt{\textcolor{darkblue}{\textbf{\ipa{pʰi˥}}}}\kern2pt]} \hypertarget{p\string_hi\string_M\string_T1}{}
\markboth{\textcolor{darkblue}{\textbf{\ipa{pʰi˧˥}}}}{}
\textcolor{teal}{\mytextsc{verb}} \hspace{4pt} Tone: MH.
\textcolor{Sepia}{\selectlanguage{english}To vomit.} \zh{呕吐。}  ¶ \textcolor{darkblue}{\textbf{\ipa{le˧-pʰi˧-ze˥}}} \textcolor{Sepia}{\selectlanguage{english}\mytextsc{accomp} \string_ \mytextsc{pfv}} \zh{呕吐了}  

\lhead{\firstmark}
\rhead{\botmark}

\subsection{\hspace{-0.5cm} {\Large \textcolor{darkblue}{\textbf{\ipa{pʰo˥}}}}\hspace{0.5cm}[\kern2pt{\textcolor{darkblue}{\textbf{\ipa{pʰo˥}}}}\kern2pt]} \hypertarget{p\string_ho\string_T1}{}
\markboth{\textcolor{darkblue}{\textbf{\ipa{pʰo˥}}}}{}
\textcolor{teal}{\mytextsc{verb}} \hspace{4pt} Tone: H.
\textcolor{Sepia}{\selectlanguage{english}To drape oneself in (a cape, a piece of fabric), without buttoning up.} \zh{披上(不系扣子)。}  ¶ \textcolor{darkblue}{\textbf{\ipa{mɤ˧-pʰo˥}}} \textcolor{Sepia}{\selectlanguage{english}\mytextsc{neg}} \zh{不披}  
 ¶ \textcolor{darkblue}{\textbf{\ipa{bɑ˩lɑ˩ qɑ˩-pʰo˩˥}}} \textcolor{Sepia}{\selectlanguage{english}to put on a shirt without buttoning up} \zh{披上衣服(不系扣子)}  

\lhead{\firstmark}
\rhead{\botmark}

\subsection{\hspace{-0.5cm} {\Large \textcolor{darkblue}{\textbf{\ipa{pʰo˧\textsubscript{b}}}}}\hspace{0.5cm}[\kern2pt{\textcolor{darkblue}{\textbf{\ipa{pʰo˥}}}}\kern2pt]} \hypertarget{p\string_ho\string_Mb1}{}
\markboth{\textcolor{darkblue}{\textbf{\ipa{pʰo˧\textsubscript{b}}}}}{}
\textcolor{teal}{\mytextsc{verb}} \hspace{4pt} Tone: M\textsubscript{b}.
\textcolor{Sepia}{\selectlanguage{english}To open (e.g. a door).} \zh{打开(例如:开门)。}  ¶ \textcolor{darkblue}{\textbf{\ipa{gɤ˩-pʰo˧ (-ze˧)}}} \textcolor{Sepia}{\selectlanguage{english}to open up} \zh{打开}  
 ¶ \textcolor{darkblue}{\textbf{\ipa{kʰi˧ pʰo˧}}} \textcolor{Sepia}{\selectlanguage{english}to open the door} \zh{开门}  
 ¶ \textcolor{darkblue}{\textbf{\ipa{kʰi˧mi˧ le˧-pʰo˧}}} \textcolor{Sepia}{\selectlanguage{english}to open the door} \zh{开门}  
 ¶ \textcolor{darkblue}{\textbf{\ipa{tso˧\textasciitilde{}tso˧ pʰo˧}}} \textcolor{Sepia}{\selectlanguage{english}to open something} \zh{打开东西}  

\lhead{\firstmark}
\rhead{\botmark}

\subsection{\hspace{-0.5cm} {\Large \textcolor{darkblue}{\textbf{\ipa{pʰo˩\textsubscript{a}}}}}\hspace{0.5cm}[\kern2pt{\textcolor{darkblue}{\textbf{\ipa{pʰo˩˥}}}}\kern2pt]} \hypertarget{p\string_ho\string_Ba1}{}
\markboth{\textcolor{darkblue}{\textbf{\ipa{pʰo˩\textsubscript{a}}}}}{}
\textcolor{teal}{\mytextsc{verb}} \hspace{4pt} Tone: L\textsubscript{a}.
\textcolor{Sepia}{\selectlanguage{english}To flee, to rush away, to escape.} \zh{逃,逃跑,逃掉。}  ¶ \textcolor{darkblue}{\textbf{\ipa{le˧-pʰo˩-ze˩}}} \textcolor{Sepia}{\selectlanguage{english}\mytextsc{accomp} \string_ \mytextsc{pfv}} \zh{逃跑了}  
 ¶ \textcolor{darkblue}{\textbf{\ipa{le˧-pʰo˩-hɯ˩-ze˩!}}} \textcolor{Sepia}{\selectlanguage{english}(She/he) has escaped!} \zh{(他)逃跑了!}  

\lhead{\firstmark}
\rhead{\botmark}

\subsection{\hspace{-0.5cm} {\Large \textcolor{darkblue}{\textbf{\ipa{pʰo˩lɑ˧˥}}}}\hspace{0.5cm}[\kern2pt{\textcolor{darkblue}{\textbf{\ipa{pʰo˧lɑ˩}}}}\kern2pt]} \hypertarget{p\string_ho\string_BlA\string_M\string_T1}{}
\markboth{\textcolor{darkblue}{\textbf{\ipa{pʰo˩lɑ˧˥}}}}{}
\textcolor{teal}{\mytextsc{verb}} \hspace{4pt} Tone: LM+MH\#.
\textcolor{Sepia}{\selectlanguage{english}To wage war.} \zh{战争、打仗。}  ¶ \textcolor{darkblue}{\textbf{\ipa{mɤ˧-pʰo˩lɑ˩}}} \textcolor{Sepia}{\selectlanguage{english}\mytextsc{neg}} \zh{不打仗}  
 ¶ \textcolor{darkblue}{\textbf{\ipa{pʰo˩lɑ˧˥ | ɖɯ˧-kʰv̩˧˥}}} \textcolor{Sepia}{\selectlanguage{english}a year of war, a year during which there was war} \zh{打仗的一年}  

\lhead{\firstmark}
\rhead{\botmark}

\subsection{\hspace{-0.5cm} {\Large \textcolor{darkblue}{\textbf{\ipa{pʰo˧˥}}}}\hspace{0.5cm}[\kern2pt{\textcolor{darkblue}{\textbf{\ipa{pʰo˧˥}}}}\kern2pt]} \hypertarget{p\string_ho\string_M\string_T1}{}
\markboth{\textcolor{darkblue}{\textbf{\ipa{pʰo˧˥}}}}{}
\textcolor{teal}{\mytextsc{verb}} \hspace{4pt} Tone: MH.
\textcolor{Sepia}{\selectlanguage{english}To sow.} \zh{撒 (撒种子)、播(种子)。}  ¶ \textcolor{darkblue}{\textbf{\ipa{ɻæ˩ pʰo˧˥}}} \textcolor{Sepia}{\selectlanguage{english}to sow seeds} \zh{撒种子}  

\lhead{\firstmark}
\rhead{\botmark}

\subsection{\hspace{-0.5cm} {\Large \textcolor{darkblue}{\textbf{\ipa{pʰo˧˥\textsubscript{a}}}}}\hspace{0.5cm}[\kern2pt{\textcolor{darkblue}{\textbf{\ipa{pʰo˧˥}}}}\kern2pt]} \hypertarget{p\string_ho\string_M\string_Ta1}{}
\markboth{\textcolor{darkblue}{\textbf{\ipa{pʰo˧˥\textsubscript{a}}}}}{}
\textcolor{teal}{\mytextsc{classifier}} \hspace{4pt} Tone: MH\textsubscript{a}.
\textcolor{Sepia}{\selectlanguage{english}A member of a pair; also used for some large domestic animals, e.g. oxen.} \zh{量词:一对中的一只(例如一只鞋),一头大牲畜(牛……)。} 
\lhead{\firstmark}
\rhead{\botmark}

\subsection{\hspace{-0.5cm} {\Large \textcolor{darkblue}{\textbf{\ipa{pʰv̩˧}}} \textsubscript{1}}\hspace{0.5cm}[\kern2pt{\textcolor{darkblue}{\textbf{\ipa{pʰv̩˥}}}}\kern2pt]} \hypertarget{p\string_hv\string_=\string_M1}{}
\markboth{\textcolor{darkblue}{\textbf{\ipa{pʰv̩˧}}} \textsubscript{1}}{}
\textcolor{teal}{\mytextsc{noun}} \hspace{4pt} Tone: M.
\textcolor{Sepia}{\selectlanguage{english}Male.} \zh{公的。}  ¶ \textcolor{darkblue}{\textbf{\ipa{ʈʂʰɯ˧, | pʰv̩˧ ɲi˩!}}} \textcolor{Sepia}{\selectlanguage{english}It's a male!} \zh{这(只动物)是公的!}  
 ¶ \textcolor{darkblue}{\textbf{\ipa{ʈʂʰɯ˧, | pʰv̩˧!}}} \textcolor{Sepia}{\selectlanguage{english}It's a male!} \zh{这(只动物)是公的!}  
 \zh{量词}: \textcolor{darkblue}{\textbf{\ipa{v̩˧}}}  \mytextsc{clf}: \textcolor{darkblue}{\textbf{\ipa{v̩˧}}} 
\lhead{\firstmark}
\rhead{\botmark}

\subsection{\hspace{-0.5cm} {\Large \textcolor{darkblue}{\textbf{\ipa{pʰv̩˧}}} \textsubscript{2}}\hspace{0.5cm}[\kern2pt{\textcolor{darkblue}{\textbf{\ipa{pʰv̩˥}}}}\kern2pt]} \hypertarget{p\string_hv\string_=\string_M2}{}
\markboth{\textcolor{darkblue}{\textbf{\ipa{pʰv̩˧}}} \textsubscript{2}}{}
\textcolor{teal}{\mytextsc{noun}} \hspace{4pt} Tone: M.
\textcolor{Sepia}{\selectlanguage{english}Price.} \zh{价格。} 
\lhead{\firstmark}
\rhead{\botmark}

\subsection{\hspace{-0.5cm} {\Large \textcolor{darkblue}{\textbf{\ipa{pʰv̩˧˥}}}}\hspace{0.5cm}[\kern2pt{\textcolor{darkblue}{\textbf{\ipa{pʰv̩˧˥}}}}\kern2pt]} \hypertarget{p\string_hv\string_=\string_M\string_T1}{}
\markboth{\textcolor{darkblue}{\textbf{\ipa{pʰv̩˧˥}}}}{}
\textcolor{teal}{\mytextsc{verb}} \hspace{4pt} Tone: MH.
\textcolor{Sepia}{\selectlanguage{english}To take off (clothes).} \zh{脱(衣服)。} 
\lhead{\firstmark}
\rhead{\botmark}

\subsection{\hspace{-0.5cm} {\Large \textcolor{darkblue}{\textbf{\ipa{pʰv̩˧˥}}} \textsubscript{1}}\hspace{0.5cm}[\kern2pt{\textcolor{darkblue}{\textbf{\ipa{pʰv̩˧˥}}}}\kern2pt]} \hypertarget{p\string_hv\string_=\string_M\string_T1}{}
\markboth{\textcolor{darkblue}{\textbf{\ipa{pʰv̩˧˥}}} \textsubscript{1}}{}
\textcolor{teal}{\mytextsc{verb}} \hspace{4pt} Tone: MH.
\textcolor{Sepia}{\selectlanguage{english}To boil, to cook in water.} \zh{煮(鸡蛋、洋芋……)。}  ¶ \textcolor{darkblue}{\textbf{\ipa{jɤ˩jo˥ F | pʰv̩˧˥! | æ˩ʁv̩˩˥ F | pʰv̩˧˥!}}} \textcolor{Sepia}{\selectlanguage{english}Potatoes can be boiled! Eggs can be boiled!} \zh{洋芋,是(可以)煮的!鸡蛋,是(可以)煮的!}  
 ¶ \textcolor{darkblue}{\textbf{\ipa{æ˩ʁv̩˩ pʰv̩˥}}} \textcolor{Sepia}{\selectlanguage{english}to cook eggs in water} \zh{煮鸡蛋}  
 ¶ \textcolor{darkblue}{\textbf{\ipa{jɤ˩jo˥ pʰv̩˩}}} \textcolor{Sepia}{\selectlanguage{english}to boil potatoes} \zh{煮洋芋}  
 ¶ \textcolor{darkblue}{\textbf{\ipa{le˧-pʰv̩˧ | le˧-mv̩˩-ze˩!}}} \textcolor{Sepia}{\selectlanguage{english}It is cooked (from boiling)! / It has been boiled to the point when it is now well-done/cooked} \zh{煮熟了!}  

\lhead{\firstmark}
\rhead{\botmark}

\subsection{\hspace{-0.5cm} {\Large \textcolor{darkblue}{\textbf{\ipa{pʰv̩˧˥}}} \textsubscript{2}}\hspace{0.5cm}[\kern2pt{\textcolor{darkblue}{\textbf{\ipa{pʰv̩˧˥}}}}\kern2pt]} \hypertarget{p\string_hv\string_=\string_M\string_T2}{}
\markboth{\textcolor{darkblue}{\textbf{\ipa{pʰv̩˧˥}}} \textsubscript{2}}{}
\textcolor{teal}{\mytextsc{verb}} \hspace{4pt} Tone: MH.
\textcolor{Sepia}{\selectlanguage{english}To pour, to spill.} \zh{倒(酒……),倒出来。}  ¶ \textcolor{darkblue}{\textbf{\ipa{ʐɯ˧ pʰv̩˧˥}}} \textcolor{Sepia}{\selectlanguage{english}to pour wine, to serve wine} \zh{倒酒}  
 ¶ \textcolor{darkblue}{\textbf{\ipa{dʑɯ˩ pʰv̩˩˥}}} \textcolor{Sepia}{\selectlanguage{english}to pour water, to serve water (as a beverage)} \zh{倒水}  
 ¶ \textcolor{darkblue}{\textbf{\ipa{mv̩˩tɕo˧ pʰv̩˧˥}}} \textcolor{Sepia}{\selectlanguage{english}to pour out, to spill on the floor} \zh{往外倒}  
 ¶ \textcolor{darkblue}{\textbf{\ipa{[F5] ɖæ˩˥ | mv̩˩tɕo˧ pʰv̩˥}}} \textcolor{Sepia}{\selectlanguage{english}to throw out garbage, to pour garbage (out of a bucket onto a dirt heap)} \zh{倒垃圾}  

\lhead{\firstmark}
\rhead{\botmark}

\subsection{\hspace{-0.5cm} {\Large \textcolor{darkblue}{\textbf{\ipa{pʰv̩˧˥}}} \textsubscript{3}}\hspace{0.5cm}[\kern2pt{\textcolor{darkblue}{\textbf{\ipa{pʰv̩˧˥}}}}\kern2pt]} \hypertarget{p\string_hv\string_=\string_M\string_T3}{}
\markboth{\textcolor{darkblue}{\textbf{\ipa{pʰv̩˧˥}}} \textsubscript{3}}{}
\textcolor{teal}{\mytextsc{verb}} \hspace{4pt} Tone: MH.
\textcolor{Sepia}{\selectlanguage{english}To turn over (when lying down).} \zh{翻身、翻来翻去。}  ¶ \textcolor{darkblue}{\textbf{\ipa{le˧-wo˧ tsɤ˥-pʰv̩˩ |}}} \textcolor{Sepia}{\selectlanguage{english}to turn over (when lying down)} \zh{翻身}  
 ¶ \textcolor{darkblue}{\textbf{\ipa{ɖɯ˧-tɕo˥ tsɤ˩-pʰv̩˩, | ʈʂʰɯ˧-tɕo˥ tsɤ˩-pʰv̩˩}}} \textcolor{Sepia}{\selectlanguage{english}to turn over this way and that (when lying down: turning over restlessly)} \zh{翻来翻去}  

\lhead{\firstmark}
\rhead{\botmark}

\subsection{\hspace{-0.5cm} {\Large \textcolor{darkblue}{\textbf{\ipa{pʰv̩˩\textsubscript{a}}}}}\hspace{0.5cm}[\kern2pt{\textcolor{darkblue}{\textbf{\ipa{pʰv̩˩˥}}}}\kern2pt]} \hypertarget{p\string_hv\string_=\string_Ba1}{}
\markboth{\textcolor{darkblue}{\textbf{\ipa{pʰv̩˩\textsubscript{a}}}}}{}
\textcolor{teal}{\mytextsc{adjective}} \hspace{4pt} Tone: L\textsubscript{a}.
\textcolor{Sepia}{\selectlanguage{english}White.} \zh{白色(脸、衣服)。}  ¶ \textcolor{darkblue}{\textbf{\ipa{pʰv̩˩-hĩ˩˥}}} \textcolor{Sepia}{\selectlanguage{english}\mytextsc{rel}} \zh{白的}  

\lhead{\firstmark}
\rhead{\botmark}

\subsection{\hspace{-0.5cm} {\Large \textcolor{darkblue}{\textbf{\ipa{pʰv̩˩\textsubscript{a}}}}}\hspace{0.5cm}[\kern2pt{\textcolor{darkblue}{\textbf{\ipa{pʰv̩˩˥}}}}\kern2pt]} \hypertarget{p\string_hv\string_=\string_Ba1}{}
\markboth{\textcolor{darkblue}{\textbf{\ipa{pʰv̩˩\textsubscript{a}}}}}{}
\textcolor{teal}{\mytextsc{classifier}} \hspace{4pt} Tone: L\textsubscript{b}.
\textcolor{Sepia}{\selectlanguage{english}Classifier for fields.} \zh{量词:田地(一块)。}  ¶ \textcolor{darkblue}{\textbf{\ipa{lv̩˧ | ɖɯ˧-pʰv̩˩}}} \textcolor{Sepia}{\selectlanguage{english}a field} \zh{一块田}  

\lhead{\firstmark}
\rhead{\botmark}

\subsection{\hspace{-0.5cm} {\Large \textcolor{darkblue}{\textbf{\ipa{pʰv̩˩\textsubscript{b}}}} \textsubscript{1}}\hspace{0.5cm}[\kern2pt{\textcolor{darkblue}{\textbf{\ipa{pʰv̩˩˥}}}}\kern2pt]} \hypertarget{p\string_hv\string_=\string_Bb1}{}
\markboth{\textcolor{darkblue}{\textbf{\ipa{pʰv̩˩\textsubscript{b}}}} \textsubscript{1}}{}
\textcolor{teal}{\mytextsc{verb}} \hspace{4pt} Tone: L\textsubscript{b}.
\textcolor{Sepia}{\selectlanguage{english}To move around.} \zh{摇动、翻滚。}  ¶ \textcolor{darkblue}{\textbf{\ipa{bo˩˥ | tʰi˧-pʰv̩˩-dʑo˩}}} \textcolor{Sepia}{\selectlanguage{english}The pig is moving around (restlessly).} \zh{猪在翻滚}  
 ¶ \textcolor{darkblue}{\textbf{\ipa{bo˩-ɳɯ˧ | pʰv̩˧\textasciitilde{}pʰv̩˩}}} \textcolor{Sepia}{\selectlanguage{english}same meaning as above} \zh{猪在翻滚}  

\lhead{\firstmark}
\rhead{\botmark}

\subsection{\hspace{-0.5cm} {\Large \textcolor{darkblue}{\textbf{\ipa{pʰv̩˩\textsubscript{b}}}} \textsubscript{2}}\hspace{0.5cm}[\kern2pt{\textcolor{darkblue}{\textbf{\ipa{pʰv̩˩˥}}}}\kern2pt]} \hypertarget{p\string_hv\string_=\string_Bb2}{}
\markboth{\textcolor{darkblue}{\textbf{\ipa{pʰv̩˩\textsubscript{b}}}} \textsubscript{2}}{}
\textcolor{teal}{\mytextsc{verb}} \hspace{4pt} Tone: L\textsubscript{b}.
\textcolor{Sepia}{\selectlanguage{english}To expand, to spread, to extend.} \zh{扩散、发展。}  ¶ \textcolor{darkblue}{\textbf{\ipa{zo˧mv̩˥ | tʰi˧-pʰv̩˩}}} \textcolor{Sepia}{\selectlanguage{english}the children spread into new territory; the family spreads, expands into new areas} \zh{孩子们扩散(到新的地方)}  

\lhead{\firstmark}
\rhead{\botmark}

\subsection{\hspace{-0.5cm} {\Large \textcolor{darkblue}{\textbf{\ipa{pʰv̩˧ɖɯ˧˥}}}}\hspace{0.5cm}[\kern2pt{\textcolor{darkblue}{\textbf{\ipa{pʰv̩˧ɖɯ˧˥}}}}\kern2pt]} \hypertarget{p\string_hv\string_=\string_Md`M\string_M\string_T1}{}
\markboth{\textcolor{darkblue}{\textbf{\ipa{pʰv̩˧ɖɯ˧˥}}}}{}
\textcolor{teal}{\mytextsc{adjective}} \hspace{4pt} Tone: MH\#.
\textit{From:} \textbf{pʰv̩˧ 2 and ɖɯ˩a} \textcolor{Sepia}{\selectlanguage{english}Expensive.} \zh{贵。}  ¶ \textcolor{darkblue}{\textbf{\ipa{pʰv̩˧ɖɯ˧ ʝi˥}}} \textcolor{Sepia}{\selectlanguage{english}to care for, to give great attention to} \zh{关心}  

\lhead{\firstmark}
\rhead{\botmark}

\subsection{\hspace{-0.5cm} {\Large \textcolor{darkblue}{\textbf{\ipa{pʰv̩˧dʑo˧-hĩ\#˥}}}}\hspace{0.5cm}[\kern2pt{\textcolor{darkblue}{\textbf{\ipa{xxxx non-correspondance entre le nombre de morphèmes et le nombre de tons de morphèmes}}}}\kern2pt]} \hypertarget{p\string_hv\string_=\string_Mdz£o\string_M-hi\string_~\#\string_T1}{}
\markboth{\textcolor{darkblue}{\textbf{\ipa{pʰv̩˧dʑo˧-hĩ\#˥}}}}{}
\textcolor{teal}{\mytextsc{noun}} \hspace{4pt} Tone: \#H.
\textcolor{Sepia}{\selectlanguage{english}Inhabitant of Labai, person from Labai.} \zh{拉伯的人。} 
\lhead{\firstmark}
\rhead{\botmark}

\subsection{\hspace{-0.5cm} {\Large \textcolor{darkblue}{\textbf{\ipa{pʰv̩˧dʑo\#˥}}}}\hspace{0.5cm}[\kern2pt{\textcolor{darkblue}{\textbf{\ipa{pʰv̩˧dʑo˧}}}}\kern2pt]} \hypertarget{p\string_hv\string_=\string_Mdz£o\#\string_T1}{}
\markboth{\textcolor{darkblue}{\textbf{\ipa{pʰv̩˧dʑo\#˥}}}}{}
\textcolor{teal}{\mytextsc{noun}} \hspace{4pt} Tone: \#H.
\textcolor{Sepia}{\selectlanguage{english}The village of Labai.} \zh{拉伯。}  ¶ \textcolor{darkblue}{\textbf{\ipa{pʰv̩˧dʑo˧ dzi˧˥}}} \textcolor{Sepia}{\selectlanguage{english}to live in Labai, to dwell in Labai} \zh{在拉柏住}  

\lhead{\firstmark}
\rhead{\botmark}

\subsection{\hspace{-0.5cm} {\Large \textcolor{darkblue}{\textbf{\ipa{pʰv̩˧kɤ˧}}}}\hspace{0.5cm}[\kern2pt{\textcolor{darkblue}{\textbf{\ipa{pʰv̩˧kɤ˧}}}}\kern2pt]} \hypertarget{p\string_hv\string_=\string_Mk7\string_M1}{}
\markboth{\textcolor{darkblue}{\textbf{\ipa{pʰv̩˧kɤ˧}}}}{}
\textcolor{teal}{\mytextsc{noun}} \hspace{4pt} Tone: M.
\textcolor{Sepia}{\selectlanguage{english}Blanket.} \zh{被子。}  \zh{量词}: \textcolor{darkblue}{\textbf{\ipa{ɭɯ˧}}}  \mytextsc{clf}: \textcolor{darkblue}{\textbf{\ipa{ɭɯ˧}}} 
\lhead{\firstmark}
\rhead{\botmark}

\subsection{\hspace{-0.5cm} {\Large \textcolor{darkblue}{\textbf{\ipa{pʰv̩˧ɭɯ˧-ʈʰæ˧qʰwɤ˥}}}}\hspace{0.5cm}[\kern2pt{\textcolor{darkblue}{\textbf{\ipa{xxxx non-correspondance entre le nombre de morphèmes et le nombre de tons de morphèmes}}}}\kern2pt]} \hypertarget{p\string_hv\string_=\string_Ml\string_RM\string_M-t`\string_h\{\string_Mq\string_hw7\string_T1}{}
\markboth{\textcolor{darkblue}{\textbf{\ipa{pʰv̩˧ɭɯ˧-ʈʰæ˧qʰwɤ˥}}}}{}
\textcolor{teal}{\mytextsc{noun}} \hspace{4pt} Tone: H\#.
\textcolor{Sepia}{\selectlanguage{english}Woolen skirt. (Not in common use in Yongning.).} \zh{羊毛裙子。}  \zh{量词}: \textcolor{darkblue}{\textbf{\ipa{ɭɯ˧}}}  \mytextsc{clf}: \textcolor{darkblue}{\textbf{\ipa{ɭɯ˧}}} 
\lhead{\firstmark}
\rhead{\botmark}

\subsection{\hspace{-0.5cm} {\Large \textcolor{darkblue}{\textbf{\ipa{pʰv̩˧ɭɯ\#˥}}}}\hspace{0.5cm}[\kern2pt{\textcolor{darkblue}{\textbf{\ipa{pʰv̩˧ɭɯ˧}}}}\kern2pt]} \hypertarget{p\string_hv\string_=\string_Ml\string_RM\#\string_T1}{}
\markboth{\textcolor{darkblue}{\textbf{\ipa{pʰv̩˧ɭɯ\#˥}}}}{}
\textcolor{teal}{\mytextsc{noun}} \hspace{4pt} Tone: \#H.
\textcolor{Sepia}{\selectlanguage{english}Tibetan wool fabric.} \zh{氆氇。}  \zh{量词}: \textcolor{darkblue}{\textbf{\ipa{ɭɯ˧}}}  \mytextsc{clf}: \textcolor{darkblue}{\textbf{\ipa{ɭɯ˧}}} 
\lhead{\firstmark}
\rhead{\botmark}

\subsection{\hspace{-0.5cm} {\Large \textcolor{darkblue}{\textbf{\ipa{pʰv̩˧ʂɯ˧}}}}\hspace{0.5cm}[\kern2pt{\textcolor{darkblue}{\textbf{\ipa{pʰv̩˧ʂɯ˧}}}}\kern2pt]} \hypertarget{p\string_hv\string_=\string_Ms`M\string_M1}{}
\markboth{\textcolor{darkblue}{\textbf{\ipa{pʰv̩˧ʂɯ˧}}}}{}
\textcolor{teal}{\mytextsc{noun}} \hspace{4pt} Tone: M.
\textcolor{Sepia}{\selectlanguage{english}Beauty cream; now used for sun cream.} \zh{美容膏。也来指防晒霜。}  ¶ \textcolor{darkblue}{\textbf{\ipa{pʰv˧ʂɯ˧ jɤ˧˥}}} \textcolor{Sepia}{\selectlanguage{english}to put on beauty cream; to apply sunscreen} \zh{抹美容膏,抹防晒霜}  
 ¶ \textcolor{darkblue}{\textbf{\ipa{pʰv˧ʂɯ˧ lɑ˧˥}}} \textcolor{Sepia}{\selectlanguage{english}to put on beauty cream; to apply sunscreen} \zh{抹美容膏,抹防晒霜}  

\lhead{\firstmark}
\rhead{\botmark}

\subsection{\hspace{-0.5cm} {\Large \textcolor{darkblue}{\textbf{\ipa{pʰv̩˩-tɕæ˩ɻæ˥}}}}\hspace{0.5cm}[\kern2pt{\textcolor{darkblue}{\textbf{\ipa{xxxx non-correspondance entre le nombre de morphèmes et le nombre de tons de morphèmes}}}}\kern2pt]} \hypertarget{p\string_hv\string_=\string_B-ts£\{\string_Br£`\{\string_T1}{}
\markboth{\textcolor{darkblue}{\textbf{\ipa{pʰv̩˩-tɕæ˩ɻæ˥}}}}{}
\textcolor{teal}{\mytextsc{adjective}} \hspace{4pt} Tone: L+H\#.
\textit{From:} \textbf{pʰv̩˩a} \textcolor{Sepia}{\selectlanguage{english}Very white.} \zh{很白(脸、衣服、头发)。}  ¶ \textcolor{darkblue}{\textbf{\ipa{pʰv̩˩tɕæ˩ɻæ˥-gv̩˩}}} \textcolor{Sepia}{\selectlanguage{english}very white} \zh{很白}  
 ¶ \textcolor{darkblue}{\textbf{\ipa{pʰv̩˩↑tɕæ˥ɻæ˩-gv̩˩}}} \textcolor{Sepia}{\selectlanguage{english}very white} \zh{很白}  
 ¶ \textcolor{darkblue}{\textbf{\ipa{pʰæ˧qʰwɤ˩ | pʰv̩˩tɕæ˩ɻæ˥-gv̩˩}}} \textcolor{Sepia}{\selectlanguage{english}the face is very white} \zh{脸很白}  

\lhead{\firstmark}
\rhead{\botmark}

\subsection{\hspace{-0.5cm} {\Large \textcolor{darkblue}{\textbf{\ipa{pʰv̩˥ʈʂʰe˩}}}}\hspace{0.5cm}[\kern2pt{\textcolor{darkblue}{\textbf{\ipa{xxxx ton non trouvé, à faire manuellement...}}}}\kern2pt]} \hypertarget{p\string_hv\string_=\string_Tt`s`\string_he\string_B1}{}
\markboth{\textcolor{darkblue}{\textbf{\ipa{pʰv̩˥ʈʂʰe˩}}}}{}
\textcolor{teal}{\mytextsc{verb}} \hspace{4pt} Tone: HL.
\textcolor{Sepia}{\selectlanguage{english}To distinguish.} \zh{分开、区分、区别开来。}  ¶ \textcolor{darkblue}{\textbf{\ipa{le˧-pʰv̩˥ʈʂʰe˩}}} \textcolor{Sepia}{\selectlanguage{english}to distinguish, to tell apart (e.g. species of mushrooms)} \zh{分开、区分、区别开来}  
 ¶ \textcolor{darkblue}{\textbf{\ipa{ɖɯ˧-pʰv̩˥ʈʂʰe˩=ɻ̍˩}}} \textcolor{Sepia}{\selectlanguage{english}\mytextsc{delimitative} \string_ \mytextsc{inceptive}} \zh{试着区分}  
 ¶ \textcolor{darkblue}{\textbf{\ipa{mɤ˧-pʰv̩˥ʈʂʰe˩}}} \textcolor{Sepia}{\selectlanguage{english}not to distinguish, not to make any difference (e.g. between different species of mushrooms)} \zh{不分开,不分,不区分}  

\lhead{\firstmark}
\rhead{\botmark}

\subsection{\hspace{-0.5cm} {\Large \textcolor{darkblue}{\textbf{\ipa{pʰv̩˧tv̩˥}}}}\hspace{0.5cm}[\kern2pt{\textcolor{darkblue}{\textbf{\ipa{pʰv̩˧tv̩˥}}}}\kern2pt]} \hypertarget{p\string_hv\string_=\string_Mtv\string_=\string_T1}{}
\markboth{\textcolor{darkblue}{\textbf{\ipa{pʰv̩˧tv̩˥}}}}{}
\textcolor{teal}{\mytextsc{noun}} \hspace{4pt} Tone: H\#.
\textcolor{Sepia}{\selectlanguage{english}Male water buffalo.} \zh{公水牛。}  ¶ \textcolor{darkblue}{\textbf{\ipa{dʑi˧mi˧-pʰv̩˩tv̩˩}}} \textcolor{Sepia}{\selectlanguage{english}same meaning: male water buffalo} \zh{同上:公水牛}  
 ¶ \textcolor{darkblue}{\textbf{\ipa{dʑi˧mi˧ ʈʂʰɯ˧-pʰo˩ dʑo˩, | pʰv̩˧tv̩˥ ɲi˩!}}} \textcolor{Sepia}{\selectlanguage{english}This buffalo is a male!} \zh{这头水牛是公的/是公水牛!}  
 \zh{量词}: \textcolor{darkblue}{\textbf{\ipa{pʰo˧˥}}}  \mytextsc{clf}: \textcolor{darkblue}{\textbf{\ipa{pʰo˧˥}}} 
\lhead{\firstmark}
\rhead{\botmark}

\subsection{\hspace{-0.5cm} {\Large \textcolor{darkblue}{\textbf{\ipa{pʰv̩˧ʐo˧˥}}}}\hspace{0.5cm}[\kern2pt{\textcolor{darkblue}{\textbf{\ipa{pʰv̩˧ʐo˧˥}}}}\kern2pt]} \hypertarget{p\string_hv\string_=\string_Mz`o\string_M\string_T1}{}
\markboth{\textcolor{darkblue}{\textbf{\ipa{pʰv̩˧ʐo˧˥}}}}{}
\textcolor{teal}{\mytextsc{adjective}} \hspace{4pt} Tone: MH\#.
\textit{From:} \textbf{pʰv̩˧ 2 and ʐo˩a} \textcolor{Sepia}{\selectlanguage{english}Cheap.} \zh{便宜。} \textit{See:} \textcolor{darkblue}{\textbf{\ipa{pʰv̩˧2; ʐo˩a2}}} 
\lhead{\firstmark}
\rhead{\botmark}

\newpage
\section*{\centering- \textcolor{darkblue}{\textbf{\ipa{q}}} -}
\subsection{\hspace{-0.5cm} {\Large \textcolor{darkblue}{\textbf{\ipa{qɑ˩\textsubscript{c}}}}}\hspace{0.5cm}[\kern2pt{\textcolor{darkblue}{\textbf{\ipa{qɑ˩˥}}}}\kern2pt]} \hypertarget{qA\string_Bc1}{}
\markboth{\textcolor{darkblue}{\textbf{\ipa{qɑ˩\textsubscript{c}}}}}{}
\textcolor{teal}{\mytextsc{classifier}} \hspace{4pt} Tone: L\textsubscript{c}.
\ding{202} \textcolor{Sepia}{\selectlanguage{english}Classifier for armfuls: of firewood, objects...} \zh{量词:抱。}  ¶ \textcolor{darkblue}{\textbf{\ipa{ʈʂʰɯ˧-qɑ˥}}} \textcolor{Sepia}{\selectlanguage{english}this armful} \zh{这一抱}  
\ding{203} \textcolor{Sepia}{\selectlanguage{english}A large bundle of cut cereals, made of about 10 sheaves. Each sheaf is tied using one stalk, then sheaves are tied together using string. A mule can carry 4 large bundles. Also for: an armful.} \zh{量词:粮食垛、干草垛。}  ¶ \textcolor{darkblue}{\textbf{\ipa{dze˧ɭɯ˧ ɖɯ˧-qɑ˩}}} \textcolor{Sepia}{\selectlanguage{english}a bundle of corn (cut cereals)} \zh{一垛小麦(收割时,将十束绑在一起成一垛)}  

\lhead{\firstmark}
\rhead{\botmark}

\subsection{\hspace{-0.5cm} {\Large \textcolor{darkblue}{\textbf{\ipa{qɑ˩\textsubscript{a}}}}}\hspace{0.5cm}[\kern2pt{\textcolor{darkblue}{\textbf{\ipa{qɑ˩˥}}}}\kern2pt]} \hypertarget{qA\string_Ba1}{}
\markboth{\textcolor{darkblue}{\textbf{\ipa{qɑ˩\textsubscript{a}}}}}{}
\textcolor{teal}{\mytextsc{verb}} \hspace{4pt} Tone: L\textsubscript{a}.
\ding{202} \textcolor{Sepia}{\selectlanguage{english}To cover (e.g. cover a pot with a lid).} \zh{盖、覆盖。}  ¶ \textcolor{darkblue}{\textbf{\ipa{le˧-qɑ˩-ze˩}}} \textcolor{Sepia}{\selectlanguage{english}\mytextsc{accomp} \string_ \mytextsc{pfv}} \zh{盖了}  
 ¶ \textcolor{darkblue}{\textbf{\ipa{tʰi˧-qɑ˩-ze˩}}} \textcolor{Sepia}{\selectlanguage{english}\mytextsc{dur} \string_ \mytextsc{pfv}} \zh{\mytextsc{dur} \string_ \mytextsc{pfv}}  
 ¶ \textcolor{darkblue}{\textbf{\ipa{ɖɯ˧-kʰwɤ˥ | tʰi˧-qɑ˥}}} \textcolor{Sepia}{\selectlanguage{english}to cover (a television set) with a piece of fabric (to protect it from dust)} \zh{用一块(布料)来盖(电视机,为了防灰)}  
 ¶ \textcolor{darkblue}{\textbf{\ipa{hæ̃˧qʰv̩˥ | tʰi˧-qɑ˩!}}} \textcolor{Sepia}{\selectlanguage{english}At night, we cover (the television set with a piece of fabric)!} \zh{晚上,(要)盖上! / 我们晚上盖电视机(为了防灰)!}  
 ¶ \textcolor{darkblue}{\textbf{\ipa{tso˧\textasciitilde{}tso˧ qɑ˥}}} \textcolor{Sepia}{\selectlanguage{english}to cover things} \zh{覆盖东西}  
\ding{203} \textcolor{Sepia}{\selectlanguage{english}To hide from view.} \zh{遮(云遮月)、遮挡。} 
\lhead{\firstmark}
\rhead{\botmark}

\subsection{\hspace{-0.5cm} {\Large \textcolor{darkblue}{\textbf{\ipa{‑qɑ˧˥}}}}\hspace{0.5cm}[\kern2pt{\textcolor{darkblue}{\textbf{\ipa{qɑ˧˥}}}}\kern2pt]} \hypertarget{‑qA\string_M\string_T1}{}
\markboth{\textcolor{darkblue}{\textbf{\ipa{‑qɑ˧˥}}}}{}
\textcolor{teal}{\mytextsc{postposition}} \hspace{4pt} Tone: MH.
\textcolor{Sepia}{\selectlanguage{english}Dative (to); comitative (with).} \zh{给、对。} 
\lhead{\firstmark}
\rhead{\botmark}

\subsection{\hspace{-0.5cm} {\Large \textcolor{darkblue}{\textbf{\ipa{qɑ˧˥}}}}\hspace{0.5cm}[\kern2pt{\textcolor{darkblue}{\textbf{\ipa{qɑ˧˥}}}}\kern2pt]} \hypertarget{qA\string_M\string_T1}{}
\markboth{\textcolor{darkblue}{\textbf{\ipa{qɑ˧˥}}}}{}
\textcolor{teal}{\mytextsc{verb}} \hspace{4pt} Tone: MH.
\textcolor{Sepia}{\selectlanguage{english}To help.} \zh{帮助。}  ¶ \textcolor{darkblue}{\textbf{\ipa{tʰi˧-qɑ˧˥}}} \textcolor{Sepia}{\selectlanguage{english}\mytextsc{dur}} \zh{\mytextsc{dur}}  
 ¶ \textcolor{darkblue}{\textbf{\ipa{qɑ˩\textasciitilde{}qɑ˧˥}}} \textcolor{Sepia}{\selectlanguage{english}\mytextsc{red}} \zh{\mytextsc{重叠:帮帮忙}}  
 ¶ \textcolor{darkblue}{\textbf{\ipa{hĩ˧ qɑ˩\textasciitilde{}qɑ˩}}} \textcolor{Sepia}{\selectlanguage{english}to help people; to go and work at someone else's place (e.g. during the harvest)} \zh{帮人,到别人家去工作(例如收庄稼的时候)}  
 ¶ \textcolor{darkblue}{\textbf{\ipa{njɤ˧ no˧ qɑ˧\textasciitilde{}qɑ˥}}} \textcolor{Sepia}{\selectlanguage{english}I help you} \zh{我帮你}  

\lhead{\firstmark}
\rhead{\botmark}

\subsection{\hspace{-0.5cm} {\Large \textcolor{darkblue}{\textbf{\ipa{qæ˥}}} \textsubscript{1}}\hspace{0.5cm}[\kern2pt{\textcolor{darkblue}{\textbf{\ipa{qæ˥}}}}\kern2pt]} \hypertarget{q\{\string_T1}{}
\markboth{\textcolor{darkblue}{\textbf{\ipa{qæ˥}}} \textsubscript{1}}{}
\textcolor{teal}{\mytextsc{verb}} \hspace{4pt} Tone: H.
\textcolor{Sepia}{\selectlanguage{english}To displace, to move (e.g. earth from one spot to another).} \zh{搬。} Local Chinese dialect:\zh{盘。} ¶ \textcolor{darkblue}{\textbf{\ipa{le˧-qæ˥}}} \textcolor{Sepia}{\selectlanguage{english}\mytextsc{accomp}} \zh{\mytextsc{accomp}}  

\lhead{\firstmark}
\rhead{\botmark}

\subsection{\hspace{-0.5cm} {\Large \textcolor{darkblue}{\textbf{\ipa{qæ˥}}} \textsubscript{2}}\hspace{0.5cm}[\kern2pt{\textcolor{darkblue}{\textbf{\ipa{qæ˥}}}}\kern2pt]} \hypertarget{q\{\string_T2}{}
\markboth{\textcolor{darkblue}{\textbf{\ipa{qæ˥}}} \textsubscript{2}}{}
\textcolor{teal}{\mytextsc{verb}} \hspace{4pt} Tone: H.
\textcolor{Sepia}{\selectlanguage{english}To change.} \zh{换。}  ¶ \textcolor{darkblue}{\textbf{\ipa{le˧-qæ˥-ze˩}}} \textcolor{Sepia}{\selectlanguage{english}\mytextsc{accomp} \string_ \mytextsc{pfv}} \zh{换了}  
 ¶ \textcolor{darkblue}{\textbf{\ipa{dʑi˧hṽ˧ qæ˧}}} \textcolor{Sepia}{\selectlanguage{english}to change clothes} \zh{换衣服}  
 ¶ \textcolor{darkblue}{\textbf{\ipa{bɑ˩lɑ˩˥ | tʰi˧-qæ˥}}} \textcolor{Sepia}{\selectlanguage{english}to change clothes} \zh{换衣服}  
 ¶ \textcolor{darkblue}{\textbf{\ipa{qæ˧\textasciitilde{}qæ˧}}} \textcolor{Sepia}{\selectlanguage{english}\mytextsc{red}: to exchange (an object for another)} \zh{\mytextsc{重叠:交换}}  
 ¶ \textcolor{darkblue}{\textbf{\ipa{qæ˧\textasciitilde{}qæ˧-ɻ̍˥}}} \textcolor{Sepia}{\selectlanguage{english}\mytextsc{red} \mytextsc{inceptive}} \zh{\mytextsc{red} \mytextsc{inceptive}}  
 ¶ \textcolor{darkblue}{\textbf{\ipa{qæ˧\textasciitilde{}qæ˧-ɻ̍˧-ze˥}}} \textcolor{Sepia}{\selectlanguage{english}\mytextsc{red} \mytextsc{inceptive} \mytextsc{pfv}} \zh{\mytextsc{red} \mytextsc{inceptive} \mytextsc{pfv}}  
 ¶ \textcolor{darkblue}{\textbf{\ipa{tso˧\textasciitilde{}tso˧ qæ˧\textasciitilde{}qæ˧}}} \textcolor{Sepia}{\selectlanguage{english}to exchange things} \zh{交换东西}  
 ¶ \textcolor{darkblue}{\textbf{\ipa{le˧-qæ˧\textasciitilde{}qæ˧(-ze˩)}}} \textcolor{Sepia}{\selectlanguage{english}\mytextsc{accomp} \mytextsc{red} (\mytextsc{pfv})} \zh{\mytextsc{accomp} \mytextsc{red} (\mytextsc{pfv})}  

\lhead{\firstmark}
\rhead{\botmark}

\subsection{\hspace{-0.5cm} {\Large \textcolor{darkblue}{\textbf{\ipa{qæ˥}}} \textsubscript{3}}\hspace{0.5cm}[\kern2pt{\textcolor{darkblue}{\textbf{\ipa{qæ˥}}}}\kern2pt]} \hypertarget{q\{\string_T3}{}
\markboth{\textcolor{darkblue}{\textbf{\ipa{qæ˥}}} \textsubscript{3}}{}
\textcolor{teal}{\mytextsc{verb}} \hspace{4pt} Tone: H.
\textcolor{Sepia}{\selectlanguage{english}To sculpt.} \zh{雕。}  ¶ \textcolor{darkblue}{\textbf{\ipa{le˧-qæ˥-ze˩}}} \textcolor{Sepia}{\selectlanguage{english}\mytextsc{accomp} \string_ \mytextsc{pfv}} \zh{雕了}  
 ¶ \textcolor{darkblue}{\textbf{\ipa{bæ˩bæ˩ qæ˥}}} \textcolor{Sepia}{\selectlanguage{english}to sculpt a flower} \zh{雕花}  

\lhead{\firstmark}
\rhead{\botmark}

\subsection{\hspace{-0.5cm} {\Large \textcolor{darkblue}{\textbf{\ipa{qæ˧do˧}}}}\hspace{0.5cm}[\kern2pt{\textcolor{darkblue}{\textbf{\ipa{qæ˩do˩˥}}}}\kern2pt]} \hypertarget{q\{\string_Mdo\string_M1}{}
\markboth{\textcolor{darkblue}{\textbf{\ipa{qæ˧do˧}}}}{}
\textcolor{teal}{\mytextsc{noun}} \hspace{4pt} Tone: M.
\textcolor{Sepia}{\selectlanguage{english}Timber, lumber.} \zh{木材、木料。}  ¶ \textcolor{darkblue}{\textbf{\ipa{ʑi˧mi˧-qæ˩do˩}}} \textcolor{Sepia}{\selectlanguage{english}lumber for the construction of the main building of a Na farm} \zh{建主房的木材}  
 ¶ \textcolor{darkblue}{\textbf{\ipa{ʑi˧qʰwɤ˧-qæ˧do\#˥}}} \textcolor{Sepia}{\selectlanguage{english}lumber for the construction of a building} \zh{建房子的木材}  
 \zh{量词}: \textcolor{darkblue}{\textbf{\ipa{kɤ˧˥}}}  \mytextsc{clf}: \textcolor{darkblue}{\textbf{\ipa{kɤ˧˥}}} \textit{Syn:} \hyperlink{}{\textcolor{darkblue}{\textbf{\ipa{qæ˧ɻ̍˧}}}}. 
\lhead{\firstmark}
\rhead{\botmark}

\subsection{\hspace{-0.5cm} {\Large \textcolor{darkblue}{\textbf{\ipa{qæ˧dzɯ˩}}}}\hspace{0.5cm}[\kern2pt{\textcolor{darkblue}{\textbf{\ipa{qæ˧dzɯ˩}}}}\kern2pt]} \hypertarget{q\{\string_MdzM\string_B1}{}
\markboth{\textcolor{darkblue}{\textbf{\ipa{qæ˧dzɯ˩}}}}{}
\textcolor{teal}{\mytextsc{noun}} \hspace{4pt} Tone: L\#.
\textcolor{Sepia}{\selectlanguage{english}A family name from Yongning. There are two families in Yongning that carry this name.} \zh{一个姓。这个姓,永宁有两家。}  ¶ \textcolor{darkblue}{\textbf{\ipa{qæ˧dzɯ˩-ɻ̍˩}}} \textcolor{Sepia}{\selectlanguage{english}the \textcolor{darkblue}{\textbf{\ipa{/qæ˧dzɯ˩/}}} clan, the \textcolor{darkblue}{\textbf{\ipa{/qæ˧dzɯ˩/}}} family} \zh{\textcolor{darkblue}{\textbf{\ipa{/qæ˧dzɯ˩/}}}家族}  
 ¶ \textcolor{darkblue}{\textbf{\ipa{qæ˧dzɯ˩ | -tsʰɯ˧ɻ̍˧}}} \textcolor{Sepia}{\selectlanguage{english}the name of a person, containing both a family name: \textcolor{darkblue}{\textbf{\ipa{/lqæ˧dzɯ˩/}}}, and a given name: \textcolor{darkblue}{\textbf{\ipa{/tsʰɯ˧ɻ\#˥/}}}} \zh{一个人的名字:姓为\textcolor{darkblue}{\textbf{\ipa{/qæ˧dzɯ˩/}}},名为\textcolor{darkblue}{\textbf{\ipa{/tsʰɯ˧ɻ\#˥/}}}}  

\lhead{\firstmark}
\rhead{\botmark}

\subsection{\hspace{-0.5cm} {\Large \textcolor{darkblue}{\textbf{\ipa{qæ˧ɻ̍˧}}}}\hspace{0.5cm}[\kern2pt{\textcolor{darkblue}{\textbf{\ipa{qæ˧ɻ̍˧}}}}\kern2pt]} \hypertarget{q\{\string_Mr£`̍\string_M1}{}
\markboth{\textcolor{darkblue}{\textbf{\ipa{qæ˧ɻ̍˧}}}}{}
\textcolor{teal}{\mytextsc{noun}} \hspace{4pt} Tone: M.
\textcolor{Sepia}{\selectlanguage{english}Timber, lumber.} \zh{木材、木料。}  \zh{量词}: \textcolor{darkblue}{\textbf{\ipa{kɤ˧˥}}}  \mytextsc{clf}: \textcolor{darkblue}{\textbf{\ipa{kɤ˧˥}}} \textit{Syn:} \hyperlink{}{\textcolor{darkblue}{\textbf{\ipa{qæ˧do˧}}}}. 
\lhead{\firstmark}
\rhead{\botmark}

\subsection{\hspace{-0.5cm} {\Large \textcolor{darkblue}{\textbf{\ipa{qæ˩\textsubscript{a}}}}}\hspace{0.5cm}[\kern2pt{\textcolor{darkblue}{\textbf{\ipa{qæ˩˥}}}}\kern2pt]} \hypertarget{q\{\string_Ba1}{}
\markboth{\textcolor{darkblue}{\textbf{\ipa{qæ˩\textsubscript{a}}}}}{}
\textcolor{teal}{\mytextsc{verb}} \hspace{4pt} Tone: L\textsubscript{a}.
\textcolor{Sepia}{\selectlanguage{english}To coax (a child).} \zh{哄(孩子)。}  ¶ \textcolor{darkblue}{\textbf{\ipa{zo˧ qæ˥}}} \textcolor{Sepia}{\selectlanguage{english}to coax a child} \zh{哄孩子}  
 ¶ \textcolor{darkblue}{\textbf{\ipa{le˧-qæ˧\textasciitilde{}qæ˥ | le˧-ʑi˧-kʰɯ˥}}} \textcolor{Sepia}{\selectlanguage{english}to put asleep by coaxing, to coax asleep} \zh{哄睡着}  

\lhead{\firstmark}
\rhead{\botmark}

\subsection{\hspace{-0.5cm} {\Large \textcolor{darkblue}{\textbf{\ipa{qæ˩\textsubscript{b}}}}}\hspace{0.5cm}[\kern2pt{\textcolor{darkblue}{\textbf{\ipa{qæ˩˥}}}}\kern2pt]} \hypertarget{q\{\string_Bb1}{}
\markboth{\textcolor{darkblue}{\textbf{\ipa{qæ˩\textsubscript{b}}}}}{}
\textcolor{teal}{\mytextsc{verb}} \hspace{4pt} Tone: L\textsubscript{b}.
\textcolor{Sepia}{\selectlanguage{english}To cheat, to deceive.} \zh{欺骗。}  ¶ \textcolor{darkblue}{\textbf{\ipa{hĩ˧ qæ˥-kv̩˩}}} \textcolor{Sepia}{\selectlanguage{english}sly, who is good at deceiving people} \zh{狡猾、很能骗人的}  
 ¶ \textcolor{darkblue}{\textbf{\ipa{hĩ˧ qæ˥ | ʐwæ˩˥}}} \textcolor{Sepia}{\selectlanguage{english}sly, who is good at deceiving people} \zh{狡猾、很能骗人的}  
 ¶ \textcolor{darkblue}{\textbf{\ipa{hĩ˧ qæ˥ mɤ˩-ɖo˩!}}} \textcolor{Sepia}{\selectlanguage{english}One must not cheat others! / One must not deceive people! (A precept taught by the main consultant's grandmother)} \zh{不要骗人!(这个信条,是发音合作人的祖母教的)}  
 ¶ \textcolor{darkblue}{\textbf{\ipa{qæ˩-mɤ˩-ɖo˩˥!}}} \textcolor{Sepia}{\selectlanguage{english}One must not cheat (others)! / One must not deceive people! (A precept taught by the main consultant's grandmother)} \zh{不要骗人!(这个信条,是发音合作人的祖母教的)}  
 ¶ \textcolor{darkblue}{\textbf{\ipa{mɤ˧-qæ˩}}} \textcolor{Sepia}{\selectlanguage{english}\mytextsc{neg}} \zh{不骗}  
 ¶ \textcolor{darkblue}{\textbf{\ipa{hĩ˧ qæ˥-tso˩\textasciitilde{}tso˩!}}} \textcolor{Sepia}{\selectlanguage{english}Shoddy stuff! (Literally: 'deceitful stuff!') (Context: a comment about thread of poor quality, bought at the market)} \zh{骗人的东西!(关于买来的一团线,质量不好)}  

\lhead{\firstmark}
\rhead{\botmark}

\subsection{\hspace{-0.5cm} {\Large \textcolor{darkblue}{\textbf{\ipa{qæ˩di˩}}}}\hspace{0.5cm}[\kern2pt{\textcolor{darkblue}{\textbf{\ipa{qæ˩di˩˥}}}}\kern2pt]} \hypertarget{q\{\string_Bdi\string_B1}{}
\markboth{\textcolor{darkblue}{\textbf{\ipa{qæ˩di˩}}}}{}
\textcolor{teal}{\mytextsc{verb}} \hspace{4pt} Tone: L.
\textcolor{Sepia}{\selectlanguage{english}To flick, to flip.} \zh{弹(弹脸)。} 
\lhead{\firstmark}
\rhead{\botmark}

\subsection{\hspace{-0.5cm} {\Large \textcolor{darkblue}{\textbf{\ipa{qæ˧˥}}} \textsubscript{1}}\hspace{0.5cm}[\kern2pt{\textcolor{darkblue}{\textbf{\ipa{qæ˧˥}}}}\kern2pt]} \hypertarget{q\{\string_M\string_T1}{}
\markboth{\textcolor{darkblue}{\textbf{\ipa{qæ˧˥}}} \textsubscript{1}}{}
\textcolor{teal}{\mytextsc{verb}} \hspace{4pt} Tone: MH.
\textcolor{Sepia}{\selectlanguage{english}To burn something, e.g. to cremate a corpse.} \zh{燃烧,如:烧尸体(进行火葬时)。}  ¶ \textcolor{darkblue}{\textbf{\ipa{mv̩˧ qæ˩-ze˩}}} \textcolor{Sepia}{\selectlanguage{english}the fire has started, the fire is blazing; a fire has caught} \zh{火烧着了 / 着火了}  
 ¶ \textcolor{darkblue}{\textbf{\ipa{mv̩˧ le˧-qæ˧˥ / mv̩˧ le˧-qæ˧-ze˥}}} \textcolor{Sepia}{\selectlanguage{english}the fire is burning; a fire has caught} \zh{火在烧 / 着火了}  
 ¶ \textcolor{darkblue}{\textbf{\ipa{mv̩˧ qæ˥-ɻ̍˩}}} \textcolor{Sepia}{\selectlanguage{english}the fire is burning} \zh{火在烧 / 火烧着了}  
 ¶ \textcolor{darkblue}{\textbf{\ipa{mv̩˧ qæ˥-ɻ̍˩ kʰɯ˩}}} \textcolor{Sepia}{\selectlanguage{english}to start a fire (as an act of destruction/war), to commit arson} \zh{(有人)放火}  
 ¶ \textcolor{darkblue}{\textbf{\ipa{mv̩˧qæ˥-ɻ̍˩-hɯ˩}}} \textcolor{Sepia}{\selectlanguage{english}a fire has started} \zh{(有人)放火了!}  

\lhead{\firstmark}
\rhead{\botmark}

\subsection{\hspace{-0.5cm} {\Large \textcolor{darkblue}{\textbf{\ipa{qæ˧˥}}} \textsubscript{2}}\hspace{0.5cm}[\kern2pt{\textcolor{darkblue}{\textbf{\ipa{qæ˧˥}}}}\kern2pt]} \hypertarget{q\{\string_M\string_T2}{}
\markboth{\textcolor{darkblue}{\textbf{\ipa{qæ˧˥}}} \textsubscript{2}}{}
\textcolor{teal}{\mytextsc{verb}} \hspace{4pt} Tone: MH.
\textcolor{Sepia}{\selectlanguage{english}To suffer, to have pain.} \zh{疼。}  ¶ \textcolor{darkblue}{\textbf{\ipa{bi˧mi˧ qæ˧˥}}} \textcolor{Sepia}{\selectlanguage{english}to have a stomach-ache} \zh{肚子疼}  
 ¶ \textcolor{darkblue}{\textbf{\ipa{ɬo˧kʰv̩˧ qæ˧˥}}} \textcolor{Sepia}{\selectlanguage{english}the waist hurts, the lower back hurts} \zh{腰疼}  
 ¶ \textcolor{darkblue}{\textbf{\ipa{ʁo˧qʰwɤ˩ qæ˩}}} \textcolor{Sepia}{\selectlanguage{english}to have a headache} \zh{头疼}  

\lhead{\firstmark}
\rhead{\botmark}

\subsection{\hspace{-0.5cm} {\Large \textcolor{darkblue}{\textbf{\ipa{qæ˩˥}}} \textsubscript{1}}\hspace{0.5cm}[\kern2pt{\textcolor{darkblue}{\textbf{\ipa{qæ˩˥}}}}\kern2pt]} \hypertarget{q\{\string_B\string_T1}{}
\markboth{\textcolor{darkblue}{\textbf{\ipa{qæ˩˥}}} \textsubscript{1}}{}
\textcolor{teal}{\mytextsc{noun}} \hspace{4pt} Tone: LH.
\textcolor{Sepia}{\selectlanguage{english}Oil; cooking oil.} \zh{油,食用油。} 
\lhead{\firstmark}
\rhead{\botmark}

\subsection{\hspace{-0.5cm} {\Large \textcolor{darkblue}{\textbf{\ipa{qæ˩˥}}} \textsubscript{2}}\hspace{0.5cm}[\kern2pt{\textcolor{darkblue}{\textbf{\ipa{qæ˩˥}}}}\kern2pt]} \hypertarget{q\{\string_B\string_T2}{}
\markboth{\textcolor{darkblue}{\textbf{\ipa{qæ˩˥}}} \textsubscript{2}}{}
\textcolor{teal}{\mytextsc{noun}} \hspace{4pt} Tone: LH.
\textcolor{Sepia}{\selectlanguage{english}Glue.} \zh{胶。}  \zh{量词}: \textcolor{darkblue}{\textbf{\ipa{kʰwɤ˥}}}  \mytextsc{clf}: \textcolor{darkblue}{\textbf{\ipa{kʰwɤ˥}}} 
\lhead{\firstmark}
\rhead{\botmark}

\subsection{\hspace{-0.5cm} {\Large \textcolor{darkblue}{\textbf{\ipa{qi˧qi˧}}}}\hspace{0.5cm}[\kern2pt{\textcolor{darkblue}{\textbf{\ipa{qi˧qi˧}}}}\kern2pt]} \hypertarget{qi\string_Mqi\string_M1}{}
\markboth{\textcolor{darkblue}{\textbf{\ipa{qi˧qi˧}}}}{}
\textcolor{teal}{\mytextsc{adverb(ial)}} \hspace{4pt} Tone: M.
\textcolor{Sepia}{\selectlanguage{english}Originally, to begin with.} \zh{原来、一开始。} 
\lhead{\firstmark}
\rhead{\botmark}

\subsection{\hspace{-0.5cm} {\Large \textcolor{darkblue}{\textbf{\ipa{qo˥}}} \textsubscript{1}}\hspace{0.5cm}[\kern2pt{\textcolor{darkblue}{\textbf{\ipa{qo˥}}}}\kern2pt]} \hypertarget{qo\string_T1}{}
\markboth{\textcolor{darkblue}{\textbf{\ipa{qo˥}}} \textsubscript{1}}{}
\textcolor{teal}{\mytextsc{verb}} \hspace{4pt} Tone: H.
\textcolor{Sepia}{\selectlanguage{english}To kneel down.} \zh{跪下。} 
\lhead{\firstmark}
\rhead{\botmark}

\subsection{\hspace{-0.5cm} {\Large \textcolor{darkblue}{\textbf{\ipa{qo˥}}} \textsubscript{2}}\hspace{0.5cm}[\kern2pt{\textcolor{darkblue}{\textbf{\ipa{qo˥}}}}\kern2pt]} \hypertarget{qo\string_T2}{}
\markboth{\textcolor{darkblue}{\textbf{\ipa{qo˥}}} \textsubscript{2}}{}
\textcolor{teal}{\mytextsc{verb}} \hspace{4pt} Tone: H.
\textcolor{Sepia}{\selectlanguage{english}To love.} \zh{爱,关心。}  ¶ \textcolor{darkblue}{\textbf{\ipa{mɤ˧-qo˧}}} \textcolor{Sepia}{\selectlanguage{english}\mytextsc{neg}} \zh{不爱}  
 ¶ \textcolor{darkblue}{\textbf{\ipa{zo˧mv̩˥zo˩ qo˩}}} \textcolor{Sepia}{\selectlanguage{english}to love (one's) children} \zh{爱孩子}  
 ¶ \textcolor{darkblue}{\textbf{\ipa{õ˧-hĩ˥ qo˩}}} \textcolor{Sepia}{\selectlanguage{english}to love one's family} \zh{爱自己家人}  

\lhead{\firstmark}
\rhead{\botmark}

\subsection{\hspace{-0.5cm} {\Large \textcolor{darkblue}{\textbf{\ipa{-qo˧}}}}\hspace{0.5cm}[\kern2pt{\textcolor{darkblue}{\textbf{\ipa{qo˥}}}}\kern2pt]} \hypertarget{-qo\string_M1}{}
\markboth{\textcolor{darkblue}{\textbf{\ipa{-qo˧}}}}{}
\textcolor{teal}{\mytextsc{postposition}} \hspace{4pt} Tone: M.
\textcolor{Sepia}{\selectlanguage{english}In, inside.} \zh{里。} \textit{See:} \hyperlink{}{\textcolor{darkblue}{\textbf{\ipa{-qo˧lo˩}}}} 
\lhead{\firstmark}
\rhead{\botmark}

\subsection{\hspace{-0.5cm} {\Large \textcolor{darkblue}{\textbf{\ipa{-qo˧lo˩}}}}\hspace{0.5cm}[\kern2pt{\textcolor{darkblue}{\textbf{\ipa{qo˧lo˩}}}}\kern2pt]} \hypertarget{-qo\string_Mlo\string_B1}{}
\markboth{\textcolor{darkblue}{\textbf{\ipa{-qo˧lo˩}}}}{}
\textcolor{teal}{\mytextsc{postposition}} \hspace{4pt} Tone: L\#.
\textcolor{Sepia}{\selectlanguage{english}In.} \zh{里面。}  ¶ \textcolor{darkblue}{\textbf{\ipa{ʈʂʰɯ˧ | ɑ˩ʁo˧-qo˧lo˩ dʑo˩}}} \textcolor{Sepia}{\selectlanguage{english}(S)he is in the house. / (S)he is inside.} \zh{他在家里。}  
\textit{See:} \hyperlink{}{\textcolor{darkblue}{\textbf{\ipa{qo˧lo˩}}}} 
\lhead{\firstmark}
\rhead{\botmark}

\subsection{\hspace{-0.5cm} {\Large \textcolor{darkblue}{\textbf{\ipa{qo˧lo˩}}}}\hspace{0.5cm}[\kern2pt{\textcolor{darkblue}{\textbf{\ipa{qo˧lo˩}}}}\kern2pt]} \hypertarget{qo\string_Mlo\string_B1}{}
\markboth{\textcolor{darkblue}{\textbf{\ipa{qo˧lo˩}}}}{}
\textcolor{teal}{\mytextsc{adverb(ial)}} \hspace{4pt} Tone: L\#.
\textcolor{Sepia}{\selectlanguage{english}Inside, within.} \zh{里面。} \textit{See:} \hyperlink{}{\textcolor{darkblue}{\textbf{\ipa{-qo˧lo˩}}}} 
\lhead{\firstmark}
\rhead{\botmark}

\subsection{\hspace{-0.5cm} {\Large \textcolor{darkblue}{\textbf{\ipa{qo˧pv̩˩}}}}\hspace{0.5cm}[\kern2pt{\textcolor{darkblue}{\textbf{\ipa{qo˧pv̩˩}}}}\kern2pt]} \hypertarget{qo\string_Mpv\string_=\string_B1}{}
\markboth{\textcolor{darkblue}{\textbf{\ipa{qo˧pv̩˩}}}}{}
\textcolor{teal}{\mytextsc{noun}} \hspace{4pt} Tone: L\#.
\textcolor{Sepia}{\selectlanguage{english}Cuckoo.} \zh{布谷鸟。}  ¶ \textcolor{darkblue}{\textbf{\ipa{qo˧pv̩˩-ɻwæ˩ | ɖɯ˧-ɲi˥}}} \textcolor{Sepia}{\selectlanguage{english}Ancestors' Day, Tomb-Sweeping Day, on the first day of the fifth month; literally: 'the day when the cuckoo sings'} \zh{清明节。直译:“布谷鸟叫的那天”}  
 \zh{量词}: \textcolor{darkblue}{\textbf{\ipa{mi˩}}}  \mytextsc{clf}: \textcolor{darkblue}{\textbf{\ipa{mi˩}}} 
\lhead{\firstmark}
\rhead{\botmark}

\subsection{\hspace{-0.5cm} {\Large \textcolor{darkblue}{\textbf{\ipa{qo˧pv̩˩-ʐwæ˩ɖʐæ˩}}}}\hspace{0.5cm}[\kern2pt{\textcolor{darkblue}{\textbf{\ipa{qo˩pv̩˧ʐwæ˧ɖʐæ˧}}}}\kern2pt]} \hypertarget{qo\string_Mpv\string_=\string_B-z`w\{\string_Bd`z`\{\string_B1}{}
\markboth{\textcolor{darkblue}{\textbf{\ipa{qo˧pv̩˩-ʐwæ˩ɖʐæ˩}}}}{}
\textcolor{teal}{\mytextsc{noun}} \hspace{4pt} Tone: LM-.
\textcolor{Sepia}{\selectlanguage{english}Jay, \textit{Garrulus glandarius sinensis}.} \zh{松鸦。} 
\lhead{\firstmark}
\rhead{\botmark}

\subsection{\hspace{-0.5cm} {\Large \textcolor{darkblue}{\textbf{\ipa{qo˧tv̩˩}}}}\hspace{0.5cm}[\kern2pt{\textcolor{darkblue}{\textbf{\ipa{qo˧tv̩˩}}}}\kern2pt]} \hypertarget{qo\string_Mtv\string_=\string_B1}{}
\markboth{\textcolor{darkblue}{\textbf{\ipa{qo˧tv̩˩}}}}{}
\textcolor{teal}{\mytextsc{noun}} \hspace{4pt} Tone: L\#/LM.
\textcolor{Sepia}{\selectlanguage{english}Kernel, fruit stone, pit.} \zh{果核。}  ¶ \textcolor{darkblue}{\textbf{\ipa{dʑi˧ʁo˩-qo˩tv̩˩}}} \textcolor{Sepia}{\selectlanguage{english}peach kernel} \zh{桃子果核}  
 \zh{量词}: \textcolor{darkblue}{\textbf{\ipa{ɭɯ˧}}}  \mytextsc{clf}: \textcolor{darkblue}{\textbf{\ipa{ɭɯ˧}}} 
\lhead{\firstmark}
\rhead{\botmark}

\subsection{\hspace{-0.5cm} {\Large \textcolor{darkblue}{\textbf{\ipa{qo˩\textsubscript{a}}}}}\hspace{0.5cm}[\kern2pt{\textcolor{darkblue}{\textbf{\ipa{qo˩˥}}}}\kern2pt]} \hypertarget{qo\string_Ba1}{}
\markboth{\textcolor{darkblue}{\textbf{\ipa{qo˩\textsubscript{a}}}}}{}
\textcolor{teal}{\mytextsc{verb}} \hspace{4pt} Tone: L\textsubscript{a}.
\textcolor{Sepia}{\selectlanguage{english}To put away, to preserve (e.g. to put leftovers in a box so flies won't land on it).} \zh{放、储存。} 
\lhead{\firstmark}
\rhead{\botmark}

\subsection{\hspace{-0.5cm} {\Large \textcolor{darkblue}{\textbf{\ipa{qo˩ho˧˥}}}}\hspace{0.5cm}[\kern2pt{\textcolor{darkblue}{\textbf{\ipa{qo˩ho˧˥}}}}\kern2pt]} \hypertarget{qo\string_Bho\string_M\string_T1}{}
\markboth{\textcolor{darkblue}{\textbf{\ipa{qo˩ho˧˥}}}}{}
\textcolor{teal}{\mytextsc{noun}} \hspace{4pt} Tone: LM+MH\#.
\textcolor{Sepia}{\selectlanguage{english}Round wicker/bamboo box used to carry gifts.} \zh{礼盒。}  \zh{量词}: \textcolor{darkblue}{\textbf{\ipa{ɭɯ˧}}}  \mytextsc{clf}: \textcolor{darkblue}{\textbf{\ipa{ɭɯ˧}}} 
\lhead{\firstmark}
\rhead{\botmark}

\subsection{\hspace{-0.5cm} {\Large \textcolor{darkblue}{\textbf{\ipa{qo˩qɑ˩}}}}\hspace{0.5cm}[\kern2pt{\textcolor{darkblue}{\textbf{\ipa{qo˩qɑ˩˥}}}}\kern2pt]} \hypertarget{qo\string_BqA\string_B1}{}
\markboth{\textcolor{darkblue}{\textbf{\ipa{qo˩qɑ˩}}}}{}
\textcolor{teal}{\mytextsc{noun}} \hspace{4pt} Tone: L.
\textcolor{Sepia}{\selectlanguage{english}Mountain pass.} \zh{垭口。}  \zh{量词}: \textcolor{darkblue}{\textbf{\ipa{ɭɯ˧}}}  \mytextsc{clf}: \textcolor{darkblue}{\textbf{\ipa{ɭɯ˧}}} 
\lhead{\firstmark}
\rhead{\botmark}

\subsection{\hspace{-0.5cm} {\Large \textcolor{darkblue}{\textbf{\ipa{qo˩tv̩˩-lv̩˥}}}}\hspace{0.5cm}[\kern2pt{\textcolor{darkblue}{\textbf{\ipa{xxxx non-correspondance entre le nombre de morphèmes et le nombre de tons de morphèmes}}}}\kern2pt]} \hypertarget{qo\string_Btv\string_=\string_B-lv\string_=\string_T1}{}
\markboth{\textcolor{darkblue}{\textbf{\ipa{qo˩tv̩˩-lv̩˥}}}}{}
\textcolor{teal}{\mytextsc{noun}} \hspace{4pt} Tone: L+H\#.
\textcolor{Sepia}{\selectlanguage{english}Ball, lump.} \zh{团。}  ¶ \textcolor{darkblue}{\textbf{\ipa{li˩-qo˩tv̩˥-lv̩˩}}} \textcolor{Sepia}{\selectlanguage{english}tea leaves compressed in bowl shape} \zh{沱茶}  
 ¶ \textcolor{darkblue}{\textbf{\ipa{li˩-qo˩tv̩˥-lv̩˩ | ɖɯ˧-qʰwɤ˧ tɕɤ˥}}} \textcolor{Sepia}{\selectlanguage{english}to make a bowl of tea, using tea leaves compressed in bowl shape} \zh{煮一碗沱茶}  
 \zh{量词}: \textcolor{darkblue}{\textbf{\ipa{ɭɯ˧}}}  \mytextsc{clf}: \textcolor{darkblue}{\textbf{\ipa{ɭɯ˧}}} 
\lhead{\firstmark}
\rhead{\botmark}

\subsection{\hspace{-0.5cm} {\Large \textcolor{darkblue}{\textbf{\ipa{qv̩˩˥}}}}\hspace{0.5cm}[\kern2pt{\textcolor{darkblue}{\textbf{\ipa{qv̩˩˥}}}}\kern2pt]} \hypertarget{qv\string_=\string_B\string_T1}{}
\markboth{\textcolor{darkblue}{\textbf{\ipa{qv̩˩˥}}}}{}
\textcolor{teal}{\mytextsc{noun}} \hspace{4pt} Tone: LH.
\textcolor{Sepia}{\selectlanguage{english}Handle.} \zh{把手。}  \zh{量词}: \textcolor{darkblue}{\textbf{\ipa{kʰwɤ˥}}}  \mytextsc{clf}: \textcolor{darkblue}{\textbf{\ipa{kʰwɤ˥}}} 
\lhead{\firstmark}
\rhead{\botmark}

\subsection{\hspace{-0.5cm} {\Large \textcolor{darkblue}{\textbf{\ipa{qv̩˧˥}}}}\hspace{0.5cm}[\kern2pt{\textcolor{darkblue}{\textbf{\ipa{qv̩˧˥}}}}\kern2pt]} \hypertarget{qv\string_=\string_M\string_T1}{}
\markboth{\textcolor{darkblue}{\textbf{\ipa{qv̩˧˥}}}}{}
\textcolor{teal}{\mytextsc{verb}} \hspace{4pt} Tone: MH.
\textcolor{Sepia}{\selectlanguage{english}To frighten.} \zh{吓(吓唬)。}  ¶ \textcolor{darkblue}{\textbf{\ipa{hĩ˧ qv̩˩}}} \textcolor{Sepia}{\selectlanguage{english}to frighten people} \zh{吓人}  
 ¶ \textcolor{darkblue}{\textbf{\ipa{no˧ | hĩ˧ qv̩˩-zo˩! / ʈʂʰɯ˧-ɳɯ˧ | hĩ˧ qv̩˩-zo˩!}}} \textcolor{Sepia}{\selectlanguage{english}You frighten people! / He frightens people!} \zh{你吓人! / 他吓人!}  
 ¶ \textcolor{darkblue}{\textbf{\ipa{ʈʂʰɯ˧ | njæ˩ qv̩˩-tsʰɯ˩˥!}}} \textcolor{Sepia}{\selectlanguage{english}He frightens people!} \zh{他吓人!}  
 ¶ \textcolor{darkblue}{\textbf{\ipa{njɤ˧ɳɯ˧ | ʈʂʰɯ˧ qv̩˩-bi˩!}}} \textcolor{Sepia}{\selectlanguage{english}I am going to frighten her/him!} \zh{我要吓唬他一下!}  
 ¶ \textcolor{darkblue}{\textbf{\ipa{tʰɑ˧-qv̩˧˥!}}} \textcolor{Sepia}{\selectlanguage{english}\mytextsc{prohib}} \zh{别吓唬(人家)!}  

\lhead{\firstmark}
\rhead{\botmark}

\subsection{\hspace{-0.5cm} {\Large \textcolor{darkblue}{\textbf{\ipa{qv̩˩\textsubscript{a}}}}}\hspace{0.5cm}[\kern2pt{\textcolor{darkblue}{\textbf{\ipa{qv̩˩˥}}}}\kern2pt]} \hypertarget{qv\string_=\string_Ba1}{}
\markboth{\textcolor{darkblue}{\textbf{\ipa{qv̩˩\textsubscript{a}}}}}{}
\textcolor{teal}{\mytextsc{verb}} \hspace{4pt} Tone: L\textsubscript{a}.
\textcolor{Sepia}{\selectlanguage{english}To wash (something) along (of water); to be carried (by water) (heavy objects, e.g. rocks are carried by a stream; the verb cannot be used for light objects, such as leaves).} \zh{冲走。}  ¶ \textcolor{darkblue}{\textbf{\ipa{le˧-qv̩˩ | le˧-po˧-tsʰɯ˧˥}}} \textcolor{Sepia}{\selectlanguage{english}to carry to a certain place, to wash along all the way to a certain place} \zh{冲到某个地方}  
 ¶ \textcolor{darkblue}{\textbf{\ipa{lv̩˧mi˧ | ɬi˧dʑɯ˩-ɳɯ˩ | qv̩˩˥.}}} \textcolor{Sepia}{\selectlanguage{english}The stones are carried (down into the plain) by (the strong current of) the river of Yongning.} \zh{石头被永宁河水冲(到坝子)}  
 ¶ \textcolor{darkblue}{\textbf{\ipa{dʑɯ˧-ɳɯ˧ | le˧-qv̩˩ | le˧-po˧-tsʰɯ˧-hĩ˥ | lv̩˧mi˧}}} \textcolor{Sepia}{\selectlanguage{english}stones carried over (to this place) by the stream} \zh{水流冲下来的石头}  

\lhead{\firstmark}
\rhead{\botmark}

\subsection{\hspace{-0.5cm} {\Large \textcolor{darkblue}{\textbf{\ipa{qv̩˧dzi˩}}}}\hspace{0.5cm}[\kern2pt{\textcolor{darkblue}{\textbf{\ipa{qv̩˧dzi˩}}}}\kern2pt]} \hypertarget{qv\string_=\string_Mdzi\string_B1}{}
\markboth{\textcolor{darkblue}{\textbf{\ipa{qv̩˧dzi˩}}}}{}
\textcolor{teal}{\mytextsc{noun}} \hspace{4pt} Tone: L\#.
\textcolor{Sepia}{\selectlanguage{english}\textit{Pinus massoniana D.Don in Lamb.}, Masson's pine, Chinese red pine, horsetail pine. Its seeds are not edible (the fish eat them, but they are poisonous for humans).} \zh{马尾松。} Local Chinese dialect:\zh{马松树。} ¶ \textcolor{darkblue}{\textbf{\ipa{qv̩˧dzi˩-lv̩˩\textasciitilde{}lv̩˩, | dzɯ˧ mɤ˧-ɖo˧!}}} \textcolor{Sepia}{\selectlanguage{english}One must not eat the seeds of Masson's pine! (It is poisonous)} \zh{马松树的果子,不要吃!(有毒)}  
 \zh{量词}: \textcolor{darkblue}{\textbf{\ipa{dzi˩, ʝi˧}}}  \mytextsc{clf}: \textcolor{darkblue}{\textbf{\ipa{dzi˩, ʝi˧}}} 
\lhead{\firstmark}
\rhead{\botmark}

\subsection{\hspace{-0.5cm} {\Large \textcolor{darkblue}{\textbf{\ipa{qv̩˧ɻ\#˥}}}}\hspace{0.5cm}[\kern2pt{\textcolor{darkblue}{\textbf{\ipa{qv̩˧ɻ˧}}}}\kern2pt]} \hypertarget{qv\string_=\string_Mr£`\#\string_T1}{}
\markboth{\textcolor{darkblue}{\textbf{\ipa{qv̩˧ɻ\#˥}}}}{}
\textcolor{teal}{\mytextsc{noun}} \hspace{4pt} Tone: \#H.
\textcolor{Sepia}{\selectlanguage{english}Name of a mountain in Yongning.} \zh{永宁的一座山。}  ¶ \textcolor{darkblue}{\textbf{\ipa{qv̩˧ɻ̍˧-ʁo˧-qʰwɤ˥}}} \textcolor{Sepia}{\selectlanguage{english}the top of the \textcolor{darkblue}{\textbf{\ipa{/qv̩˧ɻ̍˧/}}} mountain} \zh{\textcolor{darkblue}{\textbf{\ipa{/qv̩˧ɻ̍˧/}}}山的山顶}  

\lhead{\firstmark}
\rhead{\botmark}

\subsection{\hspace{-0.5cm} {\Large \textcolor{darkblue}{\textbf{\ipa{qv̩˧tɕi˥}}}}\hspace{0.5cm}[\kern2pt{\textcolor{darkblue}{\textbf{\ipa{qv̩˧tɕi˥}}}}\kern2pt]} \hypertarget{qv\string_=\string_Mts£i\string_T1}{}
\markboth{\textcolor{darkblue}{\textbf{\ipa{qv̩˧tɕi˥}}}}{}
\textcolor{teal}{\mytextsc{noun}} \hspace{4pt} Tone: H\#.
\textcolor{Sepia}{\selectlanguage{english}Spittle, phlegm, sputum.} \zh{痰。} 
\lhead{\firstmark}
\rhead{\botmark}

\subsection{\hspace{-0.5cm} {\Large \textcolor{darkblue}{\textbf{\ipa{qv̩˧ʈʂæ˧˥}}}}\hspace{0.5cm}[\kern2pt{\textcolor{darkblue}{\textbf{\ipa{qv̩˧ʈʂæ˧˥}}}}\kern2pt]} \hypertarget{qv\string_=\string_Mt`s`\{\string_M\string_T1}{}
\markboth{\textcolor{darkblue}{\textbf{\ipa{qv̩˧ʈʂæ˧˥}}}}{}
\textcolor{teal}{\mytextsc{noun}} \hspace{4pt} Tone: MH\#.
\ding{202} \textcolor{Sepia}{\selectlanguage{english}Throat.} \zh{喉咙。}  \zh{量词}: \textcolor{darkblue}{\textbf{\ipa{ɭɯ˧}}} \ding{203} \textcolor{Sepia}{\selectlanguage{english}Voice.} \zh{声音。}  ¶ \textcolor{darkblue}{\textbf{\ipa{ʈʂʰɯ˧ | qv̩˧ʈʂæ˧ dʑɤ˥!}}} \textcolor{Sepia}{\selectlanguage{english}(S)he has a beautiful voice.} \zh{他嗓子好。}  
 ¶ \textcolor{darkblue}{\textbf{\ipa{ʈʂʰɯ˧ | qv̩˧ʈʂæ˧˥ | ɖwæ˧˥ | dʑɤ˩˥!}}} \textcolor{Sepia}{\selectlanguage{english}(S)he has a really beautiful voice.} \zh{他嗓子很好。}  
 \mytextsc{clf}: \textcolor{darkblue}{\textbf{\ipa{ɭɯ˧}}} 
\lhead{\firstmark}
\rhead{\botmark}

\subsection{\hspace{-0.5cm} {\Large \textcolor{darkblue}{\textbf{\ipa{qwɑ˧mæ\#˥}}}}\hspace{0.5cm}[\kern2pt{\textcolor{darkblue}{\textbf{\ipa{qwɑ˧mæ˧}}}}\kern2pt]} \hypertarget{qwA\string_Mm\{\#\string_T1}{}
\markboth{\textcolor{darkblue}{\textbf{\ipa{qwɑ˧mæ\#˥}}}}{}
\textcolor{teal}{\mytextsc{noun}} \hspace{4pt} Tone: \#H.
\textcolor{Sepia}{\selectlanguage{english}Middle part of the main room.} \zh{主屋的中庭:在主屋上半部分与门之间的空间。}  \zh{量词}: \textcolor{darkblue}{\textbf{\ipa{kʰwɤ˥}}}  \mytextsc{clf}: \textcolor{darkblue}{\textbf{\ipa{kʰwɤ˥}}} 
\lhead{\firstmark}
\rhead{\botmark}

\subsection{\hspace{-0.5cm} {\Large \textcolor{darkblue}{\textbf{\ipa{qwæ˧}}}}\hspace{0.5cm}[\kern2pt{\textcolor{darkblue}{\textbf{\ipa{qwæ˥}}}}\kern2pt]} \hypertarget{qw\{\string_M1}{}
\markboth{\textcolor{darkblue}{\textbf{\ipa{qwæ˧}}}}{}
\textcolor{teal}{\mytextsc{noun}} \hspace{4pt} Tone: M.
\textcolor{Sepia}{\selectlanguage{english}Mat, bed mat.} \zh{床垫子。}  ¶ \textcolor{darkblue}{\textbf{\ipa{qwæ˧mi\#˥}}} \textcolor{Sepia}{\selectlanguage{english}large mat} \zh{大床垫子}  
 \zh{量词}: \textcolor{darkblue}{\textbf{\ipa{nɑ˧}}}  \mytextsc{clf}: \textcolor{darkblue}{\textbf{\ipa{nɑ˧}}} 
\lhead{\firstmark}
\rhead{\botmark}

\subsection{\hspace{-0.5cm} {\Large \textcolor{darkblue}{\textbf{\ipa{qwæ˧lo˧˥}}}}\hspace{0.5cm}[\kern2pt{\textcolor{darkblue}{\textbf{\ipa{qwæ˧lo˧}}}}\kern2pt]} \hypertarget{qw\{\string_Mlo\string_M\string_T1}{}
\markboth{\textcolor{darkblue}{\textbf{\ipa{qwæ˧lo˧˥}}}}{}
\textcolor{teal}{\mytextsc{noun}} \hspace{4pt} Tone: MH\#.
\textcolor{Sepia}{\selectlanguage{english}Passageway, small lane, small path.} \zh{过道、小道。}  ¶ \textcolor{darkblue}{\textbf{\ipa{qwæ˧lo˧-qo˥ | gɤ˩tɕo˧ le˧-jo˩}}} \textcolor{Sepia}{\selectlanguage{english}to come by the small lane} \zh{抄小道}  
 \zh{量词}: \textcolor{darkblue}{\textbf{\ipa{kʰɯ˩}}}  \mytextsc{clf}: \textcolor{darkblue}{\textbf{\ipa{kʰɯ˩}}} 
\lhead{\firstmark}
\rhead{\botmark}

\subsection{\hspace{-0.5cm} {\Large \textcolor{darkblue}{\textbf{\ipa{qwæ˧ʁo\#˥}}}}\hspace{0.5cm}[\kern2pt{\textcolor{darkblue}{\textbf{\ipa{qwæ˧ʁo˧}}}}\kern2pt]} \hypertarget{qw\{\string_MRo\#\string_T1}{}
\markboth{\textcolor{darkblue}{\textbf{\ipa{qwæ˧ʁo\#˥}}}}{}
\textcolor{teal}{\mytextsc{noun}} \hspace{4pt} Tone: \#H.
\textcolor{Sepia}{\selectlanguage{english}The bench of the main room, close to the hearth, where guests are seated.} \zh{主屋里面的长凳:客人和老人坐的地方。}  \zh{量词}: \textcolor{darkblue}{\textbf{\ipa{ɭɯ˧}}}  \mytextsc{clf}: \textcolor{darkblue}{\textbf{\ipa{ɭɯ˧}}} \textit{See:} \hyperlink{}{\textcolor{darkblue}{\textbf{\ipa{qwæ˧˥}}} \textsubscript{3}} 
\lhead{\firstmark}
\rhead{\botmark}

\subsection{\hspace{-0.5cm} {\Large \textcolor{darkblue}{\textbf{\ipa{qwæ˧ʂe\#˥}}}}\hspace{0.5cm}[\kern2pt{\textcolor{darkblue}{\textbf{\ipa{qwæ˧ʂe˥}}}}\kern2pt]} \hypertarget{qw\{\string_Ms`e\#\string_T1}{}
\markboth{\textcolor{darkblue}{\textbf{\ipa{qwæ˧ʂe\#˥}}}}{}
\textcolor{teal}{\mytextsc{noun}} \hspace{4pt} Tone: \#H.
\textcolor{Sepia}{\selectlanguage{english}Bedbug.} \zh{臭虫。}  \zh{量词}: \textcolor{darkblue}{\textbf{\ipa{mi˩}}}  \mytextsc{clf}: \textcolor{darkblue}{\textbf{\ipa{mi˩}}} 
\lhead{\firstmark}
\rhead{\botmark}

\subsection{\hspace{-0.5cm} {\Large \textcolor{darkblue}{\textbf{\ipa{qwæ˧ʂe˧lɑ˧bv̩˥}}}}\hspace{0.5cm}[\kern2pt{\textcolor{darkblue}{\textbf{\ipa{qwæ˧ʂe˧lɑ˧bv̩˧}}}}\kern2pt]} \hypertarget{qw\{\string_Ms`e\string_MlA\string_Mbv\string_=\string_T1}{}
\markboth{\textcolor{darkblue}{\textbf{\ipa{qwæ˧ʂe˧lɑ˧bv̩˥}}}}{}
\textcolor{teal}{\mytextsc{noun}} \hspace{4pt} Tone: H\#.
\textcolor{Sepia}{\selectlanguage{english}A species of worm.} \zh{一种蠕虫。}  \zh{量词}: \textcolor{darkblue}{\textbf{\ipa{mi˩}}}  \mytextsc{clf}: \textcolor{darkblue}{\textbf{\ipa{mi˩}}} 
\lhead{\firstmark}
\rhead{\botmark}

\subsection{\hspace{-0.5cm} {\Large \textcolor{darkblue}{\textbf{\ipa{qwæ˧zo˧zo˩}}}}\hspace{0.5cm}[\kern2pt{\textcolor{darkblue}{\textbf{\ipa{qwæ˧zo˧zo˥}}}}\kern2pt]} \hypertarget{qw\{\string_Mzo\string_Mzo\string_B1}{}
\markboth{\textcolor{darkblue}{\textbf{\ipa{qwæ˧zo˧zo˩}}}}{}
\textcolor{teal}{\mytextsc{noun}} \hspace{4pt} Tone: L\#.
\textcolor{Sepia}{\selectlanguage{english}The bench of the main room, close to the hearth, where guests are seated.} \zh{主屋的长凳,离火塘近。这是客人的尊座。}  \zh{量词}: \textcolor{darkblue}{\textbf{\ipa{pɤ˩}}}  \mytextsc{clf}: \textcolor{darkblue}{\textbf{\ipa{pɤ˩}}} 
\lhead{\firstmark}
\rhead{\botmark}

\subsection{\hspace{-0.5cm} {\Large \textcolor{darkblue}{\textbf{\ipa{qwæ˩ɖʐæ˩}}}}\hspace{0.5cm}[\kern2pt{\textcolor{darkblue}{\textbf{\ipa{qwæ˧ɖʐæ˧˥}}}}\kern2pt]} \hypertarget{qw\{\string_Bd`z`\{\string_B1}{}
\markboth{\textcolor{darkblue}{\textbf{\ipa{qwæ˩ɖʐæ˩}}}}{}
\textcolor{teal}{\mytextsc{noun}} \hspace{4pt} Tone: L.
\textcolor{Sepia}{\selectlanguage{english}Jaw; mouth.} \zh{颚、嘴、嘴巴、口。}  ¶ \textcolor{darkblue}{\textbf{\ipa{qwæ˩ɖʐæ˩-qo˥-ɳɯ˩ | ʈʰæ˧˥}}} \textcolor{Sepia}{\selectlanguage{english}to masticate, to gnaw} \zh{咬在嘴里}  
 \zh{量词}: \textcolor{darkblue}{\textbf{\ipa{ɭɯ˧}}}  \mytextsc{clf}: \textcolor{darkblue}{\textbf{\ipa{ɭɯ˧}}} 
\lhead{\firstmark}
\rhead{\botmark}

\subsection{\hspace{-0.5cm} {\Large \textcolor{darkblue}{\textbf{\ipa{qwæ˩\textasciitilde{}qwæ˧˥}}}}\hspace{0.5cm}[\kern2pt{\textcolor{darkblue}{\textbf{\ipa{qwæ˧qwæ˩}}}}\kern2pt]} \hypertarget{qw\{\string_B~qw\{\string_M\string_T1}{}
\markboth{\textcolor{darkblue}{\textbf{\ipa{qwæ˩\textasciitilde{}qwæ˧˥}}}}{}
\textcolor{teal}{\mytextsc{verb}} \hspace{4pt} Tone: .
\textcolor{Sepia}{\selectlanguage{english}To scratch.} \zh{抠痒。}  ¶ \textcolor{darkblue}{\textbf{\ipa{le˧-qwæ˧\textasciitilde{}qwæ˩-ze˩}}} \textcolor{Sepia}{\selectlanguage{english}\mytextsc{accomp} \string_ \mytextsc{red} \mytextsc{pfv}} \zh{\mytextsc{accomp} \string_ \mytextsc{red} \mytextsc{pfv}}  

\lhead{\firstmark}
\rhead{\botmark}

\subsection{\hspace{-0.5cm} {\Large \textcolor{darkblue}{\textbf{\ipa{qwæ˩ʂv̩˧˥}}}}\hspace{0.5cm}[\kern2pt{\textcolor{darkblue}{\textbf{\ipa{qwæ˧ʂv̩˧}}}}\kern2pt]} \hypertarget{qw\{\string_Bs`v\string_=\string_M\string_T1}{}
\markboth{\textcolor{darkblue}{\textbf{\ipa{qwæ˩ʂv̩˧˥}}}}{}
\textcolor{teal}{\mytextsc{noun}} \hspace{4pt} Tone: LM+MH\#.
\textcolor{Sepia}{\selectlanguage{english}Bit (of a bridle).} \zh{马嚼子。}  ¶ \textcolor{darkblue}{\textbf{\ipa{ʐwæ˧-qwæ˥ʂv̩˩}}} \textcolor{Sepia}{\selectlanguage{english}bit of a horse's bridle} \zh{马嚼子}  
 \zh{量词}: \textcolor{darkblue}{\textbf{\ipa{nɑ˧}}}  \mytextsc{clf}: \textcolor{darkblue}{\textbf{\ipa{nɑ˧}}} 
\lhead{\firstmark}
\rhead{\botmark}

\subsection{\hspace{-0.5cm} {\Large \textcolor{darkblue}{\textbf{\ipa{qwæ˧˥}}} \textsubscript{1}}\hspace{0.5cm}[\kern2pt{\textcolor{darkblue}{\textbf{\ipa{qwæ˧˥}}}}\kern2pt]} \hypertarget{qw\{\string_M\string_T1}{}
\markboth{\textcolor{darkblue}{\textbf{\ipa{qwæ˧˥}}} \textsubscript{1}}{}
\textcolor{teal}{\mytextsc{verb}} \hspace{4pt} Tone: MH.
\ding{202} \textcolor{Sepia}{\selectlanguage{english}To dig.} \zh{挖(土)。}  ¶ \textcolor{darkblue}{\textbf{\ipa{tv̩˧qʰv̩˧ qwæ˧˥}}} \textcolor{Sepia}{\selectlanguage{english}to dig a hole} \zh{挖洞}  
 ¶ \textcolor{darkblue}{\textbf{\ipa{ʈʂe˧ qwæ˩}}} \textcolor{Sepia}{\selectlanguage{english}to dig out the soil} \zh{挖土}  
 ¶ \textcolor{darkblue}{\textbf{\ipa{qʰæ˧lo˧ qwæ˥}}} \textcolor{Sepia}{\selectlanguage{english}to dig a ditch} \zh{挖水沟}  
 ¶ \textcolor{darkblue}{\textbf{\ipa{jɤ˩jo˥ qwæ˩}}} \textcolor{Sepia}{\selectlanguage{english}to dig out potatoes, to harvest potatoes} \zh{挖洋芋}  
\ding{203} \textcolor{Sepia}{\selectlanguage{english}To scoop (water).} \zh{舀(水)。}  ¶ \textcolor{darkblue}{\textbf{\ipa{dʑɯ˩ qwæ˩˥}}} \textcolor{Sepia}{\selectlanguage{english}to scoop water} \zh{舀水}  

\lhead{\firstmark}
\rhead{\botmark}

\subsection{\hspace{-0.5cm} {\Large \textcolor{darkblue}{\textbf{\ipa{qwæ˧˥}}} \textsubscript{2}}\hspace{0.5cm}[\kern2pt{\textcolor{darkblue}{\textbf{\ipa{qwæ˧˥}}}}\kern2pt]} \hypertarget{qw\{\string_M\string_T2}{}
\markboth{\textcolor{darkblue}{\textbf{\ipa{qwæ˧˥}}} \textsubscript{2}}{}
\textcolor{teal}{\mytextsc{verb}} \hspace{4pt} Tone: MH.
\textcolor{Sepia}{\selectlanguage{english}To engrave.} \zh{雕刻。}  ¶ \textcolor{darkblue}{\textbf{\ipa{bæ˩bæ˩ qwæ˥}}} \textcolor{Sepia}{\selectlanguage{english}to engrave a flower} \zh{刻花}  
 ¶ \textcolor{darkblue}{\textbf{\ipa{qwæ˩\textasciitilde{}qwæ˧˥}}} \textcolor{Sepia}{\selectlanguage{english}\mytextsc{red}} \zh{\mytextsc{重叠}}  
 ¶ \textcolor{darkblue}{\textbf{\ipa{bæ˩bæ˩ qwæ˥\textasciitilde{}qwæ˩}}} \textcolor{Sepia}{\selectlanguage{english}to engrave flowers} \zh{刻一朵花}  

\lhead{\firstmark}
\rhead{\botmark}

\subsection{\hspace{-0.5cm} {\Large \textcolor{darkblue}{\textbf{\ipa{qwæ˧˥}}} \textsubscript{3}}\hspace{0.5cm}[\kern2pt{\textcolor{darkblue}{\textbf{\ipa{qwæ˧˥}}}}\kern2pt]} \hypertarget{qw\{\string_M\string_T3}{}
\markboth{\textcolor{darkblue}{\textbf{\ipa{qwæ˧˥}}} \textsubscript{3}}{}
\textcolor{teal}{\mytextsc{noun}} \hspace{4pt} Tone: \#H.
\textcolor{Sepia}{\selectlanguage{english}The bench of the main room, close to the hearth, where guests are seated.} \zh{主屋里面的长凳:客人和老人坐的地方。}  \zh{量词}: \textcolor{darkblue}{\textbf{\ipa{ɭɯ˧}}}  \mytextsc{clf}: \textcolor{darkblue}{\textbf{\ipa{ɭɯ˧}}} \textit{See:} \hyperlink{}{\textcolor{darkblue}{\textbf{\ipa{qwæ˧ʁo\#˥}}}} 
\lhead{\firstmark}
\rhead{\botmark}

\subsection{\hspace{-0.5cm} {\Large \textcolor{darkblue}{\textbf{\ipa{qwæ˩˥}}}}\hspace{0.5cm}[\kern2pt{\textcolor{darkblue}{\textbf{\ipa{qwæ˩˥}}}}\kern2pt]} \hypertarget{qw\{\string_B\string_T1}{}
\markboth{\textcolor{darkblue}{\textbf{\ipa{qwæ˩˥}}}}{}
\textcolor{teal}{\mytextsc{noun}} \hspace{4pt} Tone: LH.
\textcolor{Sepia}{\selectlanguage{english}Jaw (monosyllable).} \zh{嘴巴(单音节)。}  \zh{量词}: \textcolor{darkblue}{\textbf{\ipa{ɭɯ˧}}}  \mytextsc{clf}: \textcolor{darkblue}{\textbf{\ipa{ɭɯ˧}}} 
\lhead{\firstmark}
\rhead{\botmark}

\subsection{\hspace{-0.5cm} {\Large \textcolor{darkblue}{\textbf{\ipa{qwɤ˧}}}}\hspace{0.5cm}[\kern2pt{\textcolor{darkblue}{\textbf{\ipa{qwɤ˥}}}}\kern2pt]} \hypertarget{qw7\string_M1}{}
\markboth{\textcolor{darkblue}{\textbf{\ipa{qwɤ˧}}}}{}
\textcolor{teal}{\mytextsc{noun}} \hspace{4pt} Tone: M.
\textcolor{Sepia}{\selectlanguage{english}Fire pit.} \zh{火塘。}  ¶ \textcolor{darkblue}{\textbf{\ipa{qwɤ˧, | mv̩˧ kʰɯ˩-di˩!}}} \textcolor{Sepia}{\selectlanguage{english}The fire pit is the place where one puts fire / where one does a fire!} \zh{火塘,就是升火的地方!}  
 \zh{量词}: \textcolor{darkblue}{\textbf{\ipa{ɭɯ˧}}}  \mytextsc{clf}: \textcolor{darkblue}{\textbf{\ipa{ɭɯ˧}}} 
\lhead{\firstmark}
\rhead{\botmark}

\subsection{\hspace{-0.5cm} {\Large \textcolor{darkblue}{\textbf{\ipa{qwɤ˧\textsubscript{a}}}}}\hspace{0.5cm}[\kern2pt{\textcolor{darkblue}{\textbf{\ipa{qwɤ˥}}}}\kern2pt]} \hypertarget{qw7\string_Ma1}{}
\markboth{\textcolor{darkblue}{\textbf{\ipa{qwɤ˧\textsubscript{a}}}}}{}
\textcolor{teal}{\mytextsc{verb}} \hspace{4pt} Tone: M\textsubscript{a}.
\textcolor{Sepia}{\selectlanguage{english}To accuse, to denounce.} \zh{告状。}  ¶ \textcolor{darkblue}{\textbf{\ipa{mɤ˧-qwɤ˧}}} \textcolor{Sepia}{\selectlanguage{english}\mytextsc{neg}} \zh{不告状}  
 ¶ \textcolor{darkblue}{\textbf{\ipa{hĩ˧ qwɤ˩}}} \textcolor{Sepia}{\selectlanguage{english}to accuse someone, to denounce someone} \zh{告一个人}  
 ¶ \textcolor{darkblue}{\textbf{\ipa{njɤ˧-ɳɯ˧ | qwɤ˧-bi˧!}}} \textcolor{Sepia}{\selectlanguage{english}I am going to denounce/accuse} \zh{我要告状!}  
 ¶ \textcolor{darkblue}{\textbf{\ipa{no˧ | le˧-qwɤ˧-hõ˧!}}} \textcolor{Sepia}{\selectlanguage{english}Go and denounce (him/her)!} \zh{你去告状吧!}  
 ¶ \textcolor{darkblue}{\textbf{\ipa{qwɤ˧\textasciitilde{}qwɤ˩}}} \textcolor{Sepia}{\selectlanguage{english}\mytextsc{red}} \zh{\mytextsc{重叠}}  

\lhead{\firstmark}
\rhead{\botmark}

\subsection{\hspace{-0.5cm} {\Large \textcolor{darkblue}{\textbf{\ipa{qwɤ˧ɭɯ\#˥}}}}\hspace{0.5cm}[\kern2pt{\textcolor{darkblue}{\textbf{\ipa{qwɤ˧ɭɯ˧}}}}\kern2pt]} \hypertarget{qw7\string_Ml\string_RM\#\string_T1}{}
\markboth{\textcolor{darkblue}{\textbf{\ipa{qwɤ˧ɭɯ\#˥}}}}{}
\textcolor{teal}{\mytextsc{noun}} \hspace{4pt} Tone: \#H.
\textcolor{Sepia}{\selectlanguage{english}Campfire.} \zh{营火、篝火。}  ¶ \textcolor{darkblue}{\textbf{\ipa{qwɤ˧ɭɯ˧-pʰɤ˧bɤ˥}}} \textcolor{Sepia}{\selectlanguage{english}the gifts offered to the ancestors, at the fireplace: even when building a campfire for one day only on the mountain, one offers a little food to the ancestors before beginning the meal (in the same way as is done at home)} \zh{敬给祖先的礼物:即使在山上升起篝火野餐,还是要像在家里一样,用餐前先敬给祖先一些饭。}  

\lhead{\firstmark}
\rhead{\botmark}

\subsection{\hspace{-0.5cm} {\Large \textcolor{darkblue}{\textbf{\ipa{qwɤ˩\textsubscript{a}}}}}\hspace{0.5cm}[\kern2pt{\textcolor{darkblue}{\textbf{\ipa{qwɤ˩˥}}}}\kern2pt]} \hypertarget{qw7\string_Ba1}{}
\markboth{\textcolor{darkblue}{\textbf{\ipa{qwɤ˩\textsubscript{a}}}}}{}
\textcolor{teal}{\mytextsc{verb}} \hspace{4pt} Tone: L\textsubscript{a}.
\textcolor{Sepia}{\selectlanguage{english}To grow.} \zh{生长、长。}  ¶ \textcolor{darkblue}{\textbf{\ipa{gɤ˩-qwɤ˥}}} \textcolor{Sepia}{\selectlanguage{english}to grow} \zh{长大,生长}  
 ¶ \textcolor{darkblue}{\textbf{\ipa{ʈʂʰɯ˧ | gɤ˩-qwɤ˥-ze˩!}}} \textcolor{Sepia}{\selectlanguage{english}(S)he has grown up! / (S)he has grown a lot! (About a child)} \zh{他长大了!(关于一个小孩)}  

\lhead{\firstmark}
\rhead{\botmark}

\subsection{\hspace{-0.5cm} {\Large \textcolor{darkblue}{\textbf{\ipa{qwɤ˩pi˩}}}}\hspace{0.5cm}[\kern2pt{\textcolor{darkblue}{\textbf{\ipa{qwɤ˩pi˩˥}}}}\kern2pt]} \hypertarget{qw7\string_Bpi\string_B1}{}
\markboth{\textcolor{darkblue}{\textbf{\ipa{qwɤ˩pi˩}}}}{}
\textcolor{teal}{\mytextsc{noun}} \hspace{4pt} Tone: L.
\textcolor{Sepia}{\selectlanguage{english}Mouth.} \zh{嘴巴。}  ¶ \textcolor{darkblue}{\textbf{\ipa{qwɤ˩pi˩-qo˩lo˥}}} \textcolor{Sepia}{\selectlanguage{english}inside the mouth} \zh{嘴巴里}  
 ¶ \textcolor{darkblue}{\textbf{\ipa{[F5] ko˩pi˩-ko˩lo˧}}} \textcolor{Sepia}{\selectlanguage{english}inside the mouth} \zh{嘴巴里}  
 \zh{量词}: \textcolor{darkblue}{\textbf{\ipa{ɭɯ˧}}}  \mytextsc{clf}: \textcolor{darkblue}{\textbf{\ipa{ɭɯ˧}}} 
\lhead{\firstmark}
\rhead{\botmark}

\newpage
\section*{\centering- \textcolor{darkblue}{\textbf{\ipa{qʰ}}} -}
\subsection{\hspace{-0.5cm} {\Large \textcolor{darkblue}{\textbf{\ipa{qʰɑ˥}}}}\hspace{0.5cm}[\kern2pt{\textcolor{darkblue}{\textbf{\ipa{qʰɑ˥}}}}\kern2pt]} \hypertarget{q\string_hA\string_T1}{}
\markboth{\textcolor{darkblue}{\textbf{\ipa{qʰɑ˥}}}}{}
\textcolor{teal}{\mytextsc{adjective}} \hspace{4pt} Tone: H.
\textcolor{Sepia}{\selectlanguage{english}Bitter.} \zh{苦。} 
\lhead{\firstmark}
\rhead{\botmark}

\subsection{\hspace{-0.5cm} {\Large \textcolor{darkblue}{\textbf{\ipa{qʰɑ˧-}}}}\hspace{0.5cm}[\kern2pt{\textcolor{darkblue}{\textbf{\ipa{qʰɑ˥}}}}\kern2pt]} \hypertarget{q\string_hA\string_M-1}{}
\markboth{\textcolor{darkblue}{\textbf{\ipa{qʰɑ˧-}}}}{}
\textcolor{teal}{\mytextsc{adverb(ial)}} \hspace{4pt} Tone: .
\textcolor{Sepia}{\selectlanguage{english}Very, extremely.} \zh{多么、非常。}  ¶ \textcolor{darkblue}{\textbf{\ipa{qʰɑ˧-ɖɯ˧-hĩ˧}}} \textcolor{Sepia}{\selectlanguage{english}extremely big} \zh{非常大}  
 ¶ \textcolor{darkblue}{\textbf{\ipa{qʰɑ˧-ɖɯ˧-gv̩˧}}} \textcolor{Sepia}{\selectlanguage{english}extremely large; how large!} \zh{非常大}  
 ¶ \textcolor{darkblue}{\textbf{\ipa{qʰɑ˧-ʂwæ˧-gv̩˧}}} \textcolor{Sepia}{\selectlanguage{english}extremely tall; how tall!} \zh{很高、非常高}  
 ¶ \textcolor{darkblue}{\textbf{\ipa{qʰɑ˧-ʂwæ˧-mi˧zo˥}}} \textcolor{Sepia}{\selectlanguage{english}extremely tall} \zh{很高}  
 ¶ \textcolor{darkblue}{\textbf{\ipa{qʰɑ˧-ɖɯ˧-mi˧zo˥}}} \textcolor{Sepia}{\selectlanguage{english}extremely big} \zh{很大}  
\textit{See:} \hyperlink{}{\textcolor{darkblue}{\textbf{\ipa{qʰɑ˧}}} \textsubscript{1}} 
\lhead{\firstmark}
\rhead{\botmark}

\subsection{\hspace{-0.5cm} {\Large \textcolor{darkblue}{\textbf{\ipa{qʰɑ˧}}} \textsubscript{1}}\hspace{0.5cm}[\kern2pt{\textcolor{darkblue}{\textbf{\ipa{qʰɑ˥}}}}\kern2pt]} \hypertarget{q\string_hA\string_M1}{}
\markboth{\textcolor{darkblue}{\textbf{\ipa{qʰɑ˧}}} \textsubscript{1}}{}
\textcolor{teal}{\mytextsc{pronoun/pronominal}} \hspace{4pt} Tone: M.
\textcolor{Sepia}{\selectlanguage{english}How many (small number).} \zh{几、多少。}  ¶ \textcolor{darkblue}{\textbf{\ipa{hĩ˧ | qʰɑ˧-kv̩˧˥?}}} \textcolor{Sepia}{\selectlanguage{english}how many people?} \zh{几个人?}  
 ¶ \textcolor{darkblue}{\textbf{\ipa{bæ˩bæ˩˥ | qʰɑ˧-bæ˩?}}} \textcolor{Sepia}{\selectlanguage{english}how many flowers?} \zh{几朵花?}  
 ¶ \textcolor{darkblue}{\textbf{\ipa{qʰɑ˧-ʑi˩?}}} \textcolor{Sepia}{\selectlanguage{english}how many families?} \zh{几家?}  
 ¶ \textcolor{darkblue}{\textbf{\ipa{hɑ˧ | qʰɑ˧-tɕʰi˩?}}} \textcolor{Sepia}{\selectlanguage{english}how many meals?} \zh{几顿饭?}  
 ¶ \textcolor{darkblue}{\textbf{\ipa{qʰɑ˧-ɲi˧?}}} \textcolor{Sepia}{\selectlanguage{english}how many days?} \zh{几天?}  
 ¶ \textcolor{darkblue}{\textbf{\ipa{qʰɑ˧-kʰv̩˧ gv̩˧-ze˩?}}} \textcolor{Sepia}{\selectlanguage{english}How old are you / is (s)he?} \zh{几岁了?}  
 ¶ \textcolor{darkblue}{\textbf{\ipa{qʰɑ˧-kʰv̩˧˥?}}} \textcolor{Sepia}{\selectlanguage{english}how many years?} \zh{几年?}  
 ¶ \textcolor{darkblue}{\textbf{\ipa{qʰɑ˧-kʰwɤ˧˥?}}} \textcolor{Sepia}{\selectlanguage{english}how many pieces?} \zh{几块?}  
 ¶ \textcolor{darkblue}{\textbf{\ipa{qʰɑ˧-nɑ˧?}}} \textcolor{Sepia}{\selectlanguage{english}how many (tools...)?} \zh{几把?}  
 ¶ \textcolor{darkblue}{\textbf{\ipa{sɯ˩tʰi˩˥ | qʰɑ˧-nɑ˧ dʑo˧?}}} \textcolor{Sepia}{\selectlanguage{english}How many knives are there?} \zh{有几把刀?}  
 ¶ \textcolor{darkblue}{\textbf{\ipa{qʰɑ˧-kʰɯ˩}}} \textcolor{Sepia}{\selectlanguage{english}how many (long objects)} \zh{几条}  
 ¶ \textcolor{darkblue}{\textbf{\ipa{qʰɑ˧-kʰɯ˩ dʑo˩?}}} \textcolor{Sepia}{\selectlanguage{english}How many (long objects) are there?} \zh{有几条?}  
 ¶ \textcolor{darkblue}{\textbf{\ipa{qʰɑ˧-mæ˩ dʑo˩?}}} \textcolor{Sepia}{\selectlanguage{english}How much money do (you) have?} \zh{有几块(钱)?}  
 ¶ \textcolor{darkblue}{\textbf{\ipa{si˧dzi˩ | qʰɑ˧-dzi˩?}}} \textcolor{Sepia}{\selectlanguage{english}how many trees?} \zh{几棵树?}  
 ¶ \textcolor{darkblue}{\textbf{\ipa{si˧kɤ˧˥ | qʰɑ˧-kɤ˧˥?}}} \textcolor{Sepia}{\selectlanguage{english}how many branches?} \zh{几枝树枝?}  
 ¶ \textcolor{darkblue}{\textbf{\ipa{qʰɑ˧-kʰɤ˧˥?}}} \textcolor{Sepia}{\selectlanguage{english}how many baskets?} \zh{几筐?}  
\textit{See:} \hyperlink{}{\textcolor{darkblue}{\textbf{\ipa{qʰɑ˧}}} \textsubscript{2}} 
\lhead{\firstmark}
\rhead{\botmark}

\subsection{\hspace{-0.5cm} {\Large \textcolor{darkblue}{\textbf{\ipa{qʰɑ˧}}} \textsubscript{2}}\hspace{0.5cm}[\kern2pt{\textcolor{darkblue}{\textbf{\ipa{qʰɑ˥}}}}\kern2pt]} \hypertarget{q\string_hA\string_M2}{}
\markboth{\textcolor{darkblue}{\textbf{\ipa{qʰɑ˧}}} \textsubscript{2}}{}
\textcolor{teal}{\mytextsc{adverb(ial)}} \hspace{4pt} Tone: M.
\textcolor{Sepia}{\selectlanguage{english}A few; several; some.} \zh{几(如:十几个)。}  ¶ \textcolor{darkblue}{\textbf{\ipa{tsʰe˩-qʰɑ˩˥}}} \textcolor{Sepia}{\selectlanguage{english}ten and a few more (i.e. between ten and twenty)} \zh{十几个、十来个}  
 ¶ \textcolor{darkblue}{\textbf{\ipa{tsʰe˩-qʰɑ˩-kv̩˩˥}}} \textcolor{Sepia}{\selectlanguage{english}ten and a few more (i.e. between ten and twenty)} \zh{十几个、十来个}  
\textit{See:} \hyperlink{}{\textcolor{darkblue}{\textbf{\ipa{qʰɑ˧}}} \textsubscript{1}} 
\lhead{\firstmark}
\rhead{\botmark}

\subsection{\hspace{-0.5cm} {\Large \textcolor{darkblue}{\textbf{\ipa{qʰɑ˧dze˧}}}}\hspace{0.5cm}[\kern2pt{\textcolor{darkblue}{\textbf{\ipa{qʰɑ˧dze˧}}}}\kern2pt]} \hypertarget{q\string_hA\string_Mdze\string_M1}{}
\markboth{\textcolor{darkblue}{\textbf{\ipa{qʰɑ˧dze˧}}}}{}
\textcolor{teal}{\mytextsc{noun}} \hspace{4pt} Tone: M.
\textcolor{Sepia}{\selectlanguage{english}Sweet corn; maize; Indian corn.} \zh{玉米、包谷。}  ¶ \textcolor{darkblue}{\textbf{\ipa{qʰɑ˧dze˧-kʰɯ˩ʈɯ˩}}} \textcolor{Sepia}{\selectlanguage{english}the roots of the sweetcorn plant} \zh{玉米的根}  
 ¶ \textcolor{darkblue}{\textbf{\ipa{qʰɑ˧dze˧ qʰæ˩}}} \textcolor{Sepia}{\selectlanguage{english}to cut ears of sweetcorn, to snap off ears of sweetcorn} \zh{采玉米:折断玉米棒子}  
 ¶ \textcolor{darkblue}{\textbf{\ipa{qʰɑ˧dze˧ ɖʐɤ˧˥}}} \textcolor{Sepia}{\selectlanguage{english}to harvest sweetcorn, to pick sweetcorn} \zh{采玉米}  
 ¶ \textcolor{darkblue}{\textbf{\ipa{qʰɑ˧dze˧-tsɑ˩bɤ˩ | ɖɯ˧-mɤ˩}}} \textcolor{Sepia}{\selectlanguage{english}a little sweetcorn flour} \zh{一点玉米粉}  
 ¶ \textcolor{darkblue}{\textbf{\ipa{qʰɑ˧dze˧-hɑ˧bɤ˥, | qʰɑ˧dze˧-hɑ˧ɭɯ\#˥, | qʰɑ˧dze˧-tsɑ˩bɤ˩}}} \textcolor{Sepia}{\selectlanguage{english}three forms of sweetcorn: sweetcorn ear; sweetcorn grains; sweetcorn flour} \zh{玉米的三种形态:玉米棒子,玉米粒,玉米粉}  
 \zh{量词}: \textcolor{darkblue}{\textbf{\ipa{kɤ˧˥}}}  \mytextsc{clf}: \textcolor{darkblue}{\textbf{\ipa{kɤ˧˥}}} 
\lhead{\firstmark}
\rhead{\botmark}

\subsection{\hspace{-0.5cm} {\Large \textcolor{darkblue}{\textbf{\ipa{qʰɑ˧dze˧-hwæ˩-di˩}}}}\hspace{0.5cm}[\kern2pt{\textcolor{darkblue}{\textbf{\ipa{xxxx non-correspondance entre le nombre de morphèmes et le nombre de tons de morphèmes}}}}\kern2pt]} \hypertarget{q\string_hA\string_Mdze\string_M-hw\{\string_B-di\string_B1}{}
\markboth{\textcolor{darkblue}{\textbf{\ipa{qʰɑ˧dze˧-hwæ˩-di˩}}}}{}
\textcolor{teal}{\mytextsc{noun}} \hspace{4pt} Tone: \mytextsc{L}.
\textcolor{Sepia}{\selectlanguage{english}Horizontal beams of the rack for drying grain: 'the place to hang sweetcorn'.} \zh{粮架的横梁。}  \zh{量词}: \textcolor{darkblue}{\textbf{\ipa{kɤ˧˥}}}  \mytextsc{clf}: \textcolor{darkblue}{\textbf{\ipa{kɤ˧˥}}} 
\lhead{\firstmark}
\rhead{\botmark}

\subsection{\hspace{-0.5cm} {\Large \textcolor{darkblue}{\textbf{\ipa{qʰɑ˧dze˧-lv̩˧}}}}\hspace{0.5cm}[\kern2pt{\textcolor{darkblue}{\textbf{\ipa{xxxx non-correspondance entre le nombre de morphèmes et le nombre de tons de morphèmes}}}}\kern2pt]} \hypertarget{q\string_hA\string_Mdze\string_M-lv\string_=\string_M1}{}
\markboth{\textcolor{darkblue}{\textbf{\ipa{qʰɑ˧dze˧-lv̩˧}}}}{}
\textcolor{teal}{\mytextsc{noun}} \hspace{4pt} Tone: M.
\textcolor{Sepia}{\selectlanguage{english}Maize field.} \zh{包谷田、玉米田。}  \zh{量词}: \textcolor{darkblue}{\textbf{\ipa{pʰv̩˩}}}  \mytextsc{clf}: \textcolor{darkblue}{\textbf{\ipa{pʰv̩˩}}} 
\lhead{\firstmark}
\rhead{\botmark}

\subsection{\hspace{-0.5cm} {\Large \textcolor{darkblue}{\textbf{\ipa{qʰɑ˧tɑ˧}}}}\hspace{0.5cm}[\kern2pt{\textcolor{darkblue}{\textbf{\ipa{qʰɑ˧tɑ˧}}}}\kern2pt]} \hypertarget{q\string_hA\string_MtA\string_M1}{}
\markboth{\textcolor{darkblue}{\textbf{\ipa{qʰɑ˧tɑ˧}}}}{}
\textcolor{teal}{\mytextsc{pronoun/pronominal}} \hspace{4pt} Tone: M.
\textcolor{Sepia}{\selectlanguage{english}When.} \zh{什么时候。}  ¶ \textcolor{darkblue}{\textbf{\ipa{qʰɑ˧tɑ˧ bi˧?}}} \textcolor{Sepia}{\selectlanguage{english}When will you go?} \zh{你什么时候去?}  

\lhead{\firstmark}
\rhead{\botmark}

\subsection{\hspace{-0.5cm} {\Large \textcolor{darkblue}{\textbf{\ipa{qʰɑ˩jɤ˩}}}}\hspace{0.5cm}[\kern2pt{\textcolor{darkblue}{\textbf{\ipa{qʰɑ˩jɤ˩˥}}}}\kern2pt]} \hypertarget{q\string_hA\string_Bj7\string_B1}{}
\markboth{\textcolor{darkblue}{\textbf{\ipa{qʰɑ˩jɤ˩}}}}{}
\textcolor{teal}{\mytextsc{pronoun/pronominal}} \hspace{4pt} Tone: L.
\textcolor{Sepia}{\selectlanguage{english}How many.} \zh{多少。}  ¶ \textcolor{darkblue}{\textbf{\ipa{qʰɑ˩jɤ˩ tʰi˥-ki˩?}}} \textcolor{Sepia}{\selectlanguage{english}How much does it cost?} \zh{要给多少? = 多少钱?}  
 ¶ \textcolor{darkblue}{\textbf{\ipa{qʰɑ˩jɤ˩ ɲi˧?}}} \textcolor{Sepia}{\selectlanguage{english}How much do (you) need?} \zh{要多少?}  
 ¶ \textcolor{darkblue}{\textbf{\ipa{ɖʐe˧ | qʰɑ˩jɤ˩ ɲi˧?}}} \textcolor{Sepia}{\selectlanguage{english}How much money does it cost?} \zh{要多少钱?}  

\lhead{\firstmark}
\rhead{\botmark}

\subsection{\hspace{-0.5cm} {\Large \textcolor{darkblue}{\textbf{\ipa{qʰɑ˩ne˩}}}}\hspace{0.5cm}[\kern2pt{\textcolor{darkblue}{\textbf{\ipa{qʰɑ˩ne˩˥}}}}\kern2pt]} \hypertarget{q\string_hA\string_Bne\string_B1}{}
\markboth{\textcolor{darkblue}{\textbf{\ipa{qʰɑ˩ne˩}}}}{}
\textcolor{teal}{\mytextsc{pronoun/pronominal}} \hspace{4pt} Tone: L.
\textcolor{Sepia}{\selectlanguage{english}How.} \zh{怎么样。}  ¶ \textcolor{darkblue}{\textbf{\ipa{qʰɑ˩ne˩ ʝi˥?}}} \textcolor{Sepia}{\selectlanguage{english}how to do / how is it done?} \zh{怎么做?}  
 ¶ \textcolor{darkblue}{\textbf{\ipa{qʰɑ˩ne˩ ʝi˥-tso˩-ɲi˩?}}} \textcolor{Sepia}{\selectlanguage{english}how must one do / how is it done?} \zh{要怎么做?}  
 ¶ \textcolor{darkblue}{\textbf{\ipa{qʰɑ˩ne˩ gv̩˩˥?}}} \textcolor{Sepia}{\selectlanguage{english}how to do / how is it done?} \zh{怎么做?}  
 ¶ \textcolor{darkblue}{\textbf{\ipa{qʰɑ˩ne˩ gv̩˩-ho˥-ze˩?}}} \textcolor{Sepia}{\selectlanguage{english}What happened?} \zh{怎么样了?发展到什么程度?}  

\lhead{\firstmark}
\rhead{\botmark}

\subsection{\hspace{-0.5cm} {\Large \textcolor{darkblue}{\textbf{\ipa{qʰæ˥}}}}\hspace{0.5cm}[\kern2pt{\textcolor{darkblue}{\textbf{\ipa{qʰæ˥}}}}\kern2pt]} \hypertarget{q\string_h\{\string_T1}{}
\markboth{\textcolor{darkblue}{\textbf{\ipa{qʰæ˥}}}}{}
\textcolor{teal}{\mytextsc{noun}} \hspace{4pt} Tone: \#H.
\ding{202} \textcolor{Sepia}{\selectlanguage{english}Excrements, dung, dropping.} \zh{屎、垃圾、 肥料。}  ¶ \textcolor{darkblue}{\textbf{\ipa{qʰæ˧ ɖɯ˧-pɤ˧ ʂe˧˥}}} \textcolor{Sepia}{\selectlanguage{english}to defecate} \zh{拉一泡屎}  
 ¶ \textcolor{darkblue}{\textbf{\ipa{qʰæ˧ kv̩˥}}} \textcolor{Sepia}{\selectlanguage{english}to pick up dung} \zh{捡(马……)屎}  
 ¶ \textcolor{darkblue}{\textbf{\ipa{qʰæ˧-pi˩ kv̩˩}}} \textcolor{Sepia}{\selectlanguage{english}to pick up a little dung (to fertilize the fields)} \zh{捡一点(马……)屎}  
 ¶ \textcolor{darkblue}{\textbf{\ipa{qʰæ˧ ɖɯ˧-pi˧ kv̩˥}}} \textcolor{Sepia}{\selectlanguage{english}as above: to pick up a little dung (to fertilize the fields)} \zh{同上:捡一点(马……)屎}  
 \zh{量词}: \textcolor{darkblue}{\textbf{\ipa{pɤ˥}}} \ding{203} \textcolor{Sepia}{\selectlanguage{english}Flatulence, fart.} \zh{屁。}  ¶ \textcolor{darkblue}{\textbf{\ipa{qʰæ˧ kʰɯ˩}}} \textcolor{Sepia}{\selectlanguage{english}to fart} \zh{放屁}  
 ¶ \textcolor{darkblue}{\textbf{\ipa{qʰæ˧ | ɖɯ˧-pɤ˥ kʰɯ˩}}} \textcolor{Sepia}{\selectlanguage{english}to fart, to make a fart} \zh{放一个屁}  
 \zh{量词}: \textcolor{darkblue}{\textbf{\ipa{pɤ˥}}} \ding{204} \textcolor{Sepia}{\selectlanguage{english}Refuse, garbage.} \zh{垃圾。}  \mytextsc{clf}: \textcolor{darkblue}{\textbf{\ipa{pɤ˥}}} \textcolor{darkblue}{\textbf{\ipa{pɤ˥}}} 
\lhead{\firstmark}
\rhead{\botmark}

\subsection{\hspace{-0.5cm} {\Large \textcolor{darkblue}{\textbf{\ipa{qʰæ˥}}} \textsubscript{1}}\hspace{0.5cm}[\kern2pt{\textcolor{darkblue}{\textbf{\ipa{qʰæ˥}}}}\kern2pt]} \hypertarget{q\string_h\{\string_T1}{}
\markboth{\textcolor{darkblue}{\textbf{\ipa{qʰæ˥}}} \textsubscript{1}}{}
\textcolor{teal}{\mytextsc{verb}} \hspace{4pt} Tone: H.
\textcolor{Sepia}{\selectlanguage{english}To gnaw, to nibble.} \zh{啃(啃骨头)。} 
\lhead{\firstmark}
\rhead{\botmark}

\subsection{\hspace{-0.5cm} {\Large \textcolor{darkblue}{\textbf{\ipa{qʰæ˥}}} \textsubscript{2}}\hspace{0.5cm}[\kern2pt{\textcolor{darkblue}{\textbf{\ipa{qʰæ˥}}}}\kern2pt]} \hypertarget{q\string_h\{\string_T2}{}
\markboth{\textcolor{darkblue}{\textbf{\ipa{qʰæ˥}}} \textsubscript{2}}{}
\textcolor{teal}{\mytextsc{adjective}} \hspace{4pt} Tone: H.
\textcolor{Sepia}{\selectlanguage{english}Cold (water).} \zh{冷(水)。}  ¶ \textcolor{darkblue}{\textbf{\ipa{dʑɯ˩qʰæ˩}}} \textcolor{Sepia}{\selectlanguage{english}cold water} \zh{凉水}  
 ¶ \textcolor{darkblue}{\textbf{\ipa{qʰæ˧-ɕjæ˧-gv̩˧}}} \textcolor{Sepia}{\selectlanguage{english}very cold} \zh{冷得很}  

\lhead{\firstmark}
\rhead{\botmark}

\subsection{\hspace{-0.5cm} {\Large \textcolor{darkblue}{\textbf{\ipa{qʰæ˥}}} \textsubscript{3}}\hspace{0.5cm}[\kern2pt{\textcolor{darkblue}{\textbf{\ipa{qʰæ˥}}}}\kern2pt]} \hypertarget{q\string_h\{\string_T3}{}
\markboth{\textcolor{darkblue}{\textbf{\ipa{qʰæ˥}}} \textsubscript{3}}{}
\textcolor{teal}{\mytextsc{noun}} \hspace{4pt} Tone: \#H.
\textcolor{Sepia}{\selectlanguage{english}Trench (monosyllable).} \zh{水沟(单音节)。}  \zh{量词}: \textcolor{darkblue}{\textbf{\ipa{kʰɯ˩}}}  \mytextsc{clf}: \textcolor{darkblue}{\textbf{\ipa{kʰɯ˩}}} 
\lhead{\firstmark}
\rhead{\botmark}

\subsection{\hspace{-0.5cm} {\Large \textcolor{darkblue}{\textbf{\ipa{qʰæ˧\textsubscript{b}}}}}\hspace{0.5cm}[\kern2pt{\textcolor{darkblue}{\textbf{\ipa{qʰæ˩˥}}}}\kern2pt]} \hypertarget{q\string_h\{\string_Mb1}{}
\markboth{\textcolor{darkblue}{\textbf{\ipa{qʰæ˧\textsubscript{b}}}}}{}
\textcolor{teal}{\mytextsc{verb}} \hspace{4pt} Tone: M\textsubscript{b}.
\textcolor{Sepia}{\selectlanguage{english}To break (a stick breaks).} \zh{断,破(棍子,竹竿)。}  ¶ \textcolor{darkblue}{\textbf{\ipa{le˧-qʰæ˧-ze˧}}} \textcolor{Sepia}{\selectlanguage{english}\mytextsc{accomp} \string_ \mytextsc{pfv}} \zh{断了}  
 ¶ \textcolor{darkblue}{\textbf{\ipa{si˧ qʰæ˩}}} \textcolor{Sepia}{\selectlanguage{english}to break wood} \zh{砸木头}  

\lhead{\firstmark}
\rhead{\botmark}

\subsection{\hspace{-0.5cm} {\Large \textcolor{darkblue}{\textbf{\ipa{qʰæ˧kʰwɤ\#˥}}}}\hspace{0.5cm}[\kern2pt{\textcolor{darkblue}{\textbf{\ipa{qʰæ˧kʰwɤ˧}}}}\kern2pt]} \hypertarget{q\string_h\{\string_Mk\string_hw7\#\string_T1}{}
\markboth{\textcolor{darkblue}{\textbf{\ipa{qʰæ˧kʰwɤ\#˥}}}}{}
\textcolor{teal}{\mytextsc{noun}} \hspace{4pt} Tone: \#H.
\textcolor{Sepia}{\selectlanguage{english}Small dam (in canal; made of stones and earth).} \zh{小水坝,来堵塞田地里的小水渠。}  ¶ \textcolor{darkblue}{\textbf{\ipa{qʰæ˧kʰwɤ˧ ɖɯ˧-ɭɯ˧}}} \textcolor{Sepia}{\selectlanguage{english}a small dam} \zh{一个小水坝}  
 \zh{量词}: \textcolor{darkblue}{\textbf{\ipa{ɭɯ˧}}}  \mytextsc{clf}: \textcolor{darkblue}{\textbf{\ipa{ɭɯ˧}}} 
\lhead{\firstmark}
\rhead{\botmark}

\subsection{\hspace{-0.5cm} {\Large \textcolor{darkblue}{\textbf{\ipa{qʰæ˧lo˧˥}}}}\hspace{0.5cm}[\kern2pt{\textcolor{darkblue}{\textbf{\ipa{qʰæ˧lo˧˥}}}}\kern2pt]} \hypertarget{q\string_h\{\string_Mlo\string_M\string_T1}{}
\markboth{\textcolor{darkblue}{\textbf{\ipa{qʰæ˧lo˧˥}}}}{}
\textcolor{teal}{\mytextsc{noun}} \hspace{4pt} Tone: MH\#.
\textcolor{Sepia}{\selectlanguage{english}Small gulley, small trench.} \zh{小水渠。}  \zh{量词}: \textcolor{darkblue}{\textbf{\ipa{kʰɯ˩}}}  \mytextsc{clf}: \textcolor{darkblue}{\textbf{\ipa{kʰɯ˩}}} \textit{See:} \hyperlink{}{\textcolor{darkblue}{\textbf{\ipa{qʰæ˧zo\#˥}}}} 
\lhead{\firstmark}
\rhead{\botmark}

\subsection{\hspace{-0.5cm} {\Large \textcolor{darkblue}{\textbf{\ipa{qʰæ˧mi˧}}}}\hspace{0.5cm}[\kern2pt{\textcolor{darkblue}{\textbf{\ipa{qʰæ˧mi˧}}}}\kern2pt]} \hypertarget{q\string_h\{\string_Mmi\string_M1}{}
\markboth{\textcolor{darkblue}{\textbf{\ipa{qʰæ˧mi˧}}}}{}
\textcolor{teal}{\mytextsc{noun}} \hspace{4pt} Tone: M.
\textcolor{Sepia}{\selectlanguage{english}Large trench, canal.} \zh{大水渠。}  \zh{量词}: \textcolor{darkblue}{\textbf{\ipa{kʰɯ˩}}}  \mytextsc{clf}: \textcolor{darkblue}{\textbf{\ipa{kʰɯ˩}}} 
\lhead{\firstmark}
\rhead{\botmark}

\subsection{\hspace{-0.5cm} {\Large \textcolor{darkblue}{\textbf{\ipa{qʰæ˧mo˩}}}}\hspace{0.5cm}[\kern2pt{\textcolor{darkblue}{\textbf{\ipa{qʰæ˧mo˩}}}}\kern2pt]} \hypertarget{q\string_h\{\string_Mmo\string_B1}{}
\markboth{\textcolor{darkblue}{\textbf{\ipa{qʰæ˧mo˩}}}}{}
\textcolor{teal}{\mytextsc{noun}} \hspace{4pt} Tone: L\#.
\textcolor{Sepia}{\selectlanguage{english}A poisonous mushroom.} \zh{有毒的一种菌子。} 
\lhead{\firstmark}
\rhead{\botmark}

\subsection{\hspace{-0.5cm} {\Large \textcolor{darkblue}{\textbf{\ipa{qʰæ˧tv̩˧}}}}\hspace{0.5cm}[\kern2pt{\textcolor{darkblue}{\textbf{\ipa{qʰæ˧tv̩˧}}}}\kern2pt]} \hypertarget{q\string_h\{\string_Mtv\string_=\string_M1}{}
\markboth{\textcolor{darkblue}{\textbf{\ipa{qʰæ˧tv̩˧}}}}{}
\textcolor{teal}{\mytextsc{noun}} \hspace{4pt} Tone: M.
\textcolor{Sepia}{\selectlanguage{english}Anus.} \zh{肛门。}  \zh{量词}: \textcolor{darkblue}{\textbf{\ipa{ɭɯ˧}}}  \mytextsc{clf}: \textcolor{darkblue}{\textbf{\ipa{ɭɯ˧}}} 
\lhead{\firstmark}
\rhead{\botmark}

\subsection{\hspace{-0.5cm} {\Large \textcolor{darkblue}{\textbf{\ipa{qʰæ˧tɕʰi˧}}}}\hspace{0.5cm}[\kern2pt{\textcolor{darkblue}{\textbf{\ipa{qʰæ˧tɕʰi˧}}}}\kern2pt]} \hypertarget{q\string_h\{\string_Mts£\string_hi\string_M1}{}
\markboth{\textcolor{darkblue}{\textbf{\ipa{qʰæ˧tɕʰi˧}}}}{}
\textcolor{teal}{\mytextsc{noun}} \hspace{4pt} Tone: M.
\textcolor{Sepia}{\selectlanguage{english}A village of Yongning; Chinese name: Kaiji.} \zh{开基(永宁的一个村落)。}  ¶ \textcolor{darkblue}{\textbf{\ipa{ʈʂʰɯ˧ | qʰæ˧tɕʰi˧-hĩ˧ ɲi˥!}}} \textcolor{Sepia}{\selectlanguage{english}(S)he is from the village of Kaiji!} \zh{他是开基村人!}  
 ¶ \textcolor{darkblue}{\textbf{\ipa{dʑɤ˩bv̩˧kɤ˧-sɑ˥ʁwɤ˩, | hi˩ʁwɤ˩-lo˥, | æ˩mi˧-ʁwɤ\#˥, | lɑ˧lo˧-ʁwɤ˥, | lɑ˧ŋwɤ˧, | bɤ˧tsʰo˧gv̩˥, | ə˧lɑ˧-ʁwɤ\#˥, | gæ˧ɻæ˩, | qʰæ˧tɕʰi˧, | tʰo˧ʈɯ\#˥}}} \textcolor{Sepia}{\selectlanguage{english}the ten villages traditionally considered as part of Yongning} \zh{摩梭传统地理概念中,属于永宁的十个村落}  

\lhead{\firstmark}
\rhead{\botmark}

\subsection{\hspace{-0.5cm} {\Large \textcolor{darkblue}{\textbf{\ipa{qʰæ˧ʈæ˧˥}}}}\hspace{0.5cm}[\kern2pt{\textcolor{darkblue}{\textbf{\ipa{qʰæ˧ʈæ˧˥}}}}\kern2pt]} \hypertarget{q\string_h\{\string_Mt`\{\string_M\string_T1}{}
\markboth{\textcolor{darkblue}{\textbf{\ipa{qʰæ˧ʈæ˧˥}}}}{}
\textcolor{teal}{\mytextsc{adjective}} \hspace{4pt} Tone: MH\#.
\textit{From:} \textbf{qʰæ˩a 1} \textcolor{Sepia}{\selectlanguage{english}Quiet, at peace.} \zh{安静。}  ¶ \textcolor{darkblue}{\textbf{\ipa{qʰæ˧ʈæ˧˥ | tʰi˧-dzi˩}}} \textcolor{Sepia}{\selectlanguage{english}to sit quietly, free from toil and care} \zh{安静地坐着}  
 ¶ \textcolor{darkblue}{\textbf{\ipa{qʰæ˧ʈæ˧˥ | tʰi˧-ʝi˧}}} \textcolor{Sepia}{\selectlanguage{english}to work quietly} \zh{安静地工作}  

\lhead{\firstmark}
\rhead{\botmark}

\subsection{\hspace{-0.5cm} {\Large \textcolor{darkblue}{\textbf{\ipa{qʰæ˧zo\#˥}}}}\hspace{0.5cm}[\kern2pt{\textcolor{darkblue}{\textbf{\ipa{qʰæ˧zo˧}}}}\kern2pt]} \hypertarget{q\string_h\{\string_Mzo\#\string_T1}{}
\markboth{\textcolor{darkblue}{\textbf{\ipa{qʰæ˧zo\#˥}}}}{}
\textcolor{teal}{\mytextsc{noun}} \hspace{4pt} Tone: \#H.
\textcolor{Sepia}{\selectlanguage{english}Small trench/canal.} \zh{小水渠。}  \zh{量词}: \textcolor{darkblue}{\textbf{\ipa{kʰɯ˩}}}  \mytextsc{clf}: \textcolor{darkblue}{\textbf{\ipa{kʰɯ˩}}} \textit{See:} \hyperlink{}{\textcolor{darkblue}{\textbf{\ipa{qʰæ˧lo˧˥}}}} 
\lhead{\firstmark}
\rhead{\botmark}

\subsection{\hspace{-0.5cm} {\Large \textcolor{darkblue}{\textbf{\ipa{qʰæ˩}}}}\hspace{0.5cm}[\kern2pt{\textcolor{darkblue}{\textbf{\ipa{qʰæ˩˥}}}}\kern2pt]} \hypertarget{q\string_h\{\string_B1}{}
\markboth{\textcolor{darkblue}{\textbf{\ipa{qʰæ˩}}}}{}
\textcolor{teal}{\mytextsc{verb}} \hspace{4pt} Tone: L\textsubscript{a}.
\textcolor{Sepia}{\selectlanguage{english}To crack, to snap off.} \zh{折断。}  ¶ \textcolor{darkblue}{\textbf{\ipa{qʰɑ˧dze˧ qʰæ˩}}} \textcolor{Sepia}{\selectlanguage{english}to harvest sweet corn (literally: to snap off ears of sweet corn)} \zh{采玉米}  
 ¶ \textcolor{darkblue}{\textbf{\ipa{qʰɑ˧dze˧ | le˧-qʰæ˩-ze˩}}} \textcolor{Sepia}{\selectlanguage{english}The sweetcorn has been harvested.} \zh{玉米收好了。}  
 ¶ \textcolor{darkblue}{\textbf{\ipa{qʰɑ˧dze˧ | ɖɯ˧-qʰæ˧\textasciitilde{}qʰæ˥-ɻ̍˩}}} \textcolor{Sepia}{\selectlanguage{english}to harvest some sweet corn} \zh{去采些玉米}  

\lhead{\firstmark}
\rhead{\botmark}

\subsection{\hspace{-0.5cm} {\Large \textcolor{darkblue}{\textbf{\ipa{qʰæ˩\textsubscript{a}}}} \textsubscript{1}}\hspace{0.5cm}[\kern2pt{\textcolor{darkblue}{\textbf{\ipa{qʰæ˩˥}}}}\kern2pt]} \hypertarget{q\string_h\{\string_Ba1}{}
\markboth{\textcolor{darkblue}{\textbf{\ipa{qʰæ˩\textsubscript{a}}}} \textsubscript{1}}{}
\textcolor{teal}{\mytextsc{adjective}} \hspace{4pt} Tone: L\textsubscript{a}.
\textcolor{Sepia}{\selectlanguage{english}False, fake.} \zh{假。}  ¶ \textcolor{darkblue}{\textbf{\ipa{qʰæ˩-hĩ˩˥, | tʰɑ˧-ʐwɤ˩!}}} \textcolor{Sepia}{\selectlanguage{english}Do not tell lies! / Do not tell things that are false!} \zh{假话,不要说! =不要撒谎!}  
 ¶ \textcolor{darkblue}{\textbf{\ipa{qʰæ˧ ʐwɤ˧}}} \textcolor{Sepia}{\selectlanguage{english}to tell lies} \zh{撒谎、说谎}  

\lhead{\firstmark}
\rhead{\botmark}

\subsection{\hspace{-0.5cm} {\Large \textcolor{darkblue}{\textbf{\ipa{qʰæ˩\textsubscript{a}}}} \textsubscript{2}}\hspace{0.5cm}[\kern2pt{\textcolor{darkblue}{\textbf{\ipa{qʰæ˩˥}}}}\kern2pt]} \hypertarget{q\string_h\{\string_Ba2}{}
\markboth{\textcolor{darkblue}{\textbf{\ipa{qʰæ˩\textsubscript{a}}}} \textsubscript{2}}{}
\textcolor{teal}{\mytextsc{adjective}} \hspace{4pt} Tone: L\textsubscript{a}.
\ding{202} \textcolor{Sepia}{\selectlanguage{english}Well (to feel well); quiet.} \zh{平静、安静,安乐、(身体)健康。}  ¶ \textcolor{darkblue}{\textbf{\ipa{hĩ˧ | ə˩-qʰæ˩˥?}}} \textcolor{Sepia}{\selectlanguage{english}How are you?} \zh{你好吗? / 一切好吗?}  
 ¶ \textcolor{darkblue}{\textbf{\ipa{njɤ˧ | mɤ˧-qʰæ˩.}}} \textcolor{Sepia}{\selectlanguage{english}I don't feel well.} \zh{我不舒服。}  
\ding{203} \textcolor{Sepia}{\selectlanguage{english}Light, easy (work).} \zh{轻松。}  ¶ \textcolor{darkblue}{\textbf{\ipa{qʰæ˩-hĩ˩˥}}} \textcolor{Sepia}{\selectlanguage{english}\mytextsc{rel}} \zh{轻松的}  

\lhead{\firstmark}
\rhead{\botmark}

\subsection{\hspace{-0.5cm} {\Large \textcolor{darkblue}{\textbf{\ipa{qʰæ˩bæ˩}}}}\hspace{0.5cm}[\kern2pt{\textcolor{darkblue}{\textbf{\ipa{qʰæ˧bæ˧}}}}\kern2pt]} \hypertarget{q\string_h\{\string_Bb\{\string_B1}{}
\markboth{\textcolor{darkblue}{\textbf{\ipa{qʰæ˩bæ˩}}}}{}
\textcolor{teal}{\mytextsc{noun}} \hspace{4pt} Tone: L.
\textcolor{Sepia}{\selectlanguage{english}Spoon, used for salt, tsamba... It corresponds to European teaspoons and tablespoons.} \zh{调羹。}  \zh{量词}: \textcolor{darkblue}{\textbf{\ipa{nɑ˧}}}  \mytextsc{clf}: \textcolor{darkblue}{\textbf{\ipa{nɑ˧}}} 
\lhead{\firstmark}
\rhead{\botmark}

\subsection{\hspace{-0.5cm} {\Large \textcolor{darkblue}{\textbf{\ipa{qʰæ˩ʈv̩˩ɻæ˥}}}}\hspace{0.5cm}[\kern2pt{\textcolor{darkblue}{\textbf{\ipa{qʰæ˩ʈv̩˩ɻæ˥}}}}\kern2pt]} \hypertarget{q\string_h\{\string_Bt`v\string_=\string_Br£`\{\string_T1}{}
\markboth{\textcolor{darkblue}{\textbf{\ipa{qʰæ˩ʈv̩˩ɻæ˥}}}}{}
\textcolor{teal}{\mytextsc{adjective}} \hspace{4pt} Tone: L+H\#.
\textcolor{Sepia}{\selectlanguage{english}Quiet, peaceful.} \zh{安宁。}  ¶ \textcolor{darkblue}{\textbf{\ipa{qʰæ˩ʈv̩˩ɻæ˥ | ɖɯ˧-dzi˩-ɻ̍˩}}} \textcolor{Sepia}{\selectlanguage{english}to sit quietly for a while} \zh{安静地坐一会}  
 ¶ \textcolor{darkblue}{\textbf{\ipa{qʰæ˩ʈv̩˩ɻæ˥-gv̩˩}}} \textcolor{Sepia}{\selectlanguage{english}peacefully} \zh{安宁地}  

\lhead{\firstmark}
\rhead{\botmark}

\subsection{\hspace{-0.5cm} {\Large \textcolor{darkblue}{\textbf{\ipa{qʰæ˧˥}}} \textsubscript{1}}\hspace{0.5cm}[\kern2pt{\textcolor{darkblue}{\textbf{\ipa{qʰæ˧˥}}}}\kern2pt]} \hypertarget{q\string_h\{\string_M\string_T1}{}
\markboth{\textcolor{darkblue}{\textbf{\ipa{qʰæ˧˥}}} \textsubscript{1}}{}
\textcolor{teal}{\mytextsc{verb}} \hspace{4pt} Tone: MH.
\textcolor{Sepia}{\selectlanguage{english}To come out (moon, sun).} \zh{出来(月亮,太阳)。}  ¶ \textcolor{darkblue}{\textbf{\ipa{tʰi˧-qʰæ˧-ze˥}}} \textcolor{Sepia}{\selectlanguage{english}\mytextsc{dur} \string_ \mytextsc{pfv}} \zh{\mytextsc{dur} \string_ \mytextsc{pfv}}  

\lhead{\firstmark}
\rhead{\botmark}

\subsection{\hspace{-0.5cm} {\Large \textcolor{darkblue}{\textbf{\ipa{qʰæ˧˥}}} \textsubscript{2}}\hspace{0.5cm}[\kern2pt{\textcolor{darkblue}{\textbf{\ipa{qʰæ˧˥}}}}\kern2pt]} \hypertarget{q\string_h\{\string_M\string_T2}{}
\markboth{\textcolor{darkblue}{\textbf{\ipa{qʰæ˧˥}}} \textsubscript{2}}{}
\textcolor{teal}{\mytextsc{verb}} \hspace{4pt} Tone: MH.
\textcolor{Sepia}{\selectlanguage{english}To pull down, to dismantle.} \zh{拆。}  ¶ \textcolor{darkblue}{\textbf{\ipa{ʑi˧qʰwɤ˧ qʰæ˧˥}}} \textcolor{Sepia}{\selectlanguage{english}to demolish a house} \zh{拆房子}  

\lhead{\firstmark}
\rhead{\botmark}

\subsection{\hspace{-0.5cm} {\Large \textcolor{darkblue}{\textbf{\ipa{qʰæ˧˥}}} \textsubscript{3}}\hspace{0.5cm}[\kern2pt{\textcolor{darkblue}{\textbf{\ipa{qʰæ˧˥}}}}\kern2pt]} \hypertarget{q\string_h\{\string_M\string_T3}{}
\markboth{\textcolor{darkblue}{\textbf{\ipa{qʰæ˧˥}}} \textsubscript{3}}{}
\textcolor{teal}{\mytextsc{verb}} \hspace{4pt} Tone: MH.
\textcolor{Sepia}{\selectlanguage{english}To share: several people share something among themselves; someone shares out something.} \zh{分东西、(大家)平分东西。} 
\lhead{\firstmark}
\rhead{\botmark}

\subsection{\hspace{-0.5cm} {\Large \textcolor{darkblue}{\textbf{\ipa{qʰæ˧˥}}} \textsubscript{4}}\hspace{0.5cm}[\kern2pt{\textcolor{darkblue}{\textbf{\ipa{qʰæ˧˥}}}}\kern2pt]} \hypertarget{q\string_h\{\string_M\string_T4}{}
\markboth{\textcolor{darkblue}{\textbf{\ipa{qʰæ˧˥}}} \textsubscript{4}}{}
\textcolor{teal}{\mytextsc{verb}} \hspace{4pt} Tone: MH.
\textcolor{Sepia}{\selectlanguage{english}To shoot (with a gun).} \zh{开枪。}  ¶ \textcolor{darkblue}{\textbf{\ipa{le˧-qʰæ˧-ze˥}}} \textcolor{Sepia}{\selectlanguage{english}\mytextsc{accomp} \string_ \mytextsc{pfv}} \zh{开枪了}  
 ¶ \textcolor{darkblue}{\textbf{\ipa{mv̩˧ʐe˧ qʰæ˩(-ze˩)}}} \textcolor{Sepia}{\selectlanguage{english}to shoot with a gun} \zh{开枪}  

\lhead{\firstmark}
\rhead{\botmark}

\subsection{\hspace{-0.5cm} {\Large \textcolor{darkblue}{\textbf{\ipa{qʰæ˧˥}}} \textsubscript{5}}\hspace{0.5cm}[\kern2pt{\textcolor{darkblue}{\textbf{\ipa{qʰæ˧˥}}}}\kern2pt]} \hypertarget{q\string_h\{\string_M\string_T5}{}
\markboth{\textcolor{darkblue}{\textbf{\ipa{qʰæ˧˥}}} \textsubscript{5}}{}
\textcolor{teal}{\mytextsc{adjective}} \hspace{4pt} Tone: MH.
\textcolor{Sepia}{\selectlanguage{english}Happy, content, peaceful, at peace.} \zh{幸福,安逸,平安。}  ¶ \textcolor{darkblue}{\textbf{\ipa{le˧-qʰæ˧-ze˥}}} \textcolor{Sepia}{\selectlanguage{english}\mytextsc{accomp} \string_ \mytextsc{pfv}} \zh{\mytextsc{accomp} \string_ \mytextsc{pfv}}  
 ¶ \textcolor{darkblue}{\textbf{\ipa{lo˧ qʰæ˩}}} \textcolor{Sepia}{\selectlanguage{english}to work in a quiet, relaxed manner} \zh{轻松工作}  

\lhead{\firstmark}
\rhead{\botmark}

\subsection{\hspace{-0.5cm} {\Large \textcolor{darkblue}{\textbf{\ipa{qʰæ˧˥}}} \textsubscript{6}}\hspace{0.5cm}[\kern2pt{\textcolor{darkblue}{\textbf{\ipa{qʰæ˧˥}}}}\kern2pt]} \hypertarget{q\string_h\{\string_M\string_T6}{}
\markboth{\textcolor{darkblue}{\textbf{\ipa{qʰæ˧˥}}} \textsubscript{6}}{}
\textcolor{teal}{\mytextsc{verb}} \hspace{4pt} Tone: MH.
\textcolor{Sepia}{\selectlanguage{english}To burn, to go brown: food or oil gets close to burning point (but remains edible).} \zh{糊、变黑(高温让油、食物变黑,变糊了)。}  ¶ \textcolor{darkblue}{\textbf{\ipa{le˧-qʰæ˧-ze˥}}} \textcolor{Sepia}{\selectlanguage{english}\mytextsc{accomp} \string_ \mytextsc{pfv}} \zh{\mytextsc{accomp} \string_ \mytextsc{pfv}}  
 ¶ \textcolor{darkblue}{\textbf{\ipa{mɤ˧ | le˧-qʰæ˧-ze˥}}} \textcolor{Sepia}{\selectlanguage{english}The oil has burned / has reached boiling point / has gone black!} \zh{油焦了!}  
 ¶ \textcolor{darkblue}{\textbf{\ipa{hɑ˧ | le˧-qʰæ˧-ze˥}}} \textcolor{Sepia}{\selectlanguage{english}The rice has burned / is overcooked.} \zh{饭糊了。}  
 ¶ \textcolor{darkblue}{\textbf{\ipa{v̩˩tsʰɤ˩˥ | hṽ˧\textasciitilde{}hṽ˧ F | le˧-qʰæ˧-ze˥!}}} \textcolor{Sepia}{\selectlanguage{english}The vegetables are going brown / are overcooked / are getting burnt from frying!} \zh{菜都炒糊了!}  
 ¶ \textcolor{darkblue}{\textbf{\ipa{ʂe˧ | hṽ˧\textasciitilde{}hṽ˧ F | le˧-qʰæ˧-ze˥!}}} \textcolor{Sepia}{\selectlanguage{english}The meat is going brown / is overcooked / is getting burnt from frying!} \zh{肉都炒焦了!}  

\lhead{\firstmark}
\rhead{\botmark}

\subsection{\hspace{-0.5cm} {\Large \textcolor{darkblue}{\textbf{\ipa{qʰo˧\textsubscript{a}}}}}\hspace{0.5cm}[\kern2pt{\textcolor{darkblue}{\textbf{\ipa{qʰo˥}}}}\kern2pt]} \hypertarget{q\string_ho\string_Ma1}{}
\markboth{\textcolor{darkblue}{\textbf{\ipa{qʰo˧\textsubscript{a}}}}}{}
\textcolor{teal}{\mytextsc{verb}} \hspace{4pt} Tone: M\textsubscript{a}.
\textcolor{Sepia}{\selectlanguage{english}To pile up (e.g. stones).} \zh{堆起来。}  ¶ \textcolor{darkblue}{\textbf{\ipa{lv̩˧mi˧ tʰi˧-qʰo˧}}} \textcolor{Sepia}{\selectlanguage{english}to pile up stones} \zh{石头堆起来}  

\lhead{\firstmark}
\rhead{\botmark}

\subsection{\hspace{-0.5cm} {\Large \textcolor{darkblue}{\textbf{\ipa{qʰo˧lo˧}}}}\hspace{0.5cm}[\kern2pt{\textcolor{darkblue}{\textbf{\ipa{qʰo˧lo˧}}}}\kern2pt]} \hypertarget{q\string_ho\string_Mlo\string_M1}{}
\markboth{\textcolor{darkblue}{\textbf{\ipa{qʰo˧lo˧}}}}{}
\textcolor{teal}{\mytextsc{noun}} \hspace{4pt} Tone: M.
\textcolor{Sepia}{\selectlanguage{english}Wheel.} \zh{轮子。}  \zh{量词}: \textcolor{darkblue}{\textbf{\ipa{ɭɯ˧}}}  \mytextsc{clf}: \textcolor{darkblue}{\textbf{\ipa{ɭɯ˧}}} 
\lhead{\firstmark}
\rhead{\botmark}

\subsection{\hspace{-0.5cm} {\Large \textcolor{darkblue}{\textbf{\ipa{qʰo˧mo˥}}}}\hspace{0.5cm}[\kern2pt{\textcolor{darkblue}{\textbf{\ipa{qʰo˧mo˥}}}}\kern2pt]} \hypertarget{q\string_ho\string_Mmo\string_T1}{}
\markboth{\textcolor{darkblue}{\textbf{\ipa{qʰo˧mo˥}}}}{}
\textcolor{teal}{\mytextsc{noun}} \hspace{4pt} Tone: H\#.
\textcolor{Sepia}{\selectlanguage{english}Old cow (which does not give milk anymore).} \zh{老牛(不产奶了)。}  \zh{量词}: \textcolor{darkblue}{\textbf{\ipa{pʰo˧˥}}}  \mytextsc{clf}: \textcolor{darkblue}{\textbf{\ipa{pʰo˧˥}}} 
\lhead{\firstmark}
\rhead{\botmark}

\subsection{\hspace{-0.5cm} {\Large \textcolor{darkblue}{\textbf{\ipa{qʰo˩\textsubscript{b}}}}}\hspace{0.5cm}[\kern2pt{\textcolor{darkblue}{\textbf{\ipa{qʰo˩˥}}}}\kern2pt]} \hypertarget{q\string_ho\string_Bb1}{}
\markboth{\textcolor{darkblue}{\textbf{\ipa{qʰo˩\textsubscript{b}}}}}{}
\textcolor{teal}{\mytextsc{verb}} \hspace{4pt} Tone: L\textsubscript{b}.
\textcolor{Sepia}{\selectlanguage{english}To invite, to treat.} \zh{邀请、请。}  ¶ \textcolor{darkblue}{\textbf{\ipa{hĩ˧ qʰo˧˥}}} \textcolor{Sepia}{\selectlanguage{english}to invite someone} \zh{邀请人}  
 ¶ \textcolor{darkblue}{\textbf{\ipa{hĩ˧bæ˧ qʰo˧˥}}} \textcolor{Sepia}{\selectlanguage{english}to invite a guest} \zh{邀请客人}  
 ¶ \textcolor{darkblue}{\textbf{\ipa{hĩ˧bæ˧ | qʰo˧-zo˧-ho˥}}} \textcolor{Sepia}{\selectlanguage{english}We should invite guests!} \zh{需要请一下客人!}  
 ¶ \textcolor{darkblue}{\textbf{\ipa{hĩ˧bæ˧ qʰo˧-di˧˥}}} \textcolor{Sepia}{\selectlanguage{english}Euphemism for 'rat poison'. This phrase is intended not to attract the mice's attention to these preparations.} \zh{‘待客的东西’(老鼠药的委婉语。如果说出来要买老鼠药,老鼠会知道,就不会吃的。)}  
 ¶ \textcolor{darkblue}{\textbf{\ipa{ɖɯ˧-qʰo˥\textasciitilde{}qʰo˩-ɻ̍˩}}} \textcolor{Sepia}{\selectlanguage{english}\mytextsc{delimitative} \mytextsc{red} \mytextsc{inceptive}} \zh{\mytextsc{delimitative} \mytextsc{red} \mytextsc{inceptive:请一下}}  
 ¶ \textcolor{darkblue}{\textbf{\ipa{qʰo˩-mɤ˥-qʰo˩}}} \textcolor{Sepia}{\selectlanguage{english}to invite or not} \zh{请不请}  
 ¶ \textcolor{darkblue}{\textbf{\ipa{qʰo˩-mɤ˩-ho˥}}} \textcolor{Sepia}{\selectlanguage{english}...will not invite} \zh{不请了 / 不要请了}  

\lhead{\firstmark}
\rhead{\botmark}

\subsection{\hspace{-0.5cm} {\Large \textcolor{darkblue}{\textbf{\ipa{qʰo˩dv̩˧˥}}}}\hspace{0.5cm}[\kern2pt{\textcolor{darkblue}{\textbf{\ipa{qʰo˩dv̩˧˥}}}}\kern2pt]} \hypertarget{q\string_ho\string_Bdv\string_=\string_M\string_T1}{}
\markboth{\textcolor{darkblue}{\textbf{\ipa{qʰo˩dv̩˧˥}}}}{}
\textcolor{teal}{\mytextsc{noun}} \hspace{4pt} Tone: LM+MH\#.
\textcolor{Sepia}{\selectlanguage{english}Hammer; typically a large wood hammer.} \zh{大锤子。}  ¶ \textcolor{darkblue}{\textbf{\ipa{ʂe˩-qʰo˩dv̩˧˥}}} \textcolor{Sepia}{\selectlanguage{english}iron hammer} \zh{铁锤子}  
 \zh{量词}: \textcolor{darkblue}{\textbf{\ipa{ɭɯ˧}}}  \mytextsc{clf}: \textcolor{darkblue}{\textbf{\ipa{ɭɯ˧}}} 
\lhead{\firstmark}
\rhead{\botmark}

\subsection{\hspace{-0.5cm} {\Large \textcolor{darkblue}{\textbf{\ipa{qʰo˩mv̩˩}}}}\hspace{0.5cm}[\kern2pt{\textcolor{darkblue}{\textbf{\ipa{qʰo˩mv̩˩˥}}}}\kern2pt]} \hypertarget{q\string_ho\string_Bmv\string_=\string_B1}{}
\markboth{\textcolor{darkblue}{\textbf{\ipa{qʰo˩mv̩˩}}}}{}
\textcolor{teal}{\mytextsc{noun}} \hspace{4pt} Tone: L.
\textcolor{Sepia}{\selectlanguage{english}Straw hat.} \zh{斗笠。}  \zh{量词}: \textcolor{darkblue}{\textbf{\ipa{ɭɯ˧}}}  \mytextsc{clf}: \textcolor{darkblue}{\textbf{\ipa{ɭɯ˧}}} 
\lhead{\firstmark}
\rhead{\botmark}

\subsection{\hspace{-0.5cm} {\Large \textcolor{darkblue}{\textbf{\ipa{qʰo˩tv̩˧˥}}}}\hspace{0.5cm}[\kern2pt{\textcolor{darkblue}{\textbf{\ipa{qʰo˩tv̩˧˥}}}}\kern2pt]} \hypertarget{q\string_ho\string_Btv\string_=\string_M\string_T1}{}
\markboth{\textcolor{darkblue}{\textbf{\ipa{qʰo˩tv̩˧˥}}}}{}
\textcolor{teal}{\mytextsc{noun}} \hspace{4pt} Tone: LM+MH\#.
\textcolor{Sepia}{\selectlanguage{english}Tree stump.} \zh{树墩、树桩。}  \zh{量词}: \textcolor{darkblue}{\textbf{\ipa{ɭɯ˧}}}  \mytextsc{clf}: \textcolor{darkblue}{\textbf{\ipa{ɭɯ˧}}} 
\lhead{\firstmark}
\rhead{\botmark}

\subsection{\hspace{-0.5cm} {\Large \textcolor{darkblue}{\textbf{\ipa{qʰo˧˥}}} \textsubscript{1}}\hspace{0.5cm}[\kern2pt{\textcolor{darkblue}{\textbf{\ipa{qʰo˧˥}}}}\kern2pt]} \hypertarget{q\string_ho\string_M\string_T1}{}
\markboth{\textcolor{darkblue}{\textbf{\ipa{qʰo˧˥}}} \textsubscript{1}}{}
\textcolor{teal}{\mytextsc{verb}} \hspace{4pt} Tone: MH.
\textcolor{Sepia}{\selectlanguage{english}To peck.} \zh{啄。}  ¶ \textcolor{darkblue}{\textbf{\ipa{hɑ˧ qʰo˩(-ze˩)}}} \textcolor{Sepia}{\selectlanguage{english}to peck cereals} \zh{啄粮食}  
 ¶ \textcolor{darkblue}{\textbf{\ipa{hɑ˧ qʰo˥\textasciitilde{}qʰo˩ (-dʑo˩)}}} \textcolor{Sepia}{\selectlanguage{english}to peck cereals} \zh{啄粮食}  
 ¶ \textcolor{darkblue}{\textbf{\ipa{æ˩-ɳɯ˥ | hɑ˧ qʰo˩}}} \textcolor{Sepia}{\selectlanguage{english}the chicken is pecking cereals} \zh{鸡在啄粮食}  

\lhead{\firstmark}
\rhead{\botmark}

\subsection{\hspace{-0.5cm} {\Large \textcolor{darkblue}{\textbf{\ipa{qʰo˧˥}}} \textsubscript{2}}\hspace{0.5cm}[\kern2pt{\textcolor{darkblue}{\textbf{\ipa{qʰo˧˥}}}}\kern2pt]} \hypertarget{q\string_ho\string_M\string_T2}{}
\markboth{\textcolor{darkblue}{\textbf{\ipa{qʰo˧˥}}} \textsubscript{2}}{}
\textcolor{teal}{\mytextsc{verb}} \hspace{4pt} Tone: MH.
\textcolor{Sepia}{\selectlanguage{english}To kill; to slaughter (an animal).} \zh{杀,宰牲畜。}  ¶ \textcolor{darkblue}{\textbf{\ipa{bo˩ qʰo˧˥ / bo˩ qʰo˧-ze˥}}} \textcolor{Sepia}{\selectlanguage{english}to slaughter a pig} \zh{杀猪}  
 ¶ \textcolor{darkblue}{\textbf{\ipa{bo˩˥ | le˧-qʰo˧-ze˥}}} \textcolor{Sepia}{\selectlanguage{english}the pig has been slaughtered} \zh{杀了猪}  
 ¶ \textcolor{darkblue}{\textbf{\ipa{æ˩ qʰo˧˥}}} \textcolor{Sepia}{\selectlanguage{english}to kill a chicken} \zh{杀鸡}  
 ¶ \textcolor{darkblue}{\textbf{\ipa{ʝi˧ qʰo˩}}} \textcolor{Sepia}{\selectlanguage{english}to kill a cow} \zh{杀牛}  

\lhead{\firstmark}
\rhead{\botmark}

\subsection{\hspace{-0.5cm} {\Large \textcolor{darkblue}{\textbf{\ipa{qʰv̩˧}}} \textsubscript{1}}\hspace{0.5cm}[\kern2pt{\textcolor{darkblue}{\textbf{\ipa{qʰv̩˥}}}}\kern2pt]} \hypertarget{q\string_hv\string_=\string_M1}{}
\markboth{\textcolor{darkblue}{\textbf{\ipa{qʰv̩˧}}} \textsubscript{1}}{}
\textcolor{teal}{\mytextsc{noun}} \hspace{4pt} Tone: M.
\ding{202} \textcolor{Sepia}{\selectlanguage{english}Hole.} \zh{洞。}  \zh{量词}: \textcolor{darkblue}{\textbf{\ipa{ɭɯ˧}}} \ding{203} \textcolor{Sepia}{\selectlanguage{english}Burrow.} \zh{野兽的洞穴、野兽的窝。}  ¶ \textcolor{darkblue}{\textbf{\ipa{ɖɤ˧-qʰv̩˧}}} \textcolor{Sepia}{\selectlanguage{english}fox burrow} \zh{狐狸的窝} \zh{tone: M}  
 ¶ \textcolor{darkblue}{\textbf{\ipa{ʂwæ˧ qʰv̩˧}}} \textcolor{Sepia}{\selectlanguage{english}otter's burrow} \zh{水獭的窝} \zh{tone: M}  
 \mytextsc{clf}: \textcolor{darkblue}{\textbf{\ipa{ɭɯ˧}}} 
\lhead{\firstmark}
\rhead{\botmark}

\subsection{\hspace{-0.5cm} {\Large \textcolor{darkblue}{\textbf{\ipa{qʰv̩˧}}} \textsubscript{2}}\hspace{0.5cm}[\kern2pt{\textcolor{darkblue}{\textbf{\ipa{qʰv̩˥}}}}\kern2pt]} \hypertarget{q\string_hv\string_=\string_M2}{}
\markboth{\textcolor{darkblue}{\textbf{\ipa{qʰv̩˧}}} \textsubscript{2}}{}
\textcolor{teal}{\mytextsc{noun}} \hspace{4pt} Tone: M.
\textcolor{Sepia}{\selectlanguage{english}Horn.} \zh{犄角(锯下来的)。}  ¶ \textcolor{darkblue}{\textbf{\ipa{ʝi˧-qʰv̩\#˥}}} \textcolor{Sepia}{\selectlanguage{english}Ox horn. Ox horns are used as containers for drinking.} \zh{牛角(过去,用牛角来当饮料容器)}  
 ¶ \textcolor{darkblue}{\textbf{\ipa{ʈʂʰæ˧-qʰv̩˥}}} \textcolor{Sepia}{\selectlanguage{english}stag horn} \zh{鹿角}  
 \zh{量词}: \textcolor{darkblue}{\textbf{\ipa{ɭɯ˧}}} \textcolor{darkblue}{\textbf{\ipa{dze˩}}}  \mytextsc{clf}: \textcolor{darkblue}{\textbf{\ipa{ɭɯ˧}}} \textcolor{darkblue}{\textbf{\ipa{dze˩}}} 
\lhead{\firstmark}
\rhead{\botmark}

\subsection{\hspace{-0.5cm} {\Large \textcolor{darkblue}{\textbf{\ipa{qʰv̩˧˥}}} \textsubscript{1}}\hspace{0.5cm}[\kern2pt{\textcolor{darkblue}{\textbf{\ipa{qʰv̩˧˥}}}}\kern2pt]} \hypertarget{q\string_hv\string_=\string_M\string_T1}{}
\markboth{\textcolor{darkblue}{\textbf{\ipa{qʰv̩˧˥}}} \textsubscript{1}}{}
\textcolor{teal}{\mytextsc{verb}} \hspace{4pt} Tone: MH.
\textcolor{Sepia}{\selectlanguage{english}To huddle up, to curl up.} \zh{蜷缩。}  ¶ \textcolor{darkblue}{\textbf{\ipa{ɲi˧-qʰv̩˧˥ | tʰi˧-dzi˩}}} \textcolor{Sepia}{\selectlanguage{english}to be seated, leaning forward, torso bent towards the thighs} \zh{坐着身体缩成一团}  
 ¶ \textcolor{darkblue}{\textbf{\ipa{[M23] ɲi˧-qʰv̩˧-ʝi˥ | tʰi˧-dzi˩}}} \textcolor{Sepia}{\selectlanguage{english}to be seated, leaning forward, torso bent towards the thighs} \zh{坐着身体缩成一团}  

\lhead{\firstmark}
\rhead{\botmark}

\subsection{\hspace{-0.5cm} {\Large \textcolor{darkblue}{\textbf{\ipa{qʰv̩˧˥}}} \textsubscript{2}}\hspace{0.5cm}[\kern2pt{\textcolor{darkblue}{\textbf{\ipa{qʰv̩˧˥}}}}\kern2pt]} \hypertarget{q\string_hv\string_=\string_M\string_T2}{}
\markboth{\textcolor{darkblue}{\textbf{\ipa{qʰv̩˧˥}}} \textsubscript{2}}{}
\textcolor{teal}{\mytextsc{number}} \hspace{4pt} Tone: MH.
\textcolor{Sepia}{\selectlanguage{english}6.} \zh{6。} 
\lhead{\firstmark}
\rhead{\botmark}

\subsection{\hspace{-0.5cm} {\Large \textcolor{darkblue}{\textbf{\ipa{qʰv̩˥}}}}\hspace{0.5cm}[\kern2pt{\textcolor{darkblue}{\textbf{\ipa{qʰv̩˥}}}}\kern2pt]} \hypertarget{q\string_hv\string_=\string_T1}{}
\markboth{\textcolor{darkblue}{\textbf{\ipa{qʰv̩˥}}}}{}
\textcolor{teal}{\mytextsc{noun}} \hspace{4pt} Tone: \#H.
\textcolor{Sepia}{\selectlanguage{english}Noise, sound.} \zh{声音。}  ¶ \textcolor{darkblue}{\textbf{\ipa{ʈʂʰɯ˧ | ə˧tso˧ qʰv̩˧ ɲi˥?}}} \textcolor{Sepia}{\selectlanguage{english}What is this sound?} \zh{这是什么声音?}  
 \zh{量词}: \textcolor{darkblue}{\textbf{\ipa{kʰwɤ˥}}}  \mytextsc{clf}: \textcolor{darkblue}{\textbf{\ipa{kʰwɤ˥}}} 
\lhead{\firstmark}
\rhead{\botmark}

\subsection{\hspace{-0.5cm} {\Large \textcolor{darkblue}{\textbf{\ipa{qʰv̩˥\textsubscript{a}}}}}\hspace{0.5cm}[\kern2pt{\textcolor{darkblue}{\textbf{\ipa{qʰv̩˥}}}}\kern2pt]} \hypertarget{q\string_hv\string_=\string_Ta1}{}
\markboth{\textcolor{darkblue}{\textbf{\ipa{qʰv̩˥\textsubscript{a}}}}}{}
\textcolor{teal}{\mytextsc{classifier}} \hspace{4pt} Tone: H\textsubscript{a}.
\textcolor{Sepia}{\selectlanguage{english}Classifier for hamlets / small villages.} \zh{量词:村落。}  ¶ \textcolor{darkblue}{\textbf{\ipa{ŋwɤ˧-qʰv̩˧, | tsʰe˧ɲi˧-ʑi˩}}} \textcolor{Sepia}{\selectlanguage{english}Five hamlets, twelve families! (This formula summarizes the statistics of the village of /ə˧lɑ˧-ʁwɤ\#˥/)} \zh{五个村落,十二个家庭!(描写阿拉瓦村的情况)}  

\lhead{\firstmark}
\rhead{\botmark}

\subsection{\hspace{-0.5cm} {\Large \textcolor{darkblue}{\textbf{\ipa{qʰv̩˩ɖɯ˩}}}}\hspace{0.5cm}[\kern2pt{\textcolor{darkblue}{\textbf{\ipa{qʰv̩˩ɖɯ˩˥}}}}\kern2pt]} \hypertarget{q\string_hv\string_=\string_Bd`M\string_B1}{}
\markboth{\textcolor{darkblue}{\textbf{\ipa{qʰv̩˩ɖɯ˩}}}}{}
\textcolor{teal}{\mytextsc{noun}} \hspace{4pt} Tone: L.
\textcolor{Sepia}{\selectlanguage{english}Attachment (to someone): found in the phrase 'to be attached to someone, to care for someone'.} \zh{关心。}  ¶ \textcolor{darkblue}{\textbf{\ipa{qʰv̩˩ɖɯ˩ pʰv̩˥}}} \textcolor{Sepia}{\selectlanguage{english}to care for someone, to respect, to feel attachment to someone} \zh{关心(一个人),重视(如:孩子重视父母)}  

\lhead{\firstmark}
\rhead{\botmark}

\subsection{\hspace{-0.5cm} {\Large \textcolor{darkblue}{\textbf{\ipa{qʰv̩˩ɖʐæ˩}}}}\hspace{0.5cm}[\kern2pt{\textcolor{darkblue}{\textbf{\ipa{qʰv̩˩ɖʐæ˩˥}}}}\kern2pt]} \hypertarget{q\string_hv\string_=\string_Bd`z`\{\string_B1}{}
\markboth{\textcolor{darkblue}{\textbf{\ipa{qʰv̩˩ɖʐæ˩}}}}{}
\textcolor{teal}{\mytextsc{noun}} \hspace{4pt} Tone: L.
\textcolor{Sepia}{\selectlanguage{english}String; small rope.} \zh{小绳子,细的绳子。}  ¶ \textcolor{darkblue}{\textbf{\ipa{qʰv̩˩ɖʐæ˩ ʈʂʰɯ˩-kʰɯ˥}}} \textcolor{Sepia}{\selectlanguage{english}\mytextsc{n}+\mytextsc{dem}+\mytextsc{clf}} \zh{一条细的绳子}  
 \zh{量词}: \textcolor{darkblue}{\textbf{\ipa{kʰɯ˩}}}  \mytextsc{clf}: \textcolor{darkblue}{\textbf{\ipa{kʰɯ˩}}} 
\lhead{\firstmark}
\rhead{\botmark}

\subsection{\hspace{-0.5cm} {\Large \textcolor{darkblue}{\textbf{\ipa{qʰv̩˧dʑɯ˥\$}}}}\hspace{0.5cm}[\kern2pt{\textcolor{darkblue}{\textbf{\ipa{qʰv̩˧dʑɯ˥}}}}\kern2pt]} \hypertarget{q\string_hv\string_=\string_Mdz£M\string_T\$1}{}
\markboth{\textcolor{darkblue}{\textbf{\ipa{qʰv̩˧dʑɯ˥\$}}}}{}
\textcolor{teal}{\mytextsc{noun}} \hspace{4pt} Tone: H\$.
\textcolor{Sepia}{\selectlanguage{english}Hole, cavity (e.g. mouse hole, or trap to catch large animals).} \zh{窟窿。}  ¶ \textcolor{darkblue}{\textbf{\ipa{hwæ˧tsɯ˥-qʰv̩˩dʑi˩}}} \textcolor{Sepia}{\selectlanguage{english}mousehole} \zh{耗子洞}  
 ¶ \textcolor{darkblue}{\textbf{\ipa{qʰv̩˧dʑɯ˧ tsʰi˧ (-ze˩)}}} \textcolor{Sepia}{\selectlanguage{english}to bore a hole} \zh{挖一个洞}  
 \zh{量词}: \textcolor{darkblue}{\textbf{\ipa{ɭɯ˧}}}  \mytextsc{clf}: \textcolor{darkblue}{\textbf{\ipa{ɭɯ˧}}} 
\lhead{\firstmark}
\rhead{\botmark}

\subsection{\hspace{-0.5cm} {\Large \textcolor{darkblue}{\textbf{\ipa{qʰv̩˧ɬi˧mi\#˥}}}}\hspace{0.5cm}[\kern2pt{\textcolor{darkblue}{\textbf{\ipa{qʰv̩˧ɬi˧mi˧}}}}\kern2pt]} \hypertarget{q\string_hv\string_=\string_MKi\string_Mmi\#\string_T1}{}
\markboth{\textcolor{darkblue}{\textbf{\ipa{qʰv̩˧ɬi˧mi\#˥}}}}{}
\textcolor{teal}{\mytextsc{noun}} \hspace{4pt} Tone: \#H.
\textcolor{Sepia}{\selectlanguage{english}6th month.} \zh{六月。} 
\lhead{\firstmark}
\rhead{\botmark}

\subsection{\hspace{-0.5cm} {\Large \textcolor{darkblue}{\textbf{\ipa{qʰv̩˩tsʰi˧˥}}}}\hspace{0.5cm}[\kern2pt{\textcolor{darkblue}{\textbf{\ipa{qʰv̩˩tsʰi˧˥}}}}\kern2pt]} \hypertarget{q\string_hv\string_=\string_Bts\string_hi\string_M\string_T1}{}
\markboth{\textcolor{darkblue}{\textbf{\ipa{qʰv̩˩tsʰi˧˥}}}}{}
\textcolor{teal}{\mytextsc{number}} \hspace{4pt} Tone: LM+MH\#.
\textcolor{Sepia}{\selectlanguage{english}60.} \zh{60。} 
\lhead{\firstmark}
\rhead{\botmark}

\subsection{\hspace{-0.5cm} {\Large \textcolor{darkblue}{\textbf{\ipa{qʰv̩˧tʰv̩\#˥}}}}\hspace{0.5cm}[\kern2pt{\textcolor{darkblue}{\textbf{\ipa{qʰv̩˧tʰv̩˧}}}}\kern2pt]} \hypertarget{q\string_hv\string_=\string_Mt\string_hv\string_=\#\string_T1}{}
\markboth{\textcolor{darkblue}{\textbf{\ipa{qʰv̩˧tʰv̩\#˥}}}}{}
\textcolor{teal}{\mytextsc{classifier}} \hspace{4pt} Tone: \#H.
\textcolor{Sepia}{\selectlanguage{english}Classifier: a hornful. The quantity of liquid (or powder) that can be contained in an ox's horn. Ox horns used to serve as containers for water.} \zh{量词:一个牛角的容量。}  ¶ \textcolor{darkblue}{\textbf{\ipa{ɖɯ˧-qʰv̩˧tʰv̩\#˥, | ɲi˧-qʰv̩˧tʰv̩\#˥, | so˩-qʰv̩˩tʰv̩˩˥, | ʐv̩˧-qʰv̩˧tʰv̩\#˥, | ŋwɤ˧-qʰv̩˧tʰv̩\#˥, | qʰv̩˧-qʰv̩˧tʰv̩\#˥, | ʂɯ˧-qʰv̩˧tʰv̩\#˥, | hõ˧-qʰv̩˧tʰv̩\#˥, | gv̩˧-qʰv̩˧tʰv̩\#˥, | tsʰe˩-qʰv̩˩tʰv̩˩˥}}} \textcolor{Sepia}{\selectlanguage{english}association with numerals from 1 to 10} \zh{与数词结合,一至十}  
\textit{See:} \hyperlink{}{\textcolor{darkblue}{\textbf{\ipa{qʰv̩˧tʰv˥\$}}}} 
\lhead{\firstmark}
\rhead{\botmark}

\subsection{\hspace{-0.5cm} {\Large \textcolor{darkblue}{\textbf{\ipa{qʰv̩˧tʰv˥\$}}}}\hspace{0.5cm}[\kern2pt{\textcolor{darkblue}{\textbf{\ipa{qʰv̩˧tʰv˥}}}}\kern2pt]} \hypertarget{q\string_hv\string_=\string_Mt\string_hv\string_T\$1}{}
\markboth{\textcolor{darkblue}{\textbf{\ipa{qʰv̩˧tʰv˥\$}}}}{}
\textcolor{teal}{\mytextsc{noun}} \hspace{4pt} Tone: H\$.
\textcolor{Sepia}{\selectlanguage{english}Horn.} \zh{(牛)角。}  ¶ \textcolor{darkblue}{\textbf{\ipa{qʰv˧tʰv˥ | ɖɯ˧-ɭɯ˧}}} \textcolor{Sepia}{\selectlanguage{english}a horn} \zh{一个角}  
 ¶ \textcolor{darkblue}{\textbf{\ipa{qʰv˧tʰv˧ ɲi˥}}} \textcolor{Sepia}{\selectlanguage{english}It's a horn.} \zh{是(牛)角。}  
 \zh{量词}: \textcolor{darkblue}{\textbf{\ipa{ɭɯ˧}}}  \mytextsc{clf}: \textcolor{darkblue}{\textbf{\ipa{ɭɯ˧}}} \textit{See:} \hyperlink{}{\textcolor{darkblue}{\textbf{\ipa{qʰv̩˧tʰv̩\#˥}}}} 
\lhead{\firstmark}
\rhead{\botmark}

\subsection{\hspace{-0.5cm} {\Large \textcolor{darkblue}{\textbf{\ipa{qʰv̩˩\textasciitilde{}qʰv̩˧˥}}}}\hspace{0.5cm}[\kern2pt{\textcolor{darkblue}{\textbf{\ipa{qʰv̩˧qʰv̩˧˥}}}}\kern2pt]} \hypertarget{q\string_hv\string_=\string_B~q\string_hv\string_=\string_M\string_T1}{}
\markboth{\textcolor{darkblue}{\textbf{\ipa{qʰv̩˩\textasciitilde{}qʰv̩˧˥}}}}{}
\textcolor{teal}{\mytextsc{verb}} \hspace{4pt} Tone: MH.
\textcolor{Sepia}{\selectlanguage{english}To fold (clothes).} \zh{折叠、裹起来。}  ¶ \textcolor{darkblue}{\textbf{\ipa{qʰv̩˩\textasciitilde{}qʰv̩˧-ze˥}}} \textcolor{Sepia}{\selectlanguage{english}\mytextsc{pfv}} \zh{折起来了}  
 ¶ \textcolor{darkblue}{\textbf{\ipa{le˧-qʰv̩˩\textasciitilde{}qʰv̩˩}}} \textcolor{Sepia}{\selectlanguage{english}\mytextsc{accomp}} \zh{\mytextsc{accomp}}  

\lhead{\firstmark}
\rhead{\botmark}

\subsection{\hspace{-0.5cm} {\Large \textcolor{darkblue}{\textbf{\ipa{qʰwæ˧}}}}\hspace{0.5cm}[\kern2pt{\textcolor{darkblue}{\textbf{\ipa{qʰwæ˥}}}}\kern2pt]} \hypertarget{q\string_hw\{\string_M1}{}
\markboth{\textcolor{darkblue}{\textbf{\ipa{qʰwæ˧}}}}{}
\textcolor{teal}{\mytextsc{noun}} \hspace{4pt} Tone: M.
\textcolor{Sepia}{\selectlanguage{english}Message (monosyllable).} \zh{信息,信。}  ¶ \textcolor{darkblue}{\textbf{\ipa{qʰwæ˧ po˧˥}}} \textcolor{Sepia}{\selectlanguage{english}to carry a letter; to convey a message} \zh{带信息、传信息,传一封信}  
 ¶ \textcolor{darkblue}{\textbf{\ipa{qʰwæ˧ kʰwɤ˧˥}}} \textcolor{Sepia}{\selectlanguage{english}to be in touch (with someone)} \zh{互相通信息、有联系(两个人互相通信息)}  
 ¶ \textcolor{darkblue}{\textbf{\ipa{dɑ˧pɤ˧-qʰwæ\#˥}}} \textcolor{Sepia}{\selectlanguage{english}the tales of the \textcolor{darkblue}{\textbf{\ipa{/dɑ˧pɤ˧/}}} priests} \zh{达巴的故事}  
 \zh{量词}: \textcolor{darkblue}{\textbf{\ipa{kʰwɤ˥}}}  \mytextsc{clf}: \textcolor{darkblue}{\textbf{\ipa{kʰwɤ˥}}} 
\lhead{\firstmark}
\rhead{\botmark}

\subsection{\hspace{-0.5cm} {\Large \textcolor{darkblue}{\textbf{\ipa{qʰwæ˧kʰwɤ\#˥}}}}\hspace{0.5cm}[\kern2pt{\textcolor{darkblue}{\textbf{\ipa{qʰwæ˩kʰwɤ˩˥}}}}\kern2pt]} \hypertarget{q\string_hw\{\string_Mk\string_hw7\#\string_T1}{}
\markboth{\textcolor{darkblue}{\textbf{\ipa{qʰwæ˧kʰwɤ\#˥}}}}{}
\textcolor{teal}{\mytextsc{noun}} \hspace{4pt} Tone: \#H.
\textcolor{Sepia}{\selectlanguage{english}Gossip, idle chatter.} \zh{闲话、流言、蜚语、闲言碎语、八卦。}  ¶ \textcolor{darkblue}{\textbf{\ipa{ɖɯ˧-zɯ˧ qʰwæ˧kʰwɤ˧}}} \textcolor{Sepia}{\selectlanguage{english}to tell a piece of gossip} \zh{讲一点八卦}  
 \zh{量词}: \textcolor{darkblue}{\textbf{\ipa{kʰwɤ˥}}}  \mytextsc{clf}: \textcolor{darkblue}{\textbf{\ipa{kʰwɤ˥}}} 
\lhead{\firstmark}
\rhead{\botmark}

\subsection{\hspace{-0.5cm} {\Large \textcolor{darkblue}{\textbf{\ipa{qʰwæ˧ɭɯ˧}}}}\hspace{0.5cm}[\kern2pt{\textcolor{darkblue}{\textbf{\ipa{qʰwæ˧ɭɯ˧˥}}}}\kern2pt]} \hypertarget{q\string_hw\{\string_Ml\string_RM\string_M1}{}
\markboth{\textcolor{darkblue}{\textbf{\ipa{qʰwæ˧ɭɯ˧}}}}{}
\textcolor{teal}{\mytextsc{noun}} \hspace{4pt} Tone: M.
\textcolor{Sepia}{\selectlanguage{english}Vegetable garden.} \zh{菜园。}  \zh{量词}: \textcolor{darkblue}{\textbf{\ipa{kɤ˧˥}}}  \mytextsc{clf}: \textcolor{darkblue}{\textbf{\ipa{kɤ˧˥}}} 
\lhead{\firstmark}
\rhead{\botmark}

\subsection{\hspace{-0.5cm} {\Large \textcolor{darkblue}{\textbf{\ipa{qʰwæ˧mi\#˥}}}}\hspace{0.5cm}[\kern2pt{\textcolor{darkblue}{\textbf{\ipa{qʰwæ˧mi˧}}}}\kern2pt]} \hypertarget{q\string_hw\{\string_Mmi\#\string_T1}{}
\markboth{\textcolor{darkblue}{\textbf{\ipa{qʰwæ˧mi\#˥}}}}{}
\textcolor{teal}{\mytextsc{noun}} \hspace{4pt} Tone: \#H.
\textcolor{Sepia}{\selectlanguage{english}Message, information (extended meaning: letter).} \zh{口信, 信息。}  ¶ \textcolor{darkblue}{\textbf{\ipa{qʰwæ˧mi˧ ʝi˧}}} \textcolor{Sepia}{\selectlanguage{english}to carry a message} \zh{带一个口信}  
 \zh{量词}: \textcolor{darkblue}{\textbf{\ipa{kʰwɤ˥}}}  \mytextsc{clf}: \textcolor{darkblue}{\textbf{\ipa{kʰwɤ˥}}} 
\lhead{\firstmark}
\rhead{\botmark}

\subsection{\hspace{-0.5cm} {\Large \textcolor{darkblue}{\textbf{\ipa{qʰwæ˧ʈɯ˥}}}}\hspace{0.5cm}[\kern2pt{\textcolor{darkblue}{\textbf{\ipa{qʰwæ˩ʈɯ˧˥}}}}\kern2pt]} \hypertarget{q\string_hw\{\string_Mt`M\string_T1}{}
\markboth{\textcolor{darkblue}{\textbf{\ipa{qʰwæ˧ʈɯ˥}}}}{}
\textcolor{teal}{\mytextsc{noun}} \hspace{4pt} Tone: H\#.
\textcolor{Sepia}{\selectlanguage{english}Scarf, kerchief.} \zh{头帕。}  \zh{量词}: \textcolor{darkblue}{\textbf{\ipa{bɤ˧˥}}}  \mytextsc{clf}: \textcolor{darkblue}{\textbf{\ipa{bɤ˧˥}}} 
\lhead{\firstmark}
\rhead{\botmark}

\subsection{\hspace{-0.5cm} {\Large \textcolor{darkblue}{\textbf{\ipa{qʰwæ˩}}}}\hspace{0.5cm}[\kern2pt{\textcolor{darkblue}{\textbf{\ipa{qʰwæ˥}}}}\kern2pt]} \hypertarget{q\string_hw\{\string_B1}{}
\markboth{\textcolor{darkblue}{\textbf{\ipa{qʰwæ˩}}}}{}
\textcolor{teal}{\mytextsc{noun}} \hspace{4pt} Tone: L.
\textcolor{Sepia}{\selectlanguage{english}Fence, made of bamboo or of thorny shrub branches.} \zh{篱笆。}  \zh{量词}: \textcolor{darkblue}{\textbf{\ipa{kɤ˧˥}}}  \mytextsc{clf}: \textcolor{darkblue}{\textbf{\ipa{kɤ˧˥}}} 
\lhead{\firstmark}
\rhead{\botmark}

\subsection{\hspace{-0.5cm} {\Large \textcolor{darkblue}{\textbf{\ipa{qʰwæ˩\textsubscript{a}}}}}\hspace{0.5cm}[\kern2pt{\textcolor{darkblue}{\textbf{\ipa{qʰwæ˧˥}}}}\kern2pt]} \hypertarget{q\string_hw\{\string_Ba1}{}
\markboth{\textcolor{darkblue}{\textbf{\ipa{qʰwæ˩\textsubscript{a}}}}}{}
\textcolor{teal}{\mytextsc{verb}} \hspace{4pt} Tone: L\textsubscript{a}.
\textcolor{Sepia}{\selectlanguage{english}To block.} \zh{挡住。} 
\lhead{\firstmark}
\rhead{\botmark}

\subsection{\hspace{-0.5cm} {\Large \textcolor{darkblue}{\textbf{\ipa{qʰwæ˩kɤ˩}}}}\hspace{0.5cm}[\kern2pt{\textcolor{darkblue}{\textbf{\ipa{qʰwæ˩kɤ˩˥}}}}\kern2pt]} \hypertarget{q\string_hw\{\string_Bk7\string_B1}{}
\markboth{\textcolor{darkblue}{\textbf{\ipa{qʰwæ˩kɤ˩}}}}{}
\textcolor{teal}{\mytextsc{noun}} \hspace{4pt} Tone: L.
\textcolor{Sepia}{\selectlanguage{english}A sort of shrub, reaching 1.5 to 2 meters in height.} \zh{一种灌木,1.5至2米高,可以当篱笆用。}  ¶ \textcolor{darkblue}{\textbf{\ipa{qʰwæ˩kɤ˩-dzi˩˥}}} \textcolor{Sepia}{\selectlanguage{english}same meaning} \zh{同上}  
 \zh{量词}: \textcolor{darkblue}{\textbf{\ipa{dzi˩, ʝi˧}}}  \mytextsc{clf}: \textcolor{darkblue}{\textbf{\ipa{dzi˩, ʝi˧}}} 
\lhead{\firstmark}
\rhead{\botmark}

\subsection{\hspace{-0.5cm} {\Large \textcolor{darkblue}{\textbf{\ipa{qʰwæ˧˥}}} \textsubscript{1}}\hspace{0.5cm}[\kern2pt{\textcolor{darkblue}{\textbf{\ipa{qʰwæ˥}}}}\kern2pt]} \hypertarget{q\string_hw\{\string_M\string_T1}{}
\markboth{\textcolor{darkblue}{\textbf{\ipa{qʰwæ˧˥}}} \textsubscript{1}}{}
\textcolor{teal}{\mytextsc{verb}} \hspace{4pt} Tone: MH.
\textcolor{Sepia}{\selectlanguage{english}To break (bowl, jar), to crack (nuts).} \zh{弄碎。}  ¶ \textcolor{darkblue}{\textbf{\ipa{ʁo˧do˧ qʰwæ˧˥}}} \textcolor{Sepia}{\selectlanguage{english}to crack walnuts} \zh{敲开坚果(在永宁,不用夹子:用锤子敲开)}  

\lhead{\firstmark}
\rhead{\botmark}

\subsection{\hspace{-0.5cm} {\Large \textcolor{darkblue}{\textbf{\ipa{qʰwæ˧˥}}} \textsubscript{2}}\hspace{0.5cm}[\kern2pt{\textcolor{darkblue}{\textbf{\ipa{qʰwæ˧˥}}}}\kern2pt]} \hypertarget{q\string_hw\{\string_M\string_T2}{}
\markboth{\textcolor{darkblue}{\textbf{\ipa{qʰwæ˧˥}}} \textsubscript{2}}{}
\textcolor{teal}{\mytextsc{verb}} \hspace{4pt} Tone: MH.
\textcolor{Sepia}{\selectlanguage{english}To slap.} \zh{掴、打。}  ¶ \textcolor{darkblue}{\textbf{\ipa{le˧-qʰwæ˧-ze˥}}} \textcolor{Sepia}{\selectlanguage{english}\mytextsc{accomp} \string_ \mytextsc{pfv}} \zh{掴了}  
 ¶ \textcolor{darkblue}{\textbf{\ipa{zɯ˧ɻ̍˧ qʰwæ˩}}} \textcolor{Sepia}{\selectlanguage{english}to slap/smack someone's cheek} \zh{打嘴巴}  
 ¶ \textcolor{darkblue}{\textbf{\ipa{zɯ˧ɻ̍˧ | ɖɯ˧-ɭɯ˧ | tʰi˧-qʰwæ˧-bi˥!}}} \textcolor{Sepia}{\selectlanguage{english}I'm going to slap your cheek! (Said by an adult to a child)} \zh{我要打嘴巴了!(对孩子说)}  

\lhead{\firstmark}
\rhead{\botmark}

\subsection{\hspace{-0.5cm} {\Large \textcolor{darkblue}{\textbf{\ipa{qʰwæ˧˥\textsubscript{a}}}}}\hspace{0.5cm}[\kern2pt{\textcolor{darkblue}{\textbf{\ipa{qʰwæ˩˥}}}}\kern2pt]} \hypertarget{q\string_hw\{\string_M\string_Ta1}{}
\markboth{\textcolor{darkblue}{\textbf{\ipa{qʰwæ˧˥\textsubscript{a}}}}}{}
\textcolor{teal}{\mytextsc{classifier}} \hspace{4pt} Tone: MH\textsubscript{a}.
\textcolor{Sepia}{\selectlanguage{english}Classifier for filaments of hemp before spinning.} \zh{量词:丝,如纺之前的麻丝(一根)。} 
\lhead{\firstmark}
\rhead{\botmark}

\subsection{\hspace{-0.5cm} {\Large \textcolor{darkblue}{\textbf{\ipa{qʰwɤ˧}}}}\hspace{0.5cm}[\kern2pt{\textcolor{darkblue}{\textbf{\ipa{qʰwɤ˥}}}}\kern2pt]} \hypertarget{q\string_hw7\string_M1}{}
\markboth{\textcolor{darkblue}{\textbf{\ipa{qʰwɤ˧}}}}{}
\textcolor{teal}{\mytextsc{noun}} \hspace{4pt} Tone: M.
\textcolor{Sepia}{\selectlanguage{english}Traces, track (left by an animal).} \zh{痕迹。}  \zh{量词}: \textcolor{darkblue}{\textbf{\ipa{pʰo˧˥}}}  \mytextsc{clf}: \textcolor{darkblue}{\textbf{\ipa{pʰo˧˥}}} 
\lhead{\firstmark}
\rhead{\botmark}

\subsection{\hspace{-0.5cm} {\Large \textcolor{darkblue}{\textbf{\ipa{qʰwɤ˧\textsubscript{a}}}}}\hspace{0.5cm}[\kern2pt{\textcolor{darkblue}{\textbf{\ipa{qʰwɤ˥}}}}\kern2pt]} \hypertarget{q\string_hw7\string_Ma1}{}
\markboth{\textcolor{darkblue}{\textbf{\ipa{qʰwɤ˧\textsubscript{a}}}}}{}
\textcolor{teal}{\mytextsc{verb}} \hspace{4pt} Tone: M\textsubscript{a}.
\textcolor{Sepia}{\selectlanguage{english}To heal (wound, disease, broken bone...).} \zh{治好(骨折、病)。}  ¶ \textcolor{darkblue}{\textbf{\ipa{le˧-qʰwɤ˧-ɲi˥!}}} \textcolor{Sepia}{\selectlanguage{english}It is healed! / It has healed!} \zh{治好了!}  

\lhead{\firstmark}
\rhead{\botmark}

\subsection{\hspace{-0.5cm} {\Large \textcolor{darkblue}{\textbf{\ipa{qʰwɤ˧bi˩}}}}\hspace{0.5cm}[\kern2pt{\textcolor{darkblue}{\textbf{\ipa{qʰwɤ˧bi˩}}}}\kern2pt]} \hypertarget{q\string_hw7\string_Mbi\string_B1}{}
\markboth{\textcolor{darkblue}{\textbf{\ipa{qʰwɤ˧bi˩}}}}{}
\textcolor{teal}{\mytextsc{noun}} \hspace{4pt} Tone: L\#.
\ding{202} \textcolor{Sepia}{\selectlanguage{english}Hoof (of horse); foot (of dog).} \zh{马蹄、马的脚。}  ¶ \textcolor{darkblue}{\textbf{\ipa{ʐwæ˧-qʰwɤ˧bi˥\#}}} \textcolor{Sepia}{\selectlanguage{english}horse hoof} \zh{马蹄、(马、狗……的)脚}  
 ¶ \textcolor{darkblue}{\textbf{\ipa{kʰv̩˩-qʰwɤ˩bi˥\#}}} \textcolor{Sepia}{\selectlanguage{english}dog's foot} \zh{狗脚}  
 \zh{量词}: \textcolor{darkblue}{\textbf{\ipa{bi˩}}} \textcolor{darkblue}{\textbf{\ipa{tʰv̩˧˥}}} \ding{203} \textcolor{Sepia}{\selectlanguage{english}Track, trail, spoor, footprints (of an animal).} \zh{动物脚的痕迹、行径。}  \mytextsc{clf}: \textcolor{darkblue}{\textbf{\ipa{bi˩}}} \textcolor{darkblue}{\textbf{\ipa{tʰv̩˧˥}}} 
\lhead{\firstmark}
\rhead{\botmark}

\subsection{\hspace{-0.5cm} {\Large \textcolor{darkblue}{\textbf{\ipa{qʰwɤ˧mi˥\$}}}}\hspace{0.5cm}[\kern2pt{\textcolor{darkblue}{\textbf{\ipa{qʰwɤ˧mi˥}}}}\kern2pt]} \hypertarget{q\string_hw7\string_Mmi\string_T\$1}{}
\markboth{\textcolor{darkblue}{\textbf{\ipa{qʰwɤ˧mi˥\$}}}}{}
\textcolor{teal}{\mytextsc{noun}} \hspace{4pt} Tone: H\$.
\textcolor{Sepia}{\selectlanguage{english}Large bowl; it used to be made of wood.} \zh{大碗(以前碗是用木头做的)。} \textit{See:} \hyperlink{}{\textcolor{darkblue}{\textbf{\ipa{qʰwɤ˧pɤ˥\$}}}} 
\lhead{\firstmark}
\rhead{\botmark}

\subsection{\hspace{-0.5cm} {\Large \textcolor{darkblue}{\textbf{\ipa{qʰwɤ˧pɤ˥\$}}}}\hspace{0.5cm}[\kern2pt{\textcolor{darkblue}{\textbf{\ipa{qʰwɤ˧pɤ˥}}}}\kern2pt]} \hypertarget{q\string_hw7\string_Mp7\string_T\$1}{}
\markboth{\textcolor{darkblue}{\textbf{\ipa{qʰwɤ˧pɤ˥\$}}}}{}
\textcolor{teal}{\mytextsc{noun}} \hspace{4pt} Tone: H\$.
\textcolor{Sepia}{\selectlanguage{english}Large bowl.} \zh{大碗。} \textit{See:} \hyperlink{}{\textcolor{darkblue}{\textbf{\ipa{qʰwɤ˧mi˥\$}}}} 
\lhead{\firstmark}
\rhead{\botmark}

\subsection{\hspace{-0.5cm} {\Large \textcolor{darkblue}{\textbf{\ipa{qʰwɤ˧ʂe˩}}}}\hspace{0.5cm}[\kern2pt{\textcolor{darkblue}{\textbf{\ipa{qʰwɤ˧ʂe˩}}}}\kern2pt]} \hypertarget{q\string_hw7\string_Ms`e\string_B1}{}
\markboth{\textcolor{darkblue}{\textbf{\ipa{qʰwɤ˧ʂe˩}}}}{}
\textcolor{teal}{\mytextsc{noun}} \hspace{4pt} Tone: L\#.
\textcolor{Sepia}{\selectlanguage{english}Horseshoe.} \zh{马蹄铁。}  ¶ \textcolor{darkblue}{\textbf{\ipa{ʐwæ˧-qʰwɤ˧ʂe˥ (+ɲi˩)}}} \textcolor{Sepia}{\selectlanguage{english}horseshoe} \zh{马蹄铁}  
 \zh{量词}: \textcolor{darkblue}{\textbf{\ipa{nɑ˧}}} \textcolor{darkblue}{\textbf{\ipa{pʰo˧˥}}}  \mytextsc{clf}: \textcolor{darkblue}{\textbf{\ipa{nɑ˧}}} \textcolor{darkblue}{\textbf{\ipa{pʰo˧˥}}} 
\lhead{\firstmark}
\rhead{\botmark}

\subsection{\hspace{-0.5cm} {\Large \textcolor{darkblue}{\textbf{\ipa{qʰwɤ˧to˩}}}}\hspace{0.5cm}[\kern2pt{\textcolor{darkblue}{\textbf{\ipa{qʰwɤ˧to˩}}}}\kern2pt]} \hypertarget{q\string_hw7\string_Mto\string_B1}{}
\markboth{\textcolor{darkblue}{\textbf{\ipa{qʰwɤ˧to˩}}}}{}
\textcolor{teal}{\mytextsc{noun}} \hspace{4pt} Tone: L\#.
\textcolor{Sepia}{\selectlanguage{english}Tip of the shoulder.} \zh{肩膀的末端。}  ¶ \textcolor{darkblue}{\textbf{\ipa{hĩ˧ ʈʂʰɯ˧-v̩˧, | qʰwɤ˧to˩ | ɖɯ˧-pi˧˥ | ʂwæ˧-hṽ˩-di˩!}}} \textcolor{Sepia}{\selectlanguage{english}This person's shoulders are not quite straight! / His/her shoulders don't align!} \zh{这个人的肩膀不正,一高一低!}  
 \zh{量词}: \textcolor{darkblue}{\textbf{\ipa{ɭɯ˧}}}  \mytextsc{clf}: \textcolor{darkblue}{\textbf{\ipa{ɭɯ˧}}} 
\lhead{\firstmark}
\rhead{\botmark}

\subsection{\hspace{-0.5cm} {\Large \textcolor{darkblue}{\textbf{\ipa{qʰwɤ˧tʰv̩\#˥}}}}\hspace{0.5cm}[\kern2pt{\textcolor{darkblue}{\textbf{\ipa{qʰwɤ˧tʰv̩˧}}}}\kern2pt]} \hypertarget{q\string_hw7\string_Mt\string_hv\string_=\#\string_T1}{}
\markboth{\textcolor{darkblue}{\textbf{\ipa{qʰwɤ˧tʰv̩\#˥}}}}{}
\textcolor{teal}{\mytextsc{noun}} \hspace{4pt} Tone: \#H.
\textcolor{Sepia}{\selectlanguage{english}Bamboo basket to carry water (on back).} \zh{竹篓。}  \zh{量词}: \textcolor{darkblue}{\textbf{\ipa{ɭɯ˧}}}  \mytextsc{clf}: \textcolor{darkblue}{\textbf{\ipa{ɭɯ˧}}} 
\lhead{\firstmark}
\rhead{\botmark}

\subsection{\hspace{-0.5cm} {\Large \textcolor{darkblue}{\textbf{\ipa{qʰwɤ˧tsʰi˩}}}}\hspace{0.5cm}[\kern2pt{\textcolor{darkblue}{\textbf{\ipa{qʰwɤ˧tsʰi˩}}}}\kern2pt]} \hypertarget{q\string_hw7\string_Mts\string_hi\string_B1}{}
\markboth{\textcolor{darkblue}{\textbf{\ipa{qʰwɤ˧tsʰi˩}}}}{}
\textcolor{teal}{\mytextsc{noun}} \hspace{4pt} Tone: L\#.
\textcolor{Sepia}{\selectlanguage{english}Shoulder.} \zh{肩膀。}  ¶ \textcolor{darkblue}{\textbf{\ipa{qʰwɤ˧tsʰi˩-ʁo˩ | hwæ˧pʰæ˩ | ɖɯ˧-nɑ˧-ʈʂʰɯ˧ gɤ˧˥}}} \textcolor{Sepia}{\selectlanguage{english}to carry a hoe on the shoulder} \zh{肩上扛一把锄头}  
 \zh{量词}: \textcolor{darkblue}{\textbf{\ipa{ɭɯ˧}}}  \mytextsc{clf}: \textcolor{darkblue}{\textbf{\ipa{ɭɯ˧}}} 
\lhead{\firstmark}
\rhead{\botmark}

\subsection{\hspace{-0.5cm} {\Large \textcolor{darkblue}{\textbf{\ipa{qʰwɤ˧zo˥\$}}}}\hspace{0.5cm}[\kern2pt{\textcolor{darkblue}{\textbf{\ipa{qʰwɤ˧zo˥}}}}\kern2pt]} \hypertarget{q\string_hw7\string_Mzo\string_T\$1}{}
\markboth{\textcolor{darkblue}{\textbf{\ipa{qʰwɤ˧zo˥\$}}}}{}
\textcolor{teal}{\mytextsc{noun}} \hspace{4pt} Tone: H\$.
\textcolor{Sepia}{\selectlanguage{english}Small bowl.} \zh{小碗。} 
\lhead{\firstmark}
\rhead{\botmark}

\subsection{\hspace{-0.5cm} {\Large \textcolor{darkblue}{\textbf{\ipa{qʰwɤ˩\textsubscript{a}}}} \textsubscript{1}}\hspace{0.5cm}[\kern2pt{\textcolor{darkblue}{\textbf{\ipa{qʰwɤ˩˥}}}}\kern2pt]} \hypertarget{q\string_hw7\string_Ba1}{}
\markboth{\textcolor{darkblue}{\textbf{\ipa{qʰwɤ˩\textsubscript{a}}}} \textsubscript{1}}{}
\textcolor{teal}{\mytextsc{adjective}} \hspace{4pt} Tone: L\textsubscript{a}.
\textcolor{Sepia}{\selectlanguage{english}Intelligent.} \zh{聪明。}  ¶ \textcolor{darkblue}{\textbf{\ipa{qʰwɤ˩-hĩ˩˥}}} \textcolor{Sepia}{\selectlanguage{english}\mytextsc{rel}/\mytextsc{nmlz}} \zh{聪明的}  
 ¶ \textcolor{darkblue}{\textbf{\ipa{qʰwɤ˩-le˥!}}} \textcolor{Sepia}{\selectlanguage{english}(You are/(s)he is) clever! (A comment when someone says/does something clever)} \zh{很聪明! / 太聪明了!}  
 ¶ \textcolor{darkblue}{\textbf{\ipa{ɖwæ˧˥ | qʰwɤ˩˥!}}} \textcolor{Sepia}{\selectlanguage{english}\mytextsc{intensive}.very \string_} \zh{很聪明!}  

\lhead{\firstmark}
\rhead{\botmark}

\subsection{\hspace{-0.5cm} {\Large \textcolor{darkblue}{\textbf{\ipa{qʰwɤ˩\textsubscript{a}}}} \textsubscript{2}}\hspace{0.5cm}[\kern2pt{\textcolor{darkblue}{\textbf{\ipa{qʰwɤ˩˥}}}}\kern2pt]} \hypertarget{q\string_hw7\string_Ba2}{}
\markboth{\textcolor{darkblue}{\textbf{\ipa{qʰwɤ˩\textsubscript{a}}}} \textsubscript{2}}{}
\textcolor{teal}{\mytextsc{adjective}} \hspace{4pt} Tone: L\textsubscript{a}.
\textcolor{Sepia}{\selectlanguage{english}Bad.} \zh{坏。}  ¶ \textcolor{darkblue}{\textbf{\ipa{kʰv̩˧ | qʰwɤ˩-hĩ˩˥}}} \textcolor{Sepia}{\selectlanguage{english}a bad year (a year when crops are not good)} \zh{(收成)不好的一年}  
 ¶ \textcolor{darkblue}{\textbf{\ipa{kʰv̩˧ qʰwɤ˧˥}}} \textcolor{Sepia}{\selectlanguage{english}a bad year (a year when crops are not good)} \zh{(收成)不好的一年}  
 ¶ \textcolor{darkblue}{\textbf{\ipa{tsʰi˧-ʝi˧, | kʰv̩˧qʰwɤ˧ tʰv̩˧˥!}}} \textcolor{Sepia}{\selectlanguage{english}This year is a bad year! (=a year when crops are not good)} \zh{今年,年景不好!(收成不好)}  
 ¶ \textcolor{darkblue}{\textbf{\ipa{ʈʂʰɯ˧ | nv̩˩mi˩˥ | ɖwæ˧˥ | qʰwɤ˩˥!}}} \textcolor{Sepia}{\selectlanguage{english}He has a really bad heart! / He is a really bad man!} \zh{他心很坏!}  
 ¶ \textcolor{darkblue}{\textbf{\ipa{qʰwɤ˩-ʝi˩˥}}} \textcolor{Sepia}{\selectlanguage{english}to do bad things: to damage stuff; to annoy people...} \zh{干坏事:损坏东西,干扰人家……}  
 ¶ \textcolor{darkblue}{\textbf{\ipa{ʈʂʰɯ˧-ɳɯ˧ | njɤ˧-bv̩˧ tso˧\textasciitilde{}tso˧ | le˧-qʰwɤ˩-ʝi˩-ze˩!}}} \textcolor{Sepia}{\selectlanguage{english}(S)he has damaged my stuff!} \zh{他弄坏了我的东西!}  
 ¶ \textcolor{darkblue}{\textbf{\ipa{hĩ˧ qʰwɤ˧-ʝi˥}}} \textcolor{Sepia}{\selectlanguage{english}to annoy people} \zh{干扰人家、麻烦人}  
 ¶ \textcolor{darkblue}{\textbf{\ipa{ʈʂʰɯ˧ | to˩to˧mi˥ hĩ˩ qʰwɤ˩-ʝi˩!}}} \textcolor{Sepia}{\selectlanguage{english}(S)he purposedly annoys people! / (S)he annoys people on purpose!} \zh{他故意麻烦人!}  

\lhead{\firstmark}
\rhead{\botmark}

\subsection{\hspace{-0.5cm} {\Large \textcolor{darkblue}{\textbf{\ipa{qʰwɤ˩ɖɯ˩}}}}\hspace{0.5cm}[\kern2pt{\textcolor{darkblue}{\textbf{\ipa{qʰwɤ˩ɖɯ˩˥}}}}\kern2pt]} \hypertarget{q\string_hw7\string_Bd`M\string_B1}{}
\markboth{\textcolor{darkblue}{\textbf{\ipa{qʰwɤ˩ɖɯ˩}}}}{}
\textcolor{teal}{\mytextsc{noun}} \hspace{4pt} Tone: L.
\textcolor{Sepia}{\selectlanguage{english}Relatives, members of the family.} \zh{亲戚。}  ¶ \textcolor{darkblue}{\textbf{\ipa{ə˧zɯ˩ | qʰwɤ˩ɖɯ˩ ɲi˥}}} \textcolor{Sepia}{\selectlanguage{english}We two belong to the same family.} \zh{咱们两个是一家人。}  
 ¶ \textcolor{darkblue}{\textbf{\ipa{qʰwɤ˩ɖɯ˩˥, | v̩˩dze˩˥}}} \textcolor{Sepia}{\selectlanguage{english}family and friends, extended family circle} \zh{亲人:泛指亲戚与亲密朋友们}  
 ¶ \textcolor{darkblue}{\textbf{\ipa{qʰwɤ˩ɖɯ˩ to˥}}} \textcolor{Sepia}{\selectlanguage{english}to establish family ties between two families (through marriage)} \zh{建立起两个家庭之间的联系(通过婚姻)}  
 \zh{量词}: \textcolor{darkblue}{\textbf{\ipa{v̩˧}}}  \mytextsc{clf}: \textcolor{darkblue}{\textbf{\ipa{v̩˧}}} 
\lhead{\firstmark}
\rhead{\botmark}

\subsection{\hspace{-0.5cm} {\Large \textcolor{darkblue}{\textbf{\ipa{qʰwɤ˧˥}}} \textsubscript{1}}\hspace{0.5cm}[\kern2pt{\textcolor{darkblue}{\textbf{\ipa{qʰwɤ˧˥}}}}\kern2pt]} \hypertarget{q\string_hw7\string_M\string_T1}{}
\markboth{\textcolor{darkblue}{\textbf{\ipa{qʰwɤ˧˥}}} \textsubscript{1}}{}
\textcolor{teal}{\mytextsc{noun}} \hspace{4pt} Tone: MH.
\textcolor{Sepia}{\selectlanguage{english}Bowl.} \zh{碗。}  \zh{量词}: \textcolor{darkblue}{\textbf{\ipa{ɭɯ˧}}}  \mytextsc{clf}: \textcolor{darkblue}{\textbf{\ipa{ɭɯ˧}}} 
\lhead{\firstmark}
\rhead{\botmark}

\subsection{\hspace{-0.5cm} {\Large \textcolor{darkblue}{\textbf{\ipa{qʰwɤ˧˥}}} \textsubscript{2}}\hspace{0.5cm}[\kern2pt{\textcolor{darkblue}{\textbf{\ipa{qʰwɤ˧˥}}}}\kern2pt]} \hypertarget{q\string_hw7\string_M\string_T2}{}
\markboth{\textcolor{darkblue}{\textbf{\ipa{qʰwɤ˧˥}}} \textsubscript{2}}{}
\textcolor{teal}{\mytextsc{noun}} \hspace{4pt} Tone: MH.
\textcolor{Sepia}{\selectlanguage{english}Tale, story, yarn.} \zh{故事。}  ¶ \textcolor{darkblue}{\textbf{\ipa{æ˧ʂæ˧-qʰwɤ˧˥}}} \textcolor{Sepia}{\selectlanguage{english}tale, folk tale} \zh{老故事}  
 \zh{量词}: \textcolor{darkblue}{\textbf{\ipa{kʰwɤ˥}}}  \mytextsc{clf}: \textcolor{darkblue}{\textbf{\ipa{kʰwɤ˥}}} 
\lhead{\firstmark}
\rhead{\botmark}

\subsection{\hspace{-0.5cm} {\Large \textcolor{darkblue}{\textbf{\ipa{qʰwɤ˧˥\textsubscript{a}}}}}\hspace{0.5cm}[\kern2pt{\textcolor{darkblue}{\textbf{\ipa{qʰwɤ˧˥}}}}\kern2pt]} \hypertarget{q\string_hw7\string_M\string_Ta1}{}
\markboth{\textcolor{darkblue}{\textbf{\ipa{qʰwɤ˧˥\textsubscript{a}}}}}{}
\textcolor{teal}{\mytextsc{classifier}} \hspace{4pt} Tone: MH\textsubscript{a}.
\textcolor{Sepia}{\selectlanguage{english}A bowl(ful) of.} \zh{量词:碗。} 
\lhead{\firstmark}
\rhead{\botmark}

\newpage
\section*{\centering- \textcolor{darkblue}{\textbf{\ipa{ɻ}}} -}
\subsection{\hspace{-0.5cm} {\Large \textcolor{darkblue}{\textbf{\ipa{ɻ̍˩}}}}\hspace{0.5cm}[\kern2pt{\textcolor{darkblue}{\textbf{\ipa{ɻ̍˥}}}}\kern2pt]} \hypertarget{r£`̍\string_B1}{}
\markboth{\textcolor{darkblue}{\textbf{\ipa{ɻ̍˩}}}}{}
\textcolor{teal}{\mytextsc{noun}} \hspace{4pt} Tone: L.
\textcolor{Sepia}{\selectlanguage{english}Side, direction.} \zh{面(一个四方形物品的四面)。}  ¶ \textcolor{darkblue}{\textbf{\ipa{[“Housebuilding”] ʐv̩˧-ɻ̍˥}}} \textcolor{Sepia}{\selectlanguage{english}the four directions, the four sides (e.g. of a house)} \zh{四面}  

\lhead{\firstmark}
\rhead{\botmark}

\subsection{\hspace{-0.5cm} {\Large \textcolor{darkblue}{\textbf{\ipa{ɻ̍˩\textsubscript{b}}}}}\hspace{0.5cm}[\kern2pt{\textcolor{darkblue}{\textbf{\ipa{ɻ̍˩˥}}}}\kern2pt]} \hypertarget{r£`̍\string_Bb1}{}
\markboth{\textcolor{darkblue}{\textbf{\ipa{ɻ̍˩\textsubscript{b}}}}}{}
\textcolor{teal}{\mytextsc{verb}} \hspace{4pt} Tone: L\textsubscript{b}.
\textcolor{Sepia}{\selectlanguage{english}To face, to turn toward.} \zh{对着。}  ¶ \textcolor{darkblue}{\textbf{\ipa{mɤ˧-ɻ̍˩}}} \textcolor{Sepia}{\selectlanguage{english}\mytextsc{neg}} \zh{\mytextsc{neg}}  
 ¶ \textcolor{darkblue}{\textbf{\ipa{ɖɯ˧-ɻ̍˧\textasciitilde{}ɻ̍˩}}} \textcolor{Sepia}{\selectlanguage{english}\mytextsc{delimitative} \string_ \mytextsc{red}} \zh{\mytextsc{delimitative} \string_ \mytextsc{red}}  
 ¶ \textcolor{darkblue}{\textbf{\ipa{ze˩gi˧ ɻ̍˥?}}} \textcolor{Sepia}{\selectlanguage{english}In which direction should (I) look? / Which direction should I turn to?} \zh{(我要)往哪边转?}  
 ¶ \textcolor{darkblue}{\textbf{\ipa{no˧ | ʈʂʰɯ˧tɕo˧ ɻ̍˩!}}} \textcolor{Sepia}{\selectlanguage{english}Turn this way! / Turn towards this direction!} \zh{你往这里转/往这里看!}  
 ¶ \textcolor{darkblue}{\textbf{\ipa{gɤ˩-ɻ̍˥ mv̩˩-ɻ̍˩, | ə˧tso˧ li˧?}}} \textcolor{Sepia}{\selectlanguage{english}You turn in all directions; what are you looking for/at? / What are you looking for in all directions?} \zh{你左转右转,(到底)在看什么?}  

\lhead{\firstmark}
\rhead{\botmark}

\subsection{\hspace{-0.5cm} {\Large \textcolor{darkblue}{\textbf{\ipa{ɻ̍˧bɤ˧}}}}\hspace{0.5cm}[\kern2pt{\textcolor{darkblue}{\textbf{\ipa{ɻ̍˧bɤ˧}}}}\kern2pt]} \hypertarget{r£`̍\string_Mb7\string_M1}{}
\markboth{\textcolor{darkblue}{\textbf{\ipa{ɻ̍˧bɤ˧}}}}{}
\textcolor{teal}{\mytextsc{noun}} \hspace{4pt} Tone: M.
\textcolor{Sepia}{\selectlanguage{english}The truth; the facts.} \zh{实情,真理。}  ¶ \textcolor{darkblue}{\textbf{\ipa{njɤ˧-ɳɯ˧ | ɻ̍˧bɤ˧ | ʐwɤ˩-bi˩˥!}}} \textcolor{Sepia}{\selectlanguage{english}I am going to tell the truth!} \zh{我要把实情说出来!}  
 \zh{量词}: \textcolor{darkblue}{\textbf{\ipa{kʰwɤ˥}}}  \mytextsc{clf}: \textcolor{darkblue}{\textbf{\ipa{kʰwɤ˥}}} 
\lhead{\firstmark}
\rhead{\botmark}

\subsection{\hspace{-0.5cm} {\Large \textcolor{darkblue}{\textbf{\ipa{ɻ̍˧qʰv̩˧}}}}\hspace{0.5cm}[\kern2pt{\textcolor{darkblue}{\textbf{\ipa{ɻ̍˧qʰv̩˧}}}}\kern2pt]} \hypertarget{r£`̍\string_Mq\string_hv\string_=\string_M1}{}
\markboth{\textcolor{darkblue}{\textbf{\ipa{ɻ̍˧qʰv̩˧}}}}{}
\textcolor{teal}{\mytextsc{noun}} \hspace{4pt} Tone: M.
\textcolor{Sepia}{\selectlanguage{english}Warm springs.} \zh{温泉。}  ¶ \textcolor{darkblue}{\textbf{\ipa{ɻ̍˧qʰv̩˧-dʑɯ˩}}} \textcolor{Sepia}{\selectlanguage{english}warm spring water (not drinkable)} \zh{温泉水(不可饮用)}  
 \zh{量词}: \textcolor{darkblue}{\textbf{\ipa{ɭɯ˧}}}  \mytextsc{clf}: \textcolor{darkblue}{\textbf{\ipa{ɭɯ˧}}} 
\lhead{\firstmark}
\rhead{\botmark}

\subsection{\hspace{-0.5cm} {\Large \textcolor{darkblue}{\textbf{\ipa{ɻ̍˩ɻ̍˧-lo˩}}}}\hspace{0.5cm}[\kern2pt{\textcolor{darkblue}{\textbf{\ipa{ɻ̍˩ɻ̍˧lo˧}}}}\kern2pt]} \hypertarget{r£`̍\string_Br£`̍\string_M-lo\string_B1}{}
\markboth{\textcolor{darkblue}{\textbf{\ipa{ɻ̍˩ɻ̍˧-lo˩}}}}{}
\textcolor{teal}{\mytextsc{noun}} \hspace{4pt} Tone: LM-L.
\textcolor{Sepia}{\selectlanguage{english}The horse walking in second position in a caravan.} \zh{马帮中的第二匹马。} 
\lhead{\firstmark}
\rhead{\botmark}

\subsection{\hspace{-0.5cm} {\Large \textcolor{darkblue}{\textbf{\ipa{ɻ̍˧tɑ˧}}}}\hspace{0.5cm}[\kern2pt{\textcolor{darkblue}{\textbf{\ipa{ɻ̍˧tɑ˧}}}}\kern2pt]} \hypertarget{r£`̍\string_MtA\string_M1}{}
\markboth{\textcolor{darkblue}{\textbf{\ipa{ɻ̍˧tɑ˧}}}}{}
\textcolor{teal}{\mytextsc{noun}} \hspace{4pt} Tone: M.
\textcolor{Sepia}{\selectlanguage{english}Lymph nodes, glands.} \zh{淋巴结。}  \zh{量词}: \textcolor{darkblue}{\textbf{\ipa{ɭɯ˧}}}  \mytextsc{clf}: \textcolor{darkblue}{\textbf{\ipa{ɭɯ˧}}} 
\lhead{\firstmark}
\rhead{\botmark}

\subsection{\hspace{-0.5cm} {\Large \textcolor{darkblue}{\textbf{\ipa{ɻ̍˩ʈʂʰe˧-ɖɯ˩mɑ˩}}}}\hspace{0.5cm}[\kern2pt{\textcolor{darkblue}{\textbf{\ipa{ɻ̍˩ʈʂʰe˧ɖɯ˩mɑ˩}}}}\kern2pt]} \hypertarget{r£`̍\string_Bt`s`\string_he\string_M-d`M\string_BmA\string_B1}{}
\markboth{\textcolor{darkblue}{\textbf{\ipa{ɻ̍˩ʈʂʰe˧-ɖɯ˩mɑ˩}}}}{}
\textcolor{teal}{\mytextsc{noun}} \hspace{4pt} Tone: LM-L.
\textcolor{Sepia}{\selectlanguage{english}Feminine given name.} \zh{女性名字。} 
\lhead{\firstmark}
\rhead{\botmark}

\subsection{\hspace{-0.5cm} {\Large \textcolor{darkblue}{\textbf{\ipa{ɻ̍˩ʈʂʰe˧-tsʰɯ˩ɻ̍˩}}}}\hspace{0.5cm}[\kern2pt{\textcolor{darkblue}{\textbf{\ipa{ɻ̍˩ʈʂʰe˧tsʰɯ˩ɻ̍˩}}}}\kern2pt]} \hypertarget{r£`̍\string_Bt`s`\string_he\string_M-ts\string_hM\string_Br£`̍\string_B1}{}
\markboth{\textcolor{darkblue}{\textbf{\ipa{ɻ̍˩ʈʂʰe˧-tsʰɯ˩ɻ̍˩}}}}{}
\textcolor{teal}{\mytextsc{noun}} \hspace{4pt} Tone: LM-L.
\textcolor{Sepia}{\selectlanguage{english}Masculine given name.} \zh{男性名字。} 
\lhead{\firstmark}
\rhead{\botmark}

\subsection{\hspace{-0.5cm} {\Large \textcolor{darkblue}{\textbf{\ipa{ɻ̍˩ʈʂʰe\#˥}}}}\hspace{0.5cm}[\kern2pt{\textcolor{darkblue}{\textbf{\ipa{ɻ̍˩ʈʂʰe˥}}}}\kern2pt]} \hypertarget{r£`̍\string_Bt`s`\string_he\#\string_T1}{}
\markboth{\textcolor{darkblue}{\textbf{\ipa{ɻ̍˩ʈʂʰe\#˥}}}}{}
\textcolor{teal}{\mytextsc{noun}} \hspace{4pt} Tone: LM+\#H.
\textcolor{Sepia}{\selectlanguage{english}Masculine given name.} \zh{男性名字。} 
\lhead{\firstmark}
\rhead{\botmark}

\subsection{\hspace{-0.5cm} {\Large \textcolor{darkblue}{\textbf{\ipa{‑ɻ̍˩}}}}\hspace{0.5cm}[\kern2pt{\textcolor{darkblue}{\textbf{\ipa{ɻ̍˩˥}}}}\kern2pt]} \hypertarget{‑r£`̍\string_B1}{}
\markboth{\textcolor{darkblue}{\textbf{\ipa{‑ɻ̍˩}}}}{}
\textcolor{teal}{\mytextsc{suffix}} \hspace{4pt} Tone: L.
\textcolor{Sepia}{\selectlanguage{english}\mytextsc{inceptive} (\mytextsc{inchoative}).} \zh{\mytextsc{发端。}} 
\lhead{\firstmark}
\rhead{\botmark}

\subsection{\hspace{-0.5cm} {\Large \textcolor{darkblue}{\textbf{\ipa{=ɻ̍˩}}}}\hspace{0.5cm}[\kern2pt{\textcolor{darkblue}{\textbf{\ipa{ɻ̍˩˥}}}}\kern2pt]} \hypertarget{=r£`̍\string_B1}{}
\markboth{\textcolor{darkblue}{\textbf{\ipa{=ɻ̍˩}}}}{}
\textcolor{teal}{\mytextsc{clitic}} \hspace{4pt} Tone: L.
\textcolor{Sepia}{\selectlanguage{english}Associative plural.} \zh{联想复数:一家人、一族人、一辈人……。}  ¶ \textcolor{darkblue}{\textbf{\ipa{ʈʂʰɯ˧-ʑi˧=ɻ̍˥}}} \textcolor{Sepia}{\selectlanguage{english}this household; the people of this family} \zh{这家的人}  

\lhead{\firstmark}
\rhead{\botmark}

\subsection{\hspace{-0.5cm} {\Large \textcolor{darkblue}{\textbf{\ipa{=ɻæ˩}}}}\hspace{0.5cm}[\kern2pt{\textcolor{darkblue}{\textbf{\ipa{ɻæ˩˥}}}}\kern2pt]} \hypertarget{=r£`\{\string_B1}{}
\markboth{\textcolor{darkblue}{\textbf{\ipa{=ɻæ˩}}}}{}
\textcolor{teal}{\mytextsc{clitic}} \hspace{4pt} Tone: L.
\textcolor{Sepia}{\selectlanguage{english}Plural.} \zh{多数。}  ¶ \textcolor{darkblue}{\textbf{\ipa{ʈʂʰɯ˧-ɻæ˥\$}}} \textcolor{Sepia}{\selectlanguage{english}these things, this sort of things} \zh{这类的东西,……之类}  

\lhead{\firstmark}
\rhead{\botmark}

\subsection{\hspace{-0.5cm} {\Large \textcolor{darkblue}{\textbf{\ipa{ɻæ˩\textsubscript{a}}}}}\hspace{0.5cm}[\kern2pt{\textcolor{darkblue}{\textbf{\ipa{ɻæ˩˥}}}}\kern2pt]} \hypertarget{r£`\{\string_Ba1}{}
\markboth{\textcolor{darkblue}{\textbf{\ipa{ɻæ˩\textsubscript{a}}}}}{}
\textcolor{teal}{\mytextsc{adjective}} \hspace{4pt} Tone: L\textsubscript{a}.
\textcolor{Sepia}{\selectlanguage{english}Shrivelled, flat, shrunken.} \zh{瘪。}  ¶ \textcolor{darkblue}{\textbf{\ipa{ɻæ˩-hĩ˩˥}}} \textcolor{Sepia}{\selectlanguage{english}\mytextsc{rel}/\mytextsc{nmlz}} \zh{瘪的}  
 ¶ \textcolor{darkblue}{\textbf{\ipa{ɻæ˩ti˩ɻæ˥ (-gv̩˩)}}} \textcolor{Sepia}{\selectlanguage{english}shrivelled, flat, shrunken} \zh{瘪瘪的}  

\lhead{\firstmark}
\rhead{\botmark}

\subsection{\hspace{-0.5cm} {\Large \textcolor{darkblue}{\textbf{\ipa{ɻæ˩˥}}} \textsubscript{1}}\hspace{0.5cm}[\kern2pt{\textcolor{darkblue}{\textbf{\ipa{ɻæ˩˥}}}}\kern2pt]} \hypertarget{r£`\{\string_B\string_T1}{}
\markboth{\textcolor{darkblue}{\textbf{\ipa{ɻæ˩˥}}} \textsubscript{1}}{}
\textcolor{teal}{\mytextsc{noun}} \hspace{4pt} Tone: LH.
\textcolor{Sepia}{\selectlanguage{english}Seed.} \zh{种子。}  \zh{量词}: \textcolor{darkblue}{\textbf{\ipa{ɭɯ˧}}}  \mytextsc{clf}: \textcolor{darkblue}{\textbf{\ipa{ɭɯ˧}}} 
\lhead{\firstmark}
\rhead{\botmark}

\subsection{\hspace{-0.5cm} {\Large \textcolor{darkblue}{\textbf{\ipa{ɻæ˩˥}}} \textsubscript{2}}\hspace{0.5cm}[\kern2pt{\textcolor{darkblue}{\textbf{\ipa{ɻæ˩˥}}}}\kern2pt]} \hypertarget{r£`\{\string_B\string_T2}{}
\markboth{\textcolor{darkblue}{\textbf{\ipa{ɻæ˩˥}}} \textsubscript{2}}{}
\textcolor{teal}{\mytextsc{noun}} \hspace{4pt} Tone: LH.
\textcolor{Sepia}{\selectlanguage{english}Yoke (for one or two animals).} \zh{牛轭(单行或双行)。}  ¶ \textcolor{darkblue}{\textbf{\ipa{ʝi˧-ɻæ˥}}} \textcolor{Sepia}{\selectlanguage{english}same meaning as the monosyllabic form: yoke (literally 'ox yoke')} \zh{牛轭}  
 ¶ \textcolor{darkblue}{\textbf{\ipa{ɻæ˩ ʈʂʰɯ˩-ɭɯ˥ / ɻæ˩ ʈʂʰɯ˩-ɭɯ˩˥}}} \textcolor{Sepia}{\selectlanguage{english}\mytextsc{n}+\mytextsc{dem}+\mytextsc{clf;} allows two variants} \zh{这个牛轭}  
 \zh{量词}: \textcolor{darkblue}{\textbf{\ipa{ɭɯ˧}}}  \mytextsc{clf}: \textcolor{darkblue}{\textbf{\ipa{ɭɯ˧}}} 
\lhead{\firstmark}
\rhead{\botmark}

\subsection{\hspace{-0.5cm} {\Large \textcolor{darkblue}{\textbf{\ipa{ɻwæ˥}}}}\hspace{0.5cm}[\kern2pt{\textcolor{darkblue}{\textbf{\ipa{ɻwæ˥}}}}\kern2pt]} \hypertarget{r£`w\{\string_T1}{}
\markboth{\textcolor{darkblue}{\textbf{\ipa{ɻwæ˥}}}}{}
\textcolor{teal}{\mytextsc{verb}} \hspace{4pt} Tone: H.
\ding{202} \textcolor{Sepia}{\selectlanguage{english}To cry (man, and animals: cat, cow, horse, donkey, chicken, lion, wolf…); to call out.} \zh{喊、吼、叫(人、猫、牛、猪、羊、狼、驴、狮子、老虎、豺狼……)。}  ¶ \textcolor{darkblue}{\textbf{\ipa{mɤ˧-ɻwæ˥}}} \textcolor{Sepia}{\selectlanguage{english}\mytextsc{neg}} \zh{不叫}  
 ¶ \textcolor{darkblue}{\textbf{\ipa{ɻwæ˧\textasciitilde{}ɻwæ˧}}} \textcolor{Sepia}{\selectlanguage{english}\mytextsc{red}} \zh{\mytextsc{重叠}}  
 ¶ \textcolor{darkblue}{\textbf{\ipa{hĩ˧ ɻwæ˧-dʑo˩}}} \textcolor{Sepia}{\selectlanguage{english}Someone is shouting} \zh{有人在叫。}  
 ¶ \textcolor{darkblue}{\textbf{\ipa{ɖɯ˧-ɻwæ˧-ɻ̍˥}}} \textcolor{Sepia}{\selectlanguage{english}to call out} \zh{叫一声}  
 ¶ \textcolor{darkblue}{\textbf{\ipa{hwɤ˧li˧ ɻwæ˥-dʑo˩}}} \textcolor{Sepia}{\selectlanguage{english}the cat is calling/crying} \zh{猫在叫}  
 ¶ \textcolor{darkblue}{\textbf{\ipa{æ̃˩ ɻwæ˥}}} \textcolor{Sepia}{\selectlanguage{english}the chicken is cackling} \zh{鸡在叫}  
 ¶ \textcolor{darkblue}{\textbf{\ipa{ʐwæ˧pʰæ˧di˧˥ | tʰi˧-ɻwæ˥-dʑo˩}}} \textcolor{Sepia}{\selectlanguage{english}the donkey is braying} \zh{驴在叫}  
 ¶ \textcolor{darkblue}{\textbf{\ipa{[F5] ʐwæ˧ | tʰi˧-ɻwæ˥-dʑo˩}}} \textcolor{Sepia}{\selectlanguage{english}the horse is whinnying} \zh{马在嘶}  
 ¶ \textcolor{darkblue}{\textbf{\ipa{ʐwæ˧ ɻwæ˧-dʑo˩!}}} \textcolor{Sepia}{\selectlanguage{english}the horse is whinnying} \zh{马在嘶}  
\ding{203} \textcolor{Sepia}{\selectlanguage{english}To invite, to call over.} \zh{请、叫(来)。}  ¶ \textcolor{darkblue}{\textbf{\ipa{ɖɯ˧-ɻwæ˧-ɻ̍˥}}} \textcolor{Sepia}{\selectlanguage{english}\mytextsc{delimitative} \string_ \mytextsc{inceptive}} \zh{请来一下}  
 ¶ \textcolor{darkblue}{\textbf{\ipa{tʰɑ˧-ɻwæ˥!}}} \textcolor{Sepia}{\selectlanguage{english}\mytextsc{prohib}} \zh{不要请!}  
 ¶ \textcolor{darkblue}{\textbf{\ipa{ɻwæ˧-mɤ˧-bi˧!}}} \textcolor{Sepia}{\selectlanguage{english}(We)'re not inviting (him/her)!} \zh{不请他!}  

\lhead{\firstmark}
\rhead{\botmark}

\subsection{\hspace{-0.5cm} {\Large \textcolor{darkblue}{\textbf{\ipa{ɻwæ˥\textsubscript{b}}}}}\hspace{0.5cm}[\kern2pt{\textcolor{darkblue}{\textbf{\ipa{ɻwæ˥}}}}\kern2pt]} \hypertarget{r£`w\{\string_Tb1}{}
\markboth{\textcolor{darkblue}{\textbf{\ipa{ɻwæ˥\textsubscript{b}}}}}{}
\textcolor{teal}{\mytextsc{classifier}} \hspace{4pt} Tone: H\textsubscript{b}.
\textcolor{Sepia}{\selectlanguage{english}Classifier for places.} \zh{量词:地方(一处)。}  ¶ \textcolor{darkblue}{\textbf{\ipa{tʰv̩˧-ɻwæ˧-qo˥ | mɤ˧-tʰv̩˧-sɯ˩!}}} \textcolor{Sepia}{\selectlanguage{english}...has never been to those places} \zh{还没到这些地方}  

\lhead{\firstmark}
\rhead{\botmark}

\newpage
\section*{\centering- \textcolor{darkblue}{\textbf{\ipa{ɻ̃}}} -}
\subsection{\hspace{-0.5cm} {\Large \textcolor{darkblue}{\textbf{\ipa{ɻ̃˥}}}}\hspace{0.5cm}[\kern2pt{\textcolor{darkblue}{\textbf{\ipa{ɻ̃˥}}}}\kern2pt]} \hypertarget{r£`\string_~\string_T1}{}
\markboth{\textcolor{darkblue}{\textbf{\ipa{ɻ̃˥}}}}{}
\textcolor{teal}{\mytextsc{noun}} \hspace{4pt} Tone: \#H.
\textcolor{Sepia}{\selectlanguage{english}Bone.} \zh{骨头。}  \zh{量词}: \textcolor{darkblue}{\textbf{\ipa{kɤ˧˥}}}  \mytextsc{clf}: \textcolor{darkblue}{\textbf{\ipa{kɤ˧˥}}} 
\lhead{\firstmark}
\rhead{\botmark}

\subsection{\hspace{-0.5cm} {\Large \textcolor{darkblue}{\textbf{\ipa{ɻ̃˥}}}}\hspace{0.5cm}[\kern2pt{\textcolor{darkblue}{\textbf{\ipa{ɻ̃˥}}}}\kern2pt]} \hypertarget{r£`\string_~\string_T1}{}
\markboth{\textcolor{darkblue}{\textbf{\ipa{ɻ̃˥}}}}{}
\textcolor{teal}{\mytextsc{adjective}} \hspace{4pt} Tone: H.
\textcolor{Sepia}{\selectlanguage{english}Destitute, impoverished, poor; troubled, helpless.} \zh{困难、贫穷。}  ¶ \textcolor{darkblue}{\textbf{\ipa{le˧-ɻ̃˥-ze˩!}}} \textcolor{Sepia}{\selectlanguage{english}[(S)he] is really poor/helpless!} \zh{(他)真的很穷苦!}  
 ¶ \textcolor{darkblue}{\textbf{\ipa{le˧-ɻ̃˧-bi˧}}} \textcolor{Sepia}{\selectlanguage{english}\mytextsc{accomp} \string_ \mytextsc{fut}\string_imm} \zh{\mytextsc{accomp} \string_ \mytextsc{fut}\string_imm}  
 ¶ \textcolor{darkblue}{\textbf{\ipa{mɤ˧-ɻ̃˥}}} \textcolor{Sepia}{\selectlanguage{english}\mytextsc{neg}} \zh{\mytextsc{neg}}  
 ¶ \textcolor{darkblue}{\textbf{\ipa{le˧-ɻ̃˧-zo˥, | ɻ̃˧-lɑ˩ bi˩-mɤ˩-dʑɯ˩!}}} \textcolor{Sepia}{\selectlanguage{english}“Sure, we're in poverty/we're hungry, but not to the point where bones are bare!” Play on words on 'poor, destitute' and 'bone', which are homophonous. The proverb is used to relativize people's perceived degree of misfortune.} \zh{很困难,也还没有到饿死的程度啊! / 再困难,也还没饿死!(直译:“再困难,也没有露出骨头!”这个成语,来安慰认为自己太可怜的人。)}  
 ¶ \textcolor{darkblue}{\textbf{\ipa{ɻ̃˧-ʐwɤ˧˥}}} \textcolor{Sepia}{\selectlanguage{english}to complain} \zh{诉苦、抱怨}  
 ¶ \textcolor{darkblue}{\textbf{\ipa{ɻ̃˧-ʐwɤ˧ | dɑ˧-ʐwɤ˧-ɻ̍˥}}} \textcolor{Sepia}{\selectlanguage{english}to tell one's miseries, to complain about one's fate} \zh{诉苦、讲自己的不幸}  
 ¶ \textcolor{darkblue}{\textbf{\ipa{ʈʂʰɯ˧ | mɑ˧dɑ˩-qʰwɤ˩, | ɻ̃˧-ʐwɤ˧ | dɑ˧-ʐwɤ˧-ɻ̍˥!}}} \textcolor{Sepia}{\selectlanguage{english}He is unhappy; he spends his time complaining / he is always complaining!} \zh{他不幸福,一直在讲自己怎么可怜!}  

\lhead{\firstmark}
\rhead{\botmark}

\subsection{\hspace{-0.5cm} {\Large \textcolor{darkblue}{\textbf{\ipa{ɻ̃˧}}}}\hspace{0.5cm}[\kern2pt{\textcolor{darkblue}{\textbf{\ipa{ɻ̃˥}}}}\kern2pt]} \hypertarget{r£`\string_~\string_M1}{}
\markboth{\textcolor{darkblue}{\textbf{\ipa{ɻ̃˧}}}}{}
\textcolor{teal}{\mytextsc{noun}} \hspace{4pt} Tone: M.
\textcolor{Sepia}{\selectlanguage{english}Clan.} \zh{家族。}  ¶ \textcolor{darkblue}{\textbf{\ipa{ɻ̃˧ ɖɯ˧-ɻ̃˧}}} \textcolor{Sepia}{\selectlanguage{english}one clan} \zh{一个家族}  
 \zh{量词}: \textcolor{darkblue}{\textbf{\ipa{ɻ̃˧}}}  \mytextsc{clf}: \textcolor{darkblue}{\textbf{\ipa{ɻ̃˧}}} 
\lhead{\firstmark}
\rhead{\botmark}

\subsection{\hspace{-0.5cm} {\Large \textcolor{darkblue}{\textbf{\ipa{ɻ̃˧\textsubscript{b}}}}}\hspace{0.5cm}[\kern2pt{\textcolor{darkblue}{\textbf{\ipa{ɻ̃˥}}}}\kern2pt]} \hypertarget{r£`\string_~\string_Mb1}{}
\markboth{\textcolor{darkblue}{\textbf{\ipa{ɻ̃˧\textsubscript{b}}}}}{}
\textcolor{teal}{\mytextsc{classifier}} \hspace{4pt} Tone: M\textsubscript{b}.
\textcolor{Sepia}{\selectlanguage{english}Classifier for clans / extended families; literally 'one bone'. This unit is located one level higher up than the 'family community' in Fu Maoji's (1983) terminology.} \zh{量词:家族。} 
\lhead{\firstmark}
\rhead{\botmark}

\subsection{\hspace{-0.5cm} {\Large \textcolor{darkblue}{\textbf{\ipa{ɻ̃˧hæ˩}}}}\hspace{0.5cm}[\kern2pt{\textcolor{darkblue}{\textbf{\ipa{ɻ̃˧hæ˩}}}}\kern2pt]} \hypertarget{r£`\string_~\string_Mh\{\string_B1}{}
\markboth{\textcolor{darkblue}{\textbf{\ipa{ɻ̃˧hæ˩}}}}{}
\textcolor{teal}{\mytextsc{noun}} \hspace{4pt} Tone: L\#.
\textcolor{Sepia}{\selectlanguage{english}Cartilage.} \zh{软骨。}  \zh{量词}: \textcolor{darkblue}{\textbf{\ipa{ɭɯ˧}}} \textcolor{darkblue}{\textbf{\ipa{kɤ˧˥}}}  \mytextsc{clf}: \textcolor{darkblue}{\textbf{\ipa{ɭɯ˧}}} \textcolor{darkblue}{\textbf{\ipa{kɤ˧˥}}} 
\lhead{\firstmark}
\rhead{\botmark}

\subsection{\hspace{-0.5cm} {\Large \textcolor{darkblue}{\textbf{\ipa{ɻ̃˧kɤ˩}}}}\hspace{0.5cm}[\kern2pt{\textcolor{darkblue}{\textbf{\ipa{ɻ̃˧kɤ˩}}}}\kern2pt]} \hypertarget{r£`\string_~\string_Mk7\string_B1}{}
\markboth{\textcolor{darkblue}{\textbf{\ipa{ɻ̃˧kɤ˩}}}}{}
\textcolor{teal}{\mytextsc{noun}} \hspace{4pt} Tone: L\#.
\textcolor{Sepia}{\selectlanguage{english}Backbone.} \zh{脊椎骨。}  \zh{量词}: \textcolor{darkblue}{\textbf{\ipa{kɤ˧˥}}}  \mytextsc{clf}: \textcolor{darkblue}{\textbf{\ipa{kɤ˧˥}}} 
\lhead{\firstmark}
\rhead{\botmark}

\subsection{\hspace{-0.5cm} {\Large \textcolor{darkblue}{\textbf{\ipa{ɻ̃˧ko˩}}}}\hspace{0.5cm}[\kern2pt{\textcolor{darkblue}{\textbf{\ipa{ɻ̃˧ko˩}}}}\kern2pt]} \hypertarget{r£`\string_~\string_Mko\string_B1}{}
\markboth{\textcolor{darkblue}{\textbf{\ipa{ɻ̃˧ko˩}}}}{}
\textcolor{teal}{\mytextsc{noun}} \hspace{4pt} Tone: L\#.
\textcolor{Sepia}{\selectlanguage{english}Shinbone, tibia.} \zh{胫骨。}  ¶ \textcolor{darkblue}{\textbf{\ipa{hĩ˧-dzɑ˧ | ɖʐe˧ tʰɑ˧-ʝi˥, | ɻ̃˧ko˩ mi˩ tʰɑ˩-tʰv̩˩. |}}} \textcolor{Sepia}{\selectlanguage{english}The poor must not borrow money; the shinbone must not receive wounds! (Proverb, to explain that one must avoid hitting weak/sensitive spots.)} \zh{“穷人莫借钱,胫骨莫受伤!”}  
 \zh{量词}: \textcolor{darkblue}{\textbf{\ipa{kɤ˧˥}}}  \mytextsc{clf}: \textcolor{darkblue}{\textbf{\ipa{kɤ˧˥}}} 
\lhead{\firstmark}
\rhead{\botmark}

\subsection{\hspace{-0.5cm} {\Large \textcolor{darkblue}{\textbf{\ipa{ɻ̃˧mi˧}}}}\hspace{0.5cm}[\kern2pt{\textcolor{darkblue}{\textbf{\ipa{ɻ̃˧mi˧}}}}\kern2pt]} \hypertarget{r£`\string_~\string_Mmi\string_M1}{}
\markboth{\textcolor{darkblue}{\textbf{\ipa{ɻ̃˧mi˧}}}}{}
\textcolor{teal}{\mytextsc{noun}} \hspace{4pt} Tone: M.
\textcolor{Sepia}{\selectlanguage{english}Tree trunk.} \zh{树干。}  ¶ \textcolor{darkblue}{\textbf{\ipa{si˧dzi˩-ɻ̃˩mi˩}}} \textcolor{Sepia}{\selectlanguage{english}tree trunk} \zh{树干}  
 \zh{量词}: \textcolor{darkblue}{\textbf{\ipa{kɤ˧˥}}}  \mytextsc{clf}: \textcolor{darkblue}{\textbf{\ipa{kɤ˧˥}}} 
\lhead{\firstmark}
\rhead{\botmark}

\subsection{\hspace{-0.5cm} {\Large \textcolor{darkblue}{\textbf{\ipa{ɻ̃˧ʈʂæ˩}}}}\hspace{0.5cm}[\kern2pt{\textcolor{darkblue}{\textbf{\ipa{ɻ̃˧ʈʂæ˩}}}}\kern2pt]} \hypertarget{r£`\string_~\string_Mt`s`\{\string_B1}{}
\markboth{\textcolor{darkblue}{\textbf{\ipa{ɻ̃˧ʈʂæ˩}}}}{}
\textcolor{teal}{\mytextsc{noun}} \hspace{4pt} Tone: L\#.
\textcolor{Sepia}{\selectlanguage{english}Ankle, joint (between the foot and the leg, the arm and the hand…).} \zh{关节部位,关节。}  \zh{量词}: \textcolor{darkblue}{\textbf{\ipa{ʈʂæ˧˥}}}  \mytextsc{clf}: \textcolor{darkblue}{\textbf{\ipa{ʈʂæ˧˥}}} 
\lhead{\firstmark}
\rhead{\botmark}

\subsection{\hspace{-0.5cm} {\Large \textcolor{darkblue}{\textbf{\ipa{ɻ̃˧ʈʂwæ˩}}}}\hspace{0.5cm}[\kern2pt{\textcolor{darkblue}{\textbf{\ipa{ɻ̃˧ʈʂwæ˩}}}}\kern2pt]} \hypertarget{r£`\string_~\string_Mt`s`w\{\string_B1}{}
\markboth{\textcolor{darkblue}{\textbf{\ipa{ɻ̃˧ʈʂwæ˩}}}}{}
\textcolor{teal}{\mytextsc{noun}} \hspace{4pt} Tone: L\#.
\textcolor{Sepia}{\selectlanguage{english}\textit{Toricellia angulata Oliv.}.} \zh{接骨丹。}  ¶ \textcolor{darkblue}{\textbf{\ipa{ɻ̃˧ʈʂwæ˩-si˩}}} \textcolor{Sepia}{\selectlanguage{english}same meaning} \zh{同上}  

\lhead{\firstmark}
\rhead{\botmark}

\newpage
\section*{\centering- \textcolor{darkblue}{\textbf{\ipa{ʁ}}} -}
\subsection{\hspace{-0.5cm} {\Large \textcolor{darkblue}{\textbf{\ipa{ʁɑ˥}}}}\hspace{0.5cm}[\kern2pt{\textcolor{darkblue}{\textbf{\ipa{ʁɑ˥}}}}\kern2pt]} \hypertarget{RA\string_T1}{}
\markboth{\textcolor{darkblue}{\textbf{\ipa{ʁɑ˥}}}}{}
\textcolor{teal}{\mytextsc{noun}} \hspace{4pt} Tone: \#H.
\textcolor{Sepia}{\selectlanguage{english}Strength.} \zh{力气。}  ¶ \textcolor{darkblue}{\textbf{\ipa{ʁɑ˧ ʑi˧}}} \textcolor{Sepia}{\selectlanguage{english}to have strength} \zh{有力量}  
 ¶ \textcolor{darkblue}{\textbf{\ipa{no˧ɻ̍˩ | hĩ˧tɕʰi˧ ʁɑ˧ ʑi˧!}}} \textcolor{Sepia}{\selectlanguage{english}Your family/clan is powerful!} \zh{你们家族很强大!}  
 ¶ \textcolor{darkblue}{\textbf{\ipa{ʁɑ˧ tʰv̩˧ (+ze˩)}}} \textcolor{Sepia}{\selectlanguage{english}to exert oneself, to make efforts} \zh{尽力}  

\lhead{\firstmark}
\rhead{\botmark}

\subsection{\hspace{-0.5cm} {\Large \textcolor{darkblue}{\textbf{\ipa{ʁɑ˥}}} \textsubscript{1}}\hspace{0.5cm}[\kern2pt{\textcolor{darkblue}{\textbf{\ipa{ʁɑ˥}}}}\kern2pt]} \hypertarget{RA\string_T1}{}
\markboth{\textcolor{darkblue}{\textbf{\ipa{ʁɑ˥}}} \textsubscript{1}}{}
\textcolor{teal}{\mytextsc{verb}} \hspace{4pt} Tone: H.
\textcolor{Sepia}{\selectlanguage{english}To invite, to call over.} \zh{请。} 
\lhead{\firstmark}
\rhead{\botmark}

\subsection{\hspace{-0.5cm} {\Large \textcolor{darkblue}{\textbf{\ipa{ʁɑ˥}}} \textsubscript{2}}\hspace{0.5cm}[\kern2pt{\textcolor{darkblue}{\textbf{\ipa{ʁɑ˥}}}}\kern2pt]} \hypertarget{RA\string_T2}{}
\markboth{\textcolor{darkblue}{\textbf{\ipa{ʁɑ˥}}} \textsubscript{2}}{}
\textcolor{teal}{\mytextsc{verb}} \hspace{4pt} Tone: H.
\textcolor{Sepia}{\selectlanguage{english}To win, to succeed.} \zh{赢。}  ¶ \textcolor{darkblue}{\textbf{\ipa{le˧-ʁɑ˥-ze˩}}} \textcolor{Sepia}{\selectlanguage{english}\mytextsc{accomp} \string_ \mytextsc{pfv}} \zh{赢了}  

\lhead{\firstmark}
\rhead{\botmark}

\subsection{\hspace{-0.5cm} {\Large \textcolor{darkblue}{\textbf{\ipa{ʁɑ˧}}} \textsubscript{1}}\hspace{0.5cm}[\kern2pt{\textcolor{darkblue}{\textbf{\ipa{ʁɑ˥}}}}\kern2pt]} \hypertarget{RA\string_M1}{}
\markboth{\textcolor{darkblue}{\textbf{\ipa{ʁɑ˧}}} \textsubscript{1}}{}
\textcolor{teal}{\mytextsc{adjective}} \hspace{4pt} Tone: M.
\textcolor{Sepia}{\selectlanguage{english}Good (of good quality).} \zh{好(质量好,品质好,脾气好)。}  ¶ \textcolor{darkblue}{\textbf{\ipa{mɤ˧-ʁɑ˧-hĩ˧ ʂe˧}}} \textcolor{Sepia}{\selectlanguage{english}bad meat, meat of poor quality} \zh{不好的肉(质量不好)}  
 ¶ \textcolor{darkblue}{\textbf{\ipa{mɤ˧-ʁɑ˧-hĩ˧ ʂe˧-kʰwɤ˧ ki˩}}} \textcolor{Sepia}{\selectlanguage{english}to give a piece of bad meat} \zh{给一块不好的肉}  
 ¶ \textcolor{darkblue}{\textbf{\ipa{pʰi˩ko˧ | mɤ˧-ʁɑ˧-ze˧!}}} \textcolor{Sepia}{\selectlanguage{english}The apples are not good anymore! (Context: in March, apples from the previous harvest are not good anymore: they are either rotten or sour.)} \zh{苹果不好了! / 苹果不新鲜了!(三月份,上一季收获的苹果已经不好吃的了,或者烂了,或者变酸)}  
 ¶ \textcolor{darkblue}{\textbf{\ipa{hĩ˧ ɖɯ˧-v̩˧ | ʁɑ˧-mɤ˧-ʑi˧-hĩ˧ ʐwɤ˧˥!}}} \textcolor{Sepia}{\selectlanguage{english}Someone is talking nonsense!} \zh{有人在乱说话!}  

\lhead{\firstmark}
\rhead{\botmark}

\subsection{\hspace{-0.5cm} {\Large \textcolor{darkblue}{\textbf{\ipa{ʁɑ˧}}} \textsubscript{2}}\hspace{0.5cm}[\kern2pt{\textcolor{darkblue}{\textbf{\ipa{ʁɑ˥}}}}\kern2pt]} \hypertarget{RA\string_M2}{}
\markboth{\textcolor{darkblue}{\textbf{\ipa{ʁɑ˧}}} \textsubscript{2}}{}
\textcolor{teal}{\mytextsc{verb}} \hspace{4pt} Tone: M.
\textit{\textcolor{Sepia}{\selectlanguage{english}archaic}} [\zh{古语}] \textcolor{Sepia}{\selectlanguage{english}To ask for forgiveness, to apologize.} \zh{道歉、请人家原谅。}  ¶ \textcolor{darkblue}{\textbf{\ipa{ʁɑ˧-ze˧!}}} \textcolor{Sepia}{\selectlanguage{english}Please accept my apologies! (To a person of high rank)} \zh{抱歉! / 请原谅!(对地位比自己高的人说)}  

\lhead{\firstmark}
\rhead{\botmark}

\subsection{\hspace{-0.5cm} {\Large \textcolor{darkblue}{\textbf{\ipa{ʁɑ˧\textsubscript{b}}}}}\hspace{0.5cm}[\kern2pt{\textcolor{darkblue}{\textbf{\ipa{ʁɑ˥}}}}\kern2pt]} \hypertarget{RA\string_Mb1}{}
\markboth{\textcolor{darkblue}{\textbf{\ipa{ʁɑ˧\textsubscript{b}}}}}{}
\textcolor{teal}{\mytextsc{verb}} \hspace{4pt} Tone: M\textsubscript{b}.
\textcolor{Sepia}{\selectlanguage{english}To stride over (an obstacle), to straddle, to go beyond.} \zh{跨(跨过小沟)。}  ¶ \textcolor{darkblue}{\textbf{\ipa{le˧-ʁɑ˧-ze˧}}} \textcolor{Sepia}{\selectlanguage{english}\mytextsc{accomp} \string_ \mytextsc{pfv}} \zh{跨过了}  

\lhead{\firstmark}
\rhead{\botmark}

\subsection{\hspace{-0.5cm} {\Large \textcolor{darkblue}{\textbf{\ipa{ʁɑ˧dzi˧}}}}\hspace{0.5cm}[\kern2pt{\textcolor{darkblue}{\textbf{\ipa{ʁɑ˧dzi˧}}}}\kern2pt]} \hypertarget{RA\string_Mdzi\string_M1}{}
\markboth{\textcolor{darkblue}{\textbf{\ipa{ʁɑ˧dzi˧}}}}{}
\textcolor{teal}{\mytextsc{noun}} \hspace{4pt} Tone: M.
\textcolor{Sepia}{\selectlanguage{english}Poplar.} \zh{杨树。}  \zh{量词}: \textcolor{darkblue}{\textbf{\ipa{dzi˩}}}  \mytextsc{clf}: \textcolor{darkblue}{\textbf{\ipa{dzi˩}}} 
\lhead{\firstmark}
\rhead{\botmark}

\subsection{\hspace{-0.5cm} {\Large \textcolor{darkblue}{\textbf{\ipa{ʁɑ˧ɭɯ\#˥}}}}\hspace{0.5cm}[\kern2pt{\textcolor{darkblue}{\textbf{\ipa{ʁɑ˧ɭɯ˧}}}}\kern2pt]} \hypertarget{RA\string_Ml\string_RM\#\string_T1}{}
\markboth{\textcolor{darkblue}{\textbf{\ipa{ʁɑ˧ɭɯ\#˥}}}}{}
\textcolor{teal}{\mytextsc{noun}} \hspace{4pt} Tone: \#H.
\textcolor{Sepia}{\selectlanguage{english}Cairn: a human-made pile of stones, used as trail marker. The stones of the cairn are not engraved.} \zh{石堆。}  ¶ \textcolor{darkblue}{\textbf{\ipa{qo˩qɑ˩-ʁɑ˥ɭɯ˩}}} \textcolor{Sepia}{\selectlanguage{english}mountain pass cairn: a cairn at a mountain pass} \zh{垭口石堆:垭口上的石堆}  
 \zh{量词}: \textcolor{darkblue}{\textbf{\ipa{ɭɯ˧}}}  \mytextsc{clf}: \textcolor{darkblue}{\textbf{\ipa{ɭɯ˧}}} 
\lhead{\firstmark}
\rhead{\botmark}

\subsection{\hspace{-0.5cm} {\Large \textcolor{darkblue}{\textbf{\ipa{ʁɑ˧pv̩˧}}}}\hspace{0.5cm}[\kern2pt{\textcolor{darkblue}{\textbf{\ipa{ʁɑ˧pv̩˧}}}}\kern2pt]} \hypertarget{RA\string_Mpv\string_=\string_M1}{}
\markboth{\textcolor{darkblue}{\textbf{\ipa{ʁɑ˧pv̩˧}}}}{}
\textcolor{teal}{\mytextsc{noun}} \hspace{4pt} Tone: M.
\textcolor{Sepia}{\selectlanguage{english}Chest.} \zh{胸脯、胸膛。}  \zh{量词}: \textcolor{darkblue}{\textbf{\ipa{ʈv̩˩}}}  \mytextsc{clf}: \textcolor{darkblue}{\textbf{\ipa{ʈv̩˩}}} 
\lhead{\firstmark}
\rhead{\botmark}

\subsection{\hspace{-0.5cm} {\Large \textcolor{darkblue}{\textbf{\ipa{ʁɑ˧pv̩˧-ɻ̃\#˥}}}}\hspace{0.5cm}[\kern2pt{\textcolor{darkblue}{\textbf{\ipa{xxxx non-correspondance entre le nombre de morphèmes et le nombre de tons de morphèmes}}}}\kern2pt]} \hypertarget{RA\string_Mpv\string_=\string_M-r£`\string_~\#\string_T1}{}
\markboth{\textcolor{darkblue}{\textbf{\ipa{ʁɑ˧pv̩˧-ɻ̃\#˥}}}}{}
\textcolor{teal}{\mytextsc{noun}} \hspace{4pt} Tone: \#H.
\textcolor{Sepia}{\selectlanguage{english}Clavicle; collarbone.} \zh{锁骨。}  \zh{量词}: \textcolor{darkblue}{\textbf{\ipa{pʰæ˧˥}}}  \mytextsc{clf}: \textcolor{darkblue}{\textbf{\ipa{pʰæ˧˥}}} 
\lhead{\firstmark}
\rhead{\botmark}

\subsection{\hspace{-0.5cm} {\Large \textcolor{darkblue}{\textbf{\ipa{ʁɑ˧pʰv̩\#˥}}}}\hspace{0.5cm}[\kern2pt{\textcolor{darkblue}{\textbf{\ipa{ʁɑ˧pʰv̩˧}}}}\kern2pt]} \hypertarget{RA\string_Mp\string_hv\string_=\#\string_T1}{}
\markboth{\textcolor{darkblue}{\textbf{\ipa{ʁɑ˧pʰv̩\#˥}}}}{}
\textcolor{teal}{\mytextsc{noun}} \hspace{4pt} Tone: \#H.
\textcolor{Sepia}{\selectlanguage{english}Salary, price paid for the work done by a worker.} \zh{工资, 工钱。}  \zh{量词}: \textcolor{darkblue}{\textbf{\ipa{kʰwɤ˥}}}  \mytextsc{clf}: \textcolor{darkblue}{\textbf{\ipa{kʰwɤ˥}}} 
\lhead{\firstmark}
\rhead{\botmark}

\subsection{\hspace{-0.5cm} {\Large \textcolor{darkblue}{\textbf{\ipa{ʁɑ˧-ʐwɤ˧˥}}}}\hspace{0.5cm}[\kern2pt{\textcolor{darkblue}{\textbf{\ipa{xxxx non-correspondance entre le nombre de morphèmes et le nombre de tons de morphèmes}}}}\kern2pt]} \hypertarget{RA\string_M-z`w7\string_M\string_T1}{}
\markboth{\textcolor{darkblue}{\textbf{\ipa{ʁɑ˧-ʐwɤ˧˥}}}}{}
\textcolor{teal}{\mytextsc{verb}} \hspace{4pt} Tone: MH\#.
\textcolor{Sepia}{\selectlanguage{english}To browbeat; to take advantage of; to pick on.} \zh{欺负。}  ¶ \textcolor{darkblue}{\textbf{\ipa{ʈʂʰɯ˧-v̩˧ | hĩ˧-ki˧ ʁɑ˧-ʐwɤ˧-ʝi˥!}}} \textcolor{Sepia}{\selectlanguage{english}(s)he is picking on someone} \zh{他欺负人、他对人发脾气}  
 ¶ \textcolor{darkblue}{\textbf{\ipa{ʁɑ˧ ʐwɤ˧-ɻ̍˥}}} \textcolor{Sepia}{\selectlanguage{english}as above} \zh{同上}  
 ¶ \textcolor{darkblue}{\textbf{\ipa{no˧ | ʁɑ˧ ʐwɤ˧-tʰɑ˧-ɻ̍˥!}}} \textcolor{Sepia}{\selectlanguage{english}Don't browbeat people!} \zh{你不要欺负人!}  

\lhead{\firstmark}
\rhead{\botmark}

\subsection{\hspace{-0.5cm} {\Large \textcolor{darkblue}{\textbf{\ipa{ʁɑ˩mi˥}}}}\hspace{0.5cm}[\kern2pt{\textcolor{darkblue}{\textbf{\ipa{ʁɑ˩mi˥}}}}\kern2pt]} \hypertarget{RA\string_Bmi\string_T1}{}
\markboth{\textcolor{darkblue}{\textbf{\ipa{ʁɑ˩mi˥}}}}{}
\textcolor{teal}{\mytextsc{verb}} \hspace{4pt} Tone: LH.
\textcolor{Sepia}{\selectlanguage{english}To apologize.} \zh{道歉。}  ¶ \textcolor{darkblue}{\textbf{\ipa{ʁɑ˩mi˥-ze˩!}}} \textcolor{Sepia}{\selectlanguage{english}Thank you!} \zh{谢谢!}  

\lhead{\firstmark}
\rhead{\botmark}

\subsection{\hspace{-0.5cm} {\Large \textcolor{darkblue}{\textbf{\ipa{ʁɑ˩ʂɯ˧}}}}\hspace{0.5cm}[\kern2pt{\textcolor{darkblue}{\textbf{\ipa{ʁɑ˩ʂɯ˥}}}}\kern2pt]} \hypertarget{RA\string_Bs`M\string_M1}{}
\markboth{\textcolor{darkblue}{\textbf{\ipa{ʁɑ˩ʂɯ˧}}}}{}
\textcolor{teal}{\mytextsc{adverb(ial)}} \hspace{4pt} Tone: LM.
\textcolor{Sepia}{\selectlanguage{english}In fact.} \zh{其实、事实上。} 
\lhead{\firstmark}
\rhead{\botmark}

\subsection{\hspace{-0.5cm} {\Large \textcolor{darkblue}{\textbf{\ipa{ʁæ˥}}}}\hspace{0.5cm}[\kern2pt{\textcolor{darkblue}{\textbf{\ipa{ʁæ˥}}}}\kern2pt]} \hypertarget{R\{\string_T1}{}
\markboth{\textcolor{darkblue}{\textbf{\ipa{ʁæ˥}}}}{}
\textcolor{teal}{\mytextsc{noun}} \hspace{4pt} Tone: \#H.
\textcolor{Sepia}{\selectlanguage{english}Neck (monosyllable).} \zh{脖子(单音节)。}  \zh{量词}: \textcolor{darkblue}{\textbf{\ipa{ɭɯ˧}}}  \mytextsc{clf}: \textcolor{darkblue}{\textbf{\ipa{ɭɯ˧}}} \textit{See:} \hyperlink{}{\textcolor{darkblue}{\textbf{\ipa{ʁæ˧ŋv̩˥}}}} 
\lhead{\firstmark}
\rhead{\botmark}

\subsection{\hspace{-0.5cm} {\Large \textcolor{darkblue}{\textbf{\ipa{ʁæ˧}}}}\hspace{0.5cm}[\kern2pt{\textcolor{darkblue}{\textbf{\ipa{ʁæ˥}}}}\kern2pt]} \hypertarget{R\{\string_M1}{}
\markboth{\textcolor{darkblue}{\textbf{\ipa{ʁæ˧}}}}{}
\textcolor{teal}{\mytextsc{adjective}} \hspace{4pt} Tone: M.
\textcolor{Sepia}{\selectlanguage{english}Rich.} \zh{富。} 
\lhead{\firstmark}
\rhead{\botmark}

\subsection{\hspace{-0.5cm} {\Large \textcolor{darkblue}{\textbf{\ipa{ʁæ˧bæ˧}}}}\hspace{0.5cm}[\kern2pt{\textcolor{darkblue}{\textbf{\ipa{ʁæ˧bæ˧}}}}\kern2pt]} \hypertarget{R\{\string_Mb\{\string_M1}{}
\markboth{\textcolor{darkblue}{\textbf{\ipa{ʁæ˧bæ˧}}}}{}
\textcolor{teal}{\mytextsc{noun}} \hspace{4pt} Tone: M.
\textcolor{Sepia}{\selectlanguage{english}Dish, plate.} \zh{盘子。}  \zh{量词}: \textcolor{darkblue}{\textbf{\ipa{ɭɯ˧}}}  \mytextsc{clf}: \textcolor{darkblue}{\textbf{\ipa{ɭɯ˧}}} 
\lhead{\firstmark}
\rhead{\botmark}

\subsection{\hspace{-0.5cm} {\Large \textcolor{darkblue}{\textbf{\ipa{ʁæ˧ɭɯ˥}}}}\hspace{0.5cm}[\kern2pt{\textcolor{darkblue}{\textbf{\ipa{ʁæ˧ɭɯ˥}}}}\kern2pt]} \hypertarget{R\{\string_Ml\string_RM\string_T1}{}
\markboth{\textcolor{darkblue}{\textbf{\ipa{ʁæ˧ɭɯ˥}}}}{}
\textcolor{teal}{\mytextsc{noun}} \hspace{4pt} Tone: H\#.
\textcolor{Sepia}{\selectlanguage{english}Fetters (wooden fetters, around the neck); yoke.} \zh{枷锁(木头做的)。}  ¶ \textcolor{darkblue}{\textbf{\ipa{ʁæ˧ɭɯ˥ | ɖɯ˧-ɭɯ˧ kʰɯ˧˥}}} \textcolor{Sepia}{\selectlanguage{english}to put fetters (on someone's neck)} \zh{套上一个枷锁(在一个人的脖子上)}  
 ¶ \textcolor{darkblue}{\textbf{\ipa{ʁæ˧ɭɯ˥ kʰɯ˩}}} \textcolor{Sepia}{\selectlanguage{english}to put fetters (on someone's neck)} \zh{套上枷锁(在一个人的脖子上)}  

\lhead{\firstmark}
\rhead{\botmark}

\subsection{\hspace{-0.5cm} {\Large \textcolor{darkblue}{\textbf{\ipa{ʁæ˧mi˧}}}}\hspace{0.5cm}[\kern2pt{\textcolor{darkblue}{\textbf{\ipa{ʁæ˧mi˧}}}}\kern2pt]} \hypertarget{R\{\string_Mmi\string_M1}{}
\markboth{\textcolor{darkblue}{\textbf{\ipa{ʁæ˧mi˧}}}}{}
\textcolor{teal}{\mytextsc{noun}} \hspace{4pt} Tone: M.
\textcolor{Sepia}{\selectlanguage{english}Sword.} \zh{剑。}  \zh{量词}: \textcolor{darkblue}{\textbf{\ipa{nɑ˧}}}  \mytextsc{clf}: \textcolor{darkblue}{\textbf{\ipa{nɑ˧}}} 
\lhead{\firstmark}
\rhead{\botmark}

\subsection{\hspace{-0.5cm} {\Large \textcolor{darkblue}{\textbf{\ipa{ʁæ˧ŋv̩˥}}}}\hspace{0.5cm}[\kern2pt{\textcolor{darkblue}{\textbf{\ipa{ʁæ˧ŋv̩˥}}}}\kern2pt]} \hypertarget{R\{\string_MNv\string_=\string_T1}{}
\markboth{\textcolor{darkblue}{\textbf{\ipa{ʁæ˧ŋv̩˥}}}}{}
\textcolor{teal}{\mytextsc{noun}} \hspace{4pt} Tone: H\#.
\textcolor{Sepia}{\selectlanguage{english}Silver-embellished collar (a precious part of the dress, with silver thread).} \zh{银衣领。}  \zh{量词}: \textcolor{darkblue}{\textbf{\ipa{ɭɯ˧}}}  \mytextsc{clf}: \textcolor{darkblue}{\textbf{\ipa{ɭɯ˧}}} 
\lhead{\firstmark}
\rhead{\botmark}

\subsection{\hspace{-0.5cm} {\Large \textcolor{darkblue}{\textbf{\ipa{ʁæ˧ɻ̍˥}}}}\hspace{0.5cm}[\kern2pt{\textcolor{darkblue}{\textbf{\ipa{ʁæ˧ɻ̍˥}}}}\kern2pt]} \hypertarget{R\{\string_Mr£`̍\string_T1}{}
\markboth{\textcolor{darkblue}{\textbf{\ipa{ʁæ˧ɻ̍˥}}}}{}
\textcolor{teal}{\mytextsc{noun}} \hspace{4pt} Tone: H\#.
\textcolor{Sepia}{\selectlanguage{english}Neck.} \zh{脖子。}  \zh{量词}: \textcolor{darkblue}{\textbf{\ipa{ɭɯ˧}}}  \mytextsc{clf}: \textcolor{darkblue}{\textbf{\ipa{ɭɯ˧}}} \textit{See:} \hyperlink{}{\textcolor{darkblue}{\textbf{\ipa{ʁæ˧ʈv̩˥}}}} 
\lhead{\firstmark}
\rhead{\botmark}

\subsection{\hspace{-0.5cm} {\Large \textcolor{darkblue}{\textbf{\ipa{ʁæ˧tɑ˩}}}}\hspace{0.5cm}[\kern2pt{\textcolor{darkblue}{\textbf{\ipa{ʁæ˧tɑ˩}}}}\kern2pt]} \hypertarget{R\{\string_MtA\string_B1}{}
\markboth{\textcolor{darkblue}{\textbf{\ipa{ʁæ˧tɑ˩}}}}{}
\textcolor{teal}{\mytextsc{noun}} \hspace{4pt} Tone: L\#.
\textcolor{Sepia}{\selectlanguage{english}Withers: part of the ox's body on which the yoke rests.} \zh{肩隆。}  ¶ \textcolor{darkblue}{\textbf{\ipa{ʝi˧-ʁæ˧tɑ˥}}} \textcolor{Sepia}{\selectlanguage{english}ox's withers} \zh{牛肩隆}  
 ¶ \textcolor{darkblue}{\textbf{\ipa{ʁæ˧tɑ˩ tʰv̩˩-ɭɯ˩}}} \textcolor{Sepia}{\selectlanguage{english}\mytextsc{n}+\mytextsc{dem}+\mytextsc{clf}} \zh{这只肩隆}  
 \zh{量词}: \textcolor{darkblue}{\textbf{\ipa{ɭɯ˧}}}  \mytextsc{clf}: \textcolor{darkblue}{\textbf{\ipa{ɭɯ˧}}} 
\lhead{\firstmark}
\rhead{\botmark}

\subsection{\hspace{-0.5cm} {\Large \textcolor{darkblue}{\textbf{\ipa{ʁæ˧ʈv̩˥}}}}\hspace{0.5cm}[\kern2pt{\textcolor{darkblue}{\textbf{\ipa{ʁæ˧ʈv̩˥}}}}\kern2pt]} \hypertarget{R\{\string_Mt`v\string_=\string_T1}{}
\markboth{\textcolor{darkblue}{\textbf{\ipa{ʁæ˧ʈv̩˥}}}}{}
\textcolor{teal}{\mytextsc{noun}} \hspace{4pt} Tone: H\#.
\textcolor{Sepia}{\selectlanguage{english}Neck.} \zh{脖子。}  \zh{量词}: \textcolor{darkblue}{\textbf{\ipa{ɭɯ˧}}}  \mytextsc{clf}: \textcolor{darkblue}{\textbf{\ipa{ɭɯ˧}}} \textit{See:} \hyperlink{}{\textcolor{darkblue}{\textbf{\ipa{ʁæ˧ɻ̍˥}}}} 
\lhead{\firstmark}
\rhead{\botmark}

\subsection{\hspace{-0.5cm} {\Large \textcolor{darkblue}{\textbf{\ipa{ʁæ˧zo\#˥}}}}\hspace{0.5cm}[\kern2pt{\textcolor{darkblue}{\textbf{\ipa{ʁæ˧zo˧}}}}\kern2pt]} \hypertarget{R\{\string_Mzo\#\string_T1}{}
\markboth{\textcolor{darkblue}{\textbf{\ipa{ʁæ˧zo\#˥}}}}{}
\textcolor{teal}{\mytextsc{noun}} \hspace{4pt} Tone: \#H.
\textcolor{Sepia}{\selectlanguage{english}Short sword.} \zh{短剑。} 
\lhead{\firstmark}
\rhead{\botmark}

\subsection{\hspace{-0.5cm} {\Large \textcolor{darkblue}{\textbf{\ipa{ʁæ˧ʑi˧}}}}\hspace{0.5cm}[\kern2pt{\textcolor{darkblue}{\textbf{\ipa{ʁæ˧ʑi˧}}}}\kern2pt]} \hypertarget{R\{\string_Mz£i\string_M1}{}
\markboth{\textcolor{darkblue}{\textbf{\ipa{ʁæ˧ʑi˧}}}}{}
\textcolor{teal}{\mytextsc{verb}} \hspace{4pt} Tone: M.
\textcolor{Sepia}{\selectlanguage{english}To mind something; to take into account; to take into consideration; to care about.} \zh{考虑。}  ¶ \textcolor{darkblue}{\textbf{\ipa{hĩ˧ | qʰɑ˧-kv̩˧ dʑo˧˥ | mɤ˧-ʁæ˧ʑi˧, | njɤ˧-ɳɯ˧ qʰæ˧˥! |}}} \textcolor{Sepia}{\selectlanguage{english}No matter how many people (guests) there are, I (go to participate and) help! (Context: the consultant explains how, following Na traditions, she volunteers her time to help on important occasions, such as funerals, to help other families.)} \zh{无论有多少个人,我都会去帮助!(情景:合作人描写她在永宁有大事时怎么去帮其它家庭的忙,不考虑活多么累,只考虑怎么能给予帮助)}  
 ¶ \textcolor{darkblue}{\textbf{\ipa{no˧ | mɤ˧-ʁæ˧ʑi˧!}}} \textcolor{Sepia}{\selectlanguage{english}Leave me alone! / Leave me in peace! / Mind your own business!} \zh{别管我了! / 请让我安静! / 请不要打扰我了!}  

\lhead{\firstmark}
\rhead{\botmark}

\subsection{\hspace{-0.5cm} {\Large \textcolor{darkblue}{\textbf{\ipa{ʁæ˩\textsubscript{a}}}} \textsubscript{1}}\hspace{0.5cm}[\kern2pt{\textcolor{darkblue}{\textbf{\ipa{ʁæ˩˥}}}}\kern2pt]} \hypertarget{R\{\string_Ba1}{}
\markboth{\textcolor{darkblue}{\textbf{\ipa{ʁæ˩\textsubscript{a}}}} \textsubscript{1}}{}
\textcolor{teal}{\mytextsc{verb}} \hspace{4pt} Tone: L\textsubscript{a}.
\textcolor{Sepia}{\selectlanguage{english}To fall apart, to scatter, to melt (e.g. clods of dry earth melting in water when a field is irrigated after ploughing).} \zh{散、散开,化,溶化(一块土在水里面散开)。}  ¶ \textcolor{darkblue}{\textbf{\ipa{le˧-ʁæ˩-ze˩}}} \textcolor{Sepia}{\selectlanguage{english}\mytextsc{accomp} \string_ \mytextsc{pfv}} \zh{\mytextsc{accomp} \string_ \mytextsc{pfv}}  
 ¶ \textcolor{darkblue}{\textbf{\ipa{le˧-ʁæ˧\textasciitilde{}ʁæ˥ (-ze˩ / -bi˩)}}} \textcolor{Sepia}{\selectlanguage{english}\mytextsc{red}} \zh{\mytextsc{重叠}}  
 ¶ \textcolor{darkblue}{\textbf{\ipa{ɖɯ˧-kʰwɤ˧ ʁæ˥}}} \textcolor{Sepia}{\selectlanguage{english}a lump (of earth) melts} \zh{一块(土)散开}  
 ¶ \textcolor{darkblue}{\textbf{\ipa{ʈʂe˧ʈv̩˥ | le˧-ʁæ˩-ze˩}}} \textcolor{Sepia}{\selectlanguage{english}Clods of earth fall apart (after ploughing, the fields are irrigated; clods of earth melt into the water)} \zh{土块散开在了(耕田后灌溉,土块散在水里)}  

\lhead{\firstmark}
\rhead{\botmark}

\subsection{\hspace{-0.5cm} {\Large \textcolor{darkblue}{\textbf{\ipa{ʁæ˩\textsubscript{a}}}} \textsubscript{2}}\hspace{0.5cm}[\kern2pt{\textcolor{darkblue}{\textbf{\ipa{ʁæ˩˥}}}}\kern2pt]} \hypertarget{R\{\string_Ba2}{}
\markboth{\textcolor{darkblue}{\textbf{\ipa{ʁæ˩\textsubscript{a}}}} \textsubscript{2}}{}
\textcolor{teal}{\mytextsc{adjective}} \hspace{4pt} Tone: L\textsubscript{a}.
\textcolor{Sepia}{\selectlanguage{english}Drunk.} \zh{醉。}  ¶ \textcolor{darkblue}{\textbf{\ipa{ʐɯ˧ ʁæ˩(-ze˩)}}} \textcolor{Sepia}{\selectlanguage{english}drunk} \zh{醉酒}  

\lhead{\firstmark}
\rhead{\botmark}

\subsection{\hspace{-0.5cm} {\Large \textcolor{darkblue}{\textbf{\ipa{ʁæ˩\textsubscript{a}}}} \textsubscript{3}}\hspace{0.5cm}[\kern2pt{\textcolor{darkblue}{\textbf{\ipa{ʁæ˩˥}}}}\kern2pt]} \hypertarget{R\{\string_Ba3}{}
\markboth{\textcolor{darkblue}{\textbf{\ipa{ʁæ˩\textsubscript{a}}}} \textsubscript{3}}{}
\textcolor{teal}{\mytextsc{adjective}} \hspace{4pt} Tone: L\textsubscript{a}.
\textcolor{Sepia}{\selectlanguage{english}Appropriate; auspitious.} \zh{合适,吉利。}  ¶ \textcolor{darkblue}{\textbf{\ipa{ʁæ˧ mɤ˧-ʑi˧}}} \textcolor{Sepia}{\selectlanguage{english}not propicious / not favourable} \zh{不吉利、不合适}  
 ¶ \textcolor{darkblue}{\textbf{\ipa{ʁæ˧ mɤ˧-ʑi˧, | ʝi˧ mɤ˧-tʰɑ˩! / ʝi˧-mɤ˧-ɖo˧!}}} \textcolor{Sepia}{\selectlanguage{english}It is not appropriate / the situation is not propitious; it must not / should not be done! (A phrase to caution others against doing something)} \zh{不吉利 / 不合适(的事情),不能做!/ 不要做!(警告)}  
 ¶ \textcolor{darkblue}{\textbf{\ipa{ʁæ˧ mɤ˧-ʑi˧, | ʐwɤ˩ mɤ˩-tʰɑ˥! / ʁæ˧ mɤ˧-ʑi˧, | ʐwɤ˩ mɤ˩-ɖo˩˥!}}} \textcolor{Sepia}{\selectlanguage{english}It's not appropriate; one must not talk about it! / One should not talk nonsense! (A phrase to caution others against being carelessly talkative)} \zh{不合适(的话),不能说! / 不合适(的话),不要说!(警告)}  
 ¶ \textcolor{darkblue}{\textbf{\ipa{ʁæ˧-mɤ˧-ʑi˧, | tɕi˩-mɤ˩-ɖo˩˥!}}} \textcolor{Sepia}{\selectlanguage{english}(You) must not transcribe the bad ones! / You must not transcribe the messy ones! (Context: the investigator was explaining his wish to choose, among the wealth of recorded narratives, those that are the most interesting and successful, to do a transcription and complete translation and annotation. By her answer, the consultant indicates her approval, at the same time as she shows her understanding of the idea: any materials that may be inappropriate in any way should be left out, and not put to writing.)} \zh{乱七八糟的,不要记录! / 不好的,不要记录!(情景:选择一个故事来做记音翻译等。合作人提出,要考虑好记录哪些、选择好的资料,不能什么都记录。)}  
 ¶ \textcolor{darkblue}{\textbf{\ipa{ʈʂʰɯ˧ | lo˧ | ʁæ˧-mɤ˧-ʑi˧ ʝi˧!}}} \textcolor{Sepia}{\selectlanguage{english}He does a bad job of it! / He makes a mess of his work!} \zh{他工作做得乱七八糟!}  

\lhead{\firstmark}
\rhead{\botmark}

\subsection{\hspace{-0.5cm} {\Large \textcolor{darkblue}{\textbf{\ipa{ʁæ˩\textsubscript{a}}}} \textsubscript{4}}\hspace{0.5cm}[\kern2pt{\textcolor{darkblue}{\textbf{\ipa{ʁæ˩˥}}}}\kern2pt]} \hypertarget{R\{\string_Ba4}{}
\markboth{\textcolor{darkblue}{\textbf{\ipa{ʁæ˩\textsubscript{a}}}} \textsubscript{4}}{}
\textcolor{teal}{\mytextsc{adjective}} \hspace{4pt} Tone: L\textsubscript{a}.
\textcolor{Sepia}{\selectlanguage{english}Ugly.} \zh{丑陋。} \textit{See:} \hyperlink{}{\textcolor{darkblue}{\textbf{\ipa{ɖʐv̩˩\textsubscript{a}}}} \textsubscript{1}} 
\lhead{\firstmark}
\rhead{\botmark}

\subsection{\hspace{-0.5cm} {\Large \textcolor{darkblue}{\textbf{\ipa{ʁæ˧˥}}}}\hspace{0.5cm}[\kern2pt{\textcolor{darkblue}{\textbf{\ipa{ʁæ˧˥}}}}\kern2pt]} \hypertarget{R\{\string_M\string_T1}{}
\markboth{\textcolor{darkblue}{\textbf{\ipa{ʁæ˧˥}}}}{}
\textcolor{teal}{\mytextsc{adjective}} \hspace{4pt} Tone: MH.
\textcolor{Sepia}{\selectlanguage{english}Nauseous, disgusting.} \zh{不好吃,恶心。} 
\lhead{\firstmark}
\rhead{\botmark}

\subsection{\hspace{-0.5cm} {\Large \textcolor{darkblue}{\textbf{\ipa{ʁæ˩˥}}}}\hspace{0.5cm}[\kern2pt{\textcolor{darkblue}{\textbf{\ipa{ʁæ˩˥}}}}\kern2pt]} \hypertarget{R\{\string_B\string_T1}{}
\markboth{\textcolor{darkblue}{\textbf{\ipa{ʁæ˩˥}}}}{}
\textcolor{teal}{\mytextsc{noun}} \hspace{4pt} Tone: LH.
\textcolor{Sepia}{\selectlanguage{english}Sap.} \zh{树液。}  ¶ \textcolor{darkblue}{\textbf{\ipa{tʰo˩ʁæ˩˥}}} \textcolor{Sepia}{\selectlanguage{english}same meaning} \zh{同上}  

\lhead{\firstmark}
\rhead{\botmark}

\subsection{\hspace{-0.5cm} {\Large \textcolor{darkblue}{\textbf{\ipa{ʁo˥}}}}\hspace{0.5cm}[\kern2pt{\textcolor{darkblue}{\textbf{\ipa{ʁo˥}}}}\kern2pt]} \hypertarget{Ro\string_T1}{}
\markboth{\textcolor{darkblue}{\textbf{\ipa{ʁo˥}}}}{}
\textcolor{teal}{\mytextsc{noun}} \hspace{4pt} Tone: \#H.
\ding{202} \textcolor{Sepia}{\selectlanguage{english}Head (monosyllable).} \zh{头(单音节)。}  \zh{量词}: \textcolor{darkblue}{\textbf{\ipa{ɭɯ˧}}} \ding{203} \textcolor{Sepia}{\selectlanguage{english}Beginning.} \zh{开头。}  ¶ \textcolor{darkblue}{\textbf{\ipa{ɬi˧-ʁo\#˥}}} \textcolor{Sepia}{\selectlanguage{english}the beginning of the month, the first days of the month} \zh{月初}  
 ¶ \textcolor{darkblue}{\textbf{\ipa{kʰv̩˧-ʁo˥\$}}} \textcolor{Sepia}{\selectlanguage{english}the beginning of the year} \zh{年初}  
 ¶ \textcolor{darkblue}{\textbf{\ipa{*ɲi˧-ʁo˩}}} \textcolor{Sepia}{\selectlanguage{english}*the beginning of the day} \zh{*天初}  
 \mytextsc{clf}: \textcolor{darkblue}{\textbf{\ipa{ɭɯ˧}}} 
\lhead{\firstmark}
\rhead{\botmark}

\subsection{\hspace{-0.5cm} {\Large \textcolor{darkblue}{\textbf{\ipa{ʁo˥-ʐv̩˩}}}}\hspace{0.5cm}[\kern2pt{\textcolor{darkblue}{\textbf{\ipa{xxxx non-correspondance entre le nombre de morphèmes et le nombre de tons de morphèmes}}}}\kern2pt]} \hypertarget{Ro\string_T-z`v\string_=\string_B1}{}
\markboth{\textcolor{darkblue}{\textbf{\ipa{ʁo˥-ʐv̩˩}}}}{}
\textcolor{teal}{\mytextsc{verb}} \hspace{4pt} Tone: .
\textcolor{Sepia}{\selectlanguage{english}Bless and protect.} \zh{保佑。}  ¶ \textcolor{darkblue}{\textbf{\ipa{mɤ˧-ʁo˥ʐv̩˩}}} \textcolor{Sepia}{\selectlanguage{english}\mytextsc{neg}} \zh{\mytextsc{neg}}  
 ¶ \textcolor{darkblue}{\textbf{\ipa{gɤ˧lɑ˧ | ɖɯ˧-ʁo˥ʐv̩˩-ɻ̍˩!}}} \textcolor{Sepia}{\selectlanguage{english}May the gods bless (you/us)!} \zh{菩萨保佑!}  

\lhead{\firstmark}
\rhead{\botmark}

\subsection{\hspace{-0.5cm} {\Large \textcolor{darkblue}{\textbf{\ipa{ʁo˧}}} \textsubscript{1}}\hspace{0.5cm}[\kern2pt{\textcolor{darkblue}{\textbf{\ipa{ʁo˥}}}}\kern2pt]} \hypertarget{Ro\string_M1}{}
\markboth{\textcolor{darkblue}{\textbf{\ipa{ʁo˧}}} \textsubscript{1}}{}
\textcolor{teal}{\mytextsc{verb}} \hspace{4pt} Tone: M intrans.
\textcolor{Sepia}{\selectlanguage{english}To lay eggs.} \zh{下蛋。}  ¶ \textcolor{darkblue}{\textbf{\ipa{æ˩ ʁo˥}}} \textcolor{Sepia}{\selectlanguage{english}to lay eggs} \zh{下蛋}  
 ¶ \textcolor{darkblue}{\textbf{\ipa{æ˩mi˧ tʰi˧-ʁo˧-dʑo˧!}}} \textcolor{Sepia}{\selectlanguage{english}The hen is laying eggs!} \zh{母鸡在下蛋!}  
 ¶ \textcolor{darkblue}{\textbf{\ipa{æ˩mi˧ | æ˩ ʁo˧-ze˩!}}} \textcolor{Sepia}{\selectlanguage{english}The hen has laid eggs!} \zh{母鸡下蛋了!}  

\lhead{\firstmark}
\rhead{\botmark}

\subsection{\hspace{-0.5cm} {\Large \textcolor{darkblue}{\textbf{\ipa{ʁo˧}}} \textsubscript{2}}\hspace{0.5cm}[\kern2pt{\textcolor{darkblue}{\textbf{\ipa{ʁo˥}}}}\kern2pt]} \hypertarget{Ro\string_M2}{}
\markboth{\textcolor{darkblue}{\textbf{\ipa{ʁo˧}}} \textsubscript{2}}{}
\textcolor{teal}{\mytextsc{verb}} \hspace{4pt} Tone: M intrans.
\textcolor{Sepia}{\selectlanguage{english}To be able to, to manage to.} \zh{能……、有能力做。}  ¶ \textcolor{darkblue}{\textbf{\ipa{njɤ˧ | tɕi˩-mɤ˩-ʁo˩˥!}}} \textcolor{Sepia}{\selectlanguage{english}I can't write! / I am not able to write! (Said by someone who has not learnt to write)} \zh{我写不出来! / 我不会写!}  
 ¶ \textcolor{darkblue}{\textbf{\ipa{njɤ˧ | tɕi˩-ʁo˩˥!}}} \textcolor{Sepia}{\selectlanguage{english}I can write! / I am able to write! / I know how to write!} \zh{我会写! / 我写得出来!}  

\lhead{\firstmark}
\rhead{\botmark}

\subsection{\hspace{-0.5cm} {\Large \textcolor{darkblue}{\textbf{\ipa{ʁo˧bv̩˧}}}}\hspace{0.5cm}[\kern2pt{\textcolor{darkblue}{\textbf{\ipa{ʁo˩bv̩˩˥}}}}\kern2pt]} \hypertarget{Ro\string_Mbv\string_=\string_M1}{}
\markboth{\textcolor{darkblue}{\textbf{\ipa{ʁo˧bv̩˧}}}}{}
\textcolor{teal}{\mytextsc{noun}} \hspace{4pt} Tone: M.
\textcolor{Sepia}{\selectlanguage{english}Sprout, bud.} \zh{树的萌芽、新发出来的叶子。}  ¶ \textcolor{darkblue}{\textbf{\ipa{tʰo˧-ʁo˧bv˥}}} \textcolor{Sepia}{\selectlanguage{english}bud of pine tree} \zh{小松树尖}  
 ¶ \textcolor{darkblue}{\textbf{\ipa{tʰo˩ʂv˩-ʁo˥bv˩}}} \textcolor{Sepia}{\selectlanguage{english}bud of pine tree} \zh{小松树尖}  
 \zh{量词}: \textcolor{darkblue}{\textbf{\ipa{kʰwɤ˥}}}  \mytextsc{clf}: \textcolor{darkblue}{\textbf{\ipa{kʰwɤ˥}}} 
\lhead{\firstmark}
\rhead{\botmark}

\subsection{\hspace{-0.5cm} {\Large \textcolor{darkblue}{\textbf{\ipa{ʁo˧dɑ˧}}}}\hspace{0.5cm}[\kern2pt{\textcolor{darkblue}{\textbf{\ipa{ʁo˧dɑ˧}}}}\kern2pt]} \hypertarget{Ro\string_MdA\string_M1}{}
\markboth{\textcolor{darkblue}{\textbf{\ipa{ʁo˧dɑ˧}}}}{}
\textcolor{teal}{\mytextsc{adverb(ial)}} \hspace{4pt} Tone: M.
\textcolor{Sepia}{\selectlanguage{english}In front of.} \zh{前面,之前。}  ¶ \textcolor{darkblue}{\textbf{\ipa{ʂɯ˧-kʰv̩˧-ʁo˧dɑ˧}}} \textcolor{Sepia}{\selectlanguage{english}seven years ago} \zh{七年前}  
 ¶ \textcolor{darkblue}{\textbf{\ipa{ʁo˧dɑ˧ ɖɯ˧-so˩ ɲi˩}}} \textcolor{Sepia}{\selectlanguage{english}the past few days} \zh{前几天}  

\lhead{\firstmark}
\rhead{\botmark}

\subsection{\hspace{-0.5cm} {\Large \textcolor{darkblue}{\textbf{\ipa{ʁo˧do˧}}} \textsubscript{1}}\hspace{0.5cm}[\kern2pt{\textcolor{darkblue}{\textbf{\ipa{ʁo˩do˥}}}}\kern2pt]} \hypertarget{Ro\string_Mdo\string_M1}{}
\markboth{\textcolor{darkblue}{\textbf{\ipa{ʁo˧do˧}}} \textsubscript{1}}{}
\textcolor{teal}{\mytextsc{noun}} \hspace{4pt} Tone: M.
\textcolor{Sepia}{\selectlanguage{english}Walnut.} \zh{核桃。}  ¶ \textcolor{darkblue}{\textbf{\ipa{ʁo˧do˧ qʰwæ˧˥}}} \textcolor{Sepia}{\selectlanguage{english}to crack walnuts} \zh{开核桃}  
 ¶ \textcolor{darkblue}{\textbf{\ipa{ʁo˧do˧ ʐwæ˧}}} \textcolor{Sepia}{\selectlanguage{english}to weigh walnuts} \zh{称核桃}  
 \zh{量词}: \textcolor{darkblue}{\textbf{\ipa{ɭɯ˧}}}  \mytextsc{clf}: \textcolor{darkblue}{\textbf{\ipa{ɭɯ˧}}} 
\lhead{\firstmark}
\rhead{\botmark}

\subsection{\hspace{-0.5cm} {\Large \textcolor{darkblue}{\textbf{\ipa{ʁo˧do˧}}} \textsubscript{2}}\hspace{0.5cm}[\kern2pt{\textcolor{darkblue}{\textbf{\ipa{ʁo˧do˧}}}}\kern2pt]} \hypertarget{Ro\string_Mdo\string_M2}{}
\markboth{\textcolor{darkblue}{\textbf{\ipa{ʁo˧do˧}}} \textsubscript{2}}{}
\textcolor{teal}{\mytextsc{noun}} \hspace{4pt} Tone: M.
\textcolor{Sepia}{\selectlanguage{english}Interests.} \zh{利息。}  \zh{量词}: \textcolor{darkblue}{\textbf{\ipa{kʰwɤ˥}}}  \mytextsc{clf}: \textcolor{darkblue}{\textbf{\ipa{kʰwɤ˥}}} 
\lhead{\firstmark}
\rhead{\botmark}

\subsection{\hspace{-0.5cm} {\Large \textcolor{darkblue}{\textbf{\ipa{ʁo˧dzi˥}}}}\hspace{0.5cm}[\kern2pt{\textcolor{darkblue}{\textbf{\ipa{ʁo˧dzi˥}}}}\kern2pt]} \hypertarget{Ro\string_Mdzi\string_T1}{}
\markboth{\textcolor{darkblue}{\textbf{\ipa{ʁo˧dzi˥}}}}{}
\textcolor{teal}{\mytextsc{verb}} \hspace{4pt} Tone: H\#.
\textcolor{Sepia}{\selectlanguage{english}To collide, to run into.} \zh{碰撞。}  ¶ \textcolor{darkblue}{\textbf{\ipa{le˧-ʁo˧dzi˥}}} \textcolor{Sepia}{\selectlanguage{english}\mytextsc{accomp}} \zh{\mytextsc{accomp}}  
 ¶ \textcolor{darkblue}{\textbf{\ipa{hĩ˧ | tʰi˧-ʁo˧dzi˥ tsʰɯ˩(-ze˩)}}} \textcolor{Sepia}{\selectlanguage{english}People have ran into one another} \zh{人们(互相)碰撞}  

\lhead{\firstmark}
\rhead{\botmark}

\subsection{\hspace{-0.5cm} {\Large \textcolor{darkblue}{\textbf{\ipa{ʁo˧dzi˩}}}}\hspace{0.5cm}[\kern2pt{\textcolor{darkblue}{\textbf{\ipa{ʁo˧dzi˩}}}}\kern2pt]} \hypertarget{Ro\string_Mdzi\string_B1}{}
\markboth{\textcolor{darkblue}{\textbf{\ipa{ʁo˧dzi˩}}}}{}
\textcolor{teal}{\mytextsc{noun}} \hspace{4pt} Tone: L\#.
\textcolor{Sepia}{\selectlanguage{english}Tibetan.} \zh{藏族。}  \zh{量词}: \textcolor{darkblue}{\textbf{\ipa{v̩˧}}}  \mytextsc{clf}: \textcolor{darkblue}{\textbf{\ipa{v̩˧}}} 
\lhead{\firstmark}
\rhead{\botmark}

\subsection{\hspace{-0.5cm} {\Large \textcolor{darkblue}{\textbf{\ipa{ʁo˧dzi˩-di˩}}}}\hspace{0.5cm}[\kern2pt{\textcolor{darkblue}{\textbf{\ipa{ʁo˧dzi˩di˧}}}}\kern2pt]} \hypertarget{Ro\string_Mdzi\string_B-di\string_B1}{}
\markboth{\textcolor{darkblue}{\textbf{\ipa{ʁo˧dzi˩-di˩}}}}{}
\textcolor{teal}{\mytextsc{noun}} \hspace{4pt} Tone: L\#-.
\textcolor{Sepia}{\selectlanguage{english}Tibet (literally: 'the Tibetan land').} \zh{西藏。} 
\lhead{\firstmark}
\rhead{\botmark}

\subsection{\hspace{-0.5cm} {\Large \textcolor{darkblue}{\textbf{\ipa{ʁo˧dzi˩-tʰæ˩ɻæ˩}}}}\hspace{0.5cm}[\kern2pt{\textcolor{darkblue}{\textbf{\ipa{ʁo˧dzi˩tʰæ˧ɻæ˧}}}}\kern2pt]} \hypertarget{Ro\string_Mdzi\string_B-t\string_h\{\string_Br£`\{\string_B1}{}
\markboth{\textcolor{darkblue}{\textbf{\ipa{ʁo˧dzi˩-tʰæ˩ɻæ˩}}}}{}
\textcolor{teal}{\mytextsc{noun}} \hspace{4pt} Tone: L\#-.
\textcolor{Sepia}{\selectlanguage{english}Flag, banner, pennant (literally: Tibetan writings).} \zh{旗子。}  \zh{量词}: \textcolor{darkblue}{\textbf{\ipa{pʰæ˧˥}}}  \mytextsc{clf}: \textcolor{darkblue}{\textbf{\ipa{pʰæ˧˥}}} 
\lhead{\firstmark}
\rhead{\botmark}

\subsection{\hspace{-0.5cm} {\Large \textcolor{darkblue}{\textbf{\ipa{ʁo˧dʑɯ˧}}}}\hspace{0.5cm}[\kern2pt{\textcolor{darkblue}{\textbf{\ipa{ʁo˧dʑɯ˧}}}}\kern2pt]} \hypertarget{Ro\string_Mdz£M\string_M1}{}
\markboth{\textcolor{darkblue}{\textbf{\ipa{ʁo˧dʑɯ˧}}}}{}
\textcolor{teal}{\mytextsc{noun}} \hspace{4pt} Tone: M.
\textcolor{Sepia}{\selectlanguage{english}Bridle; halter.} \zh{马笼头。}  ¶ \textcolor{darkblue}{\textbf{\ipa{ʐwæ˧-ʁo˧dʑɯ˥ (ʈʂʰɯ˧ | ʐwæ˧-ʁo˧dʑɯ˥ ɲi˩)}}} \textcolor{Sepia}{\selectlanguage{english}horse's halter} \zh{马笼头}  
 \zh{量词}: \textcolor{darkblue}{\textbf{\ipa{nɑ˧}}} \textcolor{darkblue}{\textbf{\ipa{pɤ˩}}}  \mytextsc{clf}: \textcolor{darkblue}{\textbf{\ipa{nɑ˧}}} \textcolor{darkblue}{\textbf{\ipa{pɤ˩}}} 
\lhead{\firstmark}
\rhead{\botmark}

\subsection{\hspace{-0.5cm} {\Large \textcolor{darkblue}{\textbf{\ipa{ʁo˧ɖɯ˧˥}}}}\hspace{0.5cm}[\kern2pt{\textcolor{darkblue}{\textbf{\ipa{ʁo˧ɖɯ˧}}}}\kern2pt]} \hypertarget{Ro\string_Md`M\string_M\string_T1}{}
\markboth{\textcolor{darkblue}{\textbf{\ipa{ʁo˧ɖɯ˧˥}}}}{}
\textcolor{teal}{\mytextsc{noun}} \hspace{4pt} Tone: MH\#.
\textcolor{Sepia}{\selectlanguage{english}Tadpole.} \zh{蝌蚪。} 
\lhead{\firstmark}
\rhead{\botmark}

\subsection{\hspace{-0.5cm} {\Large \textcolor{darkblue}{\textbf{\ipa{ʁo˧gv̩\#˥}}}}\hspace{0.5cm}[\kern2pt{\textcolor{darkblue}{\textbf{\ipa{ʁo˧gv̩˧}}}}\kern2pt]} \hypertarget{Ro\string_Mgv\string_=\#\string_T1}{}
\markboth{\textcolor{darkblue}{\textbf{\ipa{ʁo˧gv̩\#˥}}}}{}
\textcolor{teal}{\mytextsc{noun}} \hspace{4pt} Tone: \#H.
\textcolor{Sepia}{\selectlanguage{english}Pillow.} \zh{枕头。}  \zh{量词}: \textcolor{darkblue}{\textbf{\ipa{ɭɯ˧}}}  \mytextsc{clf}: \textcolor{darkblue}{\textbf{\ipa{ɭɯ˧}}} 
\lhead{\firstmark}
\rhead{\botmark}

\subsection{\hspace{-0.5cm} {\Large \textcolor{darkblue}{\textbf{\ipa{ʁo˧hṽ˧˥}}}}\hspace{0.5cm}[\kern2pt{\textcolor{darkblue}{\textbf{\ipa{ʁo˧hṽ˧˥}}}}\kern2pt]} \hypertarget{Ro\string_Mhv\string_~\string_M\string_T1}{}
\markboth{\textcolor{darkblue}{\textbf{\ipa{ʁo˧hṽ˧˥}}}}{}
\textcolor{teal}{\mytextsc{noun}} \hspace{4pt} Tone: MH\#.
\textcolor{Sepia}{\selectlanguage{english}Hair (of the head).} \zh{头发。}  \zh{量词}: \textcolor{darkblue}{\textbf{\ipa{kʰɯ˩}}}  \mytextsc{clf}: \textcolor{darkblue}{\textbf{\ipa{kʰɯ˩}}} 
\lhead{\firstmark}
\rhead{\botmark}

\subsection{\hspace{-0.5cm} {\Large \textcolor{darkblue}{\textbf{\ipa{ʁo˧ʝi˧}}}}\hspace{0.5cm}[\kern2pt{\textcolor{darkblue}{\textbf{\ipa{ʁo˧ʝi˧}}}}\kern2pt]} \hypertarget{Ro\string_Mj££i\string_M1}{}
\markboth{\textcolor{darkblue}{\textbf{\ipa{ʁo˧ʝi˧}}}}{}
\textcolor{teal}{\mytextsc{adverb(ial)}} \hspace{4pt} Tone: M.
\textcolor{Sepia}{\selectlanguage{english}The year after next.} \zh{后年。}  ¶ \textcolor{darkblue}{\textbf{\ipa{ʁo˧ʝi˧ ɖɯ˧-kʰv̩˧˥}}} \textcolor{Sepia}{\selectlanguage{english}the year after next} \zh{后年}  

\lhead{\firstmark}
\rhead{\botmark}

\subsection{\hspace{-0.5cm} {\Large \textcolor{darkblue}{\textbf{\ipa{ʁo˧kɤ˩}}}}\hspace{0.5cm}[\kern2pt{\textcolor{darkblue}{\textbf{\ipa{ʁo˧kɤ˩}}}}\kern2pt]} \hypertarget{Ro\string_Mk7\string_B1}{}
\markboth{\textcolor{darkblue}{\textbf{\ipa{ʁo˧kɤ˩}}}}{}
\textcolor{teal}{\mytextsc{noun}} \hspace{4pt} Tone: L\#.
\textcolor{Sepia}{\selectlanguage{english}Woven headdress for women who already have children; for young women who do not yet have children, this same item is called \textcolor{darkblue}{\textbf{\ipa{/ʁo˧ni˥/}}}.} \zh{用来将长辫缠成盘头的黑色丝头饰(已经有孩子的女人戴的)。还没有孩子的青年女人,也戴这种头饰,但称作\textcolor{darkblue}{\textbf{\ipa{/ʁo˧ni˥/}}})。}  \zh{量词}: \textcolor{darkblue}{\textbf{\ipa{kɤ˧˥}}}  \mytextsc{clf}: \textcolor{darkblue}{\textbf{\ipa{kɤ˧˥}}} 
\lhead{\firstmark}
\rhead{\botmark}

\subsection{\hspace{-0.5cm} {\Large \textcolor{darkblue}{\textbf{\ipa{ʁo˧lv̩˧}}}}\hspace{0.5cm}[\kern2pt{\textcolor{darkblue}{\textbf{\ipa{ʁo˧lv̩˧}}}}\kern2pt]} \hypertarget{Ro\string_Mlv\string_=\string_M1}{}
\markboth{\textcolor{darkblue}{\textbf{\ipa{ʁo˧lv̩˧}}}}{}
\textcolor{teal}{\mytextsc{verb}} \hspace{4pt} Tone: M.
\textcolor{Sepia}{\selectlanguage{english}To lose one's way, to become lost.} \zh{迷路。}  ¶ \textcolor{darkblue}{\textbf{\ipa{le˧-ʁo˧lv̩˧}}} \textcolor{Sepia}{\selectlanguage{english}\mytextsc{accomp}} \zh{\mytextsc{accomp}}  

\lhead{\firstmark}
\rhead{\botmark}

\subsection{\hspace{-0.5cm} {\Large \textcolor{darkblue}{\textbf{\ipa{ʁo˧ɬi˥}}}}\hspace{0.5cm}[\kern2pt{\textcolor{darkblue}{\textbf{\ipa{ʁo˧ɬi˥}}}}\kern2pt]} \hypertarget{Ro\string_MKi\string_T1}{}
\markboth{\textcolor{darkblue}{\textbf{\ipa{ʁo˧ɬi˥}}}}{}
\textcolor{teal}{\mytextsc{noun}} \hspace{4pt} Tone: H\#.
\textcolor{Sepia}{\selectlanguage{english}Large needle with which animal hide can be sewn.} \zh{大粗针,用来缝琵琶肉。}  ¶ \textcolor{darkblue}{\textbf{\ipa{ʁo˧ɬi˥, | bo˩ʈʂʰæ˧ ʐv̩˩-di˩ ɲi˩.}}} \textcolor{Sepia}{\selectlanguage{english}The large needle is used to sew pipa meat.} \zh{大针,是用来缝琵琶肉的。}  
 \zh{量词}: \textcolor{darkblue}{\textbf{\ipa{ɭɯ˧}}}  \mytextsc{clf}: \textcolor{darkblue}{\textbf{\ipa{ɭɯ˧}}} 
\lhead{\firstmark}
\rhead{\botmark}

\subsection{\hspace{-0.5cm} {\Large \textcolor{darkblue}{\textbf{\ipa{ʁo˧mi˥\$}}}}\hspace{0.5cm}[\kern2pt{\textcolor{darkblue}{\textbf{\ipa{ʁo˧mi˥}}}}\kern2pt]} \hypertarget{Ro\string_Mmi\string_T\$1}{}
\markboth{\textcolor{darkblue}{\textbf{\ipa{ʁo˧mi˥\$}}}}{}
\textcolor{teal}{\mytextsc{noun}} \hspace{4pt} Tone: H\$.
\textcolor{Sepia}{\selectlanguage{english}Large needle.} \zh{大针。}  \zh{量词}: \textcolor{darkblue}{\textbf{\ipa{ɭɯ˧}}}  \mytextsc{clf}: \textcolor{darkblue}{\textbf{\ipa{ɭɯ˧}}} 
\lhead{\firstmark}
\rhead{\botmark}

\subsection{\hspace{-0.5cm} {\Large \textcolor{darkblue}{\textbf{\ipa{ʁo˧mi˧}}}}\hspace{0.5cm}[\kern2pt{\textcolor{darkblue}{\textbf{\ipa{ʁo˧mi˧}}}}\kern2pt]} \hypertarget{Ro\string_Mmi\string_M1}{}
\markboth{\textcolor{darkblue}{\textbf{\ipa{ʁo˧mi˧}}}}{}
\textcolor{teal}{\mytextsc{noun}} \hspace{4pt} Tone: M.
\textcolor{Sepia}{\selectlanguage{english}King; high official; chief.} \zh{国王、大臣、头领。}  ¶ \textcolor{darkblue}{\textbf{\ipa{ʁo˧mi˧ ʝi˧-hĩ˧ hĩ˧}}} \textcolor{Sepia}{\selectlanguage{english}person who has a role as king/high official/chief} \zh{当国王、土司、大臣、头领……的人}  
 ¶ \textcolor{darkblue}{\textbf{\ipa{kʰv̩˧mæ˧-ʁo˧mi˧}}} \textcolor{Sepia}{\selectlanguage{english}head of (a band of) robbers} \zh{土匪的头领}  
 \zh{量词}: \textcolor{darkblue}{\textbf{\ipa{v̩˧}}}  \mytextsc{clf}: \textcolor{darkblue}{\textbf{\ipa{v̩˧}}} 
\lhead{\firstmark}
\rhead{\botmark}

\subsection{\hspace{-0.5cm} {\Large \textcolor{darkblue}{\textbf{\ipa{ʁo˧ni˥}}}}\hspace{0.5cm}[\kern2pt{\textcolor{darkblue}{\textbf{\ipa{ʁo˧ni˥}}}}\kern2pt]} \hypertarget{Ro\string_Mni\string_T1}{}
\markboth{\textcolor{darkblue}{\textbf{\ipa{ʁo˧ni˥}}}}{}
\textcolor{teal}{\mytextsc{noun}} \hspace{4pt} Tone: H\#.
\textcolor{Sepia}{\selectlanguage{english}Woven headdress for young women who do not yet have children; for women who already have children, this same item is called \textcolor{darkblue}{\textbf{\ipa{/ʁo˧kɤ˩/}}}.} \zh{用来将长辫缠成盘头的黑色丝头饰(还没有孩子的青年女人戴的)。已经有孩子的女人,也戴这种头饰,但称作\textcolor{darkblue}{\textbf{\ipa{/ʁo˧kɤ˩/}}})。}  \zh{量词}: \textcolor{darkblue}{\textbf{\ipa{bo˩}}}  \mytextsc{clf}: \textcolor{darkblue}{\textbf{\ipa{bo˩}}} 
\lhead{\firstmark}
\rhead{\botmark}

\subsection{\hspace{-0.5cm} {\Large \textcolor{darkblue}{\textbf{\ipa{ʁo˧pʰɤ˩-ʁo˩dv̩˩lv̩˩}}}}\hspace{0.5cm}[\kern2pt{\textcolor{darkblue}{\textbf{\ipa{xxxx non-correspondance entre le nombre de morphèmes et le nombre de tons de morphèmes}}}}\kern2pt]} \hypertarget{Ro\string_Mp\string_h7\string_B-Ro\string_Bdv\string_=\string_Blv\string_=\string_B1}{}
\markboth{\textcolor{darkblue}{\textbf{\ipa{ʁo˧pʰɤ˩-ʁo˩dv̩˩lv̩˩}}}}{}
\textcolor{teal}{\mytextsc{noun}} \hspace{4pt} Tone: L\#.
\textcolor{Sepia}{\selectlanguage{english}\textit{Eugeron breviscapus} (a type of daisy).} \zh{短葶飞蓬。} 
\lhead{\firstmark}
\rhead{\botmark}

\subsection{\hspace{-0.5cm} {\Large \textcolor{darkblue}{\textbf{\ipa{ʁo˧qɑ˥}}}}\hspace{0.5cm}[\kern2pt{\textcolor{darkblue}{\textbf{\ipa{ʁo˧qɑ˥}}}}\kern2pt]} \hypertarget{Ro\string_MqA\string_T1}{}
\markboth{\textcolor{darkblue}{\textbf{\ipa{ʁo˧qɑ˥}}}}{}
\textcolor{teal}{\mytextsc{noun}} \hspace{4pt} Tone: H\#.
\textcolor{Sepia}{\selectlanguage{english}Lid.} \zh{锅盖、盖子。}  \zh{量词}: \textcolor{darkblue}{\textbf{\ipa{ɭɯ˧}}}  \mytextsc{clf}: \textcolor{darkblue}{\textbf{\ipa{ɭɯ˧}}} 
\lhead{\firstmark}
\rhead{\botmark}

\subsection{\hspace{-0.5cm} {\Large \textcolor{darkblue}{\textbf{\ipa{ʁo˧qʰwɤ˩}}}}\hspace{0.5cm}[\kern2pt{\textcolor{darkblue}{\textbf{\ipa{ʁo˧qʰwɤ˩}}}}\kern2pt]} \hypertarget{Ro\string_Mq\string_hw7\string_B1}{}
\markboth{\textcolor{darkblue}{\textbf{\ipa{ʁo˧qʰwɤ˩}}}}{}
\textcolor{teal}{\mytextsc{noun}} \hspace{4pt} Tone: L\#.
\ding{202} \textcolor{Sepia}{\selectlanguage{english}Head.} \zh{头,上面部分。}  ¶ \textcolor{darkblue}{\textbf{\ipa{ʁo˧qʰwɤ˩ dzi˩}}} \textcolor{Sepia}{\selectlanguage{english}to sit in a place of honour} \zh{坐在贵宾的位置上}  
 ¶ \textcolor{darkblue}{\textbf{\ipa{õ˧-ʁo˥qʰwɤ˩}}} \textcolor{Sepia}{\selectlanguage{english}one's own head} \zh{自己的头}  
 ¶ \textcolor{darkblue}{\textbf{\ipa{õ˧-ʁo˥qʰwɤ˩ lɑ˩}}} \textcolor{Sepia}{\selectlanguage{english}to hit one's own head (context: a child hits its own head rhythmically with a stick)} \zh{打自己的头(情景:一个小孩用小棍子敲打自己的头)}  
 \zh{量词}: \textcolor{darkblue}{\textbf{\ipa{ɭɯ˧}}} \ding{203} \textcolor{Sepia}{\selectlanguage{english}Top part, upper part.} \zh{上面部分。}  \mytextsc{clf}: \textcolor{darkblue}{\textbf{\ipa{ɭɯ˧}}} 
\lhead{\firstmark}
\rhead{\botmark}

\subsection{\hspace{-0.5cm} {\Large \textcolor{darkblue}{\textbf{\ipa{ʁo˧so˩}}}}\hspace{0.5cm}[\kern2pt{\textcolor{darkblue}{\textbf{\ipa{ʁo˧so˩}}}}\kern2pt]} \hypertarget{Ro\string_Mso\string_B1}{}
\markboth{\textcolor{darkblue}{\textbf{\ipa{ʁo˧so˩}}}}{}
\textcolor{teal}{\mytextsc{adverb(ial)}} \hspace{4pt} Tone: L\#.
\textcolor{Sepia}{\selectlanguage{english}The day after tomorrow.} \zh{后天。}  ¶ \textcolor{darkblue}{\textbf{\ipa{ʁo˧so˩ | -ɖɯ˧ɲi˥}}} \textcolor{Sepia}{\selectlanguage{english}the day after tomorrow} \zh{后天}  

\lhead{\firstmark}
\rhead{\botmark}

\subsection{\hspace{-0.5cm} {\Large \textcolor{darkblue}{\textbf{\ipa{ʁo˧ʂv̩˧}}}}\hspace{0.5cm}[\kern2pt{\textcolor{darkblue}{\textbf{\ipa{ʁo˧ʂv̩˧}}}}\kern2pt]} \hypertarget{Ro\string_Ms`v\string_=\string_M1}{}
\markboth{\textcolor{darkblue}{\textbf{\ipa{ʁo˧ʂv̩˧}}}}{}
\textcolor{teal}{\mytextsc{verb}} \hspace{4pt} Tone: M.
\textcolor{Sepia}{\selectlanguage{english}To guide, to show the way.} \zh{带头、带路。}  ¶ \textcolor{darkblue}{\textbf{\ipa{ʐɤ˩mi˩ ʁo˩ʂv̩˩˥}}} \textcolor{Sepia}{\selectlanguage{english}to show the way} \zh{带路}  
 ¶ \textcolor{darkblue}{\textbf{\ipa{ɖɯ˧-ʑi˩-ɳɯ˩ | ʁo˧ʂv̩˧}}} \textcolor{Sepia}{\selectlanguage{english}a family shows the way/ sets an example (which other families follow): for instance, one family begins to harvest rice, and others follow their example} \zh{有一家带头:例如收庄稼时,一个家先开始收割,于是其它家庭也跟着开始收割。}  
 ¶ \textcolor{darkblue}{\textbf{\ipa{ʁo˧ʂv̩˧-ze˧}}} \textcolor{Sepia}{\selectlanguage{english}\mytextsc{pfv}} \zh{带了路}  
 ¶ \textcolor{darkblue}{\textbf{\ipa{njɤ˧=ɻ̍˩-ɳɯ˩ | ʁo˧ʂv̩˧!}}} \textcolor{Sepia}{\selectlanguage{english}We are showing the way! / We are setting an example for others! (Context: for agricultural activities, one household started first, and the others followed suit.)} \zh{是我们带头的!(其他家庭是跟着我们来的!)(情景:农业活动,如:收庄稼,是一个家庭先开始的,然后其他家庭也跟着来。)}  

\lhead{\firstmark}
\rhead{\botmark}

\subsection{\hspace{-0.5cm} {\Large \textcolor{darkblue}{\textbf{\ipa{-ʁo˧to˩}}}}\hspace{0.5cm}[\kern2pt{\textcolor{darkblue}{\textbf{\ipa{ʁo˧to˩}}}}\kern2pt]} \hypertarget{-Ro\string_Mto\string_B1}{}
\markboth{\textcolor{darkblue}{\textbf{\ipa{-ʁo˧to˩}}}}{}
\textcolor{teal}{\mytextsc{postposition}} \hspace{4pt} Tone: L\#.
\ding{202} \textcolor{Sepia}{\selectlanguage{english}On top of.} \zh{……之上。}  ¶ \textcolor{darkblue}{\textbf{\ipa{qo˩qɑ˩-ʁo˩to˥}}} \textcolor{Sepia}{\selectlanguage{english}at the top of the mountain pass, at the mountain pass} \zh{垭口上}  
 ¶ \textcolor{darkblue}{\textbf{\ipa{ʁo˧qʰwɤ˩-ʁo˩to˩}}} \textcolor{Sepia}{\selectlanguage{english}on the head, on top of the head} \zh{头上}  
 ¶ \textcolor{darkblue}{\textbf{\ipa{ʑi˧qʰwɤ˧-ʁo˧to˩}}} \textcolor{Sepia}{\selectlanguage{english}on the house; e.g. there is a bird's nest on the top of the house} \zh{房子上面:例如:有鸟窝在房顶上}  
\ding{203} \textcolor{Sepia}{\selectlanguage{english}While, at the time that, during the time that.} \zh{……的时候。}  ¶ \textcolor{darkblue}{\textbf{\ipa{hɑ˧dzɯ˧-ʁo˧to˩, | ʈʂʰɯ˧-ɳɯ˧ | mɤ˧-fv̩˧-ʝi˧.}}} \textcolor{Sepia}{\selectlanguage{english}During the meal, he felt displeased/he got angry.} \zh{吃饭的时候,他不高兴了/生气了。}  
\ding{204} \textcolor{Sepia}{\selectlanguage{english}To, at, towards.} \zh{向、往。} \ding{205} \textcolor{Sepia}{\selectlanguage{english}Compared to.} \zh{跟……相比。} 
\lhead{\firstmark}
\rhead{\botmark}

\subsection{\hspace{-0.5cm} {\Large \textcolor{darkblue}{\textbf{\ipa{-ʁo˧tʰo˩}}}}\hspace{0.5cm}[\kern2pt{\textcolor{darkblue}{\textbf{\ipa{ʁo˧tʰo˩}}}}\kern2pt]} \hypertarget{-Ro\string_Mt\string_ho\string_B1}{}
\markboth{\textcolor{darkblue}{\textbf{\ipa{-ʁo˧tʰo˩}}}}{}
\textcolor{teal}{\mytextsc{postposition}} \hspace{4pt} Tone: L\#.
\textcolor{Sepia}{\selectlanguage{english}Behind; since.} \zh{后面,自从。}  ¶ \textcolor{darkblue}{\textbf{\ipa{ʑi˧-tʰo˩}}} \textcolor{Sepia}{\selectlanguage{english}behind the house (=the place where there is a vegetable garden)} \zh{家后院(=菜园的地方)}  
 ¶ \textcolor{darkblue}{\textbf{\ipa{ʑi˧-ʁo˥tʰo˩}}} \textcolor{Sepia}{\selectlanguage{english}as above: behind the house} \zh{同上:家后院}  
 ¶ \textcolor{darkblue}{\textbf{\ipa{ʑi˧qʰwɤ˧-ʁo˧tʰo˩}}} \textcolor{Sepia}{\selectlanguage{english}as above: behind the house} \zh{同上:家后院}  

\lhead{\firstmark}
\rhead{\botmark}

\subsection{\hspace{-0.5cm} {\Large \textcolor{darkblue}{\textbf{\ipa{ʁo˧tɕʰɤ\#˥}}}}\hspace{0.5cm}[\kern2pt{\textcolor{darkblue}{\textbf{\ipa{ʁo˧tɕʰɤ˧}}}}\kern2pt]} \hypertarget{Ro\string_Mts£\string_h7\#\string_T1}{}
\markboth{\textcolor{darkblue}{\textbf{\ipa{ʁo˧tɕʰɤ\#˥}}}}{}
\textcolor{teal}{\mytextsc{adjective}} \hspace{4pt} Tone: \#H.
\textcolor{Sepia}{\selectlanguage{english}Sharp, pointed.} \zh{尖。}  ¶ \textcolor{darkblue}{\textbf{\ipa{[F5] ʁo˧tɕʰɤ˧\textasciitilde{}tɕʰɤ˧-gv̩˧}}} \textcolor{Sepia}{\selectlanguage{english}sharp} \zh{尖}  

\lhead{\firstmark}
\rhead{\botmark}

\subsection{\hspace{-0.5cm} {\Large \textcolor{darkblue}{\textbf{\ipa{ʁo˧tsʰe˧ʁo\#˥}}}}\hspace{0.5cm}[\kern2pt{\textcolor{darkblue}{\textbf{\ipa{ʁo˧tsʰe˧ʁo˩}}}}\kern2pt]} \hypertarget{Ro\string_Mts\string_he\string_MRo\#\string_T1}{}
\markboth{\textcolor{darkblue}{\textbf{\ipa{ʁo˧tsʰe˧ʁo\#˥}}}}{}
\textcolor{teal}{\mytextsc{noun}} \hspace{4pt} Tone: \#H.
\textcolor{Sepia}{\selectlanguage{english}Top (e.g. mountain top).} \zh{顶上,如:山顶。}  ¶ \textcolor{darkblue}{\textbf{\ipa{ʁwɤ˧-bv̩˧ | ʁo˧tsʰe˧ʁo˧}}} \textcolor{Sepia}{\selectlanguage{english}the top of the mountain, the mountain top} \zh{山的顶,山顶}  
 ¶ \textcolor{darkblue}{\textbf{\ipa{ʁo˧qʰwɤ˩-ʁo˩tsʰe˩}}} \textcolor{Sepia}{\selectlanguage{english}the top of the head} \zh{头顶}  

\lhead{\firstmark}
\rhead{\botmark}

\subsection{\hspace{-0.5cm} {\Large \textcolor{darkblue}{\textbf{\ipa{ʁo˧ʈv̩˧ʈv̩˥}}}}\hspace{0.5cm}[\kern2pt{\textcolor{darkblue}{\textbf{\ipa{ʁo˧ʈv̩˧ʈv̩˧}}}}\kern2pt]} \hypertarget{Ro\string_Mt`v\string_=\string_Mt`v\string_=\string_T1}{}
\markboth{\textcolor{darkblue}{\textbf{\ipa{ʁo˧ʈv̩˧ʈv̩˥}}}}{}
\textcolor{teal}{\mytextsc{noun}} \hspace{4pt} Tone: H\#.
\textcolor{Sepia}{\selectlanguage{english}Yi (derogatory term: “ungroomed heads”, “messy heads”).} \zh{彝族(带偏见的说法:“乱糟糟的头发”)。}  \zh{量词}: \textcolor{darkblue}{\textbf{\ipa{v̩˧}}}  \mytextsc{clf}: \textcolor{darkblue}{\textbf{\ipa{v̩˧}}} 
\lhead{\firstmark}
\rhead{\botmark}

\subsection{\hspace{-0.5cm} {\Large \textcolor{darkblue}{\textbf{\ipa{ʁo˧ʈʂe˩}}}}\hspace{0.5cm}[\kern2pt{\textcolor{darkblue}{\textbf{\ipa{ʁo˧ʈʂe˩}}}}\kern2pt]} \hypertarget{Ro\string_Mt`s`e\string_B1}{}
\markboth{\textcolor{darkblue}{\textbf{\ipa{ʁo˧ʈʂe˩}}}}{}
\textcolor{teal}{\mytextsc{noun}} \hspace{4pt} Tone: L\#.
\textcolor{Sepia}{\selectlanguage{english}Tinea, ringworm.} \zh{癣。}  ¶ \textcolor{darkblue}{\textbf{\ipa{[M23] ʁu˧ʈʂɯ˩ ɖwæ˧˥ tʰi˧ di˩!}}} \textcolor{Sepia}{\selectlanguage{english}(She/he) has a big patch of tinea!} \zh{他长了很多癣!}  
 \zh{量词}: \textcolor{darkblue}{\textbf{\ipa{pʰæ˧˥}}}  \mytextsc{clf}: \textcolor{darkblue}{\textbf{\ipa{pʰæ˧˥}}} 
\lhead{\firstmark}
\rhead{\botmark}

\subsection{\hspace{-0.5cm} {\Large \textcolor{darkblue}{\textbf{\ipa{ʁo˧zo\#˥}}}}\hspace{0.5cm}[\kern2pt{\textcolor{darkblue}{\textbf{\ipa{ʁo˧zo˧˥}}}}\kern2pt]} \hypertarget{Ro\string_Mzo\#\string_T1}{}
\markboth{\textcolor{darkblue}{\textbf{\ipa{ʁo˧zo\#˥}}}}{}
\textcolor{teal}{\mytextsc{noun}} \hspace{4pt} Tone: \#H.
\textcolor{Sepia}{\selectlanguage{english}Small needle.} \zh{小针。}  \zh{量词}: \textcolor{darkblue}{\textbf{\ipa{ɭɯ˧}}}  \mytextsc{clf}: \textcolor{darkblue}{\textbf{\ipa{ɭɯ˧}}} 
\lhead{\firstmark}
\rhead{\botmark}

\subsection{\hspace{-0.5cm} {\Large \textcolor{darkblue}{\textbf{\ipa{ʁo˧ʑi˧˥}}}}\hspace{0.5cm}[\kern2pt{\textcolor{darkblue}{\textbf{\ipa{ʁo˧ʑi˥}}}}\kern2pt]} \hypertarget{Ro\string_Mz£i\string_M\string_T1}{}
\markboth{\textcolor{darkblue}{\textbf{\ipa{ʁo˧ʑi˧˥}}}}{}
\textcolor{teal}{\mytextsc{adverb(ial)}} \hspace{4pt} Tone: MH\#.
\textcolor{Sepia}{\selectlanguage{english}As from…, starting….} \zh{从……开始。}  ¶ \textcolor{darkblue}{\textbf{\ipa{ɖɯ˧ɬi˧mi˧-ʁo˧ʑi˧˥}}} \textcolor{Sepia}{\selectlanguage{english}from the first month} \zh{一月份开始}  
 ¶ \textcolor{darkblue}{\textbf{\ipa{tsʰi˧ɲi˧-ʁo˧ʑi˧˥}}} \textcolor{Sepia}{\selectlanguage{english}as from today} \zh{今天开始}  
 ¶ \textcolor{darkblue}{\textbf{\ipa{tsʰi˧ʝi˧ ɖɯ˧-kʰv̩˧˥-ʁo˧ʑi˧˥}}} \textcolor{Sepia}{\selectlanguage{english}from this year on, as from this year} \zh{今年开始}  
 ¶ \textcolor{darkblue}{\textbf{\ipa{gv̩˩ɬi˩mi˩-ʁo˩ʑi˩˥}}} \textcolor{Sepia}{\selectlanguage{english}as from the 9th month, starting from the 9th month} \zh{九月份开始}  
 ¶ \textcolor{darkblue}{\textbf{\ipa{ʐe˧ʈæ˥ɬi˩-ʁo˩ʑi˩}}} \textcolor{Sepia}{\selectlanguage{english}as from the 11th month, starting from the 11th month} \zh{十一月份开始}  

\lhead{\firstmark}
\rhead{\botmark}

\subsection{\hspace{-0.5cm} {\Large \textcolor{darkblue}{\textbf{\ipa{ʁo˩}}}}\hspace{0.5cm}[\kern2pt{\textcolor{darkblue}{\textbf{\ipa{ʁo˩˥}}}}\kern2pt]} \hypertarget{Ro\string_B1}{}
\markboth{\textcolor{darkblue}{\textbf{\ipa{ʁo˩}}}}{}
\textcolor{teal}{\mytextsc{verb}} \hspace{4pt} Tone: L.
\textcolor{Sepia}{\selectlanguage{english}To sink (e.g. a boat slowly sinking down into a lake).} \zh{掉入、沉下去。}  ¶ \textcolor{darkblue}{\textbf{\ipa{mv̩˩tɕo˥ ʁo˩}}} \textcolor{Sepia}{\selectlanguage{english}to sink} \zh{掉入}  

\lhead{\firstmark}
\rhead{\botmark}

\subsection{\hspace{-0.5cm} {\Large \textcolor{darkblue}{\textbf{\ipa{ʁo˩\textsubscript{b}}}}}\hspace{0.5cm}[\kern2pt{\textcolor{darkblue}{\textbf{\ipa{ʁo˩˥}}}}\kern2pt]} \hypertarget{Ro\string_Bb1}{}
\markboth{\textcolor{darkblue}{\textbf{\ipa{ʁo˩\textsubscript{b}}}}}{}
\textcolor{teal}{\mytextsc{classifier}} \hspace{4pt} Tone: L\textsubscript{b}.
\textcolor{Sepia}{\selectlanguage{english}A sort of.} \zh{量词:种。}  ¶ \textcolor{darkblue}{\textbf{\ipa{ɖɯ˧-ʁo˩}}} \textcolor{Sepia}{\selectlanguage{english}one type (of clothing, food...)} \zh{一种(衣服、食物……)}  
 ¶ \textcolor{darkblue}{\textbf{\ipa{ʈʂʰɯ˧-ʁo˥}}} \textcolor{Sepia}{\selectlanguage{english}this type (of clothing, food...)} \zh{这种(衣服、食物……)}  

\lhead{\firstmark}
\rhead{\botmark}

\subsection{\hspace{-0.5cm} {\Large \textcolor{darkblue}{\textbf{\ipa{ʁo˩\textsubscript{b}}}}}\hspace{0.5cm}[\kern2pt{\textcolor{darkblue}{\textbf{\ipa{ʁo˥}}}}\kern2pt]} \hypertarget{Ro\string_Bb1}{}
\markboth{\textcolor{darkblue}{\textbf{\ipa{ʁo˩\textsubscript{b}}}}}{}
\textcolor{teal}{\mytextsc{verb}} \hspace{4pt} Tone: L\textsubscript{b}.
\textcolor{Sepia}{\selectlanguage{english}To form, to appear: e.g. a callus has formed.} \zh{出现、形成(如:出了茧子)。}  ¶ \textcolor{darkblue}{\textbf{\ipa{sɯ˧ʈv̩˥ ʁo˩-ze˩! |}}} \textcolor{Sepia}{\selectlanguage{english}A callus has formed!} \zh{磨出了茧子!}  
 ¶ \textcolor{darkblue}{\textbf{\ipa{ʁo˩-mɤ˩-ho˥}}} \textcolor{Sepia}{\selectlanguage{english}\string_ \mytextsc{neg} \mytextsc{desiderative}} \zh{不会出(茧子)}  

\lhead{\firstmark}
\rhead{\botmark}

\subsection{\hspace{-0.5cm} {\Large \textcolor{darkblue}{\textbf{\ipa{ʁo˩di˥}}}}\hspace{0.5cm}[\kern2pt{\textcolor{darkblue}{\textbf{\ipa{ʁo˧di˧}}}}\kern2pt]} \hypertarget{Ro\string_Bdi\string_T1}{}
\markboth{\textcolor{darkblue}{\textbf{\ipa{ʁo˩di˥}}}}{}
\textcolor{teal}{\mytextsc{noun}} \hspace{4pt} Tone: LH.
\textcolor{Sepia}{\selectlanguage{english}Mad person.} \zh{疯子。}  \zh{量词}: \textcolor{darkblue}{\textbf{\ipa{v̩˧}}}  \mytextsc{clf}: \textcolor{darkblue}{\textbf{\ipa{v̩˧}}} 
\lhead{\firstmark}
\rhead{\botmark}

\subsection{\hspace{-0.5cm} {\Large \textcolor{darkblue}{\textbf{\ipa{ʁo˩ɖɯ˩so˧}}}}\hspace{0.5cm}[\kern2pt{\textcolor{darkblue}{\textbf{\ipa{ʁo˧ɖɯ˧so˧˥}}}}\kern2pt]} \hypertarget{Ro\string_Bd`M\string_Bso\string_M1}{}
\markboth{\textcolor{darkblue}{\textbf{\ipa{ʁo˩ɖɯ˩so˧}}}}{}
\textcolor{teal}{\mytextsc{adverb(ial)}} \hspace{4pt} Tone: .
\textcolor{Sepia}{\selectlanguage{english}In three days.} \zh{大后天。} 
\lhead{\firstmark}
\rhead{\botmark}

\subsection{\hspace{-0.5cm} {\Large \textcolor{darkblue}{\textbf{\ipa{ʁo˩hi˩}}}}\hspace{0.5cm}[\kern2pt{\textcolor{darkblue}{\textbf{\ipa{ʁo˩hi˩˥}}}}\kern2pt]} \hypertarget{Ro\string_Bhi\string_B1}{}
\markboth{\textcolor{darkblue}{\textbf{\ipa{ʁo˩hi˩}}}}{}
\textcolor{teal}{\mytextsc{noun}} \hspace{4pt} Tone: L.
\textcolor{Sepia}{\selectlanguage{english}Molars and premolars.} \zh{臼齿+后臼齿。}  \zh{量词}: \textcolor{darkblue}{\textbf{\ipa{ɭɯ˧}}}  \mytextsc{clf}: \textcolor{darkblue}{\textbf{\ipa{ɭɯ˧}}} 
\lhead{\firstmark}
\rhead{\botmark}

\subsection{\hspace{-0.5cm} {\Large \textcolor{darkblue}{\textbf{\ipa{ʁo˩kʰv̩˩}}}}\hspace{0.5cm}[\kern2pt{\textcolor{darkblue}{\textbf{\ipa{ʁo˩kʰv̩˩˥}}}}\kern2pt]} \hypertarget{Ro\string_Bk\string_hv\string_=\string_B1}{}
\markboth{\textcolor{darkblue}{\textbf{\ipa{ʁo˩kʰv̩˩}}}}{}
\textcolor{teal}{\mytextsc{noun}} \hspace{4pt} Tone: L.
\textcolor{Sepia}{\selectlanguage{english}Sandalwood, sandlewood.} \zh{香木。} Local Chinese dialect:\zh{柏香。} ¶ \textcolor{darkblue}{\textbf{\ipa{ʁo˩kʰv̩˩-si˩}}} \textcolor{Sepia}{\selectlanguage{english}same meaning} \zh{同上}  
\textit{See:} \hyperlink{}{\textcolor{darkblue}{\textbf{\ipa{tsɤ˧di˧}}}} 
\lhead{\firstmark}
\rhead{\botmark}

\subsection{\hspace{-0.5cm} {\Large \textcolor{darkblue}{\textbf{\ipa{ʁo˧˥}}}}\hspace{0.5cm}[\kern2pt{\textcolor{darkblue}{\textbf{\ipa{ʁo˧˥}}}}\kern2pt]} \hypertarget{Ro\string_M\string_T1}{}
\markboth{\textcolor{darkblue}{\textbf{\ipa{ʁo˧˥}}}}{}
\textcolor{teal}{\mytextsc{noun}} \hspace{4pt} Tone: MH.
\textcolor{Sepia}{\selectlanguage{english}Needle.} \zh{针。}  \zh{量词}: \textcolor{darkblue}{\textbf{\ipa{ɭɯ˧}}}  \mytextsc{clf}: \textcolor{darkblue}{\textbf{\ipa{ɭɯ˧}}} \textit{See:} \textcolor{darkblue}{\textbf{\ipa{ʈʂe˥}}} 
\lhead{\firstmark}
\rhead{\botmark}

\subsection{\hspace{-0.5cm} {\Large \textcolor{darkblue}{\textbf{\ipa{ʁv̩˧˥}}}}\hspace{0.5cm}[\kern2pt{\textcolor{darkblue}{\textbf{\ipa{ʁv̩˧˥}}}}\kern2pt]} \hypertarget{Rv\string_=\string_M\string_T1}{}
\markboth{\textcolor{darkblue}{\textbf{\ipa{ʁv̩˧˥}}}}{}
\textcolor{teal}{\mytextsc{noun}} \hspace{4pt} Tone: MH.
\textcolor{Sepia}{\selectlanguage{english}Crane (a migratory bird).} \zh{黑颈鹤(候鸟)。}  ¶ \textcolor{darkblue}{\textbf{\ipa{ʁv̩˧nɑ˥mi˩}}} \textcolor{Sepia}{\selectlanguage{english}same meaning: crane} \zh{同上:黑颈鹤}  
 ¶ \textcolor{darkblue}{\textbf{\ipa{ʁv̩˧ dzɯ˥-ze˩}}} \textcolor{Sepia}{\selectlanguage{english}...ate the crane} \zh{吃了黑颈鹤}  
 ¶ \textcolor{darkblue}{\textbf{\ipa{ʁv̩˧ hwæ˥-ze˩}}} \textcolor{Sepia}{\selectlanguage{english}...bought (a/the) crane} \zh{买了黑颈鹤}  
 \zh{量词}: \textcolor{darkblue}{\textbf{\ipa{mi˩}}}  \mytextsc{clf}: \textcolor{darkblue}{\textbf{\ipa{mi˩}}} 
\lhead{\firstmark}
\rhead{\botmark}

\subsection{\hspace{-0.5cm} {\Large \textcolor{darkblue}{\textbf{\ipa{ʁv̩˥}}}}\hspace{0.5cm}[\kern2pt{\textcolor{darkblue}{\textbf{\ipa{ʁv̩˥}}}}\kern2pt]} \hypertarget{Rv\string_=\string_T1}{}
\markboth{\textcolor{darkblue}{\textbf{\ipa{ʁv̩˥}}}}{}
\textcolor{teal}{\mytextsc{verb}} \hspace{4pt} Tone: H.
\textcolor{Sepia}{\selectlanguage{english}To swallow.} \zh{吞,咽。}  ¶ \textcolor{darkblue}{\textbf{\ipa{le˧-ʁv̩˥}}} \textcolor{Sepia}{\selectlanguage{english}\mytextsc{accomp}} \zh{\mytextsc{accomp}}  

\lhead{\firstmark}
\rhead{\botmark}

\subsection{\hspace{-0.5cm} {\Large \textcolor{darkblue}{\textbf{\ipa{ʁv̩˧mi˥\$}}}}\hspace{0.5cm}[\kern2pt{\textcolor{darkblue}{\textbf{\ipa{ʁv̩˧mi˥}}}}\kern2pt]} \hypertarget{Rv\string_=\string_Mmi\string_T\$1}{}
\markboth{\textcolor{darkblue}{\textbf{\ipa{ʁv̩˧mi˥\$}}}}{}
\textcolor{teal}{\mytextsc{noun}} \hspace{4pt} Tone: H\$.
\textcolor{Sepia}{\selectlanguage{english}Female crane.} \zh{母鹤。}  \zh{量词}: \textcolor{darkblue}{\textbf{\ipa{mi˩}}}  \mytextsc{clf}: \textcolor{darkblue}{\textbf{\ipa{mi˩}}} 
\lhead{\firstmark}
\rhead{\botmark}

\subsection{\hspace{-0.5cm} {\Large \textcolor{darkblue}{\textbf{\ipa{ʁv̩˧pʰv̩\#˥}}}}\hspace{0.5cm}[\kern2pt{\textcolor{darkblue}{\textbf{\ipa{ʁv̩˧pʰv̩˧}}}}\kern2pt]} \hypertarget{Rv\string_=\string_Mp\string_hv\string_=\#\string_T1}{}
\markboth{\textcolor{darkblue}{\textbf{\ipa{ʁv̩˧pʰv̩\#˥}}}}{}
\textcolor{teal}{\mytextsc{noun}} \hspace{4pt} Tone: \#H.
\textcolor{Sepia}{\selectlanguage{english}Male crane.} \zh{公鹤。}  ¶ \textcolor{darkblue}{\textbf{\ipa{ʁv̩˧pʰv̩˧-ʁv̩˧mi\#˥}}} \textcolor{Sepia}{\selectlanguage{english}male crane and female crane} \zh{公鹤与母鹤}  
 \zh{量词}: \textcolor{darkblue}{\textbf{\ipa{mi˩}}}  \mytextsc{clf}: \textcolor{darkblue}{\textbf{\ipa{mi˩}}} 
\lhead{\firstmark}
\rhead{\botmark}

\subsection{\hspace{-0.5cm} {\Large \textcolor{darkblue}{\textbf{\ipa{ʁv̩˧zo\#˥}}}}\hspace{0.5cm}[\kern2pt{\textcolor{darkblue}{\textbf{\ipa{ʁv̩˧zo˧}}}}\kern2pt]} \hypertarget{Rv\string_=\string_Mzo\#\string_T1}{}
\markboth{\textcolor{darkblue}{\textbf{\ipa{ʁv̩˧zo\#˥}}}}{}
\textcolor{teal}{\mytextsc{noun}} \hspace{4pt} Tone: \#H.
\textcolor{Sepia}{\selectlanguage{english}Baby crane.} \zh{小鹤。}  \zh{量词}: \textcolor{darkblue}{\textbf{\ipa{ɭɯ˧}}}  \mytextsc{clf}: \textcolor{darkblue}{\textbf{\ipa{ɭɯ˧}}} 
\lhead{\firstmark}
\rhead{\botmark}

\subsection{\hspace{-0.5cm} {\Large \textcolor{darkblue}{\textbf{\ipa{ʁwæ˥}}}}\hspace{0.5cm}[\kern2pt{\textcolor{darkblue}{\textbf{\ipa{ʁwæ˥}}}}\kern2pt]} \hypertarget{Rw\{\string_T1}{}
\markboth{\textcolor{darkblue}{\textbf{\ipa{ʁwæ˥}}}}{}
\textcolor{teal}{\mytextsc{noun}} \hspace{4pt} Tone: \#H.
\textcolor{Sepia}{\selectlanguage{english}Left (monosyllable).} \zh{左边(单音节)。} 
\lhead{\firstmark}
\rhead{\botmark}

\subsection{\hspace{-0.5cm} {\Large \textcolor{darkblue}{\textbf{\ipa{ʁwæ˧gi\#˥}}}}\hspace{0.5cm}[\kern2pt{\textcolor{darkblue}{\textbf{\ipa{ʁwæ˧gi˧}}}}\kern2pt]} \hypertarget{Rw\{\string_Mgi\#\string_T1}{}
\markboth{\textcolor{darkblue}{\textbf{\ipa{ʁwæ˧gi\#˥}}}}{}
\textcolor{teal}{\mytextsc{noun}} \hspace{4pt} Tone: \#H.
\textcolor{Sepia}{\selectlanguage{english}Left side, left.} \zh{左边。} 
\lhead{\firstmark}
\rhead{\botmark}

\subsection{\hspace{-0.5cm} {\Large \textcolor{darkblue}{\textbf{\ipa{ʁwæ˧gi˧dzɤ\#˥}}}}\hspace{0.5cm}[\kern2pt{\textcolor{darkblue}{\textbf{\ipa{ʁwæ˧gi˧dzɤ˧}}}}\kern2pt]} \hypertarget{Rw\{\string_Mgi\string_Mdz7\#\string_T1}{}
\markboth{\textcolor{darkblue}{\textbf{\ipa{ʁwæ˧gi˧dzɤ\#˥}}}}{}
\textcolor{teal}{\mytextsc{noun}} \hspace{4pt} Tone: \#H.
\textcolor{Sepia}{\selectlanguage{english}Left, left side, left direction.} \zh{左、左边。} 
\lhead{\firstmark}
\rhead{\botmark}

\subsection{\hspace{-0.5cm} {\Large \textcolor{darkblue}{\textbf{\ipa{ʁwæ˧lo˥}}}}\hspace{0.5cm}[\kern2pt{\textcolor{darkblue}{\textbf{\ipa{ʁwæ˧lo˥}}}}\kern2pt]} \hypertarget{Rw\{\string_Mlo\string_T1}{}
\markboth{\textcolor{darkblue}{\textbf{\ipa{ʁwæ˧lo˥}}}}{}
\textcolor{teal}{\mytextsc{noun}} \hspace{4pt} Tone: H\#.
\textcolor{Sepia}{\selectlanguage{english}Left side, left direction.} \zh{左边,左手。} 
\lhead{\firstmark}
\rhead{\botmark}

\subsection{\hspace{-0.5cm} {\Large \textcolor{darkblue}{\textbf{\ipa{ʁwæ˧tsɯ˥}}}}\hspace{0.5cm}[\kern2pt{\textcolor{darkblue}{\textbf{\ipa{ʁwæ˧tsɯ˥}}}}\kern2pt]} \hypertarget{Rw\{\string_MtsM\string_T1}{}
\markboth{\textcolor{darkblue}{\textbf{\ipa{ʁwæ˧tsɯ˥}}}}{}
\textcolor{teal}{\mytextsc{noun}} \hspace{4pt} Tone: H\#.
\textcolor{Sepia}{\selectlanguage{english}Socks.} \zh{袜子。}  Borrowing: Chinese  \zh{袜子}

\lhead{\firstmark}
\rhead{\botmark}

\subsection{\hspace{-0.5cm} {\Large \textcolor{darkblue}{\textbf{\ipa{ʁwæ˧ʈʂʰe˩}}}}\hspace{0.5cm}[\kern2pt{\textcolor{darkblue}{\textbf{\ipa{ʁwæ˧ʈʂʰe˩}}}}\kern2pt]} \hypertarget{Rw\{\string_Mt`s`\string_he\string_B1}{}
\markboth{\textcolor{darkblue}{\textbf{\ipa{ʁwæ˧ʈʂʰe˩}}}}{}
\textcolor{teal}{\mytextsc{verb}} \hspace{4pt} Tone: L\#.
\textcolor{Sepia}{\selectlanguage{english}To accomplish, to complete.} \zh{完成(汉语借词)。}  Borrowing: Chinese  \zh{完成}
 ¶ \textcolor{darkblue}{\textbf{\ipa{le˧-ʁwæ˧ʈʂʰe˩-ze˩!}}} \textcolor{Sepia}{\selectlanguage{english}It's complete! / It's finished!} \zh{完成了!}  

\lhead{\firstmark}
\rhead{\botmark}

\subsection{\hspace{-0.5cm} {\Large \textcolor{darkblue}{\textbf{\ipa{ʁwɤ˧}}} \textsubscript{1}}\hspace{0.5cm}[\kern2pt{\textcolor{darkblue}{\textbf{\ipa{ʁwɤ˥}}}}\kern2pt]} \hypertarget{Rw7\string_M1}{}
\markboth{\textcolor{darkblue}{\textbf{\ipa{ʁwɤ˧}}} \textsubscript{1}}{}
\textcolor{teal}{\mytextsc{noun}} \hspace{4pt} Tone: M.
\textcolor{Sepia}{\selectlanguage{english}Mountain.} \zh{山。}  ¶ \textcolor{darkblue}{\textbf{\ipa{ʁo˧-ʂwæ˧}}} \textcolor{Sepia}{\selectlanguage{english}high mountain} \zh{高山}  
 \zh{量词}: \textcolor{darkblue}{\textbf{\ipa{ɭɯ˧}}}  \mytextsc{clf}: \textcolor{darkblue}{\textbf{\ipa{ɭɯ˧}}} 
\lhead{\firstmark}
\rhead{\botmark}

\subsection{\hspace{-0.5cm} {\Large \textcolor{darkblue}{\textbf{\ipa{ʁwɤ˧}}} \textsubscript{2}}\hspace{0.5cm}[\kern2pt{\textcolor{darkblue}{\textbf{\ipa{ʁwɤ˥}}}}\kern2pt]} \hypertarget{Rw7\string_M2}{}
\markboth{\textcolor{darkblue}{\textbf{\ipa{ʁwɤ˧}}} \textsubscript{2}}{}
\textcolor{teal}{\mytextsc{noun}} \hspace{4pt} Tone: M.
\textcolor{Sepia}{\selectlanguage{english}Village, hamlet.} \zh{村寨,村落。}  ¶ \textcolor{darkblue}{\textbf{\ipa{ʁwɤ˧-qo˧}}} \textcolor{Sepia}{\selectlanguage{english}in the village} \zh{村子里}  
 ¶ \textcolor{darkblue}{\textbf{\ipa{[M23] ɖɯ˧-ʁwɤ˧ mɤ˧-ɲi˩: | ʈʂʰɯ˧-ʁwɤ˧... | ʈʂʰɯ˧-ʁwɤ˧…}}} \textcolor{Sepia}{\selectlanguage{english}These do not belong to the same village: here, it is the village named...; over there, it is the village named...} \zh{它们不属于一个村落:这边,是……村,而那边,是……村。}  
 \zh{量词}: \textcolor{darkblue}{\textbf{\ipa{ʁwɤ˧}}}  \mytextsc{clf}: \textcolor{darkblue}{\textbf{\ipa{ʁwɤ˧}}} 
\lhead{\firstmark}
\rhead{\botmark}

\subsection{\hspace{-0.5cm} {\Large \textcolor{darkblue}{\textbf{\ipa{ʁwɤ˧}}} \textsubscript{3}}\hspace{0.5cm}[\kern2pt{\textcolor{darkblue}{\textbf{\ipa{ʁwɤ˥}}}}\kern2pt]} \hypertarget{Rw7\string_M3}{}
\markboth{\textcolor{darkblue}{\textbf{\ipa{ʁwɤ˧}}} \textsubscript{3}}{}
\textcolor{teal}{\mytextsc{noun}} \hspace{4pt} Tone: M.
\textcolor{Sepia}{\selectlanguage{english}Money.} \zh{钱。}  ¶ \textcolor{darkblue}{\textbf{\ipa{ɖʐe˧-ʁwɤ˧}}} \textcolor{Sepia}{\selectlanguage{english}money} \zh{钱}  

\lhead{\firstmark}
\rhead{\botmark}

\subsection{\hspace{-0.5cm} {\Large \textcolor{darkblue}{\textbf{\ipa{ʁwɤ˧\textsubscript{a}}}}}\hspace{0.5cm}[\kern2pt{\textcolor{darkblue}{\textbf{\ipa{ʁwɤ˥}}}}\kern2pt]} \hypertarget{Rw7\string_Ma1}{}
\markboth{\textcolor{darkblue}{\textbf{\ipa{ʁwɤ˧\textsubscript{a}}}}}{}
\textcolor{teal}{\mytextsc{classifier}} \hspace{4pt} Tone: M\textsubscript{a}.
\textcolor{Sepia}{\selectlanguage{english}A heap (e.g. of grains, of cut wood); literally: 'a mountain of'.} \zh{量词:堆(一堆粮食、一堆柴……)。} 
\lhead{\firstmark}
\rhead{\botmark}

\subsection{\hspace{-0.5cm} {\Large \textcolor{darkblue}{\textbf{\ipa{ʁwɤ˧\textsubscript{a}}}}}\hspace{0.5cm}[\kern2pt{\textcolor{darkblue}{\textbf{\ipa{ʁwɤ˥}}}}\kern2pt]} \hypertarget{Rw7\string_Ma1}{}
\markboth{\textcolor{darkblue}{\textbf{\ipa{ʁwɤ˧\textsubscript{a}}}}}{}
\textcolor{teal}{\mytextsc{verb}} \hspace{4pt} Tone: M\textsubscript{a}.
\textcolor{Sepia}{\selectlanguage{english}To make a heap of (e.g. cereals), to pile up.} \zh{堆 (例如:堆积泥土)。}  ¶ \textcolor{darkblue}{\textbf{\ipa{ɖɯ˧-ʁwɤ˧ tʰi˧-ʁwɤ˧}}} \textcolor{Sepia}{\selectlanguage{english}to make a heap, to heap together} \zh{堆在一起}  
 ¶ \textcolor{darkblue}{\textbf{\ipa{ɖɯ˧-ʁwɤ˧ tʰi˧-tɕɯ˥}}} \textcolor{Sepia}{\selectlanguage{english}to arrange into a heap} \zh{收拾成一堆}  
 ¶ \textcolor{darkblue}{\textbf{\ipa{tso˧\textasciitilde{}tso˧ | gɤ˩-ʁwɤ˥ lv̩˩}}} \textcolor{Sepia}{\selectlanguage{english}to pile up objects} \zh{东西堆起来}  
 ¶ \textcolor{darkblue}{\textbf{\ipa{ɖɯ˧-ʁwɤ˥-lv̩˩}}} \textcolor{Sepia}{\selectlanguage{english}to make into a heap (e.g. nuts, fruit... scattered around)} \zh{收拾成一堆(如:有果子散在地上,把它们堆在一起)}  
 ¶ \textcolor{darkblue}{\textbf{\ipa{tso˧\textasciitilde{}tso˧ ʁwɤ˩}}} \textcolor{Sepia}{\selectlanguage{english}to pile up things} \zh{东西堆在一起}  

\lhead{\firstmark}
\rhead{\botmark}

\subsection{\hspace{-0.5cm} {\Large \textcolor{darkblue}{\textbf{\ipa{ʁwɤ˧lɑ˩-bi˩}}}}\hspace{0.5cm}[\kern2pt{\textcolor{darkblue}{\textbf{\ipa{ʁwɤ˧lɑ˩bi˧}}}}\kern2pt]} \hypertarget{Rw7\string_MlA\string_B-bi\string_B1}{}
\markboth{\textcolor{darkblue}{\textbf{\ipa{ʁwɤ˧lɑ˩-bi˩}}}}{}
\textcolor{teal}{\mytextsc{noun}} \hspace{4pt} Tone: L\#-.
\textcolor{Sepia}{\selectlanguage{english}Walabie, a village of the Yongning plain. It is inhabited by both Na and Pumi.} \zh{瓦拉别(永宁的一个村落)。}  ¶ \textcolor{darkblue}{\textbf{\ipa{ə˧go˧-ʁwɤ˧, | ʁwɤ˧lɑ˩-bi˩, | bæ˧ʁwɤ˧, | tʰo˧tsʰe\#˥, | pi˧tsʰe˩-di˩, | pɤ˧dʑɤ˩-di˩, | ʁwɤ˧tv̩˧}}} \textcolor{Sepia}{\selectlanguage{english}Villages that one encounters as one leaves the plain of Yongning (away from the Lake); the first two are perceived as villages with a high proportion of Na members, and the third as a mostly Na village, whereas the next ones are Pumi (Prinmi).} \zh{永宁背向泸沽湖方向经过的村落。前两个村落拥有相当大的摩梭人口比例,第三个村落是摩梭村,最后一个是普米村。}  

\lhead{\firstmark}
\rhead{\botmark}

\subsection{\hspace{-0.5cm} {\Large \textcolor{darkblue}{\textbf{\ipa{ʁwɤ˧qʰv̩˧}}}}\hspace{0.5cm}[\kern2pt{\textcolor{darkblue}{\textbf{\ipa{ʁwɤ˧qʰv̩˧}}}}\kern2pt]} \hypertarget{Rw7\string_Mq\string_hv\string_=\string_M1}{}
\markboth{\textcolor{darkblue}{\textbf{\ipa{ʁwɤ˧qʰv̩˧}}}}{}
\textcolor{teal}{\mytextsc{noun}} \hspace{4pt} Tone: M.
\textcolor{Sepia}{\selectlanguage{english}Cave, cavern.} \zh{山洞。}  \zh{量词}: \textcolor{darkblue}{\textbf{\ipa{ɭɯ˧}}}  \mytextsc{clf}: \textcolor{darkblue}{\textbf{\ipa{ɭɯ˧}}} \textit{See:} \hyperlink{}{\textcolor{darkblue}{\textbf{\ipa{ʁwɤ˧qʰv̩˧dʑɯ\#˥}}}} 
\lhead{\firstmark}
\rhead{\botmark}

\subsection{\hspace{-0.5cm} {\Large \textcolor{darkblue}{\textbf{\ipa{ʁwɤ˧qʰv̩˧dʑɯ\#˥}}}}\hspace{0.5cm}[\kern2pt{\textcolor{darkblue}{\textbf{\ipa{ʁwɤ˧qʰv̩˧dʑɯ˧}}}}\kern2pt]} \hypertarget{Rw7\string_Mq\string_hv\string_=\string_Mdz£M\#\string_T1}{}
\markboth{\textcolor{darkblue}{\textbf{\ipa{ʁwɤ˧qʰv̩˧dʑɯ\#˥}}}}{}
\textcolor{teal}{\mytextsc{noun}} \hspace{4pt} Tone: \#H.
\textcolor{Sepia}{\selectlanguage{english}Cave, cavern.} \zh{山洞。}  \zh{量词}: \textcolor{darkblue}{\textbf{\ipa{ɭɯ˧}}}  \mytextsc{clf}: \textcolor{darkblue}{\textbf{\ipa{ɭɯ˧}}} \textit{See:} \hyperlink{}{\textcolor{darkblue}{\textbf{\ipa{ʁwɤ˧qʰv̩˧}}}} 
\lhead{\firstmark}
\rhead{\botmark}

\subsection{\hspace{-0.5cm} {\Large \textcolor{darkblue}{\textbf{\ipa{ʁwɤ˧\textasciitilde{}ʁwɤ˥\textsubscript{a}}}}}\hspace{0.5cm}[\kern2pt{\textcolor{darkblue}{\textbf{\ipa{ʁwɤ˩ʁwɤ˩˥}}}}\kern2pt]} \hypertarget{Rw7\string_M~Rw7\string_Ta1}{}
\markboth{\textcolor{darkblue}{\textbf{\ipa{ʁwɤ˧\textasciitilde{}ʁwɤ˥\textsubscript{a}}}}}{}
\textcolor{teal}{\mytextsc{verb}} \hspace{4pt} Tone: L\textsubscript{a}.
\textcolor{Sepia}{\selectlanguage{english}To discuss, to negociate.} \zh{商量。}  ¶ \textcolor{darkblue}{\textbf{\ipa{ɖɯ˧-ʁwɤ˧\textasciitilde{}ʁwɤ˥-ɻ̍˩}}} \textcolor{Sepia}{\selectlanguage{english}\mytextsc{delimitative} \string_ \mytextsc{red} \mytextsc{inceptive}} \zh{商量商量}  

\lhead{\firstmark}
\rhead{\botmark}

\subsection{\hspace{-0.5cm} {\Large \textcolor{darkblue}{\textbf{\ipa{ʁwɤ˧tv̩˧}}}}\hspace{0.5cm}[\kern2pt{\textcolor{darkblue}{\textbf{\ipa{ʁwɤ˧tv̩˧}}}}\kern2pt]} \hypertarget{Rw7\string_Mtv\string_=\string_M1}{}
\markboth{\textcolor{darkblue}{\textbf{\ipa{ʁwɤ˧tv̩˧}}}}{}
\textcolor{teal}{\mytextsc{noun}} \hspace{4pt} Tone: M.
\textcolor{Sepia}{\selectlanguage{english}A village near the Hot Springs.} \zh{温泉乡的一个村落。}  ¶ \textcolor{darkblue}{\textbf{\ipa{ʁwɤ˧tv̩˧-ʁwɤ˧}}} \textcolor{Sepia}{\selectlanguage{english}same meaning} \zh{同上}  
 ¶ \textcolor{darkblue}{\textbf{\ipa{ə˧go˧-ʁwɤ˧, | ʁwɤ˧lɑ˩-bi˩, | bæ˧ʁwɤ˧, | tʰo˧tsʰe\#˥, | pi˧tsʰe˩-di˩, | pɤ˧dʑɤ˩-di˩, | ʁwɤ˧tv̩˧}}} \textcolor{Sepia}{\selectlanguage{english}Villages that one encounters as one leaves the plain of Yongning (away from the Lake); the first two are perceived as villages with a high proportion of Na members, and the third as a mostly Na village, whereas the next ones are Pumi (Prinmi).} \zh{永宁背向泸沽湖方向经过的村落。前两个村落拥有相当大的摩梭人口比例,第三个村落是摩梭村,最后一个是普米村。}  
 ¶ \textcolor{darkblue}{\textbf{\ipa{ʁwɤ˧tv̩˧: | bɤ˧!}}} \textcolor{Sepia}{\selectlanguage{english}\textcolor{darkblue}{\textbf{\ipa{/ʁwɤ˧tv̩˧/}}} is a Pumi village!} \zh{fv:/ʁwɤ˧tv̩˧/是一个普米族村落!}  

\lhead{\firstmark}
\rhead{\botmark}

\subsection{\hspace{-0.5cm} {\Large \textcolor{darkblue}{\textbf{\ipa{ʁwɤ˧ʐv̩\#˥}}}}\hspace{0.5cm}[\kern2pt{\textcolor{darkblue}{\textbf{\ipa{ʁwɤ˧ʐv̩˧}}}}\kern2pt]} \hypertarget{Rw7\string_Mz`v\string_=\#\string_T1}{}
\markboth{\textcolor{darkblue}{\textbf{\ipa{ʁwɤ˧ʐv̩\#˥}}}}{}
\textcolor{teal}{\mytextsc{noun}} \hspace{4pt} Tone: \#H.
\textcolor{Sepia}{\selectlanguage{english}The village of Qiansuo.} \zh{前所。}  ¶ \textcolor{darkblue}{\textbf{\ipa{ʁwɤ˧ʐv̩˧-lo˩mæ˩}}} \textcolor{Sepia}{\selectlanguage{english}same meaning} \zh{同上}  
 ¶ \textcolor{darkblue}{\textbf{\ipa{ʁwɤ˧ʐv̩˧, | jɤ˧qʰɑ˧ dʑɤ˥; | hwɤ˧li˧-hɑ˧ mɤ˧-dʑo˧˥!}}} \textcolor{Sepia}{\selectlanguage{english}Adage: “In Qiansuo, bitter buckwheat grows beautifully; there's nothing for cats to eat!” Explanation: cats do not eat bitter buckwheat.} \zh{俗语:“前所,苦荞(庄稼)很好。猫,没得吃!”(说明:猫不吃苦荞。)}  
 ¶ \textcolor{darkblue}{\textbf{\ipa{ʁwɤ˧ʐv˧, | jɤ˧qʰɑ˧ dʑɤ˥, | hwɤ˧li˧˥ | hɑ˧ mɤ˧-dʑo˧!}}} \textcolor{Sepia}{\selectlanguage{english}as above} \zh{同上}  

\lhead{\firstmark}
\rhead{\botmark}

\subsection{\hspace{-0.5cm} {\Large \textcolor{darkblue}{\textbf{\ipa{*ʁwɤ˩\textsubscript{a}}}}}\hspace{0.5cm}[\kern2pt{\textcolor{darkblue}{\textbf{\ipa{ʁwɤ˩˥}}}}\kern2pt]} \hypertarget{*Rw7\string_Ba1}{}
\markboth{\textcolor{darkblue}{\textbf{\ipa{*ʁwɤ˩\textsubscript{a}}}}}{}
\textcolor{teal}{\mytextsc{verb}} \hspace{4pt} Tone: L\textsubscript{a}.
\textcolor{Sepia}{\selectlanguage{english}To negociate (monosyllabic root extracted from the reduplicated form).} \zh{商量(单音节)。} 
\lhead{\firstmark}
\rhead{\botmark}

\subsection{\hspace{-0.5cm} {\Large \textcolor{darkblue}{\textbf{\ipa{ʁwɤ˩ʁo˩}}}}\hspace{0.5cm}[\kern2pt{\textcolor{darkblue}{\textbf{\ipa{ʁwɤ˩ʁo˩˥}}}}\kern2pt]} \hypertarget{Rw7\string_BRo\string_B1}{}
\markboth{\textcolor{darkblue}{\textbf{\ipa{ʁwɤ˩ʁo˩}}}}{}
\textcolor{teal}{\mytextsc{noun}} \hspace{4pt} Tone: L.
\textcolor{Sepia}{\selectlanguage{english}Hillside.} \zh{山坡。}  ¶ \textcolor{darkblue}{\textbf{\ipa{ʁwɤ˩ʁo˩ dʑɤ˥bv̩˩ ə˩bi˩?}}} \textcolor{Sepia}{\selectlanguage{english}You want to come and have fun on the mountain? Would you like to go and take a stroll on the mountain?} \zh{去山上玩,好吗?}  
 \zh{量词}: \textcolor{darkblue}{\textbf{\ipa{ʁwɤ˧}}}  \mytextsc{clf}: \textcolor{darkblue}{\textbf{\ipa{ʁwɤ˧}}} 
\lhead{\firstmark}
\rhead{\botmark}

\newpage
\section*{\centering- \textcolor{darkblue}{\textbf{\ipa{s}}} -}
\subsection{\hspace{-0.5cm} {\Large \textcolor{darkblue}{\textbf{\ipa{sɑ˥}}} \textsubscript{1}}\hspace{0.5cm}[\kern2pt{\textcolor{darkblue}{\textbf{\ipa{sɑ˥}}}}\kern2pt]} \hypertarget{sA\string_T1}{}
\markboth{\textcolor{darkblue}{\textbf{\ipa{sɑ˥}}} \textsubscript{1}}{}
\textcolor{teal}{\mytextsc{noun}} \hspace{4pt} Tone: \#H.
\ding{202} \textcolor{Sepia}{\selectlanguage{english}Flax, \textit{Linum usitatissimum}.} \zh{亚麻。}  \zh{量词}: \textcolor{darkblue}{\textbf{\ipa{qʰwæ˧˥}}} \ding{203} \textcolor{Sepia}{\selectlanguage{english}Hemp, \textit{Cannabis sativa}.} \zh{火麻、胡麻。}  \zh{量词}: \textcolor{darkblue}{\textbf{\ipa{qʰwæ˧˥}}}  \mytextsc{clf}: \textcolor{darkblue}{\textbf{\ipa{qʰwæ˧˥}}} \textcolor{darkblue}{\textbf{\ipa{qʰwæ˧˥}}} 
\lhead{\firstmark}
\rhead{\botmark}

\subsection{\hspace{-0.5cm} {\Large \textcolor{darkblue}{\textbf{\ipa{sɑ˥}}} \textsubscript{2}}\hspace{0.5cm}[\kern2pt{\textcolor{darkblue}{\textbf{\ipa{sɑ˥}}}}\kern2pt]} \hypertarget{sA\string_T2}{}
\markboth{\textcolor{darkblue}{\textbf{\ipa{sɑ˥}}} \textsubscript{2}}{}
\textcolor{teal}{\mytextsc{classifier}} \hspace{4pt} Tone: H*.
\textcolor{Sepia}{\selectlanguage{english}A thing (no plural; only used in the negative construction “there is not a thing”).} \zh{量词:样,如:‘一样都没有’。}  ¶ \textcolor{darkblue}{\textbf{\ipa{ɖɯ˧-sɑ˥ | mɤ˧-dʑo˧!}}} \textcolor{Sepia}{\selectlanguage{english}There is simply nothing at all! (A polite statement made by the host when welcoming a guest for a meal, apologizing, in self-deprecation, for not offering a meal commensurate to one's wishes.)} \zh{一样也没有! / 没什么东西!(请客时的礼貌、自我贬低说法:请客人原谅菜不够丰盛)}  
\textit{See:} \hyperlink{}{\textcolor{darkblue}{\textbf{\ipa{so˥}}} \textsubscript{2}} 
\lhead{\firstmark}
\rhead{\botmark}

\subsection{\hspace{-0.5cm} {\Large \textcolor{darkblue}{\textbf{\ipa{sɑ˧bo\#˥}}}}\hspace{0.5cm}[\kern2pt{\textcolor{darkblue}{\textbf{\ipa{sɑ˧bo˧˥}}}}\kern2pt]} \hypertarget{sA\string_Mbo\#\string_T1}{}
\markboth{\textcolor{darkblue}{\textbf{\ipa{sɑ˧bo\#˥}}}}{}
\textcolor{teal}{\mytextsc{noun}} \hspace{4pt} Tone: \#H.
\textcolor{Sepia}{\selectlanguage{english}Distaff.} \zh{卷线杆、拉线棒。}  ¶ \textcolor{darkblue}{\textbf{\ipa{sɑ˧bo˧-di˧˥}}} \textcolor{Sepia}{\selectlanguage{english}same meaning} \zh{同上}  
 \zh{量词}: \textcolor{darkblue}{\textbf{\ipa{nɑ˧}}}  \mytextsc{clf}: \textcolor{darkblue}{\textbf{\ipa{nɑ˧}}} 
\lhead{\firstmark}
\rhead{\botmark}

\subsection{\hspace{-0.5cm} {\Large \textcolor{darkblue}{\textbf{\ipa{sɑ˧pʰv̩˧˥}}}}\hspace{0.5cm}[\kern2pt{\textcolor{darkblue}{\textbf{\ipa{sɑ˧pʰv̩˧}}}}\kern2pt]} \hypertarget{sA\string_Mp\string_hv\string_=\string_M\string_T1}{}
\markboth{\textcolor{darkblue}{\textbf{\ipa{sɑ˧pʰv̩˧˥}}}}{}
\textcolor{teal}{\mytextsc{noun}} \hspace{4pt} Tone: MH\#.
\textcolor{Sepia}{\selectlanguage{english}Thread of linen, \textit{Cannabis sativa}.} \zh{麻线。}  ¶ \textcolor{darkblue}{\textbf{\ipa{sɑ˧pʰv̩˧-sɑ˧jɤ˥}}} \textcolor{Sepia}{\selectlanguage{english}linen thread} \zh{麻线}  
 \zh{量词}: \textcolor{darkblue}{\textbf{\ipa{ɭɯ˧}}}  \mytextsc{clf}: \textcolor{darkblue}{\textbf{\ipa{ɭɯ˧}}} 
\lhead{\firstmark}
\rhead{\botmark}

\subsection{\hspace{-0.5cm} {\Large \textcolor{darkblue}{\textbf{\ipa{sɑ˧tɕɯ˧}}}}\hspace{0.5cm}[\kern2pt{\textcolor{darkblue}{\textbf{\ipa{sɑ˧tɕɯ˩}}}}\kern2pt]} \hypertarget{sA\string_Mts£M\string_M1}{}
\markboth{\textcolor{darkblue}{\textbf{\ipa{sɑ˧tɕɯ˧}}}}{}
\textcolor{teal}{\mytextsc{noun}} \hspace{4pt} Tone: M.
\textcolor{Sepia}{\selectlanguage{english}Vagina.} \zh{女生殖器。}  \zh{量词}: \textcolor{darkblue}{\textbf{\ipa{ɭɯ˧}}}  \mytextsc{clf}: \textcolor{darkblue}{\textbf{\ipa{ɭɯ˧}}} 
\lhead{\firstmark}
\rhead{\botmark}

\subsection{\hspace{-0.5cm} {\Large \textcolor{darkblue}{\textbf{\ipa{sɑ˧tsʰv̩˩}}}}\hspace{0.5cm}[\kern2pt{\textcolor{darkblue}{\textbf{\ipa{sɑ˧tsʰv̩˧}}}}\kern2pt]} \hypertarget{sA\string_Mts\string_hv\string_=\string_B1}{}
\markboth{\textcolor{darkblue}{\textbf{\ipa{sɑ˧tsʰv̩˩}}}}{}
\textcolor{teal}{\mytextsc{noun}} \hspace{4pt} Tone: L\#.
\textcolor{Sepia}{\selectlanguage{english}Vinegar.} \zh{酸醋(汉语借词)。}  Borrowing: Chinese  \zh{酸醋}
\textit{See:} \hyperlink{}{\textcolor{darkblue}{\textbf{\ipa{tsʰv̩˩˥}}}} 
\lhead{\firstmark}
\rhead{\botmark}

\subsection{\hspace{-0.5cm} {\Large \textcolor{darkblue}{\textbf{\ipa{sɑ˩mi˩}}}}\hspace{0.5cm}[\kern2pt{\textcolor{darkblue}{\textbf{\ipa{xxxx non-correspondance entre le nombre de morphèmes et le nombre de tons de morphèmes}}}}\kern2pt]} \hypertarget{sA\string_Bmi\string_B1}{}
\markboth{\textcolor{darkblue}{\textbf{\ipa{sɑ˩mi˩}}}}{}
\textcolor{teal}{\mytextsc{noun}} \hspace{4pt} Tone: L.
\textcolor{Sepia}{\selectlanguage{english}Marijuana, cannabis, \textit{Cannabis indica}.} \zh{大麻。}  ¶ \textcolor{darkblue}{\textbf{\ipa{sɑ˩mi˩-mæ˩ɻæ˥, | dzɯ˧-kv̩˩!}}} \textcolor{Sepia}{\selectlanguage{english}Cannabis oil is edible!} \zh{大麻油,是可以吃的!}  
 \zh{量词}: \textcolor{darkblue}{\textbf{\ipa{kɤ˧˥}}}  \mytextsc{clf}: \textcolor{darkblue}{\textbf{\ipa{kɤ˧˥}}} 
\lhead{\firstmark}
\rhead{\botmark}

\subsection{\hspace{-0.5cm} {\Large \textcolor{darkblue}{\textbf{\ipa{sɑ˩tsʰi˩}}} \textsubscript{1}}\hspace{0.5cm}[\kern2pt{\textcolor{darkblue}{\textbf{\ipa{sɑ˧tsʰi˧}}}}\kern2pt]} \hypertarget{sA\string_Bts\string_hi\string_B1}{}
\markboth{\textcolor{darkblue}{\textbf{\ipa{sɑ˩tsʰi˩}}} \textsubscript{1}}{}
\textcolor{teal}{\mytextsc{noun}} \hspace{4pt} Tone: L.
\textcolor{Sepia}{\selectlanguage{english}Oar.} \zh{桨。}  \zh{量词}: \textcolor{darkblue}{\textbf{\ipa{nɑ˧}}}  \mytextsc{clf}: \textcolor{darkblue}{\textbf{\ipa{nɑ˧}}} \textit{See:} \hyperlink{}{\textcolor{darkblue}{\textbf{\ipa{sɑ˩tsʰi˩}}} \textsubscript{2}} 
\lhead{\firstmark}
\rhead{\botmark}

\subsection{\hspace{-0.5cm} {\Large \textcolor{darkblue}{\textbf{\ipa{sɑ˩tsʰi˩}}} \textsubscript{2}}\hspace{0.5cm}[\kern2pt{\textcolor{darkblue}{\textbf{\ipa{sɑ˩tsʰi˩˥}}}}\kern2pt]} \hypertarget{sA\string_Bts\string_hi\string_B2}{}
\markboth{\textcolor{darkblue}{\textbf{\ipa{sɑ˩tsʰi˩}}} \textsubscript{2}}{}
\textcolor{teal}{\mytextsc{noun}} \hspace{4pt} Tone: L.
\textcolor{Sepia}{\selectlanguage{english}Wooden instrument resembling an oar, used to stir pigswill.} \zh{像桨的木头工具,来搅拌猪食。} \textit{See:} \hyperlink{}{\textcolor{darkblue}{\textbf{\ipa{sɑ˩tsʰi˩}}} \textsubscript{1}} 
\lhead{\firstmark}
\rhead{\botmark}

\subsection{\hspace{-0.5cm} {\Large \textcolor{darkblue}{\textbf{\ipa{sɑ˧˥}}}}\hspace{0.5cm}[\kern2pt{\textcolor{darkblue}{\textbf{\ipa{sɑ˧˥}}}}\kern2pt]} \hypertarget{sA\string_M\string_T1}{}
\markboth{\textcolor{darkblue}{\textbf{\ipa{sɑ˧˥}}}}{}
\textcolor{teal}{\mytextsc{verb}} \hspace{4pt} Tone: MH.
\textcolor{Sepia}{\selectlanguage{english}To deliver.} \zh{运送(货到目的地)。}  ¶ \textcolor{darkblue}{\textbf{\ipa{le˧-sɑ˧-tʰi˥-ki˩}}} \textcolor{Sepia}{\selectlanguage{english}to deliver (to someone's place)} \zh{送(东西到人家里)}  

\lhead{\firstmark}
\rhead{\botmark}

\subsection{\hspace{-0.5cm} {\Large \textcolor{darkblue}{\textbf{\ipa{sɑ˧˥\textsubscript{a}}}}}\hspace{0.5cm}[\kern2pt{\textcolor{darkblue}{\textbf{\ipa{sɑ˧˥}}}}\kern2pt]} \hypertarget{sA\string_M\string_Ta1}{}
\markboth{\textcolor{darkblue}{\textbf{\ipa{sɑ˧˥\textsubscript{a}}}}}{}
\textcolor{teal}{\mytextsc{classifier}} \hspace{4pt} Tone: MH\textsubscript{a}.
\textcolor{Sepia}{\selectlanguage{english}Classifier for salted, smoked hog legs.} \zh{量词:腊猪脚(烟熏腊猪蹄子)(一只)。}  ¶ \textcolor{darkblue}{\textbf{\ipa{ʂe˧sɑ˩ | ɖɯ˧-sɑ˧˥}}} \textcolor{Sepia}{\selectlanguage{english}a salted, smoked hog leg} \zh{一只腊猪脚}  

\lhead{\firstmark}
\rhead{\botmark}

\subsection{\hspace{-0.5cm} {\Large \textcolor{darkblue}{\textbf{\ipa{sæ˧tsʰɤ˩}}}}\hspace{0.5cm}[\kern2pt{\textcolor{darkblue}{\textbf{\ipa{sæ˩tsʰɤ˩˥}}}}\kern2pt]} \hypertarget{s\{\string_Mts\string_h7\string_B1}{}
\markboth{\textcolor{darkblue}{\textbf{\ipa{sæ˧tsʰɤ˩}}}}{}
\textcolor{teal}{\mytextsc{noun}} \hspace{4pt} Tone: L\#.
\textcolor{Sepia}{\selectlanguage{english}Pickled vegetables.} \zh{酸菜(汉语借词)、泡菜。}  Borrowing: Chinese  \zh{酸菜}

\lhead{\firstmark}
\rhead{\botmark}

\subsection{\hspace{-0.5cm} {\Large \textcolor{darkblue}{\textbf{\ipa{se˥}}}}\hspace{0.5cm}[\kern2pt{\textcolor{darkblue}{\textbf{\ipa{se˥}}}}\kern2pt]} \hypertarget{se\string_T1}{}
\markboth{\textcolor{darkblue}{\textbf{\ipa{se˥}}}}{}
\textcolor{teal}{\mytextsc{verb}} \hspace{4pt} Tone: H.
\textcolor{Sepia}{\selectlanguage{english}To walk.} \zh{走、走路。}  ¶ \textcolor{darkblue}{\textbf{\ipa{le˧-se˥-ze˩}}} \textcolor{Sepia}{\selectlanguage{english}\mytextsc{accomp} \string_ \mytextsc{pfv}} \zh{走了}  
 ¶ \textcolor{darkblue}{\textbf{\ipa{se˧-ho˥-ze˩!}}} \textcolor{Sepia}{\selectlanguage{english}[The baby] will soon walk / will soon be able to walk!} \zh{(婴儿)很快就学会走路了!}  
 ¶ \textcolor{darkblue}{\textbf{\ipa{ʐɤ˩mi˩-qo˥ | so˩-hɑ̃˩ se˩˥}}} \textcolor{Sepia}{\selectlanguage{english}to spend three days on the road, to make a trip that lasts three days} \zh{走在路上三天时间、走三天}  

\lhead{\firstmark}
\rhead{\botmark}

\subsection{\hspace{-0.5cm} {\Large \textcolor{darkblue}{\textbf{\ipa{se˧}}}}\hspace{0.5cm}[\kern2pt{\textcolor{darkblue}{\textbf{\ipa{se˩˥}}}}\kern2pt]} \hypertarget{se\string_M1}{}
\markboth{\textcolor{darkblue}{\textbf{\ipa{se˧}}}}{}
\textcolor{teal}{\mytextsc{noun}} \hspace{4pt} Tone: M.
\textcolor{Sepia}{\selectlanguage{english}Himalayan goral (\textit{Naemorhedus goral}), blue sheep.} \zh{岩羊。}  \zh{量词}: \textcolor{darkblue}{\textbf{\ipa{pʰo˧˥}}}  \mytextsc{clf}: \textcolor{darkblue}{\textbf{\ipa{pʰo˧˥}}} 
\lhead{\firstmark}
\rhead{\botmark}

\subsection{\hspace{-0.5cm} {\Large \textcolor{darkblue}{\textbf{\ipa{se˧gi\#˥}}}}\hspace{0.5cm}[\kern2pt{\textcolor{darkblue}{\textbf{\ipa{se˧gi˧}}}}\kern2pt]} \hypertarget{se\string_Mgi\#\string_T1}{}
\markboth{\textcolor{darkblue}{\textbf{\ipa{se˧gi\#˥}}}}{}
\textcolor{teal}{\mytextsc{noun}} \hspace{4pt} Tone: \#H.
\textcolor{Sepia}{\selectlanguage{english}The Tibetan name of the mountain \textcolor{darkblue}{\textbf{\ipa{/kɤ˧mv̩˧˥/}}} (Chinese name: Gemu).} \zh{格姆山的藏语名字。}  ¶ \textcolor{darkblue}{\textbf{\ipa{se˧gi˧-kɤ˩mv̩˩}}} \textcolor{Sepia}{\selectlanguage{english}same meaning} \zh{同上}  

\lhead{\firstmark}
\rhead{\botmark}

\subsection{\hspace{-0.5cm} {\Large \textcolor{darkblue}{\textbf{\ipa{se˧kʰɯ˩}}}}\hspace{0.5cm}[\kern2pt{\textcolor{darkblue}{\textbf{\ipa{se˩kʰɯ˥}}}}\kern2pt]} \hypertarget{se\string_Mk\string_hM\string_B1}{}
\markboth{\textcolor{darkblue}{\textbf{\ipa{se˧kʰɯ˩}}}}{}
\textcolor{teal}{\mytextsc{noun}} \hspace{4pt} Tone: L\#.
\textcolor{Sepia}{\selectlanguage{english}Satin.} \zh{缎子。}  ¶ \textcolor{darkblue}{\textbf{\ipa{se˧kʰɯ˩-ʁo˩ni˩}}} \textcolor{Sepia}{\selectlanguage{english}satin headdressxxxx} \zh{缎子发带}  
 \zh{量词}: \textcolor{darkblue}{\textbf{\ipa{kʰɯ˩}}}  \mytextsc{clf}: \textcolor{darkblue}{\textbf{\ipa{kʰɯ˩}}} 
\lhead{\firstmark}
\rhead{\botmark}

\subsection{\hspace{-0.5cm} {\Large \textcolor{darkblue}{\textbf{\ipa{se˧mi\#˥}}}}\hspace{0.5cm}[\kern2pt{\textcolor{darkblue}{\textbf{\ipa{se˧mi˧}}}}\kern2pt]} \hypertarget{se\string_Mmi\#\string_T1}{}
\markboth{\textcolor{darkblue}{\textbf{\ipa{se˧mi\#˥}}}}{}
\textcolor{teal}{\mytextsc{noun}} \hspace{4pt} Tone: \#H.
\textcolor{Sepia}{\selectlanguage{english}Female goral (\textit{Naemorhedus goral}), female blue sheep.} \zh{母岩羊。}  \zh{量词}: \textcolor{darkblue}{\textbf{\ipa{mi˩}}}  \mytextsc{clf}: \textcolor{darkblue}{\textbf{\ipa{mi˩}}} 
\lhead{\firstmark}
\rhead{\botmark}

\subsection{\hspace{-0.5cm} {\Large \textcolor{darkblue}{\textbf{\ipa{se˧nɑ\#˥}}}}\hspace{0.5cm}[\kern2pt{\textcolor{darkblue}{\textbf{\ipa{se˧nɑ˩}}}}\kern2pt]} \hypertarget{se\string_MnA\#\string_T1}{}
\markboth{\textcolor{darkblue}{\textbf{\ipa{se˧nɑ\#˥}}}}{}
\textcolor{teal}{\mytextsc{adjective}} \hspace{4pt} Tone: \#H.
\textcolor{Sepia}{\selectlanguage{english}Stingy, miserly.} \zh{吝啬。}  ¶ \textcolor{darkblue}{\textbf{\ipa{ʈʂʰɯ˧ | se˧nɑ˧-hĩ˧ ɖɯ˧-v̩˧ ɲi˩!}}} \textcolor{Sepia}{\selectlanguage{english}It's a stingy person!} \zh{他是一个吝啬的人!}  
 ¶ \textcolor{darkblue}{\textbf{\ipa{ʈʂʰɯ˧ | ə˧-se˧nɑ˧? - se˧nɑ˧ | ʐwæ˩˥!}}} \textcolor{Sepia}{\selectlanguage{english}Is he stingy? - Oh yes, very much so!} \zh{他吝啬吗? - 非常吝啬!}  

\lhead{\firstmark}
\rhead{\botmark}

\subsection{\hspace{-0.5cm} {\Large \textcolor{darkblue}{\textbf{\ipa{se˧pʰɤ˧}}}}\hspace{0.5cm}[\kern2pt{\textcolor{darkblue}{\textbf{\ipa{se˧pʰɤ˩}}}}\kern2pt]} \hypertarget{se\string_Mp\string_h7\string_M1}{}
\markboth{\textcolor{darkblue}{\textbf{\ipa{se˧pʰɤ˧}}}}{}
\textcolor{teal}{\mytextsc{noun}} \hspace{4pt} Tone: M.
\textcolor{Sepia}{\selectlanguage{english}Fuss.} \zh{大惊小怪,麻烦。}  ¶ \textcolor{darkblue}{\textbf{\ipa{se˧pʰɤ˧ ʝi˧}}} \textcolor{Sepia}{\selectlanguage{english}to make a big fuss about something} \zh{小事大作}  
 \zh{量词}: \textcolor{darkblue}{\textbf{\ipa{kʰwɤ˥}}}  \mytextsc{clf}: \textcolor{darkblue}{\textbf{\ipa{kʰwɤ˥}}} 
\lhead{\firstmark}
\rhead{\botmark}

\subsection{\hspace{-0.5cm} {\Large \textcolor{darkblue}{\textbf{\ipa{se˧pʰv̩\#˥}}}}\hspace{0.5cm}[\kern2pt{\textcolor{darkblue}{\textbf{\ipa{se˧pʰv̩˩}}}}\kern2pt]} \hypertarget{se\string_Mp\string_hv\string_=\#\string_T1}{}
\markboth{\textcolor{darkblue}{\textbf{\ipa{se˧pʰv̩\#˥}}}}{}
\textcolor{teal}{\mytextsc{noun}} \hspace{4pt} Tone: \#H.
\textcolor{Sepia}{\selectlanguage{english}Male goral (\textit{Naemorhedus goral}), male blue sheep.} \zh{公岩羊。}  \zh{量词}: \textcolor{darkblue}{\textbf{\ipa{mi˩}}}  \mytextsc{clf}: \textcolor{darkblue}{\textbf{\ipa{mi˩}}} 
\lhead{\firstmark}
\rhead{\botmark}

\subsection{\hspace{-0.5cm} {\Large \textcolor{darkblue}{\textbf{\ipa{se˧ʂɯ˩}}}}\hspace{0.5cm}[\kern2pt{\textcolor{darkblue}{\textbf{\ipa{se˧ʂɯ˧}}}}\kern2pt]} \hypertarget{se\string_Ms`M\string_B1}{}
\markboth{\textcolor{darkblue}{\textbf{\ipa{se˧ʂɯ˩}}}}{}
\textcolor{teal}{\mytextsc{verb}} \hspace{4pt} Tone: L\#.
\textcolor{Sepia}{\selectlanguage{english}To waste.} \zh{浪费。}  ¶ \textcolor{darkblue}{\textbf{\ipa{ɖwæ˧˥ | se˧ʂɯ˩!}}} \textcolor{Sepia}{\selectlanguage{english}It's a waste!} \zh{很浪费! / 太浪费了!}  

\lhead{\firstmark}
\rhead{\botmark}

\subsection{\hspace{-0.5cm} {\Large \textcolor{darkblue}{\textbf{\ipa{se˧tʰo˩}}}}\hspace{0.5cm}[\kern2pt{\textcolor{darkblue}{\textbf{\ipa{se˩tʰo˩˥}}}}\kern2pt]} \hypertarget{se\string_Mt\string_ho\string_B1}{}
\markboth{\textcolor{darkblue}{\textbf{\ipa{se˧tʰo˩}}}}{}
\textcolor{teal}{\mytextsc{noun}} \hspace{4pt} Tone: L.
\textcolor{Sepia}{\selectlanguage{english}Tenon.} \zh{榫头(汉语借词)。}  Borrowing: Chinese  \zh{榫头}
 ¶ \textcolor{darkblue}{\textbf{\ipa{se˧tʰo˩ | ɖɯ˧-ɭɯ˧}}} \textcolor{Sepia}{\selectlanguage{english}a tenon} \zh{一个榫头}  
 \zh{量词}: \textcolor{darkblue}{\textbf{\ipa{ɭɯ˧}}}  \mytextsc{clf}: \textcolor{darkblue}{\textbf{\ipa{ɭɯ˧}}} 
\lhead{\firstmark}
\rhead{\botmark}

\subsection{\hspace{-0.5cm} {\Large \textcolor{darkblue}{\textbf{\ipa{se˧zo\#˥}}}}\hspace{0.5cm}[\kern2pt{\textcolor{darkblue}{\textbf{\ipa{se˧zo˧}}}}\kern2pt]} \hypertarget{se\string_Mzo\#\string_T1}{}
\markboth{\textcolor{darkblue}{\textbf{\ipa{se˧zo\#˥}}}}{}
\textcolor{teal}{\mytextsc{noun}} \hspace{4pt} Tone: \#H.
\textcolor{Sepia}{\selectlanguage{english}Baby goral, baby blue sheep.} \zh{小岩羊。}  \zh{量词}: \textcolor{darkblue}{\textbf{\ipa{ɭɯ˧}}}  \mytextsc{clf}: \textcolor{darkblue}{\textbf{\ipa{ɭɯ˧}}} 
\lhead{\firstmark}
\rhead{\botmark}

\subsection{\hspace{-0.5cm} {\Large \textcolor{darkblue}{\textbf{\ipa{se˧ʐɯ˩}}}}\hspace{0.5cm}[\kern2pt{\textcolor{darkblue}{\textbf{\ipa{se˧ʐɯ˧}}}}\kern2pt]} \hypertarget{se\string_Mz`M\string_B1}{}
\markboth{\textcolor{darkblue}{\textbf{\ipa{se˧ʐɯ˩}}}}{}
\textcolor{teal}{\mytextsc{noun}} \hspace{4pt} Tone: L\#.
\textcolor{Sepia}{\selectlanguage{english}Birthday.} \zh{生日(汉语借词)。}  Borrowing: Chinese  \zh{生日}
 ¶ \textcolor{darkblue}{\textbf{\ipa{se˧ʐɯ˩ ko˩}}} \textcolor{Sepia}{\selectlanguage{english}to celebrate a birthday} \zh{过生日}  

\lhead{\firstmark}
\rhead{\botmark}

\subsection{\hspace{-0.5cm} {\Large \textcolor{darkblue}{\textbf{\ipa{‑se˩}}}}\hspace{0.5cm}[\kern2pt{\textcolor{darkblue}{\textbf{\ipa{se˩˥}}}}\kern2pt]} \hypertarget{‑se\string_B1}{}
\markboth{\textcolor{darkblue}{\textbf{\ipa{‑se˩}}}}{}
\textcolor{teal}{\mytextsc{suffix}} \hspace{4pt} Tone: L.
\textcolor{Sepia}{\selectlanguage{english}Suffix indicating the completion of an action: the action has reached its end.} \zh{\mytextsc{完成。}}  ¶ \textcolor{darkblue}{\textbf{\ipa{se˩-ze˥!}}} \textcolor{Sepia}{\selectlanguage{english}It's finished! / It's completed!} \zh{完了!}  
 ¶ \textcolor{darkblue}{\textbf{\ipa{no˧ | tʰi˧-dzi˩-kʰɯ˩-se˩-dʑo˩, | dʑɯ˩-tsʰi˧ ɖɯ˧-qʰwɤ˧ pʰv̩˥ | tʰi˧-ki˧!}}} \textcolor{Sepia}{\selectlanguage{english}After you have been seated, (I) pour out a bowl of hot water (for you).} \zh{让(你)坐下以后,(我)给你倒一杯开水。}  

\lhead{\firstmark}
\rhead{\botmark}

\subsection{\hspace{-0.5cm} {\Large \textcolor{darkblue}{\textbf{\ipa{se˩\textsubscript{a}}}}}\hspace{0.5cm}[\kern2pt{\textcolor{darkblue}{\textbf{\ipa{se˥}}}}\kern2pt]} \hypertarget{se\string_Ba1}{}
\markboth{\textcolor{darkblue}{\textbf{\ipa{se˩\textsubscript{a}}}}}{}
\textcolor{teal}{\mytextsc{verb}} \hspace{4pt} Tone: L\textsubscript{a}.
\textcolor{Sepia}{\selectlanguage{english}To finish, to complete.} \zh{完成。}  ¶ \textcolor{darkblue}{\textbf{\ipa{le˧-ʝi˥ | le˧-se˩-ze˩!}}} \textcolor{Sepia}{\selectlanguage{english}It's done and finished! / (I) have finished (the job)!} \zh{做完了! / 完成了!}  
 ¶ \textcolor{darkblue}{\textbf{\ipa{le˧-se˧\textasciitilde{}se˥-ze˩!}}} \textcolor{Sepia}{\selectlanguage{english}It's finished, it's completed! / It's now over and done with!} \zh{完成了!}  
 ¶ \textcolor{darkblue}{\textbf{\ipa{mɤ˧-se˩}}} \textcolor{Sepia}{\selectlanguage{english}It's not finished!} \zh{没有完!}  
 ¶ \textcolor{darkblue}{\textbf{\ipa{se˩˥ | -dʑo˩, | se˩-mɤ˩-tʰɑ˩˥! | dʑɤ˩˥ | -dʑo˩, | dʑɤ˩-kʰɯ˧ tʰɑ˥!}}} \textcolor{Sepia}{\selectlanguage{english}A comment about linguistic documentation, summarizing an explanation provided by a student: “One cannot complete the task (=one cannot collect everything: every single word, every single turn of phrase, every single story...); but one can do nice things (=collect stories which constitute complete, interesting documents)”.} \zh{(想做)完,但是没办法做完!不过最后还是可以做得很好!(情景:谈及收集语言的工作)}  

\lhead{\firstmark}
\rhead{\botmark}

\subsection{\hspace{-0.5cm} {\Large \textcolor{darkblue}{\textbf{\ipa{se˩di˩}}}}\hspace{0.5cm}[\kern2pt{\textcolor{darkblue}{\textbf{\ipa{se˧di˧}}}}\kern2pt]} \hypertarget{se\string_Bdi\string_B1}{}
\markboth{\textcolor{darkblue}{\textbf{\ipa{se˩di˩}}}}{}
\textcolor{teal}{\mytextsc{noun}} \hspace{4pt} Tone: L.
\textcolor{Sepia}{\selectlanguage{english}Saw.} \zh{锯。}  ¶ \textcolor{darkblue}{\textbf{\ipa{se˩di˩˥ | ɖɯ˩-hĩ˩˥ | ɖɯ˧-nɑ˧}}} \textcolor{Sepia}{\selectlanguage{english}a large saw} \zh{一把大锯}  
 ¶ \textcolor{darkblue}{\textbf{\ipa{se˩di˩˥ | tɕi˩-hĩ˩˥ | ɖɯ˧-nɑ˧}}} \textcolor{Sepia}{\selectlanguage{english}a small saw} \zh{一把小锯}  
 ¶ \textcolor{darkblue}{\textbf{\ipa{se˩di˩˥ | ɬi˧-hĩ˧ | ɖɯ˧-nɑ˩}}} \textcolor{Sepia}{\selectlanguage{english}a medium-sized saw} \zh{中间大小的锯子}  
 \zh{量词}: \textcolor{darkblue}{\textbf{\ipa{nɑ˧}}}  \mytextsc{clf}: \textcolor{darkblue}{\textbf{\ipa{nɑ˧}}} 
\lhead{\firstmark}
\rhead{\botmark}

\subsection{\hspace{-0.5cm} {\Large \textcolor{darkblue}{\textbf{\ipa{se˩gwɤ˩mi˥}}}}\hspace{0.5cm}[\kern2pt{\textcolor{darkblue}{\textbf{\ipa{se˧gwɤ˧mi˧}}}}\kern2pt]} \hypertarget{se\string_Bgw7\string_Bmi\string_T1}{}
\markboth{\textcolor{darkblue}{\textbf{\ipa{se˩gwɤ˩mi˥}}}}{}
\textcolor{teal}{\mytextsc{noun}} \hspace{4pt} Tone: L+H\#.
\textcolor{Sepia}{\selectlanguage{english}Vulture. This term is not restricted to female vultures, and hence does not provide an indication on sex.} \zh{雕(不仅来指母雕)。}  ¶ \textcolor{darkblue}{\textbf{\ipa{se˩gwɤ˩mi˥-pʰv̩˩}}} \textcolor{Sepia}{\selectlanguage{english}male vulture} \zh{公雕}  
 ¶ \textcolor{darkblue}{\textbf{\ipa{se˩gwɤ˩mi˥-zo˩}}} \textcolor{Sepia}{\selectlanguage{english}baby vulture} \zh{小雕}  
 ¶ \textcolor{darkblue}{\textbf{\ipa{se˩gwɤ˩mi˥-ʈʂʰɯ˩, | mi˩ ɲi˥!}}} \textcolor{Sepia}{\selectlanguage{english}This vulture is a female!} \zh{这只雕是母的!}  
 \zh{量词}: \textcolor{darkblue}{\textbf{\ipa{mi˩}}}  \mytextsc{clf}: \textcolor{darkblue}{\textbf{\ipa{mi˩}}} 
\lhead{\firstmark}
\rhead{\botmark}

\subsection{\hspace{-0.5cm} {\Large \textcolor{darkblue}{\textbf{\ipa{sɤ˥}}}}\hspace{0.5cm}[\kern2pt{\textcolor{darkblue}{\textbf{\ipa{sɤ˥}}}}\kern2pt]} \hypertarget{s7\string_T1}{}
\markboth{\textcolor{darkblue}{\textbf{\ipa{sɤ˥}}}}{}
\textcolor{teal}{\mytextsc{noun}} \hspace{4pt} Tone: \#H.
\textcolor{Sepia}{\selectlanguage{english}Blood.} \zh{血。}  \zh{量词}: \textcolor{darkblue}{\textbf{\ipa{ʈʰɤ˥}}}  \mytextsc{clf}: \textcolor{darkblue}{\textbf{\ipa{ʈʰɤ˥}}} 
\lhead{\firstmark}
\rhead{\botmark}

\subsection{\hspace{-0.5cm} {\Large \textcolor{darkblue}{\textbf{\ipa{sɤ˧ɭɯ˩}}}}\hspace{0.5cm}[\kern2pt{\textcolor{darkblue}{\textbf{\ipa{sɤ˧ɭɯ˧}}}}\kern2pt]} \hypertarget{s7\string_Ml\string_RM\string_B1}{}
\markboth{\textcolor{darkblue}{\textbf{\ipa{sɤ˧ɭɯ˩}}}}{}
\textcolor{teal}{\mytextsc{noun}} \hspace{4pt} Tone: L\#.
\textcolor{Sepia}{\selectlanguage{english}Pear.} \zh{梨子。}  \zh{量词}: \textcolor{darkblue}{\textbf{\ipa{kʰɤ˧˥}}} \textcolor{darkblue}{\textbf{\ipa{ɭɯ˧}}}  \mytextsc{clf}: \textcolor{darkblue}{\textbf{\ipa{kʰɤ˧˥}}} \textcolor{darkblue}{\textbf{\ipa{ɭɯ˧}}} 
\lhead{\firstmark}
\rhead{\botmark}

\subsection{\hspace{-0.5cm} {\Large \textcolor{darkblue}{\textbf{\ipa{sɤ˧sɤ˧˥}}}}\hspace{0.5cm}[\kern2pt{\textcolor{darkblue}{\textbf{\ipa{sɤ˧sɤ˧˥}}}}\kern2pt]} \hypertarget{s7\string_Ms7\string_M\string_T1}{}
\markboth{\textcolor{darkblue}{\textbf{\ipa{sɤ˧sɤ˧˥}}}}{}
\textcolor{teal}{\mytextsc{adjective}} \hspace{4pt} Tone: MH\#.
\textcolor{Sepia}{\selectlanguage{english}Pleasant (circumstances).} \zh{舒畅。}  ¶ \textcolor{darkblue}{\textbf{\ipa{si˧dzi˩-ʈʰæ˩qo˩dzi˩, | sɤ˧sɤ˧˥ | ʐwæ˩˥!}}} \textcolor{Sepia}{\selectlanguage{english}Being seated under this tree is especially pleasant!} \zh{在树下坐着,感到很舒畅!}  
 ¶ \textcolor{darkblue}{\textbf{\ipa{ʈʂʰɯ˧-ɳɯ˧ | ɖɯ˧-ɖʐɯ˩ gwɤ˩-dʑo˩, | sɤ˧sɤ˧˥ | ʐwæ˩˥!}}} \textcolor{Sepia}{\selectlanguage{english}He has sung for a while; it was really pleasant!} \zh{他唱了一会,真舒畅!}  

\lhead{\firstmark}
\rhead{\botmark}

\subsection{\hspace{-0.5cm} {\Large \textcolor{darkblue}{\textbf{\ipa{sɤ˧tʰo˧˥}}}}\hspace{0.5cm}[\kern2pt{\textcolor{darkblue}{\textbf{\ipa{sɤ˧tʰo˧˥}}}}\kern2pt]} \hypertarget{s7\string_Mt\string_ho\string_M\string_T1}{}
\markboth{\textcolor{darkblue}{\textbf{\ipa{sɤ˧tʰo˧˥}}}}{}
\textcolor{teal}{\mytextsc{noun}} \hspace{4pt} Tone: MH\#.
\textcolor{Sepia}{\selectlanguage{english}A type of pine.} \zh{一种松树。} Local Chinese dialect:\zh{阔松。} ¶ \textcolor{darkblue}{\textbf{\ipa{sɤ˧tʰo˧-dzi˧˥}}} \textcolor{Sepia}{\selectlanguage{english}same meaning} \zh{同上}  
 \zh{量词}: \textcolor{darkblue}{\textbf{\ipa{dzi˩}}}  \mytextsc{clf}: \textcolor{darkblue}{\textbf{\ipa{dzi˩}}} 
\lhead{\firstmark}
\rhead{\botmark}

\subsection{\hspace{-0.5cm} {\Large \textcolor{darkblue}{\textbf{\ipa{sɤ˧tsi˥}}}}\hspace{0.5cm}[\kern2pt{\textcolor{darkblue}{\textbf{\ipa{sɤ˧tsi˥}}}}\kern2pt]} \hypertarget{s7\string_Mtsi\string_T1}{}
\markboth{\textcolor{darkblue}{\textbf{\ipa{sɤ˧tsi˥}}}}{}
\textcolor{teal}{\mytextsc{noun}} \hspace{4pt} Tone: H\#.
\textcolor{Sepia}{\selectlanguage{english}Vein.} \zh{血管。}  \zh{量词}: \textcolor{darkblue}{\textbf{\ipa{kʰɯ˩}}}  \mytextsc{clf}: \textcolor{darkblue}{\textbf{\ipa{kʰɯ˩}}} 
\lhead{\firstmark}
\rhead{\botmark}

\subsection{\hspace{-0.5cm} {\Large \textcolor{darkblue}{\textbf{\ipa{sɤ˩˥}}}}\hspace{0.5cm}[\kern2pt{\textcolor{darkblue}{\textbf{\ipa{sɤ˩˥}}}}\kern2pt]} \hypertarget{s7\string_B\string_T1}{}
\markboth{\textcolor{darkblue}{\textbf{\ipa{sɤ˩˥}}}}{}
\textcolor{teal}{\mytextsc{noun}} \hspace{4pt} Tone: LH.
\textcolor{Sepia}{\selectlanguage{english}Mole; pigmented naevus.} \zh{黑痣。}  \zh{量词}: \textcolor{darkblue}{\textbf{\ipa{ɭɯ˧}}}  \mytextsc{clf}: \textcolor{darkblue}{\textbf{\ipa{ɭɯ˧}}} 
\lhead{\firstmark}
\rhead{\botmark}

\subsection{\hspace{-0.5cm} {\Large \textcolor{darkblue}{\textbf{\ipa{si˥}}}}\hspace{0.5cm}[\kern2pt{\textcolor{darkblue}{\textbf{\ipa{si˧˥}}}}\kern2pt]} \hypertarget{si\string_T1}{}
\markboth{\textcolor{darkblue}{\textbf{\ipa{si˥}}}}{}
\textcolor{teal}{\mytextsc{noun}} \hspace{4pt} Tone: \#H.
\textcolor{Sepia}{\selectlanguage{english}Wood.} \zh{木头。}  ¶ \textcolor{darkblue}{\textbf{\ipa{si˧-mo˩}}} \textcolor{Sepia}{\selectlanguage{english}dead wood} \zh{枯木}  
 \zh{量词}: \textcolor{darkblue}{\textbf{\ipa{kɤ˧˥}}}  \mytextsc{clf}: \textcolor{darkblue}{\textbf{\ipa{kɤ˧˥}}} 
\lhead{\firstmark}
\rhead{\botmark}

\subsection{\hspace{-0.5cm} {\Large \textcolor{darkblue}{\textbf{\ipa{si˧\textsubscript{a}}}}}\hspace{0.5cm}[\kern2pt{\textcolor{darkblue}{\textbf{\ipa{si˥}}}}\kern2pt]} \hypertarget{si\string_Ma1}{}
\markboth{\textcolor{darkblue}{\textbf{\ipa{si˧\textsubscript{a}}}}}{}
\textcolor{teal}{\mytextsc{verb}} \hspace{4pt} Tone: M\textsubscript{a}.
\textcolor{Sepia}{\selectlanguage{english}To choose, to select.} \zh{挑选。}  ¶ \textcolor{darkblue}{\textbf{\ipa{le˧-si˧-ze˧}}} \textcolor{Sepia}{\selectlanguage{english}\mytextsc{accomp} \string_ \mytextsc{pfv}} \zh{选了}  
 ¶ \textcolor{darkblue}{\textbf{\ipa{no˧ si˧-bi˧!}}} \textcolor{Sepia}{\selectlanguage{english}You choose! / Go ahead and choose!} \zh{你要选!}  
 ¶ \textcolor{darkblue}{\textbf{\ipa{njɤ˧-ɳɯ˧ si˧-bi˧!}}} \textcolor{Sepia}{\selectlanguage{english}I choose! / Let me choose!} \zh{是我来选!}  
 ¶ \textcolor{darkblue}{\textbf{\ipa{le˧-si˥\textasciitilde{}si˩}}} \textcolor{Sepia}{\selectlanguage{english}\mytextsc{accomp} \string_ \mytextsc{red}} \zh{\mytextsc{accomp} \string_ \mytextsc{red}}  
 ¶ \textcolor{darkblue}{\textbf{\ipa{tso˧\textasciitilde{}tso˧ si˩}}} \textcolor{Sepia}{\selectlanguage{english}to choose things} \zh{选东西}  
 ¶ \textcolor{darkblue}{\textbf{\ipa{tso˧\textasciitilde{}tso˧ si˧\textasciitilde{}si˥}}} \textcolor{Sepia}{\selectlanguage{english}to choose things} \zh{选选东西}  
 ¶ \textcolor{darkblue}{\textbf{\ipa{dʑɤ˩-hĩ˥ | si˧}}} \textcolor{Sepia}{\selectlanguage{english}to choose good ones} \zh{挑好的}  

\lhead{\firstmark}
\rhead{\botmark}

\subsection{\hspace{-0.5cm} {\Large \textcolor{darkblue}{\textbf{\ipa{si˧bv̩˧}}}}\hspace{0.5cm}[\kern2pt{\textcolor{darkblue}{\textbf{\ipa{si˧bv̩˧}}}}\kern2pt]} \hypertarget{si\string_Mbv\string_=\string_M1}{}
\markboth{\textcolor{darkblue}{\textbf{\ipa{si˧bv̩˧}}}}{}
\textcolor{teal}{\mytextsc{noun}} \hspace{4pt} Tone: M.
\textcolor{Sepia}{\selectlanguage{english}Evil spirit.} \zh{鬼。}  \zh{量词}: \textcolor{darkblue}{\textbf{\ipa{v̩˧}}}  \mytextsc{clf}: \textcolor{darkblue}{\textbf{\ipa{v̩˧}}} 
\lhead{\firstmark}
\rhead{\botmark}

\subsection{\hspace{-0.5cm} {\Large \textcolor{darkblue}{\textbf{\ipa{si˧bv̩˧-mi\#˥}}}}\hspace{0.5cm}[\kern2pt{\textcolor{darkblue}{\textbf{\ipa{xxxx non-correspondance entre le nombre de morphèmes et le nombre de tons de morphèmes}}}}\kern2pt]} \hypertarget{si\string_Mbv\string_=\string_M-mi\#\string_T1}{}
\markboth{\textcolor{darkblue}{\textbf{\ipa{si˧bv̩˧-mi\#˥}}}}{}
\textcolor{teal}{\mytextsc{noun}} \hspace{4pt} Tone: \#H.
\textcolor{Sepia}{\selectlanguage{english}Evil spirit (female).} \zh{妖精。}  \zh{量词}: \textcolor{darkblue}{\textbf{\ipa{v̩˧}}}  \mytextsc{clf}: \textcolor{darkblue}{\textbf{\ipa{v̩˧}}} 
\lhead{\firstmark}
\rhead{\botmark}

\subsection{\hspace{-0.5cm} {\Large \textcolor{darkblue}{\textbf{\ipa{si˧bv̩˧-zo\#˥}}}}\hspace{0.5cm}[\kern2pt{\textcolor{darkblue}{\textbf{\ipa{xxxx non-correspondance entre le nombre de morphèmes et le nombre de tons de morphèmes}}}}\kern2pt]} \hypertarget{si\string_Mbv\string_=\string_M-zo\#\string_T1}{}
\markboth{\textcolor{darkblue}{\textbf{\ipa{si˧bv̩˧-zo\#˥}}}}{}
\textcolor{teal}{\mytextsc{noun}} \hspace{4pt} Tone: \#H.
\textcolor{Sepia}{\selectlanguage{english}Evil spirit (masculine).} \zh{鬼。}  \zh{量词}: \textcolor{darkblue}{\textbf{\ipa{v̩˧}}}  \mytextsc{clf}: \textcolor{darkblue}{\textbf{\ipa{v̩˧}}} 
\lhead{\firstmark}
\rhead{\botmark}

\subsection{\hspace{-0.5cm} {\Large \textcolor{darkblue}{\textbf{\ipa{si˧ɕi˧˥}}}}\hspace{0.5cm}[\kern2pt{\textcolor{darkblue}{\textbf{\ipa{si˧ɕi˧}}}}\kern2pt]} \hypertarget{si\string_Ms£i\string_M\string_T1}{}
\markboth{\textcolor{darkblue}{\textbf{\ipa{si˧ɕi˧˥}}}}{}
\textcolor{teal}{\mytextsc{noun}} \hspace{4pt} Tone: MH\#.
\textcolor{Sepia}{\selectlanguage{english}Forest.} \zh{森林。}  ¶ \textcolor{darkblue}{\textbf{\ipa{[F5] tʰo˧ɕi˧˥}}} \textcolor{Sepia}{\selectlanguage{english}pine forest} \zh{松树森林}  
 \zh{量词}: \textcolor{darkblue}{\textbf{\ipa{pʰæ˧˥}}}  \mytextsc{clf}: \textcolor{darkblue}{\textbf{\ipa{pʰæ˧˥}}} 
\lhead{\firstmark}
\rhead{\botmark}

\subsection{\hspace{-0.5cm} {\Large \textcolor{darkblue}{\textbf{\ipa{si˧dzi˩}}}}\hspace{0.5cm}[\kern2pt{\textcolor{darkblue}{\textbf{\ipa{si˧dzi˧˥}}}}\kern2pt]} \hypertarget{si\string_Mdzi\string_B1}{}
\markboth{\textcolor{darkblue}{\textbf{\ipa{si˧dzi˩}}}}{}
\textcolor{teal}{\mytextsc{noun}} \hspace{4pt} Tone: L\#.
\textcolor{Sepia}{\selectlanguage{english}Tree.} \zh{树。}  \zh{量词}: \textcolor{darkblue}{\textbf{\ipa{dzi˩, ʝi˧}}}  \mytextsc{clf}: \textcolor{darkblue}{\textbf{\ipa{dzi˩, ʝi˧}}} 
\lhead{\firstmark}
\rhead{\botmark}

\subsection{\hspace{-0.5cm} {\Large \textcolor{darkblue}{\textbf{\ipa{si˧dzi˩-mv̩˩tsɯ˩}}}}\hspace{0.5cm}[\kern2pt{\textcolor{darkblue}{\textbf{\ipa{xxxx non-correspondance entre le nombre de morphèmes et le nombre de tons de morphèmes}}}}\kern2pt]} \hypertarget{si\string_Mdzi\string_B-mv\string_=\string_BtsM\string_B1}{}
\markboth{\textcolor{darkblue}{\textbf{\ipa{si˧dzi˩-mv̩˩tsɯ˩}}}}{}
\textcolor{teal}{\mytextsc{noun}} \hspace{4pt} Tone: \#L-L.
\textcolor{Sepia}{\selectlanguage{english}Radicel, rootlet, small root.} \zh{胚根。} 
\lhead{\firstmark}
\rhead{\botmark}

\subsection{\hspace{-0.5cm} {\Large \textcolor{darkblue}{\textbf{\ipa{si˧dʑɯ˥}}}}\hspace{0.5cm}[\kern2pt{\textcolor{darkblue}{\textbf{\ipa{xxxx non-correspondance entre le nombre de morphèmes et le nombre de tons de morphèmes}}}}\kern2pt]} \hypertarget{si\string_Mdz£M\string_T1}{}
\markboth{\textcolor{darkblue}{\textbf{\ipa{si˧dʑɯ˥}}}}{}
\textcolor{teal}{\mytextsc{noun}} \hspace{4pt} Tone: H\#.
\textcolor{Sepia}{\selectlanguage{english}Kindling.} \zh{火煤、火捻、火种、劈柴、引柴。}  \zh{量词}: \textcolor{darkblue}{\textbf{\ipa{kʰwɤ˥}}}  \mytextsc{clf}: \textcolor{darkblue}{\textbf{\ipa{kʰwɤ˥}}} 
\lhead{\firstmark}
\rhead{\botmark}

\subsection{\hspace{-0.5cm} {\Large \textcolor{darkblue}{\textbf{\ipa{si˧gɯ˧}}}}\hspace{0.5cm}[\kern2pt{\textcolor{darkblue}{\textbf{\ipa{si˧gɯ˥}}}}\kern2pt]} \hypertarget{si\string_MgM\string_M1}{}
\markboth{\textcolor{darkblue}{\textbf{\ipa{si˧gɯ˧}}}}{}
\textcolor{teal}{\mytextsc{noun}} \hspace{4pt} Tone: M.
\textcolor{Sepia}{\selectlanguage{english}Lion.} \zh{狮子。}  Borrowing: Tibetan  seng ge སེང་གེ
 \zh{量词}: \textcolor{darkblue}{\textbf{\ipa{mi˩}}}  \mytextsc{clf}: \textcolor{darkblue}{\textbf{\ipa{mi˩}}} 
\lhead{\firstmark}
\rhead{\botmark}

\subsection{\hspace{-0.5cm} {\Large \textcolor{darkblue}{\textbf{\ipa{si˧gɯ˧-mi˩}}}}\hspace{0.5cm}[\kern2pt{\textcolor{darkblue}{\textbf{\ipa{xxxx non-correspondance entre le nombre de morphèmes et le nombre de tons de morphèmes}}}}\kern2pt]} \hypertarget{si\string_MgM\string_M-mi\string_B1}{}
\markboth{\textcolor{darkblue}{\textbf{\ipa{si˧gɯ˧-mi˩}}}}{}
\textcolor{teal}{\mytextsc{noun}} \hspace{4pt} Tone: \mytextsc{L}.
\textcolor{Sepia}{\selectlanguage{english}Lioness.} \zh{母狮。}  \zh{量词}: \textcolor{darkblue}{\textbf{\ipa{mi˩}}}  \mytextsc{clf}: \textcolor{darkblue}{\textbf{\ipa{mi˩}}} 
\lhead{\firstmark}
\rhead{\botmark}

\subsection{\hspace{-0.5cm} {\Large \textcolor{darkblue}{\textbf{\ipa{si˧gɯ˧-pʰv̩\#˥}}}}\hspace{0.5cm}[\kern2pt{\textcolor{darkblue}{\textbf{\ipa{xxxx non-correspondance entre le nombre de morphèmes et le nombre de tons de morphèmes}}}}\kern2pt]} \hypertarget{si\string_MgM\string_M-p\string_hv\string_=\#\string_T1}{}
\markboth{\textcolor{darkblue}{\textbf{\ipa{si˧gɯ˧-pʰv̩\#˥}}}}{}
\textcolor{teal}{\mytextsc{noun}} \hspace{4pt} Tone: \#H.
\textcolor{Sepia}{\selectlanguage{english}Male lion.} \zh{公狮子。}  \zh{量词}: \textcolor{darkblue}{\textbf{\ipa{mi˩}}}  \mytextsc{clf}: \textcolor{darkblue}{\textbf{\ipa{mi˩}}} 
\lhead{\firstmark}
\rhead{\botmark}

\subsection{\hspace{-0.5cm} {\Large \textcolor{darkblue}{\textbf{\ipa{si˧gɯ˧-tsʰo\#˥}}}}\hspace{0.5cm}[\kern2pt{\textcolor{darkblue}{\textbf{\ipa{xxxx non-correspondance entre le nombre de morphèmes et le nombre de tons de morphèmes}}}}\kern2pt]} \hypertarget{si\string_MgM\string_M-ts\string_ho\#\string_T1}{}
\markboth{\textcolor{darkblue}{\textbf{\ipa{si˧gɯ˧-tsʰo\#˥}}}}{}
\textcolor{teal}{\mytextsc{noun}} \hspace{4pt} Tone: \#H.
\textcolor{Sepia}{\selectlanguage{english}Lion Dance: a show organized for the feudal lord.} \zh{狮子舞:土司准备的礼仪性表演。土司也亲自参与舞蹈。} 
\lhead{\firstmark}
\rhead{\botmark}

\subsection{\hspace{-0.5cm} {\Large \textcolor{darkblue}{\textbf{\ipa{si˧gɯ˧-zo\#˥}}}}\hspace{0.5cm}[\kern2pt{\textcolor{darkblue}{\textbf{\ipa{xxxx non-correspondance entre le nombre de morphèmes et le nombre de tons de morphèmes}}}}\kern2pt]} \hypertarget{si\string_MgM\string_M-zo\#\string_T1}{}
\markboth{\textcolor{darkblue}{\textbf{\ipa{si˧gɯ˧-zo\#˥}}}}{}
\textcolor{teal}{\mytextsc{noun}} \hspace{4pt} Tone: \#H.
\textcolor{Sepia}{\selectlanguage{english}Lion cub.} \zh{小狮子。}  \zh{量词}: \textcolor{darkblue}{\textbf{\ipa{ɭɯ˧}}} \textcolor{darkblue}{\textbf{\ipa{mi˩}}}  \mytextsc{clf}: \textcolor{darkblue}{\textbf{\ipa{ɭɯ˧}}} \textcolor{darkblue}{\textbf{\ipa{mi˩}}} 
\lhead{\firstmark}
\rhead{\botmark}

\subsection{\hspace{-0.5cm} {\Large \textcolor{darkblue}{\textbf{\ipa{si˧kɤ˧˥}}}}\hspace{0.5cm}[\kern2pt{\textcolor{darkblue}{\textbf{\ipa{si˧kɤ˧˥}}}}\kern2pt]} \hypertarget{si\string_Mk7\string_M\string_T1}{}
\markboth{\textcolor{darkblue}{\textbf{\ipa{si˧kɤ˧˥}}}}{}
\textcolor{teal}{\mytextsc{noun}} \hspace{4pt} Tone: MH\#.
\textcolor{Sepia}{\selectlanguage{english}Branch; rod, stick.} \zh{树枝、小树枝,棍子。}  \zh{量词}: \textcolor{darkblue}{\textbf{\ipa{kɤ˧˥}}}  \mytextsc{clf}: \textcolor{darkblue}{\textbf{\ipa{kɤ˧˥}}} 
\lhead{\firstmark}
\rhead{\botmark}

\subsection{\hspace{-0.5cm} {\Large \textcolor{darkblue}{\textbf{\ipa{si˧kwɤ˩}}}}\hspace{0.5cm}[\kern2pt{\textcolor{darkblue}{\textbf{\ipa{xxxx ton non trouvé, à faire manuellement...}}}}\kern2pt]} \hypertarget{si\string_Mkw7\string_B1}{}
\markboth{\textcolor{darkblue}{\textbf{\ipa{si˧kwɤ˩}}}}{}
\textcolor{teal}{\mytextsc{noun}} \hspace{4pt} Tone: \#L.
\textcolor{Sepia}{\selectlanguage{english}Wooden structure (of a house), carpentry.} \zh{木头框架,如:房子的木头框架。}  ¶ \textcolor{darkblue}{\textbf{\ipa{ʑi˧mi˧-si˧kwɤ˩}}} \textcolor{Sepia}{\selectlanguage{english}a house's carpentry, a house's wooden structure} \zh{房子的木头框架}  
 \zh{量词}: \textcolor{darkblue}{\textbf{\ipa{kwɤ˩}}}  \mytextsc{clf}: \textcolor{darkblue}{\textbf{\ipa{kwɤ˩}}} 
\lhead{\firstmark}
\rhead{\botmark}

\subsection{\hspace{-0.5cm} {\Large \textcolor{darkblue}{\textbf{\ipa{si˧kʰɯ\#˥}}}}\hspace{0.5cm}[\kern2pt{\textcolor{darkblue}{\textbf{\ipa{si˧kʰɯ˥}}}}\kern2pt]} \hypertarget{si\string_Mk\string_hM\#\string_T1}{}
\markboth{\textcolor{darkblue}{\textbf{\ipa{si˧kʰɯ\#˥}}}}{}
\textcolor{teal}{\mytextsc{noun}} \hspace{4pt} Tone: H\#.
\zh{色疙瘩。} Local Chinese dialect:\zh{根三香。} ¶ \textcolor{darkblue}{\textbf{\ipa{si˧kʰɯ˧-bæ˥bæ˩}}} \textcolor{Sepia}{\selectlanguage{english}flower of...} \zh{色疙瘩花}  
\textit{See:} \textcolor{darkblue}{\textbf{\ipa{si˧kʰɯ˧ɭɯ˧bv̩˥}}} 
\lhead{\firstmark}
\rhead{\botmark}

\subsection{\hspace{-0.5cm} {\Large \textcolor{darkblue}{\textbf{\ipa{si˧kʰɯ˧-ɭɯ˧bv̩˥}}}}\hspace{0.5cm}[\kern2pt{\textcolor{darkblue}{\textbf{\ipa{xxxx non-correspondance entre le nombre de morphèmes et le nombre de tons de morphèmes}}}}\kern2pt]} \hypertarget{si\string_Mk\string_hM\string_M-l\string_RM\string_Mbv\string_=\string_T1}{}
\markboth{\textcolor{darkblue}{\textbf{\ipa{si˧kʰɯ˧-ɭɯ˧bv̩˥}}}}{}
\textcolor{teal}{\mytextsc{noun}} \hspace{4pt} Tone: H\#.
\textcolor{Sepia}{\selectlanguage{english}White Chinese herbaceous peony, \textit{Paeonia lactiflora}.} \zh{白芍药。}  \zh{量词}: \textcolor{darkblue}{\textbf{\ipa{kɤ˧˥}}}  \mytextsc{clf}: \textcolor{darkblue}{\textbf{\ipa{kɤ˧˥}}} \textit{See:} \hyperlink{}{\textcolor{darkblue}{\textbf{\ipa{si˧kʰɯ\#˥}}}} 
\lhead{\firstmark}
\rhead{\botmark}

\subsection{\hspace{-0.5cm} {\Large \textcolor{darkblue}{\textbf{\ipa{si˧nɑ˥}}}}\hspace{0.5cm}[\kern2pt{\textcolor{darkblue}{\textbf{\ipa{si˧nɑ˥}}}}\kern2pt]} \hypertarget{si\string_MnA\string_T1}{}
\markboth{\textcolor{darkblue}{\textbf{\ipa{si˧nɑ˥}}}}{}
\textcolor{teal}{\mytextsc{noun}} \hspace{4pt} Tone: H\#.
\textcolor{Sepia}{\selectlanguage{english}Deep forest.} \zh{森林深处(难走路)。}  \zh{量词}: \textcolor{darkblue}{\textbf{\ipa{pʰæ˧˥}}}  \mytextsc{clf}: \textcolor{darkblue}{\textbf{\ipa{pʰæ˧˥}}} 
\lhead{\firstmark}
\rhead{\botmark}

\subsection{\hspace{-0.5cm} {\Large \textcolor{darkblue}{\textbf{\ipa{si˧-ʁæ˧bæ˥}}}}\hspace{0.5cm}[\kern2pt{\textcolor{darkblue}{\textbf{\ipa{xxxx non-correspondance entre le nombre de morphèmes et le nombre de tons de morphèmes}}}}\kern2pt]} \hypertarget{si\string_M-R\{\string_Mb\{\string_T1}{}
\markboth{\textcolor{darkblue}{\textbf{\ipa{si˧-ʁæ˧bæ˥}}}}{}
\textcolor{teal}{\mytextsc{noun}} \hspace{4pt} Tone: H\#.
\textcolor{Sepia}{\selectlanguage{english}Wooden plate.} \zh{木盘子。}  \zh{量词}: \textcolor{darkblue}{\textbf{\ipa{ɭɯ˧}}}  \mytextsc{clf}: \textcolor{darkblue}{\textbf{\ipa{ɭɯ˧}}} 
\lhead{\firstmark}
\rhead{\botmark}

\subsection{\hspace{-0.5cm} {\Large \textcolor{darkblue}{\textbf{\ipa{si˧ʁo\#˥}}}}\hspace{0.5cm}[\kern2pt{\textcolor{darkblue}{\textbf{\ipa{si˧ʁo˧}}}}\kern2pt]} \hypertarget{si\string_MRo\#\string_T1}{}
\markboth{\textcolor{darkblue}{\textbf{\ipa{si˧ʁo\#˥}}}}{}
\textcolor{teal}{\mytextsc{noun}} \hspace{4pt} Tone: \#H.
\textcolor{Sepia}{\selectlanguage{english}Fruit tree.} \zh{果树。}  \zh{量词}: \textcolor{darkblue}{\textbf{\ipa{dzi˩}}}  \mytextsc{clf}: \textcolor{darkblue}{\textbf{\ipa{dzi˩}}} 
\lhead{\firstmark}
\rhead{\botmark}

\subsection{\hspace{-0.5cm} {\Large \textcolor{darkblue}{\textbf{\ipa{si˧ʁo˧si˧ɭɯ\#˥}}}}\hspace{0.5cm}[\kern2pt{\textcolor{darkblue}{\textbf{\ipa{si˧ʁo˧si˧ɭɯ˧}}}}\kern2pt]} \hypertarget{si\string_MRo\string_Msi\string_Ml\string_RM\#\string_T1}{}
\markboth{\textcolor{darkblue}{\textbf{\ipa{si˧ʁo˧si˧ɭɯ\#˥}}}}{}
\textcolor{teal}{\mytextsc{noun}} \hspace{4pt} Tone: \#H.
\textcolor{Sepia}{\selectlanguage{english}Fruit.} \zh{水果。}  ¶ \textcolor{darkblue}{\textbf{\ipa{si˧ʁo˧si˧ɭɯ˧ ɲi˥}}} \textcolor{Sepia}{\selectlanguage{english}\mytextsc{cop}} \zh{是水果。}  
 \zh{量词}: \textcolor{darkblue}{\textbf{\ipa{ɭɯ˧}}}  \mytextsc{clf}: \textcolor{darkblue}{\textbf{\ipa{ɭɯ˧}}} 
\lhead{\firstmark}
\rhead{\botmark}

\subsection{\hspace{-0.5cm} {\Large \textcolor{darkblue}{\textbf{\ipa{si˧-sæ˥qʰv̩˩}}}}\hspace{0.5cm}[\kern2pt{\textcolor{darkblue}{\textbf{\ipa{xxxx non-correspondance entre le nombre de morphèmes et le nombre de tons de morphèmes}}}}\kern2pt]} \hypertarget{si\string_M-s\{\string_Tq\string_hv\string_=\string_B1}{}
\markboth{\textcolor{darkblue}{\textbf{\ipa{si˧-sæ˥qʰv̩˩}}}}{}
\textcolor{teal}{\mytextsc{noun}} \hspace{4pt} Tone: \#H.
\textcolor{Sepia}{\selectlanguage{english}Birch, \textit{Betula szechuanica (Betula Pendula var. szechuanica)}.} \zh{四川桦树,白桦树。} 
\lhead{\firstmark}
\rhead{\botmark}

\subsection{\hspace{-0.5cm} {\Large \textcolor{darkblue}{\textbf{\ipa{si˧tʰv̩\#˥}}}}\hspace{0.5cm}[\kern2pt{\textcolor{darkblue}{\textbf{\ipa{si˧tʰv̩˧}}}}\kern2pt]} \hypertarget{si\string_Mt\string_hv\string_=\#\string_T1}{}
\markboth{\textcolor{darkblue}{\textbf{\ipa{si˧tʰv̩\#˥}}}}{}
\textcolor{teal}{\mytextsc{noun}} \hspace{4pt} Tone: \#H.
\textcolor{Sepia}{\selectlanguage{english}A piece of furniture of the main room, which constitutes the symbolic dwelling of ancestors, and serves as an altar; on the New Year, some candles are lighted on it.} \zh{供桌:主屋里面的一个家具,是祖先的象征性住所。}  ¶ \textcolor{darkblue}{\textbf{\ipa{ʑi˧dv̩˧-nv̩˩mi˩, | si˧tʰv̩˧!}}} \textcolor{Sepia}{\selectlanguage{english}The heart of the house is the altar to the ancestors!} \zh{屋子的中心,就是祖先的供桌!}  

\lhead{\firstmark}
\rhead{\botmark}

\subsection{\hspace{-0.5cm} {\Large \textcolor{darkblue}{\textbf{\ipa{si˩qʰɑ˩}}}}\hspace{0.5cm}[\kern2pt{\textcolor{darkblue}{\textbf{\ipa{si˩qʰɑ˩˥}}}}\kern2pt]} \hypertarget{si\string_Bq\string_hA\string_B1}{}
\markboth{\textcolor{darkblue}{\textbf{\ipa{si˩qʰɑ˩}}}}{}
\textcolor{teal}{\mytextsc{noun}} \hspace{4pt} Tone: L.
\textcolor{Sepia}{\selectlanguage{english}Plum tree, prune tree.} \zh{梅子。}  ¶ \textcolor{darkblue}{\textbf{\ipa{si˩qʰɑ˩-dʑɯ˩}}} \textcolor{Sepia}{\selectlanguage{english}a liquid prepared from plums, which served as an equivalent of vinegar (vinegar was introduced late: it was bought in Chinese areas)} \zh{用梅子做的一种汁,用法类似于醋。过去,永宁没有醋,醋是从内地买来的。}  

\lhead{\firstmark}
\rhead{\botmark}

\subsection{\hspace{-0.5cm} {\Large \textcolor{darkblue}{\textbf{\ipa{si˩tsʰɤ˩}}}}\hspace{0.5cm}[\kern2pt{\textcolor{darkblue}{\textbf{\ipa{si˩tsʰɤ˩˥}}}}\kern2pt]} \hypertarget{si\string_Bts\string_h7\string_B1}{}
\markboth{\textcolor{darkblue}{\textbf{\ipa{si˩tsʰɤ˩}}}}{}
\textcolor{teal}{\mytextsc{noun}} \hspace{4pt} Tone: L.
\ding{202} \textcolor{Sepia}{\selectlanguage{english}Leaf.} \zh{叶子。}  ¶ \textcolor{darkblue}{\textbf{\ipa{si˧dzi˩-si˩tsʰɤ˩}}} \textcolor{Sepia}{\selectlanguage{english}tree leaf} \zh{树叶}  
 \zh{量词}: \textcolor{darkblue}{\textbf{\ipa{tsʰɤ˧˥}}} \ding{203} \textcolor{Sepia}{\selectlanguage{english}Cock's comb.} \zh{鸡冠。}  ¶ \textcolor{darkblue}{\textbf{\ipa{æ̃˧ʂwæ˥-si˩tsʰɤ˩}}} \textcolor{Sepia}{\selectlanguage{english}comb of (a) cock} \zh{公鸡冠}  
 \mytextsc{clf}: \textcolor{darkblue}{\textbf{\ipa{tsʰɤ˧˥}}} 
\lhead{\firstmark}
\rhead{\botmark}

\subsection{\hspace{-0.5cm} {\Large \textcolor{darkblue}{\textbf{\ipa{si˧˥}}} \textsubscript{1}}\hspace{0.5cm}[\kern2pt{\textcolor{darkblue}{\textbf{\ipa{si˩˥}}}}\kern2pt]} \hypertarget{si\string_M\string_T1}{}
\markboth{\textcolor{darkblue}{\textbf{\ipa{si˧˥}}} \textsubscript{1}}{}
\textcolor{teal}{\mytextsc{verb}} \hspace{4pt} Tone: MH.
\textcolor{Sepia}{\selectlanguage{english}To shave (the beard or the head); to scrub (e.g. to scrub earth off vegetables).} \zh{剔,刮。}  ¶ \textcolor{darkblue}{\textbf{\ipa{mo˧ si˥}}} \textcolor{Sepia}{\selectlanguage{english}to scrub mushrooms (to take off the earth, moss...)} \zh{刮菌子(刮掉污垢)}  
 ¶ \textcolor{darkblue}{\textbf{\ipa{mv̩˧tsɯ˧ si˥}}} \textcolor{Sepia}{\selectlanguage{english}to shave (one's) beard} \zh{刮胡子}  
 ¶ \textcolor{darkblue}{\textbf{\ipa{ʁo˧qʰwɤ˩ si˩}}} \textcolor{Sepia}{\selectlanguage{english}to shave one's head} \zh{剃头}  
 ¶ \textcolor{darkblue}{\textbf{\ipa{ʁo˧qʰwɤ˩ si˩-di˩}}} \textcolor{Sepia}{\selectlanguage{english}Razor: object used to shave the head or the beard. (In the main consultant's youth, not every family had a razor. One would call someone to the house to shave the head or the beard. It was mostly monks and elderly people who had their heads and beards shaved.)} \zh{理发刮刀}  

\lhead{\firstmark}
\rhead{\botmark}

\subsection{\hspace{-0.5cm} {\Large \textcolor{darkblue}{\textbf{\ipa{si˧˥}}} \textsubscript{2}}\hspace{0.5cm}[\kern2pt{\textcolor{darkblue}{\textbf{\ipa{si˧˥}}}}\kern2pt]} \hypertarget{si\string_M\string_T2}{}
\markboth{\textcolor{darkblue}{\textbf{\ipa{si˧˥}}} \textsubscript{2}}{}
\textcolor{teal}{\mytextsc{verb}} \hspace{4pt} Tone: MH.
\textcolor{Sepia}{\selectlanguage{english}To murder, to kill (a human being).} \zh{杀(人)。}  ¶ \textcolor{darkblue}{\textbf{\ipa{hĩ˧ si˩}}} \textcolor{Sepia}{\selectlanguage{english}to kill someone} \zh{杀人}  

\lhead{\firstmark}
\rhead{\botmark}

\subsection{\hspace{-0.5cm} {\Large \textcolor{darkblue}{\textbf{\ipa{si˩˥}}}}\hspace{0.5cm}[\kern2pt{\textcolor{darkblue}{\textbf{\ipa{si˩˥}}}}\kern2pt]} \hypertarget{si\string_B\string_T1}{}
\markboth{\textcolor{darkblue}{\textbf{\ipa{si˩˥}}}}{}
\textcolor{teal}{\mytextsc{noun}} \hspace{4pt} Tone: LH.
\textcolor{Sepia}{\selectlanguage{english}Liver.} \zh{肝。}  \zh{量词}: \textcolor{darkblue}{\textbf{\ipa{ɭɯ˧}}}  \mytextsc{clf}: \textcolor{darkblue}{\textbf{\ipa{ɭɯ˧}}} 
\lhead{\firstmark}
\rhead{\botmark}

\subsection{\hspace{-0.5cm} {\Large \textcolor{darkblue}{\textbf{\ipa{so˥}}} \textsubscript{1}}\hspace{0.5cm}[\kern2pt{\textcolor{darkblue}{\textbf{\ipa{so˥}}}}\kern2pt]} \hypertarget{so\string_T1}{}
\markboth{\textcolor{darkblue}{\textbf{\ipa{so˥}}} \textsubscript{1}}{}
\textcolor{teal}{\mytextsc{noun}} \hspace{4pt} Tone: \#H.
\textcolor{Sepia}{\selectlanguage{english}Offering to the gods, given to them in the morning; it comprises tea, butter, flour, and honey; it is burnt over a fire of pine needles.} \zh{早上献给神的食物(含茶、酥油、面、蜂蜜),扔进松针火里烧。}  ¶ \textcolor{darkblue}{\textbf{\ipa{so˧ dze˧ tʰi˧-qæ˩}}} \textcolor{Sepia}{\selectlanguage{english}to burn honey as an offering} \zh{烧蜂蜜献给神}  
 ¶ \textcolor{darkblue}{\textbf{\ipa{[M23] so˧ qæ˩}}} \textcolor{Sepia}{\selectlanguage{english}to burn an offering} \zh{烧献给神(食物,……)}  

\lhead{\firstmark}
\rhead{\botmark}

\subsection{\hspace{-0.5cm} {\Large \textcolor{darkblue}{\textbf{\ipa{so˥}}} \textsubscript{2}}\hspace{0.5cm}[\kern2pt{\textcolor{darkblue}{\textbf{\ipa{so˥}}}}\kern2pt]} \hypertarget{so\string_T2}{}
\markboth{\textcolor{darkblue}{\textbf{\ipa{so˥}}} \textsubscript{2}}{}
\textcolor{teal}{\mytextsc{classifier}} \hspace{4pt} Tone: H*.
\textcolor{Sepia}{\selectlanguage{english}A thing (no plural; only used in the negative construction “there is not a thing”).} \zh{量词:样东西,如:‘一样东西都没有’。}  ¶ \textcolor{darkblue}{\textbf{\ipa{ɖɯ˧-so˥ | mɤ˧-dʑo˧!}}} \textcolor{Sepia}{\selectlanguage{english}There is simply nothing at all! (A polite statement made by the host when welcoming a guest for a meal, apologizing, in self-deprecation, for not offering a meal commensurate to one's wishes.)} \zh{一样也没有! / 没什么东西!(请客时的礼貌、自我贬低说法:请客人原谅菜不够丰盛)}  
\textit{See:} \textcolor{darkblue}{\textbf{\ipa{sɑ˥}}} 
\lhead{\firstmark}
\rhead{\botmark}

\subsection{\hspace{-0.5cm} {\Large \textcolor{darkblue}{\textbf{\ipa{so˧\textsubscript{a}}}}}\hspace{0.5cm}[\kern2pt{\textcolor{darkblue}{\textbf{\ipa{so˩˥}}}}\kern2pt]} \hypertarget{so\string_Ma1}{}
\markboth{\textcolor{darkblue}{\textbf{\ipa{so˧\textsubscript{a}}}}}{}
\textcolor{teal}{\mytextsc{classifier}} \hspace{4pt} Tone: M\textsubscript{a}.
\textcolor{Sepia}{\selectlanguage{english}Classifier for mornings.} \zh{量词:早晨(一个)。}  ¶ \textcolor{darkblue}{\textbf{\ipa{mv̩˩si˧-njɤ˧˥ | ɖɯ˧-so˧, | njɤ˧le˧gv̩˧ | ɖɯ˧-ɲi˥, | mv̩˧kʰv̩˥ | ɖɯ˧-hɑ̃˧˥!}}} \textcolor{Sepia}{\selectlanguage{english}One morning; one day; [or] one night! (A sentence that summarizes the three ways to count days: a day can be referred to as “one morning”, “one day”, or “one night”.)} \zh{一个早晨,一个白天,(或者说)一个晚上!(这句话,总结数日子的三个方式:‘一天’,可以说成‘一个早晨’、‘一个白天’、或‘一个晚上’。)}  
 ¶ \textcolor{darkblue}{\textbf{\ipa{tʰv̩˧-so˩}}} \textcolor{Sepia}{\selectlanguage{english}that morning} \zh{那天早上}  

\lhead{\firstmark}
\rhead{\botmark}

\subsection{\hspace{-0.5cm} {\Large \textcolor{darkblue}{\textbf{\ipa{so˧dʑɯ\#˥}}}}\hspace{0.5cm}[\kern2pt{\textcolor{darkblue}{\textbf{\ipa{so˧dʑɯ˧˥}}}}\kern2pt]} \hypertarget{so\string_Mdz£M\#\string_T1}{}
\markboth{\textcolor{darkblue}{\textbf{\ipa{so˧dʑɯ\#˥}}}}{}
\textcolor{teal}{\mytextsc{noun}} \hspace{4pt} Tone: \#H.
\textcolor{Sepia}{\selectlanguage{english}Pitfall, pit, trap.} \zh{陷阱。}  ¶ \textcolor{darkblue}{\textbf{\ipa{so˧dʑɯ˧ | ɖɯ˧-ɭɯ˧ | qwæ˧˥}}} \textcolor{Sepia}{\selectlanguage{english}to dig a pitfall, to dig a trap} \zh{挖一个陷阱}  

\lhead{\firstmark}
\rhead{\botmark}

\subsection{\hspace{-0.5cm} {\Large \textcolor{darkblue}{\textbf{\ipa{so˧hɑ̃˩}}}}\hspace{0.5cm}[\kern2pt{\textcolor{darkblue}{\textbf{\ipa{so˧hɑ̃˧}}}}\kern2pt]} \hypertarget{so\string_MhA\string_~\string_B1}{}
\markboth{\textcolor{darkblue}{\textbf{\ipa{so˧hɑ̃˩}}}}{}
\textcolor{teal}{\mytextsc{adverb(ial)}} \hspace{4pt} Tone: L\#.
\textcolor{Sepia}{\selectlanguage{english}Tomorrow evening.} \zh{明晚。}  ¶ \textcolor{darkblue}{\textbf{\ipa{so˧hɑ̃˩ | -ɖɯ˩hɑ̃˩˥}}} \textcolor{Sepia}{\selectlanguage{english}tomorrow evening} \zh{明天晚上}  

\lhead{\firstmark}
\rhead{\botmark}

\subsection{\hspace{-0.5cm} {\Large \textcolor{darkblue}{\textbf{\ipa{so˧hwɤ˩}}}}\hspace{0.5cm}[\kern2pt{\textcolor{darkblue}{\textbf{\ipa{so˧hwɤ˩}}}}\kern2pt]} \hypertarget{so\string_Mhw7\string_B1}{}
\markboth{\textcolor{darkblue}{\textbf{\ipa{so˧hwɤ˩}}}}{}
\textcolor{teal}{\mytextsc{adverb(ial)}} \hspace{4pt} Tone: L\#.
\textcolor{Sepia}{\selectlanguage{english}Afterwards; later; from now on.} \zh{后来、以后,从此以后。} 
\lhead{\firstmark}
\rhead{\botmark}

\subsection{\hspace{-0.5cm} {\Large \textcolor{darkblue}{\textbf{\ipa{so˧ʝi˥\$}}}}\hspace{0.5cm}[\kern2pt{\textcolor{darkblue}{\textbf{\ipa{so˧ʝi˩}}}}\kern2pt]} \hypertarget{so\string_Mj££i\string_T\$1}{}
\markboth{\textcolor{darkblue}{\textbf{\ipa{so˧ʝi˥\$}}}}{}
\textcolor{teal}{\mytextsc{adverb(ial)}} \hspace{4pt} Tone: H\$.
\textcolor{Sepia}{\selectlanguage{english}Next year.} \zh{明年。} 
\lhead{\firstmark}
\rhead{\botmark}

\subsection{\hspace{-0.5cm} {\Large \textcolor{darkblue}{\textbf{\ipa{so˧lo˧}}}}\hspace{0.5cm}[\kern2pt{\textcolor{darkblue}{\textbf{\ipa{so˧lo˧}}}}\kern2pt]} \hypertarget{so\string_Mlo\string_M1}{}
\markboth{\textcolor{darkblue}{\textbf{\ipa{so˧lo˧}}}}{}
\textcolor{teal}{\mytextsc{noun}} \hspace{4pt} Tone: M.
\textcolor{Sepia}{\selectlanguage{english}Influence, example (in education).} \zh{影响,榜样。}  ¶ \textcolor{darkblue}{\textbf{\ipa{so˧lo˧ dzɑ˧! | mɤ˧-dʑɤ˩-hĩ˩ | ɖɯ˧-ʑi˩ ɲi˩!}}} \textcolor{Sepia}{\selectlanguage{english}He/she has a bad influence / he/she gives a bad example! (His/her family) is a bad family!} \zh{他(对周围的人)有一个不好的影响!(他的家庭)是个不好的家庭!}  
 ¶ \textcolor{darkblue}{\textbf{\ipa{so˧lo˧ mɤ˧-dʑɤ˩!}}} \textcolor{Sepia}{\selectlanguage{english}Same meaning as previous example: His/her example/influence is not good.} \zh{同上:(他对别人的)影响不好。}  
 ¶ \textcolor{darkblue}{\textbf{\ipa{so˧lo˧ dʑɤ˩}}} \textcolor{Sepia}{\selectlanguage{english}good influence; good example; good education} \zh{好榜样、好例子、好教育}  
 \zh{量词}: \textcolor{darkblue}{\textbf{\ipa{kʰwɤ˥}}}  \mytextsc{clf}: \textcolor{darkblue}{\textbf{\ipa{kʰwɤ˥}}} 
\lhead{\firstmark}
\rhead{\botmark}

\subsection{\hspace{-0.5cm} {\Large \textcolor{darkblue}{\textbf{\ipa{so˧ɬi˧mi˧}}}}\hspace{0.5cm}[\kern2pt{\textcolor{darkblue}{\textbf{\ipa{so˧ɬi˧mi˥}}}}\kern2pt]} \hypertarget{so\string_MKi\string_Mmi\string_M1}{}
\markboth{\textcolor{darkblue}{\textbf{\ipa{so˧ɬi˧mi˧}}}}{}
\textcolor{teal}{\mytextsc{noun}} \hspace{4pt} Tone: M.
\textcolor{Sepia}{\selectlanguage{english}Third month.} \zh{三月。} 
\lhead{\firstmark}
\rhead{\botmark}

\subsection{\hspace{-0.5cm} {\Large \textcolor{darkblue}{\textbf{\ipa{so˧ɲi˥}}}}\hspace{0.5cm}[\kern2pt{\textcolor{darkblue}{\textbf{\ipa{so˧ɲi˩}}}}\kern2pt]} \hypertarget{so\string_MJi\string_T1}{}
\markboth{\textcolor{darkblue}{\textbf{\ipa{so˧ɲi˥}}}}{}
\textcolor{teal}{\mytextsc{adverb(ial)}} \hspace{4pt} Tone: .
\textcolor{Sepia}{\selectlanguage{english}Tomorrow.} \zh{明天、第二天。} 
\lhead{\firstmark}
\rhead{\botmark}

\subsection{\hspace{-0.5cm} {\Large \textcolor{darkblue}{\textbf{\ipa{so˩}}}}\hspace{0.5cm}[\kern2pt{\textcolor{darkblue}{\textbf{\ipa{so˩˥}}}}\kern2pt]} \hypertarget{so\string_B1}{}
\markboth{\textcolor{darkblue}{\textbf{\ipa{so˩}}}}{}
\textcolor{teal}{\mytextsc{number}} \hspace{4pt} Tone: L.
\textcolor{Sepia}{\selectlanguage{english}3.} \zh{3。} 
\lhead{\firstmark}
\rhead{\botmark}

\subsection{\hspace{-0.5cm} {\Large \textcolor{darkblue}{\textbf{\ipa{so˩\textsubscript{a}}}} \textsubscript{1}}\hspace{0.5cm}[\kern2pt{\textcolor{darkblue}{\textbf{\ipa{so˩˥}}}}\kern2pt]} \hypertarget{so\string_Ba1}{}
\markboth{\textcolor{darkblue}{\textbf{\ipa{so˩\textsubscript{a}}}} \textsubscript{1}}{}
\textcolor{teal}{\mytextsc{adjective}} \hspace{4pt} Tone: L\textsubscript{a}.
\textcolor{Sepia}{\selectlanguage{english}Good, pleasant to the taste or smell.} \zh{香(吃得香,气味香)。} 
\lhead{\firstmark}
\rhead{\botmark}

\subsection{\hspace{-0.5cm} {\Large \textcolor{darkblue}{\textbf{\ipa{so˩\textsubscript{a}}}} \textsubscript{2}}\hspace{0.5cm}[\kern2pt{\textcolor{darkblue}{\textbf{\ipa{so˩˥}}}}\kern2pt]} \hypertarget{so\string_Ba2}{}
\markboth{\textcolor{darkblue}{\textbf{\ipa{so˩\textsubscript{a}}}} \textsubscript{2}}{}
\textcolor{teal}{\mytextsc{verb}} \hspace{4pt} Tone: L\textsubscript{a}.
\ding{202} \textcolor{Sepia}{\selectlanguage{english}To study.} \zh{学习。}  ¶ \textcolor{darkblue}{\textbf{\ipa{tʰæ˧ɻæ˩ so˩}}} \textcolor{Sepia}{\selectlanguage{english}to study (books)} \zh{读书、学习}  
 ¶ \textcolor{darkblue}{\textbf{\ipa{so˩ mɤ˩-se˥!}}} \textcolor{Sepia}{\selectlanguage{english}There's no end of it! / One is never done with studying! (A comment about the linguist's endeavour to study a language: unlike manual work, it is never really finished.)} \zh{学不完!(关于语言学家的工作:做不完,不像做手工可以有一个明确的终点。)}  
 ¶ \textcolor{darkblue}{\textbf{\ipa{ɖɯ˧-so˧\textasciitilde{}so˥-ɻ̍˩}}} \textcolor{Sepia}{\selectlanguage{english}to study a little} \zh{学一学}  
\ding{203} \textcolor{Sepia}{\selectlanguage{english}To follow the example of someone, to imitate someone.} \zh{学一个人、模仿一个人。}  ¶ \textcolor{darkblue}{\textbf{\ipa{tʰv̩˧ tʰɑ˧-so˧˥!}}} \textcolor{Sepia}{\selectlanguage{english}Don't follow his example! / Don't do like him!} \zh{别学他! / 别做得像他一样!}  
\ding{204} \textcolor{Sepia}{\selectlanguage{english}To teach.} \zh{教。}  ¶ \textcolor{darkblue}{\textbf{\ipa{tʰæ˧ɻæ˩ so˩}}} \textcolor{Sepia}{\selectlanguage{english}to teach} \zh{教书}  
 ¶ \textcolor{darkblue}{\textbf{\ipa{njɤ˧-ɳɯ˧ | no˧ so˧-bi˧!}}} \textcolor{Sepia}{\selectlanguage{english}I'm going to teach you! / Let me teach you!}  

\lhead{\firstmark}
\rhead{\botmark}

\subsection{\hspace{-0.5cm} {\Large \textcolor{darkblue}{\textbf{\ipa{so˩\textasciitilde{}so˧˥}}}}\hspace{0.5cm}[\kern2pt{\textcolor{darkblue}{\textbf{\ipa{so˧so˧˥}}}}\kern2pt]} \hypertarget{so\string_B~so\string_M\string_T1}{}
\markboth{\textcolor{darkblue}{\textbf{\ipa{so˩\textasciitilde{}so˧˥}}}}{}
\textcolor{teal}{\mytextsc{verb}} \hspace{4pt} Tone: MH.
\textcolor{Sepia}{\selectlanguage{english}To rub in one's hands.} \zh{揉在手里。}  ¶ \textcolor{darkblue}{\textbf{\ipa{le˧-so˩\textasciitilde{}so˩}}} \textcolor{Sepia}{\selectlanguage{english}\mytextsc{accomp} \string_ \mytextsc{red}} \zh{揉来揉去}  

\lhead{\firstmark}
\rhead{\botmark}

\subsection{\hspace{-0.5cm} {\Large \textcolor{darkblue}{\textbf{\ipa{so˧˥}}}}\hspace{0.5cm}[\kern2pt{\textcolor{darkblue}{\textbf{\ipa{so˧˥}}}}\kern2pt]} \hypertarget{so\string_M\string_T1}{}
\markboth{\textcolor{darkblue}{\textbf{\ipa{so˧˥}}}}{}
\textcolor{teal}{\mytextsc{noun}} \hspace{4pt} Tone: MH.
\ding{202} \textcolor{Sepia}{\selectlanguage{english}Breath.} \zh{(一口)气。}  \zh{量词}: \textcolor{darkblue}{\textbf{\ipa{kʰɯ˩}}} \ding{203} \textcolor{Sepia}{\selectlanguage{english}Vapour.} \zh{蒸汽。}  ¶ \textcolor{darkblue}{\textbf{\ipa{so˧ tʰv̩˥-ze˩}}} \textcolor{Sepia}{\selectlanguage{english}Vapour is coming out.} \zh{热气冒出来了。}  
 \mytextsc{clf}: \textcolor{darkblue}{\textbf{\ipa{kʰɯ˩}}} 
\lhead{\firstmark}
\rhead{\botmark}

\subsection{\hspace{-0.5cm} {\Large \textcolor{darkblue}{\textbf{\ipa{sɯ˥}}} \textsubscript{1}}\hspace{0.5cm}[\kern2pt{\textcolor{darkblue}{\textbf{\ipa{sɯ˥}}}}\kern2pt]} \hypertarget{sM\string_T1}{}
\markboth{\textcolor{darkblue}{\textbf{\ipa{sɯ˥}}} \textsubscript{1}}{}
\textcolor{teal}{\mytextsc{verb}} \hspace{4pt} Tone: H.
\textcolor{Sepia}{\selectlanguage{english}To whet.} \zh{磨(刀)。}  ¶ \textcolor{darkblue}{\textbf{\ipa{ɖɯ˧-sɯ˧\textasciitilde{}sɯ˧-ɻ̍˥}}} \textcolor{Sepia}{\selectlanguage{english}to whet a little} \zh{磨一磨}  
 ¶ \textcolor{darkblue}{\textbf{\ipa{sɯ˩tʰi˩ sɯ˩˥}}} \textcolor{Sepia}{\selectlanguage{english}to whet a knife} \zh{磨刀}  

\lhead{\firstmark}
\rhead{\botmark}

\subsection{\hspace{-0.5cm} {\Large \textcolor{darkblue}{\textbf{\ipa{sɯ˥}}} \textsubscript{2}}\hspace{0.5cm}[\kern2pt{\textcolor{darkblue}{\textbf{\ipa{sɯ˥}}}}\kern2pt]} \hypertarget{sM\string_T2}{}
\markboth{\textcolor{darkblue}{\textbf{\ipa{sɯ˥}}} \textsubscript{2}}{}
\textcolor{teal}{\mytextsc{verb}} \hspace{4pt} Tone: H.
\textcolor{Sepia}{\selectlanguage{english}To know.} \zh{知道。}  ¶ \textcolor{darkblue}{\textbf{\ipa{mɤ˧-sɯ˥}}} \textcolor{Sepia}{\selectlanguage{english}\mytextsc{neg}} \zh{不知道}  

\lhead{\firstmark}
\rhead{\botmark}

\subsection{\hspace{-0.5cm} {\Large \textcolor{darkblue}{\textbf{\ipa{‑sɯ˧}}}}\hspace{0.5cm}[\kern2pt{\textcolor{darkblue}{\textbf{\ipa{sɯ˥}}}}\kern2pt]} \hypertarget{‑sM\string_M1}{}
\markboth{\textcolor{darkblue}{\textbf{\ipa{‑sɯ˧}}}}{}
\textcolor{teal}{\mytextsc{suffix}} \hspace{4pt} Tone: M.
\textcolor{Sepia}{\selectlanguage{english}First, at first, in the first place; anymore (in “not anymore”).} \zh{首先、先。}  ¶ \textcolor{darkblue}{\textbf{\ipa{njɤ˧ ʈʂʰɯ˧-sɯ˩ | dzɯ˧-bi˧!}}} \textcolor{Sepia}{\selectlanguage{english}Let me eat this one first! / I'll eat this one first!} \zh{我要先吃这个!}  
 ¶ \textcolor{darkblue}{\textbf{\ipa{njɤ˧ | ʈʂʰɯ˧-sɯ˩ | li˧-bi˧!}}} \textcolor{Sepia}{\selectlanguage{english}I'll read this one first! (Context: examining two books, and deciding which one to read first)} \zh{我要先读这本!}  
 ¶ \textcolor{darkblue}{\textbf{\ipa{ʈʂʰɯ˧ sɯ˩ | hwæ˧-bi˧!}}} \textcolor{Sepia}{\selectlanguage{english}Let's buy this one first!} \zh{先买这个吧!}  
 ¶ \textcolor{darkblue}{\textbf{\ipa{ʈʂʰɯ˧ sɯ˩ | tɕʰi˧-bi˧!}}} \textcolor{Sepia}{\selectlanguage{english}Let's sell this one first!} \zh{先卖这个吧!}  
 ¶ \textcolor{darkblue}{\textbf{\ipa{ʈʂʰɯ˧ sɯ˩ | dzɯ˧-bi˧!}}} \textcolor{Sepia}{\selectlanguage{english}Let's eat this one first!} \zh{先吃这个吧!}  
 ¶ \textcolor{darkblue}{\textbf{\ipa{ʈʂʰɯ˧ sɯ˩ | ʑi˩-bi˩˥}}} \textcolor{Sepia}{\selectlanguage{english}Let's pick up this one first!} \zh{先拿这个吧!}  
 ¶ \textcolor{darkblue}{\textbf{\ipa{ʈʂʰɯ˧ sɯ˩ | ʈʰɯ˩-bi˩˥}}} \textcolor{Sepia}{\selectlanguage{english}Let's drink this one first!} \zh{先喝这个吧!}  
 ¶ \textcolor{darkblue}{\textbf{\ipa{ʈʂʰɯ˧ sɯ˩ | lɑ˧-bi˥}}} \textcolor{Sepia}{\selectlanguage{english}Let's beat this one first!} \zh{先打这个吧!}  
 ¶ \textcolor{darkblue}{\textbf{\ipa{tv̩˧tv̩˥ sɯ˩ | tʰi˧-tsʰi˥!}}} \textcolor{Sepia}{\selectlanguage{english}Put on your hat first! (Injunction to a little child before an outing)} \zh{你先戴上帽子!(情景:出门前,让孩子戴上帽子)}  
 ¶ \textcolor{darkblue}{\textbf{\ipa{no˧ | le˧-sɯ˧ gv̩˧\textasciitilde{}gv̩˥!}}} \textcolor{Sepia}{\selectlanguage{english}Do your own work first! / Please work on your own for a start! (Context: when I arrive for a morning class, the consultant is busy; she knows that I have various tasks to do, some of which I can do on my own, such as verifying texts that have already been transcribed; she tells me: “Please work on your own for a start!”)} \zh{你先自己工作(一会)吧!(情景:调查者早上到合作人的家,但她忙着,而她知道调查者有不同类型的工作要做,其中有一些可以自己做,比如重新核对记录过的长篇语料。她说:“你先忙自己的一会吧!”)}  

\lhead{\firstmark}
\rhead{\botmark}

\subsection{\hspace{-0.5cm} {\Large \textcolor{darkblue}{\textbf{\ipa{sɯ˧\textsubscript{a}}}}}\hspace{0.5cm}[\kern2pt{\textcolor{darkblue}{\textbf{\ipa{sɯ˥}}}}\kern2pt]} \hypertarget{sM\string_Ma1}{}
\markboth{\textcolor{darkblue}{\textbf{\ipa{sɯ˧\textsubscript{a}}}}}{}
\textcolor{teal}{\mytextsc{verb}} \hspace{4pt} Tone: M\textsubscript{a}.
\ding{202} \textcolor{Sepia}{\selectlanguage{english}To string (beads).} \zh{串(珠)。}  ¶ \textcolor{darkblue}{\textbf{\ipa{sɯ˧ɻ̍˧ sɯ˧}}} \textcolor{Sepia}{\selectlanguage{english}to string beads} \zh{串珠}  
 ¶ \textcolor{darkblue}{\textbf{\ipa{le˧-sɯ˧-se˩-ze˩}}} \textcolor{Sepia}{\selectlanguage{english}(I) have finished to string (beads)} \zh{串完了!}  
 ¶ \textcolor{darkblue}{\textbf{\ipa{tso˧\textasciitilde{}tso˧ sɯ˩}}} \textcolor{Sepia}{\selectlanguage{english}to string things} \zh{串东西}  
\ding{203} \textcolor{Sepia}{\selectlanguage{english}To put on (a skirt).} \zh{穿(裙子)。}  ¶ \textcolor{darkblue}{\textbf{\ipa{ʈʰæ˧qʰwɤ˧ sɯ˧}}} \textcolor{Sepia}{\selectlanguage{english}to put on a skirt} \zh{穿裙子}  

\lhead{\firstmark}
\rhead{\botmark}

\subsection{\hspace{-0.5cm} {\Large \textcolor{darkblue}{\textbf{\ipa{sɯ˧gv̩\#˥}}}}\hspace{0.5cm}[\kern2pt{\textcolor{darkblue}{\textbf{\ipa{sɯ˧gv̩˧}}}}\kern2pt]} \hypertarget{sM\string_Mgv\string_=\#\string_T1}{}
\markboth{\textcolor{darkblue}{\textbf{\ipa{sɯ˧gv̩\#˥}}}}{}
\textcolor{teal}{\mytextsc{noun}} \hspace{4pt} Tone: \#H.
\textcolor{Sepia}{\selectlanguage{english}Box, case.} \zh{箱子,柜子。}  \zh{量词}: \textcolor{darkblue}{\textbf{\ipa{ɭɯ˧}}}  \mytextsc{clf}: \textcolor{darkblue}{\textbf{\ipa{ɭɯ˧}}} 
\lhead{\firstmark}
\rhead{\botmark}

\subsection{\hspace{-0.5cm} {\Large \textcolor{darkblue}{\textbf{\ipa{sɯ˧kʰɯ˩}}}}\hspace{0.5cm}[\kern2pt{\textcolor{darkblue}{\textbf{\ipa{sɯ˩kʰɯ˥}}}}\kern2pt]} \hypertarget{sM\string_Mk\string_hM\string_B1}{}
\markboth{\textcolor{darkblue}{\textbf{\ipa{sɯ˧kʰɯ˩}}}}{}
\textcolor{teal}{\mytextsc{noun}} \hspace{4pt} Tone: L\#.
\textcolor{Sepia}{\selectlanguage{english}Ritual performed for the death of a female relative who left her maternal home to marry.} \zh{斯克:嫁到外边的女人去世时进行的仪式。} 
\lhead{\firstmark}
\rhead{\botmark}

\subsection{\hspace{-0.5cm} {\Large \textcolor{darkblue}{\textbf{\ipa{sɯ˧ljɤ˧}}}}\hspace{0.5cm}[\kern2pt{\textcolor{darkblue}{\textbf{\ipa{sɯ˧ljɤ˧}}}}\kern2pt]} \hypertarget{sM\string_Mlj7\string_M1}{}
\markboth{\textcolor{darkblue}{\textbf{\ipa{sɯ˧ljɤ˧}}}}{}
\textcolor{teal}{\mytextsc{noun}} \hspace{4pt} Tone: M.
\textcolor{Sepia}{\selectlanguage{english}Plastic.} \zh{塑料(汉语借词)。}  Borrowing: Chinese  \zh{塑料}

\lhead{\firstmark}
\rhead{\botmark}

\subsection{\hspace{-0.5cm} {\Large \textcolor{darkblue}{\textbf{\ipa{sɯ˧ljɤ˧tʰo˧˥}}}}\hspace{0.5cm}[\kern2pt{\textcolor{darkblue}{\textbf{\ipa{sɯ˧ljɤ˧tʰo˧}}}}\kern2pt]} \hypertarget{sM\string_Mlj7\string_Mt\string_ho\string_M\string_T1}{}
\markboth{\textcolor{darkblue}{\textbf{\ipa{sɯ˧ljɤ˧tʰo˧˥}}}}{}
\textcolor{teal}{\mytextsc{noun}} \hspace{4pt} Tone: MH\#.
\textcolor{Sepia}{\selectlanguage{english}Plastic jerrican; used to store and transport drinking water.} \zh{塑料桶(汉语借词)。}  \zh{量词}: \textcolor{darkblue}{\textbf{\ipa{ɭɯ˧}}}  \mytextsc{clf}: \textcolor{darkblue}{\textbf{\ipa{ɭɯ˧}}} 
\lhead{\firstmark}
\rhead{\botmark}

\subsection{\hspace{-0.5cm} {\Large \textcolor{darkblue}{\textbf{\ipa{sɯ˧mɤ˩}}}}\hspace{0.5cm}[\kern2pt{\textcolor{darkblue}{\textbf{\ipa{sɯ˧mɤ˧˥}}}}\kern2pt]} \hypertarget{sM\string_Mm7\string_B1}{}
\markboth{\textcolor{darkblue}{\textbf{\ipa{sɯ˧mɤ˩}}}}{}
\textcolor{teal}{\mytextsc{noun}} \hspace{4pt} Tone: L\#.
\textcolor{Sepia}{\selectlanguage{english}Purple perilla, \textit{Perilla frutescens}, akajiso.} \zh{紫苏。}  ¶ \textcolor{darkblue}{\textbf{\ipa{sɯ˧mɤ˩, | ɬi˧di˩ | tv̩˧-kv̩˧˥!}}} \textcolor{Sepia}{\selectlanguage{english}Purple perilla is cultivated in Yongning! / Purple perilla is among the crops that are grown in Yongning!} \zh{在永宁,有紫苏!/ 有人种紫苏!}  
 ¶ \textcolor{darkblue}{\textbf{\ipa{sɯ˧mɤ˩-dze˩}}} \textcolor{Sepia}{\selectlanguage{english}candy containing purple perilla seeds} \zh{含紫苏的糖果}  
 ¶ \textcolor{darkblue}{\textbf{\ipa{sɯ˧mɤ˩-mæ˩ɻæ˩}}} \textcolor{Sepia}{\selectlanguage{english}purple perilla oil} \zh{紫苏油}  

\lhead{\firstmark}
\rhead{\botmark}

\subsection{\hspace{-0.5cm} {\Large \textcolor{darkblue}{\textbf{\ipa{sɯ˧pv̩˩}}}}\hspace{0.5cm}[\kern2pt{\textcolor{darkblue}{\textbf{\ipa{sɯ˧pv̩˩}}}}\kern2pt]} \hypertarget{sM\string_Mpv\string_=\string_B1}{}
\markboth{\textcolor{darkblue}{\textbf{\ipa{sɯ˧pv̩˩}}}}{}
\textcolor{teal}{\mytextsc{noun}} \hspace{4pt} Tone: L\#.
\textcolor{Sepia}{\selectlanguage{english}Urinary bladder.} \zh{膀胱。}  \zh{量词}: \textcolor{darkblue}{\textbf{\ipa{kɤ˧˥}}}  \mytextsc{clf}: \textcolor{darkblue}{\textbf{\ipa{kɤ˧˥}}} 
\lhead{\firstmark}
\rhead{\botmark}

\subsection{\hspace{-0.5cm} {\Large \textcolor{darkblue}{\textbf{\ipa{sɯ˧pv̩˩-ni˩gv̩˩}}}}\hspace{0.5cm}[\kern2pt{\textcolor{darkblue}{\textbf{\ipa{sɯ˧pv̩˩ni˧gv̩˧}}}}\kern2pt]} \hypertarget{sM\string_Mpv\string_=\string_B-ni\string_Bgv\string_=\string_B1}{}
\markboth{\textcolor{darkblue}{\textbf{\ipa{sɯ˧pv̩˩-ni˩gv̩˩}}}}{}
\textcolor{teal}{\mytextsc{adjective}} \hspace{4pt} Tone: L\#-.
\textcolor{Sepia}{\selectlanguage{english}Swollen: literally: 'like a bladder'.} \zh{膀肿。} 
\lhead{\firstmark}
\rhead{\botmark}

\subsection{\hspace{-0.5cm} {\Large \textcolor{darkblue}{\textbf{\ipa{sɯ˧pv̩˧-sɯ˥nɑ˩}}}}\hspace{0.5cm}[\kern2pt{\textcolor{darkblue}{\textbf{\ipa{sɯ˧pv̩˧sɯ˥nɑ˩}}}}\kern2pt]} \hypertarget{sM\string_Mpv\string_=\string_M-sM\string_TnA\string_B1}{}
\markboth{\textcolor{darkblue}{\textbf{\ipa{sɯ˧pv̩˧-sɯ˥nɑ˩}}}}{}
\textcolor{teal}{\mytextsc{noun}} \hspace{4pt} Tone: \#H-.
\textcolor{Sepia}{\selectlanguage{english}Caterpillar.} \zh{毛虫。}  \zh{量词}: \textcolor{darkblue}{\textbf{\ipa{mi˩}}}  \mytextsc{clf}: \textcolor{darkblue}{\textbf{\ipa{mi˩}}} 
\lhead{\firstmark}
\rhead{\botmark}

\subsection{\hspace{-0.5cm} {\Large \textcolor{darkblue}{\textbf{\ipa{sɯ˧pʰi˧}}}}\hspace{0.5cm}[\kern2pt{\textcolor{darkblue}{\textbf{\ipa{sɯ˧pʰi˥}}}}\kern2pt]} \hypertarget{sM\string_Mp\string_hi\string_M1}{}
\markboth{\textcolor{darkblue}{\textbf{\ipa{sɯ˧pʰi˧}}}}{}
\textcolor{teal}{\mytextsc{noun}} \hspace{4pt} Tone: M.
\textcolor{Sepia}{\selectlanguage{english}Chieftain, nobleman, lord: the highest of the three castes (ranks) in feudal society.} \zh{贵族,土司,奴隶主,官。音译:“司沛”。}  ¶ \textcolor{darkblue}{\textbf{\ipa{ɖæ˩mi˧-sɯ˩pʰi˩}}} \textcolor{Sepia}{\selectlanguage{english}the noblemen at the monastery} \zh{大寺贵族}  
 ¶ \textcolor{darkblue}{\textbf{\ipa{sɯ˧pʰi˧ hĩ˩}}} \textcolor{Sepia}{\selectlanguage{english}the nobleman's subjects, the nobleman's people} \zh{贵族的臣子、贵族手下的人}  
 \zh{量词}: \textcolor{darkblue}{\textbf{\ipa{v̩˧}}}  \mytextsc{clf}: \textcolor{darkblue}{\textbf{\ipa{v̩˧}}} 
\lhead{\firstmark}
\rhead{\botmark}

\subsection{\hspace{-0.5cm} {\Large \textcolor{darkblue}{\textbf{\ipa{sɯ˧pʰi˧-zo˧}}}}\hspace{0.5cm}[\kern2pt{\textcolor{darkblue}{\textbf{\ipa{xxxx non-correspondance entre le nombre de morphèmes et le nombre de tons de morphèmes}}}}\kern2pt]} \hypertarget{sM\string_Mp\string_hi\string_M-zo\string_M1}{}
\markboth{\textcolor{darkblue}{\textbf{\ipa{sɯ˧pʰi˧-zo˧}}}}{}
\textcolor{teal}{\mytextsc{noun}} \hspace{4pt} Tone: H\#.
\textcolor{Sepia}{\selectlanguage{english}Young man of the nobility.} \zh{少爷。}  \zh{量词}: \textcolor{darkblue}{\textbf{\ipa{v̩˧}}}  \mytextsc{clf}: \textcolor{darkblue}{\textbf{\ipa{v̩˧}}} 
\lhead{\firstmark}
\rhead{\botmark}

\subsection{\hspace{-0.5cm} {\Large \textcolor{darkblue}{\textbf{\ipa{sɯ˧ɻ̍˧}}}}\hspace{0.5cm}[\kern2pt{\textcolor{darkblue}{\textbf{\ipa{sɯ˧ɻ̍˧}}}}\kern2pt]} \hypertarget{sM\string_Mr£`̍\string_M1}{}
\markboth{\textcolor{darkblue}{\textbf{\ipa{sɯ˧ɻ̍˧}}}}{}
\textcolor{teal}{\mytextsc{noun}} \hspace{4pt} Tone: M.
\textcolor{Sepia}{\selectlanguage{english}Bead, pearl.} \zh{珠,珠子,珍珠。}  \zh{量词}: \textcolor{darkblue}{\textbf{\ipa{ɭɯ˧}}}  \mytextsc{clf}: \textcolor{darkblue}{\textbf{\ipa{ɭɯ˧}}} 
\lhead{\firstmark}
\rhead{\botmark}

\subsection{\hspace{-0.5cm} {\Large \textcolor{darkblue}{\textbf{\ipa{sɯ˧ɻæ˧}}}}\hspace{0.5cm}[\kern2pt{\textcolor{darkblue}{\textbf{\ipa{sɯ˧ɻæ˧}}}}\kern2pt]} \hypertarget{sM\string_Mr£`\{\string_M1}{}
\markboth{\textcolor{darkblue}{\textbf{\ipa{sɯ˧ɻæ˧}}}}{}
\textcolor{teal}{\mytextsc{noun}} \hspace{4pt} Tone: M.
\textcolor{Sepia}{\selectlanguage{english}Table.} \zh{桌子。}  \zh{量词}: \textcolor{darkblue}{\textbf{\ipa{pɤ˩}}}  \mytextsc{clf}: \textcolor{darkblue}{\textbf{\ipa{pɤ˩}}} \textit{See:} \hyperlink{}{\textcolor{darkblue}{\textbf{\ipa{ʈʂo˧tsɯ˥}}}} 
\lhead{\firstmark}
\rhead{\botmark}

\subsection{\hspace{-0.5cm} {\Large \textcolor{darkblue}{\textbf{\ipa{sɯ˧ɻ̃\#˥}}}}\hspace{0.5cm}[\kern2pt{\textcolor{darkblue}{\textbf{\ipa{sɯ˧ɻ̃˧}}}}\kern2pt]} \hypertarget{sM\string_Mr£`\string_~\#\string_T1}{}
\markboth{\textcolor{darkblue}{\textbf{\ipa{sɯ˧ɻ̃\#˥}}}}{}
\textcolor{teal}{\mytextsc{noun}} \hspace{4pt} Tone: \#H.
\textcolor{Sepia}{\selectlanguage{english}Tree trunk.} \zh{树干。}  ¶ \textcolor{darkblue}{\textbf{\ipa{si˧dzi˩ tʰv̩˩-dzi˩, | sɯ˧ɻ̃˧ dʑɤ˥!}}} \textcolor{Sepia}{\selectlanguage{english}This tree has a good trunk! (i.e. it is suitable for use in carpentry, making furniture...)} \zh{这是棵好树!(可以用做木料)}  
 \zh{量词}: \textcolor{darkblue}{\textbf{\ipa{lo˩}}}  \mytextsc{clf}: \textcolor{darkblue}{\textbf{\ipa{lo˩}}} 
\lhead{\firstmark}
\rhead{\botmark}

\subsection{\hspace{-0.5cm} {\Large \textcolor{darkblue}{\textbf{\ipa{sɯ˧ɻ̃˧mi\#˥}}}}\hspace{0.5cm}[\kern2pt{\textcolor{darkblue}{\textbf{\ipa{sɯ˧ɻ̃˧mi˧}}}}\kern2pt]} \hypertarget{sM\string_Mr£`\string_~\string_Mmi\#\string_T1}{}
\markboth{\textcolor{darkblue}{\textbf{\ipa{sɯ˧ɻ̃˧mi\#˥}}}}{}
\textcolor{teal}{\mytextsc{noun}} \hspace{4pt} Tone: \#H.
\textcolor{Sepia}{\selectlanguage{english}Backbone.} \zh{脊椎骨。}  \zh{量词}: \textcolor{darkblue}{\textbf{\ipa{dzi˩}}}  \mytextsc{clf}: \textcolor{darkblue}{\textbf{\ipa{dzi˩}}} 
\lhead{\firstmark}
\rhead{\botmark}

\subsection{\hspace{-0.5cm} {\Large \textcolor{darkblue}{\textbf{\ipa{sɯ˧sɯ˩}}}}\hspace{0.5cm}[\kern2pt{\textcolor{darkblue}{\textbf{\ipa{sɯ˧sɯ˩}}}}\kern2pt]} \hypertarget{sM\string_MsM\string_B1}{}
\markboth{\textcolor{darkblue}{\textbf{\ipa{sɯ˧sɯ˩}}}}{}
\textcolor{teal}{\mytextsc{adjective}} \hspace{4pt} Tone: L\#.
\textcolor{Sepia}{\selectlanguage{english}Raw.} \zh{生(不熟)。}  ¶ \textcolor{darkblue}{\textbf{\ipa{ʂe˧ sɯ˧\textasciitilde{}sɯ˥}}} \textcolor{Sepia}{\selectlanguage{english}raw meat} \zh{生肉}  
 ¶ \textcolor{darkblue}{\textbf{\ipa{ʈʂe˧ sɯ˧\textasciitilde{}sɯ˥}}} \textcolor{Sepia}{\selectlanguage{english}'raw earth': immature soil, earth that has not been prepared for agriculture by adding manure, etc} \zh{‘生土’:没有经过加工(加肥料等等)的土,还不适合种农作物}  

\lhead{\firstmark}
\rhead{\botmark}

\subsection{\hspace{-0.5cm} {\Large \textcolor{darkblue}{\textbf{\ipa{sɯ˧tsɯ˧}}}}\hspace{0.5cm}[\kern2pt{\textcolor{darkblue}{\textbf{\ipa{sɯ˧tsɯ˧}}}}\kern2pt]} \hypertarget{sM\string_MtsM\string_M1}{}
\markboth{\textcolor{darkblue}{\textbf{\ipa{sɯ˧tsɯ˧}}}}{}
\textcolor{teal}{\mytextsc{noun}} \hspace{4pt} Tone: M.
\textcolor{Sepia}{\selectlanguage{english}Lion.} \zh{狮子(汉语借词)。}  Borrowing: Chinese  \zh{狮子}
 \zh{量词}: \textcolor{darkblue}{\textbf{\ipa{mi˩}}}  \mytextsc{clf}: \textcolor{darkblue}{\textbf{\ipa{mi˩}}} 
\lhead{\firstmark}
\rhead{\botmark}

\subsection{\hspace{-0.5cm} {\Large \textcolor{darkblue}{\textbf{\ipa{sɯ˧tsɯ˩}}}}\hspace{0.5cm}[\kern2pt{\textcolor{darkblue}{\textbf{\ipa{sɯ˧tsɯ˩}}}}\kern2pt]} \hypertarget{sM\string_MtsM\string_B1}{}
\markboth{\textcolor{darkblue}{\textbf{\ipa{sɯ˧tsɯ˩}}}}{}
\textcolor{teal}{\mytextsc{noun}} \hspace{4pt} Tone: L\#.
\textcolor{Sepia}{\selectlanguage{english}Camphor.} \zh{樟。}  ¶ \textcolor{darkblue}{\textbf{\ipa{sɯ˧tsɯ˩-dzi˩}}} \textcolor{Sepia}{\selectlanguage{english}camphor tree} \zh{樟树}  

\lhead{\firstmark}
\rhead{\botmark}

\subsection{\hspace{-0.5cm} {\Large \textcolor{darkblue}{\textbf{\ipa{sɯ˧ʈv̩˥}}}}\hspace{0.5cm}[\kern2pt{\textcolor{darkblue}{\textbf{\ipa{sɯ˧ʈv̩˥}}}}\kern2pt]} \hypertarget{sM\string_Mt`v\string_=\string_T1}{}
\markboth{\textcolor{darkblue}{\textbf{\ipa{sɯ˧ʈv̩˥}}}}{}
\textcolor{teal}{\mytextsc{noun}} \hspace{4pt} Tone: H\#.
\textcolor{Sepia}{\selectlanguage{english}Callus.} \zh{茧子。}  ¶ \textcolor{darkblue}{\textbf{\ipa{hĩ˧ ʈʂʰɯ˧-v̩˧-bv̩˧ | mv̩˧ɲi˧, | sɯ˧ʈv̩˥ ʁo˩!}}} \textcolor{Sepia}{\selectlanguage{english}This person's thumb has a callus / developed a callus!} \zh{这个人的拇指有茧子!}  
 ¶ \textcolor{darkblue}{\textbf{\ipa{sɯ˧ʈv̩˥ | mɤ˧-ʁo˩-ze˩!}}} \textcolor{Sepia}{\selectlanguage{english}The callus is gone! / There is no callus anymore!} \zh{没有茧子了!}  
 ¶ \textcolor{darkblue}{\textbf{\ipa{sɯ˧ʈv̩˥ ʁo˩-ze˩!}}} \textcolor{Sepia}{\selectlanguage{english}A callus has formed!} \zh{磨出了茧子!}  
 \zh{量词}: \textcolor{darkblue}{\textbf{\ipa{ɭɯ˧}}}  \mytextsc{clf}: \textcolor{darkblue}{\textbf{\ipa{ɭɯ˧}}} 
\lhead{\firstmark}
\rhead{\botmark}

\subsection{\hspace{-0.5cm} {\Large \textcolor{darkblue}{\textbf{\ipa{sɯ˧zɯ\#˥}}}}\hspace{0.5cm}[\kern2pt{\textcolor{darkblue}{\textbf{\ipa{sɯ˧zɯ˧}}}}\kern2pt]} \hypertarget{sM\string_MzM\#\string_T1}{}
\markboth{\textcolor{darkblue}{\textbf{\ipa{sɯ˧zɯ\#˥}}}}{}
\textcolor{teal}{\mytextsc{noun}} \hspace{4pt} Tone: \#H.
\textcolor{Sepia}{\selectlanguage{english}Family community.} \zh{家族、支系。}  ¶ \textcolor{darkblue}{\textbf{\ipa{sɯ˧zɯ˧ ɖɯ˧-lo˩}}} \textcolor{Sepia}{\selectlanguage{english}one family community} \zh{一个支系,一条线}  
 ¶ \textcolor{darkblue}{\textbf{\ipa{sɯ˧zɯ˧ ɖɯ˧-ʁwɤ˧}}} \textcolor{Sepia}{\selectlanguage{english}one family community} \zh{一个支系,一条线}  
 ¶ \textcolor{darkblue}{\textbf{\ipa{sɯ˧zɯ˧ ə˩-dʑo˩?}}} \textcolor{Sepia}{\selectlanguage{english}Is there a (complete) family community? / Is the family large? (Question asked as part of discussions preliminary to marriage: Will the bride have a large family around her, be surrounded by a large family? A small family is considered much less attractive than a large one.)} \zh{家族齐全吗?/ 家族,人多吗?(谈婚姻前的题目之一:男方家族人多不多。以人多为好。)}  
 \zh{量词}: \textcolor{darkblue}{\textbf{\ipa{lo˩, ʁwɤ˧}}}  \mytextsc{clf}: \textcolor{darkblue}{\textbf{\ipa{lo˩, ʁwɤ˧}}} 
\lhead{\firstmark}
\rhead{\botmark}

\subsection{\hspace{-0.5cm} {\Large \textcolor{darkblue}{\textbf{\ipa{sɯ˩\textsubscript{a}}}}}\hspace{0.5cm}[\kern2pt{\textcolor{darkblue}{\textbf{\ipa{sɯ˩˥}}}}\kern2pt]} \hypertarget{sM\string_Ba1}{}
\markboth{\textcolor{darkblue}{\textbf{\ipa{sɯ˩\textsubscript{a}}}}}{}
\textcolor{teal}{\mytextsc{verb}} \hspace{4pt} Tone: L\textsubscript{a}.
\textcolor{Sepia}{\selectlanguage{english}To live, to be alive.} \zh{活。}  ¶ \textcolor{darkblue}{\textbf{\ipa{ʈʂʰɯ˧ tʰi˧-sɯ˩-dʑo˩!}}} \textcolor{Sepia}{\selectlanguage{english}(S)he is alive!} \zh{他活着!}  
 ¶ \textcolor{darkblue}{\textbf{\ipa{ʈʂʰɯ˧ | mɤ˧-ʂɯ˧! | tʰi˧-sɯ˩-dʑo˩!}}} \textcolor{Sepia}{\selectlanguage{english}It's not dead! It's alive! (About a plant or animal that looked dead)} \zh{它没死,还活着!(一个植物、动物)}  

\lhead{\firstmark}
\rhead{\botmark}

\subsection{\hspace{-0.5cm} {\Large \textcolor{darkblue}{\textbf{\ipa{sɯ˩pv̩˩}}}}\hspace{0.5cm}[\kern2pt{\textcolor{darkblue}{\textbf{\ipa{sɯ˧pv̩˥}}}}\kern2pt]} \hypertarget{sM\string_Bpv\string_=\string_B1}{}
\markboth{\textcolor{darkblue}{\textbf{\ipa{sɯ˩pv̩˩}}}}{}
\textcolor{teal}{\mytextsc{noun}} \hspace{4pt} Tone: .
\textcolor{Sepia}{\selectlanguage{english}Raised spot, blister.} \zh{水泡(例如:开水烫了手,会形成水泡)。}  ¶ \textcolor{darkblue}{\textbf{\ipa{sɯ˩pv̩˩ qʰwæ˥-ze˩!}}} \textcolor{Sepia}{\selectlanguage{english}a raised spot has formed!} \zh{起了水泡!}  
 \zh{量词}: \textcolor{darkblue}{\textbf{\ipa{ɭɯ˧}}}  \mytextsc{clf}: \textcolor{darkblue}{\textbf{\ipa{ɭɯ˧}}} 
\lhead{\firstmark}
\rhead{\botmark}

\subsection{\hspace{-0.5cm} {\Large \textcolor{darkblue}{\textbf{\ipa{sɯ˩ɻ̍˩}}}}\hspace{0.5cm}[\kern2pt{\textcolor{darkblue}{\textbf{\ipa{sɯ˩ɻ̍˩˥}}}}\kern2pt]} \hypertarget{sM\string_Br£`̍\string_B1}{}
\markboth{\textcolor{darkblue}{\textbf{\ipa{sɯ˩ɻ̍˩}}}}{}
\textcolor{teal}{\mytextsc{noun}} \hspace{4pt} Tone: L.
\textcolor{Sepia}{\selectlanguage{english}Whetting-stone.} \zh{磨刀石。}  \zh{量词}: \textcolor{darkblue}{\textbf{\ipa{ɭɯ˧}}}  \mytextsc{clf}: \textcolor{darkblue}{\textbf{\ipa{ɭɯ˧}}} 
\lhead{\firstmark}
\rhead{\botmark}

\subsection{\hspace{-0.5cm} {\Large \textcolor{darkblue}{\textbf{\ipa{sɯ˩tʰi˩}}}}\hspace{0.5cm}[\kern2pt{\textcolor{darkblue}{\textbf{\ipa{sɯ˩tʰi˩˥}}}}\kern2pt]} \hypertarget{sM\string_Bt\string_hi\string_B1}{}
\markboth{\textcolor{darkblue}{\textbf{\ipa{sɯ˩tʰi˩}}}}{}
\textcolor{teal}{\mytextsc{noun}} \hspace{4pt} Tone: L.
\textcolor{Sepia}{\selectlanguage{english}Knife.} \zh{刀。}  \zh{量词}: \textcolor{darkblue}{\textbf{\ipa{nɑ˧}}}  \mytextsc{clf}: \textcolor{darkblue}{\textbf{\ipa{nɑ˧}}} 
\lhead{\firstmark}
\rhead{\botmark}

\subsection{\hspace{-0.5cm} {\Large \textcolor{darkblue}{\textbf{\ipa{sɯ˩tʰi˩-kʰɯ˥ʑi˩}}}}\hspace{0.5cm}[\kern2pt{\textcolor{darkblue}{\textbf{\ipa{xxxx ton non trouvé, à faire manuellement...}}}}\kern2pt]} \hypertarget{sM\string_Bt\string_hi\string_B-k\string_hM\string_Tz£i\string_B1}{}
\markboth{\textcolor{darkblue}{\textbf{\ipa{sɯ˩tʰi˩-kʰɯ˥ʑi˩}}}}{}
\textcolor{teal}{\mytextsc{noun}} \hspace{4pt} Tone: L+\#H-.
\textcolor{Sepia}{\selectlanguage{english}Knife sheath.} \zh{刀鞘。}  \zh{量词}: \textcolor{darkblue}{\textbf{\ipa{ɭɯ˧}}}  \mytextsc{clf}: \textcolor{darkblue}{\textbf{\ipa{ɭɯ˧}}} 
\lhead{\firstmark}
\rhead{\botmark}

\newpage
\section*{\centering- \textcolor{darkblue}{\textbf{\ipa{ʂ}}} -}
\subsection{\hspace{-0.5cm} {\Large \textcolor{darkblue}{\textbf{\ipa{ʂæ˥-ljɤ˩}}}}\hspace{0.5cm}[\kern2pt{\textcolor{darkblue}{\textbf{\ipa{xxxx non-correspondance entre le nombre de morphèmes et le nombre de tons de morphèmes}}}}\kern2pt]} \hypertarget{s`\{\string_T-lj7\string_B1}{}
\markboth{\textcolor{darkblue}{\textbf{\ipa{ʂæ˥-ljɤ˩}}}}{}
\textcolor{teal}{\mytextsc{verb}} \hspace{4pt} Tone: H-.
\textcolor{Sepia}{\selectlanguage{english}To discuss.} \zh{商量(汉语借词。}  Borrowing: Chinese  \zh{商量}

\lhead{\firstmark}
\rhead{\botmark}

\subsection{\hspace{-0.5cm} {\Large \textcolor{darkblue}{\textbf{\ipa{ʂæ˧}}}}\hspace{0.5cm}[\kern2pt{\textcolor{darkblue}{\textbf{\ipa{ʂæ˥}}}}\kern2pt]} \hypertarget{s`\{\string_M1}{}
\markboth{\textcolor{darkblue}{\textbf{\ipa{ʂæ˧}}}}{}
\textcolor{teal}{\mytextsc{adjective}} \hspace{4pt} Tone: M.
\ding{202} \textcolor{Sepia}{\selectlanguage{english}Long.} \zh{长。}  ¶ \textcolor{darkblue}{\textbf{\ipa{qʰɑ˧-ʂæ˧-gv̩˧}}} \textcolor{Sepia}{\selectlanguage{english}extremely long} \zh{非常长}  
 ¶ \textcolor{darkblue}{\textbf{\ipa{le˧-ʈɤ˧-le˧-ʂæ˧ (+kʰɯ˧˥)}}} \textcolor{Sepia}{\selectlanguage{english}to lengthen} \zh{拉长}  
\ding{203} \textcolor{Sepia}{\selectlanguage{english}Distant, far.} \zh{远。} 
\lhead{\firstmark}
\rhead{\botmark}

\subsection{\hspace{-0.5cm} {\Large \textcolor{darkblue}{\textbf{\ipa{ʂæ˧ɖæ\#˥}}}}\hspace{0.5cm}[\kern2pt{\textcolor{darkblue}{\textbf{\ipa{ʂæ˧ɖæ˧}}}}\kern2pt]} \hypertarget{s`\{\string_Md`\{\#\string_T1}{}
\markboth{\textcolor{darkblue}{\textbf{\ipa{ʂæ˧ɖæ\#˥}}}}{}
\textcolor{teal}{\mytextsc{noun}} \hspace{4pt} Tone: \#H.
\textcolor{Sepia}{\selectlanguage{english}Difference in length.} \zh{长度区别。}  ¶ \textcolor{darkblue}{\textbf{\ipa{ʂæ˧ɖæ˧ di˥, | mɤ˧-dʑɤ˩!}}} \textcolor{Sepia}{\selectlanguage{english}If there are differences in length, it's not good / it won't do! (Context: explaining which trees to fell when in need of timber for housebuilding; the trees need to be about the same size.)} \zh{如果长短不一,不好!/不行!(情景:解释砍树时如何选择合适的树)}  
 ¶ \textcolor{darkblue}{\textbf{\ipa{ʂæ˧ɖæ˧ | mɤ˧-di˩!}}} \textcolor{Sepia}{\selectlanguage{english}There are no differences in length! (i.e. the timber is suitable for use in construction; same context as previous example)} \zh{没有长度区别,都一样齐!(等于是好的建房木料)(情景:同上)}  
 \zh{量词}: \textcolor{darkblue}{\textbf{\ipa{kʰwɤ˥}}}  \mytextsc{clf}: \textcolor{darkblue}{\textbf{\ipa{kʰwɤ˥}}} 
\lhead{\firstmark}
\rhead{\botmark}

\subsection{\hspace{-0.5cm} {\Large \textcolor{darkblue}{\textbf{\ipa{ʂæ˧-lo˩pv˩}}}}\hspace{0.5cm}[\kern2pt{\textcolor{darkblue}{\textbf{\ipa{ʂæ˧lo˧pv˧}}}}\kern2pt]} \hypertarget{s`\{\string_M-lo\string_Bpv\string_B1}{}
\markboth{\textcolor{darkblue}{\textbf{\ipa{ʂæ˧-lo˩pv˩}}}}{}
\textcolor{teal}{\mytextsc{noun}} \hspace{4pt} Tone: -L.
\textcolor{Sepia}{\selectlanguage{english}Scabious.} \zh{山萝卜。}  Borrowing: Chinese  \zh{山萝卜}
\textit{See:} \hyperlink{}{\textcolor{darkblue}{\textbf{\ipa{hwɤ˧li˧-hwæ˧qʰæ\#˥}}}} 
\lhead{\firstmark}
\rhead{\botmark}

\subsection{\hspace{-0.5cm} {\Large \textcolor{darkblue}{\textbf{\ipa{ʂæ˧pʰi˧}}}}\hspace{0.5cm}[\kern2pt{\textcolor{darkblue}{\textbf{\ipa{ʂæ˩pʰi˩˥}}}}\kern2pt]} \hypertarget{s`\{\string_Mp\string_hi\string_M1}{}
\markboth{\textcolor{darkblue}{\textbf{\ipa{ʂæ˧pʰi˧}}}}{}
\textcolor{teal}{\mytextsc{noun}} \hspace{4pt} Tone: M.
\textcolor{Sepia}{\selectlanguage{english}Commodity, goods, merchandise.} \zh{商品。}  Borrowing: Chinese  \zh{商品}

\lhead{\firstmark}
\rhead{\botmark}

\subsection{\hspace{-0.5cm} {\Large \textcolor{darkblue}{\textbf{\ipa{ʂæ˧ʁwɤ˩}}}}\hspace{0.5cm}[\kern2pt{\textcolor{darkblue}{\textbf{\ipa{ʂæ˩ʁwɤ˩˥}}}}\kern2pt]} \hypertarget{s`\{\string_MRw7\string_B1}{}
\markboth{\textcolor{darkblue}{\textbf{\ipa{ʂæ˧ʁwɤ˩}}}}{}
\textcolor{teal}{\mytextsc{noun}} \hspace{4pt} Tone: L\#.
\textcolor{Sepia}{\selectlanguage{english}Shuhe: the name of a village in the Lijiang plain.} \zh{束河(旧称:龙泉):丽江坝子里的一个村落。由于束河商人多,经常有束河人到永宁等地,使得相当多的永宁人熟悉那个村落名。}  Borrowing: Naxi  \textcolor{darkblue}{\textbf{\ipa{/sɑ˥wɤ˧/}}}

\lhead{\firstmark}
\rhead{\botmark}

\subsection{\hspace{-0.5cm} {\Large \textcolor{darkblue}{\textbf{\ipa{ʂæ˧tsɯ˧}}}}\hspace{0.5cm}[\kern2pt{\textcolor{darkblue}{\textbf{\ipa{ʂæ˧tsɯ˩}}}}\kern2pt]} \hypertarget{s`\{\string_MtsM\string_M1}{}
\markboth{\textcolor{darkblue}{\textbf{\ipa{ʂæ˧tsɯ˧}}}}{}
\textcolor{teal}{\mytextsc{noun}} \hspace{4pt} Tone: M.
\textcolor{Sepia}{\selectlanguage{english}Kaftan: clothing that children used to wear before they came of age: a loose robe (the same for girls and boys); also worn by adult men in earlier times.} \zh{裋、卡夫坦长衣:成年前男女小孩均穿的裋,成年男人也穿。}  \zh{量词}: \textcolor{darkblue}{\textbf{\ipa{ɭɯ˧˥}}}  \mytextsc{clf}: \textcolor{darkblue}{\textbf{\ipa{ɭɯ˧˥}}} 
\lhead{\firstmark}
\rhead{\botmark}

\subsection{\hspace{-0.5cm} {\Large \textcolor{darkblue}{\textbf{\ipa{ʂæ˩ɻ̃˩}}}}\hspace{0.5cm}[\kern2pt{\textcolor{darkblue}{\textbf{\ipa{ʂæ˧ɻ̃˧˥}}}}\kern2pt]} \hypertarget{s`\{\string_Br£`\string_~\string_B1}{}
\markboth{\textcolor{darkblue}{\textbf{\ipa{ʂæ˩ɻ̃˩}}}}{}
\textcolor{teal}{\mytextsc{noun}} \hspace{4pt} Tone: L.
\textcolor{Sepia}{\selectlanguage{english}Bone.} \zh{骨头。}  \zh{量词}: \textcolor{darkblue}{\textbf{\ipa{kɤ˧˥}}}  \mytextsc{clf}: \textcolor{darkblue}{\textbf{\ipa{kɤ˧˥}}} 
\lhead{\firstmark}
\rhead{\botmark}

\subsection{\hspace{-0.5cm} {\Large \textcolor{darkblue}{\textbf{\ipa{ʂæ˧˥}}} \textsubscript{1}}\hspace{0.5cm}[\kern2pt{\textcolor{darkblue}{\textbf{\ipa{ʂæ˥}}}}\kern2pt]} \hypertarget{s`\{\string_M\string_T1}{}
\markboth{\textcolor{darkblue}{\textbf{\ipa{ʂæ˧˥}}} \textsubscript{1}}{}
\textcolor{teal}{\mytextsc{verb}} \hspace{4pt} Tone: MH.
\textcolor{Sepia}{\selectlanguage{english}To lead along (by hand, halter…).} \zh{牵(牵着牛)。}  ¶ \textcolor{darkblue}{\textbf{\ipa{kʰv̩˧ ʂæ˧˥}}} \textcolor{Sepia}{\selectlanguage{english}to lead a dog; to hunt} \zh{遛狗,狩猎}  
 ¶ \textcolor{darkblue}{\textbf{\ipa{kʰv̩˧ʂæ˧ hɯ˧˥}}} \textcolor{Sepia}{\selectlanguage{english}gone hunting, out hunting} \zh{狩猎去了}  

\lhead{\firstmark}
\rhead{\botmark}

\subsection{\hspace{-0.5cm} {\Large \textcolor{darkblue}{\textbf{\ipa{ʂæ˧˥}}} \textsubscript{2}}\hspace{0.5cm}[\kern2pt{\textcolor{darkblue}{\textbf{\ipa{ʂæ˧˥}}}}\kern2pt]} \hypertarget{s`\{\string_M\string_T2}{}
\markboth{\textcolor{darkblue}{\textbf{\ipa{ʂæ˧˥}}} \textsubscript{2}}{}
\textcolor{teal}{\mytextsc{verb}} \hspace{4pt} Tone: MH.
\ding{202} \textcolor{Sepia}{\selectlanguage{english}To tie into bundles.} \zh{捆成一包。}  ¶ \textcolor{darkblue}{\textbf{\ipa{le˧-ʂæ˧˥}}} \textcolor{Sepia}{\selectlanguage{english}\mytextsc{accomp}} \zh{\mytextsc{accomp}}  
 ¶ \textcolor{darkblue}{\textbf{\ipa{hɑ˧ ʂæ˩}}} \textcolor{Sepia}{\selectlanguage{english}to tie freshly cut rice into bundles} \zh{刚收割的稻子,捆成捆}  
\ding{203} \textcolor{Sepia}{\selectlanguage{english}To wrap, to pack.} \zh{包。}  ¶ \textcolor{darkblue}{\textbf{\ipa{ʂæ˩\textasciitilde{}ʂæ˧˥}}} \textcolor{Sepia}{\selectlanguage{english}\mytextsc{red}: to wrap, to pack} \zh{\mytextsc{重叠:包一包}}  
 ¶ \textcolor{darkblue}{\textbf{\ipa{ʂæ˩\textasciitilde{}ʂæ˧-ze˥}}} \textcolor{Sepia}{\selectlanguage{english}\mytextsc{red} \mytextsc{accomp}} \zh{\mytextsc{red} \mytextsc{accomp}}  
 ¶ \textcolor{darkblue}{\textbf{\ipa{tso˧\textasciitilde{}tso˧ ʂæ˥\textasciitilde{}ʂæ˩}}} \textcolor{Sepia}{\selectlanguage{english}to wrap things} \zh{包一包东西}  

\lhead{\firstmark}
\rhead{\botmark}

\subsection{\hspace{-0.5cm} {\Large \textcolor{darkblue}{\textbf{\ipa{ʂæ˧˥\textsubscript{a}}}}}\hspace{0.5cm}[\kern2pt{\textcolor{darkblue}{\textbf{\ipa{ʂæ˧˥}}}}\kern2pt]} \hypertarget{s`\{\string_M\string_Ta1}{}
\markboth{\textcolor{darkblue}{\textbf{\ipa{ʂæ˧˥\textsubscript{a}}}}}{}
\textcolor{teal}{\mytextsc{classifier}} \hspace{4pt} Tone: MH\textsubscript{a}.
\textcolor{Sepia}{\selectlanguage{english}A sheaf of cut rice or other crop (the amount cut at one go with a sickle and immediately tied together with one sprig).} \zh{量词:捆。}  ¶ \textcolor{darkblue}{\textbf{\ipa{zɯ˧ | ɖɯ˧-ʂæ˧˥}}} \textcolor{Sepia}{\selectlanguage{english}a sheaf of grass} \zh{一捆草}  
 ¶ \textcolor{darkblue}{\textbf{\ipa{ɕi˧ɭɯ˧ | ɖɯ˧-ʂæ˧˥}}} \textcolor{Sepia}{\selectlanguage{english}a sheaf of rice} \zh{一捆稻谷}  

\lhead{\firstmark}
\rhead{\botmark}

\subsection{\hspace{-0.5cm} {\Large \textcolor{darkblue}{\textbf{\ipa{ʂe˥}}} \textsubscript{1}}\hspace{0.5cm}[\kern2pt{\textcolor{darkblue}{\textbf{\ipa{ʂe˥}}}}\kern2pt]} \hypertarget{s`e\string_T1}{}
\markboth{\textcolor{darkblue}{\textbf{\ipa{ʂe˥}}} \textsubscript{1}}{}
\textcolor{teal}{\mytextsc{noun}} \hspace{4pt} Tone: \#H.
\textcolor{Sepia}{\selectlanguage{english}Meat, flesh.} \zh{肉,肌肉。} 
\lhead{\firstmark}
\rhead{\botmark}

\subsection{\hspace{-0.5cm} {\Large \textcolor{darkblue}{\textbf{\ipa{ʂe˥}}} \textsubscript{2}}\hspace{0.5cm}[\kern2pt{\textcolor{darkblue}{\textbf{\ipa{ʂe˥}}}}\kern2pt]} \hypertarget{s`e\string_T2}{}
\markboth{\textcolor{darkblue}{\textbf{\ipa{ʂe˥}}} \textsubscript{2}}{}
\textcolor{teal}{\mytextsc{noun}} \hspace{4pt} Tone: \#H.
\textcolor{Sepia}{\selectlanguage{english}Unripe cereals.} \zh{未熟粮食。}  ¶ \textcolor{darkblue}{\textbf{\ipa{ʂe˧do˧˥}}} \textcolor{Sepia}{\selectlanguage{english}unripe cereals} \zh{未熟粮食}  

\lhead{\firstmark}
\rhead{\botmark}

\subsection{\hspace{-0.5cm} {\Large \textcolor{darkblue}{\textbf{\ipa{ʂe˧\textsubscript{a}}}}}\hspace{0.5cm}[\kern2pt{\textcolor{darkblue}{\textbf{\ipa{ʂe˩˥}}}}\kern2pt]} \hypertarget{s`e\string_Ma1}{}
\markboth{\textcolor{darkblue}{\textbf{\ipa{ʂe˧\textsubscript{a}}}}}{}
\textcolor{teal}{\mytextsc{verb}} \hspace{4pt} Tone: M\textsubscript{a}.
\textcolor{Sepia}{\selectlanguage{english}To look for, to search for; to procure, to get.} \zh{寻找。}  ¶ \textcolor{darkblue}{\textbf{\ipa{le˧-ʂe˧ le˧-ɖɯ˧-ze˧!}}} \textcolor{Sepia}{\selectlanguage{english}(I) looked for something, and I found it!} \zh{(我)找了……就找到了! / 找到了!}  
 ¶ \textcolor{darkblue}{\textbf{\ipa{hĩ˧ ɖɯ˧-v̩˧ ʂe˧}}} \textcolor{Sepia}{\selectlanguage{english}literally 'to look for someone'; meaning: to visit someone of the opposite sex, to frequent someone of the opposite sex (this is typically a masculine activity)} \zh{直译:‘找一个人’。实际含义:去访问异性的人(一般是男人去访问女人)}  
 ¶ \textcolor{darkblue}{\textbf{\ipa{hĩ˧ ʂe˩}}} \textcolor{Sepia}{\selectlanguage{english}to take a wife, to marry a wife} \zh{娶媳妇}  
 ¶ \textcolor{darkblue}{\textbf{\ipa{tso˧\textasciitilde{}tso˧ ʂe˩}}} \textcolor{Sepia}{\selectlanguage{english}to look for things} \zh{找东西}  
 ¶ \textcolor{darkblue}{\textbf{\ipa{lo˧ mɤ˧-dʑo˧, | lo˧ ʂe˧!}}} \textcolor{Sepia}{\selectlanguage{english}[(S)he] looks for complications / creates unnecessary complications! (Literally: 'to look for work when there isn't any'.)} \zh{没事找事!}  
 ¶ \textcolor{darkblue}{\textbf{\ipa{le˧-ʂe˧ tʰi˧-tɕɯ˥}}} \textcolor{Sepia}{\selectlanguage{english}to prepare (e.g. ingredients for a recipe, luggage for travel), to get (something) ready} \zh{准备(做饭的材料、旅途用品……)}  

\lhead{\firstmark}
\rhead{\botmark}

\subsection{\hspace{-0.5cm} {\Large \textcolor{darkblue}{\textbf{\ipa{ʂe˧bæ˧}}}}\hspace{0.5cm}[\kern2pt{\textcolor{darkblue}{\textbf{\ipa{ʂe˩bæ˩˥}}}}\kern2pt]} \hypertarget{s`e\string_Mb\{\string_M1}{}
\markboth{\textcolor{darkblue}{\textbf{\ipa{ʂe˧bæ˧}}}}{}
\textcolor{teal}{\mytextsc{noun}} \hspace{4pt} Tone: M.
\textcolor{Sepia}{\selectlanguage{english}Necklace; chain.} \zh{项圈、项链,锁链。}  ¶ \textcolor{darkblue}{\textbf{\ipa{ŋv̩˩-ʂe˩bæ˥}}} \textcolor{Sepia}{\selectlanguage{english}silver necklace} \zh{银项链}  
 ¶ \textcolor{darkblue}{\textbf{\ipa{hæ̃˩-ʂe˩bæ˥}}} \textcolor{Sepia}{\selectlanguage{english}gold necklace} \zh{金项链}  
 ¶ \textcolor{darkblue}{\textbf{\ipa{ʂe˧mo˧ʂe˧bæ˧, | kʰv̩˩mi˩ pʰæ˩˥!}}} \textcolor{Sepia}{\selectlanguage{english}The iron necklace is used to tie the dog!} \zh{铁链,是来用拴狗的!}  
 ¶ \textcolor{darkblue}{\textbf{\ipa{kʰi˧-ʂe˧bæ˥, | ʂe˧mo˧ po˧-ɳɯ˧ | gv̩˩˥!}}} \textcolor{Sepia}{\selectlanguage{english}The door's chain (the chain used to lock the door) is made of iron!} \zh{铁链,是来用拴狗的!}  
 \zh{量词}: \textcolor{darkblue}{\textbf{\ipa{kʰɯ˩}}}  \mytextsc{clf}: \textcolor{darkblue}{\textbf{\ipa{kʰɯ˩}}} 
\lhead{\firstmark}
\rhead{\botmark}

\subsection{\hspace{-0.5cm} {\Large \textcolor{darkblue}{\textbf{\ipa{ʂe˧bv̩\#˥}}}}\hspace{0.5cm}[\kern2pt{\textcolor{darkblue}{\textbf{\ipa{ʂe˧bv̩˧}}}}\kern2pt]} \hypertarget{s`e\string_Mbv\string_=\#\string_T1}{}
\markboth{\textcolor{darkblue}{\textbf{\ipa{ʂe˧bv̩\#˥}}}}{}
\textcolor{teal}{\mytextsc{noun}} \hspace{4pt} Tone: \#H.
\textcolor{Sepia}{\selectlanguage{english}Sausage, dried meat preserved in intestines.} \zh{香肠,把瘦肉装在肠子里。} 
\lhead{\firstmark}
\rhead{\botmark}

\subsection{\hspace{-0.5cm} {\Large \textcolor{darkblue}{\textbf{\ipa{ʂe˧di˩}}}}\hspace{0.5cm}[\kern2pt{\textcolor{darkblue}{\textbf{\ipa{ʂe˩di˩˥}}}}\kern2pt]} \hypertarget{s`e\string_Mdi\string_B1}{}
\markboth{\textcolor{darkblue}{\textbf{\ipa{ʂe˧di˩}}}}{}
\textcolor{teal}{\mytextsc{adjective}} \hspace{4pt} Tone: L\#.
\textcolor{Sepia}{\selectlanguage{english}Fat (person).} \zh{胖。}  ¶ \textcolor{darkblue}{\textbf{\ipa{ʂe˧ di˩-ze˩!}}} \textcolor{Sepia}{\selectlanguage{english}(He/she) has put on weight!} \zh{胖了!}  

\lhead{\firstmark}
\rhead{\botmark}

\subsection{\hspace{-0.5cm} {\Large \textcolor{darkblue}{\textbf{\ipa{ʂe˧dzo\#˥}}}}\hspace{0.5cm}[\kern2pt{\textcolor{darkblue}{\textbf{\ipa{ʂe˧dzo˩}}}}\kern2pt]} \hypertarget{s`e\string_Mdzo\#\string_T1}{}
\markboth{\textcolor{darkblue}{\textbf{\ipa{ʂe˧dzo\#˥}}}}{}
\textcolor{teal}{\mytextsc{noun}} \hspace{4pt} Tone: \#H.
\textcolor{Sepia}{\selectlanguage{english}Cooking table: a wooden piece of furniture on which one places the chopping board.} \zh{放案板的家具。}  \zh{量词}: \textcolor{darkblue}{\textbf{\ipa{pɤ˩}}}  \mytextsc{clf}: \textcolor{darkblue}{\textbf{\ipa{pɤ˩}}} 
\lhead{\firstmark}
\rhead{\botmark}

\subsection{\hspace{-0.5cm} {\Large \textcolor{darkblue}{\textbf{\ipa{ʂe˧kʰɯ˧}}}}\hspace{0.5cm}[\kern2pt{\textcolor{darkblue}{\textbf{\ipa{ʂe˧kʰɯ˩}}}}\kern2pt]} \hypertarget{s`e\string_Mk\string_hM\string_M1}{}
\markboth{\textcolor{darkblue}{\textbf{\ipa{ʂe˧kʰɯ˧}}}}{}
\textcolor{teal}{\mytextsc{noun}} \hspace{4pt} Tone: M.
\textcolor{Sepia}{\selectlanguage{english}Tripod.} \zh{三脚架。}  \zh{量词}: \textcolor{darkblue}{\textbf{\ipa{nɑ˧}}}  \mytextsc{clf}: \textcolor{darkblue}{\textbf{\ipa{nɑ˧}}} 
\lhead{\firstmark}
\rhead{\botmark}

\subsection{\hspace{-0.5cm} {\Large \textcolor{darkblue}{\textbf{\ipa{ʂe˧mi˧}}}}\hspace{0.5cm}[\kern2pt{\textcolor{darkblue}{\textbf{\ipa{ʂe˩mi˥}}}}\kern2pt]} \hypertarget{s`e\string_Mmi\string_M1}{}
\markboth{\textcolor{darkblue}{\textbf{\ipa{ʂe˧mi˧}}}}{}
\textcolor{teal}{\mytextsc{noun}} \hspace{4pt} Tone: M.
\textcolor{Sepia}{\selectlanguage{english}Louse.} \zh{虱子。}  \zh{量词}: \textcolor{darkblue}{\textbf{\ipa{mi˩}}}  \mytextsc{clf}: \textcolor{darkblue}{\textbf{\ipa{mi˩}}} 
\lhead{\firstmark}
\rhead{\botmark}

\subsection{\hspace{-0.5cm} {\Large \textcolor{darkblue}{\textbf{\ipa{ʂe˧mo˧}}}}\hspace{0.5cm}[\kern2pt{\textcolor{darkblue}{\textbf{\ipa{ʂe˩mo˩˥}}}}\kern2pt]} \hypertarget{s`e\string_Mmo\string_M1}{}
\markboth{\textcolor{darkblue}{\textbf{\ipa{ʂe˧mo˧}}}}{}
\textcolor{teal}{\mytextsc{noun}} \hspace{4pt} Tone: M.
\textcolor{Sepia}{\selectlanguage{english}Iron (disyllable).} \zh{铁(双音节)。} \textit{See:} \hyperlink{}{\textcolor{darkblue}{\textbf{\ipa{ʂe˩}}}} 
\lhead{\firstmark}
\rhead{\botmark}

\subsection{\hspace{-0.5cm} {\Large \textcolor{darkblue}{\textbf{\ipa{ʂe˧nɑ˩}}}}\hspace{0.5cm}[\kern2pt{\textcolor{darkblue}{\textbf{\ipa{ʂe˧nɑ˧}}}}\kern2pt]} \hypertarget{s`e\string_MnA\string_B1}{}
\markboth{\textcolor{darkblue}{\textbf{\ipa{ʂe˧nɑ˩}}}}{}
\textcolor{teal}{\mytextsc{noun}} \hspace{4pt} Tone: L\#.
\textcolor{Sepia}{\selectlanguage{english}Lean meat.} \zh{瘦肉。} 
\lhead{\firstmark}
\rhead{\botmark}

\subsection{\hspace{-0.5cm} {\Large \textcolor{darkblue}{\textbf{\ipa{ʂe˧ɲi˩}}}}\hspace{0.5cm}[\kern2pt{\textcolor{darkblue}{\textbf{\ipa{ʂe˧ɲi˧}}}}\kern2pt]} \hypertarget{s`e\string_MJi\string_B1}{}
\markboth{\textcolor{darkblue}{\textbf{\ipa{ʂe˧ɲi˩}}}}{}
\textcolor{teal}{\mytextsc{noun}} \hspace{4pt} Tone: L\#.
\textcolor{Sepia}{\selectlanguage{english}Fire tongs.} \zh{火钳。}  \zh{量词}: \textcolor{darkblue}{\textbf{\ipa{nɑ˧}}}  \mytextsc{clf}: \textcolor{darkblue}{\textbf{\ipa{nɑ˧}}} 
\lhead{\firstmark}
\rhead{\botmark}

\subsection{\hspace{-0.5cm} {\Large \textcolor{darkblue}{\textbf{\ipa{ʂe˧pv̩˩}}}}\hspace{0.5cm}[\kern2pt{\textcolor{darkblue}{\textbf{\ipa{ʂe˧pv̩˧}}}}\kern2pt]} \hypertarget{s`e\string_Mpv\string_=\string_B1}{}
\markboth{\textcolor{darkblue}{\textbf{\ipa{ʂe˧pv̩˩}}}}{}
\textcolor{teal}{\mytextsc{noun}} \hspace{4pt} Tone: L\#.
\textcolor{Sepia}{\selectlanguage{english}Cured meat; bacon.} \zh{腊肉。} 
\lhead{\firstmark}
\rhead{\botmark}

\subsection{\hspace{-0.5cm} {\Large \textcolor{darkblue}{\textbf{\ipa{ʂe˧qʰv̩˧}}}}\hspace{0.5cm}[\kern2pt{\textcolor{darkblue}{\textbf{\ipa{ʂe˧qʰv̩˧}}}}\kern2pt]} \hypertarget{s`e\string_Mq\string_hv\string_=\string_M1}{}
\markboth{\textcolor{darkblue}{\textbf{\ipa{ʂe˧qʰv̩˧}}}}{}
\textcolor{teal}{\mytextsc{noun}} \hspace{4pt} Tone: M.
\textcolor{Sepia}{\selectlanguage{english}Iron nail; nail.} \zh{铁钉,钉子。}  ¶ \textcolor{darkblue}{\textbf{\ipa{ʂe˧qʰv̩˧ lɑ˧˥}}} \textcolor{Sepia}{\selectlanguage{english}to hammer in a nail, to hit a nail} \zh{钉钉子}  
 \zh{量词}: \textcolor{darkblue}{\textbf{\ipa{ɭɯ˧}}}  \mytextsc{clf}: \textcolor{darkblue}{\textbf{\ipa{ɭɯ˧}}} 
\lhead{\firstmark}
\rhead{\botmark}

\subsection{\hspace{-0.5cm} {\Large \textcolor{darkblue}{\textbf{\ipa{ʂe˧sɑ˩}}}}\hspace{0.5cm}[\kern2pt{\textcolor{darkblue}{\textbf{\ipa{ʂe˧sɑ˧}}}}\kern2pt]} \hypertarget{s`e\string_MsA\string_B1}{}
\markboth{\textcolor{darkblue}{\textbf{\ipa{ʂe˧sɑ˩}}}}{}
\textcolor{teal}{\mytextsc{noun}} \hspace{4pt} Tone: L\#.
\textcolor{Sepia}{\selectlanguage{english}Meat of the limbs of pig. This includes the four limbs; it usually refers to preserved meat, but can also be used to refer to the limbs of the living animal.} \zh{猪腿肉。}  \zh{量词}: \textcolor{darkblue}{\textbf{\ipa{sɑ˧˥}}}  \mytextsc{clf}: \textcolor{darkblue}{\textbf{\ipa{sɑ˧˥}}} 
\lhead{\firstmark}
\rhead{\botmark}

\subsection{\hspace{-0.5cm} {\Large \textcolor{darkblue}{\textbf{\ipa{ʂe˧-sɯ˧sɯ˥}}}}\hspace{0.5cm}[\kern2pt{\textcolor{darkblue}{\textbf{\ipa{xxxx non-correspondance entre le nombre de morphèmes et le nombre de tons de morphèmes}}}}\kern2pt]} \hypertarget{s`e\string_M-sM\string_MsM\string_T1}{}
\markboth{\textcolor{darkblue}{\textbf{\ipa{ʂe˧-sɯ˧sɯ˥}}}}{}
\textcolor{teal}{\mytextsc{noun}} \hspace{4pt} Tone: H\#.
\textcolor{Sepia}{\selectlanguage{english}Raw meat.} \zh{生肉。} 
\lhead{\firstmark}
\rhead{\botmark}

\subsection{\hspace{-0.5cm} {\Large \textcolor{darkblue}{\textbf{\ipa{ʂe˧ʂe˧}}}}\hspace{0.5cm}[\kern2pt{\textcolor{darkblue}{\textbf{\ipa{ʂe˧ʂe˩}}}}\kern2pt]} \hypertarget{s`e\string_Ms`e\string_M1}{}
\markboth{\textcolor{darkblue}{\textbf{\ipa{ʂe˧ʂe˧}}}}{}
\textcolor{teal}{\mytextsc{verb}} \hspace{4pt} Tone: M.
\textcolor{Sepia}{\selectlanguage{english}To catch a cold.} \zh{着凉。}  ¶ \textcolor{darkblue}{\textbf{\ipa{ʂe˧ʂe˧-ze˩}}} \textcolor{Sepia}{\selectlanguage{english}\mytextsc{pfv}} \zh{着凉了}  

\lhead{\firstmark}
\rhead{\botmark}

\subsection{\hspace{-0.5cm} {\Large \textcolor{darkblue}{\textbf{\ipa{ʂe˧ʈʂe˩}}}}\hspace{0.5cm}[\kern2pt{\textcolor{darkblue}{\textbf{\ipa{ʂe˩ʈʂe˩˥}}}}\kern2pt]} \hypertarget{s`e\string_Mt`s`e\string_B1}{}
\markboth{\textcolor{darkblue}{\textbf{\ipa{ʂe˧ʈʂe˩}}}}{}
\textcolor{teal}{\mytextsc{noun}} \hspace{4pt} Tone: L\#.
\textcolor{Sepia}{\selectlanguage{english}Cotton fabric, cloth.} \zh{棉布,布料。}  \zh{量词}: \textcolor{darkblue}{\textbf{\ipa{pʰæ˧˥}}} \textcolor{darkblue}{\textbf{\ipa{kʰɤ˥}}}  \mytextsc{clf}: \textcolor{darkblue}{\textbf{\ipa{pʰæ˧˥}}} \textcolor{darkblue}{\textbf{\ipa{kʰɤ˥}}} 
\lhead{\firstmark}
\rhead{\botmark}

\subsection{\hspace{-0.5cm} {\Large \textcolor{darkblue}{\textbf{\ipa{ʂe˧ʐe\#˥}}}}\hspace{0.5cm}[\kern2pt{\textcolor{darkblue}{\textbf{\ipa{ʂe˧ʐe˩}}}}\kern2pt]} \hypertarget{s`e\string_Mz`e\#\string_T1}{}
\markboth{\textcolor{darkblue}{\textbf{\ipa{ʂe˧ʐe\#˥}}}}{}
\textcolor{teal}{\mytextsc{noun}} \hspace{4pt} Tone: \#H.
\textcolor{Sepia}{\selectlanguage{english}Preserved pork meat.} \zh{腊肉,包括不同几类的腊肉,如火腿等。}  \zh{量词}: \textcolor{darkblue}{\textbf{\ipa{ʐe˥}}}  \mytextsc{clf}: \textcolor{darkblue}{\textbf{\ipa{ʐe˥}}} 
\lhead{\firstmark}
\rhead{\botmark}

\subsection{\hspace{-0.5cm} {\Large \textcolor{darkblue}{\textbf{\ipa{ʂe˩}}}}\hspace{0.5cm}[\kern2pt{\textcolor{darkblue}{\textbf{\ipa{ʂe˥}}}}\kern2pt]} \hypertarget{s`e\string_B1}{}
\markboth{\textcolor{darkblue}{\textbf{\ipa{ʂe˩}}}}{}
\textcolor{teal}{\mytextsc{noun}} \hspace{4pt} Tone: L.
\textcolor{Sepia}{\selectlanguage{english}Iron (monosyllable).} \zh{铁(单音节)。} 
\lhead{\firstmark}
\rhead{\botmark}

\subsection{\hspace{-0.5cm} {\Large \textcolor{darkblue}{\textbf{\ipa{ʂe˩\textsubscript{b}}}}}\hspace{0.5cm}[\kern2pt{\textcolor{darkblue}{\textbf{\ipa{ʂe˥}}}}\kern2pt]} \hypertarget{s`e\string_Bb1}{}
\markboth{\textcolor{darkblue}{\textbf{\ipa{ʂe˩\textsubscript{b}}}}}{}
\textcolor{teal}{\mytextsc{verb}} \hspace{4pt} Tone: L\textsubscript{b}.
\textcolor{Sepia}{\selectlanguage{english}To urinate.} \zh{小便,尿; 屙尿; 解溲; 拉(屎)。}  ¶ \textcolor{darkblue}{\textbf{\ipa{dʑi˧ ʂe˧˥}}} \textcolor{Sepia}{\selectlanguage{english}to urinate} \zh{屙尿}  
 ¶ \textcolor{darkblue}{\textbf{\ipa{qʰæ˧ ʂe˧˥}}} \textcolor{Sepia}{\selectlanguage{english}to defecate} \zh{拉屎}  
 ¶ \textcolor{darkblue}{\textbf{\ipa{le˧-ʂe˩-ze˩}}} \textcolor{Sepia}{\selectlanguage{english}\mytextsc{accomp} \string_ \mytextsc{pfv}} \zh{尿了}  
 ¶ \textcolor{darkblue}{\textbf{\ipa{ɖɯ˧-ʈʰɤ˧ ʂe˧˥}}} \textcolor{Sepia}{\selectlanguage{english}to urinate a drop} \zh{尿一滴尿}  

\lhead{\firstmark}
\rhead{\botmark}

\subsection{\hspace{-0.5cm} {\Large \textcolor{darkblue}{\textbf{\ipa{ʂe˩lɑ˩}}}}\hspace{0.5cm}[\kern2pt{\textcolor{darkblue}{\textbf{\ipa{ʂe˧lɑ˧}}}}\kern2pt]} \hypertarget{s`e\string_BlA\string_B1}{}
\markboth{\textcolor{darkblue}{\textbf{\ipa{ʂe˩lɑ˩}}}}{}
\textcolor{teal}{\mytextsc{noun}} \hspace{4pt} Tone: L.
\textcolor{Sepia}{\selectlanguage{english}To forge.} \zh{打铁。} 
\lhead{\firstmark}
\rhead{\botmark}

\subsection{\hspace{-0.5cm} {\Large \textcolor{darkblue}{\textbf{\ipa{ʂe˩-lɑ˩-hĩ˥}}}}\hspace{0.5cm}[\kern2pt{\textcolor{darkblue}{\textbf{\ipa{xxxx non-correspondance entre le nombre de morphèmes et le nombre de tons de morphèmes}}}}\kern2pt]} \hypertarget{s`e\string_B-lA\string_B-hi\string_~\string_T1}{}
\markboth{\textcolor{darkblue}{\textbf{\ipa{ʂe˩-lɑ˩-hĩ˥}}}}{}
\textcolor{teal}{\mytextsc{noun}} \hspace{4pt} Tone: L+H\#.
\textcolor{Sepia}{\selectlanguage{english}Blacksmith.} \zh{铁匠。}  ¶ \textcolor{darkblue}{\textbf{\ipa{ʂe˩lɑ˩-hĩ˥ hĩ˩}}} \textcolor{Sepia}{\selectlanguage{english}blacksmith} \zh{铁匠}  
 \zh{量词}: \textcolor{darkblue}{\textbf{\ipa{v̩˧}}}  \mytextsc{clf}: \textcolor{darkblue}{\textbf{\ipa{v̩˧}}} 
\lhead{\firstmark}
\rhead{\botmark}

\subsection{\hspace{-0.5cm} {\Large \textcolor{darkblue}{\textbf{\ipa{ʂe˩mɤ˩}}}}\hspace{0.5cm}[\kern2pt{\textcolor{darkblue}{\textbf{\ipa{ʂe˩mɤ˧˥}}}}\kern2pt]} \hypertarget{s`e\string_Bm7\string_B1}{}
\markboth{\textcolor{darkblue}{\textbf{\ipa{ʂe˩mɤ˩}}}}{}
\textcolor{teal}{\mytextsc{noun}} \hspace{4pt} Tone: L.
\textcolor{Sepia}{\selectlanguage{english}Fat meat.} \zh{肥肉。} 
\lhead{\firstmark}
\rhead{\botmark}

\subsection{\hspace{-0.5cm} {\Large \textcolor{darkblue}{\textbf{\ipa{ʂe˩-mo˧˥}}}}\hspace{0.5cm}[\kern2pt{\textcolor{darkblue}{\textbf{\ipa{xxxx non-correspondance entre le nombre de morphèmes et le nombre de tons de morphèmes}}}}\kern2pt]} \hypertarget{s`e\string_B-mo\string_M\string_T1}{}
\markboth{\textcolor{darkblue}{\textbf{\ipa{ʂe˩-mo˧˥}}}}{}
\textcolor{teal}{\mytextsc{noun}} \hspace{4pt} Tone: LM+MH\#.
\textcolor{Sepia}{\selectlanguage{english}Pine mushroom, matsutake, \textit{Tricholoma matsutake}.} \zh{松茸。} 
\lhead{\firstmark}
\rhead{\botmark}

\subsection{\hspace{-0.5cm} {\Large \textcolor{darkblue}{\textbf{\ipa{ʂe˩ʂv̩˩}}}}\hspace{0.5cm}[\kern2pt{\textcolor{darkblue}{\textbf{\ipa{ʂe˧ʂv̩˥}}}}\kern2pt]} \hypertarget{s`e\string_Bs`v\string_=\string_B1}{}
\markboth{\textcolor{darkblue}{\textbf{\ipa{ʂe˩ʂv̩˩}}}}{}
\textcolor{teal}{\mytextsc{noun}} \hspace{4pt} Tone: L.
\textcolor{Sepia}{\selectlanguage{english}Nit, egg of louse.} \zh{虮子。}  \zh{量词}: \textcolor{darkblue}{\textbf{\ipa{ɭɯ˧}}}  \mytextsc{clf}: \textcolor{darkblue}{\textbf{\ipa{ɭɯ˧}}} 
\lhead{\firstmark}
\rhead{\botmark}

\subsection{\hspace{-0.5cm} {\Large \textcolor{darkblue}{\textbf{\ipa{ʂɤ˧do˧˥}}}}\hspace{0.5cm}[\kern2pt{\textcolor{darkblue}{\textbf{\ipa{ʂɤ˩do˩˥}}}}\kern2pt]} \hypertarget{s`7\string_Mdo\string_M\string_T1}{}
\markboth{\textcolor{darkblue}{\textbf{\ipa{ʂɤ˧do˧˥}}}}{}
\textcolor{teal}{\mytextsc{adjective}} \hspace{4pt} Tone: MH\#.
\ding{202} \textcolor{Sepia}{\selectlanguage{english}Ashamed.} \zh{害羞。}  ¶ \textcolor{darkblue}{\textbf{\ipa{ʂɤ˧do˧ mɤ˧-sɯ˥!}}} \textcolor{Sepia}{\selectlanguage{english}[(S)he] is sullen / impudent / has no sense of shame} \zh{不知羞耻!}  
\ding{203} \textcolor{Sepia}{\selectlanguage{english}Modest, demure, discreet, polite.} \zh{娴静、礼貌。}  ¶ \textcolor{darkblue}{\textbf{\ipa{ʈʂʰɯ˧ ʂɤ˧do˧-zo˥! / ʂɤ˧do˧ ʝi˥!}}} \textcolor{Sepia}{\selectlanguage{english}(S)he is very modest/discreet/polite!} \zh{他/她很娴静 / 很持重!}  

\lhead{\firstmark}
\rhead{\botmark}

\subsection{\hspace{-0.5cm} {\Large \textcolor{darkblue}{\textbf{\ipa{ʂɤ˧ɲi\#˥}}}}\hspace{0.5cm}[\kern2pt{\textcolor{darkblue}{\textbf{\ipa{ʂɤ˧ɲi˧}}}}\kern2pt]} \hypertarget{s`7\string_MJi\#\string_T1}{}
\markboth{\textcolor{darkblue}{\textbf{\ipa{ʂɤ˧ɲi\#˥}}}}{}
\textcolor{teal}{\mytextsc{noun}} \hspace{4pt} Tone: \#H.
\textcolor{Sepia}{\selectlanguage{english}Advice, suggestion, recommendation.} \zh{建议、意见。}  ¶ \textcolor{darkblue}{\textbf{\ipa{ʂɤ˧ɲi˧ ʑi˧˥}}} \textcolor{Sepia}{\selectlanguage{english}to ask for advice / to ask for counsel} \zh{请求意见,求教}  
 ¶ \textcolor{darkblue}{\textbf{\ipa{no˧ | hĩ˧-ki˧ | ʂɤ˧ɲi˧ mɤ˧-ʑi˧-zo˥!}}} \textcolor{Sepia}{\selectlanguage{english}You shouldn't ask around for advice! / There is no need for you to ask for anyone's advice! (=You can make a decision by yourself.)} \zh{你不要问人家的意见!}  
 ¶ \textcolor{darkblue}{\textbf{\ipa{ə˧tse˧ʝi˧-zo˥ | ʂɤ˧ɲi˧ ʑi˧-tso˧-ɲi˥?}}} \textcolor{Sepia}{\selectlanguage{english}Why would you want to ask for (his/her) advice?} \zh{你为什么要问(他的)意见!}  
 \zh{量词}: \textcolor{darkblue}{\textbf{\ipa{kʰwɤ˥}}}  \mytextsc{clf}: \textcolor{darkblue}{\textbf{\ipa{kʰwɤ˥}}} 
\lhead{\firstmark}
\rhead{\botmark}

\subsection{\hspace{-0.5cm} {\Large \textcolor{darkblue}{\textbf{\ipa{ʂɤ˩\textsubscript{a}}}}}\hspace{0.5cm}[\kern2pt{\textcolor{darkblue}{\textbf{\ipa{ʂɤ˩˥}}}}\kern2pt]} \hypertarget{s`7\string_Ba1}{}
\markboth{\textcolor{darkblue}{\textbf{\ipa{ʂɤ˩\textsubscript{a}}}}}{}
\textcolor{teal}{\mytextsc{verb}} \hspace{4pt} Tone: L\textsubscript{a}.
\textcolor{Sepia}{\selectlanguage{english}To tear, to rip.} \zh{撕(纸……)。}  ¶ \textcolor{darkblue}{\textbf{\ipa{tso˧\textasciitilde{}tso˧ ʂɤ˥}}} \textcolor{Sepia}{\selectlanguage{english}to tear things} \zh{撕东西}  
 ¶ \textcolor{darkblue}{\textbf{\ipa{tso˧\textasciitilde{}tso˧ ʂɤ˧\textasciitilde{}ʂɤ˥ (+ze˩)}}} \textcolor{Sepia}{\selectlanguage{english}to tear things} \zh{撕东西}  
 ¶ \textcolor{darkblue}{\textbf{\ipa{le˧-ʂɤ˧\textasciitilde{}ʂɤ˥+ze˩}}} \textcolor{Sepia}{\selectlanguage{english}\mytextsc{accomp} \string_ \mytextsc{red} \mytextsc{pfv}} \zh{撕了}  

\lhead{\firstmark}
\rhead{\botmark}

\subsection{\hspace{-0.5cm} {\Large \textcolor{darkblue}{\textbf{\ipa{ʂɤ˩ŋɤ\#˥}}}}\hspace{0.5cm}[\kern2pt{\textcolor{darkblue}{\textbf{\ipa{ʂɤ˩ŋɤ˥}}}}\kern2pt]} \hypertarget{s`7\string_BN7\#\string_T1}{}
\markboth{\textcolor{darkblue}{\textbf{\ipa{ʂɤ˩ŋɤ\#˥}}}}{}
\textcolor{teal}{\mytextsc{noun}} \hspace{4pt} Tone: LM+\#H.
\textcolor{Sepia}{\selectlanguage{english}Gong.} \zh{锣。}  ¶ \textcolor{darkblue}{\textbf{\ipa{ʂɤ˩ŋɤ˧ lɑ˩}}} \textcolor{Sepia}{\selectlanguage{english}to play the gong} \zh{打锣}  
 \zh{量词}: \textcolor{darkblue}{\textbf{\ipa{ɭɯ˧}}}  \mytextsc{clf}: \textcolor{darkblue}{\textbf{\ipa{ɭɯ˧}}} 
\lhead{\firstmark}
\rhead{\botmark}

\subsection{\hspace{-0.5cm} {\Large \textcolor{darkblue}{\textbf{\ipa{ʂo˥}}}}\hspace{0.5cm}[\kern2pt{\textcolor{darkblue}{\textbf{\ipa{ʂo˥}}}}\kern2pt]} \hypertarget{s`o\string_T1}{}
\markboth{\textcolor{darkblue}{\textbf{\ipa{ʂo˥}}}}{}
\textcolor{teal}{\mytextsc{verb}} \hspace{4pt} Tone: H.
\textcolor{Sepia}{\selectlanguage{english}To reap, to gather in.} \zh{收割。}  ¶ \textcolor{darkblue}{\textbf{\ipa{le˧-ʂo˥-ze˩}}} \textcolor{Sepia}{\selectlanguage{english}\mytextsc{accomp} \string_ \mytextsc{pfv}} \zh{收割了}  
 ¶ \textcolor{darkblue}{\textbf{\ipa{ɖɯ˧-kʰv̩˥ ɖɯ˧-ʂɯ˩ | gɤ˩-ʂo˥-ze˩!}}} \textcolor{Sepia}{\selectlanguage{english}We have one harvest (of rice) every year!} \zh{每年收一次稻谷!}  

\lhead{\firstmark}
\rhead{\botmark}

\subsection{\hspace{-0.5cm} {\Large \textcolor{darkblue}{\textbf{\ipa{ʂo˧}}}}\hspace{0.5cm}[\kern2pt{\textcolor{darkblue}{\textbf{\ipa{ʂo˥}}}}\kern2pt]} \hypertarget{s`o\string_M1}{}
\markboth{\textcolor{darkblue}{\textbf{\ipa{ʂo˧}}}}{}
\textcolor{teal}{\mytextsc{verb}} \hspace{4pt} Tone: M.
\textcolor{Sepia}{\selectlanguage{english}To gather.} \zh{收集。}  ¶ \textcolor{darkblue}{\textbf{\ipa{le˧-ʂo˧\textasciitilde{}ʂo˧}}} \textcolor{Sepia}{\selectlanguage{english}\mytextsc{accomp} \string_ \mytextsc{red}} \zh{\mytextsc{accomp} \string_ \mytextsc{red}}  
 ¶ \textcolor{darkblue}{\textbf{\ipa{ʂo˧\textasciitilde{}ʂo˧-zo˧-ho˩-ze˩}}} \textcolor{Sepia}{\selectlanguage{english}We are going to have to gather (things)} \zh{该收集一些了。}  

\lhead{\firstmark}
\rhead{\botmark}

\subsection{\hspace{-0.5cm} {\Large \textcolor{darkblue}{\textbf{\ipa{ʂo˧}}}}\hspace{0.5cm}[\kern2pt{\textcolor{darkblue}{\textbf{\ipa{ʂo˥}}}}\kern2pt]} \hypertarget{s`o\string_M1}{}
\markboth{\textcolor{darkblue}{\textbf{\ipa{ʂo˧}}}}{}
\textcolor{teal}{\mytextsc{interjection}} \hspace{4pt} Tone: M.
\textcolor{Sepia}{\selectlanguage{english}Interjection to get pigs to move forward.} \zh{赶猪用的叹词:走!走!。}  ¶ \textcolor{darkblue}{\textbf{\ipa{ʂo˧! / ʂo˧bɤ˩!}}} \textcolor{Sepia}{\selectlanguage{english}interjection to get pigs to move forward} \zh{赶猪用的叹词}  

\lhead{\firstmark}
\rhead{\botmark}

\subsection{\hspace{-0.5cm} {\Large \textcolor{darkblue}{\textbf{\ipa{ʂo˩\textsubscript{a}}}}}\hspace{0.5cm}[\kern2pt{\textcolor{darkblue}{\textbf{\ipa{ʂo˥}}}}\kern2pt]} \hypertarget{s`o\string_Ba1}{}
\markboth{\textcolor{darkblue}{\textbf{\ipa{ʂo˩\textsubscript{a}}}}}{}
\textcolor{teal}{\mytextsc{adjective}} \hspace{4pt} Tone: L\textsubscript{a}.
\textcolor{Sepia}{\selectlanguage{english}Clean; clear (water).} \zh{干净、整洁,本质干净,清(水)。}  ¶ \textcolor{darkblue}{\textbf{\ipa{ʂo˩-hĩ˩˥}}} \textcolor{Sepia}{\selectlanguage{english}\mytextsc{nmlz}} \zh{干净的}  
 ¶ \textcolor{darkblue}{\textbf{\ipa{mɤ˧-ʂo˩}}} \textcolor{Sepia}{\selectlanguage{english}not clean} \zh{不干净、脏}  
 ¶ \textcolor{darkblue}{\textbf{\ipa{ʈʂʰɯ˧ | ʂo˩-hĩ˩ ɲi˥. |}}} \textcolor{Sepia}{\selectlanguage{english}This is clean.} \zh{这是干净的。}  
 ¶ \textcolor{darkblue}{\textbf{\ipa{dʑɯ˧ ʂo˧}}} \textcolor{Sepia}{\selectlanguage{english}clean water, clear water} \zh{清水、干净的水}  

\lhead{\firstmark}
\rhead{\botmark}

\subsection{\hspace{-0.5cm} {\Large \textcolor{darkblue}{\textbf{\ipa{ʂo˩qæ˩}}}}\hspace{0.5cm}[\kern2pt{\textcolor{darkblue}{\textbf{\ipa{ʂo˩qæ˩˥}}}}\kern2pt]} \hypertarget{s`o\string_Bq\{\string_B1}{}
\markboth{\textcolor{darkblue}{\textbf{\ipa{ʂo˩qæ˩}}}}{}
\textcolor{teal}{\mytextsc{adjective}} \hspace{4pt} Tone: L.
\textcolor{Sepia}{\selectlanguage{english}Very clean.} \zh{很干净。}  ¶ \textcolor{darkblue}{\textbf{\ipa{ʂo˩qæ˩˥ | -gv̩˩}}} \textcolor{Sepia}{\selectlanguage{english}very clean} \zh{很干净}  
 ¶ \textcolor{darkblue}{\textbf{\ipa{ɑ˩ʁo˧ | le˧-gv̩˧\textasciitilde{}gv̩˥ | ʂo˩qæ˩˥ | -gv̩˩}}} \textcolor{Sepia}{\selectlanguage{english}to put the house in order, that it be very clean} \zh{家收拾得干干净净}  

\lhead{\firstmark}
\rhead{\botmark}

\subsection{\hspace{-0.5cm} {\Large \textcolor{darkblue}{\textbf{\ipa{ʂo˧˥}}} \textsubscript{1}}\hspace{0.5cm}[\kern2pt{\textcolor{darkblue}{\textbf{\ipa{ʂo˧˥}}}}\kern2pt]} \hypertarget{s`o\string_M\string_T1}{}
\markboth{\textcolor{darkblue}{\textbf{\ipa{ʂo˧˥}}} \textsubscript{1}}{}
\textcolor{teal}{\mytextsc{verb}} \hspace{4pt} Tone: MH.
\textcolor{Sepia}{\selectlanguage{english}To slip, to slide.} \zh{滑,光滑(路……)。}  ¶ \textcolor{darkblue}{\textbf{\ipa{mv̩˩tɕo˧ ʂo˧˥}}} \textcolor{Sepia}{\selectlanguage{english}to slide down, to slip to the floor} \zh{滑下、滑倒}  
 ¶ \textcolor{darkblue}{\textbf{\ipa{ʈʂʰɯ˧ | le˧-ʂo˧˥, | tʰi˧-ʈwæ˧-ze˥}}} \textcolor{Sepia}{\selectlanguage{english}(S)he slipped and fell down} \zh{他滑了一跤}  
 ¶ \textcolor{darkblue}{\textbf{\ipa{ʂo˩\textasciitilde{}ʂo˧˥}}} \textcolor{Sepia}{\selectlanguage{english}\mytextsc{red}} \zh{\mytextsc{重叠}}  
 ¶ \textcolor{darkblue}{\textbf{\ipa{ɖæ˩ʂo˩˥ / ɖæ˩ʂo˩-ze˥}}} \textcolor{Sepia}{\selectlanguage{english}to slide down} \zh{往下滑}  
 ¶ \textcolor{darkblue}{\textbf{\ipa{no˧ | ɖæ˩ʂo˩\textasciitilde{}ɖæ˥ʂo˩! |}}} \textcolor{Sepia}{\selectlanguage{english}You are really cunning! (A criticism of someone who is not direct, not honest, who does not have a proper attitude: giving a slimy impression.)} \zh{你真滑头!}  
\textit{See:} \hyperlink{}{\textcolor{darkblue}{\textbf{\ipa{ʂo˧˥}}} \textsubscript{2}} 
\lhead{\firstmark}
\rhead{\botmark}

\subsection{\hspace{-0.5cm} {\Large \textcolor{darkblue}{\textbf{\ipa{ʂo˧˥}}} \textsubscript{2}}\hspace{0.5cm}[\kern2pt{\textcolor{darkblue}{\textbf{\ipa{ʂo˧˥}}}}\kern2pt]} \hypertarget{s`o\string_M\string_T2}{}
\markboth{\textcolor{darkblue}{\textbf{\ipa{ʂo˧˥}}} \textsubscript{2}}{}
\textcolor{teal}{\mytextsc{adjective}} \hspace{4pt} Tone: MH.
\textcolor{Sepia}{\selectlanguage{english}Slippery.} \zh{光滑(路……)。}  ¶ \textcolor{darkblue}{\textbf{\ipa{mɤ˩ ʂo˩-ʂo˥ |}}} \textcolor{Sepia}{\selectlanguage{english}slippery with grease} \zh{油腻腻、滑腻}  
 ¶ \textcolor{darkblue}{\textbf{\ipa{ɲi˧to˧ ɖɯ˧-ɭɯ˧ | dze˧-ʂo˧\textasciitilde{}ʂo˥}}} \textcolor{Sepia}{\selectlanguage{english}(her/his) whole mouth was slippery with sugar} \zh{他嘴巴被糖粘得黏黏的}  
\textit{See:} \hyperlink{}{\textcolor{darkblue}{\textbf{\ipa{ʂo˧˥}}} \textsubscript{1}} 
\lhead{\firstmark}
\rhead{\botmark}

\subsection{\hspace{-0.5cm} {\Large \textcolor{darkblue}{\textbf{\ipa{ʂɻ̍˧˥}}}}\hspace{0.5cm}[\kern2pt{\textcolor{darkblue}{\textbf{\ipa{ʂɻ̍˧˥}}}}\kern2pt]} \hypertarget{s`r£`̍\string_M\string_T1}{}
\markboth{\textcolor{darkblue}{\textbf{\ipa{ʂɻ̍˧˥}}}}{}
\textcolor{teal}{\mytextsc{adjective}} \hspace{4pt} Tone: MH.
\textcolor{Sepia}{\selectlanguage{english}Full.} \zh{满。}  ¶ \textcolor{darkblue}{\textbf{\ipa{le˧-ʂɻ̍˧-ze˥}}} \textcolor{Sepia}{\selectlanguage{english}\mytextsc{accomp} \string_ \mytextsc{pfv}} \zh{满了}  

\lhead{\firstmark}
\rhead{\botmark}

\subsection{\hspace{-0.5cm} {\Large \textcolor{darkblue}{\textbf{\ipa{ʂɯ˧}}}}\hspace{0.5cm}[\kern2pt{\textcolor{darkblue}{\textbf{\ipa{ʂɯ˥}}}}\kern2pt]} \hypertarget{s`M\string_M1}{}
\markboth{\textcolor{darkblue}{\textbf{\ipa{ʂɯ˧}}}}{}
\textcolor{teal}{\mytextsc{number}} \hspace{4pt} Tone: M? H\#? (pas L).
\textcolor{Sepia}{\selectlanguage{english}7.} \zh{7。} 
\lhead{\firstmark}
\rhead{\botmark}

\subsection{\hspace{-0.5cm} {\Large \textcolor{darkblue}{\textbf{\ipa{ʂɯ˧\textsubscript{a}}}} \textsubscript{1}}\hspace{0.5cm}[\kern2pt{\textcolor{darkblue}{\textbf{\ipa{ʂɯ˥}}}}\kern2pt]} \hypertarget{s`M\string_Ma1}{}
\markboth{\textcolor{darkblue}{\textbf{\ipa{ʂɯ˧\textsubscript{a}}}} \textsubscript{1}}{}
\textcolor{teal}{\mytextsc{verb}} \hspace{4pt} Tone: M\textsubscript{a}.
\textcolor{Sepia}{\selectlanguage{english}To leak.} \zh{漏。}  ¶ \textcolor{darkblue}{\textbf{\ipa{tʰi˧-ʂɯ˥\textasciitilde{}ʂɯ˩(-ze˩)}}} \textcolor{Sepia}{\selectlanguage{english}it is leaking} \zh{漏了!}  
 ¶ \textcolor{darkblue}{\textbf{\ipa{mɤ˧-ʂɯ˥\textasciitilde{}ʂɯ˩! | mɤ˧-ʑi˧!}}} \textcolor{Sepia}{\selectlanguage{english}It does not leak; it does not flow out!} \zh{没漏,没流出去!}  

\lhead{\firstmark}
\rhead{\botmark}

\subsection{\hspace{-0.5cm} {\Large \textcolor{darkblue}{\textbf{\ipa{ʂɯ˧\textsubscript{a}}}} \textsubscript{2}}\hspace{0.5cm}[\kern2pt{\textcolor{darkblue}{\textbf{\ipa{ʂɯ˥}}}}\kern2pt]} \hypertarget{s`M\string_Ma2}{}
\markboth{\textcolor{darkblue}{\textbf{\ipa{ʂɯ˧\textsubscript{a}}}} \textsubscript{2}}{}
\textcolor{teal}{\mytextsc{verb}} \hspace{4pt} Tone: M\textsubscript{a}.
\textcolor{Sepia}{\selectlanguage{english}To die.} \zh{死。}  ¶ \textcolor{darkblue}{\textbf{\ipa{le˧-ʂɯ˧-ho˩-ze˩}}} \textcolor{Sepia}{\selectlanguage{english}It's going to die! (About a sick plant or animal)} \zh{快要死了!(病了的植物、动物)}  
 ¶ \textcolor{darkblue}{\textbf{\ipa{mɤ˧-ʂɯ˧-sɯ˩!}}} \textcolor{Sepia}{\selectlanguage{english}(He/she/it) is not dead yet!} \zh{还没死!}  
 ¶ \textcolor{darkblue}{\textbf{\ipa{no˧ | le˧-ʂɯ˧-bi˧-tsæ˧-ɲi˧-ze˩!}}} \textcolor{Sepia}{\selectlanguage{english}Go and die! / May you die! (Imprecation)} \zh{你去死吧!}  

\lhead{\firstmark}
\rhead{\botmark}

\subsection{\hspace{-0.5cm} {\Large \textcolor{darkblue}{\textbf{\ipa{ʂɯ˧dʑi˧}}}}\hspace{0.5cm}[\kern2pt{\textcolor{darkblue}{\textbf{\ipa{ʂɯ˩dʑi˩˥}}}}\kern2pt]} \hypertarget{s`M\string_Mdz£i\string_M1}{}
\markboth{\textcolor{darkblue}{\textbf{\ipa{ʂɯ˧dʑi˧}}}}{}
\textcolor{teal}{\mytextsc{noun}} \hspace{4pt} Tone: M.
\textcolor{Sepia}{\selectlanguage{english}Shroud, burial suit.} \zh{寿衣。}  ¶ \textcolor{darkblue}{\textbf{\ipa{ʂɯ˧dʑi˧ ʐv̩˥}}} \textcolor{Sepia}{\selectlanguage{english}to sew the burial suit, to sew the shroud} \zh{缝寿衣}  

\lhead{\firstmark}
\rhead{\botmark}

\subsection{\hspace{-0.5cm} {\Large \textcolor{darkblue}{\textbf{\ipa{ʂɯ˧-ɬi˧mi˧}}}}\hspace{0.5cm}[\kern2pt{\textcolor{darkblue}{\textbf{\ipa{xxxx non-correspondance entre le nombre de morphèmes et le nombre de tons de morphèmes}}}}\kern2pt]} \hypertarget{s`M\string_M-Ki\string_Mmi\string_M1}{}
\markboth{\textcolor{darkblue}{\textbf{\ipa{ʂɯ˧-ɬi˧mi˧}}}}{}
\textcolor{teal}{\mytextsc{noun}} \hspace{4pt} Tone: M.
\textcolor{Sepia}{\selectlanguage{english}7th month.} \zh{七月。} 
\lhead{\firstmark}
\rhead{\botmark}

\subsection{\hspace{-0.5cm} {\Large \textcolor{darkblue}{\textbf{\ipa{ʂɯ˧ɲi˥}}}}\hspace{0.5cm}[\kern2pt{\textcolor{darkblue}{\textbf{\ipa{ʂɯ˧ɲi˩}}}}\kern2pt]} \hypertarget{s`M\string_MJi\string_T1}{}
\markboth{\textcolor{darkblue}{\textbf{\ipa{ʂɯ˧ɲi˥}}}}{}
\textcolor{teal}{\mytextsc{adverb(ial)}} \hspace{4pt} Tone: H\#.
\textcolor{Sepia}{\selectlanguage{english}The day before yesterday.} \zh{前天。}  ¶ \textcolor{darkblue}{\textbf{\ipa{ʂɯ˧ɲi˥ | -ɖɯ˧ɲi˥}}} \textcolor{Sepia}{\selectlanguage{english}the day before yesterday} \zh{前天}  

\lhead{\firstmark}
\rhead{\botmark}

\subsection{\hspace{-0.5cm} {\Large \textcolor{darkblue}{\textbf{\ipa{ʂɯ˧ʂɯ˧-dzi˩}}}}\hspace{0.5cm}[\kern2pt{\textcolor{darkblue}{\textbf{\ipa{ʂɯ˧ʂɯ˧dzi˧}}}}\kern2pt]} \hypertarget{s`M\string_Ms`M\string_M-dzi\string_B1}{}
\markboth{\textcolor{darkblue}{\textbf{\ipa{ʂɯ˧ʂɯ˧-dzi˩}}}}{}
\textcolor{teal}{\mytextsc{noun}} \hspace{4pt} Tone: -L.
\textcolor{Sepia}{\selectlanguage{english}Yyyy.} \zh{三颗针。}  \zh{量词}: \textcolor{darkblue}{\textbf{\ipa{dzi˩}}}  \mytextsc{clf}: \textcolor{darkblue}{\textbf{\ipa{dzi˩}}} 
\lhead{\firstmark}
\rhead{\botmark}

\subsection{\hspace{-0.5cm} {\Large \textcolor{darkblue}{\textbf{\ipa{ʂɯ˧tɤ˧ɻ\#˥}}}}\hspace{0.5cm}[\kern2pt{\textcolor{darkblue}{\textbf{\ipa{ʂɯ˧tɤ˧ɻ˧}}}}\kern2pt]} \hypertarget{s`M\string_Mt7\string_Mr£`\#\string_T1}{}
\markboth{\textcolor{darkblue}{\textbf{\ipa{ʂɯ˧tɤ˧ɻ\#˥}}}}{}
\textcolor{teal}{\mytextsc{adjective}} \hspace{4pt} Tone: \#H.
\textcolor{Sepia}{\selectlanguage{english}Smooth (e.g. carefully planed wood).} \zh{平滑。}  ¶ \textcolor{darkblue}{\textbf{\ipa{ʂɯ˧tɤ˧ɻ̍˧-zo˥}}} \textcolor{Sepia}{\selectlanguage{english}very smooth} \zh{很平滑}  
 ¶ \textcolor{darkblue}{\textbf{\ipa{ʂɯ˧tɤ˧ɻ̍˧ gv̩˧-ze˩}}} \textcolor{Sepia}{\selectlanguage{english}(it) was made nice and smooth} \zh{弄得平滑了}  

\lhead{\firstmark}
\rhead{\botmark}

\subsection{\hspace{-0.5cm} {\Large \textcolor{darkblue}{\textbf{\ipa{ʂɯ˧tsʰi˩}}}}\hspace{0.5cm}[\kern2pt{\textcolor{darkblue}{\textbf{\ipa{ʂɯ˧tsʰi˩}}}}\kern2pt]} \hypertarget{s`M\string_Mts\string_hi\string_B1}{}
\markboth{\textcolor{darkblue}{\textbf{\ipa{ʂɯ˧tsʰi˩}}}}{}
\textcolor{teal}{\mytextsc{number}} \hspace{4pt} Tone: L\#.
\textcolor{Sepia}{\selectlanguage{english}70.} \zh{70。} 
\lhead{\firstmark}
\rhead{\botmark}

\subsection{\hspace{-0.5cm} {\Large \textcolor{darkblue}{\textbf{\ipa{ʂɯ˩\textsubscript{b}}}}}\hspace{0.5cm}[\kern2pt{\textcolor{darkblue}{\textbf{\ipa{ʂɯ˥}}}}\kern2pt]} \hypertarget{s`M\string_Bb1}{}
\markboth{\textcolor{darkblue}{\textbf{\ipa{ʂɯ˩\textsubscript{b}}}}}{}
\textcolor{teal}{\mytextsc{classifier}} \hspace{4pt} Tone: L\textsubscript{b}.
\textcolor{Sepia}{\selectlanguage{english}Times (repeating an action: doing something n times).} \zh{量词:次数。} 
\lhead{\firstmark}
\rhead{\botmark}

\subsection{\hspace{-0.5cm} {\Large \textcolor{darkblue}{\textbf{\ipa{ʂɯ˩ʝi\#˥}}}}\hspace{0.5cm}[\kern2pt{\textcolor{darkblue}{\textbf{\ipa{ʂɯ˧ʝi˧}}}}\kern2pt]} \hypertarget{s`M\string_Bj££i\#\string_T1}{}
\markboth{\textcolor{darkblue}{\textbf{\ipa{ʂɯ˩ʝi\#˥}}}}{}
\textcolor{teal}{\mytextsc{adverb(ial)}} \hspace{4pt} Tone: LM+\#H.
\textcolor{Sepia}{\selectlanguage{english}Two years ago.} \zh{前年。}  ¶ \textcolor{darkblue}{\textbf{\ipa{ʂɯ˩ʝi˥ | ɖɯ˧-kʰv̩˧˥}}} \textcolor{Sepia}{\selectlanguage{english}two years ago} \zh{前年}  

\lhead{\firstmark}
\rhead{\botmark}

\subsection{\hspace{-0.5cm} {\Large \textcolor{darkblue}{\textbf{\ipa{ʂɯ˩kwæ˩ɻæ˥}}}}\hspace{0.5cm}[\kern2pt{\textcolor{darkblue}{\textbf{\ipa{ʂɯ˧kwæ˧ɻæ˩}}}}\kern2pt]} \hypertarget{s`M\string_Bkw\{\string_Br£`\{\string_T1}{}
\markboth{\textcolor{darkblue}{\textbf{\ipa{ʂɯ˩kwæ˩ɻæ˥}}}}{}
\textcolor{teal}{\mytextsc{adjective}} \hspace{4pt} Tone: L+H\#.
\textcolor{Sepia}{\selectlanguage{english}Yellow.} \zh{黄。}  ¶ \textcolor{darkblue}{\textbf{\ipa{ʂɯ˩kwæ˩ɻæ˥-hĩ˩ gv̩˩-ze˩}}} \textcolor{Sepia}{\selectlanguage{english}[the book] has turned yellow!} \zh{[书]变黄了!}  
 ¶ \textcolor{darkblue}{\textbf{\ipa{[F5] ʂɯ˩kwæ˩˥ | ʂɯ˩kwæ˩˥ | gv̩˩}}} \textcolor{Sepia}{\selectlanguage{english}very yellow} \zh{深黄}  

\lhead{\firstmark}
\rhead{\botmark}

\subsection{\hspace{-0.5cm} {\Large \textcolor{darkblue}{\textbf{\ipa{ʂɯ˩tsɯ˧}}}}\hspace{0.5cm}[\kern2pt{\textcolor{darkblue}{\textbf{\ipa{ʂɯ˩tsɯ˥}}}}\kern2pt]} \hypertarget{s`M\string_BtsM\string_M1}{}
\markboth{\textcolor{darkblue}{\textbf{\ipa{ʂɯ˩tsɯ˧}}}}{}
\textcolor{teal}{\mytextsc{noun}} \hspace{4pt} Tone: LM.
\textcolor{Sepia}{\selectlanguage{english}Persimmon.} \zh{柿子(汉语借词)。}  Borrowing: Chinese  \zh{柿子}
 ¶ \textcolor{darkblue}{\textbf{\ipa{ʂɯ˩tsɯ˧ | ɖɯ˧-so˩-ɭɯ˩ hwæ˩-bi˩!}}} \textcolor{Sepia}{\selectlanguage{english}Let's buy a few persimmons!} \zh{买一些柿子吧!}  

\lhead{\firstmark}
\rhead{\botmark}

\subsection{\hspace{-0.5cm} {\Large \textcolor{darkblue}{\textbf{\ipa{ʂɯ˩tsɯ˧}}}}\hspace{0.5cm}[\kern2pt{\textcolor{darkblue}{\textbf{\ipa{ʂɯ˩tsɯ˥}}}}\kern2pt]} \hypertarget{s`M\string_BtsM\string_M1}{}
\markboth{\textcolor{darkblue}{\textbf{\ipa{ʂɯ˩tsɯ˧}}}}{}
\textcolor{teal}{\mytextsc{noun}} \hspace{4pt} Tone: LM.
\textcolor{Sepia}{\selectlanguage{english}Pistol.} \zh{手枪。}  ¶ \textcolor{darkblue}{\textbf{\ipa{ʂɯ˩tsɯ˧ | ɖɯ˧-nɑ˧ | tʰi˧-pɤ˥\textasciitilde{}pɤ˩}}} \textcolor{Sepia}{\selectlanguage{english}to carry a pistol} \zh{带手枪}  

\lhead{\firstmark}
\rhead{\botmark}

\subsection{\hspace{-0.5cm} {\Large \textcolor{darkblue}{\textbf{\ipa{ʂɯ˧˥}}} \textsubscript{1}}\hspace{0.5cm}[\kern2pt{\textcolor{darkblue}{\textbf{\ipa{ʂɯ˧˥}}}}\kern2pt]} \hypertarget{s`M\string_M\string_T1}{}
\markboth{\textcolor{darkblue}{\textbf{\ipa{ʂɯ˧˥}}} \textsubscript{1}}{}
\textcolor{teal}{\mytextsc{verb}} \hspace{4pt} Tone: MH.
\textcolor{Sepia}{\selectlanguage{english}To peel (with a knife).} \zh{削(用刀)。}  ¶ \textcolor{darkblue}{\textbf{\ipa{ɣɯ˩ ʂɯ˧˥ / ɣɯ˩ʂɯ˧ ze˥}}} \textcolor{Sepia}{\selectlanguage{english}to peel, to peel off the skin} \zh{削皮}  
 ¶ \textcolor{darkblue}{\textbf{\ipa{ɣɯ˩kɯ˧ ʂɯ˥}}} \textcolor{Sepia}{\selectlanguage{english}to peel, to peel off the skin} \zh{削皮}  
 ¶ \textcolor{darkblue}{\textbf{\ipa{jɤ˩jo˧ ɣɯ˥ʂɯ˩}}} \textcolor{Sepia}{\selectlanguage{english}to peel potatoes} \zh{削洋芋皮}  
 ¶ \textcolor{darkblue}{\textbf{\ipa{[F5] tso˧tso˧ ɣɯ˥ʂɯ˩}}} \textcolor{Sepia}{\selectlanguage{english}to peel things} \zh{削东西}  

\lhead{\firstmark}
\rhead{\botmark}

\subsection{\hspace{-0.5cm} {\Large \textcolor{darkblue}{\textbf{\ipa{ʂɯ˧˥}}} \textsubscript{2}}\hspace{0.5cm}[\kern2pt{\textcolor{darkblue}{\textbf{\ipa{ʂɯ˧˥}}}}\kern2pt]} \hypertarget{s`M\string_M\string_T2}{}
\markboth{\textcolor{darkblue}{\textbf{\ipa{ʂɯ˧˥}}} \textsubscript{2}}{}
\textcolor{teal}{\mytextsc{adjective}} \hspace{4pt} Tone: MH.
\textcolor{Sepia}{\selectlanguage{english}New, fresh.} \zh{新。}  ¶ \textcolor{darkblue}{\textbf{\ipa{ʂɯ˧-hĩ˧ ɲi˥!}}} \textcolor{Sepia}{\selectlanguage{english}It's new!} \zh{是新的!}  
 ¶ \textcolor{darkblue}{\textbf{\ipa{ʂe˧ ʂɯ˩}}} \textcolor{Sepia}{\selectlanguage{english}fresh meat} \zh{新鲜的肉}  

\lhead{\firstmark}
\rhead{\botmark}

\subsection{\hspace{-0.5cm} {\Large \textcolor{darkblue}{\textbf{\ipa{ʂv̩˧˥}}}}\hspace{0.5cm}[\kern2pt{\textcolor{darkblue}{\textbf{\ipa{ʂv̩˧˥}}}}\kern2pt]} \hypertarget{s`v\string_=\string_M\string_T1}{}
\markboth{\textcolor{darkblue}{\textbf{\ipa{ʂv̩˧˥}}}}{}
\textcolor{teal}{\mytextsc{verb}} \hspace{4pt} Tone: MH.
\textcolor{Sepia}{\selectlanguage{english}To twist, to wring.} \zh{拧(拧毛巾)。}  ¶ \textcolor{darkblue}{\textbf{\ipa{le˧-ʂv̩˧-ze˥}}} \textcolor{Sepia}{\selectlanguage{english}\mytextsc{accomp} \string_ \mytextsc{pfv}} \zh{拧了}  
 ¶ \textcolor{darkblue}{\textbf{\ipa{dʑi˧hṽ˧ ʂv̩˩}}} \textcolor{Sepia}{\selectlanguage{english}to wring out clothes} \zh{拧衣服}  

\lhead{\firstmark}
\rhead{\botmark}

\subsection{\hspace{-0.5cm} {\Large \textcolor{darkblue}{\textbf{\ipa{ʂv̩˥}}}}\hspace{0.5cm}[\kern2pt{\textcolor{darkblue}{\textbf{\ipa{ʂv̩˥}}}}\kern2pt]} \hypertarget{s`v\string_=\string_T1}{}
\markboth{\textcolor{darkblue}{\textbf{\ipa{ʂv̩˥}}}}{}
\textcolor{teal}{\mytextsc{noun}} \hspace{4pt} Tone: H.
\textcolor{Sepia}{\selectlanguage{english}Dice.} \zh{骰子。}  ¶ \textcolor{darkblue}{\textbf{\ipa{ʂv̩˧ | ʐv̩˩-ɭɯ˩˥}}} \textcolor{Sepia}{\selectlanguage{english}four dice (dice came in pairs)} \zh{四个骰子}  
 \zh{量词}: \textcolor{darkblue}{\textbf{\ipa{ɭɯ˧}}}  \mytextsc{clf}: \textcolor{darkblue}{\textbf{\ipa{ɭɯ˧}}} 
\lhead{\firstmark}
\rhead{\botmark}

\subsection{\hspace{-0.5cm} {\Large \textcolor{darkblue}{\textbf{\ipa{ʂv̩˧\textsubscript{b}}}}}\hspace{0.5cm}[\kern2pt{\textcolor{darkblue}{\textbf{\ipa{ʂv̩˥}}}}\kern2pt]} \hypertarget{s`v\string_=\string_Mb1}{}
\markboth{\textcolor{darkblue}{\textbf{\ipa{ʂv̩˧\textsubscript{b}}}}}{}
\textcolor{teal}{\mytextsc{verb}} \hspace{4pt} Tone: M\textsubscript{b}.
\ding{202} \textcolor{Sepia}{\selectlanguage{english}To look after, to take care of (children).} \zh{带(孩子……)。}  ¶ \textcolor{darkblue}{\textbf{\ipa{zo˧mv̩˥ | ɖɯ˧-ɭɯ˧ ʂv̩˧}}} \textcolor{Sepia}{\selectlanguage{english}to take care of a child, to look after a child} \zh{带个孩子}  
 ¶ \textcolor{darkblue}{\textbf{\ipa{zo˧mv̩˥ ʂv̩˩}}} \textcolor{Sepia}{\selectlanguage{english}to take care of a child, to look after a child} \zh{带孩子}  
 ¶ \textcolor{darkblue}{\textbf{\ipa{le˧-ʂv̩˧ tʰi˧-kʰɯ˧˥ | tʰæ˧ɻ̍˩ so˩}}} \textcolor{Sepia}{\selectlanguage{english}to oblige to study (a mother obliges a child to study)} \zh{让他学习、要求他学习(家长管孩子,让他学习)}  
\ding{203} \textcolor{Sepia}{\selectlanguage{english}To lead (the way).} \zh{带(路)。}  ¶ \textcolor{darkblue}{\textbf{\ipa{ʈæ˧ʂɯ˧ | ɖʐv̩˧ ʂv̩˧-po˧-bi˧-ho˥!}}} \textcolor{Sepia}{\selectlanguage{english}Dashi is going to take care of his friends [taking them on a tourist trip to Yongning]} \zh{达石要管朋友(带他们去永宁旅游)}  
 ¶ \textcolor{darkblue}{\textbf{\ipa{ʈæ˧ʂɯ˧ | ɖʐv̩˧ ʂv̩˧-bi˧-ho˩!}}} \textcolor{Sepia}{\selectlanguage{english}Dashi is going to take care of his friends [taking them on a tourist trip to Yongning]} \zh{达石要管朋友(带他们去永宁旅游)}  

\lhead{\firstmark}
\rhead{\botmark}

\subsection{\hspace{-0.5cm} {\Large \textcolor{darkblue}{\textbf{\ipa{ʂv̩˧ɖv̩˧}}}}\hspace{0.5cm}[\kern2pt{\textcolor{darkblue}{\textbf{\ipa{ʂv̩˧ɖv̩˧}}}}\kern2pt]} \hypertarget{s`v\string_=\string_Md`v\string_=\string_M1}{}
\markboth{\textcolor{darkblue}{\textbf{\ipa{ʂv̩˧ɖv̩˧}}}}{}
\textcolor{teal}{\mytextsc{verb}} \hspace{4pt} Tone: M.
\ding{202} \textcolor{Sepia}{\selectlanguage{english}To think.} \zh{想。}  ¶ \textcolor{darkblue}{\textbf{\ipa{ə˧tso˧ ʂv̩˧ɖv̩˧?}}} \textcolor{Sepia}{\selectlanguage{english}What are you thinking about? / Where's your mind?} \zh{在想什么?}  
 ¶ \textcolor{darkblue}{\textbf{\ipa{njɤ˧ | ɖɯ˧ bæ˧ ʂv̩˧dv̩˧}}} \textcolor{Sepia}{\selectlanguage{english}I'm thinking about something.} \zh{我在想一件事情。}  
 ¶ \textcolor{darkblue}{\textbf{\ipa{ʂv̩˧ɖv̩˧ tʰv̩˧}}} \textcolor{Sepia}{\selectlanguage{english}to understand} \zh{明白,想起}  
 ¶ \textcolor{darkblue}{\textbf{\ipa{njɤ˧ | ʂv̩˧ɖv̩˧ tʰv̩˧}}} \textcolor{Sepia}{\selectlanguage{english}I understand.} \zh{我明白。}  
 ¶ \textcolor{darkblue}{\textbf{\ipa{ʈʂʰɯ˧ | le˧-ʂv̩˧ɖv̩˧-le˧-tʰv̩˧-ze˧}}} \textcolor{Sepia}{\selectlanguage{english}He has understood.} \zh{他明白了。}  
\ding{203} \textcolor{Sepia}{\selectlanguage{english}To remember, to recollect, to recall.} \zh{想起、回忆。}  ¶ \textcolor{darkblue}{\textbf{\ipa{ʂv̩˧ɖv̩˧ tʰv̩˧}}} \textcolor{Sepia}{\selectlanguage{english}to remember} \zh{想起}  
 ¶ \textcolor{darkblue}{\textbf{\ipa{njɤ˧ | ʂv̩˧ɖv̩˧ tʰv̩˧}}} \textcolor{Sepia}{\selectlanguage{english}I remember} \zh{我想起}  
 ¶ \textcolor{darkblue}{\textbf{\ipa{ʈʂʰɯ˧ | le˧-ʂv̩˧ɖv̩˧-le˧-tʰv̩˧-ze˧}}} \textcolor{Sepia}{\selectlanguage{english}He remembers, he recollects} \zh{他想起来了}  
\ding{204} \textcolor{Sepia}{\selectlanguage{english}To miss, to long for; to feel sorrowful, sad, grieved.} \zh{想念、感到悲哀。}  ¶ \textcolor{darkblue}{\textbf{\ipa{ʂv̩˧ɖv̩˧ tʰv̩˧ | ʐwæ˩˥}}} \textcolor{Sepia}{\selectlanguage{english}to be full of nostalgia} \zh{特别想念}  
 ¶ \textcolor{darkblue}{\textbf{\ipa{njɤ˧ | no˩ ʂv̩˩ɖv̩˩˥}}} \textcolor{Sepia}{\selectlanguage{english}I miss you!} \zh{我想你!}  
 ¶ \textcolor{darkblue}{\textbf{\ipa{ʂv̩˧ɖv̩˧ qʰwɤ˧ tʰv̩˧˥}}} \textcolor{Sepia}{\selectlanguage{english}to have a fit of nostalgia} \zh{想念}  
 ¶ \textcolor{darkblue}{\textbf{\ipa{[F5] ʂv̩˧ɖv̩˧ mɤ˧-zo˧}}} \textcolor{Sepia}{\selectlanguage{english}There's no need to worry / feel unhappy} \zh{不用发愁}  

\lhead{\firstmark}
\rhead{\botmark}

\subsection{\hspace{-0.5cm} {\Large \textcolor{darkblue}{\textbf{\ipa{ʂv̩˩gv̩˩}}}}\hspace{0.5cm}[\kern2pt{\textcolor{darkblue}{\textbf{\ipa{ʂv̩˧gv̩˧}}}}\kern2pt]} \hypertarget{s`v\string_=\string_Bgv\string_=\string_B1}{}
\markboth{\textcolor{darkblue}{\textbf{\ipa{ʂv̩˩gv̩˩}}}}{}
\textcolor{teal}{\mytextsc{noun}} \hspace{4pt} Tone: L.
\textcolor{Sepia}{\selectlanguage{english}Sickle.} \zh{镰刀。}  \zh{量词}: \textcolor{darkblue}{\textbf{\ipa{nɑ˧}}}  \mytextsc{clf}: \textcolor{darkblue}{\textbf{\ipa{nɑ˧}}} 
\lhead{\firstmark}
\rhead{\botmark}

\subsection{\hspace{-0.5cm} {\Large \textcolor{darkblue}{\textbf{\ipa{ʂv̩˧kʰɯ˩}}}}\hspace{0.5cm}[\kern2pt{\textcolor{darkblue}{\textbf{\ipa{ʂv̩˩kʰɯ˩˥}}}}\kern2pt]} \hypertarget{s`v\string_=\string_Mk\string_hM\string_B1}{}
\markboth{\textcolor{darkblue}{\textbf{\ipa{ʂv̩˧kʰɯ˩}}}}{}
\textcolor{teal}{\mytextsc{verb}} \hspace{4pt} Tone: .
\textcolor{Sepia}{\selectlanguage{english}To bet.} \zh{赌博。}  ¶ \textcolor{darkblue}{\textbf{\ipa{ʂv̩˧kʰɯ˩ | -jɤ˩po˧}}} \textcolor{Sepia}{\selectlanguage{english}same meaning} \zh{同上}  

\lhead{\firstmark}
\rhead{\botmark}

\subsection{\hspace{-0.5cm} {\Large \textcolor{darkblue}{\textbf{\ipa{ʂv̩˩njɤ˥}}}}\hspace{0.5cm}[\kern2pt{\textcolor{darkblue}{\textbf{\ipa{ʂv̩˩njɤ˥}}}}\kern2pt]} \hypertarget{s`v\string_=\string_Bnj7\string_T1}{}
\markboth{\textcolor{darkblue}{\textbf{\ipa{ʂv̩˩njɤ˥}}}}{}
\textcolor{teal}{\mytextsc{noun}} \hspace{4pt} Tone: LH.
\textcolor{Sepia}{\selectlanguage{english}Tree bur; burl.} \zh{树瘤。}  ¶ \textcolor{darkblue}{\textbf{\ipa{ʂv̩˩njɤ˥ ɲi˩}}} \textcolor{Sepia}{\selectlanguage{english}\mytextsc{cop}} \zh{是树瘤。}  
 \zh{量词}: \textcolor{darkblue}{\textbf{\ipa{ɭɯ˧}}}  \mytextsc{clf}: \textcolor{darkblue}{\textbf{\ipa{ɭɯ˧}}} 
\lhead{\firstmark}
\rhead{\botmark}

\subsection{\hspace{-0.5cm} {\Large \textcolor{darkblue}{\textbf{\ipa{ʂv̩˧ʂv̩˧˥}}}}\hspace{0.5cm}[\kern2pt{\textcolor{darkblue}{\textbf{\ipa{ʂv̩˧ʂv̩˧˥}}}}\kern2pt]} \hypertarget{s`v\string_=\string_Ms`v\string_=\string_M\string_T1}{}
\markboth{\textcolor{darkblue}{\textbf{\ipa{ʂv̩˧ʂv̩˧˥}}}}{}
\textcolor{teal}{\mytextsc{noun}} \hspace{4pt} Tone: MH\#.
\textcolor{Sepia}{\selectlanguage{english}Paper.} \zh{纸。}  ¶ \textcolor{darkblue}{\textbf{\ipa{ʂv̩˧ʂv̩˧˥ | ɖɯ˧-pʰæ˧˥}}} \textcolor{Sepia}{\selectlanguage{english}a sheet of paper} \zh{一张纸}  
 ¶ \textcolor{darkblue}{\textbf{\ipa{[F5] ʂv̩˧ʂv̩˧ ɖɯ˧ pʰæ˧˥}}} \textcolor{Sepia}{\selectlanguage{english}a sheet of paper} \zh{一张纸}  
 \zh{量词}: \textcolor{darkblue}{\textbf{\ipa{pʰæ˧˥}}}  \mytextsc{clf}: \textcolor{darkblue}{\textbf{\ipa{pʰæ˧˥}}} 
\lhead{\firstmark}
\rhead{\botmark}

\subsection{\hspace{-0.5cm} {\Large \textcolor{darkblue}{\textbf{\ipa{ʂwæ˧}}}}\hspace{0.5cm}[\kern2pt{\textcolor{darkblue}{\textbf{\ipa{ʂwæ˥}}}}\kern2pt]} \hypertarget{s`w\{\string_M1}{}
\markboth{\textcolor{darkblue}{\textbf{\ipa{ʂwæ˧}}}}{}
\textcolor{teal}{\mytextsc{noun}} \hspace{4pt} Tone: M.
\textcolor{Sepia}{\selectlanguage{english}Otter.} \zh{水獭。} Local Chinese dialect:\zh{水潭猫。} ¶ \textcolor{darkblue}{\textbf{\ipa{ʂwæ˧-ɣɯ˩}}} \textcolor{Sepia}{\selectlanguage{english}otter skin} \zh{水獭皮}  

\lhead{\firstmark}
\rhead{\botmark}

\subsection{\hspace{-0.5cm} {\Large \textcolor{darkblue}{\textbf{\ipa{ʂwæ˧}}}}\hspace{0.5cm}[\kern2pt{\textcolor{darkblue}{\textbf{\ipa{ʂwæ˥}}}}\kern2pt]} \hypertarget{s`w\{\string_M1}{}
\markboth{\textcolor{darkblue}{\textbf{\ipa{ʂwæ˧}}}}{}
\textcolor{teal}{\mytextsc{adjective}} \hspace{4pt} Tone: M.
\textcolor{Sepia}{\selectlanguage{english}Tall.} \zh{高。}  ¶ \textcolor{darkblue}{\textbf{\ipa{qʰɑ˧-ʂwæ˧-gv̩˧}}} \textcolor{Sepia}{\selectlanguage{english}very tall} \zh{非常高}  
 ¶ \textcolor{darkblue}{\textbf{\ipa{ʈʂʰɯ˧ | ə˧pɤ˧ | -ʂwæ˩-gv̩˩˥!}}} \textcolor{Sepia}{\selectlanguage{english}(S)he is extremely tall!} \zh{他非常高!}  
 ¶ \textcolor{darkblue}{\textbf{\ipa{gv̩˧mi˧ ʂwæ˧}}} \textcolor{Sepia}{\selectlanguage{english}tall; literally 'with a tall body'} \zh{高、身材高}  

\lhead{\firstmark}
\rhead{\botmark}

\subsection{\hspace{-0.5cm} {\Large \textcolor{darkblue}{\textbf{\ipa{ʂwæ˧\textsubscript{a}}}}}\hspace{0.5cm}[\kern2pt{\textcolor{darkblue}{\textbf{\ipa{ʂwæ˥}}}}\kern2pt]} \hypertarget{s`w\{\string_Ma1}{}
\markboth{\textcolor{darkblue}{\textbf{\ipa{ʂwæ˧\textsubscript{a}}}}}{}
\textcolor{teal}{\mytextsc{verb}} \hspace{4pt} Tone: M\textsubscript{a}.
\textcolor{Sepia}{\selectlanguage{english}To stir.} \zh{搅拌。}  ¶ \textcolor{darkblue}{\textbf{\ipa{le˧-ʂwæ˧}}} \textcolor{Sepia}{\selectlanguage{english}\mytextsc{accomp}} \zh{\mytextsc{accomp}}  
 ¶ \textcolor{darkblue}{\textbf{\ipa{mɤ˧-ʂwæ˧}}} \textcolor{Sepia}{\selectlanguage{english}\mytextsc{neg}} \zh{不搅拌}  
 ¶ \textcolor{darkblue}{\textbf{\ipa{tso˧\textasciitilde{}tso˧ ʂwæ˩}}} \textcolor{Sepia}{\selectlanguage{english}to stir things} \zh{搅拌东西}  

\lhead{\firstmark}
\rhead{\botmark}

\subsection{\hspace{-0.5cm} {\Large \textcolor{darkblue}{\textbf{\ipa{ʂwæ˧bæ˩}}}}\hspace{0.5cm}[\kern2pt{\textcolor{darkblue}{\textbf{\ipa{ʂwæ˧bæ˩}}}}\kern2pt]} \hypertarget{s`w\{\string_Mb\{\string_B1}{}
\markboth{\textcolor{darkblue}{\textbf{\ipa{ʂwæ˧bæ˩}}}}{}
\textcolor{teal}{\mytextsc{noun}} \hspace{4pt} Tone: L\#.
\textcolor{Sepia}{\selectlanguage{english}Camellia flower.} \zh{映山红。} Local Chinese dialect:\zh{山茶花。} ¶ \textcolor{darkblue}{\textbf{\ipa{ʂwæ˧bæ˩ bæ˩ |}}} \textcolor{Sepia}{\selectlanguage{english}The camellia flowers are in bloom.} \zh{山茶花开了。}  
 ¶ \textcolor{darkblue}{\textbf{\ipa{so˧-ɬi˧mi˧, | ʂwæ˧bæ˩ bæ˩! |}}} \textcolor{Sepia}{\selectlanguage{english}Camellia flowers bloom in the third month!} \zh{山茶花是在三月份开花的!}  
 ¶ \textcolor{darkblue}{\textbf{\ipa{ʂwæ˧bæ˩-si˩}}} \textcolor{Sepia}{\selectlanguage{english}camellia tree} \zh{山茶树}  

\lhead{\firstmark}
\rhead{\botmark}

\subsection{\hspace{-0.5cm} {\Large \textcolor{darkblue}{\textbf{\ipa{ʂwæ˧gv̩\#˥}}}}\hspace{0.5cm}[\kern2pt{\textcolor{darkblue}{\textbf{\ipa{ʂwæ˧gv̩˧}}}}\kern2pt]} \hypertarget{s`w\{\string_Mgv\string_=\#\string_T1}{}
\markboth{\textcolor{darkblue}{\textbf{\ipa{ʂwæ˧gv̩\#˥}}}}{}
\textcolor{teal}{\mytextsc{noun}} \hspace{4pt} Tone: \#H.
\textcolor{Sepia}{\selectlanguage{english}A mountain to the North-West of Yongning, called “Jiaze Mountain” in Chinese.} \zh{加泽大山(位于永宁西北的一座山)。}  ¶ \textcolor{darkblue}{\textbf{\ipa{kɤ˧mv̩˧˥, | æ˧ʂæ˧, | ŋwɤ˧hɑ̃˩, | ʂwæ˧gv̩\#˥, | nɑ˩tsʰi˩˥ | -tɕʰɤ˧pɤ˧mi\#˥, | qv̩˧ɻ̍˧-ʈʂʰɑ˧nɑ˥ |}}} \textcolor{Sepia}{\selectlanguage{english}The six mountains of Yongning that carry a name and have a definite symbolic value. The other mountains do not have comparable symbolic value, and fewer people use specific names for them.} \zh{永宁地区有固定名字的六座山。其它的山,因为没有重要的象征意义,因此没有取名。}  

\lhead{\firstmark}
\rhead{\botmark}

\subsection{\hspace{-0.5cm} {\Large \textcolor{darkblue}{\textbf{\ipa{ʂwæ˧si\#˥}}}}\hspace{0.5cm}[\kern2pt{\textcolor{darkblue}{\textbf{\ipa{ʂwæ˧si˧}}}}\kern2pt]} \hypertarget{s`w\{\string_Msi\#\string_T1}{}
\markboth{\textcolor{darkblue}{\textbf{\ipa{ʂwæ˧si\#˥}}}}{}
\textcolor{teal}{\mytextsc{noun}} \hspace{4pt} Tone: \#H.
\textcolor{Sepia}{\selectlanguage{english}Camellia tree.} \zh{山茶树。} 
\lhead{\firstmark}
\rhead{\botmark}

\subsection{\hspace{-0.5cm} {\Large \textcolor{darkblue}{\textbf{\ipa{ʂwæ˧tsɯ˥}}}}\hspace{0.5cm}[\kern2pt{\textcolor{darkblue}{\textbf{\ipa{ʂwæ˧tsɯ˥}}}}\kern2pt]} \hypertarget{s`w\{\string_MtsM\string_T1}{}
\markboth{\textcolor{darkblue}{\textbf{\ipa{ʂwæ˧tsɯ˥}}}}{}
\textcolor{teal}{\mytextsc{noun}} \hspace{4pt} Tone: H\#.
\textcolor{Sepia}{\selectlanguage{english}Brush.} \zh{刷子。}  Borrowing: Chinese  \zh{刷子}
 \zh{量词}: \textcolor{darkblue}{\textbf{\ipa{nɑ˧}}}  \mytextsc{clf}: \textcolor{darkblue}{\textbf{\ipa{nɑ˧}}} 
\lhead{\firstmark}
\rhead{\botmark}

\subsection{\hspace{-0.5cm} {\Large \textcolor{darkblue}{\textbf{\ipa{ʂwæ˩\textsubscript{a}}}}}\hspace{0.5cm}[\kern2pt{\textcolor{darkblue}{\textbf{\ipa{ʂwæ˩˥}}}}\kern2pt]} \hypertarget{s`w\{\string_Ba1}{}
\markboth{\textcolor{darkblue}{\textbf{\ipa{ʂwæ˩\textsubscript{a}}}}}{}
\textcolor{teal}{\mytextsc{verb}} \hspace{4pt} Tone: L\textsubscript{a}.
\textcolor{Sepia}{\selectlanguage{english}To cure (meat etc) with smoke.} \zh{熏。}  ¶ \textcolor{darkblue}{\textbf{\ipa{ʂe˧ ʂwæ˥}}} \textcolor{Sepia}{\selectlanguage{english}to cure meat with smoke} \zh{熏肉}  

\lhead{\firstmark}
\rhead{\botmark}

\subsection{\hspace{-0.5cm} {\Large \textcolor{darkblue}{\textbf{\ipa{ʂwæ˩gv̩˩}}}}\hspace{0.5cm}[\kern2pt{\textcolor{darkblue}{\textbf{\ipa{ʂwæ˩gv̩˩˥}}}}\kern2pt]} \hypertarget{s`w\{\string_Bgv\string_=\string_B1}{}
\markboth{\textcolor{darkblue}{\textbf{\ipa{ʂwæ˩gv̩˩}}}}{}
\textcolor{teal}{\mytextsc{noun}} \hspace{4pt} Tone: L.
\textcolor{Sepia}{\selectlanguage{english}Kitchen cabinet, kitchen dresser.} \zh{柜子。}  \zh{量词}: \textcolor{darkblue}{\textbf{\ipa{ɭɯ˧}}}  \mytextsc{clf}: \textcolor{darkblue}{\textbf{\ipa{ɭɯ˧}}} 
\lhead{\firstmark}
\rhead{\botmark}

\subsection{\hspace{-0.5cm} {\Large \textcolor{darkblue}{\textbf{\ipa{ʂwæ˧˥}}}}\hspace{0.5cm}[\kern2pt{\textcolor{darkblue}{\textbf{\ipa{ʂwæ˧˥}}}}\kern2pt]} \hypertarget{s`w\{\string_M\string_T1}{}
\markboth{\textcolor{darkblue}{\textbf{\ipa{ʂwæ˧˥}}}}{}
\textcolor{teal}{\mytextsc{verb}} \hspace{4pt} Tone: MH.
\textcolor{Sepia}{\selectlanguage{english}To defecate.} \zh{拉(屎)。}  ¶ \textcolor{darkblue}{\textbf{\ipa{qʰæ˧ ʂwæ˩}}} \textcolor{Sepia}{\selectlanguage{english}to defecate} \zh{拉屎}  

\lhead{\firstmark}
\rhead{\botmark}

\subsection{\hspace{-0.5cm} {\Large \textcolor{darkblue}{\textbf{\ipa{ʂwæ˩˥}}}}\hspace{0.5cm}[\kern2pt{\textcolor{darkblue}{\textbf{\ipa{ʂwæ˩˥}}}}\kern2pt]} \hypertarget{s`w\{\string_B\string_T1}{}
\markboth{\textcolor{darkblue}{\textbf{\ipa{ʂwæ˩˥}}}}{}
\textcolor{teal}{\mytextsc{noun}} \hspace{4pt} Tone: LH.
\textcolor{Sepia}{\selectlanguage{english}Wedge.} \zh{楔子。}  ¶ \textcolor{darkblue}{\textbf{\ipa{ʂwæ˩ lɑ˧˥ / ʂwæ˩ lɑ˧-ze˥}}} \textcolor{Sepia}{\selectlanguage{english}to strike a wedge} \zh{打一个楔子}  
 ¶ \textcolor{darkblue}{\textbf{\ipa{ʂwæ˩ hwæ˥-ze˩}}} \textcolor{Sepia}{\selectlanguage{english}...bought a wedge} \zh{买了楔子}  
 ¶ \textcolor{darkblue}{\textbf{\ipa{ʂwæ˩ tʰv̩˩-ɭɯ˩˥ / ʂwæ˩ tʰv̩˩-ɭɯ˥}}} \textcolor{Sepia}{\selectlanguage{english}\string_ \mytextsc{dem}+\mytextsc{clf}} \zh{这个楔子}  
 ¶ \textcolor{darkblue}{\textbf{\ipa{ʂwæ˩ tʰv̩˩-kʰwɤ˩˥}}} \textcolor{Sepia}{\selectlanguage{english}\string_ \mytextsc{dem}+\mytextsc{clf}} \zh{这个楔子}  
 ¶ \textcolor{darkblue}{\textbf{\ipa{[F5] ʂwæ˩ kʰɯ˥}}} \textcolor{Sepia}{\selectlanguage{english}to place a wedge, to put a wedge} \zh{放一个楔子}  
 \zh{量词}: \textcolor{darkblue}{\textbf{\ipa{kʰwɤ˥ / ɭɯ˧}}}  \mytextsc{clf}: \textcolor{darkblue}{\textbf{\ipa{kʰwɤ˥ / ɭɯ˧}}} 
\lhead{\firstmark}
\rhead{\botmark}

\newpage
\section*{\centering- \textcolor{darkblue}{\textbf{\ipa{t}}} -}
\subsection{\hspace{-0.5cm} {\Large \textcolor{darkblue}{\textbf{\ipa{tɑ˥}}}}\hspace{0.5cm}[\kern2pt{\textcolor{darkblue}{\textbf{\ipa{tɑ˥}}}}\kern2pt]} \hypertarget{tA\string_T1}{}
\markboth{\textcolor{darkblue}{\textbf{\ipa{tɑ˥}}}}{}
\textcolor{teal}{\mytextsc{adjective}} \hspace{4pt} Tone: H.
\textcolor{Sepia}{\selectlanguage{english}Reliable, trustworthy.} \zh{可靠。}  ¶ \textcolor{darkblue}{\textbf{\ipa{le˧-tɑ˥ (| ʐwæ˩˥)}}} \textcolor{Sepia}{\selectlanguage{english}very reliable} \zh{很靠谱}  
 ¶ \textcolor{darkblue}{\textbf{\ipa{le˧ mɤ˧-tɑ˥ (| ʐwæ˩˥)}}} \textcolor{Sepia}{\selectlanguage{english}not reliable at all} \zh{不靠谱}  
 ¶ \textcolor{darkblue}{\textbf{\ipa{no˧ | le˧-mɤ˧-tɑ˥-hĩ˩ ɖɯ˧-v̩˧ ɲi˩!}}} \textcolor{Sepia}{\selectlanguage{english}You are an irresponsible person! / You are not a reliable person!} \zh{你是不靠谱的人!}  

\lhead{\firstmark}
\rhead{\botmark}

\subsection{\hspace{-0.5cm} {\Large \textcolor{darkblue}{\textbf{\ipa{tɑ˥mo˩}}}}\hspace{0.5cm}[\kern2pt{\textcolor{darkblue}{\textbf{\ipa{xxxx ton non trouvé, à faire manuellement...}}}}\kern2pt]} \hypertarget{tA\string_Tmo\string_B1}{}
\markboth{\textcolor{darkblue}{\textbf{\ipa{tɑ˥mo˩}}}}{}
\textcolor{teal}{\mytextsc{verb}} \hspace{4pt} Tone: HL.
\textcolor{Sepia}{\selectlanguage{english}To wilt, to wither (flower...).} \zh{萎、萎蔫。}  ¶ \textcolor{darkblue}{\textbf{\ipa{lə˧-tɑ˥mo˩-ze˩!}}} \textcolor{Sepia}{\selectlanguage{english}It has wilted!} \zh{萎蔫了!}  

\lhead{\firstmark}
\rhead{\botmark}

\subsection{\hspace{-0.5cm} {\Large \textcolor{darkblue}{\textbf{\ipa{tɑ˧\textsubscript{a}}}}}\hspace{0.5cm}[\kern2pt{\textcolor{darkblue}{\textbf{\ipa{tɑ˩˥}}}}\kern2pt]} \hypertarget{tA\string_Ma1}{}
\markboth{\textcolor{darkblue}{\textbf{\ipa{tɑ˧\textsubscript{a}}}}}{}
\textcolor{teal}{\mytextsc{verb}} \hspace{4pt} Tone: M\textsubscript{a}.
\textcolor{Sepia}{\selectlanguage{english}To dry beside or over a fire.} \zh{烘干。}  ¶ \textcolor{darkblue}{\textbf{\ipa{kwɤ˧-kʰɯ˧ tʰi˧-tɑ˧}}} \textcolor{Sepia}{\selectlanguage{english}to warm up (food...) beside a fire} \zh{放在火炉旁边热一下(饭)}  

\lhead{\firstmark}
\rhead{\botmark}

\subsection{\hspace{-0.5cm} {\Large \textcolor{darkblue}{\textbf{\ipa{tɑ˧dzi˩}}}}\hspace{0.5cm}[\kern2pt{\textcolor{darkblue}{\textbf{\ipa{tɑ˩dzi˧˥}}}}\kern2pt]} \hypertarget{tA\string_Mdzi\string_B1}{}
\markboth{\textcolor{darkblue}{\textbf{\ipa{tɑ˧dzi˩}}}}{}
\textcolor{teal}{\mytextsc{noun}} \hspace{4pt} Tone: L\#.
\textcolor{Sepia}{\selectlanguage{english}A Na village down below Nhissei, upward from Lataddi.} \zh{村落名。}  ¶ \textcolor{darkblue}{\textbf{\ipa{ɬi˧ki˧, | ɲi˧se˩, | tɑ˧dzi˩, | mv̩˧qʰwæ˩, | lɑ˧tʰɑ˧-di˧˥}}} \textcolor{Sepia}{\selectlanguage{english}Villages that one passes when moving away from the Yongning plain, towards Lugu lake. These villages do not count as part of Yongning proper. The last, \textcolor{darkblue}{\textbf{\ipa{/lɑ˧tʰɑ˧-di˧˥/}}}, is not a village name like the preceding four: it refers to the entire Na area beyond the fourth village.} \zh{永宁到泸沽湖所经过的村落,依次是:里格、尼赛、大祖、木垮,然后到拉塔地(拉塔地指的是泸沽湖周边的摩梭地区,包括左所、洛水村等)}  

\lhead{\firstmark}
\rhead{\botmark}

\subsection{\hspace{-0.5cm} {\Large \textcolor{darkblue}{\textbf{\ipa{tɑ˧gɤ˩}}}}\hspace{0.5cm}[\kern2pt{\textcolor{darkblue}{\textbf{\ipa{tɑ˩gɤ˥}}}}\kern2pt]} \hypertarget{tA\string_Mg7\string_B1}{}
\markboth{\textcolor{darkblue}{\textbf{\ipa{tɑ˧gɤ˩}}}}{}
\textcolor{teal}{\mytextsc{adjective}} \hspace{4pt} Tone: L\#.
\textcolor{Sepia}{\selectlanguage{english}Gaunt, emaciated.} \zh{瘦弱、枯瘦。} 
\lhead{\firstmark}
\rhead{\botmark}

\subsection{\hspace{-0.5cm} {\Large \textcolor{darkblue}{\textbf{\ipa{tɑ˧ho˧}}}}\hspace{0.5cm}[\kern2pt{\textcolor{darkblue}{\textbf{\ipa{tɑ˧ho˩}}}}\kern2pt]} \hypertarget{tA\string_Mho\string_M1}{}
\markboth{\textcolor{darkblue}{\textbf{\ipa{tɑ˧ho˧}}}}{}
\textcolor{teal}{\mytextsc{adverb(ial)}} \hspace{4pt} Tone: M.
\textcolor{Sepia}{\selectlanguage{english}Together.} \zh{一起。}  ¶ \textcolor{darkblue}{\textbf{\ipa{ɖɯ˧-ʁwɤ˧ tɑ˧ho˧ kʰi˧˥}}} \textcolor{Sepia}{\selectlanguage{english}The whole village went together.} \zh{全村一起去了。}  
 ¶ \textcolor{darkblue}{\textbf{\ipa{tɑ˧ho˧ ʝi˧}}} \textcolor{Sepia}{\selectlanguage{english}to work together} \zh{一起工作}  
 ¶ \textcolor{darkblue}{\textbf{\ipa{tɑ˧ho˧ tsʰo˧}}} \textcolor{Sepia}{\selectlanguage{english}to dance together} \zh{一起跳舞}  

\lhead{\firstmark}
\rhead{\botmark}

\subsection{\hspace{-0.5cm} {\Large \textcolor{darkblue}{\textbf{\ipa{tɑ˧ko˧}}}}\hspace{0.5cm}[\kern2pt{\textcolor{darkblue}{\textbf{\ipa{tɑ˧ko˧}}}}\kern2pt]} \hypertarget{tA\string_Mko\string_M1}{}
\markboth{\textcolor{darkblue}{\textbf{\ipa{tɑ˧ko˧}}}}{}
\textcolor{teal}{\mytextsc{verb}} \hspace{4pt} Tone: M.
\textcolor{Sepia}{\selectlanguage{english}To do manual work, to get a job, to do odd jobs to make some money.} \zh{打工(汉语借词)。}  Borrowing: Chinese  \zh{打工}
 ¶ \textcolor{darkblue}{\textbf{\ipa{tɑ˧ko˧ hɯ˧-ze˩!}}} \textcolor{Sepia}{\selectlanguage{english}(S)he has gone (to the city, to another place...) to do odd jobs to make some money!} \zh{(他)打工去了!}  

\lhead{\firstmark}
\rhead{\botmark}

\subsection{\hspace{-0.5cm} {\Large \textcolor{darkblue}{\textbf{\ipa{tɑ˧ko˩}}}}\hspace{0.5cm}[\kern2pt{\textcolor{darkblue}{\textbf{\ipa{tɑ˧ko˩}}}}\kern2pt]} \hypertarget{tA\string_Mko\string_B1}{}
\markboth{\textcolor{darkblue}{\textbf{\ipa{tɑ˧ko˩}}}}{}
\textcolor{teal}{\mytextsc{verb}} \hspace{4pt} Tone: L\#.
\textcolor{Sepia}{\selectlanguage{english}To delay, to hold up.} \zh{耽误。}  ¶ \textcolor{darkblue}{\textbf{\ipa{hĩ˧ tɑ˧ko˥}}} \textcolor{Sepia}{\selectlanguage{english}to delay people} \zh{耽误人家}  
 ¶ \textcolor{darkblue}{\textbf{\ipa{ʈʂʰɯ˧ hĩ˧ tɑ˧ko˥ | ʐwæ˩˥!}}} \textcolor{Sepia}{\selectlanguage{english}(S)he delays people a lot!} \zh{他耽误大家很多!}  

\lhead{\firstmark}
\rhead{\botmark}

\subsection{\hspace{-0.5cm} {\Large \textcolor{darkblue}{\textbf{\ipa{tɑ˧nɑ˩}}}}\hspace{0.5cm}[\kern2pt{\textcolor{darkblue}{\textbf{\ipa{tɑ˧nɑ˩}}}}\kern2pt]} \hypertarget{tA\string_MnA\string_B1}{}
\markboth{\textcolor{darkblue}{\textbf{\ipa{tɑ˧nɑ˩}}}}{}
\textcolor{teal}{\mytextsc{noun}} \hspace{4pt} Tone: L\#.
\textcolor{Sepia}{\selectlanguage{english}Crossbow.} \zh{弩弓。}  \zh{量词}: \textcolor{darkblue}{\textbf{\ipa{pɤ˩}}} \textcolor{darkblue}{\textbf{\ipa{nɑ˧}}}  \mytextsc{clf}: \textcolor{darkblue}{\textbf{\ipa{pɤ˩}}} \textcolor{darkblue}{\textbf{\ipa{nɑ˧}}} 
\lhead{\firstmark}
\rhead{\botmark}

\subsection{\hspace{-0.5cm} {\Large \textcolor{darkblue}{\textbf{\ipa{tɑ˧pi˧}}}}\hspace{0.5cm}[\kern2pt{\textcolor{darkblue}{\textbf{\ipa{tɑ˧pi˧}}}}\kern2pt]} \hypertarget{tA\string_Mpi\string_M1}{}
\markboth{\textcolor{darkblue}{\textbf{\ipa{tɑ˧pi˧}}}}{}
\textcolor{teal}{\mytextsc{adjective}} \hspace{4pt} Tone: M.
\textcolor{Sepia}{\selectlanguage{english}Identical to, like, to the likeness of.} \zh{如、像、像……那样。}  ¶ \textcolor{darkblue}{\textbf{\ipa{no˧-bi˧ tɑ˩pi˩, ...}}} \textcolor{Sepia}{\selectlanguage{english}like you; following your example} \zh{像你}  
 ¶ \textcolor{darkblue}{\textbf{\ipa{njɤ˧-bi˧ tɑ˩pi˩…}}} \textcolor{Sepia}{\selectlanguage{english}like me; following my example} \zh{像我}  
 ¶ \textcolor{darkblue}{\textbf{\ipa{no˧=ɻ̍˩-bv̩˩, | njɤ˧=ɻ̍˩-bv̩˩, | tɑ˧pi˧!}}} \textcolor{Sepia}{\selectlanguage{english}Yours and ours are built on the same pattern / are identical! (Context: discussing the farms of the village: they are all built on the same model, in the same way, and thus identical.)} \zh{你家的房子,我家的房子,都是一样的!(如:一个村子里的房子,都是按同一个模式建设的。)}  
 ¶ \textcolor{darkblue}{\textbf{\ipa{no˧-ɳɯ˧ gv̩˩, | njɤ˧-ɳɯ˧-gv̩˩, | tɑ˧pi˧!}}} \textcolor{Sepia}{\selectlanguage{english}Whether it's you or me who's building [the house], it's the same / the result is the same!} \zh{无论是谁来盖房,盖出来的都一样!}  
 ¶ \textcolor{darkblue}{\textbf{\ipa{ʈʂʰɯ˧-bi˩ | tɑ˧pi˧, | njɤ˧-ɳɯ˧ dɑ˧-bi˥-ze˩!}}} \textcolor{Sepia}{\selectlanguage{english}I am going to build [a house] like that one! / I am going to build [a house] that will be identical to his!} \zh{我要盖跟这一样的房子!}  

\lhead{\firstmark}
\rhead{\botmark}

\subsection{\hspace{-0.5cm} {\Large \textcolor{darkblue}{\textbf{\ipa{tɑ˧pi˧}}}}\hspace{0.5cm}[\kern2pt{\textcolor{darkblue}{\textbf{\ipa{tɑ˧pi˧}}}}\kern2pt]} \hypertarget{tA\string_Mpi\string_M1}{}
\markboth{\textcolor{darkblue}{\textbf{\ipa{tɑ˧pi˧}}}}{}
\textcolor{teal}{\mytextsc{verb}} \hspace{4pt} Tone: M.
\textcolor{Sepia}{\selectlanguage{english}To take as an example, to draw an analog.} \zh{打比方(汉语借词:当地汉语方言‘打比’)。} Local Chinese dialect:\zh{打比。} Borrowing: Chinese  \zh{打比}
 ¶ \textcolor{darkblue}{\textbf{\ipa{tɑ˧pi˧-ze˩}}} \textcolor{Sepia}{\selectlanguage{english}\mytextsc{pfv}} \zh{打比方}  

\lhead{\firstmark}
\rhead{\botmark}

\subsection{\hspace{-0.5cm} {\Large \textcolor{darkblue}{\textbf{\ipa{tɑ˧pv̩˩}}}}\hspace{0.5cm}[\kern2pt{\textcolor{darkblue}{\textbf{\ipa{tɑ˧pv̩˩}}}}\kern2pt]} \hypertarget{tA\string_Mpv\string_=\string_B1}{}
\markboth{\textcolor{darkblue}{\textbf{\ipa{tɑ˧pv̩˩}}}}{}
\textcolor{teal}{\mytextsc{adjective}} \hspace{4pt} Tone: L\#.
\textcolor{Sepia}{\selectlanguage{english}Dry (fruit, vegetables), dried in the sun.} \zh{晒干的(水果、蔬菜……)。}  ¶ \textcolor{darkblue}{\textbf{\ipa{v̩˩tsʰɤ˧-tɑ˧pv̩˥}}} \textcolor{Sepia}{\selectlanguage{english}dry vegetables, vegetables dried in the sun} \zh{晒干的蔬菜}  

\lhead{\firstmark}
\rhead{\botmark}

\subsection{\hspace{-0.5cm} {\Large \textcolor{darkblue}{\textbf{\ipa{tɑ˧pʰi˩}}}}\hspace{0.5cm}[\kern2pt{\textcolor{darkblue}{\textbf{\ipa{tɑ˧pʰi˩}}}}\kern2pt]} \hypertarget{tA\string_Mp\string_hi\string_B1}{}
\markboth{\textcolor{darkblue}{\textbf{\ipa{tɑ˧pʰi˩}}}}{}
\textcolor{teal}{\mytextsc{noun}} \hspace{4pt} Tone: L\#.
\textcolor{Sepia}{\selectlanguage{english}Chinese mugwort, \textit{Artemisia argyi}.} \zh{艾、艾蒿。}  \zh{量词}: \textcolor{darkblue}{\textbf{\ipa{dzi˩}}}  \mytextsc{clf}: \textcolor{darkblue}{\textbf{\ipa{dzi˩}}} 
\lhead{\firstmark}
\rhead{\botmark}

\subsection{\hspace{-0.5cm} {\Large \textcolor{darkblue}{\textbf{\ipa{tɑ˧\textasciitilde{}tɑ˧}}} \textsubscript{1}}\hspace{0.5cm}[\kern2pt{\textcolor{darkblue}{\textbf{\ipa{tɑ˧tɑ˧}}}}\kern2pt]} \hypertarget{tA\string_M~tA\string_M1}{}
\markboth{\textcolor{darkblue}{\textbf{\ipa{tɑ˧\textasciitilde{}tɑ˧}}} \textsubscript{1}}{}
\textcolor{teal}{\mytextsc{adjective}} \hspace{4pt} Tone: M.
\textcolor{Sepia}{\selectlanguage{english}Serious, reliable, careful; clear (to see clearly).} \zh{严肃认真、细心、细致,(看得)清楚、清晰。}  ¶ \textcolor{darkblue}{\textbf{\ipa{mɤ˧-tɑ˧\textasciitilde{}tɑ˧}}} \textcolor{Sepia}{\selectlanguage{english}sloppy, not careful} \zh{邋遢、草率、潦草}  
 ¶ \textcolor{darkblue}{\textbf{\ipa{lo˧ ʝi˧ mɤ˧-tɑ˧\textasciitilde{}tɑ˧}}} \textcolor{Sepia}{\selectlanguage{english}to do sloppy work} \zh{工作草率}  
 ¶ \textcolor{darkblue}{\textbf{\ipa{hĩ˧ ʈʂʰɯ˧-v̩˧, | tɑ˧\textasciitilde{}tɑ˧!}}} \textcolor{Sepia}{\selectlanguage{english}(S)he works carefully!} \zh{他很认真!}  
\textit{See:} \hyperlink{}{\textcolor{darkblue}{\textbf{\ipa{tɑ˧\textasciitilde{}tɑ˧}}} \textsubscript{2}} 
\lhead{\firstmark}
\rhead{\botmark}

\subsection{\hspace{-0.5cm} {\Large \textcolor{darkblue}{\textbf{\ipa{tɑ˧\textasciitilde{}tɑ˧}}} \textsubscript{2}}\hspace{0.5cm}[\kern2pt{\textcolor{darkblue}{\textbf{\ipa{tɑ˧tɑ˧}}}}\kern2pt]} \hypertarget{tA\string_M~tA\string_M2}{}
\markboth{\textcolor{darkblue}{\textbf{\ipa{tɑ˧\textasciitilde{}tɑ˧}}} \textsubscript{2}}{}
\textcolor{teal}{\mytextsc{adverb(ial)}} \hspace{4pt} Tone: M.
\textcolor{Sepia}{\selectlanguage{english}Exactly (right), just (right).} \zh{刚(好)、正(好)。}  ¶ \textcolor{darkblue}{\textbf{\ipa{tɑ˧\textasciitilde{}tɑ˧ | ho˩˥! |}}} \textcolor{Sepia}{\selectlanguage{english}Just right, exactly right. (Example: a pair of shoes fits perfectly.)} \zh{刚刚好!(如:一双鞋刚好合适)}  
 ¶ \textcolor{darkblue}{\textbf{\ipa{le˧-li˧ tɑ˧\textasciitilde{}tɑ˧}}} \textcolor{Sepia}{\selectlanguage{english}to see clearly} \zh{看清楚}  
\textit{See:} \hyperlink{}{\textcolor{darkblue}{\textbf{\ipa{tɑ˧\textasciitilde{}tɑ˧}}} \textsubscript{1}} 
\lhead{\firstmark}
\rhead{\botmark}

\subsection{\hspace{-0.5cm} {\Large \textcolor{darkblue}{\textbf{\ipa{tɑ˩\textsubscript{a}}}}}\hspace{0.5cm}[\kern2pt{\textcolor{darkblue}{\textbf{\ipa{tɑ˧˥}}}}\kern2pt]} \hypertarget{tA\string_Ba1}{}
\markboth{\textcolor{darkblue}{\textbf{\ipa{tɑ˩\textsubscript{a}}}}}{}
\textcolor{teal}{\mytextsc{classifier}} \hspace{4pt} Tone: L\textsubscript{a}.
\textcolor{Sepia}{\selectlanguage{english}Classifier for sums of money.} \zh{量词:钱(一笔)。}  ¶ \textcolor{darkblue}{\textbf{\ipa{ɖʐe˧ | ɖɯ˧-tɑ˩}}} \textcolor{Sepia}{\selectlanguage{english}a (big) sum of money} \zh{一笔钱}  

\lhead{\firstmark}
\rhead{\botmark}

\subsection{\hspace{-0.5cm} {\Large \textcolor{darkblue}{\textbf{\ipa{tɑ˩dv̩˧˥}}}}\hspace{0.5cm}[\kern2pt{\textcolor{darkblue}{\textbf{\ipa{tɑ˩dv̩˥}}}}\kern2pt]} \hypertarget{tA\string_Bdv\string_=\string_M\string_T1}{}
\markboth{\textcolor{darkblue}{\textbf{\ipa{tɑ˩dv̩˧˥}}}}{}
\textcolor{teal}{\mytextsc{noun}} \hspace{4pt} Tone: LM+MH\#.
\textcolor{Sepia}{\selectlanguage{english}Pocket.} \zh{口袋、衣袋、兜子。}  ¶ \textcolor{darkblue}{\textbf{\ipa{njɤ˧ | tɑ˩dv̩˧-qo˥ | tsʰe˩mæ˩-tɑ˥kɤ˩-lɑ˩ dʑo˩!}}} \textcolor{Sepia}{\selectlanguage{english}I only have ten yuan in my pocket!} \zh{我兜子里只有十元钱!}  
 \zh{量词}: \textcolor{darkblue}{\textbf{\ipa{ɭɯ˧}}}  \mytextsc{clf}: \textcolor{darkblue}{\textbf{\ipa{ɭɯ˧}}} 
\lhead{\firstmark}
\rhead{\botmark}

\subsection{\hspace{-0.5cm} {\Large \textcolor{darkblue}{\textbf{\ipa{tɑ˩dʑɤ\#˥}}}}\hspace{0.5cm}[\kern2pt{\textcolor{darkblue}{\textbf{\ipa{tɑ˩dʑɤ˧˥}}}}\kern2pt]} \hypertarget{tA\string_Bdz£7\#\string_T1}{}
\markboth{\textcolor{darkblue}{\textbf{\ipa{tɑ˩dʑɤ\#˥}}}}{}
\textcolor{teal}{\mytextsc{noun}} \hspace{4pt} Tone: LM+\#H.
\textcolor{Sepia}{\selectlanguage{english}Masculine given name given to the second among twins.} \zh{男性名字,双胞胎中老二的名字。} 
\lhead{\firstmark}
\rhead{\botmark}

\subsection{\hspace{-0.5cm} {\Large \textcolor{darkblue}{\textbf{\ipa{tɑ˩ɖʐo˧dzi˧˥}}}}\hspace{0.5cm}[\kern2pt{\textcolor{darkblue}{\textbf{\ipa{tɑ˧ɖʐo˧dzi˩}}}}\kern2pt]} \hypertarget{tA\string_Bd`z`o\string_Mdzi\string_M\string_T1}{}
\markboth{\textcolor{darkblue}{\textbf{\ipa{tɑ˩ɖʐo˧dzi˧˥}}}}{}
\textcolor{teal}{\mytextsc{noun}} \hspace{4pt} Tone: LM+MH\#.
\textcolor{Sepia}{\selectlanguage{english}Small prayer flag.} \zh{小经幡。}  \zh{量词}: \textcolor{darkblue}{\textbf{\ipa{dzi˩}}}  \mytextsc{clf}: \textcolor{darkblue}{\textbf{\ipa{dzi˩}}} 
\lhead{\firstmark}
\rhead{\botmark}

\subsection{\hspace{-0.5cm} {\Large \textcolor{darkblue}{\textbf{\ipa{tɑ˩hwɤ˩}}}}\hspace{0.5cm}[\kern2pt{\textcolor{darkblue}{\textbf{\ipa{tɑ˧hwɤ˧}}}}\kern2pt]} \hypertarget{tA\string_Bhw7\string_B1}{}
\markboth{\textcolor{darkblue}{\textbf{\ipa{tɑ˩hwɤ˩}}}}{}
\textcolor{teal}{\mytextsc{verb}} \hspace{4pt} Tone: L.
 Borrowing: Chinese  \zh{打发?}
\ding{202} \textcolor{Sepia}{\selectlanguage{english}To offer gifts outside the family circle.} \zh{送礼(给家里以外的人)。}  ¶ \textcolor{darkblue}{\textbf{\ipa{hĩ˧-ki˧ | ɖɯ˧-kʰwɤ˧ tɑ˥hwɤ˩-zo˩-ʝi˩!}}} \textcolor{Sepia}{\selectlanguage{english}We shall have to make a present to people! / It's going to be an occasion to make a present to people! (For instance, when a child goes through the “Coming of age” rite.)} \zh{应该给人家送礼了!(例如,人家为孩子进行成年礼时,要送礼。)}  
 ¶ \textcolor{darkblue}{\textbf{\ipa{zo˧mv̩˥-ki˩, | tɑ˩hwɤ˩ mɤ˥-kv̩˩!}}} \textcolor{Sepia}{\selectlanguage{english}Presents are not for the kids! / We don't give big presents to children!} \zh{不会专门给孩子送(大)礼的!(说明:送礼,是送给家里的主人)}  
 ¶ \textcolor{darkblue}{\textbf{\ipa{ʐɯ˧ tɑ˩hwɤ˩}}} \textcolor{Sepia}{\selectlanguage{english}to offer wine as a present} \zh{送酒(作为礼物)}  
 ¶ \textcolor{darkblue}{\textbf{\ipa{li˩ tɑ˥hwɤ˩}}} \textcolor{Sepia}{\selectlanguage{english}to offer tea as a present} \zh{送茶(作为礼物)}  
 ¶ \textcolor{darkblue}{\textbf{\ipa{dze˧ tɑ˥hwɤ˩}}} \textcolor{Sepia}{\selectlanguage{english}to offer sweets as a present} \zh{送糖(作为礼物)}  
\ding{203} \textcolor{Sepia}{\selectlanguage{english}To give a dowry: to give goods to a young woman when she goes to her new home after her wedding. The dowry used to be brought on horseback, in two wood boxes: gifts must come in pairs, and the dowry is no exception.} \zh{送陪嫁(嫁妆、陪奁)。}  ¶ \textcolor{darkblue}{\textbf{\ipa{ə˧tso˧ tɑ˩hwɤ˩-ʝi˩? | ə˧-sɯ˩kv̩˩ (-dʑo˩), | ɖɯ˧-li˧-ɻ̍˩-bi˩!}}} \textcolor{Sepia}{\selectlanguage{english}What did they give as a dowry? Let's go and have a look! (At a wedding, the gifts given as a dowry are put on public display, for everyone to appreciate the parents' generosity.)} \zh{给的是什么嫁妆?咱们去看一看吧!(结婚的时候,陪嫁展示在大家眼前,显示女方家的大方程度)}  
 ¶ \textcolor{darkblue}{\textbf{\ipa{ti˧tsɯ˥ | qʰɑ˧-ɭɯ˧ tɑ˩hwɤ˩? - ti˧tsɯ˥ | ɲi˧-ɭɯ˧ tɑ˩hwɤ˩!}}} \textcolor{Sepia}{\selectlanguage{english}How many boxes are there in the dowry? - The dowry consists of two boxes!} \zh{陪嫁有几个木箱? - 陪嫁有两个(木箱)!}  

\lhead{\firstmark}
\rhead{\botmark}

\subsection{\hspace{-0.5cm} {\Large \textcolor{darkblue}{\textbf{\ipa{tɑ˩kɤ˧}}}}\hspace{0.5cm}[\kern2pt{\textcolor{darkblue}{\textbf{\ipa{tɑ˩kɤ˩˥}}}}\kern2pt]} \hypertarget{tA\string_Bk7\string_M1}{}
\markboth{\textcolor{darkblue}{\textbf{\ipa{tɑ˩kɤ˧}}}}{}
\textcolor{teal}{\mytextsc{verb}} \hspace{4pt} Tone: .
\textcolor{Sepia}{\selectlanguage{english}To tease (by gestures).} \zh{逗弄(动作)。} 
\lhead{\firstmark}
\rhead{\botmark}

\subsection{\hspace{-0.5cm} {\Large \textcolor{darkblue}{\textbf{\ipa{tɑ˩li˥}}}}\hspace{0.5cm}[\kern2pt{\textcolor{darkblue}{\textbf{\ipa{tɑ˩li˥}}}}\kern2pt]} \hypertarget{tA\string_Bli\string_T1}{}
\markboth{\textcolor{darkblue}{\textbf{\ipa{tɑ˩li˥}}}}{}
\textcolor{teal}{\mytextsc{noun}} \hspace{4pt} Tone: LH.
\textcolor{Sepia}{\selectlanguage{english}Dali (city name).} \zh{大理(汉语借词)。}  Borrowing: Chinese  \zh{大理}

\lhead{\firstmark}
\rhead{\botmark}

\subsection{\hspace{-0.5cm} {\Large \textcolor{darkblue}{\textbf{\ipa{tɑ˩mv̩˩}}}}\hspace{0.5cm}[\kern2pt{\textcolor{darkblue}{\textbf{\ipa{tɑ˩mv̩˩˥}}}}\kern2pt]} \hypertarget{tA\string_Bmv\string_=\string_B1}{}
\markboth{\textcolor{darkblue}{\textbf{\ipa{tɑ˩mv̩˩}}}}{}
\textcolor{teal}{\mytextsc{verb}} \hspace{4pt} Tone: L.
\textcolor{Sepia}{\selectlanguage{english}Proverb.} \zh{谚语。}  ¶ \textcolor{darkblue}{\textbf{\ipa{æ˧ʂæ˧-tɑ˩mv̩˩}}} \textcolor{Sepia}{\selectlanguage{english}same meaning: proverb (literally 'proverb of yore')} \zh{同上:谚语(直译:‘从前的老话’)}  
 ¶ \textcolor{darkblue}{\textbf{\ipa{[F5] æ˧ʂæ˧-tɑ˥mv̩˩}}} \textcolor{Sepia}{\selectlanguage{english}proverb; traditional story} \zh{谚语、传统故事}  

\lhead{\firstmark}
\rhead{\botmark}

\subsection{\hspace{-0.5cm} {\Large \textcolor{darkblue}{\textbf{\ipa{tɑ˩so˩kʰo˥}}}}\hspace{0.5cm}[\kern2pt{\textcolor{darkblue}{\textbf{\ipa{tɑ˩so˩kʰo˥}}}}\kern2pt]} \hypertarget{tA\string_Bso\string_Bk\string_ho\string_T1}{}
\markboth{\textcolor{darkblue}{\textbf{\ipa{tɑ˩so˩kʰo˥}}}}{}
\textcolor{teal}{\mytextsc{adverb(ial)}} \hspace{4pt} Tone: L+H\#.
\textcolor{Sepia}{\selectlanguage{english}Crosslegged (bodily posture).} \zh{盘腿(而坐)。}  ¶ \textcolor{darkblue}{\textbf{\ipa{tɑ˩so˩kʰo˥ | tʰi˧-dzi˩}}} \textcolor{Sepia}{\selectlanguage{english}To sit crosslegged. (This is the usual posture for monks, and also a usual posture for commoners.)} \zh{打坐、盘腿而坐(和尚的坐姿)}  

\lhead{\firstmark}
\rhead{\botmark}

\subsection{\hspace{-0.5cm} {\Large \textcolor{darkblue}{\textbf{\ipa{tɑ˧˥}}}}\hspace{0.5cm}[\kern2pt{\textcolor{darkblue}{\textbf{\ipa{tɑ˧˥}}}}\kern2pt]} \hypertarget{tA\string_M\string_T1}{}
\markboth{\textcolor{darkblue}{\textbf{\ipa{tɑ˧˥}}}}{}
\textcolor{teal}{\mytextsc{verb}} \hspace{4pt} Tone: MH.
\textcolor{Sepia}{\selectlanguage{english}To give way, to fall backward.} \zh{退后。}  ¶ \textcolor{darkblue}{\textbf{\ipa{ʁo˧tʰo˩ tɑ˩}}} \textcolor{Sepia}{\selectlanguage{english}to give way, to fall backward} \zh{往后退}  

\lhead{\firstmark}
\rhead{\botmark}

\subsection{\hspace{-0.5cm} {\Large \textcolor{darkblue}{\textbf{\ipa{tɑ˧˥\textsubscript{a}}}}}\hspace{0.5cm}[\kern2pt{\textcolor{darkblue}{\textbf{\ipa{tɑ˥}}}}\kern2pt]} \hypertarget{tA\string_M\string_Ta1}{}
\markboth{\textcolor{darkblue}{\textbf{\ipa{tɑ˧˥\textsubscript{a}}}}}{}
\textcolor{teal}{\mytextsc{classifier}} \hspace{4pt} Tone: MH\textsubscript{a}.
\textcolor{Sepia}{\selectlanguage{english}Entirely, all; everyone.} \zh{量词:全部、一切,大家。}  ¶ \textcolor{darkblue}{\textbf{\ipa{ɖɯ˧-tɑ˧˥}}} \textcolor{Sepia}{\selectlanguage{english}entirely, all; everyone} \zh{全部、一切,大家}  
 ¶ \textcolor{darkblue}{\textbf{\ipa{ɖɯ˧-tɑ˧=ɻæ˥}}} \textcolor{Sepia}{\selectlanguage{english}entirely, all; everyone (same meaning as above, with a plural morpheme)} \zh{全部、一切,大家(同上,加上多数词素)}  

\lhead{\firstmark}
\rhead{\botmark}

\subsection{\hspace{-0.5cm} {\Large \textcolor{darkblue}{\textbf{\ipa{tɑ˩˥fv˩˥}}}}\hspace{0.5cm}[\kern2pt{\textcolor{darkblue}{\textbf{\ipa{tɑ˩fv˥}}}}\kern2pt]} \hypertarget{tA\string_B\string_Tfv\string_B\string_T1}{}
\markboth{\textcolor{darkblue}{\textbf{\ipa{tɑ˩˥fv˩˥}}}}{}
\textcolor{teal}{\mytextsc{noun}} \hspace{4pt} Tone: LH.LH.
\textcolor{Sepia}{\selectlanguage{english}Excrement.} \zh{大粪(汉语借词)。}  Borrowing: Chinese  \zh{大粪}

\lhead{\firstmark}
\rhead{\botmark}

\subsection{\hspace{-0.5cm} {\Large \textcolor{darkblue}{\textbf{\ipa{tæ˧pv̩˩}}}}\hspace{0.5cm}[\kern2pt{\textcolor{darkblue}{\textbf{\ipa{tæ˧pv̩˩}}}}\kern2pt]} \hypertarget{t\{\string_Mpv\string_=\string_B1}{}
\markboth{\textcolor{darkblue}{\textbf{\ipa{tæ˧pv̩˩}}}}{}
\textcolor{teal}{\mytextsc{adjective}} \hspace{4pt} Tone: L\#.
\textcolor{Sepia}{\selectlanguage{english}Skinny, thin (person).} \zh{瘦(人很瘦)。}  ¶ \textcolor{darkblue}{\textbf{\ipa{v˩tsʰɤ˧˥ | le˧-tæ˥pv˩ kʰɯ˩}}} \textcolor{Sepia}{\selectlanguage{english}to dry vegetables} \zh{将蔬菜弄干(晒干)}  
 ¶ \textcolor{darkblue}{\textbf{\ipa{v˩tsʰɤ˧˥ | tæ˧pv˩ gv˩}}} \textcolor{Sepia}{\selectlanguage{english}to dry vegetables} \zh{将蔬菜弄干(晒干)}  
\textit{See:} \hyperlink{}{\textcolor{darkblue}{\textbf{\ipa{tɑ˧gɤ˩}}}} 
\lhead{\firstmark}
\rhead{\botmark}

\subsection{\hspace{-0.5cm} {\Large \textcolor{darkblue}{\textbf{\ipa{tæ˧ɻæ˩}}}}\hspace{0.5cm}[\kern2pt{\textcolor{darkblue}{\textbf{\ipa{tæ˧ɻæ˩}}}}\kern2pt]} \hypertarget{t\{\string_Mr£`\{\string_B1}{}
\markboth{\textcolor{darkblue}{\textbf{\ipa{tæ˧ɻæ˩}}}}{}
\textcolor{teal}{\mytextsc{noun}} \hspace{4pt} Tone: L\#.
\textcolor{Sepia}{\selectlanguage{english}Oesophagus; Adam's apple.} \zh{喉管、喉结。}  \zh{量词}: \textcolor{darkblue}{\textbf{\ipa{ɭɯ˧}}}  \mytextsc{clf}: \textcolor{darkblue}{\textbf{\ipa{ɭɯ˧}}} 
\lhead{\firstmark}
\rhead{\botmark}

\subsection{\hspace{-0.5cm} {\Large \textcolor{darkblue}{\textbf{\ipa{ti˧}}}}\hspace{0.5cm}[\kern2pt{\textcolor{darkblue}{\textbf{\ipa{ti˥}}}}\kern2pt]} \hypertarget{ti\string_M1}{}
\markboth{\textcolor{darkblue}{\textbf{\ipa{ti˧}}}}{}
\textcolor{teal}{\mytextsc{verb}} \hspace{4pt} Tone: M.
\textcolor{Sepia}{\selectlanguage{english}To become mature, to become an adult.} \zh{成熟(人成熟)。}  ¶ \textcolor{darkblue}{\textbf{\ipa{hĩ˧ tʰv̩˧-v̩˧, | gɤ˩-ti˧-ze˧!}}} \textcolor{Sepia}{\selectlanguage{english}This person has become an adult! / This person has grown up/has become mature!} \zh{这个人,成熟了! / 是大人了!}  
 ¶ \textcolor{darkblue}{\textbf{\ipa{hĩ˧ tʰv̩˧-v̩˧, | mɤ˧-ti˧-sɯ˩!}}} \textcolor{Sepia}{\selectlanguage{english}This person is not mature yet! / This person is not an adult yet!} \zh{这个人,还不成熟!}  
 ¶ \textcolor{darkblue}{\textbf{\ipa{zo˩mv̩˧ | gɤ˩-ti˧, | lo˧ hɑ˧!}}} \textcolor{Sepia}{\selectlanguage{english}For a child to grow/to become an adult is no easy business! (Refers to difficulty both for the child and for the family)} \zh{孩子长成熟(的过程),还是挺难的!}  

\lhead{\firstmark}
\rhead{\botmark}

\subsection{\hspace{-0.5cm} {\Large \textcolor{darkblue}{\textbf{\ipa{ti˧ɖo˥}}}}\hspace{0.5cm}[\kern2pt{\textcolor{darkblue}{\textbf{\ipa{ti˩ɖo˩˥}}}}\kern2pt]} \hypertarget{ti\string_Md`o\string_T1}{}
\markboth{\textcolor{darkblue}{\textbf{\ipa{ti˧ɖo˥}}}}{}
\textcolor{teal}{\mytextsc{noun}} \hspace{4pt} Tone: H\#.
\textcolor{Sepia}{\selectlanguage{english}Masculine given name.} \zh{男性名字。} 
\lhead{\firstmark}
\rhead{\botmark}

\subsection{\hspace{-0.5cm} {\Large \textcolor{darkblue}{\textbf{\ipa{ti˧pʰv̩\#˥}}}}\hspace{0.5cm}[\kern2pt{\textcolor{darkblue}{\textbf{\ipa{ti˩pʰv̩˩˥}}}}\kern2pt]} \hypertarget{ti\string_Mp\string_hv\string_=\#\string_T1}{}
\markboth{\textcolor{darkblue}{\textbf{\ipa{ti˧pʰv̩\#˥}}}}{}
\textcolor{teal}{\mytextsc{noun}} \hspace{4pt} Tone: \#H.
\textcolor{Sepia}{\selectlanguage{english}Copper cup for offerings.} \zh{铜杯盏,做仪式用的。}  \zh{量词}: \textcolor{darkblue}{\textbf{\ipa{fv̩˩}}}  \mytextsc{clf}: \textcolor{darkblue}{\textbf{\ipa{fv̩˩}}} 
\lhead{\firstmark}
\rhead{\botmark}

\subsection{\hspace{-0.5cm} {\Large \textcolor{darkblue}{\textbf{\ipa{ti˧tsɯ˥}}}}\hspace{0.5cm}[\kern2pt{\textcolor{darkblue}{\textbf{\ipa{ti˩tsɯ˥}}}}\kern2pt]} \hypertarget{ti\string_MtsM\string_T1}{}
\markboth{\textcolor{darkblue}{\textbf{\ipa{ti˧tsɯ˥}}}}{}
\textcolor{teal}{\mytextsc{noun}} \hspace{4pt} Tone: H\#.
\textcolor{Sepia}{\selectlanguage{english}Box (woven out of bamboo or wicker).} \zh{竹箱。}  \zh{量词}: \textcolor{darkblue}{\textbf{\ipa{ɭɯ˧}}}  \mytextsc{clf}: \textcolor{darkblue}{\textbf{\ipa{ɭɯ˧}}} 
\lhead{\firstmark}
\rhead{\botmark}

\subsection{\hspace{-0.5cm} {\Large \textcolor{darkblue}{\textbf{\ipa{ti˧ʈʂʰɯ˩}}}}\hspace{0.5cm}[\kern2pt{\textcolor{darkblue}{\textbf{\ipa{ti˧ʈʂʰɯ˥}}}}\kern2pt]} \hypertarget{ti\string_Mt`s`\string_hM\string_B1}{}
\markboth{\textcolor{darkblue}{\textbf{\ipa{ti˧ʈʂʰɯ˩}}}}{}
\textcolor{teal}{\mytextsc{noun}} \hspace{4pt} Tone: L\#.
\textcolor{Sepia}{\selectlanguage{english}Hammer.} \zh{铁锤。}  \zh{量词}: \textcolor{darkblue}{\textbf{\ipa{ɭɯ˧}}}  \mytextsc{clf}: \textcolor{darkblue}{\textbf{\ipa{ɭɯ˧}}} 
\lhead{\firstmark}
\rhead{\botmark}

\subsection{\hspace{-0.5cm} {\Large \textcolor{darkblue}{\textbf{\ipa{ti˩\textsubscript{a}}}}}\hspace{0.5cm}[\kern2pt{\textcolor{darkblue}{\textbf{\ipa{ti˩˥}}}}\kern2pt]} \hypertarget{ti\string_Ba1}{}
\markboth{\textcolor{darkblue}{\textbf{\ipa{ti˩\textsubscript{a}}}}}{}
\textcolor{teal}{\mytextsc{verb}} \hspace{4pt} Tone: L\textsubscript{a}.
\ding{202} \textcolor{Sepia}{\selectlanguage{english}To pound, e.g. pounding Szechuan pepper with a small metal pestle, or pounding earth to build a wall of earth.} \zh{捣(花椒、大蒜……)。}  ¶ \textcolor{darkblue}{\textbf{\ipa{læ˧tsɯ˥ ti˩}}} \textcolor{Sepia}{\selectlanguage{english}to pound hot peppers} \zh{捣辣椒}  
 ¶ \textcolor{darkblue}{\textbf{\ipa{tsʰo˧ko˧ ti˩}}} \textcolor{Sepia}{\selectlanguage{english}to pound cardamom} \zh{捣草果}  
 ¶ \textcolor{darkblue}{\textbf{\ipa{dze˩ ti˥}}} \textcolor{Sepia}{\selectlanguage{english}to pound Szechuan pepper} \zh{捣花椒}  
 ¶ \textcolor{darkblue}{\textbf{\ipa{ʈʂo˩bo˩ ti˥}}} \textcolor{Sepia}{\selectlanguage{english}to build a wall of earth, by pounding the earth} \zh{垒土墙}  
\ding{203} \textcolor{Sepia}{\selectlanguage{english}To hit, to strike lightly.} \zh{拍打。}  ¶ \textcolor{darkblue}{\textbf{\ipa{hĩ˧ ti˥}}} \textcolor{Sepia}{\selectlanguage{english}to slap someone, to hit someone mildly} \zh{拍打人}  
 ¶ \textcolor{darkblue}{\textbf{\ipa{hĩ˧ | ɖɯ˧-v̩˧ ti˩-ze˩}}} \textcolor{Sepia}{\selectlanguage{english}(She/he) has slapped someone.} \zh{(他)拍打了某人。}  

\lhead{\firstmark}
\rhead{\botmark}

\subsection{\hspace{-0.5cm} {\Large \textcolor{darkblue}{\textbf{\ipa{ti˩pʰo˩}}}}\hspace{0.5cm}[\kern2pt{\textcolor{darkblue}{\textbf{\ipa{ti˩pʰo˥}}}}\kern2pt]} \hypertarget{ti\string_Bp\string_ho\string_B1}{}
\markboth{\textcolor{darkblue}{\textbf{\ipa{ti˩pʰo˩}}}}{}
\textcolor{teal}{\mytextsc{noun}} \hspace{4pt} Tone: L.
\textcolor{Sepia}{\selectlanguage{english}Ceiling.} \zh{天花板。}  \zh{量词}: \textcolor{darkblue}{\textbf{\ipa{nɑ˧}}}  \mytextsc{clf}: \textcolor{darkblue}{\textbf{\ipa{nɑ˧}}} 
\lhead{\firstmark}
\rhead{\botmark}

\subsection{\hspace{-0.5cm} {\Large \textcolor{darkblue}{\textbf{\ipa{ti˩tje˧}}}}\hspace{0.5cm}[\kern2pt{\textcolor{darkblue}{\textbf{\ipa{ti˧tje˧}}}}\kern2pt]} \hypertarget{ti\string_Btje\string_M1}{}
\markboth{\textcolor{darkblue}{\textbf{\ipa{ti˩tje˧}}}}{}
\textcolor{teal}{\mytextsc{verb}} \hspace{4pt} Tone: LM.
\textcolor{Sepia}{\selectlanguage{english}To treat, to handle (someone).} \zh{对待(汉语借词)。}  Borrowing: Chinese  \zh{对待}

\lhead{\firstmark}
\rhead{\botmark}

\subsection{\hspace{-0.5cm} {\Large \textcolor{darkblue}{\textbf{\ipa{ti˧˥}}}}\hspace{0.5cm}[\kern2pt{\textcolor{darkblue}{\textbf{\ipa{ti˧˥}}}}\kern2pt]} \hypertarget{ti\string_M\string_T1}{}
\markboth{\textcolor{darkblue}{\textbf{\ipa{ti˧˥}}}}{}
\textcolor{teal}{\mytextsc{verb}} \hspace{4pt} Tone: MH.
\textcolor{Sepia}{\selectlanguage{english}To settle, to decide (Chinese borrowing).} \zh{决定(汉语借词:定)。}  Borrowing: Chinese  \zh{定}

\lhead{\firstmark}
\rhead{\botmark}

\subsection{\hspace{-0.5cm} {\Large \textcolor{darkblue}{\textbf{\ipa{ti˧˥\textsubscript{a}}}}}\hspace{0.5cm}[\kern2pt{\textcolor{darkblue}{\textbf{\ipa{ti˩˥}}}}\kern2pt]} \hypertarget{ti\string_M\string_Ta1}{}
\markboth{\textcolor{darkblue}{\textbf{\ipa{ti˧˥\textsubscript{a}}}}}{}
\textcolor{teal}{\mytextsc{classifier}} \hspace{4pt} Tone: MH\textsubscript{a}.
\textcolor{Sepia}{\selectlanguage{english}Classifier for layers (of dust, of boards…).} \zh{量词:层(一层灰、一层木板……)。}  ¶ \textcolor{darkblue}{\textbf{\ipa{dʑɯ˩-nɑ˩mi˩˥ | gv̩˧-ti˩-qo˩ tʰv̩˩}}} \textcolor{Sepia}{\selectlanguage{english}To arrive at the heart of the heart of the alpine forest. Literally 'in the ninth layer of alpine forest'; 'ninth' here serves as the highest numeral, to refer to an extreme; there is no such thing as a 'first layer', a 'second layer' and so on.} \zh{到深山老林的最深处。直译:‘到深山老林的第九层’。这里的‘九’作为最高的数字,表示‘极深’的意思:不能说‘深山老林的第一层’、‘第二层’等。}  

\lhead{\firstmark}
\rhead{\botmark}

\subsection{\hspace{-0.5cm} {\Large \textcolor{darkblue}{\textbf{\ipa{tjɤ˧hwɑ˧˥}}}}\hspace{0.5cm}[\kern2pt{\textcolor{darkblue}{\textbf{\ipa{tjɤ˧hwɑ˧}}}}\kern2pt]} \hypertarget{tj7\string_MhwA\string_M\string_T1}{}
\markboth{\textcolor{darkblue}{\textbf{\ipa{tjɤ˧hwɑ˧˥}}}}{}
\textcolor{teal}{\mytextsc{noun}} \hspace{4pt} Tone: MH\#.
\textcolor{Sepia}{\selectlanguage{english}Telephone.} \zh{电话(汉语借词)。}  Borrowing: Chinese  \zh{电话}
 \zh{量词}: \textcolor{darkblue}{\textbf{\ipa{ɭɯ˧}}}  \mytextsc{clf}: \textcolor{darkblue}{\textbf{\ipa{ɭɯ˧}}} 
\lhead{\firstmark}
\rhead{\botmark}

\subsection{\hspace{-0.5cm} {\Large \textcolor{darkblue}{\textbf{\ipa{tjɤ˧po˧}}}}\hspace{0.5cm}[\kern2pt{\textcolor{darkblue}{\textbf{\ipa{tjɤ˧po˧˥}}}}\kern2pt]} \hypertarget{tj7\string_Mpo\string_M1}{}
\markboth{\textcolor{darkblue}{\textbf{\ipa{tjɤ˧po˧}}}}{}
\textcolor{teal}{\mytextsc{noun}} \hspace{4pt} Tone: M.
\textcolor{Sepia}{\selectlanguage{english}Pillbox; blockhouse.} \zh{碉堡(汉语借词)。}  Borrowing: Chinese  \zh{碉堡}
 \zh{量词}: \textcolor{darkblue}{\textbf{\ipa{ɭɯ˧}}}  \mytextsc{clf}: \textcolor{darkblue}{\textbf{\ipa{ɭɯ˧}}} 
\lhead{\firstmark}
\rhead{\botmark}

\subsection{\hspace{-0.5cm} {\Large \textcolor{darkblue}{\textbf{\ipa{tjɤ˩˥ʂɯ˧}}}}\hspace{0.5cm}[\kern2pt{\textcolor{darkblue}{\textbf{\ipa{tjɤ˧ʂɯ˧}}}}\kern2pt]} \hypertarget{tj7\string_B\string_Ts`M\string_M1}{}
\markboth{\textcolor{darkblue}{\textbf{\ipa{tjɤ˩˥ʂɯ˧}}}}{}
\textcolor{teal}{\mytextsc{noun}} \hspace{4pt} Tone: LH.M.
\textcolor{Sepia}{\selectlanguage{english}Television.} \zh{电视(汉语借词)。}  Borrowing: Chinese  \zh{电视}
 ¶ \textcolor{darkblue}{\textbf{\ipa{tjɤ˩ʂɯ˧ li˥}}} \textcolor{Sepia}{\selectlanguage{english}to watch television} \zh{看电视}  
 ¶ \textcolor{darkblue}{\textbf{\ipa{tjɤ˩ʂɯ˧-qo˥}}} \textcolor{Sepia}{\selectlanguage{english}on television} \zh{电视上}  

\lhead{\firstmark}
\rhead{\botmark}

\subsection{\hspace{-0.5cm} {\Large \textcolor{darkblue}{\textbf{\ipa{to˥\textsubscript{a}}}}}\hspace{0.5cm}[\kern2pt{\textcolor{darkblue}{\textbf{\ipa{to˩˥}}}}\kern2pt]} \hypertarget{to\string_Ta1}{}
\markboth{\textcolor{darkblue}{\textbf{\ipa{to˥\textsubscript{a}}}}}{}
\textcolor{teal}{\mytextsc{classifier}} \hspace{4pt} Tone: H\textsubscript{a}.
\textcolor{Sepia}{\selectlanguage{english}An armful of.} \zh{量词:抱。}  ¶ \textcolor{darkblue}{\textbf{\ipa{qʰv̩˩ɖʐæ˩˥ | ɖɯ˧-to˥}}} \textcolor{Sepia}{\selectlanguage{english}an armful of string (i.e. a huge quantity of string; see the narrative TraderAndHisSon)} \zh{一抱绳子}  

\lhead{\firstmark}
\rhead{\botmark}

\subsection{\hspace{-0.5cm} {\Large \textcolor{darkblue}{\textbf{\ipa{to˧bɤ\#˥}}}}\hspace{0.5cm}[\kern2pt{\textcolor{darkblue}{\textbf{\ipa{to˩bɤ˥}}}}\kern2pt]} \hypertarget{to\string_Mb7\#\string_T1}{}
\markboth{\textcolor{darkblue}{\textbf{\ipa{to˧bɤ\#˥}}}}{}
\textcolor{teal}{\mytextsc{adjective}} \hspace{4pt} Tone: \#H.
\textcolor{Sepia}{\selectlanguage{english}Empty.} \zh{空。}  ¶ \textcolor{darkblue}{\textbf{\ipa{to˧bɤ˧-ze˩}}} \textcolor{Sepia}{\selectlanguage{english}\mytextsc{pfv}: it's empty, there is nothing left (e.g. a bowl is entirely emptied)} \zh{空了}  
 ¶ \textcolor{darkblue}{\textbf{\ipa{to˧bɤ˧ ɲi˥}}} \textcolor{Sepia}{\selectlanguage{english}\string_ \mytextsc{cop}: it's empty} \zh{是空的}  

\lhead{\firstmark}
\rhead{\botmark}

\subsection{\hspace{-0.5cm} {\Large \textcolor{darkblue}{\textbf{\ipa{to˧kɤ\#˥}}}}\hspace{0.5cm}[\kern2pt{\textcolor{darkblue}{\textbf{\ipa{to˧kɤ˧}}}}\kern2pt]} \hypertarget{to\string_Mk7\#\string_T1}{}
\markboth{\textcolor{darkblue}{\textbf{\ipa{to˧kɤ\#˥}}}}{}
\textcolor{teal}{\mytextsc{noun}} \hspace{4pt} Tone: \#H.
\ding{202} \textcolor{Sepia}{\selectlanguage{english}Forehead.} \zh{额头。}  \zh{量词}: \textcolor{darkblue}{\textbf{\ipa{ʈv̩˩}}} \ding{203} \textcolor{Sepia}{\selectlanguage{english}Luck, good fortune.} \zh{运气。}  ¶ \textcolor{darkblue}{\textbf{\ipa{to˧kɤ˧ dʑɤ˥}}} \textcolor{Sepia}{\selectlanguage{english}to be lucky; to have a good karma} \zh{好运气,运气好}  
 ¶ \textcolor{darkblue}{\textbf{\ipa{njɤ˧ | tsʰi˧ʝi˧ | to˧kɤ˧ dʑjɤ˥ (+ | ʐwæ˩˥)}}} \textcolor{Sepia}{\selectlanguage{english}This year, I am lucky! / This is an auspicious year for me!} \zh{我今年运气好!}  
 \mytextsc{clf}: \textcolor{darkblue}{\textbf{\ipa{ʈv̩˩}}} 
\lhead{\firstmark}
\rhead{\botmark}

\subsection{\hspace{-0.5cm} {\Large \textcolor{darkblue}{\textbf{\ipa{to˧kɤ˧qʰæ˩di˩ | -bæ˩bæ˩˥}}}}\hspace{0.5cm}[\kern2pt{\textcolor{darkblue}{\textbf{\ipa{xxxx non-correspondance entre le nombre de groupes tonals et le nombre de tons}}}}\kern2pt]} \hypertarget{to\string_Mk7\string_Mq\string_h\{\string_Bdi\string_B | -b\{\string_Bb\{\string_B\string_T1}{}
\markboth{\textcolor{darkblue}{\textbf{\ipa{to˧kɤ˧qʰæ˩di˩ | -bæ˩bæ˩˥}}}}{}
\textcolor{teal}{\mytextsc{noun}} \hspace{4pt} Tone: .
\textcolor{Sepia}{\selectlanguage{english}Plant with long filaments.} \zh{永宁的一种植物。}  \zh{量词}: \textcolor{darkblue}{\textbf{\ipa{bæ˩}}}  \mytextsc{clf}: \textcolor{darkblue}{\textbf{\ipa{bæ˩}}} 
\lhead{\firstmark}
\rhead{\botmark}

\subsection{\hspace{-0.5cm} {\Large \textcolor{darkblue}{\textbf{\ipa{to˧qɑ˧}}}}\hspace{0.5cm}[\kern2pt{\textcolor{darkblue}{\textbf{\ipa{to˧qɑ˧}}}}\kern2pt]} \hypertarget{to\string_MqA\string_M1}{}
\markboth{\textcolor{darkblue}{\textbf{\ipa{to˧qɑ˧}}}}{}
\textcolor{teal}{\mytextsc{noun}} \hspace{4pt} Tone: M.
\textcolor{Sepia}{\selectlanguage{english}Kid.} \zh{羔羊。}  \zh{量词}: \textcolor{darkblue}{\textbf{\ipa{ɭɯ˧}}}  \mytextsc{clf}: \textcolor{darkblue}{\textbf{\ipa{ɭɯ˧}}} 
\lhead{\firstmark}
\rhead{\botmark}

\subsection{\hspace{-0.5cm} {\Large \textcolor{darkblue}{\textbf{\ipa{to˧\textasciitilde{}to˧\textsubscript{b}}}}}\hspace{0.5cm}[\kern2pt{\textcolor{darkblue}{\textbf{\ipa{to˧to˥}}}}\kern2pt]} \hypertarget{to\string_M~to\string_Mb1}{}
\markboth{\textcolor{darkblue}{\textbf{\ipa{to˧\textasciitilde{}to˧\textsubscript{b}}}}}{}
\textcolor{teal}{\mytextsc{verb}} \hspace{4pt} Tone: M\textsubscript{b}.
\textcolor{Sepia}{\selectlanguage{english}To hold a child in one's arms; to hug.} \zh{抱小孩子、搂,互相拥抱。}  ¶ \textcolor{darkblue}{\textbf{\ipa{zo˧mv̩˥ to˩\textasciitilde{}to˩}}} \textcolor{Sepia}{\selectlanguage{english}to hold a child in one's arms, to hug a child} \zh{抱小孩子}  

\lhead{\firstmark}
\rhead{\botmark}

\subsection{\hspace{-0.5cm} {\Large \textcolor{darkblue}{\textbf{\ipa{to˩\textsubscript{a}}}} \textsubscript{1}}\hspace{0.5cm}[\kern2pt{\textcolor{darkblue}{\textbf{\ipa{to˧˥}}}}\kern2pt]} \hypertarget{to\string_Ba1}{}
\markboth{\textcolor{darkblue}{\textbf{\ipa{to˩\textsubscript{a}}}} \textsubscript{1}}{}
\textcolor{teal}{\mytextsc{verb}} \hspace{4pt} Tone: L\textsubscript{a}.
\textcolor{Sepia}{\selectlanguage{english}To wrestle.} \zh{摔交。}  ¶ \textcolor{darkblue}{\textbf{\ipa{le˧-to˩-ze˩}}} \textcolor{Sepia}{\selectlanguage{english}\mytextsc{accomp} \string_ \mytextsc{pfv}} \zh{\mytextsc{accomp} \string_ \mytextsc{pfv}}  
 ¶ \textcolor{darkblue}{\textbf{\ipa{ɖʐæ˧\textasciitilde{}ɖʐæ˧ to˩}}} \textcolor{Sepia}{\selectlanguage{english}to wrestle} \zh{摔交}  

\lhead{\firstmark}
\rhead{\botmark}

\subsection{\hspace{-0.5cm} {\Large \textcolor{darkblue}{\textbf{\ipa{to˩\textsubscript{a}}}} \textsubscript{2}}\hspace{0.5cm}[\kern2pt{\textcolor{darkblue}{\textbf{\ipa{to˩˥}}}}\kern2pt]} \hypertarget{to\string_Ba2}{}
\markboth{\textcolor{darkblue}{\textbf{\ipa{to˩\textsubscript{a}}}} \textsubscript{2}}{}
\textcolor{teal}{\mytextsc{verb}} \hspace{4pt} Tone: L\textsubscript{a}.
\textcolor{Sepia}{\selectlanguage{english}To lie down.} \zh{躺下。}  ¶ \textcolor{darkblue}{\textbf{\ipa{tʰi˧-to˩-ɕjɤ˩}}} \textcolor{Sepia}{\selectlanguage{english}to lie down and rest} \zh{躺着休息}  

\lhead{\firstmark}
\rhead{\botmark}

\subsection{\hspace{-0.5cm} {\Large \textcolor{darkblue}{\textbf{\ipa{to˩\textsubscript{a}}}} \textsubscript{3}}\hspace{0.5cm}[\kern2pt{\textcolor{darkblue}{\textbf{\ipa{to˩˥}}}}\kern2pt]} \hypertarget{to\string_Ba3}{}
\markboth{\textcolor{darkblue}{\textbf{\ipa{to˩\textsubscript{a}}}} \textsubscript{3}}{}
\textcolor{teal}{\mytextsc{verb}} \hspace{4pt} Tone: L\textsubscript{a}.
\textcolor{Sepia}{\selectlanguage{english}To stand in a family relationship, to have family ties.} \zh{有亲属关系。}  ¶ \textcolor{darkblue}{\textbf{\ipa{le˧-to˧\textasciitilde{}to˥}}} \textcolor{Sepia}{\selectlanguage{english}\mytextsc{accomp} \string_ \mytextsc{red}} \zh{\mytextsc{accomp} \string_ \mytextsc{red}}  
 ¶ \textcolor{darkblue}{\textbf{\ipa{qʰwɤ˩ɖɯ˩˥ | le˧-to˩-ze˩}}} \textcolor{Sepia}{\selectlanguage{english}We have acquired a family tie! (through adoption, marriage...)} \zh{我们成了亲戚!(通过领养、婚姻……)}  

\lhead{\firstmark}
\rhead{\botmark}

\subsection{\hspace{-0.5cm} {\Large \textcolor{darkblue}{\textbf{\ipa{to˩bi\#˥}}}}\hspace{0.5cm}[\kern2pt{\textcolor{darkblue}{\textbf{\ipa{to˩bi˩˥}}}}\kern2pt]} \hypertarget{to\string_Bbi\#\string_T1}{}
\markboth{\textcolor{darkblue}{\textbf{\ipa{to˩bi\#˥}}}}{}
\textcolor{teal}{\mytextsc{noun}} \hspace{4pt} Tone: LM+\#H.
\textcolor{Sepia}{\selectlanguage{english}Bottle.} \zh{瓶子。}  \zh{量词}: \textcolor{darkblue}{\textbf{\ipa{ɭɯ˧}}}  \mytextsc{clf}: \textcolor{darkblue}{\textbf{\ipa{ɭɯ˧}}} 
\lhead{\firstmark}
\rhead{\botmark}

\subsection{\hspace{-0.5cm} {\Large \textcolor{darkblue}{\textbf{\ipa{to˩bi˩}}}}\hspace{0.5cm}[\kern2pt{\textcolor{darkblue}{\textbf{\ipa{to˥}}}}\kern2pt]} \hypertarget{to\string_Bbi\string_B1}{}
\markboth{\textcolor{darkblue}{\textbf{\ipa{to˩bi˩}}}}{}
\textcolor{teal}{\mytextsc{classifier}} \hspace{4pt} Tone: L.
\textcolor{Sepia}{\selectlanguage{english}Self-classifier for bottles.} \zh{量词:瓶。}  ¶ \textcolor{darkblue}{\textbf{\ipa{ɖɯ˧-to˩bi˩, so˩-to˩bi˩˥, ʐv̩˧-to˥bi˩, qʰv̩˧-to˥bi˩, ʂɯ˧-to˩bi˩, gv̩˧-to˥bi˩, tsʰe˩-to˩bi˩˥}}} \textcolor{Sepia}{\selectlanguage{english}association with numerals from 1 to 10} \zh{与数词结合,一至十}  

\lhead{\firstmark}
\rhead{\botmark}

\subsection{\hspace{-0.5cm} {\Large \textcolor{darkblue}{\textbf{\ipa{to˩kʰv̩˩mi˥}}}}\hspace{0.5cm}[\kern2pt{\textcolor{darkblue}{\textbf{\ipa{to˩kʰv̩˩mi˥}}}}\kern2pt]} \hypertarget{to\string_Bk\string_hv\string_=\string_Bmi\string_T1}{}
\markboth{\textcolor{darkblue}{\textbf{\ipa{to˩kʰv̩˩mi˥}}}}{}
\textcolor{teal}{\mytextsc{noun}} \hspace{4pt} Tone: L+H\#.
\textcolor{Sepia}{\selectlanguage{english}Male dog.} \zh{公狗。}  \zh{量词}: \textcolor{darkblue}{\textbf{\ipa{mi˩}}} \textcolor{darkblue}{\textbf{\ipa{pʰo˧˥}}}  \mytextsc{clf}: \textcolor{darkblue}{\textbf{\ipa{mi˩}}} \textcolor{darkblue}{\textbf{\ipa{pʰo˧˥}}} 
\lhead{\firstmark}
\rhead{\botmark}

\subsection{\hspace{-0.5cm} {\Large \textcolor{darkblue}{\textbf{\ipa{to˩mi˩}}} \textsubscript{1}}\hspace{0.5cm}[\kern2pt{\textcolor{darkblue}{\textbf{\ipa{to˩mi˩˥}}}}\kern2pt]} \hypertarget{to\string_Bmi\string_B1}{}
\markboth{\textcolor{darkblue}{\textbf{\ipa{to˩mi˩}}} \textsubscript{1}}{}
\textcolor{teal}{\mytextsc{noun}} \hspace{4pt} Tone: L.
\textcolor{Sepia}{\selectlanguage{english}Pillar.} \zh{柱子。}  ¶ \textcolor{darkblue}{\textbf{\ipa{hæ̃˧ʂɯ˩-to˩mi˩}}} \textcolor{Sepia}{\selectlanguage{english}the Precious Pillars, the Golden Pillars: a solemn designation for the two pillars of the main building} \zh{‘黄金柱’、‘宝贵柱’:对主屋两个柱子的庄严称呼}  
 \zh{量词}: \textcolor{darkblue}{\textbf{\ipa{ɭɯ˧}}}  \mytextsc{clf}: \textcolor{darkblue}{\textbf{\ipa{ɭɯ˧}}} 
\lhead{\firstmark}
\rhead{\botmark}

\subsection{\hspace{-0.5cm} {\Large \textcolor{darkblue}{\textbf{\ipa{to˩mi˩}}} \textsubscript{2}}\hspace{0.5cm}[\kern2pt{\textcolor{darkblue}{\textbf{\ipa{to˩mi˩˥}}}}\kern2pt]} \hypertarget{to\string_Bmi\string_B2}{}
\markboth{\textcolor{darkblue}{\textbf{\ipa{to˩mi˩}}} \textsubscript{2}}{}
\textcolor{teal}{\mytextsc{noun}} \hspace{4pt} Tone: L.
\textcolor{Sepia}{\selectlanguage{english}Large slope.} \zh{大山坡。} 
\lhead{\firstmark}
\rhead{\botmark}

\subsection{\hspace{-0.5cm} {\Large \textcolor{darkblue}{\textbf{\ipa{to˩pi˩}}}}\hspace{0.5cm}[\kern2pt{\textcolor{darkblue}{\textbf{\ipa{to˩pi˩˥}}}}\kern2pt]} \hypertarget{to\string_Bpi\string_B1}{}
\markboth{\textcolor{darkblue}{\textbf{\ipa{to˩pi˩}}}}{}
\textcolor{teal}{\mytextsc{classifier}} \hspace{4pt} Tone: L.
\textcolor{Sepia}{\selectlanguage{english}Classifier for times: n times as many/much as….} \zh{量词:倍(多几倍、少几倍等等)。}  ¶ \textcolor{darkblue}{\textbf{\ipa{ɖɯ˧-to˩pi˩, ɲi˧-to˩pi˩, so˩-to˩pi˩˥, ʐv̩˧-to˥pi˩, qʰv̩˧-to˥pi˩, ʂɯ˧-to˩pi˩, gv̩˧-to˥pi˩, tsʰe˩-to˩pi˩˥}}} \textcolor{Sepia}{\selectlanguage{english}association with numerals from 1 to 10} \zh{与数词结合,一至十}  

\lhead{\firstmark}
\rhead{\botmark}

\subsection{\hspace{-0.5cm} {\Large \textcolor{darkblue}{\textbf{\ipa{to˩pv̩˧}}}}\hspace{0.5cm}[\kern2pt{\textcolor{darkblue}{\textbf{\ipa{to˩pv̩˥}}}}\kern2pt]} \hypertarget{to\string_Bpv\string_=\string_M1}{}
\markboth{\textcolor{darkblue}{\textbf{\ipa{to˩pv̩˧}}}}{}
\textcolor{teal}{\mytextsc{adverb(ial)}} \hspace{4pt} Tone: LM.
\textcolor{Sepia}{\selectlanguage{english}To begin with, at first, in the first place.} \zh{最初。}  Borrowing: Tibetan?

\lhead{\firstmark}
\rhead{\botmark}

\subsection{\hspace{-0.5cm} {\Large \textcolor{darkblue}{\textbf{\ipa{to˩qo˩lv̩˥}}}}\hspace{0.5cm}[\kern2pt{\textcolor{darkblue}{\textbf{\ipa{to˩qo˩lv̩˥}}}}\kern2pt]} \hypertarget{to\string_Bqo\string_Blv\string_=\string_T1}{}
\markboth{\textcolor{darkblue}{\textbf{\ipa{to˩qo˩lv̩˥}}}}{}
\textcolor{teal}{\mytextsc{adjective}} \hspace{4pt} Tone: L+H\#.
\textcolor{Sepia}{\selectlanguage{english}Round in shape.} \zh{圆形(球很圆)。}  ¶ \textcolor{darkblue}{\textbf{\ipa{to˩qo˩lv̩˥-gv̩˩}}} \textcolor{Sepia}{\selectlanguage{english}round in shape} \zh{圆形}  

\lhead{\firstmark}
\rhead{\botmark}

\subsection{\hspace{-0.5cm} {\Large \textcolor{darkblue}{\textbf{\ipa{to˩qo˧˥}}}}\hspace{0.5cm}[\kern2pt{\textcolor{darkblue}{\textbf{\ipa{xxxx ton non trouvé, à faire manuellement...}}}}\kern2pt]} \hypertarget{to\string_Bqo\string_M\string_T1}{}
\markboth{\textcolor{darkblue}{\textbf{\ipa{to˩qo˧˥}}}}{}
\textcolor{teal}{\mytextsc{verb}} \hspace{4pt} Tone: L+MH\#.
\textcolor{Sepia}{\selectlanguage{english}To turn upside down.} \zh{倒过来、倒放倒置。}  ¶ \textcolor{darkblue}{\textbf{\ipa{to˩qo˧˥ | tɕɯ˧}}} \textcolor{Sepia}{\selectlanguage{english}to put upside down} \zh{倒过来放}  
 ¶ \textcolor{darkblue}{\textbf{\ipa{njɤ˧-ɳɯ˧ | to˩qo˧-bi˧!}}} \textcolor{Sepia}{\selectlanguage{english}I am going to turn (this object) upside down!} \zh{我要(将这个东西)倒过来放!}  
 ¶ \textcolor{darkblue}{\textbf{\ipa{to˩qo˧-ze˥}}} \textcolor{Sepia}{\selectlanguage{english}\mytextsc{pfv}} \zh{倒过来了}  

\lhead{\firstmark}
\rhead{\botmark}

\subsection{\hspace{-0.5cm} {\Large \textcolor{darkblue}{\textbf{\ipa{to˩to˧mi˥}}}}\hspace{0.5cm}[\kern2pt{\textcolor{darkblue}{\textbf{\ipa{to˩to˧mi˥}}}}\kern2pt]} \hypertarget{to\string_Bto\string_Mmi\string_T1}{}
\markboth{\textcolor{darkblue}{\textbf{\ipa{to˩to˧mi˥}}}}{}
\textcolor{teal}{\mytextsc{adverb(ial)}} \hspace{4pt} Tone: LM+H\#.
\ding{202} \textcolor{Sepia}{\selectlanguage{english}Carefully.} \zh{认真地。}  ¶ \textcolor{darkblue}{\textbf{\ipa{njɤ˧-ɳɯ˧ | to˩to˧ mi˥ | ʐwɤ˩-bi˩˥! |}}} \textcolor{Sepia}{\selectlanguage{english}I will explain carefully! / I will explain very clearly, step by step!} \zh{我要认真地讲!}  
 ¶ \textcolor{darkblue}{\textbf{\ipa{to˩to˧-mi˥ | so˩˥}}} \textcolor{Sepia}{\selectlanguage{english}to study with great care} \zh{认真地学习}  
\ding{203} \textcolor{Sepia}{\selectlanguage{english}Intentionally, purposedly, on purpose.} \zh{故意地。} 
\lhead{\firstmark}
\rhead{\botmark}

\subsection{\hspace{-0.5cm} {\Large \textcolor{darkblue}{\textbf{\ipa{to˩ʈɯ˩}}}}\hspace{0.5cm}[\kern2pt{\textcolor{darkblue}{\textbf{\ipa{to˧ʈɯ˧}}}}\kern2pt]} \hypertarget{to\string_Bt`M\string_B1}{}
\markboth{\textcolor{darkblue}{\textbf{\ipa{to˩ʈɯ˩}}}}{}
\textcolor{teal}{\mytextsc{adjective}} \hspace{4pt} Tone: L.
\textcolor{Sepia}{\selectlanguage{english}Short (of person).} \zh{矮。}  ¶ \textcolor{darkblue}{\textbf{\ipa{to˩ʈɯ˩\textasciitilde{}ʈɯ˥}}} \textcolor{Sepia}{\selectlanguage{english}short} \zh{矮}  

\lhead{\firstmark}
\rhead{\botmark}

\subsection{\hspace{-0.5cm} {\Large \textcolor{darkblue}{\textbf{\ipa{to˩zo˩}}}}\hspace{0.5cm}[\kern2pt{\textcolor{darkblue}{\textbf{\ipa{to˧zo˧}}}}\kern2pt]} \hypertarget{to\string_Bzo\string_B1}{}
\markboth{\textcolor{darkblue}{\textbf{\ipa{to˩zo˩}}}}{}
\textcolor{teal}{\mytextsc{noun}} \hspace{4pt} Tone: L.
\textcolor{Sepia}{\selectlanguage{english}Small slope.} \zh{小山坡。} 
\lhead{\firstmark}
\rhead{\botmark}

\subsection{\hspace{-0.5cm} {\Large \textcolor{darkblue}{\textbf{\ipa{to˩˥}}}}\hspace{0.5cm}[\kern2pt{\textcolor{darkblue}{\textbf{\ipa{to˩˥}}}}\kern2pt]} \hypertarget{to\string_B\string_T1}{}
\markboth{\textcolor{darkblue}{\textbf{\ipa{to˩˥}}}}{}
\textcolor{teal}{\mytextsc{noun}} \hspace{4pt} Tone: LH.
\textcolor{Sepia}{\selectlanguage{english}Mountain slope, hillside.} \zh{山坡,岗。}  ¶ \textcolor{darkblue}{\textbf{\ipa{to˩ do˩˥}}} \textcolor{Sepia}{\selectlanguage{english}to climb a hillside} \zh{爬山坡}  
 ¶ \textcolor{darkblue}{\textbf{\ipa{ʁwɤ˧-to˩}}} \textcolor{Sepia}{\selectlanguage{english}mountain slope} \zh{山坡}  
 \zh{量词}: \textcolor{darkblue}{\textbf{\ipa{ɭɯ˧}}}  \mytextsc{clf}: \textcolor{darkblue}{\textbf{\ipa{ɭɯ˧}}} 
\lhead{\firstmark}
\rhead{\botmark}

\subsection{\hspace{-0.5cm} {\Large \textcolor{darkblue}{\textbf{\ipa{tv̩˧˥}}} \textsubscript{1}}\hspace{0.5cm}[\kern2pt{\textcolor{darkblue}{\textbf{\ipa{tv̩˥}}}}\kern2pt]} \hypertarget{tv\string_=\string_M\string_T1}{}
\markboth{\textcolor{darkblue}{\textbf{\ipa{tv̩˧˥}}} \textsubscript{1}}{}
\textcolor{teal}{\mytextsc{verb}} \hspace{4pt} Tone: MH.
\textcolor{Sepia}{\selectlanguage{english}To support, to stabilize, to consolidate.} \zh{搀扶、撑住、稳住。} 
\lhead{\firstmark}
\rhead{\botmark}

\subsection{\hspace{-0.5cm} {\Large \textcolor{darkblue}{\textbf{\ipa{tv̩˧˥}}} \textsubscript{2}}\hspace{0.5cm}[\kern2pt{\textcolor{darkblue}{\textbf{\ipa{tv̩˧˥}}}}\kern2pt]} \hypertarget{tv\string_=\string_M\string_T2}{}
\markboth{\textcolor{darkblue}{\textbf{\ipa{tv̩˧˥}}} \textsubscript{2}}{}
\textcolor{teal}{\mytextsc{verb}} \hspace{4pt} Tone: MH.
\textcolor{Sepia}{\selectlanguage{english}To pour (a liquid) into someone's mouth, to pour down someone's throat (e.g. pouring medicines into the throat of a sick person).} \zh{喂,喂到嘴里。}  ¶ \textcolor{darkblue}{\textbf{\ipa{ʈʂʰæ˧ɣɯ˧ | tʰi˧-tv̩˧˥}}} \textcolor{Sepia}{\selectlanguage{english}to give medicines, to pour medicines into the throat of a sick person} \zh{喂药}  

\lhead{\firstmark}
\rhead{\botmark}

\subsection{\hspace{-0.5cm} {\Large \textcolor{darkblue}{\textbf{\ipa{tv̩˧\textsubscript{a}}}}}\hspace{0.5cm}[\kern2pt{\textcolor{darkblue}{\textbf{\ipa{tv̩˩˥}}}}\kern2pt]} \hypertarget{tv\string_=\string_Ma1}{}
\markboth{\textcolor{darkblue}{\textbf{\ipa{tv̩˧\textsubscript{a}}}}}{}
\textcolor{teal}{\mytextsc{verb}} \hspace{4pt} Tone: M\textsubscript{a}.
\textcolor{Sepia}{\selectlanguage{english}To plant, to bed out (rice).} \zh{耕种、插秧。}  ¶ \textcolor{darkblue}{\textbf{\ipa{ɕi˧ tv̩˧}}} \textcolor{Sepia}{\selectlanguage{english}to bed out rice} \zh{插秧}  
 ¶ \textcolor{darkblue}{\textbf{\ipa{le˧-tv̩˧-ze˧}}}  
 ¶ \textcolor{darkblue}{\textbf{\ipa{le˧-tv̩˥-tv̩˩-ze˩}}}  

\lhead{\firstmark}
\rhead{\botmark}

\subsection{\hspace{-0.5cm} {\Large \textcolor{darkblue}{\textbf{\ipa{tv̩˧\textsubscript{a}}}} \textsubscript{1}}\hspace{0.5cm}[\kern2pt{\textcolor{darkblue}{\textbf{\ipa{tv̩˥}}}}\kern2pt]} \hypertarget{tv\string_=\string_Ma1}{}
\markboth{\textcolor{darkblue}{\textbf{\ipa{tv̩˧\textsubscript{a}}}} \textsubscript{1}}{}
\textcolor{teal}{\mytextsc{classifier}} \hspace{4pt} Tone: M\textsubscript{a}.
\textcolor{Sepia}{\selectlanguage{english}1,000.} \zh{千(数词充当量词)。}  ¶ \textcolor{darkblue}{\textbf{\ipa{ɖɯ˧-tv̩˧}}} \textcolor{Sepia}{\selectlanguage{english}one thousand} \zh{一千}  
 ¶ \textcolor{darkblue}{\textbf{\ipa{ɖɯ˧-tv̩˧ tv̩˧}}} \textcolor{Sepia}{\selectlanguage{english}one thousand thousands = one million} \zh{一千千,等于一百万}  
 ¶ \textcolor{darkblue}{\textbf{\ipa{tsʰe˩-tv̩˩ mæ˥}}} \textcolor{Sepia}{\selectlanguage{english}ten thousand times 10,000, i.e. one hundred million} \zh{十千万,等于一亿}  

\lhead{\firstmark}
\rhead{\botmark}

\subsection{\hspace{-0.5cm} {\Large \textcolor{darkblue}{\textbf{\ipa{tv̩˧\textsubscript{a}}}} \textsubscript{2}}\hspace{0.5cm}[\kern2pt{\textcolor{darkblue}{\textbf{\ipa{tv̩˥}}}}\kern2pt]} \hypertarget{tv\string_=\string_Ma2}{}
\markboth{\textcolor{darkblue}{\textbf{\ipa{tv̩˧\textsubscript{a}}}} \textsubscript{2}}{}
\textcolor{teal}{\mytextsc{classifier}} \hspace{4pt} Tone: M\textsubscript{a}.
\textcolor{Sepia}{\selectlanguage{english}Classifier: a dime, i.e. one tenth of the monetary unit.} \zh{量词:角(钱),一元的十分之一。} 
\lhead{\firstmark}
\rhead{\botmark}

\subsection{\hspace{-0.5cm} {\Large \textcolor{darkblue}{\textbf{\ipa{tv̩˧ɕi˩}}}}\hspace{0.5cm}[\kern2pt{\textcolor{darkblue}{\textbf{\ipa{tv̩˩ɕi˩˥}}}}\kern2pt]} \hypertarget{tv\string_=\string_Ms£i\string_B1}{}
\markboth{\textcolor{darkblue}{\textbf{\ipa{tv̩˧ɕi˩}}}}{}
\textcolor{teal}{\mytextsc{noun}} \hspace{4pt} Tone: L\#.
\textcolor{Sepia}{\selectlanguage{english}Centipede.} \zh{蜈蚣。}  \zh{量词}: \textcolor{darkblue}{\textbf{\ipa{mi˩}}}  \mytextsc{clf}: \textcolor{darkblue}{\textbf{\ipa{mi˩}}} 
\lhead{\firstmark}
\rhead{\botmark}

\subsection{\hspace{-0.5cm} {\Large \textcolor{darkblue}{\textbf{\ipa{tv̩˩ɭɯ˧˥}}}}\hspace{0.5cm}[\kern2pt{\textcolor{darkblue}{\textbf{\ipa{tv̩˩ɭɯ˧˥}}}}\kern2pt]} \hypertarget{tv\string_=\string_Bl\string_RM\string_M\string_T1}{}
\markboth{\textcolor{darkblue}{\textbf{\ipa{tv̩˩ɭɯ˧˥}}}}{}
\textcolor{teal}{\mytextsc{noun}} \hspace{4pt} Tone: LM+MH\#.
\textcolor{Sepia}{\selectlanguage{english}Fine, high-quality basket carried on the back; it had a trough shape. Not in use anymore at the time of fieldwork.} \zh{高级的背篓,过去用它放礼物。}  \zh{量词}: \textcolor{darkblue}{\textbf{\ipa{ɭɯ˧}}}  \mytextsc{clf}: \textcolor{darkblue}{\textbf{\ipa{ɭɯ˧}}} 
\lhead{\firstmark}
\rhead{\botmark}

\subsection{\hspace{-0.5cm} {\Large \textcolor{darkblue}{\textbf{\ipa{tv̩˧po˩}}}}\hspace{0.5cm}[\kern2pt{\textcolor{darkblue}{\textbf{\ipa{tv̩˧po˩}}}}\kern2pt]} \hypertarget{tv\string_=\string_Mpo\string_B1}{}
\markboth{\textcolor{darkblue}{\textbf{\ipa{tv̩˧po˩}}}}{}
\textcolor{teal}{\mytextsc{verb}} \hspace{4pt} Tone: L\#.
\textcolor{Sepia}{\selectlanguage{english}To gamble, to bet, to wager.} \zh{赌博(汉语借词)。}  Borrowing: Chinese  \zh{赌博}

\lhead{\firstmark}
\rhead{\botmark}

\subsection{\hspace{-0.5cm} {\Large \textcolor{darkblue}{\textbf{\ipa{tv̩˧qʰv̩˧}}}}\hspace{0.5cm}[\kern2pt{\textcolor{darkblue}{\textbf{\ipa{tv̩˧qʰv̩˧}}}}\kern2pt]} \hypertarget{tv\string_=\string_Mq\string_hv\string_=\string_M1}{}
\markboth{\textcolor{darkblue}{\textbf{\ipa{tv̩˧qʰv̩˧}}}}{}
\textcolor{teal}{\mytextsc{noun}} \hspace{4pt} Tone: M.
\textcolor{Sepia}{\selectlanguage{english}Temporary tomb, where the body is placed prior to cremation.} \zh{临时坟墓。} 
\lhead{\firstmark}
\rhead{\botmark}

\subsection{\hspace{-0.5cm} {\Large \textcolor{darkblue}{\textbf{\ipa{tv̩˧tsʰɯ˧}}}}\hspace{0.5cm}[\kern2pt{\textcolor{darkblue}{\textbf{\ipa{tv̩˧tsʰɯ˧}}}}\kern2pt]} \hypertarget{tv\string_=\string_Mts\string_hM\string_M1}{}
\markboth{\textcolor{darkblue}{\textbf{\ipa{tv̩˧tsʰɯ˧}}}}{}
\textcolor{teal}{\mytextsc{noun}} \hspace{4pt} Tone: M.
\ding{202} \textcolor{Sepia}{\selectlanguage{english}Time.} \zh{时间。}  ¶ \textcolor{darkblue}{\textbf{\ipa{[F5] njɤ˧ | tv̩˧tsʰɯ˧ mɤ˧-dʑo˧.}}} \textcolor{Sepia}{\selectlanguage{english}I don't have the time.} \zh{我没时间。}  
 ¶ \textcolor{darkblue}{\textbf{\ipa{[F5] njɤ˧ | tv̩˧tsʰɯ˧ dʑo˧.}}} \textcolor{Sepia}{\selectlanguage{english}I have time. / I have some free time. / I have the time.} \zh{我有时间。}  
 \zh{量词}: \textcolor{darkblue}{\textbf{\ipa{ɭɯ˧}}} \ding{203} \textcolor{Sepia}{\selectlanguage{english}Spell of time; hour.} \zh{时间段、小时。}  ¶ \textcolor{darkblue}{\textbf{\ipa{tv̩˧tsʰɯ˧ | ɖɯ˧-ɭɯ˧}}} \textcolor{Sepia}{\selectlanguage{english}one hour} \zh{一个小时}  
 ¶ \textcolor{darkblue}{\textbf{\ipa{tv̩˧tsʰɯ˧ ɖɯ˧-ɭɯ˧ gv̩˧-ze˧!}}} \textcolor{Sepia}{\selectlanguage{english}One hour has gone by.} \zh{一个小时过去了。}  
 ¶ \textcolor{darkblue}{\textbf{\ipa{tv̩˧tsʰɯ˧ ɖɯ˧-ɭɯ˧ le˧-hɯ˩-ze˩.}}} \textcolor{Sepia}{\selectlanguage{english}One hour has gone by.} \zh{一个小时过去了。}  
 ¶ \textcolor{darkblue}{\textbf{\ipa{[F5] tv̩˧tsʰɯ˧ qʰɑ˧-ɭɯ˧?}}} \textcolor{Sepia}{\selectlanguage{english}What time is it? (Literally: “How many hours?”)} \zh{几点了?}  
 \mytextsc{clf}: \textcolor{darkblue}{\textbf{\ipa{ɭɯ˧}}} 
\lhead{\firstmark}
\rhead{\botmark}

\subsection{\hspace{-0.5cm} {\Large \textcolor{darkblue}{\textbf{\ipa{tv̩˩tv̩˩}}}}\hspace{0.5cm}[\kern2pt{\textcolor{darkblue}{\textbf{\ipa{tv̩˩tv̩˩˥}}}}\kern2pt]} \hypertarget{tv\string_=\string_Btv\string_=\string_B1}{}
\markboth{\textcolor{darkblue}{\textbf{\ipa{tv̩˩tv̩˩}}}}{}
\textcolor{teal}{\mytextsc{adjective}} \hspace{4pt} Tone: L.
\ding{202} \textcolor{Sepia}{\selectlanguage{english}Upright.} \zh{直,笔直的(如:站直)。} \ding{203} \textcolor{Sepia}{\selectlanguage{english}Upright, righteous, honest.} \zh{耿直。} 
\lhead{\firstmark}
\rhead{\botmark}

\subsection{\hspace{-0.5cm} {\Large \textcolor{darkblue}{\textbf{\ipa{tv̩˧tv̩˥}}}}\hspace{0.5cm}[\kern2pt{\textcolor{darkblue}{\textbf{\ipa{tv̩˩tv̩˩˥}}}}\kern2pt]} \hypertarget{tv\string_=\string_Mtv\string_=\string_T1}{}
\markboth{\textcolor{darkblue}{\textbf{\ipa{tv̩˧tv̩˥}}}}{}
\textcolor{teal}{\mytextsc{noun}} \hspace{4pt} Tone: H\#.
\textcolor{Sepia}{\selectlanguage{english}Hat.} \zh{帽子。}  \zh{量词}: \textcolor{darkblue}{\textbf{\ipa{ɭɯ˧}}}  \mytextsc{clf}: \textcolor{darkblue}{\textbf{\ipa{ɭɯ˧}}} 
\lhead{\firstmark}
\rhead{\botmark}

\newpage
\section*{\centering- \textcolor{darkblue}{\textbf{\ipa{tʰ}}} -}
\subsection{\hspace{-0.5cm} {\Large \textcolor{darkblue}{\textbf{\ipa{tʰɑ˧‑}}}}\hspace{0.5cm}[\kern2pt{\textcolor{darkblue}{\textbf{\ipa{tʰɑ˥}}}}\kern2pt]} \hypertarget{t\string_hA\string_M‑1}{}
\markboth{\textcolor{darkblue}{\textbf{\ipa{tʰɑ˧‑}}}}{}
\textcolor{teal}{\mytextsc{prefix}} \hspace{4pt} Tone: M/0.
\textcolor{Sepia}{\selectlanguage{english}Prohibitive.} \zh{不要、别\mytextsc{禁止式。}}  ¶ \textcolor{darkblue}{\textbf{\ipa{tʰɑ˧-lɑ˩\textasciitilde{}lɑ˩-ze˩!}}} \textcolor{Sepia}{\selectlanguage{english}Don't quarrel!} \zh{别吵架了!}  
 ¶ \textcolor{darkblue}{\textbf{\ipa{tʰɑ˧-dzo˧\textasciitilde{}dzo˥!}}} \textcolor{Sepia}{\selectlanguage{english}Don't touch!} \zh{不要动来动去! / 不要碰来碰去!}  

\lhead{\firstmark}
\rhead{\botmark}

\subsection{\hspace{-0.5cm} {\Large \textcolor{darkblue}{\textbf{\ipa{tʰɑ˧v̩˥}}}}\hspace{0.5cm}[\kern2pt{\textcolor{darkblue}{\textbf{\ipa{tʰɑ˧v̩˥}}}}\kern2pt]} \hypertarget{t\string_hA\string_Mv\string_=\string_T1}{}
\markboth{\textcolor{darkblue}{\textbf{\ipa{tʰɑ˧v̩˥}}}}{}
\textcolor{teal}{\mytextsc{noun}} \hspace{4pt} Tone: H\#.
\textcolor{Sepia}{\selectlanguage{english}Room for guests (local Chinese word; meaning in standard Chinese: central room, main hall). There is no direct equivalent in Na because the traditional house did not have a room for guests.} \zh{堂屋(汉语借词),来指客房。}  \zh{量词}: \textcolor{darkblue}{\textbf{\ipa{ɭɯ˧}}}  \mytextsc{clf}: \textcolor{darkblue}{\textbf{\ipa{ɭɯ˧}}} 
\lhead{\firstmark}
\rhead{\botmark}

\subsection{\hspace{-0.5cm} {\Large \textcolor{darkblue}{\textbf{\ipa{tʰɑ˩lo˧}}}}\hspace{0.5cm}[\kern2pt{\textcolor{darkblue}{\textbf{\ipa{tʰɑ˩lo˥}}}}\kern2pt]} \hypertarget{t\string_hA\string_Blo\string_M1}{}
\markboth{\textcolor{darkblue}{\textbf{\ipa{tʰɑ˩lo˧}}}}{}
\textcolor{teal}{\mytextsc{noun}} \hspace{4pt} Tone: LM.
\textcolor{Sepia}{\selectlanguage{english}The name given to the plain of Yongning by the Tibetans.} \zh{永宁的藏语名称。}  Borrowing: Tibetan  thar lam
 ¶ \textcolor{darkblue}{\textbf{\ipa{tʰɑ˩lo˧-go˧bɤ˩}}} \textcolor{Sepia}{\selectlanguage{english}the temple of Thar Lam} \zh{永宁大寺}  
 ¶ \textcolor{darkblue}{\textbf{\ipa{tʰɑ˩lo˧ se˧gi˧ kɤ˩mv̩˩}}} \textcolor{Sepia}{\selectlanguage{english}mount Gemu, in Yongning} \zh{永宁格姆山}  

\lhead{\firstmark}
\rhead{\botmark}

\subsection{\hspace{-0.5cm} {\Large \textcolor{darkblue}{\textbf{\ipa{tʰɑ˩mi\#˥}}}}\hspace{0.5cm}[\kern2pt{\textcolor{darkblue}{\textbf{\ipa{tʰɑ˩mi˥}}}}\kern2pt]} \hypertarget{t\string_hA\string_Bmi\#\string_T1}{}
\markboth{\textcolor{darkblue}{\textbf{\ipa{tʰɑ˩mi\#˥}}}}{}
\textcolor{teal}{\mytextsc{noun}} \hspace{4pt} Tone: LM+\#H.
\textcolor{Sepia}{\selectlanguage{english}Female water buffalo.} \zh{母水牛。}  ¶ \textcolor{darkblue}{\textbf{\ipa{dʑi˧mi˧-tʰɑ˩mi˩}}} \textcolor{Sepia}{\selectlanguage{english}same meaning: female water buffalo} \zh{母水牛}  
 ¶ \textcolor{darkblue}{\textbf{\ipa{dʑi˧mi˧ ʈʂʰɯ˧-pʰo˩ dʑo˩, | tʰɑ˩mi˧ ɲi˥!}}} \textcolor{Sepia}{\selectlanguage{english}This buffalo is a female!} \zh{这头水牛是母的!}  
 \zh{量词}: \textcolor{darkblue}{\textbf{\ipa{pʰo˧˥}}}  \mytextsc{clf}: \textcolor{darkblue}{\textbf{\ipa{pʰo˧˥}}} 
\lhead{\firstmark}
\rhead{\botmark}

\subsection{\hspace{-0.5cm} {\Large \textcolor{darkblue}{\textbf{\ipa{tʰɑ˩pʰv̩\#˥}}}}\hspace{0.5cm}[\kern2pt{\textcolor{darkblue}{\textbf{\ipa{tʰɑ˩pʰv̩˥}}}}\kern2pt]} \hypertarget{t\string_hA\string_Bp\string_hv\string_=\#\string_T1}{}
\markboth{\textcolor{darkblue}{\textbf{\ipa{tʰɑ˩pʰv̩\#˥}}}}{}
\textcolor{teal}{\mytextsc{noun}} \hspace{4pt} Tone: LM+\#H.
\textcolor{Sepia}{\selectlanguage{english}Male water buffalo.} \zh{公水牛。}  \zh{量词}: \textcolor{darkblue}{\textbf{\ipa{pʰo˧˥}}}  \mytextsc{clf}: \textcolor{darkblue}{\textbf{\ipa{pʰo˧˥}}} 
\lhead{\firstmark}
\rhead{\botmark}

\subsection{\hspace{-0.5cm} {\Large \textcolor{darkblue}{\textbf{\ipa{tʰɑ˩tʰɑ˩}}}}\hspace{0.5cm}[\kern2pt{\textcolor{darkblue}{\textbf{\ipa{tʰɑ˩tʰɑ˩˥}}}}\kern2pt]} \hypertarget{t\string_hA\string_Bt\string_hA\string_B1}{}
\markboth{\textcolor{darkblue}{\textbf{\ipa{tʰɑ˩tʰɑ˩}}}}{}
\textcolor{teal}{\mytextsc{noun}} \hspace{4pt} Tone: L.
\textcolor{Sepia}{\selectlanguage{english}A good spot, a good place to find a certain species of plant: for instance, mushrooms that will grow there every year.} \zh{采野生植物如菌子等的好地方。}  ¶ \textcolor{darkblue}{\textbf{\ipa{tʰɑ˩tʰɑ˩˥ | ɖɯ˧-kʰwɤ˥}}} \textcolor{Sepia}{\selectlanguage{english}a good spot (for hunting a certain kind of wild plant)} \zh{一个好地方}  
 \zh{量词}: \textcolor{darkblue}{\textbf{\ipa{kʰwɤ˥}}}  \mytextsc{clf}: \textcolor{darkblue}{\textbf{\ipa{kʰwɤ˥}}} 
\lhead{\firstmark}
\rhead{\botmark}

\subsection{\hspace{-0.5cm} {\Large \textcolor{darkblue}{\textbf{\ipa{tʰɑ˩zo\#˥}}}}\hspace{0.5cm}[\kern2pt{\textcolor{darkblue}{\textbf{\ipa{tʰɑ˩zo˥}}}}\kern2pt]} \hypertarget{t\string_hA\string_Bzo\#\string_T1}{}
\markboth{\textcolor{darkblue}{\textbf{\ipa{tʰɑ˩zo\#˥}}}}{}
\textcolor{teal}{\mytextsc{noun}} \hspace{4pt} Tone: LM+\#H.
\textcolor{Sepia}{\selectlanguage{english}Baby water buffalo.} \zh{小水牛。}  \zh{量词}: \textcolor{darkblue}{\textbf{\ipa{mi˩}}}  \mytextsc{clf}: \textcolor{darkblue}{\textbf{\ipa{mi˩}}} 
\lhead{\firstmark}
\rhead{\botmark}

\subsection{\hspace{-0.5cm} {\Large \textcolor{darkblue}{\textbf{\ipa{tʰɑ˩-ʐwæ˧mi˧}}}}\hspace{0.5cm}[\kern2pt{\textcolor{darkblue}{\textbf{\ipa{tʰɑ˧ʐwæ˧mi˧}}}}\kern2pt]} \hypertarget{t\string_hA\string_B-z`w\{\string_Mmi\string_M1}{}
\markboth{\textcolor{darkblue}{\textbf{\ipa{tʰɑ˩-ʐwæ˧mi˧}}}}{}
\textcolor{teal}{\mytextsc{noun}} \hspace{4pt} Tone: L-.
\textcolor{Sepia}{\selectlanguage{english}Donkey, ass (either jack or jenny or foal).} \zh{驴子。}  \zh{量词}: \textcolor{darkblue}{\textbf{\ipa{pʰo˧˥}}}  \mytextsc{clf}: \textcolor{darkblue}{\textbf{\ipa{pʰo˧˥}}} 
\lhead{\firstmark}
\rhead{\botmark}

\subsection{\hspace{-0.5cm} {\Large \textcolor{darkblue}{\textbf{\ipa{tʰɑ˧˥}}} \textsubscript{1}}\hspace{0.5cm}[\kern2pt{\textcolor{darkblue}{\textbf{\ipa{tʰɑ˧˥}}}}\kern2pt]} \hypertarget{t\string_hA\string_M\string_T1}{}
\markboth{\textcolor{darkblue}{\textbf{\ipa{tʰɑ˧˥}}} \textsubscript{1}}{}
\textcolor{teal}{\mytextsc{adjective}} \hspace{4pt} Tone: MH.
\textcolor{Sepia}{\selectlanguage{english}Sharp, keen.} \zh{锋利。} 
\lhead{\firstmark}
\rhead{\botmark}

\subsection{\hspace{-0.5cm} {\Large \textcolor{darkblue}{\textbf{\ipa{tʰɑ˧˥}}} \textsubscript{2}}\hspace{0.5cm}[\kern2pt{\textcolor{darkblue}{\textbf{\ipa{tʰɑ˧˥}}}}\kern2pt]} \hypertarget{t\string_hA\string_M\string_T2}{}
\markboth{\textcolor{darkblue}{\textbf{\ipa{tʰɑ˧˥}}} \textsubscript{2}}{}
\textcolor{teal}{\mytextsc{verb}} \hspace{4pt} Tone: MH.
\textcolor{Sepia}{\selectlanguage{english}To be possible, to be allowed: \mytextsc{permissive}.} \zh{可以,允许。}  ¶ \textcolor{darkblue}{\textbf{\ipa{mɤ˧-tʰɑ˧˥ | mɤ˧-ʐv̩˩! | njɤ˧ | dzɯ˧-bi˧ni˧-mɤ˧-gv̩˧˥!}}} \textcolor{Sepia}{\selectlanguage{english}It's not that good! I don't like to eat that!} \zh{不怎么好吃!我不喜欢吃!}  
 ¶ \textcolor{darkblue}{\textbf{\ipa{[Mushrooms] ə˩ljɤ˩hæ̃˩ʂɯ˥-mo˩-ʈʂʰɯ˩-dʑo˩, | hĩ˧ | mɤ˧-tʰɑ˧˥ | dv̩˩-mɤ˧-kv̩˧˥!}}} \textcolor{Sepia}{\selectlanguage{english}The Golden Mushroom is not all that poisonous! / The Golden Mushroom is not really dangerous!} \zh{黄蜡伞不怎么会让人中毒!/ 毒性不太大!}  

\lhead{\firstmark}
\rhead{\botmark}

\subsection{\hspace{-0.5cm} {\Large \textcolor{darkblue}{\textbf{\ipa{tʰɑ˩˥}}}}\hspace{0.5cm}[\kern2pt{\textcolor{darkblue}{\textbf{\ipa{tʰɑ˩˥}}}}\kern2pt]} \hypertarget{t\string_hA\string_B\string_T1}{}
\markboth{\textcolor{darkblue}{\textbf{\ipa{tʰɑ˩˥}}}}{}
\textcolor{teal}{\mytextsc{noun}} \hspace{4pt} Tone: LH.
\textcolor{Sepia}{\selectlanguage{english}Water buffalo (monosyllabic form, extracted from disyllables such as \zh{/tʰɑ˩mi\#˥/} 'female buffalo').} \zh{水牛。}  \zh{量词}: \textcolor{darkblue}{\textbf{\ipa{pʰo˧˥}}}  \mytextsc{clf}: \textcolor{darkblue}{\textbf{\ipa{pʰo˧˥}}} 
\lhead{\firstmark}
\rhead{\botmark}

\subsection{\hspace{-0.5cm} {\Large \textcolor{darkblue}{\textbf{\ipa{tʰæ˧ɻæ˩}}}}\hspace{0.5cm}[\kern2pt{\textcolor{darkblue}{\textbf{\ipa{tʰæ˧ɻæ˩}}}}\kern2pt]} \hypertarget{t\string_h\{\string_Mr£`\{\string_B1}{}
\markboth{\textcolor{darkblue}{\textbf{\ipa{tʰæ˧ɻæ˩}}}}{}
\textcolor{teal}{\mytextsc{noun}} \hspace{4pt} Tone: L\#.
\textcolor{Sepia}{\selectlanguage{english}Book.} \zh{书。}  \zh{量词}: \textcolor{darkblue}{\textbf{\ipa{pɤ˩}}}  \mytextsc{clf}: \textcolor{darkblue}{\textbf{\ipa{pɤ˩}}} 
\lhead{\firstmark}
\rhead{\botmark}

\subsection{\hspace{-0.5cm} {\Large \textcolor{darkblue}{\textbf{\ipa{tʰæ˩tsɯ˧}}}}\hspace{0.5cm}[\kern2pt{\textcolor{darkblue}{\textbf{\ipa{tʰæ˩tsɯ˥}}}}\kern2pt]} \hypertarget{t\string_h\{\string_BtsM\string_M1}{}
\markboth{\textcolor{darkblue}{\textbf{\ipa{tʰæ˩tsɯ˧}}}}{}
\textcolor{teal}{\mytextsc{noun}} \hspace{4pt} Tone: LM.
\textcolor{Sepia}{\selectlanguage{english}Jar.} \zh{坛子(汉语借词)。}  Borrowing: Chinese  \zh{坛子}

\lhead{\firstmark}
\rhead{\botmark}

\subsection{\hspace{-0.5cm} {\Large \textcolor{darkblue}{\textbf{\ipa{tʰi˧‑}}}}\hspace{0.5cm}[\kern2pt{\textcolor{darkblue}{\textbf{\ipa{tʰi˩˥}}}}\kern2pt]} \hypertarget{t\string_hi\string_M‑1}{}
\markboth{\textcolor{darkblue}{\textbf{\ipa{tʰi˧‑}}}}{}
\textcolor{teal}{\mytextsc{prefix}} \hspace{4pt} Tone: M/0.
\textcolor{Sepia}{\selectlanguage{english}Durative (\mytextsc{dur}).} \zh{持续体。}  ¶ \textcolor{darkblue}{\textbf{\ipa{tʰi˧-dzɯ˥-dʑo˩!}}} \textcolor{Sepia}{\selectlanguage{english}(She) is eating!} \zh{她在吃东西!}  
 ¶ \textcolor{darkblue}{\textbf{\ipa{tʰi˧-mɤ˧-ɲi˥}}} \textcolor{Sepia}{\selectlanguage{english}otherwise, or else} \zh{否则、要不然}  

\lhead{\firstmark}
\rhead{\botmark}

\subsection{\hspace{-0.5cm} {\Large \textcolor{darkblue}{\textbf{\ipa{tʰi˧}}}}\hspace{0.5cm}[\kern2pt{\textcolor{darkblue}{\textbf{\ipa{tʰi˥}}}}\kern2pt]} \hypertarget{t\string_hi\string_M1}{}
\markboth{\textcolor{darkblue}{\textbf{\ipa{tʰi˧}}}}{}
\textcolor{teal}{\mytextsc{adjective}} \hspace{4pt} Tone: M.
\textcolor{Sepia}{\selectlanguage{english}Able; capable; competent; clever.} \zh{能干。}  ¶ \textcolor{darkblue}{\textbf{\ipa{ɖwæ˧˥ | tʰi˧}}} \textcolor{Sepia}{\selectlanguage{english}\mytextsc{intensive}.very} \zh{很能干}  
 ¶ \textcolor{darkblue}{\textbf{\ipa{mv̩˩tʰi˩ tʰv̩˩-v̩˩˥}}} \textcolor{Sepia}{\selectlanguage{english}that clever woman} \zh{那个聪明女人}  
 ¶ \textcolor{darkblue}{\textbf{\ipa{zo˧tʰi˧}}} \textcolor{Sepia}{\selectlanguage{english}clever man} \zh{聪明男人}  

\lhead{\firstmark}
\rhead{\botmark}

\subsection{\hspace{-0.5cm} {\Large \textcolor{darkblue}{\textbf{\ipa{tʰi˩\textsubscript{a}}}}}\hspace{0.5cm}[\kern2pt{\textcolor{darkblue}{\textbf{\ipa{tʰi˧˥}}}}\kern2pt]} \hypertarget{t\string_hi\string_Ba1}{}
\markboth{\textcolor{darkblue}{\textbf{\ipa{tʰi˩\textsubscript{a}}}}}{}
\textcolor{teal}{\mytextsc{verb}} \hspace{4pt} Tone: L\textsubscript{a}.
\textcolor{Sepia}{\selectlanguage{english}To plane (wood flat).} \zh{刨。}  ¶ \textcolor{darkblue}{\textbf{\ipa{tso˧\textasciitilde{}tso˧ tʰi˥(-ze˩)}}} \textcolor{Sepia}{\selectlanguage{english}to plane things} \zh{刨东西}  
 ¶ \textcolor{darkblue}{\textbf{\ipa{le˧-tʰi˩-ze˩}}} \textcolor{Sepia}{\selectlanguage{english}\mytextsc{accomp} \string_ \mytextsc{pfv}} \zh{刨了}  
 ¶ \textcolor{darkblue}{\textbf{\ipa{tso˧\textasciitilde{}tso˧ | le˧-tʰi˩(-ze˩)}}} \textcolor{Sepia}{\selectlanguage{english}to plane things} \zh{刨东西}  
 ¶ \textcolor{darkblue}{\textbf{\ipa{pæ˩pʰæ˧ tʰi˥}}} \textcolor{Sepia}{\selectlanguage{english}to plane a plank} \zh{刨木板}  

\lhead{\firstmark}
\rhead{\botmark}

\subsection{\hspace{-0.5cm} {\Large \textcolor{darkblue}{\textbf{\ipa{tʰi˩mi\#˥}}}}\hspace{0.5cm}[\kern2pt{\textcolor{darkblue}{\textbf{\ipa{tʰi˧mi˥}}}}\kern2pt]} \hypertarget{t\string_hi\string_Bmi\#\string_T1}{}
\markboth{\textcolor{darkblue}{\textbf{\ipa{tʰi˩mi\#˥}}}}{}
\textcolor{teal}{\mytextsc{noun}} \hspace{4pt} Tone: LM+\#H.
\textcolor{Sepia}{\selectlanguage{english}Large plane.} \zh{大刨。}  \zh{量词}: \textcolor{darkblue}{\textbf{\ipa{nɑ˧}}}  \mytextsc{clf}: \textcolor{darkblue}{\textbf{\ipa{nɑ˧}}} 
\lhead{\firstmark}
\rhead{\botmark}

\subsection{\hspace{-0.5cm} {\Large \textcolor{darkblue}{\textbf{\ipa{tʰi˩zo\#˥}}}}\hspace{0.5cm}[\kern2pt{\textcolor{darkblue}{\textbf{\ipa{tʰi˧zo˩}}}}\kern2pt]} \hypertarget{t\string_hi\string_Bzo\#\string_T1}{}
\markboth{\textcolor{darkblue}{\textbf{\ipa{tʰi˩zo\#˥}}}}{}
\textcolor{teal}{\mytextsc{noun}} \hspace{4pt} Tone: LM+\#H.
\textcolor{Sepia}{\selectlanguage{english}Small plane.} \zh{小刨。}  \zh{量词}: \textcolor{darkblue}{\textbf{\ipa{nɑ˧}}}  \mytextsc{clf}: \textcolor{darkblue}{\textbf{\ipa{nɑ˧}}} 
\lhead{\firstmark}
\rhead{\botmark}

\subsection{\hspace{-0.5cm} {\Large \textcolor{darkblue}{\textbf{\ipa{tʰi˩˥}}} \textsubscript{1}}\hspace{0.5cm}[\kern2pt{\textcolor{darkblue}{\textbf{\ipa{tʰi˩˥}}}}\kern2pt]} \hypertarget{t\string_hi\string_B\string_T1}{}
\markboth{\textcolor{darkblue}{\textbf{\ipa{tʰi˩˥}}} \textsubscript{1}}{}
\textcolor{teal}{\mytextsc{noun}} \hspace{4pt} Tone: LH.
\textcolor{Sepia}{\selectlanguage{english}Plane.} \zh{刨。}  \zh{量词}: \textcolor{darkblue}{\textbf{\ipa{nɑ˧}}}  \mytextsc{clf}: \textcolor{darkblue}{\textbf{\ipa{nɑ˧}}} 
\lhead{\firstmark}
\rhead{\botmark}

\subsection{\hspace{-0.5cm} {\Large \textcolor{darkblue}{\textbf{\ipa{tʰi˩˥}}} \textsubscript{2}}\hspace{0.5cm}[\kern2pt{\textcolor{darkblue}{\textbf{\ipa{tʰi˩˥}}}}\kern2pt]} \hypertarget{t\string_hi\string_B\string_T2}{}
\markboth{\textcolor{darkblue}{\textbf{\ipa{tʰi˩˥}}} \textsubscript{2}}{}
\textcolor{teal}{\mytextsc{discourse}} \textcolor{teal}{\mytextsc{particle}} \hspace{4pt} Tone: LM? LH?.
\textcolor{Sepia}{\selectlanguage{english}Discourse particle: so, then, and then.} \zh{然后。} 
\lhead{\firstmark}
\rhead{\botmark}

\subsection{\hspace{-0.5cm} {\Large \textcolor{darkblue}{\textbf{\ipa{tʰo˥\textsubscript{a}}}}}\hspace{0.5cm}[\kern2pt{\textcolor{darkblue}{\textbf{\ipa{tʰo˥}}}}\kern2pt]} \hypertarget{t\string_ho\string_Ta1}{}
\markboth{\textcolor{darkblue}{\textbf{\ipa{tʰo˥\textsubscript{a}}}}}{}
\textcolor{teal}{\mytextsc{classifier}} \hspace{4pt} Tone: H\textsubscript{a}.
\textcolor{Sepia}{\selectlanguage{english}Classifier for solutions.} \zh{量词:办法,解决的方法(一个)。}  ¶ \textcolor{darkblue}{\textbf{\ipa{ə˧tso˧ tʰo˧ dʑo˧-kv̩˩?}}} \textcolor{Sepia}{\selectlanguage{english}What can we do about it? / What can be done about it?} \zh{有什么办法?}  

\lhead{\firstmark}
\rhead{\botmark}

\subsection{\hspace{-0.5cm} {\Large \textcolor{darkblue}{\textbf{\ipa{tʰo˥\textsubscript{a}}}}}\hspace{0.5cm}[\kern2pt{\textcolor{darkblue}{\textbf{\ipa{tʰo˩˥}}}}\kern2pt]} \hypertarget{t\string_ho\string_Ta1}{}
\markboth{\textcolor{darkblue}{\textbf{\ipa{tʰo˥\textsubscript{a}}}}}{}
\textcolor{teal}{\mytextsc{classifier}} \hspace{4pt} Tone: H\textsubscript{a}.
\textcolor{Sepia}{\selectlanguage{english}Classifier for sets.} \zh{量词:套(汉语借词)。}  Borrowing: Chinese  \zh{套}

\lhead{\firstmark}
\rhead{\botmark}

\subsection{\hspace{-0.5cm} {\Large \textcolor{darkblue}{\textbf{\ipa{tʰo˧ɕi˩}}}}\hspace{0.5cm}[\kern2pt{\textcolor{darkblue}{\textbf{\ipa{tʰo˧ɕi˧}}}}\kern2pt]} \hypertarget{t\string_ho\string_Ms£i\string_B1}{}
\markboth{\textcolor{darkblue}{\textbf{\ipa{tʰo˧ɕi˩}}}}{}
\textcolor{teal}{\mytextsc{noun}} \hspace{4pt} Tone: L\#.
\textcolor{Sepia}{\selectlanguage{english}Messenger.} \zh{通信员(汉语借词)。}  Borrowing: Chinese  \zh{通信}
 \zh{量词}: \textcolor{darkblue}{\textbf{\ipa{v˧}}}  \mytextsc{clf}: \textcolor{darkblue}{\textbf{\ipa{v˧}}} 
\lhead{\firstmark}
\rhead{\botmark}

\subsection{\hspace{-0.5cm} {\Large \textcolor{darkblue}{\textbf{\ipa{tʰo˧ɕi˧˥}}}}\hspace{0.5cm}[\kern2pt{\textcolor{darkblue}{\textbf{\ipa{tʰo˧ɕi˩}}}}\kern2pt]} \hypertarget{t\string_ho\string_Ms£i\string_M\string_T1}{}
\markboth{\textcolor{darkblue}{\textbf{\ipa{tʰo˧ɕi˧˥}}}}{}
\textcolor{teal}{\mytextsc{noun}} \hspace{4pt} Tone: MH\#.
\textcolor{Sepia}{\selectlanguage{english}Forest of conifers.} \zh{松树林。}  \zh{量词}: \textcolor{darkblue}{\textbf{\ipa{pʰæ˧˥}}}  \mytextsc{clf}: \textcolor{darkblue}{\textbf{\ipa{pʰæ˧˥}}} 
\lhead{\firstmark}
\rhead{\botmark}

\subsection{\hspace{-0.5cm} {\Large \textcolor{darkblue}{\textbf{\ipa{tʰo˧dzi˩}}}}\hspace{0.5cm}[\kern2pt{\textcolor{darkblue}{\textbf{\ipa{tʰo˧dzi˧˥}}}}\kern2pt]} \hypertarget{t\string_ho\string_Mdzi\string_B1}{}
\markboth{\textcolor{darkblue}{\textbf{\ipa{tʰo˧dzi˩}}}}{}
\textcolor{teal}{\mytextsc{noun}} \hspace{4pt} Tone: L\#.
\textcolor{Sepia}{\selectlanguage{english}Pine tree.} \zh{松树。}  \zh{量词}: \textcolor{darkblue}{\textbf{\ipa{dzi˩}}}  \mytextsc{clf}: \textcolor{darkblue}{\textbf{\ipa{dzi˩}}} 
\lhead{\firstmark}
\rhead{\botmark}

\subsection{\hspace{-0.5cm} {\Large \textcolor{darkblue}{\textbf{\ipa{tʰo˧dzi˩-hwæ˩tsɯ˩}}}}\hspace{0.5cm}[\kern2pt{\textcolor{darkblue}{\textbf{\ipa{xxxx non-correspondance entre le nombre de morphèmes et le nombre de tons de morphèmes}}}}\kern2pt]} \hypertarget{t\string_ho\string_Mdzi\string_B-hw\{\string_BtsM\string_B1}{}
\markboth{\textcolor{darkblue}{\textbf{\ipa{tʰo˧dzi˩-hwæ˩tsɯ˩}}}}{}
\textcolor{teal}{\mytextsc{noun}} \hspace{4pt} Tone: LM-.
\textcolor{Sepia}{\selectlanguage{english}Hedgehog.} \zh{刺猬。} 
\lhead{\firstmark}
\rhead{\botmark}

\subsection{\hspace{-0.5cm} {\Large \textcolor{darkblue}{\textbf{\ipa{tʰo˧fv̩˧}}}}\hspace{0.5cm}[\kern2pt{\textcolor{darkblue}{\textbf{\ipa{xxxx non-correspondance entre le nombre de morphèmes et le nombre de tons de morphèmes}}}}\kern2pt]} \hypertarget{t\string_ho\string_Mfv\string_=\string_M1}{}
\markboth{\textcolor{darkblue}{\textbf{\ipa{tʰo˧fv̩˧}}}}{}
\textcolor{teal}{\mytextsc{noun}} \hspace{4pt} Tone: M.
\textcolor{Sepia}{\selectlanguage{english}Bandit, brigand.} \zh{土匪(汉语借词)。}  Borrowing: Chinese  \zh{土匪}

\lhead{\firstmark}
\rhead{\botmark}

\subsection{\hspace{-0.5cm} {\Large \textcolor{darkblue}{\textbf{\ipa{tʰo˧lɑ˧tɕi˧}}}}\hspace{0.5cm}[\kern2pt{\textcolor{darkblue}{\textbf{\ipa{tʰo˧lɑ˧tɕi˧}}}}\kern2pt]} \hypertarget{t\string_ho\string_MlA\string_Mts£i\string_M1}{}
\markboth{\textcolor{darkblue}{\textbf{\ipa{tʰo˧lɑ˧tɕi˧}}}}{}
\textcolor{teal}{\mytextsc{noun}} \hspace{4pt} Tone: M.
\textcolor{Sepia}{\selectlanguage{english}Tractor.} \zh{拖拉机(汉语借词)。}  Borrowing: Chinese  \zh{洋火}
 ¶ \textcolor{darkblue}{\textbf{\ipa{bo˩mi˧-tʰo˧lɑ˧tɕi˧}}} \textcolor{Sepia}{\selectlanguage{english}'sow-tractor': a small tractor (the first type that was introduced into Yongning)} \zh{‘母猪拖拉机’:小型拖拉机}  
 \zh{量词}: \textcolor{darkblue}{\textbf{\ipa{yyyy}}}  \mytextsc{clf}: \textcolor{darkblue}{\textbf{\ipa{yyyy}}} 
\lhead{\firstmark}
\rhead{\botmark}

\subsection{\hspace{-0.5cm} {\Large \textcolor{darkblue}{\textbf{\ipa{tʰo˧li˧}}}}\hspace{0.5cm}[\kern2pt{\textcolor{darkblue}{\textbf{\ipa{tʰo˧li˧}}}}\kern2pt]} \hypertarget{t\string_ho\string_Mli\string_M1}{}
\markboth{\textcolor{darkblue}{\textbf{\ipa{tʰo˧li˧}}}}{}
\textcolor{teal}{\mytextsc{noun}} \hspace{4pt} Tone: M.
\textcolor{Sepia}{\selectlanguage{english}Rabbit.} \zh{兔子。}  \zh{量词}: \textcolor{darkblue}{\textbf{\ipa{mi˩}}}  \mytextsc{clf}: \textcolor{darkblue}{\textbf{\ipa{mi˩}}} 
\lhead{\firstmark}
\rhead{\botmark}

\subsection{\hspace{-0.5cm} {\Large \textcolor{darkblue}{\textbf{\ipa{tʰo˧li˧kʰv̩˧˥}}}}\hspace{0.5cm}[\kern2pt{\textcolor{darkblue}{\textbf{\ipa{tʰo˧li˧kʰv̩˧˥}}}}\kern2pt]} \hypertarget{t\string_ho\string_Mli\string_Mk\string_hv\string_=\string_M\string_T1}{}
\markboth{\textcolor{darkblue}{\textbf{\ipa{tʰo˧li˧kʰv̩˧˥}}}}{}
\textcolor{teal}{\mytextsc{noun}} \hspace{4pt} Tone: MH\#.
\textcolor{Sepia}{\selectlanguage{english}Year of the rabbit.} \zh{兔年。} 
\lhead{\firstmark}
\rhead{\botmark}

\subsection{\hspace{-0.5cm} {\Large \textcolor{darkblue}{\textbf{\ipa{tʰo˧li˧-pʰv̩\#˥}}}}\hspace{0.5cm}[\kern2pt{\textcolor{darkblue}{\textbf{\ipa{xxxx non-correspondance entre le nombre de morphèmes et le nombre de tons de morphèmes}}}}\kern2pt]} \hypertarget{t\string_ho\string_Mli\string_M-p\string_hv\string_=\#\string_T1}{}
\markboth{\textcolor{darkblue}{\textbf{\ipa{tʰo˧li˧-pʰv̩\#˥}}}}{}
\textcolor{teal}{\mytextsc{noun}} \hspace{4pt} Tone: \#H.
\textcolor{Sepia}{\selectlanguage{english}Male rabbit.} \zh{公兔。}  \zh{量词}: \textcolor{darkblue}{\textbf{\ipa{mi˩}}}  \mytextsc{clf}: \textcolor{darkblue}{\textbf{\ipa{mi˩}}} 
\lhead{\firstmark}
\rhead{\botmark}

\subsection{\hspace{-0.5cm} {\Large \textcolor{darkblue}{\textbf{\ipa{tʰo˧li˧-zo\#˥}}}}\hspace{0.5cm}[\kern2pt{\textcolor{darkblue}{\textbf{\ipa{xxxx non-correspondance entre le nombre de morphèmes et le nombre de tons de morphèmes}}}}\kern2pt]} \hypertarget{t\string_ho\string_Mli\string_M-zo\#\string_T1}{}
\markboth{\textcolor{darkblue}{\textbf{\ipa{tʰo˧li˧-zo\#˥}}}}{}
\textcolor{teal}{\mytextsc{noun}} \hspace{4pt} Tone: \#H.
\textcolor{Sepia}{\selectlanguage{english}Baby rabbit.} \zh{小兔。}  \zh{量词}: \textcolor{darkblue}{\textbf{\ipa{ɭɯ˧}}}  \mytextsc{clf}: \textcolor{darkblue}{\textbf{\ipa{ɭɯ˧}}} 
\lhead{\firstmark}
\rhead{\botmark}

\subsection{\hspace{-0.5cm} {\Large \textcolor{darkblue}{\textbf{\ipa{tʰo˧-mo˩}}}}\hspace{0.5cm}[\kern2pt{\textcolor{darkblue}{\textbf{\ipa{xxxx non-correspondance entre le nombre de morphèmes et le nombre de tons de morphèmes}}}}\kern2pt]} \hypertarget{t\string_ho\string_M-mo\string_B1}{}
\markboth{\textcolor{darkblue}{\textbf{\ipa{tʰo˧-mo˩}}}}{}
\textcolor{teal}{\mytextsc{noun}} \hspace{4pt} Tone: L\#.
\textcolor{Sepia}{\selectlanguage{english}“Pine-tree mushroom”: an edible mushroom often found close to pine trees.} \zh{“松树菌”:一种菌子。} 
\lhead{\firstmark}
\rhead{\botmark}

\subsection{\hspace{-0.5cm} {\Large \textcolor{darkblue}{\textbf{\ipa{tʰo˧ɻæ˥}}}}\hspace{0.5cm}[\kern2pt{\textcolor{darkblue}{\textbf{\ipa{tʰo˧ɻæ˥}}}}\kern2pt]} \hypertarget{t\string_ho\string_Mr£`\{\string_T1}{}
\markboth{\textcolor{darkblue}{\textbf{\ipa{tʰo˧ɻæ˥}}}}{}
\textcolor{teal}{\mytextsc{noun}} \hspace{4pt} Tone: H\#.
\textcolor{Sepia}{\selectlanguage{english}Pine-nut kernel.} \zh{松子。}  \zh{量词}: \textcolor{darkblue}{\textbf{\ipa{ʈʂwɤ˧}}}  \mytextsc{clf}: \textcolor{darkblue}{\textbf{\ipa{ʈʂwɤ˧}}} 
\lhead{\firstmark}
\rhead{\botmark}

\subsection{\hspace{-0.5cm} {\Large \textcolor{darkblue}{\textbf{\ipa{tʰo˧tsʰe˧-ʁwɤ\#˥}}}}\hspace{0.5cm}[\kern2pt{\textcolor{darkblue}{\textbf{\ipa{xxxx non-correspondance entre le nombre de morphèmes et le nombre de tons de morphèmes}}}}\kern2pt]} \hypertarget{t\string_ho\string_Mts\string_he\string_M-Rw7\#\string_T1}{}
\markboth{\textcolor{darkblue}{\textbf{\ipa{tʰo˧tsʰe˧-ʁwɤ\#˥}}}}{}
\textcolor{teal}{\mytextsc{noun}} \hspace{4pt} Tone: \#H.
\textcolor{Sepia}{\selectlanguage{english}A village close to the Hot Springs.} \zh{温泉乡的一个村落。}  ¶ \textcolor{darkblue}{\textbf{\ipa{tʰo˧tsʰe\#˥}}} \textcolor{Sepia}{\selectlanguage{english}same meaning} \zh{同上}  
 ¶ \textcolor{darkblue}{\textbf{\ipa{ə˧go˧-ʁwɤ˧, | ʁwɤ˧lɑ˩-bi˩, | bæ˧ʁwɤ˧, | tʰo˧tsʰe\#˥, | pi˧tsʰe˩-di˩, | pɤ˧dʑɤ˩-di˩, | ʁwɤ˧tv̩˧}}} \textcolor{Sepia}{\selectlanguage{english}Villages that one encounters as one leaves the plain of Yongning (away from the Lake); the first two are perceived as villages with a high proportion of Na members, and the third as a mostly Na village, whereas the next ones are Pumi (Prinmi).} \zh{永宁背向泸沽湖方向经过的村落。前两个村落拥有相当大的摩梭人口比例,第三个村落是摩梭村,最后一个是普米村。}  
 ¶ \textcolor{darkblue}{\textbf{\ipa{tʰo˧tsʰe˧: | bɤ˧!}}} \textcolor{Sepia}{\selectlanguage{english}\textcolor{darkblue}{\textbf{\ipa{/tʰo˧tsʰe˧/}}} is a Pumi village!} \zh{fv:/tʰo˧tsʰe˧/是一个普米族村落!}  

\lhead{\firstmark}
\rhead{\botmark}

\subsection{\hspace{-0.5cm} {\Large \textcolor{darkblue}{\textbf{\ipa{tʰo˧ʈɯ\#˥}}}}\hspace{0.5cm}[\kern2pt{\textcolor{darkblue}{\textbf{\ipa{tʰo˩ʈɯ˩˥}}}}\kern2pt]} \hypertarget{t\string_ho\string_Mt`M\#\string_T1}{}
\markboth{\textcolor{darkblue}{\textbf{\ipa{tʰo˧ʈɯ\#˥}}}}{}
\textcolor{teal}{\mytextsc{noun}} \hspace{4pt} Tone: \#H.
\textcolor{Sepia}{\selectlanguage{english}A village in Yongning; Chinese: Tuozhikaiji.} \zh{拖支开基村(永宁的一个村落)。}  ¶ \textcolor{darkblue}{\textbf{\ipa{dʑɤ˩bv̩˧kɤ˧-sɑ˥ʁwɤ˩, | hi˩ʁwɤ˩-lo˥, | æ˩mi˧-ʁwɤ\#˥, | lɑ˧lo˧-ʁwɤ˥, | lɑ˧ŋwɤ˧, | bɤ˧tsʰo˧gv̩˥, | ə˧lɑ˧-ʁwɤ\#˥, | gæ˧ɻæ˩, | qʰæ˧tɕʰi˧, | tʰo˧ʈɯ\#˥}}} \textcolor{Sepia}{\selectlanguage{english}the ten villages traditionally considered as part of Yongning} \zh{摩梭传统地理概念中,属于永宁的十个村落}  

\lhead{\firstmark}
\rhead{\botmark}

\subsection{\hspace{-0.5cm} {\Large \textcolor{darkblue}{\textbf{\ipa{tʰo˧ʐv̩˥}}}}\hspace{0.5cm}[\kern2pt{\textcolor{darkblue}{\textbf{\ipa{tʰo˩ʐv̩˩˥}}}}\kern2pt]} \hypertarget{t\string_ho\string_Mz`v\string_=\string_T1}{}
\markboth{\textcolor{darkblue}{\textbf{\ipa{tʰo˧ʐv̩˥}}}}{}
\textcolor{teal}{\mytextsc{noun}} \hspace{4pt} Tone: H\#.
\textcolor{Sepia}{\selectlanguage{english}Pigeon.} \zh{鸽子。}  ¶ \textcolor{darkblue}{\textbf{\ipa{tʰo˧ʐv̩˥-mi˩}}} \textcolor{Sepia}{\selectlanguage{english}female pigeon} \zh{母鸽子}  
 ¶ \textcolor{darkblue}{\textbf{\ipa{tʰo˧ʐv̩˥-pʰv̩˩}}} \textcolor{Sepia}{\selectlanguage{english}male pigeon} \zh{公鸽子}  
 ¶ \textcolor{darkblue}{\textbf{\ipa{tʰo˧ʐv̩˥-zo˩}}} \textcolor{Sepia}{\selectlanguage{english}baby pigeon} \zh{小鸽子}  
 \zh{量词}: \textcolor{darkblue}{\textbf{\ipa{mi˩}}}  \mytextsc{clf}: \textcolor{darkblue}{\textbf{\ipa{mi˩}}} 
\lhead{\firstmark}
\rhead{\botmark}

\subsection{\hspace{-0.5cm} {\Large \textcolor{darkblue}{\textbf{\ipa{tʰo˩\textsubscript{a}}}}}\hspace{0.5cm}[\kern2pt{\textcolor{darkblue}{\textbf{\ipa{tʰo˥}}}}\kern2pt]} \hypertarget{t\string_ho\string_Ba1}{}
\markboth{\textcolor{darkblue}{\textbf{\ipa{tʰo˩\textsubscript{a}}}}}{}
\textcolor{teal}{\mytextsc{verb}} \hspace{4pt} Tone: L\textsubscript{a}.
\textcolor{Sepia}{\selectlanguage{english}To lean on.} \zh{靠。}  ¶ \textcolor{darkblue}{\textbf{\ipa{tʰi˧-tʰo˩}}} \textcolor{Sepia}{\selectlanguage{english}\mytextsc{dur}} \zh{\mytextsc{dur}}  
 ¶ \textcolor{darkblue}{\textbf{\ipa{tʰi˧-tʰo˩-ɻ̍˩}}} \textcolor{Sepia}{\selectlanguage{english}\mytextsc{dur} \string_ \mytextsc{inceptive}} \zh{\mytextsc{dur} \string_ \mytextsc{inceptive}}  
 ¶ \textcolor{darkblue}{\textbf{\ipa{ɖɯ˧-tʰo˩-ɻ̍˩}}} \textcolor{Sepia}{\selectlanguage{english}\mytextsc{delimitative} \string_ \mytextsc{inceptive}} \zh{\mytextsc{delimitative} \string_ \mytextsc{inceptive}}  

\lhead{\firstmark}
\rhead{\botmark}

\subsection{\hspace{-0.5cm} {\Large \textcolor{darkblue}{\textbf{\ipa{tʰo˩lo˧}}}}\hspace{0.5cm}[\kern2pt{\textcolor{darkblue}{\textbf{\ipa{tʰo˩lo˥}}}}\kern2pt]} \hypertarget{t\string_ho\string_Blo\string_M1}{}
\markboth{\textcolor{darkblue}{\textbf{\ipa{tʰo˩lo˧}}}}{}
\textcolor{teal}{\mytextsc{noun}} \hspace{4pt} Tone: LM.
\textcolor{Sepia}{\selectlanguage{english}The horse walking in front in a caravan.} \zh{头马:马帮里走在最前面的那匹马。} 
\lhead{\firstmark}
\rhead{\botmark}

\subsection{\hspace{-0.5cm} {\Large \textcolor{darkblue}{\textbf{\ipa{tʰo˩pʰv̩˧tɕʰɤ˧}}}}\hspace{0.5cm}[\kern2pt{\textcolor{darkblue}{\textbf{\ipa{tʰo˩pʰv̩˧tɕʰɤ˧}}}}\kern2pt]} \hypertarget{t\string_ho\string_Bp\string_hv\string_=\string_Mts£\string_h7\string_M1}{}
\markboth{\textcolor{darkblue}{\textbf{\ipa{tʰo˩pʰv̩˧tɕʰɤ˧}}}}{}
\textcolor{teal}{\mytextsc{noun}} \hspace{4pt} Tone: LM.
\textcolor{Sepia}{\selectlanguage{english}Gun; firelock; rifle.} \zh{枪,明火枪。}  Borrowing: Chinese?
 \zh{量词}: \textcolor{darkblue}{\textbf{\ipa{kʰɯ˩}}}  \mytextsc{clf}: \textcolor{darkblue}{\textbf{\ipa{kʰɯ˩}}} 
\lhead{\firstmark}
\rhead{\botmark}

\subsection{\hspace{-0.5cm} {\Large \textcolor{darkblue}{\textbf{\ipa{tʰo˩ʁæ˩}}}}\hspace{0.5cm}[\kern2pt{\textcolor{darkblue}{\textbf{\ipa{tʰo˩ʁæ˩˥}}}}\kern2pt]} \hypertarget{t\string_ho\string_BR\{\string_B1}{}
\markboth{\textcolor{darkblue}{\textbf{\ipa{tʰo˩ʁæ˩}}}}{}
\textcolor{teal}{\mytextsc{noun}} \hspace{4pt} Tone: L.
\textcolor{Sepia}{\selectlanguage{english}Pine resin; colophony.} \zh{松香。}  \zh{量词}: \textcolor{darkblue}{\textbf{\ipa{ʈʰɤ˥}}}  \mytextsc{clf}: \textcolor{darkblue}{\textbf{\ipa{ʈʰɤ˥}}} 
\lhead{\firstmark}
\rhead{\botmark}

\subsection{\hspace{-0.5cm} {\Large \textcolor{darkblue}{\textbf{\ipa{tʰo˩ʂv̩˩}}}}\hspace{0.5cm}[\kern2pt{\textcolor{darkblue}{\textbf{\ipa{tʰo˩ʂv̩˩˥}}}}\kern2pt]} \hypertarget{t\string_ho\string_Bs`v\string_=\string_B1}{}
\markboth{\textcolor{darkblue}{\textbf{\ipa{tʰo˩ʂv̩˩}}}}{}
\textcolor{teal}{\mytextsc{noun}} \hspace{4pt} Tone: L.
\textcolor{Sepia}{\selectlanguage{english}Pine needles.} \zh{树针。}  \zh{量词}: \textcolor{darkblue}{\textbf{\ipa{qɑ˩}}}  \mytextsc{clf}: \textcolor{darkblue}{\textbf{\ipa{qɑ˩}}} 
\lhead{\firstmark}
\rhead{\botmark}

\subsection{\hspace{-0.5cm} {\Large \textcolor{darkblue}{\textbf{\ipa{tʰo˩tɕi˧˥}}}}\hspace{0.5cm}[\kern2pt{\textcolor{darkblue}{\textbf{\ipa{tʰo˩tɕi˧˥}}}}\kern2pt]} \hypertarget{t\string_ho\string_Bts£i\string_M\string_T1}{}
\markboth{\textcolor{darkblue}{\textbf{\ipa{tʰo˩tɕi˧˥}}}}{}
\textcolor{teal}{\mytextsc{noun}} \hspace{4pt} Tone: LM+MH\#.
\textcolor{Sepia}{\selectlanguage{english}Brick.} \zh{砖。}  \zh{量词}: \textcolor{darkblue}{\textbf{\ipa{ɭɯ˧}}}  \mytextsc{clf}: \textcolor{darkblue}{\textbf{\ipa{ɭɯ˧}}} 
\lhead{\firstmark}
\rhead{\botmark}

\subsection{\hspace{-0.5cm} {\Large \textcolor{darkblue}{\textbf{\ipa{tʰv̩˧˥}}} \textsubscript{1}}\hspace{0.5cm}[\kern2pt{\textcolor{darkblue}{\textbf{\ipa{tʰv̩˧˥}}}}\kern2pt]} \hypertarget{t\string_hv\string_=\string_M\string_T1}{}
\markboth{\textcolor{darkblue}{\textbf{\ipa{tʰv̩˧˥}}} \textsubscript{1}}{}
\textcolor{teal}{\mytextsc{verb}} \hspace{4pt} Tone: MH.
\textcolor{Sepia}{\selectlanguage{english}To step on, to tread on, to trample.} \zh{踩。}  ¶ \textcolor{darkblue}{\textbf{\ipa{tʰv̩˩\textasciitilde{}tʰv̩˧˥}}} \textcolor{Sepia}{\selectlanguage{english}\mytextsc{red}} \zh{\mytextsc{重叠}}  
 ¶ \textcolor{darkblue}{\textbf{\ipa{ɖɯ˧-tʰv̩˧ tʰi˥-tʰv̩˩}}} \textcolor{Sepia}{\selectlanguage{english}to give a kick, to stamp the ground} \zh{踢一脚}  
 ¶ \textcolor{darkblue}{\textbf{\ipa{kʰɯ˧tsʰɤ˧ tʰv̩˥-tsʰɯ˩}}} \textcolor{Sepia}{\selectlanguage{english}to give a kick, to stamp the ground} \zh{踢一脚}  
 ¶ \textcolor{darkblue}{\textbf{\ipa{kʰɯ˧tsʰɤ˧ tʰv̩˥\textasciitilde{}tʰv̩˩}}} \textcolor{Sepia}{\selectlanguage{english}to give a kick, to stamp the ground} \zh{踢一脚}  
 ¶ \textcolor{darkblue}{\textbf{\ipa{kʰɯ˧tsʰɤ˧ tʰɑ˧-tʰv̩˧˥!}}} \textcolor{Sepia}{\selectlanguage{english}Do not stamp the ground! / Do not kick/tread on something!} \zh{别踢!}  

\lhead{\firstmark}
\rhead{\botmark}

\subsection{\hspace{-0.5cm} {\Large \textcolor{darkblue}{\textbf{\ipa{tʰv̩˧˥}}} \textsubscript{2}}\hspace{0.5cm}[\kern2pt{\textcolor{darkblue}{\textbf{\ipa{tʰv̩˧˥}}}}\kern2pt]} \hypertarget{t\string_hv\string_=\string_M\string_T2}{}
\markboth{\textcolor{darkblue}{\textbf{\ipa{tʰv̩˧˥}}} \textsubscript{2}}{}
\textcolor{teal}{\mytextsc{verb}} \hspace{4pt} Tone: MH.
\textcolor{Sepia}{\selectlanguage{english}To take charge of, to foot the bill (e.g. someone invites the whole village to a feast; that person provides the food, but does not necessarily do the cooking).} \zh{负担(某个活动的费用,如:请全村人吃饭)。} 
\lhead{\firstmark}
\rhead{\botmark}

\subsection{\hspace{-0.5cm} {\Large \textcolor{darkblue}{\textbf{\ipa{tʰv̩˥}}} \textsubscript{1}}\hspace{0.5cm}[\kern2pt{\textcolor{darkblue}{\textbf{\ipa{tʰv̩˧˥}}}}\kern2pt]} \hypertarget{t\string_hv\string_=\string_T1}{}
\markboth{\textcolor{darkblue}{\textbf{\ipa{tʰv̩˥}}} \textsubscript{1}}{}
\textcolor{teal}{\mytextsc{pronoun/pronominal}} \hspace{4pt} Tone: \#H.
\textcolor{Sepia}{\selectlanguage{english}That; distal demonstrative.} \zh{那\mytextsc{指示}.远指。}  ¶ \textcolor{darkblue}{\textbf{\ipa{tʰv̩˧ ɲi˥!}}} \textcolor{Sepia}{\selectlanguage{english}It's that one! / That's the one!} \zh{是那个!}  
 ¶ \textcolor{darkblue}{\textbf{\ipa{tʰv̩˧-v̩\#˥}}} \textcolor{Sepia}{\selectlanguage{english}that one} \zh{那个}  
\textit{See:} \hyperlink{}{\textcolor{darkblue}{\textbf{\ipa{tʰv̩˥}}} \textsubscript{2}} \textit{See:} \hyperlink{}{\textcolor{darkblue}{\textbf{\ipa{tʰv̩˥}}} \textsubscript{3}} 
\lhead{\firstmark}
\rhead{\botmark}

\subsection{\hspace{-0.5cm} {\Large \textcolor{darkblue}{\textbf{\ipa{tʰv̩˥}}} \textsubscript{2}}\hspace{0.5cm}[\kern2pt{\textcolor{darkblue}{\textbf{\ipa{tʰv̩˥}}}}\kern2pt]} \hypertarget{t\string_hv\string_=\string_T2}{}
\markboth{\textcolor{darkblue}{\textbf{\ipa{tʰv̩˥}}} \textsubscript{2}}{}
\textcolor{teal}{\mytextsc{pronoun/pronominal}} \hspace{4pt} Tone: \#H.
\textcolor{Sepia}{\selectlanguage{english}3rd person singular.} \zh{他。}  ¶ \textcolor{darkblue}{\textbf{\ipa{tʰv̩˧=ɻ̍˩}}} \textcolor{Sepia}{\selectlanguage{english}his family, his household, his clan, his kin} \zh{他家、他家族、他的人}  
\textit{See:} \hyperlink{}{\textcolor{darkblue}{\textbf{\ipa{tʰv̩˥}}} \textsubscript{1}} \textit{See:} \hyperlink{}{\textcolor{darkblue}{\textbf{\ipa{tʰv̩˥}}} \textsubscript{3}} 
\lhead{\firstmark}
\rhead{\botmark}

\subsection{\hspace{-0.5cm} {\Large \textcolor{darkblue}{\textbf{\ipa{tʰv̩˥}}} \textsubscript{3}}\hspace{0.5cm}[\kern2pt{\textcolor{darkblue}{\textbf{\ipa{tʰv̩˥}}}}\kern2pt]} \hypertarget{t\string_hv\string_=\string_T3}{}
\markboth{\textcolor{darkblue}{\textbf{\ipa{tʰv̩˥}}} \textsubscript{3}}{}
\textcolor{teal}{\mytextsc{suffix}} \hspace{4pt} Tone: \#H.
\textcolor{Sepia}{\selectlanguage{english}Topic marker; grammaticalized from the distal demonstrative.} \zh{\mytextsc{主题(°指示}.远指)。} \textit{See:} \hyperlink{}{\textcolor{darkblue}{\textbf{\ipa{tʰv̩˥}}} \textsubscript{1}} \textit{See:} \hyperlink{}{\textcolor{darkblue}{\textbf{\ipa{tʰv̩˥}}} \textsubscript{2}} 
\lhead{\firstmark}
\rhead{\botmark}

\subsection{\hspace{-0.5cm} {\Large \textcolor{darkblue}{\textbf{\ipa{tʰv̩˧\textsubscript{a}}}}}\hspace{0.5cm}[\kern2pt{\textcolor{darkblue}{\textbf{\ipa{tʰv̩˥}}}}\kern2pt]} \hypertarget{t\string_hv\string_=\string_Ma1}{}
\markboth{\textcolor{darkblue}{\textbf{\ipa{tʰv̩˧\textsubscript{a}}}}}{}
\textcolor{teal}{\mytextsc{verb}} \hspace{4pt} Tone: M\textsubscript{a}.
\ding{202} \textcolor{Sepia}{\selectlanguage{english}To come out.} \zh{出来。}  ¶ \textcolor{darkblue}{\textbf{\ipa{ɑ˩pʰo˩ tʰv̩˩˥}}} \textcolor{Sepia}{\selectlanguage{english}to come out: e.g. an animal comes out of its burrow} \zh{出来,如:动物从地洞里爬出来}  
 ¶ \textcolor{darkblue}{\textbf{\ipa{ɲi˧mi˧ tʰv̩˧}}} \textcolor{Sepia}{\selectlanguage{english}the sun comes out} \zh{太阳出来}  
\ding{203} \textcolor{Sepia}{\selectlanguage{english}To rise (wind).} \zh{刮(风)。} \ding{204} \textcolor{Sepia}{\selectlanguage{english}To bud, to sprout (a tree sprouts).} \zh{发芽、抽芽。}  ¶ \textcolor{darkblue}{\textbf{\ipa{si˧dzi˩ | ʁo˧bv̩˧ tʰv̩˧}}} \textcolor{Sepia}{\selectlanguage{english}the tree buds} \zh{树抽芽}  
\ding{205} \textcolor{Sepia}{\selectlanguage{english}To appear, to happen, to get (a wound).} \zh{出现。}  ¶ \textcolor{darkblue}{\textbf{\ipa{mi˧ tʰv̩˧}}} \textcolor{Sepia}{\selectlanguage{english}to get wounded} \zh{受伤}  
 ¶ \textcolor{darkblue}{\textbf{\ipa{ɖɯ˧-v̩˧ mi˧ tʰv̩˧-ze˧!}}} \textcolor{Sepia}{\selectlanguage{english}someone has got wounded!} \zh{有人受伤了!}  
\ding{206} \textcolor{Sepia}{\selectlanguage{english}To create; to found.} \zh{建立、创造、制造出来。}  ¶ \textcolor{darkblue}{\textbf{\ipa{ʑi˩ tʰv̩˩}}} \textcolor{Sepia}{\selectlanguage{english}to found a new home} \zh{分家、建立新家}  
 ¶ \textcolor{darkblue}{\textbf{\ipa{ʈʂʰɯ˧ | ʑi˩ tʰv̩˩-ze˥!}}} \textcolor{Sepia}{\selectlanguage{english}(S)he founded a new home!} \zh{他建了新家!}  
 ¶ \textcolor{darkblue}{\textbf{\ipa{ʈʂʰɯ˧ | ʑi˩ tʰv̩˩-bi˩˥!}}} \textcolor{Sepia}{\selectlanguage{english}(S)he is going to found a new home!} \zh{他要建个新家!}  
 ¶ \textcolor{darkblue}{\textbf{\ipa{lo˧ mɤ˧-dʑo˧, | lo˧ tʰv̩˧˥! / no˧ | lo˧ mɤ˧-dʑo˧, | lo˧ tʰv̩˧-ɲi˥!}}} \textcolor{Sepia}{\selectlanguage{english}[(S)he] has no obligations, and yet (s)he works a lot / (s)he finds tasks to do! (A compliment to a civil servant who could be content to pocket a salary every month but who sets goals for her/himself and looks for useful tasks to accomplish. The sentence can also be used negatively, to criticize someone who takes up unnecessary tasks instead of keeping quiet.)} \zh{自找麻烦!(这句,除贬义用法,还能用来表扬,如表扬一位当官的人努力去做好事,给自己找有意义的事情干。)}  

\lhead{\firstmark}
\rhead{\botmark}

\subsection{\hspace{-0.5cm} {\Large \textcolor{darkblue}{\textbf{\ipa{tʰv̩˧˥\textsubscript{a}}}}}\hspace{0.5cm}[\kern2pt{\textcolor{darkblue}{\textbf{\ipa{tʰv̩˥}}}}\kern2pt]} \hypertarget{t\string_hv\string_=\string_M\string_Ta1}{}
\markboth{\textcolor{darkblue}{\textbf{\ipa{tʰv̩˧˥\textsubscript{a}}}}}{}
\textcolor{teal}{\mytextsc{classifier}} \hspace{4pt} Tone: MH\textsubscript{a}.
\textcolor{Sepia}{\selectlanguage{english}Classifier for steps (in walking).} \zh{量词:步。}  ¶ \textcolor{darkblue}{\textbf{\ipa{ɖɯ˧-tʰv̩˧\textasciitilde{}ɖɯ˥-tʰv̩˩}}} \textcolor{Sepia}{\selectlanguage{english}step by step, one step after the other} \zh{一步一步}  
 ¶ \textcolor{darkblue}{\textbf{\ipa{ɖɯ˧-tʰv̩˧˥, | ɖɯ˧-tʰv̩˧˥}}} \textcolor{Sepia}{\selectlanguage{english}step by step, one step after the other; same as above, but detaching the two parts of the phrase; this is closer to repetition than to reduplication} \zh{一步又一步}  

\lhead{\firstmark}
\rhead{\botmark}

\subsection{\hspace{-0.5cm} {\Large \textcolor{darkblue}{\textbf{\ipa{tʰv̩˩\textsubscript{b}}}}}\hspace{0.5cm}[\kern2pt{\textcolor{darkblue}{\textbf{\ipa{tʰv̩˩˥}}}}\kern2pt]} \hypertarget{t\string_hv\string_=\string_Bb1}{}
\markboth{\textcolor{darkblue}{\textbf{\ipa{tʰv̩˩\textsubscript{b}}}}}{}
\textcolor{teal}{\mytextsc{classifier}} \hspace{4pt} Tone: L\textsubscript{b}.
\textcolor{Sepia}{\selectlanguage{english}Classifier for sets of tens. The term used to refer to sets of eight. The word remains in use, but its meaning has shifted towards the meaning of 'sets of ten', following the generalized use of decimal numeration.} \zh{量词:一套(有十个)。更早的意思是八个。}  ¶ \textcolor{darkblue}{\textbf{\ipa{qʰwɤ˩˥ | ɖɯ˧-tʰv̩˩}}} \textcolor{Sepia}{\selectlanguage{english}a set of ten bowls} \zh{一套十个碗}  
 ¶ \textcolor{darkblue}{\textbf{\ipa{ɖʐɯ˧ʂɯ˥ | ɖɯ˧-tʰv̩˩}}} \textcolor{Sepia}{\selectlanguage{english}a set of ten (pairs of) chopsticks} \zh{一套十(双)筷子}  

\lhead{\firstmark}
\rhead{\botmark}

\subsection{\hspace{-0.5cm} {\Large \textcolor{darkblue}{\textbf{\ipa{tʰv̩˧\textsubscript{b}}}}}\hspace{0.5cm}[\kern2pt{\textcolor{darkblue}{\textbf{\ipa{tʰv̩˩˥}}}}\kern2pt]} \hypertarget{t\string_hv\string_=\string_Mb1}{}
\markboth{\textcolor{darkblue}{\textbf{\ipa{tʰv̩˧\textsubscript{b}}}}}{}
\textcolor{teal}{\mytextsc{verb}} \hspace{4pt} Tone: M\textsubscript{b}.
\textcolor{Sepia}{\selectlanguage{english}To lend.} \zh{借给人。}  ¶ \textcolor{darkblue}{\textbf{\ipa{tso˧\textasciitilde{}tso˧ tʰv̩˧}}} \textcolor{Sepia}{\selectlanguage{english}to lend something} \zh{借东西(给人)}  

\lhead{\firstmark}
\rhead{\botmark}

\subsection{\hspace{-0.5cm} {\Large \textcolor{darkblue}{\textbf{\ipa{tʰv̩˧gi˧}}}}\hspace{0.5cm}[\kern2pt{\textcolor{darkblue}{\textbf{\ipa{tʰv̩˧gi˩}}}}\kern2pt]} \hypertarget{t\string_hv\string_=\string_Mgi\string_M1}{}
\markboth{\textcolor{darkblue}{\textbf{\ipa{tʰv̩˧gi˧}}}}{}
\textcolor{teal}{\mytextsc{adverb(ial)}} \hspace{4pt} Tone: .
\textcolor{Sepia}{\selectlanguage{english}In that direction.} \zh{那边。} 
\lhead{\firstmark}
\rhead{\botmark}

\subsection{\hspace{-0.5cm} {\Large \textcolor{darkblue}{\textbf{\ipa{tʰv̩˧ne-ʝi˥}}}}\hspace{0.5cm}[\kern2pt{\textcolor{darkblue}{\textbf{\ipa{xxxx non-correspondance entre le nombre de morphèmes et le nombre de tons de morphèmes}}}}\kern2pt]} \hypertarget{t\string_hv\string_=\string_Mne-j££i\string_T1}{}
\markboth{\textcolor{darkblue}{\textbf{\ipa{tʰv̩˧ne-ʝi˥}}}}{}
\textcolor{teal}{\mytextsc{adverb(ial)}} \hspace{4pt} Tone: MH\#.
\textcolor{Sepia}{\selectlanguage{english}In that way.} \zh{那样。} 
\lhead{\firstmark}
\rhead{\botmark}

\subsection{\hspace{-0.5cm} {\Large \textcolor{darkblue}{\textbf{\ipa{tʰv̩˧ɲi\#˥}}}}\hspace{0.5cm}[\kern2pt{\textcolor{darkblue}{\textbf{\ipa{tʰv̩˧ɲi˧}}}}\kern2pt]} \hypertarget{t\string_hv\string_=\string_MJi\#\string_T1}{}
\markboth{\textcolor{darkblue}{\textbf{\ipa{tʰv̩˧ɲi\#˥}}}}{}
\textcolor{teal}{\mytextsc{adverb(ial)}} \hspace{4pt} Tone: \#H.
\textcolor{Sepia}{\selectlanguage{english}That day.} \zh{那天。} 
\lhead{\firstmark}
\rhead{\botmark}

\subsection{\hspace{-0.5cm} {\Large \textcolor{darkblue}{\textbf{\ipa{tʰv̩˧qo˧}}}}\hspace{0.5cm}[\kern2pt{\textcolor{darkblue}{\textbf{\ipa{tʰv̩˧qo˧}}}}\kern2pt]} \hypertarget{t\string_hv\string_=\string_Mqo\string_M1}{}
\markboth{\textcolor{darkblue}{\textbf{\ipa{tʰv̩˧qo˧}}}}{}
\textcolor{teal}{\mytextsc{pronoun/pronominal}} \hspace{4pt} Tone: M.
\textcolor{Sepia}{\selectlanguage{english}There; that place.} \zh{那里、那个地方。} 
\lhead{\firstmark}
\rhead{\botmark}

\subsection{\hspace{-0.5cm} {\Large \textcolor{darkblue}{\textbf{\ipa{tʰv̩˧-si˥}}}}\hspace{0.5cm}[\kern2pt{\textcolor{darkblue}{\textbf{\ipa{xxxx non-correspondance entre le nombre de morphèmes et le nombre de tons de morphèmes}}}}\kern2pt]} \hypertarget{t\string_hv\string_=\string_M-si\string_T1}{}
\markboth{\textcolor{darkblue}{\textbf{\ipa{tʰv̩˧-si˥}}}}{}
\textcolor{teal}{\mytextsc{adverb(ial)}} \hspace{4pt} Tone: H\#.
\textcolor{Sepia}{\selectlanguage{english}Numerous.} \zh{多。}  ¶ \textcolor{darkblue}{\textbf{\ipa{mv̩˧ʁo˧=ɻ̍˥-dʑo˩, | ɻæ˩˥ | tʰv̩˧-si˥ | tʰv̩˩-jɤ˩ dʑo˩˥!}}} \textcolor{Sepia}{\selectlanguage{english}The People of the Sky had seeds in profusion! (from the narrative “Seeds”)} \zh{天上的人,有许多许多种子!(故事:“种子”)}  

\lhead{\firstmark}
\rhead{\botmark}

\subsection{\hspace{-0.5cm} {\Large \textcolor{darkblue}{\textbf{\ipa{‑tʰv̩˧}}} \textsubscript{1}}\hspace{0.5cm}[\kern2pt{\textcolor{darkblue}{\textbf{\ipa{tʰv̩˥}}}}\kern2pt]} \hypertarget{‑t\string_hv\string_=\string_M1}{}
\markboth{\textcolor{darkblue}{\textbf{\ipa{‑tʰv̩˧}}} \textsubscript{1}}{}
\textcolor{teal}{\mytextsc{postposition}} \hspace{4pt} Tone: M.
\textcolor{Sepia}{\selectlanguage{english}Temporal postposition: up to, up until.} \zh{到……为止。} 
\lhead{\firstmark}
\rhead{\botmark}

\subsection{\hspace{-0.5cm} {\Large \textcolor{darkblue}{\textbf{\ipa{‑tʰv̩˧}}} \textsubscript{2}}\hspace{0.5cm}[\kern2pt{\textcolor{darkblue}{\textbf{\ipa{tʰv̩˥}}}}\kern2pt]} \hypertarget{‑t\string_hv\string_=\string_M2}{}
\markboth{\textcolor{darkblue}{\textbf{\ipa{‑tʰv̩˧}}} \textsubscript{2}}{}
\textcolor{teal}{\mytextsc{suffix}} \hspace{4pt} Tone: M.
\textcolor{Sepia}{\selectlanguage{english}To achieve, to attain (a goal), to complete successfully (an action); grammaticalized from the verb 'to come out'.} \zh{……成。}  ¶ \textcolor{darkblue}{\textbf{\ipa{lo˧ ʝi˧-mɤ˧-tʰv̩˧}}} \textcolor{Sepia}{\selectlanguage{english}not to be able to complete one's task, to be unable to do one's work fully (example: someone is constantly being disturbed, and consequently can't achieve what they wanted to/can't work in a focused way)} \zh{活做不出来、活做不成(比如:一个人经常被打扰,所以不能集中工作,没有效率,要做的事做不成)}  

\lhead{\firstmark}
\rhead{\botmark}

\newpage
\section*{\centering- \textcolor{darkblue}{\textbf{\ipa{tɕ}}} -}
\subsection{\hspace{-0.5cm} {\Large \textcolor{darkblue}{\textbf{\ipa{tɕæ˧hæ˩}}}}\hspace{0.5cm}[\kern2pt{\textcolor{darkblue}{\textbf{\ipa{tɕæ˧hæ˩}}}}\kern2pt]} \hypertarget{ts£\{\string_Mh\{\string_B1}{}
\markboth{\textcolor{darkblue}{\textbf{\ipa{tɕæ˧hæ˩}}}}{}
\textcolor{teal}{\mytextsc{noun}} \hspace{4pt} Tone: L\#.
\textcolor{Sepia}{\selectlanguage{english}Rubber.} \zh{橡胶(汉语借词。第二个音节:未确定。)。}  Borrowing: Chinese  \zh{胶} +?
 ¶ \textcolor{darkblue}{\textbf{\ipa{tɕæ˧hæ˩-dzɑ˩qʰwɤ˩}}} \textcolor{Sepia}{\selectlanguage{english}rubber shoe, shoe with a rubber sole, sports shoe} \zh{橡胶鞋、橡胶底鞋}  
 \zh{量词}: \textcolor{darkblue}{\textbf{\ipa{dzi˧}}}  \mytextsc{clf}: \textcolor{darkblue}{\textbf{\ipa{dzi˧}}} 
\lhead{\firstmark}
\rhead{\botmark}

\subsection{\hspace{-0.5cm} {\Large \textcolor{darkblue}{\textbf{\ipa{tɕæ˧pʰv̩˩}}}}\hspace{0.5cm}[\kern2pt{\textcolor{darkblue}{\textbf{\ipa{tɕæ˧pʰv̩˩}}}}\kern2pt]} \hypertarget{ts£\{\string_Mp\string_hv\string_=\string_B1}{}
\markboth{\textcolor{darkblue}{\textbf{\ipa{tɕæ˧pʰv̩˩}}}}{}
\textcolor{teal}{\mytextsc{adjective}} \hspace{4pt} Tone: L\#.
\textcolor{Sepia}{\selectlanguage{english}White.} \zh{白(脸、衣服)。}  ¶ \textcolor{darkblue}{\textbf{\ipa{tɕæ˧pʰv̩˩-bɑ˩lɑ˩}}} \textcolor{Sepia}{\selectlanguage{english}white clothes} \zh{白的衣服}  
 ¶ \textcolor{darkblue}{\textbf{\ipa{tɕæ˧pʰv̩˩-ʈæ˩qʰwɤ˩}}} \textcolor{Sepia}{\selectlanguage{english}white skirt} \zh{白色裙子}  

\lhead{\firstmark}
\rhead{\botmark}

\subsection{\hspace{-0.5cm} {\Large \textcolor{darkblue}{\textbf{\ipa{tɕæ˧ɻæ˩}}}}\hspace{0.5cm}[\kern2pt{\textcolor{darkblue}{\textbf{\ipa{tɕæ˧ɻæ˩}}}}\kern2pt]} \hypertarget{ts£\{\string_Mr£`\{\string_B1}{}
\markboth{\textcolor{darkblue}{\textbf{\ipa{tɕæ˧ɻæ˩}}}}{}
\textcolor{teal}{\mytextsc{noun}} \hspace{4pt} Tone: L\#.
\textcolor{Sepia}{\selectlanguage{english}Pickled vegetables.} \zh{酸菜、泡菜。}  ¶ \textcolor{darkblue}{\textbf{\ipa{wo˩-tɕæ˩ɻæ˥}}} \textcolor{Sepia}{\selectlanguage{english}pickled turnip leaves} \zh{圆根叶子酸菜}  
 ¶ \textcolor{darkblue}{\textbf{\ipa{tsʰɑ˧-tɕæ˧ɻæ˥}}}  
 ¶ \textcolor{darkblue}{\textbf{\ipa{ɬi˩bi˩-tɕæ˩ɻæ˥}}} \textcolor{Sepia}{\selectlanguage{english}pickled turnip} \zh{圆根酸菜}  
 ¶ \textcolor{darkblue}{\textbf{\ipa{pɤ˧pɤ˧tsʰɯ˧-tɕæ˧ɻæ˥}}} \textcolor{Sepia}{\selectlanguage{english}picked Chinese cabbage} \zh{圆白菜酸菜}  

\lhead{\firstmark}
\rhead{\botmark}

\subsection{\hspace{-0.5cm} {\Large \textcolor{darkblue}{\textbf{\ipa{tɕɤ}}}}\hspace{0.5cm}[\kern2pt{\textcolor{darkblue}{\textbf{\ipa{[]}}}}\kern2pt]} \hypertarget{ts£71}{}
\markboth{\textcolor{darkblue}{\textbf{\ipa{tɕɤ}}}}{}
\textcolor{teal}{\mytextsc{interjection}} \hspace{4pt} Tone: 0.
\textcolor{Sepia}{\selectlanguage{english}Interjection: hey!} \zh{感叹词:嘿!。} 
\lhead{\firstmark}
\rhead{\botmark}

\subsection{\hspace{-0.5cm} {\Large \textcolor{darkblue}{\textbf{\ipa{tɕɤ˥}}}}\hspace{0.5cm}[\kern2pt{\textcolor{darkblue}{\textbf{\ipa{tɕɤ˥}}}}\kern2pt]} \hypertarget{ts£7\string_T1}{}
\markboth{\textcolor{darkblue}{\textbf{\ipa{tɕɤ˥}}}}{}
\textcolor{teal}{\mytextsc{verb}} \hspace{4pt} Tone: H.
\textcolor{Sepia}{\selectlanguage{english}To fade (of colours).} \zh{褪色。}  ¶ \textcolor{darkblue}{\textbf{\ipa{le˧-tɕɤ˥-ze˩}}} \textcolor{Sepia}{\selectlanguage{english}\mytextsc{accomp} \string_ \mytextsc{pfv}} \zh{褪色了}  

\lhead{\firstmark}
\rhead{\botmark}

\subsection{\hspace{-0.5cm} {\Large \textcolor{darkblue}{\textbf{\ipa{tɕɤ˧fv̩˩}}}}\hspace{0.5cm}[\kern2pt{\textcolor{darkblue}{\textbf{\ipa{tɕɤ˧fv̩˩}}}}\kern2pt]} \hypertarget{ts£7\string_Mfv\string_=\string_B1}{}
\markboth{\textcolor{darkblue}{\textbf{\ipa{tɕɤ˧fv̩˩}}}}{}
\textcolor{teal}{\mytextsc{noun}} \hspace{4pt} Tone: L\#.
\textcolor{Sepia}{\selectlanguage{english}Container for liquids, such as plastic jerricans; used to store and transport drinking water.} \zh{塑料桶等存水用的容器。}  \zh{量词}: \textcolor{darkblue}{\textbf{\ipa{ɭɯ˧}}}  \mytextsc{clf}: \textcolor{darkblue}{\textbf{\ipa{ɭɯ˧}}} 
\lhead{\firstmark}
\rhead{\botmark}

\subsection{\hspace{-0.5cm} {\Large \textcolor{darkblue}{\textbf{\ipa{tɕɤ˧ho˩pæ˧}}}}\hspace{0.5cm}[\kern2pt{\textcolor{darkblue}{\textbf{\ipa{xxxx ton non trouvé, à faire manuellement...}}}}\kern2pt]} \hypertarget{ts£7\string_Mho\string_Bp\{\string_M1}{}
\markboth{\textcolor{darkblue}{\textbf{\ipa{tɕɤ˧ho˩pæ˧}}}}{}
\textcolor{teal}{\mytextsc{noun}} \hspace{4pt} Tone: MLM.
\textcolor{Sepia}{\selectlanguage{english}Plywood, veneer board.} \zh{胶合板(汉语借词)。}  Borrowing: Chinese  \zh{胶合板}

\lhead{\firstmark}
\rhead{\botmark}

\subsection{\hspace{-0.5cm} {\Large \textcolor{darkblue}{\textbf{\ipa{tɕɤ˧qʰɑ\#˥}}}}\hspace{0.5cm}[\kern2pt{\textcolor{darkblue}{\textbf{\ipa{tɕɤ˧qʰɑ˧}}}}\kern2pt]} \hypertarget{ts£7\string_Mq\string_hA\#\string_T1}{}
\markboth{\textcolor{darkblue}{\textbf{\ipa{tɕɤ˧qʰɑ\#˥}}}}{}
\textcolor{teal}{\mytextsc{noun}} \hspace{4pt} Tone: \#H.
\textcolor{Sepia}{\selectlanguage{english}Mugwort, wormwood, \textit{Artemisia vulgaris}.} \zh{蒿、青蒿。} Local Chinese dialect:\zh{蒿草、蒿枝。} ¶ \textcolor{darkblue}{\textbf{\ipa{tɕɤ˧qʰɑ˧-mo˩}}} \textcolor{Sepia}{\selectlanguage{english}a type of edible mushroom, called 'mugwort mushroom' because it grows close to mugwort} \zh{一种可以吃的菌子,长在蒿附近}  
\textit{See:} \hyperlink{}{\textcolor{darkblue}{\textbf{\ipa{ho˧ʈʂɯ˧}}}} 
\lhead{\firstmark}
\rhead{\botmark}

\subsection{\hspace{-0.5cm} {\Large \textcolor{darkblue}{\textbf{\ipa{tɕɤ˧tɑ˧}}}}\hspace{0.5cm}[\kern2pt{\textcolor{darkblue}{\textbf{\ipa{tɕɤ˧tɑ˧}}}}\kern2pt]} \hypertarget{ts£7\string_MtA\string_M1}{}
\markboth{\textcolor{darkblue}{\textbf{\ipa{tɕɤ˧tɑ˧}}}}{}
\textcolor{teal}{\mytextsc{noun}} \hspace{4pt} Tone: M.
\textcolor{Sepia}{\selectlanguage{english}Yoke.} \zh{牛轭(单行)(汉语借词)。} Local Chinese dialect:\zh{牛夹担、牛枷档、牛拴。} Borrowing: Chinese  \zh{夹担}
 ¶ \textcolor{darkblue}{\textbf{\ipa{tɕɤ˧tɑ˧ tʰv̩˧-ɭɯ˧}}} \textcolor{Sepia}{\selectlanguage{english}\mytextsc{n}+\mytextsc{dem}+\mytextsc{clf}} \zh{这个牛轭}  
 \zh{量词}: \textcolor{darkblue}{\textbf{\ipa{ɭɯ˧}}}  \mytextsc{clf}: \textcolor{darkblue}{\textbf{\ipa{ɭɯ˧}}} 
\lhead{\firstmark}
\rhead{\botmark}

\subsection{\hspace{-0.5cm} {\Large \textcolor{darkblue}{\textbf{\ipa{tɕɤ˧\textasciitilde{}tɕɤ˧}}}}\hspace{0.5cm}[\kern2pt{\textcolor{darkblue}{\textbf{\ipa{tɕɤ˧tɕɤ˧}}}}\kern2pt]} \hypertarget{ts£7\string_M~ts£7\string_M1}{}
\markboth{\textcolor{darkblue}{\textbf{\ipa{tɕɤ˧\textasciitilde{}tɕɤ˧}}}}{}
\textcolor{teal}{\mytextsc{adverb(ial)}} \hspace{4pt} Tone: M.
\textcolor{Sepia}{\selectlanguage{english}Just; exactly.} \zh{将将(汉语借词)、刚刚。} Local Chinese dialect:\zh{将将。} Borrowing: Chinese  \zh{将将}

\lhead{\firstmark}
\rhead{\botmark}

\subsection{\hspace{-0.5cm} {\Large \textcolor{darkblue}{\textbf{\ipa{tɕɤ˩}}}}\hspace{0.5cm}[\kern2pt{\textcolor{darkblue}{\textbf{\ipa{xxxx groupe tonal entier sans aucun ton}}}}\kern2pt]} \hypertarget{ts£7\string_B1}{}
\markboth{\textcolor{darkblue}{\textbf{\ipa{tɕɤ˩}}}}{}
\textcolor{teal}{\mytextsc{verb}} \hspace{4pt} Tone: .
\textcolor{Sepia}{\selectlanguage{english}To bind together.} \zh{打结、系上。}  Borrowing: Chinese  \zh{架?}
 ¶ \textcolor{darkblue}{\textbf{\ipa{ʁæ˧ɻ̍˥ | tʰi˧-tɕɤ˩}}} \textcolor{Sepia}{\selectlanguage{english}to attach the yoke (to a buffalo)} \zh{系上牛轭}  

\lhead{\firstmark}
\rhead{\botmark}

\subsection{\hspace{-0.5cm} {\Large \textcolor{darkblue}{\textbf{\ipa{tɕɤ˩ho˩tsɯ˥}}}}\hspace{0.5cm}[\kern2pt{\textcolor{darkblue}{\textbf{\ipa{tɕɤ˩ho˩tsɯ˥}}}}\kern2pt]} \hypertarget{ts£7\string_Bho\string_BtsM\string_T1}{}
\markboth{\textcolor{darkblue}{\textbf{\ipa{tɕɤ˩ho˩tsɯ˥}}}}{}
\textcolor{teal}{\mytextsc{noun}} \hspace{4pt} Tone: L+H\#.
\textcolor{Sepia}{\selectlanguage{english}Swindler, cheat.} \zh{骗子。}  ¶ \textcolor{darkblue}{\textbf{\ipa{ʈʂʰɯ˧ | hĩ˧ ʈʂʰɯ˧-v̩˧ | tɕɤ˩ho˩tsɯ˥ ɲi˩.}}} \textcolor{Sepia}{\selectlanguage{english}This man is a swindler!} \zh{这个人是骗子!}  
 \zh{量词}: \textcolor{darkblue}{\textbf{\ipa{v̩˧}}}  \mytextsc{clf}: \textcolor{darkblue}{\textbf{\ipa{v̩˧}}} 
\lhead{\firstmark}
\rhead{\botmark}

\subsection{\hspace{-0.5cm} {\Large \textcolor{darkblue}{\textbf{\ipa{tɕɤ˧˥}}}}\hspace{0.5cm}[\kern2pt{\textcolor{darkblue}{\textbf{\ipa{tɕɤ˧˥}}}}\kern2pt]} \hypertarget{ts£7\string_M\string_T1}{}
\markboth{\textcolor{darkblue}{\textbf{\ipa{tɕɤ˧˥}}}}{}
\textcolor{teal}{\mytextsc{verb}} \hspace{4pt} Tone: MH.
\textcolor{Sepia}{\selectlanguage{english}To boil, to cook thoroughly; to cook in a pot.} \zh{煮。}  ¶ \textcolor{darkblue}{\textbf{\ipa{ʂe˧ tɕɤ˩}}} \textcolor{Sepia}{\selectlanguage{english}to boil meat} \zh{煮肉}  
 ¶ \textcolor{darkblue}{\textbf{\ipa{bo˩-hɑ˧ tɕɤ˩}}} \textcolor{Sepia}{\selectlanguage{english}to boil pigswill, to cook pigswill} \zh{煮猪食}  
 ¶ \textcolor{darkblue}{\textbf{\ipa{ho˧ tɕɤ˩}}} \textcolor{Sepia}{\selectlanguage{english}to cook stew} \zh{煮粥}  
 ¶ \textcolor{darkblue}{\textbf{\ipa{dʑɯ˩ʁo˩˥, | mo˧-no˥, | mo˧ tɕɤ˥-hĩ˩ lɑ˩-ɲi˩-mæ˩! |}}} \textcolor{Sepia}{\selectlanguage{english}Up on the mountain, to cook mushrooms, (we) simply cook them in a pot! (This does not refer to boiling in the sense of 'cooking in hot water': the mushrooms are put in a pot; one adds grease and salt, and the mushrooms cook in their own juice.)} \zh{在山上,菌子,就是简单煮一下而已!(放在锅里,加油、加盐。用菌子自身的水分)}  

\lhead{\firstmark}
\rhead{\botmark}

\subsection{\hspace{-0.5cm} {\Large \textcolor{darkblue}{\textbf{\ipa{tɕi˥}}}}\hspace{0.5cm}[\kern2pt{\textcolor{darkblue}{\textbf{\ipa{tɕi˥}}}}\kern2pt]} \hypertarget{ts£i\string_T1}{}
\markboth{\textcolor{darkblue}{\textbf{\ipa{tɕi˥}}}}{}
\textcolor{teal}{\mytextsc{verb}} \hspace{4pt} Tone: H.
\textcolor{Sepia}{\selectlanguage{english}To shake (e.g. clothes after washing; to shake one's head).} \zh{抖、抖动,摇动。}  ¶ \textcolor{darkblue}{\textbf{\ipa{le˧-tɕi˧\textasciitilde{}tɕi˧-ze˩}}} \textcolor{Sepia}{\selectlanguage{english}\mytextsc{accomp} \string_ \mytextsc{pfv}} \zh{\mytextsc{accomp} \string_ \mytextsc{pfv}}  
 ¶ \textcolor{darkblue}{\textbf{\ipa{tʰi˧-tɕi˧\textasciitilde{}tɕi˧+ze˩}}} \textcolor{Sepia}{\selectlanguage{english}\mytextsc{dur} \string_ \mytextsc{pfv}} \zh{\mytextsc{dur} \string_ \mytextsc{pfv}}  
 ¶ \textcolor{darkblue}{\textbf{\ipa{ʁo˧qʰwɤ˩ tɕi˩\textasciitilde{}tɕi˩}}} \textcolor{Sepia}{\selectlanguage{english}to shake one's head} \zh{摇头}  
 ¶ \textcolor{darkblue}{\textbf{\ipa{ɖɯ˧-tɕi˧\textasciitilde{}tɕi˧-ɻ̍˥}}} \textcolor{Sepia}{\selectlanguage{english}\mytextsc{demilitative} \mytextsc{red} \mytextsc{inceptive}} \zh{摇一摇}  

\lhead{\firstmark}
\rhead{\botmark}

\subsection{\hspace{-0.5cm} {\Large \textcolor{darkblue}{\textbf{\ipa{tɕi˧}}} \textsubscript{1}}\hspace{0.5cm}[\kern2pt{\textcolor{darkblue}{\textbf{\ipa{tɕi˥}}}}\kern2pt]} \hypertarget{ts£i\string_M1}{}
\markboth{\textcolor{darkblue}{\textbf{\ipa{tɕi˧}}} \textsubscript{1}}{}
\textcolor{teal}{\mytextsc{adjective}} \hspace{4pt} Tone: M.
\ding{202} \textcolor{Sepia}{\selectlanguage{english}Acid.} \zh{酸。}  ¶ \textcolor{darkblue}{\textbf{\ipa{tɕʰɯ˩-hĩ˩˥}}}  
 ¶ \textcolor{darkblue}{\textbf{\ipa{[M18] tɕi˧-hĩ˧ pʰi˩}}} \textcolor{Sepia}{\selectlanguage{english}to have acid reflux} \zh{吐酸水}  
\ding{203} \textcolor{Sepia}{\selectlanguage{english}Sour, fermented.} \zh{(通过发酵的)酸。} 
\lhead{\firstmark}
\rhead{\botmark}

\subsection{\hspace{-0.5cm} {\Large \textcolor{darkblue}{\textbf{\ipa{tɕi˧}}} \textsubscript{2}}\hspace{0.5cm}[\kern2pt{\textcolor{darkblue}{\textbf{\ipa{tɕi˥}}}}\kern2pt]} \hypertarget{ts£i\string_M2}{}
\markboth{\textcolor{darkblue}{\textbf{\ipa{tɕi˧}}} \textsubscript{2}}{}
\textcolor{teal}{\mytextsc{noun}} \hspace{4pt} Tone: M.
\textcolor{Sepia}{\selectlanguage{english}Snare, trap, trick.} \zh{圈套。}  ¶ \textcolor{darkblue}{\textbf{\ipa{tɕi˧ kʰɯ˧˥}}} \textcolor{Sepia}{\selectlanguage{english}to set a trap} \zh{设下圈套}  
 \zh{量词}: \textcolor{darkblue}{\textbf{\ipa{ɭɯ˧}}}  \mytextsc{clf}: \textcolor{darkblue}{\textbf{\ipa{ɭɯ˧}}} 
\lhead{\firstmark}
\rhead{\botmark}

\subsection{\hspace{-0.5cm} {\Large \textcolor{darkblue}{\textbf{\ipa{tɕi˧\textsubscript{a}}}}}\hspace{0.5cm}[\kern2pt{\textcolor{darkblue}{\textbf{\ipa{tɕi˥}}}}\kern2pt]} \hypertarget{ts£i\string_Ma1}{}
\markboth{\textcolor{darkblue}{\textbf{\ipa{tɕi˧\textsubscript{a}}}}}{}
\textcolor{teal}{\mytextsc{classifier}} \hspace{4pt} Tone: M\textsubscript{a}.
\textcolor{Sepia}{\selectlanguage{english}Some, a few.} \zh{量词:一些。}  ¶ \textcolor{darkblue}{\textbf{\ipa{ɖɯ˧-tɕi˧}}} \textcolor{Sepia}{\selectlanguage{english}some, a few} \zh{一些}  
 ¶ \textcolor{darkblue}{\textbf{\ipa{ʈʂʰɯ˧-tɕi˩}}} \textcolor{Sepia}{\selectlanguage{english}these few} \zh{这些}  

\lhead{\firstmark}
\rhead{\botmark}

\subsection{\hspace{-0.5cm} {\Large \textcolor{darkblue}{\textbf{\ipa{tɕi˧do˩}}}}\hspace{0.5cm}[\kern2pt{\textcolor{darkblue}{\textbf{\ipa{tɕi˧do˩}}}}\kern2pt]} \hypertarget{ts£i\string_Mdo\string_B1}{}
\markboth{\textcolor{darkblue}{\textbf{\ipa{tɕi˧do˩}}}}{}
\textcolor{teal}{\mytextsc{noun}} \hspace{4pt} Tone: L\#.
\textcolor{Sepia}{\selectlanguage{english}Tangerine.} \zh{橘子。} Local Chinese dialect:\zh{黄果。} \zh{量词}: \textcolor{darkblue}{\textbf{\ipa{ɭɯ˧}}}  \mytextsc{clf}: \textcolor{darkblue}{\textbf{\ipa{ɭɯ˧}}} 
\lhead{\firstmark}
\rhead{\botmark}

\subsection{\hspace{-0.5cm} {\Large \textcolor{darkblue}{\textbf{\ipa{tɕi˧-dʑɯ˩}}}}\hspace{0.5cm}[\kern2pt{\textcolor{darkblue}{\textbf{\ipa{xxxx non-correspondance entre le nombre de morphèmes et le nombre de tons de morphèmes}}}}\kern2pt]} \hypertarget{ts£i\string_M-dz£M\string_B1}{}
\markboth{\textcolor{darkblue}{\textbf{\ipa{tɕi˧-dʑɯ˩}}}}{}
\textcolor{teal}{\mytextsc{noun}} \hspace{4pt} Tone: L\#.
\ding{202} \textcolor{Sepia}{\selectlanguage{english}Acid potion: a preparation from sour plums or wild berries, used to make people vomit when they had food poisoning (e.g. from eating poisonous mushrooms).} \zh{用梅子等野生果子做出来的一种药品(酸水),食物中毒的情况下给病人和这种酸水让他呕吐。} \ding{203} \textcolor{Sepia}{\selectlanguage{english}Vinegar.} \zh{醋。} 
\lhead{\firstmark}
\rhead{\botmark}

\subsection{\hspace{-0.5cm} {\Large \textcolor{darkblue}{\textbf{\ipa{tɕi˧kwɤ˧}}}}\hspace{0.5cm}[\kern2pt{\textcolor{darkblue}{\textbf{\ipa{tɕi˧kwɤ˩}}}}\kern2pt]} \hypertarget{ts£i\string_Mkw7\string_M1}{}
\markboth{\textcolor{darkblue}{\textbf{\ipa{tɕi˧kwɤ˧}}}}{}
\textcolor{teal}{\mytextsc{noun}} \hspace{4pt} Tone: M.
\textcolor{Sepia}{\selectlanguage{english}Melon, gourd.} \zh{瓜。}  ¶ \textcolor{darkblue}{\textbf{\ipa{tɕi˧kwɤ˧ bv̩˧-ɻ̍˧ (+ɲi˩)}}} \textcolor{Sepia}{\selectlanguage{english}small melon} \zh{小瓜}  
 ¶ \textcolor{darkblue}{\textbf{\ipa{tɕi˧kwɤ˧ kwɤ˧mo˩}}} \textcolor{Sepia}{\selectlanguage{english}large melon} \zh{大瓜}  
 \zh{量词}: \textcolor{darkblue}{\textbf{\ipa{ɭɯ˧}}}  \mytextsc{clf}: \textcolor{darkblue}{\textbf{\ipa{ɭɯ˧}}} 
\lhead{\firstmark}
\rhead{\botmark}

\subsection{\hspace{-0.5cm} {\Large \textcolor{darkblue}{\textbf{\ipa{tɕi˧sɯ˧pɤ˧}}}}\hspace{0.5cm}[\kern2pt{\textcolor{darkblue}{\textbf{\ipa{tɕi˩sɯ˩pɤ˥}}}}\kern2pt]} \hypertarget{ts£i\string_MsM\string_Mp7\string_M1}{}
\markboth{\textcolor{darkblue}{\textbf{\ipa{tɕi˧sɯ˧pɤ˧}}}}{}
\textcolor{teal}{\mytextsc{noun}} \hspace{4pt} Tone: M.
\textcolor{Sepia}{\selectlanguage{english}Cheese made of yak milk. First, the milk is creamed, then boiled again, with an additive to make it curdle; finally, the preparation is left to dry and harden. It is used in cooking (some of it can be added to gruel), and also as a treatment for diaorrhea. It can keep for a long time.} \zh{牦牛奶酪。}  ¶ \textcolor{darkblue}{\textbf{\ipa{mv̩˧ɭɯ˩-pʰɤ˩bɤ˩, | tɕi˧sɯ˧pɤ˧!}}} \textcolor{Sepia}{\selectlanguage{english}The gift from Muli is yak cheese! / The gift that people usually bring back from their trips to Muli is yak cheese! / Yak cheese is a specialty of Muli! (Yak cheese used to be one of the delicacies that young men offered to young ladies when coming back from caravan journeys.)} \zh{木里的礼物:牦牛的奶酪! / 牦牛奶酪,是木里的特产!}  
 ¶ \textcolor{darkblue}{\textbf{\ipa{mv̩˧ɭɯ˩ pʰɤ˩bɤ˩, | tɕi˧sɯ˧pɤ˧! | ə˧ɖo˧ ʁo˧ ɖʐɯ˥\textasciitilde{}ɖʐɯ˩ ʝi˩-ze˩!}}} \textcolor{Sepia}{\selectlanguage{english}The gift from Muli is yak cheese! (My) beloved will shake her head (when tasting the delightfully acid cheese)! (Words from a song that used to be sung when travelling, imagining the return to Yongning.)} \zh{(从)木里(带回来)的礼物,就是牦牛奶酪!亲爱的(=收礼物的那个人),会摇头的!(吃、喝的时候会摇头,是因为牦牛奶奶酪比较酸)}  
 ¶ \textcolor{darkblue}{\textbf{\ipa{tɕi˧sɯ˧pɤ˧, | ɖɯ˧-tɑ˧˥ | gv̩˧-mɤ˧-kv̩˥! | ʝi˧-kʰv̩˥-lɑ˩ gv̩˩-kv̩˩!}}} \textcolor{Sepia}{\selectlanguage{english}Not everyone knew how to make yak cheese! Only a few had this know-how!} \zh{不是每个人都会做牦牛奶酪!只有少数(人)才会做!}  
 ¶ \textcolor{darkblue}{\textbf{\ipa{tɕi˧sɯ˧pɤ˧-dʑɯ˩}}} \textcolor{Sepia}{\selectlanguage{english}water in which one has diluted some yak cheese; it has medicinal properties} \zh{一种饮料:将牦牛奶酪溶化在水里}  
 ¶ \textcolor{darkblue}{\textbf{\ipa{tɕi˧sɯ˧pɤ˧ ʈʰɯ˩}}} \textcolor{Sepia}{\selectlanguage{english}to drink water in which one has diluted some yak cheese; literally: 'to drink yak cheese'} \zh{喝溶化在水里的牦牛奶酪(直译:喝牦牛奶酪)}  

\lhead{\firstmark}
\rhead{\botmark}

\subsection{\hspace{-0.5cm} {\Large \textcolor{darkblue}{\textbf{\ipa{tɕi˧tɕi˧ | læ˩sæ˧-dzi˩}}}}\hspace{0.5cm}[\kern2pt{\textcolor{darkblue}{\textbf{\ipa{xxxx non-correspondance entre le nombre de groupes tonals et le nombre de tons}}}}\kern2pt]} \hypertarget{ts£i\string_Mts£i\string_M | l\{\string_Bs\{\string_M-dzi\string_B1}{}
\markboth{\textcolor{darkblue}{\textbf{\ipa{tɕi˧tɕi˧ | læ˩sæ˧-dzi˩}}}}{}
\textcolor{teal}{\mytextsc{noun}} \hspace{4pt} Tone: M-LH-L.
\textcolor{Sepia}{\selectlanguage{english}A type of hardwood (not identified yet).} \zh{一种树,木质很硬。} 
\lhead{\firstmark}
\rhead{\botmark}

\subsection{\hspace{-0.5cm} {\Large \textcolor{darkblue}{\textbf{\ipa{tɕi˩\textsubscript{a}}}}}\hspace{0.5cm}[\kern2pt{\textcolor{darkblue}{\textbf{\ipa{tɕi˩˥}}}}\kern2pt]} \hypertarget{ts£i\string_Ba1}{}
\markboth{\textcolor{darkblue}{\textbf{\ipa{tɕi˩\textsubscript{a}}}}}{}
\textcolor{teal}{\mytextsc{adjective}} \hspace{4pt} Tone: L\textsubscript{a}.
\textcolor{Sepia}{\selectlanguage{english}Small; short (not tall).} \zh{矮,低,小。}  ¶ \textcolor{darkblue}{\textbf{\ipa{tɕi˩-hĩ˩˥}}} \textcolor{Sepia}{\selectlanguage{english}\mytextsc{nmlz}} \zh{矮的}  
 ¶ \textcolor{darkblue}{\textbf{\ipa{gv̩˧mi˧ tɕi˩}}} \textcolor{Sepia}{\selectlanguage{english}short (not tall)} \zh{矮}  

\lhead{\firstmark}
\rhead{\botmark}

\subsection{\hspace{-0.5cm} {\Large \textcolor{darkblue}{\textbf{\ipa{tɕi˩nv̩˧˥}}}}\hspace{0.5cm}[\kern2pt{\textcolor{darkblue}{\textbf{\ipa{tɕi˧nv̩˥}}}}\kern2pt]} \hypertarget{ts£i\string_Bnv\string_=\string_M\string_T1}{}
\markboth{\textcolor{darkblue}{\textbf{\ipa{tɕi˩nv̩˧˥}}}}{}
\textcolor{teal}{\mytextsc{noun}} \hspace{4pt} Tone: LM+MH\#.
\textcolor{Sepia}{\selectlanguage{english}Saddle mat.} \zh{马鞍下面的毯子。}  ¶ \textcolor{darkblue}{\textbf{\ipa{ʐwæ˧-tɕi˥nv̩˩}}} \textcolor{Sepia}{\selectlanguage{english}horse saddle mat} \zh{马鞍毯子}  
 \zh{量词}: \textcolor{darkblue}{\textbf{\ipa{pɤ˩}}}  \mytextsc{clf}: \textcolor{darkblue}{\textbf{\ipa{pɤ˩}}} 
\lhead{\firstmark}
\rhead{\botmark}

\subsection{\hspace{-0.5cm} {\Large \textcolor{darkblue}{\textbf{\ipa{tɕi˩qɑ˥}}}}\hspace{0.5cm}[\kern2pt{\textcolor{darkblue}{\textbf{\ipa{tɕi˩qɑ˧˥}}}}\kern2pt]} \hypertarget{ts£i\string_BqA\string_T1}{}
\markboth{\textcolor{darkblue}{\textbf{\ipa{tɕi˩qɑ˥}}}}{}
\textcolor{teal}{\mytextsc{noun}} \hspace{4pt} Tone: LH.
\textcolor{Sepia}{\selectlanguage{english}Carpet.} \zh{毯子。}  \zh{量词}: \textcolor{darkblue}{\textbf{\ipa{ɭɯ˧}}}  \mytextsc{clf}: \textcolor{darkblue}{\textbf{\ipa{ɭɯ˧}}} 
\lhead{\firstmark}
\rhead{\botmark}

\subsection{\hspace{-0.5cm} {\Large \textcolor{darkblue}{\textbf{\ipa{tɕi˩˥}}}}\hspace{0.5cm}[\kern2pt{\textcolor{darkblue}{\textbf{\ipa{tɕi˩˥}}}}\kern2pt]} \hypertarget{ts£i\string_B\string_T1}{}
\markboth{\textcolor{darkblue}{\textbf{\ipa{tɕi˩˥}}}}{}
\textcolor{teal}{\mytextsc{noun}} \hspace{4pt} Tone: LH.
\textcolor{Sepia}{\selectlanguage{english}Saddle.} \zh{马鞍。}  ¶ \textcolor{darkblue}{\textbf{\ipa{ʐwæ˧-tɕi˥}}} \textcolor{Sepia}{\selectlanguage{english}horse saddle} \zh{马鞍}  
 \zh{量词}: \textcolor{darkblue}{\textbf{\ipa{pɤ˩}}}  \mytextsc{clf}: \textcolor{darkblue}{\textbf{\ipa{pɤ˩}}} 
\lhead{\firstmark}
\rhead{\botmark}

\subsection{\hspace{-0.5cm} {\Large \textcolor{darkblue}{\textbf{\ipa{tɕo˥}}}}\hspace{0.5cm}[\kern2pt{\textcolor{darkblue}{\textbf{\ipa{tɕo˩˥}}}}\kern2pt]} \hypertarget{ts£o\string_T1}{}
\markboth{\textcolor{darkblue}{\textbf{\ipa{tɕo˥}}}}{}
\textcolor{teal}{\mytextsc{noun}} \hspace{4pt} Tone: H.
\textcolor{Sepia}{\selectlanguage{english}Direction.} \zh{方向。}  ¶ \textcolor{darkblue}{\textbf{\ipa{ʈʂʰɯ˧-tɕo˧}}} \textcolor{Sepia}{\selectlanguage{english}this way} \zh{这个方向,向这里}  
 ¶ \textcolor{darkblue}{\textbf{\ipa{ɖɯ˧-tɕo˥}}} \textcolor{Sepia}{\selectlanguage{english}one side, in one direction} \zh{一边}  
 ¶ \textcolor{darkblue}{\textbf{\ipa{gɤ˩-tɕo˧}}} \textcolor{Sepia}{\selectlanguage{english}upward, towards the top} \zh{向上,往上}  
 ¶ \textcolor{darkblue}{\textbf{\ipa{dv̩˩tɕo˧}}} \textcolor{Sepia}{\selectlanguage{english}that way} \zh{那边}  

\lhead{\firstmark}
\rhead{\botmark}

\subsection{\hspace{-0.5cm} {\Large \textcolor{darkblue}{\textbf{\ipa{tɕo˩ɕjo˧}}}}\hspace{0.5cm}[\kern2pt{\textcolor{darkblue}{\textbf{\ipa{tɕo˩ɕjo˥}}}}\kern2pt]} \hypertarget{ts£o\string_Bs£jo\string_M1}{}
\markboth{\textcolor{darkblue}{\textbf{\ipa{tɕo˩ɕjo˧}}}}{}
\textcolor{teal}{\mytextsc{noun}} \hspace{4pt} Tone: LM.
\textcolor{Sepia}{\selectlanguage{english}Whistle, whistling noise.} \zh{口哨。}  ¶ \textcolor{darkblue}{\textbf{\ipa{tɕo˩ɕjo˧ | ɖɯ˧-ɖʐo˩ kʰɯ˩}}} \textcolor{Sepia}{\selectlanguage{english}to whistle a little, to whistle a few notes} \zh{吹口哨、吹一声口哨}  

\lhead{\firstmark}
\rhead{\botmark}

\subsection{\hspace{-0.5cm} {\Large \textcolor{darkblue}{\textbf{\ipa{tɕo˩mv̩˧}}}}\hspace{0.5cm}[\kern2pt{\textcolor{darkblue}{\textbf{\ipa{tɕo˩mv̩˥}}}}\kern2pt]} \hypertarget{ts£o\string_Bmv\string_=\string_M1}{}
\markboth{\textcolor{darkblue}{\textbf{\ipa{tɕo˩mv̩˧}}}}{}
\textcolor{teal}{\mytextsc{noun}} \hspace{4pt} Tone: LM.
\textcolor{Sepia}{\selectlanguage{english}Wife of maternal uncle. The word consists of a Chinese borrowing, \zh{舅} 'maternal uncle', to which is added the Na word for 'woman'.} \zh{舅妈(舅:汉语借词,妈:摩梭话“女人”)。}  Borrowing: Chinese  \zh{舅}
 \zh{量词}: \textcolor{darkblue}{\textbf{\ipa{v̩˧}}}  \mytextsc{clf}: \textcolor{darkblue}{\textbf{\ipa{v̩˧}}} 
\lhead{\firstmark}
\rhead{\botmark}

\subsection{\hspace{-0.5cm} {\Large \textcolor{darkblue}{\textbf{\ipa{tɕɯ˥}}}}\hspace{0.5cm}[\kern2pt{\textcolor{darkblue}{\textbf{\ipa{tɕɯ˥}}}}\kern2pt]} \hypertarget{ts£M\string_T1}{}
\markboth{\textcolor{darkblue}{\textbf{\ipa{tɕɯ˥}}}}{}
\textcolor{teal}{\mytextsc{verb}} \hspace{4pt} Tone: H.
\ding{202} \textcolor{Sepia}{\selectlanguage{english}To put, to lay up.} \zh{放置。}  ¶ \textcolor{darkblue}{\textbf{\ipa{tʰi˧-tɕɯ˥}}} \textcolor{Sepia}{\selectlanguage{english}\mytextsc{dur}} \zh{\mytextsc{dur}}  
 ¶ \textcolor{darkblue}{\textbf{\ipa{[F5] ɖɯ˩hĩ˧ | ɖɯ˩˧ | tʰi˧-tɕɯ˥, | tɕi˩hĩ˧ | tɕi˩˧ | tʰi˧-tɕɯ˥}}} \textcolor{Sepia}{\selectlanguage{english}to put big ones with big ones, small ones with small ones} \zh{大小归类}  
\ding{203} \textcolor{Sepia}{\selectlanguage{english}To settle, to decide.} \zh{决定、定下来。}  ¶ \textcolor{darkblue}{\textbf{\ipa{le˧-ʐwɤ˩ | tʰi˧-tɕɯ˥}}} \textcolor{Sepia}{\selectlanguage{english}to settle} \zh{说好、决定}  
 ¶ \textcolor{darkblue}{\textbf{\ipa{le˧-ʐwɤ˩ | tʰi˧-tɕɯ˧-ɲi˥-tsɯ˩!}}} \textcolor{Sepia}{\selectlanguage{english}It's settled!} \zh{说好了! / 决定好了!}  

\lhead{\firstmark}
\rhead{\botmark}

\subsection{\hspace{-0.5cm} {\Large \textcolor{darkblue}{\textbf{\ipa{tɕɯ˧}}}}\hspace{0.5cm}[\kern2pt{\textcolor{darkblue}{\textbf{\ipa{tɕɯ˥}}}}\kern2pt]} \hypertarget{ts£M\string_M1}{}
\markboth{\textcolor{darkblue}{\textbf{\ipa{tɕɯ˧}}}}{}
\textcolor{teal}{\mytextsc{noun}} \hspace{4pt} Tone: M.
\textcolor{Sepia}{\selectlanguage{english}Cloud.} \zh{云。}  ¶ \textcolor{darkblue}{\textbf{\ipa{mv̩˧tɕɯ˥}}} \textcolor{Sepia}{\selectlanguage{english}the weather is cloudy} \zh{天上多云}  
 ¶ \textcolor{darkblue}{\textbf{\ipa{mv̩˧ʁo˥, | tɕɯ˧!}}} \textcolor{Sepia}{\selectlanguage{english}The sky is cloudy!} \zh{天上有云!}  
 ¶ \textcolor{darkblue}{\textbf{\ipa{mv̩˧ʁo˥ tɕɯ˩ pʰv̩˩ |}}} \textcolor{Sepia}{\selectlanguage{english}The sky is cloudy!} \zh{天上有云!}  
 ¶ \textcolor{darkblue}{\textbf{\ipa{tɕɯ˧pʰv̩˩; tɕɯ˧ | pʰv̩˩tɕæ˩˥ | -gv̩˩}}} \textcolor{Sepia}{\selectlanguage{english}white cloud} \zh{白云、白色的云}  
 ¶ \textcolor{darkblue}{\textbf{\ipa{mv̩˧nɑ˥-tɕɯ˩nɑ˩-ɻ̍˩!}}} \textcolor{Sepia}{\selectlanguage{english}the sky is dark / the sky is very cloudy} \zh{天很黑,有很多乌云}  
 \zh{量词}: \textcolor{darkblue}{\textbf{\ipa{kʰwɤ˥}}}  \mytextsc{clf}: \textcolor{darkblue}{\textbf{\ipa{kʰwɤ˥}}} 
\lhead{\firstmark}
\rhead{\botmark}

\subsection{\hspace{-0.5cm} {\Large \textcolor{darkblue}{\textbf{\ipa{tɕɯ˧\textsubscript{b}}}}}\hspace{0.5cm}[\kern2pt{\textcolor{darkblue}{\textbf{\ipa{tɕɯ˥}}}}\kern2pt]} \hypertarget{ts£M\string_Mb1}{}
\markboth{\textcolor{darkblue}{\textbf{\ipa{tɕɯ˧\textsubscript{b}}}}}{}
\textcolor{teal}{\mytextsc{verb}} \hspace{4pt} Tone: M\textsubscript{b}.
\textcolor{Sepia}{\selectlanguage{english}To shake.} \zh{摇晃。}  ¶ \textcolor{darkblue}{\textbf{\ipa{tso˧\textasciitilde{}tso˧ tɕɯ˧}}} \textcolor{Sepia}{\selectlanguage{english}to shake things} \zh{摇东西}  

\lhead{\firstmark}
\rhead{\botmark}

\subsection{\hspace{-0.5cm} {\Large \textcolor{darkblue}{\textbf{\ipa{tɕɯ˧ɭɯ˧}}}}\hspace{0.5cm}[\kern2pt{\textcolor{darkblue}{\textbf{\ipa{tɕɯ˩ɭɯ˩˥}}}}\kern2pt]} \hypertarget{ts£M\string_Ml\string_RM\string_M1}{}
\markboth{\textcolor{darkblue}{\textbf{\ipa{tɕɯ˧ɭɯ˧}}}}{}
\textcolor{teal}{\mytextsc{verb}} \hspace{4pt} Tone: M.
\textcolor{Sepia}{\selectlanguage{english}To roll, to spool, to reel.} \zh{缠绕。}  ¶ \textcolor{darkblue}{\textbf{\ipa{njɤ˧-ɳɯ˧ | tɕɯ˧ɭɯ˧-bi˧!}}} \textcolor{Sepia}{\selectlanguage{english}Let me reel! / Let me do the reeling!} \zh{让我来缠吧!}  

\lhead{\firstmark}
\rhead{\botmark}

\subsection{\hspace{-0.5cm} {\Large \textcolor{darkblue}{\textbf{\ipa{tɕɯ˧mi˥\$}}}}\hspace{0.5cm}[\kern2pt{\textcolor{darkblue}{\textbf{\ipa{tɕɯ˩mi˥}}}}\kern2pt]} \hypertarget{ts£M\string_Mmi\string_T\$1}{}
\markboth{\textcolor{darkblue}{\textbf{\ipa{tɕɯ˧mi˥\$}}}}{}
\textcolor{teal}{\mytextsc{noun}} \hspace{4pt} Tone: H\$.
\textcolor{Sepia}{\selectlanguage{english}Large scale.} \zh{大称。} 
\lhead{\firstmark}
\rhead{\botmark}

\subsection{\hspace{-0.5cm} {\Large \textcolor{darkblue}{\textbf{\ipa{tɕɯ˧pv̩˧}}}}\hspace{0.5cm}[\kern2pt{\textcolor{darkblue}{\textbf{\ipa{tɕɯ˩pv̩˥}}}}\kern2pt]} \hypertarget{ts£M\string_Mpv\string_=\string_M1}{}
\markboth{\textcolor{darkblue}{\textbf{\ipa{tɕɯ˧pv̩˧}}}}{}
\textcolor{teal}{\mytextsc{adjective}} \hspace{4pt} Tone: .
\textcolor{Sepia}{\selectlanguage{english}At ease.} \zh{轻松快乐、舒畅。}  ¶ \textcolor{darkblue}{\textbf{\ipa{ʈʂʰɯ˧qo˧ | tɕɯ˧pv̩˧-ʂe˧\textasciitilde{}ʂe˧ | ɖɯ˧-dzi˩-zo˩-ho˩!}}} \textcolor{Sepia}{\selectlanguage{english}Have a seat here, happy and relaxed!} \zh{在这边舒畅地坐一会吧!}  
 ¶ \textcolor{darkblue}{\textbf{\ipa{ʈʂʰɯ˧qo˧ | tɕɯ˧pv̩˧-ʂe˧\textasciitilde{}ʂe˧-zo˥ | ɖɯ˧-dzi˩-bi˩-ɻ̍˩!}}} \textcolor{Sepia}{\selectlanguage{english}Let's have a seat here, happy and relaxed!} \zh{在这边舒畅地坐一会吧!}  

\lhead{\firstmark}
\rhead{\botmark}

\subsection{\hspace{-0.5cm} {\Large \textcolor{darkblue}{\textbf{\ipa{tɕɯ˧sɯ˧˥}}}}\hspace{0.5cm}[\kern2pt{\textcolor{darkblue}{\textbf{\ipa{tɕɯ˧sɯ˧˥}}}}\kern2pt]} \hypertarget{ts£M\string_MsM\string_M\string_T1}{}
\markboth{\textcolor{darkblue}{\textbf{\ipa{tɕɯ˧sɯ˧˥}}}}{}
\textcolor{teal}{\mytextsc{noun}} \hspace{4pt} Tone: MH\#.
\textcolor{Sepia}{\selectlanguage{english}Mist, fog.} \zh{雾。}  ¶ \textcolor{darkblue}{\textbf{\ipa{tɕɯ˧sɯ˧mv̩˥}}} \textcolor{Sepia}{\selectlanguage{english}there is some fog, there is some mist} \zh{有雾}  
 \zh{量词}: \textcolor{darkblue}{\textbf{\ipa{ti˧˥}}}  \mytextsc{clf}: \textcolor{darkblue}{\textbf{\ipa{ti˧˥}}} 
\lhead{\firstmark}
\rhead{\botmark}

\subsection{\hspace{-0.5cm} {\Large \textcolor{darkblue}{\textbf{\ipa{tɕɯ˧wɤ˧}}}}\hspace{0.5cm}[\kern2pt{\textcolor{darkblue}{\textbf{\ipa{tɕɯ˧wɤ˧}}}}\kern2pt]} \hypertarget{ts£M\string_Mw7\string_M1}{}
\markboth{\textcolor{darkblue}{\textbf{\ipa{tɕɯ˧wɤ˧}}}}{}
\textcolor{teal}{\mytextsc{verb}} \hspace{4pt} Tone: M.
\textcolor{Sepia}{\selectlanguage{english}To reincarnate.} \zh{转生、转世。}  ¶ \textcolor{darkblue}{\textbf{\ipa{le˧-tɕɯ˧wɤ˧-ho˥!}}} \textcolor{Sepia}{\selectlanguage{english}(She/he) is going to get reincarnated! (About a deceased person)} \zh{他要转生了!}  

\lhead{\firstmark}
\rhead{\botmark}

\subsection{\hspace{-0.5cm} {\Large \textcolor{darkblue}{\textbf{\ipa{tɕɯ˧zo˥\$}}}}\hspace{0.5cm}[\kern2pt{\textcolor{darkblue}{\textbf{\ipa{tɕɯ˧zo˥}}}}\kern2pt]} \hypertarget{ts£M\string_Mzo\string_T\$1}{}
\markboth{\textcolor{darkblue}{\textbf{\ipa{tɕɯ˧zo˥\$}}}}{}
\textcolor{teal}{\mytextsc{noun}} \hspace{4pt} Tone: H\$.
\textcolor{Sepia}{\selectlanguage{english}Small scale.} \zh{小称。} 
\lhead{\firstmark}
\rhead{\botmark}

\subsection{\hspace{-0.5cm} {\Large \textcolor{darkblue}{\textbf{\ipa{tɕɯ˩\textsubscript{a}}}}}\hspace{0.5cm}[\kern2pt{\textcolor{darkblue}{\textbf{\ipa{tɕɯ˥}}}}\kern2pt]} \hypertarget{ts£M\string_Ba1}{}
\markboth{\textcolor{darkblue}{\textbf{\ipa{tɕɯ˩\textsubscript{a}}}}}{}
\textcolor{teal}{\mytextsc{verb}} \hspace{4pt} Tone: L\textsubscript{a}.
\textcolor{Sepia}{\selectlanguage{english}To write.} \zh{写。}  ¶ \textcolor{darkblue}{\textbf{\ipa{le˧-tɕɯ˩-ze˩}}} \textcolor{Sepia}{\selectlanguage{english}\mytextsc{accomp}+\mytextsc{pfv}} \zh{写了}  
 ¶ \textcolor{darkblue}{\textbf{\ipa{tʰæ˧ɻæ˩ tɕɯ˩}}} \textcolor{Sepia}{\selectlanguage{english}to write, to write a text, to write a book} \zh{写、写书}  
 ¶ \textcolor{darkblue}{\textbf{\ipa{ɖɯ˧-kʰv̩˥ | tsʰe˧-ɲi˧ ɬi˧, | njɤ˧ | tsʰe˧-ɲi˧ bæ˧ tɕɯ˩-bi˩-ʂv̩˩ɖv̩˩!}}} \textcolor{Sepia}{\selectlanguage{english}There are twelve months in one year; I would like to transcribe twelve stories (in the coming year)! (Context: the consultant notices that I completed the transcription of two texts in two months; by providing this example sentence, she suggests to me the project of keeping up the same rhythm, transcribing twelve stories in the coming year.)} \zh{一年有十二个月,我就想(一年之内)记十二个故事!(情景:我两个月内完成了两个故事的记录工作。发音合作人举这个例句,鼓励我坚持这种速度,一年内再记十二个故事。)}  
 ¶ \textcolor{darkblue}{\textbf{\ipa{ɖɯ˧-tɕɯ˧\textasciitilde{}tɕɯ˥-ɻ̍˩}}} \textcolor{Sepia}{\selectlanguage{english}\mytextsc{delimitative} \string_ \mytextsc{red} \mytextsc{inceptive}} \zh{\mytextsc{delimitative} \string_ \mytextsc{red} \mytextsc{inceptive}}  
 ¶ \textcolor{darkblue}{\textbf{\ipa{tɕɯ˩-di˩˥}}} \textcolor{Sepia}{\selectlanguage{english}brush, pen; literally 'thing to write'} \zh{笔。直译:‘(用来)书写的(东西)’}  
 ¶ \textcolor{darkblue}{\textbf{\ipa{tʰæ˧ɻæ˩-tɕɯ˩-di˩}}} \textcolor{Sepia}{\selectlanguage{english}brush, pen; literally 'thing to write books'} \zh{笔。直译:‘(用来)写书的(东西)’}  
 ¶ \textcolor{darkblue}{\textbf{\ipa{ʈʂʰɯ˧ | tʰi˧-tɕɯ˧\textasciitilde{}tɕɯ˥ dʑo˩}}} \textcolor{Sepia}{\selectlanguage{english}(S)he is writing} \zh{他正在写写东西。}  

\lhead{\firstmark}
\rhead{\botmark}

\subsection{\hspace{-0.5cm} {\Large \textcolor{darkblue}{\textbf{\ipa{tɕɯ˩lv̩˩ho˥}}}}\hspace{0.5cm}[\kern2pt{\textcolor{darkblue}{\textbf{\ipa{xxxx non-correspondance entre le nombre de morphèmes et le nombre de tons de morphèmes}}}}\kern2pt]} \hypertarget{ts£M\string_Blv\string_=\string_Bho\string_T1}{}
\markboth{\textcolor{darkblue}{\textbf{\ipa{tɕɯ˩lv̩˩ho˥}}}}{}
\textcolor{teal}{\mytextsc{noun}} \hspace{4pt} Tone: L+H\#.
\textcolor{Sepia}{\selectlanguage{english}Sling.} \zh{弹弓。}  \zh{量词}: \textcolor{darkblue}{\textbf{\ipa{ɭɯ˧}}}  \mytextsc{clf}: \textcolor{darkblue}{\textbf{\ipa{ɭɯ˧}}} 
\lhead{\firstmark}
\rhead{\botmark}

\subsection{\hspace{-0.5cm} {\Large \textcolor{darkblue}{\textbf{\ipa{tɕɯ˩ɭɯ˩}}}}\hspace{0.5cm}[\kern2pt{\textcolor{darkblue}{\textbf{\ipa{tɕɯ˧ɭɯ˧}}}}\kern2pt]} \hypertarget{ts£M\string_Bl\string_RM\string_B1}{}
\markboth{\textcolor{darkblue}{\textbf{\ipa{tɕɯ˩ɭɯ˩}}}}{}
\textcolor{teal}{\mytextsc{noun}} \hspace{4pt} Tone: L.
\textcolor{Sepia}{\selectlanguage{english}Shrike, \textit{Lanius tephronotus}.} \zh{伯劳鸟。}  \zh{量词}: \textcolor{darkblue}{\textbf{\ipa{mi˩}}}  \mytextsc{clf}: \textcolor{darkblue}{\textbf{\ipa{mi˩}}} 
\lhead{\firstmark}
\rhead{\botmark}

\subsection{\hspace{-0.5cm} {\Large \textcolor{darkblue}{\textbf{\ipa{tɕɯ˩ɭɯ˩-qʰæ˥bæ˩}}}}\hspace{0.5cm}[\kern2pt{\textcolor{darkblue}{\textbf{\ipa{xxxx non-correspondance entre le nombre de morphèmes et le nombre de tons de morphèmes}}}}\kern2pt]} \hypertarget{ts£M\string_Bl\string_RM\string_B-q\string_h\{\string_Tb\{\string_B1}{}
\markboth{\textcolor{darkblue}{\textbf{\ipa{tɕɯ˩ɭɯ˩-qʰæ˥bæ˩}}}}{}
\textcolor{teal}{\mytextsc{noun}} \hspace{4pt} Tone: L+\#H-.
\zh{终石藤。}  \zh{量词}: \textcolor{darkblue}{\textbf{\ipa{dzi˩}}}  \mytextsc{clf}: \textcolor{darkblue}{\textbf{\ipa{dzi˩}}} 
\lhead{\firstmark}
\rhead{\botmark}

\subsection{\hspace{-0.5cm} {\Large \textcolor{darkblue}{\textbf{\ipa{tɕɯ˩mi˥}}}}\hspace{0.5cm}[\kern2pt{\textcolor{darkblue}{\textbf{\ipa{tɕɯ˩mi˥}}}}\kern2pt]} \hypertarget{ts£M\string_Bmi\string_T1}{}
\markboth{\textcolor{darkblue}{\textbf{\ipa{tɕɯ˩mi˥}}}}{}
\textcolor{teal}{\mytextsc{noun}} \hspace{4pt} Tone: LH.
\textcolor{Sepia}{\selectlanguage{english}Chinese Hwamei or Melodious Laughingthrush (\textit{Leucodioptron canorum}).} \zh{画眉鸟。} Local Chinese dialect:\zh{画眉鸟。} ¶ \textcolor{darkblue}{\textbf{\ipa{tɕɯ˩mi˥ | ə˧mi˧ ɲi˩!}}} \textcolor{Sepia}{\selectlanguage{english}It's a mummy hwamei! (=a female)} \zh{是一个画眉鸟妈妈!(=是母的画眉鸟)}  
 ¶ \textcolor{darkblue}{\textbf{\ipa{tɕɯ˩mi˥ | zo˧ ɲi˥!}}} \textcolor{Sepia}{\selectlanguage{english}It's a baby hwamei!} \zh{是一个小画眉鸟!}  
 ¶ \textcolor{darkblue}{\textbf{\ipa{tɕɯ˩mi˥ | pʰv̩˧ ɲi˩!}}} \textcolor{Sepia}{\selectlanguage{english}It's a male hwamei!} \zh{是公的画眉鸟!}  
 \zh{量词}: \textcolor{darkblue}{\textbf{\ipa{mi˩}}}  \mytextsc{clf}: \textcolor{darkblue}{\textbf{\ipa{mi˩}}} 
\lhead{\firstmark}
\rhead{\botmark}

\subsection{\hspace{-0.5cm} {\Large \textcolor{darkblue}{\textbf{\ipa{tɕɯ˧˥}}} \textsubscript{1}}\hspace{0.5cm}[\kern2pt{\textcolor{darkblue}{\textbf{\ipa{tɕɯ˧˥}}}}\kern2pt]} \hypertarget{ts£M\string_M\string_T1}{}
\markboth{\textcolor{darkblue}{\textbf{\ipa{tɕɯ˧˥}}} \textsubscript{1}}{}
\textcolor{teal}{\mytextsc{verb}} \hspace{4pt} Tone: MH.
\textcolor{Sepia}{\selectlanguage{english}To pack-transport.} \zh{驮运。}  ¶ \textcolor{darkblue}{\textbf{\ipa{ʐwæ˧ tɕɯ˩}}} \textcolor{Sepia}{\selectlanguage{english}to pack-transport, to transport on horseback} \zh{用马驮运、做马帮}  
 ¶ \textcolor{darkblue}{\textbf{\ipa{ʐwæ˧ʁo˧ tʰi˧-tɕɯ˧˥}}} \textcolor{Sepia}{\selectlanguage{english}to transport on horseback} \zh{用马驮运}  
 ¶ \textcolor{darkblue}{\textbf{\ipa{ʐwæ˧-tɕɯ˩-zo˩}}} \textcolor{Sepia}{\selectlanguage{english}person who takes part in a caravan, who works in a caravan} \zh{加入马帮的男人}  

\lhead{\firstmark}
\rhead{\botmark}

\subsection{\hspace{-0.5cm} {\Large \textcolor{darkblue}{\textbf{\ipa{tɕɯ˧˥}}} \textsubscript{2}}\hspace{0.5cm}[\kern2pt{\textcolor{darkblue}{\textbf{\ipa{tɕɯ˧˥}}}}\kern2pt]} \hypertarget{ts£M\string_M\string_T2}{}
\markboth{\textcolor{darkblue}{\textbf{\ipa{tɕɯ˧˥}}} \textsubscript{2}}{}
\textcolor{teal}{\mytextsc{noun}} \hspace{4pt} Tone: MH.
\textcolor{Sepia}{\selectlanguage{english}Leech.} \zh{水蛭、蚂蟥。} Local Chinese dialect:\zh{蚂蟥。} \zh{量词}: \textcolor{darkblue}{\textbf{\ipa{mi˩}}}  \mytextsc{clf}: \textcolor{darkblue}{\textbf{\ipa{mi˩}}} 
\lhead{\firstmark}
\rhead{\botmark}

\subsection{\hspace{-0.5cm} {\Large \textcolor{darkblue}{\textbf{\ipa{tɕɯ˧˥}}} \textsubscript{3}}\hspace{0.5cm}[\kern2pt{\textcolor{darkblue}{\textbf{\ipa{tɕɯ˧˥}}}}\kern2pt]} \hypertarget{ts£M\string_M\string_T3}{}
\markboth{\textcolor{darkblue}{\textbf{\ipa{tɕɯ˧˥}}} \textsubscript{3}}{}
\textcolor{teal}{\mytextsc{noun}} \hspace{4pt} Tone: MH.
\textcolor{Sepia}{\selectlanguage{english}Wasp.} \zh{马蜂 (黄蜂)。}  ¶ \textcolor{darkblue}{\textbf{\ipa{tɕɯ˧mi˥\$}}} \textcolor{Sepia}{\selectlanguage{english}female wasp (elicited combination)} \zh{母蚂蜂(人工的词)}  
 ¶ \textcolor{darkblue}{\textbf{\ipa{tɕɯ˧pʰv̩\#˥}}} \textcolor{Sepia}{\selectlanguage{english}male wasp (elicited combination)} \zh{公马蜂}  
 ¶ \textcolor{darkblue}{\textbf{\ipa{tɕɯ˧zo\#˥}}} \textcolor{Sepia}{\selectlanguage{english}baby wasp (elicited combination)} \zh{小马蜂}  
 \zh{量词}: \textcolor{darkblue}{\textbf{\ipa{mi˩}}}  \mytextsc{clf}: \textcolor{darkblue}{\textbf{\ipa{mi˩}}} 
\lhead{\firstmark}
\rhead{\botmark}

\subsection{\hspace{-0.5cm} {\Large \textcolor{darkblue}{\textbf{\ipa{tɕɯ˧˥}}} \textsubscript{4}}\hspace{0.5cm}[\kern2pt{\textcolor{darkblue}{\textbf{\ipa{tɕɯ˧˥}}}}\kern2pt]} \hypertarget{ts£M\string_M\string_T4}{}
\markboth{\textcolor{darkblue}{\textbf{\ipa{tɕɯ˧˥}}} \textsubscript{4}}{}
\textcolor{teal}{\mytextsc{noun}} \hspace{4pt} Tone: MH.
\textcolor{Sepia}{\selectlanguage{english}Scale.} \zh{称。}  \zh{量词}: \textcolor{darkblue}{\textbf{\ipa{nɑ˧}}}  \mytextsc{clf}: \textcolor{darkblue}{\textbf{\ipa{nɑ˧}}} 
\lhead{\firstmark}
\rhead{\botmark}

\subsection{\hspace{-0.5cm} {\Large \textcolor{darkblue}{\textbf{\ipa{tɕɯ˧˥\textsubscript{a}}}} \textsubscript{1}}\hspace{0.5cm}[\kern2pt{\textcolor{darkblue}{\textbf{\ipa{tɕɯ˩˥}}}}\kern2pt]} \hypertarget{ts£M\string_M\string_Ta1}{}
\markboth{\textcolor{darkblue}{\textbf{\ipa{tɕɯ˧˥\textsubscript{a}}}} \textsubscript{1}}{}
\textcolor{teal}{\mytextsc{classifier}} \hspace{4pt} Tone: MH\textsubscript{a}.
\textcolor{Sepia}{\selectlanguage{english}Classifier for loads carried by a pack-animal.} \zh{量词:驮子(一匹)。} 
\lhead{\firstmark}
\rhead{\botmark}

\subsection{\hspace{-0.5cm} {\Large \textcolor{darkblue}{\textbf{\ipa{tɕɯ˧˥\textsubscript{a}}}} \textsubscript{2}}\hspace{0.5cm}[\kern2pt{\textcolor{darkblue}{\textbf{\ipa{tɕɯ˧˥}}}}\kern2pt]} \hypertarget{ts£M\string_M\string_Ta2}{}
\markboth{\textcolor{darkblue}{\textbf{\ipa{tɕɯ˧˥\textsubscript{a}}}} \textsubscript{2}}{}
\textcolor{teal}{\mytextsc{classifier}} \hspace{4pt} Tone: MH\textsubscript{a}.
\textcolor{Sepia}{\selectlanguage{english}Classifier: a pound of.} \zh{量词:斤(用于固体,也用于液体)(汉语借词)。}  Borrowing: Chinese  \zh{斤}
 ¶ \textcolor{darkblue}{\textbf{\ipa{ʐɯ˧ | ɖɯ˧-tɕɯ˧˥}}} \textcolor{Sepia}{\selectlanguage{english}a pint of wine} \zh{一斤酒}  

\lhead{\firstmark}
\rhead{\botmark}

\subsection{\hspace{-0.5cm} {\Large \textcolor{darkblue}{\textbf{\ipa{tɕɯ˩˥}}}}\hspace{0.5cm}[\kern2pt{\textcolor{darkblue}{\textbf{\ipa{tɕɯ˩˥}}}}\kern2pt]} \hypertarget{ts£M\string_B\string_T1}{}
\markboth{\textcolor{darkblue}{\textbf{\ipa{tɕɯ˩˥}}}}{}
\textcolor{teal}{\mytextsc{noun}} \hspace{4pt} Tone: LH.
\textcolor{Sepia}{\selectlanguage{english}Saliva.} \zh{口水、唾、唾沫、唾液。} 
\lhead{\firstmark}
\rhead{\botmark}

\subsection{\hspace{-0.5cm} {\Large \textcolor{darkblue}{\textbf{\ipa{tɕʰɤ˧pɤ˧-mi\#˥}}}}\hspace{0.5cm}[\kern2pt{\textcolor{darkblue}{\textbf{\ipa{xxxx non-correspondance entre le nombre de morphèmes et le nombre de tons de morphèmes}}}}\kern2pt]} \hypertarget{ts£\string_h7\string_Mp7\string_M-mi\#\string_T1}{}
\markboth{\textcolor{darkblue}{\textbf{\ipa{tɕʰɤ˧pɤ˧-mi\#˥}}}}{}
\textcolor{teal}{\mytextsc{noun}} \hspace{4pt} Tone: \#H.
\textcolor{Sepia}{\selectlanguage{english}The name of a sacred spring located in a cave on mount \textcolor{darkblue}{\textbf{\ipa{/nɑ˩tsʰi˩/}}}.} \zh{一处神泉。}  ¶ \textcolor{darkblue}{\textbf{\ipa{nɑ˩tsʰi˩˥ | tɕʰɤ˧pɤ˧-mi\#˥}}} \textcolor{Sepia}{\selectlanguage{english}full name of the mountain where the spring is located} \zh{神泉所在山的全称}  

\lhead{\firstmark}
\rhead{\botmark}

\subsection{\hspace{-0.5cm} {\Large \textcolor{darkblue}{\textbf{\ipa{tɕʰɤ˧ʂo\#˥}}}}\hspace{0.5cm}[\kern2pt{\textcolor{darkblue}{\textbf{\ipa{tɕʰɤ˧ʂo˧}}}}\kern2pt]} \hypertarget{ts£\string_h7\string_Ms`o\#\string_T1}{}
\markboth{\textcolor{darkblue}{\textbf{\ipa{tɕʰɤ˧ʂo\#˥}}}}{}
\textcolor{teal}{\mytextsc{noun}} \hspace{4pt} Tone: \#H.
\textcolor{Sepia}{\selectlanguage{english}Altar.} \zh{祭坛。}  \zh{量词}: \textcolor{darkblue}{\textbf{\ipa{nɑ˧}}}  \mytextsc{clf}: \textcolor{darkblue}{\textbf{\ipa{nɑ˧}}} 
\lhead{\firstmark}
\rhead{\botmark}

\subsection{\hspace{-0.5cm} {\Large \textcolor{darkblue}{\textbf{\ipa{tɕʰɤ˧ti\#˥}}}}\hspace{0.5cm}[\kern2pt{\textcolor{darkblue}{\textbf{\ipa{tɕʰɤ˧ti˧}}}}\kern2pt]} \hypertarget{ts£\string_h7\string_Mti\#\string_T1}{}
\markboth{\textcolor{darkblue}{\textbf{\ipa{tɕʰɤ˧ti\#˥}}}}{}
\textcolor{teal}{\mytextsc{noun}} \hspace{4pt} Tone: \#H.
\textcolor{Sepia}{\selectlanguage{english}Stupa, tower.} \zh{塔。}  \zh{量词}: \textcolor{darkblue}{\textbf{\ipa{ɭɯ˧}}}  \mytextsc{clf}: \textcolor{darkblue}{\textbf{\ipa{ɭɯ˧}}} 
\lhead{\firstmark}
\rhead{\botmark}

\subsection{\hspace{-0.5cm} {\Large \textcolor{darkblue}{\textbf{\ipa{tɕʰɤ˧tɕo˩}}}}\hspace{0.5cm}[\kern2pt{\textcolor{darkblue}{\textbf{\ipa{tɕʰɤ˧tɕo˩}}}}\kern2pt]} \hypertarget{ts£\string_h7\string_Mts£o\string_B1}{}
\markboth{\textcolor{darkblue}{\textbf{\ipa{tɕʰɤ˧tɕo˩}}}}{}
\textcolor{teal}{\mytextsc{noun}} \hspace{4pt} Tone: L\#.
\textcolor{Sepia}{\selectlanguage{english}Two-man saw: a saw designed for use by two sawyers.} \zh{双人锯:以前用于把圆木截成板材的大的双人锯(汉语借词)。}  Borrowing: Chinese  \zh{??}

\lhead{\firstmark}
\rhead{\botmark}

\subsection{\hspace{-0.5cm} {\Large \textcolor{darkblue}{\textbf{\ipa{tɕʰɤ˧tɕʰɤ˧˥}}}}\hspace{0.5cm}[\kern2pt{\textcolor{darkblue}{\textbf{\ipa{tɕʰɤ˧tɕʰɤ˧˥}}}}\kern2pt]} \hypertarget{ts£\string_h7\string_Mts£\string_h7\string_M\string_T1}{}
\markboth{\textcolor{darkblue}{\textbf{\ipa{tɕʰɤ˧tɕʰɤ˧˥}}}}{}
\textcolor{teal}{\mytextsc{adverb(ial)}} \hspace{4pt} Tone: MH\#.
\textcolor{Sepia}{\selectlanguage{english}Entirely, completely, totally.} \zh{彻底。} 
\lhead{\firstmark}
\rhead{\botmark}

\subsection{\hspace{-0.5cm} {\Large \textcolor{darkblue}{\textbf{\ipa{tɕʰɤ˩lv̩˩}}}}\hspace{0.5cm}[\kern2pt{\textcolor{darkblue}{\textbf{\ipa{tɕʰɤ˩lv̩˩˥}}}}\kern2pt]} \hypertarget{ts£\string_h7\string_Blv\string_=\string_B1}{}
\markboth{\textcolor{darkblue}{\textbf{\ipa{tɕʰɤ˩lv̩˩}}}}{}
\textcolor{teal}{\mytextsc{noun}} \hspace{4pt} Tone: L.
\textcolor{Sepia}{\selectlanguage{english}Lily, lily buds.} \zh{百合。}  ¶ \textcolor{darkblue}{\textbf{\ipa{tɕʰɤ˩lv̩˩-hṽ˩hṽ˩˥}}} \textcolor{Sepia}{\selectlanguage{english}stir-fried lily buds} \zh{炒百合}  
 ¶ \textcolor{darkblue}{\textbf{\ipa{tɕʰɤ˩lv̩˩˥, | kv̩˧-pʰæ˧di˥!}}} \textcolor{Sepia}{\selectlanguage{english}Lily buds look like garlic!} \zh{百合,像大蒜!}  
 ¶ \textcolor{darkblue}{\textbf{\ipa{tɕʰɤ˩lv̩˩˥, | dʑɯ˩-nɑ˩mi˩-ʁo˥ dʑɯ˩-nɑ˩mi˩-ʁo˥ di˩-kv̩˩!}}} \textcolor{Sepia}{\selectlanguage{english}Lilies grow high up on the mountain!} \zh{百合长在高山上!}  
 ¶ \textcolor{darkblue}{\textbf{\ipa{tɕʰɤ˩lv̩˩˥, | dʑɯ˩-nɑ˩mi˩-ʁo˥ | di˩-kv̩˩˥! |}}} \textcolor{Sepia}{\selectlanguage{english}Lilies grow high up on the mountain!} \zh{百合长在高山上!}  

\lhead{\firstmark}
\rhead{\botmark}

\subsection{\hspace{-0.5cm} {\Large \textcolor{darkblue}{\textbf{\ipa{tɕʰɤ˩ʈʂv̩˧}}}}\hspace{0.5cm}[\kern2pt{\textcolor{darkblue}{\textbf{\ipa{tɕʰɤ˩ʈʂv̩˥}}}}\kern2pt]} \hypertarget{ts£\string_h7\string_Bt`s`v\string_=\string_M1}{}
\markboth{\textcolor{darkblue}{\textbf{\ipa{tɕʰɤ˩ʈʂv̩˧}}}}{}
\textcolor{teal}{\mytextsc{noun}} \hspace{4pt} Tone: LM.
\textcolor{Sepia}{\selectlanguage{english}Glass used for wine.} \zh{酒杯。}  \zh{量词}: \textcolor{darkblue}{\textbf{\ipa{ɭɯ˧}}}  \mytextsc{clf}: \textcolor{darkblue}{\textbf{\ipa{ɭɯ˧}}} 
\lhead{\firstmark}
\rhead{\botmark}

\subsection{\hspace{-0.5cm} {\Large \textcolor{darkblue}{\textbf{\ipa{tɕʰɤ˩ʈʂv˧}}}}\hspace{0.5cm}[\kern2pt{\textcolor{darkblue}{\textbf{\ipa{tɕʰɤ˩ʈʂv˥}}}}\kern2pt]} \hypertarget{ts£\string_h7\string_Bt`s`v\string_M1}{}
\markboth{\textcolor{darkblue}{\textbf{\ipa{tɕʰɤ˩ʈʂv˧}}}}{}
\textcolor{teal}{\mytextsc{noun}} \hspace{4pt} Tone: LM.
\textcolor{Sepia}{\selectlanguage{english}Drinking glass, goblet.} \zh{杯子。}  ¶ \textcolor{darkblue}{\textbf{\ipa{bo˧ʐæ˧-tɕʰɤ˩ʈʂv˩}}} \textcolor{Sepia}{\selectlanguage{english}goblet for drinking tea (made of glass)} \zh{玻璃茶杯}  

\lhead{\firstmark}
\rhead{\botmark}

\subsection{\hspace{-0.5cm} {\Large \textcolor{darkblue}{\textbf{\ipa{tɕʰɤ˧˥}}}}\hspace{0.5cm}[\kern2pt{\textcolor{darkblue}{\textbf{\ipa{tɕʰɤ˧˥}}}}\kern2pt]} \hypertarget{ts£\string_h7\string_M\string_T1}{}
\markboth{\textcolor{darkblue}{\textbf{\ipa{tɕʰɤ˧˥}}}}{}
\textcolor{teal}{\mytextsc{verb}} \hspace{4pt} Tone: MH.
\textcolor{Sepia}{\selectlanguage{english}To cheat on someone, to deceive.} \zh{欺骗。}  ¶ \textcolor{darkblue}{\textbf{\ipa{le˧-tɕʰɤ˧-ze˥}}} \textcolor{Sepia}{\selectlanguage{english}\mytextsc{accomp} \string_ \mytextsc{pfv}} \zh{欺骗了}  
 ¶ \textcolor{darkblue}{\textbf{\ipa{hĩ˧ tɕʰɤ˧(-ze˩)}}} \textcolor{Sepia}{\selectlanguage{english}to cheat on people, to deceive people} \zh{骗人}  
 ¶ \textcolor{darkblue}{\textbf{\ipa{no˧ | hĩ˧ tɕʰɤ˧!}}} \textcolor{Sepia}{\selectlanguage{english}You cheat people! / You deceive people!} \zh{你骗人!}  
 ¶ \textcolor{darkblue}{\textbf{\ipa{(hĩ˧ |) no˩ tɕʰɤ˩˥!}}} \textcolor{Sepia}{\selectlanguage{english}People cheat you!} \zh{人家骗你!}  
 ¶ \textcolor{darkblue}{\textbf{\ipa{(no˧ |) njɤ˩ tɕʰɤ˩˥!}}} \textcolor{Sepia}{\selectlanguage{english}You cheat on me!} \zh{你骗我!}  

\lhead{\firstmark}
\rhead{\botmark}

\subsection{\hspace{-0.5cm} {\Large \textcolor{darkblue}{\textbf{\ipa{tɕʰi˥}}}}\hspace{0.5cm}[\kern2pt{\textcolor{darkblue}{\textbf{\ipa{tɕʰi˥}}}}\kern2pt]} \hypertarget{ts£\string_hi\string_T1}{}
\markboth{\textcolor{darkblue}{\textbf{\ipa{tɕʰi˥}}}}{}
\textcolor{teal}{\mytextsc{noun}} \hspace{4pt} Tone: \#H.
\textcolor{Sepia}{\selectlanguage{english}Thorn.} \zh{刺。}  \zh{量词}: \textcolor{darkblue}{\textbf{\ipa{kɤ˧˥}}}  \mytextsc{clf}: \textcolor{darkblue}{\textbf{\ipa{kɤ˧˥}}} 
\lhead{\firstmark}
\rhead{\botmark}

\subsection{\hspace{-0.5cm} {\Large \textcolor{darkblue}{\textbf{\ipa{tɕʰi˧\textsubscript{b}}}} \textsubscript{1}}\hspace{0.5cm}[\kern2pt{\textcolor{darkblue}{\textbf{\ipa{tɕʰi˥}}}}\kern2pt]} \hypertarget{ts£\string_hi\string_Mb1}{}
\markboth{\textcolor{darkblue}{\textbf{\ipa{tɕʰi˧\textsubscript{b}}}} \textsubscript{1}}{}
\textcolor{teal}{\mytextsc{verb}} \hspace{4pt} Tone: M\textsubscript{b}.
\textcolor{Sepia}{\selectlanguage{english}To guard, to defend (e.g. guard a house).} \zh{守卫。}  ¶ \textcolor{darkblue}{\textbf{\ipa{ɑ˩ʁo˧ tɕʰi˧}}} \textcolor{Sepia}{\selectlanguage{english}to watch over the house, to guard the house} \zh{守护家}  
 ¶ \textcolor{darkblue}{\textbf{\ipa{ɑ˩ʁo˧ tʰi˧-tɕʰi˧-dʑo˧}}} \textcolor{Sepia}{\selectlanguage{english}watching over the house} \zh{守着家}  
 ¶ \textcolor{darkblue}{\textbf{\ipa{tso˧\textasciitilde{}tso˧ tɕʰi˧}}} \textcolor{Sepia}{\selectlanguage{english}to watch over things} \zh{守着东西}  

\lhead{\firstmark}
\rhead{\botmark}

\subsection{\hspace{-0.5cm} {\Large \textcolor{darkblue}{\textbf{\ipa{tɕʰi˧\textsubscript{b}}}} \textsubscript{2}}\hspace{0.5cm}[\kern2pt{\textcolor{darkblue}{\textbf{\ipa{tɕʰi˥}}}}\kern2pt]} \hypertarget{ts£\string_hi\string_Mb2}{}
\markboth{\textcolor{darkblue}{\textbf{\ipa{tɕʰi˧\textsubscript{b}}}} \textsubscript{2}}{}
\textcolor{teal}{\mytextsc{verb}} \hspace{4pt} Tone: M\textsubscript{b}.
\textcolor{Sepia}{\selectlanguage{english}To sell.} \zh{卖。} 
\lhead{\firstmark}
\rhead{\botmark}

\subsection{\hspace{-0.5cm} {\Large \textcolor{darkblue}{\textbf{\ipa{tɕʰi˧ɖv̩\#˥}}}}\hspace{0.5cm}[\kern2pt{\textcolor{darkblue}{\textbf{\ipa{tɕʰi˧ɖv̩˧}}}}\kern2pt]} \hypertarget{ts£\string_hi\string_Md`v\string_=\#\string_T1}{}
\markboth{\textcolor{darkblue}{\textbf{\ipa{tɕʰi˧ɖv̩\#˥}}}}{}
\textcolor{teal}{\mytextsc{noun}} \hspace{4pt} Tone: \#H.
\textcolor{Sepia}{\selectlanguage{english}Feminine given name.} \zh{女性名字。} 
\lhead{\firstmark}
\rhead{\botmark}

\subsection{\hspace{-0.5cm} {\Large \textcolor{darkblue}{\textbf{\ipa{tɕʰi˧nɑ˥}}}}\hspace{0.5cm}[\kern2pt{\textcolor{darkblue}{\textbf{\ipa{tɕʰi˧nɑ˧}}}}\kern2pt]} \hypertarget{ts£\string_hi\string_MnA\string_T1}{}
\markboth{\textcolor{darkblue}{\textbf{\ipa{tɕʰi˧nɑ˥}}}}{}
\textcolor{teal}{\mytextsc{noun}} \hspace{4pt} Tone: H\#.
\textcolor{Sepia}{\selectlanguage{english}Prinsepia, \textit{Prinsepia utilis Royle}; its seeds yield a highly valued oil, for both cooking and massaging on people's bodies.} \zh{青刺果、青刺尖、阿娜斯果。}  ¶ \textcolor{darkblue}{\textbf{\ipa{tɕʰi˧nɑ˥-dzi˩}}} \textcolor{Sepia}{\selectlanguage{english}prinsepia plant} \zh{青刺尖}  
 ¶ \textcolor{darkblue}{\textbf{\ipa{tɕʰi˧nɑ˥-bæ˩bæ˩}}} \textcolor{Sepia}{\selectlanguage{english}prinsepia flower} \zh{青刺果花}  

\lhead{\firstmark}
\rhead{\botmark}

\subsection{\hspace{-0.5cm} {\Large \textcolor{darkblue}{\textbf{\ipa{tɕʰi˧ʈʂʰɤ˥}}}}\hspace{0.5cm}[\kern2pt{\textcolor{darkblue}{\textbf{\ipa{xxxx non-correspondance entre le nombre de morphèmes et le nombre de tons de morphèmes}}}}\kern2pt]} \hypertarget{ts£\string_hi\string_Mt`s`\string_h7\string_T1}{}
\markboth{\textcolor{darkblue}{\textbf{\ipa{tɕʰi˧ʈʂʰɤ˥}}}}{}
\textcolor{teal}{\mytextsc{noun}} \hspace{4pt} Tone: H\#.
\textcolor{Sepia}{\selectlanguage{english}Car.} \zh{汽车(汉语借词)。}  Borrowing: Chinese  \zh{汽车}
 \zh{量词}: \textcolor{darkblue}{\textbf{\ipa{nɑ˧}}}  \mytextsc{clf}: \textcolor{darkblue}{\textbf{\ipa{nɑ˧}}} 
\lhead{\firstmark}
\rhead{\botmark}

\subsection{\hspace{-0.5cm} {\Large \textcolor{darkblue}{\textbf{\ipa{tɕʰi˩\textsubscript{b}}}}}\hspace{0.5cm}[\kern2pt{\textcolor{darkblue}{\textbf{\ipa{tɕʰi˩˥}}}}\kern2pt]} \hypertarget{ts£\string_hi\string_Bb1}{}
\markboth{\textcolor{darkblue}{\textbf{\ipa{tɕʰi˩\textsubscript{b}}}}}{}
\textcolor{teal}{\mytextsc{classifier}} \hspace{4pt} Tone: L\textsubscript{b}.
\textcolor{Sepia}{\selectlanguage{english}Classifier for meals.} \zh{量词:饭(一顿)。}  ¶ \textcolor{darkblue}{\textbf{\ipa{ɖɯ˧-tɕʰi˩ dzɯ˩}}} \textcolor{Sepia}{\selectlanguage{english}to have a meal, to eat a meal} \zh{吃一顿}  
 ¶ \textcolor{darkblue}{\textbf{\ipa{gv̩˧-tɕʰi˥}}} \textcolor{Sepia}{\selectlanguage{english}nine meals} \zh{就顿(饭)}  
 ¶ \textcolor{darkblue}{\textbf{\ipa{tɕʰi˩ tʰv̩˩˥}}} \textcolor{Sepia}{\selectlanguage{english}to contribute food for the meals during a funeral ceremony: when one is invited to a funeral, one brings food as a contribution to the funeral} \zh{带饭,“出(一)顿(饭)”:被请参加守孝时,要给那家主人带上饭)}  
 ¶ \textcolor{darkblue}{\textbf{\ipa{tɕʰi˩tʰv̩˩-hĩ˥}}} \textcolor{Sepia}{\selectlanguage{english}the person who provides the meal at a wake (following a funeral); it is generally someone who is not from the household.} \zh{给大家供饭的那个人(不一定是主人)}  

\lhead{\firstmark}
\rhead{\botmark}

\subsection{\hspace{-0.5cm} {\Large \textcolor{darkblue}{\textbf{\ipa{tɕʰi˩tsɯ˧}}}}\hspace{0.5cm}[\kern2pt{\textcolor{darkblue}{\textbf{\ipa{tɕʰi˧tsɯ˥}}}}\kern2pt]} \hypertarget{ts£\string_hi\string_BtsM\string_M1}{}
\markboth{\textcolor{darkblue}{\textbf{\ipa{tɕʰi˩tsɯ˧}}}}{}
\textcolor{teal}{\mytextsc{noun}} \hspace{4pt} Tone: LM.
\textcolor{Sepia}{\selectlanguage{english}Eggplant.} \zh{茄子。}  Borrowing: Chinese  \zh{茄子}

\lhead{\firstmark}
\rhead{\botmark}

\subsection{\hspace{-0.5cm} {\Large \textcolor{darkblue}{\textbf{\ipa{tɕʰo˩}}}}\hspace{0.5cm}[\kern2pt{\textcolor{darkblue}{\textbf{\ipa{tɕʰo˩˥}}}}\kern2pt]} \hypertarget{ts£\string_ho\string_B1}{}
\markboth{\textcolor{darkblue}{\textbf{\ipa{tɕʰo˩}}}}{}
\textcolor{teal}{\mytextsc{classifier}} \hspace{4pt} Tone: L *.
\textcolor{Sepia}{\selectlanguage{english}Classifier: in combination with 'one', means 'together'; no plural form.} \zh{量词:一起。}  ¶ \textcolor{darkblue}{\textbf{\ipa{ɖɯ˧-tɕʰo˩}}} \textcolor{Sepia}{\selectlanguage{english}together} \zh{一起}  
 ¶ \textcolor{darkblue}{\textbf{\ipa{le˧-tɕʰo˥\textasciitilde{}tɕʰo˩}}} \textcolor{Sepia}{\selectlanguage{english}same meaning as above: together} \zh{同上:一起}  

\lhead{\firstmark}
\rhead{\botmark}

\subsection{\hspace{-0.5cm} {\Large \textcolor{darkblue}{\textbf{\ipa{tɕʰo˩\textsubscript{a}}}}}\hspace{0.5cm}[\kern2pt{\textcolor{darkblue}{\textbf{\ipa{tɕʰo˩˥}}}}\kern2pt]} \hypertarget{ts£\string_ho\string_Ba1}{}
\markboth{\textcolor{darkblue}{\textbf{\ipa{tɕʰo˩\textsubscript{a}}}}}{}
\textcolor{teal}{\mytextsc{verb}} \hspace{4pt} Tone: L\textsubscript{a}.
\textcolor{Sepia}{\selectlanguage{english}To accompany someone, to go along with someone.} \zh{陪伴、一起去、跟着。}  ¶ \textcolor{darkblue}{\textbf{\ipa{hĩ˧ tɕʰo˥}}} \textcolor{Sepia}{\selectlanguage{english}to accompany someone} \zh{陪伴某人}  
 ¶ \textcolor{darkblue}{\textbf{\ipa{ɖɯ˧-tɕʰo˩ tʰi˩-tɕʰo˩ |}}} \textcolor{Sepia}{\selectlanguage{english}to make up a set, to go with each other/one another: for instance, in the main room, the thangka above the hearth and the paintings on the cupboard that hosts the altar to the ancestors make up a set, they go with each other} \zh{陪伴某人}  

\lhead{\firstmark}
\rhead{\botmark}

\subsection{\hspace{-0.5cm} {\Large \textcolor{darkblue}{\textbf{\ipa{tɕʰo˩mi\#˥}}}}\hspace{0.5cm}[\kern2pt{\textcolor{darkblue}{\textbf{\ipa{tɕʰo˩mi˥}}}}\kern2pt]} \hypertarget{ts£\string_ho\string_Bmi\#\string_T1}{}
\markboth{\textcolor{darkblue}{\textbf{\ipa{tɕʰo˩mi\#˥}}}}{}
\textcolor{teal}{\mytextsc{noun}} \hspace{4pt} Tone: LM+\#H.
\textcolor{Sepia}{\selectlanguage{english}Large ladle.} \zh{大瓢。}  \zh{量词}: \textcolor{darkblue}{\textbf{\ipa{nɑ˧}}}  \mytextsc{clf}: \textcolor{darkblue}{\textbf{\ipa{nɑ˧}}} 
\lhead{\firstmark}
\rhead{\botmark}

\subsection{\hspace{-0.5cm} {\Large \textcolor{darkblue}{\textbf{\ipa{tɕʰo˩qʰwɤ˧}}}}\hspace{0.5cm}[\kern2pt{\textcolor{darkblue}{\textbf{\ipa{tɕʰo˩qʰwɤ˥}}}}\kern2pt]} \hypertarget{ts£\string_ho\string_Bq\string_hw7\string_M1}{}
\markboth{\textcolor{darkblue}{\textbf{\ipa{tɕʰo˩qʰwɤ˧}}}}{}
\textcolor{teal}{\mytextsc{noun}} \hspace{4pt} Tone: LM.
\textcolor{Sepia}{\selectlanguage{english}Ladle used for pigswill.} \zh{用来煮猪食的勺子。}  \zh{量词}: \textcolor{darkblue}{\textbf{\ipa{nɑ˧}}}  \mytextsc{clf}: \textcolor{darkblue}{\textbf{\ipa{nɑ˧}}} 
\lhead{\firstmark}
\rhead{\botmark}

\subsection{\hspace{-0.5cm} {\Large \textcolor{darkblue}{\textbf{\ipa{tɕʰo˩zo\#˥}}}}\hspace{0.5cm}[\kern2pt{\textcolor{darkblue}{\textbf{\ipa{tɕʰo˩zo˥}}}}\kern2pt]} \hypertarget{ts£\string_ho\string_Bzo\#\string_T1}{}
\markboth{\textcolor{darkblue}{\textbf{\ipa{tɕʰo˩zo\#˥}}}}{}
\textcolor{teal}{\mytextsc{noun}} \hspace{4pt} Tone: LM+\#H.
\textcolor{Sepia}{\selectlanguage{english}Small ladle.} \zh{小瓢。}  \zh{量词}: \textcolor{darkblue}{\textbf{\ipa{nɑ˧}}}  \mytextsc{clf}: \textcolor{darkblue}{\textbf{\ipa{nɑ˧}}} 
\lhead{\firstmark}
\rhead{\botmark}

\subsection{\hspace{-0.5cm} {\Large \textcolor{darkblue}{\textbf{\ipa{tɕʰo˧˥}}}}\hspace{0.5cm}[\kern2pt{\textcolor{darkblue}{\textbf{\ipa{tɕʰo˧˥}}}}\kern2pt]} \hypertarget{ts£\string_ho\string_M\string_T1}{}
\markboth{\textcolor{darkblue}{\textbf{\ipa{tɕʰo˧˥}}}}{}
\textcolor{teal}{\mytextsc{verb}} \hspace{4pt} Tone: MH.
\textcolor{Sepia}{\selectlanguage{english}To square (off).} \zh{(将木料)砍成方形。}  ¶ \textcolor{darkblue}{\textbf{\ipa{bi˩mi˩-ɳɯ˥ | tɕʰo˧˥}}} \textcolor{Sepia}{\selectlanguage{english}to square off with an axe} \zh{用斧头砍成方形}  

\lhead{\firstmark}
\rhead{\botmark}

\subsection{\hspace{-0.5cm} {\Large \textcolor{darkblue}{\textbf{\ipa{tɕʰo˩˧}}}}\hspace{0.5cm}[\kern2pt{\textcolor{darkblue}{\textbf{\ipa{tɕʰo˩˥}}}}\kern2pt]} \hypertarget{ts£\string_ho\string_B\string_M1}{}
\markboth{\textcolor{darkblue}{\textbf{\ipa{tɕʰo˩˧}}}}{}
\textcolor{teal}{\mytextsc{noun}} \hspace{4pt} Tone: LM.
\textcolor{Sepia}{\selectlanguage{english}Ladle, scoop used for water.} \zh{勺子、瓢。}  \zh{量词}: \textcolor{darkblue}{\textbf{\ipa{nɑ˧}}}  \mytextsc{clf}: \textcolor{darkblue}{\textbf{\ipa{nɑ˧}}} 
\lhead{\firstmark}
\rhead{\botmark}

\subsection{\hspace{-0.5cm} {\Large \textcolor{darkblue}{\textbf{\ipa{tɕʰɯ˥}}}}\hspace{0.5cm}[\kern2pt{\textcolor{darkblue}{\textbf{\ipa{tɕʰɯ˧˥}}}}\kern2pt]} \hypertarget{ts£\string_hM\string_T1}{}
\markboth{\textcolor{darkblue}{\textbf{\ipa{tɕʰɯ˥}}}}{}
\textcolor{teal}{\mytextsc{verb}} \hspace{4pt} Tone: H.
\textcolor{Sepia}{\selectlanguage{english}To pierce (e.g. a cow's nose).} \zh{穿刺、 刺破。}  ¶ \textcolor{darkblue}{\textbf{\ipa{ʝi˧ ʈʂʰɯ˧-pʰo˩, | ɲi˧ tɕʰi˧-ze˩!}}} \textcolor{Sepia}{\selectlanguage{english}This ox's nose was pierced (to put a ring)} \zh{这头牛的鼻子被穿刺(为了安一个牛鼻圈)}  

\lhead{\firstmark}
\rhead{\botmark}

\subsection{\hspace{-0.5cm} {\Large \textcolor{darkblue}{\textbf{\ipa{tɕʰɯ˧\textsubscript{a}}}} \textsubscript{1}}\hspace{0.5cm}[\kern2pt{\textcolor{darkblue}{\textbf{\ipa{tɕʰɯ˩˥}}}}\kern2pt]} \hypertarget{ts£\string_hM\string_Ma1}{}
\markboth{\textcolor{darkblue}{\textbf{\ipa{tɕʰɯ˧\textsubscript{a}}}} \textsubscript{1}}{}
\textcolor{teal}{\mytextsc{verb}} \hspace{4pt} Tone: M\textsubscript{a}.
\textcolor{Sepia}{\selectlanguage{english}To raise (one's arm).} \zh{举、抬(胳膊)。}  ¶ \textcolor{darkblue}{\textbf{\ipa{lo˩qʰwɤ˥ | gɤ˩-tɕʰɯ˧}}} \textcolor{Sepia}{\selectlanguage{english}to raise one's arm} \zh{举手、抬胳膊}  
 ¶ \textcolor{darkblue}{\textbf{\ipa{kʰɯ˧tsʰɤ˧˥ | gɤ˩-tɕʰɯ˧}}} \textcolor{Sepia}{\selectlanguage{english}to raise one's leg} \zh{抬脚}  
 ¶ \textcolor{darkblue}{\textbf{\ipa{gɤ˩-mɤ˧-tɕʰɯ˧}}} \textcolor{Sepia}{\selectlanguage{english}not to raise} \zh{不抬起来}  

\lhead{\firstmark}
\rhead{\botmark}

\subsection{\hspace{-0.5cm} {\Large \textcolor{darkblue}{\textbf{\ipa{tɕʰɯ˧\textsubscript{a}}}} \textsubscript{2}}\hspace{0.5cm}[\kern2pt{\textcolor{darkblue}{\textbf{\ipa{tɕʰɯ˥}}}}\kern2pt]} \hypertarget{ts£\string_hM\string_Ma2}{}
\markboth{\textcolor{darkblue}{\textbf{\ipa{tɕʰɯ˧\textsubscript{a}}}} \textsubscript{2}}{}
\textcolor{teal}{\mytextsc{verb}} \hspace{4pt} Tone: M\textsubscript{a}.
\ding{202} \textcolor{Sepia}{\selectlanguage{english}To guard, to keep guard.} \zh{守护。} \ding{203} \textcolor{Sepia}{\selectlanguage{english}To keep a deathwatch, to sit with others at a funeral wake.} \zh{居丧、守灵。}  ¶ \textcolor{darkblue}{\textbf{\ipa{hĩ˧ tɕʰɯ˧}}} \textcolor{Sepia}{\selectlanguage{english}same meaning: to keep a deathwatch for a deceased person} \zh{同上:守灵}  

\lhead{\firstmark}
\rhead{\botmark}

\subsection{\hspace{-0.5cm} {\Large \textcolor{darkblue}{\textbf{\ipa{tɕʰɯ˧bo˧˥}}}}\hspace{0.5cm}[\kern2pt{\textcolor{darkblue}{\textbf{\ipa{tɕʰɯ˧bo˧}}}}\kern2pt]} \hypertarget{ts£\string_hM\string_Mbo\string_M\string_T1}{}
\markboth{\textcolor{darkblue}{\textbf{\ipa{tɕʰɯ˧bo˧˥}}}}{}
\textcolor{teal}{\mytextsc{adjective}} \hspace{4pt} Tone: MH\#.
\textcolor{Sepia}{\selectlanguage{english}Fresh, cool.} \zh{凉快。} 
\lhead{\firstmark}
\rhead{\botmark}

\subsection{\hspace{-0.5cm} {\Large \textcolor{darkblue}{\textbf{\ipa{tɕʰɯ˧lo\#˥}}}}\hspace{0.5cm}[\kern2pt{\textcolor{darkblue}{\textbf{\ipa{tɕʰɯ˩lo˩˥}}}}\kern2pt]} \hypertarget{ts£\string_hM\string_Mlo\#\string_T1}{}
\markboth{\textcolor{darkblue}{\textbf{\ipa{tɕʰɯ˧lo\#˥}}}}{}
\textcolor{teal}{\mytextsc{noun}} \hspace{4pt} Tone: \#H.
\textcolor{Sepia}{\selectlanguage{english}Large plate.} \zh{大盘子。}  \zh{量词}: \textcolor{darkblue}{\textbf{\ipa{ɭɯ˧}}}  \mytextsc{clf}: \textcolor{darkblue}{\textbf{\ipa{ɭɯ˧}}} 
\lhead{\firstmark}
\rhead{\botmark}

\subsection{\hspace{-0.5cm} {\Large \textcolor{darkblue}{\textbf{\ipa{tɕʰɯ˧si˩-dʑɤ˩pv̩˩}}}}\hspace{0.5cm}[\kern2pt{\textcolor{darkblue}{\textbf{\ipa{tɕʰɯ˧si˩dʑɤ˧pv̩˧}}}}\kern2pt]} \hypertarget{ts£\string_hM\string_Msi\string_B-dz£7\string_Bpv\string_=\string_B1}{}
\markboth{\textcolor{darkblue}{\textbf{\ipa{tɕʰɯ˧si˩-dʑɤ˩pv̩˩}}}}{}
\textcolor{teal}{\mytextsc{noun}} \hspace{4pt} Tone: L\#-.
\textcolor{Sepia}{\selectlanguage{english}Monster, demon.} \zh{妖怪。}  ¶ \textcolor{darkblue}{\textbf{\ipa{no˧ | tɕʰɯ˧si˩-dʑɤ˩pv̩˩-ki˩ | le˧-hɯ˩-ɲi˩-ze˩!}}} \textcolor{Sepia}{\selectlanguage{english}You have gone away to the world of monstres (and should not come back to trouble the living)! (Speech addressed to a ghost that one beseeches should not come back)} \zh{你已经到妖怪的世界那边(就恳求你不要回来了)!(对鬼说的话)}  

\lhead{\firstmark}
\rhead{\botmark}

\subsection{\hspace{-0.5cm} {\Large \textcolor{darkblue}{\textbf{\ipa{tɕʰɯ˧sɯ˥}}}}\hspace{0.5cm}[\kern2pt{\textcolor{darkblue}{\textbf{\ipa{tɕʰɯ˧sɯ˥}}}}\kern2pt]} \hypertarget{ts£\string_hM\string_MsM\string_T1}{}
\markboth{\textcolor{darkblue}{\textbf{\ipa{tɕʰɯ˧sɯ˥}}}}{}
\textcolor{teal}{\mytextsc{adjective}} \hspace{4pt} Tone: H\#.
\textcolor{Sepia}{\selectlanguage{english}Sad, grieved.} \zh{悲哀、伤心。} 
\lhead{\firstmark}
\rhead{\botmark}

\subsection{\hspace{-0.5cm} {\Large \textcolor{darkblue}{\textbf{\ipa{tɕʰɯ˩\textsubscript{a}}}}}\hspace{0.5cm}[\kern2pt{\textcolor{darkblue}{\textbf{\ipa{tɕʰɯ˧˥}}}}\kern2pt]} \hypertarget{ts£\string_hM\string_Ba1}{}
\markboth{\textcolor{darkblue}{\textbf{\ipa{tɕʰɯ˩\textsubscript{a}}}}}{}
\textcolor{teal}{\mytextsc{adjective}} \hspace{4pt} Tone: L\textsubscript{a}.
\textcolor{Sepia}{\selectlanguage{english}Sweet.} \zh{甜。} 
\lhead{\firstmark}
\rhead{\botmark}

\subsection{\hspace{-0.5cm} {\Large \textcolor{darkblue}{\textbf{\ipa{tɕʰɯ˩di˩}}}}\hspace{0.5cm}[\kern2pt{\textcolor{darkblue}{\textbf{\ipa{tɕʰɯ˧di˧˥}}}}\kern2pt]} \hypertarget{ts£\string_hM\string_Bdi\string_B1}{}
\markboth{\textcolor{darkblue}{\textbf{\ipa{tɕʰɯ˩di˩}}}}{}
\textcolor{teal}{\mytextsc{verb}} \hspace{4pt} Tone: L.
\textcolor{Sepia}{\selectlanguage{english}To hunt.} \zh{狩猎。} \textit{See:} \textcolor{darkblue}{\textbf{\ipa{tɕʰɯ˩˥, di˧˥1}}} 
\lhead{\firstmark}
\rhead{\botmark}

\subsection{\hspace{-0.5cm} {\Large \textcolor{darkblue}{\textbf{\ipa{tɕʰɯ˩di˩kʰv̩˩}}}}\hspace{0.5cm}[\kern2pt{\textcolor{darkblue}{\textbf{\ipa{tɕʰɯ˩di˩kʰv̩˩˥}}}}\kern2pt]} \hypertarget{ts£\string_hM\string_Bdi\string_Bk\string_hv\string_=\string_B1}{}
\markboth{\textcolor{darkblue}{\textbf{\ipa{tɕʰɯ˩di˩kʰv̩˩}}}}{}
\textcolor{teal}{\mytextsc{noun}} \hspace{4pt} Tone: L.
\textcolor{Sepia}{\selectlanguage{english}Hunting dog, hound.} \zh{猎狗。}  ¶ \textcolor{darkblue}{\textbf{\ipa{tɕʰɯ˩di˩-kʰv̩˥mi˩}}} \textcolor{Sepia}{\selectlanguage{english}same meaning} \zh{猎狗}  
 \zh{量词}: \textcolor{darkblue}{\textbf{\ipa{mi˩}}}  \mytextsc{clf}: \textcolor{darkblue}{\textbf{\ipa{mi˩}}} 
\lhead{\firstmark}
\rhead{\botmark}

\subsection{\hspace{-0.5cm} {\Large \textcolor{darkblue}{\textbf{\ipa{tɕʰɯ˩mi\#˥}}}}\hspace{0.5cm}[\kern2pt{\textcolor{darkblue}{\textbf{\ipa{tɕʰɯ˧mi˥}}}}\kern2pt]} \hypertarget{ts£\string_hM\string_Bmi\#\string_T1}{}
\markboth{\textcolor{darkblue}{\textbf{\ipa{tɕʰɯ˩mi\#˥}}}}{}
\textcolor{teal}{\mytextsc{noun}} \hspace{4pt} Tone: LM+\#H / L.
\textcolor{Sepia}{\selectlanguage{english}Female muntjac.} \zh{母麂子。}  \zh{量词}: \textcolor{darkblue}{\textbf{\ipa{mi˩}}}  \mytextsc{clf}: \textcolor{darkblue}{\textbf{\ipa{mi˩}}} 
\lhead{\firstmark}
\rhead{\botmark}

\subsection{\hspace{-0.5cm} {\Large \textcolor{darkblue}{\textbf{\ipa{tɕʰɯ˩pʰv̩\#˥}}}}\hspace{0.5cm}[\kern2pt{\textcolor{darkblue}{\textbf{\ipa{tɕʰɯ˩pʰv̩˥}}}}\kern2pt]} \hypertarget{ts£\string_hM\string_Bp\string_hv\string_=\#\string_T1}{}
\markboth{\textcolor{darkblue}{\textbf{\ipa{tɕʰɯ˩pʰv̩\#˥}}}}{}
\textcolor{teal}{\mytextsc{noun}} \hspace{4pt} Tone: LM+\#H / L.
\textcolor{Sepia}{\selectlanguage{english}Male muntjac.} \zh{公麂子。}  \zh{量词}: \textcolor{darkblue}{\textbf{\ipa{mi˩}}}  \mytextsc{clf}: \textcolor{darkblue}{\textbf{\ipa{mi˩}}} 
\lhead{\firstmark}
\rhead{\botmark}

\subsection{\hspace{-0.5cm} {\Large \textcolor{darkblue}{\textbf{\ipa{tɕʰɯ˩-ʁo˩-tɕʰɯ˥!}}}}\hspace{0.5cm}[\kern2pt{\textcolor{darkblue}{\textbf{\ipa{xxxx non-correspondance entre le nombre de morphèmes et le nombre de tons de morphèmes}}}}\kern2pt]} \hypertarget{ts£\string_hM\string_B-Ro\string_B-ts£\string_hM\string_T!1}{}
\markboth{\textcolor{darkblue}{\textbf{\ipa{tɕʰɯ˩-ʁo˩-tɕʰɯ˥!}}}}{}
\textcolor{teal}{\mytextsc{adverb(ial)}} \hspace{4pt} Tone: L+H\#.
\textcolor{Sepia}{\selectlanguage{english}Bless you! (what one says when someone sneezes).} \zh{旁边的人打嚏喷时说的祝愿话。} 
\lhead{\firstmark}
\rhead{\botmark}

\subsection{\hspace{-0.5cm} {\Large \textcolor{darkblue}{\textbf{\ipa{tɕʰɯ˩\textasciitilde{}tɕʰɯ˧˥}}}}\hspace{0.5cm}[\kern2pt{\textcolor{darkblue}{\textbf{\ipa{tɕʰɯ˧tɕʰɯ˧˥}}}}\kern2pt]} \hypertarget{ts£\string_hM\string_B~ts£\string_hM\string_M\string_T1}{}
\markboth{\textcolor{darkblue}{\textbf{\ipa{tɕʰɯ˩\textasciitilde{}tɕʰɯ˧˥}}}}{}
\textcolor{teal}{\mytextsc{verb}} \hspace{4pt} Tone: MH.
\textcolor{Sepia}{\selectlanguage{english}To suck.} \zh{吸吮。}  ¶ \textcolor{darkblue}{\textbf{\ipa{lo˩mi˧ tɕʰi˩\textasciitilde{}tɕʰi˩}}} \textcolor{Sepia}{\selectlanguage{english}to suck one's thumb} \zh{吮拇指}  

\lhead{\firstmark}
\rhead{\botmark}

\subsection{\hspace{-0.5cm} {\Large \textcolor{darkblue}{\textbf{\ipa{tɕʰɯ˩zo\#˥}}}}\hspace{0.5cm}[\kern2pt{\textcolor{darkblue}{\textbf{\ipa{tɕʰɯ˩zo˥}}}}\kern2pt]} \hypertarget{ts£\string_hM\string_Bzo\#\string_T1}{}
\markboth{\textcolor{darkblue}{\textbf{\ipa{tɕʰɯ˩zo\#˥}}}}{}
\textcolor{teal}{\mytextsc{noun}} \hspace{4pt} Tone: LM+\#H / L.
\textcolor{Sepia}{\selectlanguage{english}Baby muntjac.} \zh{麂子崽子。}  \zh{量词}: \textcolor{darkblue}{\textbf{\ipa{ɭɯ˧}}}  \mytextsc{clf}: \textcolor{darkblue}{\textbf{\ipa{ɭɯ˧}}} 
\lhead{\firstmark}
\rhead{\botmark}

\subsection{\hspace{-0.5cm} {\Large \textcolor{darkblue}{\textbf{\ipa{tɕʰɯ˧˥}}}}\hspace{0.5cm}[\kern2pt{\textcolor{darkblue}{\textbf{\ipa{tɕʰɯ˩˥}}}}\kern2pt]} \hypertarget{ts£\string_hM\string_M\string_T1}{}
\markboth{\textcolor{darkblue}{\textbf{\ipa{tɕʰɯ˧˥}}}}{}
\textcolor{teal}{\mytextsc{noun}} \hspace{4pt} Tone: MH.
\textcolor{Sepia}{\selectlanguage{english}Lacquer, paint.} \zh{漆。}  Borrowing: Chinese  \zh{漆}
 ¶ \textcolor{darkblue}{\textbf{\ipa{tɕʰɯ˧ jɤ˥-zo˩-ho˩!}}} \textcolor{Sepia}{\selectlanguage{english}It's time to paint (the room, the house...)! / We're going to have to paint (the room, the house...)!} \zh{该刷漆了!}  

\lhead{\firstmark}
\rhead{\botmark}

\subsection{\hspace{-0.5cm} {\Large \textcolor{darkblue}{\textbf{\ipa{tɕʰɯ˧˥}}} \textsubscript{1}}\hspace{0.5cm}[\kern2pt{\textcolor{darkblue}{\textbf{\ipa{tɕʰɯ˧˥}}}}\kern2pt]} \hypertarget{ts£\string_hM\string_M\string_T1}{}
\markboth{\textcolor{darkblue}{\textbf{\ipa{tɕʰɯ˧˥}}} \textsubscript{1}}{}
\textcolor{teal}{\mytextsc{verb}} \hspace{4pt} Tone: MH.
\textcolor{Sepia}{\selectlanguage{english}To throw away.} \zh{扔(垃圾)。}  ¶ \textcolor{darkblue}{\textbf{\ipa{ɖæ˩˥ | tʰi˧-tɕʰɯ˧˥}}} \textcolor{Sepia}{\selectlanguage{english}to throw garbage} \zh{扔垃圾}  

\lhead{\firstmark}
\rhead{\botmark}

\subsection{\hspace{-0.5cm} {\Large \textcolor{darkblue}{\textbf{\ipa{tɕʰɯ˧˥}}} \textsubscript{2}}\hspace{0.5cm}[\kern2pt{\textcolor{darkblue}{\textbf{\ipa{tɕʰɯ˧˥}}}}\kern2pt]} \hypertarget{ts£\string_hM\string_M\string_T2}{}
\markboth{\textcolor{darkblue}{\textbf{\ipa{tɕʰɯ˧˥}}} \textsubscript{2}}{}
\textcolor{teal}{\mytextsc{verb}} \hspace{4pt} Tone: MH.
\textcolor{Sepia}{\selectlanguage{english}To spit.} \zh{吐(吐口水)。} 
\lhead{\firstmark}
\rhead{\botmark}

\subsection{\hspace{-0.5cm} {\Large \textcolor{darkblue}{\textbf{\ipa{tɕʰɯ˧˥}}} \textsubscript{3}}\hspace{0.5cm}[\kern2pt{\textcolor{darkblue}{\textbf{\ipa{tɕʰɯ˧˥}}}}\kern2pt]} \hypertarget{ts£\string_hM\string_M\string_T3}{}
\markboth{\textcolor{darkblue}{\textbf{\ipa{tɕʰɯ˧˥}}} \textsubscript{3}}{}
\textcolor{teal}{\mytextsc{verb}} \hspace{4pt} Tone: MH.
\textcolor{Sepia}{\selectlanguage{english}To lose, to misplace.} \zh{丢失、弄丢。}  ¶ \textcolor{darkblue}{\textbf{\ipa{le˧-tɕʰɯ˧-ze˥}}} \textcolor{Sepia}{\selectlanguage{english}\mytextsc{accomp} \string_ \mytextsc{pfv}} \zh{丢了}  
 ¶ \textcolor{darkblue}{\textbf{\ipa{le˧-tɕʰɯ˧-hɯ˥-ze˩!}}} \textcolor{Sepia}{\selectlanguage{english}It's lost!} \zh{丢掉了!}  

\lhead{\firstmark}
\rhead{\botmark}

\subsection{\hspace{-0.5cm} {\Large \textcolor{darkblue}{\textbf{\ipa{tɕʰɯ˧˥}}} \textsubscript{4}}\hspace{0.5cm}[\kern2pt{\textcolor{darkblue}{\textbf{\ipa{tɕʰɯ˧˥}}}}\kern2pt]} \hypertarget{ts£\string_hM\string_M\string_T4}{}
\markboth{\textcolor{darkblue}{\textbf{\ipa{tɕʰɯ˧˥}}} \textsubscript{4}}{}
\textcolor{teal}{\mytextsc{adjective}} \hspace{4pt} Tone: MH.
\textcolor{Sepia}{\selectlanguage{english}Anxious, worried.} \zh{担心。}  ¶ \textcolor{darkblue}{\textbf{\ipa{nv̩˩mi˩ tɕʰɯ˥}}} \textcolor{Sepia}{\selectlanguage{english}anxious, worried} \zh{担心}  

\lhead{\firstmark}
\rhead{\botmark}

\subsection{\hspace{-0.5cm} {\Large \textcolor{darkblue}{\textbf{\ipa{tɕʰɯ˧˥}}} \textsubscript{5}}\hspace{0.5cm}[\kern2pt{\textcolor{darkblue}{\textbf{\ipa{tɕʰɯ˧˥}}}}\kern2pt]} \hypertarget{ts£\string_hM\string_M\string_T5}{}
\markboth{\textcolor{darkblue}{\textbf{\ipa{tɕʰɯ˧˥}}} \textsubscript{5}}{}
\textcolor{teal}{\mytextsc{adjective}} \hspace{4pt} Tone: MH.
\textit{\textcolor{Sepia}{\selectlanguage{english}archaic}} [\zh{古语}] \textcolor{Sepia}{\selectlanguage{english}At ease, comfortable.} \zh{舒服。}  ¶ \textcolor{darkblue}{\textbf{\ipa{mɤ˧-tɕʰɯ˧-bi˥ / mɤ˧-tɕʰɯ˧˥ |-bi˩}}} \textcolor{Sepia}{\selectlanguage{english}even if one is in need, ...} \zh{虽然很贫穷,……}  
 ¶ \textcolor{darkblue}{\textbf{\ipa{mɤ˧-dʑo˧ mɤ˧-tɕʰɯ˧-ɻ̍˧-bi˥, | ɖwæ˩ mɤ˧-zo˧!}}} \textcolor{Sepia}{\selectlanguage{english}Even if one is in need, one should not worry! (because the Gods will do something to save us)} \zh{虽然穷,莫担心!(因为菩萨会救好人)}  
 ¶ \textcolor{darkblue}{\textbf{\ipa{hĩ˧ ʈʂʰɯ˧-v̩˧-dʑo˩, | ɖwæ˧˥ | mɤ˧-tɕʰɯ˧˥!}}} \textcolor{Sepia}{\selectlanguage{english}This person is really in need!} \zh{这个人,真的很穷!}  

\lhead{\firstmark}
\rhead{\botmark}

\subsection{\hspace{-0.5cm} {\Large \textcolor{darkblue}{\textbf{\ipa{tɕʰɯ˩˥}}}}\hspace{0.5cm}[\kern2pt{\textcolor{darkblue}{\textbf{\ipa{tɕʰɯ˥}}}}\kern2pt]} \hypertarget{ts£\string_hM\string_B\string_T1}{}
\markboth{\textcolor{darkblue}{\textbf{\ipa{tɕʰɯ˩˥}}}}{}
\textcolor{teal}{\mytextsc{noun}} \hspace{4pt} Tone: LH.
\textcolor{Sepia}{\selectlanguage{english}Muntjac.} \zh{麂子。}  ¶ \textcolor{darkblue}{\textbf{\ipa{tɕʰɯ˩ hwæ˧-ze˩}}} \textcolor{Sepia}{\selectlanguage{english}...has bought (a/the) muntjac} \zh{买麂子}  
 ¶ \textcolor{darkblue}{\textbf{\ipa{tɕʰɯ˩ dzɯ˩-ze˥}}} \textcolor{Sepia}{\selectlanguage{english}...has eaten muntjac} \zh{吃了麂子}  
 \zh{量词}: \textcolor{darkblue}{\textbf{\ipa{pʰo˧˥}}}  \mytextsc{clf}: \textcolor{darkblue}{\textbf{\ipa{pʰo˧˥}}} 
\lhead{\firstmark}
\rhead{\botmark}

\newpage
\section*{\centering- \textcolor{darkblue}{\textbf{\ipa{ts}}} -}
\subsection{\hspace{-0.5cm} {\Large \textcolor{darkblue}{\textbf{\ipa{tsɑ˧}}}}\hspace{0.5cm}[\kern2pt{\textcolor{darkblue}{\textbf{\ipa{tsɑ˩˥}}}}\kern2pt]} \hypertarget{tsA\string_M1}{}
\markboth{\textcolor{darkblue}{\textbf{\ipa{tsɑ˧}}}}{}
\textcolor{teal}{\mytextsc{adjective}} \hspace{4pt} Tone: M.
\textcolor{Sepia}{\selectlanguage{english}Busy.} \zh{忙。}  ¶ \textcolor{darkblue}{\textbf{\ipa{ɖwæ˧˥ | tsɑ˧}}} \textcolor{Sepia}{\selectlanguage{english}\mytextsc{intensive}.very: very busy} \zh{很忙}  
 ¶ \textcolor{darkblue}{\textbf{\ipa{tsɑ˧ | ʐwæ˩˥}}} \textcolor{Sepia}{\selectlanguage{english}extremely busy} \zh{非常忙}  

\lhead{\firstmark}
\rhead{\botmark}

\subsection{\hspace{-0.5cm} {\Large \textcolor{darkblue}{\textbf{\ipa{tsɑ˧bɤ˧}}}}\hspace{0.5cm}[\kern2pt{\textcolor{darkblue}{\textbf{\ipa{tsɑ˧bɤ˧˥}}}}\kern2pt]} \hypertarget{tsA\string_Mb7\string_M1}{}
\markboth{\textcolor{darkblue}{\textbf{\ipa{tsɑ˧bɤ˧}}}}{}
\textcolor{teal}{\mytextsc{noun}} \hspace{4pt} Tone: M.
\textcolor{Sepia}{\selectlanguage{english}Powder; flour.} \zh{糌粑、面粉、粉、粉末。}  Borrowing: Tibetan  rtsam pa
 ¶ \textcolor{darkblue}{\textbf{\ipa{qʰɑ˧dze˧-tsɑ˩bɤ˩}}} \textcolor{Sepia}{\selectlanguage{english}sweetcorn flour} \zh{玉米粉}  
 ¶ \textcolor{darkblue}{\textbf{\ipa{dze˧ɭɯ˧-tsɑ˩bɤ˩}}} \textcolor{Sepia}{\selectlanguage{english}wheat flour} \zh{小麦面}  
 ¶ \textcolor{darkblue}{\textbf{\ipa{lv̩˧mi˧-tsɑ˩bɤ˩}}} \textcolor{Sepia}{\selectlanguage{english}stone powder, powdered stone} \zh{石头粉、被磨成粉的石头}  
 ¶ \textcolor{darkblue}{\textbf{\ipa{tsʰi˧zi˧-tsɑ˧bɤ˥}}} \textcolor{Sepia}{\selectlanguage{english}highland barley flour} \zh{青稞面粉}  
 ¶ \textcolor{darkblue}{\textbf{\ipa{mv̩˩zɯ˩-tsɑ˩bɤ˥}}} \textcolor{Sepia}{\selectlanguage{english}oatmeal flour} \zh{燕麦面粉}  
 ¶ \textcolor{darkblue}{\textbf{\ipa{jɤ˩jo˧-tsɑ˧bɤ˥}}} \textcolor{Sepia}{\selectlanguage{english}potato flour (elicited combination)} \zh{洋芋面粉}  
 ¶ \textcolor{darkblue}{\textbf{\ipa{nv̩˩ɭɯ˧-tsɑ˩bɤ˩}}} \textcolor{Sepia}{\selectlanguage{english}soy flour} \zh{黄豆面粉}  
 ¶ \textcolor{darkblue}{\textbf{\ipa{læ˧tsɯ˥-tsɑ˩bɤ˩}}} \textcolor{Sepia}{\selectlanguage{english}chili powder} \zh{辣椒粉}  
 ¶ \textcolor{darkblue}{\textbf{\ipa{ʈʂʰæ˧ɣɯ˧-tsɑ˧bɤ˥}}} \textcolor{Sepia}{\selectlanguage{english}Medicine powder, medicine in powder form. For instance: the disinfectant commonly used in Yongning at the time of fieldwork, of the brand \zh{云南白药}} \zh{药粉,粉状药品。如:“云南白药”消毒粉。}  
 ¶ \textcolor{darkblue}{\textbf{\ipa{ʂæ˩ɻ̃˩-tsɑ˩bɤ˥}}} \textcolor{Sepia}{\selectlanguage{english}bone powder} \zh{骨头粉}  
 ¶ \textcolor{darkblue}{\textbf{\ipa{jɤ˧-tsɑ˧bɤ˧}}} \textcolor{Sepia}{\selectlanguage{english}tobacco powder} \zh{烟草粉}  
 ¶ \textcolor{darkblue}{\textbf{\ipa{jɤ˧ɻ̃˧-tsɑ˧bɤ˥}}} \textcolor{Sepia}{\selectlanguage{english}tobacco powder} \zh{烟草粉}  
 ¶ \textcolor{darkblue}{\textbf{\ipa{dze˩-tsɑ˩bɤ˥}}} \textcolor{Sepia}{\selectlanguage{english}Szechuan pepper powder} \zh{花椒粉}  
 ¶ \textcolor{darkblue}{\textbf{\ipa{dze˧-tsɑ˧bɤ˥}}} \textcolor{Sepia}{\selectlanguage{english}powdered sugar, granulated sugar} \zh{砂糖}  
 ¶ \textcolor{darkblue}{\textbf{\ipa{tsɑ˧bɤ˧ mɤ˩}}} \textcolor{Sepia}{\selectlanguage{english}to eat dry flour (made of grilled barley)}  
 ¶ \textcolor{darkblue}{\textbf{\ipa{tsɑ˧bɤ˧ gv̩˩}}} \textcolor{Sepia}{\selectlanguage{english}to cook tsamba (grilled flour)} \zh{炒糌粑,制作糌粑}  

\lhead{\firstmark}
\rhead{\botmark}

\subsection{\hspace{-0.5cm} {\Large \textcolor{darkblue}{\textbf{\ipa{tsɑ˧ʐo˩}}}}\hspace{0.5cm}[\kern2pt{\textcolor{darkblue}{\textbf{\ipa{tsɑ˧ʐo˩}}}}\kern2pt]} \hypertarget{tsA\string_Mz`o\string_B1}{}
\markboth{\textcolor{darkblue}{\textbf{\ipa{tsɑ˧ʐo˩}}}}{}
\textcolor{teal}{\mytextsc{adjective}} \hspace{4pt} Tone: L\#.
\textcolor{Sepia}{\selectlanguage{english}Diligent, conscientious.} \zh{勤快。} 
\lhead{\firstmark}
\rhead{\botmark}

\subsection{\hspace{-0.5cm} {\Large \textcolor{darkblue}{\textbf{\ipa{tsɑ˩}}}}\hspace{0.5cm}[\kern2pt{\textcolor{darkblue}{\textbf{\ipa{tsɑ˥}}}}\kern2pt]} \hypertarget{tsA\string_B1}{}
\markboth{\textcolor{darkblue}{\textbf{\ipa{tsɑ˩}}}}{}
\textcolor{teal}{\mytextsc{verb}} \hspace{4pt} Tone: L.
\textcolor{Sepia}{\selectlanguage{english}To wink (as a discreet sign of mutual understanding).} \zh{眨眼。}  ¶ \textcolor{darkblue}{\textbf{\ipa{ʈʂʰɯ˧ | njɤ˩ɭɯ˧ tsɑ˩\textasciitilde{}tsɑ˩-dʑo˩!}}} \textcolor{Sepia}{\selectlanguage{english}\mytextsc{red}: (S)he is winking!} \zh{\mytextsc{重叠:他在眨眨眼!}}  
 ¶ \textcolor{darkblue}{\textbf{\ipa{ʈʂʰɯ˧ | njɤ˩ɭɯ˧ tsɑ˩-dʑo˩!}}} \textcolor{Sepia}{\selectlanguage{english}(S)he is winking!} \zh{他在眨眼!}  
 ¶ \textcolor{darkblue}{\textbf{\ipa{tsɑ˩\textasciitilde{}tsɑ˧˥}}} \textcolor{Sepia}{\selectlanguage{english}\mytextsc{red}} \zh{\mytextsc{重叠}}  
 ¶ \textcolor{darkblue}{\textbf{\ipa{mɤ˧-tsɑ˩\textasciitilde{}tsɑ˩}}} \textcolor{Sepia}{\selectlanguage{english}\mytextsc{neg} \mytextsc{red}} \zh{不眨眼}  
\textit{See:} \hyperlink{}{\textcolor{darkblue}{\textbf{\ipa{tsɯ˩pʰɤ˩}}}} 
\lhead{\firstmark}
\rhead{\botmark}

\subsection{\hspace{-0.5cm} {\Large \textcolor{darkblue}{\textbf{\ipa{tsɑ˩tɕi˩}}}}\hspace{0.5cm}[\kern2pt{\textcolor{darkblue}{\textbf{\ipa{tsɑ˩tɕi˩˥}}}}\kern2pt]} \hypertarget{tsA\string_Bts£i\string_B1}{}
\markboth{\textcolor{darkblue}{\textbf{\ipa{tsɑ˩tɕi˩}}}}{}
\textcolor{teal}{\mytextsc{noun}} \hspace{4pt} Tone: L.
\textcolor{Sepia}{\selectlanguage{english}Various mushrooms, mixed mushrooms.} \zh{杂菌(汉语借词)。}  Borrowing: Chinese  \zh{杂菌}
 \zh{量词}: \textcolor{darkblue}{\textbf{\ipa{ɭɯ˧}}} \textcolor{darkblue}{\textbf{\ipa{mo˧˥}}}  \mytextsc{clf}: \textcolor{darkblue}{\textbf{\ipa{ɭɯ˧}}} \textcolor{darkblue}{\textbf{\ipa{mo˧˥}}} 
\lhead{\firstmark}
\rhead{\botmark}

\subsection{\hspace{-0.5cm} {\Large \textcolor{darkblue}{\textbf{\ipa{tsɑ˧˥}}} \textsubscript{1}}\hspace{0.5cm}[\kern2pt{\textcolor{darkblue}{\textbf{\ipa{tsɑ˥}}}}\kern2pt]} \hypertarget{tsA\string_M\string_T1}{}
\markboth{\textcolor{darkblue}{\textbf{\ipa{tsɑ˧˥}}} \textsubscript{1}}{}
\textcolor{teal}{\mytextsc{verb}} \hspace{4pt} Tone: MH.
\ding{202} \textcolor{Sepia}{\selectlanguage{english}To kick, to smash (clods of earth).} \zh{打碎(坷拉),踢(一脚)。}  ¶ \textcolor{darkblue}{\textbf{\ipa{le˧-tsɑ˧-ze˥}}} \textcolor{Sepia}{\selectlanguage{english}\mytextsc{accomp}+\mytextsc{pfv}} \zh{\mytextsc{accomp}+\mytextsc{pfv}}  
 ¶ \textcolor{darkblue}{\textbf{\ipa{ʈʂe˧ tsɑ˩}}} \textcolor{Sepia}{\selectlanguage{english}to smash clods of earth, after plowing (with a hand instrument, such as a hoe)} \zh{打碎土坷垃}  
 ¶ \textcolor{darkblue}{\textbf{\ipa{ɖɯ˧-tsɑ˧ tʰi˥-tsɑ˩}}} \textcolor{Sepia}{\selectlanguage{english}to kick repeatedly, to give one kick after the other} \zh{踢了又踢}  
\ding{203} \textcolor{Sepia}{\selectlanguage{english}To row (a boat).} \zh{划(船)。}  ¶ \textcolor{darkblue}{\textbf{\ipa{tsɑ˧-hɯ˥-tsɑ˩-ɻ̍˩}}} \textcolor{Sepia}{\selectlanguage{english}to row in a sustained way, to row with great vigour} \zh{用力地划船、一直划船}  

\lhead{\firstmark}
\rhead{\botmark}

\subsection{\hspace{-0.5cm} {\Large \textcolor{darkblue}{\textbf{\ipa{tsɑ˧˥}}} \textsubscript{2}}\hspace{0.5cm}[\kern2pt{\textcolor{darkblue}{\textbf{\ipa{tsɑ˧˥}}}}\kern2pt]} \hypertarget{tsA\string_M\string_T2}{}
\markboth{\textcolor{darkblue}{\textbf{\ipa{tsɑ˧˥}}} \textsubscript{2}}{}
\textcolor{teal}{\mytextsc{verb}} \hspace{4pt} Tone: MH.
\textcolor{Sepia}{\selectlanguage{english}To lay (up/down), to place.} \zh{放置、放下。}  ¶ \textcolor{darkblue}{\textbf{\ipa{mv̩˩tɕo˧ tsɑ˧˥}}} \textcolor{Sepia}{\selectlanguage{english}to lay down, to put down on the floor} \zh{放下、放在地上}  

\lhead{\firstmark}
\rhead{\botmark}

\subsection{\hspace{-0.5cm} {\Large \textcolor{darkblue}{\textbf{\ipa{‑tsæ˧}}}}\hspace{0.5cm}[\kern2pt{\textcolor{darkblue}{\textbf{\ipa{tsæ˥}}}}\kern2pt]} \hypertarget{‑ts\{\string_M1}{}
\markboth{\textcolor{darkblue}{\textbf{\ipa{‑tsæ˧}}}}{}
\textcolor{teal}{\mytextsc{suffix}} \hspace{4pt} Tone: M.
\textcolor{Sepia}{\selectlanguage{english}Causative.} \zh{\mytextsc{使动:让。}}  ¶ \textcolor{darkblue}{\textbf{\ipa{tʰi˧-dzɯ˥-kʰɯ˩-tsæ˩-ɲi˩!}}} \textcolor{Sepia}{\selectlanguage{english}(We) have to get her to eat! (Context: comment made by a family member when a young child refused to have a meal)} \zh{必须让她吃!(情景:一个小女孩拒绝吃饭,家人就说这句。)}  
 ¶ \textcolor{darkblue}{\textbf{\ipa{tʰi˧-ʐwɤ˩-kʰɯ˩-tsæ˩-ɲi˩!}}} \textcolor{Sepia}{\selectlanguage{english}(We) have to get (him/her) to talk! (Variant based on the preceding example)} \zh{必须让他说!(在以上例子的基础上编的句子)}  

\lhead{\firstmark}
\rhead{\botmark}

\subsection{\hspace{-0.5cm} {\Large \textcolor{darkblue}{\textbf{\ipa{tsæ˧qæ˥}}}}\hspace{0.5cm}[\kern2pt{\textcolor{darkblue}{\textbf{\ipa{tsæ˧qæ˥}}}}\kern2pt]} \hypertarget{ts\{\string_Mq\{\string_T1}{}
\markboth{\textcolor{darkblue}{\textbf{\ipa{tsæ˧qæ˥}}}}{}
\textcolor{teal}{\mytextsc{noun}} \hspace{4pt} Tone: H\#.
\ding{202} \textcolor{Sepia}{\selectlanguage{english}Hook.} \zh{钩子。}  \zh{量词}: \textcolor{darkblue}{\textbf{\ipa{nɑ˧}}} \textcolor{darkblue}{\textbf{\ipa{ɭɯ˧}}} \ding{203} \textcolor{Sepia}{\selectlanguage{english}Firing pin (of a gun).} \zh{撞针。}  \mytextsc{clf}: \textcolor{darkblue}{\textbf{\ipa{nɑ˧}}} \textcolor{darkblue}{\textbf{\ipa{ɭɯ˧}}} 
\lhead{\firstmark}
\rhead{\botmark}

\subsection{\hspace{-0.5cm} {\Large \textcolor{darkblue}{\textbf{\ipa{tse˧bæ˥}}}}\hspace{0.5cm}[\kern2pt{\textcolor{darkblue}{\textbf{\ipa{tse˧bæ˥}}}}\kern2pt]} \hypertarget{tse\string_Mb\{\string_T1}{}
\markboth{\textcolor{darkblue}{\textbf{\ipa{tse˧bæ˥}}}}{}
\textcolor{teal}{\mytextsc{noun}} \hspace{4pt} Tone: H\#.
\textcolor{Sepia}{\selectlanguage{english}Tinder.} \zh{火绒。}  \zh{量词}: \textcolor{darkblue}{\textbf{\ipa{kʰɯ˩}}}  \mytextsc{clf}: \textcolor{darkblue}{\textbf{\ipa{kʰɯ˩}}} 
\lhead{\firstmark}
\rhead{\botmark}

\subsection{\hspace{-0.5cm} {\Large \textcolor{darkblue}{\textbf{\ipa{tse˧bo\#˥}}}}\hspace{0.5cm}[\kern2pt{\textcolor{darkblue}{\textbf{\ipa{tse˧bo˧}}}}\kern2pt]} \hypertarget{tse\string_Mbo\#\string_T1}{}
\markboth{\textcolor{darkblue}{\textbf{\ipa{tse˧bo\#˥}}}}{}
\textcolor{teal}{\mytextsc{noun}} \hspace{4pt} Tone: \#H.
\textcolor{Sepia}{\selectlanguage{english}Small bell.} \zh{铃铛。}  \zh{量词}: \textcolor{darkblue}{\textbf{\ipa{ɭɯ˧}}}  \mytextsc{clf}: \textcolor{darkblue}{\textbf{\ipa{ɭɯ˧}}} 
\lhead{\firstmark}
\rhead{\botmark}

\subsection{\hspace{-0.5cm} {\Large \textcolor{darkblue}{\textbf{\ipa{tse˧di\#˥}}}}\hspace{0.5cm}[\kern2pt{\textcolor{darkblue}{\textbf{\ipa{tse˧di˧}}}}\kern2pt]} \hypertarget{tse\string_Mdi\#\string_T1}{}
\markboth{\textcolor{darkblue}{\textbf{\ipa{tse˧di\#˥}}}}{}
\textcolor{teal}{\mytextsc{noun}} \hspace{4pt} Tone: \#H.
\textcolor{Sepia}{\selectlanguage{english}Sandalwood, sandlewood.} \zh{檀香木、檀香、檀木。}  ¶ \textcolor{darkblue}{\textbf{\ipa{tse˧di˧-si\#˥}}} \textcolor{Sepia}{\selectlanguage{english}same meaning} \zh{同上}  

\lhead{\firstmark}
\rhead{\botmark}

\subsection{\hspace{-0.5cm} {\Large \textcolor{darkblue}{\textbf{\ipa{tse˧kʰo˩}}}}\hspace{0.5cm}[\kern2pt{\textcolor{darkblue}{\textbf{\ipa{tse˧kʰo˩}}}}\kern2pt]} \hypertarget{tse\string_Mk\string_ho\string_B1}{}
\markboth{\textcolor{darkblue}{\textbf{\ipa{tse˧kʰo˩}}}}{}
\textcolor{teal}{\mytextsc{noun}} \hspace{4pt} Tone: L\#.
\textcolor{Sepia}{\selectlanguage{english}Sanctuary (small sanctuary on the mountain).} \zh{佛龛。}  \zh{量词}: \textcolor{darkblue}{\textbf{\ipa{ɭɯ˧}}}  \mytextsc{clf}: \textcolor{darkblue}{\textbf{\ipa{ɭɯ˧}}} 
\lhead{\firstmark}
\rhead{\botmark}

\subsection{\hspace{-0.5cm} {\Large \textcolor{darkblue}{\textbf{\ipa{tse˧lv̩˥}}}}\hspace{0.5cm}[\kern2pt{\textcolor{darkblue}{\textbf{\ipa{tse˧lv̩˥}}}}\kern2pt]} \hypertarget{tse\string_Mlv\string_=\string_T1}{}
\markboth{\textcolor{darkblue}{\textbf{\ipa{tse˧lv̩˥}}}}{}
\textcolor{teal}{\mytextsc{noun}} \hspace{4pt} Tone: H\#.
\textcolor{Sepia}{\selectlanguage{english}Flint.} \zh{燧石。}  \zh{量词}: \textcolor{darkblue}{\textbf{\ipa{ɭɯ˧}}}  \mytextsc{clf}: \textcolor{darkblue}{\textbf{\ipa{ɭɯ˧}}} 
\lhead{\firstmark}
\rhead{\botmark}

\subsection{\hspace{-0.5cm} {\Large \textcolor{darkblue}{\textbf{\ipa{tse˧mi˥}}}}\hspace{0.5cm}[\kern2pt{\textcolor{darkblue}{\textbf{\ipa{tse˧mi˥}}}}\kern2pt]} \hypertarget{tse\string_Mmi\string_T1}{}
\markboth{\textcolor{darkblue}{\textbf{\ipa{tse˧mi˥}}}}{}
\textcolor{teal}{\mytextsc{noun}} \hspace{4pt} Tone: H\#.
\textcolor{Sepia}{\selectlanguage{english}Lighter.} \zh{火镰。}  \zh{量词}: \textcolor{darkblue}{\textbf{\ipa{nɑ˧}}}  \mytextsc{clf}: \textcolor{darkblue}{\textbf{\ipa{nɑ˧}}} 
\lhead{\firstmark}
\rhead{\botmark}

\subsection{\hspace{-0.5cm} {\Large \textcolor{darkblue}{\textbf{\ipa{tse˧mi˥-dʑɯ˩ʁo˩}}}}\hspace{0.5cm}[\kern2pt{\textcolor{darkblue}{\textbf{\ipa{tse˧mi˥dʑɯ˩ʁo˩}}}}\kern2pt]} \hypertarget{tse\string_Mmi\string_T-dz£M\string_BRo\string_B1}{}
\markboth{\textcolor{darkblue}{\textbf{\ipa{tse˧mi˥-dʑɯ˩ʁo˩}}}}{}
\textcolor{teal}{\mytextsc{noun}} \hspace{4pt} Tone: H\#-L.
\textcolor{Sepia}{\selectlanguage{english}The village of Wenquan, in the plain of Yongning, where hot springs are located, hence the Chinese name Wenquan, 'hot springs'.} \zh{温泉乡的主要村落。} 
\lhead{\firstmark}
\rhead{\botmark}

\subsection{\hspace{-0.5cm} {\Large \textcolor{darkblue}{\textbf{\ipa{tse˩\textsubscript{a}}}} \textsubscript{1}}\hspace{0.5cm}[\kern2pt{\textcolor{darkblue}{\textbf{\ipa{tse˩˥}}}}\kern2pt]} \hypertarget{tse\string_Ba1}{}
\markboth{\textcolor{darkblue}{\textbf{\ipa{tse˩\textsubscript{a}}}} \textsubscript{1}}{}
\textcolor{teal}{\mytextsc{verb}} \hspace{4pt} Tone: L\textsubscript{a}.
\textcolor{Sepia}{\selectlanguage{english}To chase after; to pursue.} \zh{追赶。}  ¶ \textcolor{darkblue}{\textbf{\ipa{hĩ˧ tse˥}}} \textcolor{Sepia}{\selectlanguage{english}to chase after someone} \zh{追赶某人}  

\lhead{\firstmark}
\rhead{\botmark}

\subsection{\hspace{-0.5cm} {\Large \textcolor{darkblue}{\textbf{\ipa{tse˩\textsubscript{a}}}} \textsubscript{2}}\hspace{0.5cm}[\kern2pt{\textcolor{darkblue}{\textbf{\ipa{tse˩˥}}}}\kern2pt]} \hypertarget{tse\string_Ba2}{}
\markboth{\textcolor{darkblue}{\textbf{\ipa{tse˩\textsubscript{a}}}} \textsubscript{2}}{}
\textcolor{teal}{\mytextsc{verb}} \hspace{4pt} Tone: L\textsubscript{a}.
\textcolor{Sepia}{\selectlanguage{english}To float.} \zh{漂浮 (浮在水上)。}  ¶ \textcolor{darkblue}{\textbf{\ipa{gɤ˩tse˧}}} \textcolor{Sepia}{\selectlanguage{english}as above: to float} \zh{同上:漂浮 (浮在水上)}  
 ¶ \textcolor{darkblue}{\textbf{\ipa{ɖɯ˧-tse˧\textasciitilde{}tse˥-ɻ̍˩}}} \textcolor{Sepia}{\selectlanguage{english}\mytextsc{delimitative} \string_ \mytextsc{red} \mytextsc{inceptive}} \zh{\mytextsc{delimitative} \string_ \mytextsc{red} \mytextsc{inceptive}}  
 ¶ \textcolor{darkblue}{\textbf{\ipa{dʑɯ˩ʁo˩˥ | tʰi˧-tse˩ (-dʑo˩)}}} \textcolor{Sepia}{\selectlanguage{english}to float (in a torrent) on the mountain (e.g. timber is floated downstream)} \zh{让木头漂到下游}  
 ¶ \textcolor{darkblue}{\textbf{\ipa{dʑɯ˩ʁo˩ tse˧}}} \textcolor{Sepia}{\selectlanguage{english}as above: to float down from the mountain} \zh{同上:让木头漂到下游}  

\lhead{\firstmark}
\rhead{\botmark}

\subsection{\hspace{-0.5cm} {\Large \textcolor{darkblue}{\textbf{\ipa{tse˩\textsubscript{a}}}} \textsubscript{3}}\hspace{0.5cm}[\kern2pt{\textcolor{darkblue}{\textbf{\ipa{tse˩˥}}}}\kern2pt]} \hypertarget{tse\string_Ba3}{}
\markboth{\textcolor{darkblue}{\textbf{\ipa{tse˩\textsubscript{a}}}} \textsubscript{3}}{}
\textcolor{teal}{\mytextsc{verb}} \hspace{4pt} Tone: L\textsubscript{a}.
\textcolor{Sepia}{\selectlanguage{english}To lock.} \zh{锁门。}  ¶ \textcolor{darkblue}{\textbf{\ipa{kʰi˧ tse˥(-ze˩) / kʰi˧ tʰi˧-tse˩}}} \textcolor{Sepia}{\selectlanguage{english}to lock the door} \zh{锁门}  

\lhead{\firstmark}
\rhead{\botmark}

\subsection{\hspace{-0.5cm} {\Large \textcolor{darkblue}{\textbf{\ipa{tse˩pʰæ˧˥}}}}\hspace{0.5cm}[\kern2pt{\textcolor{darkblue}{\textbf{\ipa{tse˩pʰæ˧˥}}}}\kern2pt]} \hypertarget{tse\string_Bp\string_h\{\string_M\string_T1}{}
\markboth{\textcolor{darkblue}{\textbf{\ipa{tse˩pʰæ˧˥}}}}{}
\textcolor{teal}{\mytextsc{noun}} \hspace{4pt} Tone: LM+MH\#.
\textcolor{Sepia}{\selectlanguage{english}Coins of the imperial times.} \zh{民国之前的货币。}  ¶ \textcolor{darkblue}{\textbf{\ipa{æ˧-tse˥pʰæ˩}}} \textcolor{Sepia}{\selectlanguage{english}bronze coins of the imperial times} \zh{民国之前的铜币}  
 \zh{量词}: \textcolor{darkblue}{\textbf{\ipa{pʰæ˧˥}}}  \mytextsc{clf}: \textcolor{darkblue}{\textbf{\ipa{pʰæ˧˥}}} 
\lhead{\firstmark}
\rhead{\botmark}

\subsection{\hspace{-0.5cm} {\Large \textcolor{darkblue}{\textbf{\ipa{tse˩qwæ˧˥}}}}\hspace{0.5cm}[\kern2pt{\textcolor{darkblue}{\textbf{\ipa{tse˩qwæ˧˥}}}}\kern2pt]} \hypertarget{tse\string_Bqw\{\string_M\string_T1}{}
\markboth{\textcolor{darkblue}{\textbf{\ipa{tse˩qwæ˧˥}}}}{}
\textcolor{teal}{\mytextsc{noun}} \hspace{4pt} Tone: LM+MH\#.
\textcolor{Sepia}{\selectlanguage{english}Key.} \zh{钥匙。}  \zh{量词}: \textcolor{darkblue}{\textbf{\ipa{ɭɯ˧}}}  \mytextsc{clf}: \textcolor{darkblue}{\textbf{\ipa{ɭɯ˧}}} 
\lhead{\firstmark}
\rhead{\botmark}

\subsection{\hspace{-0.5cm} {\Large \textcolor{darkblue}{\textbf{\ipa{tse˩tɑ˧˥}}}}\hspace{0.5cm}[\kern2pt{\textcolor{darkblue}{\textbf{\ipa{tse˩tɑ˧˥}}}}\kern2pt]} \hypertarget{tse\string_BtA\string_M\string_T1}{}
\markboth{\textcolor{darkblue}{\textbf{\ipa{tse˩tɑ˧˥}}}}{}
\textcolor{teal}{\mytextsc{noun}} \hspace{4pt} Tone: LM+MH\#.
\textcolor{Sepia}{\selectlanguage{english}Scissors.} \zh{剪刀。}  \zh{量词}: \textcolor{darkblue}{\textbf{\ipa{nɑ˧}}}  \mytextsc{clf}: \textcolor{darkblue}{\textbf{\ipa{nɑ˧}}} 
\lhead{\firstmark}
\rhead{\botmark}

\subsection{\hspace{-0.5cm} {\Large \textcolor{darkblue}{\textbf{\ipa{tse˩ʈʂʰv̩˩}}}}\hspace{0.5cm}[\kern2pt{\textcolor{darkblue}{\textbf{\ipa{tse˩ʈʂʰv̩˩˥}}}}\kern2pt]} \hypertarget{tse\string_Bt`s`\string_hv\string_=\string_B1}{}
\markboth{\textcolor{darkblue}{\textbf{\ipa{tse˩ʈʂʰv̩˩}}}}{}
\textcolor{teal}{\mytextsc{noun}} \hspace{4pt} Tone: L.
\textcolor{Sepia}{\selectlanguage{english}Derogatory term of address for a dog.} \zh{骂狗的话。} \textit{See:} \hyperlink{}{\textcolor{darkblue}{\textbf{\ipa{tse˩ʈʂʰv̩˩-kʰv̩˥}}}} 
\lhead{\firstmark}
\rhead{\botmark}

\subsection{\hspace{-0.5cm} {\Large \textcolor{darkblue}{\textbf{\ipa{tse˩ʈʂʰv̩˩-kʰv̩˥}}}}\hspace{0.5cm}[\kern2pt{\textcolor{darkblue}{\textbf{\ipa{xxxx non-correspondance entre le nombre de morphèmes et le nombre de tons de morphèmes}}}}\kern2pt]} \hypertarget{tse\string_Bt`s`\string_hv\string_=\string_B-k\string_hv\string_=\string_T1}{}
\markboth{\textcolor{darkblue}{\textbf{\ipa{tse˩ʈʂʰv̩˩-kʰv̩˥}}}}{}
\textcolor{teal}{\mytextsc{noun}} \hspace{4pt} Tone: L+H\#.
\textcolor{Sepia}{\selectlanguage{english}Derogatory term of address for a dog.} \zh{骂狗的话。}  ¶ \textcolor{darkblue}{\textbf{\ipa{tse˩ʈʂʰv̩˩-kʰv̩˧ ! | mv̩˩tɕo˧ se˥ !}}} \textcolor{Sepia}{\selectlanguage{english}Come down, you damn dog!} \zh{你这坏狗,下去!}  
\textit{See:} \hyperlink{}{\textcolor{darkblue}{\textbf{\ipa{tse˩ʈʂʰv̩˩}}}} 
\lhead{\firstmark}
\rhead{\botmark}

\subsection{\hspace{-0.5cm} {\Large \textcolor{darkblue}{\textbf{\ipa{tse˩˥}}}}\hspace{0.5cm}[\kern2pt{\textcolor{darkblue}{\textbf{\ipa{tse˩˥}}}}\kern2pt]} \hypertarget{tse\string_B\string_T1}{}
\markboth{\textcolor{darkblue}{\textbf{\ipa{tse˩˥}}}}{}
\textcolor{teal}{\mytextsc{noun}} \hspace{4pt} Tone: LH.
\textcolor{Sepia}{\selectlanguage{english}Lock.} \zh{锁。}  ¶ \textcolor{darkblue}{\textbf{\ipa{æ˧tse˥}}} \textcolor{Sepia}{\selectlanguage{english}bronze lock} \zh{铜锁}  
 \zh{量词}: \textcolor{darkblue}{\textbf{\ipa{nɑ˧}}}  \mytextsc{clf}: \textcolor{darkblue}{\textbf{\ipa{nɑ˧}}} 
\lhead{\firstmark}
\rhead{\botmark}

\subsection{\hspace{-0.5cm} {\Large \textcolor{darkblue}{\textbf{\ipa{tsɤ˧}}} \textsubscript{1}}\hspace{0.5cm}[\kern2pt{\textcolor{darkblue}{\textbf{\ipa{tsɤ˥}}}}\kern2pt]} \hypertarget{ts7\string_M1}{}
\markboth{\textcolor{darkblue}{\textbf{\ipa{tsɤ˧}}} \textsubscript{1}}{}
\textcolor{teal}{\mytextsc{verb}} \hspace{4pt} Tone: M intrans.
\textcolor{Sepia}{\selectlanguage{english}To become, tu turn into; to be.} \zh{形成,变成。}  ¶ \textcolor{darkblue}{\textbf{\ipa{sɯ˧pv̩˧-sɯ˥nɑ˩-ʈʂʰɯ˩ | ə˧dzɤ˧\textasciitilde{}dzɤ˥-zo˩ | pʰi˧li˩ tsɤ˩-ɲi˩-kv̩˩-tsɯ˩ | -mv̩˩!}}} \textcolor{Sepia}{\selectlanguage{english}The caterpillar gradually becomes a butterfly, doesn't it!} \zh{毛虫能慢慢变成蝴蝶,不是吗?}  
 ¶ \textcolor{darkblue}{\textbf{\ipa{ɖɯ˧-bæ˧ mɤ˧-tsɤ˧}}} \textcolor{Sepia}{\selectlanguage{english}It's not the same} \zh{有区别、不一样}  

\lhead{\firstmark}
\rhead{\botmark}

\subsection{\hspace{-0.5cm} {\Large \textcolor{darkblue}{\textbf{\ipa{tsɤ˧}}} \textsubscript{2}}\hspace{0.5cm}[\kern2pt{\textcolor{darkblue}{\textbf{\ipa{tsɤ˥}}}}\kern2pt]} \hypertarget{ts7\string_M2}{}
\markboth{\textcolor{darkblue}{\textbf{\ipa{tsɤ˧}}} \textsubscript{2}}{}
\textcolor{teal}{\mytextsc{adjective}} \hspace{4pt} Tone: M.
\textcolor{Sepia}{\selectlanguage{english}Suitable, correct.} \zh{对,合适,成。}  ¶ \textcolor{darkblue}{\textbf{\ipa{(le˧-)tsɤ˧-ze˧!}}} \textcolor{Sepia}{\selectlanguage{english}It's okay! / It's arranged! / Things have been made good!} \zh{好了! / 弄好了! / 成!}  
 ¶ \textcolor{darkblue}{\textbf{\ipa{tsɤ˧-ʝi˧!}}} \textcolor{Sepia}{\selectlanguage{english}Okay, fine! (Indication of acquiescence to an instruction)} \zh{行! / 好的!(表示同意或接受命令)}  
 ¶ \textcolor{darkblue}{\textbf{\ipa{tsɤ˧ ɲi˥!}}} \textcolor{Sepia}{\selectlanguage{english}That's fine!} \zh{好的!}  
 ¶ \textcolor{darkblue}{\textbf{\ipa{no˧ | mɤ˧-bi˧ mɤ˧-tsɤ˧!}}} \textcolor{Sepia}{\selectlanguage{english}It wouldn't be right for you not to go!} \zh{你如果不去,就不对! =你不能不去!}  
 ¶ \textcolor{darkblue}{\textbf{\ipa{ʈʂʰɯ˧ | ɖɯ˧-pi˧˥ | mɤ˧-tsɤ˧!}}} \textcolor{Sepia}{\selectlanguage{english}He is not quite OK! / There's something wrong with him!} \zh{他有一点不对劲吧!}  
 ¶ \textcolor{darkblue}{\textbf{\ipa{tsɤ˧ mɤ˧-ʝi˧-ze˧!}}} \textcolor{Sepia}{\selectlanguage{english}It won't do! / It won't work! / It's no good!} \zh{不好了!/不行了!}  
 ¶ \textcolor{darkblue}{\textbf{\ipa{hĩ˧-ɳɯ˩ | le˧-so˩, | tsɤ˧!}}} \textcolor{Sepia}{\selectlanguage{english}When people teach you something, it's fortunate / it's good / it's an opportunity to seize! (Context: discussing the behaviour of someone who would not listen to good advice.)} \zh{人家教,是好事! / 人家教,是要珍惜的! / 有人愿意教你,是件好事!}  
 ¶ \textcolor{darkblue}{\textbf{\ipa{hĩ˧-ɳɯ˩ | le˧-so˩, | tsɤ˧-kv˧˥!}}} \textcolor{Sepia}{\selectlanguage{english}as above} \zh{同上}  

\lhead{\firstmark}
\rhead{\botmark}

\subsection{\hspace{-0.5cm} {\Large \textcolor{darkblue}{\textbf{\ipa{tsɤ˧}}} \textsubscript{3}}\hspace{0.5cm}[\kern2pt{\textcolor{darkblue}{\textbf{\ipa{tsɤ˥}}}}\kern2pt]} \hypertarget{ts7\string_M3}{}
\markboth{\textcolor{darkblue}{\textbf{\ipa{tsɤ˧}}} \textsubscript{3}}{}
\textcolor{teal}{\mytextsc{adjective}} \hspace{4pt} Tone: M.
\textcolor{Sepia}{\selectlanguage{english}Fine (powder).} \zh{细(粉状)。}  ¶ \textcolor{darkblue}{\textbf{\ipa{tsɑ˧bɤ˧ tsɤ\#˥}}} \textcolor{Sepia}{\selectlanguage{english}fine flour} \zh{细粮}  

\lhead{\firstmark}
\rhead{\botmark}

\subsection{\hspace{-0.5cm} {\Large \textcolor{darkblue}{\textbf{\ipa{tsɤ˧}}} \textsubscript{4}}\hspace{0.5cm}[\kern2pt{\textcolor{darkblue}{\textbf{\ipa{tsɤ˥}}}}\kern2pt]} \hypertarget{ts7\string_M4}{}
\markboth{\textcolor{darkblue}{\textbf{\ipa{tsɤ˧}}} \textsubscript{4}}{}
\textcolor{teal}{\mytextsc{adjective}} \hspace{4pt} Tone: M.
\textcolor{Sepia}{\selectlanguage{english}Greedy.} \zh{嘴馋。} \textit{See:} \hyperlink{}{\textcolor{darkblue}{\textbf{\ipa{tsɤ˧ʁo˧-tsʰi˧ʁo˥}}}} 
\lhead{\firstmark}
\rhead{\botmark}

\subsection{\hspace{-0.5cm} {\Large \textcolor{darkblue}{\textbf{\ipa{tsɤ˧di˧}}}}\hspace{0.5cm}[\kern2pt{\textcolor{darkblue}{\textbf{\ipa{tsɤ˩di˩˥}}}}\kern2pt]} \hypertarget{ts7\string_Mdi\string_M1}{}
\markboth{\textcolor{darkblue}{\textbf{\ipa{tsɤ˧di˧}}}}{}
\textcolor{teal}{\mytextsc{noun}} \hspace{4pt} Tone: M.
\textcolor{Sepia}{\selectlanguage{english}Sandalwood, sandlewood; a tall tree, not a shrub.} \zh{香木。} Local Chinese dialect:\zh{柏香。} ¶ \textcolor{darkblue}{\textbf{\ipa{tsɤ˧di˧-dzi˩}}} \textcolor{Sepia}{\selectlanguage{english}same meaning} \zh{同上}  
\textit{See:} \textcolor{darkblue}{\textbf{\ipa{ʁo˩kʰv˩}}} 
\lhead{\firstmark}
\rhead{\botmark}

\subsection{\hspace{-0.5cm} {\Large \textcolor{darkblue}{\textbf{\ipa{tsɤ˧ɖɯ˧}}}}\hspace{0.5cm}[\kern2pt{\textcolor{darkblue}{\textbf{\ipa{tsɤ˧ɖɯ˩}}}}\kern2pt]} \hypertarget{ts7\string_Md`M\string_M1}{}
\markboth{\textcolor{darkblue}{\textbf{\ipa{tsɤ˧ɖɯ˧}}}}{}
\textcolor{teal}{\mytextsc{verb}} \hspace{4pt} Tone: .
\textcolor{Sepia}{\selectlanguage{english}To give birth (cattle).} \zh{生崽子(牛类)。}  ¶ \textcolor{darkblue}{\textbf{\ipa{tsɤ˧ɖɯ˧-ze˩}}} \textcolor{Sepia}{\selectlanguage{english}\mytextsc{pfv}} \zh{生崽子了}  
 ¶ \textcolor{darkblue}{\textbf{\ipa{(dʑi˧mi˧) tsɤ˧ɖɯ˧-ze˩}}} \textcolor{Sepia}{\selectlanguage{english}(the water buffalo) has given birth.} \zh{水牛生崽子了。}  

\lhead{\firstmark}
\rhead{\botmark}

\subsection{\hspace{-0.5cm} {\Large \textcolor{darkblue}{\textbf{\ipa{tsɤ˧ʁo˧-tsʰi˧ʁo˥}}}}\hspace{0.5cm}[\kern2pt{\textcolor{darkblue}{\textbf{\ipa{xxxx non-correspondance entre le nombre de morphèmes et le nombre de tons de morphèmes}}}}\kern2pt]} \hypertarget{ts7\string_MRo\string_M-ts\string_hi\string_MRo\string_T1}{}
\markboth{\textcolor{darkblue}{\textbf{\ipa{tsɤ˧ʁo˧-tsʰi˧ʁo˥}}}}{}
\textcolor{teal}{\mytextsc{adjective}} \hspace{4pt} Tone: H\#.
\textcolor{Sepia}{\selectlanguage{english}Greedy.} \zh{馋。}  ¶ \textcolor{darkblue}{\textbf{\ipa{tsɤ˧ʁo˧-tsʰi˧ʁo˥ tsʰi˩}}} \textcolor{Sepia}{\selectlanguage{english}to be greedy} \zh{馋}  
\textit{See:} \hyperlink{}{\textcolor{darkblue}{\textbf{\ipa{tsɤ˧}}} \textsubscript{4}} 
\lhead{\firstmark}
\rhead{\botmark}

\subsection{\hspace{-0.5cm} {\Large \textcolor{darkblue}{\textbf{\ipa{tsi˥}}}}\hspace{0.5cm}[\kern2pt{\textcolor{darkblue}{\textbf{\ipa{tsi˥}}}}\kern2pt]} \hypertarget{tsi\string_T1}{}
\markboth{\textcolor{darkblue}{\textbf{\ipa{tsi˥}}}}{}
\textcolor{teal}{\mytextsc{noun}} \hspace{4pt} Tone: \#H.
\ding{202} \textcolor{Sepia}{\selectlanguage{english}Crack.} \zh{裂缝、缝隙。}  ¶ \textcolor{darkblue}{\textbf{\ipa{tsi˧ qʰwæ˧-ze˥!}}} \textcolor{Sepia}{\selectlanguage{english}A crack has appeared!} \zh{有了裂缝!}  
 ¶ \textcolor{darkblue}{\textbf{\ipa{tsi˧ hɯ˧-ze˧!}}} \textcolor{Sepia}{\selectlanguage{english}A crack has appeared!} \zh{有了裂缝!}  
 \zh{量词}: \textcolor{darkblue}{\textbf{\ipa{pʰæ˧˥}}} \ding{203} \textcolor{Sepia}{\selectlanguage{english}Stitch.} \zh{针脚。}  \mytextsc{clf}: \textcolor{darkblue}{\textbf{\ipa{pʰæ˧˥}}} 
\lhead{\firstmark}
\rhead{\botmark}

\subsection{\hspace{-0.5cm} {\Large \textcolor{darkblue}{\textbf{\ipa{tsi˧\textsubscript{a}}}}}\hspace{0.5cm}[\kern2pt{\textcolor{darkblue}{\textbf{\ipa{tsi˥}}}}\kern2pt]} \hypertarget{tsi\string_Ma1}{}
\markboth{\textcolor{darkblue}{\textbf{\ipa{tsi˧\textsubscript{a}}}}}{}
\textcolor{teal}{\mytextsc{adjective}} \hspace{4pt} Tone: M\textsubscript{a}.
\textcolor{Sepia}{\selectlanguage{english}Spicy.} \zh{辣。}  ¶ \textcolor{darkblue}{\textbf{\ipa{ʈʂʰɯ˧ tsi˧-hĩ˧ ɲi˥!}}} \textcolor{Sepia}{\selectlanguage{english}It's spicy!} \zh{这是辣的!}  

\lhead{\firstmark}
\rhead{\botmark}

\subsection{\hspace{-0.5cm} {\Large \textcolor{darkblue}{\textbf{\ipa{tsi˧\textsubscript{b}}}}}\hspace{0.5cm}[\kern2pt{\textcolor{darkblue}{\textbf{\ipa{tsi˥}}}}\kern2pt]} \hypertarget{tsi\string_Mb1}{}
\markboth{\textcolor{darkblue}{\textbf{\ipa{tsi˧\textsubscript{b}}}}}{}
\textcolor{teal}{\mytextsc{verb}} \hspace{4pt} Tone: M\textsubscript{b}.
\textcolor{Sepia}{\selectlanguage{english}To set up, to install.} \zh{安装。}  ¶ \textcolor{darkblue}{\textbf{\ipa{tso˧\textasciitilde{}tso˧ tsi˧}}} \textcolor{Sepia}{\selectlanguage{english}to set up something} \zh{安装东西}  
 ¶ \textcolor{darkblue}{\textbf{\ipa{tso˧\textasciitilde{}tso˧ | gɤ˩-tsi˧-ɻ̍˥}}} \textcolor{Sepia}{\selectlanguage{english}to set up something} \zh{安装东西}  
 ¶ \textcolor{darkblue}{\textbf{\ipa{gɤ˩-tsi˧ tʰi˧-tɕɯ˥}}} \textcolor{Sepia}{\selectlanguage{english}to set up (vertically)} \zh{立起来}  

\lhead{\firstmark}
\rhead{\botmark}

\subsection{\hspace{-0.5cm} {\Large \textcolor{darkblue}{\textbf{\ipa{tsi˧gi˥\$}}}}\hspace{0.5cm}[\kern2pt{\textcolor{darkblue}{\textbf{\ipa{tsi˧gi˥}}}}\kern2pt]} \hypertarget{tsi\string_Mgi\string_T\$1}{}
\markboth{\textcolor{darkblue}{\textbf{\ipa{tsi˧gi˥\$}}}}{}
\textcolor{teal}{\mytextsc{noun}} \hspace{4pt} Tone: H\$.
\textcolor{Sepia}{\selectlanguage{english}Crack, fissure.} \zh{缝隙,例如:墙上的。}  ¶ \textcolor{darkblue}{\textbf{\ipa{tsi˧gi˥ | ɖɯ˧-kʰwɤ˥}}} \textcolor{Sepia}{\selectlanguage{english}a fissure} \zh{一个缝隙}  
 ¶ \textcolor{darkblue}{\textbf{\ipa{tsi˧gi˥ | ɖɯ˧-kʰwɤ˧ tʰi˧-di˥}}} \textcolor{Sepia}{\selectlanguage{english}there is a fissure} \zh{有一个缝隙}  
 \zh{量词}: \textcolor{darkblue}{\textbf{\ipa{ɭɯ˧, kʰwɤ˥}}}  \mytextsc{clf}: \textcolor{darkblue}{\textbf{\ipa{ɭɯ˧, kʰwɤ˥}}} 
\lhead{\firstmark}
\rhead{\botmark}

\subsection{\hspace{-0.5cm} {\Large \textcolor{darkblue}{\textbf{\ipa{tsi˩\textsubscript{a}}}}}\hspace{0.5cm}[\kern2pt{\textcolor{darkblue}{\textbf{\ipa{tsi˩˥}}}}\kern2pt]} \hypertarget{tsi\string_Ba1}{}
\markboth{\textcolor{darkblue}{\textbf{\ipa{tsi˩\textsubscript{a}}}}}{}
\textcolor{teal}{\mytextsc{verb}} \hspace{4pt} Tone: L\textsubscript{a}.
\textcolor{Sepia}{\selectlanguage{english}To boil, to bring to a boil.} \zh{烧开。}  ¶ \textcolor{darkblue}{\textbf{\ipa{dʑɯ˧ | le˧-tsi˩-tʰv̩˩-ze˩!}}} \textcolor{Sepia}{\selectlanguage{english}The water is boiling!} \zh{水开了!}  
 ¶ \textcolor{darkblue}{\textbf{\ipa{dʑɯ˩ tsi˩-tʰv̩˩-ze˥!}}} \textcolor{Sepia}{\selectlanguage{english}The water is boiling!} \zh{水开了!}  
 ¶ \textcolor{darkblue}{\textbf{\ipa{mɤ˧-tsi˩-tʰv̩˩-sɯ˩!}}} \textcolor{Sepia}{\selectlanguage{english}It is not boiling yet!} \zh{还没有烧开!}  
 ¶ \textcolor{darkblue}{\textbf{\ipa{ɖɯ˧-tsi˩-tʰv̩˩-ɻ̍˩-kʰɯ˩}}} \textcolor{Sepia}{\selectlanguage{english}to leave to boil for a while} \zh{煮一会儿}  
 ¶ \textcolor{darkblue}{\textbf{\ipa{ɖɯ˧-tsi˧\textasciitilde{}tsi˥-ɻ̍˩ kʰɯ˩}}} \textcolor{Sepia}{\selectlanguage{english}to boil a while} \zh{煮一会儿}  

\lhead{\firstmark}
\rhead{\botmark}

\subsection{\hspace{-0.5cm} {\Large \textcolor{darkblue}{\textbf{\ipa{tsi˩ɭɯ˩}}}}\hspace{0.5cm}[\kern2pt{\textcolor{darkblue}{\textbf{\ipa{tsi˩ɭɯ˩˥}}}}\kern2pt]} \hypertarget{tsi\string_Bl\string_RM\string_B1}{}
\markboth{\textcolor{darkblue}{\textbf{\ipa{tsi˩ɭɯ˩}}}}{}
\textcolor{teal}{\mytextsc{noun}} \hspace{4pt} Tone: L.
\textcolor{Sepia}{\selectlanguage{english}A species of small bird.} \zh{一种小鸟。}  \zh{量词}: \textcolor{darkblue}{\textbf{\ipa{mi˩}}}  \mytextsc{clf}: \textcolor{darkblue}{\textbf{\ipa{mi˩}}} 
\lhead{\firstmark}
\rhead{\botmark}

\subsection{\hspace{-0.5cm} {\Large \textcolor{darkblue}{\textbf{\ipa{‑tso˧}}}}\hspace{0.5cm}[\kern2pt{\textcolor{darkblue}{\textbf{\ipa{tso˥}}}}\kern2pt]} \hypertarget{‑tso\string_M1}{}
\markboth{\textcolor{darkblue}{\textbf{\ipa{‑tso˧}}}}{}
\textcolor{teal}{\mytextsc{suffix}} \hspace{4pt} Tone: M.
\textcolor{Sepia}{\selectlanguage{english}\mytextsc{volitive}.} \zh{\mytextsc{意志。}}  ¶ \textcolor{darkblue}{\textbf{\ipa{dʑɤ˩bv̩˥-tso˩-ɲi˩-mæ˩!}}} \textcolor{Sepia}{\selectlanguage{english}Let's go and play!} \zh{玩一玩吧!}  
 ¶ \textcolor{darkblue}{\textbf{\ipa{ə˧tso˧ tv̩˧-tso˧-ɲi˥ ?}}} \textcolor{Sepia}{\selectlanguage{english}What do you plan to plant? / Which crop are you going to plant?} \zh{要种什么东西?}  
 ¶ \textcolor{darkblue}{\textbf{\ipa{ə˧tso˧ ʝi˧-bi˧-tso˧-ɲi˥?}}} \textcolor{Sepia}{\selectlanguage{english}What do you plan to do now?} \zh{要做什么了?}  

\lhead{\firstmark}
\rhead{\botmark}

\subsection{\hspace{-0.5cm} {\Large \textcolor{darkblue}{\textbf{\ipa{tso˧kʰwɤ\#˥}}}}\hspace{0.5cm}[\kern2pt{\textcolor{darkblue}{\textbf{\ipa{tso˧kʰwɤ˧}}}}\kern2pt]} \hypertarget{tso\string_Mk\string_hw7\#\string_T1}{}
\markboth{\textcolor{darkblue}{\textbf{\ipa{tso˧kʰwɤ\#˥}}}}{}
\textcolor{teal}{\mytextsc{noun}} \hspace{4pt} Tone: \#H.
\textcolor{Sepia}{\selectlanguage{english}Bag (of fabric or leather).} \zh{袋子。}  \zh{量词}: \textcolor{darkblue}{\textbf{\ipa{ɭɯ˧}}}  \mytextsc{clf}: \textcolor{darkblue}{\textbf{\ipa{ɭɯ˧}}} 
\lhead{\firstmark}
\rhead{\botmark}

\subsection{\hspace{-0.5cm} {\Large \textcolor{darkblue}{\textbf{\ipa{tso˧lo˧-mv̩˥tso˩}}}}\hspace{0.5cm}[\kern2pt{\textcolor{darkblue}{\textbf{\ipa{tso˧lo˧mv̩˥tso˩}}}}\kern2pt]} \hypertarget{tso\string_Mlo\string_M-mv\string_=\string_Ttso\string_B1}{}
\markboth{\textcolor{darkblue}{\textbf{\ipa{tso˧lo˧-mv̩˥tso˩}}}}{}
\textcolor{teal}{\mytextsc{noun}} \hspace{4pt} Tone: \#H-.
\textcolor{Sepia}{\selectlanguage{english}Tool; thing, object.} \zh{东西,工具。}  \zh{量词}: \textcolor{darkblue}{\textbf{\ipa{nɑ˧, ɭɯ˧}}}  \mytextsc{clf}: \textcolor{darkblue}{\textbf{\ipa{nɑ˧, ɭɯ˧}}} 
\lhead{\firstmark}
\rhead{\botmark}

\subsection{\hspace{-0.5cm} {\Large \textcolor{darkblue}{\textbf{\ipa{tso˧qwɤ˧}}}}\hspace{0.5cm}[\kern2pt{\textcolor{darkblue}{\textbf{\ipa{tso˧qwɤ˧}}}}\kern2pt]} \hypertarget{tso\string_Mqw7\string_M1}{}
\markboth{\textcolor{darkblue}{\textbf{\ipa{tso˧qwɤ˧}}}}{}
\textcolor{teal}{\mytextsc{noun}} \hspace{4pt} Tone: M.
\textcolor{Sepia}{\selectlanguage{english}Sleeping corner: a part of the main room where there is bedding; some people can sit there during meals or family reunions. Newborn babies sleep there. After a decease, corpses are placed on that bed.} \zh{小床角:主屋里面的一个角落,有床垫子。用餐、招待客人的时候,会有人在上面坐。刚出生的婴儿也在此处睡觉。人去世后,尸体先放在那个地方。}  \zh{量词}: \textcolor{darkblue}{\textbf{\ipa{ɭɯ˧}}}  \mytextsc{clf}: \textcolor{darkblue}{\textbf{\ipa{ɭɯ˧}}} 
\lhead{\firstmark}
\rhead{\botmark}

\subsection{\hspace{-0.5cm} {\Large \textcolor{darkblue}{\textbf{\ipa{tso˧tso\#˥}}}}\hspace{0.5cm}[\kern2pt{\textcolor{darkblue}{\textbf{\ipa{tso˧tso˧}}}}\kern2pt]} \hypertarget{tso\string_Mtso\#\string_T1}{}
\markboth{\textcolor{darkblue}{\textbf{\ipa{tso˧tso\#˥}}}}{}
\textcolor{teal}{\mytextsc{noun}} \hspace{4pt} Tone: \#H.
\textcolor{Sepia}{\selectlanguage{english}Thing, thingummy, stuff.} \zh{东西。}  ¶ \textcolor{darkblue}{\textbf{\ipa{tso˧\textasciitilde{}tso˧-zo˧\textasciitilde{}zo˧-mv̩˧\textasciitilde{}mv̩˥}}} \textcolor{Sepia}{\selectlanguage{english}thingummy, stuff} \zh{各种东西、各种各样乱七八糟东西}  
 ¶ \textcolor{darkblue}{\textbf{\ipa{tso˧\textasciitilde{}tso˧ hwæ˩}}} \textcolor{Sepia}{\selectlanguage{english}to buy things} \zh{买东西}  
 ¶ \textcolor{darkblue}{\textbf{\ipa{tso˧\textasciitilde{}tso˧ tɕʰi˧(-ze˩)}}} \textcolor{Sepia}{\selectlanguage{english}to sell things} \zh{卖东西}  
 ¶ \textcolor{darkblue}{\textbf{\ipa{tso˧\textasciitilde{}tso˧ dzɯ˧(-ze˩)}}} \textcolor{Sepia}{\selectlanguage{english}to eat things} \zh{吃东西}  
 ¶ \textcolor{darkblue}{\textbf{\ipa{tso˧\textasciitilde{}tso˧ dze˥}}} \textcolor{Sepia}{\selectlanguage{english}to cut things} \zh{切东西}  
 ¶ \textcolor{darkblue}{\textbf{\ipa{tso˧\textasciitilde{}tso˧ ʈʰɯ˩}}} \textcolor{Sepia}{\selectlanguage{english}to drink things} \zh{喝东西}  
 ¶ \textcolor{darkblue}{\textbf{\ipa{tso˧\textasciitilde{}tso˧ lɑ˩}}} \textcolor{Sepia}{\selectlanguage{english}to beat things} \zh{打东西}  
 \zh{量词}: \textcolor{darkblue}{\textbf{\ipa{nɑ˧, ɭɯ˧}}}  \mytextsc{clf}: \textcolor{darkblue}{\textbf{\ipa{nɑ˧, ɭɯ˧}}} 
\lhead{\firstmark}
\rhead{\botmark}

\subsection{\hspace{-0.5cm} {\Large \textcolor{darkblue}{\textbf{\ipa{tso˩\textsubscript{c}}}}}\hspace{0.5cm}[\kern2pt{\textcolor{darkblue}{\textbf{\ipa{tso˩˥}}}}\kern2pt]} \hypertarget{tso\string_Bc1}{}
\markboth{\textcolor{darkblue}{\textbf{\ipa{tso˩\textsubscript{c}}}}}{}
\textcolor{teal}{\mytextsc{classifier}} \hspace{4pt} Tone: L\textsubscript{c}.
\textcolor{Sepia}{\selectlanguage{english}Classifier for rooms.} \zh{量词:间(房间,分隔间,包间)。}  ¶ \textcolor{darkblue}{\textbf{\ipa{ʈʂʰɯ˧-tso˥}}} \textcolor{Sepia}{\selectlanguage{english}this room} \zh{这间}  

\lhead{\firstmark}
\rhead{\botmark}

\subsection{\hspace{-0.5cm} {\Large \textcolor{darkblue}{\textbf{\ipa{tso˩\textsubscript{a}}}}}\hspace{0.5cm}[\kern2pt{\textcolor{darkblue}{\textbf{\ipa{tso˩˥}}}}\kern2pt]} \hypertarget{tso\string_Ba1}{}
\markboth{\textcolor{darkblue}{\textbf{\ipa{tso˩\textsubscript{a}}}}}{}
\textcolor{teal}{\mytextsc{verb}} \hspace{4pt} Tone: L\textsubscript{a}.
\textcolor{Sepia}{\selectlanguage{english}To build a wall, a bridge….} \zh{砌(墙)、建(桥梁)。}  ¶ \textcolor{darkblue}{\textbf{\ipa{ɖʐɤ˩ tso˩}}} \textcolor{Sepia}{\selectlanguage{english}to build stairs} \zh{修一座楼梯}  
 ¶ \textcolor{darkblue}{\textbf{\ipa{dzo˩ tso˩}}} \textcolor{Sepia}{\selectlanguage{english}to build a bridge} \zh{修一座桥}  
 ¶ \textcolor{darkblue}{\textbf{\ipa{ɖɯ˧-tso˧\textasciitilde{}tso˥-ɻ̍˩}}} \textcolor{Sepia}{\selectlanguage{english}to build something} \zh{修东西}  

\lhead{\firstmark}
\rhead{\botmark}

\subsection{\hspace{-0.5cm} {\Large \textcolor{darkblue}{\textbf{\ipa{tso˩qʰv̩˩ɻ̍˥}}}}\hspace{0.5cm}[\kern2pt{\textcolor{darkblue}{\textbf{\ipa{tso˧qʰv̩˧ɻ̍˥}}}}\kern2pt]} \hypertarget{tso\string_Bq\string_hv\string_=\string_Br£`̍\string_T1}{}
\markboth{\textcolor{darkblue}{\textbf{\ipa{tso˩qʰv̩˩ɻ̍˥}}}}{}
\textcolor{teal}{\mytextsc{noun}} \hspace{4pt} Tone: H\#.
\textcolor{Sepia}{\selectlanguage{english}Porch, enclosed porch, vestibule: a narrow area between the door and the courtyard, covered by the roof (and hence sheltered from rain), and, in some houses, shut off from the coutyard by a wall with one door approximately in the middle. This porch is the area that one reaches when crossing the threshold, coming out from the main room. In the main consultant's house, where the porch is not enclosed, it is exposed to sunshine until the afternoon, and tasks such as chopping vegetables are carried out sitting in this area.} \zh{玄关、门厅。}  \zh{量词}: \textcolor{darkblue}{\textbf{\ipa{kʰwɤ˥}}}  \mytextsc{clf}: \textcolor{darkblue}{\textbf{\ipa{kʰwɤ˥}}} 
\lhead{\firstmark}
\rhead{\botmark}

\subsection{\hspace{-0.5cm} {\Large \textcolor{darkblue}{\textbf{\ipa{tso˩\textasciitilde{}tso˧˥}}}}\hspace{0.5cm}[\kern2pt{\textcolor{darkblue}{\textbf{\ipa{tso˩tso˧˥}}}}\kern2pt]} \hypertarget{tso\string_B~tso\string_M\string_T1}{}
\markboth{\textcolor{darkblue}{\textbf{\ipa{tso˩\textasciitilde{}tso˧˥}}}}{}
\textcolor{teal}{\mytextsc{verb}} \hspace{4pt} Tone: LM+MH\#.
\textcolor{Sepia}{\selectlanguage{english}To mix, to prepare (dog food).} \zh{拌好狗食。} 
\lhead{\firstmark}
\rhead{\botmark}

\subsection{\hspace{-0.5cm} {\Large \textcolor{darkblue}{\textbf{\ipa{tsɯ˥}}} \textsubscript{1}}\hspace{0.5cm}[\kern2pt{\textcolor{darkblue}{\textbf{\ipa{tsɯ˥}}}}\kern2pt]} \hypertarget{tsM\string_T1}{}
\markboth{\textcolor{darkblue}{\textbf{\ipa{tsɯ˥}}} \textsubscript{1}}{}
\textcolor{teal}{\mytextsc{verb}} \hspace{4pt} Tone: H.
\ding{202} \textcolor{Sepia}{\selectlanguage{english}To tie (with a rope).} \zh{绑、捆、栓。}  ¶ \textcolor{darkblue}{\textbf{\ipa{dʑi˧mi˧ tʰi˧-tsɯ˥}}} \textcolor{Sepia}{\selectlanguage{english}to tie a water buffalo} \zh{栓水牛}  
 ¶ \textcolor{darkblue}{\textbf{\ipa{tsɯ˧\textasciitilde{}tsɯ˧}}} \textcolor{Sepia}{\selectlanguage{english}\mytextsc{red}} \zh{\mytextsc{重叠}}  
 ¶ \textcolor{darkblue}{\textbf{\ipa{le˧-tsɯ˧\textasciitilde{}tsɯ˧}}} \textcolor{Sepia}{\selectlanguage{english}\mytextsc{accomp} \mytextsc{red}} \zh{\mytextsc{accomp} \mytextsc{red}}  
 ¶ \textcolor{darkblue}{\textbf{\ipa{tʰi˧-tsɯ˧\textasciitilde{}tsɯ˧}}} \textcolor{Sepia}{\selectlanguage{english}\mytextsc{dur} \mytextsc{red}} \zh{\mytextsc{dur} \mytextsc{red}}  
\ding{203} \textcolor{Sepia}{\selectlanguage{english}To hang oneself.} \zh{上吊自杀、缢。}  ¶ \textcolor{darkblue}{\textbf{\ipa{ʁæ˧tsɯ˧ le˧-ʂɯ˧ +ze˧}}} \textcolor{Sepia}{\selectlanguage{english}to hang oneself} \zh{上吊自杀、缢}  

\lhead{\firstmark}
\rhead{\botmark}

\subsection{\hspace{-0.5cm} {\Large \textcolor{darkblue}{\textbf{\ipa{tsɯ˥}}} \textsubscript{2}}\hspace{0.5cm}[\kern2pt{\textcolor{darkblue}{\textbf{\ipa{tsɯ˥}}}}\kern2pt]} \hypertarget{tsM\string_T2}{}
\markboth{\textcolor{darkblue}{\textbf{\ipa{tsɯ˥}}} \textsubscript{2}}{}
\textcolor{teal}{\mytextsc{verb}} \hspace{4pt} Tone: H.
\textcolor{Sepia}{\selectlanguage{english}Dredge for, fish for, scoop up out of water.} \zh{打捞。} 
\lhead{\firstmark}
\rhead{\botmark}

\subsection{\hspace{-0.5cm} {\Large \textcolor{darkblue}{\textbf{\ipa{tsɯ˧}}}}\hspace{0.5cm}[\kern2pt{\textcolor{darkblue}{\textbf{\ipa{tsɯ˥}}}}\kern2pt]} \hypertarget{tsM\string_M1}{}
\markboth{\textcolor{darkblue}{\textbf{\ipa{tsɯ˧}}}}{}
\textcolor{teal}{\mytextsc{noun}} \hspace{4pt} Tone: M.
\textcolor{Sepia}{\selectlanguage{english}Letter, Chinese character.} \zh{字。}  Borrowing: Chinese  \zh{字}

\lhead{\firstmark}
\rhead{\botmark}

\subsection{\hspace{-0.5cm} {\Large \textcolor{darkblue}{\textbf{\ipa{tsɯ˩\textsubscript{a}}}}}\hspace{0.5cm}[\kern2pt{\textcolor{darkblue}{\textbf{\ipa{tsɯ˥}}}}\kern2pt]} \hypertarget{tsM\string_Ba1}{}
\markboth{\textcolor{darkblue}{\textbf{\ipa{tsɯ˩\textsubscript{a}}}}}{}
\textcolor{teal}{\mytextsc{verb}} \hspace{4pt} Tone: L\textsubscript{a}.
\textcolor{Sepia}{\selectlanguage{english}To block up.} \zh{堵塞、塞住洞口。} 
\lhead{\firstmark}
\rhead{\botmark}

\subsection{\hspace{-0.5cm} {\Large \textcolor{darkblue}{\textbf{\ipa{tsɯ˩pʰɤ˩}}}}\hspace{0.5cm}[\kern2pt{\textcolor{darkblue}{\textbf{\ipa{tsɯ˧pʰɤ˧˥}}}}\kern2pt]} \hypertarget{tsM\string_Bp\string_h7\string_B1}{}
\markboth{\textcolor{darkblue}{\textbf{\ipa{tsɯ˩pʰɤ˩}}}}{}
\textcolor{teal}{\mytextsc{verb}} \hspace{4pt} Tone: L.
\ding{202} \textcolor{Sepia}{\selectlanguage{english}To blink.} \zh{眨眼。}  ¶ \textcolor{darkblue}{\textbf{\ipa{mɤ˧-tsɯ˩pʰɤ˩}}} \textcolor{Sepia}{\selectlanguage{english}\mytextsc{neg}} \zh{不眨眼}  
 ¶ \textcolor{darkblue}{\textbf{\ipa{ɖɯ˧-tsɯ˧\textasciitilde{}tsɯ˥-ɻ̍˩}}} \textcolor{Sepia}{\selectlanguage{english}\mytextsc{delimitative} \mytextsc{red} \mytextsc{inceptive}} \zh{\mytextsc{delimitative} \mytextsc{red} \mytextsc{inceptive}}  
 ¶ \textcolor{darkblue}{\textbf{\ipa{njɤ˩ɭɯ˧ tsɯ˩pʰɤ˩}}} \textcolor{Sepia}{\selectlanguage{english}to blink} \zh{眨眼}  
\ding{203} \textcolor{Sepia}{\selectlanguage{english}To wink (as a discreet sign of mutual understanding).} \zh{眨眼。}  ¶ \textcolor{darkblue}{\textbf{\ipa{ʈʂʰɯ˧ | njɤ˩ɭɯ˧ tsɯ˩pʰɤ˩-dʑo˩!}}} \textcolor{Sepia}{\selectlanguage{english}(S)he is winking!} \zh{他在眨眼!}  

\lhead{\firstmark}
\rhead{\botmark}

\subsection{\hspace{-0.5cm} {\Large \textcolor{darkblue}{\textbf{\ipa{tsɯ˧˥}}}}\hspace{0.5cm}[\kern2pt{\textcolor{darkblue}{\textbf{\ipa{tsɯ˧˥}}}}\kern2pt]} \hypertarget{tsM\string_M\string_T1}{}
\markboth{\textcolor{darkblue}{\textbf{\ipa{tsɯ˧˥}}}}{}
\textcolor{teal}{\mytextsc{suffix}} \hspace{4pt} Tone: MH.
\textcolor{Sepia}{\selectlanguage{english}Reported/hearsay evidential: the speaker indicates that they are not in a position to vouch personally for what they are saying, that it is based on indirect knowledge, from hearsay.} \zh{据说\mytextsc{传闻据素。}} 
\lhead{\firstmark}
\rhead{\botmark}

\subsection{\hspace{-0.5cm} {\Large \textcolor{darkblue}{\textbf{\ipa{tsɯ˧˥}}}}\hspace{0.5cm}[\kern2pt{\textcolor{darkblue}{\textbf{\ipa{tsɯ˧˥}}}}\kern2pt]} \hypertarget{tsM\string_M\string_T1}{}
\markboth{\textcolor{darkblue}{\textbf{\ipa{tsɯ˧˥}}}}{}
\textcolor{teal}{\mytextsc{verb}} \hspace{4pt} Tone: MH.
\textcolor{Sepia}{\selectlanguage{english}To call, to give the name of….} \zh{叫、叫做。}  ¶ \textcolor{darkblue}{\textbf{\ipa{ʈæ˧ʂɯ˧-ɳɯ˧ | no˧-ki˥ | jɤ˩-ʐe˧ ɲi˩-tsɯ˩-mɤ˩-tsɯ˩!}}} \textcolor{Sepia}{\selectlanguage{english}\textcolor{darkblue}{\textbf{\ipa{/ʈæ˧ʂɯ˧/}}} calls you “foreigner”!} \zh{达石把你叫作“老外”!}  

\lhead{\firstmark}
\rhead{\botmark}

\subsection{\hspace{-0.5cm} {\Large \textcolor{darkblue}{\textbf{\ipa{tsʰɑ˧bo\#˥}}}}\hspace{0.5cm}[\kern2pt{\textcolor{darkblue}{\textbf{\ipa{tsʰɑ˧bo˧}}}}\kern2pt]} \hypertarget{ts\string_hA\string_Mbo\#\string_T1}{}
\markboth{\textcolor{darkblue}{\textbf{\ipa{tsʰɑ˧bo\#˥}}}}{}
\textcolor{teal}{\mytextsc{noun}} \hspace{4pt} Tone: \#H.
\textcolor{Sepia}{\selectlanguage{english}Cook.} \zh{厨师。}  ¶ \textcolor{darkblue}{\textbf{\ipa{tsʰɑ˧bo˧ lɑ˩}}} \textcolor{Sepia}{\selectlanguage{english}to be a cook, to work as a cook, to get employed as a cook} \zh{当厨师}  
 ¶ \textcolor{darkblue}{\textbf{\ipa{tsʰɑ˧bo˧ ʝi˧}}} \textcolor{Sepia}{\selectlanguage{english}to be a cook, to work as a cook, to get employed as a cook} \zh{当厨师}  

\lhead{\firstmark}
\rhead{\botmark}

\subsection{\hspace{-0.5cm} {\Large \textcolor{darkblue}{\textbf{\ipa{tsʰɑ˧kv̩˩}}}}\hspace{0.5cm}[\kern2pt{\textcolor{darkblue}{\textbf{\ipa{tsʰɑ˧kv̩˩}}}}\kern2pt]} \hypertarget{ts\string_hA\string_Mkv\string_=\string_B1}{}
\markboth{\textcolor{darkblue}{\textbf{\ipa{tsʰɑ˧kv̩˩}}}}{}
\textcolor{teal}{\mytextsc{noun}} \hspace{4pt} Tone: L\#.
\textcolor{Sepia}{\selectlanguage{english}Warehouse, storehouse.} \zh{仓库(汉语借词)。}  Borrowing: Chinese  \zh{仓库}

\lhead{\firstmark}
\rhead{\botmark}

\subsection{\hspace{-0.5cm} {\Large \textcolor{darkblue}{\textbf{\ipa{tsʰɑ˧tɕɤ˧˥}}}}\hspace{0.5cm}[\kern2pt{\textcolor{darkblue}{\textbf{\ipa{tsʰɑ˧tɕɤ˧˥}}}}\kern2pt]} \hypertarget{ts\string_hA\string_Mts£7\string_M\string_T1}{}
\markboth{\textcolor{darkblue}{\textbf{\ipa{tsʰɑ˧tɕɤ˧˥}}}}{}
\textcolor{teal}{\mytextsc{noun}} \hspace{4pt} Tone: MH.
\textcolor{Sepia}{\selectlanguage{english}Seedlings.} \zh{青菜幼苗。} 
\lhead{\firstmark}
\rhead{\botmark}

\subsection{\hspace{-0.5cm} {\Large \textcolor{darkblue}{\textbf{\ipa{tsʰɑ˩pʰɑ˩lɑ˥}}}}\hspace{0.5cm}[\kern2pt{\textcolor{darkblue}{\textbf{\ipa{tsʰɑ˩pʰɑ˩lɑ˥}}}}\kern2pt]} \hypertarget{ts\string_hA\string_Bp\string_hA\string_BlA\string_T1}{}
\markboth{\textcolor{darkblue}{\textbf{\ipa{tsʰɑ˩pʰɑ˩lɑ˥}}}}{}
\textcolor{teal}{\mytextsc{noun}} \hspace{4pt} Tone: L+H\#.
\textcolor{Sepia}{\selectlanguage{english}Husk of sweet corn (maize) cobs.} \zh{苞谷叶(玉米穰子的叶子)。}  ¶ \textcolor{darkblue}{\textbf{\ipa{qʰɑ˧dze˧-tsʰɑ˩pʰɑ˩lɑ˩}}} \textcolor{Sepia}{\selectlanguage{english}same meaning} \zh{同上}  

\lhead{\firstmark}
\rhead{\botmark}

\subsection{\hspace{-0.5cm} {\Large \textcolor{darkblue}{\textbf{\ipa{tsʰæ˧pʰv˧˥}}}}\hspace{0.5cm}[\kern2pt{\textcolor{darkblue}{\textbf{\ipa{tsʰæ˧pʰv˧˥}}}}\kern2pt]} \hypertarget{ts\string_h\{\string_Mp\string_hv\string_M\string_T1}{}
\markboth{\textcolor{darkblue}{\textbf{\ipa{tsʰæ˧pʰv˧˥}}}}{}
\textcolor{teal}{\mytextsc{noun}} \hspace{4pt} Tone: MH\#.
\textcolor{Sepia}{\selectlanguage{english}Chinese cabbage. This is a calque from the Chinese 'white vegetable', using the Chinese word for 'vegetable' in association with the Na word for 'white'.} \zh{白菜(借汉语‘白菜’的第二个音节来充当这个名词的第一个音节:按摩梭话句法,形容词在名词后面,跟汉语相反)。}  Borrowing: Chinese  \zh{菜}
 \zh{量词}: \textcolor{darkblue}{\textbf{\ipa{po˧}}}  \mytextsc{clf}: \textcolor{darkblue}{\textbf{\ipa{po˧}}} \textit{See:} \hyperlink{}{\textcolor{darkblue}{\textbf{\ipa{v̩˩tsʰɤ˧-pʰv̩˥}}}} 
\lhead{\firstmark}
\rhead{\botmark}

\subsection{\hspace{-0.5cm} {\Large \textcolor{darkblue}{\textbf{\ipa{tsʰe\#˥}}}}\hspace{0.5cm}[\kern2pt{\textcolor{darkblue}{\textbf{\ipa{tsʰe˥}}}}\kern2pt]} \hypertarget{ts\string_he\#\string_T1}{}
\markboth{\textcolor{darkblue}{\textbf{\ipa{tsʰe\#˥}}}}{}
\textcolor{teal}{\mytextsc{noun}} \hspace{4pt} Tone: \#H.
\textcolor{Sepia}{\selectlanguage{english}Salt.} \zh{盐。} 
\lhead{\firstmark}
\rhead{\botmark}

\subsection{\hspace{-0.5cm} {\Large \textcolor{darkblue}{\textbf{\ipa{tsʰe˧}}}}\hspace{0.5cm}[\kern2pt{\textcolor{darkblue}{\textbf{\ipa{tsʰe˩˥}}}}\kern2pt]} \hypertarget{ts\string_he\string_M1}{}
\markboth{\textcolor{darkblue}{\textbf{\ipa{tsʰe˧}}}}{}
\textcolor{teal}{\mytextsc{number}} \hspace{4pt} Tone: L.
\textcolor{Sepia}{\selectlanguage{english}10.} \zh{10。} 
\lhead{\firstmark}
\rhead{\botmark}

\subsection{\hspace{-0.5cm} {\Large \textcolor{darkblue}{\textbf{\ipa{tsʰe˧do˧˥}}}}\hspace{0.5cm}[\kern2pt{\textcolor{darkblue}{\textbf{\ipa{tsʰe˧do˧˥}}}}\kern2pt]} \hypertarget{ts\string_he\string_Mdo\string_M\string_T1}{}
\markboth{\textcolor{darkblue}{\textbf{\ipa{tsʰe˧do˧˥}}}}{}
\textcolor{teal}{\mytextsc{adverb(ial)}} \hspace{4pt} Tone: MH\# | L.
\textcolor{Sepia}{\selectlanguage{english}The beginning of the month.} \zh{月初。}  ¶ \textcolor{darkblue}{\textbf{\ipa{tsʰe˧do˧-ɖɯ˧ɲi\#˥ / tsʰe˧do˧-ɖɯ˧ɲi˥}}} \textcolor{Sepia}{\selectlanguage{english}the 1st day of the month} \zh{初一}  
 ¶ \textcolor{darkblue}{\textbf{\ipa{tsʰe˧do˧-ɲi˧ɲi\#˥ / tsʰe˧do˧-ɲi˧ɲi˥}}} \textcolor{Sepia}{\selectlanguage{english}the second day of the month} \zh{初二}  
 ¶ \textcolor{darkblue}{\textbf{\ipa{tsʰe˧do˧˥ | -so˩ɲi˩˥}}} \textcolor{Sepia}{\selectlanguage{english}the third day of the month} \zh{初三}  
 ¶ \textcolor{darkblue}{\textbf{\ipa{tsʰe˧do˧-ŋwɤ˥ɲi˩}}} \textcolor{Sepia}{\selectlanguage{english}the 5th day of the month} \zh{初五}  
 ¶ \textcolor{darkblue}{\textbf{\ipa{tsʰe˧do˧-hõ˥ɲi˩}}} \textcolor{Sepia}{\selectlanguage{english}the 8th day of the month} \zh{初八}  
 ¶ \textcolor{darkblue}{\textbf{\ipa{tsʰe˧do˧˥ | -tsʰe˩ɲi˩˥}}} \textcolor{Sepia}{\selectlanguage{english}the 10th day of the month} \zh{初十}  
 ¶ \textcolor{darkblue}{\textbf{\ipa{tsʰe˧do˧˥ | -tsʰe˩ɖɯ˩ɲi˩˥}}} \textcolor{Sepia}{\selectlanguage{english}the 11th day of the month} \zh{十一日}  

\lhead{\firstmark}
\rhead{\botmark}

\subsection{\hspace{-0.5cm} {\Large \textcolor{darkblue}{\textbf{\ipa{tsʰe˧hṽ˧˥}}}}\hspace{0.5cm}[\kern2pt{\textcolor{darkblue}{\textbf{\ipa{tsʰe˧hṽ˧˥}}}}\kern2pt]} \hypertarget{ts\string_he\string_Mhv\string_~\string_M\string_T1}{}
\markboth{\textcolor{darkblue}{\textbf{\ipa{tsʰe˧hṽ˧˥}}}}{}
\textcolor{teal}{\mytextsc{noun}} \hspace{4pt} Tone: MH\#.
\textcolor{Sepia}{\selectlanguage{english}Chinese evergreen.} \zh{万年青。}  ¶ \textcolor{darkblue}{\textbf{\ipa{tsʰe˧hṽ˧-dzi˧˥}}} \textcolor{Sepia}{\selectlanguage{english}Chinese evergreen tree} \zh{万年青树}  
 ¶ \textcolor{darkblue}{\textbf{\ipa{tsʰe˧hṽ˧-bæ˥bæ˩}}} \textcolor{Sepia}{\selectlanguage{english}flowers of Chinese evergreen} \zh{万年青花}  
 \zh{量词}: \textcolor{darkblue}{\textbf{\ipa{dzi˩}}}  \mytextsc{clf}: \textcolor{darkblue}{\textbf{\ipa{dzi˩}}} 
\lhead{\firstmark}
\rhead{\botmark}

\subsection{\hspace{-0.5cm} {\Large \textcolor{darkblue}{\textbf{\ipa{tsʰe˧jɤ˧mi˥}}}}\hspace{0.5cm}[\kern2pt{\textcolor{darkblue}{\textbf{\ipa{tsʰe˧jɤ˧mi˥}}}}\kern2pt]} \hypertarget{ts\string_he\string_Mj7\string_Mmi\string_T1}{}
\markboth{\textcolor{darkblue}{\textbf{\ipa{tsʰe˧jɤ˧mi˥}}}}{}
\textcolor{teal}{\mytextsc{noun}} \hspace{4pt} Tone: H\#.
\textcolor{Sepia}{\selectlanguage{english}Marsh, bog, swamp.} \zh{沼泽。}  ¶ \textcolor{darkblue}{\textbf{\ipa{tsʰe˧jɤ˧mi˥-qo˩, | ʈʰæ˧tɕi˧ɭɯ˥ | tʰi˧-di˩!}}} \textcolor{Sepia}{\selectlanguage{english}In marshes, there are clumps of wild herbs!} \zh{沼泽里,只长野草!}  
 \zh{量词}: \textcolor{darkblue}{\textbf{\ipa{pʰæ˧˥}}}  \mytextsc{clf}: \textcolor{darkblue}{\textbf{\ipa{pʰæ˧˥}}} 
\lhead{\firstmark}
\rhead{\botmark}

\subsection{\hspace{-0.5cm} {\Large \textcolor{darkblue}{\textbf{\ipa{tsʰe˧ɬi˧mi˧}}}}\hspace{0.5cm}[\kern2pt{\textcolor{darkblue}{\textbf{\ipa{tsʰe˧ɬi˧mi˧}}}}\kern2pt]} \hypertarget{ts\string_he\string_MKi\string_Mmi\string_M1}{}
\markboth{\textcolor{darkblue}{\textbf{\ipa{tsʰe˧ɬi˧mi˧}}}}{}
\textcolor{teal}{\mytextsc{noun}} \hspace{4pt} Tone: M.
\textcolor{Sepia}{\selectlanguage{english}10th month.} \zh{十月。} 
\lhead{\firstmark}
\rhead{\botmark}

\subsection{\hspace{-0.5cm} {\Large \textcolor{darkblue}{\textbf{\ipa{tsʰe˧qʰɑ˩}}}}\hspace{0.5cm}[\kern2pt{\textcolor{darkblue}{\textbf{\ipa{tsʰe˧qʰɑ˩}}}}\kern2pt]} \hypertarget{ts\string_he\string_Mq\string_hA\string_B1}{}
\markboth{\textcolor{darkblue}{\textbf{\ipa{tsʰe˧qʰɑ˩}}}}{}
\textcolor{teal}{\mytextsc{adjective}} \hspace{4pt} Tone: L\#.
\textcolor{Sepia}{\selectlanguage{english}Too salty.} \zh{太咸。} \textit{See:} \hyperlink{}{\textcolor{darkblue}{\textbf{\ipa{tsʰe˧so˧˥}}}} 
\lhead{\firstmark}
\rhead{\botmark}

\subsection{\hspace{-0.5cm} {\Large \textcolor{darkblue}{\textbf{\ipa{tsʰe˧so˧˥}}}}\hspace{0.5cm}[\kern2pt{\textcolor{darkblue}{\textbf{\ipa{tsʰe˧so˧˥}}}}\kern2pt]} \hypertarget{ts\string_he\string_Mso\string_M\string_T1}{}
\markboth{\textcolor{darkblue}{\textbf{\ipa{tsʰe˧so˧˥}}}}{}
\textcolor{teal}{\mytextsc{adjective}} \hspace{4pt} Tone: MH\#.
\textcolor{Sepia}{\selectlanguage{english}Salty (pleasantly salty).} \zh{咸。} \textit{See:} \hyperlink{}{\textcolor{darkblue}{\textbf{\ipa{tsʰe˧qʰɑ˩}}}} 
\lhead{\firstmark}
\rhead{\botmark}

\subsection{\hspace{-0.5cm} {\Large \textcolor{darkblue}{\textbf{\ipa{tsʰe˧tʰv̩\#˥}}}}\hspace{0.5cm}[\kern2pt{\textcolor{darkblue}{\textbf{\ipa{tsʰe˧tʰv̩˧}}}}\kern2pt]} \hypertarget{ts\string_he\string_Mt\string_hv\string_=\#\string_T1}{}
\markboth{\textcolor{darkblue}{\textbf{\ipa{tsʰe˧tʰv̩\#˥}}}}{}
\textcolor{teal}{\mytextsc{noun}} \hspace{4pt} Tone: \#H.
\ding{202} \textcolor{Sepia}{\selectlanguage{english}Smallpox.} \zh{天花。}  ¶ \textcolor{darkblue}{\textbf{\ipa{tsʰe˧tʰv̩˧ | bæ˩bæ˩ bæ˥-ze˩}}} \textcolor{Sepia}{\selectlanguage{english}Smallpox has broken out.} \zh{天花/麻疹犯了。}  
 \zh{量词}: \textcolor{darkblue}{\textbf{\ipa{ʂɯ˩}}} \ding{203} \textcolor{Sepia}{\selectlanguage{english}Measles.} \zh{麻疹,疹子。}  ¶ \textcolor{darkblue}{\textbf{\ipa{gv̩˧-kʰv̩˩ mɤ˩-gv̩˩, | tsʰe˧ mɤ˧-tʰv̩˧, | hĩ˧ ʈʂɤ˧-mɤ˧-kv̩˩!}}} \textcolor{Sepia}{\selectlanguage{english}If one does not catch the measles before age nine, one cannot become an adult (literally 'a person')!} \zh{九岁前不得麻疹,不能成人! / 得麻疹,就是小孩生长过程中必须要的一件事情!}  
 \mytextsc{clf}: \textcolor{darkblue}{\textbf{\ipa{ʂɯ˩}}} 
\lhead{\firstmark}
\rhead{\botmark}

\subsection{\hspace{-0.5cm} {\Large \textcolor{darkblue}{\textbf{\ipa{tsʰe˧ʈʂæ˧}}}}\hspace{0.5cm}[\kern2pt{\textcolor{darkblue}{\textbf{\ipa{tsʰe˧ʈʂæ˧}}}}\kern2pt]} \hypertarget{ts\string_he\string_Mt`s`\{\string_M1}{}
\markboth{\textcolor{darkblue}{\textbf{\ipa{tsʰe˧ʈʂæ˧}}}}{}
\textcolor{teal}{\mytextsc{noun}} \hspace{4pt} Tone: M.
\textcolor{Sepia}{\selectlanguage{english}Village head, small-ranking official.} \zh{村长。}  Borrowing: Chinese  \zh{村长}

\lhead{\firstmark}
\rhead{\botmark}

\subsection{\hspace{-0.5cm} {\Large \textcolor{darkblue}{\textbf{\ipa{tsʰe˩\textsubscript{b}}}}}\hspace{0.5cm}[\kern2pt{\textcolor{darkblue}{\textbf{\ipa{tsʰe˩˥}}}}\kern2pt]} \hypertarget{ts\string_he\string_Bb1}{}
\markboth{\textcolor{darkblue}{\textbf{\ipa{tsʰe˩\textsubscript{b}}}}}{}
\textcolor{teal}{\mytextsc{classifier}} \hspace{4pt} Tone: L\textsubscript{b}.
\textcolor{Sepia}{\selectlanguage{english}Classifier: an inch (1/3 decimeter).} \zh{量词:寸(汉语借词)。}  Borrowing: Chinese  \zh{寸}

\lhead{\firstmark}
\rhead{\botmark}

\subsection{\hspace{-0.5cm} {\Large \textcolor{darkblue}{\textbf{\ipa{tsʰe˩\textsubscript{b}}}}}\hspace{0.5cm}[\kern2pt{\textcolor{darkblue}{\textbf{\ipa{tsʰe˩˥}}}}\kern2pt]} \hypertarget{ts\string_he\string_Bb1}{}
\markboth{\textcolor{darkblue}{\textbf{\ipa{tsʰe˩\textsubscript{b}}}}}{}
\textcolor{teal}{\mytextsc{classifier}} \hspace{4pt} Tone: L\textsubscript{b}.
\textcolor{Sepia}{\selectlanguage{english}Classifier for knots, e.g. in a braid.} \zh{量词:数辫子的节(一节)。} 
\lhead{\firstmark}
\rhead{\botmark}

\subsection{\hspace{-0.5cm} {\Large \textcolor{darkblue}{\textbf{\ipa{tsʰe˩gv̩˩}}}}\hspace{0.5cm}[\kern2pt{\textcolor{darkblue}{\textbf{\ipa{tsʰe˩gv̩˩˥}}}}\kern2pt]} \hypertarget{ts\string_he\string_Bgv\string_=\string_B1}{}
\markboth{\textcolor{darkblue}{\textbf{\ipa{tsʰe˩gv̩˩}}}}{}
\textcolor{teal}{\mytextsc{number}} \hspace{4pt} Tone: L.
\textcolor{Sepia}{\selectlanguage{english}19.} \zh{19。} 
\lhead{\firstmark}
\rhead{\botmark}

\subsection{\hspace{-0.5cm} {\Large \textcolor{darkblue}{\textbf{\ipa{tsʰe˩hõ˩}}}}\hspace{0.5cm}[\kern2pt{\textcolor{darkblue}{\textbf{\ipa{tsʰe˩hõ˩˥}}}}\kern2pt]} \hypertarget{ts\string_he\string_Bho\string_~\string_B1}{}
\markboth{\textcolor{darkblue}{\textbf{\ipa{tsʰe˩hõ˩}}}}{}
\textcolor{teal}{\mytextsc{number}} \hspace{4pt} Tone: L.
\textcolor{Sepia}{\selectlanguage{english}18.} \zh{18。} 
\lhead{\firstmark}
\rhead{\botmark}

\subsection{\hspace{-0.5cm} {\Large \textcolor{darkblue}{\textbf{\ipa{tsʰe˩ŋwɤ˩}}}}\hspace{0.5cm}[\kern2pt{\textcolor{darkblue}{\textbf{\ipa{tsʰe˩ŋwɤ˩˥}}}}\kern2pt]} \hypertarget{ts\string_he\string_BNw7\string_B1}{}
\markboth{\textcolor{darkblue}{\textbf{\ipa{tsʰe˩ŋwɤ˩}}}}{}
\textcolor{teal}{\mytextsc{number}} \hspace{4pt} Tone: L.
\textcolor{Sepia}{\selectlanguage{english}15.} \zh{15。} 
\lhead{\firstmark}
\rhead{\botmark}

\subsection{\hspace{-0.5cm} {\Large \textcolor{darkblue}{\textbf{\ipa{tsʰe˩qʰv̩˩}}}}\hspace{0.5cm}[\kern2pt{\textcolor{darkblue}{\textbf{\ipa{tsʰe˩qʰv̩˩˥}}}}\kern2pt]} \hypertarget{ts\string_he\string_Bq\string_hv\string_=\string_B1}{}
\markboth{\textcolor{darkblue}{\textbf{\ipa{tsʰe˩qʰv̩˩}}}}{}
\textcolor{teal}{\mytextsc{number}} \hspace{4pt} Tone: L.
\textcolor{Sepia}{\selectlanguage{english}16.} \zh{16。} 
\lhead{\firstmark}
\rhead{\botmark}

\subsection{\hspace{-0.5cm} {\Large \textcolor{darkblue}{\textbf{\ipa{tsʰe˩ʐv̩˩}}}}\hspace{0.5cm}[\kern2pt{\textcolor{darkblue}{\textbf{\ipa{tsʰe˩ʐv̩˩˥}}}}\kern2pt]} \hypertarget{ts\string_he\string_Bz`v\string_=\string_B1}{}
\markboth{\textcolor{darkblue}{\textbf{\ipa{tsʰe˩ʐv̩˩}}}}{}
\textcolor{teal}{\mytextsc{number}} \hspace{4pt} Tone: L.
\textcolor{Sepia}{\selectlanguage{english}14.} \zh{14。} 
\lhead{\firstmark}
\rhead{\botmark}

\subsection{\hspace{-0.5cm} {\Large \textcolor{darkblue}{\textbf{\ipa{tsʰɤ˩\textsubscript{a}}}}}\hspace{0.5cm}[\kern2pt{\textcolor{darkblue}{\textbf{\ipa{tsʰɤ˩˥}}}}\kern2pt]} \hypertarget{ts\string_h7\string_Ba1}{}
\markboth{\textcolor{darkblue}{\textbf{\ipa{tsʰɤ˩\textsubscript{a}}}}}{}
\textcolor{teal}{\mytextsc{verb}} \hspace{4pt} Tone: L\textsubscript{a}.
\textcolor{Sepia}{\selectlanguage{english}To plait, to weave (hair, thread).} \zh{编(头发,线)。}  ¶ \textcolor{darkblue}{\textbf{\ipa{ʁo˧qʰwɤ˩ tsʰɤ˩}}} \textcolor{Sepia}{\selectlanguage{english}to weave the hair} \zh{编辫子}  
 ¶ \textcolor{darkblue}{\textbf{\ipa{hæ̃˧pɤ˧ le˧-tsʰɤ˩}}} \textcolor{Sepia}{\selectlanguage{english}to plait hair} \zh{梳一条辫子}  
 ¶ \textcolor{darkblue}{\textbf{\ipa{ɖɯ˧-tsʰɤ˧\textasciitilde{}tsʰɤ˥-ɻ̍˩}}} \textcolor{Sepia}{\selectlanguage{english}\mytextsc{delimitative} \string_ \mytextsc{red} \mytextsc{inceptive}} \zh{\mytextsc{delimitative} \string_ \mytextsc{red} \mytextsc{inceptive}}  

\lhead{\firstmark}
\rhead{\botmark}

\subsection{\hspace{-0.5cm} {\Large \textcolor{darkblue}{\textbf{\ipa{tsʰɤ˧˥}}} \textsubscript{1}}\hspace{0.5cm}[\kern2pt{\textcolor{darkblue}{\textbf{\ipa{tsʰɤ˧˥}}}}\kern2pt]} \hypertarget{ts\string_h7\string_M\string_T1}{}
\markboth{\textcolor{darkblue}{\textbf{\ipa{tsʰɤ˧˥}}} \textsubscript{1}}{}
\textcolor{teal}{\mytextsc{verb}} \hspace{4pt} Tone: MH.
\textcolor{Sepia}{\selectlanguage{english}To milk.} \zh{挤奶。}  ¶ \textcolor{darkblue}{\textbf{\ipa{tso˧\textasciitilde{}tso˧ tsʰɤ˩ + ze˩}}} \textcolor{Sepia}{\selectlanguage{english}to milk things} \zh{挤出东西}  
 ¶ \textcolor{darkblue}{\textbf{\ipa{ʝi˧-bv̩˧ | ɳæ˧ tsʰɤ˩}}} \textcolor{Sepia}{\selectlanguage{english}to milk a cow} \zh{挤牛奶}  

\lhead{\firstmark}
\rhead{\botmark}

\subsection{\hspace{-0.5cm} {\Large \textcolor{darkblue}{\textbf{\ipa{tsʰɤ˧˥}}} \textsubscript{2}}\hspace{0.5cm}[\kern2pt{\textcolor{darkblue}{\textbf{\ipa{tsʰɤ˧˥}}}}\kern2pt]} \hypertarget{ts\string_h7\string_M\string_T2}{}
\markboth{\textcolor{darkblue}{\textbf{\ipa{tsʰɤ˧˥}}} \textsubscript{2}}{}
\textcolor{teal}{\mytextsc{verb}} \hspace{4pt} Tone: MH.
\textcolor{Sepia}{\selectlanguage{english}To rub (e.g. rough fabric rubs against the skin).} \zh{摩擦。} 
\lhead{\firstmark}
\rhead{\botmark}

\subsection{\hspace{-0.5cm} {\Large \textcolor{darkblue}{\textbf{\ipa{tsʰɤ˧˥}}} \textsubscript{3}}\hspace{0.5cm}[\kern2pt{\textcolor{darkblue}{\textbf{\ipa{tsʰɤ˧˥}}}}\kern2pt]} \hypertarget{ts\string_h7\string_M\string_T3}{}
\markboth{\textcolor{darkblue}{\textbf{\ipa{tsʰɤ˧˥}}} \textsubscript{3}}{}
\textcolor{teal}{\mytextsc{verb}} \hspace{4pt} Tone: MH.
\textcolor{Sepia}{\selectlanguage{english}To attack, to pillage (e.g. bandits attack a caravan).} \zh{抢。} 
\lhead{\firstmark}
\rhead{\botmark}

\subsection{\hspace{-0.5cm} {\Large \textcolor{darkblue}{\textbf{\ipa{tsʰɤ˧˥}}} \textsubscript{4}}\hspace{0.5cm}[\kern2pt{\textcolor{darkblue}{\textbf{\ipa{tsʰɤ˧˥}}}}\kern2pt]} \hypertarget{ts\string_h7\string_M\string_T4}{}
\markboth{\textcolor{darkblue}{\textbf{\ipa{tsʰɤ˧˥}}} \textsubscript{4}}{}
\textcolor{teal}{\mytextsc{verb}} \hspace{4pt} Tone: MH.
\textcolor{Sepia}{\selectlanguage{english}To give back, to return.} \zh{还(东西)。} 
\lhead{\firstmark}
\rhead{\botmark}

\subsection{\hspace{-0.5cm} {\Large \textcolor{darkblue}{\textbf{\ipa{tsʰɤ˧˥\textsubscript{a}}}}}\hspace{0.5cm}[\kern2pt{\textcolor{darkblue}{\textbf{\ipa{tsʰɤ˧˥}}}}\kern2pt]} \hypertarget{ts\string_h7\string_M\string_Ta1}{}
\markboth{\textcolor{darkblue}{\textbf{\ipa{tsʰɤ˧˥\textsubscript{a}}}}}{}
\textcolor{teal}{\mytextsc{classifier}} \hspace{4pt} Tone: MH\textsubscript{a}.
\textcolor{Sepia}{\selectlanguage{english}Classifier for dented or bumpy objects: cockscombs, leaves, and bulbs of garlic.} \zh{量词:凸凹的物品,如:鸡冠(一顶)、叶子(一片)、蒜(一头)。}  ¶ \textcolor{darkblue}{\textbf{\ipa{bæ˩bæ˩˥ | ɖɯ˧-tsʰɤ˧˥}}} \textcolor{Sepia}{\selectlanguage{english}a flower} \zh{一朵花}  

\lhead{\firstmark}
\rhead{\botmark}

\subsection{\hspace{-0.5cm} {\Large \textcolor{darkblue}{\textbf{\ipa{tsʰi\#˥}}}}\hspace{0.5cm}[\kern2pt{\textcolor{darkblue}{\textbf{\ipa{tsʰi˥}}}}\kern2pt]} \hypertarget{ts\string_hi\#\string_T1}{}
\markboth{\textcolor{darkblue}{\textbf{\ipa{tsʰi\#˥}}}}{}
\textcolor{teal}{\mytextsc{noun}} \hspace{4pt} Tone: \#H.
\textcolor{Sepia}{\selectlanguage{english}Dry season (winter and spring: from the 9th lunar month to the 2nd lunar month).} \zh{旱季(冬天至春天:农历九月到来年二月)。} 
\lhead{\firstmark}
\rhead{\botmark}

\subsection{\hspace{-0.5cm} {\Large \textcolor{darkblue}{\textbf{\ipa{tsʰi˥\textsubscript{a}}}}}\hspace{0.5cm}[\kern2pt{\textcolor{darkblue}{\textbf{\ipa{tsʰi˥}}}}\kern2pt]} \hypertarget{ts\string_hi\string_Ta1}{}
\markboth{\textcolor{darkblue}{\textbf{\ipa{tsʰi˥\textsubscript{a}}}}}{}
\textcolor{teal}{\mytextsc{classifier}} \hspace{4pt} Tone: H\textsubscript{a}.
\textcolor{Sepia}{\selectlanguage{english}Classifier for pelts / hides (animal skins), and for pieces of fabric.} \zh{量词:动物皮(一张),布料(一块)。}  ¶ \textcolor{darkblue}{\textbf{\ipa{ɖɯ˧-tsʰi˥}}} \textcolor{Sepia}{\selectlanguage{english}one pelt} \zh{一张动物皮}  
 ¶ \textcolor{darkblue}{\textbf{\ipa{ɖɯ˧-tsʰi˧ ɲi˥}}} \textcolor{Sepia}{\selectlanguage{english}It's a pelt.} \zh{这是一张(动物皮)}  

\lhead{\firstmark}
\rhead{\botmark}

\subsection{\hspace{-0.5cm} {\Large \textcolor{darkblue}{\textbf{\ipa{tsʰi˧}}} \textsubscript{1}}\hspace{0.5cm}[\kern2pt{\textcolor{darkblue}{\textbf{\ipa{tsʰi˥}}}}\kern2pt]} \hypertarget{ts\string_hi\string_M1}{}
\markboth{\textcolor{darkblue}{\textbf{\ipa{tsʰi˧}}} \textsubscript{1}}{}
\textcolor{teal}{\mytextsc{adjective}} \hspace{4pt} Tone: M.
\textcolor{Sepia}{\selectlanguage{english}Hot; scalding.} \zh{热,烫。}  ¶ \textcolor{darkblue}{\textbf{\ipa{tsʰi˧-zo˧ mɤ˧-tʰɑ˧˥!}}} \textcolor{Sepia}{\selectlanguage{english}The heat is unbearable! / The weather is unbearably hot!} \zh{热得受不了!}  
\textit{See:} \hyperlink{}{\textcolor{darkblue}{\textbf{\ipa{tsʰi˧}}} \textsubscript{2}} 
\lhead{\firstmark}
\rhead{\botmark}

\subsection{\hspace{-0.5cm} {\Large \textcolor{darkblue}{\textbf{\ipa{tsʰi˧}}} \textsubscript{2}}\hspace{0.5cm}[\kern2pt{\textcolor{darkblue}{\textbf{\ipa{tsʰi˥}}}}\kern2pt]} \hypertarget{ts\string_hi\string_M2}{}
\markboth{\textcolor{darkblue}{\textbf{\ipa{tsʰi˧}}} \textsubscript{2}}{}
\textcolor{teal}{\mytextsc{adjective}} \hspace{4pt} Tone: M.
\textcolor{Sepia}{\selectlanguage{english}Bright.} \zh{明亮。}  ¶ \textcolor{darkblue}{\textbf{\ipa{ɲi˧mi˧ tsʰi˧}}} \textcolor{Sepia}{\selectlanguage{english}the sun shines, the sunlight is strong} \zh{太阳很晒}  
 ¶ \textcolor{darkblue}{\textbf{\ipa{ɬi˧mi˧ tsʰi˧}}} \textcolor{Sepia}{\selectlanguage{english}the moon shines, the moonlight is strong} \zh{月亮很亮、月光很明亮}  
\textit{See:} \hyperlink{}{\textcolor{darkblue}{\textbf{\ipa{tsʰi˧}}} \textsubscript{1}} 
\lhead{\firstmark}
\rhead{\botmark}

\subsection{\hspace{-0.5cm} {\Large \textcolor{darkblue}{\textbf{\ipa{tsʰi˧\textsubscript{b}}}}}\hspace{0.5cm}[\kern2pt{\textcolor{darkblue}{\textbf{\ipa{tsʰi˥}}}}\kern2pt]} \hypertarget{ts\string_hi\string_Mb1}{}
\markboth{\textcolor{darkblue}{\textbf{\ipa{tsʰi˧\textsubscript{b}}}}}{}
\textcolor{teal}{\mytextsc{verb}} \hspace{4pt} Tone: M\textsubscript{b}.
\textcolor{Sepia}{\selectlanguage{english}To wear (a hat).} \zh{戴帽子。}  ¶ \textcolor{darkblue}{\textbf{\ipa{tv̩˧tv̩˥ tsʰi˩}}} \textcolor{Sepia}{\selectlanguage{english}to put on a hat} \zh{戴上帽子}  

\lhead{\firstmark}
\rhead{\botmark}

\subsection{\hspace{-0.5cm} {\Large \textcolor{darkblue}{\textbf{\ipa{tsʰi˧bv̩˩}}}}\hspace{0.5cm}[\kern2pt{\textcolor{darkblue}{\textbf{\ipa{tsʰi˧bv̩˩}}}}\kern2pt]} \hypertarget{ts\string_hi\string_Mbv\string_=\string_B1}{}
\markboth{\textcolor{darkblue}{\textbf{\ipa{tsʰi˧bv̩˩}}}}{}
\textcolor{teal}{\mytextsc{adjective}} \hspace{4pt} Tone: L\#.
\textcolor{Sepia}{\selectlanguage{english}Muggy, sultry, oppressively hot.} \zh{闷热。} 
\lhead{\firstmark}
\rhead{\botmark}

\subsection{\hspace{-0.5cm} {\Large \textcolor{darkblue}{\textbf{\ipa{tsʰi˧ʝi\#˥}}}}\hspace{0.5cm}[\kern2pt{\textcolor{darkblue}{\textbf{\ipa{tsʰi˧ʝi˧}}}}\kern2pt]} \hypertarget{ts\string_hi\string_Mj££i\#\string_T1}{}
\markboth{\textcolor{darkblue}{\textbf{\ipa{tsʰi˧ʝi\#˥}}}}{}
\textcolor{teal}{\mytextsc{adverb(ial)}} \hspace{4pt} Tone: \#H.
\textcolor{Sepia}{\selectlanguage{english}This year.} \zh{今年。}  ¶ \textcolor{darkblue}{\textbf{\ipa{tsʰi˧ʝi˧-se˥, | …}}} \textcolor{Sepia}{\selectlanguage{english}Until this year, ...} \zh{到了今年,……}  

\lhead{\firstmark}
\rhead{\botmark}

\subsection{\hspace{-0.5cm} {\Large \textcolor{darkblue}{\textbf{\ipa{tsʰi˧ɲi\#˥}}}}\hspace{0.5cm}[\kern2pt{\textcolor{darkblue}{\textbf{\ipa{tsʰi˧ɲi˧}}}}\kern2pt]} \hypertarget{ts\string_hi\string_MJi\#\string_T1}{}
\markboth{\textcolor{darkblue}{\textbf{\ipa{tsʰi˧ɲi\#˥}}}}{}
\textcolor{teal}{\mytextsc{adverb(ial)}} \hspace{4pt} Tone: \#H.
\textcolor{Sepia}{\selectlanguage{english}Today.} \zh{今天。}  ¶ \textcolor{darkblue}{\textbf{\ipa{tsʰi˧ɲi˧-ʁo˧dɑ˧}}} \textcolor{Sepia}{\selectlanguage{english}before today; previously} \zh{今天之前}  

\lhead{\firstmark}
\rhead{\botmark}

\subsection{\hspace{-0.5cm} {\Large \textcolor{darkblue}{\textbf{\ipa{tsʰi˧qʰæ˧˥}}}}\hspace{0.5cm}[\kern2pt{\textcolor{darkblue}{\textbf{\ipa{tsʰi˧qʰæ˧˥}}}}\kern2pt]} \hypertarget{ts\string_hi\string_Mq\string_h\{\string_M\string_T1}{}
\markboth{\textcolor{darkblue}{\textbf{\ipa{tsʰi˧qʰæ˧˥}}}}{}
\textcolor{teal}{\mytextsc{adverb(ial)}} \hspace{4pt} Tone: MH.
\textcolor{Sepia}{\selectlanguage{english}Now, currently, these days.} \zh{现在。} 
\lhead{\firstmark}
\rhead{\botmark}

\subsection{\hspace{-0.5cm} {\Large \textcolor{darkblue}{\textbf{\ipa{tsʰi˧si˩-dʑɤ˩pv̩˩}}}}\hspace{0.5cm}[\kern2pt{\textcolor{darkblue}{\textbf{\ipa{tsʰi˧si˩dʑɤ˧pv̩˧}}}}\kern2pt]} \hypertarget{ts\string_hi\string_Msi\string_B-dz£7\string_Bpv\string_=\string_B1}{}
\markboth{\textcolor{darkblue}{\textbf{\ipa{tsʰi˧si˩-dʑɤ˩pv̩˩}}}}{}
\textcolor{teal}{\mytextsc{noun}} \hspace{4pt} Tone: L\#-.
\textcolor{Sepia}{\selectlanguage{english}The world of spirits, the world of the dead.} \zh{神灵的世界、死人的世界。} 
\lhead{\firstmark}
\rhead{\botmark}

\subsection{\hspace{-0.5cm} {\Large \textcolor{darkblue}{\textbf{\ipa{tsʰi˧ti\#˥}}}}\hspace{0.5cm}[\kern2pt{\textcolor{darkblue}{\textbf{\ipa{tsʰi˧ti˧}}}}\kern2pt]} \hypertarget{ts\string_hi\string_Mti\#\string_T1}{}
\markboth{\textcolor{darkblue}{\textbf{\ipa{tsʰi˧ti\#˥}}}}{}
\textcolor{teal}{\mytextsc{noun}} \hspace{4pt} Tone: \#H.
\textcolor{Sepia}{\selectlanguage{english}Masculine given name.} \zh{男性名字。} 
\lhead{\firstmark}
\rhead{\botmark}

\subsection{\hspace{-0.5cm} {\Large \textcolor{darkblue}{\textbf{\ipa{tsʰi˧zi\#˥}}}}\hspace{0.5cm}[\kern2pt{\textcolor{darkblue}{\textbf{\ipa{tsʰi˧zi˧}}}}\kern2pt]} \hypertarget{ts\string_hi\string_Mzi\#\string_T1}{}
\markboth{\textcolor{darkblue}{\textbf{\ipa{tsʰi˧zi\#˥}}}}{}
\textcolor{teal}{\mytextsc{noun}} \hspace{4pt} Tone: \#H.
\textcolor{Sepia}{\selectlanguage{english}Highland barley, \textit{Hordeum vulgare var. nudum Hook. f.}.} \zh{青稞。}  ¶ \textcolor{darkblue}{\textbf{\ipa{tsʰi˧zi˧ | nɑ˩-hĩ˩˥}}} \textcolor{Sepia}{\selectlanguage{english}black barley} \zh{黑青稞}  
 ¶ \textcolor{darkblue}{\textbf{\ipa{tsʰi˧zi˧ | pʰv̩˩-hĩ˩˥}}} \textcolor{Sepia}{\selectlanguage{english}white barley} \zh{白青稞}  
 \zh{量词}: \textcolor{darkblue}{\textbf{\ipa{kɤ˧˥}}}  \mytextsc{clf}: \textcolor{darkblue}{\textbf{\ipa{kɤ˧˥}}} 
\lhead{\firstmark}
\rhead{\botmark}

\subsection{\hspace{-0.5cm} {\Large \textcolor{darkblue}{\textbf{\ipa{tsʰi˧zi˧-ɻ̃\#˥}}}}\hspace{0.5cm}[\kern2pt{\textcolor{darkblue}{\textbf{\ipa{xxxx non-correspondance entre le nombre de morphèmes et le nombre de tons de morphèmes}}}}\kern2pt]} \hypertarget{ts\string_hi\string_Mzi\string_M-r£`\string_~\#\string_T1}{}
\markboth{\textcolor{darkblue}{\textbf{\ipa{tsʰi˧zi˧-ɻ̃\#˥}}}}{}
\textcolor{teal}{\mytextsc{noun}} \hspace{4pt} Tone: \#H.
\textcolor{Sepia}{\selectlanguage{english}Highland barley straw.} \zh{青稞杆。}  \zh{量词}: \textcolor{darkblue}{\textbf{\ipa{kɤ˧˥}}}  \mytextsc{clf}: \textcolor{darkblue}{\textbf{\ipa{kɤ˧˥}}} 
\lhead{\firstmark}
\rhead{\botmark}

\subsection{\hspace{-0.5cm} {\Large \textcolor{darkblue}{\textbf{\ipa{tsʰi˩\textsubscript{a}}}}}\hspace{0.5cm}[\kern2pt{\textcolor{darkblue}{\textbf{\ipa{tsʰi˩˥}}}}\kern2pt]} \hypertarget{ts\string_hi\string_Ba1}{}
\markboth{\textcolor{darkblue}{\textbf{\ipa{tsʰi˩\textsubscript{a}}}}}{}
\textcolor{teal}{\mytextsc{adjective}} \hspace{4pt} Tone: L\textsubscript{a}.
\textcolor{Sepia}{\selectlanguage{english}Fine, thin.} \zh{细(树、体型细小)。}  ¶ \textcolor{darkblue}{\textbf{\ipa{tsʰi˩-hĩ˩˥}}} \textcolor{Sepia}{\selectlanguage{english}\mytextsc{nmlz}} \zh{细的}  
 ¶ \textcolor{darkblue}{\textbf{\ipa{qʰɑ˧-tsʰi˧-gv̩˧}}} \textcolor{Sepia}{\selectlanguage{english}very thin} \zh{非常细}  
 ¶ \textcolor{darkblue}{\textbf{\ipa{dʑɤ˧˥ | tsʰi˩-njæ˩˥ | -gv̩˩!}}} \textcolor{Sepia}{\selectlanguage{english}It is really thin!} \zh{真细!}  

\lhead{\firstmark}
\rhead{\botmark}

\subsection{\hspace{-0.5cm} {\Large \textcolor{darkblue}{\textbf{\ipa{tsʰi˩mv̩˩tʰv̩˩}}}}\hspace{0.5cm}[\kern2pt{\textcolor{darkblue}{\textbf{\ipa{tsʰi˩mv̩˩tʰv̩˩˥}}}}\kern2pt]} \hypertarget{ts\string_hi\string_Bmv\string_=\string_Bt\string_hv\string_=\string_B1}{}
\markboth{\textcolor{darkblue}{\textbf{\ipa{tsʰi˩mv̩˩tʰv̩˩}}}}{}
\textcolor{teal}{\mytextsc{noun}} \hspace{4pt} Tone: L.
\textcolor{Sepia}{\selectlanguage{english}Dancing demon.} \zh{跳着的鬼。} 
\lhead{\firstmark}
\rhead{\botmark}

\subsection{\hspace{-0.5cm} {\Large \textcolor{darkblue}{\textbf{\ipa{tsʰi˩tv̩˩}}}}\hspace{0.5cm}[\kern2pt{\textcolor{darkblue}{\textbf{\ipa{tsʰi˩tv̩˩˥}}}}\kern2pt]} \hypertarget{ts\string_hi\string_Btv\string_=\string_B1}{}
\markboth{\textcolor{darkblue}{\textbf{\ipa{tsʰi˩tv̩˩}}}}{}
\textcolor{teal}{\mytextsc{noun}} \hspace{4pt} Tone: L.
\textcolor{Sepia}{\selectlanguage{english}Bone marrow.} \zh{骨髓。}  \zh{量词}: \textcolor{darkblue}{\textbf{\ipa{kʰwɤ˥}}}  \mytextsc{clf}: \textcolor{darkblue}{\textbf{\ipa{kʰwɤ˥}}} 
\lhead{\firstmark}
\rhead{\botmark}

\subsection{\hspace{-0.5cm} {\Large \textcolor{darkblue}{\textbf{\ipa{tsʰi˩tsʰi˩}}}}\hspace{0.5cm}[\kern2pt{\textcolor{darkblue}{\textbf{\ipa{tsʰi˩tsʰi˩˥}}}}\kern2pt]} \hypertarget{ts\string_hi\string_Bts\string_hi\string_B1}{}
\markboth{\textcolor{darkblue}{\textbf{\ipa{tsʰi˩tsʰi˩}}}}{}
\textcolor{teal}{\mytextsc{noun}} \hspace{4pt} Tone: L.
\textcolor{Sepia}{\selectlanguage{english}Peas, garden peas.} \zh{豌豆。}  \zh{量词}: \textcolor{darkblue}{\textbf{\ipa{kɤ˧˥}}}  \mytextsc{clf}: \textcolor{darkblue}{\textbf{\ipa{kɤ˧˥}}} 
\lhead{\firstmark}
\rhead{\botmark}

\subsection{\hspace{-0.5cm} {\Large \textcolor{darkblue}{\textbf{\ipa{tsʰi˧˥}}} \textsubscript{1}}\hspace{0.5cm}[\kern2pt{\textcolor{darkblue}{\textbf{\ipa{tsʰi˧˥}}}}\kern2pt]} \hypertarget{ts\string_hi\string_M\string_T1}{}
\markboth{\textcolor{darkblue}{\textbf{\ipa{tsʰi˧˥}}} \textsubscript{1}}{}
\textcolor{teal}{\mytextsc{verb}} \hspace{4pt} Tone: MH.
\textcolor{Sepia}{\selectlanguage{english}To construct a house, to build a house.} \zh{盖,建 (房子)。}  ¶ \textcolor{darkblue}{\textbf{\ipa{ʑi˧qʰwɤ˧ tsʰi˧˥}}} \textcolor{Sepia}{\selectlanguage{english}to build a house} \zh{建 房子}  

\lhead{\firstmark}
\rhead{\botmark}

\subsection{\hspace{-0.5cm} {\Large \textcolor{darkblue}{\textbf{\ipa{tsʰi˧˥}}} \textsubscript{2}}\hspace{0.5cm}[\kern2pt{\textcolor{darkblue}{\textbf{\ipa{tsʰi˧˥}}}}\kern2pt]} \hypertarget{ts\string_hi\string_M\string_T2}{}
\markboth{\textcolor{darkblue}{\textbf{\ipa{tsʰi˧˥}}} \textsubscript{2}}{}
\textcolor{teal}{\mytextsc{verb}} \hspace{4pt} Tone: MH.
\textcolor{Sepia}{\selectlanguage{english}To bore a hole, to punch a hole.} \zh{穿一个洞。} 
\lhead{\firstmark}
\rhead{\botmark}

\subsection{\hspace{-0.5cm} {\Large \textcolor{darkblue}{\textbf{\ipa{tsʰi˧˥}}} \textsubscript{3}}\hspace{0.5cm}[\kern2pt{\textcolor{darkblue}{\textbf{\ipa{tsʰi˧˥}}}}\kern2pt]} \hypertarget{ts\string_hi\string_M\string_T3}{}
\markboth{\textcolor{darkblue}{\textbf{\ipa{tsʰi˧˥}}} \textsubscript{3}}{}
\textcolor{teal}{\mytextsc{verb}} \hspace{4pt} Tone: MH.
\textcolor{Sepia}{\selectlanguage{english}To start (a fire).} \zh{点(火)。}  ¶ \textcolor{darkblue}{\textbf{\ipa{mv̩˧ tsʰi˧˥}}} \textcolor{Sepia}{\selectlanguage{english}to start a fire} \zh{点火}  
 ¶ \textcolor{darkblue}{\textbf{\ipa{njɤ˧-ɳɯ˧ | mv̩˧tsʰi˧-bi˥}}} \textcolor{Sepia}{\selectlanguage{english}I am going to start a fire} \zh{我要点个火}  
 ¶ \textcolor{darkblue}{\textbf{\ipa{mv̩˩tsʰo˩ tsʰi˧}}} \textcolor{Sepia}{\selectlanguage{english}to put fire to a piece of wood full of sap (to start a fire)} \zh{用含很多树脂的木头来引火}  

\lhead{\firstmark}
\rhead{\botmark}

\subsection{\hspace{-0.5cm} {\Large \textcolor{darkblue}{\textbf{\ipa{tsʰi˧˥}}} \textsubscript{4}}\hspace{0.5cm}[\kern2pt{\textcolor{darkblue}{\textbf{\ipa{tsʰi˧˥}}}}\kern2pt]} \hypertarget{ts\string_hi\string_M\string_T4}{}
\markboth{\textcolor{darkblue}{\textbf{\ipa{tsʰi˧˥}}} \textsubscript{4}}{}
\textcolor{teal}{\mytextsc{verb}} \hspace{4pt} Tone: MH.
\textcolor{Sepia}{\selectlanguage{english}To squat.} \zh{蹲。}  ¶ \textcolor{darkblue}{\textbf{\ipa{le˧-tsʰi˩\textasciitilde{}tsʰi˩ | tʰi˧-dzi˩}}} \textcolor{Sepia}{\selectlanguage{english}to sit cross-legged} \zh{盘腿坐}  
 ¶ \textcolor{darkblue}{\textbf{\ipa{gɤ˩-tsʰi˧\textasciitilde{}tsʰi˩ tʰi˧-dzi˩}}} \textcolor{Sepia}{\selectlanguage{english}as above} \zh{盘腿坐}  

\lhead{\firstmark}
\rhead{\botmark}

\subsection{\hspace{-0.5cm} {\Large \textcolor{darkblue}{\textbf{\ipa{tsʰi˧˥}}} \textsubscript{5}}\hspace{0.5cm}[\kern2pt{\textcolor{darkblue}{\textbf{\ipa{tsʰi˧˥}}}}\kern2pt]} \hypertarget{ts\string_hi\string_M\string_T5}{}
\markboth{\textcolor{darkblue}{\textbf{\ipa{tsʰi˧˥}}} \textsubscript{5}}{}
\textcolor{teal}{\mytextsc{adjective}} \hspace{4pt} Tone: MH.
\textit{\textcolor{Sepia}{\selectlanguage{english}archaic}} [\zh{古语}] \textcolor{Sepia}{\selectlanguage{english}Sick, ill.} \zh{病。}  ¶ \textcolor{darkblue}{\textbf{\ipa{mɤ˧-go˩ mɤ˩-tsʰi˩-ɻ̍˩ |}}} \textcolor{Sepia}{\selectlanguage{english}to be in good health: not sick, not suffering} \zh{健康:不病、不痛}  

\lhead{\firstmark}
\rhead{\botmark}

\subsection{\hspace{-0.5cm} {\Large \textcolor{darkblue}{\textbf{\ipa{tsʰo˥}}}}\hspace{0.5cm}[\kern2pt{\textcolor{darkblue}{\textbf{\ipa{tsʰo˥}}}}\kern2pt]} \hypertarget{ts\string_ho\string_T1}{}
\markboth{\textcolor{darkblue}{\textbf{\ipa{tsʰo˥}}}}{}
\textcolor{teal}{\mytextsc{adjective}} \hspace{4pt} Tone: H.
\textcolor{Sepia}{\selectlanguage{english}Complete, all in readiness.} \zh{齐全。}  ¶ \textcolor{darkblue}{\textbf{\ipa{ə˧tso˧-mɤ˧-ɲi˩, | tʰi˧-tsʰo˥-ze˩!}}} \textcolor{Sepia}{\selectlanguage{english}All is in readiness! Everything is now ready! (Context: preparation for a feast, a meal...)} \zh{什么都准备得很齐全!}  
 ¶ \textcolor{darkblue}{\textbf{\ipa{mɤ˧-tsʰo˧-sɯ˥! | wɤ˩˥ | ɲi˧-bæ˧ hwæ˧-zo˧-ho˩!}}} \textcolor{Sepia}{\selectlanguage{english}[Decoration] is not complete yet! [I] still need to purchase a few items! (Context: visitors admire a newly furbished apartment in town; the landlord answers their compliments by saying 'The work is not finished yet!')} \zh{还不算齐全! / 还没有装饰齐全!(情景:朋友们表扬新装修的丽江房子,主人谦虚回答:‘还不算齐全!’)}  
 ¶ \textcolor{darkblue}{\textbf{\ipa{tʰi˧-tsʰo˥-kʰɯ˩}}} \textcolor{Sepia}{\selectlanguage{english}\mytextsc{dur} \string_ \mytextsc{caus}: to complete, to bring to complete readiness} \zh{\mytextsc{dur} \string_ \mytextsc{caus:完成、弄齐全}}  

\lhead{\firstmark}
\rhead{\botmark}

\subsection{\hspace{-0.5cm} {\Large \textcolor{darkblue}{\textbf{\ipa{tsʰo˧\textsubscript{b}}}}}\hspace{0.5cm}[\kern2pt{\textcolor{darkblue}{\textbf{\ipa{tsʰo˩˥}}}}\kern2pt]} \hypertarget{ts\string_ho\string_Mb1}{}
\markboth{\textcolor{darkblue}{\textbf{\ipa{tsʰo˧\textsubscript{b}}}}}{}
\textcolor{teal}{\mytextsc{verb}} \hspace{4pt} Tone: M\textsubscript{b}.
\textcolor{Sepia}{\selectlanguage{english}To jump.} \zh{跳。}  ¶ \textcolor{darkblue}{\textbf{\ipa{bæ˧ tsʰo˧}}} \textcolor{Sepia}{\selectlanguage{english}to skip rope} \zh{跳绳}  
 ¶ \textcolor{darkblue}{\textbf{\ipa{tsʰo˧\textasciitilde{}tsʰo˧}}} \textcolor{Sepia}{\selectlanguage{english}\mytextsc{red}} \zh{\mytextsc{重叠}}  

\lhead{\firstmark}
\rhead{\botmark}

\subsection{\hspace{-0.5cm} {\Large \textcolor{darkblue}{\textbf{\ipa{tsʰo˧ɖɯ˩}}}}\hspace{0.5cm}[\kern2pt{\textcolor{darkblue}{\textbf{\ipa{tsʰo˧ɖɯ˩}}}}\kern2pt]} \hypertarget{ts\string_ho\string_Md`M\string_B1}{}
\markboth{\textcolor{darkblue}{\textbf{\ipa{tsʰo˧ɖɯ˩}}}}{}
\textcolor{teal}{\mytextsc{noun}} \hspace{4pt} Tone: L\#.
\textcolor{Sepia}{\selectlanguage{english}Group dance.} \zh{集体舞。}  ¶ \textcolor{darkblue}{\textbf{\ipa{tsʰo˧ɖɯ˩ tsʰo˩}}} \textcolor{Sepia}{\selectlanguage{english}to perform a group dance} \zh{跳一个集体舞}  

\lhead{\firstmark}
\rhead{\botmark}

\subsection{\hspace{-0.5cm} {\Large \textcolor{darkblue}{\textbf{\ipa{tsʰo˧ɖwæ\#˥}}}}\hspace{0.5cm}[\kern2pt{\textcolor{darkblue}{\textbf{\ipa{tsʰo˧ɖwæ˧}}}}\kern2pt]} \hypertarget{ts\string_ho\string_Md`w\{\#\string_T1}{}
\markboth{\textcolor{darkblue}{\textbf{\ipa{tsʰo˧ɖwæ\#˥}}}}{}
\textcolor{teal}{\mytextsc{noun}} \hspace{4pt} Tone: \#H.
\textcolor{Sepia}{\selectlanguage{english}Stone step.} \zh{石头台阶。}  \zh{量词}: \textcolor{darkblue}{\textbf{\ipa{ɖwæ˥}}}  \mytextsc{clf}: \textcolor{darkblue}{\textbf{\ipa{ɖwæ˥}}} 
\lhead{\firstmark}
\rhead{\botmark}

\subsection{\hspace{-0.5cm} {\Large \textcolor{darkblue}{\textbf{\ipa{tsʰo˧ko˧}}}}\hspace{0.5cm}[\kern2pt{\textcolor{darkblue}{\textbf{\ipa{tsʰo˧ko˧}}}}\kern2pt]} \hypertarget{ts\string_ho\string_Mko\string_M1}{}
\markboth{\textcolor{darkblue}{\textbf{\ipa{tsʰo˧ko˧}}}}{}
\textcolor{teal}{\mytextsc{noun}} \hspace{4pt} Tone: M.
\textcolor{Sepia}{\selectlanguage{english}Cardamom, \textit{Elletaria cardamomum}.} \zh{小豆蔻。} Local Chinese dialect:\zh{草果。} Borrowing: \zh{草果}
 \zh{量词}: \textcolor{darkblue}{\textbf{\ipa{ɭɯ˧}}}  \mytextsc{clf}: \textcolor{darkblue}{\textbf{\ipa{ɭɯ˧}}} 
\lhead{\firstmark}
\rhead{\botmark}

\subsection{\hspace{-0.5cm} {\Large \textcolor{darkblue}{\textbf{\ipa{tsʰo˧pæ\#˥}}}}\hspace{0.5cm}[\kern2pt{\textcolor{darkblue}{\textbf{\ipa{tsʰo˧pæ˧}}}}\kern2pt]} \hypertarget{ts\string_ho\string_Mp\{\#\string_T1}{}
\markboth{\textcolor{darkblue}{\textbf{\ipa{tsʰo˧pæ\#˥}}}}{}
\textcolor{teal}{\mytextsc{noun}} \hspace{4pt} Tone: \#H.
\textcolor{Sepia}{\selectlanguage{english}Caravan chief.} \zh{马帮头领。}  Borrowing: Tibetan  tshong.pa “merchant"

\lhead{\firstmark}
\rhead{\botmark}

\subsection{\hspace{-0.5cm} {\Large \textcolor{darkblue}{\textbf{\ipa{tsʰo˧pjɤ˧}}}}\hspace{0.5cm}[\kern2pt{\textcolor{darkblue}{\textbf{\ipa{tsʰo˧pjɤ˧}}}}\kern2pt]} \hypertarget{ts\string_ho\string_Mpj7\string_M1}{}
\markboth{\textcolor{darkblue}{\textbf{\ipa{tsʰo˧pjɤ˧}}}}{}
\textcolor{teal}{\mytextsc{noun}} \hspace{4pt} Tone: M.
\textcolor{Sepia}{\selectlanguage{english}Soap. Presumably borrowed from a language of Burma: cp. Nung /tshɑ³¹ pi⁵⁵ iɔ⁵⁵/ [Dai 1992], Luxi Achang and Lianghe Achang /tshɑu⁵⁵ pjɑu⁵⁵/ [Dai 1985], Longchuan Achang /tshau³¹ piau³¹/ [Dai 1992]. Culturally, it is not unlikely that soap was first introduced through contact/commerce with ethnic groups of Burma.} \zh{肥皂。}  \zh{量词}: \textcolor{darkblue}{\textbf{\ipa{ɭɯ˧}}}  \mytextsc{clf}: \textcolor{darkblue}{\textbf{\ipa{ɭɯ˧}}} 
\lhead{\firstmark}
\rhead{\botmark}

\subsection{\hspace{-0.5cm} {\Large \textcolor{darkblue}{\textbf{\ipa{tsʰo˧qʰwɤ˧mi\#˥}}}}\hspace{0.5cm}[\kern2pt{\textcolor{darkblue}{\textbf{\ipa{tsʰo˧qʰwɤ˧mi˧}}}}\kern2pt]} \hypertarget{ts\string_ho\string_Mq\string_hw7\string_Mmi\#\string_T1}{}
\markboth{\textcolor{darkblue}{\textbf{\ipa{tsʰo˧qʰwɤ˧mi\#˥}}}}{}
\textcolor{teal}{\mytextsc{noun}} \hspace{4pt} Tone: \#H.
\textcolor{Sepia}{\selectlanguage{english}Demon, ghost.} \zh{鬼。}  \zh{量词}: \textcolor{darkblue}{\textbf{\ipa{v̩˧}}}  \mytextsc{clf}: \textcolor{darkblue}{\textbf{\ipa{v̩˧}}} \textit{See:} \hyperlink{}{\textcolor{darkblue}{\textbf{\ipa{tsʰo˧qʰwɤ˧zo\#˥}}}} 
\lhead{\firstmark}
\rhead{\botmark}

\subsection{\hspace{-0.5cm} {\Large \textcolor{darkblue}{\textbf{\ipa{tsʰo˧qʰwɤ˧mi˧-bæ˥bæ˩}}}}\hspace{0.5cm}[\kern2pt{\textcolor{darkblue}{\textbf{\ipa{tsʰo˧qʰwɤ˧mi˧bæ˥bæ˩}}}}\kern2pt]} \hypertarget{ts\string_ho\string_Mq\string_hw7\string_Mmi\string_M-b\{\string_Tb\{\string_B1}{}
\markboth{\textcolor{darkblue}{\textbf{\ipa{tsʰo˧qʰwɤ˧mi˧-bæ˥bæ˩}}}}{}
\textcolor{teal}{\mytextsc{noun}} \hspace{4pt} Tone: \#H-.
\textcolor{Sepia}{\selectlanguage{english}A blue flower, \textit{Delphinium grandiflorum}.} \zh{翠雀花。} \textit{See:} \hyperlink{}{\textcolor{darkblue}{\textbf{\ipa{tsʰo˧qʰwɤ˧mi\#˥}}}} 
\lhead{\firstmark}
\rhead{\botmark}

\subsection{\hspace{-0.5cm} {\Large \textcolor{darkblue}{\textbf{\ipa{tsʰo˧qʰwɤ˧zo\#˥}}}}\hspace{0.5cm}[\kern2pt{\textcolor{darkblue}{\textbf{\ipa{tsʰo˧qʰwɤ˧zo˧}}}}\kern2pt]} \hypertarget{ts\string_ho\string_Mq\string_hw7\string_Mzo\#\string_T1}{}
\markboth{\textcolor{darkblue}{\textbf{\ipa{tsʰo˧qʰwɤ˧zo\#˥}}}}{}
\textcolor{teal}{\mytextsc{noun}} \hspace{4pt} Tone: \#H.
\textcolor{Sepia}{\selectlanguage{english}Demon, ghost (this word is less common than that with a feminine suffix).} \zh{鬼。} \textit{See:} \hyperlink{}{\textcolor{darkblue}{\textbf{\ipa{tsʰo˧qʰwɤ˧mi\#˥}}}} 
\lhead{\firstmark}
\rhead{\botmark}

\subsection{\hspace{-0.5cm} {\Large \textcolor{darkblue}{\textbf{\ipa{tsʰo˧qʰwɤ˩}}}}\hspace{0.5cm}[\kern2pt{\textcolor{darkblue}{\textbf{\ipa{tsʰo˧qʰwɤ˩}}}}\kern2pt]} \hypertarget{ts\string_ho\string_Mq\string_hw7\string_B1}{}
\markboth{\textcolor{darkblue}{\textbf{\ipa{tsʰo˧qʰwɤ˩}}}}{}
\textcolor{teal}{\mytextsc{adverb(ial)}} \hspace{4pt} Tone: L\#.
\textcolor{Sepia}{\selectlanguage{english}Tonight.} \zh{今晚。}  ¶ \textcolor{darkblue}{\textbf{\ipa{tsʰo˧qʰwɤ˩ | mv̩˩kʰv̩˧˥}}} \textcolor{Sepia}{\selectlanguage{english}same meaning: tonight} \zh{同上:今晚}  

\lhead{\firstmark}
\rhead{\botmark}

\subsection{\hspace{-0.5cm} {\Large \textcolor{darkblue}{\textbf{\ipa{tsʰo˧ʁo\#˥}}}}\hspace{0.5cm}[\kern2pt{\textcolor{darkblue}{\textbf{\ipa{tsʰo˧ʁo˧}}}}\kern2pt]} \hypertarget{ts\string_ho\string_MRo\#\string_T1}{}
\markboth{\textcolor{darkblue}{\textbf{\ipa{tsʰo˧ʁo\#˥}}}}{}
\textcolor{teal}{\mytextsc{noun}} \hspace{4pt} Tone: \#H.
\textcolor{Sepia}{\selectlanguage{english}Stable: the building at the entrance of the farm, through which one comes when entering the farm. It is made of wood. On the ground floor, there are stables; hay and straw are stored on the first floor.} \zh{牲畜圈:家门口的那栋楼,下为畜厩,上存饲料或另辟为房间。}  \zh{量词}: \textcolor{darkblue}{\textbf{\ipa{ɭɯ˧}}}  \mytextsc{clf}: \textcolor{darkblue}{\textbf{\ipa{ɭɯ˧}}} 
\lhead{\firstmark}
\rhead{\botmark}

\subsection{\hspace{-0.5cm} {\Large \textcolor{darkblue}{\textbf{\ipa{tsʰo˧tsɯ˧}}}}\hspace{0.5cm}[\kern2pt{\textcolor{darkblue}{\textbf{\ipa{tsʰo˧tsɯ˧}}}}\kern2pt]} \hypertarget{ts\string_ho\string_MtsM\string_M1}{}
\markboth{\textcolor{darkblue}{\textbf{\ipa{tsʰo˧tsɯ˧}}}}{}
\textcolor{teal}{\mytextsc{noun}} \hspace{4pt} Tone: M.
\textcolor{Sepia}{\selectlanguage{english}Onion; leek.} \zh{葱,韭葱。}  Borrowing: Chinese  \zh{葱子}
 \zh{量词}: \textcolor{darkblue}{\textbf{\ipa{po˧}}}  \mytextsc{clf}: \textcolor{darkblue}{\textbf{\ipa{po˧}}} 
\lhead{\firstmark}
\rhead{\botmark}

\subsection{\hspace{-0.5cm} {\Large \textcolor{darkblue}{\textbf{\ipa{tsʰo˩}}}}\hspace{0.5cm}[\kern2pt{\textcolor{darkblue}{\textbf{\ipa{tsʰo˥}}}}\kern2pt]} \hypertarget{ts\string_ho\string_B1}{}
\markboth{\textcolor{darkblue}{\textbf{\ipa{tsʰo˩}}}}{}
\textcolor{teal}{\mytextsc{noun}} \hspace{4pt} Tone: L.
\textcolor{Sepia}{\selectlanguage{english}Human beings; the human species.} \zh{人类。} 
\lhead{\firstmark}
\rhead{\botmark}

\subsection{\hspace{-0.5cm} {\Large \textcolor{darkblue}{\textbf{\ipa{tsʰo˩mo˩}}}}\hspace{0.5cm}[\kern2pt{\textcolor{darkblue}{\textbf{\ipa{tsʰo˩mo˩˥}}}}\kern2pt]} \hypertarget{ts\string_ho\string_Bmo\string_B1}{}
\markboth{\textcolor{darkblue}{\textbf{\ipa{tsʰo˩mo˩}}}}{}
\textcolor{teal}{\mytextsc{noun}} \hspace{4pt} Tone: L.
\textcolor{Sepia}{\selectlanguage{english}Old man.} \zh{老头。} 
\lhead{\firstmark}
\rhead{\botmark}

\subsection{\hspace{-0.5cm} {\Large \textcolor{darkblue}{\textbf{\ipa{tsʰo˩tsɯ˧}}}}\hspace{0.5cm}[\kern2pt{\textcolor{darkblue}{\textbf{\ipa{tsʰo˩tsɯ˥}}}}\kern2pt]} \hypertarget{ts\string_ho\string_BtsM\string_M1}{}
\markboth{\textcolor{darkblue}{\textbf{\ipa{tsʰo˩tsɯ˧}}}}{}
\textcolor{teal}{\mytextsc{noun}} \hspace{4pt} Tone: LM.
\textcolor{Sepia}{\selectlanguage{english}File (tool).} \zh{锉刀。}  Borrowing: Chinese  \zh{锉子}
 \zh{量词}: \textcolor{darkblue}{\textbf{\ipa{nɑ˧}}}  \mytextsc{clf}: \textcolor{darkblue}{\textbf{\ipa{nɑ˧}}} 
\lhead{\firstmark}
\rhead{\botmark}

\subsection{\hspace{-0.5cm} {\Large \textcolor{darkblue}{\textbf{\ipa{tsʰo˧˥}}}}\hspace{0.5cm}[\kern2pt{\textcolor{darkblue}{\textbf{\ipa{tsʰo˧˥}}}}\kern2pt]} \hypertarget{ts\string_ho\string_M\string_T1}{}
\markboth{\textcolor{darkblue}{\textbf{\ipa{tsʰo˧˥}}}}{}
\textcolor{teal}{\mytextsc{noun}} \hspace{4pt} Tone: MH.
\textcolor{Sepia}{\selectlanguage{english}Respect, attention, esteem.} \zh{重视、关心、恭敬。}  ¶ \textcolor{darkblue}{\textbf{\ipa{ʈʂʰɯ˧-ɳɯ˧ | njɤ˧-ki˧ | ɖwæ˧˥ | tsʰo˧ ʝi˥!}}} \textcolor{Sepia}{\selectlanguage{english}He/she treats me with great respect/attention.} \zh{他很重视我 / 他对我很尊敬、很关心。}  

\lhead{\firstmark}
\rhead{\botmark}

\subsection{\hspace{-0.5cm} {\Large \textcolor{darkblue}{\textbf{\ipa{tsʰɯ˧hṽ˥\$}}}}\hspace{0.5cm}[\kern2pt{\textcolor{darkblue}{\textbf{\ipa{tsʰɯ˧hṽ˧}}}}\kern2pt]} \hypertarget{ts\string_hM\string_Mhv\string_~\string_T\$1}{}
\markboth{\textcolor{darkblue}{\textbf{\ipa{tsʰɯ˧hṽ˥\$}}}}{}
\textcolor{teal}{\mytextsc{noun}} \hspace{4pt} Tone: H\$.
\textcolor{Sepia}{\selectlanguage{english}Wool.} \zh{羊毛。}  \zh{量词}: \textcolor{darkblue}{\textbf{\ipa{kʰwɤ˥}}}  \mytextsc{clf}: \textcolor{darkblue}{\textbf{\ipa{kʰwɤ˥}}} 
\lhead{\firstmark}
\rhead{\botmark}

\subsection{\hspace{-0.5cm} {\Large \textcolor{darkblue}{\textbf{\ipa{tsʰɯ˧mi˥\$}}}}\hspace{0.5cm}[\kern2pt{\textcolor{darkblue}{\textbf{\ipa{tsʰɯ˧mi˥}}}}\kern2pt]} \hypertarget{ts\string_hM\string_Mmi\string_T\$1}{}
\markboth{\textcolor{darkblue}{\textbf{\ipa{tsʰɯ˧mi˥\$}}}}{}
\textcolor{teal}{\mytextsc{noun}} \hspace{4pt} Tone: H\$.
\textcolor{Sepia}{\selectlanguage{english}Nanny goat.} \zh{母山羊。}  ¶ \textcolor{darkblue}{\textbf{\ipa{tsʰɯ˧mi˧-po˧lo˥}}} \textcolor{Sepia}{\selectlanguage{english}nanny goat and billy goat} \zh{母山羊与公山羊}  
 \zh{量词}: \textcolor{darkblue}{\textbf{\ipa{pʰo˧˥}}}  \mytextsc{clf}: \textcolor{darkblue}{\textbf{\ipa{pʰo˧˥}}} 
\lhead{\firstmark}
\rhead{\botmark}

\subsection{\hspace{-0.5cm} {\Large \textcolor{darkblue}{\textbf{\ipa{tsʰɯ˧mi˧-to˧qɑ˥\$}}}}\hspace{0.5cm}[\kern2pt{\textcolor{darkblue}{\textbf{\ipa{xxxx non-correspondance entre le nombre de morphèmes et le nombre de tons de morphèmes}}}}\kern2pt]} \hypertarget{ts\string_hM\string_Mmi\string_M-to\string_MqA\string_T\$1}{}
\markboth{\textcolor{darkblue}{\textbf{\ipa{tsʰɯ˧mi˧-to˧qɑ˥\$}}}}{}
\textcolor{teal}{\mytextsc{noun}} \hspace{4pt} Tone: H\$.
\textcolor{Sepia}{\selectlanguage{english}Male goat; also used to refer to a young male goat, or even to goats in general, male and female.} \zh{公山羊(包括公山羊羔)(可以来指所有的山羊,包括母的和公的)。}  \zh{量词}: \textcolor{darkblue}{\textbf{\ipa{pʰo˧˥}}}  \mytextsc{clf}: \textcolor{darkblue}{\textbf{\ipa{pʰo˧˥}}} 
\lhead{\firstmark}
\rhead{\botmark}

\subsection{\hspace{-0.5cm} {\Large \textcolor{darkblue}{\textbf{\ipa{tsʰɯ˧pʰv̩\#˥}}}}\hspace{0.5cm}[\kern2pt{\textcolor{darkblue}{\textbf{\ipa{tsʰɯ˩pʰv̩˩˥}}}}\kern2pt]} \hypertarget{ts\string_hM\string_Mp\string_hv\string_=\#\string_T1}{}
\markboth{\textcolor{darkblue}{\textbf{\ipa{tsʰɯ˧pʰv̩\#˥}}}}{}
\textcolor{teal}{\mytextsc{noun}} \hspace{4pt} Tone: \#H.
\textcolor{Sepia}{\selectlanguage{english}He-goat.} \zh{公山羊。}  \zh{量词}: \textcolor{darkblue}{\textbf{\ipa{pʰo˧˥}}}  \mytextsc{clf}: \textcolor{darkblue}{\textbf{\ipa{pʰo˧˥}}} 
\lhead{\firstmark}
\rhead{\botmark}

\subsection{\hspace{-0.5cm} {\Large \textcolor{darkblue}{\textbf{\ipa{tsʰɯ˧ɻ̍\#˥}}}}\hspace{0.5cm}[\kern2pt{\textcolor{darkblue}{\textbf{\ipa{tsʰɯ˧ɻ̍˧}}}}\kern2pt]} \hypertarget{ts\string_hM\string_Mr£`̍\#\string_T1}{}
\markboth{\textcolor{darkblue}{\textbf{\ipa{tsʰɯ˧ɻ̍\#˥}}}}{}
\textcolor{teal}{\mytextsc{noun}} \hspace{4pt} Tone: \#H.
\textcolor{Sepia}{\selectlanguage{english}A unixex given name: a given name used for both men and women.} \zh{男女通用名。} 
\lhead{\firstmark}
\rhead{\botmark}

\subsection{\hspace{-0.5cm} {\Large \textcolor{darkblue}{\textbf{\ipa{tsʰɯ˧ʂwæ˥}}}}\hspace{0.5cm}[\kern2pt{\textcolor{darkblue}{\textbf{\ipa{tsʰɯ˧ʂwæ˥}}}}\kern2pt]} \hypertarget{ts\string_hM\string_Ms`w\{\string_T1}{}
\markboth{\textcolor{darkblue}{\textbf{\ipa{tsʰɯ˧ʂwæ˥}}}}{}
\textcolor{teal}{\mytextsc{noun}} \hspace{4pt} Tone: H\#.
\textcolor{Sepia}{\selectlanguage{english}Wether (castrated goat, neutered goat).} \zh{阉山羊。}  \zh{量词}: \textcolor{darkblue}{\textbf{\ipa{pʰo˧˥}}}  \mytextsc{clf}: \textcolor{darkblue}{\textbf{\ipa{pʰo˧˥}}} 
\lhead{\firstmark}
\rhead{\botmark}

\subsection{\hspace{-0.5cm} {\Large \textcolor{darkblue}{\textbf{\ipa{tsʰɯ˧-to˧qɑ˥}}}}\hspace{0.5cm}[\kern2pt{\textcolor{darkblue}{\textbf{\ipa{xxxx non-correspondance entre le nombre de morphèmes et le nombre de tons de morphèmes}}}}\kern2pt]} \hypertarget{ts\string_hM\string_M-to\string_MqA\string_T1}{}
\markboth{\textcolor{darkblue}{\textbf{\ipa{tsʰɯ˧-to˧qɑ˥}}}}{}
\textcolor{teal}{\mytextsc{noun}} \hspace{4pt} Tone: H\#.
\textcolor{Sepia}{\selectlanguage{english}Kid (child of the goat).} \zh{羔羊、羔子。}  \zh{量词}: \textcolor{darkblue}{\textbf{\ipa{pʰo˧˥}}}  \mytextsc{clf}: \textcolor{darkblue}{\textbf{\ipa{pʰo˧˥}}} \textit{See:} \hyperlink{}{\textcolor{darkblue}{\textbf{\ipa{tsʰɯ˧zo˥\$}}}} 
\lhead{\firstmark}
\rhead{\botmark}

\subsection{\hspace{-0.5cm} {\Large \textcolor{darkblue}{\textbf{\ipa{tsʰɯ˧zo\#˥}}}}\hspace{0.5cm}[\kern2pt{\textcolor{darkblue}{\textbf{\ipa{tsʰɯ˧zo˧}}}}\kern2pt]} \hypertarget{ts\string_hM\string_Mzo\#\string_T1}{}
\markboth{\textcolor{darkblue}{\textbf{\ipa{tsʰɯ˧zo\#˥}}}}{}
\textcolor{teal}{\mytextsc{noun}} \hspace{4pt} Tone: \#H.
\textcolor{Sepia}{\selectlanguage{english}Young nanny goat.} \zh{母山羊羔。}  \zh{量词}: \textcolor{darkblue}{\textbf{\ipa{ɭɯ˧}}}  \mytextsc{clf}: \textcolor{darkblue}{\textbf{\ipa{ɭɯ˧}}} 
\lhead{\firstmark}
\rhead{\botmark}

\subsection{\hspace{-0.5cm} {\Large \textcolor{darkblue}{\textbf{\ipa{tsʰɯ˧zo˥\$}}}}\hspace{0.5cm}[\kern2pt{\textcolor{darkblue}{\textbf{\ipa{tsʰɯ˧zo˥}}}}\kern2pt]} \hypertarget{ts\string_hM\string_Mzo\string_T\$1}{}
\markboth{\textcolor{darkblue}{\textbf{\ipa{tsʰɯ˧zo˥\$}}}}{}
\textcolor{teal}{\mytextsc{noun}} \hspace{4pt} Tone: H\$.
\textcolor{Sepia}{\selectlanguage{english}Kid (child of the goat).} \zh{山羊羔。}  ¶ \textcolor{darkblue}{\textbf{\ipa{tsʰɯ˧zo˧-to˧qɑ˥}}} \textcolor{Sepia}{\selectlanguage{english}young nanny goat(s) and young kid(s)} \zh{母山羊羔与公山羊羔}  
 \zh{量词}: \textcolor{darkblue}{\textbf{\ipa{ɭɯ˧}}}  \mytextsc{clf}: \textcolor{darkblue}{\textbf{\ipa{ɭɯ˧}}} \textit{See:} \hyperlink{}{\textcolor{darkblue}{\textbf{\ipa{tsʰɯ˧-to˧qɑ˥}}}} 
\lhead{\firstmark}
\rhead{\botmark}

\subsection{\hspace{-0.5cm} {\Large \textcolor{darkblue}{\textbf{\ipa{tsʰɯ˩\textsubscript{a}}}}}\hspace{0.5cm}[\kern2pt{\textcolor{darkblue}{\textbf{\ipa{tsʰɯ˩˥}}}}\kern2pt]} \hypertarget{ts\string_hM\string_Ba1}{}
\markboth{\textcolor{darkblue}{\textbf{\ipa{tsʰɯ˩\textsubscript{a}}}}}{}
\textcolor{teal}{\mytextsc{verb}} \hspace{4pt} Tone: L\textsubscript{a}.
\textcolor{Sepia}{\selectlanguage{english}To come (\mytextsc{pst}).} \zh{来(过去式)。}  ¶ \textcolor{darkblue}{\textbf{\ipa{le˧-gwɤ˩\textasciitilde{}gwɤ˩ | le˧-tsʰɯ˩-ze˩}}} \textcolor{Sepia}{\selectlanguage{english}to be back from a stroll} \zh{散步回来}  
 ¶ \textcolor{darkblue}{\textbf{\ipa{le˧-tsʰɯ˩-ze˩}}} \textcolor{Sepia}{\selectlanguage{english}to be back} \zh{回来了}  
 ¶ \textcolor{darkblue}{\textbf{\ipa{ɖɯ˧-ʝi˧-ɳɯ˧ tsʰɯ˧˥, | ɖɯ˧-ki˧ tʰv̩˧!}}} \textcolor{Sepia}{\selectlanguage{english}“We have come from different places, and now we arrive in the same place / we come together!” This turn of phrase is not intelligible without prior learning, as it literally means “Coming from one place; arriving in one place”.} \zh{“我们都来自不同的地方,但现在在一起了!”}  

\lhead{\firstmark}
\rhead{\botmark}

\subsection{\hspace{-0.5cm} {\Large \textcolor{darkblue}{\textbf{\ipa{tsʰɯ˩tsʰɯ˩ɻ̃˩}}}}\hspace{0.5cm}[\kern2pt{\textcolor{darkblue}{\textbf{\ipa{tsʰɯ˩tsʰɯ˩ɻ̃˩˥}}}}\kern2pt]} \hypertarget{ts\string_hM\string_Bts\string_hM\string_Br£`\string_~\string_B1}{}
\markboth{\textcolor{darkblue}{\textbf{\ipa{tsʰɯ˩tsʰɯ˩ɻ̃˩}}}}{}
\textcolor{teal}{\mytextsc{noun}} \hspace{4pt} Tone: L.
\textcolor{Sepia}{\selectlanguage{english}Dry plant of garden peas, garden peas hay.} \zh{豌豆干草。}  ¶ \textcolor{darkblue}{\textbf{\ipa{ʈʂʰɯ˧ | tsʰɯ˩tsʰɯ˩ɻ̃˩ ɲi˥.}}} \textcolor{Sepia}{\selectlanguage{english}This is garden pea hay.} \zh{这是豌豆干草。}  
 \zh{量词}: \textcolor{darkblue}{\textbf{\ipa{kɤ˧˥}}}  \mytextsc{clf}: \textcolor{darkblue}{\textbf{\ipa{kɤ˧˥}}} 
\lhead{\firstmark}
\rhead{\botmark}

\subsection{\hspace{-0.5cm} {\Large \textcolor{darkblue}{\textbf{\ipa{tsʰɯ˧˥}}} \textsubscript{1}}\hspace{0.5cm}[\kern2pt{\textcolor{darkblue}{\textbf{\ipa{tsʰɯ˥}}}}\kern2pt]} \hypertarget{ts\string_hM\string_M\string_T1}{}
\markboth{\textcolor{darkblue}{\textbf{\ipa{tsʰɯ˧˥}}} \textsubscript{1}}{}
\textcolor{teal}{\mytextsc{verb}} \hspace{4pt} Tone: MH.
\textcolor{Sepia}{\selectlanguage{english}To cut to pieces (e.g. to cut away at a piece of clothing with scissors).} \zh{剪成片。}  ¶ \textcolor{darkblue}{\textbf{\ipa{tʰɑ˧-tsʰɯ˧˥!}}} \textcolor{Sepia}{\selectlanguage{english}\mytextsc{prohib}} \zh{\mytextsc{prohib}}  
 ¶ \textcolor{darkblue}{\textbf{\ipa{dʑi˧hṽ˧ tsʰɯ˩}}} \textcolor{Sepia}{\selectlanguage{english}to cut clothes to pieces} \zh{把衣服剪成片}  

\lhead{\firstmark}
\rhead{\botmark}

\subsection{\hspace{-0.5cm} {\Large \textcolor{darkblue}{\textbf{\ipa{tsʰɯ˧˥}}} \textsubscript{2}}\hspace{0.5cm}[\kern2pt{\textcolor{darkblue}{\textbf{\ipa{tsʰɯ˧˥}}}}\kern2pt]} \hypertarget{ts\string_hM\string_M\string_T2}{}
\markboth{\textcolor{darkblue}{\textbf{\ipa{tsʰɯ˧˥}}} \textsubscript{2}}{}
\textcolor{teal}{\mytextsc{noun}} \hspace{4pt} Tone: MH.
\textcolor{Sepia}{\selectlanguage{english}Goat (male or female).} \zh{山羊。}  \zh{量词}: \textcolor{darkblue}{\textbf{\ipa{pʰo˧˥}}}  \mytextsc{clf}: \textcolor{darkblue}{\textbf{\ipa{pʰo˧˥}}} 
\lhead{\firstmark}
\rhead{\botmark}

\subsection{\hspace{-0.5cm} {\Large \textcolor{darkblue}{\textbf{\ipa{tsʰv̩˩˥}}}}\hspace{0.5cm}[\kern2pt{\textcolor{darkblue}{\textbf{\ipa{tsʰv̩˩˥}}}}\kern2pt]} \hypertarget{ts\string_hv\string_=\string_B\string_T1}{}
\markboth{\textcolor{darkblue}{\textbf{\ipa{tsʰv̩˩˥}}}}{}
\textcolor{teal}{\mytextsc{noun}} \hspace{4pt} Tone: LH.
\textcolor{Sepia}{\selectlanguage{english}Vinegar.} \zh{醋(汉语借词)。}  Borrowing: Chinese  \zh{醋}
\textit{See:} \textcolor{darkblue}{\textbf{\ipa{sɑ˧tsʰv̩˩, tɕi˧-dʑɯ˩}}} 
\lhead{\firstmark}
\rhead{\botmark}

\newpage
\section*{\centering- \textcolor{darkblue}{\textbf{\ipa{ʈ}}} -}
\subsection{\hspace{-0.5cm} {\Large \textcolor{darkblue}{\textbf{\ipa{ʈæ˧bɤ˧}}}}\hspace{0.5cm}[\kern2pt{\textcolor{darkblue}{\textbf{\ipa{ʈæ˩bɤ˩˥}}}}\kern2pt]} \hypertarget{t`\{\string_Mb7\string_M1}{}
\markboth{\textcolor{darkblue}{\textbf{\ipa{ʈæ˧bɤ˧}}}}{}
\textcolor{teal}{\mytextsc{noun}} \hspace{4pt} Tone: M.
\textcolor{Sepia}{\selectlanguage{english}Monk, nun.} \zh{和尚,尼姑。}  ¶ \textcolor{darkblue}{\textbf{\ipa{ʈæ˧bɤ˧ʈʂʰo˧}}} \textcolor{Sepia}{\selectlanguage{english}same meaning} \zh{同上}  
 ¶ \textcolor{darkblue}{\textbf{\ipa{ʈæ˧bɤ˧ ʝi˧-hĩ˧-hĩ˧}}} \textcolor{Sepia}{\selectlanguage{english}person who is a monk} \zh{当和尚的人}  
 ¶ \textcolor{darkblue}{\textbf{\ipa{hæ˧ʈæ˩bɤ˩}}} \textcolor{Sepia}{\selectlanguage{english}Chinese monk} \zh{汉人和尚}  
 \zh{量词}: \textcolor{darkblue}{\textbf{\ipa{v̩˧}}}  \mytextsc{clf}: \textcolor{darkblue}{\textbf{\ipa{v̩˧}}} 
\lhead{\firstmark}
\rhead{\botmark}

\subsection{\hspace{-0.5cm} {\Large \textcolor{darkblue}{\textbf{\ipa{ʈæ˧kwæ˧˥}}}}\hspace{0.5cm}[\kern2pt{\textcolor{darkblue}{\textbf{\ipa{ʈæ˧kwæ˧˥}}}}\kern2pt]} \hypertarget{t`\{\string_Mkw\{\string_M\string_T1}{}
\markboth{\textcolor{darkblue}{\textbf{\ipa{ʈæ˧kwæ˧˥}}}}{}
\textcolor{teal}{\mytextsc{adjective}} \hspace{4pt} Tone: MH\#.
\textcolor{Sepia}{\selectlanguage{english}Prodigal, wasteful.} \zh{爱浪费。}  ¶ \textcolor{darkblue}{\textbf{\ipa{ʈʂʰɯ˧ | ʈæ˧kwæ˧-hĩ˥ | ɖɯ˧-v̩˧ ɲi˩.}}} \textcolor{Sepia}{\selectlanguage{english}(S)he is a prodigal person.} \zh{他是爱浪费的人。}  

\lhead{\firstmark}
\rhead{\botmark}

\subsection{\hspace{-0.5cm} {\Large \textcolor{darkblue}{\textbf{\ipa{ʈæ˧qo˧}}}}\hspace{0.5cm}[\kern2pt{\textcolor{darkblue}{\textbf{\ipa{ʈæ˧qo˧}}}}\kern2pt]} \hypertarget{t`\{\string_Mqo\string_M1}{}
\markboth{\textcolor{darkblue}{\textbf{\ipa{ʈæ˧qo˧}}}}{}
\textcolor{teal}{\mytextsc{adverb(ial)}} \hspace{4pt} Tone: M.
\textcolor{Sepia}{\selectlanguage{english}At bottom, at the bottom of.} \zh{底下。}  ¶ \textcolor{darkblue}{\textbf{\ipa{hi˩nɑ˧mi˧-ʈæ˧qo˥}}} \textcolor{Sepia}{\selectlanguage{english}at the bottom of the Lake} \zh{在湖底下}  
 ¶ \textcolor{darkblue}{\textbf{\ipa{ʈæ˧qo˧ tɕɯ˧}}} \textcolor{Sepia}{\selectlanguage{english}to place at the bottom of...} \zh{放在底下}  
 ¶ \textcolor{darkblue}{\textbf{\ipa{hi˩nɑ˧mi˧, | ʈæ˧ mɤ˧-do˩; | hĩ˧-nv̩˥mi˩, | ɳv̩˧ mɤ˧-tʰɑ˩.}}} \textcolor{Sepia}{\selectlanguage{english}“One can't see to the bottom of the Lake; one can't know the heart of men.” (Proverb that comes up in courting songs.)} \zh{“人的心,湖底藏:看不清,摸不透!” 直译:“湖,(我们)看不到(它的)底下。人的心,是知道不了的!”(情歌里的一个谚语)}  

\lhead{\firstmark}
\rhead{\botmark}

\subsection{\hspace{-0.5cm} {\Large \textcolor{darkblue}{\textbf{\ipa{ʈæ˧ʂɯ˧}}}}\hspace{0.5cm}[\kern2pt{\textcolor{darkblue}{\textbf{\ipa{ʈæ˧ʂɯ˧}}}}\kern2pt]} \hypertarget{t`\{\string_Ms`M\string_M1}{}
\markboth{\textcolor{darkblue}{\textbf{\ipa{ʈæ˧ʂɯ˧}}}}{}
\textcolor{teal}{\mytextsc{noun}} \hspace{4pt} Tone: M.
\textcolor{Sepia}{\selectlanguage{english}Masculine given name.} \zh{男性名字。} 
\lhead{\firstmark}
\rhead{\botmark}

\subsection{\hspace{-0.5cm} {\Large \textcolor{darkblue}{\textbf{\ipa{ʈæ˩\textsubscript{a}}}}}\hspace{0.5cm}[\kern2pt{\textcolor{darkblue}{\textbf{\ipa{ʈæ˧˥}}}}\kern2pt]} \hypertarget{t`\{\string_Ba1}{}
\markboth{\textcolor{darkblue}{\textbf{\ipa{ʈæ˩\textsubscript{a}}}}}{}
\textcolor{teal}{\mytextsc{verb}} \hspace{4pt} Tone: L\textsubscript{a}.
\ding{202} \textcolor{Sepia}{\selectlanguage{english}To lock up (animals), to close (a door).} \zh{关(门、羊)。}  ¶ \textcolor{darkblue}{\textbf{\ipa{bv̩˩qo˩ ʈæ˥}}} \textcolor{Sepia}{\selectlanguage{english}to close the stable} \zh{关牛圈}  
 ¶ \textcolor{darkblue}{\textbf{\ipa{tʰi˧-ʈæ˩}}} \textcolor{Sepia}{\selectlanguage{english}\mytextsc{dur}: to close} \zh{关门}  
 ¶ \textcolor{darkblue}{\textbf{\ipa{kʰi˧ ʈæ˥}}} \textcolor{Sepia}{\selectlanguage{english}to close the door} \zh{关门}  
\ding{203} \textcolor{Sepia}{\selectlanguage{english}To tie (a knot).} \zh{扣(扣子)、系、结。} 
\lhead{\firstmark}
\rhead{\botmark}

\subsection{\hspace{-0.5cm} {\Large \textcolor{darkblue}{\textbf{\ipa{ʈæ˩ɖɯ˧}}}}\hspace{0.5cm}[\kern2pt{\textcolor{darkblue}{\textbf{\ipa{ʈæ˧ɖɯ˧}}}}\kern2pt]} \hypertarget{t`\{\string_Bd`M\string_M1}{}
\markboth{\textcolor{darkblue}{\textbf{\ipa{ʈæ˩ɖɯ˧}}}}{}
\textcolor{teal}{\mytextsc{adjective}} \hspace{4pt} Tone: LM.
\textcolor{Sepia}{\selectlanguage{english}Satisfied, quiet.} \zh{安乐。}  ¶ \textcolor{darkblue}{\textbf{\ipa{mɤ˧-ʈæ˩ɖɯ˩}}} \textcolor{Sepia}{\selectlanguage{english}dissatisfied, restless} \zh{不高兴、不安}  
 ¶ \textcolor{darkblue}{\textbf{\ipa{ə˧mɑ˧ | tsʰi˧-ɲi˧ | ʈæ˩ɖɯ˧ tʰi˧-dzi˩-dʑo˩!}}} \textcolor{Sepia}{\selectlanguage{english}Today, Ama is sitting quietly!} \zh{今天,阿妈安乐地坐着。}  
 ¶ \textcolor{darkblue}{\textbf{\ipa{ʈʂʰɯ˧-ɳɯ˧ | njɤ˧-ki˧ | mɤ˧-ʈæ˩ɖɯ˩-hĩ˩ ʐwɤ˩!}}} \textcolor{Sepia}{\selectlanguage{english}He told me unpleasant things! / He told me vexing things! / He told me some things that make me frustrated/dissatisfied!} \zh{他跟我说了一些让我不安的(事情)! / 他跟我说的,让我生气!}  

\lhead{\firstmark}
\rhead{\botmark}

\subsection{\hspace{-0.5cm} {\Large \textcolor{darkblue}{\textbf{\ipa{ʈæ˩tsʰo\#˥}}} \textsubscript{1}}\hspace{0.5cm}[\kern2pt{\textcolor{darkblue}{\textbf{\ipa{ʈæ˩tsʰo˥}}}}\kern2pt]} \hypertarget{t`\{\string_Bts\string_ho\#\string_T1}{}
\markboth{\textcolor{darkblue}{\textbf{\ipa{ʈæ˩tsʰo\#˥}}} \textsubscript{1}}{}
\textcolor{teal}{\mytextsc{noun}} \hspace{4pt} Tone: LM+\#H.
\textcolor{Sepia}{\selectlanguage{english}Class, group, set (of monks).} \zh{班、小组。}  ¶ \textcolor{darkblue}{\textbf{\ipa{ʈæ˩tsʰo˧ | ɖɯ˧-ɭɯ˧}}} \textcolor{Sepia}{\selectlanguage{english}a group (of priests)} \zh{一个小组、一帮(和尚)}  
 \zh{量词}: \textcolor{darkblue}{\textbf{\ipa{ɭɯ˧}}}  \mytextsc{clf}: \textcolor{darkblue}{\textbf{\ipa{ɭɯ˧}}} \textit{See:} \hyperlink{}{\textcolor{darkblue}{\textbf{\ipa{ʈæ˩tsʰo\#˥}}} \textsubscript{2}} 
\lhead{\firstmark}
\rhead{\botmark}

\subsection{\hspace{-0.5cm} {\Large \textcolor{darkblue}{\textbf{\ipa{ʈæ˩tsʰo\#˥}}} \textsubscript{2}}\hspace{0.5cm}[\kern2pt{\textcolor{darkblue}{\textbf{\ipa{ʈæ˩tsʰo˩˥}}}}\kern2pt]} \hypertarget{t`\{\string_Bts\string_ho\#\string_T2}{}
\markboth{\textcolor{darkblue}{\textbf{\ipa{ʈæ˩tsʰo\#˥}}} \textsubscript{2}}{}
\textcolor{teal}{\mytextsc{classifier}} \hspace{4pt} Tone: L.
\textcolor{Sepia}{\selectlanguage{english}Daeco.} \zh{量词:和尚(一帮、一班)。}  ¶ \textcolor{darkblue}{\textbf{\ipa{ɖɯ˧-ʈæ˩tsʰo˩}}} \textcolor{Sepia}{\selectlanguage{english}a group (of monks)} \zh{一班(和尚)}  
\textit{See:} \hyperlink{}{\textcolor{darkblue}{\textbf{\ipa{ʈæ˩tsʰo\#˥}}} \textsubscript{1}} 
\lhead{\firstmark}
\rhead{\botmark}

\subsection{\hspace{-0.5cm} {\Large \textcolor{darkblue}{\textbf{\ipa{ʈæ˩ʈv̩\#˥}}}}\hspace{0.5cm}[\kern2pt{\textcolor{darkblue}{\textbf{\ipa{ʈæ˩ʈv̩˥}}}}\kern2pt]} \hypertarget{t`\{\string_Bt`v\string_=\#\string_T1}{}
\markboth{\textcolor{darkblue}{\textbf{\ipa{ʈæ˩ʈv̩\#˥}}}}{}
\textcolor{teal}{\mytextsc{noun}} \hspace{4pt} Tone: LM+\#H.
\textcolor{Sepia}{\selectlanguage{english}Masculine given name.} \zh{男性名字。} 
\lhead{\firstmark}
\rhead{\botmark}

\subsection{\hspace{-0.5cm} {\Large \textcolor{darkblue}{\textbf{\ipa{ʈɤ˧\textsubscript{a}}}}}\hspace{0.5cm}[\kern2pt{\textcolor{darkblue}{\textbf{\ipa{ʈɤ˥}}}}\kern2pt]} \hypertarget{t`7\string_Ma1}{}
\markboth{\textcolor{darkblue}{\textbf{\ipa{ʈɤ˧\textsubscript{a}}}}}{}
\textcolor{teal}{\mytextsc{verb}} \hspace{4pt} Tone: M\textsubscript{a}.
\textcolor{Sepia}{\selectlanguage{english}To pull.} \zh{拉、拽。}  ¶ \textcolor{darkblue}{\textbf{\ipa{tso˧\textasciitilde{}tso˧ ʈɤ˩(-ze˩)}}} \textcolor{Sepia}{\selectlanguage{english}to pull something} \zh{拉拽东西}  
 ¶ \textcolor{darkblue}{\textbf{\ipa{mv̩˧ʐe˧ qʰæ˩ | le˧-wo˧-ʈɤ˥-di˩}}} \textcolor{Sepia}{\selectlanguage{english}periphrase to refer to the trigger of a gun: the part that one pulls to shoot} \zh{扳机}  

\lhead{\firstmark}
\rhead{\botmark}

\subsection{\hspace{-0.5cm} {\Large \textcolor{darkblue}{\textbf{\ipa{ʈi˥\textsubscript{a}}}}}\hspace{0.5cm}[\kern2pt{\textcolor{darkblue}{\textbf{\ipa{ʈi˩˥}}}}\kern2pt]} \hypertarget{t`i\string_Ta1}{}
\markboth{\textcolor{darkblue}{\textbf{\ipa{ʈi˥\textsubscript{a}}}}}{}
\textcolor{teal}{\mytextsc{classifier}} \hspace{4pt} Tone: H\textsubscript{a}.
\textcolor{Sepia}{\selectlanguage{english}A handspan (between the thumb and index). The distance between the thumb and the middle finger is not commonly used.} \zh{量词:拃(大拇指和食指之间的距离。一般不用大拇指和中指之间的距离。)。} 
\lhead{\firstmark}
\rhead{\botmark}

\subsection{\hspace{-0.5cm} {\Large \textcolor{darkblue}{\textbf{\ipa{ʈi˩\textsubscript{a}}}}}\hspace{0.5cm}[\kern2pt{\textcolor{darkblue}{\textbf{\ipa{ʈi˩˥}}}}\kern2pt]} \hypertarget{t`i\string_Ba1}{}
\markboth{\textcolor{darkblue}{\textbf{\ipa{ʈi˩\textsubscript{a}}}}}{}
\textcolor{teal}{\mytextsc{verb}} \hspace{4pt} Tone: L\textsubscript{a}.
\textcolor{Sepia}{\selectlanguage{english}To get up.} \zh{起(如:起来,起床)。}  ¶ \textcolor{darkblue}{\textbf{\ipa{gɤ˩-ʈi˧}}} \textcolor{Sepia}{\selectlanguage{english}to get up} \zh{起来}  
 ¶ \textcolor{darkblue}{\textbf{\ipa{ʑi˧ ʈi˥}}} \textcolor{Sepia}{\selectlanguage{english}to wake up} \zh{醒来}  
 ¶ \textcolor{darkblue}{\textbf{\ipa{ʑi˧ gɤ˧-ʈi˩}}} \textcolor{Sepia}{\selectlanguage{english}to wake up} \zh{醒来}  
 ¶ \textcolor{darkblue}{\textbf{\ipa{gɤ˩ mɤ˥-ʈi˩}}} \textcolor{Sepia}{\selectlanguage{english}not to get up} \zh{不起床}  
 ¶ \textcolor{darkblue}{\textbf{\ipa{mɤ˧-ʈi˩-sɯ˩!}}} \textcolor{Sepia}{\selectlanguage{english}(She/he) has not got up yet / is not up yet!} \zh{还没起床!}  
 ¶ \textcolor{darkblue}{\textbf{\ipa{le˧-ʈi˩-ze˩!}}} \textcolor{Sepia}{\selectlanguage{english}(She/he) has got up!} \zh{起床了!}  
 ¶ \textcolor{darkblue}{\textbf{\ipa{ɖɯ˧-ʈi˧\textasciitilde{}ʈi˥-ɻ̍˩}}} \textcolor{Sepia}{\selectlanguage{english}\mytextsc{delimitative} \mytextsc{red} \mytextsc{inceptive}} \zh{起来一下}  

\lhead{\firstmark}
\rhead{\botmark}

\subsection{\hspace{-0.5cm} {\Large \textcolor{darkblue}{\textbf{\ipa{ʈɯ˧\textsubscript{a}}}}}\hspace{0.5cm}[\kern2pt{\textcolor{darkblue}{\textbf{\ipa{ʈɯ˧˥}}}}\kern2pt]} \hypertarget{t`M\string_Ma1}{}
\markboth{\textcolor{darkblue}{\textbf{\ipa{ʈɯ˧\textsubscript{a}}}}}{}
\textcolor{teal}{\mytextsc{verb}} \hspace{4pt} Tone: M\textsubscript{a}.
\textcolor{Sepia}{\selectlanguage{english}To set in place.} \zh{安装、摆好。}  ¶ \textcolor{darkblue}{\textbf{\ipa{ʂe˧kʰɯ˧ tʰi˧-ʈɯ˧˥, | v̩˧ | tʰi˧-ʈɯ˧}}} \textcolor{Sepia}{\selectlanguage{english}to set the tripod in place (in the hearth); to set the large pot in place (as part of the final steps in setting up the new home, after a new house has been built)} \zh{(建完新房后)安装三脚架、把锅摆好(在三脚架上)}  
 ¶ \textcolor{darkblue}{\textbf{\ipa{tsʰo˩-ɻ̃˩˥ | dʑɯ˩ mɤ˩-ʈɯ˩˥, | lɑ˧-ʂe˧ | kʰv̩˧ tʰɑ˩-ki˩!}}} \textcolor{Sepia}{\selectlanguage{english}“Human bones must not be put in water; tiger's flesh must not be given to the dog!” (Explanation: corpses were not buried in water, unlike in certain Tibetan customs. Neither the body, nor the ashes of cremation, must be put in water.)} \zh{“人骨头,莫碰水!老虎肉,莫给狗!”(这个谚语,来强调摩梭与藏族的一些不同习惯:摩梭禁止让尸体或骨灰沾水。)}  

\lhead{\firstmark}
\rhead{\botmark}

\subsection{\hspace{-0.5cm} {\Large \textcolor{darkblue}{\textbf{\ipa{ʈɯ˧ʈʰæ\#˥}}}}\hspace{0.5cm}[\kern2pt{\textcolor{darkblue}{\textbf{\ipa{ʈɯ˩ʈʰæ˩˥}}}}\kern2pt]} \hypertarget{t`M\string_Mt`\string_h\{\#\string_T1}{}
\markboth{\textcolor{darkblue}{\textbf{\ipa{ʈɯ˧ʈʰæ\#˥}}}}{}
\textcolor{teal}{\mytextsc{noun}} \hspace{4pt} Tone: \#H.
\textcolor{Sepia}{\selectlanguage{english}Patrimony, family wealth, property.} \zh{家底、财产(贵重物品)。}  ¶ \textcolor{darkblue}{\textbf{\ipa{ɑ˩ʁo˧ ʈɯ˧ʈʰæ˧!}}} \textcolor{Sepia}{\selectlanguage{english}Prosperity to the family!} \zh{祝你们家发财!}  
 ¶ \textcolor{darkblue}{\textbf{\ipa{ʈʂʰɯ˧ | ʈɯ˧ʈʰæ˧ | ɖwæ˧˥ | dʑo˧-ʝi˧!}}} \textcolor{Sepia}{\selectlanguage{english}(S)he has a large patrimony / His/her family is rich!} \zh{他家底很好! / 他家有钱!}  
 \zh{量词}: \textcolor{darkblue}{\textbf{\ipa{kʰwɤ˥}}}  \mytextsc{clf}: \textcolor{darkblue}{\textbf{\ipa{kʰwɤ˥}}} 
\lhead{\firstmark}
\rhead{\botmark}

\subsection{\hspace{-0.5cm} {\Large \textcolor{darkblue}{\textbf{\ipa{ʈɯ˧˥}}}}\hspace{0.5cm}[\kern2pt{\textcolor{darkblue}{\textbf{\ipa{ʈɯ˥}}}}\kern2pt]} \hypertarget{t`M\string_M\string_T1}{}
\markboth{\textcolor{darkblue}{\textbf{\ipa{ʈɯ˧˥}}}}{}
\textcolor{teal}{\mytextsc{verb}} \hspace{4pt} Tone: MH.
\textcolor{Sepia}{\selectlanguage{english}To blanch: to scald with boiling water, as a preliminary stage in cooking (e.g. for dried vegetables) or in making thread (from linen).} \zh{以滚水将蔬菜或亚麻灼过。}  ¶ \textcolor{darkblue}{\textbf{\ipa{tʰi˧-ʈɯ˧˥}}} \textcolor{Sepia}{\selectlanguage{english}\mytextsc{dur}} \zh{\mytextsc{dur}}  
 ¶ \textcolor{darkblue}{\textbf{\ipa{dʑɯ˩tsʰi˩-qo˥ | tʰi˧-ʈɯ˧˥ / dʑɯ˩tsʰi˩-qo˥ | ʈɯ˧˥}}} \textcolor{Sepia}{\selectlanguage{english}to blanch with boiling water} \zh{以滚水灼过}  
 ¶ \textcolor{darkblue}{\textbf{\ipa{dʑɯ˧-qo˧ | ʈɯ˧˥}}} \textcolor{Sepia}{\selectlanguage{english}to blanch with water} \zh{以水灼过}  
 ¶ \textcolor{darkblue}{\textbf{\ipa{v˩tsʰɤ˧ ʈɯ˥}}} \textcolor{Sepia}{\selectlanguage{english}to blanch vegetables} \zh{灼蔬菜}  
 ¶ \textcolor{darkblue}{\textbf{\ipa{sɑ˧, | ʈɯ˧-kv˥!}}} \textcolor{Sepia}{\selectlanguage{english}Linen needs to be blanched!} \zh{亚麻,要灼过!}  

\lhead{\firstmark}
\rhead{\botmark}

\subsection{\hspace{-0.5cm} {\Large \textcolor{darkblue}{\textbf{\ipa{ʈv̩˩}}}}\hspace{0.5cm}[\kern2pt{\textcolor{darkblue}{\textbf{\ipa{ʈv̩˥}}}}\kern2pt]} \hypertarget{t`v\string_=\string_B1}{}
\markboth{\textcolor{darkblue}{\textbf{\ipa{ʈv̩˩}}}}{}
\textcolor{teal}{\mytextsc{noun}} \hspace{4pt} Tone: L.
\textcolor{Sepia}{\selectlanguage{english}Knot.} \zh{死扣、死结。}  ¶ \textcolor{darkblue}{\textbf{\ipa{ɖɯ˧-ʈv̩˩}}} \textcolor{Sepia}{\selectlanguage{english}a knot} \zh{一个死结}  
 ¶ \textcolor{darkblue}{\textbf{\ipa{ɖɯ˧-ʈv̩˩ | tʰi˧-ʈv̩˩}}} \textcolor{Sepia}{\selectlanguage{english}to tie a knot} \zh{打一个死结}  

\lhead{\firstmark}
\rhead{\botmark}

\subsection{\hspace{-0.5cm} {\Large \textcolor{darkblue}{\textbf{\ipa{ʈv̩˩\textsubscript{a}}}} \textsubscript{1}}\hspace{0.5cm}[\kern2pt{\textcolor{darkblue}{\textbf{\ipa{ʈv̩˧˥}}}}\kern2pt]} \hypertarget{t`v\string_=\string_Ba1}{}
\markboth{\textcolor{darkblue}{\textbf{\ipa{ʈv̩˩\textsubscript{a}}}} \textsubscript{1}}{}
\textcolor{teal}{\mytextsc{verb}} \hspace{4pt} Tone: L\textsubscript{a}.
\textcolor{Sepia}{\selectlanguage{english}To weave (bamboo).} \zh{编(竹子)。}  ¶ \textcolor{darkblue}{\textbf{\ipa{qʰwɤ˧tʰv̩˧ ʈv̩˥}}} \textcolor{Sepia}{\selectlanguage{english}to weave a basket for carrying water} \zh{编背水的背篓}  
 ¶ \textcolor{darkblue}{\textbf{\ipa{mi˩ɬi˩ ʈv̩˥}}} \textcolor{Sepia}{\selectlanguage{english}to weave bamboo} \zh{编竹子}  
 ¶ \textcolor{darkblue}{\textbf{\ipa{tso˧\textasciitilde{}tso˧ ʈv̩˥}}} \textcolor{Sepia}{\selectlanguage{english}to weave things} \zh{编东西}  

\lhead{\firstmark}
\rhead{\botmark}

\subsection{\hspace{-0.5cm} {\Large \textcolor{darkblue}{\textbf{\ipa{ʈv̩˩\textsubscript{a}}}} \textsubscript{2}}\hspace{0.5cm}[\kern2pt{\textcolor{darkblue}{\textbf{\ipa{ʈv̩˩˥}}}}\kern2pt]} \hypertarget{t`v\string_=\string_Ba2}{}
\markboth{\textcolor{darkblue}{\textbf{\ipa{ʈv̩˩\textsubscript{a}}}} \textsubscript{2}}{}
\textcolor{teal}{\mytextsc{verb}} \hspace{4pt} Tone: L\textsubscript{a}.
\textcolor{Sepia}{\selectlanguage{english}To throw (a stone at someone).} \zh{掷(掷石头)。}  ¶ \textcolor{darkblue}{\textbf{\ipa{mɤ˧-ʈv̩˩}}} \textcolor{Sepia}{\selectlanguage{english}\mytextsc{neg}} \zh{\mytextsc{neg}}  
 ¶ \textcolor{darkblue}{\textbf{\ipa{lv̩˧mi˧ ʈv̩˩}}} \textcolor{Sepia}{\selectlanguage{english}to throw a stone} \zh{掷石头}  
 ¶ \textcolor{darkblue}{\textbf{\ipa{tso˧\textasciitilde{}tso˧ ʈv̩˥}}} \textcolor{Sepia}{\selectlanguage{english}to throw things} \zh{掷东西}  

\lhead{\firstmark}
\rhead{\botmark}

\subsection{\hspace{-0.5cm} {\Large \textcolor{darkblue}{\textbf{\ipa{ʈv̩˩\textsubscript{b}}}}}\hspace{0.5cm}[\kern2pt{\textcolor{darkblue}{\textbf{\ipa{ʈv̩˥}}}}\kern2pt]} \hypertarget{t`v\string_=\string_Bb1}{}
\markboth{\textcolor{darkblue}{\textbf{\ipa{ʈv̩˩\textsubscript{b}}}}}{}
\textcolor{teal}{\mytextsc{classifier}} \hspace{4pt} Tone: L\textsubscript{b}.
\textcolor{Sepia}{\selectlanguage{english}Classifier: a large chunk of: a piece larger than a mouthful. The size can range from a chunk of meat corresponding to one serving for one guest, to a quarter of meat weighing several kilos.} \zh{量词:大块,如:一块肉,从一个人的份到几公斤的重量。} 
\lhead{\firstmark}
\rhead{\botmark}

\subsection{\hspace{-0.5cm} {\Large \textcolor{darkblue}{\textbf{\ipa{ʈv̩˩qʰv̩˩}}}}\hspace{0.5cm}[\kern2pt{\textcolor{darkblue}{\textbf{\ipa{ʈv̩˩qʰv̩˩˥}}}}\kern2pt]} \hypertarget{t`v\string_=\string_Bq\string_hv\string_=\string_B1}{}
\markboth{\textcolor{darkblue}{\textbf{\ipa{ʈv̩˩qʰv̩˩}}}}{}
\textcolor{teal}{\mytextsc{noun}} \hspace{4pt} Tone: L.
\textcolor{Sepia}{\selectlanguage{english}Slipknot.} \zh{活扣。}  ¶ \textcolor{darkblue}{\textbf{\ipa{ʈv̩˩qʰv̩˩˥ | tʰi˧-ʈv̩˩}}} \textcolor{Sepia}{\selectlanguage{english}to tie a slipknot} \zh{打活扣}  
 ¶ \textcolor{darkblue}{\textbf{\ipa{ʈv̩˩qʰv̩˩˥ | ɖɯ˧-ɭɯ˧ | tʰi˧-ʈv̩˩}}} \textcolor{Sepia}{\selectlanguage{english}to tie a slipknot} \zh{打一个活扣}  

\lhead{\firstmark}
\rhead{\botmark}

\subsection{\hspace{-0.5cm} {\Large \textcolor{darkblue}{\textbf{\ipa{ʈwæ˩\textsubscript{a}}}}}\hspace{0.5cm}[\kern2pt{\textcolor{darkblue}{\textbf{\ipa{ʈwæ˧˥}}}}\kern2pt]} \hypertarget{t`w\{\string_Ba1}{}
\markboth{\textcolor{darkblue}{\textbf{\ipa{ʈwæ˩\textsubscript{a}}}}}{}
\textcolor{teal}{\mytextsc{verb}} \hspace{4pt} Tone: L\textsubscript{a}.
\textcolor{Sepia}{\selectlanguage{english}To freeze, to solidify.} \zh{冻。}  ¶ \textcolor{darkblue}{\textbf{\ipa{dʑɯ˩ ʈwæ˩˥}}} \textcolor{Sepia}{\selectlanguage{english}water freezes} \zh{水冻成冰}  
 ¶ \textcolor{darkblue}{\textbf{\ipa{dʑɯ˩pʰæ˩ ʈwæ˧-ze˩}}} \textcolor{Sepia}{\selectlanguage{english}ice has formed} \zh{水冻成冰了。}  
 ¶ \textcolor{darkblue}{\textbf{\ipa{ɖɯ˧-ʈwæ˧\textasciitilde{}ʈwæ˥-ɻ̍˩ kʰɯ˩}}} \textcolor{Sepia}{\selectlanguage{english}to put to freeze, to put in the deep freeze} \zh{冷冻,放在冷箱}  

\lhead{\firstmark}
\rhead{\botmark}

\subsection{\hspace{-0.5cm} {\Large \textcolor{darkblue}{\textbf{\ipa{ʈwæ˧˥}}}}\hspace{0.5cm}[\kern2pt{\textcolor{darkblue}{\textbf{\ipa{ʈwæ˥}}}}\kern2pt]} \hypertarget{t`w\{\string_M\string_T1}{}
\markboth{\textcolor{darkblue}{\textbf{\ipa{ʈwæ˧˥}}}}{}
\textcolor{teal}{\mytextsc{verb}} \hspace{4pt} Tone: MH.
\textcolor{Sepia}{\selectlanguage{english}To fall down (on a slippery road).} \zh{跌倒(路很滑)。} 
\lhead{\firstmark}
\rhead{\botmark}

\subsection{\hspace{-0.5cm} {\Large \textcolor{darkblue}{\textbf{\ipa{ʈwɤ˧\textsubscript{a}}}}}\hspace{0.5cm}[\kern2pt{\textcolor{darkblue}{\textbf{\ipa{ʈwɤ˩˥}}}}\kern2pt]} \hypertarget{t`w7\string_Ma1}{}
\markboth{\textcolor{darkblue}{\textbf{\ipa{ʈwɤ˧\textsubscript{a}}}}}{}
\textcolor{teal}{\mytextsc{verb}} \hspace{4pt} Tone: M\textsubscript{a}.
\textcolor{Sepia}{\selectlanguage{english}To sing (of bird), to cock-a-doodle-doo (cock).} \zh{啼,鸡叫。}  ¶ \textcolor{darkblue}{\textbf{\ipa{æ̃˩ ʈwɤ˧ (+ze˧)}}} \textcolor{Sepia}{\selectlanguage{english}The cock sings.} \zh{鸡叫。}  

\lhead{\firstmark}
\rhead{\botmark}

\newpage
\section*{\centering- \textcolor{darkblue}{\textbf{\ipa{ʈʰ}}} -}
\subsection{\hspace{-0.5cm} {\Large \textcolor{darkblue}{\textbf{\ipa{ʈʰæ˧-mɤ˧-ʝi\#˥}}}}\hspace{0.5cm}[\kern2pt{\textcolor{darkblue}{\textbf{\ipa{xxxx non-correspondance entre le nombre de morphèmes et le nombre de tons de morphèmes}}}}\kern2pt]} \hypertarget{t`\string_h\{\string_M-m7\string_M-j££i\#\string_T1}{}
\markboth{\textcolor{darkblue}{\textbf{\ipa{ʈʰæ˧-mɤ˧-ʝi\#˥}}}}{}
\textcolor{teal}{\mytextsc{adjective}} \hspace{4pt} Tone: \#H.
\textcolor{Sepia}{\selectlanguage{english}Disorderly.} \zh{乱。}  ¶ \textcolor{darkblue}{\textbf{\ipa{ɑ˩ʁo˧ | ʈʰæ˧-mɤ˧-ʝi˧! |}}} \textcolor{Sepia}{\selectlanguage{english}The house is in a mess!} \zh{家很乱!}  
 ¶ \textcolor{darkblue}{\textbf{\ipa{ʈʰæ˧-mɤ˧-ʝi˧ ɲi˥! |}}} \textcolor{Sepia}{\selectlanguage{english}It's a real mess!} \zh{真乱!}  

\lhead{\firstmark}
\rhead{\botmark}

\subsection{\hspace{-0.5cm} {\Large \textcolor{darkblue}{\textbf{\ipa{ʈʰæ˧mi˧-ɳɯ˩}}}}\hspace{0.5cm}[\kern2pt{\textcolor{darkblue}{\textbf{\ipa{xxxx non-correspondance entre le nombre de morphèmes et le nombre de tons de morphèmes}}}}\kern2pt]} \hypertarget{t`\string_h\{\string_Mmi\string_M-n`M\string_B1}{}
\markboth{\textcolor{darkblue}{\textbf{\ipa{ʈʰæ˧mi˧-ɳɯ˩}}}}{}
\textcolor{teal}{\mytextsc{adverb(ial)}} \hspace{4pt} Tone: L\#.
\textcolor{Sepia}{\selectlanguage{english}Really, actually.} \zh{真的。}  ¶ \textcolor{darkblue}{\textbf{\ipa{ʈʂʰɯ˧ | ʈʰæ˧mi˧-ɳɯ˩ | go˩˥!}}} \textcolor{Sepia}{\selectlanguage{english}(S)he is really ill!} \zh{他真的病了!}  

\lhead{\firstmark}
\rhead{\botmark}

\subsection{\hspace{-0.5cm} {\Large \textcolor{darkblue}{\textbf{\ipa{-ʈʰæ˧qo˩}}}}\hspace{0.5cm}[\kern2pt{\textcolor{darkblue}{\textbf{\ipa{ʈʰæ˧qo˩}}}}\kern2pt]} \hypertarget{-t`\string_h\{\string_Mqo\string_B1}{}
\markboth{\textcolor{darkblue}{\textbf{\ipa{-ʈʰæ˧qo˩}}}}{}
\textcolor{teal}{\mytextsc{postposition}} \hspace{4pt} Tone: L\#.
\textcolor{Sepia}{\selectlanguage{english}Under.} \zh{……之下、下面。} 
\lhead{\firstmark}
\rhead{\botmark}

\subsection{\hspace{-0.5cm} {\Large \textcolor{darkblue}{\textbf{\ipa{ʈʰæ˧qʰwɤ˧}}}}\hspace{0.5cm}[\kern2pt{\textcolor{darkblue}{\textbf{\ipa{ʈʰæ˧qʰwɤ˧}}}}\kern2pt]} \hypertarget{t`\string_h\{\string_Mq\string_hw7\string_M1}{}
\markboth{\textcolor{darkblue}{\textbf{\ipa{ʈʰæ˧qʰwɤ˧}}}}{}
\textcolor{teal}{\mytextsc{noun}} \hspace{4pt} Tone: M.
\textcolor{Sepia}{\selectlanguage{english}Skirt.} \zh{裙子。}  \zh{量词}: \textcolor{darkblue}{\textbf{\ipa{ɭɯ˧˥}}}  \mytextsc{clf}: \textcolor{darkblue}{\textbf{\ipa{ɭɯ˧˥}}} 
\lhead{\firstmark}
\rhead{\botmark}

\subsection{\hspace{-0.5cm} {\Large \textcolor{darkblue}{\textbf{\ipa{*ʈʰæ˩}}}}\hspace{0.5cm}[\kern2pt{\textcolor{darkblue}{\textbf{\ipa{ʈʰæ˥}}}}\kern2pt]} \hypertarget{*t`\string_h\{\string_B1}{}
\markboth{\textcolor{darkblue}{\textbf{\ipa{*ʈʰæ˩}}}}{}
\textcolor{teal}{\mytextsc{noun}} \hspace{4pt} Tone: L.
\textcolor{Sepia}{\selectlanguage{english}Skirt (monosyllabic form extracted from the set phrase 'to wear the skirt').} \zh{裙子(单音节)。}  ¶ \textcolor{darkblue}{\textbf{\ipa{ʈʰæ˧ | le˧-ki˩}}} \textcolor{Sepia}{\selectlanguage{english}to put on the skirt (\mytextsc{accomp})} \zh{穿上裙子}  
 ¶ \textcolor{darkblue}{\textbf{\ipa{ʈʰæ˩ ki˩˥}}} \textcolor{Sepia}{\selectlanguage{english}the ritual of coming of age for women: “putting on the skirt”} \zh{女人成年的礼仪:“穿裙子”}  

\lhead{\firstmark}
\rhead{\botmark}

\subsection{\hspace{-0.5cm} {\Large \textcolor{darkblue}{\textbf{\ipa{ʈʰæ˩ki˩}}}}\hspace{0.5cm}[\kern2pt{\textcolor{darkblue}{\textbf{\ipa{ʈʰæ˩ki˩˥}}}}\kern2pt]} \hypertarget{t`\string_h\{\string_Bki\string_B1}{}
\markboth{\textcolor{darkblue}{\textbf{\ipa{ʈʰæ˩ki˩}}}}{}
\textcolor{teal}{\mytextsc{verb}} \hspace{4pt} Tone: L.
\textcolor{Sepia}{\selectlanguage{english}To perform the ceremony for a person's coming of age.} \zh{举行女孩的成年礼。}  ¶ \textcolor{darkblue}{\textbf{\ipa{ʈʰæ˩ki˩-ze˥!}}} \textcolor{Sepia}{\selectlanguage{english}She has come of age! / The ceremony for her coming of age has been performed!} \zh{穿裙了! / 行过穿裙礼了! / 她成年了!}  

\lhead{\firstmark}
\rhead{\botmark}

\subsection{\hspace{-0.5cm} {\Large \textcolor{darkblue}{\textbf{\ipa{ʈʰæ˩\textasciitilde{}ʈʰæ˧˥}}}}\hspace{0.5cm}[\kern2pt{\textcolor{darkblue}{\textbf{\ipa{ʈʰæ˧ʈʰæ˧˥}}}}\kern2pt]} \hypertarget{t`\string_h\{\string_B~t`\string_h\{\string_M\string_T1}{}
\markboth{\textcolor{darkblue}{\textbf{\ipa{ʈʰæ˩\textasciitilde{}ʈʰæ˧˥}}}}{}
\textcolor{teal}{\mytextsc{verb}} \hspace{4pt} Tone: MH.
\textcolor{Sepia}{\selectlanguage{english}To itch.} \zh{痒。}  ¶ \textcolor{darkblue}{\textbf{\ipa{le˧-ʈʰæ˩\textasciitilde{}ʈʰæ˩-ze˩}}} \textcolor{Sepia}{\selectlanguage{english}\mytextsc{accomp} \mytextsc{red} \mytextsc{pfv}} \zh{\mytextsc{accomp} \mytextsc{red} \mytextsc{pfv}}  

\lhead{\firstmark}
\rhead{\botmark}

\subsection{\hspace{-0.5cm} {\Large \textcolor{darkblue}{\textbf{\ipa{ʈʰæ˧˥}}}}\hspace{0.5cm}[\kern2pt{\textcolor{darkblue}{\textbf{\ipa{ʈʰæ˧˥}}}}\kern2pt]} \hypertarget{t`\string_h\{\string_M\string_T1}{}
\markboth{\textcolor{darkblue}{\textbf{\ipa{ʈʰæ˧˥}}}}{}
\textcolor{teal}{\mytextsc{verb}} \hspace{4pt} Tone: MH.
\ding{202} \textcolor{Sepia}{\selectlanguage{english}To bite; to sting.} \zh{咬、叮。}  ¶ \textcolor{darkblue}{\textbf{\ipa{tso˧\textasciitilde{}tso˧ ʈʰæ˩(-ze˩)}}} \textcolor{Sepia}{\selectlanguage{english}to bite things} \zh{咬东西}  
 ¶ \textcolor{darkblue}{\textbf{\ipa{hĩ˧ ʈʰæ˩}}} \textcolor{Sepia}{\selectlanguage{english}to bite someone (e.g. a dog bites a stranger)} \zh{咬人}  
\ding{203} \textcolor{Sepia}{\selectlanguage{english}To fit, to adjust, to match: e.g. when building a house, to pieces of carpentry fit each other exactly, and 'bite' into each other to perfection.} \zh{对号、合适、相配:建房时,两块木材调剂地刚好合适,好像互相“咬紧”的样子。} 
\lhead{\firstmark}
\rhead{\botmark}

\subsection{\hspace{-0.5cm} {\Large \textcolor{darkblue}{\textbf{\ipa{ʈʰɤ˥\textsubscript{a}}}}}\hspace{0.5cm}[\kern2pt{\textcolor{darkblue}{\textbf{\ipa{ʈʰɤ˥}}}}\kern2pt]} \hypertarget{t`\string_h7\string_Ta1}{}
\markboth{\textcolor{darkblue}{\textbf{\ipa{ʈʰɤ˥\textsubscript{a}}}}}{}
\textcolor{teal}{\mytextsc{classifier}} \hspace{4pt} Tone: H\textsubscript{a}.
\textcolor{Sepia}{\selectlanguage{english}A drop (of liquid).} \zh{量词:滴。} 
\lhead{\firstmark}
\rhead{\botmark}

\subsection{\hspace{-0.5cm} {\Large \textcolor{darkblue}{\textbf{\ipa{ʈʰɤ˧˥}}}}\hspace{0.5cm}[\kern2pt{\textcolor{darkblue}{\textbf{\ipa{ʈʰɤ˩˥}}}}\kern2pt]} \hypertarget{t`\string_h7\string_M\string_T1}{}
\markboth{\textcolor{darkblue}{\textbf{\ipa{ʈʰɤ˧˥}}}}{}
\textcolor{teal}{\mytextsc{verb}} \hspace{4pt} Tone: MH.
\textcolor{Sepia}{\selectlanguage{english}To drip, to dribble.} \zh{滴(水往下滴)。}  ¶ \textcolor{darkblue}{\textbf{\ipa{tʰi˧-ʈʰɤ˩\textasciitilde{}ʈʰɤ˩}}} \textcolor{Sepia}{\selectlanguage{english}\mytextsc{dur} \mytextsc{red}} \zh{滴着滴着}  

\lhead{\firstmark}
\rhead{\botmark}

\subsection{\hspace{-0.5cm} {\Large \textcolor{darkblue}{\textbf{\ipa{ʈʰi˩\textsubscript{a}}}}}\hspace{0.5cm}[\kern2pt{\textcolor{darkblue}{\textbf{\ipa{ʈʰi˥}}}}\kern2pt]} \hypertarget{t`\string_hi\string_Ba1}{}
\markboth{\textcolor{darkblue}{\textbf{\ipa{ʈʰi˩\textsubscript{a}}}}}{}
\textcolor{teal}{\mytextsc{adjective}} \hspace{4pt} Tone: L\textsubscript{a}.
\textcolor{Sepia}{\selectlanguage{english}Tired, weary.} \zh{累、疲倦、精疲力竭。}  ¶ \textcolor{darkblue}{\textbf{\ipa{le˧-ʈʰi˩-ze˩}}} \textcolor{Sepia}{\selectlanguage{english}\mytextsc{accomp} \string_ \mytextsc{pfv}} \zh{累了}  
 ¶ \textcolor{darkblue}{\textbf{\ipa{njɤ˧ | ʈʰi˩˥!}}} \textcolor{Sepia}{\selectlanguage{english}I am tired!} \zh{我累了!}  
 ¶ \textcolor{darkblue}{\textbf{\ipa{njɤ˧ | ʈʰi˩-ze˥!}}} \textcolor{Sepia}{\selectlanguage{english}I am tired!} \zh{我累了!}  

\lhead{\firstmark}
\rhead{\botmark}

\subsection{\hspace{-0.5cm} {\Large \textcolor{darkblue}{\textbf{\ipa{ʈʰɯ˩\textsubscript{a}}}}}\hspace{0.5cm}[\kern2pt{\textcolor{darkblue}{\textbf{\ipa{ʈʰɯ˥}}}}\kern2pt]} \hypertarget{t`\string_hM\string_Ba1}{}
\markboth{\textcolor{darkblue}{\textbf{\ipa{ʈʰɯ˩\textsubscript{a}}}}}{}
\textcolor{teal}{\mytextsc{verb}} \hspace{4pt} Tone: L\textsubscript{a}.
\textcolor{Sepia}{\selectlanguage{english}To sneeze.} \zh{打喷嚏。}  ¶ \textcolor{darkblue}{\textbf{\ipa{ɖɯ˧-ʈʰɯ˧\textasciitilde{}ʈʰɯ˥}}} \textcolor{Sepia}{\selectlanguage{english}\mytextsc{inceptive} \mytextsc{red}}  

\lhead{\firstmark}
\rhead{\botmark}

\subsection{\hspace{-0.5cm} {\Large \textcolor{darkblue}{\textbf{\ipa{ʈʰɯ˩\textsubscript{b}}}}}\hspace{0.5cm}[\kern2pt{\textcolor{darkblue}{\textbf{\ipa{ʈʰɯ˩˥}}}}\kern2pt]} \hypertarget{t`\string_hM\string_Bb1}{}
\markboth{\textcolor{darkblue}{\textbf{\ipa{ʈʰɯ˩\textsubscript{b}}}}}{}
\textcolor{teal}{\mytextsc{verb}} \hspace{4pt} Tone: L\textsubscript{b}.
\textcolor{Sepia}{\selectlanguage{english}To drink.} \zh{喝。}  ¶ \textcolor{darkblue}{\textbf{\ipa{njɤ˧ | mɤ˧-ʈʰɯ˩}}} \textcolor{Sepia}{\selectlanguage{english}I don't drink} \zh{我不喝}  
 ¶ \textcolor{darkblue}{\textbf{\ipa{ʈʰɯ˩-ze˥}}} \textcolor{Sepia}{\selectlanguage{english}\mytextsc{pfv}} \zh{喝了}  
 ¶ \textcolor{darkblue}{\textbf{\ipa{le˧-ʈʰɯ˩-ze˩}}} \textcolor{Sepia}{\selectlanguage{english}\mytextsc{accomp} \string_ \mytextsc{pfv}} \zh{\mytextsc{accomp} \string_ \mytextsc{pfv}}  
 ¶ \textcolor{darkblue}{\textbf{\ipa{ʐɯ˧ ʈʰɯ˩}}} \textcolor{Sepia}{\selectlanguage{english}to drink wine} \zh{喝酒}  
 ¶ \textcolor{darkblue}{\textbf{\ipa{jɤ˧ ʈʰɯ˩}}} \textcolor{Sepia}{\selectlanguage{english}to smoke (tobacco)} \zh{抽烟}  
 ¶ \textcolor{darkblue}{\textbf{\ipa{dʑɯ˩qʰæ˩ ʈʰɯ˩˥}}} \textcolor{Sepia}{\selectlanguage{english}to drink cold water} \zh{喝凉水}  
 ¶ \textcolor{darkblue}{\textbf{\ipa{dʑɯ˩tsʰi˩ ʈʰɯ˩˥}}} \textcolor{Sepia}{\selectlanguage{english}to drink hot water} \zh{喝热水}  
 ¶ \textcolor{darkblue}{\textbf{\ipa{li˩ ʈʰɯ˩}}} \textcolor{Sepia}{\selectlanguage{english}to drink tea} \zh{喝茶}  
 ¶ \textcolor{darkblue}{\textbf{\ipa{v̩˩dʑɯ˩ ʈʰɯ˩˥}}} \textcolor{Sepia}{\selectlanguage{english}to drink soup} \zh{喝汤}  
 ¶ \textcolor{darkblue}{\textbf{\ipa{dʑɯ˧ ʈʰɯ˧}}} \textcolor{Sepia}{\selectlanguage{english}to drink water} \zh{喝水}  
 ¶ \textcolor{darkblue}{\textbf{\ipa{njɤ˧ | dʑɯ˧ ʈʰɯ˧-ze˧}}} \textcolor{Sepia}{\selectlanguage{english}I have drunk some water} \zh{我喝了水}  
 ¶ \textcolor{darkblue}{\textbf{\ipa{njɤ˧ | dʑɯ˧ ʈʰɯ˧-zo˧-ho˩}}} \textcolor{Sepia}{\selectlanguage{english}I'm going to have to drink water.} \zh{我应该喝水了。}  

\lhead{\firstmark}
\rhead{\botmark}

\newpage
\section*{\centering- \textcolor{darkblue}{\textbf{\ipa{ʈʂ}}} -}
\subsection{\hspace{-0.5cm} {\Large \textcolor{darkblue}{\textbf{\ipa{ʈʂɑ˧tɑ˥}}}}\hspace{0.5cm}[\kern2pt{\textcolor{darkblue}{\textbf{\ipa{ʈʂɑ˧tɑ˥}}}}\kern2pt]} \hypertarget{t`s`A\string_MtA\string_T1}{}
\markboth{\textcolor{darkblue}{\textbf{\ipa{ʈʂɑ˧tɑ˥}}}}{}
\textcolor{teal}{\mytextsc{noun}} \hspace{4pt} Tone: H\#.
\textcolor{Sepia}{\selectlanguage{english}Sign.} \zh{记号。}  ¶ \textcolor{darkblue}{\textbf{\ipa{ʈʂɑ˧tɑ˥ ʝi˩}}} \textcolor{Sepia}{\selectlanguage{english}to make a mark, to write a sign} \zh{写一个符号、画一个符号}  
 ¶ \textcolor{darkblue}{\textbf{\ipa{ʈʂɑ˧tɑ˥ tɕi˩}}} \textcolor{Sepia}{\selectlanguage{english}to write signs, to make marks} \zh{写符号、画符号}  
 \zh{量词}: \textcolor{darkblue}{\textbf{\ipa{kʰwɤ˥}}}  \mytextsc{clf}: \textcolor{darkblue}{\textbf{\ipa{kʰwɤ˥}}} 
\lhead{\firstmark}
\rhead{\botmark}

\subsection{\hspace{-0.5cm} {\Large \textcolor{darkblue}{\textbf{\ipa{ʈʂæ˧mo\#˥}}}}\hspace{0.5cm}[\kern2pt{\textcolor{darkblue}{\textbf{\ipa{ʈʂæ˧mo˧}}}}\kern2pt]} \hypertarget{t`s`\{\string_Mmo\#\string_T1}{}
\markboth{\textcolor{darkblue}{\textbf{\ipa{ʈʂæ˧mo\#˥}}}}{}
\textcolor{teal}{\mytextsc{noun}} \hspace{4pt} Tone: \#H.
\textcolor{Sepia}{\selectlanguage{english}A poisonous mushroom.} \zh{一种有毒的菌子。}  ¶ \textcolor{darkblue}{\textbf{\ipa{ʈʂæ˧mo˧-kʰi˧tɕʰɯ˩-mo˩ / kʰi˧tɕʰɯ˩-mo˩}}} \textcolor{Sepia}{\selectlanguage{english}same meaning} \zh{同上}  
\textit{Syn:} \hyperlink{}{\textcolor{darkblue}{\textbf{\ipa{kʰi˧tɕʰɯ˩-mo˩}}}}. 
\lhead{\firstmark}
\rhead{\botmark}

\subsection{\hspace{-0.5cm} {\Large \textcolor{darkblue}{\textbf{\ipa{ʈʂæ˧ʈʂɯ˧}}}}\hspace{0.5cm}[\kern2pt{\textcolor{darkblue}{\textbf{\ipa{ʈʂæ˧ʈʂɯ˧}}}}\kern2pt]} \hypertarget{t`s`\{\string_Mt`s`M\string_M1}{}
\markboth{\textcolor{darkblue}{\textbf{\ipa{ʈʂæ˧ʈʂɯ˧}}}}{}
\textcolor{teal}{\mytextsc{adverb(ial)}} \hspace{4pt} Tone: M.
\textcolor{Sepia}{\selectlanguage{english}Truthfully, accurately, really.} \zh{确切、真的。} 
\lhead{\firstmark}
\rhead{\botmark}

\subsection{\hspace{-0.5cm} {\Large \textcolor{darkblue}{\textbf{\ipa{ʈʂæ˧wɤ˩}}}}\hspace{0.5cm}[\kern2pt{\textcolor{darkblue}{\textbf{\ipa{ʈʂæ˧wɤ˩}}}}\kern2pt]} \hypertarget{t`s`\{\string_Mw7\string_B1}{}
\markboth{\textcolor{darkblue}{\textbf{\ipa{ʈʂæ˧wɤ˩}}}}{}
\textcolor{teal}{\mytextsc{noun}} \hspace{4pt} Tone: L\#.
\textcolor{Sepia}{\selectlanguage{english}Servant.} \zh{仆人,佣人。}  \zh{量词}: \textcolor{darkblue}{\textbf{\ipa{v̩˧}}}  \mytextsc{clf}: \textcolor{darkblue}{\textbf{\ipa{v̩˧}}} 
\lhead{\firstmark}
\rhead{\botmark}

\subsection{\hspace{-0.5cm} {\Large \textcolor{darkblue}{\textbf{\ipa{ʈʂæ˩do\#˥}}}}\hspace{0.5cm}[\kern2pt{\textcolor{darkblue}{\textbf{\ipa{ʈʂæ˧do˧}}}}\kern2pt]} \hypertarget{t`s`\{\string_Bdo\#\string_T1}{}
\markboth{\textcolor{darkblue}{\textbf{\ipa{ʈʂæ˩do\#˥}}}}{}
\textcolor{teal}{\mytextsc{noun}} \hspace{4pt} Tone: LM+\#H.
\textcolor{Sepia}{\selectlanguage{english}Container in which butter-tea is mixed; also: butter churn.} \zh{打酥油茶的罐、酥油茶搅拌器,黄油搅乳器。}  \zh{量词}: \textcolor{darkblue}{\textbf{\ipa{ɭɯ˧}}}  \mytextsc{clf}: \textcolor{darkblue}{\textbf{\ipa{ɭɯ˧}}} 
\lhead{\firstmark}
\rhead{\botmark}

\subsection{\hspace{-0.5cm} {\Large \textcolor{darkblue}{\textbf{\ipa{ʈʂæ˧˥}}} \textsubscript{1}}\hspace{0.5cm}[\kern2pt{\textcolor{darkblue}{\textbf{\ipa{ʈʂæ˧˥}}}}\kern2pt]} \hypertarget{t`s`\{\string_M\string_T1}{}
\markboth{\textcolor{darkblue}{\textbf{\ipa{ʈʂæ˧˥}}} \textsubscript{1}}{}
\textcolor{teal}{\mytextsc{verb}} \hspace{4pt} Tone: MH.
\textcolor{Sepia}{\selectlanguage{english}To rob, to steal.} \zh{抢劫、抢。}  ¶ \textcolor{darkblue}{\textbf{\ipa{le˧-ʈʂæ˧-ze˥}}} \textcolor{Sepia}{\selectlanguage{english}\mytextsc{accomp} \string_ \mytextsc{pfv}} \zh{抢了}  
 ¶ \textcolor{darkblue}{\textbf{\ipa{tso˧\textasciitilde{}tso˧ ʈʂæ˩}}} \textcolor{Sepia}{\selectlanguage{english}to steal things} \zh{抢东西}  
 ¶ \textcolor{darkblue}{\textbf{\ipa{le˧-ʈʂæ˧-po˥-hɯ˩(-ze˩)}}} \textcolor{Sepia}{\selectlanguage{english}[(S)he] stole away (something)} \zh{把东西抢走了}  
 ¶ \textcolor{darkblue}{\textbf{\ipa{hĩ˧ ʈʂæ˩}}} \textcolor{Sepia}{\selectlanguage{english}to rob people, to steal from people} \zh{抢劫}  

\lhead{\firstmark}
\rhead{\botmark}

\subsection{\hspace{-0.5cm} {\Large \textcolor{darkblue}{\textbf{\ipa{ʈʂæ˧˥}}} \textsubscript{2}}\hspace{0.5cm}[\kern2pt{\textcolor{darkblue}{\textbf{\ipa{ʈʂæ˧˥}}}}\kern2pt]} \hypertarget{t`s`\{\string_M\string_T2}{}
\markboth{\textcolor{darkblue}{\textbf{\ipa{ʈʂæ˧˥}}} \textsubscript{2}}{}
\textcolor{teal}{\mytextsc{verb}} \hspace{4pt} Tone: MH.
\textcolor{Sepia}{\selectlanguage{english}To send someone.} \zh{派人。}  ¶ \textcolor{darkblue}{\textbf{\ipa{ɖɯ˧-v̩˧ ʈʂæ˧˥}}} \textcolor{Sepia}{\selectlanguage{english}to send someone} \zh{派一个人}  
 ¶ \textcolor{darkblue}{\textbf{\ipa{hĩ˧ ʈʂæ˩}}} \textcolor{Sepia}{\selectlanguage{english}as above} \zh{同上}  

\lhead{\firstmark}
\rhead{\botmark}

\subsection{\hspace{-0.5cm} {\Large \textcolor{darkblue}{\textbf{\ipa{ʈʂæ˧˥}}} \textsubscript{3}}\hspace{0.5cm}[\kern2pt{\textcolor{darkblue}{\textbf{\ipa{ʈʂæ˧˥}}}}\kern2pt]} \hypertarget{t`s`\{\string_M\string_T3}{}
\markboth{\textcolor{darkblue}{\textbf{\ipa{ʈʂæ˧˥}}} \textsubscript{3}}{}
\textcolor{teal}{\mytextsc{verb}} \hspace{4pt} Tone: MH.
\textcolor{Sepia}{\selectlanguage{english}To set, to attach (e.g. to sew a button, to put a saddle on a horse).} \zh{安上(如:缝扣子、安上马鞍)。}  ¶ \textcolor{darkblue}{\textbf{\ipa{pv̩˩ɭɯ˥ ʈʂæ˩}}} \textcolor{Sepia}{\selectlanguage{english}to sew a button} \zh{缝扣子}  
 ¶ \textcolor{darkblue}{\textbf{\ipa{ʐwæ˧tɕi˥ ʈʂæ˩}}} \textcolor{Sepia}{\selectlanguage{english}to put a saddle on a horse, to saddle a horse} \zh{备鞍}  
 ¶ \textcolor{darkblue}{\textbf{\ipa{ɖɯ˧-ɲi˥, | so˧-ʂɯ˧ ʈʂæ˧˥!}}} \textcolor{Sepia}{\selectlanguage{english}In one day [of caravan journey], one saddles (horses) three times!} \zh{(走马帮,)一天备鞍三次!}  

\lhead{\firstmark}
\rhead{\botmark}

\subsection{\hspace{-0.5cm} {\Large \textcolor{darkblue}{\textbf{\ipa{ʈʂæ˧˥}}} \textsubscript{4}}\hspace{0.5cm}[\kern2pt{\textcolor{darkblue}{\textbf{\ipa{ʈʂæ˧˥}}}}\kern2pt]} \hypertarget{t`s`\{\string_M\string_T4}{}
\markboth{\textcolor{darkblue}{\textbf{\ipa{ʈʂæ˧˥}}} \textsubscript{4}}{}
\textcolor{teal}{\mytextsc{noun}} \hspace{4pt} Tone: MH.
\ding{202} \textcolor{Sepia}{\selectlanguage{english}Articulation.} \zh{关节。}  \zh{量词}: \textcolor{darkblue}{\textbf{\ipa{ʈʂæ˧˥}}} \ding{203} \textcolor{Sepia}{\selectlanguage{english}Period, epoch, age, era, span of time.} \zh{段(时间)、时代。}  \mytextsc{clf}: \textcolor{darkblue}{\textbf{\ipa{ʈʂæ˧˥}}} 
\lhead{\firstmark}
\rhead{\botmark}

\subsection{\hspace{-0.5cm} {\Large \textcolor{darkblue}{\textbf{\ipa{ʈʂæ˧˥\textsubscript{a}}}}}\hspace{0.5cm}[\kern2pt{\textcolor{darkblue}{\textbf{\ipa{ʈʂæ˥}}}}\kern2pt]} \hypertarget{t`s`\{\string_M\string_Ta1}{}
\markboth{\textcolor{darkblue}{\textbf{\ipa{ʈʂæ˧˥\textsubscript{a}}}}}{}
\textcolor{teal}{\mytextsc{classifier}} \hspace{4pt} Tone: MH\textsubscript{a}.
\textcolor{Sepia}{\selectlanguage{english}Classifier for ears (of sweet corn).} \zh{量词.玉米(一棒)。}  ¶ \textcolor{darkblue}{\textbf{\ipa{qʰɑ˧dze˧ | ɖɯ˧-ʈʂæ˧˥}}} \textcolor{Sepia}{\selectlanguage{english}an ear of sweet corn} \zh{一棒玉米}  
 ¶ \textcolor{darkblue}{\textbf{\ipa{qʰɑ˧dze˧ | ɖɯ˧-ʈʂæ˧ ɖʐɤ˥}}} \textcolor{Sepia}{\selectlanguage{english}to pick an ear of sweet corn} \zh{掰一棒玉米}  

\lhead{\firstmark}
\rhead{\botmark}

\subsection{\hspace{-0.5cm} {\Large \textcolor{darkblue}{\textbf{\ipa{ʈʂe˥}}} \textsubscript{1}}\hspace{0.5cm}[\kern2pt{\textcolor{darkblue}{\textbf{\ipa{ʈʂe˥}}}}\kern2pt]} \hypertarget{t`s`e\string_T1}{}
\markboth{\textcolor{darkblue}{\textbf{\ipa{ʈʂe˥}}} \textsubscript{1}}{}
\textcolor{teal}{\mytextsc{noun}} \hspace{4pt} Tone: \#H.
\textcolor{Sepia}{\selectlanguage{english}Earth.} \zh{土壤。}  ¶ \textcolor{darkblue}{\textbf{\ipa{ʈʂe˧pv̩˩}}} \textcolor{Sepia}{\selectlanguage{english}dry earth} \zh{干土}  
 ¶ \textcolor{darkblue}{\textbf{\ipa{ʈʂe˧ sɯ˧\textasciitilde{}sɯ˥}}} \textcolor{Sepia}{\selectlanguage{english}'raw earth': immature soil, earth that has not been prepared for agriculture by adding manure, etc} \zh{‘生土’:没有经过加工(加肥料等等)的土,还不适合种农作物}  

\lhead{\firstmark}
\rhead{\botmark}

\subsection{\hspace{-0.5cm} {\Large \textcolor{darkblue}{\textbf{\ipa{ʈʂe˥}}} \textsubscript{2}}\hspace{0.5cm}[\kern2pt{\textcolor{darkblue}{\textbf{\ipa{ʈʂe˥}}}}\kern2pt]} \hypertarget{t`s`e\string_T2}{}
\markboth{\textcolor{darkblue}{\textbf{\ipa{ʈʂe˥}}} \textsubscript{2}}{}
\textcolor{teal}{\mytextsc{noun}} \hspace{4pt} Tone: \#H.
\textcolor{Sepia}{\selectlanguage{english}Needle.} \zh{针(汉语借词)。}  \zh{量词}: \textcolor{darkblue}{\textbf{\ipa{ɭɯ˧}}}  \mytextsc{clf}: \textcolor{darkblue}{\textbf{\ipa{ɭɯ˧}}} \textit{See:} \hyperlink{}{\textcolor{darkblue}{\textbf{\ipa{ʁo˧˥}}}} 
\lhead{\firstmark}
\rhead{\botmark}

\subsection{\hspace{-0.5cm} {\Large \textcolor{darkblue}{\textbf{\ipa{ʈʂe˧dɑ˥}}}}\hspace{0.5cm}[\kern2pt{\textcolor{darkblue}{\textbf{\ipa{ʈʂe˧dɑ˥}}}}\kern2pt]} \hypertarget{t`s`e\string_MdA\string_T1}{}
\markboth{\textcolor{darkblue}{\textbf{\ipa{ʈʂe˧dɑ˥}}}}{}
\textcolor{teal}{\mytextsc{noun}} \hspace{4pt} Tone: H\#.
\textcolor{Sepia}{\selectlanguage{english}Partition.} \zh{隔板。}  \zh{量词}: \textcolor{darkblue}{\textbf{\ipa{do˥}}}  \mytextsc{clf}: \textcolor{darkblue}{\textbf{\ipa{do˥}}} 
\lhead{\firstmark}
\rhead{\botmark}

\subsection{\hspace{-0.5cm} {\Large \textcolor{darkblue}{\textbf{\ipa{ʈʂe˧gi˥\$}}}}\hspace{0.5cm}[\kern2pt{\textcolor{darkblue}{\textbf{\ipa{ʈʂe˧gi˥}}}}\kern2pt]} \hypertarget{t`s`e\string_Mgi\string_T\$1}{}
\markboth{\textcolor{darkblue}{\textbf{\ipa{ʈʂe˧gi˥\$}}}}{}
\textcolor{teal}{\mytextsc{adverb(ial)}} \hspace{4pt} Tone: H\$.
\textcolor{Sepia}{\selectlanguage{english}In-between, in the middle of.} \zh{中间、之间、间。}  ¶ \textcolor{darkblue}{\textbf{\ipa{ə˧-sɯ˩kv̩˩-ʈʂe˩gi˩}}} \textcolor{Sepia}{\selectlanguage{english}(in the space) between us, in the space that separates us} \zh{在咱们之间(的空间)}  

\lhead{\firstmark}
\rhead{\botmark}

\subsection{\hspace{-0.5cm} {\Large \textcolor{darkblue}{\textbf{\ipa{ʈʂe˩\textsubscript{a}}}}}\hspace{0.5cm}[\kern2pt{\textcolor{darkblue}{\textbf{\ipa{ʈʂe˩˥}}}}\kern2pt]} \hypertarget{t`s`e\string_Ba1}{}
\markboth{\textcolor{darkblue}{\textbf{\ipa{ʈʂe˩\textsubscript{a}}}}}{}
\textcolor{teal}{\mytextsc{verb}} \hspace{4pt} Tone: L\textsubscript{a}.
\textcolor{Sepia}{\selectlanguage{english}To sting, to pierce (e.g. a thorn).} \zh{刺痛。}  ¶ \textcolor{darkblue}{\textbf{\ipa{le˧-ʈʂe˩-ze˩}}} \textcolor{Sepia}{\selectlanguage{english}\mytextsc{accomp} \string_ \mytextsc{pfv}} \zh{\mytextsc{accomp} \string_ \mytextsc{pfv}}  
 ¶ \textcolor{darkblue}{\textbf{\ipa{tɕʰi˧-ɳɯ˧ ʈʂe˩-ze˩}}} \textcolor{Sepia}{\selectlanguage{english}to be pierced by a thorn, to catch a thorn} \zh{被刺所刺痛}  
 ¶ \textcolor{darkblue}{\textbf{\ipa{tso˧\textasciitilde{}tso˧ ʈʂe˥}}} \textcolor{Sepia}{\selectlanguage{english}to pierce something} \zh{刺到一个东西}  

\lhead{\firstmark}
\rhead{\botmark}

\subsection{\hspace{-0.5cm} {\Large \textcolor{darkblue}{\textbf{\ipa{ʈʂe˩kʰɯ˩}}}}\hspace{0.5cm}[\kern2pt{\textcolor{darkblue}{\textbf{\ipa{ʈʂe˩kʰɯ˩˥}}}}\kern2pt]} \hypertarget{t`s`e\string_Bk\string_hM\string_B1}{}
\markboth{\textcolor{darkblue}{\textbf{\ipa{ʈʂe˩kʰɯ˩}}}}{}
\textcolor{teal}{\mytextsc{noun}} \hspace{4pt} Tone: L.
\textcolor{Sepia}{\selectlanguage{english}Seam.} \zh{缝。}  \zh{量词}: \textcolor{darkblue}{\textbf{\ipa{ɭɯ˧}}}  \mytextsc{clf}: \textcolor{darkblue}{\textbf{\ipa{ɭɯ˧}}} 
\lhead{\firstmark}
\rhead{\botmark}

\subsection{\hspace{-0.5cm} {\Large \textcolor{darkblue}{\textbf{\ipa{ʈʂe˩ʂwæ˧˥}}}}\hspace{0.5cm}[\kern2pt{\textcolor{darkblue}{\textbf{\ipa{ʈʂe˩ʂwæ˧˥}}}}\kern2pt]} \hypertarget{t`s`e\string_Bs`w\{\string_M\string_T1}{}
\markboth{\textcolor{darkblue}{\textbf{\ipa{ʈʂe˩ʂwæ˧˥}}}}{}
\textcolor{teal}{\mytextsc{noun}} \hspace{4pt} Tone: LM+MH\#.
\textcolor{Sepia}{\selectlanguage{english}Grit, gravel.} \zh{砾石。} 
\lhead{\firstmark}
\rhead{\botmark}

\subsection{\hspace{-0.5cm} {\Large \textcolor{darkblue}{\textbf{\ipa{ʈʂɤ˧\textsubscript{a}}}}}\hspace{0.5cm}[\kern2pt{\textcolor{darkblue}{\textbf{\ipa{ʈʂɤ˥}}}}\kern2pt]} \hypertarget{t`s`7\string_Ma1}{}
\markboth{\textcolor{darkblue}{\textbf{\ipa{ʈʂɤ˧\textsubscript{a}}}}}{}
\textcolor{teal}{\mytextsc{verb}} \hspace{4pt} Tone: M\textsubscript{a}.
\ding{202} \textcolor{Sepia}{\selectlanguage{english}To count; to calculate.} \zh{数、算。}  ¶ \textcolor{darkblue}{\textbf{\ipa{ʈʂɤ˧\textasciitilde{}ʈʂɤ˩}}} \textcolor{Sepia}{\selectlanguage{english}\mytextsc{red}} \zh{\mytextsc{重叠:算一算}}  
 ¶ \textcolor{darkblue}{\textbf{\ipa{ɖɯ˧-ʈʂɤ˥\textasciitilde{}ʈʂɤ˩-ɻ̍˩}}} \textcolor{Sepia}{\selectlanguage{english}to do some counting, to take a count} \zh{算一下}  
 ¶ \textcolor{darkblue}{\textbf{\ipa{tso˧\textasciitilde{}tso˧ ʈʂɤ˩}}} \textcolor{Sepia}{\selectlanguage{english}to count things} \zh{数东西}  
 ¶ \textcolor{darkblue}{\textbf{\ipa{hĩ˧ ʈʂɤ˩}}} \textcolor{Sepia}{\selectlanguage{english}to count people} \zh{数人}  
 ¶ \textcolor{darkblue}{\textbf{\ipa{bo˩ ʈʂɤ˧}}} \textcolor{Sepia}{\selectlanguage{english}to count pigs} \zh{数猪}  
 ¶ \textcolor{darkblue}{\textbf{\ipa{le˧-ʈʂɤ˧-ze˧}}} \textcolor{Sepia}{\selectlanguage{english}\mytextsc{accomp} \string_ \mytextsc{pfv}} \zh{数了}  
\ding{203} \textcolor{Sepia}{\selectlanguage{english}To do divination, to do fortune-telling.} \zh{算命。}  ¶ \textcolor{darkblue}{\textbf{\ipa{le˧-ʈʂɤ˥\textasciitilde{}ʈʂɤ˩}}} \textcolor{Sepia}{\selectlanguage{english}to do divination, to do fortune-telling} \zh{算命}  
 ¶ \textcolor{darkblue}{\textbf{\ipa{ɖɯ˧-ʈʂɤ˥\textasciitilde{}ʈʂɤ˩-ɻ̍˩}}} \textcolor{Sepia}{\selectlanguage{english}to do some fortune-telling} \zh{算一下命}  
 ¶ \textcolor{darkblue}{\textbf{\ipa{ɲi˧ŋwɤ˩hɑ̃˩tʰɑ˩ | ɖɯ˧-ɭɯ˧ | ʈʂɤ˧-bi˧!}}} \textcolor{Sepia}{\selectlanguage{english}(We are) going to look for an auspicious day (for an event such as a wedding or the building of a house)} \zh{要掐算一下日子}  
 ¶ \textcolor{darkblue}{\textbf{\ipa{kɯ˧ ʈʂɤ˧, | hɑ̃˧ ʈʂɤ˧}}} \textcolor{Sepia}{\selectlanguage{english}to look for an auspicious day for an important event; literally: “to count stars, to count days”} \zh{掐算一下。直译:“算星星,算日子”。}  
\ding{204} \textcolor{Sepia}{\selectlanguage{english}To count as.} \zh{算是,当作。}  ¶ \textcolor{darkblue}{\textbf{\ipa{hĩ˧ ɖɯ˧-v̩˧ ʈʂɤ˧-ze˧!}}} \textcolor{Sepia}{\selectlanguage{english}(She/he) now counts as a (grown-up) person! / (She/he) can now be considered an adult! (A comment typically made when a child reaches adulthood, at age 13.)} \zh{变成大人了!(十三岁成年礼时常用的一句话)}  
 ¶ \textcolor{darkblue}{\textbf{\ipa{dʑɤ˩ ʈʂɤ˧}}} \textcolor{Sepia}{\selectlanguage{english}to be good, to count as good, to be considered as good} \zh{算是很好的}  
 ¶ \textcolor{darkblue}{\textbf{\ipa{ʈʂʰɯ˧ | õ˧-bv̩˥-õ˩ | dʑɤ˩ʈʂɤ˧ (+ | ʐwæ˩˥)}}} \textcolor{Sepia}{\selectlanguage{english}He thinks highly of himself! / He is proud of himself / conceited!} \zh{他觉得自己很了不起!}  
 ¶ \textcolor{darkblue}{\textbf{\ipa{hɤ˩ ʈʂɤ˩˥}}} \textcolor{Sepia}{\selectlanguage{english}to count as remarkable, to be considered as remarkable} \zh{算很了不起的,算很能干的}  
 ¶ \textcolor{darkblue}{\textbf{\ipa{ɖwæ˧˥ | hɤ˩ ʈʂɤ˩˥}}} \textcolor{Sepia}{\selectlanguage{english}same meaning} \zh{同上}  
 ¶ \textcolor{darkblue}{\textbf{\ipa{ʈʂʰɯ˧ | gi˧zɯ˧ ʈʂɤ˧-tso˧-ɲi˥.}}} \textcolor{Sepia}{\selectlanguage{english}He has the status of little brother! (Comment made to emphasize someone's role in the family.)} \zh{他是做弟弟的!(强调该人的社会角色)}  
 ¶ \textcolor{darkblue}{\textbf{\ipa{ʈʂʰɯ˧ | gi˧zɯ˧ ʈʂɤ˧-ɲi˥!}}} \textcolor{Sepia}{\selectlanguage{english}He has the status of little brother! (Comment made to emphasize someone's role in the family.)} \zh{他是做弟弟的!(强调该人的社会角色)}  
 ¶ \textcolor{darkblue}{\textbf{\ipa{ʈʂʰɯ˧ | bɤ˧ ʈʂɤ˧-tso˧-ɲi˥!}}} \textcolor{Sepia}{\selectlanguage{english}He/she is Pumi! (Comment made to emphasize this aspect of someone's identity.)} \zh{他是普米族!(强调该人的民族)}  
 ¶ \textcolor{darkblue}{\textbf{\ipa{ʈʂʰɯ˧ | nɑ˩ ʈʂɤ˧-tso˧-ɲi˥!}}} \textcolor{Sepia}{\selectlanguage{english}He/she is Na! (Comment made to emphasize this aspect of someone's identity.)} \zh{他是摩梭人!(强调该人的民族身份)}  
 ¶ \textcolor{darkblue}{\textbf{\ipa{ʈʂʰɯ˧ | æ˧mv̩˩ ʈʂɤ˩-ɲi˩!}}} \textcolor{Sepia}{\selectlanguage{english}She has the status of elder sister! / He has the status of elder brother! (Comment made to emphasize someone's role in the family.)} \zh{她是做姐姐的! / 他是做哥哥的!(强调该人的社会角色)}  
 ¶ \textcolor{darkblue}{\textbf{\ipa{ʈʂʰɯ˧ | gi˧zɯ˧-go˩mi˩ ʈʂɤ˩-ɲi˩!}}} \textcolor{Sepia}{\selectlanguage{english}They are brothers and sisters!} \zh{他们是(兄)弟(姐)妹!}  
 ¶ \textcolor{darkblue}{\textbf{\ipa{ʈʂʰɯ˧ | æ˧mv̩˧-go˧mi˥ | ʈʂɤ˧-tso˧ mɤ˧-ɲi˥! | mɤ˧-ʈʂɤ˧!}}} \textcolor{Sepia}{\selectlanguage{english}They can't be considered as brothers and sisters! / They are not actually brothers and sisters!} \zh{他们不算是兄弟姐妹!}  
 ¶ \textcolor{darkblue}{\textbf{\ipa{ʐwæ˩ ʈʂɤ˩}}} \textcolor{Sepia}{\selectlanguage{english}remarkable, great, exceptional, outstanding} \zh{了不起}  
 ¶ \textcolor{darkblue}{\textbf{\ipa{ʈʂʰɯ˧ | ə˧tso˧ ʐwæ˩ ʈʂɤ˩-tso˩ dʑo˩?}}} \textcolor{Sepia}{\selectlanguage{english}What's so remarkable about her/him?} \zh{他有什么了不起的?}  
 ¶ \textcolor{darkblue}{\textbf{\ipa{ʈʂʰɯ˧ | ʐwæ˩ ʈʂɤ˩˥ | ʐwæ˩˥!}}} \textcolor{Sepia}{\selectlanguage{english}It's really an outstanding person!} \zh{他非常了不起!}  

\lhead{\firstmark}
\rhead{\botmark}

\subsection{\hspace{-0.5cm} {\Large \textcolor{darkblue}{\textbf{\ipa{ʈʂo˧kʰɯ˩}}}}\hspace{0.5cm}[\kern2pt{\textcolor{darkblue}{\textbf{\ipa{ʈʂo˧kʰɯ˩}}}}\kern2pt]} \hypertarget{t`s`o\string_Mk\string_hM\string_B1}{}
\markboth{\textcolor{darkblue}{\textbf{\ipa{ʈʂo˧kʰɯ˩}}}}{}
\textcolor{teal}{\mytextsc{noun}} \hspace{4pt} Tone: L\#.
\textcolor{Sepia}{\selectlanguage{english}Ritual performed on the occasion of the death of a family member.} \zh{忠克:亲人去世时举行的仪式。} 
\lhead{\firstmark}
\rhead{\botmark}

\subsection{\hspace{-0.5cm} {\Large \textcolor{darkblue}{\textbf{\ipa{ʈʂo˧ɭɯ\#˥}}}}\hspace{0.5cm}[\kern2pt{\textcolor{darkblue}{\textbf{\ipa{ʈʂo˧ɭɯ˧}}}}\kern2pt]} \hypertarget{t`s`o\string_Ml\string_RM\#\string_T1}{}
\markboth{\textcolor{darkblue}{\textbf{\ipa{ʈʂo˧ɭɯ\#˥}}}}{}
\textcolor{teal}{\mytextsc{noun}} \hspace{4pt} Tone: \#H.
\textcolor{Sepia}{\selectlanguage{english}Hand-operated grindstone.} \zh{手推磨。}  ¶ \textcolor{darkblue}{\textbf{\ipa{ʈʂo˧ɭɯ˧-nv̩˥mi˩}}} \textcolor{Sepia}{\selectlanguage{english}the axis of the grindstone (literally: its hear)} \zh{手推磨的轴心}  
 \zh{量词}: \textcolor{darkblue}{\textbf{\ipa{nɑ˧}}}  \mytextsc{clf}: \textcolor{darkblue}{\textbf{\ipa{nɑ˧}}} 
\lhead{\firstmark}
\rhead{\botmark}

\subsection{\hspace{-0.5cm} {\Large \textcolor{darkblue}{\textbf{\ipa{ʈʂo˧ɭɯ˧ʈʂo˧˥}}}}\hspace{0.5cm}[\kern2pt{\textcolor{darkblue}{\textbf{\ipa{ʈʂo˧ɭɯ˧ʈʂo˧˥}}}}\kern2pt]} \hypertarget{t`s`o\string_Ml\string_RM\string_Mt`s`o\string_M\string_T1}{}
\markboth{\textcolor{darkblue}{\textbf{\ipa{ʈʂo˧ɭɯ˧ʈʂo˧˥}}}}{}
\textcolor{teal}{\mytextsc{noun}} \hspace{4pt} Tone: MH\#.
\textcolor{Sepia}{\selectlanguage{english}A water insect.} \zh{一种水虫。}  \zh{量词}: \textcolor{darkblue}{\textbf{\ipa{mi˩}}}  \mytextsc{clf}: \textcolor{darkblue}{\textbf{\ipa{mi˩}}} 
\lhead{\firstmark}
\rhead{\botmark}

\subsection{\hspace{-0.5cm} {\Large \textcolor{darkblue}{\textbf{\ipa{ʈʂo˧ʂɯ\#˥}}}}\hspace{0.5cm}[\kern2pt{\textcolor{darkblue}{\textbf{\ipa{ʈʂo˧ʂɯ˧}}}}\kern2pt]} \hypertarget{t`s`o\string_Ms`M\#\string_T1}{}
\markboth{\textcolor{darkblue}{\textbf{\ipa{ʈʂo˧ʂɯ\#˥}}}}{}
\textcolor{teal}{\mytextsc{noun}} \hspace{4pt} Tone: \#H.
\textcolor{Sepia}{\selectlanguage{english}Name of a village: Zhongshi.} \zh{忠实(永宁的一个村落)。}  ¶ \textcolor{darkblue}{\textbf{\ipa{ɖæ˩ʂɯ\#˥, | ʈʂo˧ʂɯ\#˥, | bɤ˩tɕʰɯ˩˥, | dɑ˧pʰo˥, | bɤ˧dzi˩, | dze˧bo˧}}} \textcolor{Sepia}{\selectlanguage{english}the six villages of the plain of Yongning, in traditional order: by order of increasing distance from the Lake} \zh{永宁坝的六个村落,按传统排序:从距离泸沽湖最近的村落说起。}  

\lhead{\firstmark}
\rhead{\botmark}

\subsection{\hspace{-0.5cm} {\Large \textcolor{darkblue}{\textbf{\ipa{ʈʂo˧tsɯ˥}}}}\hspace{0.5cm}[\kern2pt{\textcolor{darkblue}{\textbf{\ipa{ʈʂo˧tsɯ˥}}}}\kern2pt]} \hypertarget{t`s`o\string_MtsM\string_T1}{}
\markboth{\textcolor{darkblue}{\textbf{\ipa{ʈʂo˧tsɯ˥}}}}{}
\textcolor{teal}{\mytextsc{noun}} \hspace{4pt} Tone: H\#.
\textcolor{Sepia}{\selectlanguage{english}Table.} \zh{桌子(汉语借词)。}  Borrowing: Chinese  \zh{桌子}
 \zh{量词}: \textcolor{darkblue}{\textbf{\ipa{pɤ˩}}}  \mytextsc{clf}: \textcolor{darkblue}{\textbf{\ipa{pɤ˩}}} \textit{See:} \hyperlink{}{\textcolor{darkblue}{\textbf{\ipa{sɯ˧ɻæ˧}}}} 
\lhead{\firstmark}
\rhead{\botmark}

\subsection{\hspace{-0.5cm} {\Large \textcolor{darkblue}{\textbf{\ipa{ʈʂo˩}}}}\hspace{0.5cm}[\kern2pt{\textcolor{darkblue}{\textbf{\ipa{ʈʂo˩˥}}}}\kern2pt]} \hypertarget{t`s`o\string_B1}{}
\markboth{\textcolor{darkblue}{\textbf{\ipa{ʈʂo˩}}}}{}
\textcolor{teal}{\mytextsc{classifier}} \hspace{4pt} Tone: L.
\textcolor{Sepia}{\selectlanguage{english}Classifier for meals.} \zh{量词:饭(一顿)。}  ¶ \textcolor{darkblue}{\textbf{\ipa{ɖɯ˧-ʈʂo˩ tʰi˩-pæ˩ |}}} \textcolor{Sepia}{\selectlanguage{english}to serve a meal, to set a meal} \zh{摆饭,摆饭桌}  
 ¶ \textcolor{darkblue}{\textbf{\ipa{ʐo˩˥, | njɤ˧ ɖɯ˧-ʈʂo˩ pæ˩-bi˩!}}} \textcolor{Sepia}{\selectlanguage{english}For lunch, I will serve a (whole) meal! / I'm taking charge of lunch!} \zh{我来管午饭这一顿!}  
 ¶ \textcolor{darkblue}{\textbf{\ipa{hɑ˧ ɖɯ˧-ʈʂo˩}}} \textcolor{Sepia}{\selectlanguage{english}a meal} \zh{一顿饭}  

\lhead{\firstmark}
\rhead{\botmark}

\subsection{\hspace{-0.5cm} {\Large \textcolor{darkblue}{\textbf{\ipa{ʈʂo˩bo˩}}}}\hspace{0.5cm}[\kern2pt{\textcolor{darkblue}{\textbf{\ipa{ʈʂo˧bo˧}}}}\kern2pt]} \hypertarget{t`s`o\string_Bbo\string_B1}{}
\markboth{\textcolor{darkblue}{\textbf{\ipa{ʈʂo˩bo˩}}}}{}
\textcolor{teal}{\mytextsc{noun}} \hspace{4pt} Tone: L.
\textcolor{Sepia}{\selectlanguage{english}Earthen wall.} \zh{土墙。}  ¶ \textcolor{darkblue}{\textbf{\ipa{ʈʂo˩bo˩ ti˥}}} \textcolor{Sepia}{\selectlanguage{english}to build a wall of earth, by pounding the earth} \zh{垒墙}  
 \zh{量词}: \textcolor{darkblue}{\textbf{\ipa{do˥}}}  \mytextsc{clf}: \textcolor{darkblue}{\textbf{\ipa{do˥}}} 
\lhead{\firstmark}
\rhead{\botmark}

\subsection{\hspace{-0.5cm} {\Large \textcolor{darkblue}{\textbf{\ipa{ʈʂo˩mv̩˩}}}}\hspace{0.5cm}[\kern2pt{\textcolor{darkblue}{\textbf{\ipa{ʈʂo˩mv̩˩˥}}}}\kern2pt]} \hypertarget{t`s`o\string_Bmv\string_=\string_B1}{}
\markboth{\textcolor{darkblue}{\textbf{\ipa{ʈʂo˩mv̩˩}}}}{}
\textcolor{teal}{\mytextsc{noun}} \hspace{4pt} Tone: L.
\textcolor{Sepia}{\selectlanguage{english}Fine sand.} \zh{沙子。} 
\lhead{\firstmark}
\rhead{\botmark}

\subsection{\hspace{-0.5cm} {\Large \textcolor{darkblue}{\textbf{\ipa{ʈʂɻ̍˥}}}}\hspace{0.5cm}[\kern2pt{\textcolor{darkblue}{\textbf{\ipa{ʈʂɻ̍˥}}}}\kern2pt]} \hypertarget{t`s`r£`̍\string_T1}{}
\markboth{\textcolor{darkblue}{\textbf{\ipa{ʈʂɻ̍˥}}}}{}
\textcolor{teal}{\mytextsc{verb}} \hspace{4pt} Tone: H.
\textcolor{Sepia}{\selectlanguage{english}To cough.} \zh{咳嗽。}  ¶ \textcolor{darkblue}{\textbf{\ipa{ʈʂʰɯ˧ | tʰi˧-ʈʂɻ̍˥-dʑo˩}}} \textcolor{Sepia}{\selectlanguage{english}(S)he is coughing.} \zh{他在咳嗽}  

\lhead{\firstmark}
\rhead{\botmark}

\subsection{\hspace{-0.5cm} {\Large \textcolor{darkblue}{\textbf{\ipa{ʈʂɯ˧}}}}\hspace{0.5cm}[\kern2pt{\textcolor{darkblue}{\textbf{\ipa{ʈʂɯ˧˥}}}}\kern2pt]} \hypertarget{t`s`M\string_M1}{}
\markboth{\textcolor{darkblue}{\textbf{\ipa{ʈʂɯ˧}}}}{}
\textcolor{teal}{\mytextsc{noun}} \hspace{4pt} Tone: M.
\textcolor{Sepia}{\selectlanguage{english}Claws.} \zh{爪子。}  \zh{量词}: \textcolor{darkblue}{\textbf{\ipa{ɭɯ˧}}}  \mytextsc{clf}: \textcolor{darkblue}{\textbf{\ipa{ɭɯ˧}}} \textit{See:} \hyperlink{}{\textcolor{darkblue}{\textbf{\ipa{kv̩˧ʈʂɯ˧˥}}}} 
\lhead{\firstmark}
\rhead{\botmark}

\subsection{\hspace{-0.5cm} {\Large \textcolor{darkblue}{\textbf{\ipa{ʈʂɯ˧dzi˩}}}}\hspace{0.5cm}[\kern2pt{\textcolor{darkblue}{\textbf{\ipa{ʈʂɯ˩dzi˩˥}}}}\kern2pt]} \hypertarget{t`s`M\string_Mdzi\string_B1}{}
\markboth{\textcolor{darkblue}{\textbf{\ipa{ʈʂɯ˧dzi˩}}}}{}
\textcolor{teal}{\mytextsc{noun}} \hspace{4pt} Tone: L\#.
\textcolor{Sepia}{\selectlanguage{english}Lacquer tree, varnish tree.} \zh{漆树。} 
\lhead{\firstmark}
\rhead{\botmark}

\subsection{\hspace{-0.5cm} {\Large \textcolor{darkblue}{\textbf{\ipa{ʈʂɯ˧fv̩\#˥}}}}\hspace{0.5cm}[\kern2pt{\textcolor{darkblue}{\textbf{\ipa{ʈʂɯ˧fv̩˩}}}}\kern2pt]} \hypertarget{t`s`M\string_Mfv\string_=\#\string_T1}{}
\markboth{\textcolor{darkblue}{\textbf{\ipa{ʈʂɯ˧fv̩\#˥}}}}{}
\textcolor{teal}{\mytextsc{noun}} \hspace{4pt} Tone: \#H.
\textcolor{Sepia}{\selectlanguage{english}Local government.} \zh{(土)知府,如:永宁知府(汉语借词)。}  Borrowing: Chinese  \zh{知府}
 ¶ \textcolor{darkblue}{\textbf{\ipa{no˧ | ɬi˧di˩-ʈʂɯ˩fv̩˩-ni˩-zo˩!}}} \textcolor{Sepia}{\selectlanguage{english}You think you're the government, hey?! (A criticism to people who keep telling others how they should behave, as if they lorded it over everyone else.)} \zh{你像永宁土知府! / 你是永宁土知府吧!(批评独断的人、一手包办的人)}  
 ¶ \textcolor{darkblue}{\textbf{\ipa{no˧ | ʈʂɯ˧fv̩˧-mi˧-ni˧˥ | -zo˩!}}} \textcolor{Sepia}{\selectlanguage{english}You are the Princess of Yongning, hey?! (Criticism addressed to an overbearing woman)} \zh{你好像是永宁大公主! / 你好像是永宁知府女主人!(批评一个独断的女人)}  

\lhead{\firstmark}
\rhead{\botmark}

\subsection{\hspace{-0.5cm} {\Large \textcolor{darkblue}{\textbf{\ipa{ʈʂɯ˧mɤ˩}}}}\hspace{0.5cm}[\kern2pt{\textcolor{darkblue}{\textbf{\ipa{ʈʂɯ˧mɤ˥}}}}\kern2pt]} \hypertarget{t`s`M\string_Mm7\string_B1}{}
\markboth{\textcolor{darkblue}{\textbf{\ipa{ʈʂɯ˧mɤ˩}}}}{}
\textcolor{teal}{\mytextsc{noun}} \hspace{4pt} Tone: L\#.
\textcolor{Sepia}{\selectlanguage{english}Sesame.} \zh{芝麻(汉语借词)。}  Borrowing: Chinese  \zh{芝麻}
 ¶ \textcolor{darkblue}{\textbf{\ipa{ʈʂɯ˧mɤ˩, | ɬi˧di˩ | mɤ˧-tʰv̩˧-ɲi˥!}}} \textcolor{Sepia}{\selectlanguage{english}Sesame does not grow in Yongning! / Sesame isn't grown in Yongning!} \zh{永宁不产芝麻!}  

\lhead{\firstmark}
\rhead{\botmark}

\subsection{\hspace{-0.5cm} {\Large \textcolor{darkblue}{\textbf{\ipa{ʈʂɯ˧˥}}}}\hspace{0.5cm}[\kern2pt{\textcolor{darkblue}{\textbf{\ipa{ʈʂɯ˥}}}}\kern2pt]} \hypertarget{t`s`M\string_M\string_T1}{}
\markboth{\textcolor{darkblue}{\textbf{\ipa{ʈʂɯ˧˥}}}}{}
\textcolor{teal}{\mytextsc{verb}} \hspace{4pt} Tone: MH.
\textcolor{Sepia}{\selectlanguage{english}To sift.} \zh{筛。}  ¶ \textcolor{darkblue}{\textbf{\ipa{le˧-ʈʂɯ˧-ze˥}}} \textcolor{Sepia}{\selectlanguage{english}\mytextsc{accomp} \string_ \mytextsc{pfv}} \zh{\mytextsc{accomp} \string_ \mytextsc{pfv}}  
 ¶ \textcolor{darkblue}{\textbf{\ipa{ɖɯ˧-ʈʂɯ˧-ɻ̍˥}}} \textcolor{Sepia}{\selectlanguage{english}\mytextsc{delimitative} \string_ \mytextsc{inceptive}} \zh{\mytextsc{delimitative} \string_ \mytextsc{inceptive}}  

\lhead{\firstmark}
\rhead{\botmark}

\subsection{\hspace{-0.5cm} {\Large \textcolor{darkblue}{\textbf{\ipa{ʈʂv̩˩}}}}\hspace{0.5cm}[\kern2pt{\textcolor{darkblue}{\textbf{\ipa{ʈʂv̩˩˥}}}}\kern2pt]} \hypertarget{t`s`v\string_=\string_B1}{}
\markboth{\textcolor{darkblue}{\textbf{\ipa{ʈʂv̩˩}}}}{}
\textcolor{teal}{\mytextsc{adjective}} \hspace{4pt} Tone: L.
\textcolor{Sepia}{\selectlanguage{english}Peaceful, soft (astrological sign).} \zh{平和的(生肖)。}  ¶ \textcolor{darkblue}{\textbf{\ipa{kʰv̩˧ ʈʂv̩˧˥}}} \textcolor{Sepia}{\selectlanguage{english}a peaceful, soft astrological sign, such as the Ox, the Rabbit and the Goat} \zh{平和的生肖,如牛、兔、羊}  

\lhead{\firstmark}
\rhead{\botmark}

\subsection{\hspace{-0.5cm} {\Large \textcolor{darkblue}{\textbf{\ipa{ʈʂv̩˩˥}}}}\hspace{0.5cm}[\kern2pt{\textcolor{darkblue}{\textbf{\ipa{ʈʂv̩˩˥}}}}\kern2pt]} \hypertarget{t`s`v\string_=\string_B\string_T1}{}
\markboth{\textcolor{darkblue}{\textbf{\ipa{ʈʂv̩˩˥}}}}{}
\textcolor{teal}{\mytextsc{noun}} \hspace{4pt} Tone: LH.
\textcolor{Sepia}{\selectlanguage{english}Sweat (monosyllable).} \zh{汗(单音节)。}  ¶ \textcolor{darkblue}{\textbf{\ipa{ʈʂv̩˧ bv̩˧nv̩˩}}} \textcolor{Sepia}{\selectlanguage{english}smelling of sweat, reeking of sweat} \zh{有汗(臭)的味道}  

\lhead{\firstmark}
\rhead{\botmark}

\subsection{\hspace{-0.5cm} {\Large \textcolor{darkblue}{\textbf{\ipa{ʈʂv̩˩\textsubscript{a}}}} \textsubscript{1}}\hspace{0.5cm}[\kern2pt{\textcolor{darkblue}{\textbf{\ipa{ʈʂv̩˩˥}}}}\kern2pt]} \hypertarget{t`s`v\string_=\string_Ba1}{}
\markboth{\textcolor{darkblue}{\textbf{\ipa{ʈʂv̩˩\textsubscript{a}}}} \textsubscript{1}}{}
\textcolor{teal}{\mytextsc{verb}} \hspace{4pt} Tone: L\textsubscript{a}.
\textcolor{Sepia}{\selectlanguage{english}To contaminate, to infect.} \zh{传染。}  ¶ \textcolor{darkblue}{\textbf{\ipa{hĩ˧ ʈʂv̩˥-ho˩}}} \textcolor{Sepia}{\selectlanguage{english}(the disease) is going to contaminate someone / is going to contaminate people} \zh{(病毒)会传染人的}  
 ¶ \textcolor{darkblue}{\textbf{\ipa{ʈʂv̩˧\textasciitilde{}ʈʂv̩˥}}} \textcolor{Sepia}{\selectlanguage{english}\mytextsc{red}} \zh{\mytextsc{重叠}}  
 ¶ \textcolor{darkblue}{\textbf{\ipa{ʈʂv̩˧\textasciitilde{}ʈʂv̩˥-ɻ̍˩ ho˩}}} \textcolor{Sepia}{\selectlanguage{english}(the disease) is going to contaminate (people)} \zh{(病毒)会传染的。}  
 ¶ \textcolor{darkblue}{\textbf{\ipa{njɤ˧-ɳɯ˧ | no˧ ʈʂv̩˧-ʝi˥!}}} \textcolor{Sepia}{\selectlanguage{english}(Be careful,) I may contaminate you / pass on my cold to you!} \zh{(要小心:)我会传染你的!}  

\lhead{\firstmark}
\rhead{\botmark}

\subsection{\hspace{-0.5cm} {\Large \textcolor{darkblue}{\textbf{\ipa{ʈʂv̩˩\textsubscript{a}}}} \textsubscript{2}}\hspace{0.5cm}[\kern2pt{\textcolor{darkblue}{\textbf{\ipa{ʈʂv̩˩˥}}}}\kern2pt]} \hypertarget{t`s`v\string_=\string_Ba2}{}
\markboth{\textcolor{darkblue}{\textbf{\ipa{ʈʂv̩˩\textsubscript{a}}}} \textsubscript{2}}{}
\textcolor{teal}{\mytextsc{verb}} \hspace{4pt} Tone: L\textsubscript{a}.
\textcolor{Sepia}{\selectlanguage{english}To light (a candle).} \zh{点(蜡烛……)。} 
\lhead{\firstmark}
\rhead{\botmark}

\subsection{\hspace{-0.5cm} {\Large \textcolor{darkblue}{\textbf{\ipa{ʈʂv̩˧di˧˥}}}}\hspace{0.5cm}[\kern2pt{\textcolor{darkblue}{\textbf{\ipa{ʈʂv̩˧di˧˥}}}}\kern2pt]} \hypertarget{t`s`v\string_=\string_Mdi\string_M\string_T1}{}
\markboth{\textcolor{darkblue}{\textbf{\ipa{ʈʂv̩˧di˧˥}}}}{}
\textcolor{teal}{\mytextsc{noun}} \hspace{4pt} Tone: MH\#.
\textcolor{Sepia}{\selectlanguage{english}Name of a village outside the plain of Lijiang, in the vicinity of the Lake, close to \textcolor{darkblue}{\textbf{\ipa{/lɑ˧tʰɑ˧-di˧˥/}}}.} \zh{村落名。} 
\lhead{\firstmark}
\rhead{\botmark}

\subsection{\hspace{-0.5cm} {\Large \textcolor{darkblue}{\textbf{\ipa{ʈʂv̩˩dʑɯ˥}}}}\hspace{0.5cm}[\kern2pt{\textcolor{darkblue}{\textbf{\ipa{ʈʂv̩˩dʑɯ˥}}}}\kern2pt]} \hypertarget{t`s`v\string_=\string_Bdz£M\string_T1}{}
\markboth{\textcolor{darkblue}{\textbf{\ipa{ʈʂv̩˩dʑɯ˥}}}}{}
\textcolor{teal}{\mytextsc{noun}} \hspace{4pt} Tone: LH.
\textcolor{Sepia}{\selectlanguage{english}Sweat.} \zh{汗。} 
\lhead{\firstmark}
\rhead{\botmark}

\subsection{\hspace{-0.5cm} {\Large \textcolor{darkblue}{\textbf{\ipa{ʈʂv̩˧pɤ˩}}}}\hspace{0.5cm}[\kern2pt{\textcolor{darkblue}{\textbf{\ipa{ʈʂv̩˧pɤ˩}}}}\kern2pt]} \hypertarget{t`s`v\string_=\string_Mp7\string_B1}{}
\markboth{\textcolor{darkblue}{\textbf{\ipa{ʈʂv̩˧pɤ˩}}}}{}
\textcolor{teal}{\mytextsc{noun}} \hspace{4pt} Tone: L\#.
\textcolor{Sepia}{\selectlanguage{english}Cutting-board; vessel or cutting board for meat.} \zh{菜板、俎。}  \zh{量词}: \textcolor{darkblue}{\textbf{\ipa{nɑ˧}}}  \mytextsc{clf}: \textcolor{darkblue}{\textbf{\ipa{nɑ˧}}} 
\lhead{\firstmark}
\rhead{\botmark}

\subsection{\hspace{-0.5cm} {\Large \textcolor{darkblue}{\textbf{\ipa{ʈʂv̩˧tɕɯ˥}}}}\hspace{0.5cm}[\kern2pt{\textcolor{darkblue}{\textbf{\ipa{ʈʂv̩˧tɕɯ˥}}}}\kern2pt]} \hypertarget{t`s`v\string_=\string_Mts£M\string_T1}{}
\markboth{\textcolor{darkblue}{\textbf{\ipa{ʈʂv̩˧tɕɯ˥}}}}{}
\textcolor{teal}{\mytextsc{noun}} \hspace{4pt} Tone: H\#.
\textcolor{Sepia}{\selectlanguage{english}Spittle, phlegm, sputum.} \zh{痰。} 
\lhead{\firstmark}
\rhead{\botmark}

\subsection{\hspace{-0.5cm} {\Large \textcolor{darkblue}{\textbf{\ipa{ʈʂwɑ˧\textasciitilde{}ʈʂwɑ˧-nɑ˧\textasciitilde{}nɑ\#˥}}}}\hspace{0.5cm}[\kern2pt{\textcolor{darkblue}{\textbf{\ipa{xxxx non-correspondance entre le nombre de morphèmes et le nombre de tons de morphèmes}}}}\kern2pt]} \hypertarget{t`s`wA\string_M~t`s`wA\string_M-nA\string_M~nA\#\string_T1}{}
\markboth{\textcolor{darkblue}{\textbf{\ipa{ʈʂwɑ˧\textasciitilde{}ʈʂwɑ˧-nɑ˧\textasciitilde{}nɑ\#˥}}}}{}
\textcolor{teal}{\mytextsc{adjective}} \hspace{4pt} Tone: \#H.
\textcolor{Sepia}{\selectlanguage{english}Mixed; diverse, heterogeneous; messy.} \zh{杂、混杂。}  ¶ \textcolor{darkblue}{\textbf{\ipa{ʈʂwɑ˧\textasciitilde{}ʈʂwɑ˧-nɑ˧\textasciitilde{}nɑ˧-hĩ˥}}} \textcolor{Sepia}{\selectlanguage{english}\mytextsc{rel}/nmlz} \zh{混杂的}  
 ¶ \textcolor{darkblue}{\textbf{\ipa{ʈʂwɑ˧\textasciitilde{}ʈʂwɑ˧-nɑ˧\textasciitilde{}nɑ˧-ɻ̍˥}}} \textcolor{Sepia}{\selectlanguage{english}messy} \zh{混杂}  

\lhead{\firstmark}
\rhead{\botmark}

\subsection{\hspace{-0.5cm} {\Large \textcolor{darkblue}{\textbf{\ipa{ʈʂwæ˥\textsubscript{a}}}}}\hspace{0.5cm}[\kern2pt{\textcolor{darkblue}{\textbf{\ipa{ʈʂwæ˧˥}}}}\kern2pt]} \hypertarget{t`s`w\{\string_Ta1}{}
\markboth{\textcolor{darkblue}{\textbf{\ipa{ʈʂwæ˥\textsubscript{a}}}}}{}
\textcolor{teal}{\mytextsc{classifier}} \hspace{4pt} Tone: H\textsubscript{a}.
\textcolor{Sepia}{\selectlanguage{english}Classifier for journeys.} \zh{量词:征途、路程、路途、征程,趟。}  ¶ \textcolor{darkblue}{\textbf{\ipa{ɖɯ˧-ɲi˥ | ɖɯ˧-ʈʂwæ˧ bi˧}}} \textcolor{Sepia}{\selectlanguage{english}to go once a day, to go one time each day} \zh{一天去一趟}  

\lhead{\firstmark}
\rhead{\botmark}

\subsection{\hspace{-0.5cm} {\Large \textcolor{darkblue}{\textbf{\ipa{ʈʂwæ˧tʰo˩}}}}\hspace{0.5cm}[\kern2pt{\textcolor{darkblue}{\textbf{\ipa{ʈʂwæ˧tʰo˧}}}}\kern2pt]} \hypertarget{t`s`w\{\string_Mt\string_ho\string_B1}{}
\markboth{\textcolor{darkblue}{\textbf{\ipa{ʈʂwæ˧tʰo˩}}}}{}
\textcolor{teal}{\mytextsc{noun}} \hspace{4pt} Tone: L\#.
\textcolor{Sepia}{\selectlanguage{english}Brick.} \zh{砖头(汉语借词)。}  Borrowing: Chinese  \zh{砖头}
\textit{See:} \hyperlink{}{\textcolor{darkblue}{\textbf{\ipa{tʰo˩tɕi˧˥}}}} 
\lhead{\firstmark}
\rhead{\botmark}

\subsection{\hspace{-0.5cm} {\Large \textcolor{darkblue}{\textbf{\ipa{ʈʂwæ˧\textasciitilde{}ʈʂwæ˧}}}}\hspace{0.5cm}[\kern2pt{\textcolor{darkblue}{\textbf{\ipa{ʈʂwæ˧ʈʂwæ˧˥}}}}\kern2pt]} \hypertarget{t`s`w\{\string_M~t`s`w\{\string_M1}{}
\markboth{\textcolor{darkblue}{\textbf{\ipa{ʈʂwæ˧\textasciitilde{}ʈʂwæ˧}}}}{}
\textcolor{teal}{\mytextsc{verb}} \hspace{4pt} Tone: M.
\textcolor{Sepia}{\selectlanguage{english}To mix.} \zh{搅拌、使混合。} 
\lhead{\firstmark}
\rhead{\botmark}

\subsection{\hspace{-0.5cm} {\Large \textcolor{darkblue}{\textbf{\ipa{ʈʂwæ˩ho˧ɻ̍˧}}}}\hspace{0.5cm}[\kern2pt{\textcolor{darkblue}{\textbf{\ipa{ʈʂwæ˧ho˧ɻ̍˧}}}}\kern2pt]} \hypertarget{t`s`w\{\string_Bho\string_Mr£`̍\string_M1}{}
\markboth{\textcolor{darkblue}{\textbf{\ipa{ʈʂwæ˩ho˧ɻ̍˧}}}}{}
\textcolor{teal}{\mytextsc{noun}} \hspace{4pt} Tone: LM.
\textcolor{Sepia}{\selectlanguage{english}Drill.} \zh{钻子。}  Borrowing: Chinese  \zh{钻}

\lhead{\firstmark}
\rhead{\botmark}

\subsection{\hspace{-0.5cm} {\Large \textcolor{darkblue}{\textbf{\ipa{ʈʂwæ˩\textasciitilde{}ʈʂwæ˧˥}}}}\hspace{0.5cm}[\kern2pt{\textcolor{darkblue}{\textbf{\ipa{ʈʂwæ˧ʈʂwæ˧}}}}\kern2pt]} \hypertarget{t`s`w\{\string_B~t`s`w\{\string_M\string_T1}{}
\markboth{\textcolor{darkblue}{\textbf{\ipa{ʈʂwæ˩\textasciitilde{}ʈʂwæ˧˥}}}}{}
\textcolor{teal}{\mytextsc{verb}} \hspace{4pt} Tone: MH.
\textcolor{Sepia}{\selectlanguage{english}To hold by the hand, to hold each other's hands.} \zh{手拉手。}  ¶ \textcolor{darkblue}{\textbf{\ipa{le˧-ʈʂwæ˧\textasciitilde{}ʈʂwæ˧-ze˧!}}} \textcolor{Sepia}{\selectlanguage{english}\mytextsc{accomp} \mytextsc{red} \mytextsc{pfv}} \zh{\mytextsc{accomp} \mytextsc{red} \mytextsc{pfv}}  
 ¶ \textcolor{darkblue}{\textbf{\ipa{ʈʂwæ˩\textasciitilde{}ʈʂwæ˧-ɻ̍˥}}} \textcolor{Sepia}{\selectlanguage{english}\mytextsc{red} \mytextsc{inceptive}} \zh{\mytextsc{red} \mytextsc{inceptive}}  

\lhead{\firstmark}
\rhead{\botmark}

\subsection{\hspace{-0.5cm} {\Large \textcolor{darkblue}{\textbf{\ipa{ʈʂwæ˧˥}}} \textsubscript{1}}\hspace{0.5cm}[\kern2pt{\textcolor{darkblue}{\textbf{\ipa{ʈʂwæ˧˥}}}}\kern2pt]} \hypertarget{t`s`w\{\string_M\string_T1}{}
\markboth{\textcolor{darkblue}{\textbf{\ipa{ʈʂwæ˧˥}}} \textsubscript{1}}{}
\textcolor{teal}{\mytextsc{verb}} \hspace{4pt} Tone: MH.
 Borrowing: Chinese  \zh{装?}
\ding{202} \textcolor{Sepia}{\selectlanguage{english}To set up, to install.} \zh{安装。}  ¶ \textcolor{darkblue}{\textbf{\ipa{tjɤ˧hwɑ˧ ʈʂwæ˥}}} \textcolor{Sepia}{\selectlanguage{english}to set up the telephone, to put up a telephone line (in a house that did not have it before)} \zh{安装电话(座机)}  
 ¶ \textcolor{darkblue}{\textbf{\ipa{le˧-ʈʂwæ˧˥ le˧-tse˧-ze˧!}}} \textcolor{Sepia}{\selectlanguage{english}It's installed! / It is now well installed!} \zh{装好了!}  
\ding{203} \textcolor{Sepia}{\selectlanguage{english}To repair, to cure (a tooth).} \zh{补(牙)、修好(坏牙)。}  ¶ \textcolor{darkblue}{\textbf{\ipa{hi˧ ʈʂwæ˩}}} \textcolor{Sepia}{\selectlanguage{english}to cure a tooth} \zh{补牙、修好坏牙}  
 ¶ \textcolor{darkblue}{\textbf{\ipa{hi˧ | le˧-ʈʂwæ˧-ze˥!}}} \textcolor{Sepia}{\selectlanguage{english}The tooth is cured!} \zh{牙补好了!}  
\ding{204} \textcolor{Sepia}{\selectlanguage{english}To tie (a string), to make a knot to tie two pieces of thread together.} \zh{结线。}  ¶ \textcolor{darkblue}{\textbf{\ipa{kʰɯ˩ ʈʂwæ˩˥}}} \textcolor{Sepia}{\selectlanguage{english}to tie pieces of thread together} \zh{结线}  

\lhead{\firstmark}
\rhead{\botmark}

\subsection{\hspace{-0.5cm} {\Large \textcolor{darkblue}{\textbf{\ipa{ʈʂwæ˧˥}}} \textsubscript{2}}\hspace{0.5cm}[\kern2pt{\textcolor{darkblue}{\textbf{\ipa{ʈʂwæ˧˥}}}}\kern2pt]} \hypertarget{t`s`w\{\string_M\string_T2}{}
\markboth{\textcolor{darkblue}{\textbf{\ipa{ʈʂwæ˧˥}}} \textsubscript{2}}{}
\textcolor{teal}{\mytextsc{verb}} \hspace{4pt} Tone: MH.
\textcolor{Sepia}{\selectlanguage{english}To savour, to enjoy, to relish.} \zh{欣赏、品尝(饮食、音乐……)。}  ¶ \textcolor{darkblue}{\textbf{\ipa{no˧ | li˩ ʈʂwæ˧-ɻ̍˥! |}}} \textcolor{Sepia}{\selectlanguage{english}Please enjoy a cup of tea! (A polite invitation)} \zh{请您品一点茶!(礼貌说法)}  
 ¶ \textcolor{darkblue}{\textbf{\ipa{ʐɯ˧ F | ʈʂwæ˧˥! | li˩˥ F | ʈʂwæ˧˥! hɑ˧ F | ʈʂwæ˧˥!}}} \textcolor{Sepia}{\selectlanguage{english}Wine is something to be savoured! Tea is something to be savoured! Food is something to be savoured! (An explanation about the use of the verb.)} \zh{酒,是可以品尝的!茶,是可以品尝的!饭,是可以品尝的!(关于‘品尝’这个动词的说明)}  
 ¶ \textcolor{darkblue}{\textbf{\ipa{hɑ˧ ʈʂwæ˩}}} \textcolor{Sepia}{\selectlanguage{english}to savour food} \zh{品尝食物}  
 ¶ \textcolor{darkblue}{\textbf{\ipa{li˩ ʈʂwæ˧˥}}} \textcolor{Sepia}{\selectlanguage{english}to savour tea} \zh{品茶}  
 ¶ \textcolor{darkblue}{\textbf{\ipa{ʐɯ˧ ʈʂwæ˧˥}}} \textcolor{Sepia}{\selectlanguage{english}to savour wine} \zh{品酒}  
 ¶ \textcolor{darkblue}{\textbf{\ipa{ə˩kʰɯ˩ ʈʂwæ˥}}} \textcolor{Sepia}{\selectlanguage{english}to savour turnip (an ironic but fully acceptable statement)} \zh{尝尝圆根(玩笑话,因为圆根没有什么滋味)}  

\lhead{\firstmark}
\rhead{\botmark}

\subsection{\hspace{-0.5cm} {\Large \textcolor{darkblue}{\textbf{\ipa{ʈʂwɤ˧\textsubscript{a}}}}}\hspace{0.5cm}[\kern2pt{\textcolor{darkblue}{\textbf{\ipa{ʈʂwɤ˥}}}}\kern2pt]} \hypertarget{t`s`w7\string_Ma1}{}
\markboth{\textcolor{darkblue}{\textbf{\ipa{ʈʂwɤ˧\textsubscript{a}}}}}{}
\textcolor{teal}{\mytextsc{classifier}} \hspace{4pt} Tone: M\textsubscript{a}.
\textcolor{Sepia}{\selectlanguage{english}A handful (with one single hand).} \zh{量词:捧。} 
\lhead{\firstmark}
\rhead{\botmark}

\subsection{\hspace{-0.5cm} {\Large \textcolor{darkblue}{\textbf{\ipa{ʈʂwɤ˧\textsubscript{a}}}}}\hspace{0.5cm}[\kern2pt{\textcolor{darkblue}{\textbf{\ipa{ʈʂwɤ˩˥}}}}\kern2pt]} \hypertarget{t`s`w7\string_Ma1}{}
\markboth{\textcolor{darkblue}{\textbf{\ipa{ʈʂwɤ˧\textsubscript{a}}}}}{}
\textcolor{teal}{\mytextsc{verb}} \hspace{4pt} Tone: M\textsubscript{a}.
\textcolor{Sepia}{\selectlanguage{english}To scratch (with claws, e.g. of tiger).} \zh{抓(用爪子抓)。}  ¶ \textcolor{darkblue}{\textbf{\ipa{tso˧\textasciitilde{}tso˧ ʈʂwɤ˩}}} \textcolor{Sepia}{\selectlanguage{english}to scratch objects} \zh{抓东西}  

\lhead{\firstmark}
\rhead{\botmark}

\subsection{\hspace{-0.5cm} {\Large \textcolor{darkblue}{\textbf{\ipa{ʈʂwɤ˧\textasciitilde{}ʈʂwɤ˩}}}}\hspace{0.5cm}[\kern2pt{\textcolor{darkblue}{\textbf{\ipa{ʈʂwɤ˧ʈʂwɤ˧˥}}}}\kern2pt]} \hypertarget{t`s`w7\string_M~t`s`w7\string_B1}{}
\markboth{\textcolor{darkblue}{\textbf{\ipa{ʈʂwɤ˧\textasciitilde{}ʈʂwɤ˩}}}}{}
\textcolor{teal}{\mytextsc{verb}} \hspace{4pt} Tone: M.
\textcolor{Sepia}{\selectlanguage{english}To touch.} \zh{触碰。}  ¶ \textcolor{darkblue}{\textbf{\ipa{ə˧tso˧ mɤ˧-ɲi˩ ʈʂwɤ˧\textasciitilde{}ʈʂwɤ˩!}}} \textcolor{Sepia}{\selectlanguage{english}You really touch all and everything, don't you! (Mildly scolding a baby that crawls around on a table and grabs every object in turn)} \zh{你什么都碰,是吗!(小孩爬在桌子上,试着拿每个东西)}  

\lhead{\firstmark}
\rhead{\botmark}

\subsection{\hspace{-0.5cm} {\Large \textcolor{darkblue}{\textbf{\ipa{ʈʂʰɑ˧lɑ˧}}}}\hspace{0.5cm}[\kern2pt{\textcolor{darkblue}{\textbf{\ipa{ʈʂʰɑ˧lɑ˧}}}}\kern2pt]} \hypertarget{t`s`\string_hA\string_MlA\string_M1}{}
\markboth{\textcolor{darkblue}{\textbf{\ipa{ʈʂʰɑ˧lɑ˧}}}}{}
\textcolor{teal}{\mytextsc{verb}} \hspace{4pt} Tone: M.
\textcolor{Sepia}{\selectlanguage{english}To discuss, to have a talk, to chat.} \zh{商量、交谈、谈天、聊天。}  ¶ \textcolor{darkblue}{\textbf{\ipa{hĩ˧-qɑ˩ ʈʂʰɑ˩lɑ˩}}} \textcolor{Sepia}{\selectlanguage{english}to have a chat with someone} \zh{跟人聊天}  
 ¶ \textcolor{darkblue}{\textbf{\ipa{ɖɯ˧-kʰwɤ˧ ʈʂʰɑ˧lɑ˥}}} \textcolor{Sepia}{\selectlanguage{english}to have a small chat} \zh{聊聊天}  
 ¶ \textcolor{darkblue}{\textbf{\ipa{njɤ˧ | no˧-qɑ˧ ʈʂʰɑ˧lɑ˥}}} \textcolor{Sepia}{\selectlanguage{english}I tell you, I narrate to you} \zh{我给你讲、我跟你聊聊天}  

\lhead{\firstmark}
\rhead{\botmark}

\subsection{\hspace{-0.5cm} {\Large \textcolor{darkblue}{\textbf{\ipa{ʈʂʰɑ˧lɑ˧-mv̩˧lɑ˩}}}}\hspace{0.5cm}[\kern2pt{\textcolor{darkblue}{\textbf{\ipa{xxxx non-correspondance entre le nombre de morphèmes et le nombre de tons de morphèmes}}}}\kern2pt]} \hypertarget{t`s`\string_hA\string_MlA\string_M-mv\string_=\string_MlA\string_B1}{}
\markboth{\textcolor{darkblue}{\textbf{\ipa{ʈʂʰɑ˧lɑ˧-mv̩˧lɑ˩}}}}{}
\textcolor{teal}{\mytextsc{verb}} \hspace{4pt} Tone: M.
\textcolor{Sepia}{\selectlanguage{english}To discuss, to have a talk, to chat.} \zh{商量、交谈、谈天、聊天。}  ¶ \textcolor{darkblue}{\textbf{\ipa{ʈʂʰɑ˧lɑ˧-mv̩˧lɑ˩-ɻ̍˩}}} \textcolor{Sepia}{\selectlanguage{english}to have a chat} \zh{聊聊天}  

\lhead{\firstmark}
\rhead{\botmark}

\subsection{\hspace{-0.5cm} {\Large \textcolor{darkblue}{\textbf{\ipa{ʈʂʰɑ˧nɑ˥}}}}\hspace{0.5cm}[\kern2pt{\textcolor{darkblue}{\textbf{\ipa{ʈʂʰɑ˧nɑ˥}}}}\kern2pt]} \hypertarget{t`s`\string_hA\string_MnA\string_T1}{}
\markboth{\textcolor{darkblue}{\textbf{\ipa{ʈʂʰɑ˧nɑ˥}}}}{}
\textcolor{teal}{\mytextsc{noun}} \hspace{4pt} Tone: H\#.
\textcolor{Sepia}{\selectlanguage{english}The name of a sacred spring, at the foot of a cliff, on the mountain \textcolor{darkblue}{\textbf{\ipa{/qv̩˧ɻ\#˥/}}}.} \zh{一眼山泉的名字。}  ¶ \textcolor{darkblue}{\textbf{\ipa{qv̩˧ɻ̍˧-ʈʂʰɑ˧nɑ˥\#}}} \textcolor{Sepia}{\selectlanguage{english}the full name of the mountain} \zh{山的全名,包括水泉名}  
 ¶ \textcolor{darkblue}{\textbf{\ipa{kɤ˧mv̩˧˥, | æ˧ʂæ˧, | ŋwɤ˧hɑ̃˩, | ʂwæ˧gv̩\#˥, | nɑ˩tsʰi˩˥ | -tɕʰɤ˧pɤ˧mi\#˥, | qv̩˧ɻ̍˧-ʈʂʰɑ˧nɑ˥ |}}} \textcolor{Sepia}{\selectlanguage{english}The six mountains of Yongning that carry a name and have a definite symbolic value. The other mountains do not have comparable symbolic value, and fewer people use specific names for them.} \zh{永宁地区有固定名字的六座山。其它的山,因为没有重要的象征意义,因此没有取名。}  

\lhead{\firstmark}
\rhead{\botmark}

\subsection{\hspace{-0.5cm} {\Large \textcolor{darkblue}{\textbf{\ipa{ʈʂʰæ˥}}}}\hspace{0.5cm}[\kern2pt{\textcolor{darkblue}{\textbf{\ipa{ʈʂʰæ˧˥}}}}\kern2pt]} \hypertarget{t`s`\string_h\{\string_T1}{}
\markboth{\textcolor{darkblue}{\textbf{\ipa{ʈʂʰæ˥}}}}{}
\textcolor{teal}{\mytextsc{verb}} \hspace{4pt} Tone: H.
\textcolor{Sepia}{\selectlanguage{english}To wash (clothes, oneself…).} \zh{洗(洗衣服,洗澡……)。}  ¶ \textcolor{darkblue}{\textbf{\ipa{dʑi˧hṽ˧ ʈʂʰæ˧}}} \textcolor{Sepia}{\selectlanguage{english}to wash clothes} \zh{洗衣服}  
 ¶ \textcolor{darkblue}{\textbf{\ipa{bɑ˩lɑ˩ ʈʂʰæ˩˥}}} \textcolor{Sepia}{\selectlanguage{english}to wash shirts} \zh{洗上衣}  
 ¶ \textcolor{darkblue}{\textbf{\ipa{ɬi˧qʰwɤ˩ ʈʂʰæ˩}}} \textcolor{Sepia}{\selectlanguage{english}to wash trousers} \zh{洗裤子}  
 ¶ \textcolor{darkblue}{\textbf{\ipa{gv̩˧mi˧ ʈʂʰæ˧}}} \textcolor{Sepia}{\selectlanguage{english}to wash oneself, to take a bath/shower} \zh{洗澡}  
 ¶ \textcolor{darkblue}{\textbf{\ipa{gv̩˧mi˧ ʈʂʰæ˧\textasciitilde{}ʈʂʰæ˧}}} \textcolor{Sepia}{\selectlanguage{english}to wash oneself a bit, to do a quick clean-up} \zh{洗一下身体}  
 ¶ \textcolor{darkblue}{\textbf{\ipa{hɑ˧ ʈʂʰæ˧}}} \textcolor{Sepia}{\selectlanguage{english}to rinse cereals (before cooking)} \zh{淘洗粮食}  
 ¶ \textcolor{darkblue}{\textbf{\ipa{ɕi˧ʈʂʰwæ˧ ʈʂʰæ˧(-ze˩)}}} \textcolor{Sepia}{\selectlanguage{english}to rinse rice (before cooking)} \zh{淘米}  

\lhead{\firstmark}
\rhead{\botmark}

\subsection{\hspace{-0.5cm} {\Large \textcolor{darkblue}{\textbf{\ipa{ʈʂʰæ˧ɣɯ\#˥}}}}\hspace{0.5cm}[\kern2pt{\textcolor{darkblue}{\textbf{\ipa{ʈʂʰæ˧ɣɯ˧}}}}\kern2pt]} \hypertarget{t`s`\string_h\{\string_MGM\#\string_T1}{}
\markboth{\textcolor{darkblue}{\textbf{\ipa{ʈʂʰæ˧ɣɯ\#˥}}}}{}
\textcolor{teal}{\mytextsc{noun}} \hspace{4pt} Tone: \#H.
\textcolor{Sepia}{\selectlanguage{english}Medicine.} \zh{药。}  ¶ \textcolor{darkblue}{\textbf{\ipa{ʈʂʰæ˧ɣɯ˧ ʈʰɯ˧˥}}} \textcolor{Sepia}{\selectlanguage{english}to take a medicine; literally “to drink a medicine”} \zh{吃药(直译:“喝药”)}  
 ¶ \textcolor{darkblue}{\textbf{\ipa{ʈʂʰæ˧ɣɯ˧ lɑ˩}}} \textcolor{Sepia}{\selectlanguage{english}to spread pesticides (in an orchard, a vegetable garden or a field)} \zh{打农药}  

\lhead{\firstmark}
\rhead{\botmark}

\subsection{\hspace{-0.5cm} {\Large \textcolor{darkblue}{\textbf{\ipa{ʈʂʰæ˧ɣɯ˧-ki˩-hĩ˩-hĩ˩}}}}\hspace{0.5cm}[\kern2pt{\textcolor{darkblue}{\textbf{\ipa{xxxx non-correspondance entre le nombre de morphèmes et le nombre de tons de morphèmes}}}}\kern2pt]} \hypertarget{t`s`\string_h\{\string_MGM\string_M-ki\string_B-hi\string_~\string_B-hi\string_~\string_B1}{}
\markboth{\textcolor{darkblue}{\textbf{\ipa{ʈʂʰæ˧ɣɯ˧-ki˩-hĩ˩-hĩ˩}}}}{}
\textcolor{teal}{\mytextsc{noun}} \hspace{4pt} Tone: \mytextsc{L}.
\textcolor{Sepia}{\selectlanguage{english}Doctor; literally: “person who gives medicines”.} \zh{医生。}  \zh{量词}: \textcolor{darkblue}{\textbf{\ipa{v̩˧}}}  \mytextsc{clf}: \textcolor{darkblue}{\textbf{\ipa{v̩˧}}} 
\lhead{\firstmark}
\rhead{\botmark}

\subsection{\hspace{-0.5cm} {\Large \textcolor{darkblue}{\textbf{\ipa{ʈʂʰæ˧mi˥\$}}}}\hspace{0.5cm}[\kern2pt{\textcolor{darkblue}{\textbf{\ipa{ʈʂʰæ˧mi˥}}}}\kern2pt]} \hypertarget{t`s`\string_h\{\string_Mmi\string_T\$1}{}
\markboth{\textcolor{darkblue}{\textbf{\ipa{ʈʂʰæ˧mi˥\$}}}}{}
\textcolor{teal}{\mytextsc{noun}} \hspace{4pt} Tone: H\$.
\textcolor{Sepia}{\selectlanguage{english}Doe, hind.} \zh{母马鹿。}  \zh{量词}: \textcolor{darkblue}{\textbf{\ipa{pʰo˧˥}}}  \mytextsc{clf}: \textcolor{darkblue}{\textbf{\ipa{pʰo˧˥}}} 
\lhead{\firstmark}
\rhead{\botmark}

\subsection{\hspace{-0.5cm} {\Large \textcolor{darkblue}{\textbf{\ipa{ʈʂʰæ˧nɑ˥}}}}\hspace{0.5cm}[\kern2pt{\textcolor{darkblue}{\textbf{\ipa{ʈʂʰæ˧nɑ˥}}}}\kern2pt]} \hypertarget{t`s`\string_h\{\string_MnA\string_T1}{}
\markboth{\textcolor{darkblue}{\textbf{\ipa{ʈʂʰæ˧nɑ˥}}}}{}
\textcolor{teal}{\mytextsc{noun}} \hspace{4pt} Tone: H\#.
\textcolor{Sepia}{\selectlanguage{english}Black stag: a legendary species, which only spirits are able to hunt down.} \zh{黑鹿。}  \zh{量词}: \textcolor{darkblue}{\textbf{\ipa{pʰo˧˥}}}  \mytextsc{clf}: \textcolor{darkblue}{\textbf{\ipa{pʰo˧˥}}} 
\lhead{\firstmark}
\rhead{\botmark}

\subsection{\hspace{-0.5cm} {\Large \textcolor{darkblue}{\textbf{\ipa{ʈʂʰæ˧pʰv̩\#˥}}}}\hspace{0.5cm}[\kern2pt{\textcolor{darkblue}{\textbf{\ipa{ʈʂʰæ˧pʰv̩˧}}}}\kern2pt]} \hypertarget{t`s`\string_h\{\string_Mp\string_hv\string_=\#\string_T1}{}
\markboth{\textcolor{darkblue}{\textbf{\ipa{ʈʂʰæ˧pʰv̩\#˥}}}}{}
\textcolor{teal}{\mytextsc{noun}} \hspace{4pt} Tone: \#H.
\textcolor{Sepia}{\selectlanguage{english}Male deer.} \zh{公马鹿。}  \zh{量词}: \textcolor{darkblue}{\textbf{\ipa{pʰo˧˥}}}  \mytextsc{clf}: \textcolor{darkblue}{\textbf{\ipa{pʰo˧˥}}} 
\lhead{\firstmark}
\rhead{\botmark}

\subsection{\hspace{-0.5cm} {\Large \textcolor{darkblue}{\textbf{\ipa{ʈʂʰæ˧qʰv̩˥\$}}}}\hspace{0.5cm}[\kern2pt{\textcolor{darkblue}{\textbf{\ipa{ʈʂʰæ˧qʰv̩˥}}}}\kern2pt]} \hypertarget{t`s`\string_h\{\string_Mq\string_hv\string_=\string_T\$1}{}
\markboth{\textcolor{darkblue}{\textbf{\ipa{ʈʂʰæ˧qʰv̩˥\$}}}}{}
\textcolor{teal}{\mytextsc{noun}} \hspace{4pt} Tone: H\$.
\textcolor{Sepia}{\selectlanguage{english}Antlers; pilose antler (of young stags).} \zh{鹿角,鹿茸。}  \zh{量词}: \textcolor{darkblue}{\textbf{\ipa{ɭɯ˧}}}  \mytextsc{clf}: \textcolor{darkblue}{\textbf{\ipa{ɭɯ˧}}} 
\lhead{\firstmark}
\rhead{\botmark}

\subsection{\hspace{-0.5cm} {\Large \textcolor{darkblue}{\textbf{\ipa{ʈʂʰæ˧\textasciitilde{}ʈʂʰæ˧}}}}\hspace{0.5cm}[\kern2pt{\textcolor{darkblue}{\textbf{\ipa{ʈʂʰæ˧ʈʂʰæ˧}}}}\kern2pt]} \hypertarget{t`s`\string_h\{\string_M~t`s`\string_h\{\string_M1}{}
\markboth{\textcolor{darkblue}{\textbf{\ipa{ʈʂʰæ˧\textasciitilde{}ʈʂʰæ˧}}}}{}
\textcolor{teal}{\mytextsc{adjective}} \hspace{4pt} Tone: M.
\textcolor{Sepia}{\selectlanguage{english}Solid, of good quality.} \zh{结实、质量好,(东西)耐用,(人)可靠。} 
\lhead{\firstmark}
\rhead{\botmark}

\subsection{\hspace{-0.5cm} {\Large \textcolor{darkblue}{\textbf{\ipa{ʈʂʰæ˧zo\#˥}}}}\hspace{0.5cm}[\kern2pt{\textcolor{darkblue}{\textbf{\ipa{ʈʂʰæ˧zo˧}}}}\kern2pt]} \hypertarget{t`s`\string_h\{\string_Mzo\#\string_T1}{}
\markboth{\textcolor{darkblue}{\textbf{\ipa{ʈʂʰæ˧zo\#˥}}}}{}
\textcolor{teal}{\mytextsc{noun}} \hspace{4pt} Tone: \#H.
\textcolor{Sepia}{\selectlanguage{english}Baby deer.} \zh{小鹿。}  \zh{量词}: \textcolor{darkblue}{\textbf{\ipa{ɭɯ˧}}}  \mytextsc{clf}: \textcolor{darkblue}{\textbf{\ipa{ɭɯ˧}}} 
\lhead{\firstmark}
\rhead{\botmark}

\subsection{\hspace{-0.5cm} {\Large \textcolor{darkblue}{\textbf{\ipa{ʈʂʰæ˧˥}}} \textsubscript{1}}\hspace{0.5cm}[\kern2pt{\textcolor{darkblue}{\textbf{\ipa{ʈʂʰæ˧˥}}}}\kern2pt]} \hypertarget{t`s`\string_h\{\string_M\string_T1}{}
\markboth{\textcolor{darkblue}{\textbf{\ipa{ʈʂʰæ˧˥}}} \textsubscript{1}}{}
\textcolor{teal}{\mytextsc{noun}} \hspace{4pt} Tone: MH.
\textcolor{Sepia}{\selectlanguage{english}Deer, red deer, \textit{Cervus elaphus kansuensis}.} \zh{马鹿。}  \zh{量词}: \textcolor{darkblue}{\textbf{\ipa{pʰo˧˥}}}  \mytextsc{clf}: \textcolor{darkblue}{\textbf{\ipa{pʰo˧˥}}} 
\lhead{\firstmark}
\rhead{\botmark}

\subsection{\hspace{-0.5cm} {\Large \textcolor{darkblue}{\textbf{\ipa{ʈʂʰæ˧˥}}} \textsubscript{2}}\hspace{0.5cm}[\kern2pt{\textcolor{darkblue}{\textbf{\ipa{ʈʂʰæ˧˥}}}}\kern2pt]} \hypertarget{t`s`\string_h\{\string_M\string_T2}{}
\markboth{\textcolor{darkblue}{\textbf{\ipa{ʈʂʰæ˧˥}}} \textsubscript{2}}{}
\textcolor{teal}{\mytextsc{classifier}} \hspace{4pt} Tone: MH.
\textcolor{Sepia}{\selectlanguage{english}Classifier for generations.} \zh{量词:代、世、辈、世代。} 
\lhead{\firstmark}
\rhead{\botmark}

\subsection{\hspace{-0.5cm} {\Large \textcolor{darkblue}{\textbf{\ipa{ʈʂʰe˧\textsubscript{b}}}}}\hspace{0.5cm}[\kern2pt{\textcolor{darkblue}{\textbf{\ipa{ʈʂʰe˥}}}}\kern2pt]} \hypertarget{t`s`\string_he\string_Mb1}{}
\markboth{\textcolor{darkblue}{\textbf{\ipa{ʈʂʰe˧\textsubscript{b}}}}}{}
\textcolor{teal}{\mytextsc{verb}} \hspace{4pt} Tone: M\textsubscript{b}.
\textcolor{Sepia}{\selectlanguage{english}To stretch (one's hand...).} \zh{伸(伸手)。}  ¶ \textcolor{darkblue}{\textbf{\ipa{le˧-ʈʂʰe˧-ze˧}}} \textcolor{Sepia}{\selectlanguage{english}\mytextsc{accomp} \string_ \mytextsc{pfv}} \zh{\mytextsc{accomp} \string_ \mytextsc{pfv}}  
 ¶ \textcolor{darkblue}{\textbf{\ipa{mv̩˩tɕo˧ ʈʂʰe˧}}} \textcolor{Sepia}{\selectlanguage{english}to stretch down} \zh{向下伸展}  
 ¶ \textcolor{darkblue}{\textbf{\ipa{lo˩qʰwɤ˧ | ə˩pʰo˩ ʈʂʰe˩˥}}} \textcolor{Sepia}{\selectlanguage{english}to strech one's hand outside (e.g. out the window)} \zh{手伸到外边}  
 ¶ \textcolor{darkblue}{\textbf{\ipa{tso˧\textasciitilde{}tso˧ ʈʂʰe˧}}} \textcolor{Sepia}{\selectlanguage{english}to extend something, to stick out something (e.g. to extend a cane out the window of a car)} \zh{伸出一个东西,如:从车窗里伸出一个棍子}  

\lhead{\firstmark}
\rhead{\botmark}

\subsection{\hspace{-0.5cm} {\Large \textcolor{darkblue}{\textbf{\ipa{ʈʂʰe˧\textasciitilde{}ʈʂʰe˧}}}}\hspace{0.5cm}[\kern2pt{\textcolor{darkblue}{\textbf{\ipa{ʈʂʰe˧ʈʂʰe˧}}}}\kern2pt]} \hypertarget{t`s`\string_he\string_M~t`s`\string_he\string_M1}{}
\markboth{\textcolor{darkblue}{\textbf{\ipa{ʈʂʰe˧\textasciitilde{}ʈʂʰe˧}}}}{}
\textcolor{teal}{\mytextsc{classifier}} \hspace{4pt} Tone: M.
\textit{From:} \textbf{ʈʂʰe˧b} \textcolor{Sepia}{\selectlanguage{english}Classifiers for walls, i.e. the width of a room: for instance, a cupboard can be described as extending over an entire wall, i.e. occupying the entire width of a room.} \zh{量词:一面(墙)。} 
\lhead{\firstmark}
\rhead{\botmark}

\subsection{\hspace{-0.5cm} {\Large \textcolor{darkblue}{\textbf{\ipa{ʈʂʰe˩ko˧}}}}\hspace{0.5cm}[\kern2pt{\textcolor{darkblue}{\textbf{\ipa{ʈʂʰe˩ko˥}}}}\kern2pt]} \hypertarget{t`s`\string_he\string_Bko\string_M1}{}
\markboth{\textcolor{darkblue}{\textbf{\ipa{ʈʂʰe˩ko˧}}}}{}
\textcolor{teal}{\mytextsc{verb}} \hspace{4pt} Tone: LM.
\textcolor{Sepia}{\selectlanguage{english}To succeed.} \zh{成功(汉语借词)。}  Borrowing: Chinese  \zh{成功}

\lhead{\firstmark}
\rhead{\botmark}

\subsection{\hspace{-0.5cm} {\Large \textcolor{darkblue}{\textbf{\ipa{ʈʂʰɤ˧tsɯ˧}}}}\hspace{0.5cm}[\kern2pt{\textcolor{darkblue}{\textbf{\ipa{ʈʂʰɤ˧tsɯ˧}}}}\kern2pt]} \hypertarget{t`s`\string_h7\string_MtsM\string_M1}{}
\markboth{\textcolor{darkblue}{\textbf{\ipa{ʈʂʰɤ˧tsɯ˧}}}}{}
\textcolor{teal}{\mytextsc{noun}} \hspace{4pt} Tone: M.
\textcolor{Sepia}{\selectlanguage{english}Car, bus.} \zh{车子(汉语借词)。}  Borrowing: Chinese  \zh{车子}

\lhead{\firstmark}
\rhead{\botmark}

\subsection{\hspace{-0.5cm} {\Large \textcolor{darkblue}{\textbf{\ipa{ʈʂʰɤ˧zo˥-ʈʂʰɤ˩mv̩˩}}}}\hspace{0.5cm}[\kern2pt{\textcolor{darkblue}{\textbf{\ipa{ʈʂʰɤ˧zo˥ʈʂʰɤ˩mv̩˩}}}}\kern2pt]} \hypertarget{t`s`\string_h7\string_Mzo\string_T-t`s`\string_h7\string_Bmv\string_=\string_B1}{}
\markboth{\textcolor{darkblue}{\textbf{\ipa{ʈʂʰɤ˧zo˥-ʈʂʰɤ˩mv̩˩}}}}{}
\textcolor{teal}{\mytextsc{noun}} \hspace{4pt} Tone: H\#-L.
\textcolor{Sepia}{\selectlanguage{english}Love child.} \zh{私生子:没有名分的孩子、不明来路。}  ¶ \textcolor{darkblue}{\textbf{\ipa{ə˧dɑ˥ | ɲi˩-ɲi˥ | mɤ˧-sɯ˥ | ʈʂʰɯ˧-v̩˧, | ʈʂʰɤ˧zo˥-ʈʂʰɤ˩mv̩˩ mv̩˩ʈʂæ˩.}}} \textcolor{Sepia}{\selectlanguage{english}Someone who does not know who his father is, is called a “love child”.} \zh{一个人不知道他父亲是谁,就称作“私生子”。}  
 ¶ \textcolor{darkblue}{\textbf{\ipa{ə˧ʝi˧-ʂɯ˥ʝi˩, | ʈʂʰɤ˧zo˥-ʈʂʰɤ˩mv̩˩ ʐɤ˩-hĩ˩-lɑ˩ ɲi˩!}}} \textcolor{Sepia}{\selectlanguage{english}In the past, one used to bring up love children, and that was that! / In the past, one used to bring up love children without making any fuss!} \zh{过去,大家会公开把“私生子”养大,不会大惊小怪的!}  

\lhead{\firstmark}
\rhead{\botmark}

\subsection{\hspace{-0.5cm} {\Large \textcolor{darkblue}{\textbf{\ipa{ʈʂʰɤ˩\textsubscript{a}}}}}\hspace{0.5cm}[\kern2pt{\textcolor{darkblue}{\textbf{\ipa{ʈʂʰɤ˥}}}}\kern2pt]} \hypertarget{t`s`\string_h7\string_Ba1}{}
\markboth{\textcolor{darkblue}{\textbf{\ipa{ʈʂʰɤ˩\textsubscript{a}}}}}{}
\textcolor{teal}{\mytextsc{verb}} \hspace{4pt} Tone: L\textsubscript{a}.
\textcolor{Sepia}{\selectlanguage{english}To share (several people share something among themselves; someone shares out something).} \zh{分。}  ¶ \textcolor{darkblue}{\textbf{\ipa{ɖɯ˧-v̩˧ ɖɯ˧-kʰwɤ˥ | le˧-ʈʂʰɤ˧\textasciitilde{}ʈʂʰɤ˥}}} \textcolor{Sepia}{\selectlanguage{english}to share: one piece for each person} \zh{平分}  

\lhead{\firstmark}
\rhead{\botmark}

\subsection{\hspace{-0.5cm} {\Large \textcolor{darkblue}{\textbf{\ipa{ʈʂʰɤ˩ho˥}}}}\hspace{0.5cm}[\kern2pt{\textcolor{darkblue}{\textbf{\ipa{ʈʂʰɤ˩ho˥}}}}\kern2pt]} \hypertarget{t`s`\string_h7\string_Bho\string_T1}{}
\markboth{\textcolor{darkblue}{\textbf{\ipa{ʈʂʰɤ˩ho˥}}}}{}
\textcolor{teal}{\mytextsc{noun}} \hspace{4pt} Tone: LH.
\textcolor{Sepia}{\selectlanguage{english}Kettle.} \zh{水壶(汉语借词:茶壶)。}  Borrowing: \zh{茶壶}
 \zh{量词}: \textcolor{darkblue}{\textbf{\ipa{ɭɯ˧}}}  \mytextsc{clf}: \textcolor{darkblue}{\textbf{\ipa{ɭɯ˧}}} 
\lhead{\firstmark}
\rhead{\botmark}

\subsection{\hspace{-0.5cm} {\Large \textcolor{darkblue}{\textbf{\ipa{ʈʂʰɤ˩kɤ˧}}}}\hspace{0.5cm}[\kern2pt{\textcolor{darkblue}{\textbf{\ipa{ʈʂʰɤ˩kɤ˥}}}}\kern2pt]} \hypertarget{t`s`\string_h7\string_Bk7\string_M1}{}
\markboth{\textcolor{darkblue}{\textbf{\ipa{ʈʂʰɤ˩kɤ˧}}}}{}
\textcolor{teal}{\mytextsc{noun}} \hspace{4pt} Tone: LM.
\textcolor{Sepia}{\selectlanguage{english}Goblet.} \zh{缸子,杯子。}  \zh{量词}: \textcolor{darkblue}{\textbf{\ipa{ɭɯ˧}}}  \mytextsc{clf}: \textcolor{darkblue}{\textbf{\ipa{ɭɯ˧}}} 
\lhead{\firstmark}
\rhead{\botmark}

\subsection{\hspace{-0.5cm} {\Large \textcolor{darkblue}{\textbf{\ipa{ʈʂʰɤ˩qo˧}}}}\hspace{0.5cm}[\kern2pt{\textcolor{darkblue}{\textbf{\ipa{ʈʂʰɤ˩qo˥}}}}\kern2pt]} \hypertarget{t`s`\string_h7\string_Bqo\string_M1}{}
\markboth{\textcolor{darkblue}{\textbf{\ipa{ʈʂʰɤ˩qo˧}}}}{}
\textcolor{teal}{\mytextsc{noun}} \hspace{4pt} Tone: LM.
\textcolor{Sepia}{\selectlanguage{english}Attention, interest, care.} \zh{关注、关心。}  ¶ \textcolor{darkblue}{\textbf{\ipa{ʈʂʰɤ˩qo˧ kʰɯ˧˥}}} \textcolor{Sepia}{\selectlanguage{english}to pay attention to, to care for} \zh{关心、关注}  
 ¶ \textcolor{darkblue}{\textbf{\ipa{ʈʂʰɤ˩qo˧ | ɖwæ˧˥ | tʰi˧-kʰɯ˧˥}}} \textcolor{Sepia}{\selectlanguage{english}to pay great attention to, to care greatly for (e.g. a grandmother paying great attention to a little child's feeding)} \zh{很关心、很关注}  
 ¶ \textcolor{darkblue}{\textbf{\ipa{ʈʂʰɤ˩qo˧ | mɤ˧-kʰɯ˧˥}}} \textcolor{Sepia}{\selectlanguage{english}to pay little attention to, not to care for} \zh{不关心、不关注}  
 \zh{量词}: \textcolor{darkblue}{\textbf{\ipa{kʰwɤ˥}}}  \mytextsc{clf}: \textcolor{darkblue}{\textbf{\ipa{kʰwɤ˥}}} 
\lhead{\firstmark}
\rhead{\botmark}

\subsection{\hspace{-0.5cm} {\Large \textcolor{darkblue}{\textbf{\ipa{ʈʂʰɤ˩tɕʰɯ˩}}}}\hspace{0.5cm}[\kern2pt{\textcolor{darkblue}{\textbf{\ipa{ʈʂʰɤ˩tɕʰɯ˩˥}}}}\kern2pt]} \hypertarget{t`s`\string_h7\string_Bts£\string_hM\string_B1}{}
\markboth{\textcolor{darkblue}{\textbf{\ipa{ʈʂʰɤ˩tɕʰɯ˩}}}}{}
\textcolor{teal}{\mytextsc{adjective}} \hspace{4pt} Tone: L.
\textcolor{Sepia}{\selectlanguage{english}Admirable, with high qualities.} \zh{利害,值得崇拜。} 
\lhead{\firstmark}
\rhead{\botmark}

\subsection{\hspace{-0.5cm} {\Large \textcolor{darkblue}{\textbf{\ipa{ʈʂʰɤ˩\textasciitilde{}ʈʂʰɤ˧˥}}}}\hspace{0.5cm}[\kern2pt{\textcolor{darkblue}{\textbf{\ipa{ʈʂʰɤ˧ʈʂʰɤ˧˥}}}}\kern2pt]} \hypertarget{t`s`\string_h7\string_B~t`s`\string_h7\string_M\string_T1}{}
\markboth{\textcolor{darkblue}{\textbf{\ipa{ʈʂʰɤ˩\textasciitilde{}ʈʂʰɤ˧˥}}}}{}
\textcolor{teal}{\mytextsc{verb}} \hspace{4pt} Tone: MH.
\textcolor{Sepia}{\selectlanguage{english}To feel, to touch, to stroke.} \zh{抚摸。}  ¶ \textcolor{darkblue}{\textbf{\ipa{ʈʂʰɤ˩ʈʂʰɤ˧ mɤ˥-tʰɑ˩!}}} \textcolor{Sepia}{\selectlanguage{english}One must not touch!} \zh{禁止触碰!}  
 ¶ \textcolor{darkblue}{\textbf{\ipa{tʰɑ˧-ʈʂʰɤ˩ʈʂʰɤ˩!}}} \textcolor{Sepia}{\selectlanguage{english}Do not touch!} \zh{别碰!}  
 ¶ \textcolor{darkblue}{\textbf{\ipa{tso˧\textasciitilde{}tso˧ ʈʂʰɤ˥ʈʂʰɤ˩}}} \textcolor{Sepia}{\selectlanguage{english}to touch something} \zh{抚摸东西}  

\lhead{\firstmark}
\rhead{\botmark}

\subsection{\hspace{-0.5cm} {\Large \textcolor{darkblue}{\textbf{\ipa{ʈʂʰo˥}}}}\hspace{0.5cm}[\kern2pt{\textcolor{darkblue}{\textbf{\ipa{ʈʂʰo˥}}}}\kern2pt]} \hypertarget{t`s`\string_ho\string_T1}{}
\markboth{\textcolor{darkblue}{\textbf{\ipa{ʈʂʰo˥}}}}{}
\textcolor{teal}{\mytextsc{verb}} \hspace{4pt} Tone: H.
\textcolor{Sepia}{\selectlanguage{english}To pray (to a god): to recite prayers, to chant prayers.} \zh{拜(神)。}  ¶ \textcolor{darkblue}{\textbf{\ipa{ʈʂʰo˧do˩ ʈʂʰo˩}}} \textcolor{Sepia}{\selectlanguage{english}to pray to the spirit of the home} \zh{祭祀祖先}  
 ¶ \textcolor{darkblue}{\textbf{\ipa{hĩ˧-mo˥, | zo˩qo˧ ʂɯ˧, | zo˩qo˧-ɳɯ˧ ʈʂʰo˧-zo˧!}}} \textcolor{Sepia}{\selectlanguage{english}Deceased members of the family are honoured at the place where they passed away!} \zh{要在家人去世地点进行祭拜!}  

\lhead{\firstmark}
\rhead{\botmark}

\subsection{\hspace{-0.5cm} {\Large \textcolor{darkblue}{\textbf{\ipa{ʈʂʰo˧\textsubscript{b}}}}}\hspace{0.5cm}[\kern2pt{\textcolor{darkblue}{\textbf{\ipa{ʈʂʰo˥}}}}\kern2pt]} \hypertarget{t`s`\string_ho\string_Mb1}{}
\markboth{\textcolor{darkblue}{\textbf{\ipa{ʈʂʰo˧\textsubscript{b}}}}}{}
\textcolor{teal}{\mytextsc{verb}} \hspace{4pt} Tone: M\textsubscript{b}.
\textcolor{Sepia}{\selectlanguage{english}To read aloud.} \zh{朗读。}  ¶ \textcolor{darkblue}{\textbf{\ipa{le˧-ʈʂʰo˧-ze˧}}} \textcolor{Sepia}{\selectlanguage{english}\mytextsc{accomp} \string_ \mytextsc{pfv}} \zh{朗读了}  
 ¶ \textcolor{darkblue}{\textbf{\ipa{le˧-ʈʂʰo˧-le˧-se˩}}} \textcolor{Sepia}{\selectlanguage{english}(I) have finished reading aloud} \zh{朗读完了。}  
 ¶ \textcolor{darkblue}{\textbf{\ipa{tʰæ˧ɻæ˩ ʈʂʰo˩}}} \textcolor{Sepia}{\selectlanguage{english}to read a book aloud} \zh{朗读一本书}  
 ¶ \textcolor{darkblue}{\textbf{\ipa{ʈʂʰo˧\textasciitilde{}ʈʂʰo˧}}} \textcolor{Sepia}{\selectlanguage{english}\mytextsc{red}} \zh{\mytextsc{重叠}}  

\lhead{\firstmark}
\rhead{\botmark}

\subsection{\hspace{-0.5cm} {\Large \textcolor{darkblue}{\textbf{\ipa{ʈʂʰo˧bɤ\#˥}}}}\hspace{0.5cm}[\kern2pt{\textcolor{darkblue}{\textbf{\ipa{ʈʂʰo˩bɤ˩˥}}}}\kern2pt]} \hypertarget{t`s`\string_ho\string_Mb7\#\string_T1}{}
\markboth{\textcolor{darkblue}{\textbf{\ipa{ʈʂʰo˧bɤ\#˥}}}}{}
\textcolor{teal}{\mytextsc{noun}} \hspace{4pt} Tone: \#H.
\textcolor{Sepia}{\selectlanguage{english}Masculine clothing worn on special occasions.} \zh{男上衣。}  \zh{量词}: \textcolor{darkblue}{\textbf{\ipa{ɭɯ˧˥}}}  \mytextsc{clf}: \textcolor{darkblue}{\textbf{\ipa{ɭɯ˧˥}}} 
\lhead{\firstmark}
\rhead{\botmark}

\subsection{\hspace{-0.5cm} {\Large \textcolor{darkblue}{\textbf{\ipa{ʈʂʰo˧bv̩˩}}}}\hspace{0.5cm}[\kern2pt{\textcolor{darkblue}{\textbf{\ipa{ʈʂʰo˧bv̩˧}}}}\kern2pt]} \hypertarget{t`s`\string_ho\string_Mbv\string_=\string_B1}{}
\markboth{\textcolor{darkblue}{\textbf{\ipa{ʈʂʰo˧bv̩˩}}}}{}
\textcolor{teal}{\mytextsc{noun}} \hspace{4pt} Tone: L\#.
\textcolor{Sepia}{\selectlanguage{english}Calamus, sweet flag, bitterroot, \textit{Acorus calamus} (a tall wetland plant).} \zh{菖蒲。}  \zh{量词}: \textcolor{darkblue}{\textbf{\ipa{dzi˩}}}  \mytextsc{clf}: \textcolor{darkblue}{\textbf{\ipa{dzi˩}}} 
\lhead{\firstmark}
\rhead{\botmark}

\subsection{\hspace{-0.5cm} {\Large \textcolor{darkblue}{\textbf{\ipa{ʈʂʰo˧do˩}}}}\hspace{0.5cm}[\kern2pt{\textcolor{darkblue}{\textbf{\ipa{ʈʂʰo˧do˧}}}}\kern2pt]} \hypertarget{t`s`\string_ho\string_Mdo\string_B1}{}
\markboth{\textcolor{darkblue}{\textbf{\ipa{ʈʂʰo˧do˩}}}}{}
\textcolor{teal}{\mytextsc{noun}} \hspace{4pt} Tone: L\#.
\textcolor{Sepia}{\selectlanguage{english}Small eminence next to the hearth, symbolising the ancestors, on top of which some food is offered at the beginning of each meal.} \zh{火塘上面祖先灵位。}  \zh{量词}: \textcolor{darkblue}{\textbf{\ipa{ɭɯ˧}}}  \mytextsc{clf}: \textcolor{darkblue}{\textbf{\ipa{ɭɯ˧}}} 
\lhead{\firstmark}
\rhead{\botmark}

\subsection{\hspace{-0.5cm} {\Large \textcolor{darkblue}{\textbf{\ipa{ʈʂʰo˧lo\#˥}}}}\hspace{0.5cm}[\kern2pt{\textcolor{darkblue}{\textbf{\ipa{ʈʂʰo˧lo˧}}}}\kern2pt]} \hypertarget{t`s`\string_ho\string_Mlo\#\string_T1}{}
\markboth{\textcolor{darkblue}{\textbf{\ipa{ʈʂʰo˧lo\#˥}}}}{}
\textcolor{teal}{\mytextsc{noun}} \hspace{4pt} Tone: \#H.
\textcolor{Sepia}{\selectlanguage{english}Frying pan (large, with flat bottom).} \zh{平底大锅(直径大概半米),用来煎洋芋饼等等。}  \zh{量词}: \textcolor{darkblue}{\textbf{\ipa{ɭɯ˧}}}  \mytextsc{clf}: \textcolor{darkblue}{\textbf{\ipa{ɭɯ˧}}} 
\lhead{\firstmark}
\rhead{\botmark}

\subsection{\hspace{-0.5cm} {\Large \textcolor{darkblue}{\textbf{\ipa{ʈʂʰɻ̍˧}}}}\hspace{0.5cm}[\kern2pt{\textcolor{darkblue}{\textbf{\ipa{ʈʂʰɻ̍˥}}}}\kern2pt]} \hypertarget{t`s`\string_hr£`̍\string_M1}{}
\markboth{\textcolor{darkblue}{\textbf{\ipa{ʈʂʰɻ̍˧}}}}{}
\textcolor{teal}{\mytextsc{noun}} \hspace{4pt} Tone: M.
\textcolor{Sepia}{\selectlanguage{english}Ploughshare.} \zh{铧头,犁铧。}  ¶ \textcolor{darkblue}{\textbf{\ipa{ʈʂʰɻ̍˧ ʈʂʰɯ˧-ɭɯ˧}}} \textcolor{Sepia}{\selectlanguage{english}\mytextsc{n}+\mytextsc{dem}+\mytextsc{clf}} \zh{这把铧头}  
 \zh{量词}: \textcolor{darkblue}{\textbf{\ipa{ɭɯ˧}}}  \mytextsc{clf}: \textcolor{darkblue}{\textbf{\ipa{ɭɯ˧}}} 
\lhead{\firstmark}
\rhead{\botmark}

\subsection{\hspace{-0.5cm} {\Large \textcolor{darkblue}{\textbf{\ipa{ʈʂʰɻ̍˧˥}}} \textsubscript{1}}\hspace{0.5cm}[\kern2pt{\textcolor{darkblue}{\textbf{\ipa{ʈʂʰɻ̍˧˥}}}}\kern2pt]} \hypertarget{t`s`\string_hr£`̍\string_M\string_T1}{}
\markboth{\textcolor{darkblue}{\textbf{\ipa{ʈʂʰɻ̍˧˥}}} \textsubscript{1}}{}
\textcolor{teal}{\mytextsc{verb}} \hspace{4pt} Tone: MH.
\textcolor{Sepia}{\selectlanguage{english}To grasp (e.g. a sword hilt).} \zh{握 (握刀把)。}  ¶ \textcolor{darkblue}{\textbf{\ipa{sɯ˩tʰi˩˥ | (ɖɯ˧)-nɑ˧ | tʰi˧-ʈʂʰɻ̍˧˥ (+dʑo˩)}}} \textcolor{Sepia}{\selectlanguage{english}to grasp a knife} \zh{手里握刀}  
 ¶ \textcolor{darkblue}{\textbf{\ipa{ʈʂʰɻ̍˧ mɤ˧-bi˧!}}} \textcolor{Sepia}{\selectlanguage{english}I won't grasp (this knife, ...)} \zh{我不要拿(刀)!}  
 ¶ \textcolor{darkblue}{\textbf{\ipa{lo˩qʰwɤ˧ ʈʂʰɻ̍˩\textasciitilde{}ʈʂʰɻ̍˩ |}}} \textcolor{Sepia}{\selectlanguage{english}to tighten the fist} \zh{攥紧拳头}  

\lhead{\firstmark}
\rhead{\botmark}

\subsection{\hspace{-0.5cm} {\Large \textcolor{darkblue}{\textbf{\ipa{ʈʂʰɻ̍˧˥}}} \textsubscript{2}}\hspace{0.5cm}[\kern2pt{\textcolor{darkblue}{\textbf{\ipa{ʈʂʰɻ̍˧˥}}}}\kern2pt]} \hypertarget{t`s`\string_hr£`̍\string_M\string_T2}{}
\markboth{\textcolor{darkblue}{\textbf{\ipa{ʈʂʰɻ̍˧˥}}} \textsubscript{2}}{}
\textcolor{teal}{\mytextsc{noun}} \hspace{4pt} Tone: MH.
\textcolor{Sepia}{\selectlanguage{english}Lung.} \zh{肺。}  \zh{量词}: \textcolor{darkblue}{\textbf{\ipa{ɭɯ˧}}}  \mytextsc{clf}: \textcolor{darkblue}{\textbf{\ipa{ɭɯ˧}}} 
\lhead{\firstmark}
\rhead{\botmark}

\subsection{\hspace{-0.5cm} {\Large \textcolor{darkblue}{\textbf{\ipa{ʈʂʰɻ̍˧˥\textsubscript{a}}}}}\hspace{0.5cm}[\kern2pt{\textcolor{darkblue}{\textbf{\ipa{ʈʂʰɻ̍˧˥}}}}\kern2pt]} \hypertarget{t`s`\string_hr£`̍\string_M\string_Ta1}{}
\markboth{\textcolor{darkblue}{\textbf{\ipa{ʈʂʰɻ̍˧˥\textsubscript{a}}}}}{}
\textcolor{teal}{\mytextsc{classifier}} \hspace{4pt} Tone: MH\textsubscript{a}.
\textcolor{Sepia}{\selectlanguage{english}Classifier for handfuls / balls: loose substance shaped into ball form by compressing it in the hand, for instance a handful of cooked cereals.} \zh{量词:团,掐。指的是一只手里能拿的量,压成团,如:手里拿煮熟的粮食,压成饭团。} 
\lhead{\firstmark}
\rhead{\botmark}

\subsection{\hspace{-0.5cm} {\Large \textcolor{darkblue}{\textbf{\ipa{ʈʂʰɯ˥}}} \textsubscript{1}}\hspace{0.5cm}[\kern2pt{\textcolor{darkblue}{\textbf{\ipa{ʈʂʰɯ˧˥}}}}\kern2pt]} \hypertarget{t`s`\string_hM\string_T1}{}
\markboth{\textcolor{darkblue}{\textbf{\ipa{ʈʂʰɯ˥}}} \textsubscript{1}}{}
\textcolor{teal}{\mytextsc{pronoun/pronominal}} \hspace{4pt} Tone: \#H.
\textcolor{Sepia}{\selectlanguage{english}This; proximal demonstrative.} \zh{这\mytextsc{指示}.近指。}  ¶ \textcolor{darkblue}{\textbf{\ipa{ʈʂʰɯ˧ ɲi˥!}}} \textcolor{Sepia}{\selectlanguage{english}This is it! / That's it! / That's right!} \zh{是这个! / 对的!}  
 ¶ \textcolor{darkblue}{\textbf{\ipa{ʈʂʰɯ˧-v̩\#˥}}} \textcolor{Sepia}{\selectlanguage{english}this one} \zh{这个}  
 ¶ \textcolor{darkblue}{\textbf{\ipa{ʈʂʰɯ˧=ɻæ˥}}} \textcolor{Sepia}{\selectlanguage{english}these things} \zh{这些}  
\textit{See:} \hyperlink{}{\textcolor{darkblue}{\textbf{\ipa{ʈʂʰɯ˥}}} \textsubscript{2}} \textit{See:} \textcolor{darkblue}{\textbf{\ipa{-ʈʂʰɯ˥}}} 
\lhead{\firstmark}
\rhead{\botmark}

\subsection{\hspace{-0.5cm} {\Large \textcolor{darkblue}{\textbf{\ipa{ʈʂʰɯ˥}}} \textsubscript{2}}\hspace{0.5cm}[\kern2pt{\textcolor{darkblue}{\textbf{\ipa{ʈʂʰɯ˥}}}}\kern2pt]} \hypertarget{t`s`\string_hM\string_T2}{}
\markboth{\textcolor{darkblue}{\textbf{\ipa{ʈʂʰɯ˥}}} \textsubscript{2}}{}
\textcolor{teal}{\mytextsc{pronoun/pronominal}} \hspace{4pt} Tone: \#H.
\textcolor{Sepia}{\selectlanguage{english}3rd person singular pronoun.} \zh{他。}  ¶ \textcolor{darkblue}{\textbf{\ipa{ʈʂʰɯ˧ ɲi˥!}}} \textcolor{Sepia}{\selectlanguage{english}That's her/him!} \zh{是他!}  
\textit{See:} \hyperlink{}{\textcolor{darkblue}{\textbf{\ipa{ʈʂʰɯ˥}}} \textsubscript{1}} \textit{See:} \textcolor{darkblue}{\textbf{\ipa{-ʈʂʰɯ˥}}} 
\lhead{\firstmark}
\rhead{\botmark}

\subsection{\hspace{-0.5cm} {\Large \textcolor{darkblue}{\textbf{\ipa{ʈʂʰɯ˧}}}}\hspace{0.5cm}[\kern2pt{\textcolor{darkblue}{\textbf{\ipa{ʈʂʰɯ˥}}}}\kern2pt]} \hypertarget{t`s`\string_hM\string_M1}{}
\markboth{\textcolor{darkblue}{\textbf{\ipa{ʈʂʰɯ˧}}}}{}
\textcolor{teal}{\mytextsc{suffix}} \hspace{4pt} Tone: M.
\textcolor{Sepia}{\selectlanguage{english}Topic marker; grammaticalized from the proximal demonstrative.} \zh{\mytextsc{主题(°指示}.近指)。} \textit{See:} \hyperlink{}{\textcolor{darkblue}{\textbf{\ipa{ʈʂʰɯ˥}}} \textsubscript{1}} \textit{See:} \hyperlink{}{\textcolor{darkblue}{\textbf{\ipa{ʈʂʰɯ˥}}} \textsubscript{2}} 
\lhead{\firstmark}
\rhead{\botmark}

\subsection{\hspace{-0.5cm} {\Large \textcolor{darkblue}{\textbf{\ipa{ʈʂʰɯ˧-gɤ˧}}}}\hspace{0.5cm}[\kern2pt{\textcolor{darkblue}{\textbf{\ipa{xxxx non-correspondance entre le nombre de morphèmes et le nombre de tons de morphèmes}}}}\kern2pt]} \hypertarget{t`s`\string_hM\string_M-g7\string_M1}{}
\markboth{\textcolor{darkblue}{\textbf{\ipa{ʈʂʰɯ˧-gɤ˧}}}}{}
\textcolor{teal}{\mytextsc{adverb(ial)}} \hspace{4pt} Tone: M.
\textcolor{Sepia}{\selectlanguage{english}Here.} \zh{这里。} \textit{See:} \hyperlink{}{\textcolor{darkblue}{\textbf{\ipa{ʈʂʰɯ˧gi\#˥}}}} 
\lhead{\firstmark}
\rhead{\botmark}

\subsection{\hspace{-0.5cm} {\Large \textcolor{darkblue}{\textbf{\ipa{ʈʂʰɯ˧gi\#˥}}}}\hspace{0.5cm}[\kern2pt{\textcolor{darkblue}{\textbf{\ipa{ʈʂʰɯ˧gi˧}}}}\kern2pt]} \hypertarget{t`s`\string_hM\string_Mgi\#\string_T1}{}
\markboth{\textcolor{darkblue}{\textbf{\ipa{ʈʂʰɯ˧gi\#˥}}}}{}
\textcolor{teal}{\mytextsc{adverb(ial)}} \hspace{4pt} Tone: \#H.
\textcolor{Sepia}{\selectlanguage{english}Here.} \zh{这边。} \textit{See:} \hyperlink{}{\textcolor{darkblue}{\textbf{\ipa{ʈʂʰɯ˧-gɤ˧}}}} 
\lhead{\firstmark}
\rhead{\botmark}

\subsection{\hspace{-0.5cm} {\Large \textcolor{darkblue}{\textbf{\ipa{ʈʂʰɯ˧ne˧-ʝi˥}}}}\hspace{0.5cm}[\kern2pt{\textcolor{darkblue}{\textbf{\ipa{xxxx non-correspondance entre le nombre de morphèmes et le nombre de tons de morphèmes}}}}\kern2pt]} \hypertarget{t`s`\string_hM\string_Mne\string_M-j££i\string_T1}{}
\markboth{\textcolor{darkblue}{\textbf{\ipa{ʈʂʰɯ˧ne˧-ʝi˥}}}}{}
\textcolor{teal}{\mytextsc{adverb(ial)}} \hspace{4pt} Tone: MH\#.
\textcolor{Sepia}{\selectlanguage{english}Thus, in this way.} \zh{这样,这么。}  ¶ \textcolor{darkblue}{\textbf{\ipa{ʈʂʰɯ˧ne˧-ʝi˥ | le˧-ʐwɤ˩!}}} \textcolor{Sepia}{\selectlanguage{english}This is how it's said!} \zh{是这样讲的!}  
 ¶ \textcolor{darkblue}{\textbf{\ipa{ʈʂʰɯ˧ne˧-ʝi˥ | le˧-pi˥!}}} \textcolor{Sepia}{\selectlanguage{english}This is how it's said!} \zh{是这样说的!}  
 ¶ \textcolor{darkblue}{\textbf{\ipa{ʈʂʰɯ˧ne˧-ʝi˥ | le˧-ʝi˥!}}} \textcolor{Sepia}{\selectlanguage{english}This is how it's done!} \zh{是这样做的!}  

\lhead{\firstmark}
\rhead{\botmark}

\subsection{\hspace{-0.5cm} {\Large \textcolor{darkblue}{\textbf{\ipa{ʈʂʰɯ˧qɑ˧}}}}\hspace{0.5cm}[\kern2pt{\textcolor{darkblue}{\textbf{\ipa{ʈʂʰɯ˧qɑ˧}}}}\kern2pt]} \hypertarget{t`s`\string_hM\string_MqA\string_M1}{}
\markboth{\textcolor{darkblue}{\textbf{\ipa{ʈʂʰɯ˧qɑ˧}}}}{}
\textcolor{teal}{\mytextsc{adverb(ial)}} \hspace{4pt} Tone: .
\textcolor{Sepia}{\selectlanguage{english}Together.} \zh{一起。} 
\lhead{\firstmark}
\rhead{\botmark}

\subsection{\hspace{-0.5cm} {\Large \textcolor{darkblue}{\textbf{\ipa{ʈʂʰɯ˧-qo˧}}}}\hspace{0.5cm}[\kern2pt{\textcolor{darkblue}{\textbf{\ipa{xxxx non-correspondance entre le nombre de morphèmes et le nombre de tons de morphèmes}}}}\kern2pt]} \hypertarget{t`s`\string_hM\string_M-qo\string_M1}{}
\markboth{\textcolor{darkblue}{\textbf{\ipa{ʈʂʰɯ˧-qo˧}}}}{}
\textcolor{teal}{\mytextsc{adverb(ial)}} \hspace{4pt} Tone: M.
\textcolor{Sepia}{\selectlanguage{english}Here.} \zh{这里。} 
\lhead{\firstmark}
\rhead{\botmark}

\subsection{\hspace{-0.5cm} {\Large \textcolor{darkblue}{\textbf{\ipa{ʈʂʰɯ˧tɕi˩}}}}\hspace{0.5cm}[\kern2pt{\textcolor{darkblue}{\textbf{\ipa{ʈʂʰɯ˧tɕi˩}}}}\kern2pt]} \hypertarget{t`s`\string_hM\string_Mts£i\string_B1}{}
\markboth{\textcolor{darkblue}{\textbf{\ipa{ʈʂʰɯ˧tɕi˩}}}}{}
\textcolor{teal}{\mytextsc{pronoun/pronominal}} \hspace{4pt} Tone: L\#.
\textcolor{Sepia}{\selectlanguage{english}Third-person plural pronoun.} \zh{他们。} 
\lhead{\firstmark}
\rhead{\botmark}

\subsection{\hspace{-0.5cm} {\Large \textcolor{darkblue}{\textbf{\ipa{ʈʂʰɯ˧=zɯ˩}}}}\hspace{0.5cm}[\kern2pt{\textcolor{darkblue}{\textbf{\ipa{ʈʂʰɯ˧zɯ˩}}}}\kern2pt]} \hypertarget{t`s`\string_hM\string_M=zM\string_B1}{}
\markboth{\textcolor{darkblue}{\textbf{\ipa{ʈʂʰɯ˧=zɯ˩}}}}{}
\textcolor{teal}{\mytextsc{pronoun/pronominal}} \hspace{4pt} Tone: L\#.
\textcolor{Sepia}{\selectlanguage{english}Dual third-person pronoun: the two of them.} \zh{他们两个。} 
\lhead{\firstmark}
\rhead{\botmark}

\subsection{\hspace{-0.5cm} {\Large \textcolor{darkblue}{\textbf{\ipa{ʈʂʰv̩˩}}} \textsubscript{1}}\hspace{0.5cm}[\kern2pt{\textcolor{darkblue}{\textbf{\ipa{ʈʂʰv̩˧˥}}}}\kern2pt]} \hypertarget{t`s`\string_hv\string_=\string_B1}{}
\markboth{\textcolor{darkblue}{\textbf{\ipa{ʈʂʰv̩˩}}} \textsubscript{1}}{}
\textcolor{teal}{\mytextsc{verb}} \hspace{4pt} Tone: MH.
\textcolor{Sepia}{\selectlanguage{english}To complete, to finish.} \zh{完成。}  ¶ \textcolor{darkblue}{\textbf{\ipa{le˧-ʈʂʰv̩˩-se˩}}} \textcolor{Sepia}{\selectlanguage{english}\mytextsc{accomp} \string_ \mytextsc{cmpl}} \zh{完成了}  
 ¶ \textcolor{darkblue}{\textbf{\ipa{tsʰi˧-ɲi˧-bv̩˧ | lo˧ | le˧-ʈʂʰv̩˩! | le˧-se˩-ze˩!}}} \textcolor{Sepia}{\selectlanguage{english}Today's work is completed! It's finished!} \zh{今天的工作完成了!就算完工了吧!}  

\lhead{\firstmark}
\rhead{\botmark}

\subsection{\hspace{-0.5cm} {\Large \textcolor{darkblue}{\textbf{\ipa{ʈʂʰv̩˩}}} \textsubscript{2}}\hspace{0.5cm}[\kern2pt{\textcolor{darkblue}{\textbf{\ipa{ʈʂʰv̩˩˥}}}}\kern2pt]} \hypertarget{t`s`\string_hv\string_=\string_B2}{}
\markboth{\textcolor{darkblue}{\textbf{\ipa{ʈʂʰv̩˩}}} \textsubscript{2}}{}
\textcolor{teal}{\mytextsc{verb}} \hspace{4pt} Tone: L.
\textcolor{Sepia}{\selectlanguage{english}To set aside, to set apart, to distinguish.} \zh{除开。}  ¶ \textcolor{darkblue}{\textbf{\ipa{gɤ˩-ʈʂʰv̩˧, | mv̩˩-ʈʂʰv̩˧-tsæ˩-ɲi˩}}} \textcolor{Sepia}{\selectlanguage{english}to set aside, to distinguish, not to mix} \zh{不算在里面、不算在一起}  
 ¶ \textcolor{darkblue}{\textbf{\ipa{no˧-bv̩˧ | gɤ˩-ʈʂʰv̩˧! | njɤ˧-bv̩˧, | mv̩˩-ʈʂʰv̩˧!}}} \textcolor{Sepia}{\selectlanguage{english}Your stuff belongs to you; and mine belongs to me!} \zh{你的算你的,我的算我的!}  

\lhead{\firstmark}
\rhead{\botmark}

\subsection{\hspace{-0.5cm} {\Large \textcolor{darkblue}{\textbf{\ipa{ʈʂʰv̩˧}}}}\hspace{0.5cm}[\kern2pt{\textcolor{darkblue}{\textbf{\ipa{ʈʂʰv̩˥}}}}\kern2pt]} \hypertarget{t`s`\string_hv\string_=\string_M1}{}
\markboth{\textcolor{darkblue}{\textbf{\ipa{ʈʂʰv̩˧}}}}{}
\textcolor{teal}{\mytextsc{noun}} \hspace{4pt} Tone: M.
\textcolor{Sepia}{\selectlanguage{english}Breakfast.} \zh{早饭。}  ¶ \textcolor{darkblue}{\textbf{\ipa{ʈʂʰv̩˧ dzɯ˧(-ze˩)}}} \textcolor{Sepia}{\selectlanguage{english}to have breakfast, to eat breakfast} \zh{吃早饭}  
 ¶ \textcolor{darkblue}{\textbf{\ipa{bæ˧qʰæ˧ ʈʂʰv̩\#˥}}} \textcolor{Sepia}{\selectlanguage{english}The breakfast shared when coming back from the cremation ceremony. Guests stop at the house of the deceased, where they are offered breakfast before they set home.} \zh{丧礼早餐:参加火葬仪式的人留在去世的人家,一起吃一点早饭再回家。}  

\lhead{\firstmark}
\rhead{\botmark}

\subsection{\hspace{-0.5cm} {\Large \textcolor{darkblue}{\textbf{\ipa{ʈʂʰv̩˧˥}}}}\hspace{0.5cm}[\kern2pt{\textcolor{darkblue}{\textbf{\ipa{ʈʂʰv̩˧˥}}}}\kern2pt]} \hypertarget{t`s`\string_hv\string_=\string_M\string_T1}{}
\markboth{\textcolor{darkblue}{\textbf{\ipa{ʈʂʰv̩˧˥}}}}{}
\textcolor{teal}{\mytextsc{verb}} \hspace{4pt} Tone: MH.
\textcolor{Sepia}{\selectlanguage{english}To add water, to pour extra water.} \zh{掺和。}  ¶ \textcolor{darkblue}{\textbf{\ipa{le˧-ʈʂʰv̩˧-ze˥}}} \textcolor{Sepia}{\selectlanguage{english}\mytextsc{accomp} \string_ \mytextsc{pfv}} \zh{\mytextsc{accomp} \string_ \mytextsc{pfv}}  
 ¶ \textcolor{darkblue}{\textbf{\ipa{dʑɯ˩ ʈʂʰv̩˩˥}}} \textcolor{Sepia}{\selectlanguage{english}to add water (e.g. in a pot)} \zh{加水(如:往锅里添加水)}  

\lhead{\firstmark}
\rhead{\botmark}

\subsection{\hspace{-0.5cm} {\Large \textcolor{darkblue}{\textbf{\ipa{ʈʂʰv̩˩\textsubscript{a}}}}}\hspace{0.5cm}[\kern2pt{\textcolor{darkblue}{\textbf{\ipa{ʈʂʰv̩˩˥}}}}\kern2pt]} \hypertarget{t`s`\string_hv\string_=\string_Ba1}{}
\markboth{\textcolor{darkblue}{\textbf{\ipa{ʈʂʰv̩˩\textsubscript{a}}}}}{}
\textcolor{teal}{\mytextsc{verb}} \hspace{4pt} Tone: L\textsubscript{a}.
\textcolor{Sepia}{\selectlanguage{english}To dye.} \zh{染。}  ¶ \textcolor{darkblue}{\textbf{\ipa{mɤ˧-ʈʂʰv̩˩}}} \textcolor{Sepia}{\selectlanguage{english}\mytextsc{neg}} \zh{\mytextsc{neg}}  
 ¶ \textcolor{darkblue}{\textbf{\ipa{ʈʂʰv̩˩ mɤ˩-bi˩˥!}}} \textcolor{Sepia}{\selectlanguage{english}\string_ \mytextsc{neg} \mytextsc{fut}\string_imm} \zh{\string_ \mytextsc{neg} \mytextsc{fut}\string_imm}  
 ¶ \textcolor{darkblue}{\textbf{\ipa{tso˧\textasciitilde{}tso˧ ʈʂʰv̩˥}}} \textcolor{Sepia}{\selectlanguage{english}to dye things} \zh{染东西}  

\lhead{\firstmark}
\rhead{\botmark}

\subsection{\hspace{-0.5cm} {\Large \textcolor{darkblue}{\textbf{\ipa{ʈʂʰv̩˧dʑɯ˧}}}}\hspace{0.5cm}[\kern2pt{\textcolor{darkblue}{\textbf{\ipa{ʈʂʰv̩˧dʑɯ˧}}}}\kern2pt]} \hypertarget{t`s`\string_hv\string_=\string_Mdz£M\string_M1}{}
\markboth{\textcolor{darkblue}{\textbf{\ipa{ʈʂʰv̩˧dʑɯ˧}}}}{}
\textcolor{teal}{\mytextsc{noun}} \hspace{4pt} Tone: M.
\textcolor{Sepia}{\selectlanguage{english}Dye, dyestuff.} \zh{染料。}  ¶ \textcolor{darkblue}{\textbf{\ipa{dʑi˧hṽ˧-ʈʂʰv̩˧dʑɯ˧}}} \textcolor{Sepia}{\selectlanguage{english}dye for clothes} \zh{衣服染料}  
 ¶ \textcolor{darkblue}{\textbf{\ipa{ʈʂʰv̩˧dʑɯ˧ | hṽ˩-hĩ˩˥}}} \textcolor{Sepia}{\selectlanguage{english}red dye} \zh{红色的染料}  
 \zh{量词}: \textcolor{darkblue}{\textbf{\ipa{kʰwɤ˥}}}  \mytextsc{clf}: \textcolor{darkblue}{\textbf{\ipa{kʰwɤ˥}}} 
\lhead{\firstmark}
\rhead{\botmark}

\subsection{\hspace{-0.5cm} {\Large \textcolor{darkblue}{\textbf{\ipa{ʈʂʰv̩˧mi˧}}}}\hspace{0.5cm}[\kern2pt{\textcolor{darkblue}{\textbf{\ipa{ʈʂʰv̩˧mi˧}}}}\kern2pt]} \hypertarget{t`s`\string_hv\string_=\string_Mmi\string_M1}{}
\markboth{\textcolor{darkblue}{\textbf{\ipa{ʈʂʰv̩˧mi˧}}}}{}
\textcolor{teal}{\mytextsc{noun}} \hspace{4pt} Tone: M.
\textcolor{Sepia}{\selectlanguage{english}Wife.} \zh{太太、老婆、媳妇。}  \zh{量词}: \textcolor{darkblue}{\textbf{\ipa{v̩˧}}}  \mytextsc{clf}: \textcolor{darkblue}{\textbf{\ipa{v̩˧}}} 
\lhead{\firstmark}
\rhead{\botmark}

\subsection{\hspace{-0.5cm} {\Large \textcolor{darkblue}{\textbf{\ipa{ʈʂʰv̩˧ɻ̍˧qʰv̩\#˥}}}}\hspace{0.5cm}[\kern2pt{\textcolor{darkblue}{\textbf{\ipa{ʈʂʰv̩˧ɻ̍˧qʰv̩˧}}}}\kern2pt]} \hypertarget{t`s`\string_hv\string_=\string_Mr£`̍\string_Mq\string_hv\string_=\#\string_T1}{}
\markboth{\textcolor{darkblue}{\textbf{\ipa{ʈʂʰv̩˧ɻ̍˧qʰv̩\#˥}}}}{}
\textcolor{teal}{\mytextsc{noun}} \hspace{4pt} Tone: \#H.
\textcolor{Sepia}{\selectlanguage{english}Ant nest.} \zh{蚂蚁巢。}  ¶ \textcolor{darkblue}{\textbf{\ipa{ʈʂʰv̩˧ɻ̍˧qʰv̩˧ ɲi˥!}}} \textcolor{Sepia}{\selectlanguage{english}It's an ant nest!} \zh{是蚂蚁巢!}  
 \zh{量词}: \textcolor{darkblue}{\textbf{\ipa{ɭɯ˧}}}  \mytextsc{clf}: \textcolor{darkblue}{\textbf{\ipa{ɭɯ˧}}} \textit{See:} \hyperlink{}{\textcolor{darkblue}{\textbf{\ipa{ʈʂʰv̩˧ɻ̍˥\$}}}} 
\lhead{\firstmark}
\rhead{\botmark}

\subsection{\hspace{-0.5cm} {\Large \textcolor{darkblue}{\textbf{\ipa{ʈʂʰv̩˧ɻ̍˥\$}}}}\hspace{0.5cm}[\kern2pt{\textcolor{darkblue}{\textbf{\ipa{ʈʂʰv̩˧ɻ̍˥}}}}\kern2pt]} \hypertarget{t`s`\string_hv\string_=\string_Mr£`̍\string_T\$1}{}
\markboth{\textcolor{darkblue}{\textbf{\ipa{ʈʂʰv̩˧ɻ̍˥\$}}}}{}
\textcolor{teal}{\mytextsc{noun}} \hspace{4pt} Tone: H\$.
\textcolor{Sepia}{\selectlanguage{english}Ant.} \zh{蚂蚁。}  ¶ \textcolor{darkblue}{\textbf{\ipa{ʈʂʰv̩˧ɻ̍˧ | tɕi˩-hĩ˩˥}}} \textcolor{Sepia}{\selectlanguage{english}small ant} \zh{小蚂蚁}  
 \zh{量词}: \textcolor{darkblue}{\textbf{\ipa{mi˩}}}  \mytextsc{clf}: \textcolor{darkblue}{\textbf{\ipa{mi˩}}} 
\lhead{\firstmark}
\rhead{\botmark}

\subsection{\hspace{-0.5cm} {\Large \textcolor{darkblue}{\textbf{\ipa{ʈʂʰwæ˧\textsubscript{a}}}} \textsubscript{1}}\hspace{0.5cm}[\kern2pt{\textcolor{darkblue}{\textbf{\ipa{ʈʂʰwæ˩˥}}}}\kern2pt]} \hypertarget{t`s`\string_hw\{\string_Ma1}{}
\markboth{\textcolor{darkblue}{\textbf{\ipa{ʈʂʰwæ˧\textsubscript{a}}}} \textsubscript{1}}{}
\textcolor{teal}{\mytextsc{verb}} \hspace{4pt} Tone: M\textsubscript{a}.
\textcolor{Sepia}{\selectlanguage{english}To rot.} \zh{腐烂。}  ¶ \textcolor{darkblue}{\textbf{\ipa{ʈʂʰwæ˧-ze˧}}} \textcolor{Sepia}{\selectlanguage{english}\mytextsc{pfv}} \zh{烂了}  
 ¶ \textcolor{darkblue}{\textbf{\ipa{le˧-ʈʂʰwæ˧-ze˧}}} \textcolor{Sepia}{\selectlanguage{english}\mytextsc{accomp} \string_ \mytextsc{pfv}} \zh{\mytextsc{accomp} \string_ \mytextsc{pfv}}  
 ¶ \textcolor{darkblue}{\textbf{\ipa{hĩ˧-ɳɯ˩ | mɤ˧-dzɯ˥, | le˧-ʈʂʰwæ˧-ze˧! |}}} \textcolor{Sepia}{\selectlanguage{english}No one ate it, and now it's rotten! (About a water melon that was forgotten in the cupboard.)} \zh{没人吃,就烂了!(一个西瓜被忘记在橱柜里,就腐烂了)}  

\lhead{\firstmark}
\rhead{\botmark}

\subsection{\hspace{-0.5cm} {\Large \textcolor{darkblue}{\textbf{\ipa{ʈʂʰwæ˧\textsubscript{a}}}} \textsubscript{2}}\hspace{0.5cm}[\kern2pt{\textcolor{darkblue}{\textbf{\ipa{ʈʂʰwæ˥}}}}\kern2pt]} \hypertarget{t`s`\string_hw\{\string_Ma2}{}
\markboth{\textcolor{darkblue}{\textbf{\ipa{ʈʂʰwæ˧\textsubscript{a}}}} \textsubscript{2}}{}
\textcolor{teal}{\mytextsc{verb}} \hspace{4pt} Tone: M\textsubscript{a}.
\textcolor{Sepia}{\selectlanguage{english}To wake up.} \zh{醒来。}  ¶ \textcolor{darkblue}{\textbf{\ipa{le˧-ʈʂʰwæ˧-ze˧}}} \textcolor{Sepia}{\selectlanguage{english}\mytextsc{accomp} \string_ \mytextsc{pfv}} \zh{\mytextsc{accomp} \string_ \mytextsc{pfv}}  
 ¶ \textcolor{darkblue}{\textbf{\ipa{gɤ˩ʈʂʰwæ˧}}} \textcolor{Sepia}{\selectlanguage{english}to wake up} \zh{醒来}  
 ¶ \textcolor{darkblue}{\textbf{\ipa{gɤ˩ʈʂʰwæ˧-ze˧!}}} \textcolor{Sepia}{\selectlanguage{english}[(S)he] has woken up!} \zh{醒来了!}  

\lhead{\firstmark}
\rhead{\botmark}

\subsection{\hspace{-0.5cm} {\Large \textcolor{darkblue}{\textbf{\ipa{ʈʂʰwæ˧-bv̩˧nv̩\#˥}}}}\hspace{0.5cm}[\kern2pt{\textcolor{darkblue}{\textbf{\ipa{xxxx non-correspondance entre le nombre de morphèmes et le nombre de tons de morphèmes}}}}\kern2pt]} \hypertarget{t`s`\string_hw\{\string_M-bv\string_=\string_Mnv\string_=\#\string_T1}{}
\markboth{\textcolor{darkblue}{\textbf{\ipa{ʈʂʰwæ˧-bv̩˧nv̩\#˥}}}}{}
\textcolor{teal}{\mytextsc{adjective}} \hspace{4pt} Tone: \#H.
\textcolor{Sepia}{\selectlanguage{english}Bad, spoilt, rotten.} \zh{食物变味,有臭味道了。}  ¶ \textcolor{darkblue}{\textbf{\ipa{ʈʂʰwæ˧-bv̩˧nv̩˧ ɲi˥!}}} \textcolor{Sepia}{\selectlanguage{english}It stinks! / It smells rotten!} \zh{臭了、有臭味道了}  

\lhead{\firstmark}
\rhead{\botmark}

\subsection{\hspace{-0.5cm} {\Large \textcolor{darkblue}{\textbf{\ipa{ʈʂʰwæ˧kɯ˧}}}}\hspace{0.5cm}[\kern2pt{\textcolor{darkblue}{\textbf{\ipa{ʈʂʰwæ˩kɯ˥}}}}\kern2pt]} \hypertarget{t`s`\string_hw\{\string_MkM\string_M1}{}
\markboth{\textcolor{darkblue}{\textbf{\ipa{ʈʂʰwæ˧kɯ˧}}}}{}
\textcolor{teal}{\mytextsc{noun}} \hspace{4pt} Tone: M.
\textcolor{Sepia}{\selectlanguage{english}Net.} \zh{网。}  ¶ \textcolor{darkblue}{\textbf{\ipa{ʈʂʰwæ˧kɯ˧ tʰv̩˧-nɑ˩}}} \textcolor{Sepia}{\selectlanguage{english}\mytextsc{n}+\mytextsc{dem}+\mytextsc{clf}} \zh{这个网}  
 \zh{量词}: \textcolor{darkblue}{\textbf{\ipa{nɑ˧}}}  \mytextsc{clf}: \textcolor{darkblue}{\textbf{\ipa{nɑ˧}}} 
\lhead{\firstmark}
\rhead{\botmark}

\subsection{\hspace{-0.5cm} {\Large \textcolor{darkblue}{\textbf{\ipa{ʈʂʰwæ˧tsɯ˧}}}}\hspace{0.5cm}[\kern2pt{\textcolor{darkblue}{\textbf{\ipa{ʈʂʰwæ˩tsɯ˩˥}}}}\kern2pt]} \hypertarget{t`s`\string_hw\{\string_MtsM\string_M1}{}
\markboth{\textcolor{darkblue}{\textbf{\ipa{ʈʂʰwæ˧tsɯ˧}}}}{}
\textcolor{teal}{\mytextsc{noun}} \hspace{4pt} Tone: M.
\textcolor{Sepia}{\selectlanguage{english}Window.} \zh{窗户。} Local Chinese dialect:\zh{窗子。} Borrowing: Chinese  \zh{窗子}
 \zh{量词}: \textcolor{darkblue}{\textbf{\ipa{nɑ˧}}}  \mytextsc{clf}: \textcolor{darkblue}{\textbf{\ipa{nɑ˧}}} 
\lhead{\firstmark}
\rhead{\botmark}

\subsection{\hspace{-0.5cm} {\Large \textcolor{darkblue}{\textbf{\ipa{ʈʂʰwæ˧ʈʂʰwæ˧}}}}\hspace{0.5cm}[\kern2pt{\textcolor{darkblue}{\textbf{\ipa{ʈʂʰwæ˧ʈʂʰwæ˧}}}}\kern2pt]} \hypertarget{t`s`\string_hw\{\string_Mt`s`\string_hw\{\string_M1}{}
\markboth{\textcolor{darkblue}{\textbf{\ipa{ʈʂʰwæ˧ʈʂʰwæ˧}}}}{}
\textcolor{teal}{\mytextsc{noun}} \hspace{4pt} Tone: M.
\textcolor{Sepia}{\selectlanguage{english}Cymbals (Chinese borrowing).} \zh{钹。}  \zh{量词}: \textcolor{darkblue}{\textbf{\ipa{nɑ˧}}}  \mytextsc{clf}: \textcolor{darkblue}{\textbf{\ipa{nɑ˧}}} 
\lhead{\firstmark}
\rhead{\botmark}

\subsection{\hspace{-0.5cm} {\Large \textcolor{darkblue}{\textbf{\ipa{ʈʂʰwæ˩\textsubscript{a}}}}}\hspace{0.5cm}[\kern2pt{\textcolor{darkblue}{\textbf{\ipa{ʈʂʰwæ˥}}}}\kern2pt]} \hypertarget{t`s`\string_hw\{\string_Ba1}{}
\markboth{\textcolor{darkblue}{\textbf{\ipa{ʈʂʰwæ˩\textsubscript{a}}}}}{}
\textcolor{teal}{\mytextsc{adjective}} \hspace{4pt} Tone: L\textsubscript{a}.
\textcolor{Sepia}{\selectlanguage{english}Rapid, fast.} \zh{快(动作快,跑得快)。}  ¶ \textcolor{darkblue}{\textbf{\ipa{ʈʂʰwæ˩-hĩ˩˥}}} \textcolor{Sepia}{\selectlanguage{english}\mytextsc{rel}/\mytextsc{nmlz}} \zh{快的}  
 ¶ \textcolor{darkblue}{\textbf{\ipa{ɲi˧to˧ ʈʂʰwæ˩}}} \textcolor{Sepia}{\selectlanguage{english}who has a loose tongue, to talks too much} \zh{嘴快}  

\lhead{\firstmark}
\rhead{\botmark}

\subsection{\hspace{-0.5cm} {\Large \textcolor{darkblue}{\textbf{\ipa{ʈʂʰwæ˩tsʰɯ˩}}}}\hspace{0.5cm}[\kern2pt{\textcolor{darkblue}{\textbf{\ipa{ʈʂʰwæ˧tsʰɯ˩}}}}\kern2pt]} \hypertarget{t`s`\string_hw\{\string_Bts\string_hM\string_B1}{}
\markboth{\textcolor{darkblue}{\textbf{\ipa{ʈʂʰwæ˩tsʰɯ˩}}}}{}
\textcolor{teal}{\mytextsc{verb}} \hspace{4pt} Tone: L.
\textcolor{Sepia}{\selectlanguage{english}To create.} \zh{创造(汉语借词)。}  Borrowing: Chinese  \zh{创造}

\lhead{\firstmark}
\rhead{\botmark}

\subsection{\hspace{-0.5cm} {\Large \textcolor{darkblue}{\textbf{\ipa{ʈʂʰwæ˧˥}}} \textsubscript{1}}\hspace{0.5cm}[\kern2pt{\textcolor{darkblue}{\textbf{\ipa{ʈʂʰwæ˩˥}}}}\kern2pt]} \hypertarget{t`s`\string_hw\{\string_M\string_T1}{}
\markboth{\textcolor{darkblue}{\textbf{\ipa{ʈʂʰwæ˧˥}}} \textsubscript{1}}{}
\textcolor{teal}{\mytextsc{verb}} \hspace{4pt} Tone: MH.
\textcolor{Sepia}{\selectlanguage{english}To hide (an object).} \zh{藏(东西)。} 
\lhead{\firstmark}
\rhead{\botmark}

\subsection{\hspace{-0.5cm} {\Large \textcolor{darkblue}{\textbf{\ipa{ʈʂʰwæ˧˥}}} \textsubscript{2}}\hspace{0.5cm}[\kern2pt{\textcolor{darkblue}{\textbf{\ipa{ʈʂʰwæ˧˥}}}}\kern2pt]} \hypertarget{t`s`\string_hw\{\string_M\string_T2}{}
\markboth{\textcolor{darkblue}{\textbf{\ipa{ʈʂʰwæ˧˥}}} \textsubscript{2}}{}
\textcolor{teal}{\mytextsc{verb}} \hspace{4pt} Tone: MH.
\textcolor{Sepia}{\selectlanguage{english}To stab.} \zh{插、戳。} 
\lhead{\firstmark}
\rhead{\botmark}

\subsection{\hspace{-0.5cm} {\Large \textcolor{darkblue}{\textbf{\ipa{ʈʂʰwæ˩˧}}}}\hspace{0.5cm}[\kern2pt{\textcolor{darkblue}{\textbf{\ipa{ʈʂʰwæ˧˥}}}}\kern2pt]} \hypertarget{t`s`\string_hw\{\string_B\string_M1}{}
\markboth{\textcolor{darkblue}{\textbf{\ipa{ʈʂʰwæ˩˧}}}}{}
\textcolor{teal}{\mytextsc{noun}} \hspace{4pt} Tone: LM.
\textcolor{Sepia}{\selectlanguage{english}Boat.} \zh{船(汉语借词)。}  Borrowing: Chinese  \zh{船}
 \zh{量词}: \textcolor{darkblue}{\textbf{\ipa{nɑ˧}}}  \mytextsc{clf}: \textcolor{darkblue}{\textbf{\ipa{nɑ˧}}} 
\lhead{\firstmark}
\rhead{\botmark}

\subsection{\hspace{-0.5cm} {\Large \textcolor{darkblue}{\textbf{\ipa{ʈʂʰwɤ˧tsʰi˧˥}}}}\hspace{0.5cm}[\kern2pt{\textcolor{darkblue}{\textbf{\ipa{ʈʂʰwɤ˧tsʰi˧}}}}\kern2pt]} \hypertarget{t`s`\string_hw7\string_Mts\string_hi\string_M\string_T1}{}
\markboth{\textcolor{darkblue}{\textbf{\ipa{ʈʂʰwɤ˧tsʰi˧˥}}}}{}
\textcolor{teal}{\mytextsc{adjective}} \hspace{4pt} Tone: MH\#.
\textcolor{Sepia}{\selectlanguage{english}Narrow.} \zh{窄。} 
\lhead{\firstmark}
\rhead{\botmark}

\subsection{\hspace{-0.5cm} {\Large \textcolor{darkblue}{\textbf{\ipa{ʈʂʰwɤ˩}}}}\hspace{0.5cm}[\kern2pt{\textcolor{darkblue}{\textbf{\ipa{ʈʂʰwɤ˥}}}}\kern2pt]} \hypertarget{t`s`\string_hw7\string_B1}{}
\markboth{\textcolor{darkblue}{\textbf{\ipa{ʈʂʰwɤ˩}}}}{}
\textcolor{teal}{\mytextsc{noun}} \hspace{4pt} Tone: L.
\textcolor{Sepia}{\selectlanguage{english}Dinner.} \zh{晚饭。}  ¶ \textcolor{darkblue}{\textbf{\ipa{ʈʂʰwɤ˩ gv̩˩˥}}} \textcolor{Sepia}{\selectlanguage{english}to cook dinner} \zh{做晚饭}  
 ¶ \textcolor{darkblue}{\textbf{\ipa{ʈʂʰwɤ˩ tʰv̩˩˥}}} \textcolor{Sepia}{\selectlanguage{english}to take charge of dinner, to prepare dinner, to provide dinner (this can refer to providing the ingredients for the meal, not necessarily preparing it oneself)} \zh{请吃晚饭,提供晚餐(不一定自己做:意思是提供原料)}  
 ¶ \textcolor{darkblue}{\textbf{\ipa{ʈʂʰwɤ˩ dzɯ˩˥}}} \textcolor{Sepia}{\selectlanguage{english}to eat dinner} \zh{吃晚饭}  

\lhead{\firstmark}
\rhead{\botmark}

\newpage
\section*{\centering- \textcolor{darkblue}{\textbf{\ipa{u}}} -}
\subsection{\hspace{-0.5cm} {\Large \textcolor{darkblue}{\textbf{\ipa{u˧}}}}\hspace{0.5cm}[\kern2pt{\textcolor{darkblue}{\textbf{\ipa{u˥}}}}\kern2pt]} \hypertarget{u\string_M1}{}
\markboth{\textcolor{darkblue}{\textbf{\ipa{u˧}}}}{}
\textcolor{teal}{\mytextsc{pronoun/pronominal}} \hspace{4pt} Tone: M.
\textcolor{Sepia}{\selectlanguage{english}First person, associative: my family/my people. This root is only attested together with a plural or associative clitic.} \zh{我家人、我家族。}  ¶ \textcolor{darkblue}{\textbf{\ipa{u˧=ɻ˩, ʈʂʰɯ˧=ɻ˩}}} \textcolor{Sepia}{\selectlanguage{english}My clan, his clan: two terms that stand in a relation of opposition}  
 ¶ \textcolor{darkblue}{\textbf{\ipa{u˧ɻ̍˩ | ə˧si˧}}} \textcolor{Sepia}{\selectlanguage{english}my great-grandmother} \zh{我家祖母}  
 ¶ \textcolor{darkblue}{\textbf{\ipa{u˧=ɻæ˩, ʈʂʰɯ˧=ɻæ˩}}} \textcolor{Sepia}{\selectlanguage{english}Us, them: two terms that stand in a relation of opposition}  

\lhead{\firstmark}
\rhead{\botmark}

\newpage
\section*{\centering- \textcolor{darkblue}{\textbf{\ipa{v}}} \textcolor{darkblue}{\textbf{\ipa{ṽ}}} -}
\subsection{\hspace{-0.5cm} {\Large \textcolor{darkblue}{\textbf{\ipa{v̩˩}}}}\hspace{0.5cm}[\kern2pt{\textcolor{darkblue}{\textbf{\ipa{v̩˥}}}}\kern2pt]} \hypertarget{v\string_=\string_B1}{}
\markboth{\textcolor{darkblue}{\textbf{\ipa{v̩˩}}}}{}
\textcolor{teal}{\mytextsc{verb}} \hspace{4pt} Tone: L\textsubscript{a}.
\textcolor{Sepia}{\selectlanguage{english}To hug, to embrace.} \zh{搂(人的脖子)。}  ¶ \textcolor{darkblue}{\textbf{\ipa{ʁæ˧ | le˧-v̩˧\textasciitilde{}v̩˥}}} \textcolor{Sepia}{\selectlanguage{english}to embrace someone's neck} \zh{搂人的脖子}  
 ¶ \textcolor{darkblue}{\textbf{\ipa{ʁæ˧ | le˧-v̩˩}}} \textcolor{Sepia}{\selectlanguage{english}as above} \zh{同上}  
 ¶ \textcolor{darkblue}{\textbf{\ipa{ʁæ˧ v̩˥ se˩}}} \textcolor{Sepia}{\selectlanguage{english}to walk together with someone, arm curled around the neck} \zh{互相搂着走}  

\lhead{\firstmark}
\rhead{\botmark}

\subsection{\hspace{-0.5cm} {\Large \textcolor{darkblue}{\textbf{\ipa{v̩˧}}}}\hspace{0.5cm}[\kern2pt{\textcolor{darkblue}{\textbf{\ipa{v̩˩˥}}}}\kern2pt]} \hypertarget{v\string_=\string_M1}{}
\markboth{\textcolor{darkblue}{\textbf{\ipa{v̩˧}}}}{}
\textcolor{teal}{\mytextsc{classifier}} \hspace{4pt} Tone: M *.
\textcolor{Sepia}{\selectlanguage{english}Classifier for one individual (human); can only be used after the numeral 'one', i.e. either in the singular or in association with numbers whose last figure is 'one'.} \zh{量词:人(一个人)。只能用于单数。}  ¶ \textcolor{darkblue}{\textbf{\ipa{ɖɯ˧-v̩\#˥; ɖɯ˧-v̩˧ ɲi˥}}} \textcolor{Sepia}{\selectlanguage{english}one person; it is one person (elicited to verify tone)} \zh{一个人,是一个人(为了确认声调而问的短语)}  
 ¶ \textcolor{darkblue}{\textbf{\ipa{tsʰe˧ɖɯ˧-v̩˧}}} \textcolor{Sepia}{\selectlanguage{english}11 persons} \zh{十一个人}  
 ¶ \textcolor{darkblue}{\textbf{\ipa{ɲi˧tsi˧ɖɯ˧-v̩˧}}} \textcolor{Sepia}{\selectlanguage{english}21 persons} \zh{二十一个人}  
 ¶ \textcolor{darkblue}{\textbf{\ipa{so˧tsʰi˧ɖɯ˧-v̩˧}}} \textcolor{Sepia}{\selectlanguage{english}31 persons} \zh{三十一个人}  
 ¶ \textcolor{darkblue}{\textbf{\ipa{ʐv̩˧tsʰi˩ɖɯ˩-v̩˩}}} \textcolor{Sepia}{\selectlanguage{english}41 persons} \zh{四十一个人}  
 ¶ \textcolor{darkblue}{\textbf{\ipa{ŋwɤ˧tsʰi˩ɖɯ˩-v̩˩}}} \textcolor{Sepia}{\selectlanguage{english}51 persons} \zh{五十一个人}  
 ¶ \textcolor{darkblue}{\textbf{\ipa{qʰv̩˧tsʰi˧ɖɯ˧-v̩˥}}} \textcolor{Sepia}{\selectlanguage{english}61 persons} \zh{六十一个人}  
 ¶ \textcolor{darkblue}{\textbf{\ipa{ʂɯ˧tsʰi˩ɖɯ˩-v̩˩}}} \textcolor{Sepia}{\selectlanguage{english}71 persons} \zh{七十一个人}  
 ¶ \textcolor{darkblue}{\textbf{\ipa{hõ˧tsʰi˩ɖɯ˩-v̩˩}}} \textcolor{Sepia}{\selectlanguage{english}81 persons} \zh{八十一个人}  
 ¶ \textcolor{darkblue}{\textbf{\ipa{gv̩˧tsʰi˩ɖɯ˩-v̩˩}}} \textcolor{Sepia}{\selectlanguage{english}91 persons} \zh{九十一个人}  

\lhead{\firstmark}
\rhead{\botmark}

\subsection{\hspace{-0.5cm} {\Large \textcolor{darkblue}{\textbf{\ipa{v̩˥}}}}\hspace{0.5cm}[\kern2pt{\textcolor{darkblue}{\textbf{\ipa{v̩˥}}}}\kern2pt]} \hypertarget{v\string_=\string_T1}{}
\markboth{\textcolor{darkblue}{\textbf{\ipa{v̩˥}}}}{}
\textcolor{teal}{\mytextsc{noun}} \hspace{4pt} Tone: \#H.
\textcolor{Sepia}{\selectlanguage{english}Pot.} \zh{锅。}  \zh{量词}: \textcolor{darkblue}{\textbf{\ipa{ɭɯ˧}}}  \mytextsc{clf}: \textcolor{darkblue}{\textbf{\ipa{ɭɯ˧}}} 
\lhead{\firstmark}
\rhead{\botmark}

\subsection{\hspace{-0.5cm} {\Large \textcolor{darkblue}{\textbf{\ipa{v̩˥\textsubscript{b}}}}}\hspace{0.5cm}[\kern2pt{\textcolor{darkblue}{\textbf{\ipa{v̩˥}}}}\kern2pt]} \hypertarget{v\string_=\string_Tb1}{}
\markboth{\textcolor{darkblue}{\textbf{\ipa{v̩˥\textsubscript{b}}}}}{}
\textcolor{teal}{\mytextsc{classifier}} \hspace{4pt} Tone: H\textsubscript{b}.
\textcolor{Sepia}{\selectlanguage{english}Self-classifier for pots; classifier for potfuls (of food, liquid…).} \zh{量词:锅(一口),或锅的容量。} 
\lhead{\firstmark}
\rhead{\botmark}

\subsection{\hspace{-0.5cm} {\Large \textcolor{darkblue}{\textbf{\ipa{v̩˩dze˩}}}}\hspace{0.5cm}[\kern2pt{\textcolor{darkblue}{\textbf{\ipa{v̩˥}}}}\kern2pt]} \hypertarget{v\string_=\string_Bdze\string_B1}{}
\markboth{\textcolor{darkblue}{\textbf{\ipa{v̩˩dze˩}}}}{}
\textcolor{teal}{\mytextsc{noun}} \hspace{4pt} Tone: L.
\textcolor{Sepia}{\selectlanguage{english}Bird.} \zh{鸟。}  ¶ \textcolor{darkblue}{\textbf{\ipa{v̩˩dze˩-bi˥ | hṽ˧ ʑi˥}}} \textcolor{Sepia}{\selectlanguage{english}There are feathers on the bird.} \zh{鸟(身)上有(羽)毛。}  
 ¶ \textcolor{darkblue}{\textbf{\ipa{v̩˩dze˩-mi˩}}} \textcolor{Sepia}{\selectlanguage{english}female bird} \zh{母鸟}  
 ¶ \textcolor{darkblue}{\textbf{\ipa{v̩˩dze˩-pʰv̩˩}}} \textcolor{Sepia}{\selectlanguage{english}male bird} \zh{公鸟}  
 ¶ \textcolor{darkblue}{\textbf{\ipa{v̩˩dze˩-zo˩}}} \textcolor{Sepia}{\selectlanguage{english}baby bird} \zh{小鸟}  
 \zh{量词}: \textcolor{darkblue}{\textbf{\ipa{mi˩}}}  \mytextsc{clf}: \textcolor{darkblue}{\textbf{\ipa{mi˩}}} 
\lhead{\firstmark}
\rhead{\botmark}

\subsection{\hspace{-0.5cm} {\Large \textcolor{darkblue}{\textbf{\ipa{v̩˩dze˩-kʰv̩˩}}}}\hspace{0.5cm}[\kern2pt{\textcolor{darkblue}{\textbf{\ipa{xxxx non-correspondance entre le nombre de morphèmes et le nombre de tons de morphèmes}}}}\kern2pt]} \hypertarget{v\string_=\string_Bdze\string_B-k\string_hv\string_=\string_B1}{}
\markboth{\textcolor{darkblue}{\textbf{\ipa{v̩˩dze˩-kʰv̩˩}}}}{}
\textcolor{teal}{\mytextsc{noun}} \hspace{4pt} Tone: L.
\textcolor{Sepia}{\selectlanguage{english}Nest.} \zh{鸟窝,鸟巢。}  ¶ \textcolor{darkblue}{\textbf{\ipa{v̩˩dze˩kʰv̩˩ ɲi˥.}}} \textcolor{Sepia}{\selectlanguage{english}\string_ \mytextsc{cop}} \zh{是鸟窝}  
 \zh{量词}: \textcolor{darkblue}{\textbf{\ipa{ɭɯ˧}}}  \mytextsc{clf}: \textcolor{darkblue}{\textbf{\ipa{ɭɯ˧}}} 
\lhead{\firstmark}
\rhead{\botmark}

\subsection{\hspace{-0.5cm} {\Large \textcolor{darkblue}{\textbf{\ipa{v̩˧dʑo\#˥}}}}\hspace{0.5cm}[\kern2pt{\textcolor{darkblue}{\textbf{\ipa{v̩˩dʑo˩˥}}}}\kern2pt]} \hypertarget{v\string_=\string_Mdz£o\#\string_T1}{}
\markboth{\textcolor{darkblue}{\textbf{\ipa{v̩˧dʑo\#˥}}}}{}
\textcolor{teal}{\mytextsc{noun}} \hspace{4pt} Tone: \#H.
\textcolor{Sepia}{\selectlanguage{english}Wujiao township.} \zh{屋脚(村落名)。} 
\lhead{\firstmark}
\rhead{\botmark}

\subsection{\hspace{-0.5cm} {\Large \textcolor{darkblue}{\textbf{\ipa{v̩˩dʑɯ˩}}}}\hspace{0.5cm}[\kern2pt{\textcolor{darkblue}{\textbf{\ipa{v̩˧dʑɯ˧}}}}\kern2pt]} \hypertarget{v\string_=\string_Bdz£M\string_B1}{}
\markboth{\textcolor{darkblue}{\textbf{\ipa{v̩˩dʑɯ˩}}}}{}
\textcolor{teal}{\mytextsc{noun}} \hspace{4pt} Tone: L.
\textcolor{Sepia}{\selectlanguage{english}Soup.} \zh{汤。}  ¶ \textcolor{darkblue}{\textbf{\ipa{æ˩ʂe˧-v̩˥dʑɯ˩}}} \textcolor{Sepia}{\selectlanguage{english}chicken soup} \zh{鸡汤}  

\lhead{\firstmark}
\rhead{\botmark}

\subsection{\hspace{-0.5cm} {\Large \textcolor{darkblue}{\textbf{\ipa{v̩˧ko˧}}}}\hspace{0.5cm}[\kern2pt{\textcolor{darkblue}{\textbf{\ipa{v̩˩ko˩˥}}}}\kern2pt]} \hypertarget{v\string_=\string_Mko\string_M1}{}
\markboth{\textcolor{darkblue}{\textbf{\ipa{v̩˧ko˧}}}}{}
\textcolor{teal}{\mytextsc{noun}} \hspace{4pt} Tone: M.
\textcolor{Sepia}{\selectlanguage{english}Tortoise.} \zh{乌龟(汉语借词)。} 
\lhead{\firstmark}
\rhead{\botmark}

\subsection{\hspace{-0.5cm} {\Large \textcolor{darkblue}{\textbf{\ipa{v̩˧lɑ˩-ʝi˩}}}}\hspace{0.5cm}[\kern2pt{\textcolor{darkblue}{\textbf{\ipa{xxxx non-correspondance entre le nombre de morphèmes et le nombre de tons de morphèmes}}}}\kern2pt]} \hypertarget{v\string_=\string_MlA\string_B-j££i\string_B1}{}
\markboth{\textcolor{darkblue}{\textbf{\ipa{v̩˧lɑ˩-ʝi˩}}}}{}
\textcolor{teal}{\mytextsc{verb}} \hspace{4pt} Tone: L\#-.
\textcolor{Sepia}{\selectlanguage{english}To trade, to do business.} \zh{做生意。} 
\lhead{\firstmark}
\rhead{\botmark}

\subsection{\hspace{-0.5cm} {\Large \textcolor{darkblue}{\textbf{\ipa{v̩˧lɑ˩-ʝi˩-hĩ˩-hĩ˩}}}}\hspace{0.5cm}[\kern2pt{\textcolor{darkblue}{\textbf{\ipa{xxxx non-correspondance entre le nombre de morphèmes et le nombre de tons de morphèmes}}}}\kern2pt]} \hypertarget{v\string_=\string_MlA\string_B-j££i\string_B-hi\string_~\string_B-hi\string_~\string_B1}{}
\markboth{\textcolor{darkblue}{\textbf{\ipa{v̩˧lɑ˩-ʝi˩-hĩ˩-hĩ˩}}}}{}
\textcolor{teal}{\mytextsc{noun}} \hspace{4pt} Tone: L\#-.
\textcolor{Sepia}{\selectlanguage{english}Merchant.} \zh{商人。}  ¶ \textcolor{darkblue}{\textbf{\ipa{v̩˧lɑ˩-ʝi˩-hĩ˩}}} \textcolor{Sepia}{\selectlanguage{english}merchant} \zh{商人}  
 \zh{量词}: \textcolor{darkblue}{\textbf{\ipa{v̩˧}}}  \mytextsc{clf}: \textcolor{darkblue}{\textbf{\ipa{v̩˧}}} 
\lhead{\firstmark}
\rhead{\botmark}

\subsection{\hspace{-0.5cm} {\Large \textcolor{darkblue}{\textbf{\ipa{v̩˧mi\#˥}}}}\hspace{0.5cm}[\kern2pt{\textcolor{darkblue}{\textbf{\ipa{xxxx non-correspondance entre le nombre de morphèmes et le nombre de tons de morphèmes}}}}\kern2pt]} \hypertarget{v\string_=\string_Mmi\#\string_T1}{}
\markboth{\textcolor{darkblue}{\textbf{\ipa{v̩˧mi\#˥}}}}{}
\textcolor{teal}{\mytextsc{noun}} \hspace{4pt} Tone: \#H.
\textcolor{Sepia}{\selectlanguage{english}Large cooking pot.} \zh{大锅。} 
\lhead{\firstmark}
\rhead{\botmark}

\subsection{\hspace{-0.5cm} {\Large \textcolor{darkblue}{\textbf{\ipa{v̩˩tsʰɤ˧˥}}}}\hspace{0.5cm}[\kern2pt{\textcolor{darkblue}{\textbf{\ipa{v̩˧tsʰɤ˧}}}}\kern2pt]} \hypertarget{v\string_=\string_Bts\string_h7\string_M\string_T1}{}
\markboth{\textcolor{darkblue}{\textbf{\ipa{v̩˩tsʰɤ˧˥}}}}{}
\textcolor{teal}{\mytextsc{noun}} \hspace{4pt} Tone: LM+MH\#.
\textcolor{Sepia}{\selectlanguage{english}Vegetables (in a broad sense including fresh vegetables and picked vegetables).} \zh{蔬菜。}  ¶ \textcolor{darkblue}{\textbf{\ipa{[Housebuilding2] v˩tsʰɤ˧-tsʰɑ˧nɑ˥}}} \textcolor{Sepia}{\selectlanguage{english}fresh vegetables. Literally 'dark vegetables'. This does not refer to one species in particular, but to all sorts of fresh vegetables, as opposed to pickled vegetables. In the process of preserving vegetables, their original darker colours tend to fade away.} \zh{新鲜蔬菜。直译:‘绿油油的青菜’。指的不是某种具体的青菜,而是任何新鲜蔬菜,分别于酸菜。制造酸菜的过程中,蔬菜(萝卜等等)褪色:失去原来的深色。}  
 \zh{量词}: \textcolor{darkblue}{\textbf{\ipa{po˧}}}  \mytextsc{clf}: \textcolor{darkblue}{\textbf{\ipa{po˧}}} 
\lhead{\firstmark}
\rhead{\botmark}

\subsection{\hspace{-0.5cm} {\Large \textcolor{darkblue}{\textbf{\ipa{v̩˩tsʰɤ˧-bv̩\#˥}}}}\hspace{0.5cm}[\kern2pt{\textcolor{darkblue}{\textbf{\ipa{xxxx non-correspondance entre le nombre de morphèmes et le nombre de tons de morphèmes}}}}\kern2pt]} \hypertarget{v\string_=\string_Bts\string_h7\string_M-bv\string_=\#\string_T1}{}
\markboth{\textcolor{darkblue}{\textbf{\ipa{v̩˩tsʰɤ˧-bv̩\#˥}}}}{}
\textcolor{teal}{\mytextsc{noun}} \hspace{4pt} Tone: LM+\#H.
\textcolor{Sepia}{\selectlanguage{english}Ladybug, ladybird.} \zh{瓢虫。}  \zh{量词}: \textcolor{darkblue}{\textbf{\ipa{mi˩}}}  \mytextsc{clf}: \textcolor{darkblue}{\textbf{\ipa{mi˩}}} 
\lhead{\firstmark}
\rhead{\botmark}

\subsection{\hspace{-0.5cm} {\Large \textcolor{darkblue}{\textbf{\ipa{v̩˩tsʰɤ˧-pʰv̩˥}}}}\hspace{0.5cm}[\kern2pt{\textcolor{darkblue}{\textbf{\ipa{xxxx non-correspondance entre le nombre de morphèmes et le nombre de tons de morphèmes}}}}\kern2pt]} \hypertarget{v\string_=\string_Bts\string_h7\string_M-p\string_hv\string_=\string_T1}{}
\markboth{\textcolor{darkblue}{\textbf{\ipa{v̩˩tsʰɤ˧-pʰv̩˥}}}}{}
\textcolor{teal}{\mytextsc{noun}} \hspace{4pt} Tone: LM+H\#.
\textcolor{Sepia}{\selectlanguage{english}Chinese cabbage.} \zh{白菜。}  \zh{量词}: \textcolor{darkblue}{\textbf{\ipa{po˧}}}  \mytextsc{clf}: \textcolor{darkblue}{\textbf{\ipa{po˧}}} \textit{See:} \hyperlink{}{\textcolor{darkblue}{\textbf{\ipa{tsʰæ˧pʰv˧˥}}}} 
\lhead{\firstmark}
\rhead{\botmark}

\subsection{\hspace{-0.5cm} {\Large \textcolor{darkblue}{\textbf{\ipa{v̩˩tsʰɤ˧-v̩˥ɲi˩}}}}\hspace{0.5cm}[\kern2pt{\textcolor{darkblue}{\textbf{\ipa{xxxx non-correspondance entre le nombre de morphèmes et le nombre de tons de morphèmes}}}}\kern2pt]} \hypertarget{v\string_=\string_Bts\string_h7\string_M-v\string_=\string_TJi\string_B1}{}
\markboth{\textcolor{darkblue}{\textbf{\ipa{v̩˩tsʰɤ˧-v̩˥ɲi˩}}}}{}
\textcolor{teal}{\mytextsc{noun}} \hspace{4pt} Tone: LM+\#H-.
\textcolor{Sepia}{\selectlanguage{english}Vegetables.} \zh{蔬菜。}  \zh{量词}: \textcolor{darkblue}{\textbf{\ipa{qɑ˩}}}  \mytextsc{clf}: \textcolor{darkblue}{\textbf{\ipa{qɑ˩}}} 
\lhead{\firstmark}
\rhead{\botmark}

\subsection{\hspace{-0.5cm} {\Large \textcolor{darkblue}{\textbf{\ipa{v̩˧zo\#˥}}}}\hspace{0.5cm}[\kern2pt{\textcolor{darkblue}{\textbf{\ipa{xxxx non-correspondance entre le nombre de morphèmes et le nombre de tons de morphèmes}}}}\kern2pt]} \hypertarget{v\string_=\string_Mzo\#\string_T1}{}
\markboth{\textcolor{darkblue}{\textbf{\ipa{v̩˧zo\#˥}}}}{}
\textcolor{teal}{\mytextsc{noun}} \hspace{4pt} Tone: \#H.
\textcolor{Sepia}{\selectlanguage{english}Small cooking pot.} \zh{小锅。} 
\lhead{\firstmark}
\rhead{\botmark}

\subsection{\hspace{-0.5cm} {\Large \textcolor{darkblue}{\textbf{\ipa{v̩˧\textasciitilde{}v̩˧\textsubscript{a}}}}}\hspace{0.5cm}[\kern2pt{\textcolor{darkblue}{\textbf{\ipa{v̩˧v̩˧}}}}\kern2pt]} \hypertarget{v\string_=\string_M~v\string_=\string_Ma1}{}
\markboth{\textcolor{darkblue}{\textbf{\ipa{v̩˧\textasciitilde{}v̩˧\textsubscript{a}}}}}{}
\textcolor{teal}{\mytextsc{verb}} \hspace{4pt} Tone: M\textsubscript{a}.
\textcolor{Sepia}{\selectlanguage{english}To chew; to chew the cud.} \zh{嚼。}  ¶ \textcolor{darkblue}{\textbf{\ipa{le˧-v̩˧\textasciitilde{}v̩˧ +ze˩}}} \textcolor{Sepia}{\selectlanguage{english}\mytextsc{accomp}} \zh{\mytextsc{accomp}}  
 ¶ \textcolor{darkblue}{\textbf{\ipa{le˧-wo˧ v̩˧\textasciitilde{}v̩˧}}} \textcolor{Sepia}{\selectlanguage{english}to chew the cud} \zh{反刍}  

\lhead{\firstmark}
\rhead{\botmark}

\newpage
\section*{\centering- \textcolor{darkblue}{\textbf{\ipa{w}}} \textcolor{darkblue}{\textbf{\ipa{wɑ}}} \textcolor{darkblue}{\textbf{\ipa{wæ}}} \textcolor{darkblue}{\textbf{\ipa{wɤ}}} \textcolor{darkblue}{\textbf{\ipa{wo}}} \textcolor{darkblue}{\textbf{\ipa{wɤ̃}}} -}
\subsection{\hspace{-0.5cm} {\Large \textcolor{darkblue}{\textbf{\ipa{wɤ˧}}} \textsubscript{1}}\hspace{0.5cm}[\kern2pt{\textcolor{darkblue}{\textbf{\ipa{wɤ˩˥}}}}\kern2pt]} \hypertarget{w7\string_M1}{}
\markboth{\textcolor{darkblue}{\textbf{\ipa{wɤ˧}}} \textsubscript{1}}{}
\textcolor{teal}{\mytextsc{noun}} \hspace{4pt} Tone: M.
\textcolor{Sepia}{\selectlanguage{english}Serf, slave (lowest of the 3 ranks in feudal society).} \zh{奴隶,农奴。音译:“俄”。}  \zh{量词}: \textcolor{darkblue}{\textbf{\ipa{v̩˧}}}  \mytextsc{clf}: \textcolor{darkblue}{\textbf{\ipa{v̩˧}}} 
\lhead{\firstmark}
\rhead{\botmark}

\subsection{\hspace{-0.5cm} {\Large \textcolor{darkblue}{\textbf{\ipa{wɤ˧}}} \textsubscript{2}}\hspace{0.5cm}[\kern2pt{\textcolor{darkblue}{\textbf{\ipa{wɤ˥}}}}\kern2pt]} \hypertarget{w7\string_M2}{}
\markboth{\textcolor{darkblue}{\textbf{\ipa{wɤ˧}}} \textsubscript{2}}{}
\textcolor{teal}{\mytextsc{discourse}} \textcolor{teal}{\mytextsc{particle}} \hspace{4pt} Tone: M.
\textcolor{Sepia}{\selectlanguage{english}Final particle conveying exclamation, with a nuance of obviousness.} \zh{句尾助词:吧、呗。} 
\lhead{\firstmark}
\rhead{\botmark}

\subsection{\hspace{-0.5cm} {\Large \textcolor{darkblue}{\textbf{\ipa{wɤ˩\textsubscript{a}}}}}\hspace{0.5cm}[\kern2pt{\textcolor{darkblue}{\textbf{\ipa{wɤ˥}}}}\kern2pt]} \hypertarget{w7\string_Ba1}{}
\markboth{\textcolor{darkblue}{\textbf{\ipa{wɤ˩\textsubscript{a}}}}}{}
\textcolor{teal}{\mytextsc{verb}} \hspace{4pt} Tone: L\textsubscript{a}.
\textcolor{Sepia}{\selectlanguage{english}To depend on.} \zh{依赖。}  ¶ \textcolor{darkblue}{\textbf{\ipa{hĩ˧-bi˥ | wɤ˩-mɤ˩-bi˩˥!}}} \textcolor{Sepia}{\selectlanguage{english}One should not depend on others!} \zh{不要依赖别人!}  
 ¶ \textcolor{darkblue}{\textbf{\ipa{hĩ˧-bi˥ | wɤ˩-v̩˩-tʰv̩˩˥!}}} \textcolor{Sepia}{\selectlanguage{english}(Whether one wants or not) one depends on others (in some respect or other)!} \zh{(无论如何)人都会依靠别人的!(意思是:人不能完全独立,人活在人间就会或多或少需要依靠别人。)}  

\lhead{\firstmark}
\rhead{\botmark}

\subsection{\hspace{-0.5cm} {\Large \textcolor{darkblue}{\textbf{\ipa{wɤ˩\textsubscript{b}}}}}\hspace{0.5cm}[\kern2pt{\textcolor{darkblue}{\textbf{\ipa{wɤ˩˥}}}}\kern2pt]} \hypertarget{w7\string_Bb1}{}
\markboth{\textcolor{darkblue}{\textbf{\ipa{wɤ˩\textsubscript{b}}}}}{}
\textcolor{teal}{\mytextsc{classifier}} \hspace{4pt} Tone: L\textsubscript{b}.
\textcolor{Sepia}{\selectlanguage{english}Load, charge, weight.} \zh{量词:担,负荷。}  ¶ \textcolor{darkblue}{\textbf{\ipa{ɖɯ˧-wɤ˩ pɤ˩\textasciitilde{}pɤ˩ |}}} \textcolor{Sepia}{\selectlanguage{english}to carry a load} \zh{背一担}  
 ¶ \textcolor{darkblue}{\textbf{\ipa{ɖɯ˧-wɤ˩, | ɖɯ˧-wɤ˩ | le˧-kʰɯ˩\textasciitilde{}kʰɯ˩ | tʰi˧-tɕɯ˥ |}}} \textcolor{Sepia}{\selectlanguage{english}to pile up loads, one after the other} \zh{将驮的大包堆起来}  

\lhead{\firstmark}
\rhead{\botmark}

\subsection{\hspace{-0.5cm} {\Large \textcolor{darkblue}{\textbf{\ipa{wɤ˩\textasciitilde{}wɤ˩}}}}\hspace{0.5cm}[\kern2pt{\textcolor{darkblue}{\textbf{\ipa{wɤ˩wɤ˥}}}}\kern2pt]} \hypertarget{w7\string_B~w7\string_B1}{}
\markboth{\textcolor{darkblue}{\textbf{\ipa{wɤ˩\textasciitilde{}wɤ˩}}}}{}
\textcolor{teal}{\mytextsc{verb}} \hspace{4pt} Tone: L.
\textcolor{Sepia}{\selectlanguage{english}To detour past, to bypass.} \zh{绕过。}  ¶ \textcolor{darkblue}{\textbf{\ipa{le˧-wɤ˩-ze˩}}} \textcolor{Sepia}{\selectlanguage{english}\mytextsc{accomp} \string_ \mytextsc{pfv}} \zh{绕了}  
 ¶ \textcolor{darkblue}{\textbf{\ipa{ɖɯ˧-wɤ˩\textasciitilde{}wɤ˩-ɻ̍˩}}} \textcolor{Sepia}{\selectlanguage{english}\mytextsc{delimitative} \string_ \mytextsc{red} \mytextsc{inceptive}} \zh{绕一绕}  
 ¶ \textcolor{darkblue}{\textbf{\ipa{[PHONO] wɤ˩\textasciitilde{}wɤ˩ bi˩˥}}} \textcolor{Sepia}{\selectlanguage{english}\mytextsc{imm}\string_fut} \zh{\mytextsc{imm}\string_fut}  
 ¶ \textcolor{darkblue}{\textbf{\ipa{[PHONO] wɤ˩\textasciitilde{}wɤ˩-ze˥}}} \textcolor{Sepia}{\selectlanguage{english}\mytextsc{pfv}} \zh{绕了}  
 ¶ \textcolor{darkblue}{\textbf{\ipa{le˧-wɤ˩\textasciitilde{}wɤ˩ | le˧-se˥}}} \textcolor{Sepia}{\selectlanguage{english}to bypass on foot; to walk past, bypassing (a certain place)} \zh{走路绕过}  

\lhead{\firstmark}
\rhead{\botmark}

\subsection{\hspace{-0.5cm} {\Large \textcolor{darkblue}{\textbf{\ipa{wɤ˩˥}}}}\hspace{0.5cm}[\kern2pt{\textcolor{darkblue}{\textbf{\ipa{wɤ˥}}}}\kern2pt]} \hypertarget{w7\string_B\string_T1}{}
\markboth{\textcolor{darkblue}{\textbf{\ipa{wɤ˩˥}}}}{}
\textcolor{teal}{\mytextsc{adverb(ial)}} \hspace{4pt} Tone: LM? LH?.
\textcolor{Sepia}{\selectlanguage{english}Again; also.} \zh{又,再。}  ¶ \textcolor{darkblue}{\textbf{\ipa{wɤ˩˥ | ɖɯ˧-ʂɯ˩}}} \textcolor{Sepia}{\selectlanguage{english}once again, once more, one more time} \zh{再一次、又一次}  

\lhead{\firstmark}
\rhead{\botmark}

\subsection{\hspace{-0.5cm} {\Large \textcolor{darkblue}{\textbf{\ipa{wo˥}}}}\hspace{0.5cm}[\kern2pt{\textcolor{darkblue}{\textbf{\ipa{wo˧˥}}}}\kern2pt]} \hypertarget{wo\string_T1}{}
\markboth{\textcolor{darkblue}{\textbf{\ipa{wo˥}}}}{}
\textcolor{teal}{\mytextsc{adjective}} \hspace{4pt} Tone: H.
\textcolor{Sepia}{\selectlanguage{english}Hard, solid, resilient.} \zh{硬,坚硬,结实。}  ¶ \textcolor{darkblue}{\textbf{\ipa{le˧-wo˥-ze˩}}} \textcolor{Sepia}{\selectlanguage{english}\mytextsc{accomp} \string_ \mytextsc{pfv}: it hardened} \zh{硬了}  

\lhead{\firstmark}
\rhead{\botmark}

\subsection{\hspace{-0.5cm} {\Large \textcolor{darkblue}{\textbf{\ipa{wo˩\textsubscript{b}}}}}\hspace{0.5cm}[\kern2pt{\textcolor{darkblue}{\textbf{\ipa{wo˩˥}}}}\kern2pt]} \hypertarget{wo\string_Bb1}{}
\markboth{\textcolor{darkblue}{\textbf{\ipa{wo˩\textsubscript{b}}}}}{}
\textcolor{teal}{\mytextsc{classifier}} \hspace{4pt} Tone: L\textsubscript{b}.
\textcolor{Sepia}{\selectlanguage{english}Classifier for teams of oxen. In Yongning, the ard is drawn by two oxen, or two small water buffaloes, or one strong water buffalo.} \zh{量词:牛(一架)。}  ¶ \textcolor{darkblue}{\textbf{\ipa{dʑi˧mi˧ | ɲi˧-pʰo˧˥, | ɖɯ˧-wo˩!}}} \textcolor{Sepia}{\selectlanguage{english}Two water buffaloes make up one team!} \zh{两头水牛,等于一架!}  

\lhead{\firstmark}
\rhead{\botmark}

\subsection{\hspace{-0.5cm} {\Large \textcolor{darkblue}{\textbf{\ipa{wo˩kɤ\#˥}}}}\hspace{0.5cm}[\kern2pt{\textcolor{darkblue}{\textbf{\ipa{wo˩kɤ˩˥}}}}\kern2pt]} \hypertarget{wo\string_Bk7\#\string_T1}{}
\markboth{\textcolor{darkblue}{\textbf{\ipa{wo˩kɤ\#˥}}}}{}
\textcolor{teal}{\mytextsc{noun}} \hspace{4pt} Tone: LM+\#H.
\textcolor{Sepia}{\selectlanguage{english}Swing.} \zh{秋千(鞦韆)。}  ¶ \textcolor{darkblue}{\textbf{\ipa{wo˩kɤ˧-tsɑ˧-di˧˥}}} \textcolor{Sepia}{\selectlanguage{english}same meaning: swing} \zh{同上:秋千}  
 ¶ \textcolor{darkblue}{\textbf{\ipa{wo˩kɤ˧ tsɑ˧˥}}} \textcolor{Sepia}{\selectlanguage{english}same meaning: swing} \zh{同上:秋千}  
 \zh{量词}: \textcolor{darkblue}{\textbf{\ipa{nɑ˧}}}  \mytextsc{clf}: \textcolor{darkblue}{\textbf{\ipa{nɑ˧}}} 
\lhead{\firstmark}
\rhead{\botmark}

\subsection{\hspace{-0.5cm} {\Large \textcolor{darkblue}{\textbf{\ipa{wo˧˥}}}}\hspace{0.5cm}[\kern2pt{\textcolor{darkblue}{\textbf{\ipa{wo˩˥}}}}\kern2pt]} \hypertarget{wo\string_M\string_T1}{}
\markboth{\textcolor{darkblue}{\textbf{\ipa{wo˧˥}}}}{}
\textcolor{teal}{\mytextsc{verb}} \hspace{4pt} Tone: MH.
\textcolor{Sepia}{\selectlanguage{english}To do (something) over again.} \zh{重新做、再来做。}  ¶ \textcolor{darkblue}{\textbf{\ipa{le˧-wo˧ ʐwɤ˧˥}}} \textcolor{Sepia}{\selectlanguage{english}to answer} \zh{回答}  
 ¶ \textcolor{darkblue}{\textbf{\ipa{le˧-wo˧-ɻ̍˥}}} \textcolor{Sepia}{\selectlanguage{english}to turn around (e.g. in order to look back)} \zh{转身}  
 ¶ \textcolor{darkblue}{\textbf{\ipa{le˧-wo˧ li˥}}} \textcolor{Sepia}{\selectlanguage{english}to look back} \zh{往后看}  
 ¶ \textcolor{darkblue}{\textbf{\ipa{lə-˧wo˧ tʰo˥-tɕo˩}}} \textcolor{Sepia}{\selectlanguage{english}to turn around (e.g. in order to look back)} \zh{转身}  
 ¶ \textcolor{darkblue}{\textbf{\ipa{le˧-wo˧-tɕo˥!}}} \textcolor{Sepia}{\selectlanguage{english}Turn around! (Said to a baby who is about to get down a bed head first)} \zh{转身!(婴儿爬下床,头朝下。奶奶告诉她:要先转身)}  
 ¶ \textcolor{darkblue}{\textbf{\ipa{le˧-wo˧˥ | le˧-hɯ˩}}} \textcolor{Sepia}{\selectlanguage{english}has gone back, went back} \zh{回去了}  

\lhead{\firstmark}
\rhead{\botmark}

\subsection{\hspace{-0.5cm} {\Large \textcolor{darkblue}{\textbf{\ipa{wo˩˥}}}}\hspace{0.5cm}[\kern2pt{\textcolor{darkblue}{\textbf{\ipa{wo˥}}}}\kern2pt]} \hypertarget{wo\string_B\string_T1}{}
\markboth{\textcolor{darkblue}{\textbf{\ipa{wo˩˥}}}}{}
\textcolor{teal}{\mytextsc{noun}} \hspace{4pt} Tone: LH.
\textcolor{Sepia}{\selectlanguage{english}Turnip leaves; they used to be eaten as a vegetable.} \zh{圆根的叶子。}  ¶ \textcolor{darkblue}{\textbf{\ipa{wo˩bɤ˧˥}}} \textcolor{Sepia}{\selectlanguage{english}same meaning: turnip leaves} \zh{同上:圆根叶子}  
 ¶ \textcolor{darkblue}{\textbf{\ipa{[Housebuilding2] wo˩-v˥tsʰɤ˩}}} \textcolor{Sepia}{\selectlanguage{english}same meaning: turnip leaves; literally 'turnip leaves vegetable', emphasizing the fact that they are used as a vegetable: as an ingredient in a recipe} \zh{同上:圆根叶子}  
 ¶ \textcolor{darkblue}{\textbf{\ipa{[Housebuilding2] wo˩-tɕæ˩ɻæ˥}}} \textcolor{Sepia}{\selectlanguage{english}pickled turnip leaves} \zh{圆根叶子酸菜}  

\lhead{\firstmark}
\rhead{\botmark}

\newpage
\section*{\centering- \textcolor{darkblue}{\textbf{\ipa{w̃}}} \textcolor{darkblue}{\textbf{\ipa{w̃æ}}} -}
\subsection{\hspace{-0.5cm} {\Large \textcolor{darkblue}{\textbf{\ipa{w̃æ˧}}}}\hspace{0.5cm}[\kern2pt{\textcolor{darkblue}{\textbf{\ipa{w̃æ˥}}}}\kern2pt]} \hypertarget{w\string_~\{\string_M1}{}
\markboth{\textcolor{darkblue}{\textbf{\ipa{w̃æ˧}}}}{}
\textcolor{teal}{\mytextsc{verb}} \hspace{4pt} Tone: M intrans.
\textcolor{Sepia}{\selectlanguage{english}To swell, to inflate (e.g. the belly is swollen).} \zh{肿,膨胀,(肚子)胀。}  ¶ \textcolor{darkblue}{\textbf{\ipa{ɻ̍˧tɑ˧ w̃æ˧ (-ze˧)}}} \textcolor{Sepia}{\selectlanguage{english}glands are swollen} \zh{淋巴结肿了}  
 ¶ \textcolor{darkblue}{\textbf{\ipa{tso˧\textasciitilde{}tso˧ w̃æ˩}}} \textcolor{Sepia}{\selectlanguage{english}something has swollen} \zh{东西膨胀了}  

\lhead{\firstmark}
\rhead{\botmark}

\newpage
\section*{\centering- \textcolor{darkblue}{\textbf{\ipa{z}}} -}
\subsection{\hspace{-0.5cm} {\Large \textcolor{darkblue}{\textbf{\ipa{zɑ˥}}}}\hspace{0.5cm}[\kern2pt{\textcolor{darkblue}{\textbf{\ipa{xxxx groupe tonal entier sans aucun ton}}}}\kern2pt]} \hypertarget{zA\string_T1}{}
\markboth{\textcolor{darkblue}{\textbf{\ipa{zɑ˥}}}}{}
\textcolor{teal}{\mytextsc{adjective}} \hspace{4pt} Tone: H.
\textcolor{Sepia}{\selectlanguage{english}Restricted to, limited to.} \zh{仅仅。}  ¶ \textcolor{darkblue}{\textbf{\ipa{ʁwɤ˧-qo˧-ɳɯ˧-lɑ˧ mɤ˧-zɑ˥ (…)}}} \textcolor{Sepia}{\selectlanguage{english}not only the people from the village} \zh{不仅有村子里的人}  

\lhead{\firstmark}
\rhead{\botmark}

\subsection{\hspace{-0.5cm} {\Large \textcolor{darkblue}{\textbf{\ipa{zɑ˧ɭɯ˧}}}}\hspace{0.5cm}[\kern2pt{\textcolor{darkblue}{\textbf{\ipa{xxxx non-correspondance entre le nombre de morphèmes et le nombre de tons de morphèmes}}}}\kern2pt]} \hypertarget{zA\string_Ml\string_RM\string_M1}{}
\markboth{\textcolor{darkblue}{\textbf{\ipa{zɑ˧ɭɯ˧}}}}{}
\textcolor{teal}{\mytextsc{noun}} \hspace{4pt} Tone: M.
\textcolor{Sepia}{\selectlanguage{english}Barrow, castrated male pig, neutered pig.} \zh{阉猪。}  \zh{量词}: \textcolor{darkblue}{\textbf{\ipa{pʰo˧˥}}} \textcolor{darkblue}{\textbf{\ipa{v̩˧}}}  \mytextsc{clf}: \textcolor{darkblue}{\textbf{\ipa{pʰo˧˥}}} \textcolor{darkblue}{\textbf{\ipa{v̩˧}}} 
\lhead{\firstmark}
\rhead{\botmark}

\subsection{\hspace{-0.5cm} {\Large \textcolor{darkblue}{\textbf{\ipa{zɑ˧zɑ˧}}}}\hspace{0.5cm}[\kern2pt{\textcolor{darkblue}{\textbf{\ipa{zɑ˧zɑ˧}}}}\kern2pt]} \hypertarget{zA\string_MzA\string_M1}{}
\markboth{\textcolor{darkblue}{\textbf{\ipa{zɑ˧zɑ˧}}}}{}
\textcolor{teal}{\mytextsc{adjective}} \hspace{4pt} Tone: M.
\textcolor{Sepia}{\selectlanguage{english}Careful.} \zh{细心、细致。} 
\lhead{\firstmark}
\rhead{\botmark}

\subsection{\hspace{-0.5cm} {\Large \textcolor{darkblue}{\textbf{\ipa{zɑ˩\textsubscript{a}}}}}\hspace{0.5cm}[\kern2pt{\textcolor{darkblue}{\textbf{\ipa{zɑ˥}}}}\kern2pt]} \hypertarget{zA\string_Ba1}{}
\markboth{\textcolor{darkblue}{\textbf{\ipa{zɑ˩\textsubscript{a}}}}}{}
\textcolor{teal}{\mytextsc{verb}} \hspace{4pt} Tone: L\textsubscript{a}.
\textcolor{Sepia}{\selectlanguage{english}To go downward (a mountain), to descend.} \zh{下(山……)。}  ¶ \textcolor{darkblue}{\textbf{\ipa{ʁwɤ˩ zɑ˩˥}}} \textcolor{Sepia}{\selectlanguage{english}to go down the mountain} \zh{下山}  
 ¶ \textcolor{darkblue}{\textbf{\ipa{mɤ˧-zɑ˩-sɯ˩}}} \textcolor{Sepia}{\selectlanguage{english}not to go down yet} \zh{还没下来}  
 ¶ \textcolor{darkblue}{\textbf{\ipa{ɖɯ˧-zɑ˧\textasciitilde{}zɑ˥-ɻ̍˩}}} \textcolor{Sepia}{\selectlanguage{english}\mytextsc{delimitative} \string_ \mytextsc{red} \mytextsc{inceptive}} \zh{下来一下}  

\lhead{\firstmark}
\rhead{\botmark}

\subsection{\hspace{-0.5cm} {\Large \textcolor{darkblue}{\textbf{\ipa{zɑ˩-bɑ˧lɑ˩}}}}\hspace{0.5cm}[\kern2pt{\textcolor{darkblue}{\textbf{\ipa{xxxx non-correspondance entre le nombre de morphèmes et le nombre de tons de morphèmes}}}}\kern2pt]} \hypertarget{zA\string_B-bA\string_MlA\string_B1}{}
\markboth{\textcolor{darkblue}{\textbf{\ipa{zɑ˩-bɑ˧lɑ˩}}}}{}
\textcolor{teal}{\mytextsc{noun}} \hspace{4pt} Tone: L-L\#.
\ding{202} \textcolor{Sepia}{\selectlanguage{english}Religious painting (thangka) on wood, on the wall next to the hearth.} \zh{火塘旁边墙上的壁画(唐卡:内容来自藏传佛教)。} \ding{203} \textcolor{Sepia}{\selectlanguage{english}Divinity of fire, of the hearth, and of the house.} \zh{火,火塘与家的神。} 
\lhead{\firstmark}
\rhead{\botmark}

\subsection{\hspace{-0.5cm} {\Large \textcolor{darkblue}{\textbf{\ipa{zɑ˩ɲi˥-ʂɤ˩}}}}\hspace{0.5cm}[\kern2pt{\textcolor{darkblue}{\textbf{\ipa{xxxx non-correspondance entre le nombre de morphèmes et le nombre de tons de morphèmes}}}}\kern2pt]} \hypertarget{zA\string_BJi\string_T-s`7\string_B1}{}
\markboth{\textcolor{darkblue}{\textbf{\ipa{zɑ˩ɲi˥-ʂɤ˩}}}}{}
\textcolor{teal}{\mytextsc{noun}} \hspace{4pt} Tone: LH-.
\textcolor{Sepia}{\selectlanguage{english}Vampire: a demon of human shape (the size of a large person), who feeds on blood.} \zh{吸血鬼。} 
\lhead{\firstmark}
\rhead{\botmark}

\subsection{\hspace{-0.5cm} {\Large \textcolor{darkblue}{\textbf{\ipa{‑ze˧}}}}\hspace{0.5cm}[\kern2pt{\textcolor{darkblue}{\textbf{\ipa{ze˥}}}}\kern2pt]} \hypertarget{‑ze\string_M1}{}
\markboth{\textcolor{darkblue}{\textbf{\ipa{‑ze˧}}}}{}
\textcolor{teal}{\mytextsc{suffix}} \hspace{4pt} Tone: M.
\textcolor{Sepia}{\selectlanguage{english}Perfective, \mytextsc{pfv}.} \zh{\mytextsc{整体体。}} 
\lhead{\firstmark}
\rhead{\botmark}

\subsection{\hspace{-0.5cm} {\Large \textcolor{darkblue}{\textbf{\ipa{ze˩}}}}\hspace{0.5cm}[\kern2pt{\textcolor{darkblue}{\textbf{\ipa{ze˩˥}}}}\kern2pt]} \hypertarget{ze\string_B1}{}
\markboth{\textcolor{darkblue}{\textbf{\ipa{ze˩}}}}{}
\textcolor{teal}{\mytextsc{pronoun/pronominal}} \hspace{4pt} Tone: L.
\textcolor{Sepia}{\selectlanguage{english}Which.} \zh{哪。} 
\lhead{\firstmark}
\rhead{\botmark}

\subsection{\hspace{-0.5cm} {\Large \textcolor{darkblue}{\textbf{\ipa{ze˩bæ˧}}}}\hspace{0.5cm}[\kern2pt{\textcolor{darkblue}{\textbf{\ipa{ze˩bæ˥}}}}\kern2pt]} \hypertarget{ze\string_Bb\{\string_M1}{}
\markboth{\textcolor{darkblue}{\textbf{\ipa{ze˩bæ˧}}}}{}
\textcolor{teal}{\mytextsc{pronoun/pronominal}} \hspace{4pt} Tone: LM.
\textcolor{Sepia}{\selectlanguage{english}Which; which kind.} \zh{哪,哪个 (哪个碗),哪一种。}  ¶ \textcolor{darkblue}{\textbf{\ipa{ze˩bæ˧ ɲi˥?}}} \textcolor{Sepia}{\selectlanguage{english}Which one is it? / Which kind is it?} \zh{是哪个?是哪一样?}  

\lhead{\firstmark}
\rhead{\botmark}

\subsection{\hspace{-0.5cm} {\Large \textcolor{darkblue}{\textbf{\ipa{ze˩bæ˩}}}}\hspace{0.5cm}[\kern2pt{\textcolor{darkblue}{\textbf{\ipa{ze˩bæ˩˥}}}}\kern2pt]} \hypertarget{ze\string_Bb\{\string_B1}{}
\markboth{\textcolor{darkblue}{\textbf{\ipa{ze˩bæ˩}}}}{}
\textcolor{teal}{\mytextsc{noun}} \hspace{4pt} Tone: L.
\textcolor{Sepia}{\selectlanguage{english}Flash of lightning, thunderbolt.} \zh{闪电、打闪电、霹雷。}  ¶ \textcolor{darkblue}{\textbf{\ipa{ze˩bæ˩-ze˥!}}} \textcolor{Sepia}{\selectlanguage{english}There has been a flash of lightning!} \zh{打闪电了!}  
 ¶ \textcolor{darkblue}{\textbf{\ipa{ze˩bæ˩˥ | -dʑo˩!}}} \textcolor{Sepia}{\selectlanguage{english}There are flashes of lightning!} \zh{打着闪电!}  
 \zh{量词}: \textcolor{darkblue}{\textbf{\ipa{bæ˩}}}  \mytextsc{clf}: \textcolor{darkblue}{\textbf{\ipa{bæ˩}}} 
\lhead{\firstmark}
\rhead{\botmark}

\subsection{\hspace{-0.5cm} {\Large \textcolor{darkblue}{\textbf{\ipa{ze˩gɤ˧}}}}\hspace{0.5cm}[\kern2pt{\textcolor{darkblue}{\textbf{\ipa{ze˩gɤ˥}}}}\kern2pt]} \hypertarget{ze\string_Bg7\string_M1}{}
\markboth{\textcolor{darkblue}{\textbf{\ipa{ze˩gɤ˧}}}}{}
\textcolor{teal}{\mytextsc{pronoun/pronominal}} \hspace{4pt} Tone: LM.
\textcolor{Sepia}{\selectlanguage{english}At which place, where.} \zh{哪里,什么地方。} 
\lhead{\firstmark}
\rhead{\botmark}

\subsection{\hspace{-0.5cm} {\Large \textcolor{darkblue}{\textbf{\ipa{ze˩mi˩}}}}\hspace{0.5cm}[\kern2pt{\textcolor{darkblue}{\textbf{\ipa{ze˩mi˩˥}}}}\kern2pt]} \hypertarget{ze\string_Bmi\string_B1}{}
\markboth{\textcolor{darkblue}{\textbf{\ipa{ze˩mi˩}}}}{}
\textcolor{teal}{\mytextsc{noun}} \hspace{4pt} Tone: L.
\textcolor{Sepia}{\selectlanguage{english}Niece.} \zh{甥女(姐妹的女儿)。}  \zh{量词}: \textcolor{darkblue}{\textbf{\ipa{v̩˧}}}  \mytextsc{clf}: \textcolor{darkblue}{\textbf{\ipa{v̩˧}}} 
\lhead{\firstmark}
\rhead{\botmark}

\subsection{\hspace{-0.5cm} {\Large \textcolor{darkblue}{\textbf{\ipa{ze˩v̩˩}}}}\hspace{0.5cm}[\kern2pt{\textcolor{darkblue}{\textbf{\ipa{ze˩v̩˩˥}}}}\kern2pt]} \hypertarget{ze\string_Bv\string_=\string_B1}{}
\markboth{\textcolor{darkblue}{\textbf{\ipa{ze˩v̩˩}}}}{}
\textcolor{teal}{\mytextsc{noun}} \hspace{4pt} Tone: L.
\textcolor{Sepia}{\selectlanguage{english}Nephew (son of one's sister).} \zh{外甥(姐妹的儿子)。}  \zh{量词}: \textcolor{darkblue}{\textbf{\ipa{v̩˧}}}  \mytextsc{clf}: \textcolor{darkblue}{\textbf{\ipa{v̩˧}}} 
\lhead{\firstmark}
\rhead{\botmark}

\subsection{\hspace{-0.5cm} {\Large \textcolor{darkblue}{\textbf{\ipa{ze˩v̩˩-ze˧mi˩}}}}\hspace{0.5cm}[\kern2pt{\textcolor{darkblue}{\textbf{\ipa{ze˩v̩˩ze˧mi˩}}}}\kern2pt]} \hypertarget{ze\string_Bv\string_=\string_B-ze\string_Mmi\string_B1}{}
\markboth{\textcolor{darkblue}{\textbf{\ipa{ze˩v̩˩-ze˧mi˩}}}}{}
\textcolor{teal}{\mytextsc{noun}} \hspace{4pt} Tone: L-L\#.
\textcolor{Sepia}{\selectlanguage{english}Nephews and nieces.} \zh{外甥甥女(姐妹的儿女)。} 
\lhead{\firstmark}
\rhead{\botmark}

\subsection{\hspace{-0.5cm} {\Large \textcolor{darkblue}{\textbf{\ipa{zo˥}}}}\hspace{0.5cm}[\kern2pt{\textcolor{darkblue}{\textbf{\ipa{zo˩˥}}}}\kern2pt]} \hypertarget{zo\string_T1}{}
\markboth{\textcolor{darkblue}{\textbf{\ipa{zo˥}}}}{}
\textcolor{teal}{\mytextsc{noun}} \hspace{4pt} Tone: \#H.
\ding{202} \textcolor{Sepia}{\selectlanguage{english}Son.} \zh{儿子。}  ¶ \textcolor{darkblue}{\textbf{\ipa{zo˧ ɲi˥-kv̩˩}}} \textcolor{Sepia}{\selectlanguage{english}two sons} \zh{两个儿子}  
 \zh{量词}: \textcolor{darkblue}{\textbf{\ipa{v̩˧}}} \ding{203} \textcolor{Sepia}{\selectlanguage{english}Man, \textit{Vir}.} \zh{男人。}  \mytextsc{clf}: \textcolor{darkblue}{\textbf{\ipa{v̩˧}}} 
\lhead{\firstmark}
\rhead{\botmark}

\subsection{\hspace{-0.5cm} {\Large \textcolor{darkblue}{\textbf{\ipa{‑zo˧}}}}\hspace{0.5cm}[\kern2pt{\textcolor{darkblue}{\textbf{\ipa{zo˥}}}}\kern2pt]} \hypertarget{‑zo\string_M1}{}
\markboth{\textcolor{darkblue}{\textbf{\ipa{‑zo˧}}}}{}
\textcolor{teal}{\mytextsc{suffix}} \hspace{4pt} Tone: M.
\textcolor{Sepia}{\selectlanguage{english}Obligative.} \zh{应该、必须。} 
\lhead{\firstmark}
\rhead{\botmark}

\subsection{\hspace{-0.5cm} {\Large \textcolor{darkblue}{\textbf{\ipa{zo˧\textsubscript{a}}}}}\hspace{0.5cm}[\kern2pt{\textcolor{darkblue}{\textbf{\ipa{zo˩˥}}}}\kern2pt]} \hypertarget{zo\string_Ma1}{}
\markboth{\textcolor{darkblue}{\textbf{\ipa{zo˧\textsubscript{a}}}}}{}
\textcolor{teal}{\mytextsc{verb}} \hspace{4pt} Tone: M\textsubscript{a}.
\textcolor{Sepia}{\selectlanguage{english}To have to, to be necessary.} \zh{要,应该。}  ¶ \textcolor{darkblue}{\textbf{\ipa{mɤ˧-zo˧ (-ze˧)! | tʰi˧-kwɤ˩-kʰɯ˩!}}} \textcolor{Sepia}{\selectlanguage{english}It's not necessary! Forget it!} \zh{不用了!算了吧!}  
 ¶ \textcolor{darkblue}{\textbf{\ipa{ʈʂʰɯ˧ne˧-ʝi˥ | ʝi˧-zo˧-ho˥-ɲi˩!}}} \textcolor{Sepia}{\selectlanguage{english}That's how one must do! / That's how it's done!} \zh{是应该这样做的!}  

\lhead{\firstmark}
\rhead{\botmark}

\subsection{\hspace{-0.5cm} {\Large \textcolor{darkblue}{\textbf{\ipa{zo˧bæ˩}}}}\hspace{0.5cm}[\kern2pt{\textcolor{darkblue}{\textbf{\ipa{zo˧bæ˧}}}}\kern2pt]} \hypertarget{zo\string_Mb\{\string_B1}{}
\markboth{\textcolor{darkblue}{\textbf{\ipa{zo˧bæ˩}}}}{}
\textcolor{teal}{\mytextsc{noun}} \hspace{4pt} Tone: L\#.
\textcolor{Sepia}{\selectlanguage{english}Fool, idiot.} \zh{笨人、傻瓜。}  ¶ \textcolor{darkblue}{\textbf{\ipa{mɤ˧-zo˧bæ˩!}}} \textcolor{Sepia}{\selectlanguage{english}No, (you) are not an idiot! (A reassuring answer to someone who deprecates himself as an idiot.)} \zh{(你)不是笨蛋!(情景:一个人批评自己是笨蛋,人家安慰他。)}  
 ¶ \textcolor{darkblue}{\textbf{\ipa{zo˧bæ˩-mv̩˩bæ˩}}} \textcolor{Sepia}{\selectlanguage{english}silly people, idiots (of both sexes)} \zh{傻瓜们(不分男女)}  
 \zh{量词}: \textcolor{darkblue}{\textbf{\ipa{v̩˧}}}  \mytextsc{clf}: \textcolor{darkblue}{\textbf{\ipa{v̩˧}}} 
\lhead{\firstmark}
\rhead{\botmark}

\subsection{\hspace{-0.5cm} {\Large \textcolor{darkblue}{\textbf{\ipa{zo˧ɖɯ\#˥}}}}\hspace{0.5cm}[\kern2pt{\textcolor{darkblue}{\textbf{\ipa{zo˧ɖɯ˧}}}}\kern2pt]} \hypertarget{zo\string_Md`M\#\string_T1}{}
\markboth{\textcolor{darkblue}{\textbf{\ipa{zo˧ɖɯ\#˥}}}}{}
\textcolor{teal}{\mytextsc{noun}} \hspace{4pt} Tone: \#H.
\textcolor{Sepia}{\selectlanguage{english}Eldest son.} \zh{大儿子。}  ¶ \textcolor{darkblue}{\textbf{\ipa{zo˧ɖɯ˧-mv̩˥ɖɯ˩}}} \textcolor{Sepia}{\selectlanguage{english}eldest son and eldest daughter} \zh{大儿子与大女儿}  

\lhead{\firstmark}
\rhead{\botmark}

\subsection{\hspace{-0.5cm} {\Large \textcolor{darkblue}{\textbf{\ipa{zo˧hṽ˧-mv̩˥zo˩}}}}\hspace{0.5cm}[\kern2pt{\textcolor{darkblue}{\textbf{\ipa{zo˧hṽ˧˥mv̩˧zo˧}}}}\kern2pt]} \hypertarget{zo\string_Mhv\string_~\string_M-mv\string_=\string_Tzo\string_B1}{}
\markboth{\textcolor{darkblue}{\textbf{\ipa{zo˧hṽ˧-mv̩˥zo˩}}}}{}
\textcolor{teal}{\mytextsc{noun}} \hspace{4pt} Tone: MH\#-.
\textcolor{Sepia}{\selectlanguage{english}Descendants.} \zh{后代。}  ¶ \textcolor{darkblue}{\textbf{\ipa{zo˧hṽ˧mv̩˥zo˩=ɻæ˩}}} \textcolor{Sepia}{\selectlanguage{english}\string_ \mytextsc{associative}} \zh{\string_ \mytextsc{联想复数}}  

\lhead{\firstmark}
\rhead{\botmark}

\subsection{\hspace{-0.5cm} {\Large \textcolor{darkblue}{\textbf{\ipa{zo˧hṽ˧˥}}}}\hspace{0.5cm}[\kern2pt{\textcolor{darkblue}{\textbf{\ipa{zo˧hṽ˧˥}}}}\kern2pt]} \hypertarget{zo\string_Mhv\string_~\string_M\string_T1}{}
\markboth{\textcolor{darkblue}{\textbf{\ipa{zo˧hṽ˧˥}}}}{}
\textcolor{teal}{\mytextsc{noun}} \hspace{4pt} Tone: MH\#.
\ding{202} \textcolor{Sepia}{\selectlanguage{english}Son.} \zh{儿子。}  ¶ \textcolor{darkblue}{\textbf{\ipa{zo˧hṽ˧=ɻæ˥}}} \textcolor{Sepia}{\selectlanguage{english}the sons} \zh{儿子们}  
 \zh{量词}: \textcolor{darkblue}{\textbf{\ipa{v̩˧}}} \ding{203} \textcolor{Sepia}{\selectlanguage{english}Young chap, young lad, young man.} \zh{小伙子、 青年男子。}  \mytextsc{clf}: \textcolor{darkblue}{\textbf{\ipa{v̩˧}}} 
\lhead{\firstmark}
\rhead{\botmark}

\subsection{\hspace{-0.5cm} {\Large \textcolor{darkblue}{\textbf{\ipa{zo˧mv̩˥}}}}\hspace{0.5cm}[\kern2pt{\textcolor{darkblue}{\textbf{\ipa{zo˧mv̩˥}}}}\kern2pt]} \hypertarget{zo\string_Mmv\string_=\string_T1}{}
\markboth{\textcolor{darkblue}{\textbf{\ipa{zo˧mv̩˥}}}}{}
\textcolor{teal}{\mytextsc{noun}} \hspace{4pt} Tone: H\#.
\textcolor{Sepia}{\selectlanguage{english}Child.} \zh{孩子。}  ¶ \textcolor{darkblue}{\textbf{\ipa{zo˧mv̩˥ | æ˧mv̩˥tɕi˩-hĩ˩}}} \textcolor{Sepia}{\selectlanguage{english}newborn baby, infant} \zh{新生婴儿}  
 \zh{量词}: \textcolor{darkblue}{\textbf{\ipa{ɭɯ˧}}}  \mytextsc{clf}: \textcolor{darkblue}{\textbf{\ipa{ɭɯ˧}}} 
\lhead{\firstmark}
\rhead{\botmark}

\subsection{\hspace{-0.5cm} {\Large \textcolor{darkblue}{\textbf{\ipa{zo˧tv̩˧-mv̩˥tv̩˩}}}}\hspace{0.5cm}[\kern2pt{\textcolor{darkblue}{\textbf{\ipa{zo˧tv̩˧mv̩˥tv̩˩}}}}\kern2pt]} \hypertarget{zo\string_Mtv\string_=\string_M-mv\string_=\string_Ttv\string_=\string_B1}{}
\markboth{\textcolor{darkblue}{\textbf{\ipa{zo˧tv̩˧-mv̩˥tv̩˩}}}}{}
\textcolor{teal}{\mytextsc{noun}} \hspace{4pt} Tone: \#H-.
\textcolor{Sepia}{\selectlanguage{english}Only child (boy or girl).} \zh{独生子(男女通用)。} 
\lhead{\firstmark}
\rhead{\botmark}

\subsection{\hspace{-0.5cm} {\Large \textcolor{darkblue}{\textbf{\ipa{zo˧tv̩\#˥}}}}\hspace{0.5cm}[\kern2pt{\textcolor{darkblue}{\textbf{\ipa{zo˧tv̩˧}}}}\kern2pt]} \hypertarget{zo\string_Mtv\string_=\#\string_T1}{}
\markboth{\textcolor{darkblue}{\textbf{\ipa{zo˧tv̩\#˥}}}}{}
\textcolor{teal}{\mytextsc{noun}} \hspace{4pt} Tone: \#H.
\textcolor{Sepia}{\selectlanguage{english}Only son.} \zh{独生子,独生男孩。}  ¶ \textcolor{darkblue}{\textbf{\ipa{zo˧tv̩˧ ɖɯ˧-v̩˧-lɑ˧ dʑo˧˥!}}} \textcolor{Sepia}{\selectlanguage{english}(She) just has an only son!} \zh{(她)只有一个独生男孩子!}  
 ¶ \textcolor{darkblue}{\textbf{\ipa{ʂɯ˧-ɬi˧mi˧, | zo˧tv̩˧ ʐɤ˥-tʰɑ˩-se˩!}}} \textcolor{Sepia}{\selectlanguage{english}“In the seventh month, let not an only son take the road!” (The seventh month is the peak of the rainy season; it was considered as a wrong time for long travels.)} \zh{“七月份,独生子不要上路!”(七月份是大雨季,摩梭人认为七月份的路最不安全:有生命危险)}  

\lhead{\firstmark}
\rhead{\botmark}

\subsection{\hspace{-0.5cm} {\Large \textcolor{darkblue}{\textbf{\ipa{zo˧tʰi˧}}}}\hspace{0.5cm}[\kern2pt{\textcolor{darkblue}{\textbf{\ipa{zo˧tʰi˧}}}}\kern2pt]} \hypertarget{zo\string_Mt\string_hi\string_M1}{}
\markboth{\textcolor{darkblue}{\textbf{\ipa{zo˧tʰi˧}}}}{}
\textcolor{teal}{\mytextsc{noun}} \hspace{4pt} Tone: M.
\textcolor{Sepia}{\selectlanguage{english}Intelligent person.} \zh{聪明的人。}  ¶ \textcolor{darkblue}{\textbf{\ipa{zo˧tʰi˧ ɖɯ˧-v̩˧}}} \textcolor{Sepia}{\selectlanguage{english}an intelligent person} \zh{一个聪明的人}  

\lhead{\firstmark}
\rhead{\botmark}

\subsection{\hspace{-0.5cm} {\Large \textcolor{darkblue}{\textbf{\ipa{zo˧tʰi˧}}}}\hspace{0.5cm}[\kern2pt{\textcolor{darkblue}{\textbf{\ipa{zo˧tʰi˧}}}}\kern2pt]} \hypertarget{zo\string_Mt\string_hi\string_M1}{}
\markboth{\textcolor{darkblue}{\textbf{\ipa{zo˧tʰi˧}}}}{}
\textcolor{teal}{\mytextsc{adjective}} \hspace{4pt} Tone: M.
\textcolor{Sepia}{\selectlanguage{english}Intelligent.} \zh{聪明。}  ¶ \textcolor{darkblue}{\textbf{\ipa{ʈʂʰɯ˧ | zo˧tʰi˧ | ʐwæ˩˥!}}} \textcolor{Sepia}{\selectlanguage{english}He is very clever!} \zh{他很聪明!}  
 ¶ \textcolor{darkblue}{\textbf{\ipa{ʈʂʰɯ˧ | mɤ˧-tʰi˧!}}} \textcolor{Sepia}{\selectlanguage{english}He is not clever!} \zh{他不聪明!}  

\lhead{\firstmark}
\rhead{\botmark}

\subsection{\hspace{-0.5cm} {\Large \textcolor{darkblue}{\textbf{\ipa{zo˧tɕi˥}}}}\hspace{0.5cm}[\kern2pt{\textcolor{darkblue}{\textbf{\ipa{zo˧tɕi˥}}}}\kern2pt]} \hypertarget{zo\string_Mts£i\string_T1}{}
\markboth{\textcolor{darkblue}{\textbf{\ipa{zo˧tɕi˥}}}}{}
\textcolor{teal}{\mytextsc{noun}} \hspace{4pt} Tone: H\#.
\textcolor{Sepia}{\selectlanguage{english}Youngest son.} \zh{最小的儿子。}  ¶ \textcolor{darkblue}{\textbf{\ipa{zo˧tɕi˥-mv̩˩tɕi˩}}} \textcolor{Sepia}{\selectlanguage{english}youngest son and youngest daughter} \zh{最小的儿子与女儿}  

\lhead{\firstmark}
\rhead{\botmark}

\subsection{\hspace{-0.5cm} {\Large \textcolor{darkblue}{\textbf{\ipa{zo˧zo˧-mv̩˧mv̩˥}}}}\hspace{0.5cm}[\kern2pt{\textcolor{darkblue}{\textbf{\ipa{xxxx non-correspondance entre le nombre de morphèmes et le nombre de tons de morphèmes}}}}\kern2pt]} \hypertarget{zo\string_Mzo\string_M-mv\string_=\string_Mmv\string_=\string_T1}{}
\markboth{\textcolor{darkblue}{\textbf{\ipa{zo˧zo˧-mv̩˧mv̩˥}}}}{}
\textcolor{teal}{\mytextsc{noun}} \hspace{4pt} Tone: H\#.
\textcolor{Sepia}{\selectlanguage{english}Thing, thingummy.} \zh{东西。}  \zh{量词}: \textcolor{darkblue}{\textbf{\ipa{kʰwɤ˥}}}  \mytextsc{clf}: \textcolor{darkblue}{\textbf{\ipa{kʰwɤ˥}}} 
\lhead{\firstmark}
\rhead{\botmark}

\subsection{\hspace{-0.5cm} {\Large \textcolor{darkblue}{\textbf{\ipa{zo˧ʐɤ\#˥}}}}\hspace{0.5cm}[\kern2pt{\textcolor{darkblue}{\textbf{\ipa{zo˧ʐɤ˧}}}}\kern2pt]} \hypertarget{zo\string_Mz`7\#\string_T1}{}
\markboth{\textcolor{darkblue}{\textbf{\ipa{zo˧ʐɤ\#˥}}}}{}
\textcolor{teal}{\mytextsc{noun}} \hspace{4pt} Tone: \#H.
\textcolor{Sepia}{\selectlanguage{english}Adoptive son, foster son.} \zh{义子。} 
\lhead{\firstmark}
\rhead{\botmark}

\subsection{\hspace{-0.5cm} {\Large \textcolor{darkblue}{\textbf{\ipa{zo˩bv̩˥li˩}}}}\hspace{0.5cm}[\kern2pt{\textcolor{darkblue}{\textbf{\ipa{zo˧bv̩˧li˩}}}}\kern2pt]} \hypertarget{zo\string_Bbv\string_=\string_Tli\string_B1}{}
\markboth{\textcolor{darkblue}{\textbf{\ipa{zo˩bv̩˥li˩}}}}{}
\textcolor{teal}{\mytextsc{noun}} \hspace{4pt} Tone: .
\textcolor{Sepia}{\selectlanguage{english}Universe.} \zh{宇宙。}  Borrowing: Tibetan?  (Lidz 2010: 108)
 ¶ \textcolor{darkblue}{\textbf{\ipa{sɑ˧ | -zo˩bv̩˥-li˩}}} \textcolor{Sepia}{\selectlanguage{english}the universe} \zh{宇宙}  

\lhead{\firstmark}
\rhead{\botmark}

\subsection{\hspace{-0.5cm} {\Large \textcolor{darkblue}{\textbf{\ipa{zo˩no˧}}}}\hspace{0.5cm}[\kern2pt{\textcolor{darkblue}{\textbf{\ipa{zo˩no˥}}}}\kern2pt]} \hypertarget{zo\string_Bno\string_M1}{}
\markboth{\textcolor{darkblue}{\textbf{\ipa{zo˩no˧}}}}{}
\textcolor{teal}{\mytextsc{adverb(ial)}} \hspace{4pt} Tone: LM.
\textcolor{Sepia}{\selectlanguage{english}Now.} \zh{现在。}  ¶ \textcolor{darkblue}{\textbf{\ipa{zo˩no˥ | gɤ˩-ʈi˧!}}} \textcolor{Sepia}{\selectlanguage{english}She only just woke up! (Context: someone walks into the house in the afternoon, sees a little child playing, and notes: “She has got up!” The child's grandmother answers: “She only just woke up!”)} \zh{刚起床! / 刚才才起床!}  

\lhead{\firstmark}
\rhead{\botmark}

\subsection{\hspace{-0.5cm} {\Large \textcolor{darkblue}{\textbf{\ipa{zo˩qo˧}}}}\hspace{0.5cm}[\kern2pt{\textcolor{darkblue}{\textbf{\ipa{zo˩qo˥}}}}\kern2pt]} \hypertarget{zo\string_Bqo\string_M1}{}
\markboth{\textcolor{darkblue}{\textbf{\ipa{zo˩qo˧}}}}{}
\textcolor{teal}{\mytextsc{pronoun/pronominal}} \hspace{4pt} Tone: LM.
\textcolor{Sepia}{\selectlanguage{english}Where.} \zh{哪里。}  ¶ \textcolor{darkblue}{\textbf{\ipa{no˧ | zo˩qo˧ bi˧?}}} \textcolor{Sepia}{\selectlanguage{english}Where are you going?} \zh{你去哪里?}  
 ¶ \textcolor{darkblue}{\textbf{\ipa{zo˩qo˧-ɳɯ˧ | tsʰɯ˩˥?}}} \textcolor{Sepia}{\selectlanguage{english}Where (are you) coming from?} \zh{从哪里来?}  
 ¶ \textcolor{darkblue}{\textbf{\ipa{no˧ | hɑ˧ | zo˩qo˧ dzɯ˧-bi˧-pi˧, | ɖɯ˧-bæ˧ lɑ˧ ɲi˥!}}} \textcolor{Sepia}{\selectlanguage{english}No matter which one you choose: they're all the same! (Context: discussing the restaurant scene in Yongning; in the speaker's view, the newly opened restaurants all share the same qualities and shortcomings, for instance concerning hygiene.)} \zh{无论你到哪里去吃,都一样!(情景:新开的饭馆)}  
 ¶ \textcolor{darkblue}{\textbf{\ipa{zo˩qo˧ tʰv̩˧?}}} \textcolor{Sepia}{\selectlanguage{english}Where are you? (Typical question when calling someone on their mobile phone)} \zh{你到哪里了?(打手机)}  

\lhead{\firstmark}
\rhead{\botmark}

\subsection{\hspace{-0.5cm} {\Large \textcolor{darkblue}{\textbf{\ipa{zɯ˥}}}}\hspace{0.5cm}[\kern2pt{\textcolor{darkblue}{\textbf{\ipa{zɯ˥}}}}\kern2pt]} \hypertarget{zM\string_T1}{}
\markboth{\textcolor{darkblue}{\textbf{\ipa{zɯ˥}}}}{}
\textcolor{teal}{\mytextsc{noun}} \hspace{4pt} Tone: \#H.
\textcolor{Sepia}{\selectlanguage{english}Grass.} \zh{草。}  \zh{量词}: \textcolor{darkblue}{\textbf{\ipa{kʰɤ˧˥}}}  \mytextsc{clf}: \textcolor{darkblue}{\textbf{\ipa{kʰɤ˧˥}}} 
\lhead{\firstmark}
\rhead{\botmark}

\subsection{\hspace{-0.5cm} {\Large \textcolor{darkblue}{\textbf{\ipa{zɯ˧}}}}\hspace{0.5cm}[\kern2pt{\textcolor{darkblue}{\textbf{\ipa{zɯ˥}}}}\kern2pt]} \hypertarget{zM\string_M1}{}
\markboth{\textcolor{darkblue}{\textbf{\ipa{zɯ˧}}}}{}
\textcolor{teal}{\mytextsc{noun}} \hspace{4pt} Tone: M.
\textcolor{Sepia}{\selectlanguage{english}Life, existence.} \zh{生命。}  ¶ \textcolor{darkblue}{\textbf{\ipa{zɯ˧ʂæ\#˥}}} \textcolor{Sepia}{\selectlanguage{english}long life} \zh{长命、长的人生}  
 ¶ \textcolor{darkblue}{\textbf{\ipa{zɯ˧ ʂæ˧ | hɑ̃˧-ʝi˧-kʰɯ˩!}}} \textcolor{Sepia}{\selectlanguage{english}May you have a long life!} \zh{祝你长寿!}  
 ¶ \textcolor{darkblue}{\textbf{\ipa{zɯ˧ɖæ\#˥}}} \textcolor{Sepia}{\selectlanguage{english}short life} \zh{短命}  
\textit{See:} \hyperlink{}{\textcolor{darkblue}{\textbf{\ipa{zɯ˧\textsubscript{b}}}}} 
\lhead{\firstmark}
\rhead{\botmark}

\subsection{\hspace{-0.5cm} {\Large \textcolor{darkblue}{\textbf{\ipa{zɯ˧\textsubscript{b}}}}}\hspace{0.5cm}[\kern2pt{\textcolor{darkblue}{\textbf{\ipa{zɯ˥}}}}\kern2pt]} \hypertarget{zM\string_Mb1}{}
\markboth{\textcolor{darkblue}{\textbf{\ipa{zɯ˧\textsubscript{b}}}}}{}
\textcolor{teal}{\mytextsc{classifier}} \hspace{4pt} Tone: M\textsubscript{b}.
\textcolor{Sepia}{\selectlanguage{english}Self-classifier for life, existence.} \zh{量词:辈子。}  ¶ \textcolor{darkblue}{\textbf{\ipa{ɖɯ˧-zɯ˧}}} \textcolor{Sepia}{\selectlanguage{english}all of (one's) life, a lifetime} \zh{一辈子(的时间)}  
\textit{See:} \hyperlink{}{\textcolor{darkblue}{\textbf{\ipa{zɯ˧}}}} 
\lhead{\firstmark}
\rhead{\botmark}

\subsection{\hspace{-0.5cm} {\Large \textcolor{darkblue}{\textbf{\ipa{zɯ˧hṽ˩}}}}\hspace{0.5cm}[\kern2pt{\textcolor{darkblue}{\textbf{\ipa{zɯ˧hṽ˩}}}}\kern2pt]} \hypertarget{zM\string_Mhv\string_~\string_B1}{}
\markboth{\textcolor{darkblue}{\textbf{\ipa{zɯ˧hṽ˩}}}}{}
\textcolor{teal}{\mytextsc{adjective}} \hspace{4pt} Tone: L\#.
\textcolor{Sepia}{\selectlanguage{english}Green.} \zh{绿(布料、线)。}  ¶ \textcolor{darkblue}{\textbf{\ipa{zɯ˧hṽ˩-ni˩gv̩˩}}} \textcolor{Sepia}{\selectlanguage{english}green} \zh{绿}  
 ¶ \textcolor{darkblue}{\textbf{\ipa{[F5] zɯ˧hṽ˩ | \textasciitilde{}zɯ˧hṽ˩-ni˩gv̩˩}}} \textcolor{Sepia}{\selectlanguage{english}all green, green all over} \zh{全绿}  

\lhead{\firstmark}
\rhead{\botmark}

\subsection{\hspace{-0.5cm} {\Large \textcolor{darkblue}{\textbf{\ipa{zɯ˧pv̩˩}}}}\hspace{0.5cm}[\kern2pt{\textcolor{darkblue}{\textbf{\ipa{zɯ˧pv̩˩}}}}\kern2pt]} \hypertarget{zM\string_Mpv\string_=\string_B1}{}
\markboth{\textcolor{darkblue}{\textbf{\ipa{zɯ˧pv̩˩}}}}{}
\textcolor{teal}{\mytextsc{noun}} \hspace{4pt} Tone: L\#.
\textcolor{Sepia}{\selectlanguage{english}Hay, dry grass.} \zh{干草。}  \zh{量词}: \textcolor{darkblue}{\textbf{\ipa{kʰɤ˧˥}}}  \mytextsc{clf}: \textcolor{darkblue}{\textbf{\ipa{kʰɤ˧˥}}} 
\lhead{\firstmark}
\rhead{\botmark}

\subsection{\hspace{-0.5cm} {\Large \textcolor{darkblue}{\textbf{\ipa{zɯ˧-qʰɑ˧mi\#˥}}}}\hspace{0.5cm}[\kern2pt{\textcolor{darkblue}{\textbf{\ipa{xxxx non-correspondance entre le nombre de morphèmes et le nombre de tons de morphèmes}}}}\kern2pt]} \hypertarget{zM\string_M-q\string_hA\string_Mmi\#\string_T1}{}
\markboth{\textcolor{darkblue}{\textbf{\ipa{zɯ˧-qʰɑ˧mi\#˥}}}}{}
\textcolor{teal}{\mytextsc{noun}} \hspace{4pt} Tone: \#H.
\textcolor{Sepia}{\selectlanguage{english}Sabai grass, \textit{Eulaliopsis binata (Retz.) C. E. Hubb.}.} \zh{蓑草、山草、山草根、龙须草、山茅草、羊草、拟金茅。} Local Chinese dialect:\zh{狗尾巴草。} \zh{量词}: \textcolor{darkblue}{\textbf{\ipa{po˧}}}  \mytextsc{clf}: \textcolor{darkblue}{\textbf{\ipa{po˧}}} 
\lhead{\firstmark}
\rhead{\botmark}

\subsection{\hspace{-0.5cm} {\Large \textcolor{darkblue}{\textbf{\ipa{zɯ˧ɻ\#˥}}}}\hspace{0.5cm}[\kern2pt{\textcolor{darkblue}{\textbf{\ipa{zɯ˧ɻ˧}}}}\kern2pt]} \hypertarget{zM\string_Mr£`\#\string_T1}{}
\markboth{\textcolor{darkblue}{\textbf{\ipa{zɯ˧ɻ\#˥}}}}{}
\textcolor{teal}{\mytextsc{noun}} \hspace{4pt} Tone: \#H.
\textcolor{Sepia}{\selectlanguage{english}Cheek.} \zh{腮、腮帮子。}  ¶ \textcolor{darkblue}{\textbf{\ipa{zɯ˧ɻ̍˧ qʰwæ˩}}} \textcolor{Sepia}{\selectlanguage{english}to slap/smack someone's cheek} \zh{掌掴、打嘴巴}  
 \zh{量词}: \textcolor{darkblue}{\textbf{\ipa{ɭɯ˧}}}  \mytextsc{clf}: \textcolor{darkblue}{\textbf{\ipa{ɭɯ˧}}} 
\lhead{\firstmark}
\rhead{\botmark}

\subsection{\hspace{-0.5cm} {\Large \textcolor{darkblue}{\textbf{\ipa{zɯ˧\textasciitilde{}zɯ˧}}}}\hspace{0.5cm}[\kern2pt{\textcolor{darkblue}{\textbf{\ipa{zɯ˧zɯ˧}}}}\kern2pt]} \hypertarget{zM\string_M~zM\string_M1}{}
\markboth{\textcolor{darkblue}{\textbf{\ipa{zɯ˧\textasciitilde{}zɯ˧}}}}{}
\textcolor{teal}{\mytextsc{noun}} \hspace{4pt} Tone: M.
\textcolor{Sepia}{\selectlanguage{english}Life, existence.} \zh{生命。}  ¶ \textcolor{darkblue}{\textbf{\ipa{hĩ˧-zɯ˧\textasciitilde{}zɯ˥\$}}} \textcolor{Sepia}{\selectlanguage{english}human life} \zh{人生}  
 ¶ \textcolor{darkblue}{\textbf{\ipa{hĩ˧ zɯ˧ | ʂæ˧ | ʐwæ˩˥}}} \textcolor{Sepia}{\selectlanguage{english}a very long life / life is very long} \zh{很长的人生 / 人生很长}  
 \zh{量词}: \textcolor{darkblue}{\textbf{\ipa{ljɤ˩}}}  \mytextsc{clf}: \textcolor{darkblue}{\textbf{\ipa{ljɤ˩}}} 
\lhead{\firstmark}
\rhead{\botmark}

\subsection{\hspace{-0.5cm} {\Large \textcolor{darkblue}{\textbf{\ipa{zɯ˩\textasciitilde{}zɯ˩}}}}\hspace{0.5cm}[\kern2pt{\textcolor{darkblue}{\textbf{\ipa{zɯ˩zɯ˩˥}}}}\kern2pt]} \hypertarget{zM\string_B~zM\string_B1}{}
\markboth{\textcolor{darkblue}{\textbf{\ipa{zɯ˩\textasciitilde{}zɯ˩}}}}{}
\textcolor{teal}{\mytextsc{verb}} \hspace{4pt} Tone: L.
\textcolor{Sepia}{\selectlanguage{english}To be numb.} \zh{麻木。}  ¶ \textcolor{darkblue}{\textbf{\ipa{gv̩˧dv̩˧gv̩˧mi˧ | zɯ˩\textasciitilde{}zɯ˩˥}}} \textcolor{Sepia}{\selectlanguage{english}to be numb (whole body)} \zh{身体麻木、全身麻木}  
 ¶ \textcolor{darkblue}{\textbf{\ipa{gv̩˧mi˧ | zɯ˩\textasciitilde{}zɯ˩˥}}} \textcolor{Sepia}{\selectlanguage{english}to be numb (whole body)} \zh{身体麻木、全身麻木}  
 ¶ \textcolor{darkblue}{\textbf{\ipa{tʰi˧-zɯ˩\textasciitilde{}zɯ˩}}} \textcolor{Sepia}{\selectlanguage{english}\mytextsc{dur} \mytextsc{red}} \zh{\mytextsc{dur} \mytextsc{red}}  

\lhead{\firstmark}
\rhead{\botmark}

\newpage
\section*{\centering- \textcolor{darkblue}{\textbf{\ipa{ʐ}}} -}
\subsection{\hspace{-0.5cm} {\Large \textcolor{darkblue}{\textbf{\ipa{ʐ}}}}\hspace{0.5cm}[\kern2pt{\textcolor{darkblue}{\textbf{\ipa{[]}}}}\kern2pt]} \hypertarget{z`1}{}
\markboth{\textcolor{darkblue}{\textbf{\ipa{ʐ}}}}{}
\textcolor{teal}{\mytextsc{ideophone}} \hspace{4pt} Tone: 0.
\textcolor{Sepia}{\selectlanguage{english}Rumbling sound of heavy loads carried over a wooden floor, of lorries... Brroom!} \zh{形声词:轰隆隆!(拉很重的物品在地板上的隆隆声,卡车的隆隆声)。} 
\lhead{\firstmark}
\rhead{\botmark}

\subsection{\hspace{-0.5cm} {\Large \textcolor{darkblue}{\textbf{\ipa{ʐæ˧}}}}\hspace{0.5cm}[\kern2pt{\textcolor{darkblue}{\textbf{\ipa{ʐæ˩˥}}}}\kern2pt]} \hypertarget{z`\{\string_M1}{}
\markboth{\textcolor{darkblue}{\textbf{\ipa{ʐæ˧}}}}{}
\textcolor{teal}{\mytextsc{adjective}} \hspace{4pt} Tone: M.
\textcolor{Sepia}{\selectlanguage{english}Tall and big; great; impressive.} \zh{高大。}  ¶ \textcolor{darkblue}{\textbf{\ipa{ʐæ˧-ni˩gv̩˩}}} \textcolor{Sepia}{\selectlanguage{english}tall and big; great; impressive} \zh{高大}  
 ¶ \textcolor{darkblue}{\textbf{\ipa{hĩ˧ | ʈʂʰɯ˧-v̩˧, | ʐæ˧-ni˩gv̩˩!}}} \textcolor{Sepia}{\selectlanguage{english}This person looks impressive / tall and big!} \zh{这人很高大!}  
 ¶ \textcolor{darkblue}{\textbf{\ipa{ʐæ˧ni˩ | mɤ˧-gv̩˧}}} \textcolor{Sepia}{\selectlanguage{english}not tall, not impressive, not great-looking} \zh{个子不高}  
 ¶ \textcolor{darkblue}{\textbf{\ipa{ʐæ˧ | ʐwæ˩˥}}} \textcolor{Sepia}{\selectlanguage{english}very tall and big} \zh{很高大}  

\lhead{\firstmark}
\rhead{\botmark}

\subsection{\hspace{-0.5cm} {\Large \textcolor{darkblue}{\textbf{\ipa{ʐæ˧\textsubscript{a}}}}}\hspace{0.5cm}[\kern2pt{\textcolor{darkblue}{\textbf{\ipa{ʐæ˩˥}}}}\kern2pt]} \hypertarget{z`\{\string_Ma1}{}
\markboth{\textcolor{darkblue}{\textbf{\ipa{ʐæ˧\textsubscript{a}}}}}{}
\textcolor{teal}{\mytextsc{verb}} \hspace{4pt} Tone: M\textsubscript{a}.
\ding{202} \textcolor{Sepia}{\selectlanguage{english}To laugh.} \zh{笑。}  ¶ \textcolor{darkblue}{\textbf{\ipa{zo˧hṽ˥ | hĩ˧ ʐæ˧\textasciitilde{}ʐæ˥-kʰɯ˩}}} \textcolor{Sepia}{\selectlanguage{english}The kids make people laugh} \zh{孩子们把大家逗笑了。}  
 ¶ \textcolor{darkblue}{\textbf{\ipa{hĩ˧ | ʐæ˧\textasciitilde{}ʐæ˥ kʰɯ˩}}} \textcolor{Sepia}{\selectlanguage{english}to make people laugh, to amuse people} \zh{让大家笑一笑}  
 ¶ \textcolor{darkblue}{\textbf{\ipa{ʐæ˧\textasciitilde{}ʐæ˩-di˩}}} \textcolor{Sepia}{\selectlanguage{english}jokes, funny talk} \zh{笑话,好笑的话}  
\ding{203} \textcolor{Sepia}{\selectlanguage{english}To laugh at; to be impertinent; to deride, to make fun of (people).} \zh{嘲笑别人、出言不逊。}  ¶ \textcolor{darkblue}{\textbf{\ipa{le˧-ʐæ˧-ze˧}}} \textcolor{Sepia}{\selectlanguage{english}\mytextsc{accomp} \string_ \mytextsc{pfv}} \zh{出言不逊了}  
 ¶ \textcolor{darkblue}{\textbf{\ipa{le˧-ʐæ˥\textasciitilde{}ʐæ˩}}} \textcolor{Sepia}{\selectlanguage{english}\mytextsc{red}} \zh{笑一笑(别人)}  
 ¶ \textcolor{darkblue}{\textbf{\ipa{hĩ˧ ʐæ˩}}} \textcolor{Sepia}{\selectlanguage{english}to make fun of other people} \zh{嘲笑人家}  
 ¶ \textcolor{darkblue}{\textbf{\ipa{le˧-ʐæ˥\textasciitilde{}ʐæ˩-ze˩}}} \textcolor{Sepia}{\selectlanguage{english}\mytextsc{red} \mytextsc{pfv}} \zh{嘲笑了}  

\lhead{\firstmark}
\rhead{\botmark}

\subsection{\hspace{-0.5cm} {\Large \textcolor{darkblue}{\textbf{\ipa{ʐæ˧v̩˩-tʰv̩˩}}}}\hspace{0.5cm}[\kern2pt{\textcolor{darkblue}{\textbf{\ipa{ʐæ˧v̩˩tʰv̩˧}}}}\kern2pt]} \hypertarget{z`\{\string_Mv\string_=\string_B-t\string_hv\string_=\string_B1}{}
\markboth{\textcolor{darkblue}{\textbf{\ipa{ʐæ˧v̩˩-tʰv̩˩}}}}{}
\textcolor{teal}{\mytextsc{verb}} \hspace{4pt} Tone: L\#-.
\textcolor{Sepia}{\selectlanguage{english}To joke, to crack a joke.} \zh{开玩笑。}  ¶ \textcolor{darkblue}{\textbf{\ipa{ʐæ˧v̩˩-tʰv̩˩ | ʐwæ˩˥}}} \textcolor{Sepia}{\selectlanguage{english}to crack jokes all the time, to make lots of jokes} \zh{开很多玩笑、一直开玩笑}  
 ¶ \textcolor{darkblue}{\textbf{\ipa{ʐæ˧v̩˩-tʰv̩˩-hĩ˩ ʐwɤ˩}}} \textcolor{Sepia}{\selectlanguage{english}to crack a joke} \zh{开个玩笑}  

\lhead{\firstmark}
\rhead{\botmark}

\subsection{\hspace{-0.5cm} {\Large \textcolor{darkblue}{\textbf{\ipa{ʐæ˩\textsubscript{b}}}}}\hspace{0.5cm}[\kern2pt{\textcolor{darkblue}{\textbf{\ipa{ʐæ˥}}}}\kern2pt]} \hypertarget{z`\{\string_Bb1}{}
\markboth{\textcolor{darkblue}{\textbf{\ipa{ʐæ˩\textsubscript{b}}}}}{}
\textcolor{teal}{\mytextsc{verb}} \hspace{4pt} Tone: L\textsubscript{b}.
\textcolor{Sepia}{\selectlanguage{english}To mix.} \zh{搅拌合混。}  ¶ \textcolor{darkblue}{\textbf{\ipa{le˧-ʐæ˧\textasciitilde{}ʐæ˥}}} \textcolor{Sepia}{\selectlanguage{english}\mytextsc{accomp} \string_ \mytextsc{red}} \zh{搅拌}  

\lhead{\firstmark}
\rhead{\botmark}

\subsection{\hspace{-0.5cm} {\Large \textcolor{darkblue}{\textbf{\ipa{ʐæ˩mi\#˥}}}}\hspace{0.5cm}[\kern2pt{\textcolor{darkblue}{\textbf{\ipa{ʐæ˧mi˧}}}}\kern2pt]} \hypertarget{z`\{\string_Bmi\#\string_T1}{}
\markboth{\textcolor{darkblue}{\textbf{\ipa{ʐæ˩mi\#˥}}}}{}
\textcolor{teal}{\mytextsc{noun}} \hspace{4pt} Tone: LM+\#H.
\textcolor{Sepia}{\selectlanguage{english}Female leopard.} \zh{母豹子。}  ¶ \textcolor{darkblue}{\textbf{\ipa{ʐæ˩mi˧-ʐæ˥zo˩}}} \textcolor{Sepia}{\selectlanguage{english}female leopard and male leopard} \zh{母豹子与公豹子}  
 \zh{量词}: \textcolor{darkblue}{\textbf{\ipa{pʰo˧˥}}}  \mytextsc{clf}: \textcolor{darkblue}{\textbf{\ipa{pʰo˧˥}}} 
\lhead{\firstmark}
\rhead{\botmark}

\subsection{\hspace{-0.5cm} {\Large \textcolor{darkblue}{\textbf{\ipa{ʐæ˩pʰv̩˧}}}}\hspace{0.5cm}[\kern2pt{\textcolor{darkblue}{\textbf{\ipa{xxxx non-correspondance entre le nombre de morphèmes et le nombre de tons de morphèmes}}}}\kern2pt]} \hypertarget{z`\{\string_Bp\string_hv\string_=\string_M1}{}
\markboth{\textcolor{darkblue}{\textbf{\ipa{ʐæ˩pʰv̩˧}}}}{}
\textcolor{teal}{\mytextsc{noun}} \hspace{4pt} Tone: LM.
\textcolor{Sepia}{\selectlanguage{english}Male leopard.} \zh{公豹子。}  ¶ \textcolor{darkblue}{\textbf{\ipa{ʐæ˩pʰv̩˧-ʐæ˩mi˩}}} \textcolor{Sepia}{\selectlanguage{english}male leopard and female leopard} \zh{公豹子与母豹子}  
 \zh{量词}: \textcolor{darkblue}{\textbf{\ipa{pʰo˧˥}}}  \mytextsc{clf}: \textcolor{darkblue}{\textbf{\ipa{pʰo˧˥}}} 
\lhead{\firstmark}
\rhead{\botmark}

\subsection{\hspace{-0.5cm} {\Large \textcolor{darkblue}{\textbf{\ipa{ʐæ˩sɯ˩}}}}\hspace{0.5cm}[\kern2pt{\textcolor{darkblue}{\textbf{\ipa{ʐæ˩sɯ˥}}}}\kern2pt]} \hypertarget{z`\{\string_BsM\string_B1}{}
\markboth{\textcolor{darkblue}{\textbf{\ipa{ʐæ˩sɯ˩}}}}{}
\textcolor{teal}{\mytextsc{noun}} \hspace{4pt} Tone: L.
\textcolor{Sepia}{\selectlanguage{english}Rough felt made only of sheep wool. One drapes it over one's shoulders as an outdoor protection from the cold.} \zh{披毡。}  \zh{量词}: \textcolor{darkblue}{\textbf{\ipa{ɭɯ˧˥}}}  \mytextsc{clf}: \textcolor{darkblue}{\textbf{\ipa{ɭɯ˧˥}}} 
\lhead{\firstmark}
\rhead{\botmark}

\subsection{\hspace{-0.5cm} {\Large \textcolor{darkblue}{\textbf{\ipa{ʐæ˩sɯ˩-kʰwæ˩ɻæ˧}}}}\hspace{0.5cm}[\kern2pt{\textcolor{darkblue}{\textbf{\ipa{xxxx non-correspondance entre le nombre de morphèmes et le nombre de tons de morphèmes}}}}\kern2pt]} \hypertarget{z`\{\string_BsM\string_B-k\string_hw\{\string_Br£`\{\string_M1}{}
\markboth{\textcolor{darkblue}{\textbf{\ipa{ʐæ˩sɯ˩-kʰwæ˩ɻæ˧}}}}{}
\textcolor{teal}{\mytextsc{noun}} \hspace{4pt} Tone: .
\textcolor{Sepia}{\selectlanguage{english}Felt mat.} \zh{毡子(真正的毡子)做的垫子。}  \zh{量词}: \textcolor{darkblue}{\textbf{\ipa{tsʰi˥}}}  \mytextsc{clf}: \textcolor{darkblue}{\textbf{\ipa{tsʰi˥}}} 
\lhead{\firstmark}
\rhead{\botmark}

\subsection{\hspace{-0.5cm} {\Large \textcolor{darkblue}{\textbf{\ipa{ʐæ˩ʂæ˧}}}}\hspace{0.5cm}[\kern2pt{\textcolor{darkblue}{\textbf{\ipa{ʐæ˩ʂæ˥}}}}\kern2pt]} \hypertarget{z`\{\string_Bs`\{\string_M1}{}
\markboth{\textcolor{darkblue}{\textbf{\ipa{ʐæ˩ʂæ˧}}}}{}
\textcolor{teal}{\mytextsc{adjective}} \hspace{4pt} Tone: LM.
\textcolor{Sepia}{\selectlanguage{english}Far, distant.} \zh{远。}  ¶ \textcolor{darkblue}{\textbf{\ipa{no˧ | ʈʂʰɯ˧ | ə˩-ʐæ˥ʂæ˩? | dʑɤ˩kʰwɤ˧ ə˩-di˩? | - dʑɤ˩˥ | dʑɤ˩kʰwɤ˧ mɤ˧-di˥! | mɤ˧-ʐæ˩ʂæ˩!}}} \textcolor{Sepia}{\selectlanguage{english}Are you distant from him? Is there distance (between you)? - There is not much distance to speak of! We are not distant! (=we are close friends)} \zh{你们很熟吗? - 不很熟!}  

\lhead{\firstmark}
\rhead{\botmark}

\subsection{\hspace{-0.5cm} {\Large \textcolor{darkblue}{\textbf{\ipa{ʐæ˩tsɯ˧˥}}}}\hspace{0.5cm}[\kern2pt{\textcolor{darkblue}{\textbf{\ipa{ʐæ˩tsɯ˧˥}}}}\kern2pt]} \hypertarget{z`\{\string_BtsM\string_M\string_T1}{}
\markboth{\textcolor{darkblue}{\textbf{\ipa{ʐæ˩tsɯ˧˥}}}}{}
\textcolor{teal}{\mytextsc{noun}} \hspace{4pt} Tone: LM+MH\#.
\textcolor{Sepia}{\selectlanguage{english}Path, trail.} \zh{小路、径道。}  ¶ \textcolor{darkblue}{\textbf{\ipa{ʐæ˩tsɯ˧-ʐɤ˥mi˩}}} \textcolor{Sepia}{\selectlanguage{english}small trail} \zh{径道}  
 \zh{量词}: \textcolor{darkblue}{\textbf{\ipa{kʰɯ˩}}}  \mytextsc{clf}: \textcolor{darkblue}{\textbf{\ipa{kʰɯ˩}}} 
\lhead{\firstmark}
\rhead{\botmark}

\subsection{\hspace{-0.5cm} {\Large \textcolor{darkblue}{\textbf{\ipa{ʐæ˩zo\#˥}}}}\hspace{0.5cm}[\kern2pt{\textcolor{darkblue}{\textbf{\ipa{ʐæ˩zo˥}}}}\kern2pt]} \hypertarget{z`\{\string_Bzo\#\string_T1}{}
\markboth{\textcolor{darkblue}{\textbf{\ipa{ʐæ˩zo\#˥}}}}{}
\textcolor{teal}{\mytextsc{noun}} \hspace{4pt} Tone: LM+\#H.
\textcolor{Sepia}{\selectlanguage{english}Little leopard, baby leopard.} \zh{小豹子。}  ¶ \textcolor{darkblue}{\textbf{\ipa{ʐæ˩zo˧-ʐæ˥mi˩}}} \textcolor{Sepia}{\selectlanguage{english}baby leopard and female leopard} \zh{小豹子与母豹子}  
 \zh{量词}: \textcolor{darkblue}{\textbf{\ipa{ɭɯ˧}}}  \mytextsc{clf}: \textcolor{darkblue}{\textbf{\ipa{ɭɯ˧}}} 
\lhead{\firstmark}
\rhead{\botmark}

\subsection{\hspace{-0.5cm} {\Large \textcolor{darkblue}{\textbf{\ipa{ʐæ˩˥}}}}\hspace{0.5cm}[\kern2pt{\textcolor{darkblue}{\textbf{\ipa{ʐæ˥}}}}\kern2pt]} \hypertarget{z`\{\string_B\string_T1}{}
\markboth{\textcolor{darkblue}{\textbf{\ipa{ʐæ˩˥}}}}{}
\textcolor{teal}{\mytextsc{noun}} \hspace{4pt} Tone: LH.
\textcolor{Sepia}{\selectlanguage{english}Leopard, panther (note: these two terms are homonymous in English).} \zh{豹子。}  ¶ \textcolor{darkblue}{\textbf{\ipa{ʐæ˩ dzɯ˧-ze˩}}} \textcolor{Sepia}{\selectlanguage{english}...ate (a/the) leopard} \zh{吃了豹子}  
 ¶ \textcolor{darkblue}{\textbf{\ipa{ʐæ˩ hwæ˧-ze˩}}} \textcolor{Sepia}{\selectlanguage{english}...bought (a/the) leopard} \zh{买了豹子}  
 \zh{量词}: \textcolor{darkblue}{\textbf{\ipa{pʰo˧˥}}}  \mytextsc{clf}: \textcolor{darkblue}{\textbf{\ipa{pʰo˧˥}}} 
\lhead{\firstmark}
\rhead{\botmark}

\subsection{\hspace{-0.5cm} {\Large \textcolor{darkblue}{\textbf{\ipa{ʐe˥}}}}\hspace{0.5cm}[\kern2pt{\textcolor{darkblue}{\textbf{\ipa{ʐe˥}}}}\kern2pt]} \hypertarget{z`e\string_T1}{}
\markboth{\textcolor{darkblue}{\textbf{\ipa{ʐe˥}}}}{}
\textcolor{teal}{\mytextsc{classifier}} \hspace{4pt} Tone: H\textsubscript{a}.
\textcolor{Sepia}{\selectlanguage{english}Classifier for quarters of preserved meat.} \zh{量词:熏肉(一块)。} 
\lhead{\firstmark}
\rhead{\botmark}

\subsection{\hspace{-0.5cm} {\Large \textcolor{darkblue}{\textbf{\ipa{ʐe˥}}} \textsubscript{1}}\hspace{0.5cm}[\kern2pt{\textcolor{darkblue}{\textbf{\ipa{ʐe˥}}}}\kern2pt]} \hypertarget{z`e\string_T1}{}
\markboth{\textcolor{darkblue}{\textbf{\ipa{ʐe˥}}} \textsubscript{1}}{}
\textcolor{teal}{\mytextsc{noun}} \hspace{4pt} Tone: \#H.
\textcolor{Sepia}{\selectlanguage{english}Arrow.} \zh{箭。}  ¶ \textcolor{darkblue}{\textbf{\ipa{ʐe˧ɻ̃˧ | ɖɯ˧-kʰɯ˩}}} \textcolor{Sepia}{\selectlanguage{english}an arrow; also, metaphorically: a family, a lineage} \zh{一枝箭。也来指一个家庭}  
 \zh{量词}: \textcolor{darkblue}{\textbf{\ipa{kʰɯ˩}}}  \mytextsc{clf}: \textcolor{darkblue}{\textbf{\ipa{kʰɯ˩}}} 
\lhead{\firstmark}
\rhead{\botmark}

\subsection{\hspace{-0.5cm} {\Large \textcolor{darkblue}{\textbf{\ipa{ʐe˥}}} \textsubscript{2}}\hspace{0.5cm}[\kern2pt{\textcolor{darkblue}{\textbf{\ipa{ʐe˥}}}}\kern2pt]} \hypertarget{z`e\string_T2}{}
\markboth{\textcolor{darkblue}{\textbf{\ipa{ʐe˥}}} \textsubscript{2}}{}
\textcolor{teal}{\mytextsc{noun}} \hspace{4pt} Tone: \#H.
\textcolor{Sepia}{\selectlanguage{english}Rainy season (summer and autumn: from the 3rd to the 8th month of the lunar calendar).} \zh{雨季(夏天至秋天:三月份至八月份)。} 
\lhead{\firstmark}
\rhead{\botmark}

\subsection{\hspace{-0.5cm} {\Large \textcolor{darkblue}{\textbf{\ipa{ʐe˧ʈæ˥-ɬi˩}}}}\hspace{0.5cm}[\kern2pt{\textcolor{darkblue}{\textbf{\ipa{ʐe˧ʈæ˥ɬi˩}}}}\kern2pt]} \hypertarget{z`e\string_Mt`\{\string_T-Ki\string_B1}{}
\markboth{\textcolor{darkblue}{\textbf{\ipa{ʐe˧ʈæ˥-ɬi˩}}}}{}
\textcolor{teal}{\mytextsc{noun}} \hspace{4pt} Tone: H\#-L.
\textcolor{Sepia}{\selectlanguage{english}11th month.} \zh{十一月。} 
\lhead{\firstmark}
\rhead{\botmark}

\subsection{\hspace{-0.5cm} {\Large \textcolor{darkblue}{\textbf{\ipa{ʐe˧v̩\#˥}}}}\hspace{0.5cm}[\kern2pt{\textcolor{darkblue}{\textbf{\ipa{ʐe˧v̩˧}}}}\kern2pt]} \hypertarget{z`e\string_Mv\string_=\#\string_T1}{}
\markboth{\textcolor{darkblue}{\textbf{\ipa{ʐe˧v̩\#˥}}}}{}
\textcolor{teal}{\mytextsc{noun}} \hspace{4pt} Tone: \#H.
\textcolor{Sepia}{\selectlanguage{english}Castrated ox, neutered ox, steer.} \zh{阉牛。}  \zh{量词}: \textcolor{darkblue}{\textbf{\ipa{pʰo˧˥}}}  \mytextsc{clf}: \textcolor{darkblue}{\textbf{\ipa{pʰo˧˥}}} 
\lhead{\firstmark}
\rhead{\botmark}

\subsection{\hspace{-0.5cm} {\Large \textcolor{darkblue}{\textbf{\ipa{ʐe˧zo\#˥}}}}\hspace{0.5cm}[\kern2pt{\textcolor{darkblue}{\textbf{\ipa{ʐe˧zo˧}}}}\kern2pt]} \hypertarget{z`e\string_Mzo\#\string_T1}{}
\markboth{\textcolor{darkblue}{\textbf{\ipa{ʐe˧zo\#˥}}}}{}
\textcolor{teal}{\mytextsc{noun}} \hspace{4pt} Tone: \#H.
\textcolor{Sepia}{\selectlanguage{english}Arrow.} \zh{箭。}  ¶ \textcolor{darkblue}{\textbf{\ipa{ʐe˧zo˧ | ɖɯ˧-kʰɯ˩}}} \textcolor{Sepia}{\selectlanguage{english}one arrow} \zh{一枝箭}  

\lhead{\firstmark}
\rhead{\botmark}

\subsection{\hspace{-0.5cm} {\Large \textcolor{darkblue}{\textbf{\ipa{ʐe˩ʐe˧-bæ˩bæ˩}}}}\hspace{0.5cm}[\kern2pt{\textcolor{darkblue}{\textbf{\ipa{ʐe˩ʐe˧bæ˩bæ˩}}}}\kern2pt]} \hypertarget{z`e\string_Bz`e\string_M-b\{\string_Bb\{\string_B1}{}
\markboth{\textcolor{darkblue}{\textbf{\ipa{ʐe˩ʐe˧-bæ˩bæ˩}}}}{}
\textcolor{teal}{\mytextsc{noun}} \hspace{4pt} Tone: LM-L.
\textcolor{Sepia}{\selectlanguage{english}Wild cotton (literally: “Westerners' flower”).} \zh{野棉花(直译:‘洋人花’)。} \textit{See:} \hyperlink{}{\textcolor{darkblue}{\textbf{\ipa{je˩ʐe˧}}}} 
\lhead{\firstmark}
\rhead{\botmark}

\subsection{\hspace{-0.5cm} {\Large \textcolor{darkblue}{\textbf{\ipa{ʐe˩ʐe˧-læ˧tsɯ˥}}}}\hspace{0.5cm}[\kern2pt{\textcolor{darkblue}{\textbf{\ipa{ʐe˩ʐe˧læ˧tsɯ˥}}}}\kern2pt]} \hypertarget{z`e\string_Bz`e\string_M-l\{\string_MtsM\string_T1}{}
\markboth{\textcolor{darkblue}{\textbf{\ipa{ʐe˩ʐe˧-læ˧tsɯ˥}}}}{}
\textcolor{teal}{\mytextsc{noun}} \hspace{4pt} Tone: LM-H\#.
\textcolor{Sepia}{\selectlanguage{english}One of the three main types of plants used for pig fodder.} \zh{喂猪的牧草。} 
\lhead{\firstmark}
\rhead{\botmark}

\subsection{\hspace{-0.5cm} {\Large \textcolor{darkblue}{\textbf{\ipa{ʐɤ˧\textsubscript{b}}}}}\hspace{0.5cm}[\kern2pt{\textcolor{darkblue}{\textbf{\ipa{ʐɤ˩˥}}}}\kern2pt]} \hypertarget{z`7\string_Mb1}{}
\markboth{\textcolor{darkblue}{\textbf{\ipa{ʐɤ˧\textsubscript{b}}}}}{}
\textcolor{teal}{\mytextsc{verb}} \hspace{4pt} Tone: M\textsubscript{b}.
\textcolor{Sepia}{\selectlanguage{english}To raise (animals, or children); to care for (the elderly).} \zh{饲养(动物)、养(孩子)、管(老人)。}  ¶ \textcolor{darkblue}{\textbf{\ipa{bo˩ ʐɤ˧}}} \textcolor{Sepia}{\selectlanguage{english}to raise pigs} \zh{养猪}  
 ¶ \textcolor{darkblue}{\textbf{\ipa{ʐwæ˧zo˧ ʐɤ˧}}} \textcolor{Sepia}{\selectlanguage{english}to raise colts} \zh{养小马}  

\lhead{\firstmark}
\rhead{\botmark}

\subsection{\hspace{-0.5cm} {\Large \textcolor{darkblue}{\textbf{\ipa{ʐɤ˩\textsubscript{c}}}}}\hspace{0.5cm}[\kern2pt{\textcolor{darkblue}{\textbf{\ipa{ʐɤ˩˥}}}}\kern2pt]} \hypertarget{z`7\string_Bc1}{}
\markboth{\textcolor{darkblue}{\textbf{\ipa{ʐɤ˩\textsubscript{c}}}}}{}
\textcolor{teal}{\mytextsc{classifier}} \hspace{4pt} Tone: L\textsubscript{c}.
\textcolor{Sepia}{\selectlanguage{english}Classifier for lines/patterns (in weaving, drawing, painting…).} \zh{量词:图案(画画或织布)(一个)。} 
\lhead{\firstmark}
\rhead{\botmark}

\subsection{\hspace{-0.5cm} {\Large \textcolor{darkblue}{\textbf{\ipa{ʐɤ˩\textsubscript{a}}}}}\hspace{0.5cm}[\kern2pt{\textcolor{darkblue}{\textbf{\ipa{ʐɤ˩˥}}}}\kern2pt]} \hypertarget{z`7\string_Ba1}{}
\markboth{\textcolor{darkblue}{\textbf{\ipa{ʐɤ˩\textsubscript{a}}}}}{}
\textcolor{teal}{\mytextsc{adjective}} \hspace{4pt} Tone: L\textsubscript{a}.
\textcolor{Sepia}{\selectlanguage{english}Clean.} \zh{干净。}  ¶ \textcolor{darkblue}{\textbf{\ipa{ʈʂʰɯ˧ | ʐɤ˩-hĩ˩ ɲi˥}}} \textcolor{Sepia}{\selectlanguage{english}It is clean} \zh{这是干净的}  
 ¶ \textcolor{darkblue}{\textbf{\ipa{mɤ˧-ʐɤ˩}}} \textcolor{Sepia}{\selectlanguage{english}not clean, dirty} \zh{不干净}  

\lhead{\firstmark}
\rhead{\botmark}

\subsection{\hspace{-0.5cm} {\Large \textcolor{darkblue}{\textbf{\ipa{ʐɤ˩mi˩}}}}\hspace{0.5cm}[\kern2pt{\textcolor{darkblue}{\textbf{\ipa{ʐɤ˩mi˩˥}}}}\kern2pt]} \hypertarget{z`7\string_Bmi\string_B1}{}
\markboth{\textcolor{darkblue}{\textbf{\ipa{ʐɤ˩mi˩}}}}{}
\textcolor{teal}{\mytextsc{noun}} \hspace{4pt} Tone: L.
\textcolor{Sepia}{\selectlanguage{english}Road.} \zh{路。}  ¶ \textcolor{darkblue}{\textbf{\ipa{hĩ˧ | ɖɯ˧-v̩˧\textasciitilde{}ɖɯ˧-v̩˧ | le˧-se˥, | ʐɤ˩mi˩ tsɤ˩˥!}}} \textcolor{Sepia}{\selectlanguage{english}People walk, one after the other, and they create a path! (Context: when people go to fell trees on the mountain, where there was no path before, their passage open a new path, whose traces remain visible and may become a customary path.)} \zh{路是人走出来的!}  
 \zh{量词}: \textcolor{darkblue}{\textbf{\ipa{kʰɯ˩}}}  \mytextsc{clf}: \textcolor{darkblue}{\textbf{\ipa{kʰɯ˩}}} 
\lhead{\firstmark}
\rhead{\botmark}

\subsection{\hspace{-0.5cm} {\Large \textcolor{darkblue}{\textbf{\ipa{ʐɤ˩ni˩}}}}\hspace{0.5cm}[\kern2pt{\textcolor{darkblue}{\textbf{\ipa{ʐɤ˩ni˩˥}}}}\kern2pt]} \hypertarget{z`7\string_Bni\string_B1}{}
\markboth{\textcolor{darkblue}{\textbf{\ipa{ʐɤ˩ni˩}}}}{}
\textcolor{teal}{\mytextsc{adjective}} \hspace{4pt} Tone: L.
\textcolor{Sepia}{\selectlanguage{english}Near.} \zh{近。} 
\lhead{\firstmark}
\rhead{\botmark}

\subsection{\hspace{-0.5cm} {\Large \textcolor{darkblue}{\textbf{\ipa{ʐɤ˩qo˩}}}}\hspace{0.5cm}[\kern2pt{\textcolor{darkblue}{\textbf{\ipa{ʐɤ˩qo˩˥}}}}\kern2pt]} \hypertarget{z`7\string_Bqo\string_B1}{}
\markboth{\textcolor{darkblue}{\textbf{\ipa{ʐɤ˩qo˩}}}}{}
\textcolor{teal}{\mytextsc{noun}} \hspace{4pt} Tone: L.
\ding{202} \textcolor{Sepia}{\selectlanguage{english}Calf.} \zh{小牛。}  \zh{量词}: \textcolor{darkblue}{\textbf{\ipa{ɭɯ˧}}} \ding{203} \textcolor{Sepia}{\selectlanguage{english}Male pianniu (hybrid of yak and cattle).} \zh{公犏牛。}  \mytextsc{clf}: \textcolor{darkblue}{\textbf{\ipa{ɭɯ˧}}} 
\lhead{\firstmark}
\rhead{\botmark}

\subsection{\hspace{-0.5cm} {\Large \textcolor{darkblue}{\textbf{\ipa{ʐɤ˩ʐɤ˧˥}}}}\hspace{0.5cm}[\kern2pt{\textcolor{darkblue}{\textbf{\ipa{ʐɤ˩ʐɤ˧˥}}}}\kern2pt]} \hypertarget{z`7\string_Bz`7\string_M\string_T1}{}
\markboth{\textcolor{darkblue}{\textbf{\ipa{ʐɤ˩ʐɤ˧˥}}}}{}
\textcolor{teal}{\mytextsc{noun}} \hspace{4pt} Tone: LM+MH\#.
\textcolor{Sepia}{\selectlanguage{english}Lines, pattern.} \zh{花纹、图案。}  ¶ \textcolor{darkblue}{\textbf{\ipa{ʐɤ˩ʐɤ˧ tʰi˧-di˥}}} \textcolor{Sepia}{\selectlanguage{english}with lines/patterns / there are lines/patterns} \zh{有花纹}  
 ¶ \textcolor{darkblue}{\textbf{\ipa{[F5] bɑ˩lɑ˩˥ | ʈʰɯ˧-ɭɯ˥-bi˩ | ʐɤ˩ʐɤ˧ tʰi˧-di˥}}} \textcolor{Sepia}{\selectlanguage{english}There are lines/patterns on this piece of clothing.} \zh{这衣服上面有花纹。}  
 \zh{量词}: \textcolor{darkblue}{\textbf{\ipa{ʐɤ˩}}}  \mytextsc{clf}: \textcolor{darkblue}{\textbf{\ipa{ʐɤ˩}}} 
\lhead{\firstmark}
\rhead{\botmark}

\subsection{\hspace{-0.5cm} {\Large \textcolor{darkblue}{\textbf{\ipa{ʐɤ˩˧}}}}\hspace{0.5cm}[\kern2pt{\textcolor{darkblue}{\textbf{\ipa{ʐɤ˥}}}}\kern2pt]} \hypertarget{z`7\string_B\string_M1}{}
\markboth{\textcolor{darkblue}{\textbf{\ipa{ʐɤ˩˧}}}}{}
\textcolor{teal}{\mytextsc{noun}} \hspace{4pt} Tone: LM.
\textcolor{Sepia}{\selectlanguage{english}Road (monosyllable).} \zh{路(单音节)。}  ¶ \textcolor{darkblue}{\textbf{\ipa{ʐɤ˩mi˩-qo˥}}} \textcolor{Sepia}{\selectlanguage{english}on the road, on the way} \zh{路上}  
 ¶ \textcolor{darkblue}{\textbf{\ipa{ʐɤ˩mi˩-qo˥, | hĩ˧ se˧! |}}} \textcolor{Sepia}{\selectlanguage{english}People are walking on the road/path!} \zh{路上有人走!}  
 ¶ \textcolor{darkblue}{\textbf{\ipa{ʐɤ˩ se˩-zo˩˥}}} \textcolor{Sepia}{\selectlanguage{english}traveller, person who travels; specifically: person who does commerce by caravans} \zh{旅人,特别指走马帮的商人}  
 \zh{量词}: \textcolor{darkblue}{\textbf{\ipa{kʰɯ˩}}}  \mytextsc{clf}: \textcolor{darkblue}{\textbf{\ipa{kʰɯ˩}}} 
\lhead{\firstmark}
\rhead{\botmark}

\subsection{\hspace{-0.5cm} {\Large \textcolor{darkblue}{\textbf{\ipa{ʐo˩}}}}\hspace{0.5cm}[\kern2pt{\textcolor{darkblue}{\textbf{\ipa{ʐo˥}}}}\kern2pt]} \hypertarget{z`o\string_B1}{}
\markboth{\textcolor{darkblue}{\textbf{\ipa{ʐo˩}}}}{}
\textcolor{teal}{\mytextsc{noun}} \hspace{4pt} Tone: L.
\textcolor{Sepia}{\selectlanguage{english}Noon; lunch.} \zh{中午。}  ¶ \textcolor{darkblue}{\textbf{\ipa{ʐo˩ dzɯ˩˥}}} \textcolor{Sepia}{\selectlanguage{english}to have lunch} \zh{吃午饭}  

\lhead{\firstmark}
\rhead{\botmark}

\subsection{\hspace{-0.5cm} {\Large \textcolor{darkblue}{\textbf{\ipa{ʐo˩\textsubscript{a}}}} \textsubscript{1}}\hspace{0.5cm}[\kern2pt{\textcolor{darkblue}{\textbf{\ipa{ʐo˥}}}}\kern2pt]} \hypertarget{z`o\string_Ba1}{}
\markboth{\textcolor{darkblue}{\textbf{\ipa{ʐo˩\textsubscript{a}}}} \textsubscript{1}}{}
\textcolor{teal}{\mytextsc{verb}} \hspace{4pt} Tone: L\textsubscript{a}.
\textcolor{Sepia}{\selectlanguage{english}To swing back and forth.} \zh{甩来甩去。}  ¶ \textcolor{darkblue}{\textbf{\ipa{ɖɯ˧-ʐo˩-ɻ̍˩}}} \textcolor{Sepia}{\selectlanguage{english}to swing back and forth} \zh{甩来甩去}  
 ¶ \textcolor{darkblue}{\textbf{\ipa{ʐo˩\textasciitilde{}ʐo˧-ze˥}}} \textcolor{Sepia}{\selectlanguage{english}\mytextsc{red} \mytextsc{pfv}} \zh{\mytextsc{red} \mytextsc{pfv}}  
 ¶ \textcolor{darkblue}{\textbf{\ipa{[PHONO] le˧-ʐo˩\textasciitilde{}ʐo˩}}} \textcolor{Sepia}{\selectlanguage{english}\mytextsc{accomp} \mytextsc{red}} \zh{\mytextsc{accomp} \mytextsc{red}}  

\lhead{\firstmark}
\rhead{\botmark}

\subsection{\hspace{-0.5cm} {\Large \textcolor{darkblue}{\textbf{\ipa{ʐo˩\textsubscript{a}}}} \textsubscript{2}}\hspace{0.5cm}[\kern2pt{\textcolor{darkblue}{\textbf{\ipa{ʐo˩˥}}}}\kern2pt]} \hypertarget{z`o\string_Ba2}{}
\markboth{\textcolor{darkblue}{\textbf{\ipa{ʐo˩\textsubscript{a}}}} \textsubscript{2}}{}
\textcolor{teal}{\mytextsc{adjective}} \hspace{4pt} Tone: L\textsubscript{a}.
\textcolor{Sepia}{\selectlanguage{english}Light.} \zh{轻。} 
\lhead{\firstmark}
\rhead{\botmark}

\subsection{\hspace{-0.5cm} {\Large \textcolor{darkblue}{\textbf{\ipa{ʐo˩dzɯ˩}}}}\hspace{0.5cm}[\kern2pt{\textcolor{darkblue}{\textbf{\ipa{ʐo˩dzɯ˩˥}}}}\kern2pt]} \hypertarget{z`o\string_BdzM\string_B1}{}
\markboth{\textcolor{darkblue}{\textbf{\ipa{ʐo˩dzɯ˩}}}}{}
\textcolor{teal}{\mytextsc{verb}} \hspace{4pt} Tone: L.
\textcolor{Sepia}{\selectlanguage{english}To eat lunch.} \zh{吃午饭。}  ¶ \textcolor{darkblue}{\textbf{\ipa{ʐo˩ dzɯ˩˥}}} \textcolor{Sepia}{\selectlanguage{english}to have lunch} \zh{吃午饭}  
 ¶ \textcolor{darkblue}{\textbf{\ipa{ʐo˩ dzɯ˩-se˥}}} \textcolor{Sepia}{\selectlanguage{english}afternoon} \zh{下午}  

\lhead{\firstmark}
\rhead{\botmark}

\subsection{\hspace{-0.5cm} {\Large \textcolor{darkblue}{\textbf{\ipa{ʐo˩\textasciitilde{}ʐo˧˥}}}}\hspace{0.5cm}[\kern2pt{\textcolor{darkblue}{\textbf{\ipa{ʐo˧ʐo˧˥}}}}\kern2pt]} \hypertarget{z`o\string_B~z`o\string_M\string_T1}{}
\markboth{\textcolor{darkblue}{\textbf{\ipa{ʐo˩\textasciitilde{}ʐo˧˥}}}}{}
\textcolor{teal}{\mytextsc{verb}} \hspace{4pt} Tone: MH.
\textcolor{Sepia}{\selectlanguage{english}To swing.} \zh{摔、摇摆。} 
\lhead{\firstmark}
\rhead{\botmark}

\subsection{\hspace{-0.5cm} {\Large \textcolor{darkblue}{\textbf{\ipa{ʐɯ˥}}}}\hspace{0.5cm}[\kern2pt{\textcolor{darkblue}{\textbf{\ipa{ʐɯ˥}}}}\kern2pt]} \hypertarget{z`M\string_T1}{}
\markboth{\textcolor{darkblue}{\textbf{\ipa{ʐɯ˥}}}}{}
\textcolor{teal}{\mytextsc{adjective}} \hspace{4pt} Tone: H.
\textcolor{Sepia}{\selectlanguage{english}Heavy.} \zh{重。} 
\lhead{\firstmark}
\rhead{\botmark}

\subsection{\hspace{-0.5cm} {\Large \textcolor{darkblue}{\textbf{\ipa{ʐɯ˧}}}}\hspace{0.5cm}[\kern2pt{\textcolor{darkblue}{\textbf{\ipa{ʐɯ˥}}}}\kern2pt]} \hypertarget{z`M\string_M1}{}
\markboth{\textcolor{darkblue}{\textbf{\ipa{ʐɯ˧}}}}{}
\textcolor{teal}{\mytextsc{noun}} \hspace{4pt} Tone: M.
\textcolor{Sepia}{\selectlanguage{english}Fermented alcohol, wine.} \zh{酒。}  ¶ \textcolor{darkblue}{\textbf{\ipa{ʐɯ˧ pʰv̩˧˥}}} \textcolor{Sepia}{\selectlanguage{english}to pour wine} \zh{斟酒}  

\lhead{\firstmark}
\rhead{\botmark}

\subsection{\hspace{-0.5cm} {\Large \textcolor{darkblue}{\textbf{\ipa{ʐɯ˧ɭɯ˧}}}}\hspace{0.5cm}[\kern2pt{\textcolor{darkblue}{\textbf{\ipa{ʐɯ˧ɭɯ˧}}}}\kern2pt]} \hypertarget{z`M\string_Ml\string_RM\string_M1}{}
\markboth{\textcolor{darkblue}{\textbf{\ipa{ʐɯ˧ɭɯ˧}}}}{}
\textcolor{teal}{\mytextsc{verb}} \hspace{4pt} Tone: M.
\textcolor{Sepia}{\selectlanguage{english}To shake (of earth), earthquake.} \zh{地震。}  ¶ \textcolor{darkblue}{\textbf{\ipa{ʐɯ˧ɭɯ˧-ze˧!}}} \textcolor{Sepia}{\selectlanguage{english}There is an earthquake!} \zh{地震了!}  

\lhead{\firstmark}
\rhead{\botmark}

\subsection{\hspace{-0.5cm} {\Large \textcolor{darkblue}{\textbf{\ipa{ʐɯ˧nɑ˩}}}}\hspace{0.5cm}[\kern2pt{\textcolor{darkblue}{\textbf{\ipa{ʐɯ˧nɑ˩}}}}\kern2pt]} \hypertarget{z`M\string_MnA\string_B1}{}
\markboth{\textcolor{darkblue}{\textbf{\ipa{ʐɯ˧nɑ˩}}}}{}
\textcolor{teal}{\mytextsc{noun}} \hspace{4pt} Tone: L\#.
\textcolor{Sepia}{\selectlanguage{english}Strong alcohol, high-quality alcohol.} \zh{醇酒,好酒。} 
\lhead{\firstmark}
\rhead{\botmark}

\subsection{\hspace{-0.5cm} {\Large \textcolor{darkblue}{\textbf{\ipa{ʐɯ˩dzi˥}}}}\hspace{0.5cm}[\kern2pt{\textcolor{darkblue}{\textbf{\ipa{ʐɯ˩dzi˥}}}}\kern2pt]} \hypertarget{z`M\string_Bdzi\string_T1}{}
\markboth{\textcolor{darkblue}{\textbf{\ipa{ʐɯ˩dzi˥}}}}{}
\textcolor{teal}{\mytextsc{noun}} \hspace{4pt} Tone: LH.
\textcolor{Sepia}{\selectlanguage{english}Cedar.} \zh{杉树。}  \zh{量词}: \textcolor{darkblue}{\textbf{\ipa{dzi˩}}}  \mytextsc{clf}: \textcolor{darkblue}{\textbf{\ipa{dzi˩}}} 
\lhead{\firstmark}
\rhead{\botmark}

\subsection{\hspace{-0.5cm} {\Large \textcolor{darkblue}{\textbf{\ipa{ʐɯ˩gv̩˩}}}}\hspace{0.5cm}[\kern2pt{\textcolor{darkblue}{\textbf{\ipa{ʐɯ˩gv̩˩˥}}}}\kern2pt]} \hypertarget{z`M\string_Bgv\string_=\string_B1}{}
\markboth{\textcolor{darkblue}{\textbf{\ipa{ʐɯ˩gv̩˩}}}}{}
\textcolor{teal}{\mytextsc{noun}} \hspace{4pt} Tone: L.
\textcolor{Sepia}{\selectlanguage{english}Boat.} \zh{船。}  ¶ \textcolor{darkblue}{\textbf{\ipa{ʐɯ˩gv̩˩ dzi˩˥}}} \textcolor{Sepia}{\selectlanguage{english}to sit in a boat} \zh{坐船}  
 \zh{量词}: \textcolor{darkblue}{\textbf{\ipa{ɭɯ˧}}} \textcolor{darkblue}{\textbf{\ipa{nɑ˧}}}  \mytextsc{clf}: \textcolor{darkblue}{\textbf{\ipa{ɭɯ˧}}} \textcolor{darkblue}{\textbf{\ipa{nɑ˧}}} 
\lhead{\firstmark}
\rhead{\botmark}

\subsection{\hspace{-0.5cm} {\Large \textcolor{darkblue}{\textbf{\ipa{ʐɯ˩-mo˧˥}}}}\hspace{0.5cm}[\kern2pt{\textcolor{darkblue}{\textbf{\ipa{xxxx non-correspondance entre le nombre de morphèmes et le nombre de tons de morphèmes}}}}\kern2pt]} \hypertarget{z`M\string_B-mo\string_M\string_T1}{}
\markboth{\textcolor{darkblue}{\textbf{\ipa{ʐɯ˩-mo˧˥}}}}{}
\textcolor{teal}{\mytextsc{noun}} \hspace{4pt} Tone: LM+MH\#.
\textcolor{Sepia}{\selectlanguage{english}“mushroom of the cedar tree”: a sort of mushroom often found close to cedar trees.} \zh{“杉树菌”:一种菌子。}  ¶ \textcolor{darkblue}{\textbf{\ipa{tʰo˧mo˩-ʐɯ˩mo˩}}} \textcolor{Sepia}{\selectlanguage{english}Pine-tree mushroom and cedar-tree mushroom} \zh{松树菌与杉树菌}  
\textit{See:} \hyperlink{}{\textcolor{darkblue}{\textbf{\ipa{ʐɯ˩dzi˥}}}} 
\lhead{\firstmark}
\rhead{\botmark}

\subsection{\hspace{-0.5cm} {\Large \textcolor{darkblue}{\textbf{\ipa{ʐɯ˩tse˧}}}}\hspace{0.5cm}[\kern2pt{\textcolor{darkblue}{\textbf{\ipa{ʐɯ˩tse˥}}}}\kern2pt]} \hypertarget{z`M\string_Btse\string_M1}{}
\markboth{\textcolor{darkblue}{\textbf{\ipa{ʐɯ˩tse˧}}}}{}
\textcolor{teal}{\mytextsc{noun}} \hspace{4pt} Tone: LM.
\textcolor{Sepia}{\selectlanguage{english}Mountain spirit.} \zh{山神。}  \zh{量词}: \textcolor{darkblue}{\textbf{\ipa{v̩˧}}}  \mytextsc{clf}: \textcolor{darkblue}{\textbf{\ipa{v̩˧}}} 
\lhead{\firstmark}
\rhead{\botmark}

\subsection{\hspace{-0.5cm} {\Large \textcolor{darkblue}{\textbf{\ipa{ʐɯ˩tse˧-mæ˧ʂæ˩}}}}\hspace{0.5cm}[\kern2pt{\textcolor{darkblue}{\textbf{\ipa{ʐɯ˩tse˧mæ˧ʂæ˩}}}}\kern2pt]} \hypertarget{z`M\string_Btse\string_M-m\{\string_Ms`\{\string_B1}{}
\markboth{\textcolor{darkblue}{\textbf{\ipa{ʐɯ˩tse˧-mæ˧ʂæ˩}}}}{}
\textcolor{teal}{\mytextsc{noun}} \hspace{4pt} Tone: LM-L\#.
\textcolor{Sepia}{\selectlanguage{english}Golden pheasant.} \zh{锦鸡。} Local Chinese dialect:\zh{山扎拉。}\textit{See:} \hyperlink{}{\textcolor{darkblue}{\textbf{\ipa{ʐɯ˩tse˧}}}} 
\lhead{\firstmark}
\rhead{\botmark}

\subsection{\hspace{-0.5cm} {\Large \textcolor{darkblue}{\textbf{\ipa{ʐɯ˩tsɯ˧}}}}\hspace{0.5cm}[\kern2pt{\textcolor{darkblue}{\textbf{\ipa{ʐɯ˩tsɯ˥}}}}\kern2pt]} \hypertarget{z`M\string_BtsM\string_M1}{}
\markboth{\textcolor{darkblue}{\textbf{\ipa{ʐɯ˩tsɯ˧}}}}{}
\textcolor{teal}{\mytextsc{noun}} \hspace{4pt} Tone: LM.
\textcolor{Sepia}{\selectlanguage{english}Days; life; time.} \zh{日子(汉语借词)。}  Borrowing: Chinese  \zh{日子}
 ¶ \textcolor{darkblue}{\textbf{\ipa{ʐɯ˩tsɯ˧ ʈʂɤ˧}}} \textcolor{Sepia}{\selectlanguage{english}to look for an auspicious date (for building a house or other important project)} \zh{算日子(为了选择吉利的一天)}  

\lhead{\firstmark}
\rhead{\botmark}

\subsection{\hspace{-0.5cm} {\Large \textcolor{darkblue}{\textbf{\ipa{ʐɯ˩tsɯ˧mɤ˩ʈʂʰɤ˩}}}}\hspace{0.5cm}[\kern2pt{\textcolor{darkblue}{\textbf{\ipa{xxxx non-correspondance entre le nombre de morphèmes et le nombre de tons de morphèmes}}}}\kern2pt]} \hypertarget{z`M\string_BtsM\string_Mm7\string_Bt`s`\string_h7\string_B1}{}
\markboth{\textcolor{darkblue}{\textbf{\ipa{ʐɯ˩tsɯ˧mɤ˩ʈʂʰɤ˩}}}}{}
\textcolor{teal}{\mytextsc{noun}} \hspace{4pt} Tone: LM-L.
\textcolor{Sepia}{\selectlanguage{english}Mattress.} \zh{褥子。}  \zh{量词}: \textcolor{darkblue}{\textbf{\ipa{tsʰi˥}}}  \mytextsc{clf}: \textcolor{darkblue}{\textbf{\ipa{tsʰi˥}}} 
\lhead{\firstmark}
\rhead{\botmark}

\subsection{\hspace{-0.5cm} {\Large \textcolor{darkblue}{\textbf{\ipa{ʐv̩˧}}}}\hspace{0.5cm}[\kern2pt{\textcolor{darkblue}{\textbf{\ipa{ʐv̩˥}}}}\kern2pt]} \hypertarget{z`v\string_=\string_M1}{}
\markboth{\textcolor{darkblue}{\textbf{\ipa{ʐv̩˧}}}}{}
\textcolor{teal}{\mytextsc{number}} \hspace{4pt} Tone: M? H\#?.
\textcolor{Sepia}{\selectlanguage{english}4.} \zh{4。} 
\lhead{\firstmark}
\rhead{\botmark}

\subsection{\hspace{-0.5cm} {\Large \textcolor{darkblue}{\textbf{\ipa{ʐv̩˧˥}}}}\hspace{0.5cm}[\kern2pt{\textcolor{darkblue}{\textbf{\ipa{ʐv̩˧˥}}}}\kern2pt]} \hypertarget{z`v\string_=\string_M\string_T1}{}
\markboth{\textcolor{darkblue}{\textbf{\ipa{ʐv̩˧˥}}}}{}
\textcolor{teal}{\mytextsc{verb}} \hspace{4pt} Tone: MH.
\textcolor{Sepia}{\selectlanguage{english}To sew.} \zh{缝。} 
\lhead{\firstmark}
\rhead{\botmark}

\subsection{\hspace{-0.5cm} {\Large \textcolor{darkblue}{\textbf{\ipa{ʐv̩˩\textsubscript{a}}}} \textsubscript{1}}\hspace{0.5cm}[\kern2pt{\textcolor{darkblue}{\textbf{\ipa{ʐv̩˩˥}}}}\kern2pt]} \hypertarget{z`v\string_=\string_Ba1}{}
\markboth{\textcolor{darkblue}{\textbf{\ipa{ʐv̩˩\textsubscript{a}}}} \textsubscript{1}}{}
\textcolor{teal}{\mytextsc{verb}} \hspace{4pt} Tone: L\textsubscript{a}.
\ding{202} \textcolor{Sepia}{\selectlanguage{english}To knead (dough).} \zh{揉(面)。}  ¶ \textcolor{darkblue}{\textbf{\ipa{pɤ˩jɤ˧ ʐv̩˥}}} \textcolor{Sepia}{\selectlanguage{english}to knead dough} \zh{揉面}  
 ¶ \textcolor{darkblue}{\textbf{\ipa{ʐv̩˧\textasciitilde{}ʐv̩˥}}} \textcolor{Sepia}{\selectlanguage{english}\mytextsc{red}} \zh{\mytextsc{重叠}}  
 ¶ \textcolor{darkblue}{\textbf{\ipa{ɖɯ˧-kʰwɤ˧ ʐv̩˥}}} \textcolor{Sepia}{\selectlanguage{english}to knead a piece (of dough)} \zh{揉一块(面团)}  
\ding{203} \textcolor{Sepia}{\selectlanguage{english}To crease, to crumple, to wrinkle.} \zh{皱(衣服)。}  ¶ \textcolor{darkblue}{\textbf{\ipa{bɑ˩lɑ˩ ʐv̩˥(-ze˩)}}} \textcolor{Sepia}{\selectlanguage{english}to crease clothes; the clothes have been creased; the clothes are creased} \zh{衣服皱了}  
 ¶ \textcolor{darkblue}{\textbf{\ipa{ʐv̩˧\textasciitilde{}ʐv̩˥}}} \textcolor{Sepia}{\selectlanguage{english}\mytextsc{red}} \zh{\mytextsc{重叠}}  
 ¶ \textcolor{darkblue}{\textbf{\ipa{le˧-ʐv̩˧\textasciitilde{}ʐv̩˥-ze˩}}} \textcolor{Sepia}{\selectlanguage{english}\mytextsc{accomp} \string_ \mytextsc{red} \mytextsc{pfv}} \zh{\mytextsc{accomp} \string_ \mytextsc{red} \mytextsc{pfv}}  

\lhead{\firstmark}
\rhead{\botmark}

\subsection{\hspace{-0.5cm} {\Large \textcolor{darkblue}{\textbf{\ipa{ʐv̩˩\textsubscript{a}}}} \textsubscript{2}}\hspace{0.5cm}[\kern2pt{\textcolor{darkblue}{\textbf{\ipa{ʐv̩˩˥}}}}\kern2pt]} \hypertarget{z`v\string_=\string_Ba2}{}
\markboth{\textcolor{darkblue}{\textbf{\ipa{ʐv̩˩\textsubscript{a}}}} \textsubscript{2}}{}
\textcolor{teal}{\mytextsc{adjective}} \hspace{4pt} Tone: L\textsubscript{a}.
\textcolor{Sepia}{\selectlanguage{english}Delicious, good (to the taste).} \zh{好吃。} 
\lhead{\firstmark}
\rhead{\botmark}

\subsection{\hspace{-0.5cm} {\Large \textcolor{darkblue}{\textbf{\ipa{ʐv̩˧bæ˧}}}}\hspace{0.5cm}[\kern2pt{\textcolor{darkblue}{\textbf{\ipa{ʐv̩˧bæ˧}}}}\kern2pt]} \hypertarget{z`v\string_=\string_Mb\{\string_M1}{}
\markboth{\textcolor{darkblue}{\textbf{\ipa{ʐv̩˧bæ˧}}}}{}
\textcolor{teal}{\mytextsc{noun}} \hspace{4pt} Tone: M.
\textcolor{Sepia}{\selectlanguage{english}Snake, serpent.} \zh{蛇。}  ¶ \textcolor{darkblue}{\textbf{\ipa{ʐv̩˧bæ˧ ɣɯ˩ pʰv̩˩}}} \textcolor{Sepia}{\selectlanguage{english}The snake sheds skin / exuviates} \zh{蛇蜕皮}  
 \zh{量词}: \textcolor{darkblue}{\textbf{\ipa{mi˩}}}  \mytextsc{clf}: \textcolor{darkblue}{\textbf{\ipa{mi˩}}} 
\lhead{\firstmark}
\rhead{\botmark}

\subsection{\hspace{-0.5cm} {\Large \textcolor{darkblue}{\textbf{\ipa{ʐv̩˧bæ˧-bv̩˧-hɑ\#˥}}}}\hspace{0.5cm}[\kern2pt{\textcolor{darkblue}{\textbf{\ipa{xxxx non-correspondance entre le nombre de morphèmes et le nombre de tons de morphèmes}}}}\kern2pt]} \hypertarget{z`v\string_=\string_Mb\{\string_M-bv\string_=\string_M-hA\#\string_T1}{}
\markboth{\textcolor{darkblue}{\textbf{\ipa{ʐv̩˧bæ˧-bv̩˧-hɑ\#˥}}}}{}
\textcolor{teal}{\mytextsc{noun}} \hspace{4pt} Tone: \#H.
\textcolor{Sepia}{\selectlanguage{english}One of the three types of pig fodder.} \zh{能喂给猪的三种草之一。}  \zh{量词}: \textcolor{darkblue}{\textbf{\ipa{qɑ˩}}}  \mytextsc{clf}: \textcolor{darkblue}{\textbf{\ipa{qɑ˩}}} 
\lhead{\firstmark}
\rhead{\botmark}

\subsection{\hspace{-0.5cm} {\Large \textcolor{darkblue}{\textbf{\ipa{ʐv̩˧bæ˧-pʰv̩\#˥}}}}\hspace{0.5cm}[\kern2pt{\textcolor{darkblue}{\textbf{\ipa{xxxx non-correspondance entre le nombre de morphèmes et le nombre de tons de morphèmes}}}}\kern2pt]} \hypertarget{z`v\string_=\string_Mb\{\string_M-p\string_hv\string_=\#\string_T1}{}
\markboth{\textcolor{darkblue}{\textbf{\ipa{ʐv̩˧bæ˧-pʰv̩\#˥}}}}{}
\textcolor{teal}{\mytextsc{noun}} \hspace{4pt} Tone: \#H.
\textcolor{Sepia}{\selectlanguage{english}Male snake.} \zh{公蛇。}  \zh{量词}: \textcolor{darkblue}{\textbf{\ipa{mi˩}}}  \mytextsc{clf}: \textcolor{darkblue}{\textbf{\ipa{mi˩}}} 
\lhead{\firstmark}
\rhead{\botmark}

\subsection{\hspace{-0.5cm} {\Large \textcolor{darkblue}{\textbf{\ipa{ʐv̩˧bæ˧-zo\#˥}}}}\hspace{0.5cm}[\kern2pt{\textcolor{darkblue}{\textbf{\ipa{xxxx non-correspondance entre le nombre de morphèmes et le nombre de tons de morphèmes}}}}\kern2pt]} \hypertarget{z`v\string_=\string_Mb\{\string_M-zo\#\string_T1}{}
\markboth{\textcolor{darkblue}{\textbf{\ipa{ʐv̩˧bæ˧-zo\#˥}}}}{}
\textcolor{teal}{\mytextsc{noun}} \hspace{4pt} Tone: \#H.
\textcolor{Sepia}{\selectlanguage{english}Baby snake.} \zh{小蛇。}  \zh{量词}: \textcolor{darkblue}{\textbf{\ipa{ɭɯ˧}}}  \mytextsc{clf}: \textcolor{darkblue}{\textbf{\ipa{ɭɯ˧}}} 
\lhead{\firstmark}
\rhead{\botmark}

\subsection{\hspace{-0.5cm} {\Large \textcolor{darkblue}{\textbf{\ipa{ʐv̩˧bɤ\#˥}}}}\hspace{0.5cm}[\kern2pt{\textcolor{darkblue}{\textbf{\ipa{ʐv̩˧bɤ˧}}}}\kern2pt]} \hypertarget{z`v\string_=\string_Mb7\#\string_T1}{}
\markboth{\textcolor{darkblue}{\textbf{\ipa{ʐv̩˧bɤ\#˥}}}}{}
\textcolor{teal}{\mytextsc{noun}} \hspace{4pt} Tone: \#H.
\textcolor{Sepia}{\selectlanguage{english}The Pumi (Prinmi) people of the mountains.} \zh{高山普米族(永宁以北地区:木里等)。}  \zh{量词}: \textcolor{darkblue}{\textbf{\ipa{v̩˧}}}  \mytextsc{clf}: \textcolor{darkblue}{\textbf{\ipa{v̩˧}}} 
\lhead{\firstmark}
\rhead{\botmark}

\subsection{\hspace{-0.5cm} {\Large \textcolor{darkblue}{\textbf{\ipa{ʐv̩˧di˧˥}}}}\hspace{0.5cm}[\kern2pt{\textcolor{darkblue}{\textbf{\ipa{ʐv̩˧di˧˥}}}}\kern2pt]} \hypertarget{z`v\string_=\string_Mdi\string_M\string_T1}{}
\markboth{\textcolor{darkblue}{\textbf{\ipa{ʐv̩˧di˧˥}}}}{}
\textcolor{teal}{\mytextsc{noun}} \hspace{4pt} Tone: MH\#.
\textcolor{Sepia}{\selectlanguage{english}The warm area on the banks of the Yangtze river: Fengke, Labai….} \zh{金沙江边的地方(气候热)。} 
\lhead{\firstmark}
\rhead{\botmark}

\subsection{\hspace{-0.5cm} {\Large \textcolor{darkblue}{\textbf{\ipa{ʐv̩˧dzi˩}}}}\hspace{0.5cm}[\kern2pt{\textcolor{darkblue}{\textbf{\ipa{ʐv̩˧dzi˩}}}}\kern2pt]} \hypertarget{z`v\string_=\string_Mdzi\string_B1}{}
\markboth{\textcolor{darkblue}{\textbf{\ipa{ʐv̩˧dzi˩}}}}{}
\textcolor{teal}{\mytextsc{noun}} \hspace{4pt} Tone: L\#.
\textcolor{Sepia}{\selectlanguage{english}Willow tree.} \zh{柳树,杨柳。}  \zh{量词}: \textcolor{darkblue}{\textbf{\ipa{dzi˩}}}  \mytextsc{clf}: \textcolor{darkblue}{\textbf{\ipa{dzi˩}}} 
\lhead{\firstmark}
\rhead{\botmark}

\subsection{\hspace{-0.5cm} {\Large \textcolor{darkblue}{\textbf{\ipa{ʐv̩˧hĩ\#˥}}}}\hspace{0.5cm}[\kern2pt{\textcolor{darkblue}{\textbf{\ipa{ʐv̩˧hĩ˧}}}}\kern2pt]} \hypertarget{z`v\string_=\string_Mhi\string_~\#\string_T1}{}
\markboth{\textcolor{darkblue}{\textbf{\ipa{ʐv̩˧hĩ\#˥}}}}{}
\textcolor{teal}{\mytextsc{noun}} \hspace{4pt} Tone: \#H.
\textcolor{Sepia}{\selectlanguage{english}One of the designations of the Pumi (ethnic group).} \zh{普米族。}  \zh{量词}: \textcolor{darkblue}{\textbf{\ipa{v̩˧}}}  \mytextsc{clf}: \textcolor{darkblue}{\textbf{\ipa{v̩˧}}} 
\lhead{\firstmark}
\rhead{\botmark}

\subsection{\hspace{-0.5cm} {\Large \textcolor{darkblue}{\textbf{\ipa{ʐv̩˩-ɬi˩mi˩}}}}\hspace{0.5cm}[\kern2pt{\textcolor{darkblue}{\textbf{\ipa{xxxx non-correspondance entre le nombre de morphèmes et le nombre de tons de morphèmes}}}}\kern2pt]} \hypertarget{z`v\string_=\string_B-Ki\string_Bmi\string_B1}{}
\markboth{\textcolor{darkblue}{\textbf{\ipa{ʐv̩˩-ɬi˩mi˩}}}}{}
\textcolor{teal}{\mytextsc{noun}} \hspace{4pt} Tone: L.
\textcolor{Sepia}{\selectlanguage{english}4th month.} \zh{四月。} 
\lhead{\firstmark}
\rhead{\botmark}

\subsection{\hspace{-0.5cm} {\Large \textcolor{darkblue}{\textbf{\ipa{ʐv̩˩ɭɯ˥}}}}\hspace{0.5cm}[\kern2pt{\textcolor{darkblue}{\textbf{\ipa{ʐv̩˩ɭɯ˥}}}}\kern2pt]} \hypertarget{z`v\string_=\string_Bl\string_RM\string_T1}{}
\markboth{\textcolor{darkblue}{\textbf{\ipa{ʐv̩˩ɭɯ˥}}}}{}
\textcolor{teal}{\mytextsc{noun}} \hspace{4pt} Tone: LH.
\textcolor{Sepia}{\selectlanguage{english}Beam.} \zh{支撑顶板的梁。}  \zh{量词}: \textcolor{darkblue}{\textbf{\ipa{ɭɯ˧}}}  \mytextsc{clf}: \textcolor{darkblue}{\textbf{\ipa{ɭɯ˧}}} 
\lhead{\firstmark}
\rhead{\botmark}

\subsection{\hspace{-0.5cm} {\Large \textcolor{darkblue}{\textbf{\ipa{ʐv̩˩mi˩}}}}\hspace{0.5cm}[\kern2pt{\textcolor{darkblue}{\textbf{\ipa{ʐv̩˩mi˩˥}}}}\kern2pt]} \hypertarget{z`v\string_=\string_Bmi\string_B1}{}
\markboth{\textcolor{darkblue}{\textbf{\ipa{ʐv̩˩mi˩}}}}{}
\textcolor{teal}{\mytextsc{noun}} \hspace{4pt} Tone: L.
\textcolor{Sepia}{\selectlanguage{english}Bow (archery bow).} \zh{弓。}  \zh{量词}: \textcolor{darkblue}{\textbf{\ipa{nɑ˧}}}  \mytextsc{clf}: \textcolor{darkblue}{\textbf{\ipa{nɑ˧}}} 
\lhead{\firstmark}
\rhead{\botmark}

\subsection{\hspace{-0.5cm} {\Large \textcolor{darkblue}{\textbf{\ipa{ʐv̩˧mi\#˥}}}}\hspace{0.5cm}[\kern2pt{\textcolor{darkblue}{\textbf{\ipa{ʐv̩˧mi˧}}}}\kern2pt]} \hypertarget{z`v\string_=\string_Mmi\#\string_T1}{}
\markboth{\textcolor{darkblue}{\textbf{\ipa{ʐv̩˧mi\#˥}}}}{}
\textcolor{teal}{\mytextsc{noun}} \hspace{4pt} Tone: \#H.
\textcolor{Sepia}{\selectlanguage{english}Granddaughter.} \zh{孙女。}  ¶ \textcolor{darkblue}{\textbf{\ipa{njɤ˧ | ʐv̩˧mi˧ | ɖɯ˧-ɭɯ˧ dʑo˧}}} \textcolor{Sepia}{\selectlanguage{english}I have a granddaughter.} \zh{我有一个孙女。}  
 \zh{量词}: \textcolor{darkblue}{\textbf{\ipa{ɭɯ˧}}}  \mytextsc{clf}: \textcolor{darkblue}{\textbf{\ipa{ɭɯ˧}}} 
\lhead{\firstmark}
\rhead{\botmark}

\subsection{\hspace{-0.5cm} {\Large \textcolor{darkblue}{\textbf{\ipa{ʐv̩˧mv̩˧lɑ˧di˧˥}}}}\hspace{0.5cm}[\kern2pt{\textcolor{darkblue}{\textbf{\ipa{ʐv̩˧mv̩˧lɑ˧di˧˥}}}}\kern2pt]} \hypertarget{z`v\string_=\string_Mmv\string_=\string_MlA\string_Mdi\string_M\string_T1}{}
\markboth{\textcolor{darkblue}{\textbf{\ipa{ʐv̩˧mv̩˧lɑ˧di˧˥}}}}{}
\textcolor{teal}{\mytextsc{noun}} \hspace{4pt} Tone: MH\#.
\textcolor{Sepia}{\selectlanguage{english}The territory of the Pumi people on the banks of the Yangtze river. This area is perceived as less central and pleasant than Yongning.} \zh{江边普米族地区(带偏见的说法)。}  ¶ \textcolor{darkblue}{\textbf{\ipa{ʐv̩˧mv̩˧lɑ˧di˧-hĩ˥}}} \textcolor{Sepia}{\selectlanguage{english}people from the Pumi territories} \zh{普米族地区的人们}  
 \zh{量词}: \textcolor{darkblue}{\textbf{\ipa{v̩˧}}}  \mytextsc{clf}: \textcolor{darkblue}{\textbf{\ipa{v̩˧}}} 
\lhead{\firstmark}
\rhead{\botmark}

\subsection{\hspace{-0.5cm} {\Large \textcolor{darkblue}{\textbf{\ipa{ʐv̩˧-ɲi˧-ʁo˧tʰo˥}}}}\hspace{0.5cm}[\kern2pt{\textcolor{darkblue}{\textbf{\ipa{xxxx non-correspondance entre le nombre de morphèmes et le nombre de tons de morphèmes}}}}\kern2pt]} \hypertarget{z`v\string_=\string_M-Ji\string_M-Ro\string_Mt\string_ho\string_T1}{}
\markboth{\textcolor{darkblue}{\textbf{\ipa{ʐv̩˧-ɲi˧-ʁo˧tʰo˥}}}}{}
\textcolor{teal}{\mytextsc{adverb(ial)}} \hspace{4pt} Tone: H\#.
\textcolor{Sepia}{\selectlanguage{english}In four days.} \zh{四天以后。} 
\lhead{\firstmark}
\rhead{\botmark}

\subsection{\hspace{-0.5cm} {\Large \textcolor{darkblue}{\textbf{\ipa{ʐv̩˧ɻ̍˥}}}}\hspace{0.5cm}[\kern2pt{\textcolor{darkblue}{\textbf{\ipa{ʐv̩˧ɻ̍˥}}}}\kern2pt]} \hypertarget{z`v\string_=\string_Mr£`̍\string_T1}{}
\markboth{\textcolor{darkblue}{\textbf{\ipa{ʐv̩˧ɻ̍˥}}}}{}
\textcolor{teal}{\mytextsc{adjective}} \hspace{4pt} Tone: H\#.
\textcolor{Sepia}{\selectlanguage{english}Square.} \zh{正方形。}  ¶ \textcolor{darkblue}{\textbf{\ipa{ʐv̩˩-hĩ˩˥}}} \textcolor{Sepia}{\selectlanguage{english}\mytextsc{nmlz}} \zh{方形的}  
 ¶ \textcolor{darkblue}{\textbf{\ipa{ʐv̩˧ɻ̍˥-gv̩˩}}} \textcolor{Sepia}{\selectlanguage{english}square} \zh{方形的}  

\lhead{\firstmark}
\rhead{\botmark}

\subsection{\hspace{-0.5cm} {\Large \textcolor{darkblue}{\textbf{\ipa{ʐv̩˧-tsʰi˩}}}}\hspace{0.5cm}[\kern2pt{\textcolor{darkblue}{\textbf{\ipa{xxxx non-correspondance entre le nombre de morphèmes et le nombre de tons de morphèmes}}}}\kern2pt]} \hypertarget{z`v\string_=\string_M-ts\string_hi\string_B1}{}
\markboth{\textcolor{darkblue}{\textbf{\ipa{ʐv̩˧-tsʰi˩}}}}{}
\textcolor{teal}{\mytextsc{number}} \hspace{4pt} Tone: L\#.
\textcolor{Sepia}{\selectlanguage{english}40.} \zh{40。} 
\lhead{\firstmark}
\rhead{\botmark}

\subsection{\hspace{-0.5cm} {\Large \textcolor{darkblue}{\textbf{\ipa{ʐv̩˧v̩˥-ʐv̩˩mi˩}}}}\hspace{0.5cm}[\kern2pt{\textcolor{darkblue}{\textbf{\ipa{ʐv̩˧v̩˥ʐv̩˩mi˩}}}}\kern2pt]} \hypertarget{z`v\string_=\string_Mv\string_=\string_T-z`v\string_=\string_Bmi\string_B1}{}
\markboth{\textcolor{darkblue}{\textbf{\ipa{ʐv̩˧v̩˥-ʐv̩˩mi˩}}}}{}
\textcolor{teal}{\mytextsc{noun}} \hspace{4pt} Tone: H\#-.
\textcolor{Sepia}{\selectlanguage{english}Grandchildren.} \zh{孙子孙女。} 
\lhead{\firstmark}
\rhead{\botmark}

\subsection{\hspace{-0.5cm} {\Large \textcolor{darkblue}{\textbf{\ipa{ʐv̩˧v̩\#˥}}}}\hspace{0.5cm}[\kern2pt{\textcolor{darkblue}{\textbf{\ipa{ʐv̩˧v̩˧}}}}\kern2pt]} \hypertarget{z`v\string_=\string_Mv\string_=\#\string_T1}{}
\markboth{\textcolor{darkblue}{\textbf{\ipa{ʐv̩˧v̩\#˥}}}}{}
\textcolor{teal}{\mytextsc{noun}} \hspace{4pt} Tone: \#H.
\textcolor{Sepia}{\selectlanguage{english}Grandson.} \zh{孙子。}  ¶ \textcolor{darkblue}{\textbf{\ipa{njɤ˧ | ʐv̩˧v̩˧ ɖɯ˧-ɭɯ˧ dʑo˧.}}} \textcolor{Sepia}{\selectlanguage{english}I have a grandson.} \zh{我有一个孙子。}  
 \zh{量词}: \textcolor{darkblue}{\textbf{\ipa{ɭɯ˧}}}  \mytextsc{clf}: \textcolor{darkblue}{\textbf{\ipa{ɭɯ˧}}} 
\lhead{\firstmark}
\rhead{\botmark}

\subsection{\hspace{-0.5cm} {\Large \textcolor{darkblue}{\textbf{\ipa{ʐv̩˧-zo\#˥}}}}\hspace{0.5cm}[\kern2pt{\textcolor{darkblue}{\textbf{\ipa{xxxx non-correspondance entre le nombre de morphèmes et le nombre de tons de morphèmes}}}}\kern2pt]} \hypertarget{z`v\string_=\string_M-zo\#\string_T1}{}
\markboth{\textcolor{darkblue}{\textbf{\ipa{ʐv̩˧-zo\#˥}}}}{}
\textcolor{teal}{\mytextsc{noun}} \hspace{4pt} Tone: \#H.
\textcolor{Sepia}{\selectlanguage{english}Small bow (archery bow).} \zh{小弓。}  \zh{量词}: \textcolor{darkblue}{\textbf{\ipa{nɑ˧}}}  \mytextsc{clf}: \textcolor{darkblue}{\textbf{\ipa{nɑ˧}}} 
\lhead{\firstmark}
\rhead{\botmark}

\subsection{\hspace{-0.5cm} {\Large \textcolor{darkblue}{\textbf{\ipa{ʐwæ˥}}}}\hspace{0.5cm}[\kern2pt{\textcolor{darkblue}{\textbf{\ipa{ʐwæ˥}}}}\kern2pt]} \hypertarget{z`w\{\string_T1}{}
\markboth{\textcolor{darkblue}{\textbf{\ipa{ʐwæ˥}}}}{}
\textcolor{teal}{\mytextsc{noun}} \hspace{4pt} Tone: \#H.
\textcolor{Sepia}{\selectlanguage{english}Horse.} \zh{马。}  ¶ \textcolor{darkblue}{\textbf{\ipa{dʑɯ˩ʁo˩-ʐwæ˩}}} \textcolor{Sepia}{\selectlanguage{english}wild horse} \zh{野马}  
 ¶ \textcolor{darkblue}{\textbf{\ipa{o-ho-ho! ʐwæ˧-ɳɯ˩ | dzɯ˧-po˧-hɯ˥-ze˩!}}} \textcolor{Sepia}{\selectlanguage{english}Oops! The horse scoffed the lot!} \zh{啊呀嚒!马把饲料都吃光了!}  
 \zh{量词}: \textcolor{darkblue}{\textbf{\ipa{v̩˧}}}  \mytextsc{clf}: \textcolor{darkblue}{\textbf{\ipa{v̩˧}}} 
\lhead{\firstmark}
\rhead{\botmark}

\subsection{\hspace{-0.5cm} {\Large \textcolor{darkblue}{\textbf{\ipa{ʐwæ˧\textsubscript{a}}}}}\hspace{0.5cm}[\kern2pt{\textcolor{darkblue}{\textbf{\ipa{ʐwæ˥}}}}\kern2pt]} \hypertarget{z`w\{\string_Ma1}{}
\markboth{\textcolor{darkblue}{\textbf{\ipa{ʐwæ˧\textsubscript{a}}}}}{}
\textcolor{teal}{\mytextsc{verb}} \hspace{4pt} Tone: M\textsubscript{a}.
\textcolor{Sepia}{\selectlanguage{english}To weigh (with scales).} \zh{称。}  ¶ \textcolor{darkblue}{\textbf{\ipa{mɤ˧-ʐwæ˧}}} \textcolor{Sepia}{\selectlanguage{english}\mytextsc{neg}} \zh{不称}  
 ¶ \textcolor{darkblue}{\textbf{\ipa{le˧-ʐwæ˧-ze˧}}} \textcolor{Sepia}{\selectlanguage{english}\mytextsc{accomp} \string_ \mytextsc{pfv}} \zh{称了}  
 ¶ \textcolor{darkblue}{\textbf{\ipa{tso˧\textasciitilde{}tso˧ ʐwæ˩}}} \textcolor{Sepia}{\selectlanguage{english}to weigh things} \zh{称东西}  
 ¶ \textcolor{darkblue}{\textbf{\ipa{ʁo˧do˧ ʐwæ˧}}} \textcolor{Sepia}{\selectlanguage{english}to weigh walnuts} \zh{称核桃}  

\lhead{\firstmark}
\rhead{\botmark}

\subsection{\hspace{-0.5cm} {\Large \textcolor{darkblue}{\textbf{\ipa{ʐwæ˧bv̩˧˥}}}}\hspace{0.5cm}[\kern2pt{\textcolor{darkblue}{\textbf{\ipa{ʐwæ˧bv̩˧˥}}}}\kern2pt]} \hypertarget{z`w\{\string_Mbv\string_=\string_M\string_T1}{}
\markboth{\textcolor{darkblue}{\textbf{\ipa{ʐwæ˧bv̩˧˥}}}}{}
\textcolor{teal}{\mytextsc{noun}} \hspace{4pt} Tone: MH\#.
\textcolor{Sepia}{\selectlanguage{english}Horse's stable.} \zh{马圈。}  \zh{量词}: \textcolor{darkblue}{\textbf{\ipa{ɭɯ˧}}}  \mytextsc{clf}: \textcolor{darkblue}{\textbf{\ipa{ɭɯ˧}}} 
\lhead{\firstmark}
\rhead{\botmark}

\subsection{\hspace{-0.5cm} {\Large \textcolor{darkblue}{\textbf{\ipa{ʐwæ˧-hɑ\#˥}}}}\hspace{0.5cm}[\kern2pt{\textcolor{darkblue}{\textbf{\ipa{xxxx non-correspondance entre le nombre de morphèmes et le nombre de tons de morphèmes}}}}\kern2pt]} \hypertarget{z`w\{\string_M-hA\#\string_T1}{}
\markboth{\textcolor{darkblue}{\textbf{\ipa{ʐwæ˧-hɑ\#˥}}}}{}
\textcolor{teal}{\mytextsc{noun}} \hspace{4pt} Tone: \#H.
\textcolor{Sepia}{\selectlanguage{english}Horse feed.} \zh{马料、马饲料。} 
\lhead{\firstmark}
\rhead{\botmark}

\subsection{\hspace{-0.5cm} {\Large \textcolor{darkblue}{\textbf{\ipa{ʐwæ˧-kʰv̩˩}}}}\hspace{0.5cm}[\kern2pt{\textcolor{darkblue}{\textbf{\ipa{xxxx non-correspondance entre le nombre de morphèmes et le nombre de tons de morphèmes}}}}\kern2pt]} \hypertarget{z`w\{\string_M-k\string_hv\string_=\string_B1}{}
\markboth{\textcolor{darkblue}{\textbf{\ipa{ʐwæ˧-kʰv̩˩}}}}{}
\textcolor{teal}{\mytextsc{noun}} \hspace{4pt} Tone: L\#.
\textcolor{Sepia}{\selectlanguage{english}Year of the Horse.} \zh{马年。} 
\lhead{\firstmark}
\rhead{\botmark}

\subsection{\hspace{-0.5cm} {\Large \textcolor{darkblue}{\textbf{\ipa{ʐwæ˧-ɭɯ\#˥}}}}\hspace{0.5cm}[\kern2pt{\textcolor{darkblue}{\textbf{\ipa{xxxx non-correspondance entre le nombre de morphèmes et le nombre de tons de morphèmes}}}}\kern2pt]} \hypertarget{z`w\{\string_M-l\string_RM\#\string_T1}{}
\markboth{\textcolor{darkblue}{\textbf{\ipa{ʐwæ˧-ɭɯ\#˥}}}}{}
\textcolor{teal}{\mytextsc{noun}} \hspace{4pt} Tone: \#H.
\textcolor{Sepia}{\selectlanguage{english}Cereals for the horse, horse fodder.} \zh{马料(粮食)。} 
\lhead{\firstmark}
\rhead{\botmark}

\subsection{\hspace{-0.5cm} {\Large \textcolor{darkblue}{\textbf{\ipa{ʐwæ˧pʰæ˧di˧˥}}}}\hspace{0.5cm}[\kern2pt{\textcolor{darkblue}{\textbf{\ipa{ʐwæ˧pʰæ˧di˧˥}}}}\kern2pt]} \hypertarget{z`w\{\string_Mp\string_h\{\string_Mdi\string_M\string_T1}{}
\markboth{\textcolor{darkblue}{\textbf{\ipa{ʐwæ˧pʰæ˧di˧˥}}}}{}
\textcolor{teal}{\mytextsc{noun}} \hspace{4pt} Tone: MH\#.
\textcolor{Sepia}{\selectlanguage{english}Lunge, tether (for a horse).} \zh{拉马链子。}  \zh{量词}: \textcolor{darkblue}{\textbf{\ipa{ɭɯ˧}}}  \mytextsc{clf}: \textcolor{darkblue}{\textbf{\ipa{ɭɯ˧}}} 
\lhead{\firstmark}
\rhead{\botmark}

\subsection{\hspace{-0.5cm} {\Large \textcolor{darkblue}{\textbf{\ipa{ʐwæ˧-qʰæ\#˥}}}}\hspace{0.5cm}[\kern2pt{\textcolor{darkblue}{\textbf{\ipa{xxxx non-correspondance entre le nombre de morphèmes et le nombre de tons de morphèmes}}}}\kern2pt]} \hypertarget{z`w\{\string_M-q\string_h\{\#\string_T1}{}
\markboth{\textcolor{darkblue}{\textbf{\ipa{ʐwæ˧-qʰæ\#˥}}}}{}
\textcolor{teal}{\mytextsc{noun}} \hspace{4pt} Tone: \#H.
\textcolor{Sepia}{\selectlanguage{english}Horse manure.} \zh{马粪。}  \zh{量词}: \textcolor{darkblue}{\textbf{\ipa{ʁwɤ˧}}}  \mytextsc{clf}: \textcolor{darkblue}{\textbf{\ipa{ʁwɤ˧}}} 
\lhead{\firstmark}
\rhead{\botmark}

\subsection{\hspace{-0.5cm} {\Large \textcolor{darkblue}{\textbf{\ipa{ʐwæ˧ʁo˩}}}}\hspace{0.5cm}[\kern2pt{\textcolor{darkblue}{\textbf{\ipa{ʐwæ˧ʁo˩}}}}\kern2pt]} \hypertarget{z`w\{\string_MRo\string_B1}{}
\markboth{\textcolor{darkblue}{\textbf{\ipa{ʐwæ˧ʁo˩}}}}{}
\textcolor{teal}{\mytextsc{noun}} \hspace{4pt} Tone: L\#.
\textcolor{Sepia}{\selectlanguage{english}Castrated horse, gelding, neutered horse.} \zh{骟马。}  \zh{量词}: \textcolor{darkblue}{\textbf{\ipa{v̩˧}}}  \mytextsc{clf}: \textcolor{darkblue}{\textbf{\ipa{v̩˧}}} 
\lhead{\firstmark}
\rhead{\botmark}

\subsection{\hspace{-0.5cm} {\Large \textcolor{darkblue}{\textbf{\ipa{ʐwæ˧sɯ˩}}}}\hspace{0.5cm}[\kern2pt{\textcolor{darkblue}{\textbf{\ipa{ʐwæ˧sɯ˩}}}}\kern2pt]} \hypertarget{z`w\{\string_MsM\string_B1}{}
\markboth{\textcolor{darkblue}{\textbf{\ipa{ʐwæ˧sɯ˩}}}}{}
\textcolor{teal}{\mytextsc{noun}} \hspace{4pt} Tone: L\#.
\textcolor{Sepia}{\selectlanguage{english}Stallion.} \zh{公马。}  ¶ \textcolor{darkblue}{\textbf{\ipa{ʐwæ˧sɯ˩-ʐwæ˩mi˩}}} \textcolor{Sepia}{\selectlanguage{english}stallion and mare} \zh{公马与母马}  
 ¶ \textcolor{darkblue}{\textbf{\ipa{ʐwæ˧sɯ˩-ʐwæ˩zo˩}}} \textcolor{Sepia}{\selectlanguage{english}stallion and colt} \zh{公马与小马}  
 \zh{量词}: \textcolor{darkblue}{\textbf{\ipa{mi˩}}}  \mytextsc{clf}: \textcolor{darkblue}{\textbf{\ipa{mi˩}}} 
\lhead{\firstmark}
\rhead{\botmark}

\subsection{\hspace{-0.5cm} {\Large \textcolor{darkblue}{\textbf{\ipa{ʐwæ˧zo\#˥}}}}\hspace{0.5cm}[\kern2pt{\textcolor{darkblue}{\textbf{\ipa{ʐwæ˧zo˧}}}}\kern2pt]} \hypertarget{z`w\{\string_Mzo\#\string_T1}{}
\markboth{\textcolor{darkblue}{\textbf{\ipa{ʐwæ˧zo\#˥}}}}{}
\textcolor{teal}{\mytextsc{noun}} \hspace{4pt} Tone: \#H.
\textcolor{Sepia}{\selectlanguage{english}Colt, pony, filly, foal.} \zh{马驹子。}  ¶ \textcolor{darkblue}{\textbf{\ipa{ʐwæ˧zo˧-ʐwæ˥mi˩}}} \textcolor{Sepia}{\selectlanguage{english}colt and mare} \zh{马驹子与母马}  
 \zh{量词}: \textcolor{darkblue}{\textbf{\ipa{ɭɯ˧}}}  \mytextsc{clf}: \textcolor{darkblue}{\textbf{\ipa{ɭɯ˧}}} 
\lhead{\firstmark}
\rhead{\botmark}

\subsection{\hspace{-0.5cm} {\Large \textcolor{darkblue}{\textbf{\ipa{ʐwæ˧-zɯ\#˥}}}}\hspace{0.5cm}[\kern2pt{\textcolor{darkblue}{\textbf{\ipa{xxxx non-correspondance entre le nombre de morphèmes et le nombre de tons de morphèmes}}}}\kern2pt]} \hypertarget{z`w\{\string_M-zM\#\string_T1}{}
\markboth{\textcolor{darkblue}{\textbf{\ipa{ʐwæ˧-zɯ\#˥}}}}{}
\textcolor{teal}{\mytextsc{noun}} \hspace{4pt} Tone: \#H.
\textcolor{Sepia}{\selectlanguage{english}Hay for horses, horse hay.} \zh{喂马的草。} 
\lhead{\firstmark}
\rhead{\botmark}

\subsection{\hspace{-0.5cm} {\Large \textcolor{darkblue}{\textbf{\ipa{ʐwæ˧\textasciitilde{}ʐwæ˧}}}}\hspace{0.5cm}[\kern2pt{\textcolor{darkblue}{\textbf{\ipa{ʐwæ˧ʐwæ˧}}}}\kern2pt]} \hypertarget{z`w\{\string_M~z`w\{\string_M1}{}
\markboth{\textcolor{darkblue}{\textbf{\ipa{ʐwæ˧\textasciitilde{}ʐwæ˧}}}}{}
\textcolor{teal}{\mytextsc{verb}} \hspace{4pt} Tone: M.
\ding{202} \textcolor{Sepia}{\selectlanguage{english}To put (things) in order.} \zh{收拾。}  ¶ \textcolor{darkblue}{\textbf{\ipa{tso˧\textasciitilde{}tso˧ ʐwæ˧\textasciitilde{}ʐwæ˧(-ze˩)}}} \textcolor{Sepia}{\selectlanguage{english}to put things in order} \zh{收拾东西}  
 ¶ \textcolor{darkblue}{\textbf{\ipa{le˧-ʐwæ˧\textasciitilde{}ʐwæ˧ ɖɯ˧-ʝi˧-tɕɯ˥}}} \textcolor{Sepia}{\selectlanguage{english}to put things in order in one place, to arrange things together in one place} \zh{把东西收拾在一起}  
\ding{203} \textcolor{Sepia}{\selectlanguage{english}To gather (people).} \zh{聚集。} 
\lhead{\firstmark}
\rhead{\botmark}

\subsection{\hspace{-0.5cm} {\Large \textcolor{darkblue}{\textbf{\ipa{ʐwæ˩}}}}\hspace{0.5cm}[\kern2pt{\textcolor{darkblue}{\textbf{\ipa{ʐwæ˩˥}}}}\kern2pt]} \hypertarget{z`w\{\string_B1}{}
\markboth{\textcolor{darkblue}{\textbf{\ipa{ʐwæ˩}}}}{}
\textcolor{teal}{\mytextsc{adverb(ial)}} \hspace{4pt} Tone: L.
\textcolor{Sepia}{\selectlanguage{english}Extremely.} \zh{很、极。}  ¶ \textcolor{darkblue}{\textbf{\ipa{ʐwæ˩-ze˥!}}} \textcolor{Sepia}{\selectlanguage{english}That's too much! / There's too much!} \zh{太多了!}  

\lhead{\firstmark}
\rhead{\botmark}

\subsection{\hspace{-0.5cm} {\Large \textcolor{darkblue}{\textbf{\ipa{ʐwæ˩\textsubscript{a}}}} \textsubscript{1}}\hspace{0.5cm}[\kern2pt{\textcolor{darkblue}{\textbf{\ipa{ʐwæ˩˥}}}}\kern2pt]} \hypertarget{z`w\{\string_Ba1}{}
\markboth{\textcolor{darkblue}{\textbf{\ipa{ʐwæ˩\textsubscript{a}}}} \textsubscript{1}}{}
\textcolor{teal}{\mytextsc{verb}} \hspace{4pt} Tone: L\textsubscript{a}.
\textcolor{Sepia}{\selectlanguage{english}To swoon.} \zh{昏,昏厥。}  ¶ \textcolor{darkblue}{\textbf{\ipa{le˧-ʈʰi˩ | le˧-ʐwæ˩-ze˩}}} \textcolor{Sepia}{\selectlanguage{english}to be so tired as to fall into a swoon, to swoon from exhaustion} \zh{累得都昏倒了}  

\lhead{\firstmark}
\rhead{\botmark}

\subsection{\hspace{-0.5cm} {\Large \textcolor{darkblue}{\textbf{\ipa{ʐwæ˩\textsubscript{a}}}} \textsubscript{2}}\hspace{0.5cm}[\kern2pt{\textcolor{darkblue}{\textbf{\ipa{ʐwæ˩˥}}}}\kern2pt]} \hypertarget{z`w\{\string_Ba2}{}
\markboth{\textcolor{darkblue}{\textbf{\ipa{ʐwæ˩\textsubscript{a}}}} \textsubscript{2}}{}
\textcolor{teal}{\mytextsc{adjective}} \hspace{4pt} Tone: L\textsubscript{a}.
\textcolor{Sepia}{\selectlanguage{english}Good, well (working well, strongly).} \zh{好,能干。}  ¶ \textcolor{darkblue}{\textbf{\ipa{ʐwæ˩-hĩ˩˥}}} \textcolor{Sepia}{\selectlanguage{english}\mytextsc{nmlz}} \zh{能干的}  
 ¶ \textcolor{darkblue}{\textbf{\ipa{ʈʂʰɯ˧ ɖwæ˧˥ | ʐwæ˩˥!}}} \textcolor{Sepia}{\selectlanguage{english}He is very capable!} \zh{他很能干!}  

\lhead{\firstmark}
\rhead{\botmark}

\subsection{\hspace{-0.5cm} {\Large \textcolor{darkblue}{\textbf{\ipa{ʐwæ˩mi˩}}}}\hspace{0.5cm}[\kern2pt{\textcolor{darkblue}{\textbf{\ipa{ʐwæ˩mi˩˥}}}}\kern2pt]} \hypertarget{z`w\{\string_Bmi\string_B1}{}
\markboth{\textcolor{darkblue}{\textbf{\ipa{ʐwæ˩mi˩}}}}{}
\textcolor{teal}{\mytextsc{noun}} \hspace{4pt} Tone: L.
\textcolor{Sepia}{\selectlanguage{english}Mare.} \zh{母马。}  ¶ \textcolor{darkblue}{\textbf{\ipa{ʂe˩-ʐwæ˩mi˥}}} \textcolor{Sepia}{\selectlanguage{english}bicycle} \zh{自行车(“铁马”)}  
 ¶ \textcolor{darkblue}{\textbf{\ipa{ʐwæ˩mi˩-ʐwæ˩zo˩}}} \textcolor{Sepia}{\selectlanguage{english}mare and colt} \zh{母马与马驹子}  
 \zh{量词}: \textcolor{darkblue}{\textbf{\ipa{v̩˧}}} \textcolor{darkblue}{\textbf{\ipa{jɤ˧˥}}}  \mytextsc{clf}: \textcolor{darkblue}{\textbf{\ipa{v̩˧}}} \textcolor{darkblue}{\textbf{\ipa{jɤ˧˥}}} 
\lhead{\firstmark}
\rhead{\botmark}

\subsection{\hspace{-0.5cm} {\Large \textcolor{darkblue}{\textbf{\ipa{ʐwæ˧˥}}}}\hspace{0.5cm}[\kern2pt{\textcolor{darkblue}{\textbf{\ipa{ʐwæ˧˥}}}}\kern2pt]} \hypertarget{z`w\{\string_M\string_T1}{}
\markboth{\textcolor{darkblue}{\textbf{\ipa{ʐwæ˧˥}}}}{}
\textcolor{teal}{\mytextsc{verb}} \hspace{4pt} Tone: MH.
\textcolor{Sepia}{\selectlanguage{english}To hoe weeds.} \zh{薅锄、锄草。}  ¶ \textcolor{darkblue}{\textbf{\ipa{ʐwæ˩\textasciitilde{}ʐwæ˧˥}}} \textcolor{Sepia}{\selectlanguage{english}\mytextsc{red}} \zh{\mytextsc{重叠}}  
 ¶ \textcolor{darkblue}{\textbf{\ipa{jɤ˩jo˥ ʐwæ˩}}} \textcolor{Sepia}{\selectlanguage{english}to hoe potatoes, to weed a potato field} \zh{洋芋地里锄草}  
 ¶ \textcolor{darkblue}{\textbf{\ipa{jɤ˩jo˧ ʐwæ˧\textasciitilde{}ʐwæ˥}}} \textcolor{Sepia}{\selectlanguage{english}to hoe potatoes, to weed a potato field} \zh{洋芋地里锄草}  
 ¶ \textcolor{darkblue}{\textbf{\ipa{qʰɑ˧dze˧ ʐwæ˧˥}}} \textcolor{Sepia}{\selectlanguage{english}to hoe sweetcorn, to weed a sweetcorn field} \zh{苞谷地里锄草}  
 ¶ \textcolor{darkblue}{\textbf{\ipa{qʰɑ˧dze˧ ʐwæ˧\textasciitilde{}ʐwæ˥}}} \textcolor{Sepia}{\selectlanguage{english}to hoe sweetcorn, to weed a sweetcorn field} \zh{苞谷地里锄草}  

\lhead{\firstmark}
\rhead{\botmark}

\subsection{\hspace{-0.5cm} {\Large \textcolor{darkblue}{\textbf{\ipa{ʐwɤ˧}}}}\hspace{0.5cm}[\kern2pt{\textcolor{darkblue}{\textbf{\ipa{ʐwɤ˥}}}}\kern2pt]} \hypertarget{z`w7\string_M1}{}
\markboth{\textcolor{darkblue}{\textbf{\ipa{ʐwɤ˧}}}}{}
\textcolor{teal}{\mytextsc{adjective}} \hspace{4pt} Tone: M.
\textcolor{Sepia}{\selectlanguage{english}Hungry (monosyllable).} \zh{饿。}  ¶ \textcolor{darkblue}{\textbf{\ipa{hɑ˧-ʐwɤ˩}}} \textcolor{Sepia}{\selectlanguage{english}to be hungry} \zh{饿}  

\lhead{\firstmark}
\rhead{\botmark}

\subsection{\hspace{-0.5cm} {\Large \textcolor{darkblue}{\textbf{\ipa{ʐwɤ˧mv̩˧}}}}\hspace{0.5cm}[\kern2pt{\textcolor{darkblue}{\textbf{\ipa{ʐwɤ˩mv̩˩˥}}}}\kern2pt]} \hypertarget{z`w7\string_Mmv\string_=\string_M1}{}
\markboth{\textcolor{darkblue}{\textbf{\ipa{ʐwɤ˧mv̩˧}}}}{}
\textcolor{teal}{\mytextsc{noun}} \hspace{4pt} Tone: M.
\textit{From:} \textbf{ʐwɤ˩b} \textcolor{Sepia}{\selectlanguage{english}Idiom, set phrase, fixed expression.} \zh{惯用语、习惯语、习语。}  ¶ \textcolor{darkblue}{\textbf{\ipa{ʐwɤ˧mv̩˧ dʑo˧-kv̩˧˥ !}}} \textcolor{Sepia}{\selectlanguage{english}This is how they say! / There's such a set phrase!} \zh{有这么一句老话! / 有这么一个说法!}  
 ¶ \textcolor{darkblue}{\textbf{\ipa{æ˧ʂæ˧-ʐwɤ˧mv̩˧ | ɖɯ˧-kʰwɤ˥}}} \textcolor{Sepia}{\selectlanguage{english}a saying, a set phrase of the old times} \zh{一句老话、一个传统的说法}  
 \zh{量词}: \textcolor{darkblue}{\textbf{\ipa{kʰwɤ˥}}}  \mytextsc{clf}: \textcolor{darkblue}{\textbf{\ipa{kʰwɤ˥}}} 
\lhead{\firstmark}
\rhead{\botmark}

\subsection{\hspace{-0.5cm} {\Large \textcolor{darkblue}{\textbf{\ipa{ʐwɤ˩\textsubscript{b}}}}}\hspace{0.5cm}[\kern2pt{\textcolor{darkblue}{\textbf{\ipa{ʐwɤ˥}}}}\kern2pt]} \hypertarget{z`w7\string_Bb1}{}
\markboth{\textcolor{darkblue}{\textbf{\ipa{ʐwɤ˩\textsubscript{b}}}}}{}
\textcolor{teal}{\mytextsc{verb}} \hspace{4pt} Tone: L\textsubscript{b}.
\textcolor{Sepia}{\selectlanguage{english}To speak.} \zh{讲话。}  ¶ \textcolor{darkblue}{\textbf{\ipa{ʐwɤ˧\textasciitilde{}ʐwɤ˩ mɤ˩-hĩ˩}}} \textcolor{Sepia}{\selectlanguage{english}dumb person, person who is not able to speak} \zh{哑巴、不会讲话的人}  
 ¶ \textcolor{darkblue}{\textbf{\ipa{ʐwɤ˧\textasciitilde{}ʐwɤ˩ mɤ˩-hĩ˩, | ʈʂʰɯ˧-v̩˧!}}} \textcolor{Sepia}{\selectlanguage{english}(S)he is not able to speak! / (S)he won't speak!} \zh{不会讲话,这个人! / 这个人,不会讲话!}  
 ¶ \textcolor{darkblue}{\textbf{\ipa{le˧-ʐwɤ˩-ze˩}}} \textcolor{Sepia}{\selectlanguage{english}\mytextsc{accomp} \string_ \mytextsc{pfv}} \zh{讲了}  
 ¶ \textcolor{darkblue}{\textbf{\ipa{no˧ | ə˧tso˧ ʐwɤ˩-ɲi˩?}}} \textcolor{Sepia}{\selectlanguage{english}What are you saying? / What do you mean?} \zh{你说什么?}  
 ¶ \textcolor{darkblue}{\textbf{\ipa{ʐwɤ˧\textasciitilde{}ʐwɤ˩}}} \textcolor{Sepia}{\selectlanguage{english}\mytextsc{red}} \zh{\mytextsc{重叠}}  
 ¶ \textcolor{darkblue}{\textbf{\ipa{le˧-ʐwɤ˩}}} \textcolor{Sepia}{\selectlanguage{english}to answer} \zh{回答}  
 ¶ \textcolor{darkblue}{\textbf{\ipa{le˧-wo˧ ʐwɤ˧˥}}} \textcolor{Sepia}{\selectlanguage{english}to answer} \zh{回答}  
 ¶ \textcolor{darkblue}{\textbf{\ipa{ʈʂʰɯ˧ | le˧-ʐwɤ˩-bi˩-dʑo˩...}}} \textcolor{Sepia}{\selectlanguage{english}According to her/him... / From her/his point of view...} \zh{依照他的说法……}  
 ¶ \textcolor{darkblue}{\textbf{\ipa{hĩ˧-qɑ˧ ʐwɤ˧\textasciitilde{}ʐwɤ˥}}} \textcolor{Sepia}{\selectlanguage{english}to speak to people} \zh{对人家讲}  
 ¶ \textcolor{darkblue}{\textbf{\ipa{le˧-ʐwɤ˧\textasciitilde{}ʐwɤ˥-ze˩}}} \textcolor{Sepia}{\selectlanguage{english}\mytextsc{accomp} \mytextsc{red} \mytextsc{pfv}} \zh{\mytextsc{accomp} \mytextsc{red} \mytextsc{pfv}}  

\lhead{\firstmark}
\rhead{\botmark}

\newpage
\section*{\centering- \textcolor{darkblue}{\textbf{\ipa{ʑ}}} -}
\subsection{\hspace{-0.5cm} {\Large \textcolor{darkblue}{\textbf{\ipa{ʑi˥}}}}\hspace{0.5cm}[\kern2pt{\textcolor{darkblue}{\textbf{\ipa{ʑi˥}}}}\kern2pt]} \hypertarget{z£i\string_T1}{}
\markboth{\textcolor{darkblue}{\textbf{\ipa{ʑi˥}}}}{}
\textcolor{teal}{\mytextsc{verb}} \hspace{4pt} Tone: H.
\textcolor{Sepia}{\selectlanguage{english}To be present: abstract entity (courage, strength) or concrete entity (beard).} \zh{有,拥有(抽象:有力量,有勇气)。}  ¶ \textcolor{darkblue}{\textbf{\ipa{mɤ˧-ʑi˥}}} \textcolor{Sepia}{\selectlanguage{english}\mytextsc{neg}} \zh{没有}  

\lhead{\firstmark}
\rhead{\botmark}

\subsection{\hspace{-0.5cm} {\Large \textcolor{darkblue}{\textbf{\ipa{ʑi˧\textsubscript{a}}}}}\hspace{0.5cm}[\kern2pt{\textcolor{darkblue}{\textbf{\ipa{ʑi˥}}}}\kern2pt]} \hypertarget{z£i\string_Ma1}{}
\markboth{\textcolor{darkblue}{\textbf{\ipa{ʑi˧\textsubscript{a}}}}}{}
\textcolor{teal}{\mytextsc{verb}} \hspace{4pt} Tone: M\textsubscript{a}.
\ding{202} \textcolor{Sepia}{\selectlanguage{english}To leak.} \zh{漏(水)。}  ¶ \textcolor{darkblue}{\textbf{\ipa{mv̩˩tɕo˧ ʑi˧}}} \textcolor{Sepia}{\selectlanguage{english}to leak, to drip down} \zh{(水)往下漏}  
\ding{203} \textcolor{Sepia}{\selectlanguage{english}To flow (a river flows).} \zh{流(河水流着)。}  ¶ \textcolor{darkblue}{\textbf{\ipa{mv̩˩tɕo˧ ʑi˧}}} \textcolor{Sepia}{\selectlanguage{english}to flow down (the water of a brook flows down)} \zh{(河)往下游流}  

\lhead{\firstmark}
\rhead{\botmark}

\subsection{\hspace{-0.5cm} {\Large \textcolor{darkblue}{\textbf{\ipa{ʑi˧dv̩˧}}}}\hspace{0.5cm}[\kern2pt{\textcolor{darkblue}{\textbf{\ipa{ʑi˩dv̩˩˥}}}}\kern2pt]} \hypertarget{z£i\string_Mdv\string_=\string_M1}{}
\markboth{\textcolor{darkblue}{\textbf{\ipa{ʑi˧dv̩˧}}}}{}
\textcolor{teal}{\mytextsc{noun}} \hspace{4pt} Tone: M.
\ding{202} \textcolor{Sepia}{\selectlanguage{english}The household.} \zh{家。}  ¶ \textcolor{darkblue}{\textbf{\ipa{ɖɯ˧-ʑi˩dv̩˩}}} \textcolor{Sepia}{\selectlanguage{english}a household} \zh{一家人,包括所有成员}  
 ¶ \textcolor{darkblue}{\textbf{\ipa{ɑ˩ʁo˧-ʑi˧dv̩˧ ʝi˧}}} \textcolor{Sepia}{\selectlanguage{english}to take care of the house, to look after the household, to be in charge of the household} \zh{管家}  
 \zh{量词}: \textcolor{darkblue}{\textbf{\ipa{ɭɯ˧}}} \ding{203} \textcolor{Sepia}{\selectlanguage{english}The entire farmhouse, comprising the main house and the other buildings.} \zh{农舍,包括院子、人住的楼、动物住的楼等。}  \zh{量词}: \textcolor{darkblue}{\textbf{\ipa{ɭɯ˧}}}  \mytextsc{clf}: \textcolor{darkblue}{\textbf{\ipa{ɭɯ˧}}} \textcolor{darkblue}{\textbf{\ipa{ɭɯ˧}}} 
\lhead{\firstmark}
\rhead{\botmark}

\subsection{\hspace{-0.5cm} {\Large \textcolor{darkblue}{\textbf{\ipa{ʑi˧dv̩˧ʝi˧-hĩ\#˥}}}}\hspace{0.5cm}[\kern2pt{\textcolor{darkblue}{\textbf{\ipa{xxxx non-correspondance entre le nombre de morphèmes et le nombre de tons de morphèmes}}}}\kern2pt]} \hypertarget{z£i\string_Mdv\string_=\string_Mj££i\string_M-hi\string_~\#\string_T1}{}
\markboth{\textcolor{darkblue}{\textbf{\ipa{ʑi˧dv̩˧ʝi˧-hĩ\#˥}}}}{}
\textcolor{teal}{\mytextsc{noun}} \hspace{4pt} Tone: H\#.
\textcolor{Sepia}{\selectlanguage{english}The person in charge of the house, the master/mistress of the house.} \zh{一家之主、家长。} 
\lhead{\firstmark}
\rhead{\botmark}

\subsection{\hspace{-0.5cm} {\Large \textcolor{darkblue}{\textbf{\ipa{ʑi˧kv̩˧wo˧}}}}\hspace{0.5cm}[\kern2pt{\textcolor{darkblue}{\textbf{\ipa{ʑi˩kv̩˧wo˧˥}}}}\kern2pt]} \hypertarget{z£i\string_Mkv\string_=\string_Mwo\string_M1}{}
\markboth{\textcolor{darkblue}{\textbf{\ipa{ʑi˧kv̩˧wo˧}}}}{}
\textcolor{teal}{\mytextsc{noun}} \hspace{4pt} Tone: M.
\textcolor{Sepia}{\selectlanguage{english}Roof.} \zh{房顶。}  \zh{量词}: \textcolor{darkblue}{\textbf{\ipa{tsʰi˩}}}  \mytextsc{clf}: \textcolor{darkblue}{\textbf{\ipa{tsʰi˩}}} 
\lhead{\firstmark}
\rhead{\botmark}

\subsection{\hspace{-0.5cm} {\Large \textcolor{darkblue}{\textbf{\ipa{ʑi˧mi˧}}}}\hspace{0.5cm}[\kern2pt{\textcolor{darkblue}{\textbf{\ipa{ʑi˧mi˧}}}}\kern2pt]} \hypertarget{z£i\string_Mmi\string_M1}{}
\markboth{\textcolor{darkblue}{\textbf{\ipa{ʑi˧mi˧}}}}{}
\textcolor{teal}{\mytextsc{noun}} \hspace{4pt} Tone: M.
\ding{202} \textcolor{Sepia}{\selectlanguage{english}The main building of the house/farm: the building where the hearth is located.} \zh{家里有火塘的那个房子(“祖母房”)。}  \zh{量词}: \textcolor{darkblue}{\textbf{\ipa{ɭɯ˧}}} \ding{203} \textcolor{Sepia}{\selectlanguage{english}The entire house, the entire farm.} \zh{整个家园。}  \mytextsc{clf}: \textcolor{darkblue}{\textbf{\ipa{ɭɯ˧}}} 
\lhead{\firstmark}
\rhead{\botmark}

\subsection{\hspace{-0.5cm} {\Large \textcolor{darkblue}{\textbf{\ipa{ʑi˧mv̩˧˥}}}}\hspace{0.5cm}[\kern2pt{\textcolor{darkblue}{\textbf{\ipa{ʑi˩mv̩˥}}}}\kern2pt]} \hypertarget{z£i\string_Mmv\string_=\string_M\string_T1}{}
\markboth{\textcolor{darkblue}{\textbf{\ipa{ʑi˧mv̩˧˥}}}}{}
\textcolor{teal}{\mytextsc{noun}} \hspace{4pt} Tone: MH\#.
\textcolor{Sepia}{\selectlanguage{english}Dream.} \zh{梦。}  ¶ \textcolor{darkblue}{\textbf{\ipa{ʑi˧mv̩˧ qʰwɤ˧˥}}} \textcolor{Sepia}{\selectlanguage{english}to have a dream} \zh{做梦}  
 ¶ \textcolor{darkblue}{\textbf{\ipa{ʑi˧mv̩˧ sɯ˧}}} \textcolor{Sepia}{\selectlanguage{english}to sleep-walk (somnambulism); also: to speak in one's dreams} \zh{梦游,梦呓}  
 ¶ \textcolor{darkblue}{\textbf{\ipa{njɤ˧ | ə˧hwɤ˧ | ʑi˧mv̩˥ | mɤ˧-dʑɤ˩!}}} \textcolor{Sepia}{\selectlanguage{english}I have not had good dreams yesterday night! / I had a nightmare yesterday night!} \zh{我昨天做了恶梦!}  
 \zh{量词}: \textcolor{darkblue}{\textbf{\ipa{kʰwɤ˥}}}  \mytextsc{clf}: \textcolor{darkblue}{\textbf{\ipa{kʰwɤ˥}}} 
\lhead{\firstmark}
\rhead{\botmark}

\subsection{\hspace{-0.5cm} {\Large \textcolor{darkblue}{\textbf{\ipa{ʑi˧ŋɤ˥}}}}\hspace{0.5cm}[\kern2pt{\textcolor{darkblue}{\textbf{\ipa{ʑi˧ŋɤ˧˥}}}}\kern2pt]} \hypertarget{z£i\string_MN7\string_T1}{}
\markboth{\textcolor{darkblue}{\textbf{\ipa{ʑi˧ŋɤ˥}}}}{}
\textcolor{teal}{\mytextsc{verb}} \hspace{4pt} Tone: H\#.
\textcolor{Sepia}{\selectlanguage{english}To doze off, to nod.} \zh{打瞌睡。}  ¶ \textcolor{darkblue}{\textbf{\ipa{ʑi˧ŋɤ˥-ze˩}}} \textcolor{Sepia}{\selectlanguage{english}\mytextsc{pfv}} \zh{\mytextsc{pfv}}  

\lhead{\firstmark}
\rhead{\botmark}

\subsection{\hspace{-0.5cm} {\Large \textcolor{darkblue}{\textbf{\ipa{ʑi˧ŋv̩˥}}}}\hspace{0.5cm}[\kern2pt{\textcolor{darkblue}{\textbf{\ipa{ʑi˧ŋv̩˥}}}}\kern2pt]} \hypertarget{z£i\string_MNv\string_=\string_T1}{}
\markboth{\textcolor{darkblue}{\textbf{\ipa{ʑi˧ŋv̩˥}}}}{}
\textcolor{teal}{\mytextsc{verb}} \hspace{4pt} Tone: H\#.
\textcolor{Sepia}{\selectlanguage{english}To sleep.} \zh{睡觉。}  ¶ \textcolor{darkblue}{\textbf{\ipa{le˧-ʑi˧ŋv̩˥}}} \textcolor{Sepia}{\selectlanguage{english}\mytextsc{accomp}} \zh{\mytextsc{accomp}}  
 ¶ \textcolor{darkblue}{\textbf{\ipa{ʑi˧ŋv̩˥-ho˩}}} \textcolor{Sepia}{\selectlanguage{english}is going to sleep} \zh{要睡了}  

\lhead{\firstmark}
\rhead{\botmark}

\subsection{\hspace{-0.5cm} {\Large \textcolor{darkblue}{\textbf{\ipa{ʑi˧qʰwɤ˧}}}}\hspace{0.5cm}[\kern2pt{\textcolor{darkblue}{\textbf{\ipa{ʑi˩qʰwɤ˥}}}}\kern2pt]} \hypertarget{z£i\string_Mq\string_hw7\string_M1}{}
\markboth{\textcolor{darkblue}{\textbf{\ipa{ʑi˧qʰwɤ˧}}}}{}
\textcolor{teal}{\mytextsc{noun}} \hspace{4pt} Tone: M.
\textcolor{Sepia}{\selectlanguage{english}Building; houses.} \zh{房屋。}  ¶ \textcolor{darkblue}{\textbf{\ipa{ʑi˧qʰwɤ˧ gv̩˩}}} \textcolor{Sepia}{\selectlanguage{english}to build a building} \zh{建房}  
 ¶ \textcolor{darkblue}{\textbf{\ipa{ʑi˧qʰwɤ˧-lɑ˧ do˥!}}} \textcolor{Sepia}{\selectlanguage{english}One can only see buildings! (A comment by the consultant about the city of Lijiang: the plain is now thoroughly covered by buildings, and one cannot see fields anymore, unlike in Yongning, where until recently there were only a few hamlets scattered among a landscape of groves, pastures, and cultivated fields.)} \zh{只看到房子! / 能看见的只有房子!(合作者说,丽江市区都是房子,看不到田。这一点,不像永宁坝:二十世纪的永宁,只有一些小村落分散在一大片田地中。)}  
 ¶ \textcolor{darkblue}{\textbf{\ipa{ɕjo˩ɕjɤ˩-ʑi˩qʰwɤ˥}}} \textcolor{Sepia}{\selectlanguage{english}the building(s) of the school, the school buildings} \zh{学校的楼(‘学校’:汉语借词)}  
 ¶ \textcolor{darkblue}{\textbf{\ipa{ʑi˧qʰwɤ˧ ʈʂʰv̩˩}}} \textcolor{Sepia}{\selectlanguage{english}to paint a house} \zh{给房子刷颜色(直译:‘染房’)}  
 ¶ \textcolor{darkblue}{\textbf{\ipa{ʑi˧qʰwɤ˧ tɕʰi˧-hĩ˧ kʰv̩˥mi˩}}} \textcolor{Sepia}{\selectlanguage{english}watchdog} \zh{看门狗}  
 ¶ \textcolor{darkblue}{\textbf{\ipa{ʑi˧qʰwɤ˧ tɕʰi˧-hĩ˧ kʰv̩˥}}} \textcolor{Sepia}{\selectlanguage{english}watchdog} \zh{看门狗}  
 \zh{量词}: \textcolor{darkblue}{\textbf{\ipa{ɭɯ˧}}}  \mytextsc{clf}: \textcolor{darkblue}{\textbf{\ipa{ɭɯ˧}}} 
\lhead{\firstmark}
\rhead{\botmark}

\subsection{\hspace{-0.5cm} {\Large \textcolor{darkblue}{\textbf{\ipa{ʑi˧ʁæ˥\$}}}}\hspace{0.5cm}[\kern2pt{\textcolor{darkblue}{\textbf{\ipa{ʑi˧ʁæ˧}}}}\kern2pt]} \hypertarget{z£i\string_MR\{\string_T\$1}{}
\markboth{\textcolor{darkblue}{\textbf{\ipa{ʑi˧ʁæ˥\$}}}}{}
\textcolor{teal}{\mytextsc{noun}} \hspace{4pt} Tone: H\$.
\textcolor{Sepia}{\selectlanguage{english}The back of the house, the space behind the house.} \zh{房屋的上后方。}  ¶ \textcolor{darkblue}{\textbf{\ipa{ɲi˧ʈʂæ˧-ʑi˧-ʁo˧tʰo˥, | ʑi˧ʁæ˧ ɲi˥ mæ˩!}}} \textcolor{Sepia}{\selectlanguage{english}The place behind the two-storey building is called 'ʑi˧ʁæ˥\$'!} \zh{两层楼房后面(这块地方)叫做“房屋的上后方”! / 房屋背后(这块地方)叫做“房屋的上后方”!}  
 \zh{量词}: \textcolor{darkblue}{\textbf{\ipa{kʰwɤ˥}}}  \mytextsc{clf}: \textcolor{darkblue}{\textbf{\ipa{kʰwɤ˥}}} 
\lhead{\firstmark}
\rhead{\botmark}

\subsection{\hspace{-0.5cm} {\Large \textcolor{darkblue}{\textbf{\ipa{ʑi˧ʁo˥\$}}}}\hspace{0.5cm}[\kern2pt{\textcolor{darkblue}{\textbf{\ipa{ʑi˧ʁo˥}}}}\kern2pt]} \hypertarget{z£i\string_MRo\string_T\$1}{}
\markboth{\textcolor{darkblue}{\textbf{\ipa{ʑi˧ʁo˥\$}}}}{}
\textcolor{teal}{\mytextsc{noun}} \hspace{4pt} Tone: H\$.
\textcolor{Sepia}{\selectlanguage{english}Bed.} \zh{床。}  \zh{量词}: \textcolor{darkblue}{\textbf{\ipa{ɭɯ˧˥}}}  \mytextsc{clf}: \textcolor{darkblue}{\textbf{\ipa{ɭɯ˧˥}}} 
\lhead{\firstmark}
\rhead{\botmark}

\subsection{\hspace{-0.5cm} {\Large \textcolor{darkblue}{\textbf{\ipa{ʑi˩}}} \textsubscript{2}}\hspace{0.5cm}[\kern2pt{\textcolor{darkblue}{\textbf{\ipa{xxxx ton non trouvé, à faire manuellement...}}}}\kern2pt]} \hypertarget{z£i\string_B2}{}
\markboth{\textcolor{darkblue}{\textbf{\ipa{ʑi˩}}} \textsubscript{2}}{}
\textcolor{teal}{\mytextsc{classifier}} \hspace{4pt} Tone: L\textsubscript{2}.
\textcolor{Sepia}{\selectlanguage{english}Family.} \zh{家庭(一户人)。}  ¶ \textcolor{darkblue}{\textbf{\ipa{hĩ˧ | ɖɯ˧-ʑi˩}}} \textcolor{Sepia}{\selectlanguage{english}a family} \zh{一家人}  
 ¶ \textcolor{darkblue}{\textbf{\ipa{ʈʂʰɯ˧-ʑi˥}}} \textcolor{Sepia}{\selectlanguage{english}this family} \zh{这家}  
 ¶ \textcolor{darkblue}{\textbf{\ipa{ŋwæ˧-qʰv̩˧, | tsʰe˧ɲi˧-ʑi˩}}} \textcolor{Sepia}{\selectlanguage{english}Five hamlets, twelve families! (A summary of the statistics of the village of \textcolor{darkblue}{\textbf{\ipa{/ə˧lɑ˧-ʁwɤ\#˥/}}}.)} \zh{五个村落,十二个家庭!(阿拉瓦村的人口简介)}  

\lhead{\firstmark}
\rhead{\botmark}

\subsection{\hspace{-0.5cm} {\Large \textcolor{darkblue}{\textbf{\ipa{ʑi˩\textsubscript{b}}}}}\hspace{0.5cm}[\kern2pt{\textcolor{darkblue}{\textbf{\ipa{ʑi˥}}}}\kern2pt]} \hypertarget{z£i\string_Bb1}{}
\markboth{\textcolor{darkblue}{\textbf{\ipa{ʑi˩\textsubscript{b}}}}}{}
\textcolor{teal}{\mytextsc{verb}} \hspace{4pt} Tone: L\textsubscript{b}.
\ding{202} \textcolor{Sepia}{\selectlanguage{english}To clutch, to grasp, to grab.} \zh{拿,捉 (捉鸡)。}  ¶ \textcolor{darkblue}{\textbf{\ipa{æ˩ ʑi˧}}} \textcolor{Sepia}{\selectlanguage{english}to catch a chicken} \zh{捉鸡}  
 ¶ \textcolor{darkblue}{\textbf{\ipa{æ˩˥ | le˧-ʑi˩}}} \textcolor{Sepia}{\selectlanguage{english}to catch a chicken} \zh{捉鸡}  
 ¶ \textcolor{darkblue}{\textbf{\ipa{hĩ˧ ʑi˧˥}}} \textcolor{Sepia}{\selectlanguage{english}to grab someone} \zh{抓人}  
 ¶ \textcolor{darkblue}{\textbf{\ipa{ʁæ˧ ʑi˧}}} \textcolor{Sepia}{\selectlanguage{english}to hold sb in one's arms (arm over the neck)} \zh{搂(用胳膊搂脖子)}  
\ding{203} \textcolor{Sepia}{\selectlanguage{english}To take along, to bring along.} \zh{带、拿过来。}  ¶ \textcolor{darkblue}{\textbf{\ipa{tso˧\textasciitilde{}tso˧ ʑi˧˥}}} \textcolor{Sepia}{\selectlanguage{english}to bring something} \zh{带东西过来}  

\lhead{\firstmark}
\rhead{\botmark}

\subsection{\hspace{-0.5cm} {\Large \textcolor{darkblue}{\textbf{\ipa{ʑi˩hṽ\#˥}}}}\hspace{0.5cm}[\kern2pt{\textcolor{darkblue}{\textbf{\ipa{ʑi˧hṽ˥}}}}\kern2pt]} \hypertarget{z£i\string_Bhv\string_~\#\string_T1}{}
\markboth{\textcolor{darkblue}{\textbf{\ipa{ʑi˩hṽ\#˥}}}}{}
\textcolor{teal}{\mytextsc{noun}} \hspace{4pt} Tone: LM+\#H.
\textcolor{Sepia}{\selectlanguage{english}Body hair (of humans).} \zh{人身上的毛。}  \zh{量词}: \textcolor{darkblue}{\textbf{\ipa{kʰɯ˩}}}  \mytextsc{clf}: \textcolor{darkblue}{\textbf{\ipa{kʰɯ˩}}} 
\lhead{\firstmark}
\rhead{\botmark}

\subsection{\hspace{-0.5cm} {\Large \textcolor{darkblue}{\textbf{\ipa{ʑi˩-kʰv̩˧˥}}}}\hspace{0.5cm}[\kern2pt{\textcolor{darkblue}{\textbf{\ipa{xxxx non-correspondance entre le nombre de morphèmes et le nombre de tons de morphèmes}}}}\kern2pt]} \hypertarget{z£i\string_B-k\string_hv\string_=\string_M\string_T1}{}
\markboth{\textcolor{darkblue}{\textbf{\ipa{ʑi˩-kʰv̩˧˥}}}}{}
\textcolor{teal}{\mytextsc{noun}} \hspace{4pt} Tone: LM+MH\#.
\textcolor{Sepia}{\selectlanguage{english}Year of the monkey.} \zh{猴年。} 
\lhead{\firstmark}
\rhead{\botmark}

\subsection{\hspace{-0.5cm} {\Large \textcolor{darkblue}{\textbf{\ipa{ʑi˩mi\#˥}}}}\hspace{0.5cm}[\kern2pt{\textcolor{darkblue}{\textbf{\ipa{ʑi˧mi˧}}}}\kern2pt]} \hypertarget{z£i\string_Bmi\#\string_T1}{}
\markboth{\textcolor{darkblue}{\textbf{\ipa{ʑi˩mi\#˥}}}}{}
\textcolor{teal}{\mytextsc{noun}} \hspace{4pt} Tone: LM+\#H.
\textcolor{Sepia}{\selectlanguage{english}Female monkey.} \zh{母猴。}  \zh{量词}: \textcolor{darkblue}{\textbf{\ipa{mi˩}}}  \mytextsc{clf}: \textcolor{darkblue}{\textbf{\ipa{mi˩}}} 
\lhead{\firstmark}
\rhead{\botmark}

\subsection{\hspace{-0.5cm} {\Large \textcolor{darkblue}{\textbf{\ipa{ʑi˩pʰv̩\#˥}}}}\hspace{0.5cm}[\kern2pt{\textcolor{darkblue}{\textbf{\ipa{ʑi˧pʰv̩˥}}}}\kern2pt]} \hypertarget{z£i\string_Bp\string_hv\string_=\#\string_T1}{}
\markboth{\textcolor{darkblue}{\textbf{\ipa{ʑi˩pʰv̩\#˥}}}}{}
\textcolor{teal}{\mytextsc{noun}} \hspace{4pt} Tone: LM+\#H.
\textcolor{Sepia}{\selectlanguage{english}Male monkey.} \zh{公猴。}  \zh{量词}: \textcolor{darkblue}{\textbf{\ipa{mi˩}}}  \mytextsc{clf}: \textcolor{darkblue}{\textbf{\ipa{mi˩}}} 
\lhead{\firstmark}
\rhead{\botmark}

\subsection{\hspace{-0.5cm} {\Large \textcolor{darkblue}{\textbf{\ipa{ʑi˩zo\#˥}}}}\hspace{0.5cm}[\kern2pt{\textcolor{darkblue}{\textbf{\ipa{ʑi˧zo˥}}}}\kern2pt]} \hypertarget{z£i\string_Bzo\#\string_T1}{}
\markboth{\textcolor{darkblue}{\textbf{\ipa{ʑi˩zo\#˥}}}}{}
\textcolor{teal}{\mytextsc{noun}} \hspace{4pt} Tone: LM+\#H.
\textcolor{Sepia}{\selectlanguage{english}Infant monkey/ape.} \zh{小猴子。}  \zh{量词}: \textcolor{darkblue}{\textbf{\ipa{ɭɯ˧}}}  \mytextsc{clf}: \textcolor{darkblue}{\textbf{\ipa{ɭɯ˧}}} 
\lhead{\firstmark}
\rhead{\botmark}

\subsection{\hspace{-0.5cm} {\Large \textcolor{darkblue}{\textbf{\ipa{ʑi˧˥}}} \textsubscript{1}}\hspace{0.5cm}[\kern2pt{\textcolor{darkblue}{\textbf{\ipa{ʑi˧˥}}}}\kern2pt]} \hypertarget{z£i\string_M\string_T1}{}
\markboth{\textcolor{darkblue}{\textbf{\ipa{ʑi˧˥}}} \textsubscript{1}}{}
\textcolor{teal}{\mytextsc{verb}} \hspace{4pt} Tone: MH.
\textcolor{Sepia}{\selectlanguage{english}To sleep.} \zh{睡觉。}  ¶ \textcolor{darkblue}{\textbf{\ipa{le˧-ʑi˧˥}}} \textcolor{Sepia}{\selectlanguage{english}\mytextsc{accomp}} \zh{\mytextsc{accomp}}  
 ¶ \textcolor{darkblue}{\textbf{\ipa{le˧-ʑi˧-ze˥}}} \textcolor{Sepia}{\selectlanguage{english}\mytextsc{accomp}  \mytextsc{pfv}} \zh{\mytextsc{accomp}  \mytextsc{pfv}}  
 ¶ \textcolor{darkblue}{\textbf{\ipa{æ˩ ʑi˧-ze˥}}} \textcolor{Sepia}{\selectlanguage{english}the chicken has gone asleep} \zh{鸡睡觉了}  
 ¶ \textcolor{darkblue}{\textbf{\ipa{le˧-ʑi˧-bi˧-ze˩!}}} \textcolor{Sepia}{\selectlanguage{english}I'm going to sleep!} \zh{要睡觉了!}  
 ¶ \textcolor{darkblue}{\textbf{\ipa{le˧-ʑi˧˥, | ʑi˧-mɤ˥-tʰɑ˩!}}} \textcolor{Sepia}{\selectlanguage{english}Were I to try to sleep, I would not be able to! / I would like to sleep, but I can't!} \zh{想睡,但睡不了!}  
 ¶ \textcolor{darkblue}{\textbf{\ipa{pʰæ˧tɕi˥-zo˩-ɳɯ˩ | mv̩˩zo˩-qɑ˥ ʑi˩}}} \textcolor{Sepia}{\selectlanguage{english}The young man sleeps with the young woman. (This phrasing refers rather bluntly to sexual intercourse, and is considered extremely rude.)} \zh{小伙子跟年轻女人睡!(庸俗说法)}  

\lhead{\firstmark}
\rhead{\botmark}

\subsection{\hspace{-0.5cm} {\Large \textcolor{darkblue}{\textbf{\ipa{ʑi˩˥}}}}\hspace{0.5cm}[\kern2pt{\textcolor{darkblue}{\textbf{\ipa{ʑi˩˥}}}}\kern2pt]} \hypertarget{z£i\string_B\string_T1}{}
\markboth{\textcolor{darkblue}{\textbf{\ipa{ʑi˩˥}}}}{}
\textcolor{teal}{\mytextsc{noun}} \hspace{4pt} Tone: LH.
\textcolor{Sepia}{\selectlanguage{english}Monkey, ape.} \zh{猴子。}  ¶ \textcolor{darkblue}{\textbf{\ipa{ʑi˩ dzɯ˧-ze˩}}} \textcolor{Sepia}{\selectlanguage{english}...has eaten (a/the) monkey} \zh{吃了猴子}  
 ¶ \textcolor{darkblue}{\textbf{\ipa{ʑi˩ hwæ˧-ze˩}}} \textcolor{Sepia}{\selectlanguage{english}...has bought (a/the) monkey} \zh{买了猴子}  
 \zh{量词}: \textcolor{darkblue}{\textbf{\ipa{mi˩}}}  \mytextsc{clf}: \textcolor{darkblue}{\textbf{\ipa{mi˩}}} 
\lhead{\firstmark}
\rhead{\botmark}

\subsection{\hspace{-0.5cm} {\Large \textcolor{darkblue}{\textbf{\ipa{*ʑi˩˧}}}}\hspace{0.5cm}[\kern2pt{\textcolor{darkblue}{\textbf{\ipa{ʑi˩˥}}}}\kern2pt]} \hypertarget{*z£i\string_B\string_M1}{}
\markboth{\textcolor{darkblue}{\textbf{\ipa{*ʑi˩˧}}}}{}
\textcolor{teal}{\mytextsc{noun}} \hspace{4pt} Tone: LM? LH?.
\textcolor{Sepia}{\selectlanguage{english}Building; houses.} \zh{房屋。}  ¶ \textcolor{darkblue}{\textbf{\ipa{ʑi˩ tsʰi˧˥, | æ̃˩ tsʰi˥}}} \textcolor{Sepia}{\selectlanguage{english}to build a house, to build a home (set phrase)} \zh{建房立家(固定词语)}  
 ¶ \textcolor{darkblue}{\textbf{\ipa{ʑi˩ tʰv̩˩˥}}} \textcolor{Sepia}{\selectlanguage{english}to found a new home, to build a new house} \zh{分家,建立自己的新房屋比如:孩子多,一个孩子建自己的房子)}  
 ¶ \textcolor{darkblue}{\textbf{\ipa{ʑi˩ qʰæ˧˥}}} \textcolor{Sepia}{\selectlanguage{english}to demolish a house} \zh{拆房子(这个例子是调查者构造的,F4确定是可以说的。造这个例子的目的有两个:看单音节词根“家”能不能跟其它动词结合,也试着确定它的调类。)}  
 ¶ \textcolor{darkblue}{\textbf{\ipa{*ʑi˩ hwæ˧}}} \textcolor{Sepia}{\selectlanguage{english}*to buy a house (example coined by the investigator, to investigate the monosyllable's potential to combine with other verbs, and its tone category; this example was refused by speaker F4)} \zh{*买房(这个例子是调查者构造的,F4确定是不可以说的。造这个例子的目的有两个:看单音节词根“家”能不能跟其它动词结合,也试着确定它的调类。)}  

\lhead{\firstmark}
\rhead{\botmark}

\lhead{\firstmark}
\rhead{\botmark}

\lhead{\firstmark}
\rhead{\botmark}

\lhead{\firstmark}
\rhead{\botmark}

\lhead{\firstmark}
\rhead{\botmark}

\lhead{\firstmark}
\rhead{\botmark}

\lhead{\firstmark}
\rhead{\botmark}

\lhead{\firstmark}
\rhead{\botmark}

\lhead{\firstmark}
\rhead{\botmark}

\lhead{\firstmark}
\rhead{\botmark}

\lhead{\firstmark}
\rhead{\botmark}

\lhead{\firstmark}
\rhead{\botmark}

\lhead{\firstmark}
\rhead{\botmark}

\lhead{\firstmark}
\rhead{\botmark}

\lhead{\firstmark}
\rhead{\botmark}

\lhead{\firstmark}
\rhead{\botmark}

\lhead{\firstmark}
\rhead{\botmark}

\lhead{\firstmark}
\rhead{\botmark}

\lhead{\firstmark}
\rhead{\botmark}

\lhead{\firstmark}
\rhead{\botmark}

\lhead{\firstmark}
\rhead{\botmark}

\lhead{\firstmark}
\rhead{\botmark}

\end{multicols}

\newpage
\section*{\centering Words for which no close equivalent could be found}
The list that follows groups words for which no close equivalents could be found. These negative pieces of information contain hints about the consultants' Na vocabulary and its 'soft shoulders'.
\begin{center}
\begin{longtable}{r|l}
sand\string_badger & \textcolor{brown}{\zh{猪獾}} \\
lynx & \textcolor{brown}{\zh{猞猁}} \\
pangolin & \textcolor{brown}{\zh{穿山甲}} \\
earlobe & \textcolor{brown}{\zh{耳垂}} \\
temples & \textcolor{brown}{\zh{太阳穴}} \\
dried\string_cheese & \textcolor{brown}{\zh{乳扇}} \\
toothbrush\string_of\string_pig\string_hair & \textcolor{brown}{\zh{猪鬃毛牙刷}} \\
catapult & \textcolor{brown}{\zh{抛石机}} \\
slingshot & \textcolor{brown}{\zh{绷弓子}} \\
Chinese\string_angelica & \textcolor{brown}{\zh{当归}} \\
Anisodus\string_tanguticus & \textcolor{brown}{\zh{山茛菪}} \\
wax\string_gourd & \textcolor{brown}{\zh{冬瓜}} \\
celery & \textcolor{brown}{\zh{芹菜}} \\
Ephedra\string_sinica & \textcolor{brown}{\zh{草麻黄}} \\
yew & \textcolor{brown}{\zh{红豆杉}} \\
Ligularia\string_fischeria & \textcolor{brown}{\zh{山紫菀}} \\
Selaginella & \textcolor{brown}{\zh{卷柏}} \\
wild\string_man & \textcolor{brown}{\zh{野人}} \\
twins & \textcolor{brown}{\zh{双胞胎}} \\
to\string_drink\string_with\string_a\string_pipe & \textcolor{brown}{\zh{用吸管喝}} \\
to\string_regret & \textcolor{brown}{\zh{后悔}} \\
\end{longtable}\end{center}
\end{document}
