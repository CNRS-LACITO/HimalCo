\documentclass[twoside,11pt]{book}
\author{Alexis Michaud}
% ajouté en novembre 2016, en remplacement de vmargin
\usepackage[paperwidth=185mm,paperheight=260mm,top=16mm,bottom=16mm,left=15mm,right=20mm]{geometry}
\usepackage{pifont}
\usepackage{fontspec}
\usepackage{natbib}
\usepackage{booktabs}
\usepackage{xltxtra}
\usepackage{polyglossia}
\usepackage[dvipsnames,table]{xcolor}
\usepackage{longtable}
\definecolor{darkblue}{rgb}{0,0,0.75}
\usepackage{float}
\usepackage{lineno}
\usepackage{textcomp}
\usepackage{memhfixc}
\usepackage{lscape}
\usepackage{amssymb}
\usepackage{multicol}
\setlength{\columnseprule}{1pt}
\setlength{\columnsep}{1.5cm}
%\setmainfont[Mapping=tex-text,Numbers=OldStyle,Ligatures=Common]{Charis SIL} 
\setmainfont[Mapping=tex-text,Numbers=OldStyle,Ligatures=Common,ItalicFont=*,ItalicFeatures=FakeSlant,Scale=MatchLowercase]{DoulosSIL}
\newfontfamily\phon[Mapping=tex-text,Ligatures=Common,Scale=MatchLowercase]{Charis SIL} % ou DoulosSIL
\newcommand{\ipa}[1]{{\phon #1}} % API jamais en italique
\newcommand{\grise}[1]{\cellcolor{lightgray}\textbf{#1}}
\newcommand{\bleute}[1]{\cellcolor{green}\textbf{#1}}
\newcommand{\rouge}[1]{\cellcolor{red}\textbf{#1}}
\newfontfamily\cn[Mapping=tex-text,Ligatures=Common,Scale=MatchUppercase]{SimSun} % pour le chinois
\newcommand{\zh}[1]{{\cn #1}}
\newcommand{\topic}{\textsc{dem}}
\newcommand{\tete}{\textsuperscript{\textsc{head}}}
\newcommand{\rc}{\textsubscript{\textsc{rc}}}
\XeTeXlinebreaklocale "zh" % 使用中文换行
\XeTeXlinebreakskip = 0pt plus 1pt  % CIRCG
\usepackage{fancyhdr}
\pagestyle{fancy}
\fancyheadoffset{3.4em}
\fancyhead[LE,RO]{\thepage} % Numéros de page sur les côtés extérieurs
% \setlength{\oddsidemargin}{12mm}
% \setlength{\evensidemargin}{18mm}
\usepackage[dvipdfmx,xetex,bigfiles,final,activate=onclick,deactivate=onclick,transparent,passcontext]{media9}
\usepackage{graphicx}
\usepackage[bookmarks=true,colorlinks,linkcolor=blue]{hyperref}
\hypersetup{bookmarks=false,bookmarksnumbered,bookmarksopenlevel=5,bookmarksdepth=5,xetex,colorlinks=true,linkcolor=blue,citecolor=blue}
\usepackage[all]{hypcap}
%%% retirés en novembre 2016 : pas utiles. Remplacé par: geometry.
%%\usepackage{vmargin}
%%% {marge gauche}{marge en haut}{marge droite}{marge en bas}{hauteur de l'entête}{distance entre l'entête et le texte}{hauteur du pied de page}{distance entre le texte et le pied de page}
%%\setmarginsrb{2cm}{1cm}{1.5cm}{1cm}{0.5cm}{1cm}{0.5cm}{1cm}

\def\mytextsc{\bgroup\obeyspaces\mytextscaux}
\def\mytextscaux#1{\mytextscauxii #1\relax\relax\egroup}
\def\mytextscauxii#1{%
\ifx\relax#1\else \ifcat#1\@sptoken{} \expandafter\expandafter\expandafter\mytextscauxii\else
\ifnum`#1=\uccode`#1 {\normalsize #1}\else {\footnotesize \uppercase{#1}}\fi \expandafter\expandafter\expandafter\mytextscauxii\expandafter\fi\fi}
% Ne pas afficher la numérotation des sections, sous-sections etc
\setcounter{secnumdepth}{0}
\newfontfamily\englishfont{Linux Libertine O}
\newfontfamily\frenchfont{EB Garamond}



\title{\zh{摩梭-汉-英-法词典}}
\setdefaultlanguage{chinese}
\setotherlanguages{english, french}

\begin{document}
\maketitle
\newpage
\markboth{INTRODUCTION}{}

\pagenumbering{roman}
	{\LARGE \textbf{Introduction}}
	\section{About the language} \label{sec:language}

This dictionary documents the lexicon of the Na language (\ipa{nɑ˩-ʐwɤ˥}) as spoken in and around the plain of Yongning, located in Southwestern China, at the border between Yunnan and Sichuan, at a latitude of 27°50’ N and a longitude of 100°41’ E. This language is known locally as ‘Mosuo'. 

	\section{Chronology and method} \label{sec:method}

The author's fieldwork on Yongning Na began in October 2006, with tone as its main focus (lexical tone, and tonal morphology). This required examining as many lexical items as possible to ensure that no tone category was overlooked, but lexicographic work was not in itself a priority. A list of words was begun through elicitation, and gradually expanded and corrected as narratives were recorded and transcribed; addition of new words was therefore a slow process. An advantage of placing the emphasis on text collection is that a context is available to help clarify the meaning of newly encountered words, also offering a basis for further discussion of their usage. But systematic elicitation of large amounts of vocabulary was not carried out, hence the limited number of entries: currently slightly under 3,000. 

Unless otherwise stated, all the data are from one language consultant, Mrs. Latami Dashilame (\ipa{lɑ˧tʰɑ˧mi˥ ʈæ˧ʂɯ˧-lɑ˩mv˩}; Chinese: \zh{拉它米打史拉么}). She was born in 1950 in the hamlet called \ipa{ə˧lɑ˧-ʁwɤ\#˥} in Na, close to the monastery of Yongning. The administrative coordinates of this village are: Yúnnán province, Lìjiāng municipality, Nínglàng Yí autonomous county, Yǒngníng district, Ālāwǎ village (\zh{云南省丽江市宁蒗彝族自治县永宁乡阿拉瓦村}). The choice to work in one location only, and essentially with one consultant, is, again, based on the investigator's focus on the tone system. There is considerable dialectal diversity within the Na area (much more so than in the Naxi-speaking area); the tone systems of different villages are conspicuously different, and this geographical diversity combines with dramatic differences across social groups, and across generations. The obvious thing to do seemed to be an in-depth description and analysis of the language as spoken by one person (simultaneously making a few forays into other idiolects and dialects). Data from other speakers are indicated using their codes in the author's database of speakers of Naish languages. Table \ref{tab:consul} provides the speaker codes.

\begin{table}[H]
	\caption{Language consultant codes}
	\centering \label{tab:consul}
	\begin{tabular}{lllllll}
		\toprule
		speaker code &   name &  year of birth \\
		\midrule
F4 (main consultant) & \ipa{lɑ˧tʰɑ˧mi˥ ʈæ˧ʂɯ˧-lɑ˩mv˩} & 1950 \\ F5 &  \ipa{ki˧zo˧} & 1973  \\ F6 &  \ipa{tɕʰi˧ɖv\#˥} & 1987 \\ M18 &  \ipa{lɑ˧tʰɑ˧mi˥ ʈæ˧ʂɯ˧-ʈæ˩ʈv˩} & 1972 \\ M21 & \ipa{ho˧dʑɤ˧tsʰe˥} & 1942 \\ M23 & \ipa{ɖɯ˩ɖʐɯ˧} & 1974 \\
		\bottomrule
	\end{tabular}
\end{table}

The list of words as of 2011 was deposited in the STEDT database (http://stedt.berkeley.edu/). The same year, under the impetus of Guillaume Jacques and Aimée Lahaussois, plans were made to bring the word list closer to the standards of a full-fledged dictionary. A project was deposited with the Agence Nationale de la Recherche, accepted in 2012, and begun in 2013: the HimalCo project (ANR-12-CORP-0006). Céline Buret, a computing engineer, worked with the project team for two years (Nov. 2014-Oct. 2015). She converted the data to the format of the Field Linguist's Toolbox (MDF), then produced scripts for conversion to a XML format complying with the LMF standard, allowing for automatic conversion to an online format as well as to LaTeX files (with PDF as the final output for circulation). In 2015, version 1.0 of the online and PDF versions of the dictionary were produced and published online, along with the source document in MDF (Toolbox) format.

\section{Guide to using the dictionary} \label{sec:howto}

	\subsection{Formats: trilingual Na-Chinese-English or Na-Chinese French} \label{sec:versions}

Entries and examples have translations into English, Chinese and French. Two language settings are offered for the PDF and online dictionary: either Na-Chinese-English, or Na-Chinese-French. The English and the French are not typeset alongside each other in the same document because distinguishing them visually is not obvious, even with the help of typographic devices such as using different fonts and colours. In the author's own experience, it was found that the presence of four languages alongside one another made consultation more difficult; specifically, English translations tended to be a distraction slowing down access to the French translations, and English users may similarly find that French clutters the layout. On the other hand, Chinese characters are visually well-distinct from Latin-based scripts, and so it did not appear necessary to separate the Chinese and produce a Na-Chinese version. Moreover, Chinese translations are often a useful complement to the translation in English (or French), as there are often closer equivalents: for instance \ipa{gɤ˧˥} translates straightforwardly as Chinese \zh{扛} whereas the English translation is more roundabout: ‘to carry on the shoulder'. Users who wish to have access to all the information can download the original file in Toolbox (MDF) format. 

	\subsection{Format of entries} \label{sec:entries}

Each entry contains
\begin{itemize}
	\item \textit{phonological transcription:} the form of the word in phonetic alphabet; tone is indicated in terms of phonological categories
	\item \textit{part of speech:} an indication of the part of speech, using a simple set of labels
	\item \textit{tone:} the tone category of the word. This information is already present in the phonological transcription; having it repeated on its own facilitates searches
	\item \textit{definitions} in Chinese and English
	\item \textit{examples} with translations
	\item \textit{links} to related words, such as synonyms, or constituent parts of complex words 
	\item \textit{classifier:} for nouns, an indication on the more commonly associated classifiers
\end{itemize}

Among examples, those elicited to verify the output of certain combinations of tones are marked as ‘PHONO': examples elicited for the purpose of the phonological study. Proverbs and sayings are marked as ‘PROVERB'.

Some pieces of information are not shown in the PDF and online versions. These are:
\begin{itemize}
	\item An indication of \textit{semantic domain}: ‘society', ‘house', ‘body', ‘plant', ‘animal'... No attempt was made to use a fine-grained classification of the sort found in the WordNet database of English, where nouns, verbs, adjectives and adverbs are grouped into sets of cognitive synonym \citep{Fellbaum 2005}. This is simply a rough division into subsets for convenient sorting; the labelling relies partly on form, and partly on semantic contents. As for other aspects in the dictionary, choices made reflect the investigator's research priorities: for instance, the entries for ‘day’, ‘night’, ‘month’, ‘year’ were tagged as “classifiers", along with all other nouns that can appear immediately after a numeral. This allowed easy extraction of all classifiers for the purpose of a study of the tone patterns of classifiers \citep{Michaud2013}. These lexical items could just as well have been tagged as 'time', in view of their semantic field. The numbers ‘100’, ‘1,000’ and ‘10,000’ were likewise labelled as “classifiers" rather than numerals.
	\item \textit{Notes on past notations:} information tracing the history of notations, from the first fieldwork to the current version. For instance, the entry \ipa{ŋwɤ˧pʰæ˧˥} ‘tile' has a note that indicates that it was initially written with a M.H tone pattern, and with vowel \ipa{æ} in both syllables: *\ipa{ŋwæ˧pʰæ˥}. The note explains that the perception of \ipa{æ} in the first syllable is due to a phonetic tendency towards regressive vowel harmony. Verifications are also consigned in this field. About half the entries have information of this type.
	\item \textit{glosses:} glosses in English, Chinese and French, intended for the glossing of texts. The dictionary adopts the abbreviations recommended in the Leipzig Glossing Rules \citep{Comrie}; all other terms are provided in full. Glosses mostly follow the choices made by \citep{Lidz2010}.
\end{itemize}

	\subsection{Part-of-speech labelling} \label{sec:pos}
	
Dictionary entries carry a part-of-speech label. A rough-and-ready typology has been followed: see the table below. Needless to say, this system has limitations: a refined typology would require subcategories, e.g. defining classifiers as a subset among nouns; and categories such as ‘adverbs' raise greater difficulties, lacking a clear definition.
\begin{table}
	\caption{Parts of speech}
	\centering \label{tab:consul}
	\begin{tabular}{lll}
		\toprule
		label & meaning & Leipzig Glossing Rules? y/n \\
		\midrule
		adj & adjective & y \\
		clf & classifier & y \\
		clitic & (same) & n \\
		cnj & conjunction & n \\
		ideophone & (same) & n \\
		disc.PTCL & discourse particle & n \\
		intj & interjection & n \\
		lnk & linker & n \\
		n & noun & y \\
		num & numeral & n \\
		pref & prefix & n \\
		postp & postposition & n \\
		pro & pronoun & n \\
		suff & suffix & n \\
		v & verb & y \\
		\bottomrule
	\end{tabular}
\end{table}

No attempt was made at including expressive noises in the dictionary, such as the sound \ipa{ɬː}. The meaning of this sound in Na can be characterized in the same way as that of words in the dictionary: the full definition would be that  \textit{it expresses enjoyment of food or drink (‘Yummy!'), and is also used to express admiration of a beautiful object, scene, or prospect}. But a reason for leaving it out is that, unlike interjections,  \ipa{ɬː} is not pronounced on expiratory airflow, but on inspired airflow. The air flows through the sides of the mouth, which is where saliva flows when one's mouth waters. Observations about such sounds (including clicks), like that of gestures, appeared to fall outside the scope of the dictionary.

	\subsection{Loanwords} \label{sec:loan}

Borrowings from Chinese and Tibetan are indicated as such in cases where identification seems straightforward. No efforts at systematic elicitation of borrowings from either language were made, but all loanwords occurring in texts were added to the dictionary. The information provided includes: donor language; form in the donor language; and explanations. When the number of syllables in the borrowed word is the same as in the donor language, the glosses in English (and French) start by the original word followed by two colons and a translation: e.g. ‘\zh{办法}::solution' for \ipa{pæ˧˥hwɤ˧}. 

	\section{Planned improvements and mid-term perspectives} \label{sec:improv}
	
This dictionary is conceived of as work-in-progress: successive versions will be released, probably every two years or so, (i) as an online dictionary in HTML format, (ii) as PDF documents, and (iii) in database format (native Toolbox/MDF format, then, in due course, the successors of this format). 

Planned improvements for future versions include the addition of
\begin{itemize}
	\item \textit{a phonetic transcription of tone as it surfaces on the item pronounced in isolation:} a surface-phonological transcription of tone, in addition to the indication of the underlying tone category
	\item \textit{audio files for each head word:} this function has successfully been tested, but the editing of audio files still needs to be conducted
	\item \textit{links to the entire set of online recordings}: listing all textual occurrences in the lexicon entry, with links to the audio file and its aligned transcription. Textual occurrences ultimately constitute the best resource to document a word's usage. The examples currently presented in the dictionary are few in number, compared to the occurrences in texts; and their context of use may not be clear, despite efforts at clarifying their nature (singling out examples elicited for the purpose of morpho-phonological investigation by the mention ‘PHONO') and at providing contextual information for examples jotted down during fieldwork.
	\item \textit{more cross-references} between entries, pointing to synonyms, etc.
\end{itemize}

Collaborations are welcome for the following improvements :
\begin{itemize}
	\item \textit{the vocabulary of religion:} the field of religion remains mostly unexplored; the main consultant and I both lack the command of Tibetan that would be essential for this part of the investigation, and involvement of consultants from the Yongning monastery did not prove feasible in view of current restrictions on contacts with foreigners
	\item \textit{plants and animals:} as a dweller of the Yongning plain, the main consultant does not have extensive knowledge of wild plants and animals; the number of entries recorded so far remains small, and some definitions are currently limited to general indications such as ‘a type of pine'. To arrive at exact identification, and at more extensive lexicographic coverage, would require collaboration with other consultants, and with botanists.
\end{itemize}

Last but not least, Roselle Dobbs began adding a proposed orthographic representation for each head word, using a transcription that she developed with Na consultants, with a view to use within the Na community. Until this task is completed and orthography added to the online dictionary, requests for further information about orthographic developments should be sent directly to: rosellemay@hotmail.com 

	\section{Other resources about Yongning Na} \label{sec:resources}
	
	In the classical tradition of linguistic fieldwork \citep{Dixon2007}, a language description should include a dictionary, a grammar, and a collection of texts. 
	
	\begin{itemize}
		\item \textit{A set of Na recordings with time-aligned transcriptions} is available from the Pangloss Collection \citep{Michailovsky2014}; the current web address is lacito.vjf.cnrs.fr/pangloss/languages/Na\_en.htm 
		\item \textit{The grammar} is still in its early stages of preparation. A preliminary draft of a book-length study of Na morpho-tonology can be found online: https://halshs.archives-ouvertes.fr/halshs-01094049/document It also contains detailed information on the phonemic analysis.
	\end{itemize}
	
A review of the literature about Na and the other languages of the Naish  group is provided (in Chinese) by \citet{Li2015}. For an English-language introduction, see \citet{Michaud2015b}.

I would gratefully receive any comments or notifications of errors that the reader may wish to bring to my attention: please send e-mail to michaud.cnrs@gmail.com 


	\section{Acknowledgments} \label{sec:ackno}

Many thanks to Picus Ding for putting me in touch with the Mosuo scholar Latami Dashi. Special thanks to Latami Dashi for supporting and encouraging my work with his mother Latami Dashilame over the years. Many thanks to the main consultant, Latami Dashilame, and to all family members. 

Many thanks to Céline Buret and Séverine Guillaume for their much-appreciated computational expertise, and to Guillaume Jacques for suggestions all along the way. Many thanks to connoisseurs of the Na culture and language for useful exchanges: Lamu Gatusa \zh{拉木嘎吐萨} (Chinese pen-name: \zh{石高峰}), Liberty Lidz, Christine Mathieu, Pascale-Marie Milan and He Sana \zh{何梭娜}. Special thanks to Roselle Dobbs for extensive discussions and vigorous proof-reading over the years. Many thanks to A Hui \zh{阿慧} (to my knowledge the first speaker of Mosuo to read a M.A. degree in language and linguistics) for suggesting corrections. Remaining errors are my own responsibility. 

I am grateful for the opportunity allowed me by my home institution, Centre National de la Recherche Scientifique, of staying in China in 2011-2012 for extensive fieldwork, through a temporary affiliation with the CNRS’s research centre in China: CEFC (Centre d’Etudes Français sur la Chine contemporaine). From November 2012 to June 2016, I was based at the international research institute MICA, in Hanoi, in an exceptionally stimulating environment allowing for close collaboration with colleagues from Asia and elsewhere. Special thanks to the heads of the institute, Phạm Thị Ngọc Yến (succeeded in 2015 by Nguyễn Việt Sơn) and Eric Castelli, for their support and encouragement.

I am grateful to the Dongba Culture Research Institute (\zh{丽江市东巴文化研究院}) in Lijiang and the Horse-Tea Road Culture Research Centre (\zh{云南大学茶马古道文化研究所}) in Kunming for inviting me to become an Adjunct member (\zh{外聘研究员}), and for facilitating administrative and practical matters; special thanks to Li Dejing \zh{李德静} and to Mu Jihong\zh{木霁弘}. At Yunnan University, many thanks are due to Duan Bingchang \zh{段炳昌}, Wang Weidong \zh{王卫东}, Zhao Yanzhen \zh{赵燕珍} and Yang Liquan \zh{杨立权} for their careful and sensitive management of fieldwork-related administrative matters.
	
So many people have supported this project that I must apologize for those names that should be here but were inadvertently left off the list.

This work was supported financially by the ANR project HimalCo (ANR-12-CORP-0006), and constitutes a contribution to the LabEx “Empirical Foundations of Linguistics" project (ANR-10-LABX-0083).

\begin{thebibliography}{7}
	\providecommand{\natexlab}[1]{#1}
	\providecommand{\url}[1]{#1}
	\providecommand{\urlprefix}{}
	\expandafter\ifx\csname urlstyle\endcsname\relax
	\providecommand{\doi}[1]{doi:\discretionary{}{}{}#1}\else
	\providecommand{\doi}{doi:\discretionary{}{}{}\begingroup
		\urlstyle{rm}\Url}\fi
	
	\bibitem[{Li(2015)}]{Li2015}
	Li Zihe [\zh{李子鹤}]. 2015.
	\newblock
	\zh{纳西语言研究回顾------兼论语言在文化研究中的基础地位}
	[{A} review of {Naxi} language studies, with a discussion of the fundamental
	role of cultural studies for linguistic research].
	\newblock \zh{茶马古道研究期刊} 4. 125--131.
	
	\bibitem[{Comrie et~al.()Comrie, Haspelmath \& Bickel}]{Comrie}
	Comrie, Bernard, Martin Haspelmath \& Balthasar Bickel. 2008.
	\newblock Leipzig {Glossing Rules}.
	\newblock
	\urlprefix\url{http://www.eva.mpg.de/lingua/resources/glossing-rules.php}.
	
	\bibitem[{Dixon(2007)}]{Dixon2007}
	Dixon, Robert~M. 2007.
	\newblock Field linguistics: a minor manual.
	\newblock \emph{Sprachtypologie und Universalienforschung} 60(1). 12--31.
	
	\bibitem[{Lidz(2010)}]{Lidz2010}
	Lidz, Liberty. 2010.
	\newblock \emph{A descriptive grammar of {Yongning Na} ({Mosuo})}.
	\newblock Austin: University of Texas, Department of linguistics dissertation.
	\newblock
	\urlprefix\url{https://repositories.lib.utexas.edu/bitstream/handle/2152/ETD-UT-2010-12-2643/LIDZ-DISSERTATION.pdf}.
	\newblock Ph. D.
	
	\bibitem[{Michailovsky et~al.(2014)Michailovsky, Mazaudon, Michaud, Guillaume,
		Fran{\c{c}}ois \& Adamou}]{Michailovsky2014}
	Michailovsky, Boyd, Martine Mazaudon, Alexis Michaud, S{\'{e}}verine Guillaume,
	Alexandre Fran{\c{c}}ois \& Evangelia Adamou. 2014.
	\newblock Documenting and researching endangered languages: the {Pangloss
		Collection}.
	\newblock \emph{Language Documentation and Conservation} 8. 119--135.
	\newblock \urlprefix\url{http://hdl.handle.net/10125/4621}.
	
	\bibitem[{Michaud(2013)}]{Michaud2013}
	Michaud, Alexis. 2013.
	\newblock The tone patterns of numeral-plus-classifier phrases in {Yongning
		Na}: a synchronic description and analysis.
	\newblock In Nathan Hill \& Tom Owen-Smith (eds.), \emph{Transhimalayan
		{Linguistics}. {Historical} and {Descriptive} {Linguistics} of the
		{Himalayan} {Area}} (Trends in {Linguistics}. {Studies} and {Monographs}
	[{TiLSM}] 266), 275--311. Berlin: De Gruyter Mouton.
	
	\bibitem[{Michaud et~al.(2015)Michaud, Limin \& Yaoping}]{Michaud2015b}
	Michaud, Alexis, He~Limin \& Zhong Yaoping. 2015.
	\newblock Naxi / {Naish}.
	\newblock In Rint Sybesma, Wolfgang Behr, Zev Handel \& C.T.~James Huang
	(eds.), \emph{Encyclopedia of {Chinese} {Language} and {Linguistics}},
	Leiden: Brill.
	
\end{thebibliography}

\cleardoublepage


	{\LARGE \textbf{\zh{前言}}}
	\section{ \zh{缘起}} \label{sec:language}

 \zh{ 在北纬27.50度,东经100.41度的交界点上生活着一个知名的族群——摩梭人。它奇特的民族风俗使它名扬天下,但是,它的语言研究还远远跟不上它的名气。作为一名语音学家找到了摩梭语言就如同进入了阿拉丁的宝库:摩梭语音的丰富多彩,尤其是声调在整个语法系统里所起的核心作用,独特而又迷人。
 	本词典的全部信息来自于一位充满独特个人魅力的摩梭阿妈——拉它米打史拉么。1950年出生的这位阿妈,差不多与新中国同年龄,她身上的故事就是摩梭人今天与昨天的缩影。在云南省丽江市宁蒗县永宁乡平静村独自养育了四个儿女,其中一位就是摩梭著名学者拉他咪达石(拉他咪王勇)。有幸结识拉他咪达石先生与他亲爱的母亲是这本词典得以与世人见面的关键,在此向他们表示最深切的感谢。}

	\section{ \zh{词典}} \label{sec:method}

 \zh{2006年在纳西语语音研究告一段落之后,我开始了对摩梭语音的分析。一开始的初衷只是进行音系分析,但却惊奇地发现摩梭声调不仅是单纯的音系问题,同时与语法有着千丝万缕的联系。这一发现,促使我开始对摩梭话进行全方位的研究。研究方法以搜集来的长篇语料为蓝本,在搜集语料的同时进行音系实验(系统的机械问答),并以长篇语料中的词汇为基准,汇集成为本部词典。词典目前的条目数量不到三千,但我的初衷并不致力于单纯的词汇收集而是希望能在有限的条目中最大限度地呈现摩梭语音的独特性。

本词典以国际音标为注音基础。声调标注系统由于比较复杂,因此启用了一套专门符号,该符号的用法及说明见}Michaud 2015\zh{。

本词典有三种呈现格式,在线词典、PDF文本、与Toolbox资料库。内容将陆续进行更正及删补。同时欢迎读者的指正与批评。来信请寄}alexis.michaud@vjf.cnrs.fr \zh{。}

\section{ \zh{其它}} \label{sec:other}

 \zh{著名语言学家孙天心提出,语言学家有三件宝:长篇语料、词典与参考语法。以这个标准来看,长篇语料就是我囊中最大的宝贝,历时十年搜集而来的近二十个小时的资料及逐句标注翻译,已经全部在线公开。在线地址:“泛语资料库”(}Pangloss Collection \zh{),}lacito.vjf.cnrs.fr/pangloss/languages/Na\_en.htm \zh{。下一步的相关工作,是完成一本《摩梭话声调研究》,初稿也已上线,地址:}https://halshs.archives-ouvertes.fr/halshs-01094049/document \zh{。}

	\section{ \zh{致谢}} \label{sec:thks}
	
\zh{拉它米打史拉么阿妈及全家人、拉他咪达石(摩梭知名学者,出版有多部摩梭文化研究专著)、李德静(丽江市东巴文化研究院院长)、黄行(中央社会科学院民族学研究所所长)、段炳昌(云南大学人文学院院长)、木霁弘(云南大学茶马古道文化研究所所长)、王卫东(云南大学人文学院中文系系主任)、和学光(云南党校图书馆馆长)、赵燕珍(云南大学人文学院中文系教授)、杨立权(云南大学人文学院中文系教授)、丁思之教授(普米语研究专家)、拉木嘎吐萨(摩梭知名学者)、阿慧(摩梭人研究生)、}Céline Buret\zh{(工程师)、}Séverine Guillaume\zh{(工程师)、向柏霖(}Guillaume Jacques\zh{,语言学专家)、
李力(}Liberty Lidz\zh{(,摩梭话研究专家)、杜玫瑰(}Roselle Dobbs\zh{,摩梭研究者)、}Pascale-Marie Milan\zh{(人类学家,摩梭文化研究专家)、}Christine Mathieu\zh{(人类学家,摩梭文化研究专家):感谢你们的大力支持。

在此也向成书过程中曾给予帮助的许多朋友和专家们一并致谢!}




	\section{ \zh{参考书目}} \label{sec:refs}
	\begin{itemize}
		\item \zh{李子鹤. 纳西语言研究回顾——兼论语言在文化研究中的基础地位[J]. 茶马古道研究期刊, 2015, 4: 125–131.}
		\item \zh{孙天心. 藏缅语的调查[J]. 语言学论丛, 2007, 36: 98–107.}
		\item LIDZ L. A descriptive grammar of Yongning Na (Mosuo) [D]. Austin: University of Texas, Department of linguistics, 2010.
		\zh{(下载地址:}https://repositories.lib.utexas.edu/bitstream/handle/2152/ETD-UT-2010-12-2643/LIDZ-DISSERTATION.pdf\zh{)}
		\item MICHAUD A. Phrasing, prominence, and morphotonology: How utterances are divided into tone groups in Yongning Na [J]. Bulletin of Chinese Linguistics, 2015. \zh{(下载出版前版本的地址:}https://halshs.archives-ouvertes.fr/halshs-01162331\zh{)}
	\end{itemize}

\cleardoublepage
\pagenumbering{arabic}
\setcounter{page}{1}
 

	


\def\mytextsc{\bgroup\obeyspaces\mytextscaux}
\def\mytextscaux#1{\mytextscauxii #1\relax\relax\egroup}
\def\mytextscauxii#1{%
\ifx\relax#1\else \ifcat#1\@sptoken{} \expandafter\expandafter\expandafter\mytextscauxii\else
\ifnum`#1=\uccode`#1 {\normalsize #1}\else {\footnotesize \uppercase{#1}}\fi \expandafter\expandafter\expandafter\mytextscauxii\expandafter\fi\fi}

\setlength\parindent{0cm}

\addmediapath{/home/benjamin/Projets/pylmflib-1.1/examples/na/mp3}
\addmediapath{/home/benjamin/Projets/pylmflib-1.1/examples/na/mp3/mp3}
\addmediapath{/home/benjamin/Projets/pylmflib-1.1/examples/na/mp3/wav}
\graphicspath{{/home/benjamin/Projets/pylmflib-1.1/examples/na/pylmflib/output/img/}}

\newpage
\begin{multicols}{2}

\newpage
\section*{\centering- \textcolor{darkblue}{\textbf{\ipa{ɑ}}} -}
\subsection{\hspace{-0.5cm} {\Large \textcolor{darkblue}{\textbf{\ipa{ɑ˩mi\#˥}}}}\hspace{0.5cm}[\kern2pt{\textcolor{darkblue}{\textbf{\ipa{ɑ˩mi˥}}}}\kern2pt]} \hypertarget{A\string_Bmi\#\string_T1}{}
\markboth{\textcolor{darkblue}{\textbf{\ipa{ɑ˩mi\#˥}}}}{}
\textcolor{teal}{\zh{名词}} \hspace{4pt} \zh{声调类:} LM+\#H.
\zh{母鹅。} \textcolor{Sepia}{\selectlanguage{english}Female goose.} \textcolor{PineGreen}{\selectlanguage{french}Oie (femelle).}  ¶ \textcolor{darkblue}{\textbf{\ipa{ɑ˩mi˧-ɑ˥pʰv̩˩}}} \zh{公鹅与母鹅} \textcolor{Sepia}{\selectlanguage{english}goose and gander} \textcolor{PineGreen}{\selectlanguage{french}oie et jars}  
 \zh{量词}: \textcolor{darkblue}{\textbf{\ipa{mi˩}}} 
\lhead{\firstmark}
\rhead{\botmark}

\subsection{\hspace{-0.5cm} {\Large \textcolor{darkblue}{\textbf{\ipa{ɑ˩pʰo˩}}}}\hspace{0.5cm}[\kern2pt{\textcolor{darkblue}{\textbf{\ipa{ɑ˩pʰo˩˥}}}}\kern2pt]} \hypertarget{A\string_Bp\string_ho\string_B1}{}
\markboth{\textcolor{darkblue}{\textbf{\ipa{ɑ˩pʰo˩}}}}{}
\textcolor{teal}{\zh{助词}} \hspace{4pt} \zh{声调类:} L.
\zh{外面。} \textcolor{Sepia}{\selectlanguage{english}Outside.} \textcolor{PineGreen}{\selectlanguage{french}Dehors, à l'extérieur.}  ¶ \textcolor{darkblue}{\textbf{\ipa{ɑ˩pʰo˩ bi˩˥}}} \zh{出去,解手} \textcolor{Sepia}{\selectlanguage{english}to go outside; to attend to the call of nature} \textcolor{PineGreen}{\selectlanguage{french}aller dehors; aller faire ses besoins}  
 ¶ \textcolor{darkblue}{\textbf{\ipa{ɑ˩pʰo˩ bi˩-ze˥!}}} \zh{出去了! / 出去解手!} \textcolor{Sepia}{\selectlanguage{english}Let's go out! / I must answer the call of nature!} \textcolor{PineGreen}{\selectlanguage{french}On sort! / [Je] vais faire mes besoins!}  
 ¶ \textcolor{darkblue}{\textbf{\ipa{ɑ˩pʰo˩ bi˩-ʂv̩˥ɖv̩˩!}}} \zh{(她)想出去!(情景:婴儿看外边,觉得她好像想出去。)} \textcolor{Sepia}{\selectlanguage{english}[She] wants to go out! (Context: on a sunny day, a family member senses that a toddler wants to go and play outside.)} \textcolor{PineGreen}{\selectlanguage{french}[Elle] a envie de sortir! (Contexte: par une journée ensoleillée, un membre de la famille sent qu'un enfant aurait envie d'aller jouer dehors.)}  
 ¶ \textcolor{darkblue}{\textbf{\ipa{ə˧dɑ˥ | ə˩pʰo˩ hɯ˩-ze˥!}}} \zh{爸爸出去了!} \textcolor{Sepia}{\selectlanguage{english}Daddy went out! (Context: explanation provided to a little child who has just woken up and looks for her dad.)} \textcolor{PineGreen}{\selectlanguage{french}Papa est sorti! (Adressé à une petite fille qui vient de se réveiller de sa sieste et cherche son père.)}  
 ¶ \textcolor{darkblue}{\textbf{\ipa{ə˩pʰo˩-bv̩˥ | lo˧ ʝi˧}}} \zh{在外边工作:去帮别人家的忙(特别是收庄稼的时候),或者到城市里面打工} \textcolor{Sepia}{\selectlanguage{english}to work outside the household: to help other families (in particular for harvesting); to go and look for work in a city} \textcolor{PineGreen}{\selectlanguage{french}travailler à l'extérieur, aider d'autres familles (par exemple pour la récolte); aussi : aller chercher du travail à la ville}  

\lhead{\firstmark}
\rhead{\botmark}

\subsection{\hspace{-0.5cm} {\Large \textcolor{darkblue}{\textbf{\ipa{ɑ˩pʰo˩-hĩ˩}}}}\hspace{0.5cm}[\kern2pt{\textcolor{darkblue}{\textbf{\ipa{xxxx non-correspondance entre le nombre de morphèmes et le nombre de tons de morphèmes}}}}\kern2pt]} \hypertarget{A\string_Bp\string_ho\string_B-hi\string_~\string_B1}{}
\markboth{\textcolor{darkblue}{\textbf{\ipa{ɑ˩pʰo˩-hĩ˩}}}}{}
\textcolor{teal}{\zh{名词}} \hspace{4pt} \zh{声调类:} L.
\zh{外人,别人。农村人称呼所有城里人为外面的人。} \textcolor{Sepia}{\selectlanguage{english}Outsiders; strangers; other people.} \textcolor{PineGreen}{\selectlanguage{french}Quelqu'un d'autre; une personne extérieure à la famille.} 
\lhead{\firstmark}
\rhead{\botmark}

\subsection{\hspace{-0.5cm} {\Large \textcolor{darkblue}{\textbf{\ipa{ɑ˩pʰv̩˧˥}}}}\hspace{0.5cm}[\kern2pt{\textcolor{darkblue}{\textbf{\ipa{ɑ˩pʰv̩˧˥}}}}\kern2pt]} \hypertarget{A\string_Bp\string_hv\string_=\string_M\string_T1}{}
\markboth{\textcolor{darkblue}{\textbf{\ipa{ɑ˩pʰv̩˧˥}}}}{}
\textcolor{teal}{\zh{动词}} \hspace{4pt} \zh{声调类:} LM+MH\#.
\zh{打饱嗝儿。} \textcolor{Sepia}{\selectlanguage{english}To belch, to burp.} \textcolor{PineGreen}{\selectlanguage{french}Roter.} 
\lhead{\firstmark}
\rhead{\botmark}

\subsection{\hspace{-0.5cm} {\Large \textcolor{darkblue}{\textbf{\ipa{ɑ˩pʰv̩\#˥}}}}\hspace{0.5cm}[\kern2pt{\textcolor{darkblue}{\textbf{\ipa{ɑ˩pʰv̩˥}}}}\kern2pt]} \hypertarget{A\string_Bp\string_hv\string_=\#\string_T1}{}
\markboth{\textcolor{darkblue}{\textbf{\ipa{ɑ˩pʰv̩\#˥}}}}{}
\textcolor{teal}{\zh{名词}} \hspace{4pt} \zh{声调类:} LM+\#H.
\zh{公鹅。} \textcolor{Sepia}{\selectlanguage{english}Gander; male goose.} \textcolor{PineGreen}{\selectlanguage{french}Jars: mâle de l'oie.}  \zh{量词}: \textcolor{darkblue}{\textbf{\ipa{mi˩}}} 
\lhead{\firstmark}
\rhead{\botmark}

\subsection{\hspace{-0.5cm} {\Large \textcolor{darkblue}{\textbf{\ipa{ɑ˩ʁo˧}}}}\hspace{0.5cm}[\kern2pt{\textcolor{darkblue}{\textbf{\ipa{ɑ˩ʁo˥}}}}\kern2pt]} \hypertarget{A\string_BRo\string_M1}{}
\markboth{\textcolor{darkblue}{\textbf{\ipa{ɑ˩ʁo˧}}}}{}
\textcolor{teal}{\zh{名词}} \hspace{4pt} \zh{声调类:} LM.
\zh{家、家里。} \textcolor{Sepia}{\selectlanguage{english}Home, central room in the house.} \textcolor{PineGreen}{\selectlanguage{french}Le foyer, la pièce principale; la maison.}  ¶ \textcolor{darkblue}{\textbf{\ipa{ɑ˩ʁo˧-hĩ\#˥}}} \zh{家人(住在一起的家人)} \textcolor{Sepia}{\selectlanguage{english}members of the family, family members (who live under the same roof), lineage} \textcolor{PineGreen}{\selectlanguage{french}la maisonnée, les gens de la famille (proche: ceux qui habitent sous le même toit), la lignée}  
 ¶ \textcolor{darkblue}{\textbf{\ipa{ɑ˩ʁo˧=ɻæ˩}}} \zh{家人、家族(住在一起的家人)} \textcolor{Sepia}{\selectlanguage{english}the members of the family, the family group (living under the same roof)} \textcolor{PineGreen}{\selectlanguage{french}la maisonnée, les gens de la famille}  
 ¶ \textcolor{darkblue}{\textbf{\ipa{njɤ˧ | ɑ˩ʁo˧}}} \zh{我家} \textcolor{Sepia}{\selectlanguage{english}my home, my house} \textcolor{PineGreen}{\selectlanguage{french}mon foyer, ma maison}  
 ¶ \textcolor{darkblue}{\textbf{\ipa{njɤ˧ | ɑ˩ʁo˧=ɻ̍˩}}} \zh{我的家族} \textcolor{Sepia}{\selectlanguage{english}my family, my lineage, my kin} \textcolor{PineGreen}{\selectlanguage{french}ma famille, ma lignée}  
 ¶ \textcolor{darkblue}{\textbf{\ipa{no˧ | ɑ˩ʁo˧}}} \zh{你家} \textcolor{Sepia}{\selectlanguage{english}your home, your house} \textcolor{PineGreen}{\selectlanguage{french}ton foyer, ta maison}  
 ¶ \textcolor{darkblue}{\textbf{\ipa{ɑ˩ʁo˧ ʝi˧}}} \zh{管家里的事情(如:分配工作、家务等)} \textcolor{Sepia}{\selectlanguage{english}to take care of the household, to look after the affairs of the family (in particular: distributing work to the various members, and ensuring that the supplies are not running low)} \textcolor{PineGreen}{\selectlanguage{french}gérer la maisonnée, s'occuper de la famille (tâche de la personne qui répartit les travaux à accomplir, veille aux approvisionnements...)}  
 \zh{量词}: \textcolor{darkblue}{\textbf{\ipa{ɭɯ˧}}} 
\lhead{\firstmark}
\rhead{\botmark}

\subsection{\hspace{-0.5cm} {\Large \textcolor{darkblue}{\textbf{\ipa{ɑ˩zo\#˥}}}}\hspace{0.5cm}[\kern2pt{\textcolor{darkblue}{\textbf{\ipa{ɑ˩zo˥}}}}\kern2pt]} \hypertarget{A\string_Bzo\#\string_T1}{}
\markboth{\textcolor{darkblue}{\textbf{\ipa{ɑ˩zo\#˥}}}}{}
\textcolor{teal}{\zh{名词}} \hspace{4pt} \zh{声调类:} LM+\#H.
\zh{小鹅。} \textcolor{Sepia}{\selectlanguage{english}Baby goose.} \textcolor{PineGreen}{\selectlanguage{french}Oison, petit de l'oie.} 
\lhead{\firstmark}
\rhead{\botmark}

\subsection{\hspace{-0.5cm} {\Large \textcolor{darkblue}{\textbf{\ipa{ɑ˩˧}}}}\hspace{0.5cm}[\kern2pt{\textcolor{darkblue}{\textbf{\ipa{ɑ˩˥}}}}\kern2pt]} \hypertarget{A\string_B\string_M1}{}
\markboth{\textcolor{darkblue}{\textbf{\ipa{ɑ˩˧}}}}{}
\textcolor{teal}{\zh{名词}} \hspace{4pt} \zh{声调类:} LM.
\zh{鹅。} \textcolor{Sepia}{\selectlanguage{english}Goose.} \textcolor{PineGreen}{\selectlanguage{french}Oie.}  ¶ \textcolor{darkblue}{\textbf{\ipa{ɑ˩ dzɯ˥-ze˩}}} \zh{吃了鹅} \textcolor{Sepia}{\selectlanguage{english}...has eaten (a/the) goose} \textcolor{PineGreen}{\selectlanguage{french}...a mangé (une) oie}  
 ¶ \textcolor{darkblue}{\textbf{\ipa{ɑ˩ hwæ˧-ze˧}}} \zh{买了鹅} \textcolor{Sepia}{\selectlanguage{english}...has bought (a/the) goose} \textcolor{PineGreen}{\selectlanguage{french}...a acheté (une) oie}  
 \zh{量词}: \textcolor{darkblue}{\textbf{\ipa{mi˩}}} 
\lhead{\firstmark}
\rhead{\botmark}

\newpage
\section*{\centering- \textcolor{darkblue}{\textbf{\ipa{æ}}} \textcolor{darkblue}{\textbf{\ipa{æ̃}}} -}
\subsection{\hspace{-0.5cm} {\Large \textcolor{darkblue}{\textbf{\ipa{æ˧bæ˧}}}}\hspace{0.5cm}[\kern2pt{\textcolor{darkblue}{\textbf{\ipa{æ˧bæ˧}}}}\kern2pt]} \hypertarget{\{\string_Mb\{\string_M1}{}
\markboth{\textcolor{darkblue}{\textbf{\ipa{æ˧bæ˧}}}}{}
\textcolor{teal}{\zh{名词}} \hspace{4pt} \zh{声调类:} M.
\zh{甲状腺肿瘤。} \textcolor{Sepia}{\selectlanguage{english}Goitre.} \textcolor{PineGreen}{\selectlanguage{french}Goître.}  \zh{量词}: \textcolor{darkblue}{\textbf{\ipa{ɭɯ˧}}} 
\lhead{\firstmark}
\rhead{\botmark}

\subsection{\hspace{-0.5cm} {\Large \textcolor{darkblue}{\textbf{\ipa{æ˧bæ˧-ʈʂʰæ˧ɣɯ\#˥}}}}\hspace{0.5cm}[\kern2pt{\textcolor{darkblue}{\textbf{\ipa{xxxx non-correspondance entre le nombre de morphèmes et le nombre de tons de morphèmes}}}}\kern2pt]} \hypertarget{\{\string_Mb\{\string_M-t`s`\string_h\{\string_MGM\#\string_T1}{}
\markboth{\textcolor{darkblue}{\textbf{\ipa{æ˧bæ˧-ʈʂʰæ˧ɣɯ\#˥}}}}{}
\textcolor{teal}{\zh{名词}} \hspace{4pt} \zh{声调类:} \#H.
\zh{海带。} \textcolor{Sepia}{\selectlanguage{english}Kelp (literally “medicine against goiter”, because kelp was introduced in Yongning as a means to provide iodine as a diet supplement, to prevent goiters).} \textcolor{PineGreen}{\selectlanguage{french}Algue; littéralement “médicament contre le goître” car tel était le motif de la diffusion à Yongning de l'algue, qui contient de l'iode.} 
\lhead{\firstmark}
\rhead{\botmark}

\subsection{\hspace{-0.5cm} {\Large \textcolor{darkblue}{\textbf{\ipa{æ˧ʝi˩}}}}\hspace{0.5cm}[\kern2pt{\textcolor{darkblue}{\textbf{\ipa{æ˧ʝi˩}}}}\kern2pt]} \hypertarget{\{\string_Mj££i\string_B1}{}
\markboth{\textcolor{darkblue}{\textbf{\ipa{æ˧ʝi˩}}}}{}
\textcolor{teal}{\zh{名词}} \hspace{4pt} \zh{声调类:} L\#.
\zh{叫声。} \textcolor{Sepia}{\selectlanguage{english}Cry.} \textcolor{PineGreen}{\selectlanguage{french}Cri.}  ¶ \textcolor{darkblue}{\textbf{\ipa{æ˧ʝi˩ kʰɯ˩}}} \zh{叫} \textcolor{Sepia}{\selectlanguage{english}to shout} \textcolor{PineGreen}{\selectlanguage{french}crier}  
 ¶ \textcolor{darkblue}{\textbf{\ipa{no˧ | æ˧ʝi˩ tʰɑ˩-kʰɯ˩! | no˧ se˧dʑæ˩ɻæ˩-gv˩! |}}} \zh{别那么大声!} \textcolor{Sepia}{\selectlanguage{english}Don't make noise!} \textcolor{PineGreen}{\selectlanguage{french}Ne fais pas tant de bruit! (Contexte: réprimande qu'on adressait aux gens qui parlaient trop fort, qui élevaient la voix.)}  

\lhead{\firstmark}
\rhead{\botmark}

\subsection{\hspace{-0.5cm} {\Large \textcolor{darkblue}{\textbf{\ipa{æ˧ɲi\#˥}}}}\hspace{0.5cm}[\kern2pt{\textcolor{darkblue}{\textbf{\ipa{æ˧ɲi˧}}}}\kern2pt]} \hypertarget{\{\string_MJi\#\string_T1}{}
\markboth{\textcolor{darkblue}{\textbf{\ipa{æ˧ɲi\#˥}}}}{}
\textcolor{teal}{\zh{名词}} \hspace{4pt} \zh{声调类:} \#H.
\zh{唢呐。} \textcolor{Sepia}{\selectlanguage{english}Suona, trumpet.} \textcolor{PineGreen}{\selectlanguage{french}Clarinette.}  \zh{量词}: \textcolor{darkblue}{\textbf{\ipa{ɭɯ˧}}} 
\lhead{\firstmark}
\rhead{\botmark}

\subsection{\hspace{-0.5cm} {\Large \textcolor{darkblue}{\textbf{\ipa{æ˧ʁwæ˧}}}}\hspace{0.5cm}[\kern2pt{\textcolor{darkblue}{\textbf{\ipa{æ˧ʁwæ˧}}}}\kern2pt]} \hypertarget{\{\string_MRw\{\string_M1}{}
\markboth{\textcolor{darkblue}{\textbf{\ipa{æ˧ʁwæ˧}}}}{}
\textcolor{teal}{\zh{名词}} \hspace{4pt} \zh{声调类:} M.
\zh{杏。} \textcolor{Sepia}{\selectlanguage{english}Apricot.} \textcolor{PineGreen}{\selectlanguage{french}Abricot.}  ¶ \textcolor{darkblue}{\textbf{\ipa{æ˧ʁwæ˧ | ɖɯ˧-ɭɯ˧}}} \zh{一颗杏} \textcolor{Sepia}{\selectlanguage{english}an apricot} \textcolor{PineGreen}{\selectlanguage{french}un abricot}  
 \zh{量词}: \textcolor{darkblue}{\textbf{\ipa{ɭɯ˧}}} 
\lhead{\firstmark}
\rhead{\botmark}

\subsection{\hspace{-0.5cm} {\Large \textcolor{darkblue}{\textbf{\ipa{æ˧ʂæ\#˥}}}}\hspace{0.5cm}[\kern2pt{\textcolor{darkblue}{\textbf{\ipa{æ˧ʂæ˧}}}}\kern2pt]} \hypertarget{\{\string_Ms`\{\#\string_T1}{}
\markboth{\textcolor{darkblue}{\textbf{\ipa{æ˧ʂæ\#˥}}}}{}
\textcolor{teal}{\zh{助词}} \hspace{4pt} \zh{声调类:} \#H.
\zh{从前。} \textcolor{Sepia}{\selectlanguage{english}Yore, long ago.} \textcolor{PineGreen}{\selectlanguage{french}Jadis, dans le passé.}  ¶ \textcolor{darkblue}{\textbf{\ipa{æ˧ʂæ˧-kɤ˥ʈʂɯ˩}}} \zh{过去的说法,过去的口传文化} \textcolor{Sepia}{\selectlanguage{english}Sayings of the old times, oral traditions of the old times} \textcolor{PineGreen}{\selectlanguage{french}Dires du temps jadis, traditions orales}  

\lhead{\firstmark}
\rhead{\botmark}

\subsection{\hspace{-0.5cm} {\Large \textcolor{darkblue}{\textbf{\ipa{æ˧ʂæ˧}}}}\hspace{0.5cm}[\kern2pt{\textcolor{darkblue}{\textbf{\ipa{æ˧ʂæ˧}}}}\kern2pt]} \hypertarget{\{\string_Ms`\{\string_M1}{}
\markboth{\textcolor{darkblue}{\textbf{\ipa{æ˧ʂæ˧}}}}{}
\textcolor{teal}{\zh{名词}} \hspace{4pt} \zh{声调类:} M.
\zh{一座山的名字。} \textcolor{Sepia}{\selectlanguage{english}Name of a mountain: one of the two main mountains in the vicinity of the Yongning plain. It is a masculine mountain (“the young man”: \textcolor{darkblue}{\textbf{\ipa{/pʰæ˧tɕi˥/}}}), the counterpart to the feminine mountain \textcolor{darkblue}{\textbf{\ipa{/kɤ˧mv̩˧˥/}}} (“the young woman”: \textcolor{darkblue}{\textbf{\ipa{mi˩zɯ˩˥/}}}).} \textcolor{PineGreen}{\selectlanguage{french}Nom d'une montagne: l'une des deux principales montagnes autour de la plaine de Yongning, la montagne masculine (“le jeune homme”: \textcolor{darkblue}{\textbf{\ipa{/pʰæ˧tɕi˥/}}}); l'autre étant la montagne \textcolor{darkblue}{\textbf{\ipa{/kɤ˧mv̩˧˥/}}}, montagne féminine (“la jeune femme”: \textcolor{darkblue}{\textbf{\ipa{mi˩zɯ˩˥/}}}).}  ¶ \textcolor{darkblue}{\textbf{\ipa{kɤ˧mv̩˧˥, | æ˧ʂæ˧, | ŋwɤ˧hɑ̃˩, | ʂwæ˧gv̩\#˥, | nɑ˩tsʰi˩˥ | -tɕʰɤ˧pɤ˧mi\#˥, | qv̩˧ɻ̍˧-ʈʂʰɑ˧nɑ˥ |}}} \zh{永宁地区有固定名字的六座山。其它的山,因为没有重要的象征意义,因此没有取名。} \textcolor{Sepia}{\selectlanguage{english}The six mountains of Yongning that carry a name and have a definite symbolic value. The other mountains do not have comparable symbolic value, and fewer people use specific names for them.} \textcolor{PineGreen}{\selectlanguage{french}Les six montagnes de Yongning qui portent un nom. Les autres sommets du voisinage n'ont pas une valeur symbolique comparable, et ne portent pas de nom communément utilisé.}  

\lhead{\firstmark}
\rhead{\botmark}

\subsection{\hspace{-0.5cm} {\Large \textcolor{darkblue}{\textbf{\ipa{æ˧ʂæ˧-pi˧mv̩˧˥}}}}\hspace{0.5cm}[\kern2pt{\textcolor{darkblue}{\textbf{\ipa{xxxx non-correspondance entre le nombre de morphèmes et le nombre de tons de morphèmes}}}}\kern2pt]} \hypertarget{\{\string_Ms`\{\string_M-pi\string_Mmv\string_=\string_M\string_T1}{}
\markboth{\textcolor{darkblue}{\textbf{\ipa{æ˧ʂæ˧-pi˧mv̩˧˥}}}}{}
\textcolor{teal}{\zh{名词}} \hspace{4pt} \zh{声调类:} MH\#.
\zh{传统故事。} \textcolor{Sepia}{\selectlanguage{english}Folk tale, tradition; this term is more colloquial than \textcolor{darkblue}{\textbf{\ipa{/æ˧ʂæ˧-tɑ˩mv̩˩/}}}.} \textcolor{PineGreen}{\selectlanguage{french}Conte, récit du temps jadis; terme plus familier que \textcolor{darkblue}{\textbf{\ipa{/æ˧ʂæ˧-tɑ˩mv̩˩/}}}.}  \zh{量词}: \textcolor{darkblue}{\textbf{\ipa{kʰwɤ˥}}} 
\lhead{\firstmark}
\rhead{\botmark}

\subsection{\hspace{-0.5cm} {\Large \textcolor{darkblue}{\textbf{\ipa{æ˧ʂæ˧-qʰwæ\#˥}}}}\hspace{0.5cm}[\kern2pt{\textcolor{darkblue}{\textbf{\ipa{xxxx non-correspondance entre le nombre de morphèmes et le nombre de tons de morphèmes}}}}\kern2pt]} \hypertarget{\{\string_Ms`\{\string_M-q\string_hw\{\#\string_T1}{}
\markboth{\textcolor{darkblue}{\textbf{\ipa{æ˧ʂæ˧-qʰwæ\#˥}}}}{}
\textcolor{teal}{\zh{名词}} \hspace{4pt} \zh{声调类:} \#H.
\zh{口传文化。直译:“(来自)古时候的寓意”。} \textcolor{Sepia}{\selectlanguage{english}Oral tradition; literally: “messages from old times”.} \textcolor{PineGreen}{\selectlanguage{french}Tradition orale; littéralement: “messages du temps jadis”.}  ¶ \textcolor{darkblue}{\textbf{\ipa{æ˧ʂæ˧-qʰwæ˧-ɳɯ˥ | dʑo˩-ɲi˥!}}} \zh{“(这些道理,不是我个人的意见:)传统中是这样讲的! / 咱们的口传文化中就是这么讲的!”(情景:一个人提到一个谚语,也强调这些不是空话,而是重要的一个道理。)} \textcolor{Sepia}{\selectlanguage{english}“I'm not making this up:) this is part of what the old folks have passed down to us! / This is what our traditions say!” (Context: the speaker cites a proverb or saying, and emphasizes that it is to be taken seriously.)} \textcolor{PineGreen}{\selectlanguage{french}“(C'est pas moi qui invente ça:) c'est un dicton d'autrefois/ c'est quelque chose qui existe dans la tradition!” (Commentaire de quelqu'un qui cite un proverbe/dicton, et souligne qu'il ne s'agit pas de paroles en l'air, mais de vérités.)}  

\lhead{\firstmark}
\rhead{\botmark}

\subsection{\hspace{-0.5cm} {\Large \textcolor{darkblue}{\textbf{\ipa{æ˧ʂæ˧-qʰwɤ˧˥}}}}\hspace{0.5cm}[\kern2pt{\textcolor{darkblue}{\textbf{\ipa{xxxx non-correspondance entre le nombre de morphèmes et le nombre de tons de morphèmes}}}}\kern2pt]} \hypertarget{\{\string_Ms`\{\string_M-q\string_hw7\string_M\string_T1}{}
\markboth{\textcolor{darkblue}{\textbf{\ipa{æ˧ʂæ˧-qʰwɤ˧˥}}}}{}
\textcolor{teal}{\zh{名词}} \hspace{4pt} \zh{声调类:} MH\#.
\zh{故事。} \textcolor{Sepia}{\selectlanguage{english}Story, folk tale.} \textcolor{PineGreen}{\selectlanguage{french}Histoire, conte, récit traditionnel.}  ¶ \textcolor{darkblue}{\textbf{\ipa{æ˧ʂæ˧qʰwɤ˧ ʐwɤ˧˥}}} \zh{讲故事} \textcolor{Sepia}{\selectlanguage{english}to tell a story} \textcolor{PineGreen}{\selectlanguage{french}raconter une histoire}  
 ¶ \textcolor{darkblue}{\textbf{\ipa{ə˧ʝi˧-ʂɯ˥ʝi˩, | æ˧ʂæ˧qʰwɤ˧ ʐwɤ˧-kv̩˥!}}} \zh{在过去,大家经常讲故事!} \textcolor{Sepia}{\selectlanguage{english}In the old times, (people) used to tell stories!} \textcolor{PineGreen}{\selectlanguage{french}dans le temps, (on) racontait des histoires!}  
 \zh{量词}: \textcolor{darkblue}{\textbf{\ipa{kʰwɤ˥}}} 
\lhead{\firstmark}
\rhead{\botmark}

\subsection{\hspace{-0.5cm} {\Large \textcolor{darkblue}{\textbf{\ipa{æ˧tse˥-pʰæ˩}}}}\hspace{0.5cm}[\kern2pt{\textcolor{darkblue}{\textbf{\ipa{æ˧tse˥pʰæ˩}}}}\kern2pt]} \hypertarget{\{\string_Mtse\string_T-p\string_h\{\string_B1}{}
\markboth{\textcolor{darkblue}{\textbf{\ipa{æ˧tse˥-pʰæ˩}}}}{}
\textcolor{teal}{\zh{名词}} \hspace{4pt} \zh{声调类:} H\#-L.
\zh{膝盖骨。} \textcolor{Sepia}{\selectlanguage{english}Kneebone.} \textcolor{PineGreen}{\selectlanguage{french}Os du genou.}  \zh{量词}: \textcolor{darkblue}{\textbf{\ipa{ɭɯ˧}}} 
\lhead{\firstmark}
\rhead{\botmark}

\subsection{\hspace{-0.5cm} {\Large \textcolor{darkblue}{\textbf{\ipa{æ˧tsɯ˥-pɤ˩lv̩˩}}}}\hspace{0.5cm}[\kern2pt{\textcolor{darkblue}{\textbf{\ipa{æ˧tsɯ˥pɤ˩lv̩˩}}}}\kern2pt]} \hypertarget{\{\string_MtsM\string_T-p7\string_Blv\string_=\string_B1}{}
\markboth{\textcolor{darkblue}{\textbf{\ipa{æ˧tsɯ˥-pɤ˩lv̩˩}}}}{}
\textcolor{teal}{\zh{名词}} \hspace{4pt} \zh{声调类:} H\#-L.
\zh{项背 、项、脖颈儿。} \textcolor{Sepia}{\selectlanguage{english}Nape.} \textcolor{PineGreen}{\selectlanguage{french}Nuque.}  \zh{量词}: \textcolor{darkblue}{\textbf{\ipa{ɭɯ˧}}} 
\lhead{\firstmark}
\rhead{\botmark}

\subsection{\hspace{-0.5cm} {\Large \textcolor{darkblue}{\textbf{\ipa{æ˧ʈwɤ˩}}}}\hspace{0.5cm}[\kern2pt{\textcolor{darkblue}{\textbf{\ipa{æ˧ʈwɤ˩}}}}\kern2pt]} \hypertarget{\{\string_Mt`w7\string_B1}{}
\markboth{\textcolor{darkblue}{\textbf{\ipa{æ˧ʈwɤ˩}}}}{}
\textcolor{teal}{\zh{名词}} \hspace{4pt} \zh{声调类:} L\#.
\zh{清晨、一大早(鸡叫的时候)。} \textcolor{Sepia}{\selectlanguage{english}The early morning; early in the morning.} \textcolor{PineGreen}{\selectlanguage{french}Le petit matin.} 
\lhead{\firstmark}
\rhead{\botmark}

\subsection{\hspace{-0.5cm} {\Large \textcolor{darkblue}{\textbf{\ipa{æ˩gv̩˩}}}}\hspace{0.5cm}[\kern2pt{\textcolor{darkblue}{\textbf{\ipa{æ˩gv̩˩˥}}}}\kern2pt]} \hypertarget{\{\string_Bgv\string_=\string_B1}{}
\markboth{\textcolor{darkblue}{\textbf{\ipa{æ˩gv̩˩}}}}{}
\textcolor{teal}{\zh{名词}} \hspace{4pt} \zh{声调类:} L.
\zh{犁头。} \textcolor{Sepia}{\selectlanguage{english}Ard. There are no distinct words for 'ard' and 'plough'; only ards were in use at the time of fieldwork.} \textcolor{PineGreen}{\selectlanguage{french}Araire. Il n'existe pas deux termes distincts, l'un pour l'araire et l'autre pour la charrue; à la date de l'enquête, seule l'araire était en usage.}  ¶ \textcolor{darkblue}{\textbf{\ipa{æ˩gv̩˩ tʰv̩˩-nɑ˥}}} \zh{这把犁头} \textcolor{Sepia}{\selectlanguage{english}\mytextsc{n}+\mytextsc{dem}+\mytextsc{clf}} \textcolor{PineGreen}{\selectlanguage{french}\mytextsc{n}+\mytextsc{dem}+\mytextsc{clf}}  
 ¶ \textcolor{darkblue}{\textbf{\ipa{æ˩mo˥}}} \zh{陈旧的犁头(不能再用了)} \textcolor{Sepia}{\selectlanguage{english}used ard, plough which cannot be used anymore} \textcolor{PineGreen}{\selectlanguage{french}araire usagée, vieille araire (hors d'usage du fait de l'usure)}  
 ¶ \textcolor{darkblue}{\textbf{\ipa{æ˩mo˥ tʰv̩˩-nɑ˩}}} \zh{这个旧犁杆} \textcolor{Sepia}{\selectlanguage{english}\mytextsc{n}+\mytextsc{dem}+\mytextsc{clf}} \textcolor{PineGreen}{\selectlanguage{french}\mytextsc{n}+\mytextsc{dem}+\mytextsc{clf}}  
 ¶ \textcolor{darkblue}{\textbf{\ipa{æ˩-ʂɯ˩˥}}} \zh{新的犁头} \textcolor{Sepia}{\selectlanguage{english}new ard, brand new ard} \textcolor{PineGreen}{\selectlanguage{french}araire neuve}  
 \zh{量词}: \textcolor{darkblue}{\textbf{\ipa{nɑ˧}}} 
\lhead{\firstmark}
\rhead{\botmark}

\subsection{\hspace{-0.5cm} {\Large \textcolor{darkblue}{\textbf{\ipa{æ˩gv̩˩-mæ˩qo˥}}}}\hspace{0.5cm}[\kern2pt{\textcolor{darkblue}{\textbf{\ipa{xxxx non-correspondance entre le nombre de morphèmes et le nombre de tons de morphèmes}}}}\kern2pt]} \hypertarget{\{\string_Bgv\string_=\string_B-m\{\string_Bqo\string_T1}{}
\markboth{\textcolor{darkblue}{\textbf{\ipa{æ˩gv̩˩-mæ˩qo˥}}}}{}
\textcolor{teal}{\zh{名词}} \hspace{4pt} \zh{声调类:} L+H\#.
\zh{犁把。} \textcolor{Sepia}{\selectlanguage{english}Handle (stilt) of the ard, used to control the ard's direction and the furrow's depth.} \textcolor{PineGreen}{\selectlanguage{french}Mancheron de l'araire, manche de l'araire.}  ¶ \textcolor{darkblue}{\textbf{\ipa{æ̃˩gv̩˩-mæ˩ ʑi˩-hĩ˥}}} \zh{抓着犁把的人} \textcolor{Sepia}{\selectlanguage{english}the person holding the handle of the ard} \textcolor{PineGreen}{\selectlanguage{french}la personne qui tient le mancheron de la charrue}  
 ¶ \textcolor{darkblue}{\textbf{\ipa{æ̃˩gv̩˩-mæ˩qo˥ tʰv̩˩-nɑ˩}}} \zh{这个犁把} \textcolor{Sepia}{\selectlanguage{english}\mytextsc{n}+\mytextsc{dem}+\mytextsc{clf}} \textcolor{PineGreen}{\selectlanguage{french}\mytextsc{n}+\mytextsc{dem}+\mytextsc{clf}}  
 \zh{量词}: \textcolor{darkblue}{\textbf{\ipa{nɑ˧}}} 
\lhead{\firstmark}
\rhead{\botmark}

\subsection{\hspace{-0.5cm} {\Large \textcolor{darkblue}{\textbf{\ipa{æ˩mi˧-mv̩˧ʈv̩˥}}}}\hspace{0.5cm}[\kern2pt{\textcolor{darkblue}{\textbf{\ipa{xxxx non-correspondance entre le nombre de morphèmes et le nombre de tons de morphèmes}}}}\kern2pt]} \hypertarget{\{\string_Bmi\string_M-mv\string_=\string_Mt`v\string_=\string_T1}{}
\markboth{\textcolor{darkblue}{\textbf{\ipa{æ˩mi˧-mv̩˧ʈv̩˥}}}}{}
\textcolor{teal}{\zh{名词}} \hspace{4pt} \zh{声调类:} LM+H\#.
\zh{踝骨。} \textcolor{Sepia}{\selectlanguage{english}Anklebone, bone of the top of the foot.} \textcolor{PineGreen}{\selectlanguage{french}Astragale (os du dessus du pied; os “en boule” sur les deux côtés du pied, au-dessous de la cheville).}  \zh{量词}: \textcolor{darkblue}{\textbf{\ipa{ɭɯ˧}}} 
\lhead{\firstmark}
\rhead{\botmark}

\subsection{\hspace{-0.5cm} {\Large \textcolor{darkblue}{\textbf{\ipa{æ˩mi˧-ʁwɤ\#˥}}}}\hspace{0.5cm}[\kern2pt{\textcolor{darkblue}{\textbf{\ipa{xxxx non-correspondance entre le nombre de morphèmes et le nombre de tons de morphèmes}}}}\kern2pt]} \hypertarget{\{\string_Bmi\string_M-Rw7\#\string_T1}{}
\markboth{\textcolor{darkblue}{\textbf{\ipa{æ˩mi˧-ʁwɤ\#˥}}}}{}
\textcolor{teal}{\zh{名词}} \hspace{4pt} \zh{声调类:} LM+\#H.
\zh{阿咪瓦、阿米瓦(永宁的一个村落)。} \textcolor{Sepia}{\selectlanguage{english}Amiwa. This is the first village along the road from \textcolor{darkblue}{\textbf{\ipa{/qʰæ˧tɕʰi˧/}}} to \textcolor{darkblue}{\textbf{\ipa{/ʈʂo˧ʂɯ\#˥/}}}. In traditional Na geography, which takes Lugu lake as a point of origin, Amiwa is the third village of the plain.} \textcolor{PineGreen}{\selectlanguage{french}Amiwa, premier village que l'on rencontre sur la route entre \textcolor{darkblue}{\textbf{\ipa{/qʰæ˧tɕʰi˧/}}} et \textcolor{darkblue}{\textbf{\ipa{/ʈʂo˧ʂɯ\#˥/}}}. Dans la géographie traditionnelle na, qui prend comme point d'origine le lac Lugu, Amiwa est le troisième village de la plaine de Yongning.}  ¶ \textcolor{darkblue}{\textbf{\ipa{dʑɤ˩bv̩˧kɤ˧-sɑ˥ʁwɤ˩, | hi˩ʁwɤ˩-lo˥, | æ˩mi˧-ʁwɤ\#˥, | lɑ˧lo˧-ʁwɤ˥, | lɑ˧ŋwɤ˧, | bɤ˧tsʰo˧gv̩˥, | ə˧lɑ˧-ʁwɤ\#˥, | gæ˧ɻæ˩, | qʰæ˧tɕʰi˧, | tʰo˧ʈɯ\#˥}}} \zh{摩梭传统地理概念中,属于永宁的十个村落} \textcolor{Sepia}{\selectlanguage{english}the ten villages traditionally considered as part of Yongning} \textcolor{PineGreen}{\selectlanguage{french}les dix villages comptant traditionnellement comme faisant partie de Yongning}  

\lhead{\firstmark}
\rhead{\botmark}

\subsection{\hspace{-0.5cm} {\Large \textcolor{darkblue}{\textbf{\ipa{æ˩mo˧}}}}\hspace{0.5cm}[\kern2pt{\textcolor{darkblue}{\textbf{\ipa{æ˩mo˥}}}}\kern2pt]} \hypertarget{\{\string_Bmo\string_M1}{}
\markboth{\textcolor{darkblue}{\textbf{\ipa{æ˩mo˧}}}}{}
\textcolor{teal}{\zh{名词}} \hspace{4pt} \zh{声调类:} LM.
\zh{犁杆。} \textcolor{Sepia}{\selectlanguage{english}Beam of the ard, pole of the ard: a long piece of wood linking the yoke to the sole.} \textcolor{PineGreen}{\selectlanguage{french}Timon (âge, haie) de l'araire.}  ¶ \textcolor{darkblue}{\textbf{\ipa{æ˩gv̩˩-mo˥}}} \zh{同上} \textcolor{Sepia}{\selectlanguage{english}same meaning} \textcolor{PineGreen}{\selectlanguage{french}même sens}  
 \zh{量词}: \textcolor{darkblue}{\textbf{\ipa{nɑ˧}}} \zh{~【参考】~} \hyperlink{}{\textcolor{darkblue}{\textbf{\ipa{æ˩gv̩˩}}}} 
\lhead{\firstmark}
\rhead{\botmark}

\subsection{\hspace{-0.5cm} {\Large \textcolor{darkblue}{\textbf{\ipa{æ˩pʰæ˧˥}}}}\hspace{0.5cm}[\kern2pt{\textcolor{darkblue}{\textbf{\ipa{æ˩pʰæ˧˥}}}}\kern2pt]} \hypertarget{\{\string_Bp\string_h\{\string_M\string_T1}{}
\markboth{\textcolor{darkblue}{\textbf{\ipa{æ˩pʰæ˧˥}}}}{}
\textcolor{teal}{\zh{名词}} \hspace{4pt} \zh{声调类:} LM+MH\#.
\zh{悬崖、崖山、崖壁。} \textcolor{Sepia}{\selectlanguage{english}Cliff, overhanging cliff. This term designates the top of the cliff: the relatively flat ground close to the precipice. To refer to the steep (vertical) side of the cliff, one adds \textcolor{darkblue}{\textbf{\ipa{/lɑ˧bi˧/}}} 'steep slope'.} \textcolor{PineGreen}{\selectlanguage{french}Falaise. Le terme désigne spécifiquement le dos d'une falaise: l'espace relativement plat en bord de précipice. Pour faire référence à la paroi (face verticale de la falaise), on ajoute \textcolor{darkblue}{\textbf{\ipa{/lɑ˧bi˧/}}} 'escarpement'.}  ¶ \textcolor{darkblue}{\textbf{\ipa{æ˩pʰæ˧-lɑ˧bi˥}}} \zh{同上} \textcolor{Sepia}{\selectlanguage{english}same meaning} \textcolor{PineGreen}{\selectlanguage{french}même sens}  
 \zh{量词}: \textcolor{darkblue}{\textbf{\ipa{pʰæ˧˥}}} 
\lhead{\firstmark}
\rhead{\botmark}

\subsection{\hspace{-0.5cm} {\Large \textcolor{darkblue}{\textbf{\ipa{æ˩qʰv̩˥}}}}\hspace{0.5cm}[\kern2pt{\textcolor{darkblue}{\textbf{\ipa{æ˩qʰv̩˥}}}}\kern2pt]} \hypertarget{\{\string_Bq\string_hv\string_=\string_T1}{}
\markboth{\textcolor{darkblue}{\textbf{\ipa{æ˩qʰv̩˥}}}}{}
\textcolor{teal}{\zh{名词}} \hspace{4pt} \zh{声调类:} LH.
\zh{小山洞(难进去,或者钻不进去的山洞)。} \textcolor{Sepia}{\selectlanguage{english}Cave, cavern, crevice (difficult to enter, or too small for a person to enter).} \textcolor{PineGreen}{\selectlanguage{french}Crevasse, petite grotte (où il est difficile de pénétrer).}  \zh{量词}: \textcolor{darkblue}{\textbf{\ipa{ɭɯ˧}}} 
\lhead{\firstmark}
\rhead{\botmark}

\subsection{\hspace{-0.5cm} {\Large \textcolor{darkblue}{\textbf{\ipa{æ˩ʈv̩˥}}}}\hspace{0.5cm}[\kern2pt{\textcolor{darkblue}{\textbf{\ipa{æ˩ʈv̩˥}}}}\kern2pt]} \hypertarget{\{\string_Bt`v\string_=\string_T1}{}
\markboth{\textcolor{darkblue}{\textbf{\ipa{æ˩ʈv̩˥}}}}{}
\textcolor{teal}{\zh{名词}} \hspace{4pt} \zh{声调类:} LH.
\zh{大岩石。} \textcolor{Sepia}{\selectlanguage{english}Large rock.} \textcolor{PineGreen}{\selectlanguage{french}Gros rocher, roc.}  \zh{量词}: \textcolor{darkblue}{\textbf{\ipa{ʈv̩˩}}} 
\lhead{\firstmark}
\rhead{\botmark}

\subsection{\hspace{-0.5cm} {\Large \textcolor{darkblue}{\textbf{\ipa{æ̃˥}}}}\hspace{0.5cm}[\kern2pt{\textcolor{darkblue}{\textbf{\ipa{æ̃˥}}}}\kern2pt]} \hypertarget{\{\string_~\string_T1}{}
\markboth{\textcolor{darkblue}{\textbf{\ipa{æ̃˥}}}}{}
\textcolor{teal}{\zh{名词}} \hspace{4pt} \zh{声调类:} \#H.
\zh{铜,包括黄铜、红铜、青铜。} \textcolor{Sepia}{\selectlanguage{english}Brass, copper, bronze.} \textcolor{PineGreen}{\selectlanguage{french}Cuivre; bronze.}  ¶ \textcolor{darkblue}{\textbf{\ipa{æ̃˧tso˧-æ̃˧mo˩}}} \zh{铜做的工具、物品} \textcolor{Sepia}{\selectlanguage{english}instruments and objects made of brass} \textcolor{PineGreen}{\selectlanguage{french}instruments en cuivre, objets en cuivre}  
 ¶ \textcolor{darkblue}{\textbf{\ipa{æ̃˧ lɑ˩-zo˩-ɳɯ˩, | ʂe˧ mɤ˧-lɑ˧˥!}}} \zh{“打铜的人,不打铁!”这两种工作需要不同的能力:打铁需要体力,打铜需要细致。这个谚语意指:每个人有他的专业,不能随便跨越到其它领域去。} \textcolor{Sepia}{\selectlanguage{english}“The man who works copper does not work iron!” These two specialties require different skills: physical strength for working iron; and delicate gestures for working copper. This saying is used to point out that each person has her/his own area of expertise.} \textcolor{PineGreen}{\selectlanguage{french}“Celui qui travaille le cuivre, il ne doit pas travailler le fer/on ne doit pas lui confier de tâches de forgeron (=travail du fer)!” Ces deux spécialités demandent des qualités différentes: de la force physique pour le travail du fer, et du soin pour le travail du cuivre. Le dicton s'emploie pour souligner que chacun a son domaine de compétence.}  

\lhead{\firstmark}
\rhead{\botmark}

\subsection{\hspace{-0.5cm} {\Large \textcolor{darkblue}{\textbf{\ipa{æ̃˧qæ˩}}} \textsubscript{1}}\hspace{0.5cm}[\kern2pt{\textcolor{darkblue}{\textbf{\ipa{æ̃˧qæ˩}}}}\kern2pt]} \hypertarget{\{\string_~\string_Mq\{\string_B1}{}
\markboth{\textcolor{darkblue}{\textbf{\ipa{æ̃˧qæ˩}}} \textsubscript{1}}{}
\textcolor{teal}{\zh{名词}} \hspace{4pt} \zh{声调类:} L\#.
\zh{鹦鹉。} \textcolor{Sepia}{\selectlanguage{english}Parrot.} \textcolor{PineGreen}{\selectlanguage{french}Perroquet.}  \zh{量词}: \textcolor{darkblue}{\textbf{\ipa{mi˩}}} \zh{~【参考】~} \hyperlink{}{\textcolor{darkblue}{\textbf{\ipa{æ̃˧qæ˩}}} \textsubscript{2}} 
\lhead{\firstmark}
\rhead{\botmark}

\subsection{\hspace{-0.5cm} {\Large \textcolor{darkblue}{\textbf{\ipa{æ̃˧qæ˩}}} \textsubscript{2}}\hspace{0.5cm}[\kern2pt{\textcolor{darkblue}{\textbf{\ipa{æ̃˧qæ˩}}}}\kern2pt]} \hypertarget{\{\string_~\string_Mq\{\string_B2}{}
\markboth{\textcolor{darkblue}{\textbf{\ipa{æ̃˧qæ˩}}} \textsubscript{2}}{}
\textcolor{teal}{\zh{形容词}} \hspace{4pt} \zh{声调类:} L\#.
\zh{像鹦鹉的颜色:青色、蓝色、绿色。} \textcolor{Sepia}{\selectlanguage{english}Blue-green; literally 'parrot[-coloured]'.} \textcolor{PineGreen}{\selectlanguage{french}De couleur bleue/verte; couleur un peu plus légère que le vert de la plaine; équivalent du chinois \zh{青}. Littéralement: '[couleur] perroquet'.}  ¶ \textcolor{darkblue}{\textbf{\ipa{æ̃˧qæ˩-ni˩gv̩˩}}} \zh{像鹦鹉的颜色:青、蓝色、绿色} \textcolor{Sepia}{\selectlanguage{english}vivid-coloured, blue-green; literally 'like a parrot', i.e. 'parrot-coloured'} \textcolor{PineGreen}{\selectlanguage{french}couleur perroquet; littéralement 'comme un perroquet'}  
 ¶ \textcolor{darkblue}{\textbf{\ipa{[F5] æ̃˧qæ˩-bɑ˩lɑ˩}}} \zh{青、蓝色、绿色衣服} \textcolor{Sepia}{\selectlanguage{english}vivid-coloured, blue-green jacket: literally 'parrot(-coloured) jacket'} \textcolor{PineGreen}{\selectlanguage{french}vêtement bleu; littéralement vêtement 'couleur perroquet'}  
 ¶ \textcolor{darkblue}{\textbf{\ipa{[F5] æ̃˧qæ˩ni˩\textasciitilde{}æ̃˧qæ˩ni˩gv̩˩}}} \zh{\mytextsc{重叠。同上:青色}} \textcolor{Sepia}{\selectlanguage{english}\mytextsc{red;} same meaning: blue-green} \textcolor{PineGreen}{\selectlanguage{french}\mytextsc{red;} même sens: bleu-vert}  
\zh{~【参考】~} \hyperlink{}{\textcolor{darkblue}{\textbf{\ipa{æ̃˧qæ˩}}} \textsubscript{1}} 
\lhead{\firstmark}
\rhead{\botmark}

\subsection{\hspace{-0.5cm} {\Large \textcolor{darkblue}{\textbf{\ipa{æ̃˧ʂwæ˥}}}}\hspace{0.5cm}[\kern2pt{\textcolor{darkblue}{\textbf{\ipa{æ̃˧ʂwæ˥}}}}\kern2pt]} \hypertarget{\{\string_~\string_Ms`w\{\string_T1}{}
\markboth{\textcolor{darkblue}{\textbf{\ipa{æ̃˧ʂwæ˥}}}}{}
\textcolor{teal}{\zh{名词}} \hspace{4pt} \zh{声调类:} H\#.
\zh{公鸡。} \textcolor{Sepia}{\selectlanguage{english}Rooster.} \textcolor{PineGreen}{\selectlanguage{french}Coq.}  ¶ \textcolor{darkblue}{\textbf{\ipa{æ̃˧ʂwæ˥-æ̃˩mi˩}}} \zh{公鸡与母鸡} \textcolor{Sepia}{\selectlanguage{english}cock and hen} \textcolor{PineGreen}{\selectlanguage{french}coq et poule}  
 \zh{量词}: \textcolor{darkblue}{\textbf{\ipa{mi˩}}} 
\lhead{\firstmark}
\rhead{\botmark}

\subsection{\hspace{-0.5cm} {\Large \textcolor{darkblue}{\textbf{\ipa{æ̃˧tsɯ˥}}}}\hspace{0.5cm}[\kern2pt{\textcolor{darkblue}{\textbf{\ipa{æ̃˧tsɯ˥}}}}\kern2pt]} \hypertarget{\{\string_~\string_MtsM\string_T1}{}
\markboth{\textcolor{darkblue}{\textbf{\ipa{æ̃˧tsɯ˥}}}}{}
\textcolor{teal}{\zh{名词}} \hspace{4pt} \zh{声调类:} H\#.
\zh{雏鸡、稚鸡。} \textcolor{Sepia}{\selectlanguage{english}Chick.} \textcolor{PineGreen}{\selectlanguage{french}Poussin.}  \zh{量词}: \textcolor{darkblue}{\textbf{\ipa{ɭɯ˧}}} 
\lhead{\firstmark}
\rhead{\botmark}

\subsection{\hspace{-0.5cm} {\Large \textcolor{darkblue}{\textbf{\ipa{æ̃˧tsɯ˥-kʰɯ˩ʈʂɤ˩-mo˩}}}}\hspace{0.5cm}[\kern2pt{\textcolor{darkblue}{\textbf{\ipa{xxxx non-correspondance entre le nombre de morphèmes et le nombre de tons de morphèmes}}}}\kern2pt]} \hypertarget{\{\string_~\string_MtsM\string_T-k\string_hM\string_Bt`s`7\string_B-mo\string_B1}{}
\markboth{\textcolor{darkblue}{\textbf{\ipa{æ̃˧tsɯ˥-kʰɯ˩ʈʂɤ˩-mo˩}}}}{}
\textcolor{teal}{\zh{名词}} \hspace{4pt} \zh{声调类:} H\#-.
\zh{扫把菌,扫帚菌(一种菌子)。} \textcolor{Sepia}{\selectlanguage{english}“chicken-claw mushroom”: an edible mushroom.} \textcolor{PineGreen}{\selectlanguage{french}“champignon griffes-de-poulet”: champignon comestible.} 
\lhead{\firstmark}
\rhead{\botmark}

\subsection{\hspace{-0.5cm} {\Large \textcolor{darkblue}{\textbf{\ipa{æ̃˧ʈwɤ˩-mv̩˩kʰv̩˩}}}}\hspace{0.5cm}[\kern2pt{\textcolor{darkblue}{\textbf{\ipa{æ̃˧ʈwɤ˩mv̩˩kʰv̩˩}}}}\kern2pt]} \hypertarget{\{\string_~\string_Mt`w7\string_B-mv\string_=\string_Bk\string_hv\string_=\string_B1}{}
\markboth{\textcolor{darkblue}{\textbf{\ipa{æ̃˧ʈwɤ˩-mv̩˩kʰv̩˩}}}}{}
\textcolor{teal}{\zh{助词}} \hspace{4pt} \zh{声调类:} L\#-L.
\zh{一直不停地,从早到晚。直译:‘(从)早上(到)晚上’。} \textcolor{Sepia}{\selectlanguage{english}Constantly, all the time; literally: '[from] morning [till] evening'.} \textcolor{PineGreen}{\selectlanguage{french}Du matin au soir, constamment.} 
\lhead{\firstmark}
\rhead{\botmark}

\subsection{\hspace{-0.5cm} {\Large \textcolor{darkblue}{\textbf{\ipa{æ̃˧-v̩\#˥}}}}\hspace{0.5cm}[\kern2pt{\textcolor{darkblue}{\textbf{\ipa{xxxx non-correspondance entre le nombre de morphèmes et le nombre de tons de morphèmes}}}}\kern2pt]} \hypertarget{\{\string_~\string_M-v\string_=\#\string_T1}{}
\markboth{\textcolor{darkblue}{\textbf{\ipa{æ̃˧-v̩\#˥}}}}{}
\textcolor{teal}{\zh{名词}} \hspace{4pt} \zh{声调类:} \#H.
\zh{铜锅。} \textcolor{Sepia}{\selectlanguage{english}Copper pot.} \textcolor{PineGreen}{\selectlanguage{french}Casserole en cuivre.}  \zh{量词}: \textcolor{darkblue}{\textbf{\ipa{ɭɯ˧}}} 
\lhead{\firstmark}
\rhead{\botmark}

\subsection{\hspace{-0.5cm} {\Large \textcolor{darkblue}{\textbf{\ipa{æ̃˩\textsubscript{a}}}}}\hspace{0.5cm}[\kern2pt{\textcolor{darkblue}{\textbf{\ipa{æ̃˩˥}}}}\kern2pt]} \hypertarget{\{\string_~\string_Ba1}{}
\markboth{\textcolor{darkblue}{\textbf{\ipa{æ̃˩\textsubscript{a}}}}}{}
\textcolor{teal}{\zh{量词}} \hspace{4pt} \zh{声调类:} L\textsubscript{a}.
\zh{量词:火(一团)。} \textcolor{Sepia}{\selectlanguage{english}Classifier for fires.} \textcolor{PineGreen}{\selectlanguage{french}Classificateur des feux.}  ¶ \textcolor{darkblue}{\textbf{\ipa{mv̩˧ | ʈʂʰɯ˧-æ̃˥}}} \zh{这团火} \textcolor{Sepia}{\selectlanguage{english}this fire (tone: H\# / H\$)} \textcolor{PineGreen}{\selectlanguage{french}ce feu (ton: H\# / H\$)}  

\lhead{\firstmark}
\rhead{\botmark}

\subsection{\hspace{-0.5cm} {\Large \textcolor{darkblue}{\textbf{\ipa{æ̃˩\textsubscript{a}}}} \textsubscript{1}}\hspace{0.5cm}[\kern2pt{\textcolor{darkblue}{\textbf{\ipa{æ̃˩˥}}}}\kern2pt]} \hypertarget{\{\string_~\string_Ba1}{}
\markboth{\textcolor{darkblue}{\textbf{\ipa{æ̃˩\textsubscript{a}}}} \textsubscript{1}}{}
\textcolor{teal}{\zh{动词}} \hspace{4pt} \zh{声调类:} L\textsubscript{a}.
\zh{反射、辉映。} \textcolor{Sepia}{\selectlanguage{english}To reflect (a mirror reflects light).} \textcolor{PineGreen}{\selectlanguage{french}Réfléchir, renvoyer (un miroir renvoie la lumière; une cloison/toiture étanche renvoie la pluie =est étanche à la pluie).} 
\lhead{\firstmark}
\rhead{\botmark}

\subsection{\hspace{-0.5cm} {\Large \textcolor{darkblue}{\textbf{\ipa{æ̃˩\textsubscript{a}}}} \textsubscript{2}}\hspace{0.5cm}[\kern2pt{\textcolor{darkblue}{\textbf{\ipa{æ̃˩˥}}}}\kern2pt]} \hypertarget{\{\string_~\string_Ba2}{}
\markboth{\textcolor{darkblue}{\textbf{\ipa{æ̃˩\textsubscript{a}}}} \textsubscript{2}}{}
\textcolor{teal}{\zh{动词}} \hspace{4pt} \zh{声调类:} L\textsubscript{a}.
\zh{堵塞、塞。} \textcolor{Sepia}{\selectlanguage{english}To get stuck.} \textcolor{PineGreen}{\selectlanguage{french}S'enliser; se coincer, se bloquer.}  ¶ \textcolor{darkblue}{\textbf{\ipa{ʝi˩mi˩˥ | ɖʐæ˩qʰæ˧-qo˩ æ̃˩!}}} \zh{牛陷在泥巴里。} \textcolor{Sepia}{\selectlanguage{english}The cow is stuck in the mud.} \textcolor{PineGreen}{\selectlanguage{french}La vache est enlisée dans la boue.}  

\lhead{\firstmark}
\rhead{\botmark}

\subsection{\hspace{-0.5cm} {\Large \textcolor{darkblue}{\textbf{\ipa{æ̃˩bi˩}}}}\hspace{0.5cm}[\kern2pt{\textcolor{darkblue}{\textbf{\ipa{æ̃˩bi˩˥}}}}\kern2pt]} \hypertarget{\{\string_~\string_Bbi\string_B1}{}
\markboth{\textcolor{darkblue}{\textbf{\ipa{æ̃˩bi˩}}}}{}
\textcolor{teal}{\zh{名词}} \hspace{4pt} \zh{声调类:} L.
\zh{从阿拉瓦村到前所路上经过的一个村落。} \textcolor{Sepia}{\selectlanguage{english}A village just over the river on the Sichuan side of road to Qiansuo.} \textcolor{PineGreen}{\selectlanguage{french}Abi: village sur le chemin de Qiansuo.}  ¶ \textcolor{darkblue}{\textbf{\ipa{æ̃˩bi˩-ʁwɤ˥}}} \zh{同上} \textcolor{Sepia}{\selectlanguage{english}same meaning: the village of \textcolor{darkblue}{\textbf{\ipa{/æ̃˩bi˩/}}}} \textcolor{PineGreen}{\selectlanguage{french}même sens: le village de \textcolor{darkblue}{\textbf{\ipa{/æ̃˩bi˩/}}}}  
 ¶ \textcolor{darkblue}{\textbf{\ipa{æ̃˩bi˩-hĩ˥ ɲi˩!}}} \zh{是\textcolor{darkblue}{\textbf{\ipa{/æ̃˩bi˩/}}}村的人!} \textcolor{Sepia}{\selectlanguage{english}[(S)he] is from the village of \textcolor{darkblue}{\textbf{\ipa{/æ̃˩bi˩/!}}}} \textcolor{PineGreen}{\selectlanguage{french}C'est quelqu'un du village de \textcolor{darkblue}{\textbf{\ipa{/æ̃˩bi˩/!}}}}  

\lhead{\firstmark}
\rhead{\botmark}

\subsection{\hspace{-0.5cm} {\Large \textcolor{darkblue}{\textbf{\ipa{æ̃˩bv̩˥}}}}\hspace{0.5cm}[\kern2pt{\textcolor{darkblue}{\textbf{\ipa{æ̃˩bv̩˥}}}}\kern2pt]} \hypertarget{\{\string_~\string_Bbv\string_=\string_T1}{}
\markboth{\textcolor{darkblue}{\textbf{\ipa{æ̃˩bv̩˥}}}}{}
\textcolor{teal}{\zh{名词}} \hspace{4pt} \zh{声调类:} LH.
\zh{鸡圈。} \textcolor{Sepia}{\selectlanguage{english}Poultry yard.} \textcolor{PineGreen}{\selectlanguage{french}Poulailler.}  \zh{量词}: \textcolor{darkblue}{\textbf{\ipa{ɭɯ˧}}} 
\lhead{\firstmark}
\rhead{\botmark}

\subsection{\hspace{-0.5cm} {\Large \textcolor{darkblue}{\textbf{\ipa{æ̃˩-kʰv̩˧˥}}}}\hspace{0.5cm}[\kern2pt{\textcolor{darkblue}{\textbf{\ipa{xxxx non-correspondance entre le nombre de morphèmes et le nombre de tons de morphèmes}}}}\kern2pt]} \hypertarget{\{\string_~\string_B-k\string_hv\string_=\string_M\string_T1}{}
\markboth{\textcolor{darkblue}{\textbf{\ipa{æ̃˩-kʰv̩˧˥}}}}{}
\textcolor{teal}{\zh{名词}} \hspace{4pt} \zh{声调类:} LM+MH\#.
\zh{鸡年。} \textcolor{Sepia}{\selectlanguage{english}Year of the cock.} \textcolor{PineGreen}{\selectlanguage{french}Année du coq.} 
\lhead{\firstmark}
\rhead{\botmark}

\subsection{\hspace{-0.5cm} {\Large \textcolor{darkblue}{\textbf{\ipa{æ̃˩li˧pʰæ˥}}}}\hspace{0.5cm}[\kern2pt{\textcolor{darkblue}{\textbf{\ipa{æ̃˩li˧pʰæ˥}}}}\kern2pt]} \hypertarget{\{\string_~\string_Bli\string_Mp\string_h\{\string_T1}{}
\markboth{\textcolor{darkblue}{\textbf{\ipa{æ̃˩li˧pʰæ˥}}}}{}
\textcolor{teal}{\zh{名词}} \hspace{4pt} \zh{声调类:} LM+H\#.
\zh{镜子。} \textcolor{Sepia}{\selectlanguage{english}Mirror.} \textcolor{PineGreen}{\selectlanguage{french}Miroir.}  \zh{量词}: \textcolor{darkblue}{\textbf{\ipa{pʰæ˧˥}}} 
\lhead{\firstmark}
\rhead{\botmark}

\subsection{\hspace{-0.5cm} {\Large \textcolor{darkblue}{\textbf{\ipa{æ̃˩ɬi\#˥}}}}\hspace{0.5cm}[\kern2pt{\textcolor{darkblue}{\textbf{\ipa{æ̃˩ɬi˥}}}}\kern2pt]} \hypertarget{\{\string_~\string_BKi\#\string_T1}{}
\markboth{\textcolor{darkblue}{\textbf{\ipa{æ̃˩ɬi\#˥}}}}{}
\textcolor{teal}{\zh{名词}} \hspace{4pt} \zh{声调类:} LM+\#H.
\zh{灵魂、魂魄。} \textcolor{Sepia}{\selectlanguage{english}Soul.} \textcolor{PineGreen}{\selectlanguage{french}Âme.}  \zh{【借词】}\zh{藏语} bla (older form: brla)
 \zh{量词}: \textcolor{darkblue}{\textbf{\ipa{v̩˧}}} 
\lhead{\firstmark}
\rhead{\botmark}

\subsection{\hspace{-0.5cm} {\Large \textcolor{darkblue}{\textbf{\ipa{æ̃˩mi˧}}}}\hspace{0.5cm}[\kern2pt{\textcolor{darkblue}{\textbf{\ipa{æ̃˩mi˥}}}}\kern2pt]} \hypertarget{\{\string_~\string_Bmi\string_M1}{}
\markboth{\textcolor{darkblue}{\textbf{\ipa{æ̃˩mi˧}}}}{}
\textcolor{teal}{\zh{名词}} \hspace{4pt} \zh{声调类:} LM.
\zh{母鸡。} \textcolor{Sepia}{\selectlanguage{english}Hen.} \textcolor{PineGreen}{\selectlanguage{french}Poule.}  ¶ \textcolor{darkblue}{\textbf{\ipa{æ̃˩mi˧-æ̃˧ʂwæ˥\#}}} \zh{母鸡与公鸡} \textcolor{Sepia}{\selectlanguage{english}hen and rooster} \textcolor{PineGreen}{\selectlanguage{french}poule et coq}  
 ¶ \textcolor{darkblue}{\textbf{\ipa{æ̃˩mi˧-æ̃˧tsɯ˥\#}}} \zh{母鸡与稚鸡} \textcolor{Sepia}{\selectlanguage{english}hen and chick} \textcolor{PineGreen}{\selectlanguage{french}poule et poussins}  
 \zh{量词}: \textcolor{darkblue}{\textbf{\ipa{mi˩}}} 
\lhead{\firstmark}
\rhead{\botmark}

\subsection{\hspace{-0.5cm} {\Large \textcolor{darkblue}{\textbf{\ipa{æ̃˩ʁv̩˩}}}}\hspace{0.5cm}[\kern2pt{\textcolor{darkblue}{\textbf{\ipa{æ̃˩ʁv̩˩˥}}}}\kern2pt]} \hypertarget{\{\string_~\string_BRv\string_=\string_B1}{}
\markboth{\textcolor{darkblue}{\textbf{\ipa{æ̃˩ʁv̩˩}}}}{}
\textcolor{teal}{\zh{名词}} \hspace{4pt} \zh{声调类:} L.
\zh{蛋。} \textcolor{Sepia}{\selectlanguage{english}Egg.} \textcolor{PineGreen}{\selectlanguage{french}Œuf.}  ¶ \textcolor{darkblue}{\textbf{\ipa{bæ˧mi˧-æ̃˩ʁv̩˩}}} \zh{鸭子蛋} \textcolor{Sepia}{\selectlanguage{english}cane egg} \textcolor{PineGreen}{\selectlanguage{french}œuf de cane}  
 ¶ \textcolor{darkblue}{\textbf{\ipa{æ̃˩ʁv̩˩ dzɯ˩˥}}} \zh{吃蛋} \textcolor{Sepia}{\selectlanguage{english}to eat eggs} \textcolor{PineGreen}{\selectlanguage{french}manger des œufs}  
 \zh{量词}: \textcolor{darkblue}{\textbf{\ipa{ɭɯ˧}}} 
\lhead{\firstmark}
\rhead{\botmark}

\subsection{\hspace{-0.5cm} {\Large \textcolor{darkblue}{\textbf{\ipa{æ̃˩ʂe˧li˥-mo˩}}}}\hspace{0.5cm}[\kern2pt{\textcolor{darkblue}{\textbf{\ipa{æ̃˩ʂe˧li˥mo˧}}}}\kern2pt]} \hypertarget{\{\string_~\string_Bs`e\string_Mli\string_T-mo\string_B1}{}
\markboth{\textcolor{darkblue}{\textbf{\ipa{æ̃˩ʂe˧li˥-mo˩}}}}{}
\textcolor{teal}{\zh{名词}} \hspace{4pt} \zh{声调类:} LM+H\#-.
\zh{麻母鸡菌:一种可以吃的菌子,块鳞灰毒鹅膏菌。} \textcolor{Sepia}{\selectlanguage{english}“Chicken-meat mushroom”: an edible mushroom, \textit{Amanita spissa}.} \textcolor{PineGreen}{\selectlanguage{french}“champignon viande-de-poulet”: un champignon comestible, \textit{Amanita spissa}.} 
\lhead{\firstmark}
\rhead{\botmark}

\subsection{\hspace{-0.5cm} {\Large \textcolor{darkblue}{\textbf{\ipa{æ̃˩ʂe˩}}}}\hspace{0.5cm}[\kern2pt{\textcolor{darkblue}{\textbf{\ipa{æ̃˩ʂe˩˥}}}}\kern2pt]} \hypertarget{\{\string_~\string_Bs`e\string_B1}{}
\markboth{\textcolor{darkblue}{\textbf{\ipa{æ̃˩ʂe˩}}}}{}
\textcolor{teal}{\zh{名词}} \hspace{4pt} \zh{声调类:} L.
\zh{肌肉。} \textcolor{Sepia}{\selectlanguage{english}Muscle.} \textcolor{PineGreen}{\selectlanguage{french}Muscle.}  ¶ \textcolor{darkblue}{\textbf{\ipa{æ̃˩ʂe˩ tsʰi˩˥}}} \zh{发烧} \textcolor{Sepia}{\selectlanguage{english}to run a temperature, to have a fever} \textcolor{PineGreen}{\selectlanguage{french}avoir la fièvre}  
 \zh{量词}: \textcolor{darkblue}{\textbf{\ipa{kʰwɤ˥}}} 
\lhead{\firstmark}
\rhead{\botmark}

\subsection{\hspace{-0.5cm} {\Large \textcolor{darkblue}{\textbf{\ipa{æ̃˩zɯ˩}}}}\hspace{0.5cm}[\kern2pt{\textcolor{darkblue}{\textbf{\ipa{æ̃˩zɯ˩˥}}}}\kern2pt]} \hypertarget{\{\string_~\string_BzM\string_B1}{}
\markboth{\textcolor{darkblue}{\textbf{\ipa{æ̃˩zɯ˩}}}}{}
\textcolor{teal}{\zh{名词}} \hspace{4pt} \zh{声调类:} L.
\zh{玛瑙。} \textcolor{Sepia}{\selectlanguage{english}Agate. Agate of various colours is used in ornamentation. Beads range from the size of a quail egg to that of a chicken's egg.} \textcolor{PineGreen}{\selectlanguage{french}Agate. Des perles d'agate de diverses couleurs sont utilisées en orfèvrerie. Elles sont de la taille d'un oeuf de caille, les plus gros approchent la taille d'un oeuf de poule. Les perles d'agate étaient intégrées aux bijoux et vêtements, dans une tradition d'inspiration tibétaine.}  ¶ \textcolor{darkblue}{\textbf{\ipa{sɯ˧ɻ̍˧-æ̃˩zɯ˩}}} \zh{珠子形状的玛瑙} \textcolor{Sepia}{\selectlanguage{english}pearl-shaped agate bead} \textcolor{PineGreen}{\selectlanguage{french}agate en forme de perle}  
 ¶ \textcolor{darkblue}{\textbf{\ipa{æ̃˩zɯ˩-ʂo˩\textasciitilde{}ʂo˥}}} \zh{(衣服上)都镶嵌着玛瑙} \textcolor{Sepia}{\selectlanguage{english}with lots of agate on it (of a piece of clothing)} \textcolor{PineGreen}{\selectlanguage{french}tout plein d'agate, plein de morceaux d'agate (d'un vêtement)}  
 \zh{量词}: \textcolor{darkblue}{\textbf{\ipa{ɭɯ˧}}} 
\lhead{\firstmark}
\rhead{\botmark}

\subsection{\hspace{-0.5cm} {\Large \textcolor{darkblue}{\textbf{\ipa{æ̃˩˥}}}}\hspace{0.5cm}[\kern2pt{\textcolor{darkblue}{\textbf{\ipa{æ̃˩˥}}}}\kern2pt]} \hypertarget{\{\string_~\string_B\string_T1}{}
\markboth{\textcolor{darkblue}{\textbf{\ipa{æ̃˩˥}}}}{}
\textcolor{teal}{\zh{名词}} \hspace{4pt} \zh{声调类:} LH.
\zh{灵魂。} \textcolor{Sepia}{\selectlanguage{english}Soul (monosyllable).} \textcolor{PineGreen}{\selectlanguage{french}Âme (monosyllabe).}  \zh{量词}: \textcolor{darkblue}{\textbf{\ipa{v̩˧}}} 
\lhead{\firstmark}
\rhead{\botmark}

\subsection{\hspace{-0.5cm} {\Large \textcolor{darkblue}{\textbf{\ipa{æ̃˩˧}}}}\hspace{0.5cm}[\kern2pt{\textcolor{darkblue}{\textbf{\ipa{æ̃˩˥}}}}\kern2pt]} \hypertarget{\{\string_~\string_B\string_M1}{}
\markboth{\textcolor{darkblue}{\textbf{\ipa{æ̃˩˧}}}}{}
\textcolor{teal}{\zh{名词}} \hspace{4pt} \zh{声调类:} LM.
\zh{鸡。} \textcolor{Sepia}{\selectlanguage{english}Chicken.} \textcolor{PineGreen}{\selectlanguage{french}Poulet, poule.}  ¶ \textcolor{darkblue}{\textbf{\ipa{æ̃˩ dzɯ˥-ze˩}}} \zh{吃了鸡} \textcolor{Sepia}{\selectlanguage{english}...has eaten (a/some) chicken} \textcolor{PineGreen}{\selectlanguage{french}...a mangé (un/du) poulet}  
 ¶ \textcolor{darkblue}{\textbf{\ipa{æ̃˩ hwæ˧-ze˧}}} \zh{买了鸡} \textcolor{Sepia}{\selectlanguage{english}...has bought (a) chicken} \textcolor{PineGreen}{\selectlanguage{french}...a acheté (un/du) poulet}  
 ¶ \textcolor{darkblue}{\textbf{\ipa{æ̃˩˥, | kʰv̩˧, | bo˩˥, | hwɤ˧˥, | ʝi˧, | lɑ˧, | tʰo˧li˧, | mv̩˧gv̩˧, | bv̩˧ʐv̩˧, | ʐwæ˧, | jo˧, | ʑi˩˥}}} \zh{十二个生肖} \textcolor{Sepia}{\selectlanguage{english}the twelve years of the duodenary cycle} \textcolor{PineGreen}{\selectlanguage{french}les douze signes astrologiques}  
 ¶ \textcolor{darkblue}{\textbf{\ipa{æ̃˩-mɤ˥}}} \zh{鸡油} \textcolor{Sepia}{\selectlanguage{english}chicken grease, chicken fat} \textcolor{PineGreen}{\selectlanguage{french}graisse de poulet}  
 ¶ \textcolor{darkblue}{\textbf{\ipa{æ̃˩-mɤ˥ dzɯ˩}}} \zh{吃鸡油} \textcolor{Sepia}{\selectlanguage{english}to eat chicken fat} \textcolor{PineGreen}{\selectlanguage{french}manger de la graisse de poulet}  
 \zh{量词}: \textcolor{darkblue}{\textbf{\ipa{mi˩}}} 
\lhead{\firstmark}
\rhead{\botmark}

\newpage
\section*{\centering- \textcolor{darkblue}{\textbf{\ipa{b}}} -}
\subsection{\hspace{-0.5cm} {\Large \textcolor{darkblue}{\textbf{\ipa{bɑ˧lɑ˧kʰɯ˧tsʰɤ˧}}}}\hspace{0.5cm}[\kern2pt{\textcolor{darkblue}{\textbf{\ipa{bɑ˩lɑ˩kʰɯ˩tsʰɤ˩˥}}}}\kern2pt]} \hypertarget{bA\string_MlA\string_Mk\string_hM\string_Mts\string_h7\string_M1}{}
\markboth{\textcolor{darkblue}{\textbf{\ipa{bɑ˧lɑ˧kʰɯ˧tsʰɤ˧}}}}{}
\textcolor{teal}{\zh{名词}} \hspace{4pt} \zh{声调类:} M.
\zh{蜘蛛。} \textcolor{Sepia}{\selectlanguage{english}Spider.} \textcolor{PineGreen}{\selectlanguage{french}Araignée.}  \zh{量词}: \textcolor{darkblue}{\textbf{\ipa{kʰɯ˩}}} 
\lhead{\firstmark}
\rhead{\botmark}

\subsection{\hspace{-0.5cm} {\Large \textcolor{darkblue}{\textbf{\ipa{bɑ˩lɑ˩}}}}\hspace{0.5cm}[\kern2pt{\textcolor{darkblue}{\textbf{\ipa{bɑ˩lɑ˩˥}}}}\kern2pt]} \hypertarget{bA\string_BlA\string_B1}{}
\markboth{\textcolor{darkblue}{\textbf{\ipa{bɑ˩lɑ˩}}}}{}
\textcolor{teal}{\zh{名词}} \hspace{4pt} \zh{声调类:} L.
\ding{202} \zh{上衣,衣服。} \textcolor{Sepia}{\selectlanguage{english}Jacket, upper outer garment; clothes.} \textcolor{PineGreen}{\selectlanguage{french}Chemise, veste; vêtement.}  ¶ \textcolor{darkblue}{\textbf{\ipa{ɣɯ˩-bɑ˩lɑ˥ (+ɲi˩)}}} \zh{皮衣} \textcolor{Sepia}{\selectlanguage{english}leather jacket} \textcolor{PineGreen}{\selectlanguage{french}veste de cuir}  
 \zh{量词}: \textcolor{darkblue}{\textbf{\ipa{ɭɯ˧}}} \ding{203} \zh{胎盘、衣胞。} \textcolor{Sepia}{\selectlanguage{english}Placenta.} \textcolor{PineGreen}{\selectlanguage{french}Placenta.}  \zh{量词}: \textcolor{darkblue}{\textbf{\ipa{ɭɯ˧}}} 
\lhead{\firstmark}
\rhead{\botmark}

\subsection{\hspace{-0.5cm} {\Large \textcolor{darkblue}{\textbf{\ipa{bɑ˩˥}}}}\hspace{0.5cm}[\kern2pt{\textcolor{darkblue}{\textbf{\ipa{bɑ˩˥}}}}\kern2pt]} \hypertarget{bA\string_B\string_T1}{}
\markboth{\textcolor{darkblue}{\textbf{\ipa{bɑ˩˥}}}}{}
\textcolor{teal}{\zh{语气助词}} \hspace{4pt} \zh{声调类:} L?.
\zh{句尾助词,表示肯定:“……是吧。”。} \textcolor{Sepia}{\selectlanguage{english}Affirmative final particle; comparable to question-tag in English.} \textcolor{PineGreen}{\selectlanguage{french}Particule finale affirmative: “...n'est-ce pas”.} 
\lhead{\firstmark}
\rhead{\botmark}

\subsection{\hspace{-0.5cm} {\Large \textcolor{darkblue}{\textbf{\ipa{bæ˧}}} \textsubscript{1}}\hspace{0.5cm}[\kern2pt{\textcolor{darkblue}{\textbf{\ipa{bæ˥}}}}\kern2pt]} \hypertarget{b\{\string_M1}{}
\markboth{\textcolor{darkblue}{\textbf{\ipa{bæ˧}}} \textsubscript{1}}{}
\textcolor{teal}{\zh{形容词}} \hspace{4pt} \zh{声调类:} M.
\zh{傻、笨、蠢。} \textcolor{Sepia}{\selectlanguage{english}Stupid, idiot.} \textcolor{PineGreen}{\selectlanguage{french}Stupide, sot, idiot.}  ¶ \textcolor{darkblue}{\textbf{\ipa{bæ˧-hĩ˧}}} \zh{傻的} \textcolor{Sepia}{\selectlanguage{english}\mytextsc{rel}} \textcolor{PineGreen}{\selectlanguage{french}\mytextsc{rel}}  

\lhead{\firstmark}
\rhead{\botmark}

\subsection{\hspace{-0.5cm} {\Large \textcolor{darkblue}{\textbf{\ipa{bæ˧}}} \textsubscript{2}}\hspace{0.5cm}[\kern2pt{\textcolor{darkblue}{\textbf{\ipa{bæ˥}}}}\kern2pt]} \hypertarget{b\{\string_M2}{}
\markboth{\textcolor{darkblue}{\textbf{\ipa{bæ˧}}} \textsubscript{2}}{}
\textcolor{teal}{\zh{动词}} \hspace{4pt} \zh{声调类:} M intrans.
\zh{放弃。} \textcolor{Sepia}{\selectlanguage{english}To let go, to forget about something (as when providing consolation to someone who has failed, telling her/him not to feel desperate).} \textcolor{PineGreen}{\selectlanguage{french}Laisser tomber, abandonner, ne pas s'entêter.}  ¶ \textcolor{darkblue}{\textbf{\ipa{no˧ | bæ˧-ze˩!}}} \zh{你算了吧!(感叹)} \textcolor{Sepia}{\selectlanguage{english}Forget it! (Consolation to someone who has tried and failed.)} \textcolor{PineGreen}{\selectlanguage{french}Tu dois être bien déçu; allez, laisse tomber! (Ce qu'on dit à quelqu'un qui a échoué après de multiples tentatives.)}  
 ¶ \textcolor{darkblue}{\textbf{\ipa{bæ˧-ze˩ mæ˩!}}} \zh{算了嘛!(感叹)} \textcolor{Sepia}{\selectlanguage{english}Forget it! (Consolation to someone who has tried and failed.)} \textcolor{PineGreen}{\selectlanguage{french}Laisse tomber, allez! (nuance d'évidence)}  

\lhead{\firstmark}
\rhead{\botmark}

\subsection{\hspace{-0.5cm} {\Large \textcolor{darkblue}{\textbf{\ipa{bæ˧\textsubscript{a}}}}}\hspace{0.5cm}[\kern2pt{\textcolor{darkblue}{\textbf{\ipa{bæ˥}}}}\kern2pt]} \hypertarget{b\{\string_Ma1}{}
\markboth{\textcolor{darkblue}{\textbf{\ipa{bæ˧\textsubscript{a}}}}}{}
\textcolor{teal}{\zh{量词}} \hspace{4pt} \zh{声调类:} M\textsubscript{a}.
\zh{量词:东西(一样)。} \textcolor{Sepia}{\selectlanguage{english}Classifier for sorts of things; used in statements of identity: “it is the same”.} \textcolor{PineGreen}{\selectlanguage{french}Classificateur des espèces/sortes de choses. Proche de \textcolor{darkblue}{\textbf{\ipa{/ʁo˩b/}}} “sorte, variété”. S'emploie dans la construction “c'est la même chose”.}  ¶ \textcolor{darkblue}{\textbf{\ipa{ɖɯ˧-bæ˧-lɑ˧ ɲi˥!}}} \zh{是一样的!} \textcolor{Sepia}{\selectlanguage{english}It's the same!} \textcolor{PineGreen}{\selectlanguage{french}c'est pareil!/c'est la même chose!}  
 ¶ \textcolor{darkblue}{\textbf{\ipa{ʝi˧kʰv̩˥-dʑo˩, | ɲi˧-bæ˧ | ʐwɤ˩-tʰɑ˩˥! | ʝi˧kʰv̩˥-dʑo˩, | ɖɯ˧-bæ˧-lɑ˧ ʐwɤ˧-tʰɑ˥!}}} \zh{有些(词组)有两种说法,有些只有一种说法!(情景:讨论的是一些有两种不同变调发音的词组,发音合作人确定:确实有些有两种不同的变调,而有些只有一种声调模型。)} \textcolor{Sepia}{\selectlanguage{english}Some (phrases/combinations between words) can be said two different ways; whereas others can only be said in one way / only have one possible realization! (Context: the investigation bears on tonal variants for phrases, such as numeral-plus-classifier phrases; the consultant confirms that some combinations admit two variants, whereas others only have one possible tone pattern.)} \textcolor{PineGreen}{\selectlanguage{french}Il y en a certaines (=des expressions/des combinaisons de mots), on peut les prononcer de deux façons/elles ont deux schémas tonals différents! Il y en a certaines, il n'y a qu'une façon de les dire/il n'y a qu'une sorte (de réalisation tonale possible)! (commentaire au sujet d'expressions qui ont deux variantes tonales)}  
 ¶ \textcolor{darkblue}{\textbf{\ipa{ɲi˧-bæ˧-ɳɯ˧ | ɖɯ˧-bæ˧ ʝi˧}}} \zh{两者混淆,例如把两个音(两个不同的音位)写成一样,混淆两者} \textcolor{Sepia}{\selectlanguage{english}to confuse two things, e.g. to confuse two sounds (phonemes), and to write them in the same way, missing their opposition} \textcolor{PineGreen}{\selectlanguage{french}confondre deux choses (ex.: confondre deux sons, et les noter de la même façon, alors qu'ils s'opposent entre eux)}  

\lhead{\firstmark}
\rhead{\botmark}

\subsection{\hspace{-0.5cm} {\Large \textcolor{darkblue}{\textbf{\ipa{bæ˧bv̩˥}}}}\hspace{0.5cm}[\kern2pt{\textcolor{darkblue}{\textbf{\ipa{bæ˧bv̩˥}}}}\kern2pt]} \hypertarget{b\{\string_Mbv\string_=\string_T1}{}
\markboth{\textcolor{darkblue}{\textbf{\ipa{bæ˧bv̩˥}}}}{}
\textcolor{teal}{\zh{名词}} \hspace{4pt} \zh{声调类:} H\#.
\zh{猪崽。} \textcolor{Sepia}{\selectlanguage{english}Piglet.} \textcolor{PineGreen}{\selectlanguage{french}Goret, porcelet, cochonnet, petit cochon.}  ¶ \textcolor{darkblue}{\textbf{\ipa{bæ˧bv̩˥-zo˩}}} \zh{猪崽} \textcolor{Sepia}{\selectlanguage{english}same meaning: piglet} \textcolor{PineGreen}{\selectlanguage{french}même sens: goret}  
 \zh{量词}: \textcolor{darkblue}{\textbf{\ipa{ɭɯ˧}}} 
\lhead{\firstmark}
\rhead{\botmark}

\subsection{\hspace{-0.5cm} {\Large \textcolor{darkblue}{\textbf{\ipa{bæ˧ɖæ˧}}}}\hspace{0.5cm}[\kern2pt{\textcolor{darkblue}{\textbf{\ipa{bæ˧ɖæ˧}}}}\kern2pt]} \hypertarget{b\{\string_Md`\{\string_M1}{}
\markboth{\textcolor{darkblue}{\textbf{\ipa{bæ˧ɖæ˧}}}}{}
\textcolor{teal}{\zh{名词}} \hspace{4pt} \zh{声调类:} M.
\zh{细绳。} \textcolor{Sepia}{\selectlanguage{english}Small rope, string.} \textcolor{PineGreen}{\selectlanguage{french}Cordelette.}  \zh{量词}: \textcolor{darkblue}{\textbf{\ipa{kʰɯ˩}}} 
\lhead{\firstmark}
\rhead{\botmark}

\subsection{\hspace{-0.5cm} {\Large \textcolor{darkblue}{\textbf{\ipa{bæ˧mi˧}}} \textsubscript{1}}\hspace{0.5cm}[\kern2pt{\textcolor{darkblue}{\textbf{\ipa{xxxx non-correspondance entre le nombre de morphèmes et le nombre de tons de morphèmes}}}}\kern2pt]} \hypertarget{b\{\string_Mmi\string_M1}{}
\markboth{\textcolor{darkblue}{\textbf{\ipa{bæ˧mi˧}}} \textsubscript{1}}{}
\textcolor{teal}{\zh{名词}} \hspace{4pt} \zh{声调类:} M.
\ding{202} \zh{鸭子。} \textcolor{Sepia}{\selectlanguage{english}Duck (without a specification of gender).} \textcolor{PineGreen}{\selectlanguage{french}Canard (sans préciser le sexe: canard ou cane).}  \zh{量词}: \textcolor{darkblue}{\textbf{\ipa{mi˩}}} \ding{203} \zh{母鸭子。} \textcolor{Sepia}{\selectlanguage{english}Female duck.} \textcolor{PineGreen}{\selectlanguage{french}Cane.}  ¶ \textcolor{darkblue}{\textbf{\ipa{bæ˧mi˧-bæ˧pʰv̩\#˥}}} \zh{母鸭子与公鸭子} \textcolor{Sepia}{\selectlanguage{english}female duck and male duck} \textcolor{PineGreen}{\selectlanguage{french}cane et canard}  
 ¶ \textcolor{darkblue}{\textbf{\ipa{bæ˧mi˧-bæ˧zo\#˥}}} \zh{母鸭与小鸭子} \textcolor{Sepia}{\selectlanguage{english}female duck and duckling} \textcolor{PineGreen}{\selectlanguage{french}cane et caneton}  

\lhead{\firstmark}
\rhead{\botmark}

\subsection{\hspace{-0.5cm} {\Large \textcolor{darkblue}{\textbf{\ipa{bæ˧mi˧}}} \textsubscript{2}}\hspace{0.5cm}[\kern2pt{\textcolor{darkblue}{\textbf{\ipa{bæ˧mi˧}}}}\kern2pt]} \hypertarget{b\{\string_Mmi\string_M2}{}
\markboth{\textcolor{darkblue}{\textbf{\ipa{bæ˧mi˧}}} \textsubscript{2}}{}
\textcolor{teal}{\zh{名词}} \hspace{4pt} \zh{声调类:} M.
\zh{粗绳索。} \textcolor{Sepia}{\selectlanguage{english}Thick rope.} \textcolor{PineGreen}{\selectlanguage{french}Grosse corde.}  \zh{量词}: \textcolor{darkblue}{\textbf{\ipa{kʰɯ˩}}} 
\lhead{\firstmark}
\rhead{\botmark}

\subsection{\hspace{-0.5cm} {\Large \textcolor{darkblue}{\textbf{\ipa{bæ˧mi˧-pʰv̩\#˥}}}}\hspace{0.5cm}[\kern2pt{\textcolor{darkblue}{\textbf{\ipa{xxxx non-correspondance entre le nombre de morphèmes et le nombre de tons de morphèmes}}}}\kern2pt]} \hypertarget{b\{\string_Mmi\string_M-p\string_hv\string_=\#\string_T1}{}
\markboth{\textcolor{darkblue}{\textbf{\ipa{bæ˧mi˧-pʰv̩\#˥}}}}{}
\textcolor{teal}{\zh{名词}} \hspace{4pt} \zh{声调类:} \#H.
\zh{公鸭子。} \textcolor{Sepia}{\selectlanguage{english}Male duck.} \textcolor{PineGreen}{\selectlanguage{french}Canard (mâle).}  ¶ \textcolor{darkblue}{\textbf{\ipa{bæ˧mi˧-pʰv̩˧ tʰv̩˧-mi˧˥}}} \zh{这只公鸭子} \textcolor{Sepia}{\selectlanguage{english}\mytextsc{n}+\mytextsc{dem}+\mytextsc{clf}} \textcolor{PineGreen}{\selectlanguage{french}\mytextsc{n}+\mytextsc{dem}+\mytextsc{clf}}  
 \zh{量词}: \textcolor{darkblue}{\textbf{\ipa{mi˩}}} \zh{~【参考】~} \hyperlink{}{\textcolor{darkblue}{\textbf{\ipa{bæ˧pʰv̩\#˥}}}} 
\lhead{\firstmark}
\rhead{\botmark}

\subsection{\hspace{-0.5cm} {\Large \textcolor{darkblue}{\textbf{\ipa{bæ˧pʰv̩\#˥}}}}\hspace{0.5cm}[\kern2pt{\textcolor{darkblue}{\textbf{\ipa{bæ˩pʰv̩˥}}}}\kern2pt]} \hypertarget{b\{\string_Mp\string_hv\string_=\#\string_T1}{}
\markboth{\textcolor{darkblue}{\textbf{\ipa{bæ˧pʰv̩\#˥}}}}{}
\textcolor{teal}{\zh{名词}} \hspace{4pt} \zh{声调类:} \#H.
\zh{公鸭子。} \textcolor{Sepia}{\selectlanguage{english}Male duck.} \textcolor{PineGreen}{\selectlanguage{french}Canard (mâle).}  ¶ \textcolor{darkblue}{\textbf{\ipa{bæ˧pʰv̩˧ tʰv̩˧-mi˧˥ / bæ˧pʰv̩˧ tʰv̩˧-mi˥\#}}} \zh{这个公鸭子} \textcolor{Sepia}{\selectlanguage{english}\mytextsc{n}+\mytextsc{dem}+\mytextsc{clf}} \textcolor{PineGreen}{\selectlanguage{french}\mytextsc{n}+\mytextsc{dem}+\mytextsc{clf}}  
 ¶ \textcolor{darkblue}{\textbf{\ipa{bæ˧pʰv̩˧-bæ˧mi\#˥}}} \zh{公鸭子与母鸭子} \textcolor{Sepia}{\selectlanguage{english}male duck and female duck} \textcolor{PineGreen}{\selectlanguage{french}canard et cane}  
 \zh{量词}: \textcolor{darkblue}{\textbf{\ipa{mi˩}}} \zh{~【参考】~} \hyperlink{}{\textcolor{darkblue}{\textbf{\ipa{bæ˧mi˧-pʰv̩\#˥}}}} 
\lhead{\firstmark}
\rhead{\botmark}

\subsection{\hspace{-0.5cm} {\Large \textcolor{darkblue}{\textbf{\ipa{bæ˧ʁwɤ˧}}}}\hspace{0.5cm}[\kern2pt{\textcolor{darkblue}{\textbf{\ipa{bæ˩ʁwɤ˩˥}}}}\kern2pt]} \hypertarget{b\{\string_MRw7\string_M1}{}
\markboth{\textcolor{darkblue}{\textbf{\ipa{bæ˧ʁwɤ˧}}}}{}
\textcolor{teal}{\zh{名词}} \hspace{4pt} \zh{声调类:} M.
\zh{温泉乡的一个村落。} \textcolor{Sepia}{\selectlanguage{english}A village close to the Hot Springs.} \textcolor{PineGreen}{\selectlanguage{french}Un village proche des Source Chaudes.}  ¶ \textcolor{darkblue}{\textbf{\ipa{bæ˧ʁwɤ˧-ʁwɤ˧}}} \zh{同上:\textcolor{darkblue}{\textbf{\ipa{/bæ˧ʁwɤ˧/}}}村} \textcolor{Sepia}{\selectlanguage{english}same meaning: the village of \textcolor{darkblue}{\textbf{\ipa{/bæ˧ʁwɤ˧/}}}} \textcolor{PineGreen}{\selectlanguage{french}même sens: le village de \textcolor{darkblue}{\textbf{\ipa{/bæ˧ʁwɤ˧/}}}}  
 ¶ \textcolor{darkblue}{\textbf{\ipa{ə˧go˧-ʁwɤ˧, | ʁwɤ˧lɑ˩-bi˩, | bæ˧ʁwɤ˧, | tʰo˧tsʰe\#˥, | pi˧tsʰe˩-di˩, | pɤ˧dʑɤ˩-di˩, | ʁwɤ˧tv̩˧}}} \zh{永宁背向泸沽湖方向经过的村落。前两个村落拥有相当大的摩梭人口比例,第三个村落是摩梭村,最后一个是普米村。} \textcolor{Sepia}{\selectlanguage{english}Villages that one encounters as one leaves the plain of Yongning (away from the Lake); the first two are perceived as villages with a high proportion of Na members, and the third as a mostly Na village, whereas the next ones are Pumi (Prinmi).} \textcolor{PineGreen}{\selectlanguage{french}Villages au sortir de la plaine de Yongning; les deux premiers comportent une population na; le troisième est un village na; les suivants sont essentiellement des villages pumi/prinmi.}  
 ¶ \textcolor{darkblue}{\textbf{\ipa{bæ˧ʁwɤ˧: | nɑ˩˥!}}} \zh{\textcolor{darkblue}{\textbf{\ipa{/bæ˧ʁwɤ˧/}}}是一个摩梭人村落!} \textcolor{Sepia}{\selectlanguage{english}\textcolor{darkblue}{\textbf{\ipa{/bæ˧ʁwɤ˧/}}} is a Na village!} \textcolor{PineGreen}{\selectlanguage{french}\textcolor{darkblue}{\textbf{\ipa{/bæ˧ʁwɤ˧/}}}, c'est un village na!}  

\lhead{\firstmark}
\rhead{\botmark}

\subsection{\hspace{-0.5cm} {\Large \textcolor{darkblue}{\textbf{\ipa{bæ˧zo\#˥}}}}\hspace{0.5cm}[\kern2pt{\textcolor{darkblue}{\textbf{\ipa{bæ˩zo˩˥}}}}\kern2pt]} \hypertarget{b\{\string_Mzo\#\string_T1}{}
\markboth{\textcolor{darkblue}{\textbf{\ipa{bæ˧zo\#˥}}}}{}
\textcolor{teal}{\zh{名词}} \hspace{4pt} \zh{声调类:} \#H.
\zh{小鸭子。} \textcolor{Sepia}{\selectlanguage{english}Duckling.} \textcolor{PineGreen}{\selectlanguage{french}Caneton, petit canard.}  ¶ \textcolor{darkblue}{\textbf{\ipa{bæ˧zo˧ tʰv̩˧-ɭɯ\#˥}}} \zh{这只小鸭子} \textcolor{Sepia}{\selectlanguage{english}\mytextsc{n}+\mytextsc{dem}+\mytextsc{clf}} \textcolor{PineGreen}{\selectlanguage{french}\mytextsc{n}+\mytextsc{dem}+\mytextsc{clf}}  
 ¶ \textcolor{darkblue}{\textbf{\ipa{bæ˧zo˧-bæ˧mi\#˥}}} \zh{小鸭子与母鸭} \textcolor{Sepia}{\selectlanguage{english}duckling and female duck} \textcolor{PineGreen}{\selectlanguage{french}caneton et cane}  
 \zh{量词}: \textcolor{darkblue}{\textbf{\ipa{ɭɯ˧}}} 
\lhead{\firstmark}
\rhead{\botmark}

\subsection{\hspace{-0.5cm} {\Large \textcolor{darkblue}{\textbf{\ipa{bæ˩}}} \textsubscript{1}}\hspace{0.5cm}[\kern2pt{\textcolor{darkblue}{\textbf{\ipa{bæ˥}}}}\kern2pt]} \hypertarget{b\{\string_B1}{}
\markboth{\textcolor{darkblue}{\textbf{\ipa{bæ˩}}} \textsubscript{1}}{}
\textcolor{teal}{\zh{名词}} \hspace{4pt} \zh{声调类:} L.
\zh{绳子。} \textcolor{Sepia}{\selectlanguage{english}Rope.} \textcolor{PineGreen}{\selectlanguage{french}Corde.}  ¶ \textcolor{darkblue}{\textbf{\ipa{bæ˩ ʈʂʰɯ˩-kʰɯ˥}}} \zh{这条绳子} \textcolor{Sepia}{\selectlanguage{english}\mytextsc{n}+\mytextsc{dem}+\mytextsc{clf}} \textcolor{PineGreen}{\selectlanguage{french}\mytextsc{n}+\mytextsc{dem}+\mytextsc{clf}}  
 \zh{量词}: \textcolor{darkblue}{\textbf{\ipa{ʈʰɤ˥}}} \textcolor{darkblue}{\textbf{\ipa{ɖæ˩}}} \textcolor{darkblue}{\textbf{\ipa{kʰɯ˩}}} 
\lhead{\firstmark}
\rhead{\botmark}

\subsection{\hspace{-0.5cm} {\Large \textcolor{darkblue}{\textbf{\ipa{bæ˩}}} \textsubscript{2}}\hspace{0.5cm}[\kern2pt{\textcolor{darkblue}{\textbf{\ipa{bæ˩˥}}}}\kern2pt]} \hypertarget{b\{\string_B2}{}
\markboth{\textcolor{darkblue}{\textbf{\ipa{bæ˩}}} \textsubscript{2}}{}
\textcolor{teal}{\zh{动词}} \hspace{4pt} \zh{声调类:} L.
\zh{脓。} \textcolor{Sepia}{\selectlanguage{english}To fester (with pus), to suppurate, to be purulent.} \textcolor{PineGreen}{\selectlanguage{french}Pus.}  ¶ \textcolor{darkblue}{\textbf{\ipa{bæ˩ bæ˧-ze˩}}} \zh{伤口在化脓} \textcolor{Sepia}{\selectlanguage{english}the wound suppurates, the wound is pussy} \textcolor{PineGreen}{\selectlanguage{french}la blessure donne du pus, il y a du pus}  
 ¶ \textcolor{darkblue}{\textbf{\ipa{bæ˩˥ | le˧-bæ˩-ze˩}}} \zh{伤口在化脓} \textcolor{Sepia}{\selectlanguage{english}the wound suppurates, the wound is pussy} \textcolor{PineGreen}{\selectlanguage{french}la blessure donne du pus, il y a du pus}  
\zh{~【参考】~} \hyperlink{}{\textcolor{darkblue}{\textbf{\ipa{bæ˩˥}}}} 
\lhead{\firstmark}
\rhead{\botmark}

\subsection{\hspace{-0.5cm} {\Large \textcolor{darkblue}{\textbf{\ipa{bæ˩\textsubscript{a}}}} \textsubscript{1}}\hspace{0.5cm}[\kern2pt{\textcolor{darkblue}{\textbf{\ipa{bæ˩˥}}}}\kern2pt]} \hypertarget{b\{\string_Ba1}{}
\markboth{\textcolor{darkblue}{\textbf{\ipa{bæ˩\textsubscript{a}}}} \textsubscript{1}}{}
\textcolor{teal}{\zh{动词}} \hspace{4pt} \zh{声调类:} L\textsubscript{a}.
\zh{扫。} \textcolor{Sepia}{\selectlanguage{english}To sweep, to clean up.} \textcolor{PineGreen}{\selectlanguage{french}Balayer.}  ¶ \textcolor{darkblue}{\textbf{\ipa{ɖæ˩ bæ˧}}} \zh{扫地} \textcolor{Sepia}{\selectlanguage{english}to sweep the dust, to sweep the floor} \textcolor{PineGreen}{\selectlanguage{french}balayer les saletés, balayer le sol}  
 ¶ \textcolor{darkblue}{\textbf{\ipa{le˧-bæ˧\textasciitilde{}bæ˥}}} \zh{扫一扫} \textcolor{Sepia}{\selectlanguage{english}\mytextsc{accomp} \mytextsc{red}} \textcolor{PineGreen}{\selectlanguage{french}\mytextsc{accomp} \mytextsc{red}}  
 ¶ \textcolor{darkblue}{\textbf{\ipa{ɖʐɤ˩ bæ˩˥}}} \zh{扫楼梯} \textcolor{Sepia}{\selectlanguage{english}to sweep the stairs} \textcolor{PineGreen}{\selectlanguage{french}balayer l'escalier}  
 ¶ \textcolor{darkblue}{\textbf{\ipa{njɤ˧ | ɖʐɤ˩ bæ˩-zo˩-ho˥.}}} \zh{我要扫楼梯了!} \textcolor{Sepia}{\selectlanguage{english}I have to sweep the stairs!} \textcolor{PineGreen}{\selectlanguage{french}Il va falloir que je balaie l'escalier!}  
 ¶ \textcolor{darkblue}{\textbf{\ipa{gi˩ bæ˩˥}}} \zh{扫仓廪} \textcolor{Sepia}{\selectlanguage{english}to sweep the granary} \textcolor{PineGreen}{\selectlanguage{french}balayer le grenier à céréales}  
 ¶ \textcolor{darkblue}{\textbf{\ipa{njɤ˧ | gi˩ bæ˩-zo˩-ho˥.}}} \zh{我要扫仓廪了!} \textcolor{Sepia}{\selectlanguage{english}I have to sweep the granary!} \textcolor{PineGreen}{\selectlanguage{french}Il va falloir que je balaie le grenier à céréales!}  

\lhead{\firstmark}
\rhead{\botmark}

\subsection{\hspace{-0.5cm} {\Large \textcolor{darkblue}{\textbf{\ipa{bæ˩\textsubscript{a}}}} \textsubscript{2}}\hspace{0.5cm}[\kern2pt{\textcolor{darkblue}{\textbf{\ipa{bæ˩˥}}}}\kern2pt]} \hypertarget{b\{\string_Ba2}{}
\markboth{\textcolor{darkblue}{\textbf{\ipa{bæ˩\textsubscript{a}}}} \textsubscript{2}}{}
\textcolor{teal}{\zh{动词}} \hspace{4pt} \zh{声调类:} L\textsubscript{a}.
\zh{开花。} \textcolor{Sepia}{\selectlanguage{english}To bloom.} \textcolor{PineGreen}{\selectlanguage{french}S'ouvrir (fleur), fleurir.}  ¶ \textcolor{darkblue}{\textbf{\ipa{bæ˩bæ˩ bæ˥-ze˩}}} \zh{花开了。} \textcolor{Sepia}{\selectlanguage{english}The flower has bloomed.} \textcolor{PineGreen}{\selectlanguage{french}La fleur a fleuri.}  

\lhead{\firstmark}
\rhead{\botmark}

\subsection{\hspace{-0.5cm} {\Large \textcolor{darkblue}{\textbf{\ipa{bæ˩\textsubscript{a}}}} \textsubscript{3}}\hspace{0.5cm}[\kern2pt{\textcolor{darkblue}{\textbf{\ipa{bæ˩˥}}}}\kern2pt]} \hypertarget{b\{\string_Ba3}{}
\markboth{\textcolor{darkblue}{\textbf{\ipa{bæ˩\textsubscript{a}}}} \textsubscript{3}}{}
\textcolor{teal}{\zh{量词}} \hspace{4pt} \zh{声调类:} L\textsubscript{a}.
\zh{量词:花(一朵)。} \textcolor{Sepia}{\selectlanguage{english}Self-classifier for flowers.} \textcolor{PineGreen}{\selectlanguage{french}Auto-classificateur des fleurs.}  ¶ \textcolor{darkblue}{\textbf{\ipa{tʰv̩˧-bæ˥}}} \zh{\mytextsc{指示代词} \string_ :这朵(花)} \textcolor{Sepia}{\selectlanguage{english}\mytextsc{dem} \string_ (tone: H\# / H\$)} \textcolor{PineGreen}{\selectlanguage{french}\mytextsc{dem} \string_ (ton: H\# / H\$)}  

\lhead{\firstmark}
\rhead{\botmark}

\subsection{\hspace{-0.5cm} {\Large \textcolor{darkblue}{\textbf{\ipa{bæ˩bæ˩}}} \textsubscript{1}}\hspace{0.5cm}[\kern2pt{\textcolor{darkblue}{\textbf{\ipa{bæ˩bæ˩˥}}}}\kern2pt]} \hypertarget{b\{\string_Bb\{\string_B1}{}
\markboth{\textcolor{darkblue}{\textbf{\ipa{bæ˩bæ˩}}} \textsubscript{1}}{}
\textcolor{teal}{\zh{名词}} \hspace{4pt} \zh{声调类:} L.
\zh{花。} \textcolor{Sepia}{\selectlanguage{english}Flower.} \textcolor{PineGreen}{\selectlanguage{french}Fleur.}  \zh{量词}: \textcolor{darkblue}{\textbf{\ipa{bæ˩}}} \zh{~【参考】~} \hyperlink{}{\textcolor{darkblue}{\textbf{\ipa{bæ˩bæ˩}}} \textsubscript{2}} 
\lhead{\firstmark}
\rhead{\botmark}

\subsection{\hspace{-0.5cm} {\Large \textcolor{darkblue}{\textbf{\ipa{bæ˩bæ˩}}} \textsubscript{2}}\hspace{0.5cm}[\kern2pt{\textcolor{darkblue}{\textbf{\ipa{bæ˩bæ˩˥}}}}\kern2pt]} \hypertarget{b\{\string_Bb\{\string_B2}{}
\markboth{\textcolor{darkblue}{\textbf{\ipa{bæ˩bæ˩}}} \textsubscript{2}}{}
\textcolor{teal}{\zh{形容词}} \hspace{4pt} \zh{声调类:} L.
\zh{花的(蛋、石头、鸟)。} \textcolor{Sepia}{\selectlanguage{english}Spotted.} \textcolor{PineGreen}{\selectlanguage{french}Bariolé, tacheté, moucheté (ex.: un oeuf moucheté, un oiseau au pelage moucheté, une pierre ayant plusieurs couleurs).}  ¶ \textcolor{darkblue}{\textbf{\ipa{bæ˩bæ˩ tʰi˩-di˥}}} \zh{花的,有花纹} \textcolor{Sepia}{\selectlanguage{english}same meaning: spotted (e.g. an egg, a bird, a stone)} \textcolor{PineGreen}{\selectlanguage{french}même sens: bariolé, tacheté (par ex.: oeuf, oiseau, pierre)}  
\zh{~【参考】~} \hyperlink{}{\textcolor{darkblue}{\textbf{\ipa{bæ˩bæ˩}}} \textsubscript{1}} 
\lhead{\firstmark}
\rhead{\botmark}

\subsection{\hspace{-0.5cm} {\Large \textcolor{darkblue}{\textbf{\ipa{bæ˩dʑɯ˥}}}}\hspace{0.5cm}[\kern2pt{\textcolor{darkblue}{\textbf{\ipa{bæ˩dʑɯ˥}}}}\kern2pt]} \hypertarget{b\{\string_Bdz£M\string_T1}{}
\markboth{\textcolor{darkblue}{\textbf{\ipa{bæ˩dʑɯ˥}}}}{}
\textcolor{teal}{\zh{名词}} \hspace{4pt} \zh{声调类:} LH.
\zh{庄稼。} \textcolor{Sepia}{\selectlanguage{english}Crops, harvest.} \textcolor{PineGreen}{\selectlanguage{french}Récolte; plantes que l'on a semées.}  ¶ \textcolor{darkblue}{\textbf{\ipa{bæ˩dʑɯ˥ | mɤ˧-dʑɤ˩!}}} \zh{收成不好!} \textcolor{Sepia}{\selectlanguage{english}The harvest is not good!} \textcolor{PineGreen}{\selectlanguage{french}La récolte n'est pas bonne!}  
 ¶ \textcolor{darkblue}{\textbf{\ipa{bæ˩dʑɯ˧ | tv̩˧-bæ˩ le˩-mv̩˩-kʰɯ˩!}}} \zh{祝:一千棵庄稼成熟!(成年礼、过年等节庆时的祝福用语,晚辈对长辈的祝福)} \textcolor{Sepia}{\selectlanguage{english}May a thousand crops come to maturity! (A blessing to elders, used for instance during the rite of coming of age)} \textcolor{PineGreen}{\selectlanguage{french}Puissent mille récoltes venir à maturité! (Bénédiction qu'on dit aux aînés lors de cérémonies: par exemple lors du rite de passage à l'âge adulte)}  
 \zh{量词}: \textcolor{darkblue}{\textbf{\ipa{bæ˩}}} 
\lhead{\firstmark}
\rhead{\botmark}

\subsection{\hspace{-0.5cm} {\Large \textcolor{darkblue}{\textbf{\ipa{bæ˩-lɑ˩\textasciitilde{}lɑ˥}}}}\hspace{0.5cm}[\kern2pt{\textcolor{darkblue}{\textbf{\ipa{xxxx non-correspondance entre le nombre de morphèmes et le nombre de tons de morphèmes}}}}\kern2pt]} \hypertarget{b\{\string_B-lA\string_B~lA\string_T1}{}
\markboth{\textcolor{darkblue}{\textbf{\ipa{bæ˩-lɑ˩\textasciitilde{}lɑ˥}}}}{}
\textcolor{teal}{\zh{形容词}} \hspace{4pt} \zh{声调类:} L.
\zh{软,柔软、软塌塌。} \textcolor{Sepia}{\selectlanguage{english}Soft, weak, pliant.} \textcolor{PineGreen}{\selectlanguage{french}Flasque, sans consistance.} 
\lhead{\firstmark}
\rhead{\botmark}

\subsection{\hspace{-0.5cm} {\Large \textcolor{darkblue}{\textbf{\ipa{bæ˩-ljɤ˧\textasciitilde{}ljɤ˧}}}}\hspace{0.5cm}[\kern2pt{\textcolor{darkblue}{\textbf{\ipa{xxxx non-correspondance entre le nombre de morphèmes et le nombre de tons de morphèmes}}}}\kern2pt]} \hypertarget{b\{\string_B-lj7\string_M~lj7\string_M1}{}
\markboth{\textcolor{darkblue}{\textbf{\ipa{bæ˩-ljɤ˧\textasciitilde{}ljɤ˧}}}}{}
\textcolor{teal}{\zh{名词}} \hspace{4pt} \zh{声调类:} L-.
\zh{杉树果。} \textcolor{Sepia}{\selectlanguage{english}China fir cone.} \textcolor{PineGreen}{\selectlanguage{french}Pomme de pin, fruit du sapin.}  \zh{量词}: \textcolor{darkblue}{\textbf{\ipa{ɭɯ˧}}} 
\lhead{\firstmark}
\rhead{\botmark}

\subsection{\hspace{-0.5cm} {\Large \textcolor{darkblue}{\textbf{\ipa{bæ˩pʰv̩˥}}}}\hspace{0.5cm}[\kern2pt{\textcolor{darkblue}{\textbf{\ipa{bæ˧pʰv̩˧}}}}\kern2pt]} \hypertarget{b\{\string_Bp\string_hv\string_=\string_T1}{}
\markboth{\textcolor{darkblue}{\textbf{\ipa{bæ˩pʰv̩˥}}}}{}
\textcolor{teal}{\zh{名词}} \hspace{4pt} \zh{声调类:} L+H\#.
\zh{茼蒿。} \textcolor{Sepia}{\selectlanguage{english}Crowndaisy chrysanthemum, \textit{Glebionis coronaria}.} \textcolor{PineGreen}{\selectlanguage{french}Chrysanthème couronné, chrysanthème des jardins, chrysanthème comestible ou chrysanthème à couronnes, \textit{Glebionis coronaria}.}  ¶ \textcolor{darkblue}{\textbf{\ipa{bæ˩pʰv̩˥-bv̩˩ | bæ˩bæ˩˥}}} \zh{茼蒿的顶花} \textcolor{Sepia}{\selectlanguage{english}the flower of crowndaisy chrysanthemum} \textcolor{PineGreen}{\selectlanguage{french}fleur de chrysanthème couronné}  
 ¶ \textcolor{darkblue}{\textbf{\ipa{bæ˩pʰv̩˥-bæ˩bæ˩}}} \zh{茼蒿顶花} \textcolor{Sepia}{\selectlanguage{english}crowndaisy chrysanthemum flower} \textcolor{PineGreen}{\selectlanguage{french}fleur de chrysanthème couronné}  
 \zh{量词}: \textcolor{darkblue}{\textbf{\ipa{po˧}}} 
\lhead{\firstmark}
\rhead{\botmark}

\subsection{\hspace{-0.5cm} {\Large \textcolor{darkblue}{\textbf{\ipa{bæ˩-ʁwæ˩\textasciitilde{}ʁwæ˥}}}}\hspace{0.5cm}[\kern2pt{\textcolor{darkblue}{\textbf{\ipa{xxxx non-correspondance entre le nombre de morphèmes et le nombre de tons de morphèmes}}}}\kern2pt]} \hypertarget{b\{\string_B-Rw\{\string_B~Rw\{\string_T1}{}
\markboth{\textcolor{darkblue}{\textbf{\ipa{bæ˩-ʁwæ˩\textasciitilde{}ʁwæ˥}}}}{}
\textcolor{teal}{\zh{形容词}} \hspace{4pt} \zh{声调类:} L.
\zh{松。} \textcolor{Sepia}{\selectlanguage{english}Loose, slack, lax.} \textcolor{PineGreen}{\selectlanguage{french}Relâché.}  ¶ \textcolor{darkblue}{\textbf{\ipa{ʈʂʰɯ˧ | ɖwæ˧˥ | bæ˩ʁwæ˩\textasciitilde{}ʁwæ˥-ʝi˩!}}} \zh{(驮在马上面的货物没系好)松动了!} \textcolor{Sepia}{\selectlanguage{english}It's loose! / It's not well-fastened! (About a load on a mule's back)} \textcolor{PineGreen}{\selectlanguage{french}C'est tout relâché, ce n'est pas bien serré! (Au sujet d'une charge sur le dos d'un mulet)}  

\lhead{\firstmark}
\rhead{\botmark}

\subsection{\hspace{-0.5cm} {\Large \textcolor{darkblue}{\textbf{\ipa{bæ˩ʈʂo˥}}}}\hspace{0.5cm}[\kern2pt{\textcolor{darkblue}{\textbf{\ipa{bæ˧ʈʂo˧}}}}\kern2pt]} \hypertarget{b\{\string_Bt`s`o\string_T1}{}
\markboth{\textcolor{darkblue}{\textbf{\ipa{bæ˩ʈʂo˥}}}}{}
\textcolor{teal}{\zh{名词}} \hspace{4pt} \zh{声调类:} LH.
\zh{扫帚。} \textcolor{Sepia}{\selectlanguage{english}Broom.} \textcolor{PineGreen}{\selectlanguage{french}Balai.}  \zh{量词}: \textcolor{darkblue}{\textbf{\ipa{nɑ˧}}} 
\lhead{\firstmark}
\rhead{\botmark}

\subsection{\hspace{-0.5cm} {\Large \textcolor{darkblue}{\textbf{\ipa{bæ˩ʈʂwæ˩}}}}\hspace{0.5cm}[\kern2pt{\textcolor{darkblue}{\textbf{\ipa{bæ˩ʈʂwæ˥}}}}\kern2pt]} \hypertarget{b\{\string_Bt`s`w\{\string_B1}{}
\markboth{\textcolor{darkblue}{\textbf{\ipa{bæ˩ʈʂwæ˩}}}}{}
\textcolor{teal}{\zh{名词}} \hspace{4pt} \zh{声调类:} L.
\zh{缰绳。} \textcolor{Sepia}{\selectlanguage{english}Reins.} \textcolor{PineGreen}{\selectlanguage{french}Rênes.}  ¶ \textcolor{darkblue}{\textbf{\ipa{ʐwæ˧-bæ˥ʈʂwæ˩}}} \zh{马缰绳} \textcolor{Sepia}{\selectlanguage{english}horse's reins} \textcolor{PineGreen}{\selectlanguage{french}rênes du cheval}  
 \zh{量词}: \textcolor{darkblue}{\textbf{\ipa{kʰɯ˩}}} 
\lhead{\firstmark}
\rhead{\botmark}

\subsection{\hspace{-0.5cm} {\Large \textcolor{darkblue}{\textbf{\ipa{bæ˧˥}}}}\hspace{0.5cm}[\kern2pt{\textcolor{darkblue}{\textbf{\ipa{bæ˧˥}}}}\kern2pt]} \hypertarget{b\{\string_M\string_T1}{}
\markboth{\textcolor{darkblue}{\textbf{\ipa{bæ˧˥}}}}{}
\textcolor{teal}{\zh{动词}} \hspace{4pt} \zh{声调类:} MH.
\zh{跑。} \textcolor{Sepia}{\selectlanguage{english}To run.} \textcolor{PineGreen}{\selectlanguage{french}Courir.}  ¶ \textcolor{darkblue}{\textbf{\ipa{le˧-bæ˧-ze˥}}} \zh{跑了} \textcolor{Sepia}{\selectlanguage{english}\mytextsc{accomp} \string_ \mytextsc{pfv}} \textcolor{PineGreen}{\selectlanguage{french}\mytextsc{accomp} \string_ \mytextsc{pfv}}  

\lhead{\firstmark}
\rhead{\botmark}

\subsection{\hspace{-0.5cm} {\Large \textcolor{darkblue}{\textbf{\ipa{bæ˩˥}}}}\hspace{0.5cm}[\kern2pt{\textcolor{darkblue}{\textbf{\ipa{bæ˩˥}}}}\kern2pt]} \hypertarget{b\{\string_B\string_T1}{}
\markboth{\textcolor{darkblue}{\textbf{\ipa{bæ˩˥}}}}{}
\textcolor{teal}{\zh{名词}} \hspace{4pt} \zh{声调类:} LH.
\zh{脓。} \textcolor{Sepia}{\selectlanguage{english}Pus.} \textcolor{PineGreen}{\selectlanguage{french}Pus.}  ¶ \textcolor{darkblue}{\textbf{\ipa{bæ˩ bæ˧-ze˩}}} \zh{伤口在化脓} \textcolor{Sepia}{\selectlanguage{english}the wound suppurates, the wound is pussy} \textcolor{PineGreen}{\selectlanguage{french}la blessure donne du pus, il y a du pus}  
 ¶ \textcolor{darkblue}{\textbf{\ipa{bæ˩˥ | le˧-bæ˩-ze˩}}} \zh{伤口在化脓} \textcolor{Sepia}{\selectlanguage{english}the wound suppurates, the wound is pussy} \textcolor{PineGreen}{\selectlanguage{french}la blessure donne du pus, il y a du pus}  
 ¶ \textcolor{darkblue}{\textbf{\ipa{bæ˩˥ | le˧-bæ˩-ze˩}}} \textcolor{PineGreen}{\selectlanguage{french}yyyy reporter : verbe, ton ˩}  
 \zh{量词}: \textcolor{darkblue}{\textbf{\ipa{ʈʰɤ˥}}} \zh{~【参考】~} \hyperlink{}{\textcolor{darkblue}{\textbf{\ipa{bæ˩}}} \textsubscript{2}} 
\lhead{\firstmark}
\rhead{\botmark}

\subsection{\hspace{-0.5cm} {\Large \textcolor{darkblue}{\textbf{\ipa{bæ˩˧}}}}\hspace{0.5cm}[\kern2pt{\textcolor{darkblue}{\textbf{\ipa{bæ˩˥}}}}\kern2pt]} \hypertarget{b\{\string_B\string_M1}{}
\markboth{\textcolor{darkblue}{\textbf{\ipa{bæ˩˧}}}}{}
\textcolor{teal}{\zh{名词}} \hspace{4pt} \zh{声调类:} LM.
\zh{庄稼。} \textcolor{Sepia}{\selectlanguage{english}Crops.} \textcolor{PineGreen}{\selectlanguage{french}Récolte; plantes que l'on a semées.}  ¶ \textcolor{darkblue}{\textbf{\ipa{bæ˩ ɲi˧}}} \zh{是庄稼} \textcolor{Sepia}{\selectlanguage{english}\mytextsc{cop}} \textcolor{PineGreen}{\selectlanguage{french}\mytextsc{cop}}  
 ¶ \textcolor{darkblue}{\textbf{\ipa{ɖɯ˧-kʰv̩˧ ʈv̩˧-bæ˥ mv̩˩, | ɕi˧-kʰv̩˧ | le˧-mɤ˧-dzɯ˧!}}} \zh{“一年收千棵,不够吃百年!”(这个谚语,来慰藉收成不好的年份。)} \textcolor{Sepia}{\selectlanguage{english}“This year, even if we had had a thousand harvests, it would not have lasted a hundred years!” This proverb is a consolation for years of bad harvests: “If the harvest had been excellent, it would not have lasted forever anyway! Everything begins anew every year, so let us look forward!”} \textcolor{PineGreen}{\selectlanguage{french}“Quand bien même on aurait fait une récolte fabuleuse, ça ne nous durerait pas éternellement: ça se rejoue chaque année!” Littéralement: “si, une année, mille récoltes parvenaient à maturité, on n['en] mangerait pas [pour autant pendant] cent ans =on n'aurait pas à manger pour cent ans!” Le proverbe sert à se consoler d'une mauvaise récolte, qui va obliger à une année frugale: “Si belle soit la récolte, elle n'aurait de toute façon pas duré éternellement; tout est à recommencer l'année suivante, voyons donc de l'avant!”}  
 \zh{量词}: \textcolor{darkblue}{\textbf{\ipa{bæ˩}}} 
\lhead{\firstmark}
\rhead{\botmark}

\subsection{\hspace{-0.5cm} {\Large \textcolor{darkblue}{\textbf{\ipa{bɤ˥}}}}\hspace{0.5cm}[\kern2pt{\textcolor{darkblue}{\textbf{\ipa{bɤ˥}}}}\kern2pt]} \hypertarget{b7\string_T1}{}
\markboth{\textcolor{darkblue}{\textbf{\ipa{bɤ˥}}}}{}
\textcolor{teal}{\zh{名词}} \hspace{4pt} \zh{声调类:} \#H.
\zh{普米族。} \textcolor{Sepia}{\selectlanguage{english}Pumi (Prinmi) (ethnic group).} \textcolor{PineGreen}{\selectlanguage{french}Pumi (Prinmi) (groupe ethnique).}  ¶ \textcolor{darkblue}{\textbf{\ipa{bɤ˧-ʐwɤ˧ so˥}}} \zh{学普米语} \textcolor{Sepia}{\selectlanguage{english}to learn the Pumi language} \textcolor{PineGreen}{\selectlanguage{french}apprendre la langue prinmi}  
 \zh{量词}: \textcolor{darkblue}{\textbf{\ipa{v̩˧}}} 
\lhead{\firstmark}
\rhead{\botmark}

\subsection{\hspace{-0.5cm} {\Large \textcolor{darkblue}{\textbf{\ipa{bɤ˧dzi˩}}}}\hspace{0.5cm}[\kern2pt{\textcolor{darkblue}{\textbf{\ipa{xxxx non-correspondance entre le nombre de morphèmes et le nombre de tons de morphèmes}}}}\kern2pt]} \hypertarget{b7\string_Mdzi\string_B1}{}
\markboth{\textcolor{darkblue}{\textbf{\ipa{bɤ˧dzi˩}}}}{}
\textcolor{teal}{\zh{名词}} \hspace{4pt} \zh{声调类:} L\#.
\zh{八珠(永宁的一个村落)。} \textcolor{Sepia}{\selectlanguage{english}A village in Yongning.} \textcolor{PineGreen}{\selectlanguage{french}Un des villages de la plaine de Yongning.}  ¶ \textcolor{darkblue}{\textbf{\ipa{ɖæ˩ʂɯ\#˥, | ʈʂo˧ʂɯ\#˥, | bɤ˩tɕʰɯ˩˥, | dɑ˧pʰo˥, | bɤ˧dzi˩, | dze˧bo˧}}} \zh{永宁坝的六个村落,按传统排序:从距离泸沽湖最近的村落说起。} \textcolor{Sepia}{\selectlanguage{english}the six villages of the plain of Yongning, in traditional order: by order of increasing distance from the Lake} \textcolor{PineGreen}{\selectlanguage{french}les six villages de la plaine de Yongning, dans l'ordre, qui prend comme point d'origine le village le plus proche du Lac}  

\lhead{\firstmark}
\rhead{\botmark}

\subsection{\hspace{-0.5cm} {\Large \textcolor{darkblue}{\textbf{\ipa{bɤ˧kɯ˧}}}}\hspace{0.5cm}[\kern2pt{\textcolor{darkblue}{\textbf{\ipa{bɤ˩kɯ˥}}}}\kern2pt]} \hypertarget{b7\string_MkM\string_M1}{}
\markboth{\textcolor{darkblue}{\textbf{\ipa{bɤ˧kɯ˧}}}}{}
\textcolor{teal}{\zh{名词}} \hspace{4pt} \zh{声调类:} M.
\zh{筛子。} \textcolor{Sepia}{\selectlanguage{english}Sifter, sieve.} \textcolor{PineGreen}{\selectlanguage{french}Vannerie: tamis, crible, en forme de gourde; on y met des légumes, des choses à porter. Cette vannerie est commode à porter.}  \zh{量词}: \textcolor{darkblue}{\textbf{\ipa{nɑ˧}}} 
\lhead{\firstmark}
\rhead{\botmark}

\subsection{\hspace{-0.5cm} {\Large \textcolor{darkblue}{\textbf{\ipa{bɤ˧mi\#˥}}}}\hspace{0.5cm}[\kern2pt{\textcolor{darkblue}{\textbf{\ipa{xxxx non-correspondance entre le nombre de morphèmes et le nombre de tons de morphèmes}}}}\kern2pt]} \hypertarget{b7\string_Mmi\#\string_T1}{}
\markboth{\textcolor{darkblue}{\textbf{\ipa{bɤ˧mi\#˥}}}}{}
\textcolor{teal}{\zh{名词}} \hspace{4pt} \zh{声调类:} \#H.
\zh{普米族女人。} \textcolor{Sepia}{\selectlanguage{english}Pumi woman.} \textcolor{PineGreen}{\selectlanguage{french}Femme pumi.}  \zh{量词}: \textcolor{darkblue}{\textbf{\ipa{v̩˧}}} 
\lhead{\firstmark}
\rhead{\botmark}

\subsection{\hspace{-0.5cm} {\Large \textcolor{darkblue}{\textbf{\ipa{bɤ˧mi˥-ʂe˩}}}}\hspace{0.5cm}[\kern2pt{\textcolor{darkblue}{\textbf{\ipa{bɤ˩mi˧ʂe˧}}}}\kern2pt]} \hypertarget{b7\string_Mmi\string_T-s`e\string_B1}{}
\markboth{\textcolor{darkblue}{\textbf{\ipa{bɤ˧mi˥-ʂe˩}}}}{}
\textcolor{teal}{\zh{名词}} \hspace{4pt} \zh{声调类:} H\#-.
\zh{白铜。} \textcolor{Sepia}{\selectlanguage{english}Copper-nickel alloy.} \textcolor{PineGreen}{\selectlanguage{french}Cupronickel: alliage cuivre-nickel.} 
\lhead{\firstmark}
\rhead{\botmark}

\subsection{\hspace{-0.5cm} {\Large \textcolor{darkblue}{\textbf{\ipa{bɤ˧ʂɯ˩}}}}\hspace{0.5cm}[\kern2pt{\textcolor{darkblue}{\textbf{\ipa{bɤ˩ʂɯ˥}}}}\kern2pt]} \hypertarget{b7\string_Ms`M\string_B1}{}
\markboth{\textcolor{darkblue}{\textbf{\ipa{bɤ˧ʂɯ˩}}}}{}
\textcolor{teal}{\zh{名词}} \hspace{4pt} \zh{声调类:} L\#.
\zh{白沙(丽江坝子里的一个村落)。} \textcolor{Sepia}{\selectlanguage{english}Baisha: name of a village of the Lijiang plain. The village had a tradition of trading and peddling to faraway places, hence its familiarity to people in Yongning. Its Naxi name is \textcolor{darkblue}{\textbf{\ipa{/bɤ˧ʂɯ˩/}}}.} \textcolor{PineGreen}{\selectlanguage{french}Nom d'un village de la plaine de Lijiang, d'où venaient de nombreux marchands, d'où le fait que son nom soit connu à Yongning. En naxi: \textcolor{darkblue}{\textbf{\ipa{/bɤ˧ʂɯ˩/}}}.} 
\lhead{\firstmark}
\rhead{\botmark}

\subsection{\hspace{-0.5cm} {\Large \textcolor{darkblue}{\textbf{\ipa{bɤ˧tʰv̩˩}}}}\hspace{0.5cm}[\kern2pt{\textcolor{darkblue}{\textbf{\ipa{bɤ˧tʰv̩˥}}}}\kern2pt]} \hypertarget{b7\string_Mt\string_hv\string_=\string_B1}{}
\markboth{\textcolor{darkblue}{\textbf{\ipa{bɤ˧tʰv̩˩}}}}{}
\textcolor{teal}{\zh{名词}} \hspace{4pt} \zh{声调类:} L\#.
\zh{脚印。} \textcolor{Sepia}{\selectlanguage{english}Footprints.} \textcolor{PineGreen}{\selectlanguage{french}Empreintes, traces de pas, traces de pied.}  ¶ \textcolor{darkblue}{\textbf{\ipa{hĩ˧-bɤ˧tʰv̩˥}}} \zh{人的脚印} \textcolor{Sepia}{\selectlanguage{english}human footprints} \textcolor{PineGreen}{\selectlanguage{french}empreintes (de pied) d'homme}  
 ¶ \textcolor{darkblue}{\textbf{\ipa{kʰv̩˩mi˩-bɤ˩tʰv̩˥}}} \zh{狗爪印} \textcolor{Sepia}{\selectlanguage{english}dog's footprints} \textcolor{PineGreen}{\selectlanguage{french}empreintes de (pattes de) chien}  
 \zh{量词}: \textcolor{darkblue}{\textbf{\ipa{tʰv̩˧˥}}} 
\lhead{\firstmark}
\rhead{\botmark}

\subsection{\hspace{-0.5cm} {\Large \textcolor{darkblue}{\textbf{\ipa{bɤ˧tsʰo˧gv̩˥}}}}\hspace{0.5cm}[\kern2pt{\textcolor{darkblue}{\textbf{\ipa{bɤ˧tsʰo˧gv̩˩}}}}\kern2pt]} \hypertarget{b7\string_Mts\string_ho\string_Mgv\string_=\string_T1}{}
\markboth{\textcolor{darkblue}{\textbf{\ipa{bɤ˧tsʰo˧gv̩˥}}}}{}
\textcolor{teal}{\zh{名词}} \hspace{4pt} \zh{声调类:} H\#.
\zh{巴搓古(永宁的一个村落)。} \textcolor{Sepia}{\selectlanguage{english}A village of the Lijiang plain: the central village of the plain, where the marketplace was still located in the early 21st century.} \textcolor{PineGreen}{\selectlanguage{french}Un des villages de la plaine de Yongning; lieu de l'actuel marché; terme également employé pour désigner le lieu d'habitation des artisans Naxi).}  ¶ \textcolor{darkblue}{\textbf{\ipa{bɤ˧tsʰo˧gv̩˥-hĩ˩}}} \zh{从巴搓古来的一个人} \textcolor{Sepia}{\selectlanguage{english}someone from Bacuogu} \textcolor{PineGreen}{\selectlanguage{french}quelqu'un de Bacuogu}  
 ¶ \textcolor{darkblue}{\textbf{\ipa{dʑɤ˩bv̩˧kɤ˧-sɑ˥ʁwɤ˩, | hi˩ʁwɤ˩-lo˥, | æ˩mi˧-ʁwɤ\#˥, | lɑ˧lo˧-ʁwɤ˥, | lɑ˧ŋwɤ˧, | bɤ˧tsʰo˧gv̩˥, | ə˧lɑ˧-ʁwɤ\#˥, | gæ˧ɻæ˩, | qʰæ˧tɕʰi˧, | tʰo˧ʈɯ\#˥}}} \zh{摩梭传统地理概念中,属于永宁的十个村落} \textcolor{Sepia}{\selectlanguage{english}the ten villages traditionally considered as part of Yongning} \textcolor{PineGreen}{\selectlanguage{french}les dix villages comptant traditionnellement comme faisant partie de Yongning}  

\lhead{\firstmark}
\rhead{\botmark}

\subsection{\hspace{-0.5cm} {\Large \textcolor{darkblue}{\textbf{\ipa{bɤ˧zo\#˥}}}}\hspace{0.5cm}[\kern2pt{\textcolor{darkblue}{\textbf{\ipa{bɤ˧zo˧}}}}\kern2pt]} \hypertarget{b7\string_Mzo\#\string_T1}{}
\markboth{\textcolor{darkblue}{\textbf{\ipa{bɤ˧zo\#˥}}}}{}
\textcolor{teal}{\zh{名词}} \hspace{4pt} \zh{声调类:} \#H.
\zh{普米族男人。} \textcolor{Sepia}{\selectlanguage{english}Pumi man.} \textcolor{PineGreen}{\selectlanguage{french}Homme pumi.}  \zh{量词}: \textcolor{darkblue}{\textbf{\ipa{v̩˧}}} 
\lhead{\firstmark}
\rhead{\botmark}

\subsection{\hspace{-0.5cm} {\Large \textcolor{darkblue}{\textbf{\ipa{bɤ˩\textsubscript{a}}}} \textsubscript{1}}\hspace{0.5cm}[\kern2pt{\textcolor{darkblue}{\textbf{\ipa{bɤ˩˥}}}}\kern2pt]} \hypertarget{b7\string_Ba1}{}
\markboth{\textcolor{darkblue}{\textbf{\ipa{bɤ˩\textsubscript{a}}}} \textsubscript{1}}{}
\textcolor{teal}{\zh{量词}} \hspace{4pt} \zh{声调类:} L\textsubscript{a}.
\zh{量词:玉米棒子(一根)。} \textcolor{Sepia}{\selectlanguage{english}Classifier for corncobs.} \textcolor{PineGreen}{\selectlanguage{french}Classificateur des épis de maïs.}  ¶ \textcolor{darkblue}{\textbf{\ipa{hɑ˧bɤ˥ | ɖɯ˧-bɤ˩}}} \zh{一根玉米棒子} \textcolor{Sepia}{\selectlanguage{english}a corncob} \textcolor{PineGreen}{\selectlanguage{french}un épi de maïs}  
 ¶ \textcolor{darkblue}{\textbf{\ipa{tʰv̩˧-bɤ˥}}} \zh{\mytextsc{指示代词} \string_:那根(玉米棒子)} \textcolor{Sepia}{\selectlanguage{english}\mytextsc{dem} \string_ (tone: H\# / H\$)} \textcolor{PineGreen}{\selectlanguage{french}\mytextsc{dem} \string_ (ton: H\# / H\$)}  

\lhead{\firstmark}
\rhead{\botmark}

\subsection{\hspace{-0.5cm} {\Large \textcolor{darkblue}{\textbf{\ipa{bɤ˩\textsubscript{a}}}} \textsubscript{2}}\hspace{0.5cm}[\kern2pt{\textcolor{darkblue}{\textbf{\ipa{bɤ˩˥}}}}\kern2pt]} \hypertarget{b7\string_Ba2}{}
\markboth{\textcolor{darkblue}{\textbf{\ipa{bɤ˩\textsubscript{a}}}} \textsubscript{2}}{}
\textcolor{teal}{\zh{量词}} \hspace{4pt} \zh{声调类:} L\textsubscript{a}.
\zh{量词:半。} \textcolor{Sepia}{\selectlanguage{english}Classifier for halves.} \textcolor{PineGreen}{\selectlanguage{french}Classificateur des moitiés.}  ¶ \textcolor{darkblue}{\textbf{\ipa{ɖɯ˧-bɤ˩-lɑ˩ tʰv̩˩-sɯ˩! | ɖɯ˧-hu˧-ɻ̍˥!}}} \zh{我才干了一半!稍等!(情景:一个在收拾衣服,告诉对方:工作没完,还需要时间。)} \textcolor{Sepia}{\selectlanguage{english}I have only done one half as yet! Wait a bit! (Context: someone is sorting out clothes, and is midway through the task.)} \textcolor{PineGreen}{\selectlanguage{french}Je n'ai fait que la moitié! Attends un peu! (Contexte: quelqu'un est en train de trier les vêtements, et signale qu'il n'a pas fini)}  
 ¶ \textcolor{darkblue}{\textbf{\ipa{ɖɯ˧-bɤ˩-lɑ˩ tʰv̩˩-ze˩!}}} \zh{你才到了一半!(合作人对调查者学摩梭话的评定)} \textcolor{Sepia}{\selectlanguage{english}You are only half-way through! (An observation about the investigator's progress in studying the Na language. It emphasizes the road ahead; still, this is a more encouraging formulation than the previous example: this one uses the perfective, acknowledging that part of the path that has already been covered, whereas \textcolor{darkblue}{\textbf{\ipa{ɖɯ˧-bɤ˩-lɑ˩ tʰv̩˩-sɯ˩}}} would essentially emphasize all that remains ahead.)} \textcolor{PineGreen}{\selectlanguage{french}Tu es à mi-chemin! / Tu n'as parcouru que la moitié du chemin! (Réflexion de la consultante principale au sujet de mon apprentissage de la langue na. Cette formulation souligne le chemin qui reste à parcourir, mais l'emploi du perfectif apporte une note d'encouragement, là où l'exemple précédant insisterait essentiellement sur le chemin restant à parcourir: \textcolor{darkblue}{\textbf{\ipa{ɖɯ˧-bɤ˩-lɑ˩ tʰv̩˩-sɯ˩}}}.)}  
 ¶ \textcolor{darkblue}{\textbf{\ipa{ʐæ˩ʂæ˥ | ʐwæ˩˥! | le˧-se˥, | ɖɯ˧-bɤ˩-qo˩-lɑ˩ tʰv̩˩-sɯ˩!}}} \zh{真远!走啊走,才走了一半的路!} \textcolor{Sepia}{\selectlanguage{english}It's really far! We have walked (for quite some time), and only got mid-way!} \textcolor{PineGreen}{\selectlanguage{french}C'est bien loin! On a marché, et on n'est encore parvenu qu'à mi-chemin!}  
 ¶ \textcolor{darkblue}{\textbf{\ipa{ʐɤ˩mi˩˥ | ɖɯ˧-bɤ˩}}} \zh{半路} \textcolor{Sepia}{\selectlanguage{english}half the way} \textcolor{PineGreen}{\selectlanguage{french}la moitié du chemin}  
 ¶ \textcolor{darkblue}{\textbf{\ipa{ə˧mi˧! | wɤ˩˥ | ɖɯ˧-bɤ˩ dʑo˩-sɯ˩-wɤ˩!}}} \zh{啊呀嚒!还剩一半的路啊!} \textcolor{Sepia}{\selectlanguage{english}Wow! (How far!) There is still half the way to go!} \textcolor{PineGreen}{\selectlanguage{french}Houlàà! (Que c'est loin!) Il reste encore la moitié du chemin (à parcourir)!}  

\lhead{\firstmark}
\rhead{\botmark}

\subsection{\hspace{-0.5cm} {\Large \textcolor{darkblue}{\textbf{\ipa{bɤ˩tɕʰɯ˩}}}}\hspace{0.5cm}[\kern2pt{\textcolor{darkblue}{\textbf{\ipa{bɤ˧tɕʰɯ˩}}}}\kern2pt]} \hypertarget{b7\string_Bts£\string_hM\string_B1}{}
\markboth{\textcolor{darkblue}{\textbf{\ipa{bɤ˩tɕʰɯ˩}}}}{}
\textcolor{teal}{\zh{名词}} \hspace{4pt} \zh{声调类:} L.
\zh{八七(永宁的一个村落)。} \textcolor{Sepia}{\selectlanguage{english}A village in the plain of Lijiang.} \textcolor{PineGreen}{\selectlanguage{french}Un des villages de la plaine de Yongning.}  ¶ \textcolor{darkblue}{\textbf{\ipa{ɖæ˩ʂɯ\#˥, | ʈʂo˧ʂɯ\#˥, | bɤ˩tɕʰɯ˩˥, | dɑ˧pʰo˥, | bɤ˧dzi˩, | dze˧bo˧}}} \zh{永宁坝的六个村落,按传统排序:从距离泸沽湖最近的村落说起。} \textcolor{Sepia}{\selectlanguage{english}the six villages of the plain of Yongning, in traditional order: by order of increasing distance from the Lake} \textcolor{PineGreen}{\selectlanguage{french}les six villages de la plaine de Yongning, dans l'ordre, qui prend comme point d'origine le village le plus proche du Lac}  

\lhead{\firstmark}
\rhead{\botmark}

\subsection{\hspace{-0.5cm} {\Large \textcolor{darkblue}{\textbf{\ipa{bɤ˧˥\textsubscript{a}}}}}\hspace{0.5cm}[\kern2pt{\textcolor{darkblue}{\textbf{\ipa{bɤ˩˥}}}}\kern2pt]} \hypertarget{b7\string_M\string_Ta1}{}
\markboth{\textcolor{darkblue}{\textbf{\ipa{bɤ˧˥\textsubscript{a}}}}}{}
\textcolor{teal}{\zh{量词}} \hspace{4pt} \zh{声调类:} MH\textsubscript{a}.
\zh{量词:头帕(一条)。} \textcolor{Sepia}{\selectlanguage{english}Classifier for scarves.} \textcolor{PineGreen}{\selectlanguage{french}Classificateur des fichus et foulards.} 
\lhead{\firstmark}
\rhead{\botmark}

\subsection{\hspace{-0.5cm} {\Large \textcolor{darkblue}{\textbf{\ipa{‑bi}}}}\hspace{0.5cm}[\kern2pt{\textcolor{darkblue}{\textbf{\ipa{xxxx groupe tonal entier sans aucun ton}}}}\kern2pt]} \hypertarget{‑bi1}{}
\markboth{\textcolor{darkblue}{\textbf{\ipa{‑bi}}}}{}
\textcolor{teal}{\zh{连接词}} \hspace{4pt} \zh{声调类:} 0.
\zh{虽然……。} \textcolor{Sepia}{\selectlanguage{english}\mytextsc{adversative}: no matter….} \textcolor{PineGreen}{\selectlanguage{french}\mytextsc{adversatif}: bien que, même si.}  ¶ \textcolor{darkblue}{\textbf{\ipa{ʈʂʰɯ˧ | nɑ˩ ɲi˥-pi˩-bi˩-bi˩, | nɑ˩-ʐwɤ˧ | mɤ˧-kv̩˧˥!}}} \zh{他虽然是摩梭人但不会讲摩梭话。} \textcolor{Sepia}{\selectlanguage{english}Although (s)he is Na, (s)he cannot speak the Na language!} \textcolor{PineGreen}{\selectlanguage{french}Bien qu'elle/il soit Na, elle/il ne sait pas parler la langue na!}  
 ¶ \textcolor{darkblue}{\textbf{\ipa{*ʈʂʰɯ˧ | nɑ˩ ɲi˥-bi˩, … / *ʈʂʰɯ˧ | nɑ˩ ɲi˥-bi˩-bi˩}}} \zh{病句:不能这样说“他虽然是摩梭人……”} \textcolor{Sepia}{\selectlanguage{english}an ungrammatical sentence; the intended meaning was 'although (s)he is Na...'} \textcolor{PineGreen}{\selectlanguage{french}phrase non acceptable; l'intention était de dire 'bien qu'il soit Na…'}  

\lhead{\firstmark}
\rhead{\botmark}

\subsection{\hspace{-0.5cm} {\Large \textcolor{darkblue}{\textbf{\ipa{bi˥}}}}\hspace{0.5cm}[\kern2pt{\textcolor{darkblue}{\textbf{\ipa{bi˥}}}}\kern2pt]} \hypertarget{bi\string_T1}{}
\markboth{\textcolor{darkblue}{\textbf{\ipa{bi˥}}}}{}
\textcolor{teal}{\zh{名词}} \hspace{4pt} \zh{声调类:} \#H.
\zh{雪。} \textcolor{Sepia}{\selectlanguage{english}Snow.} \textcolor{PineGreen}{\selectlanguage{french}Neige.}  ¶ \textcolor{darkblue}{\textbf{\ipa{bi˧ gi˧-ze˩}}} \zh{下雪了} \textcolor{Sepia}{\selectlanguage{english}it is snowing; it has snowed} \textcolor{PineGreen}{\selectlanguage{french}il neige}  
 \zh{量词}: \textcolor{darkblue}{\textbf{\ipa{ʁwɤ˧, etc}}} 
\lhead{\firstmark}
\rhead{\botmark}

\subsection{\hspace{-0.5cm} {\Large \textcolor{darkblue}{\textbf{\ipa{bi˥}}}}\hspace{0.5cm}[\kern2pt{\textcolor{darkblue}{\textbf{\ipa{bi˥}}}}\kern2pt]} \hypertarget{bi\string_T1}{}
\markboth{\textcolor{darkblue}{\textbf{\ipa{bi˥}}}}{}
\textcolor{teal}{\zh{形容词}} \hspace{4pt} \zh{声调类:} H.
\zh{薄,浅(水浅)。} \textcolor{Sepia}{\selectlanguage{english}Thin; shallow.} \textcolor{PineGreen}{\selectlanguage{french}Mince; peu profond.}  ¶ \textcolor{darkblue}{\textbf{\ipa{bi˧ | ʐwæ˩˥!}}} \zh{很浅!} \textcolor{Sepia}{\selectlanguage{english}It's very shallow!} \textcolor{PineGreen}{\selectlanguage{french}C'est très peu profond!}  
 ¶ \textcolor{darkblue}{\textbf{\ipa{dʑɯ˧ | ɖɯ˧-pi˧ bi˧˥}}} \zh{水有点浅。} \textcolor{Sepia}{\selectlanguage{english}The water is rather shallow.} \textcolor{PineGreen}{\selectlanguage{french}L'eau est assez peu profonde.}  
 ¶ \textcolor{darkblue}{\textbf{\ipa{dʑɯ˧ bi˧-hĩ˧, | mɤ˧-ɖwæ˩!}}} \zh{水很浅,不用怕!} \textcolor{Sepia}{\selectlanguage{english}Shallow water is not frightening! / There is nothing frightening about shallow water! / Come on, don't be afraid: it's (just) shallow water!} \textcolor{PineGreen}{\selectlanguage{french}L'eau pas profonde, ça fait pas peur! / L'eau n'est pas profonde; tu vas quand même pas avoir peur!}  

\lhead{\firstmark}
\rhead{\botmark}

\subsection{\hspace{-0.5cm} {\Large \textcolor{darkblue}{\textbf{\ipa{‑bi˧}}}}\hspace{0.5cm}[\kern2pt{\textcolor{darkblue}{\textbf{\ipa{bi˥}}}}\kern2pt]} \hypertarget{‑bi\string_M1}{}
\markboth{\textcolor{darkblue}{\textbf{\ipa{‑bi˧}}}}{}
\textcolor{teal}{\zh{后缀}} \hspace{4pt} \zh{声调类:} M.
\zh{要\mytextsc{近将来。}} \textcolor{Sepia}{\selectlanguage{english}Immediate future.} \textcolor{PineGreen}{\selectlanguage{french}Futur immédiat.} 
\lhead{\firstmark}
\rhead{\botmark}

\subsection{\hspace{-0.5cm} {\Large \textcolor{darkblue}{\textbf{\ipa{bi˧}}} \textsubscript{2}}\hspace{0.5cm}[\kern2pt{\textcolor{darkblue}{\textbf{\ipa{bi˥}}}}\kern2pt]} \hypertarget{bi\string_M2}{}
\markboth{\textcolor{darkblue}{\textbf{\ipa{bi˧}}} \textsubscript{2}}{}
\textcolor{teal}{\zh{名词}} \hspace{4pt} \zh{声调类:} M.
\zh{村落,邻居、村里的人们。} \textcolor{Sepia}{\selectlanguage{english}Village; neighbours, people in the village.} \textcolor{PineGreen}{\selectlanguage{french}Le village; les gens du village, les voisins.} 
\lhead{\firstmark}
\rhead{\botmark}

\subsection{\hspace{-0.5cm} {\Large \textcolor{darkblue}{\textbf{\ipa{bi˧}}} \textsubscript{3}}\hspace{0.5cm}[\kern2pt{\textcolor{darkblue}{\textbf{\ipa{bi˥}}}}\kern2pt]} \hypertarget{bi\string_M3}{}
\markboth{\textcolor{darkblue}{\textbf{\ipa{bi˧}}} \textsubscript{3}}{}
\textcolor{teal}{\zh{动词}} \hspace{4pt} \zh{声调类:} M.
\zh{敢。} \textcolor{Sepia}{\selectlanguage{english}To dare.} \textcolor{PineGreen}{\selectlanguage{french}Oser.}  ¶ \textcolor{darkblue}{\textbf{\ipa{ʝi˧-mɤ˧-bi˧}}} \zh{不敢做} \textcolor{Sepia}{\selectlanguage{english}not to dare to do} \textcolor{PineGreen}{\selectlanguage{french}ne pas oser faire}  

\lhead{\firstmark}
\rhead{\botmark}

\subsection{\hspace{-0.5cm} {\Large \textcolor{darkblue}{\textbf{\ipa{bi˧\textsubscript{c}}}} \textsubscript{1}}\hspace{0.5cm}[\kern2pt{\textcolor{darkblue}{\textbf{\ipa{bi˩˥}}}}\kern2pt]} \hypertarget{bi\string_Mc1}{}
\markboth{\textcolor{darkblue}{\textbf{\ipa{bi˧\textsubscript{c}}}} \textsubscript{1}}{}
\textcolor{teal}{\zh{动词}} \hspace{4pt} \zh{声调类:} M\textsubscript{c}.
\zh{去。} \textcolor{Sepia}{\selectlanguage{english}To go.} \textcolor{PineGreen}{\selectlanguage{french}Aller.}  ¶ \textcolor{darkblue}{\textbf{\ipa{bi˧-tʰɑ˧!}}} \zh{可以去!} \textcolor{Sepia}{\selectlanguage{english}\mytextsc{abilitive}} \textcolor{PineGreen}{\selectlanguage{french}\mytextsc{abilitive}: On peut y aller!}  
 ¶ \textcolor{darkblue}{\textbf{\ipa{bi˧-tʰɑ˧-ze˥!}}} \zh{可以去了!} \textcolor{Sepia}{\selectlanguage{english}\mytextsc{abilitive}+\mytextsc{pfv}: We can go now!} \textcolor{PineGreen}{\selectlanguage{french}\mytextsc{abilitive}+\mytextsc{pfv}: Ca y est, on peut y aller!}  
 ¶ \textcolor{darkblue}{\textbf{\ipa{le˧-bi˩}}} \zh{回去,返回} \textcolor{Sepia}{\selectlanguage{english}to go back} \textcolor{PineGreen}{\selectlanguage{french}retourner; s'en retourner}  

\lhead{\firstmark}
\rhead{\botmark}

\subsection{\hspace{-0.5cm} {\Large \textcolor{darkblue}{\textbf{\ipa{bi˧bv̩˥}}}}\hspace{0.5cm}[\kern2pt{\textcolor{darkblue}{\textbf{\ipa{bi˩bv̩˥}}}}\kern2pt]} \hypertarget{bi\string_Mbv\string_=\string_T1}{}
\markboth{\textcolor{darkblue}{\textbf{\ipa{bi˧bv̩˥}}}}{}
\textcolor{teal}{\zh{动词}} \hspace{4pt} \zh{声调类:} H\#.
\zh{流淌,冲下去,下泻,很快地流。} \textcolor{Sepia}{\selectlanguage{english}To flow profusely.} \textcolor{PineGreen}{\selectlanguage{french}Couler à flots, couler en trombe.}  ¶ \textcolor{darkblue}{\textbf{\ipa{tʰi˧-ʈwæ˧˥, | sɤ˧ | bi˧bv̩˥-ze˩!}}} \zh{(他)摔倒了,流了很多血} \textcolor{Sepia}{\selectlanguage{english}(She/he) fell down; blood is flowing profusely!} \textcolor{PineGreen}{\selectlanguage{french}[il] est tombé [et s'est blessé]; le sang a coulé à flots!}  
 ¶ \textcolor{darkblue}{\textbf{\ipa{dʑɯ˧ | bi˧bv̩˥-ze˩!}}} \zh{水流如注!} \textcolor{Sepia}{\selectlanguage{english}The water is flowing profusely!} \textcolor{PineGreen}{\selectlanguage{french}L'eau coule à flots!}  
 ¶ \textcolor{darkblue}{\textbf{\ipa{dʑɯ˩nɑ˩mi˩ bi˩bv̩˥-ze˩-pʰæ˩di˩!}}} \zh{山上好像有了泥石流!} \textcolor{Sepia}{\selectlanguage{english}There seems to be a flood / landslide out on the mountain!} \textcolor{PineGreen}{\selectlanguage{french}On dirait qu'il y a une coulée de boue / un glissement de terrain sur la montagne!}  

\lhead{\firstmark}
\rhead{\botmark}

\subsection{\hspace{-0.5cm} {\Large \textcolor{darkblue}{\textbf{\ipa{bi˧ɕi˧kv̩˥}}}}\hspace{0.5cm}[\kern2pt{\textcolor{darkblue}{\textbf{\ipa{bi˧ɕi˧kv̩˧}}}}\kern2pt]} \hypertarget{bi\string_Ms£i\string_Mkv\string_=\string_T1}{}
\markboth{\textcolor{darkblue}{\textbf{\ipa{bi˧ɕi˧kv̩˥}}}}{}
\textcolor{teal}{\zh{名词}} \hspace{4pt} \zh{声调类:} H\#.
\zh{腮,腮帮子。} \textcolor{Sepia}{\selectlanguage{english}Cheek.} \textcolor{PineGreen}{\selectlanguage{french}Joue.}  \zh{量词}: \textcolor{darkblue}{\textbf{\ipa{ɭɯ˧}}} 
\lhead{\firstmark}
\rhead{\botmark}

\subsection{\hspace{-0.5cm} {\Large \textcolor{darkblue}{\textbf{\ipa{bi˧hæ˧˥}}}}\hspace{0.5cm}[\kern2pt{\textcolor{darkblue}{\textbf{\ipa{bi˧hæ˥}}}}\kern2pt]} \hypertarget{bi\string_Mh\{\string_M\string_T1}{}
\markboth{\textcolor{darkblue}{\textbf{\ipa{bi˧hæ˧˥}}}}{}
\textcolor{teal}{\zh{名词}} \hspace{4pt} \zh{声调类:} MH\#.
\zh{马肚带。} \textcolor{Sepia}{\selectlanguage{english}Girth (for horse).} \textcolor{PineGreen}{\selectlanguage{french}Sangle ventrale.}  ¶ \textcolor{darkblue}{\textbf{\ipa{ʐwæ˧-bi˥hæ˩}}} \zh{马肚带} \textcolor{Sepia}{\selectlanguage{english}horse's girth} \textcolor{PineGreen}{\selectlanguage{french}sangle de cheval}  
 \zh{量词}: \textcolor{darkblue}{\textbf{\ipa{kʰɯ˩}}} 
\lhead{\firstmark}
\rhead{\botmark}

\subsection{\hspace{-0.5cm} {\Large \textcolor{darkblue}{\textbf{\ipa{bi˧-lv̩˧\textasciitilde{}lv̩˥}}}}\hspace{0.5cm}[\kern2pt{\textcolor{darkblue}{\textbf{\ipa{xxxx non-correspondance entre le nombre de morphèmes et le nombre de tons de morphèmes}}}}\kern2pt]} \hypertarget{bi\string_M-lv\string_=\string_M~lv\string_=\string_T1}{}
\markboth{\textcolor{darkblue}{\textbf{\ipa{bi˧-lv̩˧\textasciitilde{}lv̩˥}}}}{}
\textcolor{teal}{\zh{名词}} \hspace{4pt} \zh{声调类:} H\#.
\zh{雪花。} \textcolor{Sepia}{\selectlanguage{english}Snow flakes.} \textcolor{PineGreen}{\selectlanguage{french}Flocons de neige.}  \zh{量词}: \textcolor{darkblue}{\textbf{\ipa{ɭɯ˧}}} 
\lhead{\firstmark}
\rhead{\botmark}

\subsection{\hspace{-0.5cm} {\Large \textcolor{darkblue}{\textbf{\ipa{bi˧mi˧}}}}\hspace{0.5cm}[\kern2pt{\textcolor{darkblue}{\textbf{\ipa{bi˩mi˩˥}}}}\kern2pt]} \hypertarget{bi\string_Mmi\string_M1}{}
\markboth{\textcolor{darkblue}{\textbf{\ipa{bi˧mi˧}}}}{}
\textcolor{teal}{\zh{名词}} \hspace{4pt} \zh{声调类:} M.
\zh{肚子。} \textcolor{Sepia}{\selectlanguage{english}Belly, abdomen.} \textcolor{PineGreen}{\selectlanguage{french}Ventre.}  ¶ \textcolor{darkblue}{\textbf{\ipa{bi˧mi˧-ɖɯ˩}}} \zh{贪心不足,贪吃} \textcolor{Sepia}{\selectlanguage{english}covetous, greedy} \textcolor{PineGreen}{\selectlanguage{french}avide, glouton; littéralement “qui a un gros ventre/gros estomac”}  
 \zh{量词}: \textcolor{darkblue}{\textbf{\ipa{ɭɯ˧}}} 
\lhead{\firstmark}
\rhead{\botmark}

\subsection{\hspace{-0.5cm} {\Large \textcolor{darkblue}{\textbf{\ipa{bi˧tɑ˧}}}}\hspace{0.5cm}[\kern2pt{\textcolor{darkblue}{\textbf{\ipa{bi˩tɑ˩˥}}}}\kern2pt]} \hypertarget{bi\string_MtA\string_M1}{}
\markboth{\textcolor{darkblue}{\textbf{\ipa{bi˧tɑ˧}}}}{}
\textcolor{teal}{\zh{名词}} \hspace{4pt} \zh{声调类:} M.
\zh{宽腰带。} \textcolor{Sepia}{\selectlanguage{english}Broad piece of fabric worn as a belt, also strapped around the shoulders.} \textcolor{PineGreen}{\selectlanguage{french}Ceinture large, en tissu.}  \zh{量词}: \textcolor{darkblue}{\textbf{\ipa{tsʰi˥}}} 
\lhead{\firstmark}
\rhead{\botmark}

\subsection{\hspace{-0.5cm} {\Large \textcolor{darkblue}{\textbf{\ipa{bi˧tɕɤ˩}}}}\hspace{0.5cm}[\kern2pt{\textcolor{darkblue}{\textbf{\ipa{bi˧tɕɤ˧}}}}\kern2pt]} \hypertarget{bi\string_Mts£7\string_B1}{}
\markboth{\textcolor{darkblue}{\textbf{\ipa{bi˧tɕɤ˩}}}}{}
\textcolor{teal}{\zh{名词}} \hspace{4pt} \zh{声调类:} L\#.
\zh{肚脐。} \textcolor{Sepia}{\selectlanguage{english}Navel.} \textcolor{PineGreen}{\selectlanguage{french}Nombril.}  \zh{量词}: \textcolor{darkblue}{\textbf{\ipa{kʰwɤ˥}}} 
\lhead{\firstmark}
\rhead{\botmark}

\subsection{\hspace{-0.5cm} {\Large \textcolor{darkblue}{\textbf{\ipa{bi˧tɕo˧}}}}\hspace{0.5cm}[\kern2pt{\textcolor{darkblue}{\textbf{\ipa{bi˧tɕo˧}}}}\kern2pt]} \hypertarget{bi\string_Mts£o\string_M1}{}
\markboth{\textcolor{darkblue}{\textbf{\ipa{bi˧tɕo˧}}}}{}
\textcolor{teal}{\zh{名词}} \hspace{4pt} \zh{声调类:} M.
\zh{周围的村落。} \textcolor{Sepia}{\selectlanguage{english}Neighbouring villages, neighbourhood.} \textcolor{PineGreen}{\selectlanguage{french}Les villages environnants; par exemple, pour la consultante principale, le village, \textcolor{darkblue}{\textbf{\ipa{/ʁwɤ˧qo˧/}}}, c'est \textcolor{darkblue}{\textbf{\ipa{/ə˧lɑ˧-ʁwɤ˧/;}}} les villages environnants constituent le voisinage, \textcolor{darkblue}{\textbf{\ipa{/bi˧tɕo˧/}}}.} 
\lhead{\firstmark}
\rhead{\botmark}

\subsection{\hspace{-0.5cm} {\Large \textcolor{darkblue}{\textbf{\ipa{bi˧zɯ˧}}}}\hspace{0.5cm}[\kern2pt{\textcolor{darkblue}{\textbf{\ipa{bi˩zɯ˥}}}}\kern2pt]} \hypertarget{bi\string_MzM\string_M1}{}
\markboth{\textcolor{darkblue}{\textbf{\ipa{bi˧zɯ˧}}}}{}
\textcolor{teal}{\zh{名词}} \hspace{4pt} \zh{声调类:} M.
\zh{小肚子。} \textcolor{Sepia}{\selectlanguage{english}Lower abdomen.} \textcolor{PineGreen}{\selectlanguage{french}Bas-ventre.}  \zh{量词}: \textcolor{darkblue}{\textbf{\ipa{ɭɯ˧}}} 
\lhead{\firstmark}
\rhead{\botmark}

\subsection{\hspace{-0.5cm} {\Large \textcolor{darkblue}{\textbf{\ipa{bi˩}}}}\hspace{0.5cm}[\kern2pt{\textcolor{darkblue}{\textbf{\ipa{bi˥}}}}\kern2pt]} \hypertarget{bi\string_B1}{}
\markboth{\textcolor{darkblue}{\textbf{\ipa{bi˩}}}}{}
\textcolor{teal}{\zh{后置词}} \hspace{4pt} \zh{声调类:} L.
\zh{向、至、往。} \textcolor{Sepia}{\selectlanguage{english}On; at.} \textcolor{PineGreen}{\selectlanguage{french}Sur; vers.}  ¶ \textcolor{darkblue}{\textbf{\ipa{gv̩˧mi˧-bi˩}}} \zh{身上} \textcolor{Sepia}{\selectlanguage{english}on the body} \textcolor{PineGreen}{\selectlanguage{french}sur le corps}  
 ¶ \textcolor{darkblue}{\textbf{\ipa{kʰɯ˧tsʰɤ˧-bi˥}}} \zh{脚上} \textcolor{Sepia}{\selectlanguage{english}on the feet} \textcolor{PineGreen}{\selectlanguage{french}sur les pieds}  
 ¶ \textcolor{darkblue}{\textbf{\ipa{ʐæ˩sɯ˩-bi˥ | tʰi˧-ʈʂʰv̩˧˥}}} \zh{抓住毡子} \textcolor{Sepia}{\selectlanguage{english}to hold grip of the felt cape} \textcolor{PineGreen}{\selectlanguage{french}[elle] a agrippé son vêtement}  
 ¶ \textcolor{darkblue}{\textbf{\ipa{lo˩qʰwɤ˧ bi˩}}} \zh{手上} \textcolor{Sepia}{\selectlanguage{english}on the hand} \textcolor{PineGreen}{\selectlanguage{french}sur la main}  
 ¶ \textcolor{darkblue}{\textbf{\ipa{pʰæ˧qʰwɤ˩ bi˩, | mɤ˩ tʰi˩-jɤ˩˥.}}} \zh{抹防晒霜} \textcolor{Sepia}{\selectlanguage{english}to put oil onto the face, to put on suncream} \textcolor{PineGreen}{\selectlanguage{french}s’étaler de l’huile sur le visage, se mettre de la crème solaire sur le visage}  

\lhead{\firstmark}
\rhead{\botmark}

\subsection{\hspace{-0.5cm} {\Large \textcolor{darkblue}{\textbf{\ipa{bi˩\textsubscript{c}}}}}\hspace{0.5cm}[\kern2pt{\textcolor{darkblue}{\textbf{\ipa{bi˥}}}}\kern2pt]} \hypertarget{bi\string_Bc1}{}
\markboth{\textcolor{darkblue}{\textbf{\ipa{bi˩\textsubscript{c}}}}}{}
\textcolor{teal}{\zh{量词}} \hspace{4pt} \zh{声调类:} L\textsubscript{c}.
\zh{量词:动物的脚或脚印(一只)。} \textcolor{Sepia}{\selectlanguage{english}Self-classifier for animal hooves; also used for footprints.} \textcolor{PineGreen}{\selectlanguage{french}Auto-classificateur des sabots d'animaux, et des traces qu'ils laissent sur le sol.} 
\lhead{\firstmark}
\rhead{\botmark}

\subsection{\hspace{-0.5cm} {\Large \textcolor{darkblue}{\textbf{\ipa{bi˩bi˧}}}}\hspace{0.5cm}[\kern2pt{\textcolor{darkblue}{\textbf{\ipa{bi˧bi˧}}}}\kern2pt]} \hypertarget{bi\string_Bbi\string_M1}{}
\markboth{\textcolor{darkblue}{\textbf{\ipa{bi˩bi˧}}}}{}
\textcolor{teal}{\zh{名词}} \hspace{4pt} \zh{声调类:} LM.
\zh{豆荚。} \textcolor{Sepia}{\selectlanguage{english}Pod (of bean).} \textcolor{PineGreen}{\selectlanguage{french}Cosse de haricot.}  ¶ \textcolor{darkblue}{\textbf{\ipa{bi˩bi˧ ɲi˩}}} \zh{是豆荚} \textcolor{Sepia}{\selectlanguage{english}\mytextsc{cop}} \textcolor{PineGreen}{\selectlanguage{french}\mytextsc{cop}}  
 ¶ \textcolor{darkblue}{\textbf{\ipa{nv̩˩ɭɯ˧-bi˩bi˩}}} \zh{黄豆荚} \textcolor{Sepia}{\selectlanguage{english}soybean pods} \textcolor{PineGreen}{\selectlanguage{french}cosses de soja}  
 \zh{量词}: \textcolor{darkblue}{\textbf{\ipa{kʰwɤ˥}}} 
\lhead{\firstmark}
\rhead{\botmark}

\subsection{\hspace{-0.5cm} {\Large \textcolor{darkblue}{\textbf{\ipa{bi˩mi˩}}}}\hspace{0.5cm}[\kern2pt{\textcolor{darkblue}{\textbf{\ipa{bi˧mi˥}}}}\kern2pt]} \hypertarget{bi\string_Bmi\string_B1}{}
\markboth{\textcolor{darkblue}{\textbf{\ipa{bi˩mi˩}}}}{}
\textcolor{teal}{\zh{名词}} \hspace{4pt} \zh{声调类:} L.
\zh{斧头。} \textcolor{Sepia}{\selectlanguage{english}Axe.} \textcolor{PineGreen}{\selectlanguage{french}Hache.}  \zh{量词}: \textcolor{darkblue}{\textbf{\ipa{nɑ˧}}} 
\lhead{\firstmark}
\rhead{\botmark}

\subsection{\hspace{-0.5cm} {\Large \textcolor{darkblue}{\textbf{\ipa{bi˩pʰv̩˧˥}}}}\hspace{0.5cm}[\kern2pt{\textcolor{darkblue}{\textbf{\ipa{bi˧pʰv̩˧}}}}\kern2pt]} \hypertarget{bi\string_Bp\string_hv\string_=\string_M\string_T1}{}
\markboth{\textcolor{darkblue}{\textbf{\ipa{bi˩pʰv̩˧˥}}}}{}
\textcolor{teal}{\zh{名词}} \hspace{4pt} \zh{声调类:} LM+MH\#.
\zh{葫芦。} \textcolor{Sepia}{\selectlanguage{english}Bottle gourd; its fruit is the calabash, which becomes hard when dry.} \textcolor{PineGreen}{\selectlanguage{french}Gourde (cucurbitacée); son fruit est la calebasse, qui devient dure en séchant.} 
\lhead{\firstmark}
\rhead{\botmark}

\subsection{\hspace{-0.5cm} {\Large \textcolor{darkblue}{\textbf{\ipa{bi˩pʰv̩˧-dʑɯ˧˥}}}}\hspace{0.5cm}[\kern2pt{\textcolor{darkblue}{\textbf{\ipa{xxxx non-correspondance entre le nombre de morphèmes et le nombre de tons de morphèmes}}}}\kern2pt]} \hypertarget{bi\string_Bp\string_hv\string_=\string_M-dz£M\string_M\string_T1}{}
\markboth{\textcolor{darkblue}{\textbf{\ipa{bi˩pʰv̩˧-dʑɯ˧˥}}}}{}
\textcolor{teal}{\zh{名词}} \hspace{4pt} \zh{声调类:} LM+MH\#.
\zh{洪水。} \textcolor{Sepia}{\selectlanguage{english}Flood.} \textcolor{PineGreen}{\selectlanguage{french}Inondation.} 
\lhead{\firstmark}
\rhead{\botmark}

\subsection{\hspace{-0.5cm} {\Large \textcolor{darkblue}{\textbf{\ipa{bi˩ʁo˥}}}}\hspace{0.5cm}[\kern2pt{\textcolor{darkblue}{\textbf{\ipa{bi˩ʁo˧˥}}}}\kern2pt]} \hypertarget{bi\string_BRo\string_T1}{}
\markboth{\textcolor{darkblue}{\textbf{\ipa{bi˩ʁo˥}}}}{}
\textcolor{teal}{\zh{名词}} \hspace{4pt} \zh{声调类:} LH.
\zh{钱包(过去:是系在腰带上的)。} \textcolor{Sepia}{\selectlanguage{english}Purse (used to be attached to the belt, on the inside).} \textcolor{PineGreen}{\selectlanguage{french}Bourse. La bourse était autrefois attachée à la ceinture, sur sa face interne.}  ¶ \textcolor{darkblue}{\textbf{\ipa{bi˩ʁo˧ tʰi˧-ʂæ˧˥}}} \zh{系上钱包(在腰带上)} \textcolor{Sepia}{\selectlanguage{english}to attach (one's) purse (to the belt)} \textcolor{PineGreen}{\selectlanguage{french}attacher sa bourse (à sa ceinture)}  
 \zh{量词}: \textcolor{darkblue}{\textbf{\ipa{ɭɯ˧}}} 
\lhead{\firstmark}
\rhead{\botmark}

\subsection{\hspace{-0.5cm} {\Large \textcolor{darkblue}{\textbf{\ipa{bi˩tɑ˩}}}}\hspace{0.5cm}[\kern2pt{\textcolor{darkblue}{\textbf{\ipa{bi˩tɑ˥}}}}\kern2pt]} \hypertarget{bi\string_BtA\string_B1}{}
\markboth{\textcolor{darkblue}{\textbf{\ipa{bi˩tɑ˩}}}}{}
\textcolor{teal}{\zh{动词}} \hspace{4pt} \zh{声调类:} L.
\zh{拖。} \textcolor{Sepia}{\selectlanguage{english}To pull, to drag.} \textcolor{PineGreen}{\selectlanguage{french}Tirer.}  ¶ \textcolor{darkblue}{\textbf{\ipa{bi˩tɑ˩-ze˥}}} \zh{拖了} \textcolor{Sepia}{\selectlanguage{english}\mytextsc{pfv}} \textcolor{PineGreen}{\selectlanguage{french}\mytextsc{pfv}}  

\lhead{\firstmark}
\rhead{\botmark}

\subsection{\hspace{-0.5cm} {\Large \textcolor{darkblue}{\textbf{\ipa{bi˩-tsɯ˧tsɯ˧}}}}\hspace{0.5cm}[\kern2pt{\textcolor{darkblue}{\textbf{\ipa{xxxx non-correspondance entre le nombre de morphèmes et le nombre de tons de morphèmes}}}}\kern2pt]} \hypertarget{bi\string_B-tsM\string_MtsM\string_M1}{}
\markboth{\textcolor{darkblue}{\textbf{\ipa{bi˩-tsɯ˧tsɯ˧}}}}{}
\textcolor{teal}{\zh{名词}} \hspace{4pt} \zh{声调类:} L-.
\zh{野草莓。} \textcolor{Sepia}{\selectlanguage{english}Wild strawberry, \textit{Fragaria vesca}.} \textcolor{PineGreen}{\selectlanguage{french}Fraise sauvage, \textit{Fragaria vesca}.} 
\lhead{\firstmark}
\rhead{\botmark}

\subsection{\hspace{-0.5cm} {\Large \textcolor{darkblue}{\textbf{\ipa{bi˩ʈʂʰɤ\#˥}}}}\hspace{0.5cm}[\kern2pt{\textcolor{darkblue}{\textbf{\ipa{bi˧ʈʂʰɤ˩}}}}\kern2pt]} \hypertarget{bi\string_Bt`s`\string_h7\#\string_T1}{}
\markboth{\textcolor{darkblue}{\textbf{\ipa{bi˩ʈʂʰɤ\#˥}}}}{}
\textcolor{teal}{\zh{名词}} \hspace{4pt} \zh{声调类:} LM+\#H.
\zh{胡须。} \textcolor{Sepia}{\selectlanguage{english}Whiskers.} \textcolor{PineGreen}{\selectlanguage{french}Favoris, rouflaquettes.}  \zh{量词}: \textcolor{darkblue}{\textbf{\ipa{kʰwɤ˥}}} 
\lhead{\firstmark}
\rhead{\botmark}

\subsection{\hspace{-0.5cm} {\Large \textcolor{darkblue}{\textbf{\ipa{bi˩wɤ˧}}}}\hspace{0.5cm}[\kern2pt{\textcolor{darkblue}{\textbf{\ipa{xxxx non-correspondance entre le nombre de morphèmes et le nombre de tons de morphèmes}}}}\kern2pt]} \hypertarget{bi\string_Bw7\string_M1}{}
\markboth{\textcolor{darkblue}{\textbf{\ipa{bi˩wɤ˧}}}}{}
\textcolor{teal}{\zh{名词}} \hspace{4pt} \zh{声调类:} LM.
\zh{酬劳。} \textcolor{Sepia}{\selectlanguage{english}Services (or money) offered as a reward to a priest.} \textcolor{PineGreen}{\selectlanguage{french}Services (ou argent) donnés en récompense à un moine.}  \zh{量词}: \textcolor{darkblue}{\textbf{\ipa{ɭɯ˧}}} 
\lhead{\firstmark}
\rhead{\botmark}

\subsection{\hspace{-0.5cm} {\Large \textcolor{darkblue}{\textbf{\ipa{bo}}}}\hspace{0.5cm}[\kern2pt{\textcolor{darkblue}{\textbf{\ipa{[]}}}}\kern2pt]} \hypertarget{bo1}{}
\markboth{\textcolor{darkblue}{\textbf{\ipa{bo}}}}{}
\textcolor{teal}{\zh{语气助词}} \hspace{4pt} \zh{声调类:} 0.
\zh{句尾助词:啵(汉语借词)。} \textcolor{Sepia}{\selectlanguage{english}Final particle expressing vigorous affirmation.} \textcolor{PineGreen}{\selectlanguage{french}Particule finale empruntée au chinois, exprimant une affirmation vigoureuse: ah que si!}  \zh{【借词】} \zh{啵}

\lhead{\firstmark}
\rhead{\botmark}

\subsection{\hspace{-0.5cm} {\Large \textcolor{darkblue}{\textbf{\ipa{bo˧}}} \textsubscript{1}}\hspace{0.5cm}[\kern2pt{\textcolor{darkblue}{\textbf{\ipa{bo˩˥}}}}\kern2pt]} \hypertarget{bo\string_M1}{}
\markboth{\textcolor{darkblue}{\textbf{\ipa{bo˧}}} \textsubscript{1}}{}
\textcolor{teal}{\zh{名词}} \hspace{4pt} \zh{声调类:} M.
\zh{田埂、小堤坝。} \textcolor{Sepia}{\selectlanguage{english}Small dike (as at the edge of a field; one can walk on it).} \textcolor{PineGreen}{\selectlanguage{french}Diguette en bord de champ, sur laquelle on peut marcher.}  ¶ \textcolor{darkblue}{\textbf{\ipa{ʈʂʰɯ˧-qo˧ | bo˧ ɖɯ˧-ɭɯ˧ tʰi˧-di˩.}}} \zh{这里有一个小堤坝。} \textcolor{Sepia}{\selectlanguage{english}Here, there is a small dike.} \textcolor{PineGreen}{\selectlanguage{french}ici, il y a une diguette.}  
 ¶ \textcolor{darkblue}{\textbf{\ipa{bo˧-kʰi˧}}} \zh{小堤坝的边沿} \textcolor{Sepia}{\selectlanguage{english}the edge of the small dike} \textcolor{PineGreen}{\selectlanguage{french}le bord de la diguette}  
 ¶ \textcolor{darkblue}{\textbf{\ipa{[F5] bo˧ ɖɯ˧-pʰæ˧˥}}} \zh{一个小堤坝} \textcolor{Sepia}{\selectlanguage{english}a small dike} \textcolor{PineGreen}{\selectlanguage{french}une diguette}  
 \zh{量词}: \textcolor{darkblue}{\textbf{\ipa{ɭɯ˧}}} 
\lhead{\firstmark}
\rhead{\botmark}

\subsection{\hspace{-0.5cm} {\Large \textcolor{darkblue}{\textbf{\ipa{bo˧}}} \textsubscript{2}}\hspace{0.5cm}[\kern2pt{\textcolor{darkblue}{\textbf{\ipa{bo˥}}}}\kern2pt]} \hypertarget{bo\string_M2}{}
\markboth{\textcolor{darkblue}{\textbf{\ipa{bo˧}}} \textsubscript{2}}{}
\textcolor{teal}{\zh{形容词}} \hspace{4pt} \zh{声调类:} M.
\zh{亮,光明。} \textcolor{Sepia}{\selectlanguage{english}Bright, shining.} \textcolor{PineGreen}{\selectlanguage{french}Lumineux.}  ¶ \textcolor{darkblue}{\textbf{\ipa{tʰi˧-bo˧-dʑo˧}}} \zh{(它)在发光。(描写灯)} \textcolor{Sepia}{\selectlanguage{english}It casts light / it is bright. (Definition of a lamp.)} \textcolor{PineGreen}{\selectlanguage{french}ça éclaire, c'est lumineux (définition d'une lampe)}  
 ¶ \textcolor{darkblue}{\textbf{\ipa{bo˧-hĩ˧}}} \zh{发亮的} \textcolor{Sepia}{\selectlanguage{english}\mytextsc{rel}} \textcolor{PineGreen}{\selectlanguage{french}\mytextsc{rel}}  

\lhead{\firstmark}
\rhead{\botmark}

\subsection{\hspace{-0.5cm} {\Large \textcolor{darkblue}{\textbf{\ipa{bo˧\textsubscript{b}}}}}\hspace{0.5cm}[\kern2pt{\textcolor{darkblue}{\textbf{\ipa{bo˩˥}}}}\kern2pt]} \hypertarget{bo\string_Mb1}{}
\markboth{\textcolor{darkblue}{\textbf{\ipa{bo˧\textsubscript{b}}}}}{}
\textcolor{teal}{\zh{动词}} \hspace{4pt} \zh{声调类:} M\textsubscript{b}.
\zh{纺(麻线),使旋转。} \textcolor{Sepia}{\selectlanguage{english}To spin.} \textcolor{PineGreen}{\selectlanguage{french}Faire tourner (la quenouille, pour filer le fil de lin).}  ¶ \textcolor{darkblue}{\textbf{\ipa{tso˧\textasciitilde{}tso˧ bo˧}}} \zh{使东西旋转} \textcolor{Sepia}{\selectlanguage{english}to spin something, to spin things} \textcolor{PineGreen}{\selectlanguage{french}faire tourner quelque chose}  
\zh{~【参考】~} \hyperlink{}{\textcolor{darkblue}{\textbf{\ipa{sɑ˧bo\#˥}}}} 
\lhead{\firstmark}
\rhead{\botmark}

\subsection{\hspace{-0.5cm} {\Large \textcolor{darkblue}{\textbf{\ipa{bo˧tsi˩}}}}\hspace{0.5cm}[\kern2pt{\textcolor{darkblue}{\textbf{\ipa{bo˩tsi˥}}}}\kern2pt]} \hypertarget{bo\string_Mtsi\string_B1}{}
\markboth{\textcolor{darkblue}{\textbf{\ipa{bo˧tsi˩}}}}{}
\textcolor{teal}{\zh{名词}} \hspace{4pt} \zh{声调类:} L\#.
\zh{(马)鬃。} \textcolor{Sepia}{\selectlanguage{english}Mane.} \textcolor{PineGreen}{\selectlanguage{french}Crinière.}  ¶ \textcolor{darkblue}{\textbf{\ipa{ʐwæ˧-bo˧tsi˥\#}}} \zh{马鬃} \textcolor{Sepia}{\selectlanguage{english}horse mane} \textcolor{PineGreen}{\selectlanguage{french}crinière du cheval}  
 ¶ \textcolor{darkblue}{\textbf{\ipa{bo˩-bo˧tsi˩}}} \zh{猪鬃} \textcolor{Sepia}{\selectlanguage{english}hog bristle} \textcolor{PineGreen}{\selectlanguage{french}soies de porc ou de sanglier}  
 \zh{量词}: \textcolor{darkblue}{\textbf{\ipa{kʰwɤ˥}}} \textcolor{darkblue}{\textbf{\ipa{qɑ˩}}} 
\lhead{\firstmark}
\rhead{\botmark}

\subsection{\hspace{-0.5cm} {\Large \textcolor{darkblue}{\textbf{\ipa{bo˧ʐæ˧}}}}\hspace{0.5cm}[\kern2pt{\textcolor{darkblue}{\textbf{\ipa{bo˩ʐæ˥}}}}\kern2pt]} \hypertarget{bo\string_Mz`\{\string_M1}{}
\markboth{\textcolor{darkblue}{\textbf{\ipa{bo˧ʐæ˧}}}}{}
\textcolor{teal}{\zh{名词}} \hspace{4pt} \zh{声调类:} M.
\zh{玻璃。} \textcolor{Sepia}{\selectlanguage{english}Glass (as a substance: window panes, drinking glasses... are made of glass).} \textcolor{PineGreen}{\selectlanguage{french}Verre (matière).}  ¶ \textcolor{darkblue}{\textbf{\ipa{bo˧ʐæ˧-tɕʰɤ˩ʈʂv˩}}} \zh{玻璃茶杯} \textcolor{Sepia}{\selectlanguage{english}goblet for drinking tea (made of glass)} \textcolor{PineGreen}{\selectlanguage{french}gobelet à thé en verre}  

\lhead{\firstmark}
\rhead{\botmark}

\subsection{\hspace{-0.5cm} {\Large \textcolor{darkblue}{\textbf{\ipa{bo˧ʐæ˧ʈæ˧qʰv̩\#˥}}}}\hspace{0.5cm}[\kern2pt{\textcolor{darkblue}{\textbf{\ipa{bo˩ʐæ˧ʈæ˧qʰv̩˧}}}}\kern2pt]} \hypertarget{bo\string_Mz`\{\string_Mt`\{\string_Mq\string_hv\string_=\#\string_T1}{}
\markboth{\textcolor{darkblue}{\textbf{\ipa{bo˧ʐæ˧ʈæ˧qʰv̩\#˥}}}}{}
\textcolor{teal}{\zh{名词}} \hspace{4pt} \zh{声调类:} \#H.
\zh{回音。} \textcolor{Sepia}{\selectlanguage{english}Echo (in some places in the mountains, there is an echo).} \textcolor{PineGreen}{\selectlanguage{french}Écho (en certains endroits des montagnes, il y a un écho).}  ¶ \textcolor{darkblue}{\textbf{\ipa{bo˧ʐæ˧ʈæ˧qʰv̩˧ | tʰi˧-ɖʐɯ˩\textasciitilde{}ɖʐɯ˩!}}} \zh{有回音} \textcolor{Sepia}{\selectlanguage{english}the echo resonates} \textcolor{PineGreen}{\selectlanguage{french}Il y a de l'écho!}  

\lhead{\firstmark}
\rhead{\botmark}

\subsection{\hspace{-0.5cm} {\Large \textcolor{darkblue}{\textbf{\ipa{bo˩\textsubscript{a}}}}}\hspace{0.5cm}[\kern2pt{\textcolor{darkblue}{\textbf{\ipa{bo˥}}}}\kern2pt]} \hypertarget{bo\string_Ba1}{}
\markboth{\textcolor{darkblue}{\textbf{\ipa{bo˩\textsubscript{a}}}}}{}
\textcolor{teal}{\zh{动词}} \hspace{4pt} \zh{声调类:} L\textsubscript{a}.
\zh{亲吻。} \textcolor{Sepia}{\selectlanguage{english}To kiss.} \textcolor{PineGreen}{\selectlanguage{french}Embrasser.}  ¶ \textcolor{darkblue}{\textbf{\ipa{ɖɯ˧-bo˩-ɻ̍˩}}} \zh{吻一下} \textcolor{Sepia}{\selectlanguage{english}\mytextsc{delimitative} \string_ \mytextsc{inceptive}} \textcolor{PineGreen}{\selectlanguage{french}\mytextsc{délimitatif} \string_ \mytextsc{inchoatif}}  
 ¶ \textcolor{darkblue}{\textbf{\ipa{ɖɯ˧-bo˧\textasciitilde{}bo˥-ɻ̍˩}}} \zh{吻一下} \textcolor{Sepia}{\selectlanguage{english}\mytextsc{delimitative} \string_ \mytextsc{red} \mytextsc{inceptive}} \textcolor{PineGreen}{\selectlanguage{french}\mytextsc{délimitatif} \string_ \mytextsc{red} \mytextsc{inchoatif}}  

\lhead{\firstmark}
\rhead{\botmark}

\subsection{\hspace{-0.5cm} {\Large \textcolor{darkblue}{\textbf{\ipa{bo˩\textsubscript{b}}}}}\hspace{0.5cm}[\kern2pt{\textcolor{darkblue}{\textbf{\ipa{bo˧˥}}}}\kern2pt]} \hypertarget{bo\string_Bb1}{}
\markboth{\textcolor{darkblue}{\textbf{\ipa{bo˩\textsubscript{b}}}}}{}
\textcolor{teal}{\zh{量词}} \hspace{4pt} \zh{声调类:} L\textsubscript{b}.
\zh{量词:缎子发带(一条)。} \textcolor{Sepia}{\selectlanguage{english}Classifier for women's traditional hair dresses / headdresses.} \textcolor{PineGreen}{\selectlanguage{french}Classificateur des coiffes (parures pour la chevelure des femmes).} 
\lhead{\firstmark}
\rhead{\botmark}

\subsection{\hspace{-0.5cm} {\Large \textcolor{darkblue}{\textbf{\ipa{bo˩-bi˧mi˧}}}}\hspace{0.5cm}[\kern2pt{\textcolor{darkblue}{\textbf{\ipa{xxxx non-correspondance entre le nombre de morphèmes et le nombre de tons de morphèmes}}}}\kern2pt]} \hypertarget{bo\string_B-bi\string_Mmi\string_M1}{}
\markboth{\textcolor{darkblue}{\textbf{\ipa{bo˩-bi˧mi˧}}}}{}
\textcolor{teal}{\zh{名词}} \hspace{4pt} \zh{声调类:} L-.
\zh{猪肚肉。} \textcolor{Sepia}{\selectlanguage{english}Meat of the pig's belly.} \textcolor{PineGreen}{\selectlanguage{french}Viande du ventre du cochon.}  \zh{量词}: \textcolor{darkblue}{\textbf{\ipa{ɭɯ˧}}} 
\lhead{\firstmark}
\rhead{\botmark}

\subsection{\hspace{-0.5cm} {\Large \textcolor{darkblue}{\textbf{\ipa{bo˩-bv̩˥}}}}\hspace{0.5cm}[\kern2pt{\textcolor{darkblue}{\textbf{\ipa{bo˧bv̩˧}}}}\kern2pt]} \hypertarget{bo\string_B-bv\string_=\string_T1}{}
\markboth{\textcolor{darkblue}{\textbf{\ipa{bo˩-bv̩˥}}}}{}
\textcolor{teal}{\zh{名词}} \hspace{4pt} \zh{声调类:} LH.
\zh{猪圈。} \textcolor{Sepia}{\selectlanguage{english}Pigsty, pigpen.} \textcolor{PineGreen}{\selectlanguage{french}Enclos des porcs.}  \zh{量词}: \textcolor{darkblue}{\textbf{\ipa{ɭɯ˧}}} 
\lhead{\firstmark}
\rhead{\botmark}

\subsection{\hspace{-0.5cm} {\Large \textcolor{darkblue}{\textbf{\ipa{bo˩dze˧}}}}\hspace{0.5cm}[\kern2pt{\textcolor{darkblue}{\textbf{\ipa{bo˩dze˥}}}}\kern2pt]} \hypertarget{bo\string_Bdze\string_M1}{}
\markboth{\textcolor{darkblue}{\textbf{\ipa{bo˩dze˧}}}}{}
\textcolor{teal}{\zh{名词}} \hspace{4pt} \zh{声调类:} LM.
\zh{百灵鸟。} \textcolor{Sepia}{\selectlanguage{english}Lark.} \textcolor{PineGreen}{\selectlanguage{french}Alouette.} \zh{~【参考】~} \hyperlink{}{\textcolor{darkblue}{\textbf{\ipa{bo˩dze˧-ko˩dze˩}}}} 
\lhead{\firstmark}
\rhead{\botmark}

\subsection{\hspace{-0.5cm} {\Large \textcolor{darkblue}{\textbf{\ipa{bo˩dze˧-ko˩dze˩}}}}\hspace{0.5cm}[\kern2pt{\textcolor{darkblue}{\textbf{\ipa{xxxx non-correspondance entre le nombre de morphèmes et le nombre de tons de morphèmes}}}}\kern2pt]} \hypertarget{bo\string_Bdze\string_M-ko\string_Bdze\string_B1}{}
\markboth{\textcolor{darkblue}{\textbf{\ipa{bo˩dze˧-ko˩dze˩}}}}{}
\textcolor{teal}{\zh{名词}} \hspace{4pt} \zh{声调类:} LM-L.
\zh{百灵鸟。} \textcolor{Sepia}{\selectlanguage{english}Lark.} \textcolor{PineGreen}{\selectlanguage{french}Alouette.} \zh{~【参考】~} \hyperlink{}{\textcolor{darkblue}{\textbf{\ipa{bo˩dze˧}}}} 
\lhead{\firstmark}
\rhead{\botmark}

\subsection{\hspace{-0.5cm} {\Large \textcolor{darkblue}{\textbf{\ipa{bo˩-ɣɯ˥}}}}\hspace{0.5cm}[\kern2pt{\textcolor{darkblue}{\textbf{\ipa{xxxx non-correspondance entre le nombre de morphèmes et le nombre de tons de morphèmes}}}}\kern2pt]} \hypertarget{bo\string_B-GM\string_T1}{}
\markboth{\textcolor{darkblue}{\textbf{\ipa{bo˩-ɣɯ˥}}}}{}
\textcolor{teal}{\zh{名词}} \hspace{4pt} \zh{声调类:} LH.
\zh{猪皮。} \textcolor{Sepia}{\selectlanguage{english}Pigskin, hogskin.} \textcolor{PineGreen}{\selectlanguage{french}Couenne.}  ¶ \textcolor{darkblue}{\textbf{\ipa{bo˩-ɣɯ˧kɯ˩}}} \zh{同上:猪皮} \textcolor{Sepia}{\selectlanguage{english}same meaning: pigskin} \textcolor{PineGreen}{\selectlanguage{french}même sens: couenne}  
 \zh{量词}: \textcolor{darkblue}{\textbf{\ipa{tsʰi˥}}} 
\lhead{\firstmark}
\rhead{\botmark}

\subsection{\hspace{-0.5cm} {\Large \textcolor{darkblue}{\textbf{\ipa{bo˩-hɑ\#˥}}}}\hspace{0.5cm}[\kern2pt{\textcolor{darkblue}{\textbf{\ipa{xxxx non-correspondance entre le nombre de morphèmes et le nombre de tons de morphèmes}}}}\kern2pt]} \hypertarget{bo\string_B-hA\#\string_T1}{}
\markboth{\textcolor{darkblue}{\textbf{\ipa{bo˩-hɑ\#˥}}}}{}
\textcolor{teal}{\zh{名词}} \hspace{4pt} \zh{声调类:} LM+\#H.
\zh{猪食。} \textcolor{Sepia}{\selectlanguage{english}Pig feed, swill.} \textcolor{PineGreen}{\selectlanguage{french}Nourriture du cochon/ pâtée des porcs.}  \zh{量词}: \textcolor{darkblue}{\textbf{\ipa{kʰɤ˧˥}}} 
\lhead{\firstmark}
\rhead{\botmark}

\subsection{\hspace{-0.5cm} {\Large \textcolor{darkblue}{\textbf{\ipa{bo˩-kʰɯ˧}}}}\hspace{0.5cm}[\kern2pt{\textcolor{darkblue}{\textbf{\ipa{xxxx non-correspondance entre le nombre de morphèmes et le nombre de tons de morphèmes}}}}\kern2pt]} \hypertarget{bo\string_B-k\string_hM\string_M1}{}
\markboth{\textcolor{darkblue}{\textbf{\ipa{bo˩-kʰɯ˧}}}}{}
\textcolor{teal}{\zh{名词}} \hspace{4pt} \zh{声调类:} LM.
\zh{猪脚腊肉:把猪腿的皮剥下来,缝成筒形,塞满瘦肉。} \textcolor{Sepia}{\selectlanguage{english}Pig's feet: dried meat preserved in the skin of the pig's foot.} \textcolor{PineGreen}{\selectlanguage{french}Pieds de porc (pièce de boucherie): viande séchée conservée dans la peau du pied de cochon.}  \zh{量词}: \textcolor{darkblue}{\textbf{\ipa{ɭɯ˧}}} 
\lhead{\firstmark}
\rhead{\botmark}

\subsection{\hspace{-0.5cm} {\Large \textcolor{darkblue}{\textbf{\ipa{bo˩-kʰv̩˧˥}}}}\hspace{0.5cm}[\kern2pt{\textcolor{darkblue}{\textbf{\ipa{xxxx non-correspondance entre le nombre de morphèmes et le nombre de tons de morphèmes}}}}\kern2pt]} \hypertarget{bo\string_B-k\string_hv\string_=\string_M\string_T1}{}
\markboth{\textcolor{darkblue}{\textbf{\ipa{bo˩-kʰv̩˧˥}}}}{}
\textcolor{teal}{\zh{名词}} \hspace{4pt} \zh{声调类:} LM+MH\#.
\zh{猪年。} \textcolor{Sepia}{\selectlanguage{english}Year of the pig.} \textcolor{PineGreen}{\selectlanguage{french}Année du porc.} 
\lhead{\firstmark}
\rhead{\botmark}

\subsection{\hspace{-0.5cm} {\Large \textcolor{darkblue}{\textbf{\ipa{bo˩lo˧}}}}\hspace{0.5cm}[\kern2pt{\textcolor{darkblue}{\textbf{\ipa{bo˩lo˥}}}}\kern2pt]} \hypertarget{bo\string_Blo\string_M1}{}
\markboth{\textcolor{darkblue}{\textbf{\ipa{bo˩lo˧}}}}{}
\textcolor{teal}{\zh{名词}} \hspace{4pt} \zh{声调类:} LM.
\zh{榫眼。} \textcolor{Sepia}{\selectlanguage{english}Mortise.} \textcolor{PineGreen}{\selectlanguage{french}Mortaise.}  ¶ \textcolor{darkblue}{\textbf{\ipa{bo˩lo˧ | ɖɯ˧-ɭɯ˧}}} \zh{一个榫眼} \textcolor{Sepia}{\selectlanguage{english}a mortise} \textcolor{PineGreen}{\selectlanguage{french}une mortaise}  
 \zh{量词}: \textcolor{darkblue}{\textbf{\ipa{ɭɯ˧}}} 
\lhead{\firstmark}
\rhead{\botmark}

\subsection{\hspace{-0.5cm} {\Large \textcolor{darkblue}{\textbf{\ipa{bo˩ɬɑ˥}}}}\hspace{0.5cm}[\kern2pt{\textcolor{darkblue}{\textbf{\ipa{bo˩ɬɑ˧˥}}}}\kern2pt]} \hypertarget{bo\string_BKA\string_T1}{}
\markboth{\textcolor{darkblue}{\textbf{\ipa{bo˩ɬɑ˥}}}}{}
\textcolor{teal}{\zh{名词}} \hspace{4pt} \zh{声调类:} LH.
\zh{种公猪。} \textcolor{Sepia}{\selectlanguage{english}Boar.} \textcolor{PineGreen}{\selectlanguage{french}Verrat.}  \zh{量词}: \textcolor{darkblue}{\textbf{\ipa{v̩˧}}} 
\lhead{\firstmark}
\rhead{\botmark}

\subsection{\hspace{-0.5cm} {\Large \textcolor{darkblue}{\textbf{\ipa{bo˩-ɬo˥}}}}\hspace{0.5cm}[\kern2pt{\textcolor{darkblue}{\textbf{\ipa{xxxx non-correspondance entre le nombre de morphèmes et le nombre de tons de morphèmes}}}}\kern2pt]} \hypertarget{bo\string_B-Ko\string_T1}{}
\markboth{\textcolor{darkblue}{\textbf{\ipa{bo˩-ɬo˥}}}}{}
\textcolor{teal}{\zh{名词}} \hspace{4pt} \zh{声调类:} LH.
\zh{猪肋骨。} \textcolor{Sepia}{\selectlanguage{english}Pork ribs.} \textcolor{PineGreen}{\selectlanguage{french}Côtes de porc.}  ¶ \textcolor{darkblue}{\textbf{\ipa{bo˩ɬo˥ | ɖɯ˧-do˥}}} \zh{一大块猪肋骨。也来比喻一个家庭,强调家所有成员之间的密切关系。} \textcolor{Sepia}{\selectlanguage{english}a large piece of pork ribs; this also designates (metaphorically) a family, to emphasize the strong bonds between all family members} \textcolor{PineGreen}{\selectlanguage{french}Un quartier de côtes de porc; désigne aussi, métaphoriquement, une famille, dans laquelle chaque individu est solidaire des autres}  
 \zh{量词}: \textcolor{darkblue}{\textbf{\ipa{ɭɯ˧}}} 
\lhead{\firstmark}
\rhead{\botmark}

\subsection{\hspace{-0.5cm} {\Large \textcolor{darkblue}{\textbf{\ipa{bo˩-mæ˧qv̩˩}}}}\hspace{0.5cm}[\kern2pt{\textcolor{darkblue}{\textbf{\ipa{xxxx non-correspondance entre le nombre de morphèmes et le nombre de tons de morphèmes}}}}\kern2pt]} \hypertarget{bo\string_B-m\{\string_Mqv\string_=\string_B1}{}
\markboth{\textcolor{darkblue}{\textbf{\ipa{bo˩-mæ˧qv̩˩}}}}{}
\textcolor{teal}{\zh{名词}} \hspace{4pt} \zh{声调类:} L-L\#.
\zh{猪尾巴。} \textcolor{Sepia}{\selectlanguage{english}Pig's tail.} \textcolor{PineGreen}{\selectlanguage{french}Queue du cochon.}  \zh{量词}: \textcolor{darkblue}{\textbf{\ipa{ɭɯ˧}}} 
\lhead{\firstmark}
\rhead{\botmark}

\subsection{\hspace{-0.5cm} {\Large \textcolor{darkblue}{\textbf{\ipa{bo˩-mɤ˥}}}}\hspace{0.5cm}[\kern2pt{\textcolor{darkblue}{\textbf{\ipa{xxxx non-correspondance entre le nombre de morphèmes et le nombre de tons de morphèmes}}}}\kern2pt]} \hypertarget{bo\string_B-m7\string_T1}{}
\markboth{\textcolor{darkblue}{\textbf{\ipa{bo˩-mɤ˥}}}}{}
\textcolor{teal}{\zh{名词}} \hspace{4pt} \zh{声调类:} LH.
\zh{猪油。} \textcolor{Sepia}{\selectlanguage{english}Lard.} \textcolor{PineGreen}{\selectlanguage{french}Saindoux (gras de porc).} 
\lhead{\firstmark}
\rhead{\botmark}

\subsection{\hspace{-0.5cm} {\Large \textcolor{darkblue}{\textbf{\ipa{bo˩mi˧}}}}\hspace{0.5cm}[\kern2pt{\textcolor{darkblue}{\textbf{\ipa{xxxx non-correspondance entre le nombre de morphèmes et le nombre de tons de morphèmes}}}}\kern2pt]} \hypertarget{bo\string_Bmi\string_M1}{}
\markboth{\textcolor{darkblue}{\textbf{\ipa{bo˩mi˧}}}}{}
\textcolor{teal}{\zh{名词}} \hspace{4pt} \zh{声调类:} LM.
\zh{母猪。} \textcolor{Sepia}{\selectlanguage{english}Sow.} \textcolor{PineGreen}{\selectlanguage{french}Truie.}  ¶ \textcolor{darkblue}{\textbf{\ipa{bo˩mi˧ ʑi˩}}} \zh{抓母猪} \textcolor{Sepia}{\selectlanguage{english}to catch (a/the) sow} \textcolor{PineGreen}{\selectlanguage{french}attraper une truie}  
 ¶ \textcolor{darkblue}{\textbf{\ipa{bo˩mi˧ do˧ (+ze˩)}}} \zh{见了母猪} \textcolor{Sepia}{\selectlanguage{english}...has seen (a/the) sow} \textcolor{PineGreen}{\selectlanguage{french}...a vu (une/la) truie}  
 ¶ \textcolor{darkblue}{\textbf{\ipa{bo˩mi˧-bæ˧bv̩˥}}} \zh{母猪与猪仔} \textcolor{Sepia}{\selectlanguage{english}sow and piglets} \textcolor{PineGreen}{\selectlanguage{french}truie et porcelets}  
 \zh{量词}: \textcolor{darkblue}{\textbf{\ipa{mi˩}}} \textcolor{darkblue}{\textbf{\ipa{v̩˧}}} 
\lhead{\firstmark}
\rhead{\botmark}

\subsection{\hspace{-0.5cm} {\Large \textcolor{darkblue}{\textbf{\ipa{bo˩mi˧-dʑɯ˩pv̩˩}}}}\hspace{0.5cm}[\kern2pt{\textcolor{darkblue}{\textbf{\ipa{xxxx non-correspondance entre le nombre de morphèmes et le nombre de tons de morphèmes}}}}\kern2pt]} \hypertarget{bo\string_Bmi\string_M-dz£M\string_Bpv\string_=\string_B1}{}
\markboth{\textcolor{darkblue}{\textbf{\ipa{bo˩mi˧-dʑɯ˩pv̩˩}}}}{}
\textcolor{teal}{\zh{名词}} \hspace{4pt} \zh{声调类:} LM-L.
\zh{龙虱。} \textcolor{Sepia}{\selectlanguage{english}\textit{(Dytiscus}, a predaceous diving beetle.} \textcolor{PineGreen}{\selectlanguage{french}Dytique, \textit{(Dytiscus}.} 
\lhead{\firstmark}
\rhead{\botmark}

\subsection{\hspace{-0.5cm} {\Large \textcolor{darkblue}{\textbf{\ipa{bo˩mi˧-dʑɯ˩pʰv̩˩}}}}\hspace{0.5cm}[\kern2pt{\textcolor{darkblue}{\textbf{\ipa{bo˩mi˧dʑɯ˩pʰv̩˩}}}}\kern2pt]} \hypertarget{bo\string_Bmi\string_M-dz£M\string_Bp\string_hv\string_=\string_B1}{}
\markboth{\textcolor{darkblue}{\textbf{\ipa{bo˩mi˧-dʑɯ˩pʰv̩˩}}}}{}
\textcolor{teal}{\zh{名词}} \hspace{4pt} \zh{声调类:} LM-L.
\zh{象鼻虫,米象。} \textcolor{Sepia}{\selectlanguage{english}Weevil, snout beetle.} \textcolor{PineGreen}{\selectlanguage{french}Charançon.} 
\lhead{\firstmark}
\rhead{\botmark}

\subsection{\hspace{-0.5cm} {\Large \textcolor{darkblue}{\textbf{\ipa{bo˩mi˧-ɳæ˧tɕʰɯ˩}}}}\hspace{0.5cm}[\kern2pt{\textcolor{darkblue}{\textbf{\ipa{bo˩mi˧ɳæ˩tɕʰɯ˩}}}}\kern2pt]} \hypertarget{bo\string_Bmi\string_M-n`\{\string_Mts£\string_hM\string_B1}{}
\markboth{\textcolor{darkblue}{\textbf{\ipa{bo˩mi˧-ɳæ˧tɕʰɯ˩}}}}{}
\textcolor{teal}{\zh{名词}} \hspace{4pt} \zh{声调类:} LM-L\#.
\zh{蒲公英。} \textcolor{Sepia}{\selectlanguage{english}Dandelion.} \textcolor{PineGreen}{\selectlanguage{french}Pissenlit.}  \zh{量词}: \textcolor{darkblue}{\textbf{\ipa{po˧}}} 
\lhead{\firstmark}
\rhead{\botmark}

\subsection{\hspace{-0.5cm} {\Large \textcolor{darkblue}{\textbf{\ipa{bo˩mi˧-ʁo˩do˩}}}}\hspace{0.5cm}[\kern2pt{\textcolor{darkblue}{\textbf{\ipa{bo˩mi˧ʁo˧do˩}}}}\kern2pt]} \hypertarget{bo\string_Bmi\string_M-Ro\string_Bdo\string_B1}{}
\markboth{\textcolor{darkblue}{\textbf{\ipa{bo˩mi˧-ʁo˩do˩}}}}{}
\textcolor{teal}{\zh{名词}} \hspace{4pt} \zh{声调类:} LM-L.
\zh{曼陀罗。} \textcolor{Sepia}{\selectlanguage{english}Yyyy.} \textcolor{PineGreen}{\selectlanguage{french}Yyyy littéralement “noix des truies”; était employé autrefois pour application sur les plaies qui suppuraient: on en extrayait le jus.}  \zh{量词}: \textcolor{darkblue}{\textbf{\ipa{dzi˩}}} 
\lhead{\firstmark}
\rhead{\botmark}

\subsection{\hspace{-0.5cm} {\Large \textcolor{darkblue}{\textbf{\ipa{bo˩pʰv̩˧}}}}\hspace{0.5cm}[\kern2pt{\textcolor{darkblue}{\textbf{\ipa{bo˩pʰv̩˥}}}}\kern2pt]} \hypertarget{bo\string_Bp\string_hv\string_=\string_M1}{}
\markboth{\textcolor{darkblue}{\textbf{\ipa{bo˩pʰv̩˧}}}}{}
\textcolor{teal}{\zh{名词}} \hspace{4pt} \zh{声调类:} LM.
\zh{种公猪。} \textcolor{Sepia}{\selectlanguage{english}Boar.} \textcolor{PineGreen}{\selectlanguage{french}Verrat.}  \zh{量词}: \textcolor{darkblue}{\textbf{\ipa{v̩˧}}} 
\lhead{\firstmark}
\rhead{\botmark}

\subsection{\hspace{-0.5cm} {\Large \textcolor{darkblue}{\textbf{\ipa{bo˩qʰæ˧-pv̩˧ʈɤ˥-ɻ̍˩}}}}\hspace{0.5cm}[\kern2pt{\textcolor{darkblue}{\textbf{\ipa{xxxx non-correspondance entre le nombre de morphèmes et le nombre de tons de morphèmes}}}}\kern2pt]} \hypertarget{bo\string_Bq\string_h\{\string_M-pv\string_=\string_Mt`7\string_T-r£`̍\string_B1}{}
\markboth{\textcolor{darkblue}{\textbf{\ipa{bo˩qʰæ˧-pv̩˧ʈɤ˥-ɻ̍˩}}}}{}
\textcolor{teal}{\zh{名词}} \hspace{4pt} \zh{声调类:} LM - H\# -.
\zh{蜣螂。} \textcolor{Sepia}{\selectlanguage{english}Dung beetle.} \textcolor{PineGreen}{\selectlanguage{french}Bousier: sorte de scarabée, qui prolifère dans les étables lorsqu'il fait chaud.} 
\lhead{\firstmark}
\rhead{\botmark}

\subsection{\hspace{-0.5cm} {\Large \textcolor{darkblue}{\textbf{\ipa{bo˩tv̩\#˥}}}}\hspace{0.5cm}[\kern2pt{\textcolor{darkblue}{\textbf{\ipa{bo˧tv̩˩}}}}\kern2pt]} \hypertarget{bo\string_Btv\string_=\#\string_T1}{}
\markboth{\textcolor{darkblue}{\textbf{\ipa{bo˩tv̩\#˥}}}}{}
\textcolor{teal}{\zh{名词}} \hspace{4pt} \zh{声调类:} LM+\#H.
\zh{野猪。} \textcolor{Sepia}{\selectlanguage{english}Wild boar.} \textcolor{PineGreen}{\selectlanguage{french}Sanglier.}  ¶ \textcolor{darkblue}{\textbf{\ipa{bo˩tv̩˧ hwæ˥}}} \zh{买野猪} \textcolor{Sepia}{\selectlanguage{english}to buy (a/the) wild boar} \textcolor{PineGreen}{\selectlanguage{french}acheter un sanglier}  
 \zh{量词}: \textcolor{darkblue}{\textbf{\ipa{mi˩}}} 
\lhead{\firstmark}
\rhead{\botmark}

\subsection{\hspace{-0.5cm} {\Large \textcolor{darkblue}{\textbf{\ipa{bo˩ʈʂʰæ˥}}}}\hspace{0.5cm}[\kern2pt{\textcolor{darkblue}{\textbf{\ipa{bo˩ʈʂʰæ˩˥}}}}\kern2pt]} \hypertarget{bo\string_Bt`s`\string_h\{\string_T1}{}
\markboth{\textcolor{darkblue}{\textbf{\ipa{bo˩ʈʂʰæ˥}}}}{}
\textcolor{teal}{\zh{名词}} \hspace{4pt} \zh{声调类:} LH.
\zh{猪膘,琵琶肉。} \textcolor{Sepia}{\selectlanguage{english}Lard, fat meat of pig; also: boneless, fleshless preserved pork: cured pork made from a whole pig by removing all its internal organs from the opened stomach, seasoned with salt and spices and then the opening is stitched together. The whole sewn pig is then pressed with a slabstone.} \textcolor{PineGreen}{\selectlanguage{french}Lard; le même terme est employé pour désigner le cochon entier désossé et conservé dans sa peau (au moyen de salpêtre et de sel), qui se conserve une décennie, appelé “viande pipa” en chinois. La glose adoptée dans les textes est: cochon-conservé-entier.} 
\lhead{\firstmark}
\rhead{\botmark}

\subsection{\hspace{-0.5cm} {\Large \textcolor{darkblue}{\textbf{\ipa{bo˩zɑ˧mi\#˥}}}}\hspace{0.5cm}[\kern2pt{\textcolor{darkblue}{\textbf{\ipa{bo˧zɑ˧mi˧}}}}\kern2pt]} \hypertarget{bo\string_BzA\string_Mmi\#\string_T1}{}
\markboth{\textcolor{darkblue}{\textbf{\ipa{bo˩zɑ˧mi\#˥}}}}{}
\textcolor{teal}{\zh{名词}} \hspace{4pt} \zh{声调类:} LM+\#H.
\zh{猪崽子(给刚出生的女孩起的名字,让鬼对她不感兴趣,不会来害小孩)。} \textcolor{Sepia}{\selectlanguage{english}“Piggy-Sow”: a term used as a temporary name for little girls, during the first months of their life, before they are given a real name. This ugly term is intended to disgust evil spirits, which will therefore turn their attention away from the infant. (In the early 21st century, the registry office requires a name to be given at birth; but this name only begins to be used by the family after the first months of life have elapsed.).} \textcolor{PineGreen}{\selectlanguage{french}“Petite Truie”: nom employé pour les petites filles pendant leurs premiers mois, avant qu'on leur donne un vrai nom. Le vilain nom dont on l'affuble vise à éviter que le nourrisson ne soit repéré par de mauvais esprits. (Actuellement, l'état-civil nécessite qu'un nom soit donné dès la naissance; mais celui-ci ne commence à être employé dans les conversations familiales qu'une fois passés les premiers mois.).} 
\lhead{\firstmark}
\rhead{\botmark}

\subsection{\hspace{-0.5cm} {\Large \textcolor{darkblue}{\textbf{\ipa{bo˩˧}}}}\hspace{0.5cm}[\kern2pt{\textcolor{darkblue}{\textbf{\ipa{xxxx groupe tonal entier sans aucun ton}}}}\kern2pt]} \hypertarget{bo\string_B\string_M1}{}
\markboth{\textcolor{darkblue}{\textbf{\ipa{bo˩˧}}}}{}
\textcolor{teal}{\zh{名词}} \hspace{4pt} \zh{声调类:} LM.
\zh{猪。} \textcolor{Sepia}{\selectlanguage{english}Pig.} \textcolor{PineGreen}{\selectlanguage{french}Porc, cochon.}  ¶ \textcolor{darkblue}{\textbf{\ipa{bo˩ hwæ˧-ze˧}}} \zh{买了猪} \textcolor{Sepia}{\selectlanguage{english}...bought (some/a) pig} \textcolor{PineGreen}{\selectlanguage{french}...a acheté (du/un) porc}  
 ¶ \textcolor{darkblue}{\textbf{\ipa{bo˩ dzɯ˥-ze˩}}} \zh{吃了猪} \textcolor{Sepia}{\selectlanguage{english}...ate (some/the) pig} \textcolor{PineGreen}{\selectlanguage{french}...a mangé (du/un) porc}  
 \zh{量词}: \textcolor{darkblue}{\textbf{\ipa{mi˩}}} \textcolor{darkblue}{\textbf{\ipa{v̩˧}}} 
\lhead{\firstmark}
\rhead{\botmark}

\subsection{\hspace{-0.5cm} {\Large \textcolor{darkblue}{\textbf{\ipa{bõ}}}}\hspace{0.5cm}[\kern2pt{\textcolor{darkblue}{\textbf{\ipa{xxxx groupe tonal entier sans aucun ton}}}}\kern2pt]} \hypertarget{bo\string_~1}{}
\markboth{\textcolor{darkblue}{\textbf{\ipa{bõ}}}}{}
\textcolor{teal}{\zh{状貌词}} \hspace{4pt} \zh{声调类:} 0.
\zh{形声词:斧头砍树。砰! / 啪!。} \textcolor{Sepia}{\selectlanguage{english}Noise of a shock between two hard objects, for instance the sound of an axe hitting a tree trunk: Bang!} \textcolor{PineGreen}{\selectlanguage{french}Bruit d'un choc entre deux objets durs, par exemple un coup de hache sur un tronc: Boum!} 
\lhead{\firstmark}
\rhead{\botmark}

\subsection{\hspace{-0.5cm} {\Large \textcolor{darkblue}{\textbf{\ipa{bv̩˩}}}}\hspace{0.5cm}[\kern2pt{\textcolor{darkblue}{\textbf{\ipa{bv̩˥}}}}\kern2pt]} \hypertarget{bv\string_=\string_B1}{}
\markboth{\textcolor{darkblue}{\textbf{\ipa{bv̩˩}}}}{}
\textcolor{teal}{\zh{名词}} \hspace{4pt} \zh{声调类:} L.
\zh{牲畜圈(单音节)。} \textcolor{Sepia}{\selectlanguage{english}Pen, corral for cattle.} \textcolor{PineGreen}{\selectlanguage{french}Enclos (monosyllabe).}  ¶ \textcolor{darkblue}{\textbf{\ipa{bv̩˩-qo˩}}} \zh{牲畜圈里面} \textcolor{Sepia}{\selectlanguage{english}inside the corral} \textcolor{PineGreen}{\selectlanguage{french}dans l'enclos}  
 ¶ \textcolor{darkblue}{\textbf{\ipa{bv̩˩qo˩ ʈæ˧}}} \zh{关在圈里} \textcolor{Sepia}{\selectlanguage{english}to enclose (cattle) inside the pen} \textcolor{PineGreen}{\selectlanguage{french}enfermer dans l'étable}  
 \zh{量词}: \textcolor{darkblue}{\textbf{\ipa{ɭɯ˧}}} 
\lhead{\firstmark}
\rhead{\botmark}

\subsection{\hspace{-0.5cm} {\Large \textcolor{darkblue}{\textbf{\ipa{bv̩˩˧}}} \textsubscript{1}}\hspace{0.5cm}[\kern2pt{\textcolor{darkblue}{\textbf{\ipa{bv̩˥}}}}\kern2pt]} \hypertarget{bv\string_=\string_B\string_M1}{}
\markboth{\textcolor{darkblue}{\textbf{\ipa{bv̩˩˧}}} \textsubscript{1}}{}
\textcolor{teal}{\zh{名词}} \hspace{4pt} \zh{声调类:} LM.
\zh{牦牛/野牦牛。} \textcolor{Sepia}{\selectlanguage{english}Yak, Bos grunniens. The same term is used for wild yaks and domesticated yaks.} \textcolor{PineGreen}{\selectlanguage{french}Yak, Bos grunniens (sauvage ou domestiqué).}  ¶ \textcolor{darkblue}{\textbf{\ipa{bv̩˩-hṽ˩˥}}} \zh{牦牛毛} \textcolor{Sepia}{\selectlanguage{english}yak hair} \textcolor{PineGreen}{\selectlanguage{french}poil de yak}  
 ¶ \textcolor{darkblue}{\textbf{\ipa{bv̩˩ dzɯ˧-ze˩}}} \zh{吃了牦牛} \textcolor{Sepia}{\selectlanguage{english}...ate (some) yak} \textcolor{PineGreen}{\selectlanguage{french}...a mangé (du) yak}  
 ¶ \textcolor{darkblue}{\textbf{\ipa{bv̩˩ hwæ˧-ze˧}}} \zh{买了牦牛} \textcolor{Sepia}{\selectlanguage{english}...bought (some) yak} \textcolor{PineGreen}{\selectlanguage{french}...a acheté (du) yak}  
 \zh{量词}: \textcolor{darkblue}{\textbf{\ipa{pʰo˧˥}}} 
\lhead{\firstmark}
\rhead{\botmark}

\subsection{\hspace{-0.5cm} {\Large \textcolor{darkblue}{\textbf{\ipa{bv̩˩˧}}} \textsubscript{2}}\hspace{0.5cm}[\kern2pt{\textcolor{darkblue}{\textbf{\ipa{bv̩˩˥}}}}\kern2pt]} \hypertarget{bv\string_=\string_B\string_M2}{}
\markboth{\textcolor{darkblue}{\textbf{\ipa{bv̩˩˧}}} \textsubscript{2}}{}
\textcolor{teal}{\zh{名词}} \hspace{4pt} \zh{声调类:} LM.
\zh{蒸笼。} \textcolor{Sepia}{\selectlanguage{english}Food steamer.} \textcolor{PineGreen}{\selectlanguage{french}Étuve.}  \zh{量词}: \textcolor{darkblue}{\textbf{\ipa{mi˩}}} \zh{~【参考】~} \hyperlink{}{\textcolor{darkblue}{\textbf{\ipa{bv̩˩di˩}}}} 
\lhead{\firstmark}
\rhead{\botmark}

\subsection{\hspace{-0.5cm} {\Large \textcolor{darkblue}{\textbf{\ipa{bv̩˧}}}}\hspace{0.5cm}[\kern2pt{\textcolor{darkblue}{\textbf{\ipa{bv̩˩˥}}}}\kern2pt]} \hypertarget{bv\string_=\string_M1}{}
\markboth{\textcolor{darkblue}{\textbf{\ipa{bv̩˧}}}}{}
\textcolor{teal}{\zh{名词}} \hspace{4pt} \zh{声调类:} M.
\zh{肠子。} \textcolor{Sepia}{\selectlanguage{english}Intestine.} \textcolor{PineGreen}{\selectlanguage{french}Intestin.}  \zh{量词}: \textcolor{darkblue}{\textbf{\ipa{kʰɯ˩}}} 
\lhead{\firstmark}
\rhead{\botmark}

\subsection{\hspace{-0.5cm} {\Large \textcolor{darkblue}{\textbf{\ipa{bv̩˥}}}}\hspace{0.5cm}[\kern2pt{\textcolor{darkblue}{\textbf{\ipa{bv̩˥}}}}\kern2pt]} \hypertarget{bv\string_=\string_T1}{}
\markboth{\textcolor{darkblue}{\textbf{\ipa{bv̩˥}}}}{}
\textcolor{teal}{\zh{名词}} \hspace{4pt} \zh{声调类:} \#H.
\zh{虫。} \textcolor{Sepia}{\selectlanguage{english}Worm; insect.} \textcolor{PineGreen}{\selectlanguage{french}Ver; insecte.}  ¶ \textcolor{darkblue}{\textbf{\ipa{bv̩˧ tʰv̩˧-mi˧˥ / bv̩˧ tʰv̩˧-mi˥\#}}} \zh{这只虫} \textcolor{Sepia}{\selectlanguage{english}\mytextsc{n}+\mytextsc{dem}+\mytextsc{clf}} \textcolor{PineGreen}{\selectlanguage{french}\mytextsc{n}+\mytextsc{dem}+\mytextsc{clf}}  
 \zh{量词}: \textcolor{darkblue}{\textbf{\ipa{mi˩}}} 
\lhead{\firstmark}
\rhead{\botmark}

\subsection{\hspace{-0.5cm} {\Large \textcolor{darkblue}{\textbf{\ipa{bv̩˥}}} \textsubscript{1}}\hspace{0.5cm}[\kern2pt{\textcolor{darkblue}{\textbf{\ipa{bv̩˥}}}}\kern2pt]} \hypertarget{bv\string_=\string_T1}{}
\markboth{\textcolor{darkblue}{\textbf{\ipa{bv̩˥}}} \textsubscript{1}}{}
\textcolor{teal}{\zh{形容词}} \hspace{4pt} \zh{声调类:} H.
\zh{粗(树粗大,粉末不精细……)。} \textcolor{Sepia}{\selectlanguage{english}Thick (tree trunk); coarse (flour, powder).} \textcolor{PineGreen}{\selectlanguage{french}Épais (tronc); grossier (farine, poudre).}  ¶ \textcolor{darkblue}{\textbf{\ipa{qʰɑ˧-bv̩˧-gv̩˧}}} \zh{很粗} \textcolor{Sepia}{\selectlanguage{english}very thick} \textcolor{PineGreen}{\selectlanguage{french}très épais}  
 ¶ \textcolor{darkblue}{\textbf{\ipa{qʰɑ˧bv̩˧\textasciitilde{}bv̩˧-gv̩˧}}} \zh{很粗(同上)} \textcolor{Sepia}{\selectlanguage{english}very thick (as above)} \textcolor{PineGreen}{\selectlanguage{french}très épais (idem ci-dessus)}  

\lhead{\firstmark}
\rhead{\botmark}

\subsection{\hspace{-0.5cm} {\Large \textcolor{darkblue}{\textbf{\ipa{bv̩˥}}} \textsubscript{2}}\hspace{0.5cm}[\kern2pt{\textcolor{darkblue}{\textbf{\ipa{bv̩˥}}}}\kern2pt]} \hypertarget{bv\string_=\string_T2}{}
\markboth{\textcolor{darkblue}{\textbf{\ipa{bv̩˥}}} \textsubscript{2}}{}
\textcolor{teal}{\zh{动词}} \hspace{4pt} \zh{声调类:} H.
\zh{分东西。} \textcolor{Sepia}{\selectlanguage{english}To distribute things, to allot things, to divide things between several persons.} \textcolor{PineGreen}{\selectlanguage{french}Partager, distribuer, répartir, diviser (ancien mot pour “donner”).}  ¶ \textcolor{darkblue}{\textbf{\ipa{ɖɯ˧-v̩˧ ɖɯ˧-kʰwɤ˥ | le˧-bv̩˧\textasciitilde{}bv̩˧}}} \zh{分给一人一块} \textcolor{Sepia}{\selectlanguage{english}to share, giving each person a piece} \textcolor{PineGreen}{\selectlanguage{french}partager, un morceau par personne}  
 ¶ \textcolor{darkblue}{\textbf{\ipa{le˧-bv̩˧\textasciitilde{}bv̩˧ tʰi˧-kwɤ˩}}} \zh{弄乱,散开} \textcolor{Sepia}{\selectlanguage{english}to scatter all over the place, to lay out in no good order} \textcolor{PineGreen}{\selectlanguage{french}littéralement “séparer et poser”; sens: mettre en désordre, disposer en désordre}  

\lhead{\firstmark}
\rhead{\botmark}

\subsection{\hspace{-0.5cm} {\Large \textcolor{darkblue}{\textbf{\ipa{bv̩˥}}} \textsubscript{3}}\hspace{0.5cm}[\kern2pt{\textcolor{darkblue}{\textbf{\ipa{bv̩˥}}}}\kern2pt]} \hypertarget{bv\string_=\string_T3}{}
\markboth{\textcolor{darkblue}{\textbf{\ipa{bv̩˥}}} \textsubscript{3}}{}
\textcolor{teal}{\zh{动词}} \hspace{4pt} \zh{声调类:} H.
\zh{烤,炙。} \textcolor{Sepia}{\selectlanguage{english}To roast, to grill.} \textcolor{PineGreen}{\selectlanguage{french}Griller, faire griller.}  ¶ \textcolor{darkblue}{\textbf{\ipa{hɑ˧ tʰi˧-bv̩˥}}} \zh{烤饭} \textcolor{Sepia}{\selectlanguage{english}to roast food, to roast cereals} \textcolor{PineGreen}{\selectlanguage{french}faire griller du riz}  

\lhead{\firstmark}
\rhead{\botmark}

\subsection{\hspace{-0.5cm} {\Large \textcolor{darkblue}{\textbf{\ipa{bv̩˩\textsubscript{a}}}} \textsubscript{1}}\hspace{0.5cm}[\kern2pt{\textcolor{darkblue}{\textbf{\ipa{bv̩˥}}}}\kern2pt]} \hypertarget{bv\string_=\string_Ba1}{}
\markboth{\textcolor{darkblue}{\textbf{\ipa{bv̩˩\textsubscript{a}}}} \textsubscript{1}}{}
\textcolor{teal}{\zh{动词}} \hspace{4pt} \zh{声调类:} L\textsubscript{a}.
\zh{孵。} \textcolor{Sepia}{\selectlanguage{english}To hatch, to incubate.} \textcolor{PineGreen}{\selectlanguage{french}Couver.}  ¶ \textcolor{darkblue}{\textbf{\ipa{æ˩mi˧ bv̩˩}}} \zh{母鸡孵蛋} \textcolor{Sepia}{\selectlanguage{english}The hen is hatching eggs.} \textcolor{PineGreen}{\selectlanguage{french}La poule couve.}  

\lhead{\firstmark}
\rhead{\botmark}

\subsection{\hspace{-0.5cm} {\Large \textcolor{darkblue}{\textbf{\ipa{bv̩˩\textsubscript{a}}}} \textsubscript{2}}\hspace{0.5cm}[\kern2pt{\textcolor{darkblue}{\textbf{\ipa{bv̩˩˥}}}}\kern2pt]} \hypertarget{bv\string_=\string_Ba2}{}
\markboth{\textcolor{darkblue}{\textbf{\ipa{bv̩˩\textsubscript{a}}}} \textsubscript{2}}{}
\textcolor{teal}{\zh{动词}} \hspace{4pt} \zh{声调类:} L\textsubscript{a}.
\zh{蒸。} \textcolor{Sepia}{\selectlanguage{english}To steam, to cook by steaming.} \textcolor{PineGreen}{\selectlanguage{french}Cuire à la vapeur, étuver.}  ¶ \textcolor{darkblue}{\textbf{\ipa{le˧-bv̩˩-ze˩}}} \zh{蒸了} \textcolor{Sepia}{\selectlanguage{english}\mytextsc{accomp} \string_ \mytextsc{pfv}} \textcolor{PineGreen}{\selectlanguage{french}\mytextsc{accomp} \string_ \mytextsc{pfv}}  
 ¶ \textcolor{darkblue}{\textbf{\ipa{pɤ˩jɤ˧ bv̩˥}}} \zh{蒸馒头} \textcolor{Sepia}{\selectlanguage{english}to steam buns} \textcolor{PineGreen}{\selectlanguage{french}cuire de la pâte à pain à la vapeur, faire des petits pains à la vapeur}  
 ¶ \textcolor{darkblue}{\textbf{\ipa{hɑ˧ bv̩˥\textasciitilde{}bv̩˩}}} \zh{蒸米饭} \textcolor{Sepia}{\selectlanguage{english}to steam rice} \textcolor{PineGreen}{\selectlanguage{french}cuire du riz à la vapeur}  

\lhead{\firstmark}
\rhead{\botmark}

\subsection{\hspace{-0.5cm} {\Large \textcolor{darkblue}{\textbf{\ipa{bv̩˩\textsubscript{a}}}} \textsubscript{3}}\hspace{0.5cm}[\kern2pt{\textcolor{darkblue}{\textbf{\ipa{bv̩˩˥}}}}\kern2pt]} \hypertarget{bv\string_=\string_Ba3}{}
\markboth{\textcolor{darkblue}{\textbf{\ipa{bv̩˩\textsubscript{a}}}} \textsubscript{3}}{}
\textcolor{teal}{\zh{动词}} \hspace{4pt} \zh{声调类:} L\textsubscript{a}.
\zh{过(日子)。} \textcolor{Sepia}{\selectlanguage{english}To live (one's life).} \textcolor{PineGreen}{\selectlanguage{french}Vivre, couler des jours, vivre (sa vie).}  ¶ \textcolor{darkblue}{\textbf{\ipa{zɯ˧ bv̩˩}}} \zh{过日子} \textcolor{Sepia}{\selectlanguage{english}to live one's life} \textcolor{PineGreen}{\selectlanguage{french}vivre sa vie}  
 ¶ \textcolor{darkblue}{\textbf{\ipa{hĩ˧-zɯ˧ bv̩˥, | lo˧hɑ˧!}}} \zh{生活,是艰难的!} \textcolor{Sepia}{\selectlanguage{english}Living one's life is hard! / Life is tough!} \textcolor{PineGreen}{\selectlanguage{french}passer sa vie/vivre la vie des hommes, c'est difficile! / la vie est dure!}  
 ¶ \textcolor{darkblue}{\textbf{\ipa{hĩ˧-zɯ˧ | le˧-bv̩˩-ze˩}}} \zh{他的日子,就结束了!(情景:一个人去世了,葬礼的时候,有人这样说。)} \textcolor{Sepia}{\selectlanguage{english}(Her/his) life has gone by! / (Her/his) life is over! (A reflection after someone's decease.)} \textcolor{PineGreen}{\selectlanguage{french}(Sa) vie a passé! / (Sa) vie est terminée! (Réflexion après le décès de quelqu'un.)}  
 ¶ \textcolor{darkblue}{\textbf{\ipa{qʰwɤ˧-ɭɯ˥, | ʈʂʰæ˧-mɤ˧-dʑɯ˧! | ʈʂʰɯ˧ ɖɯ˧-zɯ˧ bv̩˩-ze˩!}}} \zh{他从来没有洗过碗!他这辈子就是这么过来的!(关于一个官员,完全不用管家务、日常生活中的活儿:有人来管一切。)} \textcolor{Sepia}{\selectlanguage{english}He never had to do the washing-up (in his entire life)! That's how he spent his lifetime (=without any practical concerns)! (About an office-holder whose every need in daily life was attended to by servants.)} \textcolor{PineGreen}{\selectlanguage{french}Il n'a jamais eu à faire la vaisselle (de sa vie)! Voilà comment s'est passée toute sa vie! (Commentaire au sujet de la vie d'un mandarin qui s'était entièrement soustrait aux tâches manuelles.)}  
 ¶ \textcolor{darkblue}{\textbf{\ipa{ɖɯ˧-ɲi˥\textasciitilde{}ɖɯ˩-ɲi˩ | bv̩˩ lo˩ fv̩˩˥!}}} \zh{日子过得真快!} \textcolor{Sepia}{\selectlanguage{english}How easily days go by! / How time flies!} \textcolor{PineGreen}{\selectlanguage{french}Les journées passent bien vite! / Comme le temps passe!}  

\lhead{\firstmark}
\rhead{\botmark}

\subsection{\hspace{-0.5cm} {\Large \textcolor{darkblue}{\textbf{\ipa{bv̩˩\textsubscript{a}}}} \textsubscript{4}}\hspace{0.5cm}[\kern2pt{\textcolor{darkblue}{\textbf{\ipa{bv̩˩˥}}}}\kern2pt]} \hypertarget{bv\string_=\string_Ba4}{}
\markboth{\textcolor{darkblue}{\textbf{\ipa{bv̩˩\textsubscript{a}}}} \textsubscript{4}}{}
\textcolor{teal}{\zh{动词}} \hspace{4pt} \zh{声调类:} L\textsubscript{a}.
\ding{202} \zh{泼水,浇(浇菜)。} \textcolor{Sepia}{\selectlanguage{english}To sprinkle water.} \textcolor{PineGreen}{\selectlanguage{french}Asperger; arroser.}  ¶ \textcolor{darkblue}{\textbf{\ipa{le˧-bv̩˩-ze˩}}} \zh{泼了} \textcolor{Sepia}{\selectlanguage{english}\mytextsc{accomp} \string_ \mytextsc{pfv}} \textcolor{PineGreen}{\selectlanguage{french}\mytextsc{accomp} \string_ \mytextsc{pfv}}  
 ¶ \textcolor{darkblue}{\textbf{\ipa{dʑɯ˩ bv̩˩˥}}} \zh{泼水} \textcolor{Sepia}{\selectlanguage{english}to sprinkle water} \textcolor{PineGreen}{\selectlanguage{french}asperger d'eau; arroser}  
 ¶ \textcolor{darkblue}{\textbf{\ipa{ɖɯ˧-bv̩˧\textasciitilde{}bv̩˥-ɻ̍˩}}} \zh{泼一泼} \textcolor{Sepia}{\selectlanguage{english}\mytextsc{delimitative} \mytextsc{red} \mytextsc{inceptive}} \textcolor{PineGreen}{\selectlanguage{french}\mytextsc{délimitatif} \mytextsc{red} \mytextsc{inchoatif}}  
 ¶ \textcolor{darkblue}{\textbf{\ipa{le˧-bv̩˧\textasciitilde{}bv̩˥-ze˩}}} \zh{泼了一点} \textcolor{Sepia}{\selectlanguage{english}\mytextsc{accomp} \mytextsc{red} \mytextsc{pfv}} \textcolor{PineGreen}{\selectlanguage{french}\mytextsc{accomp} \mytextsc{red} \mytextsc{pfv}}  
\ding{203} \zh{撒(种子)。} \textcolor{Sepia}{\selectlanguage{english}To sow (seeds).} \textcolor{PineGreen}{\selectlanguage{french}Disperser, semer (ex.: des graines).}  ¶ \textcolor{darkblue}{\textbf{\ipa{ɻæ˩ bv̩˥}}} \zh{撒种子} \textcolor{Sepia}{\selectlanguage{english}to sow seeds} \textcolor{PineGreen}{\selectlanguage{french}semer à la volée, répandre des graines (pour les semailles)}  
 ¶ \textcolor{darkblue}{\textbf{\ipa{tʰi˧-bv̩˩-ɻ̍˩}}} \zh{撒吧!} \textcolor{Sepia}{\selectlanguage{english}Go ahead and sow!} \textcolor{PineGreen}{\selectlanguage{french}Sème donc!}  
 ¶ \textcolor{darkblue}{\textbf{\ipa{tʰi˧-bv̩˩-qɑ˩!}}} \zh{撒吧!} \textcolor{Sepia}{\selectlanguage{english}Sow!} \textcolor{PineGreen}{\selectlanguage{french}Sème!}  

\lhead{\firstmark}
\rhead{\botmark}

\subsection{\hspace{-0.5cm} {\Large \textcolor{darkblue}{\textbf{\ipa{bv̩˩\textsubscript{a}}}} \textsubscript{5}}\hspace{0.5cm}[\kern2pt{\textcolor{darkblue}{\textbf{\ipa{bv̩˩˥}}}}\kern2pt]} \hypertarget{bv\string_=\string_Ba5}{}
\markboth{\textcolor{darkblue}{\textbf{\ipa{bv̩˩\textsubscript{a}}}} \textsubscript{5}}{}
\textcolor{teal}{\zh{形容词}} \hspace{4pt} \zh{声调类:} L\textsubscript{a}.
\zh{细、薄。} \textcolor{Sepia}{\selectlanguage{english}Thin, scarce, sparse (e.g. hair).} \textcolor{PineGreen}{\selectlanguage{french}Clairsemé, à nu.}  ¶ \textcolor{darkblue}{\textbf{\ipa{ʁo˧ bv̩˧˥}}} \zh{头秃、头发很少} \textcolor{Sepia}{\selectlanguage{english}bald (literally “the head (has) scarce (hair)”)} \textcolor{PineGreen}{\selectlanguage{french}chauve (littéralement “la tête est à nu”)}  
 ¶ \textcolor{darkblue}{\textbf{\ipa{ʁo˧-bv̩˧-hĩ˥}}} \zh{秃子} \textcolor{Sepia}{\selectlanguage{english}bald person} \textcolor{PineGreen}{\selectlanguage{french}un homme chauve}  
 ¶ \textcolor{darkblue}{\textbf{\ipa{ʁo˧-bv̩˧-zo˥}}} \zh{同上} \textcolor{Sepia}{\selectlanguage{english}as above} \textcolor{PineGreen}{\selectlanguage{french}idem}  
 ¶ \textcolor{darkblue}{\textbf{\ipa{ʈʂʰɯ˧ | ʁo˧ bv̩˧-ze˥}}} \zh{他秃头了,他头发掉了} \textcolor{Sepia}{\selectlanguage{english}He lost his hair, he went bald} \textcolor{PineGreen}{\selectlanguage{french}il a perdu ses cheveux, il est devenu chauve}  
 ¶ \textcolor{darkblue}{\textbf{\ipa{ʁo˧qʰwɤ˩ | le˧-bv̩˩-ze˩}}} \zh{(他)秃头了。} \textcolor{Sepia}{\selectlanguage{english}(his) head has gone bald} \textcolor{PineGreen}{\selectlanguage{french}(sa) tête s'est dégarnie, (sa) tête est devenue chauve}  
 ¶ \textcolor{darkblue}{\textbf{\ipa{bv̩˩-hĩ˩˥}}} \zh{秃的} \textcolor{Sepia}{\selectlanguage{english}\mytextsc{rel}} \textcolor{PineGreen}{\selectlanguage{french}\mytextsc{rel}}  

\lhead{\firstmark}
\rhead{\botmark}

\subsection{\hspace{-0.5cm} {\Large \textcolor{darkblue}{\textbf{\ipa{bv̩˧ɖæ˧}}}}\hspace{0.5cm}[\kern2pt{\textcolor{darkblue}{\textbf{\ipa{bv̩˩ɖæ˩˥}}}}\kern2pt]} \hypertarget{bv\string_=\string_Md`\{\string_M1}{}
\markboth{\textcolor{darkblue}{\textbf{\ipa{bv̩˧ɖæ˧}}}}{}
\textcolor{teal}{\zh{形容词}} \hspace{4pt} \zh{声调类:} M.
\zh{脾气很坏。} \textcolor{Sepia}{\selectlanguage{english}With a bad temper. Literally 'short-intestined': in folk representation, short intestines are associated with hasty emotional reactions, whereas long intestines allow their owner to digest vexations calmly.} \textcolor{PineGreen}{\selectlanguage{french}De mauvaise humeur, ayant mauvais caractère.}  ¶ \textcolor{darkblue}{\textbf{\ipa{ʈʂʰɯ˧ | bv̩˧ɖæ˧-ze˩!}}} \zh{他现在脾气很坏!/ 他生气了!} \textcolor{Sepia}{\selectlanguage{english}He is is a bad mood now.} \textcolor{PineGreen}{\selectlanguage{french}Il est de mauvais poil/Il est de mauvaise humeur!}  

\lhead{\firstmark}
\rhead{\botmark}

\subsection{\hspace{-0.5cm} {\Large \textcolor{darkblue}{\textbf{\ipa{bv̩˩di˩}}}}\hspace{0.5cm}[\kern2pt{\textcolor{darkblue}{\textbf{\ipa{bv̩˧di˧}}}}\kern2pt]} \hypertarget{bv\string_=\string_Bdi\string_B1}{}
\markboth{\textcolor{darkblue}{\textbf{\ipa{bv̩˩di˩}}}}{}
\textcolor{teal}{\zh{名词}} \hspace{4pt} \zh{声调类:} L.
\zh{蒸笼。} \textcolor{Sepia}{\selectlanguage{english}Food steamer.} \textcolor{PineGreen}{\selectlanguage{french}Étuve.}  \zh{量词}: \textcolor{darkblue}{\textbf{\ipa{ɭɯ˧}}} \zh{~【参考】~} \hyperlink{}{\textcolor{darkblue}{\textbf{\ipa{bv̩˩˧}}} \textsubscript{2}} 
\lhead{\firstmark}
\rhead{\botmark}

\subsection{\hspace{-0.5cm} {\Large \textcolor{darkblue}{\textbf{\ipa{bv̩˩dze˩}}} \textsubscript{1}}\hspace{0.5cm}[\kern2pt{\textcolor{darkblue}{\textbf{\ipa{bv̩˩dze˩˥}}}}\kern2pt]} \hypertarget{bv\string_=\string_Bdze\string_B1}{}
\markboth{\textcolor{darkblue}{\textbf{\ipa{bv̩˩dze˩}}} \textsubscript{1}}{}
\textcolor{teal}{\zh{名词}} \hspace{4pt} \zh{声调类:} L.
\zh{大调羹。} \textcolor{Sepia}{\selectlanguage{english}Large spoon, used for rice.} \textcolor{PineGreen}{\selectlanguage{french}Grosse cuillère (pour servir le riz, etc.).}  \zh{量词}: \textcolor{darkblue}{\textbf{\ipa{nɑ˧}}} \zh{~【参考】~} \hyperlink{}{\textcolor{darkblue}{\textbf{\ipa{bv̩˩dze˩}}} \textsubscript{2}} 
\lhead{\firstmark}
\rhead{\botmark}

\subsection{\hspace{-0.5cm} {\Large \textcolor{darkblue}{\textbf{\ipa{bv̩˩dze˩}}} \textsubscript{2}}\hspace{0.5cm}[\kern2pt{\textcolor{darkblue}{\textbf{\ipa{bv̩˩dze˩˥}}}}\kern2pt]} \hypertarget{bv\string_=\string_Bdze\string_B2}{}
\markboth{\textcolor{darkblue}{\textbf{\ipa{bv̩˩dze˩}}} \textsubscript{2}}{}
\textcolor{teal}{\zh{量词}} \hspace{4pt} \zh{声调类:} L.
\zh{量词:勺。} \textcolor{Sepia}{\selectlanguage{english}Ladleful.} \textcolor{PineGreen}{\selectlanguage{french}Classificateur des cuillerée.}  ¶ \textcolor{darkblue}{\textbf{\ipa{ɖɯ˧-bv̩˩dze˩}}} \zh{一勺} \textcolor{Sepia}{\selectlanguage{english}one ladleful} \textcolor{PineGreen}{\selectlanguage{french}une louchée, une louche de}  
\zh{~【参考】~} \hyperlink{}{\textcolor{darkblue}{\textbf{\ipa{bv̩˩dze˩}}} \textsubscript{1}} 
\lhead{\firstmark}
\rhead{\botmark}

\subsection{\hspace{-0.5cm} {\Large \textcolor{darkblue}{\textbf{\ipa{bv̩˧hu˧˥}}}}\hspace{0.5cm}[\kern2pt{\textcolor{darkblue}{\textbf{\ipa{bv̩˩hu˩˥}}}}\kern2pt]} \hypertarget{bv\string_=\string_Mhu\string_M\string_T1}{}
\markboth{\textcolor{darkblue}{\textbf{\ipa{bv̩˧hu˧˥}}}}{}
\textcolor{teal}{\zh{名词}} \hspace{4pt} \zh{声调类:} MH\#.
\zh{胃与肠。} \textcolor{Sepia}{\selectlanguage{english}Bowels: intestine+stomach.} \textcolor{PineGreen}{\selectlanguage{french}Tube digestif: estomac+intestin.}  \zh{量词}: \textcolor{darkblue}{\textbf{\ipa{kwɤ˩}}} 
\lhead{\firstmark}
\rhead{\botmark}

\subsection{\hspace{-0.5cm} {\Large \textcolor{darkblue}{\textbf{\ipa{bv̩˩hwɤ˩}}}}\hspace{0.5cm}[\kern2pt{\textcolor{darkblue}{\textbf{\ipa{bv̩˧hwɤ˧˥}}}}\kern2pt]} \hypertarget{bv\string_=\string_Bhw7\string_B1}{}
\markboth{\textcolor{darkblue}{\textbf{\ipa{bv̩˩hwɤ˩}}}}{}
\textcolor{teal}{\zh{名词}} \hspace{4pt} \zh{声调类:} L.
\zh{山上放牧的人暂时住的木头小房。} \textcolor{Sepia}{\selectlanguage{english}Wooden hut where shepherds stay while herding their flock on the mountain.} \textcolor{PineGreen}{\selectlanguage{french}Cabane de berger, sur la montagne; n'est pas occupée à l'année, mais est assez solide pour être utilisée année après année, à la différence des cabanes provisoires construites lorsqu'on doit rester qq jours sur la montagne pour couper du bois.}  \zh{量词}: \textcolor{darkblue}{\textbf{\ipa{ɭɯ˧}}} 
\lhead{\firstmark}
\rhead{\botmark}

\subsection{\hspace{-0.5cm} {\Large \textcolor{darkblue}{\textbf{\ipa{bv̩˧kʰɯ˧˥}}}}\hspace{0.5cm}[\kern2pt{\textcolor{darkblue}{\textbf{\ipa{bv̩˩kʰɯ˩˥}}}}\kern2pt]} \hypertarget{bv\string_=\string_Mk\string_hM\string_M\string_T1}{}
\markboth{\textcolor{darkblue}{\textbf{\ipa{bv̩˧kʰɯ˧˥}}}}{}
\textcolor{teal}{\zh{名词}} \hspace{4pt} \zh{声调类:} MH\#.
\zh{蚕。} \textcolor{Sepia}{\selectlanguage{english}Silkworm.} \textcolor{PineGreen}{\selectlanguage{french}Ver à soie.}  \zh{量词}: \textcolor{darkblue}{\textbf{\ipa{kʰɯ˩}}} 
\lhead{\firstmark}
\rhead{\botmark}

\subsection{\hspace{-0.5cm} {\Large \textcolor{darkblue}{\textbf{\ipa{bv̩˩ɭɯ˩}}}}\hspace{0.5cm}[\kern2pt{\textcolor{darkblue}{\textbf{\ipa{bv̩˧ɭɯ˧˥}}}}\kern2pt]} \hypertarget{bv\string_=\string_Bl\string_RM\string_B1}{}
\markboth{\textcolor{darkblue}{\textbf{\ipa{bv̩˩ɭɯ˩}}}}{}
\textcolor{teal}{\zh{名词}} \hspace{4pt} \zh{声调类:} L.
\zh{肾。} \textcolor{Sepia}{\selectlanguage{english}Kidneys.} \textcolor{PineGreen}{\selectlanguage{french}Rein.}  \zh{量词}: \textcolor{darkblue}{\textbf{\ipa{ɭɯ˧}}} 
\lhead{\firstmark}
\rhead{\botmark}

\subsection{\hspace{-0.5cm} {\Large \textcolor{darkblue}{\textbf{\ipa{bv̩˧mi˧}}} \textsubscript{1}}\hspace{0.5cm}[\kern2pt{\textcolor{darkblue}{\textbf{\ipa{bv̩˩mi˩˥}}}}\kern2pt]} \hypertarget{bv\string_=\string_Mmi\string_M1}{}
\markboth{\textcolor{darkblue}{\textbf{\ipa{bv̩˧mi˧}}} \textsubscript{1}}{}
\textcolor{teal}{\zh{名词}} \hspace{4pt} \zh{声调类:} M.
\zh{母牦牛。} \textcolor{Sepia}{\selectlanguage{english}Female yak, dri, drimo, nak.} \textcolor{PineGreen}{\selectlanguage{french}Yak femelle, dri, drimo, nak.}  ¶ \textcolor{darkblue}{\textbf{\ipa{bv̩˧mi˧-bv̩˩ʂwæ˩}}} \zh{母牦牛与阉割牦牛} \textcolor{Sepia}{\selectlanguage{english}female yak and castrated yak} \textcolor{PineGreen}{\selectlanguage{french}yak femelle et yak châtré}  
 ¶ \textcolor{darkblue}{\textbf{\ipa{bv̩˧mi˧-bv̩˧zo\#˥}}} \zh{母牦牛与小牦牛} \textcolor{Sepia}{\selectlanguage{english}female yak and yak calf (baby yak)} \textcolor{PineGreen}{\selectlanguage{french}maman yack et petit yack}  
 \zh{量词}: \textcolor{darkblue}{\textbf{\ipa{mi˩}}} 
\lhead{\firstmark}
\rhead{\botmark}

\subsection{\hspace{-0.5cm} {\Large \textcolor{darkblue}{\textbf{\ipa{bv̩˧mi˧}}} \textsubscript{2}}\hspace{0.5cm}[\kern2pt{\textcolor{darkblue}{\textbf{\ipa{bv̩˧mi˧}}}}\kern2pt]} \hypertarget{bv\string_=\string_Mmi\string_M2}{}
\markboth{\textcolor{darkblue}{\textbf{\ipa{bv̩˧mi˧}}} \textsubscript{2}}{}
\textcolor{teal}{\zh{名词}} \hspace{4pt} \zh{声调类:} M.
\zh{大蒸笼。} \textcolor{Sepia}{\selectlanguage{english}Large food steamer.} \textcolor{PineGreen}{\selectlanguage{french}Grande étuve.}  \zh{量词}: \textcolor{darkblue}{\textbf{\ipa{mi˩}}} 
\lhead{\firstmark}
\rhead{\botmark}

\subsection{\hspace{-0.5cm} {\Large \textcolor{darkblue}{\textbf{\ipa{bv̩˧-nɑ˥mi˩}}}}\hspace{0.5cm}[\kern2pt{\textcolor{darkblue}{\textbf{\ipa{xxxx non-correspondance entre le nombre de morphèmes et le nombre de tons de morphèmes}}}}\kern2pt]} \hypertarget{bv\string_=\string_M-nA\string_Tmi\string_B1}{}
\markboth{\textcolor{darkblue}{\textbf{\ipa{bv̩˧-nɑ˥mi˩}}}}{}
\textcolor{teal}{\zh{名词}} \hspace{4pt} \zh{声调类:} \#H-.
\zh{玉米黏虫。} \textcolor{Sepia}{\selectlanguage{english}\textit{Mythimna separata (Walker)}.} \textcolor{PineGreen}{\selectlanguage{french}\textit{Mythimna separata (Walker)}.} 
\lhead{\firstmark}
\rhead{\botmark}

\subsection{\hspace{-0.5cm} {\Large \textcolor{darkblue}{\textbf{\ipa{bv̩˧nv̩˧}}} \textsubscript{1}}\hspace{0.5cm}[\kern2pt{\textcolor{darkblue}{\textbf{\ipa{xxxx non-correspondance entre le nombre de morphèmes et le nombre de tons de morphèmes}}}}\kern2pt]} \hypertarget{bv\string_=\string_Mnv\string_=\string_M1}{}
\markboth{\textcolor{darkblue}{\textbf{\ipa{bv̩˧nv̩˧}}} \textsubscript{1}}{}
\textcolor{teal}{\zh{动词}} \hspace{4pt} \zh{声调类:} M intrans.
\zh{闻到(嗅觉)。} \textcolor{Sepia}{\selectlanguage{english}To smell, to perceive by smelling.} \textcolor{PineGreen}{\selectlanguage{french}Sentir (par l'odorat).}  ¶ \textcolor{darkblue}{\textbf{\ipa{no˧ | ɖɯ˧-bv̩˧nv̩˧-ɻ̍˩! | ɖwæ˩˥ | ɕjɤ˧!}}} \zh{你闻一闻吧!好香!} \textcolor{Sepia}{\selectlanguage{english}Have a smell! It smells great!} \textcolor{PineGreen}{\selectlanguage{french}Sens donc! ça sent très bon/c'est très odorant/parfumé!}  
\zh{~【参考】~} \hyperlink{}{\textcolor{darkblue}{\textbf{\ipa{bv̩˧nv̩˧}}} \textsubscript{2}} 
\lhead{\firstmark}
\rhead{\botmark}

\subsection{\hspace{-0.5cm} {\Large \textcolor{darkblue}{\textbf{\ipa{bv̩˧nv̩˧}}} \textsubscript{2}}\hspace{0.5cm}[\kern2pt{\textcolor{darkblue}{\textbf{\ipa{bv̩˧nv̩˧}}}}\kern2pt]} \hypertarget{bv\string_=\string_Mnv\string_=\string_M2}{}
\markboth{\textcolor{darkblue}{\textbf{\ipa{bv̩˧nv̩˧}}} \textsubscript{2}}{}
\textcolor{teal}{\zh{形容词}} \hspace{4pt} \zh{声调类:} M intrans.
\zh{臭。} \textcolor{Sepia}{\selectlanguage{english}Stinking, smelly.} \textcolor{PineGreen}{\selectlanguage{french}Malodorant, puant, qui a une mauvaise odeur.}  ¶ \textcolor{darkblue}{\textbf{\ipa{bv̩˧nv̩˧-ze˧}}} \zh{臭了} \textcolor{Sepia}{\selectlanguage{english}\mytextsc{pfv}} \textcolor{PineGreen}{\selectlanguage{french}\mytextsc{pfv}}  
\zh{~【参考】~} \hyperlink{}{\textcolor{darkblue}{\textbf{\ipa{bv̩˧nv̩˧}}} \textsubscript{1}} 
\lhead{\firstmark}
\rhead{\botmark}

\subsection{\hspace{-0.5cm} {\Large \textcolor{darkblue}{\textbf{\ipa{bv̩˧pʰv̩˧}}}}\hspace{0.5cm}[\kern2pt{\textcolor{darkblue}{\textbf{\ipa{bv̩˧pʰv̩˧}}}}\kern2pt]} \hypertarget{bv\string_=\string_Mp\string_hv\string_=\string_M1}{}
\markboth{\textcolor{darkblue}{\textbf{\ipa{bv̩˧pʰv̩˧}}}}{}
\textcolor{teal}{\zh{名词}} \hspace{4pt} \zh{声调类:} M.
\zh{公牦牛。} \textcolor{Sepia}{\selectlanguage{english}Male yak (elicited form; the commonly used form is \textcolor{darkblue}{\textbf{\ipa{/bv̩˩ʂwæ˩/}}}).} \textcolor{PineGreen}{\selectlanguage{french}Yak mâle. Ce mot est une forme élicitée; la forme usuelle est: \textcolor{darkblue}{\textbf{\ipa{/bv̩˩ʂwæ˩/}}}.}  \zh{量词}: \textcolor{darkblue}{\textbf{\ipa{mi˩}}} 
\lhead{\firstmark}
\rhead{\botmark}

\subsection{\hspace{-0.5cm} {\Large \textcolor{darkblue}{\textbf{\ipa{bv̩˩-qʰæ˩}}}}\hspace{0.5cm}[\kern2pt{\textcolor{darkblue}{\textbf{\ipa{xxxx non-correspondance entre le nombre de morphèmes et le nombre de tons de morphèmes}}}}\kern2pt]} \hypertarget{bv\string_=\string_B-q\string_h\{\string_B1}{}
\markboth{\textcolor{darkblue}{\textbf{\ipa{bv̩˩-qʰæ˩}}}}{}
\textcolor{teal}{\zh{名词}} \hspace{4pt} \zh{声调类:} L.
\zh{肥料、粪。} \textcolor{Sepia}{\selectlanguage{english}Manure, dung.} \textcolor{PineGreen}{\selectlanguage{french}Fumier.}  ¶ \textcolor{darkblue}{\textbf{\ipa{bv̩˩qʰæ˩ tʰv̩˩-ʁwɤ˥}}} \zh{这堆肥料} \textcolor{Sepia}{\selectlanguage{english}\mytextsc{n}+\mytextsc{dem}+\mytextsc{clf}} \textcolor{PineGreen}{\selectlanguage{french}\mytextsc{n}+\mytextsc{dem}+\mytextsc{clf}}  
 \zh{量词}: \textcolor{darkblue}{\textbf{\ipa{ʁwɤ˧}}} 
\lhead{\firstmark}
\rhead{\botmark}

\subsection{\hspace{-0.5cm} {\Large \textcolor{darkblue}{\textbf{\ipa{bv̩˩qo˩-bv̩˧qʰæ˩}}}}\hspace{0.5cm}[\kern2pt{\textcolor{darkblue}{\textbf{\ipa{xxxx non-correspondance entre le nombre de morphèmes et le nombre de tons de morphèmes}}}}\kern2pt]} \hypertarget{bv\string_=\string_Bqo\string_B-bv\string_=\string_Mq\string_h\{\string_B1}{}
\markboth{\textcolor{darkblue}{\textbf{\ipa{bv̩˩qo˩-bv̩˧qʰæ˩}}}}{}
\textcolor{teal}{\zh{名词}} \hspace{4pt} \zh{声调类:} L-L\#.
\zh{肥料、粪。} \textcolor{Sepia}{\selectlanguage{english}Manure, dung.} \textcolor{PineGreen}{\selectlanguage{french}Fumier.}  \zh{量词}: \textcolor{darkblue}{\textbf{\ipa{ʁwɤ˧}}} 
\lhead{\firstmark}
\rhead{\botmark}

\subsection{\hspace{-0.5cm} {\Large \textcolor{darkblue}{\textbf{\ipa{bv̩˩qo˩-qʰæ˩}}}}\hspace{0.5cm}[\kern2pt{\textcolor{darkblue}{\textbf{\ipa{bv̩˩qo˩qʰæ˩˥}}}}\kern2pt]} \hypertarget{bv\string_=\string_Bqo\string_B-q\string_h\{\string_B1}{}
\markboth{\textcolor{darkblue}{\textbf{\ipa{bv̩˩qo˩-qʰæ˩}}}}{}
\textcolor{teal}{\zh{名词}} \hspace{4pt} \zh{声调类:} L.
\zh{肥料 , 粪。} \textcolor{Sepia}{\selectlanguage{english}Manure, excrement.} \textcolor{PineGreen}{\selectlanguage{french}Fumier.}  ¶ \textcolor{darkblue}{\textbf{\ipa{bv̩˩qo˩-qʰæ˩ tʰv̩˩-ʁwɤ˥}}} \zh{这堆肥料} \textcolor{Sepia}{\selectlanguage{english}\mytextsc{n}+\mytextsc{dem}+\mytextsc{clf}} \textcolor{PineGreen}{\selectlanguage{french}\mytextsc{n}+\mytextsc{dem}+\mytextsc{clf}}  
 \zh{量词}: \textcolor{darkblue}{\textbf{\ipa{ʁwɤ˧}}} 
\lhead{\firstmark}
\rhead{\botmark}

\subsection{\hspace{-0.5cm} {\Large \textcolor{darkblue}{\textbf{\ipa{bv̩˩qʰv̩˩}}}}\hspace{0.5cm}[\kern2pt{\textcolor{darkblue}{\textbf{\ipa{bv̩˩qʰv̩˩˥}}}}\kern2pt]} \hypertarget{bv\string_=\string_Bq\string_hv\string_=\string_B1}{}
\markboth{\textcolor{darkblue}{\textbf{\ipa{bv̩˩qʰv̩˩}}}}{}
\textcolor{teal}{\zh{名词}} \hspace{4pt} \zh{声调类:} L.
\ding{202} \zh{法螺、海螺、螺号。} \textcolor{Sepia}{\selectlanguage{english}Conch shell, \textit{Turbinella pyrum L.}. It is used in ceremonies. Each family has a pair of conchs.} \textcolor{PineGreen}{\selectlanguage{french}Conque, \textit{Turbinella pyrum L.}. On y souffle, comme dans une trompe, lors des cérémonies. Chaque famille en possède une paire.}  \zh{量词}: \textcolor{darkblue}{\textbf{\ipa{dze˩}}} \ding{203} \zh{掌纹。} \textcolor{Sepia}{\selectlanguage{english}Lines of the hand.} \textcolor{PineGreen}{\selectlanguage{french}Lignes de la main (dont la forme en spirale évoque les conques, objet symbolique important dans la culture na).}  ¶ \textcolor{darkblue}{\textbf{\ipa{lo˩qʰwɤ˧-bv̩˧ | bv̩˩qʰv̩˩˥}}} \zh{掌纹} \textcolor{Sepia}{\selectlanguage{english}the lines of the hand} \textcolor{PineGreen}{\selectlanguage{french}les lignes de la main}  

\lhead{\firstmark}
\rhead{\botmark}

\subsection{\hspace{-0.5cm} {\Large \textcolor{darkblue}{\textbf{\ipa{bv̩˧qʰv̩˧ʑi˩-hĩ˩}}}}\hspace{0.5cm}[\kern2pt{\textcolor{darkblue}{\textbf{\ipa{xxxx non-correspondance entre le nombre de morphèmes et le nombre de tons de morphèmes}}}}\kern2pt]} \hypertarget{bv\string_=\string_Mq\string_hv\string_=\string_Mz£i\string_B-hi\string_~\string_B1}{}
\markboth{\textcolor{darkblue}{\textbf{\ipa{bv̩˧qʰv̩˧ʑi˩-hĩ˩}}}}{}
\textcolor{teal}{\zh{名词}} \hspace{4pt} \zh{声调类:} L\#.
\zh{蜗牛,螺蛳。} \textcolor{Sepia}{\selectlanguage{english}Snail.} \textcolor{PineGreen}{\selectlanguage{french}Escargot.}  \zh{量词}: \textcolor{darkblue}{\textbf{\ipa{mi˩}}} 
\lhead{\firstmark}
\rhead{\botmark}

\subsection{\hspace{-0.5cm} {\Large \textcolor{darkblue}{\textbf{\ipa{bv̩˧ɻ\#˥}}}}\hspace{0.5cm}[\kern2pt{\textcolor{darkblue}{\textbf{\ipa{bv̩˧ɻ˩}}}}\kern2pt]} \hypertarget{bv\string_=\string_Mr£`\#\string_T1}{}
\markboth{\textcolor{darkblue}{\textbf{\ipa{bv̩˧ɻ\#˥}}}}{}
\textcolor{teal}{\zh{名词}} \hspace{4pt} \zh{声调类:} \#H.
\zh{苍蝇。} \textcolor{Sepia}{\selectlanguage{english}Fly.} \textcolor{PineGreen}{\selectlanguage{french}Mouche.}  ¶ \textcolor{darkblue}{\textbf{\ipa{bv̩˧ɻ̍˧ ʈʂʰɯ˧-mi˧˥ / bv̩˧ɻ̍˧ ʈʂʰɯ˧-mi˥\#}}} \zh{这只苍蝇} \textcolor{Sepia}{\selectlanguage{english}\mytextsc{n}+\mytextsc{dem}+\mytextsc{clf}} \textcolor{PineGreen}{\selectlanguage{french}\mytextsc{n}+\mytextsc{dem}+\mytextsc{clf}}  
 \zh{量词}: \textcolor{darkblue}{\textbf{\ipa{mi˩}}} 
\lhead{\firstmark}
\rhead{\botmark}

\subsection{\hspace{-0.5cm} {\Large \textcolor{darkblue}{\textbf{\ipa{bv̩˧ʂæ˧}}}}\hspace{0.5cm}[\kern2pt{\textcolor{darkblue}{\textbf{\ipa{bv̩˧ʂæ˧}}}}\kern2pt]} \hypertarget{bv\string_=\string_Ms`\{\string_M1}{}
\markboth{\textcolor{darkblue}{\textbf{\ipa{bv̩˧ʂæ˧}}}}{}
\textcolor{teal}{\zh{形容词}} \hspace{4pt} \zh{声调类:} M.
\zh{脾气好。} \textcolor{Sepia}{\selectlanguage{english}Good-tempered, with a good mood, good-humoured.} \textcolor{PineGreen}{\selectlanguage{french}De bonne humeur, ayant bon caractère.}  ¶ \textcolor{darkblue}{\textbf{\ipa{ʈʂʰɯ˧ | bv̩˧ʂæ˧-ze˩}}} \zh{他现在脾气好。/ 他高兴了。} \textcolor{Sepia}{\selectlanguage{english}He is in a good mood now.} \textcolor{PineGreen}{\selectlanguage{french}il est de bonne humeur}  
 ¶ \textcolor{darkblue}{\textbf{\ipa{bv̩˧ʂæ˧ | ʐwæ˩˥}}} \zh{脾气很好} \textcolor{Sepia}{\selectlanguage{english}in a very good mood} \textcolor{PineGreen}{\selectlanguage{french}de très bonne humeur}  

\lhead{\firstmark}
\rhead{\botmark}

\subsection{\hspace{-0.5cm} {\Large \textcolor{darkblue}{\textbf{\ipa{bv̩˩ʂwæ˩}}}}\hspace{0.5cm}[\kern2pt{\textcolor{darkblue}{\textbf{\ipa{bv̩˧ʂwæ˧}}}}\kern2pt]} \hypertarget{bv\string_=\string_Bs`w\{\string_B1}{}
\markboth{\textcolor{darkblue}{\textbf{\ipa{bv̩˩ʂwæ˩}}}}{}
\textcolor{teal}{\zh{名词}} \hspace{4pt} \zh{声调类:} L.
\zh{阉割过的牦牛。} \textcolor{Sepia}{\selectlanguage{english}Castrated yak.} \textcolor{PineGreen}{\selectlanguage{french}Yak châtré.}  ¶ \textcolor{darkblue}{\textbf{\ipa{bv̩˩ʂwæ˩-bv̩˥mi˩}}} \zh{阉割过的公牦牛与母牦牛} \textcolor{Sepia}{\selectlanguage{english}castrated yak and female yak} \textcolor{PineGreen}{\selectlanguage{french}yak châtré et yak femelle}  
 \zh{量词}: \textcolor{darkblue}{\textbf{\ipa{pʰo˧˥}}} 
\lhead{\firstmark}
\rhead{\botmark}

\subsection{\hspace{-0.5cm} {\Large \textcolor{darkblue}{\textbf{\ipa{bv̩˧tɕi˧}}}}\hspace{0.5cm}[\kern2pt{\textcolor{darkblue}{\textbf{\ipa{bv̩˩tɕi˩˥}}}}\kern2pt]} \hypertarget{bv\string_=\string_Mts£i\string_M1}{}
\markboth{\textcolor{darkblue}{\textbf{\ipa{bv̩˧tɕi˧}}}}{}
\textcolor{teal}{\zh{名词}} \hspace{4pt} \zh{声调类:} M.
\zh{毛桃。} \textcolor{Sepia}{\selectlanguage{english}Wild peach.} \textcolor{PineGreen}{\selectlanguage{french}Pêche sauvage (de petite taille).}  \zh{量词}: \textcolor{darkblue}{\textbf{\ipa{tɕi˧˥}}} \textcolor{darkblue}{\textbf{\ipa{ɭɯ˧}}} 
\lhead{\firstmark}
\rhead{\botmark}

\subsection{\hspace{-0.5cm} {\Large \textcolor{darkblue}{\textbf{\ipa{bv̩˧ʈʂɯ˥}}}}\hspace{0.5cm}[\kern2pt{\textcolor{darkblue}{\textbf{\ipa{bv̩˧ʈʂɯ˧}}}}\kern2pt]} \hypertarget{bv\string_=\string_Mt`s`M\string_T1}{}
\markboth{\textcolor{darkblue}{\textbf{\ipa{bv̩˧ʈʂɯ˥}}}}{}
\textcolor{teal}{\zh{名词}} \hspace{4pt} \zh{声调类:} H\#.
\zh{筛子。} \textcolor{Sepia}{\selectlanguage{english}Sifter, sieve.} \textcolor{PineGreen}{\selectlanguage{french}Vanneries: tamis où l'on fait sécher les graines de courge et autres produits de la ferme.}  \zh{量词}: \textcolor{darkblue}{\textbf{\ipa{nɑ˧}}} 
\lhead{\firstmark}
\rhead{\botmark}

\subsection{\hspace{-0.5cm} {\Large \textcolor{darkblue}{\textbf{\ipa{bv̩˧ʈʂʰv̩˧}}}}\hspace{0.5cm}[\kern2pt{\textcolor{darkblue}{\textbf{\ipa{bv̩˧ʈʂʰv̩˥}}}}\kern2pt]} \hypertarget{bv\string_=\string_Mt`s`\string_hv\string_=\string_M1}{}
\markboth{\textcolor{darkblue}{\textbf{\ipa{bv̩˧ʈʂʰv̩˧}}}}{}
\textcolor{teal}{\zh{名词}} \hspace{4pt} \zh{声调类:} M.
\zh{钹。} \textcolor{Sepia}{\selectlanguage{english}Cymbals.} \textcolor{PineGreen}{\selectlanguage{french}Cymbales.}  \zh{量词}: \textcolor{darkblue}{\textbf{\ipa{nɑ˧}}} 
\lhead{\firstmark}
\rhead{\botmark}

\subsection{\hspace{-0.5cm} {\Large \textcolor{darkblue}{\textbf{\ipa{bv̩˩zo˩}}}}\hspace{0.5cm}[\kern2pt{\textcolor{darkblue}{\textbf{\ipa{bv̩˧zo˧}}}}\kern2pt]} \hypertarget{bv\string_=\string_Bzo\string_B1}{}
\markboth{\textcolor{darkblue}{\textbf{\ipa{bv̩˩zo˩}}}}{}
\textcolor{teal}{\zh{名词}} \hspace{4pt} \zh{声调类:} L.
\zh{小蒸笼。} \textcolor{Sepia}{\selectlanguage{english}Small food steamer.} \textcolor{PineGreen}{\selectlanguage{french}Petite étuve.}  \zh{量词}: \textcolor{darkblue}{\textbf{\ipa{ɭɯ˧}}} 
\lhead{\firstmark}
\rhead{\botmark}

\subsection{\hspace{-0.5cm} {\Large \textcolor{darkblue}{\textbf{\ipa{bv̩˧zo\#˥}}}}\hspace{0.5cm}[\kern2pt{\textcolor{darkblue}{\textbf{\ipa{bv̩˩zo˩˥}}}}\kern2pt]} \hypertarget{bv\string_=\string_Mzo\#\string_T1}{}
\markboth{\textcolor{darkblue}{\textbf{\ipa{bv̩˧zo\#˥}}}}{}
\textcolor{teal}{\zh{名词}} \hspace{4pt} \zh{声调类:} \#H.
\zh{小牦牛。} \textcolor{Sepia}{\selectlanguage{english}Yak calf (baby yak).} \textcolor{PineGreen}{\selectlanguage{french}Petit du yak.}  ¶ \textcolor{darkblue}{\textbf{\ipa{bv̩˧zo˧ tʰv̩˧-mi˧˥ / bv̩˧zo˧ tʰv̩˧-mi˥\#}}} \zh{这头小牦牛} \textcolor{Sepia}{\selectlanguage{english}\mytextsc{n}+\mytextsc{dem}+\mytextsc{clf}} \textcolor{PineGreen}{\selectlanguage{french}\mytextsc{n}+\mytextsc{dem}+\mytextsc{clf}}  
 \zh{量词}: \textcolor{darkblue}{\textbf{\ipa{mi˩}}} 
\lhead{\firstmark}
\rhead{\botmark}

\subsection{\hspace{-0.5cm} {\Large \textcolor{darkblue}{\textbf{\ipa{bv̩˩ʐv̩˩-dzi˩}}}}\hspace{0.5cm}[\kern2pt{\textcolor{darkblue}{\textbf{\ipa{xxxx non-correspondance entre le nombre de morphèmes et le nombre de tons de morphèmes}}}}\kern2pt]} \hypertarget{bv\string_=\string_Bz`v\string_=\string_B-dzi\string_B1}{}
\markboth{\textcolor{darkblue}{\textbf{\ipa{bv̩˩ʐv̩˩-dzi˩}}}}{}
\textcolor{teal}{\zh{名词}} \hspace{4pt} \zh{声调类:} L.
\zh{常春藤。} \textcolor{Sepia}{\selectlanguage{english}Ivy.} \textcolor{PineGreen}{\selectlanguage{french}Lierre.}  \zh{量词}: \textcolor{darkblue}{\textbf{\ipa{dzi˩}}} 
\lhead{\firstmark}
\rhead{\botmark}

\subsection{\hspace{-0.5cm} {\Large \textcolor{darkblue}{\textbf{\ipa{bv̩˧ʐv̩˧-kʰv̩˧˥}}}}\hspace{0.5cm}[\kern2pt{\textcolor{darkblue}{\textbf{\ipa{xxxx non-correspondance entre le nombre de morphèmes et le nombre de tons de morphèmes}}}}\kern2pt]} \hypertarget{bv\string_=\string_Mz`v\string_=\string_M-k\string_hv\string_=\string_M\string_T1}{}
\markboth{\textcolor{darkblue}{\textbf{\ipa{bv̩˧ʐv̩˧-kʰv̩˧˥}}}}{}
\textcolor{teal}{\zh{名词}} \hspace{4pt} \zh{声调类:} MH\#.
\zh{蛇年。} \textcolor{Sepia}{\selectlanguage{english}Year of the serpent.} \textcolor{PineGreen}{\selectlanguage{french}Année du serpent.} \zh{~【参考】~} \hyperlink{}{\textcolor{darkblue}{\textbf{\ipa{ʐv̩˧bæ˧}}}} 
\lhead{\firstmark}
\rhead{\botmark}

\subsection{\hspace{-0.5cm} {\Large \textcolor{darkblue}{\textbf{\ipa{‑bv˧}}}}\hspace{0.5cm}[\kern2pt{\textcolor{darkblue}{\textbf{\ipa{xxxx groupe tonal entier sans aucun ton}}}}\kern2pt]} \hypertarget{‑bv\string_M1}{}
\markboth{\textcolor{darkblue}{\textbf{\ipa{‑bv˧}}}}{}
\textcolor{teal}{\zh{后缀}} \hspace{4pt} \zh{声调类:} 0.
\zh{属式:的。} \textcolor{Sepia}{\selectlanguage{english}Possessive.} \textcolor{PineGreen}{\selectlanguage{french}Possessif.} 
\lhead{\firstmark}
\rhead{\botmark}

\newpage
\section*{\centering- \textcolor{darkblue}{\textbf{\ipa{ɕ}}} -}
\subsection{\hspace{-0.5cm} {\Large \textcolor{darkblue}{\textbf{\ipa{ɕi˥\textsubscript{a}}}}}\hspace{0.5cm}[\kern2pt{\textcolor{darkblue}{\textbf{\ipa{ɕi˥}}}}\kern2pt]} \hypertarget{s£i\string_Ta1}{}
\markboth{\textcolor{darkblue}{\textbf{\ipa{ɕi˥\textsubscript{a}}}}}{}
\textcolor{teal}{\zh{量词}} \hspace{4pt} \zh{声调类:} H\textsubscript{a}.
\zh{百。} \textcolor{Sepia}{\selectlanguage{english}100.} \textcolor{PineGreen}{\selectlanguage{french}100.}  ¶ \textcolor{darkblue}{\textbf{\ipa{ɖɯ˧-ɕi˥}}} \zh{一百} \textcolor{Sepia}{\selectlanguage{english}one hundred} \textcolor{PineGreen}{\selectlanguage{french}cent}  
 ¶ \textcolor{darkblue}{\textbf{\ipa{ɖɯ˧-ɕi˧ kʰv̩˧˥}}} \zh{一百年} \textcolor{Sepia}{\selectlanguage{english}one hundred years} \textcolor{PineGreen}{\selectlanguage{french}cent ans, un siècle}  
 ¶ \textcolor{darkblue}{\textbf{\ipa{ɖɯ˧-ɕi˧ kʰv̩˧\textasciitilde{}ɖɯ˥-ɕi˩ kʰv̩˩}}} \zh{一百年又一百年} \textcolor{Sepia}{\selectlanguage{english}century after century} \textcolor{PineGreen}{\selectlanguage{french}siècle après siècle}  
 ¶ \textcolor{darkblue}{\textbf{\ipa{ɕi˧-kʰv̩˧˥}}} \zh{百年(“一百年”的省略说法)} \textcolor{Sepia}{\selectlanguage{english}a century, one hundred years (abridged formulation)} \textcolor{PineGreen}{\selectlanguage{french}cent ans (formulation abrégée)}  

\lhead{\firstmark}
\rhead{\botmark}

\subsection{\hspace{-0.5cm} {\Large \textcolor{darkblue}{\textbf{\ipa{ɕi˥\textsubscript{b}}}}}\hspace{0.5cm}[\kern2pt{\textcolor{darkblue}{\textbf{\ipa{ɕi˥}}}}\kern2pt]} \hypertarget{s£i\string_Tb1}{}
\markboth{\textcolor{darkblue}{\textbf{\ipa{ɕi˥\textsubscript{b}}}}}{}
\textcolor{teal}{\zh{量词}} \hspace{4pt} \zh{声调类:} H\textsubscript{b}.
\zh{量词:分(一分钱)。} \textcolor{Sepia}{\selectlanguage{english}One hundredth of a yuan, one penny.} \textcolor{PineGreen}{\selectlanguage{french}Centième d'unité monétaire.} 
\lhead{\firstmark}
\rhead{\botmark}

\subsection{\hspace{-0.5cm} {\Large \textcolor{darkblue}{\textbf{\ipa{ɕi˧}}}}\hspace{0.5cm}[\kern2pt{\textcolor{darkblue}{\textbf{\ipa{ɕi˩˥}}}}\kern2pt]} \hypertarget{s£i\string_M1}{}
\markboth{\textcolor{darkblue}{\textbf{\ipa{ɕi˧}}}}{}
\textcolor{teal}{\zh{名词}} \hspace{4pt} \zh{声调类:} M.
\zh{米(单音节)。} \textcolor{Sepia}{\selectlanguage{english}Rice (monosyllable).} \textcolor{PineGreen}{\selectlanguage{french}Riz (monosyllabe).} 
\lhead{\firstmark}
\rhead{\botmark}

\subsection{\hspace{-0.5cm} {\Large \textcolor{darkblue}{\textbf{\ipa{ɕi˧ɕi˩-lo˩}}}}\hspace{0.5cm}[\kern2pt{\textcolor{darkblue}{\textbf{\ipa{xxxx non-correspondance entre le nombre de morphèmes et le nombre de tons de morphèmes}}}}\kern2pt]} \hypertarget{s£i\string_Ms£i\string_B-lo\string_B1}{}
\markboth{\textcolor{darkblue}{\textbf{\ipa{ɕi˧ɕi˩-lo˩}}}}{}
\textcolor{teal}{\zh{名词}} \hspace{4pt} \zh{声调类:} L\#-.
\zh{最细的肋骨。} \textcolor{Sepia}{\selectlanguage{english}The smallest cutlets.} \textcolor{PineGreen}{\selectlanguage{french}Les plus petites des côtelettes.} 
\lhead{\firstmark}
\rhead{\botmark}

\subsection{\hspace{-0.5cm} {\Large \textcolor{darkblue}{\textbf{\ipa{ɕi˧-ho˩ʂɯ˩}}}}\hspace{0.5cm}[\kern2pt{\textcolor{darkblue}{\textbf{\ipa{xxxx non-correspondance entre le nombre de morphèmes et le nombre de tons de morphèmes}}}}\kern2pt]} \hypertarget{s£i\string_M-ho\string_Bs`M\string_B1}{}
\markboth{\textcolor{darkblue}{\textbf{\ipa{ɕi˧-ho˩ʂɯ˩}}}}{}
\textcolor{teal}{\zh{名词}} \hspace{4pt} \zh{声调类:} -L.
\zh{西红柿(汉语借词)。} \textcolor{Sepia}{\selectlanguage{english}Tomato.} \textcolor{PineGreen}{\selectlanguage{french}Tomate.}  \zh{【借词】} \zh{西红柿}

\lhead{\firstmark}
\rhead{\botmark}

\subsection{\hspace{-0.5cm} {\Large \textcolor{darkblue}{\textbf{\ipa{ɕi˧lv̩˧}}}}\hspace{0.5cm}[\kern2pt{\textcolor{darkblue}{\textbf{\ipa{ɕi˧lv̩˧}}}}\kern2pt]} \hypertarget{s£i\string_Mlv\string_=\string_M1}{}
\markboth{\textcolor{darkblue}{\textbf{\ipa{ɕi˧lv̩˧}}}}{}
\textcolor{teal}{\zh{名词}} \hspace{4pt} \zh{声调类:} M.
\zh{水田。} \textcolor{Sepia}{\selectlanguage{english}Paddy field.} \textcolor{PineGreen}{\selectlanguage{french}Champs de riz.}  \zh{量词}: \textcolor{darkblue}{\textbf{\ipa{pʰæ˧˥, kɤ˧˥}}} 
\lhead{\firstmark}
\rhead{\botmark}

\subsection{\hspace{-0.5cm} {\Large \textcolor{darkblue}{\textbf{\ipa{ɕi˧ɭɯ˧}}}}\hspace{0.5cm}[\kern2pt{\textcolor{darkblue}{\textbf{\ipa{xxxx non-correspondance entre le nombre de morphèmes et le nombre de tons de morphèmes}}}}\kern2pt]} \hypertarget{s£i\string_Ml\string_RM\string_M1}{}
\markboth{\textcolor{darkblue}{\textbf{\ipa{ɕi˧ɭɯ˧}}}}{}
\textcolor{teal}{\zh{名词}} \hspace{4pt} \zh{声调类:} M.
\zh{稻子,稻田。} \textcolor{Sepia}{\selectlanguage{english}Paddy rice; by extension: paddy field.} \textcolor{PineGreen}{\selectlanguage{french}Riz paddy; par extension: champ de riz.}  \zh{量词}: \textcolor{darkblue}{\textbf{\ipa{kɤ˧˥ (pour le grain)}}} \textcolor{darkblue}{\textbf{\ipa{pʰæ˧˥ (pour un champ)}}} 
\lhead{\firstmark}
\rhead{\botmark}

\subsection{\hspace{-0.5cm} {\Large \textcolor{darkblue}{\textbf{\ipa{ɕi˧ɭɯ˧-lv̩˧pʰv̩˩}}}}\hspace{0.5cm}[\kern2pt{\textcolor{darkblue}{\textbf{\ipa{xxxx non-correspondance entre le nombre de morphèmes et le nombre de tons de morphèmes}}}}\kern2pt]} \hypertarget{s£i\string_Ml\string_RM\string_M-lv\string_=\string_Mp\string_hv\string_=\string_B1}{}
\markboth{\textcolor{darkblue}{\textbf{\ipa{ɕi˧ɭɯ˧-lv̩˧pʰv̩˩}}}}{}
\textcolor{teal}{\zh{名词}} \hspace{4pt} \zh{声调类:} \mytextsc{L}\#.
\zh{水田。} \textcolor{Sepia}{\selectlanguage{english}Paddy field.} \textcolor{PineGreen}{\selectlanguage{french}Champs de riz.}  \zh{量词}: \textcolor{darkblue}{\textbf{\ipa{pʰæ˧˥, kɤ˧˥}}} 
\lhead{\firstmark}
\rhead{\botmark}

\subsection{\hspace{-0.5cm} {\Large \textcolor{darkblue}{\textbf{\ipa{ɕi˧tɕʰi\#˥}}}}\hspace{0.5cm}[\kern2pt{\textcolor{darkblue}{\textbf{\ipa{ɕi˧tɕʰi˧}}}}\kern2pt]} \hypertarget{s£i\string_Mts£\string_hi\#\string_T1}{}
\markboth{\textcolor{darkblue}{\textbf{\ipa{ɕi˧tɕʰi\#˥}}}}{}
\textcolor{teal}{\zh{名词}} \hspace{4pt} \zh{声调类:} \#H.
\zh{米糠。} \textcolor{Sepia}{\selectlanguage{english}Chaff; bran; husk (of rice).} \textcolor{PineGreen}{\selectlanguage{french}Son de riz.}  \zh{量词}: \textcolor{darkblue}{\textbf{\ipa{mɤ˩}}} 
\lhead{\firstmark}
\rhead{\botmark}

\subsection{\hspace{-0.5cm} {\Large \textcolor{darkblue}{\textbf{\ipa{ɕi˧ʈʂʰwæ˧}}}}\hspace{0.5cm}[\kern2pt{\textcolor{darkblue}{\textbf{\ipa{ɕi˧ʈʂʰwæ˧}}}}\kern2pt]} \hypertarget{s£i\string_Mt`s`\string_hw\{\string_M1}{}
\markboth{\textcolor{darkblue}{\textbf{\ipa{ɕi˧ʈʂʰwæ˧}}}}{}
\textcolor{teal}{\zh{名词}} \hspace{4pt} \zh{声调类:} M.
\zh{米。} \textcolor{Sepia}{\selectlanguage{english}Husked rice.} \textcolor{PineGreen}{\selectlanguage{french}Riz décortiqué.}  ¶ \textcolor{darkblue}{\textbf{\ipa{ɕi˧ʈʂʰwæ˧-hɑ˧}}} \zh{米饭} \textcolor{Sepia}{\selectlanguage{english}cooked rice; literally “cooked-rice food”, specifying the term \textcolor{darkblue}{\textbf{\ipa{/hɑ˥/}}}, which refers to food in general.} \textcolor{PineGreen}{\selectlanguage{french}riz cuit; littéralement: “nourriture-riz cuit”; formulation employée pour préciser le terme \textcolor{darkblue}{\textbf{\ipa{/hɑ˥/}}}, qui désigne toutes les nourritures.}  

\lhead{\firstmark}
\rhead{\botmark}

\subsection{\hspace{-0.5cm} {\Large \textcolor{darkblue}{\textbf{\ipa{ɕi˩}}}}\hspace{0.5cm}[\kern2pt{\textcolor{darkblue}{\textbf{\ipa{ɕi˧˥}}}}\kern2pt]} \hypertarget{s£i\string_B1}{}
\markboth{\textcolor{darkblue}{\textbf{\ipa{ɕi˩}}}}{}
\textcolor{teal}{\zh{动词}} \hspace{4pt} \zh{声调类:} L\textsubscript{a}?.
\textit{\zh{古语}} [\zh{古语}] \zh{怕、害怕。} \textcolor{Sepia}{\selectlanguage{english}To be afraid of.} \textcolor{PineGreen}{\selectlanguage{french}Craindre, avoir peur de. Verbe qui paraît suranné; il ne se trouve que dans quelques expressions.}  ¶ \textcolor{darkblue}{\textbf{\ipa{njɤ˧ | no˩ ɕi˩ tʰɑ˥-mɤ˩-ʝi˩! | njɤ˧ | no˩ ɖwæ˩ tʰɑ˥-mɤ˩-ʝi˩!}}} \zh{不要以为我害怕你!(挑衅的话)} \textcolor{Sepia}{\selectlanguage{english}Don't you fancy I am afraid of you! / Don't you imagine you frighten me!} \textcolor{PineGreen}{\selectlanguage{french}Ne va pas croire que tu me fasses peur! / Si tu crois que j'ai peur de toi! Si tu crois que je te crains/que tu me fais peur! (Formule de défi.)}  
 ¶ \textcolor{darkblue}{\textbf{\ipa{njɤ˧ | no˩ ɕi˩-mɤ˩-ʝi˥!}}} \zh{你不让我害怕 / 我不害怕你!} \textcolor{Sepia}{\selectlanguage{english}You don't frighten me! / I'm not afraid of you!} \textcolor{PineGreen}{\selectlanguage{french}Tu ne me fais pas peur!}  
 ¶ \textcolor{darkblue}{\textbf{\ipa{njɤ˧ | tʰv̩˧ ɕi˩-mɤ˩-ʝi˩!}}} \zh{我不怕他!} \textcolor{Sepia}{\selectlanguage{english}I'm not afraid of him!} \textcolor{PineGreen}{\selectlanguage{french}Il ne me fait pas peur!}  
 ¶ \textcolor{darkblue}{\textbf{\ipa{njɤ˧ | ʈʂʰɯ˧-v̩˧ do˧˥, | ʁo˧ ɕi˧˥ | ʐwæ˩˥!}}} \zh{我见他,非常害怕!} \textcolor{Sepia}{\selectlanguage{english}When I see him, I'm terribly afraid! / When I see him, I get frightened!} \textcolor{PineGreen}{\selectlanguage{french}Quand je le vois, j'ai peur!/Il me fait très peur!}  

\lhead{\firstmark}
\rhead{\botmark}

\subsection{\hspace{-0.5cm} {\Large \textcolor{darkblue}{\textbf{\ipa{ɕi˩dv̩˥}}}}\hspace{0.5cm}[\kern2pt{\textcolor{darkblue}{\textbf{\ipa{xxxx non-correspondance entre le nombre de morphèmes et le nombre de tons de morphèmes}}}}\kern2pt]} \hypertarget{s£i\string_Bdv\string_=\string_T1}{}
\markboth{\textcolor{darkblue}{\textbf{\ipa{ɕi˩dv̩˥}}}}{}
\textcolor{teal}{\zh{名词}} \hspace{4pt} \zh{声调类:} LH.
\zh{香,香火。} \textcolor{Sepia}{\selectlanguage{english}Incense.} \textcolor{PineGreen}{\selectlanguage{french}Encens; bâtonnet d'encens.}  ¶ \textcolor{darkblue}{\textbf{\ipa{ɕi˩dv̩˥ qæ˩}}} \zh{烧香} \textcolor{Sepia}{\selectlanguage{english}to burn incense} \textcolor{PineGreen}{\selectlanguage{french}brûler de l'encens}  

\lhead{\firstmark}
\rhead{\botmark}

\subsection{\hspace{-0.5cm} {\Large \textcolor{darkblue}{\textbf{\ipa{ɕi˩dzi˥}}}}\hspace{0.5cm}[\kern2pt{\textcolor{darkblue}{\textbf{\ipa{ɕi˩dzi˥}}}}\kern2pt]} \hypertarget{s£i\string_Bdzi\string_T1}{}
\markboth{\textcolor{darkblue}{\textbf{\ipa{ɕi˩dzi˥}}}}{}
\textcolor{teal}{\zh{名词}} \hspace{4pt} \zh{声调类:} LH.
\zh{柏树。} \textcolor{Sepia}{\selectlanguage{english}Cypress.} \textcolor{PineGreen}{\selectlanguage{french}Genévrier; arbre dont des branchages sont employés lors des rituels (suivant la tradition tibétaine).}  \zh{量词}: \textcolor{darkblue}{\textbf{\ipa{dzi˩}}} 
\lhead{\firstmark}
\rhead{\botmark}

\subsection{\hspace{-0.5cm} {\Large \textcolor{darkblue}{\textbf{\ipa{ɕi˩ʈʰæ˧˥}}} \textsubscript{1}}\hspace{0.5cm}[\kern2pt{\textcolor{darkblue}{\textbf{\ipa{ɕi˩ʈʰæ˧˥}}}}\kern2pt]} \hypertarget{s£i\string_Bt`\string_h\{\string_M\string_T1}{}
\markboth{\textcolor{darkblue}{\textbf{\ipa{ɕi˩ʈʰæ˧˥}}} \textsubscript{1}}{}
\textcolor{teal}{\zh{形容词}} \hspace{4pt} \zh{声调类:} LM+MH\#.
\zh{结巴。} \textcolor{Sepia}{\selectlanguage{english}To be a stammerer; to have a stammer.} \textcolor{PineGreen}{\selectlanguage{french}Bègue, qui a un bégaiement.}  ¶ \textcolor{darkblue}{\textbf{\ipa{ʈʂʰɯ˧ | ɖɯ˧-pi˧˥ | ɕi˩ʈʰæ˧˥}}} \zh{他有一点结巴。} \textcolor{Sepia}{\selectlanguage{english}(S)he has a stammer.} \textcolor{PineGreen}{\selectlanguage{french}Il/elle est un peu bègue.}  
 ¶ \textcolor{darkblue}{\textbf{\ipa{ʈʂʰɯ˧ | ɕi˩ʈʰæ˧-zo˥.}}} \zh{他很结巴。} \textcolor{Sepia}{\selectlanguage{english}(S)he stammers a lot.} \textcolor{PineGreen}{\selectlanguage{french}Il est très bègue / il bégaie beaucoup / il a un fort bégaiement.}  
\zh{~【参考】~} \hyperlink{}{\textcolor{darkblue}{\textbf{\ipa{ɕi˩ʈʰæ˧˥}}} \textsubscript{2}} 
\lhead{\firstmark}
\rhead{\botmark}

\subsection{\hspace{-0.5cm} {\Large \textcolor{darkblue}{\textbf{\ipa{ɕi˩ʈʰæ˧˥}}} \textsubscript{2}}\hspace{0.5cm}[\kern2pt{\textcolor{darkblue}{\textbf{\ipa{ɕi˩ʈʰæ˧˥}}}}\kern2pt]} \hypertarget{s£i\string_Bt`\string_h\{\string_M\string_T2}{}
\markboth{\textcolor{darkblue}{\textbf{\ipa{ɕi˩ʈʰæ˧˥}}} \textsubscript{2}}{}
\textcolor{teal}{\zh{名词}} \hspace{4pt} \zh{声调类:} LM+MH\#.
\zh{结巴。} \textcolor{Sepia}{\selectlanguage{english}Stammerer, stutterer.} \textcolor{PineGreen}{\selectlanguage{french}Bègue.}  ¶ \textcolor{darkblue}{\textbf{\ipa{ʈʂʰɯ˧ ɕi˩ʈʰæ˧ ɲi˥}}} \zh{他是结巴。} \textcolor{Sepia}{\selectlanguage{english}(S)he is a stammerer.} \textcolor{PineGreen}{\selectlanguage{french}Il/elle est bègue.}  
\zh{~【参考】~} \hyperlink{}{\textcolor{darkblue}{\textbf{\ipa{ɕi˩ʈʰæ˧˥}}} \textsubscript{1}} 
\lhead{\firstmark}
\rhead{\botmark}

\subsection{\hspace{-0.5cm} {\Large \textcolor{darkblue}{\textbf{\ipa{ɕi˩˥}}}}\hspace{0.5cm}[\kern2pt{\textcolor{darkblue}{\textbf{\ipa{ɕi˥}}}}\kern2pt]} \hypertarget{s£i\string_B\string_T1}{}
\markboth{\textcolor{darkblue}{\textbf{\ipa{ɕi˩˥}}}}{}
\textcolor{teal}{\zh{名词}} \hspace{4pt} \zh{声调类:} LH.
\zh{香(单音节)。} \textcolor{Sepia}{\selectlanguage{english}Incense (second syllable).} \textcolor{PineGreen}{\selectlanguage{french}Encens (monosyllabe).}  ¶ \textcolor{darkblue}{\textbf{\ipa{ɕi˩ qæ˧˥}}} \zh{烧香} \textcolor{Sepia}{\selectlanguage{english}to burn incense} \textcolor{PineGreen}{\selectlanguage{french}brûler de l'encens}  

\lhead{\firstmark}
\rhead{\botmark}

\subsection{\hspace{-0.5cm} {\Large \textcolor{darkblue}{\textbf{\ipa{ɕjɤ˥}}}}\hspace{0.5cm}[\kern2pt{\textcolor{darkblue}{\textbf{\ipa{ɕjɤ˥}}}}\kern2pt]} \hypertarget{s£j7\string_T1}{}
\markboth{\textcolor{darkblue}{\textbf{\ipa{ɕjɤ˥}}}}{}
\textcolor{teal}{\zh{动词}} \hspace{4pt} \zh{声调类:} H.
\zh{发明、想出、找到(办法)。} \textcolor{Sepia}{\selectlanguage{english}To invent, to think out/up, to come up with (an idea, a solution).} \textcolor{PineGreen}{\selectlanguage{french}Inventer, trouver.}  ¶ \textcolor{darkblue}{\textbf{\ipa{le˧-ɕjɤ˥}}} \zh{想出了} \textcolor{Sepia}{\selectlanguage{english}\mytextsc{accomp}} \textcolor{PineGreen}{\selectlanguage{french}\mytextsc{accomp}}  
 ¶ \textcolor{darkblue}{\textbf{\ipa{ʈʂʰɯ˧ | pæ˧˥hwɤ˧ | ɕjɤ˧ ɣɯ˧!}}} \zh{他很会想办法的!} \textcolor{Sepia}{\selectlanguage{english}(S)he knows to find solutions under all circumstances! / (S)he is good of finding solutions to all problems!} \textcolor{PineGreen}{\selectlanguage{french}Il/elle a une solution à tout/ sait trouver une solution en toutes circonstances!}  

\lhead{\firstmark}
\rhead{\botmark}

\subsection{\hspace{-0.5cm} {\Large \textcolor{darkblue}{\textbf{\ipa{ɕjɤ˧-bv̩˧nv̩˧}}}}\hspace{0.5cm}[\kern2pt{\textcolor{darkblue}{\textbf{\ipa{xxxx non-correspondance entre le nombre de morphèmes et le nombre de tons de morphèmes}}}}\kern2pt]} \hypertarget{s£j7\string_M-bv\string_=\string_Mnv\string_=\string_M1}{}
\markboth{\textcolor{darkblue}{\textbf{\ipa{ɕjɤ˧-bv̩˧nv̩˧}}}}{}
\textcolor{teal}{\zh{形容词}} \hspace{4pt} \zh{声调类:} M.
\zh{香(气味)。} \textcolor{Sepia}{\selectlanguage{english}Good (smell), fragrant.} \textcolor{PineGreen}{\selectlanguage{french}Bonne (odeur).}  ¶ \textcolor{darkblue}{\textbf{\ipa{ʈʂʰɯ˧ ɕjɤ˧-bv̩˧nv̩˧ ɲi˩.}}} \zh{这很香(气味香)。} \textcolor{Sepia}{\selectlanguage{english}It smells good.} \textcolor{PineGreen}{\selectlanguage{french}ça sent bon!}  

\lhead{\firstmark}
\rhead{\botmark}

\subsection{\hspace{-0.5cm} {\Large \textcolor{darkblue}{\textbf{\ipa{ɕjɤ˩\textasciitilde{}ɕjɤ˩}}}}\hspace{0.5cm}[\kern2pt{\textcolor{darkblue}{\textbf{\ipa{ɕjɤ˩ɕjɤ˩˥}}}}\kern2pt]} \hypertarget{s£j7\string_B~s£j7\string_B1}{}
\markboth{\textcolor{darkblue}{\textbf{\ipa{ɕjɤ˩\textasciitilde{}ɕjɤ˩}}}}{}
\textcolor{teal}{\zh{动词}} \hspace{4pt} \zh{声调类:} L.
\zh{欺负。} \textcolor{Sepia}{\selectlanguage{english}To browbeat, to ill-treat.} \textcolor{PineGreen}{\selectlanguage{french}Maltraiter.}  ¶ \textcolor{darkblue}{\textbf{\ipa{hĩ˧ ɕjɤ˥\textasciitilde{}ɕjɤ˩}}} \zh{欺负人} \textcolor{Sepia}{\selectlanguage{english}to ill-treat someone, to ill-treat people} \textcolor{PineGreen}{\selectlanguage{french}maltraiter quelqu'un}  
 ¶ \textcolor{darkblue}{\textbf{\ipa{no˧ | njɤ˩ ɕjɤ˩\textasciitilde{}ɕjɤ˩-mv̩˩-zo˩˥! / no˧ | njɤ˩ ɕjɤ˩\textasciitilde{}ɕjɤ˩˥!}}} \zh{你对我不好!你欺负我!} \textcolor{Sepia}{\selectlanguage{english}You are treating me badly! / You are bullying me!} \textcolor{PineGreen}{\selectlanguage{french}Vous me maltraitez!}  
 ¶ \textcolor{darkblue}{\textbf{\ipa{no˧ | njɤ˩ ɕjɤ˩\textasciitilde{}ɕjɤ˩-ze˥!}}} \zh{你欺负了我!} \textcolor{Sepia}{\selectlanguage{english}You have treated me badly! / You have bullied me!} \textcolor{PineGreen}{\selectlanguage{french}Vous m'avez maltraité!}  

\lhead{\firstmark}
\rhead{\botmark}

\subsection{\hspace{-0.5cm} {\Large \textcolor{darkblue}{\textbf{\ipa{ɕjɤ˩jo˩}}}}\hspace{0.5cm}[\kern2pt{\textcolor{darkblue}{\textbf{\ipa{ɕjɤ˩jo˩˥}}}}\kern2pt]} \hypertarget{s£j7\string_Bjo\string_B1}{}
\markboth{\textcolor{darkblue}{\textbf{\ipa{ɕjɤ˩jo˩}}}}{}
\textcolor{teal}{\zh{名词}} \hspace{4pt} \zh{声调类:} L.
\zh{贝母。} \textcolor{Sepia}{\selectlanguage{english}\textit{Fritillaria cirrhosa}.} \textcolor{PineGreen}{\selectlanguage{french}\textit{Fritillaria cirrhosa}.}  \zh{量词}: \textcolor{darkblue}{\textbf{\ipa{ɭɯ˧}}} 
\lhead{\firstmark}
\rhead{\botmark}

\subsection{\hspace{-0.5cm} {\Large \textcolor{darkblue}{\textbf{\ipa{ɕjɤ˩tʰv̩˧˥}}}}\hspace{0.5cm}[\kern2pt{\textcolor{darkblue}{\textbf{\ipa{ɕjɤ˩tʰv̩˧˥}}}}\kern2pt]} \hypertarget{s£j7\string_Bt\string_hv\string_=\string_M\string_T1}{}
\markboth{\textcolor{darkblue}{\textbf{\ipa{ɕjɤ˩tʰv̩˧˥}}}}{}
\textcolor{teal}{\zh{动词}} \hspace{4pt} \zh{声调类:} LM+MH\#.
\zh{骂,批评。} \textcolor{Sepia}{\selectlanguage{english}To insult; to criticize.} \textcolor{PineGreen}{\selectlanguage{french}Insulter, maudire, se moquer; réprimander, gronder.}  ¶ \textcolor{darkblue}{\textbf{\ipa{hĩ˧ ɕjɤ˥tʰv̩˩}}} \zh{骂人、批评人} \textcolor{Sepia}{\selectlanguage{english}to insult people; to criticize people} \textcolor{PineGreen}{\selectlanguage{french}insulter quelqu'un/ critiquer quelqu'un}  

\lhead{\firstmark}
\rhead{\botmark}

\subsection{\hspace{-0.5cm} {\Large \textcolor{darkblue}{\textbf{\ipa{ɕjɤ˧˥}}}}\hspace{0.5cm}[\kern2pt{\textcolor{darkblue}{\textbf{\ipa{ɕjɤ˧˥}}}}\kern2pt]} \hypertarget{s£j7\string_M\string_T1}{}
\markboth{\textcolor{darkblue}{\textbf{\ipa{ɕjɤ˧˥}}}}{}
\textcolor{teal}{\zh{动词}} \hspace{4pt} \zh{声调类:} MH.
\zh{尝试、体会、经过。} \textcolor{Sepia}{\selectlanguage{english}To try; to taste.} \textcolor{PineGreen}{\selectlanguage{french}Essayer, goûter, expérimenter.}  ¶ \textcolor{darkblue}{\textbf{\ipa{le˧-ɕjɤ˧-ze˥}}} \zh{试了} \textcolor{Sepia}{\selectlanguage{english}\mytextsc{accomp} \string_ \mytextsc{pfv}} \textcolor{PineGreen}{\selectlanguage{french}\mytextsc{accomp} \string_ \mytextsc{pfv}}  
 ¶ \textcolor{darkblue}{\textbf{\ipa{tso˧\textasciitilde{}tso˧ ɕjɤ˩}}} \zh{尝一个东西} \textcolor{Sepia}{\selectlanguage{english}to taste something} \textcolor{PineGreen}{\selectlanguage{french}goûter quelque chose}  
 ¶ \textcolor{darkblue}{\textbf{\ipa{no˧ ɖɯ˧-kʰwɤ˥ ɕjɤ˩!}}} \zh{你尝一口吧!} \textcolor{Sepia}{\selectlanguage{english}Have a taste! / Taste a bite!} \textcolor{PineGreen}{\selectlanguage{french}goûte un peu! goûte un morceau!}  
 ¶ \textcolor{darkblue}{\textbf{\ipa{ɖɯ˧-ɕjɤ˧-ɻ̍˥!}}} \zh{尝一尝吧! / 试一试吧!} \textcolor{Sepia}{\selectlanguage{english}Have a try!} \textcolor{PineGreen}{\selectlanguage{french}Goûte voir! / Essaie voir!}  

\lhead{\firstmark}
\rhead{\botmark}

\subsection{\hspace{-0.5cm} {\Large \textcolor{darkblue}{\textbf{\ipa{ɕjo˩li\#˥}}}}\hspace{0.5cm}[\kern2pt{\textcolor{darkblue}{\textbf{\ipa{ɕjo˩li˥}}}}\kern2pt]} \hypertarget{s£jo\string_Bli\#\string_T1}{}
\markboth{\textcolor{darkblue}{\textbf{\ipa{ɕjo˩li\#˥}}}}{}
\textcolor{teal}{\zh{名词}} \hspace{4pt} \zh{声调类:} LM+\#H.
\zh{笛子。} \textcolor{Sepia}{\selectlanguage{english}Flute.} \textcolor{PineGreen}{\selectlanguage{french}Flûte (type “flûte traversière” et non “flûte à bec”).}  \zh{量词}: \textcolor{darkblue}{\textbf{\ipa{ɭɯ˧}}} 
\lhead{\firstmark}
\rhead{\botmark}

\subsection{\hspace{-0.5cm} {\Large \textcolor{darkblue}{\textbf{\ipa{ɕɯ˩\textsubscript{a}}}}}\hspace{0.5cm}[\kern2pt{\textcolor{darkblue}{\textbf{\ipa{ɕɯ˩˥}}}}\kern2pt]} \hypertarget{s£M\string_Ba1}{}
\markboth{\textcolor{darkblue}{\textbf{\ipa{ɕɯ˩\textsubscript{a}}}}}{}
\textcolor{teal}{\zh{动词}} \hspace{4pt} \zh{声调类:} L\textsubscript{a}.
\zh{养。} \textcolor{Sepia}{\selectlanguage{english}To raise.} \textcolor{PineGreen}{\selectlanguage{french}Élever (terme plus relevé que \zh{ʐɤ˧}).}  ¶ \textcolor{darkblue}{\textbf{\ipa{ɕɯ˩zo\#˥}}} \zh{养儿} \textcolor{Sepia}{\selectlanguage{english}adopted child} \textcolor{PineGreen}{\selectlanguage{french}enfant adopté}  
 ¶ \textcolor{darkblue}{\textbf{\ipa{ho˧zo˧-ɕɯ˧zo˥, | æ̃˩ mɤ˧-tsɤ˧! | hĩ˧-zo˧mv˥, | ʐɤ˧ tʰɑ˧-mɤ˧-ʝi˧!}}} \zh{养的小雉,不会变成鸡!人家的孩子,不要养!(指的不是领养孤儿,而是养别人的孩子:无论多么关心孩子,他还是会更爱自己原来的家人。)} \textcolor{Sepia}{\selectlanguage{english}The adopted baby pheasant does not become a chicken (=does not become domesticated)! One should not bring up other people's children! (Proverb which does not apply to the adoption of children who have lost ties with their biological family, but to the adoption of children who remain in touch with their relatives: no matter how much care one puts into bringing them up, they remain more attached to their lineage.)} \textcolor{PineGreen}{\selectlanguage{french}Un bébé faisan qu'on élève chez soi ne devient pas un poulet (n'est pas domestiqué pour autant)! Il ne faut pas élever les enfants d'autrui! (Proverbe qui ne s'applique pas à l'adoption d'enfants qui ont perdu leurs attaches à leur famille biologique, mais à l'adoption d'enfants qui restent en contact avec leurs proches: quelque soin que l'on consacre à leur éducation, ils restent plus attachés à leur famille d'origine.)}  

\lhead{\firstmark}
\rhead{\botmark}

\newpage
\section*{\centering- \textcolor{darkblue}{\textbf{\ipa{d}}} -}
\subsection{\hspace{-0.5cm} {\Large \textcolor{darkblue}{\textbf{\ipa{dɑ˧ʝi˩}}}}\hspace{0.5cm}[\kern2pt{\textcolor{darkblue}{\textbf{\ipa{dɑ˧ʝi˩}}}}\kern2pt]} \hypertarget{dA\string_Mj££i\string_B1}{}
\markboth{\textcolor{darkblue}{\textbf{\ipa{dɑ˧ʝi˩}}}}{}
\textcolor{teal}{\zh{名词}} \hspace{4pt} \zh{声调类:} L\#.
\zh{骡子。} \textcolor{Sepia}{\selectlanguage{english}Mule.} \textcolor{PineGreen}{\selectlanguage{french}Mule.}  ¶ \textcolor{darkblue}{\textbf{\ipa{dɑ˧ʝi˩-dʑo˩, | ɖɯ˩mi˧ dʑo˧-kv̩˥-mæ˩! | ɖɯ˩zo˧ dʑo˧-kv̩˥-mæ˩!}}} \zh{骡子呢,有母骡子!(也)有公骡子! / 骡子,分母的和公的!(这个说明是给一个不懂畜牧业的城里人听)} \textcolor{Sepia}{\selectlanguage{english}As for mules, there exist female mules, (and) male mules! / Among mules, there is a distinction between females and males! (Explanation provided to a city dweller on a visit, who knew precious little about animal breeding.)} \textcolor{PineGreen}{\selectlanguage{french}Les mules, ça se répartit en mules mâles et mules femelles! / Il existe une distinction de sexe parmi les mules! (Explication fournie à un visiteur citadin peu au fait de l'élevage des animaux.)}  
 \zh{量词}: \textcolor{darkblue}{\textbf{\ipa{v̩˧}}} 
\lhead{\firstmark}
\rhead{\botmark}

\subsection{\hspace{-0.5cm} {\Large \textcolor{darkblue}{\textbf{\ipa{dɑ˧pɤ˧}}}}\hspace{0.5cm}[\kern2pt{\textcolor{darkblue}{\textbf{\ipa{dɑ˧pɤ˧}}}}\kern2pt]} \hypertarget{dA\string_Mp7\string_M1}{}
\markboth{\textcolor{darkblue}{\textbf{\ipa{dɑ˧pɤ˧}}}}{}
\textcolor{teal}{\zh{名词}} \hspace{4pt} \zh{声调类:} M.
\zh{宗教礼师。音译:达巴。} \textcolor{Sepia}{\selectlanguage{english}Priest of the local religion.} \textcolor{PineGreen}{\selectlanguage{french}Prêtre de la religion locale.}  ¶ \textcolor{darkblue}{\textbf{\ipa{dɑ˧pɤ˧ ʝi˧-hĩ˧ hĩ˧}}} \zh{当达巴的人} \textcolor{Sepia}{\selectlanguage{english}priest, person who performs the function of priest} \textcolor{PineGreen}{\selectlanguage{french}prêtre, personne qui joue le rôle de prêtre/qui est prêtre}  
 \zh{量词}: \textcolor{darkblue}{\textbf{\ipa{v̩˧}}} 
\lhead{\firstmark}
\rhead{\botmark}

\subsection{\hspace{-0.5cm} {\Large \textcolor{darkblue}{\textbf{\ipa{dɑ˧pv̩\#˥}}}}\hspace{0.5cm}[\kern2pt{\textcolor{darkblue}{\textbf{\ipa{dɑ˧pv̩˧}}}}\kern2pt]} \hypertarget{dA\string_Mpv\string_=\#\string_T1}{}
\markboth{\textcolor{darkblue}{\textbf{\ipa{dɑ˧pv̩\#˥}}}}{}
\textcolor{teal}{\zh{名词}} \hspace{4pt} \zh{声调类:} \#H.
\zh{主人。} \textcolor{Sepia}{\selectlanguage{english}Host.} \textcolor{PineGreen}{\selectlanguage{french}Maître de maison, hôte (personne qui accueille).}  ¶ \textcolor{darkblue}{\textbf{\ipa{ʑi˧dv̩˧ dɑ˧pv̩˧}}} \zh{家的主人} \textcolor{Sepia}{\selectlanguage{english}the family host, the member of the family who has the role of host} \textcolor{PineGreen}{\selectlanguage{french}l'hôte de la maison}  
 ¶ \textcolor{darkblue}{\textbf{\ipa{ʑi˧dv̩˧-ʝi˧-hĩ˧ dɑ˧pv̩˧}}} \zh{同上} \textcolor{Sepia}{\selectlanguage{english}ditto} \textcolor{PineGreen}{\selectlanguage{french}idem}  
 \zh{量词}: \textcolor{darkblue}{\textbf{\ipa{v̩˧}}} 
\lhead{\firstmark}
\rhead{\botmark}

\subsection{\hspace{-0.5cm} {\Large \textcolor{darkblue}{\textbf{\ipa{dɑ˧pʰo˥}}}}\hspace{0.5cm}[\kern2pt{\textcolor{darkblue}{\textbf{\ipa{dɑ˩pʰo˥}}}}\kern2pt]} \hypertarget{dA\string_Mp\string_ho\string_T1}{}
\markboth{\textcolor{darkblue}{\textbf{\ipa{dɑ˧pʰo˥}}}}{}
\textcolor{teal}{\zh{名词}} \hspace{4pt} \zh{声调类:} LH.
\zh{达坡(永宁的一个村落)。} \textcolor{Sepia}{\selectlanguage{english}Dapo.} \textcolor{PineGreen}{\selectlanguage{french}Dapo (nom de village).}  ¶ \textcolor{darkblue}{\textbf{\ipa{ɖæ˩ʂɯ\#˥, | ʈʂo˧ʂɯ\#˥, | bɤ˩tɕʰɯ˩˥, | dɑ˧pʰo˥, | bɤ˧dzi˩, | dze˧bo˧}}} \zh{永宁坝的六个村落,按传统排序:从距离泸沽湖最近的村落说起。} \textcolor{Sepia}{\selectlanguage{english}the six villages of the plain of Yongning, in traditional order: by order of increasing distance from the Lake} \textcolor{PineGreen}{\selectlanguage{french}les six villages de la plaine de Yongning, dans l'ordre, qui prend comme point d'origine le village le plus proche du Lac}  

\lhead{\firstmark}
\rhead{\botmark}

\subsection{\hspace{-0.5cm} {\Large \textcolor{darkblue}{\textbf{\ipa{dɑ˧ʁwɤ\#˥}}}}\hspace{0.5cm}[\kern2pt{\textcolor{darkblue}{\textbf{\ipa{dɑ˧ʁwɤ˧}}}}\kern2pt]} \hypertarget{dA\string_MRw7\#\string_T1}{}
\markboth{\textcolor{darkblue}{\textbf{\ipa{dɑ˧ʁwɤ\#˥}}}}{}
\textcolor{teal}{\zh{名词}} \hspace{4pt} \zh{声调类:} \#H.
\zh{一个村落,在前所的下游。据说那边的方言跟丽江坝比较接近。} \textcolor{Sepia}{\selectlanguage{english}A village downstream from Qiansuo; the language spoken there is reported to be close to that of the Yongning plain.} \textcolor{PineGreen}{\selectlanguage{french}Un village en aval de Qiansuo; la langue parlée là-bas serait relativement proche de celle de la plaine de Yongning.} 
\lhead{\firstmark}
\rhead{\botmark}

\subsection{\hspace{-0.5cm} {\Large \textcolor{darkblue}{\textbf{\ipa{dɑ˩}}}}\hspace{0.5cm}[\kern2pt{\textcolor{darkblue}{\textbf{\ipa{dɑ˩˥}}}}\kern2pt]} \hypertarget{dA\string_B1}{}
\markboth{\textcolor{darkblue}{\textbf{\ipa{dɑ˩}}}}{}
\textcolor{teal}{\zh{形容词}} \hspace{4pt} \zh{声调类:} L.
\textit{\zh{古语}} [\zh{古语}] \zh{幸福、平安、安好。} \textcolor{Sepia}{\selectlanguage{english}Happy.} \textcolor{PineGreen}{\selectlanguage{french}Heureux.}  ¶ \textcolor{darkblue}{\textbf{\ipa{mɤ˧-dɑ˩-qʰwɤ˩}}} \zh{悲情歌,讲述自己日子痛苦} \textcolor{Sepia}{\selectlanguage{english}Melancholy song; song telling of one's unhappiness. This is one of the genres of singing, lamenting one's hardships.} \textcolor{PineGreen}{\selectlanguage{french}chanson mélancolique, récit des malheurs}  
 ¶ \textcolor{darkblue}{\textbf{\ipa{mɤ˧-dɑ˩!}}} \zh{悲情歌的开头词} \textcolor{Sepia}{\selectlanguage{english}Introductory formula for melancholy songs, and sometimes at the beginning of stories. (The same formula is also used in the Laze language.)} \textcolor{PineGreen}{\selectlanguage{french}Formule employée en début de chanson mélancolique, et parfois au début d'un conte. (La même formule est en usage dans la langue lazé.)}  
 ¶ \textcolor{darkblue}{\textbf{\ipa{mɤ˧-dɑ˩-mi˩}}} \zh{同上} \textcolor{Sepia}{\selectlanguage{english}As above: same meaning as the form without a \zh{/-mi/} suffix.} \textcolor{PineGreen}{\selectlanguage{french}Comme ci-dessus: même sens que la forme sans suffixe \zh{/-mi/}.}  
 ¶ \textcolor{darkblue}{\textbf{\ipa{ɖwæ˧˥ | hɤ˩-dɑ˥! | ɖwæ˧˥ | hɤ˩˥!}}} \zh{很了不起啊!(情景:表扬一个小孩子成功地爬上了一个家具)} \textcolor{Sepia}{\selectlanguage{english}Good job! / Well done! (Context: compliment to a toddler who has climbed on a piece of furniture.)} \textcolor{PineGreen}{\selectlanguage{french}Bravo, bravo! (Contexte: compliment saluant l'exploit d'une petite fille parvenue à grimper sur un meuble.)}  
 ¶ \textcolor{darkblue}{\textbf{\ipa{ɖʐɯ˩dɑ˥-kʰɤ˩dɑ˩-ɻ̍˩}}} \zh{一切都安好。(如:来指一段时间没有饥荒、地震、流行病、战争等灾难)} \textcolor{Sepia}{\selectlanguage{english}All is well. / All is for the best. (Used for instance to describe a period without food shortage, earthquake, epidemic, war or other catastrophe)} \textcolor{PineGreen}{\selectlanguage{french}Tout va bien, tout est pour le mieux. (S'emploie par exemple pour décrire une période sans disette, ni tremblement de terre, ni épidémie, ni guerre ou autre catastrophe.)}  

\lhead{\firstmark}
\rhead{\botmark}

\subsection{\hspace{-0.5cm} {\Large \textcolor{darkblue}{\textbf{\ipa{dɑ˩\textsubscript{b}}}}}\hspace{0.5cm}[\kern2pt{\textcolor{darkblue}{\textbf{\ipa{dɑ˩˥}}}}\kern2pt]} \hypertarget{dA\string_Bb1}{}
\markboth{\textcolor{darkblue}{\textbf{\ipa{dɑ˩\textsubscript{b}}}}}{}
\textcolor{teal}{\zh{动词}} \hspace{4pt} \zh{声调类:} L\textsubscript{b}.
\zh{织。} \textcolor{Sepia}{\selectlanguage{english}To weave.} \textcolor{PineGreen}{\selectlanguage{french}Tisser.}  ¶ \textcolor{darkblue}{\textbf{\ipa{ɣɯ˧ dɑ˩}}} \zh{织布} \textcolor{Sepia}{\selectlanguage{english}to weave fabric} \textcolor{PineGreen}{\selectlanguage{french}tisser du tissu}  
 ¶ \textcolor{darkblue}{\textbf{\ipa{ɣɯ˧ | le˧-dɑ˩}}} \zh{织布} \textcolor{Sepia}{\selectlanguage{english}to weave fabric} \textcolor{PineGreen}{\selectlanguage{french}tisser du tissu}  
 ¶ \textcolor{darkblue}{\textbf{\ipa{tso˧\textasciitilde{}tso˧ dɑ˩}}} \zh{织东西} \textcolor{Sepia}{\selectlanguage{english}to weave things} \textcolor{PineGreen}{\selectlanguage{french}tisser des choses}  
 ¶ \textcolor{darkblue}{\textbf{\ipa{ɖɯ˧-dɑ˧\textasciitilde{}dɑ˩-ɻ̍˩}}} \zh{织一下} \textcolor{Sepia}{\selectlanguage{english}\mytextsc{delimitative} \string_ \mytextsc{red} \mytextsc{inceptive}} \textcolor{PineGreen}{\selectlanguage{french}\mytextsc{délimitatif} \string_ \mytextsc{red} \mytextsc{inchoatif}}  

\lhead{\firstmark}
\rhead{\botmark}

\subsection{\hspace{-0.5cm} {\Large \textcolor{darkblue}{\textbf{\ipa{dɑ˩kʰɤ˩}}}}\hspace{0.5cm}[\kern2pt{\textcolor{darkblue}{\textbf{\ipa{dɑ˩kʰɤ˩˥}}}}\kern2pt]} \hypertarget{dA\string_Bk\string_h7\string_B1}{}
\markboth{\textcolor{darkblue}{\textbf{\ipa{dɑ˩kʰɤ˩}}}}{}
\textcolor{teal}{\zh{名词}} \hspace{4pt} \zh{声调类:} L.
\zh{鼓。} \textcolor{Sepia}{\selectlanguage{english}Drum.} \textcolor{PineGreen}{\selectlanguage{french}Tambour.}  ¶ \textcolor{darkblue}{\textbf{\ipa{dɑ˩kʰɤ˩ lɑ˥(-ze˩)}}} \zh{打鼓} \textcolor{Sepia}{\selectlanguage{english}to play a drum} \textcolor{PineGreen}{\selectlanguage{french}jouer du tambour}  
 \zh{量词}: \textcolor{darkblue}{\textbf{\ipa{ɭɯ˧}}} 
\lhead{\firstmark}
\rhead{\botmark}

\subsection{\hspace{-0.5cm} {\Large \textcolor{darkblue}{\textbf{\ipa{dɑ˩to\#˥}}}}\hspace{0.5cm}[\kern2pt{\textcolor{darkblue}{\textbf{\ipa{dɑ˩to˥}}}}\kern2pt]} \hypertarget{dA\string_Bto\#\string_T1}{}
\markboth{\textcolor{darkblue}{\textbf{\ipa{dɑ˩to\#˥}}}}{}
\textcolor{teal}{\zh{助词}} \hspace{4pt} \zh{声调类:} LM+\#H.
\zh{客气地。} \textcolor{Sepia}{\selectlanguage{english}Politely. This term was only observed in association with the verb 'to say', with the meaning 'to say polite words, polite small-talk'.} \textcolor{PineGreen}{\selectlanguage{french}Poliment.}  ¶ \textcolor{darkblue}{\textbf{\ipa{dɑ˩to˧ ʐwɤ˧˥}}} \zh{说客气话} \textcolor{Sepia}{\selectlanguage{english}to say some polite things} \textcolor{PineGreen}{\selectlanguage{french}faire des politesses}  
 ¶ \textcolor{darkblue}{\textbf{\ipa{dɑ˩to˧ ʐwɤ˧-hĩ˥-lɑ˩ ɲi˩!}}} \zh{这只是客气话而已!} \textcolor{Sepia}{\selectlanguage{english}It's just polite words! (Comment about an invitation by a neighbour, which was intended to be declined: it was not a true invitation.)} \textcolor{PineGreen}{\selectlanguage{french}C'est juste pour être poli! / C'est juste une façon de dire! (Commentaire de quelqu'un au sujet d'une invitation lancée par un voisin, qui est une simple politesse et pas une vraie invitation; il convient de la décliner.)}  

\lhead{\firstmark}
\rhead{\botmark}

\subsection{\hspace{-0.5cm} {\Large \textcolor{darkblue}{\textbf{\ipa{dɑ˩to˩}}}}\hspace{0.5cm}[\kern2pt{\textcolor{darkblue}{\textbf{\ipa{dɑ˩to˩˥}}}}\kern2pt]} \hypertarget{dA\string_Bto\string_B1}{}
\markboth{\textcolor{darkblue}{\textbf{\ipa{dɑ˩to˩}}}}{}
\textcolor{teal}{\zh{助词}} \hspace{4pt} \zh{声调类:} L.
\zh{说到底,根本上,归根结底。} \textcolor{Sepia}{\selectlanguage{english}At bottom, in reality, when all is said and done.} \textcolor{PineGreen}{\selectlanguage{french}Au fond, en réalité, en définitive.} 
\lhead{\firstmark}
\rhead{\botmark}

\subsection{\hspace{-0.5cm} {\Large \textcolor{darkblue}{\textbf{\ipa{dɑ˧˥}}}}\hspace{0.5cm}[\kern2pt{\textcolor{darkblue}{\textbf{\ipa{dɑ˧˥}}}}\kern2pt]} \hypertarget{dA\string_M\string_T1}{}
\markboth{\textcolor{darkblue}{\textbf{\ipa{dɑ˧˥}}}}{}
\textcolor{teal}{\zh{名词}} \hspace{4pt} \zh{声调类:} MH.
\textcolor{Sepia}{\selectlanguage{english}Misfortune, mishaps.} \textcolor{PineGreen}{\selectlanguage{french}Infortune, malheur.} \zh{当地汉语方言:}\zh{苦。} ¶ \textcolor{darkblue}{\textbf{\ipa{dɑ˧-ʐwɤ˧˥}}} \zh{诉苦} \textcolor{Sepia}{\selectlanguage{english}to complain, to tell one's misfortunes} \textcolor{PineGreen}{\selectlanguage{french}se plaindre de son infortune, gémir sur son sort}  
 ¶ \textcolor{darkblue}{\textbf{\ipa{ɻ̃˧-ʐwɤ˧ | dɑ˧-ʐwɤ˧-ɻ̍˥}}} \zh{讲自己的不幸} \textcolor{Sepia}{\selectlanguage{english}to bemoan one's misfortunes} \textcolor{PineGreen}{\selectlanguage{french}raconter ses malheurs; se plaindre}  
 ¶ \textcolor{darkblue}{\textbf{\ipa{ʈʂʰɯ˧ | mɑ˧dɑ˩-qʰwɤ˩, | ɻ̃˧-ʐwɤ˧ | dɑ˧-ʐwɤ˧-ɻ̍˥!}}} \zh{他不幸福,他一直在诉苦!} \textcolor{Sepia}{\selectlanguage{english}(S)he is unhappy; (s)he is constantly complaining!} \textcolor{PineGreen}{\selectlanguage{french}Il est malheureux; il passe son temps à se plaindre!}  

\lhead{\firstmark}
\rhead{\botmark}

\subsection{\hspace{-0.5cm} {\Large \textcolor{darkblue}{\textbf{\ipa{dɑ˧˥}}} \textsubscript{1}}\hspace{0.5cm}[\kern2pt{\textcolor{darkblue}{\textbf{\ipa{dɑ˧˥}}}}\kern2pt]} \hypertarget{dA\string_M\string_T1}{}
\markboth{\textcolor{darkblue}{\textbf{\ipa{dɑ˧˥}}} \textsubscript{1}}{}
\textcolor{teal}{\zh{动词}} \hspace{4pt} \zh{声调类:} MH.
\zh{建(房子)。} \textcolor{Sepia}{\selectlanguage{english}To build (a house...).} \textcolor{PineGreen}{\selectlanguage{french}Construire (une maison…).}  ¶ \textcolor{darkblue}{\textbf{\ipa{ʑi˧mi˧ dɑ˧˥}}} \zh{建房} \textcolor{Sepia}{\selectlanguage{english}to build a house} \textcolor{PineGreen}{\selectlanguage{french}construire une maison}  

\lhead{\firstmark}
\rhead{\botmark}

\subsection{\hspace{-0.5cm} {\Large \textcolor{darkblue}{\textbf{\ipa{dɑ˧˥}}} \textsubscript{2}}\hspace{0.5cm}[\kern2pt{\textcolor{darkblue}{\textbf{\ipa{dɑ˧˥}}}}\kern2pt]} \hypertarget{dA\string_M\string_T2}{}
\markboth{\textcolor{darkblue}{\textbf{\ipa{dɑ˧˥}}} \textsubscript{2}}{}
\textcolor{teal}{\zh{动词}} \hspace{4pt} \zh{声调类:} MH.
\zh{砍(树),割(肉)。} \textcolor{Sepia}{\selectlanguage{english}To fell (a tree); to cut into pieces (a large piece of meat); to create a breach (in a dike).} \textcolor{PineGreen}{\selectlanguage{french}Couper un arbre, abattre un arbre; ouvrir une brèche (dans une digue).}  ¶ \textcolor{darkblue}{\textbf{\ipa{le˧-dɑ˧-ze˥}}} \zh{砍了(树),割了(肉)} \textcolor{Sepia}{\selectlanguage{english}\mytextsc{accomp} \string_ \mytextsc{pfv}} \textcolor{PineGreen}{\selectlanguage{french}\mytextsc{accomp} \string_ \mytextsc{pfv}}  
 ¶ \textcolor{darkblue}{\textbf{\ipa{ɖɯ˧-dɑ˧ tʰi˥-dɑ˩}}} \zh{砍一下} \textcolor{Sepia}{\selectlanguage{english}to hit a blow} \textcolor{PineGreen}{\selectlanguage{french}donner un coup}  
 ¶ \textcolor{darkblue}{\textbf{\ipa{dɑ˩\textasciitilde{}dɑ˧˥}}} \zh{\mytextsc{重叠}} \textcolor{Sepia}{\selectlanguage{english}\mytextsc{red}} \textcolor{PineGreen}{\selectlanguage{french}\mytextsc{red}}  
 ¶ \textcolor{darkblue}{\textbf{\ipa{le˧-dɑ˩\textasciitilde{}dɑ˩(-ze˩)}}} \zh{(我把一只鸡)割成块了} \textcolor{Sepia}{\selectlanguage{english}(I) have cut (e.g. a chicken) into pieces} \textcolor{PineGreen}{\selectlanguage{french}(j'ai) découpé (ex.: le poulet) en morceaux}  
 ¶ \textcolor{darkblue}{\textbf{\ipa{ʂe˧ dɑ˥\textasciitilde{}dɑ˩}}} \zh{把肉剁碎} \textcolor{Sepia}{\selectlanguage{english}to cut meat to pieces, to mince meat} \textcolor{PineGreen}{\selectlanguage{french}hacher de la viande, couper de la viande en morceaux}  

\lhead{\firstmark}
\rhead{\botmark}

\subsection{\hspace{-0.5cm} {\Large \textcolor{darkblue}{\textbf{\ipa{dɑ˧˥\textsubscript{b}}}}}\hspace{0.5cm}[\kern2pt{\textcolor{darkblue}{\textbf{\ipa{dɑ˧˥}}}}\kern2pt]} \hypertarget{dA\string_M\string_Tb1}{}
\markboth{\textcolor{darkblue}{\textbf{\ipa{dɑ˧˥\textsubscript{b}}}}}{}
\textcolor{teal}{\zh{量词}} \hspace{4pt} \zh{声调类:} MH\textsubscript{b}.
\zh{量词:下(打一下)。} \textcolor{Sepia}{\selectlanguage{english}Self-classifier for blows.} \textcolor{PineGreen}{\selectlanguage{french}Auto-classificateur des coups.}  ¶ \textcolor{darkblue}{\textbf{\ipa{ɖɯ˧-dɑ˧˥}}} \zh{当头一棒} \textcolor{Sepia}{\selectlanguage{english}a blow} \textcolor{PineGreen}{\selectlanguage{french}un coup}  
 ¶ \textcolor{darkblue}{\textbf{\ipa{ɖɯ˧-dɑ˧ tʰi˥-dɑ˩}}} \zh{打一下} \textcolor{Sepia}{\selectlanguage{english}to strike a blow, to give a blow} \textcolor{PineGreen}{\selectlanguage{french}donner un coup}  

\lhead{\firstmark}
\rhead{\botmark}

\subsection{\hspace{-0.5cm} {\Large \textcolor{darkblue}{\textbf{\ipa{dɤ˧-qo˧}}}}\hspace{0.5cm}[\kern2pt{\textcolor{darkblue}{\textbf{\ipa{xxxx non-correspondance entre le nombre de morphèmes et le nombre de tons de morphèmes}}}}\kern2pt]} \hypertarget{d7\string_M-qo\string_M1}{}
\markboth{\textcolor{darkblue}{\textbf{\ipa{dɤ˧-qo˧}}}}{}
\textcolor{teal}{\zh{助词}} \hspace{4pt} \zh{声调类:} M.
\zh{那里(远指)。} \textcolor{Sepia}{\selectlanguage{english}Way over there.} \textcolor{PineGreen}{\selectlanguage{french}Par là-bas tout au loin, tout au loin là-bas.} 
\lhead{\firstmark}
\rhead{\botmark}

\subsection{\hspace{-0.5cm} {\Large \textcolor{darkblue}{\textbf{\ipa{dɤ˧-tʰv̩˧-gi\#˥}}}}\hspace{0.5cm}[\kern2pt{\textcolor{darkblue}{\textbf{\ipa{xxxx non-correspondance entre le nombre de morphèmes et le nombre de tons de morphèmes}}}}\kern2pt]} \hypertarget{d7\string_M-t\string_hv\string_=\string_M-gi\#\string_T1}{}
\markboth{\textcolor{darkblue}{\textbf{\ipa{dɤ˧-tʰv̩˧-gi\#˥}}}}{}
\textcolor{teal}{\zh{助词}} \hspace{4pt} \zh{声调类:} \#H.
\zh{那边(远指)。} \textcolor{Sepia}{\selectlanguage{english}Way over there.} \textcolor{PineGreen}{\selectlanguage{french}Au loin, de ce côté-là.} 
\lhead{\firstmark}
\rhead{\botmark}

\subsection{\hspace{-0.5cm} {\Large \textcolor{darkblue}{\textbf{\ipa{dɤ˧-tʰv̩˧qo˧}}}}\hspace{0.5cm}[\kern2pt{\textcolor{darkblue}{\textbf{\ipa{xxxx non-correspondance entre le nombre de morphèmes et le nombre de tons de morphèmes}}}}\kern2pt]} \hypertarget{d7\string_M-t\string_hv\string_=\string_Mqo\string_M1}{}
\markboth{\textcolor{darkblue}{\textbf{\ipa{dɤ˧-tʰv̩˧qo˧}}}}{}
\textcolor{teal}{\zh{助词}} \hspace{4pt} \zh{声调类:} M.
\zh{那边(远指)。} \textcolor{Sepia}{\selectlanguage{english}Way over there.} \textcolor{PineGreen}{\selectlanguage{french}Par là-bas tout au loin.} \zh{~【参考】~} \hyperlink{}{\textcolor{darkblue}{\textbf{\ipa{dɤ˧-ʈʂʰɯ˧qo˧}}}} 
\lhead{\firstmark}
\rhead{\botmark}

\subsection{\hspace{-0.5cm} {\Large \textcolor{darkblue}{\textbf{\ipa{dɤ˧-ʈʂʰɯ˧qo˧}}}}\hspace{0.5cm}[\kern2pt{\textcolor{darkblue}{\textbf{\ipa{xxxx non-correspondance entre le nombre de morphèmes et le nombre de tons de morphèmes}}}}\kern2pt]} \hypertarget{d7\string_M-t`s`\string_hM\string_Mqo\string_M1}{}
\markboth{\textcolor{darkblue}{\textbf{\ipa{dɤ˧-ʈʂʰɯ˧qo˧}}}}{}
\textcolor{teal}{\zh{助词}} \hspace{4pt} \zh{声调类:} M.
\zh{那边(远指)。} \textcolor{Sepia}{\selectlanguage{english}Way over there.} \textcolor{PineGreen}{\selectlanguage{french}Par là-bas tout au loin.} \zh{~【参考】~} \hyperlink{}{\textcolor{darkblue}{\textbf{\ipa{dɤ˧-tʰv̩˧qo˧}}}} 
\lhead{\firstmark}
\rhead{\botmark}

\subsection{\hspace{-0.5cm} {\Large \textcolor{darkblue}{\textbf{\ipa{di˧mi˧}}}}\hspace{0.5cm}[\kern2pt{\textcolor{darkblue}{\textbf{\ipa{di˧mi˧}}}}\kern2pt]} \hypertarget{di\string_Mmi\string_M1}{}
\markboth{\textcolor{darkblue}{\textbf{\ipa{di˧mi˧}}}}{}
\textcolor{teal}{\zh{名词}} \hspace{4pt} \zh{声调类:} M.
\zh{平坝。} \textcolor{Sepia}{\selectlanguage{english}Large plain.} \textcolor{PineGreen}{\selectlanguage{french}Grande plaine.}  ¶ \textcolor{darkblue}{\textbf{\ipa{ɬi˧di˩-di˩mi˩}}} \zh{永宁坝} \textcolor{Sepia}{\selectlanguage{english}the plain of Yongning} \textcolor{PineGreen}{\selectlanguage{french}la plaine de Yongning}  
 \zh{量词}: \textcolor{darkblue}{\textbf{\ipa{di˩}}} 
\lhead{\firstmark}
\rhead{\botmark}

\subsection{\hspace{-0.5cm} {\Large \textcolor{darkblue}{\textbf{\ipa{di˧qo˧}}}}\hspace{0.5cm}[\kern2pt{\textcolor{darkblue}{\textbf{\ipa{di˧qo˧}}}}\kern2pt]} \hypertarget{di\string_Mqo\string_M1}{}
\markboth{\textcolor{darkblue}{\textbf{\ipa{di˧qo˧}}}}{}
\textcolor{teal}{\zh{名词}} \hspace{4pt} \zh{声调类:} M.
\zh{平坝。} \textcolor{Sepia}{\selectlanguage{english}Plain.} \textcolor{PineGreen}{\selectlanguage{french}Plaine.}  \zh{量词}: \textcolor{darkblue}{\textbf{\ipa{di˩}}} 
\lhead{\firstmark}
\rhead{\botmark}

\subsection{\hspace{-0.5cm} {\Large \textcolor{darkblue}{\textbf{\ipa{di˧ɻæ˧}}}}\hspace{0.5cm}[\kern2pt{\textcolor{darkblue}{\textbf{\ipa{di˧ɻæ˧}}}}\kern2pt]} \hypertarget{di\string_Mr£`\{\string_M1}{}
\markboth{\textcolor{darkblue}{\textbf{\ipa{di˧ɻæ˧}}}}{}
\textcolor{teal}{\zh{名词}} \hspace{4pt} \zh{声调类:} M.
\zh{平坝。} \textcolor{Sepia}{\selectlanguage{english}Plain.} \textcolor{PineGreen}{\selectlanguage{french}Plaine.}  \zh{量词}: \textcolor{darkblue}{\textbf{\ipa{di˩}}} 
\lhead{\firstmark}
\rhead{\botmark}

\subsection{\hspace{-0.5cm} {\Large \textcolor{darkblue}{\textbf{\ipa{‑di˩}}}}\hspace{0.5cm}[\kern2pt{\textcolor{darkblue}{\textbf{\ipa{di˩˥}}}}\kern2pt]} \hypertarget{‑di\string_B1}{}
\markboth{\textcolor{darkblue}{\textbf{\ipa{‑di˩}}}}{}
\textcolor{teal}{\zh{后缀}} \hspace{4pt} \zh{声调类:} L.
\zh{\mytextsc{名物化}/\mytextsc{处所格}/\mytextsc{目的格。}} \textcolor{Sepia}{\selectlanguage{english}Nominalizer; locative or purposive.} \textcolor{PineGreen}{\selectlanguage{french}Nominalisateur; locatif; purposif.}  ¶ \textcolor{darkblue}{\textbf{\ipa{tso˧\textasciitilde{}tso˧-tɕɯ˧-di˧˥}}} \zh{可以摆东西的(家具)} \textcolor{Sepia}{\selectlanguage{english}(piece of furniture/object) on which one can put things} \textcolor{PineGreen}{\selectlanguage{french}(meuble/objet) sur lequel on pose des choses}  

\lhead{\firstmark}
\rhead{\botmark}

\subsection{\hspace{-0.5cm} {\Large \textcolor{darkblue}{\textbf{\ipa{di˩\textsubscript{c}}}}}\hspace{0.5cm}[\kern2pt{\textcolor{darkblue}{\textbf{\ipa{di˩˥}}}}\kern2pt]} \hypertarget{di\string_Bc1}{}
\markboth{\textcolor{darkblue}{\textbf{\ipa{di˩\textsubscript{c}}}}}{}
\textcolor{teal}{\zh{量词}} \hspace{4pt} \zh{声调类:} L\textsubscript{c}.
\zh{量词:坝子、地方(一个)。} \textcolor{Sepia}{\selectlanguage{english}Self-classifier for plains, and places.} \textcolor{PineGreen}{\selectlanguage{french}Classificateur pour les plaines, les étendues de terre, les lieux.}  ¶ \textcolor{darkblue}{\textbf{\ipa{ɖɯ˧-v̩˧ | ɖɯ˧-di˩ hɯ˩}}} \zh{分开,每个人去不同的地方} \textcolor{Sepia}{\selectlanguage{english}to separate, each going their separate ways} \textcolor{PineGreen}{\selectlanguage{french}se séparer, partir chacun de son côté (par exemple: des frères se séparent et chacun va son chemin)}  

\lhead{\firstmark}
\rhead{\botmark}

\subsection{\hspace{-0.5cm} {\Large \textcolor{darkblue}{\textbf{\ipa{di˩\textsubscript{a}}}}}\hspace{0.5cm}[\kern2pt{\textcolor{darkblue}{\textbf{\ipa{di˩˥}}}}\kern2pt]} \hypertarget{di\string_Ba1}{}
\markboth{\textcolor{darkblue}{\textbf{\ipa{di˩\textsubscript{a}}}}}{}
\textcolor{teal}{\zh{动词}} \hspace{4pt} \zh{声调类:} L\textsubscript{a}.
\zh{存在动词:有,拥有。例如:有家,有污垢在衣服上,有长短区别(两个物品有长短区别)。} \textcolor{Sepia}{\selectlanguage{english}Existential verb: to have (a home); to have dirt on one's clothes; to have a different in length (two objects have a difference in length).} \textcolor{PineGreen}{\selectlanguage{french}Existentiel; posséder. Possession non amovible aussi bien que transitoire: avoir une maison, aussi bien que: avoir une tache de graisse sur la joue; avoir une différence de longueur (deux objets ont une différence de longueur).}  ¶ \textcolor{darkblue}{\textbf{\ipa{ʈʰɯ˧ | ɑ˩ʁo˧ mɤ˧-di˩-hĩ˩.}}} \zh{他没有家。} \textcolor{Sepia}{\selectlanguage{english}(S)he does not have a home. / (S)he is homeless.} \textcolor{PineGreen}{\selectlanguage{french}Elle/il n'a pas de maison, elle/il est sans domicile}  
 ¶ \textcolor{darkblue}{\textbf{\ipa{mɤ˧ tʰi˧-di˩}}} \zh{有油(例如:一个人的嘴巴周围油乎乎,有油)} \textcolor{Sepia}{\selectlanguage{english}there is grease (eg there is grease around the mouth of someone who has been biting away at large slabs of meat)} \textcolor{PineGreen}{\selectlanguage{french}il y a de la graisse (ex.: autour de la bouche de quelqu'un qui vient de croquer de la viande à belles dents)}  
 ¶ \textcolor{darkblue}{\textbf{\ipa{ɖɯ˧-kʰwɤ˧ di˥}}} \zh{有一块东西} \textcolor{Sepia}{\selectlanguage{english}there is something} \textcolor{PineGreen}{\selectlanguage{french}il y a quelque chose}  

\lhead{\firstmark}
\rhead{\botmark}

\subsection{\hspace{-0.5cm} {\Large \textcolor{darkblue}{\textbf{\ipa{di˩-gɤ˩lɑ˥}}}}\hspace{0.5cm}[\kern2pt{\textcolor{darkblue}{\textbf{\ipa{xxxx non-correspondance entre le nombre de morphèmes et le nombre de tons de morphèmes}}}}\kern2pt]} \hypertarget{di\string_B-g7\string_BlA\string_T1}{}
\markboth{\textcolor{darkblue}{\textbf{\ipa{di˩-gɤ˩lɑ˥}}}}{}
\textcolor{teal}{\zh{名词}} \hspace{4pt} \zh{声调类:} L+H\#.
\zh{地菩萨。} \textcolor{Sepia}{\selectlanguage{english}Earth spirit.} \textcolor{PineGreen}{\selectlanguage{french}Esprit de la terre, Bodhisattva terrestre.}  \zh{量词}: \textcolor{darkblue}{\textbf{\ipa{v̩˧}}} 
\lhead{\firstmark}
\rhead{\botmark}

\subsection{\hspace{-0.5cm} {\Large \textcolor{darkblue}{\textbf{\ipa{di˩li˩}}}}\hspace{0.5cm}[\kern2pt{\textcolor{darkblue}{\textbf{\ipa{di˩li˩˥}}}}\kern2pt]} \hypertarget{di\string_Bli\string_B1}{}
\markboth{\textcolor{darkblue}{\textbf{\ipa{di˩li˩}}}}{}
\textcolor{teal}{\zh{名词}} \hspace{4pt} \zh{声调类:} L.
\zh{海菜花。} \textcolor{Sepia}{\selectlanguage{english}Dandy: \textit{Ottelia Acuminata, Boottia acuminata, Ottelia yunnanensis}.} \textcolor{PineGreen}{\selectlanguage{french}Ottélie: une plante verte comestible: \textit{Ottelia Acuminata, Boottia acuminata, Ottelia yunnanensis}.} \zh{当地汉语方言:}\zh{龙爪菜。} ¶ \textcolor{darkblue}{\textbf{\ipa{di˩li˩-ʁo˩bv̩˥ (ton: L+H\#)}}} \zh{海菜花的萌芽} \textcolor{Sepia}{\selectlanguage{english}dandy shoots} \textcolor{PineGreen}{\selectlanguage{french}pousses d'ottélie}  
 ¶ \textcolor{darkblue}{\textbf{\ipa{di˩li˩-ʁo˩bv̩˥ hwæ˩}}} \zh{买海菜花萌芽} \textcolor{Sepia}{\selectlanguage{english}to buy dandy shoots} \textcolor{PineGreen}{\selectlanguage{french}acheter des pousses d'ottélie (verbe au ton M)}  
 ¶ \textcolor{darkblue}{\textbf{\ipa{di˩li˩-ʁo˩bv̩˥ tɕʰi˩}}} \zh{卖海菜花萌芽} \textcolor{Sepia}{\selectlanguage{english}to sell dandy shoots} \textcolor{PineGreen}{\selectlanguage{french}vendre des pousses d'ottélie}  
 ¶ \textcolor{darkblue}{\textbf{\ipa{di˩li˩-ʁo˩bv̩˥ dzɯ˩}}} \zh{吃海菜花萌芽} \textcolor{Sepia}{\selectlanguage{english}to eat dandy shoots} \textcolor{PineGreen}{\selectlanguage{french}manger des pousses d'ottélie}  
 ¶ \textcolor{darkblue}{\textbf{\ipa{di˩li˩-ʁo˩bv̩˥ dze˩}}} \zh{割海菜花萌芽} \textcolor{Sepia}{\selectlanguage{english}to cut dandy shoots} \textcolor{PineGreen}{\selectlanguage{french}couper des pousses d'ottélie}  
 ¶ \textcolor{darkblue}{\textbf{\ipa{di˩li˩-ʁo˩bv̩˥ tɕɤ˩}}} \zh{煮海菜花萌芽} \textcolor{Sepia}{\selectlanguage{english}to boil dandy shoots} \textcolor{PineGreen}{\selectlanguage{french}faire bouillir des pousses d'ottélie}  
 \zh{量词}: \textcolor{darkblue}{\textbf{\ipa{qɑ˩}}} 
\lhead{\firstmark}
\rhead{\botmark}

\subsection{\hspace{-0.5cm} {\Large \textcolor{darkblue}{\textbf{\ipa{di˧˥}}} \textsubscript{1}}\hspace{0.5cm}[\kern2pt{\textcolor{darkblue}{\textbf{\ipa{di˧˥}}}}\kern2pt]} \hypertarget{di\string_M\string_T1}{}
\markboth{\textcolor{darkblue}{\textbf{\ipa{di˧˥}}} \textsubscript{1}}{}
\textcolor{teal}{\zh{动词}} \hspace{4pt} \zh{声调类:} MH.
\zh{打散,驱赶,撵,赶,打猎。} \textcolor{Sepia}{\selectlanguage{english}To hunt; to scatter, to drive out, to drive away.} \textcolor{PineGreen}{\selectlanguage{french}Poursuivre, chasser; disperser, repousser, faire déguerpir.}  ¶ \textcolor{darkblue}{\textbf{\ipa{tɕʰɯ˩di˩˥}}} \zh{赶麂子,狩猎} \textcolor{Sepia}{\selectlanguage{english}to hunt the muntjac; to hunt} \textcolor{PineGreen}{\selectlanguage{french}chasser le muntjac; chasser}  
 ¶ \textcolor{darkblue}{\textbf{\ipa{tɕʰɯ˩di˩-bi˩-ni˩gv̩˩˥}}} \zh{有打猎的习惯、喜欢打猎} \textcolor{Sepia}{\selectlanguage{english}to have a habit of hunting, to have a fondness for hunting} \textcolor{PineGreen}{\selectlanguage{french}avoir l'habitude de chasser}  
 ¶ \textcolor{darkblue}{\textbf{\ipa{di˩\textasciitilde{}di˧˥ / di˩\textasciitilde{}di˧-ze˥}}} \zh{\mytextsc{重叠:跟着、追着}} \textcolor{Sepia}{\selectlanguage{english}\mytextsc{red}: to hunt, to track} \textcolor{PineGreen}{\selectlanguage{french}\mytextsc{red}: suivre à la trace, pister}  
 ¶ \textcolor{darkblue}{\textbf{\ipa{tʰi˧-di˩\textasciitilde{}di˩}}} \zh{\mytextsc{dur} \mytextsc{red}} \textcolor{Sepia}{\selectlanguage{english}\mytextsc{dur} \mytextsc{red}} \textcolor{PineGreen}{\selectlanguage{french}\mytextsc{dur} \mytextsc{red}}  

\lhead{\firstmark}
\rhead{\botmark}

\subsection{\hspace{-0.5cm} {\Large \textcolor{darkblue}{\textbf{\ipa{di˧˥}}} \textsubscript{2}}\hspace{0.5cm}[\kern2pt{\textcolor{darkblue}{\textbf{\ipa{di˧˥}}}}\kern2pt]} \hypertarget{di\string_M\string_T2}{}
\markboth{\textcolor{darkblue}{\textbf{\ipa{di˧˥}}} \textsubscript{2}}{}
\textcolor{teal}{\zh{动词}} \hspace{4pt} \zh{声调类:} MH.
\zh{拉(肚子)。} \textcolor{Sepia}{\selectlanguage{english}To run; to have a runny belly = to have diarrhea.} \textcolor{PineGreen}{\selectlanguage{french}S'écouler, couler; avoir la courante = avoir la diarrhée.}  ¶ \textcolor{darkblue}{\textbf{\ipa{bi˧mi˧ di˧˥}}} \zh{拉肚子} \textcolor{Sepia}{\selectlanguage{english}to have diarrhea} \textcolor{PineGreen}{\selectlanguage{french}avoir la diarrhée}  

\lhead{\firstmark}
\rhead{\botmark}

\subsection{\hspace{-0.5cm} {\Large \textcolor{darkblue}{\textbf{\ipa{di˩˥}}}}\hspace{0.5cm}[\kern2pt{\textcolor{darkblue}{\textbf{\ipa{di˩˥}}}}\kern2pt]} \hypertarget{di\string_B\string_T1}{}
\markboth{\textcolor{darkblue}{\textbf{\ipa{di˩˥}}}}{}
\textcolor{teal}{\zh{名词}} \hspace{4pt} \zh{声调类:} LH.
\zh{地(天地的地)。} \textcolor{Sepia}{\selectlanguage{english}Earth (as in: the sky and the earth).} \textcolor{PineGreen}{\selectlanguage{french}Terre (le ciel et la terre).}  ¶ \textcolor{darkblue}{\textbf{\ipa{di˩ dv̩˩-ze˥}}} \zh{挖土} \textcolor{Sepia}{\selectlanguage{english}to dig the earth} \textcolor{PineGreen}{\selectlanguage{french}creuser la terre}  
 ¶ \textcolor{darkblue}{\textbf{\ipa{di˩ hwæ˧-ze˩}}} \zh{买了土} \textcolor{Sepia}{\selectlanguage{english}bought some earth} \textcolor{PineGreen}{\selectlanguage{french}a acheté de la terre}  
 ¶ \textcolor{darkblue}{\textbf{\ipa{ɖɯ˧-di˩ ɖɯ˩-bæ˩!}}} \zh{一个地方,一个样! = 每个地方有自己的特色(如:每个村落有自己的摩梭方言/土语)} \textcolor{Sepia}{\selectlanguage{english}Each place is different! (In particular, each place has its own pronunciation: its own dialect of the Na language)} \textcolor{PineGreen}{\selectlanguage{french}C'est différent à chaque endroit/chaque endroit a ses choses propres (par exemple, en matière de langues, chaque village a sa prononciation, son dialecte)}  
 \zh{量词}: \textcolor{darkblue}{\textbf{\ipa{di˩}}} 
\lhead{\firstmark}
\rhead{\botmark}

\subsection{\hspace{-0.5cm} {\Large \textcolor{darkblue}{\textbf{\ipa{do˥}}} \textsubscript{1}}\hspace{0.5cm}[\kern2pt{\textcolor{darkblue}{\textbf{\ipa{do˥}}}}\kern2pt]} \hypertarget{do\string_T1}{}
\markboth{\textcolor{darkblue}{\textbf{\ipa{do˥}}} \textsubscript{1}}{}
\textcolor{teal}{\zh{动词}} \hspace{4pt} \zh{声调类:} H.
\zh{爬,上去,上山。} \textcolor{Sepia}{\selectlanguage{english}To climb.} \textcolor{PineGreen}{\selectlanguage{french}Grimper, monter, escalader, gravir.}  ¶ \textcolor{darkblue}{\textbf{\ipa{ʈʂo˩bo˩ do˩˥}}} \zh{爬墙} \textcolor{Sepia}{\selectlanguage{english}to climb a wall} \textcolor{PineGreen}{\selectlanguage{french}grimper un mur, gravir un mur}  
 ¶ \textcolor{darkblue}{\textbf{\ipa{gɤ˩-do˧}}} \zh{爬上} \textcolor{Sepia}{\selectlanguage{english}to escalate, to climb up} \textcolor{PineGreen}{\selectlanguage{french}grimper, gravir}  
 ¶ \textcolor{darkblue}{\textbf{\ipa{ʁwɤ˩ do˩˥}}} \zh{爬山} \textcolor{Sepia}{\selectlanguage{english}to climb a mountain, to go hiking in the mountains} \textcolor{PineGreen}{\selectlanguage{french}gravir une montagne, grimper la montagne, faire de la montagne}  
 ¶ \textcolor{darkblue}{\textbf{\ipa{to˩ do˩˥}}} \zh{爬上一个山坡} \textcolor{Sepia}{\selectlanguage{english}to ascend a slope, to climb a slope} \textcolor{PineGreen}{\selectlanguage{french}grimper la pente/ grimper une pente}  

\lhead{\firstmark}
\rhead{\botmark}

\subsection{\hspace{-0.5cm} {\Large \textcolor{darkblue}{\textbf{\ipa{do˥}}} \textsubscript{2}}\hspace{0.5cm}[\kern2pt{\textcolor{darkblue}{\textbf{\ipa{do˥}}}}\kern2pt]} \hypertarget{do\string_T2}{}
\markboth{\textcolor{darkblue}{\textbf{\ipa{do˥}}} \textsubscript{2}}{}
\textcolor{teal}{\zh{动词}} \hspace{4pt} \zh{声调类:} H.
\zh{交配、交尾。} \textcolor{Sepia}{\selectlanguage{english}To mate with; to pair (of animal).} \textcolor{PineGreen}{\selectlanguage{french}S'accoupler (animaux).}  ¶ \textcolor{darkblue}{\textbf{\ipa{bo˩ɬɑ˥ | bo˩mi˧ do˧}}} \zh{公猪与母猪交配。} \textcolor{Sepia}{\selectlanguage{english}The pig mates with the sow.} \textcolor{PineGreen}{\selectlanguage{french}Le verrat s'accouple avec la truie.}  

\lhead{\firstmark}
\rhead{\botmark}

\subsection{\hspace{-0.5cm} {\Large \textcolor{darkblue}{\textbf{\ipa{do˥\textsubscript{a}}}}}\hspace{0.5cm}[\kern2pt{\textcolor{darkblue}{\textbf{\ipa{do˥}}}}\kern2pt]} \hypertarget{do\string_Ta1}{}
\markboth{\textcolor{darkblue}{\textbf{\ipa{do˥\textsubscript{a}}}}}{}
\textcolor{teal}{\zh{量词}} \hspace{4pt} \zh{声调类:} H\textsubscript{a}.
\zh{\mytextsc{量词}.墙壁(一堵)。} \textcolor{Sepia}{\selectlanguage{english}Classifier for partitions/walls.} \textcolor{PineGreen}{\selectlanguage{french}Classificateur des cloisons et murs.}  ¶ \textcolor{darkblue}{\textbf{\ipa{ʈʂʰɯ˧-do\#˥}}} \zh{这堵(墙壁)} \textcolor{Sepia}{\selectlanguage{english}this partition/wall} \textcolor{PineGreen}{\selectlanguage{french}cette cloison}  

\lhead{\firstmark}
\rhead{\botmark}

\subsection{\hspace{-0.5cm} {\Large \textcolor{darkblue}{\textbf{\ipa{do˧}}} \textsubscript{1}}\hspace{0.5cm}[\kern2pt{\textcolor{darkblue}{\textbf{\ipa{do˥}}}}\kern2pt]} \hypertarget{do\string_M1}{}
\markboth{\textcolor{darkblue}{\textbf{\ipa{do˧}}} \textsubscript{1}}{}
\textcolor{teal}{\zh{形容词}} \hspace{4pt} \zh{声调类:} M.
\zh{笨、愚蠢。} \textcolor{Sepia}{\selectlanguage{english}Stupid, silly, idiotic.} \textcolor{PineGreen}{\selectlanguage{french}Bête, stupide.}  ¶ \textcolor{darkblue}{\textbf{\ipa{zo˩ do˩˥}}} \zh{傻瓜} \textcolor{Sepia}{\selectlanguage{english}idiot, village idiot} \textcolor{PineGreen}{\selectlanguage{french}un idiot, un fou du village; perçu comme: “un homme qui n'a pas grandi”, “quelqu'un qui est resté enfant”}  

\lhead{\firstmark}
\rhead{\botmark}

\subsection{\hspace{-0.5cm} {\Large \textcolor{darkblue}{\textbf{\ipa{do˧}}} \textsubscript{2}}\hspace{0.5cm}[\kern2pt{\textcolor{darkblue}{\textbf{\ipa{do˥}}}}\kern2pt]} \hypertarget{do\string_M2}{}
\markboth{\textcolor{darkblue}{\textbf{\ipa{do˧}}} \textsubscript{2}}{}
\textcolor{teal}{\zh{形容词}} \hspace{4pt} \zh{声调类:} M.
\zh{不能生育。} \textcolor{Sepia}{\selectlanguage{english}Sterile.} \textcolor{PineGreen}{\selectlanguage{french}Stérile.} 
\lhead{\firstmark}
\rhead{\botmark}

\subsection{\hspace{-0.5cm} {\Large \textcolor{darkblue}{\textbf{\ipa{do˧bæ˧}}}}\hspace{0.5cm}[\kern2pt{\textcolor{darkblue}{\textbf{\ipa{do˧bæ˧}}}}\kern2pt]} \hypertarget{do\string_Mb\{\string_M1}{}
\markboth{\textcolor{darkblue}{\textbf{\ipa{do˧bæ˧}}}}{}
\textcolor{teal}{\zh{名词}} \hspace{4pt} \zh{声调类:} M.
\zh{大腿。} \textcolor{Sepia}{\selectlanguage{english}Thigh.} \textcolor{PineGreen}{\selectlanguage{french}Cuisse.}  ¶ \textcolor{darkblue}{\textbf{\ipa{do˧bæ˧ | ɖɯ˩-hĩ˩˥}}} \zh{大腿} \textcolor{Sepia}{\selectlanguage{english}thigh} \textcolor{PineGreen}{\selectlanguage{french}cuisse}  
 ¶ \textcolor{darkblue}{\textbf{\ipa{do˧bæ˧ | tɕi˩-hĩ˩˥}}} \zh{小腿} \textcolor{Sepia}{\selectlanguage{english}calf} \textcolor{PineGreen}{\selectlanguage{french}mollet}  
 \zh{量词}: \textcolor{darkblue}{\textbf{\ipa{ʈv̩˩}}} 
\lhead{\firstmark}
\rhead{\botmark}

\subsection{\hspace{-0.5cm} {\Large \textcolor{darkblue}{\textbf{\ipa{do˧bv̩˧}}}}\hspace{0.5cm}[\kern2pt{\textcolor{darkblue}{\textbf{\ipa{do˧bv̩˧}}}}\kern2pt]} \hypertarget{do\string_Mbv\string_=\string_M1}{}
\markboth{\textcolor{darkblue}{\textbf{\ipa{do˧bv̩˧}}}}{}
\textcolor{teal}{\zh{名词}} \hspace{4pt} \zh{声调类:} M.
\zh{屁股。} \textcolor{Sepia}{\selectlanguage{english}Buttocks.} \textcolor{PineGreen}{\selectlanguage{french}Fesse.}  \zh{量词}: \textcolor{darkblue}{\textbf{\ipa{ɭɯ˧}}} 
\lhead{\firstmark}
\rhead{\botmark}

\subsection{\hspace{-0.5cm} {\Large \textcolor{darkblue}{\textbf{\ipa{do˩}}}}\hspace{0.5cm}[\kern2pt{\textcolor{darkblue}{\textbf{\ipa{do˩˥}}}}\kern2pt]} \hypertarget{do\string_B1}{}
\markboth{\textcolor{darkblue}{\textbf{\ipa{do˩}}}}{}
\textcolor{teal}{\zh{形容词}} \hspace{4pt} \zh{声调类:} L.
\textit{\zh{古语}} [\zh{古语}] \zh{不成熟、晚熟。} \textcolor{Sepia}{\selectlanguage{english}Immature, lacking maturity.} \textcolor{PineGreen}{\selectlanguage{french}Immature.}  ¶ \textcolor{darkblue}{\textbf{\ipa{ŋwɤ˩ɬi˩-mi˩˥, | ʂe˧ mɤ˧-mv̩˥, | ʂe˧ do˧˥! | tsʰe˩ŋwɤ˩ kʰv̩˥, | zo˧ mɤ˧-ti˩, | zo˧ do˧˥!}}} \zh{五月份,谷物还是小草(还不出谷粒),算是晚熟!男人十五岁还不成熟(=还不见姑娘),算是晚熟!} \textcolor{Sepia}{\selectlanguage{english}In the fifth month, if cereals are still green (=if they do not yet yield grain), the crop is immature (and may not yield any harvest). At age 15, if a boy does not become an adult (=if a boy does not visit girls), he is immature (he is not developing normally)!} \textcolor{PineGreen}{\selectlanguage{french}Au cinquième mois, une céréale qui ne mûrit pas/qui ne donne pas de grain, c'est une récolte stérile/qui reste en herbe! A quinze ans, le garçon qui n'a pas encore acquis de maturité (=qui ne fréquente pas encore les filles), c'est qu'il a un problème/c'est un attardé!}  

\lhead{\firstmark}
\rhead{\botmark}

\subsection{\hspace{-0.5cm} {\Large \textcolor{darkblue}{\textbf{\ipa{do˩\textsubscript{b}}}}}\hspace{0.5cm}[\kern2pt{\textcolor{darkblue}{\textbf{\ipa{do˩˥}}}}\kern2pt]} \hypertarget{do\string_Bb1}{}
\markboth{\textcolor{darkblue}{\textbf{\ipa{do˩\textsubscript{b}}}}}{}
\textcolor{teal}{\zh{动词}} \hspace{4pt} \zh{声调类:} L\textsubscript{b}.
\zh{看见,遇见,见。} \textcolor{Sepia}{\selectlanguage{english}To see; to come across someone.} \textcolor{PineGreen}{\selectlanguage{french}Voir, apercevoir.}  ¶ \textcolor{darkblue}{\textbf{\ipa{ɖɯ˧-do˥\textasciitilde{}do˩-ɻ̍˩}}} \zh{见一见} \textcolor{Sepia}{\selectlanguage{english}\mytextsc{delimitative} \string_ \mytextsc{red} \mytextsc{inceptive}} \textcolor{PineGreen}{\selectlanguage{french}\mytextsc{délimitatif} \string_ \mytextsc{red} \mytextsc{inchoatif}}  
 ¶ \textcolor{darkblue}{\textbf{\ipa{ɖɯ˧-kʰwɤ˧ do˧˥}}} \zh{看见一块(东西)} \textcolor{Sepia}{\selectlanguage{english}to see a piece} \textcolor{PineGreen}{\selectlanguage{french}apercevoir un bout/un morceau}  
 ¶ \textcolor{darkblue}{\textbf{\ipa{tso˧\textasciitilde{}tso˧ do˧˥}}} \zh{看见东西} \textcolor{Sepia}{\selectlanguage{english}to see things, to see something} \textcolor{PineGreen}{\selectlanguage{french}apercevoir des choses/apercevoir quelque chose}  
 ¶ \textcolor{darkblue}{\textbf{\ipa{do˩-mɤ˩-ho˥}}} \zh{不许(看)见} \textcolor{Sepia}{\selectlanguage{english}\string_ \mytextsc{neg} \mytextsc{desiderative}} \textcolor{PineGreen}{\selectlanguage{french}\string_ \mytextsc{neg} \mytextsc{désidératif}}  
 ¶ \textcolor{darkblue}{\textbf{\ipa{bo˩mi˧ do˩ (+ze˩)}}} \zh{看见了母猪} \textcolor{Sepia}{\selectlanguage{english}...has seen (a/the) sow} \textcolor{PineGreen}{\selectlanguage{french}...a vu (une/la) truie}  

\lhead{\firstmark}
\rhead{\botmark}

\subsection{\hspace{-0.5cm} {\Large \textcolor{darkblue}{\textbf{\ipa{do˩bv̩\#˥}}}}\hspace{0.5cm}[\kern2pt{\textcolor{darkblue}{\textbf{\ipa{do˩bv̩˥}}}}\kern2pt]} \hypertarget{do\string_Bbv\string_=\#\string_T1}{}
\markboth{\textcolor{darkblue}{\textbf{\ipa{do˩bv̩\#˥}}}}{}
\textcolor{teal}{\zh{名词}} \hspace{4pt} \zh{声调类:} LM+\#H.
\zh{嘛呢堆。} \textcolor{Sepia}{\selectlanguage{english}Mani wall, Mani pile: pile built of rubble and sand, with carved stone tablets, most with the inscription Om Mani Padme Hum. A Mani wall should be passed or circumvented from the left side, the clockwise direction in which the universe revolves, according to Buddhist doctrine.} \textcolor{PineGreen}{\selectlanguage{french}Mur de mani (le nom désigne l'ensemble du mur, pas seulement une des tablettes qui s'y trouvent). Le mur de mani est un mur de pierre sèche et de sable, comportant des tablettes de pierre sur lesquelles est gravé une inscription: le plus souvent Om Mani Padme Hum. Un mur de mani doit être contourné dans le sens des aiguilles d'une montre: le sens de rotation de l'univers, selon la doctrine bouddhiste.}  \zh{量词}: \textcolor{darkblue}{\textbf{\ipa{ɭɯ˧}}} \zh{~【参考】~} \textcolor{darkblue}{\textbf{\ipa{mɑ˩ɳɯ˧-do˥bv˩, mɑ˩ɳɯ\#˥}}} 
\lhead{\firstmark}
\rhead{\botmark}

\subsection{\hspace{-0.5cm} {\Large \textcolor{darkblue}{\textbf{\ipa{do˩kv̩\#˥}}}}\hspace{0.5cm}[\kern2pt{\textcolor{darkblue}{\textbf{\ipa{do˩kv̩˥}}}}\kern2pt]} \hypertarget{do\string_Bkv\string_=\#\string_T1}{}
\markboth{\textcolor{darkblue}{\textbf{\ipa{do˩kv̩\#˥}}}}{}
\textcolor{teal}{\zh{名词}} \hspace{4pt} \zh{声调类:} LM+\#H.
\zh{小梁子,作为楼上(第二层)木地板的底。} \textcolor{Sepia}{\selectlanguage{english}Small beams upholding the ceiling of the ground floor.} \textcolor{PineGreen}{\selectlanguage{french}Poutrelles soutenant le plancher du premier étage.}  \zh{量词}: \textcolor{darkblue}{\textbf{\ipa{ɭɯ˧}}} 
\lhead{\firstmark}
\rhead{\botmark}

\subsection{\hspace{-0.5cm} {\Large \textcolor{darkblue}{\textbf{\ipa{dv̩˩}}} \textsubscript{1}}\hspace{0.5cm}[\kern2pt{\textcolor{darkblue}{\textbf{\ipa{dv̩˩˥}}}}\kern2pt]} \hypertarget{dv\string_=\string_B1}{}
\markboth{\textcolor{darkblue}{\textbf{\ipa{dv̩˩}}} \textsubscript{1}}{}
\textcolor{teal}{\zh{量词}} \hspace{4pt} \zh{声调类:} L *.
\zh{量词:人、牲畜(一群、一队)。} \textcolor{Sepia}{\selectlanguage{english}Classifier for flocks of cattle; only used in the singular.} \textcolor{PineGreen}{\selectlanguage{french}Classificateur des troupeaux; ne s'utilise qu'au singulier.} \zh{~【参考】~} \hyperlink{}{\textcolor{darkblue}{\textbf{\ipa{dɤ˧-tʰv̩˧-gi\#˥}}}} 
\lhead{\firstmark}
\rhead{\botmark}

\subsection{\hspace{-0.5cm} {\Large \textcolor{darkblue}{\textbf{\ipa{dv̩˩}}} \textsubscript{2}}\hspace{0.5cm}[\kern2pt{\textcolor{darkblue}{\textbf{\ipa{dv̩˩˥}}}}\kern2pt]} \hypertarget{dv\string_=\string_B2}{}
\markboth{\textcolor{darkblue}{\textbf{\ipa{dv̩˩}}} \textsubscript{2}}{}
\textcolor{teal}{\zh{代词}} \hspace{4pt} \zh{声调类:} L?.
\zh{指示代词:那边(远指),从‘那个方向’这个短语提取出来的。} \textcolor{Sepia}{\selectlanguage{english}Distal demonstrative, appearing in the phrase “this way, in this direction”.} \textcolor{PineGreen}{\selectlanguage{french}Démonstratif distal, qui apparaît dans l'indication de direction “par ici, dans cette direction”.}  ¶ \textcolor{darkblue}{\textbf{\ipa{dv̩˩-tɕo˧}}} \zh{那个方向} \textcolor{Sepia}{\selectlanguage{english}that way} \textcolor{PineGreen}{\selectlanguage{french}cette direction-là}  
 ¶ \textcolor{darkblue}{\textbf{\ipa{dv̩˩tɕo˧ fæ˧}}} \zh{那个方向} \textcolor{Sepia}{\selectlanguage{english}that way} \textcolor{PineGreen}{\selectlanguage{french}cette direction-là}  
\zh{~【参考】~} \hyperlink{}{\textcolor{darkblue}{\textbf{\ipa{dɤ˧-tʰv̩˧-gi\#˥}}}} 
\lhead{\firstmark}
\rhead{\botmark}

\subsection{\hspace{-0.5cm} {\Large \textcolor{darkblue}{\textbf{\ipa{dv̩˩˧}}} \textsubscript{1}}\hspace{0.5cm}[\kern2pt{\textcolor{darkblue}{\textbf{\ipa{dv̩˩˥}}}}\kern2pt]} \hypertarget{dv\string_=\string_B\string_M1}{}
\markboth{\textcolor{darkblue}{\textbf{\ipa{dv̩˩˧}}} \textsubscript{1}}{}
\textcolor{teal}{\zh{名词}} \hspace{4pt} \zh{声调类:} LM.
\zh{黄鼠狼,黄喉貂。} \textcolor{Sepia}{\selectlanguage{english}Weasel.} \textcolor{PineGreen}{\selectlanguage{french}Belette.}  ¶ \textcolor{darkblue}{\textbf{\ipa{dv̩˩ hwæ˧-ze˧}}} \zh{买了黄鼠狼} \textcolor{Sepia}{\selectlanguage{english}...bought (a) weasel} \textcolor{PineGreen}{\selectlanguage{french}...a acheté une belette}  
 ¶ \textcolor{darkblue}{\textbf{\ipa{dv̩˩ dzɯ˧-ze˩}}} \zh{吃了黄鼠狼} \textcolor{Sepia}{\selectlanguage{english}...ate a weasel} \textcolor{PineGreen}{\selectlanguage{french}...a mangé une belette}  
 \zh{量词}: \textcolor{darkblue}{\textbf{\ipa{mi˩}}} 
\lhead{\firstmark}
\rhead{\botmark}

\subsection{\hspace{-0.5cm} {\Large \textcolor{darkblue}{\textbf{\ipa{dv̩˩˧}}} \textsubscript{2}}\hspace{0.5cm}[\kern2pt{\textcolor{darkblue}{\textbf{\ipa{dv̩˩˥}}}}\kern2pt]} \hypertarget{dv\string_=\string_B\string_M2}{}
\markboth{\textcolor{darkblue}{\textbf{\ipa{dv̩˩˧}}} \textsubscript{2}}{}
\textcolor{teal}{\zh{名词}} \hspace{4pt} \zh{声调类:} LM.
\zh{毒。} \textcolor{Sepia}{\selectlanguage{english}Poison.} \textcolor{PineGreen}{\selectlanguage{french}Poison.} \zh{~【参考】~} \hyperlink{}{\textcolor{darkblue}{\textbf{\ipa{dv̩˩\textsubscript{a}}}}} 
\lhead{\firstmark}
\rhead{\botmark}

\subsection{\hspace{-0.5cm} {\Large \textcolor{darkblue}{\textbf{\ipa{dv̩˥}}}}\hspace{0.5cm}[\kern2pt{\textcolor{darkblue}{\textbf{\ipa{dv̩˥}}}}\kern2pt]} \hypertarget{dv\string_=\string_T1}{}
\markboth{\textcolor{darkblue}{\textbf{\ipa{dv̩˥}}}}{}
\textcolor{teal}{\zh{动词}} \hspace{4pt} \zh{声调类:} H.
\zh{挖。} \textcolor{Sepia}{\selectlanguage{english}To dig.} \textcolor{PineGreen}{\selectlanguage{french}Creuser.}  ¶ \textcolor{darkblue}{\textbf{\ipa{ʈʂe˧ dv̩˧(-ze˩)}}} \zh{挖土} \textcolor{Sepia}{\selectlanguage{english}to dig the earth} \textcolor{PineGreen}{\selectlanguage{french}piocher la terre, creuser la terre}  

\lhead{\firstmark}
\rhead{\botmark}

\subsection{\hspace{-0.5cm} {\Large \textcolor{darkblue}{\textbf{\ipa{dv̩˩\textsubscript{a}}}}}\hspace{0.5cm}[\kern2pt{\textcolor{darkblue}{\textbf{\ipa{dv̩˩˥}}}}\kern2pt]} \hypertarget{dv\string_=\string_Ba1}{}
\markboth{\textcolor{darkblue}{\textbf{\ipa{dv̩˩\textsubscript{a}}}}}{}
\textcolor{teal}{\zh{动词}} \hspace{4pt} \zh{声调类:} L\textsubscript{a}.
\ding{202} \zh{毒害、毒化。} \textcolor{Sepia}{\selectlanguage{english}To poison.} \textcolor{PineGreen}{\selectlanguage{french}Empoisonner, rendre malade.}  ¶ \textcolor{darkblue}{\textbf{\ipa{ʈʂʰɯ˧, | hĩ˧ dv̩˥-mɤ˩-kv̩˩! | ʈʂʰɯ˧, | hĩ˧ dv̩˥-kv̩˩!}}} \zh{这个,不会让人中毒!那个(反倒)会让人中毒!(情景:谈不同菌子种类。)} \textcolor{Sepia}{\selectlanguage{english}This one is not poisonous / is edible (literally “this one does not poison people”)! That one [on the other hand] is poisonous / is not edible! (About different sorts of mushrooms.)} \textcolor{PineGreen}{\selectlanguage{french}Celui-ci, il n'est pas dangereux / il est comestible! Celui-là [en revanche], il est vénéneux / il est dangereux / il peut vous rendre malade / il peut vous empoisonner / il n'est pas comestible! (Au sujet de diverses sortes de champignons.)}  
 ¶ \textcolor{darkblue}{\textbf{\ipa{ʈʂʰɯ˧, | dv̩˩-mɤ˩-kv̩˥!}}} \zh{这个,不会让人中毒!(情景:谈不同菌子种类。)} \textcolor{Sepia}{\selectlanguage{english}This one is not poisonous / is edible (literally “this one does not poison people”)! (About a mushroom species.)} \textcolor{PineGreen}{\selectlanguage{french}C'est inoffensif/comestible/pas vénéneux/pas dangereux! (Au sujet d'une sorte de champignon.)}  
\ding{203} \zh{讨厌、恨。} \textcolor{Sepia}{\selectlanguage{english}To hate, to detest.} \textcolor{PineGreen}{\selectlanguage{french}Détester.}  ¶ \textcolor{darkblue}{\textbf{\ipa{le˧-dv̩˩-ze˩}}} \textcolor{PineGreen}{\selectlanguage{french}\mytextsc{accomp} \string_ \mytextsc{pfv}}  
 ¶ \textcolor{darkblue}{\textbf{\ipa{njɤ˧ | ʈʂʰɯ˧ dv̩˥ | ʐwæ˩˥!}}} \zh{我很讨厌他!} \textcolor{Sepia}{\selectlanguage{english}I hate him/her!} \textcolor{PineGreen}{\selectlanguage{french}je le déteste!}  
 ¶ \textcolor{darkblue}{\textbf{\ipa{dv̩˩-zo˧-mɤ˧-tʰɑ˧˥}}} \zh{讨厌得不行} \textcolor{Sepia}{\selectlanguage{english}to hate deeply} \textcolor{PineGreen}{\selectlanguage{french}détester à mort}  
\zh{~【参考】~} \hyperlink{}{\textcolor{darkblue}{\textbf{\ipa{dv̩˩˧}}} \textsubscript{2}} 
\lhead{\firstmark}
\rhead{\botmark}

\subsection{\hspace{-0.5cm} {\Large \textcolor{darkblue}{\textbf{\ipa{dv̩˩\textsubscript{b}}}}}\hspace{0.5cm}[\kern2pt{\textcolor{darkblue}{\textbf{\ipa{dv̩˩˥}}}}\kern2pt]} \hypertarget{dv\string_=\string_Bb1}{}
\markboth{\textcolor{darkblue}{\textbf{\ipa{dv̩˩\textsubscript{b}}}}}{}
\textcolor{teal}{\zh{量词}} \hspace{4pt} \zh{声调类:} L\textsubscript{b}.
\zh{量词:人(一些)。} \textcolor{Sepia}{\selectlanguage{english}Classifier for small groups of people: 3 or more.} \textcolor{PineGreen}{\selectlanguage{french}Classificateur des petits groupes (de personnes): quelques-uns (plus de 3).}  ¶ \textcolor{darkblue}{\textbf{\ipa{hĩ˧ ɖɯ˧-dv̩˩}}} \zh{一些人} \textcolor{Sepia}{\selectlanguage{english}a few people, a group of people} \textcolor{PineGreen}{\selectlanguage{french}quelques personnes, un groupe de personnes}  
 ¶ \textcolor{darkblue}{\textbf{\ipa{hĩ˧ ʈʂʰɯ˧-dv̩˥}}} \zh{这些人} \textcolor{Sepia}{\selectlanguage{english}these people, this group of people} \textcolor{PineGreen}{\selectlanguage{french}ce groupe de gens, ces qq personnes}  

\lhead{\firstmark}
\rhead{\botmark}

\subsection{\hspace{-0.5cm} {\Large \textcolor{darkblue}{\textbf{\ipa{dv̩˩bi˩}}}}\hspace{0.5cm}[\kern2pt{\textcolor{darkblue}{\textbf{\ipa{dv̩˩bi˩˥}}}}\kern2pt]} \hypertarget{dv\string_=\string_Bbi\string_B1}{}
\markboth{\textcolor{darkblue}{\textbf{\ipa{dv̩˩bi˩}}}}{}
\textcolor{teal}{\zh{助词}} \hspace{4pt} \zh{声调类:} L.
\zh{对面。} \textcolor{Sepia}{\selectlanguage{english}Opposite.} \textcolor{PineGreen}{\selectlanguage{french}En face.} 
\lhead{\firstmark}
\rhead{\botmark}

\subsection{\hspace{-0.5cm} {\Large \textcolor{darkblue}{\textbf{\ipa{dv̩˩mi\#˥}}}}\hspace{0.5cm}[\kern2pt{\textcolor{darkblue}{\textbf{\ipa{dv̩˩mi˩˥}}}}\kern2pt]} \hypertarget{dv\string_=\string_Bmi\#\string_T1}{}
\markboth{\textcolor{darkblue}{\textbf{\ipa{dv̩˩mi\#˥}}}}{}
\textcolor{teal}{\zh{名词}} \hspace{4pt} \zh{声调类:} LM+\#H.
\zh{母黄鼠狼。} \textcolor{Sepia}{\selectlanguage{english}Female weasel.} \textcolor{PineGreen}{\selectlanguage{french}Belette femelle.}  ¶ \textcolor{darkblue}{\textbf{\ipa{dv̩˩mi˧-dv̩˥pʰv̩˩}}} \zh{母黄鼠狼与公黄鼠狼} \textcolor{Sepia}{\selectlanguage{english}female weasel and male weasel} \textcolor{PineGreen}{\selectlanguage{french}belette femelle et belette mâle}  
 \zh{量词}: \textcolor{darkblue}{\textbf{\ipa{mi˩}}} 
\lhead{\firstmark}
\rhead{\botmark}

\subsection{\hspace{-0.5cm} {\Large \textcolor{darkblue}{\textbf{\ipa{dv̩˩pʰæ˧}}}}\hspace{0.5cm}[\kern2pt{\textcolor{darkblue}{\textbf{\ipa{dv̩˩pʰæ˥}}}}\kern2pt]} \hypertarget{dv\string_=\string_Bp\string_h\{\string_M1}{}
\markboth{\textcolor{darkblue}{\textbf{\ipa{dv̩˩pʰæ˧}}}}{}
\textcolor{teal}{\zh{名词}} \hspace{4pt} \zh{声调类:} LM.
\zh{仓廪。摩梭话音译:‘独帕’。} \textcolor{Sepia}{\selectlanguage{english}The room in the main building of the farm where cereals were kept: the granary.} \textcolor{PineGreen}{\selectlanguage{french}Partie du bâtiment principal dans laquelle étaient conservées les céréales: le grenier à céréales.}  \zh{量词}: \textcolor{darkblue}{\textbf{\ipa{ɭɯ˧}}} 
\lhead{\firstmark}
\rhead{\botmark}

\subsection{\hspace{-0.5cm} {\Large \textcolor{darkblue}{\textbf{\ipa{dv̩˩pʰv̩\#˥}}}}\hspace{0.5cm}[\kern2pt{\textcolor{darkblue}{\textbf{\ipa{dv̩˩pʰv̩˥}}}}\kern2pt]} \hypertarget{dv\string_=\string_Bp\string_hv\string_=\#\string_T1}{}
\markboth{\textcolor{darkblue}{\textbf{\ipa{dv̩˩pʰv̩\#˥}}}}{}
\textcolor{teal}{\zh{名词}} \hspace{4pt} \zh{声调类:} LM+\#H / LM.
\zh{公黄鼠狼。} \textcolor{Sepia}{\selectlanguage{english}Male weasel.} \textcolor{PineGreen}{\selectlanguage{french}Belette mâle.}  \zh{量词}: \textcolor{darkblue}{\textbf{\ipa{mi˩}}} 
\lhead{\firstmark}
\rhead{\botmark}

\subsection{\hspace{-0.5cm} {\Large \textcolor{darkblue}{\textbf{\ipa{dv̩˩zo\#˥}}}}\hspace{0.5cm}[\kern2pt{\textcolor{darkblue}{\textbf{\ipa{dv̩˧zo˧}}}}\kern2pt]} \hypertarget{dv\string_=\string_Bzo\#\string_T1}{}
\markboth{\textcolor{darkblue}{\textbf{\ipa{dv̩˩zo\#˥}}}}{}
\textcolor{teal}{\zh{名词}} \hspace{4pt} \zh{声调类:} LM+\#H / LM.
\zh{黄鼠狼的崽子。} \textcolor{Sepia}{\selectlanguage{english}Baby weasel.} \textcolor{PineGreen}{\selectlanguage{french}Bébé belette.} 
\lhead{\firstmark}
\rhead{\botmark}

\newpage
\section*{\centering- \textcolor{darkblue}{\textbf{\ipa{dz}}} -}
\subsection{\hspace{-0.5cm} {\Large \textcolor{darkblue}{\textbf{\ipa{dzɑ˥}}}}\hspace{0.5cm}[\kern2pt{\textcolor{darkblue}{\textbf{\ipa{dzɑ˥}}}}\kern2pt]} \hypertarget{dzA\string_T1}{}
\markboth{\textcolor{darkblue}{\textbf{\ipa{dzɑ˥}}}}{}
\textcolor{teal}{\zh{形容词}} \hspace{4pt} \zh{声调类:} H.
\ding{202} \zh{坏、差、下级(行为……)。} \textcolor{Sepia}{\selectlanguage{english}Bad, mean (action), inferior.} \textcolor{PineGreen}{\selectlanguage{french}Mauvais (action…), inférieur, indigne.}  ¶ \textcolor{darkblue}{\textbf{\ipa{ʈʂʰɯ˧-ɳɯ˧ | njɤ˧-ki˧ | dzɑ˧-ʝi˧ | ʐwæ˩˥!}}} \zh{他很瞧不起我!} \textcolor{Sepia}{\selectlanguage{english}He really despises me!} \textcolor{PineGreen}{\selectlanguage{french}il me méprise vraiment!}  
 ¶ \textcolor{darkblue}{\textbf{\ipa{hĩ˧ ʈʂʰɯ˧-v̩˧ dʑo˩, | õ˧-ki˥ | dzɑ˧-ʝi˧-ze˩!}}} \zh{这个人,不尊重自己!} \textcolor{Sepia}{\selectlanguage{english}This person has no self-respect! (literally: This person is doing herself harm)} \textcolor{PineGreen}{\selectlanguage{french}cette personne ne se respecte pas!}  
 ¶ \textcolor{darkblue}{\textbf{\ipa{mv̩˧ dzɑ˧.}}} \zh{天气很坏。} \textcolor{Sepia}{\selectlanguage{english}The weather is bad.} \textcolor{PineGreen}{\selectlanguage{french}Il fait mauvais temps.}  
 ¶ \textcolor{darkblue}{\textbf{\ipa{mv̩˧ dzɑ˧-ze˩}}} \zh{天气变坏了。} \textcolor{Sepia}{\selectlanguage{english}The weather is getting bad.} \textcolor{PineGreen}{\selectlanguage{french}Le temps se met au mauvais, il commence à faire mauvais temps.}  
 ¶ \textcolor{darkblue}{\textbf{\ipa{lo˧ dzɑ˧}}} \zh{(工作)差} \textcolor{Sepia}{\selectlanguage{english}poor (work), bad (job: e.g. someone has done a bad job)} \textcolor{PineGreen}{\selectlanguage{french}bâclé, mal fait (travail)}  
\ding{203} \zh{穷(人)。} \textcolor{Sepia}{\selectlanguage{english}Poor (person).} \textcolor{PineGreen}{\selectlanguage{french}Indigent, pauvre (personne…).}  ¶ \textcolor{darkblue}{\textbf{\ipa{dzɑ˧ | -ʐwæ˩-ze˥!}}} \zh{他很穷!} \textcolor{Sepia}{\selectlanguage{english}(He/she) is really poor!} \textcolor{PineGreen}{\selectlanguage{french}(il/elle est) très pauvre!}  
 ¶ \textcolor{darkblue}{\textbf{\ipa{ɑ˩ʁo˧ | bo˩ʈʂʰæ˧ mɤ˧-dʑo˧, | dzɑ˧ ʈʂɤ˧-kv̩˩!}}} \zh{如果家里没有猪膘,会显得很穷!} \textcolor{Sepia}{\selectlanguage{english}If there is no fleshless preserved pork at home, it appears as if the family is really destitute!} \textcolor{PineGreen}{\selectlanguage{french}Ne pas avoir de cochon-entier-conservé à la maison, ça fait vraiment mauvais effet/ça fait vraiment indigent/c'est la honte!}  
 ¶ \textcolor{darkblue}{\textbf{\ipa{dzɑ˧ ʈʂɤ˧ | ʐwæ˩˥!}}} \zh{真羞耻啊!} \textcolor{Sepia}{\selectlanguage{english}It's really a shame / it's really something to be ashamed of! (Talking about a socially stigmatized situation, such as not having the required food items or pieces of clothing for important ceremonies.)} \textcolor{PineGreen}{\selectlanguage{french}C'est vraiment la honte/on paraît vraiment à la rue! (Au sujet de situations stigmatisées socialement, comme de ne pas posséder les nourritures ou vêtement nécessaires aux principaux rituels.)}  

\lhead{\firstmark}
\rhead{\botmark}

\subsection{\hspace{-0.5cm} {\Large \textcolor{darkblue}{\textbf{\ipa{dzɑ˩qʰwɤ˩}}}}\hspace{0.5cm}[\kern2pt{\textcolor{darkblue}{\textbf{\ipa{dzɑ˧qʰwɤ˥}}}}\kern2pt]} \hypertarget{dzA\string_Bq\string_hw7\string_B1}{}
\markboth{\textcolor{darkblue}{\textbf{\ipa{dzɑ˩qʰwɤ˩}}}}{}
\textcolor{teal}{\zh{名词}} \hspace{4pt} \zh{声调类:} L.
\zh{鞋、鞋子。} \textcolor{Sepia}{\selectlanguage{english}Shoe.} \textcolor{PineGreen}{\selectlanguage{french}Chaussure.}  \zh{量词}: \textcolor{darkblue}{\textbf{\ipa{dzi˧}}} 
\lhead{\firstmark}
\rhead{\botmark}

\subsection{\hspace{-0.5cm} {\Large \textcolor{darkblue}{\textbf{\ipa{dze˥}}}}\hspace{0.5cm}[\kern2pt{\textcolor{darkblue}{\textbf{\ipa{dze˩˥}}}}\kern2pt]} \hypertarget{dze\string_T1}{}
\markboth{\textcolor{darkblue}{\textbf{\ipa{dze˥}}}}{}
\textcolor{teal}{\zh{名词}} \hspace{4pt} \zh{声调类:} \#H.
\zh{糖。} \textcolor{Sepia}{\selectlanguage{english}Sugar.} \textcolor{PineGreen}{\selectlanguage{french}Sucre.} 
\lhead{\firstmark}
\rhead{\botmark}

\subsection{\hspace{-0.5cm} {\Large \textcolor{darkblue}{\textbf{\ipa{dze˧bɤ˩}}}}\hspace{0.5cm}[\kern2pt{\textcolor{darkblue}{\textbf{\ipa{dze˧bɤ˧}}}}\kern2pt]} \hypertarget{dze\string_Mb7\string_B1}{}
\markboth{\textcolor{darkblue}{\textbf{\ipa{dze˧bɤ˩}}}}{}
\textcolor{teal}{\zh{名词}} \hspace{4pt} \zh{声调类:} L\#.
\zh{蝙蝠、飞鼠。} \textcolor{Sepia}{\selectlanguage{english}Bat; used for all species, including the flying squirrel.} \textcolor{PineGreen}{\selectlanguage{french}Chauve-souris; s'emploie pour toutes les espèces, y compris le renard volant.}  ¶ \textcolor{darkblue}{\textbf{\ipa{dze˧bɤ˩-zo˩ | ɖɯ˧-ɭɯ˧}}} \zh{一只小蝙蝠} \textcolor{Sepia}{\selectlanguage{english}a baby bat} \textcolor{PineGreen}{\selectlanguage{french}une petite chauve-souris, un bébé chauve-souris}  
 ¶ \textcolor{darkblue}{\textbf{\ipa{dze˧bɤ˩-pʰv̩˩ | ɖɯ˧-mi˩}}} \zh{一只公蝙蝠} \textcolor{Sepia}{\selectlanguage{english}a male bat} \textcolor{PineGreen}{\selectlanguage{french}une chauve-souris mâle}  
 ¶ \textcolor{darkblue}{\textbf{\ipa{dze˧bɤ˩-mi˩ | ɖɯ˧-mi˩}}} \zh{一只母蝙蝠} \textcolor{Sepia}{\selectlanguage{english}a female bat} \textcolor{PineGreen}{\selectlanguage{french}une chauve-souris femelle}  
 \zh{量词}: \textcolor{darkblue}{\textbf{\ipa{mi˩}}} 
\lhead{\firstmark}
\rhead{\botmark}

\subsection{\hspace{-0.5cm} {\Large \textcolor{darkblue}{\textbf{\ipa{dze˧bo˧}}}}\hspace{0.5cm}[\kern2pt{\textcolor{darkblue}{\textbf{\ipa{dze˩bo˩˥}}}}\kern2pt]} \hypertarget{dze\string_Mbo\string_M1}{}
\markboth{\textcolor{darkblue}{\textbf{\ipa{dze˧bo˧}}}}{}
\textcolor{teal}{\zh{名词}} \hspace{4pt} \zh{声调类:} M.
\ding{202} \zh{者波(姓)。这个家族有三个家庭。} \textcolor{Sepia}{\selectlanguage{english}A family name from Yongning.} \textcolor{PineGreen}{\selectlanguage{french}Nom de clan/famille étendue qui compte 3 familles. C'est également le nom d'un village de la plaine de Yongning.}  ¶ \textcolor{darkblue}{\textbf{\ipa{dze˧bo˧=ɻ̍˩}}} \zh{者波家族} \textcolor{Sepia}{\selectlanguage{english}the \textcolor{darkblue}{\textbf{\ipa{/dze˧bo˧/}}} clan, the \textcolor{darkblue}{\textbf{\ipa{/dze˧bo˧/}}} family} \textcolor{PineGreen}{\selectlanguage{french}le clan \textcolor{darkblue}{\textbf{\ipa{/dze˧bo˧/}}}, la famille \textcolor{darkblue}{\textbf{\ipa{/dze˧bo˧/}}}}  
\ding{203} \zh{者波(永宁的一个村落)。村落有两个部分,\textcolor{darkblue}{\textbf{\ipa{/gɤ˩ʁwɤ˧/}}}‘上村’与\textcolor{darkblue}{\textbf{\ipa{/mv̩˩ʁwɤ˧/}}}‘下村’.} \textcolor{Sepia}{\selectlanguage{english}A village in the Yongning plain. It consists of two parts, “upper” and “lower: \textcolor{darkblue}{\textbf{\ipa{/gɤ˩ʁwɤ˧/}}} and \textcolor{darkblue}{\textbf{\ipa{/mv̩˩ʁwɤ˧/}}}.} \textcolor{PineGreen}{\selectlanguage{french}Zhebo, un village de la plaine de Yongning. Il est divisé en deux parties, ”du haut“ et ”du bas": \textcolor{darkblue}{\textbf{\ipa{/gɤ˩ʁwɤ˧/}}} et \textcolor{darkblue}{\textbf{\ipa{/mv̩˩ʁwɤ˧/}}}.}  ¶ \textcolor{darkblue}{\textbf{\ipa{ɖæ˩ʂɯ\#˥, | ʈʂo˧ʂɯ\#˥, | bɤ˩tɕʰɯ˩˥, | dɑ˧pʰo˥, | bɤ˧dzi˩, | dze˧bo˧}}} \zh{永宁坝的六个村落,按传统排序:从距离泸沽湖最近的村落说起。} \textcolor{Sepia}{\selectlanguage{english}the six villages of the plain of Yongning, in traditional order: by order of increasing distance from the Lake} \textcolor{PineGreen}{\selectlanguage{french}les six villages de la plaine de Yongning, dans l'ordre, qui prend comme point d'origine le village le plus proche du Lac}  

\lhead{\firstmark}
\rhead{\botmark}

\subsection{\hspace{-0.5cm} {\Large \textcolor{darkblue}{\textbf{\ipa{dze˧dv̩˩}}}}\hspace{0.5cm}[\kern2pt{\textcolor{darkblue}{\textbf{\ipa{dze˧dv̩˩}}}}\kern2pt]} \hypertarget{dze\string_Mdv\string_=\string_B1}{}
\markboth{\textcolor{darkblue}{\textbf{\ipa{dze˧dv̩˩}}}}{}
\textcolor{teal}{\zh{名词}} \hspace{4pt} \zh{声调类:} L\#.
\zh{饼。} \textcolor{Sepia}{\selectlanguage{english}Cake, bread.} \textcolor{PineGreen}{\selectlanguage{french}Galette de céréale (blé, avoine…), pain.}  ¶ \textcolor{darkblue}{\textbf{\ipa{dze˧dv̩˩-pɤ˩jɤ˩}}} \zh{粮食饼} \textcolor{Sepia}{\selectlanguage{english}cake of cereals} \textcolor{PineGreen}{\selectlanguage{french}galette de céréale (même sens)}  

\lhead{\firstmark}
\rhead{\botmark}

\subsection{\hspace{-0.5cm} {\Large \textcolor{darkblue}{\textbf{\ipa{dze˧hi˧}}}}\hspace{0.5cm}[\kern2pt{\textcolor{darkblue}{\textbf{\ipa{dze˧hi˧}}}}\kern2pt]} \hypertarget{dze\string_Mhi\string_M1}{}
\markboth{\textcolor{darkblue}{\textbf{\ipa{dze˧hi˧}}}}{}
\textcolor{teal}{\zh{名词}} \hspace{4pt} \zh{声调类:} M.
\zh{丈人。} \textcolor{Sepia}{\selectlanguage{english}In-laws.} \textcolor{PineGreen}{\selectlanguage{french}Beaux-parents.}  ¶ \textcolor{darkblue}{\textbf{\ipa{njɤ˧ | dze˧hi˧-ki˩ bi˩!}}} \zh{我去我丈人(那边)!} \textcolor{Sepia}{\selectlanguage{english}I'm going to my in-laws' place! / I'm going to visit my in-laws!} \textcolor{PineGreen}{\selectlanguage{french}Je vais chez mes beaux-parents!}  
 ¶ \textcolor{darkblue}{\textbf{\ipa{no˧ | dze˧hi˧ | ə˩-to˩-ze˥? - le˧-to˩-ze˩!}}} \zh{你有丈人吗?(=你结婚了吗?)-有的!(=结婚了!)} \textcolor{Sepia}{\selectlanguage{english}Do you have in-laws? / Do you stand in an 'in-law' relationship? (=Are you married?) - Yes, I have entered into such a relationship! (=Yes, I am married!)} \textcolor{PineGreen}{\selectlanguage{french}Tu as une belle-famille? =Tu es marié(e)? -Oui!}  
 ¶ \textcolor{darkblue}{\textbf{\ipa{no˧ | dze˧hi˧ to˩ ə˩-bi˩?}}} \zh{你打算结婚吗?} \textcolor{Sepia}{\selectlanguage{english}Do you have plans to get married? (Literally: Are you going to enter an 'in-law' relationship?)} \textcolor{PineGreen}{\selectlanguage{french}Tu comptes te marier? (Littéralement: Tu vas te lier avec une belle-famille?)}  

\lhead{\firstmark}
\rhead{\botmark}

\subsection{\hspace{-0.5cm} {\Large \textcolor{darkblue}{\textbf{\ipa{dze˧kʰɤ˧˥}}}}\hspace{0.5cm}[\kern2pt{\textcolor{darkblue}{\textbf{\ipa{dze˧kʰɤ˧}}}}\kern2pt]} \hypertarget{dze\string_Mk\string_h7\string_M\string_T1}{}
\markboth{\textcolor{darkblue}{\textbf{\ipa{dze˧kʰɤ˧˥}}}}{}
\textcolor{teal}{\zh{名词}} \hspace{4pt} \zh{声调类:} MH\#.
\zh{百姓。音译:“责卡”。} \textcolor{Sepia}{\selectlanguage{english}Commoner (second of the three ranks in feudal society).} \textcolor{PineGreen}{\selectlanguage{french}Roturier, la 2e des 3 castes de la société ancienne, intermédiaire entre la noblesse et les serfs.}  \zh{量词}: \textcolor{darkblue}{\textbf{\ipa{v̩˧}}} 
\lhead{\firstmark}
\rhead{\botmark}

\subsection{\hspace{-0.5cm} {\Large \textcolor{darkblue}{\textbf{\ipa{dze˧ɭɯ˧}}}}\hspace{0.5cm}[\kern2pt{\textcolor{darkblue}{\textbf{\ipa{dze˧ɭɯ˧˥}}}}\kern2pt]} \hypertarget{dze\string_Ml\string_RM\string_M1}{}
\markboth{\textcolor{darkblue}{\textbf{\ipa{dze˧ɭɯ˧}}}}{}
\textcolor{teal}{\zh{名词}} \hspace{4pt} \zh{声调类:} M.
\zh{小麦。} \textcolor{Sepia}{\selectlanguage{english}Wheat.} \textcolor{PineGreen}{\selectlanguage{french}Blé, froment.} 
\lhead{\firstmark}
\rhead{\botmark}

\subsection{\hspace{-0.5cm} {\Large \textcolor{darkblue}{\textbf{\ipa{dze˧ɭɯ˧-ɻ̃\#˥}}}}\hspace{0.5cm}[\kern2pt{\textcolor{darkblue}{\textbf{\ipa{xxxx non-correspondance entre le nombre de morphèmes et le nombre de tons de morphèmes}}}}\kern2pt]} \hypertarget{dze\string_Ml\string_RM\string_M-r£`\string_~\#\string_T1}{}
\markboth{\textcolor{darkblue}{\textbf{\ipa{dze˧ɭɯ˧-ɻ̃\#˥}}}}{}
\textcolor{teal}{\zh{名词}} \hspace{4pt} \zh{声调类:} \#H.
\zh{麦杆。} \textcolor{Sepia}{\selectlanguage{english}Wheat straw.} \textcolor{PineGreen}{\selectlanguage{french}Paille de blé.} 
\lhead{\firstmark}
\rhead{\botmark}

\subsection{\hspace{-0.5cm} {\Large \textcolor{darkblue}{\textbf{\ipa{dze˧-ɻ̃\#˥}}}}\hspace{0.5cm}[\kern2pt{\textcolor{darkblue}{\textbf{\ipa{xxxx non-correspondance entre le nombre de morphèmes et le nombre de tons de morphèmes}}}}\kern2pt]} \hypertarget{dze\string_M-r£`\string_~\#\string_T1}{}
\markboth{\textcolor{darkblue}{\textbf{\ipa{dze˧-ɻ̃\#˥}}}}{}
\textcolor{teal}{\zh{名词}} \hspace{4pt} \zh{声调类:} \#H.
\zh{小麦秆。} \textcolor{Sepia}{\selectlanguage{english}Wheat straw.} \textcolor{PineGreen}{\selectlanguage{french}Paille de blé.} 
\lhead{\firstmark}
\rhead{\botmark}

\subsection{\hspace{-0.5cm} {\Large \textcolor{darkblue}{\textbf{\ipa{dze˧-tɕʰi\#˥}}}}\hspace{0.5cm}[\kern2pt{\textcolor{darkblue}{\textbf{\ipa{xxxx non-correspondance entre le nombre de morphèmes et le nombre de tons de morphèmes}}}}\kern2pt]} \hypertarget{dze\string_M-ts£\string_hi\#\string_T1}{}
\markboth{\textcolor{darkblue}{\textbf{\ipa{dze˧-tɕʰi\#˥}}}}{}
\textcolor{teal}{\zh{名词}} \hspace{4pt} \zh{声调类:} \#H.
\zh{麦芒。} \textcolor{Sepia}{\selectlanguage{english}Wheat beard.} \textcolor{PineGreen}{\selectlanguage{french}Barbe de blé.} 
\lhead{\firstmark}
\rhead{\botmark}

\subsection{\hspace{-0.5cm} {\Large \textcolor{darkblue}{\textbf{\ipa{dze˧-ʈʂæ˥}}}}\hspace{0.5cm}[\kern2pt{\textcolor{darkblue}{\textbf{\ipa{xxxx non-correspondance entre le nombre de morphèmes et le nombre de tons de morphèmes}}}}\kern2pt]} \hypertarget{dze\string_M-t`s`\{\string_T1}{}
\markboth{\textcolor{darkblue}{\textbf{\ipa{dze˧-ʈʂæ˥}}}}{}
\textcolor{teal}{\zh{名词}} \hspace{4pt} \zh{声调类:} H\#.
\zh{蜜蜂的螫針。} \textcolor{Sepia}{\selectlanguage{english}Sting organ.} \textcolor{PineGreen}{\selectlanguage{french}Dard de l'abeille.}  \zh{量词}: \textcolor{darkblue}{\textbf{\ipa{ɭɯ˧}}} 
\lhead{\firstmark}
\rhead{\botmark}

\subsection{\hspace{-0.5cm} {\Large \textcolor{darkblue}{\textbf{\ipa{dze˧ʈʂɯ˧}}}}\hspace{0.5cm}[\kern2pt{\textcolor{darkblue}{\textbf{\ipa{dze˧ʈʂɯ˥}}}}\kern2pt]} \hypertarget{dze\string_Mt`s`M\string_M1}{}
\markboth{\textcolor{darkblue}{\textbf{\ipa{dze˧ʈʂɯ˧}}}}{}
\textcolor{teal}{\zh{名词}} \hspace{4pt} \zh{声调类:} M.
\zh{筛子。} \textcolor{Sepia}{\selectlanguage{english}Sifter, sieve.} \textcolor{PineGreen}{\selectlanguage{french}Vanneries: tamis, crible.}  \zh{量词}: \textcolor{darkblue}{\textbf{\ipa{nɑ˧}}} 
\lhead{\firstmark}
\rhead{\botmark}

\subsection{\hspace{-0.5cm} {\Large \textcolor{darkblue}{\textbf{\ipa{dze˧ʈʂʰɤ\$˥}}}}\hspace{0.5cm}[\kern2pt{\textcolor{darkblue}{\textbf{\ipa{dze˧ʈʂʰɤ˥}}}}\kern2pt]} \hypertarget{dze\string_Mt`s`\string_h7\$\string_T1}{}
\markboth{\textcolor{darkblue}{\textbf{\ipa{dze˧ʈʂʰɤ\$˥}}}}{}
\textcolor{teal}{\zh{名词}} \hspace{4pt} \zh{声调类:} H\$.
\zh{粮食。现在,这个词的含义受到汉语‘五谷’这个词的影响,用来指代‘五谷杂粮’,相当于所有粮食类,如:小米类、稻谷、麦子、玉米以及豆类与薯类。} \textcolor{Sepia}{\selectlanguage{english}Cereals; the main cereal crop used to be barley, but the meaning of this word currently tends to become identified with the five main sorts of grains referred to in Chinese as 'the five cereals', \zh{五谷}, namely rice, two kinds of millet, wheat, and beans.} \textcolor{PineGreen}{\selectlanguage{french}Céréales; la céréale traditionnelle était l'orge, mais le sens de l'expression tend actuellement à se confondre avec celui de l'expression chinoise \zh{五谷} 'les cinq céréales': le riz; deux sortes de millet; le blé; et les fèves.} 
\lhead{\firstmark}
\rhead{\botmark}

\subsection{\hspace{-0.5cm} {\Large \textcolor{darkblue}{\textbf{\ipa{dze˩}}}}\hspace{0.5cm}[\kern2pt{\textcolor{darkblue}{\textbf{\ipa{dze˩˥}}}}\kern2pt]} \hypertarget{dze\string_B1}{}
\markboth{\textcolor{darkblue}{\textbf{\ipa{dze˩}}}}{}
\textcolor{teal}{\zh{动词}} \hspace{4pt} \zh{声调类:} L.
\zh{剩下(饭或饮料)。} \textcolor{Sepia}{\selectlanguage{english}To be left over (food or drink).} \textcolor{PineGreen}{\selectlanguage{french}Rester, être en trop, devenir un reste (nourriture, boisson).}  ¶ \textcolor{darkblue}{\textbf{\ipa{dzɯ˧-dze˥-ze˩!}}} \zh{剩了一些饭!/ 剩了一些吃的!} \textcolor{Sepia}{\selectlanguage{english}There are some leftovers! / The food has not been eaten up!} \textcolor{PineGreen}{\selectlanguage{french}il y a des restes! / on n'a pas achevé de manger (un plat)!}  
 ¶ \textcolor{darkblue}{\textbf{\ipa{gɤ˩-dze˥ +ze˩!}}} \zh{有剩下的!} \textcolor{Sepia}{\selectlanguage{english}There are some leftovers!} \textcolor{PineGreen}{\selectlanguage{french}Il en reste / il y a des restes!}  
 ¶ \textcolor{darkblue}{\textbf{\ipa{ʈʰɯ˩ dze˩-ze˥}}} \zh{喝剩了、没喝完} \textcolor{Sepia}{\selectlanguage{english}Some of the drink is left over! / (The drink) has not been drunk up!} \textcolor{PineGreen}{\selectlanguage{french}Il en reste / on n'a pas achevé de boire (un verre…); ne pas être entièrement bu}  
 ¶ \textcolor{darkblue}{\textbf{\ipa{le˧-se˩-ze˩! | gɤ˩-mɤ˧-dze˩!}}} \zh{完了!(=全部吃/喝完了!)没有剩!} \textcolor{Sepia}{\selectlanguage{english}It's completely finished (=eaten up / drunk up)! There are no leftovers!} \textcolor{PineGreen}{\selectlanguage{french}on a tout fini, il n'y a pas de restes! / tout a été mangé, bu..., il n'en reste plus !}  

\lhead{\firstmark}
\rhead{\botmark}

\subsection{\hspace{-0.5cm} {\Large \textcolor{darkblue}{\textbf{\ipa{dze˩\textsubscript{a}}}}}\hspace{0.5cm}[\kern2pt{\textcolor{darkblue}{\textbf{\ipa{dze˥}}}}\kern2pt]} \hypertarget{dze\string_Ba1}{}
\markboth{\textcolor{darkblue}{\textbf{\ipa{dze˩\textsubscript{a}}}}}{}
\textcolor{teal}{\zh{量词}} \hspace{4pt} \zh{声调类:} L\textsubscript{a}.
\zh{量词:瓶子、锅(一对)。} \textcolor{Sepia}{\selectlanguage{english}Classifier for pairs of separable objects: a pair of pots, a pair of bottles….} \textcolor{PineGreen}{\selectlanguage{french}Classificateur des lots de deux objets non indissociables; par ex.: lot de 2 casseroles; paire de haut-parleurs… Pour les paires non dissociables (ex.: paire de chaussures), on utilise: /dzi˧/.}  ¶ \textcolor{darkblue}{\textbf{\ipa{zo˧mv̩˥ | ɖɯ˧-dze˩}}} \zh{双胞胎(直译:“一对孩子”)} \textcolor{Sepia}{\selectlanguage{english}twins (literally: 'a pair of children') (F5)} \textcolor{PineGreen}{\selectlanguage{french}des jumeaux (littéralement “une paire d'enfants”)}  
 ¶ \textcolor{darkblue}{\textbf{\ipa{ʈʂʰɯ˧-dze˥}}} \zh{\mytextsc{指示代词} \string_} \textcolor{Sepia}{\selectlanguage{english}\mytextsc{dem} \string_ (tone: H\# / H\$)} \textcolor{PineGreen}{\selectlanguage{french}\mytextsc{dem} \string_ (ton: H\# / H\$)}  

\lhead{\firstmark}
\rhead{\botmark}

\subsection{\hspace{-0.5cm} {\Large \textcolor{darkblue}{\textbf{\ipa{dze˩\textsubscript{a}}}} \textsubscript{1}}\hspace{0.5cm}[\kern2pt{\textcolor{darkblue}{\textbf{\ipa{dze˩˥}}}}\kern2pt]} \hypertarget{dze\string_Ba1}{}
\markboth{\textcolor{darkblue}{\textbf{\ipa{dze˩\textsubscript{a}}}} \textsubscript{1}}{}
\textcolor{teal}{\zh{动词}} \hspace{4pt} \zh{声调类:} L\textsubscript{a}.
\zh{飞。} \textcolor{Sepia}{\selectlanguage{english}To fly.} \textcolor{PineGreen}{\selectlanguage{french}Voler (dans les airs).}  ¶ \textcolor{darkblue}{\textbf{\ipa{le˧-dze˩-hɯ˩-ze˩}}} \zh{(鸟)飞走了。} \textcolor{Sepia}{\selectlanguage{english}(The bird) has flown away.} \textcolor{PineGreen}{\selectlanguage{french}(L'oiseau) est parti à tire-d'aile.}  
 ¶ \textcolor{darkblue}{\textbf{\ipa{mv̩˧ʁo˧ dze˧˥}}} \zh{在天空中飞} \textcolor{Sepia}{\selectlanguage{english}to fly in the sky} \textcolor{PineGreen}{\selectlanguage{french}voler dans le ciel}  

\lhead{\firstmark}
\rhead{\botmark}

\subsection{\hspace{-0.5cm} {\Large \textcolor{darkblue}{\textbf{\ipa{dze˩\textsubscript{a}}}} \textsubscript{2}}\hspace{0.5cm}[\kern2pt{\textcolor{darkblue}{\textbf{\ipa{dze˩˥}}}}\kern2pt]} \hypertarget{dze\string_Ba2}{}
\markboth{\textcolor{darkblue}{\textbf{\ipa{dze˩\textsubscript{a}}}} \textsubscript{2}}{}
\textcolor{teal}{\zh{动词}} \hspace{4pt} \zh{声调类:} L\textsubscript{a}.
\zh{切(用刀)。} \textcolor{Sepia}{\selectlanguage{english}To cut (with a knife).} \textcolor{PineGreen}{\selectlanguage{french}Couper (avec un couteau).}  ¶ \textcolor{darkblue}{\textbf{\ipa{le˧-dze˩}}} \zh{\mytextsc{accomp}} \textcolor{Sepia}{\selectlanguage{english}\mytextsc{accomp}} \textcolor{PineGreen}{\selectlanguage{french}\mytextsc{accomp}}  
 ¶ \textcolor{darkblue}{\textbf{\ipa{dze˧\textasciitilde{}dze˥}}} \zh{\mytextsc{red}} \textcolor{Sepia}{\selectlanguage{english}\mytextsc{red}} \textcolor{PineGreen}{\selectlanguage{french}\mytextsc{red}}  
 ¶ \textcolor{darkblue}{\textbf{\ipa{le˧-dze˧\textasciitilde{}dze˥}}} \zh{\mytextsc{accomp} \string_ \mytextsc{red}} \textcolor{Sepia}{\selectlanguage{english}\mytextsc{accomp} \string_ \mytextsc{red}} \textcolor{PineGreen}{\selectlanguage{french}\mytextsc{accomp} \string_ \mytextsc{red}}  
 ¶ \textcolor{darkblue}{\textbf{\ipa{v̩˩tsʰɤ˧ dze˧\textasciitilde{}dze˥}}} \zh{切菜} \textcolor{Sepia}{\selectlanguage{english}to cut vegetables} \textcolor{PineGreen}{\selectlanguage{french}découper des légumes}  
 ¶ \textcolor{darkblue}{\textbf{\ipa{nv̩˩dʑɯ˥ dze˩\textasciitilde{}dze˩}}} \zh{切豆腐} \textcolor{Sepia}{\selectlanguage{english}to cut tofu} \textcolor{PineGreen}{\selectlanguage{french}découper du tofu}  

\lhead{\firstmark}
\rhead{\botmark}

\subsection{\hspace{-0.5cm} {\Large \textcolor{darkblue}{\textbf{\ipa{dze˩dʑɯ˧˥}}}}\hspace{0.5cm}[\kern2pt{\textcolor{darkblue}{\textbf{\ipa{dze˧dʑɯ˩}}}}\kern2pt]} \hypertarget{dze\string_Bdz£M\string_M\string_T1}{}
\markboth{\textcolor{darkblue}{\textbf{\ipa{dze˩dʑɯ˧˥}}}}{}
\textcolor{teal}{\zh{形容词}} \hspace{4pt} \zh{声调类:} LM+MH\#.
\zh{骄傲,自以为好。} \textcolor{Sepia}{\selectlanguage{english}Arrogant, conceited.} \textcolor{PineGreen}{\selectlanguage{french}Orgueilleux, arrogant.}  ¶ \textcolor{darkblue}{\textbf{\ipa{ʈʂʰɯ˧ | hĩ˧-bi˥ | mɤ˧-li˧! | dze˩dʑɯ˧˥ | ʐwæ˧˥!}}} \zh{他看不起别人!他很骄傲!} \textcolor{Sepia}{\selectlanguage{english}He despises others! He is very arrogant!} \textcolor{PineGreen}{\selectlanguage{french}il méprise les autres! il est très orgueilleux!}  

\lhead{\firstmark}
\rhead{\botmark}

\subsection{\hspace{-0.5cm} {\Large \textcolor{darkblue}{\textbf{\ipa{dze˩mi˧}}}}\hspace{0.5cm}[\kern2pt{\textcolor{darkblue}{\textbf{\ipa{dze˧mi˧}}}}\kern2pt]} \hypertarget{dze\string_Bmi\string_M1}{}
\markboth{\textcolor{darkblue}{\textbf{\ipa{dze˩mi˧}}}}{}
\textcolor{teal}{\zh{名词}} \hspace{4pt} \zh{声调类:} LM.
\zh{蜜蜂。} \textcolor{Sepia}{\selectlanguage{english}Bee.} \textcolor{PineGreen}{\selectlanguage{french}Abeille.}  \zh{量词}: \textcolor{darkblue}{\textbf{\ipa{mi˩}}} 
\lhead{\firstmark}
\rhead{\botmark}

\subsection{\hspace{-0.5cm} {\Large \textcolor{darkblue}{\textbf{\ipa{dze˩mi˧-bæ˩bæ˩}}}}\hspace{0.5cm}[\kern2pt{\textcolor{darkblue}{\textbf{\ipa{xxxx non-correspondance entre le nombre de morphèmes et le nombre de tons de morphèmes}}}}\kern2pt]} \hypertarget{dze\string_Bmi\string_M-b\{\string_Bb\{\string_B1}{}
\markboth{\textcolor{darkblue}{\textbf{\ipa{dze˩mi˧-bæ˩bæ˩}}}}{}
\textcolor{teal}{\zh{名词}} \hspace{4pt} \zh{声调类:} L\#.
\zh{茶绒蒿。} \textcolor{Sepia}{\selectlanguage{english}\textit{Artemisia suboligata}.} \textcolor{PineGreen}{\selectlanguage{french}\textit{Artemisia suboligata}; littéralement “la fleur aux abeilles”.}  \zh{量词}: \textcolor{darkblue}{\textbf{\ipa{bæ˩}}} 
\lhead{\firstmark}
\rhead{\botmark}

\subsection{\hspace{-0.5cm} {\Large \textcolor{darkblue}{\textbf{\ipa{dze˩mi˧-dze\#˥}}}}\hspace{0.5cm}[\kern2pt{\textcolor{darkblue}{\textbf{\ipa{xxxx non-correspondance entre le nombre de morphèmes et le nombre de tons de morphèmes}}}}\kern2pt]} \hypertarget{dze\string_Bmi\string_M-dze\#\string_T1}{}
\markboth{\textcolor{darkblue}{\textbf{\ipa{dze˩mi˧-dze\#˥}}}}{}
\textcolor{teal}{\zh{名词}} \hspace{4pt} \zh{声调类:} LM+\#H.
\zh{蜂蜜。} \textcolor{Sepia}{\selectlanguage{english}Honey.} \textcolor{PineGreen}{\selectlanguage{french}Miel.}  ¶ \textcolor{darkblue}{\textbf{\ipa{dze˩mi˧dze˧ dzɯ˧}}} \zh{吃蜂蜜} \textcolor{Sepia}{\selectlanguage{english}to eat honey} \textcolor{PineGreen}{\selectlanguage{french}manger du miel}  
 \zh{量词}: \textcolor{darkblue}{\textbf{\ipa{kʰwɤ˥}}} 
\lhead{\firstmark}
\rhead{\botmark}

\subsection{\hspace{-0.5cm} {\Large \textcolor{darkblue}{\textbf{\ipa{dze˩mi˧-kʰv̩˩}}}}\hspace{0.5cm}[\kern2pt{\textcolor{darkblue}{\textbf{\ipa{xxxx non-correspondance entre le nombre de morphèmes et le nombre de tons de morphèmes}}}}\kern2pt]} \hypertarget{dze\string_Bmi\string_M-k\string_hv\string_=\string_B1}{}
\markboth{\textcolor{darkblue}{\textbf{\ipa{dze˩mi˧-kʰv̩˩}}}}{}
\textcolor{teal}{\zh{名词}} \hspace{4pt} \zh{声调类:} LM-L.
\zh{蜂窝。} \textcolor{Sepia}{\selectlanguage{english}Beehive, honeycomb.} \textcolor{PineGreen}{\selectlanguage{french}Ruche.}  \zh{量词}: \textcolor{darkblue}{\textbf{\ipa{ɭɯ˧}}} 
\lhead{\firstmark}
\rhead{\botmark}

\subsection{\hspace{-0.5cm} {\Large \textcolor{darkblue}{\textbf{\ipa{dze˩mi˧-pv̩˥ɻ̍˩}}}}\hspace{0.5cm}[\kern2pt{\textcolor{darkblue}{\textbf{\ipa{dze˩mi˧pv̩˩ɻ̍˩}}}}\kern2pt]} \hypertarget{dze\string_Bmi\string_M-pv\string_=\string_Tr£`̍\string_B1}{}
\markboth{\textcolor{darkblue}{\textbf{\ipa{dze˩mi˧-pv̩˥ɻ̍˩}}}}{}
\textcolor{teal}{\zh{名词}} \hspace{4pt} \zh{声调类:} LM+\#H-.
\zh{沙参。} \textcolor{Sepia}{\selectlanguage{english}Terme générique pour les fleurs qu'affectionnent les abeilles: diverses fleurs dont les vertus médicinales ne sont pas connues. Par exemple: \textit{Adenophora sp.}: root of straight ladybell (flower). yyyy.} \textcolor{PineGreen}{\selectlanguage{french}\textit{Adenophora sp.}.} \zh{当地汉语方言:}\zh{yyyy fusionner les 2 entrées; est un terme générique ; tʰi˧-pv˥ɻ˩ = tʰi˧-hɑ̃˧˥ se reposer qq part, se poser quelque part。} ¶ \textcolor{darkblue}{\textbf{\ipa{dze˩mi˧-pv̩˥ɻ̍˩-kʰɯ˩ʈɯ˩}}} \zh{沙参根} \textcolor{Sepia}{\selectlanguage{english}root of straight ladybell} \textcolor{PineGreen}{\selectlanguage{french}racine d'\textit{Adenophora sp.}}  

\lhead{\firstmark}
\rhead{\botmark}

\subsection{\hspace{-0.5cm} {\Large \textcolor{darkblue}{\textbf{\ipa{dze˩mi˧-pv̩˥ɻ̍˩}}}}\hspace{0.5cm}[\kern2pt{\textcolor{darkblue}{\textbf{\ipa{dze˩mi˧pv̩˥ɻ̍˩}}}}\kern2pt]} \hypertarget{dze\string_Bmi\string_M-pv\string_=\string_Tr£`̍\string_B1}{}
\markboth{\textcolor{darkblue}{\textbf{\ipa{dze˩mi˧-pv̩˥ɻ̍˩}}}}{}
\textcolor{teal}{\zh{名词}} \hspace{4pt} \zh{声调类:} LM+\#H-.
\zh{秦艽。} \textcolor{Sepia}{\selectlanguage{english}Large-leaved gentian.} \textcolor{PineGreen}{\selectlanguage{french}Gentiane.}  ¶ \textcolor{darkblue}{\textbf{\ipa{dʑɯ˧qʰɑ˧-bæ˩bæ˩}}} \zh{秦艽花} \textcolor{Sepia}{\selectlanguage{english}gentian flowers} \textcolor{PineGreen}{\selectlanguage{french}fleurs de gentiane}  
 \zh{量词}: \textcolor{darkblue}{\textbf{\ipa{qɑ˩}}} 
\lhead{\firstmark}
\rhead{\botmark}

\subsection{\hspace{-0.5cm} {\Large \textcolor{darkblue}{\textbf{\ipa{*dze˩˧}}}}\hspace{0.5cm}[\kern2pt{\textcolor{darkblue}{\textbf{\ipa{dze˩˥}}}}\kern2pt]} \hypertarget{*dze\string_B\string_M1}{}
\markboth{\textcolor{darkblue}{\textbf{\ipa{*dze˩˧}}}}{}
\textcolor{teal}{\zh{名词}} \hspace{4pt} \zh{声调类:} LM.
\zh{蜜蜂。} \textcolor{Sepia}{\selectlanguage{english}Bee.} \textcolor{PineGreen}{\selectlanguage{french}Abeille (racine déduite du disyllabe).} 
\lhead{\firstmark}
\rhead{\botmark}

\subsection{\hspace{-0.5cm} {\Large \textcolor{darkblue}{\textbf{\ipa{dze˩˧}}}}\hspace{0.5cm}[\kern2pt{\textcolor{darkblue}{\textbf{\ipa{dze˥}}}}\kern2pt]} \hypertarget{dze\string_B\string_M1}{}
\markboth{\textcolor{darkblue}{\textbf{\ipa{dze˩˧}}}}{}
\textcolor{teal}{\zh{名词}} \hspace{4pt} \zh{声调类:} LM.
\zh{花椒。} \textcolor{Sepia}{\selectlanguage{english}Wild pepper, Szechuan pepper.} \textcolor{PineGreen}{\selectlanguage{french}Xanthoxyle, poivre de Chine, poivre du Sichuan.}  \zh{量词}: \textcolor{darkblue}{\textbf{\ipa{mɤ˩}}} 
\lhead{\firstmark}
\rhead{\botmark}

\subsection{\hspace{-0.5cm} {\Large \textcolor{darkblue}{\textbf{\ipa{dzɤ˥\textsubscript{b}}}}}\hspace{0.5cm}[\kern2pt{\textcolor{darkblue}{\textbf{\ipa{dzɤ˥}}}}\kern2pt]} \hypertarget{dz7\string_Tb1}{}
\markboth{\textcolor{darkblue}{\textbf{\ipa{dzɤ˥\textsubscript{b}}}}}{}
\textcolor{teal}{\zh{量词}} \hspace{4pt} \zh{声调类:} H\textsubscript{b}.
\zh{量词:面。} \textcolor{Sepia}{\selectlanguage{english}Classifier for sides.} \textcolor{PineGreen}{\selectlanguage{french}Côté.}  ¶ \textcolor{darkblue}{\textbf{\ipa{ʈʂʰɯ˧-dzɤ˧}}} \zh{这面} \textcolor{Sepia}{\selectlanguage{english}this side} \textcolor{PineGreen}{\selectlanguage{french}ce côté-ci}  
 ¶ \textcolor{darkblue}{\textbf{\ipa{ɖɯ˧-dzɤ˥}}} \zh{一面} \textcolor{Sepia}{\selectlanguage{english}one side} \textcolor{PineGreen}{\selectlanguage{french}un côté}  

\lhead{\firstmark}
\rhead{\botmark}

\subsection{\hspace{-0.5cm} {\Large \textcolor{darkblue}{\textbf{\ipa{dzɤ˩\textsubscript{a}}}}}\hspace{0.5cm}[\kern2pt{\textcolor{darkblue}{\textbf{\ipa{dzɤ˧˥}}}}\kern2pt]} \hypertarget{dz7\string_Ba1}{}
\markboth{\textcolor{darkblue}{\textbf{\ipa{dzɤ˩\textsubscript{a}}}}}{}
\textcolor{teal}{\zh{动词}} \hspace{4pt} \zh{声调类:} L\textsubscript{a}.
\zh{塌毁,倒塌 ,倒。} \textcolor{Sepia}{\selectlanguage{english}To collapse, to topple over, to fall into ruin.} \textcolor{PineGreen}{\selectlanguage{french}S’écrouler, s'effondrer (mur); tomber (arbre), se renverser, s'abattre.}  ¶ \textcolor{darkblue}{\textbf{\ipa{mv̩˩tɕo˧ dzɤ˩}}} \zh{同上:塌毁} \textcolor{Sepia}{\selectlanguage{english}same meaning: to collapse} \textcolor{PineGreen}{\selectlanguage{french}même sens: s'écrouler}  
 ¶ \textcolor{darkblue}{\textbf{\ipa{le˧-dzɤ˩-ze˩}}} \zh{塌毁了} \textcolor{Sepia}{\selectlanguage{english}\mytextsc{accomp} \string_ \mytextsc{pfv}} \textcolor{PineGreen}{\selectlanguage{french}\mytextsc{accomp} \string_ \mytextsc{pfv}}  

\lhead{\firstmark}
\rhead{\botmark}

\subsection{\hspace{-0.5cm} {\Large \textcolor{darkblue}{\textbf{\ipa{dzi˥}}}}\hspace{0.5cm}[\kern2pt{\textcolor{darkblue}{\textbf{\ipa{dzi˧˥}}}}\kern2pt]} \hypertarget{dzi\string_T1}{}
\markboth{\textcolor{darkblue}{\textbf{\ipa{dzi˥}}}}{}
\textcolor{teal}{\zh{名词}} \hspace{4pt} \zh{声调类:} \#H.
\zh{凿子。} \textcolor{Sepia}{\selectlanguage{english}Chisel.} \textcolor{PineGreen}{\selectlanguage{french}Burin, ciseau.}  \zh{量词}: \textcolor{darkblue}{\textbf{\ipa{ɭɯ˧ (*nɑ˧)}}} 
\lhead{\firstmark}
\rhead{\botmark}

\subsection{\hspace{-0.5cm} {\Large \textcolor{darkblue}{\textbf{\ipa{dzi˧\textsubscript{b}}}}}\hspace{0.5cm}[\kern2pt{\textcolor{darkblue}{\textbf{\ipa{dzi˩˥}}}}\kern2pt]} \hypertarget{dzi\string_Mb1}{}
\markboth{\textcolor{darkblue}{\textbf{\ipa{dzi˧\textsubscript{b}}}}}{}
\textcolor{teal}{\zh{量词}} \hspace{4pt} \zh{声调类:} M\textsubscript{b}.
\zh{量词:鞋(一双)。} \textcolor{Sepia}{\selectlanguage{english}Classifier for pairs of objects, when the pair makes up a unit: e.g. a pair of shoes.} \textcolor{PineGreen}{\selectlanguage{french}Paire d'objets qui constitue une unité: par exemple une paire de chaussures.}  ¶ \textcolor{darkblue}{\textbf{\ipa{ɣɯ˩-dzɑ˩qʰwɤ˥ | ɖɯ˧-dzi˧}}} \zh{一双皮鞋} \textcolor{Sepia}{\selectlanguage{english}a pair of leather shoes} \textcolor{PineGreen}{\selectlanguage{french}une paire de chaussures en cuir}  

\lhead{\firstmark}
\rhead{\botmark}

\subsection{\hspace{-0.5cm} {\Large \textcolor{darkblue}{\textbf{\ipa{dzi˧dzi˧}}}}\hspace{0.5cm}[\kern2pt{\textcolor{darkblue}{\textbf{\ipa{dzi˧dzi˧}}}}\kern2pt]} \hypertarget{dzi\string_Mdzi\string_M1}{}
\markboth{\textcolor{darkblue}{\textbf{\ipa{dzi˧dzi˧}}}}{}
\textcolor{teal}{\zh{名词}} \hspace{4pt} \zh{声调类:} M.
\zh{青冈树、槲栎。} \textcolor{Sepia}{\selectlanguage{english}Oriental white oak.} \textcolor{PineGreen}{\selectlanguage{french}Chêne blanc oriental.}  ¶ \textcolor{darkblue}{\textbf{\ipa{dzi˧dzi˧, | si˧dzi˩-mv̩˩!}}} \zh{\textcolor{darkblue}{\textbf{\ipa{dzi˧dzi˧}}}是一种树的名字!} \textcolor{Sepia}{\selectlanguage{english}\textcolor{darkblue}{\textbf{\ipa{/dzi˧dzi˧/}}} is the name of a tree!} \textcolor{PineGreen}{\selectlanguage{french}\textcolor{darkblue}{\textbf{\ipa{/dzi˧dzi˧/}}}, c'est un nom d'arbre! / C'est le nom d'un arbre!}  
\zh{~【同义词】~} \hyperlink{}{\textcolor{darkblue}{\textbf{\ipa{dʑɯ˩si˩}}}}. 
\lhead{\firstmark}
\rhead{\botmark}

\subsection{\hspace{-0.5cm} {\Large \textcolor{darkblue}{\textbf{\ipa{dzi˧dzi˧-mo˧˥}}}}\hspace{0.5cm}[\kern2pt{\textcolor{darkblue}{\textbf{\ipa{xxxx non-correspondance entre le nombre de morphèmes et le nombre de tons de morphèmes}}}}\kern2pt]} \hypertarget{dzi\string_Mdzi\string_M-mo\string_M\string_T1}{}
\markboth{\textcolor{darkblue}{\textbf{\ipa{dzi˧dzi˧-mo˧˥}}}}{}
\textcolor{teal}{\zh{名词}} \hspace{4pt} \zh{声调类:} MH\#.
\zh{一种可以吃的菌子,长在枯木上。} \textcolor{Sepia}{\selectlanguage{english}A species of edible mushroom that grows on fallen trees; it is used as a medicine against stomach-ache.} \textcolor{PineGreen}{\selectlanguage{french}Champignon comestible, qui ne pousse pas sur la terre, seulement sur les arbres tombés; est utilisé comme médicament pour les maux d'estomac.} 
\lhead{\firstmark}
\rhead{\botmark}

\subsection{\hspace{-0.5cm} {\Large \textcolor{darkblue}{\textbf{\ipa{dzi˧ɖæ˧}}}}\hspace{0.5cm}[\kern2pt{\textcolor{darkblue}{\textbf{\ipa{dzi˧ɖæ˧}}}}\kern2pt]} \hypertarget{dzi\string_Md`\{\string_M1}{}
\markboth{\textcolor{darkblue}{\textbf{\ipa{dzi˧ɖæ˧}}}}{}
\textcolor{teal}{\zh{名词}} \hspace{4pt} \zh{声调类:} M.
\zh{位置、所在地。} \textcolor{Sepia}{\selectlanguage{english}Location.} \textcolor{PineGreen}{\selectlanguage{french}Emplacement, localisation (emploi typique: emplacement d'une maison).}  \zh{量词}: \textcolor{darkblue}{\textbf{\ipa{kʰwɤ˥}}} 
\lhead{\firstmark}
\rhead{\botmark}

\subsection{\hspace{-0.5cm} {\Large \textcolor{darkblue}{\textbf{\ipa{dzi˩}}} \textsubscript{1}}\hspace{0.5cm}[\kern2pt{\textcolor{darkblue}{\textbf{\ipa{dzi˥}}}}\kern2pt]} \hypertarget{dzi\string_B1}{}
\markboth{\textcolor{darkblue}{\textbf{\ipa{dzi˩}}} \textsubscript{1}}{}
\textcolor{teal}{\zh{动词}} \hspace{4pt} \zh{声调类:} L.
\zh{来(晚上来了)。} \textcolor{Sepia}{\selectlanguage{english}To fall, to come (of night); to get (dark).} \textcolor{PineGreen}{\selectlanguage{french}Tomber, venir (la nuit tombe, la nuit vient).}  ¶ \textcolor{darkblue}{\textbf{\ipa{nɑ˩˥ | le˧-dzi˩-ze˩!}}} \zh{天黑了!} \textcolor{Sepia}{\selectlanguage{english}The night has fallen! / It has got dark!} \textcolor{PineGreen}{\selectlanguage{french}la nuit est tombée! Il fait noir!}  
 ¶ \textcolor{darkblue}{\textbf{\ipa{nɑ˩˥ | le˧-dzi˩ | le˧-se˩-ze˩!}}} \zh{天完全黑了!} \textcolor{Sepia}{\selectlanguage{english}It has got completely dark!} \textcolor{PineGreen}{\selectlanguage{french}il fait tout à fait nuit!}  

\lhead{\firstmark}
\rhead{\botmark}

\subsection{\hspace{-0.5cm} {\Large \textcolor{darkblue}{\textbf{\ipa{dzi˩}}} \textsubscript{2}}\hspace{0.5cm}[\kern2pt{\textcolor{darkblue}{\textbf{\ipa{dzi˩˥}}}}\kern2pt]} \hypertarget{dzi\string_B2}{}
\markboth{\textcolor{darkblue}{\textbf{\ipa{dzi˩}}} \textsubscript{2}}{}
\textcolor{teal}{\zh{量词}} \hspace{4pt} \zh{声调类:} L\textsubscript{c}.
\zh{量词:衣服(一套)。} \textcolor{Sepia}{\selectlanguage{english}Classifier for entire dresses.} \textcolor{PineGreen}{\selectlanguage{french}Un costume entier: tous les vêtements qu'on porte.}  ¶ \textcolor{darkblue}{\textbf{\ipa{dʑi˧hṽ˥ | ɖɯ˧-dzi˩}}} \zh{一套衣服} \textcolor{Sepia}{\selectlanguage{english}a full set of dress, a complete dress} \textcolor{PineGreen}{\selectlanguage{french}un costume entier, un vêtement}  
 ¶ \textcolor{darkblue}{\textbf{\ipa{dʑi˧hṽ˧ ɖɯ˧-dzi˩}}} \zh{一套衣服(同上,但将短语合在一起,构成一个单一的声调短语)} \textcolor{Sepia}{\selectlanguage{english}a full set of dress, a complete dress (same as preceding example, integrated into a single tone group)} \textcolor{PineGreen}{\selectlanguage{french}un costume entier, un vêtement (même sens que l'exemple précédent; intégration en un seul groupe tonal)}  

\lhead{\firstmark}
\rhead{\botmark}

\subsection{\hspace{-0.5cm} {\Large \textcolor{darkblue}{\textbf{\ipa{dzi˩\textsubscript{a}}}} \textsubscript{1}}\hspace{0.5cm}[\kern2pt{\textcolor{darkblue}{\textbf{\ipa{dzi˥}}}}\kern2pt]} \hypertarget{dzi\string_Ba1}{}
\markboth{\textcolor{darkblue}{\textbf{\ipa{dzi˩\textsubscript{a}}}} \textsubscript{1}}{}
\textcolor{teal}{\zh{动词}} \hspace{4pt} \zh{声调类:} L\textsubscript{a}.
\ding{202} \zh{坐。} \textcolor{Sepia}{\selectlanguage{english}To sit.} \textcolor{PineGreen}{\selectlanguage{french}S'asseoir, être assis.}  ¶ \textcolor{darkblue}{\textbf{\ipa{tʰi˧-dzi˩!}}} \zh{坐下!} \textcolor{Sepia}{\selectlanguage{english}Sit down!} \textcolor{PineGreen}{\selectlanguage{french}Asseyez-vous!}  
 ¶ \textcolor{darkblue}{\textbf{\ipa{hĩ˧bæ˧ ʈʂʰɯ˧-qo˧ dzi˩.}}} \zh{客人是坐在这边的。} \textcolor{Sepia}{\selectlanguage{english}The guest sits here.} \textcolor{PineGreen}{\selectlanguage{french}L'invité s'asseoit ici.}  
 ¶ \textcolor{darkblue}{\textbf{\ipa{(ʈʂʰɯ˧ | ) tʰi˧-dzi˩-kʰɯ˩-se˩.}}} \zh{他坐下了。} \textcolor{Sepia}{\selectlanguage{english}(S)he has got seated.} \textcolor{PineGreen}{\selectlanguage{french}Il est assis/installé, il a pris sa place.}  
 ¶ \textcolor{darkblue}{\textbf{\ipa{le˧-dzi˧\textasciitilde{}dzi˥}}} \zh{坐一坐。来指:居丧、守灵(委婉语)} \textcolor{Sepia}{\selectlanguage{english}to remain seated; used as a euphemism to mean: to sit with others at a funeral wake, to keep a deathwatch} \textcolor{PineGreen}{\selectlanguage{french}se tenir assis; s'emploie, par euphémisme, pour désigner la participation à une veillée funèbre}  
\ding{203} \zh{住。} \textcolor{Sepia}{\selectlanguage{english}To dwell, to live at a place.} \textcolor{PineGreen}{\selectlanguage{french}Demeurer, habiter.}  ¶ \textcolor{darkblue}{\textbf{\ipa{dzi˩-bi˩-ni˩gv̩˩}}} \zh{习惯(一个新的环境、一个地方的饮食……)} \textcolor{Sepia}{\selectlanguage{english}to be accustomed to; to get accustomed to, to feel at ease, to adapt (to an environment)} \textcolor{PineGreen}{\selectlanguage{french}s'habituer (à un environnement; à des habitudes de vie: nourriture, boisson...; à quelque chose; à quelqu'un)}  
 ¶ \textcolor{darkblue}{\textbf{\ipa{dzi˩-bi˩-ni˩-mɤ˩-gv̩˩˥}}} \zh{不习惯} \textcolor{Sepia}{\selectlanguage{english}not to get accustomed; to feel awkward} \textcolor{PineGreen}{\selectlanguage{french}je n'aime pas/je ne m'y fais pas/ça ne me plaît pas (ex.: quelqu'un ne veut pas rester quelque part, il s'y trouve mal et a la nostalgie d'ailleurs)}  
 ¶ \textcolor{darkblue}{\textbf{\ipa{njɤ˧ | ʈʂʰɯ˧-qo˧ | dzi˩-bi˩-ni˩-mɤ˩-gv̩˩˥}}} \zh{我不习惯这里! / 我不喜欢这里! / 我不想待了!} \textcolor{Sepia}{\selectlanguage{english}I can't get accustomed (to this place)! / (I) can't make myself at ease here! / (I) don't like it here!} \textcolor{PineGreen}{\selectlanguage{french}Je ne me fais pas à ici! / Je ne suis pas bien ici! (Paraphrase proposée par M23: “Je ne veux pas rester ici!”)}  

\lhead{\firstmark}
\rhead{\botmark}

\subsection{\hspace{-0.5cm} {\Large \textcolor{darkblue}{\textbf{\ipa{dzi˩\textsubscript{a}}}} \textsubscript{2}}\hspace{0.5cm}[\kern2pt{\textcolor{darkblue}{\textbf{\ipa{dzi˩˥}}}}\kern2pt]} \hypertarget{dzi\string_Ba2}{}
\markboth{\textcolor{darkblue}{\textbf{\ipa{dzi˩\textsubscript{a}}}} \textsubscript{2}}{}
\textcolor{teal}{\zh{动词}} \hspace{4pt} \zh{声调类:} L\textsubscript{a}.
\zh{聚集。} \textcolor{Sepia}{\selectlanguage{english}To gather, to assemble (people gather together).} \textcolor{PineGreen}{\selectlanguage{french}Se rassembler (groupe de personnes).}  ¶ \textcolor{darkblue}{\textbf{\ipa{ɖɯ˧-ʁwɤ˧ | le˧-dzi˧\textasciitilde{}dzi˥}}} \zh{全村都聚集在一起} \textcolor{Sepia}{\selectlanguage{english}the whole village gathered together} \textcolor{PineGreen}{\selectlanguage{french}tout le village se rassemble}  
 ¶ \textcolor{darkblue}{\textbf{\ipa{hĩ˧ ɖɯ˧-v̩˧ | le˧-ʂɯ˧-ze˧! le˧-dzi˧\textasciitilde{}dzi˥ jo˩!}}} \zh{一个人去世了!来参加丧礼吧!} \textcolor{Sepia}{\selectlanguage{english}Someone has passed away! Come and join the gathering!} \textcolor{PineGreen}{\selectlanguage{french}Quelqu'un est mort! Venez participer à la réunion/au rassemblement!}  

\lhead{\firstmark}
\rhead{\botmark}

\subsection{\hspace{-0.5cm} {\Large \textcolor{darkblue}{\textbf{\ipa{dzi˩\textsubscript{a}}}} \textsubscript{3}}\hspace{0.5cm}[\kern2pt{\textcolor{darkblue}{\textbf{\ipa{dzi˩˥}}}}\kern2pt]} \hypertarget{dzi\string_Ba3}{}
\markboth{\textcolor{darkblue}{\textbf{\ipa{dzi˩\textsubscript{a}}}} \textsubscript{3}}{}
\textcolor{teal}{\zh{动词}} \hspace{4pt} \zh{声调类:} L\textsubscript{a}.
\zh{掉入、沉下去。} \textcolor{Sepia}{\selectlanguage{english}To drop, to fall, to sink (e.g. boat slowly sinking down into a lake).} \textcolor{PineGreen}{\selectlanguage{french}Tomber, sombrer (ex.: un bateau qui sombre peu à peu dans le lac).}  ¶ \textcolor{darkblue}{\textbf{\ipa{mv̩˩tɕo˧ dzi˩}}} \zh{往下掉入、沉下去} \textcolor{Sepia}{\selectlanguage{english}to sink down} \textcolor{PineGreen}{\selectlanguage{french}\mytextsc{directionnel} \string_ ; même sens: sombrer, couler}  

\lhead{\firstmark}
\rhead{\botmark}

\subsection{\hspace{-0.5cm} {\Large \textcolor{darkblue}{\textbf{\ipa{dzi˩\textsubscript{b}}}}}\hspace{0.5cm}[\kern2pt{\textcolor{darkblue}{\textbf{\ipa{dzi˩˥}}}}\kern2pt]} \hypertarget{dzi\string_Bb1}{}
\markboth{\textcolor{darkblue}{\textbf{\ipa{dzi˩\textsubscript{b}}}}}{}
\textcolor{teal}{\zh{量词}} \hspace{4pt} \zh{声调类:} L\textsubscript{b}.
\zh{量词:树(一棵),竹子(一根)。} \textcolor{Sepia}{\selectlanguage{english}Classifier for trees, bamboos….} \textcolor{PineGreen}{\selectlanguage{french}Classificateur des arbres.}  ¶ \textcolor{darkblue}{\textbf{\ipa{si˧dzi˩ | ɖɯ˧-dzi˩}}} \zh{一棵树} \textcolor{Sepia}{\selectlanguage{english}a tree} \textcolor{PineGreen}{\selectlanguage{french}un arbre}  
 ¶ \textcolor{darkblue}{\textbf{\ipa{tʰv̩˧-dzi˧˥}}} \zh{那棵树} \textcolor{Sepia}{\selectlanguage{english}that tree} \textcolor{PineGreen}{\selectlanguage{french}cet arbre}  

\lhead{\firstmark}
\rhead{\botmark}

\subsection{\hspace{-0.5cm} {\Large \textcolor{darkblue}{\textbf{\ipa{dzi˩ʁo˩}}}}\hspace{0.5cm}[\kern2pt{\textcolor{darkblue}{\textbf{\ipa{dzi˧ʁo˧}}}}\kern2pt]} \hypertarget{dzi\string_BRo\string_B1}{}
\markboth{\textcolor{darkblue}{\textbf{\ipa{dzi˩ʁo˩}}}}{}
\textcolor{teal}{\zh{名词}} \hspace{4pt} \zh{声调类:} L.
\zh{座位。} \textcolor{Sepia}{\selectlanguage{english}Seat, place.} \textcolor{PineGreen}{\selectlanguage{french}Place, place assise.}  \zh{量词}: \textcolor{darkblue}{\textbf{\ipa{kʰwɤ˥}}} 
\lhead{\firstmark}
\rhead{\botmark}

\subsection{\hspace{-0.5cm} {\Large \textcolor{darkblue}{\textbf{\ipa{dzi˩tsʰɤ˩}}}}\hspace{0.5cm}[\kern2pt{\textcolor{darkblue}{\textbf{\ipa{dzi˩tsʰɤ˩˥}}}}\kern2pt]} \hypertarget{dzi\string_Bts\string_h7\string_B1}{}
\markboth{\textcolor{darkblue}{\textbf{\ipa{dzi˩tsʰɤ˩}}}}{}
\textcolor{teal}{\zh{名词}} \hspace{4pt} \zh{声调类:} L.
\zh{永宁的一种灌木。} \textcolor{Sepia}{\selectlanguage{english}A shrub with sharp thorns.} \textcolor{PineGreen}{\selectlanguage{french}Arbuste: sorte de houx, de grande taille.}  \zh{量词}: \textcolor{darkblue}{\textbf{\ipa{dzi˩}}} 
\lhead{\firstmark}
\rhead{\botmark}

\subsection{\hspace{-0.5cm} {\Large \textcolor{darkblue}{\textbf{\ipa{dzi˧˥}}}}\hspace{0.5cm}[\kern2pt{\textcolor{darkblue}{\textbf{\ipa{dzi˩˥}}}}\kern2pt]} \hypertarget{dzi\string_M\string_T1}{}
\markboth{\textcolor{darkblue}{\textbf{\ipa{dzi˧˥}}}}{}
\textcolor{teal}{\zh{动词}} \hspace{4pt} \zh{声调类:} MH.
\zh{颤抖、抖动。} \textcolor{Sepia}{\selectlanguage{english}To tremble, to shake.} \textcolor{PineGreen}{\selectlanguage{french}Trembler.}  ¶ \textcolor{darkblue}{\textbf{\ipa{njɤ˩ dzi˧˥}}} \zh{眼皮跳} \textcolor{Sepia}{\selectlanguage{english}the eyelid trembles (literally 'the eye trembles')} \textcolor{PineGreen}{\selectlanguage{french}la paupière tremble}  
 ¶ \textcolor{darkblue}{\textbf{\ipa{njɤ˩ dzi˧-ze˥}}} \zh{眼皮跳} \textcolor{Sepia}{\selectlanguage{english}the eyelid trembles} \textcolor{PineGreen}{\selectlanguage{french}la paupière tremble}  
 ¶ \textcolor{darkblue}{\textbf{\ipa{njɤ˩ dzi˧˥ | ʐwæ˩˥}}} \zh{眼皮跳得很厉害} \textcolor{Sepia}{\selectlanguage{english}the eyelid trembles terribly} \textcolor{PineGreen}{\selectlanguage{french}la paupière tremble très fort}  
 ¶ \textcolor{darkblue}{\textbf{\ipa{njɤ˩˥ | le˧-dzi˧-ze˥}}} \zh{眼皮跳} \textcolor{Sepia}{\selectlanguage{english}the eyelid trembles} \textcolor{PineGreen}{\selectlanguage{french}la paupière tremble}  
 ¶ \textcolor{darkblue}{\textbf{\ipa{njɤ˩˥ | mɤ˧-dzi˧˥}}} \zh{眼皮不跳} \textcolor{Sepia}{\selectlanguage{english}the eyelid does not tremble} \textcolor{PineGreen}{\selectlanguage{french}la paupière ne tremble pas}  

\lhead{\firstmark}
\rhead{\botmark}

\subsection{\hspace{-0.5cm} {\Large \textcolor{darkblue}{\textbf{\ipa{dzo˥}}}}\hspace{0.5cm}[\kern2pt{\textcolor{darkblue}{\textbf{\ipa{dzo˩˥}}}}\kern2pt]} \hypertarget{dzo\string_T1}{}
\markboth{\textcolor{darkblue}{\textbf{\ipa{dzo˥}}}}{}
\textcolor{teal}{\zh{名词}} \hspace{4pt} \zh{声调类:} \#H.
\zh{冰雹。} \textcolor{Sepia}{\selectlanguage{english}Hail.} \textcolor{PineGreen}{\selectlanguage{french}Grêle.}  ¶ \textcolor{darkblue}{\textbf{\ipa{dzo˧ lɑ˩}}} \zh{下冰雹} \textcolor{Sepia}{\selectlanguage{english}there is some hail} \textcolor{PineGreen}{\selectlanguage{french}il grêle}  
 ¶ \textcolor{darkblue}{\textbf{\ipa{dzo˧ gi˧-ze˩}}} \zh{下冰雹了} \textcolor{Sepia}{\selectlanguage{english}there is some hail} \textcolor{PineGreen}{\selectlanguage{french}il tombe de la grêle}  

\lhead{\firstmark}
\rhead{\botmark}

\subsection{\hspace{-0.5cm} {\Large \textcolor{darkblue}{\textbf{\ipa{dzo˧-lv̩˧\textasciitilde{}lv̩˥}}}}\hspace{0.5cm}[\kern2pt{\textcolor{darkblue}{\textbf{\ipa{xxxx non-correspondance entre le nombre de morphèmes et le nombre de tons de morphèmes}}}}\kern2pt]} \hypertarget{dzo\string_M-lv\string_=\string_M~lv\string_=\string_T1}{}
\markboth{\textcolor{darkblue}{\textbf{\ipa{dzo˧-lv̩˧\textasciitilde{}lv̩˥}}}}{}
\textcolor{teal}{\zh{名词}} \hspace{4pt} \zh{声调类:} H\#.
\zh{冰雹。} \textcolor{Sepia}{\selectlanguage{english}Hailstone.} \textcolor{PineGreen}{\selectlanguage{french}Grêlon.}  \zh{量词}: \textcolor{darkblue}{\textbf{\ipa{ɭɯ˧}}} 
\lhead{\firstmark}
\rhead{\botmark}

\subsection{\hspace{-0.5cm} {\Large \textcolor{darkblue}{\textbf{\ipa{dzo˧mi˧}}}}\hspace{0.5cm}[\kern2pt{\textcolor{darkblue}{\textbf{\ipa{dzo˧mi˥}}}}\kern2pt]} \hypertarget{dzo\string_Mmi\string_M1}{}
\markboth{\textcolor{darkblue}{\textbf{\ipa{dzo˧mi˧}}}}{}
\textcolor{teal}{\zh{名词}} \hspace{4pt} \zh{声调类:} M.
\zh{大桶。} \textcolor{Sepia}{\selectlanguage{english}Large vat.} \textcolor{PineGreen}{\selectlanguage{french}Grande cuve, grand tonneau (sens très proche du précédent, peuvent s'employer de façon interchangeable pour certains des tonneaux).}  \zh{量词}: \textcolor{darkblue}{\textbf{\ipa{ɭɯ˧}}} 
\lhead{\firstmark}
\rhead{\botmark}

\subsection{\hspace{-0.5cm} {\Large \textcolor{darkblue}{\textbf{\ipa{dzo˧zo\#˥}}}}\hspace{0.5cm}[\kern2pt{\textcolor{darkblue}{\textbf{\ipa{dzo˩zo˥}}}}\kern2pt]} \hypertarget{dzo\string_Mzo\#\string_T1}{}
\markboth{\textcolor{darkblue}{\textbf{\ipa{dzo˧zo\#˥}}}}{}
\textcolor{teal}{\zh{名词}} \hspace{4pt} \zh{声调类:} \#H.
\zh{小桶。} \textcolor{Sepia}{\selectlanguage{english}Small vat.} \textcolor{PineGreen}{\selectlanguage{french}Petite cuve.}  \zh{量词}: \textcolor{darkblue}{\textbf{\ipa{ɭɯ˧}}} 
\lhead{\firstmark}
\rhead{\botmark}

\subsection{\hspace{-0.5cm} {\Large \textcolor{darkblue}{\textbf{\ipa{dzo˩}}}}\hspace{0.5cm}[\kern2pt{\textcolor{darkblue}{\textbf{\ipa{dzo˥}}}}\kern2pt]} \hypertarget{dzo\string_B1}{}
\markboth{\textcolor{darkblue}{\textbf{\ipa{dzo˩}}}}{}
\textcolor{teal}{\zh{名词}} \hspace{4pt} \zh{声调类:} L.
\zh{桥。} \textcolor{Sepia}{\selectlanguage{english}Bridge.} \textcolor{PineGreen}{\selectlanguage{french}Pont.}  ¶ \textcolor{darkblue}{\textbf{\ipa{dzo˧ | ɖɯ˧ pɤ˩}}} \zh{一辆桥} \textcolor{Sepia}{\selectlanguage{english}a bridge} \textcolor{PineGreen}{\selectlanguage{french}un pont}  
 ¶ \textcolor{darkblue}{\textbf{\ipa{dzo˩ bæ˩˥}}} \zh{扫桥} \textcolor{Sepia}{\selectlanguage{english}to sweep (a/the) bridge} \textcolor{PineGreen}{\selectlanguage{french}balayer le pont}  
 ¶ \textcolor{darkblue}{\textbf{\ipa{njɤ˧ | dzo˩ bæ˩-zo˩-ho˥.}}} \zh{我要扫桥了。} \textcolor{Sepia}{\selectlanguage{english}I have to sweep the bridge.} \textcolor{PineGreen}{\selectlanguage{french}Il va falloir que je balaie le pont.}  
 ¶ \textcolor{darkblue}{\textbf{\ipa{dzo˩ gv̩˩˥}}} \zh{建一辆桥} \textcolor{Sepia}{\selectlanguage{english}to build (/repair) a bridge} \textcolor{PineGreen}{\selectlanguage{french}construire (ou réparer) un pont}  
 ¶ \textcolor{darkblue}{\textbf{\ipa{njɤ˧ | dzo˩ gv̩˩-zo˩-ho˥.}}} \textcolor{Sepia}{\selectlanguage{english}I have to build (/repair) a bridge.} \textcolor{PineGreen}{\selectlanguage{french}Il va falloir que je construise (/répare) un pont.}  
 \zh{量词}: \textcolor{darkblue}{\textbf{\ipa{pɤ˩}}} \textcolor{darkblue}{\textbf{\ipa{nɑ˧}}} 
\lhead{\firstmark}
\rhead{\botmark}

\subsection{\hspace{-0.5cm} {\Large \textcolor{darkblue}{\textbf{\ipa{dzo˩\textasciitilde{}dzo˧˥}}}}\hspace{0.5cm}[\kern2pt{\textcolor{darkblue}{\textbf{\ipa{dzo˧dzo˧}}}}\kern2pt]} \hypertarget{dzo\string_B~dzo\string_M\string_T1}{}
\markboth{\textcolor{darkblue}{\textbf{\ipa{dzo˩\textasciitilde{}dzo˧˥}}}}{}
\textcolor{teal}{\zh{动词}} \hspace{4pt} \zh{声调类:} MH.
\zh{嘲笑、取笑。} \textcolor{Sepia}{\selectlanguage{english}To laugh at, to poke fun at, to mock, to ridicule.} \textcolor{PineGreen}{\selectlanguage{french}Se moquer de, rire de, ridiculiser.}  ¶ \textcolor{darkblue}{\textbf{\ipa{hĩ˧ dzo˧-dzo˥-ho˩-ni˩zo˩!}}} \zh{他好像要开始取笑人家了!} \textcolor{Sepia}{\selectlanguage{english}It looks like (he/she) is going to poke fun (at...)} \textcolor{PineGreen}{\selectlanguage{french}On dirait qu'(elle/il) va tourner quelqu'un en dérision}  
 ¶ \textcolor{darkblue}{\textbf{\ipa{tʰɑ˧ dzo˩\textasciitilde{}dzo˩!}}} \zh{别嘲笑(人家)!} \textcolor{Sepia}{\selectlanguage{english}Don't laugh at people!} \textcolor{PineGreen}{\selectlanguage{french}Ne vous moquez pas!/ Pas de sarcasmes!}  

\lhead{\firstmark}
\rhead{\botmark}

\subsection{\hspace{-0.5cm} {\Large \textcolor{darkblue}{\textbf{\ipa{dzo˩mi\#˥}}}}\hspace{0.5cm}[\kern2pt{\textcolor{darkblue}{\textbf{\ipa{dzo˧mi˧}}}}\kern2pt]} \hypertarget{dzo\string_Bmi\#\string_T1}{}
\markboth{\textcolor{darkblue}{\textbf{\ipa{dzo˩mi\#˥}}}}{}
\textcolor{teal}{\zh{名词}} \hspace{4pt} \zh{声调类:} LM+\#H.
\zh{母壁虎。} \textcolor{Sepia}{\selectlanguage{english}Female lizard.} \textcolor{PineGreen}{\selectlanguage{french}Lézard femelle.}  ¶ \textcolor{darkblue}{\textbf{\ipa{dzo˧mi˧-dzo˩pʰv̩˩}}} \zh{母壁虎与公壁虎} \textcolor{Sepia}{\selectlanguage{english}female lizard and male lizard} \textcolor{PineGreen}{\selectlanguage{french}lézard femelle et lézard mâle}  
 \zh{量词}: \textcolor{darkblue}{\textbf{\ipa{mi˩}}} 
\lhead{\firstmark}
\rhead{\botmark}

\subsection{\hspace{-0.5cm} {\Large \textcolor{darkblue}{\textbf{\ipa{dzo˩pʰv̩˩}}}}\hspace{0.5cm}[\kern2pt{\textcolor{darkblue}{\textbf{\ipa{dzo˩pʰv̩˥}}}}\kern2pt]} \hypertarget{dzo\string_Bp\string_hv\string_=\string_B1}{}
\markboth{\textcolor{darkblue}{\textbf{\ipa{dzo˩pʰv̩˩}}}}{}
\textcolor{teal}{\zh{名词}} \hspace{4pt} \zh{声调类:} L.
\zh{公壁虎。} \textcolor{Sepia}{\selectlanguage{english}Male lizard.} \textcolor{PineGreen}{\selectlanguage{french}Lézard mâle.}  ¶ \textcolor{darkblue}{\textbf{\ipa{dzo˩pʰv̩˩-dzo˩mi˥}}} \zh{公壁虎与母壁虎} \textcolor{Sepia}{\selectlanguage{english}male lizard and female lizard} \textcolor{PineGreen}{\selectlanguage{french}lézard mâle et lézard femelle}  
 \zh{量词}: \textcolor{darkblue}{\textbf{\ipa{mi˩}}} \textcolor{darkblue}{\textbf{\ipa{ɭɯ˧}}} 
\lhead{\firstmark}
\rhead{\botmark}

\subsection{\hspace{-0.5cm} {\Large \textcolor{darkblue}{\textbf{\ipa{dzo˩zo\#˥}}}}\hspace{0.5cm}[\kern2pt{\textcolor{darkblue}{\textbf{\ipa{dzo˩zo˥}}}}\kern2pt]} \hypertarget{dzo\string_Bzo\#\string_T1}{}
\markboth{\textcolor{darkblue}{\textbf{\ipa{dzo˩zo\#˥}}}}{}
\textcolor{teal}{\zh{名词}} \hspace{4pt} \zh{声调类:} LM+\#H.
\zh{小壁虎。} \textcolor{Sepia}{\selectlanguage{english}Baby lizard.} \textcolor{PineGreen}{\selectlanguage{french}Bébé lézard.}  \zh{量词}: \textcolor{darkblue}{\textbf{\ipa{mi˩}}} \textcolor{darkblue}{\textbf{\ipa{ɭɯ˧}}} 
\lhead{\firstmark}
\rhead{\botmark}

\subsection{\hspace{-0.5cm} {\Large \textcolor{darkblue}{\textbf{\ipa{dzo˩˧}}}}\hspace{0.5cm}[\kern2pt{\textcolor{darkblue}{\textbf{\ipa{dzo˥}}}}\kern2pt]} \hypertarget{dzo\string_B\string_M1}{}
\markboth{\textcolor{darkblue}{\textbf{\ipa{dzo˩˧}}}}{}
\textcolor{teal}{\zh{名词}} \hspace{4pt} \zh{声调类:} LM.
\zh{壁虎,蜥蜴,四脚蛇。} \textcolor{Sepia}{\selectlanguage{english}Lizard.} \textcolor{PineGreen}{\selectlanguage{french}Lézard.}  ¶ \textcolor{darkblue}{\textbf{\ipa{dzo˩ hwæ˧-ze˧}}} \zh{买了壁虎} \textcolor{Sepia}{\selectlanguage{english}...bought (a) lizard} \textcolor{PineGreen}{\selectlanguage{french}...a acheté (un) lézard}  
 ¶ \textcolor{darkblue}{\textbf{\ipa{dzo˩ dzɯ˧-ze˩}}} \zh{吃了壁虎} \textcolor{Sepia}{\selectlanguage{english}...ate (a) lizard} \textcolor{PineGreen}{\selectlanguage{french}...a mangé (un) lézard}  
 \zh{量词}: \textcolor{darkblue}{\textbf{\ipa{mi˩}}} 
\lhead{\firstmark}
\rhead{\botmark}

\subsection{\hspace{-0.5cm} {\Large \textcolor{darkblue}{\textbf{\ipa{dzɯ˥}}}}\hspace{0.5cm}[\kern2pt{\textcolor{darkblue}{\textbf{\ipa{dzɯ˧˥}}}}\kern2pt]} \hypertarget{dzM\string_T1}{}
\markboth{\textcolor{darkblue}{\textbf{\ipa{dzɯ˥}}}}{}
\textcolor{teal}{\zh{动词}} \hspace{4pt} \zh{声调类:} H.
\zh{吃。} \textcolor{Sepia}{\selectlanguage{english}To eat.} \textcolor{PineGreen}{\selectlanguage{french}Manger.}  ¶ \textcolor{darkblue}{\textbf{\ipa{le˧-dzɯ˥}}} \zh{\mytextsc{实施}} \textcolor{Sepia}{\selectlanguage{english}\mytextsc{accomp}} \textcolor{PineGreen}{\selectlanguage{french}\mytextsc{accomp}}  
 ¶ \textcolor{darkblue}{\textbf{\ipa{hɑ˧ dzɯ˧}}} \zh{吃饭} \textcolor{Sepia}{\selectlanguage{english}to have a meal; to take some food} \textcolor{PineGreen}{\selectlanguage{french}prendre un repas, manger de la nourriture}  
 ¶ \textcolor{darkblue}{\textbf{\ipa{njɤ˧ | hɑ˧ le˧-dzɯ˥-ze˩}}} \zh{我吃过饭了。} \textcolor{Sepia}{\selectlanguage{english}I have eaten. / I have had a meal.} \textcolor{PineGreen}{\selectlanguage{french}J'ai mangé. J'ai pris mon repas.}  
 ¶ \textcolor{darkblue}{\textbf{\ipa{dzɯ˧-di˧˥}}} \zh{吃的(东西)} \textcolor{Sepia}{\selectlanguage{english}food, thing to eat} \textcolor{PineGreen}{\selectlanguage{french}nourriture, chose à manger}  
 ¶ \textcolor{darkblue}{\textbf{\ipa{dzɯ˧-bi˩-ze˩!}}} \zh{要吃饭了!} \textcolor{Sepia}{\selectlanguage{english}Let's eat! / It's time to eat!} \textcolor{PineGreen}{\selectlanguage{french}(On) va manger! / C'est l'heure de manger!}  

\lhead{\firstmark}
\rhead{\botmark}

\subsection{\hspace{-0.5cm} {\Large \textcolor{darkblue}{\textbf{\ipa{dzɯ˧tsɯ˧˥}}}}\hspace{0.5cm}[\kern2pt{\textcolor{darkblue}{\textbf{\ipa{dzɯ˧tsɯ˧˥}}}}\kern2pt]} \hypertarget{dzM\string_MtsM\string_M\string_T1}{}
\markboth{\textcolor{darkblue}{\textbf{\ipa{dzɯ˧tsɯ˧˥}}}}{}
\textcolor{teal}{\zh{名词}} \hspace{4pt} \zh{声调类:} MH\#.
\zh{永宁的一种灌木。} \textcolor{Sepia}{\selectlanguage{english}A shrub that grows in Yongning.} \textcolor{PineGreen}{\selectlanguage{french}Sorte de houx, petites feuilles à piquants.}  \zh{量词}: \textcolor{darkblue}{\textbf{\ipa{pʰæ˧˥}}} 
\lhead{\firstmark}
\rhead{\botmark}

\newpage
\section*{\centering- \textcolor{darkblue}{\textbf{\ipa{dʑ}}} -}
\subsection{\hspace{-0.5cm} {\Large \textcolor{darkblue}{\textbf{\ipa{dʑɤ˧bo˩}}}}\hspace{0.5cm}[\kern2pt{\textcolor{darkblue}{\textbf{\ipa{dʑɤ˩bo˩˥}}}}\kern2pt]} \hypertarget{dz£7\string_Mbo\string_B1}{}
\markboth{\textcolor{darkblue}{\textbf{\ipa{dʑɤ˧bo˩}}}}{}
\textcolor{teal}{\zh{名词}} \hspace{4pt} \zh{声调类:} L\#.
\zh{粮仓。} \textcolor{Sepia}{\selectlanguage{english}Granary (a special building).} \textcolor{PineGreen}{\selectlanguage{french}Grenier à céréale; selon M23, n'existe plus: était, dans le temps, un bâtiment exclusivement consacré à la conservation des céréales: un grenier.}  \zh{量词}: \textcolor{darkblue}{\textbf{\ipa{ɭɯ˧}}} 
\lhead{\firstmark}
\rhead{\botmark}

\subsection{\hspace{-0.5cm} {\Large \textcolor{darkblue}{\textbf{\ipa{dʑɤ˧do˩}}}}\hspace{0.5cm}[\kern2pt{\textcolor{darkblue}{\textbf{\ipa{dʑɤ˩do˩˥}}}}\kern2pt]} \hypertarget{dz£7\string_Mdo\string_B1}{}
\markboth{\textcolor{darkblue}{\textbf{\ipa{dʑɤ˧do˩}}}}{}
\textcolor{teal}{\zh{名词}} \hspace{4pt} \zh{声调类:} L\#.
\zh{中甸。} \textcolor{Sepia}{\selectlanguage{english}Zhongdian (place name).} \textcolor{PineGreen}{\selectlanguage{french}Zhongdian (nom de lieu).}  ¶ \textcolor{darkblue}{\textbf{\ipa{dʑɤ˧do˩-bɤ˩}}} \zh{中甸普米族} \textcolor{Sepia}{\selectlanguage{english}the Pumi people of Zhongdian} \textcolor{PineGreen}{\selectlanguage{french}les Pumi de Zhongdian}  

\lhead{\firstmark}
\rhead{\botmark}

\subsection{\hspace{-0.5cm} {\Large \textcolor{darkblue}{\textbf{\ipa{dʑɤ˩\textsubscript{b}}}}}\hspace{0.5cm}[\kern2pt{\textcolor{darkblue}{\textbf{\ipa{dʑɤ˥}}}}\kern2pt]} \hypertarget{dz£7\string_Bb1}{}
\markboth{\textcolor{darkblue}{\textbf{\ipa{dʑɤ˩\textsubscript{b}}}}}{}
\textcolor{teal}{\zh{形容词}} \hspace{4pt} \zh{声调类:} L\textsubscript{b}.
\zh{好。} \textcolor{Sepia}{\selectlanguage{english}Good (good decision).} \textcolor{PineGreen}{\selectlanguage{french}Bon (bonne décision).}  ¶ \textcolor{darkblue}{\textbf{\ipa{mɤ˧-dʑɤ˩}}} \zh{坏} \textcolor{Sepia}{\selectlanguage{english}bad} \textcolor{PineGreen}{\selectlanguage{french}pas bien, mauvais}  
 ¶ \textcolor{darkblue}{\textbf{\ipa{dʑɤ˩-hĩ˥}}} \zh{\mytextsc{rel}} \textcolor{Sepia}{\selectlanguage{english}\mytextsc{rel}} \textcolor{PineGreen}{\selectlanguage{french}\mytextsc{rel}}  
 ¶ \textcolor{darkblue}{\textbf{\ipa{(no˧) ɖwæ˧˥ | dʑɤ˩˥!}}} \zh{你很好!} \textcolor{Sepia}{\selectlanguage{english}You're great!} \textcolor{PineGreen}{\selectlanguage{french}Tu es quelqu'un de bien!}  
 ¶ \textcolor{darkblue}{\textbf{\ipa{dʑɤ˩-kʰɯ˥!}}} \zh{新年祝福:“祝一切好! / 万事如意!”} \textcolor{Sepia}{\selectlanguage{english}A benediction used on the New Year: “Let there be good (things)!”, i.e. “Prosperity!”, “All the best for the New Year!”} \textcolor{PineGreen}{\selectlanguage{french}bénédiction dite au Nouvel An: “Bonnes (choses)!” =“Prospérité!” =“Bonne année!”}  
 ¶ \textcolor{darkblue}{\textbf{\ipa{no˧ | le˧-ʝi˥ | dʑɤ˩˥, | hĩ˧-ɳɯ˩ | do˩˥! | ʈʂʰɯ˧ | le˧-ʝi˥ | mɤ˧-dʑɤ˩, | hĩ˧-ɳɯ˩ | ʐwɤ˩˥!}}} \zh{你做得好,人家(会)发现!他做的不好,人家(会)说(他)!} \textcolor{Sepia}{\selectlanguage{english}“If you do well, people see it / people realize so! If (s)he does badly, people say so!” = “A job well done earns recognition; a job badly done earns criticism!” (Context: talking about a bad book. In the Na world view as remembered by the consultant, there is no question that it is better to do good, that good deeds and good attitudes eventually get rewarded, and bad deeds and bad attitudes eventually get punished.} \textcolor{PineGreen}{\selectlanguage{french}“Si tu travailles bien, les gens s'en rendent compte! S'il travail mal, les gens le disent!” = “Le bon travail est reconnu; le mauvais travail reçoit des critiques!” (Contexte: commentaire au sujet d'un mauvais livre. C'est ce qu'on disait autrefois: la belle ouvrage est reconnue; le mauvais travail vous attire des critiques.)}  

\lhead{\firstmark}
\rhead{\botmark}

\subsection{\hspace{-0.5cm} {\Large \textcolor{darkblue}{\textbf{\ipa{dʑɤ˩bv̩˥}}}}\hspace{0.5cm}[\kern2pt{\textcolor{darkblue}{\textbf{\ipa{dʑɤ˧bv̩˩}}}}\kern2pt]} \hypertarget{dz£7\string_Bbv\string_=\string_T1}{}
\markboth{\textcolor{darkblue}{\textbf{\ipa{dʑɤ˩bv̩˥}}}}{}
\textcolor{teal}{\zh{动词}} \hspace{4pt} \zh{声调类:} LH.
\zh{玩,玩耍。} \textcolor{Sepia}{\selectlanguage{english}To play.} \textcolor{PineGreen}{\selectlanguage{french}Jouer.}  ¶ \textcolor{darkblue}{\textbf{\ipa{dʑɤ˩bv̩˥ -bi˩/-ze˩}}} \zh{要玩耍 / 玩耍了} \textcolor{Sepia}{\selectlanguage{english}\string_ \mytextsc{fut}.imm/\mytextsc{pfv}} \textcolor{PineGreen}{\selectlanguage{french}\string_ \mytextsc{fut}.imm/\mytextsc{pfv}}  
 ¶ \textcolor{darkblue}{\textbf{\ipa{le˧-dʑɤ˩bv̩˩ +ze˩}}} \zh{玩耍了} \textcolor{Sepia}{\selectlanguage{english}\mytextsc{accomp} \string_ \mytextsc{pfv}} \textcolor{PineGreen}{\selectlanguage{french}\mytextsc{accomp} \string_ \mytextsc{pfv}}  
 ¶ \textcolor{darkblue}{\textbf{\ipa{tʰi˧-dʑɤ˩bv̩˩, | tʰi˧-dʑɤ˩bv̩˩, | le˧-fv̩˧!}}} \zh{他们玩着玩着,很高兴!(情景:几个小孩子一起玩)} \textcolor{Sepia}{\selectlanguage{english}They play, they play... they're happy! / (By) playing on and on, they get really happy! (About children playing together)} \textcolor{PineGreen}{\selectlanguage{french}ils jouent, ils jouent… ils sont contents! (Au sujet d'enfants qui jouent ensemble)}  

\lhead{\firstmark}
\rhead{\botmark}

\subsection{\hspace{-0.5cm} {\Large \textcolor{darkblue}{\textbf{\ipa{dʑɤ˩bv̩˥-di˩}}}}\hspace{0.5cm}[\kern2pt{\textcolor{darkblue}{\textbf{\ipa{xxxx non-correspondance entre le nombre de morphèmes et le nombre de tons de morphèmes}}}}\kern2pt]} \hypertarget{dz£7\string_Bbv\string_=\string_T-di\string_B1}{}
\markboth{\textcolor{darkblue}{\textbf{\ipa{dʑɤ˩bv̩˥-di˩}}}}{}
\textcolor{teal}{\zh{名词}} \hspace{4pt} \zh{声调类:} LH-.
\zh{玩具。} \textcolor{Sepia}{\selectlanguage{english}Toy.} \textcolor{PineGreen}{\selectlanguage{french}Jouet.} 
\lhead{\firstmark}
\rhead{\botmark}

\subsection{\hspace{-0.5cm} {\Large \textcolor{darkblue}{\textbf{\ipa{dʑɤ˩bv̩˧kɤ˧-sɑ˥ʁwɤ˩}}}}\hspace{0.5cm}[\kern2pt{\textcolor{darkblue}{\textbf{\ipa{dʑɤ˩bv̩˩kɤ˥sɑ˧ʁwɤ˧}}}}\kern2pt]} \hypertarget{dz£7\string_Bbv\string_=\string_Mk7\string_M-sA\string_TRw7\string_B1}{}
\markboth{\textcolor{darkblue}{\textbf{\ipa{dʑɤ˩bv̩˧kɤ˧-sɑ˥ʁwɤ˩}}}}{}
\textcolor{teal}{\zh{名词}} \hspace{4pt} \zh{声调类:} LM+MH\#-.
\zh{高明 (摩梭话名称的汉译:嘎撒瓦)(永宁的一个村落)。} \textcolor{Sepia}{\selectlanguage{english}Gaoming, a village north-east of Yongning).} \textcolor{PineGreen}{\selectlanguage{french}Gaoming, un village au nord-est de Yongning.}  ¶ \textcolor{darkblue}{\textbf{\ipa{dʑɤ˩bv̩˧kɤ˧-sɑ˥ʁwɤ˩, | hi˩ʁwɤ˩-lo˥, | æ˩mi˧-ʁwɤ\#˥, | lɑ˧lo˧-ʁwɤ˥, | lɑ˧ŋwɤ˧, | bɤ˧tsʰo˧gv̩˥, | ə˧lɑ˧-ʁwɤ\#˥, | gæ˧ɻæ˩, | qʰæ˧tɕʰi˧, | tʰo˧ʈɯ\#˥}}} \zh{摩梭传统地理概念中,属于永宁的十个村落} \textcolor{Sepia}{\selectlanguage{english}the ten villages traditionally considered as part of Yongning} \textcolor{PineGreen}{\selectlanguage{french}les dix villages comptant traditionnellement comme faisant partie de Yongning}  

\lhead{\firstmark}
\rhead{\botmark}

\subsection{\hspace{-0.5cm} {\Large \textcolor{darkblue}{\textbf{\ipa{dʑɤ˩bv̩˥-ʁwɤ˩}}}}\hspace{0.5cm}[\kern2pt{\textcolor{darkblue}{\textbf{\ipa{dʑɤ˩bv̩˧˥ʁwɤ˧}}}}\kern2pt]} \hypertarget{dz£7\string_Bbv\string_=\string_T-Rw7\string_B1}{}
\markboth{\textcolor{darkblue}{\textbf{\ipa{dʑɤ˩bv̩˥-ʁwɤ˩}}}}{}
\textcolor{teal}{\zh{名词}} \hspace{4pt} \zh{声调类:} LH-.
\zh{甲波瓦(永宁的一个村落)。} \textcolor{Sepia}{\selectlanguage{english}Jiabowa (name of a village).} \textcolor{PineGreen}{\selectlanguage{french}Jiabowa (nom de village).} 
\lhead{\firstmark}
\rhead{\botmark}

\subsection{\hspace{-0.5cm} {\Large \textcolor{darkblue}{\textbf{\ipa{dʑɤ˩ɕjɤ˩}}}}\hspace{0.5cm}[\kern2pt{\textcolor{darkblue}{\textbf{\ipa{xxxx non-correspondance entre le nombre de morphèmes et le nombre de tons de morphèmes}}}}\kern2pt]} \hypertarget{dz£7\string_Bs£j7\string_B1}{}
\markboth{\textcolor{darkblue}{\textbf{\ipa{dʑɤ˩ɕjɤ˩}}}}{}
\textcolor{teal}{\zh{名词}} \hspace{4pt} \zh{声调类:} L.
\zh{鞋垫。} \textcolor{Sepia}{\selectlanguage{english}Shoe-pad; insole.} \textcolor{PineGreen}{\selectlanguage{french}Semelle (en paille); chausson (en paille).} 
\lhead{\firstmark}
\rhead{\botmark}

\subsection{\hspace{-0.5cm} {\Large \textcolor{darkblue}{\textbf{\ipa{dʑɤ˩kʰwɤ˧}}}}\hspace{0.5cm}[\kern2pt{\textcolor{darkblue}{\textbf{\ipa{xxxx ton non trouvé, à faire manuellement...}}}}\kern2pt]} \hypertarget{dz£7\string_Bk\string_hw7\string_M1}{}
\markboth{\textcolor{darkblue}{\textbf{\ipa{dʑɤ˩kʰwɤ˧}}}}{}
\textcolor{teal}{\zh{名词}} \hspace{4pt} \zh{声调类:} LM.
\zh{距离。} \textcolor{Sepia}{\selectlanguage{english}Distance.} \textcolor{PineGreen}{\selectlanguage{french}Distance.}  ¶ \textcolor{darkblue}{\textbf{\ipa{no˧ | ʈʂʰɯ˧ | ə˩-ʐæ˥ʂæ˩? | dʑɤ˩kʰwɤ˧ ə˩-di˩? | - dʑɤ˩˥ | dʑɤ˩kʰwɤ˧ mɤ˧-di˥! | mɤ˧-ʐæ˩ʂæ˩!}}} \zh{你们很熟吗?/ 你们很亲吗? - 不很熟!/ 不很亲!} \textcolor{Sepia}{\selectlanguage{english}Are you distant from him? Is there distance (between you)? - There is not much distance to speak of! We are not distant! (=we are close friends)} \textcolor{PineGreen}{\selectlanguage{french}tu es loin de lui? Y a-t-il de la distance entre vous? (=vous êtes proches/intimes, ou pas?) - Non, il n'y a guère de distance! Nous ne somme pas éloignés!}  

\lhead{\firstmark}
\rhead{\botmark}

\subsection{\hspace{-0.5cm} {\Large \textcolor{darkblue}{\textbf{\ipa{dʑɤ˩pi\#˥}}}}\hspace{0.5cm}[\kern2pt{\textcolor{darkblue}{\textbf{\ipa{dʑɤ˩pi˥}}}}\kern2pt]} \hypertarget{dz£7\string_Bpi\#\string_T1}{}
\markboth{\textcolor{darkblue}{\textbf{\ipa{dʑɤ˩pi\#˥}}}}{}
\textcolor{teal}{\zh{形容词}} \hspace{4pt} \zh{声调类:} LM+\#H.
\zh{多。} \textcolor{Sepia}{\selectlanguage{english}Plenty of.} \textcolor{PineGreen}{\selectlanguage{french}Beaucoup.}  ¶ \textcolor{darkblue}{\textbf{\ipa{dʑɤ˩pi˧ ʝi˧}}} \zh{很有用、有很多用处} \textcolor{Sepia}{\selectlanguage{english}It's very useful!} \textcolor{PineGreen}{\selectlanguage{french}C'est très utile}  
 ¶ \textcolor{darkblue}{\textbf{\ipa{dʑɤ˩pi˧ dʑo˧!}}} \zh{我有很多!} \textcolor{Sepia}{\selectlanguage{english}(I) have plenty! / (I) have a lot! (possession)} \textcolor{PineGreen}{\selectlanguage{french}J'en ai beaucoup! (possession)}  
 ¶ \textcolor{darkblue}{\textbf{\ipa{dʑɤ˩pi˧ dʑo˧˥}}} \zh{有很多。} \textcolor{Sepia}{\selectlanguage{english}There is plenty / there is a lot. (Note: existence, and not possession)} \textcolor{PineGreen}{\selectlanguage{french}Il en existe beaucoup, il y en a beaucoup. (Note: existentiel, et non possession)}  

\lhead{\firstmark}
\rhead{\botmark}

\subsection{\hspace{-0.5cm} {\Large \textcolor{darkblue}{\textbf{\ipa{dʑɤ˩qʰɑ˥}}}}\hspace{0.5cm}[\kern2pt{\textcolor{darkblue}{\textbf{\ipa{dʑɤ˩qʰɑ˩˥}}}}\kern2pt]} \hypertarget{dz£7\string_Bq\string_hA\string_T1}{}
\markboth{\textcolor{darkblue}{\textbf{\ipa{dʑɤ˩qʰɑ˥}}}}{}
\textcolor{teal}{\zh{助词}} \hspace{4pt} \zh{声调类:} LH.
\zh{一直、一个劲地。} \textcolor{Sepia}{\selectlanguage{english}Continuously, with full might.} \textcolor{PineGreen}{\selectlanguage{french}Continuellement, par un effort soutenu.}  ¶ \textcolor{darkblue}{\textbf{\ipa{dʑɤ˩qʰɑ˥ ʈɤ˩}}} \zh{一个劲地拉} \textcolor{Sepia}{\selectlanguage{english}to pull with full might} \textcolor{PineGreen}{\selectlanguage{french}tirer en un effort continu}  
 ¶ \textcolor{darkblue}{\textbf{\ipa{dʑɤ˩qʰɑ˥ mi˩}}} \zh{一个劲地推} \textcolor{Sepia}{\selectlanguage{english}to push with full might} \textcolor{PineGreen}{\selectlanguage{french}appuyer en un effort continu}  
 ¶ \textcolor{darkblue}{\textbf{\ipa{dʑɤ˩qʰɑ˥ lɑ˩}}} \zh{一个劲地打} \textcolor{Sepia}{\selectlanguage{english}to beat with full might} \textcolor{PineGreen}{\selectlanguage{french}frapper continuellement, en un effort continu}  

\lhead{\firstmark}
\rhead{\botmark}

\subsection{\hspace{-0.5cm} {\Large \textcolor{darkblue}{\textbf{\ipa{dʑɤ˩so˧}}}}\hspace{0.5cm}[\kern2pt{\textcolor{darkblue}{\textbf{\ipa{dʑɤ˧so˧}}}}\kern2pt]} \hypertarget{dz£7\string_Bso\string_M1}{}
\markboth{\textcolor{darkblue}{\textbf{\ipa{dʑɤ˩so˧}}}}{}
\textcolor{teal}{\zh{形容词}} \hspace{4pt} \zh{声调类:} LM.
\zh{好几(个)。} \textcolor{Sepia}{\selectlanguage{english}Many, a great number of.} \textcolor{PineGreen}{\selectlanguage{french}En abondance, beaucoup.}  ¶ \textcolor{darkblue}{\textbf{\ipa{dʑɤ˩-so˧ ɲi˧}}} \zh{好几天} \textcolor{Sepia}{\selectlanguage{english}many days; a long time} \textcolor{PineGreen}{\selectlanguage{french}beaucoup de jours, longtemps}  

\lhead{\firstmark}
\rhead{\botmark}

\subsection{\hspace{-0.5cm} {\Large \textcolor{darkblue}{\textbf{\ipa{dʑɤ˩tsʰi\#˥}}}}\hspace{0.5cm}[\kern2pt{\textcolor{darkblue}{\textbf{\ipa{xxxx non-correspondance entre le nombre de morphèmes et le nombre de tons de morphèmes}}}}\kern2pt]} \hypertarget{dz£7\string_Bts\string_hi\#\string_T1}{}
\markboth{\textcolor{darkblue}{\textbf{\ipa{dʑɤ˩tsʰi\#˥}}}}{}
\textcolor{teal}{\zh{名词}} \hspace{4pt} \zh{声调类:} LM+\#H.
\zh{男性名字。} \textcolor{Sepia}{\selectlanguage{english}Masculine given name.} \textcolor{PineGreen}{\selectlanguage{french}Prénom masculin.} 
\lhead{\firstmark}
\rhead{\botmark}

\subsection{\hspace{-0.5cm} {\Large \textcolor{darkblue}{\textbf{\ipa{dʑɤ˩tsʰi˧-ɖɯ˩mɑ˩}}}}\hspace{0.5cm}[\kern2pt{\textcolor{darkblue}{\textbf{\ipa{xxxx non-correspondance entre le nombre de morphèmes et le nombre de tons de morphèmes}}}}\kern2pt]} \hypertarget{dz£7\string_Bts\string_hi\string_M-d`M\string_BmA\string_B1}{}
\markboth{\textcolor{darkblue}{\textbf{\ipa{dʑɤ˩tsʰi˧-ɖɯ˩mɑ˩}}}}{}
\textcolor{teal}{\zh{名词}} \hspace{4pt} \zh{声调类:} LM-L.
\zh{女性名字。} \textcolor{Sepia}{\selectlanguage{english}Feminine given name.} \textcolor{PineGreen}{\selectlanguage{french}Prénom féminin.} 
\lhead{\firstmark}
\rhead{\botmark}

\subsection{\hspace{-0.5cm} {\Large \textcolor{darkblue}{\textbf{\ipa{dʑɤ˩tsʰi˧-tsi˩mv̩˩}}}}\hspace{0.5cm}[\kern2pt{\textcolor{darkblue}{\textbf{\ipa{dʑɤ˩tsʰi˧tsi˩mv̩˩}}}}\kern2pt]} \hypertarget{dz£7\string_Bts\string_hi\string_M-tsi\string_Bmv\string_=\string_B1}{}
\markboth{\textcolor{darkblue}{\textbf{\ipa{dʑɤ˩tsʰi˧-tsi˩mv̩˩}}}}{}
\textcolor{teal}{\zh{名词}} \hspace{4pt} \zh{声调类:} LM-L.
\zh{经幡、风马旗(挂在家旁边的树上,或房顶上)。} \textcolor{Sepia}{\selectlanguage{english}Prayer flag.} \textcolor{PineGreen}{\selectlanguage{french}Drapeau de prière; on en attache sur un arbre proche de la maison, ou sur le sommet de la maison.}  \zh{量词}: \textcolor{darkblue}{\textbf{\ipa{dzi˩}}} 
\lhead{\firstmark}
\rhead{\botmark}

\subsection{\hspace{-0.5cm} {\Large \textcolor{darkblue}{\textbf{\ipa{dʑɤ˩tsʰo˧}}}}\hspace{0.5cm}[\kern2pt{\textcolor{darkblue}{\textbf{\ipa{dʑɤ˩tsʰo˥}}}}\kern2pt]} \hypertarget{dz£7\string_Bts\string_ho\string_M1}{}
\markboth{\textcolor{darkblue}{\textbf{\ipa{dʑɤ˩tsʰo˧}}}}{}
\textcolor{teal}{\zh{动词}} \hspace{4pt} \zh{声调类:} LM.
\zh{跳舞。} \textcolor{Sepia}{\selectlanguage{english}To dance.} \textcolor{PineGreen}{\selectlanguage{french}Danser.}  ¶ \textcolor{darkblue}{\textbf{\ipa{ʈʂʰɯ˧ | dʑɤ˩tsʰo˧ mɤ˧-dʑɤ˩!}}} \zh{他舞跳得不好!} \textcolor{Sepia}{\selectlanguage{english}(S)he does not dance well!} \textcolor{PineGreen}{\selectlanguage{french}il/elle ne danse pas bien!}  
 ¶ \textcolor{darkblue}{\textbf{\ipa{dʑɤ˩tsʰo˧ | ɖɯ˧-hɑ̃˧ tsʰo˧}}} \zh{跳一整夜舞} \textcolor{Sepia}{\selectlanguage{english}to dance all evening, to dance a whole night} \textcolor{PineGreen}{\selectlanguage{french}danser (toute) une soirée}  
 ¶ \textcolor{darkblue}{\textbf{\ipa{ʑi˧qʰwɤ˧-ʂɯ˧-qo˧ | dʑɤ˩tsʰo˧ ʁɑ˧ ʂe˩}}} \zh{在新房子举办乔迁宴会} \textcolor{Sepia}{\selectlanguage{english}to throw a housewarming party in a newly built house} \textcolor{PineGreen}{\selectlanguage{french}organiser une pendaison de crémaillère dans une nouvelle maison}  

\lhead{\firstmark}
\rhead{\botmark}

\subsection{\hspace{-0.5cm} {\Large \textcolor{darkblue}{\textbf{\ipa{dʑi˥}}} \textsubscript{1}}\hspace{0.5cm}[\kern2pt{\textcolor{darkblue}{\textbf{\ipa{dʑi˥}}}}\kern2pt]} \hypertarget{dz£i\string_T1}{}
\markboth{\textcolor{darkblue}{\textbf{\ipa{dʑi˥}}} \textsubscript{1}}{}
\textcolor{teal}{\zh{名词}} \hspace{4pt} \zh{声调类:} \#H.
\zh{尿。} \textcolor{Sepia}{\selectlanguage{english}Urine.} \textcolor{PineGreen}{\selectlanguage{french}Urine.}  ¶ \textcolor{darkblue}{\textbf{\ipa{dʑi˧ bæ˥}}} \zh{扫尿} \textcolor{Sepia}{\selectlanguage{english}to sweep urine} \textcolor{PineGreen}{\selectlanguage{french}balayer l'urine}  
 ¶ \textcolor{darkblue}{\textbf{\ipa{dʑi˧-lɑ˩ | qʰæ˧}}} \zh{大小便的统称} \textcolor{Sepia}{\selectlanguage{english}excrements: urine and faeces} \textcolor{PineGreen}{\selectlanguage{french}excréments: urine et fèces}  

\lhead{\firstmark}
\rhead{\botmark}

\subsection{\hspace{-0.5cm} {\Large \textcolor{darkblue}{\textbf{\ipa{dʑi˥}}} \textsubscript{2}}\hspace{0.5cm}[\kern2pt{\textcolor{darkblue}{\textbf{\ipa{dʑi˥}}}}\kern2pt]} \hypertarget{dz£i\string_T2}{}
\markboth{\textcolor{darkblue}{\textbf{\ipa{dʑi˥}}} \textsubscript{2}}{}
\textcolor{teal}{\zh{名词}} \hspace{4pt} \zh{声调类:} H.
\zh{衣服。} \textcolor{Sepia}{\selectlanguage{english}Clothes, clothing (monosyllabic form).} \textcolor{PineGreen}{\selectlanguage{french}Habit, vêtement (mot monosyllabique).}  ¶ \textcolor{darkblue}{\textbf{\ipa{kʰv̩˧ʂɯ˥, | dʑi˧ qæ˧!}}} \zh{过年,换衣服! / 过年,要穿新衣服!} \textcolor{Sepia}{\selectlanguage{english}On New Year's Eve, one changes one's clothing / one wears new clothes!} \textcolor{PineGreen}{\selectlanguage{french}Au Nouvel An, on change de vêtements/ on porte des vêtements neufs!}  
 ¶ \textcolor{darkblue}{\textbf{\ipa{dʑi˧ qæ˧-ze˩!}}} \zh{换衣服了!} \textcolor{Sepia}{\selectlanguage{english}(He/she) has changed clothes!} \textcolor{PineGreen}{\selectlanguage{french}(il/elle) a changé de vêtements!}  
 ¶ \textcolor{darkblue}{\textbf{\ipa{nɑ˩-dʑi\#˥}}} \zh{摩梭服装} \textcolor{Sepia}{\selectlanguage{english}Na clothing} \textcolor{PineGreen}{\selectlanguage{french}le vêtement des Na}  
 ¶ \textcolor{darkblue}{\textbf{\ipa{hæ˧-dʑi\#˥}}} \zh{汉族服装} \textcolor{Sepia}{\selectlanguage{english}Chinese (Han) clothing} \textcolor{PineGreen}{\selectlanguage{french}le vêtement des Chinois (Han)}  
 \zh{量词}: \textcolor{darkblue}{\textbf{\ipa{ɭɯ˧}}} 
\lhead{\firstmark}
\rhead{\botmark}

\subsection{\hspace{-0.5cm} {\Large \textcolor{darkblue}{\textbf{\ipa{dʑi˧hṽ˥\$}}}}\hspace{0.5cm}[\kern2pt{\textcolor{darkblue}{\textbf{\ipa{dʑi˧hṽ˧˥}}}}\kern2pt]} \hypertarget{dz£i\string_Mhv\string_~\string_T\$1}{}
\markboth{\textcolor{darkblue}{\textbf{\ipa{dʑi˧hṽ˥\$}}}}{}
\textcolor{teal}{\zh{名词}} \hspace{4pt} \zh{声调类:} H\$.
\zh{衣服。} \textcolor{Sepia}{\selectlanguage{english}Clothes, clothing.} \textcolor{PineGreen}{\selectlanguage{french}Habit, vêtement (terme générique).}  \zh{量词}: \textcolor{darkblue}{\textbf{\ipa{ɭɯ˧}}} 
\lhead{\firstmark}
\rhead{\botmark}

\subsection{\hspace{-0.5cm} {\Large \textcolor{darkblue}{\textbf{\ipa{dʑi˧mi\#˥}}}}\hspace{0.5cm}[\kern2pt{\textcolor{darkblue}{\textbf{\ipa{dʑi˧mi˥}}}}\kern2pt]} \hypertarget{dz£i\string_Mmi\#\string_T1}{}
\markboth{\textcolor{darkblue}{\textbf{\ipa{dʑi˧mi\#˥}}}}{}
\textcolor{teal}{\zh{名词}} \hspace{4pt} \zh{声调类:} \#H.
\zh{母水牛。} \textcolor{Sepia}{\selectlanguage{english}Female water buffalo.} \textcolor{PineGreen}{\selectlanguage{french}Buffle (femelle).}  ¶ \textcolor{darkblue}{\textbf{\ipa{dʑi˧mi˧ tʰv̩˧-pʰo˩}}} \zh{这头母水牛} \textcolor{Sepia}{\selectlanguage{english}\mytextsc{n}+\mytextsc{dem}+\mytextsc{clf}} \textcolor{PineGreen}{\selectlanguage{french}\mytextsc{n}+\mytextsc{dem}+\mytextsc{clf}}  
 ¶ \textcolor{darkblue}{\textbf{\ipa{dʑi˧mi˧-dʑi˧zo\#˥ / dʑi˧mi˧-dʑi˥zo˩}}} \zh{母水牛与公水牛} \textcolor{Sepia}{\selectlanguage{english}female water buffalo and male water buffalo} \textcolor{PineGreen}{\selectlanguage{french}buffles femelle et mâle}  
 \zh{量词}: \textcolor{darkblue}{\textbf{\ipa{pʰo˧˥}}} 
\lhead{\firstmark}
\rhead{\botmark}

\subsection{\hspace{-0.5cm} {\Large \textcolor{darkblue}{\textbf{\ipa{dʑi˧zo\#˥}}}}\hspace{0.5cm}[\kern2pt{\textcolor{darkblue}{\textbf{\ipa{dʑi˩zo˩˥}}}}\kern2pt]} \hypertarget{dz£i\string_Mzo\#\string_T1}{}
\markboth{\textcolor{darkblue}{\textbf{\ipa{dʑi˧zo\#˥}}}}{}
\textcolor{teal}{\zh{名词}} \hspace{4pt} \zh{声调类:} \#H.
\zh{小水牛(水牛崽子),一般指公的小水牛。} \textcolor{Sepia}{\selectlanguage{english}Male baby buffalo.} \textcolor{PineGreen}{\selectlanguage{french}Buffle (enfant mâle).}  ¶ \textcolor{darkblue}{\textbf{\ipa{dʑi˧zo˧ tʰv̩˧-ɭɯ\#˥}}} \zh{这只水牛崽子} \textcolor{Sepia}{\selectlanguage{english}\mytextsc{n}+\mytextsc{dem}+\mytextsc{clf}} \textcolor{PineGreen}{\selectlanguage{french}\mytextsc{n}+\mytextsc{dem}+\mytextsc{clf}}  
 \zh{量词}: \textcolor{darkblue}{\textbf{\ipa{ɭɯ˧}}} 
\lhead{\firstmark}
\rhead{\botmark}

\subsection{\hspace{-0.5cm} {\Large \textcolor{darkblue}{\textbf{\ipa{dʑi˩wɤ˩}}}}\hspace{0.5cm}[\kern2pt{\textcolor{darkblue}{\textbf{\ipa{dʑi˩wɤ˩˥}}}}\kern2pt]} \hypertarget{dz£i\string_Bw7\string_B1}{}
\markboth{\textcolor{darkblue}{\textbf{\ipa{dʑi˩wɤ˩}}}}{}
\textcolor{teal}{\zh{名词}} \hspace{4pt} \zh{声调类:} L.
\zh{马镫。} \textcolor{Sepia}{\selectlanguage{english}Stirrup.} \textcolor{PineGreen}{\selectlanguage{french}Étriers.}  \zh{量词}: \textcolor{darkblue}{\textbf{\ipa{dze˩}}} 
\lhead{\firstmark}
\rhead{\botmark}

\subsection{\hspace{-0.5cm} {\Large \textcolor{darkblue}{\textbf{\ipa{‑dʑo˥}}}}\hspace{0.5cm}[\kern2pt{\textcolor{darkblue}{\textbf{\ipa{dʑo˥}}}}\kern2pt]} \hypertarget{‑dz£o\string_T1}{}
\markboth{\textcolor{darkblue}{\textbf{\ipa{‑dʑo˥}}}}{}
\textcolor{teal}{\zh{后缀}} \hspace{4pt} \zh{声调类:} H.
\zh{\mytextsc{主题。}} \textcolor{Sepia}{\selectlanguage{english}Topic marker.} \textcolor{PineGreen}{\selectlanguage{french}Marqueur de topic.} 
\lhead{\firstmark}
\rhead{\botmark}

\subsection{\hspace{-0.5cm} {\Large \textcolor{darkblue}{\textbf{\ipa{‑dʑo˧}}}}\hspace{0.5cm}[\kern2pt{\textcolor{darkblue}{\textbf{\ipa{dʑo˥}}}}\kern2pt]} \hypertarget{‑dz£o\string_M1}{}
\markboth{\textcolor{darkblue}{\textbf{\ipa{‑dʑo˧}}}}{}
\textcolor{teal}{\zh{后缀}} \hspace{4pt} \zh{声调类:} M.
\zh{\mytextsc{进行式。}} \textcolor{Sepia}{\selectlanguage{english}Progressive aspect.} \textcolor{PineGreen}{\selectlanguage{french}Aspect progressif.} 
\lhead{\firstmark}
\rhead{\botmark}

\subsection{\hspace{-0.5cm} {\Large \textcolor{darkblue}{\textbf{\ipa{dʑo˧\textsubscript{b}}}}}\hspace{0.5cm}[\kern2pt{\textcolor{darkblue}{\textbf{\ipa{dʑo˩˥}}}}\kern2pt]} \hypertarget{dz£o\string_Mb1}{}
\markboth{\textcolor{darkblue}{\textbf{\ipa{dʑo˧\textsubscript{b}}}}}{}
\textcolor{teal}{\zh{动词}} \hspace{4pt} \zh{声调类:} M\textsubscript{b}.
\zh{有,拥有。} \textcolor{Sepia}{\selectlanguage{english}To possess.} \textcolor{PineGreen}{\selectlanguage{french}Posséder; y avoir; avoir un objet, avoir une pensée…; autrefois, il y avait une mère et sa fille...}  ¶ \textcolor{darkblue}{\textbf{\ipa{mɤ˧-dʑo˧-ze˧!}}} \zh{没有了!} \textcolor{Sepia}{\selectlanguage{english}There isn't any left!} \textcolor{PineGreen}{\selectlanguage{french}Il n'y en a plus!}  
 ¶ \textcolor{darkblue}{\textbf{\ipa{le˧-dʑo˧-ze˧!}}} \zh{有了!} \textcolor{Sepia}{\selectlanguage{english}There is some, now!} \textcolor{PineGreen}{\selectlanguage{french}ça y est, il y en a!}  
 ¶ \textcolor{darkblue}{\textbf{\ipa{ʈʂʰɯ˧ | ɑ˩ʁo˧ | ɖɯ˧-sɑ˥ | mɤ˧-dʑo˧!}}} \zh{他家什么也没有 = 他家贫穷} \textcolor{Sepia}{\selectlanguage{english}At his home, they have nothing at all = he is needy} \textcolor{PineGreen}{\selectlanguage{french}Il n'y a rien chez lui = sa maison est dans l'indigence}  
 ¶ \textcolor{darkblue}{\textbf{\ipa{njɤ˧ | mv̩˩zɯ˩-ni˥mi˩ | ŋi˧-kv̩˧ dʑo˧˥!}}} \zh{我有两个兄弟姐妹!} \textcolor{Sepia}{\selectlanguage{english}I have two siblings!} \textcolor{PineGreen}{\selectlanguage{french}j’ai deux frères et sœurs!}  
 ¶ \textcolor{darkblue}{\textbf{\ipa{dʑo˧-tʰɑ˧˥!}}} \zh{会有的!} \textcolor{Sepia}{\selectlanguage{english}There could be some! / It's possible that there will be some!} \textcolor{PineGreen}{\selectlanguage{french}Cela arrive / il peut y en avoir / c'est possible qu'il y en ait!}  
 ¶ \textcolor{darkblue}{\textbf{\ipa{tso˧\textasciitilde{}tso˧ dʑo˧}}} \zh{他有东西} \textcolor{Sepia}{\selectlanguage{english}he has some things} \textcolor{PineGreen}{\selectlanguage{french}il y a des choses}  
 ¶ \textcolor{darkblue}{\textbf{\ipa{njɤ˧-ɻ̍˩, | ɖɯ˧-ɭɯ˧-lɑ˧ dʑo˥!}}} \zh{我们只有一个(孩子)!} \textcolor{Sepia}{\selectlanguage{english}We only have one (child)!} \textcolor{PineGreen}{\selectlanguage{french}Nous, on n'en a qu'un(, d'enfant)!}  
 ¶ \textcolor{darkblue}{\textbf{\ipa{ɖwæ˧˥ | dʑo˧-ɲi˥!}}} \zh{有很多!(如:准备建房,积累的木材有很多)} \textcolor{Sepia}{\selectlanguage{english}There are lots! (For instance, when preparing to build a house, one needs large quantities of lumber; someone may comment: “There are lots!”)} \textcolor{PineGreen}{\selectlanguage{french}Il y en a des quantités! (ex.: au sujet du bois de construction qu'on prépare en vue de la construction d'une maison)}  

\lhead{\firstmark}
\rhead{\botmark}

\subsection{\hspace{-0.5cm} {\Large \textcolor{darkblue}{\textbf{\ipa{dʑo˩\textsubscript{b}}}}}\hspace{0.5cm}[\kern2pt{\textcolor{darkblue}{\textbf{\ipa{dʑo˥}}}}\kern2pt]} \hypertarget{dz£o\string_Bb1}{}
\markboth{\textcolor{darkblue}{\textbf{\ipa{dʑo˩\textsubscript{b}}}}}{}
\textcolor{teal}{\zh{动词}} \hspace{4pt} \zh{声调类:} L\textsubscript{b}.
\zh{存在动词:有,存在着。如:某人在家或家里有几口人。} \textcolor{Sepia}{\selectlanguage{english}Existential: there is someone in the toilets/at home; there are n people in the family.} \textcolor{PineGreen}{\selectlanguage{french}Existentiel pour les êtres animés (dont les personnes).}  ¶ \textcolor{darkblue}{\textbf{\ipa{mɤ˧-dʑo˩}}} \zh{没有、不在} \textcolor{Sepia}{\selectlanguage{english}\mytextsc{neg}} \textcolor{PineGreen}{\selectlanguage{french}\mytextsc{neg}}  
 ¶ \textcolor{darkblue}{\textbf{\ipa{ʈʂʰɯ˧ | ɑ˩ʁo˧ mɤ˧-dʑo˩!}}} \zh{他不在家!} \textcolor{Sepia}{\selectlanguage{english}(S)he is not at home!} \textcolor{PineGreen}{\selectlanguage{french}Il/elle n'est pas à la maison!}  

\lhead{\firstmark}
\rhead{\botmark}

\subsection{\hspace{-0.5cm} {\Large \textcolor{darkblue}{\textbf{\ipa{‑dʑɯ˧}}}}\hspace{0.5cm}[\kern2pt{\textcolor{darkblue}{\textbf{\ipa{dʑɯ˥}}}}\kern2pt]} \hypertarget{‑dz£M\string_M1}{}
\markboth{\textcolor{darkblue}{\textbf{\ipa{‑dʑɯ˧}}}}{}
\textcolor{teal}{\zh{后缀}} \hspace{4pt} \zh{声调类:} M.
\zh{……过。} \textcolor{Sepia}{\selectlanguage{english}\mytextsc{experiential}.} \textcolor{PineGreen}{\selectlanguage{french}\mytextsc{expérientiel}.} 
\lhead{\firstmark}
\rhead{\botmark}

\subsection{\hspace{-0.5cm} {\Large \textcolor{darkblue}{\textbf{\ipa{dʑɯ˧}}}}\hspace{0.5cm}[\kern2pt{\textcolor{darkblue}{\textbf{\ipa{dʑɯ˥}}}}\kern2pt]} \hypertarget{dz£M\string_M1}{}
\markboth{\textcolor{darkblue}{\textbf{\ipa{dʑɯ˧}}}}{}
\textcolor{teal}{\zh{名词}} \hspace{4pt} \zh{声调类:} M.
\zh{……的时间。} \textcolor{Sepia}{\selectlanguage{english}Moment, time (of a certain event).} \textcolor{PineGreen}{\selectlanguage{french}Le moment (de), l'heure (de).}  ¶ \textcolor{darkblue}{\textbf{\ipa{ʈʂʰwɤ˩ dzɯ˩-bi˩-dʑɯ˩˥}}} \zh{吃晚餐的时间} \textcolor{Sepia}{\selectlanguage{english}dinner-time} \textcolor{PineGreen}{\selectlanguage{french}l'heure du repas du soir}  
 ¶ \textcolor{darkblue}{\textbf{\ipa{ɑ˩pʰo˩ bi˩-dʑɯ˩˥}}} \zh{出去的(合适)时间} \textcolor{Sepia}{\selectlanguage{english}time to go out; the right time to go outside} \textcolor{PineGreen}{\selectlanguage{french}l'heure d'aller dehors, le moment d'aller dehors}  
 ¶ \textcolor{darkblue}{\textbf{\ipa{le˧-ʑi˧-bi˧-dʑɯ˧ tʰv̩˧-ze˩!}}} \zh{睡觉的时间到了!} \textcolor{Sepia}{\selectlanguage{english}It is time to go to sleep!} \textcolor{PineGreen}{\selectlanguage{french}Il est l'heure d'aller dormir!}  
 ¶ \textcolor{darkblue}{\textbf{\ipa{ʐo˩ dzɯ˩-bi˩-dʑɯ˩˥}}} \zh{午饭的时间} \textcolor{Sepia}{\selectlanguage{english}lunch-time} \textcolor{PineGreen}{\selectlanguage{french}l'heure du déjeuner}  

\lhead{\firstmark}
\rhead{\botmark}

\subsection{\hspace{-0.5cm} {\Large \textcolor{darkblue}{\textbf{\ipa{dʑɯ˧dv̩˧}}}}\hspace{0.5cm}[\kern2pt{\textcolor{darkblue}{\textbf{\ipa{xxxx non-correspondance entre le nombre de morphèmes et le nombre de tons de morphèmes}}}}\kern2pt]} \hypertarget{dz£M\string_Mdv\string_=\string_M1}{}
\markboth{\textcolor{darkblue}{\textbf{\ipa{dʑɯ˧dv̩˧}}}}{}
\textcolor{teal}{\zh{名词}} \hspace{4pt} \zh{声调类:} M.
\zh{蚯蚓。} \textcolor{Sepia}{\selectlanguage{english}Earthworm.} \textcolor{PineGreen}{\selectlanguage{french}Ver de terre.}  ¶ \textcolor{darkblue}{\textbf{\ipa{dʑɯ˧dv̩˧-mi˩, | ə˩-dʑo˩˥?}}} \zh{有母蚯蚓吗?} \textcolor{Sepia}{\selectlanguage{english}Do female earthworms exist? (An artificially designed question, so as to elicit a form of 'earthworm' with the 'female' suffix, with a view to understanding the synchronically productive tone assignment rules for the gender suffixes.)} \textcolor{PineGreen}{\selectlanguage{french}Les vers de terre femelle, ça existe? (Cette phrase permet d'éliciter une forme associant 'ver de terre' au suffixe 'femelle', dans l'idée d'étudier les règles tonales productives en synchronie pour les suffixes de genre.)}  
 ¶ \textcolor{darkblue}{\textbf{\ipa{dʑɯ˧dv̩˧-pʰv̩˩, | ə˩-dʑo˩˥?}}} \zh{有公蚯蚓吗?} \textcolor{Sepia}{\selectlanguage{english}Do male earthworms exist? (An artificially designed question, so as to elicit a form of 'earthworm' with the 'male' suffix, with a view to understanding the synchronically productive tone assignment rules for the gender suffixes.)} \textcolor{PineGreen}{\selectlanguage{french}Les vers de terre mâle, ça existe? (Cette phrase permet d'éliciter une forme associant 'ver de terre' au suffixe 'mâle', dans l'idée d'étudier les règles tonales productives en synchronie pour les suffixes de genre.)}  
 \zh{量词}: \textcolor{darkblue}{\textbf{\ipa{kʰɯ˩}}} 
\lhead{\firstmark}
\rhead{\botmark}

\subsection{\hspace{-0.5cm} {\Large \textcolor{darkblue}{\textbf{\ipa{dʑɯ˧dze˧mi\#˥}}}}\hspace{0.5cm}[\kern2pt{\textcolor{darkblue}{\textbf{\ipa{dʑɯ˩dze˩mi˩˥}}}}\kern2pt]} \hypertarget{dz£M\string_Mdze\string_Mmi\#\string_T1}{}
\markboth{\textcolor{darkblue}{\textbf{\ipa{dʑɯ˧dze˧mi\#˥}}}}{}
\textcolor{teal}{\zh{名词}} \hspace{4pt} \zh{声调类:} \#H.
\zh{蝉。} \textcolor{Sepia}{\selectlanguage{english}Cicada.} \textcolor{PineGreen}{\selectlanguage{french}Cigale.}  ¶ \textcolor{darkblue}{\textbf{\ipa{dʑɯ˧dze˧-mi˧ tʰv̩˧-mi˧˥ / dʑɯ˧dze˧-mi˧ tʰv̩˧-mi˥\#}}} \zh{这只蝉} \textcolor{Sepia}{\selectlanguage{english}\mytextsc{n}+\mytextsc{dem}+\mytextsc{clf}} \textcolor{PineGreen}{\selectlanguage{french}\mytextsc{n}+\mytextsc{dem}+\mytextsc{clf}}  
 \zh{量词}: \textcolor{darkblue}{\textbf{\ipa{mi˩}}} 
\lhead{\firstmark}
\rhead{\botmark}

\subsection{\hspace{-0.5cm} {\Large \textcolor{darkblue}{\textbf{\ipa{dʑɯ˧ki˥}}}}\hspace{0.5cm}[\kern2pt{\textcolor{darkblue}{\textbf{\ipa{dʑɯ˩ki˩˥}}}}\kern2pt]} \hypertarget{dz£M\string_Mki\string_T1}{}
\markboth{\textcolor{darkblue}{\textbf{\ipa{dʑɯ˧ki˥}}}}{}
\textcolor{teal}{\zh{名词}} \hspace{4pt} \zh{声调类:} H\#.
\zh{布带子,用来背小孩的带子,腰带。} \textcolor{Sepia}{\selectlanguage{english}Girdle, waistband (a large piece of fabric worn at the waist; can be used to carry a child); belt.} \textcolor{PineGreen}{\selectlanguage{french}Gaine: large ceinture en tissu, qui peut servir à porter un enfant; aussi: ceinture (terme générique).}  \zh{量词}: \textcolor{darkblue}{\textbf{\ipa{kʰɯ˩}}} 
\lhead{\firstmark}
\rhead{\botmark}

\subsection{\hspace{-0.5cm} {\Large \textcolor{darkblue}{\textbf{\ipa{dʑɯ˧-li˧}}}}\hspace{0.5cm}[\kern2pt{\textcolor{darkblue}{\textbf{\ipa{xxxx non-correspondance entre le nombre de morphèmes et le nombre de tons de morphèmes}}}}\kern2pt]} \hypertarget{dz£M\string_M-li\string_M1}{}
\markboth{\textcolor{darkblue}{\textbf{\ipa{dʑɯ˧-li˧}}}}{}
\textcolor{teal}{\zh{动词}} \hspace{4pt} \zh{声调类:} M.
\zh{灌溉。} \textcolor{Sepia}{\selectlanguage{english}To irrigate.} \textcolor{PineGreen}{\selectlanguage{french}Irriguer.}  ¶ \textcolor{darkblue}{\textbf{\ipa{dʑɯ˧-li˧-ze˧}}} \zh{灌溉了} \textcolor{Sepia}{\selectlanguage{english}\mytextsc{pfv}} \textcolor{PineGreen}{\selectlanguage{french}\mytextsc{pfv}}  
 ¶ \textcolor{darkblue}{\textbf{\ipa{dʑɯ˧-mɤ˧-li˧-hĩ˧ lv̩˧}}} \zh{旱田:不灌溉的田} \textcolor{Sepia}{\selectlanguage{english}dry farmland, dry field: a field that is not irrigated} \textcolor{PineGreen}{\selectlanguage{french}champ sec/pluvial: “champ qu'on n'irrigue pas”}  

\lhead{\firstmark}
\rhead{\botmark}

\subsection{\hspace{-0.5cm} {\Large \textcolor{darkblue}{\textbf{\ipa{dʑɯ˧ɭɯ˧}}}}\hspace{0.5cm}[\kern2pt{\textcolor{darkblue}{\textbf{\ipa{dʑɯ˧ɭɯ˧}}}}\kern2pt]} \hypertarget{dz£M\string_Ml\string_RM\string_M1}{}
\markboth{\textcolor{darkblue}{\textbf{\ipa{dʑɯ˧ɭɯ˧}}}}{}
\textcolor{teal}{\zh{名词}} \hspace{4pt} \zh{声调类:} M.
\zh{黍,小米。} \textcolor{Sepia}{\selectlanguage{english}Broomcorn millet, \textit{Panicum miliaceum}.} \textcolor{PineGreen}{\selectlanguage{french}Millet, \textit{Panicum miliaceum}.}  ¶ \textcolor{darkblue}{\textbf{\ipa{dʑɯ˧ɭɯ˧-ho\#˥}}} \zh{小米粥} \textcolor{Sepia}{\selectlanguage{english}millet gruel} \textcolor{PineGreen}{\selectlanguage{french}gruau de millet}  
\zh{~【参考】~} \textcolor{darkblue}{\textbf{\ipa{dʑɯ˧njɤ˧, dʑɯ˧ʈʂʰwæ\#˥}}} 
\lhead{\firstmark}
\rhead{\botmark}

\subsection{\hspace{-0.5cm} {\Large \textcolor{darkblue}{\textbf{\ipa{dʑɯ˧mi˧}}}}\hspace{0.5cm}[\kern2pt{\textcolor{darkblue}{\textbf{\ipa{dʑɯ˧mi˧}}}}\kern2pt]} \hypertarget{dz£M\string_Mmi\string_M1}{}
\markboth{\textcolor{darkblue}{\textbf{\ipa{dʑɯ˧mi˧}}}}{}
\textcolor{teal}{\zh{名词}} \hspace{4pt} \zh{声调类:} M.
\zh{大河。} \textcolor{Sepia}{\selectlanguage{english}Large river.} \textcolor{PineGreen}{\selectlanguage{french}Grande rivière.}  \zh{量词}: \textcolor{darkblue}{\textbf{\ipa{kʰɯ˩}}} 
\lhead{\firstmark}
\rhead{\botmark}

\subsection{\hspace{-0.5cm} {\Large \textcolor{darkblue}{\textbf{\ipa{dʑɯ˧njɤ˧}}}}\hspace{0.5cm}[\kern2pt{\textcolor{darkblue}{\textbf{\ipa{dʑɯ˧njɤ˧}}}}\kern2pt]} \hypertarget{dz£M\string_Mnj7\string_M1}{}
\markboth{\textcolor{darkblue}{\textbf{\ipa{dʑɯ˧njɤ˧}}}}{}
\textcolor{teal}{\zh{名词}} \hspace{4pt} \zh{声调类:} M.
\zh{黍,小米。} \textcolor{Sepia}{\selectlanguage{english}Broomcorn millet, \textit{Panicum miliaceum}.} \textcolor{PineGreen}{\selectlanguage{french}Millet, \textit{Panicum miliaceum}.}  ¶ \textcolor{darkblue}{\textbf{\ipa{dʑɯ˧njɤ˧, | ʐɯ˧ tɕɤ˧˥!}}} \zh{小米,用来酿酒!} \textcolor{Sepia}{\selectlanguage{english}Millet is used to make wine!} \textcolor{PineGreen}{\selectlanguage{french}Le millet, on s'en sert pour faire du vin!}  
 ¶ \textcolor{darkblue}{\textbf{\ipa{dʑɯ˧njɤ˧-hɑ\#˥}}} \zh{小米饭} \textcolor{Sepia}{\selectlanguage{english}cooked millet} \textcolor{PineGreen}{\selectlanguage{french}millet cuit}  
\zh{~【参考】~} \textcolor{darkblue}{\textbf{\ipa{dʑɯ˧ɭɯ˧, dʑɯ˧ʈʂʰwæ\#˥}}} 
\lhead{\firstmark}
\rhead{\botmark}

\subsection{\hspace{-0.5cm} {\Large \textcolor{darkblue}{\textbf{\ipa{dʑɯ˧qʰɑ˧}}}}\hspace{0.5cm}[\kern2pt{\textcolor{darkblue}{\textbf{\ipa{dʑɯ˧qʰɑ˧}}}}\kern2pt]} \hypertarget{dz£M\string_Mq\string_hA\string_M1}{}
\markboth{\textcolor{darkblue}{\textbf{\ipa{dʑɯ˧qʰɑ˧}}}}{}
\textcolor{teal}{\zh{名词}} \hspace{4pt} \zh{声调类:} M.
\zh{夏枯草。} \textcolor{Sepia}{\selectlanguage{english}Selfheal (a plant used in Chinese medicine).} \textcolor{PineGreen}{\selectlanguage{french}Yyyy.}  ¶ \textcolor{darkblue}{\textbf{\ipa{dʑɯ˧qʰɑ˧-bæ˩bæ˩}}} \zh{夏枯草花} \textcolor{Sepia}{\selectlanguage{english}selfheal flowers} \textcolor{PineGreen}{\selectlanguage{french}fleurs de yyyy}  
 \zh{量词}: \textcolor{darkblue}{\textbf{\ipa{qɑ˩}}} 
\lhead{\firstmark}
\rhead{\botmark}

\subsection{\hspace{-0.5cm} {\Large \textcolor{darkblue}{\textbf{\ipa{dʑɯ˧qʰv̩˩}}}}\hspace{0.5cm}[\kern2pt{\textcolor{darkblue}{\textbf{\ipa{dʑɯ˧qʰv̩˩}}}}\kern2pt]} \hypertarget{dz£M\string_Mq\string_hv\string_=\string_B1}{}
\markboth{\textcolor{darkblue}{\textbf{\ipa{dʑɯ˧qʰv̩˩}}}}{}
\textcolor{teal}{\zh{名词}} \hspace{4pt} \zh{声调类:} L\#.
\zh{永宁的一种植物。} \textcolor{Sepia}{\selectlanguage{english}A wild plant of Yongning.} \textcolor{PineGreen}{\selectlanguage{french}Plante sauvage dont les graines forment de grosses boules de graines.}  ¶ \textcolor{darkblue}{\textbf{\ipa{dʑɯ˧qʰv̩˩-lv̩˩lv̩˩}}} \zh{这种植物的种子} \textcolor{Sepia}{\selectlanguage{english}the grains of this plant} \textcolor{PineGreen}{\selectlanguage{french}graines de la plante en question}  
 \zh{量词}: \textcolor{darkblue}{\textbf{\ipa{ɭɯ˧}}} 
\lhead{\firstmark}
\rhead{\botmark}

\subsection{\hspace{-0.5cm} {\Large \textcolor{darkblue}{\textbf{\ipa{dʑɯ˧qʰv̩˧}}}}\hspace{0.5cm}[\kern2pt{\textcolor{darkblue}{\textbf{\ipa{dʑɯ˧qʰv̩˧}}}}\kern2pt]} \hypertarget{dz£M\string_Mq\string_hv\string_=\string_M1}{}
\markboth{\textcolor{darkblue}{\textbf{\ipa{dʑɯ˧qʰv̩˧}}}}{}
\textcolor{teal}{\zh{名词}} \hspace{4pt} \zh{声调类:} M.
\zh{井、水井。} \textcolor{Sepia}{\selectlanguage{english}Well.} \textcolor{PineGreen}{\selectlanguage{french}Puits.}  ¶ \textcolor{darkblue}{\textbf{\ipa{ɑ˩ʁo˥ | dʑɯ˧qʰv̩˧ tʰi˧-di˩.}}} \zh{家里有水井。} \textcolor{Sepia}{\selectlanguage{english}There is a well at home / in the farm.} \textcolor{PineGreen}{\selectlanguage{french}il y a un puits à la maison/dans la ferme.}  
 \zh{量词}: \textcolor{darkblue}{\textbf{\ipa{ɭɯ˧}}} 
\lhead{\firstmark}
\rhead{\botmark}

\subsection{\hspace{-0.5cm} {\Large \textcolor{darkblue}{\textbf{\ipa{dʑɯ˧ʁo˩}}}}\hspace{0.5cm}[\kern2pt{\textcolor{darkblue}{\textbf{\ipa{dʑɯ˧ʁo˩}}}}\kern2pt]} \hypertarget{dz£M\string_MRo\string_B1}{}
\markboth{\textcolor{darkblue}{\textbf{\ipa{dʑɯ˧ʁo˩}}}}{}
\textcolor{teal}{\zh{名词}} \hspace{4pt} \zh{声调类:} L\#.
\zh{桃子。} \textcolor{Sepia}{\selectlanguage{english}Peach.} \textcolor{PineGreen}{\selectlanguage{french}Pêche.} 
\lhead{\firstmark}
\rhead{\botmark}

\subsection{\hspace{-0.5cm} {\Large \textcolor{darkblue}{\textbf{\ipa{dʑɯ˧ʈʂʰwæ\#˥}}}}\hspace{0.5cm}[\kern2pt{\textcolor{darkblue}{\textbf{\ipa{dʑɯ˧ʈʂʰwæ˧}}}}\kern2pt]} \hypertarget{dz£M\string_Mt`s`\string_hw\{\#\string_T1}{}
\markboth{\textcolor{darkblue}{\textbf{\ipa{dʑɯ˧ʈʂʰwæ\#˥}}}}{}
\textcolor{teal}{\zh{名词}} \hspace{4pt} \zh{声调类:} \#H.
\zh{已碾的小米。} \textcolor{Sepia}{\selectlanguage{english}Husked broomcorn millet, \textit{Panicum miliaceum}.} \textcolor{PineGreen}{\selectlanguage{french}Millet décortiqué, \textit{Panicum miliaceum}.} \zh{~【参考】~} \textcolor{darkblue}{\textbf{\ipa{dʑɯ˧ɭɯ˧, dʑɯ˧njɤ˧}}} 
\lhead{\firstmark}
\rhead{\botmark}

\subsection{\hspace{-0.5cm} {\Large \textcolor{darkblue}{\textbf{\ipa{dʑɯ˩}}}}\hspace{0.5cm}[\kern2pt{\textcolor{darkblue}{\textbf{\ipa{dʑɯ˥}}}}\kern2pt]} \hypertarget{dz£M\string_B1}{}
\markboth{\textcolor{darkblue}{\textbf{\ipa{dʑɯ˩}}}}{}
\textcolor{teal}{\zh{名词}} \hspace{4pt} \zh{声调类:} L.
\ding{202} \zh{水。} \textcolor{Sepia}{\selectlanguage{english}Water.} \textcolor{PineGreen}{\selectlanguage{french}Eau.}  ¶ \textcolor{darkblue}{\textbf{\ipa{dʑɯ˧ ʈʰɯ˧}}} \zh{喝水} \textcolor{Sepia}{\selectlanguage{english}to drink water} \textcolor{PineGreen}{\selectlanguage{french}boire de l'eau}  
 ¶ \textcolor{darkblue}{\textbf{\ipa{ʈʂʰɯ˧ dʑɯ˧ ʈʰɯ˧-dʑo˧!}}} \zh{他在喝水} \textcolor{Sepia}{\selectlanguage{english}(S)he is drinking water} \textcolor{PineGreen}{\selectlanguage{french}il est en train de boire de l'eau!}  
 ¶ \textcolor{darkblue}{\textbf{\ipa{dʑɯ˧ | ɖɯ˧-ʈʰɤ˧ ʈʰɯ˧˥}}} \zh{喝一点水(直译:‘一滴水’)} \textcolor{Sepia}{\selectlanguage{english}to drink a little water (literally 'a drop of water')} \textcolor{PineGreen}{\selectlanguage{french}boire un peu d'eau (littéralement “une goutte d'eau”)}  
 ¶ \textcolor{darkblue}{\textbf{\ipa{dʑɯ˩ kʰɯ˩˥}}} \zh{放水} \textcolor{Sepia}{\selectlanguage{english}to put water} \textcolor{PineGreen}{\selectlanguage{french}mettre de l'eau}  
 ¶ \textcolor{darkblue}{\textbf{\ipa{dʑɯ˩ mæ˩˥}}} \zh{浇灌、灌溉} \textcolor{Sepia}{\selectlanguage{english}to irrigate, to water} \textcolor{PineGreen}{\selectlanguage{french}irriguer, arroser, mettre de l’eau}  
 ¶ \textcolor{darkblue}{\textbf{\ipa{dʑɯ˩ qæ˩, | hɑ˩ qæ˩˥ |}}} \zh{‘换水换土’:这个短语描述旅人到他人乡的情况,带来水土不服的危险。为了预防这类不良反应,摩梭旅人习惯水煮一点当地的土,喝下去,为了适应当地的水土。} \textcolor{Sepia}{\selectlanguage{english}a description of the traveller's changes in environment: 'to change water, to change food'. This requires using strategies to avoid ailments: in particular, it was customary to boil in water a little earth of the place where one had arrived, and to drink this preparation.} \textcolor{PineGreen}{\selectlanguage{french}description du dépaysement que connaît le voyageur qui arrive en pays étranger et doit 'changer d'eau, changer de nourriture'. Ce dépaysement commande des stratégies de prévention de soucis de santé: en particulier, il était usuel de faire bouillir un peu de terre locale dans de l'eau, et de boire cette préparation de façon à s'accoutumer.}  
 ¶ \textcolor{darkblue}{\textbf{\ipa{[F5] dʑɯ˧ | mv̩˩tɕo˧ dɑ˧˥}}} \zh{水往下流} \textcolor{Sepia}{\selectlanguage{english}the water flows downwards} \textcolor{PineGreen}{\selectlanguage{french}l'eau coule vers le bas}  
 \zh{量词}: \textcolor{darkblue}{\textbf{\ipa{kʰɯ˩}}} \ding{203} \zh{河流。} \textcolor{Sepia}{\selectlanguage{english}River, waterway.} \textcolor{PineGreen}{\selectlanguage{french}Rivière.} 
\lhead{\firstmark}
\rhead{\botmark}

\subsection{\hspace{-0.5cm} {\Large \textcolor{darkblue}{\textbf{\ipa{dʑɯ˩\textsubscript{a}}}}}\hspace{0.5cm}[\kern2pt{\textcolor{darkblue}{\textbf{\ipa{dʑɯ˥}}}}\kern2pt]} \hypertarget{dz£M\string_Ba1}{}
\markboth{\textcolor{darkblue}{\textbf{\ipa{dʑɯ˩\textsubscript{a}}}}}{}
\textcolor{teal}{\zh{动词}} \hspace{4pt} \zh{声调类:} L\textsubscript{a}.
\zh{搓(搓绳子)。} \textcolor{Sepia}{\selectlanguage{english}To twist (strings) together (to make a rope).} \textcolor{PineGreen}{\selectlanguage{french}Rouler, tordre (par exemple: rouler des brins, pour en faire une corde; ne s'emploie pas pour les fibres très fines, pour lesquelles on dit: \textcolor{darkblue}{\textbf{\ipa{/ʈʂwæ˧˥/}}}).}  ¶ \textcolor{darkblue}{\textbf{\ipa{le˧-dʑɯ˩-ze˩}}} \zh{搓了} \textcolor{Sepia}{\selectlanguage{english}\mytextsc{accomp} \string_ \mytextsc{pfv}} \textcolor{PineGreen}{\selectlanguage{french}\mytextsc{accomp} \string_ \mytextsc{pfv}}  
 ¶ \textcolor{darkblue}{\textbf{\ipa{bæ˩ dʑɯ˩˥}}} \zh{搓绳子} \textcolor{Sepia}{\selectlanguage{english}to twist (strings into) a rope} \textcolor{PineGreen}{\selectlanguage{french}tordre une corde}  
 ¶ \textcolor{darkblue}{\textbf{\ipa{qʰv̩˩ɖʐæ˩ dʑɯ˥}}} \zh{搓一根小绳子} \textcolor{Sepia}{\selectlanguage{english}to twist a string, a small rope} \textcolor{PineGreen}{\selectlanguage{french}faire une ficelle, une petite cordelette}  
 ¶ \textcolor{darkblue}{\textbf{\ipa{ɖɯ˧-kʰwɤ˧ dʑɯ˥}}} \zh{搓一下} \textcolor{Sepia}{\selectlanguage{english}to twist a little} \textcolor{PineGreen}{\selectlanguage{french}tordre un peu / tordre quelque chose}  

\lhead{\firstmark}
\rhead{\botmark}

\subsection{\hspace{-0.5cm} {\Large \textcolor{darkblue}{\textbf{\ipa{dʑɯ˩-æ̃˩tsɯ˧}}}}\hspace{0.5cm}[\kern2pt{\textcolor{darkblue}{\textbf{\ipa{xxxx non-correspondance entre le nombre de morphèmes et le nombre de tons de morphèmes}}}}\kern2pt]} \hypertarget{dz£M\string_B-\{\string_~\string_BtsM\string_M1}{}
\markboth{\textcolor{darkblue}{\textbf{\ipa{dʑɯ˩-æ̃˩tsɯ˧}}}}{}
\textcolor{teal}{\zh{名词}} \hspace{4pt} \zh{声调类:} L-LM.
\zh{水禽,包括几种不同的小鸟,如:鹬。} \textcolor{Sepia}{\selectlanguage{english}Water fowl: used as a cover term for a variety of birds including sandpiper (\textit{Calidris}), avocet, Baillon's crake, and blue-breasted banded rail.} \textcolor{PineGreen}{\selectlanguage{french}Gibier d'eau, sauvagine; employé pour divers oiseaux tels que: bécasseau, chevalier (\textit{Calidris}), avocette, marouette, et râle.}  \zh{量词}: \textcolor{darkblue}{\textbf{\ipa{ɭɯ˧}}} 
\lhead{\firstmark}
\rhead{\botmark}

\subsection{\hspace{-0.5cm} {\Large \textcolor{darkblue}{\textbf{\ipa{dʑɯ˩dze˩}}}}\hspace{0.5cm}[\kern2pt{\textcolor{darkblue}{\textbf{\ipa{dʑɯ˧dze˧}}}}\kern2pt]} \hypertarget{dz£M\string_Bdze\string_B1}{}
\markboth{\textcolor{darkblue}{\textbf{\ipa{dʑɯ˩dze˩}}}}{}
\textcolor{teal}{\zh{名词}} \hspace{4pt} \zh{声调类:} L.
\zh{舀汤的勺子。} \textcolor{Sepia}{\selectlanguage{english}Ladle used for people's food.} \textcolor{PineGreen}{\selectlanguage{french}Louche utilisée pour faire la cuisine, distribuer la soupe.}  \zh{量词}: \textcolor{darkblue}{\textbf{\ipa{nɑ˧}}} 
\lhead{\firstmark}
\rhead{\botmark}

\subsection{\hspace{-0.5cm} {\Large \textcolor{darkblue}{\textbf{\ipa{dʑɯ˩gɤ˩di˩}}}}\hspace{0.5cm}[\kern2pt{\textcolor{darkblue}{\textbf{\ipa{dʑɯ˧gɤ˧di˧}}}}\kern2pt]} \hypertarget{dz£M\string_Bg7\string_Bdi\string_B1}{}
\markboth{\textcolor{darkblue}{\textbf{\ipa{dʑɯ˩gɤ˩di˩}}}}{}
\textcolor{teal}{\zh{名词}} \hspace{4pt} \zh{声调类:} L.
\zh{扁担。} \textcolor{Sepia}{\selectlanguage{english}Carrying/shoulder pole.} \textcolor{PineGreen}{\selectlanguage{french}Palanche.}  \zh{量词}: \textcolor{darkblue}{\textbf{\ipa{nɑ˧}}} 
\lhead{\firstmark}
\rhead{\botmark}

\subsection{\hspace{-0.5cm} {\Large \textcolor{darkblue}{\textbf{\ipa{dʑɯ˩gv̩˩}}}}\hspace{0.5cm}[\kern2pt{\textcolor{darkblue}{\textbf{\ipa{dʑɯ˩gv̩˩˥}}}}\kern2pt]} \hypertarget{dz£M\string_Bgv\string_=\string_B1}{}
\markboth{\textcolor{darkblue}{\textbf{\ipa{dʑɯ˩gv̩˩}}}}{}
\textcolor{teal}{\zh{名词}} \hspace{4pt} \zh{声调类:} L.
\zh{大水桶,水槽。} \textcolor{Sepia}{\selectlanguage{english}Large barrel where drinking water is kept; water trough.} \textcolor{PineGreen}{\selectlanguage{french}Cuve où l'on conserve l'eau potable, tonneau d'eau. A la date de l'enquête, il s'agissait d'un baril en fer.}  ¶ \textcolor{darkblue}{\textbf{\ipa{[F5] pv̩˩-dʑɯ˩gv̩˩˥}}} \textcolor{PineGreen}{\selectlanguage{french}même sens}  
 \zh{量词}: \textcolor{darkblue}{\textbf{\ipa{ɭɯ˧}}} 
\lhead{\firstmark}
\rhead{\botmark}

\subsection{\hspace{-0.5cm} {\Large \textcolor{darkblue}{\textbf{\ipa{dʑɯ˩gv̩˥}}}}\hspace{0.5cm}[\kern2pt{\textcolor{darkblue}{\textbf{\ipa{dʑɯ˩gv̩˩˥}}}}\kern2pt]} \hypertarget{dz£M\string_Bgv\string_=\string_T1}{}
\markboth{\textcolor{darkblue}{\textbf{\ipa{dʑɯ˩gv̩˥}}}}{}
\textcolor{teal}{\zh{形容词}} \hspace{4pt} \zh{声调类:} LH.
\zh{驼背。} \textcolor{Sepia}{\selectlanguage{english}Round-shouldered, stooping.} \textcolor{PineGreen}{\selectlanguage{french}Voûté, qui a le dos rond, courbé.} 
\lhead{\firstmark}
\rhead{\botmark}

\subsection{\hspace{-0.5cm} {\Large \textcolor{darkblue}{\textbf{\ipa{dʑɯ˩hṽ˧˥}}}}\hspace{0.5cm}[\kern2pt{\textcolor{darkblue}{\textbf{\ipa{dʑɯ˩hṽ˥}}}}\kern2pt]} \hypertarget{dz£M\string_Bhv\string_~\string_M\string_T1}{}
\markboth{\textcolor{darkblue}{\textbf{\ipa{dʑɯ˩hṽ˧˥}}}}{}
\textcolor{teal}{\zh{名词}} \hspace{4pt} \zh{声调类:} LM+MH\#.
\zh{面和水和成的浆糊。} \textcolor{Sepia}{\selectlanguage{english}Dough made of flour and water.} \textcolor{PineGreen}{\selectlanguage{french}Mélange d'eau et de farine: par ex. de la pâte à pain, du tsamba avec de l'eau….} 
\lhead{\firstmark}
\rhead{\botmark}

\subsection{\hspace{-0.5cm} {\Large \textcolor{darkblue}{\textbf{\ipa{dʑɯ˩-hwæ˩tsɯ˥}}}}\hspace{0.5cm}[\kern2pt{\textcolor{darkblue}{\textbf{\ipa{xxxx non-correspondance entre le nombre de morphèmes et le nombre de tons de morphèmes}}}}\kern2pt]} \hypertarget{dz£M\string_B-hw\{\string_BtsM\string_T1}{}
\markboth{\textcolor{darkblue}{\textbf{\ipa{dʑɯ˩-hwæ˩tsɯ˥}}}}{}
\textcolor{teal}{\zh{名词}} \hspace{4pt} \zh{声调类:} L+H\#.
\zh{尖鼠、鼩鼱。} \textcolor{Sepia}{\selectlanguage{english}Shrew: the consultant uses a periphrasis: “wild mouse”.} \textcolor{PineGreen}{\selectlanguage{french}Musaraigne; la locutrice emploie une périphrase: “souris sauvage”.} 
\lhead{\firstmark}
\rhead{\botmark}

\subsection{\hspace{-0.5cm} {\Large \textcolor{darkblue}{\textbf{\ipa{dʑɯ˩kʰi˩}}}}\hspace{0.5cm}[\kern2pt{\textcolor{darkblue}{\textbf{\ipa{dʑɯ˩kʰi˥}}}}\kern2pt]} \hypertarget{dz£M\string_Bk\string_hi\string_B1}{}
\markboth{\textcolor{darkblue}{\textbf{\ipa{dʑɯ˩kʰi˩}}}}{}
\textcolor{teal}{\zh{名词}} \hspace{4pt} \zh{声调类:} L.
\zh{水边。} \textcolor{Sepia}{\selectlanguage{english}Water's edge.} \textcolor{PineGreen}{\selectlanguage{french}Bord de l'eau.} 
\lhead{\firstmark}
\rhead{\botmark}

\subsection{\hspace{-0.5cm} {\Large \textcolor{darkblue}{\textbf{\ipa{dʑɯ˩kʰv̩˩}}}}\hspace{0.5cm}[\kern2pt{\textcolor{darkblue}{\textbf{\ipa{dʑɯ˩kʰv̩˩˥}}}}\kern2pt]} \hypertarget{dz£M\string_Bk\string_hv\string_=\string_B1}{}
\markboth{\textcolor{darkblue}{\textbf{\ipa{dʑɯ˩kʰv̩˩}}}}{}
\textcolor{teal}{\zh{名词}} \hspace{4pt} \zh{声调类:} L.
\zh{青苔。} \textcolor{Sepia}{\selectlanguage{english}Moss.} \textcolor{PineGreen}{\selectlanguage{french}Mousse.} 
\lhead{\firstmark}
\rhead{\botmark}

\subsection{\hspace{-0.5cm} {\Large \textcolor{darkblue}{\textbf{\ipa{dʑɯ˩nɑ˩hæ̃˩tʰɑ˩}}}}\hspace{0.5cm}[\kern2pt{\textcolor{darkblue}{\textbf{\ipa{dʑɯ˩nɑ˩hæ̃˩tʰɑ˩˥}}}}\kern2pt]} \hypertarget{dz£M\string_BnA\string_Bh\{\string_~\string_Bt\string_hA\string_B1}{}
\markboth{\textcolor{darkblue}{\textbf{\ipa{dʑɯ˩nɑ˩hæ̃˩tʰɑ˩}}}}{}
\textcolor{teal}{\zh{名词}} \hspace{4pt} \zh{声调类:} L.
\zh{水磨。} \textcolor{Sepia}{\selectlanguage{english}Water-mill.} \textcolor{PineGreen}{\selectlanguage{french}Moulin à eau.}  \zh{量词}: \textcolor{darkblue}{\textbf{\ipa{pɤ˩}}} 
\lhead{\firstmark}
\rhead{\botmark}

\subsection{\hspace{-0.5cm} {\Large \textcolor{darkblue}{\textbf{\ipa{dʑɯ˩nɑ˩mi˩}}}}\hspace{0.5cm}[\kern2pt{\textcolor{darkblue}{\textbf{\ipa{dʑɯ˩nɑ˩mi˩˥}}}}\kern2pt]} \hypertarget{dz£M\string_BnA\string_Bmi\string_B1}{}
\markboth{\textcolor{darkblue}{\textbf{\ipa{dʑɯ˩nɑ˩mi˩}}}}{}
\textcolor{teal}{\zh{名词}} \hspace{4pt} \zh{声调类:} L.
\zh{深山老林、高山上的地方。} \textcolor{Sepia}{\selectlanguage{english}Mountain areas (uncultivated), mountain forest, wilderness.} \textcolor{PineGreen}{\selectlanguage{french}Forêt d'altitude, régions sauvages en altitude.} \zh{~【参考】~} \textcolor{darkblue}{\textbf{\ipa{dʑɯ˩nɑ˩mi˩-ʁo˩, dʑɯ˩ʁo˩}}} 
\lhead{\firstmark}
\rhead{\botmark}

\subsection{\hspace{-0.5cm} {\Large \textcolor{darkblue}{\textbf{\ipa{dʑɯ˩nɑ˩mi˩-ʁo˩}}}}\hspace{0.5cm}[\kern2pt{\textcolor{darkblue}{\textbf{\ipa{xxxx non-correspondance entre le nombre de morphèmes et le nombre de tons de morphèmes}}}}\kern2pt]} \hypertarget{dz£M\string_BnA\string_Bmi\string_B-Ro\string_B1}{}
\markboth{\textcolor{darkblue}{\textbf{\ipa{dʑɯ˩nɑ˩mi˩-ʁo˩}}}}{}
\textcolor{teal}{\zh{名词}} \hspace{4pt} \zh{声调类:} L.
\zh{深山老林、高山上的地方。} \textcolor{Sepia}{\selectlanguage{english}Mountain areas (uncultivated), mountain forest, wilderness.} \textcolor{PineGreen}{\selectlanguage{french}Forêt d'altitude, régions sauvages en altitude.} \zh{~【参考】~} \textcolor{darkblue}{\textbf{\ipa{dʑɯ˩nɑ˩mi˩, dʑɯ˩ʁo˩}}} 
\lhead{\firstmark}
\rhead{\botmark}

\subsection{\hspace{-0.5cm} {\Large \textcolor{darkblue}{\textbf{\ipa{dʑɯ˩pɤ˩-kv̩˧hĩ˩}}}}\hspace{0.5cm}[\kern2pt{\textcolor{darkblue}{\textbf{\ipa{dʑɯ˩pɤ˩kv̩˧hĩ˩}}}}\kern2pt]} \hypertarget{dz£M\string_Bp7\string_B-kv\string_=\string_Mhi\string_~\string_B1}{}
\markboth{\textcolor{darkblue}{\textbf{\ipa{dʑɯ˩pɤ˩-kv̩˧hĩ˩}}}}{}
\textcolor{teal}{\zh{名词}} \hspace{4pt} \zh{声调类:} L-L\#.
\zh{水泉、山泉。} \textcolor{Sepia}{\selectlanguage{english}Spring.} \textcolor{PineGreen}{\selectlanguage{french}Source.}  ¶ \textcolor{darkblue}{\textbf{\ipa{dʑɯ˩pɤ˩-kv̩˧hĩ˩ | tʰi˧-di˩}}} \zh{有水泉} \textcolor{Sepia}{\selectlanguage{english}there is a spring} \textcolor{PineGreen}{\selectlanguage{french}il y a une source}  
\zh{~【参考】~} \textcolor{darkblue}{\textbf{\ipa{dʑɯ˩pɤ˩qʰv̩˩, dʑɯ˩pɤ˩tv̩˩qʰv̩˥}}} 
\lhead{\firstmark}
\rhead{\botmark}

\subsection{\hspace{-0.5cm} {\Large \textcolor{darkblue}{\textbf{\ipa{dʑɯ˩pɤ˩qʰv̩˩}}}}\hspace{0.5cm}[\kern2pt{\textcolor{darkblue}{\textbf{\ipa{dʑɯ˩pɤ˩qʰv̩˩˥}}}}\kern2pt]} \hypertarget{dz£M\string_Bp7\string_Bq\string_hv\string_=\string_B1}{}
\markboth{\textcolor{darkblue}{\textbf{\ipa{dʑɯ˩pɤ˩qʰv̩˩}}}}{}
\textcolor{teal}{\zh{名词}} \hspace{4pt} \zh{声调类:} L.
\zh{水泉、山泉。} \textcolor{Sepia}{\selectlanguage{english}Spring.} \textcolor{PineGreen}{\selectlanguage{french}Source.} \zh{~【参考】~} \textcolor{darkblue}{\textbf{\ipa{dʑɯ˩pɤ˩-kv̩˧hĩ˩, dʑɯ˩pɤ˩tv̩˩qʰv̩˥}}} 
\lhead{\firstmark}
\rhead{\botmark}

\subsection{\hspace{-0.5cm} {\Large \textcolor{darkblue}{\textbf{\ipa{dʑɯ˩pɤ˩tv̩˩qʰv̩˥}}}}\hspace{0.5cm}[\kern2pt{\textcolor{darkblue}{\textbf{\ipa{dʑɯ˩pɤ˩tv̩˩qʰv̩˥}}}}\kern2pt]} \hypertarget{dz£M\string_Bp7\string_Btv\string_=\string_Bq\string_hv\string_=\string_T1}{}
\markboth{\textcolor{darkblue}{\textbf{\ipa{dʑɯ˩pɤ˩tv̩˩qʰv̩˥}}}}{}
\textcolor{teal}{\zh{名词}} \hspace{4pt} \zh{声调类:} L+H\#.
\zh{水泉、山泉。} \textcolor{Sepia}{\selectlanguage{english}Spring.} \textcolor{PineGreen}{\selectlanguage{french}Source.} \zh{~【参考】~} \textcolor{darkblue}{\textbf{\ipa{dʑɯ˩pɤ˩qʰv̩˩, dʑɯ˩pɤ˩-kv̩˧hĩ˩}}} 
\lhead{\firstmark}
\rhead{\botmark}

\subsection{\hspace{-0.5cm} {\Large \textcolor{darkblue}{\textbf{\ipa{dʑɯ˩pʰæ˩}}}}\hspace{0.5cm}[\kern2pt{\textcolor{darkblue}{\textbf{\ipa{dʑɯ˩pʰæ˩˥}}}}\kern2pt]} \hypertarget{dz£M\string_Bp\string_h\{\string_B1}{}
\markboth{\textcolor{darkblue}{\textbf{\ipa{dʑɯ˩pʰæ˩}}}}{}
\textcolor{teal}{\zh{名词}} \hspace{4pt} \zh{声调类:} L.
\zh{冰。} \textcolor{Sepia}{\selectlanguage{english}Ice.} \textcolor{PineGreen}{\selectlanguage{french}Glace.}  \zh{量词}: \textcolor{darkblue}{\textbf{\ipa{pʰæ˧˥}}} 
\lhead{\firstmark}
\rhead{\botmark}

\subsection{\hspace{-0.5cm} {\Large \textcolor{darkblue}{\textbf{\ipa{dʑɯ˩qʰæ˩}}}}\hspace{0.5cm}[\kern2pt{\textcolor{darkblue}{\textbf{\ipa{dʑɯ˩qʰæ˩˥}}}}\kern2pt]} \hypertarget{dz£M\string_Bq\string_h\{\string_B1}{}
\markboth{\textcolor{darkblue}{\textbf{\ipa{dʑɯ˩qʰæ˩}}}}{}
\textcolor{teal}{\zh{名词}} \hspace{4pt} \zh{声调类:} L.
\zh{凉水。} \textcolor{Sepia}{\selectlanguage{english}Cold water.} \textcolor{PineGreen}{\selectlanguage{french}Eau froide.} 
\lhead{\firstmark}
\rhead{\botmark}

\subsection{\hspace{-0.5cm} {\Large \textcolor{darkblue}{\textbf{\ipa{dʑɯ˩qʰwɤ˩-lv̩˩}}}}\hspace{0.5cm}[\kern2pt{\textcolor{darkblue}{\textbf{\ipa{xxxx non-correspondance entre le nombre de morphèmes et le nombre de tons de morphèmes}}}}\kern2pt]} \hypertarget{dz£M\string_Bq\string_hw7\string_B-lv\string_=\string_B1}{}
\markboth{\textcolor{darkblue}{\textbf{\ipa{dʑɯ˩qʰwɤ˩-lv̩˩}}}}{}
\textcolor{teal}{\zh{名词}} \hspace{4pt} \zh{声调类:} L.
\zh{沼泽、湿地。} \textcolor{Sepia}{\selectlanguage{english}Marsh, bog, swamp (unsuitable for agriculture).} \textcolor{PineGreen}{\selectlanguage{french}Marais (terre impropre à la culture).} \zh{当地汉语方言:}\zh{潮地。} \zh{量词}: \textcolor{darkblue}{\textbf{\ipa{kɤ˧˥}}} 
\lhead{\firstmark}
\rhead{\botmark}

\subsection{\hspace{-0.5cm} {\Large \textcolor{darkblue}{\textbf{\ipa{dʑɯ˩ʁo˩}}}}\hspace{0.5cm}[\kern2pt{\textcolor{darkblue}{\textbf{\ipa{dʑɯ˩ʁo˩˥}}}}\kern2pt]} \hypertarget{dz£M\string_BRo\string_B1}{}
\markboth{\textcolor{darkblue}{\textbf{\ipa{dʑɯ˩ʁo˩}}}}{}
\textcolor{teal}{\zh{名词}} \hspace{4pt} \zh{声调类:} L.
\zh{深山老林、高山上的地方。} \textcolor{Sepia}{\selectlanguage{english}Mountain areas (uncultivated), mountain forest, wilderness.} \textcolor{PineGreen}{\selectlanguage{french}Forêt d'altitude, régions sauvages en altitude.} \zh{~【参考】~} \textcolor{darkblue}{\textbf{\ipa{dʑɯ˩nɑ˩mi˩, dʑɯ˩nɑ˩mi˩-ʁo˩}}} 
\lhead{\firstmark}
\rhead{\botmark}

\subsection{\hspace{-0.5cm} {\Large \textcolor{darkblue}{\textbf{\ipa{dʑɯ˩ʁo˩-æ̃˧}}}}\hspace{0.5cm}[\kern2pt{\textcolor{darkblue}{\textbf{\ipa{dʑɯ˩ʁo˩æ̃˥}}}}\kern2pt]} \hypertarget{dz£M\string_BRo\string_B-\{\string_~\string_M1}{}
\markboth{\textcolor{darkblue}{\textbf{\ipa{dʑɯ˩ʁo˩-æ̃˧}}}}{}
\textcolor{teal}{\zh{名词}} \hspace{4pt} \zh{声调类:} L-M.
\zh{鹌鹑。} \textcolor{Sepia}{\selectlanguage{english}Quail, rail, \textit{Coturnix}; used when identifying pictures of various species of \textit{Turnix}, \textit{Coturnix}, and \textit{Crex}.} \textcolor{PineGreen}{\selectlanguage{french}Caille, \textit{Coturnix}; terme employé pour divers oiseaux, dont des râles (\textit{Crex}).}  \zh{量词}: \textcolor{darkblue}{\textbf{\ipa{mi˩}}} 
\lhead{\firstmark}
\rhead{\botmark}

\subsection{\hspace{-0.5cm} {\Large \textcolor{darkblue}{\textbf{\ipa{dʑɯ˩ʁo˩-bo˧}}}}\hspace{0.5cm}[\kern2pt{\textcolor{darkblue}{\textbf{\ipa{dʑɯ˩ʁo˩bo˥}}}}\kern2pt]} \hypertarget{dz£M\string_BRo\string_B-bo\string_M1}{}
\markboth{\textcolor{darkblue}{\textbf{\ipa{dʑɯ˩ʁo˩-bo˧}}}}{}
\textcolor{teal}{\zh{名词}} \hspace{4pt} \zh{声调类:} L-M.
\zh{野猪。} \textcolor{Sepia}{\selectlanguage{english}Wild boar.} \textcolor{PineGreen}{\selectlanguage{french}Sanglier, porc sauvage.}  \zh{量词}: \textcolor{darkblue}{\textbf{\ipa{mi˩}}} \zh{~【同义词】~} \hyperlink{}{\textcolor{darkblue}{\textbf{\ipa{bo˩tv̩\#˥}}}}. 
\lhead{\firstmark}
\rhead{\botmark}

\subsection{\hspace{-0.5cm} {\Large \textcolor{darkblue}{\textbf{\ipa{dʑɯ˩ʁo˩-dze˧}}}}\hspace{0.5cm}[\kern2pt{\textcolor{darkblue}{\textbf{\ipa{dʑɯ˩ʁo˩dze˥}}}}\kern2pt]} \hypertarget{dz£M\string_BRo\string_B-dze\string_M1}{}
\markboth{\textcolor{darkblue}{\textbf{\ipa{dʑɯ˩ʁo˩-dze˧}}}}{}
\textcolor{teal}{\zh{名词}} \hspace{4pt} \zh{声调类:} L-M.
\zh{野花椒。} \textcolor{Sepia}{\selectlanguage{english}Wild pepper.} \textcolor{PineGreen}{\selectlanguage{french}Xanthoxyle sauvage.} 
\lhead{\firstmark}
\rhead{\botmark}

\subsection{\hspace{-0.5cm} {\Large \textcolor{darkblue}{\textbf{\ipa{dʑɯ˩ʁo˩-hwɤ˩li˧}}}}\hspace{0.5cm}[\kern2pt{\textcolor{darkblue}{\textbf{\ipa{dʑɯ˩ʁo˩hwɤ˩li˥}}}}\kern2pt]} \hypertarget{dz£M\string_BRo\string_B-hw7\string_Bli\string_M1}{}
\markboth{\textcolor{darkblue}{\textbf{\ipa{dʑɯ˩ʁo˩-hwɤ˩li˧}}}}{}
\textcolor{teal}{\zh{名词}} \hspace{4pt} \zh{声调类:} L-LM.
\zh{野猫。} \textcolor{Sepia}{\selectlanguage{english}Yunnan wild cat, \textit{Felis temincki}.} \textcolor{PineGreen}{\selectlanguage{french}Chat sauvage, \textit{Felis temincki}.}  \zh{量词}: \textcolor{darkblue}{\textbf{\ipa{mi˩}}} 
\lhead{\firstmark}
\rhead{\botmark}

\subsection{\hspace{-0.5cm} {\Large \textcolor{darkblue}{\textbf{\ipa{dʑɯ˩ʁo˩-ɬi˩bi˧}}}}\hspace{0.5cm}[\kern2pt{\textcolor{darkblue}{\textbf{\ipa{dʑɯ˩ʁo˩ɬi˩bi˥}}}}\kern2pt]} \hypertarget{dz£M\string_BRo\string_B-Ki\string_Bbi\string_M1}{}
\markboth{\textcolor{darkblue}{\textbf{\ipa{dʑɯ˩ʁo˩-ɬi˩bi˧}}}}{}
\textcolor{teal}{\zh{名词}} \hspace{4pt} \zh{声调类:} L-LM.
\zh{山药。} \textcolor{Sepia}{\selectlanguage{english}Chinese yam (shan-yao).} \textcolor{PineGreen}{\selectlanguage{french}Igname de Chine (shan-yao).}  \zh{量词}: \textcolor{darkblue}{\textbf{\ipa{ɭɯ˧}}} 
\lhead{\firstmark}
\rhead{\botmark}

\subsection{\hspace{-0.5cm} {\Large \textcolor{darkblue}{\textbf{\ipa{dʑɯ˩ʁo˩-zɯ˩}}}}\hspace{0.5cm}[\kern2pt{\textcolor{darkblue}{\textbf{\ipa{xxxx non-correspondance entre le nombre de morphèmes et le nombre de tons de morphèmes}}}}\kern2pt]} \hypertarget{dz£M\string_BRo\string_B-zM\string_B1}{}
\markboth{\textcolor{darkblue}{\textbf{\ipa{dʑɯ˩ʁo˩-zɯ˩}}}}{}
\textcolor{teal}{\zh{名词}} \hspace{4pt} \zh{声调类:} L.
\zh{野草。} \textcolor{Sepia}{\selectlanguage{english}Wild herbs.} \textcolor{PineGreen}{\selectlanguage{french}Herbes de la montagne, herbes sauvages, foin poussant sur l'alpage.}  ¶ \textcolor{darkblue}{\textbf{\ipa{ʈʂʰɯ˧ | dʑɯ˩ʁo˩-zɯ˩ ɲi˥.}}} \zh{\mytextsc{指示代词} \string_ \mytextsc{系词}} \textcolor{Sepia}{\selectlanguage{english}\mytextsc{dem} \string_ \mytextsc{cop}} \textcolor{PineGreen}{\selectlanguage{french}\mytextsc{dem} \string_ \mytextsc{cop}}  
 \zh{量词}: \textcolor{darkblue}{\textbf{\ipa{qɑ˩}}} \textcolor{darkblue}{\textbf{\ipa{po˧}}} 
\lhead{\firstmark}
\rhead{\botmark}

\subsection{\hspace{-0.5cm} {\Large \textcolor{darkblue}{\textbf{\ipa{dʑɯ˩si˩}}}}\hspace{0.5cm}[\kern2pt{\textcolor{darkblue}{\textbf{\ipa{dʑɯ˩si˩˥}}}}\kern2pt]} \hypertarget{dz£M\string_Bsi\string_B1}{}
\markboth{\textcolor{darkblue}{\textbf{\ipa{dʑɯ˩si˩}}}}{}
\textcolor{teal}{\zh{名词}} \hspace{4pt} \zh{声调类:} L.
\zh{青冈树、槲栎。} \textcolor{Sepia}{\selectlanguage{english}Oriental white oak.} \textcolor{PineGreen}{\selectlanguage{french}Chêne blanc oriental.} \zh{~【同义词】~} \hyperlink{}{\textcolor{darkblue}{\textbf{\ipa{dzi˧dzi˧}}}}. 
\lhead{\firstmark}
\rhead{\botmark}

\subsection{\hspace{-0.5cm} {\Large \textcolor{darkblue}{\textbf{\ipa{dʑɯ˩so˩}}}}\hspace{0.5cm}[\kern2pt{\textcolor{darkblue}{\textbf{\ipa{dʑɯ˩so˩˥}}}}\kern2pt]} \hypertarget{dz£M\string_Bso\string_B1}{}
\markboth{\textcolor{darkblue}{\textbf{\ipa{dʑɯ˩so˩}}}}{}
\textcolor{teal}{\zh{名词}} \hspace{4pt} \zh{声调类:} L.
\zh{波浪。} \textcolor{Sepia}{\selectlanguage{english}Wave.} \textcolor{PineGreen}{\selectlanguage{french}Vague.}  ¶ \textcolor{darkblue}{\textbf{\ipa{dʑɯ˩so˩ pʰv̩˩˥}}} \zh{有波浪} \textcolor{Sepia}{\selectlanguage{english}there is a wave, a wave breaks} \textcolor{PineGreen}{\selectlanguage{french}il y a une vague, une vague déferle}  
 \zh{量词}: \textcolor{darkblue}{\textbf{\ipa{pʰæ˧˥}}} 
\lhead{\firstmark}
\rhead{\botmark}

\subsection{\hspace{-0.5cm} {\Large \textcolor{darkblue}{\textbf{\ipa{dʑɯ˩ʂo˥}}}}\hspace{0.5cm}[\kern2pt{\textcolor{darkblue}{\textbf{\ipa{dʑɯ˩ʂo˥}}}}\kern2pt]} \hypertarget{dz£M\string_Bs`o\string_T1}{}
\markboth{\textcolor{darkblue}{\textbf{\ipa{dʑɯ˩ʂo˥}}}}{}
\textcolor{teal}{\zh{名词}} \hspace{4pt} \zh{声调类:} L+H\#.
\zh{一项仪式。} \textcolor{Sepia}{\selectlanguage{english}Name of a ritual.} \textcolor{PineGreen}{\selectlanguage{french}Nom d'un rituel.}  ¶ \textcolor{darkblue}{\textbf{\ipa{dʑɯ˩ʂo˥-tsɑ˩bɤ˩}}} \zh{做仪式时所使用的面粉。这种面粉里不要含有燕麦。仪式结束后,面粉被扔掉。} \textcolor{Sepia}{\selectlanguage{english}flour used for ceremonies; it must not contain oatmeal. After the ceremony, it is thrown away (not eaten).} \textcolor{PineGreen}{\selectlanguage{french}farine (tsamba) pouvant servir aux cérémonies; elle ne doit pas contenir d'avoine. A la fin de la cérémonie, on la jette}  

\lhead{\firstmark}
\rhead{\botmark}

\subsection{\hspace{-0.5cm} {\Large \textcolor{darkblue}{\textbf{\ipa{dʑɯ˩ʂwæ˩}}}}\hspace{0.5cm}[\kern2pt{\textcolor{darkblue}{\textbf{\ipa{dʑɯ˩ʂwæ˩˥}}}}\kern2pt]} \hypertarget{dz£M\string_Bs`w\{\string_B1}{}
\markboth{\textcolor{darkblue}{\textbf{\ipa{dʑɯ˩ʂwæ˩}}}}{}
\textcolor{teal}{\zh{名词}} \hspace{4pt} \zh{声调类:} L.
\zh{过路黄。} \textcolor{Sepia}{\selectlanguage{english}\textit{Lysimachia christinae Hance} yyyy translation into English.} \textcolor{PineGreen}{\selectlanguage{french}\textit{Lysimachia christinae Hance}.}  ¶ \textcolor{darkblue}{\textbf{\ipa{dʑɯ˩ʂwæ˩-bæ˥bæ˩}}} \zh{过路黄花} \textcolor{Sepia}{\selectlanguage{english}flower of yyyy} \textcolor{PineGreen}{\selectlanguage{french}fleur de yyyyy}  

\lhead{\firstmark}
\rhead{\botmark}

\subsection{\hspace{-0.5cm} {\Large \textcolor{darkblue}{\textbf{\ipa{dʑɯ˩tɤ˩ɻ̍˥}}}}\hspace{0.5cm}[\kern2pt{\textcolor{darkblue}{\textbf{\ipa{dʑɯ˩tɤ˩ɻ̍˥}}}}\kern2pt]} \hypertarget{dz£M\string_Bt7\string_Br£`̍\string_T1}{}
\markboth{\textcolor{darkblue}{\textbf{\ipa{dʑɯ˩tɤ˩ɻ̍˥}}}}{}
\textcolor{teal}{\zh{形容词}} \hspace{4pt} \zh{声调类:} L+H\#.
\zh{湿。} \textcolor{Sepia}{\selectlanguage{english}Humid, moist.} \textcolor{PineGreen}{\selectlanguage{french}Mouillé, humide.}  ¶ \textcolor{darkblue}{\textbf{\ipa{dʑɯ˩tɤ˩ɻ̍˥ gv̩˩-ze˩}}} \zh{湿了!} \textcolor{Sepia}{\selectlanguage{english}It got wet.} \textcolor{PineGreen}{\selectlanguage{french}ça s'est mouillé}  

\lhead{\firstmark}
\rhead{\botmark}

\subsection{\hspace{-0.5cm} {\Large \textcolor{darkblue}{\textbf{\ipa{dʑɯ˩tɕʰɯ˩lɑ˩qʰɑ˥}}}}\hspace{0.5cm}[\kern2pt{\textcolor{darkblue}{\textbf{\ipa{dʑɯ˩tɕʰɯ˩lɑ˩qʰɑ˥}}}}\kern2pt]} \hypertarget{dz£M\string_Bts£\string_hM\string_BlA\string_Bq\string_hA\string_T1}{}
\markboth{\textcolor{darkblue}{\textbf{\ipa{dʑɯ˩tɕʰɯ˩lɑ˩qʰɑ˥}}}}{}
\textcolor{teal}{\zh{名词}} \hspace{4pt} \zh{声调类:} L+H\#.
\zh{一种梅子。} \textcolor{Sepia}{\selectlanguage{english}Plum.} \textcolor{PineGreen}{\selectlanguage{french}Sorte de prunelle très acide, qui pousse à l'état sauvage; utilisée en décoction, en association avec du gingembre et de petites pommes séchées, contre les maux de gorge.} 
\lhead{\firstmark}
\rhead{\botmark}

\subsection{\hspace{-0.5cm} {\Large \textcolor{darkblue}{\textbf{\ipa{dʑɯ˩tɕʰɯ˩lɑ˩qʰæ˥}}}}\hspace{0.5cm}[\kern2pt{\textcolor{darkblue}{\textbf{\ipa{dʑɯ˩tɕʰɯ˩lɑ˩qʰæ˥}}}}\kern2pt]} \hypertarget{dz£M\string_Bts£\string_hM\string_BlA\string_Bq\string_h\{\string_T1}{}
\markboth{\textcolor{darkblue}{\textbf{\ipa{dʑɯ˩tɕʰɯ˩lɑ˩qʰæ˥}}}}{}
\textcolor{teal}{\zh{名词}} \hspace{4pt} \zh{声调类:} L+H\#.
\zh{沙棘。} \textcolor{Sepia}{\selectlanguage{english}Buckthorn, \textit{Hippophae rhamnoides Linn.}.} \textcolor{PineGreen}{\selectlanguage{french}Argousier, \textit{Hippophae rhamnoides Linn.}.} 
\lhead{\firstmark}
\rhead{\botmark}

\subsection{\hspace{-0.5cm} {\Large \textcolor{darkblue}{\textbf{\ipa{dʑɯ˩tsʰi˩}}}}\hspace{0.5cm}[\kern2pt{\textcolor{darkblue}{\textbf{\ipa{dʑɯ˩tsʰi˩˥}}}}\kern2pt]} \hypertarget{dz£M\string_Bts\string_hi\string_B1}{}
\markboth{\textcolor{darkblue}{\textbf{\ipa{dʑɯ˩tsʰi˩}}}}{}
\textcolor{teal}{\zh{名词}} \hspace{4pt} \zh{声调类:} L.
\zh{开水,热水。} \textcolor{Sepia}{\selectlanguage{english}Boiled water, hot water.} \textcolor{PineGreen}{\selectlanguage{french}Eau bouillante, eau chaude.} 
\lhead{\firstmark}
\rhead{\botmark}

\subsection{\hspace{-0.5cm} {\Large \textcolor{darkblue}{\textbf{\ipa{dʑɯ˩tsʰi˩ʈʰɯ˩di˩}}}}\hspace{0.5cm}[\kern2pt{\textcolor{darkblue}{\textbf{\ipa{dʑɯ˩tsʰi˩ʈʰɯ˩di˩˥}}}}\kern2pt]} \hypertarget{dz£M\string_Bts\string_hi\string_Bt`\string_hM\string_Bdi\string_B1}{}
\markboth{\textcolor{darkblue}{\textbf{\ipa{dʑɯ˩tsʰi˩ʈʰɯ˩di˩}}}}{}
\textcolor{teal}{\zh{名词}} \hspace{4pt} \zh{声调类:} L.
\zh{口杯。} \textcolor{Sepia}{\selectlanguage{english}Small container for hot water (for 1 person).} \textcolor{PineGreen}{\selectlanguage{french}Petite gourde thermos individuelle.}  \zh{量词}: \textcolor{darkblue}{\textbf{\ipa{ɭɯ˧}}} 
\lhead{\firstmark}
\rhead{\botmark}

\subsection{\hspace{-0.5cm} {\Large \textcolor{darkblue}{\textbf{\ipa{dʑɯ˩ʈv̩˧}}}}\hspace{0.5cm}[\kern2pt{\textcolor{darkblue}{\textbf{\ipa{dʑɯ˩ʈv̩˥}}}}\kern2pt]} \hypertarget{dz£M\string_Bt`v\string_=\string_M1}{}
\markboth{\textcolor{darkblue}{\textbf{\ipa{dʑɯ˩ʈv̩˧}}}}{}
\textcolor{teal}{\zh{形容词}} \hspace{4pt} \zh{声调类:} LM.
\zh{驼背(严重的病)。} \textcolor{Sepia}{\selectlanguage{english}To be a hunchback/humpback.} \textcolor{PineGreen}{\selectlanguage{french}Être gravement voûté, avoir une bosse.}  ¶ \textcolor{darkblue}{\textbf{\ipa{dʑɯ˩ʈv̩˧-ze˩}}} \zh{驼背了} \textcolor{Sepia}{\selectlanguage{english}\mytextsc{pfv}} \textcolor{PineGreen}{\selectlanguage{french}\mytextsc{pfv}}  

\lhead{\firstmark}
\rhead{\botmark}

\subsection{\hspace{-0.5cm} {\Large \textcolor{darkblue}{\textbf{\ipa{dʑɯ˩zo˩}}}}\hspace{0.5cm}[\kern2pt{\textcolor{darkblue}{\textbf{\ipa{dʑɯ˩zo˩˥}}}}\kern2pt]} \hypertarget{dz£M\string_Bzo\string_B1}{}
\markboth{\textcolor{darkblue}{\textbf{\ipa{dʑɯ˩zo˩}}}}{}
\textcolor{teal}{\zh{名词}} \hspace{4pt} \zh{声调类:} L.
\zh{溪流。} \textcolor{Sepia}{\selectlanguage{english}Brook, rivulet.} \textcolor{PineGreen}{\selectlanguage{french}Ruisseau.}  \zh{量词}: \textcolor{darkblue}{\textbf{\ipa{kʰɯ˩}}} 
\lhead{\firstmark}
\rhead{\botmark}

\subsection{\hspace{-0.5cm} {\Large \textcolor{darkblue}{\textbf{\ipa{dʑɯ˩ʐv̩˩}}}}\hspace{0.5cm}[\kern2pt{\textcolor{darkblue}{\textbf{\ipa{dʑɯ˩ʐv̩˩˥}}}}\kern2pt]} \hypertarget{dz£M\string_Bz`v\string_=\string_B1}{}
\markboth{\textcolor{darkblue}{\textbf{\ipa{dʑɯ˩ʐv̩˩}}}}{}
\textcolor{teal}{\zh{动词}} \hspace{4pt} \zh{声调类:} L.
\zh{游泳。} \textcolor{Sepia}{\selectlanguage{english}To swim.} \textcolor{PineGreen}{\selectlanguage{french}Nager.} 
\lhead{\firstmark}
\rhead{\botmark}

\subsection{\hspace{-0.5cm} {\Large \textcolor{darkblue}{\textbf{\ipa{dʑɯ˧˥}}}}\hspace{0.5cm}[\kern2pt{\textcolor{darkblue}{\textbf{\ipa{dʑɯ˥}}}}\kern2pt]} \hypertarget{dz£M\string_M\string_T1}{}
\markboth{\textcolor{darkblue}{\textbf{\ipa{dʑɯ˧˥}}}}{}
\textcolor{teal}{\zh{形容词}} \hspace{4pt} \zh{声调类:} MH.
\zh{多。} \textcolor{Sepia}{\selectlanguage{english}Many, much.} \textcolor{PineGreen}{\selectlanguage{french}Nombreux, beaucoup (dénombrable).}  ¶ \textcolor{darkblue}{\textbf{\ipa{hĩ˧ dʑɯ˩}}} \zh{人多。} \textcolor{Sepia}{\selectlanguage{english}There are many people.} \textcolor{PineGreen}{\selectlanguage{french}les gens sont nombreux, il y a beaucoup de monde}  

\lhead{\firstmark}
\rhead{\botmark}

\newpage
\section*{\centering- \textcolor{darkblue}{\textbf{\ipa{ɖ}}} -}
\subsection{\hspace{-0.5cm} {\Large \textcolor{darkblue}{\textbf{\ipa{ɖæ˥}}}}\hspace{0.5cm}[\kern2pt{\textcolor{darkblue}{\textbf{\ipa{ɖæ˥}}}}\kern2pt]} \hypertarget{d`\{\string_T1}{}
\markboth{\textcolor{darkblue}{\textbf{\ipa{ɖæ˥}}}}{}
\textcolor{teal}{\zh{形容词}} \hspace{4pt} \zh{声调类:} H.
\zh{短。} \textcolor{Sepia}{\selectlanguage{english}Short.} \textcolor{PineGreen}{\selectlanguage{french}Court.} 
\lhead{\firstmark}
\rhead{\botmark}

\subsection{\hspace{-0.5cm} {\Large \textcolor{darkblue}{\textbf{\ipa{ɖæ˧\textasciitilde{}ɖæ˩}}}}\hspace{0.5cm}[\kern2pt{\textcolor{darkblue}{\textbf{\ipa{ɖæ˧ɖæ˩}}}}\kern2pt]} \hypertarget{d`\{\string_M~d`\{\string_B1}{}
\markboth{\textcolor{darkblue}{\textbf{\ipa{ɖæ˧\textasciitilde{}ɖæ˩}}}}{}
\textcolor{teal}{\zh{形容词}} \hspace{4pt} \zh{声调类:} L\#.
\zh{横着(横躺在路上)。} \textcolor{Sepia}{\selectlanguage{english}Horizontal.} \textcolor{PineGreen}{\selectlanguage{french}Horizontal.}  ¶ \textcolor{darkblue}{\textbf{\ipa{ɖæ˧\textasciitilde{}ɖæ˩ | tʰi˧-tɕɯ˥}}} \zh{横着放} \textcolor{Sepia}{\selectlanguage{english}to lay flat} \textcolor{PineGreen}{\selectlanguage{french}poser à plat}  

\lhead{\firstmark}
\rhead{\botmark}

\subsection{\hspace{-0.5cm} {\Large \textcolor{darkblue}{\textbf{\ipa{ɖæ˩\textsubscript{a}}}}}\hspace{0.5cm}[\kern2pt{\textcolor{darkblue}{\textbf{\ipa{ɖæ˩˥}}}}\kern2pt]} \hypertarget{d`\{\string_Ba1}{}
\markboth{\textcolor{darkblue}{\textbf{\ipa{ɖæ˩\textsubscript{a}}}}}{}
\textcolor{teal}{\zh{量词}} \hspace{4pt} \zh{声调类:} L\textsubscript{a}.
\zh{量词:路(段)/布(匹)。} \textcolor{Sepia}{\selectlanguage{english}A section of (road); a bolt of cloth.} \textcolor{PineGreen}{\selectlanguage{french}Classificateur des sections, pour objets pouvant être divisés dans le sens de la longueur. Pour une route, cette unité correspond à 1/2 journée de marche; pour du tissu, elle correspond à une pièce.}  ¶ \textcolor{darkblue}{\textbf{\ipa{ʐɤ˩mi˩˥ | ɖɯ˧-ɖæ˩}}} \zh{一段路} \textcolor{Sepia}{\selectlanguage{english}a section of road, a stretch of road} \textcolor{PineGreen}{\selectlanguage{french}un bout de chemin, un bout de route}  
 ¶ \textcolor{darkblue}{\textbf{\ipa{ɲi˧, ɲi˩-ɖæ˩! |}}} \zh{一天两段路!(说明:早上出发,如果午饭前能到目的地,距离算是一段,如果下午晚上才到,算两段。)} \textcolor{Sepia}{\selectlanguage{english}Two stretches a day! (Set phrase: in one day, one can cover a distance of two 'stretches'. If one can get somewhere before lunch, the distance counts as one stretch/section; if one can only arrive there in the afternoon, it counts as two stretches/sections.)} \textcolor{PineGreen}{\selectlanguage{french}formule figée traditionnel: un jour, ça fait deux étapes! (si on peut parvenir quelque part avant le déjeuner, la distance est considérée comme une seule étape; si on y parvient dans l'après-midi, on compte deux étapes)}  
 ¶ \textcolor{darkblue}{\textbf{\ipa{ʈʂʰɯ˧-ɖæ˥}}} \zh{\mytextsc{指示代词} \string_} \textcolor{Sepia}{\selectlanguage{english}\mytextsc{dem} \string_ (tone: H\# / H\$)} \textcolor{PineGreen}{\selectlanguage{french}\mytextsc{dem} \string_ (ton: H\# / H\$)}  

\lhead{\firstmark}
\rhead{\botmark}

\subsection{\hspace{-0.5cm} {\Large \textcolor{darkblue}{\textbf{\ipa{ɖæ˩\textsubscript{a}}}}}\hspace{0.5cm}[\kern2pt{\textcolor{darkblue}{\textbf{\ipa{ɖæ˩˥}}}}\kern2pt]} \hypertarget{d`\{\string_Ba1}{}
\markboth{\textcolor{darkblue}{\textbf{\ipa{ɖæ˩\textsubscript{a}}}}}{}
\textcolor{teal}{\zh{动词}} \hspace{4pt} \zh{声调类:} L\textsubscript{a}.
\zh{渡(坐船渡河……)。} \textcolor{Sepia}{\selectlanguage{english}To pass over, to cross (a river on a boat, a mountain…).} \textcolor{PineGreen}{\selectlanguage{french}Passer, traverser (une rivière en bateau, une montagne…).}  ¶ \textcolor{darkblue}{\textbf{\ipa{dʑɯ˩ ɖæ˩˥ / dʑɯ˩ ɖæ˩-ze˥}}} \zh{渡河} \textcolor{Sepia}{\selectlanguage{english}to cross a river} \textcolor{PineGreen}{\selectlanguage{french}passer une rivière}  
 ¶ \textcolor{darkblue}{\textbf{\ipa{dʑɯ˧ | ɖɯ˧-kʰɯ˩ ɖæ˩}}} \zh{同上} \textcolor{Sepia}{\selectlanguage{english}as above} \textcolor{PineGreen}{\selectlanguage{french}idem}  

\lhead{\firstmark}
\rhead{\botmark}

\subsection{\hspace{-0.5cm} {\Large \textcolor{darkblue}{\textbf{\ipa{ɖæ˩dʑɯ˥}}}}\hspace{0.5cm}[\kern2pt{\textcolor{darkblue}{\textbf{\ipa{ɖæ˩dʑɯ˥}}}}\kern2pt]} \hypertarget{d`\{\string_Bdz£M\string_T1}{}
\markboth{\textcolor{darkblue}{\textbf{\ipa{ɖæ˩dʑɯ˥}}}}{}
\textcolor{teal}{\zh{名词}} \hspace{4pt} \zh{声调类:} LH.
\zh{污垢。} \textcolor{Sepia}{\selectlanguage{english}Dirt, filth.} \textcolor{PineGreen}{\selectlanguage{french}Détritus, saletés, crasse.}  \zh{量词}: \textcolor{darkblue}{\textbf{\ipa{ʁwɤ˧, etc}}} 
\lhead{\firstmark}
\rhead{\botmark}

\subsection{\hspace{-0.5cm} {\Large \textcolor{darkblue}{\textbf{\ipa{ɖæ˩-lɑ˧so˧}}}}\hspace{0.5cm}[\kern2pt{\textcolor{darkblue}{\textbf{\ipa{ɖæ˧lɑ˧so˧}}}}\kern2pt]} \hypertarget{d`\{\string_B-lA\string_Mso\string_M1}{}
\markboth{\textcolor{darkblue}{\textbf{\ipa{ɖæ˩-lɑ˧so˧}}}}{}
\textcolor{teal}{\zh{名词}} \hspace{4pt} \zh{声调类:} L-.
\zh{一种祈福仪式,和尚在过年时主持行礼。} \textcolor{Sepia}{\selectlanguage{english}Name of a ceremony conducted at home once a year, during the first two weeks of the year, by one or two monks invited to the farm: offering grain (or fruit) to the gods. The aim is to ensure prosperity for the household.} \textcolor{PineGreen}{\selectlanguage{french}Nom de cérémonie que les moines (un ou deux) pratiquent une fois par an (pendant la première quinzaine de l'année) au domicile de la personne qui les y invite: offrande de céréales (ou de fruits) aux divinités. L'objet de la cérémonie est d'assurer la prospérité financière de la maisonnée.}  \zh{【借词】}\zh{藏语} 'bras lha gsol

\lhead{\firstmark}
\rhead{\botmark}

\subsection{\hspace{-0.5cm} {\Large \textcolor{darkblue}{\textbf{\ipa{ɖæ˩mi˧}}}}\hspace{0.5cm}[\kern2pt{\textcolor{darkblue}{\textbf{\ipa{ɖæ˩mi˥}}}}\kern2pt]} \hypertarget{d`\{\string_Bmi\string_M1}{}
\markboth{\textcolor{darkblue}{\textbf{\ipa{ɖæ˩mi˧}}}}{}
\textcolor{teal}{\zh{名词}} \hspace{4pt} \zh{声调类:} LM.
\zh{永宁大寺。} \textcolor{Sepia}{\selectlanguage{english}The Yongning monastery.} \textcolor{PineGreen}{\selectlanguage{french}Nom du temple de Yongning.}  \zh{【借词】}\zh{藏语} dgra med
 ¶ \textcolor{darkblue}{\textbf{\ipa{ɖæ˩mi˧-ʈæ˩bɤ˩}}} \zh{永宁大寺的和尚} \textcolor{Sepia}{\selectlanguage{english}a priest from the monastery} \textcolor{PineGreen}{\selectlanguage{french}un prêtre du monastère}  

\lhead{\firstmark}
\rhead{\botmark}

\subsection{\hspace{-0.5cm} {\Large \textcolor{darkblue}{\textbf{\ipa{ɖæ˩mi˧-go˧bɤ˩}}}}\hspace{0.5cm}[\kern2pt{\textcolor{darkblue}{\textbf{\ipa{ɖæ˩mi˧go˧bɤ˩}}}}\kern2pt]} \hypertarget{d`\{\string_Bmi\string_M-go\string_Mb7\string_B1}{}
\markboth{\textcolor{darkblue}{\textbf{\ipa{ɖæ˩mi˧-go˧bɤ˩}}}}{}
\textcolor{teal}{\zh{名词}} \hspace{4pt} \zh{声调类:} LM-L\#.
\zh{永宁大寺。} \textcolor{Sepia}{\selectlanguage{english}Yongning temple.} \textcolor{PineGreen}{\selectlanguage{french}Le temple de Yongning.}  \zh{【借词】}\zh{藏语} dgra med dgon pa

\lhead{\firstmark}
\rhead{\botmark}

\subsection{\hspace{-0.5cm} {\Large \textcolor{darkblue}{\textbf{\ipa{ɖæ˩pʰv̩˥}}}}\hspace{0.5cm}[\kern2pt{\textcolor{darkblue}{\textbf{\ipa{ɖæ˩pʰv̩˥}}}}\kern2pt]} \hypertarget{d`\{\string_Bp\string_hv\string_=\string_T1}{}
\markboth{\textcolor{darkblue}{\textbf{\ipa{ɖæ˩pʰv̩˥}}}}{}
\textcolor{teal}{\zh{名词}} \hspace{4pt} \zh{声调类:} LH.
\zh{灰尘。} \textcolor{Sepia}{\selectlanguage{english}Dust, dirt.} \textcolor{PineGreen}{\selectlanguage{french}Poussière.}  \zh{量词}: \textcolor{darkblue}{\textbf{\ipa{ti˧˥}}} 
\lhead{\firstmark}
\rhead{\botmark}

\subsection{\hspace{-0.5cm} {\Large \textcolor{darkblue}{\textbf{\ipa{ɖæ˩ʂɯ\#˥}}}}\hspace{0.5cm}[\kern2pt{\textcolor{darkblue}{\textbf{\ipa{ɖæ˩ʂɯ˥}}}}\kern2pt]} \hypertarget{d`\{\string_Bs`M\#\string_T1}{}
\markboth{\textcolor{darkblue}{\textbf{\ipa{ɖæ˩ʂɯ\#˥}}}}{}
\textcolor{teal}{\zh{名词}} \hspace{4pt} \zh{声调类:} LM+\#H.
\zh{扎实(永宁的一个村落)。} \textcolor{Sepia}{\selectlanguage{english}A village name.} \textcolor{PineGreen}{\selectlanguage{french}Nom de village.}  ¶ \textcolor{darkblue}{\textbf{\ipa{ɖæ˩ʂɯ˧-ʁwɤ\#˥}}} \zh{同上:扎实村} \textcolor{Sepia}{\selectlanguage{english}same meaning} \textcolor{PineGreen}{\selectlanguage{french}même sens}  
 ¶ \textcolor{darkblue}{\textbf{\ipa{ɖæ˩ʂɯ\#˥, | ʈʂo˧ʂɯ\#˥, | bɤ˩tɕʰɯ˩˥, | dɑ˧pʰo˥, | bɤ˧dzi˩, | dze˧bo˧}}} \zh{永宁坝的六个村落,按传统排序:从距离泸沽湖最近的村落说起。} \textcolor{Sepia}{\selectlanguage{english}the six villages of the plain of Yongning, in traditional order: by order of increasing distance from the Lake} \textcolor{PineGreen}{\selectlanguage{french}les six villages de la plaine de Yongning, dans l'ordre, qui prend comme point d'origine le village le plus proche du Lac}  

\lhead{\firstmark}
\rhead{\botmark}

\subsection{\hspace{-0.5cm} {\Large \textcolor{darkblue}{\textbf{\ipa{ɖæ˩˧}}}}\hspace{0.5cm}[\kern2pt{\textcolor{darkblue}{\textbf{\ipa{ɖæ˩˥}}}}\kern2pt]} \hypertarget{d`\{\string_B\string_M1}{}
\markboth{\textcolor{darkblue}{\textbf{\ipa{ɖæ˩˧}}}}{}
\textcolor{teal}{\zh{名词}} \hspace{4pt} \zh{声调类:} LM.
\ding{202} \zh{尘土。} \textcolor{Sepia}{\selectlanguage{english}Dust.} \textcolor{PineGreen}{\selectlanguage{french}Poussière.}  ¶ \textcolor{darkblue}{\textbf{\ipa{ɖæ˩˥ | ɖɯ˧-ti˧ tʰi˧-di˥}}} \zh{有一层灰} \textcolor{Sepia}{\selectlanguage{english}there is a layer of dust} \textcolor{PineGreen}{\selectlanguage{french}il y a une couche de poussière}  
 ¶ \textcolor{darkblue}{\textbf{\ipa{ɖæ˩ bæ˧}}} \zh{扫灰} \textcolor{Sepia}{\selectlanguage{english}to sweep the dust} \textcolor{PineGreen}{\selectlanguage{french}balayer la poussière}  
 \zh{量词}: \textcolor{darkblue}{\textbf{\ipa{ti˧˥}}} \ding{203} \zh{污垢。} \textcolor{Sepia}{\selectlanguage{english}Dirt, filth.} \textcolor{PineGreen}{\selectlanguage{french}Saletés.} 
\lhead{\firstmark}
\rhead{\botmark}

\subsection{\hspace{-0.5cm} {\Large \textcolor{darkblue}{\textbf{\ipa{ɖɤ˥}}}}\hspace{0.5cm}[\kern2pt{\textcolor{darkblue}{\textbf{\ipa{ɖɤ˥}}}}\kern2pt]} \hypertarget{d`7\string_T1}{}
\markboth{\textcolor{darkblue}{\textbf{\ipa{ɖɤ˥}}}}{}
\textcolor{teal}{\zh{动词}} \hspace{4pt} \zh{声调类:} H.
\zh{爬行,匍匐。} \textcolor{Sepia}{\selectlanguage{english}To crawl, to creep.} \textcolor{PineGreen}{\selectlanguage{french}Ramper.}  ¶ \textcolor{darkblue}{\textbf{\ipa{ɖɤ˧\textasciitilde{}ɖɤ˧ (-ze˩)}}} \zh{\mytextsc{重叠:爬一爬}} \textcolor{Sepia}{\selectlanguage{english}\mytextsc{red}} \textcolor{PineGreen}{\selectlanguage{french}\mytextsc{red}}  
 ¶ \textcolor{darkblue}{\textbf{\ipa{ʈʂʰɯ˧ | ɖɤ˧\textasciitilde{}ɖɤ˧-ʁo˧-ze˩!}}} \zh{她会爬了!} \textcolor{Sepia}{\selectlanguage{english}She can crawl! / She knows how to crawl! (About a baby that crawls around.)} \textcolor{PineGreen}{\selectlanguage{french}Elle sait ramper! (Au sujet d'un bébé qui se déplace en rampant.)}  

\lhead{\firstmark}
\rhead{\botmark}

\subsection{\hspace{-0.5cm} {\Large \textcolor{darkblue}{\textbf{\ipa{ɖɤ˧mi˧}}}}\hspace{0.5cm}[\kern2pt{\textcolor{darkblue}{\textbf{\ipa{ɖɤ˧mi˧}}}}\kern2pt]} \hypertarget{d`7\string_Mmi\string_M1}{}
\markboth{\textcolor{darkblue}{\textbf{\ipa{ɖɤ˧mi˧}}}}{}
\textcolor{teal}{\zh{名词}} \hspace{4pt} \zh{声调类:} M.
\zh{狐狸。} \textcolor{Sepia}{\selectlanguage{english}Fox.} \textcolor{PineGreen}{\selectlanguage{french}Renard.}  ¶ \textcolor{darkblue}{\textbf{\ipa{ɖɤ˧mi˧-zo\#˥}}} \zh{小狐狸} \textcolor{Sepia}{\selectlanguage{english}little fox, baby fox} \textcolor{PineGreen}{\selectlanguage{french}petit renard, renardeau}  
 ¶ \textcolor{darkblue}{\textbf{\ipa{ɖɤ˧mi˧-pʰv̩\#˥}}} \zh{公狐狸} \textcolor{Sepia}{\selectlanguage{english}male fox} \textcolor{PineGreen}{\selectlanguage{french}renard mâle}  
 ¶ \textcolor{darkblue}{\textbf{\ipa{ɖɤ˧mi˧, | mi˩ ɲi˥!}}} \zh{这只狐狸是母的!} \textcolor{Sepia}{\selectlanguage{english}This fox is a female!} \textcolor{PineGreen}{\selectlanguage{french}Ce renard, c'est une femelle!}  
 \zh{量词}: \textcolor{darkblue}{\textbf{\ipa{pʰo˧˥}}} 
\lhead{\firstmark}
\rhead{\botmark}

\subsection{\hspace{-0.5cm} {\Large \textcolor{darkblue}{\textbf{\ipa{ɖɤ˩\textsubscript{a}}}}}\hspace{0.5cm}[\kern2pt{\textcolor{darkblue}{\textbf{\ipa{ɖɤ˩˥}}}}\kern2pt]} \hypertarget{d`7\string_Ba1}{}
\markboth{\textcolor{darkblue}{\textbf{\ipa{ɖɤ˩\textsubscript{a}}}}}{}
\textcolor{teal}{\zh{形容词}} \hspace{4pt} \zh{声调类:} L\textsubscript{a}.
\zh{很热(天气),阳光强烈。} \textcolor{Sepia}{\selectlanguage{english}Hot (weather).} \textcolor{PineGreen}{\selectlanguage{french}Brûlant, ardent (soleil), chaud (temps).}  ¶ \textcolor{darkblue}{\textbf{\ipa{ɲi˧mi˧ | ɖɤ˩-ze˥!}}} \zh{太阳很大、很强烈} \textcolor{Sepia}{\selectlanguage{english}The sun is burning hot, scalding} \textcolor{PineGreen}{\selectlanguage{french}le soleil est torride, le soleil est très vif}  
 ¶ \textcolor{darkblue}{\textbf{\ipa{ɖɤ˩-hĩ˩˥}}} \zh{热的} \textcolor{Sepia}{\selectlanguage{english}\mytextsc{rel}} \textcolor{PineGreen}{\selectlanguage{french}\mytextsc{rel}}  

\lhead{\firstmark}
\rhead{\botmark}

\subsection{\hspace{-0.5cm} {\Large \textcolor{darkblue}{\textbf{\ipa{ɖo˧}}}}\hspace{0.5cm}[\kern2pt{\textcolor{darkblue}{\textbf{\ipa{ɖo˥}}}}\kern2pt]} \hypertarget{d`o\string_M1}{}
\markboth{\textcolor{darkblue}{\textbf{\ipa{ɖo˧}}}}{}
\textcolor{teal}{\zh{动词}} \hspace{4pt} \zh{声调类:} M intrans.
\zh{让,指使、使唤。} \textcolor{Sepia}{\selectlanguage{english}To allow, to permit; also: to order about; to run errands for.} \textcolor{PineGreen}{\selectlanguage{french}Devoir, être obligé de; permettre, autoriser, accorder; ordonner, donner un ordre.}  ¶ \textcolor{darkblue}{\textbf{\ipa{po˧ mɤ˧-ɖo˧!}}} \zh{不许拿!} \textcolor{Sepia}{\selectlanguage{english}(You) are not allowed to take it! / You must not take it! (eg telling a child that (s)he is not allowed to take a knife)} \textcolor{PineGreen}{\selectlanguage{french}(Tu n'as) pas le droit de le prendre! (ex.: un enfant n'est pas autorisé à jouer avec un couteau)}  
 ¶ \textcolor{darkblue}{\textbf{\ipa{ʈʂʰɯ˧, | po˧ ɖo˧!}}} \zh{那个,是可以拿的! / 那个,是可以玩的!(情景同上:告诉一个小孩子什么东西可以拿,什么不可以拿。)} \textcolor{Sepia}{\selectlanguage{english}That one, you can have it / you can take it / you can play with it! (Context: as above: telling a child what (s)he can and cannot toy with.)} \textcolor{PineGreen}{\selectlanguage{french}Ca, (tu) peux le prendre / tu peux jouer avec! (Même contexte que ci-dessus: on indique à un enfant ce qu'on a le droit de prendre, et ce qu'on n'a pas le droit de prendre.)}  
 ¶ \textcolor{darkblue}{\textbf{\ipa{gɤ˩ do˧ mɤ˧-ɖo˧!}}} \zh{不许爬上(桌子……)} \textcolor{Sepia}{\selectlanguage{english}(You) are not allowed to climb (on the table,...)} \textcolor{PineGreen}{\selectlanguage{french}(tu) n'as pas le droit de grimper/monter sur (une table…)}  
 ¶ \textcolor{darkblue}{\textbf{\ipa{lɑ˧-kʰv̩˧˥, | ʑi˧qʰwɤ˧ tsʰi˧-mɤ˧-ɖo˧! | ʑi˩-kʰv̩˩˥, | ʑi˧qʰwɤ˧ tsʰi˧-mɤ˧-ɖo˧! |}}} \zh{虎年,不要建房!猴年,不要建房!(这样的年,被认为是太‘硬’的。)} \textcolor{Sepia}{\selectlanguage{english}(During) the year of the Tiger, one should not build a house! (During) the year of the Monkey, one should not build a house! (These years are considered too “hard”, \textcolor{darkblue}{\textbf{\ipa{/wu˧/}}}, by astrology.)} \textcolor{PineGreen}{\selectlanguage{french}L'année du Tigre, l'année du Singe, il ne faut pas construire de maison/il ne faut pas se lancer dans la construction d'une maison! (Ce sont des années trop “dures”, \textcolor{darkblue}{\textbf{\ipa{/wu˧/}}}, selon l'astrologie traditionnelle)}  
 ¶ \textcolor{darkblue}{\textbf{\ipa{ʝi˧ mɤ˧-ɖo˧!}}} \zh{不要做!} \textcolor{Sepia}{\selectlanguage{english}(One) must not do (that)!} \textcolor{PineGreen}{\selectlanguage{french}Il ne faut pas faire (ça)!}  

\lhead{\firstmark}
\rhead{\botmark}

\subsection{\hspace{-0.5cm} {\Large \textcolor{darkblue}{\textbf{\ipa{ɖɯ˧-}}}}\hspace{0.5cm}[\kern2pt{\textcolor{darkblue}{\textbf{\ipa{ɖɯ˥}}}}\kern2pt]} \hypertarget{d`M\string_M-1}{}
\markboth{\textcolor{darkblue}{\textbf{\ipa{ɖɯ˧-}}}}{}
\textcolor{teal}{\zh{介词}} \hspace{4pt} \zh{声调类:} M.
\zh{\mytextsc{进行时态。}} \textcolor{Sepia}{\selectlanguage{english}\mytextsc{delimitative}.} \textcolor{PineGreen}{\selectlanguage{french}\mytextsc{délimitatif}.} 
\lhead{\firstmark}
\rhead{\botmark}

\subsection{\hspace{-0.5cm} {\Large \textcolor{darkblue}{\textbf{\ipa{ɖɯ˧\textsubscript{b}}}}}\hspace{0.5cm}[\kern2pt{\textcolor{darkblue}{\textbf{\ipa{ɖɯ˥}}}}\kern2pt]} \hypertarget{d`M\string_Mb1}{}
\markboth{\textcolor{darkblue}{\textbf{\ipa{ɖɯ˧\textsubscript{b}}}}}{}
\textcolor{teal}{\zh{动词}} \hspace{4pt} \zh{声调类:} M\textsubscript{b}.
\zh{得到。} \textcolor{Sepia}{\selectlanguage{english}To obtain, to get.} \textcolor{PineGreen}{\selectlanguage{french}Obtenir, trouver.}  ¶ \textcolor{darkblue}{\textbf{\ipa{le˧-ʂe˧ le˧-ɖɯ˧-ze˧!}}} \zh{找到了!} \textcolor{Sepia}{\selectlanguage{english}(I) have looked for something and found it! / I have found (something by looking around for it)!} \textcolor{PineGreen}{\selectlanguage{french}j'ai cherché et je l'ai trouvé! j'ai trouvé (en cherchant)!}  
 ¶ \textcolor{darkblue}{\textbf{\ipa{ɖɯ˧-tʰɑ˧˥!}}} \zh{可以得到的!} \textcolor{Sepia}{\selectlanguage{english}It is possible to obtain it! / It can be obtained!} \textcolor{PineGreen}{\selectlanguage{french}On peut obtenir!}  
 ¶ \textcolor{darkblue}{\textbf{\ipa{ɖɯ˧-tʰɑ˧-ze˥!}}} \zh{(我们)成功地得到了!} \textcolor{Sepia}{\selectlanguage{english}We have managed to obtain it! / We found it possible to obtain it!} \textcolor{PineGreen}{\selectlanguage{french}On a réussi à obtenir!}  
 ¶ \textcolor{darkblue}{\textbf{\ipa{tso˧\textasciitilde{}tso˧ ɖɯ˧ (+ze˧)}}} \zh{得到东西} \textcolor{Sepia}{\selectlanguage{english}to obtain something} \textcolor{PineGreen}{\selectlanguage{french}obtenir quelque chose}  

\lhead{\firstmark}
\rhead{\botmark}

\subsection{\hspace{-0.5cm} {\Large \textcolor{darkblue}{\textbf{\ipa{ɖɯ˧-ɬi˧mi˧}}}}\hspace{0.5cm}[\kern2pt{\textcolor{darkblue}{\textbf{\ipa{xxxx non-correspondance entre le nombre de morphèmes et le nombre de tons de morphèmes}}}}\kern2pt]} \hypertarget{d`M\string_M-Ki\string_Mmi\string_M1}{}
\markboth{\textcolor{darkblue}{\textbf{\ipa{ɖɯ˧-ɬi˧mi˧}}}}{}
\textcolor{teal}{\zh{名词}} \hspace{4pt} \zh{声调类:} M.
\zh{正月。} \textcolor{Sepia}{\selectlanguage{english}1st month.} \textcolor{PineGreen}{\selectlanguage{french}1er mois.} 
\lhead{\firstmark}
\rhead{\botmark}

\subsection{\hspace{-0.5cm} {\Large \textcolor{darkblue}{\textbf{\ipa{ɖɯ˧-njɤ˧}}}}\hspace{0.5cm}[\kern2pt{\textcolor{darkblue}{\textbf{\ipa{xxxx non-correspondance entre le nombre de morphèmes et le nombre de tons de morphèmes}}}}\kern2pt]} \hypertarget{d`M\string_M-nj7\string_M1}{}
\markboth{\textcolor{darkblue}{\textbf{\ipa{ɖɯ˧-njɤ˧}}}}{}
\textcolor{teal}{\zh{助词}} \hspace{4pt} \zh{声调类:} M.
\zh{一直、一直不停。} \textcolor{Sepia}{\selectlanguage{english}Continuously, ceaselessly.} \textcolor{PineGreen}{\selectlanguage{french}Sans cesse; sans arrêt; toujours.}  ¶ \textcolor{darkblue}{\textbf{\ipa{ɖɯ˧-njɤ˧ | so˩˥}}} \zh{一直不停地学习} \textcolor{Sepia}{\selectlanguage{english}to study ceaselessly} \textcolor{PineGreen}{\selectlanguage{french}étudier sans arrêt}  
 ¶ \textcolor{darkblue}{\textbf{\ipa{ɖɯ˧-njɤ˧ | lo˧ ʝi˧}}} \zh{一直不停地工作} \textcolor{Sepia}{\selectlanguage{english}to work ceaselessly} \textcolor{PineGreen}{\selectlanguage{french}travailler sans arrêt}  
 ¶ \textcolor{darkblue}{\textbf{\ipa{ɖɯ˧-njɤ˧-zo˥}}} \zh{经常、常} \textcolor{Sepia}{\selectlanguage{english}often} \textcolor{PineGreen}{\selectlanguage{french}souvent}  
 ¶ \textcolor{darkblue}{\textbf{\ipa{ɖɯ˧-njɤ˧ hwæ˩; ɖɯ˧-njɤ˧ tɕʰi˧; ɖɯ˧-njɤ˧ dzɯ˧; ɖɯ˧-njɤ˧ dze˧˥; ɖɯ˧-njɤ˧ ʐwɤ˧˥; ɖɯ˧-njɤ˧ lɑ˧˥}}} \zh{跟六个调类的动词结合:买,卖,吃,切,说,打} \textcolor{Sepia}{\selectlanguage{english}combinations with verbs in the six tones: to buy, to sell, to eat, to cut, to speak, to strike} \textcolor{PineGreen}{\selectlanguage{french}avec des verbes aux six tons, pour étudier les tons : acheter; vendre; manger; couper; parler; frapper}  
 ¶ \textcolor{darkblue}{\textbf{\ipa{ɖɯ˧-njɤ˧ | hwæ˧; ɖɯ˧-njɤ˧ | tɕʰi˧; ɖɯ˧-njɤ˧ | dzɯ˧; ɖɯ˧-njɤ˧ | dze˩˥; ɖɯ˧-njɤ˧ | ʐwɤ˩˥; ɖɯ˧-njɤ˧ | lɑ˧˥}}} \zh{跟六个调类的动词结合:买,卖,吃,切,说,打} \textcolor{Sepia}{\selectlanguage{english}combinations with verbs in the six tones: to buy, to sell, to eat, to cut, to speak, to strike (separating into two tone groups)} \textcolor{PineGreen}{\selectlanguage{french}avec des verbes aux six tons, pour étudier les tons : acheter; vendre; manger; couper; parler; frapper; en séparant en groupes tonals}  

\lhead{\firstmark}
\rhead{\botmark}

\subsection{\hspace{-0.5cm} {\Large \textcolor{darkblue}{\textbf{\ipa{ɖɯ˧-ɲi˧-ɖɯ˥-hɑ̃˩}}}}\hspace{0.5cm}[\kern2pt{\textcolor{darkblue}{\textbf{\ipa{xxxx non-correspondance entre le nombre de morphèmes et le nombre de tons de morphèmes}}}}\kern2pt]} \hypertarget{d`M\string_M-Ji\string_M-d`M\string_T-hA\string_~\string_B1}{}
\markboth{\textcolor{darkblue}{\textbf{\ipa{ɖɯ˧-ɲi˧-ɖɯ˥-hɑ̃˩}}}}{}
\textcolor{teal}{\zh{名词}} \hspace{4pt} \zh{声调类:} \#H-.
\zh{一天一夜。} \textcolor{Sepia}{\selectlanguage{english}One day and one night.} \textcolor{PineGreen}{\selectlanguage{french}Un jour et une nuit.} 
\lhead{\firstmark}
\rhead{\botmark}

\subsection{\hspace{-0.5cm} {\Large \textcolor{darkblue}{\textbf{\ipa{ɖɯ˧-so˩}}}}\hspace{0.5cm}[\kern2pt{\textcolor{darkblue}{\textbf{\ipa{xxxx non-correspondance entre le nombre de morphèmes et le nombre de tons de morphèmes}}}}\kern2pt]} \hypertarget{d`M\string_M-so\string_B1}{}
\markboth{\textcolor{darkblue}{\textbf{\ipa{ɖɯ˧-so˩}}}}{}
\textcolor{teal}{\zh{名词}} \hspace{4pt} \zh{声调类:} L\textsubscript{a}.
\zh{一些、两三个(直译:‘一三(个)’。} \textcolor{Sepia}{\selectlanguage{english}Some, a few. Made up of 'one' and 'three'.} \textcolor{PineGreen}{\selectlanguage{french}Un petit nombre de, quelques-uns.}  ¶ \textcolor{darkblue}{\textbf{\ipa{hĩ˧ | ɖɯ˧-so˩ kv̩˩}}} \zh{几个人} \textcolor{Sepia}{\selectlanguage{english}a few people} \textcolor{PineGreen}{\selectlanguage{french}quelques personnes (deux, trois...)}  
 ¶ \textcolor{darkblue}{\textbf{\ipa{ɖɯ˧-so˩ ɲi˩}}} \zh{几天} \textcolor{Sepia}{\selectlanguage{english}a few days} \textcolor{PineGreen}{\selectlanguage{french}quelques jours}  

\lhead{\firstmark}
\rhead{\botmark}

\subsection{\hspace{-0.5cm} {\Large \textcolor{darkblue}{\textbf{\ipa{ɖɯ˧ʈæ˩}}}}\hspace{0.5cm}[\kern2pt{\textcolor{darkblue}{\textbf{\ipa{ɖɯ˧ʈæ˩}}}}\kern2pt]} \hypertarget{d`M\string_Mt`\{\string_B1}{}
\markboth{\textcolor{darkblue}{\textbf{\ipa{ɖɯ˧ʈæ˩}}}}{}
\textcolor{teal}{\zh{名词}} \hspace{4pt} \zh{声调类:} L\#.
\zh{一个葬礼仪式,由和尚主持。} \textcolor{Sepia}{\selectlanguage{english}A ritual performed by monks after someone's decease.} \textcolor{PineGreen}{\selectlanguage{french}Rite pratiqué par les moines du monastère après un décès.} 
\lhead{\firstmark}
\rhead{\botmark}

\subsection{\hspace{-0.5cm} {\Large \textcolor{darkblue}{\textbf{\ipa{ɖɯ˩\textsubscript{a}}}}}\hspace{0.5cm}[\kern2pt{\textcolor{darkblue}{\textbf{\ipa{ɖɯ˩˥}}}}\kern2pt]} \hypertarget{d`M\string_Ba1}{}
\markboth{\textcolor{darkblue}{\textbf{\ipa{ɖɯ˩\textsubscript{a}}}}}{}
\textcolor{teal}{\zh{形容词}} \hspace{4pt} \zh{声调类:} L\textsubscript{a}.
\zh{大。} \textcolor{Sepia}{\selectlanguage{english}Big, large.} \textcolor{PineGreen}{\selectlanguage{french}Grand.}  ¶ \textcolor{darkblue}{\textbf{\ipa{mɤ˧-ɖɯ˩}}} \zh{\mytextsc{neg}} \textcolor{Sepia}{\selectlanguage{english}\mytextsc{neg}} \textcolor{PineGreen}{\selectlanguage{french}\mytextsc{neg}}  
 ¶ \textcolor{darkblue}{\textbf{\ipa{ɖɯ˩-hĩ˩˥}}} \zh{\mytextsc{rel}} \textcolor{Sepia}{\selectlanguage{english}\mytextsc{rel}} \textcolor{PineGreen}{\selectlanguage{french}\mytextsc{rel}}  
 ¶ \textcolor{darkblue}{\textbf{\ipa{le˧-ɖɯ˩(-ze˩)}}} \zh{\mytextsc{accomp}+\mytextsc{pfv}} \textcolor{Sepia}{\selectlanguage{english}\mytextsc{accomp}+\mytextsc{pfv}} \textcolor{PineGreen}{\selectlanguage{french}\mytextsc{accomp}+\mytextsc{pfv}: ça a grandi!/ il/elle a grandi!}  
 ¶ \textcolor{darkblue}{\textbf{\ipa{ə˧pɤ˥ɖɯ˩-gv̩˩}}} \zh{好大、大得很} \textcolor{Sepia}{\selectlanguage{english}very big} \textcolor{PineGreen}{\selectlanguage{french}très grand}  

\lhead{\firstmark}
\rhead{\botmark}

\subsection{\hspace{-0.5cm} {\Large \textcolor{darkblue}{\textbf{\ipa{ɖɯ˩ɖʐɯ˧}}}}\hspace{0.5cm}[\kern2pt{\textcolor{darkblue}{\textbf{\ipa{ɖɯ˩ɖʐɯ˥}}}}\kern2pt]} \hypertarget{d`M\string_Bd`z`M\string_M1}{}
\markboth{\textcolor{darkblue}{\textbf{\ipa{ɖɯ˩ɖʐɯ˧}}}}{}
\textcolor{teal}{\zh{名词}} \hspace{4pt} \zh{声调类:} LM.
\zh{男性名字:独知。} \textcolor{Sepia}{\selectlanguage{english}Masculine given name.} \textcolor{PineGreen}{\selectlanguage{french}Prénom masculin.}  \zh{【借词】}\zh{藏语}

\lhead{\firstmark}
\rhead{\botmark}

\subsection{\hspace{-0.5cm} {\Large \textcolor{darkblue}{\textbf{\ipa{ɖɯ˩ɖʐɯ˧-tsʰɯ˩ɻ̍˩}}}}\hspace{0.5cm}[\kern2pt{\textcolor{darkblue}{\textbf{\ipa{ɖɯ˩ɖʐɯ˧tsʰɯ˩ɻ̍˩}}}}\kern2pt]} \hypertarget{d`M\string_Bd`z`M\string_M-ts\string_hM\string_Br£`̍\string_B1}{}
\markboth{\textcolor{darkblue}{\textbf{\ipa{ɖɯ˩ɖʐɯ˧-tsʰɯ˩ɻ̍˩}}}}{}
\textcolor{teal}{\zh{名词}} \hspace{4pt} \zh{声调类:} LM-L.
\zh{男性名字。} \textcolor{Sepia}{\selectlanguage{english}Masculine given name.} \textcolor{PineGreen}{\selectlanguage{french}Prénom masculin.} 
\lhead{\firstmark}
\rhead{\botmark}

\subsection{\hspace{-0.5cm} {\Large \textcolor{darkblue}{\textbf{\ipa{ɖɯ˩hĩ˩}}}}\hspace{0.5cm}[\kern2pt{\textcolor{darkblue}{\textbf{\ipa{ɖɯ˩hĩ˩˥}}}}\kern2pt]} \hypertarget{d`M\string_Bhi\string_~\string_B1}{}
\markboth{\textcolor{darkblue}{\textbf{\ipa{ɖɯ˩hĩ˩}}}}{}
\textcolor{teal}{\zh{名词}} \hspace{4pt} \zh{声调类:} L.
\zh{大人、重要的人(包括长辈)。} \textcolor{Sepia}{\selectlanguage{english}Important people (including elders).} \textcolor{PineGreen}{\selectlanguage{french}Gens importants.}  \zh{量词}: \textcolor{darkblue}{\textbf{\ipa{v̩˧}}} 
\lhead{\firstmark}
\rhead{\botmark}

\subsection{\hspace{-0.5cm} {\Large \textcolor{darkblue}{\textbf{\ipa{ɖɯ˩lo\#˥}}}}\hspace{0.5cm}[\kern2pt{\textcolor{darkblue}{\textbf{\ipa{ɖɯ˩lo˥}}}}\kern2pt]} \hypertarget{d`M\string_Blo\#\string_T1}{}
\markboth{\textcolor{darkblue}{\textbf{\ipa{ɖɯ˩lo\#˥}}}}{}
\textcolor{teal}{\zh{名词}} \hspace{4pt} \zh{声调类:} LM+\#H.
\ding{202} \zh{传统。} \textcolor{Sepia}{\selectlanguage{english}Tradition.} \textcolor{PineGreen}{\selectlanguage{french}Coutume, tradition.}  ¶ \textcolor{darkblue}{\textbf{\ipa{ɖɯ˩lo˧ ɖɯ˧-kʰwɤ˥ | tʰi˧-so˥-ɻ̍˩}}} \zh{教授一个传统、一个习俗} \textcolor{Sepia}{\selectlanguage{english}to teach a custom} \textcolor{PineGreen}{\selectlanguage{french}enseigner une coutume}  
 \zh{量词}: \textcolor{darkblue}{\textbf{\ipa{kʰwɤ˥}}} \ding{203} \zh{礼仪、礼貌。} \textcolor{Sepia}{\selectlanguage{english}Good manners.} \textcolor{PineGreen}{\selectlanguage{french}Savoir-vivre.}  ¶ \textcolor{darkblue}{\textbf{\ipa{ʈʂʰɯ˧ | ɖɯ˩lo˧ dʑɤ˥!}}} \zh{他懂礼貌、他会做人} \textcolor{Sepia}{\selectlanguage{english}(S)he knows the customs / (s)he has good manners} \textcolor{PineGreen}{\selectlanguage{french}Il/elle sait vivre/connaît les coutumes/a du savoir-vivre!}  
\ding{204} \zh{道理。} \textcolor{Sepia}{\selectlanguage{english}The order of things.} \textcolor{PineGreen}{\selectlanguage{french}Ordre des choses.} 
\lhead{\firstmark}
\rhead{\botmark}

\subsection{\hspace{-0.5cm} {\Large \textcolor{darkblue}{\textbf{\ipa{ɖɯ˩mɑ\#˥}}}}\hspace{0.5cm}[\kern2pt{\textcolor{darkblue}{\textbf{\ipa{ɖɯ˩mɑ˥}}}}\kern2pt]} \hypertarget{d`M\string_BmA\#\string_T1}{}
\markboth{\textcolor{darkblue}{\textbf{\ipa{ɖɯ˩mɑ\#˥}}}}{}
\textcolor{teal}{\zh{名词}} \hspace{4pt} \zh{声调类:} LM+\#H.
\zh{女性名字。} \textcolor{Sepia}{\selectlanguage{english}Feminine given name.} \textcolor{PineGreen}{\selectlanguage{french}Prénom féminin.} 
\lhead{\firstmark}
\rhead{\botmark}

\subsection{\hspace{-0.5cm} {\Large \textcolor{darkblue}{\textbf{\ipa{ɖɯ˩mɑ˧-ɬɑ˩tsʰo˩}}}}\hspace{0.5cm}[\kern2pt{\textcolor{darkblue}{\textbf{\ipa{ɖɯ˩mɑ˧ɬɑ˩tsʰo˩}}}}\kern2pt]} \hypertarget{d`M\string_BmA\string_M-KA\string_Bts\string_ho\string_B1}{}
\markboth{\textcolor{darkblue}{\textbf{\ipa{ɖɯ˩mɑ˧-ɬɑ˩tsʰo˩}}}}{}
\textcolor{teal}{\zh{名词}} \hspace{4pt} \zh{声调类:} LM-L.
\zh{女性名字。} \textcolor{Sepia}{\selectlanguage{english}Feminine given name.} \textcolor{PineGreen}{\selectlanguage{french}Prénom féminin.} 
\lhead{\firstmark}
\rhead{\botmark}

\subsection{\hspace{-0.5cm} {\Large \textcolor{darkblue}{\textbf{\ipa{ɖɯ˩mɑ˧-pv̩˩ʈʰɯ˩}}}}\hspace{0.5cm}[\kern2pt{\textcolor{darkblue}{\textbf{\ipa{ɖɯ˩mɑ˧pv̩˩ʈʰɯ˩}}}}\kern2pt]} \hypertarget{d`M\string_BmA\string_M-pv\string_=\string_Bt`\string_hM\string_B1}{}
\markboth{\textcolor{darkblue}{\textbf{\ipa{ɖɯ˩mɑ˧-pv̩˩ʈʰɯ˩}}}}{}
\textcolor{teal}{\zh{名词}} \hspace{4pt} \zh{声调类:} LM-L.
\zh{女性名字。} \textcolor{Sepia}{\selectlanguage{english}Feminine given name.} \textcolor{PineGreen}{\selectlanguage{french}Prénom féminin.} 
\lhead{\firstmark}
\rhead{\botmark}

\subsection{\hspace{-0.5cm} {\Large \textcolor{darkblue}{\textbf{\ipa{ɖɯ˩mi\#˥}}}}\hspace{0.5cm}[\kern2pt{\textcolor{darkblue}{\textbf{\ipa{ɖɯ˩mi˥}}}}\kern2pt]} \hypertarget{d`M\string_Bmi\#\string_T1}{}
\markboth{\textcolor{darkblue}{\textbf{\ipa{ɖɯ˩mi\#˥}}}}{}
\textcolor{teal}{\zh{名词}} \hspace{4pt} \zh{声调类:} LM+\#H.
\zh{母骡子、母马骡。} \textcolor{Sepia}{\selectlanguage{english}Female mule. This is a sterile animal. It is docile, and suitable for tasks such as leading a caravan. It is therefore a highly valued animal.} \textcolor{PineGreen}{\selectlanguage{french}Mule femelle. C'est un animal stérile. Il est docile et fort, et peut accomplir des tâches importantes comme d'être l'animal de tête dans une caravane. C'est donc un animal de grand prix.}  ¶ \textcolor{darkblue}{\textbf{\ipa{ɖɯ˩mi˧-ɖɯ˥zo˩}}} \zh{母骡子与公骡子} \textcolor{Sepia}{\selectlanguage{english}Female mule and male mule} \textcolor{PineGreen}{\selectlanguage{french}mule femelle et mule mâle}  
 \zh{量词}: \textcolor{darkblue}{\textbf{\ipa{mi˩}}} 
\lhead{\firstmark}
\rhead{\botmark}

\subsection{\hspace{-0.5cm} {\Large \textcolor{darkblue}{\textbf{\ipa{ɖɯ˩zo\#˥}}}}\hspace{0.5cm}[\kern2pt{\textcolor{darkblue}{\textbf{\ipa{ɖɯ˩zo˥}}}}\kern2pt]} \hypertarget{d`M\string_Bzo\#\string_T1}{}
\markboth{\textcolor{darkblue}{\textbf{\ipa{ɖɯ˩zo\#˥}}}}{}
\textcolor{teal}{\zh{名词}} \hspace{4pt} \zh{声调类:} LM+\#H.
\zh{公骡子。} \textcolor{Sepia}{\selectlanguage{english}Male mule.} \textcolor{PineGreen}{\selectlanguage{french}Mule mâle (animal moins prisé que la femelle).}  ¶ \textcolor{darkblue}{\textbf{\ipa{ɖɯ˩zo˧-ɖɯ˥mi˩}}} \zh{公骡子与母骡子} \textcolor{Sepia}{\selectlanguage{english}male mule and female mule} \textcolor{PineGreen}{\selectlanguage{french}mule mâle et mule femelle}  
 \zh{量词}: \textcolor{darkblue}{\textbf{\ipa{ɭɯ˧}}} 
\lhead{\firstmark}
\rhead{\botmark}

\subsection{\hspace{-0.5cm} {\Large \textcolor{darkblue}{\textbf{\ipa{ɖɯ˧˥}}}}\hspace{0.5cm}[\kern2pt{\textcolor{darkblue}{\textbf{\ipa{ɖɯ˧˥}}}}\kern2pt]} \hypertarget{d`M\string_M\string_T1}{}
\markboth{\textcolor{darkblue}{\textbf{\ipa{ɖɯ˧˥}}}}{}
\textcolor{teal}{\zh{数词}} \hspace{4pt} \zh{声调类:} MH.
\zh{1。} \textcolor{Sepia}{\selectlanguage{english}1.} \textcolor{PineGreen}{\selectlanguage{french}1.} 
\lhead{\firstmark}
\rhead{\botmark}

\subsection{\hspace{-0.5cm} {\Large \textcolor{darkblue}{\textbf{\ipa{ɖv̩˩}}}}\hspace{0.5cm}[\kern2pt{\textcolor{darkblue}{\textbf{\ipa{ɖv̩˥}}}}\kern2pt]} \hypertarget{d`v\string_=\string_B1}{}
\markboth{\textcolor{darkblue}{\textbf{\ipa{ɖv̩˩}}}}{}
\textcolor{teal}{\zh{名词}} \hspace{4pt} \zh{声调类:} L.
\zh{翅膀。} \textcolor{Sepia}{\selectlanguage{english}Wing (monosyllabic form; the disyllabic form is preferred).} \textcolor{PineGreen}{\selectlanguage{french}Ailes (forme monosyllabique; la forme disyllabique est préférée).}  \zh{量词}: \textcolor{darkblue}{\textbf{\ipa{dze˩}}} 
\lhead{\firstmark}
\rhead{\botmark}

\subsection{\hspace{-0.5cm} {\Large \textcolor{darkblue}{\textbf{\ipa{ɖv̩˧qæ˧}}}}\hspace{0.5cm}[\kern2pt{\textcolor{darkblue}{\textbf{\ipa{ɖv̩˩qæ˥}}}}\kern2pt]} \hypertarget{d`v\string_=\string_Mq\{\string_M1}{}
\markboth{\textcolor{darkblue}{\textbf{\ipa{ɖv̩˧qæ˧}}}}{}
\textcolor{teal}{\zh{名词}} \hspace{4pt} \zh{声调类:} M.
\zh{翅膀。} \textcolor{Sepia}{\selectlanguage{english}Wing.} \textcolor{PineGreen}{\selectlanguage{french}Ailes.}  ¶ \textcolor{darkblue}{\textbf{\ipa{kɤ˩nɑ˧mi˧-ɖv̩˧qæ˥}}} \zh{老鹰翅膀} \textcolor{Sepia}{\selectlanguage{english}eagle wings} \textcolor{PineGreen}{\selectlanguage{french}aile d'aigle}  
 \zh{量词}: \textcolor{darkblue}{\textbf{\ipa{dze˩}}} 
\lhead{\firstmark}
\rhead{\botmark}

\subsection{\hspace{-0.5cm} {\Large \textcolor{darkblue}{\textbf{\ipa{ɖwæ˥}}}}\hspace{0.5cm}[\kern2pt{\textcolor{darkblue}{\textbf{\ipa{ɖwæ˥}}}}\kern2pt]} \hypertarget{d`w\{\string_T1}{}
\markboth{\textcolor{darkblue}{\textbf{\ipa{ɖwæ˥}}}}{}
\textcolor{teal}{\zh{名词}} \hspace{4pt} \zh{声调类:} \#H.
\ding{202} \zh{池塘。} \textcolor{Sepia}{\selectlanguage{english}Pond.} \textcolor{PineGreen}{\selectlanguage{french}Mare.}  ¶ \textcolor{darkblue}{\textbf{\ipa{[F5] ɖwæ˩ɬo˩mi˧}}} \zh{大池塘} \textcolor{Sepia}{\selectlanguage{english}large pond} \textcolor{PineGreen}{\selectlanguage{french}grand étang}  
 \zh{量词}: \textcolor{darkblue}{\textbf{\ipa{ɭɯ˧}}} \ding{203} \zh{水坑。} \textcolor{Sepia}{\selectlanguage{english}Pool (artificial).} \textcolor{PineGreen}{\selectlanguage{french}Réserve d'eau (artificielle).} 
\lhead{\firstmark}
\rhead{\botmark}

\subsection{\hspace{-0.5cm} {\Large \textcolor{darkblue}{\textbf{\ipa{ɖwæ˥}}}}\hspace{0.5cm}[\kern2pt{\textcolor{darkblue}{\textbf{\ipa{ɖwæ˧˥}}}}\kern2pt]} \hypertarget{d`w\{\string_T1}{}
\markboth{\textcolor{darkblue}{\textbf{\ipa{ɖwæ˥}}}}{}
\textcolor{teal}{\zh{形容词}} \hspace{4pt} \zh{声调类:} H.
\zh{浑浊 (水)。} \textcolor{Sepia}{\selectlanguage{english}Muddy, turbid.} \textcolor{PineGreen}{\selectlanguage{french}Trouble (le même terme est employé pour l'eau vive et pour l'eau stagnante).}  ¶ \textcolor{darkblue}{\textbf{\ipa{dʑɯ˧ ɖwæ\#˥}}} \zh{浑浊的水} \textcolor{Sepia}{\selectlanguage{english}turbid water} \textcolor{PineGreen}{\selectlanguage{french}eau trouble}  
 ¶ \textcolor{darkblue}{\textbf{\ipa{dʑɯ˧ | ɖwæ˧-ze˩!}}} \zh{水浑浊了。} \textcolor{Sepia}{\selectlanguage{english}The water has become turbid.} \textcolor{PineGreen}{\selectlanguage{french}l'eau s'est troublée! l'eau est devenue trouble!}  

\lhead{\firstmark}
\rhead{\botmark}

\subsection{\hspace{-0.5cm} {\Large \textcolor{darkblue}{\textbf{\ipa{ɖwæ˥\textsubscript{a}}}}}\hspace{0.5cm}[\kern2pt{\textcolor{darkblue}{\textbf{\ipa{ɖwæ˩˥}}}}\kern2pt]} \hypertarget{d`w\{\string_Ta1}{}
\markboth{\textcolor{darkblue}{\textbf{\ipa{ɖwæ˥\textsubscript{a}}}}}{}
\textcolor{teal}{\zh{量词}} \hspace{4pt} \zh{声调类:} H\textsubscript{a}.
\zh{量词:梯级、楼梯(一节)。} \textcolor{Sepia}{\selectlanguage{english}Classifier for steps (of stairs).} \textcolor{PineGreen}{\selectlanguage{french}Classificateur des marches d'escalier.}  ¶ \textcolor{darkblue}{\textbf{\ipa{ɖɯ˧-ɖwæ˧ ɲi˥}}} \zh{是一节/一节阶梯。(引出这句是为了了解这个词在不同语境的声调变化。)} \textcolor{Sepia}{\selectlanguage{english}It's a step (of stairs). (Elicited to investigate the word's tonal behaviour)} \textcolor{PineGreen}{\selectlanguage{french}c'est une marche}  
 ¶ \textcolor{darkblue}{\textbf{\ipa{ʈʂʰɯ˧-ɖwæ\#˥}}} \zh{这节阶梯} \textcolor{Sepia}{\selectlanguage{english}this step} \textcolor{PineGreen}{\selectlanguage{french}cette marche}  

\lhead{\firstmark}
\rhead{\botmark}

\subsection{\hspace{-0.5cm} {\Large \textcolor{darkblue}{\textbf{\ipa{ɖwæ˧-pɤ˧ɭɯ˥}}}}\hspace{0.5cm}[\kern2pt{\textcolor{darkblue}{\textbf{\ipa{xxxx non-correspondance entre le nombre de morphèmes et le nombre de tons de morphèmes}}}}\kern2pt]} \hypertarget{d`w\{\string_M-p7\string_Ml\string_RM\string_T1}{}
\markboth{\textcolor{darkblue}{\textbf{\ipa{ɖwæ˧-pɤ˧ɭɯ˥}}}}{}
\textcolor{teal}{\zh{名词}} \hspace{4pt} \zh{声调类:} H\#.
\zh{水潭。} \textcolor{Sepia}{\selectlanguage{english}Puddle, pool (natural).} \textcolor{PineGreen}{\selectlanguage{french}Flaque (naturelle).}  ¶ \textcolor{darkblue}{\textbf{\ipa{ɖwæ˧ tʰi˧-pɤ˥ɭɯ˩}}} \zh{有水潭} \textcolor{Sepia}{\selectlanguage{english}there is water in the pool; a puddle has formed} \textcolor{PineGreen}{\selectlanguage{french}une flaque/petite mare s'est formée, il y a une flaque}  
 \zh{量词}: \textcolor{darkblue}{\textbf{\ipa{ɭɯ˧}}} 
\lhead{\firstmark}
\rhead{\botmark}

\subsection{\hspace{-0.5cm} {\Large \textcolor{darkblue}{\textbf{\ipa{ɖwæ˩\textsubscript{a}}}}}\hspace{0.5cm}[\kern2pt{\textcolor{darkblue}{\textbf{\ipa{ɖwæ˥}}}}\kern2pt]} \hypertarget{d`w\{\string_Ba1}{}
\markboth{\textcolor{darkblue}{\textbf{\ipa{ɖwæ˩\textsubscript{a}}}}}{}
\textcolor{teal}{\zh{动词}} \hspace{4pt} \zh{声调类:} L\textsubscript{a}.
\zh{害怕。} \textcolor{Sepia}{\selectlanguage{english}To be afraid.} \textcolor{PineGreen}{\selectlanguage{french}Avoir peur.}  ¶ \textcolor{darkblue}{\textbf{\ipa{njɤ˧ | ɖwæ˩˥!}}} \zh{我害怕!} \textcolor{Sepia}{\selectlanguage{english}I am afraid!} \textcolor{PineGreen}{\selectlanguage{french}J'ai peur!}  
 ¶ \textcolor{darkblue}{\textbf{\ipa{njɤ˧ | ʈʂʰɯ˧-v̩˧ | ɖwæ˩˥ | ʐwæ˩˥!}}} \zh{我很害怕那个人!} \textcolor{Sepia}{\selectlanguage{english}I am really afraid of this person!} \textcolor{PineGreen}{\selectlanguage{french}J'ai très peur de lui!}  

\lhead{\firstmark}
\rhead{\botmark}

\subsection{\hspace{-0.5cm} {\Large \textcolor{darkblue}{\textbf{\ipa{ɖwæ˧˥}}} \textsubscript{1}}\hspace{0.5cm}[\kern2pt{\textcolor{darkblue}{\textbf{\ipa{ɖwæ˩˥}}}}\kern2pt]} \hypertarget{d`w\{\string_M\string_T1}{}
\markboth{\textcolor{darkblue}{\textbf{\ipa{ɖwæ˧˥}}} \textsubscript{1}}{}
\textcolor{teal}{\zh{动词}} \hspace{4pt} \zh{声调类:} MH.
\zh{鞭打、抽打、加鞭。} \textcolor{Sepia}{\selectlanguage{english}To whip.} \textcolor{PineGreen}{\selectlanguage{french}Fouetter, donner des coups (ex.: un tigre fouette le sol avec sa queue).}  ¶ \textcolor{darkblue}{\textbf{\ipa{mæ˧qv̩˩-po˩-ɳɯ˩ | ɖwæ˧˥}}} \zh{用尾巴来抽打(如:老虎用尾巴来抽打地面)} \textcolor{Sepia}{\selectlanguage{english}to whip with the tail (e.g. a tiger whips the ground with its tail)} \textcolor{PineGreen}{\selectlanguage{french}donner des coups de queue (ex.: le tigre fouette le sol de sa queue)}  

\lhead{\firstmark}
\rhead{\botmark}

\subsection{\hspace{-0.5cm} {\Large \textcolor{darkblue}{\textbf{\ipa{ɖwæ˧˥}}} \textsubscript{2}}\hspace{0.5cm}[\kern2pt{\textcolor{darkblue}{\textbf{\ipa{ɖwæ˧˥}}}}\kern2pt]} \hypertarget{d`w\{\string_M\string_T2}{}
\markboth{\textcolor{darkblue}{\textbf{\ipa{ɖwæ˧˥}}} \textsubscript{2}}{}
\textcolor{teal}{\zh{助词}} \hspace{4pt} \zh{声调类:} MH.
\zh{很、极。} \textcolor{Sepia}{\selectlanguage{english}Intensive: very.} \textcolor{PineGreen}{\selectlanguage{french}Intensif: très.}  ¶ \textcolor{darkblue}{\textbf{\ipa{ʈʂʰɯ˧ | ɖwæ˧˥ | æ˧mv̩˩ fv̩˩!}}} \zh{她很喜欢她姐姐!} \textcolor{Sepia}{\selectlanguage{english}She likes her elder sister very much!} \textcolor{PineGreen}{\selectlanguage{french}elle aime beaucoup sa grande sœur!}  

\lhead{\firstmark}
\rhead{\botmark}

\newpage
\section*{\centering- \textcolor{darkblue}{\textbf{\ipa{ɖʐ}}} -}
\subsection{\hspace{-0.5cm} {\Large \textcolor{darkblue}{\textbf{\ipa{ɖʐæ˧\textsubscript{b}}}}}\hspace{0.5cm}[\kern2pt{\textcolor{darkblue}{\textbf{\ipa{ɖʐæ˩˥}}}}\kern2pt]} \hypertarget{d`z`\{\string_Mb1}{}
\markboth{\textcolor{darkblue}{\textbf{\ipa{ɖʐæ˧\textsubscript{b}}}}}{}
\textcolor{teal}{\zh{动词}} \hspace{4pt} \zh{声调类:} M\textsubscript{b}.
\zh{骑马。} \textcolor{Sepia}{\selectlanguage{english}To ride (a horse).} \textcolor{PineGreen}{\selectlanguage{french}Monter à cheval.}  ¶ \textcolor{darkblue}{\textbf{\ipa{le˧-ɖʐæ˧-ze˧}}} \zh{\mytextsc{accomp} \string_ \mytextsc{pfv}} \textcolor{Sepia}{\selectlanguage{english}\mytextsc{accomp} \string_ \mytextsc{pfv}} \textcolor{PineGreen}{\selectlanguage{french}\mytextsc{accomp} \string_ \mytextsc{pfv}}  
 ¶ \textcolor{darkblue}{\textbf{\ipa{ʐwæ˧ ɖʐæ˧}}} \zh{骑马} \textcolor{Sepia}{\selectlanguage{english}to ride a horse} \textcolor{PineGreen}{\selectlanguage{french}monter à cheval}  
 ¶ \textcolor{darkblue}{\textbf{\ipa{ɖʐæ˧-tʰɑ˧˥!}}} \zh{可以骑的!} \textcolor{Sepia}{\selectlanguage{english}It's possible to ride (it)! / It's OK to ride (it)!} \textcolor{PineGreen}{\selectlanguage{french}On peut le monter!}  

\lhead{\firstmark}
\rhead{\botmark}

\subsection{\hspace{-0.5cm} {\Large \textcolor{darkblue}{\textbf{\ipa{ɖʐæ˧qʰæ˥\$}}}}\hspace{0.5cm}[\kern2pt{\textcolor{darkblue}{\textbf{\ipa{ɖʐæ˩qʰæ˩˥}}}}\kern2pt]} \hypertarget{d`z`\{\string_Mq\string_h\{\string_T\$1}{}
\markboth{\textcolor{darkblue}{\textbf{\ipa{ɖʐæ˧qʰæ˥\$}}}}{}
\textcolor{teal}{\zh{名词}} \hspace{4pt} \zh{声调类:} H\$.
\zh{泥巴。} \textcolor{Sepia}{\selectlanguage{english}Mud.} \textcolor{PineGreen}{\selectlanguage{french}Boue.}  ¶ \textcolor{darkblue}{\textbf{\ipa{ɖʐæ˧qʰæ˧ ʐæ˥(-ze˩)}}} \zh{有泥巴了。} \textcolor{Sepia}{\selectlanguage{english}There is mud; mud has formed.} \textcolor{PineGreen}{\selectlanguage{french}De la boue s'est formée; il y a de la boue, c'est tout boueux. (Littéralement: “de la boue s'est mélangée”.)}  
 ¶ \textcolor{darkblue}{\textbf{\ipa{[F5] ɖʐæ˧qʰæ˧ ʐæ˧\textasciitilde{}ʐæ˥}}} \zh{有泥巴了} \textcolor{Sepia}{\selectlanguage{english}There is mud; mud has formed.} \textcolor{PineGreen}{\selectlanguage{french}De la boue s'est formée; il y a de la boue, c'est tout boueux. (Littéralement: “de la boue s'est mélangée”.)}  

\lhead{\firstmark}
\rhead{\botmark}

\subsection{\hspace{-0.5cm} {\Large \textcolor{darkblue}{\textbf{\ipa{ɖʐæ˩\textsubscript{a}}}}}\hspace{0.5cm}[\kern2pt{\textcolor{darkblue}{\textbf{\ipa{ɖʐæ˥}}}}\kern2pt]} \hypertarget{d`z`\{\string_Ba1}{}
\markboth{\textcolor{darkblue}{\textbf{\ipa{ɖʐæ˩\textsubscript{a}}}}}{}
\textcolor{teal}{\zh{动词}} \hspace{4pt} \zh{声调类:} L\textsubscript{a}.
\zh{融化。} \textcolor{Sepia}{\selectlanguage{english}To melt; to thaw.} \textcolor{PineGreen}{\selectlanguage{french}Fondre.}  ¶ \textcolor{darkblue}{\textbf{\ipa{mɤ˧ | le˧-ɖʐæ˩-ze˩}}} \zh{油融化了。} \textcolor{Sepia}{\selectlanguage{english}The grease has melted.} \textcolor{PineGreen}{\selectlanguage{french}la graisse a fondu (ex.: du saindoux qui fond dans un chaudron)}  
 ¶ \textcolor{darkblue}{\textbf{\ipa{dʑi˩pʰæ˩˥ | le˧-ɖʐæ˩-ze˩}}} \zh{冰融化了。} \textcolor{Sepia}{\selectlanguage{english}The ice has melted.} \textcolor{PineGreen}{\selectlanguage{french}La glace a fondu.}  

\lhead{\firstmark}
\rhead{\botmark}

\subsection{\hspace{-0.5cm} {\Large \textcolor{darkblue}{\textbf{\ipa{ɖʐæ˩bv˩}}}}\hspace{0.5cm}[\kern2pt{\textcolor{darkblue}{\textbf{\ipa{ɖʐæ˧bv˧}}}}\kern2pt]} \hypertarget{d`z`\{\string_Bbv\string_B1}{}
\markboth{\textcolor{darkblue}{\textbf{\ipa{ɖʐæ˩bv˩}}}}{}
\textcolor{teal}{\zh{名词}} \hspace{4pt} \zh{声调类:} L.
\zh{法师。} \textcolor{Sepia}{\selectlanguage{english}Sorcerer.} \textcolor{PineGreen}{\selectlanguage{french}Sorcier.}  ¶ \textcolor{darkblue}{\textbf{\ipa{ə˧pʰv˧-ɖʐæ˩bv˩}}} \zh{‘法师爷爷’:对年龄高(或被认为本事很大)的法师的尊重称呼} \textcolor{Sepia}{\selectlanguage{english}'Grandfather sorcerer': a respectful term of address for a sorcerer who is advanced in years or considered to have great powers} \textcolor{PineGreen}{\selectlanguage{french}'Grand-père sorcier': terme d'adresse respectueux pour un sorcier d'âge avancé, ou considéré comme ayant des pouvoirs considérables}  
 ¶ \textcolor{darkblue}{\textbf{\ipa{ə˧v˧-ɖʐæ˥bv˩}}} \zh{‘法师舅舅’:对法师的尊重称呼} \textcolor{Sepia}{\selectlanguage{english}'Uncle sorcerer': a respectful term of address for a sorcerer} \textcolor{PineGreen}{\selectlanguage{french}'Oncle sorcier': terme d'adresse respectueux pour un sorcier}  
 \zh{量词}: \textcolor{darkblue}{\textbf{\ipa{v̩˧}}} 
\lhead{\firstmark}
\rhead{\botmark}

\subsection{\hspace{-0.5cm} {\Large \textcolor{darkblue}{\textbf{\ipa{ɖʐe˧}}}}\hspace{0.5cm}[\kern2pt{\textcolor{darkblue}{\textbf{\ipa{ɖʐe˥}}}}\kern2pt]} \hypertarget{d`z`e\string_M1}{}
\markboth{\textcolor{darkblue}{\textbf{\ipa{ɖʐe˧}}}}{}
\textcolor{teal}{\zh{名词}} \hspace{4pt} \zh{声调类:} M.
\zh{钱。} \textcolor{Sepia}{\selectlanguage{english}Money.} \textcolor{PineGreen}{\selectlanguage{french}Argent (avoir de l'argent).} 
\lhead{\firstmark}
\rhead{\botmark}

\subsection{\hspace{-0.5cm} {\Large \textcolor{darkblue}{\textbf{\ipa{ɖʐe˧gɯ˧}}}}\hspace{0.5cm}[\kern2pt{\textcolor{darkblue}{\textbf{\ipa{ɖʐe˩gɯ˧˥}}}}\kern2pt]} \hypertarget{d`z`e\string_MgM\string_M1}{}
\markboth{\textcolor{darkblue}{\textbf{\ipa{ɖʐe˧gɯ˧}}}}{}
\textcolor{teal}{\zh{名词}} \hspace{4pt} \zh{声调类:} M.
\zh{永胜(地名)。} \textcolor{Sepia}{\selectlanguage{english}Yongsheng (place name).} \textcolor{PineGreen}{\selectlanguage{french}Yongsheng (nom de comté).}  ¶ \textcolor{darkblue}{\textbf{\ipa{ɖʐe˧gɯ˧-to˩mi˩}}} \zh{永胜的一座高山} \textcolor{Sepia}{\selectlanguage{english}a high mountain located in Yongsheng} \textcolor{PineGreen}{\selectlanguage{french}une haute montagne située à Yongsheng}  
 ¶ \textcolor{darkblue}{\textbf{\ipa{ɖʐe˧gɯ˧-hæ˧}}} \zh{永胜汉族} \textcolor{Sepia}{\selectlanguage{english}Yongsheng Chinese (Han) (note: Yongsheng is mainly populated by Han Chinese)} \textcolor{PineGreen}{\selectlanguage{french}Chinois de Yongsheng (note: le comté de Yongsheng était peuplé majoritairement de Chinois (Han).)}  
 ¶ \textcolor{darkblue}{\textbf{\ipa{ɖʐe˧gɯ˧-dʑo˧, | hæ˧-ʂo˧\textasciitilde{}ʂo˩!}}} \zh{永胜,汉族群多!} \textcolor{Sepia}{\selectlanguage{english}In Yongsheng, there are lots of Chinese (Han) people!} \textcolor{PineGreen}{\selectlanguage{french}A Yongsheng, il y a plein de Chinois (Han)!}  

\lhead{\firstmark}
\rhead{\botmark}

\subsection{\hspace{-0.5cm} {\Large \textcolor{darkblue}{\textbf{\ipa{ɖʐe˧ʁwɤ˧}}}}\hspace{0.5cm}[\kern2pt{\textcolor{darkblue}{\textbf{\ipa{xxxx non-correspondance entre le nombre de morphèmes et le nombre de tons de morphèmes}}}}\kern2pt]} \hypertarget{d`z`e\string_MRw7\string_M1}{}
\markboth{\textcolor{darkblue}{\textbf{\ipa{ɖʐe˧ʁwɤ˧}}}}{}
\textcolor{teal}{\zh{名词}} \hspace{4pt} \zh{声调类:} M.
\zh{钱。} \textcolor{Sepia}{\selectlanguage{english}Money, wealth.} \textcolor{PineGreen}{\selectlanguage{french}Argent (monnaie); richesse.} 
\lhead{\firstmark}
\rhead{\botmark}

\subsection{\hspace{-0.5cm} {\Large \textcolor{darkblue}{\textbf{\ipa{ɖʐɤ˧qʰwɤ˧}}}}\hspace{0.5cm}[\kern2pt{\textcolor{darkblue}{\textbf{\ipa{ɖʐɤ˩qʰwɤ˥}}}}\kern2pt]} \hypertarget{d`z`7\string_Mq\string_hw7\string_M1}{}
\markboth{\textcolor{darkblue}{\textbf{\ipa{ɖʐɤ˧qʰwɤ˧}}}}{}
\textcolor{teal}{\zh{名词}} \hspace{4pt} \zh{声调类:} M.
\zh{感冒。} \textcolor{Sepia}{\selectlanguage{english}Cold, flu.} \textcolor{PineGreen}{\selectlanguage{french}Rhume.}  ¶ \textcolor{darkblue}{\textbf{\ipa{ɖʐɤ˧qʰwɤ˧ go˩}}} \zh{感冒} \textcolor{Sepia}{\selectlanguage{english}to have a cold; to have a flu} \textcolor{PineGreen}{\selectlanguage{french}avoir un rhume, être enrhumé}  
 ¶ \textcolor{darkblue}{\textbf{\ipa{ɖʐɤ˧qʰwɤ˧ mɤ˧-go˩}}} \zh{没感冒} \textcolor{Sepia}{\selectlanguage{english}...has no cold / does not have a cold} \textcolor{PineGreen}{\selectlanguage{french}...n'est pas enrhumé}  
 \zh{量词}: \textcolor{darkblue}{\textbf{\ipa{ʂɯ˩}}} 
\lhead{\firstmark}
\rhead{\botmark}

\subsection{\hspace{-0.5cm} {\Large \textcolor{darkblue}{\textbf{\ipa{ɖʐɤ˧qʰwɤ˧ʈʂe\#˥}}}}\hspace{0.5cm}[\kern2pt{\textcolor{darkblue}{\textbf{\ipa{ɖʐɤ˧qʰwɤ˧ʈʂe˧}}}}\kern2pt]} \hypertarget{d`z`7\string_Mq\string_hw7\string_Mt`s`e\#\string_T1}{}
\markboth{\textcolor{darkblue}{\textbf{\ipa{ɖʐɤ˧qʰwɤ˧ʈʂe\#˥}}}}{}
\textcolor{teal}{\zh{名词}} \hspace{4pt} \zh{声调类:} \#H.
\zh{锥、锥子。} \textcolor{Sepia}{\selectlanguage{english}Awl.} \textcolor{PineGreen}{\selectlanguage{french}Poinçon, alène.}  \zh{量词}: \textcolor{darkblue}{\textbf{\ipa{ɭɯ˧}}} 
\lhead{\firstmark}
\rhead{\botmark}

\subsection{\hspace{-0.5cm} {\Large \textcolor{darkblue}{\textbf{\ipa{ɖʐɤ˩}}}}\hspace{0.5cm}[\kern2pt{\textcolor{darkblue}{\textbf{\ipa{ɖʐɤ˥}}}}\kern2pt]} \hypertarget{d`z`7\string_B1}{}
\markboth{\textcolor{darkblue}{\textbf{\ipa{ɖʐɤ˩}}}}{}
\textcolor{teal}{\zh{名词}} \hspace{4pt} \zh{声调类:} L.
\ding{202} \zh{梯子。} \textcolor{Sepia}{\selectlanguage{english}Ladder.} \textcolor{PineGreen}{\selectlanguage{french}Échelle.}  ¶ \textcolor{darkblue}{\textbf{\ipa{ɖʐɤ˩ do˧}}} \zh{爬上一个梯子} \textcolor{Sepia}{\selectlanguage{english}to climb a ladder} \textcolor{PineGreen}{\selectlanguage{french}gravir une échelle}  
 ¶ \textcolor{darkblue}{\textbf{\ipa{ɖʐɤ˧ | gɤ˩-do˧}}} \zh{爬上一个梯子} \textcolor{Sepia}{\selectlanguage{english}to climb up a ladder} \textcolor{PineGreen}{\selectlanguage{french}même sens, avec ajout d'un directionnel: gravir une échelle}  
 \zh{量词}: \textcolor{darkblue}{\textbf{\ipa{pɤ˩}}} \ding{203} \zh{楼梯。} \textcolor{Sepia}{\selectlanguage{english}Stairs.} \textcolor{PineGreen}{\selectlanguage{french}Escalier (en bois, sauf indication contraire).}  ¶ \textcolor{darkblue}{\textbf{\ipa{lv̩˧mi˧-ɖʐɤ˩ (+ɲi˩)}}} \zh{石头楼梯} \textcolor{Sepia}{\selectlanguage{english}stone stairs} \textcolor{PineGreen}{\selectlanguage{french}escalier en pierre}  

\lhead{\firstmark}
\rhead{\botmark}

\subsection{\hspace{-0.5cm} {\Large \textcolor{darkblue}{\textbf{\ipa{ɖʐɤ˩ɖwæ˩}}}}\hspace{0.5cm}[\kern2pt{\textcolor{darkblue}{\textbf{\ipa{ɖʐɤ˧ɖwæ˩}}}}\kern2pt]} \hypertarget{d`z`7\string_Bd`w\{\string_B1}{}
\markboth{\textcolor{darkblue}{\textbf{\ipa{ɖʐɤ˩ɖwæ˩}}}}{}
\textcolor{teal}{\zh{名词}} \hspace{4pt} \zh{声调类:} L.
\zh{台阶。} \textcolor{Sepia}{\selectlanguage{english}Step of stairs.} \textcolor{PineGreen}{\selectlanguage{french}Marche d'escalier.}  ¶ \textcolor{darkblue}{\textbf{\ipa{lv̩˧mi˧-ɖʐɤ˩ɖwæ˩}}} \zh{石头台阶} \textcolor{Sepia}{\selectlanguage{english}stone step} \textcolor{PineGreen}{\selectlanguage{french}marche en pierre}  
 \zh{量词}: \textcolor{darkblue}{\textbf{\ipa{ɖwæ˥}}} 
\lhead{\firstmark}
\rhead{\botmark}

\subsection{\hspace{-0.5cm} {\Large \textcolor{darkblue}{\textbf{\ipa{ɖʐɤ˩kɤ˥\$}}}}\hspace{0.5cm}[\kern2pt{\textcolor{darkblue}{\textbf{\ipa{ɖʐɤ˩kɤ˩˥}}}}\kern2pt]} \hypertarget{d`z`7\string_Bk7\string_T\$1}{}
\markboth{\textcolor{darkblue}{\textbf{\ipa{ɖʐɤ˩kɤ˥\$}}}}{}
\textcolor{teal}{\zh{名词}} \hspace{4pt} \zh{声调类:} LM+H\$.
\zh{一个姓。这个姓,永宁有两家。} \textcolor{Sepia}{\selectlanguage{english}A family name from Yongning. There are two families in Yongning that carry this name.} \textcolor{PineGreen}{\selectlanguage{french}Nom de clan/famille étendue. Deux familles portent ce nom à Yongning.}  ¶ \textcolor{darkblue}{\textbf{\ipa{ɖʐɤ˩kɤ˧=ɻ̍˥\$}}} \zh{\textcolor{darkblue}{\textbf{\ipa{/ɖʐɤ˩kɤ˥\$/}}}家族} \textcolor{Sepia}{\selectlanguage{english}the \textcolor{darkblue}{\textbf{\ipa{/ɖʐɤ˩kɤ˥\$/}}} clan} \textcolor{PineGreen}{\selectlanguage{french}le clan \textcolor{darkblue}{\textbf{\ipa{/ɖʐɤ˩kɤ˥\$/}}}}  

\lhead{\firstmark}
\rhead{\botmark}

\subsection{\hspace{-0.5cm} {\Large \textcolor{darkblue}{\textbf{\ipa{ɖʐɤ˧˥}}} \textsubscript{1}}\hspace{0.5cm}[\kern2pt{\textcolor{darkblue}{\textbf{\ipa{ɖʐɤ˩˥}}}}\kern2pt]} \hypertarget{d`z`7\string_M\string_T1}{}
\markboth{\textcolor{darkblue}{\textbf{\ipa{ɖʐɤ˧˥}}} \textsubscript{1}}{}
\textcolor{teal}{\zh{动词}} \hspace{4pt} \zh{声调类:} MH.
\ding{202} \zh{摘(果子、蔬菜)、扯(草)。} \textcolor{Sepia}{\selectlanguage{english}To pluck (fruit, weeds), to pick (vegetables).} \textcolor{PineGreen}{\selectlanguage{french}Cueillir (des fruits, des légumes); arracher (des mauvaises herbes).}  ¶ \textcolor{darkblue}{\textbf{\ipa{le˧-ɖʐɤ˧-po˥-jo˩!}}} \zh{(你)去给摘(一些)过来吧!} \textcolor{Sepia}{\selectlanguage{english}Pluck some (fruit) and pass them over (to us)!} \textcolor{PineGreen}{\selectlanguage{french}cueille-m'en qq-unes!/cueilles-en et passe-les(-nous) par ici!}  
 ¶ \textcolor{darkblue}{\textbf{\ipa{v̩˩tsʰɤ˧ ɖʐɤ˥}}} \zh{摘蔬菜} \textcolor{Sepia}{\selectlanguage{english}to pick vegetables} \textcolor{PineGreen}{\selectlanguage{french}cueillir des légumes}  
 ¶ \textcolor{darkblue}{\textbf{\ipa{le˧-ɖʐɤ˧˥, | mv̩˩-tɕo˧ kwɤ˩}}} \zh{扯(荒草),扔掉} \textcolor{Sepia}{\selectlanguage{english}to pluck and throw away (weeds)} \textcolor{PineGreen}{\selectlanguage{french}arracher et jeter (les mauvaises herbes)}  
\ding{203} \zh{拆(线),拔,捣碎。} \textcolor{Sepia}{\selectlanguage{english}To snap, to cut (thread); to smash; to destroy (a building).} \textcolor{PineGreen}{\selectlanguage{french}Déchirer, couper (fil); briser; broyer; détruire (une maison).}  ¶ \textcolor{darkblue}{\textbf{\ipa{le˧-ɖʐɤ˩\textasciitilde{}ɖʐɤ˩}}} \zh{\mytextsc{重叠:拆拆}} \textcolor{Sepia}{\selectlanguage{english}\mytextsc{red}} \textcolor{PineGreen}{\selectlanguage{french}\mytextsc{red}}  
 ¶ \textcolor{darkblue}{\textbf{\ipa{ʑi˧qʰwɤ˧ ɖʐɤ˧˥}}} \zh{拆房子} \textcolor{Sepia}{\selectlanguage{english}to destroy a house} \textcolor{PineGreen}{\selectlanguage{french}détruire une maison, démolir une maison}  
 ¶ \textcolor{darkblue}{\textbf{\ipa{le˧-ɖʐɤ˧˥ | ɲi˧-gi˧ gv̩˧}}} \zh{拆成两半} \textcolor{Sepia}{\selectlanguage{english}to tear into two pieces} \textcolor{PineGreen}{\selectlanguage{french}déchirer en deux morceaux}  

\lhead{\firstmark}
\rhead{\botmark}

\subsection{\hspace{-0.5cm} {\Large \textcolor{darkblue}{\textbf{\ipa{ɖʐɤ˧˥}}} \textsubscript{2}}\hspace{0.5cm}[\kern2pt{\textcolor{darkblue}{\textbf{\ipa{ɖʐɤ˧˥}}}}\kern2pt]} \hypertarget{d`z`7\string_M\string_T2}{}
\markboth{\textcolor{darkblue}{\textbf{\ipa{ɖʐɤ˧˥}}} \textsubscript{2}}{}
\textcolor{teal}{\zh{动词}} \hspace{4pt} \zh{声调类:} MH.
\zh{撑开(帐篷)。} \textcolor{Sepia}{\selectlanguage{english}To prop open (a tent).} \textcolor{PineGreen}{\selectlanguage{french}Déployer, ouvrir en soutenant; ex.: déployer la tente.}  ¶ \textcolor{darkblue}{\textbf{\ipa{le˧-ɖʐɤ˩\textasciitilde{}ɖʐɤ˩}}} \zh{\mytextsc{重叠}} \textcolor{Sepia}{\selectlanguage{english}\mytextsc{red}} \textcolor{PineGreen}{\selectlanguage{french}\mytextsc{red}}  

\lhead{\firstmark}
\rhead{\botmark}

\subsection{\hspace{-0.5cm} {\Large \textcolor{darkblue}{\textbf{\ipa{ɖʐo˥}}}}\hspace{0.5cm}[\kern2pt{\textcolor{darkblue}{\textbf{\ipa{ɖʐo˥}}}}\kern2pt]} \hypertarget{d`z`o\string_T1}{}
\markboth{\textcolor{darkblue}{\textbf{\ipa{ɖʐo˥}}}}{}
\textcolor{teal}{\zh{名词}} \hspace{4pt} \zh{声调类:} \#H.
\zh{大梁。} \textcolor{Sepia}{\selectlanguage{english}Major roof beam.} \textcolor{PineGreen}{\selectlanguage{french}Pièce de charpente carrée (côté: environ 18 cm), dans les parties du bâtiment qui n'ont pas de piliers: \textcolor{darkblue}{\textbf{\ipa{/gæ˩pʰæ˧/}}}, \textcolor{darkblue}{\textbf{\ipa{/mv̩˩pʰæ˧/}}}. Elles supportent la charpente. (M18 pense que ce terme désigne toute la structure du bâtiment.).}  \zh{量词}: \textcolor{darkblue}{\textbf{\ipa{ɖʐo˥}}} 
\lhead{\firstmark}
\rhead{\botmark}

\subsection{\hspace{-0.5cm} {\Large \textcolor{darkblue}{\textbf{\ipa{ɖʐo˥}}}}\hspace{0.5cm}[\kern2pt{\textcolor{darkblue}{\textbf{\ipa{ɖʐo˥}}}}\kern2pt]} \hypertarget{d`z`o\string_T1}{}
\markboth{\textcolor{darkblue}{\textbf{\ipa{ɖʐo˥}}}}{}
\textcolor{teal}{\zh{形容词}} \hspace{4pt} \zh{声调类:} H.
\zh{冷(天气……)。} \textcolor{Sepia}{\selectlanguage{english}Cold (weather).} \textcolor{PineGreen}{\selectlanguage{french}Froid.} 
\lhead{\firstmark}
\rhead{\botmark}

\subsection{\hspace{-0.5cm} {\Large \textcolor{darkblue}{\textbf{\ipa{ɖʐo˥\textsubscript{a}}}}}\hspace{0.5cm}[\kern2pt{\textcolor{darkblue}{\textbf{\ipa{ɖʐo˩˥}}}}\kern2pt]} \hypertarget{d`z`o\string_Ta1}{}
\markboth{\textcolor{darkblue}{\textbf{\ipa{ɖʐo˥\textsubscript{a}}}}}{}
\textcolor{teal}{\zh{量词}} \hspace{4pt} \zh{声调类:} H\textsubscript{a}.
\zh{量词:梁(一根)。} \textcolor{Sepia}{\selectlanguage{english}Classifier for beams (in carpentry).} \textcolor{PineGreen}{\selectlanguage{french}Classificateur des poutres.} 
\lhead{\firstmark}
\rhead{\botmark}

\subsection{\hspace{-0.5cm} {\Large \textcolor{darkblue}{\textbf{\ipa{ɖʐo˩\textsubscript{b}}}}}\hspace{0.5cm}[\kern2pt{\textcolor{darkblue}{\textbf{\ipa{ɖʐo˩˥}}}}\kern2pt]} \hypertarget{d`z`o\string_Bb1}{}
\markboth{\textcolor{darkblue}{\textbf{\ipa{ɖʐo˩\textsubscript{b}}}}}{}
\textcolor{teal}{\zh{动词}} \hspace{4pt} \zh{声调类:} L\textsubscript{b}.
\zh{弄碎(用牙齿、手磨)。} \textcolor{Sepia}{\selectlanguage{english}To crush, to crumble (with the teeth or with a grindstone).} \textcolor{PineGreen}{\selectlanguage{french}Écraser (au moulin; ou avec les dents).}  ¶ \textcolor{darkblue}{\textbf{\ipa{ʈʂo˧ɭɯ˧ ɖʐo˧˥}}} \zh{用手磨弄碎} \textcolor{Sepia}{\selectlanguage{english}to crush with a grindstone} \textcolor{PineGreen}{\selectlanguage{french}écraser avec un moulin (/ʈʂu˧ɭɯ\#˥/: moulin)}  
 ¶ \textcolor{darkblue}{\textbf{\ipa{ɖɯ˧-kʰwɤ˧ ɖʐo˧˥}}} \zh{弄碎一块} \textcolor{Sepia}{\selectlanguage{english}to crush a piece (of something)} \textcolor{PineGreen}{\selectlanguage{french}écraser un morceau (de quelque chose)}  
 ¶ \textcolor{darkblue}{\textbf{\ipa{ɖɯ˧-mɤ˩ ɖʐo˩}}} \zh{弄碎一点(东西)} \textcolor{Sepia}{\selectlanguage{english}to crush a little (of something)} \textcolor{PineGreen}{\selectlanguage{french}écraser un peu (de quelque chose)}  

\lhead{\firstmark}
\rhead{\botmark}

\subsection{\hspace{-0.5cm} {\Large \textcolor{darkblue}{\textbf{\ipa{ɖʐɯ˥}}}}\hspace{0.5cm}[\kern2pt{\textcolor{darkblue}{\textbf{\ipa{ɖʐɯ˧˥}}}}\kern2pt]} \hypertarget{d`z`M\string_T1}{}
\markboth{\textcolor{darkblue}{\textbf{\ipa{ɖʐɯ˥}}}}{}
\textcolor{teal}{\zh{名词}} \hspace{4pt} \zh{声调类:} \#H.
\zh{集市(圩场,街子)。} \textcolor{Sepia}{\selectlanguage{english}Market.} \textcolor{PineGreen}{\selectlanguage{french}Marché.}  \zh{量词}: \textcolor{darkblue}{\textbf{\ipa{ɖʐɯ˩}}} 
\lhead{\firstmark}
\rhead{\botmark}

\subsection{\hspace{-0.5cm} {\Large \textcolor{darkblue}{\textbf{\ipa{ɖʐɯ˥\textsubscript{a}}}}}\hspace{0.5cm}[\kern2pt{\textcolor{darkblue}{\textbf{\ipa{ɖʐɯ˩˥}}}}\kern2pt]} \hypertarget{d`z`M\string_Ta1}{}
\markboth{\textcolor{darkblue}{\textbf{\ipa{ɖʐɯ˥\textsubscript{a}}}}}{}
\textcolor{teal}{\zh{量词}} \hspace{4pt} \zh{声调类:} H\textsubscript{a}.
\zh{量词:市场(一个),城市(一个)。} \textcolor{Sepia}{\selectlanguage{english}Self-classifier for marketplaces/towns.} \textcolor{PineGreen}{\selectlanguage{french}Auto-classificateur des marchés/villes.}  ¶ \textcolor{darkblue}{\textbf{\ipa{ɖʐɯ˧ | ɖɯ˧-ɖʐɯ˥}}} \zh{一个市场} \textcolor{Sepia}{\selectlanguage{english}a marketplace, a town} \textcolor{PineGreen}{\selectlanguage{french}un marché}  

\lhead{\firstmark}
\rhead{\botmark}

\subsection{\hspace{-0.5cm} {\Large \textcolor{darkblue}{\textbf{\ipa{ɖʐɯ˥kʰɤ˩}}}}\hspace{0.5cm}[\kern2pt{\textcolor{darkblue}{\textbf{\ipa{ɖʐɯ˩kʰɤ˥}}}}\kern2pt]} \hypertarget{d`z`M\string_Tk\string_h7\string_B1}{}
\markboth{\textcolor{darkblue}{\textbf{\ipa{ɖʐɯ˥kʰɤ˩}}}}{}
\textcolor{teal}{\zh{名词}} \hspace{4pt} \zh{声调类:} .
\zh{(一)会儿。} \textcolor{Sepia}{\selectlanguage{english}Moment.} \textcolor{PineGreen}{\selectlanguage{french}Un moment.}  ¶ \textcolor{darkblue}{\textbf{\ipa{ɖɯ˧-ɖʐɯ˥kʰɤ˩}}} \zh{一会儿} \textcolor{Sepia}{\selectlanguage{english}a moment} \textcolor{PineGreen}{\selectlanguage{french}un moment}  

\lhead{\firstmark}
\rhead{\botmark}

\subsection{\hspace{-0.5cm} {\Large \textcolor{darkblue}{\textbf{\ipa{ɖʐɯ˧qo˩}}}}\hspace{0.5cm}[\kern2pt{\textcolor{darkblue}{\textbf{\ipa{ɖʐɯ˧qo˩}}}}\kern2pt]} \hypertarget{d`z`M\string_Mqo\string_B1}{}
\markboth{\textcolor{darkblue}{\textbf{\ipa{ɖʐɯ˧qo˩}}}}{}
\textcolor{teal}{\zh{助词}} \hspace{4pt} \zh{声调类:} L\#.
\zh{在城里、在市里。} \textcolor{Sepia}{\selectlanguage{english}In town, in the street.} \textcolor{PineGreen}{\selectlanguage{french}En ville.}  ¶ \textcolor{darkblue}{\textbf{\ipa{ɖʐɯ˧qo˩ kʰi˩}}} \zh{上街} \textcolor{Sepia}{\selectlanguage{english}to go into town} \textcolor{PineGreen}{\selectlanguage{french}aller dans la rue, faire un tour en ville}  

\lhead{\firstmark}
\rhead{\botmark}

\subsection{\hspace{-0.5cm} {\Large \textcolor{darkblue}{\textbf{\ipa{ɖʐɯ˧ʂɯ˥}}}}\hspace{0.5cm}[\kern2pt{\textcolor{darkblue}{\textbf{\ipa{ɖʐɯ˧ʂɯ˥}}}}\kern2pt]} \hypertarget{d`z`M\string_Ms`M\string_T1}{}
\markboth{\textcolor{darkblue}{\textbf{\ipa{ɖʐɯ˧ʂɯ˥}}}}{}
\textcolor{teal}{\zh{名词}} \hspace{4pt} \zh{声调类:} H\#.
\zh{筷子。} \textcolor{Sepia}{\selectlanguage{english}Chopsticks.} \textcolor{PineGreen}{\selectlanguage{french}Baguettes.}  \zh{量词}: \textcolor{darkblue}{\textbf{\ipa{dzi˧}}} 
\lhead{\firstmark}
\rhead{\botmark}

\subsection{\hspace{-0.5cm} {\Large \textcolor{darkblue}{\textbf{\ipa{ɖʐɯ˧ʈʂɯ˥}}}}\hspace{0.5cm}[\kern2pt{\textcolor{darkblue}{\textbf{\ipa{ɖʐɯ˧ʈʂɯ˥}}}}\kern2pt]} \hypertarget{d`z`M\string_Mt`s`M\string_T1}{}
\markboth{\textcolor{darkblue}{\textbf{\ipa{ɖʐɯ˧ʈʂɯ˥}}}}{}
\textcolor{teal}{\zh{名词}} \hspace{4pt} \zh{声调类:} H\#.
\zh{筛子。} \textcolor{Sepia}{\selectlanguage{english}Sifter, sieve.} \textcolor{PineGreen}{\selectlanguage{french}Vannerie.}  \zh{量词}: \textcolor{darkblue}{\textbf{\ipa{nɑ˧}}} 
\lhead{\firstmark}
\rhead{\botmark}

\subsection{\hspace{-0.5cm} {\Large \textcolor{darkblue}{\textbf{\ipa{ɖʐɯ˩\textasciitilde{}ɖʐɯ˧˥}}}}\hspace{0.5cm}[\kern2pt{\textcolor{darkblue}{\textbf{\ipa{ɖʐɯ˧ɖʐɯ˧˥}}}}\kern2pt]} \hypertarget{d`z`M\string_B~d`z`M\string_M\string_T1}{}
\markboth{\textcolor{darkblue}{\textbf{\ipa{ɖʐɯ˩\textasciitilde{}ɖʐɯ˧˥}}}}{}
\textcolor{teal}{\zh{动词}} \hspace{4pt} \zh{声调类:} MH.
\zh{摇(头)。} \textcolor{Sepia}{\selectlanguage{english}To shake (one's head).} \textcolor{PineGreen}{\selectlanguage{french}Secouer (la tête).}  ¶ \textcolor{darkblue}{\textbf{\ipa{ʁo˧ ɖʐɯ˥\textasciitilde{}ɖʐɯ˩}}} \zh{摇头} \textcolor{Sepia}{\selectlanguage{english}to shake one's head} \textcolor{PineGreen}{\selectlanguage{french}secouer la tête}  
 ¶ \textcolor{darkblue}{\textbf{\ipa{ʁo˧ | le˧-ɖʐɯ˩\textasciitilde{}ɖʐɯ˩-ze˩}}} \zh{摇了头} \textcolor{Sepia}{\selectlanguage{english}shook (her/his) head} \textcolor{PineGreen}{\selectlanguage{french}a secoué la tête}  

\lhead{\firstmark}
\rhead{\botmark}

\subsection{\hspace{-0.5cm} {\Large \textcolor{darkblue}{\textbf{\ipa{ɖʐɯ˩kʰɤ˥}}}}\hspace{0.5cm}[\kern2pt{\textcolor{darkblue}{\textbf{\ipa{ɖʐɯ˧kʰɤ˥}}}}\kern2pt]} \hypertarget{d`z`M\string_Bk\string_h7\string_T1}{}
\markboth{\textcolor{darkblue}{\textbf{\ipa{ɖʐɯ˩kʰɤ˥}}}}{}
\textcolor{teal}{\zh{名词}} \hspace{4pt} \zh{声调类:} LH.
\zh{时代。} \textcolor{Sepia}{\selectlanguage{english}Period of time, era.} \textcolor{PineGreen}{\selectlanguage{french}Époque, ère, état de la société.}  \zh{量词}: \textcolor{darkblue}{\textbf{\ipa{ɖʐɯ˩}}} 
\lhead{\firstmark}
\rhead{\botmark}

\subsection{\hspace{-0.5cm} {\Large \textcolor{darkblue}{\textbf{\ipa{ɖʐɯ˩tso\#˥}}}}\hspace{0.5cm}[\kern2pt{\textcolor{darkblue}{\textbf{\ipa{ɖʐɯ˩tso˥}}}}\kern2pt]} \hypertarget{d`z`M\string_Btso\#\string_T1}{}
\markboth{\textcolor{darkblue}{\textbf{\ipa{ɖʐɯ˩tso\#˥}}}}{}
\textcolor{teal}{\zh{名词}} \hspace{4pt} \zh{声调类:} LM+\#H.
\zh{社会规矩。} \textcolor{Sepia}{\selectlanguage{english}Rules of society.} \textcolor{PineGreen}{\selectlanguage{french}Règles de conduite sociale, règles régissant la société (politique, société).}  ¶ \textcolor{darkblue}{\textbf{\ipa{ɖʐɯ˩tso˥ | hĩ˧-qo˩-ɳɯ˩ | le˧-tsʰɯ˩-ɲi˩-tsɯ˩-mæ˩!}}} \zh{社会规矩,是通过人类的经验形成的! / 社会规矩,是人(按一代代的经验)创造的!} \textcolor{Sepia}{\selectlanguage{english}The rules of society, the moral teachings (including proverbs, tales...) come from people / are human creations / are the fruit of human experience!} \textcolor{PineGreen}{\selectlanguage{french}ces morales (les contes, les proverbes...) ça provient des hommes! / la morale (des histoires, ...) c'est le fruit de l'expérience des hommes!}  
 \zh{量词}: \textcolor{darkblue}{\textbf{\ipa{kʰwɤ˥}}} 
\lhead{\firstmark}
\rhead{\botmark}

\subsection{\hspace{-0.5cm} {\Large \textcolor{darkblue}{\textbf{\ipa{ɖʐv̩˧}}} \textsubscript{1}}\hspace{0.5cm}[\kern2pt{\textcolor{darkblue}{\textbf{\ipa{ɖʐv̩˥}}}}\kern2pt]} \hypertarget{d`z`v\string_=\string_M1}{}
\markboth{\textcolor{darkblue}{\textbf{\ipa{ɖʐv̩˧}}} \textsubscript{1}}{}
\textcolor{teal}{\zh{动词}} \hspace{4pt} \zh{声调类:} M intrans.
\zh{燃烧。} \textcolor{Sepia}{\selectlanguage{english}To burn, to catch fire.} \textcolor{PineGreen}{\selectlanguage{french}Brûler; prendre feu.}  ¶ \textcolor{darkblue}{\textbf{\ipa{mv̩˧ ɖʐv̩˧-ze˩!}}} \zh{着火了!} \textcolor{Sepia}{\selectlanguage{english}It has caught fire!} \textcolor{PineGreen}{\selectlanguage{french}Ca a pris feu! / Au feu!}  
 ¶ \textcolor{darkblue}{\textbf{\ipa{mv̩˧ le˧-ɖʐv̩˧-ze˧!}}} \zh{开始着火了!} \textcolor{Sepia}{\selectlanguage{english}The fire has caught!} \textcolor{PineGreen}{\selectlanguage{french}Le feu a pris!}  
 ¶ \textcolor{darkblue}{\textbf{\ipa{tʰi˧-ɖʐv̩˧-dʑo˧!}}} \zh{火在燃烧!} \textcolor{Sepia}{\selectlanguage{english}The fire is burning!} \textcolor{PineGreen}{\selectlanguage{french}C'est en train de brûler! / Le feu est en train de brûler!}  

\lhead{\firstmark}
\rhead{\botmark}

\subsection{\hspace{-0.5cm} {\Large \textcolor{darkblue}{\textbf{\ipa{ɖʐv̩˧}}} \textsubscript{2}}\hspace{0.5cm}[\kern2pt{\textcolor{darkblue}{\textbf{\ipa{ɖʐv̩˥}}}}\kern2pt]} \hypertarget{d`z`v\string_=\string_M2}{}
\markboth{\textcolor{darkblue}{\textbf{\ipa{ɖʐv̩˧}}} \textsubscript{2}}{}
\textcolor{teal}{\zh{名词}} \hspace{4pt} \zh{声调类:} M.
\zh{朋友、伙伴、伴侣。} \textcolor{Sepia}{\selectlanguage{english}Friend, companion, partner.} \textcolor{PineGreen}{\selectlanguage{french}Ami/amie, compagnon/compagne.}  ¶ \textcolor{darkblue}{\textbf{\ipa{njɤ˧ | ɖʐv̩˧ ɲi˩.}}} \zh{是我朋友。} \textcolor{Sepia}{\selectlanguage{english}[(S)he] is my friend.} \textcolor{PineGreen}{\selectlanguage{french}C'est mon ami(e).}  
 ¶ \textcolor{darkblue}{\textbf{\ipa{õ˧ ɖʐv̩˥, õ˩ li˩! |}}} \zh{“大家都容易受朋友的影响!”(直译:“自己的朋友,自己看(=自己爱学他们的习惯)”)} \textcolor{Sepia}{\selectlanguage{english}'One is easily influenced by one's friends!' (Literally: 'One's friends, one observes'.) The proverb refers to influence from friends, good or bad: good friends exert good influences; bad friends exert bad influences.} \textcolor{PineGreen}{\selectlanguage{french}'On est influencé par ses amis!' (Littéralement: 'On observe ses amis!') Le proverbe souligne l'influence des amis, en bien ou en mal selon qu'on a ou non choisi judicieusement.}  
 \zh{量词}: \textcolor{darkblue}{\textbf{\ipa{v̩˧}}} 
\lhead{\firstmark}
\rhead{\botmark}

\subsection{\hspace{-0.5cm} {\Large \textcolor{darkblue}{\textbf{\ipa{ɖʐv̩˧}}} \textsubscript{3}}\hspace{0.5cm}[\kern2pt{\textcolor{darkblue}{\textbf{\ipa{ɖʐv̩˥}}}}\kern2pt]} \hypertarget{d`z`v\string_=\string_M3}{}
\markboth{\textcolor{darkblue}{\textbf{\ipa{ɖʐv̩˧}}} \textsubscript{3}}{}
\textcolor{teal}{\zh{名词}} \hspace{4pt} \zh{声调类:} M.
\zh{事故,(不幸的)大事。} \textcolor{Sepia}{\selectlanguage{english}An important and unfortunate event, such as a serious accident.} \textcolor{PineGreen}{\selectlanguage{french}Accident (grave).}  ¶ \textcolor{darkblue}{\textbf{\ipa{ɖʐv̩˧ kʰɯ˧˥}}} \zh{犯错误,出大事} \textcolor{Sepia}{\selectlanguage{english}to cause an accident; to commit a fault; something serious happens} \textcolor{PineGreen}{\selectlanguage{french}causer un accident, commettre une faute; il se passe quelque chose de grave}  
 ¶ \textcolor{darkblue}{\textbf{\ipa{ɖʐv̩˧ kʰɯ˧-ze˥}}} \zh{同上,加上\mytextsc{pfv语素}} \textcolor{Sepia}{\selectlanguage{english}As above, with the \mytextsc{pfv} morpheme} \textcolor{PineGreen}{\selectlanguage{french}Comme ci-dessus, avec ajout du \mytextsc{pfv}}  
 ¶ \textcolor{darkblue}{\textbf{\ipa{ɖʐv̩˧ ɖɯ˧-ɖʐv̩˧ | kʰɯ˧-ze˥!}}} \zh{出大事了!} \textcolor{Sepia}{\selectlanguage{english}An accident has happened! / There's been an accident!} \textcolor{PineGreen}{\selectlanguage{french}il est arrivé un accident!}  
 \zh{量词}: \textcolor{darkblue}{\textbf{\ipa{ɖʐv̩˧}}} 
\lhead{\firstmark}
\rhead{\botmark}

\subsection{\hspace{-0.5cm} {\Large \textcolor{darkblue}{\textbf{\ipa{ɖʐv̩˧}}} \textsubscript{4}}\hspace{0.5cm}[\kern2pt{\textcolor{darkblue}{\textbf{\ipa{ɖʐv̩˥}}}}\kern2pt]} \hypertarget{d`z`v\string_=\string_M4}{}
\markboth{\textcolor{darkblue}{\textbf{\ipa{ɖʐv̩˧}}} \textsubscript{4}}{}
\textcolor{teal}{\zh{名词}} \hspace{4pt} \zh{声调类:} M.
\zh{露水。} \textcolor{Sepia}{\selectlanguage{english}Dew.} \textcolor{PineGreen}{\selectlanguage{french}Rosée.} \zh{~【参考】~} \hyperlink{}{\textcolor{darkblue}{\textbf{\ipa{ɖʐv̩˧qʰɑ˧}}}} 
\lhead{\firstmark}
\rhead{\botmark}

\subsection{\hspace{-0.5cm} {\Large \textcolor{darkblue}{\textbf{\ipa{ɖʐv̩˥}}}}\hspace{0.5cm}[\kern2pt{\textcolor{darkblue}{\textbf{\ipa{ɖʐv̩˥}}}}\kern2pt]} \hypertarget{d`z`v\string_=\string_T1}{}
\markboth{\textcolor{darkblue}{\textbf{\ipa{ɖʐv̩˥}}}}{}
\textcolor{teal}{\zh{名词}} \hspace{4pt} \zh{声调类:} \#H.
\zh{动脉。} \textcolor{Sepia}{\selectlanguage{english}Large vein, artery.} \textcolor{PineGreen}{\selectlanguage{french}Artère.}  \zh{量词}: \textcolor{darkblue}{\textbf{\ipa{kʰɯ˩}}} \zh{~【参考】~} \hyperlink{}{\textcolor{darkblue}{\textbf{\ipa{ɖʐv̩˧tsi˥}}}} 
\lhead{\firstmark}
\rhead{\botmark}

\subsection{\hspace{-0.5cm} {\Large \textcolor{darkblue}{\textbf{\ipa{ɖʐv̩˥}}}}\hspace{0.5cm}[\kern2pt{\textcolor{darkblue}{\textbf{\ipa{ɖʐv̩˥}}}}\kern2pt]} \hypertarget{d`z`v\string_=\string_T1}{}
\markboth{\textcolor{darkblue}{\textbf{\ipa{ɖʐv̩˥}}}}{}
\textcolor{teal}{\zh{形容词}} \hspace{4pt} \zh{声调类:} H.
\zh{湿。} \textcolor{Sepia}{\selectlanguage{english}Moist, wet, damp, humid.} \textcolor{PineGreen}{\selectlanguage{french}Humide, mouillé.}  ¶ \textcolor{darkblue}{\textbf{\ipa{le˧-ɖʐv̩˥-ze˩}}} \zh{\mytextsc{accomp} \string_ \mytextsc{pfv}} \textcolor{Sepia}{\selectlanguage{english}\mytextsc{accomp} \string_ \mytextsc{pfv}} \textcolor{PineGreen}{\selectlanguage{french}\mytextsc{accomp} \string_ \mytextsc{pfv}}  
 ¶ \textcolor{darkblue}{\textbf{\ipa{ɖʐv̩˧\textasciitilde{}ɖʐv̩˧}}} \zh{\mytextsc{red}} \textcolor{Sepia}{\selectlanguage{english}\mytextsc{red}} \textcolor{PineGreen}{\selectlanguage{french}\mytextsc{red}}  
 ¶ \textcolor{darkblue}{\textbf{\ipa{ʈʂe˧ ɖʐv̩˧-ze˩!}}} \zh{土湿了。} \textcolor{Sepia}{\selectlanguage{english}The earth is damp!} \textcolor{PineGreen}{\selectlanguage{french}la terre est mouillée!}  

\lhead{\firstmark}
\rhead{\botmark}

\subsection{\hspace{-0.5cm} {\Large \textcolor{darkblue}{\textbf{\ipa{ɖʐv̩˥}}}}\hspace{0.5cm}[\kern2pt{\textcolor{darkblue}{\textbf{\ipa{ɖʐv̩˥}}}}\kern2pt]} \hypertarget{d`z`v\string_=\string_T1}{}
\markboth{\textcolor{darkblue}{\textbf{\ipa{ɖʐv̩˥}}}}{}
\textcolor{teal}{\zh{动词}} \hspace{4pt} \zh{声调类:} H.
\zh{涨。} \textcolor{Sepia}{\selectlanguage{english}To rise, to go up, to increase.} \textcolor{PineGreen}{\selectlanguage{french}Augmenter.}  ¶ \textcolor{darkblue}{\textbf{\ipa{hĩ˧ ɖʐv̩˧}}} \zh{人变多} \textcolor{Sepia}{\selectlanguage{english}people become numerous} \textcolor{PineGreen}{\selectlanguage{french}Les gens deviennent nombreux, se multiplient}  
 ¶ \textcolor{darkblue}{\textbf{\ipa{mo˧ ɖʐv̩˥}}} \zh{菌子长得多} \textcolor{Sepia}{\selectlanguage{english}mushrooms multiply, become numerous} \textcolor{PineGreen}{\selectlanguage{french}les champignons se multiplient}  

\lhead{\firstmark}
\rhead{\botmark}

\subsection{\hspace{-0.5cm} {\Large \textcolor{darkblue}{\textbf{\ipa{ɖʐv̩˩\textsubscript{a}}}} \textsubscript{1}}\hspace{0.5cm}[\kern2pt{\textcolor{darkblue}{\textbf{\ipa{ɖʐv̩˩˥}}}}\kern2pt]} \hypertarget{d`z`v\string_=\string_Ba1}{}
\markboth{\textcolor{darkblue}{\textbf{\ipa{ɖʐv̩˩\textsubscript{a}}}} \textsubscript{1}}{}
\textcolor{teal}{\zh{形容词}} \hspace{4pt} \zh{声调类:} L\textsubscript{a}.
\zh{丑陋。} \textcolor{Sepia}{\selectlanguage{english}Ugly.} \textcolor{PineGreen}{\selectlanguage{french}Laid, vilain.}  ¶ \textcolor{darkblue}{\textbf{\ipa{ɖʐv̩˩-hĩ˩˥}}} \zh{丑的} \textcolor{Sepia}{\selectlanguage{english}\mytextsc{rel}/\mytextsc{nmlz}} \textcolor{PineGreen}{\selectlanguage{french}\mytextsc{rel}/\mytextsc{nmlz}}  
 ¶ \textcolor{darkblue}{\textbf{\ipa{ʈʂʰɯ˧-v̩˧ | ɖwæ˧˥ | ɖʐv̩˩˥!}}} \zh{这个好丑!} \textcolor{Sepia}{\selectlanguage{english}This one is really ugly!} \textcolor{PineGreen}{\selectlanguage{french}celui-là/celle-là est vraiment méchant/mauvais}  

\lhead{\firstmark}
\rhead{\botmark}

\subsection{\hspace{-0.5cm} {\Large \textcolor{darkblue}{\textbf{\ipa{ɖʐv̩˩\textsubscript{a}}}} \textsubscript{2}}\hspace{0.5cm}[\kern2pt{\textcolor{darkblue}{\textbf{\ipa{ɖʐv̩˩˥}}}}\kern2pt]} \hypertarget{d`z`v\string_=\string_Ba2}{}
\markboth{\textcolor{darkblue}{\textbf{\ipa{ɖʐv̩˩\textsubscript{a}}}} \textsubscript{2}}{}
\textcolor{teal}{\zh{动词}} \hspace{4pt} \zh{声调类:} L\textsubscript{a}.
\zh{决定、选择、拿主意。} \textcolor{Sepia}{\selectlanguage{english}To decide, to make a decision.} \textcolor{PineGreen}{\selectlanguage{french}Décider, choisir.}  ¶ \textcolor{darkblue}{\textbf{\ipa{njɤ˧-ɳɯ˧ | ɖʐv̩˧ ʝi˧-bi˧!}}} \zh{我来决定吧!} \textcolor{Sepia}{\selectlanguage{english}I'm going to decide!} \textcolor{PineGreen}{\selectlanguage{french}C'est moi qui vais décider!}  

\lhead{\firstmark}
\rhead{\botmark}

\subsection{\hspace{-0.5cm} {\Large \textcolor{darkblue}{\textbf{\ipa{ɖʐv̩˧\textsubscript{b}}}}}\hspace{0.5cm}[\kern2pt{\textcolor{darkblue}{\textbf{\ipa{ɖʐv̩˥}}}}\kern2pt]} \hypertarget{d`z`v\string_=\string_Mb1}{}
\markboth{\textcolor{darkblue}{\textbf{\ipa{ɖʐv̩˧\textsubscript{b}}}}}{}
\textcolor{teal}{\zh{量词}} \hspace{4pt} \zh{声调类:} M\textsubscript{b}.
\zh{量词:事故(一场)。} \textcolor{Sepia}{\selectlanguage{english}Self-classifier for accidents.} \textcolor{PineGreen}{\selectlanguage{french}Auto-classificateur des accidents.} 
\lhead{\firstmark}
\rhead{\botmark}

\subsection{\hspace{-0.5cm} {\Large \textcolor{darkblue}{\textbf{\ipa{ɖʐv̩˧-nɑ˥mi˩}}}}\hspace{0.5cm}[\kern2pt{\textcolor{darkblue}{\textbf{\ipa{ɖʐv̩˧nɑ˥mi˩}}}}\kern2pt]} \hypertarget{d`z`v\string_=\string_M-nA\string_Tmi\string_B1}{}
\markboth{\textcolor{darkblue}{\textbf{\ipa{ɖʐv̩˧-nɑ˥mi˩}}}}{}
\textcolor{teal}{\zh{名词}} \hspace{4pt} \zh{声调类:} \#H-.
\zh{鹳。} \textcolor{Sepia}{\selectlanguage{english}Heron.} \textcolor{PineGreen}{\selectlanguage{french}Héron: oiseau échassier, non migrateur.}  \zh{量词}: \textcolor{darkblue}{\textbf{\ipa{mi˩}}} 
\lhead{\firstmark}
\rhead{\botmark}

\subsection{\hspace{-0.5cm} {\Large \textcolor{darkblue}{\textbf{\ipa{ɖʐv̩˧qʰɑ˧}}}}\hspace{0.5cm}[\kern2pt{\textcolor{darkblue}{\textbf{\ipa{ɖʐv̩˧qʰɑ˧}}}}\kern2pt]} \hypertarget{d`z`v\string_=\string_Mq\string_hA\string_M1}{}
\markboth{\textcolor{darkblue}{\textbf{\ipa{ɖʐv̩˧qʰɑ˧}}}}{}
\textcolor{teal}{\zh{名词}} \hspace{4pt} \zh{声调类:} M.
\zh{露水。} \textcolor{Sepia}{\selectlanguage{english}Dew.} \textcolor{PineGreen}{\selectlanguage{french}Rosée.} \zh{~【参考】~} \hyperlink{}{\textcolor{darkblue}{\textbf{\ipa{ɖʐv̩˧}}} \textsubscript{4}} 
\lhead{\firstmark}
\rhead{\botmark}

\subsection{\hspace{-0.5cm} {\Large \textcolor{darkblue}{\textbf{\ipa{ɖʐv̩˩ti\#˥}}}}\hspace{0.5cm}[\kern2pt{\textcolor{darkblue}{\textbf{\ipa{ɖʐv̩˩ti˥}}}}\kern2pt]} \hypertarget{d`z`v\string_=\string_Bti\#\string_T1}{}
\markboth{\textcolor{darkblue}{\textbf{\ipa{ɖʐv̩˩ti\#˥}}}}{}
\textcolor{teal}{\zh{名词}} \hspace{4pt} \zh{声调类:} LM+\#H.
\zh{矛。} \textcolor{Sepia}{\selectlanguage{english}Spear.} \textcolor{PineGreen}{\selectlanguage{french}Lance.} 
\lhead{\firstmark}
\rhead{\botmark}

\subsection{\hspace{-0.5cm} {\Large \textcolor{darkblue}{\textbf{\ipa{ɖʐv̩˧tsi˥}}}}\hspace{0.5cm}[\kern2pt{\textcolor{darkblue}{\textbf{\ipa{ɖʐv̩˧tsi˥}}}}\kern2pt]} \hypertarget{d`z`v\string_=\string_Mtsi\string_T1}{}
\markboth{\textcolor{darkblue}{\textbf{\ipa{ɖʐv̩˧tsi˥}}}}{}
\textcolor{teal}{\zh{名词}} \hspace{4pt} \zh{声调类:} H\#.
\ding{202} \zh{动脉。} \textcolor{Sepia}{\selectlanguage{english}Artery.} \textcolor{PineGreen}{\selectlanguage{french}Artère (du corps humain).}  \zh{量词}: \textcolor{darkblue}{\textbf{\ipa{kʰɯ˩}}} \ding{203} \zh{茎。} \textcolor{Sepia}{\selectlanguage{english}Stem, stalk.} \textcolor{PineGreen}{\selectlanguage{french}Tige (d'une plante).} \zh{~【参考】~} \textcolor{darkblue}{\textbf{\ipa{ɖʐv̩˥}}} 
\lhead{\firstmark}
\rhead{\botmark}

\subsection{\hspace{-0.5cm} {\Large \textcolor{darkblue}{\textbf{\ipa{ɖʐv̩˧ʐv̩˧-ɖʐv̩˧mi\#˥}}}}\hspace{0.5cm}[\kern2pt{\textcolor{darkblue}{\textbf{\ipa{xxxx non-correspondance entre le nombre de morphèmes et le nombre de tons de morphèmes}}}}\kern2pt]} \hypertarget{d`z`v\string_=\string_Mz`v\string_=\string_M-d`z`v\string_=\string_Mmi\#\string_T1}{}
\markboth{\textcolor{darkblue}{\textbf{\ipa{ɖʐv̩˧ʐv̩˧-ɖʐv̩˧mi\#˥}}}}{}
\textcolor{teal}{\zh{名词}} \hspace{4pt} \zh{声调类:} \#H.
\zh{朋友、伙伴、伴侣。} \textcolor{Sepia}{\selectlanguage{english}Friend, companion, partner.} \textcolor{PineGreen}{\selectlanguage{french}Ami(e).} 
\lhead{\firstmark}
\rhead{\botmark}

\subsection{\hspace{-0.5cm} {\Large \textcolor{darkblue}{\textbf{\ipa{ɖʐwæ˥}}}}\hspace{0.5cm}[\kern2pt{\textcolor{darkblue}{\textbf{\ipa{ɖʐwæ˥}}}}\kern2pt]} \hypertarget{d`z`w\{\string_T1}{}
\markboth{\textcolor{darkblue}{\textbf{\ipa{ɖʐwæ˥}}}}{}
\textcolor{teal}{\zh{名词}} \hspace{4pt} \zh{声调类:} \#H.
\zh{锄头。} \textcolor{Sepia}{\selectlanguage{english}Small hoe (smaller than \textcolor{darkblue}{\textbf{\ipa{/hwæ˧pʰæ˩/}}}).} \textcolor{PineGreen}{\selectlanguage{french}Petite houe (plus petite que \textcolor{darkblue}{\textbf{\ipa{/hwæ˧pʰæ˩/}}}).}  \zh{量词}: \textcolor{darkblue}{\textbf{\ipa{nɑ˧}}} 
\lhead{\firstmark}
\rhead{\botmark}

\subsection{\hspace{-0.5cm} {\Large \textcolor{darkblue}{\textbf{\ipa{ɖʐwæ˧lɑ˧-ʁo˧ɖɯ˧˥}}}}\hspace{0.5cm}[\kern2pt{\textcolor{darkblue}{\textbf{\ipa{xxxx non-correspondance entre le nombre de morphèmes et le nombre de tons de morphèmes}}}}\kern2pt]} \hypertarget{d`z`w\{\string_MlA\string_M-Ro\string_Md`M\string_M\string_T1}{}
\markboth{\textcolor{darkblue}{\textbf{\ipa{ɖʐwæ˧lɑ˧-ʁo˧ɖɯ˧˥}}}}{}
\textcolor{teal}{\zh{名词}} \hspace{4pt} \zh{声调类:} MH\#.
\zh{雀。} \textcolor{Sepia}{\selectlanguage{english}A type of sparrow.} \textcolor{PineGreen}{\selectlanguage{french}Oiseau ressemblant à un moineau, au corps blanc et noir; M23 croît le reconnaître dans: Pericrocotus divaricatus, mais cette espèce n'existe que dans le nord de la Chine.}  \zh{量词}: \textcolor{darkblue}{\textbf{\ipa{mi˩}}} 
\lhead{\firstmark}
\rhead{\botmark}

\subsection{\hspace{-0.5cm} {\Large \textcolor{darkblue}{\textbf{\ipa{ɖʐwæ˧mi˧}}}}\hspace{0.5cm}[\kern2pt{\textcolor{darkblue}{\textbf{\ipa{ɖʐwæ˧mi˧}}}}\kern2pt]} \hypertarget{d`z`w\{\string_Mmi\string_M1}{}
\markboth{\textcolor{darkblue}{\textbf{\ipa{ɖʐwæ˧mi˧}}}}{}
\textcolor{teal}{\zh{名词}} \hspace{4pt} \zh{声调类:} M.
\zh{麻雀。} \textcolor{Sepia}{\selectlanguage{english}Sparrow.} \textcolor{PineGreen}{\selectlanguage{french}Moineau.}  \zh{量词}: \textcolor{darkblue}{\textbf{\ipa{mi˩}}} 
\lhead{\firstmark}
\rhead{\botmark}

\subsection{\hspace{-0.5cm} {\Large \textcolor{darkblue}{\textbf{\ipa{ɖʐwæ˧pʰv̩\#˥}}}}\hspace{0.5cm}[\kern2pt{\textcolor{darkblue}{\textbf{\ipa{ɖʐwæ˧pʰv̩˧}}}}\kern2pt]} \hypertarget{d`z`w\{\string_Mp\string_hv\string_=\#\string_T1}{}
\markboth{\textcolor{darkblue}{\textbf{\ipa{ɖʐwæ˧pʰv̩\#˥}}}}{}
\textcolor{teal}{\zh{名词}} \hspace{4pt} \zh{声调类:} \#H.
\zh{公麻雀。} \textcolor{Sepia}{\selectlanguage{english}Male sparrow.} \textcolor{PineGreen}{\selectlanguage{french}Moineau mâle.}  ¶ \textcolor{darkblue}{\textbf{\ipa{ɖʐwæ˧pʰv̩˧ tʰv̩˧-mi˧˥ / ɖʐwæ˧pʰv̩˧ tʰv̩˧-mi˥\#}}} \zh{这只公麻雀} \textcolor{Sepia}{\selectlanguage{english}\mytextsc{n}+\mytextsc{dem}+\mytextsc{clf}} \textcolor{PineGreen}{\selectlanguage{french}\mytextsc{n}+\mytextsc{dem}+\mytextsc{clf}}  
 \zh{量词}: \textcolor{darkblue}{\textbf{\ipa{mi˩}}} 
\lhead{\firstmark}
\rhead{\botmark}

\subsection{\hspace{-0.5cm} {\Large \textcolor{darkblue}{\textbf{\ipa{ɖʐwæ˧zo\#˥}}}}\hspace{0.5cm}[\kern2pt{\textcolor{darkblue}{\textbf{\ipa{ɖʐwæ˧zo˧}}}}\kern2pt]} \hypertarget{d`z`w\{\string_Mzo\#\string_T1}{}
\markboth{\textcolor{darkblue}{\textbf{\ipa{ɖʐwæ˧zo\#˥}}}}{}
\textcolor{teal}{\zh{名词}} \hspace{4pt} \zh{声调类:} \#H.
\zh{小麻雀。} \textcolor{Sepia}{\selectlanguage{english}Baby sparrow, little sparrow.} \textcolor{PineGreen}{\selectlanguage{french}Moinillon, petit moineau, bébé moineau.}  \zh{量词}: \textcolor{darkblue}{\textbf{\ipa{v̩˧, mi˩}}} 
\lhead{\firstmark}
\rhead{\botmark}

\subsection{\hspace{-0.5cm} {\Large \textcolor{darkblue}{\textbf{\ipa{ɖʐwæ˩\textsubscript{a}}}}}\hspace{0.5cm}[\kern2pt{\textcolor{darkblue}{\textbf{\ipa{ɖʐwæ˩˥}}}}\kern2pt]} \hypertarget{d`z`w\{\string_Ba1}{}
\markboth{\textcolor{darkblue}{\textbf{\ipa{ɖʐwæ˩\textsubscript{a}}}}}{}
\textcolor{teal}{\zh{动词}} \hspace{4pt} \zh{声调类:} L\textsubscript{a}.
\zh{吵架。} \textcolor{Sepia}{\selectlanguage{english}To quarrel.} \textcolor{PineGreen}{\selectlanguage{french}Se disputer (monosyllabe).}  ¶ \textcolor{darkblue}{\textbf{\ipa{ɖʐwæ˧\textasciitilde{}ɖʐwæ˥}}} \zh{\mytextsc{重叠}} \textcolor{Sepia}{\selectlanguage{english}\mytextsc{red}} \textcolor{PineGreen}{\selectlanguage{french}\mytextsc{red}}  

\lhead{\firstmark}
\rhead{\botmark}

\subsection{\hspace{-0.5cm} {\Large \textcolor{darkblue}{\textbf{\ipa{ɖʐwæ˩hi˩}}}}\hspace{0.5cm}[\kern2pt{\textcolor{darkblue}{\textbf{\ipa{ɖʐwæ˩hi˩˥}}}}\kern2pt]} \hypertarget{d`z`w\{\string_Bhi\string_B1}{}
\markboth{\textcolor{darkblue}{\textbf{\ipa{ɖʐwæ˩hi˩}}}}{}
\textcolor{teal}{\zh{名词}} \hspace{4pt} \zh{声调类:} L.
\ding{202} \zh{獠牙。} \textcolor{Sepia}{\selectlanguage{english}Canine tooth, fang.} \textcolor{PineGreen}{\selectlanguage{french}Canine (dent), croc.}  \zh{量词}: \textcolor{darkblue}{\textbf{\ipa{ɭɯ˧}}} \ding{203} \zh{动物的牙(犬牙)。} \textcolor{Sepia}{\selectlanguage{english}Fang.} \textcolor{PineGreen}{\selectlanguage{french}Crocs (de bête).} 
\lhead{\firstmark}
\rhead{\botmark}

\subsection{\hspace{-0.5cm} {\Large \textcolor{darkblue}{\textbf{\ipa{ɖʐwæ˧˥}}}}\hspace{0.5cm}[\kern2pt{\textcolor{darkblue}{\textbf{\ipa{ɖʐwæ˧˥}}}}\kern2pt]} \hypertarget{d`z`w\{\string_M\string_T1}{}
\markboth{\textcolor{darkblue}{\textbf{\ipa{ɖʐwæ˧˥}}}}{}
\textcolor{teal}{\zh{动词}} \hspace{4pt} \zh{声调类:} MH.
\zh{掉下。} \textcolor{Sepia}{\selectlanguage{english}To fall down; to release, to drop.} \textcolor{PineGreen}{\selectlanguage{french}Tomber; laisser tomber, lâcher (un objet qu'on tenait à la main).}  ¶ \textcolor{darkblue}{\textbf{\ipa{mv̩˩tɕo˧ ɖʐwæ˧˥ / mv̩˩tɕo˧ ɖʐwæ˧-ze˥}}} \zh{掉下去(+了)} \textcolor{Sepia}{\selectlanguage{english}to fall down} \textcolor{PineGreen}{\selectlanguage{french}tomber par terre; littéralement “tomber vers le bas”}  

\lhead{\firstmark}
\rhead{\botmark}

\subsection{\hspace{-0.5cm} {\Large \textcolor{darkblue}{\textbf{\ipa{ɖʐwæ˩˧}}}}\hspace{0.5cm}[\kern2pt{\textcolor{darkblue}{\textbf{\ipa{ɖʐwæ˩˥}}}}\kern2pt]} \hypertarget{d`z`w\{\string_B\string_M1}{}
\markboth{\textcolor{darkblue}{\textbf{\ipa{ɖʐwæ˩˧}}}}{}
\textcolor{teal}{\zh{名词}} \hspace{4pt} \zh{声调类:} LM.
\zh{麻雀。} \textcolor{Sepia}{\selectlanguage{english}Sparrow (monosyllabic form; not in common use).} \textcolor{PineGreen}{\selectlanguage{french}Moineau (forme monosyllabique; n'est pas d'usage courant).}  \zh{量词}: \textcolor{darkblue}{\textbf{\ipa{mi˩}}} 
\lhead{\firstmark}
\rhead{\botmark}

\newpage
\section*{\centering- \textcolor{darkblue}{\textbf{\ipa{ə}}} -}
\subsection{\hspace{-0.5cm} {\Large \textcolor{darkblue}{\textbf{\ipa{ə˧bɑ˩-lɑ˩bɑ˩}}}}\hspace{0.5cm}[\kern2pt{\textcolor{darkblue}{\textbf{\ipa{ə˧bɑ˩lɑ˧bɑ˧}}}}\kern2pt]} \hypertarget{@\string_MbA\string_B-lA\string_BbA\string_B1}{}
\markboth{\textcolor{darkblue}{\textbf{\ipa{ə˧bɑ˩-lɑ˩bɑ˩}}}}{}
\textcolor{teal}{\zh{名词}} \hspace{4pt} \zh{声调类:} L\#-.
\zh{仙人掌。} \textcolor{Sepia}{\selectlanguage{english}Cactus.} \textcolor{PineGreen}{\selectlanguage{french}Cactus.}  ¶ \textcolor{darkblue}{\textbf{\ipa{ə˧bɑ˩-lɑ˩bɑ˩ | ɖɯ˧-dzi˩}}} \zh{一棵仙人掌} \textcolor{Sepia}{\selectlanguage{english}a cactus plant} \textcolor{PineGreen}{\selectlanguage{french}un cactus}  

\lhead{\firstmark}
\rhead{\botmark}

\subsection{\hspace{-0.5cm} {\Large \textcolor{darkblue}{\textbf{\ipa{ə˧bo˥\$}}}}\hspace{0.5cm}[\kern2pt{\textcolor{darkblue}{\textbf{\ipa{ə˧bo˥}}}}\kern2pt]} \hypertarget{@\string_Mbo\string_T\$1}{}
\markboth{\textcolor{darkblue}{\textbf{\ipa{ə˧bo˥\$}}}}{}
\textcolor{teal}{\zh{名词}} \hspace{4pt} \zh{声调类:} H\$.
\zh{父亲的兄弟。} \textcolor{Sepia}{\selectlanguage{english}Paternal uncle.} \textcolor{PineGreen}{\selectlanguage{french}Oncle paternel=frère du père (sens vérifié: renvoie à la famille du père).}  ¶ \textcolor{darkblue}{\textbf{\ipa{ə˧bo˧-ɖɯ˧˥}}} \zh{伯父:父亲的哥哥} \textcolor{Sepia}{\selectlanguage{english}paternal uncle, father's elder brother} \textcolor{PineGreen}{\selectlanguage{french}oncle paternel aîné du père}  
 ¶ \textcolor{darkblue}{\textbf{\ipa{ə˧bo˧-tɕi˥ (+ɲi˩)}}} \zh{叔叔:父亲的弟弟} \textcolor{Sepia}{\selectlanguage{english}paternal uncle, father's younger brother} \textcolor{PineGreen}{\selectlanguage{french}oncle paternel cadet du père}  
 \zh{量词}: \textcolor{darkblue}{\textbf{\ipa{v̩˧}}} 
\lhead{\firstmark}
\rhead{\botmark}

\subsection{\hspace{-0.5cm} {\Large \textcolor{darkblue}{\textbf{\ipa{ə˧bo˧tɕo˧li˧}}}}\hspace{0.5cm}[\kern2pt{\textcolor{darkblue}{\textbf{\ipa{ə˧bo˧tɕo˧li˧}}}}\kern2pt]} \hypertarget{@\string_Mbo\string_Mts£o\string_Mli\string_M1}{}
\markboth{\textcolor{darkblue}{\textbf{\ipa{ə˧bo˧tɕo˧li˧}}}}{}
\textcolor{teal}{\zh{名词}} \hspace{4pt} \zh{声调类:} M.
\zh{蟋蟀。} \textcolor{Sepia}{\selectlanguage{english}Cricket.} \textcolor{PineGreen}{\selectlanguage{french}Criquet.}  \zh{量词}: \textcolor{darkblue}{\textbf{\ipa{mi˩}}} 
\lhead{\firstmark}
\rhead{\botmark}

\subsection{\hspace{-0.5cm} {\Large \textcolor{darkblue}{\textbf{\ipa{ə˧bv̩˩}}}}\hspace{0.5cm}[\kern2pt{\textcolor{darkblue}{\textbf{\ipa{ə˧bv̩˩}}}}\kern2pt]} \hypertarget{@\string_Mbv\string_=\string_B1}{}
\markboth{\textcolor{darkblue}{\textbf{\ipa{ə˧bv̩˩}}}}{}
\textcolor{teal}{\zh{名词}} \hspace{4pt} \zh{声调类:} L\#.
\zh{烤砖、陶器等用的烤炉。} \textcolor{Sepia}{\selectlanguage{english}Oven to make bricks, ceramics...} \textcolor{PineGreen}{\selectlanguage{french}Four où on cuit les briques, les objets en céramique….}  \zh{量词}: \textcolor{darkblue}{\textbf{\ipa{ɭɯ˧}}} 
\lhead{\firstmark}
\rhead{\botmark}

\subsection{\hspace{-0.5cm} {\Large \textcolor{darkblue}{\textbf{\ipa{ə˧bv̩˧-ʁwɤ˧}}}}\hspace{0.5cm}[\kern2pt{\textcolor{darkblue}{\textbf{\ipa{xxxx non-correspondance entre le nombre de morphèmes et le nombre de tons de morphèmes}}}}\kern2pt]} \hypertarget{@\string_Mbv\string_=\string_M-Rw7\string_M1}{}
\markboth{\textcolor{darkblue}{\textbf{\ipa{ə˧bv̩˧-ʁwɤ˧}}}}{}
\textcolor{teal}{\zh{名词}} \hspace{4pt} \zh{声调类:} M.
\zh{阿布瓦村。} \textcolor{Sepia}{\selectlanguage{english}Name of a village.} \textcolor{PineGreen}{\selectlanguage{french}Abuwa (nom de village).} 
\lhead{\firstmark}
\rhead{\botmark}

\subsection{\hspace{-0.5cm} {\Large \textcolor{darkblue}{\textbf{\ipa{ə˧ɕjɤ˩}}}}\hspace{0.5cm}[\kern2pt{\textcolor{darkblue}{\textbf{\ipa{ə˧ɕjɤ˩}}}}\kern2pt]} \hypertarget{@\string_Ms£j7\string_B1}{}
\markboth{\textcolor{darkblue}{\textbf{\ipa{ə˧ɕjɤ˩}}}}{}
\textcolor{teal}{\zh{名词}} \hspace{4pt} \zh{声调类:} L\#.
\zh{情人。} \textcolor{Sepia}{\selectlanguage{english}Lover, boy/girl-friend.} \textcolor{PineGreen}{\selectlanguage{french}Petit(e) ami(e), amant(e).}  \zh{量词}: \textcolor{darkblue}{\textbf{\ipa{v̩˧}}} \zh{~【参考】~} \hyperlink{}{\textcolor{darkblue}{\textbf{\ipa{ə˧ɖo˧}}}} 
\lhead{\firstmark}
\rhead{\botmark}

\subsection{\hspace{-0.5cm} {\Large \textcolor{darkblue}{\textbf{\ipa{ə˧ɕjo˩}}}}\hspace{0.5cm}[\kern2pt{\textcolor{darkblue}{\textbf{\ipa{ə˧ɕjo˩}}}}\kern2pt]} \hypertarget{@\string_Ms£jo\string_B1}{}
\markboth{\textcolor{darkblue}{\textbf{\ipa{ə˧ɕjo˩}}}}{}
\textcolor{teal}{\zh{名词}} \hspace{4pt} \zh{声调类:} L\#.
\zh{一个姓。这个姓,永宁有两家。} \textcolor{Sepia}{\selectlanguage{english}A family name from Yongning. There are two families in Yongning that carry this name.} \textcolor{PineGreen}{\selectlanguage{french}Nom de clan/famille étendue. Deux familles portent ce nom à Yongning.}  ¶ \textcolor{darkblue}{\textbf{\ipa{ə˧ɕjo˩=ɻ̍˩}}} \zh{\textcolor{darkblue}{\textbf{\ipa{/ə˧ɕjo˩/}}}家族} \textcolor{Sepia}{\selectlanguage{english}the \textcolor{darkblue}{\textbf{\ipa{/ə˧ɕjo˩/}}} clan} \textcolor{PineGreen}{\selectlanguage{french}le clan \textcolor{darkblue}{\textbf{\ipa{/ə˧ɕjo˩/}}}}  

\lhead{\firstmark}
\rhead{\botmark}

\subsection{\hspace{-0.5cm} {\Large \textcolor{darkblue}{\textbf{\ipa{ə˧dɑ˥\$}}}}\hspace{0.5cm}[\kern2pt{\textcolor{darkblue}{\textbf{\ipa{ə˧dɑ˥}}}}\kern2pt]} \hypertarget{@\string_MdA\string_T\$1}{}
\markboth{\textcolor{darkblue}{\textbf{\ipa{ə˧dɑ˥\$}}}}{}
\textcolor{teal}{\zh{名词}} \hspace{4pt} \zh{声调类:} H\$.
\zh{父亲。} \textcolor{Sepia}{\selectlanguage{english}Father.} \textcolor{PineGreen}{\selectlanguage{french}Père.}  \zh{量词}: \textcolor{darkblue}{\textbf{\ipa{v̩˧}}} 
\lhead{\firstmark}
\rhead{\botmark}

\subsection{\hspace{-0.5cm} {\Large \textcolor{darkblue}{\textbf{\ipa{ə˧dɑ˧-ə˧mi\#˥}}}}\hspace{0.5cm}[\kern2pt{\textcolor{darkblue}{\textbf{\ipa{xxxx non-correspondance entre le nombre de morphèmes et le nombre de tons de morphèmes}}}}\kern2pt]} \hypertarget{@\string_MdA\string_M-@\string_Mmi\#\string_T1}{}
\markboth{\textcolor{darkblue}{\textbf{\ipa{ə˧dɑ˧-ə˧mi\#˥}}}}{}
\textcolor{teal}{\zh{名词}} \hspace{4pt} \zh{声调类:} \#H.
\zh{父母。} \textcolor{Sepia}{\selectlanguage{english}Father and mother.} \textcolor{PineGreen}{\selectlanguage{french}Père et mère.}  ¶ \textcolor{darkblue}{\textbf{\ipa{ə˧dɑ˧-ə˧mi˧ ɲi˥-kv̩˩}}} \zh{父母亲} \textcolor{Sepia}{\selectlanguage{english}the father and mother, as a pair} \textcolor{PineGreen}{\selectlanguage{french}le père et la mère, tous les deux; le couple formé du père et de la mère}  

\lhead{\firstmark}
\rhead{\botmark}

\subsection{\hspace{-0.5cm} {\Large \textcolor{darkblue}{\textbf{\ipa{ə˧dɑ˧-zo\#˥}}}}\hspace{0.5cm}[\kern2pt{\textcolor{darkblue}{\textbf{\ipa{xxxx non-correspondance entre le nombre de morphèmes et le nombre de tons de morphèmes}}}}\kern2pt]} \hypertarget{@\string_MdA\string_M-zo\#\string_T1}{}
\markboth{\textcolor{darkblue}{\textbf{\ipa{ə˧dɑ˧-zo\#˥}}}}{}
\textcolor{teal}{\zh{名词}} \hspace{4pt} \zh{声调类:} \#H.
\zh{父子。} \textcolor{Sepia}{\selectlanguage{english}Father and son.} \textcolor{PineGreen}{\selectlanguage{french}Père et fils.} 
\lhead{\firstmark}
\rhead{\botmark}

\subsection{\hspace{-0.5cm} {\Large \textcolor{darkblue}{\textbf{\ipa{ə˧dze˧}}}}\hspace{0.5cm}[\kern2pt{\textcolor{darkblue}{\textbf{\ipa{ə˧dze˧}}}}\kern2pt]} \hypertarget{@\string_Mdze\string_M1}{}
\markboth{\textcolor{darkblue}{\textbf{\ipa{ə˧dze˧}}}}{}
\textcolor{teal}{\zh{名词}} \hspace{4pt} \zh{声调类:} M.
\zh{紫草。} \textcolor{Sepia}{\selectlanguage{english}Purple gromwell, red-root gromwell, \textit{Lithospermum erythrorhizon Sieb. et Zucc.}.} \textcolor{PineGreen}{\selectlanguage{french}Grémil des teinturiers, \textit{Lithospermum erythrorhizon Sieb. et Zucc.}.} \zh{当地汉语方言:}\zh{紫红草。} ¶ \textcolor{darkblue}{\textbf{\ipa{ə˧dze˧-njɤ˩hṽ˩}}} \zh{紫草} \textcolor{Sepia}{\selectlanguage{english}same meaning: purple gromwell} \textcolor{PineGreen}{\selectlanguage{french}même sens}  
 ¶ \textcolor{darkblue}{\textbf{\ipa{ə˧dze˧-bæ˩bæ˩}}} \zh{紫草花} \textcolor{Sepia}{\selectlanguage{english}gromwell flowers} \textcolor{PineGreen}{\selectlanguage{french}fleurs de grémil}  

\lhead{\firstmark}
\rhead{\botmark}

\subsection{\hspace{-0.5cm} {\Large \textcolor{darkblue}{\textbf{\ipa{ə˧-dzɤ˥\$}}}}\hspace{0.5cm}[\kern2pt{\textcolor{darkblue}{\textbf{\ipa{xxxx non-correspondance entre le nombre de morphèmes et le nombre de tons de morphèmes}}}}\kern2pt]} \hypertarget{@\string_M-dz7\string_T\$1}{}
\markboth{\textcolor{darkblue}{\textbf{\ipa{ə˧-dzɤ˥\$}}}}{}
\textcolor{teal}{\zh{助词}} \hspace{4pt} \zh{声调类:} H\$.
\zh{慢。} \textcolor{Sepia}{\selectlanguage{english}Slowly.} \textcolor{PineGreen}{\selectlanguage{french}Lentement, doucement.}  ¶ \textcolor{darkblue}{\textbf{\ipa{ə˧-dzɤ˧ ʝi˧}}} \zh{慢慢做} \textcolor{Sepia}{\selectlanguage{english}to work slowly, to do slowly} \textcolor{PineGreen}{\selectlanguage{french}travailler lentement, faire lentement}  
 ¶ \textcolor{darkblue}{\textbf{\ipa{ə˧dzɤ˧ le˧-hõ˩! |}}} \zh{慢走!} \textcolor{Sepia}{\selectlanguage{english}Goodbye! (Said by the host to their guest. Literally: “Walk slowly!” = “Take your time on the way!”)} \textcolor{PineGreen}{\selectlanguage{french}Au revoir! (Dit par l'hôte à la personne qui s'en va. Littéralement: “Allez doucement!” / “Prenez votre temps en chemin!”)}  
 ¶ \textcolor{darkblue}{\textbf{\ipa{ə˧dzɤ˥ | le˧-hõ˩! |}}} \zh{慢走!} \textcolor{Sepia}{\selectlanguage{english}Goodbye!} \textcolor{PineGreen}{\selectlanguage{french}Au revoir!}  
 ¶ \textcolor{darkblue}{\textbf{\ipa{ə˧dzɤ˧ le˧-dzi˩! |}}} \zh{慢慢坐!} \textcolor{Sepia}{\selectlanguage{english}Goodbye! (Said by the guest to their host. Literally: “Sit quietly!” = “Take it easy!”)} \textcolor{PineGreen}{\selectlanguage{french}Au revoir! (Dit par l'invité à son hôte. Littéralement: “Restez assis doucement = tranquillement!”}  
\zh{~【参考】~} \textcolor{darkblue}{\textbf{\ipa{ə˧ze˧, ə˧-dzɤ˧\textasciitilde{}dzɤ˥}}} 
\lhead{\firstmark}
\rhead{\botmark}

\subsection{\hspace{-0.5cm} {\Large \textcolor{darkblue}{\textbf{\ipa{ə˧-dzɤ˧\textasciitilde{}dzɤ˥}}}}\hspace{0.5cm}[\kern2pt{\textcolor{darkblue}{\textbf{\ipa{xxxx non-correspondance entre le nombre de morphèmes et le nombre de tons de morphèmes}}}}\kern2pt]} \hypertarget{@\string_M-dz7\string_M~dz7\string_T1}{}
\markboth{\textcolor{darkblue}{\textbf{\ipa{ə˧-dzɤ˧\textasciitilde{}dzɤ˥}}}}{}
\textcolor{teal}{\zh{助词}} \hspace{4pt} \zh{声调类:} H\#.
\zh{慢慢地。} \textcolor{Sepia}{\selectlanguage{english}Slowly.} \textcolor{PineGreen}{\selectlanguage{french}Lentement, doucement.}  ¶ \textcolor{darkblue}{\textbf{\ipa{ʈʂʰɯ˧ | ɖwæ˧˥ | ə˧-dzɤ˧\textasciitilde{}dzɤ˥ ʝi˩-kv̩˩!}}} \zh{他工作很细致。(直译:‘他工作很慢’,但不是批评:意味着那个人懂得慢慢来做,做得更仔细。)} \textcolor{Sepia}{\selectlanguage{english}(S)he works very carefully. (The literal meaning is 'very slowly'; this is not a criticism, however: it means that they know to take their time in order to do a good job.)} \textcolor{PineGreen}{\selectlanguage{french}Il/elle travaille avec grand soin. (Le sens littéral est “Il/elle travaille très lentement”, mais cela n'est pas une critique: cela signifie qu'il/elle sait prendre le temps pour réaliser du bon travail.)}  
 ¶ \textcolor{darkblue}{\textbf{\ipa{[F5] ə˧-dzɤ˧\textasciitilde{}dzɤ˥ ʝi˩}}} \zh{慢慢地做} \textcolor{Sepia}{\selectlanguage{english}to do (something) slowly} \textcolor{PineGreen}{\selectlanguage{french}travailler lentement, faire lentement}  
 ¶ \textcolor{darkblue}{\textbf{\ipa{[M21] ə˧-zɤ˧\textasciitilde{}zɤ˥ ʝi˩}}} \zh{慢慢地做} \textcolor{Sepia}{\selectlanguage{english}to do (something) slowly} \textcolor{PineGreen}{\selectlanguage{french}Travaille doucement! / Prends ton temps! / travailler lentement, faire lentement}  
\zh{~【参考】~} \hyperlink{}{\textcolor{darkblue}{\textbf{\ipa{ə˧-dzɤ˥\$}}}} 
\lhead{\firstmark}
\rhead{\botmark}

\subsection{\hspace{-0.5cm} {\Large \textcolor{darkblue}{\textbf{\ipa{ə˧ɖo˧}}}}\hspace{0.5cm}[\kern2pt{\textcolor{darkblue}{\textbf{\ipa{ə˧ɖo˧}}}}\kern2pt]} \hypertarget{@\string_Md`o\string_M1}{}
\markboth{\textcolor{darkblue}{\textbf{\ipa{ə˧ɖo˧}}}}{}
\textcolor{teal}{\zh{名词}} \hspace{4pt} \zh{声调类:} M.
\zh{情人(音译:阿注)。} \textcolor{Sepia}{\selectlanguage{english}Lover, boy/girl-friend.} \textcolor{PineGreen}{\selectlanguage{french}Petit ami, petite amie, amant(e).}  \zh{量词}: \textcolor{darkblue}{\textbf{\ipa{v̩˧}}} \zh{~【参考】~} \hyperlink{}{\textcolor{darkblue}{\textbf{\ipa{ə˧ɕjɤ˩}}}} 
\lhead{\firstmark}
\rhead{\botmark}

\subsection{\hspace{-0.5cm} {\Large \textcolor{darkblue}{\textbf{\ipa{ə˧go˧}}}}\hspace{0.5cm}[\kern2pt{\textcolor{darkblue}{\textbf{\ipa{ə˧go˧}}}}\kern2pt]} \hypertarget{@\string_Mgo\string_M1}{}
\markboth{\textcolor{darkblue}{\textbf{\ipa{ə˧go˧}}}}{}
\textcolor{teal}{\zh{名词}} \hspace{4pt} \zh{声调类:} M.
\zh{一个姓。这个姓,永宁有三家。} \textcolor{Sepia}{\selectlanguage{english}A family name from Yongning. There are three families in Yongning that carry this name.} \textcolor{PineGreen}{\selectlanguage{french}Nom de clan/famille étendue. Trois familles portent ce nom à Yongning.}  ¶ \textcolor{darkblue}{\textbf{\ipa{ə˧go˧=ɻ̍˩}}} \zh{\textcolor{darkblue}{\textbf{\ipa{/ə˧go˧/}}}家族} \textcolor{Sepia}{\selectlanguage{english}the \textcolor{darkblue}{\textbf{\ipa{/ə˧go˧/}}} clan} \textcolor{PineGreen}{\selectlanguage{french}le clan \textcolor{darkblue}{\textbf{\ipa{/ə˧go˧/}}}}  
 ¶ \textcolor{darkblue}{\textbf{\ipa{ə˧go˧ | dʑɤ˩tsʰi˧}}} \zh{\textcolor{darkblue}{\textbf{\ipa{/ə˧go˧/}}} 家族名叫\textcolor{darkblue}{\textbf{\ipa{/dʑɤ˩tsʰi\#˥/}}}那个人} \textcolor{Sepia}{\selectlanguage{english}The person called \textcolor{darkblue}{\textbf{\ipa{/dʑɤ˩tsʰi\#˥/}}}, of the \textcolor{darkblue}{\textbf{\ipa{/ə˧go˧/}}} clan} \textcolor{PineGreen}{\selectlanguage{french}la personne prénommée \textcolor{darkblue}{\textbf{\ipa{/dʑɤ˩tsʰi\#˥/}}}, du clan \textcolor{darkblue}{\textbf{\ipa{/ə˧go˧/}}}}  

\lhead{\firstmark}
\rhead{\botmark}

\subsection{\hspace{-0.5cm} {\Large \textcolor{darkblue}{\textbf{\ipa{ə˧go˧-ʁwɤ˧}}}}\hspace{0.5cm}[\kern2pt{\textcolor{darkblue}{\textbf{\ipa{xxxx non-correspondance entre le nombre de morphèmes et le nombre de tons de morphèmes}}}}\kern2pt]} \hypertarget{@\string_Mgo\string_M-Rw7\string_M1}{}
\markboth{\textcolor{darkblue}{\textbf{\ipa{ə˧go˧-ʁwɤ˧}}}}{}
\textcolor{teal}{\zh{名词}} \hspace{4pt} \zh{声调类:} M.
\zh{温泉乡的一个村落。} \textcolor{Sepia}{\selectlanguage{english}Name of a village of the Hot Springs area.} \textcolor{PineGreen}{\selectlanguage{french}Un village proche de Wenquan.}  ¶ \textcolor{darkblue}{\textbf{\ipa{ə˧go˧-ʁwɤ˧, | ʁwɤ˧lɑ˩-bi˩, | bæ˧ʁwɤ˧, | tʰo˧tsʰe\#˥, | pi˧tsʰe˩-di˩, | pɤ˧dʑɤ˩-di˩, | ʁwɤ˧tv̩˧}}} \zh{永宁背向泸沽湖方向经过的村落。前两个村落拥有相当大的摩梭人口比例,第三个村落是摩梭村,最后一个是普米村。} \textcolor{Sepia}{\selectlanguage{english}Villages that one encounters as one leaves the plain of Yongning (away from the Lake); the first two are perceived as villages with a high proportion of Na members, and the third as a mostly Na village, whereas the next ones are Pumi (Prinmi).} \textcolor{PineGreen}{\selectlanguage{french}Villages au sortir de la plaine de Yongning; les deux premiers comportent une population na; le troisième est un village na; les suivants sont essentiellement des villages pumi/prinmi.}  

\lhead{\firstmark}
\rhead{\botmark}

\subsection{\hspace{-0.5cm} {\Large \textcolor{darkblue}{\textbf{\ipa{ə˧gɯ˩}}}}\hspace{0.5cm}[\kern2pt{\textcolor{darkblue}{\textbf{\ipa{ə˧gɯ˩}}}}\kern2pt]} \hypertarget{@\string_MgM\string_B1}{}
\markboth{\textcolor{darkblue}{\textbf{\ipa{ə˧gɯ˩}}}}{}
\textcolor{teal}{\zh{名词}} \hspace{4pt} \zh{声调类:} L\#.
\zh{薄荷。} \textcolor{Sepia}{\selectlanguage{english}Peppermint.} \textcolor{PineGreen}{\selectlanguage{french}Menthe.}  \zh{量词}: \textcolor{darkblue}{\textbf{\ipa{po˧}}} 
\lhead{\firstmark}
\rhead{\botmark}

\subsection{\hspace{-0.5cm} {\Large \textcolor{darkblue}{\textbf{\ipa{ə˧hɑ˩-bɑ˩lɑ˩}}}}\hspace{0.5cm}[\kern2pt{\textcolor{darkblue}{\textbf{\ipa{ə˧hɑ˩bɑ˧lɑ˧}}}}\kern2pt]} \hypertarget{@\string_MhA\string_B-bA\string_BlA\string_B1}{}
\markboth{\textcolor{darkblue}{\textbf{\ipa{ə˧hɑ˩-bɑ˩lɑ˩}}}}{}
\textcolor{teal}{\zh{名词}} \hspace{4pt} \zh{声调类:} L\#-.
\zh{民歌。} \textcolor{Sepia}{\selectlanguage{english}Traditional song.} \textcolor{PineGreen}{\selectlanguage{french}Chanson traditionnelle.}  ¶ \textcolor{darkblue}{\textbf{\ipa{ə˧hɑ˩bɑ˩lɑ˩ | ɖɯ˧-ɖʐo˩ gwɤ˩}}} \zh{唱一首摩梭歌} \textcolor{Sepia}{\selectlanguage{english}to sing a song} \textcolor{PineGreen}{\selectlanguage{french}chanter une chanson}  
 \zh{量词}: \textcolor{darkblue}{\textbf{\ipa{ɖʐo˩}}} 
\lhead{\firstmark}
\rhead{\botmark}

\subsection{\hspace{-0.5cm} {\Large \textcolor{darkblue}{\textbf{\ipa{ə˧hĩ˥}}}}\hspace{0.5cm}[\kern2pt{\textcolor{darkblue}{\textbf{\ipa{ə˧hĩ˥}}}}\kern2pt]} \hypertarget{@\string_Mhi\string_~\string_T1}{}
\markboth{\textcolor{darkblue}{\textbf{\ipa{ə˧hĩ˥}}}}{}
\textcolor{teal}{\zh{助词}} \hspace{4pt} \zh{声调类:} H\#.
\zh{一会儿、待会儿、等一下。} \textcolor{Sepia}{\selectlanguage{english}In a moment.} \textcolor{PineGreen}{\selectlanguage{french}Dans un moment.}  ¶ \textcolor{darkblue}{\textbf{\ipa{ə˧hĩ˥-ɳɯ˩, | li˧-kʰɯ˧-bi˥!}}} \zh{待会儿,我给你看吧!} \textcolor{Sepia}{\selectlanguage{english}I will show you in a moment!} \textcolor{PineGreen}{\selectlanguage{french}Tout à l’heure je vais te montrer! / dans un moment, je te montrerai!}  

\lhead{\firstmark}
\rhead{\botmark}

\subsection{\hspace{-0.5cm} {\Large \textcolor{darkblue}{\textbf{\ipa{ə˧hwɤ˧}}}}\hspace{0.5cm}[\kern2pt{\textcolor{darkblue}{\textbf{\ipa{ə˧hwɤ˧}}}}\kern2pt]} \hypertarget{@\string_Mhw7\string_M1}{}
\markboth{\textcolor{darkblue}{\textbf{\ipa{ə˧hwɤ˧}}}}{}
\textcolor{teal}{\zh{助词}} \hspace{4pt} \zh{声调类:} M.
\zh{昨晚。} \textcolor{Sepia}{\selectlanguage{english}Yesterday evening.} \textcolor{PineGreen}{\selectlanguage{french}Hier soir.}  ¶ \textcolor{darkblue}{\textbf{\ipa{ə˧hwɤ˧ | mv̩˩kʰv̩˧˥}}} \zh{昨晚,夜里} \textcolor{Sepia}{\selectlanguage{english}yesterday evening, during the night} \textcolor{PineGreen}{\selectlanguage{french}hier au soir, dans la nuit}  

\lhead{\firstmark}
\rhead{\botmark}

\subsection{\hspace{-0.5cm} {\Large \textcolor{darkblue}{\textbf{\ipa{ə˧jɤ˩}}}}\hspace{0.5cm}[\kern2pt{\textcolor{darkblue}{\textbf{\ipa{ə˧jɤ˩}}}}\kern2pt]} \hypertarget{@\string_Mj7\string_B1}{}
\markboth{\textcolor{darkblue}{\textbf{\ipa{ə˧jɤ˩}}}}{}
\textcolor{teal}{\zh{名词}} \hspace{4pt} \zh{声调类:} L\#.
\zh{姨母 (比母亲大)。} \textcolor{Sepia}{\selectlanguage{english}Maternal aunt (mother's elder sister).} \textcolor{PineGreen}{\selectlanguage{french}Tante maternelle (sœur aînée de la mère).}  \zh{量词}: \textcolor{darkblue}{\textbf{\ipa{v̩˧}}} 
\lhead{\firstmark}
\rhead{\botmark}

\subsection{\hspace{-0.5cm} {\Large \textcolor{darkblue}{\textbf{\ipa{ə˧ʝi˥\$}}}}\hspace{0.5cm}[\kern2pt{\textcolor{darkblue}{\textbf{\ipa{ə˧ʝi˥}}}}\kern2pt]} \hypertarget{@\string_Mj££i\string_T\$1}{}
\markboth{\textcolor{darkblue}{\textbf{\ipa{ə˧ʝi˥\$}}}}{}
\textcolor{teal}{\zh{助词}} \hspace{4pt} \zh{声调类:} H\$.
\zh{去年。} \textcolor{Sepia}{\selectlanguage{english}Last year.} \textcolor{PineGreen}{\selectlanguage{french}L'année dernière, l'année passée, l'an passé, l'an dernier.} 
\lhead{\firstmark}
\rhead{\botmark}

\subsection{\hspace{-0.5cm} {\Large \textcolor{darkblue}{\textbf{\ipa{ə˧ʝi˧-ʂɯ˥ʝi˩}}}}\hspace{0.5cm}[\kern2pt{\textcolor{darkblue}{\textbf{\ipa{ə˧ʝi˧ʂɯ˥ʝi˩}}}}\kern2pt]} \hypertarget{@\string_Mj££i\string_M-s`M\string_Tj££i\string_B1}{}
\markboth{\textcolor{darkblue}{\textbf{\ipa{ə˧ʝi˧-ʂɯ˥ʝi˩}}}}{}
\textcolor{teal}{\zh{助词}} \hspace{4pt} \zh{声调类:} \#H-.
\zh{很久以前,古时候,传说古代。} \textcolor{Sepia}{\selectlanguage{english}Long ago; in the past; once upon a time.} \textcolor{PineGreen}{\selectlanguage{french}Jadis, aux temps anciens, il était une fois.} 
\lhead{\firstmark}
\rhead{\botmark}

\subsection{\hspace{-0.5cm} {\Large \textcolor{darkblue}{\textbf{\ipa{ə˧ʝi˧-tsʰi˧ʝi\#˥}}}}\hspace{0.5cm}[\kern2pt{\textcolor{darkblue}{\textbf{\ipa{xxxx non-correspondance entre le nombre de morphèmes et le nombre de tons de morphèmes}}}}\kern2pt]} \hypertarget{@\string_Mj££i\string_M-ts\string_hi\string_Mj££i\#\string_T1}{}
\markboth{\textcolor{darkblue}{\textbf{\ipa{ə˧ʝi˧-tsʰi˧ʝi\#˥}}}}{}
\textcolor{teal}{\zh{助词}} \hspace{4pt} \zh{声调类:} \#H.
\zh{这几年、现在这个时代。} \textcolor{Sepia}{\selectlanguage{english}These years, currently.} \textcolor{PineGreen}{\selectlanguage{french}Ces années-ci, actuellement.} 
\lhead{\firstmark}
\rhead{\botmark}

\subsection{\hspace{-0.5cm} {\Large \textcolor{darkblue}{\textbf{\ipa{ə˧lɑ˧}}}}\hspace{0.5cm}[\kern2pt{\textcolor{darkblue}{\textbf{\ipa{ə˧lɑ˧}}}}\kern2pt]} \hypertarget{@\string_MlA\string_M1}{}
\markboth{\textcolor{darkblue}{\textbf{\ipa{ə˧lɑ˧}}}}{}
\textcolor{teal}{\zh{名词}} \hspace{4pt} \zh{声调类:} M.
\zh{一个姓。这个姓,永宁有三家。} \textcolor{Sepia}{\selectlanguage{english}A family name from Yongning. There are three families in Yongning that carry this name. This is one of the three first Na clans who settled in the vicinity of the monastery, the other two being \textcolor{darkblue}{\textbf{\ipa{/kɤ˧˥tʰɑ˩/}}} and \textcolor{darkblue}{\textbf{\ipa{/lɑ˧tʰɑ˧mi˥\$/}}}.} \textcolor{PineGreen}{\selectlanguage{french}Nom de clan/famille étendue. Trois familles portent ce nom à Yongning. C'est l'un des trois clans qui se sont établis les premiers dans le voisinage du monastère de Yongning, les deux autres étant \textcolor{darkblue}{\textbf{\ipa{/kɤ˧˥tʰɑ˩/}}} et \textcolor{darkblue}{\textbf{\ipa{/lɑ˧tʰɑ˧mi˥\$/}}}.}  ¶ \textcolor{darkblue}{\textbf{\ipa{ə˧lɑ˧=ɻ̍˩}}} \zh{\textcolor{darkblue}{\textbf{\ipa{/ə˧lɑ˧/}}}家族} \textcolor{Sepia}{\selectlanguage{english}the \textcolor{darkblue}{\textbf{\ipa{/ə˧lɑ˧/}}} clan} \textcolor{PineGreen}{\selectlanguage{french}le clan \textcolor{darkblue}{\textbf{\ipa{/ə˧lɑ˧/}}}}  

\lhead{\firstmark}
\rhead{\botmark}

\subsection{\hspace{-0.5cm} {\Large \textcolor{darkblue}{\textbf{\ipa{ə˧lɑ˧-ʁwɤ\#˥}}}}\hspace{0.5cm}[\kern2pt{\textcolor{darkblue}{\textbf{\ipa{xxxx non-correspondance entre le nombre de morphèmes et le nombre de tons de morphèmes}}}}\kern2pt]} \hypertarget{@\string_MlA\string_M-Rw7\#\string_T1}{}
\markboth{\textcolor{darkblue}{\textbf{\ipa{ə˧lɑ˧-ʁwɤ\#˥}}}}{}
\textcolor{teal}{\zh{名词}} \hspace{4pt} \zh{声调类:} \#H.
\zh{永宁寺旁边的村落(主合作人住的地方)。(音译:阿拉瓦,旧名:七家村,因为村落在1960年左右有七个家庭)。} \textcolor{Sepia}{\selectlanguage{english}A hamlet of Yongning, close to the monastery.} \textcolor{PineGreen}{\selectlanguage{french}Un hameau de Yongning, proche du monastère (lieu de naissance de la consultante principale). Nom chinois: Alawa.}  ¶ \textcolor{darkblue}{\textbf{\ipa{dʑɤ˩bv̩˧kɤ˧-sɑ˥ʁwɤ˩, | hi˩ʁwɤ˩-lo˥, | æ˩mi˧-ʁwɤ\#˥, | lɑ˧lo˧-ʁwɤ˥, | lɑ˧ŋwɤ˧, | bɤ˧tsʰo˧gv̩˥, | ə˧lɑ˧-ʁwɤ\#˥, | gæ˧ɻæ˩, | qʰæ˧tɕʰi˧, | tʰo˧ʈɯ\#˥}}} \zh{摩梭传统地理概念中,属于永宁的十个村落} \textcolor{Sepia}{\selectlanguage{english}the ten villages traditionally considered as part of Yongning} \textcolor{PineGreen}{\selectlanguage{french}les dix villages comptant traditionnellement comme faisant partie de Yongning}  

\lhead{\firstmark}
\rhead{\botmark}

\subsection{\hspace{-0.5cm} {\Large \textcolor{darkblue}{\textbf{\ipa{ə˧mɑ˧}}}}\hspace{0.5cm}[\kern2pt{\textcolor{darkblue}{\textbf{\ipa{ə˧mɑ˧}}}}\kern2pt]} \hypertarget{@\string_MmA\string_M1}{}
\markboth{\textcolor{darkblue}{\textbf{\ipa{ə˧mɑ˧}}}}{}
\textcolor{teal}{\zh{名词}} \hspace{4pt} \zh{声调类:} M.
\zh{阿妈(孩子对母亲的称呼)。} \textcolor{Sepia}{\selectlanguage{english}Mother (term of address used by children).} \textcolor{PineGreen}{\selectlanguage{french}Mère (terme d'adresse).}  \zh{量词}: \textcolor{darkblue}{\textbf{\ipa{v̩˧}}} 
\lhead{\firstmark}
\rhead{\botmark}

\subsection{\hspace{-0.5cm} {\Large \textcolor{darkblue}{\textbf{\ipa{ə˧mi˧}}}}\hspace{0.5cm}[\kern2pt{\textcolor{darkblue}{\textbf{\ipa{ə˧mi˧}}}}\kern2pt]} \hypertarget{@\string_Mmi\string_M1}{}
\markboth{\textcolor{darkblue}{\textbf{\ipa{ə˧mi˧}}}}{}
\textcolor{teal}{\zh{名词}} \hspace{4pt} \zh{声调类:} M.
\zh{母亲、姑母、姨母、伯母、叔母、大娘、婶、大妈、姨、伯母、舅母、大婶、大姨、阿姨、妗母、妗子、舅妈、婶母、婶娘、婶子、叔母、姨妈、姨母、姨娘。} \textcolor{Sepia}{\selectlanguage{english}Mother; aunt.} \textcolor{PineGreen}{\selectlanguage{french}Mère; le terme s'emploie aussi pour désigner les tantes.}  ¶ \textcolor{darkblue}{\textbf{\ipa{ə˧mi˧=ɻæ˩}}} \zh{母亲们 =长辈女性} \textcolor{Sepia}{\selectlanguage{english}\string_ \mytextsc{associative}} \textcolor{PineGreen}{\selectlanguage{french}\string_ \mytextsc{associatif}: les mères =les femmes de la génération supérieure}  
 \zh{量词}: \textcolor{darkblue}{\textbf{\ipa{v̩˧}}} \textcolor{darkblue}{\textbf{\ipa{jɤ˧˥}}} 
\lhead{\firstmark}
\rhead{\botmark}

\subsection{\hspace{-0.5cm} {\Large \textcolor{darkblue}{\textbf{\ipa{ə˧mi˧-ɖɯ˩}}}}\hspace{0.5cm}[\kern2pt{\textcolor{darkblue}{\textbf{\ipa{xxxx non-correspondance entre le nombre de morphèmes et le nombre de tons de morphèmes}}}}\kern2pt]} \hypertarget{@\string_Mmi\string_M-d`M\string_B1}{}
\markboth{\textcolor{darkblue}{\textbf{\ipa{ə˧mi˧-ɖɯ˩}}}}{}
\textcolor{teal}{\zh{名词}} \hspace{4pt} \zh{声调类:} L\#.
\zh{姨母 (比母亲大)。} \textcolor{Sepia}{\selectlanguage{english}Maternal aunt (mother's elder sister).} \textcolor{PineGreen}{\selectlanguage{french}Tante maternelle (sœur aînée de la mère).}  \zh{量词}: \textcolor{darkblue}{\textbf{\ipa{v̩˧}}} 
\lhead{\firstmark}
\rhead{\botmark}

\subsection{\hspace{-0.5cm} {\Large \textcolor{darkblue}{\textbf{\ipa{ə˧mi˧-tɕi˩}}}}\hspace{0.5cm}[\kern2pt{\textcolor{darkblue}{\textbf{\ipa{xxxx non-correspondance entre le nombre de morphèmes et le nombre de tons de morphèmes}}}}\kern2pt]} \hypertarget{@\string_Mmi\string_M-ts£i\string_B1}{}
\markboth{\textcolor{darkblue}{\textbf{\ipa{ə˧mi˧-tɕi˩}}}}{}
\textcolor{teal}{\zh{名词}} \hspace{4pt} \zh{声调类:} L\#.
\zh{姨母 (比母亲小)。} \textcolor{Sepia}{\selectlanguage{english}Maternal aunt (mother's younger sister).} \textcolor{PineGreen}{\selectlanguage{french}Tante (soeur cadette de la mère).}  \zh{量词}: \textcolor{darkblue}{\textbf{\ipa{v̩˧}}} 
\lhead{\firstmark}
\rhead{\botmark}

\subsection{\hspace{-0.5cm} {\Large \textcolor{darkblue}{\textbf{\ipa{ə˧mi˧-zo\#˥}}}}\hspace{0.5cm}[\kern2pt{\textcolor{darkblue}{\textbf{\ipa{xxxx non-correspondance entre le nombre de morphèmes et le nombre de tons de morphèmes}}}}\kern2pt]} \hypertarget{@\string_Mmi\string_M-zo\#\string_T1}{}
\markboth{\textcolor{darkblue}{\textbf{\ipa{ə˧mi˧-zo\#˥}}}}{}
\textcolor{teal}{\zh{名词}} \hspace{4pt} \zh{声调类:} \#H.
\zh{母子。} \textcolor{Sepia}{\selectlanguage{english}Mother and son.} \textcolor{PineGreen}{\selectlanguage{french}Mère et fils.} 
\lhead{\firstmark}
\rhead{\botmark}

\subsection{\hspace{-0.5cm} {\Large \textcolor{darkblue}{\textbf{\ipa{ə˧mv̩˩}}}}\hspace{0.5cm}[\kern2pt{\textcolor{darkblue}{\textbf{\ipa{ə˧mv̩˩}}}}\kern2pt]} \hypertarget{@\string_Mmv\string_=\string_B1}{}
\markboth{\textcolor{darkblue}{\textbf{\ipa{ə˧mv̩˩}}}}{}
\textcolor{teal}{\zh{名词}} \hspace{4pt} \zh{声调类:} L\#.
\zh{哥哥,姐姐(也指堂哥堂姐)。} \textcolor{Sepia}{\selectlanguage{english}Elder sibling (brother or sister).} \textcolor{PineGreen}{\selectlanguage{french}Aîné: grand frère, grande sœur (employé aussi entre cousins).}  ¶ \textcolor{darkblue}{\textbf{\ipa{æ˧mv̩˩=ɻæ˩}}} \zh{联想复数:哥哥们、姐姐们} \textcolor{Sepia}{\selectlanguage{english}\mytextsc{associative}: elder siblings} \textcolor{PineGreen}{\selectlanguage{french}\mytextsc{associatif}: les aînés dans la fratrie: sœurs et frères aînés}  
 \zh{量词}: \textcolor{darkblue}{\textbf{\ipa{v̩˧}}} 
\lhead{\firstmark}
\rhead{\botmark}

\subsection{\hspace{-0.5cm} {\Large \textcolor{darkblue}{\textbf{\ipa{ə˧mv̩˧-gi˥zɯ˩}}}}\hspace{0.5cm}[\kern2pt{\textcolor{darkblue}{\textbf{\ipa{ə˧mv̩˧gi˥zɯ˩}}}}\kern2pt]} \hypertarget{@\string_Mmv\string_=\string_M-gi\string_TzM\string_B1}{}
\markboth{\textcolor{darkblue}{\textbf{\ipa{ə˧mv̩˧-gi˥zɯ˩}}}}{}
\textcolor{teal}{\zh{名词}} \hspace{4pt} \zh{声调类:} \#H-.
\zh{兄弟(哥哥们与弟弟们)。} \textcolor{Sepia}{\selectlanguage{english}Brothers, irrespective of age (elder or younger).} \textcolor{PineGreen}{\selectlanguage{french}Frères, quel que soit leur âge (aînés ou cadets).} 
\lhead{\firstmark}
\rhead{\botmark}

\subsection{\hspace{-0.5cm} {\Large \textcolor{darkblue}{\textbf{\ipa{ə˧mv̩˥-tɕi˩}}}}\hspace{0.5cm}[\kern2pt{\textcolor{darkblue}{\textbf{\ipa{xxxx non-correspondance entre le nombre de morphèmes et le nombre de tons de morphèmes}}}}\kern2pt]} \hypertarget{@\string_Mmv\string_=\string_T-ts£i\string_B1}{}
\markboth{\textcolor{darkblue}{\textbf{\ipa{ə˧mv̩˥-tɕi˩}}}}{}
\textcolor{teal}{\zh{形容词}} \hspace{4pt} \zh{声调类:} H\#.
\zh{小、细小。} \textcolor{Sepia}{\selectlanguage{english}Very small, diminutive.} \textcolor{PineGreen}{\selectlanguage{french}Petit, tout petit, riquiqui.}  ¶ \textcolor{darkblue}{\textbf{\ipa{ə˧mv̩˥-tɕi˩-gv̩˩}}} \zh{小、细小} \textcolor{Sepia}{\selectlanguage{english}very small, diminutive} \textcolor{PineGreen}{\selectlanguage{french}tout petit}  
 ¶ \textcolor{darkblue}{\textbf{\ipa{ə˧mv̩˥-tɕi˩-hĩ˩}}} \zh{细小的} \textcolor{Sepia}{\selectlanguage{english}very small} \textcolor{PineGreen}{\selectlanguage{french}tout petit}  
 ¶ \textcolor{darkblue}{\textbf{\ipa{ə˧mv̩˥tɕi˩ | ɖɯ˧-kʰwɤ˥}}} \zh{一小块} \textcolor{Sepia}{\selectlanguage{english}a little piece, a little bit} \textcolor{PineGreen}{\selectlanguage{french}un petit morceau}  

\lhead{\firstmark}
\rhead{\botmark}

\subsection{\hspace{-0.5cm} {\Large \textcolor{darkblue}{\textbf{\ipa{ə˧ɲi˥\$}}}}\hspace{0.5cm}[\kern2pt{\textcolor{darkblue}{\textbf{\ipa{ə˧ɲi˥}}}}\kern2pt]} \hypertarget{@\string_MJi\string_T\$1}{}
\markboth{\textcolor{darkblue}{\textbf{\ipa{ə˧ɲi˥\$}}}}{}
\textcolor{teal}{\zh{助词}} \hspace{4pt} \zh{声调类:} H\$.
\zh{昨天。} \textcolor{Sepia}{\selectlanguage{english}Yesterday.} \textcolor{PineGreen}{\selectlanguage{french}Hier.} 
\lhead{\firstmark}
\rhead{\botmark}

\subsection{\hspace{-0.5cm} {\Large \textcolor{darkblue}{\textbf{\ipa{ə˧ɲi˥-tsæ˩qæ˩}}}}\hspace{0.5cm}[\kern2pt{\textcolor{darkblue}{\textbf{\ipa{ə˧ɲi˥tsæ˩qæ˩}}}}\kern2pt]} \hypertarget{@\string_MJi\string_T-ts\{\string_Bq\{\string_B1}{}
\markboth{\textcolor{darkblue}{\textbf{\ipa{ə˧ɲi˥-tsæ˩qæ˩}}}}{}
\textcolor{teal}{\zh{名词}} \hspace{4pt} \zh{声调类:} H\#-L.
\zh{小指。} \textcolor{Sepia}{\selectlanguage{english}Little finger.} \textcolor{PineGreen}{\selectlanguage{french}Auriculaire.}  \zh{量词}: \textcolor{darkblue}{\textbf{\ipa{ɭɯ˧}}} 
\lhead{\firstmark}
\rhead{\botmark}

\subsection{\hspace{-0.5cm} {\Large \textcolor{darkblue}{\textbf{\ipa{ə˧ɲi˧-tsʰi˧ɲi\#˥}}}}\hspace{0.5cm}[\kern2pt{\textcolor{darkblue}{\textbf{\ipa{xxxx non-correspondance entre le nombre de morphèmes et le nombre de tons de morphèmes}}}}\kern2pt]} \hypertarget{@\string_MJi\string_M-ts\string_hi\string_MJi\#\string_T1}{}
\markboth{\textcolor{darkblue}{\textbf{\ipa{ə˧ɲi˧-tsʰi˧ɲi\#˥}}}}{}
\textcolor{teal}{\zh{助词}} \hspace{4pt} \zh{声调类:} \#H.
\zh{近来。} \textcolor{Sepia}{\selectlanguage{english}These days.} \textcolor{PineGreen}{\selectlanguage{french}Ces temps-ci, ces jours-ci.} 
\lhead{\firstmark}
\rhead{\botmark}

\subsection{\hspace{-0.5cm} {\Large \textcolor{darkblue}{\textbf{\ipa{ə˧pʰv̩˧}}}}\hspace{0.5cm}[\kern2pt{\textcolor{darkblue}{\textbf{\ipa{ə˧pʰv̩˧}}}}\kern2pt]} \hypertarget{@\string_Mp\string_hv\string_=\string_M1}{}
\markboth{\textcolor{darkblue}{\textbf{\ipa{ə˧pʰv̩˧}}}}{}
\textcolor{teal}{\zh{名词}} \hspace{4pt} \zh{声调类:} M.
\zh{舅姥爷:姥姥的哥哥或弟弟(也就是母亲的舅舅)。泛指:“祖父”。} \textcolor{Sepia}{\selectlanguage{english}Grandmother's brother=mother's uncle (on her mother's side); extended meaning: male elder two generations above oneself.} \textcolor{PineGreen}{\selectlanguage{french}Frère de la grand-mère = oncle de l'oncle maternel = oncle de la mère; sens étendu: personnage masculin important 2 générations au-dessus de soi.}  \zh{量词}: \textcolor{darkblue}{\textbf{\ipa{v̩˧}}} 
\lhead{\firstmark}
\rhead{\botmark}

\subsection{\hspace{-0.5cm} {\Large \textcolor{darkblue}{\textbf{\ipa{ə˧si˧}}}}\hspace{0.5cm}[\kern2pt{\textcolor{darkblue}{\textbf{\ipa{ə˧si˧}}}}\kern2pt]} \hypertarget{@\string_Msi\string_M1}{}
\markboth{\textcolor{darkblue}{\textbf{\ipa{ə˧si˧}}}}{}
\textcolor{teal}{\zh{名词}} \hspace{4pt} \zh{声调类:} M.
\zh{祖母。泛指:祖母与其兄弟姐妹。} \textcolor{Sepia}{\selectlanguage{english}Great-grandmother (on the mother's side); extended meaning: great-grandmother and her brothers and sisters: third-generation elders.} \textcolor{PineGreen}{\selectlanguage{french}Arrière-grand-mère (bisaïeule); sens étendu: bisaïeule et ses frères et soeurs, c'est-à-dire les membres importants de la famille à la 3e génération.}  \zh{量词}: \textcolor{darkblue}{\textbf{\ipa{v̩˧}}} 
\lhead{\firstmark}
\rhead{\botmark}

\subsection{\hspace{-0.5cm} {\Large \textcolor{darkblue}{\textbf{\ipa{ə˧si˧-ə˧pʰv̩\#˥}}}}\hspace{0.5cm}[\kern2pt{\textcolor{darkblue}{\textbf{\ipa{xxxx non-correspondance entre le nombre de morphèmes et le nombre de tons de morphèmes}}}}\kern2pt]} \hypertarget{@\string_Msi\string_M-@\string_Mp\string_hv\string_=\#\string_T1}{}
\markboth{\textcolor{darkblue}{\textbf{\ipa{ə˧si˧-ə˧pʰv̩\#˥}}}}{}
\textcolor{teal}{\zh{名词}} \hspace{4pt} \zh{声调类:} \#H.
\zh{祖宗(三、四代以前)。} \textcolor{Sepia}{\selectlanguage{english}Ancestors of the third and fourth generations.} \textcolor{PineGreen}{\selectlanguage{french}Ancêtres aux 3e et 4e générations.} 
\lhead{\firstmark}
\rhead{\botmark}

\subsection{\hspace{-0.5cm} {\Large \textcolor{darkblue}{\textbf{\ipa{ə˧so˧}}}}\hspace{0.5cm}[\kern2pt{\textcolor{darkblue}{\textbf{\ipa{ə˧so˧}}}}\kern2pt]} \hypertarget{@\string_Mso\string_M1}{}
\markboth{\textcolor{darkblue}{\textbf{\ipa{ə˧so˧}}}}{}
\textcolor{teal}{\zh{助词}} \hspace{4pt} \zh{声调类:} M.
\zh{刚才。} \textcolor{Sepia}{\selectlanguage{english}A short time ago, a moment ago.} \textcolor{PineGreen}{\selectlanguage{french}Tout à l'heure, il y a un moment.} 
\lhead{\firstmark}
\rhead{\botmark}

\subsection{\hspace{-0.5cm} {\Large \textcolor{darkblue}{\textbf{\ipa{ə˧ʂɯ˧ɲi˥-ɖɯ˧ɲi˥}}}}\hspace{0.5cm}[\kern2pt{\textcolor{darkblue}{\textbf{\ipa{ə˧ʂɯ˧ɲi˥ɖɯ˩ɲi˩}}}}\kern2pt]} \hypertarget{@\string_Ms`M\string_MJi\string_T-d`M\string_MJi\string_T1}{}
\markboth{\textcolor{darkblue}{\textbf{\ipa{ə˧ʂɯ˧ɲi˥-ɖɯ˧ɲi˥}}}}{}
\textcolor{teal}{\zh{助词}} \hspace{4pt} \zh{声调类:} H\#-H\$.
\zh{前几天。} \textcolor{Sepia}{\selectlanguage{english}The past few days.} \textcolor{PineGreen}{\selectlanguage{french}Ces derniers jours, les jours passés.} 
\lhead{\firstmark}
\rhead{\botmark}

\subsection{\hspace{-0.5cm} {\Large \textcolor{darkblue}{\textbf{\ipa{ə˧ti˥-dzi˩}}}}\hspace{0.5cm}[\kern2pt{\textcolor{darkblue}{\textbf{\ipa{ə˧ti˥dzi˩}}}}\kern2pt]} \hypertarget{@\string_Mti\string_T-dzi\string_B1}{}
\markboth{\textcolor{darkblue}{\textbf{\ipa{ə˧ti˥-dzi˩}}}}{}
\textcolor{teal}{\zh{名词}} \hspace{4pt} \zh{声调类:} H\#-.
\zh{维西。} \textcolor{Sepia}{\selectlanguage{english}The city of Weixi, in Yunnan.} \textcolor{PineGreen}{\selectlanguage{french}Weixi (localité du Yunnan).} 
\lhead{\firstmark}
\rhead{\botmark}

\subsection{\hspace{-0.5cm} {\Large \textcolor{darkblue}{\textbf{\ipa{ə˧tɕi˩}}}}\hspace{0.5cm}[\kern2pt{\textcolor{darkblue}{\textbf{\ipa{ə˧tɕi˩}}}}\kern2pt]} \hypertarget{@\string_Mts£i\string_B1}{}
\markboth{\textcolor{darkblue}{\textbf{\ipa{ə˧tɕi˩}}}}{}
\textcolor{teal}{\zh{名词}} \hspace{4pt} \zh{声调类:} L\#.
\zh{姨母 (比母亲小)。} \textcolor{Sepia}{\selectlanguage{english}Maternal aunt (mother's younger sister).} \textcolor{PineGreen}{\selectlanguage{french}Tante (soeur cadette de la mère).}  ¶ \textcolor{darkblue}{\textbf{\ipa{ə˧tɕi˩=ɻæ˩}}} \zh{姨母们} \textcolor{Sepia}{\selectlanguage{english}\mytextsc{associative}: aunts} \textcolor{PineGreen}{\selectlanguage{french}\mytextsc{associatif}: les tantes}  
 \zh{量词}: \textcolor{darkblue}{\textbf{\ipa{v̩˧}}} 
\lhead{\firstmark}
\rhead{\botmark}

\subsection{\hspace{-0.5cm} {\Large \textcolor{darkblue}{\textbf{\ipa{ə˧tse˥\$}}}}\hspace{0.5cm}[\kern2pt{\textcolor{darkblue}{\textbf{\ipa{ə˧tse˥}}}}\kern2pt]} \hypertarget{@\string_Mtse\string_T\$1}{}
\markboth{\textcolor{darkblue}{\textbf{\ipa{ə˧tse˥\$}}}}{}
\textcolor{teal}{\zh{助词}} \hspace{4pt} \zh{声调类:} H\$.
\zh{为什么。} \textcolor{Sepia}{\selectlanguage{english}Why.} \textcolor{PineGreen}{\selectlanguage{french}Pourquoi.}  ¶ \textcolor{darkblue}{\textbf{\ipa{ə˧tse˧-ʝi˥ / ə˧tse˧-ʝi˧}}} \zh{为什么?(有两个变体,意思一致)} \textcolor{Sepia}{\selectlanguage{english}Why? (Two variants; same meaning)} \textcolor{PineGreen}{\selectlanguage{french}Pourquoi? (Deux variantes, même sens)}  
 ¶ \textcolor{darkblue}{\textbf{\ipa{no˧ | ə˧tse˧-ʝi˥ | mɤ˧-tsʰɯ˩ ɲi˩? / no˧ | ə˧tse˧-ʝi˧-zo˥ | mɤ˧-tsʰɯ˩ ɲi˩?}}} \zh{你为什么没有来?} \textcolor{Sepia}{\selectlanguage{english}Why didn't you come?} \textcolor{PineGreen}{\selectlanguage{french}pourquoi tu ne viens pas/n'es pas venu?}  
 ¶ \textcolor{darkblue}{\textbf{\ipa{no˧ | ə˧tse˧-ʝi˥ | mɤ˧-dzɯ˥? = no˧ | ə˧tse˧-ʝi˥ | mɤ˧-dzɯ˧-ɲi˥?}}} \zh{你为什么不吃?} \textcolor{Sepia}{\selectlanguage{english}Why don't you eat?} \textcolor{PineGreen}{\selectlanguage{french}pourquoi tu ne manges pas?}  
 ¶ \textcolor{darkblue}{\textbf{\ipa{ʈʂʰɯ˧ | ə˧tse˧-ɲi˥-hɯ˩?}}} \zh{这是怎么一回事?} \textcolor{Sepia}{\selectlanguage{english}What does that mean?} \textcolor{PineGreen}{\selectlanguage{french}qu'est-ce que ça veut dire?}  

\lhead{\firstmark}
\rhead{\botmark}

\subsection{\hspace{-0.5cm} {\Large \textcolor{darkblue}{\textbf{\ipa{ə˧tso˧}}}}\hspace{0.5cm}[\kern2pt{\textcolor{darkblue}{\textbf{\ipa{ə˧tso˧}}}}\kern2pt]} \hypertarget{@\string_Mtso\string_M1}{}
\markboth{\textcolor{darkblue}{\textbf{\ipa{ə˧tso˧}}}}{}
\textcolor{teal}{\zh{代词}} \hspace{4pt} \zh{声调类:} M.
\zh{什么。} \textcolor{Sepia}{\selectlanguage{english}What.} \textcolor{PineGreen}{\selectlanguage{french}\mytextsc{interrog}.quoi (quoi, pronom interrogatif).}  ¶ \textcolor{darkblue}{\textbf{\ipa{ə˧tso˧ ɲi˩?}}} \zh{是什么?} \textcolor{Sepia}{\selectlanguage{english}What is it?} \textcolor{PineGreen}{\selectlanguage{french}Qu'est-ce que c'est?}  
 ¶ \textcolor{darkblue}{\textbf{\ipa{no˧ | ə˧tso˧ ʝi˧ bi˧?}}} \zh{你要做什么?} \textcolor{Sepia}{\selectlanguage{english}What are you going to do?} \textcolor{PineGreen}{\selectlanguage{french}Qu'est-ce que tu vas faire? (Cette phrase peut se substituer à “où tu vas?”, \textcolor{darkblue}{\textbf{\ipa{/zo˩qo˧ bi˧?/}}}, comme salutation adressée à quelqu'un qui est en chemin)}  
 ¶ \textcolor{darkblue}{\textbf{\ipa{no˧ | ə˧tse˧ bi˧?}}} \zh{上面例子的缩短格式:\textcolor{darkblue}{\textbf{\ipa{/tso˧/}}}与\textcolor{darkblue}{\textbf{\ipa{/ʝi˧/}}}合成一个音节,\textcolor{darkblue}{\textbf{\ipa{[tse˧]}}}。} \textcolor{Sepia}{\selectlanguage{english}contracted form of the previous example: \textcolor{darkblue}{\textbf{\ipa{/tso˧/}}} and the following \textcolor{darkblue}{\textbf{\ipa{/ʝi˧/}}} fuse into a single syllable, \textcolor{darkblue}{\textbf{\ipa{[tse˧]}}}.} \textcolor{PineGreen}{\selectlanguage{french}forme contractée de 2: dans \textcolor{darkblue}{\textbf{\ipa{[tse˧]}}}, \textcolor{darkblue}{\textbf{\ipa{/tso˧/}}} et le \textcolor{darkblue}{\textbf{\ipa{/ʝi˧/}}} suivant fusionnent en une seule syllabe.}  

\lhead{\firstmark}
\rhead{\botmark}

\subsection{\hspace{-0.5cm} {\Large \textcolor{darkblue}{\textbf{\ipa{ə˧tso˧\textasciitilde{}ə˧tso˥}}}}\hspace{0.5cm}[\kern2pt{\textcolor{darkblue}{\textbf{\ipa{ə˧tso˧ə˧tso˥}}}}\kern2pt]} \hypertarget{@\string_Mtso\string_M~@\string_Mtso\string_T1}{}
\markboth{\textcolor{darkblue}{\textbf{\ipa{ə˧tso˧\textasciitilde{}ə˧tso˥}}}}{}
\textcolor{teal}{\zh{代词}} \hspace{4pt} \zh{声调类:} H\#.
\zh{什么(重叠)。} \textcolor{Sepia}{\selectlanguage{english}What (reduplicated).} \textcolor{PineGreen}{\selectlanguage{french}\mytextsc{interrog}.quoi, rédupliqué.} 
\lhead{\firstmark}
\rhead{\botmark}

\subsection{\hspace{-0.5cm} {\Large \textcolor{darkblue}{\textbf{\ipa{ə˧tso˧-mɤ˧-ɲi˩}}}}\hspace{0.5cm}[\kern2pt{\textcolor{darkblue}{\textbf{\ipa{xxxx non-correspondance entre le nombre de morphèmes et le nombre de tons de morphèmes}}}}\kern2pt]} \hypertarget{@\string_Mtso\string_M-m7\string_M-Ji\string_B1}{}
\markboth{\textcolor{darkblue}{\textbf{\ipa{ə˧tso˧-mɤ˧-ɲi˩}}}}{}
\textcolor{teal}{\zh{代词}} \hspace{4pt} \zh{声调类:} L\#.
\zh{各种。} \textcolor{Sepia}{\selectlanguage{english}All, all sorts of.} \textcolor{PineGreen}{\selectlanguage{french}Tout, toutes les sortes de.} 
\lhead{\firstmark}
\rhead{\botmark}

\subsection{\hspace{-0.5cm} {\Large \textcolor{darkblue}{\textbf{\ipa{ə˧v̩˧˥}}} \textsubscript{1}}\hspace{0.5cm}[\kern2pt{\textcolor{darkblue}{\textbf{\ipa{ə˧v̩˧˥}}}}\kern2pt]} \hypertarget{@\string_Mv\string_=\string_M\string_T1}{}
\markboth{\textcolor{darkblue}{\textbf{\ipa{ə˧v̩˧˥}}} \textsubscript{1}}{}
\textcolor{teal}{\zh{形容词}} \hspace{4pt} \zh{声调类:} MH\#.
\zh{美,好看,美丽。} \textcolor{Sepia}{\selectlanguage{english}Beautiful, pretty.} \textcolor{PineGreen}{\selectlanguage{french}Beau, joli.}  ¶ \textcolor{darkblue}{\textbf{\ipa{ɖwæ˧˥ | ə˧v̩˧˥}}} \zh{很好看!} \textcolor{Sepia}{\selectlanguage{english}\mytextsc{intensive}.very} \textcolor{PineGreen}{\selectlanguage{french}\mytextsc{intensif}.très}  
 ¶ \textcolor{darkblue}{\textbf{\ipa{ə˧-mɤ˧-v̩˧˥}}} \zh{丑陋} \textcolor{Sepia}{\selectlanguage{english}\mytextsc{neg}: ugly} \textcolor{PineGreen}{\selectlanguage{french}\mytextsc{neg}: vilain, laid}  

\lhead{\firstmark}
\rhead{\botmark}

\subsection{\hspace{-0.5cm} {\Large \textcolor{darkblue}{\textbf{\ipa{ə˧v̩˧˥}}} \textsubscript{2}}\hspace{0.5cm}[\kern2pt{\textcolor{darkblue}{\textbf{\ipa{ə˧v̩˧˥}}}}\kern2pt]} \hypertarget{@\string_Mv\string_=\string_M\string_T2}{}
\markboth{\textcolor{darkblue}{\textbf{\ipa{ə˧v̩˧˥}}} \textsubscript{2}}{}
\textcolor{teal}{\zh{名词}} \hspace{4pt} \zh{声调类:} MH\#.
\zh{舅舅、舅父 (比母亲大或比母亲小不区分)。} \textcolor{Sepia}{\selectlanguage{english}Maternal uncle (mother's brother: same term for older brother and younger brother).} \textcolor{PineGreen}{\selectlanguage{french}Oncle maternel =frère de la mère (aîné ou cadet).}  ¶ \textcolor{darkblue}{\textbf{\ipa{ə˧v̩˧-ɖɯ˧˥}}} \zh{比母亲大的舅舅} \textcolor{Sepia}{\selectlanguage{english}mother's elder brother} \textcolor{PineGreen}{\selectlanguage{french}oncle, aîné de la mère}  
 ¶ \textcolor{darkblue}{\textbf{\ipa{ə˧v̩˧-tɕi˥}}} \zh{比母亲小的舅舅} \textcolor{Sepia}{\selectlanguage{english}mother's younger brother} \textcolor{PineGreen}{\selectlanguage{french}oncle, cadet de la mère}  
 ¶ \textcolor{darkblue}{\textbf{\ipa{mv˧ʁo˥ | tʰi˧-dze˩, | kɤ˩-nɑ˧mi˧ ɖɯ˧˥ ! | di˧qo˧ ʈʰɯ˧-dʑo˩, | ə˧v˧ ɖɯ˧˥!}}} \zh{“天上飞的,是老鹰最大。天下走的,是舅舅最大。”} \textcolor{Sepia}{\selectlanguage{english}“As the Eagle is greatest of all that fly in the sky, so the Uncle is greatest of all that walk the earth.”} \textcolor{PineGreen}{\selectlanguage{french}“Parmi tout ce qui vole dans le ciel, l'aigle est le plus grand; parmi tout ce qui marche sur la terre, l'oncle est le plus grand.”}  
 ¶ \textcolor{darkblue}{\textbf{\ipa{mv˧ʁo˥ dze˩hĩ˩-dʑo˥, | kɤ˩-nɑ˧mi˧; | di˧qo˧ se˧-dʑo˩, | ə˧v˧˥!}}} \zh{“天上飞的,是老鹰最大。天下走的,是舅舅最大。”} \textcolor{Sepia}{\selectlanguage{english}“As the Eagle is greatest of all that fly in the sky, so the Uncle is greatest of all that walk the earth.”} \textcolor{PineGreen}{\selectlanguage{french}“Parmi tout ce qui vole dans le ciel, l'aigle est le plus grand; parmi tout ce qui marche sur la terre, l'oncle est le plus grand.”}  
 ¶ \textcolor{darkblue}{\textbf{\ipa{mv˧ʁo˥ dze˩hĩ˩˥ | -dʑo˥, | kɤ˩-nɑ˧mi˧; | di˧qo˧ se˧-dʑo˩, | ə˧v˧˥!}}} \zh{“天上飞的,是老鹰最大。天下走的,是舅舅最大。”} \textcolor{Sepia}{\selectlanguage{english}“As the Eagle is greatest of all that fly in the sky, so the Uncle is greatest of all that walk the earth.”} \textcolor{PineGreen}{\selectlanguage{french}“Parmi tout ce qui vole dans le ciel, l'aigle est le plus grand; parmi tout ce qui marche sur la terre, l'oncle est le plus grand.”}  
 ¶ \textcolor{darkblue}{\textbf{\ipa{mv˧ʁo˥ dze˩hĩ˩-dʑo˥, | kɤ˩-nɑ˧mi˧; | di˧qo˧-dʑo˧, | ə˧v˧˥!}}} \zh{“天上飞的,是老鹰最大。天下走的,是舅舅最大。”} \textcolor{Sepia}{\selectlanguage{english}“As the Eagle is greatest of all that fly in the sky, so the Uncle is greatest of all that walk the earth.”} \textcolor{PineGreen}{\selectlanguage{french}“Parmi tout ce qui vole dans le ciel, l'aigle est le plus grand; parmi tout ce qui marche sur la terre, l'oncle est le plus grand.”}  
 ¶ \textcolor{darkblue}{\textbf{\ipa{mv˧ʁo˥ | dze˩-hĩ˩-dʑo˥, | ɖɯ˩-hĩ˩-dʑo˥, | kɤ˩-nɑ˧mi˧! | mv˧di˧-qo˥ | ɖɯ˩-hĩ˩-dʑo˥, | ə˧v˧˥!}}} \zh{“天上飞的,是老鹰最大。天下走的,是舅舅最大。”} \textcolor{Sepia}{\selectlanguage{english}“As the Eagle is greatest of all that fly in the sky, so the Uncle is greatest of all that walk the earth.”} \textcolor{PineGreen}{\selectlanguage{french}“Parmi tout ce qui vole dans le ciel, l'aigle est le plus grand; parmi tout ce qui marche sur la terre, l'oncle est le plus grand.”}  
 \zh{量词}: \textcolor{darkblue}{\textbf{\ipa{v̩˧}}} 
\lhead{\firstmark}
\rhead{\botmark}

\subsection{\hspace{-0.5cm} {\Large \textcolor{darkblue}{\textbf{\ipa{ə˧v̩˧-ze˥v̩˩}}}}\hspace{0.5cm}[\kern2pt{\textcolor{darkblue}{\textbf{\ipa{ə˧v̩˥ze˩v̩˩}}}}\kern2pt]} \hypertarget{@\string_Mv\string_=\string_M-ze\string_Tv\string_=\string_B1}{}
\markboth{\textcolor{darkblue}{\textbf{\ipa{ə˧v̩˧-ze˥v̩˩}}}}{}
\textcolor{teal}{\zh{名词}} \hspace{4pt} \zh{声调类:} H\#-.
\zh{叔叔侄子。} \textcolor{Sepia}{\selectlanguage{english}Uncle and nephew.} \textcolor{PineGreen}{\selectlanguage{french}Oncle et neveu.} 
\lhead{\firstmark}
\rhead{\botmark}

\subsection{\hspace{-0.5cm} {\Large \textcolor{darkblue}{\textbf{\ipa{ə˧ze˧}}}}\hspace{0.5cm}[\kern2pt{\textcolor{darkblue}{\textbf{\ipa{ə˧ze˥}}}}\kern2pt]} \hypertarget{@\string_Mze\string_M1}{}
\markboth{\textcolor{darkblue}{\textbf{\ipa{ə˧ze˧}}}}{}
\textcolor{teal}{\zh{助词}} \hspace{4pt} \zh{声调类:} H\#.
\zh{慢慢地。} \textcolor{Sepia}{\selectlanguage{english}Slowly.} \textcolor{PineGreen}{\selectlanguage{french}Lentement, doucement.}  ¶ \textcolor{darkblue}{\textbf{\ipa{ə˧ze˧ le˧-hõ˩!}}} \zh{慢走!} \textcolor{Sepia}{\selectlanguage{english}Walk slowly! / Take your time on the road! / Have a quiet and pleasant journey! (Polite salutation to someone who is leaving.)} \textcolor{PineGreen}{\selectlanguage{french}Salutations à quelqu'un qui s'en va: “Au revoir!”, littéralement “Marche doucement!”}  
 ¶ \textcolor{darkblue}{\textbf{\ipa{ə˧ze˧ le˧-dzi˩!}}} \zh{慢慢坐!} \textcolor{Sepia}{\selectlanguage{english}Just stay seated! (Polite salutation when leaving someone.)} \textcolor{PineGreen}{\selectlanguage{french}Salutation lorsqu'on quitte quelqu'un: 'Au revoir!', littéralement 'Reste tranquillement assis!'}  
\zh{~【参考】~} \hyperlink{}{\textcolor{darkblue}{\textbf{\ipa{ə˧-dzɤ˥\$}}}} 
\lhead{\firstmark}
\rhead{\botmark}

\subsection{\hspace{-0.5cm} {\Large \textcolor{darkblue}{\textbf{\ipa{ə˧zo˩-ʁwɤ˩}}}}\hspace{0.5cm}[\kern2pt{\textcolor{darkblue}{\textbf{\ipa{ə˧zo˩ʁwɤ˧}}}}\kern2pt]} \hypertarget{@\string_Mzo\string_B-Rw7\string_B1}{}
\markboth{\textcolor{darkblue}{\textbf{\ipa{ə˧zo˩-ʁwɤ˩}}}}{}
\textcolor{teal}{\zh{名词}} \hspace{4pt} \zh{声调类:} L\#-.
\zh{温泉乡的一个村落。} \textcolor{Sepia}{\selectlanguage{english}A village close to the Hot Springs.} \textcolor{PineGreen}{\selectlanguage{french}Un village proche des Sources Chaudes.} 
\lhead{\firstmark}
\rhead{\botmark}

\subsection{\hspace{-0.5cm} {\Large \textcolor{darkblue}{\textbf{\ipa{ə˧=zɯ˩}}}}\hspace{0.5cm}[\kern2pt{\textcolor{darkblue}{\textbf{\ipa{ə˧zɯ˩}}}}\kern2pt]} \hypertarget{@\string_M=zM\string_B1}{}
\markboth{\textcolor{darkblue}{\textbf{\ipa{ə˧=zɯ˩}}}}{}
\textcolor{teal}{\zh{代词}} \hspace{4pt} \zh{声调类:} L\# / L.
\zh{咱们两个。} \textcolor{Sepia}{\selectlanguage{english}Dual inclusive first person pronoun: us two, the two of us (the speaker and the addressee).} \textcolor{PineGreen}{\selectlanguage{french}Pronom duel inclusif: nous deux (le locuteur et l'interlocuteur).} 
\lhead{\firstmark}
\rhead{\botmark}

\subsection{\hspace{-0.5cm} {\Large \textcolor{darkblue}{\textbf{\ipa{ə˧ʐv̩˩}}}}\hspace{0.5cm}[\kern2pt{\textcolor{darkblue}{\textbf{\ipa{ə˧ʐv̩˩}}}}\kern2pt]} \hypertarget{@\string_Mz`v\string_=\string_B1}{}
\markboth{\textcolor{darkblue}{\textbf{\ipa{ə˧ʐv̩˩}}}}{}
\textcolor{teal}{\zh{形容词}} \hspace{4pt} \zh{声调类:} L\#.
\zh{陈旧。} \textcolor{Sepia}{\selectlanguage{english}Old, used.} \textcolor{PineGreen}{\selectlanguage{french}Ancien, usagé.}  ¶ \textcolor{darkblue}{\textbf{\ipa{ʂe˧ ʐv̩˥}}} \zh{陈肉、不新鲜的肉} \textcolor{Sepia}{\selectlanguage{english}old meat, meat that is not fresh} \textcolor{PineGreen}{\selectlanguage{french}de la vieille viande, de la viande pas fraîche}  

\lhead{\firstmark}
\rhead{\botmark}

\subsection{\hspace{-0.5cm} {\Large \textcolor{darkblue}{\textbf{\ipa{ə˧ʑi˧-ə˧pʰv̩˧˥}}}}\hspace{0.5cm}[\kern2pt{\textcolor{darkblue}{\textbf{\ipa{xxxx non-correspondance entre le nombre de morphèmes et le nombre de tons de morphèmes}}}}\kern2pt]} \hypertarget{@\string_Mz£i\string_M-@\string_Mp\string_hv\string_=\string_M\string_T1}{}
\markboth{\textcolor{darkblue}{\textbf{\ipa{ə˧ʑi˧-ə˧pʰv̩˧˥}}}}{}
\textcolor{teal}{\zh{名词}} \hspace{4pt} \zh{声调类:} MH\#.
\zh{奶奶与她的兄弟。} \textcolor{Sepia}{\selectlanguage{english}Elders by two generations: the grandmother and her brothers.} \textcolor{PineGreen}{\selectlanguage{french}La grand-mère et ses frères: les aînés 2 génération au-dessus de soi.} 
\lhead{\firstmark}
\rhead{\botmark}

\subsection{\hspace{-0.5cm} {\Large \textcolor{darkblue}{\textbf{\ipa{ə˧ʑi˧-ʐv̩˥mi˩}}}}\hspace{0.5cm}[\kern2pt{\textcolor{darkblue}{\textbf{\ipa{ə˧ʑi˥ʐv̩˩mi˩}}}}\kern2pt]} \hypertarget{@\string_Mz£i\string_M-z`v\string_=\string_Tmi\string_B1}{}
\markboth{\textcolor{darkblue}{\textbf{\ipa{ə˧ʑi˧-ʐv̩˥mi˩}}}}{}
\textcolor{teal}{\zh{名词}} \hspace{4pt} \zh{声调类:} H\#-.
\zh{奶奶与孙女。} \textcolor{Sepia}{\selectlanguage{english}Grandmother and granddaughter.} \textcolor{PineGreen}{\selectlanguage{french}Grand-mère et petite-fille.} 
\lhead{\firstmark}
\rhead{\botmark}

\subsection{\hspace{-0.5cm} {\Large \textcolor{darkblue}{\textbf{\ipa{ə˧ʑi˧˥}}}}\hspace{0.5cm}[\kern2pt{\textcolor{darkblue}{\textbf{\ipa{ə˧ʑi˧˥}}}}\kern2pt]} \hypertarget{@\string_Mz£i\string_M\string_T1}{}
\markboth{\textcolor{darkblue}{\textbf{\ipa{ə˧ʑi˧˥}}}}{}
\textcolor{teal}{\zh{名词}} \hspace{4pt} \zh{声调类:} MH\#.
\zh{祖母,姥姥,老妪。} \textcolor{Sepia}{\selectlanguage{english}Grandmother (on mother's side); elderly woman.} \textcolor{PineGreen}{\selectlanguage{french}Grand-mère, aïeule; vieille femme.}  ¶ \textcolor{darkblue}{\textbf{\ipa{ə˧ʑi˧ ʝi˧ so˥-zo˩-ho˩-ze˩!}}} \zh{我要学习当老太太了!(情景:一位医生建议合作人不要坐在小凳子或者软沙发上了,而要坐更高的木头椅子。她幽默地说:“看来我是老年人了!”)} \textcolor{Sepia}{\selectlanguage{english}I shall have to learn to be a grandmother! / I shall have to learn to behave as a grandmother! (Humorous remark by the main consultant, after a doctor has advised her to avoid low, soft seats such as sofas and to adopt a taller wooden chair. Paraphrase: “I guess I have entered the category of elderly persons!”)} \textcolor{PineGreen}{\selectlanguage{french}Il va falloir que j'apprenne à me comporter (sagement) comme une grand-mère! (Contexte: remarque teintée d'humour de la consultante principale face à l'âge qui vient et ses soucis: un médecin lui déconseille les sofas/assises molles et lui recommande une chaise haute en bois; elle se fait la réflexion qu'elle a vieilli et doit maintenant apprendre à prendre des précautions.)}  
 \zh{量词}: \textcolor{darkblue}{\textbf{\ipa{v̩˧}}} 
\lhead{\firstmark}
\rhead{\botmark}

\subsection{\hspace{-0.5cm} {\Large \textcolor{darkblue}{\textbf{\ipa{ə˩‑}}}}\hspace{0.5cm}[\kern2pt{\textcolor{darkblue}{\textbf{\ipa{ə˩˥}}}}\kern2pt]} \hypertarget{@\string_B‑1}{}
\markboth{\textcolor{darkblue}{\textbf{\ipa{ə˩‑}}}}{}
\textcolor{teal}{\zh{代词}} \hspace{4pt} \zh{声调类:} L.
\zh{……吗?。} \textcolor{Sepia}{\selectlanguage{english}Total interrogation.} \textcolor{PineGreen}{\selectlanguage{french}Interrogation totale.}  ¶ \textcolor{darkblue}{\textbf{\ipa{dʑɯ˧ | ə˩-dʑo˧?}}} \zh{有谁吗?} \textcolor{Sepia}{\selectlanguage{english}Is there any water?} \textcolor{PineGreen}{\selectlanguage{french}est-ce qu’il y a de l’eau ?}  
 ¶ \textcolor{darkblue}{\textbf{\ipa{ə˩-ŋi˩˥ ?}}} \zh{对吗? / 对吧?} \textcolor{Sepia}{\selectlanguage{english}Is that right? / Is that correct? / ... isn't it?} \textcolor{PineGreen}{\selectlanguage{french}Est-ce que c’est ça ?/ C'est bien ça? ... n’est-ce pas ?}  

\lhead{\firstmark}
\rhead{\botmark}

\subsection{\hspace{-0.5cm} {\Large \textcolor{darkblue}{\textbf{\ipa{ə˩kʰɯ˩}}}}\hspace{0.5cm}[\kern2pt{\textcolor{darkblue}{\textbf{\ipa{ə˩kʰɯ˩˥}}}}\kern2pt]} \hypertarget{@\string_Bk\string_hM\string_B1}{}
\markboth{\textcolor{darkblue}{\textbf{\ipa{ə˩kʰɯ˩}}}}{}
\textcolor{teal}{\zh{名词}} \hspace{4pt} \zh{声调类:} L.
\zh{芜菁 、扁萝卜、大头菜、蔓菁。} \textcolor{Sepia}{\selectlanguage{english}Turnip, wild cabbage, \textit{Brassica rapa}.} \textcolor{PineGreen}{\selectlanguage{french}Navet, \textit{Brassica rapa}.}  ¶ \textcolor{darkblue}{\textbf{\ipa{ə˩kʰɯ˩-bv̩˧ | kʰɯ˩ʈɯ˩˥}}} \zh{芜菁的根} \textcolor{Sepia}{\selectlanguage{english}the root of wild cabbage} \textcolor{PineGreen}{\selectlanguage{french}racine de navet}  
 \zh{量词}: \textcolor{darkblue}{\textbf{\ipa{ɭɯ˧}}} 
\lhead{\firstmark}
\rhead{\botmark}

\subsection{\hspace{-0.5cm} {\Large \textcolor{darkblue}{\textbf{\ipa{ə˩ljɤ˩hæ̃˩ʂɯ˥-mo˩}}}}\hspace{0.5cm}[\kern2pt{\textcolor{darkblue}{\textbf{\ipa{ə˩ljɤ˩hæ̃˩ʂɯ˥mo˧}}}}\kern2pt]} \hypertarget{@\string_Blj7\string_Bh\{\string_~\string_Bs`M\string_T-mo\string_B1}{}
\markboth{\textcolor{darkblue}{\textbf{\ipa{ə˩ljɤ˩hæ̃˩ʂɯ˥-mo˩}}}}{}
\textcolor{teal}{\zh{名词}} \hspace{4pt} \zh{声调类:} L+H\#-.
\zh{柠檬黄蜡伞(一种菌子)。} \textcolor{Sepia}{\selectlanguage{english}A sort of mushroom: \textit{Hygrophorus lucorum Kalc hbr.}.} \textcolor{PineGreen}{\selectlanguage{french}Grand champignon jaune vif, comestible: \textit{Hygrophorus lucorum Kalc hbr.}. Littéralement “champignon doré”.} \zh{当地汉语方言:}\zh{黄蜡伞。}
\lhead{\firstmark}
\rhead{\botmark}

\subsection{\hspace{-0.5cm} {\Large \textcolor{darkblue}{\textbf{\ipa{ə˩qo˥}}}}\hspace{0.5cm}[\kern2pt{\textcolor{darkblue}{\textbf{\ipa{ə˩qo˥}}}}\kern2pt]} \hypertarget{@\string_Bqo\string_T1}{}
\markboth{\textcolor{darkblue}{\textbf{\ipa{ə˩qo˥}}}}{}
\textcolor{teal}{\zh{助词}} \hspace{4pt} \zh{声调类:} LH.
\zh{往里。} \textcolor{Sepia}{\selectlanguage{english}Inward.} \textcolor{PineGreen}{\selectlanguage{french}À l'intérieur, vers l'intérieur.} 
\lhead{\firstmark}
\rhead{\botmark}

\newpage
\section*{\centering- \textcolor{darkblue}{\textbf{\ipa{f}}} -}
\subsection{\hspace{-0.5cm} {\Large \textcolor{darkblue}{\textbf{\ipa{fɑ˧tɑ˧˥}}}}\hspace{0.5cm}[\kern2pt{\textcolor{darkblue}{\textbf{\ipa{fɑ˧tɑ˧˥}}}}\kern2pt]} \hypertarget{fA\string_MtA\string_M\string_T1}{}
\markboth{\textcolor{darkblue}{\textbf{\ipa{fɑ˧tɑ˧˥}}}}{}
\textcolor{teal}{\zh{形容词}} \hspace{4pt} \zh{声调类:} MH.
\zh{发达。} \textcolor{Sepia}{\selectlanguage{english}Developed, flourishing.} \textcolor{PineGreen}{\selectlanguage{french}Développé, florissant.}  \zh{【借词】} \zh{发达}
 ¶ \textcolor{darkblue}{\textbf{\ipa{fɑ˧tɑ˧-ze˥}}} \zh{很发达的了} \textcolor{Sepia}{\selectlanguage{english}\mytextsc{pfv}} \textcolor{PineGreen}{\selectlanguage{french}\mytextsc{pfv}}  

\lhead{\firstmark}
\rhead{\botmark}

\subsection{\hspace{-0.5cm} {\Large \textcolor{darkblue}{\textbf{\ipa{fɑ˩\textsubscript{a}}}}}\hspace{0.5cm}[\kern2pt{\textcolor{darkblue}{\textbf{\ipa{fɑ˩˥}}}}\kern2pt]} \hypertarget{fA\string_Ba1}{}
\markboth{\textcolor{darkblue}{\textbf{\ipa{fɑ˩\textsubscript{a}}}}}{}
\textcolor{teal}{\zh{动词}} \hspace{4pt} \zh{声调类:} L\textsubscript{a}.
\zh{发酵(汉语借词:发)。} \textcolor{Sepia}{\selectlanguage{english}To ferment.} \textcolor{PineGreen}{\selectlanguage{french}Fermenter.}  \zh{【借词】} \zh{发(酵)}
 ¶ \textcolor{darkblue}{\textbf{\ipa{tsɑ˧bɤ˧ ɖɯ˧-mɤ˩ | tʰi˧-fɑ˩}}} \zh{发一点面} \textcolor{Sepia}{\selectlanguage{english}to make a little flour ferment, to prepare a little bread dough} \textcolor{PineGreen}{\selectlanguage{french}faire lever un peu de farine}  
 ¶ \textcolor{darkblue}{\textbf{\ipa{tsɑ˧bɤ˧ tʰi˧-fɑ˩! | pɤ˩jɤ˧ gv̩˥-bi˩!}}} \zh{你发一点面吧!要做馒头!} \textcolor{Sepia}{\selectlanguage{english}Make some flour to ferment! We're going to prepare buns!} \textcolor{PineGreen}{\selectlanguage{french}Fais lever de la farine, on va faire des petits pains!}  

\lhead{\firstmark}
\rhead{\botmark}

\subsection{\hspace{-0.5cm} {\Large \textcolor{darkblue}{\textbf{\ipa{fæ˧}}}}\hspace{0.5cm}[\kern2pt{\textcolor{darkblue}{\textbf{\ipa{fæ˥}}}}\kern2pt]} \hypertarget{f\{\string_M1}{}
\markboth{\textcolor{darkblue}{\textbf{\ipa{fæ˧}}}}{}
\textcolor{teal}{\zh{名词}} \hspace{4pt} \zh{声调类:} M.
\zh{方(方向的方)(汉语借词)。} \textcolor{Sepia}{\selectlanguage{english}Direction.} \textcolor{PineGreen}{\selectlanguage{french}Direction.}  \zh{【借词】} \zh{方}
 ¶ \textcolor{darkblue}{\textbf{\ipa{dv̩˩tɕo˧ fæ˧}}} \zh{那个方向} \textcolor{Sepia}{\selectlanguage{english}that way} \textcolor{PineGreen}{\selectlanguage{french}cette direction-là}  
\zh{~【参考】~} \hyperlink{}{\textcolor{darkblue}{\textbf{\ipa{dɤ˧-tʰv̩˧-gi\#˥}}}} 
\lhead{\firstmark}
\rhead{\botmark}

\subsection{\hspace{-0.5cm} {\Large \textcolor{darkblue}{\textbf{\ipa{fv̩˩˧}}}}\hspace{0.5cm}[\kern2pt{\textcolor{darkblue}{\textbf{\ipa{fv̩˩˥}}}}\kern2pt]} \hypertarget{fv\string_=\string_B\string_M1}{}
\markboth{\textcolor{darkblue}{\textbf{\ipa{fv̩˩˧}}}}{}
\textcolor{teal}{\zh{名词}} \hspace{4pt} \zh{声调类:} LM.
\zh{邻居,村里的人们。} \textcolor{Sepia}{\selectlanguage{english}Neighbours.} \textcolor{PineGreen}{\selectlanguage{french}Le voisinage, les voisins.} 
\lhead{\firstmark}
\rhead{\botmark}

\subsection{\hspace{-0.5cm} {\Large \textcolor{darkblue}{\textbf{\ipa{fv̩˧}}}}\hspace{0.5cm}[\kern2pt{\textcolor{darkblue}{\textbf{\ipa{fv̩˥}}}}\kern2pt]} \hypertarget{fv\string_=\string_M1}{}
\markboth{\textcolor{darkblue}{\textbf{\ipa{fv̩˧}}}}{}
\textcolor{teal}{\zh{形容词}} \hspace{4pt} \zh{声调类:} M.
\zh{高兴、起劲,喜欢、爱、愿意。} \textcolor{Sepia}{\selectlanguage{english}Glad, pleased, happy, delighted; to like.} \textcolor{PineGreen}{\selectlanguage{french}Content, joyeux; agréable; aimer, apprécier.}  ¶ \textcolor{darkblue}{\textbf{\ipa{ɖwæ˧˥ | fv̩˧}}} \zh{很高兴} \textcolor{Sepia}{\selectlanguage{english}\mytextsc{intensive}.very: really glad, very happy} \textcolor{PineGreen}{\selectlanguage{french}\mytextsc{intensif}.très: très content, tout content}  
 ¶ \textcolor{darkblue}{\textbf{\ipa{dʑɤ˩˥ | fv̩˧}}} \zh{很高兴} \textcolor{Sepia}{\selectlanguage{english}really glad, very happy} \textcolor{PineGreen}{\selectlanguage{french}très content, tout content}  
 ¶ \textcolor{darkblue}{\textbf{\ipa{mɤ˧-fv̩˧ ʝi˧}}} \zh{生气} \textcolor{Sepia}{\selectlanguage{english}to get angry, to lose one's temper, to air one's anger} \textcolor{PineGreen}{\selectlanguage{french}se mettre en colère, s'énerver}  
 ¶ \textcolor{darkblue}{\textbf{\ipa{ʈʂʰɯ˧ mɤ˧-fv̩˧ ʝi˧!}}} \zh{他在生气。} \textcolor{Sepia}{\selectlanguage{english}He/she is angry.} \textcolor{PineGreen}{\selectlanguage{french}Il/elle est mécontent(e) / en colère.}  
 ¶ \textcolor{darkblue}{\textbf{\ipa{[F5] ɖwæ˧˥ | fv̩˧hĩ˧ ɖɯ˧-v̩˧ ɲi˩}}} \zh{他是很善良的人。} \textcolor{Sepia}{\selectlanguage{english}It's a very agreeable person.} \textcolor{PineGreen}{\selectlanguage{french}c'est quelqu'un de très agréable}  

\lhead{\firstmark}
\rhead{\botmark}

\subsection{\hspace{-0.5cm} {\Large \textcolor{darkblue}{\textbf{\ipa{fv̩˩bi˩}}}}\hspace{0.5cm}[\kern2pt{\textcolor{darkblue}{\textbf{\ipa{fv̩˩bi˩˥}}}}\kern2pt]} \hypertarget{fv\string_=\string_Bbi\string_B1}{}
\markboth{\textcolor{darkblue}{\textbf{\ipa{fv̩˩bi˩}}}}{}
\textcolor{teal}{\zh{名词}} \hspace{4pt} \zh{声调类:} L.
\zh{邻里、邻村:大家族居住的那片地方,包括几个小村落。} \textcolor{Sepia}{\selectlanguage{english}Neighbourhood (in the extended sense: encompasses several small villages).} \textcolor{PineGreen}{\selectlanguage{french}Contrée, voisinage, ensemble de villages où habitent des gens de la famille étendue.} 
\lhead{\firstmark}
\rhead{\botmark}

\subsection{\hspace{-0.5cm} {\Large \textcolor{darkblue}{\textbf{\ipa{fv̩˧kʰo˥}}}}\hspace{0.5cm}[\kern2pt{\textcolor{darkblue}{\textbf{\ipa{fv̩˧kʰo˥}}}}\kern2pt]} \hypertarget{fv\string_=\string_Mk\string_ho\string_T1}{}
\markboth{\textcolor{darkblue}{\textbf{\ipa{fv̩˧kʰo˥}}}}{}
\textcolor{teal}{\zh{名词}} \hspace{4pt} \zh{声调类:} H\#.
\zh{奉科(金沙江边的一个地区)。} \textcolor{Sepia}{\selectlanguage{english}Fengke: a village located close to the Yangtze river, on the right bank.} \textcolor{PineGreen}{\selectlanguage{french}Fengke: village situé au bord du Yang-tsé, sur la rive droite.} 
\lhead{\firstmark}
\rhead{\botmark}

\subsection{\hspace{-0.5cm} {\Large \textcolor{darkblue}{\textbf{\ipa{fv̩˧ʂɯ˩}}}}\hspace{0.5cm}[\kern2pt{\textcolor{darkblue}{\textbf{\ipa{fv̩˧ʂɯ˩}}}}\kern2pt]} \hypertarget{fv\string_=\string_Ms`M\string_B1}{}
\markboth{\textcolor{darkblue}{\textbf{\ipa{fv̩˧ʂɯ˩}}}}{}
\textcolor{teal}{\zh{名词}} \hspace{4pt} \zh{声调类:} L\#.
\zh{风湿(汉语借词)。} \textcolor{Sepia}{\selectlanguage{english}Rhumatism.} \textcolor{PineGreen}{\selectlanguage{french}Rhumatismes.}  \zh{【借词】} \zh{风湿}
 ¶ \textcolor{darkblue}{\textbf{\ipa{fv̩˧ʂɯ˩ go˩}}} \zh{有风湿、得风湿} \textcolor{Sepia}{\selectlanguage{english}to suffer from rhumatism, to have rhumatism} \textcolor{PineGreen}{\selectlanguage{french}souffrir de rhumatismes, avoir des rhumatismes}  

\lhead{\firstmark}
\rhead{\botmark}

\newpage
\section*{\centering- \textcolor{darkblue}{\textbf{\ipa{g}}} -}
\subsection{\hspace{-0.5cm} {\Large \textcolor{darkblue}{\textbf{\ipa{gæ˧ɻæ˩}}}}\hspace{0.5cm}[\kern2pt{\textcolor{darkblue}{\textbf{\ipa{gæ˧ɻæ˩}}}}\kern2pt]} \hypertarget{g\{\string_Mr£`\{\string_B1}{}
\markboth{\textcolor{darkblue}{\textbf{\ipa{gæ˧ɻæ˩}}}}{}
\textcolor{teal}{\zh{名词}} \hspace{4pt} \zh{声调类:} L\#.
\zh{嘎尔村。} \textcolor{Sepia}{\selectlanguage{english}The name of a village located about 1,500 meters West of \textcolor{darkblue}{\textbf{\ipa{/ə˧lɑ˧-ʁwɤ\#˥/:}}} to the left when leaving the plain of Yongning towards Eya; Chinese: Gaer.} \textcolor{PineGreen}{\selectlanguage{french}Village situé à environ 1,5 km à l'ouest de \textcolor{darkblue}{\textbf{\ipa{/ə˧lɑ˧-ʁwɤ\#˥/:}}} à main gauche en sortant de la vallée de Yongning, en direction de Eya. En chinois: Gaer.}  ¶ \textcolor{darkblue}{\textbf{\ipa{dʑɤ˩bv̩˧kɤ˧-sɑ˥ʁwɤ˩, | hi˩ʁwɤ˩-lo˥, | æ˩mi˧-ʁwɤ\#˥, | lɑ˧lo˧-ʁwɤ˥, | lɑ˧ŋwɤ˧, | bɤ˧tsʰo˧gv̩˥, | ə˧lɑ˧-ʁwɤ\#˥, | gæ˧ɻæ˩, | qʰæ˧tɕʰi˧, | tʰo˧ʈɯ\#˥}}} \zh{摩梭传统地理概念中,属于永宁的十个村落} \textcolor{Sepia}{\selectlanguage{english}the ten villages traditionally considered as part of Yongning} \textcolor{PineGreen}{\selectlanguage{french}les dix villages comptant traditionnellement comme faisant partie de Yongning}  

\lhead{\firstmark}
\rhead{\botmark}

\subsection{\hspace{-0.5cm} {\Large \textcolor{darkblue}{\textbf{\ipa{gæ˩ɖæ˧}}}}\hspace{0.5cm}[\kern2pt{\textcolor{darkblue}{\textbf{\ipa{gæ˩ɖæ˥}}}}\kern2pt]} \hypertarget{g\{\string_Bd`\{\string_M1}{}
\markboth{\textcolor{darkblue}{\textbf{\ipa{gæ˩ɖæ˧}}}}{}
\textcolor{teal}{\zh{名词}} \hspace{4pt} \zh{声调类:} LM.
\zh{上半身。} \textcolor{Sepia}{\selectlanguage{english}Top part of body.} \textcolor{PineGreen}{\selectlanguage{french}Le haut du corps.} 
\lhead{\firstmark}
\rhead{\botmark}

\subsection{\hspace{-0.5cm} {\Large \textcolor{darkblue}{\textbf{\ipa{gæ˩pʰæ˧}}}}\hspace{0.5cm}[\kern2pt{\textcolor{darkblue}{\textbf{\ipa{gæ˩pʰæ˥}}}}\kern2pt]} \hypertarget{g\{\string_Bp\string_h\{\string_M1}{}
\markboth{\textcolor{darkblue}{\textbf{\ipa{gæ˩pʰæ˧}}}}{}
\textcolor{teal}{\zh{名词}} \hspace{4pt} \zh{声调类:} LM.
\zh{储藏室、库房:存粮食、火腿的房间。} \textcolor{Sepia}{\selectlanguage{english}Storeroom, larder: a room where food is kept.} \textcolor{PineGreen}{\selectlanguage{french}Resserre, pièce où on conserve certains produits: dans le même bâtiment que la cuisine-foyer-salle à manger, à sa gauche (vu depuis la cour).}  \zh{量词}: \textcolor{darkblue}{\textbf{\ipa{tso˩}}} 
\lhead{\firstmark}
\rhead{\botmark}

\subsection{\hspace{-0.5cm} {\Large \textcolor{darkblue}{\textbf{\ipa{-gɤ˧}}}}\hspace{0.5cm}[\kern2pt{\textcolor{darkblue}{\textbf{\ipa{gɤ˥}}}}\kern2pt]} \hypertarget{-g7\string_M1}{}
\markboth{\textcolor{darkblue}{\textbf{\ipa{-gɤ˧}}}}{}
\textcolor{teal}{\zh{名词}} \hspace{4pt} \zh{声调类:} M.
\ding{202} \zh{地方。} \textcolor{Sepia}{\selectlanguage{english}Place.} \textcolor{PineGreen}{\selectlanguage{french}Lieu, endroit.}  ¶ \textcolor{darkblue}{\textbf{\ipa{njɤ˧ | ɖɯ˧-ʝi˧ (-gɤ˧) bi˧-zo˧-ho˩!}}} \zh{我要去一个别的地方! / 我要换一个地方了! / 我要走了!} \textcolor{Sepia}{\selectlanguage{english}I have to go somewhere! / I have to make a trip! / I'm off!} \textcolor{PineGreen}{\selectlanguage{french}Je dois me rendre quelque part! / Je dois faire un voyage! / Je m'en vais! (Contexte: lorsqu'on se prépare réellement à un voyage; ou lors d'une dispute, lorsqu'on menace de quitter la maison.)}  
 ¶ \textcolor{darkblue}{\textbf{\ipa{ze˩ gɤ˧}}} \zh{什么地方} \textcolor{Sepia}{\selectlanguage{english}which place} \textcolor{PineGreen}{\selectlanguage{french}quel endroit}  
 ¶ \textcolor{darkblue}{\textbf{\ipa{ʈʂʰɯ˧-gɤ˧}}} \zh{这个地方} \textcolor{Sepia}{\selectlanguage{english}this place} \textcolor{PineGreen}{\selectlanguage{french}cet endroit-ci}  
 ¶ \textcolor{darkblue}{\textbf{\ipa{tʰv̩˧-gɤ˧}}} \zh{那个地方} \textcolor{Sepia}{\selectlanguage{english}that place} \textcolor{PineGreen}{\selectlanguage{french}cet endroit-là}  
\ding{203} \zh{时候。} \textcolor{Sepia}{\selectlanguage{english}Moment.} \textcolor{PineGreen}{\selectlanguage{french}Moment.}  ¶ \textcolor{darkblue}{\textbf{\ipa{ʂɯ˧-ɬi˧mi˧-qo˧-gɤ˧ tʰv̩˧}}} \zh{七月到了的时候} \textcolor{Sepia}{\selectlanguage{english}when the seventh month has come, when one is in the seventh month} \textcolor{PineGreen}{\selectlanguage{french}quand est venu le septième mois, quand on en est au septième mois}  
 ¶ \textcolor{darkblue}{\textbf{\ipa{ʂɯ˧-ɬi˧mi˧-qo˧-gɤ˧-dʑo˥}}} \zh{七月的时候} \textcolor{Sepia}{\selectlanguage{english}in the seventh month, during the seventh month} \textcolor{PineGreen}{\selectlanguage{french}pendant le septième mois, au cours du septième mois}  

\lhead{\firstmark}
\rhead{\botmark}

\subsection{\hspace{-0.5cm} {\Large \textcolor{darkblue}{\textbf{\ipa{gɤ˧\textsubscript{b}}}}}\hspace{0.5cm}[\kern2pt{\textcolor{darkblue}{\textbf{\ipa{gɤ˩˥}}}}\kern2pt]} \hypertarget{g7\string_Mb1}{}
\markboth{\textcolor{darkblue}{\textbf{\ipa{gɤ˧\textsubscript{b}}}}}{}
\textcolor{teal}{\zh{动词}} \hspace{4pt} \zh{声调类:} M\textsubscript{b}.
\zh{缺乏。} \textcolor{Sepia}{\selectlanguage{english}To lack something (someone lacks a certain ability).} \textcolor{PineGreen}{\selectlanguage{french}Manquer de.}  ¶ \textcolor{darkblue}{\textbf{\ipa{mɤ˧-gɤ˧}}} \zh{不缺乏} \textcolor{Sepia}{\selectlanguage{english}\mytextsc{neg}: not to lack} \textcolor{PineGreen}{\selectlanguage{french}\mytextsc{neg}: ne pas manquer de}  

\lhead{\firstmark}
\rhead{\botmark}

\subsection{\hspace{-0.5cm} {\Large \textcolor{darkblue}{\textbf{\ipa{gɤ˧bɤ˧}}}}\hspace{0.5cm}[\kern2pt{\textcolor{darkblue}{\textbf{\ipa{gɤ˧bɤ˩}}}}\kern2pt]} \hypertarget{g7\string_Mb7\string_M1}{}
\markboth{\textcolor{darkblue}{\textbf{\ipa{gɤ˧bɤ˧}}}}{}
\textcolor{teal}{\zh{名词}} \hspace{4pt} \zh{声调类:} M.
\zh{影子。} \textcolor{Sepia}{\selectlanguage{english}Shadow.} \textcolor{PineGreen}{\selectlanguage{french}Ombre.}  ¶ \textcolor{darkblue}{\textbf{\ipa{gɤ˧bɤ˧ li˧}}} \zh{看电视} \textcolor{Sepia}{\selectlanguage{english}to watch television (coinage to avoid the loanword 'television')} \textcolor{PineGreen}{\selectlanguage{french}regarder la télé (néologisme)}  
 \zh{量词}: \textcolor{darkblue}{\textbf{\ipa{v̩˧}}} 
\lhead{\firstmark}
\rhead{\botmark}

\subsection{\hspace{-0.5cm} {\Large \textcolor{darkblue}{\textbf{\ipa{-gɤ˧bi\#˥}}}}\hspace{0.5cm}[\kern2pt{\textcolor{darkblue}{\textbf{\ipa{gɤ˩bi˥}}}}\kern2pt]} \hypertarget{-g7\string_Mbi\#\string_T1}{}
\markboth{\textcolor{darkblue}{\textbf{\ipa{-gɤ˧bi\#˥}}}}{}
\textcolor{teal}{\zh{后置词}} \hspace{4pt} \zh{声调类:} \#H.
\zh{上面。} \textcolor{Sepia}{\selectlanguage{english}On top.} \textcolor{PineGreen}{\selectlanguage{french}Sur, dessus.}  ¶ \textcolor{darkblue}{\textbf{\ipa{ʑi˧qʰwɤ˧-gɤ˧bi˧}}} \zh{在房頂上} \textcolor{Sepia}{\selectlanguage{english}on the (roof of) the house = on the roof} \textcolor{PineGreen}{\selectlanguage{french}sur la maison, sur le toit}  

\lhead{\firstmark}
\rhead{\botmark}

\subsection{\hspace{-0.5cm} {\Large \textcolor{darkblue}{\textbf{\ipa{gɤ˧lɑ˧}}}}\hspace{0.5cm}[\kern2pt{\textcolor{darkblue}{\textbf{\ipa{gɤ˩lɑ˥}}}}\kern2pt]} \hypertarget{g7\string_MlA\string_M1}{}
\markboth{\textcolor{darkblue}{\textbf{\ipa{gɤ˧lɑ˧}}}}{}
\textcolor{teal}{\zh{名词}} \hspace{4pt} \zh{声调类:} M.
\zh{神,菩萨,佛。} \textcolor{Sepia}{\selectlanguage{english}God, Pusa, Buddha, Bodhisattva.} \textcolor{PineGreen}{\selectlanguage{french}Dieu, bouddha, bodhisattva.}  \zh{【借词】}\zh{藏语} lha
 \zh{量词}: \textcolor{darkblue}{\textbf{\ipa{v̩˧}}} 
\lhead{\firstmark}
\rhead{\botmark}

\subsection{\hspace{-0.5cm} {\Large \textcolor{darkblue}{\textbf{\ipa{gɤ˧lɑ˧-pɤ\#˥}}}}\hspace{0.5cm}[\kern2pt{\textcolor{darkblue}{\textbf{\ipa{xxxx non-correspondance entre le nombre de morphèmes et le nombre de tons de morphèmes}}}}\kern2pt]} \hypertarget{g7\string_MlA\string_M-p7\#\string_T1}{}
\markboth{\textcolor{darkblue}{\textbf{\ipa{gɤ˧lɑ˧-pɤ\#˥}}}}{}
\textcolor{teal}{\zh{名词}} \hspace{4pt} \zh{声调类:} \#H.
\zh{佛像。} \textcolor{Sepia}{\selectlanguage{english}Image of Buddha.} \textcolor{PineGreen}{\selectlanguage{french}Image du bouddha.}  \zh{量词}: \textcolor{darkblue}{\textbf{\ipa{pɤ˥}}} 
\lhead{\firstmark}
\rhead{\botmark}

\subsection{\hspace{-0.5cm} {\Large \textcolor{darkblue}{\textbf{\ipa{gɤ˧lɑ˧-ʑi˩}}}}\hspace{0.5cm}[\kern2pt{\textcolor{darkblue}{\textbf{\ipa{xxxx non-correspondance entre le nombre de morphèmes et le nombre de tons de morphèmes}}}}\kern2pt]} \hypertarget{g7\string_MlA\string_M-z£i\string_B1}{}
\markboth{\textcolor{darkblue}{\textbf{\ipa{gɤ˧lɑ˧-ʑi˩}}}}{}
\textcolor{teal}{\zh{名词}} \hspace{4pt} \zh{声调类:} \mytextsc{L}.
\zh{经堂(拜佛、拜祖先的房间)。} \textcolor{Sepia}{\selectlanguage{english}Room where the ancestors are worshipped.} \textcolor{PineGreen}{\selectlanguage{french}Pièce du culte: pièce des esprits, pièce des ancêtres, où se trouve un autel. Un rituel y est effectué chaque matin. Le nom désigne par extension l'intégralité d'un des quatre bâtiments de la ferme traditionnelle na.}  ¶ \textcolor{darkblue}{\textbf{\ipa{gɤ˧lɑ˧-ʑi˩-di˩}}} \zh{同上} \textcolor{Sepia}{\selectlanguage{english}same meaning} \textcolor{PineGreen}{\selectlanguage{french}même sens}  
 \zh{量词}: \textcolor{darkblue}{\textbf{\ipa{ɭɯ˧}}} 
\lhead{\firstmark}
\rhead{\botmark}

\subsection{\hspace{-0.5cm} {\Large \textcolor{darkblue}{\textbf{\ipa{gɤ˧qo˥}}}}\hspace{0.5cm}[\kern2pt{\textcolor{darkblue}{\textbf{\ipa{gɤ˧qo˧˥}}}}\kern2pt]} \hypertarget{g7\string_Mqo\string_T1}{}
\markboth{\textcolor{darkblue}{\textbf{\ipa{gɤ˧qo˥}}}}{}
\textcolor{teal}{\zh{名词}} \hspace{4pt} \zh{声调类:} MH.
\zh{主屋的高处:人吃饭的地方。} \textcolor{Sepia}{\selectlanguage{english}Higher part of the main room.} \textcolor{PineGreen}{\selectlanguage{french}Haut du foyer: partie de la pièce où l’on prend les repas, autour du foyer; c'est une structure en bois, surélevée d’une vingtaine de centimètres par rapport au sol cimenté.}  \zh{量词}: \textcolor{darkblue}{\textbf{\ipa{ɭɯ˧}}} 
\lhead{\firstmark}
\rhead{\botmark}

\subsection{\hspace{-0.5cm} {\Large \textcolor{darkblue}{\textbf{\ipa{gɤ˩‑}}}}\hspace{0.5cm}[\kern2pt{\textcolor{darkblue}{\textbf{\ipa{gɤ˩˥}}}}\kern2pt]} \hypertarget{g7\string_B‑1}{}
\markboth{\textcolor{darkblue}{\textbf{\ipa{gɤ˩‑}}}}{}
\textcolor{teal}{\zh{助词}} \hspace{4pt} \zh{声调类:} L.
\zh{向上、往上。} \textcolor{Sepia}{\selectlanguage{english}Directional prefix: upward.} \textcolor{PineGreen}{\selectlanguage{french}Préfixe directionnel: vers le haut.} \zh{~【参考】~} \textcolor{darkblue}{\textbf{\ipa{gɤ˩-qo˧, gɤ˩-tʰv̩˧qo˧, gɤ˩-ʈʂʰɯ˧qo˧}}} 
\lhead{\firstmark}
\rhead{\botmark}

\subsection{\hspace{-0.5cm} {\Large \textcolor{darkblue}{\textbf{\ipa{gɤ˩}}}}\hspace{0.5cm}[\kern2pt{\textcolor{darkblue}{\textbf{\ipa{gɤ˥}}}}\kern2pt]} \hypertarget{g7\string_B1}{}
\markboth{\textcolor{darkblue}{\textbf{\ipa{gɤ˩}}}}{}
\textcolor{teal}{\zh{形容词}} \hspace{4pt} \zh{声调类:} L.
\zh{爱吵架。} \textcolor{Sepia}{\selectlanguage{english}Quarrelsome. This term is used to describe the personality associated with certain astrological signs: some, such as the Tiger and the Monkey, are considered as quarrelsome, making the people born during the corresponding years less suitable for participating in certain rites (e.g. the Coming of Age rite), and more suitable for certain other rites and occasions.} \textcolor{PineGreen}{\selectlanguage{french}Querelleur, belliqueux, batailleur. Ce terme s'emploie au sujet des signes astrologiques: certains sont considérés comme 'bagarreurs', comme le Tigre et le Singe, ce qui rend les personnes nées cette année-là peu appropriées pour certains rites (ex.: lors du rite de passage à l'âge adulte), et au contraire très prisés pour d'autres.}  ¶ \textcolor{darkblue}{\textbf{\ipa{kʰv̩˧ gɤ˧˥}}} \zh{爱打架的年份/生肖:十二个生肖中,虎、猴……被认为是爱打架的。} \textcolor{Sepia}{\selectlanguage{english}“quarrelsome year”: a year whose astrological sign is a quarrelsome animal. Astrological signs such as the Tiger and the Monkey are considered as quarrelsome; people born during one of these years are said to be tough and quarrelsome.} \textcolor{PineGreen}{\selectlanguage{french}signe belliqueux (concept astrologique: certains signes confèrent aux gens nés l'année correspondante un caractère dur/belliqueux)}  
 ¶ \textcolor{darkblue}{\textbf{\ipa{kʰv̩˧ gɤ˧-hĩ˥}}} \zh{属一个爱打架的年份/生肖的人。十二个生肖中,虎、猴……被认为是爱打架的。} \textcolor{Sepia}{\selectlanguage{english}person whose astrological sign is a quarrelsome animal. Astrological signs such as the Tiger and the Monkey are considered as quarrelsome.} \textcolor{PineGreen}{\selectlanguage{french}personne d'une année batailleuse}  
 ¶ \textcolor{darkblue}{\textbf{\ipa{ʑi˩hṽ˥, | lɑ˧ : | kʰv̩˧ gɤ˧˥!}}} \zh{属猴和属虎的人很爱吵架!} \textcolor{Sepia}{\selectlanguage{english}The Monkey and the Ape are quarrelsome birth signs!} \textcolor{PineGreen}{\selectlanguage{french}Les signes astrologiques du Singe et du Tigre sont des signes batailleurs!}  
 ¶ \textcolor{darkblue}{\textbf{\ipa{ʑi˩˥, | lɑ˧, | kʰv̩˧ gɤ˧˥!}}} \zh{同上} \textcolor{Sepia}{\selectlanguage{english}Same meaning as above; the investigator substituted the colloquial term for 'ape, monkey'.} \textcolor{PineGreen}{\selectlanguage{french}Même sens que ci-dessus; formulation modernisée par l'enquêteur, utilisant le terme usuel pour 'singe'.}  

\lhead{\firstmark}
\rhead{\botmark}

\subsection{\hspace{-0.5cm} {\Large \textcolor{darkblue}{\textbf{\ipa{gɤ˩\textsubscript{a}}}} \textsubscript{1}}\hspace{0.5cm}[\kern2pt{\textcolor{darkblue}{\textbf{\ipa{gɤ˩˥}}}}\kern2pt]} \hypertarget{g7\string_Ba1}{}
\markboth{\textcolor{darkblue}{\textbf{\ipa{gɤ˩\textsubscript{a}}}} \textsubscript{1}}{}
\textcolor{teal}{\zh{动词}} \hspace{4pt} \zh{声调类:} L\textsubscript{a}.
\zh{灭,熄。} \textcolor{Sepia}{\selectlanguage{english}To go out (fire).} \textcolor{PineGreen}{\selectlanguage{french}S’éteindre.}  ¶ \textcolor{darkblue}{\textbf{\ipa{mv̩˧ | le˧-gɤ˩(-ze˩)}}} \zh{火灭了。} \textcolor{Sepia}{\selectlanguage{english}The fire has gone out. / The fire went out.} \textcolor{PineGreen}{\selectlanguage{french}Le feu s'est éteint.}  

\lhead{\firstmark}
\rhead{\botmark}

\subsection{\hspace{-0.5cm} {\Large \textcolor{darkblue}{\textbf{\ipa{gɤ˩\textsubscript{a}}}} \textsubscript{2}}\hspace{0.5cm}[\kern2pt{\textcolor{darkblue}{\textbf{\ipa{gɤ˩˥}}}}\kern2pt]} \hypertarget{g7\string_Ba2}{}
\markboth{\textcolor{darkblue}{\textbf{\ipa{gɤ˩\textsubscript{a}}}} \textsubscript{2}}{}
\textcolor{teal}{\zh{动词}} \hspace{4pt} \zh{声调类:} L\textsubscript{a}.
\zh{满意,幸福,甘心,服气。} \textcolor{Sepia}{\selectlanguage{english}To be satisfied/happy; to feel that things are fair.} \textcolor{PineGreen}{\selectlanguage{french}Être satisfait, content (de son sort), heureux.}  ¶ \textcolor{darkblue}{\textbf{\ipa{hɤ˩-zo˥, | le˧-gɤ˩-ze˩!}}} \zh{很成功,真高兴! / 他成功了,很满意!} \textcolor{Sepia}{\selectlanguage{english}(S)he has made a good job of it; (s)he is satisfied/happy!} \textcolor{PineGreen}{\selectlanguage{french}(il a réussi) habilement (ce qu'il voulait faire), il est content/satisfait!}  
 ¶ \textcolor{darkblue}{\textbf{\ipa{ʈʂʰɯ˧ | ɖwæ˧˥ | le˧-gɤ˩-ze˩!}}} \zh{他很满意!} \textcolor{Sepia}{\selectlanguage{english}(S)he is very satisfied/happy!} \textcolor{PineGreen}{\selectlanguage{french}il est très content!}  
 ¶ \textcolor{darkblue}{\textbf{\ipa{no˩-se˥, | ɖwæ˧˥ | le˧-gɤ˩-ze˩: | zo˧mv̩˥ hɤ˩-zo˩!}}} \zh{你呢,(应该)很满意:(你的)孩子很成功!} \textcolor{Sepia}{\selectlanguage{english}You have grounds for satisfaction: your children are really bright!} \textcolor{PineGreen}{\selectlanguage{french}Vous, vous avez bien de la chance/vous avez toutes raisons d'être satisfait(e)/vous avez des sujets de satisfaction: vos enfants sont brillants/habiles!}  
 ¶ \textcolor{darkblue}{\textbf{\ipa{mɤ˧-gɤ˩}}} \zh{不满意、不甘心、不服气} \textcolor{Sepia}{\selectlanguage{english}to be dissatisfied, not resigned, recalcitrant} \textcolor{PineGreen}{\selectlanguage{french}être mécontent, ne pas se résigner, être récalcitrant}  

\lhead{\firstmark}
\rhead{\botmark}

\subsection{\hspace{-0.5cm} {\Large \textcolor{darkblue}{\textbf{\ipa{gɤ˩\textsubscript{a}}}} \textsubscript{3}}\hspace{0.5cm}[\kern2pt{\textcolor{darkblue}{\textbf{\ipa{gɤ˩˥}}}}\kern2pt]} \hypertarget{g7\string_Ba3}{}
\markboth{\textcolor{darkblue}{\textbf{\ipa{gɤ˩\textsubscript{a}}}} \textsubscript{3}}{}
\textcolor{teal}{\zh{形容词}} \hspace{4pt} \zh{声调类:} L\textsubscript{a}.
\zh{震惊。} \textcolor{Sepia}{\selectlanguage{english}Startled, amazed, shocked, awestruck; terrified.} \textcolor{PineGreen}{\selectlanguage{french}Surpris, étonné, abasourdi; terrifié.}  ¶ \textcolor{darkblue}{\textbf{\ipa{le˧-gɤ˩-ze˩}}} \zh{震惊了} \textcolor{Sepia}{\selectlanguage{english}\mytextsc{accomp} \string_ \mytextsc{pfv}} \textcolor{PineGreen}{\selectlanguage{french}\mytextsc{accomp} \string_ \mytextsc{pfv}}  
 ¶ \textcolor{darkblue}{\textbf{\ipa{no˧ | hĩ˧ gɤ˧-kʰɯ˥!}}} \zh{你让人害怕!} \textcolor{Sepia}{\selectlanguage{english}You frighten people! / People are afraid of you!} \textcolor{PineGreen}{\selectlanguage{french}Tu fais peur aux gens!}  

\lhead{\firstmark}
\rhead{\botmark}

\subsection{\hspace{-0.5cm} {\Large \textcolor{darkblue}{\textbf{\ipa{gɤ˩bv̩˧}}}}\hspace{0.5cm}[\kern2pt{\textcolor{darkblue}{\textbf{\ipa{gɤ˧bv̩˧}}}}\kern2pt]} \hypertarget{g7\string_Bbv\string_=\string_M1}{}
\markboth{\textcolor{darkblue}{\textbf{\ipa{gɤ˩bv̩˧}}}}{}
\textcolor{teal}{\zh{动词}} \hspace{4pt} \zh{声调类:} LM.
\zh{溢出来。} \textcolor{Sepia}{\selectlanguage{english}To overflow.} \textcolor{PineGreen}{\selectlanguage{french}Déborder.}  ¶ \textcolor{darkblue}{\textbf{\ipa{gɤ˩bv̩˧-ze˩}}} \zh{溢出来了} \textcolor{Sepia}{\selectlanguage{english}\mytextsc{pfv}} \textcolor{PineGreen}{\selectlanguage{french}\mytextsc{pfv}}  

\lhead{\firstmark}
\rhead{\botmark}

\subsection{\hspace{-0.5cm} {\Large \textcolor{darkblue}{\textbf{\ipa{gɤ˩dzɤ˧}}}}\hspace{0.5cm}[\kern2pt{\textcolor{darkblue}{\textbf{\ipa{gɤ˩dzɤ˥}}}}\kern2pt]} \hypertarget{g7\string_Bdz7\string_M1}{}
\markboth{\textcolor{darkblue}{\textbf{\ipa{gɤ˩dzɤ˧}}}}{}
\textcolor{teal}{\zh{助词}} \hspace{4pt} \zh{声调类:} LM.
\zh{在上部分,上座。} \textcolor{Sepia}{\selectlanguage{english}At the top part: inside a room, at a table..., this is the place of honour.} \textcolor{PineGreen}{\selectlanguage{french}Haut, partie supérieure, partie noble (d'une salle, d'une tablée…) (symboliquement: “la tête”).}  ¶ \textcolor{darkblue}{\textbf{\ipa{gɤ˩dzɤ˧ dzi˧˥}}} \zh{坐上座} \textcolor{Sepia}{\selectlanguage{english}to sit at a place of honour, to sit at the superior part (of a table, a room...)} \textcolor{PineGreen}{\selectlanguage{french}être assis à une place d'honneur}  
 ¶ \textcolor{darkblue}{\textbf{\ipa{no˧ | gɤ˩dzɤ˧ dzi˧˥!}}} \zh{请您坐在上座!} \textcolor{Sepia}{\selectlanguage{english}Please be seated at the place of honour!} \textcolor{PineGreen}{\selectlanguage{french}Veuillez vous installer à l'une des premières places! / Veuillez prendre l'une des places d'honneur!}  
\zh{~【参考】~} \textcolor{darkblue}{\textbf{\ipa{gɤ˩-}}} 
\lhead{\firstmark}
\rhead{\botmark}

\subsection{\hspace{-0.5cm} {\Large \textcolor{darkblue}{\textbf{\ipa{gɤ˩-qo˧}}}}\hspace{0.5cm}[\kern2pt{\textcolor{darkblue}{\textbf{\ipa{xxxx non-correspondance entre le nombre de morphèmes et le nombre de tons de morphèmes}}}}\kern2pt]} \hypertarget{g7\string_B-qo\string_M1}{}
\markboth{\textcolor{darkblue}{\textbf{\ipa{gɤ˩-qo˧}}}}{}
\textcolor{teal}{\zh{助词}} \hspace{4pt} \zh{声调类:} M.
\zh{那上面(指高处)。} \textcolor{Sepia}{\selectlanguage{english}Way up there.} \textcolor{PineGreen}{\selectlanguage{french}Par là-bas tout en haut.} \zh{~【参考】~} \textcolor{darkblue}{\textbf{\ipa{gɤ˩-, gɤ˩-ʈʂʰɯ˧qo˧, gɤ˩-tʰv̩˧qo˧,}}} 
\lhead{\firstmark}
\rhead{\botmark}

\subsection{\hspace{-0.5cm} {\Large \textcolor{darkblue}{\textbf{\ipa{gɤ˩qwɤ˧}}}}\hspace{0.5cm}[\kern2pt{\textcolor{darkblue}{\textbf{\ipa{gɤ˩qwɤ˥}}}}\kern2pt]} \hypertarget{g7\string_Bqw7\string_M1}{}
\markboth{\textcolor{darkblue}{\textbf{\ipa{gɤ˩qwɤ˧}}}}{}
\textcolor{teal}{\zh{名词}} \hspace{4pt} \zh{声调类:} LM.
\zh{火炉旁边的祭坛(上面摆礼物等)。} \textcolor{Sepia}{\selectlanguage{english}Altar above the hearth (where gifts made to the family are displayed).} \textcolor{PineGreen}{\selectlanguage{french}Autel en contrehaut du foyer, où on dépose les présents qu'apportent les invités/les membres de la famille, les offrant ainsi aux ancêtres.}  \zh{量词}: \textcolor{darkblue}{\textbf{\ipa{ɭɯ˧}}} 
\lhead{\firstmark}
\rhead{\botmark}

\subsection{\hspace{-0.5cm} {\Large \textcolor{darkblue}{\textbf{\ipa{gɤ˩ʁwɤ\#˥}}}}\hspace{0.5cm}[\kern2pt{\textcolor{darkblue}{\textbf{\ipa{gɤ˩ʁwɤ˥}}}}\kern2pt]} \hypertarget{g7\string_BRw7\#\string_T1}{}
\markboth{\textcolor{darkblue}{\textbf{\ipa{gɤ˩ʁwɤ\#˥}}}}{}
\textcolor{teal}{\zh{名词}} \hspace{4pt} \zh{声调类:} LM+\#H.
\zh{格瓦村:永宁的一个村落。直译:上村。音译:格瓦。} \textcolor{Sepia}{\selectlanguage{english}The village of Gewa.} \textcolor{PineGreen}{\selectlanguage{french}Nom de village; en chinois: Gewa.} 
\lhead{\firstmark}
\rhead{\botmark}

\subsection{\hspace{-0.5cm} {\Large \textcolor{darkblue}{\textbf{\ipa{gɤ˩ʁwɤ˧}}}}\hspace{0.5cm}[\kern2pt{\textcolor{darkblue}{\textbf{\ipa{gɤ˩ʁwɤ˥}}}}\kern2pt]} \hypertarget{g7\string_BRw7\string_M1}{}
\markboth{\textcolor{darkblue}{\textbf{\ipa{gɤ˩ʁwɤ˧}}}}{}
\textcolor{teal}{\zh{名词}} \hspace{4pt} \zh{声调类:} LM.
\zh{上游。} \textcolor{Sepia}{\selectlanguage{english}Upper reaches of a river; upstream.} \textcolor{PineGreen}{\selectlanguage{french}Cours supérieur (d'une rivière), amont.} 
\lhead{\firstmark}
\rhead{\botmark}

\subsection{\hspace{-0.5cm} {\Large \textcolor{darkblue}{\textbf{\ipa{gɤ˩-tʰv̩˧-gi\#˥}}}}\hspace{0.5cm}[\kern2pt{\textcolor{darkblue}{\textbf{\ipa{xxxx non-correspondance entre le nombre de morphèmes et le nombre de tons de morphèmes}}}}\kern2pt]} \hypertarget{g7\string_B-t\string_hv\string_=\string_M-gi\#\string_T1}{}
\markboth{\textcolor{darkblue}{\textbf{\ipa{gɤ˩-tʰv̩˧-gi\#˥}}}}{}
\textcolor{teal}{\zh{助词}} \hspace{4pt} \zh{声调类:} L-\#H.
\zh{那里(指高处)。} \textcolor{Sepia}{\selectlanguage{english}Way up there.} \textcolor{PineGreen}{\selectlanguage{french}Au loin par là-haut, de ce côté tout là-haut.} \zh{~【参考】~} \textcolor{darkblue}{\textbf{\ipa{gɤ˩‑, gɤ˩-qo˧, gɤ˩-tʰv̩˧qo˧, gɤ˩-ʈʂʰɯ˧qo˧}}} 
\lhead{\firstmark}
\rhead{\botmark}

\subsection{\hspace{-0.5cm} {\Large \textcolor{darkblue}{\textbf{\ipa{gɤ˩-tʰv̩˧qo˧}}}}\hspace{0.5cm}[\kern2pt{\textcolor{darkblue}{\textbf{\ipa{gɤ˩tʰv̩˧qo˧}}}}\kern2pt]} \hypertarget{g7\string_B-t\string_hv\string_=\string_Mqo\string_M1}{}
\markboth{\textcolor{darkblue}{\textbf{\ipa{gɤ˩-tʰv̩˧qo˧}}}}{}
\textcolor{teal}{\zh{助词}} \hspace{4pt} \zh{声调类:} L-\#H.
\zh{那里(指高处)。} \textcolor{Sepia}{\selectlanguage{english}Way up there.} \textcolor{PineGreen}{\selectlanguage{french}Au loin par là-haut, de ce côté tout là-haut.} \zh{~【参考】~} \textcolor{darkblue}{\textbf{\ipa{gɤ˩‑, gɤ˩-qo˧, gɤ˩-ʈʂʰɯ˧qo˧}}} 
\lhead{\firstmark}
\rhead{\botmark}

\subsection{\hspace{-0.5cm} {\Large \textcolor{darkblue}{\textbf{\ipa{gɤ˩ʈʂæ˧˥}}}}\hspace{0.5cm}[\kern2pt{\textcolor{darkblue}{\textbf{\ipa{gɤ˩ʈʂæ˧˥}}}}\kern2pt]} \hypertarget{g7\string_Bt`s`\{\string_M\string_T1}{}
\markboth{\textcolor{darkblue}{\textbf{\ipa{gɤ˩ʈʂæ˧˥}}}}{}
\textcolor{teal}{\zh{名词}} \hspace{4pt} \zh{声调类:} LM+MH\#.
\zh{上半(身)。} \textcolor{Sepia}{\selectlanguage{english}Top part (of the body=above the waist).} \textcolor{PineGreen}{\selectlanguage{french}Haut du corps, partie supérieure du corps.} 
\lhead{\firstmark}
\rhead{\botmark}

\subsection{\hspace{-0.5cm} {\Large \textcolor{darkblue}{\textbf{\ipa{gɤ˩ʈʂʰæ˧-hĩ˧˥}}}}\hspace{0.5cm}[\kern2pt{\textcolor{darkblue}{\textbf{\ipa{xxxx non-correspondance entre le nombre de morphèmes et le nombre de tons de morphèmes}}}}\kern2pt]} \hypertarget{g7\string_Bt`s`\string_h\{\string_M-hi\string_~\string_M\string_T1}{}
\markboth{\textcolor{darkblue}{\textbf{\ipa{gɤ˩ʈʂʰæ˧-hĩ˧˥}}}}{}
\textcolor{teal}{\zh{名词}} \hspace{4pt} \zh{声调类:} LM+MH\#.
\zh{祖先。} \textcolor{Sepia}{\selectlanguage{english}Ancestors, past generations.} \textcolor{PineGreen}{\selectlanguage{french}Les générations passées, les ancêtres.} 
\lhead{\firstmark}
\rhead{\botmark}

\subsection{\hspace{-0.5cm} {\Large \textcolor{darkblue}{\textbf{\ipa{gɤ˩-ʈʂʰɯ˧-gi\#˥}}}}\hspace{0.5cm}[\kern2pt{\textcolor{darkblue}{\textbf{\ipa{xxxx non-correspondance entre le nombre de morphèmes et le nombre de tons de morphèmes}}}}\kern2pt]} \hypertarget{g7\string_B-t`s`\string_hM\string_M-gi\#\string_T1}{}
\markboth{\textcolor{darkblue}{\textbf{\ipa{gɤ˩-ʈʂʰɯ˧-gi\#˥}}}}{}
\textcolor{teal}{\zh{助词}} \hspace{4pt} \zh{声调类:} L-\#H.
\zh{那里(指高处)。} \textcolor{Sepia}{\selectlanguage{english}Way up there.} \textcolor{PineGreen}{\selectlanguage{french}Au loin par là-haut, de ce côté tout là-haut.} \zh{~【参考】~} \textcolor{darkblue}{\textbf{\ipa{gɤ˩-, gɤ˩-qo˧, gɤ˩-tʰv̩˧qo˧, gɤ˩-ʈʂʰɯ˧qo˧}}} 
\lhead{\firstmark}
\rhead{\botmark}

\subsection{\hspace{-0.5cm} {\Large \textcolor{darkblue}{\textbf{\ipa{gɤ˩-ʈʂʰɯ˧qo˧}}}}\hspace{0.5cm}[\kern2pt{\textcolor{darkblue}{\textbf{\ipa{gɤ˩ʈʂʰɯ˧qo˧}}}}\kern2pt]} \hypertarget{g7\string_B-t`s`\string_hM\string_Mqo\string_M1}{}
\markboth{\textcolor{darkblue}{\textbf{\ipa{gɤ˩-ʈʂʰɯ˧qo˧}}}}{}
\textcolor{teal}{\zh{助词}} \hspace{4pt} \zh{声调类:} L-\#H.
\zh{那里(指高处)。} \textcolor{Sepia}{\selectlanguage{english}Way up there.} \textcolor{PineGreen}{\selectlanguage{french}Au loin par là-haut, de ce côté tout là-haut.} \zh{~【参考】~} \textcolor{darkblue}{\textbf{\ipa{gɤ˩-, gɤ˩-qo˧, gɤ˩-tʰv̩˧qo˧}}} 
\lhead{\firstmark}
\rhead{\botmark}

\subsection{\hspace{-0.5cm} {\Large \textcolor{darkblue}{\textbf{\ipa{gɤ˧˥}}}}\hspace{0.5cm}[\kern2pt{\textcolor{darkblue}{\textbf{\ipa{gɤ˥}}}}\kern2pt]} \hypertarget{g7\string_M\string_T1}{}
\markboth{\textcolor{darkblue}{\textbf{\ipa{gɤ˧˥}}}}{}
\textcolor{teal}{\zh{动词}} \hspace{4pt} \zh{声调类:} MH.
\zh{扛,担。} \textcolor{Sepia}{\selectlanguage{english}To carry on the shoulder; to carry on a shoulder pole.} \textcolor{PineGreen}{\selectlanguage{french}Porter à l’épaule; porter sur une palanche.}  ¶ \textcolor{darkblue}{\textbf{\ipa{tʰi˧-gɤ˧˥}}} \zh{\mytextsc{dur}} \textcolor{Sepia}{\selectlanguage{english}\mytextsc{dur}} \textcolor{PineGreen}{\selectlanguage{french}\mytextsc{dur}}  
 ¶ \textcolor{darkblue}{\textbf{\ipa{tʰi˧-gɤ˧-ze˥}}} \zh{\mytextsc{dur} \string_ \mytextsc{pfv}} \textcolor{Sepia}{\selectlanguage{english}\mytextsc{dur} \string_ \mytextsc{pfv}} \textcolor{PineGreen}{\selectlanguage{french}\mytextsc{dur} \string_ \mytextsc{pfv}}  
 ¶ \textcolor{darkblue}{\textbf{\ipa{le˧-gɤ˧-ze˥}}} \zh{扛了} \textcolor{Sepia}{\selectlanguage{english}\mytextsc{accomp} \string_ \mytextsc{pfv}} \textcolor{PineGreen}{\selectlanguage{french}\mytextsc{accomp} \string_ \mytextsc{pfv}}  
 ¶ \textcolor{darkblue}{\textbf{\ipa{tso˧\textasciitilde{}tso˧ gɤ˩}}} \zh{扛东西} \textcolor{Sepia}{\selectlanguage{english}to carry something on the shoulder} \textcolor{PineGreen}{\selectlanguage{french}porter quelque chose à l'épaule}  
 ¶ \textcolor{darkblue}{\textbf{\ipa{njɤ˧(-ɳɯ˧) | gɤ˧-bi˥!}}} \zh{我来扛吧!} \textcolor{Sepia}{\selectlanguage{english}Let me carry it!} \textcolor{PineGreen}{\selectlanguage{french}C'est moi qui porte!}  

\lhead{\firstmark}
\rhead{\botmark}

\subsection{\hspace{-0.5cm} {\Large \textcolor{darkblue}{\textbf{\ipa{gi˥}}} \textsubscript{1}}\hspace{0.5cm}[\kern2pt{\textcolor{darkblue}{\textbf{\ipa{gi˥}}}}\kern2pt]} \hypertarget{gi\string_T1}{}
\markboth{\textcolor{darkblue}{\textbf{\ipa{gi˥}}} \textsubscript{1}}{}
\textcolor{teal}{\zh{动词}} \hspace{4pt} \zh{声调类:} H.
\zh{下(雨,雪)。} \textcolor{Sepia}{\selectlanguage{english}To fall (snow, rain), to snow/to rain.} \textcolor{PineGreen}{\selectlanguage{french}Tomber (neige, pluie), neiger, pleuvoir.}  ¶ \textcolor{darkblue}{\textbf{\ipa{bi˧ gi˧. / bi˧ gi˧-ze˩.}}} \zh{下雪。 / 下雪了。} \textcolor{Sepia}{\selectlanguage{english}It snows. / It has snowed.} \textcolor{PineGreen}{\selectlanguage{french}Il neige. / Il a neigé.}  
 ¶ \textcolor{darkblue}{\textbf{\ipa{hi˩ gi˩˥. / hi˩ gi˩-ze˥.}}} \zh{下雨。 / 下雨了。} \textcolor{Sepia}{\selectlanguage{english}It rains. / It has rained.} \textcolor{PineGreen}{\selectlanguage{french}Il pleut. / Il a plu.}  
 ¶ \textcolor{darkblue}{\textbf{\ipa{tsʰi˧-ɲi˧-dʑo˩, | hi˩ gi˩-ze˥, | le˧-gɤ˩-ze˩!}}} \zh{今天,下雨了,真好!(情景:大旱灾过后,雨季终于来了,这对庄稼很好。)} \textcolor{Sepia}{\selectlanguage{english}Today, it is raining; that's good! / it's a good thing! (A comment made at the beginning of the rainy season, after a long drought.)} \textcolor{PineGreen}{\selectlanguage{french}Aujourd'hui, il s'est mis à pleuvoir / il a plu; c'est bien! (Commentaire au sujet de la pluie qui est venue, après une longue période de sécheresse.)}  

\lhead{\firstmark}
\rhead{\botmark}

\subsection{\hspace{-0.5cm} {\Large \textcolor{darkblue}{\textbf{\ipa{gi˥}}} \textsubscript{2}}\hspace{0.5cm}[\kern2pt{\textcolor{darkblue}{\textbf{\ipa{gi˥}}}}\kern2pt]} \hypertarget{gi\string_T2}{}
\markboth{\textcolor{darkblue}{\textbf{\ipa{gi˥}}} \textsubscript{2}}{}
\textcolor{teal}{\zh{动词}} \hspace{4pt} \zh{声调类:} H.
\zh{欠(钱)。} \textcolor{Sepia}{\selectlanguage{english}To owe money.} \textcolor{PineGreen}{\selectlanguage{french}Devoir de l'argent, avoir des dettes.}  ¶ \textcolor{darkblue}{\textbf{\ipa{ɖʐe˧ | tʰi˧-gi˥}}} \zh{欠钱} \textcolor{Sepia}{\selectlanguage{english}to owe money} \textcolor{PineGreen}{\selectlanguage{french}devoir de l'argent}  

\lhead{\firstmark}
\rhead{\botmark}

\subsection{\hspace{-0.5cm} {\Large \textcolor{darkblue}{\textbf{\ipa{gi˥\textsubscript{a}}}}}\hspace{0.5cm}[\kern2pt{\textcolor{darkblue}{\textbf{\ipa{gi˥}}}}\kern2pt]} \hypertarget{gi\string_Ta1}{}
\markboth{\textcolor{darkblue}{\textbf{\ipa{gi˥\textsubscript{a}}}}}{}
\textcolor{teal}{\zh{量词}} \hspace{4pt} \zh{声调类:} H\textsubscript{a}.
\ding{202} \zh{量词:一半。} \textcolor{Sepia}{\selectlanguage{english}A half.} \textcolor{PineGreen}{\selectlanguage{french}Une moitié, un demi.}  ¶ \textcolor{darkblue}{\textbf{\ipa{ɖɯ˧-gi˥}}} \zh{一半} \textcolor{Sepia}{\selectlanguage{english}one half} \textcolor{PineGreen}{\selectlanguage{french}une moitié}  
 ¶ \textcolor{darkblue}{\textbf{\ipa{tsʰe˩ʐv̩˩-gi˥}}} \zh{十四个半(注:这是为了确定调类而问的短语)} \textcolor{Sepia}{\selectlanguage{english}fourteen halves (combination elicited to determine the tonal category of the classifier)} \textcolor{PineGreen}{\selectlanguage{french}quatorze moitiés (combinaison permettant de déterminer la catégorie tonale de ce classificateur: elle établit que le ton est H1 et non H2)}  
 ¶ \textcolor{darkblue}{\textbf{\ipa{tv̩˧tsʰɯ˧ | ɖɯ˧-gi˥}}} \zh{一半的时间} \textcolor{Sepia}{\selectlanguage{english}half the time, half the duration} \textcolor{PineGreen}{\selectlanguage{french}la moitié du temps, la moitié de la durée}  
\ding{203} \zh{量词:一面(房屋的一面)。} \textcolor{Sepia}{\selectlanguage{english}A side; a direction.} \textcolor{PineGreen}{\selectlanguage{french}Un côté (d'une pièce, d'une maison…); une direction.}  ¶ \textcolor{darkblue}{\textbf{\ipa{ɖɯ˧-gi˧ hõ˧}}} \zh{往一个方向走、走自己的方向} \textcolor{Sepia}{\selectlanguage{english}to go in a certain direction, to go one's way} \textcolor{PineGreen}{\selectlanguage{french}aller d'un certain côté, aller de son côté}  
 ¶ \textcolor{darkblue}{\textbf{\ipa{ɖɯ˧-v̩˧ | ɖɯ˧-gi˧ hɯ˧}}} \zh{分开:每个人去自己的方向} \textcolor{Sepia}{\selectlanguage{english}to go each one's separate way; to go each in a different direction} \textcolor{PineGreen}{\selectlanguage{french}aller chacun de son côté; se séparer}  

\lhead{\firstmark}
\rhead{\botmark}

\subsection{\hspace{-0.5cm} {\Large \textcolor{darkblue}{\textbf{\ipa{gi˧dʑɯ˧}}}}\hspace{0.5cm}[\kern2pt{\textcolor{darkblue}{\textbf{\ipa{gi˥}}}}\kern2pt]} \hypertarget{gi\string_Mdz£M\string_M1}{}
\markboth{\textcolor{darkblue}{\textbf{\ipa{gi˧dʑɯ˧}}}}{}
\textcolor{teal}{\zh{名词}} \hspace{4pt} \zh{声调类:} M.
\zh{金沙江。} \textcolor{Sepia}{\selectlanguage{english}The Yangtze river (Yellow Sands river).} \textcolor{PineGreen}{\selectlanguage{french}Le fleuve Yangtze.}  ¶ \textcolor{darkblue}{\textbf{\ipa{gi˧dʑɯ˧-kʰi\#˥}}} \zh{金沙江边:奉科,拉伯……} \textcolor{Sepia}{\selectlanguage{english}the banks of the Yangtze river: Fengke, Labai...} \textcolor{PineGreen}{\selectlanguage{french}le bord du fleuve Yangtze: Fengke, Labai...}  
 ¶ \textcolor{darkblue}{\textbf{\ipa{gi˧dʑɯ˧-kʰi˧-hĩ\#˥}}} \zh{金沙江边的人:奉科人,拉伯人……} \textcolor{Sepia}{\selectlanguage{english}inhabitants of the banks of the Yangtze: people of Labai, Fengke...} \textcolor{PineGreen}{\selectlanguage{french}les riverains du Yangtze: gens de Labai, Fengke...}  

\lhead{\firstmark}
\rhead{\botmark}

\subsection{\hspace{-0.5cm} {\Large \textcolor{darkblue}{\textbf{\ipa{gi˧-nɑ˧mi\#˥}}}}\hspace{0.5cm}[\kern2pt{\textcolor{darkblue}{\textbf{\ipa{xxxx non-correspondance entre le nombre de morphèmes et le nombre de tons de morphèmes}}}}\kern2pt]} \hypertarget{gi\string_M-nA\string_Mmi\#\string_T1}{}
\markboth{\textcolor{darkblue}{\textbf{\ipa{gi˧-nɑ˧mi\#˥}}}}{}
\textcolor{teal}{\zh{名词}} \hspace{4pt} \zh{声调类:} \#H.
\zh{熊,母熊。} \textcolor{Sepia}{\selectlanguage{english}Bear; she-bear. There is no way to refer unambiguously to a female bear, as the same term is used for bears irrespective of sex.} \textcolor{PineGreen}{\selectlanguage{french}Ours (mâle ou femelle). Il n'existe pas de terme désignant de façon non ambiguë une ourse.}  ¶ \textcolor{darkblue}{\textbf{\ipa{gi˧-nɑ˧mi˧ tʰv̩˧-pʰo˩}}} \zh{这只熊} \textcolor{Sepia}{\selectlanguage{english}\mytextsc{n}+\mytextsc{dem}+\mytextsc{clf}} \textcolor{PineGreen}{\selectlanguage{french}\mytextsc{n}+\mytextsc{dem}+\mytextsc{clf}}  
 \zh{量词}: \textcolor{darkblue}{\textbf{\ipa{pʰo˧˥}}} 
\lhead{\firstmark}
\rhead{\botmark}

\subsection{\hspace{-0.5cm} {\Large \textcolor{darkblue}{\textbf{\ipa{gi˧-nɑ˧mi˧-pʰv̩\#˥}}}}\hspace{0.5cm}[\kern2pt{\textcolor{darkblue}{\textbf{\ipa{xxxx non-correspondance entre le nombre de morphèmes et le nombre de tons de morphèmes}}}}\kern2pt]} \hypertarget{gi\string_M-nA\string_Mmi\string_M-p\string_hv\string_=\#\string_T1}{}
\markboth{\textcolor{darkblue}{\textbf{\ipa{gi˧-nɑ˧mi˧-pʰv̩\#˥}}}}{}
\textcolor{teal}{\zh{名词}} \hspace{4pt} \zh{声调类:} \#H.
\zh{公熊。} \textcolor{Sepia}{\selectlanguage{english}He-bear, male bear.} \textcolor{PineGreen}{\selectlanguage{french}Ours (mâle).}  ¶ \textcolor{darkblue}{\textbf{\ipa{gi˧-nɑ˧mi˧-pʰv̩˧ tʰv̩˧-pʰo˩}}} \zh{这只公熊} \textcolor{Sepia}{\selectlanguage{english}\mytextsc{n}+\mytextsc{dem}+\mytextsc{clf}} \textcolor{PineGreen}{\selectlanguage{french}\mytextsc{n}+\mytextsc{dem}+\mytextsc{clf}}  
 \zh{量词}: \textcolor{darkblue}{\textbf{\ipa{pʰo˧˥}}} 
\lhead{\firstmark}
\rhead{\botmark}

\subsection{\hspace{-0.5cm} {\Large \textcolor{darkblue}{\textbf{\ipa{gi˧-nɑ˧mi˧-zo\#˥}}}}\hspace{0.5cm}[\kern2pt{\textcolor{darkblue}{\textbf{\ipa{xxxx non-correspondance entre le nombre de morphèmes et le nombre de tons de morphèmes}}}}\kern2pt]} \hypertarget{gi\string_M-nA\string_Mmi\string_M-zo\#\string_T1}{}
\markboth{\textcolor{darkblue}{\textbf{\ipa{gi˧-nɑ˧mi˧-zo\#˥}}}}{}
\textcolor{teal}{\zh{名词}} \hspace{4pt} \zh{声调类:} \#H.
\zh{小熊。} \textcolor{Sepia}{\selectlanguage{english}Little bear, bear cub.} \textcolor{PineGreen}{\selectlanguage{french}Ourson (de sexe masculin).}  ¶ \textcolor{darkblue}{\textbf{\ipa{gi˧-nɑ˧mi˧-zo˧ tʰv̩˧-ɭɯ\#˥}}} \zh{这只小熊} \textcolor{Sepia}{\selectlanguage{english}\mytextsc{n}+\mytextsc{dem}+\mytextsc{clf}} \textcolor{PineGreen}{\selectlanguage{french}\mytextsc{n}+\mytextsc{dem}+\mytextsc{clf}}  
 \zh{量词}: \textcolor{darkblue}{\textbf{\ipa{ɭɯ˧}}} 
\lhead{\firstmark}
\rhead{\botmark}

\subsection{\hspace{-0.5cm} {\Large \textcolor{darkblue}{\textbf{\ipa{gi˧zɯ\#˥}}}}\hspace{0.5cm}[\kern2pt{\textcolor{darkblue}{\textbf{\ipa{gi˧zɯ˧}}}}\kern2pt]} \hypertarget{gi\string_MzM\#\string_T1}{}
\markboth{\textcolor{darkblue}{\textbf{\ipa{gi˧zɯ\#˥}}}}{}
\textcolor{teal}{\zh{名词}} \hspace{4pt} \zh{声调类:} \#H.
\zh{弟弟(也可指更年轻的表弟)。} \textcolor{Sepia}{\selectlanguage{english}Little brother, younger brother; the term is also used to refer to younger cousins.} \textcolor{PineGreen}{\selectlanguage{french}Petit frère (employé aussi entre cousins).}  ¶ \textcolor{darkblue}{\textbf{\ipa{gi˧zɯ˧=ɻæ˥}}} \zh{联想复数:弟弟们,表弟们} \textcolor{Sepia}{\selectlanguage{english}\mytextsc{associative}: younger brothers, cousins...} \textcolor{PineGreen}{\selectlanguage{french}\mytextsc{associatif}: les petits frères}  
 \zh{量词}: \textcolor{darkblue}{\textbf{\ipa{v̩˧}}} 
\lhead{\firstmark}
\rhead{\botmark}

\subsection{\hspace{-0.5cm} {\Large \textcolor{darkblue}{\textbf{\ipa{gi˧zɯ˧-go˧mi\#˥}}}}\hspace{0.5cm}[\kern2pt{\textcolor{darkblue}{\textbf{\ipa{xxxx non-correspondance entre le nombre de morphèmes et le nombre de tons de morphèmes}}}}\kern2pt]} \hypertarget{gi\string_MzM\string_M-go\string_Mmi\#\string_T1}{}
\markboth{\textcolor{darkblue}{\textbf{\ipa{gi˧zɯ˧-go˧mi\#˥}}}}{}
\textcolor{teal}{\zh{名词}} \hspace{4pt} \zh{声调类:} \#H.
\zh{弟弟妹妹。} \textcolor{Sepia}{\selectlanguage{english}Younger siblings (brothers and sisters).} \textcolor{PineGreen}{\selectlanguage{french}Cadets: petits frères+ petites sœurs.} 
\lhead{\firstmark}
\rhead{\botmark}

\subsection{\hspace{-0.5cm} {\Large \textcolor{darkblue}{\textbf{\ipa{gi˩}}}}\hspace{0.5cm}[\kern2pt{\textcolor{darkblue}{\textbf{\ipa{gi˥}}}}\kern2pt]} \hypertarget{gi\string_B1}{}
\markboth{\textcolor{darkblue}{\textbf{\ipa{gi˩}}}}{}
\textcolor{teal}{\zh{名词}} \hspace{4pt} \zh{声调类:} L.
\zh{大熊。} \textcolor{Sepia}{\selectlanguage{english}Bear.} \textcolor{PineGreen}{\selectlanguage{french}Ours.} 
\lhead{\firstmark}
\rhead{\botmark}

\subsection{\hspace{-0.5cm} {\Large \textcolor{darkblue}{\textbf{\ipa{gi˩}}}}\hspace{0.5cm}[\kern2pt{\textcolor{darkblue}{\textbf{\ipa{gi˥}}}}\kern2pt]} \hypertarget{gi\string_B1}{}
\markboth{\textcolor{darkblue}{\textbf{\ipa{gi˩}}}}{}
\textcolor{teal}{\zh{名词}} \hspace{4pt} \zh{声调类:} L.
\zh{粮仓。} \textcolor{Sepia}{\selectlanguage{english}Granary (room within the house where grain is stored).} \textcolor{PineGreen}{\selectlanguage{french}Grenier à céréales; selon M23, est le lieu dans la maison où on stocke les céréales.}  ¶ \textcolor{darkblue}{\textbf{\ipa{gi˧mi˧}}} \zh{大粮仓} \textcolor{Sepia}{\selectlanguage{english}large granary} \textcolor{PineGreen}{\selectlanguage{french}grand grenier}  
 ¶ \textcolor{darkblue}{\textbf{\ipa{gi˩zo˩˥}}} \zh{小粮仓} \textcolor{Sepia}{\selectlanguage{english}small granary} \textcolor{PineGreen}{\selectlanguage{french}petit grenier}  
 ¶ \textcolor{darkblue}{\textbf{\ipa{njɤ˧ | gi˩ gv̩˩-zo˩-ho˥}}} \zh{我应该修粮仓!} \textcolor{Sepia}{\selectlanguage{english}I shall have to repair the granary!} \textcolor{PineGreen}{\selectlanguage{french}il va falloir que je répare le grenier à céréales!}  
 \zh{量词}: \textcolor{darkblue}{\textbf{\ipa{ɭɯ˧}}} 
\lhead{\firstmark}
\rhead{\botmark}

\subsection{\hspace{-0.5cm} {\Large \textcolor{darkblue}{\textbf{\ipa{gi˩\textsubscript{a}}}}}\hspace{0.5cm}[\kern2pt{\textcolor{darkblue}{\textbf{\ipa{gi˩˥}}}}\kern2pt]} \hypertarget{gi\string_Ba1}{}
\markboth{\textcolor{darkblue}{\textbf{\ipa{gi˩\textsubscript{a}}}}}{}
\textcolor{teal}{\zh{形容词}} \hspace{4pt} \zh{声调类:} L\textsubscript{a}.
\zh{真,真的。} \textcolor{Sepia}{\selectlanguage{english}True, real; really, truly.} \textcolor{PineGreen}{\selectlanguage{french}Vrai, vraiment.}  ¶ \textcolor{darkblue}{\textbf{\ipa{mɤ˧-gi˩!}}} \zh{不是的! / 不是真的!} \textcolor{Sepia}{\selectlanguage{english}(It) is not true!} \textcolor{PineGreen}{\selectlanguage{french}c'est pas vrai!}  
 ¶ \textcolor{darkblue}{\textbf{\ipa{ə˩-gi˩˥?}}} \zh{对吧? / 对吗?} \textcolor{Sepia}{\selectlanguage{english}Right? / Is that true? / It is true, isn't it?} \textcolor{PineGreen}{\selectlanguage{french}c'est vrai? / n'est-ce pas?}  
 ¶ \textcolor{darkblue}{\textbf{\ipa{ə˩-gi˩˥ ? – gi˩˥!}}} \zh{对吧? -对的!} \textcolor{Sepia}{\selectlanguage{english}Is that right? - Yes, it is! (One speaker asks for confirmation; the other provides confirmation.)} \textcolor{PineGreen}{\selectlanguage{french}N'est-ce pas? - Oui-da! (Le locuteur demande à son interlocuteur de confirmer qu’il adhère à son propos; l'autre donne son assentiment.)}  
 ¶ \textcolor{darkblue}{\textbf{\ipa{gi˩-hĩ˩ ʐwɤ˥}}} \zh{说实话,老实说} \textcolor{Sepia}{\selectlanguage{english}to speak the truth, to tell the truth} \textcolor{PineGreen}{\selectlanguage{french}dire vrai}  
 ¶ \textcolor{darkblue}{\textbf{\ipa{gi˩˥ | -gɯ˩˥}}} \zh{真的,真正的} \textcolor{Sepia}{\selectlanguage{english}truly, veritably} \textcolor{PineGreen}{\selectlanguage{french}vraiment, véritablement}  

\lhead{\firstmark}
\rhead{\botmark}

\subsection{\hspace{-0.5cm} {\Large \textcolor{darkblue}{\textbf{\ipa{gi˩kɯ˩}}}}\hspace{0.5cm}[\kern2pt{\textcolor{darkblue}{\textbf{\ipa{gi˧kɯ˧}}}}\kern2pt]} \hypertarget{gi\string_BkM\string_B1}{}
\markboth{\textcolor{darkblue}{\textbf{\ipa{gi˩kɯ˩}}}}{}
\textcolor{teal}{\zh{名词}} \hspace{4pt} \zh{声调类:} L.
\zh{麝香(直译:大熊胆)。} \textcolor{Sepia}{\selectlanguage{english}Musk (literally: 'bear's gall').} \textcolor{PineGreen}{\selectlanguage{french}Musc (littéralement: 'bile d'ours').} 
\lhead{\firstmark}
\rhead{\botmark}

\subsection{\hspace{-0.5cm} {\Large \textcolor{darkblue}{\textbf{\ipa{‑gi˧˥}}}}\hspace{0.5cm}[\kern2pt{\textcolor{darkblue}{\textbf{\ipa{gi˧˥}}}}\kern2pt]} \hypertarget{‑gi\string_M\string_T1}{}
\markboth{\textcolor{darkblue}{\textbf{\ipa{‑gi˧˥}}}}{}
\textcolor{teal}{\zh{后置词}} \hspace{4pt} \zh{声调类:} MH.
\zh{后面,(最)后。} \textcolor{Sepia}{\selectlanguage{english}Behind.} \textcolor{PineGreen}{\selectlanguage{french}Derrière.}  ¶ \textcolor{darkblue}{\textbf{\ipa{ə˧mɑ˧-gi˧˥}}} \zh{妈妈后面} \textcolor{Sepia}{\selectlanguage{english}behind mummy} \textcolor{PineGreen}{\selectlanguage{french}derrière maman}  
 ¶ \textcolor{darkblue}{\textbf{\ipa{lɑ˧-gi˧˥}}} \zh{老虎后面} \textcolor{Sepia}{\selectlanguage{english}behind the tiger} \textcolor{PineGreen}{\selectlanguage{french}derrière le tigre}  
 ¶ \textcolor{darkblue}{\textbf{\ipa{bo˩-gi˥}}} \zh{猪后面} \textcolor{Sepia}{\selectlanguage{english}behind the pig} \textcolor{PineGreen}{\selectlanguage{french}derrière le cochon}  
 ¶ \textcolor{darkblue}{\textbf{\ipa{mv̩˩-gi˥}}} \zh{女儿后面} \textcolor{Sepia}{\selectlanguage{english}behind the daughter} \textcolor{PineGreen}{\selectlanguage{french}derrière la fille}  
 ¶ \textcolor{darkblue}{\textbf{\ipa{ʐwæ˧-gi˥}}} \zh{马后面} \textcolor{Sepia}{\selectlanguage{english}behind the horse} \textcolor{PineGreen}{\selectlanguage{french}derrière le cheval}  
 ¶ \textcolor{darkblue}{\textbf{\ipa{ʈʂʰɯ˧-gi˥ | tʰi˧-tɕʰo˩}}} \zh{藏那后面} \textcolor{Sepia}{\selectlanguage{english}to hide in there (literally 'behind there')} \textcolor{PineGreen}{\selectlanguage{french}se cacher là-derrière}  
 ¶ \textcolor{darkblue}{\textbf{\ipa{no˧-gi˧ njɤ˥ ʈʂwæ˩!}}} \zh{我跟你走! / 我都按你说的来做吧!} \textcolor{Sepia}{\selectlanguage{english}I follow in your footsteps! / I follow you! / I imitate you!} \textcolor{PineGreen}{\selectlanguage{french}je te suis, je marche dans tes pas; je t'imite}  
 ¶ \textcolor{darkblue}{\textbf{\ipa{ɖɯ˧-v̩˧-gi˧˥, | ɖɯ˧-v̩˧ hwæ˧!}}} \zh{一个接着一个地买(情景:一个人接二连三地买马,最后组成自己的马帮队)} \textcolor{Sepia}{\selectlanguage{english}to buy one after the other (context: someone buys one horse after the other, to put together a complete caravan of his own)} \textcolor{PineGreen}{\selectlanguage{french}en acheter un après l'autre (contexte: un caravanier achète des chevaux l'un après l'autre, afin de se constituer sa propre caravane)}  
 ¶ \textcolor{darkblue}{\textbf{\ipa{[F5] gi˧˥ | ɖɯ˧-qɑ˩ gv̩˩-bi˩!}}} \zh{再做一捆吧!(情景:女人们在纺麻线,工作了很久,一个人就说:“再做最后一捆(就收工吧)!”)} \textcolor{Sepia}{\selectlanguage{english}Let's do one last bundle! (Context: women are extracting flax fiber, processing bundle after bundle; towards the end of a long work session, someone says: “Let's do one last bundle! / One last bundle and we shall call it a day!”)} \textcolor{PineGreen}{\selectlanguage{french}On va en faire une dernière botte! (contexte: on travaille le lin, botte après botte; vers la fin d'une longue séance de travail, quelqu'un annonce: “On va en faire une dernière botte! / Une dernière botte, et on s'arrête!”)}  
 ¶ \textcolor{darkblue}{\textbf{\ipa{gi˧-se˧}}} \zh{在后面走,在后面跟着} \textcolor{Sepia}{\selectlanguage{english}to walk after, to follow after} \textcolor{PineGreen}{\selectlanguage{french}marcher derrière, suivre derrière}  

\lhead{\firstmark}
\rhead{\botmark}

\subsection{\hspace{-0.5cm} {\Large \textcolor{darkblue}{\textbf{\ipa{go˧bɤ˩}}}}\hspace{0.5cm}[\kern2pt{\textcolor{darkblue}{\textbf{\ipa{go˩bɤ˥}}}}\kern2pt]} \hypertarget{go\string_Mb7\string_B1}{}
\markboth{\textcolor{darkblue}{\textbf{\ipa{go˧bɤ˩}}}}{}
\textcolor{teal}{\zh{名词}} \hspace{4pt} \zh{声调类:} L\#.
\zh{庙,寺。} \textcolor{Sepia}{\selectlanguage{english}Temple, monastery.} \textcolor{PineGreen}{\selectlanguage{french}Temple, monastère.}  \zh{【借词】}\zh{藏语} dgon pa
 \zh{量词}: \textcolor{darkblue}{\textbf{\ipa{ɭɯ˧}}} 
\lhead{\firstmark}
\rhead{\botmark}

\subsection{\hspace{-0.5cm} {\Large \textcolor{darkblue}{\textbf{\ipa{go˧mi˧}}}}\hspace{0.5cm}[\kern2pt{\textcolor{darkblue}{\textbf{\ipa{go˧mi˧}}}}\kern2pt]} \hypertarget{go\string_Mmi\string_M1}{}
\markboth{\textcolor{darkblue}{\textbf{\ipa{go˧mi˧}}}}{}
\textcolor{teal}{\zh{名词}} \hspace{4pt} \zh{声调类:} M.
\zh{妹妹。} \textcolor{Sepia}{\selectlanguage{english}Younger sister.} \textcolor{PineGreen}{\selectlanguage{french}Petite soeur (employé aussi pour les cousines plus jeunes).}  ¶ \textcolor{darkblue}{\textbf{\ipa{go˧mi˧=ɻæ˩}}} \zh{联想复数:妹妹们,表妹们} \textcolor{Sepia}{\selectlanguage{english}\mytextsc{associative}: younger sisters, younger cousins} \textcolor{PineGreen}{\selectlanguage{french}\mytextsc{associatif}: les petites sœurs, les jeunes cousines}  
 \zh{量词}: \textcolor{darkblue}{\textbf{\ipa{v̩˧}}} 
\lhead{\firstmark}
\rhead{\botmark}

\subsection{\hspace{-0.5cm} {\Large \textcolor{darkblue}{\textbf{\ipa{go˩\textsubscript{a}}}}}\hspace{0.5cm}[\kern2pt{\textcolor{darkblue}{\textbf{\ipa{go˧˥}}}}\kern2pt]} \hypertarget{go\string_Ba1}{}
\markboth{\textcolor{darkblue}{\textbf{\ipa{go˩\textsubscript{a}}}}}{}
\textcolor{teal}{\zh{动词}} \hspace{4pt} \zh{声调类:} L\textsubscript{a}.
\zh{痛,病 (生病)。} \textcolor{Sepia}{\selectlanguage{english}To suffer; to be sick, to be ill.} \textcolor{PineGreen}{\selectlanguage{french}Souffrir, avoir mal; être malade.}  ¶ \textcolor{darkblue}{\textbf{\ipa{njɤ˧ | go˩˥!}}} \zh{我痛!} \textcolor{Sepia}{\selectlanguage{english}I am suffering! / It hurts!} \textcolor{PineGreen}{\selectlanguage{french}J'ai mal!}  
 ¶ \textcolor{darkblue}{\textbf{\ipa{njɤ˧ | go˩˥ | ʐwæ˩˥!}}} \zh{我好疼!} \textcolor{Sepia}{\selectlanguage{english}I am suffering a lot! / It hurts a lot!} \textcolor{PineGreen}{\selectlanguage{french}J'ai très mal!}  
 ¶ \textcolor{darkblue}{\textbf{\ipa{go˩-hĩ˩˥}}} \zh{病人,病的(人)} \textcolor{Sepia}{\selectlanguage{english}\mytextsc{nmlz}: patient, sick person} \textcolor{PineGreen}{\selectlanguage{french}\mytextsc{nmlz}: patient, malade}  
 ¶ \textcolor{darkblue}{\textbf{\ipa{hĩ˧ | go˩-hĩ˩˥}}} \zh{病人} \textcolor{Sepia}{\selectlanguage{english}sick person, person who is ill, patient} \textcolor{PineGreen}{\selectlanguage{french}patient, personne malade, malade}  
 ¶ \textcolor{darkblue}{\textbf{\ipa{bi˧mi˧ go˩}}} \zh{肚子疼} \textcolor{Sepia}{\selectlanguage{english}to have stomach-ache} \textcolor{PineGreen}{\selectlanguage{french}avoir mal au ventre}  

\lhead{\firstmark}
\rhead{\botmark}

\subsection{\hspace{-0.5cm} {\Large \textcolor{darkblue}{\textbf{\ipa{go˩bi˧}}}}\hspace{0.5cm}[\kern2pt{\textcolor{darkblue}{\textbf{\ipa{go˧bi˧}}}}\kern2pt]} \hypertarget{go\string_Bbi\string_M1}{}
\markboth{\textcolor{darkblue}{\textbf{\ipa{go˩bi˧}}}}{}
\textcolor{teal}{\zh{名词}} \hspace{4pt} \zh{声调类:} LM.
\zh{丽江城。} \textcolor{Sepia}{\selectlanguage{english}The city of Lijiang.} \textcolor{PineGreen}{\selectlanguage{french}Lijiang (la ville).}  ¶ \textcolor{darkblue}{\textbf{\ipa{go˩bi˧-ɖʐɯ˧qo˩}}} \zh{丽江城} \textcolor{Sepia}{\selectlanguage{english}the city of Lijiang} \textcolor{PineGreen}{\selectlanguage{french}la ville de Lijiang}  

\lhead{\firstmark}
\rhead{\botmark}

\subsection{\hspace{-0.5cm} {\Large \textcolor{darkblue}{\textbf{\ipa{go˩bo˥}}}}\hspace{0.5cm}[\kern2pt{\textcolor{darkblue}{\textbf{\ipa{go˧bo˧}}}}\kern2pt]} \hypertarget{go\string_Bbo\string_T1}{}
\markboth{\textcolor{darkblue}{\textbf{\ipa{go˩bo˥}}}}{}
\textcolor{teal}{\zh{名词}} \hspace{4pt} \zh{声调类:} LH.
\zh{牲畜。} \textcolor{Sepia}{\selectlanguage{english}Livestock.} \textcolor{PineGreen}{\selectlanguage{french}Bétail, animaux domestiques.}  \zh{量词}: \textcolor{darkblue}{\textbf{\ipa{pʰo˧˥}}} 
\lhead{\firstmark}
\rhead{\botmark}

\subsection{\hspace{-0.5cm} {\Large \textcolor{darkblue}{\textbf{\ipa{gɯ˩\textsubscript{a}}}} \textsubscript{1}}\hspace{0.5cm}[\kern2pt{\textcolor{darkblue}{\textbf{\ipa{gɯ˩˥}}}}\kern2pt]} \hypertarget{gM\string_Ba1}{}
\markboth{\textcolor{darkblue}{\textbf{\ipa{gɯ˩\textsubscript{a}}}} \textsubscript{1}}{}
\textcolor{teal}{\zh{动词}} \hspace{4pt} \zh{声调类:} L\textsubscript{a}.
\zh{相信。} \textcolor{Sepia}{\selectlanguage{english}To believe.} \textcolor{PineGreen}{\selectlanguage{french}Croire.}  ¶ \textcolor{darkblue}{\textbf{\ipa{ʈʂʰɯ˧-ɳɯ˧ ʐwɤ˩-hĩ˩, | njɤ˧ | mɤ˧-gɯ˩!}}} \zh{他说的话,我不相信!} \textcolor{Sepia}{\selectlanguage{english}I do not believe what (s)he says!} \textcolor{PineGreen}{\selectlanguage{french}Je ne crois pas ce qu'il dit! / Je ne le crois pas! / Je ne crois pas un mot de ce qu'il raconte!}  

\lhead{\firstmark}
\rhead{\botmark}

\subsection{\hspace{-0.5cm} {\Large \textcolor{darkblue}{\textbf{\ipa{gɯ˩\textsubscript{a}}}} \textsubscript{2}}\hspace{0.5cm}[\kern2pt{\textcolor{darkblue}{\textbf{\ipa{gɯ˩˥}}}}\kern2pt]} \hypertarget{gM\string_Ba2}{}
\markboth{\textcolor{darkblue}{\textbf{\ipa{gɯ˩\textsubscript{a}}}} \textsubscript{2}}{}
\textcolor{teal}{\zh{形容词}} \hspace{4pt} \zh{声调类:} L\textsubscript{a}.
\zh{真,真的。} \textcolor{Sepia}{\selectlanguage{english}True, authentic, veritable.} \textcolor{PineGreen}{\selectlanguage{french}Vrai, authentique, véritable.}  ¶ \textcolor{darkblue}{\textbf{\ipa{mɤ˧-gɯ˩}}} \zh{不是真的} \textcolor{Sepia}{\selectlanguage{english}not true} \textcolor{PineGreen}{\selectlanguage{french}pas vrai}  
 ¶ \textcolor{darkblue}{\textbf{\ipa{gɯ˩-hĩ˩˥}}} \zh{真的} \textcolor{Sepia}{\selectlanguage{english}\mytextsc{nmlz}} \textcolor{PineGreen}{\selectlanguage{french}\mytextsc{nmlz}}  
 ¶ \textcolor{darkblue}{\textbf{\ipa{ə˩-gɯ˩˥?}}} \zh{真的吗?} \textcolor{Sepia}{\selectlanguage{english}Is that true?} \textcolor{PineGreen}{\selectlanguage{french}c'est vrai?}  
 ¶ \textcolor{darkblue}{\textbf{\ipa{gɯ˩ wɤ˩-ɻ̍˥!}}} \zh{就是真的啊! / 的确是这样啊!} \textcolor{Sepia}{\selectlanguage{english}It's really like that! / Yes, it is indeed true!} \textcolor{PineGreen}{\selectlanguage{french}C'est bien ça! / C'est réellement ainsi!}  
 ¶ \textcolor{darkblue}{\textbf{\ipa{gɯ˩-ʝi˥?}}} \zh{原来是这样吗?} \textcolor{Sepia}{\selectlanguage{english}Really?} \textcolor{PineGreen}{\selectlanguage{french}C'est vrai? Vraiment?}  
 ¶ \textcolor{darkblue}{\textbf{\ipa{gɯ˩ ʂv̩˩ɖv̩˩˥}}} \zh{相信} \textcolor{Sepia}{\selectlanguage{english}to believe in (something); literally: 'to think (that something is) true'} \textcolor{PineGreen}{\selectlanguage{french}croire (quelqu'un, quelque chose): littéralement “penser que c'est vrai”}  
 ¶ \textcolor{darkblue}{\textbf{\ipa{gɯ˧ ʐwɤ˧}}} \zh{说实话} \textcolor{Sepia}{\selectlanguage{english}to say the truth} \textcolor{PineGreen}{\selectlanguage{french}dire la vérité}  

\lhead{\firstmark}
\rhead{\botmark}

\subsection{\hspace{-0.5cm} {\Large \textcolor{darkblue}{\textbf{\ipa{gɯ˩ɭɯ˧˥}}}}\hspace{0.5cm}[\kern2pt{\textcolor{darkblue}{\textbf{\ipa{gɯ˩ɭɯ˧˥}}}}\kern2pt]} \hypertarget{gM\string_Bl\string_RM\string_M\string_T1}{}
\markboth{\textcolor{darkblue}{\textbf{\ipa{gɯ˩ɭɯ˧˥}}}}{}
\textcolor{teal}{\zh{动词}} \hspace{4pt} \zh{声调类:} LM+MH\#.
\zh{揉。} \textcolor{Sepia}{\selectlanguage{english}To rub, to knead (e.g. rub one's hands).} \textcolor{PineGreen}{\selectlanguage{french}Frotter (ex.: se frotter les yeux, frotter un vêtement).}  ¶ \textcolor{darkblue}{\textbf{\ipa{gɯ˩ɭɯ˧-ze˥}}} \zh{揉了} \textcolor{Sepia}{\selectlanguage{english}\mytextsc{pfv}} \textcolor{PineGreen}{\selectlanguage{french}\mytextsc{pfv}}  
 ¶ \textcolor{darkblue}{\textbf{\ipa{le˧-gɯ˩ɭɯ˩+ze˩}}} \zh{揉了} \textcolor{Sepia}{\selectlanguage{english}\mytextsc{accomp} \string_ \mytextsc{pfv}} \textcolor{PineGreen}{\selectlanguage{french}\mytextsc{accomp} \string_ \mytextsc{pfv}}  
 ¶ \textcolor{darkblue}{\textbf{\ipa{le˧-gɯ˩ɭɯ˩\textasciitilde{}le˧-gɯ˩ɭɯ˩}}} \zh{揉一揉} \textcolor{Sepia}{\selectlanguage{english}\mytextsc{accomp} \mytextsc{red} \mytextsc{pfv}} \textcolor{PineGreen}{\selectlanguage{french}\mytextsc{accomp} \mytextsc{red} \mytextsc{pfv}}  

\lhead{\firstmark}
\rhead{\botmark}

\subsection{\hspace{-0.5cm} {\Large \textcolor{darkblue}{\textbf{\ipa{gv̩˧}}} \textsubscript{1}}\hspace{0.5cm}[\kern2pt{\textcolor{darkblue}{\textbf{\ipa{gv̩˥}}}}\kern2pt]} \hypertarget{gv\string_=\string_M1}{}
\markboth{\textcolor{darkblue}{\textbf{\ipa{gv̩˧}}} \textsubscript{1}}{}
\textcolor{teal}{\zh{动词}} \hspace{4pt} \zh{声调类:} M\textsubscript{c}.
\zh{过去 (时间)、过,发生。} \textcolor{Sepia}{\selectlanguage{english}To flow, to go by, to elapse (time); to take place, to occur (event).} \textcolor{PineGreen}{\selectlanguage{french}S'écouler, passer (le temps passe); se passer (un événement).}  ¶ \textcolor{darkblue}{\textbf{\ipa{le˧-gv̩˩-ze˩}}} \zh{已经过去了} \textcolor{Sepia}{\selectlanguage{english}\mytextsc{accomp} \string_ \mytextsc{pfv}} \textcolor{PineGreen}{\selectlanguage{french}\mytextsc{accomp} \string_ \mytextsc{pfv}}  
 ¶ \textcolor{darkblue}{\textbf{\ipa{ɖɯ˧-ɭɯ˧ gv̩˧}}} \zh{一个小时过去了} \textcolor{Sepia}{\selectlanguage{english}an hour goes by} \textcolor{PineGreen}{\selectlanguage{french}une heure se passe}  
 ¶ \textcolor{darkblue}{\textbf{\ipa{tsʰe˩-ɲi˩ gv̩˩-ze˥!}}} \zh{十天过去了} \textcolor{Sepia}{\selectlanguage{english}Ten days have gone by/ten days have elapsed} \textcolor{PineGreen}{\selectlanguage{french}Dix jours ont passé!}  
 ¶ \textcolor{darkblue}{\textbf{\ipa{mɤ˧-gv̩˧-ze˧!}}} \zh{不好了! / 不行了!} \textcolor{Sepia}{\selectlanguage{english}It won't do! / It won't work! / It's no good!} \textcolor{PineGreen}{\selectlanguage{french}(ah là là,) ça ne va plus!}  
 ¶ \textcolor{darkblue}{\textbf{\ipa{ʈʂʰɯ˧ne˧-ʝi˥ | gv̩˧, -tsɯ˩-mv̩˩!}}} \zh{据说是这样发生的!} \textcolor{Sepia}{\selectlanguage{english}They say that's how it happened!} \textcolor{PineGreen}{\selectlanguage{french}Ca c'est passé comme ça, à ce qu'on raconte!}  

\lhead{\firstmark}
\rhead{\botmark}

\subsection{\hspace{-0.5cm} {\Large \textcolor{darkblue}{\textbf{\ipa{gv̩˧}}} \textsubscript{2}}\hspace{0.5cm}[\kern2pt{\textcolor{darkblue}{\textbf{\ipa{gv̩˥}}}}\kern2pt]} \hypertarget{gv\string_=\string_M2}{}
\markboth{\textcolor{darkblue}{\textbf{\ipa{gv̩˧}}} \textsubscript{2}}{}
\textcolor{teal}{\zh{形容词}} \hspace{4pt} \zh{声调类:} M.
\zh{好(心好)。} \textcolor{Sepia}{\selectlanguage{english}Good (good heart).} \textcolor{PineGreen}{\selectlanguage{french}Bon (bon cœur).}  ¶ \textcolor{darkblue}{\textbf{\ipa{ɖwæ˧˥ | gv̩˧!}}} \zh{很好!} \textcolor{Sepia}{\selectlanguage{english}\mytextsc{intensive}.very} \textcolor{PineGreen}{\selectlanguage{french}\mytextsc{intensif}.très}  
 ¶ \textcolor{darkblue}{\textbf{\ipa{mɤ˧-gv̩˧!}}} \zh{不好} \textcolor{Sepia}{\selectlanguage{english}\mytextsc{neg}} \textcolor{PineGreen}{\selectlanguage{french}\mytextsc{neg}}  

\lhead{\firstmark}
\rhead{\botmark}

\subsection{\hspace{-0.5cm} {\Large \textcolor{darkblue}{\textbf{\ipa{gv̩˧}}} \textsubscript{3}}\hspace{0.5cm}[\kern2pt{\textcolor{darkblue}{\textbf{\ipa{gv̩˥}}}}\kern2pt]} \hypertarget{gv\string_=\string_M3}{}
\markboth{\textcolor{darkblue}{\textbf{\ipa{gv̩˧}}} \textsubscript{3}}{}
\textcolor{teal}{\zh{数词}} \hspace{4pt} \zh{声调类:} M? H\#? (pas L).
\zh{9。} \textcolor{Sepia}{\selectlanguage{english}9.} \textcolor{PineGreen}{\selectlanguage{french}9.} 
\lhead{\firstmark}
\rhead{\botmark}

\subsection{\hspace{-0.5cm} {\Large \textcolor{darkblue}{\textbf{\ipa{gv̩˧}}} \textsubscript{4}}\hspace{0.5cm}[\kern2pt{\textcolor{darkblue}{\textbf{\ipa{gv̩˥}}}}\kern2pt]} \hypertarget{gv\string_=\string_M4}{}
\markboth{\textcolor{darkblue}{\textbf{\ipa{gv̩˧}}} \textsubscript{4}}{}
\textcolor{teal}{\zh{动词}} \hspace{4pt} \zh{声调类:} M.
\zh{系词。} \textcolor{Sepia}{\selectlanguage{english}To be; to become.} \textcolor{PineGreen}{\selectlanguage{french}Être, devenir (verbe statif).}  ¶ \textcolor{darkblue}{\textbf{\ipa{ʈʂʰɯ˧ | no˧ | ɲi˧gɤ˧ | ʂwæ˧-mɤ˧-gv̩˧!}}} \zh{她的鼻子没有你的直!(关于一个鼻子比较扁的小女孩)} \textcolor{Sepia}{\selectlanguage{english}Her nose is less straight than yours! (About a little girl whose nose does not resemble her father's straight nose)} \textcolor{PineGreen}{\selectlanguage{french}elle a le nez moins droit que toi! (Au sujet d'une petite fille dont le nez ne ressemble pas au nez droit de son père)}  
 ¶ \textcolor{darkblue}{\textbf{\ipa{ʐæ˧ni˩ | mɤ˧-gv̩˧}}} \zh{个子不高} \textcolor{Sepia}{\selectlanguage{english}not tall, not impressive, not great-looking} \textcolor{PineGreen}{\selectlanguage{french}pas bien grand (en taille), pas bien impressionnant}  

\lhead{\firstmark}
\rhead{\botmark}

\subsection{\hspace{-0.5cm} {\Large \textcolor{darkblue}{\textbf{\ipa{gv̩˥}}}}\hspace{0.5cm}[\kern2pt{\textcolor{darkblue}{\textbf{\ipa{gv̩˥}}}}\kern2pt]} \hypertarget{gv\string_=\string_T1}{}
\markboth{\textcolor{darkblue}{\textbf{\ipa{gv̩˥}}}}{}
\textcolor{teal}{\zh{动词}} \hspace{4pt} \zh{声调类:} H.
\zh{过(一条河、一个湖……)。} \textcolor{Sepia}{\selectlanguage{english}To cross, to get over (a river, a lake…).} \textcolor{PineGreen}{\selectlanguage{french}Passer, traverser (un cours d'eau, un lac…).}  ¶ \textcolor{darkblue}{\textbf{\ipa{dʑɯ˩ gv̩˩˥}}} \zh{过河} \textcolor{Sepia}{\selectlanguage{english}to cross a river} \textcolor{PineGreen}{\selectlanguage{french}traverser l'eau/traverser la rivière}  

\lhead{\firstmark}
\rhead{\botmark}

\subsection{\hspace{-0.5cm} {\Large \textcolor{darkblue}{\textbf{\ipa{gv̩˥}}}}\hspace{0.5cm}[\kern2pt{\textcolor{darkblue}{\textbf{\ipa{gv̩˥}}}}\kern2pt]} \hypertarget{gv\string_=\string_T1}{}
\markboth{\textcolor{darkblue}{\textbf{\ipa{gv̩˥}}}}{}
\textcolor{teal}{\zh{名词}} \hspace{4pt} \zh{声调类:} \#H.
\zh{马槽。} \textcolor{Sepia}{\selectlanguage{english}Manger.} \textcolor{PineGreen}{\selectlanguage{french}Auge, mangeoire.}  ¶ \textcolor{darkblue}{\textbf{\ipa{ʐwæ˧gv̩\#˥}}} \zh{马槽} \textcolor{Sepia}{\selectlanguage{english}horse's manger} \textcolor{PineGreen}{\selectlanguage{french}auge du cheval}  
 \zh{量词}: \textcolor{darkblue}{\textbf{\ipa{ɭɯ˧}}} 
\lhead{\firstmark}
\rhead{\botmark}

\subsection{\hspace{-0.5cm} {\Large \textcolor{darkblue}{\textbf{\ipa{gv̩˩\textsubscript{a}}}} \textsubscript{1}}\hspace{0.5cm}[\kern2pt{\textcolor{darkblue}{\textbf{\ipa{gv̩˩˥}}}}\kern2pt]} \hypertarget{gv\string_=\string_Ba1}{}
\markboth{\textcolor{darkblue}{\textbf{\ipa{gv̩˩\textsubscript{a}}}} \textsubscript{1}}{}
\textcolor{teal}{\zh{动词}} \hspace{4pt} \zh{声调类:} L\textsubscript{a}.
\ding{202} \zh{做(饭)。} \textcolor{Sepia}{\selectlanguage{english}To prepare (a meal), to cook.} \textcolor{PineGreen}{\selectlanguage{french}Cuisiner, préparer (un repas, de la nourriture).}  ¶ \textcolor{darkblue}{\textbf{\ipa{hɑ˧ gv̩˥}}} \zh{做饭} \textcolor{Sepia}{\selectlanguage{english}to cook, to prepare a meal} \textcolor{PineGreen}{\selectlanguage{french}faire la cuisine, cuisiner}  
 ¶ \textcolor{darkblue}{\textbf{\ipa{le˧-gv̩˩-ze˩}}} \zh{做(饭)了} \textcolor{Sepia}{\selectlanguage{english}\mytextsc{accomp} \string_ \mytextsc{pfv}} \textcolor{PineGreen}{\selectlanguage{french}\mytextsc{accomp} \string_ \mytextsc{pfv}}  
 ¶ \textcolor{darkblue}{\textbf{\ipa{njɤ˧ | hɑ˧ gv̩˥-bi˩!}}} \zh{我来做饭吧!} \textcolor{Sepia}{\selectlanguage{english}Let me do the cooking! / I'm doing the cooking!} \textcolor{PineGreen}{\selectlanguage{french}je vais faire la cuisine!}  
\ding{203} \zh{盖、建 (房子)。} \textcolor{Sepia}{\selectlanguage{english}To construct, to build (a house).} \textcolor{PineGreen}{\selectlanguage{french}Construire (une maison).}  ¶ \textcolor{darkblue}{\textbf{\ipa{ʑi˧qʰwɤ˧ gv̩˩}}} \zh{建房} \textcolor{Sepia}{\selectlanguage{english}to build a house} \textcolor{PineGreen}{\selectlanguage{french}construire un bâtiment}  
\ding{204} \zh{修理、做出来(工具)。} \textcolor{Sepia}{\selectlanguage{english}To repair; to make (a tool, a machine…).} \textcolor{PineGreen}{\selectlanguage{french}Fabriquer ou réparer (un outil).}  ¶ \textcolor{darkblue}{\textbf{\ipa{le˧-gv̩˧\textasciitilde{}gv̩˥}}} \zh{\mytextsc{重叠:修理}} \textcolor{Sepia}{\selectlanguage{english}\mytextsc{red}: to repair} \textcolor{PineGreen}{\selectlanguage{french}\mytextsc{red}: réparer}  
 ¶ \textcolor{darkblue}{\textbf{\ipa{le˧-gv̩˩ | le˧-tʰv̩˧-ze˧!}}} \zh{修理好了!/ 修理出来了!} \textcolor{Sepia}{\selectlanguage{english}It's repaired! / It's done! / I have finished doing it!} \textcolor{PineGreen}{\selectlanguage{french}Ca y est, c'est réparé/c'est fabriqué/c'est fini!}  

\lhead{\firstmark}
\rhead{\botmark}

\subsection{\hspace{-0.5cm} {\Large \textcolor{darkblue}{\textbf{\ipa{gv̩˩\textsubscript{a}}}} \textsubscript{2}}\hspace{0.5cm}[\kern2pt{\textcolor{darkblue}{\textbf{\ipa{gv̩˩˥}}}}\kern2pt]} \hypertarget{gv\string_=\string_Ba2}{}
\markboth{\textcolor{darkblue}{\textbf{\ipa{gv̩˩\textsubscript{a}}}} \textsubscript{2}}{}
\textcolor{teal}{\zh{动词}} \hspace{4pt} \zh{声调类:} L\textsubscript{a}.
\zh{收拾。} \textcolor{Sepia}{\selectlanguage{english}To tidy up, to sort out.} \textcolor{PineGreen}{\selectlanguage{french}Ranger.}  ¶ \textcolor{darkblue}{\textbf{\ipa{tʰi˧-gv̩˧\textasciitilde{}gv̩˥}}} \zh{\mytextsc{dur}} \textcolor{Sepia}{\selectlanguage{english}\mytextsc{dur}} \textcolor{PineGreen}{\selectlanguage{french}\mytextsc{dur}}  
 ¶ \textcolor{darkblue}{\textbf{\ipa{ɖɯ˧-gv̩˧\textasciitilde{}gv̩˥-ɻ̍˩}}} \zh{收拾一下} \textcolor{Sepia}{\selectlanguage{english}to clear up a little} \textcolor{PineGreen}{\selectlanguage{french}ranger un peu}  
 ¶ \textcolor{darkblue}{\textbf{\ipa{le˧-gv̩˧\textasciitilde{}gv̩˥ | tʰi˧-tɕɯ˥}}} \zh{收拾,摆好} \textcolor{Sepia}{\selectlanguage{english}to tidy up and put (everything) into place} \textcolor{PineGreen}{\selectlanguage{french}ranger et bien mettre à sa place}  

\lhead{\firstmark}
\rhead{\botmark}

\subsection{\hspace{-0.5cm} {\Large \textcolor{darkblue}{\textbf{\ipa{gv̩˩\textsubscript{a}}}} \textsubscript{3}}\hspace{0.5cm}[\kern2pt{\textcolor{darkblue}{\textbf{\ipa{gv̩˩˥}}}}\kern2pt]} \hypertarget{gv\string_=\string_Ba3}{}
\markboth{\textcolor{darkblue}{\textbf{\ipa{gv̩˩\textsubscript{a}}}} \textsubscript{3}}{}
\textcolor{teal}{\zh{动词}} \hspace{4pt} \zh{声调类:} L\textsubscript{a}.
\zh{落下(太阳落山)。} \textcolor{Sepia}{\selectlanguage{english}To set (the sun sets), to decline.} \textcolor{PineGreen}{\selectlanguage{french}Se coucher (le soleil se couche), décliner.}  ¶ \textcolor{darkblue}{\textbf{\ipa{ɲi˧mi˧ gv̩˩-se˩}}} \zh{在太阳落山之后,在太阳落山了以后} \textcolor{Sepia}{\selectlanguage{english}after the sun has set, after sunset} \textcolor{PineGreen}{\selectlanguage{french}à la nuit tombée, une fois la nuit tombée, après le coucher du soleil}  
 ¶ \textcolor{darkblue}{\textbf{\ipa{ɲi˧mi˧ | le˧-gv̩˩-ze˩.}}} \zh{太阳落山了。} \textcolor{Sepia}{\selectlanguage{english}The sun has set.} \textcolor{PineGreen}{\selectlanguage{french}Le soleil s'est couché.}  
 ¶ \textcolor{darkblue}{\textbf{\ipa{ɲi˧mi˧ | mɤ˧-gv̩˩-sɯ˩.}}} \zh{太阳还没有落。} \textcolor{Sepia}{\selectlanguage{english}The sun has not set yet.} \textcolor{PineGreen}{\selectlanguage{french}Le soleil ne s'est pas encore couché.}  

\lhead{\firstmark}
\rhead{\botmark}

\subsection{\hspace{-0.5cm} {\Large \textcolor{darkblue}{\textbf{\ipa{gv̩˧dv̩˧}}}}\hspace{0.5cm}[\kern2pt{\textcolor{darkblue}{\textbf{\ipa{gv̩˩dv̩˩˥}}}}\kern2pt]} \hypertarget{gv\string_=\string_Mdv\string_=\string_M1}{}
\markboth{\textcolor{darkblue}{\textbf{\ipa{gv̩˧dv̩˧}}}}{}
\textcolor{teal}{\zh{名词}} \hspace{4pt} \zh{声调类:} M.
\zh{脊背。} \textcolor{Sepia}{\selectlanguage{english}Back.} \textcolor{PineGreen}{\selectlanguage{french}Dos.}  \zh{量词}: \textcolor{darkblue}{\textbf{\ipa{ʈv̩˩}}} 
\lhead{\firstmark}
\rhead{\botmark}

\subsection{\hspace{-0.5cm} {\Large \textcolor{darkblue}{\textbf{\ipa{gv̩˧dv̩˧-gv̩˧mi˧}}}}\hspace{0.5cm}[\kern2pt{\textcolor{darkblue}{\textbf{\ipa{xxxx non-correspondance entre le nombre de morphèmes et le nombre de tons de morphèmes}}}}\kern2pt]} \hypertarget{gv\string_=\string_Mdv\string_=\string_M-gv\string_=\string_Mmi\string_M1}{}
\markboth{\textcolor{darkblue}{\textbf{\ipa{gv̩˧dv̩˧-gv̩˧mi˧}}}}{}
\textcolor{teal}{\zh{名词}} \hspace{4pt} \zh{声调类:} M.
\zh{身体。} \textcolor{Sepia}{\selectlanguage{english}Body.} \textcolor{PineGreen}{\selectlanguage{french}Corps.}  \zh{量词}: \textcolor{darkblue}{\textbf{\ipa{ɭɯ˧}}} 
\lhead{\firstmark}
\rhead{\botmark}

\subsection{\hspace{-0.5cm} {\Large \textcolor{darkblue}{\textbf{\ipa{gv̩˩dʑɯ˩}}}}\hspace{0.5cm}[\kern2pt{\textcolor{darkblue}{\textbf{\ipa{gv̩˧dʑɯ˧}}}}\kern2pt]} \hypertarget{gv\string_=\string_Bdz£M\string_B1}{}
\markboth{\textcolor{darkblue}{\textbf{\ipa{gv̩˩dʑɯ˩}}}}{}
\textcolor{teal}{\zh{形容词}} \hspace{4pt} \zh{声调类:} L.
\zh{生气。} \textcolor{Sepia}{\selectlanguage{english}Angry; afflicted.} \textcolor{PineGreen}{\selectlanguage{french}En colère, affligé.} 
\lhead{\firstmark}
\rhead{\botmark}

\subsection{\hspace{-0.5cm} {\Large \textcolor{darkblue}{\textbf{\ipa{gv̩˧kv̩˩}}}}\hspace{0.5cm}[\kern2pt{\textcolor{darkblue}{\textbf{\ipa{gv̩˩kv̩˩˥}}}}\kern2pt]} \hypertarget{gv\string_=\string_Mkv\string_=\string_B1}{}
\markboth{\textcolor{darkblue}{\textbf{\ipa{gv̩˧kv̩˩}}}}{}
\textcolor{teal}{\zh{名词}} \hspace{4pt} \zh{声调类:} L\#.
\zh{语调,声调。} \textcolor{Sepia}{\selectlanguage{english}Intonation, way of speaking; can be used, by extension, to refer to tones.} \textcolor{PineGreen}{\selectlanguage{french}Intonation, manière de s'exprimer; par extension, peut désigner les tons.}  ¶ \textcolor{darkblue}{\textbf{\ipa{gv̩˧kv̩˩-gv̩˩li˩ | ʐwɤ˩˥}}} \zh{说话说得好听、有口才、口若悬河、能言善辩} \textcolor{Sepia}{\selectlanguage{english}to speak with a pleasant style, to deliver one's speech with elegance} \textcolor{PineGreen}{\selectlanguage{french}parler avec une élocution soignée/agréable}  

\lhead{\firstmark}
\rhead{\botmark}

\subsection{\hspace{-0.5cm} {\Large \textcolor{darkblue}{\textbf{\ipa{gv̩˩ɬi˩mi˩}}}}\hspace{0.5cm}[\kern2pt{\textcolor{darkblue}{\textbf{\ipa{gv̩˧ɬi˧mi˩}}}}\kern2pt]} \hypertarget{gv\string_=\string_BKi\string_Bmi\string_B1}{}
\markboth{\textcolor{darkblue}{\textbf{\ipa{gv̩˩ɬi˩mi˩}}}}{}
\textcolor{teal}{\zh{名词}} \hspace{4pt} \zh{声调类:} L.
\zh{九月。} \textcolor{Sepia}{\selectlanguage{english}9th month.} \textcolor{PineGreen}{\selectlanguage{french}9e mois.} 
\lhead{\firstmark}
\rhead{\botmark}

\subsection{\hspace{-0.5cm} {\Large \textcolor{darkblue}{\textbf{\ipa{gv̩˧mɑ˧}}}}\hspace{0.5cm}[\kern2pt{\textcolor{darkblue}{\textbf{\ipa{gv̩˩mɑ˩˥}}}}\kern2pt]} \hypertarget{gv\string_=\string_MmA\string_M1}{}
\markboth{\textcolor{darkblue}{\textbf{\ipa{gv̩˧mɑ˧}}}}{}
\textcolor{teal}{\zh{名词}} \hspace{4pt} \zh{声调类:} M.
\zh{男性名字。} \textcolor{Sepia}{\selectlanguage{english}Masculine given name.} \textcolor{PineGreen}{\selectlanguage{french}Prénom masculin.}  ¶ \textcolor{darkblue}{\textbf{\ipa{hĩ˧ | ʈʂʰɯ˧-v̩˧, | gv̩˧mɑ˧ mv̩˧ʈʂæ˧˥!}}} \zh{这个人,名叫\textcolor{darkblue}{\textbf{\ipa{/gv̩˧mɑ˧/}}}!} \textcolor{Sepia}{\selectlanguage{english}This person is called \textcolor{darkblue}{\textbf{\ipa{/gv̩˧mɑ˧/!}}}} \textcolor{PineGreen}{\selectlanguage{french}Cette personne s'appelle \textcolor{darkblue}{\textbf{\ipa{/gv̩˧mɑ˧/}}} !}  

\lhead{\firstmark}
\rhead{\botmark}

\subsection{\hspace{-0.5cm} {\Large \textcolor{darkblue}{\textbf{\ipa{gv̩˧mi˧}}}}\hspace{0.5cm}[\kern2pt{\textcolor{darkblue}{\textbf{\ipa{gv̩˧mi˧}}}}\kern2pt]} \hypertarget{gv\string_=\string_Mmi\string_M1}{}
\markboth{\textcolor{darkblue}{\textbf{\ipa{gv̩˧mi˧}}}}{}
\textcolor{teal}{\zh{名词}} \hspace{4pt} \zh{声调类:} M.
\zh{身体。} \textcolor{Sepia}{\selectlanguage{english}Body.} \textcolor{PineGreen}{\selectlanguage{french}Corps.}  \zh{量词}: \textcolor{darkblue}{\textbf{\ipa{ɭɯ˧}}} 
\lhead{\firstmark}
\rhead{\botmark}

\subsection{\hspace{-0.5cm} {\Large \textcolor{darkblue}{\textbf{\ipa{gv̩˩pʰæ˩}}}}\hspace{0.5cm}[\kern2pt{\textcolor{darkblue}{\textbf{\ipa{gv̩˧pʰæ˧}}}}\kern2pt]} \hypertarget{gv\string_=\string_Bp\string_h\{\string_B1}{}
\markboth{\textcolor{darkblue}{\textbf{\ipa{gv̩˩pʰæ˩}}}}{}
\textcolor{teal}{\zh{名词}} \hspace{4pt} \zh{声调类:} L.
\zh{相当薄的木板。} \textcolor{Sepia}{\selectlanguage{english}Thin plank.} \textcolor{PineGreen}{\selectlanguage{french}Planche de bois fine: trois ou quatre centimètres.}  \zh{量词}: \textcolor{darkblue}{\textbf{\ipa{pʰæ˧˥}}} 
\lhead{\firstmark}
\rhead{\botmark}

\subsection{\hspace{-0.5cm} {\Large \textcolor{darkblue}{\textbf{\ipa{gv̩˧sɯ˥-pv̩˩}}}}\hspace{0.5cm}[\kern2pt{\textcolor{darkblue}{\textbf{\ipa{xxxx non-correspondance entre le nombre de morphèmes et le nombre de tons de morphèmes}}}}\kern2pt]} \hypertarget{gv\string_=\string_MsM\string_T-pv\string_=\string_B1}{}
\markboth{\textcolor{darkblue}{\textbf{\ipa{gv̩˧sɯ˥-pv̩˩}}}}{}
\textcolor{teal}{\zh{名词}} \hspace{4pt} \zh{声调类:} H\#-L.
\zh{肩胛骨。} \textcolor{Sepia}{\selectlanguage{english}Shoulderblade, scapula.} \textcolor{PineGreen}{\selectlanguage{french}Omoplate.}  \zh{量词}: \textcolor{darkblue}{\textbf{\ipa{kʰwɤ˥}}} 
\lhead{\firstmark}
\rhead{\botmark}

\subsection{\hspace{-0.5cm} {\Large \textcolor{darkblue}{\textbf{\ipa{gv̩˧tɕʰɯ˧˥}}}}\hspace{0.5cm}[\kern2pt{\textcolor{darkblue}{\textbf{\ipa{xxxx non-correspondance entre le nombre de morphèmes et le nombre de tons de morphèmes}}}}\kern2pt]} \hypertarget{gv\string_=\string_Mts£\string_hM\string_M\string_T1}{}
\markboth{\textcolor{darkblue}{\textbf{\ipa{gv̩˧tɕʰɯ˧˥}}}}{}
\textcolor{teal}{\zh{动词}} \hspace{4pt} \zh{声调类:} MH\#.
\zh{着凉。} \textcolor{Sepia}{\selectlanguage{english}To catch a cold.} \textcolor{PineGreen}{\selectlanguage{french}Prendre froid, attraper un rhume, attraper froid.} 
\lhead{\firstmark}
\rhead{\botmark}

\subsection{\hspace{-0.5cm} {\Large \textcolor{darkblue}{\textbf{\ipa{gv̩˧tsʰi˩}}}}\hspace{0.5cm}[\kern2pt{\textcolor{darkblue}{\textbf{\ipa{gv̩˧tsʰi˧˥}}}}\kern2pt]} \hypertarget{gv\string_=\string_Mts\string_hi\string_B1}{}
\markboth{\textcolor{darkblue}{\textbf{\ipa{gv̩˧tsʰi˩}}}}{}
\textcolor{teal}{\zh{数词}} \hspace{4pt} \zh{声调类:} L\#.
\zh{90。} \textcolor{Sepia}{\selectlanguage{english}90.} \textcolor{PineGreen}{\selectlanguage{french}90.} 
\lhead{\firstmark}
\rhead{\botmark}

\subsection{\hspace{-0.5cm} {\Large \textcolor{darkblue}{\textbf{\ipa{gwɤ˩\textsubscript{a}}}}}\hspace{0.5cm}[\kern2pt{\textcolor{darkblue}{\textbf{\ipa{gwɤ˩˥}}}}\kern2pt]} \hypertarget{gw7\string_Ba1}{}
\markboth{\textcolor{darkblue}{\textbf{\ipa{gwɤ˩\textsubscript{a}}}}}{}
\textcolor{teal}{\zh{动词}} \hspace{4pt} \zh{声调类:} L\textsubscript{a}.
\zh{唱、唱歌。} \textcolor{Sepia}{\selectlanguage{english}To sing.} \textcolor{PineGreen}{\selectlanguage{french}Chanter.}  ¶ \textcolor{darkblue}{\textbf{\ipa{njɤ˧ | ɖɯ˧-ɖʐo˩ | gwɤ˩-ze˥!}}} \zh{我唱了一首歌!} \textcolor{Sepia}{\selectlanguage{english}I have sung a song!} \textcolor{PineGreen}{\selectlanguage{french}j'ai chanté une chanson!}  
 ¶ \textcolor{darkblue}{\textbf{\ipa{no˧ | ɖɯ˧-ɖʐo˩ gwɤ˩!}}} \zh{你唱一首吧!} \textcolor{Sepia}{\selectlanguage{english}Please sing a song! / Go ahead and sing us a song!} \textcolor{PineGreen}{\selectlanguage{french}chante-nous une chanson!}  
 ¶ \textcolor{darkblue}{\textbf{\ipa{ɖɯ˧-kʰwɤ˧ gwɤ˥}}} \zh{唱一下} \textcolor{Sepia}{\selectlanguage{english}to sing a song} \textcolor{PineGreen}{\selectlanguage{french}chanter une chanson}  
 ¶ \textcolor{darkblue}{\textbf{\ipa{ɖɯ˧-kʰwɤ˧ gwɤ˥-ɻ̍˩}}} \zh{唱一下} \textcolor{Sepia}{\selectlanguage{english}to sing a song} \textcolor{PineGreen}{\selectlanguage{french}chanter une chanson}  
 ¶ \textcolor{darkblue}{\textbf{\ipa{nɑ˩-gwɤ˥}}} \zh{摩梭民歌} \textcolor{Sepia}{\selectlanguage{english}Na songs} \textcolor{PineGreen}{\selectlanguage{french}les chansons des Na}  
 ¶ \textcolor{darkblue}{\textbf{\ipa{ʈʂʰɯ˧ | nɑ˩-gwɤ˥ F | kv̩˧˥! | hæ˧-gwɤ˩ F | kv̩˧˥! | ʁo˧dzi˩-gwɤ˩ F | kv̩˧-ʝi˥! |}}} \zh{他会唱很多种风格的歌曲:摩梭的,会唱!汉族的,会唱!藏族的,会唱!} \textcolor{Sepia}{\selectlanguage{english}He can sing (lots of different styles:) Na songs! and also Chinese (Han) songs! and also Tibetan songs!} \textcolor{PineGreen}{\selectlanguage{french}Il sait chanter (toutes sortes de styles:) les chansons na! les chansons chinoises! les chansons tibétaines!}  

\lhead{\firstmark}
\rhead{\botmark}

\subsection{\hspace{-0.5cm} {\Large \textcolor{darkblue}{\textbf{\ipa{gwɤ˩\textasciitilde{}gwɤ˧˥}}}}\hspace{0.5cm}[\kern2pt{\textcolor{darkblue}{\textbf{\ipa{gwɤ˩gwɤ˥}}}}\kern2pt]} \hypertarget{gw7\string_B~gw7\string_M\string_T1}{}
\markboth{\textcolor{darkblue}{\textbf{\ipa{gwɤ˩\textasciitilde{}gwɤ˧˥}}}}{}
\textcolor{teal}{\zh{动词}} \hspace{4pt} \zh{声调类:} L.
\zh{逛,玩,游。} \textcolor{Sepia}{\selectlanguage{english}To stroll, to ramble, to roam.} \textcolor{PineGreen}{\selectlanguage{french}Se promener, se divertir, faire du tourisme.}  ¶ \textcolor{darkblue}{\textbf{\ipa{le˧-gwɤ˩\textasciitilde{}gwɤ˩ | le˧-tsʰɯ˩-ze˩!}}} \zh{你已经散步回来了!} \textcolor{Sepia}{\selectlanguage{english}So you are back from a stroll! / You are back from your little walk, eh?} \textcolor{PineGreen}{\selectlanguage{french}(tu) reviens de promenade!/ Alors comme ça te voilà revenu de ta promenade!}  
 ¶ \textcolor{darkblue}{\textbf{\ipa{ʈʂʰɯ˧ | gwɤ˩\textasciitilde{}gwɤ˩-hɯ˩-ze˥!}}} \zh{他散步去了!} \textcolor{Sepia}{\selectlanguage{english}(S)he has gone out for a walk!} \textcolor{PineGreen}{\selectlanguage{french}Il/elle est parti(e) se promener!}  
 ¶ \textcolor{darkblue}{\textbf{\ipa{æ˧ʂæ˧ gwɤ˩; | qv̩˧ɻ̍˧ gwɤ˥; | nɑ˩tsʰi˩ gwɤ˥}}} \zh{绕\textcolor{darkblue}{\textbf{\ipa{æ˧ʂæ˧}}}山,绕\textcolor{darkblue}{\textbf{\ipa{qv̩˧ɻ\#˥}}}山,绕\textcolor{darkblue}{\textbf{\ipa{nɑ˩tsʰi˩}}}山(做“绕山”仪式,为了求生子等)} \textcolor{Sepia}{\selectlanguage{english}“to walk around Mount \textcolor{darkblue}{\textbf{\ipa{/æ˧ʂæ˧/;}}} to walk around Mount \textcolor{darkblue}{\textbf{\ipa{/qv̩˧ɻ\#˥/;}}} to walk around Mount \textcolor{darkblue}{\textbf{\ipa{/nɑ˩tsʰi˩/”:}}} i.e. to do rituals on these mountains, in particular to obtain fertility, or to obtain a cure for a child who did not learn to speak.} \textcolor{PineGreen}{\selectlanguage{french}“faire le tour de la montagne \textcolor{darkblue}{\textbf{\ipa{/æ˧ʂæ˧/;}}} faire le tour de la montagne \textcolor{darkblue}{\textbf{\ipa{/qv̩˧ɻ\#˥/;}}} faire le tour de la montagne \textcolor{darkblue}{\textbf{\ipa{/nɑ˩tsʰi˩/”:}}} façon de désigner les rites qu'on pratiquait sur ces montagnes: pour des enfants qui tardaient à apprendre à parler, etc.}  

\lhead{\firstmark}
\rhead{\botmark}

\subsection{\hspace{-0.5cm} {\Large \textcolor{darkblue}{\textbf{\ipa{gwɤ˩ʝi˧}}}}\hspace{0.5cm}[\kern2pt{\textcolor{darkblue}{\textbf{\ipa{gwɤ˩ʝi˩˥}}}}\kern2pt]} \hypertarget{gw7\string_Bj££i\string_M1}{}
\markboth{\textcolor{darkblue}{\textbf{\ipa{gwɤ˩ʝi˧}}}}{}
\textcolor{teal}{\zh{助词}} \hspace{4pt} \zh{声调类:} LM.
\zh{整齐。} \textcolor{Sepia}{\selectlanguage{english}In good order.} \textcolor{PineGreen}{\selectlanguage{french}En ordre/rangé.}  ¶ \textcolor{darkblue}{\textbf{\ipa{tso˧\textasciitilde{}tso˧ | gwɤ˩ʝi˧ tʰi˧-tɕɯ˥ |}}} \zh{把东西摆整齐} \textcolor{Sepia}{\selectlanguage{english}to put things in good order} \textcolor{PineGreen}{\selectlanguage{french}mettre des choses en ordre, ranger des choses}  

\lhead{\firstmark}
\rhead{\botmark}

\newpage
\section*{\centering- \textcolor{darkblue}{\textbf{\ipa{ɣ}}} -}
\subsection{\hspace{-0.5cm} {\Large \textcolor{darkblue}{\textbf{\ipa{ɣɯ˥}}}}\hspace{0.5cm}[\kern2pt{\textcolor{darkblue}{\textbf{\ipa{ɣɯ˥}}}}\kern2pt]} \hypertarget{GM\string_T1}{}
\markboth{\textcolor{darkblue}{\textbf{\ipa{ɣɯ˥}}}}{}
\textcolor{teal}{\zh{形容词}} \hspace{4pt} \zh{声调类:} H.
\zh{能干、好(做事情做得好)。} \textcolor{Sepia}{\selectlanguage{english}Competent, able.} \textcolor{PineGreen}{\selectlanguage{french}Habile, compétent, bon.}  ¶ \textcolor{darkblue}{\textbf{\ipa{mɤ˧-ɣɯ˥}}} \zh{不能干} \textcolor{Sepia}{\selectlanguage{english}\mytextsc{neg}} \textcolor{PineGreen}{\selectlanguage{french}\mytextsc{neg}}  
 ¶ \textcolor{darkblue}{\textbf{\ipa{ʈʂʰɯ˧-ɳɯ˧, | bɑ˩lɑ˩ hwæ˧ | ɣɯ˧!}}} \zh{他很会买衣服!} \textcolor{Sepia}{\selectlanguage{english}He/she is very good at buying clothes! / He/she has talent for choosing clothes!} \textcolor{PineGreen}{\selectlanguage{french}Il/elle s'entend à acheter des vêtements! / Il/elle a du talent pour acheter des vêtements!}  

\lhead{\firstmark}
\rhead{\botmark}

\subsection{\hspace{-0.5cm} {\Large \textcolor{darkblue}{\textbf{\ipa{ɣɯ˧}}}}\hspace{0.5cm}[\kern2pt{\textcolor{darkblue}{\textbf{\ipa{ɣɯ˥}}}}\kern2pt]} \hypertarget{GM\string_M1}{}
\markboth{\textcolor{darkblue}{\textbf{\ipa{ɣɯ˧}}}}{}
\textcolor{teal}{\zh{名词}} \hspace{4pt} \zh{声调类:} \#H.
\zh{布料。} \textcolor{Sepia}{\selectlanguage{english}Cloth.} \textcolor{PineGreen}{\selectlanguage{french}Tissu.}  ¶ \textcolor{darkblue}{\textbf{\ipa{ɣɯ˧dzo˩, | ɣɯ˧ni˧˥, | ɣɯ˧, | ɖɯ˧-ʑi˩ ɲi˩-ze˩!}}} \zh{织布机、竹子的框(让线不乱混)、布料,属于同一类!(直译:“都是一家的!”)} \textcolor{Sepia}{\selectlanguage{english}The weaving-machine, the bamboo structure keeping the threads together, and fabric: these belong to the same family! / these are all part of the same sphere!} \textcolor{PineGreen}{\selectlanguage{french}Le métier à tisser, la structure en bambou qui maintient les fils, le tissu, c'est de la même famille! / ça forme une famille! (Commentaire sémantique de la locutrice, au cours d'une séance où il était question de tissus et de tissage)}  
 \zh{量词}: \textcolor{darkblue}{\textbf{\ipa{bo˩}}} 
\lhead{\firstmark}
\rhead{\botmark}

\subsection{\hspace{-0.5cm} {\Large \textcolor{darkblue}{\textbf{\ipa{ɣɯ˧bo˩}}}}\hspace{0.5cm}[\kern2pt{\textcolor{darkblue}{\textbf{\ipa{ɣɯ˧bo˧˥}}}}\kern2pt]} \hypertarget{GM\string_Mbo\string_B1}{}
\markboth{\textcolor{darkblue}{\textbf{\ipa{ɣɯ˧bo˩}}}}{}
\textcolor{teal}{\zh{名词}} \hspace{4pt} \zh{声调类:} MH\#.
\zh{纬线、纬纱。} \textcolor{Sepia}{\selectlanguage{english}Weft, weft thread, pick.} \textcolor{PineGreen}{\selectlanguage{french}Trame (lorsqu'on tisse, il y a du fil de trame, et du fil de chaîne); la trame désigne l'ensemble des fils de trame.}  \zh{量词}: \textcolor{darkblue}{\textbf{\ipa{bo˩}}} 
\lhead{\firstmark}
\rhead{\botmark}

\subsection{\hspace{-0.5cm} {\Large \textcolor{darkblue}{\textbf{\ipa{ɣɯ˧dzo˩}}}}\hspace{0.5cm}[\kern2pt{\textcolor{darkblue}{\textbf{\ipa{ɣɯ˧dzo˩}}}}\kern2pt]} \hypertarget{GM\string_Mdzo\string_B1}{}
\markboth{\textcolor{darkblue}{\textbf{\ipa{ɣɯ˧dzo˩}}}}{}
\textcolor{teal}{\zh{名词}} \hspace{4pt} \zh{声调类:} L\#.
\zh{织布机。} \textcolor{Sepia}{\selectlanguage{english}Loom.} \textcolor{PineGreen}{\selectlanguage{french}Métier à tisser, appareil à tisser.}  \zh{量词}: \textcolor{darkblue}{\textbf{\ipa{nɑ˧}}} 
\lhead{\firstmark}
\rhead{\botmark}

\subsection{\hspace{-0.5cm} {\Large \textcolor{darkblue}{\textbf{\ipa{ɣɯ˧ni˧˥}}}}\hspace{0.5cm}[\kern2pt{\textcolor{darkblue}{\textbf{\ipa{ɣɯ˧ni˧˥}}}}\kern2pt]} \hypertarget{GM\string_Mni\string_M\string_T1}{}
\markboth{\textcolor{darkblue}{\textbf{\ipa{ɣɯ˧ni˧˥}}}}{}
\textcolor{teal}{\zh{名词}} \hspace{4pt} \zh{声调类:} MH\#.
\zh{织布机的一部分:竹子的框,让线不乱混。} \textcolor{Sepia}{\selectlanguage{english}A part of the loom: a small bamboo structure hanging from the top of the loom, keeping the threads together.} \textcolor{PineGreen}{\selectlanguage{french}Petite structure en bambou maintenant les fils du métier à tisser; ses fils (blancs) sont verticaux, et passent à travers la trame.}  \zh{量词}: \textcolor{darkblue}{\textbf{\ipa{dze˩}}} 
\lhead{\firstmark}
\rhead{\botmark}

\subsection{\hspace{-0.5cm} {\Large \textcolor{darkblue}{\textbf{\ipa{ɣɯ˩kɯ˧˥}}}}\hspace{0.5cm}[\kern2pt{\textcolor{darkblue}{\textbf{\ipa{ɣɯ˩kɯ˧˥}}}}\kern2pt]} \hypertarget{GM\string_BkM\string_M\string_T1}{}
\markboth{\textcolor{darkblue}{\textbf{\ipa{ɣɯ˩kɯ˧˥}}}}{}
\textcolor{teal}{\zh{名词}} \hspace{4pt} \zh{声调类:} LM+MH\#.
\ding{202} \zh{皮、鸡蛋壳、麦麸。} \textcolor{Sepia}{\selectlanguage{english}Peel, rind.} \textcolor{PineGreen}{\selectlanguage{french}Pelure de fruit ou de légume.}  ¶ \textcolor{darkblue}{\textbf{\ipa{pʰi˩ko˧-ɣɯ˩kɯ˩}}} \zh{苹果皮} \textcolor{Sepia}{\selectlanguage{english}peel of an apple} \textcolor{PineGreen}{\selectlanguage{french}pelure de pomme}  
 ¶ \textcolor{darkblue}{\textbf{\ipa{jɤ˩jo˧-ɣɯ˥kɯ˩}}} \zh{洋芋皮} \textcolor{Sepia}{\selectlanguage{english}potato peel} \textcolor{PineGreen}{\selectlanguage{french}pelure de pomme de terre}  
 \zh{量词}: \textcolor{darkblue}{\textbf{\ipa{kʰwɤ˥}}} \ding{203} \zh{皮。} \textcolor{Sepia}{\selectlanguage{english}Fur, pelt, skin (of animal).} \textcolor{PineGreen}{\selectlanguage{french}Fourrure, peau d'animal.}  ¶ \textcolor{darkblue}{\textbf{\ipa{ʂe˧-ɣɯ˥kɯ˩}}} \zh{肉皮:鸡皮、猪肉的皮……} \textcolor{Sepia}{\selectlanguage{english}skin of meat, i.e. skin on a piece of meat} \textcolor{PineGreen}{\selectlanguage{french}peau de la viande (peau de poulet, couenne de porc...)}  
\ding{204} \zh{蛋壳。} \textcolor{Sepia}{\selectlanguage{english}Eggshell.} \textcolor{PineGreen}{\selectlanguage{french}Coquille (d'oeuf).} \ding{205} \zh{麸。} \textcolor{Sepia}{\selectlanguage{english}Bran.} \textcolor{PineGreen}{\selectlanguage{french}Son (de céréale).}  ¶ \textcolor{darkblue}{\textbf{\ipa{dze˧ɭɯ˧-ɣɯ˩kɯ˩}}} \zh{小麦麸} \textcolor{Sepia}{\selectlanguage{english}wheat bran} \textcolor{PineGreen}{\selectlanguage{french}son de blé}  

\lhead{\firstmark}
\rhead{\botmark}

\subsection{\hspace{-0.5cm} {\Large \textcolor{darkblue}{\textbf{\ipa{ɣɯ˩-nɑ˥mi˩}}}}\hspace{0.5cm}[\kern2pt{\textcolor{darkblue}{\textbf{\ipa{ɣɯ˩˥nɑ˧mi˧}}}}\kern2pt]} \hypertarget{GM\string_B-nA\string_Tmi\string_B1}{}
\markboth{\textcolor{darkblue}{\textbf{\ipa{ɣɯ˩-nɑ˥mi˩}}}}{}
\textcolor{teal}{\zh{名词}} \hspace{4pt} \zh{声调类:} LH-.
\zh{彝族(带偏见的说法)。} \textcolor{Sepia}{\selectlanguage{english}Yi (derogatory term).} \textcolor{PineGreen}{\selectlanguage{french}Terme péjoratif pour désigner les Yi (groupe ethnique): “les peaux-noires”.}  ¶ \textcolor{darkblue}{\textbf{\ipa{ɣɯ˩-nɑ˥mi˩-zo˩}}} \zh{彝族男人} \textcolor{Sepia}{\selectlanguage{english}Yi man} \textcolor{PineGreen}{\selectlanguage{french}homme yi}  
 ¶ \textcolor{darkblue}{\textbf{\ipa{ɣɯ˩-nɑ˥mi˩-mv̩˩}}} \zh{彝族女人} \textcolor{Sepia}{\selectlanguage{english}Yi woman} \textcolor{PineGreen}{\selectlanguage{french}femme yi}  
 \zh{量词}: \textcolor{darkblue}{\textbf{\ipa{v̩˧}}} 
\lhead{\firstmark}
\rhead{\botmark}

\subsection{\hspace{-0.5cm} {\Large \textcolor{darkblue}{\textbf{\ipa{ɣɯ˩˥}}}}\hspace{0.5cm}[\kern2pt{\textcolor{darkblue}{\textbf{\ipa{ɣɯ˩˥}}}}\kern2pt]} \hypertarget{GM\string_B\string_T1}{}
\markboth{\textcolor{darkblue}{\textbf{\ipa{ɣɯ˩˥}}}}{}
\textcolor{teal}{\zh{名词}} \hspace{4pt} \zh{声调类:} LH.
\zh{皮肤。} \textcolor{Sepia}{\selectlanguage{english}Skin.} \textcolor{PineGreen}{\selectlanguage{french}Peau.}  ¶ \textcolor{darkblue}{\textbf{\ipa{ɣɯ˩ dzɯ˩˥}}} \zh{吃皮} \textcolor{Sepia}{\selectlanguage{english}to eat skin} \textcolor{PineGreen}{\selectlanguage{french}manger de la peau}  
 ¶ \textcolor{darkblue}{\textbf{\ipa{ɣɯ˩˥ | ɖɯ˧-ʂɯ˩ pʰv˩}}} \zh{直译:‘脱皮一次’。意思:疲劳而受伤(因为做了很辛苦的工作,如:在深山老林砍树、扛树干回到坝子)} \textcolor{Sepia}{\selectlanguage{english}literally 'to shed one's skin once'; meaning: to be worn out and physically hurt (by an exhausting task, such as felling trees high up on the mountains and carrying lumber back to the plain)} \textcolor{PineGreen}{\selectlanguage{french}littéralement 'perdre sa peau'; sens: être épuisé et blessé par un travail ardu (par exemple: abattre des arbres sur la montagne et ramener les grumes jusque dans la plaine)}  
 \zh{量词}: \textcolor{darkblue}{\textbf{\ipa{tsʰi˥}}} 
\lhead{\firstmark}
\rhead{\botmark}

\newpage
\section*{\centering- \textcolor{darkblue}{\textbf{\ipa{h}}} -}
\subsection{\hspace{-0.5cm} {\Large \textcolor{darkblue}{\textbf{\ipa{hɑ˥}}}}\hspace{0.5cm}[\kern2pt{\textcolor{darkblue}{\textbf{\ipa{hɑ˧˥}}}}\kern2pt]} \hypertarget{hA\string_T1}{}
\markboth{\textcolor{darkblue}{\textbf{\ipa{hɑ˥}}}}{}
\textcolor{teal}{\zh{名词}} \hspace{4pt} \zh{声调类:} \#H.
\zh{饭,米饭。} \textcolor{Sepia}{\selectlanguage{english}Food.} \textcolor{PineGreen}{\selectlanguage{french}Nourriture.}  ¶ \textcolor{darkblue}{\textbf{\ipa{hɑ˧-ʈv̩˧\textasciitilde{}ʈv̩˥}}} \zh{饭坨坨、饭团} \textcolor{Sepia}{\selectlanguage{english}ball of cereals} \textcolor{PineGreen}{\selectlanguage{french}boule de céréale (riz ou autre)}  
 ¶ \textcolor{darkblue}{\textbf{\ipa{hɑ˧ dzɯ˧}}} \zh{吃饭} \textcolor{Sepia}{\selectlanguage{english}to eat} \textcolor{PineGreen}{\selectlanguage{french}manger}  
 ¶ \textcolor{darkblue}{\textbf{\ipa{ʈʂʰɯ˧ | hɑ˧ dzɯ˧-dʑo˩!}}} \zh{他在吃饭!} \textcolor{Sepia}{\selectlanguage{english}(S)he is eating!} \textcolor{PineGreen}{\selectlanguage{french}Il/elle est en train de manger!}  
 ¶ \textcolor{darkblue}{\textbf{\ipa{hɑ˧ʂɯ˩}}} \zh{新鲜的粮食(可以用来烤很香的饼)} \textcolor{Sepia}{\selectlanguage{english}fresh cereals (freshly reaped; they yield especially good-tasting cakes)} \textcolor{PineGreen}{\selectlanguage{french}céréales fraîches (juste après la cueillette; on en prépare des galettes particulièrement savoureuses)}  

\lhead{\firstmark}
\rhead{\botmark}

\subsection{\hspace{-0.5cm} {\Large \textcolor{darkblue}{\textbf{\ipa{hɑ˧bɤ˥}}}}\hspace{0.5cm}[\kern2pt{\textcolor{darkblue}{\textbf{\ipa{xxxx non-correspondance entre le nombre de morphèmes et le nombre de tons de morphèmes}}}}\kern2pt]} \hypertarget{hA\string_Mb7\string_T1}{}
\markboth{\textcolor{darkblue}{\textbf{\ipa{hɑ˧bɤ˥}}}}{}
\textcolor{teal}{\zh{名词}} \hspace{4pt} \zh{声调类:} H\#.
\zh{玉米棒子。} \textcolor{Sepia}{\selectlanguage{english}Corncob.} \textcolor{PineGreen}{\selectlanguage{french}Épi de maïs.}  ¶ \textcolor{darkblue}{\textbf{\ipa{qʰɑ˧dze˧-hɑ˧bɤ˥}}} \zh{玉米棒子} \textcolor{Sepia}{\selectlanguage{english}sweetcorn ear} \textcolor{PineGreen}{\selectlanguage{french}maïs en épi; épi de maïs}  
 \zh{量词}: \textcolor{darkblue}{\textbf{\ipa{bɤ˩}}} 
\lhead{\firstmark}
\rhead{\botmark}

\subsection{\hspace{-0.5cm} {\Large \textcolor{darkblue}{\textbf{\ipa{hɑ˧-bv̩˥\textasciitilde{}bv̩˩-di˩}}}}\hspace{0.5cm}[\kern2pt{\textcolor{darkblue}{\textbf{\ipa{xxxx non-correspondance entre le nombre de morphèmes et le nombre de tons de morphèmes}}}}\kern2pt]} \hypertarget{hA\string_M-bv\string_=\string_T~bv\string_=\string_B-di\string_B1}{}
\markboth{\textcolor{darkblue}{\textbf{\ipa{hɑ˧-bv̩˥\textasciitilde{}bv̩˩-di˩}}}}{}
\textcolor{teal}{\zh{名词}} \hspace{4pt} \zh{声调类:} \#H--.
\zh{甑。} \textcolor{Sepia}{\selectlanguage{english}Rice steamer.} \textcolor{PineGreen}{\selectlanguage{french}Étuve pour le riz.}  \zh{量词}: \textcolor{darkblue}{\textbf{\ipa{ɭɯ˧}}} 
\lhead{\firstmark}
\rhead{\botmark}

\subsection{\hspace{-0.5cm} {\Large \textcolor{darkblue}{\textbf{\ipa{hɑ˧-gv̩˥-di˩}}}}\hspace{0.5cm}[\kern2pt{\textcolor{darkblue}{\textbf{\ipa{xxxx non-correspondance entre le nombre de morphèmes et le nombre de tons de morphèmes}}}}\kern2pt]} \hypertarget{hA\string_M-gv\string_=\string_T-di\string_B1}{}
\markboth{\textcolor{darkblue}{\textbf{\ipa{hɑ˧-gv̩˥-di˩}}}}{}
\textcolor{teal}{\zh{名词}} \hspace{4pt} \zh{声调类:} H\#-.
\zh{炉子、灶头。} \textcolor{Sepia}{\selectlanguage{english}Stove.} \textcolor{PineGreen}{\selectlanguage{french}Fourneau.} 
\lhead{\firstmark}
\rhead{\botmark}

\subsection{\hspace{-0.5cm} {\Large \textcolor{darkblue}{\textbf{\ipa{hɑ˧ɭɯ\#˥}}}}\hspace{0.5cm}[\kern2pt{\textcolor{darkblue}{\textbf{\ipa{hɑ˧ɭɯ˧}}}}\kern2pt]} \hypertarget{hA\string_Ml\string_RM\#\string_T1}{}
\markboth{\textcolor{darkblue}{\textbf{\ipa{hɑ˧ɭɯ\#˥}}}}{}
\textcolor{teal}{\zh{名词}} \hspace{4pt} \zh{声调类:} \#H.
\zh{粮食。} \textcolor{Sepia}{\selectlanguage{english}Cereals.} \textcolor{PineGreen}{\selectlanguage{french}Céréales.} 
\lhead{\firstmark}
\rhead{\botmark}

\subsection{\hspace{-0.5cm} {\Large \textcolor{darkblue}{\textbf{\ipa{hɑ˧mi˥}}}}\hspace{0.5cm}[\kern2pt{\textcolor{darkblue}{\textbf{\ipa{hɑ˧mi˥}}}}\kern2pt]} \hypertarget{hA\string_Mmi\string_T1}{}
\markboth{\textcolor{darkblue}{\textbf{\ipa{hɑ˧mi˥}}}}{}
\textcolor{teal}{\zh{动词}} \hspace{4pt} \zh{声调类:} H\#.
\zh{讨饭。} \textcolor{Sepia}{\selectlanguage{english}To beg.} \textcolor{PineGreen}{\selectlanguage{french}Mendier.}  ¶ \textcolor{darkblue}{\textbf{\ipa{hɑ˧mi˥-hĩ˩}}} \zh{要饭的、乞丐} \textcolor{Sepia}{\selectlanguage{english}\string_ \mytextsc{rel}: beggar, [person] who begs} \textcolor{PineGreen}{\selectlanguage{french}\string_ \mytextsc{rel}: mendiant, [personne] qui mendie}  
\zh{~【参考】~} \hyperlink{}{\textcolor{darkblue}{\textbf{\ipa{mi˩\textsubscript{a}}}}} 
\lhead{\firstmark}
\rhead{\botmark}

\subsection{\hspace{-0.5cm} {\Large \textcolor{darkblue}{\textbf{\ipa{hɑ˧pv̩˩}}}}\hspace{0.5cm}[\kern2pt{\textcolor{darkblue}{\textbf{\ipa{hɑ˧pv̩˩}}}}\kern2pt]} \hypertarget{hA\string_Mpv\string_=\string_B1}{}
\markboth{\textcolor{darkblue}{\textbf{\ipa{hɑ˧pv̩˩}}}}{}
\textcolor{teal}{\zh{名词}} \hspace{4pt} \zh{声调类:} L\#.
\zh{干的米饭(与稀饭不同)。} \textcolor{Sepia}{\selectlanguage{english}'dry' cooked rice: the type of rice usually served at meals, as distinct from watery rice gruel.} \textcolor{PineGreen}{\selectlanguage{french}Riz cuit 'sec': le riz tel qu'il est servi aux repas, par opposition avec le gruau de riz.} 
\lhead{\firstmark}
\rhead{\botmark}

\subsection{\hspace{-0.5cm} {\Large \textcolor{darkblue}{\textbf{\ipa{hɑ˧ʂɯ˥}}}}\hspace{0.5cm}[\kern2pt{\textcolor{darkblue}{\textbf{\ipa{hɑ˧ʂɯ˥}}}}\kern2pt]} \hypertarget{hA\string_Ms`M\string_T1}{}
\markboth{\textcolor{darkblue}{\textbf{\ipa{hɑ˧ʂɯ˥}}}}{}
\textcolor{teal}{\zh{连词}} \hspace{4pt} \zh{声调类:} H\#.
\zh{还是(汉语借词)。} \textcolor{Sepia}{\selectlanguage{english}Gap-filler, borrowed from the Chinese: “still/also...”.} \textcolor{PineGreen}{\selectlanguage{french}Explétif, emprunté au chinois: “quand même/aussi...”.}  \zh{【借词】} \zh{还是}

\lhead{\firstmark}
\rhead{\botmark}

\subsection{\hspace{-0.5cm} {\Large \textcolor{darkblue}{\textbf{\ipa{hɑ˧-ʐwɤ˩}}}}\hspace{0.5cm}[\kern2pt{\textcolor{darkblue}{\textbf{\ipa{xxxx non-correspondance entre le nombre de morphèmes et le nombre de tons de morphèmes}}}}\kern2pt]} \hypertarget{hA\string_M-z`w7\string_B1}{}
\markboth{\textcolor{darkblue}{\textbf{\ipa{hɑ˧-ʐwɤ˩}}}}{}
\textcolor{teal}{\zh{形容词}} \hspace{4pt} \zh{声调类:} L\#.
\zh{饿(饭)。} \textcolor{Sepia}{\selectlanguage{english}Hungry.} \textcolor{PineGreen}{\selectlanguage{french}Avoir faim.} \zh{~【参考】~} \hyperlink{}{\textcolor{darkblue}{\textbf{\ipa{ʐwɤ˧}}}} 
\lhead{\firstmark}
\rhead{\botmark}

\subsection{\hspace{-0.5cm} {\Large \textcolor{darkblue}{\textbf{\ipa{hɑ˩\textsubscript{a}}}}}\hspace{0.5cm}[\kern2pt{\textcolor{darkblue}{\textbf{\ipa{hɑ˧˥}}}}\kern2pt]} \hypertarget{hA\string_Ba1}{}
\markboth{\textcolor{darkblue}{\textbf{\ipa{hɑ˩\textsubscript{a}}}}}{}
\textcolor{teal}{\zh{动词}} \hspace{4pt} \zh{声调类:} L\textsubscript{a}.
\zh{睁开(眼睛)。} \textcolor{Sepia}{\selectlanguage{english}To open (one's eyes).} \textcolor{PineGreen}{\selectlanguage{french}Ouvrir (les yeux); s'ouvrir (un sac).}  ¶ \textcolor{darkblue}{\textbf{\ipa{tʰi˧-hɑ˩}}} \zh{\mytextsc{dur}} \textcolor{Sepia}{\selectlanguage{english}\mytextsc{dur}} \textcolor{PineGreen}{\selectlanguage{french}\mytextsc{dur}}  
 ¶ \textcolor{darkblue}{\textbf{\ipa{njɤ˩ɭɯ˥ | gɤ˩-hɑ˥ |}}} \zh{睁开眼睛} \textcolor{Sepia}{\selectlanguage{english}to open one's eyes} \textcolor{PineGreen}{\selectlanguage{french}ouvrir les yeux}  
 ¶ \textcolor{darkblue}{\textbf{\ipa{njɤ˩ɭɯ˧ hɑ˩}}} \zh{睁开眼睛} \textcolor{Sepia}{\selectlanguage{english}to open one's eyes} \textcolor{PineGreen}{\selectlanguage{french}ouvrir les yeux}  

\lhead{\firstmark}
\rhead{\botmark}

\subsection{\hspace{-0.5cm} {\Large \textcolor{darkblue}{\textbf{\ipa{hɑ̃˧mo˥}}}}\hspace{0.5cm}[\kern2pt{\textcolor{darkblue}{\textbf{\ipa{hɑ̃˧mo˥}}}}\kern2pt]} \hypertarget{hA\string_~\string_Mmo\string_T1}{}
\markboth{\textcolor{darkblue}{\textbf{\ipa{hɑ̃˧mo˥}}}}{}
\textcolor{teal}{\zh{形容词}} \hspace{4pt} \zh{声调类:} H\#.
\zh{年老。} \textcolor{Sepia}{\selectlanguage{english}Old (person).} \textcolor{PineGreen}{\selectlanguage{french}Âgé, vieux (personne humaine).}  ¶ \textcolor{darkblue}{\textbf{\ipa{hĩ˧ ʈʂʰɯ˧-v̩˧ | hɑ̃˧mo˥ | ʐwæ˩˥!}}} \zh{这个人,年纪非常大!} \textcolor{Sepia}{\selectlanguage{english}This person is extremely old/extremely advanced in years!} \textcolor{PineGreen}{\selectlanguage{french}Cette personne est très âgée!}  

\lhead{\firstmark}
\rhead{\botmark}

\subsection{\hspace{-0.5cm} {\Large \textcolor{darkblue}{\textbf{\ipa{hɑ̃˧˥}}} \textsubscript{1}}\hspace{0.5cm}[\kern2pt{\textcolor{darkblue}{\textbf{\ipa{hɑ̃˧˥}}}}\kern2pt]} \hypertarget{hA\string_~\string_M\string_T1}{}
\markboth{\textcolor{darkblue}{\textbf{\ipa{hɑ̃˧˥}}} \textsubscript{1}}{}
\textcolor{teal}{\zh{动词}} \hspace{4pt} \zh{声调类:} MH.
\zh{把火炉灭了。} \textcolor{Sepia}{\selectlanguage{english}To put out, to extinguish (e.g. the fire of the stove).} \textcolor{PineGreen}{\selectlanguage{french}Éteindre (par exemple le feu du foyer).}  ¶ \textcolor{darkblue}{\textbf{\ipa{mv̩˧ le˧-hɑ̃˧˥}}} \zh{灭火} \textcolor{Sepia}{\selectlanguage{english}to put out the fire} \textcolor{PineGreen}{\selectlanguage{french}éteindre le feu}  

\lhead{\firstmark}
\rhead{\botmark}

\subsection{\hspace{-0.5cm} {\Large \textcolor{darkblue}{\textbf{\ipa{hɑ̃˧˥}}} \textsubscript{2}}\hspace{0.5cm}[\kern2pt{\textcolor{darkblue}{\textbf{\ipa{hɑ̃˧˥}}}}\kern2pt]} \hypertarget{hA\string_~\string_M\string_T2}{}
\markboth{\textcolor{darkblue}{\textbf{\ipa{hɑ̃˧˥}}} \textsubscript{2}}{}
\textcolor{teal}{\zh{动词}} \hspace{4pt} \zh{声调类:} MH.
\zh{过夜。} \textcolor{Sepia}{\selectlanguage{english}To spend the night (at a certain place).} \textcolor{PineGreen}{\selectlanguage{french}Passer la nuit.}  ¶ \textcolor{darkblue}{\textbf{\ipa{ɖɯ˧-hɑ̃˧ tʰi˥-hɑ̃˩ |}}} \zh{过夜} \textcolor{Sepia}{\selectlanguage{english}to spend a night (somewhere), to stay for the night} \textcolor{PineGreen}{\selectlanguage{french}se reposer une soirée, passer une soirée/nuitée (qq part)}  
 ¶ \textcolor{darkblue}{\textbf{\ipa{ʑi˧qʰwɤ˧ ɖɯ˧-ɭɯ˧-qo˧ hɑ̃˧˥}}} \zh{在一个人家过夜} \textcolor{Sepia}{\selectlanguage{english}to spend the night in a house} \textcolor{PineGreen}{\selectlanguage{french}passer la nuit dans une maison}  

\lhead{\firstmark}
\rhead{\botmark}

\subsection{\hspace{-0.5cm} {\Large \textcolor{darkblue}{\textbf{\ipa{hɑ̃˧˥\textsubscript{a}}}}}\hspace{0.5cm}[\kern2pt{\textcolor{darkblue}{\textbf{\ipa{hɑ̃˩˥}}}}\kern2pt]} \hypertarget{hA\string_~\string_M\string_Ta1}{}
\markboth{\textcolor{darkblue}{\textbf{\ipa{hɑ̃˧˥\textsubscript{a}}}}}{}
\textcolor{teal}{\zh{量词}} \hspace{4pt} \zh{声调类:} MH\textsubscript{a}.
\zh{量词:夜。} \textcolor{Sepia}{\selectlanguage{english}Classifier for nights.} \textcolor{PineGreen}{\selectlanguage{french}Nuit; par extension: utilisé pour le décompte des jours.}  ¶ \textcolor{darkblue}{\textbf{\ipa{ɖɯ˧-hɑ̃˧˥}}} \zh{一夜} \textcolor{Sepia}{\selectlanguage{english}one night} \textcolor{PineGreen}{\selectlanguage{french}une nuit}  
 ¶ \textcolor{darkblue}{\textbf{\ipa{tsʰe˩-hɑ̃˩˥}}} \zh{十夜(等于十天)} \textcolor{Sepia}{\selectlanguage{english}ten nights = ten days} \textcolor{PineGreen}{\selectlanguage{french}dix soirées=10 jours}  
 ¶ \textcolor{darkblue}{\textbf{\ipa{ɖɯ˧-hɑ̃˧ lɑ˥-dʑo˩!}}} \zh{只有一个晚上了!} \textcolor{Sepia}{\selectlanguage{english}There is only one day left!} \textcolor{PineGreen}{\selectlanguage{french}Il ne reste qu'une soirée!}  
 ¶ \textcolor{darkblue}{\textbf{\ipa{[F5] ɖɯ˧-hɑ̃˧-ɳɯ˥ | le˧-li˧-le˧-se˩-ze˩!}}} \zh{一个晚上就读完了!/一天之内都读完了!(情景:送一个人一本书,他马上全部读完)} \textcolor{Sepia}{\selectlanguage{english}He has entirely read it in one night! / He has read the whole (book) in just one night! (Imagined context: someone is given a book; he finishes reading it within a day)} \textcolor{PineGreen}{\selectlanguage{french}(Il) a tout lu en deux jours! (contexte imaginé: on offre un livre à quelqu'un; en deux jours il a tout lu)}  

\lhead{\firstmark}
\rhead{\botmark}

\subsection{\hspace{-0.5cm} {\Large \textcolor{darkblue}{\textbf{\ipa{hæ˧}}}}\hspace{0.5cm}[\kern2pt{\textcolor{darkblue}{\textbf{\ipa{hæ˧˥}}}}\kern2pt]} \hypertarget{h\{\string_M1}{}
\markboth{\textcolor{darkblue}{\textbf{\ipa{hæ˧}}}}{}
\textcolor{teal}{\zh{名词}} \hspace{4pt} \zh{声调类:} M.
\zh{汉人。} \textcolor{Sepia}{\selectlanguage{english}Chinese (Han).} \textcolor{PineGreen}{\selectlanguage{french}Chinois (Han).}  ¶ \textcolor{darkblue}{\textbf{\ipa{hæ˧-mi\#˥}}} \zh{汉族女人} \textcolor{Sepia}{\selectlanguage{english}a Chinese woman, a Han Chinese woman} \textcolor{PineGreen}{\selectlanguage{french}une femme chinoise, une Chinoise (Han)}  
 ¶ \textcolor{darkblue}{\textbf{\ipa{hæ˧-mv̩˧ hæ˧-di˧˥}}} \zh{汉族地区,包括成都、昆明等等} \textcolor{Sepia}{\selectlanguage{english}(Han) Chinese territory: Chengdu, Kunming...} \textcolor{PineGreen}{\selectlanguage{french}le territoire des Chinois (Han): Chengdu, Kunming...}  
 ¶ \textcolor{darkblue}{\textbf{\ipa{hæ˧-di˩}}} \zh{汉族地区,包括成都、昆明等等,来代指南方} \textcolor{Sepia}{\selectlanguage{english}(Han) Chinese territory: Chengdu, Kunming...; used to mean 'the south'} \textcolor{PineGreen}{\selectlanguage{french}le territoire des Chinois (Han): Chengdu, Kunming...; l'expression est employée pour désigner la direction du sud}  
 ¶ \textcolor{darkblue}{\textbf{\ipa{hæ˧-zo˧bæ˩}}} \zh{汉男人(带偏见的称呼)} \textcolor{Sepia}{\selectlanguage{english}Han Chinese man (derogatory: literally 'Chinese idiot')} \textcolor{PineGreen}{\selectlanguage{french}homme chinois han (terme péjoratif: littéralement 'idiot de Chinois')}  
 \zh{量词}: \textcolor{darkblue}{\textbf{\ipa{v̩˧}}} 
\lhead{\firstmark}
\rhead{\botmark}

\subsection{\hspace{-0.5cm} {\Large \textcolor{darkblue}{\textbf{\ipa{hæ˧di˩-ʈæ˩bɤ˩}}}}\hspace{0.5cm}[\kern2pt{\textcolor{darkblue}{\textbf{\ipa{xxxx non-correspondance entre le nombre de morphèmes et le nombre de tons de morphèmes}}}}\kern2pt]} \hypertarget{h\{\string_Mdi\string_B-t`\{\string_Bb7\string_B1}{}
\markboth{\textcolor{darkblue}{\textbf{\ipa{hæ˧di˩-ʈæ˩bɤ˩}}}}{}
\textcolor{teal}{\zh{名词}} \hspace{4pt} \zh{声调类:} \mytextsc{L}.
\zh{比丘、游僧。} \textcolor{Sepia}{\selectlanguage{english}Beggar-monk (of the Buddhist religion).} \textcolor{PineGreen}{\selectlanguage{french}Bhiksu, moine mendiant.} 
\lhead{\firstmark}
\rhead{\botmark}

\subsection{\hspace{-0.5cm} {\Large \textcolor{darkblue}{\textbf{\ipa{hæ˧ɭɯ\#˥}}}}\hspace{0.5cm}[\kern2pt{\textcolor{darkblue}{\textbf{\ipa{hæ˧ɭɯ˧}}}}\kern2pt]} \hypertarget{h\{\string_Ml\string_RM\#\string_T1}{}
\markboth{\textcolor{darkblue}{\textbf{\ipa{hæ˧ɭɯ\#˥}}}}{}
\textcolor{teal}{\zh{名词}} \hspace{4pt} \zh{声调类:} \#H.
\zh{高粱。} \textcolor{Sepia}{\selectlanguage{english}Chinese sorghum.} \textcolor{PineGreen}{\selectlanguage{french}Sorgho, gaoliang; céréale dont on se sert pour faire du vin.} \zh{~【参考】~} \hyperlink{}{\textcolor{darkblue}{\textbf{\ipa{kɤ˧ljɤ˩}}}} 
\lhead{\firstmark}
\rhead{\botmark}

\subsection{\hspace{-0.5cm} {\Large \textcolor{darkblue}{\textbf{\ipa{hæ˧se˧}}}}\hspace{0.5cm}[\kern2pt{\textcolor{darkblue}{\textbf{\ipa{hæ˧se˧}}}}\kern2pt]} \hypertarget{h\{\string_Mse\string_M1}{}
\markboth{\textcolor{darkblue}{\textbf{\ipa{hæ˧se˧}}}}{}
\textcolor{teal}{\zh{名词}} \hspace{4pt} \zh{声调类:} M.
\zh{石花菜、海参。} \textcolor{Sepia}{\selectlanguage{english}Agar.} \textcolor{PineGreen}{\selectlanguage{french}Agar (ressemble à une algue).}  \zh{【借词】} dialect \zh{海参}

\lhead{\firstmark}
\rhead{\botmark}

\subsection{\hspace{-0.5cm} {\Large \textcolor{darkblue}{\textbf{\ipa{hæ˧ʐwɤ˩}}}}\hspace{0.5cm}[\kern2pt{\textcolor{darkblue}{\textbf{\ipa{hæ˧ʐwɤ˩}}}}\kern2pt]} \hypertarget{h\{\string_Mz`w7\string_B1}{}
\markboth{\textcolor{darkblue}{\textbf{\ipa{hæ˧ʐwɤ˩}}}}{}
\textcolor{teal}{\zh{名词}} \hspace{4pt} \zh{声调类:} L\#.
\zh{汉语。} \textcolor{Sepia}{\selectlanguage{english}The Chinese language.} \textcolor{PineGreen}{\selectlanguage{french}La langue chinoise.} 
\lhead{\firstmark}
\rhead{\botmark}

\subsection{\hspace{-0.5cm} {\Large \textcolor{darkblue}{\textbf{\ipa{hæ˩\textsubscript{a}}}}}\hspace{0.5cm}[\kern2pt{\textcolor{darkblue}{\textbf{\ipa{hæ˥}}}}\kern2pt]} \hypertarget{h\{\string_Ba1}{}
\markboth{\textcolor{darkblue}{\textbf{\ipa{hæ˩\textsubscript{a}}}}}{}
\textcolor{teal}{\zh{动词}} \hspace{4pt} \zh{声调类:} L\textsubscript{a}.
\zh{祸害、害。} \textcolor{Sepia}{\selectlanguage{english}To harm, to cause trouble.} \textcolor{PineGreen}{\selectlanguage{french}Causer du tort à (note: ce mot n'est pas un emprunt, la ressemble avec le mandarin est une coïncidence).}  ¶ \textcolor{darkblue}{\textbf{\ipa{hĩ˧ hæ˥}}} \zh{害人} \textcolor{Sepia}{\selectlanguage{english}to harm people} \textcolor{PineGreen}{\selectlanguage{french}causer du tort aux gens}  
 ¶ \textcolor{darkblue}{\textbf{\ipa{hĩ˧ hæ˥-kv̩˩}}} \zh{会害人的、残忍、凶狠} \textcolor{Sepia}{\selectlanguage{english}who can harm people; cruel} \textcolor{PineGreen}{\selectlanguage{french}qui est susceptible de causer du tort/d'être cruel}  
 ¶ \textcolor{darkblue}{\textbf{\ipa{hĩ˧ hæ˥-zo˩}}} \zh{可怕的人} \textcolor{Sepia}{\selectlanguage{english}terrifying person, frightening person, creepy person} \textcolor{PineGreen}{\selectlanguage{french}homme terrifiant}  

\lhead{\firstmark}
\rhead{\botmark}

\subsection{\hspace{-0.5cm} {\Large \textcolor{darkblue}{\textbf{\ipa{hæ˧˥}}} \textsubscript{1}}\hspace{0.5cm}[\kern2pt{\textcolor{darkblue}{\textbf{\ipa{hæ˥}}}}\kern2pt]} \hypertarget{h\{\string_M\string_T1}{}
\markboth{\textcolor{darkblue}{\textbf{\ipa{hæ˧˥}}} \textsubscript{1}}{}
\textcolor{teal}{\zh{形容词}} \hspace{4pt} \zh{声调类:} MH.
\ding{202} \zh{软、柔软(树枝……)。} \textcolor{Sepia}{\selectlanguage{english}Supple, lithe.} \textcolor{PineGreen}{\selectlanguage{french}Souple, mou (branche…).}  ¶ \textcolor{darkblue}{\textbf{\ipa{hæ˧njæ˧˥ | -gv̩˩}}} \zh{软、柔软(树枝……)} \textcolor{Sepia}{\selectlanguage{english}soft, lithe, supple} \textcolor{PineGreen}{\selectlanguage{french}souple}  
\ding{203} \zh{稀(粥、汤)。} \textcolor{Sepia}{\selectlanguage{english}Thin, watery (soup, gruel).} \textcolor{PineGreen}{\selectlanguage{french}Léger, clair, délayé (gruau, soupe...).} 
\lhead{\firstmark}
\rhead{\botmark}

\subsection{\hspace{-0.5cm} {\Large \textcolor{darkblue}{\textbf{\ipa{hæ˧˥}}} \textsubscript{2}}\hspace{0.5cm}[\kern2pt{\textcolor{darkblue}{\textbf{\ipa{hæ˧˥}}}}\kern2pt]} \hypertarget{h\{\string_M\string_T2}{}
\markboth{\textcolor{darkblue}{\textbf{\ipa{hæ˧˥}}} \textsubscript{2}}{}
\textcolor{teal}{\zh{名词}} \hspace{4pt} \zh{声调类:} MH.
\zh{石灰。} \textcolor{Sepia}{\selectlanguage{english}Lime.} \textcolor{PineGreen}{\selectlanguage{french}Chaux.}  ¶ \textcolor{darkblue}{\textbf{\ipa{hæ˧ hwæ˥}}} \zh{买石灰} \textcolor{Sepia}{\selectlanguage{english}to buy lime} \textcolor{PineGreen}{\selectlanguage{french}acheter de la chaux}  
 ¶ \textcolor{darkblue}{\textbf{\ipa{hæ˧ tɕʰi˥}}} \zh{卖石灰} \textcolor{Sepia}{\selectlanguage{english}to sell lime} \textcolor{PineGreen}{\selectlanguage{french}vendre de la chaux}  
 ¶ \textcolor{darkblue}{\textbf{\ipa{hæ˧ ki˥}}} \zh{给石灰} \textcolor{Sepia}{\selectlanguage{english}to give lime} \textcolor{PineGreen}{\selectlanguage{french}donner de la chaux}  
 ¶ \textcolor{darkblue}{\textbf{\ipa{hæ˧ dv̩˥}}} \zh{挖石灰} \textcolor{Sepia}{\selectlanguage{english}to dig lime} \textcolor{PineGreen}{\selectlanguage{french}piocher de la chaux}  
 ¶ \textcolor{darkblue}{\textbf{\ipa{hæ˧ bæ˥}}} \zh{扫石灰} \textcolor{Sepia}{\selectlanguage{english}to sweep lime} \textcolor{PineGreen}{\selectlanguage{french}balayer de la chaux}  
 ¶ \textcolor{darkblue}{\textbf{\ipa{hæ˧ gɤ˥}}} \zh{扛石灰} \textcolor{Sepia}{\selectlanguage{english}to carry lime} \textcolor{PineGreen}{\selectlanguage{french}porter de la chaux}  

\lhead{\firstmark}
\rhead{\botmark}

\subsection{\hspace{-0.5cm} {\Large \textcolor{darkblue}{\textbf{\ipa{hæ̃˧}}}}\hspace{0.5cm}[\kern2pt{\textcolor{darkblue}{\textbf{\ipa{hæ̃˥}}}}\kern2pt]} \hypertarget{h\{\string_~\string_M1}{}
\markboth{\textcolor{darkblue}{\textbf{\ipa{hæ̃˧}}}}{}
\textcolor{teal}{\zh{名词}} \hspace{4pt} \zh{声调类:} M.
\zh{风。} \textcolor{Sepia}{\selectlanguage{english}Wind.} \textcolor{PineGreen}{\selectlanguage{french}Vent.}  ¶ \textcolor{darkblue}{\textbf{\ipa{hæ̃˧ tʰv̩˧ / hæ̃˧ tʰv̩˧-ze˧}}} \zh{刮风了} \textcolor{Sepia}{\selectlanguage{english}the wind has risen, the wind is blowing} \textcolor{PineGreen}{\selectlanguage{french}il y a du vent, le vent souffle}  
 ¶ \textcolor{darkblue}{\textbf{\ipa{wɤ˩˥ | hæ̃˧ tʰv̩˧-ho˩-ze˩!}}} \zh{风又要刮起来了!} \textcolor{Sepia}{\selectlanguage{english}There's going to be some wind again!} \textcolor{PineGreen}{\selectlanguage{french}Le vent va se lever à nouveau!/ On dirait que le vent va se remettre à souffler!}  
 \zh{量词}: \textcolor{darkblue}{\textbf{\ipa{kʰwɤ˥}}} 
\lhead{\firstmark}
\rhead{\botmark}

\subsection{\hspace{-0.5cm} {\Large \textcolor{darkblue}{\textbf{\ipa{hæ̃˧\textsubscript{a}}}}}\hspace{0.5cm}[\kern2pt{\textcolor{darkblue}{\textbf{\ipa{hæ̃˥}}}}\kern2pt]} \hypertarget{h\{\string_~\string_Ma1}{}
\markboth{\textcolor{darkblue}{\textbf{\ipa{hæ̃˧\textsubscript{a}}}}}{}
\textcolor{teal}{\zh{动词}} \hspace{4pt} \zh{声调类:} M\textsubscript{a}.
\zh{扬(粮食)。} \textcolor{Sepia}{\selectlanguage{english}To use the wind to winnow.} \textcolor{PineGreen}{\selectlanguage{french}Vanner: verser doucement dans une vannerie; la balle s'envole à mesure, emportée par le vent.}  ¶ \textcolor{darkblue}{\textbf{\ipa{hɑ˧ hæ̃˩}}} \zh{扬粮食} \textcolor{Sepia}{\selectlanguage{english}to winnow cereals} \textcolor{PineGreen}{\selectlanguage{french}vanner du grain}  
 ¶ \textcolor{darkblue}{\textbf{\ipa{tso˧\textasciitilde{}tso˧ hæ̃˩}}} \zh{扬东西} \textcolor{Sepia}{\selectlanguage{english}to winnow things} \textcolor{PineGreen}{\selectlanguage{french}vanner des choses}  

\lhead{\firstmark}
\rhead{\botmark}

\subsection{\hspace{-0.5cm} {\Large \textcolor{darkblue}{\textbf{\ipa{hæ̃˧do˧}}}}\hspace{0.5cm}[\kern2pt{\textcolor{darkblue}{\textbf{\ipa{hæ̃˧do˧}}}}\kern2pt]} \hypertarget{h\{\string_~\string_Mdo\string_M1}{}
\markboth{\textcolor{darkblue}{\textbf{\ipa{hæ̃˧do˧}}}}{}
\textcolor{teal}{\zh{名词}} \hspace{4pt} \zh{声调类:} M.
\zh{打场。} \textcolor{Sepia}{\selectlanguage{english}Threshing ground.} \textcolor{PineGreen}{\selectlanguage{french}Aire à battre le grain.}  ¶ \textcolor{darkblue}{\textbf{\ipa{hæ̃˧do˧ bæ˩}}} \zh{清扫打场} \textcolor{Sepia}{\selectlanguage{english}to sweep the threshing ground} \textcolor{PineGreen}{\selectlanguage{french}balayer l'aire à battre le grain}  
 \zh{量词}: \textcolor{darkblue}{\textbf{\ipa{ɭɯ˧}}} 
\lhead{\firstmark}
\rhead{\botmark}

\subsection{\hspace{-0.5cm} {\Large \textcolor{darkblue}{\textbf{\ipa{hæ̃˧kʰɤ˧˥}}}}\hspace{0.5cm}[\kern2pt{\textcolor{darkblue}{\textbf{\ipa{hæ̃˧kʰɤ˧˥}}}}\kern2pt]} \hypertarget{h\{\string_~\string_Mk\string_h7\string_M\string_T1}{}
\markboth{\textcolor{darkblue}{\textbf{\ipa{hæ̃˧kʰɤ˧˥}}}}{}
\textcolor{teal}{\zh{名词}} \hspace{4pt} \zh{声调类:} MH\#.
\zh{椽子。} \textcolor{Sepia}{\selectlanguage{english}Rafter; beam.} \textcolor{PineGreen}{\selectlanguage{french}Pièce de charpente: poutrelles de toiture: poutrelles courtes, installées en inclinaison, dans le sens de la largeur du bâtiment, sur les poutres horizontales, ʐv̩˩ɭɯ˥. Les tuiles (autrefois: les planches) reposent sur ces poutrelles.}  \zh{量词}: \textcolor{darkblue}{\textbf{\ipa{ɭɯ˧}}} 
\lhead{\firstmark}
\rhead{\botmark}

\subsection{\hspace{-0.5cm} {\Large \textcolor{darkblue}{\textbf{\ipa{hæ̃˧kʰo˧}}}}\hspace{0.5cm}[\kern2pt{\textcolor{darkblue}{\textbf{\ipa{hæ̃˧kʰo˧}}}}\kern2pt]} \hypertarget{h\{\string_~\string_Mk\string_ho\string_M1}{}
\markboth{\textcolor{darkblue}{\textbf{\ipa{hæ̃˧kʰo˧}}}}{}
\textcolor{teal}{\zh{名词}} \hspace{4pt} \zh{声调类:} M.
\zh{小姐、公主。} \textcolor{Sepia}{\selectlanguage{english}Princess, young lady of the nobility.} \textcolor{PineGreen}{\selectlanguage{french}Demoiselle de la noblesse.}  ¶ \textcolor{darkblue}{\textbf{\ipa{hæ̃˧kʰo˧-mi˧}}} \zh{同上:小姐、公主} \textcolor{Sepia}{\selectlanguage{english}same meaning: young lady} \textcolor{PineGreen}{\selectlanguage{french}même sens: demoiselle, princesse}  
 \zh{量词}: \textcolor{darkblue}{\textbf{\ipa{v̩˧}}} 
\lhead{\firstmark}
\rhead{\botmark}

\subsection{\hspace{-0.5cm} {\Large \textcolor{darkblue}{\textbf{\ipa{hæ̃˧pɤ˧}}}}\hspace{0.5cm}[\kern2pt{\textcolor{darkblue}{\textbf{\ipa{hæ̃˧pɤ˧}}}}\kern2pt]} \hypertarget{h\{\string_~\string_Mp7\string_M1}{}
\markboth{\textcolor{darkblue}{\textbf{\ipa{hæ̃˧pɤ˧}}}}{}
\textcolor{teal}{\zh{名词}} \hspace{4pt} \zh{声调类:} M.
\zh{辫子。} \textcolor{Sepia}{\selectlanguage{english}Plait; braid.} \textcolor{PineGreen}{\selectlanguage{french}Tresse.}  \zh{量词}: \textcolor{darkblue}{\textbf{\ipa{kʰɯ˩}}} 
\lhead{\firstmark}
\rhead{\botmark}

\subsection{\hspace{-0.5cm} {\Large \textcolor{darkblue}{\textbf{\ipa{hæ̃˧qʰv̩˥\$}}}}\hspace{0.5cm}[\kern2pt{\textcolor{darkblue}{\textbf{\ipa{hæ̃˧qʰv̩˥}}}}\kern2pt]} \hypertarget{h\{\string_~\string_Mq\string_hv\string_=\string_T\$1}{}
\markboth{\textcolor{darkblue}{\textbf{\ipa{hæ̃˧qʰv̩˥\$}}}}{}
\textcolor{teal}{\zh{助词}} \hspace{4pt} \zh{声调类:} H\$.
\zh{半夜。} \textcolor{Sepia}{\selectlanguage{english}Late at night, in the middle of night.} \textcolor{PineGreen}{\selectlanguage{french}En pleine nuit, tard dans la nuit.} 
\lhead{\firstmark}
\rhead{\botmark}

\subsection{\hspace{-0.5cm} {\Large \textcolor{darkblue}{\textbf{\ipa{hæ̃˧ʂɯ˩-}}}}\hspace{0.5cm}[\kern2pt{\textcolor{darkblue}{\textbf{\ipa{hæ̃˧ʂɯ˩}}}}\kern2pt]} \hypertarget{h\{\string_~\string_Ms`M\string_B-1}{}
\markboth{\textcolor{darkblue}{\textbf{\ipa{hæ̃˧ʂɯ˩-}}}}{}
\textcolor{teal}{\zh{名词}} \hspace{4pt} \zh{声调类:} L\#.
\textcolor{Sepia}{\selectlanguage{english}'precious': a prefix added to certain nouns to coin a prestige term. This prefix is not currently productive: it cannot be added to terms such as 'mother', 'house'...} \textcolor{PineGreen}{\selectlanguage{french}'précieux': préfixe ajouté à certains noms pour construire une appellation prestigieuse. ; n'est pas productif: on ne peut l'ajouter à: /ə˧mɑ˧/ 'mère', /ɑ˩ʁo˧/ 'maison', etc.}  ¶ \textcolor{darkblue}{\textbf{\ipa{hæ̃˧ʂɯ˩-to˩mi˩}}} \zh{‘黄金柱’、‘宝贵柱’:对主屋两个柱子的庄严称呼} \textcolor{Sepia}{\selectlanguage{english}the Precious Pillars, the Golden Pillars: a solemn designation for the two pillars of the main building} \textcolor{PineGreen}{\selectlanguage{french}les Piliers d'Or, les Précieux Piliers: appellation solennelle pour les deux piliers de la maison}  
 \zh{量词}: \textcolor{darkblue}{\textbf{\ipa{nɑ˧}}} 
\lhead{\firstmark}
\rhead{\botmark}

\subsection{\hspace{-0.5cm} {\Large \textcolor{darkblue}{\textbf{\ipa{hæ̃˧ʂɯ˩-pæ˩pʰæ˩}}}}\hspace{0.5cm}[\kern2pt{\textcolor{darkblue}{\textbf{\ipa{hæ̃˧ʂɯ˩pæ˧pʰæ˧}}}}\kern2pt]} \hypertarget{h\{\string_~\string_Ms`M\string_B-p\{\string_Bp\string_h\{\string_B1}{}
\markboth{\textcolor{darkblue}{\textbf{\ipa{hæ̃˧ʂɯ˩-pæ˩pʰæ˩}}}}{}
\textcolor{teal}{\zh{名词}} \hspace{4pt} \zh{声调类:} L\#-.
\zh{粮架。} \textcolor{Sepia}{\selectlanguage{english}Rack for drying grain.} \textcolor{PineGreen}{\selectlanguage{french}Espalier en bois, dans la cour des fermes, pour faire sécher les épis de maïs avant égrenage.}  \zh{量词}: \textcolor{darkblue}{\textbf{\ipa{pʰæ˧˥}}} 
\lhead{\firstmark}
\rhead{\botmark}

\subsection{\hspace{-0.5cm} {\Large \textcolor{darkblue}{\textbf{\ipa{hæ̃˧ʂv̩˧pɤ˥}}}}\hspace{0.5cm}[\kern2pt{\textcolor{darkblue}{\textbf{\ipa{hæ̃˧ʂv̩˧pɤ˥}}}}\kern2pt]} \hypertarget{h\{\string_~\string_Ms`v\string_=\string_Mp7\string_T1}{}
\markboth{\textcolor{darkblue}{\textbf{\ipa{hæ̃˧ʂv̩˧pɤ˥}}}}{}
\textcolor{teal}{\zh{名词}} \hspace{4pt} \zh{声调类:} H\#.
\zh{丈夫。} \textcolor{Sepia}{\selectlanguage{english}Husband.} \textcolor{PineGreen}{\selectlanguage{french}Mari.} 
\lhead{\firstmark}
\rhead{\botmark}

\subsection{\hspace{-0.5cm} {\Large \textcolor{darkblue}{\textbf{\ipa{hæ̃˧ʐɤ˥}}}}\hspace{0.5cm}[\kern2pt{\textcolor{darkblue}{\textbf{\ipa{hæ̃˧ʐɤ˥}}}}\kern2pt]} \hypertarget{h\{\string_~\string_Mz`7\string_T1}{}
\markboth{\textcolor{darkblue}{\textbf{\ipa{hæ̃˧ʐɤ˥}}}}{}
\textcolor{teal}{\zh{名词}} \hspace{4pt} \zh{声调类:} H\#.
\zh{切割的痕迹。} \textcolor{Sepia}{\selectlanguage{english}Trace of cutting.} \textcolor{PineGreen}{\selectlanguage{french}Trace de découpe, marque de coupure.}  ¶ \textcolor{darkblue}{\textbf{\ipa{hæ̃˧ʐɤ˥ tʰv̩˩-kʰwɤ˩}}} \zh{这道割痕} \textcolor{Sepia}{\selectlanguage{english}\mytextsc{n}+\mytextsc{dem}+\mytextsc{clf}: this trace of cutting} \textcolor{PineGreen}{\selectlanguage{french}\mytextsc{n}+\mytextsc{dem}+\mytextsc{clf}: cette trace de découpe}  
 \zh{量词}: \textcolor{darkblue}{\textbf{\ipa{kʰwɤ˥}}} 
\lhead{\firstmark}
\rhead{\botmark}

\subsection{\hspace{-0.5cm} {\Large \textcolor{darkblue}{\textbf{\ipa{hæ̃˩}}}}\hspace{0.5cm}[\kern2pt{\textcolor{darkblue}{\textbf{\ipa{hæ̃˥}}}}\kern2pt]} \hypertarget{h\{\string_~\string_B1}{}
\markboth{\textcolor{darkblue}{\textbf{\ipa{hæ̃˩}}}}{}
\textcolor{teal}{\zh{名词}} \hspace{4pt} \zh{声调类:} L.
\zh{金子。} \textcolor{Sepia}{\selectlanguage{english}Gold.} \textcolor{PineGreen}{\selectlanguage{french}Or (métal).}  \zh{量词}: \textcolor{darkblue}{\textbf{\ipa{ʈv̩˩}}} 
\lhead{\firstmark}
\rhead{\botmark}

\subsection{\hspace{-0.5cm} {\Large \textcolor{darkblue}{\textbf{\ipa{hæ̃˩-bɑ˧lɑ˩}}}}\hspace{0.5cm}[\kern2pt{\textcolor{darkblue}{\textbf{\ipa{xxxx non-correspondance entre le nombre de morphèmes et le nombre de tons de morphèmes}}}}\kern2pt]} \hypertarget{h\{\string_~\string_B-bA\string_MlA\string_B1}{}
\markboth{\textcolor{darkblue}{\textbf{\ipa{hæ̃˩-bɑ˧lɑ˩}}}}{}
\textcolor{teal}{\zh{名词}} \hspace{4pt} \zh{声调类:} L-L\#.
\zh{丝绸。} \textcolor{Sepia}{\selectlanguage{english}Silk.} \textcolor{PineGreen}{\selectlanguage{french}Soie.}  \zh{量词}: \textcolor{darkblue}{\textbf{\ipa{ɭɯ˧}}} 
\lhead{\firstmark}
\rhead{\botmark}

\subsection{\hspace{-0.5cm} {\Large \textcolor{darkblue}{\textbf{\ipa{hæ̃˩bæ˩}}}}\hspace{0.5cm}[\kern2pt{\textcolor{darkblue}{\textbf{\ipa{hæ̃˩bæ˩˥}}}}\kern2pt]} \hypertarget{h\{\string_~\string_Bb\{\string_B1}{}
\markboth{\textcolor{darkblue}{\textbf{\ipa{hæ̃˩bæ˩}}}}{}
\textcolor{teal}{\zh{动词}} \hspace{4pt} \zh{声调类:} L.
\zh{跳大神。} \textcolor{Sepia}{\selectlanguage{english}To dance a ritual dance.} \textcolor{PineGreen}{\selectlanguage{french}Effectuer une danse rituelle.} 
\lhead{\firstmark}
\rhead{\botmark}

\subsection{\hspace{-0.5cm} {\Large \textcolor{darkblue}{\textbf{\ipa{hæ̃˩di˩}}}}\hspace{0.5cm}[\kern2pt{\textcolor{darkblue}{\textbf{\ipa{xxxx non-correspondance entre le nombre de morphèmes et le nombre de tons de morphèmes}}}}\kern2pt]} \hypertarget{h\{\string_~\string_Bdi\string_B1}{}
\markboth{\textcolor{darkblue}{\textbf{\ipa{hæ̃˩di˩}}}}{}
\textcolor{teal}{\zh{名词}} \hspace{4pt} \zh{声调类:} L.
\zh{尺。} \textcolor{Sepia}{\selectlanguage{english}Ruler.} \textcolor{PineGreen}{\selectlanguage{french}Règle.}  \zh{量词}: \textcolor{darkblue}{\textbf{\ipa{nɑ˧}}} 
\lhead{\firstmark}
\rhead{\botmark}

\subsection{\hspace{-0.5cm} {\Large \textcolor{darkblue}{\textbf{\ipa{hæ̃˩qʰwɤ˩}}}}\hspace{0.5cm}[\kern2pt{\textcolor{darkblue}{\textbf{\ipa{hæ̃˩qʰwɤ˩˥}}}}\kern2pt]} \hypertarget{h\{\string_~\string_Bq\string_hw7\string_B1}{}
\markboth{\textcolor{darkblue}{\textbf{\ipa{hæ̃˩qʰwɤ˩}}}}{}
\textcolor{teal}{\zh{名词}} \hspace{4pt} \zh{声调类:} L.
\zh{薰衣草(永宁的一种植物)。} \textcolor{Sepia}{\selectlanguage{english}Lavender.} \textcolor{PineGreen}{\selectlanguage{french}Lavande.}  \zh{量词}: \textcolor{darkblue}{\textbf{\ipa{ɭɯ˧}}} 
\lhead{\firstmark}
\rhead{\botmark}

\subsection{\hspace{-0.5cm} {\Large \textcolor{darkblue}{\textbf{\ipa{hæ̃˩sɤ˩}}}}\hspace{0.5cm}[\kern2pt{\textcolor{darkblue}{\textbf{\ipa{hæ̃˩sɤ˩˥}}}}\kern2pt]} \hypertarget{h\{\string_~\string_Bs7\string_B1}{}
\markboth{\textcolor{darkblue}{\textbf{\ipa{hæ̃˩sɤ˩}}}}{}
\textcolor{teal}{\zh{名词}} \hspace{4pt} \zh{声调类:} L.
\zh{喜鹊。} \textcolor{Sepia}{\selectlanguage{english}Magpie.} \textcolor{PineGreen}{\selectlanguage{french}Pie.}  \zh{量词}: \textcolor{darkblue}{\textbf{\ipa{mi˩}}} 
\lhead{\firstmark}
\rhead{\botmark}

\subsection{\hspace{-0.5cm} {\Large \textcolor{darkblue}{\textbf{\ipa{hæ̃˧˥}}}}\hspace{0.5cm}[\kern2pt{\textcolor{darkblue}{\textbf{\ipa{hæ̃˥}}}}\kern2pt]} \hypertarget{h\{\string_~\string_M\string_T1}{}
\markboth{\textcolor{darkblue}{\textbf{\ipa{hæ̃˧˥}}}}{}
\textcolor{teal}{\zh{动词}} \hspace{4pt} \zh{声调类:} MH.
\ding{202} \zh{切,裁。} \textcolor{Sepia}{\selectlanguage{english}To cut (with a blade: sword…), e.g. to cut cloth (to make clothes).} \textcolor{PineGreen}{\selectlanguage{french}Trancher, couper au moyen d'un instrument tranchant: couteau, épée…; ex.: tailler un vêtement.}  ¶ \textcolor{darkblue}{\textbf{\ipa{le˧-hæ̃˧-ze˥}}} \zh{切了} \textcolor{Sepia}{\selectlanguage{english}\mytextsc{accomp} \string_ \mytextsc{pfv}} \textcolor{PineGreen}{\selectlanguage{french}\mytextsc{accomp} \string_ \mytextsc{pfv}}  
 ¶ \textcolor{darkblue}{\textbf{\ipa{tʰɑ˧-hæ̃˧˥!}}} \zh{别切!} \textcolor{Sepia}{\selectlanguage{english}\mytextsc{prohib}} \textcolor{PineGreen}{\selectlanguage{french}\mytextsc{prohib}}  
 ¶ \textcolor{darkblue}{\textbf{\ipa{bɑ˩lɑ˩˥ | le˧-hæ̃˧˥, | le˧-ʐv̩˧˥}}} \zh{裁(布料来做)衣服,又缝(衣服) / 先裁布料,再缝衣服} \textcolor{Sepia}{\selectlanguage{english}to cut cloth to make clothes, and to sew clothes} \textcolor{PineGreen}{\selectlanguage{french}tailler des vêtements et les coudre}  
\ding{203} \zh{阉割。} \textcolor{Sepia}{\selectlanguage{english}To castrate.} \textcolor{PineGreen}{\selectlanguage{french}Castrer, châtrer.} 
\lhead{\firstmark}
\rhead{\botmark}

\subsection{\hspace{-0.5cm} {\Large \textcolor{darkblue}{\textbf{\ipa{hɤ˧}}}}\hspace{0.5cm}[\kern2pt{\textcolor{darkblue}{\textbf{\ipa{hɤ˥}}}}\kern2pt]} \hypertarget{h7\string_M1}{}
\markboth{\textcolor{darkblue}{\textbf{\ipa{hɤ˧}}}}{}
\textcolor{teal}{\zh{名词}} \hspace{4pt} \zh{声调类:} M.
\zh{全部。} \textcolor{Sepia}{\selectlanguage{english}All.} \textcolor{PineGreen}{\selectlanguage{french}Tout.}  ¶ \textcolor{darkblue}{\textbf{\ipa{ɖɯ˧-hɤ˧ | mɤ˧-go˩}}} \zh{一点也没病、没有任何痛苦} \textcolor{Sepia}{\selectlanguage{english}to have no ailment at all, to be free from any pain} \textcolor{PineGreen}{\selectlanguage{french}n'avoir aucune maladie, ne souffrir de rien}  
 ¶ \textcolor{darkblue}{\textbf{\ipa{ɖɯ˧-hɤ˧ | mɤ˧-sɯ˥}}} \zh{什么也不知道} \textcolor{Sepia}{\selectlanguage{english}to be ignorant of everything (literally: not to know a thing)} \textcolor{PineGreen}{\selectlanguage{french}ne pas savoir quoi que ce soit, être ignorant de tout, ne rien savoir du tout}  
 ¶ \textcolor{darkblue}{\textbf{\ipa{ʈʂʰɯ˧ | ɖɯ˧-hɤ˧ hwæ˧}}} \zh{他全部都买。/他什么都买。} \textcolor{Sepia}{\selectlanguage{english}(S)he buys everything / buys the lot} \textcolor{PineGreen}{\selectlanguage{french}(il/elle) achète (le) tout}  

\lhead{\firstmark}
\rhead{\botmark}

\subsection{\hspace{-0.5cm} {\Large \textcolor{darkblue}{\textbf{\ipa{hɤ˩\textsubscript{a}}}} \textsubscript{1}}\hspace{0.5cm}[\kern2pt{\textcolor{darkblue}{\textbf{\ipa{hɤ˩˥}}}}\kern2pt]} \hypertarget{h7\string_Ba1}{}
\markboth{\textcolor{darkblue}{\textbf{\ipa{hɤ˩\textsubscript{a}}}} \textsubscript{1}}{}
\textcolor{teal}{\zh{动词}} \hspace{4pt} \zh{声调类:} L\textsubscript{a}.
\zh{烘干。} \textcolor{Sepia}{\selectlanguage{english}To dry beside or over a fire.} \textcolor{PineGreen}{\selectlanguage{french}Chauffer au feu, sécher au feu.}  ¶ \textcolor{darkblue}{\textbf{\ipa{tʰi˧-hɤ˩}}} \zh{\mytextsc{dur}} \textcolor{Sepia}{\selectlanguage{english}\mytextsc{dur}} \textcolor{PineGreen}{\selectlanguage{french}\mytextsc{dur}}  
 ¶ \textcolor{darkblue}{\textbf{\ipa{le˧-hɤ˩}}} \zh{\mytextsc{accomp}} \textcolor{Sepia}{\selectlanguage{english}\mytextsc{accomp}} \textcolor{PineGreen}{\selectlanguage{french}\mytextsc{accomp}}  
 ¶ \textcolor{darkblue}{\textbf{\ipa{ɖɯ˧-hɤ˩-ɻ̍˩}}} \zh{烘干一下} \textcolor{Sepia}{\selectlanguage{english}\mytextsc{delimitative} \string_ \mytextsc{inceptive}} \textcolor{PineGreen}{\selectlanguage{french}\mytextsc{délimitatif} \string_ \mytextsc{inchoatif}: chauffer un coup, chauffer un peu}  
 ¶ \textcolor{darkblue}{\textbf{\ipa{le˧-hɤ˩-ze˩, | le˧-pv̩˧-ze˧!}}} \zh{烘干了,(现在)干了!} \textcolor{Sepia}{\selectlanguage{english}It was dried beside the fire, and it got dry / and it is now dry!} \textcolor{PineGreen}{\selectlanguage{french}on l'a chauffé au feu, ça a séché}  
 ¶ \textcolor{darkblue}{\textbf{\ipa{ɖɯ˧-kʰwɤ˧ hɤ˥}}} \zh{烘干一个东西} \textcolor{Sepia}{\selectlanguage{english}to dry something beside the fire} \textcolor{PineGreen}{\selectlanguage{french}chauffer quelque chose}  

\lhead{\firstmark}
\rhead{\botmark}

\subsection{\hspace{-0.5cm} {\Large \textcolor{darkblue}{\textbf{\ipa{hɤ˩\textsubscript{a}}}} \textsubscript{2}}\hspace{0.5cm}[\kern2pt{\textcolor{darkblue}{\textbf{\ipa{hɤ˩˥}}}}\kern2pt]} \hypertarget{h7\string_Ba2}{}
\markboth{\textcolor{darkblue}{\textbf{\ipa{hɤ˩\textsubscript{a}}}} \textsubscript{2}}{}
\textcolor{teal}{\zh{动词}} \hspace{4pt} \zh{声调类:} L\textsubscript{a}.
\zh{去,\mytextsc{过去式,°整体体。}} \textcolor{Sepia}{\selectlanguage{english}To go, past perfective form: has gone.} \textcolor{PineGreen}{\selectlanguage{french}Partir, forme passée perfective.} 
\lhead{\firstmark}
\rhead{\botmark}

\subsection{\hspace{-0.5cm} {\Large \textcolor{darkblue}{\textbf{\ipa{hɤ˩\textsubscript{a}}}} \textsubscript{3}}\hspace{0.5cm}[\kern2pt{\textcolor{darkblue}{\textbf{\ipa{hɤ˩˥}}}}\kern2pt]} \hypertarget{h7\string_Ba3}{}
\markboth{\textcolor{darkblue}{\textbf{\ipa{hɤ˩\textsubscript{a}}}} \textsubscript{3}}{}
\textcolor{teal}{\zh{形容词}} \hspace{4pt} \zh{声调类:} L\textsubscript{a}.
\zh{好(技巧好),好(表扬一个人的行为)。} \textcolor{Sepia}{\selectlanguage{english}Appropriate, good; of person: able, good at a certain technique.} \textcolor{PineGreen}{\selectlanguage{french}Bon, approprié, bien; capable (adjectif), habile à une technique.}  ¶ \textcolor{darkblue}{\textbf{\ipa{ɖwæ˧˥ | hɤ˩˥!}}} \zh{很好!} \textcolor{Sepia}{\selectlanguage{english}\mytextsc{intensive}.very} \textcolor{PineGreen}{\selectlanguage{french}\mytextsc{intensif}.très: c'est très bien!}  
 ¶ \textcolor{darkblue}{\textbf{\ipa{mɤ˧-hɤ˩}}} \zh{不好} \textcolor{Sepia}{\selectlanguage{english}\mytextsc{neg}} \textcolor{PineGreen}{\selectlanguage{french}\mytextsc{neg}}  
 ¶ \textcolor{darkblue}{\textbf{\ipa{hɤ˩-hĩ˩˥}}} \zh{好的} \textcolor{Sepia}{\selectlanguage{english}\mytextsc{rel}/\mytextsc{nmlz}} \textcolor{PineGreen}{\selectlanguage{french}\mytextsc{rel}/\mytextsc{nmlz}}  

\lhead{\firstmark}
\rhead{\botmark}

\subsection{\hspace{-0.5cm} {\Large \textcolor{darkblue}{\textbf{\ipa{hi˥}}}}\hspace{0.5cm}[\kern2pt{\textcolor{darkblue}{\textbf{\ipa{hi˥}}}}\kern2pt]} \hypertarget{hi\string_T1}{}
\markboth{\textcolor{darkblue}{\textbf{\ipa{hi˥}}}}{}
\textcolor{teal}{\zh{名词}} \hspace{4pt} \zh{声调类:} \#H.
\zh{牙齿。} \textcolor{Sepia}{\selectlanguage{english}Tooth.} \textcolor{PineGreen}{\selectlanguage{french}Dent.}  ¶ \textcolor{darkblue}{\textbf{\ipa{hi˧ go˧˥}}} \zh{牙疼} \textcolor{Sepia}{\selectlanguage{english}(a) tooth aches; to have a tooth-ache} \textcolor{PineGreen}{\selectlanguage{french}avoir mal aux dents}  
 \zh{量词}: \textcolor{darkblue}{\textbf{\ipa{ɭɯ˧}}} 
\lhead{\firstmark}
\rhead{\botmark}

\subsection{\hspace{-0.5cm} {\Large \textcolor{darkblue}{\textbf{\ipa{*hi˧}}}}\hspace{0.5cm}[\kern2pt{\textcolor{darkblue}{\textbf{\ipa{hi˥}}}}\kern2pt]} \hypertarget{*hi\string_M1}{}
\markboth{\textcolor{darkblue}{\textbf{\ipa{*hi˧}}}}{}
\textcolor{teal}{\zh{形容词}} \hspace{4pt} \zh{声调类:} H?.
\textit{\zh{古语}} [\zh{古语}] \zh{快。} \textcolor{Sepia}{\selectlanguage{english}Fast.} \textcolor{PineGreen}{\selectlanguage{french}Rapide, rapidement (racine extraite de la forme disyllabique).}  ¶ \textcolor{darkblue}{\textbf{\ipa{hi˧le˩ ʝi˩}}} \zh{快速做} \textcolor{Sepia}{\selectlanguage{english}to do quickly} \textcolor{PineGreen}{\selectlanguage{french}faire rapidement}  
 ¶ \textcolor{darkblue}{\textbf{\ipa{hi˧le˩ | le˧-jo˩!}}} \zh{快来!} \textcolor{Sepia}{\selectlanguage{english}Come quickly!} \textcolor{PineGreen}{\selectlanguage{french}viens vite!}  
 ¶ \textcolor{darkblue}{\textbf{\ipa{ʈʂʰɯ˧ | ɖwæ˧˥ | hi˧le˩ | ʝi˧-kv̩˩!}}} \zh{他做事很麻利!} \textcolor{Sepia}{\selectlanguage{english}(S)he knows how to work really fast!} \textcolor{PineGreen}{\selectlanguage{french}Lui, il sait travailler vite!}  

\lhead{\firstmark}
\rhead{\botmark}

\subsection{\hspace{-0.5cm} {\Large \textcolor{darkblue}{\textbf{\ipa{hi˧dʑi˧}}}}\hspace{0.5cm}[\kern2pt{\textcolor{darkblue}{\textbf{\ipa{hi˧dʑi˧}}}}\kern2pt]} \hypertarget{hi\string_Mdz£i\string_M1}{}
\markboth{\textcolor{darkblue}{\textbf{\ipa{hi˧dʑi˧}}}}{}
\textcolor{teal}{\zh{名词}} \hspace{4pt} \zh{声调类:} M.
\zh{蓑衣。} \textcolor{Sepia}{\selectlanguage{english}Rain cape, rainware made from straw, rush….} \textcolor{PineGreen}{\selectlanguage{french}Cape de pluie, vêtement qui protège de la pluie (en paille, écorce...).}  ¶ \textcolor{darkblue}{\textbf{\ipa{hi˩ gi˩-ze˥, | hi˧dʑi˧ tʰi˧-mv̩˧.}}} \zh{下雨了,披蓑衣(雨衣)吧。} \textcolor{Sepia}{\selectlanguage{english}It has begun to rain / it's raining; put on a rain cape.} \textcolor{PineGreen}{\selectlanguage{french}Il pleut, mets une cape de pluie.}  
 \zh{量词}: \textcolor{darkblue}{\textbf{\ipa{ɭɯ˧}}} 
\lhead{\firstmark}
\rhead{\botmark}

\subsection{\hspace{-0.5cm} {\Large \textcolor{darkblue}{\textbf{\ipa{hi˧kʰɯ\#˥}}}}\hspace{0.5cm}[\kern2pt{\textcolor{darkblue}{\textbf{\ipa{hi˧kʰɯ˧}}}}\kern2pt]} \hypertarget{hi\string_Mk\string_hM\#\string_T1}{}
\markboth{\textcolor{darkblue}{\textbf{\ipa{hi˧kʰɯ\#˥}}}}{}
\textcolor{teal}{\zh{名词}} \hspace{4pt} \zh{声调类:} \#H.
\zh{牙龈。} \textcolor{Sepia}{\selectlanguage{english}Gum; gingiva.} \textcolor{PineGreen}{\selectlanguage{french}Gencive.}  ¶ \textcolor{darkblue}{\textbf{\ipa{hi˧kʰɯ˧ ʈʂʰæ˧}}} \zh{刷牙} \textcolor{Sepia}{\selectlanguage{english}to brush one's teeth} \textcolor{PineGreen}{\selectlanguage{french}se brosser les dents; \textcolor{darkblue}{\textbf{\ipa{/hi˧kʰɯ˧/}}} peut désigner tout ce qu'on lave quand on se brosse les dents: gencives et dents.}  
 ¶ \textcolor{darkblue}{\textbf{\ipa{hi˧kʰɯ˧-ʈv̩˥}}} \zh{牙根} \textcolor{Sepia}{\selectlanguage{english}root of the teeth} \textcolor{PineGreen}{\selectlanguage{french}racine des dents}  

\lhead{\firstmark}
\rhead{\botmark}

\subsection{\hspace{-0.5cm} {\Large \textcolor{darkblue}{\textbf{\ipa{hi˧qʰwɤ˩}}}}\hspace{0.5cm}[\kern2pt{\textcolor{darkblue}{\textbf{\ipa{hi˧qʰwɤ˩}}}}\kern2pt]} \hypertarget{hi\string_Mq\string_hw7\string_B1}{}
\markboth{\textcolor{darkblue}{\textbf{\ipa{hi˧qʰwɤ˩}}}}{}
\textcolor{teal}{\zh{名词}} \hspace{4pt} \zh{声调类:} L\#.
\zh{蛀牙。} \textcolor{Sepia}{\selectlanguage{english}Decayed teeth; dental caries.} \textcolor{PineGreen}{\selectlanguage{french}Dent gâtée, dent cariée, carie.}  \zh{量词}: \textcolor{darkblue}{\textbf{\ipa{ɭɯ˧}}} 
\lhead{\firstmark}
\rhead{\botmark}

\subsection{\hspace{-0.5cm} {\Large \textcolor{darkblue}{\textbf{\ipa{hi˧tʰɑ˩}}}}\hspace{0.5cm}[\kern2pt{\textcolor{darkblue}{\textbf{\ipa{hi˧tʰɑ˩}}}}\kern2pt]} \hypertarget{hi\string_Mt\string_hA\string_B1}{}
\markboth{\textcolor{darkblue}{\textbf{\ipa{hi˧tʰɑ˩}}}}{}
\textcolor{teal}{\zh{形容词}} \hspace{4pt} \zh{声调类:} L\#.
\zh{锋利。} \textcolor{Sepia}{\selectlanguage{english}Sharp, keen (blade).} \textcolor{PineGreen}{\selectlanguage{french}Aiguisé, qui coupe bien, affûté.} 
\lhead{\firstmark}
\rhead{\botmark}

\subsection{\hspace{-0.5cm} {\Large \textcolor{darkblue}{\textbf{\ipa{hi˧tʰo˧˥}}}}\hspace{0.5cm}[\kern2pt{\textcolor{darkblue}{\textbf{\ipa{hi˧tʰo˧˥}}}}\kern2pt]} \hypertarget{hi\string_Mt\string_ho\string_M\string_T1}{}
\markboth{\textcolor{darkblue}{\textbf{\ipa{hi˧tʰo˧˥}}}}{}
\textcolor{teal}{\zh{名词}} \hspace{4pt} \zh{声调类:} MH\#.
\zh{牙齿。} \textcolor{Sepia}{\selectlanguage{english}Tooth.} \textcolor{PineGreen}{\selectlanguage{french}Dent.}  \zh{量词}: \textcolor{darkblue}{\textbf{\ipa{ɭɯ˧}}} 
\lhead{\firstmark}
\rhead{\botmark}

\subsection{\hspace{-0.5cm} {\Large \textcolor{darkblue}{\textbf{\ipa{hi˧tsɯ˩}}}}\hspace{0.5cm}[\kern2pt{\textcolor{darkblue}{\textbf{\ipa{hi˧tsɯ˩}}}}\kern2pt]} \hypertarget{hi\string_MtsM\string_B1}{}
\markboth{\textcolor{darkblue}{\textbf{\ipa{hi˧tsɯ˩}}}}{}
\textcolor{teal}{\zh{名词}} \hspace{4pt} \zh{声调类:} L\#.
\zh{门牙。} \textcolor{Sepia}{\selectlanguage{english}Incisors, front teeth.} \textcolor{PineGreen}{\selectlanguage{french}Incisives (dents).}  \zh{量词}: \textcolor{darkblue}{\textbf{\ipa{ɭɯ˧}}} 
\lhead{\firstmark}
\rhead{\botmark}

\subsection{\hspace{-0.5cm} {\Large \textcolor{darkblue}{\textbf{\ipa{hi˩}}} \textsubscript{1}}\hspace{0.5cm}[\kern2pt{\textcolor{darkblue}{\textbf{\ipa{hi˥}}}}\kern2pt]} \hypertarget{hi\string_B1}{}
\markboth{\textcolor{darkblue}{\textbf{\ipa{hi˩}}} \textsubscript{1}}{}
\textcolor{teal}{\zh{名词}} \hspace{4pt} \zh{声调类:} L.
\zh{湖、海(单音节)。} \textcolor{Sepia}{\selectlanguage{english}Lake (monosyllabic word).} \textcolor{PineGreen}{\selectlanguage{french}Lac (monosyllabe).}  \zh{量词}: \textcolor{darkblue}{\textbf{\ipa{ɭɯ˧}}} 
\lhead{\firstmark}
\rhead{\botmark}

\subsection{\hspace{-0.5cm} {\Large \textcolor{darkblue}{\textbf{\ipa{hi˩}}} \textsubscript{2}}\hspace{0.5cm}[\kern2pt{\textcolor{darkblue}{\textbf{\ipa{hi˩˥}}}}\kern2pt]} \hypertarget{hi\string_B2}{}
\markboth{\textcolor{darkblue}{\textbf{\ipa{hi˩}}} \textsubscript{2}}{}
\textcolor{teal}{\zh{动词}} \hspace{4pt} \zh{声调类:} L.
\zh{存在动词:固定不动的物体,如:泸沽湖。} \textcolor{Sepia}{\selectlanguage{english}Existential verb, for unmovable objects: e.g.the Lake exists/is at a certain place.} \textcolor{PineGreen}{\selectlanguage{french}Exister, se trouver: verbe d'existence pour objets non mobiles, par exemple le Lac existe/se trouve à un endroit.}  ¶ \textcolor{darkblue}{\textbf{\ipa{hi˩nɑ˧mi˧ | tʰi˧-hi˩}}} \zh{有(泸沽)湖(在那儿)} \textcolor{Sepia}{\selectlanguage{english}the Lake exists/is there} \textcolor{PineGreen}{\selectlanguage{french}le Lac se trouve là/existe là/se trouve là, immuable}  

\lhead{\firstmark}
\rhead{\botmark}

\subsection{\hspace{-0.5cm} {\Large \textcolor{darkblue}{\textbf{\ipa{hi˩dʑɯ˩}}}}\hspace{0.5cm}[\kern2pt{\textcolor{darkblue}{\textbf{\ipa{hi˩dʑɯ˩˥}}}}\kern2pt]} \hypertarget{hi\string_Bdz£M\string_B1}{}
\markboth{\textcolor{darkblue}{\textbf{\ipa{hi˩dʑɯ˩}}}}{}
\textcolor{teal}{\zh{名词}} \hspace{4pt} \zh{声调类:} L.
\zh{炭。} \textcolor{Sepia}{\selectlanguage{english}Charcoal.} \textcolor{PineGreen}{\selectlanguage{french}Charbon de bois.}  \zh{量词}: \textcolor{darkblue}{\textbf{\ipa{kʰɤ˧˥}}} 
\lhead{\firstmark}
\rhead{\botmark}

\subsection{\hspace{-0.5cm} {\Large \textcolor{darkblue}{\textbf{\ipa{hi˩mi˩}}}}\hspace{0.5cm}[\kern2pt{\textcolor{darkblue}{\textbf{\ipa{hi˩mi˩˥}}}}\kern2pt]} \hypertarget{hi\string_Bmi\string_B1}{}
\markboth{\textcolor{darkblue}{\textbf{\ipa{hi˩mi˩}}}}{}
\textcolor{teal}{\zh{名词}} \hspace{4pt} \zh{声调类:} L.
\zh{舌头。} \textcolor{Sepia}{\selectlanguage{english}Tongue.} \textcolor{PineGreen}{\selectlanguage{french}Langue.}  ¶ \textcolor{darkblue}{\textbf{\ipa{hi˩mi˩˥, | ɻ̃˧ mɤ˧-ʑi˧! | ə˧tso˧ ʐwɤ˩-bi˩, | õ˧-lɑ˥ ɖʐv̩˩!}}} \zh{“舌头没有骨头。讲的是什么(=是否真的),只有自己才知道!”(谚语)} \textcolor{Sepia}{\selectlanguage{english}The tongue has no bone! Only oneself knows what one is going to say! (Proverb meaning that one is responsible for one's speech: only oneself knows whether one is telling the truth or not.)} \textcolor{PineGreen}{\selectlanguage{french}“La langue n'a pas d'os! Ce qu'on dit, soi seul sait (si c'est la vérité)!”}  
 \zh{量词}: \textcolor{darkblue}{\textbf{\ipa{ɭɯ˧}}} 
\lhead{\firstmark}
\rhead{\botmark}

\subsection{\hspace{-0.5cm} {\Large \textcolor{darkblue}{\textbf{\ipa{hi˩nɑ˧mi\#˥}}}}\hspace{0.5cm}[\kern2pt{\textcolor{darkblue}{\textbf{\ipa{hi˩nɑ˧mi˧}}}}\kern2pt]} \hypertarget{hi\string_BnA\string_Mmi\#\string_T1}{}
\markboth{\textcolor{darkblue}{\textbf{\ipa{hi˩nɑ˧mi\#˥}}}}{}
\textcolor{teal}{\zh{名词}} \hspace{4pt} \zh{声调类:} LM+\#H.
\zh{湖。} \textcolor{Sepia}{\selectlanguage{english}Lake.} \textcolor{PineGreen}{\selectlanguage{french}Lac.}  \zh{量词}: \textcolor{darkblue}{\textbf{\ipa{ɭɯ˧}}} 
\lhead{\firstmark}
\rhead{\botmark}

\subsection{\hspace{-0.5cm} {\Large \textcolor{darkblue}{\textbf{\ipa{hi˩ɲi˩zo˩}}}}\hspace{0.5cm}[\kern2pt{\textcolor{darkblue}{\textbf{\ipa{hi˩ɲi˩zo˩˥}}}}\kern2pt]} \hypertarget{hi\string_BJi\string_Bzo\string_B1}{}
\markboth{\textcolor{darkblue}{\textbf{\ipa{hi˩ɲi˩zo˩}}}}{}
\textcolor{teal}{\zh{名词}} \hspace{4pt} \zh{声调类:} L.
\zh{娃娃鱼。} \textcolor{Sepia}{\selectlanguage{english}Salamander.} \textcolor{PineGreen}{\selectlanguage{french}Salamandre.}  \zh{量词}: \textcolor{darkblue}{\textbf{\ipa{mi˩}}} 
\lhead{\firstmark}
\rhead{\botmark}

\subsection{\hspace{-0.5cm} {\Large \textcolor{darkblue}{\textbf{\ipa{hi˩qʰɑ˩}}}}\hspace{0.5cm}[\kern2pt{\textcolor{darkblue}{\textbf{\ipa{hi˩qʰɑ˩˥}}}}\kern2pt]} \hypertarget{hi\string_Bq\string_hA\string_B1}{}
\markboth{\textcolor{darkblue}{\textbf{\ipa{hi˩qʰɑ˩}}}}{}
\textcolor{teal}{\zh{名词}} \hspace{4pt} \zh{声调类:} L.
\zh{暴雨。} \textcolor{Sepia}{\selectlanguage{english}Torrential rain, cloudburst.} \textcolor{PineGreen}{\selectlanguage{french}Orage.}  ¶ \textcolor{darkblue}{\textbf{\ipa{hi˩qʰɑ˩ lɑ˥(-ze˩)}}} \zh{下暴雨了} \textcolor{Sepia}{\selectlanguage{english}torrential rain is falling} \textcolor{PineGreen}{\selectlanguage{french}l'orage éclate, il y a de l'orage}  
 \zh{量词}: \textcolor{darkblue}{\textbf{\ipa{ʂɯ˩}}} 
\lhead{\firstmark}
\rhead{\botmark}

\subsection{\hspace{-0.5cm} {\Large \textcolor{darkblue}{\textbf{\ipa{hi˩ʁwɤ˩-lo˧}}}}\hspace{0.5cm}[\kern2pt{\textcolor{darkblue}{\textbf{\ipa{hi˩ʁwɤ˩lo˥}}}}\kern2pt]} \hypertarget{hi\string_BRw7\string_B-lo\string_M1}{}
\markboth{\textcolor{darkblue}{\textbf{\ipa{hi˩ʁwɤ˩-lo˧}}}}{}
\textcolor{teal}{\zh{名词}} \hspace{4pt} \zh{声调类:} L-.
\zh{永宁的一个村落。} \textcolor{Sepia}{\selectlanguage{english}The name of a village in the plain of Yongning.} \textcolor{PineGreen}{\selectlanguage{french}Un village de la plaine de Yongning.}  ¶ \textcolor{darkblue}{\textbf{\ipa{dʑɤ˩bv̩˧kɤ˧-sɑ˥ʁwɤ˩, | hi˩ʁwɤ˩-lo˥, | æ˩mi˧-ʁwɤ\#˥, | lɑ˧lo˧-ʁwɤ˥, | lɑ˧ŋwɤ˧, | bɤ˧tsʰo˧gv̩˥, | ə˧lɑ˧-ʁwɤ\#˥, | gæ˧ɻæ˩, | qʰæ˧tɕʰi˧, | tʰo˧ʈɯ\#˥}}} \zh{摩梭传统地理概念中,属于永宁的十个村落} \textcolor{Sepia}{\selectlanguage{english}the ten villages traditionally considered as part of Yongning} \textcolor{PineGreen}{\selectlanguage{french}les dix villages comptant traditionnellement comme faisant partie de Yongning}  

\lhead{\firstmark}
\rhead{\botmark}

\subsection{\hspace{-0.5cm} {\Large \textcolor{darkblue}{\textbf{\ipa{hi˩ʐæ˥}}}}\hspace{0.5cm}[\kern2pt{\textcolor{darkblue}{\textbf{\ipa{hi˩ʐæ˥}}}}\kern2pt]} \hypertarget{hi\string_Bz`\{\string_T1}{}
\markboth{\textcolor{darkblue}{\textbf{\ipa{hi˩ʐæ˥}}}}{}
\textcolor{teal}{\zh{名词}} \hspace{4pt} \zh{声调类:} LH.
\ding{202} \zh{小舌。} \textcolor{Sepia}{\selectlanguage{english}Uvula.} \textcolor{PineGreen}{\selectlanguage{french}Luette.}  ¶ \textcolor{darkblue}{\textbf{\ipa{qv̩˧ʈʂæ˧-bv̩˥ | hi˩ʐæ˧}}} \zh{小舌} \textcolor{Sepia}{\selectlanguage{english}the uvula; specifying 'the throat's...' disambiguates between the uvula and the tendon of the tongue, which are referred to by the same term, \textcolor{darkblue}{\textbf{\ipa{/hi˩ʐæ˥/}}}.} \textcolor{PineGreen}{\selectlanguage{french}la luette; la précision '...de la gorge' permet de lever l'ambiguïté lorsqu'il pourrait s'agir du tendon de la langue, lui aussi désigné comme \textcolor{darkblue}{\textbf{\ipa{/hi˩ʐæ˥/}}}.}  
 \zh{量词}: \textcolor{darkblue}{\textbf{\ipa{ɭɯ˧}}} \ding{203} \zh{舌头的筋。} \textcolor{Sepia}{\selectlanguage{english}Tendon of the tongue.} \textcolor{PineGreen}{\selectlanguage{french}Tendon de la langue.}  ¶ \textcolor{darkblue}{\textbf{\ipa{hi˩mi˩-bv̩˧ | hi˩ʐæ˧}}} \zh{舌头的筋} \textcolor{Sepia}{\selectlanguage{english}the tendon of the tongue; specifying 'the tongue's...' disambiguates between the uvula and the tendon of the tongue, which are referred to by the same term, \textcolor{darkblue}{\textbf{\ipa{/hi˩ʐæ˥/}}}.} \textcolor{PineGreen}{\selectlanguage{french}le tendon de la langue; la précision 'de la langue' permet de lever l'ambiguïté dans les cas où il pourrait aussi s'agir de la luette, elle aussi désignée comme \textcolor{darkblue}{\textbf{\ipa{/hi˩ʐæ˥/}}}.}  

\lhead{\firstmark}
\rhead{\botmark}

\subsection{\hspace{-0.5cm} {\Large \textcolor{darkblue}{\textbf{\ipa{hi˩˥}}}}\hspace{0.5cm}[\kern2pt{\textcolor{darkblue}{\textbf{\ipa{hi˩˥}}}}\kern2pt]} \hypertarget{hi\string_B\string_T1}{}
\markboth{\textcolor{darkblue}{\textbf{\ipa{hi˩˥}}}}{}
\textcolor{teal}{\zh{名词}} \hspace{4pt} \zh{声调类:} LH.
\zh{雨。} \textcolor{Sepia}{\selectlanguage{english}Rain.} \textcolor{PineGreen}{\selectlanguage{french}Pluie.}  ¶ \textcolor{darkblue}{\textbf{\ipa{hi˩ gi˩˥ / hi˩ gi˩-ze˥}}} \zh{下雨了} \textcolor{Sepia}{\selectlanguage{english}it's raining} \textcolor{PineGreen}{\selectlanguage{french}il pleut}  
 \zh{量词}: \textcolor{darkblue}{\textbf{\ipa{ʂɯ˩}}} 
\lhead{\firstmark}
\rhead{\botmark}

\subsection{\hspace{-0.5cm} {\Large \textcolor{darkblue}{\textbf{\ipa{‑hĩ˥}}}}\hspace{0.5cm}[\kern2pt{\textcolor{darkblue}{\textbf{\ipa{xxxx groupe tonal entier sans aucun ton}}}}\kern2pt]} \hypertarget{‑hi\string_~\string_T1}{}
\markboth{\textcolor{darkblue}{\textbf{\ipa{‑hĩ˥}}}}{}
\textcolor{teal}{\zh{连接词}} \hspace{4pt} \zh{声调类:} 0.
\zh{关系从句/名词化。} \textcolor{Sepia}{\selectlanguage{english}Relativizer and nominalizer.} \textcolor{PineGreen}{\selectlanguage{french}Relativisateur et nominalisateur.} 
\lhead{\firstmark}
\rhead{\botmark}

\subsection{\hspace{-0.5cm} {\Large \textcolor{darkblue}{\textbf{\ipa{hĩ˥}}}}\hspace{0.5cm}[\kern2pt{\textcolor{darkblue}{\textbf{\ipa{hĩ˥}}}}\kern2pt]} \hypertarget{hi\string_~\string_T1}{}
\markboth{\textcolor{darkblue}{\textbf{\ipa{hĩ˥}}}}{}
\textcolor{teal}{\zh{名词}} \hspace{4pt} \zh{声调类:} \#H.
\zh{人。} \textcolor{Sepia}{\selectlanguage{english}Person, human being, man (without any indication of gender).} \textcolor{PineGreen}{\selectlanguage{french}Personne; être humain; homme (sans indication de genre).}  ¶ \textcolor{darkblue}{\textbf{\ipa{hĩ˧ | ɖɯ˧-v̩˧}}} \zh{一个人} \textcolor{Sepia}{\selectlanguage{english}one person, an individual} \textcolor{PineGreen}{\selectlanguage{french}une personne}  
 ¶ \textcolor{darkblue}{\textbf{\ipa{hĩ˧-ɻ̃˧ | ɖɯ˧-lo˩}}} \zh{一个家族} \textcolor{Sepia}{\selectlanguage{english}a lineage, a family} \textcolor{PineGreen}{\selectlanguage{french}une lignée, une famille}  
 ¶ \textcolor{darkblue}{\textbf{\ipa{hĩ˧-mv˥ hĩ˩-di˩}}} \zh{人家的地方,人家的故乡(不是自己的地方)} \textcolor{Sepia}{\selectlanguage{english}other people's home, other people's place (as opposed to one's home place)} \textcolor{PineGreen}{\selectlanguage{french}les terres étrangères, les terres d'autres gens (par opposition avec sa propre terre natale)}  
 ¶ \textcolor{darkblue}{\textbf{\ipa{hĩ˧-mv˥ hĩ˩-di˩ | qʰɑ˧-dʑɤ˥\textasciitilde{}dʑɤ˩, | õ˧-mv˥ õ˩-di˩ tsʰe˩ mɤ˩-gv˩!}}} \zh{其他人的地方怎么好,也比不过自己的地方!} \textcolor{Sepia}{\selectlanguage{english}No matter how beautiful other people's places are, they can never be equal to one's own homeland!} \textcolor{PineGreen}{\selectlanguage{french}Si belles soient les terres d'autrui, elles n'auront jamais la beauté de ses propres terres / de la terre natale !}  
 \zh{量词}: \textcolor{darkblue}{\textbf{\ipa{v̩˧}}} 
\lhead{\firstmark}
\rhead{\botmark}

\subsection{\hspace{-0.5cm} {\Large \textcolor{darkblue}{\textbf{\ipa{hĩ˧bæ\#˥}}}}\hspace{0.5cm}[\kern2pt{\textcolor{darkblue}{\textbf{\ipa{hĩ˧bæ˧}}}}\kern2pt]} \hypertarget{hi\string_~\string_Mb\{\#\string_T1}{}
\markboth{\textcolor{darkblue}{\textbf{\ipa{hĩ˧bæ\#˥}}}}{}
\textcolor{teal}{\zh{名词}} \hspace{4pt} \zh{声调类:} \#H.
\zh{客人。} \textcolor{Sepia}{\selectlanguage{english}Guest, visitor.} \textcolor{PineGreen}{\selectlanguage{french}Invité, visiteur, hôte.}  ¶ \textcolor{darkblue}{\textbf{\ipa{hĩ˧bæ˧ ʝi˧}}} \zh{做客} \textcolor{Sepia}{\selectlanguage{english}to be a guest, to be invited, to attend a party} \textcolor{PineGreen}{\selectlanguage{french}participer à une fête en tant qu'invité, se rendre à une fête/à une invitation}  
 ¶ \textcolor{darkblue}{\textbf{\ipa{hĩ˧bæ˧ tsʰɯ˧-ze˥ ! |}}} \zh{客人来了!} \textcolor{Sepia}{\selectlanguage{english}A guest has arrived!} \textcolor{PineGreen}{\selectlanguage{french}Un invité est arrivé!}  
 \zh{量词}: \textcolor{darkblue}{\textbf{\ipa{v̩˧}}} 
\lhead{\firstmark}
\rhead{\botmark}

\subsection{\hspace{-0.5cm} {\Large \textcolor{darkblue}{\textbf{\ipa{hĩ˧hĩ\#˥}}}}\hspace{0.5cm}[\kern2pt{\textcolor{darkblue}{\textbf{\ipa{hĩ˧hĩ˧}}}}\kern2pt]} \hypertarget{hi\string_~\string_Mhi\string_~\#\string_T1}{}
\markboth{\textcolor{darkblue}{\textbf{\ipa{hĩ˧hĩ\#˥}}}}{}
\textcolor{teal}{\zh{名词}} \hspace{4pt} \zh{声调类:} \#H.
\zh{外人。} \textcolor{Sepia}{\selectlanguage{english}Strangers, people outside the family.} \textcolor{PineGreen}{\selectlanguage{french}Les gens extérieurs à la famille (s'oppose à: “les gens de la famille”).}  \zh{量词}: \textcolor{darkblue}{\textbf{\ipa{v̩˧}}} 
\lhead{\firstmark}
\rhead{\botmark}

\subsection{\hspace{-0.5cm} {\Large \textcolor{darkblue}{\textbf{\ipa{hĩ˧-lɑ˩-kv̩˩-hĩ˩}}}}\hspace{0.5cm}[\kern2pt{\textcolor{darkblue}{\textbf{\ipa{hĩ˧lɑ˧kv̩˧hĩ˧}}}}\kern2pt]} \hypertarget{hi\string_~\string_M-lA\string_B-kv\string_=\string_B-hi\string_~\string_B1}{}
\markboth{\textcolor{darkblue}{\textbf{\ipa{hĩ˧-lɑ˩-kv̩˩-hĩ˩}}}}{}
\textcolor{teal}{\zh{名词}} \hspace{4pt} \zh{声调类:} -L--.
\zh{危险的人,仇人,敌人。} \textcolor{Sepia}{\selectlanguage{english}Dangerous person; enemy.} \textcolor{PineGreen}{\selectlanguage{french}Personne dangereuse, ennemi, bandit; littéralement: “personne susceptible de frapper les gens”.} 
\lhead{\firstmark}
\rhead{\botmark}

\subsection{\hspace{-0.5cm} {\Large \textcolor{darkblue}{\textbf{\ipa{hĩ˧mo˥}}}}\hspace{0.5cm}[\kern2pt{\textcolor{darkblue}{\textbf{\ipa{hĩ˧mo˥}}}}\kern2pt]} \hypertarget{hi\string_~\string_Mmo\string_T1}{}
\markboth{\textcolor{darkblue}{\textbf{\ipa{hĩ˧mo˥}}}}{}
\textcolor{teal}{\zh{名词}} \hspace{4pt} \zh{声调类:} H\#.
\zh{老人。} \textcolor{Sepia}{\selectlanguage{english}Elderly person.} \textcolor{PineGreen}{\selectlanguage{french}Personne âgée, vieillard, vieillarde.}  ¶ \textcolor{darkblue}{\textbf{\ipa{hĩ˧mo˥-hĩ˩}}} \zh{老人、老的人} \textcolor{Sepia}{\selectlanguage{english}\string_ \mytextsc{rel;} same meaning} \textcolor{PineGreen}{\selectlanguage{french}\string_ \mytextsc{rel;} même sens}  
 \zh{量词}: \textcolor{darkblue}{\textbf{\ipa{v̩˧}}} 
\lhead{\firstmark}
\rhead{\botmark}

\subsection{\hspace{-0.5cm} {\Large \textcolor{darkblue}{\textbf{\ipa{hĩ˧mo˩}}}}\hspace{0.5cm}[\kern2pt{\textcolor{darkblue}{\textbf{\ipa{hĩ˧mo˩}}}}\kern2pt]} \hypertarget{hi\string_~\string_Mmo\string_B1}{}
\markboth{\textcolor{darkblue}{\textbf{\ipa{hĩ˧mo˩}}}}{}
\textcolor{teal}{\zh{名词}} \hspace{4pt} \zh{声调类:} L\#.
\ding{202} \zh{尸体。} \textcolor{Sepia}{\selectlanguage{english}Corpse.} \textcolor{PineGreen}{\selectlanguage{french}Cadavre.}  ¶ \textcolor{darkblue}{\textbf{\ipa{hĩ˧mo˩-kʰɯ˩-di˩}}} \zh{棺材} \textcolor{Sepia}{\selectlanguage{english}coffin; literally 'thing (in which) to put a corpse'} \textcolor{PineGreen}{\selectlanguage{french}cercueil (périphrase: “objet (dans lequel) on met le cadavre”)}  
 \zh{量词}: \textcolor{darkblue}{\textbf{\ipa{mo˧}}} \ding{203} \zh{坟墓。} \textcolor{Sepia}{\selectlanguage{english}Tomb.} \textcolor{PineGreen}{\selectlanguage{french}Tombe, tombeau.} 
\lhead{\firstmark}
\rhead{\botmark}

\subsection{\hspace{-0.5cm} {\Large \textcolor{darkblue}{\textbf{\ipa{hĩ˧-tɕʰɯ\#˥}}}}\hspace{0.5cm}[\kern2pt{\textcolor{darkblue}{\textbf{\ipa{xxxx non-correspondance entre le nombre de morphèmes et le nombre de tons de morphèmes}}}}\kern2pt]} \hypertarget{hi\string_~\string_M-ts£\string_hM\#\string_T1}{}
\markboth{\textcolor{darkblue}{\textbf{\ipa{hĩ˧-tɕʰɯ\#˥}}}}{}
\textcolor{teal}{\zh{名词}} \hspace{4pt} \zh{声调类:} \#H.
\zh{同一辈的亲戚:兄弟姐妹、堂兄弟姐妹。} \textcolor{Sepia}{\selectlanguage{english}Family member belonging to the same generation: brother, sister, or cousin (on the mother's side).} \textcolor{PineGreen}{\selectlanguage{french}Membre de la famille de même génération: frère, sœur, cousin(e) (du côté maternel).}  ¶ \textcolor{darkblue}{\textbf{\ipa{hĩ˧-tɕʰɯ˧ - hĩ˧-ʈʂɤ\#˥}}} \zh{同一辈的亲戚:兄弟姐妹、堂兄弟姐妹} \textcolor{Sepia}{\selectlanguage{english}same meaning: the family members belonging to the same generation} \textcolor{PineGreen}{\selectlanguage{french}même sens: les gens de la même génération, dans la famille: frères, sœurs, mais aussi cousins du côté maternel}  
 ¶ \textcolor{darkblue}{\textbf{\ipa{ʈʂʰɯ˧ | njɤ˧ | hĩ˧ tɕʰɯ˧ ɲi˥!}}} \zh{他是跟我同一辈的亲戚!(=堂兄弟姐妹)} \textcolor{Sepia}{\selectlanguage{english}(S)he is someone of my generation! (=my cousin, my brother/sister...)} \textcolor{PineGreen}{\selectlanguage{french}C'est mon cousin/ma cousine/quelqu'un de ma fratrie!}  
 ¶ \textcolor{darkblue}{\textbf{\ipa{hĩ˧-tɕʰɯ˧ mɤ˧-ɲi˥ F | hĩ˧-tɕʰɯ˧ ʝi˧ tʰɑ˩-kv̩˩!}}} \zh{“不是亲戚,也可以变成亲戚!”这个俗语来形容朋友之间的深情,变成像家人之间的感情。} \textcolor{Sepia}{\selectlanguage{english}“Even if one is not (born) a family member, it is possible to become one!” A saying that refers to quasi-family links between friends, which amount to a form of adoption into the family circle.} \textcolor{PineGreen}{\selectlanguage{french}“Même si on n'est pas de la même famille (au départ), on peut le devenir!” Formule traditionnelle pour désigner les liens quasi-familiaux tissés entre amis, qui reviennent à des formes d'adoption au sein du cercle familial.}  
\zh{~【同义词】~} \hyperlink{}{\textcolor{darkblue}{\textbf{\ipa{hĩ˧-ʈʂɤ\#˥}}}}. 
\lhead{\firstmark}
\rhead{\botmark}

\subsection{\hspace{-0.5cm} {\Large \textcolor{darkblue}{\textbf{\ipa{hĩ˧-ʈʂɤ\#˥}}}}\hspace{0.5cm}[\kern2pt{\textcolor{darkblue}{\textbf{\ipa{xxxx non-correspondance entre le nombre de morphèmes et le nombre de tons de morphèmes}}}}\kern2pt]} \hypertarget{hi\string_~\string_M-t`s`7\#\string_T1}{}
\markboth{\textcolor{darkblue}{\textbf{\ipa{hĩ˧-ʈʂɤ\#˥}}}}{}
\textcolor{teal}{\zh{名词}} \hspace{4pt} \zh{声调类:} \#H.
\zh{同一辈的亲戚:兄弟姐妹、堂兄弟姐妹。} \textcolor{Sepia}{\selectlanguage{english}Family member belonging to the same generation: brother, sister, or cousin (on the mother's side).} \textcolor{PineGreen}{\selectlanguage{french}Membre de la famille de même génération: frère, sœur, cousin(e) (du côté maternel).}  ¶ \textcolor{darkblue}{\textbf{\ipa{hĩ˧-tɕʰɯ˧ - hĩ˧-ʈʂɤ\#˥}}} \zh{同一辈的亲戚:兄弟姐妹、堂兄弟姐妹} \textcolor{Sepia}{\selectlanguage{english}same meaning: the family members belonging to the same generation} \textcolor{PineGreen}{\selectlanguage{french}même sens: les gens de la même génération, dans la famille: frères, sœurs, mais aussi cousins du côté maternel}  
\zh{~【同义词】~} \hyperlink{}{\textcolor{darkblue}{\textbf{\ipa{hĩ˧-tɕʰɯ\#˥}}}}. 
\lhead{\firstmark}
\rhead{\botmark}

\subsection{\hspace{-0.5cm} {\Large \textcolor{darkblue}{\textbf{\ipa{hĩ˧˥}}} \textsubscript{1}}\hspace{0.5cm}[\kern2pt{\textcolor{darkblue}{\textbf{\ipa{hĩ˧˥}}}}\kern2pt]} \hypertarget{hi\string_~\string_M\string_T1}{}
\markboth{\textcolor{darkblue}{\textbf{\ipa{hĩ˧˥}}} \textsubscript{1}}{}
\textcolor{teal}{\zh{动词}} \hspace{4pt} \zh{声调类:} MH.
\zh{站(站立)。} \textcolor{Sepia}{\selectlanguage{english}To stand, to stand upright.} \textcolor{PineGreen}{\selectlanguage{french}Être debout, se tenir debout.} 
\lhead{\firstmark}
\rhead{\botmark}

\subsection{\hspace{-0.5cm} {\Large \textcolor{darkblue}{\textbf{\ipa{hĩ˧˥}}} \textsubscript{2}}\hspace{0.5cm}[\kern2pt{\textcolor{darkblue}{\textbf{\ipa{hĩ˧˥}}}}\kern2pt]} \hypertarget{hi\string_~\string_M\string_T2}{}
\markboth{\textcolor{darkblue}{\textbf{\ipa{hĩ˧˥}}} \textsubscript{2}}{}
\textcolor{teal}{\zh{动词}} \hspace{4pt} \zh{声调类:} MH.
\zh{应该。} \textcolor{Sepia}{\selectlanguage{english}To have to, to be necessary.} \textcolor{PineGreen}{\selectlanguage{french}Devoir, falloir.}  ¶ \textcolor{darkblue}{\textbf{\ipa{mɤ˧-hĩ˧}}} \zh{\mytextsc{否定}} \textcolor{Sepia}{\selectlanguage{english}\mytextsc{neg}} \textcolor{PineGreen}{\selectlanguage{french}\mytextsc{neg}}  
 ¶ \textcolor{darkblue}{\textbf{\ipa{no˧ | ʝi˧-hĩ˧˥!}}} \zh{你应该做!} \textcolor{Sepia}{\selectlanguage{english}You have to do it!} \textcolor{PineGreen}{\selectlanguage{french}c'est à toi de le faire! / il faut que tu le fasses!}  
 ¶ \textcolor{darkblue}{\textbf{\ipa{njɤ˧ | ʝi˧-mɤ˧-hĩ˧-hĩ˥ | (ɖɯ˧-pi˧) ʝi˧-ze˩! |}}} \zh{我做了一件不应该做的事!} \textcolor{Sepia}{\selectlanguage{english}I have done something I shouldn't have!} \textcolor{PineGreen}{\selectlanguage{french}j'ai fait quelque chose que j'aurais pas dû!}  
 ¶ \textcolor{darkblue}{\textbf{\ipa{no˧ | lo˧ ʝi˧-hĩ˧!}}} \zh{你应该工作啊!} \textcolor{Sepia}{\selectlanguage{english}You have to work! / You must work!} \textcolor{PineGreen}{\selectlanguage{french}Il faut que tu travailles!}  

\lhead{\firstmark}
\rhead{\botmark}

\subsection{\hspace{-0.5cm} {\Large \textcolor{darkblue}{\textbf{\ipa{ho˥}}} \textsubscript{1}}\hspace{0.5cm}[\kern2pt{\textcolor{darkblue}{\textbf{\ipa{ho˧˥}}}}\kern2pt]} \hypertarget{ho\string_T1}{}
\markboth{\textcolor{darkblue}{\textbf{\ipa{ho˥}}} \textsubscript{1}}{}
\textcolor{teal}{\zh{名词}} \hspace{4pt} \zh{声调类:} \#H.
\zh{雉。} \textcolor{Sepia}{\selectlanguage{english}Partridge.} \textcolor{PineGreen}{\selectlanguage{french}Faisan (utilisé aussi pour: cailles, et certaines poules sauvages).}  ¶ \textcolor{darkblue}{\textbf{\ipa{ho˧ tʰv̩˧-mi˧˥ / ho˧ tʰv̩˧-mi˥\#}}} \zh{这只雉} \textcolor{Sepia}{\selectlanguage{english}\mytextsc{n}+\mytextsc{dem}+\mytextsc{clf}} \textcolor{PineGreen}{\selectlanguage{french}\mytextsc{n}+\mytextsc{dem}+\mytextsc{clf}}  
 \zh{量词}: \textcolor{darkblue}{\textbf{\ipa{mi˩}}} 
\lhead{\firstmark}
\rhead{\botmark}

\subsection{\hspace{-0.5cm} {\Large \textcolor{darkblue}{\textbf{\ipa{ho˥}}} \textsubscript{2}}\hspace{0.5cm}[\kern2pt{\textcolor{darkblue}{\textbf{\ipa{ho˥}}}}\kern2pt]} \hypertarget{ho\string_T2}{}
\markboth{\textcolor{darkblue}{\textbf{\ipa{ho˥}}} \textsubscript{2}}{}
\textcolor{teal}{\zh{名词}} \hspace{4pt} \zh{声调类:} \#H.
\zh{粥。} \textcolor{Sepia}{\selectlanguage{english}Porridge, gruel, congee.} \textcolor{PineGreen}{\selectlanguage{french}Gruau.}  ¶ \textcolor{darkblue}{\textbf{\ipa{ho˧ ʈʰɯ˧˥}}} \zh{喝粥} \textcolor{Sepia}{\selectlanguage{english}to drink gruel} \textcolor{PineGreen}{\selectlanguage{french}boire du gruau}  

\lhead{\firstmark}
\rhead{\botmark}

\subsection{\hspace{-0.5cm} {\Large \textcolor{darkblue}{\textbf{\ipa{ho˧ɕjæ˩}}}}\hspace{0.5cm}[\kern2pt{\textcolor{darkblue}{\textbf{\ipa{ho˩ɕjæ˥}}}}\kern2pt]} \hypertarget{ho\string_Ms£j\{\string_B1}{}
\markboth{\textcolor{darkblue}{\textbf{\ipa{ho˧ɕjæ˩}}}}{}
\textcolor{teal}{\zh{名词}} \hspace{4pt} \zh{声调类:} LM.
\zh{火绳,导火索。} \textcolor{Sepia}{\selectlanguage{english}Cord to which fire is put in order to shoot.} \textcolor{PineGreen}{\selectlanguage{french}Mèche.} \zh{当地汉语方言:}\zh{火线。} \zh{【借词】} \zh{火线}

\lhead{\firstmark}
\rhead{\botmark}

\subsection{\hspace{-0.5cm} {\Large \textcolor{darkblue}{\textbf{\ipa{ho˧di˧}}}}\hspace{0.5cm}[\kern2pt{\textcolor{darkblue}{\textbf{\ipa{ho˩di˥}}}}\kern2pt]} \hypertarget{ho\string_Mdi\string_M1}{}
\markboth{\textcolor{darkblue}{\textbf{\ipa{ho˧di˧}}}}{}
\textcolor{teal}{\zh{名词}} \hspace{4pt} \zh{声调类:} M.
\zh{四川(盐源、盐边、西昌……)。} \textcolor{Sepia}{\selectlanguage{english}Chinese (Han) areas of Sichuan: Yanyuan, Yanbian, Xichang...} \textcolor{PineGreen}{\selectlanguage{french}Régions chinoises (Han) du Sichuan: Yanyuan, Yanbian, Xichang...} 
\lhead{\firstmark}
\rhead{\botmark}

\subsection{\hspace{-0.5cm} {\Large \textcolor{darkblue}{\textbf{\ipa{ho˧dʑɯ˧tɤ˥ɻ̍˩}}}}\hspace{0.5cm}[\kern2pt{\textcolor{darkblue}{\textbf{\ipa{ho˧dʑɯ˧tɤ˧ɻ̍˧˥}}}}\kern2pt]} \hypertarget{ho\string_Mdz£M\string_Mt7\string_Tr£`̍\string_B1}{}
\markboth{\textcolor{darkblue}{\textbf{\ipa{ho˧dʑɯ˧tɤ˥ɻ̍˩}}}}{}
\textcolor{teal}{\zh{名词}} \hspace{4pt} \zh{声调类:} \#H-.
\zh{浆糊,浆子。} \textcolor{Sepia}{\selectlanguage{english}Paste; starch.} \textcolor{PineGreen}{\selectlanguage{french}Pâte, colle à base de farine, liquide visqueux.} \zh{~【参考】~} \hyperlink{}{\textcolor{darkblue}{\textbf{\ipa{ho˧dʑɯ˧˥}}}} 
\lhead{\firstmark}
\rhead{\botmark}

\subsection{\hspace{-0.5cm} {\Large \textcolor{darkblue}{\textbf{\ipa{ho˧dʑɯ˧˥}}}}\hspace{0.5cm}[\kern2pt{\textcolor{darkblue}{\textbf{\ipa{ho˩dʑɯ˩˥}}}}\kern2pt]} \hypertarget{ho\string_Mdz£M\string_M\string_T1}{}
\markboth{\textcolor{darkblue}{\textbf{\ipa{ho˧dʑɯ˧˥}}}}{}
\textcolor{teal}{\zh{名词}} \hspace{4pt} \zh{声调类:} MH\#.
\zh{浆糊,浆子。} \textcolor{Sepia}{\selectlanguage{english}Paste; starch.} \textcolor{PineGreen}{\selectlanguage{french}Pâte, colle à base de farine, liquide visqueux.} \zh{~【参考】~} \hyperlink{}{\textcolor{darkblue}{\textbf{\ipa{ho˧dʑɯ˧tɤ˥ɻ̍˩}}}} 
\lhead{\firstmark}
\rhead{\botmark}

\subsection{\hspace{-0.5cm} {\Large \textcolor{darkblue}{\textbf{\ipa{ho˧ko˧}}}}\hspace{0.5cm}[\kern2pt{\textcolor{darkblue}{\textbf{\ipa{xxxx non-correspondance entre le nombre de morphèmes et le nombre de tons de morphèmes}}}}\kern2pt]} \hypertarget{ho\string_Mko\string_M1}{}
\markboth{\textcolor{darkblue}{\textbf{\ipa{ho˧ko˧}}}}{}
\textcolor{teal}{\zh{名词}} \hspace{4pt} \zh{声调类:} M.
\zh{火锅(汉语借词)。} \textcolor{Sepia}{\selectlanguage{english}Cooking pot for making hotpot; traditionally made of copper, with a hole in the centre.} \textcolor{PineGreen}{\selectlanguage{french}Grand récipient pour faire la fondue mongole.}  \zh{【借词】} \zh{火锅}
 ¶ \textcolor{darkblue}{\textbf{\ipa{æ̃˧-ho˧ko˥}}} \zh{铜火锅} \textcolor{Sepia}{\selectlanguage{english}copper pot for hotpot} \textcolor{PineGreen}{\selectlanguage{french}récipient pour fondue en cuivre}  

\lhead{\firstmark}
\rhead{\botmark}

\subsection{\hspace{-0.5cm} {\Large \textcolor{darkblue}{\textbf{\ipa{ho˧mi\#˥}}}}\hspace{0.5cm}[\kern2pt{\textcolor{darkblue}{\textbf{\ipa{ho˩mi˥}}}}\kern2pt]} \hypertarget{ho\string_Mmi\#\string_T1}{}
\markboth{\textcolor{darkblue}{\textbf{\ipa{ho˧mi\#˥}}}}{}
\textcolor{teal}{\zh{名词}} \hspace{4pt} \zh{声调类:} \#H.
\zh{母雉。} \textcolor{Sepia}{\selectlanguage{english}Female pheasant.} \textcolor{PineGreen}{\selectlanguage{french}Faisan femelle.}  ¶ \textcolor{darkblue}{\textbf{\ipa{ho˧mi˧ tʰv̩˧-mi˧˥ / ho˧mi˧ tʰv̩˧-mi˥\#}}} \zh{这个母雉} \textcolor{Sepia}{\selectlanguage{english}\mytextsc{n}+\mytextsc{dem}+\mytextsc{clf}} \textcolor{PineGreen}{\selectlanguage{french}\mytextsc{n}+\mytextsc{dem}+\mytextsc{clf}}  
 ¶ \textcolor{darkblue}{\textbf{\ipa{ho˧mi˧-ho˧pʰv̩˥ / ho˧mi˧-ho˥pʰv̩˩}}} \zh{母雉与公雉} \textcolor{Sepia}{\selectlanguage{english}female and male pheasant} \textcolor{PineGreen}{\selectlanguage{french}faisan femelle et faisan mâle}  
 \zh{量词}: \textcolor{darkblue}{\textbf{\ipa{mi˩}}} 
\lhead{\firstmark}
\rhead{\botmark}

\subsection{\hspace{-0.5cm} {\Large \textcolor{darkblue}{\textbf{\ipa{ho˧pʰv̩\#˥}}}}\hspace{0.5cm}[\kern2pt{\textcolor{darkblue}{\textbf{\ipa{ho˧pʰv̩˧}}}}\kern2pt]} \hypertarget{ho\string_Mp\string_hv\string_=\#\string_T1}{}
\markboth{\textcolor{darkblue}{\textbf{\ipa{ho˧pʰv̩\#˥}}}}{}
\textcolor{teal}{\zh{名词}} \hspace{4pt} \zh{声调类:} \#H.
\zh{公雉。} \textcolor{Sepia}{\selectlanguage{english}Male pheasant.} \textcolor{PineGreen}{\selectlanguage{french}Faisan mâle.}  ¶ \textcolor{darkblue}{\textbf{\ipa{ho˧pʰv̩˧ tʰv̩˧-mi˧˥ / ho˧pʰv̩˧ tʰv̩˧-mi˥\#}}} \zh{这只公雉} \textcolor{Sepia}{\selectlanguage{english}\mytextsc{n}+\mytextsc{dem}+\mytextsc{clf}} \textcolor{PineGreen}{\selectlanguage{french}\mytextsc{n}+\mytextsc{dem}+\mytextsc{clf}}  
 \zh{量词}: \textcolor{darkblue}{\textbf{\ipa{mi˩}}} 
\lhead{\firstmark}
\rhead{\botmark}

\subsection{\hspace{-0.5cm} {\Large \textcolor{darkblue}{\textbf{\ipa{ho˧ʈʂɯ˧}}}}\hspace{0.5cm}[\kern2pt{\textcolor{darkblue}{\textbf{\ipa{ho˩ʈʂɯ˧˥}}}}\kern2pt]} \hypertarget{ho\string_Mt`s`M\string_M1}{}
\markboth{\textcolor{darkblue}{\textbf{\ipa{ho˧ʈʂɯ˧}}}}{}
\textcolor{teal}{\zh{名词}} \hspace{4pt} \zh{声调类:} M.
\zh{蒿(汉语借词:蒿枝)。} \textcolor{Sepia}{\selectlanguage{english}Mugwort, wormwood, \textit{Artemisia vulgaris}.} \textcolor{PineGreen}{\selectlanguage{french}Armoise, \textit{Artemisia vulgaris}.} \zh{当地汉语方言:}\zh{蒿草、蒿枝。} \zh{【借词】} \zh{蒿枝}
\zh{~【参考】~} \hyperlink{}{\textcolor{darkblue}{\textbf{\ipa{tɕɤ˧qʰɑ\#˥}}}} 
\lhead{\firstmark}
\rhead{\botmark}

\subsection{\hspace{-0.5cm} {\Large \textcolor{darkblue}{\textbf{\ipa{ho˧zo\#˥}}}}\hspace{0.5cm}[\kern2pt{\textcolor{darkblue}{\textbf{\ipa{ho˧zo˧}}}}\kern2pt]} \hypertarget{ho\string_Mzo\#\string_T1}{}
\markboth{\textcolor{darkblue}{\textbf{\ipa{ho˧zo\#˥}}}}{}
\textcolor{teal}{\zh{名词}} \hspace{4pt} \zh{声调类:} \#H.
\zh{小雉。} \textcolor{Sepia}{\selectlanguage{english}Baby pheasant.} \textcolor{PineGreen}{\selectlanguage{french}Bébé faisan.}  ¶ \textcolor{darkblue}{\textbf{\ipa{ho˧zo˧ tʰv̩˧-ɭɯ\#˥}}} \zh{这只小雉} \textcolor{Sepia}{\selectlanguage{english}\mytextsc{n}+\mytextsc{dem}+\mytextsc{clf}} \textcolor{PineGreen}{\selectlanguage{french}\mytextsc{n}+\mytextsc{dem}+\mytextsc{clf}}  
 \zh{量词}: \textcolor{darkblue}{\textbf{\ipa{ɭɯ˧}}} 
\lhead{\firstmark}
\rhead{\botmark}

\subsection{\hspace{-0.5cm} {\Large \textcolor{darkblue}{\textbf{\ipa{‑ho˩}}}}\hspace{0.5cm}[\kern2pt{\textcolor{darkblue}{\textbf{\ipa{ho˩˥}}}}\kern2pt]} \hypertarget{‑ho\string_B1}{}
\markboth{\textcolor{darkblue}{\textbf{\ipa{‑ho˩}}}}{}
\textcolor{teal}{\zh{后缀}} \hspace{4pt} \zh{声调类:} L.
\zh{\mytextsc{未来}\string_愿望。} \textcolor{Sepia}{\selectlanguage{english}Future / desiderative / conjecture.} \textcolor{PineGreen}{\selectlanguage{french}Future / desiderative / conjecture.}  ¶ \textcolor{darkblue}{\textbf{\ipa{hi˩gi˩ ə˥-ho˩? - hi˩ gi˩ ho˥!}}} \zh{要下雨了吗? - 是的,要下雨了!} \textcolor{Sepia}{\selectlanguage{english}Is it going to rain? - Yes, it is going to rain!} \textcolor{PineGreen}{\selectlanguage{french}Va-t-il pleuvoir ? - Oui!}  
 ¶ \textcolor{darkblue}{\textbf{\ipa{ʈʂʰɯ˧ | so˧ɲi˥ | le˧-jo˩ ho˩-hĩ˩ | ə˩-ɲi˩˥ ? - ʈʂʰɯ˧ | so˧ɲi˥ | le˧-jo˩-ho˩-ɲi˩-mæ˩.}}} \zh{他明天要来买? - (是的,)他明天会来的。(回答表示:比较肯定。)} \textcolor{Sepia}{\selectlanguage{english}Is he going to come tomorrow? - (Yes,) he will come tomorrow.} \textcolor{PineGreen}{\selectlanguage{french}Viendra-t-il demain ? - (Oui) je pense qu’il viendra demain. (Qd on est presque sûr)}  
 ¶ \textcolor{darkblue}{\textbf{\ipa{so˧ɲi˥ | le˧-ɬi˥ | mɤ˧-ho˥!}}} \zh{明天就不休息了!} \textcolor{Sepia}{\selectlanguage{english}Tomorrow, they won't be on holiday anymore! (Context: on a Sunday, talking about a kindergarten that has been on holiday during the previous week.)} \textcolor{PineGreen}{\selectlanguage{french}Demain, ils ne seront plus en vacances! (contexte: on parle d'une crèche qui a été en vacances la semaine précédente à l'occasion de la Fête des enseignants)}  
 ¶ \textcolor{darkblue}{\textbf{\ipa{tɕʰi˧ ə˧-ho˩?}}} \zh{要卖吗?/ 会卖吗?} \textcolor{Sepia}{\selectlanguage{english}Is (she/he) going to sell?} \textcolor{PineGreen}{\selectlanguage{french}va(-t-il) vendre?}  
 ¶ \textcolor{darkblue}{\textbf{\ipa{hwæ˧ ə˧-ho˥?}}} \zh{要买吗?/ 会买马?} \textcolor{Sepia}{\selectlanguage{english}Is (she/he) going to buy?} \textcolor{PineGreen}{\selectlanguage{french}va(-t-il) acheter?}  

\lhead{\firstmark}
\rhead{\botmark}

\subsection{\hspace{-0.5cm} {\Large \textcolor{darkblue}{\textbf{\ipa{ho˩\textsubscript{a}}}}}\hspace{0.5cm}[\kern2pt{\textcolor{darkblue}{\textbf{\ipa{ho˥}}}}\kern2pt]} \hypertarget{ho\string_Ba1}{}
\markboth{\textcolor{darkblue}{\textbf{\ipa{ho˩\textsubscript{a}}}}}{}
\textcolor{teal}{\zh{形容词}} \hspace{4pt} \zh{声调类:} L\textsubscript{a}.
\zh{准确,合适。} \textcolor{Sepia}{\selectlanguage{english}Correct; suitable, appropriate.} \textcolor{PineGreen}{\selectlanguage{french}Exact, correct; adapté, convenable.}  \zh{【借词】}chinois ancien: \zh{合} ?
 ¶ \textcolor{darkblue}{\textbf{\ipa{mɤ˧-ho˩}}} \zh{不合适,不准,不对} \textcolor{Sepia}{\selectlanguage{english}\mytextsc{neg}} \textcolor{PineGreen}{\selectlanguage{french}\mytextsc{neg}: faux, erroné, inapproprié}  
 ¶ \textcolor{darkblue}{\textbf{\ipa{ho˩-ze˧!}}} \zh{对了! / 准确!} \textcolor{Sepia}{\selectlanguage{english}\mytextsc{pfv}} \textcolor{PineGreen}{\selectlanguage{french}\mytextsc{pfv}}  
 ¶ \textcolor{darkblue}{\textbf{\ipa{ho˩-hĩ˩˥}}} \zh{准确的} \textcolor{Sepia}{\selectlanguage{english}\mytextsc{rel}/\mytextsc{nmlz}} \textcolor{PineGreen}{\selectlanguage{french}\mytextsc{rel}/\mytextsc{nmlz}}  

\lhead{\firstmark}
\rhead{\botmark}

\subsection{\hspace{-0.5cm} {\Large \textcolor{darkblue}{\textbf{\ipa{ho˩ɕjæ˧}}}}\hspace{0.5cm}[\kern2pt{\textcolor{darkblue}{\textbf{\ipa{ho˩ɕjæ˩˥}}}}\kern2pt]} \hypertarget{ho\string_Bs£j\{\string_M1}{}
\markboth{\textcolor{darkblue}{\textbf{\ipa{ho˩ɕjæ˧}}}}{}
\textcolor{teal}{\zh{名词}} \hspace{4pt} \zh{声调类:} LM.
\zh{藿香。} \textcolor{Sepia}{\selectlanguage{english}Wrinkled giant hyssop, \textit{Elshotzia sp.}.} \textcolor{PineGreen}{\selectlanguage{french}Hysope, \textit{Elshotzia sp.}.}  \zh{【借词】} \zh{藿香}

\lhead{\firstmark}
\rhead{\botmark}

\subsection{\hspace{-0.5cm} {\Large \textcolor{darkblue}{\textbf{\ipa{ho˩dʑɯ˩}}}}\hspace{0.5cm}[\kern2pt{\textcolor{darkblue}{\textbf{\ipa{ho˧dʑɯ˧}}}}\kern2pt]} \hypertarget{ho\string_Bdz£M\string_B1}{}
\markboth{\textcolor{darkblue}{\textbf{\ipa{ho˩dʑɯ˩}}}}{}
\textcolor{teal}{\zh{形容词}} \hspace{4pt} \zh{声调类:} L.
\zh{穷苦、凋敝、寒苦、竭蹶、穷乏。} \textcolor{Sepia}{\selectlanguage{english}Destitute, impoverished, down and out.} \textcolor{PineGreen}{\selectlanguage{french}Indigent.}  ¶ \textcolor{darkblue}{\textbf{\ipa{ho˩dʑɯ˩-ze˥}}} \zh{变穷苦了} \textcolor{Sepia}{\selectlanguage{english}\mytextsc{pfv}} \textcolor{PineGreen}{\selectlanguage{french}\mytextsc{pfv}: qui se retrouve à la rue, qui devient démuni}  

\lhead{\firstmark}
\rhead{\botmark}

\subsection{\hspace{-0.5cm} {\Large \textcolor{darkblue}{\textbf{\ipa{ho˩lo˧pv̩˥}}}}\hspace{0.5cm}[\kern2pt{\textcolor{darkblue}{\textbf{\ipa{ho˧lo˧pv̩˧}}}}\kern2pt]} \hypertarget{ho\string_Blo\string_Mpv\string_=\string_T1}{}
\markboth{\textcolor{darkblue}{\textbf{\ipa{ho˩lo˧pv̩˥}}}}{}
\textcolor{teal}{\zh{名词}} \hspace{4pt} \zh{声调类:} LM+H\#.
\zh{胡萝卜。} \textcolor{Sepia}{\selectlanguage{english}Carrot.} \textcolor{PineGreen}{\selectlanguage{french}Carotte.}  \zh{【借词】} \zh{胡萝卜}
 \zh{量词}: \textcolor{darkblue}{\textbf{\ipa{ɭɯ˧}}} 
\lhead{\firstmark}
\rhead{\botmark}

\subsection{\hspace{-0.5cm} {\Large \textcolor{darkblue}{\textbf{\ipa{ho˧˥}}}}\hspace{0.5cm}[\kern2pt{\textcolor{darkblue}{\textbf{\ipa{ho˧˥}}}}\kern2pt]} \hypertarget{ho\string_M\string_T1}{}
\markboth{\textcolor{darkblue}{\textbf{\ipa{ho˧˥}}}}{}
\textcolor{teal}{\zh{动词}} \hspace{4pt} \zh{声调类:} MH.
\zh{小口地喝。} \textcolor{Sepia}{\selectlanguage{english}To sip: to drink by small mouthfuls.} \textcolor{PineGreen}{\selectlanguage{french}Siroter, boire à petites gorgées.}  ¶ \textcolor{darkblue}{\textbf{\ipa{ʐɯ˧ ho˧˥}}} \zh{小口地喝酒} \textcolor{Sepia}{\selectlanguage{english}to sip wine} \textcolor{PineGreen}{\selectlanguage{french}siroter du vin}  
 ¶ \textcolor{darkblue}{\textbf{\ipa{ʐɯ˧ ho˧\textasciitilde{}ho˥}}} \zh{小口地喝酒} \textcolor{Sepia}{\selectlanguage{english}to sip wine} \textcolor{PineGreen}{\selectlanguage{french}siroter du vin}  
 ¶ \textcolor{darkblue}{\textbf{\ipa{ʐɯ˧ | ɖɯ˧-ho˧\textasciitilde{}ho˥}}} \zh{喝一小口酒} \textcolor{Sepia}{\selectlanguage{english}to sip wine} \textcolor{PineGreen}{\selectlanguage{french}siroter du vin}  

\lhead{\firstmark}
\rhead{\botmark}

\subsection{\hspace{-0.5cm} {\Large \textcolor{darkblue}{\textbf{\ipa{hõ˧}}}}\hspace{0.5cm}[\kern2pt{\textcolor{darkblue}{\textbf{\ipa{hõ˥}}}}\kern2pt]} \hypertarget{ho\string_~\string_M1}{}
\markboth{\textcolor{darkblue}{\textbf{\ipa{hõ˧}}}}{}
\textcolor{teal}{\zh{动词}} \hspace{4pt} \zh{声调类:} M intrans.
\ding{202} \zh{走(离开),\mytextsc{命令式。}} \textcolor{Sepia}{\selectlanguage{english}To go away (imperative).} \textcolor{PineGreen}{\selectlanguage{french}Partir (impératif).}  ¶ \textcolor{darkblue}{\textbf{\ipa{no˧ hõ˧!}}} \zh{你走吧!} \textcolor{Sepia}{\selectlanguage{english}Go!} \textcolor{PineGreen}{\selectlanguage{french}vas-y!}  
 ¶ \textcolor{darkblue}{\textbf{\ipa{no˧ | le˧-hõ˧!}}} \zh{你走吧!} \textcolor{Sepia}{\selectlanguage{english}Go!} \textcolor{PineGreen}{\selectlanguage{french}Marche!/Vas-y!}  
 ¶ \textcolor{darkblue}{\textbf{\ipa{ə˧-ze˧\textasciitilde{}ze˥ hõ˩! / ə˧-dzɤ˥ | le˧-hõ˧!}}} \zh{慢走!} \textcolor{Sepia}{\selectlanguage{english}Walk slowly! / Take your time on the road! / Have a quiet and pleasant journey! (Polite salutation to someone who is leaving.)} \textcolor{PineGreen}{\selectlanguage{french}salutation respectueuse à quelqu'un qui se met en chemin: Prends ton temps en chemin!}  
 ¶ \textcolor{darkblue}{\textbf{\ipa{ɑ˩pʰo˩ hõ˩˥!}}} \zh{出去!走开!滚出去!} \textcolor{Sepia}{\selectlanguage{english}Get out!} \textcolor{PineGreen}{\selectlanguage{french}Dehors! / Dégage!}  
\ding{203} \zh{\mytextsc{命令式。}} \textcolor{Sepia}{\selectlanguage{english}Imperative.} \textcolor{PineGreen}{\selectlanguage{french}Impératif.}  ¶ \textcolor{darkblue}{\textbf{\ipa{no˧ | dzɯ˧-hõ˧!}}} \zh{你吃吧!} \textcolor{Sepia}{\selectlanguage{english}Eat!} \textcolor{PineGreen}{\selectlanguage{french}Mange!}  

\lhead{\firstmark}
\rhead{\botmark}

\subsection{\hspace{-0.5cm} {\Large \textcolor{darkblue}{\textbf{\ipa{hõ˧-ɬi˧mi\#˥}}}}\hspace{0.5cm}[\kern2pt{\textcolor{darkblue}{\textbf{\ipa{xxxx non-correspondance entre le nombre de morphèmes et le nombre de tons de morphèmes}}}}\kern2pt]} \hypertarget{ho\string_~\string_M-Ki\string_Mmi\#\string_T1}{}
\markboth{\textcolor{darkblue}{\textbf{\ipa{hõ˧-ɬi˧mi\#˥}}}}{}
\textcolor{teal}{\zh{名词}} \hspace{4pt} \zh{声调类:} \#H.
\zh{八月。} \textcolor{Sepia}{\selectlanguage{english}8th month.} \textcolor{PineGreen}{\selectlanguage{french}8e mois.} 
\lhead{\firstmark}
\rhead{\botmark}

\subsection{\hspace{-0.5cm} {\Large \textcolor{darkblue}{\textbf{\ipa{hõ˩tsʰi˧˥}}}}\hspace{0.5cm}[\kern2pt{\textcolor{darkblue}{\textbf{\ipa{hõ˧tsʰi˧}}}}\kern2pt]} \hypertarget{ho\string_~\string_Bts\string_hi\string_M\string_T1}{}
\markboth{\textcolor{darkblue}{\textbf{\ipa{hõ˩tsʰi˧˥}}}}{}
\textcolor{teal}{\zh{数词}} \hspace{4pt} \zh{声调类:} LM+MH\#.
\zh{80。} \textcolor{Sepia}{\selectlanguage{english}80.} \textcolor{PineGreen}{\selectlanguage{french}80.} 
\lhead{\firstmark}
\rhead{\botmark}

\subsection{\hspace{-0.5cm} {\Large \textcolor{darkblue}{\textbf{\ipa{hõ˧˥}}}}\hspace{0.5cm}[\kern2pt{\textcolor{darkblue}{\textbf{\ipa{hõ˥}}}}\kern2pt]} \hypertarget{ho\string_~\string_M\string_T1}{}
\markboth{\textcolor{darkblue}{\textbf{\ipa{hõ˧˥}}}}{}
\textcolor{teal}{\zh{数词}} \hspace{4pt} \zh{声调类:} MH.
\zh{8。} \textcolor{Sepia}{\selectlanguage{english}8.} \textcolor{PineGreen}{\selectlanguage{french}8.} 
\lhead{\firstmark}
\rhead{\botmark}

\subsection{\hspace{-0.5cm} {\Large \textcolor{darkblue}{\textbf{\ipa{hu˥}}}}\hspace{0.5cm}[\kern2pt{\textcolor{darkblue}{\textbf{\ipa{hu˧˥}}}}\kern2pt]} \hypertarget{hu\string_T1}{}
\markboth{\textcolor{darkblue}{\textbf{\ipa{hu˥}}}}{}
\textcolor{teal}{\zh{动词}} \hspace{4pt} \zh{声调类:} H.
\zh{等候。} \textcolor{Sepia}{\selectlanguage{english}To wait.} \textcolor{PineGreen}{\selectlanguage{french}Attendre.}  ¶ \textcolor{darkblue}{\textbf{\ipa{le˧-hu˥-ze˩}}} \zh{等了} \textcolor{Sepia}{\selectlanguage{english}\mytextsc{accomp} \string_ \mytextsc{pfv}} \textcolor{PineGreen}{\selectlanguage{french}\mytextsc{accomp} \string_ \mytextsc{pfv}}  
 ¶ \textcolor{darkblue}{\textbf{\ipa{ɖɯ˧-hu˥ / ɖɯ˧-hu˧-ɻ̍˥}}} \zh{等一下 / 请等一下!} \textcolor{Sepia}{\selectlanguage{english}to wait a while / Wait a while!} \textcolor{PineGreen}{\selectlanguage{french}attendre un peu / Attends un peu!}  
 ¶ \textcolor{darkblue}{\textbf{\ipa{hĩ˧ hu˧}}} \zh{等人} \textcolor{Sepia}{\selectlanguage{english}to wait for someone} \textcolor{PineGreen}{\selectlanguage{french}attendre quelqu'un}  

\lhead{\firstmark}
\rhead{\botmark}

\subsection{\hspace{-0.5cm} {\Large \textcolor{darkblue}{\textbf{\ipa{hu˧mi˥\$}}}}\hspace{0.5cm}[\kern2pt{\textcolor{darkblue}{\textbf{\ipa{hu˧mi˧}}}}\kern2pt]} \hypertarget{hu\string_Mmi\string_T\$1}{}
\markboth{\textcolor{darkblue}{\textbf{\ipa{hu˧mi˥\$}}}}{}
\textcolor{teal}{\zh{名词}} \hspace{4pt} \zh{声调类:} H\$.
\zh{胃。} \textcolor{Sepia}{\selectlanguage{english}Stomach.} \textcolor{PineGreen}{\selectlanguage{french}Estomac.}  \zh{量词}: \textcolor{darkblue}{\textbf{\ipa{ɭɯ˧}}} 
\lhead{\firstmark}
\rhead{\botmark}

\subsection{\hspace{-0.5cm} {\Large \textcolor{darkblue}{\textbf{\ipa{hu˧˥}}} \textsubscript{1}}\hspace{0.5cm}[\kern2pt{\textcolor{darkblue}{\textbf{\ipa{hu˥}}}}\kern2pt]} \hypertarget{hu\string_M\string_T1}{}
\markboth{\textcolor{darkblue}{\textbf{\ipa{hu˧˥}}} \textsubscript{1}}{}
\textcolor{teal}{\zh{动词}} \hspace{4pt} \zh{声调类:} MH.
\zh{想念。} \textcolor{Sepia}{\selectlanguage{english}To miss, to long for, to have the nostalgia of.} \textcolor{PineGreen}{\selectlanguage{french}Avoir la nostalgie de.}  ¶ \textcolor{darkblue}{\textbf{\ipa{ə˧mi˧ hu˧˥}}} \zh{想念母亲} \textcolor{Sepia}{\selectlanguage{english}to miss (one's) mother} \textcolor{PineGreen}{\selectlanguage{french}avoir la nostalgie de sa mère}  

\lhead{\firstmark}
\rhead{\botmark}

\subsection{\hspace{-0.5cm} {\Large \textcolor{darkblue}{\textbf{\ipa{hu˧˥}}} \textsubscript{2}}\hspace{0.5cm}[\kern2pt{\textcolor{darkblue}{\textbf{\ipa{hu˧˥}}}}\kern2pt]} \hypertarget{hu\string_M\string_T2}{}
\markboth{\textcolor{darkblue}{\textbf{\ipa{hu˧˥}}} \textsubscript{2}}{}
\textcolor{teal}{\zh{名词}} \hspace{4pt} \zh{声调类:} MH.
\zh{内脏:胃、肠子等。} \textcolor{Sepia}{\selectlanguage{english}Entrails.} \textcolor{PineGreen}{\selectlanguage{french}Estomac au sens large: entrailles, tripes (tout le système digestif).}  \zh{量词}: \textcolor{darkblue}{\textbf{\ipa{ɭɯ˧}}} 
\lhead{\firstmark}
\rhead{\botmark}

\subsection{\hspace{-0.5cm} {\Large \textcolor{darkblue}{\textbf{\ipa{hɯ˧}}}}\hspace{0.5cm}[\kern2pt{\textcolor{darkblue}{\textbf{\ipa{hɯ˥}}}}\kern2pt]} \hypertarget{hM\string_M1}{}
\markboth{\textcolor{darkblue}{\textbf{\ipa{hɯ˧}}}}{}
\textcolor{teal}{\zh{动词}} \hspace{4pt} \zh{声调类:} M\textsubscript{c}.
\zh{走(过去式)。} \textcolor{Sepia}{\selectlanguage{english}To go, past form.} \textcolor{PineGreen}{\selectlanguage{french}Aller, forme passée.}  ¶ \textcolor{darkblue}{\textbf{\ipa{(ki˧zo˧) | lo˧ ʝi˧-hɯ˧(-ze˩)!}}} \zh{给若(人名)干活去了!} \textcolor{Sepia}{\selectlanguage{english}Kizo has gone to work!} \textcolor{PineGreen}{\selectlanguage{french}Kizo est partie travailler!}  
 ¶ \textcolor{darkblue}{\textbf{\ipa{le˧-hɯ˩-hĩ˩ hĩ˩}}} \zh{委婉语:‘走了的人’=去世了的人} \textcolor{Sepia}{\selectlanguage{english}euphemism for 'deceased person': literally 'person who has gone'} \textcolor{PineGreen}{\selectlanguage{french}personne décédée; littéralement "personne qui est partie}  

\lhead{\firstmark}
\rhead{\botmark}

\subsection{\hspace{-0.5cm} {\Large \textcolor{darkblue}{\textbf{\ipa{hɯ˧\textsubscript{b}}}}}\hspace{0.5cm}[\kern2pt{\textcolor{darkblue}{\textbf{\ipa{hɯ˥}}}}\kern2pt]} \hypertarget{hM\string_Mb1}{}
\markboth{\textcolor{darkblue}{\textbf{\ipa{hɯ˧\textsubscript{b}}}}}{}
\textcolor{teal}{\zh{量词}} \hspace{4pt} \zh{声调类:} M\textsubscript{b}.
\zh{量词:一点。} \textcolor{Sepia}{\selectlanguage{english}A few, some, a little.} \textcolor{PineGreen}{\selectlanguage{french}Quelques-uns, un peu, une petite quantité de.}  ¶ \textcolor{darkblue}{\textbf{\ipa{ʈʂʰæ˧ɣɯ˧ ɖɯ˧-hɯ˧}}} \zh{一些药} \textcolor{Sepia}{\selectlanguage{english}some medicines, a few medicines} \textcolor{PineGreen}{\selectlanguage{french}quelques médicaments}  

\lhead{\firstmark}
\rhead{\botmark}

\subsection{\hspace{-0.5cm} {\Large \textcolor{darkblue}{\textbf{\ipa{hṽ˥}}}}\hspace{0.5cm}[\kern2pt{\textcolor{darkblue}{\textbf{\ipa{hṽ˥}}}}\kern2pt]} \hypertarget{hv\string_~\string_T1}{}
\markboth{\textcolor{darkblue}{\textbf{\ipa{hṽ˥}}}}{}
\textcolor{teal}{\zh{动词}} \hspace{4pt} \zh{声调类:} H.
\zh{炒(肉、菜)。} \textcolor{Sepia}{\selectlanguage{english}To stir-fry.} \textcolor{PineGreen}{\selectlanguage{french}Frire (viande, légumes...), cuire au wok.}  ¶ \textcolor{darkblue}{\textbf{\ipa{hṽ˧\textasciitilde{}hṽ˧}}} \zh{重叠} \textcolor{Sepia}{\selectlanguage{english}\mytextsc{red}} \textcolor{PineGreen}{\selectlanguage{french}\mytextsc{red}}  
 ¶ \textcolor{darkblue}{\textbf{\ipa{le˧-hṽ˧\textasciitilde{}hṽ˧}}} \zh{\mytextsc{accomp} \mytextsc{red}} \textcolor{Sepia}{\selectlanguage{english}\mytextsc{accomp} \mytextsc{red}} \textcolor{PineGreen}{\selectlanguage{french}\mytextsc{accomp} \mytextsc{red}}  
 ¶ \textcolor{darkblue}{\textbf{\ipa{hṽ˧\textasciitilde{}hṽ˧-ze˩}}} \zh{炒了} \textcolor{Sepia}{\selectlanguage{english}\mytextsc{red} \mytextsc{pfv}} \textcolor{PineGreen}{\selectlanguage{french}\mytextsc{red} \mytextsc{pfv}}  
 ¶ \textcolor{darkblue}{\textbf{\ipa{ʂe˧ hṽ˧\textasciitilde{}hṽ˧}}} \zh{炒肉} \textcolor{Sepia}{\selectlanguage{english}to stir-fry some meat} \textcolor{PineGreen}{\selectlanguage{french}frire de la viande}  
 ¶ \textcolor{darkblue}{\textbf{\ipa{v̩˩tsʰɤ˧ hṽ˧\textasciitilde{}hṽ˧}}} \zh{炒蔬菜} \textcolor{Sepia}{\selectlanguage{english}to stir-fry some vegetables} \textcolor{PineGreen}{\selectlanguage{french}frire des légumes}  
 ¶ \textcolor{darkblue}{\textbf{\ipa{læ˧tsɯ˥ hṽ˩\textasciitilde{}hṽ˩}}} \zh{炒辣椒} \textcolor{Sepia}{\selectlanguage{english}to stir-fry chili peppers} \textcolor{PineGreen}{\selectlanguage{french}frire des piments}  
 ¶ \textcolor{darkblue}{\textbf{\ipa{hɑ˧ hṽ˧\textasciitilde{}hṽ˧}}} \zh{炒饭} \textcolor{Sepia}{\selectlanguage{english}to stir-fry some rice, to stir-fry some food} \textcolor{PineGreen}{\selectlanguage{french}frire un plat/faire la cuisine/frire du riz/de la nourriture}  

\lhead{\firstmark}
\rhead{\botmark}

\subsection{\hspace{-0.5cm} {\Large \textcolor{darkblue}{\textbf{\ipa{hṽ˥}}}}\hspace{0.5cm}[\kern2pt{\textcolor{darkblue}{\textbf{\ipa{hṽ˥}}}}\kern2pt]} \hypertarget{hv\string_~\string_T1}{}
\markboth{\textcolor{darkblue}{\textbf{\ipa{hṽ˥}}}}{}
\textcolor{teal}{\zh{名词}} \hspace{4pt} \zh{声调类:} \#H.
\zh{毛、豪猪的翎。} \textcolor{Sepia}{\selectlanguage{english}Body hair; animal's hair; porcupine's quills.} \textcolor{PineGreen}{\selectlanguage{french}Poils (pour les animaux, y compris les épines du hérisson; aussi pour les hommes).}  \zh{量词}: \textcolor{darkblue}{\textbf{\ipa{kʰɯ˩}}} 
\lhead{\firstmark}
\rhead{\botmark}

\subsection{\hspace{-0.5cm} {\Large \textcolor{darkblue}{\textbf{\ipa{hṽ˧dɤ˧ɻ\#˥}}}}\hspace{0.5cm}[\kern2pt{\textcolor{darkblue}{\textbf{\ipa{hṽ˩dɤ˩ɻ˩˥}}}}\kern2pt]} \hypertarget{hv\string_~\string_Md7\string_Mr£`\#\string_T1}{}
\markboth{\textcolor{darkblue}{\textbf{\ipa{hṽ˧dɤ˧ɻ\#˥}}}}{}
\textcolor{teal}{\zh{形容词}} \hspace{4pt} \zh{声调类:} \#H.
\zh{笨拙,经常损坏东西。} \textcolor{Sepia}{\selectlanguage{english}Clumsy, incapable.} \textcolor{PineGreen}{\selectlanguage{french}Maladroit, incapable.}  ¶ \textcolor{darkblue}{\textbf{\ipa{hṽ˩-hĩ˩˥}}} \zh{笨拙的} \textcolor{Sepia}{\selectlanguage{english}\mytextsc{rel}} \textcolor{PineGreen}{\selectlanguage{french}\mytextsc{rel}}  
 ¶ \textcolor{darkblue}{\textbf{\ipa{hṽ˧dɤ˧ɻ̍˧\textasciitilde{}hṽ˧dɤ˧ɻ̍˧-zo˥}}} \zh{笨手笨脚的(男)人} \textcolor{Sepia}{\selectlanguage{english}clumsy person / clumsy boy} \textcolor{PineGreen}{\selectlanguage{french}un maladroit}  
 ¶ \textcolor{darkblue}{\textbf{\ipa{[F5] hṽ˩ɖɻ̍˩\textasciitilde{}hṽ˧ɖɻ̍˧-gv̩˧}}} \zh{重叠:笨笨的} \textcolor{Sepia}{\selectlanguage{english}\mytextsc{red}} \textcolor{PineGreen}{\selectlanguage{french}\mytextsc{red}}  

\lhead{\firstmark}
\rhead{\botmark}

\subsection{\hspace{-0.5cm} {\Large \textcolor{darkblue}{\textbf{\ipa{hṽ˧\textasciitilde{}hṽ˩-ɖʐæ˩\textasciitilde{}ɖʐæ˩}}}}\hspace{0.5cm}[\kern2pt{\textcolor{darkblue}{\textbf{\ipa{xxxx non-correspondance entre le nombre de morphèmes et le nombre de tons de morphèmes}}}}\kern2pt]} \hypertarget{hv\string_~\string_M~hv\string_~\string_B-d`z`\{\string_B~d`z`\{\string_B1}{}
\markboth{\textcolor{darkblue}{\textbf{\ipa{hṽ˧\textasciitilde{}hṽ˩-ɖʐæ˩\textasciitilde{}ɖʐæ˩}}}}{}
\textcolor{teal}{\zh{形容词}} \hspace{4pt} \zh{声调类:} L\#-.
\zh{红红的。} \textcolor{Sepia}{\selectlanguage{english}Intensely red, red all over.} \textcolor{PineGreen}{\selectlanguage{french}Tout rouge.} 
\lhead{\firstmark}
\rhead{\botmark}

\subsection{\hspace{-0.5cm} {\Large \textcolor{darkblue}{\textbf{\ipa{hṽ˧nɑ˩}}}}\hspace{0.5cm}[\kern2pt{\textcolor{darkblue}{\textbf{\ipa{hṽ˩nɑ˥}}}}\kern2pt]} \hypertarget{hv\string_~\string_MnA\string_B1}{}
\markboth{\textcolor{darkblue}{\textbf{\ipa{hṽ˧nɑ˩}}}}{}
\textcolor{teal}{\zh{名词}} \hspace{4pt} \zh{声调类:} L\#.
\zh{野兽。} \textcolor{Sepia}{\selectlanguage{english}Wild animal.} \textcolor{PineGreen}{\selectlanguage{french}Bête sauvage.}  \zh{量词}: \textcolor{darkblue}{\textbf{\ipa{mi˩}}} 
\lhead{\firstmark}
\rhead{\botmark}

\subsection{\hspace{-0.5cm} {\Large \textcolor{darkblue}{\textbf{\ipa{hṽ˩\textsubscript{a}}}} \textsubscript{1}}\hspace{0.5cm}[\kern2pt{\textcolor{darkblue}{\textbf{\ipa{hṽ˥}}}}\kern2pt]} \hypertarget{hv\string_~\string_Ba1}{}
\markboth{\textcolor{darkblue}{\textbf{\ipa{hṽ˩\textsubscript{a}}}} \textsubscript{1}}{}
\textcolor{teal}{\zh{形容词}} \hspace{4pt} \zh{声调类:} L\textsubscript{a}.
\zh{红色的。} \textcolor{Sepia}{\selectlanguage{english}Red.} \textcolor{PineGreen}{\selectlanguage{french}Rouge (ex.: vêtement rouge, sang rouge).} 
\lhead{\firstmark}
\rhead{\botmark}

\subsection{\hspace{-0.5cm} {\Large \textcolor{darkblue}{\textbf{\ipa{hṽ˩\textsubscript{a}}}} \textsubscript{2}}\hspace{0.5cm}[\kern2pt{\textcolor{darkblue}{\textbf{\ipa{hṽ˩˥}}}}\kern2pt]} \hypertarget{hv\string_~\string_Ba2}{}
\markboth{\textcolor{darkblue}{\textbf{\ipa{hṽ˩\textsubscript{a}}}} \textsubscript{2}}{}
\textcolor{teal}{\zh{形容词}} \hspace{4pt} \zh{声调类:} L\textsubscript{a}.
\zh{矮、低。} \textcolor{Sepia}{\selectlanguage{english}Of small stature, small in stature, short, not tall; low.} \textcolor{PineGreen}{\selectlanguage{french}De petite taille, de petite stature; bas.} 
\lhead{\firstmark}
\rhead{\botmark}

\subsection{\hspace{-0.5cm} {\Large \textcolor{darkblue}{\textbf{\ipa{hṽ˩-ɖʐæ˩ɻæ˥}}}}\hspace{0.5cm}[\kern2pt{\textcolor{darkblue}{\textbf{\ipa{xxxx non-correspondance entre le nombre de morphèmes et le nombre de tons de morphèmes}}}}\kern2pt]} \hypertarget{hv\string_~\string_B-d`z`\{\string_Br£`\{\string_T1}{}
\markboth{\textcolor{darkblue}{\textbf{\ipa{hṽ˩-ɖʐæ˩ɻæ˥}}}}{}
\textcolor{teal}{\zh{形容词}} \hspace{4pt} \zh{声调类:} L+H\#.
\zh{红红的。} \textcolor{Sepia}{\selectlanguage{english}Intensely red, red all over.} \textcolor{PineGreen}{\selectlanguage{french}Tout rouge.}  ¶ \textcolor{darkblue}{\textbf{\ipa{hṽ˩ɖʐæ˩ɻæ˥-gv̩˩}}} \zh{红红的} \textcolor{Sepia}{\selectlanguage{english}intensely red, red all over} \textcolor{PineGreen}{\selectlanguage{french}tout rouge}  
 ¶ \textcolor{darkblue}{\textbf{\ipa{[F5] hṽ˩ɖʐæ˩˥ | hṽ˩ɖʐæ˩˥ gv̩˩}}} \zh{重叠} \textcolor{Sepia}{\selectlanguage{english}\mytextsc{red;} the first two syllables are higher-pitched than the following two} \textcolor{PineGreen}{\selectlanguage{french}\mytextsc{red;} les deux premières syllabes ont une fréquence fondamentale nettement plus haute que les deux suivantes}  

\lhead{\firstmark}
\rhead{\botmark}

\subsection{\hspace{-0.5cm} {\Large \textcolor{darkblue}{\textbf{\ipa{hṽ˩\textasciitilde{}hṽ˩}}}}\hspace{0.5cm}[\kern2pt{\textcolor{darkblue}{\textbf{\ipa{hṽ˧hṽ˩}}}}\kern2pt]} \hypertarget{hv\string_~\string_B~hv\string_~\string_B1}{}
\markboth{\textcolor{darkblue}{\textbf{\ipa{hṽ˩\textasciitilde{}hṽ˩}}}}{}
\textcolor{teal}{\zh{动词}} \hspace{4pt} \zh{声调类:} L.
\zh{干扰、防碍。} \textcolor{Sepia}{\selectlanguage{english}To disturb, to interfere, to hinder, to obstruct, to impede.} \textcolor{PineGreen}{\selectlanguage{french}Ennuyer, empêcher, faire obstruction à.}  ¶ \textcolor{darkblue}{\textbf{\ipa{hĩ˧ hṽ˥\textasciitilde{}hṽ˩}}} \zh{干扰人家} \textcolor{Sepia}{\selectlanguage{english}to annoy people} \textcolor{PineGreen}{\selectlanguage{french}ennuyer les gens}  

\lhead{\firstmark}
\rhead{\botmark}

\subsection{\hspace{-0.5cm} {\Large \textcolor{darkblue}{\textbf{\ipa{hwɑ˩kwɤ˧}}}}\hspace{0.5cm}[\kern2pt{\textcolor{darkblue}{\textbf{\ipa{hwɑ˧kwɤ˥}}}}\kern2pt]} \hypertarget{hwA\string_Bkw7\string_M1}{}
\markboth{\textcolor{darkblue}{\textbf{\ipa{hwɑ˩kwɤ˧}}}}{}
\textcolor{teal}{\zh{名词}} \hspace{4pt} \zh{声调类:} LM.
\zh{黄瓜(汉语借词)。} \textcolor{Sepia}{\selectlanguage{english}Cucumber.} \textcolor{PineGreen}{\selectlanguage{french}Concombre.}  \zh{【借词】} \zh{黄瓜}
 \zh{量词}: \textcolor{darkblue}{\textbf{\ipa{ɭɯ˧}}} 
\lhead{\firstmark}
\rhead{\botmark}

\subsection{\hspace{-0.5cm} {\Large \textcolor{darkblue}{\textbf{\ipa{hwæ˧\textsubscript{a}}}}}\hspace{0.5cm}[\kern2pt{\textcolor{darkblue}{\textbf{\ipa{hwæ˩˥}}}}\kern2pt]} \hypertarget{hw\{\string_Ma1}{}
\markboth{\textcolor{darkblue}{\textbf{\ipa{hwæ˧\textsubscript{a}}}}}{}
\textcolor{teal}{\zh{动词}} \hspace{4pt} \zh{声调类:} M\textsubscript{a}.
\zh{买。} \textcolor{Sepia}{\selectlanguage{english}To buy.} \textcolor{PineGreen}{\selectlanguage{french}Acheter.}  ¶ \textcolor{darkblue}{\textbf{\ipa{le˧-hwæ˧}}} \zh{\mytextsc{accomp}} \textcolor{Sepia}{\selectlanguage{english}\mytextsc{accomp}} \textcolor{PineGreen}{\selectlanguage{french}\mytextsc{accomp}}  
 ¶ \textcolor{darkblue}{\textbf{\ipa{tso˧\textasciitilde{}tso˧ hwæ˩}}} \zh{买东西} \textcolor{Sepia}{\selectlanguage{english}to buy things} \textcolor{PineGreen}{\selectlanguage{french}acheter des choses}  
 ¶ \textcolor{darkblue}{\textbf{\ipa{ɖɯ˧-kʰwɤ˥ hwæ˩}}} \zh{买一块} \textcolor{Sepia}{\selectlanguage{english}to buy a piece (of something)} \textcolor{PineGreen}{\selectlanguage{french}acheter un morceau}  
 ¶ \textcolor{darkblue}{\textbf{\ipa{hwæ˧\textasciitilde{}hwæ˩}}} \zh{\mytextsc{重叠}} \textcolor{Sepia}{\selectlanguage{english}\mytextsc{red}} \textcolor{PineGreen}{\selectlanguage{french}\mytextsc{red}}  

\lhead{\firstmark}
\rhead{\botmark}

\subsection{\hspace{-0.5cm} {\Large \textcolor{darkblue}{\textbf{\ipa{hwæ˧ɖʐæ˥}}} \textsubscript{1}}\hspace{0.5cm}[\kern2pt{\textcolor{darkblue}{\textbf{\ipa{hwæ˧ɖʐæ˧}}}}\kern2pt]} \hypertarget{hw\{\string_Md`z`\{\string_T1}{}
\markboth{\textcolor{darkblue}{\textbf{\ipa{hwæ˧ɖʐæ˥}}} \textsubscript{1}}{}
\textcolor{teal}{\zh{名词}} \hspace{4pt} \zh{声调类:} H\#.
\zh{松鼠,灰鼠。} \textcolor{Sepia}{\selectlanguage{english}Squirrel.} \textcolor{PineGreen}{\selectlanguage{french}Écureuil.}  ¶ \textcolor{darkblue}{\textbf{\ipa{hwæ˧ɖʐæ˥-pʰv̩˩}}} \zh{公松鼠} \textcolor{Sepia}{\selectlanguage{english}male squirrel} \textcolor{PineGreen}{\selectlanguage{french}écureuil mâle}  
 ¶ \textcolor{darkblue}{\textbf{\ipa{hwæ˧ɖʐæ˥-mi˩}}} \zh{母松鼠} \textcolor{Sepia}{\selectlanguage{english}female squirrel} \textcolor{PineGreen}{\selectlanguage{french}écureuil femelle}  
 \zh{量词}: \textcolor{darkblue}{\textbf{\ipa{mi˩}}} 
\lhead{\firstmark}
\rhead{\botmark}

\subsection{\hspace{-0.5cm} {\Large \textcolor{darkblue}{\textbf{\ipa{hwæ˧ɖʐæ˥}}} \textsubscript{2}}\hspace{0.5cm}[\kern2pt{\textcolor{darkblue}{\textbf{\ipa{hwæ˧ɖʐæ˥}}}}\kern2pt]} \hypertarget{hw\{\string_Md`z`\{\string_T2}{}
\markboth{\textcolor{darkblue}{\textbf{\ipa{hwæ˧ɖʐæ˥}}} \textsubscript{2}}{}
\textcolor{teal}{\zh{名词}} \hspace{4pt} \zh{声调类:} H\#.
\zh{瘊子、肉赘。} \textcolor{Sepia}{\selectlanguage{english}Wart.} \textcolor{PineGreen}{\selectlanguage{french}Verrue.}  ¶ \textcolor{darkblue}{\textbf{\ipa{hwæ˧ʈʂæ˥ tʰv̩˩}}} \zh{长瘊子} \textcolor{Sepia}{\selectlanguage{english}a wart forms} \textcolor{PineGreen}{\selectlanguage{french}une verrue se forme; attraper une verrue}  
 ¶ \textcolor{darkblue}{\textbf{\ipa{hwæ˧ʈʂæ˥ | le˧-tʰv̩˧-ze˧!}}} \zh{长瘊子了!} \textcolor{Sepia}{\selectlanguage{english}A wart has formed!} \textcolor{PineGreen}{\selectlanguage{french}Une verrue s'est formée!}  
 \zh{量词}: \textcolor{darkblue}{\textbf{\ipa{mi˩}}} 
\lhead{\firstmark}
\rhead{\botmark}

\subsection{\hspace{-0.5cm} {\Large \textcolor{darkblue}{\textbf{\ipa{hwæ˧pʰæ˥}}}}\hspace{0.5cm}[\kern2pt{\textcolor{darkblue}{\textbf{\ipa{hwæ˧pʰæ˩}}}}\kern2pt]} \hypertarget{hw\{\string_Mp\string_h\{\string_T1}{}
\markboth{\textcolor{darkblue}{\textbf{\ipa{hwæ˧pʰæ˥}}}}{}
\textcolor{teal}{\zh{名词}} \hspace{4pt} \zh{声调类:} H\#.
\zh{一块布。} \textcolor{Sepia}{\selectlanguage{english}A piece of cloth.} \textcolor{PineGreen}{\selectlanguage{french}Pièce de tissu.}  \zh{量词}: \textcolor{darkblue}{\textbf{\ipa{pʰæ˧˥}}} 
\lhead{\firstmark}
\rhead{\botmark}

\subsection{\hspace{-0.5cm} {\Large \textcolor{darkblue}{\textbf{\ipa{hwæ˧pʰæ˩}}}}\hspace{0.5cm}[\kern2pt{\textcolor{darkblue}{\textbf{\ipa{hwæ˩pʰæ˥}}}}\kern2pt]} \hypertarget{hw\{\string_Mp\string_h\{\string_B1}{}
\markboth{\textcolor{darkblue}{\textbf{\ipa{hwæ˧pʰæ˩}}}}{}
\textcolor{teal}{\zh{名词}} \hspace{4pt} \zh{声调类:} L\#.
\zh{大锄。} \textcolor{Sepia}{\selectlanguage{english}Large hoe.} \textcolor{PineGreen}{\selectlanguage{french}Grosse houe.} \zh{当地汉语方言:}\zh{挖锄。} ¶ \textcolor{darkblue}{\textbf{\ipa{hwæ˧pʰæ˩ tʰv̩˩-nɑ˩}}} \zh{这把大锄} \textcolor{Sepia}{\selectlanguage{english}\mytextsc{n}+\mytextsc{dem}+\mytextsc{clf}} \textcolor{PineGreen}{\selectlanguage{french}\mytextsc{n}+\mytextsc{dem}+\mytextsc{clf}}  
 \zh{量词}: \textcolor{darkblue}{\textbf{\ipa{nɑ˧}}} 
\lhead{\firstmark}
\rhead{\botmark}

\subsection{\hspace{-0.5cm} {\Large \textcolor{darkblue}{\textbf{\ipa{hwæ˧pʰæ˩-gv̩˩-di˩}}}}\hspace{0.5cm}[\kern2pt{\textcolor{darkblue}{\textbf{\ipa{xxxx non-correspondance entre le nombre de morphèmes et le nombre de tons de morphèmes}}}}\kern2pt]} \hypertarget{hw\{\string_Mp\string_h\{\string_B-gv\string_=\string_B-di\string_B1}{}
\markboth{\textcolor{darkblue}{\textbf{\ipa{hwæ˧pʰæ˩-gv̩˩-di˩}}}}{}
\textcolor{teal}{\zh{名词}} \hspace{4pt} \zh{声调类:} L\#--.
\zh{织布机。} \textcolor{Sepia}{\selectlanguage{english}Loom.} \textcolor{PineGreen}{\selectlanguage{french}Métier à tisser.}  ¶ \textcolor{darkblue}{\textbf{\ipa{hwæ˧pʰæ˩gv̩˩di˩-tɕi˩tɕʰi˧}}} \zh{工业织布机。直译:“织布机器”(在摩梭词后面加上汉语的“机器”)} \textcolor{Sepia}{\selectlanguage{english}industrial sewing machine (formed of the Na word plus the Chinese word for 'engine')} \textcolor{PineGreen}{\selectlanguage{french}métier à tisser industriel, machine à faire du tissu (formé de 'métier à tisser' + le mot chinois pour 'machine')}  
 \zh{量词}: \textcolor{darkblue}{\textbf{\ipa{nɑ˧}}} 
\lhead{\firstmark}
\rhead{\botmark}

\subsection{\hspace{-0.5cm} {\Large \textcolor{darkblue}{\textbf{\ipa{hwæ˧tsɯ˥}}}}\hspace{0.5cm}[\kern2pt{\textcolor{darkblue}{\textbf{\ipa{xxxx non-correspondance entre le nombre de morphèmes et le nombre de tons de morphèmes}}}}\kern2pt]} \hypertarget{hw\{\string_MtsM\string_T1}{}
\markboth{\textcolor{darkblue}{\textbf{\ipa{hwæ˧tsɯ˥}}}}{}
\textcolor{teal}{\zh{名词}} \hspace{4pt} \zh{声调类:} H\#.
\zh{老鼠(汉语借词)。} \textcolor{Sepia}{\selectlanguage{english}Rat.} \textcolor{PineGreen}{\selectlanguage{french}Rat.} \zh{当地汉语方言:}\zh{耗子。} \zh{【借词】} \zh{耗子}
 ¶ \textcolor{darkblue}{\textbf{\ipa{hwæ˧tsɯ˥-pʰv̩˩}}} \zh{公老鼠} \textcolor{Sepia}{\selectlanguage{english}male rat} \textcolor{PineGreen}{\selectlanguage{french}rat mâle}  
 ¶ \textcolor{darkblue}{\textbf{\ipa{hwæ˧tsɯ˥-mi˩}}} \zh{母老鼠} \textcolor{Sepia}{\selectlanguage{english}female rat} \textcolor{PineGreen}{\selectlanguage{french}rat femelle}  
 \zh{量词}: \textcolor{darkblue}{\textbf{\ipa{mi˩}}} 
\lhead{\firstmark}
\rhead{\botmark}

\subsection{\hspace{-0.5cm} {\Large \textcolor{darkblue}{\textbf{\ipa{hwæ˧tsɯ˥-njɤ˩di˩}}}}\hspace{0.5cm}[\kern2pt{\textcolor{darkblue}{\textbf{\ipa{xxxx non-correspondance entre le nombre de morphèmes et le nombre de tons de morphèmes}}}}\kern2pt]} \hypertarget{hw\{\string_MtsM\string_T-nj7\string_Bdi\string_B1}{}
\markboth{\textcolor{darkblue}{\textbf{\ipa{hwæ˧tsɯ˥-njɤ˩di˩}}}}{}
\textcolor{teal}{\zh{名词}} \hspace{4pt} \zh{声调类:} H\#-.
\zh{大蓟。} \textcolor{Sepia}{\selectlanguage{english}Setose thisle.} \textcolor{PineGreen}{\selectlanguage{french}Chardon.}  \zh{量词}: \textcolor{darkblue}{\textbf{\ipa{dzi˩}}} 
\lhead{\firstmark}
\rhead{\botmark}

\subsection{\hspace{-0.5cm} {\Large \textcolor{darkblue}{\textbf{\ipa{hwæ˧tsɯ˥-njɤ˩di˩-si˩dzi˩}}}}\hspace{0.5cm}[\kern2pt{\textcolor{darkblue}{\textbf{\ipa{xxxx non-correspondance entre le nombre de morphèmes et le nombre de tons de morphèmes}}}}\kern2pt]} \hypertarget{hw\{\string_MtsM\string_T-nj7\string_Bdi\string_B-si\string_Bdzi\string_B1}{}
\markboth{\textcolor{darkblue}{\textbf{\ipa{hwæ˧tsɯ˥-njɤ˩di˩-si˩dzi˩}}}}{}
\textcolor{teal}{\zh{名词}} \hspace{4pt} \zh{声调类:} H\#--.
\zh{牛蒡。} \textcolor{Sepia}{\selectlanguage{english}Burdock.} \textcolor{PineGreen}{\selectlanguage{french}Bardane: \textit{Arctium lappa}, plante dont les graines adhèrent à la laine et la queue des moutons. On en tire un médicament contre le rhume.} \zh{当地汉语方言:}\zh{牛蒡子。}
\lhead{\firstmark}
\rhead{\botmark}

\subsection{\hspace{-0.5cm} {\Large \textcolor{darkblue}{\textbf{\ipa{hwæ˩\textsubscript{a}}}} \textsubscript{1}}\hspace{0.5cm}[\kern2pt{\textcolor{darkblue}{\textbf{\ipa{xxxx non-correspondance entre le nombre de morphèmes et le nombre de tons de morphèmes}}}}\kern2pt]} \hypertarget{hw\{\string_Ba1}{}
\markboth{\textcolor{darkblue}{\textbf{\ipa{hwæ˩\textsubscript{a}}}} \textsubscript{1}}{}
\textcolor{teal}{\zh{动词}} \hspace{4pt} \zh{声调类:} L\textsubscript{a}.
\zh{关(出门,就关门)。} \textcolor{Sepia}{\selectlanguage{english}To close the door (from outside).} \textcolor{PineGreen}{\selectlanguage{french}Fermer (la porte).}  ¶ \textcolor{darkblue}{\textbf{\ipa{kʰi˧ | tʰi˧-hwæ˩!}}} \zh{关门吧!} \textcolor{Sepia}{\selectlanguage{english}Close the door!} \textcolor{PineGreen}{\selectlanguage{french}Ferme la porte!}  
 ¶ \textcolor{darkblue}{\textbf{\ipa{ʂe˧bæ˧ | le˧-wo˧-hwæ˥}}} \textcolor{PineGreen}{\selectlanguage{french}mettre la chaîne à la porte (quand on sort de la maison, on ferme la porte avec une chaîne de fer, et un verrou)}  
 ¶ \textcolor{darkblue}{\textbf{\ipa{kʰi˧-bi˥ di˩-hĩ˩ ʂe˩bæ˩}}}  
 ¶  
\zh{~【参考】~} \hyperlink{}{\textcolor{darkblue}{\textbf{\ipa{ʈæ˩\textsubscript{a}}}}} 
\lhead{\firstmark}
\rhead{\botmark}

\subsection{\hspace{-0.5cm} {\Large \textcolor{darkblue}{\textbf{\ipa{hwæ˩\textsubscript{a}}}} \textsubscript{2}}\hspace{0.5cm}[\kern2pt{\textcolor{darkblue}{\textbf{\ipa{hwæ˩˥}}}}\kern2pt]} \hypertarget{hw\{\string_Ba2}{}
\markboth{\textcolor{darkblue}{\textbf{\ipa{hwæ˩\textsubscript{a}}}} \textsubscript{2}}{}
\textcolor{teal}{\zh{动词}} \hspace{4pt} \zh{声调类:} L\textsubscript{a}.
\zh{悬挂、挂在墙上。} \textcolor{Sepia}{\selectlanguage{english}To suspend, to hang (in a place).} \textcolor{PineGreen}{\selectlanguage{french}Accrocher, suspendre; être accroché, suspendu à; se tenir à.}  ¶ \textcolor{darkblue}{\textbf{\ipa{tso˧\textasciitilde{}tso˧ | gɤ˧bi˧ hwæ˥}}} \zh{挂东西在上面} \textcolor{Sepia}{\selectlanguage{english}to suspend things up high (e.g. on a hook)} \textcolor{PineGreen}{\selectlanguage{french}accrocher des choses en hauteur}  
 ¶ \textcolor{darkblue}{\textbf{\ipa{tso˧\textasciitilde{}tso˧ hwæ˥}}} \zh{挂东西} \textcolor{Sepia}{\selectlanguage{english}to suspend things} \textcolor{PineGreen}{\selectlanguage{french}accrocher des choses}  
 ¶ \textcolor{darkblue}{\textbf{\ipa{ʂe˧ | tʰi˧-hwæ˩}}} \zh{挂肉(在火塘上,为了熏肉)} \textcolor{Sepia}{\selectlanguage{english}to hang meat (above the hearth, to smoke it)} \textcolor{PineGreen}{\selectlanguage{french}accrocher de la viande (au-dessus du foyer, pour la fumer)}  
 ¶ \textcolor{darkblue}{\textbf{\ipa{tso˧\textasciitilde{}tso˧ | tʰi˧-hwæ˩}}} \zh{挂东西} \textcolor{Sepia}{\selectlanguage{english}to suspend things} \textcolor{PineGreen}{\selectlanguage{french}accrocher des choses}  
 ¶ \textcolor{darkblue}{\textbf{\ipa{ʂe˧-hwæ˥-di˩}}} \zh{主屋里面的小梁(大梁下面),用来挂肉,熏肉。直译:“挂肉的东西”。} \textcolor{Sepia}{\selectlanguage{english}A small beam in the main room (right under the main beams) where meat is suspended to smoke it. The word literally means “thing to hang meat”.} \textcolor{PineGreen}{\selectlanguage{french}Poutrelle de la pièce principale, juste en-dessous des poutres maîtresses, servant à suspendre de la viande qui sèche. Littéralement “objet pour accrocher de la viande”}  
 ¶ \textcolor{darkblue}{\textbf{\ipa{tso˧\textasciitilde{}tso˧-hwæ˥-di˩}}} \zh{挂(东西)用的(东西),如:钩子、用来挂肉的小梁……} \textcolor{Sepia}{\selectlanguage{english}object used to suspend things; this can refer to any object from a hook to a beam on which things are hung} \textcolor{PineGreen}{\selectlanguage{french}objet servant à suspendre des choses; cette périphrase peut par exemple désigner la poutrelle servant à accrocher de la viande, dans la pièce principale}  

\lhead{\firstmark}
\rhead{\botmark}

\subsection{\hspace{-0.5cm} {\Large \textcolor{darkblue}{\textbf{\ipa{hwɤ̃˩\textsubscript{a}}}}}\hspace{0.5cm}[\kern2pt{\textcolor{darkblue}{\textbf{\ipa{hwɤ̃˥}}}}\kern2pt]} \hypertarget{hw7\string_~\string_Ba1}{}
\markboth{\textcolor{darkblue}{\textbf{\ipa{hwɤ̃˩\textsubscript{a}}}}}{}
\textcolor{teal}{\zh{形容词}} \hspace{4pt} \zh{声调类:} L\textsubscript{a}.
\zh{迟,晚。} \textcolor{Sepia}{\selectlanguage{english}Late.} \textcolor{PineGreen}{\selectlanguage{french}En retard.}  ¶ \textcolor{darkblue}{\textbf{\ipa{hwɤ̃˩-hĩ˩˥}}} \zh{迟的} \textcolor{Sepia}{\selectlanguage{english}\mytextsc{rel}/\mytextsc{nmlz}} \textcolor{PineGreen}{\selectlanguage{french}\mytextsc{rel}/\mytextsc{nmlz}}  
 ¶ \textcolor{darkblue}{\textbf{\ipa{ʈʂʰɯ˧ ʑi˧-ʈi˥ hwɤ̃˩!}}} \zh{他起床起得晚!} \textcolor{Sepia}{\selectlanguage{english}He/she gets up late!} \textcolor{PineGreen}{\selectlanguage{french}il se lève tard}  

\lhead{\firstmark}
\rhead{\botmark}

\subsection{\hspace{-0.5cm} {\Large \textcolor{darkblue}{\textbf{\ipa{hwɤ˥}}}}\hspace{0.5cm}[\kern2pt{\textcolor{darkblue}{\textbf{\ipa{hwɤ˥}}}}\kern2pt]} \hypertarget{hw7\string_T1}{}
\markboth{\textcolor{darkblue}{\textbf{\ipa{hwɤ˥}}}}{}
\textcolor{teal}{\zh{名词}} \hspace{4pt} \zh{声调类:} \#H.
\zh{在大事发生的时候送的礼物(近期一般给钱):婚礼、葬礼。} \textcolor{Sepia}{\selectlanguage{english}Gift made on important occasions (weddings, etc).} \textcolor{PineGreen}{\selectlanguage{french}Cadeau (souvent financier) qu'on offre à l'occasion des grands événements: décès, mariages.}  ¶ \textcolor{darkblue}{\textbf{\ipa{hwɤ˧ | ɖɯ˧-kʰwɤ˥}}} \zh{一份大礼物} \textcolor{Sepia}{\selectlanguage{english}a gift, a donation} \textcolor{PineGreen}{\selectlanguage{french}un cadeau, une donation}  
 \zh{量词}: \textcolor{darkblue}{\textbf{\ipa{kʰwɤ˥}}} 
\lhead{\firstmark}
\rhead{\botmark}

\subsection{\hspace{-0.5cm} {\Large \textcolor{darkblue}{\textbf{\ipa{hwɤ˥}}}}\hspace{0.5cm}[\kern2pt{\textcolor{darkblue}{\textbf{\ipa{hwɤ˧˥}}}}\kern2pt]} \hypertarget{hw7\string_T1}{}
\markboth{\textcolor{darkblue}{\textbf{\ipa{hwɤ˥}}}}{}
\textcolor{teal}{\zh{动词}} \hspace{4pt} \zh{声调类:} H.
\zh{执绋送丧。} \textcolor{Sepia}{\selectlanguage{english}To participate in a funeral ceremony (literally 'to see [the deceased] out').} \textcolor{PineGreen}{\selectlanguage{french}Participer à une cérémonie funèbre; littéralement “envoyer quelqu'un”, c'est-à-dire accompagner quelqu'un vers l'au-delà.} 
\lhead{\firstmark}
\rhead{\botmark}

\subsection{\hspace{-0.5cm} {\Large \textcolor{darkblue}{\textbf{\ipa{hwɤ˧}}}}\hspace{0.5cm}[\kern2pt{\textcolor{darkblue}{\textbf{\ipa{hwɤ˩˥}}}}\kern2pt]} \hypertarget{hw7\string_M1}{}
\markboth{\textcolor{darkblue}{\textbf{\ipa{hwɤ˧}}}}{}
\textcolor{teal}{\zh{形容词}} \hspace{4pt} \zh{声调类:} M.
\zh{宽,辽阔,宽敞。} \textcolor{Sepia}{\selectlanguage{english}Broad, vast, extensive; big (plain; piece of cloth, vegetable…).} \textcolor{PineGreen}{\selectlanguage{french}Vaste (plaine), étendu, grand (pièce de tissu), de grande taille (objet, légume...).}  ¶ \textcolor{darkblue}{\textbf{\ipa{qʰɑ˧-hwɤ˧-gv̩˧}}} \zh{非常宽敞} \textcolor{Sepia}{\selectlanguage{english}extremely vast} \textcolor{PineGreen}{\selectlanguage{french}très vaste}  

\lhead{\firstmark}
\rhead{\botmark}

\subsection{\hspace{-0.5cm} {\Large \textcolor{darkblue}{\textbf{\ipa{hwɤ˧kʰv̩˥}}}}\hspace{0.5cm}[\kern2pt{\textcolor{darkblue}{\textbf{\ipa{hwɤ˩kʰv̩˥}}}}\kern2pt]} \hypertarget{hw7\string_Mk\string_hv\string_=\string_T1}{}
\markboth{\textcolor{darkblue}{\textbf{\ipa{hwɤ˧kʰv̩˥}}}}{}
\textcolor{teal}{\zh{名词}} \hspace{4pt} \zh{声调类:} H\#.
\zh{鼠年(摩梭话称作“猫年”)。} \textcolor{Sepia}{\selectlanguage{english}Year of the Cat (corresponding to the Chinese year of the Rat).} \textcolor{PineGreen}{\selectlanguage{french}Année du Chat (correspondant à l'année chinoise du Rat).} 
\lhead{\firstmark}
\rhead{\botmark}

\subsection{\hspace{-0.5cm} {\Large \textcolor{darkblue}{\textbf{\ipa{hwɤ˧li˧-bv̩˥}}}}\hspace{0.5cm}[\kern2pt{\textcolor{darkblue}{\textbf{\ipa{xxxx non-correspondance entre le nombre de morphèmes et le nombre de tons de morphèmes}}}}\kern2pt]} \hypertarget{hw7\string_Mli\string_M-bv\string_=\string_T1}{}
\markboth{\textcolor{darkblue}{\textbf{\ipa{hwɤ˧li˧-bv̩˥}}}}{}
\textcolor{teal}{\zh{名词}} \hspace{4pt} \zh{声调类:} H\#.
\zh{夹层:主屋的夹层。因为烟多,所以人不能将这个空间当卧室。只有一层薄的木地板。} \textcolor{Sepia}{\selectlanguage{english}Lower balcony, mezzanine.} \textcolor{PineGreen}{\selectlanguage{french}Mezzanine: espace de la pièce principale où un plancher est aménagé sous la charpente, formant comme une mezzanine, mais que les habitants humains n'utilisent pas: l'endroit étrant très enfumé, on n'y place qu'un plancher peu solide et sans rambarde; d'où le nom: “(la pièce) du chat”. On y laisse parfois des objets (vanneries par exemples), qui y sont relativement préservés des insectes par la fumée.}  \zh{量词}: \textcolor{darkblue}{\textbf{\ipa{kɤ˧˥}}} \zh{~【同义词】~} \hyperlink{}{\textcolor{darkblue}{\textbf{\ipa{hwɤ˧li˧-se˧-di˧˥}}}}. 
\lhead{\firstmark}
\rhead{\botmark}

\subsection{\hspace{-0.5cm} {\Large \textcolor{darkblue}{\textbf{\ipa{hwɤ˧li˧-hwæ˧qʰæ\#˥}}}}\hspace{0.5cm}[\kern2pt{\textcolor{darkblue}{\textbf{\ipa{xxxx non-correspondance entre le nombre de morphèmes et le nombre de tons de morphèmes}}}}\kern2pt]} \hypertarget{hw7\string_Mli\string_M-hw\{\string_Mq\string_h\{\#\string_T1}{}
\markboth{\textcolor{darkblue}{\textbf{\ipa{hwɤ˧li˧-hwæ˧qʰæ\#˥}}}}{}
\textcolor{teal}{\zh{名词}} \hspace{4pt} \zh{声调类:} \#H.
\zh{山萝卜。} \textcolor{Sepia}{\selectlanguage{english}Scabious.} \textcolor{PineGreen}{\selectlanguage{french}Yyyy.} 
\lhead{\firstmark}
\rhead{\botmark}

\subsection{\hspace{-0.5cm} {\Large \textcolor{darkblue}{\textbf{\ipa{hwɤ˧li˧-se˧-di˧˥}}}}\hspace{0.5cm}[\kern2pt{\textcolor{darkblue}{\textbf{\ipa{xxxx non-correspondance entre le nombre de morphèmes et le nombre de tons de morphèmes}}}}\kern2pt]} \hypertarget{hw7\string_Mli\string_M-se\string_M-di\string_M\string_T1}{}
\markboth{\textcolor{darkblue}{\textbf{\ipa{hwɤ˧li˧-se˧-di˧˥}}}}{}
\textcolor{teal}{\zh{名词}} \hspace{4pt} \zh{声调类:} MH\#.
\zh{夹层:主屋的夹层。因为烟多,所以人不能将这个空间当卧室。只有一层薄的木地板。} \textcolor{Sepia}{\selectlanguage{english}Lower balcony, mezzanine.} \textcolor{PineGreen}{\selectlanguage{french}Mezzanine: espace de la pièce principale où un plancher est aménagé sous la charpente, formant comme une mezzanine, mais que les habitants humains n'utilisent pas: l'endroit étrant très enfumé, on n'y place qu'un plancher peu solide et sans rambarde; d'où le nom: “(la pièce) du chat”. On y laisse parfois des objets (vanneries par exemples), qui y sont relativement préservés des insectes par la fumée.}  \zh{量词}: \textcolor{darkblue}{\textbf{\ipa{kɤ˧˥}}} \zh{~【同义词】~} \hyperlink{}{\textcolor{darkblue}{\textbf{\ipa{hwɤ˧li˧-bv̩˥}}}}. 
\lhead{\firstmark}
\rhead{\botmark}

\subsection{\hspace{-0.5cm} {\Large \textcolor{darkblue}{\textbf{\ipa{hwɤ˧li˧-ʂɯ˧mo˥}}}}\hspace{0.5cm}[\kern2pt{\textcolor{darkblue}{\textbf{\ipa{xxxx non-correspondance entre le nombre de morphèmes et le nombre de tons de morphèmes}}}}\kern2pt]} \hypertarget{hw7\string_Mli\string_M-s`M\string_Mmo\string_T1}{}
\markboth{\textcolor{darkblue}{\textbf{\ipa{hwɤ˧li˧-ʂɯ˧mo˥}}}}{}
\textcolor{teal}{\zh{名词}} \hspace{4pt} \zh{声调类:} H\#.
\zh{老猫(不分公、母)。} \textcolor{Sepia}{\selectlanguage{english}Old cat (male or female).} \textcolor{PineGreen}{\selectlanguage{french}Vieux chat, vieux matou (de l'un ou l'autre sexe).}  \zh{量词}: \textcolor{darkblue}{\textbf{\ipa{mi˩}}} 
\lhead{\firstmark}
\rhead{\botmark}

\subsection{\hspace{-0.5cm} {\Large \textcolor{darkblue}{\textbf{\ipa{hwɤ˧li˧-zo˧˥}}}}\hspace{0.5cm}[\kern2pt{\textcolor{darkblue}{\textbf{\ipa{xxxx non-correspondance entre le nombre de morphèmes et le nombre de tons de morphèmes}}}}\kern2pt]} \hypertarget{hw7\string_Mli\string_M-zo\string_M\string_T1}{}
\markboth{\textcolor{darkblue}{\textbf{\ipa{hwɤ˧li˧-zo˧˥}}}}{}
\textcolor{teal}{\zh{名词}} \hspace{4pt} \zh{声调类:} MH\#.
\zh{小猫。} \textcolor{Sepia}{\selectlanguage{english}Kitten, cub.} \textcolor{PineGreen}{\selectlanguage{french}Chaton.}  \zh{量词}: \textcolor{darkblue}{\textbf{\ipa{ɭɯ˧}}} 
\lhead{\firstmark}
\rhead{\botmark}

\subsection{\hspace{-0.5cm} {\Large \textcolor{darkblue}{\textbf{\ipa{hwɤ˧li˧˥}}}}\hspace{0.5cm}[\kern2pt{\textcolor{darkblue}{\textbf{\ipa{hwɤ˧li˥}}}}\kern2pt]} \hypertarget{hw7\string_Mli\string_M\string_T1}{}
\markboth{\textcolor{darkblue}{\textbf{\ipa{hwɤ˧li˧˥}}}}{}
\textcolor{teal}{\zh{名词}} \hspace{4pt} \zh{声调类:} MH\#.
\zh{猫。} \textcolor{Sepia}{\selectlanguage{english}Cat.} \textcolor{PineGreen}{\selectlanguage{french}Chat.}  \zh{量词}: \textcolor{darkblue}{\textbf{\ipa{mi˩}}} 
\lhead{\firstmark}
\rhead{\botmark}

\subsection{\hspace{-0.5cm} {\Large \textcolor{darkblue}{\textbf{\ipa{hwɤ˧mi˥\$}}}}\hspace{0.5cm}[\kern2pt{\textcolor{darkblue}{\textbf{\ipa{hwɤ˧mi˧˥}}}}\kern2pt]} \hypertarget{hw7\string_Mmi\string_T\$1}{}
\markboth{\textcolor{darkblue}{\textbf{\ipa{hwɤ˧mi˥\$}}}}{}
\textcolor{teal}{\zh{名词}} \hspace{4pt} \zh{声调类:} H\$.
\zh{母猫。} \textcolor{Sepia}{\selectlanguage{english}She-cat, queen.} \textcolor{PineGreen}{\selectlanguage{french}Chatte.}  ¶ \textcolor{darkblue}{\textbf{\ipa{hwɤ˧mi˧-hwɤ˥pʰv̩˩ / hwɤ˧mi˧-hwɤ˧pʰv̩˥\#}}} \zh{母猫与公猫} \textcolor{Sepia}{\selectlanguage{english}she-cat and tom-cat} \textcolor{PineGreen}{\selectlanguage{french}chatte et matou}  
 \zh{量词}: \textcolor{darkblue}{\textbf{\ipa{mi˩}}} 
\lhead{\firstmark}
\rhead{\botmark}

\subsection{\hspace{-0.5cm} {\Large \textcolor{darkblue}{\textbf{\ipa{hwɤ˧pʰv̩\#˥}}}}\hspace{0.5cm}[\kern2pt{\textcolor{darkblue}{\textbf{\ipa{hwɤ˧pʰv̩˥}}}}\kern2pt]} \hypertarget{hw7\string_Mp\string_hv\string_=\#\string_T1}{}
\markboth{\textcolor{darkblue}{\textbf{\ipa{hwɤ˧pʰv̩\#˥}}}}{}
\textcolor{teal}{\zh{名词}} \hspace{4pt} \zh{声调类:} \#H.
\zh{公猫。} \textcolor{Sepia}{\selectlanguage{english}Tom-cat, tom.} \textcolor{PineGreen}{\selectlanguage{french}Matou, chat mâle.}  ¶ \textcolor{darkblue}{\textbf{\ipa{hwɤ˧pʰv̩˧ tʰv̩˧-mi˥\#}}} \zh{那个公猫} \textcolor{Sepia}{\selectlanguage{english}\mytextsc{n}+\mytextsc{dem}+\mytextsc{clf}} \textcolor{PineGreen}{\selectlanguage{french}\mytextsc{n}+\mytextsc{dem}+\mytextsc{clf}}  
 ¶ \textcolor{darkblue}{\textbf{\ipa{hwɤ˧pʰv̩˧-hwɤ˧mi˥}}} \zh{公猫与母猫} \textcolor{Sepia}{\selectlanguage{english}tom-cat and she-cat} \textcolor{PineGreen}{\selectlanguage{french}matou et chatte}  
 \zh{量词}: \textcolor{darkblue}{\textbf{\ipa{mi˩}}} 
\lhead{\firstmark}
\rhead{\botmark}

\subsection{\hspace{-0.5cm} {\Large \textcolor{darkblue}{\textbf{\ipa{hwɤ˧se˧}}}}\hspace{0.5cm}[\kern2pt{\textcolor{darkblue}{\textbf{\ipa{hwɤ˧se˧}}}}\kern2pt]} \hypertarget{hw7\string_Mse\string_M1}{}
\markboth{\textcolor{darkblue}{\textbf{\ipa{hwɤ˧se˧}}}}{}
\textcolor{teal}{\zh{名词}} \hspace{4pt} \zh{声调类:} M.
\zh{花生。} \textcolor{Sepia}{\selectlanguage{english}Peanuts.} \textcolor{PineGreen}{\selectlanguage{french}Cacahuètes.}  \zh{【借词】} \zh{花生}
 ¶ \textcolor{darkblue}{\textbf{\ipa{hwɤ˧se˧-qo˧tv̩˩}}} \zh{花生米} \textcolor{Sepia}{\selectlanguage{english}peanuts} \textcolor{PineGreen}{\selectlanguage{french}même sens; littéralement “graines de cacahuètes”}  

\lhead{\firstmark}
\rhead{\botmark}

\subsection{\hspace{-0.5cm} {\Large \textcolor{darkblue}{\textbf{\ipa{hwɤ˧tɕi˥}}}}\hspace{0.5cm}[\kern2pt{\textcolor{darkblue}{\textbf{\ipa{hwɤ˧tɕi˧}}}}\kern2pt]} \hypertarget{hw7\string_Mts£i\string_T1}{}
\markboth{\textcolor{darkblue}{\textbf{\ipa{hwɤ˧tɕi˥}}}}{}
\textcolor{teal}{\zh{名词}} \hspace{4pt} \zh{声调类:} H\#.
\zh{土大黄(学名:皱叶酸模)(喂猪的牧草)。} \textcolor{Sepia}{\selectlanguage{english}Curly dock, \textit{Rumex crispus}. It is one of three sorts of plants used as pig fodder; it is also used as food for humans.} \textcolor{PineGreen}{\selectlanguage{french}Parelle sauvage, oseille crépue, patience crépue, patience sauvage, \textit{Rumex crispus}. Cette plante constitue l'une des trois sortes de fourrage utilisées pour les cochons; elle est aussi consommée par les humains.}  ¶ \textcolor{darkblue}{\textbf{\ipa{hwɤ˧tɕi˥-bæ˩bæ˩}}} \zh{同上} \textcolor{Sepia}{\selectlanguage{english}same meaning} \textcolor{PineGreen}{\selectlanguage{french}même sens}  
 ¶ \textcolor{darkblue}{\textbf{\ipa{hwɤ˧tɕʰi˥-ʁo˩bv̩˩}}} \zh{土大黄的嫩芽} \textcolor{Sepia}{\selectlanguage{english}sprouts of curly dock} \textcolor{PineGreen}{\selectlanguage{french}pousses de parelle sauvage}  
 \zh{量词}: \textcolor{darkblue}{\textbf{\ipa{po˧}}} 
\lhead{\firstmark}
\rhead{\botmark}

\subsection{\hspace{-0.5cm} {\Large \textcolor{darkblue}{\textbf{\ipa{hwɤ˧zo\#˥}}}}\hspace{0.5cm}[\kern2pt{\textcolor{darkblue}{\textbf{\ipa{hwɤ˩zo˥}}}}\kern2pt]} \hypertarget{hw7\string_Mzo\#\string_T1}{}
\markboth{\textcolor{darkblue}{\textbf{\ipa{hwɤ˧zo\#˥}}}}{}
\textcolor{teal}{\zh{名词}} \hspace{4pt} \zh{声调类:} \#H.
\zh{小猫。} \textcolor{Sepia}{\selectlanguage{english}Kitten.} \textcolor{PineGreen}{\selectlanguage{french}Chaton.}  ¶ \textcolor{darkblue}{\textbf{\ipa{hwɤ˧zo˧ tʰv̩˧-ɭɯ\#˥}}} \zh{那个小猫} \textcolor{Sepia}{\selectlanguage{english}\mytextsc{n}+\mytextsc{dem}+\mytextsc{clf}} \textcolor{PineGreen}{\selectlanguage{french}\mytextsc{n}+\mytextsc{dem}+\mytextsc{clf}}  
 ¶ \textcolor{darkblue}{\textbf{\ipa{hwɤ˧zo˧-hwɤ˧mi˥}}} \zh{猫,包括小猫、母猫和公猫} \textcolor{Sepia}{\selectlanguage{english}cats, the cat family: kitten and parents} \textcolor{PineGreen}{\selectlanguage{french}chats (toute la famille: chatons et parents)}  
 \zh{量词}: \textcolor{darkblue}{\textbf{\ipa{ɭɯ˧}}} 
\lhead{\firstmark}
\rhead{\botmark}

\subsection{\hspace{-0.5cm} {\Large \textcolor{darkblue}{\textbf{\ipa{hwɤ˩}}}}\hspace{0.5cm}[\kern2pt{\textcolor{darkblue}{\textbf{\ipa{xxxx non-correspondance entre le nombre de morphèmes et le nombre de tons de morphèmes}}}}\kern2pt]} \hypertarget{hw7\string_B1}{}
\markboth{\textcolor{darkblue}{\textbf{\ipa{hwɤ˩}}}}{}
\textcolor{teal}{\zh{动词}} \hspace{4pt} \zh{声调类:} L\textsubscript{a}.
\ding{202} \zh{捆(捆成捆儿)。} \textcolor{Sepia}{\selectlanguage{english}To pack, to tie together into a bundle.} \textcolor{PineGreen}{\selectlanguage{french}Attacher, emballer, préparer un fardeau/une charge, constituer un ballot (avec de l'herbe, des objets…), mettre en botte, mettre en ballot.}  ¶ \textcolor{darkblue}{\textbf{\ipa{zɯ˧-wɤ˧ hwɤ˥}}} \zh{将草捆成一垛、捆一垛草} \textcolor{Sepia}{\selectlanguage{english}to make a bundle of hay} \textcolor{PineGreen}{\selectlanguage{french}faire un ballot d'herbes}  
 ¶ \textcolor{darkblue}{\textbf{\ipa{si˧-wɤ˧ hwɤ˥}}} \zh{将木头捆成一堆、捆一堆木头} \textcolor{Sepia}{\selectlanguage{english}to make a bundle of wood} \textcolor{PineGreen}{\selectlanguage{french}faire un ballot de bois}  
 ¶ \textcolor{darkblue}{\textbf{\ipa{hɑ˧-wɤ˧ hwɤ˥}}} \zh{将粮食捆成一包、捆一包粮食} \textcolor{Sepia}{\selectlanguage{english}to make a bundle of cut cereals} \textcolor{PineGreen}{\selectlanguage{french}faire un ballot de céréales}  
 ¶ \textcolor{darkblue}{\textbf{\ipa{wɤ˩ hwɤ˩˥}}} \zh{捆成一包} \textcolor{Sepia}{\selectlanguage{english}to tie together into a bundle, to make a bundle} \textcolor{PineGreen}{\selectlanguage{french}préparer un fardeau, mettre en ballot / mettre (des objets, des choses) en paquet, de façon à ce qu'une personne puisse le porter}  
 ¶ \textcolor{darkblue}{\textbf{\ipa{wɤ˩˥ | tʰi˧-hwɤ˩}}} \zh{然后,捆成一包!} \textcolor{Sepia}{\selectlanguage{english}and then, (we) tie (it) into a bundle!} \textcolor{PineGreen}{\selectlanguage{french}et après, on en fait un ballot / on en fait un paquet / on attache ça ensemble!}  
 ¶ \textcolor{darkblue}{\textbf{\ipa{wɤ˩˥ | ɖɯ˧-wɤ˩ hwɤ˩}}} \zh{又捆一包} \textcolor{Sepia}{\selectlanguage{english}to tie another bundle, to make (yet) another bundle} \textcolor{PineGreen}{\selectlanguage{french}faire un ballot de plus}  
\ding{203} \zh{有困难、像把自己捆起来一样。} \textcolor{Sepia}{\selectlanguage{english}To be in trouble, to put oneself into trouble (figurative sense: as if one were all tied up with a rope, unable to move, to live one's life normally).} \textcolor{PineGreen}{\selectlanguage{french}Sens figuré: avoir des embarras, se créer des embarras sur les bras, s'empêtrer.}  ¶ \textcolor{darkblue}{\textbf{\ipa{hĩ˧, | wɤ˩ hwɤ˧ ʝi˧-ni˥gv̩˩!}}} \zh{人家难受,像被捆一样} \textcolor{Sepia}{\selectlanguage{english}The people seem to be unhappy / under strain! (Figuratively: they look all tied up, as if they were tied with a rope, unable to move = to live a normal life.)} \textcolor{PineGreen}{\selectlanguage{french}Les gens, ils étaient accablés / ils étaient malheureux! (Image: les gens étaient comme ficelés, incapables de se mouvoir normalement, de vivre leur vie normalement.)}  
 ¶ \textcolor{darkblue}{\textbf{\ipa{wɤ˩hwɤ˧ ʝi˧-ni˥gv̩˩-ɲi˩-ze˩!}}} \zh{我给自己找麻烦了!} \textcolor{Sepia}{\selectlanguage{english}I have put myself in trouble! / I have made a lot of trouble for myself! / I have put my knickers in a twist!} \textcolor{PineGreen}{\selectlanguage{french}(je me) suis mis des grosses complications sur les bras!}  

\lhead{\firstmark}
\rhead{\botmark}

\subsection{\hspace{-0.5cm} {\Large \textcolor{darkblue}{\textbf{\ipa{hwɤ˩\textsubscript{a}}}}}\hspace{0.5cm}[\kern2pt{\textcolor{darkblue}{\textbf{\ipa{hwɤ˩˥}}}}\kern2pt]} \hypertarget{hw7\string_Ba1}{}
\markboth{\textcolor{darkblue}{\textbf{\ipa{hwɤ˩\textsubscript{a}}}}}{}
\textcolor{teal}{\zh{动词}} \hspace{4pt} \zh{声调类:} L\textsubscript{a}.
\zh{递过去。} \textcolor{Sepia}{\selectlanguage{english}To hand over, to pass over, to send.} \textcolor{PineGreen}{\selectlanguage{french}Passer un objet, envoyer un objet (à quelqu'un).}  ¶ \textcolor{darkblue}{\textbf{\ipa{hĩ˧-ki˧ | tso˧\textasciitilde{}tso˧ hwɤ˥}}} \zh{给人家寄东西} \textcolor{Sepia}{\selectlanguage{english}to send some stuff to someone} \textcolor{PineGreen}{\selectlanguage{french}envoyer des choses à quelqu'un}  

\lhead{\firstmark}
\rhead{\botmark}

\subsection{\hspace{-0.5cm} {\Large \textcolor{darkblue}{\textbf{\ipa{hwɤ˩dʑɯ˩}}}}\hspace{0.5cm}[\kern2pt{\textcolor{darkblue}{\textbf{\ipa{hwɤ˩dʑɯ˩˥}}}}\kern2pt]} \hypertarget{hw7\string_Bdz£M\string_B1}{}
\markboth{\textcolor{darkblue}{\textbf{\ipa{hwɤ˩dʑɯ˩}}}}{}
\textcolor{teal}{\zh{名词}} \hspace{4pt} \zh{声调类:} L.
\zh{山上过夜的小木房。} \textcolor{Sepia}{\selectlanguage{english}Cabin, hut.} \textcolor{PineGreen}{\selectlanguage{french}Cabane, hutte (qu'on construit sur la montagne quand on doit y passer qq nuits, par exemple pour couper du bois).}  \zh{量词}: \textcolor{darkblue}{\textbf{\ipa{ɭɯ˧}}} 
\lhead{\firstmark}
\rhead{\botmark}

\subsection{\hspace{-0.5cm} {\Large \textcolor{darkblue}{\textbf{\ipa{hwɤ˩kæ˧}}}}\hspace{0.5cm}[\kern2pt{\textcolor{darkblue}{\textbf{\ipa{hwɤ˩kæ˩˥}}}}\kern2pt]} \hypertarget{hw7\string_Bk\{\string_M1}{}
\markboth{\textcolor{darkblue}{\textbf{\ipa{hwɤ˩kæ˧}}}}{}
\textcolor{teal}{\zh{名词}} \hspace{4pt} \zh{声调类:} LM.
\zh{红桦树。} \textcolor{Sepia}{\selectlanguage{english}Red birch; its wood is good: it is used to make ards.} \textcolor{PineGreen}{\selectlanguage{french}Bouleau rouge; son bois est bon, on s'en sert pour fabriquer des araires.}  ¶ \textcolor{darkblue}{\textbf{\ipa{hwɤ˩kæ˧-si˧dzi˩}}} \zh{同上:红桦树} \textcolor{Sepia}{\selectlanguage{english}same meaning: red birch} \textcolor{PineGreen}{\selectlanguage{french}même sens: bouleau rouge}  

\lhead{\firstmark}
\rhead{\botmark}

\subsection{\hspace{-0.5cm} {\Large \textcolor{darkblue}{\textbf{\ipa{hwɤ˩ʈi˥}}}}\hspace{0.5cm}[\kern2pt{\textcolor{darkblue}{\textbf{\ipa{hwɤ˧ʈi˥}}}}\kern2pt]} \hypertarget{hw7\string_Bt`i\string_T1}{}
\markboth{\textcolor{darkblue}{\textbf{\ipa{hwɤ˩ʈi˥}}}}{}
\textcolor{teal}{\zh{动词}} \hspace{4pt} \zh{声调类:} LH.
\zh{生锈。} \textcolor{Sepia}{\selectlanguage{english}To become rusty, to get rusty, to rust.} \textcolor{PineGreen}{\selectlanguage{french}Rouiller.}  ¶ \textcolor{darkblue}{\textbf{\ipa{hwɤ˩ʈi˥-ze˩}}} \zh{生锈了} \textcolor{Sepia}{\selectlanguage{english}\mytextsc{pfv}: it has become rusty} \textcolor{PineGreen}{\selectlanguage{french}\mytextsc{pfv}: ça a rouillé}  

\lhead{\firstmark}
\rhead{\botmark}

\subsection{\hspace{-0.5cm} {\Large \textcolor{darkblue}{\textbf{\ipa{hwɤ˧˥}}} \textsubscript{1}}\hspace{0.5cm}[\kern2pt{\textcolor{darkblue}{\textbf{\ipa{hwɤ˥}}}}\kern2pt]} \hypertarget{hw7\string_M\string_T1}{}
\markboth{\textcolor{darkblue}{\textbf{\ipa{hwɤ˧˥}}} \textsubscript{1}}{}
\textcolor{teal}{\zh{动词}} \hspace{4pt} \zh{声调类:} MH.
\zh{补。} \textcolor{Sepia}{\selectlanguage{english}To mend, to patch.} \textcolor{PineGreen}{\selectlanguage{french}Repriser, raccommoder (vêtement).}  ¶ \textcolor{darkblue}{\textbf{\ipa{le˧-hwɤ˧˥}}} \zh{\mytextsc{accomp}} \textcolor{Sepia}{\selectlanguage{english}\mytextsc{accomp}} \textcolor{PineGreen}{\selectlanguage{french}\mytextsc{accomp}}  
 ¶ \textcolor{darkblue}{\textbf{\ipa{bɑ˩lɑ˩ hwɤ˥}}} \zh{补衣服} \textcolor{Sepia}{\selectlanguage{english}to mend clothes} \textcolor{PineGreen}{\selectlanguage{french}réparer un vêtement, recoudre un vêtement, rapetasser un vêtement}  

\lhead{\firstmark}
\rhead{\botmark}

\subsection{\hspace{-0.5cm} {\Large \textcolor{darkblue}{\textbf{\ipa{hwɤ˧˥}}} \textsubscript{2}}\hspace{0.5cm}[\kern2pt{\textcolor{darkblue}{\textbf{\ipa{hwɤ˧˥}}}}\kern2pt]} \hypertarget{hw7\string_M\string_T2}{}
\markboth{\textcolor{darkblue}{\textbf{\ipa{hwɤ˧˥}}} \textsubscript{2}}{}
\textcolor{teal}{\zh{名词}} \hspace{4pt} \zh{声调类:} MH.
\zh{猫(单音节)。} \textcolor{Sepia}{\selectlanguage{english}Cat (monosyllable).} \textcolor{PineGreen}{\selectlanguage{french}Chat (monosyllabique).}  \zh{量词}: \textcolor{darkblue}{\textbf{\ipa{mi˩}}} 
\lhead{\firstmark}
\rhead{\botmark}

\subsection{\hspace{-0.5cm} {\Large \textcolor{darkblue}{\textbf{\ipa{hwɤ˧˥}}} \textsubscript{3}}\hspace{0.5cm}[\kern2pt{\textcolor{darkblue}{\textbf{\ipa{hwɤ˧˥}}}}\kern2pt]} \hypertarget{hw7\string_M\string_T3}{}
\markboth{\textcolor{darkblue}{\textbf{\ipa{hwɤ˧˥}}} \textsubscript{3}}{}
\textcolor{teal}{\zh{名词}} \hspace{4pt} \zh{声调类:} MH.
\zh{锈(单音节)。} \textcolor{Sepia}{\selectlanguage{english}Rust (monosyllable).} \textcolor{PineGreen}{\selectlanguage{french}Rouille (monosyllabe).} 
\lhead{\firstmark}
\rhead{\botmark}

\newpage
\section*{\centering- \textcolor{darkblue}{\textbf{\ipa{ĩ}}} -}
\subsection{\hspace{-0.5cm} {\Large \textcolor{darkblue}{\textbf{\ipa{ĩ˧}}}}\hspace{0.5cm}[\kern2pt{\textcolor{darkblue}{\textbf{\ipa{ĩ˥}}}}\kern2pt]} \hypertarget{i\string_~\string_M1}{}
\markboth{\textcolor{darkblue}{\textbf{\ipa{ĩ˧}}}}{}
\textcolor{teal}{\zh{感叹词}} \hspace{4pt} \zh{声调类:} M.
\zh{是的,好的。} \textcolor{Sepia}{\selectlanguage{english}Yes, OK.} \textcolor{PineGreen}{\selectlanguage{french}Oui, d'accord.}  ¶ \textcolor{darkblue}{\textbf{\ipa{ʈʂʰɯ˧-ɳɯ˧ | “ĩ˧! ĩ˧!” | pi˧. |}}} \zh{他说:“是的,是的!”} \textcolor{Sepia}{\selectlanguage{english}(S)he said “Yes! yes!”} \textcolor{PineGreen}{\selectlanguage{french}il a dit: “Oui, oui!”}  

\lhead{\firstmark}
\rhead{\botmark}

\newpage
\section*{\centering- \textcolor{darkblue}{\textbf{\ipa{j}}} \textcolor{darkblue}{\textbf{\ipa{jæ}}} \textcolor{darkblue}{\textbf{\ipa{jɤ}}} \textcolor{darkblue}{\textbf{\ipa{jo}}} \textcolor{darkblue}{\textbf{\ipa{je}}} -}
\subsection{\hspace{-0.5cm} {\Large \textcolor{darkblue}{\textbf{\ipa{jɤ˧}}} \textsubscript{1}}\hspace{0.5cm}[\kern2pt{\textcolor{darkblue}{\textbf{\ipa{jɤ˥}}}}\kern2pt]} \hypertarget{j7\string_M1}{}
\markboth{\textcolor{darkblue}{\textbf{\ipa{jɤ˧}}} \textsubscript{1}}{}
\textcolor{teal}{\zh{形容词}} \hspace{4pt} \zh{声调类:} M.
\zh{好(只出现在否定词后面)。} \textcolor{Sepia}{\selectlanguage{english}Good (only appears in negative construction).} \textcolor{PineGreen}{\selectlanguage{french}Bien (ne s'utilise qu'en tournure négative).}  ¶ \textcolor{darkblue}{\textbf{\ipa{mɤ˧-jɤ˧}}} \zh{不好(形容一个人的行为)} \textcolor{Sepia}{\selectlanguage{english}\mytextsc{neg}: it's not good! It's not right! (About someone's behaviour)} \textcolor{PineGreen}{\selectlanguage{french}\mytextsc{neg}: ce n'est pas bien! / c'est pas beau, ça! (Au sujet du comportement de quelqu'un)}  

\lhead{\firstmark}
\rhead{\botmark}

\subsection{\hspace{-0.5cm} {\Large \textcolor{darkblue}{\textbf{\ipa{jɤ˧}}} \textsubscript{2}}\hspace{0.5cm}[\kern2pt{\textcolor{darkblue}{\textbf{\ipa{jɤ˥}}}}\kern2pt]} \hypertarget{j7\string_M2}{}
\markboth{\textcolor{darkblue}{\textbf{\ipa{jɤ˧}}} \textsubscript{2}}{}
\textcolor{teal}{\zh{形容词}} \hspace{4pt} \zh{声调类:} M.
\zh{平(土地)。} \textcolor{Sepia}{\selectlanguage{english}Flat.} \textcolor{PineGreen}{\selectlanguage{french}Plat.}  ¶ \textcolor{darkblue}{\textbf{\ipa{mɤ˧-jɤ˧}}} \zh{不平} \textcolor{Sepia}{\selectlanguage{english}\mytextsc{neg}: not flat; uneven} \textcolor{PineGreen}{\selectlanguage{french}\mytextsc{neg}: pas plat; inégal}  

\lhead{\firstmark}
\rhead{\botmark}

\subsection{\hspace{-0.5cm} {\Large \textcolor{darkblue}{\textbf{\ipa{jɤ˧}}} \textsubscript{3}}\hspace{0.5cm}[\kern2pt{\textcolor{darkblue}{\textbf{\ipa{jɤ˥}}}}\kern2pt]} \hypertarget{j7\string_M3}{}
\markboth{\textcolor{darkblue}{\textbf{\ipa{jɤ˧}}} \textsubscript{3}}{}
\textcolor{teal}{\zh{名词}} \hspace{4pt} \zh{声调类:} M.
\zh{烟。} \textcolor{Sepia}{\selectlanguage{english}Tobacco, cigarettes.} \textcolor{PineGreen}{\selectlanguage{french}Tabac, cigarette.}  \zh{【借词】} \zh{烟?}
 ¶ \textcolor{darkblue}{\textbf{\ipa{jɤ˧ ʈʰɯ˩}}} \zh{抽烟} \textcolor{Sepia}{\selectlanguage{english}to smoke tobacco} \textcolor{PineGreen}{\selectlanguage{french}fumer}  
 \zh{量词}: \textcolor{darkblue}{\textbf{\ipa{ko˧}}} 
\lhead{\firstmark}
\rhead{\botmark}

\subsection{\hspace{-0.5cm} {\Large \textcolor{darkblue}{\textbf{\ipa{jɤ˧gɯ˩}}}}\hspace{0.5cm}[\kern2pt{\textcolor{darkblue}{\textbf{\ipa{jɤ˧gɯ˩}}}}\kern2pt]} \hypertarget{j7\string_MgM\string_B1}{}
\markboth{\textcolor{darkblue}{\textbf{\ipa{jɤ˧gɯ˩}}}}{}
\textcolor{teal}{\zh{名词}} \hspace{4pt} \zh{声调类:} L\#.
\zh{甜荞/荞麦/花荞。} \textcolor{Sepia}{\selectlanguage{english}Buckwheat, \textit{Fagopyrum esculentum}.} \textcolor{PineGreen}{\selectlanguage{french}Sarrasin, blé noir, \textit{Fagopyrum esculentum}.}  \zh{量词}: \textcolor{darkblue}{\textbf{\ipa{kɤ˧˥}}} \zh{~【参考】~} \hyperlink{}{\textcolor{darkblue}{\textbf{\ipa{jɤ˧qʰɑ\#˥}}}} 
\lhead{\firstmark}
\rhead{\botmark}

\subsection{\hspace{-0.5cm} {\Large \textcolor{darkblue}{\textbf{\ipa{jɤ˧ŋɤ˧}}}}\hspace{0.5cm}[\kern2pt{\textcolor{darkblue}{\textbf{\ipa{jɤ˧ŋɤ˧}}}}\kern2pt]} \hypertarget{j7\string_MN7\string_M1}{}
\markboth{\textcolor{darkblue}{\textbf{\ipa{jɤ˧ŋɤ˧}}}}{}
\textcolor{teal}{\zh{名词}} \hspace{4pt} \zh{声调类:} M.
\zh{成都。} \textcolor{Sepia}{\selectlanguage{english}The city of Chengdu, in Sichuan.} \textcolor{PineGreen}{\selectlanguage{french}La ville de Chengdu, dans le Sichuan.}  ¶ \textcolor{darkblue}{\textbf{\ipa{ho˧di˧-jɤ˧ŋɤ˧}}} \zh{同上} \textcolor{Sepia}{\selectlanguage{english}same meaning} \textcolor{PineGreen}{\selectlanguage{french}même sens}  

\lhead{\firstmark}
\rhead{\botmark}

\subsection{\hspace{-0.5cm} {\Large \textcolor{darkblue}{\textbf{\ipa{jɤ˧qʰɑ\#˥}}}}\hspace{0.5cm}[\kern2pt{\textcolor{darkblue}{\textbf{\ipa{jɤ˧qʰɑ˧}}}}\kern2pt]} \hypertarget{j7\string_Mq\string_hA\#\string_T1}{}
\markboth{\textcolor{darkblue}{\textbf{\ipa{jɤ˧qʰɑ\#˥}}}}{}
\textcolor{teal}{\zh{名词}} \hspace{4pt} \zh{声调类:} \#H.
\zh{苦荞。} \textcolor{Sepia}{\selectlanguage{english}Bitter buckwheat, \textit{Fagopyrum tataricum Gaertn}.} \textcolor{PineGreen}{\selectlanguage{french}Sarrasin amer, \textit{Fagopyrum tataricum Gaertn}.}  \zh{量词}: \textcolor{darkblue}{\textbf{\ipa{kɤ˧˥}}} \zh{~【参考】~} \hyperlink{}{\textcolor{darkblue}{\textbf{\ipa{jɤ˧gɯ˩}}}} 
\lhead{\firstmark}
\rhead{\botmark}

\subsection{\hspace{-0.5cm} {\Large \textcolor{darkblue}{\textbf{\ipa{jɤ˧qʰɑ˧-pɤ˥jɤ˩-mo˩}}}}\hspace{0.5cm}[\kern2pt{\textcolor{darkblue}{\textbf{\ipa{xxxx non-correspondance entre le nombre de morphèmes et le nombre de tons de morphèmes}}}}\kern2pt]} \hypertarget{j7\string_Mq\string_hA\string_M-p7\string_Tj7\string_B-mo\string_B1}{}
\markboth{\textcolor{darkblue}{\textbf{\ipa{jɤ˧qʰɑ˧-pɤ˥jɤ˩-mo˩}}}}{}
\textcolor{teal}{\zh{名词}} \hspace{4pt} \zh{声调类:} \#H\mytextsc{}.
\zh{牛肝菌。} \textcolor{Sepia}{\selectlanguage{english}Cep, penny bun, porcino, \textit{Boletus edulis} (a type of edible fungus); literally “buckwheat bun mushroom”, due to its texture.} \textcolor{PineGreen}{\selectlanguage{french}Bolet, cèpe, \textit{Boletus edulis}; littéralement “champignon-galette de sarrasin”, du fait de sa texture.} \zh{~【参考】~} \hyperlink{}{\textcolor{darkblue}{\textbf{\ipa{njo˩kæ˧tɕi˩˥}}}} 
\lhead{\firstmark}
\rhead{\botmark}

\subsection{\hspace{-0.5cm} {\Large \textcolor{darkblue}{\textbf{\ipa{jɤ˧wo˧˥}}}}\hspace{0.5cm}[\kern2pt{\textcolor{darkblue}{\textbf{\ipa{jɤ˧wo˧˥}}}}\kern2pt]} \hypertarget{j7\string_Mwo\string_M\string_T1}{}
\markboth{\textcolor{darkblue}{\textbf{\ipa{jɤ˧wo˧˥}}}}{}
\textcolor{teal}{\zh{动词}} \hspace{4pt} \zh{声调类:} MH\#.
\zh{倒退、退步。} \textcolor{Sepia}{\selectlanguage{english}To regress.} \textcolor{PineGreen}{\selectlanguage{french}Régresser (le contraire de: faire des progrès).}  ¶ \textcolor{darkblue}{\textbf{\ipa{no˧ | jɤ˧wo˧˥ | sɯ˧ɖʐæ˧! / no˧ | le˧-wo˥ | sɯ˧ɖʐæ˧!}}} \zh{你这是在退步!(情景:一个小孩已经几个礼拜有了上厕所的习惯,那天又把屎拉在裤头里)} \textcolor{Sepia}{\selectlanguage{english}You are regressing! (Said to a child who had already developed a habit of going to the loo in the previous weeks, but who, that day, pooed in her trousers.)} \textcolor{PineGreen}{\selectlanguage{french}tu régresses! (adressé à un petit enfant qui a fait caca dans sa culotte, alors que depuis plusieurs semaines il avait pris l'habitude du pot)}  

\lhead{\firstmark}
\rhead{\botmark}

\subsection{\hspace{-0.5cm} {\Large \textcolor{darkblue}{\textbf{\ipa{jɤ˩\textsubscript{a}}}} \textsubscript{1}}\hspace{0.5cm}[\kern2pt{\textcolor{darkblue}{\textbf{\ipa{jɤ˩˥}}}}\kern2pt]} \hypertarget{j7\string_Ba1}{}
\markboth{\textcolor{darkblue}{\textbf{\ipa{jɤ˩\textsubscript{a}}}} \textsubscript{1}}{}
\textcolor{teal}{\zh{动词}} \hspace{4pt} \zh{声调类:} L\textsubscript{a}.
\zh{盘、盘绕(线)。} \textcolor{Sepia}{\selectlanguage{english}To coil.} \textcolor{PineGreen}{\selectlanguage{french}Enrouler (ex.: des fibres de lin).}  ¶ \textcolor{darkblue}{\textbf{\ipa{sɑ˧ jɤ˥}}} \zh{盘麻线} \textcolor{Sepia}{\selectlanguage{english}to coil linen thread} \textcolor{PineGreen}{\selectlanguage{french}enrouler du fil de lin}  
 ¶ \textcolor{darkblue}{\textbf{\ipa{sɑ˧ | le˧-jɤ˩}}} \zh{盘麻线} \textcolor{Sepia}{\selectlanguage{english}to coil linen thread} \textcolor{PineGreen}{\selectlanguage{french}enrouler du fil de lin}  
\zh{~【参考】~} \hyperlink{}{\textcolor{darkblue}{\textbf{\ipa{tɕɯ˧ɭɯ˧}}}} 
\lhead{\firstmark}
\rhead{\botmark}

\subsection{\hspace{-0.5cm} {\Large \textcolor{darkblue}{\textbf{\ipa{jɤ˩\textsubscript{a}}}} \textsubscript{2}}\hspace{0.5cm}[\kern2pt{\textcolor{darkblue}{\textbf{\ipa{jɤ˩˥}}}}\kern2pt]} \hypertarget{j7\string_Ba2}{}
\markboth{\textcolor{darkblue}{\textbf{\ipa{jɤ˩\textsubscript{a}}}} \textsubscript{2}}{}
\textcolor{teal}{\zh{形容词}} \hspace{4pt} \zh{声调类:} L\textsubscript{a}.
\zh{(煮)烂。} \textcolor{Sepia}{\selectlanguage{english}Overcooked, overdone, mushy, sodden, mushed.} \textcolor{PineGreen}{\selectlanguage{french}Trop cuit, dissous, tout décomposé: ex.: pommes de terre qui éclatent, pois qui se décomposent en purée.}  ¶ \textcolor{darkblue}{\textbf{\ipa{le˧-tɕɤ˧˥ | le˧-jɤ˩-ze˩!}}} \zh{煮烂了!} \textcolor{Sepia}{\selectlanguage{english}It got sodden after boiling! / After boiling, it got all mushy/overdone!} \textcolor{PineGreen}{\selectlanguage{french}ça s'est décomposé à force de bouillir!}  
 ¶ \textcolor{darkblue}{\textbf{\ipa{jɤ˩-hĩ˩˥}}} \zh{烂的} \textcolor{Sepia}{\selectlanguage{english}\mytextsc{rel}/\mytextsc{nmlz}} \textcolor{PineGreen}{\selectlanguage{french}\mytextsc{rel}/\mytextsc{nmlz}}  

\lhead{\firstmark}
\rhead{\botmark}

\subsection{\hspace{-0.5cm} {\Large \textcolor{darkblue}{\textbf{\ipa{jɤ˩\textsubscript{b}}}} \textsubscript{1}}\hspace{0.5cm}[\kern2pt{\textcolor{darkblue}{\textbf{\ipa{jɤ˩˥}}}}\kern2pt]} \hypertarget{j7\string_Bb1}{}
\markboth{\textcolor{darkblue}{\textbf{\ipa{jɤ˩\textsubscript{b}}}} \textsubscript{1}}{}
\textcolor{teal}{\zh{动词}} \hspace{4pt} \zh{声调类:} L\textsubscript{b}.
\zh{没精神。} \textcolor{Sepia}{\selectlanguage{english}To be listless, to be dejected.} \textcolor{PineGreen}{\selectlanguage{french}Être fatigué, être sans entrain, être à la masse.}  ¶ \textcolor{darkblue}{\textbf{\ipa{tʰi˧-jɤ˩-ho˩-ze˩!}}} \zh{他没精神了!} \textcolor{Sepia}{\selectlanguage{english}(S)he is getting listless/dispirited!} \textcolor{PineGreen}{\selectlanguage{french}(Il/elle) est à la masse!}  
 ¶ \textcolor{darkblue}{\textbf{\ipa{ɑ˩ʁo˧ ʂv̩˧ɖv̩˧ | tʰi˧-jɤ˩-ho˩-tsɯ˩!}}} \zh{想家的时候,没精神!} \textcolor{Sepia}{\selectlanguage{english}When one misses home, one gets listless/dispirited!} \textcolor{PineGreen}{\selectlanguage{french}Quand on a la nostalgie, on est sans entrain!}  
 ¶ \textcolor{darkblue}{\textbf{\ipa{ɑ˩ʁo˧ ʂv̩˧ɖv̩˧-zo˥, | tʰi˧-jɤ˩-ho˩!}}} \zh{想家的时候,没精神!} \textcolor{Sepia}{\selectlanguage{english}When one misses home, one gets listless/dispirited!} \textcolor{PineGreen}{\selectlanguage{french}Quand on a la nostalgie, on est sans entrain!}  
 ¶ \textcolor{darkblue}{\textbf{\ipa{ɲi˧mi˧ tsʰi˧-zo˩, | tʰi˧-jɤ˩-ho˩!}}} \zh{天气很热,没精神!} \textcolor{Sepia}{\selectlanguage{english}When the weather is hot, one gets listless/dispirited!} \textcolor{PineGreen}{\selectlanguage{french}Quand il fait très chaud, on est sans entrain!}  
 ¶ \textcolor{darkblue}{\textbf{\ipa{jɤ˩-mɤ˥-jɤ˩}}} \zh{\string_ \mytextsc{neg} \string_} \textcolor{Sepia}{\selectlanguage{english}\string_ \mytextsc{neg} \string_} \textcolor{PineGreen}{\selectlanguage{french}\string_ \mytextsc{neg} \string_}  

\lhead{\firstmark}
\rhead{\botmark}

\subsection{\hspace{-0.5cm} {\Large \textcolor{darkblue}{\textbf{\ipa{jɤ˩\textsubscript{b}}}} \textsubscript{2}}\hspace{0.5cm}[\kern2pt{\textcolor{darkblue}{\textbf{\ipa{jɤ˩˥}}}}\kern2pt]} \hypertarget{j7\string_Bb2}{}
\markboth{\textcolor{darkblue}{\textbf{\ipa{jɤ˩\textsubscript{b}}}} \textsubscript{2}}{}
\textcolor{teal}{\zh{量词}} \hspace{4pt} \zh{声调类:} L\textsubscript{b}.
\zh{量词:排(一排菜)。} \textcolor{Sepia}{\selectlanguage{english}Row: classifier for rows of vegetables.} \textcolor{PineGreen}{\selectlanguage{french}Classificateur des rangées de légumes (dans un potager, dans un champ).}  ¶ \textcolor{darkblue}{\textbf{\ipa{v˩tsʰɤ˧˥ | ɖɯ˧-jɤ˩ tʰi˩-pʰo˩}}} \zh{种一排菜} \textcolor{Sepia}{\selectlanguage{english}to plant a row of vegetables} \textcolor{PineGreen}{\selectlanguage{french}planter une rangée de légumes}  

\lhead{\firstmark}
\rhead{\botmark}

\subsection{\hspace{-0.5cm} {\Large \textcolor{darkblue}{\textbf{\ipa{jɤ˩ɕjo˧-dzɑ˧qʰwɤ˩}}}}\hspace{0.5cm}[\kern2pt{\textcolor{darkblue}{\textbf{\ipa{jɤ˩ɕjo˧dzɑ˧qʰwɤ˩}}}}\kern2pt]} \hypertarget{j7\string_Bs£jo\string_M-dzA\string_Mq\string_hw7\string_B1}{}
\markboth{\textcolor{darkblue}{\textbf{\ipa{jɤ˩ɕjo˧-dzɑ˧qʰwɤ˩}}}}{}
\textcolor{teal}{\zh{名词}} \hspace{4pt} \zh{声调类:} LM-L\#.
\zh{凉鞋。} \textcolor{Sepia}{\selectlanguage{english}Sandal.} \textcolor{PineGreen}{\selectlanguage{french}Sandale.}  \zh{量词}: \textcolor{darkblue}{\textbf{\ipa{dzi˧}}} 
\lhead{\firstmark}
\rhead{\botmark}

\subsection{\hspace{-0.5cm} {\Large \textcolor{darkblue}{\textbf{\ipa{jɤ˩ho˧}}}}\hspace{0.5cm}[\kern2pt{\textcolor{darkblue}{\textbf{\ipa{jɤ˩ho˥}}}}\kern2pt]} \hypertarget{j7\string_Bho\string_M1}{}
\markboth{\textcolor{darkblue}{\textbf{\ipa{jɤ˩ho˧}}}}{}
\textcolor{teal}{\zh{名词}} \hspace{4pt} \zh{声调类:} LM.
\zh{火柴(洋火)。} \textcolor{Sepia}{\selectlanguage{english}Matches.} \textcolor{PineGreen}{\selectlanguage{french}Allumette.}  \zh{【借词】} \zh{洋火}
 \zh{量词}: \textcolor{darkblue}{\textbf{\ipa{po˩}}} 
\lhead{\firstmark}
\rhead{\botmark}

\subsection{\hspace{-0.5cm} {\Large \textcolor{darkblue}{\textbf{\ipa{jɤ˩jo\#˥}}}}\hspace{0.5cm}[\kern2pt{\textcolor{darkblue}{\textbf{\ipa{jɤ˩jo˥}}}}\kern2pt]} \hypertarget{j7\string_Bjo\#\string_T1}{}
\markboth{\textcolor{darkblue}{\textbf{\ipa{jɤ˩jo\#˥}}}}{}
\textcolor{teal}{\zh{名词}} \hspace{4pt} \zh{声调类:} LM+\#H.
\zh{洋芋、土豆 、马铃薯(汉语借词)。} \textcolor{Sepia}{\selectlanguage{english}Potato.} \textcolor{PineGreen}{\selectlanguage{french}Pomme de terre.}  \zh{【借词】} \zh{洋芋}
 \zh{量词}: \textcolor{darkblue}{\textbf{\ipa{kɤ˧˥}}} 
\lhead{\firstmark}
\rhead{\botmark}

\subsection{\hspace{-0.5cm} {\Large \textcolor{darkblue}{\textbf{\ipa{jɤ˩jo˧-bv̩\#˥}}}}\hspace{0.5cm}[\kern2pt{\textcolor{darkblue}{\textbf{\ipa{xxxx non-correspondance entre le nombre de morphèmes et le nombre de tons de morphèmes}}}}\kern2pt]} \hypertarget{j7\string_Bjo\string_M-bv\string_=\#\string_T1}{}
\markboth{\textcolor{darkblue}{\textbf{\ipa{jɤ˩jo˧-bv̩\#˥}}}}{}
\textcolor{teal}{\zh{名词}} \hspace{4pt} \zh{声调类:} LM+\#H.
\zh{蛴螬。} \textcolor{Sepia}{\selectlanguage{english}Potato grub, \textit{Agriotes lineatus}.} \textcolor{PineGreen}{\selectlanguage{french}Larve de taupin, ver fil de fer, \textit{Agriotes lineatus}: ver qui mange les tubercules.}  \zh{量词}: \textcolor{darkblue}{\textbf{\ipa{mi˩}}} 
\lhead{\firstmark}
\rhead{\botmark}

\subsection{\hspace{-0.5cm} {\Large \textcolor{darkblue}{\textbf{\ipa{jɤ˩pæ˧sɯ˥\$}}}}\hspace{0.5cm}[\kern2pt{\textcolor{darkblue}{\textbf{\ipa{xxxx ton non trouvé, à faire manuellement...}}}}\kern2pt]} \hypertarget{j7\string_Bp\{\string_MsM\string_T\$1}{}
\markboth{\textcolor{darkblue}{\textbf{\ipa{jɤ˩pæ˧sɯ˥\$}}}}{}
\textcolor{teal}{\zh{名词}} \hspace{4pt} \zh{声调类:} LM+H\$.
\zh{杨把事。这个姓,由两部分组成的:‘杨’姓(汉语借词)与封建社会最小领导层次:‘把事’。} \textcolor{Sepia}{\selectlanguage{english}Yang Chieftain: a family name from Yongning, containing a name borrowed from Chinese (Yang \zh{杨}) plus a term referring to the lowest degree in the hierarchy of feudal leaders: the hamlet chieftain, \zh{把事}. Only one family in Yongning carries this name.} \textcolor{PineGreen}{\selectlanguage{french}Petit Chef Yang: nom de famille constitué de l'expression chinoise \zh{杨把事}, formé du patronyme \zh{杨}, suivi du terme chinois renvoyant au plus bas degré de la hiérarchie féodale: le chef de hameau, \zh{把事}. Ce nom est propre à une seule famille de Yongning.}  ¶ \textcolor{darkblue}{\textbf{\ipa{jɤ˩pæ˧sɯ˧=ɻ̍˥\$}}} \zh{杨把事家族} \textcolor{Sepia}{\selectlanguage{english}\string_ \mytextsc{associative}: the people of the Yang Chieftain family} \textcolor{PineGreen}{\selectlanguage{french}\string_ \mytextsc{associatif}: les gens de la famille Petit Chef Yang}  
 ¶ \textcolor{darkblue}{\textbf{\ipa{jɤ˩pɑ˧sɯ˥ | ʈæ˧ʂɯ˧}}} \zh{杨把事家的一个人的名字:杨把事•达石} \textcolor{Sepia}{\selectlanguage{english}the proper name of a person of the Yang Chieftain family (given name: Dashi): 'Dashi of the Yang Chieftain family'.} \textcolor{PineGreen}{\selectlanguage{french}nom propre d'une personne de la famille Petit Chef Yang: 'Dashi de la famille Petit Chef Yang'.}  
\zh{~【参考】~} \hyperlink{}{\textcolor{darkblue}{\textbf{\ipa{pæ˧sɯ˧}}}} 
\lhead{\firstmark}
\rhead{\botmark}

\subsection{\hspace{-0.5cm} {\Large \textcolor{darkblue}{\textbf{\ipa{jɤ˩po˧}}}}\hspace{0.5cm}[\kern2pt{\textcolor{darkblue}{\textbf{\ipa{jɤ˩po˥}}}}\kern2pt]} \hypertarget{j7\string_Bpo\string_M1}{}
\markboth{\textcolor{darkblue}{\textbf{\ipa{jɤ˩po˧}}}}{}
\textcolor{teal}{\zh{动词}} \hspace{4pt} \zh{声调类:} LM.
\zh{赌博、打赌。} \textcolor{Sepia}{\selectlanguage{english}To gamble, to bet, to wager.} \textcolor{PineGreen}{\selectlanguage{french}Parier.} \textcolor{PineGreen}{\selectlanguage{french}Parier.} 
\lhead{\firstmark}
\rhead{\botmark}

\subsection{\hspace{-0.5cm} {\Large \textcolor{darkblue}{\textbf{\ipa{jɤ˩tʰi˧-ʁæ˩bæ˩}}}}\hspace{0.5cm}[\kern2pt{\textcolor{darkblue}{\textbf{\ipa{jɤ˩tʰi˧ʁæ˩bæ˩}}}}\kern2pt]} \hypertarget{j7\string_Bt\string_hi\string_M-R\{\string_Bb\{\string_B1}{}
\markboth{\textcolor{darkblue}{\textbf{\ipa{jɤ˩tʰi˧-ʁæ˩bæ˩}}}}{}
\textcolor{teal}{\zh{名词}} \hspace{4pt} \zh{声调类:} LM-L.
\zh{瓷盘。} \textcolor{Sepia}{\selectlanguage{english}Porcelain plate.} \textcolor{PineGreen}{\selectlanguage{french}Assiette en faïence/porcelaine.}  \zh{量词}: \textcolor{darkblue}{\textbf{\ipa{ɭɯ˧}}} 
\lhead{\firstmark}
\rhead{\botmark}

\subsection{\hspace{-0.5cm} {\Large \textcolor{darkblue}{\textbf{\ipa{jɤ˧˥}}} \textsubscript{1}}\hspace{0.5cm}[\kern2pt{\textcolor{darkblue}{\textbf{\ipa{jɤ˧˥}}}}\kern2pt]} \hypertarget{j7\string_M\string_T1}{}
\markboth{\textcolor{darkblue}{\textbf{\ipa{jɤ˧˥}}} \textsubscript{1}}{}
\textcolor{teal}{\zh{动词}} \hspace{4pt} \zh{声调类:} MH.
\zh{舔。} \textcolor{Sepia}{\selectlanguage{english}To lick.} \textcolor{PineGreen}{\selectlanguage{french}Lécher.}  ¶ \textcolor{darkblue}{\textbf{\ipa{tso˧\textasciitilde{}tso˧ jɤ˩}}} \zh{舔东西} \textcolor{Sepia}{\selectlanguage{english}to lick something} \textcolor{PineGreen}{\selectlanguage{french}lécher quelque chose}  
 ¶ \textcolor{darkblue}{\textbf{\ipa{dzɯ˧-di˧ jɤ˥}}} \zh{舔食品} \textcolor{Sepia}{\selectlanguage{english}to lick food} \textcolor{PineGreen}{\selectlanguage{french}lécher de la nourriture}  
 ¶ \textcolor{darkblue}{\textbf{\ipa{[F5] tso˧tso˧ ɖɯ˧-kʰwɤ˥ jɤ˩-ze˩}}} \zh{他舔了一个东西。} \textcolor{Sepia}{\selectlanguage{english}(S)he has licked something.} \textcolor{PineGreen}{\selectlanguage{french}(elle/il) a léché quelque chose}  

\lhead{\firstmark}
\rhead{\botmark}

\subsection{\hspace{-0.5cm} {\Large \textcolor{darkblue}{\textbf{\ipa{jɤ˧˥}}} \textsubscript{2}}\hspace{0.5cm}[\kern2pt{\textcolor{darkblue}{\textbf{\ipa{jɤ˧˥}}}}\kern2pt]} \hypertarget{j7\string_M\string_T2}{}
\markboth{\textcolor{darkblue}{\textbf{\ipa{jɤ˧˥}}} \textsubscript{2}}{}
\textcolor{teal}{\zh{名词}} \hspace{4pt} \zh{声调类:} MH.
\zh{红萝卜菜:一种山上的野菜。春天的时候,菜园的蔬菜还没有成熟的时候,永宁的人吃红萝卜菜。彝族在高山上采下来,在永宁卖。} \textcolor{Sepia}{\selectlanguage{english}A wild radish that grows on the mountains; it is edible; it is picked and eaten in the Spring, when vegetables are not ripe yet. Yi people harvest it and sell it in the plain.} \textcolor{PineGreen}{\selectlanguage{french}Radis sauvage qui pousse en montagne; on le consomme surtout au printemps, à une époque où il n'y a pas encore de légumes. Ce radis est récoltée par les Yi et vendu dans la plaine.} \zh{当地汉语方言:}\zh{野山菜,\textcolor{darkblue}{\textbf{\ipa{/ʝi˧ʂæ˧tsʰɤ˩/}}}。} ¶ \textcolor{darkblue}{\textbf{\ipa{jɤ˧ dzɯ˧ | qʰɑ˧-sɯ˥\textasciitilde{}sɯ˩, | jɤ˧ ʈʂɤ˥ ŋv̩˩-ɭɯ˩\textasciitilde{}ɭɯ˩!}}} \zh{“红萝卜菜,味道苦,去摘也要流眼泪! / 红萝卜菜,吃起来苦,摘起来也苦!”摘红萝卜菜,需要爬高山,寻找时间长,永宁坝子的农民觉得这比较苦。} \textcolor{Sepia}{\selectlanguage{english}“The wild radish is bitter; and its harvest costs tears! / The wild radish tastes bitter; and its harvest is bitter, too! / The wild radish is all bitterness inside, and all bitterness at the harvest!” This proverb evokes the difficulty of the harvest, which requires long wanderings up high on the mountain.} \textcolor{PineGreen}{\selectlanguage{french}“Le radis sauvage, ça a un goût amer quand on le mange, et ça vous fait pleurer pour le récolter!” (de: qʰɑ˧ 'amer'+expressif) / “récolter le radis sauvage, ça fait pleurer!” Non pas à cause de la plante elle-même, pas comme un oignon qui piquerait les yeux: mais parce qu'on s'épuisait à aller le chercher en haute montagne.}  
\zh{~【参考】~} \hyperlink{}{\textcolor{darkblue}{\textbf{\ipa{ʝi˧ʂæ˧tsʰɤ˩}}}} 
\lhead{\firstmark}
\rhead{\botmark}

\subsection{\hspace{-0.5cm} {\Large \textcolor{darkblue}{\textbf{\ipa{jɤ˧˥}}} \textsubscript{3}}\hspace{0.5cm}[\kern2pt{\textcolor{darkblue}{\textbf{\ipa{jɤ˧˥}}}}\kern2pt]} \hypertarget{j7\string_M\string_T3}{}
\markboth{\textcolor{darkblue}{\textbf{\ipa{jɤ˧˥}}} \textsubscript{3}}{}
\textcolor{teal}{\zh{动词}} \hspace{4pt} \zh{声调类:} MH.
\zh{抹、涂抹。} \textcolor{Sepia}{\selectlanguage{english}To spread, to put on, to smear.} \textcolor{PineGreen}{\selectlanguage{french}Étendre, appliquer, mettre (ex.: appliquer un onguent).}  ¶ \textcolor{darkblue}{\textbf{\ipa{pʰv˧ʂɯ˧ jɤ˧˥}}} \zh{抹防晒霜} \textcolor{Sepia}{\selectlanguage{english}to put on beauty cream or sunscreen} \textcolor{PineGreen}{\selectlanguage{french}appliquer une crème de beauté ou de la crème solaire}  
 ¶ \textcolor{darkblue}{\textbf{\ipa{mɤ˩ jɤ˩˥}}} \zh{涂抹油} \textcolor{Sepia}{\selectlanguage{english}to apply grease (e.g. to the skin)} \textcolor{PineGreen}{\selectlanguage{french}appliquer de la graisse (ex.: sur une peau sèche)}  
 ¶ \textcolor{darkblue}{\textbf{\ipa{tʰi˧-jɤ˧˥}}} \textcolor{PineGreen}{\selectlanguage{french}\mytextsc{dur} \string_}  

\lhead{\firstmark}
\rhead{\botmark}

\subsection{\hspace{-0.5cm} {\Large \textcolor{darkblue}{\textbf{\ipa{jɤ˧˥\textsubscript{a}}}} \textsubscript{1}}\hspace{0.5cm}[\kern2pt{\textcolor{darkblue}{\textbf{\ipa{jɤ˧˥}}}}\kern2pt]} \hypertarget{j7\string_M\string_Ta1}{}
\markboth{\textcolor{darkblue}{\textbf{\ipa{jɤ˧˥\textsubscript{a}}}} \textsubscript{1}}{}
\textcolor{teal}{\zh{量词}} \hspace{4pt} \zh{声调类:} MH\textsubscript{a}.
\zh{量词:母性、雌性(人或动物)(一个/一只)。} \textcolor{Sepia}{\selectlanguage{english}Classifier used for women, and for some female domestic animals; it does not carry any hint of deprecation, nor does it convey any hint of respect by itself.} \textcolor{PineGreen}{\selectlanguage{french}Classificateur des créatures femelles; employé pour les personnes de sexe féminin (appellation qui ne marque pas de respect, mais n'est pas injurieuse), et pour certains animaux domestiques.} 
\lhead{\firstmark}
\rhead{\botmark}

\subsection{\hspace{-0.5cm} {\Large \textcolor{darkblue}{\textbf{\ipa{jɤ˧˥\textsubscript{a}}}} \textsubscript{2}}\hspace{0.5cm}[\kern2pt{\textcolor{darkblue}{\textbf{\ipa{jɤ˧˥}}}}\kern2pt]} \hypertarget{j7\string_M\string_Ta2}{}
\markboth{\textcolor{darkblue}{\textbf{\ipa{jɤ˧˥\textsubscript{a}}}} \textsubscript{2}}{}
\textcolor{teal}{\zh{量词}} \hspace{4pt} \zh{声调类:} MH\textsubscript{a}.
\zh{量词:面(一团),茶饼(一个)等。(一团面,是和了一个鸡蛋的面团的量。)。} \textcolor{Sepia}{\selectlanguage{english}Classifier for dough balls and teacakes. One dough ball is the quantity of dough that can be prepared with one egg. Tea consumed in the Yongning area in the first half of the 20th century was green tea from a large leaf variety of Camellia sinensis (C. sinensis assamica) found in the mountains of southern Yunnan; it used to be pressed into 'teacake' shape.} \textcolor{PineGreen}{\selectlanguage{french}Classificateur pour la pâte à pain: quantité de pâte aux œufs que l'on peut préparer avec un œuf. Ce classificateur est également utilisé pour le thé tassé en galettes.}  ¶ \textcolor{darkblue}{\textbf{\ipa{æ˩ʁv̩˩-pɤ˥jɤ˩ | ɖɯ˧-jɤ˧˥}}} \zh{一个鸡蛋面团} \textcolor{Sepia}{\selectlanguage{english}a ball of egg dough} \textcolor{PineGreen}{\selectlanguage{french}une boule de pâte à pain à l'oeuf}  
 ¶ \textcolor{darkblue}{\textbf{\ipa{ʝi˧-jɤ˧˥}}} \zh{一个团/并} \textcolor{Sepia}{\selectlanguage{english}one ball/cake} \textcolor{PineGreen}{\selectlanguage{french}une boule/galette}  

\lhead{\firstmark}
\rhead{\botmark}

\subsection{\hspace{-0.5cm} {\Large \textcolor{darkblue}{\textbf{\ipa{jo˥}}}}\hspace{0.5cm}[\kern2pt{\textcolor{darkblue}{\textbf{\ipa{jo˥}}}}\kern2pt]} \hypertarget{jo\string_T1}{}
\markboth{\textcolor{darkblue}{\textbf{\ipa{jo˥}}}}{}
\textcolor{teal}{\zh{名词}} \hspace{4pt} \zh{声调类:} \#H.
\zh{玉石。} \textcolor{Sepia}{\selectlanguage{english}Jade.} \textcolor{PineGreen}{\selectlanguage{french}Jade (matière, pierre).}  \zh{【借词】} \zh{玉}
 \zh{量词}: \textcolor{darkblue}{\textbf{\ipa{pʰo˧˥}}} 
\lhead{\firstmark}
\rhead{\botmark}

\subsection{\hspace{-0.5cm} {\Large \textcolor{darkblue}{\textbf{\ipa{jo˧}}}}\hspace{0.5cm}[\kern2pt{\textcolor{darkblue}{\textbf{\ipa{jo˥}}}}\kern2pt]} \hypertarget{jo\string_M1}{}
\markboth{\textcolor{darkblue}{\textbf{\ipa{jo˧}}}}{}
\textcolor{teal}{\zh{动词}} \hspace{4pt} \zh{声调类:} M intrans.
\zh{来。} \textcolor{Sepia}{\selectlanguage{english}To come; to come in.} \textcolor{PineGreen}{\selectlanguage{french}Venir, entrer.}  ¶ \textcolor{darkblue}{\textbf{\ipa{le˧-jo˧-ze˧!}}} \zh{来了!} \textcolor{Sepia}{\selectlanguage{english}\mytextsc{accomp} \string_ \mytextsc{pfv}: (s)he has come} \textcolor{PineGreen}{\selectlanguage{french}\mytextsc{accomp} \string_ \mytextsc{pfv}: (il/est) est arrivé(e) / est entré(e)!}  

\lhead{\firstmark}
\rhead{\botmark}

\subsection{\hspace{-0.5cm} {\Large \textcolor{darkblue}{\textbf{\ipa{jo˧gv̩˧}}}}\hspace{0.5cm}[\kern2pt{\textcolor{darkblue}{\textbf{\ipa{jo˧gv̩˧}}}}\kern2pt]} \hypertarget{jo\string_Mgv\string_=\string_M1}{}
\markboth{\textcolor{darkblue}{\textbf{\ipa{jo˧gv̩˧}}}}{}
\textcolor{teal}{\zh{名词}} \hspace{4pt} \zh{声调类:} M.
\zh{丽江(包括丽江坝子)。} \textcolor{Sepia}{\selectlanguage{english}Lijiang.} \textcolor{PineGreen}{\selectlanguage{french}Lijiang (toute la région: la ville, et la plaine environnante).} 
\lhead{\firstmark}
\rhead{\botmark}

\subsection{\hspace{-0.5cm} {\Large \textcolor{darkblue}{\textbf{\ipa{jo˧gv̩˧-ŋv̩˧lv̩˧}}}}\hspace{0.5cm}[\kern2pt{\textcolor{darkblue}{\textbf{\ipa{xxxx non-correspondance entre le nombre de morphèmes et le nombre de tons de morphèmes}}}}\kern2pt]} \hypertarget{jo\string_Mgv\string_=\string_M-Nv\string_=\string_Mlv\string_=\string_M1}{}
\markboth{\textcolor{darkblue}{\textbf{\ipa{jo˧gv̩˧-ŋv̩˧lv̩˧}}}}{}
\textcolor{teal}{\zh{名词}} \hspace{4pt} \zh{声调类:} M.
\zh{玉龙雪山。} \textcolor{Sepia}{\selectlanguage{english}Yulong snow mountain; literally 'Lijiang's snow mountain'.} \textcolor{PineGreen}{\selectlanguage{french}La montagne Yulong: principale montagne de Lijiang.} 
\lhead{\firstmark}
\rhead{\botmark}

\subsection{\hspace{-0.5cm} {\Large \textcolor{darkblue}{\textbf{\ipa{jo˧mi˧}}}}\hspace{0.5cm}[\kern2pt{\textcolor{darkblue}{\textbf{\ipa{jo˧mi˧}}}}\kern2pt]} \hypertarget{jo\string_Mmi\string_M1}{}
\markboth{\textcolor{darkblue}{\textbf{\ipa{jo˧mi˧}}}}{}
\textcolor{teal}{\zh{名词}} \hspace{4pt} \zh{声调类:} M.
\zh{母绵羊。} \textcolor{Sepia}{\selectlanguage{english}Ewe.} \textcolor{PineGreen}{\selectlanguage{french}Brebis.}  ¶ \textcolor{darkblue}{\textbf{\ipa{jo˧mi˧-po˧lo˧}}} \zh{母绵羊与公羊} \textcolor{Sepia}{\selectlanguage{english}ewe and ram} \textcolor{PineGreen}{\selectlanguage{french}brebis et bélier}  
 \zh{量词}: \textcolor{darkblue}{\textbf{\ipa{pʰo˧˥}}} 
\lhead{\firstmark}
\rhead{\botmark}

\subsection{\hspace{-0.5cm} {\Large \textcolor{darkblue}{\textbf{\ipa{jo˧mi˧-ʁwɤ˧}}}}\hspace{0.5cm}[\kern2pt{\textcolor{darkblue}{\textbf{\ipa{xxxx non-correspondance entre le nombre de morphèmes et le nombre de tons de morphèmes}}}}\kern2pt]} \hypertarget{jo\string_Mmi\string_M-Rw7\string_M1}{}
\markboth{\textcolor{darkblue}{\textbf{\ipa{jo˧mi˧-ʁwɤ˧}}}}{}
\textcolor{teal}{\zh{名词}} \hspace{4pt} \zh{声调类:} M.
\zh{有米瓦村。} \textcolor{Sepia}{\selectlanguage{english}The second village that one crosses when going from \textcolor{darkblue}{\textbf{\ipa{/qʰæ˧tɕʰi˧/}}} to \textcolor{darkblue}{\textbf{\ipa{/ʈʂo˧ʂɯ\#˥/}}}.} \textcolor{PineGreen}{\selectlanguage{french}Le second village que l'on rencontre sur le trajet entre \textcolor{darkblue}{\textbf{\ipa{/qʰæ˧tɕʰi˧/}}} et \textcolor{darkblue}{\textbf{\ipa{/ʈʂo˧ʂɯ\#˥/}}}.} 
\lhead{\firstmark}
\rhead{\botmark}

\subsection{\hspace{-0.5cm} {\Large \textcolor{darkblue}{\textbf{\ipa{jo˩}}}}\hspace{0.5cm}[\kern2pt{\textcolor{darkblue}{\textbf{\ipa{jo˥}}}}\kern2pt]} \hypertarget{jo\string_B1}{}
\markboth{\textcolor{darkblue}{\textbf{\ipa{jo˩}}}}{}
\textcolor{teal}{\zh{名词}} \hspace{4pt} \zh{声调类:} L.
\zh{绵羊。} \textcolor{Sepia}{\selectlanguage{english}Sheep.} \textcolor{PineGreen}{\selectlanguage{french}Mouton.}  ¶ \textcolor{darkblue}{\textbf{\ipa{jo˩-ɣɯ˩˥}}} \zh{羊皮} \textcolor{Sepia}{\selectlanguage{english}sheep skin} \textcolor{PineGreen}{\selectlanguage{french}peau de mouton}  
 \zh{量词}: \textcolor{darkblue}{\textbf{\ipa{pʰo˧˥}}} 
\lhead{\firstmark}
\rhead{\botmark}

\subsection{\hspace{-0.5cm} {\Large \textcolor{darkblue}{\textbf{\ipa{jo˩\textsubscript{b}}}}}\hspace{0.5cm}[\kern2pt{\textcolor{darkblue}{\textbf{\ipa{jo˩˥}}}}\kern2pt]} \hypertarget{jo\string_Bb1}{}
\markboth{\textcolor{darkblue}{\textbf{\ipa{jo˩\textsubscript{b}}}}}{}
\textcolor{teal}{\zh{量词}} \hspace{4pt} \zh{声调类:} L\textsubscript{b}.
\zh{量词:两(一两)。} \textcolor{Sepia}{\selectlanguage{english}An ounce.} \textcolor{PineGreen}{\selectlanguage{french}Unité de poids: once.} 
\lhead{\firstmark}
\rhead{\botmark}

\subsection{\hspace{-0.5cm} {\Large \textcolor{darkblue}{\textbf{\ipa{jo˩gi˩}}}}\hspace{0.5cm}[\kern2pt{\textcolor{darkblue}{\textbf{\ipa{jo˩gi˩˥}}}}\kern2pt]} \hypertarget{jo\string_Bgi\string_B1}{}
\markboth{\textcolor{darkblue}{\textbf{\ipa{jo˩gi˩}}}}{}
\textcolor{teal}{\zh{名词}} \hspace{4pt} \zh{声调类:} L.
\zh{右边。} \textcolor{Sepia}{\selectlanguage{english}Right (opposite of left).} \textcolor{PineGreen}{\selectlanguage{french}Droite (contraire de: gauche).}  ¶ \textcolor{darkblue}{\textbf{\ipa{jo˩gi˩dzɤ˩}}} \zh{右边} \textcolor{Sepia}{\selectlanguage{english}the side to the right, the right} \textcolor{PineGreen}{\selectlanguage{french}du côté droit, à droite}  
\zh{~【参考】~} \hyperlink{}{\textcolor{darkblue}{\textbf{\ipa{jo˩˧}}}} 
\lhead{\firstmark}
\rhead{\botmark}

\subsection{\hspace{-0.5cm} {\Large \textcolor{darkblue}{\textbf{\ipa{jo˩kʰv̩˩}}}}\hspace{0.5cm}[\kern2pt{\textcolor{darkblue}{\textbf{\ipa{jo˩kʰv̩˩˥}}}}\kern2pt]} \hypertarget{jo\string_Bk\string_hv\string_=\string_B1}{}
\markboth{\textcolor{darkblue}{\textbf{\ipa{jo˩kʰv̩˩}}}}{}
\textcolor{teal}{\zh{名词}} \hspace{4pt} \zh{声调类:} L.
\zh{羊年。} \textcolor{Sepia}{\selectlanguage{english}Year of the goat.} \textcolor{PineGreen}{\selectlanguage{french}Année du mouton.} 
\lhead{\firstmark}
\rhead{\botmark}

\subsection{\hspace{-0.5cm} {\Large \textcolor{darkblue}{\textbf{\ipa{jo˩lo˩}}}}\hspace{0.5cm}[\kern2pt{\textcolor{darkblue}{\textbf{\ipa{jo˩lo˩˥}}}}\kern2pt]} \hypertarget{jo\string_Blo\string_B1}{}
\markboth{\textcolor{darkblue}{\textbf{\ipa{jo˩lo˩}}}}{}
\textcolor{teal}{\zh{名词}} \hspace{4pt} \zh{声调类:} L.
\zh{右边。} \textcolor{Sepia}{\selectlanguage{english}Right (opposite of left).} \textcolor{PineGreen}{\selectlanguage{french}Droite (contraire de: gauche).} \zh{~【参考】~} \hyperlink{}{\textcolor{darkblue}{\textbf{\ipa{jo˩˧}}}} 
\lhead{\firstmark}
\rhead{\botmark}

\subsection{\hspace{-0.5cm} {\Large \textcolor{darkblue}{\textbf{\ipa{jo˩pv̩˧}}}}\hspace{0.5cm}[\kern2pt{\textcolor{darkblue}{\textbf{\ipa{jo˩pv̩˥}}}}\kern2pt]} \hypertarget{jo\string_Bpv\string_=\string_M1}{}
\markboth{\textcolor{darkblue}{\textbf{\ipa{jo˩pv̩˧}}}}{}
\textcolor{teal}{\zh{名词}} \hspace{4pt} \zh{声调类:} LM / LM+MH\#.
\zh{油布。} \textcolor{Sepia}{\selectlanguage{english}Oilcloth; tarpaulin.} \textcolor{PineGreen}{\selectlanguage{french}Toile cirée.}  \zh{【借词】} \zh{油布}
 ¶ \textcolor{darkblue}{\textbf{\ipa{jo˩pv̩˧˥}}} \zh{油布(声调变体)} \textcolor{Sepia}{\selectlanguage{english}oilcloth (tonal variant)} \textcolor{PineGreen}{\selectlanguage{french}toile cirée (variante tonale)}  
 \zh{量词}: \textcolor{darkblue}{\textbf{\ipa{tsʰi˥}}} 
\lhead{\firstmark}
\rhead{\botmark}

\subsection{\hspace{-0.5cm} {\Large \textcolor{darkblue}{\textbf{\ipa{jo˩pʰv̩˩}}}}\hspace{0.5cm}[\kern2pt{\textcolor{darkblue}{\textbf{\ipa{jo˩pʰv̩˩˥}}}}\kern2pt]} \hypertarget{jo\string_Bp\string_hv\string_=\string_B1}{}
\markboth{\textcolor{darkblue}{\textbf{\ipa{jo˩pʰv̩˩}}}}{}
\textcolor{teal}{\zh{名词}} \hspace{4pt} \zh{声调类:} L.
\zh{公绵羊。} \textcolor{Sepia}{\selectlanguage{english}Male sheep.} \textcolor{PineGreen}{\selectlanguage{french}Bélier.}  ¶ \textcolor{darkblue}{\textbf{\ipa{jo˧pʰv̩˧ tʰv̩˧-mi˥\#}}} \zh{这头公羊} \textcolor{Sepia}{\selectlanguage{english}\string_ \mytextsc{dem} \mytextsc{clf}: that ram} \textcolor{PineGreen}{\selectlanguage{french}\string_ \mytextsc{dem} \mytextsc{clf}: ce bélier}  
 \zh{量词}: \textcolor{darkblue}{\textbf{\ipa{pʰo˧˥}}} \textcolor{darkblue}{\textbf{\ipa{mi˩}}} \zh{~【参考】~} \hyperlink{}{\textcolor{darkblue}{\textbf{\ipa{po˧lo˧}}}} 
\lhead{\firstmark}
\rhead{\botmark}

\subsection{\hspace{-0.5cm} {\Large \textcolor{darkblue}{\textbf{\ipa{jo˩ʂwæ˩}}}}\hspace{0.5cm}[\kern2pt{\textcolor{darkblue}{\textbf{\ipa{jo˩ʂwæ˩˥}}}}\kern2pt]} \hypertarget{jo\string_Bs`w\{\string_B1}{}
\markboth{\textcolor{darkblue}{\textbf{\ipa{jo˩ʂwæ˩}}}}{}
\textcolor{teal}{\zh{名词}} \hspace{4pt} \zh{声调类:} L.
\zh{阉羊。} \textcolor{Sepia}{\selectlanguage{english}Wether (castrated ram, neutered ram).} \textcolor{PineGreen}{\selectlanguage{french}Bélier châtré.}  \zh{量词}: \textcolor{darkblue}{\textbf{\ipa{pʰo˧˥}}} 
\lhead{\firstmark}
\rhead{\botmark}

\subsection{\hspace{-0.5cm} {\Large \textcolor{darkblue}{\textbf{\ipa{jo˩zo˩}}}}\hspace{0.5cm}[\kern2pt{\textcolor{darkblue}{\textbf{\ipa{jo˩zo˩˥}}}}\kern2pt]} \hypertarget{jo\string_Bzo\string_B1}{}
\markboth{\textcolor{darkblue}{\textbf{\ipa{jo˩zo˩}}}}{}
\textcolor{teal}{\zh{名词}} \hspace{4pt} \zh{声调类:} L.
\zh{绵羊羔。} \textcolor{Sepia}{\selectlanguage{english}Lamb.} \textcolor{PineGreen}{\selectlanguage{french}Agneau.}  \zh{量词}: \textcolor{darkblue}{\textbf{\ipa{ɭɯ˧}}} 
\lhead{\firstmark}
\rhead{\botmark}

\subsection{\hspace{-0.5cm} {\Large \textcolor{darkblue}{\textbf{\ipa{jo˧˥}}}}\hspace{0.5cm}[\kern2pt{\textcolor{darkblue}{\textbf{\ipa{jo˧˥}}}}\kern2pt]} \hypertarget{jo\string_M\string_T1}{}
\markboth{\textcolor{darkblue}{\textbf{\ipa{jo˧˥}}}}{}
\textcolor{teal}{\zh{动词}} \hspace{4pt} \zh{声调类:} MH.
\zh{赠给。} \textcolor{Sepia}{\selectlanguage{english}To offer.} \textcolor{PineGreen}{\selectlanguage{french}Offrir.} 
\lhead{\firstmark}
\rhead{\botmark}

\subsection{\hspace{-0.5cm} {\Large \textcolor{darkblue}{\textbf{\ipa{jo˩˧}}}}\hspace{0.5cm}[\kern2pt{\textcolor{darkblue}{\textbf{\ipa{jo˩˥}}}}\kern2pt]} \hypertarget{jo\string_B\string_M1}{}
\markboth{\textcolor{darkblue}{\textbf{\ipa{jo˩˧}}}}{}
\textcolor{teal}{\zh{名词}} \hspace{4pt} \zh{声调类:} LM.
\zh{右边。} \textcolor{Sepia}{\selectlanguage{english}Right (opposite of: left).} \textcolor{PineGreen}{\selectlanguage{french}Droite (opposé de: gauche).} \zh{~【参考】~} \hyperlink{}{\textcolor{darkblue}{\textbf{\ipa{jo˩gi˩}}}} 
\lhead{\firstmark}
\rhead{\botmark}

\subsection{\hspace{-0.5cm} {\Large \textcolor{darkblue}{\textbf{\ipa{je˧pʰi˧-jɤ\#˥}}}}\hspace{0.5cm}[\kern2pt{\textcolor{darkblue}{\textbf{\ipa{xxxx non-correspondance entre le nombre de morphèmes et le nombre de tons de morphèmes}}}}\kern2pt]} \hypertarget{je\string_Mp\string_hi\string_M-j7\#\string_T1}{}
\markboth{\textcolor{darkblue}{\textbf{\ipa{je˧pʰi˧-jɤ\#˥}}}}{}
\textcolor{teal}{\zh{名词}} \hspace{4pt} \zh{声调类:} \#H.
\zh{鸦片(汉语借词)。} \textcolor{Sepia}{\selectlanguage{english}Opium.} \textcolor{PineGreen}{\selectlanguage{french}Opium.}  \zh{【借词】} \zh{鸦片}

\lhead{\firstmark}
\rhead{\botmark}

\subsection{\hspace{-0.5cm} {\Large \textcolor{darkblue}{\textbf{\ipa{je˩ʐe˧}}}}\hspace{0.5cm}[\kern2pt{\textcolor{darkblue}{\textbf{\ipa{je˧ʐe˧}}}}\kern2pt]} \hypertarget{je\string_Bz`e\string_M1}{}
\markboth{\textcolor{darkblue}{\textbf{\ipa{je˩ʐe˧}}}}{}
\textcolor{teal}{\zh{名词}} \hspace{4pt} \zh{声调类:} LM.
\zh{西方人(“洋人”)(汉语借词)。} \textcolor{Sepia}{\selectlanguage{english}Westerner.} \textcolor{PineGreen}{\selectlanguage{french}Occidental.} \zh{当地汉语方言:}\zh{洋人。} \zh{【借词】} \zh{洋人}
 \zh{量词}: \textcolor{darkblue}{\textbf{\ipa{v̩˧}}} 
\lhead{\firstmark}
\rhead{\botmark}

\newpage
\section*{\centering- \textcolor{darkblue}{\textbf{\ipa{ʝ}}} -}
\subsection{\hspace{-0.5cm} {\Large \textcolor{darkblue}{\textbf{\ipa{ʝi˥}}} \textsubscript{1}}\hspace{0.5cm}[\kern2pt{\textcolor{darkblue}{\textbf{\ipa{ʝi˥}}}}\kern2pt]} \hypertarget{j££i\string_T1}{}
\markboth{\textcolor{darkblue}{\textbf{\ipa{ʝi˥}}} \textsubscript{1}}{}
\textcolor{teal}{\zh{名词}} \hspace{4pt} \zh{声调类:} \#H.
\zh{牛。} \textcolor{Sepia}{\selectlanguage{english}Ox.} \textcolor{PineGreen}{\selectlanguage{french}Vache, boeuf.}  ¶ \textcolor{darkblue}{\textbf{\ipa{ʝi˧-ɣɯ˥}}} \zh{牛皮} \textcolor{Sepia}{\selectlanguage{english}ox skin} \textcolor{PineGreen}{\selectlanguage{french}peau de vache}  
 ¶ \textcolor{darkblue}{\textbf{\ipa{ʝi˧ tʰv̩˧-pʰo˩}}} \zh{那头牛} \textcolor{Sepia}{\selectlanguage{english}\mytextsc{n}+\mytextsc{dem}+\mytextsc{clf}} \textcolor{PineGreen}{\selectlanguage{french}\mytextsc{n}+\mytextsc{dem}+\mytextsc{clf}}  
 \zh{量词}: \textcolor{darkblue}{\textbf{\ipa{pʰo˧˥}}} 
\lhead{\firstmark}
\rhead{\botmark}

\subsection{\hspace{-0.5cm} {\Large \textcolor{darkblue}{\textbf{\ipa{ʝi˥}}} \textsubscript{2}}\hspace{0.5cm}[\kern2pt{\textcolor{darkblue}{\textbf{\ipa{ʝi˥}}}}\kern2pt]} \hypertarget{j££i\string_T2}{}
\markboth{\textcolor{darkblue}{\textbf{\ipa{ʝi˥}}} \textsubscript{2}}{}
\textcolor{teal}{\zh{动词}} \hspace{4pt} \zh{声调类:} H.
\zh{做,工作。} \textcolor{Sepia}{\selectlanguage{english}To do, to work.} \textcolor{PineGreen}{\selectlanguage{french}Travailler, faire.}  ¶ \textcolor{darkblue}{\textbf{\ipa{ɖwæ˧˥ | lo˧ ʝi˧}}} \zh{勤劳、努力} \textcolor{Sepia}{\selectlanguage{english}hard-working, who works hard} \textcolor{PineGreen}{\selectlanguage{french}travailleur, assidu, qui travaille beaucoup}  
 ¶ \textcolor{darkblue}{\textbf{\ipa{ɖɯ˧-sɑ˥ | mɤ˧-ʝi˥}}} \zh{什么也不干} \textcolor{Sepia}{\selectlanguage{english}to do nothing at all} \textcolor{PineGreen}{\selectlanguage{french}ne rien faire du tout}  
 ¶ \textcolor{darkblue}{\textbf{\ipa{ə˧tso˧-mɤ˧-ɲi˩ | ʝi˧-bi˧-zo˧-ho˥!}}} \zh{什么都要做! / 我什么都要干(/管)!} \textcolor{Sepia}{\selectlanguage{english}[I/we] will have to take charge of everything / [I/we] will have to do all the work!} \textcolor{PineGreen}{\selectlanguage{french}Il faut tout faire! / On va devoir m'occuper de tout!}  
 ¶ \textcolor{darkblue}{\textbf{\ipa{ʈʂʰɯ˧ne-ʝi˥ | ʝi˧-zo˧-ho˥-ɲi˩!}}} \zh{应该这样做的!} \textcolor{Sepia}{\selectlanguage{english}This is how it must be done! / This is how it is done!} \textcolor{PineGreen}{\selectlanguage{french}Voilà comment il faut faire! / C'est comme ça qu'on fait!}  
 ¶ \textcolor{darkblue}{\textbf{\ipa{ɑ˩ʁo˧ ʝi˧}}} \zh{管理家里的大小事情(如:分配工作、家务等)} \textcolor{Sepia}{\selectlanguage{english}to take care of the household, to look after the affairs of the family; in particular: distributing work to the various members, and ensuring that the supplies are not running low} \textcolor{PineGreen}{\selectlanguage{french}gérer la maisonnée, s'occuper de la famille (tâche de la personne qui répartit les travaux à accomplir, veille aux approvisionnements...)}  

\lhead{\firstmark}
\rhead{\botmark}

\subsection{\hspace{-0.5cm} {\Large \textcolor{darkblue}{\textbf{\ipa{ʝi˥}}} \textsubscript{3}}\hspace{0.5cm}[\kern2pt{\textcolor{darkblue}{\textbf{\ipa{ʝi˥}}}}\kern2pt]} \hypertarget{j££i\string_T3}{}
\markboth{\textcolor{darkblue}{\textbf{\ipa{ʝi˥}}} \textsubscript{3}}{}
\textcolor{teal}{\zh{动词}} \hspace{4pt} \zh{声调类:} H.
\zh{画。} \textcolor{Sepia}{\selectlanguage{english}To draw.} \textcolor{PineGreen}{\selectlanguage{french}Dessiner, tracer.}  ¶ \textcolor{darkblue}{\textbf{\ipa{mɤ˧-ʝi˥}}} \zh{不画} \textcolor{Sepia}{\selectlanguage{english}\mytextsc{neg}} \textcolor{PineGreen}{\selectlanguage{french}\mytextsc{neg}}  
 ¶ \textcolor{darkblue}{\textbf{\ipa{tʰɑ˧-ʝi˥!}}} \zh{别画!} \textcolor{Sepia}{\selectlanguage{english}\mytextsc{prohib}} \textcolor{PineGreen}{\selectlanguage{french}\mytextsc{prohib}}  
 ¶ \textcolor{darkblue}{\textbf{\ipa{ʈʂɑ˧tɑ˥ ʝi˩}}} \zh{画一个符号} \textcolor{Sepia}{\selectlanguage{english}to draw a sign} \textcolor{PineGreen}{\selectlanguage{french}tracer un signe, dessiner un signe}  

\lhead{\firstmark}
\rhead{\botmark}

\subsection{\hspace{-0.5cm} {\Large \textcolor{darkblue}{\textbf{\ipa{ʝi˥}}} \textsubscript{4}}\hspace{0.5cm}[\kern2pt{\textcolor{darkblue}{\textbf{\ipa{ʝi˥}}}}\kern2pt]} \hypertarget{j££i\string_T4}{}
\markboth{\textcolor{darkblue}{\textbf{\ipa{ʝi˥}}} \textsubscript{4}}{}
\textcolor{teal}{\zh{名词}} \hspace{4pt} \zh{声调类:} \#H.
\zh{坛子,罐子 (陶器)。} \textcolor{Sepia}{\selectlanguage{english}Earthen jar.} \textcolor{PineGreen}{\selectlanguage{french}Jarre en terre cuite.}  \zh{量词}: \textcolor{darkblue}{\textbf{\ipa{ɭɯ˧}}} 
\lhead{\firstmark}
\rhead{\botmark}

\subsection{\hspace{-0.5cm} {\Large \textcolor{darkblue}{\textbf{\ipa{ʝi˥}}} \textsubscript{5}}\hspace{0.5cm}[\kern2pt{\textcolor{darkblue}{\textbf{\ipa{ʝi˥}}}}\kern2pt]} \hypertarget{j££i\string_T5}{}
\markboth{\textcolor{darkblue}{\textbf{\ipa{ʝi˥}}} \textsubscript{5}}{}
\textcolor{teal}{\zh{动词}} \hspace{4pt} \zh{声调类:} H.
\zh{通知、告诉。} \textcolor{Sepia}{\selectlanguage{english}To inform, to tell.} \textcolor{PineGreen}{\selectlanguage{french}Informer.}  ¶ \textcolor{darkblue}{\textbf{\ipa{le˧-ʝi˥-ze˩}}} \zh{通知了} \textcolor{Sepia}{\selectlanguage{english}\mytextsc{accomp} \string_ \mytextsc{pfv}} \textcolor{PineGreen}{\selectlanguage{french}\mytextsc{accomp} \string_ \mytextsc{pfv}}  
 ¶ \textcolor{darkblue}{\textbf{\ipa{qʰwæ˧ mi˧ ʝi˧}}} \zh{告诉(一个)消息} \textcolor{Sepia}{\selectlanguage{english}to provide a piece of news, to provide some information} \textcolor{PineGreen}{\selectlanguage{french}donner une nouvelle}  
 ¶ \textcolor{darkblue}{\textbf{\ipa{njɤ˧ | hĩ˧-ki˧ | qʰwæ˧mi˧ ʝi˧-ze˩}}} \zh{我告诉了人家(那个消息)。} \textcolor{Sepia}{\selectlanguage{english}I have told people the news.} \textcolor{PineGreen}{\selectlanguage{french}j'ai annoncé la nouvelle aux gens / j'ai annoncé une nouvelle à quelqu'un}  

\lhead{\firstmark}
\rhead{\botmark}

\subsection{\hspace{-0.5cm} {\Large \textcolor{darkblue}{\textbf{\ipa{ʝi˥}}} \textsubscript{6}}\hspace{0.5cm}[\kern2pt{\textcolor{darkblue}{\textbf{\ipa{ʝi˥}}}}\kern2pt]} \hypertarget{j££i\string_T6}{}
\markboth{\textcolor{darkblue}{\textbf{\ipa{ʝi˥}}} \textsubscript{6}}{}
\textcolor{teal}{\zh{名词}} \hspace{4pt} \zh{声调类:} \#H.
\textit{\zh{古语}} [\zh{古语}] \zh{男人。} \textcolor{Sepia}{\selectlanguage{english}Man, male person.} \textcolor{PineGreen}{\selectlanguage{french}Homme \textit{(vir)}.} 
\lhead{\firstmark}
\rhead{\botmark}

\subsection{\hspace{-0.5cm} {\Large \textcolor{darkblue}{\textbf{\ipa{ʝi˥}}} \textsubscript{7}}\hspace{0.5cm}[\kern2pt{\textcolor{darkblue}{\textbf{\ipa{ʝi˥}}}}\kern2pt]} \hypertarget{j££i\string_T7}{}
\markboth{\textcolor{darkblue}{\textbf{\ipa{ʝi˥}}} \textsubscript{7}}{}
\textcolor{teal}{\zh{动词}} \hspace{4pt} \zh{声调类:} H.
\zh{存在动词:有(可移动物品)。} \textcolor{Sepia}{\selectlanguage{english}Verb of existence, for movable things.} \textcolor{PineGreen}{\selectlanguage{french}Verbe d'existence: choses amovibles.}  ¶ \textcolor{darkblue}{\textbf{\ipa{ə˧tso˧-mɤ˧-ɲi˩, | le˧-ʂe˧, | le˧-ʝi˥!}}} \zh{所有(的东西都)找,(就)有了 = 所有的东西都备好了} \textcolor{Sepia}{\selectlanguage{english}We get all sorts of things (all the necessary paraphernalia for a ritual, a feast...) and we have it (at hand for when we need it) / We get all sorts of things ready (for the ritual / the feast)!} \textcolor{PineGreen}{\selectlanguage{french}(En vue d'un rituel, d'une fête…) on rassemble toutes sortes de choses; on en a (sous la main)/on a fait une provision! / On prépare tout par avance (pour le rituel/la fête)!}  

\lhead{\firstmark}
\rhead{\botmark}

\subsection{\hspace{-0.5cm} {\Large \textcolor{darkblue}{\textbf{\ipa{ʝi˧}}} \textsubscript{1}}\hspace{0.5cm}[\kern2pt{\textcolor{darkblue}{\textbf{\ipa{ʝi˩˥}}}}\kern2pt]} \hypertarget{j££i\string_M1}{}
\markboth{\textcolor{darkblue}{\textbf{\ipa{ʝi˧}}} \textsubscript{1}}{}
\textcolor{teal}{\zh{动词}} \hspace{4pt} \zh{声调类:} M\textsubscript{c}.
\zh{来。} \textcolor{Sepia}{\selectlanguage{english}To come.} \textcolor{PineGreen}{\selectlanguage{french}Venir.}  ¶ \textcolor{darkblue}{\textbf{\ipa{lɑ˧ ʝi˧-ze˧!}}} \zh{老虎来了!} \textcolor{Sepia}{\selectlanguage{english}A tiger is coming! / A tiger has come round!} \textcolor{PineGreen}{\selectlanguage{french}Voilà le tigre! / Un tigre arrive!}  
 ¶ \textcolor{darkblue}{\textbf{\ipa{lɑ˧ le˧-ʝi˩-ze˩!}}} \zh{老虎又来了!} \textcolor{Sepia}{\selectlanguage{english}The tiger is coming back! / The tiger is coming again!} \textcolor{PineGreen}{\selectlanguage{french}Voilà le tigre qui revient! / Le tigre est revenu!/ Le tigre est de retour!}  
 ¶ \textcolor{darkblue}{\textbf{\ipa{mɤ˧-ʝi˧-ze˧!}}} \zh{不好了!不行了!} \textcolor{Sepia}{\selectlanguage{english}It's going wrong! / Something is going wrong! / We're in for trouble!} \textcolor{PineGreen}{\selectlanguage{french}Ca ne va plus! / C'est la catastrophe!}  

\lhead{\firstmark}
\rhead{\botmark}

\subsection{\hspace{-0.5cm} {\Large \textcolor{darkblue}{\textbf{\ipa{ʝi˧}}} \textsubscript{2}}\hspace{0.5cm}[\kern2pt{\textcolor{darkblue}{\textbf{\ipa{ʝi˥}}}}\kern2pt]} \hypertarget{j££i\string_M2}{}
\markboth{\textcolor{darkblue}{\textbf{\ipa{ʝi˧}}} \textsubscript{2}}{}
\textcolor{teal}{\zh{名词}} \hspace{4pt} \zh{声调类:} M.
\zh{一。} \textcolor{Sepia}{\selectlanguage{english}One (restricted use: only in association with /ɭɯ˧/).} \textcolor{PineGreen}{\selectlanguage{french}Un (numéral, à emploi restreint; ne se combine qu'avec le classificateur /ɭɯ˧/).}  ¶ \textcolor{darkblue}{\textbf{\ipa{zo˧mv̩˥ | ʝi˧-ɭɯ˧ ʂv̩˧}}} \zh{管一个孩子} \textcolor{Sepia}{\selectlanguage{english}to take care of a child} \textcolor{PineGreen}{\selectlanguage{french}s'occuper d'un enfant}  

\lhead{\firstmark}
\rhead{\botmark}

\subsection{\hspace{-0.5cm} {\Large \textcolor{darkblue}{\textbf{\ipa{ʝi˧\textsubscript{b}}}}}\hspace{0.5cm}[\kern2pt{\textcolor{darkblue}{\textbf{\ipa{ʝi˥}}}}\kern2pt]} \hypertarget{j££i\string_Mb1}{}
\markboth{\textcolor{darkblue}{\textbf{\ipa{ʝi˧\textsubscript{b}}}}}{}
\textcolor{teal}{\zh{量词}} \hspace{4pt} \zh{声调类:} M\textsubscript{b}.
\zh{量词:地方(一个)。} \textcolor{Sepia}{\selectlanguage{english}Classifier for places.} \textcolor{PineGreen}{\selectlanguage{french}Classificateur des lieux.}  ¶ \textcolor{darkblue}{\textbf{\ipa{ɖɯ˧-ʝi˧}}} \zh{一个地方} \textcolor{Sepia}{\selectlanguage{english}a place, somewhere} \textcolor{PineGreen}{\selectlanguage{french}un endroit; qq part}  
 ¶ \textcolor{darkblue}{\textbf{\ipa{ɖɯ˧-ʝi˧ dzi˩}}} \zh{住在一个地方,搬家到一个地方} \textcolor{Sepia}{\selectlanguage{english}to live somewhere; to move to somewhere} \textcolor{PineGreen}{\selectlanguage{french}habiter quelque part; emménager quelque part/déménager vers quelque part}  
 ¶ \textcolor{darkblue}{\textbf{\ipa{ɖɯ˧-v˧ | ɖɯ˧-ʝi˧ hɯ˧}}} \zh{个去个的地方!/ 每个人去不同的地方!(情景:由于工作原因,一家的成员经常需要去不同的城市工作。)} \textcolor{Sepia}{\selectlanguage{english}each goes her/his own way (context: explaining that, in many families, people go to live in different cities for professional reasons)} \textcolor{PineGreen}{\selectlanguage{french}chacun s'en va de son côté (contexte: les membres d'une famille vont habiter en des lieux différents pour raisons professionnelles)}  

\lhead{\firstmark}
\rhead{\botmark}

\subsection{\hspace{-0.5cm} {\Large \textcolor{darkblue}{\textbf{\ipa{ʝi˧-bv̩˧˥}}}}\hspace{0.5cm}[\kern2pt{\textcolor{darkblue}{\textbf{\ipa{xxxx non-correspondance entre le nombre de morphèmes et le nombre de tons de morphèmes}}}}\kern2pt]} \hypertarget{j££i\string_M-bv\string_=\string_M\string_T1}{}
\markboth{\textcolor{darkblue}{\textbf{\ipa{ʝi˧-bv̩˧˥}}}}{}
\textcolor{teal}{\zh{名词}} \hspace{4pt} \zh{声调类:} MH\#.
\zh{牛圈。} \textcolor{Sepia}{\selectlanguage{english}Cow pen.} \textcolor{PineGreen}{\selectlanguage{french}Étable (des vaches).}  \zh{量词}: \textcolor{darkblue}{\textbf{\ipa{ɭɯ˧}}} 
\lhead{\firstmark}
\rhead{\botmark}

\subsection{\hspace{-0.5cm} {\Large \textcolor{darkblue}{\textbf{\ipa{ʝi˧kʰv̩˩}}}}\hspace{0.5cm}[\kern2pt{\textcolor{darkblue}{\textbf{\ipa{ʝi˩kʰv̩˥}}}}\kern2pt]} \hypertarget{j££i\string_Mk\string_hv\string_=\string_B1}{}
\markboth{\textcolor{darkblue}{\textbf{\ipa{ʝi˧kʰv̩˩}}}}{}
\textcolor{teal}{\zh{名词}} \hspace{4pt} \zh{声调类:} L\#.
\zh{牛年。} \textcolor{Sepia}{\selectlanguage{english}Year of the ox.} \textcolor{PineGreen}{\selectlanguage{french}Année du bœuf / année du boeuf.} 
\lhead{\firstmark}
\rhead{\botmark}

\subsection{\hspace{-0.5cm} {\Large \textcolor{darkblue}{\textbf{\ipa{ʝi˧kʰv̩˥}}}}\hspace{0.5cm}[\kern2pt{\textcolor{darkblue}{\textbf{\ipa{ʝi˧kʰv̩˩}}}}\kern2pt]} \hypertarget{j££i\string_Mk\string_hv\string_=\string_T1}{}
\markboth{\textcolor{darkblue}{\textbf{\ipa{ʝi˧kʰv̩˥}}}}{}
\textcolor{teal}{\zh{代词}} \hspace{4pt} \zh{声调类:} H\#.
\zh{一些。} \textcolor{Sepia}{\selectlanguage{english}Some, a few.} \textcolor{PineGreen}{\selectlanguage{french}Certains.}  ¶ \textcolor{darkblue}{\textbf{\ipa{hĩ˧ ʝi˧kʰv̩˥}}} \zh{一些人} \textcolor{Sepia}{\selectlanguage{english}some people, part of the people} \textcolor{PineGreen}{\selectlanguage{french}certaines personnes, une partie des gens}  

\lhead{\firstmark}
\rhead{\botmark}

\subsection{\hspace{-0.5cm} {\Large \textcolor{darkblue}{\textbf{\ipa{ʝi˧kʰwɤ˥\$}}}}\hspace{0.5cm}[\kern2pt{\textcolor{darkblue}{\textbf{\ipa{ʝi˧kʰwɤ˥}}}}\kern2pt]} \hypertarget{j££i\string_Mk\string_hw7\string_T\$1}{}
\markboth{\textcolor{darkblue}{\textbf{\ipa{ʝi˧kʰwɤ˥\$}}}}{}
\textcolor{teal}{\zh{代词}} \hspace{4pt} \zh{声调类:} H\$.
\zh{一点。} \textcolor{Sepia}{\selectlanguage{english}A little, some.} \textcolor{PineGreen}{\selectlanguage{french}Un peu.} 
\lhead{\firstmark}
\rhead{\botmark}

\subsection{\hspace{-0.5cm} {\Large \textcolor{darkblue}{\textbf{\ipa{ʝi˧lo\#˥}}}}\hspace{0.5cm}[\kern2pt{\textcolor{darkblue}{\textbf{\ipa{ʝi˧lo˥}}}}\kern2pt]} \hypertarget{j££i\string_Mlo\#\string_T1}{}
\markboth{\textcolor{darkblue}{\textbf{\ipa{ʝi˧lo\#˥}}}}{}
\textcolor{teal}{\zh{名词}} \hspace{4pt} \zh{声调类:} \#H.
\zh{态度、对待的态度。} \textcolor{Sepia}{\selectlanguage{english}Attitude towards others.} \textcolor{PineGreen}{\selectlanguage{french}Traitement (d'autrui), attitude.}  ¶ \textcolor{darkblue}{\textbf{\ipa{ʝi˧lo˧ dʑɤ˥!}}} \zh{态度积极} \textcolor{Sepia}{\selectlanguage{english}(He/she) has a good attitude!} \textcolor{PineGreen}{\selectlanguage{french}(Il / elle) a une attitude positive}  
 ¶ \textcolor{darkblue}{\textbf{\ipa{ʈʂʰɯ˧ | ʝi˧lo˧ | dʑɤ˩˥! | hĩ˧-ki˧ | dʑɤ˩-ʝi˥!}}} \zh{他(对人)态度好!对人好/做好事!} \textcolor{Sepia}{\selectlanguage{english}He/she has a good attitude towards people! He/she is kind to people / does some good around him/her!} \textcolor{PineGreen}{\selectlanguage{french}Il/elle traite bien les gens! Il/elle fait de bonnes actions!}  
 ¶ \textcolor{darkblue}{\textbf{\ipa{ʝi˧lo˧ dzɑ˧}}} \zh{态度不好} \textcolor{Sepia}{\selectlanguage{english}(to have) a bad attitude: to be lazy, dissipated...} \textcolor{PineGreen}{\selectlanguage{french}(avoir une) mauvaise attitude: paresseuse, dissipée…}  
 ¶ \textcolor{darkblue}{\textbf{\ipa{njɤ˧-ɳɯ˧ hɑ˧ gv̩˥, | ʝi˧lo˧ dzɑ˧!}}} \zh{我做饭,集中不了精神 / 做的乱七八糟!} \textcolor{Sepia}{\selectlanguage{english}When I cook, I don't make a good job of it / I don't (manage to) put any heart into it / I make a mess of it!} \textcolor{PineGreen}{\selectlanguage{french}Quand je fais la cuisine, je ne suis pas bien concentrée/je travaille n'importe comment!}  
 ¶ \textcolor{darkblue}{\textbf{\ipa{ʈʂʰɯ˧ | ə˧tso˧ ʝi˧lo˧ ɲi˥?}}} \zh{这是什么态度啊?(批评一个人的态度)} \textcolor{Sepia}{\selectlanguage{english}What sort of an attitude is this? (Criticism of someone who does not have a proper attitude)} \textcolor{PineGreen}{\selectlanguage{french}Qu'est-ce que c'est que cette attitude? (critique adressée à quelqu'un qui fait n'importe quoi)}  

\lhead{\firstmark}
\rhead{\botmark}

\subsection{\hspace{-0.5cm} {\Large \textcolor{darkblue}{\textbf{\ipa{ʝi˧mi˧}}}}\hspace{0.5cm}[\kern2pt{\textcolor{darkblue}{\textbf{\ipa{ʝi˩mi˩˥}}}}\kern2pt]} \hypertarget{j££i\string_Mmi\string_M1}{}
\markboth{\textcolor{darkblue}{\textbf{\ipa{ʝi˧mi˧}}}}{}
\textcolor{teal}{\zh{名词}} \hspace{4pt} \zh{声调类:} M.
\zh{坛子,罐子 (陶器)。} \textcolor{Sepia}{\selectlanguage{english}Jar.} \textcolor{PineGreen}{\selectlanguage{french}Jarre.}  \zh{量词}: \textcolor{darkblue}{\textbf{\ipa{ɭɯ˧}}} 
\lhead{\firstmark}
\rhead{\botmark}

\subsection{\hspace{-0.5cm} {\Large \textcolor{darkblue}{\textbf{\ipa{ʝi˧pʰv̩\#˥}}}}\hspace{0.5cm}[\kern2pt{\textcolor{darkblue}{\textbf{\ipa{ʝi˩pʰv̩˧˥}}}}\kern2pt]} \hypertarget{j££i\string_Mp\string_hv\string_=\#\string_T1}{}
\markboth{\textcolor{darkblue}{\textbf{\ipa{ʝi˧pʰv̩\#˥}}}}{}
\textcolor{teal}{\zh{名词}} \hspace{4pt} \zh{声调类:} \#H.
\zh{公牛。} \textcolor{Sepia}{\selectlanguage{english}Male ox, bull.} \textcolor{PineGreen}{\selectlanguage{french}Taureau.}  ¶ \textcolor{darkblue}{\textbf{\ipa{ʝi˧pʰv̩˧ tʰv̩˧-mi˥\#}}} \zh{那头公牛} \textcolor{Sepia}{\selectlanguage{english}\mytextsc{n}+\mytextsc{dem}+\mytextsc{clf}} \textcolor{PineGreen}{\selectlanguage{french}\mytextsc{n}+\mytextsc{dem}+\mytextsc{clf}}  
 ¶ \textcolor{darkblue}{\textbf{\ipa{ʝi˧pʰv̩˧ tʰv̩˧-ɭɯ\#˥}}} \zh{那头公牛} \textcolor{Sepia}{\selectlanguage{english}\mytextsc{n}+\mytextsc{dem}+\mytextsc{clf}.animaux} \textcolor{PineGreen}{\selectlanguage{french}\mytextsc{n}+\mytextsc{dem}+\mytextsc{clf}.animaux}  
 \zh{量词}: \textcolor{darkblue}{\textbf{\ipa{ɭɯ˧ / mi˩}}} 
\lhead{\firstmark}
\rhead{\botmark}

\subsection{\hspace{-0.5cm} {\Large \textcolor{darkblue}{\textbf{\ipa{ʝi˧qv̩˥}}}}\hspace{0.5cm}[\kern2pt{\textcolor{darkblue}{\textbf{\ipa{ʝi˩qv̩˩˥}}}}\kern2pt]} \hypertarget{j££i\string_Mqv\string_=\string_T1}{}
\markboth{\textcolor{darkblue}{\textbf{\ipa{ʝi˧qv̩˥}}}}{}
\textcolor{teal}{\zh{名词}} \hspace{4pt} \zh{声调类:} H\#.
\zh{轭的一个部分,将牛轭安在牛的脖子上。} \textcolor{Sepia}{\selectlanguage{english}Neck strap: a part of the buffalo's harness for ploughing: a strap that fastens the yoke.} \textcolor{PineGreen}{\selectlanguage{french}Collier: une partie du harnais utilisé pour les labours, qui maintient le joug en place; cette corde passe sous le cou de l'animal, et est fixée au joug.}  \zh{量词}: \textcolor{darkblue}{\textbf{\ipa{ɭɯ˧}}} 
\lhead{\firstmark}
\rhead{\botmark}

\subsection{\hspace{-0.5cm} {\Large \textcolor{darkblue}{\textbf{\ipa{ʝi˧ʁæ˥}}}}\hspace{0.5cm}[\kern2pt{\textcolor{darkblue}{\textbf{\ipa{ʝi˧ʁæ˥}}}}\kern2pt]} \hypertarget{j££i\string_MR\{\string_T1}{}
\markboth{\textcolor{darkblue}{\textbf{\ipa{ʝi˧ʁæ˥}}}}{}
\textcolor{teal}{\zh{名词}} \hspace{4pt} \zh{声调类:} H\#.
\zh{黄牛。} \textcolor{Sepia}{\selectlanguage{english}Cow, beef.} \textcolor{PineGreen}{\selectlanguage{french}Vache, boeuf.}  \zh{量词}: \textcolor{darkblue}{\textbf{\ipa{pʰo˧˥}}} 
\lhead{\firstmark}
\rhead{\botmark}

\subsection{\hspace{-0.5cm} {\Large \textcolor{darkblue}{\textbf{\ipa{ʝi˧ʁo\#˥}}}}\hspace{0.5cm}[\kern2pt{\textcolor{darkblue}{\textbf{\ipa{ʝi˧ʁo˥}}}}\kern2pt]} \hypertarget{j££i\string_MRo\#\string_T1}{}
\markboth{\textcolor{darkblue}{\textbf{\ipa{ʝi˧ʁo\#˥}}}}{}
\textcolor{teal}{\zh{形容词}} \hspace{4pt} \zh{声调类:} \#H.
\zh{能干、不缺劳力。} \textcolor{Sepia}{\selectlanguage{english}Able, capable, able-bodied.} \textcolor{PineGreen}{\selectlanguage{french}Capable; littéralement “qui sait faire”.}  ¶ \textcolor{darkblue}{\textbf{\ipa{ʈʂʰɯ˧ | ʝi˧ʁo˧-hĩ˧ | ɖɯ˧-v̩˧ ɲi˩}}} \zh{他是一个能干/称职的人。} \textcolor{Sepia}{\selectlanguage{english}It's an able/competent person.} \textcolor{PineGreen}{\selectlanguage{french}c'est quelqu'un d'habile/de capable}  
 ¶ \textcolor{darkblue}{\textbf{\ipa{ʝi˧ʁo˧-zo˥}}} \zh{一个能干的男人} \textcolor{Sepia}{\selectlanguage{english}a competent lad, a capable fellow} \textcolor{PineGreen}{\selectlanguage{french}un homme capable/habile, un gaillard compétent}  
 ¶ \textcolor{darkblue}{\textbf{\ipa{ʝi˧ʁo˧ ɲi˥}}} \zh{\mytextsc{cop}} \textcolor{Sepia}{\selectlanguage{english}\mytextsc{cop}} \textcolor{PineGreen}{\selectlanguage{french}\mytextsc{cop}}  

\lhead{\firstmark}
\rhead{\botmark}

\subsection{\hspace{-0.5cm} {\Large \textcolor{darkblue}{\textbf{\ipa{ʝi˧sɑ˧}}}}\hspace{0.5cm}[\kern2pt{\textcolor{darkblue}{\textbf{\ipa{ʝi˧sɑ˧}}}}\kern2pt]} \hypertarget{j££i\string_MsA\string_M1}{}
\markboth{\textcolor{darkblue}{\textbf{\ipa{ʝi˧sɑ˧}}}}{}
\textcolor{teal}{\zh{名词}} \hspace{4pt} \zh{声调类:} M.
\zh{雨伞。} \textcolor{Sepia}{\selectlanguage{english}Umbrella.} \textcolor{PineGreen}{\selectlanguage{french}Parapluie (emprunt).}  \zh{【借词】} \zh{雨伞}
 \zh{量词}: \textcolor{darkblue}{\textbf{\ipa{nɑ˧}}} 
\lhead{\firstmark}
\rhead{\botmark}

\subsection{\hspace{-0.5cm} {\Large \textcolor{darkblue}{\textbf{\ipa{ʝi˧se˧}}} \textsubscript{1}}\hspace{0.5cm}[\kern2pt{\textcolor{darkblue}{\textbf{\ipa{ʝi˧se˩}}}}\kern2pt]} \hypertarget{j££i\string_Mse\string_M1}{}
\markboth{\textcolor{darkblue}{\textbf{\ipa{ʝi˧se˧}}} \textsubscript{1}}{}
\textcolor{teal}{\zh{名词}} \hspace{4pt} \zh{声调类:} M.
\zh{医生(汉语借词)。} \textcolor{Sepia}{\selectlanguage{english}Doctor.} \textcolor{PineGreen}{\selectlanguage{french}Médecin.}  \zh{【借词】} \zh{医生}

\lhead{\firstmark}
\rhead{\botmark}

\subsection{\hspace{-0.5cm} {\Large \textcolor{darkblue}{\textbf{\ipa{ʝi˧se˧}}} \textsubscript{2}}\hspace{0.5cm}[\kern2pt{\textcolor{darkblue}{\textbf{\ipa{ʝi˧se˧}}}}\kern2pt]} \hypertarget{j££i\string_Mse\string_M2}{}
\markboth{\textcolor{darkblue}{\textbf{\ipa{ʝi˧se˧}}} \textsubscript{2}}{}
\textcolor{teal}{\zh{形容词}} \hspace{4pt} \zh{声调类:} M.
\zh{野生(汉语借词)。} \textcolor{Sepia}{\selectlanguage{english}Wild (as opposed to: cultivated; e.g. wild plants, wild animals).} \textcolor{PineGreen}{\selectlanguage{french}Sauvage, spontané: plantes qui poussent spontanément (par opposition aux plantes cultivées), animaux sauvages.}  \zh{【借词】} \zh{野生}
 ¶ \textcolor{darkblue}{\textbf{\ipa{ʝi˧se˧-hĩ˧}}} \zh{野生的} \textcolor{Sepia}{\selectlanguage{english}\string_ \mytextsc{rel}/\mytextsc{nmlz}} \textcolor{PineGreen}{\selectlanguage{french}\string_ \mytextsc{rel}/\mytextsc{nmlz}}  

\lhead{\firstmark}
\rhead{\botmark}

\subsection{\hspace{-0.5cm} {\Large \textcolor{darkblue}{\textbf{\ipa{ʝi˧sɯ˥}}}}\hspace{0.5cm}[\kern2pt{\textcolor{darkblue}{\textbf{\ipa{ʝi˧sɯ˧}}}}\kern2pt]} \hypertarget{j££i\string_MsM\string_T1}{}
\markboth{\textcolor{darkblue}{\textbf{\ipa{ʝi˧sɯ˥}}}}{}
\textcolor{teal}{\zh{名词}} \hspace{4pt} \zh{声调类:} H\#.
\zh{意思(汉语借词)。} \textcolor{Sepia}{\selectlanguage{english}Meaning, sense.} \textcolor{PineGreen}{\selectlanguage{french}Signification, sens.}  \zh{【借词】} \zh{意思}

\lhead{\firstmark}
\rhead{\botmark}

\subsection{\hspace{-0.5cm} {\Large \textcolor{darkblue}{\textbf{\ipa{ʝi˧ʂæ˧tsʰɤ˩}}}}\hspace{0.5cm}[\kern2pt{\textcolor{darkblue}{\textbf{\ipa{ʝi˧ʂæ˧tsʰɤ˧}}}}\kern2pt]} \hypertarget{j££i\string_Ms`\{\string_Mts\string_h7\string_B1}{}
\markboth{\textcolor{darkblue}{\textbf{\ipa{ʝi˧ʂæ˧tsʰɤ˩}}}}{}
\textcolor{teal}{\zh{名词}} \hspace{4pt} \zh{声调类:} L\#.
\zh{红萝卜菜(汉语借词:野山菜):一种山上的野菜。春天的时候,菜园的蔬菜还没有成熟的时候,永宁的人吃红萝卜菜。彝族人从高山上采下来,在永宁卖。} \textcolor{Sepia}{\selectlanguage{english}A wild radish that grows on the mountains; it is edible; it is picked and eaten in the Spring, when vegetables are not ripe yet. Yi people harvest it and sell it in the plain.} \textcolor{PineGreen}{\selectlanguage{french}Radis sauvage qui pousse en montagne; on le consomme surtout au printemps, à une époque où il n'y a pas encore de légumes. Ce radis est récoltée par les Yi et vendu dans la plaine.} \zh{当地汉语方言:}\zh{野山菜。} \zh{【借词】} \zh{野山菜}
\zh{~【参考】~} \hyperlink{}{\textcolor{darkblue}{\textbf{\ipa{jɤ˧˥}}} \textsubscript{2}} 
\lhead{\firstmark}
\rhead{\botmark}

\subsection{\hspace{-0.5cm} {\Large \textcolor{darkblue}{\textbf{\ipa{ʝi˧ʂɯ˥}}}}\hspace{0.5cm}[\kern2pt{\textcolor{darkblue}{\textbf{\ipa{ʝi˧ʂɯ˥}}}}\kern2pt]} \hypertarget{j££i\string_Ms`M\string_T1}{}
\markboth{\textcolor{darkblue}{\textbf{\ipa{ʝi˧ʂɯ˥}}}}{}
\textcolor{teal}{\zh{名词}} \hspace{4pt} \zh{声调类:} H\#.
\zh{男性名字。} \textcolor{Sepia}{\selectlanguage{english}Masculine given name.} \textcolor{PineGreen}{\selectlanguage{french}Prénom masculin.} 
\lhead{\firstmark}
\rhead{\botmark}

\subsection{\hspace{-0.5cm} {\Large \textcolor{darkblue}{\textbf{\ipa{ʝi˧tɕi˧}}}}\hspace{0.5cm}[\kern2pt{\textcolor{darkblue}{\textbf{\ipa{ʝi˧tɕi˥}}}}\kern2pt]} \hypertarget{j££i\string_Mts£i\string_M1}{}
\markboth{\textcolor{darkblue}{\textbf{\ipa{ʝi˧tɕi˧}}}}{}
\textcolor{teal}{\zh{名词}} \hspace{4pt} \zh{声调类:} M.
\zh{女性名字。} \textcolor{Sepia}{\selectlanguage{english}Feminine given name.} \textcolor{PineGreen}{\selectlanguage{french}Prénom féminin.} 
\lhead{\firstmark}
\rhead{\botmark}

\subsection{\hspace{-0.5cm} {\Large \textcolor{darkblue}{\textbf{\ipa{ʝi˧tɕi˧-ɖɯ˩mɑ˩}}}}\hspace{0.5cm}[\kern2pt{\textcolor{darkblue}{\textbf{\ipa{xxxx non-correspondance entre le nombre de morphèmes et le nombre de tons de morphèmes}}}}\kern2pt]} \hypertarget{j££i\string_Mts£i\string_M-d`M\string_BmA\string_B1}{}
\markboth{\textcolor{darkblue}{\textbf{\ipa{ʝi˧tɕi˧-ɖɯ˩mɑ˩}}}}{}
\textcolor{teal}{\zh{名词}} \hspace{4pt} \zh{声调类:} \mytextsc{L}.
\zh{女性名字。} \textcolor{Sepia}{\selectlanguage{english}Feminine given name.} \textcolor{PineGreen}{\selectlanguage{french}Prénom féminin.} 
\lhead{\firstmark}
\rhead{\botmark}

\subsection{\hspace{-0.5cm} {\Large \textcolor{darkblue}{\textbf{\ipa{ʝi˧tsɯ˧}}}}\hspace{0.5cm}[\kern2pt{\textcolor{darkblue}{\textbf{\ipa{ʝi˧tsɯ˧}}}}\kern2pt]} \hypertarget{j££i\string_MtsM\string_M1}{}
\markboth{\textcolor{darkblue}{\textbf{\ipa{ʝi˧tsɯ˧}}}}{}
\textcolor{teal}{\zh{名词}} \hspace{4pt} \zh{声调类:} M.
\zh{椅子。} \textcolor{Sepia}{\selectlanguage{english}Chair (borrowing).} \textcolor{PineGreen}{\selectlanguage{french}Chaise (emprunt).}  \zh{【借词】} \zh{椅子}
 \zh{量词}: \textcolor{darkblue}{\textbf{\ipa{nɑ˧}}} 
\lhead{\firstmark}
\rhead{\botmark}

\subsection{\hspace{-0.5cm} {\Large \textcolor{darkblue}{\textbf{\ipa{ʝi˧ʈʂʰe˥-mi˩}}}}\hspace{0.5cm}[\kern2pt{\textcolor{darkblue}{\textbf{\ipa{ʝi˧ʈʂʰe˥mi˩}}}}\kern2pt]} \hypertarget{j££i\string_Mt`s`\string_he\string_T-mi\string_B1}{}
\markboth{\textcolor{darkblue}{\textbf{\ipa{ʝi˧ʈʂʰe˥-mi˩}}}}{}
\textcolor{teal}{\zh{名词}} \hspace{4pt} \zh{声调类:} H\#-.
\zh{南方。} \textcolor{Sepia}{\selectlanguage{english}South.} \textcolor{PineGreen}{\selectlanguage{french}Sud.}  ¶ \textcolor{darkblue}{\textbf{\ipa{ʝi˧ʈʂʰe˥mi˩-gi˩dzɤ˩ se˩}}} \zh{往南方走} \textcolor{Sepia}{\selectlanguage{english}to walk towards the south} \textcolor{PineGreen}{\selectlanguage{french}marcher en direction du sud}  

\lhead{\firstmark}
\rhead{\botmark}

\subsection{\hspace{-0.5cm} {\Large \textcolor{darkblue}{\textbf{\ipa{ʝi˧zo\#˥}}}}\hspace{0.5cm}[\kern2pt{\textcolor{darkblue}{\textbf{\ipa{ʝi˧zo˧}}}}\kern2pt]} \hypertarget{j££i\string_Mzo\#\string_T1}{}
\markboth{\textcolor{darkblue}{\textbf{\ipa{ʝi˧zo\#˥}}}}{}
\textcolor{teal}{\zh{名词}} \hspace{4pt} \zh{声调类:} \#H.
\zh{小牛。} \textcolor{Sepia}{\selectlanguage{english}Calf.} \textcolor{PineGreen}{\selectlanguage{french}Veau.}  ¶ \textcolor{darkblue}{\textbf{\ipa{ʝi˧zo˧ tʰv̩˧-ɭɯ\#˥}}} \zh{那头小牛} \textcolor{Sepia}{\selectlanguage{english}\mytextsc{n}+\mytextsc{dem}+\mytextsc{clf}} \textcolor{PineGreen}{\selectlanguage{french}\mytextsc{n}+\mytextsc{dem}+\mytextsc{clf}}  
 \zh{量词}: \textcolor{darkblue}{\textbf{\ipa{pʰo˧˥ / ɭɯ˧}}} 
\lhead{\firstmark}
\rhead{\botmark}

\subsection{\hspace{-0.5cm} {\Large \textcolor{darkblue}{\textbf{\ipa{ʝi˩bv̩˩}}} \textsubscript{1}}\hspace{0.5cm}[\kern2pt{\textcolor{darkblue}{\textbf{\ipa{ʝi˧bv̩˧}}}}\kern2pt]} \hypertarget{j££i\string_Bbv\string_=\string_B1}{}
\markboth{\textcolor{darkblue}{\textbf{\ipa{ʝi˩bv̩˩}}} \textsubscript{1}}{}
\textcolor{teal}{\zh{名词}} \hspace{4pt} \zh{声调类:} L.
\zh{麻子。} \textcolor{Sepia}{\selectlanguage{english}Pockmarked person.} \textcolor{PineGreen}{\selectlanguage{french}Grêlé.}  ¶ \textcolor{darkblue}{\textbf{\ipa{ʝi˩bv̩˩-ʝi˧ʈv̩˩ʈv̩˩}}} \zh{同上} \textcolor{Sepia}{\selectlanguage{english}same meaning} \textcolor{PineGreen}{\selectlanguage{french}même sens}  
 \zh{量词}: \textcolor{darkblue}{\textbf{\ipa{v̩˧}}} 
\lhead{\firstmark}
\rhead{\botmark}

\subsection{\hspace{-0.5cm} {\Large \textcolor{darkblue}{\textbf{\ipa{ʝi˩bv̩˩}}} \textsubscript{2}}\hspace{0.5cm}[\kern2pt{\textcolor{darkblue}{\textbf{\ipa{ʝi˩bv̩˩˥}}}}\kern2pt]} \hypertarget{j££i\string_Bbv\string_=\string_B2}{}
\markboth{\textcolor{darkblue}{\textbf{\ipa{ʝi˩bv̩˩}}} \textsubscript{2}}{}
\textcolor{teal}{\zh{名词}} \hspace{4pt} \zh{声调类:} L.
\zh{公牛。} \textcolor{Sepia}{\selectlanguage{english}Bull (male).} \textcolor{PineGreen}{\selectlanguage{french}Taureau.}  \zh{量词}: \textcolor{darkblue}{\textbf{\ipa{pʰo˧˥}}} 
\lhead{\firstmark}
\rhead{\botmark}

\subsection{\hspace{-0.5cm} {\Large \textcolor{darkblue}{\textbf{\ipa{ʝi˩di˩-mi˥}}}}\hspace{0.5cm}[\kern2pt{\textcolor{darkblue}{\textbf{\ipa{xxxx non-correspondance entre le nombre de morphèmes et le nombre de tons de morphèmes}}}}\kern2pt]} \hypertarget{j££i\string_Bdi\string_B-mi\string_T1}{}
\markboth{\textcolor{darkblue}{\textbf{\ipa{ʝi˩di˩-mi˥}}}}{}
\textcolor{teal}{\zh{名词}} \hspace{4pt} \zh{声调类:} L+H\#.
\zh{小牝牛(包括黄牛和小母犏牛)。} \textcolor{Sepia}{\selectlanguage{english}Heifer; also used for a female pianniu: hybrid of yak and cattle.} \textcolor{PineGreen}{\selectlanguage{french}Génisse; s'emploie pour une petite vache, aussi bien que pour le pianniu (tibétain: dzomo).}  \zh{量词}: \textcolor{darkblue}{\textbf{\ipa{ɭɯ˧}}} \textcolor{darkblue}{\textbf{\ipa{pʰo˧˥}}} 
\lhead{\firstmark}
\rhead{\botmark}

\subsection{\hspace{-0.5cm} {\Large \textcolor{darkblue}{\textbf{\ipa{ʝi˩mi˩}}}}\hspace{0.5cm}[\kern2pt{\textcolor{darkblue}{\textbf{\ipa{ʝi˧mi˧}}}}\kern2pt]} \hypertarget{j££i\string_Bmi\string_B1}{}
\markboth{\textcolor{darkblue}{\textbf{\ipa{ʝi˩mi˩}}}}{}
\textcolor{teal}{\zh{名词}} \hspace{4pt} \zh{声调类:} L.
\zh{母牛。} \textcolor{Sepia}{\selectlanguage{english}Cow (female).} \textcolor{PineGreen}{\selectlanguage{french}Vache.}  ¶ \textcolor{darkblue}{\textbf{\ipa{ʝi˩mi˩-ʐɤ˥qo˩}}} \zh{母牛与小牛} \textcolor{Sepia}{\selectlanguage{english}cow and calf} \textcolor{PineGreen}{\selectlanguage{french}vache et veau}  
 \zh{量词}: \textcolor{darkblue}{\textbf{\ipa{pʰo˧˥}}} 
\lhead{\firstmark}
\rhead{\botmark}

\subsection{\hspace{-0.5cm} {\Large \textcolor{darkblue}{\textbf{\ipa{ʝi˩næ˩-se˧}}}}\hspace{0.5cm}[\kern2pt{\textcolor{darkblue}{\textbf{\ipa{xxxx non-correspondance entre le nombre de morphèmes et le nombre de tons de morphèmes}}}}\kern2pt]} \hypertarget{j££i\string_Bn\{\string_B-se\string_M1}{}
\markboth{\textcolor{darkblue}{\textbf{\ipa{ʝi˩næ˩-se˧}}}}{}
\textcolor{teal}{\zh{名词}} \hspace{4pt} \zh{声调类:} L-M.
\zh{云南,昆明……。} \textcolor{Sepia}{\selectlanguage{english}Kunming, and the Eastern part of the province of Yunnan.} \textcolor{PineGreen}{\selectlanguage{french}Kunming et la partie orientale du Yunnan, une fois passés Lijiang et Dali.}  \zh{【借词】} \zh{云南省}
 ¶ \textcolor{darkblue}{\textbf{\ipa{sɯ˧pʰi˧ | ʝi˩næ˩-se˧-qo˧ hɯ˧-ɲi˥!}}} \zh{土司到昆明去了!} \textcolor{Sepia}{\selectlanguage{english}The (feudal) lord has gone to Kunming!} \textcolor{PineGreen}{\selectlanguage{french}Le seigneur est parti à Kunming!}  

\lhead{\firstmark}
\rhead{\botmark}

\subsection{\hspace{-0.5cm} {\Large \textcolor{darkblue}{\textbf{\ipa{ʝi˩ŋɤ˧˥}}}}\hspace{0.5cm}[\kern2pt{\textcolor{darkblue}{\textbf{\ipa{xxxx non-correspondance entre le nombre de morphèmes et le nombre de tons de morphèmes}}}}\kern2pt]} \hypertarget{j££i\string_BN7\string_M\string_T1}{}
\markboth{\textcolor{darkblue}{\textbf{\ipa{ʝi˩ŋɤ˧˥}}}}{}
\textcolor{teal}{\zh{动词}} \hspace{4pt} \zh{声调类:} LM+MH\#.
\zh{往后仰。} \textcolor{Sepia}{\selectlanguage{english}To bend (one's body) backwards.} \textcolor{PineGreen}{\selectlanguage{french}Se courber vers l’arrière.}  ¶ \textcolor{darkblue}{\textbf{\ipa{ʝi˩ŋɤ˧-ze˥}}} \zh{往后仰了} \textcolor{Sepia}{\selectlanguage{english}\mytextsc{pfv}} \textcolor{PineGreen}{\selectlanguage{french}\mytextsc{pfv}}  
 ¶ \textcolor{darkblue}{\textbf{\ipa{ʝi˩ŋɤ˧˥ | tʰi˧-dzi˩}}} \zh{坐着往后仰} \textcolor{Sepia}{\selectlanguage{english}to be seated leaning backwards, to lean against the back of one's seat} \textcolor{PineGreen}{\selectlanguage{french}être assis en s'inclinant vers l'arrière}  

\lhead{\firstmark}
\rhead{\botmark}

\subsection{\hspace{-0.5cm} {\Large \textcolor{darkblue}{\textbf{\ipa{ʝi˩qʰv̩˩}}}}\hspace{0.5cm}[\kern2pt{\textcolor{darkblue}{\textbf{\ipa{ʝi˧qʰv̩˧}}}}\kern2pt]} \hypertarget{j££i\string_Bq\string_hv\string_=\string_B1}{}
\markboth{\textcolor{darkblue}{\textbf{\ipa{ʝi˩qʰv̩˩}}}}{}
\textcolor{teal}{\zh{名词}} \hspace{4pt} \zh{声调类:} L.
\zh{袖子。} \textcolor{Sepia}{\selectlanguage{english}Sleeve.} \textcolor{PineGreen}{\selectlanguage{french}Manche.}  \zh{量词}: \textcolor{darkblue}{\textbf{\ipa{ɭɯ˧}}} 
\lhead{\firstmark}
\rhead{\botmark}

\subsection{\hspace{-0.5cm} {\Large \textcolor{darkblue}{\textbf{\ipa{ʝi˩ʈʂæ˧˥}}}}\hspace{0.5cm}[\kern2pt{\textcolor{darkblue}{\textbf{\ipa{ʝi˩ʈʂæ˧˥}}}}\kern2pt]} \hypertarget{j££i\string_Bt`s`\{\string_M\string_T1}{}
\markboth{\textcolor{darkblue}{\textbf{\ipa{ʝi˩ʈʂæ˧˥}}}}{}
\textcolor{teal}{\zh{名词}} \hspace{4pt} \zh{声调类:} LM+MH\#.
\zh{腰。} \textcolor{Sepia}{\selectlanguage{english}Waist.} \textcolor{PineGreen}{\selectlanguage{french}Taille.}  \zh{量词}: \textcolor{darkblue}{\textbf{\ipa{ɭɯ˧}}} 
\lhead{\firstmark}
\rhead{\botmark}

\subsection{\hspace{-0.5cm} {\Large \textcolor{darkblue}{\textbf{\ipa{ʝi˩˥}}}}\hspace{0.5cm}[\kern2pt{\textcolor{darkblue}{\textbf{\ipa{ʝi˩˥}}}}\kern2pt]} \hypertarget{j££i\string_B\string_T1}{}
\markboth{\textcolor{darkblue}{\textbf{\ipa{ʝi˩˥}}}}{}
\textcolor{teal}{\zh{名词}} \hspace{4pt} \zh{声调类:} LH.
\zh{痘痘。} \textcolor{Sepia}{\selectlanguage{english}Spot, pimple.} \textcolor{PineGreen}{\selectlanguage{french}Bouton.}  ¶ \textcolor{darkblue}{\textbf{\ipa{ʝi˩ tʰv̩˩˥}}} \zh{出痘痘} \textcolor{Sepia}{\selectlanguage{english}to have spots, to get pimples} \textcolor{PineGreen}{\selectlanguage{french}avoir des boutons}  
 \zh{量词}: \textcolor{darkblue}{\textbf{\ipa{ɭɯ˧}}} 
\lhead{\firstmark}
\rhead{\botmark}

\newpage
\section*{\centering- \textcolor{darkblue}{\textbf{\ipa{k}}} -}
\subsection{\hspace{-0.5cm} {\Large \textcolor{darkblue}{\textbf{\ipa{kæ˧ʈʂe˧}}}}\hspace{0.5cm}[\kern2pt{\textcolor{darkblue}{\textbf{\ipa{kæ˧ʈʂe˧}}}}\kern2pt]} \hypertarget{k\{\string_Mt`s`e\string_M1}{}
\markboth{\textcolor{darkblue}{\textbf{\ipa{kæ˧ʈʂe˧}}}}{}
\textcolor{teal}{\zh{名词}} \hspace{4pt} \zh{声调类:} M.
\zh{针灸(汉语借词:干针)。} \textcolor{Sepia}{\selectlanguage{english}Acupuncture needles; acupuncture.} \textcolor{PineGreen}{\selectlanguage{french}Acupuncture.} \zh{当地汉语方言:}\zh{干针。} \zh{【借词】} \zh{干针}
 ¶ \textcolor{darkblue}{\textbf{\ipa{kæ˧ʈʂe˧ lɑ˧˥}}} \zh{扎针灸} \textcolor{Sepia}{\selectlanguage{english}to do an acupuncture session, to use acupuncture needles} \textcolor{PineGreen}{\selectlanguage{french}faire une séance d'acupuncture, placer des aiguilles d'acupuncture}  

\lhead{\firstmark}
\rhead{\botmark}

\subsection{\hspace{-0.5cm} {\Large \textcolor{darkblue}{\textbf{\ipa{kæ˧ʈʂɯ˧}}}}\hspace{0.5cm}[\kern2pt{\textcolor{darkblue}{\textbf{\ipa{kæ˧ʈʂɯ˧}}}}\kern2pt]} \hypertarget{k\{\string_Mt`s`M\string_M1}{}
\markboth{\textcolor{darkblue}{\textbf{\ipa{kæ˧ʈʂɯ˧}}}}{}
\textcolor{teal}{\zh{名词}} \hspace{4pt} \zh{声调类:} M.
\zh{甘蔗。} \textcolor{Sepia}{\selectlanguage{english}Sugar cane.} \textcolor{PineGreen}{\selectlanguage{french}Canne à sucre.}  \zh{【借词】} \zh{甘蔗}

\lhead{\firstmark}
\rhead{\botmark}

\subsection{\hspace{-0.5cm} {\Large \textcolor{darkblue}{\textbf{\ipa{kɤ˧dzi˧}}}}\hspace{0.5cm}[\kern2pt{\textcolor{darkblue}{\textbf{\ipa{kɤ˧dzi˧}}}}\kern2pt]} \hypertarget{k7\string_Mdzi\string_M1}{}
\markboth{\textcolor{darkblue}{\textbf{\ipa{kɤ˧dzi˧}}}}{}
\textcolor{teal}{\zh{动词}} \hspace{4pt} \zh{声调类:} M.
\zh{坐下(在饭桌)。} \textcolor{Sepia}{\selectlanguage{english}To take a seat, to get seated.} \textcolor{PineGreen}{\selectlanguage{french}Prendre place (lors d'un repas, d'une cérémonie...).}  ¶ \textcolor{darkblue}{\textbf{\ipa{ɑ˩ʁo˧-hĩ˧ | kɤ˧dzi˧-ze˧.}}} \zh{家人入座了。} \textcolor{Sepia}{\selectlanguage{english}The members of the family took their seats / got seated.} \textcolor{PineGreen}{\selectlanguage{french}Les gens de la famille se sont assis/ont pris place.}  

\lhead{\firstmark}
\rhead{\botmark}

\subsection{\hspace{-0.5cm} {\Large \textcolor{darkblue}{\textbf{\ipa{kɤ˧kɤ˩}}}}\hspace{0.5cm}[\kern2pt{\textcolor{darkblue}{\textbf{\ipa{kɤ˧kɤ˩}}}}\kern2pt]} \hypertarget{k7\string_Mk7\string_B1}{}
\markboth{\textcolor{darkblue}{\textbf{\ipa{kɤ˧kɤ˩}}}}{}
\textcolor{teal}{\zh{助词}} \hspace{4pt} \zh{声调类:} L\#.
\zh{挨着(坐……)。} \textcolor{Sepia}{\selectlanguage{english}Next to, close to.} \textcolor{PineGreen}{\selectlanguage{french}Proche de, à côté de.}  ¶ \textcolor{darkblue}{\textbf{\ipa{(tso˧\textasciitilde{}tso˧) kɤ˧kɤ˩ | tʰi˧-tɕɯ˥}}} \zh{摆整齐、使均匀,如:一排排挨着} \textcolor{Sepia}{\selectlanguage{english}to arrange, to put in good order} \textcolor{PineGreen}{\selectlanguage{french}ranger des choses en bon ordre}  
 ¶ \textcolor{darkblue}{\textbf{\ipa{[Tiger2] kɤ˧kɤ˩ | tʰi˧-se˥}}} \zh{并排走} \textcolor{Sepia}{\selectlanguage{english}to walk in a line, one behind the other} \textcolor{PineGreen}{\selectlanguage{french}marcher en file indienne}  
 ¶ \textcolor{darkblue}{\textbf{\ipa{kɤ˧kɤ˩ | tʰi˧-dzi˩}}} \zh{挨着坐} \textcolor{Sepia}{\selectlanguage{english}to sit close to one another} \textcolor{PineGreen}{\selectlanguage{french}être assis les uns à côté des autres, proches les uns des autres}  

\lhead{\firstmark}
\rhead{\botmark}

\subsection{\hspace{-0.5cm} {\Large \textcolor{darkblue}{\textbf{\ipa{kɤ˧ljɤ˩}}}}\hspace{0.5cm}[\kern2pt{\textcolor{darkblue}{\textbf{\ipa{kɤ˧ljɤ˩}}}}\kern2pt]} \hypertarget{k7\string_Mlj7\string_B1}{}
\markboth{\textcolor{darkblue}{\textbf{\ipa{kɤ˧ljɤ˩}}}}{}
\textcolor{teal}{\zh{名词}} \hspace{4pt} \zh{声调类:} L\#.
\zh{高粱(汉语借词)。} \textcolor{Sepia}{\selectlanguage{english}Chinese sorghum.} \textcolor{PineGreen}{\selectlanguage{french}Sorgho, gaoliang; céréale dont on se sert pour faire du vin.}  \zh{【借词】} \zh{高粱}
\zh{~【参考】~} \hyperlink{}{\textcolor{darkblue}{\textbf{\ipa{hæ˧ɭɯ\#˥}}}} 
\lhead{\firstmark}
\rhead{\botmark}

\subsection{\hspace{-0.5cm} {\Large \textcolor{darkblue}{\textbf{\ipa{kɤ˧mi˧}}} \textsubscript{1}}\hspace{0.5cm}[\kern2pt{\textcolor{darkblue}{\textbf{\ipa{kɤ˧mi˧}}}}\kern2pt]} \hypertarget{k7\string_Mmi\string_M1}{}
\markboth{\textcolor{darkblue}{\textbf{\ipa{kɤ˧mi˧}}} \textsubscript{1}}{}
\textcolor{teal}{\zh{名词}} \hspace{4pt} \zh{声调类:} M.
\zh{母隼。} \textcolor{Sepia}{\selectlanguage{english}Female falcon.} \textcolor{PineGreen}{\selectlanguage{french}Faucon femelle.}  ¶ \textcolor{darkblue}{\textbf{\ipa{kɤ˩mi˩-kɤ˩pʰv̩˥}}} \zh{母隼与公隼} \textcolor{Sepia}{\selectlanguage{english}female falcon and male falcon} \textcolor{PineGreen}{\selectlanguage{french}faucon femelle et faucon mâle}  
 \zh{量词}: \textcolor{darkblue}{\textbf{\ipa{mi˩}}} 
\lhead{\firstmark}
\rhead{\botmark}

\subsection{\hspace{-0.5cm} {\Large \textcolor{darkblue}{\textbf{\ipa{kɤ˧mi˧}}} \textsubscript{2}}\hspace{0.5cm}[\kern2pt{\textcolor{darkblue}{\textbf{\ipa{kɤ˧mi˧}}}}\kern2pt]} \hypertarget{k7\string_Mmi\string_M2}{}
\markboth{\textcolor{darkblue}{\textbf{\ipa{kɤ˧mi˧}}} \textsubscript{2}}{}
\textcolor{teal}{\zh{名词}} \hspace{4pt} \zh{声调类:} M.
\zh{大坛子,大瓶。} \textcolor{Sepia}{\selectlanguage{english}A large jar; a large bottle.} \textcolor{PineGreen}{\selectlanguage{french}Grande jarre; grande bouteille.}  \zh{量词}: \textcolor{darkblue}{\textbf{\ipa{ɭɯ˧}}} 
\lhead{\firstmark}
\rhead{\botmark}

\subsection{\hspace{-0.5cm} {\Large \textcolor{darkblue}{\textbf{\ipa{kɤ˧mv̩˧˥}}}}\hspace{0.5cm}[\kern2pt{\textcolor{darkblue}{\textbf{\ipa{kɤ˧mv̩˧˥}}}}\kern2pt]} \hypertarget{k7\string_Mmv\string_=\string_M\string_T1}{}
\markboth{\textcolor{darkblue}{\textbf{\ipa{kɤ˧mv̩˧˥}}}}{}
\textcolor{teal}{\zh{名词}} \hspace{4pt} \zh{声调类:} MH\#.
\zh{格母山。} \textcolor{Sepia}{\selectlanguage{english}The Gemu mountain (Yongning).} \textcolor{PineGreen}{\selectlanguage{french}La montagne Gemu (Yongning).}  ¶ \textcolor{darkblue}{\textbf{\ipa{ɬi˧di˩-kɤ˩mv̩˩}}} \zh{永宁格姆山} \textcolor{Sepia}{\selectlanguage{english}Mount Gemu, in Yongning} \textcolor{PineGreen}{\selectlanguage{french}la montagne Gemu de Yongning}  
 ¶ \textcolor{darkblue}{\textbf{\ipa{kɤ˧mv̩˧-hæ̃˧kʰo˥}}} \zh{格姆公主:格姆山别名(格姆山被看作女神)} \textcolor{Sepia}{\selectlanguage{english}the Gemu princess: another name for Mount Gemu, considered as a female deity} \textcolor{PineGreen}{\selectlanguage{french}“la princesse Gemu”; autre nom de la montagne Gemu, considérée comme une divinité féminine}  
 ¶ \textcolor{darkblue}{\textbf{\ipa{kɤ˧mv̩˧˥, | æ˧ʂæ˧, | ŋwɤ˧hɑ̃˩, | ʂwæ˧gv̩\#˥, | nɑ˩tsʰi˩˥ | -tɕʰɤ˧pɤ˧mi\#˥, | qv̩˧ɻ̍˧-ʈʂʰɑ˧nɑ˥ |}}} \zh{永宁地区有固定名字的六座山。其它山,没有重要的象征意义,因此也没有固定名称。} \textcolor{Sepia}{\selectlanguage{english}The six mountains of Yongning that carry a name and have a definite symbolic value. The other mountains do not have comparable symbolic value, and fewer people use specific names for them.} \textcolor{PineGreen}{\selectlanguage{french}Les six montagnes de Yongning qui portent un nom. Les autres sommets du voisinage n'ont pas une valeur symbolique comparable, et ne portent pas de nom communément utilisé.}  

\lhead{\firstmark}
\rhead{\botmark}

\subsection{\hspace{-0.5cm} {\Large \textcolor{darkblue}{\textbf{\ipa{kɤ˧ʈʂɯ˩}}} \textsubscript{1}}\hspace{0.5cm}[\kern2pt{\textcolor{darkblue}{\textbf{\ipa{kɤ˧ʈʂɯ˩}}}}\kern2pt]} \hypertarget{k7\string_Mt`s`M\string_B1}{}
\markboth{\textcolor{darkblue}{\textbf{\ipa{kɤ˧ʈʂɯ˩}}} \textsubscript{1}}{}
\textcolor{teal}{\zh{动词}} \hspace{4pt} \zh{声调类:} L\#.
\zh{讲。} \textcolor{Sepia}{\selectlanguage{english}To tell.} \textcolor{PineGreen}{\selectlanguage{french}Parler, raconter.}  ¶ \textcolor{darkblue}{\textbf{\ipa{hĩ˧-ki˧ | tʰɑ˧-kɤ˧ʈʂɯ˩!}}} \zh{不要告诉人家!} \textcolor{Sepia}{\selectlanguage{english}Don't tell it! / Don't tell anyone!} \textcolor{PineGreen}{\selectlanguage{french}il ne faut pas le dire aux gens! / c'est secret!}  
 ¶ \textcolor{darkblue}{\textbf{\ipa{kɤ˧-tʰɑ˥-ʈʂɯ˩!}}} \zh{不要告诉人家!} \textcolor{Sepia}{\selectlanguage{english}Don't tell it! / Don't tell anyone!} \textcolor{PineGreen}{\selectlanguage{french}il ne faut pas le dire! / c'est secret!}  
 ¶ \textcolor{darkblue}{\textbf{\ipa{njɤ˧-ɳɯ˧ | kɤ˧ʈʂɯ˩-bi˩!}}} \zh{我要说一点事情!} \textcolor{Sepia}{\selectlanguage{english}I'm going to (jump in and) say something!} \textcolor{PineGreen}{\selectlanguage{french}je vais intervenir/je vais dire quelque chose!}  
 ¶ \textcolor{darkblue}{\textbf{\ipa{no˧ | kɤ˧ʈʂɯ˩ dʑo˩-ɲi˩!}}} \zh{你得说话啊!} \textcolor{Sepia}{\selectlanguage{english}You have to say something!} \textcolor{PineGreen}{\selectlanguage{french}il faut que tu dises quelque chose!}  
 ¶ \textcolor{darkblue}{\textbf{\ipa{ʈʂʰɯ˧ | kɤ˧ʈʂɯ˩ | dʑɤ˩˥ | mɤ˧-mv̩˧-sɯ˥! / ʈʂʰɯ˧ | kɤ˧ʈʂɯ˩ dʑɤ˩˥ | mɤ˧-mv̩˧\textasciitilde{}mv̩˧-sɯ˥!}}} \zh{她还不怎么听得懂话!(关于一个不会说话的两岁小孩)} \textcolor{Sepia}{\selectlanguage{english}She does not really understand yet! (About a toddler aged 2 who does not yet speak distinctly or follow conversations)} \textcolor{PineGreen}{\selectlanguage{french}elle ne comprend pas encore grand'chose à ce qu'on dit! / elle ne sait pas encore comprendre ce qu'on dit! (au sujet d'une fillette de moins de 2 ans qui ne parle pas encore)}  
 ¶ \textcolor{darkblue}{\textbf{\ipa{tʰɑ˧-kɤ˧ʈʂɯ˩! | hĩ˧ ɳv̩˧ tʰɑ˧-kʰɯ˩!}}} \zh{不要告诉(人家)!别让人家知道!} \textcolor{Sepia}{\selectlanguage{english}Don't talk about it! Don't let people know!} \textcolor{PineGreen}{\selectlanguage{french}N'en parle pas! il ne faut pas que les gens le sachent!}  
 ¶ \textcolor{darkblue}{\textbf{\ipa{hĩ˧-ki˧ | kɤ˧-mɤ˧-ʈʂɯ˩}}} \zh{不跟人家说(自己做什么事)} \textcolor{Sepia}{\selectlanguage{english}not to tell people; (to do something secretly) without telling anyone} \textcolor{PineGreen}{\selectlanguage{french}(faire quelque chose) en cachette, sans le dire à personne}  
 ¶ \textcolor{darkblue}{\textbf{\ipa{kɤ˧ʈʂɯ˩ ɲi˩}}} \zh{听话,乖(来形容一个孩子)} \textcolor{Sepia}{\selectlanguage{english}well-behaved, obedient (child) (literally: who listens to what (s)he is told)} \textcolor{PineGreen}{\selectlanguage{french}sage (au sujet d'un enfant) (littéralement: qui écoute ce qu'on lui dit)}  
\zh{~【参考】~} \hyperlink{}{\textcolor{darkblue}{\textbf{\ipa{kɤ˧ʈʂɯ˩}}} \textsubscript{2}} 
\lhead{\firstmark}
\rhead{\botmark}

\subsection{\hspace{-0.5cm} {\Large \textcolor{darkblue}{\textbf{\ipa{kɤ˧ʈʂɯ˩}}} \textsubscript{2}}\hspace{0.5cm}[\kern2pt{\textcolor{darkblue}{\textbf{\ipa{kɤ˧ʈʂɯ˩}}}}\kern2pt]} \hypertarget{k7\string_Mt`s`M\string_B2}{}
\markboth{\textcolor{darkblue}{\textbf{\ipa{kɤ˧ʈʂɯ˩}}} \textsubscript{2}}{}
\textcolor{teal}{\zh{名词}} \hspace{4pt} \zh{声调类:} L\#.
\zh{话。} \textcolor{Sepia}{\selectlanguage{english}Speech.} \textcolor{PineGreen}{\selectlanguage{french}Parole.}  ¶ \textcolor{darkblue}{\textbf{\ipa{kɤ˧ʈʂɯ˩ ʝi˩}}} \zh{答应,誓、发誓。两个人发生矛盾的时候,如果无法协调,他们会去大寺,在神像前讲述他们各自的观点,发誓他们自己讲的是真的。神会惩罚说谎的人(他家会有祸害)。} \textcolor{Sepia}{\selectlanguage{english}to promise; to make an oath; also: to swear before the gods: when people had a disagreement that they were unable to settle, they would go to the monastery and present their point of view before the gods, swearing that they were telling the truth; the gods would then punish the guilty one (through plagues and misfortunes).} \textcolor{PineGreen}{\selectlanguage{french}promettre; aussi: jurer ses grands dieux, prêter serment devant les Dieux: lorsque deux personnes avaient un différend qu'elles ne parvenaient pas à trancher, elles allaient raconter chacune sa version des faits devant les Dieux (au monastère); ceux-ci punissaient ensuite le coupable (par des calamités qui frappaient la famille du coupable).}  
 ¶ \textcolor{darkblue}{\textbf{\ipa{ʈʂʰɯ˧ | kɤ˧ʈʂɯ˩-ʝi˩}}} \zh{他答应} \textcolor{Sepia}{\selectlanguage{english}(s)he promises} \textcolor{PineGreen}{\selectlanguage{french}il/elle promet}  
 ¶ \textcolor{darkblue}{\textbf{\ipa{ʈʂʰɯ˧ | kɤ˧ʈʂɯ˩ | mɤ˧-ʝi˥!}}} \zh{他没有答应!} \textcolor{Sepia}{\selectlanguage{english}(s)he has not promised!} \textcolor{PineGreen}{\selectlanguage{french}il n'a pas promis!}  
 ¶ \textcolor{darkblue}{\textbf{\ipa{hĩ˧-kɤ˧ʈʂɯ˥ ɲi˩}}} \zh{听别人的建议、把别人的话当回事} \textcolor{Sepia}{\selectlanguage{english}to listen to people's advice, to pay attention to what other people say (a good attitude in the consultant's view)} \textcolor{PineGreen}{\selectlanguage{french}écouter les conseils d'autrui, prêter attention à la parole d'autrui, écouter les bons conseils (attitude jugée positive et souhaitable par la consultante)}  
 ¶ \textcolor{darkblue}{\textbf{\ipa{hĩ˧-kɤ˧ʈʂɯ˥ | le˧-ɲi˥}}} \zh{同上} \textcolor{Sepia}{\selectlanguage{english}as above} \textcolor{PineGreen}{\selectlanguage{french}même sens}  
 ¶ \textcolor{darkblue}{\textbf{\ipa{hĩ˧-kɤ˧ʈʂɯ˥ | mɤ˧-ɲi˥}}} \zh{听不进去别人的意见与建议} \textcolor{Sepia}{\selectlanguage{english}to fail to listen to people's advice} \textcolor{PineGreen}{\selectlanguage{french}ne pas écouter les bons conseils, ne pas prêter attention à ce qu'on vous dit}  
 \zh{量词}: \textcolor{darkblue}{\textbf{\ipa{kʰwɤ˥}}} \zh{~【参考】~} \hyperlink{}{\textcolor{darkblue}{\textbf{\ipa{kɤ˧ʈʂɯ˩}}} \textsubscript{1}} 
\lhead{\firstmark}
\rhead{\botmark}

\subsection{\hspace{-0.5cm} {\Large \textcolor{darkblue}{\textbf{\ipa{kɤ˧v̩\#˥}}}}\hspace{0.5cm}[\kern2pt{\textcolor{darkblue}{\textbf{\ipa{kɤ˧v̩˧}}}}\kern2pt]} \hypertarget{k7\string_Mv\string_=\#\string_T1}{}
\markboth{\textcolor{darkblue}{\textbf{\ipa{kɤ˧v̩\#˥}}}}{}
\textcolor{teal}{\zh{名词}} \hspace{4pt} \zh{声调类:} \#H.
\zh{护符,护身符。} \textcolor{Sepia}{\selectlanguage{english}Amulet.} \textcolor{PineGreen}{\selectlanguage{french}Amulette.}  \zh{量词}: \textcolor{darkblue}{\textbf{\ipa{ɭɯ˧}}} 
\lhead{\firstmark}
\rhead{\botmark}

\subsection{\hspace{-0.5cm} {\Large \textcolor{darkblue}{\textbf{\ipa{kɤ˧wɤ\#˥}}}}\hspace{0.5cm}[\kern2pt{\textcolor{darkblue}{\textbf{\ipa{kɤ˧wɤ˧}}}}\kern2pt]} \hypertarget{k7\string_Mw7\#\string_T1}{}
\markboth{\textcolor{darkblue}{\textbf{\ipa{kɤ˧wɤ\#˥}}}}{}
\textcolor{teal}{\zh{名词}} \hspace{4pt} \zh{声调类:} \#H.
\zh{缘分、共同命运。} \textcolor{Sepia}{\selectlanguage{english}Predestination, predestined affinity.} \textcolor{PineGreen}{\selectlanguage{french}Destinée, affinité prédestinée.}  ¶ \textcolor{darkblue}{\textbf{\ipa{kɤ˧wɤ˧-ljɤ˧˥}}} \zh{有缘分、有共同命运} \textcolor{Sepia}{\selectlanguage{english}to have a predestined affinity; to have a common destiny} \textcolor{PineGreen}{\selectlanguage{french}avoir une affinité prédestinée, avoir un destin commun}  

\lhead{\firstmark}
\rhead{\botmark}

\subsection{\hspace{-0.5cm} {\Large \textcolor{darkblue}{\textbf{\ipa{kɤ˧zo\#˥}}}}\hspace{0.5cm}[\kern2pt{\textcolor{darkblue}{\textbf{\ipa{kɤ˧zo˧}}}}\kern2pt]} \hypertarget{k7\string_Mzo\#\string_T1}{}
\markboth{\textcolor{darkblue}{\textbf{\ipa{kɤ˧zo\#˥}}}}{}
\textcolor{teal}{\zh{名词}} \hspace{4pt} \zh{声调类:} \#H.
\zh{男性名字。} \textcolor{Sepia}{\selectlanguage{english}Masculine given name.} \textcolor{PineGreen}{\selectlanguage{french}Prénom masculin.} 
\lhead{\firstmark}
\rhead{\botmark}

\subsection{\hspace{-0.5cm} {\Large \textcolor{darkblue}{\textbf{\ipa{kɤ˩}}}}\hspace{0.5cm}[\kern2pt{\textcolor{darkblue}{\textbf{\ipa{kɤ˥}}}}\kern2pt]} \hypertarget{k7\string_B1}{}
\markboth{\textcolor{darkblue}{\textbf{\ipa{kɤ˩}}}}{}
\textcolor{teal}{\zh{名词}} \hspace{4pt} \zh{声调类:} L.
\zh{瓶子。} \textcolor{Sepia}{\selectlanguage{english}Bottle.} \textcolor{PineGreen}{\selectlanguage{french}Bouteille.}  ¶ \textcolor{darkblue}{\textbf{\ipa{ʐɯ˧-kɤ˩}}} \zh{酒瓶} \textcolor{Sepia}{\selectlanguage{english}wine bottle} \textcolor{PineGreen}{\selectlanguage{french}bouteille d'alcool}  
 ¶ \textcolor{darkblue}{\textbf{\ipa{ʐɯ˧ ɖɯ˧-kɤ˩}}} \zh{一瓶酒} \textcolor{Sepia}{\selectlanguage{english}one bottle of wine} \textcolor{PineGreen}{\selectlanguage{french}une bouteille d'alcool}  
 \zh{量词}: \textcolor{darkblue}{\textbf{\ipa{ɭɯ˧}}} 
\lhead{\firstmark}
\rhead{\botmark}

\subsection{\hspace{-0.5cm} {\Large \textcolor{darkblue}{\textbf{\ipa{kɤ˩\textsubscript{a}}}}}\hspace{0.5cm}[\kern2pt{\textcolor{darkblue}{\textbf{\ipa{kɤ˩˥}}}}\kern2pt]} \hypertarget{k7\string_Ba1}{}
\markboth{\textcolor{darkblue}{\textbf{\ipa{kɤ˩\textsubscript{a}}}}}{}
\textcolor{teal}{\zh{量词}} \hspace{4pt} \zh{声调类:} L\textsubscript{a}.
\zh{量词:瓶。} \textcolor{Sepia}{\selectlanguage{english}A bottle of.} \textcolor{PineGreen}{\selectlanguage{french}Classificateur des bouteilles.}  ¶ \textcolor{darkblue}{\textbf{\ipa{kɤ˩zo˩˥}}} \zh{一小瓶} \textcolor{Sepia}{\selectlanguage{english}small bottle} \textcolor{PineGreen}{\selectlanguage{french}petite bouteille}  
 ¶ \textcolor{darkblue}{\textbf{\ipa{ʈʂʰɯ˧-kɤ˥}}} \zh{\mytextsc{指示代词} \string_} \textcolor{Sepia}{\selectlanguage{english}\mytextsc{dem} \string_ (tone: H\# / H\$)} \textcolor{PineGreen}{\selectlanguage{french}\mytextsc{dem} \string_ (tone: H\# / H\$)}  

\lhead{\firstmark}
\rhead{\botmark}

\subsection{\hspace{-0.5cm} {\Large \textcolor{darkblue}{\textbf{\ipa{kɤ˩\textasciitilde{}kɤ˧˥}}}}\hspace{0.5cm}[\kern2pt{\textcolor{darkblue}{\textbf{\ipa{kɤ˧kɤ˧˥}}}}\kern2pt]} \hypertarget{k7\string_B~k7\string_M\string_T1}{}
\markboth{\textcolor{darkblue}{\textbf{\ipa{kɤ˩\textasciitilde{}kɤ˧˥}}}}{}
\textcolor{teal}{\zh{动词}} \hspace{4pt} \zh{声调类:} MH.
\zh{敲、拍。} \textcolor{Sepia}{\selectlanguage{english}To knock, to tap, to poke.} \textcolor{PineGreen}{\selectlanguage{french}Tapoter.}  ¶ \textcolor{darkblue}{\textbf{\ipa{kʰi˧ kɤ˥\textasciitilde{}kɤ˩}}} \zh{敲门} \textcolor{Sepia}{\selectlanguage{english}to knock at the door} \textcolor{PineGreen}{\selectlanguage{french}frapper à la porte}  
 ¶ \textcolor{darkblue}{\textbf{\ipa{njɤ˧-ɳɯ˧ | no˧ | kɤ˩\textasciitilde{}kɤ˧-bi˥!}}} \zh{我要打你屁股了!(大人对孩子说)} \textcolor{Sepia}{\selectlanguage{english}I am going to slap your buttocks! (An adult threatens a child.)} \textcolor{PineGreen}{\selectlanguage{french}Je vais te donner une tape/une fessée! (Menace d'un adulte à un enfant)}  
 ¶ \textcolor{darkblue}{\textbf{\ipa{ʈʂo˧tsɯ˥ kɤ˩\textasciitilde{}kɤ˩ (-ze˩/-bi˩)}}} \zh{拍拍桌子} \textcolor{Sepia}{\selectlanguage{english}to tap the table, to rap on the table} \textcolor{PineGreen}{\selectlanguage{french}heurter la table, taper sur la table}  
 ¶ \textcolor{darkblue}{\textbf{\ipa{gv̩˧dv̩˧ kɤ˧\textasciitilde{}kɤ˩}}} \zh{敲敲背} \textcolor{Sepia}{\selectlanguage{english}to tap someone's back (to relieve back pain)} \textcolor{PineGreen}{\selectlanguage{french}tapoter sur le dos de quelqu'un (pour soulager un mal de dos)}  
\zh{~【参考】~} \hyperlink{}{\textcolor{darkblue}{\textbf{\ipa{kɤ˧˥}}}} 
\lhead{\firstmark}
\rhead{\botmark}

\subsection{\hspace{-0.5cm} {\Large \textcolor{darkblue}{\textbf{\ipa{kɤ˩lo˧˥}}}}\hspace{0.5cm}[\kern2pt{\textcolor{darkblue}{\textbf{\ipa{kɤ˩lo˧˥}}}}\kern2pt]} \hypertarget{k7\string_Blo\string_M\string_T1}{}
\markboth{\textcolor{darkblue}{\textbf{\ipa{kɤ˩lo˧˥}}}}{}
\textcolor{teal}{\zh{名词}} \hspace{4pt} \zh{声调类:} LM+MH\#.
\zh{树枝。} \textcolor{Sepia}{\selectlanguage{english}Branch.} \textcolor{PineGreen}{\selectlanguage{french}Branche.}  ¶ \textcolor{darkblue}{\textbf{\ipa{si˧dzi˩-kɤ˩lo˩}}} \zh{树枝} \textcolor{Sepia}{\selectlanguage{english}branch of tree} \textcolor{PineGreen}{\selectlanguage{french}branche d'arbre}  
 ¶ \textcolor{darkblue}{\textbf{\ipa{si˧-kɤ˥lo˩}}} \zh{同上} \textcolor{Sepia}{\selectlanguage{english}as above} \textcolor{PineGreen}{\selectlanguage{french}idem}  
 \zh{量词}: \textcolor{darkblue}{\textbf{\ipa{kɤ˧˥}}} 
\lhead{\firstmark}
\rhead{\botmark}

\subsection{\hspace{-0.5cm} {\Large \textcolor{darkblue}{\textbf{\ipa{kɤ˩-nɑ˧mi˧}}}}\hspace{0.5cm}[\kern2pt{\textcolor{darkblue}{\textbf{\ipa{kɤ˧nɑ˧mi˧}}}}\kern2pt]} \hypertarget{k7\string_B-nA\string_Mmi\string_M1}{}
\markboth{\textcolor{darkblue}{\textbf{\ipa{kɤ˩-nɑ˧mi˧}}}}{}
\textcolor{teal}{\zh{名词}} \hspace{4pt} \zh{声调类:} L-.
\zh{老鹰。} \textcolor{Sepia}{\selectlanguage{english}Eagle.} \textcolor{PineGreen}{\selectlanguage{french}Aigle.}  \zh{量词}: \textcolor{darkblue}{\textbf{\ipa{mi˩}}} 
\lhead{\firstmark}
\rhead{\botmark}

\subsection{\hspace{-0.5cm} {\Large \textcolor{darkblue}{\textbf{\ipa{kɤ˩pʰv̩˩}}}}\hspace{0.5cm}[\kern2pt{\textcolor{darkblue}{\textbf{\ipa{kɤ˩pʰv̩˩˥}}}}\kern2pt]} \hypertarget{k7\string_Bp\string_hv\string_=\string_B1}{}
\markboth{\textcolor{darkblue}{\textbf{\ipa{kɤ˩pʰv̩˩}}}}{}
\textcolor{teal}{\zh{名词}} \hspace{4pt} \zh{声调类:} L.
\zh{公隼。} \textcolor{Sepia}{\selectlanguage{english}Male falcon.} \textcolor{PineGreen}{\selectlanguage{french}Faucon mâle.}  ¶ \textcolor{darkblue}{\textbf{\ipa{kɤ˩pʰv̩˩-kɤ˩mi˥}}} \zh{公隼与母隼} \textcolor{Sepia}{\selectlanguage{english}male falcon and female falcon} \textcolor{PineGreen}{\selectlanguage{french}faucon mâle et faucon femelle}  
 \zh{量词}: \textcolor{darkblue}{\textbf{\ipa{mi˩}}} 
\lhead{\firstmark}
\rhead{\botmark}

\subsection{\hspace{-0.5cm} {\Large \textcolor{darkblue}{\textbf{\ipa{kɤ˩-tjɤ˧ljɤ\#˥}}}}\hspace{0.5cm}[\kern2pt{\textcolor{darkblue}{\textbf{\ipa{kɤ˧tjɤ˧ljɤ˧}}}}\kern2pt]} \hypertarget{k7\string_B-tj7\string_Mlj7\#\string_T1}{}
\markboth{\textcolor{darkblue}{\textbf{\ipa{kɤ˩-tjɤ˧ljɤ\#˥}}}}{}
\textcolor{teal}{\zh{名词}} \hspace{4pt} \zh{声调类:} L-\#H.
\zh{铃铛。} \textcolor{Sepia}{\selectlanguage{english}Small bell hung to an animal's neck (e.g. horse's bell).} \textcolor{PineGreen}{\selectlanguage{french}Clochette s'accrochant autour du cou (ex.: clochette d'un cheval).}  \zh{量词}: \textcolor{darkblue}{\textbf{\ipa{ɭɯ˧}}} 
\lhead{\firstmark}
\rhead{\botmark}

\subsection{\hspace{-0.5cm} {\Large \textcolor{darkblue}{\textbf{\ipa{kɤ˩zo˩}}} \textsubscript{1}}\hspace{0.5cm}[\kern2pt{\textcolor{darkblue}{\textbf{\ipa{kɤ˩zo˩˥}}}}\kern2pt]} \hypertarget{k7\string_Bzo\string_B1}{}
\markboth{\textcolor{darkblue}{\textbf{\ipa{kɤ˩zo˩}}} \textsubscript{1}}{}
\textcolor{teal}{\zh{名词}} \hspace{4pt} \zh{声调类:} L.
\zh{小隼。} \textcolor{Sepia}{\selectlanguage{english}Baby falcon.} \textcolor{PineGreen}{\selectlanguage{french}Bébé faucon.} 
\lhead{\firstmark}
\rhead{\botmark}

\subsection{\hspace{-0.5cm} {\Large \textcolor{darkblue}{\textbf{\ipa{kɤ˩zo˩}}} \textsubscript{2}}\hspace{0.5cm}[\kern2pt{\textcolor{darkblue}{\textbf{\ipa{kɤ˩zo˩˥}}}}\kern2pt]} \hypertarget{k7\string_Bzo\string_B2}{}
\markboth{\textcolor{darkblue}{\textbf{\ipa{kɤ˩zo˩}}} \textsubscript{2}}{}
\textcolor{teal}{\zh{名词}} \hspace{4pt} \zh{声调类:} L.
\zh{小瓶子。} \textcolor{Sepia}{\selectlanguage{english}Small bottle.} \textcolor{PineGreen}{\selectlanguage{french}Petite bouteille.} 
\lhead{\firstmark}
\rhead{\botmark}

\subsection{\hspace{-0.5cm} {\Large \textcolor{darkblue}{\textbf{\ipa{kɤ˩xxxx}}}}\hspace{0.5cm}[\kern2pt{\textcolor{darkblue}{\textbf{\ipa{xxxx groupe tonal entier sans aucun ton}}}}\kern2pt]} \hypertarget{k7\string_Bxxxx1}{}
\markboth{\textcolor{darkblue}{\textbf{\ipa{kɤ˩xxxx}}}}{}
\textcolor{teal}{\zh{动词}} \hspace{4pt} \zh{声调类:} xxxx vérifier : L\textsubscript{a}? L\textsubscript{b}?.
\zh{锯(木头)。} \textcolor{Sepia}{\selectlanguage{english}To saw.} \textcolor{PineGreen}{\selectlanguage{french}Scier (du bois).} 
\lhead{\firstmark}
\rhead{\botmark}

\subsection{\hspace{-0.5cm} {\Large \textcolor{darkblue}{\textbf{\ipa{kɤ˧˥}}}}\hspace{0.5cm}[\kern2pt{\textcolor{darkblue}{\textbf{\ipa{kɤ˧˥}}}}\kern2pt]} \hypertarget{k7\string_M\string_T1}{}
\markboth{\textcolor{darkblue}{\textbf{\ipa{kɤ˧˥}}}}{}
\textcolor{teal}{\zh{动词}} \hspace{4pt} \zh{声调类:} MH.
\zh{敲门。} \textcolor{Sepia}{\selectlanguage{english}To knock on the door.} \textcolor{PineGreen}{\selectlanguage{french}Frapper à la porte, heurter à la porte.}  ¶ \textcolor{darkblue}{\textbf{\ipa{tʰi˧-kɤ˧˥}}} \zh{\mytextsc{dur}} \textcolor{Sepia}{\selectlanguage{english}\mytextsc{dur}} \textcolor{PineGreen}{\selectlanguage{french}\mytextsc{dur}}  
\zh{~【参考】~} \textcolor{darkblue}{\textbf{\ipa{kɤ˩kɤ˧˥}}} 
\lhead{\firstmark}
\rhead{\botmark}

\subsection{\hspace{-0.5cm} {\Large \textcolor{darkblue}{\textbf{\ipa{kɤ˧˥\textsubscript{a}}}} \textsubscript{1}}\hspace{0.5cm}[\kern2pt{\textcolor{darkblue}{\textbf{\ipa{kɤ˧˥}}}}\kern2pt]} \hypertarget{k7\string_M\string_Ta1}{}
\markboth{\textcolor{darkblue}{\textbf{\ipa{kɤ˧˥\textsubscript{a}}}} \textsubscript{1}}{}
\textcolor{teal}{\zh{量词}} \hspace{4pt} \zh{声调类:} MH\textsubscript{a}.
\zh{量词:棍子、树枝(一根)。} \textcolor{Sepia}{\selectlanguage{english}Classifier for sticks/rods.} \textcolor{PineGreen}{\selectlanguage{french}Classificateur des bâtons.}  ¶ \textcolor{darkblue}{\textbf{\ipa{si˧-kɤ˧˥ | ɖɯ˧-kɤ˧˥}}} \zh{一根树枝} \textcolor{Sepia}{\selectlanguage{english}a branch (of a tree)} \textcolor{PineGreen}{\selectlanguage{french}une branche (d'arbre)}  

\lhead{\firstmark}
\rhead{\botmark}

\subsection{\hspace{-0.5cm} {\Large \textcolor{darkblue}{\textbf{\ipa{kɤ˧˥\textsubscript{a}}}} \textsubscript{2}}\hspace{0.5cm}[\kern2pt{\textcolor{darkblue}{\textbf{\ipa{kɤ˧˥}}}}\kern2pt]} \hypertarget{k7\string_M\string_Ta2}{}
\markboth{\textcolor{darkblue}{\textbf{\ipa{kɤ˧˥\textsubscript{a}}}} \textsubscript{2}}{}
\textcolor{teal}{\zh{量词}} \hspace{4pt} \zh{声调类:} MH\textsubscript{a}.
\zh{量词:地(一片)。} \textcolor{Sepia}{\selectlanguage{english}A tract of land.} \textcolor{PineGreen}{\selectlanguage{french}Une étendue de terre.} 
\lhead{\firstmark}
\rhead{\botmark}

\subsection{\hspace{-0.5cm} {\Large \textcolor{darkblue}{\textbf{\ipa{kɤ˧˥tʰɑ˩}}}}\hspace{0.5cm}[\kern2pt{\textcolor{darkblue}{\textbf{\ipa{kɤ˧tʰɑ˧˥}}}}\kern2pt]} \hypertarget{k7\string_M\string_Tt\string_hA\string_B1}{}
\markboth{\textcolor{darkblue}{\textbf{\ipa{kɤ˧˥tʰɑ˩}}}}{}
\textcolor{teal}{\zh{名词}} \hspace{4pt} \zh{声调类:} MH+L.
\zh{一个姓。这个姓,永宁有两家。} \textcolor{Sepia}{\selectlanguage{english}A family name from Yongning. There are two families in Yongning that carry this name. This is one of the first three clans who settled in the vicinity of the Yongning monastery, the other two being \textcolor{darkblue}{\textbf{\ipa{/ə˧lɑ˧/}}} and \textcolor{darkblue}{\textbf{\ipa{/lɑ˧tʰɑ˧mi˥\$/}}}.} \textcolor{PineGreen}{\selectlanguage{french}Nom de clan/famille étendue. Deux familles portent ce nom à Yongning. C'est l'un des trois premiers clans à s'être établis à proximité du monastère de Yongning, les deux autres étant \textcolor{darkblue}{\textbf{\ipa{/ə˧lɑ˧/}}} et \textcolor{darkblue}{\textbf{\ipa{/lɑ˧tʰɑ˧mi˥\$/}}}.}  ¶ \textcolor{darkblue}{\textbf{\ipa{kɤ˧˥tʰɑ˩=ɻ̍˩}}} \zh{\textcolor{darkblue}{\textbf{\ipa{/kɤ˧˥tʰɑ˩/}}}家族} \textcolor{Sepia}{\selectlanguage{english}the \textcolor{darkblue}{\textbf{\ipa{/kɤ˧˥tʰɑ˩/}}} clan} \textcolor{PineGreen}{\selectlanguage{french}le clan \textcolor{darkblue}{\textbf{\ipa{/kɤ˧˥tʰɑ˩/}}}}  

\lhead{\firstmark}
\rhead{\botmark}

\subsection{\hspace{-0.5cm} {\Large \textcolor{darkblue}{\textbf{\ipa{kɤ˩˧}}}}\hspace{0.5cm}[\kern2pt{\textcolor{darkblue}{\textbf{\ipa{kɤ˩˥}}}}\kern2pt]} \hypertarget{k7\string_B\string_M1}{}
\markboth{\textcolor{darkblue}{\textbf{\ipa{kɤ˩˧}}}}{}
\textcolor{teal}{\zh{名词}} \hspace{4pt} \zh{声调类:} LM.
\zh{隼、“小鹰”。} \textcolor{Sepia}{\selectlanguage{english}Falcon.} \textcolor{PineGreen}{\selectlanguage{french}Buse, faucon.}  ¶ \textcolor{darkblue}{\textbf{\ipa{kɤ˩ hwæ˧-ze˧}}} \zh{买了隼} \textcolor{Sepia}{\selectlanguage{english}...bought (a) falcon} \textcolor{PineGreen}{\selectlanguage{french}...a acheté (un) faucon}  
 ¶ \textcolor{darkblue}{\textbf{\ipa{kɤ˩ dzɯ˧-ze˩}}} \zh{吃了隼} \textcolor{Sepia}{\selectlanguage{english}...ate (a) falcon} \textcolor{PineGreen}{\selectlanguage{french}...a mangé (un) faucon}  
 \zh{量词}: \textcolor{darkblue}{\textbf{\ipa{mi˩}}} 
\lhead{\firstmark}
\rhead{\botmark}

\subsection{\hspace{-0.5cm} {\Large \textcolor{darkblue}{\textbf{\ipa{ki˥\textsubscript{a}}}}}\hspace{0.5cm}[\kern2pt{\textcolor{darkblue}{\textbf{\ipa{ki˥}}}}\kern2pt]} \hypertarget{ki\string_Ta1}{}
\markboth{\textcolor{darkblue}{\textbf{\ipa{ki˥\textsubscript{a}}}}}{}
\textcolor{teal}{\zh{量词}} \hspace{4pt} \zh{声调类:} H\textsubscript{a}.
\zh{量词:加上数词‘一’,这个量词表示‘一起’。} \textcolor{Sepia}{\selectlanguage{english}In association with the numeral 'one', this classifier means 'together'.} \textcolor{PineGreen}{\selectlanguage{french}En association avec le numéral 'un', ce classificateur signifie 'ensemble'.}  ¶ \textcolor{darkblue}{\textbf{\ipa{ɖɯ˧-ki˥}}} \zh{一起(共事)} \textcolor{Sepia}{\selectlanguage{english}together} \textcolor{PineGreen}{\selectlanguage{french}ensemble}  
 ¶ \textcolor{darkblue}{\textbf{\ipa{ɖɯ˧-ki˧ tʰv̩˧}}} \zh{同时到达} \textcolor{Sepia}{\selectlanguage{english}to arrive together/at the same time} \textcolor{PineGreen}{\selectlanguage{french}arriver ensemble/en même temps}  
 ¶ \textcolor{darkblue}{\textbf{\ipa{ɖɯ˧-ʝi˧-ɳɯ˧ tsʰɯ˧˥, | ɖɯ˧-ki˧ tʰv̩˧!}}} \zh{从一个地方,一起到!} \textcolor{Sepia}{\selectlanguage{english}Coming from the same place, we arrive together!} \textcolor{PineGreen}{\selectlanguage{french}Venant du même endroit, (on) arrive ensemble!}  
 ¶ \textcolor{darkblue}{\textbf{\ipa{ɖɯ˧-ki˧ dzi˧˥}}} \zh{住在一起} \textcolor{Sepia}{\selectlanguage{english}to live together} \textcolor{PineGreen}{\selectlanguage{french}habiter ensemble}  

\lhead{\firstmark}
\rhead{\botmark}

\subsection{\hspace{-0.5cm} {\Large \textcolor{darkblue}{\textbf{\ipa{‑ki˧}}}}\hspace{0.5cm}[\kern2pt{\textcolor{darkblue}{\textbf{\ipa{ki˥}}}}\kern2pt]} \hypertarget{‑ki\string_M1}{}
\markboth{\textcolor{darkblue}{\textbf{\ipa{‑ki˧}}}}{}
\textcolor{teal}{\zh{后缀}} \hspace{4pt} \zh{声调类:} M.
\zh{对\mytextsc{与格}/向格。} \textcolor{Sepia}{\selectlanguage{english}Dative (particle indicating the recipient) / allative (indicating a direction).} \textcolor{PineGreen}{\selectlanguage{french}Datif/allatif.}  ¶ \textcolor{darkblue}{\textbf{\ipa{ʈʂʰɯ˧-ki˧ ʐwɤ˧˥}}} \zh{给他讲} \textcolor{Sepia}{\selectlanguage{english}to speak to her/him} \textcolor{PineGreen}{\selectlanguage{french}lui dire, lui parler}  
 ¶ \textcolor{darkblue}{\textbf{\ipa{ʈʂʰɯ˧-ki˧ ʐwɤ˧-bi˥}}} \zh{要给他讲} \textcolor{Sepia}{\selectlanguage{english}as above, with immediate future} \textcolor{PineGreen}{\selectlanguage{french}idem+futur immédiat}  
 ¶ \textcolor{darkblue}{\textbf{\ipa{ə˧mɑ˧-ɳɯ˧ | njɤ˧-ki˧ | nɑ˩-ʐwɤ˧ so˩!}}} \zh{阿妈教我摩梭话!} \textcolor{Sepia}{\selectlanguage{english}Ama teaches me the Na language!} \textcolor{PineGreen}{\selectlanguage{french}Ama m'enseigne la langue na!}  

\lhead{\firstmark}
\rhead{\botmark}

\subsection{\hspace{-0.5cm} {\Large \textcolor{darkblue}{\textbf{\ipa{ki˧\textsubscript{a}}}}}\hspace{0.5cm}[\kern2pt{\textcolor{darkblue}{\textbf{\ipa{ki˥}}}}\kern2pt]} \hypertarget{ki\string_Ma1}{}
\markboth{\textcolor{darkblue}{\textbf{\ipa{ki˧\textsubscript{a}}}}}{}
\textcolor{teal}{\zh{动词}} \hspace{4pt} \zh{声调类:} M\textsubscript{a}.
\zh{给、传、献给、发(工资),嫁给。} \textcolor{Sepia}{\selectlanguage{english}To give, to pass on, to transmit, to offer.} \textcolor{PineGreen}{\selectlanguage{french}Donner, passer, transmettre.}  ¶ \textcolor{darkblue}{\textbf{\ipa{ki˧\textasciitilde{}ki˩}}} \zh{重叠} \textcolor{Sepia}{\selectlanguage{english}\mytextsc{red}} \textcolor{PineGreen}{\selectlanguage{french}\mytextsc{red}}  
 ¶ \textcolor{darkblue}{\textbf{\ipa{tso˧\textasciitilde{}tso˧-ki˩}}} \zh{给东西} \textcolor{Sepia}{\selectlanguage{english}to give things} \textcolor{PineGreen}{\selectlanguage{french}donner des choses}  
 ¶ \textcolor{darkblue}{\textbf{\ipa{tso˧\textasciitilde{}tso˧ ki˧\textasciitilde{}ki˥}}} \zh{给东西} \textcolor{Sepia}{\selectlanguage{english}to give things} \textcolor{PineGreen}{\selectlanguage{french}donner des choses}  
 ¶ \textcolor{darkblue}{\textbf{\ipa{hĩ˧-ki˧ ki˩}}} \zh{1.许配给人家。2.嫁给人} \textcolor{Sepia}{\selectlanguage{english}1. to give to someone. 2. to give oneself (in marriage) to someone, to marry someone (for a woman)} \textcolor{PineGreen}{\selectlanguage{french}1. donner à quelqu'un. 2. se donner en mariage à quelqu'un (pour une femme)}  
 ¶ \textcolor{darkblue}{\textbf{\ipa{hĩ˧-ki˧ | ɖwæ˧˥ | tʰi˧-ki˧}}} \zh{大方} \textcolor{Sepia}{\selectlanguage{english}to be generous, to be open-handed} \textcolor{PineGreen}{\selectlanguage{french}être généreux}  
 ¶ \textcolor{darkblue}{\textbf{\ipa{hĩ˧-ki˧ ki˩ fv̩˩}}} \zh{爱送礼,爱给别人送东西} \textcolor{Sepia}{\selectlanguage{english}to like to make gifts, to like to give things to people} \textcolor{PineGreen}{\selectlanguage{french}qui aime faire des cadeaux, qui aime donner des choses aux gens}  
 ¶ \textcolor{darkblue}{\textbf{\ipa{pʰɤ˧bɤ˧ ki˧ (-bi˧)}}} \zh{送礼物} \textcolor{Sepia}{\selectlanguage{english}to offer gifts} \textcolor{PineGreen}{\selectlanguage{french}donner des cadeaux}  
 ¶ \textcolor{darkblue}{\textbf{\ipa{hɑ˧ ki˩}}} \zh{喂饭} \textcolor{Sepia}{\selectlanguage{english}to feed, to give food} \textcolor{PineGreen}{\selectlanguage{french}nourrir, donner à manger}  

\lhead{\firstmark}
\rhead{\botmark}

\subsection{\hspace{-0.5cm} {\Large \textcolor{darkblue}{\textbf{\ipa{ki˧li˥}}}}\hspace{0.5cm}[\kern2pt{\textcolor{darkblue}{\textbf{\ipa{ki˧li˥}}}}\kern2pt]} \hypertarget{ki\string_Mli\string_T1}{}
\markboth{\textcolor{darkblue}{\textbf{\ipa{ki˧li˥}}}}{}
\textcolor{teal}{\zh{助词}} \hspace{4pt} \zh{声调类:} H\#.
\zh{乱七八糟。} \textcolor{Sepia}{\selectlanguage{english}In a mess.} \textcolor{PineGreen}{\selectlanguage{french}En désordre (formulation expressive, quasi-onomatopéique).} 
\lhead{\firstmark}
\rhead{\botmark}

\subsection{\hspace{-0.5cm} {\Large \textcolor{darkblue}{\textbf{\ipa{ki˧zo\#˥}}}}\hspace{0.5cm}[\kern2pt{\textcolor{darkblue}{\textbf{\ipa{ki˧zo˧}}}}\kern2pt]} \hypertarget{ki\string_Mzo\#\string_T1}{}
\markboth{\textcolor{darkblue}{\textbf{\ipa{ki˧zo\#˥}}}}{}
\textcolor{teal}{\zh{名词}} \hspace{4pt} \zh{声调类:} \#H.
\zh{男女通用名。} \textcolor{Sepia}{\selectlanguage{english}A unixex given name: a given name used for both men and women.} \textcolor{PineGreen}{\selectlanguage{french}Prénom unisexe: prénom utilisé pour les deux sexes.} 
\lhead{\firstmark}
\rhead{\botmark}

\subsection{\hspace{-0.5cm} {\Large \textcolor{darkblue}{\textbf{\ipa{ki˩\textsubscript{a}}}}}\hspace{0.5cm}[\kern2pt{\textcolor{darkblue}{\textbf{\ipa{ki˩˥}}}}\kern2pt]} \hypertarget{ki\string_Ba1}{}
\markboth{\textcolor{darkblue}{\textbf{\ipa{ki˩\textsubscript{a}}}}}{}
\textcolor{teal}{\zh{动词}} \hspace{4pt} \zh{声调类:} L\textsubscript{a}.
\zh{穿上(裤子、袜子、鞋子)。} \textcolor{Sepia}{\selectlanguage{english}To put on (a skirt, trousers...).} \textcolor{PineGreen}{\selectlanguage{french}Enfiler, porter, mettre (une robe, un pantalon...).}  ¶ \textcolor{darkblue}{\textbf{\ipa{ɬi˧qʰwɤ˩ | ɖɯ˧-ɭɯ˧ ki˩}}} \zh{穿上裤子} \textcolor{Sepia}{\selectlanguage{english}to put on trousers} \textcolor{PineGreen}{\selectlanguage{french}enfiler un pantalon}  
 ¶ \textcolor{darkblue}{\textbf{\ipa{dzɑ˩qʰwɤ˩˥ | ɖɯ˧-dzi˧ ki˩}}} \zh{穿上一双鞋} \textcolor{Sepia}{\selectlanguage{english}to put on a pair of shoes} \textcolor{PineGreen}{\selectlanguage{french}enfiler une paire de chaussures}  
 ¶ \textcolor{darkblue}{\textbf{\ipa{ʈʰæ˩ ki˩˥}}} \zh{“穿裙”:这是成年礼的名称(穿裙礼:女孩满13岁,即为成人)} \textcolor{Sepia}{\selectlanguage{english}'to put on a skirt'; this is the name of the ritual of entry into adulthood, after a girl has reached age 13} \textcolor{PineGreen}{\selectlanguage{french}porter la jupe; nom du rituel de passage à l'âge adulte (à treize ans révolus) pour les jeunes filles}  
 ¶ \textcolor{darkblue}{\textbf{\ipa{ɬi˧ ki˥}}} \zh{“穿裤”:这是成年礼的名称(穿裤礼:男孩满了13岁,即为成人)} \textcolor{Sepia}{\selectlanguage{english}'to put on trousers'; this is the name of the ritual of entry into adulthood, after a boy has reached age 13} \textcolor{PineGreen}{\selectlanguage{french}porter le pantalon; nom du rituel de passage à l'âge adulte (à treize ans révolus) pour les jeunes gens}  

\lhead{\firstmark}
\rhead{\botmark}

\subsection{\hspace{-0.5cm} {\Large \textcolor{darkblue}{\textbf{\ipa{ki˩mi˧}}}}\hspace{0.5cm}[\kern2pt{\textcolor{darkblue}{\textbf{\ipa{ki˩mi˥}}}}\kern2pt]} \hypertarget{ki\string_Bmi\string_M1}{}
\markboth{\textcolor{darkblue}{\textbf{\ipa{ki˩mi˧}}}}{}
\textcolor{teal}{\zh{名词}} \hspace{4pt} \zh{声调类:} LM.
\zh{绿头苍蝇。} \textcolor{Sepia}{\selectlanguage{english}A large fly with a green head; its larvas are particularly harmful.} \textcolor{PineGreen}{\selectlanguage{french}Grosse mouche, dont les larves sont redoutées.}  \zh{量词}: \textcolor{darkblue}{\textbf{\ipa{mi˩}}} 
\lhead{\firstmark}
\rhead{\botmark}

\subsection{\hspace{-0.5cm} {\Large \textcolor{darkblue}{\textbf{\ipa{ki˩tɑ\#˥}}}}\hspace{0.5cm}[\kern2pt{\textcolor{darkblue}{\textbf{\ipa{ki˩tɑ˥}}}}\kern2pt]} \hypertarget{ki\string_BtA\#\string_T1}{}
\markboth{\textcolor{darkblue}{\textbf{\ipa{ki˩tɑ\#˥}}}}{}
\textcolor{teal}{\zh{名词}} \hspace{4pt} \zh{声调类:} LM+\#H.
\zh{皮袋,来装家里的财物:金币、银币……这个皮袋,埋在房子里的一个保密地方,防贼。为了让它很结实,袋子有四、五层麻布内衬。可以保存很长时间。} \textcolor{Sepia}{\selectlanguage{english}Bag made of leather and linen, in which silver coins used to be kept, buried somewhere in the house to hide it from robbers.} \textcolor{PineGreen}{\selectlanguage{french}Sac de cuir et de lin, dans lequel on plaçait ce que possédait la maisonnée: or, argent… qu'on enterrait quelque part dans la maison, pour se prémunir contre les voleurs; les matières choisies, cuir et lin, se conservaient très longtemps; le sac avait quatre ou cinq épaisseurs de tissu, pour le rendre plus solide.}  ¶ \textcolor{darkblue}{\textbf{\ipa{æ˧-tse˥pʰæ˩ | ɖɯ˧-ki˩tɑ˩}}} \zh{一袋铜币} \textcolor{Sepia}{\selectlanguage{english}a bag of bronze coins} \textcolor{PineGreen}{\selectlanguage{french}un sac de pièces de cuivre}  

\lhead{\firstmark}
\rhead{\botmark}

\subsection{\hspace{-0.5cm} {\Large \textcolor{darkblue}{\textbf{\ipa{ki˩ti\#˥}}}}\hspace{0.5cm}[\kern2pt{\textcolor{darkblue}{\textbf{\ipa{ki˩ti˥}}}}\kern2pt]} \hypertarget{ki\string_Bti\#\string_T1}{}
\markboth{\textcolor{darkblue}{\textbf{\ipa{ki˩ti\#˥}}}}{}
\textcolor{teal}{\zh{名词}} \hspace{4pt} \zh{声调类:} LM+\#H.
\zh{皮腰带。} \textcolor{Sepia}{\selectlanguage{english}Leather belt.} \textcolor{PineGreen}{\selectlanguage{french}Ceinture en cuir.}  \zh{量词}: \textcolor{darkblue}{\textbf{\ipa{kʰɯ˩}}} 
\lhead{\firstmark}
\rhead{\botmark}

\subsection{\hspace{-0.5cm} {\Large \textcolor{darkblue}{\textbf{\ipa{ko˥}}}}\hspace{0.5cm}[\kern2pt{\textcolor{darkblue}{\textbf{\ipa{ko˥}}}}\kern2pt]} \hypertarget{ko\string_T1}{}
\markboth{\textcolor{darkblue}{\textbf{\ipa{ko˥}}}}{}
\textcolor{teal}{\zh{名词}} \hspace{4pt} \zh{声调类:} \#H.
\zh{小山。} \textcolor{Sepia}{\selectlanguage{english}Hill, small mountain.} \textcolor{PineGreen}{\selectlanguage{french}Colline, petite montagne.}  \zh{量词}: \textcolor{darkblue}{\textbf{\ipa{ɭɯ˧}}} 
\lhead{\firstmark}
\rhead{\botmark}

\subsection{\hspace{-0.5cm} {\Large \textcolor{darkblue}{\textbf{\ipa{ko˧\textsubscript{a}}}}}\hspace{0.5cm}[\kern2pt{\textcolor{darkblue}{\textbf{\ipa{ko˥}}}}\kern2pt]} \hypertarget{ko\string_Ma1}{}
\markboth{\textcolor{darkblue}{\textbf{\ipa{ko˧\textsubscript{a}}}}}{}
\textcolor{teal}{\zh{量词}} \hspace{4pt} \zh{声调类:} M\textsubscript{a}.
\zh{量词:小东西,例如烟(一只)。} \textcolor{Sepia}{\selectlanguage{english}Classifier for small objects, e.g. cigarettes.} \textcolor{PineGreen}{\selectlanguage{french}Classificateur des petits objets, tels que des cigarettes.} 
\lhead{\firstmark}
\rhead{\botmark}

\subsection{\hspace{-0.5cm} {\Large \textcolor{darkblue}{\textbf{\ipa{ko˧ɖæ\#˥}}}}\hspace{0.5cm}[\kern2pt{\textcolor{darkblue}{\textbf{\ipa{ko˧ɖæ˧}}}}\kern2pt]} \hypertarget{ko\string_Md`\{\#\string_T1}{}
\markboth{\textcolor{darkblue}{\textbf{\ipa{ko˧ɖæ\#˥}}}}{}
\textcolor{teal}{\zh{名词}} \hspace{4pt} \zh{声调类:} \#H.
\zh{佛像。} \textcolor{Sepia}{\selectlanguage{english}Sculpture of Buddha (Tibetan borrowing).} \textcolor{PineGreen}{\selectlanguage{french}Statue de bouddha.}  \zh{【借词】}\zh{藏语} sku-vdra (sku-'dra)
 ¶ \textcolor{darkblue}{\textbf{\ipa{ko˧ɖæ˧-zo˧}}} \zh{小佛像} \textcolor{Sepia}{\selectlanguage{english}small statue of Buddha} \textcolor{PineGreen}{\selectlanguage{french}statuette du Bouddha}  
 \zh{量词}: \textcolor{darkblue}{\textbf{\ipa{ɭɯ˧}}} 
\lhead{\firstmark}
\rhead{\botmark}

\subsection{\hspace{-0.5cm} {\Large \textcolor{darkblue}{\textbf{\ipa{ko˧li\#˥}}}}\hspace{0.5cm}[\kern2pt{\textcolor{darkblue}{\textbf{\ipa{ko˧li˧}}}}\kern2pt]} \hypertarget{ko\string_Mli\#\string_T1}{}
\markboth{\textcolor{darkblue}{\textbf{\ipa{ko˧li\#˥}}}}{}
\textcolor{teal}{\zh{名词}} \hspace{4pt} \zh{声调类:} \#H.
\zh{吹火筒,用来吹火的小管子。} \textcolor{Sepia}{\selectlanguage{english}Blow tube: tube to blow on a fire.} \textcolor{PineGreen}{\selectlanguage{french}Soufflet à bouche: un tube dans lequel on souffle pour attiser le feu.}  \zh{量词}: \textcolor{darkblue}{\textbf{\ipa{ɭɯ˧}}} 
\lhead{\firstmark}
\rhead{\botmark}

\subsection{\hspace{-0.5cm} {\Large \textcolor{darkblue}{\textbf{\ipa{ko˧no˧-ʁo\#˥}}}}\hspace{0.5cm}[\kern2pt{\textcolor{darkblue}{\textbf{\ipa{xxxx non-correspondance entre le nombre de morphèmes et le nombre de tons de morphèmes}}}}\kern2pt]} \hypertarget{ko\string_Mno\string_M-Ro\#\string_T1}{}
\markboth{\textcolor{darkblue}{\textbf{\ipa{ko˧no˧-ʁo\#˥}}}}{}
\textcolor{teal}{\zh{名词}} \hspace{4pt} \zh{声调类:} \#H.
\zh{山梁。} \textcolor{Sepia}{\selectlanguage{english}Mountain ridge; bridge in the mountains.} \textcolor{PineGreen}{\selectlanguage{french}Crête, ligne de faîte (en montagne).}  \zh{量词}: \textcolor{darkblue}{\textbf{\ipa{kʰwɤ˥}}} 
\lhead{\firstmark}
\rhead{\botmark}

\subsection{\hspace{-0.5cm} {\Large \textcolor{darkblue}{\textbf{\ipa{ko˧sɯ\#˥}}}}\hspace{0.5cm}[\kern2pt{\textcolor{darkblue}{\textbf{\ipa{ko˧sɯ˧}}}}\kern2pt]} \hypertarget{ko\string_MsM\#\string_T1}{}
\markboth{\textcolor{darkblue}{\textbf{\ipa{ko˧sɯ\#˥}}}}{}
\textcolor{teal}{\zh{名词}} \hspace{4pt} \zh{声调类:} \#H.
\zh{商店、小卖部(汉语借词:公司)。} \textcolor{Sepia}{\selectlanguage{english}Shop.} \textcolor{PineGreen}{\selectlanguage{french}Boutique.}  \zh{【借词】} \zh{公司}

\lhead{\firstmark}
\rhead{\botmark}

\subsection{\hspace{-0.5cm} {\Large \textcolor{darkblue}{\textbf{\ipa{ko˩\textsubscript{a}}}}}\hspace{0.5cm}[\kern2pt{\textcolor{darkblue}{\textbf{\ipa{ko˩˥}}}}\kern2pt]} \hypertarget{ko\string_Ba1}{}
\markboth{\textcolor{darkblue}{\textbf{\ipa{ko˩\textsubscript{a}}}}}{}
\textcolor{teal}{\zh{动词}} \hspace{4pt} \zh{声调类:} L\textsubscript{a}.
\zh{烤火取暖,晒太阳。} \textcolor{Sepia}{\selectlanguage{english}To warm oneself at a fire; to bask in the sun.} \textcolor{PineGreen}{\selectlanguage{french}Se chauffer au feu ou au soleil; prendre le soleil.}  ¶ \textcolor{darkblue}{\textbf{\ipa{mv̩˧ ko˥}}} \zh{烤火取暖} \textcolor{Sepia}{\selectlanguage{english}to warm oneself at a fire} \textcolor{PineGreen}{\selectlanguage{french}se chauffer au feu}  
 ¶ \textcolor{darkblue}{\textbf{\ipa{le˧-ko˩-ze˩}}} \zh{烤火了} \textcolor{Sepia}{\selectlanguage{english}\mytextsc{accomp} \string_ \mytextsc{pfv}} \textcolor{PineGreen}{\selectlanguage{french}\mytextsc{accomp} \string_ \mytextsc{pfv}}  
 ¶ \textcolor{darkblue}{\textbf{\ipa{ɲi˧mi˧ ko˩}}} \zh{晒太阳} \textcolor{Sepia}{\selectlanguage{english}to bask in the sun} \textcolor{PineGreen}{\selectlanguage{french}se réchauffer au soleil}  
 ¶ \textcolor{darkblue}{\textbf{\ipa{ɲi˧mi˧ ɖɯ˧-ko˩-ɻ̍˩}}} \zh{晒晒太阳} \textcolor{Sepia}{\selectlanguage{english}to bask in the sun for a while} \textcolor{PineGreen}{\selectlanguage{french}se réchauffer un moment au soleil}  
 ¶ \textcolor{darkblue}{\textbf{\ipa{ɲi˧mi˧ ɖɯ˧-ko˧\textasciitilde{}ko˥-ɻ̍˩}}} \zh{晒晒太阳} \textcolor{Sepia}{\selectlanguage{english}to bask in the sun for a while} \textcolor{PineGreen}{\selectlanguage{french}se réchauffer un moment au soleil}  

\lhead{\firstmark}
\rhead{\botmark}

\subsection{\hspace{-0.5cm} {\Large \textcolor{darkblue}{\textbf{\ipa{ko˩dze˧}}}}\hspace{0.5cm}[\kern2pt{\textcolor{darkblue}{\textbf{\ipa{ko˩dze˥}}}}\kern2pt]} \hypertarget{ko\string_Bdze\string_M1}{}
\markboth{\textcolor{darkblue}{\textbf{\ipa{ko˩dze˧}}}}{}
\textcolor{teal}{\zh{名词}} \hspace{4pt} \zh{声调类:} LM.
\zh{一种鸽子。} \textcolor{Sepia}{\selectlanguage{english}A sort of dove.} \textcolor{PineGreen}{\selectlanguage{french}Sorte de colombe.}  \zh{量词}: \textcolor{darkblue}{\textbf{\ipa{mi˩}}} 
\lhead{\firstmark}
\rhead{\botmark}

\subsection{\hspace{-0.5cm} {\Large \textcolor{darkblue}{\textbf{\ipa{ko˩ɖʐo˩}}}}\hspace{0.5cm}[\kern2pt{\textcolor{darkblue}{\textbf{\ipa{ko˩ɖʐo˩˥}}}}\kern2pt]} \hypertarget{ko\string_Bd`z`o\string_B1}{}
\markboth{\textcolor{darkblue}{\textbf{\ipa{ko˩ɖʐo˩}}}}{}
\textcolor{teal}{\zh{名词}} \hspace{4pt} \zh{声调类:} L.
\zh{连枷。} \textcolor{Sepia}{\selectlanguage{english}Flail.} \textcolor{PineGreen}{\selectlanguage{french}Fléau pour battre le grain.}  \zh{量词}: \textcolor{darkblue}{\textbf{\ipa{nɑ˧}}} 
\lhead{\firstmark}
\rhead{\botmark}

\subsection{\hspace{-0.5cm} {\Large \textcolor{darkblue}{\textbf{\ipa{ko˩qʰɑ˧-dʑɯ\#˥}}}}\hspace{0.5cm}[\kern2pt{\textcolor{darkblue}{\textbf{\ipa{xxxx non-correspondance entre le nombre de morphèmes et le nombre de tons de morphèmes}}}}\kern2pt]} \hypertarget{ko\string_Bq\string_hA\string_M-dz£M\#\string_T1}{}
\markboth{\textcolor{darkblue}{\textbf{\ipa{ko˩qʰɑ˧-dʑɯ\#˥}}}}{}
\textcolor{teal}{\zh{名词}} \hspace{4pt} \zh{声调类:} LM+\#H.
\zh{金梅花。} \textcolor{Sepia}{\selectlanguage{english}Yyyy.} \textcolor{PineGreen}{\selectlanguage{french}Yyy.}  ¶ \textcolor{darkblue}{\textbf{\ipa{ko˩qʰɑ˧-dʑɯ˧-bæ˥bæ˩}}} \zh{金梅花的花} \textcolor{Sepia}{\selectlanguage{english}flower of...} \textcolor{PineGreen}{\selectlanguage{french}fleur de...}  

\lhead{\firstmark}
\rhead{\botmark}

\subsection{\hspace{-0.5cm} {\Large \textcolor{darkblue}{\textbf{\ipa{ko˧˥}}} \textsubscript{1}}\hspace{0.5cm}[\kern2pt{\textcolor{darkblue}{\textbf{\ipa{ko˧˥}}}}\kern2pt]} \hypertarget{ko\string_M\string_T1}{}
\markboth{\textcolor{darkblue}{\textbf{\ipa{ko˧˥}}} \textsubscript{1}}{}
\textcolor{teal}{\zh{助词}} \hspace{4pt} \zh{声调类:} MH.
\zh{过于,太(汉语借词)。} \textcolor{Sepia}{\selectlanguage{english}Too much, excessively.} \textcolor{PineGreen}{\selectlanguage{french}Trop, excessivement.}  \zh{【借词】} \zh{过}

\lhead{\firstmark}
\rhead{\botmark}

\subsection{\hspace{-0.5cm} {\Large \textcolor{darkblue}{\textbf{\ipa{ko˧˥}}} \textsubscript{2}}\hspace{0.5cm}[\kern2pt{\textcolor{darkblue}{\textbf{\ipa{ko˧˥}}}}\kern2pt]} \hypertarget{ko\string_M\string_T2}{}
\markboth{\textcolor{darkblue}{\textbf{\ipa{ko˧˥}}} \textsubscript{2}}{}
\textcolor{teal}{\zh{动词}} \hspace{4pt} \zh{声调类:} MH.
\zh{过(汉语借词)。} \textcolor{Sepia}{\selectlanguage{english}To happen, to take place, to pass, to go by (days, existence).} \textcolor{PineGreen}{\selectlanguage{french}Se passer, avoir lieu: les jours passent, la vie se passe; couler (des jours/des années).}  \zh{【借词】} \zh{过}
 ¶ \textcolor{darkblue}{\textbf{\ipa{se˧ʐɯ˩ ko˩}}} \zh{过生日} \textcolor{Sepia}{\selectlanguage{english}to celebrate a birthday} \textcolor{PineGreen}{\selectlanguage{french}fêter un anniversaire}  

\lhead{\firstmark}
\rhead{\botmark}

\subsection{\hspace{-0.5cm} {\Large \textcolor{darkblue}{\textbf{\ipa{kɯ˥}}}}\hspace{0.5cm}[\kern2pt{\textcolor{darkblue}{\textbf{\ipa{kɯ˥}}}}\kern2pt]} \hypertarget{kM\string_T1}{}
\markboth{\textcolor{darkblue}{\textbf{\ipa{kɯ˥}}}}{}
\textcolor{teal}{\zh{名词}} \hspace{4pt} \zh{声调类:} \#H.
\ding{202} \zh{胆。} \textcolor{Sepia}{\selectlanguage{english}Gallbladder.} \textcolor{PineGreen}{\selectlanguage{french}Vésicule biliaire.}  \zh{量词}: \textcolor{darkblue}{\textbf{\ipa{ɭɯ˧}}} \ding{203} \zh{胆汁。} \textcolor{Sepia}{\selectlanguage{english}Gall.} \textcolor{PineGreen}{\selectlanguage{french}Bile.} 
\lhead{\firstmark}
\rhead{\botmark}

\subsection{\hspace{-0.5cm} {\Large \textcolor{darkblue}{\textbf{\ipa{kɯ˥}}}}\hspace{0.5cm}[\kern2pt{\textcolor{darkblue}{\textbf{\ipa{kɯ˥}}}}\kern2pt]} \hypertarget{kM\string_T1}{}
\markboth{\textcolor{darkblue}{\textbf{\ipa{kɯ˥}}}}{}
\textcolor{teal}{\zh{形容词}} \hspace{4pt} \zh{声调类:} H.
\zh{紧。} \textcolor{Sepia}{\selectlanguage{english}Tight, tense.} \textcolor{PineGreen}{\selectlanguage{french}Serré, tendu.}  ¶ \textcolor{darkblue}{\textbf{\ipa{le˧-tsɯ˥ | le˧-kɯ˥-kʰɯ˩}}} \zh{绑紧} \textcolor{Sepia}{\selectlanguage{english}to attach tightly, to attach so that it will be quite tight} \textcolor{PineGreen}{\selectlanguage{french}attacher très serré}  
 ¶ \textcolor{darkblue}{\textbf{\ipa{le˧-kɯ˥-se˩}}} \zh{紧了} \textcolor{Sepia}{\selectlanguage{english}\mytextsc{accomp} \string_ \mytextsc{pfv}} \textcolor{PineGreen}{\selectlanguage{french}\mytextsc{accomp} \string_ \mytextsc{pfv}}  

\lhead{\firstmark}
\rhead{\botmark}

\subsection{\hspace{-0.5cm} {\Large \textcolor{darkblue}{\textbf{\ipa{kɯ˧}}}}\hspace{0.5cm}[\kern2pt{\textcolor{darkblue}{\textbf{\ipa{kɯ˥}}}}\kern2pt]} \hypertarget{kM\string_M1}{}
\markboth{\textcolor{darkblue}{\textbf{\ipa{kɯ˧}}}}{}
\textcolor{teal}{\zh{名词}} \hspace{4pt} \zh{声调类:} M.
\zh{星星。} \textcolor{Sepia}{\selectlanguage{english}Star.} \textcolor{PineGreen}{\selectlanguage{french}Étoile.}  ¶ \textcolor{darkblue}{\textbf{\ipa{mv̩˧ʁo˥ | kɯ˧}}} \zh{天上有星星、天上看得见星星} \textcolor{Sepia}{\selectlanguage{english}there are stars in the sky, one can see stars} \textcolor{PineGreen}{\selectlanguage{french}le ciel est étoilé, on voit les étoiles du ciel}  
 ¶ \textcolor{darkblue}{\textbf{\ipa{nɑ˩-ʈʂʰɯ˥, | kɯ˧ mɤ˧-li˧! | di˧mi˧-lɑ˧ li˥!}}} \zh{摩梭呢,不看星星,只看平坝(=永宁坝子)!(合作人说明为什么她不知道星星、星座的名字:摩梭人本来对天文不太感兴趣。)} \textcolor{Sepia}{\selectlanguage{english}The Na do not look at the stars! They only look at the plain (=at the plain of Yongning)! (A comment by the consultant about her lack of knowledge of the names of stars and constellations.)} \textcolor{PineGreen}{\selectlanguage{french}Les Na, ils ne regardent pas les étoiles! Ils ne regardent que la plaine (=leur plaine: la plaine de Yongning)! (commentaire de F4 au sujet de son ignorance des noms d'étoiles et de constellations)}  
 \zh{量词}: \textcolor{darkblue}{\textbf{\ipa{ɭɯ˧, kɯ˧}}} 
\lhead{\firstmark}
\rhead{\botmark}

\subsection{\hspace{-0.5cm} {\Large \textcolor{darkblue}{\textbf{\ipa{kɯ˧\textsubscript{b}}}}}\hspace{0.5cm}[\kern2pt{\textcolor{darkblue}{\textbf{\ipa{kɯ˩˥}}}}\kern2pt]} \hypertarget{kM\string_Mb1}{}
\markboth{\textcolor{darkblue}{\textbf{\ipa{kɯ˧\textsubscript{b}}}}}{}
\textcolor{teal}{\zh{量词}} \hspace{4pt} \zh{声调类:} M\textsubscript{b}.
\zh{量词:星星(一个)。} \textcolor{Sepia}{\selectlanguage{english}Self-classifier for stars.} \textcolor{PineGreen}{\selectlanguage{french}Auto-classificateur des étoiles; classificateur des jours propices.} 
\lhead{\firstmark}
\rhead{\botmark}

\subsection{\hspace{-0.5cm} {\Large \textcolor{darkblue}{\textbf{\ipa{kɯ˧ɭɯ˧}}}}\hspace{0.5cm}[\kern2pt{\textcolor{darkblue}{\textbf{\ipa{kɯ˧ɭɯ˧˥}}}}\kern2pt]} \hypertarget{kM\string_Ml\string_RM\string_M1}{}
\markboth{\textcolor{darkblue}{\textbf{\ipa{kɯ˧ɭɯ˧}}}}{}
\textcolor{teal}{\zh{名词}} \hspace{4pt} \zh{声调类:} M.
\zh{神。} \textcolor{Sepia}{\selectlanguage{english}Spirit.} \textcolor{PineGreen}{\selectlanguage{french}Esprit bienfaisant.}  ¶ \textcolor{darkblue}{\textbf{\ipa{kɯ˧ɭɯ˧ | ɖɯ˧-dze˩}}} \zh{两个(好)神} \textcolor{Sepia}{\selectlanguage{english}a pair of (good) spirits, two (good) spirits} \textcolor{PineGreen}{\selectlanguage{french}deux esprits bienfaisants}  
 \zh{量词}: \textcolor{darkblue}{\textbf{\ipa{dze˩}}} 
\lhead{\firstmark}
\rhead{\botmark}

\subsection{\hspace{-0.5cm} {\Large \textcolor{darkblue}{\textbf{\ipa{kɯ˧qʰæ˧ʂe˧˥}}}}\hspace{0.5cm}[\kern2pt{\textcolor{darkblue}{\textbf{\ipa{kɯ˧qʰæ˧ʂe˧˥}}}}\kern2pt]} \hypertarget{kM\string_Mq\string_h\{\string_Ms`e\string_M\string_T1}{}
\markboth{\textcolor{darkblue}{\textbf{\ipa{kɯ˧qʰæ˧ʂe˧˥}}}}{}
\textcolor{teal}{\zh{名词}} \hspace{4pt} \zh{声调类:} MH\#.
\zh{流星。} \textcolor{Sepia}{\selectlanguage{english}Comet.} \textcolor{PineGreen}{\selectlanguage{french}Comète (littéralement “les étoiles défèquent”).}  \zh{量词}: \textcolor{darkblue}{\textbf{\ipa{ʂɯ˩}}} 
\lhead{\firstmark}
\rhead{\botmark}

\subsection{\hspace{-0.5cm} {\Large \textcolor{darkblue}{\textbf{\ipa{kɯ˩\textsubscript{a}}}}}\hspace{0.5cm}[\kern2pt{\textcolor{darkblue}{\textbf{\ipa{kɯ˧˥}}}}\kern2pt]} \hypertarget{kM\string_Ba1}{}
\markboth{\textcolor{darkblue}{\textbf{\ipa{kɯ˩\textsubscript{a}}}}}{}
\textcolor{teal}{\zh{动词}} \hspace{4pt} \zh{声调类:} L\textsubscript{a}.
\zh{不理需要帮忙的人:知道一个人需要帮助,自己也有能力帮忙,但假装没看见、什么事没有。} \textcolor{Sepia}{\selectlanguage{english}To ignore someone who would need help, to leave someone alone with difficulties one could help with. This is a term for which no straightforward Chinese equivalent has been found; it refers to a situation where lack of real attachment to someone shows up in the lack of impulse to go out of one's way and help them.} \textcolor{PineGreen}{\selectlanguage{french}Laisser quelqu'un en rade, laisser quelqu'un seul au moment où il aurait besoin d'aide, faire mine d'ignorer quelqu'un qui aurait besoin d'aide, négliger d'assister quelqu'un. Il n'a pas été trouvé d'équivalent chinois simple pour ce terme, qui renvoie à la situation où un manque de réel sympathie pour quelqu'un se traduit par le fait qu'on n'est pas tenté de faire l'effort de l'aider lorsqu'il en aurait besoin: on fait alors comme si de rien n'était, comme si on n'était pas au courant de la situation de cette personne.}  ¶ \textcolor{darkblue}{\textbf{\ipa{hĩ˧ kɯ˥}}} \zh{同上} \textcolor{Sepia}{\selectlanguage{english}same meaning} \textcolor{PineGreen}{\selectlanguage{french}même sens}  
 ¶ \textcolor{darkblue}{\textbf{\ipa{hĩ˧-ɳɯ˩ | kɯ˩-kv̩˥!}}} \zh{人家在你需要帮忙的时候就会不理你的!(如果处不好关系,人家对你没有什么好感,到时候你需要帮忙他们就不理你了。)} \textcolor{Sepia}{\selectlanguage{english}People will sometimes ignore you when you are in need! / (You will realize that, in cases where you need help) people will sometimes ignore you and leave you alone with your difficulties!} \textcolor{PineGreen}{\selectlanguage{french}Il arrive que les gens te laissent tomber! / Il arrive que les gens ne t'apportent pas leur aide quand tu en aurais besoin!}  
 ¶ \textcolor{darkblue}{\textbf{\ipa{kɯ˩-mɤ˩-kv̩˥!}}} \zh{人家在你需要帮忙的时候就会不理你的!} \textcolor{Sepia}{\selectlanguage{english}(They) are not going to help you! / You're not going to get any help (from them)!} \textcolor{PineGreen}{\selectlanguage{french}(ils/elles) n(e t)'aideront pas!}  
 ¶ \textcolor{darkblue}{\textbf{\ipa{hĩ˧-ɳɯ˩ | kɯ˩-tʰɑ˩-kʰɯ˥!}}} \zh{别让人家(在你需要帮忙的时候)不理你!} \textcolor{Sepia}{\selectlanguage{english}Don't (behave in such a way as to) let people ignore you when you are in need! (Explanation: one should build trust for oneself, making others feel real trust and gratitude, so that they will help as a matter of course when the need for it arises; otherwise they will ignore us when we are in need of help.)} \textcolor{PineGreen}{\selectlanguage{french}Fais en sorte que les gens ne te laissent pas en rade (quand tu auras besoin d'aide)! (Explication: il faut faire en sorte de gagner une estime et une sympathie réelles de la part des gens que l'on connaît; de la sorte, ils vous aideront spontanément lorsque vous en aurez besoin. Sinon, leur manque de réelle sympathie se traduira par le fait qu'ils ne feront pas l'effort de donner un coup de main lorsqu'on en aura besoin: ils feront alors comme si de rien n'était, comme s'ils n'étaient pas conscients de notre besoin.)}  
 ¶ \textcolor{darkblue}{\textbf{\ipa{njɤ˧ | no˩ kɯ˩-hĩ˥ mɤ˩-ɲi˩! | njɤ˧ | no˧-ki˧ | dʑɤ˩-so˥-ɲi˩!}}} \zh{我不是不重视你!(刚好相反:)我是用心教你的 / 我努力教你最好的!(情景:一名学生认为老师忽视他,老师发现学生不高兴,就说明。)} \textcolor{Sepia}{\selectlanguage{english}I am not neglecting you at all! (On the contrary) I am teaching you good things / I am doing my best to teach you! (Context: a student considers himself neglected by a teacher; the teacher realizes that the student is dissatisfied, and provides a clarification.)} \textcolor{PineGreen}{\selectlanguage{french}Je ne te néglige pas du tout! (Bien au contraire:) je t'enseigne bien / je fais de mon mieux pour t'apprendre des choses! (Contexte imaginé: un étudiant s'estime négligé par un enseignant; celui-ci se rend compte du mécontentement de l'étudiant, et lui dit qu'il interprète mal.)}  

\lhead{\firstmark}
\rhead{\botmark}

\subsection{\hspace{-0.5cm} {\Large \textcolor{darkblue}{\textbf{\ipa{kɯ˩ɻ̍˧}}}}\hspace{0.5cm}[\kern2pt{\textcolor{darkblue}{\textbf{\ipa{kɯ˩ɻ̍˥}}}}\kern2pt]} \hypertarget{kM\string_Br£`̍\string_M1}{}
\markboth{\textcolor{darkblue}{\textbf{\ipa{kɯ˩ɻ̍˧}}}}{}
\textcolor{teal}{\zh{名词}} \hspace{4pt} \zh{声调类:} LM.
\zh{胡琴,二胡。} \textcolor{Sepia}{\selectlanguage{english}Two-string violin, erhu.} \textcolor{PineGreen}{\selectlanguage{french}Violon à deux cordes, erhu.}  ¶ \textcolor{darkblue}{\textbf{\ipa{kɯ˩ɻ̍˧ ʈɤ˧}}} \zh{拉二胡} \textcolor{Sepia}{\selectlanguage{english}to play erhu} \textcolor{PineGreen}{\selectlanguage{french}jouer du erhu}  
 \zh{量词}: \textcolor{darkblue}{\textbf{\ipa{nɑ˧}}} 
\lhead{\firstmark}
\rhead{\botmark}

\subsection{\hspace{-0.5cm} {\Large \textcolor{darkblue}{\textbf{\ipa{kv̩˧˥}}}}\hspace{0.5cm}[\kern2pt{\textcolor{darkblue}{\textbf{\ipa{kv̩˧˥}}}}\kern2pt]} \hypertarget{kv\string_=\string_M\string_T1}{}
\markboth{\textcolor{darkblue}{\textbf{\ipa{kv̩˧˥}}}}{}
\textcolor{teal}{\zh{动词}} \hspace{4pt} \zh{声调类:} MH.
\zh{会、有能力做。} \textcolor{Sepia}{\selectlanguage{english}To be able to.} \textcolor{PineGreen}{\selectlanguage{french}Pouvoir, être capable de, avoir la compétence pour (verbe de modalité épistémique).} 
\lhead{\firstmark}
\rhead{\botmark}

\subsection{\hspace{-0.5cm} {\Large \textcolor{darkblue}{\textbf{\ipa{kv̩˥}}}}\hspace{0.5cm}[\kern2pt{\textcolor{darkblue}{\textbf{\ipa{kv̩˥}}}}\kern2pt]} \hypertarget{kv\string_=\string_T1}{}
\markboth{\textcolor{darkblue}{\textbf{\ipa{kv̩˥}}}}{}
\textcolor{teal}{\zh{名词}} \hspace{4pt} \zh{声调类:} \#H.
\zh{大蒜。} \textcolor{Sepia}{\selectlanguage{english}Garlic, \textit{Allium sativum}.} \textcolor{PineGreen}{\selectlanguage{french}Ail, \textit{Allium sativum}.}  \zh{量词}: \textcolor{darkblue}{\textbf{\ipa{ɭɯ˧}}} \textcolor{darkblue}{\textbf{\ipa{tsʰɤ˧˥}}} 
\lhead{\firstmark}
\rhead{\botmark}

\subsection{\hspace{-0.5cm} {\Large \textcolor{darkblue}{\textbf{\ipa{kv̩˩\textsubscript{a}}}} \textsubscript{1}}\hspace{0.5cm}[\kern2pt{\textcolor{darkblue}{\textbf{\ipa{kv̩˥}}}}\kern2pt]} \hypertarget{kv\string_=\string_Ba1}{}
\markboth{\textcolor{darkblue}{\textbf{\ipa{kv̩˩\textsubscript{a}}}} \textsubscript{1}}{}
\textcolor{teal}{\zh{动词}} \hspace{4pt} \zh{声调类:} L\textsubscript{a}.
\ding{202} \zh{捡起来,拾。} \textcolor{Sepia}{\selectlanguage{english}To pick up (from the ground), to gather.} \textcolor{PineGreen}{\selectlanguage{french}Ramasser; cueillir (des baies, des choses qu'on se baisse pour cueillir).}  ¶ \textcolor{darkblue}{\textbf{\ipa{kv̩˧\textasciitilde{}kv̩˥}}} \zh{\mytextsc{重叠}} \textcolor{Sepia}{\selectlanguage{english}\mytextsc{red}} \textcolor{PineGreen}{\selectlanguage{french}\mytextsc{red}}  
 ¶ \textcolor{darkblue}{\textbf{\ipa{gɤ˩-kv̩˧\textasciitilde{}kv̩˥}}} \zh{捡起来(地上的东西)} \textcolor{Sepia}{\selectlanguage{english}to pick up (something that was on the ground)} \textcolor{PineGreen}{\selectlanguage{french}ramasser (quelque chose qui se trouvait à terre)}  
 ¶ \textcolor{darkblue}{\textbf{\ipa{le˧-kv̩˧\textasciitilde{}kv̩˥}}} \zh{捡起来(地上的东西)} \textcolor{Sepia}{\selectlanguage{english}to pick up (something that was on the ground)} \textcolor{PineGreen}{\selectlanguage{french}ramasser (quelque chose qui se trouvait à terre)}  
\ding{203} \zh{钓鱼。} \textcolor{Sepia}{\selectlanguage{english}To fish.} \textcolor{PineGreen}{\selectlanguage{french}Pêcher.}  ¶ \textcolor{darkblue}{\textbf{\ipa{ɲi˧zo˧ kv̩˥}}} \zh{钓鱼} \textcolor{Sepia}{\selectlanguage{english}to fish} \textcolor{PineGreen}{\selectlanguage{french}pêcher du poisson}  

\lhead{\firstmark}
\rhead{\botmark}

\subsection{\hspace{-0.5cm} {\Large \textcolor{darkblue}{\textbf{\ipa{kv̩˩\textsubscript{a}}}} \textsubscript{2}}\hspace{0.5cm}[\kern2pt{\textcolor{darkblue}{\textbf{\ipa{kv̩˩˥}}}}\kern2pt]} \hypertarget{kv\string_=\string_Ba2}{}
\markboth{\textcolor{darkblue}{\textbf{\ipa{kv̩˩\textsubscript{a}}}} \textsubscript{2}}{}
\textcolor{teal}{\zh{动词}} \hspace{4pt} \zh{声调类:} L\textsubscript{a}.
\zh{过。} \textcolor{Sepia}{\selectlanguage{english}To cross.} \textcolor{PineGreen}{\selectlanguage{french}Traverser.}  ¶ \textcolor{darkblue}{\textbf{\ipa{ʈʂʰwæ˩ kv̩˥}}} \zh{坐船过(河)} \textcolor{Sepia}{\selectlanguage{english}to cross (a river) in a boat} \textcolor{PineGreen}{\selectlanguage{french}traverser en bateau}  

\lhead{\firstmark}
\rhead{\botmark}

\subsection{\hspace{-0.5cm} {\Large \textcolor{darkblue}{\textbf{\ipa{kv̩˧dʑɯ˧˥}}}}\hspace{0.5cm}[\kern2pt{\textcolor{darkblue}{\textbf{\ipa{kv̩˧dʑɯ˧˥}}}}\kern2pt]} \hypertarget{kv\string_=\string_Mdz£M\string_M\string_T1}{}
\markboth{\textcolor{darkblue}{\textbf{\ipa{kv̩˧dʑɯ˧˥}}}}{}
\textcolor{teal}{\zh{名词}} \hspace{4pt} \zh{声调类:} MH.
\zh{帐篷。} \textcolor{Sepia}{\selectlanguage{english}Tent.} \textcolor{PineGreen}{\selectlanguage{french}Tente.}  ¶ \textcolor{darkblue}{\textbf{\ipa{kv̩˧dʑɯ˧ lɑ˥}}} \zh{安装帐篷、搭建帐篷} \textcolor{Sepia}{\selectlanguage{english}to put up a tent, to set up a tent} \textcolor{PineGreen}{\selectlanguage{french}déplier une tente, installer une tente}  
 \zh{量词}: \textcolor{darkblue}{\textbf{\ipa{nɑ˧}}} 
\lhead{\firstmark}
\rhead{\botmark}

\subsection{\hspace{-0.5cm} {\Large \textcolor{darkblue}{\textbf{\ipa{kv̩˧ʝi˥}}}}\hspace{0.5cm}[\kern2pt{\textcolor{darkblue}{\textbf{\ipa{kv̩˧ʝi˧˥}}}}\kern2pt]} \hypertarget{kv\string_=\string_Mj££i\string_T1}{}
\markboth{\textcolor{darkblue}{\textbf{\ipa{kv̩˧ʝi˥}}}}{}
\textcolor{teal}{\zh{助词}} \hspace{4pt} \zh{声调类:} H\#.
\zh{真的、的确、确实。} \textcolor{Sepia}{\selectlanguage{english}Truly, really, for good.} \textcolor{PineGreen}{\selectlanguage{french}Pour de vrai, réellement.} 
\lhead{\firstmark}
\rhead{\botmark}

\subsection{\hspace{-0.5cm} {\Large \textcolor{darkblue}{\textbf{\ipa{kv̩˧ʝi˥\$}}}}\hspace{0.5cm}[\kern2pt{\textcolor{darkblue}{\textbf{\ipa{kv̩˧ʝi˥}}}}\kern2pt]} \hypertarget{kv\string_=\string_Mj££i\string_T\$1}{}
\markboth{\textcolor{darkblue}{\textbf{\ipa{kv̩˧ʝi˥\$}}}}{}
\textcolor{teal}{\zh{名词}} \hspace{4pt} \zh{声调类:} H\$.
\zh{生命。} \textcolor{Sepia}{\selectlanguage{english}Life, existence, lifetime.} \textcolor{PineGreen}{\selectlanguage{french}Vie, existence.} 
\lhead{\firstmark}
\rhead{\botmark}

\subsection{\hspace{-0.5cm} {\Large \textcolor{darkblue}{\textbf{\ipa{kv̩˩kv̩˩}}}}\hspace{0.5cm}[\kern2pt{\textcolor{darkblue}{\textbf{\ipa{kv̩˩kv̩˩˥}}}}\kern2pt]} \hypertarget{kv\string_=\string_Bkv\string_=\string_B1}{}
\markboth{\textcolor{darkblue}{\textbf{\ipa{kv̩˩kv̩˩}}}}{}
\textcolor{teal}{\zh{名词}} \hspace{4pt} \zh{声调类:} L.
\zh{颧骨。} \textcolor{Sepia}{\selectlanguage{english}Cheekbone.} \textcolor{PineGreen}{\selectlanguage{french}Pommettes.}  \zh{量词}: \textcolor{darkblue}{\textbf{\ipa{ɭɯ˧}}} \zh{~【参考】~} \hyperlink{}{\textcolor{darkblue}{\textbf{\ipa{njɤ˧kv̩˩}}}} 
\lhead{\firstmark}
\rhead{\botmark}

\subsection{\hspace{-0.5cm} {\Large \textcolor{darkblue}{\textbf{\ipa{kv̩˧lv̩˧lv̩˥}}}}\hspace{0.5cm}[\kern2pt{\textcolor{darkblue}{\textbf{\ipa{kv̩˧lv̩˧lv̩˧}}}}\kern2pt]} \hypertarget{kv\string_=\string_Mlv\string_=\string_Mlv\string_=\string_T1}{}
\markboth{\textcolor{darkblue}{\textbf{\ipa{kv̩˧lv̩˧lv̩˥}}}}{}
\textcolor{teal}{\zh{名词}} \hspace{4pt} \zh{声调类:} H\#.
\zh{蒜瓣。} \textcolor{Sepia}{\selectlanguage{english}Garlic braid: garlic bulbs with long leaves, braided into a large clump.} \textcolor{PineGreen}{\selectlanguage{french}Tresse d'ail, ail tressé.}  \zh{量词}: \textcolor{darkblue}{\textbf{\ipa{ɭɯ˧}}} 
\lhead{\firstmark}
\rhead{\botmark}

\subsection{\hspace{-0.5cm} {\Large \textcolor{darkblue}{\textbf{\ipa{kv̩˩nɑ˧˥}}}}\hspace{0.5cm}[\kern2pt{\textcolor{darkblue}{\textbf{\ipa{kv̩˧nɑ˥}}}}\kern2pt]} \hypertarget{kv\string_=\string_BnA\string_M\string_T1}{}
\markboth{\textcolor{darkblue}{\textbf{\ipa{kv̩˩nɑ˧˥}}}}{}
\textcolor{teal}{\zh{名词}} \hspace{4pt} \zh{声调类:} LM+MH\#.
\zh{丝绸。} \textcolor{Sepia}{\selectlanguage{english}Silk.} \textcolor{PineGreen}{\selectlanguage{french}Soie.}  ¶ \textcolor{darkblue}{\textbf{\ipa{kv̩˩nɑ˧-bɑ˧lɑ˥}}} \zh{丝绸衣服} \textcolor{Sepia}{\selectlanguage{english}silk garment} \textcolor{PineGreen}{\selectlanguage{french}vêtement en soie}  
 \zh{量词}: \textcolor{darkblue}{\textbf{\ipa{tsʰi˥}}} 
\lhead{\firstmark}
\rhead{\botmark}

\subsection{\hspace{-0.5cm} {\Large \textcolor{darkblue}{\textbf{\ipa{kv̩˧ɲi˥}}}}\hspace{0.5cm}[\kern2pt{\textcolor{darkblue}{\textbf{\ipa{kv̩˧ɲi˥}}}}\kern2pt]} \hypertarget{kv\string_=\string_MJi\string_T1}{}
\markboth{\textcolor{darkblue}{\textbf{\ipa{kv̩˧ɲi˥}}}}{}
\textcolor{teal}{\zh{形容词}} \hspace{4pt} \zh{声调类:} H\#.
\zh{空手,空。} \textcolor{Sepia}{\selectlanguage{english}Empty.} \textcolor{PineGreen}{\selectlanguage{french}Vide, sans rien.}  ¶ \textcolor{darkblue}{\textbf{\ipa{bi˩ʁo˧ | kv̩˧ɲi˥-kʰɯ˩}}} \zh{(把一个人的)钱包弄空} \textcolor{Sepia}{\selectlanguage{english}to empty (someone's) purse, i.e. to take someone's money} \textcolor{PineGreen}{\selectlanguage{french}vider la bourse (de quelqu'un), c'est-à-dire lui prendre son argent)}  
 ¶ \textcolor{darkblue}{\textbf{\ipa{tɕʰɯ˩ di˩-hɯ˩˥, | mɤ˧-ɖɯ˧, | kv̩˧ɲi˥ | le˧-tsʰɯ˩!}}} \zh{他去狩猎,没得(任何猎物),空手回来!} \textcolor{Sepia}{\selectlanguage{english}He went to hunt the muntjac, but did not kill any, and came back empty-handed!} \textcolor{PineGreen}{\selectlanguage{french}Il est parti chasser le muntjac, il n'en a pas tué, et est revenu bredouille!}  

\lhead{\firstmark}
\rhead{\botmark}

\subsection{\hspace{-0.5cm} {\Large \textcolor{darkblue}{\textbf{\ipa{kv̩˧ʁo˧bv̩˥}}}}\hspace{0.5cm}[\kern2pt{\textcolor{darkblue}{\textbf{\ipa{kv̩˧ʁo˧bv̩˧˥}}}}\kern2pt]} \hypertarget{kv\string_=\string_MRo\string_Mbv\string_=\string_T1}{}
\markboth{\textcolor{darkblue}{\textbf{\ipa{kv̩˧ʁo˧bv̩˥}}}}{}
\textcolor{teal}{\zh{名词}} \hspace{4pt} \zh{声调类:} H\#.
\zh{蒜苗。} \textcolor{Sepia}{\selectlanguage{english}Garlic sprouts (consumed as a vegetable).} \textcolor{PineGreen}{\selectlanguage{french}Pousses d'ail (aliment).}  \zh{量词}: \textcolor{darkblue}{\textbf{\ipa{kɤ˧˥}}} 
\lhead{\firstmark}
\rhead{\botmark}

\subsection{\hspace{-0.5cm} {\Large \textcolor{darkblue}{\textbf{\ipa{kv̩˧ʂe˥\$}}}}\hspace{0.5cm}[\kern2pt{\textcolor{darkblue}{\textbf{\ipa{kv̩˧ʂe˧˥}}}}\kern2pt]} \hypertarget{kv\string_=\string_Ms`e\string_T\$1}{}
\markboth{\textcolor{darkblue}{\textbf{\ipa{kv̩˧ʂe˥\$}}}}{}
\textcolor{teal}{\zh{名词}} \hspace{4pt} \zh{声调类:} H\$.
\zh{跳蚤。} \textcolor{Sepia}{\selectlanguage{english}Flea.} \textcolor{PineGreen}{\selectlanguage{french}Puce.}  \zh{量词}: \textcolor{darkblue}{\textbf{\ipa{mi˩}}} 
\lhead{\firstmark}
\rhead{\botmark}

\subsection{\hspace{-0.5cm} {\Large \textcolor{darkblue}{\textbf{\ipa{kv̩˩tɑ˩}}}}\hspace{0.5cm}[\kern2pt{\textcolor{darkblue}{\textbf{\ipa{kv̩˩tɑ˩˥}}}}\kern2pt]} \hypertarget{kv\string_=\string_BtA\string_B1}{}
\markboth{\textcolor{darkblue}{\textbf{\ipa{kv̩˩tɑ˩}}}}{}
\textcolor{teal}{\zh{动词}} \hspace{4pt} \zh{声调类:} L.
\zh{集中在一起(如:砍木材后,把木材堆在一起)。} \textcolor{Sepia}{\selectlanguage{english}To assemble, to group, to bring together (e.g. after felling trees, putting pieces of timber together).} \textcolor{PineGreen}{\selectlanguage{french}Regrouper, rassembler (ex.: des troncs, après leur abattage).} 
\lhead{\firstmark}
\rhead{\botmark}

\subsection{\hspace{-0.5cm} {\Large \textcolor{darkblue}{\textbf{\ipa{kv̩˧tsʰɑ˥\$}}}}\hspace{0.5cm}[\kern2pt{\textcolor{darkblue}{\textbf{\ipa{kv̩˧tsʰɑ˥}}}}\kern2pt]} \hypertarget{kv\string_=\string_Mts\string_hA\string_T\$1}{}
\markboth{\textcolor{darkblue}{\textbf{\ipa{kv̩˧tsʰɑ˥\$}}}}{}
\textcolor{teal}{\zh{名词}} \hspace{4pt} \zh{声调类:} H\$.
\zh{一个姓(木里土司,普米族,的姓)。} \textcolor{Sepia}{\selectlanguage{english}Family name (that of the Muli feudal lord, belonging to the Pumi/Prinmi ethnic group).} \textcolor{PineGreen}{\selectlanguage{french}Nom de clan/famille étendue. Ce nom était celui de la famille des seigneurs (pumi/prinmi) de Muli.}  ¶ \textcolor{darkblue}{\textbf{\ipa{kv̩˧tsʰɑ˧=ɻ̍˥\$}}} \zh{\textcolor{darkblue}{\textbf{\ipa{/kv̩˧tsʰɑ˥\$/}}}家族} \textcolor{Sepia}{\selectlanguage{english}the \textcolor{darkblue}{\textbf{\ipa{/kv̩˧tsʰɑ˥\$/}}} clan, the \textcolor{darkblue}{\textbf{\ipa{/kv̩˧tsʰɑ˥\$/}}} family} \textcolor{PineGreen}{\selectlanguage{french}le clan \textcolor{darkblue}{\textbf{\ipa{/kv̩˧tsʰɑ˥\$/}}}, la famille \textcolor{darkblue}{\textbf{\ipa{/kv̩˧tsʰɑ˥\$/}}}}  
 ¶ \textcolor{darkblue}{\textbf{\ipa{kv̩˧tsʰɑ˧=ɻ̍˧ pi˥-zo˩!}}} \zh{人家把他们称作“\textcolor{darkblue}{\textbf{\ipa{/kv̩˧tsʰɑ˥\$/}}}家族”!} \textcolor{Sepia}{\selectlanguage{english}They were called “the \textcolor{darkblue}{\textbf{\ipa{/kv̩˧tsʰɑ˥\$/}}} family”!} \textcolor{PineGreen}{\selectlanguage{french}On les appelait “les \textcolor{darkblue}{\textbf{\ipa{/kv̩˧tsʰɑ˥\$/”!}}}}  

\lhead{\firstmark}
\rhead{\botmark}

\subsection{\hspace{-0.5cm} {\Large \textcolor{darkblue}{\textbf{\ipa{kv̩˧tsʰɤ˩}}}}\hspace{0.5cm}[\kern2pt{\textcolor{darkblue}{\textbf{\ipa{kv̩˧tsʰɤ˩}}}}\kern2pt]} \hypertarget{kv\string_=\string_Mts\string_h7\string_B1}{}
\markboth{\textcolor{darkblue}{\textbf{\ipa{kv̩˧tsʰɤ˩}}}}{}
\textcolor{teal}{\zh{名词}} \hspace{4pt} \zh{声调类:} L\#.
\zh{蒜头。} \textcolor{Sepia}{\selectlanguage{english}Head of garlic.} \textcolor{PineGreen}{\selectlanguage{french}Tête d'ail.}  \zh{量词}: \textcolor{darkblue}{\textbf{\ipa{tsʰɤ˧˥}}} 
\lhead{\firstmark}
\rhead{\botmark}

\subsection{\hspace{-0.5cm} {\Large \textcolor{darkblue}{\textbf{\ipa{kv̩˧ʈʂɯ˧˥}}}}\hspace{0.5cm}[\kern2pt{\textcolor{darkblue}{\textbf{\ipa{kv̩˧ʈʂɯ˧˥}}}}\kern2pt]} \hypertarget{kv\string_=\string_Mt`s`M\string_M\string_T1}{}
\markboth{\textcolor{darkblue}{\textbf{\ipa{kv̩˧ʈʂɯ˧˥}}}}{}
\textcolor{teal}{\zh{名词}} \hspace{4pt} \zh{声调类:} MH\#.
\zh{指甲。} \textcolor{Sepia}{\selectlanguage{english}(finger)nail, (toe)nail.} \textcolor{PineGreen}{\selectlanguage{french}Ongle.}  \zh{量词}: \textcolor{darkblue}{\textbf{\ipa{ɭɯ˧}}} 
\lhead{\firstmark}
\rhead{\botmark}

\subsection{\hspace{-0.5cm} {\Large \textcolor{darkblue}{\textbf{\ipa{‑kv̩˧˥}}}}\hspace{0.5cm}[\kern2pt{\textcolor{darkblue}{\textbf{\ipa{kv̩˧˥}}}}\kern2pt]} \hypertarget{‑kv\string_=\string_M\string_T1}{}
\markboth{\textcolor{darkblue}{\textbf{\ipa{‑kv̩˧˥}}}}{}
\textcolor{teal}{\zh{后缀}} \hspace{4pt} \zh{声调类:} MH.
\zh{能。} \textcolor{Sepia}{\selectlanguage{english}\mytextsc{abilitive;} also indicates future.} \textcolor{PineGreen}{\selectlanguage{french}\mytextsc{abilitive;} a aussi des emplois de futur.} 
\lhead{\firstmark}
\rhead{\botmark}

\subsection{\hspace{-0.5cm} {\Large \textcolor{darkblue}{\textbf{\ipa{kwɑ˧fæ˩}}}}\hspace{0.5cm}[\kern2pt{\textcolor{darkblue}{\textbf{\ipa{kwɑ˧fæ˩}}}}\kern2pt]} \hypertarget{kwA\string_Mf\{\string_B1}{}
\markboth{\textcolor{darkblue}{\textbf{\ipa{kwɑ˧fæ˩}}}}{}
\textcolor{teal}{\zh{名词}} \hspace{4pt} \zh{声调类:} L\#.
\zh{官房(汉语借词),酒店名称。} \textcolor{Sepia}{\selectlanguage{english}Name of a hotel.} \textcolor{PineGreen}{\selectlanguage{french}Nom d'un hôtel.}  \zh{【借词】} \zh{官房}
 ¶ \textcolor{darkblue}{\textbf{\ipa{kwɑ˧fæ˩}}} \zh{丽江官房大酒店的简称。注:发音合作人的女儿在丽江官房大酒店工作。} \textcolor{Sepia}{\selectlanguage{english}the abridged name of a five-star hotel where one of the main consultant's daughters works} \textcolor{PineGreen}{\selectlanguage{french}le nom abrégé d'un hôtel cinq étoiles où travaille l'une des filles de la consultante principale.}  

\lhead{\firstmark}
\rhead{\botmark}

\subsection{\hspace{-0.5cm} {\Large \textcolor{darkblue}{\textbf{\ipa{kwɑ˧tsʰɑ˧}}}}\hspace{0.5cm}[\kern2pt{\textcolor{darkblue}{\textbf{\ipa{kwɑ˧tsʰɑ˧}}}}\kern2pt]} \hypertarget{kwA\string_Mts\string_hA\string_M1}{}
\markboth{\textcolor{darkblue}{\textbf{\ipa{kwɑ˧tsʰɑ˧}}}}{}
\textcolor{teal}{\zh{名词}} \hspace{4pt} \zh{声调类:} M.
\zh{棺材(汉语借词)。} \textcolor{Sepia}{\selectlanguage{english}Coffin.} \textcolor{PineGreen}{\selectlanguage{french}Cercueil.}  \zh{【借词】} \zh{棺材}
 ¶ \textcolor{darkblue}{\textbf{\ipa{kwɑ˧tsʰɑ˧, | hĩ˧-mo˩-kʰɯ˩-di˩ ɲi˩!}}} \zh{棺材,是装尸体的! / 棺材,是用来装尸体的!} \textcolor{Sepia}{\selectlanguage{english}The coffin is the thing in which the corpse is put! / The coffin is the thing to put the corpse!} \textcolor{PineGreen}{\selectlanguage{french}Le cercueil, c'est là où on met le cadavre! / Le cercueil, c'est l'objet qui accueille le cadavre!}  
 \zh{量词}: \textcolor{darkblue}{\textbf{\ipa{ɭɯ˧}}} 
\lhead{\firstmark}
\rhead{\botmark}

\subsection{\hspace{-0.5cm} {\Large \textcolor{darkblue}{\textbf{\ipa{kwæ˧}}}}\hspace{0.5cm}[\kern2pt{\textcolor{darkblue}{\textbf{\ipa{kwæ˥}}}}\kern2pt]} \hypertarget{kw\{\string_M1}{}
\markboth{\textcolor{darkblue}{\textbf{\ipa{kwæ˧}}}}{}
\textcolor{teal}{\zh{动词}} \hspace{4pt} \zh{声调类:} M.
\zh{管(汉语借词)。} \textcolor{Sepia}{\selectlanguage{english}To take care of, to take charge of.} \textcolor{PineGreen}{\selectlanguage{french}S'occuper de, se charger de.}  \zh{【借词】} \zh{管}

\lhead{\firstmark}
\rhead{\botmark}

\subsection{\hspace{-0.5cm} {\Large \textcolor{darkblue}{\textbf{\ipa{kwæ˧fæ˥}}}}\hspace{0.5cm}[\kern2pt{\textcolor{darkblue}{\textbf{\ipa{kwæ˧fæ˥}}}}\kern2pt]} \hypertarget{kw\{\string_Mf\{\string_T1}{}
\markboth{\textcolor{darkblue}{\textbf{\ipa{kwæ˧fæ˥}}}}{}
\textcolor{teal}{\zh{名词}} \hspace{4pt} \zh{声调类:} H\#.
\zh{中等大小的梁。} \textcolor{Sepia}{\selectlanguage{english}Medium-sized beam.} \textcolor{PineGreen}{\selectlanguage{french}Poutre intermédiaire: pièce de charpente horizontale, posée sur une poutre maîtresse, et soutenant deux des poutres du toit, /ʐv̩˩ɭɯ˧/.}  \zh{量词}: \textcolor{darkblue}{\textbf{\ipa{pʰæ˧˥}}} 
\lhead{\firstmark}
\rhead{\botmark}

\subsection{\hspace{-0.5cm} {\Large \textcolor{darkblue}{\textbf{\ipa{kwæ˧pæ˥}}}}\hspace{0.5cm}[\kern2pt{\textcolor{darkblue}{\textbf{\ipa{kwæ˧pæ˥}}}}\kern2pt]} \hypertarget{kw\{\string_Mp\{\string_T1}{}
\markboth{\textcolor{darkblue}{\textbf{\ipa{kwæ˧pæ˥}}}}{}
\textcolor{teal}{\zh{名词}} \hspace{4pt} \zh{声调类:} H\#.
\zh{耙(可能是汉语借词。原来借来的词:刮板?? 刮耙??)。} \textcolor{Sepia}{\selectlanguage{english}Harrow (made of wood).} \textcolor{PineGreen}{\selectlanguage{french}Herse en bois (vraisemblablement un emprunt au chinois; terme emprunté: pas identifié avec certitude).}  \zh{量词}: \textcolor{darkblue}{\textbf{\ipa{nɑ˧}}} 
\lhead{\firstmark}
\rhead{\botmark}

\subsection{\hspace{-0.5cm} {\Large \textcolor{darkblue}{\textbf{\ipa{kwæ˧tsɯ˧}}}}\hspace{0.5cm}[\kern2pt{\textcolor{darkblue}{\textbf{\ipa{kwæ˧tsɯ˧}}}}\kern2pt]} \hypertarget{kw\{\string_MtsM\string_M1}{}
\markboth{\textcolor{darkblue}{\textbf{\ipa{kwæ˧tsɯ˧}}}}{}
\textcolor{teal}{\zh{名词}} \hspace{4pt} \zh{声调类:} M.
\zh{葵花瓜籽(汉语借词)。} \textcolor{Sepia}{\selectlanguage{english}Sunflower seed.} \textcolor{PineGreen}{\selectlanguage{french}Graine de tournesol.}  \zh{【借词】} \zh{瓜子}

\lhead{\firstmark}
\rhead{\botmark}

\subsection{\hspace{-0.5cm} {\Large \textcolor{darkblue}{\textbf{\ipa{‑kwɤ}}}}\hspace{0.5cm}[\kern2pt{\textcolor{darkblue}{\textbf{\ipa{xxxx groupe tonal entier sans aucun ton}}}}\kern2pt]} \hypertarget{‑kw71}{}
\markboth{\textcolor{darkblue}{\textbf{\ipa{‑kwɤ}}}}{}
\textcolor{teal}{\zh{连接词}} \hspace{4pt} \zh{声调类:} 0.
\zh{……的时候。} \textcolor{Sepia}{\selectlanguage{english}When.} \textcolor{PineGreen}{\selectlanguage{french}Lorsque; ne peut s'employer seul, mais apparaît dans la formule \textcolor{darkblue}{\textbf{\ipa{/kwɤ-tɕɯ-lɑ/}}}.} 
\lhead{\firstmark}
\rhead{\botmark}

\subsection{\hspace{-0.5cm} {\Large \textcolor{darkblue}{\textbf{\ipa{kwɤ˧ɭɯ˩}}}}\hspace{0.5cm}[\kern2pt{\textcolor{darkblue}{\textbf{\ipa{kwɤ˧ɭɯ˩}}}}\kern2pt]} \hypertarget{kw7\string_Ml\string_RM\string_B1}{}
\markboth{\textcolor{darkblue}{\textbf{\ipa{kwɤ˧ɭɯ˩}}}}{}
\textcolor{teal}{\zh{名词}} \hspace{4pt} \zh{声调类:} L\#.
\zh{坛子、罐子 (陶器),宝贝。} \textcolor{Sepia}{\selectlanguage{english}Jug; jar; pitcher; also: treasure, valuable possession.} \textcolor{PineGreen}{\selectlanguage{french}Jarre; trésor, objet de grande valeur.}  ¶ \textcolor{darkblue}{\textbf{\ipa{ʈʂʰɯ˧ | njɤ˧ kwɤ˧ɭɯ˩ ɲi˩!}}} \zh{他是我宝贝!(母亲说孩子是她的宝贝)} \textcolor{Sepia}{\selectlanguage{english}(S)he is my treasure! (About a child)} \textcolor{PineGreen}{\selectlanguage{french}C'est mon petit trésor! (dit au sujet d'un enfant)}  
 \zh{量词}: \textcolor{darkblue}{\textbf{\ipa{ɭɯ˧}}} 
\lhead{\firstmark}
\rhead{\botmark}

\subsection{\hspace{-0.5cm} {\Large \textcolor{darkblue}{\textbf{\ipa{kwɤ˧pɤ˧}}}}\hspace{0.5cm}[\kern2pt{\textcolor{darkblue}{\textbf{\ipa{kwɤ˧pɤ˧}}}}\kern2pt]} \hypertarget{kw7\string_Mp7\string_M1}{}
\markboth{\textcolor{darkblue}{\textbf{\ipa{kwɤ˧pɤ˧}}}}{}
\textcolor{teal}{\zh{名词}} \hspace{4pt} \zh{声调类:} M.
\zh{解释,教导、教诲。} \textcolor{Sepia}{\selectlanguage{english}Teaching, explanation.} \textcolor{PineGreen}{\selectlanguage{french}Explication, enseignement.}  ¶ \textcolor{darkblue}{\textbf{\ipa{kwɤ˧pɤ˧ ɖɯ˧-kʰwɤ˥ lɑ˩}}} \zh{解释一个道理、教一件事} \textcolor{Sepia}{\selectlanguage{english}to provide an explanation, to teach something} \textcolor{PineGreen}{\selectlanguage{french}enseigner quelque chose à quelqu'un, expliquer quelque chose à quelqu'un}  
 ¶ \textcolor{darkblue}{\textbf{\ipa{kwɤ˧pɤ˧ ɖɯ˧-kʰwɤ˥ | tʰi˧-lɑ˩-ɻ̍˩}}} \zh{同上:解释一个道理、教一件事} \textcolor{Sepia}{\selectlanguage{english}As above: to provide an explanation, to teach something} \textcolor{PineGreen}{\selectlanguage{french}comme ci-dessus: enseigner quelque chose à quelqu'un, expliquer quelque chose à quelqu'un}  
 ¶ \textcolor{darkblue}{\textbf{\ipa{[M23] kwɤ˧pɤ˧ lɑ˧˥}}} \zh{教、解释} \textcolor{Sepia}{\selectlanguage{english}to teach} \textcolor{PineGreen}{\selectlanguage{french}enseigner}  
 \zh{量词}: \textcolor{darkblue}{\textbf{\ipa{kʰwɤ˥}}} 
\lhead{\firstmark}
\rhead{\botmark}

\subsection{\hspace{-0.5cm} {\Large \textcolor{darkblue}{\textbf{\ipa{‑kwɤ˧tɕɯ˥}}}}\hspace{0.5cm}[\kern2pt{\textcolor{darkblue}{\textbf{\ipa{kwɤ˧tɕɯ˥}}}}\kern2pt]} \hypertarget{‑kw7\string_Mts£M\string_T1}{}
\markboth{\textcolor{darkblue}{\textbf{\ipa{‑kwɤ˧tɕɯ˥}}}}{}
\textcolor{teal}{\zh{连接词}} \hspace{4pt} \zh{声调类:} H\#.
\zh{因为,由于,既然。} \textcolor{Sepia}{\selectlanguage{english}After; because, since, as.} \textcolor{PineGreen}{\selectlanguage{french}Comme; après; puisque.}  ¶ \textcolor{darkblue}{\textbf{\ipa{-kwɤ˧tɕɯ˥-lɑ˩}}} \zh{同上} \textcolor{Sepia}{\selectlanguage{english}same meaning} \textcolor{PineGreen}{\selectlanguage{french}même sens}  
 ¶ \textcolor{darkblue}{\textbf{\ipa{ʈʂʰɯ˧ | go˩-kwɤ˩tɕɯ˥-lɑ˩, | hɑ˧ mɤ˧-dzɯ˥.}}} \zh{他病了,吃不下饭。} \textcolor{Sepia}{\selectlanguage{english}Because (s)he is ill, (s)he does not eat.} \textcolor{PineGreen}{\selectlanguage{french}Comme il est malade, il ne mange pas.}  
 ¶ \textcolor{darkblue}{\textbf{\ipa{[M18] ʈʂʰɯ˧ne˧-ʝi˥ | pi˧-kwɤ˩tɕɯ˩-lɑ˩, | wɤ˩˥ | lɑ˧hɑ˥ | ɖɯ˧-kʰwɤ˧ ʐwɤ˧˥.}}} \zh{他这样说完以后,又讲了些其它的。} \textcolor{Sepia}{\selectlanguage{english}After he said that, he went on to say something different / he changed his mind and said something quite different.} \textcolor{PineGreen}{\selectlanguage{french}ayant dit cela, il dit à nouveau autre chose (après avoir dit ça, il a dit ajouté autre chose!)}  
 ¶ \textcolor{darkblue}{\textbf{\ipa{[M18] ʈʂʰɯ˧ | tʰi˧-dzi˩-kwɤ˩-tɕɯ˩, | ɖɯ˧-kʰwɤ˧ ʐwɤ˧-ɻ̍˥: | “sɤ˧sɤ˧˥ | ʐwæ˧˥!”}}} \zh{他坐下后,说了这么一句:“真舒服!”} \textcolor{Sepia}{\selectlanguage{english}After he got seated, he said the following: “How comfortable!”} \textcolor{PineGreen}{\selectlanguage{french}après s’être assis/lorsqu’il fut assis, il dit une phrase: “quel confort!”}  

\lhead{\firstmark}
\rhead{\botmark}

\subsection{\hspace{-0.5cm} {\Large \textcolor{darkblue}{\textbf{\ipa{kwɤ˩\textsubscript{a}}}}}\hspace{0.5cm}[\kern2pt{\textcolor{darkblue}{\textbf{\ipa{kwɤ˩˥}}}}\kern2pt]} \hypertarget{kw7\string_Ba1}{}
\markboth{\textcolor{darkblue}{\textbf{\ipa{kwɤ˩\textsubscript{a}}}}}{}
\textcolor{teal}{\zh{量词}} \hspace{4pt} \zh{声调类:} L\textsubscript{a}.
\zh{量词:串。} \textcolor{Sepia}{\selectlanguage{english}A string, a cluster of.} \textcolor{PineGreen}{\selectlanguage{french}Classificateur des objets tressés, enfilés ou liés ensemble.}  ¶ \textcolor{darkblue}{\textbf{\ipa{kv̩˧ | ɖɯ˧-kwɤ˩}}} \zh{一辫大蒜} \textcolor{Sepia}{\selectlanguage{english}a braid of garlic} \textcolor{PineGreen}{\selectlanguage{french}une tresse d'aïl}  
 ¶ \textcolor{darkblue}{\textbf{\ipa{lɑ˧tsɯ˥ | ɖɯ˧-kwɤ˩}}} \zh{一辫辣椒} \textcolor{Sepia}{\selectlanguage{english}a braid of hot peppers} \textcolor{PineGreen}{\selectlanguage{french}une ligature de piments}  
 ¶ \textcolor{darkblue}{\textbf{\ipa{ʈʂʰɯ˧-kwɤ˥}}} \zh{\mytextsc{指示代词} \string_} \textcolor{Sepia}{\selectlanguage{english}\mytextsc{dem} \string_ (tone: H\# / H\$)} \textcolor{PineGreen}{\selectlanguage{french}\mytextsc{dem} \string_ (ton: H\# / H\$)}  

\lhead{\firstmark}
\rhead{\botmark}

\subsection{\hspace{-0.5cm} {\Large \textcolor{darkblue}{\textbf{\ipa{kwɤ˩\textsubscript{a}}}} \textsubscript{1}}\hspace{0.5cm}[\kern2pt{\textcolor{darkblue}{\textbf{\ipa{kwɤ˩˥}}}}\kern2pt]} \hypertarget{kw7\string_Ba1}{}
\markboth{\textcolor{darkblue}{\textbf{\ipa{kwɤ˩\textsubscript{a}}}} \textsubscript{1}}{}
\textcolor{teal}{\zh{动词}} \hspace{4pt} \zh{声调类:} L\textsubscript{a}.
\zh{扔掉。} \textcolor{Sepia}{\selectlanguage{english}To throw away (rubbish).} \textcolor{PineGreen}{\selectlanguage{french}Jeter.}  ¶ \textcolor{darkblue}{\textbf{\ipa{mv̩˩tɕo˧ kwɤ˩}}} \zh{扔掉(垃圾)} \textcolor{Sepia}{\selectlanguage{english}to throw away (rubbish)} \textcolor{PineGreen}{\selectlanguage{french}jeter (détritus); littéralement: “mettre en bas”}  
 ¶ \textcolor{darkblue}{\textbf{\ipa{tso˧\textasciitilde{}tso˧ kwɤ˥}}} \zh{扔东西} \textcolor{Sepia}{\selectlanguage{english}to throw stuff away} \textcolor{PineGreen}{\selectlanguage{french}jeter des choses}  

\lhead{\firstmark}
\rhead{\botmark}

\subsection{\hspace{-0.5cm} {\Large \textcolor{darkblue}{\textbf{\ipa{kwɤ˩\textsubscript{a}}}} \textsubscript{2}}\hspace{0.5cm}[\kern2pt{\textcolor{darkblue}{\textbf{\ipa{kwɤ˩˥}}}}\kern2pt]} \hypertarget{kw7\string_Ba2}{}
\markboth{\textcolor{darkblue}{\textbf{\ipa{kwɤ˩\textsubscript{a}}}} \textsubscript{2}}{}
\textcolor{teal}{\zh{动词}} \hspace{4pt} \zh{声调类:} L\textsubscript{a}.
\zh{管(汉语借词)。} \textcolor{Sepia}{\selectlanguage{english}To manage, to be in charge of, to take care of.} \textcolor{PineGreen}{\selectlanguage{french}S'occuper de, gérer, superviser.}  ¶ \textcolor{darkblue}{\textbf{\ipa{ɖɯ˧-kʰwɤ˧ kwɤ˥}}} \zh{管一些} \textcolor{Sepia}{\selectlanguage{english}to supervise a bit} \textcolor{PineGreen}{\selectlanguage{french}s'occuper un peu/s'occuper d'une partie (combinaison élicitée pour vérifier que le ton est L et non LM)}  

\lhead{\firstmark}
\rhead{\botmark}

\subsection{\hspace{-0.5cm} {\Large \textcolor{darkblue}{\textbf{\ipa{kwɤ˩-tjɤ˧ljɤ\#˥}}}}\hspace{0.5cm}[\kern2pt{\textcolor{darkblue}{\textbf{\ipa{kwɤ˧tjɤ˧ljɤ˧}}}}\kern2pt]} \hypertarget{kw7\string_B-tj7\string_Mlj7\#\string_T1}{}
\markboth{\textcolor{darkblue}{\textbf{\ipa{kwɤ˩-tjɤ˧ljɤ\#˥}}}}{}
\textcolor{teal}{\zh{名词}} \hspace{4pt} \zh{声调类:} L-\#H.
\zh{铃铛。} \textcolor{Sepia}{\selectlanguage{english}Small bell.} \textcolor{PineGreen}{\selectlanguage{french}Clochette s'accrochant autour du cou (ex.: clochette d'un cheval).}  \zh{量词}: \textcolor{darkblue}{\textbf{\ipa{ɭɯ˧}}} 
\lhead{\firstmark}
\rhead{\botmark}

\subsection{\hspace{-0.5cm} {\Large \textcolor{darkblue}{\textbf{\ipa{kwɤ˩to˥}}}}\hspace{0.5cm}[\kern2pt{\textcolor{darkblue}{\textbf{\ipa{kwɤ˩to˥}}}}\kern2pt]} \hypertarget{kw7\string_Bto\string_T1}{}
\markboth{\textcolor{darkblue}{\textbf{\ipa{kwɤ˩to˥}}}}{}
\textcolor{teal}{\zh{名词}} \hspace{4pt} \zh{声调类:} LH.
\zh{颌骨。} \textcolor{Sepia}{\selectlanguage{english}Jawbone, mandible, lower jaw.} \textcolor{PineGreen}{\selectlanguage{french}Mandibule, mâchoire inférieure.}  \zh{量词}: \textcolor{darkblue}{\textbf{\ipa{ɭɯ˧}}} 
\lhead{\firstmark}
\rhead{\botmark}

\newpage
\section*{\centering- \textcolor{darkblue}{\textbf{\ipa{kʰ}}} -}
\subsection{\hspace{-0.5cm} {\Large \textcolor{darkblue}{\textbf{\ipa{kʰɤ˧mi˥\$}}}}\hspace{0.5cm}[\kern2pt{\textcolor{darkblue}{\textbf{\ipa{xxxx ton non trouvé, à faire manuellement...}}}}\kern2pt]} \hypertarget{k\string_h7\string_Mmi\string_T\$1}{}
\markboth{\textcolor{darkblue}{\textbf{\ipa{kʰɤ˧mi˥\$}}}}{}
\textcolor{teal}{\zh{名词}} \hspace{4pt} \zh{声调类:} \$H.
\zh{大背篓。} \textcolor{Sepia}{\selectlanguage{english}Large basket carried on the back.} \textcolor{PineGreen}{\selectlanguage{french}Grande hotte.}  \zh{量词}: \textcolor{darkblue}{\textbf{\ipa{kʰɤ˧˥}}} 
\lhead{\firstmark}
\rhead{\botmark}

\subsection{\hspace{-0.5cm} {\Large \textcolor{darkblue}{\textbf{\ipa{kʰɤ˧ʂɯ˧}}}}\hspace{0.5cm}[\kern2pt{\textcolor{darkblue}{\textbf{\ipa{kʰɤ˧ʂɯ˧}}}}\kern2pt]} \hypertarget{k\string_h7\string_Ms`M\string_M1}{}
\markboth{\textcolor{darkblue}{\textbf{\ipa{kʰɤ˧ʂɯ˧}}}}{}
\textcolor{teal}{\zh{动词}} \hspace{4pt} \zh{声调类:} M.
\zh{开始(汉语借词)。} \textcolor{Sepia}{\selectlanguage{english}To begin.} \textcolor{PineGreen}{\selectlanguage{french}Commencer.}  \zh{【借词】} \zh{开始}

\lhead{\firstmark}
\rhead{\botmark}

\subsection{\hspace{-0.5cm} {\Large \textcolor{darkblue}{\textbf{\ipa{kʰɤ˧zo˥\$}}}}\hspace{0.5cm}[\kern2pt{\textcolor{darkblue}{\textbf{\ipa{xxxx ton non trouvé, à faire manuellement...}}}}\kern2pt]} \hypertarget{k\string_h7\string_Mzo\string_T\$1}{}
\markboth{\textcolor{darkblue}{\textbf{\ipa{kʰɤ˧zo˥\$}}}}{}
\textcolor{teal}{\zh{名词}} \hspace{4pt} \zh{声调类:} \$H.
\zh{小背篓。} \textcolor{Sepia}{\selectlanguage{english}Small basket carried on the back.} \textcolor{PineGreen}{\selectlanguage{french}Petite hotte.}  \zh{量词}: \textcolor{darkblue}{\textbf{\ipa{kʰɤ˧˥}}} 
\lhead{\firstmark}
\rhead{\botmark}

\subsection{\hspace{-0.5cm} {\Large \textcolor{darkblue}{\textbf{\ipa{kʰɤ˩njɤ˩\textasciitilde{}kʰɤ˧njɤ˧}}}}\hspace{0.5cm}[\kern2pt{\textcolor{darkblue}{\textbf{\ipa{xxxx non-correspondance entre le nombre de morphèmes et le nombre de tons de morphèmes}}}}\kern2pt]} \hypertarget{k\string_h7\string_Bnj7\string_B~k\string_h7\string_Mnj7\string_M1}{}
\markboth{\textcolor{darkblue}{\textbf{\ipa{kʰɤ˩njɤ˩\textasciitilde{}kʰɤ˧njɤ˧}}}}{}
\textcolor{teal}{\zh{形容词}} \hspace{4pt} \zh{声调类:} L-.
\zh{柔软(动作)。} \textcolor{Sepia}{\selectlanguage{english}Supple (movement).} \textcolor{PineGreen}{\selectlanguage{french}Souple (mouvement).} 
\lhead{\firstmark}
\rhead{\botmark}

\subsection{\hspace{-0.5cm} {\Large \textcolor{darkblue}{\textbf{\ipa{kʰɤ˧˥}}} \textsubscript{1}}\hspace{0.5cm}[\kern2pt{\textcolor{darkblue}{\textbf{\ipa{kʰɤ˧˥}}}}\kern2pt]} \hypertarget{k\string_h7\string_M\string_T1}{}
\markboth{\textcolor{darkblue}{\textbf{\ipa{kʰɤ˧˥}}} \textsubscript{1}}{}
\textcolor{teal}{\zh{动词}} \hspace{4pt} \zh{声调类:} MH.
\zh{灭(火)。} \textcolor{Sepia}{\selectlanguage{english}To put out (a fire).} \textcolor{PineGreen}{\selectlanguage{french}Éteindre le foyer.} \zh{~【参考】~} \hyperlink{}{\textcolor{darkblue}{\textbf{\ipa{hɑ̃˧˥}}} \textsubscript{1}} 
\lhead{\firstmark}
\rhead{\botmark}

\subsection{\hspace{-0.5cm} {\Large \textcolor{darkblue}{\textbf{\ipa{kʰɤ˧˥}}} \textsubscript{2}}\hspace{0.5cm}[\kern2pt{\textcolor{darkblue}{\textbf{\ipa{kʰɤ˧˥}}}}\kern2pt]} \hypertarget{k\string_h7\string_M\string_T2}{}
\markboth{\textcolor{darkblue}{\textbf{\ipa{kʰɤ˧˥}}} \textsubscript{2}}{}
\textcolor{teal}{\zh{名词}} \hspace{4pt} \zh{声调类:} MH.
\zh{背篓。} \textcolor{Sepia}{\selectlanguage{english}Basket carried on the back.} \textcolor{PineGreen}{\selectlanguage{french}Hotte.}  \zh{量词}: \textcolor{darkblue}{\textbf{\ipa{kʰɤ˧˥}}} 
\lhead{\firstmark}
\rhead{\botmark}

\subsection{\hspace{-0.5cm} {\Large \textcolor{darkblue}{\textbf{\ipa{kʰɤ˧˥\textsubscript{a}}}}}\hspace{0.5cm}[\kern2pt{\textcolor{darkblue}{\textbf{\ipa{kʰɤ˧˥}}}}\kern2pt]} \hypertarget{k\string_h7\string_M\string_Ta1}{}
\markboth{\textcolor{darkblue}{\textbf{\ipa{kʰɤ˧˥\textsubscript{a}}}}}{}
\textcolor{teal}{\zh{量词}} \hspace{4pt} \zh{声调类:} MH\textsubscript{a}.
\zh{量词:筐。} \textcolor{Sepia}{\selectlanguage{english}A basket of.} \textcolor{PineGreen}{\selectlanguage{french}Classificateur des cageots.} 
\lhead{\firstmark}
\rhead{\botmark}

\subsection{\hspace{-0.5cm} {\Large \textcolor{darkblue}{\textbf{\ipa{kʰi˥}}}}\hspace{0.5cm}[\kern2pt{\textcolor{darkblue}{\textbf{\ipa{kʰi˥}}}}\kern2pt]} \hypertarget{k\string_hi\string_T1}{}
\markboth{\textcolor{darkblue}{\textbf{\ipa{kʰi˥}}}}{}
\textcolor{teal}{\zh{动词}} \hspace{4pt} \zh{声调类:} H.
\zh{拆开、分离(几根线)。} \textcolor{Sepia}{\selectlanguage{english}To separate, to take apart (e.g. to separate fibers of linen to make thread).} \textcolor{PineGreen}{\selectlanguage{french}Séparer, défaire (des fibres de lin: on sépare les fibres pour faire du fil).}  ¶ \textcolor{darkblue}{\textbf{\ipa{sɑ˧ | le˧-kʰi˥}}} \zh{拆开粗麻(为了纺出细麻线)} \textcolor{Sepia}{\selectlanguage{english}to separate linen fibres (to make thread)} \textcolor{PineGreen}{\selectlanguage{french}défaire des fibres de lin (pour fabriquer du fil)}  

\lhead{\firstmark}
\rhead{\botmark}

\subsection{\hspace{-0.5cm} {\Large \textcolor{darkblue}{\textbf{\ipa{kʰi˥}}} \textsubscript{1}}\hspace{0.5cm}[\kern2pt{\textcolor{darkblue}{\textbf{\ipa{kʰi˥}}}}\kern2pt]} \hypertarget{k\string_hi\string_T1}{}
\markboth{\textcolor{darkblue}{\textbf{\ipa{kʰi˥}}} \textsubscript{1}}{}
\textcolor{teal}{\zh{名词}} \hspace{4pt} \zh{声调类:} \#H.
\zh{门。} \textcolor{Sepia}{\selectlanguage{english}Door.} \textcolor{PineGreen}{\selectlanguage{french}Porte.}  ¶ \textcolor{darkblue}{\textbf{\ipa{kʰi˧-zo\#˥}}} \zh{小门} \textcolor{Sepia}{\selectlanguage{english}small door} \textcolor{PineGreen}{\selectlanguage{french}petite porte}  
 \zh{量词}: \textcolor{darkblue}{\textbf{\ipa{̩pɤ˩}}} 
\lhead{\firstmark}
\rhead{\botmark}

\subsection{\hspace{-0.5cm} {\Large \textcolor{darkblue}{\textbf{\ipa{kʰi˥}}} \textsubscript{2}}\hspace{0.5cm}[\kern2pt{\textcolor{darkblue}{\textbf{\ipa{kʰi˥}}}}\kern2pt]} \hypertarget{k\string_hi\string_T2}{}
\markboth{\textcolor{darkblue}{\textbf{\ipa{kʰi˥}}} \textsubscript{2}}{}
\textcolor{teal}{\zh{名词}} \hspace{4pt} \zh{声调类:} \#H.
\zh{边(单音节)。} \textcolor{Sepia}{\selectlanguage{english}Edge (monosyllable).} \textcolor{PineGreen}{\selectlanguage{french}Bord (monosyllabe).} 
\lhead{\firstmark}
\rhead{\botmark}

\subsection{\hspace{-0.5cm} {\Large \textcolor{darkblue}{\textbf{\ipa{kʰi˧bɤ\#˥}}}}\hspace{0.5cm}[\kern2pt{\textcolor{darkblue}{\textbf{\ipa{kʰi˧bɤ˧}}}}\kern2pt]} \hypertarget{k\string_hi\string_Mb7\#\string_T1}{}
\markboth{\textcolor{darkblue}{\textbf{\ipa{kʰi˧bɤ\#˥}}}}{}
\textcolor{teal}{\zh{名词}} \hspace{4pt} \zh{声调类:} \#H.
\zh{门槛。} \textcolor{Sepia}{\selectlanguage{english}Threshold.} \textcolor{PineGreen}{\selectlanguage{french}Seuil.}  \zh{量词}: \textcolor{darkblue}{\textbf{\ipa{ɭɯ˧}}} 
\lhead{\firstmark}
\rhead{\botmark}

\subsection{\hspace{-0.5cm} {\Large \textcolor{darkblue}{\textbf{\ipa{-kʰi˧\textasciitilde{}kʰi˧}}}}\hspace{0.5cm}[\kern2pt{\textcolor{darkblue}{\textbf{\ipa{kʰi˧kʰi˧}}}}\kern2pt]} \hypertarget{-k\string_hi\string_M~k\string_hi\string_M1}{}
\markboth{\textcolor{darkblue}{\textbf{\ipa{-kʰi˧\textasciitilde{}kʰi˧}}}}{}
\textcolor{teal}{\zh{后置词}} \hspace{4pt} \zh{声调类:} \#H.
\zh{周围、左右、旁边。} \textcolor{Sepia}{\selectlanguage{english}Around, close to, near, nearby.} \textcolor{PineGreen}{\selectlanguage{french}Aux alentours de, au bord de, auprès de.}  ¶ \textcolor{darkblue}{\textbf{\ipa{ʑi˧qʰwɤ˧-kʰi˧\textasciitilde{}kʰi˧}}} \zh{房子周围} \textcolor{Sepia}{\selectlanguage{english}near the house, in the vicinity of the house} \textcolor{PineGreen}{\selectlanguage{french}aux alentours de la maison}  
 ¶ \textcolor{darkblue}{\textbf{\ipa{[F5] njɤ˧-bv̩˧ | kʰi˧\textasciitilde{}kʰi˧}}} \zh{我的周围} \textcolor{Sepia}{\selectlanguage{english}near me, around me} \textcolor{PineGreen}{\selectlanguage{french}à côté de moi, autour de moi}  
 ¶ \textcolor{darkblue}{\textbf{\ipa{[M21] ʐɤ˩mi˩-kʰi˩\textasciitilde{}kʰi˩ se˩˥}}} \zh{走在马路边} \textcolor{Sepia}{\selectlanguage{english}to walk on the roadside, to walk by the side of the road} \textcolor{PineGreen}{\selectlanguage{french}marcher au bord de la route}  

\lhead{\firstmark}
\rhead{\botmark}

\subsection{\hspace{-0.5cm} {\Large \textcolor{darkblue}{\textbf{\ipa{kʰi˧mi˧}}}}\hspace{0.5cm}[\kern2pt{\textcolor{darkblue}{\textbf{\ipa{kʰi˧mi˧}}}}\kern2pt]} \hypertarget{k\string_hi\string_Mmi\string_M1}{}
\markboth{\textcolor{darkblue}{\textbf{\ipa{kʰi˧mi˧}}}}{}
\textcolor{teal}{\zh{名词}} \hspace{4pt} \zh{声调类:} M.
\zh{大门。} \textcolor{Sepia}{\selectlanguage{english}Main entrance, main door.} \textcolor{PineGreen}{\selectlanguage{french}Grande porte (ex.: porte d'entrée d'une maison).}  \zh{量词}: \textcolor{darkblue}{\textbf{\ipa{pɤ˩}}} 
\lhead{\firstmark}
\rhead{\botmark}

\subsection{\hspace{-0.5cm} {\Large \textcolor{darkblue}{\textbf{\ipa{kʰi˧qʰv̩\#˥}}}}\hspace{0.5cm}[\kern2pt{\textcolor{darkblue}{\textbf{\ipa{kʰi˧qʰv̩˧}}}}\kern2pt]} \hypertarget{k\string_hi\string_Mq\string_hv\string_=\#\string_T1}{}
\markboth{\textcolor{darkblue}{\textbf{\ipa{kʰi˧qʰv̩\#˥}}}}{}
\textcolor{teal}{\zh{名词}} \hspace{4pt} \zh{声调类:} \#H.
\zh{门。} \textcolor{Sepia}{\selectlanguage{english}Door.} \textcolor{PineGreen}{\selectlanguage{french}Porte.}  ¶ \textcolor{darkblue}{\textbf{\ipa{kʰi˧qʰv˧ tʰv˧-ɲi˥}}} \zh{到达(家)门(情景:从远方回家,到达家门)} \textcolor{Sepia}{\selectlanguage{english}to reach the door, to get to the door (context: reaching the door of one's home, getting back home from a long trip)} \textcolor{PineGreen}{\selectlanguage{french}parvenir à la porte, atteindre la porte (contexte: retour d'un lointain périple)}  
 ¶ \textcolor{darkblue}{\textbf{\ipa{ɑ˩ʁo˧ kʰi˧qʰv˧ tʰv˧}}} \zh{到达家门} \textcolor{Sepia}{\selectlanguage{english}to reach the door of the house} \textcolor{PineGreen}{\selectlanguage{french}parvenir à la porte de la maison}  

\lhead{\firstmark}
\rhead{\botmark}

\subsection{\hspace{-0.5cm} {\Large \textcolor{darkblue}{\textbf{\ipa{kʰi˧-qʰwɤ˩}}}}\hspace{0.5cm}[\kern2pt{\textcolor{darkblue}{\textbf{\ipa{xxxx non-correspondance entre le nombre de morphèmes et le nombre de tons de morphèmes}}}}\kern2pt]} \hypertarget{k\string_hi\string_M-q\string_hw7\string_B1}{}
\markboth{\textcolor{darkblue}{\textbf{\ipa{kʰi˧-qʰwɤ˩}}}}{}
\textcolor{teal}{\zh{名词}} \hspace{4pt} \zh{声调类:} L\#.
\zh{门的合页。} \textcolor{Sepia}{\selectlanguage{english}Hinge.} \textcolor{PineGreen}{\selectlanguage{french}Gonds (d'une porte).}  \zh{量词}: \textcolor{darkblue}{\textbf{\ipa{ɭɯ˧}}} \zh{~【同义词】~} \hyperlink{}{\textcolor{darkblue}{\textbf{\ipa{kʰi˧-bv̩˧lv̩˩}}}}. 
\lhead{\firstmark}
\rhead{\botmark}

\subsection{\hspace{-0.5cm} {\Large \textcolor{darkblue}{\textbf{\ipa{kʰi˧tɕʰɯ˩-mo˩}}}}\hspace{0.5cm}[\kern2pt{\textcolor{darkblue}{\textbf{\ipa{kʰi˧tɕʰɯ˩mo˧}}}}\kern2pt]} \hypertarget{k\string_hi\string_Mts£\string_hM\string_B-mo\string_B1}{}
\markboth{\textcolor{darkblue}{\textbf{\ipa{kʰi˧tɕʰɯ˩-mo˩}}}}{}
\textcolor{teal}{\zh{名词}} \hspace{4pt} \zh{声调类:} L\#-.
\zh{一种有毒的菌子。} \textcolor{Sepia}{\selectlanguage{english}A poisonous mushroom.} \textcolor{PineGreen}{\selectlanguage{french}Un champignon vénéneux.}  ¶ \textcolor{darkblue}{\textbf{\ipa{ʈʂæ˧mo˧-kʰi˧tɕʰɯ˩-mo˩}}} \zh{同上} \textcolor{Sepia}{\selectlanguage{english}same meaning} \textcolor{PineGreen}{\selectlanguage{french}même sens}  
\zh{~【同义词】~} \hyperlink{}{\textcolor{darkblue}{\textbf{\ipa{ʈʂæ˧mo\#˥}}}}. 
\lhead{\firstmark}
\rhead{\botmark}

\subsection{\hspace{-0.5cm} {\Large \textcolor{darkblue}{\textbf{\ipa{kʰi˧˥}}}}\hspace{0.5cm}[\kern2pt{\textcolor{darkblue}{\textbf{\ipa{kʰi˧˥}}}}\kern2pt]} \hypertarget{k\string_hi\string_M\string_T1}{}
\markboth{\textcolor{darkblue}{\textbf{\ipa{kʰi˧˥}}}}{}
\textcolor{teal}{\zh{动词}} \hspace{4pt} \zh{声调类:} MH.
\zh{走(过去式)。} \textcolor{Sepia}{\selectlanguage{english}Past form of verb 'to leave'.} \textcolor{PineGreen}{\selectlanguage{french}Forme passée du verbe 'partir'.}  ¶ \textcolor{darkblue}{\textbf{\ipa{ʈʂʰɯ˧ | zo˩qo˧ kʰi˧?}}} \zh{他到哪里去了?} \textcolor{Sepia}{\selectlanguage{english}Where has (s)he gone to? / Where has (s)he left for?} \textcolor{PineGreen}{\selectlanguage{french}Elle/il est parti où?}  
 ¶ \textcolor{darkblue}{\textbf{\ipa{[M23] no˧ | tsʰi˧ɲi˧ | ɑ˩pʰo˩˥ | ə˩-kʰi˩˥?}}} \zh{你今天出去了吗?} \textcolor{Sepia}{\selectlanguage{english}Did you go outside today? / Did you take a stroll today?} \textcolor{PineGreen}{\selectlanguage{french}tu es allé faire un tour dehors, aujourd'hui?/tu es sorti, aujourd'hui? (Contexte: question posée par un consultant alors que je le raccompagne après une séance de travail vespérale)}  

\lhead{\firstmark}
\rhead{\botmark}

\subsection{\hspace{-0.5cm} {\Large \textcolor{darkblue}{\textbf{\ipa{kʰo˥}}}}\hspace{0.5cm}[\kern2pt{\textcolor{darkblue}{\textbf{\ipa{kʰo˥}}}}\kern2pt]} \hypertarget{k\string_ho\string_T1}{}
\markboth{\textcolor{darkblue}{\textbf{\ipa{kʰo˥}}}}{}
\textcolor{teal}{\zh{动词}} \hspace{4pt} \zh{声调类:} H.
\zh{铺(床……)、铺得满地(果子、工具……)。} \textcolor{Sepia}{\selectlanguage{english}To spread (e.g. to do a bed; to spread/scatter objects all over the floor).} \textcolor{PineGreen}{\selectlanguage{french}Étendre (un matelas), étaler (des fruits, des outils... partout par terre).}  ¶ \textcolor{darkblue}{\textbf{\ipa{kʰwæ˧ɻæ˧ kʰo˧}}} \zh{铺垫子} \textcolor{Sepia}{\selectlanguage{english}to spread a mat} \textcolor{PineGreen}{\selectlanguage{french}étendre une natte}  

\lhead{\firstmark}
\rhead{\botmark}

\subsection{\hspace{-0.5cm} {\Large \textcolor{darkblue}{\textbf{\ipa{kʰo˧bɤ˧}}}}\hspace{0.5cm}[\kern2pt{\textcolor{darkblue}{\textbf{\ipa{kʰo˧bɤ˧}}}}\kern2pt]} \hypertarget{k\string_ho\string_Mb7\string_M1}{}
\markboth{\textcolor{darkblue}{\textbf{\ipa{kʰo˧bɤ˧}}}}{}
\textcolor{teal}{\zh{名词}} \hspace{4pt} \zh{声调类:} M.
\zh{家(文言):母亲生活的空间:有家人,有火塘,有母亲在那里生活的那个空间。} \textcolor{Sepia}{\selectlanguage{english}Home (solemn, formal word).} \textcolor{PineGreen}{\selectlanguage{french}Foyer. Mot ancien, qui n'est utilisé que dans un registre soutenu; il désigne un espace de vie.}  ¶ \textcolor{darkblue}{\textbf{\ipa{dʑi˧kʰi˧ le˧-gwɤ˩ | qo˩ tɑ˧-ze˥, | njɤ˧-ɕi˩ ə˩mɑ˩ kʰo˩bɤ˩ dʑɤ˩. |}}} \textcolor{PineGreen}{\selectlanguage{french}xxxx traduction à affiner (chanson au sujet de la nostalgie du foyer)}  
 ¶ \textcolor{darkblue}{\textbf{\ipa{dʑi˧kʰi˧ le˧-gwɤ˩ | qo˩ tɑ˧-ze˥, | njɤ˧-ɕi˩ ə˩mɑ˩ kʰo˩bɤ˩-qo˩. |}}} \textcolor{PineGreen}{\selectlanguage{french}idem ci-dessus}  
 ¶ \textcolor{darkblue}{\textbf{\ipa{dʑi˧kʰi˧ le˧-gwɤ˩ qo˩ tɑ˩-ze˩, | njɤ˧-ɕi˩ ə˩mɑ˩ kʰo˩bɤ˩ dʑɤ˩. |}}} \textcolor{PineGreen}{\selectlanguage{french}idem}  
 \zh{量词}: \textcolor{darkblue}{\textbf{\ipa{ɭɯ˧}}} 
\lhead{\firstmark}
\rhead{\botmark}

\subsection{\hspace{-0.5cm} {\Large \textcolor{darkblue}{\textbf{\ipa{kʰo˧lo˧}}}}\hspace{0.5cm}[\kern2pt{\textcolor{darkblue}{\textbf{\ipa{kʰo˧lo˧}}}}\kern2pt]} \hypertarget{k\string_ho\string_Mlo\string_M1}{}
\markboth{\textcolor{darkblue}{\textbf{\ipa{kʰo˧lo˧}}}}{}
\textcolor{teal}{\zh{名词}} \hspace{4pt} \zh{声调类:} M.
\zh{转经筒。} \textcolor{Sepia}{\selectlanguage{english}Prayer wheel.} \textcolor{PineGreen}{\selectlanguage{french}Moulin à prière (aussi bien les très grands, fixés à des axes verticaux dans les monastères, que les plus petits, tenus à la main).} \zh{当地汉语方言:}\zh{祈祷轱辘。} \zh{量词}: \textcolor{darkblue}{\textbf{\ipa{ɭɯ˧}}} 
\lhead{\firstmark}
\rhead{\botmark}

\subsection{\hspace{-0.5cm} {\Large \textcolor{darkblue}{\textbf{\ipa{kʰɯ˧di˧˥}}}}\hspace{0.5cm}[\kern2pt{\textcolor{darkblue}{\textbf{\ipa{kʰɯ˩di˩˥}}}}\kern2pt]} \hypertarget{k\string_hM\string_Mdi\string_M\string_T1}{}
\markboth{\textcolor{darkblue}{\textbf{\ipa{kʰɯ˧di˧˥}}}}{}
\textcolor{teal}{\zh{名词}} \hspace{4pt} \zh{声调类:} MH\#.
\zh{容器。} \textcolor{Sepia}{\selectlanguage{english}Container (general term).} \textcolor{PineGreen}{\selectlanguage{french}Récipient (terme générique).}  \zh{量词}: \textcolor{darkblue}{\textbf{\ipa{ɭɯ˧}}} 
\lhead{\firstmark}
\rhead{\botmark}

\subsection{\hspace{-0.5cm} {\Large \textcolor{darkblue}{\textbf{\ipa{kʰɯ˧dv̩\#˥}}}}\hspace{0.5cm}[\kern2pt{\textcolor{darkblue}{\textbf{\ipa{kʰɯ˧dv̩˧˥}}}}\kern2pt]} \hypertarget{k\string_hM\string_Mdv\string_=\#\string_T1}{}
\markboth{\textcolor{darkblue}{\textbf{\ipa{kʰɯ˧dv̩\#˥}}}}{}
\textcolor{teal}{\zh{名词}} \hspace{4pt} \zh{声调类:} \#H.
\zh{跛。} \textcolor{Sepia}{\selectlanguage{english}Cripple, lame person.} \textcolor{PineGreen}{\selectlanguage{french}Boiteux.}  ¶ \textcolor{darkblue}{\textbf{\ipa{kʰɯ˧dv̩˧-hĩ˧}}} \zh{跛} \textcolor{Sepia}{\selectlanguage{english}cripple} \textcolor{PineGreen}{\selectlanguage{french}boiteux}  
 ¶ \textcolor{darkblue}{\textbf{\ipa{kʰɯ˧dv̩˧-tsʰo˧qʰwɤ˧}}} \zh{跛鬼} \textcolor{Sepia}{\selectlanguage{english}lame demon} \textcolor{PineGreen}{\selectlanguage{french}démon boiteux}  
 \zh{量词}: \textcolor{darkblue}{\textbf{\ipa{v̩˧}}} 
\lhead{\firstmark}
\rhead{\botmark}

\subsection{\hspace{-0.5cm} {\Large \textcolor{darkblue}{\textbf{\ipa{kʰɯ˧dʑɯ˧˥}}}}\hspace{0.5cm}[\kern2pt{\textcolor{darkblue}{\textbf{\ipa{kʰɯ˧dʑɯ˧}}}}\kern2pt]} \hypertarget{k\string_hM\string_Mdz£M\string_M\string_T1}{}
\markboth{\textcolor{darkblue}{\textbf{\ipa{kʰɯ˧dʑɯ˧˥}}}}{}
\textcolor{teal}{\zh{名词}} \hspace{4pt} \zh{声调类:} MH\#.
\zh{裹腿。} \textcolor{Sepia}{\selectlanguage{english}Leggings, puttee.} \textcolor{PineGreen}{\selectlanguage{french}Bande molletière.}  \zh{量词}: \textcolor{darkblue}{\textbf{\ipa{dzi˧}}} 
\lhead{\firstmark}
\rhead{\botmark}

\subsection{\hspace{-0.5cm} {\Large \textcolor{darkblue}{\textbf{\ipa{kʰɯ˧pi˧}}}}\hspace{0.5cm}[\kern2pt{\textcolor{darkblue}{\textbf{\ipa{kʰɯ˧pi˧}}}}\kern2pt]} \hypertarget{k\string_hM\string_Mpi\string_M1}{}
\markboth{\textcolor{darkblue}{\textbf{\ipa{kʰɯ˧pi˧}}}}{}
\textcolor{teal}{\zh{动词}} \hspace{4pt} \zh{声调类:} .
\zh{绊。} \textcolor{Sepia}{\selectlanguage{english}To stumble, to trip.} \textcolor{PineGreen}{\selectlanguage{french}Trébucher.}  ¶ \textcolor{darkblue}{\textbf{\ipa{njɤ˧ kʰɯ˧pi˧-ze˧!}}} \zh{我绊了一跤!} \textcolor{Sepia}{\selectlanguage{english}I have stumbled!} \textcolor{PineGreen}{\selectlanguage{french}j'ai trébuché!}  

\lhead{\firstmark}
\rhead{\botmark}

\subsection{\hspace{-0.5cm} {\Large \textcolor{darkblue}{\textbf{\ipa{kʰɯ˧pʰv̩˩}}}}\hspace{0.5cm}[\kern2pt{\textcolor{darkblue}{\textbf{\ipa{kʰɯ˧pʰv̩˩}}}}\kern2pt]} \hypertarget{k\string_hM\string_Mp\string_hv\string_=\string_B1}{}
\markboth{\textcolor{darkblue}{\textbf{\ipa{kʰɯ˧pʰv̩˩}}}}{}
\textcolor{teal}{\zh{名词}} \hspace{4pt} \zh{声调类:} L\#.
\textit{\zh{古语}} [\zh{古语}] \zh{汉族。} \textcolor{Sepia}{\selectlanguage{english}Chinese, Han.} \textcolor{PineGreen}{\selectlanguage{french}Chinois.}  \zh{量词}: \textcolor{darkblue}{\textbf{\ipa{v̩˧}}} 
\lhead{\firstmark}
\rhead{\botmark}

\subsection{\hspace{-0.5cm} {\Large \textcolor{darkblue}{\textbf{\ipa{kʰɯ˧tʰo˧˥}}}}\hspace{0.5cm}[\kern2pt{\textcolor{darkblue}{\textbf{\ipa{kʰɯ˧tʰo˥}}}}\kern2pt]} \hypertarget{k\string_hM\string_Mt\string_ho\string_M\string_T1}{}
\markboth{\textcolor{darkblue}{\textbf{\ipa{kʰɯ˧tʰo˧˥}}}}{}
\textcolor{teal}{\zh{名词}} \hspace{4pt} \zh{声调类:} H\#.
\zh{脚链。} \textcolor{Sepia}{\selectlanguage{english}Chains (to tie a criminal's feet), made of iron.} \textcolor{PineGreen}{\selectlanguage{french}Chaîne de fer, pour attacher les chevilles d'un criminel.}  ¶ \textcolor{darkblue}{\textbf{\ipa{kʰɯ˧tʰo˧ kʰɯ˥}}} \zh{戴上脚链(在一个人的脚上)} \textcolor{Sepia}{\selectlanguage{english}to put chains (on someone's feet)} \textcolor{PineGreen}{\selectlanguage{french}mettre les chaînes (aux pieds de quelqu'un)}  

\lhead{\firstmark}
\rhead{\botmark}

\subsection{\hspace{-0.5cm} {\Large \textcolor{darkblue}{\textbf{\ipa{kʰɯ˧tʰv̩\#˥}}}}\hspace{0.5cm}[\kern2pt{\textcolor{darkblue}{\textbf{\ipa{kʰɯ˧tʰv̩˧}}}}\kern2pt]} \hypertarget{k\string_hM\string_Mt\string_hv\string_=\#\string_T1}{}
\markboth{\textcolor{darkblue}{\textbf{\ipa{kʰɯ˧tʰv̩\#˥}}}}{}
\textcolor{teal}{\zh{名词}} \hspace{4pt} \zh{声调类:} \#H.
\zh{织布机的脚蹬子=踏板。} \textcolor{Sepia}{\selectlanguage{english}Pedal of the loom (to invert the vertical position of the threads).} \textcolor{PineGreen}{\selectlanguage{french}Pédale du métier à tisser (pour inverser la position verticale des fils de trame entre 2 passages du volant).}  \zh{量词}: \textcolor{darkblue}{\textbf{\ipa{dze˩}}} 
\lhead{\firstmark}
\rhead{\botmark}

\subsection{\hspace{-0.5cm} {\Large \textcolor{darkblue}{\textbf{\ipa{kʰɯ˧tsɯ˧bæ˥}}}}\hspace{0.5cm}[\kern2pt{\textcolor{darkblue}{\textbf{\ipa{kʰɯ˧tsɯ˧bæ˥}}}}\kern2pt]} \hypertarget{k\string_hM\string_MtsM\string_Mb\{\string_T1}{}
\markboth{\textcolor{darkblue}{\textbf{\ipa{kʰɯ˧tsɯ˧bæ˥}}}}{}
\textcolor{teal}{\zh{名词}} \hspace{4pt} \zh{声调类:} H\#.
\zh{绑腿布:用来绑裤腿的一块缠布,也有装饰功能(从藏族地区传过来的)。} \textcolor{Sepia}{\selectlanguage{english}Strip of fabric used to tie the bottom part of trousers, which were wide; in addition to this function, this piece of clothing was also decorative; it came from Tibetan regions.} \textcolor{PineGreen}{\selectlanguage{french}Bande de tissu large d'une dizaine de centimères, utilisée autrefois pour serrer le pantalon, qui était très ample; c'était un élément fonctionnel mais également décoratif, préparé en belle étoffe; il provenait des régions tibétaines.} 
\lhead{\firstmark}
\rhead{\botmark}

\subsection{\hspace{-0.5cm} {\Large \textcolor{darkblue}{\textbf{\ipa{kʰɯ˧tsʰɤ˧˥}}}}\hspace{0.5cm}[\kern2pt{\textcolor{darkblue}{\textbf{\ipa{kʰɯ˧tsʰɤ˧˥}}}}\kern2pt]} \hypertarget{k\string_hM\string_Mts\string_h7\string_M\string_T1}{}
\markboth{\textcolor{darkblue}{\textbf{\ipa{kʰɯ˧tsʰɤ˧˥}}}}{}
\textcolor{teal}{\zh{名词}} \hspace{4pt} \zh{声调类:} MH\#.
\zh{腿,脚。} \textcolor{Sepia}{\selectlanguage{english}Leg.} \textcolor{PineGreen}{\selectlanguage{french}Jambe.}  \zh{量词}: \textcolor{darkblue}{\textbf{\ipa{pʰo˧˥}}} 
\lhead{\firstmark}
\rhead{\botmark}

\subsection{\hspace{-0.5cm} {\Large \textcolor{darkblue}{\textbf{\ipa{kʰɯ˧ʈʂæ˧˥}}}}\hspace{0.5cm}[\kern2pt{\textcolor{darkblue}{\textbf{\ipa{kʰɯ˧ʈʂæ˧˥}}}}\kern2pt]} \hypertarget{k\string_hM\string_Mt`s`\{\string_M\string_T1}{}
\markboth{\textcolor{darkblue}{\textbf{\ipa{kʰɯ˧ʈʂæ˧˥}}}}{}
\textcolor{teal}{\zh{名词}} \hspace{4pt} \zh{声调类:} MH\#.
\zh{踝关节。} \textcolor{Sepia}{\selectlanguage{english}Ankle.} \textcolor{PineGreen}{\selectlanguage{french}Cheville.}  \zh{量词}: \textcolor{darkblue}{\textbf{\ipa{ʈʂæ˧˥}}} 
\lhead{\firstmark}
\rhead{\botmark}

\subsection{\hspace{-0.5cm} {\Large \textcolor{darkblue}{\textbf{\ipa{kʰɯ˧ʈʂɤ\#˥}}}}\hspace{0.5cm}[\kern2pt{\textcolor{darkblue}{\textbf{\ipa{kʰɯ˧ʈʂɤ˧}}}}\kern2pt]} \hypertarget{k\string_hM\string_Mt`s`7\#\string_T1}{}
\markboth{\textcolor{darkblue}{\textbf{\ipa{kʰɯ˧ʈʂɤ\#˥}}}}{}
\textcolor{teal}{\zh{名词}} \hspace{4pt} \zh{声调类:} \#H.
\zh{鸡爪。} \textcolor{Sepia}{\selectlanguage{english}Chicken feet.} \textcolor{PineGreen}{\selectlanguage{french}Griffes de poulet.}  ¶ \textcolor{darkblue}{\textbf{\ipa{kʰɯ˧ʈʂɤ˧ tʰv̩˧-ɭɯ\#˥}}} \zh{这只鸡爪} \textcolor{Sepia}{\selectlanguage{english}\mytextsc{n}+\mytextsc{dem}+\mytextsc{clf}} \textcolor{PineGreen}{\selectlanguage{french}\mytextsc{n}+\mytextsc{dem}+\mytextsc{clf}}  
 ¶ \textcolor{darkblue}{\textbf{\ipa{kʰɯ˧ʈʂɤ˧ tʰv̩˧-ʈv̩˥\#}}} \zh{这只鸡爪} \textcolor{Sepia}{\selectlanguage{english}\mytextsc{n}+\mytextsc{dem}+\mytextsc{clf}} \textcolor{PineGreen}{\selectlanguage{french}\mytextsc{n}+\mytextsc{dem}+\mytextsc{clf}}  
 \zh{量词}: \textcolor{darkblue}{\textbf{\ipa{ʈv̩˩ / ɭɯ˧}}} 
\lhead{\firstmark}
\rhead{\botmark}

\subsection{\hspace{-0.5cm} {\Large \textcolor{darkblue}{\textbf{\ipa{kʰɯ˧ʐɯ˥\$}}}}\hspace{0.5cm}[\kern2pt{\textcolor{darkblue}{\textbf{\ipa{kʰɯ˧ʐɯ˥}}}}\kern2pt]} \hypertarget{k\string_hM\string_Mz`M\string_T\$1}{}
\markboth{\textcolor{darkblue}{\textbf{\ipa{kʰɯ˧ʐɯ˥\$}}}}{}
\textcolor{teal}{\zh{名词}} \hspace{4pt} \zh{声调类:} H\$.
\zh{黄酒。} \textcolor{Sepia}{\selectlanguage{english}Rice wine (low alcohol).} \textcolor{PineGreen}{\selectlanguage{french}Vin de riz (faiblement alcoolisé).}  \zh{量词}: \textcolor{darkblue}{\textbf{\ipa{qʰwɤ˧˥}}} 
\lhead{\firstmark}
\rhead{\botmark}

\subsection{\hspace{-0.5cm} {\Large \textcolor{darkblue}{\textbf{\ipa{kʰɯ˩}}}}\hspace{0.5cm}[\kern2pt{\textcolor{darkblue}{\textbf{\ipa{kʰɯ˥}}}}\kern2pt]} \hypertarget{k\string_hM\string_B1}{}
\markboth{\textcolor{darkblue}{\textbf{\ipa{kʰɯ˩}}}}{}
\textcolor{teal}{\zh{名词}} \hspace{4pt} \zh{声调类:} L.
\zh{线。} \textcolor{Sepia}{\selectlanguage{english}Thread.} \textcolor{PineGreen}{\selectlanguage{french}Fil.}  \zh{量词}: \textcolor{darkblue}{\textbf{\ipa{kʰɯ˩}}} 
\lhead{\firstmark}
\rhead{\botmark}

\subsection{\hspace{-0.5cm} {\Large \textcolor{darkblue}{\textbf{\ipa{kʰɯ˩\textsubscript{b}}}}}\hspace{0.5cm}[\kern2pt{\textcolor{darkblue}{\textbf{\ipa{kʰɯ˥}}}}\kern2pt]} \hypertarget{k\string_hM\string_Bb1}{}
\markboth{\textcolor{darkblue}{\textbf{\ipa{kʰɯ˩\textsubscript{b}}}}}{}
\textcolor{teal}{\zh{量词}} \hspace{4pt} \zh{声调类:} L\textsubscript{b}.
\zh{量词:线(一根、一条)。} \textcolor{Sepia}{\selectlanguage{english}Classifier for threads.} \textcolor{PineGreen}{\selectlanguage{french}Brin (d'herbe, de fil, de ficelle…).}  ¶ \textcolor{darkblue}{\textbf{\ipa{kʰɯ˧ | ɖɯ˧-kʰɯ˩}}} \zh{一根线} \textcolor{Sepia}{\selectlanguage{english}a thread of string} \textcolor{PineGreen}{\selectlanguage{french}un brin de fil}  
 ¶ \textcolor{darkblue}{\textbf{\ipa{zɯ˧ | ɖɯ˧-kʰɯ˩}}} \zh{一根草} \textcolor{Sepia}{\selectlanguage{english}a blade of grass} \textcolor{PineGreen}{\selectlanguage{french}un brin d'herbe}  
 ¶ \textcolor{darkblue}{\textbf{\ipa{bæ˩ ɖɯ˥-kʰɯ˩}}} \zh{一条绳子} \textcolor{Sepia}{\selectlanguage{english}a thread of rope} \textcolor{PineGreen}{\selectlanguage{french}un brin de corde, un bout de corde}  
 ¶ \textcolor{darkblue}{\textbf{\ipa{kʰɯ˧ | ʈʂʰɯ˧-kʰɯ˧˥}}} \zh{这根线} \textcolor{Sepia}{\selectlanguage{english}this thread (note: irregular tone pattern)} \textcolor{PineGreen}{\selectlanguage{french}ce brin (note: schéma tonal irrégulier)}  

\lhead{\firstmark}
\rhead{\botmark}

\subsection{\hspace{-0.5cm} {\Large \textcolor{darkblue}{\textbf{\ipa{kʰɯ˩pv̩˩}}}}\hspace{0.5cm}[\kern2pt{\textcolor{darkblue}{\textbf{\ipa{kʰɯ˩pv̩˩˥}}}}\kern2pt]} \hypertarget{k\string_hM\string_Bpv\string_=\string_B1}{}
\markboth{\textcolor{darkblue}{\textbf{\ipa{kʰɯ˩pv̩˩}}}}{}
\textcolor{teal}{\zh{名词}} \hspace{4pt} \zh{声调类:} L.
\zh{梭,梭子。} \textcolor{Sepia}{\selectlanguage{english}Shuttle.} \textcolor{PineGreen}{\selectlanguage{french}Navette du métier à tisser; elle est actuellement confectionnée au plus simple, en prenant une tige de tournesol ou un bambou fin.}  \zh{量词}: \textcolor{darkblue}{\textbf{\ipa{ɭɯ˧}}} \zh{~【参考】~} \hyperlink{}{\textcolor{darkblue}{\textbf{\ipa{pv̩˧qʰwɤ˥}}}} 
\lhead{\firstmark}
\rhead{\botmark}

\subsection{\hspace{-0.5cm} {\Large \textcolor{darkblue}{\textbf{\ipa{kʰɯ˩ʈɯ˩}}}}\hspace{0.5cm}[\kern2pt{\textcolor{darkblue}{\textbf{\ipa{kʰɯ˩ʈɯ˩˥}}}}\kern2pt]} \hypertarget{k\string_hM\string_Bt`M\string_B1}{}
\markboth{\textcolor{darkblue}{\textbf{\ipa{kʰɯ˩ʈɯ˩}}}}{}
\textcolor{teal}{\zh{名词}} \hspace{4pt} \zh{声调类:} L.
\zh{根。} \textcolor{Sepia}{\selectlanguage{english}Root.} \textcolor{PineGreen}{\selectlanguage{french}Racine.}  ¶ \textcolor{darkblue}{\textbf{\ipa{si˧dzi˩-kʰɯ˩ʈɯ˩}}} \zh{树根} \textcolor{Sepia}{\selectlanguage{english}tree root} \textcolor{PineGreen}{\selectlanguage{french}racines d'arbre}  
 \zh{量词}: \textcolor{darkblue}{\textbf{\ipa{ʈv̩˩}}} 
\lhead{\firstmark}
\rhead{\botmark}

\subsection{\hspace{-0.5cm} {\Large \textcolor{darkblue}{\textbf{\ipa{kʰɯ˧˥}}} \textsubscript{1}}\hspace{0.5cm}[\kern2pt{\textcolor{darkblue}{\textbf{\ipa{kʰɯ˧˥}}}}\kern2pt]} \hypertarget{k\string_hM\string_M\string_T1}{}
\markboth{\textcolor{darkblue}{\textbf{\ipa{kʰɯ˧˥}}} \textsubscript{1}}{}
\textcolor{teal}{\zh{动词}} \hspace{4pt} \zh{声调类:} MH.
\ding{202} \zh{放,装(如:装进袋里),点种,收下。} \textcolor{Sepia}{\selectlanguage{english}To put into (e.g. to put into a bag); to dibble in seeds.} \textcolor{PineGreen}{\selectlanguage{french}Mettre, mettre dans (ex.: mettre de la farine dans une casserole); libérer, lâcher (ex.: un poulet qu'on tenait par les pattes); semer en enfonçant les graines; ranger, remettre à sa place.}  ¶ \textcolor{darkblue}{\textbf{\ipa{kʰɯ˩\textasciitilde{}kʰɯ˧˥}}} \zh{\mytextsc{red}} \textcolor{Sepia}{\selectlanguage{english}\mytextsc{red}} \textcolor{PineGreen}{\selectlanguage{french}\mytextsc{red}}  
 ¶ \textcolor{darkblue}{\textbf{\ipa{qwɤ˧-qo˧ | si˧ tʰi˧-kʰɯ˧˥}}} \zh{放木头在火中} \textcolor{Sepia}{\selectlanguage{english}to add wood into the fire} \textcolor{PineGreen}{\selectlanguage{french}mettre/ajouter du bois dans le feu}  
\ding{203} \zh{让,\mytextsc{使动。}} \textcolor{Sepia}{\selectlanguage{english}To allow; to let; to cause (causative value).} \textcolor{PineGreen}{\selectlanguage{french}Autoriser; a aussi valeur causative, mais ce causatif issu du verbe “mettre” paraît avoir un sens plus proche de “laisser”: par exemple “laisser sécher au soleil” plutôt que “faire sécher au soleil”.}  ¶ \textcolor{darkblue}{\textbf{\ipa{kʰv̩˩mi˩ zɯ˩\textasciitilde{}zɯ˩˥, | le˧-ɖæ˥-kʰɯ˩! | hĩ˧-zɯ˧\textasciitilde{}zɯ˥, | le˧-ʂæ˧-kʰɯ˥!}}} \zh{够的寿命,变短了/使得变短!(而)人的寿命,变长了/使得变长!(《狗和人交换寿命》故事的一个提要)} \textcolor{Sepia}{\selectlanguage{english}Dog's lifespan was made shorter, and man's lifespan was made longer! (A summary of the legend “How dog and man exchanged their lifespans”.)} \textcolor{PineGreen}{\selectlanguage{french}La vie des chiens s'en est trouvée écourtée, et celle des hommes allongée! (Résumé en quelques mots du récit “Le chien échange sa longévité avec l'homme”)}  
 ¶ \textcolor{darkblue}{\textbf{\ipa{hwæ˧ kʰɯ˧ ə˥-bi˩? | - hwæ˧ kʰɯ˧-bi˥!}}} \zh{(你)让买吗? - 让买!} \textcolor{Sepia}{\selectlanguage{english}Do you agree to buy? - Yes!} \textcolor{PineGreen}{\selectlanguage{french}Tu es d'accord pour acheter? - Oui!}  
 ¶ \textcolor{darkblue}{\textbf{\ipa{tɕʰi˧ kʰɯ˧ ə˥-bi˩?}}} \zh{(你)让卖吗?} \textcolor{Sepia}{\selectlanguage{english}Do you agree to sell?} \textcolor{PineGreen}{\selectlanguage{french}Tu es d'accord pour vendre?}  
 ¶ \textcolor{darkblue}{\textbf{\ipa{dzɯ˧ kʰɯ˩ ə˩-bi˩?}}} \zh{(你)让吃吗?} \textcolor{Sepia}{\selectlanguage{english}Do you agree to eat?} \textcolor{PineGreen}{\selectlanguage{french}Tu es d'accord pour manger?}  
 ¶ \textcolor{darkblue}{\textbf{\ipa{tɕi˩ kʰɯ˥ ə˩-bi˩?}}} \zh{(你)让写吗?} \textcolor{Sepia}{\selectlanguage{english}Do you agree to write?} \textcolor{PineGreen}{\selectlanguage{french}Tu es d'accord pour écrire?}  
 ¶ \textcolor{darkblue}{\textbf{\ipa{ʈʰɯ˩ kʰɯ˩ ə˥-bi˩?}}} \zh{(你)让喝吗?} \textcolor{Sepia}{\selectlanguage{english}Do you agree to drink?} \textcolor{PineGreen}{\selectlanguage{french}Tu es d'accord pour boire?}  
 ¶ \textcolor{darkblue}{\textbf{\ipa{ʐv̩˧ kʰɯ˥ ə˩-bi˩?}}} \zh{(你)让缝吗?} \textcolor{Sepia}{\selectlanguage{english}Do you agree to sew?} \textcolor{PineGreen}{\selectlanguage{french}Tu es d'accord pour coudre?}  

\lhead{\firstmark}
\rhead{\botmark}

\subsection{\hspace{-0.5cm} {\Large \textcolor{darkblue}{\textbf{\ipa{kʰɯ˧˥}}} \textsubscript{2}}\hspace{0.5cm}[\kern2pt{\textcolor{darkblue}{\textbf{\ipa{kʰɯ˧˥}}}}\kern2pt]} \hypertarget{k\string_hM\string_M\string_T2}{}
\markboth{\textcolor{darkblue}{\textbf{\ipa{kʰɯ˧˥}}} \textsubscript{2}}{}
\textcolor{teal}{\zh{动词}} \hspace{4pt} \zh{声调类:} MH.
\zh{甩、扔(石头)。} \textcolor{Sepia}{\selectlanguage{english}To throw.} \textcolor{PineGreen}{\selectlanguage{french}Lancer, jeter.}  ¶ \textcolor{darkblue}{\textbf{\ipa{le˧-kʰɯ˧-ze˥}}} \zh{甩了} \textcolor{Sepia}{\selectlanguage{english}\mytextsc{accomp} \string_ \mytextsc{pfv}} \textcolor{PineGreen}{\selectlanguage{french}\mytextsc{accomp} \string_ \mytextsc{pfv}}  
 ¶ \textcolor{darkblue}{\textbf{\ipa{lv̩˧mi˧ kʰɯ˧˥}}} \zh{扔石头} \textcolor{Sepia}{\selectlanguage{english}to throw a stone} \textcolor{PineGreen}{\selectlanguage{french}jeter une pierre}  

\lhead{\firstmark}
\rhead{\botmark}

\subsection{\hspace{-0.5cm} {\Large \textcolor{darkblue}{\textbf{\ipa{kʰɯ˧˥}}} \textsubscript{3}}\hspace{0.5cm}[\kern2pt{\textcolor{darkblue}{\textbf{\ipa{kʰɯ˧˥}}}}\kern2pt]} \hypertarget{k\string_hM\string_M\string_T3}{}
\markboth{\textcolor{darkblue}{\textbf{\ipa{kʰɯ˧˥}}} \textsubscript{3}}{}
\textcolor{teal}{\zh{动词}} \hspace{4pt} \zh{声调类:} MH.
\zh{戴(手镯)。} \textcolor{Sepia}{\selectlanguage{english}To wear (a bracelet).} \textcolor{PineGreen}{\selectlanguage{french}Porter (un bracelet).}  ¶ \textcolor{darkblue}{\textbf{\ipa{le˧-kʰɯ˧-ze˥}}} \zh{戴了} \textcolor{Sepia}{\selectlanguage{english}\mytextsc{accomp} \string_ \mytextsc{pfv}} \textcolor{PineGreen}{\selectlanguage{french}\mytextsc{accomp} \string_ \mytextsc{pfv}}  
 ¶ \textcolor{darkblue}{\textbf{\ipa{lo˩dʑo˧ kʰɯ˩}}} \zh{戴手镯} \textcolor{Sepia}{\selectlanguage{english}to wear a bracelet} \textcolor{PineGreen}{\selectlanguage{french}porter un bracelet}  

\lhead{\firstmark}
\rhead{\botmark}

\subsection{\hspace{-0.5cm} {\Large \textcolor{darkblue}{\textbf{\ipa{kʰv̩˧˥}}}}\hspace{0.5cm}[\kern2pt{\textcolor{darkblue}{\textbf{\ipa{kʰv̩˧˥}}}}\kern2pt]} \hypertarget{k\string_hv\string_=\string_M\string_T1}{}
\markboth{\textcolor{darkblue}{\textbf{\ipa{kʰv̩˧˥}}}}{}
\textcolor{teal}{\zh{名词}} \hspace{4pt} \zh{声调类:} MH.
\ding{202} \zh{年、岁。} \textcolor{Sepia}{\selectlanguage{english}Year; year of age.} \textcolor{PineGreen}{\selectlanguage{french}Année, an.}  ¶ \textcolor{darkblue}{\textbf{\ipa{kʰv̩˧-mæ˥}}} \zh{年尾} \textcolor{Sepia}{\selectlanguage{english}end of the year} \textcolor{PineGreen}{\selectlanguage{french}fin de l'année}  
 ¶ \textcolor{darkblue}{\textbf{\ipa{kʰv̩˧-mæ˥ ʂæ˩}}} \zh{闰年(有13个月)} \textcolor{Sepia}{\selectlanguage{english}intercalary year: a year with 13 months; this happens every 4 years or so} \textcolor{PineGreen}{\selectlanguage{french}année longue, de 13 mois; cela a lieu tous les 4 ans environ}  
 ¶ \textcolor{darkblue}{\textbf{\ipa{kʰv̩˧-mæ˥ ɖæ˩}}} \zh{正常的年份,普通年:一年十二个月} \textcolor{Sepia}{\selectlanguage{english}normal year, usual year: a year that has 12 months} \textcolor{PineGreen}{\selectlanguage{french}année normale, à douze mois}  
\ding{203} \zh{生肖。} \textcolor{Sepia}{\selectlanguage{english}Astrological sign.} \textcolor{PineGreen}{\selectlanguage{french}Signe astrologique.}  ¶ \textcolor{darkblue}{\textbf{\ipa{no˧ | ə˧tso˧ kʰv̩˧ ɲi˥?}}} \zh{你是属什么的?} \textcolor{Sepia}{\selectlanguage{english}What is your astrological sign?} \textcolor{PineGreen}{\selectlanguage{french}De quel signe es-tu?}  

\lhead{\firstmark}
\rhead{\botmark}

\subsection{\hspace{-0.5cm} {\Large \textcolor{darkblue}{\textbf{\ipa{kʰv̩˥}}} \textsubscript{1}}\hspace{0.5cm}[\kern2pt{\textcolor{darkblue}{\textbf{\ipa{kʰv̩˧˥}}}}\kern2pt]} \hypertarget{k\string_hv\string_=\string_T1}{}
\markboth{\textcolor{darkblue}{\textbf{\ipa{kʰv̩˥}}} \textsubscript{1}}{}
\textcolor{teal}{\zh{名词}} \hspace{4pt} \zh{声调类:} \#H.
\zh{(鸟)巢。} \textcolor{Sepia}{\selectlanguage{english}Nest (monosyllable).} \textcolor{PineGreen}{\selectlanguage{french}Nid (monosyllabe).}  ¶ \textcolor{darkblue}{\textbf{\ipa{kʰv̩˧ ʈʂʰɯ˧-ɭɯ\#˥}}} \zh{这只鸟巢} \textcolor{Sepia}{\selectlanguage{english}\mytextsc{n}+\mytextsc{dem}+\mytextsc{clf}} \textcolor{PineGreen}{\selectlanguage{french}\mytextsc{n}+\mytextsc{dem}+\mytextsc{clf}}  
 \zh{量词}: \textcolor{darkblue}{\textbf{\ipa{ɭɯ˧}}} 
\lhead{\firstmark}
\rhead{\botmark}

\subsection{\hspace{-0.5cm} {\Large \textcolor{darkblue}{\textbf{\ipa{kʰv̩˥}}} \textsubscript{2}}\hspace{0.5cm}[\kern2pt{\textcolor{darkblue}{\textbf{\ipa{kʰv̩˥}}}}\kern2pt]} \hypertarget{k\string_hv\string_=\string_T2}{}
\markboth{\textcolor{darkblue}{\textbf{\ipa{kʰv̩˥}}} \textsubscript{2}}{}
\textcolor{teal}{\zh{动词}} \hspace{4pt} \zh{声调类:} H.
\zh{割(草)。} \textcolor{Sepia}{\selectlanguage{english}To harvest grass, to cut grass.} \textcolor{PineGreen}{\selectlanguage{french}Couper (ex.: de l'herbe) pour récolter.}  ¶ \textcolor{darkblue}{\textbf{\ipa{le˧-kʰv̩˥-ze˩}}} \zh{割了} \textcolor{Sepia}{\selectlanguage{english}\mytextsc{accomp} \string_ \mytextsc{pfv}} \textcolor{PineGreen}{\selectlanguage{french}\mytextsc{accomp} \string_ \mytextsc{pfv}}  
 ¶ \textcolor{darkblue}{\textbf{\ipa{zɯ˧-kʰv̩˧}}} \zh{割草} \textcolor{Sepia}{\selectlanguage{english}to cut grass} \textcolor{PineGreen}{\selectlanguage{french}couper de l'herbe}  

\lhead{\firstmark}
\rhead{\botmark}

\subsection{\hspace{-0.5cm} {\Large \textcolor{darkblue}{\textbf{\ipa{kʰv̩˥}}} \textsubscript{3}}\hspace{0.5cm}[\kern2pt{\textcolor{darkblue}{\textbf{\ipa{kʰv̩˥}}}}\kern2pt]} \hypertarget{k\string_hv\string_=\string_T3}{}
\markboth{\textcolor{darkblue}{\textbf{\ipa{kʰv̩˥}}} \textsubscript{3}}{}
\textcolor{teal}{\zh{名词}} \hspace{4pt} \zh{声调类:} \#H.
\zh{狗。} \textcolor{Sepia}{\selectlanguage{english}Dog (monosyllable).} \textcolor{PineGreen}{\selectlanguage{french}Chien (monosyllabe).}  ¶ \textcolor{darkblue}{\textbf{\ipa{kʰv̩˧-ʂe˧ dzɯ˧}}} \zh{吃狗肉} \textcolor{Sepia}{\selectlanguage{english}to eat dog meat (a practice which is strongly antagonistic to Na culture, which considers dog as man's benefactor)} \textcolor{PineGreen}{\selectlanguage{french}manger de la viande de chien (pratique qui va droit à l'encontre de la culture na, dans laquelle le chien est considéré comme bienfaiteur de l'homme)}  
 ¶ \textcolor{darkblue}{\textbf{\ipa{kʰv̩˧-zɯ˧\textasciitilde{}zɯ˥}}} \zh{狗的生命(传说狗与人交换了生命)} \textcolor{Sepia}{\selectlanguage{english}dog's existence, dog's life (which dog exchanged with man, according to the legend)} \textcolor{PineGreen}{\selectlanguage{french}l'existence du chien, la vie du chien (qu'il a échangée avec l'homme, selon la légende)}  
 ¶ \textcolor{darkblue}{\textbf{\ipa{kʰv̩˧ tʰv̩˧-mi˥\#}}} \zh{那条狗} \textcolor{Sepia}{\selectlanguage{english}\mytextsc{n}+\mytextsc{dem}+\mytextsc{clf}} \textcolor{PineGreen}{\selectlanguage{french}\mytextsc{n}+\mytextsc{dem}+\mytextsc{clf}}  
 ¶ \textcolor{darkblue}{\textbf{\ipa{kʰv̩˧-gɤ˥ljɤ˩}}} \zh{流浪狗} \textcolor{Sepia}{\selectlanguage{english}roving dog} \textcolor{PineGreen}{\selectlanguage{french}chien errant}  
 \zh{量词}: \textcolor{darkblue}{\textbf{\ipa{mi˩}}} \textcolor{darkblue}{\textbf{\ipa{v̩˧}}} \textcolor{darkblue}{\textbf{\ipa{jɤ˧˥}}} 
\lhead{\firstmark}
\rhead{\botmark}

\subsection{\hspace{-0.5cm} {\Large \textcolor{darkblue}{\textbf{\ipa{kʰv̩˥}}} \textsubscript{4}}\hspace{0.5cm}[\kern2pt{\textcolor{darkblue}{\textbf{\ipa{kʰv̩˥}}}}\kern2pt]} \hypertarget{k\string_hv\string_=\string_T4}{}
\markboth{\textcolor{darkblue}{\textbf{\ipa{kʰv̩˥}}} \textsubscript{4}}{}
\textcolor{teal}{\zh{动词}} \hspace{4pt} \zh{声调类:} H.
\zh{偷。} \textcolor{Sepia}{\selectlanguage{english}To steal.} \textcolor{PineGreen}{\selectlanguage{french}Voler.}  ¶ \textcolor{darkblue}{\textbf{\ipa{hĩ˧-bv̩˧ tso˧\textasciitilde{}tso˧ kʰv̩˧}}} \zh{偷别人的东西} \textcolor{Sepia}{\selectlanguage{english}to steal someone's stuff, to steal someone else's property} \textcolor{PineGreen}{\selectlanguage{french}voler les affaires de quelqu'un}  

\lhead{\firstmark}
\rhead{\botmark}

\subsection{\hspace{-0.5cm} {\Large \textcolor{darkblue}{\textbf{\ipa{kʰv̩˧˥\textsubscript{a}}}}}\hspace{0.5cm}[\kern2pt{\textcolor{darkblue}{\textbf{\ipa{kʰv̩˩˥}}}}\kern2pt]} \hypertarget{k\string_hv\string_=\string_M\string_Ta1}{}
\markboth{\textcolor{darkblue}{\textbf{\ipa{kʰv̩˧˥\textsubscript{a}}}}}{}
\textcolor{teal}{\zh{量词}} \hspace{4pt} \zh{声调类:} MH\textsubscript{a}.
\zh{量词:年、岁。} \textcolor{Sepia}{\selectlanguage{english}Year; year of age.} \textcolor{PineGreen}{\selectlanguage{french}Année.}  ¶ \textcolor{darkblue}{\textbf{\ipa{ɖɯ˧-kʰv̩˧˥}}} \zh{一年} \textcolor{Sepia}{\selectlanguage{english}one year} \textcolor{PineGreen}{\selectlanguage{french}une année}  

\lhead{\firstmark}
\rhead{\botmark}

\subsection{\hspace{-0.5cm} {\Large \textcolor{darkblue}{\textbf{\ipa{kʰv̩˧bv̩˧˥}}}}\hspace{0.5cm}[\kern2pt{\textcolor{darkblue}{\textbf{\ipa{kʰv̩˧bv̩˧˥}}}}\kern2pt]} \hypertarget{k\string_hv\string_=\string_Mbv\string_=\string_M\string_T1}{}
\markboth{\textcolor{darkblue}{\textbf{\ipa{kʰv̩˧bv̩˧˥}}}}{}
\textcolor{teal}{\zh{名词}} \hspace{4pt} \zh{声调类:} MH\#.
\zh{狗窝。} \textcolor{Sepia}{\selectlanguage{english}Kennel, doghouse.} \textcolor{PineGreen}{\selectlanguage{french}Chenil.}  \zh{量词}: \textcolor{darkblue}{\textbf{\ipa{ɭɯ˧}}} 
\lhead{\firstmark}
\rhead{\botmark}

\subsection{\hspace{-0.5cm} {\Large \textcolor{darkblue}{\textbf{\ipa{kʰv̩˩-kʰɤ˩}}}}\hspace{0.5cm}[\kern2pt{\textcolor{darkblue}{\textbf{\ipa{xxxx non-correspondance entre le nombre de morphèmes et le nombre de tons de morphèmes}}}}\kern2pt]} \hypertarget{k\string_hv\string_=\string_B-k\string_h7\string_B1}{}
\markboth{\textcolor{darkblue}{\textbf{\ipa{kʰv̩˩-kʰɤ˩}}}}{}
\textcolor{teal}{\zh{名词}} \hspace{4pt} \zh{声调类:} L.
\zh{鸡窝。} \textcolor{Sepia}{\selectlanguage{english}Chicken nest.} \textcolor{PineGreen}{\selectlanguage{french}Nid de poule, pondoir, endroit où la poule pond.}  \zh{量词}: \textcolor{darkblue}{\textbf{\ipa{ɭɯ˧}}} 
\lhead{\firstmark}
\rhead{\botmark}

\subsection{\hspace{-0.5cm} {\Large \textcolor{darkblue}{\textbf{\ipa{kʰv̩˧kʰv̩˩}}}}\hspace{0.5cm}[\kern2pt{\textcolor{darkblue}{\textbf{\ipa{kʰv̩˩kʰv̩˩˥}}}}\kern2pt]} \hypertarget{k\string_hv\string_=\string_Mk\string_hv\string_=\string_B1}{}
\markboth{\textcolor{darkblue}{\textbf{\ipa{kʰv̩˧kʰv̩˩}}}}{}
\textcolor{teal}{\zh{名词}} \hspace{4pt} \zh{声调类:} L\#.
\zh{狗年。} \textcolor{Sepia}{\selectlanguage{english}Year of the dog.} \textcolor{PineGreen}{\selectlanguage{french}Année du chien.} 
\lhead{\firstmark}
\rhead{\botmark}

\subsection{\hspace{-0.5cm} {\Large \textcolor{darkblue}{\textbf{\ipa{kʰv̩˧kwæ˧}}}}\hspace{0.5cm}[\kern2pt{\textcolor{darkblue}{\textbf{\ipa{kʰv̩˧kwæ˩}}}}\kern2pt]} \hypertarget{k\string_hv\string_=\string_Mkw\{\string_M1}{}
\markboth{\textcolor{darkblue}{\textbf{\ipa{kʰv̩˧kwæ˧}}}}{}
\textcolor{teal}{\zh{名词}} \hspace{4pt} \zh{声调类:} M.
\zh{苦瓜。} \textcolor{Sepia}{\selectlanguage{english}Bitter melon.} \textcolor{PineGreen}{\selectlanguage{french}Concombre amer.}  \zh{【借词】} \zh{苦瓜}
 \zh{量词}: \textcolor{darkblue}{\textbf{\ipa{ɭɯ˧}}} 
\lhead{\firstmark}
\rhead{\botmark}

\subsection{\hspace{-0.5cm} {\Large \textcolor{darkblue}{\textbf{\ipa{kʰv̩˧mæ˧}}}}\hspace{0.5cm}[\kern2pt{\textcolor{darkblue}{\textbf{\ipa{kʰv̩˧mæ˥}}}}\kern2pt]} \hypertarget{k\string_hv\string_=\string_Mm\{\string_M1}{}
\markboth{\textcolor{darkblue}{\textbf{\ipa{kʰv̩˧mæ˧}}}}{}
\textcolor{teal}{\zh{名词}} \hspace{4pt} \zh{声调类:} M.
\zh{强盗。} \textcolor{Sepia}{\selectlanguage{english}Robber, bandit.} \textcolor{PineGreen}{\selectlanguage{french}Voleur, bandit.}  ¶ \textcolor{darkblue}{\textbf{\ipa{kʰv̩˧mæ˧ ʝi˧-hĩ˧-hĩ˧}}} \zh{当强盗的人=强盗} \textcolor{Sepia}{\selectlanguage{english}person who robs, robber} \textcolor{PineGreen}{\selectlanguage{french}personne qui est un bandit, bandit}  
 ¶ \textcolor{darkblue}{\textbf{\ipa{kʰv̩˧mæ˧-ni˩-zo˩! | hĩ˧ lɑ˩-ho˩!}}} \zh{他像强盗似的!会打人的!} \textcolor{Sepia}{\selectlanguage{english}He's like a bandit! He may hit people!} \textcolor{PineGreen}{\selectlanguage{french}Ca doit être un voleur! Il se pourrait qu'il frappe les gens!}  
 ¶ \textcolor{darkblue}{\textbf{\ipa{kʰv̩˧mæ˧-ʑi˩}}} \zh{监狱。直译:“贼家”} \textcolor{Sepia}{\selectlanguage{english}prison: literally “house for thieves”} \textcolor{PineGreen}{\selectlanguage{french}prison: littéralement “maison des voleurs”}  
 ¶ \textcolor{darkblue}{\textbf{\ipa{kʰv̩˧mæ˧-ʝi˧-hĩ˧, | lo˧ʑi˥bv̩˩-qo˩ ʈæ˩!}}} \zh{贼,被关在监狱!} \textcolor{Sepia}{\selectlanguage{english}Thieves are tied up in prisons / are sent to prison!} \textcolor{PineGreen}{\selectlanguage{french}les voleurs, on les met en prison!}  
 ¶ \textcolor{darkblue}{\textbf{\ipa{no˧ | kʰv̩˧mæ˧-pʰæ˧qʰwɤ˩-ne˩-ʝi˩-zo˩!}}} \zh{你有一张贼脸!(控告一个人)} \textcolor{Sepia}{\selectlanguage{english}You have the face of a thief! / You really look like a thief! (An accusation about someone one thinks is a thief)} \textcolor{PineGreen}{\selectlanguage{french}Toi, tu m'as une tête de voleur! (accusation lancée à quelqu'un qu'on pense être un voleur)}  
 \zh{量词}: \textcolor{darkblue}{\textbf{\ipa{v̩˧}}} 
\lhead{\firstmark}
\rhead{\botmark}

\subsection{\hspace{-0.5cm} {\Large \textcolor{darkblue}{\textbf{\ipa{kʰv̩˩mi˩}}}}\hspace{0.5cm}[\kern2pt{\textcolor{darkblue}{\textbf{\ipa{kʰv̩˧mi˧}}}}\kern2pt]} \hypertarget{k\string_hv\string_=\string_Bmi\string_B1}{}
\markboth{\textcolor{darkblue}{\textbf{\ipa{kʰv̩˩mi˩}}}}{}
\textcolor{teal}{\zh{名词}} \hspace{4pt} \zh{声调类:} L.
\zh{狗。} \textcolor{Sepia}{\selectlanguage{english}Dog (either he-dog or she-dog).} \textcolor{PineGreen}{\selectlanguage{french}Chien (sans spécifier le sexe).}  ¶ \textcolor{darkblue}{\textbf{\ipa{kʰv̩˩mi˩ ʈʂʰɯ˩-jɤ˧}}} \zh{这条狗} \textcolor{Sepia}{\selectlanguage{english}\mytextsc{n}+\mytextsc{dem}+\mytextsc{clf}} \textcolor{PineGreen}{\selectlanguage{french}\mytextsc{n}+\mytextsc{dem}+\mytextsc{clf}}  
 ¶ \textcolor{darkblue}{\textbf{\ipa{di˧qo˧-kʰv̩˩mi˩}}} \zh{平坝的狗} \textcolor{Sepia}{\selectlanguage{english}the dogs of the plain (which, unlike dogs in mountain hamlets, get to see lots of passers-by, and are less likely to bite strangers)} \textcolor{PineGreen}{\selectlanguage{french}les chiens de la plaine (qui à la différence des chiens des petits hameaux de montagne voient beaucoup de passage et sont moins susceptibles de mordre les inconnus de passage)}  
 ¶ \textcolor{darkblue}{\textbf{\ipa{kʰv̩˩mi˩-gɤ˥ljɤ˩}}} \zh{流浪的狗} \textcolor{Sepia}{\selectlanguage{english}roving dog} \textcolor{PineGreen}{\selectlanguage{french}chien errant}  
 \zh{量词}: \textcolor{darkblue}{\textbf{\ipa{v̩˧, terme respectueux (le même que pour les humains)}}} \textcolor{darkblue}{\textbf{\ipa{on peut aussi dire: jɤ˧˥}}} 
\lhead{\firstmark}
\rhead{\botmark}

\subsection{\hspace{-0.5cm} {\Large \textcolor{darkblue}{\textbf{\ipa{kʰv̩˧mv̩˥}}}}\hspace{0.5cm}[\kern2pt{\textcolor{darkblue}{\textbf{\ipa{kʰv̩˩mv̩˩˥}}}}\kern2pt]} \hypertarget{k\string_hv\string_=\string_Mmv\string_=\string_T1}{}
\markboth{\textcolor{darkblue}{\textbf{\ipa{kʰv̩˧mv̩˥}}}}{}
\textcolor{teal}{\zh{名词}} \hspace{4pt} \zh{声调类:} H\#.
\zh{小母狗(给刚出生的女孩起的名字,让鬼对她不感兴趣,不会来害小孩)。} \textcolor{Sepia}{\selectlanguage{english}Female puppy. The term is also used as a temporary name for little girls, during the first months of their life, before they are given a real name. This ugly term is intended to disgust evil spirits, which will therefore turn their attention away from the infant. (In the early 21st century, the registry office requires a name to be given at birth; but this name only begins to be used by the family after the first months of life have elapsed.).} \textcolor{PineGreen}{\selectlanguage{french}Chienne, petit chiot femelle. Le terme est également employé comme nom provisoire pour les fillettes pendant leurs premiers mois, avant qu'on ne leur donne un vrai nom. Le vilain nom dont on l'affuble vise à éviter que le nourrisson ne soit repéré par de mauvais esprits. (Actuellement, l'état-civil nécessite qu'un nom soit donné dès la naissance; mais celui-ci ne commence à être employé dans les conversations familiales qu'une fois passés les premiers mois.).}  \zh{量词}: \textcolor{darkblue}{\textbf{\ipa{v̩˧}}} 
\lhead{\firstmark}
\rhead{\botmark}

\subsection{\hspace{-0.5cm} {\Large \textcolor{darkblue}{\textbf{\ipa{kʰv̩˧nɑ˥}}}}\hspace{0.5cm}[\kern2pt{\textcolor{darkblue}{\textbf{\ipa{kʰv̩˩nɑ˧˥}}}}\kern2pt]} \hypertarget{k\string_hv\string_=\string_MnA\string_T1}{}
\markboth{\textcolor{darkblue}{\textbf{\ipa{kʰv̩˧nɑ˥}}}}{}
\textcolor{teal}{\zh{名词}} \hspace{4pt} \zh{声调类:} H\#.
\zh{狗。} \textcolor{Sepia}{\selectlanguage{english}Dog (formal word, used in elevated speech).} \textcolor{PineGreen}{\selectlanguage{french}Chien (registre de langage relevé).}  \zh{量词}: \textcolor{darkblue}{\textbf{\ipa{mi˩}}} 
\lhead{\firstmark}
\rhead{\botmark}

\subsection{\hspace{-0.5cm} {\Large \textcolor{darkblue}{\textbf{\ipa{kʰv̩˧pʰæ˧}}}}\hspace{0.5cm}[\kern2pt{\textcolor{darkblue}{\textbf{\ipa{kʰv̩˧pʰæ˥}}}}\kern2pt]} \hypertarget{k\string_hv\string_=\string_Mp\string_h\{\string_M1}{}
\markboth{\textcolor{darkblue}{\textbf{\ipa{kʰv̩˧pʰæ˧}}}}{}
\textcolor{teal}{\zh{名词}} \hspace{4pt} \zh{声调类:} M.
\zh{年龄。} \textcolor{Sepia}{\selectlanguage{english}Age.} \textcolor{PineGreen}{\selectlanguage{french}Âge.}  ¶ \textcolor{darkblue}{\textbf{\ipa{kʰv̩˧pʰæ˧ tɕi˩}}} \zh{年轻} \textcolor{Sepia}{\selectlanguage{english}young} \textcolor{PineGreen}{\selectlanguage{french}jeune}  
 ¶ \textcolor{darkblue}{\textbf{\ipa{kʰv̩˧pʰæ˧ | tɕi˩-hĩ˩˥}}} \zh{年轻的} \textcolor{Sepia}{\selectlanguage{english}young} \textcolor{PineGreen}{\selectlanguage{french}jeune}  

\lhead{\firstmark}
\rhead{\botmark}

\subsection{\hspace{-0.5cm} {\Large \textcolor{darkblue}{\textbf{\ipa{kʰv̩˧-pʰo˥}}}}\hspace{0.5cm}[\kern2pt{\textcolor{darkblue}{\textbf{\ipa{xxxx non-correspondance entre le nombre de morphèmes et le nombre de tons de morphèmes}}}}\kern2pt]} \hypertarget{k\string_hv\string_=\string_M-p\string_ho\string_T1}{}
\markboth{\textcolor{darkblue}{\textbf{\ipa{kʰv̩˧-pʰo˥}}}}{}
\textcolor{teal}{\zh{名词}} \hspace{4pt} \zh{声调类:} H\#.
\zh{半年。} \textcolor{Sepia}{\selectlanguage{english}Half a year.} \textcolor{PineGreen}{\selectlanguage{french}Une demi-année.}  ¶ \textcolor{darkblue}{\textbf{\ipa{ɖɯ˧-kʰv̩˧-kʰv̩˥-pʰo˩}}} \zh{一年半} \textcolor{Sepia}{\selectlanguage{english}one year and a half} \textcolor{PineGreen}{\selectlanguage{french}un an et demi}  

\lhead{\firstmark}
\rhead{\botmark}

\subsection{\hspace{-0.5cm} {\Large \textcolor{darkblue}{\textbf{\ipa{kʰv̩˧pʰv̩\#˥}}}}\hspace{0.5cm}[\kern2pt{\textcolor{darkblue}{\textbf{\ipa{kʰv̩˧pʰv̩˥}}}}\kern2pt]} \hypertarget{k\string_hv\string_=\string_Mp\string_hv\string_=\#\string_T1}{}
\markboth{\textcolor{darkblue}{\textbf{\ipa{kʰv̩˧pʰv̩\#˥}}}}{}
\textcolor{teal}{\zh{名词}} \hspace{4pt} \zh{声调类:} \#H.
\zh{公狗。} \textcolor{Sepia}{\selectlanguage{english}He-dog.} \textcolor{PineGreen}{\selectlanguage{french}Chien mâle (forme élicitée).}  ¶ \textcolor{darkblue}{\textbf{\ipa{kʰv̩˧pʰv̩˧ ʈʂʰɯ˧-ɭɯ\#˥}}} \zh{这只公狗} \textcolor{Sepia}{\selectlanguage{english}\mytextsc{n}+\mytextsc{dem}+\mytextsc{clf}} \textcolor{PineGreen}{\selectlanguage{french}\mytextsc{n}+\mytextsc{dem}+\mytextsc{clf}}  
 ¶ \textcolor{darkblue}{\textbf{\ipa{kʰv̩˧pʰv̩˧ tʰv̩˧-mi˧˥}}} \zh{这只公狗} \textcolor{Sepia}{\selectlanguage{english}\mytextsc{n}+\mytextsc{dem}+\mytextsc{clf}} \textcolor{PineGreen}{\selectlanguage{french}\mytextsc{n}+\mytextsc{dem}+\mytextsc{clf}}  
 ¶ \textcolor{darkblue}{\textbf{\ipa{kʰv̩˧pʰv̩˧ tʰv̩˧-v̩\#˥}}} \zh{这个公狗} \textcolor{Sepia}{\selectlanguage{english}\mytextsc{n}+\mytextsc{dem}+\mytextsc{clf}} \textcolor{PineGreen}{\selectlanguage{french}\mytextsc{n}+\mytextsc{dem}+\mytextsc{clf}}  
 \zh{量词}: \textcolor{darkblue}{\textbf{\ipa{v̩˧ / mi˩ / ɭɯ˧}}} 
\lhead{\firstmark}
\rhead{\botmark}

\subsection{\hspace{-0.5cm} {\Large \textcolor{darkblue}{\textbf{\ipa{kʰv̩˧qʰwɤ˧˥}}}}\hspace{0.5cm}[\kern2pt{\textcolor{darkblue}{\textbf{\ipa{kʰv̩˧qʰwɤ˧}}}}\kern2pt]} \hypertarget{k\string_hv\string_=\string_Mq\string_hw7\string_M\string_T1}{}
\markboth{\textcolor{darkblue}{\textbf{\ipa{kʰv̩˧qʰwɤ˧˥}}}}{}
\textcolor{teal}{\zh{名词}} \hspace{4pt} \zh{声调类:} MH\#.
\zh{庄稼收成不好的(一)年。} \textcolor{Sepia}{\selectlanguage{english}Bad year, year when the crops are bad.} \textcolor{PineGreen}{\selectlanguage{french}Mauvaise année, année de disette.}  ¶ \textcolor{darkblue}{\textbf{\ipa{kʰv̩˧qʰwɤ˧ tʰv̩˧˥}}} \zh{今年,收成不好。} \textcolor{Sepia}{\selectlanguage{english}the year is bad; crops are bad this year; a bad year has come} \textcolor{PineGreen}{\selectlanguage{french}une mauvaise année a lieu, une année de mauvaise récolte/de disette}  

\lhead{\firstmark}
\rhead{\botmark}

\subsection{\hspace{-0.5cm} {\Large \textcolor{darkblue}{\textbf{\ipa{kʰv̩˧ʂæ˧˥}}}}\hspace{0.5cm}[\kern2pt{\textcolor{darkblue}{\textbf{\ipa{kʰv̩˧ʂæ˥}}}}\kern2pt]} \hypertarget{k\string_hv\string_=\string_Ms`\{\string_M\string_T1}{}
\markboth{\textcolor{darkblue}{\textbf{\ipa{kʰv̩˧ʂæ˧˥}}}}{}
\textcolor{teal}{\zh{动词}} \hspace{4pt} \zh{声调类:} MH.
\zh{打猎、赶走、驱逐。} \textcolor{Sepia}{\selectlanguage{english}To hunt (leading a dog).} \textcolor{PineGreen}{\selectlanguage{french}Chasser; mener un chien de chasse.}  ¶ \textcolor{darkblue}{\textbf{\ipa{kʰv̩˧ʂæ˧ hɯ˧˥}}} \zh{狩猎去了} \textcolor{Sepia}{\selectlanguage{english}(He/she) has gone hunting} \textcolor{PineGreen}{\selectlanguage{french}(Elle/il) est parti(e) chasser}  

\lhead{\firstmark}
\rhead{\botmark}

\subsection{\hspace{-0.5cm} {\Large \textcolor{darkblue}{\textbf{\ipa{kʰv̩˧ʂɯ˥}}}}\hspace{0.5cm}[\kern2pt{\textcolor{darkblue}{\textbf{\ipa{kʰv̩˧ʂɯ˥}}}}\kern2pt]} \hypertarget{k\string_hv\string_=\string_Ms`M\string_T1}{}
\markboth{\textcolor{darkblue}{\textbf{\ipa{kʰv̩˧ʂɯ˥}}}}{}
\textcolor{teal}{\zh{动词}} \hspace{4pt} \zh{声调类:} .
\zh{过年。} \textcolor{Sepia}{\selectlanguage{english}To celebrate the New Year.} \textcolor{PineGreen}{\selectlanguage{french}Fêter le Nouvel An.} 
\lhead{\firstmark}
\rhead{\botmark}

\subsection{\hspace{-0.5cm} {\Large \textcolor{darkblue}{\textbf{\ipa{kʰv̩˧sɯ˧sɯ˩}}}}\hspace{0.5cm}[\kern2pt{\textcolor{darkblue}{\textbf{\ipa{kʰv̩˧sɯ˧sɯ˩}}}}\kern2pt]} \hypertarget{k\string_hv\string_=\string_MsM\string_MsM\string_B1}{}
\markboth{\textcolor{darkblue}{\textbf{\ipa{kʰv̩˧sɯ˧sɯ˩}}}}{}
\textcolor{teal}{\zh{名词}} \hspace{4pt} \zh{声调类:} L\#.
\zh{球穗千斤拔、半灌木千斤拔、大苞千斤拔。} \textcolor{Sepia}{\selectlanguage{english}A flowering plant in the legume family: \textit{Flemingia strobilifera}, also known as \textit{Moghania fruticulosa}.} \textcolor{PineGreen}{\selectlanguage{french}Plante à fleurs, \textit{Flemingia strobilifera}, aussi appelée \textit{Moghania fruticulosa} (nom en chinois local: “oreille de souris”, du fait de la forme de la feuille).} \zh{当地汉语方言:}\zh{耗子耳朵。} \zh{量词}: \textcolor{darkblue}{\textbf{\ipa{kɤ˧˥}}} 
\lhead{\firstmark}
\rhead{\botmark}

\subsection{\hspace{-0.5cm} {\Large \textcolor{darkblue}{\textbf{\ipa{kʰv̩˧tɕʰi˥\$}}}}\hspace{0.5cm}[\kern2pt{\textcolor{darkblue}{\textbf{\ipa{kʰv̩˧tɕʰi˥}}}}\kern2pt]} \hypertarget{k\string_hv\string_=\string_Mts£\string_hi\string_T\$1}{}
\markboth{\textcolor{darkblue}{\textbf{\ipa{kʰv̩˧tɕʰi˥\$}}}}{}
\textcolor{teal}{\zh{名词}} \hspace{4pt} \zh{声调类:} H\$.
\zh{办法。} \textcolor{Sepia}{\selectlanguage{english}Solution, way out.} \textcolor{PineGreen}{\selectlanguage{french}Solution, méthode.}  ¶ \textcolor{darkblue}{\textbf{\ipa{kʰv̩˧tɕʰi˥ | mɤ˧-dʑo˧-ze˧! | ɻ̃˧-ɻ̍˧ tʰo˩!}}} \zh{没有办法了!糟糕了!} \textcolor{Sepia}{\selectlanguage{english}There is nothing we can do anymore! It's a catastrophe!} \textcolor{PineGreen}{\selectlanguage{french}Il n'y a plus rien à faire! C'est la catastrophe!}  

\lhead{\firstmark}
\rhead{\botmark}

\subsection{\hspace{-0.5cm} {\Large \textcolor{darkblue}{\textbf{\ipa{kʰv̩˧tsʰi˧-bo˥tsʰi˩}}}}\hspace{0.5cm}[\kern2pt{\textcolor{darkblue}{\textbf{\ipa{kʰv̩˧tsʰi˧bo˥tsʰi˩}}}}\kern2pt]} \hypertarget{k\string_hv\string_=\string_Mts\string_hi\string_M-bo\string_Tts\string_hi\string_B1}{}
\markboth{\textcolor{darkblue}{\textbf{\ipa{kʰv̩˧tsʰi˧-bo˥tsʰi˩}}}}{}
\textcolor{teal}{\zh{名词}} \hspace{4pt} \zh{声调类:} \#H-.
\zh{鼹鼠。} \textcolor{Sepia}{\selectlanguage{english}Mole shrew.} \textcolor{PineGreen}{\selectlanguage{french}Taupe.}  \zh{量词}: \textcolor{darkblue}{\textbf{\ipa{pʰo˧˥}}} \textcolor{darkblue}{\textbf{\ipa{v̩˧}}} 
\lhead{\firstmark}
\rhead{\botmark}

\subsection{\hspace{-0.5cm} {\Large \textcolor{darkblue}{\textbf{\ipa{kʰv̩˩tsɤ˩mi˥}}}}\hspace{0.5cm}[\kern2pt{\textcolor{darkblue}{\textbf{\ipa{kʰv̩˩tsɤ˩mi˥}}}}\kern2pt]} \hypertarget{k\string_hv\string_=\string_Bts7\string_Bmi\string_T1}{}
\markboth{\textcolor{darkblue}{\textbf{\ipa{kʰv̩˩tsɤ˩mi˥}}}}{}
\textcolor{teal}{\zh{名词}} \hspace{4pt} \zh{声调类:} L+H\#.
\zh{母狗。} \textcolor{Sepia}{\selectlanguage{english}She-dog.} \textcolor{PineGreen}{\selectlanguage{french}Chienne.}  \zh{量词}: \textcolor{darkblue}{\textbf{\ipa{v̩˧}}} 
\lhead{\firstmark}
\rhead{\botmark}

\subsection{\hspace{-0.5cm} {\Large \textcolor{darkblue}{\textbf{\ipa{kʰv̩˧zo˥\$}}}}\hspace{0.5cm}[\kern2pt{\textcolor{darkblue}{\textbf{\ipa{kʰv̩˧zo˥}}}}\kern2pt]} \hypertarget{k\string_hv\string_=\string_Mzo\string_T\$1}{}
\markboth{\textcolor{darkblue}{\textbf{\ipa{kʰv̩˧zo˥\$}}}}{}
\textcolor{teal}{\zh{名词}} \hspace{4pt} \zh{声调类:} H\$.
\zh{一个姓。这个姓,永宁有两家。} \textcolor{Sepia}{\selectlanguage{english}A family name from Yongning. There are two families in Yongning that carry this name.} \textcolor{PineGreen}{\selectlanguage{french}Nom de clan/famille étendue. Deux familles portent ce nom à Yongning.}  ¶ \textcolor{darkblue}{\textbf{\ipa{kʰv̩˧zo˧=ɻ̍˥\$}}} \zh{\textcolor{darkblue}{\textbf{\ipa{/kʰv̩˧zo˥\$/}}}家族} \textcolor{Sepia}{\selectlanguage{english}the \textcolor{darkblue}{\textbf{\ipa{/kʰv̩˧zo˥\$/}}} clan, the \textcolor{darkblue}{\textbf{\ipa{/kʰv̩˧zo˥\$/}}} family} \textcolor{PineGreen}{\selectlanguage{french}La famille \textcolor{darkblue}{\textbf{\ipa{/kʰv̩˧zo˥\$/}}}, les \textcolor{darkblue}{\textbf{\ipa{/kʰv̩˧zo˥\$/}}}}  
 ¶ \textcolor{darkblue}{\textbf{\ipa{kʰv̩˧zo˥-tsʰɯ˩ɻ̍˩}}} \zh{一个人的名字:姓为\textcolor{darkblue}{\textbf{\ipa{/kʰv̩˧zo˥\$/}}},名为\textcolor{darkblue}{\textbf{\ipa{/tsʰɯ˧ɻ\#˥/}}}} \textcolor{Sepia}{\selectlanguage{english}the name of a person, containing both a family name: \textcolor{darkblue}{\textbf{\ipa{/kʰv̩˧zo˥\$/}}}, and a given name: \textcolor{darkblue}{\textbf{\ipa{/tsʰɯ˧ɻ\#˥/}}}} \textcolor{PineGreen}{\selectlanguage{french}nom d'une personne, comportant un nom de famille (\textcolor{darkblue}{\textbf{\ipa{/kʰv̩˧zo˥\$/}}}) et un prénom (\textcolor{darkblue}{\textbf{\ipa{/tsʰɯ˧ɻ\#˥/}}})}  

\lhead{\firstmark}
\rhead{\botmark}

\subsection{\hspace{-0.5cm} {\Large \textcolor{darkblue}{\textbf{\ipa{kʰv̩˧zo\#˥}}}}\hspace{0.5cm}[\kern2pt{\textcolor{darkblue}{\textbf{\ipa{kʰv̩˧zo˧}}}}\kern2pt]} \hypertarget{k\string_hv\string_=\string_Mzo\#\string_T1}{}
\markboth{\textcolor{darkblue}{\textbf{\ipa{kʰv̩˧zo\#˥}}}}{}
\textcolor{teal}{\zh{名词}} \hspace{4pt} \zh{声调类:} \#H.
\zh{公狗(给刚出生的男孩子的名字,让鬼对他不感兴趣,不过来害小孩)。} \textcolor{Sepia}{\selectlanguage{english}Male dog. The term is also used as a temporary name for little boys, during the first months of their life, before they are given a real name. This ugly term is intended to disgust evil spirits, which will therefore turn their attention away from the infant. (In the early 21st century, the registry office requires a name to be given at birth; but this name only begins to be used by the family after the first months of life have elapsed.).} \textcolor{PineGreen}{\selectlanguage{french}Chien (mâle), chiot. Le terme est également employé comme nom provisoire pour les garçonnets pendant leurs premiers mois, avant qu'on ne leur donne un vrai nom. Le vilain nom dont on l'affuble vise à éviter que le nourrisson ne soit repéré par de mauvais esprits. (Actuellement, l'état-civil nécessite qu'un nom soit donné dès la naissance; mais celui-ci ne commence à être employé dans les conversations familiales qu'une fois passés les premiers mois.).}  ¶ \textcolor{darkblue}{\textbf{\ipa{kʰv̩˧zo˧ ʈʂʰɯ˧-ɭɯ\#˥}}} \zh{这只公狗} \textcolor{Sepia}{\selectlanguage{english}\mytextsc{n}+\mytextsc{dem}+\mytextsc{clf}} \textcolor{PineGreen}{\selectlanguage{french}\mytextsc{n}+\mytextsc{dem}+\mytextsc{clf}}  
 ¶ \textcolor{darkblue}{\textbf{\ipa{kʰv̩˧zo˧ tʰv̩˧-mi˧˥}}} \zh{这只公狗} \textcolor{Sepia}{\selectlanguage{english}\mytextsc{n}+\mytextsc{dem}+\mytextsc{clf}} \textcolor{PineGreen}{\selectlanguage{french}\mytextsc{n}+\mytextsc{dem}+\mytextsc{clf}}  
 ¶ \textcolor{darkblue}{\textbf{\ipa{kʰv̩˧zo˧ tʰv̩˧-v̩\#˥}}} \zh{这只公狗} \textcolor{Sepia}{\selectlanguage{english}\mytextsc{n}+\mytextsc{dem}+\mytextsc{clf}} \textcolor{PineGreen}{\selectlanguage{french}\mytextsc{n}+\mytextsc{dem}+\mytextsc{clf}}  
 ¶ \textcolor{darkblue}{\textbf{\ipa{kʰv̩˧zo˥-kʰv̩˩mv̩˩}}} \zh{小狗与母狗} \textcolor{Sepia}{\selectlanguage{english}puppy and she-dog} \textcolor{PineGreen}{\selectlanguage{french}chien et chienne}  
 \zh{量词}: \textcolor{darkblue}{\textbf{\ipa{v̩˧ / mi˩ / ɭɯ˧}}} 
\lhead{\firstmark}
\rhead{\botmark}

\subsection{\hspace{-0.5cm} {\Large \textcolor{darkblue}{\textbf{\ipa{kʰwæ˧ɻæ\#˥}}}}\hspace{0.5cm}[\kern2pt{\textcolor{darkblue}{\textbf{\ipa{kʰwæ˧ɻæ˧}}}}\kern2pt]} \hypertarget{k\string_hw\{\string_Mr£`\{\#\string_T1}{}
\markboth{\textcolor{darkblue}{\textbf{\ipa{kʰwæ˧ɻæ\#˥}}}}{}
\textcolor{teal}{\zh{名词}} \hspace{4pt} \zh{声调类:} \#H.
\zh{毡子。也用来指席子,垫子等。} \textcolor{Sepia}{\selectlanguage{english}Felt; extended use: mat (even if not made of felt), cushion….} \textcolor{PineGreen}{\selectlanguage{french}Feutre; par extension: natte, tapis (même en vannerie), coussin….}  ¶ \textcolor{darkblue}{\textbf{\ipa{kʰwæ˧ɻæ˧ tʰi˧-kʰo˥}}} \zh{铺席子} \textcolor{Sepia}{\selectlanguage{english}to spread a mat} \textcolor{PineGreen}{\selectlanguage{french}étendre la natte}  
 \zh{量词}: \textcolor{darkblue}{\textbf{\ipa{tsʰi˥}}} 
\lhead{\firstmark}
\rhead{\botmark}

\subsection{\hspace{-0.5cm} {\Large \textcolor{darkblue}{\textbf{\ipa{kʰwɤ˥\textsubscript{a}}}}}\hspace{0.5cm}[\kern2pt{\textcolor{darkblue}{\textbf{\ipa{kʰwɤ˥}}}}\kern2pt]} \hypertarget{k\string_hw7\string_Ta1}{}
\markboth{\textcolor{darkblue}{\textbf{\ipa{kʰwɤ˥\textsubscript{a}}}}}{}
\textcolor{teal}{\zh{量词}} \hspace{4pt} \zh{声调类:} H\textsubscript{a}.
\zh{量词:块。一块肉、一口饭。} \textcolor{Sepia}{\selectlanguage{english}A piece of, a chunk of; a mouthful of.} \textcolor{PineGreen}{\selectlanguage{french}Classificateur des morceaux/bouchées.}  ¶ \textcolor{darkblue}{\textbf{\ipa{ɖɯ˧-kʰwɤ˥\textasciitilde{}ɖɯ˩-kʰwɤ˩}}} \zh{一块一块地} \textcolor{Sepia}{\selectlanguage{english}chunk by chunk, one chunk after the other} \textcolor{PineGreen}{\selectlanguage{french}par petites bouchées, par petits morceaux}  
 ¶ \textcolor{darkblue}{\textbf{\ipa{kʰwɤ˧ | ɖɯ˧-ʂe˧-ɻ̍˩!}}} \zh{你们得要做出决定!} \textcolor{Sepia}{\selectlanguage{english}Go ahead and decide! / Please make a decision!} \textcolor{PineGreen}{\selectlanguage{french}Décidez! / Il faut vous décider!}  
 ¶ \textcolor{darkblue}{\textbf{\ipa{ɖɯ˧-kʰwɤ˧ so˧˥, | ɖɯ˧-kʰwɤ˥ fv̩˩!}}} \zh{学一点,就高兴一点!(评说语言调查工作:合作人看着本词典的初稿,说:这是一项很大的工程,关键的是调查者要有兴趣,欣赏每个新学的语言信息。)} \textcolor{Sepia}{\selectlanguage{english}Each new word is a new joy! (A comment by the consultant about the investigator's enjoyment of fieldwork. She takes a look at a draft dictionary, and comments that it represents a great deal of work, and that what matters is that the investigator should feel an interest in it, considering each new 'piece' – each addition to the dictionary – as a source of joy.)} \textcolor{PineGreen}{\selectlanguage{french}Chaque mot appris représente une joie de plus! (Commentaire de la locutrice au sujet du travail de l'enquêteur. Tenant en main le manuscrit de ce dictionnaire, elle commente: cela représente un travail immense; l'important est que l'enquêteur y trouve de l'intérêt: chaque information nouvelle – chaque “morceau” de langue – ajoutée au dictionnaire est une joie pour l'enquêteur.)}  

\lhead{\firstmark}
\rhead{\botmark}

\subsection{\hspace{-0.5cm} {\Large \textcolor{darkblue}{\textbf{\ipa{kʰwɤ˧pʰv̩˧}}}}\hspace{0.5cm}[\kern2pt{\textcolor{darkblue}{\textbf{\ipa{kʰwɤ˧pʰv̩˧}}}}\kern2pt]} \hypertarget{k\string_hw7\string_Mp\string_hv\string_=\string_M1}{}
\markboth{\textcolor{darkblue}{\textbf{\ipa{kʰwɤ˧pʰv̩˧}}}}{}
\textcolor{teal}{\zh{名词}} \hspace{4pt} \zh{声调类:} M.
\zh{草坪、草地。} \textcolor{Sepia}{\selectlanguage{english}Meadow.} \textcolor{PineGreen}{\selectlanguage{french}Pré: soit prairie de plaine, soit prairie d'altitude (alpage).} 
\lhead{\firstmark}
\rhead{\botmark}

\subsection{\hspace{-0.5cm} {\Large \textcolor{darkblue}{\textbf{\ipa{kʰwɤ˧pʰv̩˧-mo˧˥}}}}\hspace{0.5cm}[\kern2pt{\textcolor{darkblue}{\textbf{\ipa{xxxx non-correspondance entre le nombre de morphèmes et le nombre de tons de morphèmes}}}}\kern2pt]} \hypertarget{k\string_hw7\string_Mp\string_hv\string_=\string_M-mo\string_M\string_T1}{}
\markboth{\textcolor{darkblue}{\textbf{\ipa{kʰwɤ˧pʰv̩˧-mo˧˥}}}}{}
\textcolor{teal}{\zh{名词}} \hspace{4pt} \zh{声调类:} MH\#.
\zh{可以吃的一种菌子:可能是四孢蘑菇。直译:“草坪菌”。} \textcolor{Sepia}{\selectlanguage{english}Meadow mushroom: a sort of edible mushroom that grows on meadows (not yet identified; perhaps \textit{Agaricus campestris}).} \textcolor{PineGreen}{\selectlanguage{french}Champignon des prés: une sorte de champignon comestible (pas encore identifiée): agaric champêtre ou rosé des prés, \textit{Agaricus campestris}?} 
\lhead{\firstmark}
\rhead{\botmark}

\newpage
\section*{\centering- \textcolor{darkblue}{\textbf{\ipa{l}}} \textcolor{darkblue}{\textbf{\ipa{ɭ}}} -}
\subsection{\hspace{-0.5cm} {\Large \textcolor{darkblue}{\textbf{\ipa{‑lɑ˧}}} \textsubscript{1}}\hspace{0.5cm}[\kern2pt{\textcolor{darkblue}{\textbf{\ipa{xxxx groupe tonal entier sans aucun ton}}}}\kern2pt]} \hypertarget{‑lA\string_M1}{}
\markboth{\textcolor{darkblue}{\textbf{\ipa{‑lɑ˧}}} \textsubscript{1}}{}
\textcolor{teal}{\zh{助词}} \hspace{4pt} \zh{声调类:} 0.
\zh{只,才。} \textcolor{Sepia}{\selectlanguage{english}Only.} \textcolor{PineGreen}{\selectlanguage{french}Seulement.}  ¶ \textcolor{darkblue}{\textbf{\ipa{ʈʂʰɯ˧-lɑ˩ ɲi˩-ze˩-mæ˩!}}} \zh{就这些了! / 就这些而已! / 就这样!} \textcolor{Sepia}{\selectlanguage{english}That's all!} \textcolor{PineGreen}{\selectlanguage{french}C'est tout ! / Voilà tout !}  

\lhead{\firstmark}
\rhead{\botmark}

\subsection{\hspace{-0.5cm} {\Large \textcolor{darkblue}{\textbf{\ipa{‑lɑ˧}}} \textsubscript{2}}\hspace{0.5cm}[\kern2pt{\textcolor{darkblue}{\textbf{\ipa{xxxx groupe tonal entier sans aucun ton}}}}\kern2pt]} \hypertarget{‑lA\string_M2}{}
\markboth{\textcolor{darkblue}{\textbf{\ipa{‑lɑ˧}}} \textsubscript{2}}{}
\textcolor{teal}{\zh{助词}} \hspace{4pt} \zh{声调类:} 0.
\zh{和、与、跟。} \textcolor{Sepia}{\selectlanguage{english}Too, also, and.} \textcolor{PineGreen}{\selectlanguage{french}Et, aussi.}  ¶ \textcolor{darkblue}{\textbf{\ipa{ɖɯ˧-kʰv̩˧-lɑ˥ | so˩-ɬi˩˥}}} \zh{一岁三个月} \textcolor{Sepia}{\selectlanguage{english}one year and three months (context: indicating the age of an infant)} \textcolor{PineGreen}{\selectlanguage{french}un an et trois mois (contexte: on indique l'âge d'un petit enfant)}  
 ¶ \textcolor{darkblue}{\textbf{\ipa{ʈʂʰɯ˧-lɑ˧ | mɤ˧-bi˧, | njɤ˧-lɑ˧ mɤ˧-bi˧!}}} \zh{他不去(的话),我也不去!} \textcolor{Sepia}{\selectlanguage{english}(If) (s)he does not go, I'm not going either!} \textcolor{PineGreen}{\selectlanguage{french}s'il n'y va pas, moi non plus!}  
 ¶ \textcolor{darkblue}{\textbf{\ipa{hĩ˧-lɑ˩ | dʑɤ˧˥, | mv̩˧di˧-lɑ˥ | dʑɤ˧˥! / hĩ˧-lɑ˩ | dʑɤ˧˥, | lv̩˧-lɑ˧ | dʑɤ˧˥!}}} \zh{人也好,田也好!(习语:将女孩嫁出去前,一家人打听对方家如何,推荐的人保证:“他们家,人也好,田也好!”)} \textcolor{Sepia}{\selectlanguage{english}The people are good; and the land is good! / The people are good; and the fields are good! (A set phrase to recommend a family which a young woman is considering joining through marriage: the people are good, and their land is good.)} \textcolor{PineGreen}{\selectlanguage{french}les gens (y) sont bons, (et) la terre (y) est bonne (formule de recommandation pour la famille que va rejoindre une jeune femme lors de son mariage)}  
 ¶ \textcolor{darkblue}{\textbf{\ipa{hĩ˧ F | dʑɤ˧˥, | mv̩˧di˧˥ F | dʑɤ˧˥!}}} \zh{同上} \textcolor{Sepia}{\selectlanguage{english}as above} \textcolor{PineGreen}{\selectlanguage{french}même sens}  
 ¶ \textcolor{darkblue}{\textbf{\ipa{mɤ˧-lɑ˧ dʑɤ˧˥!}}} \zh{猪油也好!(按照上面例子的变体)} \textcolor{Sepia}{\selectlanguage{english}The grease too is good! (Elicited variant on the preceding examples)} \textcolor{PineGreen}{\selectlanguage{french}la graisse aussi (y) est bonne! (variation élicitée à partir des exemples qui précèdent)}  
 ¶ \textcolor{darkblue}{\textbf{\ipa{qæ˩-lɑ˥ | dʑɤ˧˥!}}} \zh{油也好!} \textcolor{Sepia}{\selectlanguage{english}The oil too is good! (Elicited variant on the preceding examples)} \textcolor{PineGreen}{\selectlanguage{french}l’huile aussi (y) est bonne! (variation à partir de l'exemple qui précède)}  
 ¶ \textcolor{darkblue}{\textbf{\ipa{ʈʂʰɯ˧-lɑ˧ | mɤ˧-bi˧, | njɤ˧ | mɤ˧-bi˧-ze˧! / ʈʰɯ˧ mɤ˧-bi˧-ze˧-dʑo˧, | njɤ˧-lɑ˧ | mɤ˧-bi˧-ze˧!}}} \zh{他如果不去,我也不去!} \textcolor{Sepia}{\selectlanguage{english}If he doesn't go, I'm not going either!} \textcolor{PineGreen}{\selectlanguage{french}s'il n'y va pas, moi non plus!}  

\lhead{\firstmark}
\rhead{\botmark}

\subsection{\hspace{-0.5cm} {\Large \textcolor{darkblue}{\textbf{\ipa{lɑ˧}}}}\hspace{0.5cm}[\kern2pt{\textcolor{darkblue}{\textbf{\ipa{lɑ˥}}}}\kern2pt]} \hypertarget{lA\string_M1}{}
\markboth{\textcolor{darkblue}{\textbf{\ipa{lɑ˧}}}}{}
\textcolor{teal}{\zh{名词}} \hspace{4pt} \zh{声调类:} M.
\zh{老虎。} \textcolor{Sepia}{\selectlanguage{english}Tiger.} \textcolor{PineGreen}{\selectlanguage{french}Tigre.}  \zh{量词}: \textcolor{darkblue}{\textbf{\ipa{pʰo˧˥}}} 
\lhead{\firstmark}
\rhead{\botmark}

\subsection{\hspace{-0.5cm} {\Large \textcolor{darkblue}{\textbf{\ipa{lɑ˧bi\#˥}}}}\hspace{0.5cm}[\kern2pt{\textcolor{darkblue}{\textbf{\ipa{lɑ˧bi˧}}}}\kern2pt]} \hypertarget{lA\string_Mbi\#\string_T1}{}
\markboth{\textcolor{darkblue}{\textbf{\ipa{lɑ˧bi\#˥}}}}{}
\textcolor{teal}{\zh{名词}} \hspace{4pt} \zh{声调类:} \#H.
\zh{陡坡、土坡、斜坡。} \textcolor{Sepia}{\selectlanguage{english}Steep slope.} \textcolor{PineGreen}{\selectlanguage{french}Escarpement, pente raide, terrain escarpé.}  ¶ \textcolor{darkblue}{\textbf{\ipa{lɑ˧bi˧-tsɤ˧}}} \zh{‘像陡坡’,等于:很陡} \textcolor{Sepia}{\selectlanguage{english}steep (literally 'like a steep slope')} \textcolor{PineGreen}{\selectlanguage{french}raide, escarpé (littéralement 'comme un escarpement')}  
 ¶ \textcolor{darkblue}{\textbf{\ipa{lɑ˧bi˧-tsɤ˧ | ʐwæ˩˥!}}} \zh{陡得很!} \textcolor{Sepia}{\selectlanguage{english}It is really steep!} \textcolor{PineGreen}{\selectlanguage{french}C'est très pentu! (Littéralement: 'Ca ressemble vraiment à une pente raide!')}  

\lhead{\firstmark}
\rhead{\botmark}

\subsection{\hspace{-0.5cm} {\Large \textcolor{darkblue}{\textbf{\ipa{lɑ˧do\#˥}}}}\hspace{0.5cm}[\kern2pt{\textcolor{darkblue}{\textbf{\ipa{lɑ˧do˧}}}}\kern2pt]} \hypertarget{lA\string_Mdo\#\string_T1}{}
\markboth{\textcolor{darkblue}{\textbf{\ipa{lɑ˧do\#˥}}}}{}
\textcolor{teal}{\zh{名词}} \hspace{4pt} \zh{声调类:} \#H.
\zh{马夫(参加马帮)。} \textcolor{Sepia}{\selectlanguage{english}Horse groom.} \textcolor{PineGreen}{\selectlanguage{french}Palefrenier, caravanier (employé, pas chef de caravane).} 
\lhead{\firstmark}
\rhead{\botmark}

\subsection{\hspace{-0.5cm} {\Large \textcolor{darkblue}{\textbf{\ipa{lɑ˧hwɤ˩}}}}\hspace{0.5cm}[\kern2pt{\textcolor{darkblue}{\textbf{\ipa{lɑ˧hwɤ˩}}}}\kern2pt]} \hypertarget{lA\string_Mhw7\string_B1}{}
\markboth{\textcolor{darkblue}{\textbf{\ipa{lɑ˧hwɤ˩}}}}{}
\textcolor{teal}{\zh{名词}} \hspace{4pt} \zh{声调类:} L\#.
\zh{村落名。} \textcolor{Sepia}{\selectlanguage{english}A Na village outside the Yongning plain, close to the Lake, not far from \textcolor{darkblue}{\textbf{\ipa{/lɑ˧tʰɑ˧-di˧˥/}}}.} \textcolor{PineGreen}{\selectlanguage{french}Village na hors de la plaine de Yongning, vers le Lac, non loin de \textcolor{darkblue}{\textbf{\ipa{/lɑ˧tʰɑ˧-di˧˥/}}}.} 
\lhead{\firstmark}
\rhead{\botmark}

\subsection{\hspace{-0.5cm} {\Large \textcolor{darkblue}{\textbf{\ipa{lɑ˧kɤ˩}}}}\hspace{0.5cm}[\kern2pt{\textcolor{darkblue}{\textbf{\ipa{lɑ˧kɤ˩}}}}\kern2pt]} \hypertarget{lA\string_Mk7\string_B1}{}
\markboth{\textcolor{darkblue}{\textbf{\ipa{lɑ˧kɤ˩}}}}{}
\textcolor{teal}{\zh{名词}} \hspace{4pt} \zh{声调类:} L\#.
\zh{小坛子,用来存酒。} \textcolor{Sepia}{\selectlanguage{english}Small jar used to preserve wine.} \textcolor{PineGreen}{\selectlanguage{french}Petite cruche, petit pot pour l'alcool; sert pour le conserver longtemps, pas seulement pour le verser.}  \zh{量词}: \textcolor{darkblue}{\textbf{\ipa{ɭɯ˧}}} 
\lhead{\firstmark}
\rhead{\botmark}

\subsection{\hspace{-0.5cm} {\Large \textcolor{darkblue}{\textbf{\ipa{lɑ˧kʰv̩˧˥}}}}\hspace{0.5cm}[\kern2pt{\textcolor{darkblue}{\textbf{\ipa{lɑ˧kʰv̩˧˥}}}}\kern2pt]} \hypertarget{lA\string_Mk\string_hv\string_=\string_M\string_T1}{}
\markboth{\textcolor{darkblue}{\textbf{\ipa{lɑ˧kʰv̩˧˥}}}}{}
\textcolor{teal}{\zh{名词}} \hspace{4pt} \zh{声调类:} MH\#.
\zh{虎年。} \textcolor{Sepia}{\selectlanguage{english}Year of the Tiger.} \textcolor{PineGreen}{\selectlanguage{french}Année du Tigre.} 
\lhead{\firstmark}
\rhead{\botmark}

\subsection{\hspace{-0.5cm} {\Large \textcolor{darkblue}{\textbf{\ipa{lɑ˧\textasciitilde{}lɑ˧}}}}\hspace{0.5cm}[\kern2pt{\textcolor{darkblue}{\textbf{\ipa{lɑ˧lɑ˧}}}}\kern2pt]} \hypertarget{lA\string_M~lA\string_M1}{}
\markboth{\textcolor{darkblue}{\textbf{\ipa{lɑ˧\textasciitilde{}lɑ˧}}}}{}
\textcolor{teal}{\zh{形容词}} \hspace{4pt} \zh{声调类:} M.
\zh{松弛。} \textcolor{Sepia}{\selectlanguage{english}Flaccid, flabby.} \textcolor{PineGreen}{\selectlanguage{french}Ballant, flasque.} 
\lhead{\firstmark}
\rhead{\botmark}

\subsection{\hspace{-0.5cm} {\Large \textcolor{darkblue}{\textbf{\ipa{lɑ˧\textasciitilde{}lɑ˧\textsubscript{b}}}}}\hspace{0.5cm}[\kern2pt{\textcolor{darkblue}{\textbf{\ipa{lɑ˧lɑ˧}}}}\kern2pt]} \hypertarget{lA\string_M~lA\string_Mb1}{}
\markboth{\textcolor{darkblue}{\textbf{\ipa{lɑ˧\textasciitilde{}lɑ˧\textsubscript{b}}}}}{}
\textcolor{teal}{\zh{动词}} \hspace{4pt} \zh{声调类:} M\textsubscript{b}.
\zh{掺水。} \textcolor{Sepia}{\selectlanguage{english}To dilute in water.} \textcolor{PineGreen}{\selectlanguage{french}Diluer (dans l’eau).}  ¶ \textcolor{darkblue}{\textbf{\ipa{(dʑɯ˧-qo˧) le˧-lɑ˧\textasciitilde{}lɑ˧}}} \zh{掺水} \textcolor{Sepia}{\selectlanguage{english}to dilute in water} \textcolor{PineGreen}{\selectlanguage{french}diluer dans de l’eau}  

\lhead{\firstmark}
\rhead{\botmark}

\subsection{\hspace{-0.5cm} {\Large \textcolor{darkblue}{\textbf{\ipa{lɑ˧lo˧-ʁwɤ˥}}}}\hspace{0.5cm}[\kern2pt{\textcolor{darkblue}{\textbf{\ipa{xxxx non-correspondance entre le nombre de morphèmes et le nombre de tons de morphèmes}}}}\kern2pt]} \hypertarget{lA\string_Mlo\string_M-Rw7\string_T1}{}
\markboth{\textcolor{darkblue}{\textbf{\ipa{lɑ˧lo˧-ʁwɤ˥}}}}{}
\textcolor{teal}{\zh{名词}} \hspace{4pt} \zh{声调类:} H\#.
\zh{拉洛瓦村(永宁的一个村落)。} \textcolor{Sepia}{\selectlanguage{english}A village of Yongning; Chinese name: Laluowa.} \textcolor{PineGreen}{\selectlanguage{french}Un village de Yongning; prononciation chinoise: Laluowa.}  ¶ \textcolor{darkblue}{\textbf{\ipa{dʑɤ˩bv̩˧kɤ˧-sɑ˥ʁwɤ˩, | hi˩ʁwɤ˩-lo˥, | æ˩mi˧-ʁwɤ\#˥, | lɑ˧lo˧-ʁwɤ˥, | lɑ˧ŋwɤ˧, | bɤ˧tsʰo˧gv̩˥, | ə˧lɑ˧-ʁwɤ\#˥, | gæ˧ɻæ˩, | qʰæ˧tɕʰi˧, | tʰo˧ʈɯ\#˥}}} \zh{摩梭传统地理概念中,属于永宁的十个村落} \textcolor{Sepia}{\selectlanguage{english}the ten villages traditionally considered as part of Yongning} \textcolor{PineGreen}{\selectlanguage{french}les dix villages comptant traditionnellement comme faisant partie de Yongning}  

\lhead{\firstmark}
\rhead{\botmark}

\subsection{\hspace{-0.5cm} {\Large \textcolor{darkblue}{\textbf{\ipa{lɑ˧ɬɑ˧˥}}}}\hspace{0.5cm}[\kern2pt{\textcolor{darkblue}{\textbf{\ipa{lɑ˧ɬɑ˧˥}}}}\kern2pt]} \hypertarget{lA\string_MKA\string_M\string_T1}{}
\markboth{\textcolor{darkblue}{\textbf{\ipa{lɑ˧ɬɑ˧˥}}}}{}
\textcolor{teal}{\zh{连接词}} \hspace{4pt} \zh{声调类:} MH\#.
\zh{这以外。} \textcolor{Sepia}{\selectlanguage{english}Apart from, aside of, other than.} \textcolor{PineGreen}{\selectlanguage{french}À part, en dehors de.}  ¶ \textcolor{darkblue}{\textbf{\ipa{tsɑ˧bɤ˧ mɤ˧-pʰv̩˧ɖɯ˧! | lɑ˧ɬɑ˧˥, | ə˧tso˧-mɤ˧-ɲi˩ | pʰv̩˩ɖɯ˩˥!}}} \zh{面粉不贵。其它的呢,什么都贵!(题目:讲今日永宁食品物价)} \textcolor{Sepia}{\selectlanguage{english}Flour is not expensive; apart from it, everything is expensive! / Flour is cheap; but everything else is expensive! (An observation about the cost of living in early 21st-century Yongning)} \textcolor{PineGreen}{\selectlanguage{french}La farine n'est pas chère; à part ça, tout est cher! (Réflexion au sujet du coût de la vie dans la région aujourd'hui)}  

\lhead{\firstmark}
\rhead{\botmark}

\subsection{\hspace{-0.5cm} {\Large \textcolor{darkblue}{\textbf{\ipa{lɑ˧ɬɑ˧˥}}} \textsubscript{1}}\hspace{0.5cm}[\kern2pt{\textcolor{darkblue}{\textbf{\ipa{lɑ˧ɬɑ˧˥}}}}\kern2pt]} \hypertarget{lA\string_MKA\string_M\string_T1}{}
\markboth{\textcolor{darkblue}{\textbf{\ipa{lɑ˧ɬɑ˧˥}}} \textsubscript{1}}{}
\textcolor{teal}{\zh{代词}} \hspace{4pt} \zh{声调类:} MH\#.
\zh{别的。} \textcolor{Sepia}{\selectlanguage{english}Other.} \textcolor{PineGreen}{\selectlanguage{french}Autre, autres.}  ¶ \textcolor{darkblue}{\textbf{\ipa{lɑ˧ɬɑ˧˥ | ɖɯ˧-tɕi˥}}} \zh{其它一些} \textcolor{Sepia}{\selectlanguage{english}some others, a few others} \textcolor{PineGreen}{\selectlanguage{french}quelques autres}  
 ¶ \textcolor{darkblue}{\textbf{\ipa{lɑ˧ɬɑ˧˥ | ʈʂʰɯ˧-tɕi˩}}} \zh{其它的那些} \textcolor{Sepia}{\selectlanguage{english}those other, those few others, the few that remained} \textcolor{PineGreen}{\selectlanguage{french}ces quelques autres, ceux qui restent}  
 ¶ \textcolor{darkblue}{\textbf{\ipa{lɑ˧ɬɑ˧˥ | ɖɯ˧-ʁo˩ ɲi˩!}}} \zh{是另一回事! / 是另一码事!} \textcolor{Sepia}{\selectlanguage{english}It's something different! / That's a different matter!} \textcolor{PineGreen}{\selectlanguage{french}C'est autre chose! / Ca, c'est différent!}  
\zh{~【参考】~} \hyperlink{}{\textcolor{darkblue}{\textbf{\ipa{lɑ˧ɬɑ˧˥}}} \textsubscript{2}} 
\lhead{\firstmark}
\rhead{\botmark}

\subsection{\hspace{-0.5cm} {\Large \textcolor{darkblue}{\textbf{\ipa{lɑ˧ɬɑ˧˥}}} \textsubscript{2}}\hspace{0.5cm}[\kern2pt{\textcolor{darkblue}{\textbf{\ipa{lɑ˧ɬɑ˧˥}}}}\kern2pt]} \hypertarget{lA\string_MKA\string_M\string_T2}{}
\markboth{\textcolor{darkblue}{\textbf{\ipa{lɑ˧ɬɑ˧˥}}} \textsubscript{2}}{}
\textcolor{teal}{\zh{形容词}} \hspace{4pt} \zh{声调类:} MH\#.
\zh{别的。} \textcolor{Sepia}{\selectlanguage{english}Other.} \textcolor{PineGreen}{\selectlanguage{french}Autre.}  ¶ \textcolor{darkblue}{\textbf{\ipa{lɑ˧ɬɑ˧ hĩ˥}}} \zh{其它人} \textcolor{Sepia}{\selectlanguage{english}other people} \textcolor{PineGreen}{\selectlanguage{french}les autres gens}  
 ¶ \textcolor{darkblue}{\textbf{\ipa{ɖɯ˧-bæ˧ | le˧-se˩, | ɖɯ˧-bæ˧ ʝi˧! / ɖɯ˧-bæ˧ | le˧-se˩, | wɤ˩˥ | lɑ˧ɬɑ˧˥ | ɖɯ˧-bæ˧ ʝi˧! |}}} \zh{做完一件事情,就轮到另一个!} \textcolor{Sepia}{\selectlanguage{english}When one has finished one task, one moves on to another!} \textcolor{PineGreen}{\selectlanguage{french}Quand on a fini une chose/une tâche, on en fait une autre / on passe à une autre!}  
\zh{~【参考】~} \hyperlink{}{\textcolor{darkblue}{\textbf{\ipa{lɑ˧ɬɑ˧˥}}} \textsubscript{1}} 
\lhead{\firstmark}
\rhead{\botmark}

\subsection{\hspace{-0.5cm} {\Large \textcolor{darkblue}{\textbf{\ipa{lɑ˧mɑ˧}}}}\hspace{0.5cm}[\kern2pt{\textcolor{darkblue}{\textbf{\ipa{lɑ˧mɑ˧}}}}\kern2pt]} \hypertarget{lA\string_MmA\string_M1}{}
\markboth{\textcolor{darkblue}{\textbf{\ipa{lɑ˧mɑ˧}}}}{}
\textcolor{teal}{\zh{名词}} \hspace{4pt} \zh{声调类:} M.
\zh{喇嘛。} \textcolor{Sepia}{\selectlanguage{english}Lama.} \textcolor{PineGreen}{\selectlanguage{french}Lama.}  \zh{【借词】}\zh{藏语} bla-ma
 ¶ \textcolor{darkblue}{\textbf{\ipa{hæ˧-lɑ˩mɑ˩}}} \zh{汉族喇嘛} \textcolor{Sepia}{\selectlanguage{english}Chinese lama} \textcolor{PineGreen}{\selectlanguage{french}lama chinois}  
 \zh{量词}: \textcolor{darkblue}{\textbf{\ipa{v̩˧}}} 
\lhead{\firstmark}
\rhead{\botmark}

\subsection{\hspace{-0.5cm} {\Large \textcolor{darkblue}{\textbf{\ipa{lɑ˧mi\#˥}}}}\hspace{0.5cm}[\kern2pt{\textcolor{darkblue}{\textbf{\ipa{lɑ˧mi˧}}}}\kern2pt]} \hypertarget{lA\string_Mmi\#\string_T1}{}
\markboth{\textcolor{darkblue}{\textbf{\ipa{lɑ˧mi\#˥}}}}{}
\textcolor{teal}{\zh{名词}} \hspace{4pt} \zh{声调类:} \#H.
\zh{母老虎。} \textcolor{Sepia}{\selectlanguage{english}Female tiger.} \textcolor{PineGreen}{\selectlanguage{french}Tigresse.}  ¶ \textcolor{darkblue}{\textbf{\ipa{lɑ˧mi˧ tʰv̩˧-mi˧˥ / lɑ˧mi˧ tʰv̩˧-mi˥\#}}} \zh{那只老虎} \textcolor{Sepia}{\selectlanguage{english}\mytextsc{n}+\mytextsc{dem}+\mytextsc{clf}} \textcolor{PineGreen}{\selectlanguage{french}\mytextsc{n}+\mytextsc{dem}+\mytextsc{clf}}  
 \zh{量词}: \textcolor{darkblue}{\textbf{\ipa{pʰo˧˥ / mi˩}}} 
\lhead{\firstmark}
\rhead{\botmark}

\subsection{\hspace{-0.5cm} {\Large \textcolor{darkblue}{\textbf{\ipa{lɑ˧ŋwɤ˧}}}}\hspace{0.5cm}[\kern2pt{\textcolor{darkblue}{\textbf{\ipa{lɑ˧ŋwɤ˧}}}}\kern2pt]} \hypertarget{lA\string_MNw7\string_M1}{}
\markboth{\textcolor{darkblue}{\textbf{\ipa{lɑ˧ŋwɤ˧}}}}{}
\textcolor{teal}{\zh{名词}} \hspace{4pt} \zh{声调类:} M.
\zh{一座山的名字。} \textcolor{Sepia}{\selectlanguage{english}The name of a mountain on the way from Yongning to Wujiao; this name is also used to refer to the hamlets on the slope of this mountain.} \textcolor{PineGreen}{\selectlanguage{french}Nom de montagne, sur le chemin de Yongning à Wujiao; est aussi le nom qui désigne les hameaux qui se trouvent sur cette montagne.}  ¶ \textcolor{darkblue}{\textbf{\ipa{dʑɤ˩bv̩˧kɤ˧-sɑ˥ʁwɤ˩, | hi˩ʁwɤ˩-lo˥, | æ˩mi˧-ʁwɤ\#˥, | lɑ˧lo˧-ʁwɤ˥, | lɑ˧ŋwɤ˧, | bɤ˧tsʰo˧gv̩˥, | ə˧lɑ˧-ʁwɤ\#˥, | gæ˧ɻæ˩, | qʰæ˧tɕʰi˧, | tʰo˧ʈɯ\#˥}}} \zh{摩梭传统地理概念中,属于永宁的十个村落} \textcolor{Sepia}{\selectlanguage{english}the ten villages traditionally considered as part of Yongning} \textcolor{PineGreen}{\selectlanguage{french}les dix villages comptant traditionnellement comme faisant partie de Yongning}  

\lhead{\firstmark}
\rhead{\botmark}

\subsection{\hspace{-0.5cm} {\Large \textcolor{darkblue}{\textbf{\ipa{lɑ˧pʰv̩\#˥}}}}\hspace{0.5cm}[\kern2pt{\textcolor{darkblue}{\textbf{\ipa{lɑ˧pʰv̩˧}}}}\kern2pt]} \hypertarget{lA\string_Mp\string_hv\string_=\#\string_T1}{}
\markboth{\textcolor{darkblue}{\textbf{\ipa{lɑ˧pʰv̩\#˥}}}}{}
\textcolor{teal}{\zh{名词}} \hspace{4pt} \zh{声调类:} \#H.
\zh{公老虎。} \textcolor{Sepia}{\selectlanguage{english}Male tiger.} \textcolor{PineGreen}{\selectlanguage{french}Tigre (mâle).}  ¶ \textcolor{darkblue}{\textbf{\ipa{lɑ˧pʰv̩˧ tʰv̩˧-ɭɯ\#˥}}} \zh{那只老虎} \textcolor{Sepia}{\selectlanguage{english}\mytextsc{n}+\mytextsc{dem}+\mytextsc{clf}} \textcolor{PineGreen}{\selectlanguage{french}\mytextsc{n}+\mytextsc{dem}+\mytextsc{clf}}  
 \zh{量词}: \textcolor{darkblue}{\textbf{\ipa{pʰo˧˥ / ɭɯ˧}}} 
\lhead{\firstmark}
\rhead{\botmark}

\subsection{\hspace{-0.5cm} {\Large \textcolor{darkblue}{\textbf{\ipa{lɑ˧tʰɑ˧-di˧˥}}}}\hspace{0.5cm}[\kern2pt{\textcolor{darkblue}{\textbf{\ipa{xxxx non-correspondance entre le nombre de morphèmes et le nombre de tons de morphèmes}}}}\kern2pt]} \hypertarget{lA\string_Mt\string_hA\string_M-di\string_M\string_T1}{}
\markboth{\textcolor{darkblue}{\textbf{\ipa{lɑ˧tʰɑ˧-di˧˥}}}}{}
\textcolor{teal}{\zh{名词}} \hspace{4pt} \zh{声调类:} MH\#.
\zh{泸沽湖周边的摩梭地区,包括左所(今为泸沽湖镇)、洛水村等。} \textcolor{Sepia}{\selectlanguage{english}The entire Na area around Lugu lake, including Zuosuo (currently Luguhu Zhen) and the village of Luoshui.} \textcolor{PineGreen}{\selectlanguage{french}La région na qui entoure le lac Lugu: Zuosuo (actuel Luguhu Zhen), le village de Luoshui, et les autres localités du bord du Lac.}  ¶ \textcolor{darkblue}{\textbf{\ipa{ɬi˧ki˧, | ɲi˧se˩, | tɑ˧dzi˩, | mv̩˧qʰwæ˩, | lɑ˧tʰɑ˧-di˧˥}}} \zh{从永宁往泸沽湖所经过的村落,依次是:里格、尼赛、大祖、木垮,然后到拉塔地(拉塔地指的是泸沽湖周边的摩梭地区,包括左所、洛水村等)} \textcolor{Sepia}{\selectlanguage{english}Villages that one passes when moving away from the Yongning plain, towards Lugu lake. These villages do not count as part of Yongning proper. The last, \textcolor{darkblue}{\textbf{\ipa{/lɑ˧tʰɑ˧-di˧˥/}}}, is not a village name like the preceding four: it refers to the entire Na area beyond the fourth village.} \textcolor{PineGreen}{\selectlanguage{french}Villages dans l'ordre, après la plaine de Yongning, ne comptant pas comme faisant partie de Yongning. Le dernier, \textcolor{darkblue}{\textbf{\ipa{/lɑ˧tʰɑ˧-di˧˥/}}}, désigne toute la région na au-delà du quatrième village.}  

\lhead{\firstmark}
\rhead{\botmark}

\subsection{\hspace{-0.5cm} {\Large \textcolor{darkblue}{\textbf{\ipa{lɑ˧tʰɑ˧mi˥\$}}}}\hspace{0.5cm}[\kern2pt{\textcolor{darkblue}{\textbf{\ipa{lɑ˧tʰɑ˧mi˥}}}}\kern2pt]} \hypertarget{lA\string_Mt\string_hA\string_Mmi\string_T\$1}{}
\markboth{\textcolor{darkblue}{\textbf{\ipa{lɑ˧tʰɑ˧mi˥\$}}}}{}
\textcolor{teal}{\zh{名词}} \hspace{4pt} \zh{声调类:} H\$.
\zh{一个姓。这个姓,永宁有五个家。音译:拉他咪。} \textcolor{Sepia}{\selectlanguage{english}A family name from Yongning. There are five families in Yongning that carry this name. This is one of the first three clans who settled in the vicinity of the Yongning monastery, the other two being \textcolor{darkblue}{\textbf{\ipa{/kɤ˧˥tʰɑ˩/}}} and \textcolor{darkblue}{\textbf{\ipa{/ə˧lɑ˧/}}}.} \textcolor{PineGreen}{\selectlanguage{french}Nom de clan/famille étendue. Cinq familles portent ce nom à Yongning. C'est l'un des trois premiers clans à s'être établis à proximité du monastère de Yongning, les deux autres étant \textcolor{darkblue}{\textbf{\ipa{/kɤ˧˥tʰɑ˩/}}} et \textcolor{darkblue}{\textbf{\ipa{/ə˧lɑ˧/}}}.}  ¶ \textcolor{darkblue}{\textbf{\ipa{lɑ˧tʰɑ˧mi˧=ɻ̍˥\$}}} \zh{\textcolor{darkblue}{\textbf{\ipa{/lɑ˧tʰɑ˧mi˥\$/}}}家族} \textcolor{Sepia}{\selectlanguage{english}the \textcolor{darkblue}{\textbf{\ipa{/lɑ˧tʰɑ˧mi˥\$/}}} clan, the \textcolor{darkblue}{\textbf{\ipa{/lɑ˧tʰɑ˧mi˥\$/}}} family} \textcolor{PineGreen}{\selectlanguage{french}le clan \textcolor{darkblue}{\textbf{\ipa{/lɑ˧tʰɑ˧mi˥\$/}}}, la famille \textcolor{darkblue}{\textbf{\ipa{/lɑ˧tʰɑ˧mi˥\$/}}}}  

\lhead{\firstmark}
\rhead{\botmark}

\subsection{\hspace{-0.5cm} {\Large \textcolor{darkblue}{\textbf{\ipa{lɑ˧tʰɑ˧mi˥-ʈæ˧ʂɯ˧-lɑ˩mv̩˩}}}}\hspace{0.5cm}[\kern2pt{\textcolor{darkblue}{\textbf{\ipa{lɑ˧tʰɑ˧mi˥ʈæ˩ʂɯ˩lɑ˩mv̩˩}}}}\kern2pt]} \hypertarget{lA\string_Mt\string_hA\string_Mmi\string_T-t`\{\string_Ms`M\string_M-lA\string_Bmv\string_=\string_B1}{}
\markboth{\textcolor{darkblue}{\textbf{\ipa{lɑ˧tʰɑ˧mi˥-ʈæ˧ʂɯ˧-lɑ˩mv̩˩}}}}{}
\textcolor{teal}{\zh{名词}} \hspace{4pt} \zh{声调类:} H\#-M-L.
\zh{拉他咪•达石拉么:本著作的标准发音合作人。} \textcolor{Sepia}{\selectlanguage{english}Proper name of the main consultant (reference speaker) for this volume (speaker code: F4).} \textcolor{PineGreen}{\selectlanguage{french}Nom propre (nom de famille et prénom) de la consultante de référence du présent travail (code locutrice: F4).} 
\lhead{\firstmark}
\rhead{\botmark}

\subsection{\hspace{-0.5cm} {\Large \textcolor{darkblue}{\textbf{\ipa{lɑ˧zi˥}}}}\hspace{0.5cm}[\kern2pt{\textcolor{darkblue}{\textbf{\ipa{lɑ˧zi˥}}}}\kern2pt]} \hypertarget{lA\string_Mzi\string_T1}{}
\markboth{\textcolor{darkblue}{\textbf{\ipa{lɑ˧zi˥}}}}{}
\textcolor{teal}{\zh{名词}} \hspace{4pt} \zh{声调类:} H\#.
\zh{画家。} \textcolor{Sepia}{\selectlanguage{english}Painter.} \textcolor{PineGreen}{\selectlanguage{french}Peintre (activité qui n'est pas réservée aux moines).}  ¶ \textcolor{darkblue}{\textbf{\ipa{ʈʂʰɯ˧-v̩˧, | lɑ˧zi˥ ɲi˩!}}} \zh{他是画家!} \textcolor{Sepia}{\selectlanguage{english}(S)he is a painter! / (S)he can paint!} \textcolor{PineGreen}{\selectlanguage{french}elle/il est peintre! / elle/il sait peindre!}  

\lhead{\firstmark}
\rhead{\botmark}

\subsection{\hspace{-0.5cm} {\Large \textcolor{darkblue}{\textbf{\ipa{lɑ˧zo\#˥}}}}\hspace{0.5cm}[\kern2pt{\textcolor{darkblue}{\textbf{\ipa{lɑ˧zo˧}}}}\kern2pt]} \hypertarget{lA\string_Mzo\#\string_T1}{}
\markboth{\textcolor{darkblue}{\textbf{\ipa{lɑ˧zo\#˥}}}}{}
\textcolor{teal}{\zh{名词}} \hspace{4pt} \zh{声调类:} \#H.
\zh{小老虎。} \textcolor{Sepia}{\selectlanguage{english}Baby tiger, child of tiger.} \textcolor{PineGreen}{\selectlanguage{french}Petit tigre.}  ¶ \textcolor{darkblue}{\textbf{\ipa{lɑ˧zo˧ tʰv̩˧-ɭɯ\#˥}}} \zh{那只小老虎} \textcolor{Sepia}{\selectlanguage{english}\mytextsc{n}+\mytextsc{dem}+\mytextsc{clf}} \textcolor{PineGreen}{\selectlanguage{french}\mytextsc{n}+\mytextsc{dem}+\mytextsc{clf}}  
 \zh{量词}: \textcolor{darkblue}{\textbf{\ipa{ɭɯ˧}}} 
\lhead{\firstmark}
\rhead{\botmark}

\subsection{\hspace{-0.5cm} {\Large \textcolor{darkblue}{\textbf{\ipa{lɑ˩gv̩˧}}}}\hspace{0.5cm}[\kern2pt{\textcolor{darkblue}{\textbf{\ipa{lɑ˩gv̩˥}}}}\kern2pt]} \hypertarget{lA\string_Bgv\string_=\string_M1}{}
\markboth{\textcolor{darkblue}{\textbf{\ipa{lɑ˩gv̩˧}}}}{}
\textcolor{teal}{\zh{形容词}} \hspace{4pt} \zh{声调类:} LM.
\zh{弯(树...)。} \textcolor{Sepia}{\selectlanguage{english}Curved, crooked, bent (e.g. tree).} \textcolor{PineGreen}{\selectlanguage{french}Recourbé, tordu, courbe.}  ¶ \textcolor{darkblue}{\textbf{\ipa{si˧dzi˩ | lɑ˩-gv̩˧-ze˩}}} \zh{树弯了。} \textcolor{Sepia}{\selectlanguage{english}The tree got crooked.} \textcolor{PineGreen}{\selectlanguage{french}L'arbre est devenu courbé.}  

\lhead{\firstmark}
\rhead{\botmark}

\subsection{\hspace{-0.5cm} {\Large \textcolor{darkblue}{\textbf{\ipa{lɑ˩gv̩˧-lɑ˩ɲi˩}}}}\hspace{0.5cm}[\kern2pt{\textcolor{darkblue}{\textbf{\ipa{lɑ˩gv̩˧lɑ˩ɲi˩}}}}\kern2pt]} \hypertarget{lA\string_Bgv\string_=\string_M-lA\string_BJi\string_B1}{}
\markboth{\textcolor{darkblue}{\textbf{\ipa{lɑ˩gv̩˧-lɑ˩ɲi˩}}}}{}
\textcolor{teal}{\zh{形容词}} \hspace{4pt} \zh{声调类:} LM-L.
\zh{弯(路,植物,人的四肢)。} \textcolor{Sepia}{\selectlanguage{english}Crooked, curved, bent (e.g. road, person's limbs).} \textcolor{PineGreen}{\selectlanguage{french}Tout tordu, tout recourbé.} 
\lhead{\firstmark}
\rhead{\botmark}

\subsection{\hspace{-0.5cm} {\Large \textcolor{darkblue}{\textbf{\ipa{lɑ˩jɤ˧-ɬi˧}}}}\hspace{0.5cm}[\kern2pt{\textcolor{darkblue}{\textbf{\ipa{lɑ˩jɤ˧ɬi˧}}}}\kern2pt]} \hypertarget{lA\string_Bj7\string_M-Ki\string_M1}{}
\markboth{\textcolor{darkblue}{\textbf{\ipa{lɑ˩jɤ˧-ɬi˧}}}}{}
\textcolor{teal}{\zh{名词}} \hspace{4pt} \zh{声调类:} LM-.
\zh{十二月。} \textcolor{Sepia}{\selectlanguage{english}12th month.} \textcolor{PineGreen}{\selectlanguage{french}Le douzième mois.} 
\lhead{\firstmark}
\rhead{\botmark}

\subsection{\hspace{-0.5cm} {\Large \textcolor{darkblue}{\textbf{\ipa{lɑ˩\textasciitilde{}lɑ˧˥}}}}\hspace{0.5cm}[\kern2pt{\textcolor{darkblue}{\textbf{\ipa{lɑ˧lɑ˧˥}}}}\kern2pt]} \hypertarget{lA\string_B~lA\string_M\string_T1}{}
\markboth{\textcolor{darkblue}{\textbf{\ipa{lɑ˩\textasciitilde{}lɑ˧˥}}}}{}
\textcolor{teal}{\zh{动词}} \hspace{4pt} \zh{声调类:} MH.
\zh{打架、吵架。} \textcolor{Sepia}{\selectlanguage{english}To fight, to scuffle, to come to blows.} \textcolor{PineGreen}{\selectlanguage{french}Se disputer, se battre.}  ¶ \textcolor{darkblue}{\textbf{\ipa{lɑ˩lɑ˧-hĩ˥ | ʈʂʰɯ˧-tɕi˩}}} \zh{打架的这些(人)} \textcolor{Sepia}{\selectlanguage{english}those people who are fighting} \textcolor{PineGreen}{\selectlanguage{french}ces (gens) qui se disputent}  
\zh{~【参考】~} \textcolor{darkblue}{\textbf{\ipa{lɑ˧˥}}} 
\lhead{\firstmark}
\rhead{\botmark}

\subsection{\hspace{-0.5cm} {\Large \textcolor{darkblue}{\textbf{\ipa{lɑ˩mɑ˩}}}}\hspace{0.5cm}[\kern2pt{\textcolor{darkblue}{\textbf{\ipa{lɑ˩mɑ˩˥}}}}\kern2pt]} \hypertarget{lA\string_BmA\string_B1}{}
\markboth{\textcolor{darkblue}{\textbf{\ipa{lɑ˩mɑ˩}}}}{}
\textcolor{teal}{\zh{名词}} \hspace{4pt} \zh{声调类:} L.
\zh{一个姓。这个姓,永宁有四个家。} \textcolor{Sepia}{\selectlanguage{english}A family name from Yongning. There are four families in Yongning that carry this name.} \textcolor{PineGreen}{\selectlanguage{french}Nom de clan/famille étendue. Quatre familles portent ce nom à Yongning.}  ¶ \textcolor{darkblue}{\textbf{\ipa{lɑ˩mɑ˩=ɻ̍˥\$}}} \zh{\textcolor{darkblue}{\textbf{\ipa{/lɑ˩mɑ˩/}}}家族} \textcolor{Sepia}{\selectlanguage{english}the \textcolor{darkblue}{\textbf{\ipa{/lɑ˩mɑ˩/}}} clan, the \textcolor{darkblue}{\textbf{\ipa{/lɑ˩mɑ˩/}}} family} \textcolor{PineGreen}{\selectlanguage{french}le clan \textcolor{darkblue}{\textbf{\ipa{/lɑ˩mɑ˩/}}}, la famille \textcolor{darkblue}{\textbf{\ipa{/lɑ˩mɑ˩/}}}}  
 ¶ \textcolor{darkblue}{\textbf{\ipa{lɑ˩mɑ˩-gv̩˥mɑ˩}}} \zh{一个人的名字:姓为\textcolor{darkblue}{\textbf{\ipa{/lɑ˩mɑ˩/}}},名为\textcolor{darkblue}{\textbf{\ipa{/gv̩˧mɑ˧/}}}} \textcolor{Sepia}{\selectlanguage{english}the name of a person, containing both a family name: \textcolor{darkblue}{\textbf{\ipa{/lɑ˩mɑ˩/}}}, and a given name: \textcolor{darkblue}{\textbf{\ipa{/gv̩˧mɑ˧/}}}} \textcolor{PineGreen}{\selectlanguage{french}nom d'une personne, comportant un nom de famille (\textcolor{darkblue}{\textbf{\ipa{/lɑ˩mɑ˩/}}}) et un prénom (\textcolor{darkblue}{\textbf{\ipa{/gv̩˧mɑ˧/}}})}  

\lhead{\firstmark}
\rhead{\botmark}

\subsection{\hspace{-0.5cm} {\Large \textcolor{darkblue}{\textbf{\ipa{lɑ˩tɑ˧}}}}\hspace{0.5cm}[\kern2pt{\textcolor{darkblue}{\textbf{\ipa{lɑ˩tɑ˥}}}}\kern2pt]} \hypertarget{lA\string_BtA\string_M1}{}
\markboth{\textcolor{darkblue}{\textbf{\ipa{lɑ˩tɑ˧}}}}{}
\textcolor{teal}{\zh{形容词}} \hspace{4pt} \zh{声调类:} LM.
\zh{歪,偏 (帽子戴得歪)。} \textcolor{Sepia}{\selectlanguage{english}Askew, slanting (e.g. hat).} \textcolor{PineGreen}{\selectlanguage{french}De biais, de travers (ex.: porter son chapeau de travers).} 
\lhead{\firstmark}
\rhead{\botmark}

\subsection{\hspace{-0.5cm} {\Large \textcolor{darkblue}{\textbf{\ipa{-lɑ˩tɑ˩}}}}\hspace{0.5cm}[\kern2pt{\textcolor{darkblue}{\textbf{\ipa{lɑ˩tɑ˩˥}}}}\kern2pt]} \hypertarget{-lA\string_BtA\string_B1}{}
\markboth{\textcolor{darkblue}{\textbf{\ipa{-lɑ˩tɑ˩}}}}{}
\textcolor{teal}{\zh{后置词}} \hspace{4pt} \zh{声调类:} L.
\textcolor{Sepia}{\selectlanguage{english}Close to.} \textcolor{PineGreen}{\selectlanguage{french}À proximité de.}  ¶ \textcolor{darkblue}{\textbf{\ipa{ɑ˩ʁo˧ | -lɑ˩tɑ˩˥}}} \zh{家的面积} \textcolor{Sepia}{\selectlanguage{english}the perimeter of the house: the surface on which the house (farm) extends} \textcolor{PineGreen}{\selectlanguage{french}le périmètre de la maison, là où s'étend le domaine de la maison}  

\lhead{\firstmark}
\rhead{\botmark}

\subsection{\hspace{-0.5cm} {\Large \textcolor{darkblue}{\textbf{\ipa{lɑ˩ʈʂv̩˩}}}}\hspace{0.5cm}[\kern2pt{\textcolor{darkblue}{\textbf{\ipa{lɑ˩ʈʂv̩˩˥}}}}\kern2pt]} \hypertarget{lA\string_Bt`s`v\string_=\string_B1}{}
\markboth{\textcolor{darkblue}{\textbf{\ipa{lɑ˩ʈʂv̩˩}}}}{}
\textcolor{teal}{\zh{名词}} \hspace{4pt} \zh{声调类:} L.
\zh{蜡烛。} \textcolor{Sepia}{\selectlanguage{english}Candle.} \textcolor{PineGreen}{\selectlanguage{french}Bougie.}  \zh{【借词】} \zh{蜡烛}
 \zh{量词}: \textcolor{darkblue}{\textbf{\ipa{kɤ˧˥}}} 
\lhead{\firstmark}
\rhead{\botmark}

\subsection{\hspace{-0.5cm} {\Large \textcolor{darkblue}{\textbf{\ipa{lɑ˧˥}}} \textsubscript{1}}\hspace{0.5cm}[\kern2pt{\textcolor{darkblue}{\textbf{\ipa{lɑ˧˥}}}}\kern2pt]} \hypertarget{lA\string_M\string_T1}{}
\markboth{\textcolor{darkblue}{\textbf{\ipa{lɑ˧˥}}} \textsubscript{1}}{}
\textcolor{teal}{\zh{动词}} \hspace{4pt} \zh{声调类:} MH.
\zh{打(打人,钉钉子……)。} \textcolor{Sepia}{\selectlanguage{english}To strike someone, to beat someone.} \textcolor{PineGreen}{\selectlanguage{french}Battre quelque chose, frapper quelque chose, enfoncer un clou, casser des cailloux; donner (une injonction…).}  ¶ \textcolor{darkblue}{\textbf{\ipa{hĩ˧ lɑ˩}}} \zh{打人} \textcolor{Sepia}{\selectlanguage{english}to strike someone} \textcolor{PineGreen}{\selectlanguage{french}frapper quelqu'un}  
 ¶ \textcolor{darkblue}{\textbf{\ipa{hɑ˧ lɑ˩}}} \zh{打粮食} \textcolor{Sepia}{\selectlanguage{english}to beat the grain} \textcolor{PineGreen}{\selectlanguage{french}battre le grain}  
 ¶ \textcolor{darkblue}{\textbf{\ipa{nv̩˩ɭɯ˧ lɑ˧}}} \zh{打大豆} \textcolor{Sepia}{\selectlanguage{english}to beat soy} \textcolor{PineGreen}{\selectlanguage{french}battre les cosses de soja}  
 ¶ \textcolor{darkblue}{\textbf{\ipa{sɯ˩tʰi˩-po˥-ɳɯ˩ | lɑ˧˥}}} \zh{用刀子来砍(沱茶、茶饼)} \textcolor{Sepia}{\selectlanguage{english}to break with a knife (brick tea: compressed tea leaves)} \textcolor{PineGreen}{\selectlanguage{french}casser au moyen d'un couteau (du thé compressé en galettes ou en briques, à l'ancienne)}  
 ¶ \textcolor{darkblue}{\textbf{\ipa{ə˧ʝi˧-ʂɯ˥ʝi˩, | ɬi˧di˩-dʑo˩, | æ˧ lɑ˩-hĩ˩ F | dʑo˩˥! | ʂe˧ lɑ˧-hĩ˥ F | dʑo˩˥! | hæ̃˩ lɑ˩-hĩ˥ F | dʑo˩˥! | ŋv̩˩ lɑ˩-hĩ˥ F | dʑo˩˥!}}} \zh{过去,在永宁,有铜匠、铁匠、金匠、银匠。} \textcolor{Sepia}{\selectlanguage{english}In the past, in Yongning, there were craftsmen who forged copper! craftsmen who forged iron! craftsmen who forged gold! and craftsmen who forged silver!} \textcolor{PineGreen}{\selectlanguage{french}Autrefois, à Yongning, il y avait des artisans qui travaillaient le cuivre! Il y avait des artisans qui travaillaient le fer! Il y avait des artisans qui travaillaient l'or! Il y avait des artisans qui travaillaient l'argent!}  
 ¶ \textcolor{darkblue}{\textbf{\ipa{ə˧ʝi˧-ʂɯ˥ʝi˩, | ɬi˧di˩-dʑo˩, | æ˧ lɑ˩-hĩ˩ dʑo˩, | ʂe˧ lɑ˧-hĩ˥ dʑo˩, | hæ̃˩ lɑ˩-hĩ˥ dʑo˩, | ŋv̩˩ lɑ˩-hĩ˥ dʑo˩.}}} \zh{过去,在永宁,有铜匠、铁匠、金匠、银匠。} \textcolor{Sepia}{\selectlanguage{english}In the past, in Yongning, there were craftsmen who forged copper; craftsmen who forged iron; craftsmen who forged gold; and craftsmen who forged silver.} \textcolor{PineGreen}{\selectlanguage{french}Autrefois, à Yongning, il y avait des artisans qui travaillaient le cuivre. Il y avait des artisans qui travaillaient le fer. Il y avait des artisans qui travaillaient l'or. Il y avait des artisans qui travaillaient l'argent.}  
\zh{~【参考】~} \hyperlink{}{\textcolor{darkblue}{\textbf{\ipa{lɑ˩\textasciitilde{}lɑ˧˥}}}} 
\lhead{\firstmark}
\rhead{\botmark}

\subsection{\hspace{-0.5cm} {\Large \textcolor{darkblue}{\textbf{\ipa{lɑ˧˥}}} \textsubscript{2}}\hspace{0.5cm}[\kern2pt{\textcolor{darkblue}{\textbf{\ipa{lɑ˧˥}}}}\kern2pt]} \hypertarget{lA\string_M\string_T2}{}
\markboth{\textcolor{darkblue}{\textbf{\ipa{lɑ˧˥}}} \textsubscript{2}}{}
\textcolor{teal}{\zh{动词}} \hspace{4pt} \zh{声调类:} MH.
\zh{有,结(露水)。} \textcolor{Sepia}{\selectlanguage{english}To form, to be there, to have appeared (dew).} \textcolor{PineGreen}{\selectlanguage{french}Apparaître, y avoir (de la rosée).}  ¶ \textcolor{darkblue}{\textbf{\ipa{ɖʐv̩˧ lɑ˧˥}}} \zh{结露水了。} \textcolor{Sepia}{\selectlanguage{english}Some dew has appeared; there is some dew} \textcolor{PineGreen}{\selectlanguage{french}Il y a de la rosée; de la rosée s'est formée}  
 ¶ \textcolor{darkblue}{\textbf{\ipa{ɖʐv̩˧qʰɑ˧ lɑ˧˥}}} \zh{结露水了。} \textcolor{Sepia}{\selectlanguage{english}Some dew has appeared; there is some dew} \textcolor{PineGreen}{\selectlanguage{french}Il y a de la rosée; de la rosée s'est formée}  

\lhead{\firstmark}
\rhead{\botmark}

\subsection{\hspace{-0.5cm} {\Large \textcolor{darkblue}{\textbf{\ipa{‑læ˧}}}}\hspace{0.5cm}[\kern2pt{\textcolor{darkblue}{\textbf{\ipa{læ˥}}}}\kern2pt]} \hypertarget{‑l\{\string_M1}{}
\markboth{\textcolor{darkblue}{\textbf{\ipa{‑læ˧}}}}{}
\textcolor{teal}{\zh{后缀}} \hspace{4pt} \zh{声调类:} M.
\zh{\mytextsc{主题:……的话、关于……。}} \textcolor{Sepia}{\selectlanguage{english}This \mytextsc{top} marker introduces a new element, without necessarily contrasting it with others. Possible gloss: concerning… .} \textcolor{PineGreen}{\selectlanguage{french}Topique, introduisant un élément nouveau, pas nécessairement en contraste avec ce qui précède. Gloses possibles: pour ce qui est de, en ce qui concerne, quant à.}  ¶ \textcolor{darkblue}{\textbf{\ipa{ɖɯ˩mɑ˧ | -læ˧…}}} \zh{关于独妈呢,……} \textcolor{Sepia}{\selectlanguage{english}Concerning (my granddaughter) \textcolor{darkblue}{\textbf{\ipa{ɖɯ˩mɑ˧}}}, …} \textcolor{PineGreen}{\selectlanguage{french}pour ce qui est de ma petite-fille \textcolor{darkblue}{\textbf{\ipa{ɖɯ˩mɑ˧}}}, eh bien…}  
 ¶ \textcolor{darkblue}{\textbf{\ipa{lɑ˩mv̩˩˥ | -læ˧...}}} \zh{关于拉姆呢,……} \textcolor{Sepia}{\selectlanguage{english}Concerning \textcolor{darkblue}{\textbf{\ipa{lɑ˩mv̩˩˥}}} [a given name], ...} \textcolor{PineGreen}{\selectlanguage{french}pour ce qui est de \textcolor{darkblue}{\textbf{\ipa{lɑ˩mv̩˩˥}}} [nom propre], ...}  
 ¶ \textcolor{darkblue}{\textbf{\ipa{ti˧ɖo˥ | -læ˧…}}} \zh{关于\textcolor{darkblue}{\textbf{\ipa{ti˧ɖo˥}}}[人的名字]呢,……} \textcolor{Sepia}{\selectlanguage{english}Concerning \textcolor{darkblue}{\textbf{\ipa{ti˧ɖo˥}}} [a given name], …} \textcolor{PineGreen}{\selectlanguage{french}pour ce qui est de \textcolor{darkblue}{\textbf{\ipa{ti˧ɖo˥}}} [nom propre], …}  

\lhead{\firstmark}
\rhead{\botmark}

\subsection{\hspace{-0.5cm} {\Large \textcolor{darkblue}{\textbf{\ipa{læ˧dæ˧qæ˥}}}}\hspace{0.5cm}[\kern2pt{\textcolor{darkblue}{\textbf{\ipa{læ˧dæ˧qæ˥}}}}\kern2pt]} \hypertarget{l\{\string_Md\{\string_Mq\{\string_T1}{}
\markboth{\textcolor{darkblue}{\textbf{\ipa{læ˧dæ˧qæ˥}}}}{}
\textcolor{teal}{\zh{名词}} \hspace{4pt} \zh{声调类:} H\#.
\zh{腋下。} \textcolor{Sepia}{\selectlanguage{english}Armpit.} \textcolor{PineGreen}{\selectlanguage{french}Aisselle.} \zh{当地汉语方言:}\zh{膈肢窝。} \zh{量词}: \textcolor{darkblue}{\textbf{\ipa{ɭɯ˧}}} 
\lhead{\firstmark}
\rhead{\botmark}

\subsection{\hspace{-0.5cm} {\Large \textcolor{darkblue}{\textbf{\ipa{læ˧ʁæ˥\$}}}}\hspace{0.5cm}[\kern2pt{\textcolor{darkblue}{\textbf{\ipa{læ˧ʁæ˥}}}}\kern2pt]} \hypertarget{l\{\string_MR\{\string_T\$1}{}
\markboth{\textcolor{darkblue}{\textbf{\ipa{læ˧ʁæ˥\$}}}}{}
\textcolor{teal}{\zh{名词}} \hspace{4pt} \zh{声调类:} H\$.
\zh{乌鸦。} \textcolor{Sepia}{\selectlanguage{english}Raven.} \textcolor{PineGreen}{\selectlanguage{french}Corbeau.}  \zh{量词}: \textcolor{darkblue}{\textbf{\ipa{mi˩}}} 
\lhead{\firstmark}
\rhead{\botmark}

\subsection{\hspace{-0.5cm} {\Large \textcolor{darkblue}{\textbf{\ipa{læ˧ʁæ˧mi˥\$}}}}\hspace{0.5cm}[\kern2pt{\textcolor{darkblue}{\textbf{\ipa{læ˧ʁæ˧mi˥}}}}\kern2pt]} \hypertarget{l\{\string_MR\{\string_Mmi\string_T\$1}{}
\markboth{\textcolor{darkblue}{\textbf{\ipa{læ˧ʁæ˧mi˥\$}}}}{}
\textcolor{teal}{\zh{名词}} \hspace{4pt} \zh{声调类:} H\$.
\zh{母乌鸦。} \textcolor{Sepia}{\selectlanguage{english}Female raven.} \textcolor{PineGreen}{\selectlanguage{french}Corbeau femelle.}  \zh{量词}: \textcolor{darkblue}{\textbf{\ipa{mi˩}}} 
\lhead{\firstmark}
\rhead{\botmark}

\subsection{\hspace{-0.5cm} {\Large \textcolor{darkblue}{\textbf{\ipa{læ˧ʁæ˧-pʰv̩\#˥}}}}\hspace{0.5cm}[\kern2pt{\textcolor{darkblue}{\textbf{\ipa{xxxx non-correspondance entre le nombre de morphèmes et le nombre de tons de morphèmes}}}}\kern2pt]} \hypertarget{l\{\string_MR\{\string_M-p\string_hv\string_=\#\string_T1}{}
\markboth{\textcolor{darkblue}{\textbf{\ipa{læ˧ʁæ˧-pʰv̩\#˥}}}}{}
\textcolor{teal}{\zh{名词}} \hspace{4pt} \zh{声调类:} \#H.
\zh{公乌鸦。} \textcolor{Sepia}{\selectlanguage{english}Male raven.} \textcolor{PineGreen}{\selectlanguage{french}Corbeau mâle.}  ¶ \textcolor{darkblue}{\textbf{\ipa{læ˧ʁæ˧-pʰv̩˧ tʰv̩˧-mi˥\$}}} \zh{那只公乌鸦} \textcolor{Sepia}{\selectlanguage{english}\mytextsc{n}+\mytextsc{dem}+\mytextsc{clf}} \textcolor{PineGreen}{\selectlanguage{french}\mytextsc{n}+\mytextsc{dem}+\mytextsc{clf}}  
 \zh{量词}: \textcolor{darkblue}{\textbf{\ipa{mi˩}}} 
\lhead{\firstmark}
\rhead{\botmark}

\subsection{\hspace{-0.5cm} {\Large \textcolor{darkblue}{\textbf{\ipa{læ˧tsɯ˥}}}}\hspace{0.5cm}[\kern2pt{\textcolor{darkblue}{\textbf{\ipa{læ˧tsɯ˥}}}}\kern2pt]} \hypertarget{l\{\string_MtsM\string_T1}{}
\markboth{\textcolor{darkblue}{\textbf{\ipa{læ˧tsɯ˥}}}}{}
\textcolor{teal}{\zh{名词}} \hspace{4pt} \zh{声调类:} H\#.
\zh{辣椒(汉语借词:辣子)。} \textcolor{Sepia}{\selectlanguage{english}Chilly peppers.} \textcolor{PineGreen}{\selectlanguage{french}Piment.} \zh{当地汉语方言:}\zh{辣子。} \zh{【借词】} \zh{辣子}
 ¶ \textcolor{darkblue}{\textbf{\ipa{læ˧tsɯ˥ hṽ˩\textasciitilde{}hṽ˩}}} \zh{炒辣椒} \textcolor{Sepia}{\selectlanguage{english}to fry chilly peppers} \textcolor{PineGreen}{\selectlanguage{french}frire des piments}  
 \zh{量词}: \textcolor{darkblue}{\textbf{\ipa{ɭɯ˧}}} 
\lhead{\firstmark}
\rhead{\botmark}

\subsection{\hspace{-0.5cm} {\Large \textcolor{darkblue}{\textbf{\ipa{le˧‑}}}}\hspace{0.5cm}[\kern2pt{\textcolor{darkblue}{\textbf{\ipa{le˥}}}}\kern2pt]} \hypertarget{le\string_M‑1}{}
\markboth{\textcolor{darkblue}{\textbf{\ipa{le˧‑}}}}{}
\textcolor{teal}{\zh{前缀}} \hspace{4pt} \zh{声调类:} M/0.
\zh{\mytextsc{实施。}} \textcolor{Sepia}{\selectlanguage{english}\mytextsc{accomplished} aspect.} \textcolor{PineGreen}{\selectlanguage{french}\mytextsc{accomp}.} 
\lhead{\firstmark}
\rhead{\botmark}

\subsection{\hspace{-0.5cm} {\Large \textcolor{darkblue}{\textbf{\ipa{le˧-tɑ˧˥}}}}\hspace{0.5cm}[\kern2pt{\textcolor{darkblue}{\textbf{\ipa{xxxx non-correspondance entre le nombre de morphèmes et le nombre de tons de morphèmes}}}}\kern2pt]} \hypertarget{le\string_M-tA\string_M\string_T1}{}
\markboth{\textcolor{darkblue}{\textbf{\ipa{le˧-tɑ˧˥}}}}{}
\textcolor{teal}{\zh{连接词}} \hspace{4pt} \zh{声调类:} MH.
\zh{到……为止、一直到……、连……。} \textcolor{Sepia}{\selectlanguage{english}Up to, all the way to; even.} \textcolor{PineGreen}{\selectlanguage{french}Jusqu'à; même.} 
\lhead{\firstmark}
\rhead{\botmark}

\subsection{\hspace{-0.5cm} {\Large \textcolor{darkblue}{\textbf{\ipa{le˧-wo˥}}}}\hspace{0.5cm}[\kern2pt{\textcolor{darkblue}{\textbf{\ipa{xxxx non-correspondance entre le nombre de morphèmes et le nombre de tons de morphèmes}}}}\kern2pt]} \hypertarget{le\string_M-wo\string_T1}{}
\markboth{\textcolor{darkblue}{\textbf{\ipa{le˧-wo˥}}}}{}
\textcolor{teal}{\zh{助词}} \hspace{4pt} \zh{声调类:} H\#.
\zh{再、又、重新。} \textcolor{Sepia}{\selectlanguage{english}Over again, once over again; back.} \textcolor{PineGreen}{\selectlanguage{french}À nouveau, de nouveau.} 
\lhead{\firstmark}
\rhead{\botmark}

\subsection{\hspace{-0.5cm} {\Large \textcolor{darkblue}{\textbf{\ipa{le˧-wo˧}}}}\hspace{0.5cm}[\kern2pt{\textcolor{darkblue}{\textbf{\ipa{xxxx non-correspondance entre le nombre de morphèmes et le nombre de tons de morphèmes}}}}\kern2pt]} \hypertarget{le\string_M-wo\string_M1}{}
\markboth{\textcolor{darkblue}{\textbf{\ipa{le˧-wo˧}}}}{}
\textcolor{teal}{\zh{助词}} \hspace{4pt} \zh{声调类:} M.
\zh{又,……回去。} \textcolor{Sepia}{\selectlanguage{english}Again; back.} \textcolor{PineGreen}{\selectlanguage{french}À nouveau.}  ¶ \textcolor{darkblue}{\textbf{\ipa{le˧-wo˧ jo˧}}} \zh{回} \textcolor{Sepia}{\selectlanguage{english}to come back} \textcolor{PineGreen}{\selectlanguage{french}revenir}  
 ¶ \textcolor{darkblue}{\textbf{\ipa{le˧-wo˧ le˧-gv̩˩}}} \zh{从头开始} \textcolor{Sepia}{\selectlanguage{english}to do over again} \textcolor{PineGreen}{\selectlanguage{french}recommencer}  
 ¶ \textcolor{darkblue}{\textbf{\ipa{le˧-wo˧ le˧-gv̩˧\textasciitilde{}gv̩˥}}} \zh{重新做、重新建} \textcolor{Sepia}{\selectlanguage{english}to build anew, to make anew, to rebuild} \textcolor{PineGreen}{\selectlanguage{french}refaire, réinstaller, reconstruire}  
 ¶ \textcolor{darkblue}{\textbf{\ipa{le˧-wo˧ le˥-tɕo˩ ʐwɤ˩}}} \zh{重复讲说过的话} \textcolor{Sepia}{\selectlanguage{english}to speak over and over again, to rant, to repeat ceaselessly} \textcolor{PineGreen}{\selectlanguage{french}répéter sans arrêt}  

\lhead{\firstmark}
\rhead{\botmark}

\subsection{\hspace{-0.5cm} {\Large \textcolor{darkblue}{\textbf{\ipa{le˩}}}}\hspace{0.5cm}[\kern2pt{\textcolor{darkblue}{\textbf{\ipa{le˩˥}}}}\kern2pt]} \hypertarget{le\string_B1}{}
\markboth{\textcolor{darkblue}{\textbf{\ipa{le˩}}}}{}
\textcolor{teal}{\zh{语气助词}} \hspace{4pt} \zh{声调类:} L?.
\zh{句尾助词:感叹。} \textcolor{Sepia}{\selectlanguage{english}Exclamative final particle.} \textcolor{PineGreen}{\selectlanguage{french}Particule finale exclamative.}  ¶ \textcolor{darkblue}{\textbf{\ipa{dʑɤ˩ le˥!}}} \zh{好了!/ 太好了!} \textcolor{Sepia}{\selectlanguage{english}Well done! / Great!} \textcolor{PineGreen}{\selectlanguage{french}Bravo!}  

\lhead{\firstmark}
\rhead{\botmark}

\subsection{\hspace{-0.5cm} {\Large \textcolor{darkblue}{\textbf{\ipa{li˧\textsubscript{a}}}}}\hspace{0.5cm}[\kern2pt{\textcolor{darkblue}{\textbf{\ipa{li˥}}}}\kern2pt]} \hypertarget{li\string_Ma1}{}
\markboth{\textcolor{darkblue}{\textbf{\ipa{li˧\textsubscript{a}}}}}{}
\textcolor{teal}{\zh{动词}} \hspace{4pt} \zh{声调类:} M\textsubscript{a}.
\ding{202} \zh{看。} \textcolor{Sepia}{\selectlanguage{english}To look at.} \textcolor{PineGreen}{\selectlanguage{french}Regarder.}  ¶ \textcolor{darkblue}{\textbf{\ipa{tʰi˧-li˧-dʑo˧}}} \zh{正在看} \textcolor{Sepia}{\selectlanguage{english}\mytextsc{dur} \string_ \mytextsc{prog}} \textcolor{PineGreen}{\selectlanguage{french}\mytextsc{dur} \string_ \mytextsc{prog}}  
 ¶ \textcolor{darkblue}{\textbf{\ipa{tso˧\textasciitilde{}tso˧ li˩}}} \zh{看东西} \textcolor{Sepia}{\selectlanguage{english}to look at things} \textcolor{PineGreen}{\selectlanguage{french}regarder des choses}  
\ding{203} \zh{管理。} \textcolor{Sepia}{\selectlanguage{english}To manage, to be in charge of.} \textcolor{PineGreen}{\selectlanguage{french}S'occuper de.}  ¶ \textcolor{darkblue}{\textbf{\ipa{ɑ˩ʁo˧ li˧}}} \zh{管家、管家里的事情,看守家} \textcolor{Sepia}{\selectlanguage{english}to manage the household, to look after the house; to keep watch over the house} \textcolor{PineGreen}{\selectlanguage{french}s'occuper de la maison, veiller aux affaires de la maison; surveiller la maison}  
\ding{204} \zh{访问。} \textcolor{Sepia}{\selectlanguage{english}To visit, to go and see (someone).} \textcolor{PineGreen}{\selectlanguage{french}Rendre visite à, aller voir (quelqu'un).}  ¶ \textcolor{darkblue}{\textbf{\ipa{pʰæ˧tɕi˥-zo˩-ɳɯ˩ | mv̩˩zo˩ li˥}}} \zh{小伙子去拜访年轻女人(委婉语,指性交)} \textcolor{Sepia}{\selectlanguage{english}The young man sees the young woman. (Euphemistic phrasing, meaning “the young man has sexual intercourse with the young woman”.)} \textcolor{PineGreen}{\selectlanguage{french}Le jeune homme voit (=va voir) la jeune femme. (Euphémisme pour désigner les relations sexuelles.)}  

\lhead{\firstmark}
\rhead{\botmark}

\subsection{\hspace{-0.5cm} {\Large \textcolor{darkblue}{\textbf{\ipa{li˧ʐv̩˩}}}}\hspace{0.5cm}[\kern2pt{\textcolor{darkblue}{\textbf{\ipa{li˩ʐv̩˩˥}}}}\kern2pt]} \hypertarget{li\string_Mz`v\string_=\string_B1}{}
\markboth{\textcolor{darkblue}{\textbf{\ipa{li˧ʐv̩˩}}}}{}
\textcolor{teal}{\zh{名词}} \hspace{4pt} \zh{声调类:} L\#.
\zh{里脊肉。} \textcolor{Sepia}{\selectlanguage{english}Tenderloins.} \textcolor{PineGreen}{\selectlanguage{french}Filet-mignon.} 
\lhead{\firstmark}
\rhead{\botmark}

\subsection{\hspace{-0.5cm} {\Large \textcolor{darkblue}{\textbf{\ipa{li˩pi˥}}}}\hspace{0.5cm}[\kern2pt{\textcolor{darkblue}{\textbf{\ipa{li˧pi˥}}}}\kern2pt]} \hypertarget{li\string_Bpi\string_T1}{}
\markboth{\textcolor{darkblue}{\textbf{\ipa{li˩pi˥}}}}{}
\textcolor{teal}{\zh{名词}} \hspace{4pt} \zh{声调类:} LH.
\zh{已经泡了太久的茶叶。} \textcolor{Sepia}{\selectlanguage{english}Tea that has infused for too long, tea dregs.} \textcolor{PineGreen}{\selectlanguage{french}Feuille de thé qui a trop infusé, qui est bonne à jeter.}  \zh{量词}: \textcolor{darkblue}{\textbf{\ipa{kʰwɤ˥}}} 
\lhead{\firstmark}
\rhead{\botmark}

\subsection{\hspace{-0.5cm} {\Large \textcolor{darkblue}{\textbf{\ipa{li˩pʰv̩˩}}}}\hspace{0.5cm}[\kern2pt{\textcolor{darkblue}{\textbf{\ipa{li˧pʰv̩˩}}}}\kern2pt]} \hypertarget{li\string_Bp\string_hv\string_=\string_B1}{}
\markboth{\textcolor{darkblue}{\textbf{\ipa{li˩pʰv̩˩}}}}{}
\textcolor{teal}{\zh{名词}} \hspace{4pt} \zh{声调类:} L.
\zh{雪茶。} \textcolor{Sepia}{\selectlanguage{english}Whiteworm Lichen, \textit{Thamnolia vermicularis}; it used to be gathered on the seventh lunar month. It was used as a herbal tea.} \textcolor{PineGreen}{\selectlanguage{french}Un lichen de montagne, \textit{Thamnolia vermicularis}, employé en décoction. Au souvenir de F4, ce lichen se cueillait au septième mois, seule période où il était abondant; on allait le cueillir en haute montagne.}  ¶ \textcolor{darkblue}{\textbf{\ipa{ŋwɤ˧hɑ̃˩-li˩pʰv˩}}} \zh{\textcolor{darkblue}{\textbf{\ipa{ŋwɤ˧hɑ̃˩}}} 山的雪茶(说明:这种苔藓在那座山上多,七月份人家去采)} \textcolor{Sepia}{\selectlanguage{english}lichen tea from the mountain \textcolor{darkblue}{\textbf{\ipa{ŋwɤ˧hɑ̃˩}}} (this type of lichen grows abundantly on that mountain, and was generally harvested there)} \textcolor{PineGreen}{\selectlanguage{french}thé de lichen de la montagne \textcolor{darkblue}{\textbf{\ipa{ŋwɤ˧hɑ̃˩}}} (ce lichen y est abondant; c'est généralement là-bas qu'on le cueillait)}  

\lhead{\firstmark}
\rhead{\botmark}

\subsection{\hspace{-0.5cm} {\Large \textcolor{darkblue}{\textbf{\ipa{li˩˥}}}}\hspace{0.5cm}[\kern2pt{\textcolor{darkblue}{\textbf{\ipa{li˩˥}}}}\kern2pt]} \hypertarget{li\string_B\string_T1}{}
\markboth{\textcolor{darkblue}{\textbf{\ipa{li˩˥}}}}{}
\textcolor{teal}{\zh{名词}} \hspace{4pt} \zh{声调类:} LH.
\zh{茶。} \textcolor{Sepia}{\selectlanguage{english}Tea.} \textcolor{PineGreen}{\selectlanguage{french}Thé.}  ¶ \textcolor{darkblue}{\textbf{\ipa{li˩qʰɑ˩}}} \zh{‘苦茶’:用白芍药来泡的饮料,没有茶的时候就喝这种‘苦茶’。它有医疗作用,帮助消化。} \textcolor{Sepia}{\selectlanguage{english}'bitter tea': herbal tea prepared with leaves of Chinese peony, when there was no tea available; it had medicinal properties} \textcolor{PineGreen}{\selectlanguage{french}'thé amer': décoction de feuilles de pivoine blanche de Chine, que l'on buvait lorsqu'il n'y avait pas de thé à la maison; cela possédait des vertus médicinales.}  
 \zh{量词}: \textcolor{darkblue}{\textbf{\ipa{qʰwɤ˧˥}}} 
\lhead{\firstmark}
\rhead{\botmark}

\subsection{\hspace{-0.5cm} {\Large \textcolor{darkblue}{\textbf{\ipa{ljɤ˩\textsubscript{a}}}}}\hspace{0.5cm}[\kern2pt{\textcolor{darkblue}{\textbf{\ipa{ljɤ˩˥}}}}\kern2pt]} \hypertarget{lj7\string_Ba1}{}
\markboth{\textcolor{darkblue}{\textbf{\ipa{ljɤ˩\textsubscript{a}}}}}{}
\textcolor{teal}{\zh{量词}} \hspace{4pt} \zh{声调类:} L\textsubscript{a}.
\zh{量词:命、命运。} \textcolor{Sepia}{\selectlanguage{english}Self-classifier for lives/destinies.} \textcolor{PineGreen}{\selectlanguage{french}Auto-classificateur des vies/destins.}  ¶ \textcolor{darkblue}{\textbf{\ipa{ʈʂʰɯ˧-ljɤ˥}}} \zh{\mytextsc{指示代词} \string_} \textcolor{Sepia}{\selectlanguage{english}\mytextsc{dem} \string_ (tone: H\# / H\$)} \textcolor{PineGreen}{\selectlanguage{french}\mytextsc{dem} \string_ (tone: H\# / H\$)}  

\lhead{\firstmark}
\rhead{\botmark}

\subsection{\hspace{-0.5cm} {\Large \textcolor{darkblue}{\textbf{\ipa{ljɤ˩mi˥}}}}\hspace{0.5cm}[\kern2pt{\textcolor{darkblue}{\textbf{\ipa{ljɤ˩mi˩˥}}}}\kern2pt]} \hypertarget{lj7\string_Bmi\string_T1}{}
\markboth{\textcolor{darkblue}{\textbf{\ipa{ljɤ˩mi˥}}}}{}
\textcolor{teal}{\zh{名词}} \hspace{4pt} \zh{声调类:} LH.
\zh{大梁。} \textcolor{Sepia}{\selectlanguage{english}Major (supporting) beam.} \textcolor{PineGreen}{\selectlanguage{french}Poutre importante.}  \zh{量词}: \textcolor{darkblue}{\textbf{\ipa{pʰæ˧˥}}} 
\lhead{\firstmark}
\rhead{\botmark}

\subsection{\hspace{-0.5cm} {\Large \textcolor{darkblue}{\textbf{\ipa{ljɤ˩mi˥-ʈæ˩qo˩}}}}\hspace{0.5cm}[\kern2pt{\textcolor{darkblue}{\textbf{\ipa{xxxx non-correspondance entre le nombre de morphèmes et le nombre de tons de morphèmes}}}}\kern2pt]} \hypertarget{lj7\string_Bmi\string_T-t`\{\string_Bqo\string_B1}{}
\markboth{\textcolor{darkblue}{\textbf{\ipa{ljɤ˩mi˥-ʈæ˩qo˩}}}}{}
\textcolor{teal}{\zh{名词}} \hspace{4pt} \zh{声调类:} LH-.
\zh{大梁的装饰:大梁的‘耳朵’。} \textcolor{Sepia}{\selectlanguage{english}Decoration of major (supporting) beam: symbolically, this is the beam's 'ear'.} \textcolor{PineGreen}{\selectlanguage{french}Enjolivement sous une poutre; est perçu symboliquement comme 'l'oreille' de la poutre.}  \zh{量词}: \textcolor{darkblue}{\textbf{\ipa{pʰæ˧˥}}} 
\lhead{\firstmark}
\rhead{\botmark}

\subsection{\hspace{-0.5cm} {\Large \textcolor{darkblue}{\textbf{\ipa{ljɤ˩ʂɯ˩}}}}\hspace{0.5cm}[\kern2pt{\textcolor{darkblue}{\textbf{\ipa{xxxx non-correspondance entre le nombre de morphèmes et le nombre de tons de morphèmes}}}}\kern2pt]} \hypertarget{lj7\string_Bs`M\string_B1}{}
\markboth{\textcolor{darkblue}{\textbf{\ipa{ljɤ˩ʂɯ˩}}}}{}
\textcolor{teal}{\zh{名词}} \hspace{4pt} \zh{声调类:} L.
\zh{粮食(汉语借词)。} \textcolor{Sepia}{\selectlanguage{english}Cereals.} \textcolor{PineGreen}{\selectlanguage{french}Céréales.}  \zh{【借词】} \zh{粮食}

\lhead{\firstmark}
\rhead{\botmark}

\subsection{\hspace{-0.5cm} {\Large \textcolor{darkblue}{\textbf{\ipa{ljɤ˩˥}}} \textsubscript{1}}\hspace{0.5cm}[\kern2pt{\textcolor{darkblue}{\textbf{\ipa{ljɤ˩˥}}}}\kern2pt]} \hypertarget{lj7\string_B\string_T1}{}
\markboth{\textcolor{darkblue}{\textbf{\ipa{ljɤ˩˥}}} \textsubscript{1}}{}
\textcolor{teal}{\zh{名词}} \hspace{4pt} \zh{声调类:} LH.
\zh{梁。} \textcolor{Sepia}{\selectlanguage{english}Beam.} \textcolor{PineGreen}{\selectlanguage{french}Poutre.}  \zh{量词}: \textcolor{darkblue}{\textbf{\ipa{pʰæ˧˥}}} 
\lhead{\firstmark}
\rhead{\botmark}

\subsection{\hspace{-0.5cm} {\Large \textcolor{darkblue}{\textbf{\ipa{ljɤ˩˥}}} \textsubscript{2}}\hspace{0.5cm}[\kern2pt{\textcolor{darkblue}{\textbf{\ipa{ljɤ˩˥}}}}\kern2pt]} \hypertarget{lj7\string_B\string_T2}{}
\markboth{\textcolor{darkblue}{\textbf{\ipa{ljɤ˩˥}}} \textsubscript{2}}{}
\textcolor{teal}{\zh{名词}} \hspace{4pt} \zh{声调类:} LM?LH?.
\zh{命、生命、命运。} \textcolor{Sepia}{\selectlanguage{english}Life, existence, destiny, fate.} \textcolor{PineGreen}{\selectlanguage{french}Sort, lot, existence, vie, destinée.}  ¶ \textcolor{darkblue}{\textbf{\ipa{no˧ | ljɤ˩ ʈʂʰɯ˧-ljɤ˩-dʑo˩, | qʰæ˩˥ | ʐwæ˩˥!}}} \zh{你命好!} \textcolor{Sepia}{\selectlanguage{english}You really have a happy lot!} \textcolor{PineGreen}{\selectlanguage{french}Tu as une belle vie! Tu as la vie belle!}  
 ¶ \textcolor{darkblue}{\textbf{\ipa{hĩ˧-ljɤ˥}}} \zh{人命、人类的命运} \textcolor{Sepia}{\selectlanguage{english}human existence, the human lot} \textcolor{PineGreen}{\selectlanguage{french}l'existence humaine}  
 \zh{量词}: \textcolor{darkblue}{\textbf{\ipa{ljɤ˩}}} 
\lhead{\firstmark}
\rhead{\botmark}

\subsection{\hspace{-0.5cm} {\Large \textcolor{darkblue}{\textbf{\ipa{lje˩fe˧}}}}\hspace{0.5cm}[\kern2pt{\textcolor{darkblue}{\textbf{\ipa{lje˧fe˩}}}}\kern2pt]} \hypertarget{lje\string_Bfe\string_M1}{}
\markboth{\textcolor{darkblue}{\textbf{\ipa{lje˩fe˧}}}}{}
\textcolor{teal}{\zh{名词}} \hspace{4pt} \zh{声调类:} LM.
\zh{凉粉。} \textcolor{Sepia}{\selectlanguage{english}Mungo bean jelly.} \textcolor{PineGreen}{\selectlanguage{french}Liangfen: spécialité de Dali et des environs.}  \zh{【借词】} \zh{凉粉}

\lhead{\firstmark}
\rhead{\botmark}

\subsection{\hspace{-0.5cm} {\Large \textcolor{darkblue}{\textbf{\ipa{lo˧}}}}\hspace{0.5cm}[\kern2pt{\textcolor{darkblue}{\textbf{\ipa{lo˥}}}}\kern2pt]} \hypertarget{lo\string_M1}{}
\markboth{\textcolor{darkblue}{\textbf{\ipa{lo˧}}}}{}
\textcolor{teal}{\zh{名词}} \hspace{4pt} \zh{声调类:} M.
\ding{202} \zh{事情。} \textcolor{Sepia}{\selectlanguage{english}Work, occupation, task.} \textcolor{PineGreen}{\selectlanguage{french}Occupation, travail, tâche.}  ¶ \textcolor{darkblue}{\textbf{\ipa{lo˧ dʑo˧}}} \zh{忙,有活要干} \textcolor{Sepia}{\selectlanguage{english}to be busy, to have work to do} \textcolor{PineGreen}{\selectlanguage{french}avoir du travail, être occupé}  
 ¶ \textcolor{darkblue}{\textbf{\ipa{njɤ˧ | lo˧ mɤ˧-dʑo˧.}}} \zh{我不忙。} \textcolor{Sepia}{\selectlanguage{english}I am not busy. / I have some free time.} \textcolor{PineGreen}{\selectlanguage{french}Je ne suis pas occupé. / J'ai du temps libre. / Je suis disponible.}  
 \zh{量词}: \textcolor{darkblue}{\textbf{\ipa{lo˧}}} \ding{203} \zh{用处。} \textcolor{Sepia}{\selectlanguage{english}Usefulness.} \textcolor{PineGreen}{\selectlanguage{french}Utilité.}  ¶ \textcolor{darkblue}{\textbf{\ipa{lo˧ mɤ˧-dʑo˧}}} \zh{没有用!(情景:谈到常春藤,说它是没有用处的植物)} \textcolor{Sepia}{\selectlanguage{english}It's no use / it does not have any usefulness. (Context: talking about ivy, which cannot be fed to cattle and is not used for medical purposes, or for firewood, or for making ropes, tools...)} \textcolor{PineGreen}{\selectlanguage{french}C'est inutile / ça n'a aucune utilité. (Contexte: discussion au sujet du lierre, plante qui n'est utilisée ni comme fourrage, ni comme médicament, ni comme combustible, ni pour la confection de cordes ou autres outils ou objets)}  

\lhead{\firstmark}
\rhead{\botmark}

\subsection{\hspace{-0.5cm} {\Large \textcolor{darkblue}{\textbf{\ipa{lo˧\textsubscript{b}}}}}\hspace{0.5cm}[\kern2pt{\textcolor{darkblue}{\textbf{\ipa{lo˩˥}}}}\kern2pt]} \hypertarget{lo\string_Mb1}{}
\markboth{\textcolor{darkblue}{\textbf{\ipa{lo˧\textsubscript{b}}}}}{}
\textcolor{teal}{\zh{量词}} \hspace{4pt} \zh{声调类:} M\textsubscript{b}.
\zh{量词:事情(一件)、活(一个)。} \textcolor{Sepia}{\selectlanguage{english}Self-classifier for tasks/occupations.} \textcolor{PineGreen}{\selectlanguage{french}Auto-classificateur des travaux/occupations.} 
\lhead{\firstmark}
\rhead{\botmark}

\subsection{\hspace{-0.5cm} {\Large \textcolor{darkblue}{\textbf{\ipa{lo˧bæ˧˥}}}}\hspace{0.5cm}[\kern2pt{\textcolor{darkblue}{\textbf{\ipa{lo˧bæ˧}}}}\kern2pt]} \hypertarget{lo\string_Mb\{\string_M\string_T1}{}
\markboth{\textcolor{darkblue}{\textbf{\ipa{lo˧bæ˧˥}}}}{}
\textcolor{teal}{\zh{名词}} \hspace{4pt} \zh{声调类:} MH\#.
\zh{索桥,溜索。} \textcolor{Sepia}{\selectlanguage{english}Suspended bridge; zip line, flying fox.} \textcolor{PineGreen}{\selectlanguage{french}Pont suspendu; pont de corde. La corde du pont suspendu aurait été faite d'écorces d'arbres, non de chanvre, car les cordes en chanvre se détériorent rapidement quand elles sont exposées à la pluie.} 
\lhead{\firstmark}
\rhead{\botmark}

\subsection{\hspace{-0.5cm} {\Large \textcolor{darkblue}{\textbf{\ipa{lo˧bv̩˩-ʈʂʰɯ˩}}}}\hspace{0.5cm}[\kern2pt{\textcolor{darkblue}{\textbf{\ipa{lo˩bv̩˧ʈʂʰɯ˧}}}}\kern2pt]} \hypertarget{lo\string_Mbv\string_=\string_B-t`s`\string_hM\string_B1}{}
\markboth{\textcolor{darkblue}{\textbf{\ipa{lo˧bv̩˩-ʈʂʰɯ˩}}}}{}
\textcolor{teal}{\zh{名词}} \hspace{4pt} \zh{声调类:} L\#-.
\zh{象、大象。} \textcolor{Sepia}{\selectlanguage{english}Elephant.} \textcolor{PineGreen}{\selectlanguage{french}Éléphant.}  \zh{【借词】}\zh{藏语}
 \zh{量词}: \textcolor{darkblue}{\textbf{\ipa{pʰo˧˥}}} \textcolor{darkblue}{\textbf{\ipa{v̩˧}}} 
\lhead{\firstmark}
\rhead{\botmark}

\subsection{\hspace{-0.5cm} {\Large \textcolor{darkblue}{\textbf{\ipa{lo˧ɖʐɤ˩}}}}\hspace{0.5cm}[\kern2pt{\textcolor{darkblue}{\textbf{\ipa{lo˩ɖʐɤ˥}}}}\kern2pt]} \hypertarget{lo\string_Md`z`7\string_B1}{}
\markboth{\textcolor{darkblue}{\textbf{\ipa{lo˧ɖʐɤ˩}}}}{}
\textcolor{teal}{\zh{名词}} \hspace{4pt} \zh{声调类:} L\#.
\zh{三齿耙。} \textcolor{Sepia}{\selectlanguage{english}Weeding hoe: hand instrument with three spikes perpendicular to the handle, to loosen the soil. At the time of fieldwork, this tool had a metal head.} \textcolor{PineGreen}{\selectlanguage{french}Serfouette, croc à trois dents: instrument à trois dents perpendiculaires au manche, pour ameublir la terre. Les modèles actuellement utilisés ont une tête en métal.}  ¶ \textcolor{darkblue}{\textbf{\ipa{lo˧ɖʐɤ˩ ʈʂʰɯ˩-nɑ˩}}} \zh{这把三齿耙} \textcolor{Sepia}{\selectlanguage{english}\mytextsc{n}+\mytextsc{dem}+\mytextsc{clf}} \textcolor{PineGreen}{\selectlanguage{french}\mytextsc{n}+\mytextsc{dem}+\mytextsc{clf}}  
 \zh{量词}: \textcolor{darkblue}{\textbf{\ipa{nɑ˧}}} 
\lhead{\firstmark}
\rhead{\botmark}

\subsection{\hspace{-0.5cm} {\Large \textcolor{darkblue}{\textbf{\ipa{lo˧fv̩˧}}}}\hspace{0.5cm}[\kern2pt{\textcolor{darkblue}{\textbf{\ipa{lo˧fv̩˩}}}}\kern2pt]} \hypertarget{lo\string_Mfv\string_=\string_M1}{}
\markboth{\textcolor{darkblue}{\textbf{\ipa{lo˧fv̩˧}}}}{}
\textcolor{teal}{\zh{形容词}} \hspace{4pt} \zh{声调类:} .
\zh{容易,容易做。} \textcolor{Sepia}{\selectlanguage{english}Easy.} \textcolor{PineGreen}{\selectlanguage{french}Facile à faire.}  ¶ \textcolor{darkblue}{\textbf{\ipa{lo˧fv̩˧ | ʐwæ˩˥}}} \zh{很容易} \textcolor{Sepia}{\selectlanguage{english}very easy} \textcolor{PineGreen}{\selectlanguage{french}très facile}  

\lhead{\firstmark}
\rhead{\botmark}

\subsection{\hspace{-0.5cm} {\Large \textcolor{darkblue}{\textbf{\ipa{lo˧gv̩˩}}}}\hspace{0.5cm}[\kern2pt{\textcolor{darkblue}{\textbf{\ipa{lo˧gv̩˩}}}}\kern2pt]} \hypertarget{lo\string_Mgv\string_=\string_B1}{}
\markboth{\textcolor{darkblue}{\textbf{\ipa{lo˧gv̩˩}}}}{}
\textcolor{teal}{\zh{名词}} \hspace{4pt} \zh{声调类:} L\#.
\zh{宁蒗。} \textcolor{Sepia}{\selectlanguage{english}Ninglang.} \textcolor{PineGreen}{\selectlanguage{french}Ninglang; actuellement utilisé pour désigner un village na du comté de Ninglang, relativement proche du centre administratif.}  ¶ \textcolor{darkblue}{\textbf{\ipa{lo˧gv̩˩-di˩mi˩}}} \zh{宁蒗坝子} \textcolor{Sepia}{\selectlanguage{english}the Ninglang plain} \textcolor{PineGreen}{\selectlanguage{french}la plaine de Ninglang}  

\lhead{\firstmark}
\rhead{\botmark}

\subsection{\hspace{-0.5cm} {\Large \textcolor{darkblue}{\textbf{\ipa{lo˧ʝi˧-hĩ˧-hĩ\#˥}}}}\hspace{0.5cm}[\kern2pt{\textcolor{darkblue}{\textbf{\ipa{xxxx non-correspondance entre le nombre de morphèmes et le nombre de tons de morphèmes}}}}\kern2pt]} \hypertarget{lo\string_Mj££i\string_M-hi\string_~\string_M-hi\string_~\#\string_T1}{}
\markboth{\textcolor{darkblue}{\textbf{\ipa{lo˧ʝi˧-hĩ˧-hĩ\#˥}}}}{}
\textcolor{teal}{\zh{名词}} \hspace{4pt} \zh{声调类:} \#H.
\zh{劳动人民,农民。} \textcolor{Sepia}{\selectlanguage{english}Worker (in the fields or elsewhere).} \textcolor{PineGreen}{\selectlanguage{french}Travailleur (paysan, ouvrier…).}  \zh{量词}: \textcolor{darkblue}{\textbf{\ipa{v̩˧}}} 
\lhead{\firstmark}
\rhead{\botmark}

\subsection{\hspace{-0.5cm} {\Large \textcolor{darkblue}{\textbf{\ipa{lo˧lo˧}}}}\hspace{0.5cm}[\kern2pt{\textcolor{darkblue}{\textbf{\ipa{lo˧lo˧}}}}\kern2pt]} \hypertarget{lo\string_Mlo\string_M1}{}
\markboth{\textcolor{darkblue}{\textbf{\ipa{lo˧lo˧}}}}{}
\textcolor{teal}{\zh{名词}} \hspace{4pt} \zh{声调类:} M.
\zh{彝族。} \textcolor{Sepia}{\selectlanguage{english}Yi (ethnic group).} \textcolor{PineGreen}{\selectlanguage{french}Yi (groupe ethnique).}  \zh{量词}: \textcolor{darkblue}{\textbf{\ipa{v̩˧}}} 
\lhead{\firstmark}
\rhead{\botmark}

\subsection{\hspace{-0.5cm} {\Large \textcolor{darkblue}{\textbf{\ipa{lo˧ɲi˥}}}}\hspace{0.5cm}[\kern2pt{\textcolor{darkblue}{\textbf{\ipa{lo˧ɲi˥}}}}\kern2pt]} \hypertarget{lo\string_MJi\string_T1}{}
\markboth{\textcolor{darkblue}{\textbf{\ipa{lo˧ɲi˥}}}}{}
\textcolor{teal}{\zh{名词}} \hspace{4pt} \zh{声调类:} H\#.
\zh{手指。} \textcolor{Sepia}{\selectlanguage{english}Finger.} \textcolor{PineGreen}{\selectlanguage{french}Doigt.}  \zh{量词}: \textcolor{darkblue}{\textbf{\ipa{ɭɯ˧}}} 
\lhead{\firstmark}
\rhead{\botmark}

\subsection{\hspace{-0.5cm} {\Large \textcolor{darkblue}{\textbf{\ipa{lo˧ɲi˥ | ɖɯ˧-ɭɯ˧}}}}\hspace{0.5cm}[\kern2pt{\textcolor{darkblue}{\textbf{\ipa{xxxx non-correspondance entre le nombre de groupes tonals et le nombre de tons}}}}\kern2pt]} \hypertarget{lo\string_MJi\string_T | d`M\string_M-l\string_RM\string_M1}{}
\markboth{\textcolor{darkblue}{\textbf{\ipa{lo˧ɲi˥ | ɖɯ˧-ɭɯ˧}}}}{}
\textcolor{teal}{\zh{名词}} \hspace{4pt} \zh{声调类:} H\# | M.
\zh{食指。} \textcolor{Sepia}{\selectlanguage{english}Index.} \textcolor{PineGreen}{\selectlanguage{french}Index.} 
\lhead{\firstmark}
\rhead{\botmark}

\subsection{\hspace{-0.5cm} {\Large \textcolor{darkblue}{\textbf{\ipa{lo˧ɲi˥ | ɲi˧-ɭɯ˧}}}}\hspace{0.5cm}[\kern2pt{\textcolor{darkblue}{\textbf{\ipa{xxxx non-correspondance entre le nombre de groupes tonals et le nombre de tons}}}}\kern2pt]} \hypertarget{lo\string_MJi\string_T | Ji\string_M-l\string_RM\string_M1}{}
\markboth{\textcolor{darkblue}{\textbf{\ipa{lo˧ɲi˥ | ɲi˧-ɭɯ˧}}}}{}
\textcolor{teal}{\zh{名词}} \hspace{4pt} \zh{声调类:} H\# | M.
\zh{中指。} \textcolor{Sepia}{\selectlanguage{english}Middle finger.} \textcolor{PineGreen}{\selectlanguage{french}Majeur.} 
\lhead{\firstmark}
\rhead{\botmark}

\subsection{\hspace{-0.5cm} {\Large \textcolor{darkblue}{\textbf{\ipa{lo˧ʂv̩˩}}}}\hspace{0.5cm}[\kern2pt{\textcolor{darkblue}{\textbf{\ipa{lo˧ʂv̩˩}}}}\kern2pt]} \hypertarget{lo\string_Ms`v\string_=\string_B1}{}
\markboth{\textcolor{darkblue}{\textbf{\ipa{lo˧ʂv̩˩}}}}{}
\textcolor{teal}{\zh{名词}} \hspace{4pt} \zh{声调类:} L\#.
\zh{洛水村。} \textcolor{Sepia}{\selectlanguage{english}The village of Luoshui.} \textcolor{PineGreen}{\selectlanguage{french}Luoshui: village du bord du Lac.} 
\lhead{\firstmark}
\rhead{\botmark}

\subsection{\hspace{-0.5cm} {\Large \textcolor{darkblue}{\textbf{\ipa{lo˧ʂv̩˩ | -hi˩-nɑ˧mi˧}}}}\hspace{0.5cm}[\kern2pt{\textcolor{darkblue}{\textbf{\ipa{xxxx non-correspondance entre le nombre de groupes tonals et le nombre de tons}}}}\kern2pt]} \hypertarget{lo\string_Ms`v\string_=\string_B | -hi\string_B-nA\string_Mmi\string_M1}{}
\markboth{\textcolor{darkblue}{\textbf{\ipa{lo˧ʂv̩˩ | -hi˩-nɑ˧mi˧}}}}{}
\textcolor{teal}{\zh{名词}} \hspace{4pt} \zh{声调类:} L\# | L-.
\zh{泸沽湖。} \textcolor{Sepia}{\selectlanguage{english}Lugu lake.} \textcolor{PineGreen}{\selectlanguage{french}Lac Lugu.} 
\lhead{\firstmark}
\rhead{\botmark}

\subsection{\hspace{-0.5cm} {\Large \textcolor{darkblue}{\textbf{\ipa{lo˧tɑ˧-lo˧tɕi\#˥}}}}\hspace{0.5cm}[\kern2pt{\textcolor{darkblue}{\textbf{\ipa{xxxx non-correspondance entre le nombre de morphèmes et le nombre de tons de morphèmes}}}}\kern2pt]} \hypertarget{lo\string_MtA\string_M-lo\string_Mts£i\#\string_T1}{}
\markboth{\textcolor{darkblue}{\textbf{\ipa{lo˧tɑ˧-lo˧tɕi\#˥}}}}{}
\textcolor{teal}{\zh{名词}} \hspace{4pt} \zh{声调类:} \#H.
\zh{经幡、风马旗(挂在山上)。} \textcolor{Sepia}{\selectlanguage{english}Streamer of scriptures.} \textcolor{PineGreen}{\selectlanguage{french}Drapeau de prières.}  \zh{【借词】}\zh{藏语} rlung rta
 ¶ \textcolor{darkblue}{\textbf{\ipa{lo˧tɑ˧-lo˧tɕi˧ | le˧-lɑ˧˥}}} \zh{直译:印出一个经幡。也来指准备经幡的工作(到山上去挂之前)} \textcolor{Sepia}{\selectlanguage{english}to print a streamer of scriptures; more generally: to string together a streamer of scriptures} \textcolor{PineGreen}{\selectlanguage{french}imprimer un drapeau de prières; sens plus général: confectionner un drapeau de prières (chez soi, avant de se rendre sur le lieu où on l'installe)}  
 \zh{量词}: \textcolor{darkblue}{\textbf{\ipa{pɤ˥}}} \textcolor{darkblue}{\textbf{\ipa{pʰæ˧˥}}} 
\lhead{\firstmark}
\rhead{\botmark}

\subsection{\hspace{-0.5cm} {\Large \textcolor{darkblue}{\textbf{\ipa{lo˩}}} \textsubscript{1}}\hspace{0.5cm}[\kern2pt{\textcolor{darkblue}{\textbf{\ipa{lo˩˥}}}}\kern2pt]} \hypertarget{lo\string_B1}{}
\markboth{\textcolor{darkblue}{\textbf{\ipa{lo˩}}} \textsubscript{1}}{}
\textcolor{teal}{\zh{动词}} \hspace{4pt} \zh{声调类:} L.
\zh{过(垭口)。} \textcolor{Sepia}{\selectlanguage{english}To cross (a mountain pass).} \textcolor{PineGreen}{\selectlanguage{french}Passer, franchir (un col).}  ¶ \textcolor{darkblue}{\textbf{\ipa{mv̩˩tɕo˧-lo˩}}} \zh{往下过去(过了垭口以后)} \textcolor{Sepia}{\selectlanguage{english}to go down (after crossing a mountain pass)} \textcolor{PineGreen}{\selectlanguage{french}descendre (après avoir passé un col)}  

\lhead{\firstmark}
\rhead{\botmark}

\subsection{\hspace{-0.5cm} {\Large \textcolor{darkblue}{\textbf{\ipa{lo˩}}} \textsubscript{2}}\hspace{0.5cm}[\kern2pt{\textcolor{darkblue}{\textbf{\ipa{lo˥}}}}\kern2pt]} \hypertarget{lo\string_B2}{}
\markboth{\textcolor{darkblue}{\textbf{\ipa{lo˩}}} \textsubscript{2}}{}
\textcolor{teal}{\zh{名词}} \hspace{4pt} \zh{声调类:} L.
\zh{山谷。} \textcolor{Sepia}{\selectlanguage{english}Mountain valley.} \textcolor{PineGreen}{\selectlanguage{french}Vallée de montagne.}  ¶ \textcolor{darkblue}{\textbf{\ipa{lo˧-qo˧}}} \zh{山谷里} \textcolor{Sepia}{\selectlanguage{english}in the valley} \textcolor{PineGreen}{\selectlanguage{french}dans la vallée}  
 \zh{量词}: \textcolor{darkblue}{\textbf{\ipa{lo˩}}} 
\lhead{\firstmark}
\rhead{\botmark}

\subsection{\hspace{-0.5cm} {\Large \textcolor{darkblue}{\textbf{\ipa{lo˩\textsubscript{b}}}}}\hspace{0.5cm}[\kern2pt{\textcolor{darkblue}{\textbf{\ipa{lo˥}}}}\kern2pt]} \hypertarget{lo\string_Bb1}{}
\markboth{\textcolor{darkblue}{\textbf{\ipa{lo˩\textsubscript{b}}}}}{}
\textcolor{teal}{\zh{量词}} \hspace{4pt} \zh{声调类:} L\textsubscript{b}.
\zh{量词:谷。} \textcolor{Sepia}{\selectlanguage{english}Self-classifier for valleys.} \textcolor{PineGreen}{\selectlanguage{french}Classificateur des vallées.}  ¶ \textcolor{darkblue}{\textbf{\ipa{hĩ˧-ɻ̃˧ | ɖɯ˧-lo˩}}} \zh{住在一座山谷里的所有人(直译:‘一山谷的人’)} \textcolor{Sepia}{\selectlanguage{english}literally 'a valley of people', to mean: all the population of that valley} \textcolor{PineGreen}{\selectlanguage{french}tous les membres d'une grande famille: '[toute la population d']une vallée'}  
 ¶ \textcolor{darkblue}{\textbf{\ipa{si˧dzi˩ | ɖɯ˧-lo˩}}} \zh{一山谷的树,一片森林(直译:‘一山谷的树’)} \textcolor{Sepia}{\selectlanguage{english}'a valley [of/covered with] trees', i.e. a large tract of woodland} \textcolor{PineGreen}{\selectlanguage{french}une grande quantité d'arbres: 'une vallée [couverte] d'arbres'}  

\lhead{\firstmark}
\rhead{\botmark}

\subsection{\hspace{-0.5cm} {\Large \textcolor{darkblue}{\textbf{\ipa{lo˩bɤ˩}}}}\hspace{0.5cm}[\kern2pt{\textcolor{darkblue}{\textbf{\ipa{lo˧bɤ˧˥}}}}\kern2pt]} \hypertarget{lo\string_Bb7\string_B1}{}
\markboth{\textcolor{darkblue}{\textbf{\ipa{lo˩bɤ˩}}}}{}
\textcolor{teal}{\zh{名词}} \hspace{4pt} \zh{声调类:} L.
\zh{手掌。} \textcolor{Sepia}{\selectlanguage{english}Palm of the hand.} \textcolor{PineGreen}{\selectlanguage{french}Paume.}  \zh{量词}: \textcolor{darkblue}{\textbf{\ipa{ɭɯ˧}}} 
\lhead{\firstmark}
\rhead{\botmark}

\subsection{\hspace{-0.5cm} {\Large \textcolor{darkblue}{\textbf{\ipa{lo˩bv̩˧-ɭɯ˩}}}}\hspace{0.5cm}[\kern2pt{\textcolor{darkblue}{\textbf{\ipa{xxxx non-correspondance entre le nombre de morphèmes et le nombre de tons de morphèmes}}}}\kern2pt]} \hypertarget{lo\string_Bbv\string_=\string_M-l\string_RM\string_B1}{}
\markboth{\textcolor{darkblue}{\textbf{\ipa{lo˩bv̩˧-ɭɯ˩}}}}{}
\textcolor{teal}{\zh{名词}} \hspace{4pt} \zh{声调类:} LM-L.
\zh{肘。} \textcolor{Sepia}{\selectlanguage{english}Elbow.} \textcolor{PineGreen}{\selectlanguage{french}Partie saillante du coude, qd le bras est replié.}  \zh{量词}: \textcolor{darkblue}{\textbf{\ipa{ɭɯ˧}}} 
\lhead{\firstmark}
\rhead{\botmark}

\subsection{\hspace{-0.5cm} {\Large \textcolor{darkblue}{\textbf{\ipa{lo˩dv̩\#˥}}}}\hspace{0.5cm}[\kern2pt{\textcolor{darkblue}{\textbf{\ipa{lo˩dv̩˥}}}}\kern2pt]} \hypertarget{lo\string_Bdv\string_=\#\string_T1}{}
\markboth{\textcolor{darkblue}{\textbf{\ipa{lo˩dv̩\#˥}}}}{}
\textcolor{teal}{\zh{名词}} \hspace{4pt} \zh{声调类:} LM+\#H.
\zh{独臂人:缺一只胳膊(手)的人。} \textcolor{Sepia}{\selectlanguage{english}Person with a single arm or hand, one-armed (or one-handed) person.} \textcolor{PineGreen}{\selectlanguage{french}Manchot.}  \zh{量词}: \textcolor{darkblue}{\textbf{\ipa{v̩˧}}} 
\lhead{\firstmark}
\rhead{\botmark}

\subsection{\hspace{-0.5cm} {\Large \textcolor{darkblue}{\textbf{\ipa{lo˩dzi˩}}}}\hspace{0.5cm}[\kern2pt{\textcolor{darkblue}{\textbf{\ipa{lo˩dzi˥}}}}\kern2pt]} \hypertarget{lo\string_Bdzi\string_B1}{}
\markboth{\textcolor{darkblue}{\textbf{\ipa{lo˩dzi˩}}}}{}
\textcolor{teal}{\zh{量词}} \hspace{4pt} \zh{声调类:} L.
\zh{量词:捧(用两只手)。} \textcolor{Sepia}{\selectlanguage{english}A handful (using both hands).} \textcolor{PineGreen}{\selectlanguage{french}Classificateur des poignées (à deux mains).} 
\lhead{\firstmark}
\rhead{\botmark}

\subsection{\hspace{-0.5cm} {\Large \textcolor{darkblue}{\textbf{\ipa{lo˩dʑo˥}}}}\hspace{0.5cm}[\kern2pt{\textcolor{darkblue}{\textbf{\ipa{lo˩dʑo˩˥}}}}\kern2pt]} \hypertarget{lo\string_Bdz£o\string_T1}{}
\markboth{\textcolor{darkblue}{\textbf{\ipa{lo˩dʑo˥}}}}{}
\textcolor{teal}{\zh{名词}} \hspace{4pt} \zh{声调类:} LH.
\zh{手镯。} \textcolor{Sepia}{\selectlanguage{english}Bracelet.} \textcolor{PineGreen}{\selectlanguage{french}Bracelet.}  ¶ \textcolor{darkblue}{\textbf{\ipa{ŋv̩˩-lo˩dʑo˧ (+ɲi˩)}}} \zh{银手镯} \textcolor{Sepia}{\selectlanguage{english}silver bracelet} \textcolor{PineGreen}{\selectlanguage{french}bracelet en argent}  
 ¶ \textcolor{darkblue}{\textbf{\ipa{hæ̃˩-lo˩dʑo˥ (+ɲi˩)}}} \zh{金手镯} \textcolor{Sepia}{\selectlanguage{english}gold bracelet} \textcolor{PineGreen}{\selectlanguage{french}bracelet en or}  
 ¶ \textcolor{darkblue}{\textbf{\ipa{jo˧-lo˥dʑo˩}}} \zh{玉手镯} \textcolor{Sepia}{\selectlanguage{english}jade bracelet} \textcolor{PineGreen}{\selectlanguage{french}bracelet en jade}  
 ¶ \textcolor{darkblue}{\textbf{\ipa{lo˩dʑo˥ kʰɯ˩}}} \zh{戴上手镯} \textcolor{Sepia}{\selectlanguage{english}to put on a bracelet} \textcolor{PineGreen}{\selectlanguage{french}mettre un bracelet}  
 \zh{量词}: \textcolor{darkblue}{\textbf{\ipa{pʰo˧˥}}} 
\lhead{\firstmark}
\rhead{\botmark}

\subsection{\hspace{-0.5cm} {\Large \textcolor{darkblue}{\textbf{\ipa{lo˩ɖɯ˧}}}}\hspace{0.5cm}[\kern2pt{\textcolor{darkblue}{\textbf{\ipa{xxxx non-correspondance entre le nombre de morphèmes et le nombre de tons de morphèmes}}}}\kern2pt]} \hypertarget{lo\string_Bd`M\string_M1}{}
\markboth{\textcolor{darkblue}{\textbf{\ipa{lo˩ɖɯ˧}}}}{}
\textcolor{teal}{\zh{形容词}} \hspace{4pt} \zh{声调类:} LM.
\zh{大方。} \textcolor{Sepia}{\selectlanguage{english}Generous.} \textcolor{PineGreen}{\selectlanguage{french}Généreux.} 
\lhead{\firstmark}
\rhead{\botmark}

\subsection{\hspace{-0.5cm} {\Large \textcolor{darkblue}{\textbf{\ipa{lo˩-gv̩˧dv̩˧}}}}\hspace{0.5cm}[\kern2pt{\textcolor{darkblue}{\textbf{\ipa{lo˧gv̩˧dv̩˧}}}}\kern2pt]} \hypertarget{lo\string_B-gv\string_=\string_Mdv\string_=\string_M1}{}
\markboth{\textcolor{darkblue}{\textbf{\ipa{lo˩-gv̩˧dv̩˧}}}}{}
\textcolor{teal}{\zh{名词}} \hspace{4pt} \zh{声调类:} L-.
\zh{手背。} \textcolor{Sepia}{\selectlanguage{english}Back of the hand.} \textcolor{PineGreen}{\selectlanguage{french}Dos de la main.}  \zh{量词}: \textcolor{darkblue}{\textbf{\ipa{kʰwɤ˥}}} 
\lhead{\firstmark}
\rhead{\botmark}

\subsection{\hspace{-0.5cm} {\Large \textcolor{darkblue}{\textbf{\ipa{lo˩jɤ˧}}}}\hspace{0.5cm}[\kern2pt{\textcolor{darkblue}{\textbf{\ipa{lo˩jɤ˥}}}}\kern2pt]} \hypertarget{lo\string_Bj7\string_M1}{}
\markboth{\textcolor{darkblue}{\textbf{\ipa{lo˩jɤ˧}}}}{}
\textcolor{teal}{\zh{名词}} \hspace{4pt} \zh{声调类:} LM.
\zh{银元。} \textcolor{Sepia}{\selectlanguage{english}Silver coin, silver yuan.} \textcolor{PineGreen}{\selectlanguage{french}Pièce d'argent.}  ¶ \textcolor{darkblue}{\textbf{\ipa{lo˩jɤ˧ | ɖɯ˧-pʰæ˧˥}}} \zh{一块银元} \textcolor{Sepia}{\selectlanguage{english}one silver coin} \textcolor{PineGreen}{\selectlanguage{french}une pièce d'argent}  

\lhead{\firstmark}
\rhead{\botmark}

\subsection{\hspace{-0.5cm} {\Large \textcolor{darkblue}{\textbf{\ipa{lo˩ko˧}}}}\hspace{0.5cm}[\kern2pt{\textcolor{darkblue}{\textbf{\ipa{lo˩ko˥}}}}\kern2pt]} \hypertarget{lo\string_Bko\string_M1}{}
\markboth{\textcolor{darkblue}{\textbf{\ipa{lo˩ko˧}}}}{}
\textcolor{teal}{\zh{名词}} \hspace{4pt} \zh{声调类:} LM.
\zh{煮饭或煮汤的锣锅。在过去,锣锅一般是铜做的。} \textcolor{Sepia}{\selectlanguage{english}Pot for cooking rice, soup...; used to be made of copper.} \textcolor{PineGreen}{\selectlanguage{french}Casserole, pour cuire les céréales, les légumes, les soupes... Elle était autrefois en cuivre.}  \zh{【借词】} \zh{锣锅}
 ¶ \textcolor{darkblue}{\textbf{\ipa{lo˩ko˧: | hɑ˧ tɕɤ˩-di˩! | æ˧-v̩˧, | dʑɯ˩-kʰɯ˩-di˩˥! | ʈʂʰɤ˧ho˥, | dʑɯ˩ tɕɯ˩-di˩˥! |}}} \zh{锣锅,是用来煮饭的!铜锅,是放水用的!水壶,是来煮水的!(描写永宁二十世纪中使用的三种锅)} \textcolor{Sepia}{\selectlanguage{english}The cooking pot is for cooking cereals! The copper pot is for putting water! The boiler is for boiling water! (A summary of the respective uses of the three types of pots in use in Yongning about the middle of the 20th century.)} \textcolor{PineGreen}{\selectlanguage{french}La casserole (\textcolor{darkblue}{\textbf{\ipa{/lo˩ko˧/}}}), ça sert à cuire la nourriture! La casserole de cuivre (\textcolor{darkblue}{\textbf{\ipa{/æ˧-v̩˧/}}}), ça sert à mettre de l'eau! La bouilloire (\textcolor{darkblue}{\textbf{\ipa{/ʈʂʰɤ˩ho˥/}}}), ça sert à faire bouillir l'eau! (Résumé des emplois des trois sortes de casseroles qui étaient en usage à Yongning vers le milieu du XXe siècle.)}  
 \zh{量词}: \textcolor{darkblue}{\textbf{\ipa{ɭɯ˧}}} 
\lhead{\firstmark}
\rhead{\botmark}

\subsection{\hspace{-0.5cm} {\Large \textcolor{darkblue}{\textbf{\ipa{lo˩mi˧}}}}\hspace{0.5cm}[\kern2pt{\textcolor{darkblue}{\textbf{\ipa{lo˩mi˥}}}}\kern2pt]} \hypertarget{lo\string_Bmi\string_M1}{}
\markboth{\textcolor{darkblue}{\textbf{\ipa{lo˩mi˧}}}}{}
\textcolor{teal}{\zh{名词}} \hspace{4pt} \zh{声调类:} LM.
\zh{大拇指。} \textcolor{Sepia}{\selectlanguage{english}Thumb.} \textcolor{PineGreen}{\selectlanguage{french}Pouce.}  \zh{量词}: \textcolor{darkblue}{\textbf{\ipa{ɭɯ˧}}} 
\lhead{\firstmark}
\rhead{\botmark}

\subsection{\hspace{-0.5cm} {\Large \textcolor{darkblue}{\textbf{\ipa{lo˩mi˧-qɑ˩}}}}\hspace{0.5cm}[\kern2pt{\textcolor{darkblue}{\textbf{\ipa{lo˩mi˧qɑ˧}}}}\kern2pt]} \hypertarget{lo\string_Bmi\string_M-qA\string_B1}{}
\markboth{\textcolor{darkblue}{\textbf{\ipa{lo˩mi˧-qɑ˩}}}}{}
\textcolor{teal}{\zh{名词}} \hspace{4pt} \zh{声调类:} LM-L.
\zh{虎口。} \textcolor{Sepia}{\selectlanguage{english}Space between thumb and index finger.} \textcolor{PineGreen}{\selectlanguage{french}Espace entre le pouce et l'index.}  \zh{量词}: \textcolor{darkblue}{\textbf{\ipa{ɭɯ˧}}} 
\lhead{\firstmark}
\rhead{\botmark}

\subsection{\hspace{-0.5cm} {\Large \textcolor{darkblue}{\textbf{\ipa{lo˩pv̩˧˥}}}}\hspace{0.5cm}[\kern2pt{\textcolor{darkblue}{\textbf{\ipa{lo˩pv̩˧˥}}}}\kern2pt]} \hypertarget{lo\string_Bpv\string_=\string_M\string_T1}{}
\markboth{\textcolor{darkblue}{\textbf{\ipa{lo˩pv̩˧˥}}}}{}
\textcolor{teal}{\zh{名词}} \hspace{4pt} \zh{声调类:} LM+MH\#.
\zh{戒指。} \textcolor{Sepia}{\selectlanguage{english}Ring.} \textcolor{PineGreen}{\selectlanguage{french}Anneau.}  ¶ \textcolor{darkblue}{\textbf{\ipa{ŋv̩˩-lo˩pv̩˩}}} \zh{银戒指} \textcolor{Sepia}{\selectlanguage{english}silver ring} \textcolor{PineGreen}{\selectlanguage{french}anneau en argent}  
 ¶ \textcolor{darkblue}{\textbf{\ipa{hæ̃˩-lo˩pv̩˩}}} \zh{金戒指} \textcolor{Sepia}{\selectlanguage{english}gold ring} \textcolor{PineGreen}{\selectlanguage{french}anneau en or}  
 \zh{量词}: \textcolor{darkblue}{\textbf{\ipa{ɭɯ˧}}} 
\lhead{\firstmark}
\rhead{\botmark}

\subsection{\hspace{-0.5cm} {\Large \textcolor{darkblue}{\textbf{\ipa{lo˩qʰv̩˩}}}}\hspace{0.5cm}[\kern2pt{\textcolor{darkblue}{\textbf{\ipa{lo˩qʰv̩˩˥}}}}\kern2pt]} \hypertarget{lo\string_Bq\string_hv\string_=\string_B1}{}
\markboth{\textcolor{darkblue}{\textbf{\ipa{lo˩qʰv̩˩}}}}{}
\textcolor{teal}{\zh{名词}} \hspace{4pt} \zh{声调类:} L.
\zh{山沟。} \textcolor{Sepia}{\selectlanguage{english}Gully; ravine; valley.} \textcolor{PineGreen}{\selectlanguage{french}Vallée, gorge, ravin.}  \zh{量词}: \textcolor{darkblue}{\textbf{\ipa{lo˩}}} 
\lhead{\firstmark}
\rhead{\botmark}

\subsection{\hspace{-0.5cm} {\Large \textcolor{darkblue}{\textbf{\ipa{lo˩qʰwɤ˧}}}}\hspace{0.5cm}[\kern2pt{\textcolor{darkblue}{\textbf{\ipa{lo˩qʰwɤ˥}}}}\kern2pt]} \hypertarget{lo\string_Bq\string_hw7\string_M1}{}
\markboth{\textcolor{darkblue}{\textbf{\ipa{lo˩qʰwɤ˧}}}}{}
\textcolor{teal}{\zh{名词}} \hspace{4pt} \zh{声调类:} LM.
\ding{202} \zh{胳膊。} \textcolor{Sepia}{\selectlanguage{english}Arm.} \textcolor{PineGreen}{\selectlanguage{french}Bras.}  ¶ \textcolor{darkblue}{\textbf{\ipa{lo˩qʰwɤ˧ li˧}}} \zh{看胳膊} \textcolor{Sepia}{\selectlanguage{english}to look at (the) arm} \textcolor{PineGreen}{\selectlanguage{french}regarder le bras}  
 \zh{量词}: \textcolor{darkblue}{\textbf{\ipa{pʰo˧˥}}} \ding{203} \zh{手。} \textcolor{Sepia}{\selectlanguage{english}Hand.} \textcolor{PineGreen}{\selectlanguage{french}Main.}  ¶ \textcolor{darkblue}{\textbf{\ipa{lo˩qʰwɤ˧ ʈʂʰæ˧}}} \zh{洗手} \textcolor{Sepia}{\selectlanguage{english}to wash one's hands} \textcolor{PineGreen}{\selectlanguage{french}se laver les mains}  

\lhead{\firstmark}
\rhead{\botmark}

\subsection{\hspace{-0.5cm} {\Large \textcolor{darkblue}{\textbf{\ipa{lo˩qʰwɤ˧-kʰɯ˧ʑi˧˥}}}}\hspace{0.5cm}[\kern2pt{\textcolor{darkblue}{\textbf{\ipa{xxxx non-correspondance entre le nombre de morphèmes et le nombre de tons de morphèmes}}}}\kern2pt]} \hypertarget{lo\string_Bq\string_hw7\string_M-k\string_hM\string_Mz£i\string_M\string_T1}{}
\markboth{\textcolor{darkblue}{\textbf{\ipa{lo˩qʰwɤ˧-kʰɯ˧ʑi˧˥}}}}{}
\textcolor{teal}{\zh{名词}} \hspace{4pt} \zh{声调类:} LM+MH\#.
\zh{手套。} \textcolor{Sepia}{\selectlanguage{english}Glove.} \textcolor{PineGreen}{\selectlanguage{french}Gant.} 
\lhead{\firstmark}
\rhead{\botmark}

\subsection{\hspace{-0.5cm} {\Large \textcolor{darkblue}{\textbf{\ipa{lo˩ʁwæ\#˥}}}}\hspace{0.5cm}[\kern2pt{\textcolor{darkblue}{\textbf{\ipa{lo˩ʁwæ˥}}}}\kern2pt]} \hypertarget{lo\string_BRw\{\#\string_T1}{}
\markboth{\textcolor{darkblue}{\textbf{\ipa{lo˩ʁwæ\#˥}}}}{}
\textcolor{teal}{\zh{名词}} \hspace{4pt} \zh{声调类:} LM+\#H.
\zh{左撇子。} \textcolor{Sepia}{\selectlanguage{english}Left-handed person.} \textcolor{PineGreen}{\selectlanguage{french}Gaucher.} 
\lhead{\firstmark}
\rhead{\botmark}

\subsection{\hspace{-0.5cm} {\Large \textcolor{darkblue}{\textbf{\ipa{lo˩tʰo˧}}}}\hspace{0.5cm}[\kern2pt{\textcolor{darkblue}{\textbf{\ipa{lo˩tʰo˥}}}}\kern2pt]} \hypertarget{lo\string_Bt\string_ho\string_M1}{}
\markboth{\textcolor{darkblue}{\textbf{\ipa{lo˩tʰo˧}}}}{}
\textcolor{teal}{\zh{名词}} \hspace{4pt} \zh{声调类:} LM.
\zh{手铐。} \textcolor{Sepia}{\selectlanguage{english}Handcuffs, chains to tie a criminal's hands.} \textcolor{PineGreen}{\selectlanguage{french}Menottes: chaîne de fer pour attacher les mains d'un criminel.}  ¶ \textcolor{darkblue}{\textbf{\ipa{lo˩tʰo˧ kʰɯ˧˥}}} \zh{戴上手铐} \textcolor{Sepia}{\selectlanguage{english}to put handcuffs, to put on chains (on someone's hands)} \textcolor{PineGreen}{\selectlanguage{french}passer les menottes à quelqu'un}  

\lhead{\firstmark}
\rhead{\botmark}

\subsection{\hspace{-0.5cm} {\Large \textcolor{darkblue}{\textbf{\ipa{lo˩tsʰɯ˥-sɑ˩}}}}\hspace{0.5cm}[\kern2pt{\textcolor{darkblue}{\textbf{\ipa{lo˩tsʰɯ˥sɑ˧}}}}\kern2pt]} \hypertarget{lo\string_Bts\string_hM\string_T-sA\string_B1}{}
\markboth{\textcolor{darkblue}{\textbf{\ipa{lo˩tsʰɯ˥-sɑ˩}}}}{}
\textcolor{teal}{\zh{名词}} \hspace{4pt} \zh{声调类:} LH-.
\zh{牲畜前腿的肉。} \textcolor{Sepia}{\selectlanguage{english}Meat of the anterior limbs of cattle.} \textcolor{PineGreen}{\selectlanguage{french}Viande des membres antérieurs.} 
\lhead{\firstmark}
\rhead{\botmark}

\subsection{\hspace{-0.5cm} {\Large \textcolor{darkblue}{\textbf{\ipa{lo˩ʈv̩˧}}}}\hspace{0.5cm}[\kern2pt{\textcolor{darkblue}{\textbf{\ipa{lo˩ʈv̩˥}}}}\kern2pt]} \hypertarget{lo\string_Bt`v\string_=\string_M1}{}
\markboth{\textcolor{darkblue}{\textbf{\ipa{lo˩ʈv̩˧}}}}{}
\textcolor{teal}{\zh{名词}} \hspace{4pt} \zh{声调类:} LM.
\zh{拳。} \textcolor{Sepia}{\selectlanguage{english}Fist.} \textcolor{PineGreen}{\selectlanguage{french}Poing.}  \zh{量词}: \textcolor{darkblue}{\textbf{\ipa{ʈv̩˩}}} 
\lhead{\firstmark}
\rhead{\botmark}

\subsection{\hspace{-0.5cm} {\Large \textcolor{darkblue}{\textbf{\ipa{lo˩ʈʰɯ˧}}}}\hspace{0.5cm}[\kern2pt{\textcolor{darkblue}{\textbf{\ipa{lo˩ʈʰɯ˥}}}}\kern2pt]} \hypertarget{lo\string_Bt`\string_hM\string_M1}{}
\markboth{\textcolor{darkblue}{\textbf{\ipa{lo˩ʈʰɯ˧}}}}{}
\textcolor{teal}{\zh{名词}} \hspace{4pt} \zh{声调类:} LM.
\zh{肘。} \textcolor{Sepia}{\selectlanguage{english}Elbow.} \textcolor{PineGreen}{\selectlanguage{french}Coude.}  \zh{量词}: \textcolor{darkblue}{\textbf{\ipa{ʈv̩˩}}} 
\lhead{\firstmark}
\rhead{\botmark}

\subsection{\hspace{-0.5cm} {\Large \textcolor{darkblue}{\textbf{\ipa{lo˩ʈʂæ˧˥}}}}\hspace{0.5cm}[\kern2pt{\textcolor{darkblue}{\textbf{\ipa{lo˩ʈʂæ˧˥}}}}\kern2pt]} \hypertarget{lo\string_Bt`s`\{\string_M\string_T1}{}
\markboth{\textcolor{darkblue}{\textbf{\ipa{lo˩ʈʂæ˧˥}}}}{}
\textcolor{teal}{\zh{名词}} \hspace{4pt} \zh{声调类:} LM+MH\#.
\zh{手臂的关节(手腕、肘弯)。} \textcolor{Sepia}{\selectlanguage{english}Joints of the arm: wrist, elbow.} \textcolor{PineGreen}{\selectlanguage{french}Articulations du bras: le poignet, mais aussi le coude.}  \zh{量词}: \textcolor{darkblue}{\textbf{\ipa{ʈʂæ˧˥}}} 
\lhead{\firstmark}
\rhead{\botmark}

\subsection{\hspace{-0.5cm} {\Large \textcolor{darkblue}{\textbf{\ipa{lo˧˥}}} \textsubscript{1}}\hspace{0.5cm}[\kern2pt{\textcolor{darkblue}{\textbf{\ipa{lo˥}}}}\kern2pt]} \hypertarget{lo\string_M\string_T1}{}
\markboth{\textcolor{darkblue}{\textbf{\ipa{lo˧˥}}} \textsubscript{1}}{}
\textcolor{teal}{\zh{形容词}} \hspace{4pt} \zh{声调类:} MH.
\zh{厚。} \textcolor{Sepia}{\selectlanguage{english}Thick.} \textcolor{PineGreen}{\selectlanguage{french}Épais.}  ¶ \textcolor{darkblue}{\textbf{\ipa{ʈʂʰɯ˧ | lo˧-pæ˧-ɻæ˥-gv̩˩!}}} \zh{很厚啊!} \textcolor{Sepia}{\selectlanguage{english}It's really thick!} \textcolor{PineGreen}{\selectlanguage{french}c'est très épais!}  

\lhead{\firstmark}
\rhead{\botmark}

\subsection{\hspace{-0.5cm} {\Large \textcolor{darkblue}{\textbf{\ipa{lo˧˥}}} \textsubscript{2}}\hspace{0.5cm}[\kern2pt{\textcolor{darkblue}{\textbf{\ipa{lo˧˥}}}}\kern2pt]} \hypertarget{lo\string_M\string_T2}{}
\markboth{\textcolor{darkblue}{\textbf{\ipa{lo˧˥}}} \textsubscript{2}}{}
\textcolor{teal}{\zh{动词}} \hspace{4pt} \zh{声调类:} MH.
\zh{拱手作揖。} \textcolor{Sepia}{\selectlanguage{english}To join hands in an indication of submission.} \textcolor{PineGreen}{\selectlanguage{french}Joindre les mains en signe de soumission.}  ¶ \textcolor{darkblue}{\textbf{\ipa{tsʰɤ˧tsʰɤ˧ lo˧˥}}} \zh{拱手作揖} \textcolor{Sepia}{\selectlanguage{english}to join hands in an indication of submission} \textcolor{PineGreen}{\selectlanguage{french}rendre hommage à, joindre les mains en signe de soumission/respect}  
 ¶ \textcolor{darkblue}{\textbf{\ipa{tsʰɤ˧tsʰɤ˧ | le˧-lo˧-ze˥}}} \zh{\mytextsc{accomp} \string_ \mytextsc{pfv}} \textcolor{Sepia}{\selectlanguage{english}\mytextsc{accomp} \string_ \mytextsc{pfv}} \textcolor{PineGreen}{\selectlanguage{french}\mytextsc{accomp} \string_ \mytextsc{pfv}}  

\lhead{\firstmark}
\rhead{\botmark}

\subsection{\hspace{-0.5cm} {\Large \textcolor{darkblue}{\textbf{\ipa{*lo˩˧}}}}\hspace{0.5cm}[\kern2pt{\textcolor{darkblue}{\textbf{\ipa{lo˩˥}}}}\kern2pt]} \hypertarget{*lo\string_B\string_M1}{}
\markboth{\textcolor{darkblue}{\textbf{\ipa{*lo˩˧}}}}{}
\textcolor{teal}{\zh{名词}} \hspace{4pt} \zh{声调类:} LM.
\zh{大拇指(单音节,按照双音节词构拟出来的)。} \textcolor{Sepia}{\selectlanguage{english}Thumb.} \textcolor{PineGreen}{\selectlanguage{french}Pouce (forme reconstruite d'après le disyllabe).} 
\lhead{\firstmark}
\rhead{\botmark}

\subsection{\hspace{-0.5cm} {\Large \textcolor{darkblue}{\textbf{\ipa{lv̩˧}}} \textsubscript{1}}\hspace{0.5cm}[\kern2pt{\textcolor{darkblue}{\textbf{\ipa{lv̩˥}}}}\kern2pt]} \hypertarget{lv\string_=\string_M1}{}
\markboth{\textcolor{darkblue}{\textbf{\ipa{lv̩˧}}} \textsubscript{1}}{}
\textcolor{teal}{\zh{名词}} \hspace{4pt} \zh{声调类:} M.
\zh{田地。} \textcolor{Sepia}{\selectlanguage{english}Field.} \textcolor{PineGreen}{\selectlanguage{french}Champs.}  \zh{量词}: \textcolor{darkblue}{\textbf{\ipa{kɤ˧˥}}} 
\lhead{\firstmark}
\rhead{\botmark}

\subsection{\hspace{-0.5cm} {\Large \textcolor{darkblue}{\textbf{\ipa{lv̩˧}}} \textsubscript{2}}\hspace{0.5cm}[\kern2pt{\textcolor{darkblue}{\textbf{\ipa{lv̩˥}}}}\kern2pt]} \hypertarget{lv\string_=\string_M2}{}
\markboth{\textcolor{darkblue}{\textbf{\ipa{lv̩˧}}} \textsubscript{2}}{}
\textcolor{teal}{\zh{名词}} \hspace{4pt} \zh{声调类:} M.
\zh{喂给马或牛的粮食。} \textcolor{Sepia}{\selectlanguage{english}Cereals for horses or cattle.} \textcolor{PineGreen}{\selectlanguage{french}Grain (pour chevaux ou vaches), picotin.}  ¶ \textcolor{darkblue}{\textbf{\ipa{ʐwæ˧-lv̩˧}}} \zh{喂给马的粮食} \textcolor{Sepia}{\selectlanguage{english}cereals fed to horses; same meaning as \textcolor{darkblue}{\textbf{\ipa{/ʐwæ˧-ɭɯ\#˥/}}}} \textcolor{PineGreen}{\selectlanguage{french}grain pour pour cheval, picotin; même sens que: \textcolor{darkblue}{\textbf{\ipa{/ʐwæ˧-ɭɯ\#˥/}}}}  
\zh{~【参考】~} \hyperlink{}{\textcolor{darkblue}{\textbf{\ipa{ʐwæ˧-ɭɯ\#˥}}}} 
\lhead{\firstmark}
\rhead{\botmark}

\subsection{\hspace{-0.5cm} {\Large \textcolor{darkblue}{\textbf{\ipa{lv̩˧˥}}}}\hspace{0.5cm}[\kern2pt{\textcolor{darkblue}{\textbf{\ipa{lv̩˧˥}}}}\kern2pt]} \hypertarget{lv\string_=\string_M\string_T1}{}
\markboth{\textcolor{darkblue}{\textbf{\ipa{lv̩˧˥}}}}{}
\textcolor{teal}{\zh{名词}} \hspace{4pt} \zh{声调类:} MH.
\zh{蛆。} \textcolor{Sepia}{\selectlanguage{english}Maggot.} \textcolor{PineGreen}{\selectlanguage{french}Larve.} 
\lhead{\firstmark}
\rhead{\botmark}

\subsection{\hspace{-0.5cm} {\Large \textcolor{darkblue}{\textbf{\ipa{lv̩˧˥}}} \textsubscript{1}}\hspace{0.5cm}[\kern2pt{\textcolor{darkblue}{\textbf{\ipa{lv̩˧˥}}}}\kern2pt]} \hypertarget{lv\string_=\string_M\string_T1}{}
\markboth{\textcolor{darkblue}{\textbf{\ipa{lv̩˧˥}}} \textsubscript{1}}{}
\textcolor{teal}{\zh{动词}} \hspace{4pt} \zh{声调类:} MH.
\zh{放牧。} \textcolor{Sepia}{\selectlanguage{english}To herd.} \textcolor{PineGreen}{\selectlanguage{french}Garder les animaux, mener paître les animaux.}  ¶ \textcolor{darkblue}{\textbf{\ipa{go˩bo˥ lv̩˩}}} \zh{放牧牲畜} \textcolor{Sepia}{\selectlanguage{english}to graze cattle, to herd cattle} \textcolor{PineGreen}{\selectlanguage{french}mener paître le bétail, garder le bétail}  
 ¶ \textcolor{darkblue}{\textbf{\ipa{ʐwæ˧ lv̩˩}}} \zh{放马} \textcolor{Sepia}{\selectlanguage{english}to graze horses, to herd horses} \textcolor{PineGreen}{\selectlanguage{french}mener paître les chevaux}  
 ¶ \textcolor{darkblue}{\textbf{\ipa{ʝi˧ lv̩˩}}} \zh{放牛} \textcolor{Sepia}{\selectlanguage{english}to graze cows, to herd cows} \textcolor{PineGreen}{\selectlanguage{french}mener paître les vaches}  
 ¶ \textcolor{darkblue}{\textbf{\ipa{bo˩ lv̩˩˥}}} \zh{放猪} \textcolor{Sepia}{\selectlanguage{english}to herd pigs} \textcolor{PineGreen}{\selectlanguage{french}garder les cochons}  
 ¶ \textcolor{darkblue}{\textbf{\ipa{tsʰɯ˧ lv̩˥}}} \zh{放山羊} \textcolor{Sepia}{\selectlanguage{english}to graze goats, to herd goats} \textcolor{PineGreen}{\selectlanguage{french}mener paître les chèvres}  
 ¶ \textcolor{darkblue}{\textbf{\ipa{ɖɯ˧-hɤ˧ mɤ˧-lv̩˩\textasciitilde{}lv̩˩}}} \zh{懒,什么也不管} \textcolor{Sepia}{\selectlanguage{english}lazy, who does not take care of anything} \textcolor{PineGreen}{\selectlanguage{french}paresseux, qui ne s'occupe de rien}  

\lhead{\firstmark}
\rhead{\botmark}

\subsection{\hspace{-0.5cm} {\Large \textcolor{darkblue}{\textbf{\ipa{lv̩˧˥}}} \textsubscript{2}}\hspace{0.5cm}[\kern2pt{\textcolor{darkblue}{\textbf{\ipa{lv̩˧˥}}}}\kern2pt]} \hypertarget{lv\string_=\string_M\string_T2}{}
\markboth{\textcolor{darkblue}{\textbf{\ipa{lv̩˧˥}}} \textsubscript{2}}{}
\textcolor{teal}{\zh{动词}} \hspace{4pt} \zh{声调类:} MH.
\zh{逃跑,逃掉。} \textcolor{Sepia}{\selectlanguage{english}To escape, to flee.} \textcolor{PineGreen}{\selectlanguage{french}S'enfuir.} 
\lhead{\firstmark}
\rhead{\botmark}

\subsection{\hspace{-0.5cm} {\Large \textcolor{darkblue}{\textbf{\ipa{lv̩˥}}}}\hspace{0.5cm}[\kern2pt{\textcolor{darkblue}{\textbf{\ipa{lv̩˥}}}}\kern2pt]} \hypertarget{lv\string_=\string_T1}{}
\markboth{\textcolor{darkblue}{\textbf{\ipa{lv̩˥}}}}{}
\textcolor{teal}{\zh{动词}} \hspace{4pt} \zh{声调类:} H.
\zh{缠(线……)、裹(毡子……)。} \textcolor{Sepia}{\selectlanguage{english}To wind, to coil, to wrap.} \textcolor{PineGreen}{\selectlanguage{french}Enrouler (du fil); emballer.}  ¶ \textcolor{darkblue}{\textbf{\ipa{le˧-qo˥-lv̩˩}}} \zh{裹起来} \textcolor{Sepia}{\selectlanguage{english}to wrap, to coil} \textcolor{PineGreen}{\selectlanguage{french}enrouler}  
 ¶ \textcolor{darkblue}{\textbf{\ipa{kʰɯ˧ qo˧-lv̩˥}}} \zh{缠线} \textcolor{Sepia}{\selectlanguage{english}to wind a thread} \textcolor{PineGreen}{\selectlanguage{french}enrouler du fil}  
 ¶ \textcolor{darkblue}{\textbf{\ipa{qo˧-lv̩˩}}} \zh{裹} \textcolor{Sepia}{\selectlanguage{english}to wrap, to coil} \textcolor{PineGreen}{\selectlanguage{french}même sens: enrouler}  

\lhead{\firstmark}
\rhead{\botmark}

\subsection{\hspace{-0.5cm} {\Large \textcolor{darkblue}{\textbf{\ipa{lv̩˩\textsubscript{a}}}} \textsubscript{1}}\hspace{0.5cm}[\kern2pt{\textcolor{darkblue}{\textbf{\ipa{lv̩˧˥}}}}\kern2pt]} \hypertarget{lv\string_=\string_Ba1}{}
\markboth{\textcolor{darkblue}{\textbf{\ipa{lv̩˩\textsubscript{a}}}} \textsubscript{1}}{}
\textcolor{teal}{\zh{动词}} \hspace{4pt} \zh{声调类:} L\textsubscript{a}.
\zh{狗吠。} \textcolor{Sepia}{\selectlanguage{english}To bark (a dog barks).} \textcolor{PineGreen}{\selectlanguage{french}Aboyer.}  ¶ \textcolor{darkblue}{\textbf{\ipa{kʰv̩˩mi˩ lv̩˥ |}}} \zh{狗吠} \textcolor{Sepia}{\selectlanguage{english}the dog barks} \textcolor{PineGreen}{\selectlanguage{french}le chien aboie}  
 ¶ \textcolor{darkblue}{\textbf{\ipa{kʰv̩˩ lv̩˥-dʑo˩ |}}} \zh{狗在叫} \textcolor{Sepia}{\selectlanguage{english}the dog is barking} \textcolor{PineGreen}{\selectlanguage{french}le chien est en train d'aboyer}  
 ¶ \textcolor{darkblue}{\textbf{\ipa{ɖɯ˧-lv̩˧\textasciitilde{}lv̩˥-ɻ̍˩}}} \zh{叫一叫} \textcolor{Sepia}{\selectlanguage{english}\mytextsc{delimitative} \string_ \mytextsc{red} \mytextsc{inceptive}} \textcolor{PineGreen}{\selectlanguage{french}\mytextsc{délimitatif} \string_ \mytextsc{red} \mytextsc{inchoatif}}  

\lhead{\firstmark}
\rhead{\botmark}

\subsection{\hspace{-0.5cm} {\Large \textcolor{darkblue}{\textbf{\ipa{lv̩˩\textsubscript{a}}}} \textsubscript{2}}\hspace{0.5cm}[\kern2pt{\textcolor{darkblue}{\textbf{\ipa{lv̩˩˥}}}}\kern2pt]} \hypertarget{lv\string_=\string_Ba2}{}
\markboth{\textcolor{darkblue}{\textbf{\ipa{lv̩˩\textsubscript{a}}}} \textsubscript{2}}{}
\textcolor{teal}{\zh{动词}} \hspace{4pt} \zh{声调类:} L\textsubscript{a}.
\zh{把布卷起来。} \textcolor{Sepia}{\selectlanguage{english}To roll, to coil (fabric).} \textcolor{PineGreen}{\selectlanguage{french}Enrouler (un tissu).}  ¶ \textcolor{darkblue}{\textbf{\ipa{le˧-qæ˥-lv̩˩}}} \zh{卷起来} \textcolor{Sepia}{\selectlanguage{english}to coil} \textcolor{PineGreen}{\selectlanguage{french}enrouler}  
 ¶ \textcolor{darkblue}{\textbf{\ipa{le˧-lv̩˧\textasciitilde{}lv̩˧}}} \textcolor{PineGreen}{\selectlanguage{french}\mytextsc{accomp} \mytextsc{red}}  
 ¶ \textcolor{darkblue}{\textbf{\ipa{tso˧\textasciitilde{}tso˧ lv̩˧\textasciitilde{}lv̩˧}}} \zh{卷东西} \textcolor{Sepia}{\selectlanguage{english}to coil things} \textcolor{PineGreen}{\selectlanguage{french}enrouler des choses}  
 ¶ \textcolor{darkblue}{\textbf{\ipa{ɖɯ˧-kʰwɤ˧ lv̩˥}}} \zh{卷一块(东西)} \textcolor{Sepia}{\selectlanguage{english}to coil something} \textcolor{PineGreen}{\selectlanguage{french}enrouler quelque chose}  

\lhead{\firstmark}
\rhead{\botmark}

\subsection{\hspace{-0.5cm} {\Large \textcolor{darkblue}{\textbf{\ipa{lv̩˩\textsubscript{a}}}} \textsubscript{3}}\hspace{0.5cm}[\kern2pt{\textcolor{darkblue}{\textbf{\ipa{lv̩˩˥}}}}\kern2pt]} \hypertarget{lv\string_=\string_Ba3}{}
\markboth{\textcolor{darkblue}{\textbf{\ipa{lv̩˩\textsubscript{a}}}} \textsubscript{3}}{}
\textcolor{teal}{\zh{动词}} \hspace{4pt} \zh{声调类:} L\textsubscript{a}.
\zh{耕种。} \textcolor{Sepia}{\selectlanguage{english}To plough, to till.} \textcolor{PineGreen}{\selectlanguage{french}Labourer.}  ¶ \textcolor{darkblue}{\textbf{\ipa{le˧-lv̩˩-ze˩}}} \zh{耕种了} \textcolor{Sepia}{\selectlanguage{english}\mytextsc{accomp} \string_ \mytextsc{pfv}} \textcolor{PineGreen}{\selectlanguage{french}\mytextsc{accomp} \string_ \mytextsc{pfv}}  
 ¶ \textcolor{darkblue}{\textbf{\ipa{ʝi˧-lv̩˧˥}}} \zh{耕种} \textcolor{Sepia}{\selectlanguage{english}to plough} \textcolor{PineGreen}{\selectlanguage{french}labourer}  
 ¶ \textcolor{darkblue}{\textbf{\ipa{dʑi˧mi˧ lv̩˧˥ / dʑi˧mi˧ lv̩˧-ze˥}}} \zh{用水牛耕田} \textcolor{Sepia}{\selectlanguage{english}to plough with a water buffalo} \textcolor{PineGreen}{\selectlanguage{french}labourer avec un buffle}  
 ¶ \textcolor{darkblue}{\textbf{\ipa{ʝi˧ ɖɯ˧-lv̩˧\textasciitilde{}lv̩˥-ɻ̍˩}}} \zh{耕一耕} \textcolor{Sepia}{\selectlanguage{english}to plough a little} \textcolor{PineGreen}{\selectlanguage{french}labourer un peu}  

\lhead{\firstmark}
\rhead{\botmark}

\subsection{\hspace{-0.5cm} {\Large \textcolor{darkblue}{\textbf{\ipa{lv̩˩\textsubscript{a}}}} \textsubscript{4}}\hspace{0.5cm}[\kern2pt{\textcolor{darkblue}{\textbf{\ipa{lv̩˩˥}}}}\kern2pt]} \hypertarget{lv\string_=\string_Ba4}{}
\markboth{\textcolor{darkblue}{\textbf{\ipa{lv̩˩\textsubscript{a}}}} \textsubscript{4}}{}
\textcolor{teal}{\zh{动词}} \hspace{4pt} \zh{声调类:} L\textsubscript{a}.
\zh{足够。} \textcolor{Sepia}{\selectlanguage{english}To suffice, to be enough.} \textcolor{PineGreen}{\selectlanguage{french}Suffire.}  ¶ \textcolor{darkblue}{\textbf{\ipa{ə˩-lv̩˩˥? / ə˩-lv̩˩-ze˥?}}} \zh{够了吗?} \textcolor{Sepia}{\selectlanguage{english}Is it enough? Is it sufficient?} \textcolor{PineGreen}{\selectlanguage{french}est-ce que ça (te) suffit ?}  

\lhead{\firstmark}
\rhead{\botmark}

\subsection{\hspace{-0.5cm} {\Large \textcolor{darkblue}{\textbf{\ipa{lv̩˧dʑɯ˥}}}}\hspace{0.5cm}[\kern2pt{\textcolor{darkblue}{\textbf{\ipa{lv̩˧dʑɯ˧}}}}\kern2pt]} \hypertarget{lv\string_=\string_Mdz£M\string_T1}{}
\markboth{\textcolor{darkblue}{\textbf{\ipa{lv̩˧dʑɯ˥}}}}{}
\textcolor{teal}{\zh{名词}} \hspace{4pt} \zh{声调类:} H\#.
\zh{零碎的石头块。} \textcolor{Sepia}{\selectlanguage{english}Stone chips, little slabs of stone.} \textcolor{PineGreen}{\selectlanguage{french}Éclats de pierre, débris de pierre, petits bouts de pierre (ne veut pas dire “sable”).}  \zh{量词}: \textcolor{darkblue}{\textbf{\ipa{ʈʂwɤ˧}}} 
\lhead{\firstmark}
\rhead{\botmark}

\subsection{\hspace{-0.5cm} {\Large \textcolor{darkblue}{\textbf{\ipa{lv̩˩ʝi˧}}}}\hspace{0.5cm}[\kern2pt{\textcolor{darkblue}{\textbf{\ipa{lv̩˧ʝi˧}}}}\kern2pt]} \hypertarget{lv\string_=\string_Bj££i\string_M1}{}
\markboth{\textcolor{darkblue}{\textbf{\ipa{lv̩˩ʝi˧}}}}{}
\textcolor{teal}{\zh{动词}} \hspace{4pt} \zh{声调类:} LM.
\zh{录音(汉语借词)。} \textcolor{Sepia}{\selectlanguage{english}To record sound.} \textcolor{PineGreen}{\selectlanguage{french}Enregistrer.}  \zh{【借词】} \zh{录音}
 ¶ \textcolor{darkblue}{\textbf{\ipa{hɑ˧ le˧-dzɯ˧-se˥, | lv̩˩ ʝi˧-bi˧ !}}} \zh{吃完饭,就录音吧! / 吃完饭就可以录音了!} \textcolor{Sepia}{\selectlanguage{english}After the meal, we'll do a recording!} \textcolor{PineGreen}{\selectlanguage{french}Quand (on) aura fini de manger, (on) fera un enregistrement !}  

\lhead{\firstmark}
\rhead{\botmark}

\subsection{\hspace{-0.5cm} {\Large \textcolor{darkblue}{\textbf{\ipa{lv̩˧mi˧}}}}\hspace{0.5cm}[\kern2pt{\textcolor{darkblue}{\textbf{\ipa{lv̩˩mi˥}}}}\kern2pt]} \hypertarget{lv\string_=\string_Mmi\string_M1}{}
\markboth{\textcolor{darkblue}{\textbf{\ipa{lv̩˧mi˧}}}}{}
\textcolor{teal}{\zh{名词}} \hspace{4pt} \zh{声调类:} M.
\zh{石头。} \textcolor{Sepia}{\selectlanguage{english}Stone.} \textcolor{PineGreen}{\selectlanguage{french}Pierre.}  ¶ \textcolor{darkblue}{\textbf{\ipa{kʰv̩˧pʰæ˧tɕi˩, | lv̩˧mi˧ dzɯ˧-bi˧-ʁo˧-ho˩!}}} \zh{‘年轻人,石头都能吃!’(意思:年轻人消化好,吃什么都行,而人变老就不那么容易消化了,要注意吃什么。)} \textcolor{Sepia}{\selectlanguage{english}'When one is young, one could eat stones!' (Meaning: when one is young, one can eat anything, one has an excellent digestion; as one gets old, one is less tolerant of food that is not easy to digest.)} \textcolor{PineGreen}{\selectlanguage{french}'Quand on est jeune, on mangerait des pierres!' (Signification: quand on est jeune, on mange de tout, on a une digestion solide; tandis que quand on est vieux, on a facilement mal au ventre, dès qu'on mange quelque chose d'un peu indigeste, une nourriture “trop dure”.)}  
 \zh{量词}: \textcolor{darkblue}{\textbf{\ipa{ɭɯ˧}}} 
\lhead{\firstmark}
\rhead{\botmark}

\subsection{\hspace{-0.5cm} {\Large \textcolor{darkblue}{\textbf{\ipa{lv̩˧mi˧-bo\#˥}}}}\hspace{0.5cm}[\kern2pt{\textcolor{darkblue}{\textbf{\ipa{xxxx non-correspondance entre le nombre de morphèmes et le nombre de tons de morphèmes}}}}\kern2pt]} \hypertarget{lv\string_=\string_Mmi\string_M-bo\#\string_T1}{}
\markboth{\textcolor{darkblue}{\textbf{\ipa{lv̩˧mi˧-bo\#˥}}}}{}
\textcolor{teal}{\zh{名词}} \hspace{4pt} \zh{声调类:} \#H.
\zh{石墙。} \textcolor{Sepia}{\selectlanguage{english}Stone wall.} \textcolor{PineGreen}{\selectlanguage{french}Mur en pierre.}  \zh{量词}: \textcolor{darkblue}{\textbf{\ipa{ɭɯ˧}}} 
\lhead{\firstmark}
\rhead{\botmark}

\subsection{\hspace{-0.5cm} {\Large \textcolor{darkblue}{\textbf{\ipa{lv̩˧mi˧-dʑɯ˧dʑɯ˩}}}}\hspace{0.5cm}[\kern2pt{\textcolor{darkblue}{\textbf{\ipa{xxxx non-correspondance entre le nombre de morphèmes et le nombre de tons de morphèmes}}}}\kern2pt]} \hypertarget{lv\string_=\string_Mmi\string_M-dz£M\string_Mdz£M\string_B1}{}
\markboth{\textcolor{darkblue}{\textbf{\ipa{lv̩˧mi˧-dʑɯ˧dʑɯ˩}}}}{}
\textcolor{teal}{\zh{名词}} \hspace{4pt} \zh{声调类:} \mytextsc{L}\#.
\zh{零碎的石块。} \textcolor{Sepia}{\selectlanguage{english}Little slabs of stone, stone chips.} \textcolor{PineGreen}{\selectlanguage{french}Éclats de pierre, débris de pierre, petits bouts de pierre (ne veut pas dire “sable”).}  \zh{量词}: \textcolor{darkblue}{\textbf{\ipa{kʰwɤ˥}}} 
\lhead{\firstmark}
\rhead{\botmark}

\subsection{\hspace{-0.5cm} {\Large \textcolor{darkblue}{\textbf{\ipa{lv̩˧pʰv̩˩}}} \textsubscript{2}}\hspace{0.5cm}[\kern2pt{\textcolor{darkblue}{\textbf{\ipa{lv̩˧pʰv̩˩}}}}\kern2pt]} \hypertarget{lv\string_=\string_Mp\string_hv\string_=\string_B2}{}
\markboth{\textcolor{darkblue}{\textbf{\ipa{lv̩˧pʰv̩˩}}} \textsubscript{2}}{}
\textcolor{teal}{\zh{名词}} \hspace{4pt} \zh{声调类:} L\#.
\zh{水田。} \textcolor{Sepia}{\selectlanguage{english}Paddy field.} \textcolor{PineGreen}{\selectlanguage{french}Champs inondés.}  \zh{量词}: \textcolor{darkblue}{\textbf{\ipa{pʰv̩˩}}} 
\lhead{\firstmark}
\rhead{\botmark}

\subsection{\hspace{-0.5cm} {\Large \textcolor{darkblue}{\textbf{\ipa{lv̩˧qæ\#˥}}}}\hspace{0.5cm}[\kern2pt{\textcolor{darkblue}{\textbf{\ipa{xxxx groupe tonal entier sans aucun ton}}}}\kern2pt]} \hypertarget{lv\string_=\string_Mq\{\#\string_T1}{}
\markboth{\textcolor{darkblue}{\textbf{\ipa{lv̩˧qæ\#˥}}}}{}
\textcolor{teal}{\zh{名词}} \hspace{4pt} \zh{声调类:} \#˥.
\zh{地界:不同家庭田地之间的界限。} \textcolor{Sepia}{\selectlanguage{english}Limit, boundary between fields belonging to different families. It is typically materialized by a small dike.} \textcolor{PineGreen}{\selectlanguage{french}Limite de propriété: limite entre les champs appartenant à des familles différentes. Elle est souvent matérialisée par une diguette.} 
\lhead{\firstmark}
\rhead{\botmark}

\subsection{\hspace{-0.5cm} {\Large \textcolor{darkblue}{\textbf{\ipa{lv̩˧sɯ˥}}}}\hspace{0.5cm}[\kern2pt{\textcolor{darkblue}{\textbf{\ipa{lv̩˧sɯ˥}}}}\kern2pt]} \hypertarget{lv\string_=\string_MsM\string_T1}{}
\markboth{\textcolor{darkblue}{\textbf{\ipa{lv̩˧sɯ˥}}}}{}
\textcolor{teal}{\zh{名词}} \hspace{4pt} \zh{声调类:} H\#.
\zh{傈僳族。} \textcolor{Sepia}{\selectlanguage{english}Lisu (ethnic group).} \textcolor{PineGreen}{\selectlanguage{french}Lisu (groupe ethnique).}  \zh{量词}: \textcolor{darkblue}{\textbf{\ipa{v̩˧}}} 
\lhead{\firstmark}
\rhead{\botmark}

\subsection{\hspace{-0.5cm} {\Large \textcolor{darkblue}{\textbf{\ipa{lv̩˩tɕʰɯ˧}}}}\hspace{0.5cm}[\kern2pt{\textcolor{darkblue}{\textbf{\ipa{lv̩˩tɕʰɯ˥}}}}\kern2pt]} \hypertarget{lv\string_=\string_Bts£\string_hM\string_M1}{}
\markboth{\textcolor{darkblue}{\textbf{\ipa{lv̩˩tɕʰɯ˧}}}}{}
\textcolor{teal}{\zh{名词}} \hspace{4pt} \zh{声调类:} LM.
\zh{六区,今奉科乡(汉语借词)。} \textcolor{Sepia}{\selectlanguage{english}The village of Fengke (close to the Yangtze river): this is the former name of the area in Chinese.} \textcolor{PineGreen}{\selectlanguage{french}Fengke: nom chinois ancien du village de Fengke, au bord du Yangtze.}  \zh{【借词】} \zh{六区}

\lhead{\firstmark}
\rhead{\botmark}

\subsection{\hspace{-0.5cm} {\Large \textcolor{darkblue}{\textbf{\ipa{lv̩˩tɕʰɯ˧-hĩ\#˥}}}}\hspace{0.5cm}[\kern2pt{\textcolor{darkblue}{\textbf{\ipa{xxxx non-correspondance entre le nombre de morphèmes et le nombre de tons de morphèmes}}}}\kern2pt]} \hypertarget{lv\string_=\string_Bts£\string_hM\string_M-hi\string_~\#\string_T1}{}
\markboth{\textcolor{darkblue}{\textbf{\ipa{lv̩˩tɕʰɯ˧-hĩ\#˥}}}}{}
\textcolor{teal}{\zh{名词}} \hspace{4pt} \zh{声调类:} LM+\#H.
\zh{奉科的人。} \textcolor{Sepia}{\selectlanguage{english}The inhabitants of the village of Fengke (Fv-kho).} \textcolor{PineGreen}{\selectlanguage{french}Gens de Fengke (Fv-kho).}  \zh{【借词】} \zh{六区}

\lhead{\firstmark}
\rhead{\botmark}

\subsection{\hspace{-0.5cm} {\Large \textcolor{darkblue}{\textbf{\ipa{lv̩˧tsɯ˥}}}}\hspace{0.5cm}[\kern2pt{\textcolor{darkblue}{\textbf{\ipa{lv̩˧tsɯ˥}}}}\kern2pt]} \hypertarget{lv\string_=\string_MtsM\string_T1}{}
\markboth{\textcolor{darkblue}{\textbf{\ipa{lv̩˧tsɯ˥}}}}{}
\textcolor{teal}{\zh{名词}} \hspace{4pt} \zh{声调类:} H\#.
\zh{炉子(汉语借词)。} \textcolor{Sepia}{\selectlanguage{english}Oven.} \textcolor{PineGreen}{\selectlanguage{french}Four.}  \zh{【借词】} \zh{炉子}
 \zh{量词}: \textcolor{darkblue}{\textbf{\ipa{nɑ˧}}} 
\lhead{\firstmark}
\rhead{\botmark}

\subsection{\hspace{-0.5cm} {\Large \textcolor{darkblue}{\textbf{\ipa{lv̩˩\textasciitilde{}lv̩˧˥}}}}\hspace{0.5cm}[\kern2pt{\textcolor{darkblue}{\textbf{\ipa{lv̩˧lv̩˧˥}}}}\kern2pt]} \hypertarget{lv\string_=\string_B~lv\string_=\string_M\string_T1}{}
\markboth{\textcolor{darkblue}{\textbf{\ipa{lv̩˩\textasciitilde{}lv̩˧˥}}}}{}
\textcolor{teal}{\zh{动词}} \hspace{4pt} \zh{声调类:} MH.
\zh{动(虫、桌子、小孩子动)。} \textcolor{Sepia}{\selectlanguage{english}To move.} \textcolor{PineGreen}{\selectlanguage{french}Bouger, faire des mouvements.}  ¶ \textcolor{darkblue}{\textbf{\ipa{lv̩˩\textasciitilde{}lv̩˧-ze˥}}} \zh{动了} \textcolor{Sepia}{\selectlanguage{english}\mytextsc{pfv}} \textcolor{PineGreen}{\selectlanguage{french}\mytextsc{pfv}}  
 ¶ \textcolor{darkblue}{\textbf{\ipa{tʰi˧-lv̩˩\textasciitilde{}lv̩˩(-ze˩)}}} \zh{\mytextsc{dur} \mytextsc{red}} \textcolor{Sepia}{\selectlanguage{english}\mytextsc{dur} \mytextsc{red}} \textcolor{PineGreen}{\selectlanguage{french}\mytextsc{dur} \mytextsc{red}}  
 ¶ \textcolor{darkblue}{\textbf{\ipa{tʰi˧-lv̩˩\textasciitilde{}lv̩˩ | se˧}}} \zh{歪着走、扭着走、例如:残疾人走路有困难} \textcolor{Sepia}{\selectlanguage{english}to walk askance, to walk askew: e.g. a lame person walks with difficulty} \textcolor{PineGreen}{\selectlanguage{french}marcher en se trémoussant, marcher de travers, marcher en se contorsionnant}  
 ¶ \textcolor{darkblue}{\textbf{\ipa{kʰɯ˧tsʰɤ˧ lv̩˥\textasciitilde{}lv̩˩}}} \zh{活动一下(自己的)腿} \textcolor{Sepia}{\selectlanguage{english}to move one's leg around} \textcolor{PineGreen}{\selectlanguage{french}bouger la jambe, remuer la jambe}  

\lhead{\firstmark}
\rhead{\botmark}

\subsection{\hspace{-0.5cm} {\Large \textcolor{darkblue}{\textbf{\ipa{lv˧bv˧}}}}\hspace{0.5cm}[\kern2pt{\textcolor{darkblue}{\textbf{\ipa{lv˩bv˩˥}}}}\kern2pt]} \hypertarget{lv\string_Mbv\string_M1}{}
\markboth{\textcolor{darkblue}{\textbf{\ipa{lv˧bv˧}}}}{}
\textcolor{teal}{\zh{名词}} \hspace{4pt} \zh{声调类:} M.
\zh{菜畦。} \textcolor{Sepia}{\selectlanguage{english}Vegetable bed.} \textcolor{PineGreen}{\selectlanguage{french}Lit à légumes (dans le potager).}  ¶ \textcolor{darkblue}{\textbf{\ipa{v˩tsʰɤ˧-lv˧bv̩\#˥}}} \zh{同上:菜畦} \textcolor{Sepia}{\selectlanguage{english}same meaning: vegetable bed (in a vegetable garden)} \textcolor{PineGreen}{\selectlanguage{french}même sens: lit à légumes (dans le potager)}  
 ¶ \textcolor{darkblue}{\textbf{\ipa{qʰwæ˧ɭɯ˧-qo˧ | v˩tsʰɤ˧-lv˧bv˧ | le˧-gv˩, v˩tsʰɤ˧˥ | ɖɯ˧-jɤ˩ tʰi˩-pʰo˩}}} \zh{菜园里建菜畦,种一排菜} \textcolor{Sepia}{\selectlanguage{english}to make a vegetable bed in the vegetable garden, and to sow a row of vegetables} \textcolor{PineGreen}{\selectlanguage{french}bâtir un lit à légumes dans le potager, et semer une rangée de légumes}  
 \zh{量词}: \textcolor{darkblue}{\textbf{\ipa{kɤ˧˥}}} 
\lhead{\firstmark}
\rhead{\botmark}

\subsection{\hspace{-0.5cm} {\Large \textcolor{darkblue}{\textbf{\ipa{lwæ˩pʰv̩˩}}}}\hspace{0.5cm}[\kern2pt{\textcolor{darkblue}{\textbf{\ipa{lwæ˩pʰv̩˩˥}}}}\kern2pt]} \hypertarget{lw\{\string_Bp\string_hv\string_=\string_B1}{}
\markboth{\textcolor{darkblue}{\textbf{\ipa{lwæ˩pʰv̩˩}}}}{}
\textcolor{teal}{\zh{名词}} \hspace{4pt} \zh{声调类:} L.
\zh{灰。} \textcolor{Sepia}{\selectlanguage{english}Ashes.} \textcolor{PineGreen}{\selectlanguage{french}Cendres.}  ¶ \textcolor{darkblue}{\textbf{\ipa{[F5] lwæ˩pʰv̩˩-ni˥gv̩˩}}} \zh{灰色的(直译:“像白灰”)} \textcolor{Sepia}{\selectlanguage{english}grey; literally: “like ashes”} \textcolor{PineGreen}{\selectlanguage{french}de couleur grise; littéralement “comme de la cendre”}  

\lhead{\firstmark}
\rhead{\botmark}

\subsection{\hspace{-0.5cm} {\Large \textcolor{darkblue}{\textbf{\ipa{lwɤ˩˥}}}}\hspace{0.5cm}[\kern2pt{\textcolor{darkblue}{\textbf{\ipa{lwɤ˩˥}}}}\kern2pt]} \hypertarget{lw7\string_B\string_T1}{}
\markboth{\textcolor{darkblue}{\textbf{\ipa{lwɤ˩˥}}}}{}
\textcolor{teal}{\zh{名词}} \hspace{4pt} \zh{声调类:} LH.
\zh{灰,灰烬(包括草木灰等等)。} \textcolor{Sepia}{\selectlanguage{english}Ashes (of plants, charcoal...), cinders.} \textcolor{PineGreen}{\selectlanguage{french}Cendre (cendre végétale ou cendre de charbon); scories.}  ¶ \textcolor{darkblue}{\textbf{\ipa{lwɤ˩-pʰæ˧di˩}}} \zh{像灰烬,灰色} \textcolor{Sepia}{\selectlanguage{english}like ashes; gray-coloured} \textcolor{PineGreen}{\selectlanguage{french}comme de la cendre}  
 \zh{量词}: \textcolor{darkblue}{\textbf{\ipa{ʈʂwɤ˧}}} 
\lhead{\firstmark}
\rhead{\botmark}

\subsection{\hspace{-0.5cm} {\Large \textcolor{darkblue}{\textbf{\ipa{ɭɯ˧\textsubscript{b}}}}}\hspace{0.5cm}[\kern2pt{\textcolor{darkblue}{\textbf{\ipa{ɭɯ˥}}}}\kern2pt]} \hypertarget{l\string_RM\string_Mb1}{}
\markboth{\textcolor{darkblue}{\textbf{\ipa{ɭɯ˧\textsubscript{b}}}}}{}
\textcolor{teal}{\zh{量词}} \hspace{4pt} \zh{声调类:} M\textsubscript{b}.
\zh{最常用的量词,相当于汉语中‘个’的用法。本意是圆形颗粒。一粒(米……),一个(碗……),件(衣服……)。} \textcolor{Sepia}{\selectlanguage{english}Originally a classifier for round objects: grains, bowls… Now a generic classifier, used e.g. for pieces of clothing.} \textcolor{PineGreen}{\selectlanguage{french}Classificateur générique; à l'origine, classificateur pour les objets ronds, à l'emploi maintenant élargi.}  ¶ \textcolor{darkblue}{\textbf{\ipa{ɕi˧ ɖɯ˧-ɭɯ˧ |}}} \zh{一粒米} \textcolor{Sepia}{\selectlanguage{english}a grain of rice} \textcolor{PineGreen}{\selectlanguage{french}un grain de riz}  
 ¶ \textcolor{darkblue}{\textbf{\ipa{hõ˧-ɭɯ˥}}} \zh{八粒} \textcolor{Sepia}{\selectlanguage{english}eight grains} \textcolor{PineGreen}{\selectlanguage{french}huit grains}  
 ¶ \textcolor{darkblue}{\textbf{\ipa{ɖɯ˧-ɭɯ˧ hwæ˧-mɤ˧-ɖo˧! | le˧-qʰwæ˧-kv̩˥!}}} \zh{不要(只)买一个!会碎的!(东西要一对一对买:2、4、6、8、10……,单数不吉利,东西会碎的。)} \textcolor{Sepia}{\selectlanguage{english}Don't buy just one / Don't buy a single one: it would break! (Explanation: objects must be bought by pairs: 2, 4, 6, 8, 10..., not by sets of odd numbers (1, 3, 5, 7, 9...), otherwise it bears ill luck and the objects get broken or lost)} \textcolor{PineGreen}{\selectlanguage{french}N'en achète pas un (unique)! Ca va se casser! (Explication: il faut acheter les objets par paires: 2, 4, 6, 8, 10…, pas en nombre impair, sinon cela porte malheur et les objets cassent, se perdent…)}  
 ¶ \textcolor{darkblue}{\textbf{\ipa{ʈʂʰɯ˧ | zo˧hṽ˥ | dʑɤ˩-ɭɯ˥ dʑo˩!}}} \zh{她有个很漂亮的孩子!} \textcolor{Sepia}{\selectlanguage{english}She has a really pretty child! (Context: the main consultant had a polite conversation with a neighbour who had a lovely grandson; later on, she told me: “She has a really pretty child!”)} \textcolor{PineGreen}{\selectlanguage{french}Elle a un bien bel enfant! (Contexte: lors d'une sortie, la consultante principale dit quelques politesses à une voisine qui se promenait avec un petit-fils attendrissant; elle me dit ensuite: “Elle a un bien bel enfant!” Littéralement: “elle, (d')enfant(s), (elle) en a un (de) bien!”)}  

\lhead{\firstmark}
\rhead{\botmark}

\subsection{\hspace{-0.5cm} {\Large \textcolor{darkblue}{\textbf{\ipa{ɭɯ˧˥\textsubscript{a}}}}}\hspace{0.5cm}[\kern2pt{\textcolor{darkblue}{\textbf{\ipa{ɭɯ˧˥}}}}\kern2pt]} \hypertarget{l\string_RM\string_M\string_Ta1}{}
\markboth{\textcolor{darkblue}{\textbf{\ipa{ɭɯ˧˥\textsubscript{a}}}}}{}
\textcolor{teal}{\zh{量词}} \hspace{4pt} \zh{声调类:} MH\textsubscript{a}.
\zh{量词:衣服(一件)。} \textcolor{PineGreen}{\selectlanguage{french}Classificateur des vêtements.}  ¶ \textcolor{darkblue}{\textbf{\ipa{ʈʰæ˧qʰwɤ˧ ɖɯ˧-ɭɯ˧˥}}} \zh{一件裙子} \textcolor{Sepia}{\selectlanguage{english}a skirt} \textcolor{PineGreen}{\selectlanguage{french}une jupe}  
 ¶ \textcolor{darkblue}{\textbf{\ipa{bɑ˩lɑ˩˥ | ɖɯ˧-ɭɯ˧˥ |}}} \zh{一件衣服} \textcolor{Sepia}{\selectlanguage{english}a piece of clothing; a shirt} \textcolor{PineGreen}{\selectlanguage{french}un vêtement}  
 ¶ \textcolor{darkblue}{\textbf{\ipa{*dʑi˧hṽ˥\$+ɖɯ˧-ɭɯ˧˥}}} \zh{(这个量词不能与/dʑi˧hṽ˥\$/结合。)} \textcolor{Sepia}{\selectlanguage{english}This classifier cannot combine with /dʑi˧hṽ˥\$/, which takes /ɖɯ˧-dzi˩/ as its classifier.} \textcolor{PineGreen}{\selectlanguage{french}Ce classificateur ne se combine pas avec /dʑi˧hṽ˥\$/, qui prend pour classificateur: /ɖɯ˧-dzi˩/.}  

\lhead{\firstmark}
\rhead{\botmark}

\newpage
\section*{\centering- \textcolor{darkblue}{\textbf{\ipa{ɬ}}} -}
\subsection{\hspace{-0.5cm} {\Large \textcolor{darkblue}{\textbf{\ipa{ɬɑ˧mv̩˥\$}}}}\hspace{0.5cm}[\kern2pt{\textcolor{darkblue}{\textbf{\ipa{ɬɑ˧mv̩˥}}}}\kern2pt]} \hypertarget{KA\string_Mmv\string_=\string_T\$1}{}
\markboth{\textcolor{darkblue}{\textbf{\ipa{ɬɑ˧mv̩˥\$}}}}{}
\textcolor{teal}{\zh{名词}} \hspace{4pt} \zh{声调类:} H\$.
\zh{女性名字。} \textcolor{Sepia}{\selectlanguage{english}Feminine given name.} \textcolor{PineGreen}{\selectlanguage{french}Prénom féminin.} 
\lhead{\firstmark}
\rhead{\botmark}

\subsection{\hspace{-0.5cm} {\Large \textcolor{darkblue}{\textbf{\ipa{ɬɑ˧pɤ˩}}}}\hspace{0.5cm}[\kern2pt{\textcolor{darkblue}{\textbf{\ipa{ɬɑ˧pɤ˩}}}}\kern2pt]} \hypertarget{KA\string_Mp7\string_B1}{}
\markboth{\textcolor{darkblue}{\textbf{\ipa{ɬɑ˧pɤ˩}}}}{}
\textcolor{teal}{\zh{助词}} \hspace{4pt} \zh{声调类:} L\#.
\zh{多、使劲。} \textcolor{Sepia}{\selectlanguage{english}A lot, hard.} \textcolor{PineGreen}{\selectlanguage{french}Beaucoup.}  ¶ \textcolor{darkblue}{\textbf{\ipa{ɬɑ˧pɤ˩ ʝi˩}}} \zh{使劲工作、使劲干} \textcolor{Sepia}{\selectlanguage{english}to work hard, to work in a concentrated manner} \textcolor{PineGreen}{\selectlanguage{french}en faire beaucoup}  
 ¶ \textcolor{darkblue}{\textbf{\ipa{ɬɑ˧pɤ˩ | ɖɯ˧-kʰwɤ˧ ʝi˧}}} \zh{使劲工作一下} \textcolor{Sepia}{\selectlanguage{english}to work hard for a while, to get some solid work done} \textcolor{PineGreen}{\selectlanguage{french}en mettre un coup, beaucoup travailler, bien avancer dans son travail}  
 ¶ \textcolor{darkblue}{\textbf{\ipa{ɬɑ˧pɤ˩ | ɖɯ˧-kʰwɤ˧ so˥}}} \zh{努力学习一下} \textcolor{Sepia}{\selectlanguage{english}to study hard, to make headway in one's studies} \textcolor{PineGreen}{\selectlanguage{french}beaucoup étudier, faire un bon progrès dans l'étude}  

\lhead{\firstmark}
\rhead{\botmark}

\subsection{\hspace{-0.5cm} {\Large \textcolor{darkblue}{\textbf{\ipa{ɬɑ˧sɑ˧}}}}\hspace{0.5cm}[\kern2pt{\textcolor{darkblue}{\textbf{\ipa{ɬɑ˧sɑ˧}}}}\kern2pt]} \hypertarget{KA\string_MsA\string_M1}{}
\markboth{\textcolor{darkblue}{\textbf{\ipa{ɬɑ˧sɑ˧}}}}{}
\textcolor{teal}{\zh{名词}} \hspace{4pt} \zh{声调类:} M.
\zh{拉萨。} \textcolor{Sepia}{\selectlanguage{english}Lhasa.} \textcolor{PineGreen}{\selectlanguage{french}Lhasa (capitale du Tibet).} 
\lhead{\firstmark}
\rhead{\botmark}

\subsection{\hspace{-0.5cm} {\Large \textcolor{darkblue}{\textbf{\ipa{ɬɑ˧tɑ˥}}}}\hspace{0.5cm}[\kern2pt{\textcolor{darkblue}{\textbf{\ipa{ɬɑ˧tɑ˥}}}}\kern2pt]} \hypertarget{KA\string_MtA\string_T1}{}
\markboth{\textcolor{darkblue}{\textbf{\ipa{ɬɑ˧tɑ˥}}}}{}
\textcolor{teal}{\zh{名词}} \hspace{4pt} \zh{声调类:} H\#.
\textit{\zh{古语}} [\zh{古语}] \zh{皮革背心。} \textcolor{Sepia}{\selectlanguage{english}Jerkin, leather vest.} \textcolor{PineGreen}{\selectlanguage{french}Gilet de cuir (mot sorti d'usage, n'apparaît que dans un proverbe).}  \zh{量词}: \textcolor{darkblue}{\textbf{\ipa{ɭɯ˧}}} 
\lhead{\firstmark}
\rhead{\botmark}

\subsection{\hspace{-0.5cm} {\Large \textcolor{darkblue}{\textbf{\ipa{ɬɑ˧tsʰo\#˥}}}}\hspace{0.5cm}[\kern2pt{\textcolor{darkblue}{\textbf{\ipa{ɬɑ˧tsʰo˧}}}}\kern2pt]} \hypertarget{KA\string_Mts\string_ho\#\string_T1}{}
\markboth{\textcolor{darkblue}{\textbf{\ipa{ɬɑ˧tsʰo\#˥}}}}{}
\textcolor{teal}{\zh{名词}} \hspace{4pt} \zh{声调类:} \#H.
\zh{女性名字。} \textcolor{Sepia}{\selectlanguage{english}Feminine given name.} \textcolor{PineGreen}{\selectlanguage{french}Prénom féminin.} 
\lhead{\firstmark}
\rhead{\botmark}

\subsection{\hspace{-0.5cm} {\Large \textcolor{darkblue}{\textbf{\ipa{ɬɑ˧˥}}}}\hspace{0.5cm}[\kern2pt{\textcolor{darkblue}{\textbf{\ipa{ɬɑ˧˥}}}}\kern2pt]} \hypertarget{KA\string_M\string_T1}{}
\markboth{\textcolor{darkblue}{\textbf{\ipa{ɬɑ˧˥}}}}{}
\textcolor{teal}{\zh{形容词}} \hspace{4pt} \zh{声调类:} MH.
\zh{多、丰富、充分。} \textcolor{Sepia}{\selectlanguage{english}Numerous, abundant, plentiful.} \textcolor{PineGreen}{\selectlanguage{french}Abondant, nombreux.}  ¶ \textcolor{darkblue}{\textbf{\ipa{dʑɤ˩-hĩ˩˥, | le˧-ɳɯ˥! | mɤ˧-dʑɤ˩-hĩ˩, | le˧-ɬɑ˧˥!}}} \zh{好的,不多!不好的,就很多了!(情景:谈高中学生想入大学)} \textcolor{Sepia}{\selectlanguage{english}Good ones are few! Not-so-good ones are numerous! (Context: discussing universities, among which high-school graduates choose.)} \textcolor{PineGreen}{\selectlanguage{french}Les bons, il n'y en a guère; les médiocres, il y en a en quantité! (Contexte: au sujet des établissements universitaires entre lesquels les titulaires du baccalauréat chinois ont à choisir)}  

\lhead{\firstmark}
\rhead{\botmark}

\subsection{\hspace{-0.5cm} {\Large \textcolor{darkblue}{\textbf{\ipa{ɬi˥}}}}\hspace{0.5cm}[\kern2pt{\textcolor{darkblue}{\textbf{\ipa{ɬi˥}}}}\kern2pt]} \hypertarget{Ki\string_T1}{}
\markboth{\textcolor{darkblue}{\textbf{\ipa{ɬi˥}}}}{}
\textcolor{teal}{\zh{动词}} \hspace{4pt} \zh{声调类:} H.
\zh{休息,松懈。} \textcolor{Sepia}{\selectlanguage{english}To rest, to relax.} \textcolor{PineGreen}{\selectlanguage{french}Se reposer, se détendre.}  ¶ \textcolor{darkblue}{\textbf{\ipa{le˧-ɬi˥}}} \zh{\mytextsc{accomp} \string_} \textcolor{Sepia}{\selectlanguage{english}\mytextsc{accomp} \string_} \textcolor{PineGreen}{\selectlanguage{french}\mytextsc{accomp} \string_}  

\lhead{\firstmark}
\rhead{\botmark}

\subsection{\hspace{-0.5cm} {\Large \textcolor{darkblue}{\textbf{\ipa{ɬi˧\textsubscript{b}}}}}\hspace{0.5cm}[\kern2pt{\textcolor{darkblue}{\textbf{\ipa{ɬi˩˥}}}}\kern2pt]} \hypertarget{Ki\string_Mb1}{}
\markboth{\textcolor{darkblue}{\textbf{\ipa{ɬi˧\textsubscript{b}}}}}{}
\textcolor{teal}{\zh{量词}} \hspace{4pt} \zh{声调类:} M\textsubscript{b}.
\zh{量词:月。} \textcolor{Sepia}{\selectlanguage{english}Month.} \textcolor{PineGreen}{\selectlanguage{french}Mois.} 
\lhead{\firstmark}
\rhead{\botmark}

\subsection{\hspace{-0.5cm} {\Large \textcolor{darkblue}{\textbf{\ipa{ɬi˧bo\#˥}}}}\hspace{0.5cm}[\kern2pt{\textcolor{darkblue}{\textbf{\ipa{ɬi˩bo˩˥}}}}\kern2pt]} \hypertarget{Ki\string_Mbo\#\string_T1}{}
\markboth{\textcolor{darkblue}{\textbf{\ipa{ɬi˧bo\#˥}}}}{}
\textcolor{teal}{\zh{名词}} \hspace{4pt} \zh{声调类:} \#H.
\zh{聋子。} \textcolor{Sepia}{\selectlanguage{english}Deaf person.} \textcolor{PineGreen}{\selectlanguage{french}Sourd, personne sourde.}  ¶ \textcolor{darkblue}{\textbf{\ipa{ɬi˧bo˧-hĩ˧}}} \zh{耳朵聋的人} \textcolor{Sepia}{\selectlanguage{english}deaf person} \textcolor{PineGreen}{\selectlanguage{french}personne sourde}  
 \zh{量词}: \textcolor{darkblue}{\textbf{\ipa{v̩˧}}} 
\lhead{\firstmark}
\rhead{\botmark}

\subsection{\hspace{-0.5cm} {\Large \textcolor{darkblue}{\textbf{\ipa{ɬi˧bv̩˧}}}}\hspace{0.5cm}[\kern2pt{\textcolor{darkblue}{\textbf{\ipa{ɬi˧bv̩˧}}}}\kern2pt]} \hypertarget{Ki\string_Mbv\string_=\string_M1}{}
\markboth{\textcolor{darkblue}{\textbf{\ipa{ɬi˧bv̩˧}}}}{}
\textcolor{teal}{\zh{名词}} \hspace{4pt} \zh{声调类:} M.
\zh{白族。} \textcolor{Sepia}{\selectlanguage{english}Bai (ethnic group).} \textcolor{PineGreen}{\selectlanguage{french}Bai (groupe ethnique).}  \zh{量词}: \textcolor{darkblue}{\textbf{\ipa{v̩˧}}} 
\lhead{\firstmark}
\rhead{\botmark}

\subsection{\hspace{-0.5cm} {\Large \textcolor{darkblue}{\textbf{\ipa{ɬi˧bv̩˩ | dʑɤ˩tsʰi˧-si\#˥}}}}\hspace{0.5cm}[\kern2pt{\textcolor{darkblue}{\textbf{\ipa{xxxx non-correspondance entre le nombre de groupes tonals et le nombre de tons}}}}\kern2pt]} \hypertarget{Ki\string_Mbv\string_=\string_B | dz£7\string_Bts\string_hi\string_M-si\#\string_T1}{}
\markboth{\textcolor{darkblue}{\textbf{\ipa{ɬi˧bv̩˩ | dʑɤ˩tsʰi˧-si\#˥}}}}{}
\textcolor{teal}{\zh{名词}} \hspace{4pt} \zh{声调类:} L\# | LM+\#H.
\zh{香椿、香椿树。} \textcolor{Sepia}{\selectlanguage{english}Chinese toon, fragrant cedar, \textit{Ailanthus chinensis}.} \textcolor{PineGreen}{\selectlanguage{french}Ailante, \textit{Ailanthus chinensis}; arbre très odorant.} 
\lhead{\firstmark}
\rhead{\botmark}

\subsection{\hspace{-0.5cm} {\Large \textcolor{darkblue}{\textbf{\ipa{ɬi˧bv̩˧-mi\#˥}}}}\hspace{0.5cm}[\kern2pt{\textcolor{darkblue}{\textbf{\ipa{xxxx non-correspondance entre le nombre de morphèmes et le nombre de tons de morphèmes}}}}\kern2pt]} \hypertarget{Ki\string_Mbv\string_=\string_M-mi\#\string_T1}{}
\markboth{\textcolor{darkblue}{\textbf{\ipa{ɬi˧bv̩˧-mi\#˥}}}}{}
\textcolor{teal}{\zh{名词}} \hspace{4pt} \zh{声调类:} \#H.
\zh{白族女人。} \textcolor{Sepia}{\selectlanguage{english}Bai woman.} \textcolor{PineGreen}{\selectlanguage{french}Femme bai.}  \zh{量词}: \textcolor{darkblue}{\textbf{\ipa{v̩˧}}} 
\lhead{\firstmark}
\rhead{\botmark}

\subsection{\hspace{-0.5cm} {\Large \textcolor{darkblue}{\textbf{\ipa{ɬi˧bv̩˧-zo\#˥}}}}\hspace{0.5cm}[\kern2pt{\textcolor{darkblue}{\textbf{\ipa{xxxx non-correspondance entre le nombre de morphèmes et le nombre de tons de morphèmes}}}}\kern2pt]} \hypertarget{Ki\string_Mbv\string_=\string_M-zo\#\string_T1}{}
\markboth{\textcolor{darkblue}{\textbf{\ipa{ɬi˧bv̩˧-zo\#˥}}}}{}
\textcolor{teal}{\zh{名词}} \hspace{4pt} \zh{声调类:} \#H.
\textcolor{Sepia}{\selectlanguage{english}Bai man.} \textcolor{PineGreen}{\selectlanguage{french}Homme bai.} \zh{当地汉语方言:}\zh{白族男人。} \zh{量词}: \textcolor{darkblue}{\textbf{\ipa{v̩˧}}} 
\lhead{\firstmark}
\rhead{\botmark}

\subsection{\hspace{-0.5cm} {\Large \textcolor{darkblue}{\textbf{\ipa{ɬi˧di˩}}}}\hspace{0.5cm}[\kern2pt{\textcolor{darkblue}{\textbf{\ipa{ɬi˧di˧}}}}\kern2pt]} \hypertarget{Ki\string_Mdi\string_B1}{}
\markboth{\textcolor{darkblue}{\textbf{\ipa{ɬi˧di˩}}}}{}
\textcolor{teal}{\zh{名词}} \hspace{4pt} \zh{声调类:} L\#.
\zh{永宁(地名)。} \textcolor{Sepia}{\selectlanguage{english}Yongning (place name).} \textcolor{PineGreen}{\selectlanguage{french}Yongning (nom de lieu).} 
\lhead{\firstmark}
\rhead{\botmark}

\subsection{\hspace{-0.5cm} {\Large \textcolor{darkblue}{\textbf{\ipa{ɬi˧di˩-hĩ˩}}}}\hspace{0.5cm}[\kern2pt{\textcolor{darkblue}{\textbf{\ipa{xxxx non-correspondance entre le nombre de morphèmes et le nombre de tons de morphèmes}}}}\kern2pt]} \hypertarget{Ki\string_Mdi\string_B-hi\string_~\string_B1}{}
\markboth{\textcolor{darkblue}{\textbf{\ipa{ɬi˧di˩-hĩ˩}}}}{}
\textcolor{teal}{\zh{名词}} \hspace{4pt} \zh{声调类:} L\#-.
\zh{永宁人(纳人)。} \textcolor{Sepia}{\selectlanguage{english}People of Yongning. Unless otherwise specified, this is mainly understood as referring to the Na (Mosuo).} \textcolor{PineGreen}{\selectlanguage{french}Les gens de Yongning.}  \zh{量词}: \textcolor{darkblue}{\textbf{\ipa{v̩˧}}} 
\lhead{\firstmark}
\rhead{\botmark}

\subsection{\hspace{-0.5cm} {\Large \textcolor{darkblue}{\textbf{\ipa{ɬi˧dʑɯ˩}}}}\hspace{0.5cm}[\kern2pt{\textcolor{darkblue}{\textbf{\ipa{xxxx non-correspondance entre le nombre de morphèmes et le nombre de tons de morphèmes}}}}\kern2pt]} \hypertarget{Ki\string_Mdz£M\string_B1}{}
\markboth{\textcolor{darkblue}{\textbf{\ipa{ɬi˧dʑɯ˩}}}}{}
\textcolor{teal}{\zh{名词}} \hspace{4pt} \zh{声调类:} L\#.
\zh{永宁坝子的河流。} \textcolor{Sepia}{\selectlanguage{english}The river that flows through the plain of Yongning.} \textcolor{PineGreen}{\selectlanguage{french}La rivière qui traverse la plaine de Yongning.}  \zh{量词}: \textcolor{darkblue}{\textbf{\ipa{kʰɯ˩}}} 
\lhead{\firstmark}
\rhead{\botmark}

\subsection{\hspace{-0.5cm} {\Large \textcolor{darkblue}{\textbf{\ipa{ɬi˧gv̩\#˥}}}}\hspace{0.5cm}[\kern2pt{\textcolor{darkblue}{\textbf{\ipa{ɬi˧gv̩˩}}}}\kern2pt]} \hypertarget{Ki\string_Mgv\string_=\#\string_T1}{}
\markboth{\textcolor{darkblue}{\textbf{\ipa{ɬi˧gv̩\#˥}}}}{}
\textcolor{teal}{\zh{名词}} \hspace{4pt} \zh{声调类:} \#H.
\zh{中部,中间。} \textcolor{Sepia}{\selectlanguage{english}Middle part; (in) the centre.} \textcolor{PineGreen}{\selectlanguage{french}Partie intermédiaire, milieu; au milieu.}  ¶ \textcolor{darkblue}{\textbf{\ipa{ɬi˧gv̩˧ dzi˥}}} \zh{坐在中间} \textcolor{Sepia}{\selectlanguage{english}to sit in the centre} \textcolor{PineGreen}{\selectlanguage{french}être assis au milieu}  

\lhead{\firstmark}
\rhead{\botmark}

\subsection{\hspace{-0.5cm} {\Large \textcolor{darkblue}{\textbf{\ipa{ɬi˧hĩ\#˥}}} \textsubscript{1}}\hspace{0.5cm}[\kern2pt{\textcolor{darkblue}{\textbf{\ipa{ɬi˧hĩ˧}}}}\kern2pt]} \hypertarget{Ki\string_Mhi\string_~\#\string_T1}{}
\markboth{\textcolor{darkblue}{\textbf{\ipa{ɬi˧hĩ\#˥}}} \textsubscript{1}}{}
\textcolor{teal}{\zh{名词}} \hspace{4pt} \zh{声调类:} \#H.
\zh{兄弟里面夹中的男孩(上有哥哥下有弟弟的孩子)。} \textcolor{Sepia}{\selectlanguage{english}Man in middle position among siblings: neither eldest brother nor youngest brother; literal translation: “person in the middle”.} \textcolor{PineGreen}{\selectlanguage{french}Homme en position intermédiaire dans la fratrie: ni aîné ni cadet; traduction littérale: “personne du milieu”.} 
\lhead{\firstmark}
\rhead{\botmark}

\subsection{\hspace{-0.5cm} {\Large \textcolor{darkblue}{\textbf{\ipa{ɬi˧hĩ\#˥}}} \textsubscript{2}}\hspace{0.5cm}[\kern2pt{\textcolor{darkblue}{\textbf{\ipa{ɬi˧hĩ˧}}}}\kern2pt]} \hypertarget{Ki\string_Mhi\string_~\#\string_T2}{}
\markboth{\textcolor{darkblue}{\textbf{\ipa{ɬi˧hĩ\#˥}}} \textsubscript{2}}{}
\textcolor{teal}{\zh{名词}} \hspace{4pt} \zh{声调类:} \#H.
\zh{永宁的人。} \textcolor{Sepia}{\selectlanguage{english}Inhabitant of Yongning; as used by the main consultant, the term includes Pumi (Prinmi) people along with Na people.} \textcolor{PineGreen}{\selectlanguage{french}Habitant de Yongning. Peut désigner les Prinmi qui habitent dans la plaine, aussi bien que les Na.} 
\lhead{\firstmark}
\rhead{\botmark}

\subsection{\hspace{-0.5cm} {\Large \textcolor{darkblue}{\textbf{\ipa{ɬi˧ki\#˥}}}}\hspace{0.5cm}[\kern2pt{\textcolor{darkblue}{\textbf{\ipa{ɬi˧ki˥}}}}\kern2pt]} \hypertarget{Ki\string_Mki\#\string_T1}{}
\markboth{\textcolor{darkblue}{\textbf{\ipa{ɬi˧ki\#˥}}}}{}
\textcolor{teal}{\zh{名词}} \hspace{4pt} \zh{声调类:} \#H.
\zh{泸沽湖附近的一个村落。} \textcolor{Sepia}{\selectlanguage{english}The name of a Na village, outside the plain of Yongning, close to the Lake.} \textcolor{PineGreen}{\selectlanguage{french}Village na, hors de la plaine, proche du Lac.}  ¶ \textcolor{darkblue}{\textbf{\ipa{ɬi˧ki˧, | ɲi˧se˩, | tɑ˧dzi˩, | mv̩˧qʰwæ˩, | lɑ˧tʰɑ˧-di˧˥}}} \zh{永宁到泸沽湖所经过的村落,依次是:里格、尼赛、大祖、木垮,然后到拉塔地(拉塔地指的是泸沽湖周边的摩梭地区,包括左所、洛水村等)} \textcolor{Sepia}{\selectlanguage{english}Villages that one passes when moving away from the Yongning plain, towards Lugu lake. These villages do not count as part of Yongning proper. The last, \textcolor{darkblue}{\textbf{\ipa{/lɑ˧tʰɑ˧-di˧˥/}}}, is not a village name like the preceding four: it refers to the entire Na area beyond the fourth village.} \textcolor{PineGreen}{\selectlanguage{french}Villages dans l'ordre, après la plaine de Yongning, ne comptant pas comme faisant partie de Yongning. Le dernier, \textcolor{darkblue}{\textbf{\ipa{/lɑ˧tʰɑ˧-di˧˥/}}}, désigne toute la région na au-delà du quatrième village.}  

\lhead{\firstmark}
\rhead{\botmark}

\subsection{\hspace{-0.5cm} {\Large \textcolor{darkblue}{\textbf{\ipa{ɬi˧ki˥}}}}\hspace{0.5cm}[\kern2pt{\textcolor{darkblue}{\textbf{\ipa{ɬi˧ki˧}}}}\kern2pt]} \hypertarget{Ki\string_Mki\string_T1}{}
\markboth{\textcolor{darkblue}{\textbf{\ipa{ɬi˧ki˥}}}}{}
\textcolor{teal}{\zh{名词}} \hspace{4pt} \zh{声调类:} H\#.
\zh{男性成年礼:直译“穿裤”。} \textcolor{Sepia}{\selectlanguage{english}Ritual for boys coming of age, i.e. reaching the age of 13 years: “wearing trousers”; at that age adolescents begin to wear trousers instead of children's robes.} \textcolor{PineGreen}{\selectlanguage{french}Cérémonie pour les garçons atteignant 13 ans: littéralement “porter/enfiler/mettre le pantalon”. Après cette cérémonie, l'adolescent porte un pantalon, au lieu du vêtement unisexe des enfants. (Rituel parallèle avec fv{/ʈʰæ˩ ki˩˥/}, “porter/enfiler/mettre la jupe”, pour les jeunes filles.).} 
\lhead{\firstmark}
\rhead{\botmark}

\subsection{\hspace{-0.5cm} {\Large \textcolor{darkblue}{\textbf{\ipa{ɬi˧mi˧}}} \textsubscript{1}}\hspace{0.5cm}[\kern2pt{\textcolor{darkblue}{\textbf{\ipa{ɬi˧mi˧}}}}\kern2pt]} \hypertarget{Ki\string_Mmi\string_M1}{}
\markboth{\textcolor{darkblue}{\textbf{\ipa{ɬi˧mi˧}}} \textsubscript{1}}{}
\textcolor{teal}{\zh{名词}} \hspace{4pt} \zh{声调类:} M.
\ding{202} \zh{月亮(双音节)。} \textcolor{Sepia}{\selectlanguage{english}Moon (disyllable).} \textcolor{PineGreen}{\selectlanguage{french}Lune (disyllabe).}  \zh{量词}: \textcolor{darkblue}{\textbf{\ipa{ɭɯ˧}}} \ding{203} \zh{月(双音节)。} \textcolor{Sepia}{\selectlanguage{english}Month (disyllable).} \textcolor{PineGreen}{\selectlanguage{french}Mois.}  ¶ \textcolor{darkblue}{\textbf{\ipa{ɬi˧mi˧ ɖɯ˧-gi˥}}} \zh{半个月} \textcolor{Sepia}{\selectlanguage{english}half a month} \textcolor{PineGreen}{\selectlanguage{french}une quinzaine, la moitié d'un mois}  
 ¶ \textcolor{darkblue}{\textbf{\ipa{ɬi˧mi˧ le˧-gv̩˩}}} \zh{下半月份} \textcolor{Sepia}{\selectlanguage{english}the latter half of the month; literally 'the declining half of the month'} \textcolor{PineGreen}{\selectlanguage{french}le mois décroît; expression qui peut désigner la seconde période du mois}  

\lhead{\firstmark}
\rhead{\botmark}

\subsection{\hspace{-0.5cm} {\Large \textcolor{darkblue}{\textbf{\ipa{ɬi˧mi˧}}} \textsubscript{2}}\hspace{0.5cm}[\kern2pt{\textcolor{darkblue}{\textbf{\ipa{ɬi˧mi˧}}}}\kern2pt]} \hypertarget{Ki\string_Mmi\string_M2}{}
\markboth{\textcolor{darkblue}{\textbf{\ipa{ɬi˧mi˧}}} \textsubscript{2}}{}
\textcolor{teal}{\zh{名词}} \hspace{4pt} \zh{声调类:} M.
\zh{母獐子。} \textcolor{Sepia}{\selectlanguage{english}Female roebuck.} \textcolor{PineGreen}{\selectlanguage{french}Chevrotain femelle.}  \zh{量词}: \textcolor{darkblue}{\textbf{\ipa{v̩˧}}} 
\lhead{\firstmark}
\rhead{\botmark}

\subsection{\hspace{-0.5cm} {\Large \textcolor{darkblue}{\textbf{\ipa{ɬi˧mi˧dɑ˧dzɯ\#˥}}}}\hspace{0.5cm}[\kern2pt{\textcolor{darkblue}{\textbf{\ipa{ɬi˧mi˧dɑ˧dzɯ˧}}}}\kern2pt]} \hypertarget{Ki\string_Mmi\string_MdA\string_MdzM\#\string_T1}{}
\markboth{\textcolor{darkblue}{\textbf{\ipa{ɬi˧mi˧dɑ˧dzɯ\#˥}}}}{}
\textcolor{teal}{\zh{名词}} \hspace{4pt} \zh{声调类:} \#H.
\zh{月蚀。} \textcolor{Sepia}{\selectlanguage{english}Lunar eclipse.} \textcolor{PineGreen}{\selectlanguage{french}Éclipse de lune.}  ¶ \textcolor{darkblue}{\textbf{\ipa{ɬi˧mi˧dɑ˧dzɯ˧ tʰv̩˧}}} \zh{有月蚀} \textcolor{Sepia}{\selectlanguage{english}there is a lunar eclipse} \textcolor{PineGreen}{\selectlanguage{french}il y a une éclipse de lune}  
 ¶ \textcolor{darkblue}{\textbf{\ipa{ʈʂʰɯ˧ | ɬi˧mi˧dɑ˧dzɯ˧ ɲi˥!}}} \zh{这是月蚀!(一个人问:‘这是怎么回事?’,另一个回答:‘这是月蚀!’)} \textcolor{Sepia}{\selectlanguage{english}This is a lunar eclipse! (Answer to the question 'What is happening? / What is that supposed to mean?')} \textcolor{PineGreen}{\selectlanguage{french}c'est une éclipse de lune! (réponse à la question 'Qu'est-ce qui se passe?')}  
 \zh{量词}: \textcolor{darkblue}{\textbf{\ipa{ʂɯ˩}}} 
\lhead{\firstmark}
\rhead{\botmark}

\subsection{\hspace{-0.5cm} {\Large \textcolor{darkblue}{\textbf{\ipa{ɬi˧ɳæ˩}}}}\hspace{0.5cm}[\kern2pt{\textcolor{darkblue}{\textbf{\ipa{ɬi˧ɳæ˧}}}}\kern2pt]} \hypertarget{Ki\string_Mn`\{\string_B1}{}
\markboth{\textcolor{darkblue}{\textbf{\ipa{ɬi˧ɳæ˩}}}}{}
\textcolor{teal}{\zh{名词}} \hspace{4pt} \zh{声调类:} L\#.
\zh{月经。} \textcolor{Sepia}{\selectlanguage{english}Menses; period.} \textcolor{PineGreen}{\selectlanguage{french}Menstrues.}  ¶ \textcolor{darkblue}{\textbf{\ipa{ʈʂʰɯ˧ | ɬi˧ɳæ˩-ze˩}}} \zh{她来了月经。} \textcolor{Sepia}{\selectlanguage{english}She is having her menses.} \textcolor{PineGreen}{\selectlanguage{french}Elle est en train d'avoir ses règles.}  
 ¶ \textcolor{darkblue}{\textbf{\ipa{ɬi˧ɳæ˩ go˩}}} \zh{来了月经,疼} \textcolor{Sepia}{\selectlanguage{english}to have painful menses} \textcolor{PineGreen}{\selectlanguage{french}avoir des menstrues douloureuses}  
 \zh{量词}: \textcolor{darkblue}{\textbf{\ipa{ɬi˧}}} 
\lhead{\firstmark}
\rhead{\botmark}

\subsection{\hspace{-0.5cm} {\Large \textcolor{darkblue}{\textbf{\ipa{ɬi˧pæ˥}}}}\hspace{0.5cm}[\kern2pt{\textcolor{darkblue}{\textbf{\ipa{ɬi˧pæ˩}}}}\kern2pt]} \hypertarget{Ki\string_Mp\{\string_T1}{}
\markboth{\textcolor{darkblue}{\textbf{\ipa{ɬi˧pæ˥}}}}{}
\textcolor{teal}{\zh{名词}} \hspace{4pt} \zh{声调类:} H\#.
\zh{耳环。} \textcolor{Sepia}{\selectlanguage{english}Earring.} \textcolor{PineGreen}{\selectlanguage{french}Boucle d'oreille.}  ¶ \textcolor{darkblue}{\textbf{\ipa{ŋv̩˩-ɬi˩pæ˥ (+ɲi˩)}}} \zh{银耳环} \textcolor{Sepia}{\selectlanguage{english}silver earring} \textcolor{PineGreen}{\selectlanguage{french}boucle d'oreille en argent}  
 ¶ \textcolor{darkblue}{\textbf{\ipa{hæ̃˩-ɬi˩pæ˥ (+ɲi˩)}}} \zh{金耳环} \textcolor{Sepia}{\selectlanguage{english}gold earring} \textcolor{PineGreen}{\selectlanguage{french}boucle d'oreille en or}  
 \zh{量词}: \textcolor{darkblue}{\textbf{\ipa{dze˩}}} 
\lhead{\firstmark}
\rhead{\botmark}

\subsection{\hspace{-0.5cm} {\Large \textcolor{darkblue}{\textbf{\ipa{ɬi˧pi˩}}}}\hspace{0.5cm}[\kern2pt{\textcolor{darkblue}{\textbf{\ipa{ɬi˩pi˥}}}}\kern2pt]} \hypertarget{Ki\string_Mpi\string_B1}{}
\markboth{\textcolor{darkblue}{\textbf{\ipa{ɬi˧pi˩}}}}{}
\textcolor{teal}{\zh{名词}} \hspace{4pt} \zh{声调类:} L\#.
\zh{耳朵。} \textcolor{Sepia}{\selectlanguage{english}Ear.} \textcolor{PineGreen}{\selectlanguage{french}Oreille.}  \zh{量词}: \textcolor{darkblue}{\textbf{\ipa{pʰo˧˥}}} 
\lhead{\firstmark}
\rhead{\botmark}

\subsection{\hspace{-0.5cm} {\Large \textcolor{darkblue}{\textbf{\ipa{ɬi˧pv̩˧lv̩˥}}}}\hspace{0.5cm}[\kern2pt{\textcolor{darkblue}{\textbf{\ipa{ɬi˩pv̩˩lv̩˩˥}}}}\kern2pt]} \hypertarget{Ki\string_Mpv\string_=\string_Mlv\string_=\string_T1}{}
\markboth{\textcolor{darkblue}{\textbf{\ipa{ɬi˧pv̩˧lv̩˥}}}}{}
\textcolor{teal}{\zh{名词}} \hspace{4pt} \zh{声调类:} H\#.
\zh{耳朵瘤。} \textcolor{Sepia}{\selectlanguage{english}Ear tumour, pathological excrescence of the ear.} \textcolor{PineGreen}{\selectlanguage{french}Tumeur de l'oreille, excroissance pathologique de l'oreille.}  \zh{量词}: \textcolor{darkblue}{\textbf{\ipa{ɭɯ˧}}} 
\lhead{\firstmark}
\rhead{\botmark}

\subsection{\hspace{-0.5cm} {\Large \textcolor{darkblue}{\textbf{\ipa{ɬi˧pʰv̩\#˥}}}}\hspace{0.5cm}[\kern2pt{\textcolor{darkblue}{\textbf{\ipa{ɬi˧pʰv̩˥}}}}\kern2pt]} \hypertarget{Ki\string_Mp\string_hv\string_=\#\string_T1}{}
\markboth{\textcolor{darkblue}{\textbf{\ipa{ɬi˧pʰv̩\#˥}}}}{}
\textcolor{teal}{\zh{名词}} \hspace{4pt} \zh{声调类:} \#H.
\zh{公獐子。} \textcolor{Sepia}{\selectlanguage{english}Male roebuck, male hornless river deer.} \textcolor{PineGreen}{\selectlanguage{french}Chevrotain mâle.}  ¶ \textcolor{darkblue}{\textbf{\ipa{ɬi˧pʰv̩˧ tʰv̩˧-mi˥\# / ɬi˧pʰv̩˧ tʰv̩˧-mi˧˥}}} \zh{那只公獐子} \textcolor{Sepia}{\selectlanguage{english}\mytextsc{n}+\mytextsc{dem}+\mytextsc{clf}} \textcolor{PineGreen}{\selectlanguage{french}\mytextsc{n}+\mytextsc{dem}+\mytextsc{clf}}  
 \zh{量词}: \textcolor{darkblue}{\textbf{\ipa{v̩˧}}} \textcolor{darkblue}{\textbf{\ipa{ɭɯ˧}}} \textcolor{darkblue}{\textbf{\ipa{mi˩}}} 
\lhead{\firstmark}
\rhead{\botmark}

\subsection{\hspace{-0.5cm} {\Large \textcolor{darkblue}{\textbf{\ipa{ɬi˧qʰæ\#˥}}}}\hspace{0.5cm}[\kern2pt{\textcolor{darkblue}{\textbf{\ipa{ɬi˧qʰæ˧}}}}\kern2pt]} \hypertarget{Ki\string_Mq\string_h\{\#\string_T1}{}
\markboth{\textcolor{darkblue}{\textbf{\ipa{ɬi˧qʰæ\#˥}}}}{}
\textcolor{teal}{\zh{名词}} \hspace{4pt} \zh{声调类:} \#H.
\zh{耳垢。} \textcolor{Sepia}{\selectlanguage{english}Earwax.} \textcolor{PineGreen}{\selectlanguage{french}Cérumen.}  \zh{量词}: \textcolor{darkblue}{\textbf{\ipa{kʰwɤ˥}}} 
\lhead{\firstmark}
\rhead{\botmark}

\subsection{\hspace{-0.5cm} {\Large \textcolor{darkblue}{\textbf{\ipa{ɬi˧qʰv̩\#˥}}}}\hspace{0.5cm}[\kern2pt{\textcolor{darkblue}{\textbf{\ipa{ɬi˧qʰv̩˧}}}}\kern2pt]} \hypertarget{Ki\string_Mq\string_hv\string_=\#\string_T1}{}
\markboth{\textcolor{darkblue}{\textbf{\ipa{ɬi˧qʰv̩\#˥}}}}{}
\textcolor{teal}{\zh{名词}} \hspace{4pt} \zh{声调类:} \#H.
\zh{耳孔。} \textcolor{Sepia}{\selectlanguage{english}Auditory canal.} \textcolor{PineGreen}{\selectlanguage{french}Conduit auditif.}  ¶ \textcolor{darkblue}{\textbf{\ipa{ʈʂʰɯ˧ | ɬi˧qʰv̩˧ | ɖɯ˧-pi˧˥ | tʰɑ˩˥!}}} \zh{她耳朵很好使! / 她耳朵很尖!(情景:一有客人到家的声音,睡午觉的两岁女孩子立即醒来。)} \textcolor{Sepia}{\selectlanguage{english}She has a sensitive ear! (Context: about a 2-year old girl who wakes up from her siesta as soon as guests come in.)} \textcolor{PineGreen}{\selectlanguage{french}Elle a l'oreille fine! (Contexte: au sujet d'une petite fille de 2 ans qui se réveille aussitôt de sa sieste lorsqu'elle entend l'arrivée de visiteurs.)}  
 \zh{量词}: \textcolor{darkblue}{\textbf{\ipa{ɭɯ˧}}} 
\lhead{\firstmark}
\rhead{\botmark}

\subsection{\hspace{-0.5cm} {\Large \textcolor{darkblue}{\textbf{\ipa{ɬi˧ʈv̩˥}}}}\hspace{0.5cm}[\kern2pt{\textcolor{darkblue}{\textbf{\ipa{ɬi˩ʈv̩˥}}}}\kern2pt]} \hypertarget{Ki\string_Mt`v\string_=\string_T1}{}
\markboth{\textcolor{darkblue}{\textbf{\ipa{ɬi˧ʈv̩˥}}}}{}
\textcolor{teal}{\zh{名词}} \hspace{4pt} \zh{声调类:} H\#.
\zh{车前草。} \textcolor{Sepia}{\selectlanguage{english}Asiatic plantain.} \textcolor{PineGreen}{\selectlanguage{french}Plantain (utilisé par les Na pour ses vertus médicinales; est abondant à Yongning).}  \zh{量词}: \textcolor{darkblue}{\textbf{\ipa{po˧}}} 
\lhead{\firstmark}
\rhead{\botmark}

\subsection{\hspace{-0.5cm} {\Large \textcolor{darkblue}{\textbf{\ipa{ɬi˩}}} \textsubscript{1}}\hspace{0.5cm}[\kern2pt{\textcolor{darkblue}{\textbf{\ipa{ɬi˩˥}}}}\kern2pt]} \hypertarget{Ki\string_B1}{}
\markboth{\textcolor{darkblue}{\textbf{\ipa{ɬi˩}}} \textsubscript{1}}{}
\textcolor{teal}{\zh{动词}} \hspace{4pt} \zh{声调类:} L\textsubscript{a}.
\zh{量(一块布料……)有多长:有多少庹。} \textcolor{Sepia}{\selectlanguage{english}To measure (e.g. a piece of fabric) to find its length, in armspans.} \textcolor{PineGreen}{\selectlanguage{french}Toiser: mesurer (une pièce de tissu…) à l'aune de la toise: distance entre les deux bras écartés.}  ¶ \textcolor{darkblue}{\textbf{\ipa{ɬi˩-se˥ (-ze˩)}}} \zh{量完(了)} \textcolor{Sepia}{\selectlanguage{english}\string_ \mytextsc{achev} (\mytextsc{pfv})} \textcolor{PineGreen}{\selectlanguage{french}\string_ \mytextsc{achev} (\mytextsc{pfv})}  

\lhead{\firstmark}
\rhead{\botmark}

\subsection{\hspace{-0.5cm} {\Large \textcolor{darkblue}{\textbf{\ipa{ɬi˩}}} \textsubscript{2}}\hspace{0.5cm}[\kern2pt{\textcolor{darkblue}{\textbf{\ipa{ɬi˥}}}}\kern2pt]} \hypertarget{Ki\string_B2}{}
\markboth{\textcolor{darkblue}{\textbf{\ipa{ɬi˩}}} \textsubscript{2}}{}
\textcolor{teal}{\zh{名词}} \hspace{4pt} \zh{声调类:} L.
\zh{獐子。} \textcolor{Sepia}{\selectlanguage{english}Roebuck, hornless river deer.} \textcolor{PineGreen}{\selectlanguage{french}Chevrotain.}  \zh{量词}: \textcolor{darkblue}{\textbf{\ipa{pʰo˧˥}}} \textcolor{darkblue}{\textbf{\ipa{mi˩}}} 
\lhead{\firstmark}
\rhead{\botmark}

\subsection{\hspace{-0.5cm} {\Large \textcolor{darkblue}{\textbf{\ipa{ɬi˩\textsubscript{b}}}}}\hspace{0.5cm}[\kern2pt{\textcolor{darkblue}{\textbf{\ipa{ɬi˥}}}}\kern2pt]} \hypertarget{Ki\string_Bb1}{}
\markboth{\textcolor{darkblue}{\textbf{\ipa{ɬi˩\textsubscript{b}}}}}{}
\textcolor{teal}{\zh{量词}} \hspace{4pt} \zh{声调类:} L\textsubscript{b}.
\zh{量词:庹。} \textcolor{Sepia}{\selectlanguage{english}A span, an armspread.} \textcolor{PineGreen}{\selectlanguage{french}Toise: envergure des bras =longueur des deux bras écartés. Cette unité correspond à environ 5 pieds chinois (1 mètre 78).}  ¶ \textcolor{darkblue}{\textbf{\ipa{tsʰe˧-ɬi˧}}} \zh{十庹} \textcolor{Sepia}{\selectlanguage{english}10 spans, 10 armspreads} \textcolor{PineGreen}{\selectlanguage{french}10 toises}  

\lhead{\firstmark}
\rhead{\botmark}

\subsection{\hspace{-0.5cm} {\Large \textcolor{darkblue}{\textbf{\ipa{ɬi˩bi˩}}}}\hspace{0.5cm}[\kern2pt{\textcolor{darkblue}{\textbf{\ipa{ɬi˧bi˧}}}}\kern2pt]} \hypertarget{Ki\string_Bbi\string_B1}{}
\markboth{\textcolor{darkblue}{\textbf{\ipa{ɬi˩bi˩}}}}{}
\textcolor{teal}{\zh{名词}} \hspace{4pt} \zh{声调类:} L.
\zh{萝卜。} \textcolor{Sepia}{\selectlanguage{english}Turnip; radish.} \textcolor{PineGreen}{\selectlanguage{french}Navet, gros radis.}  \zh{量词}: \textcolor{darkblue}{\textbf{\ipa{ɭɯ˧}}} 
\lhead{\firstmark}
\rhead{\botmark}

\subsection{\hspace{-0.5cm} {\Large \textcolor{darkblue}{\textbf{\ipa{ɬi˩qʰwɤ˩}}}}\hspace{0.5cm}[\kern2pt{\textcolor{darkblue}{\textbf{\ipa{ɬi˧qʰwɤ˧}}}}\kern2pt]} \hypertarget{Ki\string_Bq\string_hw7\string_B1}{}
\markboth{\textcolor{darkblue}{\textbf{\ipa{ɬi˩qʰwɤ˩}}}}{}
\textcolor{teal}{\zh{名词}} \hspace{4pt} \zh{声调类:} L.
\zh{裤子。} \textcolor{Sepia}{\selectlanguage{english}Trousers.} \textcolor{PineGreen}{\selectlanguage{french}Pantalon.}  \zh{量词}: \textcolor{darkblue}{\textbf{\ipa{ɭɯ˧}}} 
\lhead{\firstmark}
\rhead{\botmark}

\subsection{\hspace{-0.5cm} {\Large \textcolor{darkblue}{\textbf{\ipa{ɬi˩ʁɑ˩}}}}\hspace{0.5cm}[\kern2pt{\textcolor{darkblue}{\textbf{\ipa{ɬi˩ʁɑ˩˥}}}}\kern2pt]} \hypertarget{Ki\string_BRA\string_B1}{}
\markboth{\textcolor{darkblue}{\textbf{\ipa{ɬi˩ʁɑ˩}}}}{}
\textcolor{teal}{\zh{形容词}} \hspace{4pt} \zh{声调类:} L.
\zh{大发雷霆。} \textcolor{Sepia}{\selectlanguage{english}Infuriated, in a rage (connotation: attitude of a violent and overbearing person).} \textcolor{PineGreen}{\selectlanguage{french}Furieux, en rage (attitude d'une personne violente et présomptueuse).}  ¶ \textcolor{darkblue}{\textbf{\ipa{ɬi˩ʁɑ˩ ʝi˧}}} \zh{大发雷霆} \textcolor{Sepia}{\selectlanguage{english}to abandon oneself to one's rage} \textcolor{PineGreen}{\selectlanguage{french}se livrer au courroux}  

\lhead{\firstmark}
\rhead{\botmark}

\subsection{\hspace{-0.5cm} {\Large \textcolor{darkblue}{\textbf{\ipa{ɬi˩ʈɯ˩mæ˥}}}}\hspace{0.5cm}[\kern2pt{\textcolor{darkblue}{\textbf{\ipa{ɬi˩ʈɯ˩mæ˩˥}}}}\kern2pt]} \hypertarget{Ki\string_Bt`M\string_Bm\{\string_T1}{}
\markboth{\textcolor{darkblue}{\textbf{\ipa{ɬi˩ʈɯ˩mæ˥}}}}{}
\textcolor{teal}{\zh{名词}} \hspace{4pt} \zh{声调类:} L+H\#.
\zh{主屋里面没有火铺的地方:没有木地板、小狗可以偶尔进来的地方(家人就给它扔骨头)。} \textcolor{Sepia}{\selectlanguage{english}Lower part of the main room.} \textcolor{PineGreen}{\selectlanguage{french}Contrebas du foyer: place dans la salle principale entre le foyer et la porte (où les chiens sont tolérés en fin de repas; on leur y jette des os et autres débris de nourriture; dans la maison de F4, à la date de l’enquête, c’est un endroit où rien ne recouvre le sol cimenté.}  ¶ \textcolor{darkblue}{\textbf{\ipa{u˧=ɻ̍˩, | kʰv̩˩mi˩ ʈʂʰɯ˩-jɤ˥ | ɖɯ˧-njɤ˧-zo˥ | ɬi˩ʈɯ˩mæ˥ hĩ˩ dʑo˩.}}} \zh{咱们家这只狗经常呆在主屋火塘下面的地方。} \textcolor{Sepia}{\selectlanguage{english}Us (=in our family), this dog is often seated in the lower part of the room.} \textcolor{PineGreen}{\selectlanguage{french}Nous (=dans notre maison), ce chien, il se tient souvent assis en contrebas du foyer.}  
 \zh{量词}: \textcolor{darkblue}{\textbf{\ipa{kʰwɤ˥}}} 
\lhead{\firstmark}
\rhead{\botmark}

\subsection{\hspace{-0.5cm} {\Large \textcolor{darkblue}{\textbf{\ipa{ɬi˩zo˩}}}}\hspace{0.5cm}[\kern2pt{\textcolor{darkblue}{\textbf{\ipa{ɬi˧zo˥}}}}\kern2pt]} \hypertarget{Ki\string_Bzo\string_B1}{}
\markboth{\textcolor{darkblue}{\textbf{\ipa{ɬi˩zo˩}}}}{}
\textcolor{teal}{\zh{名词}} \hspace{4pt} \zh{声调类:} L.
\zh{小獐子。} \textcolor{Sepia}{\selectlanguage{english}Baby roebuck.} \textcolor{PineGreen}{\selectlanguage{french}Bébé chevrotain.} 
\lhead{\firstmark}
\rhead{\botmark}

\subsection{\hspace{-0.5cm} {\Large \textcolor{darkblue}{\textbf{\ipa{ɬi˧˥}}}}\hspace{0.5cm}[\kern2pt{\textcolor{darkblue}{\textbf{\ipa{ɬi˧˥}}}}\kern2pt]} \hypertarget{Ki\string_M\string_T1}{}
\markboth{\textcolor{darkblue}{\textbf{\ipa{ɬi˧˥}}}}{}
\textcolor{teal}{\zh{动词}} \hspace{4pt} \zh{声调类:} MH.
\zh{晒干。} \textcolor{Sepia}{\selectlanguage{english}To dry in the sun.} \textcolor{PineGreen}{\selectlanguage{french}Faire sécher au soleil.}  ¶ \textcolor{darkblue}{\textbf{\ipa{le˧-pv̩˧ tʰi˧-ɬi˧˥}}} \zh{晒干} \textcolor{Sepia}{\selectlanguage{english}to put in the sun to dry} \textcolor{PineGreen}{\selectlanguage{french}exposer au soleil afin de faire sécher}  

\lhead{\firstmark}
\rhead{\botmark}

\subsection{\hspace{-0.5cm} {\Large \textcolor{darkblue}{\textbf{\ipa{ɬo˥}}}}\hspace{0.5cm}[\kern2pt{\textcolor{darkblue}{\textbf{\ipa{ɬo˧˥}}}}\kern2pt]} \hypertarget{Ko\string_T1}{}
\markboth{\textcolor{darkblue}{\textbf{\ipa{ɬo˥}}}}{}
\textcolor{teal}{\zh{名词}} \hspace{4pt} \zh{声调类:} \#H.
\zh{肋骨。} \textcolor{Sepia}{\selectlanguage{english}Rib.} \textcolor{PineGreen}{\selectlanguage{french}Côte.}  ¶ \textcolor{darkblue}{\textbf{\ipa{bo˩ɬo˧}}} \zh{猪肋骨} \textcolor{Sepia}{\selectlanguage{english}pork rib} \textcolor{PineGreen}{\selectlanguage{french}côtes de porc}  
 \zh{量词}: \textcolor{darkblue}{\textbf{\ipa{ɭɯ˧}}} 
\lhead{\firstmark}
\rhead{\botmark}

\subsection{\hspace{-0.5cm} {\Large \textcolor{darkblue}{\textbf{\ipa{ɬo˧kʰv̩˧}}}}\hspace{0.5cm}[\kern2pt{\textcolor{darkblue}{\textbf{\ipa{ɬo˧kʰv̩˧}}}}\kern2pt]} \hypertarget{Ko\string_Mk\string_hv\string_=\string_M1}{}
\markboth{\textcolor{darkblue}{\textbf{\ipa{ɬo˧kʰv̩˧}}}}{}
\textcolor{teal}{\zh{名词}} \hspace{4pt} \zh{声调类:} M.
\zh{胯。} \textcolor{Sepia}{\selectlanguage{english}Hip.} \textcolor{PineGreen}{\selectlanguage{french}Hanche.}  \zh{量词}: \textcolor{darkblue}{\textbf{\ipa{ɭɯ˧}}} \zh{~【同义词】~} \hyperlink{}{\textcolor{darkblue}{\textbf{\ipa{ɬo˩tsʰe˩mæ˥}}}}. 
\lhead{\firstmark}
\rhead{\botmark}

\subsection{\hspace{-0.5cm} {\Large \textcolor{darkblue}{\textbf{\ipa{ɬo˧pɤ˥}}}}\hspace{0.5cm}[\kern2pt{\textcolor{darkblue}{\textbf{\ipa{ɬo˧pɤ˥}}}}\kern2pt]} \hypertarget{Ko\string_Mp7\string_T1}{}
\markboth{\textcolor{darkblue}{\textbf{\ipa{ɬo˧pɤ˥}}}}{}
\textcolor{teal}{\zh{名词}} \hspace{4pt} \zh{声调类:} H\#.
\zh{水泡。} \textcolor{Sepia}{\selectlanguage{english}Blister (on the hands or feet).} \textcolor{PineGreen}{\selectlanguage{french}Ampoule.}  ¶ \textcolor{darkblue}{\textbf{\ipa{ɬo˧pɤ˥ qʰwæ˩-ze˩!}}} \zh{起了水泡!} \textcolor{Sepia}{\selectlanguage{english}(I/you/(s)he) got a blister!} \textcolor{PineGreen}{\selectlanguage{french}(Il s'est/Tu t'es/Je me suis) fait une ampoule!}  
 ¶ \textcolor{darkblue}{\textbf{\ipa{ɬo˧pɤ˥ | ɖɯ˧-ɭɯ˧ | qʰwæ˧-ze˥!}}} \zh{起了一个水泡!} \textcolor{Sepia}{\selectlanguage{english}(I/you/(s)he) got a blister!} \textcolor{PineGreen}{\selectlanguage{french}(Il s'est/Tu t'es/Je me suis) fait une ampoule!}  
 ¶ \textcolor{darkblue}{\textbf{\ipa{ɬo˧pɤ˥ | ʁo˩-po˥-ɳɯ˩ | ʈʂe˩˥}}} \zh{用针来扎水泡} \textcolor{Sepia}{\selectlanguage{english}to pierce a blister with a needle} \textcolor{PineGreen}{\selectlanguage{french}percer une ampoule à l'aide d'une aiguille}  
 \zh{量词}: \textcolor{darkblue}{\textbf{\ipa{ɭɯ˧}}} 
\lhead{\firstmark}
\rhead{\botmark}

\subsection{\hspace{-0.5cm} {\Large \textcolor{darkblue}{\textbf{\ipa{ɬo˧pv̩˥}}}}\hspace{0.5cm}[\kern2pt{\textcolor{darkblue}{\textbf{\ipa{ɬo˧pv̩˥}}}}\kern2pt]} \hypertarget{Ko\string_Mpv\string_=\string_T1}{}
\markboth{\textcolor{darkblue}{\textbf{\ipa{ɬo˧pv̩˥}}}}{}
\textcolor{teal}{\zh{名词}} \hspace{4pt} \zh{声调类:} H\#.
\zh{跪下磕头 (叩头)。} \textcolor{Sepia}{\selectlanguage{english}Kow-tow.} \textcolor{PineGreen}{\selectlanguage{french}Prosternation, kow-tow (très probablement emprunt tibétain).}  ¶ \textcolor{darkblue}{\textbf{\ipa{ɬo˧pv̩˥ ti˩}}} \zh{跪下磕头} \textcolor{Sepia}{\selectlanguage{english}to kow-tow} \textcolor{PineGreen}{\selectlanguage{french}se prosterner}  
 ¶ \textcolor{darkblue}{\textbf{\ipa{ɬo˧pv̩˥ | le˧-ti˩}}} \zh{跪下磕头} \textcolor{Sepia}{\selectlanguage{english}to kow-tow} \textcolor{PineGreen}{\selectlanguage{french}se prosterner}  
\zh{~【参考】~} \hyperlink{}{\textcolor{darkblue}{\textbf{\ipa{ɬo˧˥}}}} 
\lhead{\firstmark}
\rhead{\botmark}

\subsection{\hspace{-0.5cm} {\Large \textcolor{darkblue}{\textbf{\ipa{ɬo˧tɑ˧}}}}\hspace{0.5cm}[\kern2pt{\textcolor{darkblue}{\textbf{\ipa{ɬo˧tɑ˧}}}}\kern2pt]} \hypertarget{Ko\string_MtA\string_M1}{}
\markboth{\textcolor{darkblue}{\textbf{\ipa{ɬo˧tɑ˧}}}}{}
\textcolor{teal}{\zh{介词}} \hspace{4pt} \zh{声调类:} M.
\zh{旁边。} \textcolor{Sepia}{\selectlanguage{english}On the side of, beside.} \textcolor{PineGreen}{\selectlanguage{french}À côté de, sur le côté de.}  ¶ \textcolor{darkblue}{\textbf{\ipa{ɬo˧tɑ˧ ɻ̍˩}}} \zh{向侧面转} \textcolor{Sepia}{\selectlanguage{english}to turn to the side} \textcolor{PineGreen}{\selectlanguage{french}se tourner vers le côté, se tourner de côté}  
 ¶ \textcolor{darkblue}{\textbf{\ipa{ʁo˧qʰwɤ˩ | ɬo˧tɑ˧ | go˩˥}}} \zh{头疼,太阳穴阵痛} \textcolor{Sepia}{\selectlanguage{english}to have a headache; one's temples are throbbing (literally: 'to hurt on the sides of the head')} \textcolor{PineGreen}{\selectlanguage{french}avoir mal sur le côté de la tête, avoir les tempes qui bourdonnent (littéralement: 'avoir mal sur les côtés de la tête')}  

\lhead{\firstmark}
\rhead{\botmark}

\subsection{\hspace{-0.5cm} {\Large \textcolor{darkblue}{\textbf{\ipa{ɬo˩kɤ˩}}}}\hspace{0.5cm}[\kern2pt{\textcolor{darkblue}{\textbf{\ipa{ɬo˩kɤ˩˥}}}}\kern2pt]} \hypertarget{Ko\string_Bk7\string_B1}{}
\markboth{\textcolor{darkblue}{\textbf{\ipa{ɬo˩kɤ˩}}}}{}
\textcolor{teal}{\zh{名词}} \hspace{4pt} \zh{声调类:} L.
\zh{肋骨。} \textcolor{Sepia}{\selectlanguage{english}Rib.} \textcolor{PineGreen}{\selectlanguage{french}Côte (partie du corps).}  \zh{量词}: \textcolor{darkblue}{\textbf{\ipa{kɤ˧˥}}} 
\lhead{\firstmark}
\rhead{\botmark}

\subsection{\hspace{-0.5cm} {\Large \textcolor{darkblue}{\textbf{\ipa{ɬo˩tsʰe˩mæ˥}}}}\hspace{0.5cm}[\kern2pt{\textcolor{darkblue}{\textbf{\ipa{ɬo˩tsʰe˩mæ˥}}}}\kern2pt]} \hypertarget{Ko\string_Bts\string_he\string_Bm\{\string_T1}{}
\markboth{\textcolor{darkblue}{\textbf{\ipa{ɬo˩tsʰe˩mæ˥}}}}{}
\textcolor{teal}{\zh{名词}} \hspace{4pt} \zh{声调类:} L+H\#.
\textit{\zh{古语}} [\zh{古语}] \zh{胯。} \textcolor{Sepia}{\selectlanguage{english}Hip.} \textcolor{PineGreen}{\selectlanguage{french}Hanche.}  \zh{量词}: \textcolor{darkblue}{\textbf{\ipa{ɭɯ˧}}} \zh{~【同义词】~} \hyperlink{}{\textcolor{darkblue}{\textbf{\ipa{ɬo˧kʰv̩˧}}}}. 
\lhead{\firstmark}
\rhead{\botmark}

\subsection{\hspace{-0.5cm} {\Large \textcolor{darkblue}{\textbf{\ipa{ɬo˧˥}}}}\hspace{0.5cm}[\kern2pt{\textcolor{darkblue}{\textbf{\ipa{ɬo˧˥}}}}\kern2pt]} \hypertarget{Ko\string_M\string_T1}{}
\markboth{\textcolor{darkblue}{\textbf{\ipa{ɬo˧˥}}}}{}
\textcolor{teal}{\zh{形容词}} \hspace{4pt} \zh{声调类:} MH.
\zh{深(水深)。} \textcolor{Sepia}{\selectlanguage{english}Deep (water).} \textcolor{PineGreen}{\selectlanguage{french}Profond (eau).} 
\lhead{\firstmark}
\rhead{\botmark}

\subsection{\hspace{-0.5cm} {\Large \textcolor{darkblue}{\textbf{\ipa{ɬv̩˧˥}}}}\hspace{0.5cm}[\kern2pt{\textcolor{darkblue}{\textbf{\ipa{ɬv̩˧˥}}}}\kern2pt]} \hypertarget{Kv\string_=\string_M\string_T1}{}
\markboth{\textcolor{darkblue}{\textbf{\ipa{ɬv̩˧˥}}}}{}
\textcolor{teal}{\zh{名词}} \hspace{4pt} \zh{声调类:} MH.
\ding{202} \zh{脑子、脑髓。} \textcolor{Sepia}{\selectlanguage{english}Brains.} \textcolor{PineGreen}{\selectlanguage{french}Cerveau, cervelle.}  \zh{量词}: \textcolor{darkblue}{\textbf{\ipa{ʈv̩˩}}} \ding{203} \zh{骨髓。} \textcolor{Sepia}{\selectlanguage{english}Marrow.} \textcolor{PineGreen}{\selectlanguage{french}Moëlle (des os).} 
\lhead{\firstmark}
\rhead{\botmark}

\subsection{\hspace{-0.5cm} {\Large \textcolor{darkblue}{\textbf{\ipa{ɬv̩˩\textsubscript{a}}}} \textsubscript{1}}\hspace{0.5cm}[\kern2pt{\textcolor{darkblue}{\textbf{\ipa{ɬv̩˩˥}}}}\kern2pt]} \hypertarget{Kv\string_=\string_Ba1}{}
\markboth{\textcolor{darkblue}{\textbf{\ipa{ɬv̩˩\textsubscript{a}}}} \textsubscript{1}}{}
\textcolor{teal}{\zh{动词}} \hspace{4pt} \zh{声调类:} L\textsubscript{a}.
\zh{含在嘴里、在嘴巴里溶化。} \textcolor{Sepia}{\selectlanguage{english}To hold in the mouth; to let melt in the mouth.} \textcolor{PineGreen}{\selectlanguage{french}Garder dans la bouche, laisser fondre dans la bouche.}  ¶ \textcolor{darkblue}{\textbf{\ipa{tso˧\textasciitilde{}tso˧ ɬv̩˥}}} \zh{含在嘴里(情景:一个小孩把不能吃的东西含在嘴巴里)} \textcolor{Sepia}{\selectlanguage{english}to hold something in the mouth, to have something in the mouth (context: a small child who does not yet know to distinguish between food and non-edible stuff puts things in its mouth)} \textcolor{PineGreen}{\selectlanguage{french}mettre des choses dans sa bouche (contexte: un petit enfant qui ne fait pas encore la différence entre nourriture et choses non comestibles met des choses dans sa bouche)}  

\lhead{\firstmark}
\rhead{\botmark}

\subsection{\hspace{-0.5cm} {\Large \textcolor{darkblue}{\textbf{\ipa{ɬv̩˩\textsubscript{a}}}} \textsubscript{2}}\hspace{0.5cm}[\kern2pt{\textcolor{darkblue}{\textbf{\ipa{ɬv̩˩˥}}}}\kern2pt]} \hypertarget{Kv\string_=\string_Ba2}{}
\markboth{\textcolor{darkblue}{\textbf{\ipa{ɬv̩˩\textsubscript{a}}}} \textsubscript{2}}{}
\textcolor{teal}{\zh{形容词}} \hspace{4pt} \zh{声调类:} L\textsubscript{a}.
\zh{温暖,暖和。} \textcolor{Sepia}{\selectlanguage{english}Warm.} \textcolor{PineGreen}{\selectlanguage{french}Chaud, tiède (agréablement tiède, pas froid).}  ¶ \textcolor{darkblue}{\textbf{\ipa{dʑɤ˩˥ | ɬv̩˩˥}}} \zh{温暖} \textcolor{Sepia}{\selectlanguage{english}nice and warm} \textcolor{PineGreen}{\selectlanguage{french}très tiède, bien tiède}  
 ¶ \textcolor{darkblue}{\textbf{\ipa{ɖwæ˧˥ | ɬv̩˩˥}}} \zh{很暖和} \textcolor{Sepia}{\selectlanguage{english}\mytextsc{intensive}.very} \textcolor{PineGreen}{\selectlanguage{french}\mytextsc{intensif}.très: très tiède}  
 ¶ \textcolor{darkblue}{\textbf{\ipa{ɬv̩˩-hĩ˩˥}}} \zh{温暖的} \textcolor{Sepia}{\selectlanguage{english}\mytextsc{rel}/\mytextsc{nmlz}} \textcolor{PineGreen}{\selectlanguage{french}\mytextsc{rel}/\mytextsc{nmlz}}  

\lhead{\firstmark}
\rhead{\botmark}

\subsection{\hspace{-0.5cm} {\Large \textcolor{darkblue}{\textbf{\ipa{ɬv̩˩\textsubscript{a}}}} \textsubscript{3}}\hspace{0.5cm}[\kern2pt{\textcolor{darkblue}{\textbf{\ipa{ɬv̩˩˥}}}}\kern2pt]} \hypertarget{Kv\string_=\string_Ba3}{}
\markboth{\textcolor{darkblue}{\textbf{\ipa{ɬv̩˩\textsubscript{a}}}} \textsubscript{3}}{}
\textcolor{teal}{\zh{动词}} \hspace{4pt} \zh{声调类:} L\textsubscript{a}.
\zh{热饭。} \textcolor{Sepia}{\selectlanguage{english}To warm up (food).} \textcolor{PineGreen}{\selectlanguage{french}Réchauffer de la nourriture.}  ¶ \textcolor{darkblue}{\textbf{\ipa{hɑ˧ ɬv̩˧˥}}} \zh{热饭} \textcolor{Sepia}{\selectlanguage{english}to warm up rice / food} \textcolor{PineGreen}{\selectlanguage{french}réchauffer du riz / de la nourriture}  
 ¶ \textcolor{darkblue}{\textbf{\ipa{hɑ˧ | le˧-ɬv̩˩}}} \zh{热饭} \textcolor{Sepia}{\selectlanguage{english}to warm up rice / food} \textcolor{PineGreen}{\selectlanguage{french}réchauffer du riz / de la nourriture}  
 ¶ \textcolor{darkblue}{\textbf{\ipa{hɑ˧ | ɖɯ˧-ɬv̩˧\textasciitilde{}ɬv̩˥-ɻ̍˩}}} \zh{饭热一热} \textcolor{Sepia}{\selectlanguage{english}to warm up food a little} \textcolor{PineGreen}{\selectlanguage{french}réchauffer un peu la nourriture}  

\lhead{\firstmark}
\rhead{\botmark}

\subsection{\hspace{-0.5cm} {\Large \textcolor{darkblue}{\textbf{\ipa{ɬv̩˧gv̩\#˥}}}}\hspace{0.5cm}[\kern2pt{\textcolor{darkblue}{\textbf{\ipa{ɬv̩˧gv̩˥}}}}\kern2pt]} \hypertarget{Kv\string_=\string_Mgv\string_=\#\string_T1}{}
\markboth{\textcolor{darkblue}{\textbf{\ipa{ɬv̩˧gv̩\#˥}}}}{}
\textcolor{teal}{\zh{名词}} \hspace{4pt} \zh{声调类:} \#H.
\zh{火葬后第七天的送食物仪式。} \textcolor{Sepia}{\selectlanguage{english}Ritual offering of food to the deceased, seven days after cremation.} \textcolor{PineGreen}{\selectlanguage{french}Nourriture qu'on offre rituellement au défunt, sept jours après sa crémation.} 
\lhead{\firstmark}
\rhead{\botmark}

\subsection{\hspace{-0.5cm} {\Large \textcolor{darkblue}{\textbf{\ipa{ɬv̩˧mi˧mæ˧dv̩˧mi\#˥}}}}\hspace{0.5cm}[\kern2pt{\textcolor{darkblue}{\textbf{\ipa{ɬv̩˧mi˧mæ˧dv̩˧mi˧}}}}\kern2pt]} \hypertarget{Kv\string_=\string_Mmi\string_Mm\{\string_Mdv\string_=\string_Mmi\#\string_T1}{}
\markboth{\textcolor{darkblue}{\textbf{\ipa{ɬv̩˧mi˧mæ˧dv̩˧mi\#˥}}}}{}
\textcolor{teal}{\zh{名词}} \hspace{4pt} \zh{声调类:} \#H.
\zh{螳螂。} \textcolor{Sepia}{\selectlanguage{english}Praying mantis.} \textcolor{PineGreen}{\selectlanguage{french}Mante religieuse.}  ¶ \textcolor{darkblue}{\textbf{\ipa{ɬv̩˧mi˧mæ˧dv̩˧mi˧ tʰv̩˧-mi˧˥ / ɬv̩˧mi˧mæ˧dv̩˧mi˧ tʰv̩˧-mi˥\#}}} \zh{那只螳螂} \textcolor{Sepia}{\selectlanguage{english}\mytextsc{n}+\mytextsc{dem}+\mytextsc{clf}} \textcolor{PineGreen}{\selectlanguage{french}\mytextsc{n}+\mytextsc{dem}+\mytextsc{clf}}  
 \zh{量词}: \textcolor{darkblue}{\textbf{\ipa{mi˩}}} 
\lhead{\firstmark}
\rhead{\botmark}

\subsection{\hspace{-0.5cm} {\Large \textcolor{darkblue}{\textbf{\ipa{ɬv̩˧ʁwɤ\#˥}}}}\hspace{0.5cm}[\kern2pt{\textcolor{darkblue}{\textbf{\ipa{ɬv̩˧ʁwɤ˧}}}}\kern2pt]} \hypertarget{Kv\string_=\string_MRw7\#\string_T1}{}
\markboth{\textcolor{darkblue}{\textbf{\ipa{ɬv̩˧ʁwɤ\#˥}}}}{}
\textcolor{teal}{\zh{名词}} \hspace{4pt} \zh{声调类:} \#H.
\zh{村落名。} \textcolor{Sepia}{\selectlanguage{english}Village name.} \textcolor{PineGreen}{\selectlanguage{french}Village en aval de Qiansuo; leur langue serait relativement proche de celle de la vallée de Yongning.} 
\lhead{\firstmark}
\rhead{\botmark}

\newpage
\section*{\centering- \textcolor{darkblue}{\textbf{\ipa{m}}} -}
\subsection{\hspace{-0.5cm} {\Large \textcolor{darkblue}{\textbf{\ipa{mɑ˧pʰv̩˧}}}}\hspace{0.5cm}[\kern2pt{\textcolor{darkblue}{\textbf{\ipa{mɑ˧pʰv̩˧}}}}\kern2pt]} \hypertarget{mA\string_Mp\string_hv\string_=\string_M1}{}
\markboth{\textcolor{darkblue}{\textbf{\ipa{mɑ˧pʰv̩˧}}}}{}
\textcolor{teal}{\zh{名词}} \hspace{4pt} \zh{声调类:} M.
\zh{酥油。} \textcolor{Sepia}{\selectlanguage{english}Butter.} \textcolor{PineGreen}{\selectlanguage{french}Beurre (pour la préparation du thé au beurre).} 
\lhead{\firstmark}
\rhead{\botmark}

\subsection{\hspace{-0.5cm} {\Large \textcolor{darkblue}{\textbf{\ipa{mɑ˧tsɑ˥}}}}\hspace{0.5cm}[\kern2pt{\textcolor{darkblue}{\textbf{\ipa{mɑ˧tsɑ˧}}}}\kern2pt]} \hypertarget{mA\string_MtsA\string_T1}{}
\markboth{\textcolor{darkblue}{\textbf{\ipa{mɑ˧tsɑ˥}}}}{}
\textcolor{teal}{\zh{名词}} \hspace{4pt} \zh{声调类:} H\#.
\zh{来历、发源地、深层原因/来源、来龙去脉、脉络。} \textcolor{Sepia}{\selectlanguage{english}Origin, distant cause, remote cause.} \textcolor{PineGreen}{\selectlanguage{french}Origine, cause (lointaine).}  ¶ \textcolor{darkblue}{\textbf{\ipa{mɑ˧tsɑ˥ | ʈʂʰɯ˧-qo˧ le˧-tsʰɯ˩-ɲi˩! |}}} \zh{这(件事情)出处很远! / 有它的来龙去脉(=不是突然一下子出现的)!} \textcolor{Sepia}{\selectlanguage{english}(Of an event:) It comes from afar! / It does not take place simply by chance: there is a long story behind it!} \textcolor{PineGreen}{\selectlanguage{french}(Au sujet d'un événement) Ca vient de loin! / ça a une origine/ça n'arrive pas par hasard!}  
 ¶ \textcolor{darkblue}{\textbf{\ipa{mɑ˧tsɑ˥ ʈʂʰɯ˩-kʰwɤ˩ |}}} \zh{这个来历} \textcolor{Sepia}{\selectlanguage{english}\mytextsc{n}+\mytextsc{dem}+\mytextsc{clf}: this cause, this origin} \textcolor{PineGreen}{\selectlanguage{french}\mytextsc{n}+\mytextsc{dem}+\mytextsc{clf}: cette cause, cette origine}  
 \zh{量词}: \textcolor{darkblue}{\textbf{\ipa{kʰwɤ˥}}} 
\lhead{\firstmark}
\rhead{\botmark}

\subsection{\hspace{-0.5cm} {\Large \textcolor{darkblue}{\textbf{\ipa{mɑ˩dzɑ˩}}}}\hspace{0.5cm}[\kern2pt{\textcolor{darkblue}{\textbf{\ipa{mɑ˧dzɑ˧}}}}\kern2pt]} \hypertarget{mA\string_BdzA\string_B1}{}
\markboth{\textcolor{darkblue}{\textbf{\ipa{mɑ˩dzɑ˩}}}}{}
\textcolor{teal}{\zh{名词}} \hspace{4pt} \zh{声调类:} L.
\zh{墨。} \textcolor{Sepia}{\selectlanguage{english}Ink (solid).} \textcolor{PineGreen}{\selectlanguage{french}Encre (solide).}  \zh{量词}: \textcolor{darkblue}{\textbf{\ipa{qʰwɤ˧˥}}} 
\lhead{\firstmark}
\rhead{\botmark}

\subsection{\hspace{-0.5cm} {\Large \textcolor{darkblue}{\textbf{\ipa{mɑ˩ɳɯ\#˥}}}}\hspace{0.5cm}[\kern2pt{\textcolor{darkblue}{\textbf{\ipa{xxxx non-correspondance entre le nombre de morphèmes et le nombre de tons de morphèmes}}}}\kern2pt]} \hypertarget{mA\string_Bn`M\#\string_T1}{}
\markboth{\textcolor{darkblue}{\textbf{\ipa{mɑ˩ɳɯ\#˥}}}}{}
\textcolor{teal}{\zh{名词}} \hspace{4pt} \zh{声调类:} LM+\#H.
\zh{嘛呢堆。} \textcolor{Sepia}{\selectlanguage{english}Mani wall, Mani pile: pile built of rubble and sand, with carved stone tablets, most with the inscription Om Mani Padme Hum. A Mani wall should be passed or circumvented from the left side, the clockwise direction in which the universe revolves, according to Buddhist doctrine.} \textcolor{PineGreen}{\selectlanguage{french}Mur de mani: mur de pierre sèche et de sable, comportant des tablettes de pierre sur lesquelles est gravé une inscription: le plus souvent Om Mani Padme Hum. Un mur de mani doit être contourné dans le sens des aiguilles d'une montre: le sens de rotation de l'univers, selon la doctrine bouddhiste.}  \zh{量词}: \textcolor{darkblue}{\textbf{\ipa{ɭɯ˧}}} \zh{~【参考】~} \textcolor{darkblue}{\textbf{\ipa{mɑ˩ɳɯ˧-do˥bv˩, do˩bv̩\#˥}}} 
\lhead{\firstmark}
\rhead{\botmark}

\subsection{\hspace{-0.5cm} {\Large \textcolor{darkblue}{\textbf{\ipa{mɑ˩ɳɯ˧-do˥bv̩˩}}}}\hspace{0.5cm}[\kern2pt{\textcolor{darkblue}{\textbf{\ipa{xxxx non-correspondance entre le nombre de morphèmes et le nombre de tons de morphèmes}}}}\kern2pt]} \hypertarget{mA\string_Bn`M\string_M-do\string_Tbv\string_=\string_B1}{}
\markboth{\textcolor{darkblue}{\textbf{\ipa{mɑ˩ɳɯ˧-do˥bv̩˩}}}}{}
\textcolor{teal}{\zh{名词}} \hspace{4pt} \zh{声调类:} LM+\#H-.
\zh{嘛呢堆。} \textcolor{Sepia}{\selectlanguage{english}Mani wall, Mani pile: pile built of rubble and sand, with carved stone tablets, most with the inscription Om Mani Padme Hum. A Mani wall should be passed or circumvented from the left side, the clockwise direction in which the universe revolves, according to Buddhist doctrine.} \textcolor{PineGreen}{\selectlanguage{french}Mur de mani: mur de pierre sèche et de sable, comportant des tablettes de pierre sur lesquelles est gravé une inscription: le plus souvent Om Mani Padme Hum. Un mur de mani doit être contourné dans le sens des aiguilles d'une montre: le sens de rotation de l'univers, selon la doctrine bouddhiste.}  \zh{量词}: \textcolor{darkblue}{\textbf{\ipa{ɭɯ˧}}} \zh{~【参考】~} \textcolor{darkblue}{\textbf{\ipa{mɑ˩ɳɯ\#˥, do˩bv̩\#˥}}} 
\lhead{\firstmark}
\rhead{\botmark}

\subsection{\hspace{-0.5cm} {\Large \textcolor{darkblue}{\textbf{\ipa{mæ˧}}}}\hspace{0.5cm}[\kern2pt{\textcolor{darkblue}{\textbf{\ipa{mæ˥}}}}\kern2pt]} \hypertarget{m\{\string_M1}{}
\markboth{\textcolor{darkblue}{\textbf{\ipa{mæ˧}}}}{}
\textcolor{teal}{\zh{动词}} \hspace{4pt} \zh{声调类:} M.
\zh{……成、……成功。} \textcolor{Sepia}{\selectlanguage{english}To achieve, to succeed in, to complete (a task).} \textcolor{PineGreen}{\selectlanguage{french}Parvenir à, réussir à.}  ¶ \textcolor{darkblue}{\textbf{\ipa{njɤ˧ ɖʐɤ˧˥ | tʰi˧-mɤ˧-mæ˧!}}} \zh{我够不着!(例如:够不着树枝上的果子)} \textcolor{Sepia}{\selectlanguage{english}I can't fetch it!} \textcolor{PineGreen}{\selectlanguage{french}je ne parviens pas à attraper (ex.: un fruit sur une branche trop élevée)}  

\lhead{\firstmark}
\rhead{\botmark}

\subsection{\hspace{-0.5cm} {\Large \textcolor{darkblue}{\textbf{\ipa{mæ˧}}}}\hspace{0.5cm}[\kern2pt{\textcolor{darkblue}{\textbf{\ipa{mæ˥}}}}\kern2pt]} \hypertarget{m\{\string_M1}{}
\markboth{\textcolor{darkblue}{\textbf{\ipa{mæ˧}}}}{}
\textcolor{teal}{\zh{语气助词}} \hspace{4pt} \zh{声调类:} M.
\zh{句尾助词,表示显然、理所当然:“……呗!”。} \textcolor{Sepia}{\selectlanguage{english}Final particle conveying obviousness.} \textcolor{PineGreen}{\selectlanguage{french}Particule finale exprimant l'évidence.}  ¶ \textcolor{darkblue}{\textbf{\ipa{[Healing.66] hu˧mi˧-ʈʂʰæ˧ɣɯ˧ | le˧-ʈʰɯ˩, | le˧-qʰwɤ˧-ze˧ mæ˧! |}}} \zh{吃了胃药,就好了呗!} \textcolor{Sepia}{\selectlanguage{english}[Nowadays] one simply takes medicines for the stomach, and one is healed! [unlike in the old times, when there were no hospitals]} \textcolor{PineGreen}{\selectlanguage{french}On prend des médicaments pour l'estomac, et ça guérit!}  

\lhead{\firstmark}
\rhead{\botmark}

\subsection{\hspace{-0.5cm} {\Large \textcolor{darkblue}{\textbf{\ipa{mæ˧}}} \textsubscript{1}}\hspace{0.5cm}[\kern2pt{\textcolor{darkblue}{\textbf{\ipa{mæ˥}}}}\kern2pt]} \hypertarget{m\{\string_M1}{}
\markboth{\textcolor{darkblue}{\textbf{\ipa{mæ˧}}} \textsubscript{1}}{}
\textcolor{teal}{\zh{动词}} \hspace{4pt} \zh{声调类:} M.
\zh{(有)空。} \textcolor{Sepia}{\selectlanguage{english}To be free to, to have the time to.} \textcolor{PineGreen}{\selectlanguage{french}Avoir le temps de, être libre.}  ¶ \textcolor{darkblue}{\textbf{\ipa{njɤ˧ | mɤ˧-mæ˧.}}} \zh{我忙、我没有空} \textcolor{Sepia}{\selectlanguage{english}I do not have the time; I am busy} \textcolor{PineGreen}{\selectlanguage{french}je suis occupé, je n'ai pas le temps}  
 ¶ \textcolor{darkblue}{\textbf{\ipa{njɤ˧ | mæ˧-mɤ˧-ho˩.}}} \zh{我不会有时间。} \textcolor{Sepia}{\selectlanguage{english}I won't have the time.} \textcolor{PineGreen}{\selectlanguage{french}Je ne vais pas avoir le temps.}  

\lhead{\firstmark}
\rhead{\botmark}

\subsection{\hspace{-0.5cm} {\Large \textcolor{darkblue}{\textbf{\ipa{mæ˧}}} \textsubscript{2}}\hspace{0.5cm}[\kern2pt{\textcolor{darkblue}{\textbf{\ipa{mæ˥}}}}\kern2pt]} \hypertarget{m\{\string_M2}{}
\markboth{\textcolor{darkblue}{\textbf{\ipa{mæ˧}}} \textsubscript{2}}{}
\textcolor{teal}{\zh{动词}} \hspace{4pt} \zh{声调类:} M.
\zh{能够(做)。} \textcolor{Sepia}{\selectlanguage{english}To manage (to do something).} \textcolor{PineGreen}{\selectlanguage{french}Parvenir à.}  ¶ \textcolor{darkblue}{\textbf{\ipa{ɖɯ˩-hĩ˩ qʰɑ˥ mæ˩\textasciitilde{}mæ˩! | tɕi˩-hĩ˩ lə˥-mɤ˩-mæ˩! / ɖɯ˩-hĩ˩˥, | qʰɑ˧ mæ˥\textasciitilde{}mæ˩! | tɕi˩-hĩ˩˥, | le˧-mɤ˧-mæ˧!}}} \zh{“大人管干活,小孩管玩耍!”这个谚语的意思是:不要让孩子干活,每个年龄有自己的事,孩子的事就是玩。成年人的活儿,不是他们的事!} \textcolor{Sepia}{\selectlanguage{english}“Adults can manage all sorts of things, (whereas) children can't manage (that much) yet!” This saying is used when someone puts high demands on children or adolescents: Let the children play! To each age its occupations: children should play, not work. Adults' tasks are not their business!} \textcolor{PineGreen}{\selectlanguage{french}“Les adultes peuvent tout faire; les enfants, eux, n'y arrivent pas/n'en sont pas capables!” Sens: s'adresse à quelqu'un qui assigne des tâches aux enfants et adolescents: Laissez les enfants jouer! A chacun ses occupations: les adultes travaillent; les enfants, leur tâche, c'est de s'amuser entre eux, pas de travailler. Le travail des grands, c'est pas leur affaire!}  

\lhead{\firstmark}
\rhead{\botmark}

\subsection{\hspace{-0.5cm} {\Large \textcolor{darkblue}{\textbf{\ipa{mæ˧\textsubscript{a}}}}}\hspace{0.5cm}[\kern2pt{\textcolor{darkblue}{\textbf{\ipa{mæ˥}}}}\kern2pt]} \hypertarget{m\{\string_Ma1}{}
\markboth{\textcolor{darkblue}{\textbf{\ipa{mæ˧\textsubscript{a}}}}}{}
\textcolor{teal}{\zh{动词}} \hspace{4pt} \zh{声调类:} M\textsubscript{a}.
\ding{202} \zh{钩住(东西)。} \textcolor{Sepia}{\selectlanguage{english}To clutch, to catch hold of.} \textcolor{PineGreen}{\selectlanguage{french}Attraper (un objet en hauteur).}  ¶ \textcolor{darkblue}{\textbf{\ipa{tʰi˧-mæ˧-ze˧}}} \zh{钩住了} \textcolor{Sepia}{\selectlanguage{english}\mytextsc{dur} \string_ \mytextsc{pfv}} \textcolor{PineGreen}{\selectlanguage{french}\mytextsc{dur} \string_ \mytextsc{pfv}}  
\ding{203} \zh{跟上。} \textcolor{Sepia}{\selectlanguage{english}To catch up with (someone).} \textcolor{PineGreen}{\selectlanguage{french}Rattraper, rejoindre (quelqu'un qui est plus avant sur un chemin/une route).} 
\lhead{\firstmark}
\rhead{\botmark}

\subsection{\hspace{-0.5cm} {\Large \textcolor{darkblue}{\textbf{\ipa{mæ˧pæ˧}}}}\hspace{0.5cm}[\kern2pt{\textcolor{darkblue}{\textbf{\ipa{mæ˩pæ˥}}}}\kern2pt]} \hypertarget{m\{\string_Mp\{\string_M1}{}
\markboth{\textcolor{darkblue}{\textbf{\ipa{mæ˧pæ˧}}}}{}
\textcolor{teal}{\zh{名词}} \hspace{4pt} \zh{声调类:} M.
\zh{大筛子。} \textcolor{Sepia}{\selectlanguage{english}Large sifter.} \textcolor{PineGreen}{\selectlanguage{french}Vannerie.}  \zh{量词}: \textcolor{darkblue}{\textbf{\ipa{nɑ˧}}} 
\lhead{\firstmark}
\rhead{\botmark}

\subsection{\hspace{-0.5cm} {\Large \textcolor{darkblue}{\textbf{\ipa{-mæ˧qo˩}}}}\hspace{0.5cm}[\kern2pt{\textcolor{darkblue}{\textbf{\ipa{mæ˧qo˩}}}}\kern2pt]} \hypertarget{-m\{\string_Mqo\string_B1}{}
\markboth{\textcolor{darkblue}{\textbf{\ipa{-mæ˧qo˩}}}}{}
\textcolor{teal}{\zh{后置词}} \hspace{4pt} \zh{声调类:} L\#.
\zh{下面,后面。} \textcolor{Sepia}{\selectlanguage{english}Below, behind.} \textcolor{PineGreen}{\selectlanguage{french}Derrière, sous.} \zh{~【参考】~} \hyperlink{}{\textcolor{darkblue}{\textbf{\ipa{mæ˧qo˩}}}} 
\lhead{\firstmark}
\rhead{\botmark}

\subsection{\hspace{-0.5cm} {\Large \textcolor{darkblue}{\textbf{\ipa{mæ˧qo˩}}}}\hspace{0.5cm}[\kern2pt{\textcolor{darkblue}{\textbf{\ipa{mæ˧qo˧}}}}\kern2pt]} \hypertarget{m\{\string_Mqo\string_B1}{}
\markboth{\textcolor{darkblue}{\textbf{\ipa{mæ˧qo˩}}}}{}
\textcolor{teal}{\zh{助词}} \hspace{4pt} \zh{声调类:} L\#.
\zh{在尽头、在极点,在下面、在后面。} \textcolor{Sepia}{\selectlanguage{english}At the extremity, at the end; at the bottom, in the lower part.} \textcolor{PineGreen}{\selectlanguage{french}En bas, au fond; à l'arrière, derrière.} \zh{~【参考】~} \hyperlink{}{\textcolor{darkblue}{\textbf{\ipa{-mæ˧qo˩}}}} 
\lhead{\firstmark}
\rhead{\botmark}

\subsection{\hspace{-0.5cm} {\Large \textcolor{darkblue}{\textbf{\ipa{mæ˧qv̩˩}}}}\hspace{0.5cm}[\kern2pt{\textcolor{darkblue}{\textbf{\ipa{mæ˧qv̩˩}}}}\kern2pt]} \hypertarget{m\{\string_Mqv\string_=\string_B1}{}
\markboth{\textcolor{darkblue}{\textbf{\ipa{mæ˧qv̩˩}}}}{}
\textcolor{teal}{\zh{名词}} \hspace{4pt} \zh{声调类:} L\#.
\zh{尾巴。} \textcolor{Sepia}{\selectlanguage{english}Tail.} \textcolor{PineGreen}{\selectlanguage{french}Queue.}  ¶ \textcolor{darkblue}{\textbf{\ipa{ʝi˧-mæ˧qv̩˥}}} \zh{牛尾巴} \textcolor{Sepia}{\selectlanguage{english}cow's tail} \textcolor{PineGreen}{\selectlanguage{french}queue de la vache}  
 \zh{量词}: \textcolor{darkblue}{\textbf{\ipa{ɭɯ˧}}} 
\lhead{\firstmark}
\rhead{\botmark}

\subsection{\hspace{-0.5cm} {\Large \textcolor{darkblue}{\textbf{\ipa{mæ˧ɻæ˩}}}}\hspace{0.5cm}[\kern2pt{\textcolor{darkblue}{\textbf{\ipa{mæ˧ɻæ˩}}}}\kern2pt]} \hypertarget{m\{\string_Mr£`\{\string_B1}{}
\markboth{\textcolor{darkblue}{\textbf{\ipa{mæ˧ɻæ˩}}}}{}
\textcolor{teal}{\zh{名词}} \hspace{4pt} \zh{声调类:} L\#.
\zh{植物油。} \textcolor{Sepia}{\selectlanguage{english}Vegetable oil.} \textcolor{PineGreen}{\selectlanguage{french}Huile végétale.} 
\lhead{\firstmark}
\rhead{\botmark}

\subsection{\hspace{-0.5cm} {\Large \textcolor{darkblue}{\textbf{\ipa{mæ˧ɻ̃\#˥}}}}\hspace{0.5cm}[\kern2pt{\textcolor{darkblue}{\textbf{\ipa{mæ˧ɻ̃˩}}}}\kern2pt]} \hypertarget{m\{\string_Mr£`\string_~\#\string_T1}{}
\markboth{\textcolor{darkblue}{\textbf{\ipa{mæ˧ɻ̃\#˥}}}}{}
\textcolor{teal}{\zh{名词}} \hspace{4pt} \zh{声调类:} \#H.
\zh{尾椎骨。} \textcolor{Sepia}{\selectlanguage{english}Coccyx.} \textcolor{PineGreen}{\selectlanguage{french}Coccyx.} \zh{当地汉语方言:}\zh{尾结骨。} \zh{量词}: \textcolor{darkblue}{\textbf{\ipa{ɭɯ˧}}} 
\lhead{\firstmark}
\rhead{\botmark}

\subsection{\hspace{-0.5cm} {\Large \textcolor{darkblue}{\textbf{\ipa{mæ˩}}}}\hspace{0.5cm}[\kern2pt{\textcolor{darkblue}{\textbf{\ipa{mæ˩˥}}}}\kern2pt]} \hypertarget{m\{\string_B1}{}
\markboth{\textcolor{darkblue}{\textbf{\ipa{mæ˩}}}}{}
\textcolor{teal}{\zh{动词}} \hspace{4pt} \zh{声调类:} L.
\zh{灌溉。} \textcolor{Sepia}{\selectlanguage{english}To water, to irrigate (making small trenches and pouring water into them).} \textcolor{PineGreen}{\selectlanguage{french}Irriguer (en faisant couler de l'eau dans de petites tranchées).}  ¶ \textcolor{darkblue}{\textbf{\ipa{dʑɯ˩ mæ˩˥}}} \zh{浇灌} \textcolor{Sepia}{\selectlanguage{english}to irrigate, to water} \textcolor{PineGreen}{\selectlanguage{french}irriguer, arroser, mettre de l’eau}  
 ¶ \textcolor{darkblue}{\textbf{\ipa{dʑɯ˧ | le˧-mæ˩}}} \zh{浇灌} \textcolor{Sepia}{\selectlanguage{english}\mytextsc{accomp}: to water, to irrigate} \textcolor{PineGreen}{\selectlanguage{french}\mytextsc{accomp}: irriguer, arroser}  

\lhead{\firstmark}
\rhead{\botmark}

\subsection{\hspace{-0.5cm} {\Large \textcolor{darkblue}{\textbf{\ipa{mæ˩\textsubscript{a}}}}}\hspace{0.5cm}[\kern2pt{\textcolor{darkblue}{\textbf{\ipa{mæ˩˥}}}}\kern2pt]} \hypertarget{m\{\string_Ba1}{}
\markboth{\textcolor{darkblue}{\textbf{\ipa{mæ˩\textsubscript{a}}}}}{}
\textcolor{teal}{\zh{动词}} \hspace{4pt} \zh{声调类:} L\textsubscript{a}.
\zh{瞄准,指。} \textcolor{Sepia}{\selectlanguage{english}To aim at; to point at.} \textcolor{PineGreen}{\selectlanguage{french}Viser; pointer, montrer du doigt.}  ¶ \textcolor{darkblue}{\textbf{\ipa{tʰi˧-mæ˩-ze˩, | qʰæ˧-bi˥-ze˩.}}} \zh{瞄准了,要开枪了。} \textcolor{Sepia}{\selectlanguage{english}[(S)he] has aimed; [(s)he] will now shoot.} \textcolor{PineGreen}{\selectlanguage{french}(Il) a visé, (il) va tirer.}  
 ¶ \textcolor{darkblue}{\textbf{\ipa{lo˧ɲi˥ mæ˩}}} \zh{用手指出} \textcolor{Sepia}{\selectlanguage{english}to point at with the finger} \textcolor{PineGreen}{\selectlanguage{french}montrer du doigt}  
 ¶ \textcolor{darkblue}{\textbf{\ipa{tso˧\textasciitilde{}tso˧ mæ˥}}} \zh{指东西} \textcolor{Sepia}{\selectlanguage{english}to point at things} \textcolor{PineGreen}{\selectlanguage{french}pointer des choses du doigt}  

\lhead{\firstmark}
\rhead{\botmark}

\subsection{\hspace{-0.5cm} {\Large \textcolor{darkblue}{\textbf{\ipa{mæ˩\textsubscript{a}}}} \textsubscript{1}}\hspace{0.5cm}[\kern2pt{\textcolor{darkblue}{\textbf{\ipa{mæ˩˥}}}}\kern2pt]} \hypertarget{m\{\string_Ba1}{}
\markboth{\textcolor{darkblue}{\textbf{\ipa{mæ˩\textsubscript{a}}}} \textsubscript{1}}{}
\textcolor{teal}{\zh{量词}} \hspace{4pt} \zh{声调类:} L\textsubscript{a}.
\zh{量词:钱(一元)。} \textcolor{Sepia}{\selectlanguage{english}Monetary unit: yuan.} \textcolor{PineGreen}{\selectlanguage{french}Unité monétaire: un yuan.}  ¶ \textcolor{darkblue}{\textbf{\ipa{ʈʂʰɯ˧-mæ˥}}} \zh{\mytextsc{指示代词} \string_} \textcolor{Sepia}{\selectlanguage{english}\mytextsc{dem} \string_ (tone: H\# / H\$)} \textcolor{PineGreen}{\selectlanguage{french}\mytextsc{dem} \string_ (tone: H\# / H\$)}  

\lhead{\firstmark}
\rhead{\botmark}

\subsection{\hspace{-0.5cm} {\Large \textcolor{darkblue}{\textbf{\ipa{mæ˩\textsubscript{a}}}} \textsubscript{2}}\hspace{0.5cm}[\kern2pt{\textcolor{darkblue}{\textbf{\ipa{mæ˩˥}}}}\kern2pt]} \hypertarget{m\{\string_Ba2}{}
\markboth{\textcolor{darkblue}{\textbf{\ipa{mæ˩\textsubscript{a}}}} \textsubscript{2}}{}
\textcolor{teal}{\zh{量词}} \hspace{4pt} \zh{声调类:} L\textsubscript{a}.
\zh{万(数词充当量词)。} \textcolor{Sepia}{\selectlanguage{english}10,000.} \textcolor{PineGreen}{\selectlanguage{french}10.000.}  \zh{【借词】} \zh{元}, MC *mjonH (Baxter 2000)
 ¶ \textcolor{darkblue}{\textbf{\ipa{ɖɯ˧-mæ˩}}} \zh{一万} \textcolor{Sepia}{\selectlanguage{english}10,000} \textcolor{PineGreen}{\selectlanguage{french}10.000}  
 ¶ \textcolor{darkblue}{\textbf{\ipa{tsʰe˩-tv̩˩ mæ˥}}} \zh{十千万,等于一亿} \textcolor{Sepia}{\selectlanguage{english}ten thousand times 10,000, i.e. one hundred million} \textcolor{PineGreen}{\selectlanguage{french}dix mille fois 10.000, soit cent millions}  
 ¶ \textcolor{darkblue}{\textbf{\ipa{ɖɯ˧-ɕi˧ mæ˩}}} \zh{一百万} \textcolor{Sepia}{\selectlanguage{english}one hundred times 10,000, i.e. one million} \textcolor{PineGreen}{\selectlanguage{french}cent fois 10.000, soit un million}  

\lhead{\firstmark}
\rhead{\botmark}

\subsection{\hspace{-0.5cm} {\Large \textcolor{darkblue}{\textbf{\ipa{mæ˩ɖʐo˥}}}}\hspace{0.5cm}[\kern2pt{\textcolor{darkblue}{\textbf{\ipa{mæ˩ɖʐo˩˥}}}}\kern2pt]} \hypertarget{m\{\string_Bd`z`o\string_T1}{}
\markboth{\textcolor{darkblue}{\textbf{\ipa{mæ˩ɖʐo˥}}}}{}
\textcolor{teal}{\zh{名词}} \hspace{4pt} \zh{声调类:} LH.
\zh{鞭子。} \textcolor{Sepia}{\selectlanguage{english}Whip.} \textcolor{PineGreen}{\selectlanguage{french}Fouet.}  ¶ \textcolor{darkblue}{\textbf{\ipa{ʐwæ˧-mæ˥ɖʐo˩}}} \zh{马鞭} \textcolor{Sepia}{\selectlanguage{english}horse whip} \textcolor{PineGreen}{\selectlanguage{french}fouet de cheval}  
 \zh{量词}: \textcolor{darkblue}{\textbf{\ipa{kʰɯ˩}}} 
\lhead{\firstmark}
\rhead{\botmark}

\subsection{\hspace{-0.5cm} {\Large \textcolor{darkblue}{\textbf{\ipa{mæ˩ko˥}}}}\hspace{0.5cm}[\kern2pt{\textcolor{darkblue}{\textbf{\ipa{mæ˩ko˥}}}}\kern2pt]} \hypertarget{m\{\string_Bko\string_T1}{}
\markboth{\textcolor{darkblue}{\textbf{\ipa{mæ˩ko˥}}}}{}
\textcolor{teal}{\zh{名词}} \hspace{4pt} \zh{声调类:} LH.
\zh{挽具,后鞧。} \textcolor{Sepia}{\selectlanguage{english}Harness.} \textcolor{PineGreen}{\selectlanguage{french}Harnais.}  ¶ \textcolor{darkblue}{\textbf{\ipa{ʐwæ˧-mæ˥ko˩}}} \zh{马后鞧} \textcolor{Sepia}{\selectlanguage{english}horse harness} \textcolor{PineGreen}{\selectlanguage{french}harnais de cheval}  
 \zh{量词}: \textcolor{darkblue}{\textbf{\ipa{ɭɯ˧}}} 
\lhead{\firstmark}
\rhead{\botmark}

\subsection{\hspace{-0.5cm} {\Large \textcolor{darkblue}{\textbf{\ipa{mæ˧˥}}}}\hspace{0.5cm}[\kern2pt{\textcolor{darkblue}{\textbf{\ipa{mæ˧˥}}}}\kern2pt]} \hypertarget{m\{\string_M\string_T1}{}
\markboth{\textcolor{darkblue}{\textbf{\ipa{mæ˧˥}}}}{}
\textcolor{teal}{\zh{动词}} \hspace{4pt} \zh{声调类:} MH.
\ding{202} \zh{闭(嘴)。} \textcolor{Sepia}{\selectlanguage{english}To close (the mouth).} \textcolor{PineGreen}{\selectlanguage{french}Fermer (la bouche).}  ¶ \textcolor{darkblue}{\textbf{\ipa{ɲi˧to˧ | tʰi˧-mæ˧˥}}} \zh{闭嘴} \textcolor{Sepia}{\selectlanguage{english}to close the mouth} \textcolor{PineGreen}{\selectlanguage{french}fermer la bouche}  
 ¶ \textcolor{darkblue}{\textbf{\ipa{mæ˩\textasciitilde{}mæ˧˥}}} \zh{\mytextsc{重叠}} \textcolor{Sepia}{\selectlanguage{english}\mytextsc{red}} \textcolor{PineGreen}{\selectlanguage{french}\mytextsc{red}}  
 ¶ \textcolor{darkblue}{\textbf{\ipa{ɲi˧to˧ | tʰi˧-mæ˩\textasciitilde{}mæ˩}}} \zh{闭嘴} \textcolor{Sepia}{\selectlanguage{english}to close the mouth} \textcolor{PineGreen}{\selectlanguage{french}fermer la bouche}  
\ding{203} \zh{抿(嘴巴)。} \textcolor{Sepia}{\selectlanguage{english}To purse (one's lips).} \textcolor{PineGreen}{\selectlanguage{french}Pincer (les lèvres).} 
\lhead{\firstmark}
\rhead{\botmark}

\subsection{\hspace{-0.5cm} {\Large \textcolor{darkblue}{\textbf{\ipa{mɤ˧‑}}}}\hspace{0.5cm}[\kern2pt{\textcolor{darkblue}{\textbf{\ipa{mɤ˥}}}}\kern2pt]} \hypertarget{m7\string_M‑1}{}
\markboth{\textcolor{darkblue}{\textbf{\ipa{mɤ˧‑}}}}{}
\textcolor{teal}{\zh{前缀}} \hspace{4pt} \zh{声调类:} M.
\zh{否定:不,没。} \textcolor{Sepia}{\selectlanguage{english}Negation.} \textcolor{PineGreen}{\selectlanguage{french}Negation.} 
\lhead{\firstmark}
\rhead{\botmark}

\subsection{\hspace{-0.5cm} {\Large \textcolor{darkblue}{\textbf{\ipa{mɤ˧ʈʂʰɤ˧}}}}\hspace{0.5cm}[\kern2pt{\textcolor{darkblue}{\textbf{\ipa{mɤ˧ʈʂʰɤ˧}}}}\kern2pt]} \hypertarget{m7\string_Mt`s`\string_h7\string_M1}{}
\markboth{\textcolor{darkblue}{\textbf{\ipa{mɤ˧ʈʂʰɤ˧}}}}{}
\textcolor{teal}{\zh{名词}} \hspace{4pt} \zh{声调类:} M.
\zh{马车(汉语借词)。} \textcolor{Sepia}{\selectlanguage{english}Cart.} \textcolor{PineGreen}{\selectlanguage{french}Charrette.}  \zh{【借词】} \zh{马车}

\lhead{\firstmark}
\rhead{\botmark}

\subsection{\hspace{-0.5cm} {\Large \textcolor{darkblue}{\textbf{\ipa{mɤ˩}}}}\hspace{0.5cm}[\kern2pt{\textcolor{darkblue}{\textbf{\ipa{mɤ˥}}}}\kern2pt]} \hypertarget{m7\string_B1}{}
\markboth{\textcolor{darkblue}{\textbf{\ipa{mɤ˩}}}}{}
\textcolor{teal}{\zh{名词}} \hspace{4pt} \zh{声调类:} L.
\zh{动物油。} \textcolor{Sepia}{\selectlanguage{english}Animal fat.} \textcolor{PineGreen}{\selectlanguage{french}Huile animale, graisse.}  ¶ \textcolor{darkblue}{\textbf{\ipa{njɤ˧ | mɤ˩ mɤ˩ dzɯ˩˥!}}} \zh{我不吃猪油!(这是调查者的特点之一)} \textcolor{Sepia}{\selectlanguage{english}I don't eat animal fat! (One of the investigator's peculiarities)} \textcolor{PineGreen}{\selectlanguage{french}Je ne mange pas de graisse/de saindoux! (C'est là l'une des particularités de l'enquêteur)}  

\lhead{\firstmark}
\rhead{\botmark}

\subsection{\hspace{-0.5cm} {\Large \textcolor{darkblue}{\textbf{\ipa{mɤ˩\textsubscript{a}}}}}\hspace{0.5cm}[\kern2pt{\textcolor{darkblue}{\textbf{\ipa{mɤ˩˥}}}}\kern2pt]} \hypertarget{m7\string_Ba1}{}
\markboth{\textcolor{darkblue}{\textbf{\ipa{mɤ˩\textsubscript{a}}}}}{}
\textcolor{teal}{\zh{量词}} \hspace{4pt} \zh{声调类:} L\textsubscript{a}.
\zh{量词:一些、一点。} \textcolor{Sepia}{\selectlanguage{english}A few, a little.} \textcolor{PineGreen}{\selectlanguage{french}Classificateur des petites quantités: quelques-uns, quelque peu de, un peu de.}  ¶ \textcolor{darkblue}{\textbf{\ipa{ɕi˧ɭɯ˧-ɻæ˩ | ɖɯ˧-mɤ˩}}} \zh{一些稻谷种子} \textcolor{Sepia}{\selectlanguage{english}a few seeds of rice} \textcolor{PineGreen}{\selectlanguage{french}quelques graines de riz}  
 ¶ \textcolor{darkblue}{\textbf{\ipa{ɻæ˩˥ | ɖɯ˧-mɤ˩}}} \zh{一些种子} \textcolor{Sepia}{\selectlanguage{english}a few seeds} \textcolor{PineGreen}{\selectlanguage{french}quelques graines}  
 ¶ \textcolor{darkblue}{\textbf{\ipa{tsɑ˧bɤ˧ | ɖɯ˧-mɤ˩, | tsɑ˧bɤ˧ | ɲi˧-mɤ˩}}} \zh{一小捧面粉、两小捧面粉……} \textcolor{Sepia}{\selectlanguage{english}a small quantity of flour; two small quantities of flour; etc} \textcolor{PineGreen}{\selectlanguage{french}un peu de farine; deux poignées/petites quantités de farine; etc.}  
 ¶ \textcolor{darkblue}{\textbf{\ipa{ʈʂʰɯ˧-mɤ˥}}} \zh{\mytextsc{指示代词} \string_} \textcolor{Sepia}{\selectlanguage{english}\mytextsc{dem} \string_ (tone: H\# / H\$)} \textcolor{PineGreen}{\selectlanguage{french}\mytextsc{dem} \string_ (tone: H\# / H\$)}  

\lhead{\firstmark}
\rhead{\botmark}

\subsection{\hspace{-0.5cm} {\Large \textcolor{darkblue}{\textbf{\ipa{mɤ˩\textsubscript{b}}}}}\hspace{0.5cm}[\kern2pt{\textcolor{darkblue}{\textbf{\ipa{mɤ˩˥}}}}\kern2pt]} \hypertarget{m7\string_Bb1}{}
\markboth{\textcolor{darkblue}{\textbf{\ipa{mɤ˩\textsubscript{b}}}}}{}
\textcolor{teal}{\zh{动词}} \hspace{4pt} \zh{声调类:} L\textsubscript{b}.
\zh{将粉状的食品放在嘴里(如:干糌粑)。} \textcolor{Sepia}{\selectlanguage{english}To eat food in powder form, typically tsamba.} \textcolor{PineGreen}{\selectlanguage{french}Prendre dans la bouche un aliment en poudre.}  ¶ \textcolor{darkblue}{\textbf{\ipa{tsɑ˧bɤ˧ mɤ˩}}} \zh{吃干糌粑} \textcolor{Sepia}{\selectlanguage{english}to eat dry tsamba: one takes a spoonful, pours it into the mouth, and lets it get impregnated with saliva} \textcolor{PineGreen}{\selectlanguage{french}manger du tsamba sec: on en prend une cuillère qu'on renverse dans sa bouche, et on laisse la farine s'imprégner de salive}  
 ¶ \textcolor{darkblue}{\textbf{\ipa{tsɑ˧bɤ˧ | ɖɯ˧-mɤ˧\textasciitilde{}mɤ˩-ɻ̍˩}}} \zh{品干糌粑、慢慢享受一点干糌粑} \textcolor{Sepia}{\selectlanguage{english}to eat a little dry tsamba, to take the time to appreciate some dry tsamba} \textcolor{PineGreen}{\selectlanguage{french}savourer un peu de tsamba}  

\lhead{\firstmark}
\rhead{\botmark}

\subsection{\hspace{-0.5cm} {\Large \textcolor{darkblue}{\textbf{\ipa{mɤ˩ɬi˩}}}}\hspace{0.5cm}[\kern2pt{\textcolor{darkblue}{\textbf{\ipa{mɤ˩ɬi˩˥}}}}\kern2pt]} \hypertarget{m7\string_BKi\string_B1}{}
\markboth{\textcolor{darkblue}{\textbf{\ipa{mɤ˩ɬi˩}}}}{}
\textcolor{teal}{\zh{名词}} \hspace{4pt} \zh{声调类:} L.
\zh{酥油茶。} \textcolor{Sepia}{\selectlanguage{english}Butter tea.} \textcolor{PineGreen}{\selectlanguage{french}Thé au beurre.}  \zh{量词}: \textcolor{darkblue}{\textbf{\ipa{qʰwɤ˧˥}}} 
\lhead{\firstmark}
\rhead{\botmark}

\subsection{\hspace{-0.5cm} {\Large \textcolor{darkblue}{\textbf{\ipa{mɤ˩mv̩˩}}}}\hspace{0.5cm}[\kern2pt{\textcolor{darkblue}{\textbf{\ipa{mɤ˩mv̩˩˥}}}}\kern2pt]} \hypertarget{m7\string_Bmv\string_=\string_B1}{}
\markboth{\textcolor{darkblue}{\textbf{\ipa{mɤ˩mv̩˩}}}}{}
\textcolor{teal}{\zh{名词}} \hspace{4pt} \zh{声调类:} L.
\zh{烛台。} \textcolor{Sepia}{\selectlanguage{english}Candle holder.} \textcolor{PineGreen}{\selectlanguage{french}Porte-bougie: objet en cuivre dans lequel on verse de la paraffine fondue, ou de la graisse, et dans lequel on place une mèche; est utilisé dans les rituels.}  \zh{量词}: \textcolor{darkblue}{\textbf{\ipa{qʰwɤ˧˥}}} 
\lhead{\firstmark}
\rhead{\botmark}

\subsection{\hspace{-0.5cm} {\Large \textcolor{darkblue}{\textbf{\ipa{mɤ˩tʰɑ˧}}}}\hspace{0.5cm}[\kern2pt{\textcolor{darkblue}{\textbf{\ipa{mɤ˩tʰɑ˥}}}}\kern2pt]} \hypertarget{m7\string_Bt\string_hA\string_M1}{}
\markboth{\textcolor{darkblue}{\textbf{\ipa{mɤ˩tʰɑ˧}}}}{}
\textcolor{teal}{\zh{名词}} \hspace{4pt} \zh{声调类:} LM.
\zh{麻糖(汉语借词)。} \textcolor{Sepia}{\selectlanguage{english}Sesame candy.} \textcolor{PineGreen}{\selectlanguage{french}Confiserie au sésame.}  \zh{【借词】} \zh{麻糖}

\lhead{\firstmark}
\rhead{\botmark}

\subsection{\hspace{-0.5cm} {\Large \textcolor{darkblue}{\textbf{\ipa{mi˧}}}}\hspace{0.5cm}[\kern2pt{\textcolor{darkblue}{\textbf{\ipa{mi˩˥}}}}\kern2pt]} \hypertarget{mi\string_M1}{}
\markboth{\textcolor{darkblue}{\textbf{\ipa{mi˧}}}}{}
\textcolor{teal}{\zh{名词}} \hspace{4pt} \zh{声调类:} M.
\zh{伤口。} \textcolor{Sepia}{\selectlanguage{english}Wound.} \textcolor{PineGreen}{\selectlanguage{french}Blessure, plaie.}  \zh{量词}: \textcolor{darkblue}{\textbf{\ipa{kʰwɤ˥}}} 
\lhead{\firstmark}
\rhead{\botmark}

\subsection{\hspace{-0.5cm} {\Large \textcolor{darkblue}{\textbf{\ipa{mi˧kʰwɤ\#˥}}}}\hspace{0.5cm}[\kern2pt{\textcolor{darkblue}{\textbf{\ipa{mi˩kʰwɤ˥}}}}\kern2pt]} \hypertarget{mi\string_Mk\string_hw7\#\string_T1}{}
\markboth{\textcolor{darkblue}{\textbf{\ipa{mi˧kʰwɤ\#˥}}}}{}
\textcolor{teal}{\zh{名词}} \hspace{4pt} \zh{声调类:} \#H.
\ding{202} \zh{伤口。} \textcolor{Sepia}{\selectlanguage{english}Wound.} \textcolor{PineGreen}{\selectlanguage{french}Blessure, plaie.}  \zh{量词}: \textcolor{darkblue}{\textbf{\ipa{kʰwɤ˥}}} \ding{203} \zh{疮。} \textcolor{Sepia}{\selectlanguage{english}Ulcer.} \textcolor{PineGreen}{\selectlanguage{french}Ulcère.} 
\lhead{\firstmark}
\rhead{\botmark}

\subsection{\hspace{-0.5cm} {\Large \textcolor{darkblue}{\textbf{\ipa{mi˧ɬo\#˥}}}}\hspace{0.5cm}[\kern2pt{\textcolor{darkblue}{\textbf{\ipa{mi˩ɬo˥}}}}\kern2pt]} \hypertarget{mi\string_MKo\#\string_T1}{}
\markboth{\textcolor{darkblue}{\textbf{\ipa{mi˧ɬo\#˥}}}}{}
\textcolor{teal}{\zh{名词}} \hspace{4pt} \zh{声调类:} \#H.
\zh{祈祷。} \textcolor{Sepia}{\selectlanguage{english}Prayer.} \textcolor{PineGreen}{\selectlanguage{french}Prière.}  ¶ \textcolor{darkblue}{\textbf{\ipa{mi˧ɬo˧ lɑ˩}}} \zh{祈祷} \textcolor{Sepia}{\selectlanguage{english}to pray} \textcolor{PineGreen}{\selectlanguage{french}prier}  
 \zh{量词}: \textcolor{darkblue}{\textbf{\ipa{kʰwɤ˥}}} \zh{~【参考】~} \hyperlink{}{\textcolor{darkblue}{\textbf{\ipa{ɬo˧˥}}}} 
\lhead{\firstmark}
\rhead{\botmark}

\subsection{\hspace{-0.5cm} {\Large \textcolor{darkblue}{\textbf{\ipa{mi˧mi˧}}}}\hspace{0.5cm}[\kern2pt{\textcolor{darkblue}{\textbf{\ipa{mi˧mi˧}}}}\kern2pt]} \hypertarget{mi\string_Mmi\string_M1}{}
\markboth{\textcolor{darkblue}{\textbf{\ipa{mi˧mi˧}}}}{}
\textcolor{teal}{\zh{名词}} \hspace{4pt} \zh{声调类:} M.
\zh{核,仁。} \textcolor{Sepia}{\selectlanguage{english}Kernel (of a seed).} \textcolor{PineGreen}{\selectlanguage{french}Amande (d'un noyau).}  \zh{【借词】} (dialectal) \zh{米米}

\lhead{\firstmark}
\rhead{\botmark}

\subsection{\hspace{-0.5cm} {\Large \textcolor{darkblue}{\textbf{\ipa{mi˧pɤ\#˥}}}}\hspace{0.5cm}[\kern2pt{\textcolor{darkblue}{\textbf{\ipa{mi˩pɤ˩˥}}}}\kern2pt]} \hypertarget{mi\string_Mp7\#\string_T1}{}
\markboth{\textcolor{darkblue}{\textbf{\ipa{mi˧pɤ\#˥}}}}{}
\textcolor{teal}{\zh{名词}} \hspace{4pt} \zh{声调类:} \#H.
\zh{疤。} \textcolor{Sepia}{\selectlanguage{english}Scar.} \textcolor{PineGreen}{\selectlanguage{french}Cicatrice.}  \zh{量词}: \textcolor{darkblue}{\textbf{\ipa{kʰwɤ˥}}} 
\lhead{\firstmark}
\rhead{\botmark}

\subsection{\hspace{-0.5cm} {\Large \textcolor{darkblue}{\textbf{\ipa{mi˧tʰv̩\#˥}}}}\hspace{0.5cm}[\kern2pt{\textcolor{darkblue}{\textbf{\ipa{mi˩tʰv̩˩˥}}}}\kern2pt]} \hypertarget{mi\string_Mt\string_hv\string_=\#\string_T1}{}
\markboth{\textcolor{darkblue}{\textbf{\ipa{mi˧tʰv̩\#˥}}}}{}
\textcolor{teal}{\zh{名词}} \hspace{4pt} \zh{声调类:} \#H.
\zh{拐棍。} \textcolor{Sepia}{\selectlanguage{english}Walking-stick.} \textcolor{PineGreen}{\selectlanguage{french}Bâton, canne pour marcher.}  \zh{量词}: \textcolor{darkblue}{\textbf{\ipa{kɤ˧˥}}} 
\lhead{\firstmark}
\rhead{\botmark}

\subsection{\hspace{-0.5cm} {\Large \textcolor{darkblue}{\textbf{\ipa{mi˩\textsubscript{a}}}}}\hspace{0.5cm}[\kern2pt{\textcolor{darkblue}{\textbf{\ipa{mi˧˥}}}}\kern2pt]} \hypertarget{mi\string_Ba1}{}
\markboth{\textcolor{darkblue}{\textbf{\ipa{mi˩\textsubscript{a}}}}}{}
\textcolor{teal}{\zh{动词}} \hspace{4pt} \zh{声调类:} L\textsubscript{a}.
\zh{请求、要,讨饭。} \textcolor{Sepia}{\selectlanguage{english}To ask for.} \textcolor{PineGreen}{\selectlanguage{french}Demander, quémander.}  ¶ \textcolor{darkblue}{\textbf{\ipa{hɑ˧ mi˥}}} \zh{讨饭} \textcolor{Sepia}{\selectlanguage{english}to beg (literally: 'to ask for food')} \textcolor{PineGreen}{\selectlanguage{french}mendier (littéralement: 'demander à manger')}  
 ¶ \textcolor{darkblue}{\textbf{\ipa{hɑ˧ | ɖɯ˧-mi˧\textasciitilde{}mi˥-ɻ̍˩}}} \zh{讨点饭} \textcolor{Sepia}{\selectlanguage{english}to beg a little, to ask around for some food} \textcolor{PineGreen}{\selectlanguage{french}mendier un peu}  

\lhead{\firstmark}
\rhead{\botmark}

\subsection{\hspace{-0.5cm} {\Large \textcolor{darkblue}{\textbf{\ipa{mi˩\textsubscript{b}}}}}\hspace{0.5cm}[\kern2pt{\textcolor{darkblue}{\textbf{\ipa{mi˩˥}}}}\kern2pt]} \hypertarget{mi\string_Bb1}{}
\markboth{\textcolor{darkblue}{\textbf{\ipa{mi˩\textsubscript{b}}}}}{}
\textcolor{teal}{\zh{量词}} \hspace{4pt} \zh{声调类:} L\textsubscript{b}.
\zh{量词:小动物(一只)。} \textcolor{Sepia}{\selectlanguage{english}Classifier for small animals.} \textcolor{PineGreen}{\selectlanguage{french}Classificateur des petits animaux (poules…).}  ¶ \textcolor{darkblue}{\textbf{\ipa{ʈʂʰɯ˧-mi˧˥}}} \zh{这只} \textcolor{Sepia}{\selectlanguage{english}this animal} \textcolor{PineGreen}{\selectlanguage{french}cet animal}  

\lhead{\firstmark}
\rhead{\botmark}

\subsection{\hspace{-0.5cm} {\Large \textcolor{darkblue}{\textbf{\ipa{mi˩hwɑ˧}}}}\hspace{0.5cm}[\kern2pt{\textcolor{darkblue}{\textbf{\ipa{mi˩hwɑ˩˥}}}}\kern2pt]} \hypertarget{mi\string_BhwA\string_M1}{}
\markboth{\textcolor{darkblue}{\textbf{\ipa{mi˩hwɑ˧}}}}{}
\textcolor{teal}{\zh{名词}} \hspace{4pt} \zh{声调类:} LM.
\zh{棉花(汉语借词)。} \textcolor{Sepia}{\selectlanguage{english}Cotton.} \textcolor{PineGreen}{\selectlanguage{french}Coton.}  \zh{【借词】}\zh{棉花}
 ¶ \textcolor{darkblue}{\textbf{\ipa{mi˩hwɑ˧-bɑ˩lɑ˩}}} \zh{棉布衣服} \textcolor{Sepia}{\selectlanguage{english}cotton clothes} \textcolor{PineGreen}{\selectlanguage{french}vêtement de coton}  

\lhead{\firstmark}
\rhead{\botmark}

\subsection{\hspace{-0.5cm} {\Large \textcolor{darkblue}{\textbf{\ipa{mi˩ɬi˩}}}}\hspace{0.5cm}[\kern2pt{\textcolor{darkblue}{\textbf{\ipa{mi˧ɬi˧}}}}\kern2pt]} \hypertarget{mi\string_BKi\string_B1}{}
\markboth{\textcolor{darkblue}{\textbf{\ipa{mi˩ɬi˩}}}}{}
\textcolor{teal}{\zh{名词}} \hspace{4pt} \zh{声调类:} L.
\zh{大竹子。} \textcolor{Sepia}{\selectlanguage{english}Large bamboo.} \textcolor{PineGreen}{\selectlanguage{french}Grand bambou.}  ¶ \textcolor{darkblue}{\textbf{\ipa{mi˩ɬi˩-bæ˩ʈʂo˥}}} \zh{竹扫帚} \textcolor{Sepia}{\selectlanguage{english}bamboo broom} \textcolor{PineGreen}{\selectlanguage{french}balai en petites tiges de bambou}  
 ¶ \textcolor{darkblue}{\textbf{\ipa{mi˩ɬi˩-ʈʂæ˥do˩}}} \zh{竹桶,用来背水} \textcolor{Sepia}{\selectlanguage{english}bamboo bucket to carry water (on one's back)} \textcolor{PineGreen}{\selectlanguage{french}seau en bambou pour porter de l'eau (sur le dos)}  
 \zh{量词}: \textcolor{darkblue}{\textbf{\ipa{dzi˩}}} 
\lhead{\firstmark}
\rhead{\botmark}

\subsection{\hspace{-0.5cm} {\Large \textcolor{darkblue}{\textbf{\ipa{mi˩ɬi˩-ʁo˩bv̩˥}}}}\hspace{0.5cm}[\kern2pt{\textcolor{darkblue}{\textbf{\ipa{xxxx non-correspondance entre le nombre de morphèmes et le nombre de tons de morphèmes}}}}\kern2pt]} \hypertarget{mi\string_BKi\string_B-Ro\string_Bbv\string_=\string_T1}{}
\markboth{\textcolor{darkblue}{\textbf{\ipa{mi˩ɬi˩-ʁo˩bv̩˥}}}}{}
\textcolor{teal}{\zh{名词}} \hspace{4pt} \zh{声调类:} L+H\#.
\zh{竹笋。} \textcolor{Sepia}{\selectlanguage{english}Bamboo shoot.} \textcolor{PineGreen}{\selectlanguage{french}Pousse de bambou.}  \zh{量词}: \textcolor{darkblue}{\textbf{\ipa{kɤ˧˥}}} 
\lhead{\firstmark}
\rhead{\botmark}

\subsection{\hspace{-0.5cm} {\Large \textcolor{darkblue}{\textbf{\ipa{mi˩mo˩}}}}\hspace{0.5cm}[\kern2pt{\textcolor{darkblue}{\textbf{\ipa{mi˧mo˧}}}}\kern2pt]} \hypertarget{mi\string_Bmo\string_B1}{}
\markboth{\textcolor{darkblue}{\textbf{\ipa{mi˩mo˩}}}}{}
\textcolor{teal}{\zh{名词}} \hspace{4pt} \zh{声调类:} L.
\zh{小筛子。} \textcolor{Sepia}{\selectlanguage{english}Small sifter.} \textcolor{PineGreen}{\selectlanguage{french}Petit crible.}  \zh{量词}: \textcolor{darkblue}{\textbf{\ipa{nɑ˧}}} 
\lhead{\firstmark}
\rhead{\botmark}

\subsection{\hspace{-0.5cm} {\Large \textcolor{darkblue}{\textbf{\ipa{mi˩pʰv̩˩}}}}\hspace{0.5cm}[\kern2pt{\textcolor{darkblue}{\textbf{\ipa{mi˧pʰv̩˧}}}}\kern2pt]} \hypertarget{mi\string_Bp\string_hv\string_=\string_B1}{}
\markboth{\textcolor{darkblue}{\textbf{\ipa{mi˩pʰv̩˩}}}}{}
\textcolor{teal}{\zh{名词}} \hspace{4pt} \zh{声调类:} L.
\zh{荨麻。} \textcolor{Sepia}{\selectlanguage{english}Nettle.} \textcolor{PineGreen}{\selectlanguage{french}Ortie.}  \zh{量词}: \textcolor{darkblue}{\textbf{\ipa{dzi˩}}} 
\lhead{\firstmark}
\rhead{\botmark}

\subsection{\hspace{-0.5cm} {\Large \textcolor{darkblue}{\textbf{\ipa{mi˩zɯ˩}}}}\hspace{0.5cm}[\kern2pt{\textcolor{darkblue}{\textbf{\ipa{mi˧zɯ˧}}}}\kern2pt]} \hypertarget{mi\string_BzM\string_B1}{}
\markboth{\textcolor{darkblue}{\textbf{\ipa{mi˩zɯ˩}}}}{}
\textcolor{teal}{\zh{名词}} \hspace{4pt} \zh{声调类:} L.
\zh{女人。主屋的第二个柱子(代表女性的那个柱子)也是用这个名字。} \textcolor{Sepia}{\selectlanguage{english}Woman; also the name of the second pillar in the main room (the feminine pillar).} \textcolor{PineGreen}{\selectlanguage{french}Femme; aussi: nom du deuxième pilier de la maison (le pilier féminin).}  \zh{量词}: \textcolor{darkblue}{\textbf{\ipa{v̩˧}}} 
\lhead{\firstmark}
\rhead{\botmark}

\subsection{\hspace{-0.5cm} {\Large \textcolor{darkblue}{\textbf{\ipa{mi˧˥}}}}\hspace{0.5cm}[\kern2pt{\textcolor{darkblue}{\textbf{\ipa{mi˥}}}}\kern2pt]} \hypertarget{mi\string_M\string_T1}{}
\markboth{\textcolor{darkblue}{\textbf{\ipa{mi˧˥}}}}{}
\textcolor{teal}{\zh{动词}} \hspace{4pt} \zh{声调类:} MH.
\zh{推、拥挤。} \textcolor{Sepia}{\selectlanguage{english}To push.} \textcolor{PineGreen}{\selectlanguage{french}Pousser.}  ¶ \textcolor{darkblue}{\textbf{\ipa{le˧-mi˧-ze˥}}} \zh{推开了} \textcolor{Sepia}{\selectlanguage{english}\mytextsc{accomp} \string_ \mytextsc{pfv}} \textcolor{PineGreen}{\selectlanguage{french}\mytextsc{accomp} \string_ \mytextsc{pfv}}  
 ¶ \textcolor{darkblue}{\textbf{\ipa{le˧-mi˧˥}}} \zh{推} \textcolor{Sepia}{\selectlanguage{english}\mytextsc{accomp}} \textcolor{PineGreen}{\selectlanguage{french}\mytextsc{accomp}}  
 ¶ \textcolor{darkblue}{\textbf{\ipa{tʰi˧-mi˧˥}}} \zh{推} \textcolor{Sepia}{\selectlanguage{english}\mytextsc{dur}} \textcolor{PineGreen}{\selectlanguage{french}\mytextsc{dur}}  
 ¶ \textcolor{darkblue}{\textbf{\ipa{tso˧\textasciitilde{}tso˧ mi˩}}} \zh{推开一个东西} \textcolor{Sepia}{\selectlanguage{english}to push something} \textcolor{PineGreen}{\selectlanguage{french}pousser quelque chose}  
 ¶ \textcolor{darkblue}{\textbf{\ipa{mi˩\textasciitilde{}mi˧˥}}} \zh{\mytextsc{重叠:推、拥挤}} \textcolor{Sepia}{\selectlanguage{english}\mytextsc{red}: to push and squeeze} \textcolor{PineGreen}{\selectlanguage{french}\mytextsc{red}}  
 ¶ \textcolor{darkblue}{\textbf{\ipa{mi˩\textasciitilde{}mi˧-ɻ̍˥}}} \zh{\mytextsc{重叠:推、拥挤}} \textcolor{Sepia}{\selectlanguage{english}\mytextsc{red} \mytextsc{inceptive}} \textcolor{PineGreen}{\selectlanguage{french}\mytextsc{red} \mytextsc{inchoatif}}  

\lhead{\firstmark}
\rhead{\botmark}

\subsection{\hspace{-0.5cm} {\Large \textcolor{darkblue}{\textbf{\ipa{-mi˩˧}}}}\hspace{0.5cm}[\kern2pt{\textcolor{darkblue}{\textbf{\ipa{mi˩˥}}}}\kern2pt]} \hypertarget{-mi\string_B\string_M1}{}
\markboth{\textcolor{darkblue}{\textbf{\ipa{-mi˩˧}}}}{}
\textcolor{teal}{\zh{后缀}} \hspace{4pt} \zh{声调类:} LM.
\ding{202} \zh{阴性后缀。} \textcolor{Sepia}{\selectlanguage{english}Feminine suffix.} \textcolor{PineGreen}{\selectlanguage{french}Suffixe féminin.}  \zh{量词}: \textcolor{darkblue}{\textbf{\ipa{v̩˧}}} \ding{203} \zh{指大词。} \textcolor{Sepia}{\selectlanguage{english}Augmentative suffix.} \textcolor{PineGreen}{\selectlanguage{french}Suffixe augmentatif.}  \zh{量词}: \textcolor{darkblue}{\textbf{\ipa{v̩˧}}} 
\lhead{\firstmark}
\rhead{\botmark}

\subsection{\hspace{-0.5cm} {\Large \textcolor{darkblue}{\textbf{\ipa{mi˩˧}}}}\hspace{0.5cm}[\kern2pt{\textcolor{darkblue}{\textbf{\ipa{mi˥}}}}\kern2pt]} \hypertarget{mi\string_B\string_M1}{}
\markboth{\textcolor{darkblue}{\textbf{\ipa{mi˩˧}}}}{}
\textcolor{teal}{\zh{名词}} \hspace{4pt} \zh{声调类:} LM.
\zh{母的(动物)。} \textcolor{Sepia}{\selectlanguage{english}Female (animal).} \textcolor{PineGreen}{\selectlanguage{french}Femelle (animal femelle).}  ¶ \textcolor{darkblue}{\textbf{\ipa{ʈʂʰɯ˧, | mi˩˥! / ʈʂʰɯ˧, | mi˩ ɲi˥!}}} \zh{是母的!} \textcolor{Sepia}{\selectlanguage{english}It's a female!} \textcolor{PineGreen}{\selectlanguage{french}C'est une femelle!}  
 \zh{量词}: \textcolor{darkblue}{\textbf{\ipa{v̩˧}}} 
\lhead{\firstmark}
\rhead{\botmark}

\subsection{\hspace{-0.5cm} {\Large \textcolor{darkblue}{\textbf{\ipa{mje˧˥}}}}\hspace{0.5cm}[\kern2pt{\textcolor{darkblue}{\textbf{\ipa{mje˥}}}}\kern2pt]} \hypertarget{mje\string_M\string_T1}{}
\markboth{\textcolor{darkblue}{\textbf{\ipa{mje˧˥}}}}{}
\textcolor{teal}{\zh{名词}} \hspace{4pt} \zh{声调类:} .
\zh{面条。} \textcolor{Sepia}{\selectlanguage{english}Noodles.} \textcolor{PineGreen}{\selectlanguage{french}Nouilles, pâtes alimentaires.}  \zh{【借词】} \zh{面}
 ¶ \textcolor{darkblue}{\textbf{\ipa{mjæ˧˥ | dzɯ˧-bi˧! \textasciitilde{} mjæ˧ dzɯ˧-bi˧! \textasciitilde{} mjæ˧ dzɯ˥-bi˩!}}} \zh{吃面吧!} \textcolor{Sepia}{\selectlanguage{english}Let's eat noodles!} \textcolor{PineGreen}{\selectlanguage{french}On va manger des nouilles!}  
 ¶ \textcolor{darkblue}{\textbf{\ipa{mjæ˧˥ | ɖɯ˧-qʰwɤ˧ tɕɤ˥}}} \zh{煮一碗面} \textcolor{Sepia}{\selectlanguage{english}to boil a bowl of noodles, to cook a bowl of noodles} \textcolor{PineGreen}{\selectlanguage{french}faire cuire un bol de nouilles}  
 ¶ \textcolor{darkblue}{\textbf{\ipa{mjæ˧ hwæ˥-bi˩}}} \zh{要买面} \textcolor{Sepia}{\selectlanguage{english}(we) will buy noodles} \textcolor{PineGreen}{\selectlanguage{french}(on) va acheter des nouilles}  

\lhead{\firstmark}
\rhead{\botmark}

\subsection{\hspace{-0.5cm} {\Large \textcolor{darkblue}{\textbf{\ipa{mo˥\textsubscript{a}}}}}\hspace{0.5cm}[\kern2pt{\textcolor{darkblue}{\textbf{\ipa{mo˥}}}}\kern2pt]} \hypertarget{mo\string_Ta1}{}
\markboth{\textcolor{darkblue}{\textbf{\ipa{mo˥\textsubscript{a}}}}}{}
\textcolor{teal}{\zh{量词}} \hspace{4pt} \zh{声调类:} H\textsubscript{a}.
\zh{量词:地(一亩地)(汉语借词)。} \textcolor{Sepia}{\selectlanguage{english}One Chinese acre, amounting to one-sixth of an acre.} \textcolor{PineGreen}{\selectlanguage{french}Acre chinois: un sixième d'acre; 0,0667 hectare.}  \zh{【借词】} \zh{亩}

\lhead{\firstmark}
\rhead{\botmark}

\subsection{\hspace{-0.5cm} {\Large \textcolor{darkblue}{\textbf{\ipa{mo˧\textsubscript{a}}}}}\hspace{0.5cm}[\kern2pt{\textcolor{darkblue}{\textbf{\ipa{mo˥}}}}\kern2pt]} \hypertarget{mo\string_Ma1}{}
\markboth{\textcolor{darkblue}{\textbf{\ipa{mo˧\textsubscript{a}}}}}{}
\textcolor{teal}{\zh{量词}} \hspace{4pt} \zh{声调类:} M\textsubscript{a}.
\zh{量词:尸体。} \textcolor{Sepia}{\selectlanguage{english}Classifier for corpses.} \textcolor{PineGreen}{\selectlanguage{french}Classificateur des cadavres et tombeaux.} 
\lhead{\firstmark}
\rhead{\botmark}

\subsection{\hspace{-0.5cm} {\Large \textcolor{darkblue}{\textbf{\ipa{mo˧ɖʐv̩˥}}}}\hspace{0.5cm}[\kern2pt{\textcolor{darkblue}{\textbf{\ipa{mo˧ɖʐv̩˥}}}}\kern2pt]} \hypertarget{mo\string_Md`z`v\string_=\string_T1}{}
\markboth{\textcolor{darkblue}{\textbf{\ipa{mo˧ɖʐv̩˥}}}}{}
\textcolor{teal}{\zh{名词}} \hspace{4pt} \zh{声调类:} H\#.
\zh{羊肚菌。} \textcolor{Sepia}{\selectlanguage{english}Morel, hickory chick: an edible mushroom.} \textcolor{PineGreen}{\selectlanguage{french}Morille: champignon comestible, particulièrement apprécié pour sa texture.} \zh{当地汉语方言:}\zh{羊菌。} ¶ \textcolor{darkblue}{\textbf{\ipa{ʂɯ˧-ɬi˧mi˧, | mo˧ɖʐv̩˥!}}} \zh{七月份,是羊肚菌的季节!} \textcolor{Sepia}{\selectlanguage{english}The seventh month is the season of morels!} \textcolor{PineGreen}{\selectlanguage{french}Le septième mois, c'est la saison des morilles! (cette sorte de champignon) (Il pousse des paquets de champignons si compacts qu'on n'arrive même pas à les séparer.)}  
 ¶ \textcolor{darkblue}{\textbf{\ipa{ʂɯ˧-ɬi˧mi˧ | mo˧ɖʐv̩˥-ne˩-ʝi˩-zo˩!}}} \zh{(你们家孩子)生得像七月份的羊肚菌一样!(来形容一家有很多孩子出生,一个又一个。在永宁地区,七月份羊肚菌很多。)} \textcolor{Sepia}{\selectlanguage{english}'[They have kids] like (=as numerous as) morels in the seventh month!', i.e. they have children in great abundance. This is a humorous comment made about people who had one child after the other. The abundance of morels in the seventh month is spectacular and proverbial.} \textcolor{PineGreen}{\selectlanguage{french}“Il vous en vient comme des morilles au septième mois!” Commentaire humoristique: ce qu'on disait au sujet des gens qui avaient beaucoup d'enfants, qui avaient enfant après enfant: “Ca prolifère comme les morilles au septième mois!”}  

\lhead{\firstmark}
\rhead{\botmark}

\subsection{\hspace{-0.5cm} {\Large \textcolor{darkblue}{\textbf{\ipa{mo˧jo˩-mi˩}}}}\hspace{0.5cm}[\kern2pt{\textcolor{darkblue}{\textbf{\ipa{mo˧jo˩mi˧}}}}\kern2pt]} \hypertarget{mo\string_Mjo\string_B-mi\string_B1}{}
\markboth{\textcolor{darkblue}{\textbf{\ipa{mo˧jo˩-mi˩}}}}{}
\textcolor{teal}{\zh{名词}} \hspace{4pt} \zh{声调类:} L\#-.
\zh{猫头鹰。} \textcolor{Sepia}{\selectlanguage{english}Owl.} \textcolor{PineGreen}{\selectlanguage{french}Chouette, hibou (toutes les espèces de \textit{bubo} et \textit{strix}).}  \zh{量词}: \textcolor{darkblue}{\textbf{\ipa{mi˩}}} 
\lhead{\firstmark}
\rhead{\botmark}

\subsection{\hspace{-0.5cm} {\Large \textcolor{darkblue}{\textbf{\ipa{mo˧jo˩mi˩-pʰv̩˩}}}}\hspace{0.5cm}[\kern2pt{\textcolor{darkblue}{\textbf{\ipa{mo˧jo˧mi˩pʰv̩˧}}}}\kern2pt]} \hypertarget{mo\string_Mjo\string_Bmi\string_B-p\string_hv\string_=\string_B1}{}
\markboth{\textcolor{darkblue}{\textbf{\ipa{mo˧jo˩mi˩-pʰv̩˩}}}}{}
\textcolor{teal}{\zh{名词}} \hspace{4pt} \zh{声调类:} L\#-.
\zh{公猫头鹰。} \textcolor{Sepia}{\selectlanguage{english}Male owl.} \textcolor{PineGreen}{\selectlanguage{french}Hibou mâle.}  \zh{量词}: \textcolor{darkblue}{\textbf{\ipa{mi˩}}} 
\lhead{\firstmark}
\rhead{\botmark}

\subsection{\hspace{-0.5cm} {\Large \textcolor{darkblue}{\textbf{\ipa{mo˧jo˩mi˩-zo˩}}}}\hspace{0.5cm}[\kern2pt{\textcolor{darkblue}{\textbf{\ipa{mo˧jo˧mi˩zo˧}}}}\kern2pt]} \hypertarget{mo\string_Mjo\string_Bmi\string_B-zo\string_B1}{}
\markboth{\textcolor{darkblue}{\textbf{\ipa{mo˧jo˩mi˩-zo˩}}}}{}
\textcolor{teal}{\zh{名词}} \hspace{4pt} \zh{声调类:} L\#-.
\zh{小的猫头鹰。} \textcolor{Sepia}{\selectlanguage{english}Baby owl.} \textcolor{PineGreen}{\selectlanguage{french}Bébé hibou.}  \zh{量词}: \textcolor{darkblue}{\textbf{\ipa{ɭɯ˧}}} 
\lhead{\firstmark}
\rhead{\botmark}

\subsection{\hspace{-0.5cm} {\Large \textcolor{darkblue}{\textbf{\ipa{mo˧kɤ˩}}}}\hspace{0.5cm}[\kern2pt{\textcolor{darkblue}{\textbf{\ipa{mo˧kɤ˩}}}}\kern2pt]} \hypertarget{mo\string_Mk7\string_B1}{}
\markboth{\textcolor{darkblue}{\textbf{\ipa{mo˧kɤ˩}}}}{}
\textcolor{teal}{\zh{名词}} \hspace{4pt} \zh{声调类:} L\#.
\zh{杜鹃花、踯躅、山石榴、照山红、唐杜鹃。} \textcolor{Sepia}{\selectlanguage{english}Azalea.} \textcolor{PineGreen}{\selectlanguage{french}Azalée. Cette plante est perçue comme vénéneuse; on ne consomme pas les champignons qui poussent dans son voisinage.} \zh{当地汉语方言:}\zh{杨花木。} ¶ \textcolor{darkblue}{\textbf{\ipa{mo˧kɤ˩-bæ˩bæ˩}}} \zh{杜鹃花} \textcolor{Sepia}{\selectlanguage{english}azalea flowers} \textcolor{PineGreen}{\selectlanguage{french}fleurs d'azalée}  

\lhead{\firstmark}
\rhead{\botmark}

\subsection{\hspace{-0.5cm} {\Large \textcolor{darkblue}{\textbf{\ipa{mo˧ɬɑ˥}}}}\hspace{0.5cm}[\kern2pt{\textcolor{darkblue}{\textbf{\ipa{mo˧ɬɑ˥}}}}\kern2pt]} \hypertarget{mo\string_MKA\string_T1}{}
\markboth{\textcolor{darkblue}{\textbf{\ipa{mo˧ɬɑ˥}}}}{}
\textcolor{teal}{\zh{名词}} \hspace{4pt} \zh{声调类:} H\#.
\zh{诬蔑、坏话。} \textcolor{Sepia}{\selectlanguage{english}Slander.} \textcolor{PineGreen}{\selectlanguage{french}Médisance, calomnie.}  ¶ \textcolor{darkblue}{\textbf{\ipa{mo˧ɬɑ˥ ʐwɤ˩}}} \zh{说人的坏话} \textcolor{Sepia}{\selectlanguage{english}to slander, to speak ill of others} \textcolor{PineGreen}{\selectlanguage{french}médire de quelqu'un, calomnier quelqu'un}  

\lhead{\firstmark}
\rhead{\botmark}

\subsection{\hspace{-0.5cm} {\Large \textcolor{darkblue}{\textbf{\ipa{mo˧mo˥}}}}\hspace{0.5cm}[\kern2pt{\textcolor{darkblue}{\textbf{\ipa{mo˧mo˥}}}}\kern2pt]} \hypertarget{mo\string_Mmo\string_T1}{}
\markboth{\textcolor{darkblue}{\textbf{\ipa{mo˧mo˥}}}}{}
\textcolor{teal}{\zh{名词}} \hspace{4pt} \zh{声调类:} H\#.
\zh{馒头、包子。} \textcolor{Sepia}{\selectlanguage{english}Steamed bun.} \textcolor{PineGreen}{\selectlanguage{french}Petits pains (pouvant contenir de la farine de maïs; mais surtout farine de blé) cuits à la vapeur.}  \zh{量词}: \textcolor{darkblue}{\textbf{\ipa{ɭɯ˧}}} 
\lhead{\firstmark}
\rhead{\botmark}

\subsection{\hspace{-0.5cm} {\Large \textcolor{darkblue}{\textbf{\ipa{mo˧nɑ˥}}} \textsubscript{1}}\hspace{0.5cm}[\kern2pt{\textcolor{darkblue}{\textbf{\ipa{mo˧nɑ˥}}}}\kern2pt]} \hypertarget{mo\string_MnA\string_T1}{}
\markboth{\textcolor{darkblue}{\textbf{\ipa{mo˧nɑ˥}}} \textsubscript{1}}{}
\textcolor{teal}{\zh{名词}} \hspace{4pt} \zh{声调类:} H\#.
\zh{闲话。} \textcolor{Sepia}{\selectlanguage{english}Gossip.} \textcolor{PineGreen}{\selectlanguage{french}Médisance, ragot.}  ¶ \textcolor{darkblue}{\textbf{\ipa{mo˧nɑ˥ ʐwɤ˩}}} \zh{八卦、讲别人的坏话} \textcolor{Sepia}{\selectlanguage{english}to indulge in gossip, to speak badly of others} \textcolor{PineGreen}{\selectlanguage{french}ragoter, médire}  
 ¶ \textcolor{darkblue}{\textbf{\ipa{ʈʂʰɯ˧ | ɖɯ˧-ɲi˥ | mo˧nɑ˥ ʐwɤ˩-dʑo˩!}}} \zh{他一天到晚都在八卦!} \textcolor{Sepia}{\selectlanguage{english}(S)he gossips all day!} \textcolor{PineGreen}{\selectlanguage{french}Il/elle ragote toute la journée!}  
 ¶ \textcolor{darkblue}{\textbf{\ipa{mo˧nɑ˥-ɕi˩mi˩}}} \zh{同上:八卦、坏话} \textcolor{Sepia}{\selectlanguage{english}same meaning: gossip} \textcolor{PineGreen}{\selectlanguage{french}même sens: ragot, médisance}  
 ¶ \textcolor{darkblue}{\textbf{\ipa{mo˧nɑ˥-ɕi˩mi˩ ʐwɤ˩}}} \zh{八卦、讲别人的坏话} \textcolor{Sepia}{\selectlanguage{english}to indulge in gossip, to speak badly of others} \textcolor{PineGreen}{\selectlanguage{french}ragoter, médire}  
 ¶ \textcolor{darkblue}{\textbf{\ipa{hĩ˧ | ʈʂʰɯ˧-v̩˧, | mo˧nɑ˥-ɕi˩mi˩ | ɖɯ˧-v̩˧ ɲi˩!}}} \zh{他爱八卦、爱说别人坏话} \textcolor{Sepia}{\selectlanguage{english}He's a gossiper, he talks badly of other people} \textcolor{PineGreen}{\selectlanguage{french}C'est un ragoteur, il est médisant}  

\lhead{\firstmark}
\rhead{\botmark}

\subsection{\hspace{-0.5cm} {\Large \textcolor{darkblue}{\textbf{\ipa{mo˧nɑ˥}}} \textsubscript{2}}\hspace{0.5cm}[\kern2pt{\textcolor{darkblue}{\textbf{\ipa{mo˧nɑ˥}}}}\kern2pt]} \hypertarget{mo\string_MnA\string_T2}{}
\markboth{\textcolor{darkblue}{\textbf{\ipa{mo˧nɑ˥}}} \textsubscript{2}}{}
\textcolor{teal}{\zh{名词}} \hspace{4pt} \zh{声调类:} H\#.
\zh{剁成粉的秸杆。} \textcolor{Sepia}{\selectlanguage{english}Chopped straw, used when preparing pickled vegetables.} \textcolor{PineGreen}{\selectlanguage{french}Paille hachée, utilisée dans la préparation des légumes en saumure.}  ¶ \textcolor{darkblue}{\textbf{\ipa{mv˩zɯ˩-mo˩nɑ˥}}} \zh{剁成粉的燕麦秸杆} \textcolor{Sepia}{\selectlanguage{english}chopped oat straw} \textcolor{PineGreen}{\selectlanguage{french}paille d'avoine hachée}  

\lhead{\firstmark}
\rhead{\botmark}

\subsection{\hspace{-0.5cm} {\Large \textcolor{darkblue}{\textbf{\ipa{mo˧qʰwɤ˥}}}}\hspace{0.5cm}[\kern2pt{\textcolor{darkblue}{\textbf{\ipa{mo˧qʰwɤ˥}}}}\kern2pt]} \hypertarget{mo\string_Mq\string_hw7\string_T1}{}
\markboth{\textcolor{darkblue}{\textbf{\ipa{mo˧qʰwɤ˥}}}}{}
\textcolor{teal}{\zh{名词}} \hspace{4pt} \zh{声调类:} H\#.
\zh{溜索上往返移动的木头梭。} \textcolor{Sepia}{\selectlanguage{english}Wooden shuttle of zip line (flying fox): it glides along the rope; the passenger, horse, or load of goods is tied to the shuttle.} \textcolor{PineGreen}{\selectlanguage{french}Navette en bois d'un pont de corde: la navette coulisse sur la corde; passager, cheval ou chargement y sont attachés.}  \zh{量词}: \textcolor{darkblue}{\textbf{\ipa{ɭɯ˧}}} 
\lhead{\firstmark}
\rhead{\botmark}

\subsection{\hspace{-0.5cm} {\Large \textcolor{darkblue}{\textbf{\ipa{mo˧qʰwɤ˧˥}}}}\hspace{0.5cm}[\kern2pt{\textcolor{darkblue}{\textbf{\ipa{mo˧qʰwɤ˧˥}}}}\kern2pt]} \hypertarget{mo\string_Mq\string_hw7\string_M\string_T1}{}
\markboth{\textcolor{darkblue}{\textbf{\ipa{mo˧qʰwɤ˧˥}}}}{}
\textcolor{teal}{\zh{形容词}} \hspace{4pt} \zh{声调类:} MH.
\zh{胃口好,或:贪吃。} \textcolor{Sepia}{\selectlanguage{english}Fond of food; voracious (can range from neutral uses to clearly negative uses).} \textcolor{PineGreen}{\selectlanguage{french}Qui a bon appétit, qui a un solide appétit; gourmand, vorace (peut être neutre, ou franchement négatif).} 
\lhead{\firstmark}
\rhead{\botmark}

\subsection{\hspace{-0.5cm} {\Large \textcolor{darkblue}{\textbf{\ipa{mo˩}}}}\hspace{0.5cm}[\kern2pt{\textcolor{darkblue}{\textbf{\ipa{mo˩˥}}}}\kern2pt]} \hypertarget{mo\string_B1}{}
\markboth{\textcolor{darkblue}{\textbf{\ipa{mo˩}}}}{}
\textcolor{teal}{\zh{语气助词}} \hspace{4pt} \zh{声调类:} L.
\zh{句尾助词:请……。} \textcolor{Sepia}{\selectlanguage{english}Final particle indicating invitation/suggestion to do something.} \textcolor{PineGreen}{\selectlanguage{french}Particule indiquant l'invitation à faire quelque chose.}  ¶ \textcolor{darkblue}{\textbf{\ipa{no˧ | ɖɯ˧-ʈʰɯ˩-ɻ̍˩ mo˩!}}} \zh{请你喝一点!} \textcolor{Sepia}{\selectlanguage{english}Please drink a little! / Do have a sip!} \textcolor{PineGreen}{\selectlanguage{french}Bois donc un peu!}  

\lhead{\firstmark}
\rhead{\botmark}

\subsection{\hspace{-0.5cm} {\Large \textcolor{darkblue}{\textbf{\ipa{mo˩\textsubscript{a}}}} \textsubscript{1}}\hspace{0.5cm}[\kern2pt{\textcolor{darkblue}{\textbf{\ipa{mo˩˥}}}}\kern2pt]} \hypertarget{mo\string_Ba1}{}
\markboth{\textcolor{darkblue}{\textbf{\ipa{mo˩\textsubscript{a}}}} \textsubscript{1}}{}
\textcolor{teal}{\zh{形容词}} \hspace{4pt} \zh{声调类:} L\textsubscript{a}.
\zh{年老。} \textcolor{Sepia}{\selectlanguage{english}Old.} \textcolor{PineGreen}{\selectlanguage{french}Vieux, âgé.}  ¶ \textcolor{darkblue}{\textbf{\ipa{mo˩ hĩ˩˥}}} \zh{老人} \textcolor{Sepia}{\selectlanguage{english}old person} \textcolor{PineGreen}{\selectlanguage{french}vieille personne}  
 ¶ \textcolor{darkblue}{\textbf{\ipa{si˧ mo˥}}} \zh{老树、老木头} \textcolor{Sepia}{\selectlanguage{english}old wood; old tree} \textcolor{PineGreen}{\selectlanguage{french}vieux bois, vieil arbre}  
 ¶ \textcolor{darkblue}{\textbf{\ipa{le˧-mo˩-ze˩}}} \zh{\mytextsc{accomp} \string_ \mytextsc{pfv}} \textcolor{Sepia}{\selectlanguage{english}\mytextsc{accomp} \string_ \mytextsc{pfv}: (he/she) has become old / has aged.} \textcolor{PineGreen}{\selectlanguage{french}\mytextsc{accomp} \string_ \mytextsc{pfv}: (il/elle) a vieilli}  
 ¶ \textcolor{darkblue}{\textbf{\ipa{le˧-mo˩-hĩ˩}}} \zh{老了的人} \textcolor{Sepia}{\selectlanguage{english}Old person, person who has become old} \textcolor{PineGreen}{\selectlanguage{french}vieille personne, personne qui a vieilli}  
 ¶ \textcolor{darkblue}{\textbf{\ipa{hĩ˧ mo˥, | õ˧-di˧ fv̩˥! | ʐwæ˧ mo˥, | to˩ do˩ ɖwæ˥!}}} \zh{老人爱自家,老马怕山坡!(谚语,描写不爱到处跑的老年人)} \textcolor{Sepia}{\selectlanguage{english}Old folk like their own place (=their own home); old horses are afraid to climb slopes! (Proverb.)} \textcolor{PineGreen}{\selectlanguage{french}“Les vieilles personnes aiment leur chez-eux; les vieux chevaux ont peur de grimper les pentes!” (Sens: avec l'âge, on devient moins entreprenant.)}  
 ¶ \textcolor{darkblue}{\textbf{\ipa{lv̩˧ mo˥ F | dʑɯ˧ | le˧-qv̩˩; | si˧ mo˥ F | le˧-dze˩-kv̩˩! | no˧ F | ə˧tse˧ | le˧-ʂɯ˧-mɤ˧-tʰɑ˧˥ | di˩!}}} \zh{老石头要被河流冲走,老木头要被砍掉。你呢,怎么还不死? (嘲笑一个年龄很高的人。摩梭传统中,人的寿命是六十岁:过了七十岁的人,被认为是已经到达了命的尽头。)} \textcolor{Sepia}{\selectlanguage{english}Old stones are carried away by the stream; and old wood gets chopped down! And you, why can't you die? (Mocking an elderly person. Na tradition assigns man a lifespan of sixty years; people getting past seventy are considered to have reached the end of their lifespan.)} \textcolor{PineGreen}{\selectlanguage{french}Les vieilles pierres, le courant les emporte; le vieux bois, on le coupe! Alors pourquoi toi te voilà qui ne veux pas mourir! (Moquerie à l'égard d'une personne très âgée.)}  

\lhead{\firstmark}
\rhead{\botmark}

\subsection{\hspace{-0.5cm} {\Large \textcolor{darkblue}{\textbf{\ipa{mo˩\textsubscript{a}}}} \textsubscript{2}}\hspace{0.5cm}[\kern2pt{\textcolor{darkblue}{\textbf{\ipa{mo˩˥}}}}\kern2pt]} \hypertarget{mo\string_Ba2}{}
\markboth{\textcolor{darkblue}{\textbf{\ipa{mo˩\textsubscript{a}}}} \textsubscript{2}}{}
\textcolor{teal}{\zh{动词}} \hspace{4pt} \zh{声调类:} L\textsubscript{a}.
\zh{死、去世。} \textcolor{Sepia}{\selectlanguage{english}To die.} \textcolor{PineGreen}{\selectlanguage{french}Mourir.}  ¶ \textcolor{darkblue}{\textbf{\ipa{mɤ˧-mo˩-sɯ˩!}}} \zh{还没死!} \textcolor{Sepia}{\selectlanguage{english}(She/he/it) is not dead yet!} \textcolor{PineGreen}{\selectlanguage{french}(Il n'est) pas encore mort!}  
 ¶ \textcolor{darkblue}{\textbf{\ipa{si˧ mo˩}}} \zh{老干柴(直译:死了的木头)} \textcolor{Sepia}{\selectlanguage{english}dead wood} \textcolor{PineGreen}{\selectlanguage{french}bois mort}  

\lhead{\firstmark}
\rhead{\botmark}

\subsection{\hspace{-0.5cm} {\Large \textcolor{darkblue}{\textbf{\ipa{mo˩kv̩\#˥}}}}\hspace{0.5cm}[\kern2pt{\textcolor{darkblue}{\textbf{\ipa{mo˩kv̩˥}}}}\kern2pt]} \hypertarget{mo\string_Bkv\string_=\#\string_T1}{}
\markboth{\textcolor{darkblue}{\textbf{\ipa{mo˩kv̩\#˥}}}}{}
\textcolor{teal}{\zh{名词}} \hspace{4pt} \zh{声调类:} LM+\#H.
\zh{蘑菇:长在倒在地上的树(如青冈等树木)上的菌子(汉语借词)。} \textcolor{Sepia}{\selectlanguage{english}Mushrooms that grows on fallen trunks, e.g. oaks.} \textcolor{PineGreen}{\selectlanguage{french}Sorte de champignon qui pousse sur les chênes (sur les troncs tombés, sur les arbres morts).}  \zh{【借词】} \zh{蘑菇}
 ¶ \textcolor{darkblue}{\textbf{\ipa{mo˩kv̩˥, | si˧dzi˩-mo˩!}}} \zh{\textcolor{darkblue}{\textbf{\ipa{/mo˩kv̩\#˥/}}},指的是长在(倒在地上的)树上的菌子!} \textcolor{Sepia}{\selectlanguage{english}\textcolor{darkblue}{\textbf{\ipa{/mo˩kv̩\#˥/}}} refers to mushrooms that grow on trees!} \textcolor{PineGreen}{\selectlanguage{french}\textcolor{darkblue}{\textbf{\ipa{/mo˩kv̩\#˥/}}}, ça désigne les champignons qui pousse sur les arbres! (littéralement: “les champignons d'arbres”, par opposition aux “champignons de terre”)}  

\lhead{\firstmark}
\rhead{\botmark}

\subsection{\hspace{-0.5cm} {\Large \textcolor{darkblue}{\textbf{\ipa{mo˩ɻ\#˥}}}}\hspace{0.5cm}[\kern2pt{\textcolor{darkblue}{\textbf{\ipa{mo˩ɻ˥}}}}\kern2pt]} \hypertarget{mo\string_Br£`\#\string_T1}{}
\markboth{\textcolor{darkblue}{\textbf{\ipa{mo˩ɻ\#˥}}}}{}
\textcolor{teal}{\zh{名词}} \hspace{4pt} \zh{声调类:} LM+\#H.
\zh{木耳(汉语借词)。} \textcolor{Sepia}{\selectlanguage{english}Black mushroom, 'wood ear' (an edible fungus).} \textcolor{PineGreen}{\selectlanguage{french}Champignon noir.}  \zh{【借词】} \zh{木耳}

\lhead{\firstmark}
\rhead{\botmark}

\subsection{\hspace{-0.5cm} {\Large \textcolor{darkblue}{\textbf{\ipa{mo˩zo\#˥}}}}\hspace{0.5cm}[\kern2pt{\textcolor{darkblue}{\textbf{\ipa{mo˩zo˥}}}}\kern2pt]} \hypertarget{mo\string_Bzo\#\string_T1}{}
\markboth{\textcolor{darkblue}{\textbf{\ipa{mo˩zo\#˥}}}}{}
\textcolor{teal}{\zh{名词}} \hspace{4pt} \zh{声调类:} LM+\#H.
\zh{士兵。} \textcolor{Sepia}{\selectlanguage{english}Soldier.} \textcolor{PineGreen}{\selectlanguage{french}Militaire, soldat.}  ¶ \textcolor{darkblue}{\textbf{\ipa{mo˩zo˧ ʝi˧-hɯ˧ ɲi˥!}}} \zh{当兵去了!} \textcolor{Sepia}{\selectlanguage{english}He went to the army! / He joined the army! / He became a soldier!} \textcolor{PineGreen}{\selectlanguage{french}Il est parti à l'armée! / Il s'est fait soldat!}  
 \zh{量词}: \textcolor{darkblue}{\textbf{\ipa{v̩˧}}} 
\lhead{\firstmark}
\rhead{\botmark}

\subsection{\hspace{-0.5cm} {\Large \textcolor{darkblue}{\textbf{\ipa{mo˧˥}}}}\hspace{0.5cm}[\kern2pt{\textcolor{darkblue}{\textbf{\ipa{mo˧˥}}}}\kern2pt]} \hypertarget{mo\string_M\string_T1}{}
\markboth{\textcolor{darkblue}{\textbf{\ipa{mo˧˥}}}}{}
\textcolor{teal}{\zh{名词}} \hspace{4pt} \zh{声调类:} MH.
\zh{菌子、蘑菇。} \textcolor{Sepia}{\selectlanguage{english}Mushroom.} \textcolor{PineGreen}{\selectlanguage{french}Champignon.}  \zh{量词}: \textcolor{darkblue}{\textbf{\ipa{ɭɯ˧}}} \textcolor{darkblue}{\textbf{\ipa{mo˧˥}}} \zh{~【参考】~} \hyperlink{}{\textcolor{darkblue}{\textbf{\ipa{mo˧˥\textsubscript{a}}}}} 
\lhead{\firstmark}
\rhead{\botmark}

\subsection{\hspace{-0.5cm} {\Large \textcolor{darkblue}{\textbf{\ipa{mo˧˥\textsubscript{a}}}}}\hspace{0.5cm}[\kern2pt{\textcolor{darkblue}{\textbf{\ipa{mo˧˥}}}}\kern2pt]} \hypertarget{mo\string_M\string_Ta1}{}
\markboth{\textcolor{darkblue}{\textbf{\ipa{mo˧˥\textsubscript{a}}}}}{}
\textcolor{teal}{\zh{量词}} \hspace{4pt} \zh{声调类:} MH\textsubscript{a}.
\zh{量词:蘑菇(一只)。} \textcolor{Sepia}{\selectlanguage{english}Self-classifier for mushrooms.} \textcolor{PineGreen}{\selectlanguage{french}Auto-classificateur des champignons.} \zh{~【参考】~} \hyperlink{}{\textcolor{darkblue}{\textbf{\ipa{mo˧˥}}}} 
\lhead{\firstmark}
\rhead{\botmark}

\subsection{\hspace{-0.5cm} {\Large \textcolor{darkblue}{\textbf{\ipa{mv̩˩˥}}}}\hspace{0.5cm}[\kern2pt{\textcolor{darkblue}{\textbf{\ipa{mv̩˩˥}}}}\kern2pt]} \hypertarget{mv\string_=\string_B\string_T1}{}
\markboth{\textcolor{darkblue}{\textbf{\ipa{mv̩˩˥}}}}{}
\textcolor{teal}{\zh{名词}} \hspace{4pt} \zh{声调类:} LH.
\zh{女儿。} \textcolor{Sepia}{\selectlanguage{english}Daughter.} \textcolor{PineGreen}{\selectlanguage{french}Fille.}  \zh{量词}: \textcolor{darkblue}{\textbf{\ipa{v̩˧}}} 
\lhead{\firstmark}
\rhead{\botmark}

\subsection{\hspace{-0.5cm} {\Large \textcolor{darkblue}{\textbf{\ipa{mv̩˧}}}}\hspace{0.5cm}[\kern2pt{\textcolor{darkblue}{\textbf{\ipa{mv̩˥}}}}\kern2pt]} \hypertarget{mv\string_=\string_M1}{}
\markboth{\textcolor{darkblue}{\textbf{\ipa{mv̩˧}}}}{}
\textcolor{teal}{\zh{名词}} \hspace{4pt} \zh{声调类:} M.
\zh{姓名。} \textcolor{Sepia}{\selectlanguage{english}Name (given name or family name).} \textcolor{PineGreen}{\selectlanguage{french}Nom (nom de famille ou prénom: nom donné à un individu).}  ¶ \textcolor{darkblue}{\textbf{\ipa{ɑ˩ʁo˧-bv̩˧ | mv̩˧ (+ɲi˩)}}} \zh{这是家里的姓! / 这是我家的姓!} \textcolor{Sepia}{\selectlanguage{english}This is the family name! / This is my family name!} \textcolor{PineGreen}{\selectlanguage{french}c'est le nom de la famille / c'est mon nom de famille!}  
 ¶ \textcolor{darkblue}{\textbf{\ipa{njɤ˧ | mv̩˧ ɖɯ˧-kʰwɤ˥ | ʂe˧-zo˧-ho˩!}}} \zh{我得去(向大寺里的和尚)求一个名字(给刚出生的孩子起名)} \textcolor{Sepia}{\selectlanguage{english}I have to go and get a name (from the monks at the monastery) (for a newborn child)} \textcolor{PineGreen}{\selectlanguage{french}Il va falloir que j'aille chercher un nom (auprès des moines du monastère) (pour un enfant qui vient de naître)!}  

\lhead{\firstmark}
\rhead{\botmark}

\subsection{\hspace{-0.5cm} {\Large \textcolor{darkblue}{\textbf{\ipa{mv̩˧}}}}\hspace{0.5cm}[\kern2pt{\textcolor{darkblue}{\textbf{\ipa{mv̩˥}}}}\kern2pt]} \hypertarget{mv\string_=\string_M1}{}
\markboth{\textcolor{darkblue}{\textbf{\ipa{mv̩˧}}}}{}
\textcolor{teal}{\zh{语气助词}} \hspace{4pt} \zh{声调类:} M.
\zh{句尾助词,表示肯定:“嘛”。} \textcolor{Sepia}{\selectlanguage{english}Affirmative final particle.} \textcolor{PineGreen}{\selectlanguage{french}Particule finale affirmative.} 
\lhead{\firstmark}
\rhead{\botmark}

\subsection{\hspace{-0.5cm} {\Large \textcolor{darkblue}{\textbf{\ipa{mv̩˥}}}}\hspace{0.5cm}[\kern2pt{\textcolor{darkblue}{\textbf{\ipa{mv̩˥}}}}\kern2pt]} \hypertarget{mv\string_=\string_T1}{}
\markboth{\textcolor{darkblue}{\textbf{\ipa{mv̩˥}}}}{}
\textcolor{teal}{\zh{动词}} \hspace{4pt} \zh{声调类:} H.
\ding{202} \zh{懂,听见。} \textcolor{Sepia}{\selectlanguage{english}To hear.} \textcolor{PineGreen}{\selectlanguage{french}Entendre.}  ¶ \textcolor{darkblue}{\textbf{\ipa{njɤ˧ | le˧-mv̩˥-ze˩}}} \zh{我听见了} \textcolor{Sepia}{\selectlanguage{english}I have heard} \textcolor{PineGreen}{\selectlanguage{french}j'ai entendu}  
\ding{203} \zh{懂。} \textcolor{Sepia}{\selectlanguage{english}To understand.} \textcolor{PineGreen}{\selectlanguage{french}Comprendre.}  ¶ \textcolor{darkblue}{\textbf{\ipa{njɤ˧ | le˧-mv̩˥-ze˩}}} \zh{我懂了} \textcolor{Sepia}{\selectlanguage{english}I have understood} \textcolor{PineGreen}{\selectlanguage{french}j'ai compris}  

\lhead{\firstmark}
\rhead{\botmark}

\subsection{\hspace{-0.5cm} {\Large \textcolor{darkblue}{\textbf{\ipa{mv̩˥}}} \textsubscript{1}}\hspace{0.5cm}[\kern2pt{\textcolor{darkblue}{\textbf{\ipa{mv̩˥}}}}\kern2pt]} \hypertarget{mv\string_=\string_T1}{}
\markboth{\textcolor{darkblue}{\textbf{\ipa{mv̩˥}}} \textsubscript{1}}{}
\textcolor{teal}{\zh{名词}} \hspace{4pt} \zh{声调类:} \#H.
\zh{天。} \textcolor{Sepia}{\selectlanguage{english}Sky.} \textcolor{PineGreen}{\selectlanguage{french}Ciel.}  ¶ \textcolor{darkblue}{\textbf{\ipa{mv̩˧tʰv̩˧(-ze˩)}}} \zh{天晴,天色亮} \textcolor{Sepia}{\selectlanguage{english}the day is bright, the sky is clear} \textcolor{PineGreen}{\selectlanguage{french}il fait clair, il fait grand jour, le ciel est clair}  
 ¶ \textcolor{darkblue}{\textbf{\ipa{hĩ˧-ɳɯ˩ mɤ˩-do˩, | mv̩˧-ɳɯ˩ | do˩˥!}}} \zh{“人看不见的,老天能看见!”} \textcolor{Sepia}{\selectlanguage{english}“What humans do not see, the Heavens see it!” (Meaning: a good deed earns one happiness in future; and a count of bad deeds, even those that go unseen by humans, is also kept in the Heavens.)} \textcolor{PineGreen}{\selectlanguage{french}“Ce que les hommes ne voient pas, le ciel le voit!” (Sens: une bonne action n'est jamais perdue, et une mauvaise reçoit sa punition dans le monde d'en haut.)}  
 \zh{量词}: \textcolor{darkblue}{\textbf{\ipa{ɭɯ˧}}} 
\lhead{\firstmark}
\rhead{\botmark}

\subsection{\hspace{-0.5cm} {\Large \textcolor{darkblue}{\textbf{\ipa{mv̩˥}}} \textsubscript{2}}\hspace{0.5cm}[\kern2pt{\textcolor{darkblue}{\textbf{\ipa{mv̩˥}}}}\kern2pt]} \hypertarget{mv\string_=\string_T2}{}
\markboth{\textcolor{darkblue}{\textbf{\ipa{mv̩˥}}} \textsubscript{2}}{}
\textcolor{teal}{\zh{名词}} \hspace{4pt} \zh{声调类:} \#H.
\zh{火。} \textcolor{Sepia}{\selectlanguage{english}Fire.} \textcolor{PineGreen}{\selectlanguage{french}Feu.}  ¶ \textcolor{darkblue}{\textbf{\ipa{mv̩˧ kʰɯ˩}}} \zh{点火} \textcolor{Sepia}{\selectlanguage{english}to light a fire, to do a fire} \textcolor{PineGreen}{\selectlanguage{french}allumer un feu, faire un feu}  
 \zh{量词}: \textcolor{darkblue}{\textbf{\ipa{æ̃˩}}} 
\lhead{\firstmark}
\rhead{\botmark}

\subsection{\hspace{-0.5cm} {\Large \textcolor{darkblue}{\textbf{\ipa{mv̩˩\textsubscript{a}}}} \textsubscript{1}}\hspace{0.5cm}[\kern2pt{\textcolor{darkblue}{\textbf{\ipa{mv̩˥}}}}\kern2pt]} \hypertarget{mv\string_=\string_Ba1}{}
\markboth{\textcolor{darkblue}{\textbf{\ipa{mv̩˩\textsubscript{a}}}} \textsubscript{1}}{}
\textcolor{teal}{\zh{动词}} \hspace{4pt} \zh{声调类:} L\textsubscript{a}.
\zh{吹(灰,乐器)。} \textcolor{Sepia}{\selectlanguage{english}To blow (e.g. to blow the fire, to blow a horn).} \textcolor{PineGreen}{\selectlanguage{french}Souffler (ex.: souffler sur le feu, attiser le feu; souffler dans un instrument à vent).}  ¶ \textcolor{darkblue}{\textbf{\ipa{mv̩˧\textasciitilde{}mv̩˥(-ze˩)}}} \zh{\mytextsc{重叠:吹吹}} \textcolor{Sepia}{\selectlanguage{english}\mytextsc{red}} \textcolor{PineGreen}{\selectlanguage{french}\mytextsc{red}}  
 ¶ \textcolor{darkblue}{\textbf{\ipa{ʝi˧qʰv̩˧ mv̩˥}}} \zh{吹号角} \textcolor{Sepia}{\selectlanguage{english}to blow a horn} \textcolor{PineGreen}{\selectlanguage{french}souffler dans une corne}  

\lhead{\firstmark}
\rhead{\botmark}

\subsection{\hspace{-0.5cm} {\Large \textcolor{darkblue}{\textbf{\ipa{mv̩˩\textsubscript{a}}}} \textsubscript{2}}\hspace{0.5cm}[\kern2pt{\textcolor{darkblue}{\textbf{\ipa{mv̩˩˥}}}}\kern2pt]} \hypertarget{mv\string_=\string_Ba2}{}
\markboth{\textcolor{darkblue}{\textbf{\ipa{mv̩˩\textsubscript{a}}}} \textsubscript{2}}{}
\textcolor{teal}{\zh{动词}} \hspace{4pt} \zh{声调类:} L\textsubscript{a}.
\zh{冲(走)。} \textcolor{Sepia}{\selectlanguage{english}To flush away, to carry away: a strong current carries a swimmer away.} \textcolor{PineGreen}{\selectlanguage{french}Emporter (le courant emporte un nageur), balayer (une vague balaie une épave de bateau).} 
\lhead{\firstmark}
\rhead{\botmark}

\subsection{\hspace{-0.5cm} {\Large \textcolor{darkblue}{\textbf{\ipa{mv̩˩\textsubscript{a}}}} \textsubscript{3}}\hspace{0.5cm}[\kern2pt{\textcolor{darkblue}{\textbf{\ipa{mv̩˩˥}}}}\kern2pt]} \hypertarget{mv\string_=\string_Ba3}{}
\markboth{\textcolor{darkblue}{\textbf{\ipa{mv̩˩\textsubscript{a}}}} \textsubscript{3}}{}
\textcolor{teal}{\zh{形容词}} \hspace{4pt} \zh{声调类:} L\textsubscript{a}.
\ding{202} \zh{熟、成熟(植物、水果)。} \textcolor{Sepia}{\selectlanguage{english}Ripe.} \textcolor{PineGreen}{\selectlanguage{french}Mûr (produit agricole).}  ¶ \textcolor{darkblue}{\textbf{\ipa{mv̩˩-hĩ˩˥}}} \zh{熟的} \textcolor{Sepia}{\selectlanguage{english}\mytextsc{rel}} \textcolor{PineGreen}{\selectlanguage{french}\mytextsc{rel}}  
\ding{203} \zh{熟(食物)。} \textcolor{Sepia}{\selectlanguage{english}Cooked, well-cooked, done.} \textcolor{PineGreen}{\selectlanguage{french}Cuit (aliment).} 
\lhead{\firstmark}
\rhead{\botmark}

\subsection{\hspace{-0.5cm} {\Large \textcolor{darkblue}{\textbf{\ipa{mv̩˩\textsubscript{a}}}} \textsubscript{4}}\hspace{0.5cm}[\kern2pt{\textcolor{darkblue}{\textbf{\ipa{mv̩˩˥}}}}\kern2pt]} \hypertarget{mv\string_=\string_Ba4}{}
\markboth{\textcolor{darkblue}{\textbf{\ipa{mv̩˩\textsubscript{a}}}} \textsubscript{4}}{}
\textcolor{teal}{\zh{动词}} \hspace{4pt} \zh{声调类:} L\textsubscript{a}.
\zh{燃烧。} \textcolor{Sepia}{\selectlanguage{english}To burn, to become consumed (e.g. a body on the funeral pyre becomes consumed).} \textcolor{PineGreen}{\selectlanguage{french}Brûler, se consumer (ex.: un corps sur le bûcher).} 
\lhead{\firstmark}
\rhead{\botmark}

\subsection{\hspace{-0.5cm} {\Large \textcolor{darkblue}{\textbf{\ipa{mv̩˧\textsubscript{a}}}}}\hspace{0.5cm}[\kern2pt{\textcolor{darkblue}{\textbf{\ipa{mv̩˩˥}}}}\kern2pt]} \hypertarget{mv\string_=\string_Ma1}{}
\markboth{\textcolor{darkblue}{\textbf{\ipa{mv̩˧\textsubscript{a}}}}}{}
\textcolor{teal}{\zh{动词}} \hspace{4pt} \zh{声调类:} M\textsubscript{a}.
\zh{穿(衣服、上衣)。} \textcolor{Sepia}{\selectlanguage{english}To put on (a shirt, a jacket).} \textcolor{PineGreen}{\selectlanguage{french}Mettre, porter, enfiler, endosser (une chemise, une veste); se vêtir d'un habit.}  ¶ \textcolor{darkblue}{\textbf{\ipa{bɑ˩lɑ˩ mv̩˥}}} \zh{穿衣服} \textcolor{Sepia}{\selectlanguage{english}to put on a shirt/jacket} \textcolor{PineGreen}{\selectlanguage{french}mettre une chemise/veste}  
 ¶ \textcolor{darkblue}{\textbf{\ipa{bɑ˩lɑ˩˥ | tʰi˧-mv̩˧}}} \zh{穿衣服} \textcolor{Sepia}{\selectlanguage{english}to put on a shirt/jacket} \textcolor{PineGreen}{\selectlanguage{french}mettre une chemise/veste}  
 ¶ \textcolor{darkblue}{\textbf{\ipa{dʑi˧hṽ˧ mv̩˩}}} \zh{穿衣服} \textcolor{Sepia}{\selectlanguage{english}to put on clothes} \textcolor{PineGreen}{\selectlanguage{french}enfiler un habit}  
\zh{~【参考】~} \hyperlink{}{\textcolor{darkblue}{\textbf{\ipa{ki˩\textsubscript{a}}}}} 
\lhead{\firstmark}
\rhead{\botmark}

\subsection{\hspace{-0.5cm} {\Large \textcolor{darkblue}{\textbf{\ipa{mv̩˩-bæ˧mi˩}}}}\hspace{0.5cm}[\kern2pt{\textcolor{darkblue}{\textbf{\ipa{xxxx non-correspondance entre le nombre de morphèmes et le nombre de tons de morphèmes}}}}\kern2pt]} \hypertarget{mv\string_=\string_B-b\{\string_Mmi\string_B1}{}
\markboth{\textcolor{darkblue}{\textbf{\ipa{mv̩˩-bæ˧mi˩}}}}{}
\textcolor{teal}{\zh{名词}} \hspace{4pt} \zh{声调类:} L-L\#.
\zh{傻女人、笨女人。} \textcolor{Sepia}{\selectlanguage{english}Fool, idiot (female).} \textcolor{PineGreen}{\selectlanguage{french}Imbécile, idiote.}  \zh{量词}: \textcolor{darkblue}{\textbf{\ipa{v̩˧}}} 
\lhead{\firstmark}
\rhead{\botmark}

\subsection{\hspace{-0.5cm} {\Large \textcolor{darkblue}{\textbf{\ipa{mv̩˧bɤ\#˥}}}}\hspace{0.5cm}[\kern2pt{\textcolor{darkblue}{\textbf{\ipa{xxxx non-correspondance entre le nombre de morphèmes et le nombre de tons de morphèmes}}}}\kern2pt]} \hypertarget{mv\string_=\string_Mb7\#\string_T1}{}
\markboth{\textcolor{darkblue}{\textbf{\ipa{mv̩˧bɤ\#˥}}}}{}
\textcolor{teal}{\zh{名词}} \hspace{4pt} \zh{声调类:} \#H.
\zh{脚底。} \textcolor{Sepia}{\selectlanguage{english}Sole of the foot.} \textcolor{PineGreen}{\selectlanguage{french}Plante du pied.}  \zh{量词}: \textcolor{darkblue}{\textbf{\ipa{kʰwɤ˥}}} 
\lhead{\firstmark}
\rhead{\botmark}

\subsection{\hspace{-0.5cm} {\Large \textcolor{darkblue}{\textbf{\ipa{mv̩˧bv̩˧ʐv̩˥}}}}\hspace{0.5cm}[\kern2pt{\textcolor{darkblue}{\textbf{\ipa{mv̩˧bv̩˧ʐv̩˧}}}}\kern2pt]} \hypertarget{mv\string_=\string_Mbv\string_=\string_Mz`v\string_=\string_T1}{}
\markboth{\textcolor{darkblue}{\textbf{\ipa{mv̩˧bv̩˧ʐv̩˥}}}}{}
\textcolor{teal}{\zh{名词}} \hspace{4pt} \zh{声调类:} H\#.
\zh{龙。} \textcolor{Sepia}{\selectlanguage{english}Dragon.} \textcolor{PineGreen}{\selectlanguage{french}Dragon.}  \zh{量词}: \textcolor{darkblue}{\textbf{\ipa{mi˩}}} 
\lhead{\firstmark}
\rhead{\botmark}

\subsection{\hspace{-0.5cm} {\Large \textcolor{darkblue}{\textbf{\ipa{mv̩˧ɕi˥}}}}\hspace{0.5cm}[\kern2pt{\textcolor{darkblue}{\textbf{\ipa{mv̩˧ɕi˥}}}}\kern2pt]} \hypertarget{mv\string_=\string_Ms£i\string_T1}{}
\markboth{\textcolor{darkblue}{\textbf{\ipa{mv̩˧ɕi˥}}}}{}
\textcolor{teal}{\zh{名词}} \hspace{4pt} \zh{声调类:} H\#.
\zh{火花。} \textcolor{Sepia}{\selectlanguage{english}Spark.} \textcolor{PineGreen}{\selectlanguage{french}Étincelle.}  \zh{量词}: \textcolor{darkblue}{\textbf{\ipa{æ̃˩}}} 
\lhead{\firstmark}
\rhead{\botmark}

\subsection{\hspace{-0.5cm} {\Large \textcolor{darkblue}{\textbf{\ipa{mv̩˧ɕi˥dʑɯ˩ʈʰɯ˩}}}}\hspace{0.5cm}[\kern2pt{\textcolor{darkblue}{\textbf{\ipa{mv̩˧ɕi˧dʑɯ˧ʈʰɯ˥}}}}\kern2pt]} \hypertarget{mv\string_=\string_Ms£i\string_Tdz£M\string_Bt`\string_hM\string_B1}{}
\markboth{\textcolor{darkblue}{\textbf{\ipa{mv̩˧ɕi˥dʑɯ˩ʈʰɯ˩}}}}{}
\textcolor{teal}{\zh{名词}} \hspace{4pt} \zh{声调类:} H\#-L.
\zh{彩虹。} \textcolor{Sepia}{\selectlanguage{english}Rainbow.} \textcolor{PineGreen}{\selectlanguage{french}Arc-en-ciel.}  \zh{量词}: \textcolor{darkblue}{\textbf{\ipa{kʰɯ˩}}} 
\lhead{\firstmark}
\rhead{\botmark}

\subsection{\hspace{-0.5cm} {\Large \textcolor{darkblue}{\textbf{\ipa{mv̩˩ɖæ˧}}}}\hspace{0.5cm}[\kern2pt{\textcolor{darkblue}{\textbf{\ipa{xxxx non-correspondance entre le nombre de morphèmes et le nombre de tons de morphèmes}}}}\kern2pt]} \hypertarget{mv\string_=\string_Bd`\{\string_M1}{}
\markboth{\textcolor{darkblue}{\textbf{\ipa{mv̩˩ɖæ˧}}}}{}
\textcolor{teal}{\zh{名词}} \hspace{4pt} \zh{声调类:} LM.
\zh{下半身。} \textcolor{Sepia}{\selectlanguage{english}Bottom part of body.} \textcolor{PineGreen}{\selectlanguage{french}Le bas du corps.} 
\lhead{\firstmark}
\rhead{\botmark}

\subsection{\hspace{-0.5cm} {\Large \textcolor{darkblue}{\textbf{\ipa{mv̩˧di˧˥}}}}\hspace{0.5cm}[\kern2pt{\textcolor{darkblue}{\textbf{\ipa{mv̩˩di˥}}}}\kern2pt]} \hypertarget{mv\string_=\string_Mdi\string_M\string_T1}{}
\markboth{\textcolor{darkblue}{\textbf{\ipa{mv̩˧di˧˥}}}}{}
\textcolor{teal}{\zh{名词}} \hspace{4pt} \zh{声调类:} MH\#.
\ding{202} \zh{田地。} \textcolor{Sepia}{\selectlanguage{english}Field.} \textcolor{PineGreen}{\selectlanguage{french}Champs (quel que soit ce qu'on y cultive).}  \zh{量词}: \textcolor{darkblue}{\textbf{\ipa{kɤ˧˥}}} \ding{203} \zh{天下。} \textcolor{Sepia}{\selectlanguage{english}The Earth, the place where mankind lives (as opposed to the Heavens).} \textcolor{PineGreen}{\selectlanguage{french}La Terre, là où habitent les hommes (par opposition au ciel).} 
\lhead{\firstmark}
\rhead{\botmark}

\subsection{\hspace{-0.5cm} {\Large \textcolor{darkblue}{\textbf{\ipa{mv̩˩do˩}}}}\hspace{0.5cm}[\kern2pt{\textcolor{darkblue}{\textbf{\ipa{mv̩˧do˧˥}}}}\kern2pt]} \hypertarget{mv\string_=\string_Bdo\string_B1}{}
\markboth{\textcolor{darkblue}{\textbf{\ipa{mv̩˩do˩}}}}{}
\textcolor{teal}{\zh{动词}} \hspace{4pt} \zh{声调类:} L.
\zh{问。} \textcolor{Sepia}{\selectlanguage{english}To ask.} \textcolor{PineGreen}{\selectlanguage{french}Demander.}  ¶ \textcolor{darkblue}{\textbf{\ipa{le˧-mv̩˩do˩}}} \zh{\mytextsc{accomp}} \textcolor{Sepia}{\selectlanguage{english}\mytextsc{accomp}} \textcolor{PineGreen}{\selectlanguage{french}\mytextsc{accomp}}  
 ¶ \textcolor{darkblue}{\textbf{\ipa{mv̩˩do˩-ze˥}}} \zh{问了} \textcolor{Sepia}{\selectlanguage{english}\mytextsc{pfv}} \textcolor{PineGreen}{\selectlanguage{french}\mytextsc{pfv}}  
 ¶ \textcolor{darkblue}{\textbf{\ipa{ə˧tso˧ mv̩˩do˩-bi˩? |}}} \zh{要问什么呢?} \textcolor{Sepia}{\selectlanguage{english}What would [you] like to ask? / What is your question?} \textcolor{PineGreen}{\selectlanguage{french}qu'est-ce que (tu) vas demander?}  

\lhead{\firstmark}
\rhead{\botmark}

\subsection{\hspace{-0.5cm} {\Large \textcolor{darkblue}{\textbf{\ipa{mv̩˩ɖɯ˩}}}}\hspace{0.5cm}[\kern2pt{\textcolor{darkblue}{\textbf{\ipa{mv̩˩ɖɯ˩˥}}}}\kern2pt]} \hypertarget{mv\string_=\string_Bd`M\string_B1}{}
\markboth{\textcolor{darkblue}{\textbf{\ipa{mv̩˩ɖɯ˩}}}}{}
\textcolor{teal}{\zh{名词}} \hspace{4pt} \zh{声调类:} L.
\zh{大女儿。} \textcolor{Sepia}{\selectlanguage{english}Eldest daughter.} \textcolor{PineGreen}{\selectlanguage{french}Fille aînée.}  ¶ \textcolor{darkblue}{\textbf{\ipa{zo˧ɖɯ˧-mv̩˥ɖɯ˩}}} \zh{大儿子与大女儿} \textcolor{Sepia}{\selectlanguage{english}eldest son and eldest daughter (i.e. eldest male and female siblings)} \textcolor{PineGreen}{\selectlanguage{french}fils aîné et fille aînée: les aînés}  

\lhead{\firstmark}
\rhead{\botmark}

\subsection{\hspace{-0.5cm} {\Large \textcolor{darkblue}{\textbf{\ipa{mv̩˧dze˧}}}}\hspace{0.5cm}[\kern2pt{\textcolor{darkblue}{\textbf{\ipa{mv̩˩dze˩˥}}}}\kern2pt]} \hypertarget{mv\string_=\string_Mdze\string_M1}{}
\markboth{\textcolor{darkblue}{\textbf{\ipa{mv̩˧dze˧}}}}{}
\textcolor{teal}{\zh{名词}} \hspace{4pt} \zh{声调类:} M.
\zh{大麦。} \textcolor{Sepia}{\selectlanguage{english}Barley, \textit{Hordeum vulgare L}.} \textcolor{PineGreen}{\selectlanguage{french}Orge, \textit{Hordeum vulgare L}.}  \zh{量词}: \textcolor{darkblue}{\textbf{\ipa{kɤ˧˥}}} 
\lhead{\firstmark}
\rhead{\botmark}

\subsection{\hspace{-0.5cm} {\Large \textcolor{darkblue}{\textbf{\ipa{mv̩˧dze˧-tɕʰi\#˥}}}}\hspace{0.5cm}[\kern2pt{\textcolor{darkblue}{\textbf{\ipa{xxxx non-correspondance entre le nombre de morphèmes et le nombre de tons de morphèmes}}}}\kern2pt]} \hypertarget{mv\string_=\string_Mdze\string_M-ts£\string_hi\#\string_T1}{}
\markboth{\textcolor{darkblue}{\textbf{\ipa{mv̩˧dze˧-tɕʰi\#˥}}}}{}
\textcolor{teal}{\zh{名词}} \hspace{4pt} \zh{声调类:} \#H.
\zh{青稞芒。} \textcolor{Sepia}{\selectlanguage{english}Highland barley beard.} \textcolor{PineGreen}{\selectlanguage{french}Barbe d'orge.} 
\lhead{\firstmark}
\rhead{\botmark}

\subsection{\hspace{-0.5cm} {\Large \textcolor{darkblue}{\textbf{\ipa{mv̩˩dzɤ˧}}}}\hspace{0.5cm}[\kern2pt{\textcolor{darkblue}{\textbf{\ipa{mv̩˧dzɤ˧}}}}\kern2pt]} \hypertarget{mv\string_=\string_Bdz7\string_M1}{}
\markboth{\textcolor{darkblue}{\textbf{\ipa{mv̩˩dzɤ˧}}}}{}
\textcolor{teal}{\zh{名词}} \hspace{4pt} \zh{声调类:} LM.
\zh{下面部分。} \textcolor{Sepia}{\selectlanguage{english}Bottom part (symbolically: “the tail”).} \textcolor{PineGreen}{\selectlanguage{french}Bas, partie inférieure (symboliquement: “la queue”).}  ¶ \textcolor{darkblue}{\textbf{\ipa{mv̩˩dzɤ˧ dzi˧˥}}} \zh{坐在(房间的)下面部分} \textcolor{Sepia}{\selectlanguage{english}to be seated in the bottom part (of the room)} \textcolor{PineGreen}{\selectlanguage{french}être assis au fond de la salle}  
 ¶ \textcolor{darkblue}{\textbf{\ipa{no˧ | mv̩˩dzɤ˧ dzi˧˥!}}} \zh{你到下面去坐!} \textcolor{Sepia}{\selectlanguage{english}Go and get seated in the bottom part (of the room)!} \textcolor{PineGreen}{\selectlanguage{french}Assieds-toi en bas!}  

\lhead{\firstmark}
\rhead{\botmark}

\subsection{\hspace{-0.5cm} {\Large \textcolor{darkblue}{\textbf{\ipa{mv̩˧gɤ˥gɤ˩}}}}\hspace{0.5cm}[\kern2pt{\textcolor{darkblue}{\textbf{\ipa{mv̩˩gɤ˧gɤ˧}}}}\kern2pt]} \hypertarget{mv\string_=\string_Mg7\string_Tg7\string_B1}{}
\markboth{\textcolor{darkblue}{\textbf{\ipa{mv̩˧gɤ˥gɤ˩}}}}{}
\textcolor{teal}{\zh{名词}} \hspace{4pt} \zh{声调类:} .
\zh{下一代、后裔、后人。} \textcolor{Sepia}{\selectlanguage{english}Descendants.} \textcolor{PineGreen}{\selectlanguage{french}Les descendants, la descendance.} 
\lhead{\firstmark}
\rhead{\botmark}

\subsection{\hspace{-0.5cm} {\Large \textcolor{darkblue}{\textbf{\ipa{mv̩˧-gɤ˧lɑ˥}}}}\hspace{0.5cm}[\kern2pt{\textcolor{darkblue}{\textbf{\ipa{xxxx non-correspondance entre le nombre de morphèmes et le nombre de tons de morphèmes}}}}\kern2pt]} \hypertarget{mv\string_=\string_M-g7\string_MlA\string_T1}{}
\markboth{\textcolor{darkblue}{\textbf{\ipa{mv̩˧-gɤ˧lɑ˥}}}}{}
\textcolor{teal}{\zh{名词}} \hspace{4pt} \zh{声调类:} H\#.
\zh{天宫菩萨。} \textcolor{Sepia}{\selectlanguage{english}Sky spirit.} \textcolor{PineGreen}{\selectlanguage{french}Esprit du ciel, Bodhisattva céleste.}  \zh{量词}: \textcolor{darkblue}{\textbf{\ipa{v̩˧}}} 
\lhead{\firstmark}
\rhead{\botmark}

\subsection{\hspace{-0.5cm} {\Large \textcolor{darkblue}{\textbf{\ipa{mv̩˧-gv̩˧dv̩˧}}}}\hspace{0.5cm}[\kern2pt{\textcolor{darkblue}{\textbf{\ipa{xxxx non-correspondance entre le nombre de morphèmes et le nombre de tons de morphèmes}}}}\kern2pt]} \hypertarget{mv\string_=\string_M-gv\string_=\string_Mdv\string_=\string_M1}{}
\markboth{\textcolor{darkblue}{\textbf{\ipa{mv̩˧-gv̩˧dv̩˧}}}}{}
\textcolor{teal}{\zh{名词}} \hspace{4pt} \zh{声调类:} M.
\zh{脚背。} \textcolor{Sepia}{\selectlanguage{english}Instep, top part of the foot.} \textcolor{PineGreen}{\selectlanguage{french}Partie supérieure du pied.}  \zh{量词}: \textcolor{darkblue}{\textbf{\ipa{ɭɯ˧}}} 
\lhead{\firstmark}
\rhead{\botmark}

\subsection{\hspace{-0.5cm} {\Large \textcolor{darkblue}{\textbf{\ipa{mv̩˧gv̩˧-kʰv̩˩}}}}\hspace{0.5cm}[\kern2pt{\textcolor{darkblue}{\textbf{\ipa{xxxx non-correspondance entre le nombre de morphèmes et le nombre de tons de morphèmes}}}}\kern2pt]} \hypertarget{mv\string_=\string_Mgv\string_=\string_M-k\string_hv\string_=\string_B1}{}
\markboth{\textcolor{darkblue}{\textbf{\ipa{mv̩˧gv̩˧-kʰv̩˩}}}}{}
\textcolor{teal}{\zh{名词}} \hspace{4pt} \zh{声调类:} L\#.
\zh{龙年。} \textcolor{Sepia}{\selectlanguage{english}Year of the dragon.} \textcolor{PineGreen}{\selectlanguage{french}Année du dragon.} 
\lhead{\firstmark}
\rhead{\botmark}

\subsection{\hspace{-0.5cm} {\Large \textcolor{darkblue}{\textbf{\ipa{mv̩˧gv̩\#˥}}}}\hspace{0.5cm}[\kern2pt{\textcolor{darkblue}{\textbf{\ipa{mv̩˧gv̩˧}}}}\kern2pt]} \hypertarget{mv\string_=\string_Mgv\string_=\#\string_T1}{}
\markboth{\textcolor{darkblue}{\textbf{\ipa{mv̩˧gv̩\#˥}}}}{}
\textcolor{teal}{\zh{名词}} \hspace{4pt} \zh{声调类:} \#H.
\zh{雷、雷声。} \textcolor{Sepia}{\selectlanguage{english}Clap of thunder.} \textcolor{PineGreen}{\selectlanguage{french}Tonnerre.}  ¶ \textcolor{darkblue}{\textbf{\ipa{mv̩˧gv̩˧ | gv̩˧-ze˩}}} \zh{打雷了} \textcolor{Sepia}{\selectlanguage{english}there is a clap of thunder} \textcolor{PineGreen}{\selectlanguage{french}le tonnerre gronde}  
 ¶ \textcolor{darkblue}{\textbf{\ipa{mv̩˧gv̩˧ lɑ˩}}} \zh{打雷了} \textcolor{Sepia}{\selectlanguage{english}there is a clap of thunder} \textcolor{PineGreen}{\selectlanguage{french}il y a un coup de tonnerre}  
 \zh{量词}: \textcolor{darkblue}{\textbf{\ipa{ɭɯ˧}}} 
\lhead{\firstmark}
\rhead{\botmark}

\subsection{\hspace{-0.5cm} {\Large \textcolor{darkblue}{\textbf{\ipa{mv̩˩kʰv̩˧˥}}}}\hspace{0.5cm}[\kern2pt{\textcolor{darkblue}{\textbf{\ipa{mv̩˩kʰv̩˧˥}}}}\kern2pt]} \hypertarget{mv\string_=\string_Bk\string_hv\string_=\string_M\string_T1}{}
\markboth{\textcolor{darkblue}{\textbf{\ipa{mv̩˩kʰv̩˧˥}}}}{}
\textcolor{teal}{\zh{名词}} \hspace{4pt} \zh{声调类:} LM+MH\#.
\zh{晚上。} \textcolor{Sepia}{\selectlanguage{english}Evening (starting when it begins to get dark).} \textcolor{PineGreen}{\selectlanguage{french}Soir, soirée (dès 17h, 18h, quand approche la tombée de la nuit).} 
\lhead{\firstmark}
\rhead{\botmark}

\subsection{\hspace{-0.5cm} {\Large \textcolor{darkblue}{\textbf{\ipa{mv̩˧kʰv̩˧˥}}}}\hspace{0.5cm}[\kern2pt{\textcolor{darkblue}{\textbf{\ipa{mv̩˧kʰv̩˧˥}}}}\kern2pt]} \hypertarget{mv\string_=\string_Mk\string_hv\string_=\string_M\string_T1}{}
\markboth{\textcolor{darkblue}{\textbf{\ipa{mv̩˧kʰv̩˧˥}}}}{}
\textcolor{teal}{\zh{名词}} \hspace{4pt} \zh{声调类:} MH\#.
\zh{烟。} \textcolor{Sepia}{\selectlanguage{english}Smoke.} \textcolor{PineGreen}{\selectlanguage{french}Fumée.}  ¶ \textcolor{darkblue}{\textbf{\ipa{mv̩˧kʰv̩˧ lv̩˥}}} \zh{烟很多} \textcolor{Sepia}{\selectlanguage{english}there is a lot of smoke} \textcolor{PineGreen}{\selectlanguage{french}ça enfume tout le monde}  
 \zh{量词}: \textcolor{darkblue}{\textbf{\ipa{æ̃˩}}} 
\lhead{\firstmark}
\rhead{\botmark}

\subsection{\hspace{-0.5cm} {\Large \textcolor{darkblue}{\textbf{\ipa{mv̩˩ɬi˥}}}}\hspace{0.5cm}[\kern2pt{\textcolor{darkblue}{\textbf{\ipa{mv̩˩ɬi˥}}}}\kern2pt]} \hypertarget{mv\string_=\string_BKi\string_T1}{}
\markboth{\textcolor{darkblue}{\textbf{\ipa{mv̩˩ɬi˥}}}}{}
\textcolor{teal}{\zh{名词}} \hspace{4pt} \zh{声调类:} LH.
\zh{二女儿。} \textcolor{Sepia}{\selectlanguage{english}Second daughter; literally “middle daughter”.} \textcolor{PineGreen}{\selectlanguage{french}Cadette, puinée (fille deuxième née); littéralement: “fille du milieu”.} 
\lhead{\firstmark}
\rhead{\botmark}

\subsection{\hspace{-0.5cm} {\Large \textcolor{darkblue}{\textbf{\ipa{mv̩˧ɭɯ˩}}}}\hspace{0.5cm}[\kern2pt{\textcolor{darkblue}{\textbf{\ipa{mv̩˧ɭɯ˩}}}}\kern2pt]} \hypertarget{mv\string_=\string_Ml\string_RM\string_B1}{}
\markboth{\textcolor{darkblue}{\textbf{\ipa{mv̩˧ɭɯ˩}}}}{}
\textcolor{teal}{\zh{名词}} \hspace{4pt} \zh{声调类:} L\#.
\zh{木里。} \textcolor{Sepia}{\selectlanguage{english}Muli county.} \textcolor{PineGreen}{\selectlanguage{french}Muli (localité dans le Sichuan, proche de Yongning).} 
\lhead{\firstmark}
\rhead{\botmark}

\subsection{\hspace{-0.5cm} {\Large \textcolor{darkblue}{\textbf{\ipa{mv̩˧mi˧}}}}\hspace{0.5cm}[\kern2pt{\textcolor{darkblue}{\textbf{\ipa{mv̩˧mi˧}}}}\kern2pt]} \hypertarget{mv\string_=\string_Mmi\string_M1}{}
\markboth{\textcolor{darkblue}{\textbf{\ipa{mv̩˧mi˧}}}}{}
\textcolor{teal}{\zh{名词}} \hspace{4pt} \zh{声调类:} M.
\zh{女人。} \textcolor{Sepia}{\selectlanguage{english}Woman.} \textcolor{PineGreen}{\selectlanguage{french}Femme.}  ¶ \textcolor{darkblue}{\textbf{\ipa{mv̩˧mi˧ so˩tsʰi˩-kʰv̩˩, | qʰo˧mo˥ gi˩ le˩-ʈɤ˩! | ʝi˧=ɻæ˧ qʰv̩˧tsʰi˧-kʰv̩˩, | bɤ˧di˩ lɑ˩ hṽ˩ ɖʐæ˩!}}} \zh{“女人,到三十岁就算是得拉着的老牛。男人,到六十岁还能在普米山上骑老虎!”这个谚语讲男人与女人老化过程,特别描写相互吸引的程度:三十岁女人算是老了,六十岁男人还认为自己有伟大的威力。女人可以用这个谚语隐蔽地嘲弄一个老男人。} \textcolor{Sepia}{\selectlanguage{english}“A woman of thirty must be pulled along like an old cow; a man of sixty stills rides tigers bareback in the land of the Pumi!” This proverb is about ageing in both sexes, with special emphasis on the appeal that they exert on the opposite sex: at thirty, a woman is old; at sixty, a man is still ready for the greatest exploits. The proverb is reported to be used by women, as an ironic (covertly mocking) comment about an ageing man.} \textcolor{PineGreen}{\selectlanguage{french}“A trente ans, la femme est déjà comme une vieille vache qu'il faut tirer pour qu'elle avance (=à trente ans, une femme, c'est déjà une vieille); à soixante, l'homme chevauche sur une peau [littéralement: des poils] de tigre au pays des Pumi!” (=pour l'homme, soixante ans c'est un âge qui permet encore les exploits) (Dicton au sujet de la façon dont vieillissent les deux sexes, au plan de l'attirance qu'ils exercent sur l'autre sexe; employé par une femme, peut véhiculer une nuance de moquerie à l'égard d'un homme âgé)}  
 \zh{量词}: \textcolor{darkblue}{\textbf{\ipa{v̩˧}}} 
\lhead{\firstmark}
\rhead{\botmark}

\subsection{\hspace{-0.5cm} {\Large \textcolor{darkblue}{\textbf{\ipa{mv̩˧-mv̩˥-di˩}}}}\hspace{0.5cm}[\kern2pt{\textcolor{darkblue}{\textbf{\ipa{xxxx non-correspondance entre le nombre de morphèmes et le nombre de tons de morphèmes}}}}\kern2pt]} \hypertarget{mv\string_=\string_M-mv\string_=\string_T-di\string_B1}{}
\markboth{\textcolor{darkblue}{\textbf{\ipa{mv̩˧-mv̩˥-di˩}}}}{}
\textcolor{teal}{\zh{名词}} \hspace{4pt} \zh{声调类:} H\#-.
\zh{风箱。} \textcolor{Sepia}{\selectlanguage{english}Bellows.} \textcolor{PineGreen}{\selectlanguage{french}Soufflet.}  \zh{量词}: \textcolor{darkblue}{\textbf{\ipa{ɭɯ˧}}} 
\lhead{\firstmark}
\rhead{\botmark}

\subsection{\hspace{-0.5cm} {\Large \textcolor{darkblue}{\textbf{\ipa{mv̩˧ɲi˧}}}}\hspace{0.5cm}[\kern2pt{\textcolor{darkblue}{\textbf{\ipa{mv̩˧ɲi˧}}}}\kern2pt]} \hypertarget{mv\string_=\string_MJi\string_M1}{}
\markboth{\textcolor{darkblue}{\textbf{\ipa{mv̩˧ɲi˧}}}}{}
\textcolor{teal}{\zh{名词}} \hspace{4pt} \zh{声调类:} M.
\zh{脚趾。} \textcolor{Sepia}{\selectlanguage{english}Toe.} \textcolor{PineGreen}{\selectlanguage{french}Orteil.}  \zh{量词}: \textcolor{darkblue}{\textbf{\ipa{ɭɯ˧}}} 
\lhead{\firstmark}
\rhead{\botmark}

\subsection{\hspace{-0.5cm} {\Large \textcolor{darkblue}{\textbf{\ipa{mv̩˩pʰæ˧}}}}\hspace{0.5cm}[\kern2pt{\textcolor{darkblue}{\textbf{\ipa{mv̩˩pʰæ˥}}}}\kern2pt]} \hypertarget{mv\string_=\string_Bp\string_h\{\string_M1}{}
\markboth{\textcolor{darkblue}{\textbf{\ipa{mv̩˩pʰæ˧}}}}{}
\textcolor{teal}{\zh{名词}} \hspace{4pt} \zh{声调类:} LM.
\zh{备料房:煮猪食、煮酒的地方,有时候也在那边准备人的饭。} \textcolor{Sepia}{\selectlanguage{english}Kitchen: the room where pig swill is cooked, where wine is distilled, and where some of the dishes of people are prepared too.} \textcolor{PineGreen}{\selectlanguage{french}Office, cuisine: pièce où on cuisine la pâtée des cochons, où on distille le vin, et où on prépare certains des plats pour les humains. Elle est située dans le même bâtiment que le foyer-salle à manger, à sa droite (vu depuis la cour).}  \zh{量词}: \textcolor{darkblue}{\textbf{\ipa{ɭɯ˧}}} 
\lhead{\firstmark}
\rhead{\botmark}

\subsection{\hspace{-0.5cm} {\Large \textcolor{darkblue}{\textbf{\ipa{mv̩˧qo˩}}}}\hspace{0.5cm}[\kern2pt{\textcolor{darkblue}{\textbf{\ipa{mv̩˧qo˩}}}}\kern2pt]} \hypertarget{mv\string_=\string_Mqo\string_B1}{}
\markboth{\textcolor{darkblue}{\textbf{\ipa{mv̩˧qo˩}}}}{}
\textcolor{teal}{\zh{名词}} \hspace{4pt} \zh{声调类:} L\#.
\zh{木瓜。} \textcolor{Sepia}{\selectlanguage{english}Papaya.} \textcolor{PineGreen}{\selectlanguage{french}Papaye.}  ¶ \textcolor{darkblue}{\textbf{\ipa{mv̩˧qo˩-dʑɯ˩}}} \zh{用木瓜做的一种汁,用法类似于醋。过去,永宁没有醋,醋是从内地(汉族地区)买来的。} \textcolor{Sepia}{\selectlanguage{english}a liquid prepared from the papaya, which served as an equivalent of vinegar (vinegar was introduced late: it was bought in Chinese areas)} \textcolor{PineGreen}{\selectlanguage{french}un liquide préparé à base de papaye, servant d'équivalent de vinaigre (le vinaigre a été introduit tardivement; il était acheté en pays chinois)}  
 \zh{量词}: \textcolor{darkblue}{\textbf{\ipa{ɭɯ˧}}} 
\lhead{\firstmark}
\rhead{\botmark}

\subsection{\hspace{-0.5cm} {\Large \textcolor{darkblue}{\textbf{\ipa{mv̩˧qʰwæ˩}}}}\hspace{0.5cm}[\kern2pt{\textcolor{darkblue}{\textbf{\ipa{mv̩˧qʰwæ˩}}}}\kern2pt]} \hypertarget{mv\string_=\string_Mq\string_hw\{\string_B1}{}
\markboth{\textcolor{darkblue}{\textbf{\ipa{mv̩˧qʰwæ˩}}}}{}
\textcolor{teal}{\zh{名词}} \hspace{4pt} \zh{声调类:} L\#.
\zh{木垮:村落名。} \textcolor{Sepia}{\selectlanguage{english}The name of a village outside the plain of Yongning, close to the Lake.} \textcolor{PineGreen}{\selectlanguage{french}Village na hors de la plaine de Yongning, vers le Lac.}  ¶ \textcolor{darkblue}{\textbf{\ipa{ɬi˧ki˧, | ɲi˧se˩, | tɑ˧dzi˩, | mv̩˧qʰwæ˩, | lɑ˧tʰɑ˧-di˧˥}}} \zh{永宁到泸沽湖所经过的村落,依次是:里格、尼赛、大祖、木垮,然后到拉塔地(拉塔地指的是泸沽湖周边的摩梭地区,包括左所、洛水村等)} \textcolor{Sepia}{\selectlanguage{english}Villages that one passes when moving away from the Yongning plain, towards Lugu lake. These villages do not count as part of Yongning proper. The last, \textcolor{darkblue}{\textbf{\ipa{/lɑ˧tʰɑ˧-di˧˥/}}}, is not a village name like the preceding four: it refers to the entire Na area beyond the fourth village.} \textcolor{PineGreen}{\selectlanguage{french}Villages dans l'ordre, après la plaine de Yongning, ne comptant pas comme faisant partie de Yongning. Le dernier, \textcolor{darkblue}{\textbf{\ipa{/lɑ˧tʰɑ˧-di˧˥/}}}, désigne toute la région na au-delà du quatrième village.}  

\lhead{\firstmark}
\rhead{\botmark}

\subsection{\hspace{-0.5cm} {\Large \textcolor{darkblue}{\textbf{\ipa{mv̩˧ʁo˥\$}}}}\hspace{0.5cm}[\kern2pt{\textcolor{darkblue}{\textbf{\ipa{mv̩˧ʁo˥}}}}\kern2pt]} \hypertarget{mv\string_=\string_MRo\string_T\$1}{}
\markboth{\textcolor{darkblue}{\textbf{\ipa{mv̩˧ʁo˥\$}}}}{}
\textcolor{teal}{\zh{名词}} \hspace{4pt} \zh{声调类:} H\$.
\zh{天空。} \textcolor{Sepia}{\selectlanguage{english}Heavens, sky.} \textcolor{PineGreen}{\selectlanguage{french}Le ciel, les cieux.}  \zh{量词}: \textcolor{darkblue}{\textbf{\ipa{ɭɯ˧}}} 
\lhead{\firstmark}
\rhead{\botmark}

\subsection{\hspace{-0.5cm} {\Large \textcolor{darkblue}{\textbf{\ipa{mv̩˩ʁwɤ˧}}} \textsubscript{1}}\hspace{0.5cm}[\kern2pt{\textcolor{darkblue}{\textbf{\ipa{mv̩˩ʁwɤ˥}}}}\kern2pt]} \hypertarget{mv\string_=\string_BRw7\string_M1}{}
\markboth{\textcolor{darkblue}{\textbf{\ipa{mv̩˩ʁwɤ˧}}} \textsubscript{1}}{}
\textcolor{teal}{\zh{名词}} \hspace{4pt} \zh{声调类:} LM.
\zh{下游。} \textcolor{Sepia}{\selectlanguage{english}Lower reaches of a river; downstream.} \textcolor{PineGreen}{\selectlanguage{french}Cours inférieur, aval.} 
\lhead{\firstmark}
\rhead{\botmark}

\subsection{\hspace{-0.5cm} {\Large \textcolor{darkblue}{\textbf{\ipa{mv̩˩ʁwɤ˧}}} \textsubscript{2}}\hspace{0.5cm}[\kern2pt{\textcolor{darkblue}{\textbf{\ipa{mv̩˩ʁwɤ˥}}}}\kern2pt]} \hypertarget{mv\string_=\string_BRw7\string_M2}{}
\markboth{\textcolor{darkblue}{\textbf{\ipa{mv̩˩ʁwɤ˧}}} \textsubscript{2}}{}
\textcolor{teal}{\zh{名词}} \hspace{4pt} \zh{声调类:} LM.
\zh{下村,比如者波下村(永宁的一个村落)。} \textcolor{Sepia}{\selectlanguage{english}The name of a village.} \textcolor{PineGreen}{\selectlanguage{french}“le village du bas”: nom courant pour désigner un hameau d'un village, ou un village entier, par exemple le hameau du bas du village de Zhubo.} 
\lhead{\firstmark}
\rhead{\botmark}

\subsection{\hspace{-0.5cm} {\Large \textcolor{darkblue}{\textbf{\ipa{mv̩˩si˧˥}}}}\hspace{0.5cm}[\kern2pt{\textcolor{darkblue}{\textbf{\ipa{mv̩˩si˧˥}}}}\kern2pt]} \hypertarget{mv\string_=\string_Bsi\string_M\string_T1}{}
\markboth{\textcolor{darkblue}{\textbf{\ipa{mv̩˩si˧˥}}}}{}
\textcolor{teal}{\zh{名词}} \hspace{4pt} \zh{声调类:} LM+MH\#.
\zh{早晨。} \textcolor{Sepia}{\selectlanguage{english}Morning.} \textcolor{PineGreen}{\selectlanguage{french}Matin.} 
\lhead{\firstmark}
\rhead{\botmark}

\subsection{\hspace{-0.5cm} {\Large \textcolor{darkblue}{\textbf{\ipa{mv̩˩si˧-njɤ˧˥}}}}\hspace{0.5cm}[\kern2pt{\textcolor{darkblue}{\textbf{\ipa{xxxx non-correspondance entre le nombre de morphèmes et le nombre de tons de morphèmes}}}}\kern2pt]} \hypertarget{mv\string_=\string_Bsi\string_M-nj7\string_M\string_T1}{}
\markboth{\textcolor{darkblue}{\textbf{\ipa{mv̩˩si˧-njɤ˧˥}}}}{}
\textcolor{teal}{\zh{助词}} \hspace{4pt} \zh{声调类:} LM+MH\#.
\zh{一大早。} \textcolor{Sepia}{\selectlanguage{english}Early in the morning.} \textcolor{PineGreen}{\selectlanguage{french}Tôt le matin.} 
\lhead{\firstmark}
\rhead{\botmark}

\subsection{\hspace{-0.5cm} {\Large \textcolor{darkblue}{\textbf{\ipa{mv̩˩tɑ\#˥}}}}\hspace{0.5cm}[\kern2pt{\textcolor{darkblue}{\textbf{\ipa{mv̩˩tɑ˥}}}}\kern2pt]} \hypertarget{mv\string_=\string_BtA\#\string_T1}{}
\markboth{\textcolor{darkblue}{\textbf{\ipa{mv̩˩tɑ\#˥}}}}{}
\textcolor{teal}{\zh{动词}} \hspace{4pt} \zh{声调类:} LM+\#H.
\zh{表扬。} \textcolor{Sepia}{\selectlanguage{english}To praise, to commend.} \textcolor{PineGreen}{\selectlanguage{french}Louer, faire l'éloge de.}  ¶ \textcolor{darkblue}{\textbf{\ipa{mv̩˩tɑ˧ ʝi˧}}} \zh{表扬} \textcolor{Sepia}{\selectlanguage{english}to praise} \textcolor{PineGreen}{\selectlanguage{french}louer, faire l'éloge de}  
 ¶ \textcolor{darkblue}{\textbf{\ipa{hĩ˧-ɳɯ˩ | mv̩˩tɑ˥ F | ʝi˧ le˧-hɯ˩-ze˩.}}} \zh{(他做了好事情,于是)人家大大地表扬他了。} \textcolor{Sepia}{\selectlanguage{english}(She/he did some good things, and) people praised him.} \textcolor{PineGreen}{\selectlanguage{french}(Il a fait de bonnes choses, et) les gens l'ont loué/ ont chanté ses louanges}  

\lhead{\firstmark}
\rhead{\botmark}

\subsection{\hspace{-0.5cm} {\Large \textcolor{darkblue}{\textbf{\ipa{mv̩˧ʈʰæ\#˥}}}}\hspace{0.5cm}[\kern2pt{\textcolor{darkblue}{\textbf{\ipa{mv̩˧ʈʰæ˧}}}}\kern2pt]} \hypertarget{mv\string_=\string_Mt`\string_h\{\#\string_T1}{}
\markboth{\textcolor{darkblue}{\textbf{\ipa{mv̩˧ʈʰæ\#˥}}}}{}
\textcolor{teal}{\zh{助词}} \hspace{4pt} \zh{声调类:} \#H.
\zh{下面。} \textcolor{Sepia}{\selectlanguage{english}Under.} \textcolor{PineGreen}{\selectlanguage{french}Dessous, en bas.}  ¶ \textcolor{darkblue}{\textbf{\ipa{ʈʂʰɯ˧ | mv̩˧ʈʰæ˧-lɑ˩ li˩! | gɤ˧bi˧ mɤ˧-li˩!}}} \zh{他老低头是往下看,不往上看!(情景:有人经常脖子疼、头疼,阿妈提出,这应该跟工作姿势不对有关:那个人一直坐在办公桌前,低着头)} \textcolor{Sepia}{\selectlanguage{english}He only looks down, he never glances up! (About someone who constantly sits at his desk, and complains about headaches and a sore neck: the speaker points out that it may be due to a bad posture while at work.)} \textcolor{PineGreen}{\selectlanguage{french}Il regarde tout le temps vers le bas! il ne regarde pas vers le haut! (au sujet d'une personne constamment assise à son bureau, et qui se plaint de mots de tête)}  

\lhead{\firstmark}
\rhead{\botmark}

\subsection{\hspace{-0.5cm} {\Large \textcolor{darkblue}{\textbf{\ipa{mv̩˩tɕi˥}}}}\hspace{0.5cm}[\kern2pt{\textcolor{darkblue}{\textbf{\ipa{mv̩˩tɕi˥}}}}\kern2pt]} \hypertarget{mv\string_=\string_Bts£i\string_T1}{}
\markboth{\textcolor{darkblue}{\textbf{\ipa{mv̩˩tɕi˥}}}}{}
\textcolor{teal}{\zh{名词}} \hspace{4pt} \zh{声调类:} LH.
\zh{最小的女儿。} \textcolor{Sepia}{\selectlanguage{english}Youngest daughter.} \textcolor{PineGreen}{\selectlanguage{french}Benjamine, plus jeune fille.} 
\lhead{\firstmark}
\rhead{\botmark}

\subsection{\hspace{-0.5cm} {\Large \textcolor{darkblue}{\textbf{\ipa{mv̩˩tɕo˧}}}}\hspace{0.5cm}[\kern2pt{\textcolor{darkblue}{\textbf{\ipa{mv̩˩tɕo˥}}}}\kern2pt]} \hypertarget{mv\string_=\string_Bts£o\string_M1}{}
\markboth{\textcolor{darkblue}{\textbf{\ipa{mv̩˩tɕo˧}}}}{}
\textcolor{teal}{\zh{助词}} \hspace{4pt} \zh{声调类:} LM.
\zh{往下。} \textcolor{Sepia}{\selectlanguage{english}Downward.} \textcolor{PineGreen}{\selectlanguage{french}Vers le bas.}  ¶ \textcolor{darkblue}{\textbf{\ipa{mv̩˩tɕo˧ kwɤ˩}}} \zh{往下扔} \textcolor{Sepia}{\selectlanguage{english}to throw down} \textcolor{PineGreen}{\selectlanguage{french}jeter vers le bas}  
 ¶ \textcolor{darkblue}{\textbf{\ipa{mv̩˩tɕo˧ se˧!}}} \zh{下去!(命令狗从主屋的地板下去:狗不准来上面)} \textcolor{Sepia}{\selectlanguage{english}Get down! Go down! (Command to the dog if it climbs onto the floorboard of the house, contrary to the rule)} \textcolor{PineGreen}{\selectlanguage{french}Descends! (Ce qu'on dit au chien qui monte sur la partie haute de la cuisine, contrevenant à la règle)}  

\lhead{\firstmark}
\rhead{\botmark}

\subsection{\hspace{-0.5cm} {\Large \textcolor{darkblue}{\textbf{\ipa{mv̩˩tʰi˩}}}}\hspace{0.5cm}[\kern2pt{\textcolor{darkblue}{\textbf{\ipa{mv̩˩tʰi˩˥}}}}\kern2pt]} \hypertarget{mv\string_=\string_Bt\string_hi\string_B1}{}
\markboth{\textcolor{darkblue}{\textbf{\ipa{mv̩˩tʰi˩}}}}{}
\textcolor{teal}{\zh{形容词}} \hspace{4pt} \zh{声调类:} L.
\zh{聪明。} \textcolor{Sepia}{\selectlanguage{english}Intelligent.} \textcolor{PineGreen}{\selectlanguage{french}Intelligente (d'une femme).}  ¶ \textcolor{darkblue}{\textbf{\ipa{ʈʂʰɯ˧ | mv̩˩tʰi˩˥ | ʐwæ˩˥!}}} \zh{她很聪明!} \textcolor{Sepia}{\selectlanguage{english}She is very intelligent!} \textcolor{PineGreen}{\selectlanguage{french}elle est très intelligente!}  

\lhead{\firstmark}
\rhead{\botmark}

\subsection{\hspace{-0.5cm} {\Large \textcolor{darkblue}{\textbf{\ipa{mv̩˩ʈʂæ˧˥}}}}\hspace{0.5cm}[\kern2pt{\textcolor{darkblue}{\textbf{\ipa{mv̩˩ʈʂæ˧˥}}}}\kern2pt]} \hypertarget{mv\string_=\string_Bt`s`\{\string_M\string_T1}{}
\markboth{\textcolor{darkblue}{\textbf{\ipa{mv̩˩ʈʂæ˧˥}}}}{}
\textcolor{teal}{\zh{名词}} \hspace{4pt} \zh{声调类:} LM+MH\#.
\zh{下半(身)。} \textcolor{Sepia}{\selectlanguage{english}Lower part (of the body=below the waist).} \textcolor{PineGreen}{\selectlanguage{french}Bas du corps, partie inférieure du corps.} 
\lhead{\firstmark}
\rhead{\botmark}

\subsection{\hspace{-0.5cm} {\Large \textcolor{darkblue}{\textbf{\ipa{mv̩˧ʈʂæ˧˥}}}}\hspace{0.5cm}[\kern2pt{\textcolor{darkblue}{\textbf{\ipa{mv̩˧ʈʂæ˧˥}}}}\kern2pt]} \hypertarget{mv\string_=\string_Mt`s`\{\string_M\string_T1}{}
\markboth{\textcolor{darkblue}{\textbf{\ipa{mv̩˧ʈʂæ˧˥}}}}{}
\textcolor{teal}{\zh{动词}} \hspace{4pt} \zh{声调类:} MH\#.
\zh{叫做、称作、名叫。} \textcolor{Sepia}{\selectlanguage{english}To call, to give the name... to, to refer to... as...} \textcolor{PineGreen}{\selectlanguage{french}S'appeler, avoir... pour nom.}  ¶ \textcolor{darkblue}{\textbf{\ipa{(ʈʂʰɯ˧ | ) ə˧tso˧ mv̩˧ʈʂæ˧˥?}}} \zh{他叫什么名字?} \textcolor{Sepia}{\selectlanguage{english}What's her/his name? / What is (it/he/she) called?} \textcolor{PineGreen}{\selectlanguage{french}comment il s'appelle? Quel est son nom?}  
 ¶ \textcolor{darkblue}{\textbf{\ipa{njɤ˧ | ... mv̩˧ʈʂæ˧˥}}} \zh{我名字叫……} \textcolor{Sepia}{\selectlanguage{english}My name is...} \textcolor{PineGreen}{\selectlanguage{french}Je m'appelle…}  

\lhead{\firstmark}
\rhead{\botmark}

\subsection{\hspace{-0.5cm} {\Large \textcolor{darkblue}{\textbf{\ipa{mv̩˧tsʰi\#˥}}}}\hspace{0.5cm}[\kern2pt{\textcolor{darkblue}{\textbf{\ipa{mv̩˧tsʰi˧}}}}\kern2pt]} \hypertarget{mv\string_=\string_Mts\string_hi\#\string_T1}{}
\markboth{\textcolor{darkblue}{\textbf{\ipa{mv̩˧tsʰi\#˥}}}}{}
\textcolor{teal}{\zh{名词}} \hspace{4pt} \zh{声调类:} \#H.
\zh{旱季(冬天与春天:农历九月到二月)。} \textcolor{Sepia}{\selectlanguage{english}Dry season (winter and spring: from the 9th lunar month to the 2nd lunar month).} \textcolor{PineGreen}{\selectlanguage{french}Saison sèche (hiver et printemps; du 9e mois au 2e mois du calendrier lunaire compris).}  ¶ \textcolor{darkblue}{\textbf{\ipa{mv̩˧tsʰi˧-qo˩}}} \zh{旱季的时候} \textcolor{Sepia}{\selectlanguage{english}during the dry season} \textcolor{PineGreen}{\selectlanguage{french}durant la saison sèche}  

\lhead{\firstmark}
\rhead{\botmark}

\subsection{\hspace{-0.5cm} {\Large \textcolor{darkblue}{\textbf{\ipa{mv̩˩tsʰo˩}}}}\hspace{0.5cm}[\kern2pt{\textcolor{darkblue}{\textbf{\ipa{mv̩˩tsʰo˩˥}}}}\kern2pt]} \hypertarget{mv\string_=\string_Bts\string_ho\string_B1}{}
\markboth{\textcolor{darkblue}{\textbf{\ipa{mv̩˩tsʰo˩}}}}{}
\textcolor{teal}{\zh{名词}} \hspace{4pt} \zh{声调类:} L.
\zh{含很多树脂的木头,用来引火。} \textcolor{Sepia}{\selectlanguage{english}Firewood full of resin, used to start a fire.} \textcolor{PineGreen}{\selectlanguage{french}Bois gorgé de résine, pour faire partir le feu (pièces de la taille d'une bûche, qu'on débite en petits morceaux pour faire partir le feu).}  \zh{量词}: \textcolor{darkblue}{\textbf{\ipa{kɤ˧˥}}} 
\lhead{\firstmark}
\rhead{\botmark}

\subsection{\hspace{-0.5cm} {\Large \textcolor{darkblue}{\textbf{\ipa{mv̩˧ʈʂʰɤ˩}}}}\hspace{0.5cm}[\kern2pt{\textcolor{darkblue}{\textbf{\ipa{mv̩˧ʈʂʰɤ˩}}}}\kern2pt]} \hypertarget{mv\string_=\string_Mt`s`\string_h7\string_B1}{}
\markboth{\textcolor{darkblue}{\textbf{\ipa{mv̩˧ʈʂʰɤ˩}}}}{}
\textcolor{teal}{\zh{名词}} \hspace{4pt} \zh{声调类:} L\#.
\zh{下巴。} \textcolor{Sepia}{\selectlanguage{english}Chin.} \textcolor{PineGreen}{\selectlanguage{french}Menton.}  \zh{量词}: \textcolor{darkblue}{\textbf{\ipa{kʰwɤ˥}}} 
\lhead{\firstmark}
\rhead{\botmark}

\subsection{\hspace{-0.5cm} {\Large \textcolor{darkblue}{\textbf{\ipa{mv̩˧tsɯ˧˥}}}}\hspace{0.5cm}[\kern2pt{\textcolor{darkblue}{\textbf{\ipa{mv̩˧tsɯ˧˥}}}}\kern2pt]} \hypertarget{mv\string_=\string_MtsM\string_M\string_T1}{}
\markboth{\textcolor{darkblue}{\textbf{\ipa{mv̩˧tsɯ˧˥}}}}{}
\textcolor{teal}{\zh{名词}} \hspace{4pt} \zh{声调类:} MH\#.
\zh{胡子。} \textcolor{Sepia}{\selectlanguage{english}Beard.} \textcolor{PineGreen}{\selectlanguage{french}Barbe.}  ¶ \textcolor{darkblue}{\textbf{\ipa{mv̩˧tsɯ˧ ʑi˥}}} \zh{有胡子} \textcolor{Sepia}{\selectlanguage{english}to have a beard} \textcolor{PineGreen}{\selectlanguage{french}avoir de la barbe}  
 \zh{量词}: \textcolor{darkblue}{\textbf{\ipa{kʰwɤ˥}}} 
\lhead{\firstmark}
\rhead{\botmark}

\subsection{\hspace{-0.5cm} {\Large \textcolor{darkblue}{\textbf{\ipa{mv̩˧ʈʂv̩˩}}}}\hspace{0.5cm}[\kern2pt{\textcolor{darkblue}{\textbf{\ipa{mv̩˧ʈʂv̩˩}}}}\kern2pt]} \hypertarget{mv\string_=\string_Mt`s`v\string_=\string_B1}{}
\markboth{\textcolor{darkblue}{\textbf{\ipa{mv̩˧ʈʂv̩˩}}}}{}
\textcolor{teal}{\zh{名词}} \hspace{4pt} \zh{声调类:} L\#.
\zh{臼。} \textcolor{Sepia}{\selectlanguage{english}Mortar.} \textcolor{PineGreen}{\selectlanguage{french}Mortier.}  \zh{量词}: \textcolor{darkblue}{\textbf{\ipa{nɑ˧}}} 
\lhead{\firstmark}
\rhead{\botmark}

\subsection{\hspace{-0.5cm} {\Large \textcolor{darkblue}{\textbf{\ipa{mv̩˧ʈʂv̩˥}}} \textsubscript{1}}\hspace{0.5cm}[\kern2pt{\textcolor{darkblue}{\textbf{\ipa{mv̩˧ʈʂv̩˥}}}}\kern2pt]} \hypertarget{mv\string_=\string_Mt`s`v\string_=\string_T1}{}
\markboth{\textcolor{darkblue}{\textbf{\ipa{mv̩˧ʈʂv̩˥}}} \textsubscript{1}}{}
\textcolor{teal}{\zh{形容词}} \hspace{4pt} \zh{声调类:} H\#.
\ding{202} \zh{皱(衣服)。} \textcolor{Sepia}{\selectlanguage{english}Creased.} \textcolor{PineGreen}{\selectlanguage{french}Plissé, froissé.} \ding{203} \zh{(脸)有皱纹。} \textcolor{Sepia}{\selectlanguage{english}Wrinkled.} \textcolor{PineGreen}{\selectlanguage{french}Ridé.}  ¶ \textcolor{darkblue}{\textbf{\ipa{to˧kɤ˧ | mv̩˧ʈʂv̩˥ ze˩.}}} \zh{他的前额有了皱纹。} \textcolor{Sepia}{\selectlanguage{english}(His/her) forehead became wrinkled.} \textcolor{PineGreen}{\selectlanguage{french}(Son) front s'est ridé / son front a pris des rides.}  
 ¶ \textcolor{darkblue}{\textbf{\ipa{to˧kɤ˧ | le˧-mv̩˧ʈʂv̩˥}}} \zh{他的前额有皱纹。} \textcolor{Sepia}{\selectlanguage{english}(His/her) forehead is wrinkled.} \textcolor{PineGreen}{\selectlanguage{french}(Son) front est ridé.}  
 ¶ \textcolor{darkblue}{\textbf{\ipa{[F5] æ˩ʂe˩˥ | le˧-mv̩˧ʈʂv̩˥}}} \zh{皮肤有皱纹(直译:“肉有皱纹”)} \textcolor{Sepia}{\selectlanguage{english}The skin is wrinkled (literally “the flesh is wrinkled”)} \textcolor{PineGreen}{\selectlanguage{french}La peau est ridée (littéralement: “la chair est ridée”)}  
\ding{204} \zh{谢(花谢了)。} \textcolor{Sepia}{\selectlanguage{english}Withered.} \textcolor{PineGreen}{\selectlanguage{french}Fané.}  ¶ \textcolor{darkblue}{\textbf{\ipa{bæ˩bæ˩˥ | le˧-mv̩˧ʈʂv̩˥-ze˩}}} \zh{花谢了。} \textcolor{Sepia}{\selectlanguage{english}The flower has withered.} \textcolor{PineGreen}{\selectlanguage{french}La fleur s'est fanée.}  
\zh{~【参考】~} \hyperlink{}{\textcolor{darkblue}{\textbf{\ipa{mv̩˧ʈʂv̩˥}}} \textsubscript{2}} 
\lhead{\firstmark}
\rhead{\botmark}

\subsection{\hspace{-0.5cm} {\Large \textcolor{darkblue}{\textbf{\ipa{mv̩˧ʈʂv̩˥}}} \textsubscript{2}}\hspace{0.5cm}[\kern2pt{\textcolor{darkblue}{\textbf{\ipa{mv̩˧ʈʂv̩˥}}}}\kern2pt]} \hypertarget{mv\string_=\string_Mt`s`v\string_=\string_T2}{}
\markboth{\textcolor{darkblue}{\textbf{\ipa{mv̩˧ʈʂv̩˥}}} \textsubscript{2}}{}
\textcolor{teal}{\zh{名词}} \hspace{4pt} \zh{声调类:} H\#.
\zh{皱纹。} \textcolor{Sepia}{\selectlanguage{english}Wrinkle.} \textcolor{PineGreen}{\selectlanguage{french}Rides.}  \zh{量词}: \textcolor{darkblue}{\textbf{\ipa{kʰɯ˩}}} \zh{~【参考】~} \hyperlink{}{\textcolor{darkblue}{\textbf{\ipa{mv̩˧ʈʂv̩˥}}} \textsubscript{1}} 
\lhead{\firstmark}
\rhead{\botmark}

\subsection{\hspace{-0.5cm} {\Large \textcolor{darkblue}{\textbf{\ipa{mv̩˧ʈʂv̩˩-nv̩˩mi˩}}}}\hspace{0.5cm}[\kern2pt{\textcolor{darkblue}{\textbf{\ipa{xxxx non-correspondance entre le nombre de morphèmes et le nombre de tons de morphèmes}}}}\kern2pt]} \hypertarget{mv\string_=\string_Mt`s`v\string_=\string_B-nv\string_=\string_Bmi\string_B1}{}
\markboth{\textcolor{darkblue}{\textbf{\ipa{mv̩˧ʈʂv̩˩-nv̩˩mi˩}}}}{}
\textcolor{teal}{\zh{名词}} \hspace{4pt} \zh{声调类:} L\#-.
\zh{杵。} \textcolor{Sepia}{\selectlanguage{english}Small pestle.} \textcolor{PineGreen}{\selectlanguage{french}Petit pilon.}  \zh{量词}: \textcolor{darkblue}{\textbf{\ipa{nɑ˧}}} 
\lhead{\firstmark}
\rhead{\botmark}

\subsection{\hspace{-0.5cm} {\Large \textcolor{darkblue}{\textbf{\ipa{mv̩˧ʈʰɯ˧}}}}\hspace{0.5cm}[\kern2pt{\textcolor{darkblue}{\textbf{\ipa{xxxx non-correspondance entre le nombre de morphèmes et le nombre de tons de morphèmes}}}}\kern2pt]} \hypertarget{mv\string_=\string_Mt`\string_hM\string_M1}{}
\markboth{\textcolor{darkblue}{\textbf{\ipa{mv̩˧ʈʰɯ˧}}}}{}
\textcolor{teal}{\zh{名词}} \hspace{4pt} \zh{声调类:} M.
\zh{脚跟。} \textcolor{Sepia}{\selectlanguage{english}Heel.} \textcolor{PineGreen}{\selectlanguage{french}Talon.}  \zh{量词}: \textcolor{darkblue}{\textbf{\ipa{kʰwɤ˥}}} 
\lhead{\firstmark}
\rhead{\botmark}

\subsection{\hspace{-0.5cm} {\Large \textcolor{darkblue}{\textbf{\ipa{mv̩˩tv̩˩}}}}\hspace{0.5cm}[\kern2pt{\textcolor{darkblue}{\textbf{\ipa{mv̩˧tv̩˧}}}}\kern2pt]} \hypertarget{mv\string_=\string_Btv\string_=\string_B1}{}
\markboth{\textcolor{darkblue}{\textbf{\ipa{mv̩˩tv̩˩}}}}{}
\textcolor{teal}{\zh{名词}} \hspace{4pt} \zh{声调类:} L.
\zh{独生女。} \textcolor{Sepia}{\selectlanguage{english}Only daughter.} \textcolor{PineGreen}{\selectlanguage{french}Fille unique.}  ¶ \textcolor{darkblue}{\textbf{\ipa{mv̩˩tv̩˩˥ | ɖɯ˧-v̩˧-lɑ˧ dʑo˧˥!}}} \zh{她只有一个独生女!} \textcolor{Sepia}{\selectlanguage{english}(She) just has an only daughter!} \textcolor{PineGreen}{\selectlanguage{french}(elle) n'a qu'une fille unique!}  

\lhead{\firstmark}
\rhead{\botmark}

\subsection{\hspace{-0.5cm} {\Large \textcolor{darkblue}{\textbf{\ipa{mv̩˧tʰv̩˧˥}}}}\hspace{0.5cm}[\kern2pt{\textcolor{darkblue}{\textbf{\ipa{mv̩˩tʰv̩˩˥}}}}\kern2pt]} \hypertarget{mv\string_=\string_Mt\string_hv\string_=\string_M\string_T1}{}
\markboth{\textcolor{darkblue}{\textbf{\ipa{mv̩˧tʰv̩˧˥}}}}{}
\textcolor{teal}{\zh{名词}} \hspace{4pt} \zh{声调类:} MH\#.
\zh{火把。} \textcolor{Sepia}{\selectlanguage{english}Torch.} \textcolor{PineGreen}{\selectlanguage{french}Torche.}  \zh{量词}: \textcolor{darkblue}{\textbf{\ipa{qɑ˩}}} 
\lhead{\firstmark}
\rhead{\botmark}

\subsection{\hspace{-0.5cm} {\Large \textcolor{darkblue}{\textbf{\ipa{mv̩˧ʐe˧˥}}} \textsubscript{1}}\hspace{0.5cm}[\kern2pt{\textcolor{darkblue}{\textbf{\ipa{mv̩˧ʐe˧˥}}}}\kern2pt]} \hypertarget{mv\string_=\string_Mz`e\string_M\string_T1}{}
\markboth{\textcolor{darkblue}{\textbf{\ipa{mv̩˧ʐe˧˥}}} \textsubscript{1}}{}
\textcolor{teal}{\zh{名词}} \hspace{4pt} \zh{声调类:} MH\#.
\zh{雨季(夏天与秋天:三月份至八月份)。} \textcolor{Sepia}{\selectlanguage{english}Rainy season (summer and autumn: from the 3rd to the 8th month of the lunar calendar).} \textcolor{PineGreen}{\selectlanguage{french}Saison des pluies (été et automne: du 3e au 8e mois du calendrier lunaire).}  ¶ \textcolor{darkblue}{\textbf{\ipa{mv̩˧ʐe˧-qo˥}}} \zh{雨季的时候} \textcolor{Sepia}{\selectlanguage{english}during the rainy season} \textcolor{PineGreen}{\selectlanguage{french}pendant la saison des pluies}  

\lhead{\firstmark}
\rhead{\botmark}

\subsection{\hspace{-0.5cm} {\Large \textcolor{darkblue}{\textbf{\ipa{mv̩˧ʐe˧-ʈʂʰæ˧ɣɯ\#˥}}}}\hspace{0.5cm}[\kern2pt{\textcolor{darkblue}{\textbf{\ipa{xxxx non-correspondance entre le nombre de morphèmes et le nombre de tons de morphèmes}}}}\kern2pt]} \hypertarget{mv\string_=\string_Mz`e\string_M-t`s`\string_h\{\string_MGM\#\string_T1}{}
\markboth{\textcolor{darkblue}{\textbf{\ipa{mv̩˧ʐe˧-ʈʂʰæ˧ɣɯ\#˥}}}}{}
\textcolor{teal}{\zh{名词}} \hspace{4pt} \zh{声调类:} \#H.
\zh{火药。} \textcolor{Sepia}{\selectlanguage{english}Gunpowder.} \textcolor{PineGreen}{\selectlanguage{french}Poudre à canon.}  \zh{量词}: \textcolor{darkblue}{\textbf{\ipa{po˩}}} 
\lhead{\firstmark}
\rhead{\botmark}

\subsection{\hspace{-0.5cm} {\Large \textcolor{darkblue}{\textbf{\ipa{mv̩˧ʐe\#˥}}} \textsubscript{2}}\hspace{0.5cm}[\kern2pt{\textcolor{darkblue}{\textbf{\ipa{mv̩˧ʐe˧}}}}\kern2pt]} \hypertarget{mv\string_=\string_Mz`e\#\string_T2}{}
\markboth{\textcolor{darkblue}{\textbf{\ipa{mv̩˧ʐe\#˥}}} \textsubscript{2}}{}
\textcolor{teal}{\zh{名词}} \hspace{4pt} \zh{声调类:} \#H.
\zh{枪,明火枪。} \textcolor{Sepia}{\selectlanguage{english}Gun; firelock.} \textcolor{PineGreen}{\selectlanguage{french}Arme à feu, fusil; arquebuse.}  \zh{量词}: \textcolor{darkblue}{\textbf{\ipa{kʰɯ˩}}} 
\lhead{\firstmark}
\rhead{\botmark}

\subsection{\hspace{-0.5cm} {\Large \textcolor{darkblue}{\textbf{\ipa{mv̩˧ʑi˩}}}}\hspace{0.5cm}[\kern2pt{\textcolor{darkblue}{\textbf{\ipa{mv̩˧ʑi˧}}}}\kern2pt]} \hypertarget{mv\string_=\string_Mz£i\string_B1}{}
\markboth{\textcolor{darkblue}{\textbf{\ipa{mv̩˧ʑi˩}}}}{}
\textcolor{teal}{\zh{名词}} \hspace{4pt} \zh{声调类:} L.
\zh{消息、闲话、八卦。} \textcolor{Sepia}{\selectlanguage{english}News, gossip.} \textcolor{PineGreen}{\selectlanguage{french}Nouvelle, ragot, information, histoire.}  ¶ \textcolor{darkblue}{\textbf{\ipa{mv̩˧ʑi˩ | ɖɯ˧-kʰwɤ˥}}} \zh{一个八卦} \textcolor{Sepia}{\selectlanguage{english}a piece of gossip} \textcolor{PineGreen}{\selectlanguage{french}une nouvelle, un ragot, une information}  
 \zh{量词}: \textcolor{darkblue}{\textbf{\ipa{kʰwɤ˥}}} 
\lhead{\firstmark}
\rhead{\botmark}

\subsection{\hspace{-0.5cm} {\Large \textcolor{darkblue}{\textbf{\ipa{mv̩˩zo˩}}}}\hspace{0.5cm}[\kern2pt{\textcolor{darkblue}{\textbf{\ipa{mv̩˩zo˩˥}}}}\kern2pt]} \hypertarget{mv\string_=\string_Bzo\string_B1}{}
\markboth{\textcolor{darkblue}{\textbf{\ipa{mv̩˩zo˩}}}}{}
\textcolor{teal}{\zh{名词}} \hspace{4pt} \zh{声调类:} L.
\zh{姑娘。} \textcolor{Sepia}{\selectlanguage{english}Young lady.} \textcolor{PineGreen}{\selectlanguage{french}Jeune fille.}  ¶ \textcolor{darkblue}{\textbf{\ipa{mv̩˩zo˩=ɻæ˧}}} \zh{姑娘们} \textcolor{Sepia}{\selectlanguage{english}young ladies} \textcolor{PineGreen}{\selectlanguage{french}les jeunes filles}  
 \zh{量词}: \textcolor{darkblue}{\textbf{\ipa{ɭɯ˧}}} \textcolor{darkblue}{\textbf{\ipa{v̩˧}}} 
\lhead{\firstmark}
\rhead{\botmark}

\subsection{\hspace{-0.5cm} {\Large \textcolor{darkblue}{\textbf{\ipa{mv̩˩ʐɤ˩}}}}\hspace{0.5cm}[\kern2pt{\textcolor{darkblue}{\textbf{\ipa{mv̩˩ʐɤ˩˥}}}}\kern2pt]} \hypertarget{mv\string_=\string_Bz`7\string_B1}{}
\markboth{\textcolor{darkblue}{\textbf{\ipa{mv̩˩ʐɤ˩}}}}{}
\textcolor{teal}{\zh{名词}} \hspace{4pt} \zh{声调类:} L.
\zh{义女。} \textcolor{Sepia}{\selectlanguage{english}Adopted daughter.} \textcolor{PineGreen}{\selectlanguage{french}Fille adoptive.} 
\lhead{\firstmark}
\rhead{\botmark}

\subsection{\hspace{-0.5cm} {\Large \textcolor{darkblue}{\textbf{\ipa{mv̩˩zo˩-ə˩mi˥}}}}\hspace{0.5cm}[\kern2pt{\textcolor{darkblue}{\textbf{\ipa{xxxx non-correspondance entre le nombre de morphèmes et le nombre de tons de morphèmes}}}}\kern2pt]} \hypertarget{mv\string_=\string_Bzo\string_B-@\string_Bmi\string_T1}{}
\markboth{\textcolor{darkblue}{\textbf{\ipa{mv̩˩zo˩-ə˩mi˥}}}}{}
\textcolor{teal}{\zh{名词}} \hspace{4pt} \zh{声调类:} L+H\#.
\zh{姑娘与母亲。} \textcolor{Sepia}{\selectlanguage{english}A young lady and her mother.} \textcolor{PineGreen}{\selectlanguage{french}(une) jeune fille et (sa) mère.} 
\lhead{\firstmark}
\rhead{\botmark}

\subsection{\hspace{-0.5cm} {\Large \textcolor{darkblue}{\textbf{\ipa{mv̩˩zɯ˩}}} \textsubscript{1}}\hspace{0.5cm}[\kern2pt{\textcolor{darkblue}{\textbf{\ipa{mv̩˩zɯ˥}}}}\kern2pt]} \hypertarget{mv\string_=\string_BzM\string_B1}{}
\markboth{\textcolor{darkblue}{\textbf{\ipa{mv̩˩zɯ˩}}} \textsubscript{1}}{}
\textcolor{teal}{\zh{名词}} \hspace{4pt} \zh{声调类:} L.
\zh{兄弟(哥哥们与弟弟们)。} \textcolor{Sepia}{\selectlanguage{english}Brothers.} \textcolor{PineGreen}{\selectlanguage{french}Frères (aînés ou cadets).}  ¶ \textcolor{darkblue}{\textbf{\ipa{ʈʂʰɯ˧ | nɑ˧dʑi˧-bv̩˧ | mv̩˩zɯ˩-ʝi˥-hĩ˩ ɲi˩!}}} \zh{他是\textcolor{darkblue}{\textbf{\ipa{nɑ˧dʑi˧/}}}的兄弟!} \textcolor{Sepia}{\selectlanguage{english}He is \textcolor{darkblue}{\textbf{\ipa{nɑ˧dʑi˧/'s}}} brother!} \textcolor{PineGreen}{\selectlanguage{french}il est frère de \textcolor{darkblue}{\textbf{\ipa{nɑ˧dʑi˧/!}}}}  
 \zh{量词}: \textcolor{darkblue}{\textbf{\ipa{v̩˧}}} 
\lhead{\firstmark}
\rhead{\botmark}

\subsection{\hspace{-0.5cm} {\Large \textcolor{darkblue}{\textbf{\ipa{mv̩˩zɯ˩}}} \textsubscript{2}}\hspace{0.5cm}[\kern2pt{\textcolor{darkblue}{\textbf{\ipa{mv̩˩zɯ˩˥}}}}\kern2pt]} \hypertarget{mv\string_=\string_BzM\string_B2}{}
\markboth{\textcolor{darkblue}{\textbf{\ipa{mv̩˩zɯ˩}}} \textsubscript{2}}{}
\textcolor{teal}{\zh{名词}} \hspace{4pt} \zh{声调类:} L.
\zh{燕麦。} \textcolor{Sepia}{\selectlanguage{english}Oats.} \textcolor{PineGreen}{\selectlanguage{french}Avoine.}  \zh{量词}: \textcolor{darkblue}{\textbf{\ipa{kɤ˧˥}}} 
\lhead{\firstmark}
\rhead{\botmark}

\subsection{\hspace{-0.5cm} {\Large \textcolor{darkblue}{\textbf{\ipa{mv̩˩zɯ˩-ni˥mi˩}}}}\hspace{0.5cm}[\kern2pt{\textcolor{darkblue}{\textbf{\ipa{xxxx non-correspondance entre le nombre de morphèmes et le nombre de tons de morphèmes}}}}\kern2pt]} \hypertarget{mv\string_=\string_BzM\string_B-ni\string_Tmi\string_B1}{}
\markboth{\textcolor{darkblue}{\textbf{\ipa{mv̩˩zɯ˩-ni˥mi˩}}}}{}
\textcolor{teal}{\zh{名词}} \hspace{4pt} \zh{声调类:} L+\#H-.
\zh{兄弟姐妹,堂兄弟姐妹。} \textcolor{Sepia}{\selectlanguage{english}Brothers and sisters, siblings.} \textcolor{PineGreen}{\selectlanguage{french}Frères et sœurs (tous les frères et sœurs; s'applique aussi aux cousins).} 
\lhead{\firstmark}
\rhead{\botmark}

\subsection{\hspace{-0.5cm} {\Large \textcolor{darkblue}{\textbf{\ipa{mv̩˧‑}}}}\hspace{0.5cm}[\kern2pt{\textcolor{darkblue}{\textbf{\ipa{xxxx non-correspondance entre le nombre de morphèmes et le nombre de tons de morphèmes}}}}\kern2pt]} \hypertarget{mv\string_=\string_M‑1}{}
\markboth{\textcolor{darkblue}{\textbf{\ipa{mv̩˧‑}}}}{}
\textcolor{teal}{\zh{前缀}} \hspace{4pt} \zh{声调类:} M.
\zh{即将、快要、马上会、立即。} \textcolor{Sepia}{\selectlanguage{english}Aspect/mood: the event is about to take place: the event is imminent.} \textcolor{PineGreen}{\selectlanguage{french}Aspect/mode: l'événement est imminent: sur le point de se produire.}  ¶ \textcolor{darkblue}{\textbf{\ipa{ʈʂʰɯ˧ | mv̩˧-dzɯ˧-kwɤ˩-tɕɯ˩!}}} \zh{你吃完吧!} \textcolor{Sepia}{\selectlanguage{english}Come on, eat it up! / Come on, finish your bowl!} \textcolor{PineGreen}{\selectlanguage{french}Mange-le donc! / Finis donc ça! (Contexte: à table, quelqu'un ne finit pas son bol; sa mère ou grand-mère lui enjoint de finir, pour ne pas gaspiller de nourriture.)}  
 ¶ \textcolor{darkblue}{\textbf{\ipa{tʰi˧-mv̩˧-dzɯ˧-kwɤ˩-tɕɯ˩!}}} \zh{同上} \textcolor{Sepia}{\selectlanguage{english}Same as previous example, with the \mytextsc{durative}} \textcolor{PineGreen}{\selectlanguage{french}Comme l'exemple précédent, avec le \mytextsc{duratif}}  
 ¶ \textcolor{darkblue}{\textbf{\ipa{[M18] ʈʂʰɯ˧ mv̩˧-ʂɯ˧ bi˩-ni˩gv̩˩! njɤ˧ | gv̩˩dʑɯ˩˥ | ʐwæ˩˥! |}}} \zh{他要死了!我很伤心!} \textcolor{Sepia}{\selectlanguage{english}(S)he is going to die! I am devastated!} \textcolor{PineGreen}{\selectlanguage{french}Il/elle va mourir! Je suis au désespoir!}  
 ¶ \textcolor{darkblue}{\textbf{\ipa{hĩ˧ ʈʂʰɯ˧-v̩˧ tʰv̩˧ mv̩˧-ʂɯ˧-kwɤ˧tɕɯ˥-lɑ˩...}}} \zh{因为这个人快要去世……} \textcolor{Sepia}{\selectlanguage{english}as this person is going to die soon...} \textcolor{PineGreen}{\selectlanguage{french}du fait que cette personne va mourir très bientôt...}  
 ¶ \textcolor{darkblue}{\textbf{\ipa{mv̩˧-dzɯ˧-bi˩-ze˩!}}} \zh{马上要吃了!} \textcolor{Sepia}{\selectlanguage{english}[We are] about to eat! / We are going to eat right now!} \textcolor{PineGreen}{\selectlanguage{french}[On] va manger tout de suite!}  
 ¶ \textcolor{darkblue}{\textbf{\ipa{mv̩˧-hwæ˧}}} \zh{即将买} \textcolor{Sepia}{\selectlanguage{english}about to buy} \textcolor{PineGreen}{\selectlanguage{french}sur le point d'acheter}  
 ¶ \textcolor{darkblue}{\textbf{\ipa{mv̩˧-tɕʰi˧}}} \zh{即将卖} \textcolor{Sepia}{\selectlanguage{english}about to sell} \textcolor{PineGreen}{\selectlanguage{french}sur le point de vendre}  
 ¶ \textcolor{darkblue}{\textbf{\ipa{mv̩˧-dzɯ˧-kwɤ˧tɕɯ˥-lɑ˩...}}} \zh{因为马上要吃……} \textcolor{Sepia}{\selectlanguage{english}since (she/he) is about to eat...} \textcolor{PineGreen}{\selectlanguage{french}puisqu'elle/il est sur le point de manger...}  
 ¶ \textcolor{darkblue}{\textbf{\ipa{mv̩˧-lɑ˩-kwɤ˩tɕɯ˩-lɑ˩...}}} \zh{因为要打……} \textcolor{Sepia}{\selectlanguage{english}since (she/he) is about to strike...} \textcolor{PineGreen}{\selectlanguage{french}puisqu'elle/il est sur le point de frapper...}  

\lhead{\firstmark}
\rhead{\botmark}

\subsection{\hspace{-0.5cm} {\Large \textcolor{darkblue}{\textbf{\ipa{mv̩˧\textasciitilde{}mv̩\#˥}}}}\hspace{0.5cm}[\kern2pt{\textcolor{darkblue}{\textbf{\ipa{mv̩˧mv̩˧}}}}\kern2pt]} \hypertarget{mv\string_=\string_M~mv\string_=\#\string_T1}{}
\markboth{\textcolor{darkblue}{\textbf{\ipa{mv̩˧\textasciitilde{}mv̩\#˥}}}}{}
\textcolor{teal}{\zh{形容词}} \hspace{4pt} \zh{声调类:} .
\zh{清楚(话、事情)。} \textcolor{Sepia}{\selectlanguage{english}Clear (speech).} \textcolor{PineGreen}{\selectlanguage{french}Clair (parole, événement…).}  ¶ \textcolor{darkblue}{\textbf{\ipa{ʐwɤ˧ mv̩˧\textasciitilde{}mv̩˧}}} \zh{讲清楚} \textcolor{Sepia}{\selectlanguage{english}to speak clearly; clear speech} \textcolor{PineGreen}{\selectlanguage{french}parler clairement}  
 ¶ \textcolor{darkblue}{\textbf{\ipa{le˧-mv̩˧\textasciitilde{}mv̩˧-kʰɯ˩}}} \zh{弄明白、讲清楚} \textcolor{Sepia}{\selectlanguage{english}to clarify, to explain} \textcolor{PineGreen}{\selectlanguage{french}éclaircir, tirer au clair, expliquer}  

\lhead{\firstmark}
\rhead{\botmark}

\newpage
\section*{\centering- \textcolor{darkblue}{\textbf{\ipa{n}}} -}
\subsection{\hspace{-0.5cm} {\Large \textcolor{darkblue}{\textbf{\ipa{nɑ˥}}}}\hspace{0.5cm}[\kern2pt{\textcolor{darkblue}{\textbf{\ipa{nɑ˥}}}}\kern2pt]} \hypertarget{nA\string_T1}{}
\markboth{\textcolor{darkblue}{\textbf{\ipa{nɑ˥}}}}{}
\textcolor{teal}{\zh{形容词}} \hspace{4pt} \zh{声调类:} H.
\zh{严重,重要。} \textcolor{Sepia}{\selectlanguage{english}Important, serious (e.g. a wound).} \textcolor{PineGreen}{\selectlanguage{french}Grave, sérieux (ex.: une blessure).}  ¶ \textcolor{darkblue}{\textbf{\ipa{mɤ˧-nɑ˥}}} \zh{不严重} \textcolor{Sepia}{\selectlanguage{english}not serious} \textcolor{PineGreen}{\selectlanguage{french}bénin, pas grave, sans conséquence (ex.: une écorchure)}  

\lhead{\firstmark}
\rhead{\botmark}

\subsection{\hspace{-0.5cm} {\Large \textcolor{darkblue}{\textbf{\ipa{nɑ˧\textsubscript{a}}}}}\hspace{0.5cm}[\kern2pt{\textcolor{darkblue}{\textbf{\ipa{nɑ˥}}}}\kern2pt]} \hypertarget{nA\string_Ma1}{}
\markboth{\textcolor{darkblue}{\textbf{\ipa{nɑ˧\textsubscript{a}}}}}{}
\textcolor{teal}{\zh{量词}} \hspace{4pt} \zh{声调类:} M\textsubscript{a}.
\zh{量词:工具(一把)。} \textcolor{Sepia}{\selectlanguage{english}Classifier for tools.} \textcolor{PineGreen}{\selectlanguage{french}Classificateur des outils.}  ¶ \textcolor{darkblue}{\textbf{\ipa{ɖɯ˧-nɑ˧ dʑo˧}}} \zh{有一把(工具)} \textcolor{Sepia}{\selectlanguage{english}there is one (tool)} \textcolor{PineGreen}{\selectlanguage{french}il y en a un; il y a un outil}  

\lhead{\firstmark}
\rhead{\botmark}

\subsection{\hspace{-0.5cm} {\Large \textcolor{darkblue}{\textbf{\ipa{nɑ˧dʑi\#˥}}}}\hspace{0.5cm}[\kern2pt{\textcolor{darkblue}{\textbf{\ipa{nɑ˩dʑi˥}}}}\kern2pt]} \hypertarget{nA\string_Mdz£i\#\string_T1}{}
\markboth{\textcolor{darkblue}{\textbf{\ipa{nɑ˧dʑi\#˥}}}}{}
\textcolor{teal}{\zh{名词}} \hspace{4pt} \zh{声调类:} \#H.
\zh{女性名字。} \textcolor{Sepia}{\selectlanguage{english}Feminine given name.} \textcolor{PineGreen}{\selectlanguage{french}Prénom féminin.} 
\lhead{\firstmark}
\rhead{\botmark}

\subsection{\hspace{-0.5cm} {\Large \textcolor{darkblue}{\textbf{\ipa{nɑ˧mi\#˥}}}}\hspace{0.5cm}[\kern2pt{\textcolor{darkblue}{\textbf{\ipa{nɑ˩mi˥}}}}\kern2pt]} \hypertarget{nA\string_Mmi\#\string_T1}{}
\markboth{\textcolor{darkblue}{\textbf{\ipa{nɑ˧mi\#˥}}}}{}
\textcolor{teal}{\zh{名词}} \hspace{4pt} \zh{声调类:} \#H.
\zh{受累、劳累、辛苦、困难、艰难、艰苦。} \textcolor{Sepia}{\selectlanguage{english}Difficulties, complications, hardship, overwork, great fatigue.} \textcolor{PineGreen}{\selectlanguage{french}Épuisement, misère, difficultés.}  ¶ \textcolor{darkblue}{\textbf{\ipa{nɑ˧mi˧ tʰv̩˧!}}} \zh{现在是艰苦的时候! / 现在很贫困!} \textcolor{Sepia}{\selectlanguage{english}Hardship has come!} \textcolor{PineGreen}{\selectlanguage{french}Misère! / Des difficultés surviennent, on rencontre des difficultés; on est dans une période difficile}  
 \zh{量词}: \textcolor{darkblue}{\textbf{\ipa{kʰwɤ˥}}} 
\lhead{\firstmark}
\rhead{\botmark}

\subsection{\hspace{-0.5cm} {\Large \textcolor{darkblue}{\textbf{\ipa{nɑ˩\textsubscript{b}}}}}\hspace{0.5cm}[\kern2pt{\textcolor{darkblue}{\textbf{\ipa{nɑ˥}}}}\kern2pt]} \hypertarget{nA\string_Bb1}{}
\markboth{\textcolor{darkblue}{\textbf{\ipa{nɑ˩\textsubscript{b}}}}}{}
\textcolor{teal}{\zh{形容词}} \hspace{4pt} \zh{声调类:} L\textsubscript{b}.
\zh{黑,暗(颜色,天色)。} \textcolor{Sepia}{\selectlanguage{english}Black.} \textcolor{PineGreen}{\selectlanguage{french}Noir, sombre.}  ¶ \textcolor{darkblue}{\textbf{\ipa{nɑ˩-hĩ˥}}} \zh{黑的} \textcolor{Sepia}{\selectlanguage{english}\mytextsc{rel}} \textcolor{PineGreen}{\selectlanguage{french}\mytextsc{rel}}  
 ¶ \textcolor{darkblue}{\textbf{\ipa{mɤ˧-nɑ˩}}} \zh{不黑} \textcolor{Sepia}{\selectlanguage{english}\mytextsc{neg}} \textcolor{PineGreen}{\selectlanguage{french}\mytextsc{neg}}  

\lhead{\firstmark}
\rhead{\botmark}

\subsection{\hspace{-0.5cm} {\Large \textcolor{darkblue}{\textbf{\ipa{nɑ˩bɑ˧-ʁɑ˧ɭɯ\#˥}}}}\hspace{0.5cm}[\kern2pt{\textcolor{darkblue}{\textbf{\ipa{xxxx non-correspondance entre le nombre de morphèmes et le nombre de tons de morphèmes}}}}\kern2pt]} \hypertarget{nA\string_BbA\string_M-RA\string_Ml\string_RM\#\string_T1}{}
\markboth{\textcolor{darkblue}{\textbf{\ipa{nɑ˩bɑ˧-ʁɑ˧ɭɯ\#˥}}}}{}
\textcolor{teal}{\zh{名词}} \hspace{4pt} \zh{声调类:} LM+\#H.
\zh{一座山的名字。} \textcolor{Sepia}{\selectlanguage{english}Name of a mountain.} \textcolor{PineGreen}{\selectlanguage{french}Nom d'une montagne de Yongning.} 
\lhead{\firstmark}
\rhead{\botmark}

\subsection{\hspace{-0.5cm} {\Large \textcolor{darkblue}{\textbf{\ipa{nɑ˩dzi˧}}}}\hspace{0.5cm}[\kern2pt{\textcolor{darkblue}{\textbf{\ipa{nɑ˩dzi˥}}}}\kern2pt]} \hypertarget{nA\string_Bdzi\string_M1}{}
\markboth{\textcolor{darkblue}{\textbf{\ipa{nɑ˩dzi˧}}}}{}
\textcolor{teal}{\zh{助词}} \hspace{4pt} \zh{声调类:} LM.
\zh{暗(黄昏/暮的时候,天变暗)。} \textcolor{Sepia}{\selectlanguage{english}Dark (at twilight, dusk).} \textcolor{PineGreen}{\selectlanguage{french}Sombre (au crépuscule, il se met à faire sombre).}  ¶ \textcolor{darkblue}{\textbf{\ipa{nɑ˩dzi˧-ze˩!}}} \zh{天变暗了! / 黄昏到了!} \textcolor{Sepia}{\selectlanguage{english}It has got dark! Twilight has come!} \textcolor{PineGreen}{\selectlanguage{french}le crépuscule est venu! / c'est le crépuscule!}  
 ¶ \textcolor{darkblue}{\textbf{\ipa{nɑ˩dzi˧-ho˩-ze˩!}}} \zh{(天)要变暗了!} \textcolor{Sepia}{\selectlanguage{english}It's going to get dark!} \textcolor{PineGreen}{\selectlanguage{french}Il va faire sombre! La nuit va commencer à tomber!}  

\lhead{\firstmark}
\rhead{\botmark}

\subsection{\hspace{-0.5cm} {\Large \textcolor{darkblue}{\textbf{\ipa{nɑ˩hĩ\#˥}}}}\hspace{0.5cm}[\kern2pt{\textcolor{darkblue}{\textbf{\ipa{nɑ˧hĩ˧}}}}\kern2pt]} \hypertarget{nA\string_Bhi\string_~\#\string_T1}{}
\markboth{\textcolor{darkblue}{\textbf{\ipa{nɑ˩hĩ\#˥}}}}{}
\textcolor{teal}{\zh{名词}} \hspace{4pt} \zh{声调类:} LM+\#H.
\zh{纳西族。} \textcolor{Sepia}{\selectlanguage{english}Naxi (ethnic group).} \textcolor{PineGreen}{\selectlanguage{french}Naxi (groupe ethnique).}  ¶ \textcolor{darkblue}{\textbf{\ipa{nɑ˩hĩ˧-mi˧ ɲi˥!}}} \zh{她是纳西族女人!} \textcolor{Sepia}{\selectlanguage{english}She is a Naxi women! / It's a Naxi woman!} \textcolor{PineGreen}{\selectlanguage{french}c'est une femme naxi!}  
 ¶ \textcolor{darkblue}{\textbf{\ipa{nɑ˩hĩ˧-bɑ˧lɑ˥}}} \zh{纳西族服装} \textcolor{Sepia}{\selectlanguage{english}the Naxi costume, Naxi garments} \textcolor{PineGreen}{\selectlanguage{french}vêtements naxi, costume naxi}  
 ¶ \textcolor{darkblue}{\textbf{\ipa{nɑ˩hĩ˧-ʐwɤ˧ so˥}}} \zh{学纳西语} \textcolor{Sepia}{\selectlanguage{english}to study the Naxi language} \textcolor{PineGreen}{\selectlanguage{french}apprendre la langue naxi}  
 ¶ \textcolor{darkblue}{\textbf{\ipa{nɑ˩hĩ˧-tʰæ˧ɻæ˥}}} \zh{纳西族的书} \textcolor{Sepia}{\selectlanguage{english}Naxi books} \textcolor{PineGreen}{\selectlanguage{french}livres naxi}  
 \zh{量词}: \textcolor{darkblue}{\textbf{\ipa{v̩˧}}} 
\lhead{\firstmark}
\rhead{\botmark}

\subsection{\hspace{-0.5cm} {\Large \textcolor{darkblue}{\textbf{\ipa{nɑ˩kwɤ˧}}}}\hspace{0.5cm}[\kern2pt{\textcolor{darkblue}{\textbf{\ipa{nɑ˧kwɤ˩}}}}\kern2pt]} \hypertarget{nA\string_Bkw7\string_M1}{}
\markboth{\textcolor{darkblue}{\textbf{\ipa{nɑ˩kwɤ˧}}}}{}
\textcolor{teal}{\zh{名词}} \hspace{4pt} \zh{声调类:} LM.
\zh{南瓜。} \textcolor{Sepia}{\selectlanguage{english}Pumpkin; cushaw.} \textcolor{PineGreen}{\selectlanguage{french}Potiron.}  \zh{量词}: \textcolor{darkblue}{\textbf{\ipa{ɭɯ˧}}} 
\lhead{\firstmark}
\rhead{\botmark}

\subsection{\hspace{-0.5cm} {\Large \textcolor{darkblue}{\textbf{\ipa{nɑ˩mi\#˥}}}}\hspace{0.5cm}[\kern2pt{\textcolor{darkblue}{\textbf{\ipa{nɑ˩mi˥}}}}\kern2pt]} \hypertarget{nA\string_Bmi\#\string_T1}{}
\markboth{\textcolor{darkblue}{\textbf{\ipa{nɑ˩mi\#˥}}}}{}
\textcolor{teal}{\zh{名词}} \hspace{4pt} \zh{声调类:} LM+\#H.
\zh{摩梭女人。} \textcolor{Sepia}{\selectlanguage{english}Na woman.} \textcolor{PineGreen}{\selectlanguage{french}Une femme Na.} 
\lhead{\firstmark}
\rhead{\botmark}

\subsection{\hspace{-0.5cm} {\Large \textcolor{darkblue}{\textbf{\ipa{nɑ˩mv̩˥-nɑ˩dzi˩dzi˩}}}}\hspace{0.5cm}[\kern2pt{\textcolor{darkblue}{\textbf{\ipa{xxxx non-correspondance entre le nombre de morphèmes et le nombre de tons de morphèmes}}}}\kern2pt]} \hypertarget{nA\string_Bmv\string_=\string_T-nA\string_Bdzi\string_Bdzi\string_B1}{}
\markboth{\textcolor{darkblue}{\textbf{\ipa{nɑ˩mv̩˥-nɑ˩dzi˩dzi˩}}}}{}
\textcolor{teal}{\zh{形容词}} \hspace{4pt} \zh{声调类:} .
\zh{很暗(天变得很暗)。} \textcolor{Sepia}{\selectlanguage{english}All dark, quite dark (at twilight, dusk).} \textcolor{PineGreen}{\selectlanguage{french}Tout sombre, tout noir (il fait nuit noire).} 
\lhead{\firstmark}
\rhead{\botmark}

\subsection{\hspace{-0.5cm} {\Large \textcolor{darkblue}{\textbf{\ipa{nɑ˩pv̩˧-qʰwɤ˧}}}}\hspace{0.5cm}[\kern2pt{\textcolor{darkblue}{\textbf{\ipa{nɑ˩pv̩˧qʰwɤ˧}}}}\kern2pt]} \hypertarget{nA\string_Bpv\string_=\string_M-q\string_hw7\string_M1}{}
\markboth{\textcolor{darkblue}{\textbf{\ipa{nɑ˩pv̩˧-qʰwɤ˧}}}}{}
\textcolor{teal}{\zh{名词}} \hspace{4pt} \zh{声调类:} LM-.
\zh{皇帝。} \textcolor{Sepia}{\selectlanguage{english}Emperor (borrowed from the Mongolian?).} \textcolor{PineGreen}{\selectlanguage{french}Empereur (emprunt au mongole?).}  ¶ \textcolor{darkblue}{\textbf{\ipa{ʈʂʰɯ˧ | nɑ˩pʰv̩˧-qʰwɤ˧-ni˩gv̩˩!}}} \zh{他摆出做皇帝的样子! / 他以为他是皇帝吧!(嘲笑一个自以为是的人)} \textcolor{Sepia}{\selectlanguage{english}He's got an empereror's looks! / He thinks he's the emperor! (Mocking someone who thinks he or she can impose his/her decisions to everyone, who thinks (s)he is a great leader.)} \textcolor{PineGreen}{\selectlanguage{french}Il vous prend des airs d'empereur! / Il se prend pour l'empereur! (Façon de se moquer d'un personnage qui veut en imposer à tous, qui se prend pour un grand chef.)}  

\lhead{\firstmark}
\rhead{\botmark}

\subsection{\hspace{-0.5cm} {\Large \textcolor{darkblue}{\textbf{\ipa{nɑ˩tsʰi˩}}}}\hspace{0.5cm}[\kern2pt{\textcolor{darkblue}{\textbf{\ipa{nɑ˩tsʰi˩˥}}}}\kern2pt]} \hypertarget{nA\string_Bts\string_hi\string_B1}{}
\markboth{\textcolor{darkblue}{\textbf{\ipa{nɑ˩tsʰi˩}}}}{}
\textcolor{teal}{\zh{名词}} \hspace{4pt} \zh{声调类:} L.
\zh{一座山的名字。} \textcolor{Sepia}{\selectlanguage{english}Name of a mountain.} \textcolor{PineGreen}{\selectlanguage{french}Nom d'une montagne de Yongning.}  ¶ \textcolor{darkblue}{\textbf{\ipa{kɤ˧mv̩˧˥, | æ˧ʂæ˧, | ŋwɤ˧hɑ̃˩, | ʂwæ˧gv̩\#˥, | nɑ˩tsʰi˩˥ | -tɕʰɤ˧pɤ˧mi\#˥, | qv̩˧ɻ̍˧-ʈʂʰɑ˧nɑ˥ |}}} \zh{永宁地区有固定名字的六座山。其它的山,因为没有重要的象征意义,因此没有取名。} \textcolor{Sepia}{\selectlanguage{english}The six mountains of Yongning that carry a name and have a definite symbolic value. The other mountains do not have comparable symbolic value, and fewer people use specific names for them.} \textcolor{PineGreen}{\selectlanguage{french}Les six montagnes de Yongning qui portent un nom. Les autres sommets du voisinage n'ont pas une valeur symbolique comparable, et ne portent pas de nom communément utilisé.}  

\lhead{\firstmark}
\rhead{\botmark}

\subsection{\hspace{-0.5cm} {\Large \textcolor{darkblue}{\textbf{\ipa{nɑ˩zo\#˥}}}}\hspace{0.5cm}[\kern2pt{\textcolor{darkblue}{\textbf{\ipa{nɑ˩zo˥}}}}\kern2pt]} \hypertarget{nA\string_Bzo\#\string_T1}{}
\markboth{\textcolor{darkblue}{\textbf{\ipa{nɑ˩zo\#˥}}}}{}
\textcolor{teal}{\zh{名词}} \hspace{4pt} \zh{声调类:} LM+\#H.
\zh{摩梭男人。} \textcolor{Sepia}{\selectlanguage{english}Na man.} \textcolor{PineGreen}{\selectlanguage{french}Un homme Na.} 
\lhead{\firstmark}
\rhead{\botmark}

\subsection{\hspace{-0.5cm} {\Large \textcolor{darkblue}{\textbf{\ipa{nɑ˩-ʐwɤ˥}}}}\hspace{0.5cm}[\kern2pt{\textcolor{darkblue}{\textbf{\ipa{xxxx non-correspondance entre le nombre de morphèmes et le nombre de tons de morphèmes}}}}\kern2pt]} \hypertarget{nA\string_B-z`w7\string_T1}{}
\markboth{\textcolor{darkblue}{\textbf{\ipa{nɑ˩-ʐwɤ˥}}}}{}
\textcolor{teal}{\zh{名词}} \hspace{4pt} \zh{声调类:} LH.
\zh{本语言:摩梭话(纳语)。} \textcolor{Sepia}{\selectlanguage{english}Autonym of the language: the Na language.} \textcolor{PineGreen}{\selectlanguage{french}Langue na: endonyme de la langue na.} 
\lhead{\firstmark}
\rhead{\botmark}

\subsection{\hspace{-0.5cm} {\Large \textcolor{darkblue}{\textbf{\ipa{nɑ˧˥}}}}\hspace{0.5cm}[\kern2pt{\textcolor{darkblue}{\textbf{\ipa{nɑ˧˥}}}}\kern2pt]} \hypertarget{nA\string_M\string_T1}{}
\markboth{\textcolor{darkblue}{\textbf{\ipa{nɑ˧˥}}}}{}
\textcolor{teal}{\zh{动词}} \hspace{4pt} \zh{声调类:} MH.
\zh{发抖,颤抖。} \textcolor{Sepia}{\selectlanguage{english}To tremble.} \textcolor{PineGreen}{\selectlanguage{french}Trembler.}  ¶ \textcolor{darkblue}{\textbf{\ipa{nɑ˩\textasciitilde{}nɑ˧-ze˥}}} \zh{发抖了} \textcolor{Sepia}{\selectlanguage{english}\mytextsc{red} \mytextsc{pfv}} \textcolor{PineGreen}{\selectlanguage{french}\mytextsc{red} \mytextsc{pfv}}  
 ¶ \textcolor{darkblue}{\textbf{\ipa{le˧-nɑ˩\textasciitilde{}nɑ˩}}} \zh{\mytextsc{accomp} \mytextsc{red}} \textcolor{Sepia}{\selectlanguage{english}\mytextsc{accomp} \mytextsc{red}} \textcolor{PineGreen}{\selectlanguage{french}\mytextsc{accomp} \mytextsc{red}}  
 ¶ \textcolor{darkblue}{\textbf{\ipa{lo˩qʰwɤ˥ | nɑ˩\textasciitilde{}nɑ˧˥}}} \zh{手抖} \textcolor{Sepia}{\selectlanguage{english}the hand trembles} \textcolor{PineGreen}{\selectlanguage{french}la main tremble}  

\lhead{\firstmark}
\rhead{\botmark}

\subsection{\hspace{-0.5cm} {\Large \textcolor{darkblue}{\textbf{\ipa{nɑ˩˧}}}}\hspace{0.5cm}[\kern2pt{\textcolor{darkblue}{\textbf{\ipa{nɑ˩˥}}}}\kern2pt]} \hypertarget{nA\string_B\string_M1}{}
\markboth{\textcolor{darkblue}{\textbf{\ipa{nɑ˩˧}}}}{}
\textcolor{teal}{\zh{名词}} \hspace{4pt} \zh{声调类:} LM.
\zh{自称:摩梭族。} \textcolor{Sepia}{\selectlanguage{english}Endonym: Na.} \textcolor{PineGreen}{\selectlanguage{french}Endonyme: les Na.}  ¶ \textcolor{darkblue}{\textbf{\ipa{nɑ˩-mv̩˧ nɑ˥-di˩ |}}} \zh{摩梭人地区} \textcolor{Sepia}{\selectlanguage{english}Na territory} \textcolor{PineGreen}{\selectlanguage{french}le territoire des Na}  
 ¶ \textcolor{darkblue}{\textbf{\ipa{ə˧ʝi˧-ʂɯ˥ʝi˩, | nɑ˩zo˧-tɑ˥mv̩˩-ɳɯ˩ | dʑo˧-ɲi˥-tsɯ˩!}}} \zh{过去,摩梭人的传统(里)有(关于这些问题的说法)嘛!} \textcolor{Sepia}{\selectlanguage{english}Na traditions used to mention this! / There used to be Na traditions about this! (Context: when reference is made to local customs, to explain what is allowed and what is not.)} \textcolor{PineGreen}{\selectlanguage{french}Autrefois, notre tradition, elle en parlait ! / Notre tradition, elle en parle! (Contexte: quand on fait référence à la coutume locale: ce qu'il est interdit de faire, ce qu'on est autorisé à faire…)}  
 \zh{量词}: \textcolor{darkblue}{\textbf{\ipa{v̩˧}}} 
\lhead{\firstmark}
\rhead{\botmark}

\subsection{\hspace{-0.5cm} {\Large \textcolor{darkblue}{\textbf{\ipa{ni˥}}}}\hspace{0.5cm}[\kern2pt{\textcolor{darkblue}{\textbf{\ipa{ni˥}}}}\kern2pt]} \hypertarget{ni\string_T1}{}
\markboth{\textcolor{darkblue}{\textbf{\ipa{ni˥}}}}{}
\textcolor{teal}{\zh{名词}} \hspace{4pt} \zh{声调类:} \#H.
\zh{苋米。} \textcolor{Sepia}{\selectlanguage{english}Amaranth.} \textcolor{PineGreen}{\selectlanguage{french}Amaranthe, \textit{Amaranthus}: minuscule graine qui n'est pas une céréale mais a une valeur nutritionnelle comparable aux céréales.}  \zh{量词}: \textcolor{darkblue}{\textbf{\ipa{po˧}}} 
\lhead{\firstmark}
\rhead{\botmark}

\subsection{\hspace{-0.5cm} {\Large \textcolor{darkblue}{\textbf{\ipa{ni˧fv̩˥}}}}\hspace{0.5cm}[\kern2pt{\textcolor{darkblue}{\textbf{\ipa{ni˧fv̩˥}}}}\kern2pt]} \hypertarget{ni\string_Mfv\string_=\string_T1}{}
\markboth{\textcolor{darkblue}{\textbf{\ipa{ni˧fv̩˥}}}}{}
\textcolor{teal}{\zh{名词}} \hspace{4pt} \zh{声调类:} H\#.
\zh{大包:用来包装物品的皮包(马帮用的),或者来装尸体的麻布包(为了在火葬前暂时存放尸体)。} \textcolor{Sepia}{\selectlanguage{english}A very large bag, either made of leather (to carry products over long distances by caravan) or of linen (to wrap up a corpse for temporary inhumation).} \textcolor{PineGreen}{\selectlanguage{french}Très grand sac/très grande poche; en cuir, pour emballer les produits que l'on transportait sur de longues distances; s'emploie aussi pour désigner le sac de toile dans lequel on place le corps d'un défunt pendant son inhumation provisoire.}  ¶ \textcolor{darkblue}{\textbf{\ipa{jɤ˧ŋɤ˧-ni˧fv̩˥}}} \zh{成都大包。(据说这类的包一般是成都地区生产的。)} \textcolor{Sepia}{\selectlanguage{english}Chengdu bag (note: this kind of large, solid bag was often purchased in the area of Chengdu, hence their association with this place name.)} \textcolor{PineGreen}{\selectlanguage{french}grand sac de Chengdu. Expression employée car les sacs de ce type provenaient généralement de la région de Chengdu.}  
 \zh{量词}: \textcolor{darkblue}{\textbf{\ipa{ɭɯ˧}}} 
\lhead{\firstmark}
\rhead{\botmark}

\subsection{\hspace{-0.5cm} {\Large \textcolor{darkblue}{\textbf{\ipa{‑ni˧gv̩˧˥}}}}\hspace{0.5cm}[\kern2pt{\textcolor{darkblue}{\textbf{\ipa{ni˧gv̩˧˥}}}}\kern2pt]} \hypertarget{‑ni\string_Mgv\string_=\string_M\string_T1}{}
\markboth{\textcolor{darkblue}{\textbf{\ipa{‑ni˧gv̩˧˥}}}}{}
\textcolor{teal}{\zh{助词}} \hspace{4pt} \zh{声调类:} MH\#.
\zh{如、像。} \textcolor{Sepia}{\selectlanguage{english}To be like, to seem like.} \textcolor{PineGreen}{\selectlanguage{french}Comme (être comme, être semblable à).}  ¶ \textcolor{darkblue}{\textbf{\ipa{zɯ˧hṽ˩-ni˩gv̩˩}}} \zh{像草,等于绿色} \textcolor{Sepia}{\selectlanguage{english}like grass, i.e. vivid green} \textcolor{PineGreen}{\selectlanguage{french}comme de l’herbe, c'est-à-dire vert}  
 ¶ \textcolor{darkblue}{\textbf{\ipa{æ̃˧qæ˩-ni˩gv̩˩}}} \zh{像鹦鹉,等于青色} \textcolor{Sepia}{\selectlanguage{english}like a parrot (i.e. blue/green-coloured)} \textcolor{PineGreen}{\selectlanguage{french}bleu-vert; littéralement “comme un perroquet”, c'est-à-dire “couleur perroquet”}  
 ¶ \textcolor{darkblue}{\textbf{\ipa{lwæ˩pʰv̩˩-ni˥gv̩˩}}} \zh{灰色} \textcolor{Sepia}{\selectlanguage{english}like ashes, i.e. grey-coloured} \textcolor{PineGreen}{\selectlanguage{french}comme de la cendre = de couleur grise}  
 ¶ \textcolor{darkblue}{\textbf{\ipa{sɯ˧pv̩˩-ni˩gv̩˩}}} \zh{像膀胱} \textcolor{Sepia}{\selectlanguage{english}like a urinary bladder, in the shape of a bladder} \textcolor{PineGreen}{\selectlanguage{french}comme une vessie, en forme de vessie}  
 ¶ \textcolor{darkblue}{\textbf{\ipa{(nv̩˩mi˩˥ | ) ɖɯ˧-v̩˧-ni˩gv̩˩}}} \zh{一条心,想得一致} \textcolor{Sepia}{\selectlanguage{english}(their heart is) like one, as one} \textcolor{PineGreen}{\selectlanguage{french}comme un seul cœur, (leur) cœur est à l'unisson}  
 ¶ \textcolor{darkblue}{\textbf{\ipa{dzi˩bi˩-ni˩gv̩˩˥}}} \zh{习惯(一个环境)} \textcolor{Sepia}{\selectlanguage{english}to be accustomed to, to get accustomed to} \textcolor{PineGreen}{\selectlanguage{french}s’habituer, s'accoutumer, prendre ses habitudes (dans un environnement)}  
 ¶ \textcolor{darkblue}{\textbf{\ipa{[élicitation lors de la transcription de Agriculture.70] li˩ ʈʰɯ˩-bi˩-ni˩-gv̩˩˥}}} \zh{习惯喝茶、有喝茶的习惯} \textcolor{Sepia}{\selectlanguage{english}to have a habit of drinking tea, to be a tea-drinker} \textcolor{PineGreen}{\selectlanguage{french}avoir l'habitude de boire du thé, être un buveur de thé}  
 ¶ \textcolor{darkblue}{\textbf{\ipa{ʈʂʰɯ˧ | ʂɯ˧-bi˧-ni˩gv̩˩!}}} \zh{他好像要死了!} \textcolor{Sepia}{\selectlanguage{english}It looks like it's going to die! / It looks as if it were going to die!} \textcolor{PineGreen}{\selectlanguage{french}On dirait qu'il/elle va mourir!}  

\lhead{\firstmark}
\rhead{\botmark}

\subsection{\hspace{-0.5cm} {\Large \textcolor{darkblue}{\textbf{\ipa{ni˧mi\#˥}}}}\hspace{0.5cm}[\kern2pt{\textcolor{darkblue}{\textbf{\ipa{ni˧mi˧}}}}\kern2pt]} \hypertarget{ni\string_Mmi\#\string_T1}{}
\markboth{\textcolor{darkblue}{\textbf{\ipa{ni˧mi\#˥}}}}{}
\textcolor{teal}{\zh{名词}} \hspace{4pt} \zh{声调类:} \#H.
\zh{姐妹。} \textcolor{Sepia}{\selectlanguage{english}Sisters.} \textcolor{PineGreen}{\selectlanguage{french}Soeurs (aînées ou cadettes).}  ¶ \textcolor{darkblue}{\textbf{\ipa{ʈʂʰɯ˧ | ʈæ˧ʂɯ˧-bv̩˧ | ni˧mi˧ ɲi˥.}}} \zh{她是达石的姐姐(或妹妹)} \textcolor{Sepia}{\selectlanguage{english}She is \textcolor{darkblue}{\textbf{\ipa{/ʈæ˧ʂɯ˧/’s}}} sister.} \textcolor{PineGreen}{\selectlanguage{french}Elle est soeur de \textcolor{darkblue}{\textbf{\ipa{/ʈæ˧ʂɯ˧/}}}.}  
 \zh{量词}: \textcolor{darkblue}{\textbf{\ipa{v̩˧}}} 
\lhead{\firstmark}
\rhead{\botmark}

\subsection{\hspace{-0.5cm} {\Large \textcolor{darkblue}{\textbf{\ipa{njæ˥-qv̩˩}}}}\hspace{0.5cm}[\kern2pt{\textcolor{darkblue}{\textbf{\ipa{njæ˥qv̩˩}}}}\kern2pt]} \hypertarget{nj\{\string_T-qv\string_=\string_B1}{}
\markboth{\textcolor{darkblue}{\textbf{\ipa{njæ˥-qv̩˩}}}}{}
\textcolor{teal}{\zh{动词}} \hspace{4pt} \zh{声调类:} H\#-.
\zh{看别的方向(蔑视态度)。} \textcolor{Sepia}{\selectlanguage{english}To look away from.} \textcolor{PineGreen}{\selectlanguage{french}Détourner le regard, détourner la tête, se détourner.}  ¶ \textcolor{darkblue}{\textbf{\ipa{hĩ˧ njæ˧qv̩˥}}} \zh{看别的方向,不直接看(蔑视态度)} \textcolor{Sepia}{\selectlanguage{english}to turn away the head from, to look away from (someone that one despises, hates...)} \textcolor{PineGreen}{\selectlanguage{french}se détourner de quelqu'un, détourner la tête face à quelqu'un (que l'on méprise, déteste...)}  
 ¶ \textcolor{darkblue}{\textbf{\ipa{mɤ˧-njæ˥qv̩˩}}} \zh{\mytextsc{neg}} \textcolor{Sepia}{\selectlanguage{english}\mytextsc{neg}} \textcolor{PineGreen}{\selectlanguage{french}\mytextsc{neg}: ne pas détourner le regard (face à quelqu'un)}  

\lhead{\firstmark}
\rhead{\botmark}

\subsection{\hspace{-0.5cm} {\Large \textcolor{darkblue}{\textbf{\ipa{njæ˧bæ˥}}}}\hspace{0.5cm}[\kern2pt{\textcolor{darkblue}{\textbf{\ipa{njæ˧bæ˥}}}}\kern2pt]} \hypertarget{nj\{\string_Mb\{\string_T1}{}
\markboth{\textcolor{darkblue}{\textbf{\ipa{njæ˧bæ˥}}}}{}
\textcolor{teal}{\zh{名词}} \hspace{4pt} \zh{声调类:} H\#.
\zh{眼泪。} \textcolor{Sepia}{\selectlanguage{english}Tear.} \textcolor{PineGreen}{\selectlanguage{french}Larme.}  \zh{量词}: \textcolor{darkblue}{\textbf{\ipa{ʈʰɤ˥}}} 
\lhead{\firstmark}
\rhead{\botmark}

\subsection{\hspace{-0.5cm} {\Large \textcolor{darkblue}{\textbf{\ipa{njæ˧tsɯ˩}}}}\hspace{0.5cm}[\kern2pt{\textcolor{darkblue}{\textbf{\ipa{njæ˧tsɯ˩}}}}\kern2pt]} \hypertarget{nj\{\string_MtsM\string_B1}{}
\markboth{\textcolor{darkblue}{\textbf{\ipa{njæ˧tsɯ˩}}}}{}
\textcolor{teal}{\zh{名词}} \hspace{4pt} \zh{声调类:} L\#.
\ding{202} \zh{眉毛。} \textcolor{Sepia}{\selectlanguage{english}Eyebrow.} \textcolor{PineGreen}{\selectlanguage{french}Sourcil.}  ¶ \textcolor{darkblue}{\textbf{\ipa{njæ˧tsɯ˩-ɖæ˩}}} \zh{眉毛} \textcolor{Sepia}{\selectlanguage{english}eyebrow (this formulation avoids ambiguity between 'eyebrow' and 'eyelashes')} \textcolor{PineGreen}{\selectlanguage{french}sourcil (formulation permettant de lever l'ambiguïté du terme, qui peut signifier 'sourcil' aussi bien que 'cil')}  
 ¶ \textcolor{darkblue}{\textbf{\ipa{njæ˧tsɯ˩ | mv̩˩tɕo˧ kʰɯ˧˥}}} \zh{皱眉毛} \textcolor{Sepia}{\selectlanguage{english}to knit the brows (literally “to lower the brows”)} \textcolor{PineGreen}{\selectlanguage{french}froncer les sourcils (littéralement “abaisser les sourcils”)}  
 \zh{量词}: \textcolor{darkblue}{\textbf{\ipa{kʰwɤ˥}}} \ding{203} \zh{睫毛、眼睫毛、眼毛。} \textcolor{Sepia}{\selectlanguage{english}Eyelashes.} \textcolor{PineGreen}{\selectlanguage{french}Cil.}  ¶ \textcolor{darkblue}{\textbf{\ipa{njæ˧tsɯ˩-ʂæ˩}}} \textcolor{PineGreen}{\selectlanguage{french}cil (formulation permettant de lever l'ambiguïté du terme, qui peut signifier “sourcil” aussi bien que “cil”)}  

\lhead{\firstmark}
\rhead{\botmark}

\subsection{\hspace{-0.5cm} {\Large \textcolor{darkblue}{\textbf{\ipa{njæ˧=zɯ˩}}}}\hspace{0.5cm}[\kern2pt{\textcolor{darkblue}{\textbf{\ipa{njæ˧zɯ˩}}}}\kern2pt]} \hypertarget{nj\{\string_M=zM\string_B1}{}
\markboth{\textcolor{darkblue}{\textbf{\ipa{njæ˧=zɯ˩}}}}{}
\textcolor{teal}{\zh{代词}} \hspace{4pt} \zh{声调类:} L\#.
\zh{我们两个。} \textcolor{Sepia}{\selectlanguage{english}Dual exclusive first person pronoun: us two, the two of us (the speaker plus another person who is not the addressee).} \textcolor{PineGreen}{\selectlanguage{french}Pronom de première personne duelle exclusive: nous deux (le locuteur et une autre personne qui n'est pas l'interlocuteur).}  ¶ \textcolor{darkblue}{\textbf{\ipa{ɑ˩ʁo˧(-hĩ˧) | njæ˧zɯ˩ ho˩-dʑo˩!}}} \zh{(我们两个不能再呆在这里了,)家里在等我们!} \textcolor{Sepia}{\selectlanguage{english}(We cannot stay any longer because) our family is waiting for us!} \textcolor{PineGreen}{\selectlanguage{french}(On ne peut pas rester car) les gens de la famille nous attendent!}  

\lhead{\firstmark}
\rhead{\botmark}

\subsection{\hspace{-0.5cm} {\Large \textcolor{darkblue}{\textbf{\ipa{njæ˩pʰv̩˧}}}}\hspace{0.5cm}[\kern2pt{\textcolor{darkblue}{\textbf{\ipa{njæ˩pʰv̩˥}}}}\kern2pt]} \hypertarget{nj\{\string_Bp\string_hv\string_=\string_M1}{}
\markboth{\textcolor{darkblue}{\textbf{\ipa{njæ˩pʰv̩˧}}}}{}
\textcolor{teal}{\zh{名词}} \hspace{4pt} \zh{声调类:} LM.
\zh{白眼球。} \textcolor{Sepia}{\selectlanguage{english}White of the eye.} \textcolor{PineGreen}{\selectlanguage{french}Blanc des yeux.}  \zh{量词}: \textcolor{darkblue}{\textbf{\ipa{ɭɯ˧}}} 
\lhead{\firstmark}
\rhead{\botmark}

\subsection{\hspace{-0.5cm} {\Large \textcolor{darkblue}{\textbf{\ipa{njæ˩qwæ˧˥}}}}\hspace{0.5cm}[\kern2pt{\textcolor{darkblue}{\textbf{\ipa{njæ˩qwæ˧˥}}}}\kern2pt]} \hypertarget{nj\{\string_Bqw\{\string_M\string_T1}{}
\markboth{\textcolor{darkblue}{\textbf{\ipa{njæ˩qwæ˧˥}}}}{}
\textcolor{teal}{\zh{形容词}} \hspace{4pt} \zh{声调类:} LM+MH\#.
\zh{眼睛瞎了。} \textcolor{Sepia}{\selectlanguage{english}Blind.} \textcolor{PineGreen}{\selectlanguage{french}Aveugle.}  ¶ \textcolor{darkblue}{\textbf{\ipa{ʈʂʰɯ˧ | njæ˩qwæ˧-ze˥}}} \zh{他眼睛瞎了。} \textcolor{Sepia}{\selectlanguage{english}(S)he went blind.} \textcolor{PineGreen}{\selectlanguage{french}Elle/il est devenu(e) aveugle.}  
 ¶ \textcolor{darkblue}{\textbf{\ipa{ʈʂʰɯ˧ | njæ˩qwæ˧ ɲi˥.}}} \zh{他是瞎子。} \textcolor{Sepia}{\selectlanguage{english}(S)he is blind.} \textcolor{PineGreen}{\selectlanguage{french}Elle/il est aveugle.}  
 ¶ \textcolor{darkblue}{\textbf{\ipa{njæ˩qwæ˧-mi\#˥}}} \zh{眼睛瞎了的女人} \textcolor{Sepia}{\selectlanguage{english}blind woman} \textcolor{PineGreen}{\selectlanguage{french}femme aveugle}  
 ¶ \textcolor{darkblue}{\textbf{\ipa{njæ˩qwæ˧-zo\#˥}}} \zh{眼睛瞎了的男人} \textcolor{Sepia}{\selectlanguage{english}blind man} \textcolor{PineGreen}{\selectlanguage{french}homme aveugle}  
 ¶ \textcolor{darkblue}{\textbf{\ipa{njæ˩qwæ˧-hĩ\#˥}}} \zh{瞎子} \textcolor{Sepia}{\selectlanguage{english}blind person} \textcolor{PineGreen}{\selectlanguage{french}personne aveugle}  
 \zh{量词}: \textcolor{darkblue}{\textbf{\ipa{v̩˧}}} 
\lhead{\firstmark}
\rhead{\botmark}

\subsection{\hspace{-0.5cm} {\Large \textcolor{darkblue}{\textbf{\ipa{njæ˩qʰæ\#˥}}}}\hspace{0.5cm}[\kern2pt{\textcolor{darkblue}{\textbf{\ipa{njæ˩qʰæ˥}}}}\kern2pt]} \hypertarget{nj\{\string_Bq\string_h\{\#\string_T1}{}
\markboth{\textcolor{darkblue}{\textbf{\ipa{njæ˩qʰæ\#˥}}}}{}
\textcolor{teal}{\zh{名词}} \hspace{4pt} \zh{声调类:} LM+\#H.
\zh{眼屎。} \textcolor{Sepia}{\selectlanguage{english}Eye sand, gum in the eyes, rheum.} \textcolor{PineGreen}{\selectlanguage{french}Chassie.}  \zh{量词}: \textcolor{darkblue}{\textbf{\ipa{kʰwɤ˥}}} 
\lhead{\firstmark}
\rhead{\botmark}

\subsection{\hspace{-0.5cm} {\Large \textcolor{darkblue}{\textbf{\ipa{njɤ˧di˧˥}}}}\hspace{0.5cm}[\kern2pt{\textcolor{darkblue}{\textbf{\ipa{njɤ˧di˧˥}}}}\kern2pt]} \hypertarget{nj7\string_Mdi\string_M\string_T1}{}
\markboth{\textcolor{darkblue}{\textbf{\ipa{njɤ˧di˧˥}}}}{}
\textcolor{teal}{\zh{名词}} \hspace{4pt} \zh{声调类:} MH\#.
\zh{胶。} \textcolor{Sepia}{\selectlanguage{english}Glue.} \textcolor{PineGreen}{\selectlanguage{french}Colle.}  \zh{量词}: \textcolor{darkblue}{\textbf{\ipa{kʰwɤ˥}}} 
\lhead{\firstmark}
\rhead{\botmark}

\subsection{\hspace{-0.5cm} {\Large \textcolor{darkblue}{\textbf{\ipa{njɤ˧kv̩˩}}}}\hspace{0.5cm}[\kern2pt{\textcolor{darkblue}{\textbf{\ipa{njɤ˧kv̩˩}}}}\kern2pt]} \hypertarget{nj7\string_Mkv\string_=\string_B1}{}
\markboth{\textcolor{darkblue}{\textbf{\ipa{njɤ˧kv̩˩}}}}{}
\textcolor{teal}{\zh{名词}} \hspace{4pt} \zh{声调类:} L\#.
\zh{颧骨。} \textcolor{Sepia}{\selectlanguage{english}Cheekbone.} \textcolor{PineGreen}{\selectlanguage{french}Pommettes.}  \zh{量词}: \textcolor{darkblue}{\textbf{\ipa{ɭɯ˧}}} \zh{~【参考】~} \hyperlink{}{\textcolor{darkblue}{\textbf{\ipa{kv̩˩kv̩˩}}}} 
\lhead{\firstmark}
\rhead{\botmark}

\subsection{\hspace{-0.5cm} {\Large \textcolor{darkblue}{\textbf{\ipa{njɤ˧kv̩˩-njɤ˩tsʰɤ˩}}}}\hspace{0.5cm}[\kern2pt{\textcolor{darkblue}{\textbf{\ipa{njɤ˧kv̩˩njɤ˧tsʰɤ˧}}}}\kern2pt]} \hypertarget{nj7\string_Mkv\string_=\string_B-nj7\string_Bts\string_h7\string_B1}{}
\markboth{\textcolor{darkblue}{\textbf{\ipa{njɤ˧kv̩˩-njɤ˩tsʰɤ˩}}}}{}
\textcolor{teal}{\zh{形容词}} \hspace{4pt} \zh{声调类:} L\#-.
\zh{美丽、面貌美。} \textcolor{Sepia}{\selectlanguage{english}Beautiful; with a pretty face (of a woman).} \textcolor{PineGreen}{\selectlanguage{french}Belle; qui a un beau visage, qui a des traits gracieux.}  ¶ \textcolor{darkblue}{\textbf{\ipa{ə˧mi˧! | mv̩˩zo˩ ʈʂʰɯ˩-ɭɯ˥ | njɤ˧kv̩˩-njɤ˩tsʰɤ˩! | ɖwæ˧˥ | ə˧v̩˧˥!}}} \zh{啊呀,这个少女真美丽!很漂亮!} \textcolor{Sepia}{\selectlanguage{english}Wow! This young lady is really beautiful! Very pretty!} \textcolor{PineGreen}{\selectlanguage{french}Eh bien, cette jeune fille est vraiment belle! Très jolie!}  
 ¶ \textcolor{darkblue}{\textbf{\ipa{njɤ˧kv̩˩njɤ˩tsʰɤ˩ | ʐwæ˩˥}}} \zh{非常美} \textcolor{Sepia}{\selectlanguage{english}extremely beautiful} \textcolor{PineGreen}{\selectlanguage{french}particulièrement belle}  
\zh{~【参考】~} \textcolor{darkblue}{\textbf{\ipa{njɤ˧kv̩˩, tsʰɤ˧˥a}}} 
\lhead{\firstmark}
\rhead{\botmark}

\subsection{\hspace{-0.5cm} {\Large \textcolor{darkblue}{\textbf{\ipa{njɤ˧le˧gv̩\#˥}}}}\hspace{0.5cm}[\kern2pt{\textcolor{darkblue}{\textbf{\ipa{njɤ˧le˧gv̩˧}}}}\kern2pt]} \hypertarget{nj7\string_Mle\string_Mgv\string_=\#\string_T1}{}
\markboth{\textcolor{darkblue}{\textbf{\ipa{njɤ˧le˧gv̩\#˥}}}}{}
\textcolor{teal}{\zh{名词}} \hspace{4pt} \zh{声调类:} \#H.
\zh{白天、大白天。} \textcolor{Sepia}{\selectlanguage{english}Daytime.} \textcolor{PineGreen}{\selectlanguage{french}Journée, plein jour.}  ¶ \textcolor{darkblue}{\textbf{\ipa{ɲi˧mi˧-njɤ˩le˩gv̩˩}}} \zh{白天} \textcolor{Sepia}{\selectlanguage{english}daytime} \textcolor{PineGreen}{\selectlanguage{french}même sens}  

\lhead{\firstmark}
\rhead{\botmark}

\subsection{\hspace{-0.5cm} {\Large \textcolor{darkblue}{\textbf{\ipa{njɤ˧mv̩˥}}}}\hspace{0.5cm}[\kern2pt{\textcolor{darkblue}{\textbf{\ipa{njɤ˧mv̩˥}}}}\kern2pt]} \hypertarget{nj7\string_Mmv\string_=\string_T1}{}
\markboth{\textcolor{darkblue}{\textbf{\ipa{njɤ˧mv̩˥}}}}{}
\textcolor{teal}{\zh{名词}} \hspace{4pt} \zh{声调类:} H\#.
\zh{灰条菜、灰灰菜:喂猪的牧草。} \textcolor{Sepia}{\selectlanguage{english}A plant used as fodder for the pigs, \textit{Chenopodium album}.} \textcolor{PineGreen}{\selectlanguage{french}Sorte de fourrage pour les cochons, \textit{Chenopodium album}. (Il y a en tout trois sortes de fourrage pour les cochons.).} \zh{当地汉语方言:}\zh{灰凋。} \zh{量词}: \textcolor{darkblue}{\textbf{\ipa{qɑ˩}}} 
\lhead{\firstmark}
\rhead{\botmark}

\subsection{\hspace{-0.5cm} {\Large \textcolor{darkblue}{\textbf{\ipa{njɤ˧mv̩˥-mi˩}}}}\hspace{0.5cm}[\kern2pt{\textcolor{darkblue}{\textbf{\ipa{njɤ˧mv̩˥mi˩}}}}\kern2pt]} \hypertarget{nj7\string_Mmv\string_=\string_T-mi\string_B1}{}
\markboth{\textcolor{darkblue}{\textbf{\ipa{njɤ˧mv̩˥-mi˩}}}}{}
\textcolor{teal}{\zh{名词}} \hspace{4pt} \zh{声调类:} H\#-L.
\zh{骆驼。} \textcolor{Sepia}{\selectlanguage{english}Camel.} \textcolor{PineGreen}{\selectlanguage{french}Chameau.}  ¶ \textcolor{darkblue}{\textbf{\ipa{njɤ˧mv̩˥mi˩-zo˩}}} \zh{小骆驼} \textcolor{Sepia}{\selectlanguage{english}baby camel} \textcolor{PineGreen}{\selectlanguage{french}enfant du chameau, petit chameau}  
 ¶ \textcolor{darkblue}{\textbf{\ipa{njɤ˧mv̩˥mi˩-pʰv̩˩}}} \zh{公骆驼} \textcolor{Sepia}{\selectlanguage{english}male camel} \textcolor{PineGreen}{\selectlanguage{french}chameau mâle}  
 \zh{量词}: \textcolor{darkblue}{\textbf{\ipa{mi˩}}} 
\lhead{\firstmark}
\rhead{\botmark}

\subsection{\hspace{-0.5cm} {\Large \textcolor{darkblue}{\textbf{\ipa{njɤ˧nɑ˩}}}}\hspace{0.5cm}[\kern2pt{\textcolor{darkblue}{\textbf{\ipa{njɤ˧nɑ˩}}}}\kern2pt]} \hypertarget{nj7\string_MnA\string_B1}{}
\markboth{\textcolor{darkblue}{\textbf{\ipa{njɤ˧nɑ˩}}}}{}
\textcolor{teal}{\zh{名词}} \hspace{4pt} \zh{声调类:} L\#.
\zh{眼珠。} \textcolor{Sepia}{\selectlanguage{english}Eyeball.} \textcolor{PineGreen}{\selectlanguage{french}Prunelle.}  \zh{量词}: \textcolor{darkblue}{\textbf{\ipa{ɭɯ˧}}} 
\lhead{\firstmark}
\rhead{\botmark}

\subsection{\hspace{-0.5cm} {\Large \textcolor{darkblue}{\textbf{\ipa{njɤ˧ʈʂɤ˥}}}}\hspace{0.5cm}[\kern2pt{\textcolor{darkblue}{\textbf{\ipa{njɤ˧ʈʂɤ˥}}}}\kern2pt]} \hypertarget{nj7\string_Mt`s`7\string_T1}{}
\markboth{\textcolor{darkblue}{\textbf{\ipa{njɤ˧ʈʂɤ˥}}}}{}
\textcolor{teal}{\zh{名词}} \hspace{4pt} \zh{声调类:} H\#.
\zh{鬼针草。} \textcolor{Sepia}{\selectlanguage{english}Black-jack, beggar-ticks, cobbler's pegs, Spanish needle, \textit{Bidens pilosa L.}, a species of flowering plant in the aster family. The barbed awns of the seeds catch onto fur or clothing, and can injure flesh.} \textcolor{PineGreen}{\selectlanguage{french}Sornet, herbe à aiguilles, \textit{Bidens pilosa L.}: plante de la famille des Asteraceae, dont les graines noires, fines et allongées, de 5 à 10 mm, s'accrochent aux vêtements et aux poils d'animaux par deux piquants fins, situés à l'une de leurs extrémités.} 
\lhead{\firstmark}
\rhead{\botmark}

\subsection{\hspace{-0.5cm} {\Large \textcolor{darkblue}{\textbf{\ipa{njɤ˩}}}}\hspace{0.5cm}[\kern2pt{\textcolor{darkblue}{\textbf{\ipa{njɤ˩˥}}}}\kern2pt]} \hypertarget{nj7\string_B1}{}
\markboth{\textcolor{darkblue}{\textbf{\ipa{njɤ˩}}}}{}
\textcolor{teal}{\zh{代词}} \hspace{4pt} \zh{声调类:} L.
\zh{我。} \textcolor{Sepia}{\selectlanguage{english}1st singular pronoun, \mytextsc{1sg}.} \textcolor{PineGreen}{\selectlanguage{french}Pronom de 1e personne du singulier.}  ¶ \textcolor{darkblue}{\textbf{\ipa{njɤ˩ ɲi˩˥!}}} \zh{是我!(情景:一个人敲门,里面的人问是谁,人家回答:“是我!”)} \textcolor{Sepia}{\selectlanguage{english}It's me! (Typical answer at the door)} \textcolor{PineGreen}{\selectlanguage{french}C'est moi! (Contexte: quelqu'un frappe à la porte, on demande qui c'est, et on reçoit pour réponse: “C'est moi!”)}  
 ¶ \textcolor{darkblue}{\textbf{\ipa{njɤ˧ no˧ lɑ˧˥}}} \zh{我打你} \textcolor{Sepia}{\selectlanguage{english}I strike you} \textcolor{PineGreen}{\selectlanguage{french}je te frappe}  

\lhead{\firstmark}
\rhead{\botmark}

\subsection{\hspace{-0.5cm} {\Large \textcolor{darkblue}{\textbf{\ipa{njɤ˩\textsubscript{b}}}}}\hspace{0.5cm}[\kern2pt{\textcolor{darkblue}{\textbf{\ipa{njɤ˩˥}}}}\kern2pt]} \hypertarget{nj7\string_Bb1}{}
\markboth{\textcolor{darkblue}{\textbf{\ipa{njɤ˩\textsubscript{b}}}}}{}
\textcolor{teal}{\zh{动词}} \hspace{4pt} \zh{声调类:} L\textsubscript{b}.
\zh{舂米。} \textcolor{Sepia}{\selectlanguage{english}To husk.} \textcolor{PineGreen}{\selectlanguage{french}Décortiquer le riz.}  ¶ \textcolor{darkblue}{\textbf{\ipa{le˧-njɤ˩-ze˩}}} \zh{舂了} \textcolor{Sepia}{\selectlanguage{english}\mytextsc{accomp} \string_ \mytextsc{pfv}} \textcolor{PineGreen}{\selectlanguage{french}\mytextsc{accomp} \string_ \mytextsc{pfv}}  
 ¶ \textcolor{darkblue}{\textbf{\ipa{hɑ˧ njɤ˧˥}}} \zh{舂米} \textcolor{Sepia}{\selectlanguage{english}to husk rice} \textcolor{PineGreen}{\selectlanguage{french}décortiquer du riz}  
 ¶ \textcolor{darkblue}{\textbf{\ipa{hɑ˧ | le˧-njɤ˩}}} \zh{舂米} \textcolor{Sepia}{\selectlanguage{english}to husk rice} \textcolor{PineGreen}{\selectlanguage{french}décortiquer du riz}  
 ¶ \textcolor{darkblue}{\textbf{\ipa{hɑ˧ | ɖɯ˧-njɤ˧\textasciitilde{}njɤ˩-ɻ̍˩}}} \zh{把米舂一舂} \textcolor{Sepia}{\selectlanguage{english}rice - \mytextsc{delimitative} \mytextsc{red} \mytextsc{inceptive}} \textcolor{PineGreen}{\selectlanguage{french}riz - \mytextsc{délimitatif} \string_ \mytextsc{red} \mytextsc{inchoatif} : décortiquer un peu le riz}  

\lhead{\firstmark}
\rhead{\botmark}

\subsection{\hspace{-0.5cm} {\Large \textcolor{darkblue}{\textbf{\ipa{njɤ˩bi˥}}}}\hspace{0.5cm}[\kern2pt{\textcolor{darkblue}{\textbf{\ipa{njɤ˩bi˥}}}}\kern2pt]} \hypertarget{nj7\string_Bbi\string_T1}{}
\markboth{\textcolor{darkblue}{\textbf{\ipa{njɤ˩bi˥}}}}{}
\textcolor{teal}{\zh{名词}} \hspace{4pt} \zh{声调类:} LH.
\zh{上眼皮。} \textcolor{Sepia}{\selectlanguage{english}Eyelid (top eyelid).} \textcolor{PineGreen}{\selectlanguage{french}Paupière supérieure.}  \zh{量词}: \textcolor{darkblue}{\textbf{\ipa{ɭɯ˧}}} 
\lhead{\firstmark}
\rhead{\botmark}

\subsection{\hspace{-0.5cm} {\Large \textcolor{darkblue}{\textbf{\ipa{njɤ˩-gɤ˧lɑ˩}}}}\hspace{0.5cm}[\kern2pt{\textcolor{darkblue}{\textbf{\ipa{njɤ˧gɤ˧lɑ˩}}}}\kern2pt]} \hypertarget{nj7\string_B-g7\string_MlA\string_B1}{}
\markboth{\textcolor{darkblue}{\textbf{\ipa{njɤ˩-gɤ˧lɑ˩}}}}{}
\textcolor{teal}{\zh{名词}} \hspace{4pt} \zh{声调类:} L-L\#.
\zh{眼珠。} \textcolor{Sepia}{\selectlanguage{english}Eyeball.} \textcolor{PineGreen}{\selectlanguage{french}Prunelle.}  \zh{量词}: \textcolor{darkblue}{\textbf{\ipa{ɭɯ˧}}} 
\lhead{\firstmark}
\rhead{\botmark}

\subsection{\hspace{-0.5cm} {\Large \textcolor{darkblue}{\textbf{\ipa{njɤ˩kʰi\#˥}}}}\hspace{0.5cm}[\kern2pt{\textcolor{darkblue}{\textbf{\ipa{njɤ˩kʰi˥}}}}\kern2pt]} \hypertarget{nj7\string_Bk\string_hi\#\string_T1}{}
\markboth{\textcolor{darkblue}{\textbf{\ipa{njɤ˩kʰi\#˥}}}}{}
\textcolor{teal}{\zh{名词}} \hspace{4pt} \zh{声调类:} LM+\#H.
\zh{下眼皮。} \textcolor{Sepia}{\selectlanguage{english}Bottom eyelid.} \textcolor{PineGreen}{\selectlanguage{french}Paupière inférieure.}  \zh{量词}: \textcolor{darkblue}{\textbf{\ipa{kʰwɤ˥}}} 
\lhead{\firstmark}
\rhead{\botmark}

\subsection{\hspace{-0.5cm} {\Large \textcolor{darkblue}{\textbf{\ipa{njɤ˩ɭɯ˧}}}}\hspace{0.5cm}[\kern2pt{\textcolor{darkblue}{\textbf{\ipa{njɤ˩ɭɯ˥}}}}\kern2pt]} \hypertarget{nj7\string_Bl\string_RM\string_M1}{}
\markboth{\textcolor{darkblue}{\textbf{\ipa{njɤ˩ɭɯ˧}}}}{}
\textcolor{teal}{\zh{名词}} \hspace{4pt} \zh{声调类:} LM.
\zh{眼睛。} \textcolor{Sepia}{\selectlanguage{english}Eye.} \textcolor{PineGreen}{\selectlanguage{french}Œil.}  \zh{量词}: \textcolor{darkblue}{\textbf{\ipa{ɭɯ˧}}} 
\lhead{\firstmark}
\rhead{\botmark}

\subsection{\hspace{-0.5cm} {\Large \textcolor{darkblue}{\textbf{\ipa{njɤ˩qʰwɤ˧˥}}}}\hspace{0.5cm}[\kern2pt{\textcolor{darkblue}{\textbf{\ipa{njɤ˩qʰwɤ˧˥}}}}\kern2pt]} \hypertarget{nj7\string_Bq\string_hw7\string_M\string_T1}{}
\markboth{\textcolor{darkblue}{\textbf{\ipa{njɤ˩qʰwɤ˧˥}}}}{}
\textcolor{teal}{\zh{名词}} \hspace{4pt} \zh{声调类:} LM+MH\#.
\zh{眼眶。} \textcolor{Sepia}{\selectlanguage{english}Orbit; eye socket.} \textcolor{PineGreen}{\selectlanguage{french}Orbite (de l'œil).}  \zh{量词}: \textcolor{darkblue}{\textbf{\ipa{ɭɯ˧}}} 
\lhead{\firstmark}
\rhead{\botmark}

\subsection{\hspace{-0.5cm} {\Large \textcolor{darkblue}{\textbf{\ipa{njɤ˩-tse˧\textasciitilde{}tse˩}}}}\hspace{0.5cm}[\kern2pt{\textcolor{darkblue}{\textbf{\ipa{njɤ˧tse˧tse˩}}}}\kern2pt]} \hypertarget{nj7\string_B-tse\string_M~tse\string_B1}{}
\markboth{\textcolor{darkblue}{\textbf{\ipa{njɤ˩-tse˧\textasciitilde{}tse˩}}}}{}
\textcolor{teal}{\zh{名词}} \hspace{4pt} \zh{声调类:} L-L\#.
\zh{节节草。} \textcolor{Sepia}{\selectlanguage{english}Branched horsetail, \textit{Equisetum ramosissimum Desf.} This is a wild herb used in traditional medicine; its stem consists of small segments; when pulled/plucked, the stem breaks at one of these articulations.} \textcolor{PineGreen}{\selectlanguage{french}Prêle ramifiée, prêle rameuse, \textit{Equisetum ramosissimum Desf.} Herbe sauvage, utilisée en pharmacopée traditionnelle; sa tige est divisée en petits segments, et elle se brise à l'une de ces articulations si on l'arrache.} \zh{当地汉语方言:}\zh{节节高。} \zh{量词}: \textcolor{darkblue}{\textbf{\ipa{po˧}}} 
\lhead{\firstmark}
\rhead{\botmark}

\subsection{\hspace{-0.5cm} {\Large \textcolor{darkblue}{\textbf{\ipa{njɤ˩ʈʂv̩˧˥}}}}\hspace{0.5cm}[\kern2pt{\textcolor{darkblue}{\textbf{\ipa{njɤ˩ʈʂv̩˧˥}}}}\kern2pt]} \hypertarget{nj7\string_Bt`s`v\string_=\string_M\string_T1}{}
\markboth{\textcolor{darkblue}{\textbf{\ipa{njɤ˩ʈʂv̩˧˥}}}}{}
\textcolor{teal}{\zh{名词}} \hspace{4pt} \zh{声调类:} LM+MH\#.
\zh{泥鳅。} \textcolor{Sepia}{\selectlanguage{english}Loach (a kind of fish).} \textcolor{PineGreen}{\selectlanguage{french}Loche (poisson).}  \zh{量词}: \textcolor{darkblue}{\textbf{\ipa{mi˩}}} 
\lhead{\firstmark}
\rhead{\botmark}

\subsection{\hspace{-0.5cm} {\Large \textcolor{darkblue}{\textbf{\ipa{njɤ˧˥}}} \textsubscript{1}}\hspace{0.5cm}[\kern2pt{\textcolor{darkblue}{\textbf{\ipa{njɤ˧˥}}}}\kern2pt]} \hypertarget{nj7\string_M\string_T1}{}
\markboth{\textcolor{darkblue}{\textbf{\ipa{njɤ˧˥}}} \textsubscript{1}}{}
\textcolor{teal}{\zh{动词}} \hspace{4pt} \zh{声调类:} MH.
\zh{贴。} \textcolor{Sepia}{\selectlanguage{english}To glue (two objects together).} \textcolor{PineGreen}{\selectlanguage{french}Coller (2 objets ensemble).}  ¶ \textcolor{darkblue}{\textbf{\ipa{le˧-njɤ˧-ze˥!}}} \zh{粘在一起了!} \textcolor{Sepia}{\selectlanguage{english}\mytextsc{accomp} \string_ \mytextsc{pfv}: It's glued!} \textcolor{PineGreen}{\selectlanguage{french}C'est recollé! / Ca y est, c'est collé!}  
 ¶ \textcolor{darkblue}{\textbf{\ipa{tso˧\textasciitilde{}tso˧ le˧-ɖʐɤ˧, | le˧-njɤ˧˥!}}} \zh{东西撕破了,粘在一起(就好了)} \textcolor{Sepia}{\selectlanguage{english}When something (e.g. a book) is torn, (we) glue it together!} \textcolor{PineGreen}{\selectlanguage{french}(pour un livre, par ex.) Quand un truc est déchiré, on le recolle!}  
\zh{~【参考】~} \hyperlink{}{\textcolor{darkblue}{\textbf{\ipa{njɤ˧˥}}} \textsubscript{2}} 
\lhead{\firstmark}
\rhead{\botmark}

\subsection{\hspace{-0.5cm} {\Large \textcolor{darkblue}{\textbf{\ipa{njɤ˧˥}}} \textsubscript{2}}\hspace{0.5cm}[\kern2pt{\textcolor{darkblue}{\textbf{\ipa{njɤ˧˥}}}}\kern2pt]} \hypertarget{nj7\string_M\string_T2}{}
\markboth{\textcolor{darkblue}{\textbf{\ipa{njɤ˧˥}}} \textsubscript{2}}{}
\textcolor{teal}{\zh{形容词}} \hspace{4pt} \zh{声调类:} MH.
\zh{黏(胶,树脂)。} \textcolor{Sepia}{\selectlanguage{english}Sticky.} \textcolor{PineGreen}{\selectlanguage{french}Poisseux, collant, visqueux (colle, résine...).} \zh{~【参考】~} \hyperlink{}{\textcolor{darkblue}{\textbf{\ipa{njɤ˧˥}}} \textsubscript{1}} 
\lhead{\firstmark}
\rhead{\botmark}

\subsection{\hspace{-0.5cm} {\Large \textcolor{darkblue}{\textbf{\ipa{njɤ˧˥}}} \textsubscript{3}}\hspace{0.5cm}[\kern2pt{\textcolor{darkblue}{\textbf{\ipa{njɤ˧˥}}}}\kern2pt]} \hypertarget{nj7\string_M\string_T3}{}
\markboth{\textcolor{darkblue}{\textbf{\ipa{njɤ˧˥}}} \textsubscript{3}}{}
\textcolor{teal}{\zh{形容词}} \hspace{4pt} \zh{声调类:} MH.
\zh{早。} \textcolor{Sepia}{\selectlanguage{english}Early (to rise early).} \textcolor{PineGreen}{\selectlanguage{french}Tôt.} 
\lhead{\firstmark}
\rhead{\botmark}

\subsection{\hspace{-0.5cm} {\Large \textcolor{darkblue}{\textbf{\ipa{njɤ˩˥}}}}\hspace{0.5cm}[\kern2pt{\textcolor{darkblue}{\textbf{\ipa{njɤ˩˥}}}}\kern2pt]} \hypertarget{nj7\string_B\string_T1}{}
\markboth{\textcolor{darkblue}{\textbf{\ipa{njɤ˩˥}}}}{}
\textcolor{teal}{\zh{名词}} \hspace{4pt} \zh{声调类:} LH.
\zh{眼睛(单音节)。} \textcolor{Sepia}{\selectlanguage{english}Eye (monosyllable).} \textcolor{PineGreen}{\selectlanguage{french}Œil (monosyllabe).}  \zh{量词}: \textcolor{darkblue}{\textbf{\ipa{ɭɯ˧}}} 
\lhead{\firstmark}
\rhead{\botmark}

\subsection{\hspace{-0.5cm} {\Large \textcolor{darkblue}{\textbf{\ipa{njo˥}}}}\hspace{0.5cm}[\kern2pt{\textcolor{darkblue}{\textbf{\ipa{njo˥}}}}\kern2pt]} \hypertarget{njo\string_T1}{}
\markboth{\textcolor{darkblue}{\textbf{\ipa{njo˥}}}}{}
\textcolor{teal}{\zh{名词}} \hspace{4pt} \zh{声调类:} \#H.
\zh{葫芦丝、葫芦箫。} \textcolor{Sepia}{\selectlanguage{english}Cucurbit flute, hulusi: a free reed wind instrument.} \textcolor{PineGreen}{\selectlanguage{french}Flûte à calebasse, hulusi: instrument à vent à anche libre.}  ¶ \textcolor{darkblue}{\textbf{\ipa{njo˧ mv̩˥}}} \zh{吹响葫芦丝} \textcolor{Sepia}{\selectlanguage{english}to play the cucurbit flute} \textcolor{PineGreen}{\selectlanguage{french}jouer de la flûte à calebasse}  

\lhead{\firstmark}
\rhead{\botmark}

\subsection{\hspace{-0.5cm} {\Large \textcolor{darkblue}{\textbf{\ipa{njo˧}}}}\hspace{0.5cm}[\kern2pt{\textcolor{darkblue}{\textbf{\ipa{njo˥}}}}\kern2pt]} \hypertarget{njo\string_M1}{}
\markboth{\textcolor{darkblue}{\textbf{\ipa{njo˧}}}}{}
\textcolor{teal}{\zh{名词}} \hspace{4pt} \zh{声调类:} M.
\zh{谷穗。} \textcolor{Sepia}{\selectlanguage{english}Ear (of wheat, barley).} \textcolor{PineGreen}{\selectlanguage{french}Épi (de blé, d'orge, de riz...).}  ¶ \textcolor{darkblue}{\textbf{\ipa{hɑ˧-njo˩}}} \zh{谷穗} \textcolor{Sepia}{\selectlanguage{english}ear of cereals} \textcolor{PineGreen}{\selectlanguage{french}épi de céréales}  
 ¶ \textcolor{darkblue}{\textbf{\ipa{mv̩˧dze˧-njo˧ (+ɲi˩)}}} \zh{大麦穗} \textcolor{Sepia}{\selectlanguage{english}ear of barley} \textcolor{PineGreen}{\selectlanguage{french}épi d'orge}  
 ¶ \textcolor{darkblue}{\textbf{\ipa{tsʰi˧zi˧-hɑ˧njo˥ (+ɲi˩)}}} \zh{青稞穗} \textcolor{Sepia}{\selectlanguage{english}ear of highland barley} \textcolor{PineGreen}{\selectlanguage{french}épi d'orge d'altitude}  

\lhead{\firstmark}
\rhead{\botmark}

\subsection{\hspace{-0.5cm} {\Large \textcolor{darkblue}{\textbf{\ipa{njo˧bi˧li˥}}}}\hspace{0.5cm}[\kern2pt{\textcolor{darkblue}{\textbf{\ipa{njo˧bi˧li˥}}}}\kern2pt]} \hypertarget{njo\string_Mbi\string_Mli\string_T1}{}
\markboth{\textcolor{darkblue}{\textbf{\ipa{njo˧bi˧li˥}}}}{}
\textcolor{teal}{\zh{名词}} \hspace{4pt} \zh{声调类:} H\#.
\zh{嘴唇。} \textcolor{Sepia}{\selectlanguage{english}Lips.} \textcolor{PineGreen}{\selectlanguage{french}Lèvres.}  \zh{量词}: \textcolor{darkblue}{\textbf{\ipa{ɭɯ˧}}} 
\lhead{\firstmark}
\rhead{\botmark}

\subsection{\hspace{-0.5cm} {\Large \textcolor{darkblue}{\textbf{\ipa{njo˩bi˥}}}}\hspace{0.5cm}[\kern2pt{\textcolor{darkblue}{\textbf{\ipa{njo˩bi˥}}}}\kern2pt]} \hypertarget{njo\string_Bbi\string_T1}{}
\markboth{\textcolor{darkblue}{\textbf{\ipa{njo˩bi˥}}}}{}
\textcolor{teal}{\zh{名词}} \hspace{4pt} \zh{声调类:} LH.
\zh{乳房。} \textcolor{Sepia}{\selectlanguage{english}Breast.} \textcolor{PineGreen}{\selectlanguage{french}Sein, mamelle.}  ¶ \textcolor{darkblue}{\textbf{\ipa{ʝi˧-njo˥bi˩}}} \zh{牛的奶头} \textcolor{Sepia}{\selectlanguage{english}cow's breast} \textcolor{PineGreen}{\selectlanguage{french}mamelle de la vache}  
 ¶ \textcolor{darkblue}{\textbf{\ipa{[F5] njo˩bi˧-ʁo˧qʰwɤ˩}}} \zh{乳头} \textcolor{Sepia}{\selectlanguage{english}nipple, teat} \textcolor{PineGreen}{\selectlanguage{french}téton}  
 \zh{量词}: \textcolor{darkblue}{\textbf{\ipa{ɭɯ˧}}} 
\lhead{\firstmark}
\rhead{\botmark}

\subsection{\hspace{-0.5cm} {\Large \textcolor{darkblue}{\textbf{\ipa{njo˩kæ˧tɕi˩˥}}}}\hspace{0.5cm}[\kern2pt{\textcolor{darkblue}{\textbf{\ipa{xxxx ton non trouvé, à faire manuellement...}}}}\kern2pt]} \hypertarget{njo\string_Bk\{\string_Mts£i\string_B\string_T1}{}
\markboth{\textcolor{darkblue}{\textbf{\ipa{njo˩kæ˧tɕi˩˥}}}}{}
\textcolor{teal}{\zh{名词}} \hspace{4pt} \zh{声调类:} LM+LH.
\zh{牛肝菌(汉语借词)。} \textcolor{Sepia}{\selectlanguage{english}Cep, \textit{Boletus edulis}.} \textcolor{PineGreen}{\selectlanguage{french}Bolet, cèpe, \textit{Boletus edulis}; littéralement “champignon-galette de sarrasin”, du fait de sa texture.}  \zh{【借词】} \zh{牛肝菌}
\zh{~【参考】~} \hyperlink{}{\textcolor{darkblue}{\textbf{\ipa{jɤ˧qʰɑ˧-pɤ˥jɤ˩-mo˩}}}} 
\lhead{\firstmark}
\rhead{\botmark}

\subsection{\hspace{-0.5cm} {\Large \textcolor{darkblue}{\textbf{\ipa{njo˩pɤ˩lv̩˥}}}}\hspace{0.5cm}[\kern2pt{\textcolor{darkblue}{\textbf{\ipa{njo˩pɤ˩lv̩˥}}}}\kern2pt]} \hypertarget{njo\string_Bp7\string_Blv\string_=\string_T1}{}
\markboth{\textcolor{darkblue}{\textbf{\ipa{njo˩pɤ˩lv̩˥}}}}{}
\textcolor{teal}{\zh{名词}} \hspace{4pt} \zh{声调类:} L+H\#.
\zh{牛的奶头。} \textcolor{Sepia}{\selectlanguage{english}Udder.} \textcolor{PineGreen}{\selectlanguage{french}Pis de la vache.}  \zh{量词}: \textcolor{darkblue}{\textbf{\ipa{ɭɯ˧}}} 
\lhead{\firstmark}
\rhead{\botmark}

\subsection{\hspace{-0.5cm} {\Large \textcolor{darkblue}{\textbf{\ipa{njo˩˥}}}}\hspace{0.5cm}[\kern2pt{\textcolor{darkblue}{\textbf{\ipa{njo˩˥}}}}\kern2pt]} \hypertarget{njo\string_B\string_T1}{}
\markboth{\textcolor{darkblue}{\textbf{\ipa{njo˩˥}}}}{}
\textcolor{teal}{\zh{名词}} \hspace{4pt} \zh{声调类:} LH.
\zh{奶汁。} \textcolor{Sepia}{\selectlanguage{english}Milk.} \textcolor{PineGreen}{\selectlanguage{french}Lait.}  ¶ \textcolor{darkblue}{\textbf{\ipa{njo˩ ki˧}}} \zh{给(喂)奶} \textcolor{Sepia}{\selectlanguage{english}to breast-feed (literally 'to give milk')} \textcolor{PineGreen}{\selectlanguage{french}donner le sein, donner la tétée, nourrir (un nourrisson); littéralement '“donner du lait”}  
 ¶ \textcolor{darkblue}{\textbf{\ipa{njo˩ ʈʰɯ˩˥}}} \zh{喝奶} \textcolor{Sepia}{\selectlanguage{english}to drink milk} \textcolor{PineGreen}{\selectlanguage{french}boire du lait}  

\lhead{\firstmark}
\rhead{\botmark}

\subsection{\hspace{-0.5cm} {\Large \textcolor{darkblue}{\textbf{\ipa{no˧no˧}}}}\hspace{0.5cm}[\kern2pt{\textcolor{darkblue}{\textbf{\ipa{no˧no˧}}}}\kern2pt]} \hypertarget{no\string_Mno\string_M1}{}
\markboth{\textcolor{darkblue}{\textbf{\ipa{no˧no˧}}}}{}
\textcolor{teal}{\zh{名词}} \hspace{4pt} \zh{声调类:} M.
\zh{女性名字。} \textcolor{Sepia}{\selectlanguage{english}Feminine given name.} \textcolor{PineGreen}{\selectlanguage{french}Prénom féminin.} 
\lhead{\firstmark}
\rhead{\botmark}

\subsection{\hspace{-0.5cm} {\Large \textcolor{darkblue}{\textbf{\ipa{no˧=ɻ̍˩}}}}\hspace{0.5cm}[\kern2pt{\textcolor{darkblue}{\textbf{\ipa{no˧ɻ̍˩}}}}\kern2pt]} \hypertarget{no\string_M=r£`̍\string_B1}{}
\markboth{\textcolor{darkblue}{\textbf{\ipa{no˧=ɻ̍˩}}}}{}
\textcolor{teal}{\zh{代词}} \hspace{4pt} \zh{声调类:} L\#.
\zh{你们。} \textcolor{Sepia}{\selectlanguage{english}Second person plural.} \textcolor{PineGreen}{\selectlanguage{french}Deuxième personne du pluriel.} \zh{~【参考】~} \hyperlink{}{\textcolor{darkblue}{\textbf{\ipa{no˧-sɯ˩kv̩˩}}}} 
\lhead{\firstmark}
\rhead{\botmark}

\subsection{\hspace{-0.5cm} {\Large \textcolor{darkblue}{\textbf{\ipa{no˧=zɯ˩}}}}\hspace{0.5cm}[\kern2pt{\textcolor{darkblue}{\textbf{\ipa{no˧zɯ˩}}}}\kern2pt]} \hypertarget{no\string_M=zM\string_B1}{}
\markboth{\textcolor{darkblue}{\textbf{\ipa{no˧=zɯ˩}}}}{}
\textcolor{teal}{\zh{代词}} \hspace{4pt} \zh{声调类:} L\#.
\zh{你们俩。} \textcolor{Sepia}{\selectlanguage{english}Dual second person pronoun: you two.} \textcolor{PineGreen}{\selectlanguage{french}Pronom personnel de deuxième personne duel: vous deux.} \zh{~【参考】~} \hyperlink{}{\textcolor{darkblue}{\textbf{\ipa{no˩zɯ˧˥}}}} 
\lhead{\firstmark}
\rhead{\botmark}

\subsection{\hspace{-0.5cm} {\Large \textcolor{darkblue}{\textbf{\ipa{no˩}}}}\hspace{0.5cm}[\kern2pt{\textcolor{darkblue}{\textbf{\ipa{no˩˥}}}}\kern2pt]} \hypertarget{no\string_B1}{}
\markboth{\textcolor{darkblue}{\textbf{\ipa{no˩}}}}{}
\textcolor{teal}{\zh{代词}} \hspace{4pt} \zh{声调类:} L.
\zh{你。} \textcolor{Sepia}{\selectlanguage{english}Second person singular pronoun.} \textcolor{PineGreen}{\selectlanguage{french}Pronom de deuxième personne du singulier.}  ¶ \textcolor{darkblue}{\textbf{\ipa{no˩ ɲi˩˥}}} \zh{是你!} \textcolor{Sepia}{\selectlanguage{english}It's you!} \textcolor{PineGreen}{\selectlanguage{french}c'est toi!}  

\lhead{\firstmark}
\rhead{\botmark}

\subsection{\hspace{-0.5cm} {\Large \textcolor{darkblue}{\textbf{\ipa{no˩bv̩˧}}}}\hspace{0.5cm}[\kern2pt{\textcolor{darkblue}{\textbf{\ipa{no˩bv̩˥}}}}\kern2pt]} \hypertarget{no\string_Bbv\string_=\string_M1}{}
\markboth{\textcolor{darkblue}{\textbf{\ipa{no˩bv̩˧}}}}{}
\textcolor{teal}{\zh{名词}} \hspace{4pt} \zh{声调类:} LM.
\zh{男性名字。} \textcolor{Sepia}{\selectlanguage{english}Masculine given name.} \textcolor{PineGreen}{\selectlanguage{french}Prénom masculin.} 
\lhead{\firstmark}
\rhead{\botmark}

\subsection{\hspace{-0.5cm} {\Large \textcolor{darkblue}{\textbf{\ipa{no˩qo˥}}}}\hspace{0.5cm}[\kern2pt{\textcolor{darkblue}{\textbf{\ipa{no˩qo˥}}}}\kern2pt]} \hypertarget{no\string_Bqo\string_T1}{}
\markboth{\textcolor{darkblue}{\textbf{\ipa{no˩qo˥}}}}{}
\textcolor{teal}{\zh{助词}} \hspace{4pt} \zh{声调类:} LH.
\zh{……附近。} \textcolor{Sepia}{\selectlanguage{english}Close to, next to.} \textcolor{PineGreen}{\selectlanguage{french}À proximité de, à côté de.} 
\lhead{\firstmark}
\rhead{\botmark}

\subsection{\hspace{-0.5cm} {\Large \textcolor{darkblue}{\textbf{\ipa{no˩zɯ˧˥}}}}\hspace{0.5cm}[\kern2pt{\textcolor{darkblue}{\textbf{\ipa{no˩zɯ˧˥}}}}\kern2pt]} \hypertarget{no\string_BzM\string_M\string_T1}{}
\markboth{\textcolor{darkblue}{\textbf{\ipa{no˩zɯ˧˥}}}}{}
\textcolor{teal}{\zh{代词}} \hspace{4pt} \zh{声调类:} LM+MH\#.
\zh{你们俩。} \textcolor{Sepia}{\selectlanguage{english}Dual second person pronoun: you two.} \textcolor{PineGreen}{\selectlanguage{french}Pronom personnel de deuxième personne duel: vous deux.} \zh{~【参考】~} \hyperlink{}{\textcolor{darkblue}{\textbf{\ipa{no˧=zɯ˩}}}} 
\lhead{\firstmark}
\rhead{\botmark}

\subsection{\hspace{-0.5cm} {\Large \textcolor{darkblue}{\textbf{\ipa{no˩xx}}}}\hspace{0.5cm}[\kern2pt{\textcolor{darkblue}{\textbf{\ipa{xxxx ton non trouvé, à faire manuellement...}}}}\kern2pt]} \hypertarget{no\string_Bxx1}{}
\markboth{\textcolor{darkblue}{\textbf{\ipa{no˩xx}}}}{}
\textcolor{teal}{\zh{动词}} \hspace{4pt} \zh{声调类:} Lxx.
\zh{搀。} \textcolor{Sepia}{\selectlanguage{english}To add, to blend in, to mix.} \textcolor{PineGreen}{\selectlanguage{french}Mélanger, ajouter.}  ¶ \textcolor{darkblue}{\textbf{\ipa{ɳæ˧ | tsɑ˧bɤ˧-qo˧ tʰi˧-no˩}}} \zh{在糌粑里搀奶、糌粑里搀奶} \textcolor{Sepia}{\selectlanguage{english}to mix grilled flour with milk, to add milk to grilled flour} \textcolor{PineGreen}{\selectlanguage{french}mettre du lait dans la farine grillée, mélanger la farine grillée avec du lait}  
 ¶ \textcolor{darkblue}{\textbf{\ipa{le˧-no˩}}} \zh{\mytextsc{accomp} \string_} \textcolor{Sepia}{\selectlanguage{english}\mytextsc{accomp} \string_} \textcolor{PineGreen}{\selectlanguage{french}\mytextsc{accomp} \string_}  
 ¶ \textcolor{darkblue}{\textbf{\ipa{hɑ˧-qo˩ | tɕæ˧ɻæ˩ tʰi˩-no˩: hɑ˧-qo˩ | tɕæ˧ɻæ˩ tʰi˩-kʰɯ˩}}} \zh{关于这个动词的说明:‘饭里面搀泡菜,就是说:在饭里面放泡菜。’} \textcolor{Sepia}{\selectlanguage{english}a paraphrase to explain the verb's meaning: 'to blend pickled vegetables into the food/rice, that means: to add picked vegetables to the food.'} \textcolor{PineGreen}{\selectlanguage{french}paraphrase pour expliquer le sens du verbe: 'ajouter des légumes en saumure dans la nourriture, ça veut dire: mettre des légumes en saumure dans la nourriture.'}  
 ¶ \textcolor{darkblue}{\textbf{\ipa{hɑ˧-qo˩ | tɕæ˧ɻæ˩ tʰi˩-no˩: tɕæ˧ɻæ˩-lɑ˩ | hɑ˧ | ɖɯ˧-tɕʰo˩ dzɯ˩}}} \zh{关于这个动词的说明:‘饭里面搀泡菜,就是说:把泡菜和饭一起吃。’} \textcolor{Sepia}{\selectlanguage{english}a paraphrase to explain the verb's meaning: 'to blend pickled vegetables into the food/rice, that means: to eat picked vegetables and food/rice together.'} \textcolor{PineGreen}{\selectlanguage{french}paraphrase pour expliquer le sens du verbe: 'ajouter des légumes en saumure dans la nourriture, ça veut dire: manger des légumes en saumure et de la nourriture (du riz) ensemble.'}  

\lhead{\firstmark}
\rhead{\botmark}

\subsection{\hspace{-0.5cm} {\Large \textcolor{darkblue}{\textbf{\ipa{‑no˧˥}}}}\hspace{0.5cm}[\kern2pt{\textcolor{darkblue}{\textbf{\ipa{no˧˥}}}}\kern2pt]} \hypertarget{‑no\string_M\string_T1}{}
\markboth{\textcolor{darkblue}{\textbf{\ipa{‑no˧˥}}}}{}
\textcolor{teal}{\zh{语气助词}} \hspace{4pt} \zh{声调类:} MH.
\zh{\mytextsc{主题:对于、关于。}} \textcolor{Sepia}{\selectlanguage{english}Contrastive topic.} \textcolor{PineGreen}{\selectlanguage{french}Topique contrastif. Gloses possibles: …, en revanche, ...; …, pour sa part, ...; quant à ….}  ¶ \textcolor{darkblue}{\textbf{\ipa{qæ˧do˧ | -no˧˥}}} \zh{关于木材,……} \textcolor{Sepia}{\selectlanguage{english}As for lumber, ...} \textcolor{PineGreen}{\selectlanguage{french}en ce qui concerne le bois de construction, …}  
 ¶ \textcolor{darkblue}{\textbf{\ipa{ʑi˧qʰwɤ˧ | -no˧˥}}} \zh{关于主屋……} \textcolor{Sepia}{\selectlanguage{english}As for the main room, ...} \textcolor{PineGreen}{\selectlanguage{french}en ce qui concerne la pièce principale, …}  

\lhead{\firstmark}
\rhead{\botmark}

\subsection{\hspace{-0.5cm} {\Large \textcolor{darkblue}{\textbf{\ipa{nv̩˥}}} \textsubscript{1}}\hspace{0.5cm}[\kern2pt{\textcolor{darkblue}{\textbf{\ipa{nv̩˥}}}}\kern2pt]} \hypertarget{nv\string_=\string_T1}{}
\markboth{\textcolor{darkblue}{\textbf{\ipa{nv̩˥}}} \textsubscript{1}}{}
\textcolor{teal}{\zh{动词}} \hspace{4pt} \zh{声调类:} H.
\zh{追赶。} \textcolor{Sepia}{\selectlanguage{english}To chase after, to pursue.} \textcolor{PineGreen}{\selectlanguage{french}Suivre à la trace, pister.}  ¶ \textcolor{darkblue}{\textbf{\ipa{le˧-nv̩˥}}} \zh{\mytextsc{accomp}} \textcolor{Sepia}{\selectlanguage{english}\mytextsc{accomp}} \textcolor{PineGreen}{\selectlanguage{french}\mytextsc{accomp}}  
 ¶ \textcolor{darkblue}{\textbf{\ipa{ʈʂɤ˩nv̩˩}}} \zh{追赶} \textcolor{Sepia}{\selectlanguage{english}to chase after, to pursue} \textcolor{PineGreen}{\selectlanguage{french}suivre à la trace, pister}  
 ¶ \textcolor{darkblue}{\textbf{\ipa{le˧-ʈʂɤ˩nv̩˩}}} \zh{追赶} \textcolor{Sepia}{\selectlanguage{english}to chase after, to pursue} \textcolor{PineGreen}{\selectlanguage{french}suivre à la trace, pister}  
 ¶ \textcolor{darkblue}{\textbf{\ipa{le˧-ʈʂɤ˩nv̩˩ | le˧-hɯ˩}}} \zh{追赶去了} \textcolor{Sepia}{\selectlanguage{english}He went to chase after} \textcolor{PineGreen}{\selectlanguage{french}Il est parti à la poursuite de / à la chasse de}  

\lhead{\firstmark}
\rhead{\botmark}

\subsection{\hspace{-0.5cm} {\Large \textcolor{darkblue}{\textbf{\ipa{nv̩˥}}} \textsubscript{2}}\hspace{0.5cm}[\kern2pt{\textcolor{darkblue}{\textbf{\ipa{nv̩˥}}}}\kern2pt]} \hypertarget{nv\string_=\string_T2}{}
\markboth{\textcolor{darkblue}{\textbf{\ipa{nv̩˥}}} \textsubscript{2}}{}
\textcolor{teal}{\zh{动词}} \hspace{4pt} \zh{声调类:} H.
\zh{埋。} \textcolor{Sepia}{\selectlanguage{english}To bury.} \textcolor{PineGreen}{\selectlanguage{french}Enterrer.} 
\lhead{\firstmark}
\rhead{\botmark}

\subsection{\hspace{-0.5cm} {\Large \textcolor{darkblue}{\textbf{\ipa{nv̩˩dʑɯ˥}}}}\hspace{0.5cm}[\kern2pt{\textcolor{darkblue}{\textbf{\ipa{nv̩˩dʑɯ˩˥}}}}\kern2pt]} \hypertarget{nv\string_=\string_Bdz£M\string_T1}{}
\markboth{\textcolor{darkblue}{\textbf{\ipa{nv̩˩dʑɯ˥}}}}{}
\textcolor{teal}{\zh{名词}} \hspace{4pt} \zh{声调类:} LH.
\zh{豆腐。} \textcolor{Sepia}{\selectlanguage{english}Tofu, bean curd.} \textcolor{PineGreen}{\selectlanguage{french}Tofu.}  \zh{量词}: \textcolor{darkblue}{\textbf{\ipa{v̩˥}}} 
\lhead{\firstmark}
\rhead{\botmark}

\subsection{\hspace{-0.5cm} {\Large \textcolor{darkblue}{\textbf{\ipa{nv̩˩ho\#˥}}}}\hspace{0.5cm}[\kern2pt{\textcolor{darkblue}{\textbf{\ipa{nv̩˩ho˥}}}}\kern2pt]} \hypertarget{nv\string_=\string_Bho\#\string_T1}{}
\markboth{\textcolor{darkblue}{\textbf{\ipa{nv̩˩ho\#˥}}}}{}
\textcolor{teal}{\zh{名词}} \hspace{4pt} \zh{声调类:} LM+\#H.
\zh{豆花。} \textcolor{Sepia}{\selectlanguage{english}Long-boiled soft beancurd.} \textcolor{PineGreen}{\selectlanguage{french}Tofu léger, longuement bouilli.}  \zh{量词}: \textcolor{darkblue}{\textbf{\ipa{v̩˥}}} 
\lhead{\firstmark}
\rhead{\botmark}

\subsection{\hspace{-0.5cm} {\Large \textcolor{darkblue}{\textbf{\ipa{nv̩˧hṽ˩}}}}\hspace{0.5cm}[\kern2pt{\textcolor{darkblue}{\textbf{\ipa{nv̩˩hṽ˥}}}}\kern2pt]} \hypertarget{nv\string_=\string_Mhv\string_~\string_B1}{}
\markboth{\textcolor{darkblue}{\textbf{\ipa{nv̩˧hṽ˩}}}}{}
\textcolor{teal}{\zh{名词}} \hspace{4pt} \zh{声调类:} L\#.
\zh{豆子,四季豆,花腰豆。} \textcolor{Sepia}{\selectlanguage{english}Bean; string bean, kidney bean.} \textcolor{PineGreen}{\selectlanguage{french}Haricot: terme générique.}  \zh{量词}: \textcolor{darkblue}{\textbf{\ipa{v̩˥}}} 
\lhead{\firstmark}
\rhead{\botmark}

\subsection{\hspace{-0.5cm} {\Large \textcolor{darkblue}{\textbf{\ipa{nv̩˧hṽ˩-bi˩bi˩}}}}\hspace{0.5cm}[\kern2pt{\textcolor{darkblue}{\textbf{\ipa{xxxx non-correspondance entre le nombre de morphèmes et le nombre de tons de morphèmes}}}}\kern2pt]} \hypertarget{nv\string_=\string_Mhv\string_~\string_B-bi\string_Bbi\string_B1}{}
\markboth{\textcolor{darkblue}{\textbf{\ipa{nv̩˧hṽ˩-bi˩bi˩}}}}{}
\textcolor{teal}{\zh{名词}} \hspace{4pt} \zh{声调类:} L\#-.
\zh{四季豆、玉豆、帶莢豌豆、菜豆、刀豆、豆角、敏豆仔、敏豆。} \textcolor{Sepia}{\selectlanguage{english}Green bean, snap bean, string bean; one consumes the pod with the seed inside.} \textcolor{PineGreen}{\selectlanguage{french}Haricot vert; on consomme la cosse fraîche et la graine qu'il contient.}  \zh{量词}: \textcolor{darkblue}{\textbf{\ipa{kʰɤ˧˥}}} 
\lhead{\firstmark}
\rhead{\botmark}

\subsection{\hspace{-0.5cm} {\Large \textcolor{darkblue}{\textbf{\ipa{nv̩˩ɭɯ˧}}}}\hspace{0.5cm}[\kern2pt{\textcolor{darkblue}{\textbf{\ipa{xxxx non-correspondance entre le nombre de morphèmes et le nombre de tons de morphèmes}}}}\kern2pt]} \hypertarget{nv\string_=\string_Bl\string_RM\string_M1}{}
\markboth{\textcolor{darkblue}{\textbf{\ipa{nv̩˩ɭɯ˧}}}}{}
\textcolor{teal}{\zh{名词}} \hspace{4pt} \zh{声调类:} LM.
\zh{黄豆。} \textcolor{Sepia}{\selectlanguage{english}Soy beans, soya beans.} \textcolor{PineGreen}{\selectlanguage{french}Soja.}  \zh{量词}: \textcolor{darkblue}{\textbf{\ipa{wɤ˩}}} 
\lhead{\firstmark}
\rhead{\botmark}

\subsection{\hspace{-0.5cm} {\Large \textcolor{darkblue}{\textbf{\ipa{nv̩˩mi˩}}}}\hspace{0.5cm}[\kern2pt{\textcolor{darkblue}{\textbf{\ipa{nv̩˩mi˥}}}}\kern2pt]} \hypertarget{nv\string_=\string_Bmi\string_B1}{}
\markboth{\textcolor{darkblue}{\textbf{\ipa{nv̩˩mi˩}}}}{}
\textcolor{teal}{\zh{名词}} \hspace{4pt} \zh{声调类:} L.
\ding{202} \zh{心脏。} \textcolor{Sepia}{\selectlanguage{english}Heart.} \textcolor{PineGreen}{\selectlanguage{french}Cœur.}  ¶ \textcolor{darkblue}{\textbf{\ipa{hĩ˧ ʈʂʰɯ˧-v̩˧, | nv̩˩mi˩ tɕi˥! |}}} \zh{这个人,胆小!(直译:“心小”)} \textcolor{Sepia}{\selectlanguage{english}This person lacks courage! (literally “...(his/her) heart is small”)} \textcolor{PineGreen}{\selectlanguage{french}Celui-là, il manque de courage! (littéralement “(a) un petit coeur”, “son coeur est petit”)}  
 ¶ \textcolor{darkblue}{\textbf{\ipa{nv̩˩mi˩˥ | ɖɯ˧-ɭɯ˧ tsɤ˧ |}}} \zh{情投意合} \textcolor{Sepia}{\selectlanguage{english}in sympathy, in unison} \textcolor{PineGreen}{\selectlanguage{french}être en sympathie, à l'unisson}  
 ¶ \textcolor{darkblue}{\textbf{\ipa{nv̩˩mi˩˥ | tʰi˧-tɕɯ˥ | so˩˥}}} \zh{耐心地学习 / 耐心地教} \textcolor{Sepia}{\selectlanguage{english}to study patiently / to teach patiently} \textcolor{PineGreen}{\selectlanguage{french}enseigner patiemment/apprendre patiemment}  
 ¶ \textcolor{darkblue}{\textbf{\ipa{nv̩˩mi˩-qo˥ | tʰi˧-ʑi˥}}} \zh{记住、记得(直译:‘心里存着’、‘心里有’)} \textcolor{Sepia}{\selectlanguage{english}to remember, to keep in mind} \textcolor{PineGreen}{\selectlanguage{french}se souvenir de, garder à l'esprit, avoir à l'esprit}  
 ¶ \textcolor{darkblue}{\textbf{\ipa{nv̩˩mi˩-qo˥ | tʰi˧-kʰɯ˧˥}}} \zh{记住(直译:‘放在心里’)} \textcolor{Sepia}{\selectlanguage{english}to make an effort to remember, to carry in mind} \textcolor{PineGreen}{\selectlanguage{french}faire l'effort de se souvenir, garder à l'esprit, garder en mémoire}  
 \zh{量词}: \textcolor{darkblue}{\textbf{\ipa{ɭɯ˧}}} \ding{203} \zh{心情。} \textcolor{Sepia}{\selectlanguage{english}State of mind.} \textcolor{PineGreen}{\selectlanguage{french}État d'esprit.}  ¶ \textcolor{darkblue}{\textbf{\ipa{nv̩˩mi˩ dzɯ˩\textasciitilde{}dzɯ˩-ɻ̍˥}}} \zh{经常吵架、过不到一起去} \textcolor{Sepia}{\selectlanguage{english}not to get along well; to quarrel all the time; to poison each other's life} \textcolor{PineGreen}{\selectlanguage{french}ne pas bien s'entendre; se chamailler sans répit; s'empoisonner mutuellement la vie, se faire la vie impossible, se bouffer le nez}  

\lhead{\firstmark}
\rhead{\botmark}

\subsection{\hspace{-0.5cm} {\Large \textcolor{darkblue}{\textbf{\ipa{nv̩˩mi˩-ɖɯ˩}}}}\hspace{0.5cm}[\kern2pt{\textcolor{darkblue}{\textbf{\ipa{xxxx non-correspondance entre le nombre de morphèmes et le nombre de tons de morphèmes}}}}\kern2pt]} \hypertarget{nv\string_=\string_Bmi\string_B-d`M\string_B1}{}
\markboth{\textcolor{darkblue}{\textbf{\ipa{nv̩˩mi˩-ɖɯ˩}}}}{}
\textcolor{teal}{\zh{形容词}} \hspace{4pt} \zh{声调类:} L.
\zh{勇敢、有勇气的。} \textcolor{Sepia}{\selectlanguage{english}Courageous, brave.} \textcolor{PineGreen}{\selectlanguage{french}Courageux, audacieux.}  ¶ \textcolor{darkblue}{\textbf{\ipa{ʈʂʰɯ˧ | nv̩˩mi˩˥ | ɖwæ˧˥ | ɖɯ˩˥! | hĩ˧ | mɤ˧-ɖwæ˥!}}} \zh{他很勇敢!谁也不怕!} \textcolor{Sepia}{\selectlanguage{english}(S)he is very brave! (S)he is not afraid of others / (s)he fears no one!} \textcolor{PineGreen}{\selectlanguage{french}il est très courageux! il n'a peur de personne!}  
 ¶ \textcolor{darkblue}{\textbf{\ipa{pɤ˩mi˩˥ | nv̩˩mi˩ ɖɯ˩˥, | ʝi˧-ɳɯ˧ tʰv̩˧˥; | mi˩zɯ˩ nv̩˥mi˩ ɖɯ˩ (-dʑo˩), | hĩ˧-ɳɯ˧ lɑ˧˥!}}} \zh{“勇敢的青蛙,被牛压死。勇敢的女人,被人家打!”(说明:青蛙太勇敢,上马路,就容易被压死,而女人太勇敢,容易跟别人发生矛盾,最后就打不过男人。)} \textcolor{Sepia}{\selectlanguage{english}“If the frog is brave, it gets stamped on by the ox; if the woman is brave, she gets beaten!” (Explanation: weaker creatures must not be too brave: if a frog fears nothing and ventures onto the roads, it can easily get crushed to death; if a woman behaves with masculine self-assurance and courage, she gets into situations where people come to hands, and she eventually has the lower hand.)} \textcolor{PineGreen}{\selectlanguage{french}“La grenouille courageuse, elle se fait écraser par le bœuf; la femme courageuse, elle se fait frapper!” (Explication: les créatures qui ne sont pas les plus fortes doivent se garder d'être trop courageuses: la grenouille qui n'a peur de rien et s'aventure sur le grand chemin, elle se fait écraser; la femme qui se comporte avec une mâle assurance, elle finit par entrer dans des conflits où on en vient aux mains et où elle a finalement le dessous.)}  

\lhead{\firstmark}
\rhead{\botmark}

\subsection{\hspace{-0.5cm} {\Large \textcolor{darkblue}{\textbf{\ipa{nv̩˩mi˩-ki˧ki˩}}}}\hspace{0.5cm}[\kern2pt{\textcolor{darkblue}{\textbf{\ipa{xxxx non-correspondance entre le nombre de morphèmes et le nombre de tons de morphèmes}}}}\kern2pt]} \hypertarget{nv\string_=\string_Bmi\string_B-ki\string_Mki\string_B1}{}
\markboth{\textcolor{darkblue}{\textbf{\ipa{nv̩˩mi˩-ki˧ki˩}}}}{}
\textcolor{teal}{\zh{形容词}} \hspace{4pt} \zh{声调类:} .
\zh{心意相通。} \textcolor{Sepia}{\selectlanguage{english}With similar mood/frame of mind.} \textcolor{PineGreen}{\selectlanguage{french}En harmonie, à l'unisson.} 
\lhead{\firstmark}
\rhead{\botmark}

\subsection{\hspace{-0.5cm} {\Large \textcolor{darkblue}{\textbf{\ipa{nv̩˩mi˩-ʈʰi˩}}}}\hspace{0.5cm}[\kern2pt{\textcolor{darkblue}{\textbf{\ipa{xxxx non-correspondance entre le nombre de morphèmes et le nombre de tons de morphèmes}}}}\kern2pt]} \hypertarget{nv\string_=\string_Bmi\string_B-t`\string_hi\string_B1}{}
\markboth{\textcolor{darkblue}{\textbf{\ipa{nv̩˩mi˩-ʈʰi˩}}}}{}
\textcolor{teal}{\zh{形容词}} \hspace{4pt} \zh{声调类:} L.
\zh{累得没精神了。} \textcolor{Sepia}{\selectlanguage{english}Weak, worn out.} \textcolor{PineGreen}{\selectlanguage{french}Découragé, nostalgique, mélancolique.} 
\lhead{\firstmark}
\rhead{\botmark}

\subsection{\hspace{-0.5cm} {\Large \textcolor{darkblue}{\textbf{\ipa{nv̩˩pi˧}}}}\hspace{0.5cm}[\kern2pt{\textcolor{darkblue}{\textbf{\ipa{nv̩˩pi˥}}}}\kern2pt]} \hypertarget{nv\string_=\string_Bpi\string_M1}{}
\markboth{\textcolor{darkblue}{\textbf{\ipa{nv̩˩pi˧}}}}{}
\textcolor{teal}{\zh{名词}} \hspace{4pt} \zh{声调类:} LM.
\zh{豆粕。} \textcolor{Sepia}{\selectlanguage{english}Soybean dregs.} \textcolor{PineGreen}{\selectlanguage{french}Tourteau de soja: le reste du soja, après qu'on en a tiré le lait de soja; sert de nourriture pour les porcs.} 
\lhead{\firstmark}
\rhead{\botmark}

\subsection{\hspace{-0.5cm} {\Large \textcolor{darkblue}{\textbf{\ipa{nv̩˧pɤ˩}}}}\hspace{0.5cm}[\kern2pt{\textcolor{darkblue}{\textbf{\ipa{nv̩˧pɤ˩}}}}\kern2pt]} \hypertarget{nv\string_=\string_Mp7\string_B1}{}
\markboth{\textcolor{darkblue}{\textbf{\ipa{nv̩˧pɤ˩}}}}{}
\textcolor{teal}{\zh{名词}} \hspace{4pt} \zh{声调类:} L\#.
\zh{蚕豆。} \textcolor{Sepia}{\selectlanguage{english}Broad beans; lima beans.} \textcolor{PineGreen}{\selectlanguage{french}Fèves.} 
\lhead{\firstmark}
\rhead{\botmark}

\subsection{\hspace{-0.5cm} {\Large \textcolor{darkblue}{\textbf{\ipa{nv̩˩tɕʰi\#˥}}}}\hspace{0.5cm}[\kern2pt{\textcolor{darkblue}{\textbf{\ipa{nv̩˩tɕʰi˥}}}}\kern2pt]} \hypertarget{nv\string_=\string_Bts£\string_hi\#\string_T1}{}
\markboth{\textcolor{darkblue}{\textbf{\ipa{nv̩˩tɕʰi\#˥}}}}{}
\textcolor{teal}{\zh{名词}} \hspace{4pt} \zh{声调类:} LM+\#H.
\zh{豆类的细糠秕,来喂牛。} \textcolor{Sepia}{\selectlanguage{english}Fine chaff of beans (used to feed cows).} \textcolor{PineGreen}{\selectlanguage{french}Balle de légumineuse (fine, pour nourrir les bovidés).}  \zh{量词}: \textcolor{darkblue}{\textbf{\ipa{kʰɤ˧˥}}} 
\lhead{\firstmark}
\rhead{\botmark}

\subsection{\hspace{-0.5cm} {\Large \textcolor{darkblue}{\textbf{\ipa{nv̩˩tsɑ˧˥}}}}\hspace{0.5cm}[\kern2pt{\textcolor{darkblue}{\textbf{\ipa{nv̩˩tsɑ˧˥}}}}\kern2pt]} \hypertarget{nv\string_=\string_BtsA\string_M\string_T1}{}
\markboth{\textcolor{darkblue}{\textbf{\ipa{nv̩˩tsɑ˧˥}}}}{}
\textcolor{teal}{\zh{名词}} \hspace{4pt} \zh{声调类:} LM+MH\#.
\zh{粗的豆糠。} \textcolor{Sepia}{\selectlanguage{english}Coarse chaff of beans.} \textcolor{PineGreen}{\selectlanguage{french}Balle grossière de légumineuses.}  \zh{量词}: \textcolor{darkblue}{\textbf{\ipa{mɤ˩, etc}}} 
\lhead{\firstmark}
\rhead{\botmark}

\subsection{\hspace{-0.5cm} {\Large \textcolor{darkblue}{\textbf{\ipa{nv̩˧tv̩˥}}}}\hspace{0.5cm}[\kern2pt{\textcolor{darkblue}{\textbf{\ipa{nv̩˧tv̩˥}}}}\kern2pt]} \hypertarget{nv\string_=\string_Mtv\string_=\string_T1}{}
\markboth{\textcolor{darkblue}{\textbf{\ipa{nv̩˧tv̩˥}}}}{}
\textcolor{teal}{\zh{名词}} \hspace{4pt} \zh{声调类:} H\#.
\zh{(挂在马脖子下面的)饲料袋子、马粮袋子。} \textcolor{Sepia}{\selectlanguage{english}Nosebag.} \textcolor{PineGreen}{\selectlanguage{french}Musette à grain, sac à grain: sac dans lequel on donnait à manger au cheval; le sac est pendu au cou du cheval.}  \zh{量词}: \textcolor{darkblue}{\textbf{\ipa{ɭɯ˧}}} 
\lhead{\firstmark}
\rhead{\botmark}

\subsection{\hspace{-0.5cm} {\Large \textcolor{darkblue}{\textbf{\ipa{nv̩˩ze˧}}}}\hspace{0.5cm}[\kern2pt{\textcolor{darkblue}{\textbf{\ipa{nv̩˩ze˥}}}}\kern2pt]} \hypertarget{nv\string_=\string_Bze\string_M1}{}
\markboth{\textcolor{darkblue}{\textbf{\ipa{nv̩˩ze˧}}}}{}
\textcolor{teal}{\zh{名词}} \hspace{4pt} \zh{声调类:} LM.
\zh{鹰嘴豆、桃尔豆、鸡豆、鸡心豆。} \textcolor{Sepia}{\selectlanguage{english}Chickpea, \textit{Cicer arietinum}, black-coloured; the dish \zh{黑色凉粉} is made out of this pea.} \textcolor{PineGreen}{\selectlanguage{french}Pois chiche, \textit{Cicer arietinum}, de couleur noire, dont on prépare la spécialité de Dali: \zh{黑色凉粉}.} \zh{当地汉语方言:}\zh{鸡豆。}
\lhead{\firstmark}
\rhead{\botmark}

\newpage
\section*{\centering- \textcolor{darkblue}{\textbf{\ipa{ɳ}}} -}
\subsection{\hspace{-0.5cm} {\Large \textcolor{darkblue}{\textbf{\ipa{ɳæ˥}}}}\hspace{0.5cm}[\kern2pt{\textcolor{darkblue}{\textbf{\ipa{ɳæ˧˥}}}}\kern2pt]} \hypertarget{n`\{\string_T1}{}
\markboth{\textcolor{darkblue}{\textbf{\ipa{ɳæ˥}}}}{}
\textcolor{teal}{\zh{动词}} \hspace{4pt} \zh{声调类:} H.
\zh{躲藏。} \textcolor{Sepia}{\selectlanguage{english}To hide, to conceal oneself.} \textcolor{PineGreen}{\selectlanguage{french}Se cacher.}  ¶ \textcolor{darkblue}{\textbf{\ipa{tʰi˧-ɳæ˥}}} \zh{\mytextsc{dur} \string_} \textcolor{Sepia}{\selectlanguage{english}\mytextsc{dur} \string_} \textcolor{PineGreen}{\selectlanguage{french}\mytextsc{dur}}  

\lhead{\firstmark}
\rhead{\botmark}

\subsection{\hspace{-0.5cm} {\Large \textcolor{darkblue}{\textbf{\ipa{ɳæ˧=ɻ̍˩}}}}\hspace{0.5cm}[\kern2pt{\textcolor{darkblue}{\textbf{\ipa{ɳæ˧ɻ̍˩}}}}\kern2pt]} \hypertarget{n`\{\string_M=r£`̍\string_B1}{}
\markboth{\textcolor{darkblue}{\textbf{\ipa{ɳæ˧=ɻ̍˩}}}}{}
\textcolor{teal}{\zh{代词}} \hspace{4pt} \zh{声调类:} L\#.
\zh{你们。这是\textcolor{darkblue}{\textbf{\ipa{/no˧=ɻ̍˩/}}}的一个变体。\textcolor{darkblue}{\textbf{\ipa{/no˧=ɻ̍˩/}}}被认为是更标准的。} \textcolor{Sepia}{\selectlanguage{english}Second person plural. This is a variant of \textcolor{darkblue}{\textbf{\ipa{/no˧=ɻ̍˩/}}}; the form \textcolor{darkblue}{\textbf{\ipa{/no˧=ɻ̍˩/}}} is considered more correct.} \textcolor{PineGreen}{\selectlanguage{french}Deuxième personne du pluriel: vous autres. Variante de \textcolor{darkblue}{\textbf{\ipa{/no˧=ɻ̍˩/}}}; la forme \textcolor{darkblue}{\textbf{\ipa{/no˧=ɻ̍˩/}}} est jugée plus correcte.} \zh{~【参考】~} \textcolor{darkblue}{\textbf{\ipa{ɳæ˩=ɻæ˧, no˧=ɻ̍˩}}} 
\lhead{\firstmark}
\rhead{\botmark}

\subsection{\hspace{-0.5cm} {\Large \textcolor{darkblue}{\textbf{\ipa{ɳæ˩=ɻæ˧}}}}\hspace{0.5cm}[\kern2pt{\textcolor{darkblue}{\textbf{\ipa{ɳæ˩ɻæ˥}}}}\kern2pt]} \hypertarget{n`\{\string_B=r£`\{\string_M1}{}
\markboth{\textcolor{darkblue}{\textbf{\ipa{ɳæ˩=ɻæ˧}}}}{}
\textcolor{teal}{\zh{代词}} \hspace{4pt} \zh{声调类:} LM.
\zh{你们家、你们家族。} \textcolor{Sepia}{\selectlanguage{english}Second person associative plural.} \textcolor{PineGreen}{\selectlanguage{french}Deuxième personne, pluriel associatif: vous autres.} \zh{~【参考】~} \textcolor{darkblue}{\textbf{\ipa{ɳæ˧=ɻ̍˩, no˧=ɻ̍˩}}} 
\lhead{\firstmark}
\rhead{\botmark}

\subsection{\hspace{-0.5cm} {\Large \textcolor{darkblue}{\textbf{\ipa{ɳæ˧˥}}}}\hspace{0.5cm}[\kern2pt{\textcolor{darkblue}{\textbf{\ipa{ɳæ˧˥}}}}\kern2pt]} \hypertarget{n`\{\string_M\string_T1}{}
\markboth{\textcolor{darkblue}{\textbf{\ipa{ɳæ˧˥}}}}{}
\textcolor{teal}{\zh{动词}} \hspace{4pt} \zh{声调类:} MH.
\ding{202} \zh{按(用手)、压扁、挤压。} \textcolor{Sepia}{\selectlanguage{english}To press, to push down (with the hand); to press flat, to flatten; to squeeze.} \textcolor{PineGreen}{\selectlanguage{french}Aplatir; appuyer, peser sur; presser.}  ¶ \textcolor{darkblue}{\textbf{\ipa{mv̩˩tɕo˧ ɳæ˧˥}}} \zh{往下按} \textcolor{Sepia}{\selectlanguage{english}to push down, to press down} \textcolor{PineGreen}{\selectlanguage{french}appuyer vers le bas, peser sur}  
 ¶ \textcolor{darkblue}{\textbf{\ipa{le˧-ɳæ˩\textasciitilde{}ɳæ˩}}} \textcolor{PineGreen}{\selectlanguage{french}\mytextsc{accomp} \mytextsc{red}}  
\ding{203} \zh{压迫。} \textcolor{Sepia}{\selectlanguage{english}To oppress.} \textcolor{PineGreen}{\selectlanguage{french}Opprimer, accabler, écraser de son autorité, en imposer par la violence.}  ¶ \textcolor{darkblue}{\textbf{\ipa{hĩ˧ kʰv̩˧, | hĩ˧ ɳæ˩}}} \zh{偷和迫(描述专制统治者的行为)} \textcolor{Sepia}{\selectlanguage{english}to steal and oppress (description of a despot's behaviour)} \textcolor{PineGreen}{\selectlanguage{french}voler et oppresser (description du comportement d'un despote)}  

\lhead{\firstmark}
\rhead{\botmark}

\subsection{\hspace{-0.5cm} {\Large \textcolor{darkblue}{\textbf{\ipa{ɳɯ˥}}}}\hspace{0.5cm}[\kern2pt{\textcolor{darkblue}{\textbf{\ipa{ɳɯ˥}}}}\kern2pt]} \hypertarget{n`M\string_T1}{}
\markboth{\textcolor{darkblue}{\textbf{\ipa{ɳɯ˥}}}}{}
\textcolor{teal}{\zh{形容词}} \hspace{4pt} \zh{声调类:} H.
\zh{少。} \textcolor{Sepia}{\selectlanguage{english}Few.} \textcolor{PineGreen}{\selectlanguage{french}Peu, peu nombreux (dénombrable).}  ¶ \textcolor{darkblue}{\textbf{\ipa{hĩ˧ ɳɯ˧}}} \zh{好的,不多!不好的,就很多了!} \textcolor{Sepia}{\selectlanguage{english}people are few / there are few people} \textcolor{PineGreen}{\selectlanguage{french}les gens sont peu nombreux}  
 ¶ \textcolor{darkblue}{\textbf{\ipa{tso˧\textasciitilde{}tso˧ | ɳɯ˧-ze˩}}} \zh{东西(变)少了} \textcolor{Sepia}{\selectlanguage{english}there are fewer things, the amount of things has decreased} \textcolor{PineGreen}{\selectlanguage{french}il y a moins de choses, la quantité a diminué}  
 ¶ \textcolor{darkblue}{\textbf{\ipa{dʑɤ˩-hĩ˩˥, | le˧-ɳɯ˥! | mɤ˧-dʑɤ˩-hĩ˩, | le˧-dʑɯ˧˥!}}} \zh{好的少,不好的多!(关于大学:高考后,学生要报志愿)} \textcolor{Sepia}{\selectlanguage{english}Good one are few; bad ones are many! / There are few good ones, but many bad ones! (A comment about higher education institutions, among which laureates of the national University entrance examination are given a choice.)} \textcolor{PineGreen}{\selectlanguage{french}Les bons, il n'y en a guère; les médiocres, il y en a en quantité! (contexte: au sujet des établissements universitaires entre lesquels les lauréats du concours national d'entrée à l'université ont à choisir)}  

\lhead{\firstmark}
\rhead{\botmark}

\subsection{\hspace{-0.5cm} {\Large \textcolor{darkblue}{\textbf{\ipa{‑ɳɯ˧}}} \textsubscript{1}}\hspace{0.5cm}[\kern2pt{\textcolor{darkblue}{\textbf{\ipa{ɳɯ˥}}}}\kern2pt]} \hypertarget{‑n`M\string_M1}{}
\markboth{\textcolor{darkblue}{\textbf{\ipa{‑ɳɯ˧}}} \textsubscript{1}}{}
\textcolor{teal}{\zh{后置词}} \hspace{4pt} \zh{声调类:} M.
\zh{离格,施动者,主题。接近汉语的‘由’。} \textcolor{Sepia}{\selectlanguage{english}Ablative, agent, and topic marker.} \textcolor{PineGreen}{\selectlanguage{french}Ablatif, agent, et marqueur de topique.} 
\lhead{\firstmark}
\rhead{\botmark}

\subsection{\hspace{-0.5cm} {\Large \textcolor{darkblue}{\textbf{\ipa{ɳɯ˧˥}}}}\hspace{0.5cm}[\kern2pt{\textcolor{darkblue}{\textbf{\ipa{ɳɯ˧˥}}}}\kern2pt]} \hypertarget{n`M\string_M\string_T1}{}
\markboth{\textcolor{darkblue}{\textbf{\ipa{ɳɯ˧˥}}}}{}
\textcolor{teal}{\zh{动词}} \hspace{4pt} \zh{声调类:} MH.
\zh{拧。} \textcolor{Sepia}{\selectlanguage{english}To wring, to tighten, to clamp.} \textcolor{PineGreen}{\selectlanguage{french}Serrer.}  ¶ \textcolor{darkblue}{\textbf{\ipa{le˧-ɳɯ˧-ze˥}}} \zh{拧了} \textcolor{Sepia}{\selectlanguage{english}\mytextsc{accomp} \string_ \mytextsc{pfv}} \textcolor{PineGreen}{\selectlanguage{french}\mytextsc{accomp} \string_ \mytextsc{pfv}}  
 ¶ \textcolor{darkblue}{\textbf{\ipa{ʁo˧qɑ˥ | ʈʰɯ˧-ɭɯ˧ | le˧-ɳɯ˧-qɑ˥-jo˩!}}} \zh{(你)拧一下盖子吧!} \textcolor{Sepia}{\selectlanguage{english}Tighten the lid! (of a glass jar, used as drinking glass)} \textcolor{PineGreen}{\selectlanguage{french}Serre donc ce couvercle! (celui d'un bocal en verre, utilisé comme verre)}  

\lhead{\firstmark}
\rhead{\botmark}

\subsection{\hspace{-0.5cm} {\Large \textcolor{darkblue}{\textbf{\ipa{ɳv̩˩˧}}}}\hspace{0.5cm}[\kern2pt{\textcolor{darkblue}{\textbf{\ipa{ɳv̩˩˥}}}}\kern2pt]} \hypertarget{n`v\string_=\string_B\string_M1}{}
\markboth{\textcolor{darkblue}{\textbf{\ipa{ɳv̩˩˧}}}}{}
\textcolor{teal}{\zh{名词}} \hspace{4pt} \zh{声调类:} LM.
\zh{蛀虫。} \textcolor{Sepia}{\selectlanguage{english}Moth; insect that eats into wood, books, clothes etc.} \textcolor{PineGreen}{\selectlanguage{french}Mite (insecte qui mange les vêments).}  \zh{量词}: \textcolor{darkblue}{\textbf{\ipa{mi˩}}} 
\lhead{\firstmark}
\rhead{\botmark}

\subsection{\hspace{-0.5cm} {\Large \textcolor{darkblue}{\textbf{\ipa{ɳv̩˥}}}}\hspace{0.5cm}[\kern2pt{\textcolor{darkblue}{\textbf{\ipa{ɳv̩˥}}}}\kern2pt]} \hypertarget{n`v\string_=\string_T1}{}
\markboth{\textcolor{darkblue}{\textbf{\ipa{ɳv̩˥}}}}{}
\textcolor{teal}{\zh{动词}} \hspace{4pt} \zh{声调类:} H.
\ding{202} \zh{闻嗅。} \textcolor{Sepia}{\selectlanguage{english}To sniff.} \textcolor{PineGreen}{\selectlanguage{french}Sentir, renifler.} \ding{203} \zh{听到(消息)、风闻。} \textcolor{Sepia}{\selectlanguage{english}To hear, to get to know (good news…).} \textcolor{PineGreen}{\selectlanguage{french}Apprendre une nouvelle; être au courant de.}  ¶ \textcolor{darkblue}{\textbf{\ipa{mɤ˧-ɳv̩˥}}} \zh{(我)不知道这个消息!} \textcolor{Sepia}{\selectlanguage{english}I am not aware of this piece of news! / I didn't know about that!} \textcolor{PineGreen}{\selectlanguage{french}\mytextsc{neg}: je ne suis pas au courant!}  
 ¶ \textcolor{darkblue}{\textbf{\ipa{no˧ ə˧tso˧ ɳv̩˥?}}} \zh{你听到了什么消息呢?} \textcolor{Sepia}{\selectlanguage{english}Which piece of news did you get? / What did you get to know?} \textcolor{PineGreen}{\selectlanguage{french}Quelle nouvelle as-tu apprise?}  

\lhead{\firstmark}
\rhead{\botmark}

\newpage
\section*{\centering- \textcolor{darkblue}{\textbf{\ipa{ɲ}}} -}
\subsection{\hspace{-0.5cm} {\Large \textcolor{darkblue}{\textbf{\ipa{ɲi˥}}} \textsubscript{1}}\hspace{0.5cm}[\kern2pt{\textcolor{darkblue}{\textbf{\ipa{ɲi˥}}}}\kern2pt]} \hypertarget{Ji\string_T1}{}
\markboth{\textcolor{darkblue}{\textbf{\ipa{ɲi˥}}} \textsubscript{1}}{}
\textcolor{teal}{\zh{动词}} \hspace{4pt} \zh{声调类:} H.
\zh{听。} \textcolor{Sepia}{\selectlanguage{english}To listen.} \textcolor{PineGreen}{\selectlanguage{french}Écouter.}  ¶ \textcolor{darkblue}{\textbf{\ipa{tʰi˧-ɲi˥}}} \zh{\mytextsc{dur}} \textcolor{Sepia}{\selectlanguage{english}\mytextsc{dur}} \textcolor{PineGreen}{\selectlanguage{french}\mytextsc{dur}}  
 ¶ \textcolor{darkblue}{\textbf{\ipa{tso˧\textasciitilde{}tso˧ ɲi˧}}} \zh{听东西} \textcolor{Sepia}{\selectlanguage{english}to listen to things} \textcolor{PineGreen}{\selectlanguage{french}écouter des choses}  
 ¶ \textcolor{darkblue}{\textbf{\ipa{le˧-ɲi˥-ze˩}}} \zh{听了} \textcolor{Sepia}{\selectlanguage{english}\mytextsc{accomp} \string_ \mytextsc{pfv}} \textcolor{PineGreen}{\selectlanguage{french}\mytextsc{accomp} \string_ \mytextsc{pfv}}  

\lhead{\firstmark}
\rhead{\botmark}

\subsection{\hspace{-0.5cm} {\Large \textcolor{darkblue}{\textbf{\ipa{ɲi˥}}} \textsubscript{2}}\hspace{0.5cm}[\kern2pt{\textcolor{darkblue}{\textbf{\ipa{ɲi˥}}}}\kern2pt]} \hypertarget{Ji\string_T2}{}
\markboth{\textcolor{darkblue}{\textbf{\ipa{ɲi˥}}} \textsubscript{2}}{}
\textcolor{teal}{\zh{动词}} \hspace{4pt} \zh{声调类:} H.
\zh{向别人借。} \textcolor{Sepia}{\selectlanguage{english}To borrow from someone.} \textcolor{PineGreen}{\selectlanguage{french}Emprunter (un objet).}  ¶ \textcolor{darkblue}{\textbf{\ipa{hĩ˧-ki˧ | tso˧\textasciitilde{}tso˧ ɲi˧ |}}} \zh{向别人借东西} \textcolor{Sepia}{\selectlanguage{english}to borrow things from someone} \textcolor{PineGreen}{\selectlanguage{french}emprunter des choses à quelqu'un}  

\lhead{\firstmark}
\rhead{\botmark}

\subsection{\hspace{-0.5cm} {\Large \textcolor{darkblue}{\textbf{\ipa{ɲi˥}}} \textsubscript{3}}\hspace{0.5cm}[\kern2pt{\textcolor{darkblue}{\textbf{\ipa{ɲi˥}}}}\kern2pt]} \hypertarget{Ji\string_T3}{}
\markboth{\textcolor{darkblue}{\textbf{\ipa{ɲi˥}}} \textsubscript{3}}{}
\textcolor{teal}{\zh{动词}} \hspace{4pt} \zh{声调类:} H.
\zh{败、输。} \textcolor{Sepia}{\selectlanguage{english}To lose, to be defeated.} \textcolor{PineGreen}{\selectlanguage{french}Échouer, perdre.} 
\lhead{\firstmark}
\rhead{\botmark}

\subsection{\hspace{-0.5cm} {\Large \textcolor{darkblue}{\textbf{\ipa{ɲi˥\textsubscript{b}}}}}\hspace{0.5cm}[\kern2pt{\textcolor{darkblue}{\textbf{\ipa{ɲi˥}}}}\kern2pt]} \hypertarget{Ji\string_Tb1}{}
\markboth{\textcolor{darkblue}{\textbf{\ipa{ɲi˥\textsubscript{b}}}}}{}
\textcolor{teal}{\zh{量词}} \hspace{4pt} \zh{声调类:} H\textsubscript{b}.
\zh{日、天。} \textcolor{Sepia}{\selectlanguage{english}Day.} \textcolor{PineGreen}{\selectlanguage{french}Un jour.}  ¶ \textcolor{darkblue}{\textbf{\ipa{ɖɯ˧-ɲi˥}}} \zh{一天} \textcolor{Sepia}{\selectlanguage{english}one day} \textcolor{PineGreen}{\selectlanguage{french}un jour}  

\lhead{\firstmark}
\rhead{\botmark}

\subsection{\hspace{-0.5cm} {\Large \textcolor{darkblue}{\textbf{\ipa{ɲi˧}}}}\hspace{0.5cm}[\kern2pt{\textcolor{darkblue}{\textbf{\ipa{ɲi˥}}}}\kern2pt]} \hypertarget{Ji\string_M1}{}
\markboth{\textcolor{darkblue}{\textbf{\ipa{ɲi˧}}}}{}
\textcolor{teal}{\zh{形容词}} \hspace{4pt} \zh{声调类:} M.
\zh{饱。} \textcolor{Sepia}{\selectlanguage{english}Full, satiated.} \textcolor{PineGreen}{\selectlanguage{french}Rassasié, repu.}  ¶ \textcolor{darkblue}{\textbf{\ipa{le˧-ɲi˧-ze˧}}} \zh{饱了} \textcolor{Sepia}{\selectlanguage{english}\mytextsc{accomp} \string_ \mytextsc{pfv}} \textcolor{PineGreen}{\selectlanguage{french}\mytextsc{accomp} \string_ \mytextsc{pfv}}  
 ¶ \textcolor{darkblue}{\textbf{\ipa{hɑ˧-ɲi˧(-ze˩)}}} \zh{吃饱了。 / 吃饱饭了。} \textcolor{Sepia}{\selectlanguage{english}I am full. / I am satiated.} \textcolor{PineGreen}{\selectlanguage{french}(je) suis rassasié}  
 ¶ \textcolor{darkblue}{\textbf{\ipa{njɤ˧ | le˧-ɲi˧-ze˧!}}} \zh{我饱了。} \textcolor{Sepia}{\selectlanguage{english}I am full. / I am satiated.} \textcolor{PineGreen}{\selectlanguage{french}je suis rassasié}  

\lhead{\firstmark}
\rhead{\botmark}

\subsection{\hspace{-0.5cm} {\Large \textcolor{darkblue}{\textbf{\ipa{ɲi˧\textsubscript{a}}}}}\hspace{0.5cm}[\kern2pt{\textcolor{darkblue}{\textbf{\ipa{ɲi˥}}}}\kern2pt]} \hypertarget{Ji\string_Ma1}{}
\markboth{\textcolor{darkblue}{\textbf{\ipa{ɲi˧\textsubscript{a}}}}}{}
\textcolor{teal}{\zh{动词}} \hspace{4pt} \zh{声调类:} M\textsubscript{a}.
\zh{需要。} \textcolor{Sepia}{\selectlanguage{english}To need.} \textcolor{PineGreen}{\selectlanguage{french}Avoir besoin de, vouloir.}  ¶ \textcolor{darkblue}{\textbf{\ipa{no˧ | ə˩-ɲi˧? | mɤ˧-ɲi˧!}}} \zh{你要吗?- 不要!} \textcolor{Sepia}{\selectlanguage{english}Do you want (some)? - No!} \textcolor{PineGreen}{\selectlanguage{french}Tu en veux? - (Non,) je n'en veux pas/je n'en ai pas besoin!}  

\lhead{\firstmark}
\rhead{\botmark}

\subsection{\hspace{-0.5cm} {\Large \textcolor{darkblue}{\textbf{\ipa{ɲi˧dʑɯ˧}}}}\hspace{0.5cm}[\kern2pt{\textcolor{darkblue}{\textbf{\ipa{ɲi˧dʑɯ˥}}}}\kern2pt]} \hypertarget{Ji\string_Mdz£M\string_M1}{}
\markboth{\textcolor{darkblue}{\textbf{\ipa{ɲi˧dʑɯ˧}}}}{}
\textcolor{teal}{\zh{名词}} \hspace{4pt} \zh{声调类:} H\#.
\zh{男生殖器。} \textcolor{Sepia}{\selectlanguage{english}Penis.} \textcolor{PineGreen}{\selectlanguage{french}Pénis, organe sexuel masculin.}  \zh{量词}: \textcolor{darkblue}{\textbf{\ipa{ɭɯ˧}}} 
\lhead{\firstmark}
\rhead{\botmark}

\subsection{\hspace{-0.5cm} {\Large \textcolor{darkblue}{\textbf{\ipa{ɲi˧gɤ\#˥}}}}\hspace{0.5cm}[\kern2pt{\textcolor{darkblue}{\textbf{\ipa{ɲi˧gɤ˧}}}}\kern2pt]} \hypertarget{Ji\string_Mg7\#\string_T1}{}
\markboth{\textcolor{darkblue}{\textbf{\ipa{ɲi˧gɤ\#˥}}}}{}
\textcolor{teal}{\zh{名词}} \hspace{4pt} \zh{声调类:} \#H.
\zh{鼻子。} \textcolor{Sepia}{\selectlanguage{english}Nose.} \textcolor{PineGreen}{\selectlanguage{french}Nez.}  \zh{量词}: \textcolor{darkblue}{\textbf{\ipa{ɭɯ˧}}} 
\lhead{\firstmark}
\rhead{\botmark}

\subsection{\hspace{-0.5cm} {\Large \textcolor{darkblue}{\textbf{\ipa{ɲi˧gɤ˧-bæ˧˥}}}}\hspace{0.5cm}[\kern2pt{\textcolor{darkblue}{\textbf{\ipa{xxxx non-correspondance entre le nombre de morphèmes et le nombre de tons de morphèmes}}}}\kern2pt]} \hypertarget{Ji\string_Mg7\string_M-b\{\string_M\string_T1}{}
\markboth{\textcolor{darkblue}{\textbf{\ipa{ɲi˧gɤ˧-bæ˧˥}}}}{}
\textcolor{teal}{\zh{名词}} \hspace{4pt} \zh{声调类:} MH\#.
\zh{牛鼻绳。也可以来指牛鼻圈。} \textcolor{Sepia}{\selectlanguage{english}Rope attached to a cow's nasal ring.} \textcolor{PineGreen}{\selectlanguage{french}Corde accrochée à l'anneau nasal, longe de vache; aussi utilisé par extension pour l'anneau nasal, pour lequel aucun terme propre n'existe.}  \zh{量词}: \textcolor{darkblue}{\textbf{\ipa{kʰɯ˩}}} 
\lhead{\firstmark}
\rhead{\botmark}

\subsection{\hspace{-0.5cm} {\Large \textcolor{darkblue}{\textbf{\ipa{ɲi˧ɬi˧mi˧}}}}\hspace{0.5cm}[\kern2pt{\textcolor{darkblue}{\textbf{\ipa{ɲi˧ɬi˧mi˧}}}}\kern2pt]} \hypertarget{Ji\string_MKi\string_Mmi\string_M1}{}
\markboth{\textcolor{darkblue}{\textbf{\ipa{ɲi˧ɬi˧mi˧}}}}{}
\textcolor{teal}{\zh{名词}} \hspace{4pt} \zh{声调类:} M.
\zh{二月。} \textcolor{Sepia}{\selectlanguage{english}Second month.} \textcolor{PineGreen}{\selectlanguage{french}Le deuxième mois.} 
\lhead{\firstmark}
\rhead{\botmark}

\subsection{\hspace{-0.5cm} {\Large \textcolor{darkblue}{\textbf{\ipa{ɲi˧mi˧}}}}\hspace{0.5cm}[\kern2pt{\textcolor{darkblue}{\textbf{\ipa{ɲi˧mi˧}}}}\kern2pt]} \hypertarget{Ji\string_Mmi\string_M1}{}
\markboth{\textcolor{darkblue}{\textbf{\ipa{ɲi˧mi˧}}}}{}
\textcolor{teal}{\zh{名词}} \hspace{4pt} \zh{声调类:} M.
\ding{202} \zh{太阳。} \textcolor{Sepia}{\selectlanguage{english}Sun.} \textcolor{PineGreen}{\selectlanguage{french}Soleil.}  ¶ \textcolor{darkblue}{\textbf{\ipa{ɲi˧mi˧ tʰv̩˧}}} \zh{太阳出来、日出} \textcolor{Sepia}{\selectlanguage{english}the sun rises} \textcolor{PineGreen}{\selectlanguage{french}le soleil se lève}  
 \zh{量词}: \textcolor{darkblue}{\textbf{\ipa{ɭɯ˧}}} \ding{203} \zh{日、时间。} \textcolor{Sepia}{\selectlanguage{english}Day; daytime; time.} \textcolor{PineGreen}{\selectlanguage{french}La journée; le temps.} 
\lhead{\firstmark}
\rhead{\botmark}

\subsection{\hspace{-0.5cm} {\Large \textcolor{darkblue}{\textbf{\ipa{ɲi˧mi˧dɑ˧dzɯ\#˥}}}}\hspace{0.5cm}[\kern2pt{\textcolor{darkblue}{\textbf{\ipa{ɲi˧mi˧dɑ˧dzɯ˧}}}}\kern2pt]} \hypertarget{Ji\string_Mmi\string_MdA\string_MdzM\#\string_T1}{}
\markboth{\textcolor{darkblue}{\textbf{\ipa{ɲi˧mi˧dɑ˧dzɯ\#˥}}}}{}
\textcolor{teal}{\zh{名词}} \hspace{4pt} \zh{声调类:} \#H.
\zh{日蚀。} \textcolor{Sepia}{\selectlanguage{english}Solar eclipse.} \textcolor{PineGreen}{\selectlanguage{french}Éclipse solaire.}  ¶ \textcolor{darkblue}{\textbf{\ipa{ɲi˧mi˧dɑ˧dzɯ˧ tʰv̩˧}}} \zh{有日蚀} \textcolor{Sepia}{\selectlanguage{english}there is a solar eclipse} \textcolor{PineGreen}{\selectlanguage{french}il y a une éclipse de soleil}  
 ¶ \textcolor{darkblue}{\textbf{\ipa{ɲi˧mi˧dɑ˧dzɯ˧ ɲi˥!}}} \zh{是的,是日蚀!} \textcolor{Sepia}{\selectlanguage{english}Yes, it's a solar eclipse!} \textcolor{PineGreen}{\selectlanguage{french}Oui, c'est bien une éclipse de soleil!}  
 \zh{量词}: \textcolor{darkblue}{\textbf{\ipa{ʂɯ˩}}} 
\lhead{\firstmark}
\rhead{\botmark}

\subsection{\hspace{-0.5cm} {\Large \textcolor{darkblue}{\textbf{\ipa{ɲi˧mi˧-gv̩˩}}}}\hspace{0.5cm}[\kern2pt{\textcolor{darkblue}{\textbf{\ipa{xxxx non-correspondance entre le nombre de morphèmes et le nombre de tons de morphèmes}}}}\kern2pt]} \hypertarget{Ji\string_Mmi\string_M-gv\string_=\string_B1}{}
\markboth{\textcolor{darkblue}{\textbf{\ipa{ɲi˧mi˧-gv̩˩}}}}{}
\textcolor{teal}{\zh{名词}} \hspace{4pt} \zh{声调类:} \mytextsc{L}.
\zh{西方。} \textcolor{Sepia}{\selectlanguage{english}West: “[the direction where] the sun sets”.} \textcolor{PineGreen}{\selectlanguage{french}Ouest; “[la direction dans laquelle] le soleil se couche”.}  ¶ \textcolor{darkblue}{\textbf{\ipa{ɲi˧mi˧gv̩˩-gi˩-dzɤ˩ se˩}}} \zh{往西边走} \textcolor{Sepia}{\selectlanguage{english}to walk towards the west} \textcolor{PineGreen}{\selectlanguage{french}marcher vers l'ouest}  

\lhead{\firstmark}
\rhead{\botmark}

\subsection{\hspace{-0.5cm} {\Large \textcolor{darkblue}{\textbf{\ipa{ɲi˧mi˧-kʰɯ˧ʂɯ˧}}}}\hspace{0.5cm}[\kern2pt{\textcolor{darkblue}{\textbf{\ipa{xxxx non-correspondance entre le nombre de morphèmes et le nombre de tons de morphèmes}}}}\kern2pt]} \hypertarget{Ji\string_Mmi\string_M-k\string_hM\string_Ms`M\string_M1}{}
\markboth{\textcolor{darkblue}{\textbf{\ipa{ɲi˧mi˧-kʰɯ˧ʂɯ˧}}}}{}
\textcolor{teal}{\zh{名词}} \hspace{4pt} \zh{声调类:} M.
\zh{太阳的光线。} \textcolor{Sepia}{\selectlanguage{english}Rays (of sunshine).} \textcolor{PineGreen}{\selectlanguage{french}Rayons du soleil.}  \zh{量词}: \textcolor{darkblue}{\textbf{\ipa{kʰɯ˩}}} 
\lhead{\firstmark}
\rhead{\botmark}

\subsection{\hspace{-0.5cm} {\Large \textcolor{darkblue}{\textbf{\ipa{ɲi˧mi˧tʰv̩\#˥}}}}\hspace{0.5cm}[\kern2pt{\textcolor{darkblue}{\textbf{\ipa{ɲi˧mi˧tʰv̩˧}}}}\kern2pt]} \hypertarget{Ji\string_Mmi\string_Mt\string_hv\string_=\#\string_T1}{}
\markboth{\textcolor{darkblue}{\textbf{\ipa{ɲi˧mi˧tʰv̩\#˥}}}}{}
\textcolor{teal}{\zh{名词}} \hspace{4pt} \zh{声调类:} \#H.
\zh{东方。} \textcolor{Sepia}{\selectlanguage{english}East, orient.} \textcolor{PineGreen}{\selectlanguage{french}Est, orient.}  ¶ \textcolor{darkblue}{\textbf{\ipa{ɲi˧mi˧tʰv̩˧-gi˧}}} \zh{东方方向} \textcolor{Sepia}{\selectlanguage{english}the direction of the east} \textcolor{PineGreen}{\selectlanguage{french}la direction de l'est}  
 ¶ \textcolor{darkblue}{\textbf{\ipa{ɲi˧mi˧tʰv̩˧-gi˧ | se˧}}} \zh{向东面走} \textcolor{Sepia}{\selectlanguage{english}to walk towards the east} \textcolor{PineGreen}{\selectlanguage{french}marcher vers l'est}  
 ¶ \textcolor{darkblue}{\textbf{\ipa{ɲi˧mi˧tʰv̩˧-gi˧ | dʑo˩˥}}} \zh{住在东方(合作人想象我在欧洲,想着她说:‘她住在东方’。)} \textcolor{Sepia}{\selectlanguage{english}to live in the East, to live in the Orient. (Context: the consultant imagines that I am in Europe, thinking of her, saying: 'She lives in the Orient'.)} \textcolor{PineGreen}{\selectlanguage{french}se trouver à l'est, habiter en Orient (contexte: la locutrice m'imagine, depuis l'Europe, pensant à elle, et disant: “elle habite en Orient”.}  

\lhead{\firstmark}
\rhead{\botmark}

\subsection{\hspace{-0.5cm} {\Large \textcolor{darkblue}{\textbf{\ipa{ɲi˧mi˧-ʈæ˧ʂɯ˧}}}}\hspace{0.5cm}[\kern2pt{\textcolor{darkblue}{\textbf{\ipa{xxxx non-correspondance entre le nombre de morphèmes et le nombre de tons de morphèmes}}}}\kern2pt]} \hypertarget{Ji\string_Mmi\string_M-t`\{\string_Ms`M\string_M1}{}
\markboth{\textcolor{darkblue}{\textbf{\ipa{ɲi˧mi˧-ʈæ˧ʂɯ˧}}}}{}
\textcolor{teal}{\zh{名词}} \hspace{4pt} \zh{声调类:} M.
\zh{葵花。} \textcolor{Sepia}{\selectlanguage{english}Sunflower.} \textcolor{PineGreen}{\selectlanguage{french}Tournesol.}  \zh{量词}: \textcolor{darkblue}{\textbf{\ipa{dzi˩}}} 
\lhead{\firstmark}
\rhead{\botmark}

\subsection{\hspace{-0.5cm} {\Large \textcolor{darkblue}{\textbf{\ipa{ɲi˧nɑ˩}}}}\hspace{0.5cm}[\kern2pt{\textcolor{darkblue}{\textbf{\ipa{ɲi˧nɑ˩}}}}\kern2pt]} \hypertarget{Ji\string_MnA\string_B1}{}
\markboth{\textcolor{darkblue}{\textbf{\ipa{ɲi˧nɑ˩}}}}{}
\textcolor{teal}{\zh{名词}} \hspace{4pt} \zh{声调类:} L\#.
\zh{藤子。} \textcolor{Sepia}{\selectlanguage{english}Cane; rattan.} \textcolor{PineGreen}{\selectlanguage{french}Liane, rattan, vigne vierge, lierre.} 
\lhead{\firstmark}
\rhead{\botmark}

\subsection{\hspace{-0.5cm} {\Large \textcolor{darkblue}{\textbf{\ipa{ɲi˧pʰv̩˩}}}}\hspace{0.5cm}[\kern2pt{\textcolor{darkblue}{\textbf{\ipa{ɲi˧pʰv̩˩}}}}\kern2pt]} \hypertarget{Ji\string_Mp\string_hv\string_=\string_B1}{}
\markboth{\textcolor{darkblue}{\textbf{\ipa{ɲi˧pʰv̩˩}}}}{}
\textcolor{teal}{\zh{名词}} \hspace{4pt} \zh{声调类:} L\#.
\zh{霜。} \textcolor{Sepia}{\selectlanguage{english}Frost.} \textcolor{PineGreen}{\selectlanguage{french}Givre.}  ¶ \textcolor{darkblue}{\textbf{\ipa{ɲi˧pʰv̩˩ lɑ˩-ze˩}}} \zh{有霜} \textcolor{Sepia}{\selectlanguage{english}there is some frost} \textcolor{PineGreen}{\selectlanguage{french}il y a du givre}  

\lhead{\firstmark}
\rhead{\botmark}

\subsection{\hspace{-0.5cm} {\Large \textcolor{darkblue}{\textbf{\ipa{ɲi˧qʰv̩˧}}}}\hspace{0.5cm}[\kern2pt{\textcolor{darkblue}{\textbf{\ipa{ɲi˧qʰv̩˧}}}}\kern2pt]} \hypertarget{Ji\string_Mq\string_hv\string_=\string_M1}{}
\markboth{\textcolor{darkblue}{\textbf{\ipa{ɲi˧qʰv̩˧}}}}{}
\textcolor{teal}{\zh{名词}} \hspace{4pt} \zh{声调类:} M.
\ding{202} \zh{鼻孔。} \textcolor{Sepia}{\selectlanguage{english}Nostril.} \textcolor{PineGreen}{\selectlanguage{french}Narine.}  \zh{量词}: \textcolor{darkblue}{\textbf{\ipa{ɭɯ˧}}} \ding{203} \zh{鼻涕。} \textcolor{Sepia}{\selectlanguage{english}Snivel, nasal mucus.} \textcolor{PineGreen}{\selectlanguage{french}Mucus, morve.} 
\lhead{\firstmark}
\rhead{\botmark}

\subsection{\hspace{-0.5cm} {\Large \textcolor{darkblue}{\textbf{\ipa{ɲi˧se˩}}}}\hspace{0.5cm}[\kern2pt{\textcolor{darkblue}{\textbf{\ipa{ɲi˧se˩}}}}\kern2pt]} \hypertarget{Ji\string_Mse\string_B1}{}
\markboth{\textcolor{darkblue}{\textbf{\ipa{ɲi˧se˩}}}}{}
\textcolor{teal}{\zh{名词}} \hspace{4pt} \zh{声调类:} L\#.
\zh{小落水(村落名)。} \textcolor{Sepia}{\selectlanguage{english}The name of a village.} \textcolor{PineGreen}{\selectlanguage{french}Un village du bord du Lac.}  ¶ \textcolor{darkblue}{\textbf{\ipa{ɲi˧se˩, | nɑ˩-lɑ˧ ɲi˥!}}} \zh{小落水,是纯摩梭的一个村落!} \textcolor{Sepia}{\selectlanguage{english}Nhissei is a thoroughly Na village! / Na is populated entirely by Na people!} \textcolor{PineGreen}{\selectlanguage{french}Nhissei, c'est un village entièrement Na!}  
 ¶ \textcolor{darkblue}{\textbf{\ipa{ɬi˧ki˧, | ɲi˧se˩, | tɑ˧dzi˩, | mv̩˧qʰwæ˩, | lɑ˧tʰɑ˧-di˧˥}}} \zh{永宁到泸沽湖所经过的村落,依次是:里格、尼赛、大祖、木垮,然后到拉塔地(拉塔地指的是泸沽湖周边的摩梭地区,包括左所、洛水村等)} \textcolor{Sepia}{\selectlanguage{english}Villages that one passes when moving away from the Yongning plain, towards Lugu lake. These villages do not count as part of Yongning proper. The last, \textcolor{darkblue}{\textbf{\ipa{/lɑ˧tʰɑ˧-di˧˥/}}}, is not a village name like the preceding four: it refers to the entire Na area beyond the fourth village.} \textcolor{PineGreen}{\selectlanguage{french}Villages dans l'ordre, après la plaine de Yongning, ne comptant pas comme faisant partie de Yongning. Le dernier, \textcolor{darkblue}{\textbf{\ipa{/lɑ˧tʰɑ˧-di˧˥/}}}, désigne toute la région na au-delà du quatrième village.}  

\lhead{\firstmark}
\rhead{\botmark}

\subsection{\hspace{-0.5cm} {\Large \textcolor{darkblue}{\textbf{\ipa{ɲi˧to˧}}}}\hspace{0.5cm}[\kern2pt{\textcolor{darkblue}{\textbf{\ipa{ɲi˧to˧}}}}\kern2pt]} \hypertarget{Ji\string_Mto\string_M1}{}
\markboth{\textcolor{darkblue}{\textbf{\ipa{ɲi˧to˧}}}}{}
\textcolor{teal}{\zh{名词}} \hspace{4pt} \zh{声调类:} M.
\zh{嘴巴,包括嘴周围的部位:颚等。} \textcolor{Sepia}{\selectlanguage{english}The mouth, seen as including the part of the face surrounding the mouth, in particular the jaw.} \textcolor{PineGreen}{\selectlanguage{french}Bouche/pourtour de la bouche (autour des lèvre).}  ¶ \textcolor{darkblue}{\textbf{\ipa{ɲi˧to˧ ʈʂʰwæ˩}}} \zh{多嘴、拉不断扯不断(直译:“嘴快”)} \textcolor{Sepia}{\selectlanguage{english}talkative} \textcolor{PineGreen}{\selectlanguage{french}bavard (littéralement “bouche rapide”)}  
 \zh{量词}: \textcolor{darkblue}{\textbf{\ipa{kʰwɤ˥}}} 
\lhead{\firstmark}
\rhead{\botmark}

\subsection{\hspace{-0.5cm} {\Large \textcolor{darkblue}{\textbf{\ipa{ɲi˧-ʈʂæ˧-ʑi˧˥}}}}\hspace{0.5cm}[\kern2pt{\textcolor{darkblue}{\textbf{\ipa{xxxx non-correspondance entre le nombre de morphèmes et le nombre de tons de morphèmes}}}}\kern2pt]} \hypertarget{Ji\string_M-t`s`\{\string_M-z£i\string_M\string_T1}{}
\markboth{\textcolor{darkblue}{\textbf{\ipa{ɲi˧-ʈʂæ˧-ʑi˧˥}}}}{}
\textcolor{teal}{\zh{名词}} \hspace{4pt} \zh{声调类:} MH\#.
\zh{二层房:农场里面的一栋楼,正对着农场大门。} \textcolor{Sepia}{\selectlanguage{english}The building inside the farm where the bedrooms are located, and a living-room (downstairs in the centre). Literally 'the two-floor building', as this is the only building that has rooms on two floors.} \textcolor{PineGreen}{\selectlanguage{french}Bâtiment d'habitation; littéralement 'le bâtiment à deux étages', car c'est le seul qui ait des salles sur deux étages. Ce bâtiment se trouve face à l'entrée de la ferme.}  ¶ \textcolor{darkblue}{\textbf{\ipa{ɲi˧-ʈʂæ˧-ʑi˧-di˥}}} \zh{同上} \textcolor{Sepia}{\selectlanguage{english}same meaning} \textcolor{PineGreen}{\selectlanguage{french}même sens}  
 \zh{量词}: \textcolor{darkblue}{\textbf{\ipa{ɭɯ˧}}} 
\lhead{\firstmark}
\rhead{\botmark}

\subsection{\hspace{-0.5cm} {\Large \textcolor{darkblue}{\textbf{\ipa{ɲi˧zo\#˥}}}}\hspace{0.5cm}[\kern2pt{\textcolor{darkblue}{\textbf{\ipa{ɲi˧zo˧}}}}\kern2pt]} \hypertarget{Ji\string_Mzo\#\string_T1}{}
\markboth{\textcolor{darkblue}{\textbf{\ipa{ɲi˧zo\#˥}}}}{}
\textcolor{teal}{\zh{名词}} \hspace{4pt} \zh{声调类:} \#H.
\zh{鱼。} \textcolor{Sepia}{\selectlanguage{english}Fish.} \textcolor{PineGreen}{\selectlanguage{french}Poisson.}  ¶ \textcolor{darkblue}{\textbf{\ipa{ɲi˧zo˧ tʰv̩˧-mi˥\# / ɲi˧zo˧ tʰv̩˧-mi˧˥}}} \zh{那条鱼} \textcolor{Sepia}{\selectlanguage{english}\mytextsc{n}+\mytextsc{dem}+\mytextsc{clf}} \textcolor{PineGreen}{\selectlanguage{french}\mytextsc{n}+\mytextsc{dem}+\mytextsc{clf}}  
 ¶ \textcolor{darkblue}{\textbf{\ipa{ɲi˧zo˧-tɑ˧pv̩˥}}} \zh{干鱼} \textcolor{Sepia}{\selectlanguage{english}dried fish} \textcolor{PineGreen}{\selectlanguage{french}poisson séché}  
 \zh{量词}: \textcolor{darkblue}{\textbf{\ipa{mi˩}}} 
\lhead{\firstmark}
\rhead{\botmark}

\subsection{\hspace{-0.5cm} {\Large \textcolor{darkblue}{\textbf{\ipa{ɲi˩}}} \textsubscript{1}}\hspace{0.5cm}[\kern2pt{\textcolor{darkblue}{\textbf{\ipa{ɲi˩˥}}}}\kern2pt]} \hypertarget{Ji\string_B1}{}
\markboth{\textcolor{darkblue}{\textbf{\ipa{ɲi˩}}} \textsubscript{1}}{}
\textcolor{teal}{\zh{动词}} \hspace{4pt} \zh{声调类:} L\textsubscript{a}.
\zh{夹、夹持。} \textcolor{Sepia}{\selectlanguage{english}To press, to hold (clamped under the arm, between the legs...).} \textcolor{PineGreen}{\selectlanguage{french}Serrer, tenir (ex.: tenir quelque chose serré sous le bras, serrer quelque chose entre les jambes).}  ¶ \textcolor{darkblue}{\textbf{\ipa{ɖɯ˧-ɲi˧\textasciitilde{}ɲi˥-ɻ̍˩}}} \zh{夹一点} \textcolor{Sepia}{\selectlanguage{english}to press  a little} \textcolor{PineGreen}{\selectlanguage{french}serrer un peu}  

\lhead{\firstmark}
\rhead{\botmark}

\subsection{\hspace{-0.5cm} {\Large \textcolor{darkblue}{\textbf{\ipa{ɲi˩}}} \textsubscript{2}}\hspace{0.5cm}[\kern2pt{\textcolor{darkblue}{\textbf{\ipa{ɲi˩˥}}}}\kern2pt]} \hypertarget{Ji\string_B2}{}
\markboth{\textcolor{darkblue}{\textbf{\ipa{ɲi˩}}} \textsubscript{2}}{}
\textcolor{teal}{\zh{代词}} \hspace{4pt} \zh{声调类:} L.
\zh{谁。} \textcolor{Sepia}{\selectlanguage{english}Who.} \textcolor{PineGreen}{\selectlanguage{french}Pronom interrogatif: qui?}  ¶ \textcolor{darkblue}{\textbf{\ipa{ɲi˩ ɲi˧?}}} \zh{是谁?} \textcolor{Sepia}{\selectlanguage{english}Who is that?} \textcolor{PineGreen}{\selectlanguage{french}C'est qui?}  
 ¶ \textcolor{darkblue}{\textbf{\ipa{no˧ | ɲi˩ ɲi˧?}}} \zh{你是谁?} \textcolor{Sepia}{\selectlanguage{english}Who are you?} \textcolor{PineGreen}{\selectlanguage{french}Qui êtes-vous?}  
 ¶ \textcolor{darkblue}{\textbf{\ipa{ʈʂʰɯ˧ | ɲi˩ ɲi˧?}}} \zh{他是谁?} \textcolor{Sepia}{\selectlanguage{english}Who is this person?} \textcolor{PineGreen}{\selectlanguage{french}Qui est-ce?}  
 ¶ \textcolor{darkblue}{\textbf{\ipa{no˧ | ɲi˩˥ | -ki˩ bi˩-pi˩, | ɖɯ˧-bæ˧ lɑ˧ ɲi˥!}}} \zh{无论你去谁(家),都一样!} \textcolor{Sepia}{\selectlanguage{english}No matter where you go, it's the same everywhere!} \textcolor{PineGreen}{\selectlanguage{french}Peu importe chez qui tu vas, c'est pareil partout!}  
 ¶ \textcolor{darkblue}{\textbf{\ipa{no˧ | ɲi˩-ki˥ bi˩-pi˩, | ɖɯ˧-bæ˧ lɑ˧ ɲi˥!}}} \zh{同上,声调段界不同} \textcolor{Sepia}{\selectlanguage{english}As previous example, with a different division into tone groups} \textcolor{PineGreen}{\selectlanguage{french}Comme l'exemple précédent, avec une division en groupes tonals différente}  

\lhead{\firstmark}
\rhead{\botmark}

\subsection{\hspace{-0.5cm} {\Large \textcolor{darkblue}{\textbf{\ipa{-ɲi˩}}}}\hspace{0.5cm}[\kern2pt{\textcolor{darkblue}{\textbf{\ipa{ɲi˩˥}}}}\kern2pt]} \hypertarget{-Ji\string_B1}{}
\markboth{\textcolor{darkblue}{\textbf{\ipa{-ɲi˩}}}}{}
\textcolor{teal}{\zh{语气助词}} \hspace{4pt} \zh{声调类:} L.
\zh{\mytextsc{肯定(°系词)。}} \textcolor{Sepia}{\selectlanguage{english}A particle derived from the copula, described by L. Lidz (2010:497) as conveying “an epistemic strategy that marks a high degree of certitude”.} \textcolor{PineGreen}{\selectlanguage{french}Particule indiquant la certitude; dérivée du verbe copule.} 
\lhead{\firstmark}
\rhead{\botmark}

\subsection{\hspace{-0.5cm} {\Large \textcolor{darkblue}{\textbf{\ipa{ɲi˩\textsubscript{a}}}} \textsubscript{1}}\hspace{0.5cm}[\kern2pt{\textcolor{darkblue}{\textbf{\ipa{ɲi˩˥}}}}\kern2pt]} \hypertarget{Ji\string_Ba1}{}
\markboth{\textcolor{darkblue}{\textbf{\ipa{ɲi˩\textsubscript{a}}}} \textsubscript{1}}{}
\textcolor{teal}{\zh{动词}} \hspace{4pt} \zh{声调类:} L\textsubscript{a}.
\zh{捻,缠线。} \textcolor{Sepia}{\selectlanguage{english}To twine, to wind; twist with the fingers (e.g. linen, to make thread).} \textcolor{PineGreen}{\selectlanguage{french}Tordre avec les doigts, enrouler, filer (pour fabriquer du fil de lin, pour tisser des vêtements).}  ¶ \textcolor{darkblue}{\textbf{\ipa{le˧-ɲi˩}}} \zh{\mytextsc{accomp}} \textcolor{Sepia}{\selectlanguage{english}\mytextsc{accomp}} \textcolor{PineGreen}{\selectlanguage{french}\mytextsc{accomp}}  
 ¶ \textcolor{darkblue}{\textbf{\ipa{sɑ˧ ɲi˥}}} \zh{捻麻} \textcolor{Sepia}{\selectlanguage{english}to twine hemp (to make thread)} \textcolor{PineGreen}{\selectlanguage{french}tordre le chanvre/le lin (pour faire du fil)}  
 ¶ \textcolor{darkblue}{\textbf{\ipa{ɖɯ˧-ɲi˧\textasciitilde{}ɲi˥-ɻ̍˩}}} \zh{捻一捻} \textcolor{Sepia}{\selectlanguage{english}\mytextsc{delimitative} \mytextsc{red} \mytextsc{inceptive}} \textcolor{PineGreen}{\selectlanguage{french}\mytextsc{délimitatif} \mytextsc{red} \mytextsc{inchoatif}}  

\lhead{\firstmark}
\rhead{\botmark}

\subsection{\hspace{-0.5cm} {\Large \textcolor{darkblue}{\textbf{\ipa{ɲi˩\textsubscript{a}}}} \textsubscript{2}}\hspace{0.5cm}[\kern2pt{\textcolor{darkblue}{\textbf{\ipa{ɲi˩˥}}}}\kern2pt]} \hypertarget{Ji\string_Ba2}{}
\markboth{\textcolor{darkblue}{\textbf{\ipa{ɲi˩\textsubscript{a}}}} \textsubscript{2}}{}
\textcolor{teal}{\zh{动词}} \hspace{4pt} \zh{声调类:} L\textsubscript{a}.
\zh{设备坏了。} \textcolor{Sepia}{\selectlanguage{english}To break (tool), to be broken.} \textcolor{PineGreen}{\selectlanguage{french}S'abîmer, se casser; tomber en panne (ex.: appareil photo).}  ¶ \textcolor{darkblue}{\textbf{\ipa{le˧-ɲi˩-ze˩}}} \zh{坏了!/破了!} \textcolor{Sepia}{\selectlanguage{english}\mytextsc{accomp} \string_ \mytextsc{pfv}: it's broken!} \textcolor{PineGreen}{\selectlanguage{french}\mytextsc{accomp} \string_ \mytextsc{pfv}: c'est cassé!}  
 ¶ \textcolor{darkblue}{\textbf{\ipa{tso˧\textasciitilde{}tso˧ ɲi˥}}} \zh{东西坏了} \textcolor{Sepia}{\selectlanguage{english}to break things} \textcolor{PineGreen}{\selectlanguage{french}casser des choses}  

\lhead{\firstmark}
\rhead{\botmark}

\subsection{\hspace{-0.5cm} {\Large \textcolor{darkblue}{\textbf{\ipa{ɲi˩\textsubscript{a}}}} \textsubscript{3}}\hspace{0.5cm}[\kern2pt{\textcolor{darkblue}{\textbf{\ipa{ɲi˩˥}}}}\kern2pt]} \hypertarget{Ji\string_Ba3}{}
\markboth{\textcolor{darkblue}{\textbf{\ipa{ɲi˩\textsubscript{a}}}} \textsubscript{3}}{}
\textcolor{teal}{\zh{动词}} \hspace{4pt} \zh{声调类:} L\textsubscript{a}.
\zh{是\mytextsc{系词。}} \textcolor{Sepia}{\selectlanguage{english}Copula.} \textcolor{PineGreen}{\selectlanguage{french}Verbe copule.} 
\lhead{\firstmark}
\rhead{\botmark}

\subsection{\hspace{-0.5cm} {\Large \textcolor{darkblue}{\textbf{\ipa{ɲi˩bv̩˩}}}}\hspace{0.5cm}[\kern2pt{\textcolor{darkblue}{\textbf{\ipa{ɲi˩bv̩˩˥}}}}\kern2pt]} \hypertarget{Ji\string_Bbv\string_=\string_B1}{}
\markboth{\textcolor{darkblue}{\textbf{\ipa{ɲi˩bv̩˩}}}}{}
\textcolor{teal}{\zh{名词}} \hspace{4pt} \zh{声调类:} L.
\zh{蟋蟀。} \textcolor{Sepia}{\selectlanguage{english}Grasshopper, cricket.} \textcolor{PineGreen}{\selectlanguage{french}Criquet.}  \zh{量词}: \textcolor{darkblue}{\textbf{\ipa{mi˩}}} 
\lhead{\firstmark}
\rhead{\botmark}

\subsection{\hspace{-0.5cm} {\Large \textcolor{darkblue}{\textbf{\ipa{ɲi˩bv̩˩-ʂe˩sɑ˧}}}}\hspace{0.5cm}[\kern2pt{\textcolor{darkblue}{\textbf{\ipa{xxxx non-correspondance entre le nombre de morphèmes et le nombre de tons de morphèmes}}}}\kern2pt]} \hypertarget{Ji\string_Bbv\string_=\string_B-s`e\string_BsA\string_M1}{}
\markboth{\textcolor{darkblue}{\textbf{\ipa{ɲi˩bv̩˩-ʂe˩sɑ˧}}}}{}
\textcolor{teal}{\zh{名词}} \hspace{4pt} \zh{声调类:} LM.
\zh{蜻蜓。} \textcolor{Sepia}{\selectlanguage{english}Dragonfly.} \textcolor{PineGreen}{\selectlanguage{french}Libellule.}  \zh{量词}: \textcolor{darkblue}{\textbf{\ipa{mi˩}}} 
\lhead{\firstmark}
\rhead{\botmark}

\subsection{\hspace{-0.5cm} {\Large \textcolor{darkblue}{\textbf{\ipa{ɲi˩mɑ\#˥}}}}\hspace{0.5cm}[\kern2pt{\textcolor{darkblue}{\textbf{\ipa{ɲi˩mɑ˥}}}}\kern2pt]} \hypertarget{Ji\string_BmA\#\string_T1}{}
\markboth{\textcolor{darkblue}{\textbf{\ipa{ɲi˩mɑ\#˥}}}}{}
\textcolor{teal}{\zh{名词}} \hspace{4pt} \zh{声调类:} LM+\#H.
\zh{男性名字,起给双胞胎中的老大。} \textcolor{Sepia}{\selectlanguage{english}Masculine given name used for the elder of two twins (the child who is born first).} \textcolor{PineGreen}{\selectlanguage{french}Prénom masculin pour l'aîné des jumeaux (l'enfant né en premier).} 
\lhead{\firstmark}
\rhead{\botmark}

\subsection{\hspace{-0.5cm} {\Large \textcolor{darkblue}{\textbf{\ipa{ɲi˩pʰv̩˩}}}}\hspace{0.5cm}[\kern2pt{\textcolor{darkblue}{\textbf{\ipa{ɲi˩pʰv̩˩˥}}}}\kern2pt]} \hypertarget{Ji\string_Bp\string_hv\string_=\string_B1}{}
\markboth{\textcolor{darkblue}{\textbf{\ipa{ɲi˩pʰv̩˩}}}}{}
\textcolor{teal}{\zh{名词}} \hspace{4pt} \zh{声调类:} L.
\zh{一种植物。合作人看水薄荷的图片就觉得像这种植物,但很可能不是。李达珠等(2015:98)翻译为“野牡丹”但这好像也不准确。} \textcolor{Sepia}{\selectlanguage{english}A mountain plant; the consultant proposes this term for water mint, \textit{Mentha aquatica, Mentha hirsuta Huds.} but this is unlikely to be the correct identification.} \textcolor{PineGreen}{\selectlanguage{french}Une plante de montagne; la locutrice pense la reconnaître sur une photo de menthe aquatique, \textit{Mentha aquatica, Mentha hirsuta Huds.} mais ce n'est vraisemblablement pas la bonne identification.}  ¶ \textcolor{darkblue}{\textbf{\ipa{ɲi˩pʰv̩˩-bæ˥bæ˩}}} \zh{这种植物的花} \textcolor{Sepia}{\selectlanguage{english}the flower of this plant} \textcolor{PineGreen}{\selectlanguage{french}la fleur de cette plante}  

\lhead{\firstmark}
\rhead{\botmark}

\subsection{\hspace{-0.5cm} {\Large \textcolor{darkblue}{\textbf{\ipa{ɲi˩=ɻ̍˥}}}}\hspace{0.5cm}[\kern2pt{\textcolor{darkblue}{\textbf{\ipa{ɲi˩ɻ̍˥}}}}\kern2pt]} \hypertarget{Ji\string_B=r£`̍\string_T1}{}
\markboth{\textcolor{darkblue}{\textbf{\ipa{ɲi˩=ɻ̍˥}}}}{}
\textcolor{teal}{\zh{代词}} \hspace{4pt} \zh{声调类:} LM+H\#.
\zh{第二人称,联想复数:你与周边的人(家人、家族、亲戚、朋友们……)。} \textcolor{Sepia}{\selectlanguage{english}Second person associative pronoun: you and your clan/family/friends.} \textcolor{PineGreen}{\selectlanguage{french}Pronom de 2e personne associatif: toi et les tiens.} 
\lhead{\firstmark}
\rhead{\botmark}

\subsection{\hspace{-0.5cm} {\Large \textcolor{darkblue}{\textbf{\ipa{ɲi˩tsɯ\#˥}}}}\hspace{0.5cm}[\kern2pt{\textcolor{darkblue}{\textbf{\ipa{ɲi˩tsɯ˥}}}}\kern2pt]} \hypertarget{Ji\string_BtsM\#\string_T1}{}
\markboth{\textcolor{darkblue}{\textbf{\ipa{ɲi˩tsɯ\#˥}}}}{}
\textcolor{teal}{\zh{名词}} \hspace{4pt} \zh{声调类:} LM+\#H.
\zh{苗族。} \textcolor{Sepia}{\selectlanguage{english}Hmong (ethnic group).} \textcolor{PineGreen}{\selectlanguage{french}Hmông (groupe ethnique).}  \zh{量词}: \textcolor{darkblue}{\textbf{\ipa{v̩˧}}} 
\lhead{\firstmark}
\rhead{\botmark}

\subsection{\hspace{-0.5cm} {\Large \textcolor{darkblue}{\textbf{\ipa{ɲi˩ʈʂe˩}}}}\hspace{0.5cm}[\kern2pt{\textcolor{darkblue}{\textbf{\ipa{ɲi˩ʈʂe˩˥}}}}\kern2pt]} \hypertarget{Ji\string_Bt`s`e\string_B1}{}
\markboth{\textcolor{darkblue}{\textbf{\ipa{ɲi˩ʈʂe˩}}}}{}
\textcolor{teal}{\zh{名词}} \hspace{4pt} \zh{声调类:} L.
\zh{门闩。} \textcolor{Sepia}{\selectlanguage{english}Door bar.} \textcolor{PineGreen}{\selectlanguage{french}Barre de porte: barre pour fermer la porte principale de la ferme.}  ¶ \textcolor{darkblue}{\textbf{\ipa{ɲi˩ʈʂe˩ tʰi˥-kʰɯ˩, | tʰi˧-ʈæ˩!}}} \zh{放门闩,(好好)锁(门)!} \textcolor{Sepia}{\selectlanguage{english}Put on the door bar, to lock (the main door)!} \textcolor{PineGreen}{\selectlanguage{french}On met la barre à la porte; on verrouille! / On met la barre à la porte, et c'est fermé!}  

\lhead{\firstmark}
\rhead{\botmark}

\subsection{\hspace{-0.5cm} {\Large \textcolor{darkblue}{\textbf{\ipa{ɲi˧˥}}}}\hspace{0.5cm}[\kern2pt{\textcolor{darkblue}{\textbf{\ipa{ɲi˧˥}}}}\kern2pt]} \hypertarget{Ji\string_M\string_T1}{}
\markboth{\textcolor{darkblue}{\textbf{\ipa{ɲi˧˥}}}}{}
\textcolor{teal}{\zh{数词}} \hspace{4pt} \zh{声调类:} MH.
\zh{2。} \textcolor{Sepia}{\selectlanguage{english}2.} \textcolor{PineGreen}{\selectlanguage{french}2.} 
\lhead{\firstmark}
\rhead{\botmark}

\newpage
\section*{\centering- \textcolor{darkblue}{\textbf{\ipa{ŋ}}} -}
\subsection{\hspace{-0.5cm} {\Large \textcolor{darkblue}{\textbf{\ipa{ŋæ˧ʝi˩}}}}\hspace{0.5cm}[\kern2pt{\textcolor{darkblue}{\textbf{\ipa{ŋæ˩ʝi˥}}}}\kern2pt]} \hypertarget{N\{\string_Mj££i\string_B1}{}
\markboth{\textcolor{darkblue}{\textbf{\ipa{ŋæ˧ʝi˩}}}}{}
\textcolor{teal}{\zh{形容词}} \hspace{4pt} \zh{声调类:} L\#.
\zh{安逸(汉语借词)。} \textcolor{Sepia}{\selectlanguage{english}Easy and comfortable, at ease.} \textcolor{PineGreen}{\selectlanguage{french}À l'aise, dans le confort, dans l'abondance.}  \zh{【借词】} \zh{安逸}

\lhead{\firstmark}
\rhead{\botmark}

\subsection{\hspace{-0.5cm} {\Large \textcolor{darkblue}{\textbf{\ipa{ŋɤ˩ŋɤ˩}}}}\hspace{0.5cm}[\kern2pt{\textcolor{darkblue}{\textbf{\ipa{ŋɤ˩ŋɤ˩˥}}}}\kern2pt]} \hypertarget{N7\string_BN7\string_B1}{}
\markboth{\textcolor{darkblue}{\textbf{\ipa{ŋɤ˩ŋɤ˩}}}}{}
\textcolor{teal}{\zh{名词}} \hspace{4pt} \zh{声调类:} L.
\zh{上腭。} \textcolor{Sepia}{\selectlanguage{english}Palate.} \textcolor{PineGreen}{\selectlanguage{french}Palais.}  \zh{量词}: \textcolor{darkblue}{\textbf{\ipa{kʰwɤ˥}}} 
\lhead{\firstmark}
\rhead{\botmark}

\subsection{\hspace{-0.5cm} {\Large \textcolor{darkblue}{\textbf{\ipa{ŋv̩˩}}}}\hspace{0.5cm}[\kern2pt{\textcolor{darkblue}{\textbf{\ipa{ŋv̩˥}}}}\kern2pt]} \hypertarget{Nv\string_=\string_B1}{}
\markboth{\textcolor{darkblue}{\textbf{\ipa{ŋv̩˩}}}}{}
\textcolor{teal}{\zh{名词}} \hspace{4pt} \zh{声调类:} L.
\ding{202} \zh{银子。} \textcolor{Sepia}{\selectlanguage{english}Silver; money.} \textcolor{PineGreen}{\selectlanguage{french}Argent (métal).}  ¶ \textcolor{darkblue}{\textbf{\ipa{ŋv˧hæ̃˩/ or et ærgent càd ærgent, pætrimoine}}} \zh{金钱、钱财、财富。直译:‘银子与金子’} \textcolor{Sepia}{\selectlanguage{english}money, wealth; literally 'silver and gold'} \textcolor{PineGreen}{\selectlanguage{french}argent, patrimoine, fortune; littéralement 'or et argent'}  
\ding{203} \zh{钱。} \textcolor{Sepia}{\selectlanguage{english}Money.} \textcolor{PineGreen}{\selectlanguage{french}Argent (argent-papier et pièces de monnaie).} 
\lhead{\firstmark}
\rhead{\botmark}

\subsection{\hspace{-0.5cm} {\Large \textcolor{darkblue}{\textbf{\ipa{ŋv̩˩\textsubscript{a}}}}}\hspace{0.5cm}[\kern2pt{\textcolor{darkblue}{\textbf{\ipa{ŋv̩˩˥}}}}\kern2pt]} \hypertarget{Nv\string_=\string_Ba1}{}
\markboth{\textcolor{darkblue}{\textbf{\ipa{ŋv̩˩\textsubscript{a}}}}}{}
\textcolor{teal}{\zh{动词}} \hspace{4pt} \zh{声调类:} L\textsubscript{a}.
\zh{哭。} \textcolor{Sepia}{\selectlanguage{english}To cry, to weep.} \textcolor{PineGreen}{\selectlanguage{french}Pleurer.}  ¶ \textcolor{darkblue}{\textbf{\ipa{(tʰi˧-)ŋv̩˧\textasciitilde{}ŋv̩˥}}} \zh{哭一场} \textcolor{Sepia}{\selectlanguage{english}\mytextsc{dur} \mytextsc{red}} \textcolor{PineGreen}{\selectlanguage{french}\mytextsc{dur} \mytextsc{red}}  

\lhead{\firstmark}
\rhead{\botmark}

\subsection{\hspace{-0.5cm} {\Large \textcolor{darkblue}{\textbf{\ipa{ŋwæ˧qʰv̩˧}}}}\hspace{0.5cm}[\kern2pt{\textcolor{darkblue}{\textbf{\ipa{ŋwæ˧qʰv̩˧}}}}\kern2pt]} \hypertarget{Nw\{\string_Mq\string_hv\string_=\string_M1}{}
\markboth{\textcolor{darkblue}{\textbf{\ipa{ŋwæ˧qʰv̩˧}}}}{}
\textcolor{teal}{\zh{名词}} \hspace{4pt} \zh{声调类:} M.
\zh{烧瓦的烤炉。} \textcolor{Sepia}{\selectlanguage{english}Oven to make bricks.} \textcolor{PineGreen}{\selectlanguage{french}Four où on cuit les tuiles.}  ¶ \textcolor{darkblue}{\textbf{\ipa{ŋwæ˧qʰv̩˧ ʂɯ˧-ʑi˩}}} \zh{‘瓦炉七家’:过去来指阿拉瓦村的人,当时那里只有七家住} \textcolor{Sepia}{\selectlanguage{english}'the seven families of the Brick Oven': an expression formerly used to designate the people from Alawa village, at a time when there were only seven families living there.} \textcolor{PineGreen}{\selectlanguage{french}'les sept familles du Four à tuiles': expression dont on désignait autrefois les gens du village de Alawa, du temps où il n'y avait là que sept familles}  
 ¶ \textcolor{darkblue}{\textbf{\ipa{ə˧lɑ˧-ʁwɤ˧ | ŋwæ˧qʰv̩˧ | tsʰe˧ɲi˧ ʑi˩}}} \zh{‘阿拉瓦瓦炉十二家’:过去来指阿拉瓦村的人,当时那里住的人家,从七家已经增加到十二家} \textcolor{Sepia}{\selectlanguage{english}'the twelve families of Alawa and the Brick Oven': an expression formerly used to designate the people from Alawa village, at a time when the number of families had increased from seven to twelve through migration.} \textcolor{PineGreen}{\selectlanguage{french}“les douze familles de Alawa”: expression dont on désignait autrefois les gens du village de Alawa, du temps où le nombre de familles était passé de sept à douze par l'arrivée de nouveaux venus.}  
 \zh{量词}: \textcolor{darkblue}{\textbf{\ipa{ɭɯ˧}}} 
\lhead{\firstmark}
\rhead{\botmark}

\subsection{\hspace{-0.5cm} {\Large \textcolor{darkblue}{\textbf{\ipa{ŋwɤ˧}}}}\hspace{0.5cm}[\kern2pt{\textcolor{darkblue}{\textbf{\ipa{ŋwɤ˥}}}}\kern2pt]} \hypertarget{Nw7\string_M1}{}
\markboth{\textcolor{darkblue}{\textbf{\ipa{ŋwɤ˧}}}}{}
\textcolor{teal}{\zh{数词}} \hspace{4pt} \zh{声调类:} M? H\#? (pas L).
\zh{5。} \textcolor{Sepia}{\selectlanguage{english}5.} \textcolor{PineGreen}{\selectlanguage{french}5.} 
\lhead{\firstmark}
\rhead{\botmark}

\subsection{\hspace{-0.5cm} {\Large \textcolor{darkblue}{\textbf{\ipa{ŋwɤ˧hɑ̃˩}}}}\hspace{0.5cm}[\kern2pt{\textcolor{darkblue}{\textbf{\ipa{ŋwɤ˧hɑ̃˩}}}}\kern2pt]} \hypertarget{Nw7\string_MhA\string_~\string_B1}{}
\markboth{\textcolor{darkblue}{\textbf{\ipa{ŋwɤ˧hɑ̃˩}}}}{}
\textcolor{teal}{\zh{名词}} \hspace{4pt} \zh{声调类:} L\#.
\zh{位于永宁西南的一座山。} \textcolor{Sepia}{\selectlanguage{english}A mountain to the South-West of Yongning.} \textcolor{PineGreen}{\selectlanguage{french}Nom d'une montagne au sud-ouest de Yongning.}  ¶ \textcolor{darkblue}{\textbf{\ipa{kɤ˧mv̩˧˥, | æ˧ʂæ˧, | ŋwɤ˧hɑ̃˩, | ʂwæ˧gv̩\#˥, | nɑ˩tsʰi˩˥ | -tɕʰɤ˧pɤ˧mi\#˥, | qv̩˧ɻ̍˧-ʈʂʰɑ˧nɑ˥ |}}} \zh{永宁地区有固定名字的六座山。其它的山,因为没有重要的象征意义,因此没有取名。} \textcolor{Sepia}{\selectlanguage{english}The six mountains of Yongning that carry a name and have a definite symbolic value. The other mountains do not have comparable symbolic value, and fewer people use specific names for them.} \textcolor{PineGreen}{\selectlanguage{french}Les six montagnes de Yongning qui portent un nom. Les autres sommets du voisinage n'ont pas une valeur symbolique comparable, et ne portent pas de nom communément utilisé.}  

\lhead{\firstmark}
\rhead{\botmark}

\subsection{\hspace{-0.5cm} {\Large \textcolor{darkblue}{\textbf{\ipa{ŋwɤ˧pʰæ˧˥}}}}\hspace{0.5cm}[\kern2pt{\textcolor{darkblue}{\textbf{\ipa{ŋwɤ˧pʰæ˧˥}}}}\kern2pt]} \hypertarget{Nw7\string_Mp\string_h\{\string_M\string_T1}{}
\markboth{\textcolor{darkblue}{\textbf{\ipa{ŋwɤ˧pʰæ˧˥}}}}{}
\textcolor{teal}{\zh{名词}} \hspace{4pt} \zh{声调类:} MH\#.
\zh{瓦(汉语借词)。} \textcolor{Sepia}{\selectlanguage{english}Tile.} \textcolor{PineGreen}{\selectlanguage{french}Tuile.}  \zh{【借词】} \zh{瓦}
 \zh{量词}: \textcolor{darkblue}{\textbf{\ipa{pʰæ˧˥}}} 
\lhead{\firstmark}
\rhead{\botmark}

\subsection{\hspace{-0.5cm} {\Large \textcolor{darkblue}{\textbf{\ipa{ŋwɤ˧qo˥}}}}\hspace{0.5cm}[\kern2pt{\textcolor{darkblue}{\textbf{\ipa{ŋwɤ˧qo˥}}}}\kern2pt]} \hypertarget{Nw7\string_Mqo\string_T1}{}
\markboth{\textcolor{darkblue}{\textbf{\ipa{ŋwɤ˧qo˥}}}}{}
\textcolor{teal}{\zh{名词}} \hspace{4pt} \zh{声调类:} H\#.
\zh{膝盖。} \textcolor{Sepia}{\selectlanguage{english}Knee.} \textcolor{PineGreen}{\selectlanguage{french}Genou.}  \zh{量词}: \textcolor{darkblue}{\textbf{\ipa{ɭɯ˧}}} \zh{~【参考】~} \hyperlink{}{\textcolor{darkblue}{\textbf{\ipa{ŋwɤ˩ɬv̩˧˥}}}} 
\lhead{\firstmark}
\rhead{\botmark}

\subsection{\hspace{-0.5cm} {\Large \textcolor{darkblue}{\textbf{\ipa{ŋwɤ˧tsʰi˩}}}}\hspace{0.5cm}[\kern2pt{\textcolor{darkblue}{\textbf{\ipa{ŋwɤ˧tsʰi˩}}}}\kern2pt]} \hypertarget{Nw7\string_Mts\string_hi\string_B1}{}
\markboth{\textcolor{darkblue}{\textbf{\ipa{ŋwɤ˧tsʰi˩}}}}{}
\textcolor{teal}{\zh{数词}} \hspace{4pt} \zh{声调类:} L\#.
\zh{50。} \textcolor{Sepia}{\selectlanguage{english}50.} \textcolor{PineGreen}{\selectlanguage{french}50.} 
\lhead{\firstmark}
\rhead{\botmark}

\subsection{\hspace{-0.5cm} {\Large \textcolor{darkblue}{\textbf{\ipa{ŋwɤ˩ɭɯ˧-tse˥pʰæ˩}}}}\hspace{0.5cm}[\kern2pt{\textcolor{darkblue}{\textbf{\ipa{ŋwɤ˩ɭɯ˧tse˥pʰæ˩}}}}\kern2pt]} \hypertarget{Nw7\string_Bl\string_RM\string_M-tse\string_Tp\string_h\{\string_B1}{}
\markboth{\textcolor{darkblue}{\textbf{\ipa{ŋwɤ˩ɭɯ˧-tse˥pʰæ˩}}}}{}
\textcolor{teal}{\zh{名词}} \hspace{4pt} \zh{声调类:} LM+\#H-.
\zh{膝盖骨。} \textcolor{Sepia}{\selectlanguage{english}Kneebone.} \textcolor{PineGreen}{\selectlanguage{french}Os du genou.}  \zh{量词}: \textcolor{darkblue}{\textbf{\ipa{ɭɯ˧}}} 
\lhead{\firstmark}
\rhead{\botmark}

\subsection{\hspace{-0.5cm} {\Large \textcolor{darkblue}{\textbf{\ipa{ŋwɤ˩ɬi˩mi˩}}}}\hspace{0.5cm}[\kern2pt{\textcolor{darkblue}{\textbf{\ipa{ŋwɤ˩ɬi˩mi˩˥}}}}\kern2pt]} \hypertarget{Nw7\string_BKi\string_Bmi\string_B1}{}
\markboth{\textcolor{darkblue}{\textbf{\ipa{ŋwɤ˩ɬi˩mi˩}}}}{}
\textcolor{teal}{\zh{名词}} \hspace{4pt} \zh{声调类:} L.
\zh{五月。} \textcolor{Sepia}{\selectlanguage{english}5th month.} \textcolor{PineGreen}{\selectlanguage{french}5e mois.} 
\lhead{\firstmark}
\rhead{\botmark}

\subsection{\hspace{-0.5cm} {\Large \textcolor{darkblue}{\textbf{\ipa{ŋwɤ˩ɬv̩˧˥}}}}\hspace{0.5cm}[\kern2pt{\textcolor{darkblue}{\textbf{\ipa{ŋwɤ˩ɬv̩˧˥}}}}\kern2pt]} \hypertarget{Nw7\string_BKv\string_=\string_M\string_T1}{}
\markboth{\textcolor{darkblue}{\textbf{\ipa{ŋwɤ˩ɬv̩˧˥}}}}{}
\textcolor{teal}{\zh{名词}} \hspace{4pt} \zh{声调类:} LM+MH\#.
\zh{膝盖(直译:“膝盖髓”)。这个说法强调膝盖的脆弱。} \textcolor{Sepia}{\selectlanguage{english}Cartilages of the knee; literally “marrow of the knee”. This expression emphasizes the fragility of this articulation.} \textcolor{PineGreen}{\selectlanguage{french}Genou, cartilages du genou, articulation du genou: littéralement “moëlle du genou”. L'expression insiste sur le caractère fragile de cette articulation.}  ¶ \textcolor{darkblue}{\textbf{\ipa{[M23] ŋwɤ˩ɬv̩˧-ko˧lo˥ go˩.}}} \zh{膝盖疼。} \textcolor{Sepia}{\selectlanguage{english}to feel pain inside the knee} \textcolor{PineGreen}{\selectlanguage{french}avoir mal dans le genou}  
 \zh{量词}: \textcolor{darkblue}{\textbf{\ipa{ɭɯ˧}}} \zh{~【参考】~} \hyperlink{}{\textcolor{darkblue}{\textbf{\ipa{ŋwɤ˧qo˥}}}} 
\lhead{\firstmark}
\rhead{\botmark}

\subsection{\hspace{-0.5cm} {\Large \textcolor{darkblue}{\textbf{\ipa{ŋwɤ˧˥}}}}\hspace{0.5cm}[\kern2pt{\textcolor{darkblue}{\textbf{\ipa{ŋwɤ˧˥}}}}\kern2pt]} \hypertarget{Nw7\string_M\string_T1}{}
\markboth{\textcolor{darkblue}{\textbf{\ipa{ŋwɤ˧˥}}}}{}
\textcolor{teal}{\zh{动词}} \hspace{4pt} \zh{声调类:} MH.
\zh{刺痛。} \textcolor{Sepia}{\selectlanguage{english}To sting, to pierce.} \textcolor{PineGreen}{\selectlanguage{french}Percer, piquer.}  ¶ \textcolor{darkblue}{\textbf{\ipa{tɕʰi˧ ŋwɤ˩-ze˩}}} \zh{(他)被刺扎疼了。} \textcolor{Sepia}{\selectlanguage{english}(He/she) was stung by a thorn} \textcolor{PineGreen}{\selectlanguage{french}(Elle/il) a pris une écharde}  

\lhead{\firstmark}
\rhead{\botmark}

\newpage
\section*{\centering- \textcolor{darkblue}{\textbf{\ipa{õ}}} -}
\subsection{\hspace{-0.5cm} {\Large \textcolor{darkblue}{\textbf{\ipa{õ˧dɤ˧ɻ̍˧}}}}\hspace{0.5cm}[\kern2pt{\textcolor{darkblue}{\textbf{\ipa{õ˧dɤ˧ɻ̍˧}}}}\kern2pt]} \hypertarget{o\string_~\string_Md7\string_Mr£`̍\string_M1}{}
\markboth{\textcolor{darkblue}{\textbf{\ipa{õ˧dɤ˧ɻ̍˧}}}}{}
\textcolor{teal}{\zh{名词}} \hspace{4pt} \zh{声调类:} M.
\zh{根本。} \textcolor{Sepia}{\selectlanguage{english}Foundation, fundamentals.} \textcolor{PineGreen}{\selectlanguage{french}Fondement/fondamentalement.}  ¶ \textcolor{darkblue}{\textbf{\ipa{õ˧dɤ˧ɻ̍˧-ɳɯ˧, | hĩ˧ ʈʂʰɯ˧-v̩˧ | ʈʂʰɯ˧ne˧ gv̩˧˥ | -ɲi˩!}}} \zh{他原来是这样做事情的! / 他原来这么不懂事!} \textcolor{Sepia}{\selectlanguage{english}So that is how he really behaves / does! (Comment on someone whose behaviour is not respectful of good manners)} \textcolor{PineGreen}{\selectlanguage{french}Voilà comment il se comporte en réalité/au fond! (Se dit de quelqu'un dont le comportement est irrespectueux des règles de savoir-vivre)}  

\lhead{\firstmark}
\rhead{\botmark}

\subsection{\hspace{-0.5cm} {\Large \textcolor{darkblue}{\textbf{\ipa{õ˧ʈʂwɤ˧}}}}\hspace{0.5cm}[\kern2pt{\textcolor{darkblue}{\textbf{\ipa{õ˧ʈʂwɤ˧}}}}\kern2pt]} \hypertarget{o\string_~\string_Mt`s`w7\string_M1}{}
\markboth{\textcolor{darkblue}{\textbf{\ipa{õ˧ʈʂwɤ˧}}}}{}
\textcolor{teal}{\zh{名词}} \hspace{4pt} \zh{声调类:} M.
\zh{蚊子。} \textcolor{Sepia}{\selectlanguage{english}Mosquito.} \textcolor{PineGreen}{\selectlanguage{french}Moustique.}  ¶ \textcolor{darkblue}{\textbf{\ipa{õ˧ʈʂwɤ˧ le˧-tʰv̩˧-ze˧!}}} \zh{有一只蚊子!} \textcolor{Sepia}{\selectlanguage{english}Here comes a mosquito! / A mosquito has come in! (=into the room, into the mosquito net...)} \textcolor{PineGreen}{\selectlanguage{french}voilà un moustique! / un moustique est entré (dans la pièce, sous la moustiquaire…)}  
 ¶ \textcolor{darkblue}{\textbf{\ipa{ʂɯ˧-ɬi˧mi˧, | õ˧ʈʂwɤ˧! |}}} \zh{七月份,蚊子多! / 七月份,是蚊子多的一个月!} \textcolor{Sepia}{\selectlanguage{english}In the seventh month, there are lots of mosquitoes!} \textcolor{PineGreen}{\selectlanguage{french}Le septième mois, c'est un mois à moustiques!}  
 \zh{量词}: \textcolor{darkblue}{\textbf{\ipa{mi˩}}} 
\lhead{\firstmark}
\rhead{\botmark}

\subsection{\hspace{-0.5cm} {\Large \textcolor{darkblue}{\textbf{\ipa{õ˧ʈʂʰɯ˧ne˧-ʝi˥}}}}\hspace{0.5cm}[\kern2pt{\textcolor{darkblue}{\textbf{\ipa{xxxx non-correspondance entre le nombre de morphèmes et le nombre de tons de morphèmes}}}}\kern2pt]} \hypertarget{o\string_~\string_Mt`s`\string_hM\string_Mne\string_M-j££i\string_T1}{}
\markboth{\textcolor{darkblue}{\textbf{\ipa{õ˧ʈʂʰɯ˧ne˧-ʝi˥}}}}{}
\textcolor{teal}{\zh{助词}} \hspace{4pt} \zh{声调类:} MH\#.
\zh{那样。} \textcolor{Sepia}{\selectlanguage{english}In that way.} \textcolor{PineGreen}{\selectlanguage{french}De cette façon-là.} 
\lhead{\firstmark}
\rhead{\botmark}

\subsection{\hspace{-0.5cm} {\Large \textcolor{darkblue}{\textbf{\ipa{õ˩dv̩˧˥}}}}\hspace{0.5cm}[\kern2pt{\textcolor{darkblue}{\textbf{\ipa{õ˩dv̩˧˥}}}}\kern2pt]} \hypertarget{o\string_~\string_Bdv\string_=\string_M\string_T1}{}
\markboth{\textcolor{darkblue}{\textbf{\ipa{õ˩dv̩˧˥}}}}{}
\textcolor{teal}{\zh{名词}} \hspace{4pt} \zh{声调类:} LM+MH\#.
\zh{狼。} \textcolor{Sepia}{\selectlanguage{english}Wolf.} \textcolor{PineGreen}{\selectlanguage{french}Loup.}  \zh{量词}: \textcolor{darkblue}{\textbf{\ipa{mi˩}}} 
\lhead{\firstmark}
\rhead{\botmark}

\subsection{\hspace{-0.5cm} {\Large \textcolor{darkblue}{\textbf{\ipa{õ˩dv̩˧-kʰv̩˥mi˩}}}}\hspace{0.5cm}[\kern2pt{\textcolor{darkblue}{\textbf{\ipa{õ˩dv̩˧˥kʰv̩˧mi˧}}}}\kern2pt]} \hypertarget{o\string_~\string_Bdv\string_=\string_M-k\string_hv\string_=\string_Tmi\string_B1}{}
\markboth{\textcolor{darkblue}{\textbf{\ipa{õ˩dv̩˧-kʰv̩˥mi˩}}}}{}
\textcolor{teal}{\zh{名词}} \hspace{4pt} \zh{声调类:} LM+MH\#-.
\zh{狼狗。} \textcolor{Sepia}{\selectlanguage{english}Wolfhound.} \textcolor{PineGreen}{\selectlanguage{french}Chien-loup.}  \zh{量词}: \textcolor{darkblue}{\textbf{\ipa{mi˩}}} 
\lhead{\firstmark}
\rhead{\botmark}

\subsection{\hspace{-0.5cm} {\Large \textcolor{darkblue}{\textbf{\ipa{õ˩dv̩˧-mi˥}}}}\hspace{0.5cm}[\kern2pt{\textcolor{darkblue}{\textbf{\ipa{xxxx non-correspondance entre le nombre de morphèmes et le nombre de tons de morphèmes}}}}\kern2pt]} \hypertarget{o\string_~\string_Bdv\string_=\string_M-mi\string_T1}{}
\markboth{\textcolor{darkblue}{\textbf{\ipa{õ˩dv̩˧-mi˥}}}}{}
\textcolor{teal}{\zh{名词}} \hspace{4pt} \zh{声调类:} LM+H\#.
\zh{母狼。} \textcolor{Sepia}{\selectlanguage{english}Female wolf.} \textcolor{PineGreen}{\selectlanguage{french}Louve.}  \zh{量词}: \textcolor{darkblue}{\textbf{\ipa{mi˩}}} 
\lhead{\firstmark}
\rhead{\botmark}

\subsection{\hspace{-0.5cm} {\Large \textcolor{darkblue}{\textbf{\ipa{õ˩dv̩˧-pʰv̩\#˥}}}}\hspace{0.5cm}[\kern2pt{\textcolor{darkblue}{\textbf{\ipa{xxxx non-correspondance entre le nombre de morphèmes et le nombre de tons de morphèmes}}}}\kern2pt]} \hypertarget{o\string_~\string_Bdv\string_=\string_M-p\string_hv\string_=\#\string_T1}{}
\markboth{\textcolor{darkblue}{\textbf{\ipa{õ˩dv̩˧-pʰv̩\#˥}}}}{}
\textcolor{teal}{\zh{名词}} \hspace{4pt} \zh{声调类:} LM+\#H.
\zh{公狼。} \textcolor{Sepia}{\selectlanguage{english}Male wolf.} \textcolor{PineGreen}{\selectlanguage{french}Loup mâle.}  \zh{量词}: \textcolor{darkblue}{\textbf{\ipa{mi˩}}} 
\lhead{\firstmark}
\rhead{\botmark}

\subsection{\hspace{-0.5cm} {\Large \textcolor{darkblue}{\textbf{\ipa{õ˩dv̩˧-zo\#˥}}}}\hspace{0.5cm}[\kern2pt{\textcolor{darkblue}{\textbf{\ipa{õ˩dv̩˧zo˥}}}}\kern2pt]} \hypertarget{o\string_~\string_Bdv\string_=\string_M-zo\#\string_T1}{}
\markboth{\textcolor{darkblue}{\textbf{\ipa{õ˩dv̩˧-zo\#˥}}}}{}
\textcolor{teal}{\zh{名词}} \hspace{4pt} \zh{声调类:} LM+\#H-.
\zh{小狼。} \textcolor{Sepia}{\selectlanguage{english}Little wolf.} \textcolor{PineGreen}{\selectlanguage{french}Louveteau.} 
\lhead{\firstmark}
\rhead{\botmark}

\subsection{\hspace{-0.5cm} {\Large \textcolor{darkblue}{\textbf{\ipa{õ˧˥}}}}\hspace{0.5cm}[\kern2pt{\textcolor{darkblue}{\textbf{\ipa{õ˧˥}}}}\kern2pt]} \hypertarget{o\string_~\string_M\string_T1}{}
\markboth{\textcolor{darkblue}{\textbf{\ipa{õ˧˥}}}}{}
\textcolor{teal}{\zh{代词}} \hspace{4pt} \zh{声调类:} MH.
\zh{自己。} \textcolor{Sepia}{\selectlanguage{english}(one)self.} \textcolor{PineGreen}{\selectlanguage{french}Soi-même, propre.}  ¶ \textcolor{darkblue}{\textbf{\ipa{õ˧-ɑ˥ʁo˩}}} \zh{自己家} \textcolor{Sepia}{\selectlanguage{english}one's house} \textcolor{PineGreen}{\selectlanguage{french}sa propre maison}  
 ¶ \textcolor{darkblue}{\textbf{\ipa{õ˧-dʑɯ˥, õ˩ ʈʰɯ˩! |}}} \zh{自己喝自己的!(情景:一个婴儿抓另一个婴儿的奶瓶。)} \textcolor{Sepia}{\selectlanguage{english}Each drinks from her own bottle! (Context: a toddler has grabbed another's milk bottle; parents prevent her from drinking from it.)} \textcolor{PineGreen}{\selectlanguage{french}Chacun boit sa propre boisson! (Contexte: un petit enfant s'empare du biberon d'un autre et s'apprête à boire; on l'en empêche.)}  
 ¶ \textcolor{darkblue}{\textbf{\ipa{õ˧-ʂe˥, õ˩ ʈʰæ˩! |}}} \zh{自己吃自己的(那块)肉!(关于饮食习惯:吃饭的时候,每人分得一块肉,自己吃完。当地人认为,汉族没有这种分吃的习惯。)} \textcolor{Sepia}{\selectlanguage{english}Each person eats their own slab of meat! (Describing table manners: each person used to receive one slice of meat and eat it up, unlike Chinese custom, in which each guest picks food mouthful by mouthful, with chopsticks, from the dishes placed on the table.)} \textcolor{PineGreen}{\selectlanguage{french}chacun mange son propre morceau de viande! (Description des manières de table: dans le temps, on donnait un bout de viande à chacun et chacun mangeait son morceau, pas comme la coutume chinoise qui veut qu'on prélève bouchée par bouchée, avec ses baguettes, dans les bols/assiettes posés sur la table.)}  
 ¶ \textcolor{darkblue}{\textbf{\ipa{õ˧-bv̩˥-õ˩ ʝi˩-ɳɯ˩ | sɯ˧-kv̩˩!}}} \zh{自己做,就能学会!/ 要学会,就得自己熟练!} \textcolor{Sepia}{\selectlanguage{english}One learns by practising oneself! / It's by practising oneself that one really masters a skill!} \textcolor{PineGreen}{\selectlanguage{french}c'est en faisant soi-même qu'on apprend!}  
 ¶ \textcolor{darkblue}{\textbf{\ipa{õ˧-bv̩˥-õ˩ +N |}}} \zh{自己的(+名词)} \textcolor{Sepia}{\selectlanguage{english}one's own N} \textcolor{PineGreen}{\selectlanguage{french}son propre N (soi-même+\mytextsc{poss}+soi-même)}  
 ¶ \textcolor{darkblue}{\textbf{\ipa{õ˧-bv̩˥-õ˩ ʐwæ˩}}} \zh{自己的马} \textcolor{Sepia}{\selectlanguage{english}one's own horse} \textcolor{PineGreen}{\selectlanguage{french}son propre cheval}  
 ¶ \textcolor{darkblue}{\textbf{\ipa{õ˧-bv̩˥-õ˩ ʝi˩}}} \zh{自己的牛} \textcolor{Sepia}{\selectlanguage{english}one's own cow} \textcolor{PineGreen}{\selectlanguage{french}sa propre vache}  
 ¶ \textcolor{darkblue}{\textbf{\ipa{õ˧-bv̩˥-õ˩ lv̩˩}}} \zh{自己的田地} \textcolor{Sepia}{\selectlanguage{english}one's own field} \textcolor{PineGreen}{\selectlanguage{french}son propre champ}  
 ¶ \textcolor{darkblue}{\textbf{\ipa{õ˧-bv̩˥-õ˩ ɖʐe˩}}} \zh{自己的钱} \textcolor{Sepia}{\selectlanguage{english}one's own money} \textcolor{PineGreen}{\selectlanguage{french}son propre argent}  
 ¶ \textcolor{darkblue}{\textbf{\ipa{õ˧mv̩˥-õ˩di˩}}} \zh{出生的地方、老家、故乡} \textcolor{Sepia}{\selectlanguage{english}birth place} \textcolor{PineGreen}{\selectlanguage{french}lieu de naissance, lieu d'origine}  
 ¶ \textcolor{darkblue}{\textbf{\ipa{hĩ˧-mv˥ hĩ˩-di˩ | qʰɑ˧-dʑɤ˥\textasciitilde{}dʑɤ˩, | õ˧-mv˥ õ˩-di˩ tsʰe˩ mɤ˩-gv˩!}}} \zh{其他人的地方怎么好,也比不过自己的地方!} \textcolor{Sepia}{\selectlanguage{english}No matter how beautiful other people's places are, they can never be equal to one's own homeland!} \textcolor{PineGreen}{\selectlanguage{french}Si belles soient les terres d'autrui, elles n'auront jamais la beauté de ses propres terres / de la terre natale !}  
 ¶ \textcolor{darkblue}{\textbf{\ipa{õ˧-ə˧mv̩˥ / õ˧-ə˥mv̩˩ / õ˧-ə˧mv̩˧˥}}} \zh{自家姐姐(或哥哥)} \textcolor{Sepia}{\selectlanguage{english}one's own elder (brother or sister)} \textcolor{PineGreen}{\selectlanguage{french}son propre aîné (frère ou soeur)}  
 ¶ \textcolor{darkblue}{\textbf{\ipa{õ˧-ə˧v̩˥ / õ˧-ə˥v̩˩}}} \zh{自家舅舅(母亲的兄弟)} \textcolor{Sepia}{\selectlanguage{english}one's own maternal uncle} \textcolor{PineGreen}{\selectlanguage{french}son propre oncle}  
 ¶ \textcolor{darkblue}{\textbf{\ipa{õ˧-ʐɤ˥mi˩, õ˩ ɲi˩! |}}} \zh{自己的道路,就是自己!/ 每个人有自己的命运!} \textcolor{Sepia}{\selectlanguage{english}One's path, that is one's identity / one's destiny! / The path you choose is your destiny!} \textcolor{PineGreen}{\selectlanguage{french}Chacun a son chemin! / Chacun vit sa vie! / A chacun sa destinée!}  

\lhead{\firstmark}
\rhead{\botmark}

\newpage
\section*{\centering- \textcolor{darkblue}{\textbf{\ipa{p}}} -}
\subsection{\hspace{-0.5cm} {\Large \textcolor{darkblue}{\textbf{\ipa{pɑ˧tɕɤ˧}}}}\hspace{0.5cm}[\kern2pt{\textcolor{darkblue}{\textbf{\ipa{pɑ˧tɕɤ˥}}}}\kern2pt]} \hypertarget{pA\string_Mts£7\string_M1}{}
\markboth{\textcolor{darkblue}{\textbf{\ipa{pɑ˧tɕɤ˧}}}}{}
\textcolor{teal}{\zh{名词}} \hspace{4pt} \zh{声调类:} M.
\zh{芭蕉(汉语借词)。} \textcolor{Sepia}{\selectlanguage{english}Plantain.} \textcolor{PineGreen}{\selectlanguage{french}Bananier plantain.}  \zh{【借词】} \zh{芭蕉}

\lhead{\firstmark}
\rhead{\botmark}

\subsection{\hspace{-0.5cm} {\Large \textcolor{darkblue}{\textbf{\ipa{pæ˥}}}}\hspace{0.5cm}[\kern2pt{\textcolor{darkblue}{\textbf{\ipa{pæ˥}}}}\kern2pt]} \hypertarget{p\{\string_T1}{}
\markboth{\textcolor{darkblue}{\textbf{\ipa{pæ˥}}}}{}
\textcolor{teal}{\zh{动词}} \hspace{4pt} \zh{声调类:} H.
\zh{搬(家)。} \textcolor{Sepia}{\selectlanguage{english}To move house.} \textcolor{PineGreen}{\selectlanguage{french}Déménager.}  \zh{【借词】} \zh{搬?}

\lhead{\firstmark}
\rhead{\botmark}

\subsection{\hspace{-0.5cm} {\Large \textcolor{darkblue}{\textbf{\ipa{pæ˥\textsubscript{a}}}}}\hspace{0.5cm}[\kern2pt{\textcolor{darkblue}{\textbf{\ipa{pæ˥}}}}\kern2pt]} \hypertarget{p\{\string_Ta1}{}
\markboth{\textcolor{darkblue}{\textbf{\ipa{pæ˥\textsubscript{a}}}}}{}
\textcolor{teal}{\zh{量词}} \hspace{4pt} \zh{声调类:} H\textsubscript{a}.
\zh{量词:马、军人……(一队)。} \textcolor{Sepia}{\selectlanguage{english}Classifier for packs/herds (of horses...), troops (of soldiers)...} \textcolor{PineGreen}{\selectlanguage{french}Troupe (de chevaux, de soldats…).} 
\lhead{\firstmark}
\rhead{\botmark}

\subsection{\hspace{-0.5cm} {\Large \textcolor{darkblue}{\textbf{\ipa{pæ˧kʰwɤ\#˥}}}}\hspace{0.5cm}[\kern2pt{\textcolor{darkblue}{\textbf{\ipa{pæ˧kʰwɤ˧˥}}}}\kern2pt]} \hypertarget{p\{\string_Mk\string_hw7\#\string_T1}{}
\markboth{\textcolor{darkblue}{\textbf{\ipa{pæ˧kʰwɤ\#˥}}}}{}
\textcolor{teal}{\zh{名词}} \hspace{4pt} \zh{声调类:} \#H.
\zh{民国之前的银币。} \textcolor{Sepia}{\selectlanguage{english}Silver coin of the imperial times.} \textcolor{PineGreen}{\selectlanguage{french}Pièce d'argent de l'époque impériale.}  ¶ \textcolor{darkblue}{\textbf{\ipa{ə˧mi˧! | pæ˧kʰwɤ˧ so˧-ɭɯ˥ ki˩-mæ˩!}}} \zh{哇!(他)给三块银币!(在一个孩子成年时,亲戚会给银币。给一块,不合适,因为礼物不能只给一个,要给两个。给两块银币,是合适的,也是够的。给三块银币,超出期望,是大礼物了。按现在的标准/说法,三个银币等于半个月的工资左右。)} \textcolor{Sepia}{\selectlanguage{english}Wow! [(S)he] is giving you three silver coins!! (According to the main consultant's memories, this is the type of comment that uncles and aunts would make when a child who turned 13 received significant amounts of money on the occasion of their coming of age. The equivalent today would be about half a month's salary. To give only one coin would not be right, because gifts have to come in pairs. To give two coins is fully sufficient: a beautiful gift. To give three coins is an impressive gift, beyond expectations.)} \textcolor{PineGreen}{\selectlanguage{french}Waouuu! [Il/elle] te donne trois pièces d'argent! (D'après le souvenir qu'en a la consultante principale, c'est le type de commentaire que faisaient autrefois les tantes ou oncles d'un enfant à qui on offrait une forte somme d'argent à l'occasion de son passage à l'âge adulte, à treize ans. Cela correspondrait aujourd'hui à la moitié d'un mois de salaire. Donner une seule pièce, c'est symboliquement inapproprié: on offre par paires. Donner deux pièces, c'est un beau cadeau, approprié et suffisant. Donner trois pièces, c'est un cadeau considérable, qui dépasse les attentes.)}  
 ¶ \textcolor{darkblue}{\textbf{\ipa{pæ˧kʰwɤ˧ ɖɯ˧-ɭɯ˥\# ; pæ˧kʰwɤ˧ ɲi˧-ɭɯ˥\# ; pæ˧kʰwɤ˧ so˧-ɭɯ˥\#}}} \zh{一块银币,两块银币,三块银币} \textcolor{Sepia}{\selectlanguage{english}one silver coin, two silver coins, three silver coins} \textcolor{PineGreen}{\selectlanguage{french}une pièce d'argent; deux pièces d'argent; trois pièces d'argent}  
 ¶ \textcolor{darkblue}{\textbf{\ipa{pæ˧kʰwɤ˧ ɖɯ˧-ki˩tɑ˩}}} \zh{一包银币(埋在地里,为了藏)} \textcolor{Sepia}{\selectlanguage{english}a bag of silver coins (to be interred in a secret place)} \textcolor{PineGreen}{\selectlanguage{french}un sac de pièces d'argent, destiné à être caché/enterré}  
 \zh{量词}: \textcolor{darkblue}{\textbf{\ipa{ɭɯ˧}}} 
\lhead{\firstmark}
\rhead{\botmark}

\subsection{\hspace{-0.5cm} {\Large \textcolor{darkblue}{\textbf{\ipa{pæ˧li˩}}}}\hspace{0.5cm}[\kern2pt{\textcolor{darkblue}{\textbf{\ipa{pæ˧li˧}}}}\kern2pt]} \hypertarget{p\{\string_Mli\string_B1}{}
\markboth{\textcolor{darkblue}{\textbf{\ipa{pæ˧li˩}}}}{}
\textcolor{teal}{\zh{名词}} \hspace{4pt} \zh{声调类:} L\#.
\zh{板栗。} \textcolor{Sepia}{\selectlanguage{english}Chinese chestnut.} \textcolor{PineGreen}{\selectlanguage{french}Châtaigne.}  \zh{【借词】} \zh{板栗}
 ¶ \textcolor{darkblue}{\textbf{\ipa{pæ˧li˩-si˩dzi˩}}} \zh{板栗树} \textcolor{Sepia}{\selectlanguage{english}chestnut tree} \textcolor{PineGreen}{\selectlanguage{french}châtaignier}  
 ¶ \textcolor{darkblue}{\textbf{\ipa{pæ˧li˩-dzi˩}}} \zh{板栗树} \textcolor{Sepia}{\selectlanguage{english}chestnut tree} \textcolor{PineGreen}{\selectlanguage{french}châtaignier}  

\lhead{\firstmark}
\rhead{\botmark}

\subsection{\hspace{-0.5cm} {\Large \textcolor{darkblue}{\textbf{\ipa{pæ˧ɻæ˩-ʈʂʰo˩}}}}\hspace{0.5cm}[\kern2pt{\textcolor{darkblue}{\textbf{\ipa{xxxx non-correspondance entre le nombre de morphèmes et le nombre de tons de morphèmes}}}}\kern2pt]} \hypertarget{p\{\string_Mr£`\{\string_B-t`s`\string_ho\string_B1}{}
\markboth{\textcolor{darkblue}{\textbf{\ipa{pæ˧ɻæ˩-ʈʂʰo˩}}}}{}
\textcolor{teal}{\zh{名词}} \hspace{4pt} \zh{声调类:} L\#-.
\zh{红桥。} \textcolor{Sepia}{\selectlanguage{english}Hongqiao, a (mostly Han Chinese) village on the road from Ninglang to Yongning.} \textcolor{PineGreen}{\selectlanguage{french}Hongqiao, village sur la route entre Ninglang et Yongning (principalement peuplé de Chinois Han).}  ¶ \textcolor{darkblue}{\textbf{\ipa{no˧ | pæ˧ɻæ˩ʈʂʰo˩-hĩ˩-ni˩-zo˩!}}} \zh{解放前用的侮辱语句:“你像红桥人!”=“你很丑!”摩梭民间文化中,红桥(马帮路过的一个乡)的人被认为难看,面貌不“眉清目秀”,比如有扁鼻子。} \textcolor{Sepia}{\selectlanguage{english}“You look like someone from Hongqiao!” This is an insult, meaning “You are ugly”. Popular Na geography had it that the people of Hongqiao (a village which the caravans crossed) had coarse, unlovely physical features, such as big snub noses.} \textcolor{PineGreen}{\selectlanguage{french}“Tu ressembles à quelqu'un de Hongqiao!” Insulte, pour dire de quelqu'un qu'il a un physique disgracieux. La géographie populaire na attribuait des traits grossiers aux gens de Hongqiao (localité que traversaient les caravanes): gros nez camus, en particulier.}  

\lhead{\firstmark}
\rhead{\botmark}

\subsection{\hspace{-0.5cm} {\Large \textcolor{darkblue}{\textbf{\ipa{pæ˧sɯ˧}}}}\hspace{0.5cm}[\kern2pt{\textcolor{darkblue}{\textbf{\ipa{xxxx non-correspondance entre le nombre de morphèmes et le nombre de tons de morphèmes}}}}\kern2pt]} \hypertarget{p\{\string_MsM\string_M1}{}
\markboth{\textcolor{darkblue}{\textbf{\ipa{pæ˧sɯ˧}}}}{}
\textcolor{teal}{\zh{名词}} \hspace{4pt} \zh{声调类:} M.
\zh{把事(封建官员系统中的最低等级)(汉语借词)。} \textcolor{Sepia}{\selectlanguage{english}The lowest rank in the hierarchy of feudal officials.} \textcolor{PineGreen}{\selectlanguage{french}Rang (le plus bas) dans la hiérarchie des fonctionnaires féodaux.}  \zh{【借词】} \zh{把事}

\lhead{\firstmark}
\rhead{\botmark}

\subsection{\hspace{-0.5cm} {\Large \textcolor{darkblue}{\textbf{\ipa{pæ˧te˩}}}}\hspace{0.5cm}[\kern2pt{\textcolor{darkblue}{\textbf{\ipa{pæ˧te˧}}}}\kern2pt]} \hypertarget{p\{\string_Mte\string_B1}{}
\markboth{\textcolor{darkblue}{\textbf{\ipa{pæ˧te˩}}}}{}
\textcolor{teal}{\zh{名词}} \hspace{4pt} \zh{声调类:} L\#.
\zh{板凳。} \textcolor{Sepia}{\selectlanguage{english}Bench, stool.} \textcolor{PineGreen}{\selectlanguage{french}Banc, tabouret.}  \zh{【借词】} \zh{板凳}
 \zh{量词}: \textcolor{darkblue}{\textbf{\ipa{ɭɯ˧}}} 
\lhead{\firstmark}
\rhead{\botmark}

\subsection{\hspace{-0.5cm} {\Large \textcolor{darkblue}{\textbf{\ipa{pæ˩\textsubscript{a}}}}}\hspace{0.5cm}[\kern2pt{\textcolor{darkblue}{\textbf{\ipa{pæ˥}}}}\kern2pt]} \hypertarget{p\{\string_Ba1}{}
\markboth{\textcolor{darkblue}{\textbf{\ipa{pæ˩\textsubscript{a}}}}}{}
\textcolor{teal}{\zh{动词}} \hspace{4pt} \zh{声调类:} M\textsubscript{a}.
\zh{摆桌子、供应饭菜。} \textcolor{Sepia}{\selectlanguage{english}To lay (the table).} \textcolor{PineGreen}{\selectlanguage{french}Mettre (la table), servir.}  \zh{【借词】} \zh{摆?}
 ¶ \textcolor{darkblue}{\textbf{\ipa{hɑ˧ tʰi˧-pæ˩ tsæ˩-ɲi˩-ze˩! | hɑ˧ dzɯ˧-bi˧-ze˩!}}} \zh{饭摆好了!吃饭了!} \textcolor{Sepia}{\selectlanguage{english}The table is set / everything is ready! Let's eat!} \textcolor{PineGreen}{\selectlanguage{french}C'est servi! A table!}  

\lhead{\firstmark}
\rhead{\botmark}

\subsection{\hspace{-0.5cm} {\Large \textcolor{darkblue}{\textbf{\ipa{pæ˩pʰæ˧˥}}} \textsubscript{1}}\hspace{0.5cm}[\kern2pt{\textcolor{darkblue}{\textbf{\ipa{pæ˧pʰæ˩}}}}\kern2pt]} \hypertarget{p\{\string_Bp\string_h\{\string_M\string_T1}{}
\markboth{\textcolor{darkblue}{\textbf{\ipa{pæ˩pʰæ˧˥}}} \textsubscript{1}}{}
\textcolor{teal}{\zh{名词}} \hspace{4pt} \zh{声调类:} LM+MH\#.
\ding{202} \zh{厚的木板、 木板子。} \textcolor{Sepia}{\selectlanguage{english}Thick wood plank. A well-prepared plank, used in construction, could last a hundred years.} \textcolor{PineGreen}{\selectlanguage{french}Grosse planche de bois, épaisse d'une dizaine de centimètres, utilisée pour la charpente des maisons.}  \zh{量词}: \textcolor{darkblue}{\textbf{\ipa{pʰæ˧˥}}} \ding{203} \zh{耙。} \textcolor{Sepia}{\selectlanguage{english}Harrow; the term is the same as that for 'plank', as the harrow essentially consisted in a large, squared piece of lumber, without teeth.} \textcolor{PineGreen}{\selectlanguage{french}Herse en bois, qui consiste essentiellement en une grosse pièce de bois, sans dents, d'où l'emploi (par extension) du terme qui signifie “planche”.} 
\lhead{\firstmark}
\rhead{\botmark}

\subsection{\hspace{-0.5cm} {\Large \textcolor{darkblue}{\textbf{\ipa{pæ˩pʰæ˧˥}}} \textsubscript{2}}\hspace{0.5cm}[\kern2pt{\textcolor{darkblue}{\textbf{\ipa{pæ˩pʰæ˧˥}}}}\kern2pt]} \hypertarget{p\{\string_Bp\string_h\{\string_M\string_T2}{}
\markboth{\textcolor{darkblue}{\textbf{\ipa{pæ˩pʰæ˧˥}}} \textsubscript{2}}{}
\textcolor{teal}{\zh{名词}} \hspace{4pt} \zh{声调类:} LM+MH\#.
\zh{男性名字。} \textcolor{Sepia}{\selectlanguage{english}Masculine given name.} \textcolor{PineGreen}{\selectlanguage{french}Prénom masculin.} 
\lhead{\firstmark}
\rhead{\botmark}

\subsection{\hspace{-0.5cm} {\Large \textcolor{darkblue}{\textbf{\ipa{pæ˧˥}}} \textsubscript{1}}\hspace{0.5cm}[\kern2pt{\textcolor{darkblue}{\textbf{\ipa{pæ˧˥}}}}\kern2pt]} \hypertarget{p\{\string_M\string_T1}{}
\markboth{\textcolor{darkblue}{\textbf{\ipa{pæ˧˥}}} \textsubscript{1}}{}
\textcolor{teal}{\zh{动词}} \hspace{4pt} \zh{声调类:} MH.
\zh{种(地)。} \textcolor{Sepia}{\selectlanguage{english}To cultivate land.} \textcolor{PineGreen}{\selectlanguage{french}Cultiver (une terre).} 
\lhead{\firstmark}
\rhead{\botmark}

\subsection{\hspace{-0.5cm} {\Large \textcolor{darkblue}{\textbf{\ipa{pæ˧˥}}} \textsubscript{2}}\hspace{0.5cm}[\kern2pt{\textcolor{darkblue}{\textbf{\ipa{pæ˧˥}}}}\kern2pt]} \hypertarget{p\{\string_M\string_T2}{}
\markboth{\textcolor{darkblue}{\textbf{\ipa{pæ˧˥}}} \textsubscript{2}}{}
\textcolor{teal}{\zh{动词}} \hspace{4pt} \zh{声调类:} MH.
\zh{超过,错过。} \textcolor{Sepia}{\selectlanguage{english}To exceed; to let slip.} \textcolor{PineGreen}{\selectlanguage{french}Dépasser, outrepasser; laisser passer (une occasion).}  ¶ \textcolor{darkblue}{\textbf{\ipa{pæ˧˥ | -kʰɯ˩-pi˩, | mɤ˧-tsɤ˧! |}}} \zh{错过(一个吉日),不好!} \textcolor{Sepia}{\selectlanguage{english}It's not good to let (an auspicious day) slip by! / It's not good to miss the opportunity (of an auspicious days; for the building of a house, for instance)} \textcolor{PineGreen}{\selectlanguage{french}Ce n'est pas bien de laisser passer (un jour propice: pour la construction d'une maison, par exemple)!}  
 ¶ \textcolor{darkblue}{\textbf{\ipa{pæ˧˥ | -tʰɑ˧-kʰɯ˩}}} \zh{不要错过(机会)!} \textcolor{Sepia}{\selectlanguage{english}Don't let (this opportunity) slip by! / (You/we) mustn't miss this opportunity!} \textcolor{PineGreen}{\selectlanguage{french}Il ne faut pas laisser passer/filer (une occasion/un moment propice)!}  
 ¶ \textcolor{darkblue}{\textbf{\ipa{le˧-pæ˧-ze˥!}}} \zh{错过了!} \textcolor{Sepia}{\selectlanguage{english}It's too late! / We have let the opportunity slip by!} \textcolor{PineGreen}{\selectlanguage{french}(On) a laissé filer (une occasion)/ c'est passé, c'est trop tard!}  

\lhead{\firstmark}
\rhead{\botmark}

\subsection{\hspace{-0.5cm} {\Large \textcolor{darkblue}{\textbf{\ipa{pæ˧˥hwɤ˧}}}}\hspace{0.5cm}[\kern2pt{\textcolor{darkblue}{\textbf{\ipa{pæ˧hwɤ˧}}}}\kern2pt]} \hypertarget{p\{\string_M\string_Thw7\string_M1}{}
\markboth{\textcolor{darkblue}{\textbf{\ipa{pæ˧˥hwɤ˧}}}}{}
\textcolor{teal}{\zh{名词}} \hspace{4pt} \zh{声调类:} MH.M.
\zh{办法(早期汉语借词)。} \textcolor{Sepia}{\selectlanguage{english}Solution, method (early borrowing from Chinese).} \textcolor{PineGreen}{\selectlanguage{french}Solution, méthode (emprunt chinois ancien).}  \zh{【借词】} \zh{办法}
 ¶ \textcolor{darkblue}{\textbf{\ipa{ʈʂʰɯ˧ | pæ˧˥hwɤ˧ | ɕjɤ˩ ɣɯ˧ (+ | ʐwæ˩˥)!}}} \zh{他很会想办法的!} \textcolor{Sepia}{\selectlanguage{english}He/she is great at finding solutions / at handling all sorts of difficult situations!} \textcolor{PineGreen}{\selectlanguage{french}Il/elle excelle à trouver des solutions/ il a une solution à tout!}  
 \zh{量词}: \textcolor{darkblue}{\textbf{\ipa{kʰwɤ˥}}} 
\lhead{\firstmark}
\rhead{\botmark}

\subsection{\hspace{-0.5cm} {\Large \textcolor{darkblue}{\textbf{\ipa{pe˧ʂe˧}}}}\hspace{0.5cm}[\kern2pt{\textcolor{darkblue}{\textbf{\ipa{pe˧ʂe˧}}}}\kern2pt]} \hypertarget{pe\string_Ms`e\string_M1}{}
\markboth{\textcolor{darkblue}{\textbf{\ipa{pe˧ʂe˧}}}}{}
\textcolor{teal}{\zh{助词}} \hspace{4pt} \zh{声调类:} M.
\zh{本身(汉语借词)。} \textcolor{Sepia}{\selectlanguage{english}Itself, per se.} \textcolor{PineGreen}{\selectlanguage{french}En soi.}  \zh{【借词】} \zh{本身}

\lhead{\firstmark}
\rhead{\botmark}

\subsection{\hspace{-0.5cm} {\Large \textcolor{darkblue}{\textbf{\ipa{pɤ˥}}}}\hspace{0.5cm}[\kern2pt{\textcolor{darkblue}{\textbf{\ipa{pɤ˥}}}}\kern2pt]} \hypertarget{p7\string_T1}{}
\markboth{\textcolor{darkblue}{\textbf{\ipa{pɤ˥}}}}{}
\textcolor{teal}{\zh{名词}} \hspace{4pt} \zh{声调类:} \#H.
\zh{画。} \textcolor{Sepia}{\selectlanguage{english}Drawing, painting.} \textcolor{PineGreen}{\selectlanguage{french}Dessin, peinture.}  \zh{量词}: \textcolor{darkblue}{\textbf{\ipa{pɤ˥}}} \textcolor{darkblue}{\textbf{\ipa{pʰæ˧˥}}} 
\lhead{\firstmark}
\rhead{\botmark}

\subsection{\hspace{-0.5cm} {\Large \textcolor{darkblue}{\textbf{\ipa{pɤ˥}}}}\hspace{0.5cm}[\kern2pt{\textcolor{darkblue}{\textbf{\ipa{pɤ˥}}}}\kern2pt]} \hypertarget{p7\string_T1}{}
\markboth{\textcolor{darkblue}{\textbf{\ipa{pɤ˥}}}}{}
\textcolor{teal}{\zh{动词}} \hspace{4pt} \zh{声调类:} H.
\zh{蜷曲、蜷缩。} \textcolor{Sepia}{\selectlanguage{english}To curl up; to hunch, to huddle up.} \textcolor{PineGreen}{\selectlanguage{french}S'accroupir, se mettre en boule, se recroqueviller sur soi-même.}  ¶ \textcolor{darkblue}{\textbf{\ipa{æ˩ ʈʂʰɯ˧-mi˥ | si˧dzi˩-ʈʰæ˩qo˩ | tʰi˧-pɤ˥-dʑo˩!}}} \zh{那只鸡,在树下蜷缩着!} \textcolor{Sepia}{\selectlanguage{english}The hen has huddled up under a tree!} \textcolor{PineGreen}{\selectlanguage{french}La poule est recroquevillée sous l'arbre/est accroupie sous l'arbre!}  
 ¶ \textcolor{darkblue}{\textbf{\ipa{ʈʂʰɯ˧-qo˧ ɖɯ˧-pɤ˥ ɕjɤ˩-ɻ̍˩!}}} \zh{过来这边躺一下!} \textcolor{Sepia}{\selectlanguage{english}Come and lay here (for a rest)!} \textcolor{PineGreen}{\selectlanguage{french}Viens t'allonger par ici (pour te reposer)!}  

\lhead{\firstmark}
\rhead{\botmark}

\subsection{\hspace{-0.5cm} {\Large \textcolor{darkblue}{\textbf{\ipa{pɤ˥\textsubscript{b}}}}}\hspace{0.5cm}[\kern2pt{\textcolor{darkblue}{\textbf{\ipa{pɤ˥}}}}\kern2pt]} \hypertarget{p7\string_Tb1}{}
\markboth{\textcolor{darkblue}{\textbf{\ipa{pɤ˥\textsubscript{b}}}}}{}
\textcolor{teal}{\zh{量词}} \hspace{4pt} \zh{声调类:} H\textsubscript{b}.
\zh{量词:雕像,如:佛像(一尊)。} \textcolor{Sepia}{\selectlanguage{english}Classifier for statues, paintings...} \textcolor{PineGreen}{\selectlanguage{french}Classificateur des images, peintures….} 
\lhead{\firstmark}
\rhead{\botmark}

\subsection{\hspace{-0.5cm} {\Large \textcolor{darkblue}{\textbf{\ipa{pɤ˧\textsubscript{a}}}}}\hspace{0.5cm}[\kern2pt{\textcolor{darkblue}{\textbf{\ipa{pɤ˥}}}}\kern2pt]} \hypertarget{p7\string_Ma1}{}
\markboth{\textcolor{darkblue}{\textbf{\ipa{pɤ˧\textsubscript{a}}}}}{}
\textcolor{teal}{\zh{动词}} \hspace{4pt} \zh{声调类:} M\textsubscript{a}.
\zh{背(水、柴、孩子……)。} \textcolor{Sepia}{\selectlanguage{english}To carry on one's back.} \textcolor{PineGreen}{\selectlanguage{french}Porter sur son dos (le bois, …).}  ¶ \textcolor{darkblue}{\textbf{\ipa{pɤ˧\textasciitilde{}pɤ˧}}} \zh{\mytextsc{重叠:背一背}} \textcolor{Sepia}{\selectlanguage{english}\mytextsc{red}} \textcolor{PineGreen}{\selectlanguage{french}\mytextsc{red}}  
 ¶ \textcolor{darkblue}{\textbf{\ipa{tʰi˧-pɤ˥\textasciitilde{}pɤ˩}}} \zh{背一背} \textcolor{Sepia}{\selectlanguage{english}\mytextsc{dur} \mytextsc{red}} \textcolor{PineGreen}{\selectlanguage{french}\mytextsc{dur} \mytextsc{red}}  
 ¶ \textcolor{darkblue}{\textbf{\ipa{qʰæ˧ pɤ˧\textasciitilde{}pɤ˥}}} \zh{背肥料} \textcolor{Sepia}{\selectlanguage{english}to carry manure} \textcolor{PineGreen}{\selectlanguage{french}porter des engrais/ du fumier}  
 ¶ \textcolor{darkblue}{\textbf{\ipa{kʰɤ˧ pɤ˧\textasciitilde{}pɤ˥}}} \zh{背背篓} \textcolor{Sepia}{\selectlanguage{english}to carry a dorsal basket} \textcolor{PineGreen}{\selectlanguage{french}porter un panier dorsal}  
 ¶ \textcolor{darkblue}{\textbf{\ipa{zɯ˧ pɤ˧\textasciitilde{}pɤ˥}}} \zh{背草} \textcolor{Sepia}{\selectlanguage{english}to carry grass} \textcolor{PineGreen}{\selectlanguage{french}porter de l'herbe}  
 ¶ \textcolor{darkblue}{\textbf{\ipa{tso˧\textasciitilde{}tso˧ pɤ˧\textasciitilde{}pɤ˥}}} \zh{背东西} \textcolor{Sepia}{\selectlanguage{english}to carry things} \textcolor{PineGreen}{\selectlanguage{french}porter des choses}  
 ¶ \textcolor{darkblue}{\textbf{\ipa{*tso˧\textasciitilde{}tso˧ pɤ˩}}} \zh{背东西(语法上,这个短语没有问题,但发音合作人不那么说。)} \textcolor{Sepia}{\selectlanguage{english}to carry things (this expression is well-formed syntactically, but apparently not in use)} \textcolor{PineGreen}{\selectlanguage{french}porter des choses (l'expression est bien formée, mais pas usitée)}  
 ¶ \textcolor{darkblue}{\textbf{\ipa{njɤ˧-ɳɯ˧ pɤ˧\textasciitilde{}pɤ˩ (+bi˩)!}}} \zh{我来背!} \textcolor{Sepia}{\selectlanguage{english}I'll do the carrying! / Let me carry (it)!} \textcolor{PineGreen}{\selectlanguage{french}c'est moi qui porte!}  
 ¶ \textcolor{darkblue}{\textbf{\ipa{dʑɯ˩ pɤ˩\textasciitilde{}pɤ˥}}} \zh{背水} \textcolor{Sepia}{\selectlanguage{english}to carry water} \textcolor{PineGreen}{\selectlanguage{french}porter de l'eau}  
 ¶ \textcolor{darkblue}{\textbf{\ipa{zo˧mv̩˥ pɤ˩\textasciitilde{}pɤ˩}}} \zh{背孩子} \textcolor{Sepia}{\selectlanguage{english}to carry a child on the back} \textcolor{PineGreen}{\selectlanguage{french}porter un enfant sur le dos}  

\lhead{\firstmark}
\rhead{\botmark}

\subsection{\hspace{-0.5cm} {\Large \textcolor{darkblue}{\textbf{\ipa{pɤ˧dʑɤ˩-di˩}}}}\hspace{0.5cm}[\kern2pt{\textcolor{darkblue}{\textbf{\ipa{pɤ˧dʑɤ˩di˧}}}}\kern2pt]} \hypertarget{p7\string_Mdz£7\string_B-di\string_B1}{}
\markboth{\textcolor{darkblue}{\textbf{\ipa{pɤ˧dʑɤ˩-di˩}}}}{}
\textcolor{teal}{\zh{名词}} \hspace{4pt} \zh{声调类:} L\#-.
\zh{温泉乡的一个村落。} \textcolor{Sepia}{\selectlanguage{english}A village close to the Hot Springs.} \textcolor{PineGreen}{\selectlanguage{french}Un village proche des Sources Chaudes.}  ¶ \textcolor{darkblue}{\textbf{\ipa{ə˧go˧-ʁwɤ˧, | ʁwɤ˧lɑ˩-bi˩, | bæ˧ʁwɤ˧, | tʰo˧tsʰe\#˥, | pi˧tsʰe˩-di˩, | pɤ˧dʑɤ˩-di˩, | ʁwɤ˧tv̩˧}}} \zh{永宁背向泸沽湖方向经过的村落。前两个村落拥有相当大的摩梭人口比例,第三个村落是摩梭村,最后一个是普米村。} \textcolor{Sepia}{\selectlanguage{english}Villages that one encounters as one leaves the plain of Yongning (away from the Lake); the first two are perceived as villages with a high proportion of Na members, and the third as a mostly Na village, whereas the next ones are Pumi (Prinmi).} \textcolor{PineGreen}{\selectlanguage{french}Villages au sortir de la plaine de Yongning; les deux premiers comportent une population na; le troisième est un village na; les suivants sont essentiellement des villages pumi/prinmi.}  

\lhead{\firstmark}
\rhead{\botmark}

\subsection{\hspace{-0.5cm} {\Large \textcolor{darkblue}{\textbf{\ipa{pɤ˧lɑ˩}}}}\hspace{0.5cm}[\kern2pt{\textcolor{darkblue}{\textbf{\ipa{xxxx non-correspondance entre le nombre de morphèmes et le nombre de tons de morphèmes}}}}\kern2pt]} \hypertarget{p7\string_MlA\string_B1}{}
\markboth{\textcolor{darkblue}{\textbf{\ipa{pɤ˧lɑ˩}}}}{}
\textcolor{teal}{\zh{名词}} \hspace{4pt} \zh{声调类:} L\#.
\zh{相片,照片。} \textcolor{Sepia}{\selectlanguage{english}Photo, photography (newly coined word).} \textcolor{PineGreen}{\selectlanguage{french}Photo, photographie (néologisme).}  \zh{量词}: \textcolor{darkblue}{\textbf{\ipa{pʰæ˧˥}}} 
\lhead{\firstmark}
\rhead{\botmark}

\subsection{\hspace{-0.5cm} {\Large \textcolor{darkblue}{\textbf{\ipa{pɤ˧ʁɑ˧}}}}\hspace{0.5cm}[\kern2pt{\textcolor{darkblue}{\textbf{\ipa{pɤ˧ʁɑ˧}}}}\kern2pt]} \hypertarget{p7\string_MRA\string_M1}{}
\markboth{\textcolor{darkblue}{\textbf{\ipa{pɤ˧ʁɑ˧}}}}{}
\textcolor{teal}{\zh{量词}} \hspace{4pt} \zh{声调类:} M.
\zh{量词:一大步。} \textcolor{Sepia}{\selectlanguage{english}A big step.} \textcolor{PineGreen}{\selectlanguage{french}Un grand pas.}  ¶ \textcolor{darkblue}{\textbf{\ipa{ɖɯ˧-pɤ˧ʁɑ˧\textasciitilde{}ɖɯ˧-pɤ˧ʁɑ˧}}} \zh{大步流星地} \textcolor{Sepia}{\selectlanguage{english}with great strides} \textcolor{PineGreen}{\selectlanguage{french}à grands pas}  
 ¶ \textcolor{darkblue}{\textbf{\ipa{ɲi˧-pɤ˧ʁɑ˧}}} \zh{两大步} \textcolor{Sepia}{\selectlanguage{english}two great strides} \textcolor{PineGreen}{\selectlanguage{french}deux grandes enjambées}  

\lhead{\firstmark}
\rhead{\botmark}

\subsection{\hspace{-0.5cm} {\Large \textcolor{darkblue}{\textbf{\ipa{‑pɤ˧to˩}}}}\hspace{0.5cm}[\kern2pt{\textcolor{darkblue}{\textbf{\ipa{pɤ˧to˩}}}}\kern2pt]} \hypertarget{‑p7\string_Mto\string_B1}{}
\markboth{\textcolor{darkblue}{\textbf{\ipa{‑pɤ˧to˩}}}}{}
\textcolor{teal}{\zh{连接词}} \hspace{4pt} \zh{声调类:} L\#.
\zh{连。} \textcolor{Sepia}{\selectlanguage{english}Even.} \textcolor{PineGreen}{\selectlanguage{french}Même.}  ¶ \textcolor{darkblue}{\textbf{\ipa{ʈʂʰɯ˧ | li˩-pɤ˥to˩ | ʈʰɯ˩-ɲi˥!}}} \zh{她连茶都喝!(关于一个一岁孩子的饮食习惯)} \textcolor{Sepia}{\selectlanguage{english}She even drinks tea! (About the eating and drinking habits of a one-year-old child)} \textcolor{PineGreen}{\selectlanguage{french}Elle boit même du thé! (au sujet de l'alimentation d'un enfant d'un an)}  
 ¶ \textcolor{darkblue}{\textbf{\ipa{ʈʂʰɯ˧ | pɤ˩jɤ˧-pɤ˥to˩ | dzɯ˩-ɲi˥!}}} \zh{她连面包都吃!} \textcolor{Sepia}{\selectlanguage{english}She even eats bread! (About the eating and drinking habits of a one-year-old child)} \textcolor{PineGreen}{\selectlanguage{french}Elle mange même du pain! (au sujet de l'alimentation d'un enfant d'un an)}  
 ¶ \textcolor{darkblue}{\textbf{\ipa{hæ˧, | kʰv̩˩mi˩-ʂe˩-pɤ˥to˩ dzɯ˩-kv̩˩!}}} \zh{汉族连狗肉都吃!(注:摩梭人不吃狗肉)} \textcolor{Sepia}{\selectlanguage{english}The (Han) Chinese even eat dog meat! (Note: consumption of dog meat is forbidden in Na culture)} \textcolor{PineGreen}{\selectlanguage{french}les Chinois, ils mangent même du chien! (Note: l'un des interdits alimentaires na concerne la viande de chien, le chien étant un animal sacré dans la culture na.)}  
 ¶ \textcolor{darkblue}{\textbf{\ipa{hæ˧, | kʰv̩˩mi˩-ʂe˩˥ F dzɯ˩-kv̩˩!}}} \zh{同上} \textcolor{Sepia}{\selectlanguage{english}as above} \textcolor{PineGreen}{\selectlanguage{french}même sens}  
 ¶ \textcolor{darkblue}{\textbf{\ipa{bo˩-pɤ˥to˩; lɑ˧-pɤ˧to˩; mv̩˩-pɤ˥to˩; ʐwæ˧-pɤ˧to˩; ɬi˧mi˧-pɤ˧to˩; ɲi˧mi˧-pɤ˧to˩; hwɤ˧li˧-pɤ˥-to˩; hwɤ˧mi˧-pɤ˥to˩; kʰv̩˩mi˩-pɤ˥-to˩; ʁo˧dzi˩-pɤ˩to˩; ʝi˩ʈʂæ˧-pɤ˥to˩; nɑ˩hĩ˧-pɤ˧to˩; bo˩mi˧-pɤ˧to˩; bo˩ɬɑ˧-pɤ˩to˩; ʁæ˧ʈv̩˥-pɤ˩to˩}}} \zh{与不同声调类的名词结合} \textcolor{Sepia}{\selectlanguage{english}combinations with nouns of the various tone categories} \textcolor{PineGreen}{\selectlanguage{french}en association avec des noms des diverses catégories tonales}  

\lhead{\firstmark}
\rhead{\botmark}

\subsection{\hspace{-0.5cm} {\Large \textcolor{darkblue}{\textbf{\ipa{pɤ˧tv̩˥}}}}\hspace{0.5cm}[\kern2pt{\textcolor{darkblue}{\textbf{\ipa{pɤ˧tv̩˥}}}}\kern2pt]} \hypertarget{p7\string_Mtv\string_=\string_T1}{}
\markboth{\textcolor{darkblue}{\textbf{\ipa{pɤ˧tv̩˥}}}}{}
\textcolor{teal}{\zh{名词}} \hspace{4pt} \zh{声调类:} H\#.
\zh{篮子、竹篮。} \textcolor{Sepia}{\selectlanguage{english}Wickerwork basket.} \textcolor{PineGreen}{\selectlanguage{french}Panier de vannerie.}  ¶ \textcolor{darkblue}{\textbf{\ipa{ɖʐɯ˧ʂɯ˥-pɤ˩tv̩˩}}} \zh{筷子篮} \textcolor{Sepia}{\selectlanguage{english}small basket where chopsticks are kept} \textcolor{PineGreen}{\selectlanguage{french}panier (traditionnellement: en vannerie) dans lequel on range les baguettes}  

\lhead{\firstmark}
\rhead{\botmark}

\subsection{\hspace{-0.5cm} {\Large \textcolor{darkblue}{\textbf{\ipa{pɤ˧tʰi˩}}}}\hspace{0.5cm}[\kern2pt{\textcolor{darkblue}{\textbf{\ipa{pɤ˧tʰi˩}}}}\kern2pt]} \hypertarget{p7\string_Mt\string_hi\string_B1}{}
\markboth{\textcolor{darkblue}{\textbf{\ipa{pɤ˧tʰi˩}}}}{}
\textcolor{teal}{\zh{名词}} \hspace{4pt} \zh{声调类:} L\#.
\zh{一个姓。这个姓,永宁有两家。} \textcolor{Sepia}{\selectlanguage{english}A family name from Yongning. There are two families in Yongning that carry this name.} \textcolor{PineGreen}{\selectlanguage{french}Nom de clan/famille étendue. Deux familles portent ce nom à Yongning.}  ¶ \textcolor{darkblue}{\textbf{\ipa{pɤ˧tʰi˩=ɻ̍˩}}} \zh{\textcolor{darkblue}{\textbf{\ipa{/pɤ˧tʰi˩/}}}家族} \textcolor{Sepia}{\selectlanguage{english}the \textcolor{darkblue}{\textbf{\ipa{/pɤ˧tʰi˩/}}} clan, the \textcolor{darkblue}{\textbf{\ipa{/pɤ˧tʰi˩/}}} family} \textcolor{PineGreen}{\selectlanguage{french}le clan \textcolor{darkblue}{\textbf{\ipa{/pɤ˧tʰi˩/}}}, la famille \textcolor{darkblue}{\textbf{\ipa{/pɤ˧tʰi˩/}}}}  

\lhead{\firstmark}
\rhead{\botmark}

\subsection{\hspace{-0.5cm} {\Large \textcolor{darkblue}{\textbf{\ipa{pɤ˩\textsubscript{a}}}}}\hspace{0.5cm}[\kern2pt{\textcolor{darkblue}{\textbf{\ipa{pɤ˩˥}}}}\kern2pt]} \hypertarget{p7\string_Ba1}{}
\markboth{\textcolor{darkblue}{\textbf{\ipa{pɤ˩\textsubscript{a}}}}}{}
\textcolor{teal}{\zh{动词}} \hspace{4pt} \zh{声调类:} L\textsubscript{a}.
\zh{出现、出来、浮现。} \textcolor{Sepia}{\selectlanguage{english}To come out, to emerge, to appear.} \textcolor{PineGreen}{\selectlanguage{french}Sortir, émerger, apparaître.}  ¶ \textcolor{darkblue}{\textbf{\ipa{dʑɯ˩ pɤ˩˥}}} \zh{涌出水来} \textcolor{Sepia}{\selectlanguage{english}some water comes out} \textcolor{PineGreen}{\selectlanguage{french}de l'eau sort}  
 ¶ \textcolor{darkblue}{\textbf{\ipa{dʑɯ˧qʰv̩˧-qo˧ | dʑɯ˩ pɤ˩-ze˥}}} \zh{水泉里面,涌出水来。} \textcolor{Sepia}{\selectlanguage{english}Water emerges at the source.} \textcolor{PineGreen}{\selectlanguage{french}De l'eau apparaît à la source / de l'eau coule à la source}  
 ¶ \textcolor{darkblue}{\textbf{\ipa{tʰi˧-pɤ˩-dʑo˩}}} \zh{正在涌出水来} \textcolor{Sepia}{\selectlanguage{english}\mytextsc{dur} \string_ \mytextsc{prog}: it is emerging} \textcolor{PineGreen}{\selectlanguage{french}\mytextsc{dur} \string_ \mytextsc{prog}: ça sort, ça coule, ça émerge (ex.: de l'eau de source)}  
 ¶ \textcolor{darkblue}{\textbf{\ipa{gɤ˩-pɤ˥}}} \zh{出现、上来:太阳出来} \textcolor{Sepia}{\selectlanguage{english}to emerge, to come up, to appear (e.g. the sun comes out)} \textcolor{PineGreen}{\selectlanguage{french}émerger, se lever: le soleil se lève}  

\lhead{\firstmark}
\rhead{\botmark}

\subsection{\hspace{-0.5cm} {\Large \textcolor{darkblue}{\textbf{\ipa{pɤ˩\textsubscript{b}}}}}\hspace{0.5cm}[\kern2pt{\textcolor{darkblue}{\textbf{\ipa{pɤ˩˥}}}}\kern2pt]} \hypertarget{p7\string_Bb1}{}
\markboth{\textcolor{darkblue}{\textbf{\ipa{pɤ˩\textsubscript{b}}}}}{}
\textcolor{teal}{\zh{量词}} \hspace{4pt} \zh{声调类:} L\textsubscript{b}.
\zh{量词:木工件,如梯子、门等等(一扇门,一把梯子)。} \textcolor{Sepia}{\selectlanguage{english}Classifier for ladders, doors….} \textcolor{PineGreen}{\selectlanguage{french}Classificateur des éléments de menuiserie/charpente: échelles, portes….} 
\lhead{\firstmark}
\rhead{\botmark}

\subsection{\hspace{-0.5cm} {\Large \textcolor{darkblue}{\textbf{\ipa{pɤ˩dʑɯ˩}}}}\hspace{0.5cm}[\kern2pt{\textcolor{darkblue}{\textbf{\ipa{pɤ˩dʑɯ˩˥}}}}\kern2pt]} \hypertarget{p7\string_Bdz£M\string_B1}{}
\markboth{\textcolor{darkblue}{\textbf{\ipa{pɤ˩dʑɯ˩}}}}{}
\textcolor{teal}{\zh{名词}} \hspace{4pt} \zh{声调类:} L.
\zh{泉水。} \textcolor{Sepia}{\selectlanguage{english}Spring water.} \textcolor{PineGreen}{\selectlanguage{french}Eau de source.} 
\lhead{\firstmark}
\rhead{\botmark}

\subsection{\hspace{-0.5cm} {\Large \textcolor{darkblue}{\textbf{\ipa{pɤ˩-ho˩\textasciitilde{}ho˥}}}}\hspace{0.5cm}[\kern2pt{\textcolor{darkblue}{\textbf{\ipa{xxxx non-correspondance entre le nombre de morphèmes et le nombre de tons de morphèmes}}}}\kern2pt]} \hypertarget{p7\string_B-ho\string_B~ho\string_T1}{}
\markboth{\textcolor{darkblue}{\textbf{\ipa{pɤ˩-ho˩\textasciitilde{}ho˥}}}}{}
\textcolor{teal}{\zh{形容词}} \hspace{4pt} \zh{声调类:} L+H\#.
\zh{柔软。} \textcolor{Sepia}{\selectlanguage{english}Soft.} \textcolor{PineGreen}{\selectlanguage{french}Mou.}  ¶ \textcolor{darkblue}{\textbf{\ipa{pɤ˩-ho˩\textasciitilde{}ho˥-gv̩˩}}} \zh{柔软} \textcolor{Sepia}{\selectlanguage{english}soft} \textcolor{PineGreen}{\selectlanguage{french}mou}  
 ¶ \textcolor{darkblue}{\textbf{\ipa{ʁo˧qʰwɤ˩ | pɤ˩-ho˩\textasciitilde{}ho˥-gv̩˩-hĩ˩ | tʰv̩˧-kʰwɤ˥}}} \zh{头上软软的那块 =囟门} \textcolor{Sepia}{\selectlanguage{english}the place where the head is soft =the fontanel} \textcolor{PineGreen}{\selectlanguage{french}l'endroit où la tête est toute molle =la fontanelle, chez les bébés}  

\lhead{\firstmark}
\rhead{\botmark}

\subsection{\hspace{-0.5cm} {\Large \textcolor{darkblue}{\textbf{\ipa{pɤ˩jɤ˧bv̩˥-di˩}}}}\hspace{0.5cm}[\kern2pt{\textcolor{darkblue}{\textbf{\ipa{xxxx non-correspondance entre le nombre de morphèmes et le nombre de tons de morphèmes}}}}\kern2pt]} \hypertarget{p7\string_Bj7\string_Mbv\string_=\string_T-di\string_B1}{}
\markboth{\textcolor{darkblue}{\textbf{\ipa{pɤ˩jɤ˧bv̩˥-di˩}}}}{}
\textcolor{teal}{\zh{名词}} \hspace{4pt} \zh{声调类:} LM+\#H-.
\zh{用来蒸面团(馒头等等)的蒸笼。} \textcolor{Sepia}{\selectlanguage{english}Steamer used for bread (buns).} \textcolor{PineGreen}{\selectlanguage{french}Étuve pour cuire la pâte/le pain.}  \zh{量词}: \textcolor{darkblue}{\textbf{\ipa{ɭɯ˧}}} 
\lhead{\firstmark}
\rhead{\botmark}

\subsection{\hspace{-0.5cm} {\Large \textcolor{darkblue}{\textbf{\ipa{pɤ˩jɤ˧˥}}}}\hspace{0.5cm}[\kern2pt{\textcolor{darkblue}{\textbf{\ipa{pɤ˩jɤ˧˥}}}}\kern2pt]} \hypertarget{p7\string_Bj7\string_M\string_T1}{}
\markboth{\textcolor{darkblue}{\textbf{\ipa{pɤ˩jɤ˧˥}}}}{}
\textcolor{teal}{\zh{名词}} \hspace{4pt} \zh{声调类:} LM+MH\#.
\ding{202} \zh{做面包的面团(可以蒸成馒头)。} \textcolor{Sepia}{\selectlanguage{english}Dough for making steamed bread.} \textcolor{PineGreen}{\selectlanguage{french}Pâte à pain (à cuire à la vapeur, pour obtenir des petits pains blancs).}  \zh{量词}: \textcolor{darkblue}{\textbf{\ipa{jɤ˧˥}}} \ding{203} \zh{饼。} \textcolor{Sepia}{\selectlanguage{english}Round flat cake.} \textcolor{PineGreen}{\selectlanguage{french}Galette.}  ¶ \textcolor{darkblue}{\textbf{\ipa{li˩-pɤ˥jɤ˩ | ɖɯ˧-ɭɯ˧}}} \zh{一块茶饼} \textcolor{Sepia}{\selectlanguage{english}a piece of brick tea, a brick of tea (tea leaves pressed into the shape of a round flat cake)} \textcolor{PineGreen}{\selectlanguage{french}une galette de thé (feuilles de thé pressées en forme de galette)}  
 ¶ \textcolor{darkblue}{\textbf{\ipa{ɕi˧ʈʂʰwæ˧-pɤ˩jɤ˩}}} \zh{米饼} \textcolor{Sepia}{\selectlanguage{english}rice cake} \textcolor{PineGreen}{\selectlanguage{french}galette de riz}  
 ¶ \textcolor{darkblue}{\textbf{\ipa{dze˧ɭɯ˧-pɤ˩jɤ˩}}} \zh{小麦饼} \textcolor{Sepia}{\selectlanguage{english}wheat cake} \textcolor{PineGreen}{\selectlanguage{french}galette de froment, galette à la farine de blé}  
 ¶ \textcolor{darkblue}{\textbf{\ipa{qʰɑ˧dze˧-pɤ˩jɤ˩}}} \zh{玉米饼} \textcolor{Sepia}{\selectlanguage{english}sweetcorn cake} \textcolor{PineGreen}{\selectlanguage{french}galette de maïs, galette à la farine de maïs}  
 ¶ \textcolor{darkblue}{\textbf{\ipa{tsʰi˧zi˧-pɤ˥jɤ˩}}} \zh{青稞饼} \textcolor{Sepia}{\selectlanguage{english}highland barley cake} \textcolor{PineGreen}{\selectlanguage{french}galette à l'orge d'altitude}  
 ¶ \textcolor{darkblue}{\textbf{\ipa{jɤ˧gɯ˩-pɤ˩jɤ˩}}} \zh{甜荞饼} \textcolor{Sepia}{\selectlanguage{english}buckwheat cake} \textcolor{PineGreen}{\selectlanguage{french}galette de sarrasin}  
 ¶ \textcolor{darkblue}{\textbf{\ipa{jɤ˧qʰɑ˧-pɤ˥jɤ˩}}} \zh{苦荞饼} \textcolor{Sepia}{\selectlanguage{english}bitter buckwheat cake} \textcolor{PineGreen}{\selectlanguage{french}galette de sarrasin amer}  
 \zh{量词}: \textcolor{darkblue}{\textbf{\ipa{jɤ˧˥}}} 
\lhead{\firstmark}
\rhead{\botmark}

\subsection{\hspace{-0.5cm} {\Large \textcolor{darkblue}{\textbf{\ipa{pɤ˩lv̩˩}}}}\hspace{0.5cm}[\kern2pt{\textcolor{darkblue}{\textbf{\ipa{pɤ˧lv̩˧}}}}\kern2pt]} \hypertarget{p7\string_Blv\string_=\string_B1}{}
\markboth{\textcolor{darkblue}{\textbf{\ipa{pɤ˩lv̩˩}}}}{}
\textcolor{teal}{\zh{名词}} \hspace{4pt} \zh{声调类:} L.
\zh{项背 、项、脖颈儿。} \textcolor{Sepia}{\selectlanguage{english}Nape.} \textcolor{PineGreen}{\selectlanguage{french}Nuque.}  \zh{量词}: \textcolor{darkblue}{\textbf{\ipa{ɭɯ˧}}} 
\lhead{\firstmark}
\rhead{\botmark}

\subsection{\hspace{-0.5cm} {\Large \textcolor{darkblue}{\textbf{\ipa{pɤ˩lv̩˧}}}}\hspace{0.5cm}[\kern2pt{\textcolor{darkblue}{\textbf{\ipa{pɤ˩lv̩˩˥}}}}\kern2pt]} \hypertarget{p7\string_Blv\string_=\string_M1}{}
\markboth{\textcolor{darkblue}{\textbf{\ipa{pɤ˩lv̩˧}}}}{}
\textcolor{teal}{\zh{名词}} \hspace{4pt} \zh{声调类:} LM.
\zh{仓库:主屋对面的房子,只有一层。用来收藏大工具,例如犁,或者腊肉。} \textcolor{Sepia}{\selectlanguage{english}Warehouse, storehouse: a one-floor building, opposite the main building (\textcolor{darkblue}{\textbf{\ipa{/ʑi˧mi˧/}}}); it is used for storing objects, such as the ard, and preserved meat.} \textcolor{PineGreen}{\selectlanguage{french}Réserve, magasin: bâtiment à un seul étage, face au bâtiment principal (\textcolor{darkblue}{\textbf{\ipa{/ʑi˧mi˧/}}}), dans lequel on range les gros outils, tels que l'araire, et la viande séchée.} 
\lhead{\firstmark}
\rhead{\botmark}

\subsection{\hspace{-0.5cm} {\Large \textcolor{darkblue}{\textbf{\ipa{pɤ˩mi˩}}}}\hspace{0.5cm}[\kern2pt{\textcolor{darkblue}{\textbf{\ipa{pɤ˩mi˥}}}}\kern2pt]} \hypertarget{p7\string_Bmi\string_B1}{}
\markboth{\textcolor{darkblue}{\textbf{\ipa{pɤ˩mi˩}}}}{}
\textcolor{teal}{\zh{名词}} \hspace{4pt} \zh{声调类:} L.
\zh{青蛙。} \textcolor{Sepia}{\selectlanguage{english}Frog.} \textcolor{PineGreen}{\selectlanguage{french}Grenouille.}  ¶ \textcolor{darkblue}{\textbf{\ipa{pɤ˩mi˩-pɤ˥pʰv̩˩}}} \zh{母青蛙与公青蛙} \textcolor{Sepia}{\selectlanguage{english}female frog and male frog} \textcolor{PineGreen}{\selectlanguage{french}grenouille femelle et grenouille mâle}  
 ¶ \textcolor{darkblue}{\textbf{\ipa{pɤ˩mi˩-ʝi˥pʰv̩˩}}} \zh{一种大青蛙,在永宁坝子很常见。这是发音合作人认识的三种蛙之一。纳西族人不吃这种动物(摩梭人不吃任何蛙类动物)。} \textcolor{Sepia}{\selectlanguage{english}A species of large frog or toad, which is abundant in the Yongning plain. This is one of three species distinguished by the consultant. It is not eaten by the Naxi (nor by the Na, who do not eat any sort of frog). This is the term used for \textit{Kaloula verrucosa} and \textit{Rana chaochiaoensis}.} \textcolor{PineGreen}{\selectlanguage{french}grosse grenouille (ou crapaud); animal très courant dans la plaine. C'est l'une des trois sortes de grenouilles que connaît la locutrice. Cet animal n'est pas consommé par les Naxi (ni par les Na, qui ne mangent aucune grenouille). La locutrice emploie ce terme pour \textit{Kaloula verrucosa} et \textit{Rana chaochiaoensis}.}  
 ¶ \textcolor{darkblue}{\textbf{\ipa{pɤ˩mi˩-ʝi˥pʰv̩˩-mi˩}}} \zh{同上} \textcolor{Sepia}{\selectlanguage{english}same meaning} \textcolor{PineGreen}{\selectlanguage{french}même sens}  
 ¶ \textcolor{darkblue}{\textbf{\ipa{hæ̃˧ʂɯ˩-pɤ˩mi˩}}} \zh{一种很美的青蛙,身体很长。只出现在山上森林里。这是发音合作人认识的第二种青蛙。} \textcolor{Sepia}{\selectlanguage{english}A beautiful species of frog, with a long body. It is only found in the forest, on the mountain. This is the second of three species distinguished by the consultant.} \textcolor{PineGreen}{\selectlanguage{french}Belle grenouille, de longue taille. Elle ne s'observe qu'en forêt, dans la montagne. C'est la deuxième des trois sortes de grenouilles que connaît la locutrice.}  
 ¶ \textcolor{darkblue}{\textbf{\ipa{dʑɯ˩-pɤ˩mi˩˥}}} \zh{一种青蛙,头小、眼睛大。这是发音合作人认识的第三种青蛙。纳西族吃这种青蛙。} \textcolor{Sepia}{\selectlanguage{english}A species of frog with a small head and large eyes, considered by the consultant as spending most of the time in the water. This is the third of three species of frogs distinguished by the consultant. The Naxi hunt it, especially in the fifth month.} \textcolor{PineGreen}{\selectlanguage{french}Grenouille ayant une petite tête et de grands yeux, qui passerait le plus clair de son temps dans l'eau. C'est la troisième des trois sortes de grenouilles que connaît la locutrice. Les Naxi la chassent, la dénichant sous les cailloux des ruisseaux, surtout au cinquième mois.}  
 ¶ \textcolor{darkblue}{\textbf{\ipa{nɑ˩hĩ˥ | pɤ˧-ʂe˧ dzɯ˧; | pɤ˧-ɣɯ˧ | ɬɑ˧tɑ˥ mv̩˩! | pɤ˧-mæ˧, | bæ˧ʈʂo˥ ʝi˩!}}} \zh{谚语:“纳西人吃青蛙,披青蛙皮衣,蛙尾巴当扫帚!”} \textcolor{Sepia}{\selectlanguage{english}Proverb: “The Naxi eat frog meat; they wear vests made of frog skin; and they make brooms with frog tails!”} \textcolor{PineGreen}{\selectlanguage{french}“Les Naxi mangent de la viande de grenouille; ils se vêtent de gilets en peau de grenouille; ils se font des balais avec la queue des grenouilles!”}  
 \zh{量词}: \textcolor{darkblue}{\textbf{\ipa{mi˩}}} 
\lhead{\firstmark}
\rhead{\botmark}

\subsection{\hspace{-0.5cm} {\Large \textcolor{darkblue}{\textbf{\ipa{pɤ˩pʰv̩˩}}}}\hspace{0.5cm}[\kern2pt{\textcolor{darkblue}{\textbf{\ipa{pɤ˧pʰv̩˧˥}}}}\kern2pt]} \hypertarget{p7\string_Bp\string_hv\string_=\string_B1}{}
\markboth{\textcolor{darkblue}{\textbf{\ipa{pɤ˩pʰv̩˩}}}}{}
\textcolor{teal}{\zh{名词}} \hspace{4pt} \zh{声调类:} L.
\zh{公青蛙。} \textcolor{Sepia}{\selectlanguage{english}Male frog.} \textcolor{PineGreen}{\selectlanguage{french}Grenouille mâle.}  \zh{量词}: \textcolor{darkblue}{\textbf{\ipa{mi˩}}} 
\lhead{\firstmark}
\rhead{\botmark}

\subsection{\hspace{-0.5cm} {\Large \textcolor{darkblue}{\textbf{\ipa{pɤ˩ti\#˥}}}}\hspace{0.5cm}[\kern2pt{\textcolor{darkblue}{\textbf{\ipa{pɤ˩ti˥}}}}\kern2pt]} \hypertarget{p7\string_Bti\#\string_T1}{}
\markboth{\textcolor{darkblue}{\textbf{\ipa{pɤ˩ti\#˥}}}}{}
\textcolor{teal}{\zh{名词}} \hspace{4pt} \zh{声调类:} LM+\#H.
\zh{凳子。} \textcolor{Sepia}{\selectlanguage{english}Stool, small bench.} \textcolor{PineGreen}{\selectlanguage{french}Tabouret, petit banc.}  \zh{量词}: \textcolor{darkblue}{\textbf{\ipa{ɭɯ˧}}} 
\lhead{\firstmark}
\rhead{\botmark}

\subsection{\hspace{-0.5cm} {\Large \textcolor{darkblue}{\textbf{\ipa{pɤ˩tɕɯ˧-pɤ˥mi˩}}}}\hspace{0.5cm}[\kern2pt{\textcolor{darkblue}{\textbf{\ipa{pɤ˩tɕɯ˧pɤ˥mi˩}}}}\kern2pt]} \hypertarget{p7\string_Bts£M\string_M-p7\string_Tmi\string_B1}{}
\markboth{\textcolor{darkblue}{\textbf{\ipa{pɤ˩tɕɯ˧-pɤ˥mi˩}}}}{}
\textcolor{teal}{\zh{名词}} \hspace{4pt} \zh{声调类:} LM+\#H-.
\zh{蝌蚪。} \textcolor{Sepia}{\selectlanguage{english}Tadpole.} \textcolor{PineGreen}{\selectlanguage{french}Têtard.}  \zh{量词}: \textcolor{darkblue}{\textbf{\ipa{mi˩}}} \zh{~【参考】~} \textcolor{darkblue}{\textbf{\ipa{pɤ˩tɕɯ˧˥, pɤ˩tɕɯ˧-ʁo˧ɖɯ˧˥}}} 
\lhead{\firstmark}
\rhead{\botmark}

\subsection{\hspace{-0.5cm} {\Large \textcolor{darkblue}{\textbf{\ipa{pɤ˩tɕɯ˧-ʁo˧ɖɯ˧˥}}}}\hspace{0.5cm}[\kern2pt{\textcolor{darkblue}{\textbf{\ipa{xxxx non-correspondance entre le nombre de morphèmes et le nombre de tons de morphèmes}}}}\kern2pt]} \hypertarget{p7\string_Bts£M\string_M-Ro\string_Md`M\string_M\string_T1}{}
\markboth{\textcolor{darkblue}{\textbf{\ipa{pɤ˩tɕɯ˧-ʁo˧ɖɯ˧˥}}}}{}
\textcolor{teal}{\zh{名词}} \hspace{4pt} \zh{声调类:} LM+MH\#.
\zh{蝌蚪。} \textcolor{Sepia}{\selectlanguage{english}Tadpole.} \textcolor{PineGreen}{\selectlanguage{french}Têtard.}  \zh{量词}: \textcolor{darkblue}{\textbf{\ipa{mi˩}}} \zh{~【参考】~} \textcolor{darkblue}{\textbf{\ipa{pɤ˩tɕɯ˧˥, pɤ˩tɕɯ˧-pɤ˥mi˩}}} 
\lhead{\firstmark}
\rhead{\botmark}

\subsection{\hspace{-0.5cm} {\Large \textcolor{darkblue}{\textbf{\ipa{pɤ˩tɕɯ˧˥}}}}\hspace{0.5cm}[\kern2pt{\textcolor{darkblue}{\textbf{\ipa{pɤ˩tɕɯ˧˥}}}}\kern2pt]} \hypertarget{p7\string_Bts£M\string_M\string_T1}{}
\markboth{\textcolor{darkblue}{\textbf{\ipa{pɤ˩tɕɯ˧˥}}}}{}
\textcolor{teal}{\zh{名词}} \hspace{4pt} \zh{声调类:} LM+MH\#.
\zh{蝌蚪。} \textcolor{Sepia}{\selectlanguage{english}Tadpole.} \textcolor{PineGreen}{\selectlanguage{french}Têtard.}  \zh{量词}: \textcolor{darkblue}{\textbf{\ipa{mi˩}}} \zh{~【参考】~} \textcolor{darkblue}{\textbf{\ipa{pɤ˩tɕɯ˧-ʁo˧ɖɯ˧˥, pɤ˩tɕɯ˧-pɤ˥mi˩}}} 
\lhead{\firstmark}
\rhead{\botmark}

\subsection{\hspace{-0.5cm} {\Large \textcolor{darkblue}{\textbf{\ipa{pɤ˧˥}}}}\hspace{0.5cm}[\kern2pt{\textcolor{darkblue}{\textbf{\ipa{pɤ˧˥}}}}\kern2pt]} \hypertarget{p7\string_M\string_T1}{}
\markboth{\textcolor{darkblue}{\textbf{\ipa{pɤ˧˥}}}}{}
\textcolor{teal}{\zh{动词}} \hspace{4pt} \zh{声调类:} MH.
\zh{耙地。} \textcolor{Sepia}{\selectlanguage{english}To harrow.} \textcolor{PineGreen}{\selectlanguage{french}Passer la herse, aplanir (à l'aide d'une herse/instrument permettant de lisser le champ après labourage, afin qu'il soit prêt pour qu'on y repique le riz).}  ¶ \textcolor{darkblue}{\textbf{\ipa{ʝi˧ pɤ˥}}} \zh{耙地} \textcolor{Sepia}{\selectlanguage{english}to harrow} \textcolor{PineGreen}{\selectlanguage{french}passer la herse}  
 ¶ \textcolor{darkblue}{\textbf{\ipa{ɕi˧ tv̩˧-dʑo˧, | ʝi˧ le˧-pɤ˩!}}} \zh{种稻谷,要(先)耙地!} \textcolor{Sepia}{\selectlanguage{english}When one plants rice, one must harrow the field (first)!} \textcolor{PineGreen}{\selectlanguage{french}Quand on plante du riz (=avant de planter le riz), il faut passer la herse!}  

\lhead{\firstmark}
\rhead{\botmark}

\subsection{\hspace{-0.5cm} {\Large \textcolor{darkblue}{\textbf{\ipa{pɤ˩˧ʐv̩˩}}}}\hspace{0.5cm}[\kern2pt{\textcolor{darkblue}{\textbf{\ipa{xxxx non-correspondance entre le nombre de morphèmes et le nombre de tons de morphèmes}}}}\kern2pt]} \hypertarget{p7\string_B\string_Mz`v\string_=\string_B1}{}
\markboth{\textcolor{darkblue}{\textbf{\ipa{pɤ˩˧ʐv̩˩}}}}{}
\textcolor{teal}{\zh{名词}} \hspace{4pt} \zh{声调类:} LM-L.
\zh{褥子(汉语借词:被褥)。} \textcolor{Sepia}{\selectlanguage{english}Mattress.} \textcolor{PineGreen}{\selectlanguage{french}Matelas.} \zh{当地汉语方言:}\zh{被褥。} \zh{【借词】} \zh{被褥}
 \zh{量词}: \textcolor{darkblue}{\textbf{\ipa{tsʰi˥}}} 
\lhead{\firstmark}
\rhead{\botmark}

\subsection{\hspace{-0.5cm} {\Large \textcolor{darkblue}{\textbf{\ipa{pi˥}}}}\hspace{0.5cm}[\kern2pt{\textcolor{darkblue}{\textbf{\ipa{pi˩˥}}}}\kern2pt]} \hypertarget{pi\string_T1}{}
\markboth{\textcolor{darkblue}{\textbf{\ipa{pi˥}}}}{}
\textcolor{teal}{\zh{动词}} \hspace{4pt} \zh{声调类:} H.
\zh{说。} \textcolor{Sepia}{\selectlanguage{english}To say.} \textcolor{PineGreen}{\selectlanguage{french}Dire.}  ¶ \textcolor{darkblue}{\textbf{\ipa{tʰɑ˧-pi˥!}}} \zh{别说!} \textcolor{Sepia}{\selectlanguage{english}Don't say it! / Don't speak about it!} \textcolor{PineGreen}{\selectlanguage{french}Il ne faut pas (le) dire!}  
 ¶ \textcolor{darkblue}{\textbf{\ipa{ə˧tso˧ pi˧?}}} \zh{(你刚才)说什么?(请人家重新说一遍)} \textcolor{Sepia}{\selectlanguage{english}What did you say? (Call for repetition)} \textcolor{PineGreen}{\selectlanguage{french}Que dis-tu? (employé pour demander à quelqu'un de répéter)}  
 ¶ \textcolor{darkblue}{\textbf{\ipa{ə˧tso˧ pi˧-ɲi˥?}}} \zh{(你刚才)说什么?(请人家重新说一遍)} \textcolor{Sepia}{\selectlanguage{english}What did you say? (Call for repetition)} \textcolor{PineGreen}{\selectlanguage{french}Que dis-tu? (employé pour demander à quelqu'un de répéter)}  

\lhead{\firstmark}
\rhead{\botmark}

\subsection{\hspace{-0.5cm} {\Large \textcolor{darkblue}{\textbf{\ipa{pi˧lv̩\#˥}}}}\hspace{0.5cm}[\kern2pt{\textcolor{darkblue}{\textbf{\ipa{pi˧lv̩˩}}}}\kern2pt]} \hypertarget{pi\string_Mlv\string_=\#\string_T1}{}
\markboth{\textcolor{darkblue}{\textbf{\ipa{pi˧lv̩\#˥}}}}{}
\textcolor{teal}{\zh{名词}} \hspace{4pt} \zh{声调类:} \#H.
\zh{酒糟:煮酒剩下的渣滓(一般给猪吃)。} \textcolor{Sepia}{\selectlanguage{english}Residue left by the production of alcohol, distiller's grains: grains that are fed to the pigs.} \textcolor{PineGreen}{\selectlanguage{french}Déchet de la distillation: ce qui reste après la production de l'alcool; grain qu'on donne aux animaux.} \zh{当地汉语方言:}\zh{酒糟。} ¶ \textcolor{darkblue}{\textbf{\ipa{pi˧lv̩˧, | hĩ˧ | dzɯ˧-mɤ˧-kv̩˩!}}} \zh{酒糟,人不能吃!} \textcolor{Sepia}{\selectlanguage{english}Distiller's grains are not suitable for human consumption! / People don't eat distiller's grains!} \textcolor{PineGreen}{\selectlanguage{french}Les grains après distillation, ça ne se mange pas! / ce n'est pas propre à la consommation humaine!}  

\lhead{\firstmark}
\rhead{\botmark}

\subsection{\hspace{-0.5cm} {\Large \textcolor{darkblue}{\textbf{\ipa{pi˧mɑ˧}}}}\hspace{0.5cm}[\kern2pt{\textcolor{darkblue}{\textbf{\ipa{pi˧mɑ˧}}}}\kern2pt]} \hypertarget{pi\string_MmA\string_M1}{}
\markboth{\textcolor{darkblue}{\textbf{\ipa{pi˧mɑ˧}}}}{}
\textcolor{teal}{\zh{名词}} \hspace{4pt} \zh{声调类:} M.
\zh{男女通用名。} \textcolor{Sepia}{\selectlanguage{english}A unixex given name: a given name used for both men and women.} \textcolor{PineGreen}{\selectlanguage{french}Prénom unisexe: prénom utilisé pour les deux sexes.} 
\lhead{\firstmark}
\rhead{\botmark}

\subsection{\hspace{-0.5cm} {\Large \textcolor{darkblue}{\textbf{\ipa{pi˧mɑ˧-ɬɑ˩mv̩˩}}}}\hspace{0.5cm}[\kern2pt{\textcolor{darkblue}{\textbf{\ipa{xxxx non-correspondance entre le nombre de morphèmes et le nombre de tons de morphèmes}}}}\kern2pt]} \hypertarget{pi\string_MmA\string_M-KA\string_Bmv\string_=\string_B1}{}
\markboth{\textcolor{darkblue}{\textbf{\ipa{pi˧mɑ˧-ɬɑ˩mv̩˩}}}}{}
\textcolor{teal}{\zh{名词}} \hspace{4pt} \zh{声调类:} \mytextsc{L}.
\zh{女性名字。} \textcolor{Sepia}{\selectlanguage{english}Feminine given name.} \textcolor{PineGreen}{\selectlanguage{french}Prénom féminin.} 
\lhead{\firstmark}
\rhead{\botmark}

\subsection{\hspace{-0.5cm} {\Large \textcolor{darkblue}{\textbf{\ipa{pi˧mv̩˥\$}}}}\hspace{0.5cm}[\kern2pt{\textcolor{darkblue}{\textbf{\ipa{pi˧mv̩˥}}}}\kern2pt]} \hypertarget{pi\string_Mmv\string_=\string_T\$1}{}
\markboth{\textcolor{darkblue}{\textbf{\ipa{pi˧mv̩˥\$}}}}{}
\textcolor{teal}{\zh{名词}} \hspace{4pt} \zh{声调类:} H\$.
\zh{成语、俗语。} \textcolor{Sepia}{\selectlanguage{english}Set phrase, idiom, adage.} \textcolor{PineGreen}{\selectlanguage{french}Dicton, parole du temps jadis, adage.} 
\lhead{\firstmark}
\rhead{\botmark}

\subsection{\hspace{-0.5cm} {\Large \textcolor{darkblue}{\textbf{\ipa{pi˧tsʰe˩-di˩}}}}\hspace{0.5cm}[\kern2pt{\textcolor{darkblue}{\textbf{\ipa{pi˧tsʰe˩di˧}}}}\kern2pt]} \hypertarget{pi\string_Mts\string_he\string_B-di\string_B1}{}
\markboth{\textcolor{darkblue}{\textbf{\ipa{pi˧tsʰe˩-di˩}}}}{}
\textcolor{teal}{\zh{名词}} \hspace{4pt} \zh{声调类:} L\#-.
\zh{温泉乡的一个村落。} \textcolor{Sepia}{\selectlanguage{english}A village close to the Hot Springs.} \textcolor{PineGreen}{\selectlanguage{french}Un village proche des Sources Chaudes.}  ¶ \textcolor{darkblue}{\textbf{\ipa{ə˧go˧-ʁwɤ˧, | ʁwɤ˧lɑ˩-bi˩, | bæ˧ʁwɤ˧, | tʰo˧tsʰe\#˥, | pi˧tsʰe˩-di˩, | pɤ˧dʑɤ˩-di˩, | ʁwɤ˧tv̩˧}}} \zh{永宁背向泸沽湖方向经过的村落。前两个村落拥有相当大的摩梭人口比例,第三个村落是摩梭村,最后一个是普米村。} \textcolor{Sepia}{\selectlanguage{english}Villages that one encounters as one leaves the plain of Yongning (away from the Lake); the first two are perceived as villages with a high proportion of Na members, and the third as a mostly Na village, whereas the next ones are Pumi (Prinmi).} \textcolor{PineGreen}{\selectlanguage{french}Villages au sortir de la plaine de Yongning; les deux premiers comportent une population na; le troisième est un village na; les suivants sont essentiellement des villages pumi/prinmi.}  
 ¶ \textcolor{darkblue}{\textbf{\ipa{pi˧tsʰe˩: bɤ˩! |}}} \zh{fv:/pi˧tsʰe˩/是一个普米族村落!} \textcolor{Sepia}{\selectlanguage{english}\textcolor{darkblue}{\textbf{\ipa{/pi˧tsʰe˧/}}} is a Pumi village!} \textcolor{PineGreen}{\selectlanguage{french}\textcolor{darkblue}{\textbf{\ipa{/pi˧tsʰe˩/}}}, c'est un village pumi!}  

\lhead{\firstmark}
\rhead{\botmark}

\subsection{\hspace{-0.5cm} {\Large \textcolor{darkblue}{\textbf{\ipa{pi˩ɻ̍˥}}}}\hspace{0.5cm}[\kern2pt{\textcolor{darkblue}{\textbf{\ipa{pi˩ɻ̍˥}}}}\kern2pt]} \hypertarget{pi\string_Br£`̍\string_T1}{}
\markboth{\textcolor{darkblue}{\textbf{\ipa{pi˩ɻ̍˥}}}}{}
\textcolor{teal}{\zh{名词}} \hspace{4pt} \zh{声调类:} LH.
\zh{双下巴。} \textcolor{Sepia}{\selectlanguage{english}Double chin; flesh under the chin.} \textcolor{PineGreen}{\selectlanguage{french}Double menton, bourrelet de chair sous le menton.}  \zh{量词}: \textcolor{darkblue}{\textbf{\ipa{ɭɯ˧}}} 
\lhead{\firstmark}
\rhead{\botmark}

\subsection{\hspace{-0.5cm} {\Large \textcolor{darkblue}{\textbf{\ipa{pi˩ti\#˥}}}}\hspace{0.5cm}[\kern2pt{\textcolor{darkblue}{\textbf{\ipa{pi˩ti˥}}}}\kern2pt]} \hypertarget{pi\string_Bti\#\string_T1}{}
\markboth{\textcolor{darkblue}{\textbf{\ipa{pi˩ti\#˥}}}}{}
\textcolor{teal}{\zh{名词}} \hspace{4pt} \zh{声调类:} LM+\#H.
\zh{银块。} \textcolor{Sepia}{\selectlanguage{english}Silver nugget, piece of raw silver.} \textcolor{PineGreen}{\selectlanguage{french}Pépite d'argent.}  \zh{量词}: \textcolor{darkblue}{\textbf{\ipa{ɭɯ˧}}} 
\lhead{\firstmark}
\rhead{\botmark}

\subsection{\hspace{-0.5cm} {\Large \textcolor{darkblue}{\textbf{\ipa{pi˧˥\textsubscript{a}}}}}\hspace{0.5cm}[\kern2pt{\textcolor{darkblue}{\textbf{\ipa{pi˧˥}}}}\kern2pt]} \hypertarget{pi\string_M\string_Ta1}{}
\markboth{\textcolor{darkblue}{\textbf{\ipa{pi˧˥\textsubscript{a}}}}}{}
\textcolor{teal}{\zh{量词}} \hspace{4pt} \zh{声调类:} MH\textsubscript{a}.
\zh{量词:少。} \textcolor{Sepia}{\selectlanguage{english}A little (noncount); mostly appears in combination with the numeral 'one'.} \textcolor{PineGreen}{\selectlanguage{french}Peu (indénombrable), un peu; souvent employé comme hypocoristique.}  ¶ \textcolor{darkblue}{\textbf{\ipa{ɖɯ˧-pi˧˥}}} \zh{一点} \textcolor{Sepia}{\selectlanguage{english}a little} \textcolor{PineGreen}{\selectlanguage{french}un peu}  
 ¶ \textcolor{darkblue}{\textbf{\ipa{qʰæ˧-pi˩}}} \zh{一点粪肥} \textcolor{Sepia}{\selectlanguage{english}a little manure} \textcolor{PineGreen}{\selectlanguage{french}un peu de crottin (ramassé comme engrais)}  
 ¶ \textcolor{darkblue}{\textbf{\ipa{ŋv̩˧-pi˧}}} \zh{一点钱} \textcolor{Sepia}{\selectlanguage{english}a little money} \textcolor{PineGreen}{\selectlanguage{french}un peu d’argent}  
 ¶ \textcolor{darkblue}{\textbf{\ipa{ŋv̩˧ | ɖɯ˧-pi˧˥}}} \zh{一点钱} \textcolor{Sepia}{\selectlanguage{english}a little money} \textcolor{PineGreen}{\selectlanguage{french}un peu d’argent}  
 ¶ \textcolor{darkblue}{\textbf{\ipa{lwɤ˧˥ | ɖɯ˧ pi˧˥}}} \zh{一点灰} \textcolor{Sepia}{\selectlanguage{english}a little ashes} \textcolor{PineGreen}{\selectlanguage{french}un peu de cendre; on ne peut dire: *lwɤ˧-pi˥, non plus que: *tsʰe˧-pi˩ (pour “un peu de sel”)}  
 ¶ \textcolor{darkblue}{\textbf{\ipa{ʈʂʰɯ˧ | ɖʐe˧ ɖɯ˧-pi˧ dʑo˧!}}} \zh{他有一些钱!} \textcolor{Sepia}{\selectlanguage{english}He has a little money! / He is rather affluent!} \textcolor{PineGreen}{\selectlanguage{french}il a un peu d'argent!}  

\lhead{\firstmark}
\rhead{\botmark}

\subsection{\hspace{-0.5cm} {\Large \textcolor{darkblue}{\textbf{\ipa{pi˩˥}}}}\hspace{0.5cm}[\kern2pt{\textcolor{darkblue}{\textbf{\ipa{pi˥}}}}\kern2pt]} \hypertarget{pi\string_B\string_T1}{}
\markboth{\textcolor{darkblue}{\textbf{\ipa{pi˩˥}}}}{}
\textcolor{teal}{\zh{名词}} \hspace{4pt} \zh{声调类:} LH.
\zh{笔。} \textcolor{Sepia}{\selectlanguage{english}Brush (Chinese borrowing).} \textcolor{PineGreen}{\selectlanguage{french}Pinceau (emprunt ancien).}  \zh{【借词】} \zh{笔}
 ¶ \textcolor{darkblue}{\textbf{\ipa{tʰæ˧ɻæ˩ tɕɯ˩-di˩, | pi˩˥!}}} \zh{用来写字的那个东西,(叫做)“笔”!} \textcolor{Sepia}{\selectlanguage{english}The thing used to write is called “pen”!} \textcolor{PineGreen}{\selectlanguage{french}Le truc pour écrire, ça s'appelle “pinceau”!}  

\lhead{\firstmark}
\rhead{\botmark}

\subsection{\hspace{-0.5cm} {\Large \textcolor{darkblue}{\textbf{\ipa{pjɤ˥}}}}\hspace{0.5cm}[\kern2pt{\textcolor{darkblue}{\textbf{\ipa{pjɤ˥}}}}\kern2pt]} \hypertarget{pj7\string_T1}{}
\markboth{\textcolor{darkblue}{\textbf{\ipa{pjɤ˥}}}}{}
\textcolor{teal}{\zh{形容词}} \hspace{4pt} \zh{声调类:} H.
\zh{方形的。} \textcolor{Sepia}{\selectlanguage{english}Square.} \textcolor{PineGreen}{\selectlanguage{french}Carré/anguleux (visage, pilier…).}  ¶ \textcolor{darkblue}{\textbf{\ipa{tɑ˧-pjɤ˧\textasciitilde{}pjɤ˥ (-zo˩)}}} \zh{(脸、物品)太方,不圆滑} \textcolor{Sepia}{\selectlanguage{english}(of a face or object) unpleasantly squarish, lacking smoothness} \textcolor{PineGreen}{\selectlanguage{french}un peu anguleux/carré (terme péjoratif: objet / physique trop peu lisse pour être plaisant au regard ou au toucher)}  

\lhead{\firstmark}
\rhead{\botmark}

\subsection{\hspace{-0.5cm} {\Large \textcolor{darkblue}{\textbf{\ipa{po˥}}}}\hspace{0.5cm}[\kern2pt{\textcolor{darkblue}{\textbf{\ipa{po˧˥}}}}\kern2pt]} \hypertarget{po\string_T1}{}
\markboth{\textcolor{darkblue}{\textbf{\ipa{po˥}}}}{}
\textcolor{teal}{\zh{动词}} \hspace{4pt} \zh{声调类:} .
\zh{包(量词)(汉语借词)。} \textcolor{Sepia}{\selectlanguage{english}To pack.} \textcolor{PineGreen}{\selectlanguage{french}Emballer.}  \zh{【借词】} \zh{包}
 ¶ \textcolor{darkblue}{\textbf{\ipa{le˧-po˥}}} \zh{\mytextsc{accomp}} \textcolor{Sepia}{\selectlanguage{english}\mytextsc{accomp}} \textcolor{PineGreen}{\selectlanguage{french}\mytextsc{accomp}}  

\lhead{\firstmark}
\rhead{\botmark}

\subsection{\hspace{-0.5cm} {\Large \textcolor{darkblue}{\textbf{\ipa{po˧\textsubscript{a}}}}}\hspace{0.5cm}[\kern2pt{\textcolor{darkblue}{\textbf{\ipa{po˥}}}}\kern2pt]} \hypertarget{po\string_Ma1}{}
\markboth{\textcolor{darkblue}{\textbf{\ipa{po˧\textsubscript{a}}}}}{}
\textcolor{teal}{\zh{量词}} \hspace{4pt} \zh{声调类:} M\textsubscript{a}.
\zh{量词:有根的植物,衣服(一棵,一件)。} \textcolor{Sepia}{\selectlanguage{english}Classifier for plants with a stalk; also used for pieces of clothing.} \textcolor{PineGreen}{\selectlanguage{french}Classificateur des plantes à tiges (fleurs, poireaux…); aussi utilisé pour compter les types/catégories de vêtements.} 
\lhead{\firstmark}
\rhead{\botmark}

\subsection{\hspace{-0.5cm} {\Large \textcolor{darkblue}{\textbf{\ipa{po˧ɖʐɯ\#˥}}}}\hspace{0.5cm}[\kern2pt{\textcolor{darkblue}{\textbf{\ipa{po˧ɖʐɯ˧}}}}\kern2pt]} \hypertarget{po\string_Md`z`M\#\string_T1}{}
\markboth{\textcolor{darkblue}{\textbf{\ipa{po˧ɖʐɯ\#˥}}}}{}
\textcolor{teal}{\zh{名词}} \hspace{4pt} \zh{声调类:} \#H.
\zh{工匠。} \textcolor{Sepia}{\selectlanguage{english}Craftsman.} \textcolor{PineGreen}{\selectlanguage{french}Artisan.}  ¶ \textcolor{darkblue}{\textbf{\ipa{po˧ɖʐɯ˧ ʝi˧-hĩ˧-hĩ˧}}} \zh{当工匠的人} \textcolor{Sepia}{\selectlanguage{english}person who works as a craftsman} \textcolor{PineGreen}{\selectlanguage{french}personne qui travaille comme artisan}  
 \zh{量词}: \textcolor{darkblue}{\textbf{\ipa{v̩˧}}} 
\lhead{\firstmark}
\rhead{\botmark}

\subsection{\hspace{-0.5cm} {\Large \textcolor{darkblue}{\textbf{\ipa{po˧lo˧}}}}\hspace{0.5cm}[\kern2pt{\textcolor{darkblue}{\textbf{\ipa{po˩lo˧˥}}}}\kern2pt]} \hypertarget{po\string_Mlo\string_M1}{}
\markboth{\textcolor{darkblue}{\textbf{\ipa{po˧lo˧}}}}{}
\textcolor{teal}{\zh{名词}} \hspace{4pt} \zh{声调类:} M.
\zh{公绵羊。} \textcolor{Sepia}{\selectlanguage{english}Ram.} \textcolor{PineGreen}{\selectlanguage{french}Bélier; bouc.}  ¶ \textcolor{darkblue}{\textbf{\ipa{po˧lo˧ lɑ˧˥}}} \zh{打公绵羊} \textcolor{Sepia}{\selectlanguage{english}to strike a ram} \textcolor{PineGreen}{\selectlanguage{french}frapper un bélier}  
 \zh{量词}: \textcolor{darkblue}{\textbf{\ipa{pʰo˧˥}}} 
\lhead{\firstmark}
\rhead{\botmark}

\subsection{\hspace{-0.5cm} {\Large \textcolor{darkblue}{\textbf{\ipa{po˧po˧}}}}\hspace{0.5cm}[\kern2pt{\textcolor{darkblue}{\textbf{\ipa{po˩po˩˥}}}}\kern2pt]} \hypertarget{po\string_Mpo\string_M1}{}
\markboth{\textcolor{darkblue}{\textbf{\ipa{po˧po˧}}}}{}
\textcolor{teal}{\zh{名词}} \hspace{4pt} \zh{声调类:} M.
\zh{球。} \textcolor{Sepia}{\selectlanguage{english}Ball.} \textcolor{PineGreen}{\selectlanguage{french}Ballon.}  ¶ \textcolor{darkblue}{\textbf{\ipa{[F5] po˧po˧ lɑ˧˥}}} \zh{打球} \textcolor{Sepia}{\selectlanguage{english}to play (foot)ball} \textcolor{PineGreen}{\selectlanguage{french}jouer au ballon}  
 \zh{量词}: \textcolor{darkblue}{\textbf{\ipa{ɭɯ˧}}} 
\lhead{\firstmark}
\rhead{\botmark}

\subsection{\hspace{-0.5cm} {\Large \textcolor{darkblue}{\textbf{\ipa{po˧po˧tsʰɤ˧˥}}}}\hspace{0.5cm}[\kern2pt{\textcolor{darkblue}{\textbf{\ipa{po˧po˧tsʰɤ˧}}}}\kern2pt]} \hypertarget{po\string_Mpo\string_Mts\string_h7\string_M\string_T1}{}
\markboth{\textcolor{darkblue}{\textbf{\ipa{po˧po˧tsʰɤ˧˥}}}}{}
\textcolor{teal}{\zh{名词}} \hspace{4pt} \zh{声调类:} MH\#.
\zh{圆白菜。} \textcolor{Sepia}{\selectlanguage{english}Cabbage.} \textcolor{PineGreen}{\selectlanguage{french}Chou.} \zh{当地汉语方言:}\zh{包包菜。} \zh{【借词】} \zh{包包菜}
 \zh{量词}: \textcolor{darkblue}{\textbf{\ipa{ɭɯ˧}}} 
\lhead{\firstmark}
\rhead{\botmark}

\subsection{\hspace{-0.5cm} {\Large \textcolor{darkblue}{\textbf{\ipa{po˩\textsubscript{b}}}}}\hspace{0.5cm}[\kern2pt{\textcolor{darkblue}{\textbf{\ipa{po˩˥}}}}\kern2pt]} \hypertarget{po\string_Bb1}{}
\markboth{\textcolor{darkblue}{\textbf{\ipa{po˩\textsubscript{b}}}}}{}
\textcolor{teal}{\zh{量词}} \hspace{4pt} \zh{声调类:} L\textsubscript{b}.
\zh{量词:包(汉语借词)。} \textcolor{Sepia}{\selectlanguage{english}Classifier for packs (e.g. a pack of cigarettes).} \textcolor{PineGreen}{\selectlanguage{french}Classificateur des paquets.}  \zh{【借词】} \zh{包}

\lhead{\firstmark}
\rhead{\botmark}

\subsection{\hspace{-0.5cm} {\Large \textcolor{darkblue}{\textbf{\ipa{po˧˥}}}}\hspace{0.5cm}[\kern2pt{\textcolor{darkblue}{\textbf{\ipa{po˧˥}}}}\kern2pt]} \hypertarget{po\string_M\string_T1}{}
\markboth{\textcolor{darkblue}{\textbf{\ipa{po˧˥}}}}{}
\textcolor{teal}{\zh{动词}} \hspace{4pt} \zh{声调类:} MH.
\ding{202} \zh{寄信、服送、带过来、拿、送。} \textcolor{Sepia}{\selectlanguage{english}To bring; to send (a letter), to deliver (a message).} \textcolor{PineGreen}{\selectlanguage{french}Amener, apporter; ramener, rapporter; faire cadeau de; envoyer (un message), transmettre; utiliser.}  ¶ \textcolor{darkblue}{\textbf{\ipa{qʰwæ˧ po˧˥}}} \zh{带来一封信/一个消息} \textcolor{Sepia}{\selectlanguage{english}to bring a letter/a message} \textcolor{PineGreen}{\selectlanguage{french}amener une lettre/un message}  
 ¶ \textcolor{darkblue}{\textbf{\ipa{tso˧\textasciitilde{}tso˧ tʰi˧-po˧˥}}} \zh{带来一个东西} \textcolor{Sepia}{\selectlanguage{english}to bring something} \textcolor{PineGreen}{\selectlanguage{french}amener quelque chose}  
 ¶ \textcolor{darkblue}{\textbf{\ipa{hĩ˧ ɖɯ˧-v̩˧ | tso˧\textasciitilde{}tso˧ ɖɯ˧-kʰwɤ˥ | tʰi˧-po˧˥}}} \zh{有人带东西过来} \textcolor{Sepia}{\selectlanguage{english}someone brings something} \textcolor{PineGreen}{\selectlanguage{french}quelqu'un prend/amène quelque chose}  
 ¶ \textcolor{darkblue}{\textbf{\ipa{ʈʂʰwæ˧˥ | po˧-jo˥!}}} \zh{快拿过来吧!/ 快带过来吧!} \textcolor{Sepia}{\selectlanguage{english}Bring it over, quick!} \textcolor{PineGreen}{\selectlanguage{french}amène vite!}  
\ding{203} \zh{怀孕。} \textcolor{Sepia}{\selectlanguage{english}To carry (a child), i.e. to be pregnant.} \textcolor{PineGreen}{\selectlanguage{french}Porter un enfant, c'est-à-dire être enceinte.}  ¶ \textcolor{darkblue}{\textbf{\ipa{ʈʂʰɯ˧ | zo˧mv̩˥ po˩.}}} \zh{她怀孕了。} \textcolor{Sepia}{\selectlanguage{english}She is pregnant.} \textcolor{PineGreen}{\selectlanguage{french}Elle est enceinte.}  
 ¶ \textcolor{darkblue}{\textbf{\ipa{zo˧ po˩ (+ze˩)}}} \zh{怀孕} \textcolor{Sepia}{\selectlanguage{english}to carry a child, i.e. to be pregnant} \textcolor{PineGreen}{\selectlanguage{french}porter un enfant, être enceinte.}  

\lhead{\firstmark}
\rhead{\botmark}

\subsection{\hspace{-0.5cm} {\Large \textcolor{darkblue}{\textbf{\ipa{pv̩˩}}}}\hspace{0.5cm}[\kern2pt{\textcolor{darkblue}{\textbf{\ipa{pv̩˩˥}}}}\kern2pt]} \hypertarget{pv\string_=\string_B1}{}
\markboth{\textcolor{darkblue}{\textbf{\ipa{pv̩˩}}}}{}
\textcolor{teal}{\zh{动词}} \hspace{4pt} \zh{声调类:} L.
\zh{过、过去(时间过去、日子过去)。} \textcolor{Sepia}{\selectlanguage{english}To go by, to flow (of time).} \textcolor{PineGreen}{\selectlanguage{french}Passer, s'écouler: le temps passe, les jours passent.}  ¶ \textcolor{darkblue}{\textbf{\ipa{ɲi˧mi˧ pv̩˩}}} \zh{时间过去。直译:(一)天(慢慢)过(去)} \textcolor{Sepia}{\selectlanguage{english}time goes by; literally: the day goes by} \textcolor{PineGreen}{\selectlanguage{french}le temps passe, la journée passe}  
 ¶ \textcolor{darkblue}{\textbf{\ipa{ɲi˧mi˧ | le˧-pv̩˩-ze˩}}} \zh{时间过去了,(一)天过去了} \textcolor{Sepia}{\selectlanguage{english}time has gone by, the day has gone by} \textcolor{PineGreen}{\selectlanguage{french}le temps a passé, la journée a passé}  
 ¶ \textcolor{darkblue}{\textbf{\ipa{dʑɤ˩-dzɯ˧ qʰwɤ˧-dzɯ˥, | bi˧mi˧ ʂv̩˧-qʰwɤ˧-ɻ̍˥; | dʑɤ˩-ʐwɤ˥ qʰwɤ˩-ʐwɤ˩, | ɲi˧mi˧ ʂæ˧ pv̩˩-di˩!}}} \zh{“吃好吃坏,(都)能填满肚子/(都)能吃饱!说好说坏,(都)能让一天(轻松)过去!”(这个谚语,说闲聊的好。)} \textcolor{Sepia}{\selectlanguage{english}Whether one eats good stuff or bad stuff, that fills the stomach / that does the trick of filling your belly! Whether one tells good stories or bad ones, that helps make the long day go by / that does the trick of chipping a long (and tedious) day away/of filling a day pleasantly! (A laid-back proverb in praise of small talk and gossip.)} \textcolor{PineGreen}{\selectlanguage{french}Qu'on mange bien ou mal, on arrive à se remplir le ventre / Que la nourriture soit bonne ou mauvaise, peu importe au fond, tant qu'on a le ventre plein; qu'on dise des bonnes choses (=des éloges d'autrui) ou des mauvaises (=des ragots), on arrive à passer la journée / le jour se passe agréablement! (Proverbe qui fait l'éloge des vertus du bavardage et du commérage.)}  
 ¶ \textcolor{darkblue}{\textbf{\ipa{dʑɤ˩-dzɯ˧ qʰwɤ˧-dzɯ˥, | bi˧mi˧ ʂv̩˧˥; | dʑɤ˩-ʐwɤ˥ qʰwɤ˩-ʐwɤ˩, | ɲi˧mi˧ ʂæ˧-pv̩˩-di˩!}}} \zh{上述谚语的变体} \textcolor{Sepia}{\selectlanguage{english}Variant of the above proverb.} \textcolor{PineGreen}{\selectlanguage{french}Variante du proverbe ci-dessus.}  

\lhead{\firstmark}
\rhead{\botmark}

\subsection{\hspace{-0.5cm} {\Large \textcolor{darkblue}{\textbf{\ipa{pv̩˧}}} \textsubscript{1}}\hspace{0.5cm}[\kern2pt{\textcolor{darkblue}{\textbf{\ipa{pv̩˥}}}}\kern2pt]} \hypertarget{pv\string_=\string_M1}{}
\markboth{\textcolor{darkblue}{\textbf{\ipa{pv̩˧}}} \textsubscript{1}}{}
\textcolor{teal}{\zh{动词}} \hspace{4pt} \zh{声调类:} M\textsubscript{c}.
\zh{祭。} \textcolor{Sepia}{\selectlanguage{english}To perform (a sacrifice, a ritual), to celebrate (a festival), to chant (a ritual).} \textcolor{PineGreen}{\selectlanguage{french}Faire un sacrifice, faire un rituel, psalmodier.}  ¶ \textcolor{darkblue}{\textbf{\ipa{kʰv̩˧ pv̩˥}}} \zh{做过年的祭礼} \textcolor{Sepia}{\selectlanguage{english}to do the New Year ceremony, to celebrate the New Year} \textcolor{PineGreen}{\selectlanguage{french}faire la veillée du Nouvel An (la veille de la nouvelle année), célébrer la veillée du Nouvel An}  
 ¶ \textcolor{darkblue}{\textbf{\ipa{tsʰi˧ɲi˧, | kʰv̩˧ pv̩˥-tso˩-ɲi˩!}}} \zh{今天就要过年了!} \textcolor{Sepia}{\selectlanguage{english}Tonight, we are going to celebrate the New Year!} \textcolor{PineGreen}{\selectlanguage{french}ce soir, on va fêter le Nouvel An!}  

\lhead{\firstmark}
\rhead{\botmark}

\subsection{\hspace{-0.5cm} {\Large \textcolor{darkblue}{\textbf{\ipa{pv̩˧}}} \textsubscript{2}}\hspace{0.5cm}[\kern2pt{\textcolor{darkblue}{\textbf{\ipa{pv̩˥}}}}\kern2pt]} \hypertarget{pv\string_=\string_M2}{}
\markboth{\textcolor{darkblue}{\textbf{\ipa{pv̩˧}}} \textsubscript{2}}{}
\textcolor{teal}{\zh{形容词}} \hspace{4pt} \zh{声调类:} M.
\zh{干燥。} \textcolor{Sepia}{\selectlanguage{english}Dry.} \textcolor{PineGreen}{\selectlanguage{french}Sec.}  ¶ \textcolor{darkblue}{\textbf{\ipa{le˧-pv̩˧-ze˧}}} \zh{干了} \textcolor{Sepia}{\selectlanguage{english}\mytextsc{accomp} \string_ \mytextsc{pfv}} \textcolor{PineGreen}{\selectlanguage{french}\mytextsc{accomp} \string_ \mytextsc{pfv}}  
 ¶ \textcolor{darkblue}{\textbf{\ipa{le˧-pv̩˧ le˧-ʐwæ˩-ze˩}}} \zh{干透了} \textcolor{Sepia}{\selectlanguage{english}It has dried up / it has completely dried / it is now completely dry} \textcolor{PineGreen}{\selectlanguage{french}ça a complètement séché/c'est entièrement sec/c'est parfaitement sec}  
 ¶ \textcolor{darkblue}{\textbf{\ipa{pv̩˧-kæ˧-ɻæ˩-gv̩˩}}} \zh{全干、完全干} \textcolor{Sepia}{\selectlanguage{english}all dry, completely dry} \textcolor{PineGreen}{\selectlanguage{french}tout sec}  

\lhead{\firstmark}
\rhead{\botmark}

\subsection{\hspace{-0.5cm} {\Large \textcolor{darkblue}{\textbf{\ipa{pv̩˧˥}}} \textsubscript{1}}\hspace{0.5cm}[\kern2pt{\textcolor{darkblue}{\textbf{\ipa{pv̩˧˥}}}}\kern2pt]} \hypertarget{pv\string_=\string_M\string_T1}{}
\markboth{\textcolor{darkblue}{\textbf{\ipa{pv̩˧˥}}} \textsubscript{1}}{}
\textcolor{teal}{\zh{动词}} \hspace{4pt} \zh{声调类:} MH.
\zh{拔、扯(草)。} \textcolor{Sepia}{\selectlanguage{english}To pull out (weeds), to weed.} \textcolor{PineGreen}{\selectlanguage{french}Enlever, arracher (les mauvaises herbes); couper du fourrage pour les animaux domestiques.}  ¶ \textcolor{darkblue}{\textbf{\ipa{zɯ˧ pv̩˩}}} \zh{拔草} \textcolor{Sepia}{\selectlanguage{english}to pull out (weeds), to weed; to cut grass for cattle} \textcolor{PineGreen}{\selectlanguage{french}arracher les mauvaises herbes; couper du fourrage pour les animaux domestiques}  
 ¶ \textcolor{darkblue}{\textbf{\ipa{zɯ˧ | le˧-pv̩˧˥}}} \zh{拔草} \textcolor{Sepia}{\selectlanguage{english}to pull out (weeds), to weed; to cut grass for cattle} \textcolor{PineGreen}{\selectlanguage{french}arracher les mauvaises herbes; couper du fourrage pour les animaux domestiques}  

\lhead{\firstmark}
\rhead{\botmark}

\subsection{\hspace{-0.5cm} {\Large \textcolor{darkblue}{\textbf{\ipa{pv̩˧˥}}} \textsubscript{2}}\hspace{0.5cm}[\kern2pt{\textcolor{darkblue}{\textbf{\ipa{pv̩˧˥}}}}\kern2pt]} \hypertarget{pv\string_=\string_M\string_T2}{}
\markboth{\textcolor{darkblue}{\textbf{\ipa{pv̩˧˥}}} \textsubscript{2}}{}
\textcolor{teal}{\zh{动词}} \hspace{4pt} \zh{声调类:} MH.
\zh{拉出(剑……)。} \textcolor{Sepia}{\selectlanguage{english}To draw (a weapon), to take out of its sheath.} \textcolor{PineGreen}{\selectlanguage{french}Dégainer (une arme blanche), sortir du fourreau.}  ¶ \textcolor{darkblue}{\textbf{\ipa{ʁæ˧mi˧ | tʰi˧-pv̩˧˥}}} \zh{拉出剑} \textcolor{Sepia}{\selectlanguage{english}to draw a sword} \textcolor{PineGreen}{\selectlanguage{french}dégainer une épée}  
 ¶ \textcolor{darkblue}{\textbf{\ipa{gæ˩-pv̩˧˥}}} \zh{拉出(剑……)} \textcolor{Sepia}{\selectlanguage{english}to draw (a weapon), to take out of its sheath} \textcolor{PineGreen}{\selectlanguage{french}dégainer, sortir une arme de son fourreau}  
 ¶ \textcolor{darkblue}{\textbf{\ipa{ʁæ˧mi˧ | gæ˩-pv̩˧˥}}} \zh{拉出剑} \textcolor{Sepia}{\selectlanguage{english}to draw a sword} \textcolor{PineGreen}{\selectlanguage{french}dégainer une épée}  

\lhead{\firstmark}
\rhead{\botmark}

\subsection{\hspace{-0.5cm} {\Large \textcolor{darkblue}{\textbf{\ipa{pv̩˧˥}}} \textsubscript{3}}\hspace{0.5cm}[\kern2pt{\textcolor{darkblue}{\textbf{\ipa{pv̩˧˥}}}}\kern2pt]} \hypertarget{pv\string_=\string_M\string_T3}{}
\markboth{\textcolor{darkblue}{\textbf{\ipa{pv̩˧˥}}} \textsubscript{3}}{}
\textcolor{teal}{\zh{量词}} \hspace{4pt} \zh{声调类:} MH\textsubscript{a}.
\zh{量词:步。} \textcolor{Sepia}{\selectlanguage{english}Classifier for steps / strides.} \textcolor{PineGreen}{\selectlanguage{french}Classificateur des pas/enjambées; emprunt au chinois.}  \zh{【借词】} \zh{步}

\lhead{\firstmark}
\rhead{\botmark}

\subsection{\hspace{-0.5cm} {\Large \textcolor{darkblue}{\textbf{\ipa{pv̩˩\textsubscript{a}}}} \textsubscript{1}}\hspace{0.5cm}[\kern2pt{\textcolor{darkblue}{\textbf{\ipa{pv̩˩˥}}}}\kern2pt]} \hypertarget{pv\string_=\string_Ba1}{}
\markboth{\textcolor{darkblue}{\textbf{\ipa{pv̩˩\textsubscript{a}}}} \textsubscript{1}}{}
\textcolor{teal}{\zh{动词}} \hspace{4pt} \zh{声调类:} L\textsubscript{a}.
\zh{送行。} \textcolor{Sepia}{\selectlanguage{english}To see off.} \textcolor{PineGreen}{\selectlanguage{french}Raccompagner; escorter; mener, conduire (du bétail).}  ¶ \textcolor{darkblue}{\textbf{\ipa{hĩ˧bæ˧ pv̩˥}}} \zh{送客} \textcolor{Sepia}{\selectlanguage{english}to see a guest off} \textcolor{PineGreen}{\selectlanguage{french}raccompagner un invité}  

\lhead{\firstmark}
\rhead{\botmark}

\subsection{\hspace{-0.5cm} {\Large \textcolor{darkblue}{\textbf{\ipa{pv̩˩\textsubscript{a}}}} \textsubscript{2}}\hspace{0.5cm}[\kern2pt{\textcolor{darkblue}{\textbf{\ipa{pv̩˩˥}}}}\kern2pt]} \hypertarget{pv\string_=\string_Ba2}{}
\markboth{\textcolor{darkblue}{\textbf{\ipa{pv̩˩\textsubscript{a}}}} \textsubscript{2}}{}
\textcolor{teal}{\zh{动词}} \hspace{4pt} \zh{声调类:} L\textsubscript{a}.
\zh{让,安排,投资,要求。} \textcolor{Sepia}{\selectlanguage{english}To allow, to authorize (someone to do something, e.g. to marry); to finance (i.e. to invest money in a caravan); to require (someone to do something).} \textcolor{PineGreen}{\selectlanguage{french}Autoriser (ex.: un mariage); demander (à quelqu'un de faire quelque chose), faire faire; commanditer, être commanditaire/investisseur (ex.: pour une caravane).}  ¶ \textcolor{darkblue}{\textbf{\ipa{sɯ˧pʰi˧-ɳɯ˧ | pv̩˩-kʰɯ˥-ɲi˩!}}} \zh{(马帮)是土司来投资的!} \textcolor{Sepia}{\selectlanguage{english}It was the feudal lord who financed (the caravan)!} \textcolor{PineGreen}{\selectlanguage{french}c'est le seigneur qui était le commanditaire!}  
 ¶ \textcolor{darkblue}{\textbf{\ipa{ʈʂʰɯ˧ | ɖʐe˧ ʂe˧ pv̩˩-kʰɯ˩-tso˩-ɲi˩!}}} \zh{是他来投资的!(如:马帮)} \textcolor{Sepia}{\selectlanguage{english}(S)he is bringing the capital! / (S)he is financing the whole thing! (e.g. a caravan)} \textcolor{PineGreen}{\selectlanguage{french}c'est elle/lui qui apporte le capital/qui commandite! (ex.: pour une caravane)}  
 ¶ \textcolor{darkblue}{\textbf{\ipa{hĩ˧-ɳɯ˩ | pv̩˩-mɤ˩-kʰɯ˥!}}} \zh{人家不让去!} \textcolor{Sepia}{\selectlanguage{english}People do not allow access! / Access is not allowed! (Context: a discussion about difficulties for the investigator to be allowed to stay in an area of Sichuan where Naish languages are spoken. The consultant summarizes as: “Access is not allowed!”)} \textcolor{PineGreen}{\selectlanguage{french}On n'est pas autorisé à y aller! (Contexte: discussion au sujet des difficultés pour l'enquêteur d'accès à une localité où sont parlées des langues naish, dans le Sichuan. La consultante résume: “On n'est pas autorisé à y aller! / L'accès n'est pas autorisé!”)}  

\lhead{\firstmark}
\rhead{\botmark}

\subsection{\hspace{-0.5cm} {\Large \textcolor{darkblue}{\textbf{\ipa{pv̩˩\textsubscript{a}}}} \textsubscript{3}}\hspace{0.5cm}[\kern2pt{\textcolor{darkblue}{\textbf{\ipa{pv̩˩˥}}}}\kern2pt]} \hypertarget{pv\string_=\string_Ba3}{}
\markboth{\textcolor{darkblue}{\textbf{\ipa{pv̩˩\textsubscript{a}}}} \textsubscript{3}}{}
\textcolor{teal}{\zh{动词}} \hspace{4pt} \zh{声调类:} L\textsubscript{a}.
\zh{梳。} \textcolor{Sepia}{\selectlanguage{english}To comb.} \textcolor{PineGreen}{\selectlanguage{french}Peigner.}  ¶ \textcolor{darkblue}{\textbf{\ipa{ʁo˧qʰwɤ˩ pv̩˩}}} \zh{梳头} \textcolor{Sepia}{\selectlanguage{english}to comb one's head} \textcolor{PineGreen}{\selectlanguage{french}se peigner}  
 ¶ \textcolor{darkblue}{\textbf{\ipa{ʁo˧ pv̩˥}}} \zh{梳头} \textcolor{Sepia}{\selectlanguage{english}to comb one's head} \textcolor{PineGreen}{\selectlanguage{french}se peigner}  

\lhead{\firstmark}
\rhead{\botmark}

\subsection{\hspace{-0.5cm} {\Large \textcolor{darkblue}{\textbf{\ipa{pv̩˩ɭɯ˥}}}}\hspace{0.5cm}[\kern2pt{\textcolor{darkblue}{\textbf{\ipa{pv̩˩ɭɯ˥}}}}\kern2pt]} \hypertarget{pv\string_=\string_Bl\string_RM\string_T1}{}
\markboth{\textcolor{darkblue}{\textbf{\ipa{pv̩˩ɭɯ˥}}}}{}
\textcolor{teal}{\zh{名词}} \hspace{4pt} \zh{声调类:} LH.
\zh{扣子。} \textcolor{Sepia}{\selectlanguage{english}Button.} \textcolor{PineGreen}{\selectlanguage{french}Bouton (sur un vêtement).}  \zh{量词}: \textcolor{darkblue}{\textbf{\ipa{ɭɯ˧}}} 
\lhead{\firstmark}
\rhead{\botmark}

\subsection{\hspace{-0.5cm} {\Large \textcolor{darkblue}{\textbf{\ipa{pv̩˧lv̩˧}}}}\hspace{0.5cm}[\kern2pt{\textcolor{darkblue}{\textbf{\ipa{pv̩˧lv̩˧}}}}\kern2pt]} \hypertarget{pv\string_=\string_Mlv\string_=\string_M1}{}
\markboth{\textcolor{darkblue}{\textbf{\ipa{pv̩˧lv̩˧}}}}{}
\textcolor{teal}{\zh{名词}} \hspace{4pt} \zh{声调类:} M.
\zh{旱地。} \textcolor{Sepia}{\selectlanguage{english}Nonirrigated farmland; dry land.} \textcolor{PineGreen}{\selectlanguage{french}Champ sec/pluvial.}  \zh{量词}: \textcolor{darkblue}{\textbf{\ipa{pʰv̩˩}}} 
\lhead{\firstmark}
\rhead{\botmark}

\subsection{\hspace{-0.5cm} {\Large \textcolor{darkblue}{\textbf{\ipa{pv̩˩mi˩}}}}\hspace{0.5cm}[\kern2pt{\textcolor{darkblue}{\textbf{\ipa{pv̩˩mi˩˥}}}}\kern2pt]} \hypertarget{pv\string_=\string_Bmi\string_B1}{}
\markboth{\textcolor{darkblue}{\textbf{\ipa{pv̩˩mi˩}}}}{}
\textcolor{teal}{\zh{名词}} \hspace{4pt} \zh{声调类:} L.
\zh{粗齿梳子。} \textcolor{Sepia}{\selectlanguage{english}Comb (coarse).} \textcolor{PineGreen}{\selectlanguage{french}Peigne grossier, à dents relativement écartées.}  \zh{量词}: \textcolor{darkblue}{\textbf{\ipa{nɑ˧}}} 
\lhead{\firstmark}
\rhead{\botmark}

\subsection{\hspace{-0.5cm} {\Large \textcolor{darkblue}{\textbf{\ipa{pv̩˩pv̩˧}}}}\hspace{0.5cm}[\kern2pt{\textcolor{darkblue}{\textbf{\ipa{pv̩˩pv̩˥}}}}\kern2pt]} \hypertarget{pv\string_=\string_Bpv\string_=\string_M1}{}
\markboth{\textcolor{darkblue}{\textbf{\ipa{pv̩˩pv̩˧}}}}{}
\textcolor{teal}{\zh{名词}} \hspace{4pt} \zh{声调类:} LM.
\zh{衣兜。} \textcolor{Sepia}{\selectlanguage{english}Pocket.} \textcolor{PineGreen}{\selectlanguage{french}Poche.}  ¶ \textcolor{darkblue}{\textbf{\ipa{bɑ˩lɑ˩-pv̩˥pv̩˩}}} \zh{上衣兜子} \textcolor{Sepia}{\selectlanguage{english}pocket of the shirt; it can contain small objects such as tobacco and coins.} \textcolor{PineGreen}{\selectlanguage{french}poche intérieure de chemise; on y serrait de petits objets: pièces de monnaie, tabac…}  
\zh{~【同义词】~} \hyperlink{}{\textcolor{darkblue}{\textbf{\ipa{tɑ˩dv̩˧˥}}}}. 
\lhead{\firstmark}
\rhead{\botmark}

\subsection{\hspace{-0.5cm} {\Large \textcolor{darkblue}{\textbf{\ipa{pv̩˧qʰwɤ˥}}}}\hspace{0.5cm}[\kern2pt{\textcolor{darkblue}{\textbf{\ipa{pv̩˧qʰwɤ˥}}}}\kern2pt]} \hypertarget{pv\string_=\string_Mq\string_hw7\string_T1}{}
\markboth{\textcolor{darkblue}{\textbf{\ipa{pv̩˧qʰwɤ˥}}}}{}
\textcolor{teal}{\zh{名词}} \hspace{4pt} \zh{声调类:} H\#.
\zh{梭,梭子(传统的,木头做的)。} \textcolor{Sepia}{\selectlanguage{english}Shuttle (traditional shuttle made of wood).} \textcolor{PineGreen}{\selectlanguage{french}Navette du métier à tisser: navette traditionnelle, en bois (n'est plus en usage actuellement, remplacée par une navette plus simple).}  ¶ \textcolor{darkblue}{\textbf{\ipa{ɣɯ˧dzo˩-bv̩˩ | pv̩˧qʰwɤ˥}}} \zh{织布机的梭子} \textcolor{Sepia}{\selectlanguage{english}the shuttle of the loom} \textcolor{PineGreen}{\selectlanguage{french}la navette du métier à tisser}  
 \zh{量词}: \textcolor{darkblue}{\textbf{\ipa{ɭɯ˧}}} \zh{~【参考】~} \hyperlink{}{\textcolor{darkblue}{\textbf{\ipa{kʰɯ˩pv̩˩}}}} 
\lhead{\firstmark}
\rhead{\botmark}

\subsection{\hspace{-0.5cm} {\Large \textcolor{darkblue}{\textbf{\ipa{pv̩˧ɻ\#˥}}}}\hspace{0.5cm}[\kern2pt{\textcolor{darkblue}{\textbf{\ipa{pv̩˧ɻ˧}}}}\kern2pt]} \hypertarget{pv\string_=\string_Mr£`\#\string_T1}{}
\markboth{\textcolor{darkblue}{\textbf{\ipa{pv̩˧ɻ\#˥}}}}{}
\textcolor{teal}{\zh{名词}} \hspace{4pt} \zh{声调类:} \#H.
\zh{氆氇。} \textcolor{Sepia}{\selectlanguage{english}Tibetan wool fabric.} \textcolor{PineGreen}{\selectlanguage{french}Habit tibétain en laine (vêtement de grand prix).}  \zh{量词}: \textcolor{darkblue}{\textbf{\ipa{tsʰi˥}}} 
\lhead{\firstmark}
\rhead{\botmark}

\subsection{\hspace{-0.5cm} {\Large \textcolor{darkblue}{\textbf{\ipa{pv̩˧ʂɯ˩}}}}\hspace{0.5cm}[\kern2pt{\textcolor{darkblue}{\textbf{\ipa{pv̩˧ʂɯ˩}}}}\kern2pt]} \hypertarget{pv\string_=\string_Ms`M\string_B1}{}
\markboth{\textcolor{darkblue}{\textbf{\ipa{pv̩˧ʂɯ˩}}}}{}
\textcolor{teal}{\zh{名词}} \hspace{4pt} \zh{声调类:} L\#.
\zh{琥珀。} \textcolor{Sepia}{\selectlanguage{english}Amber.} \textcolor{PineGreen}{\selectlanguage{french}Ambre.}  \zh{量词}: \textcolor{darkblue}{\textbf{\ipa{ɭɯ˧}}} 
\lhead{\firstmark}
\rhead{\botmark}

\subsection{\hspace{-0.5cm} {\Large \textcolor{darkblue}{\textbf{\ipa{pv̩˩tɑ˩}}}}\hspace{0.5cm}[\kern2pt{\textcolor{darkblue}{\textbf{\ipa{pv̩˩tɑ˩˥}}}}\kern2pt]} \hypertarget{pv\string_=\string_BtA\string_B1}{}
\markboth{\textcolor{darkblue}{\textbf{\ipa{pv̩˩tɑ˩}}}}{}
\textcolor{teal}{\zh{名词}} \hspace{4pt} \zh{声调类:} L.
\zh{桶。} \textcolor{Sepia}{\selectlanguage{english}Bucket, pail.} \textcolor{PineGreen}{\selectlanguage{french}Seau.}  \zh{量词}: \textcolor{darkblue}{\textbf{\ipa{ɭɯ˧}}} 
\lhead{\firstmark}
\rhead{\botmark}

\subsection{\hspace{-0.5cm} {\Large \textcolor{darkblue}{\textbf{\ipa{pv̩˩tsɯ˧˥}}}}\hspace{0.5cm}[\kern2pt{\textcolor{darkblue}{\textbf{\ipa{pv̩˩tsɯ˧˥}}}}\kern2pt]} \hypertarget{pv\string_=\string_BtsM\string_M\string_T1}{}
\markboth{\textcolor{darkblue}{\textbf{\ipa{pv̩˩tsɯ˧˥}}}}{}
\textcolor{teal}{\zh{名词}} \hspace{4pt} \zh{声调类:} LM+MH\#.
\ding{202} \zh{篦子(用来梳虱子)。} \textcolor{Sepia}{\selectlanguage{english}Fine comb (used to comb out lice).} \textcolor{PineGreen}{\selectlanguage{french}Peigne fin (utilisé pour épouiller).}  \zh{量词}: \textcolor{darkblue}{\textbf{\ipa{nɑ˧}}} \ding{203} \zh{用来夯实布料的木头架子,里面有铁丝。} \textcolor{Sepia}{\selectlanguage{english}Iron threads in a wooden frame (like a comb in which the weft is caught), used to tamp down the threads while weaving.} \textcolor{PineGreen}{\selectlanguage{french}Fils de fer dans un cadre de bois: sorte de peigne dans lequel la trame est emprisonnée, et qui sert à tasser les fils à mesure que l'on tisse.} 
\lhead{\firstmark}
\rhead{\botmark}

\subsection{\hspace{-0.5cm} {\Large \textcolor{darkblue}{\textbf{\ipa{pv̩˧ʈʂɯ˩}}}}\hspace{0.5cm}[\kern2pt{\textcolor{darkblue}{\textbf{\ipa{pv̩˧ʈʂɯ˩}}}}\kern2pt]} \hypertarget{pv\string_=\string_Mt`s`M\string_B1}{}
\markboth{\textcolor{darkblue}{\textbf{\ipa{pv̩˧ʈʂɯ˩}}}}{}
\textcolor{teal}{\zh{动词}} \hspace{4pt} \zh{声调类:} L\#.
\zh{挤、挤压。} \textcolor{Sepia}{\selectlanguage{english}To press, to squeeze.} \textcolor{PineGreen}{\selectlanguage{french}Presser, serrer.}  ¶ \textcolor{darkblue}{\textbf{\ipa{njɤ˧-ɳɯ˧ | pv̩˧ʈʂɯ˩-bi˩!}}} \zh{我来压吧!} \textcolor{Sepia}{\selectlanguage{english}I'm going to press (it)! / Let me press it!} \textcolor{PineGreen}{\selectlanguage{french}Je vais presser ça! / je m'occupe de serrer ça/presser ça!}  
 ¶ \textcolor{darkblue}{\textbf{\ipa{le˧-pv̩˥ʈʂɯ˩}}} \zh{\mytextsc{accomp}} \textcolor{Sepia}{\selectlanguage{english}\mytextsc{accomp}} \textcolor{PineGreen}{\selectlanguage{french}\mytextsc{accomp}}  

\lhead{\firstmark}
\rhead{\botmark}

\subsection{\hspace{-0.5cm} {\Large \textcolor{darkblue}{\textbf{\ipa{pv̩˩tsɯ˧-pv̩˥mi˩}}}}\hspace{0.5cm}[\kern2pt{\textcolor{darkblue}{\textbf{\ipa{pv̩˩tsɯ˧pv̩˥mi˩}}}}\kern2pt]} \hypertarget{pv\string_=\string_BtsM\string_M-pv\string_=\string_Tmi\string_B1}{}
\markboth{\textcolor{darkblue}{\textbf{\ipa{pv̩˩tsɯ˧-pv̩˥mi˩}}}}{}
\textcolor{teal}{\zh{名词}} \hspace{4pt} \zh{声调类:} LM+\#H-.
\zh{梳子(总称)。} \textcolor{Sepia}{\selectlanguage{english}Combs.} \textcolor{PineGreen}{\selectlanguage{french}Peignes.} 
\lhead{\firstmark}
\rhead{\botmark}

\subsection{\hspace{-0.5cm} {\Large \textcolor{darkblue}{\textbf{\ipa{pv̩˩ʈʰɯ˧}}}}\hspace{0.5cm}[\kern2pt{\textcolor{darkblue}{\textbf{\ipa{pv̩˩ʈʰɯ˥}}}}\kern2pt]} \hypertarget{pv\string_=\string_Bt`\string_hM\string_M1}{}
\markboth{\textcolor{darkblue}{\textbf{\ipa{pv̩˩ʈʰɯ˧}}}}{}
\textcolor{teal}{\zh{名词}} \hspace{4pt} \zh{声调类:} LM.
\zh{女性名字。} \textcolor{Sepia}{\selectlanguage{english}Feminine given name.} \textcolor{PineGreen}{\selectlanguage{french}Prénom féminin.} 
\lhead{\firstmark}
\rhead{\botmark}

\subsection{\hspace{-0.5cm} {\Large \textcolor{darkblue}{\textbf{\ipa{pv˧tsɤ\#˥}}}}\hspace{0.5cm}[\kern2pt{\textcolor{darkblue}{\textbf{\ipa{pv˧tsɤ˧}}}}\kern2pt]} \hypertarget{pv\string_Mts7\#\string_T1}{}
\markboth{\textcolor{darkblue}{\textbf{\ipa{pv˧tsɤ\#˥}}}}{}
\textcolor{teal}{\zh{名词}} \hspace{4pt} \zh{声调类:} \#H.
\zh{榫眼。} \textcolor{Sepia}{\selectlanguage{english}Mortise.} \textcolor{PineGreen}{\selectlanguage{french}Mortaise.}  ¶ \textcolor{darkblue}{\textbf{\ipa{pv˧tsɤ˧ | ɖɯ˧-ɭɯ˧}}} \zh{一个榫} \textcolor{Sepia}{\selectlanguage{english}a mortise} \textcolor{PineGreen}{\selectlanguage{french}une mortaise}  
 \zh{量词}: \textcolor{darkblue}{\textbf{\ipa{ɭɯ˧}}} 
\lhead{\firstmark}
\rhead{\botmark}

\newpage
\section*{\centering- \textcolor{darkblue}{\textbf{\ipa{pʰ}}} -}
\subsection{\hspace{-0.5cm} {\Large \textcolor{darkblue}{\textbf{\ipa{pʰæ˧\textsubscript{b}}}}}\hspace{0.5cm}[\kern2pt{\textcolor{darkblue}{\textbf{\ipa{pʰæ˥}}}}\kern2pt]} \hypertarget{p\string_h\{\string_Mb1}{}
\markboth{\textcolor{darkblue}{\textbf{\ipa{pʰæ˧\textsubscript{b}}}}}{}
\textcolor{teal}{\zh{动词}} \hspace{4pt} \zh{声调类:} M\textsubscript{b}.
\ding{202} \zh{拴(牛……)。} \textcolor{Sepia}{\selectlanguage{english}To tie, to fasten (an animal).} \textcolor{PineGreen}{\selectlanguage{french}Attacher (un animal).}  ¶ \textcolor{darkblue}{\textbf{\ipa{tʰi˧-pʰæ˧+ze˧}}} \zh{\mytextsc{dur} \string_ \mytextsc{pfv}} \textcolor{Sepia}{\selectlanguage{english}\mytextsc{dur} \string_ \mytextsc{pfv}} \textcolor{PineGreen}{\selectlanguage{french}\mytextsc{dur} \string_ \mytextsc{pfv}}  
 ¶ \textcolor{darkblue}{\textbf{\ipa{pʰæ˧\textasciitilde{}pʰæ˧}}} \zh{\mytextsc{重叠}} \textcolor{Sepia}{\selectlanguage{english}\mytextsc{red}} \textcolor{PineGreen}{\selectlanguage{french}\mytextsc{red}}  
\ding{203} \zh{有联系,有缘分,有深的关系。} \textcolor{Sepia}{\selectlanguage{english}To be linked, to have ties: e.g. the members of a family have ties.} \textcolor{PineGreen}{\selectlanguage{french}Être lié, avoir des liens étroits: par exemple, les membres d'une famille ont des liens profonds.}  ¶ \textcolor{darkblue}{\textbf{\ipa{pʰæ˧\textasciitilde{}pʰæ˧=ɻæ˩ ɲi˩!}}} \zh{他们有联系了/他们成了一俩了!(关于两个年轻人)} \textcolor{Sepia}{\selectlanguage{english}They are united / they make up a couple / they are united into a couple (about two young people)} \textcolor{PineGreen}{\selectlanguage{french}[Ils] sont unis, ils sont en couple! (au sujet de deux jeunes personnes)}  

\lhead{\firstmark}
\rhead{\botmark}

\subsection{\hspace{-0.5cm} {\Large \textcolor{darkblue}{\textbf{\ipa{pʰæ˧qʰwɤ˩}}}}\hspace{0.5cm}[\kern2pt{\textcolor{darkblue}{\textbf{\ipa{pʰæ˩qʰwɤ˧˥}}}}\kern2pt]} \hypertarget{p\string_h\{\string_Mq\string_hw7\string_B1}{}
\markboth{\textcolor{darkblue}{\textbf{\ipa{pʰæ˧qʰwɤ˩}}}}{}
\textcolor{teal}{\zh{名词}} \hspace{4pt} \zh{声调类:} L\#.
\zh{脸。} \textcolor{Sepia}{\selectlanguage{english}Face.} \textcolor{PineGreen}{\selectlanguage{french}Visage.}  \zh{量词}: \textcolor{darkblue}{\textbf{\ipa{ɭɯ˧}}} 
\lhead{\firstmark}
\rhead{\botmark}

\subsection{\hspace{-0.5cm} {\Large \textcolor{darkblue}{\textbf{\ipa{pʰæ˧ʂv̩˧-di˧˥}}}}\hspace{0.5cm}[\kern2pt{\textcolor{darkblue}{\textbf{\ipa{xxxx non-correspondance entre le nombre de morphèmes et le nombre de tons de morphèmes}}}}\kern2pt]} \hypertarget{p\string_h\{\string_Ms`v\string_=\string_M-di\string_M\string_T1}{}
\markboth{\textcolor{darkblue}{\textbf{\ipa{pʰæ˧ʂv̩˧-di˧˥}}}}{}
\textcolor{teal}{\zh{名词}} \hspace{4pt} \zh{声调类:} .
\zh{围巾。} \textcolor{Sepia}{\selectlanguage{english}Scarf.} \textcolor{PineGreen}{\selectlanguage{french}Foulard (périphrase); autrefois, on utilisait un fichu, \textcolor{darkblue}{\textbf{\ipa{/qʰwæ˧ʈɯ˥/}}}.} 
\lhead{\firstmark}
\rhead{\botmark}

\subsection{\hspace{-0.5cm} {\Large \textcolor{darkblue}{\textbf{\ipa{pʰæ˧tɕi˥}}}}\hspace{0.5cm}[\kern2pt{\textcolor{darkblue}{\textbf{\ipa{pʰæ˧tɕi˥}}}}\kern2pt]} \hypertarget{p\string_h\{\string_Mts£i\string_T1}{}
\markboth{\textcolor{darkblue}{\textbf{\ipa{pʰæ˧tɕi˥}}}}{}
\textcolor{teal}{\zh{名词}} \hspace{4pt} \zh{声调类:} H\#.
\ding{202} \zh{小伙子、 青年男子。} \textcolor{Sepia}{\selectlanguage{english}Young man, young chap, young lad.} \textcolor{PineGreen}{\selectlanguage{french}Jeune homme, petit gars.}  ¶ \textcolor{darkblue}{\textbf{\ipa{pʰæ˧tɕi˥-zo˩}}} \zh{小伙子} \textcolor{Sepia}{\selectlanguage{english}young man} \textcolor{PineGreen}{\selectlanguage{french}jeune homme}  
 ¶ \textcolor{darkblue}{\textbf{\ipa{pʰæ˧tɕi˥=ɻæ˩}}} \zh{小伙子们} \textcolor{Sepia}{\selectlanguage{english}young men} \textcolor{PineGreen}{\selectlanguage{french}jeunes hommes; les jeunes hommes}  
 \zh{量词}: \textcolor{darkblue}{\textbf{\ipa{v̩˧}}} \ding{203} \zh{第一根柱子的名称(代表男人、男性的那根柱子)。} \textcolor{Sepia}{\selectlanguage{english}Name of the first pillar in the main room, the one closest to the door (masculine pillar, the other one being feminine).} \textcolor{PineGreen}{\selectlanguage{french}Nom du premier pilier (il y a deux grands piliers dans la maison traditionnelle), celui qui est le plus près de la porte: c'est le pilier masculin, “le jeune homme”, le second étant féminin, “la jeune femme”.}  \zh{量词}: \textcolor{darkblue}{\textbf{\ipa{v̩˧}}} 
\lhead{\firstmark}
\rhead{\botmark}

\subsection{\hspace{-0.5cm} {\Large \textcolor{darkblue}{\textbf{\ipa{pʰæ˧ʈʂʰæ˧lo\#˥}}}}\hspace{0.5cm}[\kern2pt{\textcolor{darkblue}{\textbf{\ipa{pʰæ˧ʈʂʰæ˧lo˩}}}}\kern2pt]} \hypertarget{p\string_h\{\string_Mt`s`\string_h\{\string_Mlo\#\string_T1}{}
\markboth{\textcolor{darkblue}{\textbf{\ipa{pʰæ˧ʈʂʰæ˧lo\#˥}}}}{}
\textcolor{teal}{\zh{名词}} \hspace{4pt} \zh{声调类:} \#H.
\zh{脸盆,木盆。} \textcolor{Sepia}{\selectlanguage{english}Washbasin, basin to wash one's face.} \textcolor{PineGreen}{\selectlanguage{french}Bassine pour se laver le visage; le même ustensile est utilisé pour servir les nourritures si un bol serait trop petit: pour servir le riz, les soupes...}  \zh{量词}: \textcolor{darkblue}{\textbf{\ipa{ɭɯ˧}}} 
\lhead{\firstmark}
\rhead{\botmark}

\subsection{\hspace{-0.5cm} {\Large \textcolor{darkblue}{\textbf{\ipa{pʰæ˧˥}}}}\hspace{0.5cm}[\kern2pt{\textcolor{darkblue}{\textbf{\ipa{pʰæ˧˥}}}}\kern2pt]} \hypertarget{p\string_h\{\string_M\string_T1}{}
\markboth{\textcolor{darkblue}{\textbf{\ipa{pʰæ˧˥}}}}{}
\textcolor{teal}{\zh{动词}} \hspace{4pt} \zh{声调类:} MH.
\zh{推搡。} \textcolor{Sepia}{\selectlanguage{english}To shove, to push away.} \textcolor{PineGreen}{\selectlanguage{french}Écarter, pousser, jouer des coudes.}  ¶ \textcolor{darkblue}{\textbf{\ipa{ɖɯ˩-tɕo˧ pʰæ˧˥, | ʈʂʰɯ˧-tɕo˧ pʰæ˧˥}}} \zh{东推西挤} \textcolor{Sepia}{\selectlanguage{english}to shove on this side, to shove on that side (e.g. at a station, when lots of people are shoving their way towards the ticket counter)} \textcolor{PineGreen}{\selectlanguage{french}pousser par ici, pousser par là / jouer des coudes par ci, jouer des coudes par là (ex.: à la gare, quand il y a presse pour acheter un billet de train)}  
 ¶ \textcolor{darkblue}{\textbf{\ipa{[Housebuilding2] ʈʂe˧ | le˧-pʰæ˩\textasciitilde{}pʰæ˩}}} \zh{将土扔这里扔那里:一只鸡在抓地找吃的,让土飞这里飞那里} \textcolor{Sepia}{\selectlanguage{english}to throw earth here and there: a chicken is scratching the soil to find food, and sends spurts of earth here and there} \textcolor{PineGreen}{\selectlanguage{french}rejeter la terre de droite et de gauche : une poule gratte la terre à la recherche de nourriture, et fait voler de la terre de droite et de gauche}  

\lhead{\firstmark}
\rhead{\botmark}

\subsection{\hspace{-0.5cm} {\Large \textcolor{darkblue}{\textbf{\ipa{pʰæ˧˥\textsubscript{a}}}}}\hspace{0.5cm}[\kern2pt{\textcolor{darkblue}{\textbf{\ipa{pʰæ˧˥}}}}\kern2pt]} \hypertarget{p\string_h\{\string_M\string_Ta1}{}
\markboth{\textcolor{darkblue}{\textbf{\ipa{pʰæ˧˥\textsubscript{a}}}}}{}
\textcolor{teal}{\zh{量词}} \hspace{4pt} \zh{声调类:} MH\textsubscript{a}.
\zh{量词:平面的东西,如:纸(一张、一片)。} \textcolor{Sepia}{\selectlanguage{english}Classifier for flat objects, e.g. a sheet (of paper).} \textcolor{PineGreen}{\selectlanguage{french}Classificateur des objets plats: feuilles de papier….} 
\lhead{\firstmark}
\rhead{\botmark}

\subsection{\hspace{-0.5cm} {\Large \textcolor{darkblue}{\textbf{\ipa{pʰe˧}}}}\hspace{0.5cm}[\kern2pt{\textcolor{darkblue}{\textbf{\ipa{pʰe˥}}}}\kern2pt]} \hypertarget{p\string_he\string_M1}{}
\markboth{\textcolor{darkblue}{\textbf{\ipa{pʰe˧}}}}{}
\textcolor{teal}{\zh{感叹词}} \hspace{4pt} \zh{声调类:} M.
\zh{呸!(表示唾弃的感叹词)。} \textcolor{Sepia}{\selectlanguage{english}Interjection: No way! The speaker signals that the interlocutor is making wrong statements, and that (s)he (the speaker) will now put forward different views.} \textcolor{PineGreen}{\selectlanguage{french}Mais pas du tout! Mais non, enfin! Interjection par laquelle le locuteur signale qu'il reprend la main: que ses interlocuteurs lui paraissent être dans l'erreur, et qu'il va rectifier.} 
\lhead{\firstmark}
\rhead{\botmark}

\subsection{\hspace{-0.5cm} {\Large \textcolor{darkblue}{\textbf{\ipa{pʰɤ˧bɤ˧}}}}\hspace{0.5cm}[\kern2pt{\textcolor{darkblue}{\textbf{\ipa{pʰɤ˧bɤ˧}}}}\kern2pt]} \hypertarget{p\string_h7\string_Mb7\string_M1}{}
\markboth{\textcolor{darkblue}{\textbf{\ipa{pʰɤ˧bɤ˧}}}}{}
\textcolor{teal}{\zh{名词}} \hspace{4pt} \zh{声调类:} M.
\zh{礼物。} \textcolor{Sepia}{\selectlanguage{english}Gift, present (typical gifts are tobacco, tea leaf, candies, and wine; one does not usually offer clothes, apart from specific ritual occasions).} \textcolor{PineGreen}{\selectlanguage{french}Cadeau (choses à manger ou boire; essentiellement: tabac, thé, bonbons, vin; on n'offre généralement pas de vêtements).}  ¶ \textcolor{darkblue}{\textbf{\ipa{pʰɤ˧bɤ˧ po˧-tsʰɯ˧˥}}} \zh{带礼物} \textcolor{Sepia}{\selectlanguage{english}to bring gifts} \textcolor{PineGreen}{\selectlanguage{french}amener des cadeaux}  
 ¶ \textcolor{darkblue}{\textbf{\ipa{ʈʂʰɯ˧ | ʈæ˧ʂɯ˧ ki˧-hĩ˧ pʰɤ˧bɤ˧ ŋi˩.}}} \zh{这是达石给的礼物!} \textcolor{Sepia}{\selectlanguage{english}This is a gift from Dashi!} \textcolor{PineGreen}{\selectlanguage{french}C'est un cadeau que m'a donné Dashi!}  
 ¶ \textcolor{darkblue}{\textbf{\ipa{ʈʂʰɯ˧ | ʈæ˧ʂɯ˧ tʰi˧-ki˧-hĩ˧ pʰɤ˧bɤ˧ ŋi˩.}}} \zh{这是达石送你的礼物!} \textcolor{Sepia}{\selectlanguage{english}This is a gift from Dashi! / Here is a gift for you from Dashi!} \textcolor{PineGreen}{\selectlanguage{french}C'est un cadeau que Dashi te fait! Voici un cadeau de la part de Dashi!}  
 \zh{量词}: \textcolor{darkblue}{\textbf{\ipa{kʰwɤ˥}}} 
\lhead{\firstmark}
\rhead{\botmark}

\subsection{\hspace{-0.5cm} {\Large \textcolor{darkblue}{\textbf{\ipa{pʰɤ˧fv̩˩}}}}\hspace{0.5cm}[\kern2pt{\textcolor{darkblue}{\textbf{\ipa{pʰɤ˧fv̩˩}}}}\kern2pt]} \hypertarget{p\string_h7\string_Mfv\string_=\string_B1}{}
\markboth{\textcolor{darkblue}{\textbf{\ipa{pʰɤ˧fv̩˩}}}}{}
\textcolor{teal}{\zh{名词}} \hspace{4pt} \zh{声调类:} L\#.
\zh{茶壶。} \textcolor{Sepia}{\selectlanguage{english}Teapot.} \textcolor{PineGreen}{\selectlanguage{french}Théière.}  \zh{【借词】} \zh{壶?}

\lhead{\firstmark}
\rhead{\botmark}

\subsection{\hspace{-0.5cm} {\Large \textcolor{darkblue}{\textbf{\ipa{pʰɤ˧pʰv̩\#˥}}}}\hspace{0.5cm}[\kern2pt{\textcolor{darkblue}{\textbf{\ipa{pʰɤ˩pʰv̩˩˥}}}}\kern2pt]} \hypertarget{p\string_h7\string_Mp\string_hv\string_=\#\string_T1}{}
\markboth{\textcolor{darkblue}{\textbf{\ipa{pʰɤ˧pʰv̩\#˥}}}}{}
\textcolor{teal}{\zh{名词}} \hspace{4pt} \zh{声调类:} \#H.
\zh{公豺。} \textcolor{Sepia}{\selectlanguage{english}Male jackal.} \textcolor{PineGreen}{\selectlanguage{french}Chacal mâle.}  \zh{量词}: \textcolor{darkblue}{\textbf{\ipa{mi˩}}} 
\lhead{\firstmark}
\rhead{\botmark}

\subsection{\hspace{-0.5cm} {\Large \textcolor{darkblue}{\textbf{\ipa{pʰɤ˩mi˩}}}}\hspace{0.5cm}[\kern2pt{\textcolor{darkblue}{\textbf{\ipa{pʰɤ˩mi˩˥}}}}\kern2pt]} \hypertarget{p\string_h7\string_Bmi\string_B1}{}
\markboth{\textcolor{darkblue}{\textbf{\ipa{pʰɤ˩mi˩}}}}{}
\textcolor{teal}{\zh{名词}} \hspace{4pt} \zh{声调类:} L.
\zh{母豺。} \textcolor{Sepia}{\selectlanguage{english}Female jackal.} \textcolor{PineGreen}{\selectlanguage{french}Femelle chacal.}  \zh{量词}: \textcolor{darkblue}{\textbf{\ipa{mi˩}}} 
\lhead{\firstmark}
\rhead{\botmark}

\subsection{\hspace{-0.5cm} {\Large \textcolor{darkblue}{\textbf{\ipa{pʰɤ˩-so˩\textasciitilde{}so˥}}}}\hspace{0.5cm}[\kern2pt{\textcolor{darkblue}{\textbf{\ipa{xxxx non-correspondance entre le nombre de morphèmes et le nombre de tons de morphèmes}}}}\kern2pt]} \hypertarget{p\string_h7\string_B-so\string_B~so\string_T1}{}
\markboth{\textcolor{darkblue}{\textbf{\ipa{pʰɤ˩-so˩\textasciitilde{}so˥}}}}{}
\textcolor{teal}{\zh{形容词}} \hspace{4pt} \zh{声调类:} .
\zh{松(土)。} \textcolor{Sepia}{\selectlanguage{english}Loose (the soil is loose after being forked over).} \textcolor{PineGreen}{\selectlanguage{french}Meuble: la terre est meuble.}  ¶ \textcolor{darkblue}{\textbf{\ipa{ʈʂe˧ | pʰɤ˩-so˩\textasciitilde{}so˥-gv̩˩}}} \zh{土是松的} \textcolor{Sepia}{\selectlanguage{english}the soil is loose, the soil has been loosened} \textcolor{PineGreen}{\selectlanguage{french}la terre est meuble, la terre a été ameublie}  

\lhead{\firstmark}
\rhead{\botmark}

\subsection{\hspace{-0.5cm} {\Large \textcolor{darkblue}{\textbf{\ipa{pʰɤ˩zo˩}}}}\hspace{0.5cm}[\kern2pt{\textcolor{darkblue}{\textbf{\ipa{pʰɤ˩zo˩˥}}}}\kern2pt]} \hypertarget{p\string_h7\string_Bzo\string_B1}{}
\markboth{\textcolor{darkblue}{\textbf{\ipa{pʰɤ˩zo˩}}}}{}
\textcolor{teal}{\zh{名词}} \hspace{4pt} \zh{声调类:} L.
\zh{豺崽子。} \textcolor{Sepia}{\selectlanguage{english}Baby jackal.} \textcolor{PineGreen}{\selectlanguage{french}Petit chacal.}  \zh{量词}: \textcolor{darkblue}{\textbf{\ipa{ɭɯ˧}}} 
\lhead{\firstmark}
\rhead{\botmark}

\subsection{\hspace{-0.5cm} {\Large \textcolor{darkblue}{\textbf{\ipa{pʰɤ˩˧}}}}\hspace{0.5cm}[\kern2pt{\textcolor{darkblue}{\textbf{\ipa{pʰɤ˩˥}}}}\kern2pt]} \hypertarget{p\string_h7\string_B\string_M1}{}
\markboth{\textcolor{darkblue}{\textbf{\ipa{pʰɤ˩˧}}}}{}
\textcolor{teal}{\zh{名词}} \hspace{4pt} \zh{声调类:} LM.
\zh{豺。} \textcolor{Sepia}{\selectlanguage{english}Jackal, hyena.} \textcolor{PineGreen}{\selectlanguage{french}Hyène, chacal.}  ¶ \textcolor{darkblue}{\textbf{\ipa{pʰɤ˩ hwæ˧-ze˧}}} \zh{买了豺} \textcolor{Sepia}{\selectlanguage{english}...bought (a/the) jackal} \textcolor{PineGreen}{\selectlanguage{french}...a acheté (une) hyène}  
 ¶ \textcolor{darkblue}{\textbf{\ipa{pʰɤ˩ dzɯ˧-ze˩}}} \zh{吃了豺} \textcolor{Sepia}{\selectlanguage{english}...ate jackal} \textcolor{PineGreen}{\selectlanguage{french}...a mangé (une) hyène}  
 \zh{量词}: \textcolor{darkblue}{\textbf{\ipa{mi˩}}} 
\lhead{\firstmark}
\rhead{\botmark}

\subsection{\hspace{-0.5cm} {\Large \textcolor{darkblue}{\textbf{\ipa{pʰi˧}}}}\hspace{0.5cm}[\kern2pt{\textcolor{darkblue}{\textbf{\ipa{pʰi˥}}}}\kern2pt]} \hypertarget{p\string_hi\string_M1}{}
\markboth{\textcolor{darkblue}{\textbf{\ipa{pʰi˧}}}}{}
\textcolor{teal}{\zh{名词}} \hspace{4pt} \zh{声调类:} M.
\zh{麻布,亚麻布。} \textcolor{Sepia}{\selectlanguage{english}Linen cloth.} \textcolor{PineGreen}{\selectlanguage{french}Tissu de lin; anciennement le tissu dont étaient faits ts les vêtements (cf récit “travaux”).}  ¶ \textcolor{darkblue}{\textbf{\ipa{pʰi˩ dɑ˩˥}}} \zh{织麻布} \textcolor{Sepia}{\selectlanguage{english}to weave linen} \textcolor{PineGreen}{\selectlanguage{french}tisser le lin, faire du tissu de lin}  
 \zh{量词}: \textcolor{darkblue}{\textbf{\ipa{kʰwɤ˥}}} 
\lhead{\firstmark}
\rhead{\botmark}

\subsection{\hspace{-0.5cm} {\Large \textcolor{darkblue}{\textbf{\ipa{pʰi˧\textsubscript{b}}}}}\hspace{0.5cm}[\kern2pt{\textcolor{darkblue}{\textbf{\ipa{pʰi˧˥}}}}\kern2pt]} \hypertarget{p\string_hi\string_Mb1}{}
\markboth{\textcolor{darkblue}{\textbf{\ipa{pʰi˧\textsubscript{b}}}}}{}
\textcolor{teal}{\zh{动词}} \hspace{4pt} \zh{声调类:} M\textsubscript{b}.
\zh{簸。} \textcolor{Sepia}{\selectlanguage{english}To winnow with a fan.} \textcolor{PineGreen}{\selectlanguage{french}Vanner à l'aide d'un crible (vannerie ronde): on fait “sauter” le grain dans un crible, et le vent emporte la balle.}  ¶ \textcolor{darkblue}{\textbf{\ipa{hɑ˧ pʰi˧}}} \zh{簸粮食} \textcolor{Sepia}{\selectlanguage{english}to winnow cereals} \textcolor{PineGreen}{\selectlanguage{french}vanner du grain}  
 ¶ \textcolor{darkblue}{\textbf{\ipa{pʰi˧\textasciitilde{}pʰi˧}}} \zh{\mytextsc{重叠:簸一簸}} \textcolor{Sepia}{\selectlanguage{english}\mytextsc{red}} \textcolor{PineGreen}{\selectlanguage{french}\mytextsc{red}}  
 ¶ \textcolor{darkblue}{\textbf{\ipa{le˧-pʰi˧(-ze˧)}}} \zh{簸了} \textcolor{Sepia}{\selectlanguage{english}\mytextsc{accomp} \string_ (\mytextsc{pfv})} \textcolor{PineGreen}{\selectlanguage{french}\mytextsc{accomp} \string_ (\mytextsc{pfv})}  

\lhead{\firstmark}
\rhead{\botmark}

\subsection{\hspace{-0.5cm} {\Large \textcolor{darkblue}{\textbf{\ipa{pʰi˧kʰv̩˧}}}}\hspace{0.5cm}[\kern2pt{\textcolor{darkblue}{\textbf{\ipa{pʰi˩kʰv̩˥}}}}\kern2pt]} \hypertarget{p\string_hi\string_Mk\string_hv\string_=\string_M1}{}
\markboth{\textcolor{darkblue}{\textbf{\ipa{pʰi˧kʰv̩˧}}}}{}
\textcolor{teal}{\zh{名词}} \hspace{4pt} \zh{声调类:} M.
\zh{畚箕。} \textcolor{Sepia}{\selectlanguage{english}Dustpan, wicker scoop, dirt-scooping implement.} \textcolor{PineGreen}{\selectlanguage{french}Pelle à poussière.}  \zh{量词}: \textcolor{darkblue}{\textbf{\ipa{nɑ˧}}} 
\lhead{\firstmark}
\rhead{\botmark}

\subsection{\hspace{-0.5cm} {\Large \textcolor{darkblue}{\textbf{\ipa{pʰi˧kʰv̩˧}}}}\hspace{0.5cm}[\kern2pt{\textcolor{darkblue}{\textbf{\ipa{pʰi˧kʰv̩˧}}}}\kern2pt]} \hypertarget{p\string_hi\string_Mk\string_hv\string_=\string_M1}{}
\markboth{\textcolor{darkblue}{\textbf{\ipa{pʰi˧kʰv̩˧}}}}{}
\textcolor{teal}{\zh{名词}} \hspace{4pt} \zh{声调类:} M.
\zh{贝壳。} \textcolor{Sepia}{\selectlanguage{english}Clamshell.} \textcolor{PineGreen}{\selectlanguage{french}Coquillage.} 
\lhead{\firstmark}
\rhead{\botmark}

\subsection{\hspace{-0.5cm} {\Large \textcolor{darkblue}{\textbf{\ipa{pʰi˧li˩}}}}\hspace{0.5cm}[\kern2pt{\textcolor{darkblue}{\textbf{\ipa{pʰi˧li˧}}}}\kern2pt]} \hypertarget{p\string_hi\string_Mli\string_B1}{}
\markboth{\textcolor{darkblue}{\textbf{\ipa{pʰi˧li˩}}}}{}
\textcolor{teal}{\zh{名词}} \hspace{4pt} \zh{声调类:} L\#.
\zh{蝴蝶。} \textcolor{Sepia}{\selectlanguage{english}Butterfly.} \textcolor{PineGreen}{\selectlanguage{french}Papillon.}  \zh{量词}: \textcolor{darkblue}{\textbf{\ipa{mi˩}}} 
\lhead{\firstmark}
\rhead{\botmark}

\subsection{\hspace{-0.5cm} {\Large \textcolor{darkblue}{\textbf{\ipa{pʰi˧mo˩}}} \textsubscript{1}}\hspace{0.5cm}[\kern2pt{\textcolor{darkblue}{\textbf{\ipa{pʰi˧mo˩}}}}\kern2pt]} \hypertarget{p\string_hi\string_Mmo\string_B1}{}
\markboth{\textcolor{darkblue}{\textbf{\ipa{pʰi˧mo˩}}} \textsubscript{1}}{}
\textcolor{teal}{\zh{名词}} \hspace{4pt} \zh{声调类:} L\#.
\zh{簸箕(用来簸粮食等)。} \textcolor{Sepia}{\selectlanguage{english}Winnowing fan.} \textcolor{PineGreen}{\selectlanguage{french}Vanneuse.}  \zh{量词}: \textcolor{darkblue}{\textbf{\ipa{nɑ˧}}} \zh{~【参考】~} \hyperlink{}{\textcolor{darkblue}{\textbf{\ipa{pʰi˧mo˩}}} \textsubscript{2}} 
\lhead{\firstmark}
\rhead{\botmark}

\subsection{\hspace{-0.5cm} {\Large \textcolor{darkblue}{\textbf{\ipa{pʰi˧mo˩}}} \textsubscript{2}}\hspace{0.5cm}[\kern2pt{\textcolor{darkblue}{\textbf{\ipa{pʰi˧mo˩}}}}\kern2pt]} \hypertarget{p\string_hi\string_Mmo\string_B2}{}
\markboth{\textcolor{darkblue}{\textbf{\ipa{pʰi˧mo˩}}} \textsubscript{2}}{}
\textcolor{teal}{\zh{名词}} \hspace{4pt} \zh{声调类:} L\#.
\zh{抓鸟的圈套。} \textcolor{Sepia}{\selectlanguage{english}Snare to catch birds.} \textcolor{PineGreen}{\selectlanguage{french}Piège pour attraper des oiseaux.}  ¶ \textcolor{darkblue}{\textbf{\ipa{v̩˩dze˩ qo˥-di˩, | pʰi˧mo˩!}}} \zh{抓鸟的东西,(叫做)圈套!} \textcolor{Sepia}{\selectlanguage{english}The thing to catch birds is called “snare”!} \textcolor{PineGreen}{\selectlanguage{french}Ce dont on se sert pour attraper les oiseaux, on appelle ça “piège pour oiseaux”!}  
\zh{~【参考】~} \hyperlink{}{\textcolor{darkblue}{\textbf{\ipa{pʰi˧mo˩}}} \textsubscript{1}} 
\lhead{\firstmark}
\rhead{\botmark}

\subsection{\hspace{-0.5cm} {\Large \textcolor{darkblue}{\textbf{\ipa{pʰi˧tsʰo\#˥}}}}\hspace{0.5cm}[\kern2pt{\textcolor{darkblue}{\textbf{\ipa{pʰi˧tsʰo˧}}}}\kern2pt]} \hypertarget{p\string_hi\string_Mts\string_ho\#\string_T1}{}
\markboth{\textcolor{darkblue}{\textbf{\ipa{pʰi˧tsʰo\#˥}}}}{}
\textcolor{teal}{\zh{名词}} \hspace{4pt} \zh{声调类:} \#H.
\zh{男性名字。} \textcolor{Sepia}{\selectlanguage{english}Masculine given name.} \textcolor{PineGreen}{\selectlanguage{french}Prénom masculin.} 
\lhead{\firstmark}
\rhead{\botmark}

\subsection{\hspace{-0.5cm} {\Large \textcolor{darkblue}{\textbf{\ipa{pʰi˧ʈʂæ˧}}}}\hspace{0.5cm}[\kern2pt{\textcolor{darkblue}{\textbf{\ipa{pʰi˧ʈʂæ˧}}}}\kern2pt]} \hypertarget{p\string_hi\string_Mt`s`\{\string_M1}{}
\markboth{\textcolor{darkblue}{\textbf{\ipa{pʰi˧ʈʂæ˧}}}}{}
\textcolor{teal}{\zh{动词}} \hspace{4pt} \zh{声调类:} M.
\zh{披毡(汉语借词)。} \textcolor{Sepia}{\selectlanguage{english}To drape oneself in a felt cape, to drape a piece of felt over one's shoulders.} \textcolor{PineGreen}{\selectlanguage{french}Se draper d'un feutre.}  \zh{【借词】} \zh{披毡}

\lhead{\firstmark}
\rhead{\botmark}

\subsection{\hspace{-0.5cm} {\Large \textcolor{darkblue}{\textbf{\ipa{pʰi˩}}}}\hspace{0.5cm}[\kern2pt{\textcolor{darkblue}{\textbf{\ipa{pʰi˥}}}}\kern2pt]} \hypertarget{p\string_hi\string_B1}{}
\markboth{\textcolor{darkblue}{\textbf{\ipa{pʰi˩}}}}{}
\textcolor{teal}{\zh{形容词}} \hspace{4pt} \zh{声调类:} L.
\zh{平(汉语借词)。} \textcolor{Sepia}{\selectlanguage{english}Flat.} \textcolor{PineGreen}{\selectlanguage{french}Plat.}  \zh{【借词】} \zh{平}

\lhead{\firstmark}
\rhead{\botmark}

\subsection{\hspace{-0.5cm} {\Large \textcolor{darkblue}{\textbf{\ipa{pʰi˩hæ˩}}}}\hspace{0.5cm}[\kern2pt{\textcolor{darkblue}{\textbf{\ipa{pʰi˧hæ˧}}}}\kern2pt]} \hypertarget{p\string_hi\string_Bh\{\string_B1}{}
\markboth{\textcolor{darkblue}{\textbf{\ipa{pʰi˩hæ˩}}}}{}
\textcolor{teal}{\zh{名词}} \hspace{4pt} \zh{声调类:} L.
\zh{凉鞋。汉语借词:第一个音节:皮,第二个音节:不明确,同\textcolor{darkblue}{\textbf{\ipa{/tɕæ˧hæ˩/}}}。} \textcolor{Sepia}{\selectlanguage{english}Sandal.} \textcolor{PineGreen}{\selectlanguage{french}Sandale.}  \zh{【借词】} \zh{皮} (second syllable: not identified)
 \zh{量词}: \textcolor{darkblue}{\textbf{\ipa{dzi˧}}} 
\lhead{\firstmark}
\rhead{\botmark}

\subsection{\hspace{-0.5cm} {\Large \textcolor{darkblue}{\textbf{\ipa{pʰi˩ko˧}}}}\hspace{0.5cm}[\kern2pt{\textcolor{darkblue}{\textbf{\ipa{pʰi˩ko˩˥}}}}\kern2pt]} \hypertarget{p\string_hi\string_Bko\string_M1}{}
\markboth{\textcolor{darkblue}{\textbf{\ipa{pʰi˩ko˧}}}}{}
\textcolor{teal}{\zh{名词}} \hspace{4pt} \zh{声调类:} LM.
\zh{苹果。} \textcolor{Sepia}{\selectlanguage{english}Apple.} \textcolor{PineGreen}{\selectlanguage{french}Pomme.}  \zh{【借词】} \zh{苹果}
 \zh{量词}: \textcolor{darkblue}{\textbf{\ipa{ɭɯ˧}}} 
\lhead{\firstmark}
\rhead{\botmark}

\subsection{\hspace{-0.5cm} {\Large \textcolor{darkblue}{\textbf{\ipa{pʰi˩tʰɑ˩}}}}\hspace{0.5cm}[\kern2pt{\textcolor{darkblue}{\textbf{\ipa{pʰi˩tʰɑ˩˥}}}}\kern2pt]} \hypertarget{p\string_hi\string_Bt\string_hA\string_B1}{}
\markboth{\textcolor{darkblue}{\textbf{\ipa{pʰi˩tʰɑ˩}}}}{}
\textcolor{teal}{\zh{名词}} \hspace{4pt} \zh{声调类:} L.
\zh{火草。} \textcolor{Sepia}{\selectlanguage{english}Tinder, touchwood.} \textcolor{PineGreen}{\selectlanguage{french}Amadou.}  \zh{量词}: \textcolor{darkblue}{\textbf{\ipa{po˧}}} 
\lhead{\firstmark}
\rhead{\botmark}

\subsection{\hspace{-0.5cm} {\Large \textcolor{darkblue}{\textbf{\ipa{pʰi˧˥}}}}\hspace{0.5cm}[\kern2pt{\textcolor{darkblue}{\textbf{\ipa{pʰi˥}}}}\kern2pt]} \hypertarget{p\string_hi\string_M\string_T1}{}
\markboth{\textcolor{darkblue}{\textbf{\ipa{pʰi˧˥}}}}{}
\textcolor{teal}{\zh{动词}} \hspace{4pt} \zh{声调类:} MH.
\zh{呕吐。} \textcolor{Sepia}{\selectlanguage{english}To vomit.} \textcolor{PineGreen}{\selectlanguage{french}Vomir.}  ¶ \textcolor{darkblue}{\textbf{\ipa{le˧-pʰi˧-ze˥}}} \zh{呕吐了} \textcolor{Sepia}{\selectlanguage{english}\mytextsc{accomp} \string_ \mytextsc{pfv}} \textcolor{PineGreen}{\selectlanguage{french}\mytextsc{accomp} \string_ \mytextsc{pfv}}  

\lhead{\firstmark}
\rhead{\botmark}

\subsection{\hspace{-0.5cm} {\Large \textcolor{darkblue}{\textbf{\ipa{pʰo˥}}}}\hspace{0.5cm}[\kern2pt{\textcolor{darkblue}{\textbf{\ipa{pʰo˥}}}}\kern2pt]} \hypertarget{p\string_ho\string_T1}{}
\markboth{\textcolor{darkblue}{\textbf{\ipa{pʰo˥}}}}{}
\textcolor{teal}{\zh{动词}} \hspace{4pt} \zh{声调类:} H.
\zh{披上(不系扣子)。} \textcolor{Sepia}{\selectlanguage{english}To drape oneself in (a cape, a piece of fabric), without buttoning up.} \textcolor{PineGreen}{\selectlanguage{french}Se draper de, endosser, mettre sur son dos. Le fait de porter un vêtement sur les épaules sans le boutonner était considéré comme mal élevé: seuls les voleurs gardent la veste ouverte, pour y fourrer subrepticement leur butin.}  ¶ \textcolor{darkblue}{\textbf{\ipa{mɤ˧-pʰo˥}}} \zh{不披} \textcolor{Sepia}{\selectlanguage{english}\mytextsc{neg}} \textcolor{PineGreen}{\selectlanguage{french}\mytextsc{neg}}  
 ¶ \textcolor{darkblue}{\textbf{\ipa{bɑ˩lɑ˩ qɑ˩-pʰo˩˥}}} \zh{披上衣服(不系扣子)} \textcolor{Sepia}{\selectlanguage{english}to put on a shirt without buttoning up} \textcolor{PineGreen}{\selectlanguage{french}endosser un habit, se mettre un habit sur les épaules (sans boutonner)}  

\lhead{\firstmark}
\rhead{\botmark}

\subsection{\hspace{-0.5cm} {\Large \textcolor{darkblue}{\textbf{\ipa{pʰo˧\textsubscript{b}}}}}\hspace{0.5cm}[\kern2pt{\textcolor{darkblue}{\textbf{\ipa{pʰo˥}}}}\kern2pt]} \hypertarget{p\string_ho\string_Mb1}{}
\markboth{\textcolor{darkblue}{\textbf{\ipa{pʰo˧\textsubscript{b}}}}}{}
\textcolor{teal}{\zh{动词}} \hspace{4pt} \zh{声调类:} M\textsubscript{b}.
\zh{打开(例如:开门)。} \textcolor{Sepia}{\selectlanguage{english}To open (e.g. a door).} \textcolor{PineGreen}{\selectlanguage{french}Ouvrir (ex.: porte).}  ¶ \textcolor{darkblue}{\textbf{\ipa{gɤ˩-pʰo˧ (-ze˧)}}} \zh{打开} \textcolor{Sepia}{\selectlanguage{english}to open up} \textcolor{PineGreen}{\selectlanguage{french}ouvrir}  
 ¶ \textcolor{darkblue}{\textbf{\ipa{kʰi˧ pʰo˧}}} \zh{开门} \textcolor{Sepia}{\selectlanguage{english}to open the door} \textcolor{PineGreen}{\selectlanguage{french}ouvrir la porte}  
 ¶ \textcolor{darkblue}{\textbf{\ipa{kʰi˧mi˧ le˧-pʰo˧}}} \zh{开门} \textcolor{Sepia}{\selectlanguage{english}to open the door} \textcolor{PineGreen}{\selectlanguage{french}ouvrir la porte}  
 ¶ \textcolor{darkblue}{\textbf{\ipa{tso˧\textasciitilde{}tso˧ pʰo˧}}} \zh{打开东西} \textcolor{Sepia}{\selectlanguage{english}to open something} \textcolor{PineGreen}{\selectlanguage{french}ouvrir quelque chose}  

\lhead{\firstmark}
\rhead{\botmark}

\subsection{\hspace{-0.5cm} {\Large \textcolor{darkblue}{\textbf{\ipa{pʰo˩\textsubscript{a}}}}}\hspace{0.5cm}[\kern2pt{\textcolor{darkblue}{\textbf{\ipa{pʰo˩˥}}}}\kern2pt]} \hypertarget{p\string_ho\string_Ba1}{}
\markboth{\textcolor{darkblue}{\textbf{\ipa{pʰo˩\textsubscript{a}}}}}{}
\textcolor{teal}{\zh{动词}} \hspace{4pt} \zh{声调类:} L\textsubscript{a}.
\zh{逃,逃跑,逃掉。} \textcolor{Sepia}{\selectlanguage{english}To flee, to rush away, to escape.} \textcolor{PineGreen}{\selectlanguage{french}S'échapper, s'enfuir; détaler.}  ¶ \textcolor{darkblue}{\textbf{\ipa{le˧-pʰo˩-ze˩}}} \zh{逃跑了} \textcolor{Sepia}{\selectlanguage{english}\mytextsc{accomp} \string_ \mytextsc{pfv}} \textcolor{PineGreen}{\selectlanguage{french}\mytextsc{accomp} \string_ \mytextsc{pfv}}  
 ¶ \textcolor{darkblue}{\textbf{\ipa{le˧-pʰo˩-hɯ˩-ze˩!}}} \zh{(他)逃跑了!} \textcolor{Sepia}{\selectlanguage{english}(She/he) has escaped!} \textcolor{PineGreen}{\selectlanguage{french}(Elle/il) s'est enfui(e)!}  

\lhead{\firstmark}
\rhead{\botmark}

\subsection{\hspace{-0.5cm} {\Large \textcolor{darkblue}{\textbf{\ipa{pʰo˩lɑ˧˥}}}}\hspace{0.5cm}[\kern2pt{\textcolor{darkblue}{\textbf{\ipa{pʰo˧lɑ˩}}}}\kern2pt]} \hypertarget{p\string_ho\string_BlA\string_M\string_T1}{}
\markboth{\textcolor{darkblue}{\textbf{\ipa{pʰo˩lɑ˧˥}}}}{}
\textcolor{teal}{\zh{动词}} \hspace{4pt} \zh{声调类:} LM+MH\#.
\zh{战争、打仗。} \textcolor{Sepia}{\selectlanguage{english}To wage war.} \textcolor{PineGreen}{\selectlanguage{french}Faire la guerre.}  ¶ \textcolor{darkblue}{\textbf{\ipa{mɤ˧-pʰo˩lɑ˩}}} \zh{不打仗} \textcolor{Sepia}{\selectlanguage{english}\mytextsc{neg}} \textcolor{PineGreen}{\selectlanguage{french}\mytextsc{neg}}  
 ¶ \textcolor{darkblue}{\textbf{\ipa{pʰo˩lɑ˧˥ | ɖɯ˧-kʰv̩˧˥}}} \zh{打仗的一年} \textcolor{Sepia}{\selectlanguage{english}a year of war, a year during which there was war} \textcolor{PineGreen}{\selectlanguage{french}une année de guerre}  

\lhead{\firstmark}
\rhead{\botmark}

\subsection{\hspace{-0.5cm} {\Large \textcolor{darkblue}{\textbf{\ipa{pʰo˧˥}}}}\hspace{0.5cm}[\kern2pt{\textcolor{darkblue}{\textbf{\ipa{pʰo˧˥}}}}\kern2pt]} \hypertarget{p\string_ho\string_M\string_T1}{}
\markboth{\textcolor{darkblue}{\textbf{\ipa{pʰo˧˥}}}}{}
\textcolor{teal}{\zh{动词}} \hspace{4pt} \zh{声调类:} MH.
\zh{撒 (撒种子)、播(种子)。} \textcolor{Sepia}{\selectlanguage{english}To sow.} \textcolor{PineGreen}{\selectlanguage{french}Semer à la volée.}  ¶ \textcolor{darkblue}{\textbf{\ipa{ɻæ˩ pʰo˧˥}}} \zh{撒种子} \textcolor{Sepia}{\selectlanguage{english}to sow seeds} \textcolor{PineGreen}{\selectlanguage{french}semer des graines à la volée}  

\lhead{\firstmark}
\rhead{\botmark}

\subsection{\hspace{-0.5cm} {\Large \textcolor{darkblue}{\textbf{\ipa{pʰo˧˥\textsubscript{a}}}}}\hspace{0.5cm}[\kern2pt{\textcolor{darkblue}{\textbf{\ipa{pʰo˧˥}}}}\kern2pt]} \hypertarget{p\string_ho\string_M\string_Ta1}{}
\markboth{\textcolor{darkblue}{\textbf{\ipa{pʰo˧˥\textsubscript{a}}}}}{}
\textcolor{teal}{\zh{量词}} \hspace{4pt} \zh{声调类:} MH\textsubscript{a}.
\zh{量词:一对中的一只(例如一只鞋),一头大牲畜(牛……)。} \textcolor{Sepia}{\selectlanguage{english}A member of a pair; also used for some large domestic animals, e.g. oxen.} \textcolor{PineGreen}{\selectlanguage{french}Classificateur des membres d'une paire. Par exemple: une chaussure, pas une paire. Ce classificateur est également employé pour le gros bétail: vaches, buffles….} 
\lhead{\firstmark}
\rhead{\botmark}

\subsection{\hspace{-0.5cm} {\Large \textcolor{darkblue}{\textbf{\ipa{pʰv̩˧}}} \textsubscript{1}}\hspace{0.5cm}[\kern2pt{\textcolor{darkblue}{\textbf{\ipa{pʰv̩˥}}}}\kern2pt]} \hypertarget{p\string_hv\string_=\string_M1}{}
\markboth{\textcolor{darkblue}{\textbf{\ipa{pʰv̩˧}}} \textsubscript{1}}{}
\textcolor{teal}{\zh{名词}} \hspace{4pt} \zh{声调类:} M.
\zh{公的。} \textcolor{Sepia}{\selectlanguage{english}Male.} \textcolor{PineGreen}{\selectlanguage{french}Mâle.}  ¶ \textcolor{darkblue}{\textbf{\ipa{ʈʂʰɯ˧, | pʰv̩˧ ɲi˩!}}} \zh{这(只动物)是公的!} \textcolor{Sepia}{\selectlanguage{english}It's a male!} \textcolor{PineGreen}{\selectlanguage{french}C'est un mâle!}  
 ¶ \textcolor{darkblue}{\textbf{\ipa{ʈʂʰɯ˧, | pʰv̩˧!}}} \zh{这(只动物)是公的!} \textcolor{Sepia}{\selectlanguage{english}It's a male!} \textcolor{PineGreen}{\selectlanguage{french}C'est un mâle!}  
 \zh{量词}: \textcolor{darkblue}{\textbf{\ipa{v̩˧}}} 
\lhead{\firstmark}
\rhead{\botmark}

\subsection{\hspace{-0.5cm} {\Large \textcolor{darkblue}{\textbf{\ipa{pʰv̩˧}}} \textsubscript{2}}\hspace{0.5cm}[\kern2pt{\textcolor{darkblue}{\textbf{\ipa{pʰv̩˥}}}}\kern2pt]} \hypertarget{p\string_hv\string_=\string_M2}{}
\markboth{\textcolor{darkblue}{\textbf{\ipa{pʰv̩˧}}} \textsubscript{2}}{}
\textcolor{teal}{\zh{名词}} \hspace{4pt} \zh{声调类:} M.
\zh{价格。} \textcolor{Sepia}{\selectlanguage{english}Price.} \textcolor{PineGreen}{\selectlanguage{french}Prix.} 
\lhead{\firstmark}
\rhead{\botmark}

\subsection{\hspace{-0.5cm} {\Large \textcolor{darkblue}{\textbf{\ipa{pʰv̩˧˥}}}}\hspace{0.5cm}[\kern2pt{\textcolor{darkblue}{\textbf{\ipa{pʰv̩˧˥}}}}\kern2pt]} \hypertarget{p\string_hv\string_=\string_M\string_T1}{}
\markboth{\textcolor{darkblue}{\textbf{\ipa{pʰv̩˧˥}}}}{}
\textcolor{teal}{\zh{动词}} \hspace{4pt} \zh{声调类:} MH.
\zh{脱(衣服)。} \textcolor{Sepia}{\selectlanguage{english}To take off (clothes).} \textcolor{PineGreen}{\selectlanguage{french}Ôter, retirer (un vêtement).} 
\lhead{\firstmark}
\rhead{\botmark}

\subsection{\hspace{-0.5cm} {\Large \textcolor{darkblue}{\textbf{\ipa{pʰv̩˧˥}}} \textsubscript{1}}\hspace{0.5cm}[\kern2pt{\textcolor{darkblue}{\textbf{\ipa{pʰv̩˧˥}}}}\kern2pt]} \hypertarget{p\string_hv\string_=\string_M\string_T1}{}
\markboth{\textcolor{darkblue}{\textbf{\ipa{pʰv̩˧˥}}} \textsubscript{1}}{}
\textcolor{teal}{\zh{动词}} \hspace{4pt} \zh{声调类:} MH.
\zh{煮(鸡蛋、洋芋……)。} \textcolor{Sepia}{\selectlanguage{english}To boil, to cook in water.} \textcolor{PineGreen}{\selectlanguage{french}Faire bouillir, faire cuire à l'eau (œuf, patates…).}  ¶ \textcolor{darkblue}{\textbf{\ipa{jɤ˩jo˥ F | pʰv̩˧˥! | æ˩ʁv̩˩˥ F | pʰv̩˧˥!}}} \zh{洋芋,是(可以)煮的!鸡蛋,是(可以)煮的!} \textcolor{Sepia}{\selectlanguage{english}Potatoes can be boiled! Eggs can be boiled!} \textcolor{PineGreen}{\selectlanguage{french}Les pommes de terre, ça se cuit à l'eau! Les œufs, ça se cuit à l'eau!}  
 ¶ \textcolor{darkblue}{\textbf{\ipa{æ˩ʁv̩˩ pʰv̩˥}}} \zh{煮鸡蛋} \textcolor{Sepia}{\selectlanguage{english}to cook eggs in water} \textcolor{PineGreen}{\selectlanguage{french}cuire des œufs à l'eau, faire des oeufs durs}  
 ¶ \textcolor{darkblue}{\textbf{\ipa{jɤ˩jo˥ pʰv̩˩}}} \zh{煮洋芋} \textcolor{Sepia}{\selectlanguage{english}to boil potatoes} \textcolor{PineGreen}{\selectlanguage{french}cuire des pommes de terre à l'eau}  
 ¶ \textcolor{darkblue}{\textbf{\ipa{le˧-pʰv̩˧ | le˧-mv̩˩-ze˩!}}} \zh{煮熟了!} \textcolor{Sepia}{\selectlanguage{english}It is cooked (from boiling)! / It has been boiled to the point when it is now well-done/cooked} \textcolor{PineGreen}{\selectlanguage{french}C'est cuit (à l'eau)! Résultatif: ça a été suffisamment bouilli pour que ce soit maintenant cuit}  

\lhead{\firstmark}
\rhead{\botmark}

\subsection{\hspace{-0.5cm} {\Large \textcolor{darkblue}{\textbf{\ipa{pʰv̩˧˥}}} \textsubscript{2}}\hspace{0.5cm}[\kern2pt{\textcolor{darkblue}{\textbf{\ipa{pʰv̩˧˥}}}}\kern2pt]} \hypertarget{p\string_hv\string_=\string_M\string_T2}{}
\markboth{\textcolor{darkblue}{\textbf{\ipa{pʰv̩˧˥}}} \textsubscript{2}}{}
\textcolor{teal}{\zh{动词}} \hspace{4pt} \zh{声调类:} MH.
\zh{倒(酒……),倒出来。} \textcolor{Sepia}{\selectlanguage{english}To pour, to spill.} \textcolor{PineGreen}{\selectlanguage{french}Verser; renverser; répandre; jeter.}  ¶ \textcolor{darkblue}{\textbf{\ipa{ʐɯ˧ pʰv̩˧˥}}} \zh{倒酒} \textcolor{Sepia}{\selectlanguage{english}to pour wine, to serve wine} \textcolor{PineGreen}{\selectlanguage{french}verser du vin, servir du vin}  
 ¶ \textcolor{darkblue}{\textbf{\ipa{dʑɯ˩ pʰv̩˩˥}}} \zh{倒水} \textcolor{Sepia}{\selectlanguage{english}to pour water, to serve water (as a beverage)} \textcolor{PineGreen}{\selectlanguage{french}verser de l'eau}  
 ¶ \textcolor{darkblue}{\textbf{\ipa{mv̩˩tɕo˧ pʰv̩˧˥}}} \zh{往外倒} \textcolor{Sepia}{\selectlanguage{english}to pour out, to spill on the floor} \textcolor{PineGreen}{\selectlanguage{french}renverser, verser à terre, jeter à terre}  
 ¶ \textcolor{darkblue}{\textbf{\ipa{[F5] ɖæ˩˥ | mv̩˩tɕo˧ pʰv̩˥}}} \zh{倒垃圾} \textcolor{Sepia}{\selectlanguage{english}to throw out garbage, to pour garbage (out of a bucket onto a dirt heap)} \textcolor{PineGreen}{\selectlanguage{french}jeter des ordures}  

\lhead{\firstmark}
\rhead{\botmark}

\subsection{\hspace{-0.5cm} {\Large \textcolor{darkblue}{\textbf{\ipa{pʰv̩˧˥}}} \textsubscript{3}}\hspace{0.5cm}[\kern2pt{\textcolor{darkblue}{\textbf{\ipa{pʰv̩˧˥}}}}\kern2pt]} \hypertarget{p\string_hv\string_=\string_M\string_T3}{}
\markboth{\textcolor{darkblue}{\textbf{\ipa{pʰv̩˧˥}}} \textsubscript{3}}{}
\textcolor{teal}{\zh{动词}} \hspace{4pt} \zh{声调类:} MH.
\zh{翻身、翻来翻去。} \textcolor{Sepia}{\selectlanguage{english}To turn over (when lying down).} \textcolor{PineGreen}{\selectlanguage{french}Retourner; se retourner (quelqu'un est allongé et se retourne).}  ¶ \textcolor{darkblue}{\textbf{\ipa{le˧-wo˧ tsɤ˥-pʰv̩˩ |}}} \zh{翻身} \textcolor{Sepia}{\selectlanguage{english}to turn over (when lying down)} \textcolor{PineGreen}{\selectlanguage{french}se retourner}  
 ¶ \textcolor{darkblue}{\textbf{\ipa{ɖɯ˧-tɕo˥ tsɤ˩-pʰv̩˩, | ʈʂʰɯ˧-tɕo˥ tsɤ˩-pʰv̩˩}}} \zh{翻来翻去} \textcolor{Sepia}{\selectlanguage{english}to turn over this way and that (when lying down: turning over restlessly)} \textcolor{PineGreen}{\selectlanguage{french}se retourner par-ci, se retourner par-là}  

\lhead{\firstmark}
\rhead{\botmark}

\subsection{\hspace{-0.5cm} {\Large \textcolor{darkblue}{\textbf{\ipa{pʰv̩˩\textsubscript{a}}}}}\hspace{0.5cm}[\kern2pt{\textcolor{darkblue}{\textbf{\ipa{pʰv̩˩˥}}}}\kern2pt]} \hypertarget{p\string_hv\string_=\string_Ba1}{}
\markboth{\textcolor{darkblue}{\textbf{\ipa{pʰv̩˩\textsubscript{a}}}}}{}
\textcolor{teal}{\zh{形容词}} \hspace{4pt} \zh{声调类:} L\textsubscript{a}.
\zh{白色(脸、衣服)。} \textcolor{Sepia}{\selectlanguage{english}White.} \textcolor{PineGreen}{\selectlanguage{french}Blanc (visage, habits, cheveux...).}  ¶ \textcolor{darkblue}{\textbf{\ipa{pʰv̩˩-hĩ˩˥}}} \zh{白的} \textcolor{Sepia}{\selectlanguage{english}\mytextsc{rel}} \textcolor{PineGreen}{\selectlanguage{french}\mytextsc{rel}}  

\lhead{\firstmark}
\rhead{\botmark}

\subsection{\hspace{-0.5cm} {\Large \textcolor{darkblue}{\textbf{\ipa{pʰv̩˩\textsubscript{a}}}}}\hspace{0.5cm}[\kern2pt{\textcolor{darkblue}{\textbf{\ipa{pʰv̩˩˥}}}}\kern2pt]} \hypertarget{p\string_hv\string_=\string_Ba1}{}
\markboth{\textcolor{darkblue}{\textbf{\ipa{pʰv̩˩\textsubscript{a}}}}}{}
\textcolor{teal}{\zh{量词}} \hspace{4pt} \zh{声调类:} L\textsubscript{b}.
\zh{量词:田地(一块)。} \textcolor{Sepia}{\selectlanguage{english}Classifier for fields.} \textcolor{PineGreen}{\selectlanguage{french}Classificateur des parcelles de terre, des champs.}  ¶ \textcolor{darkblue}{\textbf{\ipa{lv̩˧ | ɖɯ˧-pʰv̩˩}}} \zh{一块田} \textcolor{Sepia}{\selectlanguage{english}a field} \textcolor{PineGreen}{\selectlanguage{french}un champ; une parcelle}  

\lhead{\firstmark}
\rhead{\botmark}

\subsection{\hspace{-0.5cm} {\Large \textcolor{darkblue}{\textbf{\ipa{pʰv̩˩\textsubscript{b}}}} \textsubscript{1}}\hspace{0.5cm}[\kern2pt{\textcolor{darkblue}{\textbf{\ipa{pʰv̩˩˥}}}}\kern2pt]} \hypertarget{p\string_hv\string_=\string_Bb1}{}
\markboth{\textcolor{darkblue}{\textbf{\ipa{pʰv̩˩\textsubscript{b}}}} \textsubscript{1}}{}
\textcolor{teal}{\zh{动词}} \hspace{4pt} \zh{声调类:} L\textsubscript{b}.
\zh{摇动、翻滚。} \textcolor{Sepia}{\selectlanguage{english}To move around.} \textcolor{PineGreen}{\selectlanguage{french}S'agiter.}  ¶ \textcolor{darkblue}{\textbf{\ipa{bo˩˥ | tʰi˧-pʰv̩˩-dʑo˩}}} \zh{猪在翻滚} \textcolor{Sepia}{\selectlanguage{english}The pig is moving around (restlessly).} \textcolor{PineGreen}{\selectlanguage{french}le cochon s'agite dans son box}  
 ¶ \textcolor{darkblue}{\textbf{\ipa{bo˩-ɳɯ˧ | pʰv̩˧\textasciitilde{}pʰv̩˩}}} \zh{猪在翻滚} \textcolor{Sepia}{\selectlanguage{english}same meaning as above} \textcolor{PineGreen}{\selectlanguage{french}même sens}  

\lhead{\firstmark}
\rhead{\botmark}

\subsection{\hspace{-0.5cm} {\Large \textcolor{darkblue}{\textbf{\ipa{pʰv̩˩\textsubscript{b}}}} \textsubscript{2}}\hspace{0.5cm}[\kern2pt{\textcolor{darkblue}{\textbf{\ipa{pʰv̩˩˥}}}}\kern2pt]} \hypertarget{p\string_hv\string_=\string_Bb2}{}
\markboth{\textcolor{darkblue}{\textbf{\ipa{pʰv̩˩\textsubscript{b}}}} \textsubscript{2}}{}
\textcolor{teal}{\zh{动词}} \hspace{4pt} \zh{声调类:} L\textsubscript{b}.
\zh{扩散、发展。} \textcolor{Sepia}{\selectlanguage{english}To expand, to spread, to extend.} \textcolor{PineGreen}{\selectlanguage{french}Connaître une expansion, se répandre, s'étendre.}  ¶ \textcolor{darkblue}{\textbf{\ipa{zo˧mv̩˥ | tʰi˧-pʰv̩˩}}} \zh{孩子们扩散(到新的地方)} \textcolor{Sepia}{\selectlanguage{english}the children spread into new territory; the family spreads, expands into new areas} \textcolor{PineGreen}{\selectlanguage{french}les enfants se répandent, la famille essaime}  

\lhead{\firstmark}
\rhead{\botmark}

\subsection{\hspace{-0.5cm} {\Large \textcolor{darkblue}{\textbf{\ipa{pʰv̩˧ɖɯ˧˥}}}}\hspace{0.5cm}[\kern2pt{\textcolor{darkblue}{\textbf{\ipa{pʰv̩˧ɖɯ˧˥}}}}\kern2pt]} \hypertarget{p\string_hv\string_=\string_Md`M\string_M\string_T1}{}
\markboth{\textcolor{darkblue}{\textbf{\ipa{pʰv̩˧ɖɯ˧˥}}}}{}
\textcolor{teal}{\zh{形容词}} \hspace{4pt} \zh{声调类:} MH\#.
\zh{贵。} \textcolor{Sepia}{\selectlanguage{english}Expensive.} \textcolor{PineGreen}{\selectlanguage{french}Coûteux, cher.}  ¶ \textcolor{darkblue}{\textbf{\ipa{pʰv̩˧ɖɯ˧ ʝi˥}}} \zh{关心} \textcolor{Sepia}{\selectlanguage{english}to care for, to give great attention to} \textcolor{PineGreen}{\selectlanguage{french}prêter attention à}  

\lhead{\firstmark}
\rhead{\botmark}

\subsection{\hspace{-0.5cm} {\Large \textcolor{darkblue}{\textbf{\ipa{pʰv̩˧dʑo˧-hĩ\#˥}}}}\hspace{0.5cm}[\kern2pt{\textcolor{darkblue}{\textbf{\ipa{xxxx non-correspondance entre le nombre de morphèmes et le nombre de tons de morphèmes}}}}\kern2pt]} \hypertarget{p\string_hv\string_=\string_Mdz£o\string_M-hi\string_~\#\string_T1}{}
\markboth{\textcolor{darkblue}{\textbf{\ipa{pʰv̩˧dʑo˧-hĩ\#˥}}}}{}
\textcolor{teal}{\zh{名词}} \hspace{4pt} \zh{声调类:} \#H.
\zh{拉伯的人。} \textcolor{Sepia}{\selectlanguage{english}Inhabitant of Labai, person from Labai.} \textcolor{PineGreen}{\selectlanguage{french}Personne de Labai, gens de Labai.} 
\lhead{\firstmark}
\rhead{\botmark}

\subsection{\hspace{-0.5cm} {\Large \textcolor{darkblue}{\textbf{\ipa{pʰv̩˧dʑo\#˥}}}}\hspace{0.5cm}[\kern2pt{\textcolor{darkblue}{\textbf{\ipa{pʰv̩˧dʑo˧}}}}\kern2pt]} \hypertarget{p\string_hv\string_=\string_Mdz£o\#\string_T1}{}
\markboth{\textcolor{darkblue}{\textbf{\ipa{pʰv̩˧dʑo\#˥}}}}{}
\textcolor{teal}{\zh{名词}} \hspace{4pt} \zh{声调类:} \#H.
\zh{拉伯。} \textcolor{Sepia}{\selectlanguage{english}The village of Labai.} \textcolor{PineGreen}{\selectlanguage{french}Village de Labai.}  ¶ \textcolor{darkblue}{\textbf{\ipa{pʰv̩˧dʑo˧ dzi˧˥}}} \zh{在拉柏住} \textcolor{Sepia}{\selectlanguage{english}to live in Labai, to dwell in Labai} \textcolor{PineGreen}{\selectlanguage{french}habiter à Labai}  

\lhead{\firstmark}
\rhead{\botmark}

\subsection{\hspace{-0.5cm} {\Large \textcolor{darkblue}{\textbf{\ipa{pʰv̩˧kɤ˧}}}}\hspace{0.5cm}[\kern2pt{\textcolor{darkblue}{\textbf{\ipa{pʰv̩˧kɤ˧}}}}\kern2pt]} \hypertarget{p\string_hv\string_=\string_Mk7\string_M1}{}
\markboth{\textcolor{darkblue}{\textbf{\ipa{pʰv̩˧kɤ˧}}}}{}
\textcolor{teal}{\zh{名词}} \hspace{4pt} \zh{声调类:} M.
\zh{被子。} \textcolor{Sepia}{\selectlanguage{english}Blanket.} \textcolor{PineGreen}{\selectlanguage{french}Couverture.}  \zh{量词}: \textcolor{darkblue}{\textbf{\ipa{ɭɯ˧}}} 
\lhead{\firstmark}
\rhead{\botmark}

\subsection{\hspace{-0.5cm} {\Large \textcolor{darkblue}{\textbf{\ipa{pʰv̩˧ɭɯ˧-ʈʰæ˧qʰwɤ˥}}}}\hspace{0.5cm}[\kern2pt{\textcolor{darkblue}{\textbf{\ipa{xxxx non-correspondance entre le nombre de morphèmes et le nombre de tons de morphèmes}}}}\kern2pt]} \hypertarget{p\string_hv\string_=\string_Ml\string_RM\string_M-t`\string_h\{\string_Mq\string_hw7\string_T1}{}
\markboth{\textcolor{darkblue}{\textbf{\ipa{pʰv̩˧ɭɯ˧-ʈʰæ˧qʰwɤ˥}}}}{}
\textcolor{teal}{\zh{名词}} \hspace{4pt} \zh{声调类:} H\#.
\zh{羊毛裙子。} \textcolor{Sepia}{\selectlanguage{english}Woolen skirt. (Not in common use in Yongning.).} \textcolor{PineGreen}{\selectlanguage{french}Jupe de laine. (Ce n'est pas un vêtement courant à Yongning.).}  \zh{量词}: \textcolor{darkblue}{\textbf{\ipa{ɭɯ˧}}} 
\lhead{\firstmark}
\rhead{\botmark}

\subsection{\hspace{-0.5cm} {\Large \textcolor{darkblue}{\textbf{\ipa{pʰv̩˧ɭɯ\#˥}}}}\hspace{0.5cm}[\kern2pt{\textcolor{darkblue}{\textbf{\ipa{pʰv̩˧ɭɯ˧}}}}\kern2pt]} \hypertarget{p\string_hv\string_=\string_Ml\string_RM\#\string_T1}{}
\markboth{\textcolor{darkblue}{\textbf{\ipa{pʰv̩˧ɭɯ\#˥}}}}{}
\textcolor{teal}{\zh{名词}} \hspace{4pt} \zh{声调类:} \#H.
\zh{氆氇。} \textcolor{Sepia}{\selectlanguage{english}Tibetan wool fabric.} \textcolor{PineGreen}{\selectlanguage{french}Tissu de laine tibétain.}  \zh{量词}: \textcolor{darkblue}{\textbf{\ipa{ɭɯ˧}}} 
\lhead{\firstmark}
\rhead{\botmark}

\subsection{\hspace{-0.5cm} {\Large \textcolor{darkblue}{\textbf{\ipa{pʰv̩˧ʂɯ˧}}}}\hspace{0.5cm}[\kern2pt{\textcolor{darkblue}{\textbf{\ipa{pʰv̩˧ʂɯ˧}}}}\kern2pt]} \hypertarget{p\string_hv\string_=\string_Ms`M\string_M1}{}
\markboth{\textcolor{darkblue}{\textbf{\ipa{pʰv̩˧ʂɯ˧}}}}{}
\textcolor{teal}{\zh{名词}} \hspace{4pt} \zh{声调类:} M.
\zh{美容膏。也来指防晒霜。} \textcolor{Sepia}{\selectlanguage{english}Beauty cream; now used for sun cream.} \textcolor{PineGreen}{\selectlanguage{french}Crème de beauté; s'emploie aussi pour la crème solaire.}  ¶ \textcolor{darkblue}{\textbf{\ipa{pʰv˧ʂɯ˧ jɤ˧˥}}} \zh{抹美容膏,抹防晒霜} \textcolor{Sepia}{\selectlanguage{english}to put on beauty cream; to apply sunscreen} \textcolor{PineGreen}{\selectlanguage{french}étaler de la crème solaire, mettre de la crème solaire}  
 ¶ \textcolor{darkblue}{\textbf{\ipa{pʰv˧ʂɯ˧ lɑ˧˥}}} \zh{抹美容膏,抹防晒霜} \textcolor{Sepia}{\selectlanguage{english}to put on beauty cream; to apply sunscreen} \textcolor{PineGreen}{\selectlanguage{french}étaler de la crème solaire, mettre de la crème solaire}  

\lhead{\firstmark}
\rhead{\botmark}

\subsection{\hspace{-0.5cm} {\Large \textcolor{darkblue}{\textbf{\ipa{pʰv̩˩-tɕæ˩ɻæ˥}}}}\hspace{0.5cm}[\kern2pt{\textcolor{darkblue}{\textbf{\ipa{xxxx non-correspondance entre le nombre de morphèmes et le nombre de tons de morphèmes}}}}\kern2pt]} \hypertarget{p\string_hv\string_=\string_B-ts£\{\string_Br£`\{\string_T1}{}
\markboth{\textcolor{darkblue}{\textbf{\ipa{pʰv̩˩-tɕæ˩ɻæ˥}}}}{}
\textcolor{teal}{\zh{形容词}} \hspace{4pt} \zh{声调类:} L+H\#.
\zh{很白(脸、衣服、头发)。} \textcolor{Sepia}{\selectlanguage{english}Very white.} \textcolor{PineGreen}{\selectlanguage{french}Blanc (visage, habits, cheveux...).}  ¶ \textcolor{darkblue}{\textbf{\ipa{pʰv̩˩tɕæ˩ɻæ˥-gv̩˩}}} \zh{很白} \textcolor{Sepia}{\selectlanguage{english}very white} \textcolor{PineGreen}{\selectlanguage{french}tout blanc}  
 ¶ \textcolor{darkblue}{\textbf{\ipa{pʰv̩˩↑tɕæ˥ɻæ˩-gv̩˩}}} \zh{很白} \textcolor{Sepia}{\selectlanguage{english}very white} \textcolor{PineGreen}{\selectlanguage{french}tout blanc}  
 ¶ \textcolor{darkblue}{\textbf{\ipa{pʰæ˧qʰwɤ˩ | pʰv̩˩tɕæ˩ɻæ˥-gv̩˩}}} \zh{脸很白} \textcolor{Sepia}{\selectlanguage{english}the face is very white} \textcolor{PineGreen}{\selectlanguage{french}le visage est très blanc}  

\lhead{\firstmark}
\rhead{\botmark}

\subsection{\hspace{-0.5cm} {\Large \textcolor{darkblue}{\textbf{\ipa{pʰv̩˥ʈʂʰe˩}}}}\hspace{0.5cm}[\kern2pt{\textcolor{darkblue}{\textbf{\ipa{xxxx ton non trouvé, à faire manuellement...}}}}\kern2pt]} \hypertarget{p\string_hv\string_=\string_Tt`s`\string_he\string_B1}{}
\markboth{\textcolor{darkblue}{\textbf{\ipa{pʰv̩˥ʈʂʰe˩}}}}{}
\textcolor{teal}{\zh{动词}} \hspace{4pt} \zh{声调类:} HL.
\zh{分开、区分、区别开来。} \textcolor{Sepia}{\selectlanguage{english}To distinguish.} \textcolor{PineGreen}{\selectlanguage{french}Distinguer, voir la différence (par exemple entre diverses espèces de champignons).}  ¶ \textcolor{darkblue}{\textbf{\ipa{le˧-pʰv̩˥ʈʂʰe˩}}} \zh{分开、区分、区别开来} \textcolor{Sepia}{\selectlanguage{english}to distinguish, to tell apart (e.g. species of mushrooms)} \textcolor{PineGreen}{\selectlanguage{french}distinguer, voir la différence (par exemple entre diverses espèces de champignons)}  
 ¶ \textcolor{darkblue}{\textbf{\ipa{ɖɯ˧-pʰv̩˥ʈʂʰe˩=ɻ̍˩}}} \zh{试着区分} \textcolor{Sepia}{\selectlanguage{english}\mytextsc{delimitative} \string_ \mytextsc{inceptive}} \textcolor{PineGreen}{\selectlanguage{french}\mytextsc{délimitatif} \string_ \mytextsc{inchoatif}}  
 ¶ \textcolor{darkblue}{\textbf{\ipa{mɤ˧-pʰv̩˥ʈʂʰe˩}}} \zh{不分开,不分,不区分} \textcolor{Sepia}{\selectlanguage{english}not to distinguish, not to make any difference (e.g. between different species of mushrooms)} \textcolor{PineGreen}{\selectlanguage{french}ne pas distinguer, ne pas faire de différence, ne pas voir la différence (par exemple entre diverses espèces de champignons)}  

\lhead{\firstmark}
\rhead{\botmark}

\subsection{\hspace{-0.5cm} {\Large \textcolor{darkblue}{\textbf{\ipa{pʰv̩˧tv̩˥}}}}\hspace{0.5cm}[\kern2pt{\textcolor{darkblue}{\textbf{\ipa{pʰv̩˧tv̩˥}}}}\kern2pt]} \hypertarget{p\string_hv\string_=\string_Mtv\string_=\string_T1}{}
\markboth{\textcolor{darkblue}{\textbf{\ipa{pʰv̩˧tv̩˥}}}}{}
\textcolor{teal}{\zh{名词}} \hspace{4pt} \zh{声调类:} H\#.
\zh{公水牛。} \textcolor{Sepia}{\selectlanguage{english}Male water buffalo.} \textcolor{PineGreen}{\selectlanguage{french}Buffle mâle.}  ¶ \textcolor{darkblue}{\textbf{\ipa{dʑi˧mi˧-pʰv̩˩tv̩˩}}} \zh{同上:公水牛} \textcolor{Sepia}{\selectlanguage{english}same meaning: male water buffalo} \textcolor{PineGreen}{\selectlanguage{french}même sens: buffle mâle}  
 ¶ \textcolor{darkblue}{\textbf{\ipa{dʑi˧mi˧ ʈʂʰɯ˧-pʰo˩ dʑo˩, | pʰv̩˧tv̩˥ ɲi˩!}}} \zh{这头水牛是公的/是公水牛!} \textcolor{Sepia}{\selectlanguage{english}This buffalo is a male!} \textcolor{PineGreen}{\selectlanguage{french}ce buffle, c'est un mâle!}  
 \zh{量词}: \textcolor{darkblue}{\textbf{\ipa{pʰo˧˥}}} 
\lhead{\firstmark}
\rhead{\botmark}

\subsection{\hspace{-0.5cm} {\Large \textcolor{darkblue}{\textbf{\ipa{pʰv̩˧ʐo˧˥}}}}\hspace{0.5cm}[\kern2pt{\textcolor{darkblue}{\textbf{\ipa{pʰv̩˧ʐo˧˥}}}}\kern2pt]} \hypertarget{p\string_hv\string_=\string_Mz`o\string_M\string_T1}{}
\markboth{\textcolor{darkblue}{\textbf{\ipa{pʰv̩˧ʐo˧˥}}}}{}
\textcolor{teal}{\zh{形容词}} \hspace{4pt} \zh{声调类:} MH\#.
\zh{便宜。} \textcolor{Sepia}{\selectlanguage{english}Cheap.} \textcolor{PineGreen}{\selectlanguage{french}Bon marché.} \zh{~【参考】~} \textcolor{darkblue}{\textbf{\ipa{pʰv̩˧2; ʐo˩a2}}} 
\lhead{\firstmark}
\rhead{\botmark}

\newpage
\section*{\centering- \textcolor{darkblue}{\textbf{\ipa{q}}} -}
\subsection{\hspace{-0.5cm} {\Large \textcolor{darkblue}{\textbf{\ipa{qɑ˩\textsubscript{c}}}}}\hspace{0.5cm}[\kern2pt{\textcolor{darkblue}{\textbf{\ipa{qɑ˩˥}}}}\kern2pt]} \hypertarget{qA\string_Bc1}{}
\markboth{\textcolor{darkblue}{\textbf{\ipa{qɑ˩\textsubscript{c}}}}}{}
\textcolor{teal}{\zh{量词}} \hspace{4pt} \zh{声调类:} L\textsubscript{c}.
\ding{202} \zh{量词:抱。} \textcolor{Sepia}{\selectlanguage{english}Classifier for armfuls: of firewood, objects...} \textcolor{PineGreen}{\selectlanguage{french}Classificateur : une brassée (de bois coupé pour le feu, d'objets...).}  ¶ \textcolor{darkblue}{\textbf{\ipa{ʈʂʰɯ˧-qɑ˥}}} \zh{这一抱} \textcolor{Sepia}{\selectlanguage{english}this armful} \textcolor{PineGreen}{\selectlanguage{french}cette brassée}  
\ding{203} \zh{量词:粮食垛、干草垛。} \textcolor{Sepia}{\selectlanguage{english}A large bundle of cut cereals, made of about 10 sheaves. Each sheaf is tied using one stalk, then sheaves are tied together using string. A mule can carry 4 large bundles. Also for: an armful.} \textcolor{PineGreen}{\selectlanguage{french}Classificateur des bottes de céréales coupées, faites d'une dizaine de gerbes. Chaque gerbe est nouée à l'aide d'une tige, puis les gerbes sont liées ensemble avec de la ficelle. Une mule peut porter 4 bottes.}  ¶ \textcolor{darkblue}{\textbf{\ipa{dze˧ɭɯ˧ ɖɯ˧-qɑ˩}}} \zh{一垛小麦(收割时,将十束绑在一起成一垛)} \textcolor{Sepia}{\selectlanguage{english}a bundle of corn (cut cereals)} \textcolor{PineGreen}{\selectlanguage{french}une botte de blé (lors de la récolte)}  

\lhead{\firstmark}
\rhead{\botmark}

\subsection{\hspace{-0.5cm} {\Large \textcolor{darkblue}{\textbf{\ipa{qɑ˩\textsubscript{a}}}}}\hspace{0.5cm}[\kern2pt{\textcolor{darkblue}{\textbf{\ipa{qɑ˩˥}}}}\kern2pt]} \hypertarget{qA\string_Ba1}{}
\markboth{\textcolor{darkblue}{\textbf{\ipa{qɑ˩\textsubscript{a}}}}}{}
\textcolor{teal}{\zh{动词}} \hspace{4pt} \zh{声调类:} L\textsubscript{a}.
\ding{202} \zh{盖、覆盖。} \textcolor{Sepia}{\selectlanguage{english}To cover (e.g. cover a pot with a lid).} \textcolor{PineGreen}{\selectlanguage{french}Couvrir: par exemple mettre un couvercle, ou couvrir un plat d'une coupelle pour éviter que les mouches ne s'y posent.}  ¶ \textcolor{darkblue}{\textbf{\ipa{le˧-qɑ˩-ze˩}}} \zh{盖了} \textcolor{Sepia}{\selectlanguage{english}\mytextsc{accomp} \string_ \mytextsc{pfv}} \textcolor{PineGreen}{\selectlanguage{french}\mytextsc{accomp} \string_ \mytextsc{pfv}}  
 ¶ \textcolor{darkblue}{\textbf{\ipa{tʰi˧-qɑ˩-ze˩}}} \zh{\mytextsc{dur} \string_ \mytextsc{pfv}} \textcolor{Sepia}{\selectlanguage{english}\mytextsc{dur} \string_ \mytextsc{pfv}} \textcolor{PineGreen}{\selectlanguage{french}\mytextsc{dur} \string_ \mytextsc{pfv}}  
 ¶ \textcolor{darkblue}{\textbf{\ipa{ɖɯ˧-kʰwɤ˥ | tʰi˧-qɑ˥}}} \zh{用一块(布料)来盖(电视机,为了防灰)} \textcolor{Sepia}{\selectlanguage{english}to cover (a television set) with a piece of fabric (to protect it from dust)} \textcolor{PineGreen}{\selectlanguage{french}couvrir (un téléviseur) d'un morceau (de tissu) (pour le préserver de la poussière)}  
 ¶ \textcolor{darkblue}{\textbf{\ipa{hæ̃˧qʰv̩˥ | tʰi˧-qɑ˩!}}} \zh{晚上,(要)盖上! / 我们晚上盖电视机(为了防灰)!} \textcolor{Sepia}{\selectlanguage{english}At night, we cover (the television set with a piece of fabric)!} \textcolor{PineGreen}{\selectlanguage{french}le soir, on recouvre (le téléviseur d'un tissu)! / (on) (le) met le soir (sur le téléviseur)/ (on) (en) recouvre (le téléviseur) le soir!}  
 ¶ \textcolor{darkblue}{\textbf{\ipa{tso˧\textasciitilde{}tso˧ qɑ˥}}} \zh{覆盖东西} \textcolor{Sepia}{\selectlanguage{english}to cover things} \textcolor{PineGreen}{\selectlanguage{french}recouvrir quelque chose}  
\ding{203} \zh{遮(云遮月)、遮挡。} \textcolor{Sepia}{\selectlanguage{english}To hide from view.} \textcolor{PineGreen}{\selectlanguage{french}Voiler, bloquer (la lumière), cacher au regard.} 
\lhead{\firstmark}
\rhead{\botmark}

\subsection{\hspace{-0.5cm} {\Large \textcolor{darkblue}{\textbf{\ipa{‑qɑ˧˥}}}}\hspace{0.5cm}[\kern2pt{\textcolor{darkblue}{\textbf{\ipa{qɑ˧˥}}}}\kern2pt]} \hypertarget{‑qA\string_M\string_T1}{}
\markboth{\textcolor{darkblue}{\textbf{\ipa{‑qɑ˧˥}}}}{}
\textcolor{teal}{\zh{后置词}} \hspace{4pt} \zh{声调类:} MH.
\zh{给、对。} \textcolor{Sepia}{\selectlanguage{english}Dative (to); comitative (with).} \textcolor{PineGreen}{\selectlanguage{french}À (datif); avec (comitatif).} 
\lhead{\firstmark}
\rhead{\botmark}

\subsection{\hspace{-0.5cm} {\Large \textcolor{darkblue}{\textbf{\ipa{qɑ˧˥}}}}\hspace{0.5cm}[\kern2pt{\textcolor{darkblue}{\textbf{\ipa{qɑ˧˥}}}}\kern2pt]} \hypertarget{qA\string_M\string_T1}{}
\markboth{\textcolor{darkblue}{\textbf{\ipa{qɑ˧˥}}}}{}
\textcolor{teal}{\zh{动词}} \hspace{4pt} \zh{声调类:} MH.
\zh{帮助。} \textcolor{Sepia}{\selectlanguage{english}To help.} \textcolor{PineGreen}{\selectlanguage{french}Aider.}  ¶ \textcolor{darkblue}{\textbf{\ipa{tʰi˧-qɑ˧˥}}} \zh{\mytextsc{dur}} \textcolor{Sepia}{\selectlanguage{english}\mytextsc{dur}} \textcolor{PineGreen}{\selectlanguage{french}\mytextsc{dur}}  
 ¶ \textcolor{darkblue}{\textbf{\ipa{qɑ˩\textasciitilde{}qɑ˧˥}}} \zh{\mytextsc{重叠:帮帮忙}} \textcolor{Sepia}{\selectlanguage{english}\mytextsc{red}} \textcolor{PineGreen}{\selectlanguage{french}\mytextsc{red}}  
 ¶ \textcolor{darkblue}{\textbf{\ipa{hĩ˧ qɑ˩\textasciitilde{}qɑ˩}}} \zh{帮人,到别人家去工作(例如收庄稼的时候)} \textcolor{Sepia}{\selectlanguage{english}to help people; to go and work at someone else's place (e.g. during the harvest)} \textcolor{PineGreen}{\selectlanguage{french}aider des gens; aller travailler chez autrui (par ex. pendant les récoltes)}  
 ¶ \textcolor{darkblue}{\textbf{\ipa{njɤ˧ no˧ qɑ˧\textasciitilde{}qɑ˥}}} \zh{我帮你} \textcolor{Sepia}{\selectlanguage{english}I help you} \textcolor{PineGreen}{\selectlanguage{french}je t'aide}  

\lhead{\firstmark}
\rhead{\botmark}

\subsection{\hspace{-0.5cm} {\Large \textcolor{darkblue}{\textbf{\ipa{qæ˥}}} \textsubscript{1}}\hspace{0.5cm}[\kern2pt{\textcolor{darkblue}{\textbf{\ipa{qæ˥}}}}\kern2pt]} \hypertarget{q\{\string_T1}{}
\markboth{\textcolor{darkblue}{\textbf{\ipa{qæ˥}}} \textsubscript{1}}{}
\textcolor{teal}{\zh{动词}} \hspace{4pt} \zh{声调类:} H.
\zh{搬。} \textcolor{Sepia}{\selectlanguage{english}To displace, to move (e.g. earth from one spot to another).} \textcolor{PineGreen}{\selectlanguage{french}Déplacer, transporter (ex.: transporter de la terre, après avoir creusé).} \zh{当地汉语方言:}\zh{盘。} ¶ \textcolor{darkblue}{\textbf{\ipa{le˧-qæ˥}}} \zh{\mytextsc{accomp}} \textcolor{Sepia}{\selectlanguage{english}\mytextsc{accomp}} \textcolor{PineGreen}{\selectlanguage{french}\mytextsc{accomp}}  

\lhead{\firstmark}
\rhead{\botmark}

\subsection{\hspace{-0.5cm} {\Large \textcolor{darkblue}{\textbf{\ipa{qæ˥}}} \textsubscript{2}}\hspace{0.5cm}[\kern2pt{\textcolor{darkblue}{\textbf{\ipa{qæ˥}}}}\kern2pt]} \hypertarget{q\{\string_T2}{}
\markboth{\textcolor{darkblue}{\textbf{\ipa{qæ˥}}} \textsubscript{2}}{}
\textcolor{teal}{\zh{动词}} \hspace{4pt} \zh{声调类:} H.
\zh{换。} \textcolor{Sepia}{\selectlanguage{english}To change.} \textcolor{PineGreen}{\selectlanguage{french}Changer.}  ¶ \textcolor{darkblue}{\textbf{\ipa{le˧-qæ˥-ze˩}}} \zh{换了} \textcolor{Sepia}{\selectlanguage{english}\mytextsc{accomp} \string_ \mytextsc{pfv}} \textcolor{PineGreen}{\selectlanguage{french}\mytextsc{accomp} \string_ \mytextsc{pfv}}  
 ¶ \textcolor{darkblue}{\textbf{\ipa{dʑi˧hṽ˧ qæ˧}}} \zh{换衣服} \textcolor{Sepia}{\selectlanguage{english}to change clothes} \textcolor{PineGreen}{\selectlanguage{french}changer de vêtements}  
 ¶ \textcolor{darkblue}{\textbf{\ipa{bɑ˩lɑ˩˥ | tʰi˧-qæ˥}}} \zh{换衣服} \textcolor{Sepia}{\selectlanguage{english}to change clothes} \textcolor{PineGreen}{\selectlanguage{french}changer de vêtements}  
 ¶ \textcolor{darkblue}{\textbf{\ipa{qæ˧\textasciitilde{}qæ˧}}} \zh{\mytextsc{重叠:交换}} \textcolor{Sepia}{\selectlanguage{english}\mytextsc{red}: to exchange (an object for another)} \textcolor{PineGreen}{\selectlanguage{french}\mytextsc{red}: échanger (un objet contre un autre)}  
 ¶ \textcolor{darkblue}{\textbf{\ipa{qæ˧\textasciitilde{}qæ˧-ɻ̍˥}}} \zh{\mytextsc{red} \mytextsc{inceptive}} \textcolor{Sepia}{\selectlanguage{english}\mytextsc{red} \mytextsc{inceptive}} \textcolor{PineGreen}{\selectlanguage{french}\mytextsc{red} \mytextsc{inchoatif}}  
 ¶ \textcolor{darkblue}{\textbf{\ipa{qæ˧\textasciitilde{}qæ˧-ɻ̍˧-ze˥}}} \zh{\mytextsc{red} \mytextsc{inceptive} \mytextsc{pfv}} \textcolor{Sepia}{\selectlanguage{english}\mytextsc{red} \mytextsc{inceptive} \mytextsc{pfv}} \textcolor{PineGreen}{\selectlanguage{french}\mytextsc{red} \mytextsc{inchoatif} \mytextsc{pfv}}  
 ¶ \textcolor{darkblue}{\textbf{\ipa{tso˧\textasciitilde{}tso˧ qæ˧\textasciitilde{}qæ˧}}} \zh{交换东西} \textcolor{Sepia}{\selectlanguage{english}to exchange things} \textcolor{PineGreen}{\selectlanguage{french}échanger des choses}  
 ¶ \textcolor{darkblue}{\textbf{\ipa{le˧-qæ˧\textasciitilde{}qæ˧(-ze˩)}}} \zh{\mytextsc{accomp} \mytextsc{red} (\mytextsc{pfv})} \textcolor{Sepia}{\selectlanguage{english}\mytextsc{accomp} \mytextsc{red} (\mytextsc{pfv})} \textcolor{PineGreen}{\selectlanguage{french}\mytextsc{accomp} \mytextsc{red} (\mytextsc{pfv})}  

\lhead{\firstmark}
\rhead{\botmark}

\subsection{\hspace{-0.5cm} {\Large \textcolor{darkblue}{\textbf{\ipa{qæ˥}}} \textsubscript{3}}\hspace{0.5cm}[\kern2pt{\textcolor{darkblue}{\textbf{\ipa{qæ˥}}}}\kern2pt]} \hypertarget{q\{\string_T3}{}
\markboth{\textcolor{darkblue}{\textbf{\ipa{qæ˥}}} \textsubscript{3}}{}
\textcolor{teal}{\zh{动词}} \hspace{4pt} \zh{声调类:} H.
\zh{雕。} \textcolor{Sepia}{\selectlanguage{english}To sculpt.} \textcolor{PineGreen}{\selectlanguage{french}Sculpter.}  ¶ \textcolor{darkblue}{\textbf{\ipa{le˧-qæ˥-ze˩}}} \zh{雕了} \textcolor{Sepia}{\selectlanguage{english}\mytextsc{accomp} \string_ \mytextsc{pfv}} \textcolor{PineGreen}{\selectlanguage{french}\mytextsc{accomp} \string_ \mytextsc{pfv}}  
 ¶ \textcolor{darkblue}{\textbf{\ipa{bæ˩bæ˩ qæ˥}}} \zh{雕花} \textcolor{Sepia}{\selectlanguage{english}to sculpt a flower} \textcolor{PineGreen}{\selectlanguage{french}sculpter une fleur}  

\lhead{\firstmark}
\rhead{\botmark}

\subsection{\hspace{-0.5cm} {\Large \textcolor{darkblue}{\textbf{\ipa{qæ˧do˧}}}}\hspace{0.5cm}[\kern2pt{\textcolor{darkblue}{\textbf{\ipa{qæ˩do˩˥}}}}\kern2pt]} \hypertarget{q\{\string_Mdo\string_M1}{}
\markboth{\textcolor{darkblue}{\textbf{\ipa{qæ˧do˧}}}}{}
\textcolor{teal}{\zh{名词}} \hspace{4pt} \zh{声调类:} M.
\zh{木材、木料。} \textcolor{Sepia}{\selectlanguage{english}Timber, lumber.} \textcolor{PineGreen}{\selectlanguage{french}Bois de charpente, tronc coupé.}  ¶ \textcolor{darkblue}{\textbf{\ipa{ʑi˧mi˧-qæ˩do˩}}} \zh{建主房的木材} \textcolor{Sepia}{\selectlanguage{english}lumber for the construction of the main building of a Na farm} \textcolor{PineGreen}{\selectlanguage{french}bois de charpente utilisé pour le bâtiment principal}  
 ¶ \textcolor{darkblue}{\textbf{\ipa{ʑi˧qʰwɤ˧-qæ˧do\#˥}}} \zh{建房子的木材} \textcolor{Sepia}{\selectlanguage{english}lumber for the construction of a building} \textcolor{PineGreen}{\selectlanguage{french}bois de charpente, bois pour la construction d'un bâtiment}  
 \zh{量词}: \textcolor{darkblue}{\textbf{\ipa{kɤ˧˥}}} \zh{~【同义词】~} \hyperlink{}{\textcolor{darkblue}{\textbf{\ipa{qæ˧ɻ̍˧}}}}. 
\lhead{\firstmark}
\rhead{\botmark}

\subsection{\hspace{-0.5cm} {\Large \textcolor{darkblue}{\textbf{\ipa{qæ˧dzɯ˩}}}}\hspace{0.5cm}[\kern2pt{\textcolor{darkblue}{\textbf{\ipa{qæ˧dzɯ˩}}}}\kern2pt]} \hypertarget{q\{\string_MdzM\string_B1}{}
\markboth{\textcolor{darkblue}{\textbf{\ipa{qæ˧dzɯ˩}}}}{}
\textcolor{teal}{\zh{名词}} \hspace{4pt} \zh{声调类:} L\#.
\zh{一个姓。这个姓,永宁有两家。} \textcolor{Sepia}{\selectlanguage{english}A family name from Yongning. There are two families in Yongning that carry this name.} \textcolor{PineGreen}{\selectlanguage{french}Nom de clan/famille étendue. Deux familles portent ce nom à Yongning.}  ¶ \textcolor{darkblue}{\textbf{\ipa{qæ˧dzɯ˩-ɻ̍˩}}} \zh{\textcolor{darkblue}{\textbf{\ipa{/qæ˧dzɯ˩/}}}家族} \textcolor{Sepia}{\selectlanguage{english}the \textcolor{darkblue}{\textbf{\ipa{/qæ˧dzɯ˩/}}} clan, the \textcolor{darkblue}{\textbf{\ipa{/qæ˧dzɯ˩/}}} family} \textcolor{PineGreen}{\selectlanguage{french}le clan \textcolor{darkblue}{\textbf{\ipa{/qæ˧dzɯ˩/}}}, la famille \textcolor{darkblue}{\textbf{\ipa{/qæ˧dzɯ˩/}}}}  
 ¶ \textcolor{darkblue}{\textbf{\ipa{qæ˧dzɯ˩ | -tsʰɯ˧ɻ̍˧}}} \zh{一个人的名字:姓为\textcolor{darkblue}{\textbf{\ipa{/qæ˧dzɯ˩/}}},名为\textcolor{darkblue}{\textbf{\ipa{/tsʰɯ˧ɻ\#˥/}}}} \textcolor{Sepia}{\selectlanguage{english}the name of a person, containing both a family name: \textcolor{darkblue}{\textbf{\ipa{/lqæ˧dzɯ˩/}}}, and a given name: \textcolor{darkblue}{\textbf{\ipa{/tsʰɯ˧ɻ\#˥/}}}} \textcolor{PineGreen}{\selectlanguage{french}nom d'une personne, comportant un nom de famille (\textcolor{darkblue}{\textbf{\ipa{/qæ˧dzɯ˩/}}}) et un prénom (\textcolor{darkblue}{\textbf{\ipa{/tsʰɯ˧ɻ\#˥/}}})}  

\lhead{\firstmark}
\rhead{\botmark}

\subsection{\hspace{-0.5cm} {\Large \textcolor{darkblue}{\textbf{\ipa{qæ˧ɻ̍˧}}}}\hspace{0.5cm}[\kern2pt{\textcolor{darkblue}{\textbf{\ipa{qæ˧ɻ̍˧}}}}\kern2pt]} \hypertarget{q\{\string_Mr£`̍\string_M1}{}
\markboth{\textcolor{darkblue}{\textbf{\ipa{qæ˧ɻ̍˧}}}}{}
\textcolor{teal}{\zh{名词}} \hspace{4pt} \zh{声调类:} M.
\zh{木材、木料。} \textcolor{Sepia}{\selectlanguage{english}Timber, lumber.} \textcolor{PineGreen}{\selectlanguage{french}Bois de charpente, tronc coupé.}  \zh{量词}: \textcolor{darkblue}{\textbf{\ipa{kɤ˧˥}}} \zh{~【同义词】~} \hyperlink{}{\textcolor{darkblue}{\textbf{\ipa{qæ˧do˧}}}}. 
\lhead{\firstmark}
\rhead{\botmark}

\subsection{\hspace{-0.5cm} {\Large \textcolor{darkblue}{\textbf{\ipa{qæ˩\textsubscript{a}}}}}\hspace{0.5cm}[\kern2pt{\textcolor{darkblue}{\textbf{\ipa{qæ˩˥}}}}\kern2pt]} \hypertarget{q\{\string_Ba1}{}
\markboth{\textcolor{darkblue}{\textbf{\ipa{qæ˩\textsubscript{a}}}}}{}
\textcolor{teal}{\zh{动词}} \hspace{4pt} \zh{声调类:} L\textsubscript{a}.
\zh{哄(孩子)。} \textcolor{Sepia}{\selectlanguage{english}To coax (a child).} \textcolor{PineGreen}{\selectlanguage{french}Cajoler un enfant.}  ¶ \textcolor{darkblue}{\textbf{\ipa{zo˧ qæ˥}}} \zh{哄孩子} \textcolor{Sepia}{\selectlanguage{english}to coax a child} \textcolor{PineGreen}{\selectlanguage{french}cajoler un enfant}  
 ¶ \textcolor{darkblue}{\textbf{\ipa{le˧-qæ˧\textasciitilde{}qæ˥ | le˧-ʑi˧-kʰɯ˥}}} \zh{哄睡着} \textcolor{Sepia}{\selectlanguage{english}to put asleep by coaxing, to coax asleep} \textcolor{PineGreen}{\selectlanguage{french}endormir (un enfant) en le cajolant}  

\lhead{\firstmark}
\rhead{\botmark}

\subsection{\hspace{-0.5cm} {\Large \textcolor{darkblue}{\textbf{\ipa{qæ˩\textsubscript{b}}}}}\hspace{0.5cm}[\kern2pt{\textcolor{darkblue}{\textbf{\ipa{qæ˩˥}}}}\kern2pt]} \hypertarget{q\{\string_Bb1}{}
\markboth{\textcolor{darkblue}{\textbf{\ipa{qæ˩\textsubscript{b}}}}}{}
\textcolor{teal}{\zh{动词}} \hspace{4pt} \zh{声调类:} L\textsubscript{b}.
\zh{欺骗。} \textcolor{Sepia}{\selectlanguage{english}To cheat, to deceive.} \textcolor{PineGreen}{\selectlanguage{french}Tromper.}  ¶ \textcolor{darkblue}{\textbf{\ipa{hĩ˧ qæ˥-kv̩˩}}} \zh{狡猾、很能骗人的} \textcolor{Sepia}{\selectlanguage{english}sly, who is good at deceiving people} \textcolor{PineGreen}{\selectlanguage{french}rusé, qui sait tromper son monde}  
 ¶ \textcolor{darkblue}{\textbf{\ipa{hĩ˧ qæ˥ | ʐwæ˩˥}}} \zh{狡猾、很能骗人的} \textcolor{Sepia}{\selectlanguage{english}sly, who is good at deceiving people} \textcolor{PineGreen}{\selectlanguage{french}qui trompe magistralement son monde}  
 ¶ \textcolor{darkblue}{\textbf{\ipa{hĩ˧ qæ˥ mɤ˩-ɖo˩!}}} \zh{不要骗人!(这个信条,是发音合作人的祖母教的)} \textcolor{Sepia}{\selectlanguage{english}One must not cheat others! / One must not deceive people! (A precept taught by the main consultant's grandmother)} \textcolor{PineGreen}{\selectlanguage{french}il ne faut pas tromper (autrui)! (précepte inculqué à la locutrice par sa grand-mère)}  
 ¶ \textcolor{darkblue}{\textbf{\ipa{qæ˩-mɤ˩-ɖo˩˥!}}} \zh{不要骗人!(这个信条,是发音合作人的祖母教的)} \textcolor{Sepia}{\selectlanguage{english}One must not cheat (others)! / One must not deceive people! (A precept taught by the main consultant's grandmother)} \textcolor{PineGreen}{\selectlanguage{french}il ne faut pas tromper (autrui)! (précepte inculqué à la locutrice par sa grand-mère)}  
 ¶ \textcolor{darkblue}{\textbf{\ipa{mɤ˧-qæ˩}}} \zh{不骗} \textcolor{Sepia}{\selectlanguage{english}\mytextsc{neg}} \textcolor{PineGreen}{\selectlanguage{french}\mytextsc{neg}}  
 ¶ \textcolor{darkblue}{\textbf{\ipa{hĩ˧ qæ˥-tso˩\textasciitilde{}tso˩!}}} \zh{骗人的东西!(关于买来的一团线,质量不好)} \textcolor{Sepia}{\selectlanguage{english}Shoddy stuff! (Literally: 'deceitful stuff!') (Context: a comment about thread of poor quality, bought at the market)} \textcolor{PineGreen}{\selectlanguage{french}C'est de la camelote! / C'est un truc d'arnaqueurs! (au sujet d'une bobine de fil de mauvaise qualité, achetée dans le commerce)}  

\lhead{\firstmark}
\rhead{\botmark}

\subsection{\hspace{-0.5cm} {\Large \textcolor{darkblue}{\textbf{\ipa{qæ˩di˩}}}}\hspace{0.5cm}[\kern2pt{\textcolor{darkblue}{\textbf{\ipa{qæ˩di˩˥}}}}\kern2pt]} \hypertarget{q\{\string_Bdi\string_B1}{}
\markboth{\textcolor{darkblue}{\textbf{\ipa{qæ˩di˩}}}}{}
\textcolor{teal}{\zh{动词}} \hspace{4pt} \zh{声调类:} L.
\zh{弹(弹脸)。} \textcolor{Sepia}{\selectlanguage{english}To flick, to flip.} \textcolor{PineGreen}{\selectlanguage{french}Donner une chiquenaude.} 
\lhead{\firstmark}
\rhead{\botmark}

\subsection{\hspace{-0.5cm} {\Large \textcolor{darkblue}{\textbf{\ipa{qæ˧˥}}} \textsubscript{1}}\hspace{0.5cm}[\kern2pt{\textcolor{darkblue}{\textbf{\ipa{qæ˧˥}}}}\kern2pt]} \hypertarget{q\{\string_M\string_T1}{}
\markboth{\textcolor{darkblue}{\textbf{\ipa{qæ˧˥}}} \textsubscript{1}}{}
\textcolor{teal}{\zh{动词}} \hspace{4pt} \zh{声调类:} MH.
\zh{燃烧,如:烧尸体(进行火葬时)。} \textcolor{Sepia}{\selectlanguage{english}To burn something, e.g. to cremate a corpse.} \textcolor{PineGreen}{\selectlanguage{french}Brûler quelque chose; par exemple: incinérer un corps.}  ¶ \textcolor{darkblue}{\textbf{\ipa{mv̩˧ qæ˩-ze˩}}} \zh{火烧着了 / 着火了} \textcolor{Sepia}{\selectlanguage{english}the fire has started, the fire is blazing; a fire has caught} \textcolor{PineGreen}{\selectlanguage{french}le feu est parti, ça brûle, ça flambe; un incendie est parti}  
 ¶ \textcolor{darkblue}{\textbf{\ipa{mv̩˧ le˧-qæ˧˥ / mv̩˧ le˧-qæ˧-ze˥}}} \zh{火在烧 / 着火了} \textcolor{Sepia}{\selectlanguage{english}the fire is burning; a fire has caught} \textcolor{PineGreen}{\selectlanguage{french}ça brûle; il y a un incendie}  
 ¶ \textcolor{darkblue}{\textbf{\ipa{mv̩˧ qæ˥-ɻ̍˩}}} \zh{火在烧 / 火烧着了} \textcolor{Sepia}{\selectlanguage{english}the fire is burning} \textcolor{PineGreen}{\selectlanguage{french}ça brûle; il y a un incendie}  
 ¶ \textcolor{darkblue}{\textbf{\ipa{mv̩˧ qæ˥-ɻ̍˩ kʰɯ˩}}} \zh{(有人)放火} \textcolor{Sepia}{\selectlanguage{english}to start a fire (as an act of destruction/war), to commit arson} \textcolor{PineGreen}{\selectlanguage{french}lancer un incendie, déclencher un incendie, mettre le feu (acte criminel)}  
 ¶ \textcolor{darkblue}{\textbf{\ipa{mv̩˧qæ˥-ɻ̍˩-hɯ˩}}} \zh{(有人)放火了!} \textcolor{Sepia}{\selectlanguage{english}a fire has started} \textcolor{PineGreen}{\selectlanguage{french}un incendie est parti}  

\lhead{\firstmark}
\rhead{\botmark}

\subsection{\hspace{-0.5cm} {\Large \textcolor{darkblue}{\textbf{\ipa{qæ˧˥}}} \textsubscript{2}}\hspace{0.5cm}[\kern2pt{\textcolor{darkblue}{\textbf{\ipa{qæ˧˥}}}}\kern2pt]} \hypertarget{q\{\string_M\string_T2}{}
\markboth{\textcolor{darkblue}{\textbf{\ipa{qæ˧˥}}} \textsubscript{2}}{}
\textcolor{teal}{\zh{动词}} \hspace{4pt} \zh{声调类:} MH.
\zh{疼。} \textcolor{Sepia}{\selectlanguage{english}To suffer, to have pain.} \textcolor{PineGreen}{\selectlanguage{french}Souffrir, avoir mal.}  ¶ \textcolor{darkblue}{\textbf{\ipa{bi˧mi˧ qæ˧˥}}} \zh{肚子疼} \textcolor{Sepia}{\selectlanguage{english}to have a stomach-ache} \textcolor{PineGreen}{\selectlanguage{french}avoir mal au ventre}  
 ¶ \textcolor{darkblue}{\textbf{\ipa{ɬo˧kʰv̩˧ qæ˧˥}}} \zh{腰疼} \textcolor{Sepia}{\selectlanguage{english}the waist hurts, the lower back hurts} \textcolor{PineGreen}{\selectlanguage{french}avoir mal à la hanche}  
 ¶ \textcolor{darkblue}{\textbf{\ipa{ʁo˧qʰwɤ˩ qæ˩}}} \zh{头疼} \textcolor{Sepia}{\selectlanguage{english}to have a headache} \textcolor{PineGreen}{\selectlanguage{french}avoir mal à la tête}  

\lhead{\firstmark}
\rhead{\botmark}

\subsection{\hspace{-0.5cm} {\Large \textcolor{darkblue}{\textbf{\ipa{qæ˩˥}}} \textsubscript{1}}\hspace{0.5cm}[\kern2pt{\textcolor{darkblue}{\textbf{\ipa{qæ˩˥}}}}\kern2pt]} \hypertarget{q\{\string_B\string_T1}{}
\markboth{\textcolor{darkblue}{\textbf{\ipa{qæ˩˥}}} \textsubscript{1}}{}
\textcolor{teal}{\zh{名词}} \hspace{4pt} \zh{声调类:} LH.
\zh{油,食用油。} \textcolor{Sepia}{\selectlanguage{english}Oil; cooking oil.} \textcolor{PineGreen}{\selectlanguage{french}Huile (terme générique; huile de friture).} 
\lhead{\firstmark}
\rhead{\botmark}

\subsection{\hspace{-0.5cm} {\Large \textcolor{darkblue}{\textbf{\ipa{qæ˩˥}}} \textsubscript{2}}\hspace{0.5cm}[\kern2pt{\textcolor{darkblue}{\textbf{\ipa{qæ˩˥}}}}\kern2pt]} \hypertarget{q\{\string_B\string_T2}{}
\markboth{\textcolor{darkblue}{\textbf{\ipa{qæ˩˥}}} \textsubscript{2}}{}
\textcolor{teal}{\zh{名词}} \hspace{4pt} \zh{声调类:} LH.
\zh{胶。} \textcolor{Sepia}{\selectlanguage{english}Glue.} \textcolor{PineGreen}{\selectlanguage{french}Colle.}  \zh{量词}: \textcolor{darkblue}{\textbf{\ipa{kʰwɤ˥}}} 
\lhead{\firstmark}
\rhead{\botmark}

\subsection{\hspace{-0.5cm} {\Large \textcolor{darkblue}{\textbf{\ipa{qi˧qi˧}}}}\hspace{0.5cm}[\kern2pt{\textcolor{darkblue}{\textbf{\ipa{qi˧qi˧}}}}\kern2pt]} \hypertarget{qi\string_Mqi\string_M1}{}
\markboth{\textcolor{darkblue}{\textbf{\ipa{qi˧qi˧}}}}{}
\textcolor{teal}{\zh{助词}} \hspace{4pt} \zh{声调类:} M.
\zh{原来、一开始。} \textcolor{Sepia}{\selectlanguage{english}Originally, to begin with.} \textcolor{PineGreen}{\selectlanguage{french}À l'origine.} 
\lhead{\firstmark}
\rhead{\botmark}

\subsection{\hspace{-0.5cm} {\Large \textcolor{darkblue}{\textbf{\ipa{qo˥}}} \textsubscript{1}}\hspace{0.5cm}[\kern2pt{\textcolor{darkblue}{\textbf{\ipa{qo˥}}}}\kern2pt]} \hypertarget{qo\string_T1}{}
\markboth{\textcolor{darkblue}{\textbf{\ipa{qo˥}}} \textsubscript{1}}{}
\textcolor{teal}{\zh{动词}} \hspace{4pt} \zh{声调类:} H.
\zh{跪下。} \textcolor{Sepia}{\selectlanguage{english}To kneel down.} \textcolor{PineGreen}{\selectlanguage{french}S'agenouiller (les mains au sol).} 
\lhead{\firstmark}
\rhead{\botmark}

\subsection{\hspace{-0.5cm} {\Large \textcolor{darkblue}{\textbf{\ipa{qo˥}}} \textsubscript{2}}\hspace{0.5cm}[\kern2pt{\textcolor{darkblue}{\textbf{\ipa{qo˥}}}}\kern2pt]} \hypertarget{qo\string_T2}{}
\markboth{\textcolor{darkblue}{\textbf{\ipa{qo˥}}} \textsubscript{2}}{}
\textcolor{teal}{\zh{动词}} \hspace{4pt} \zh{声调类:} H.
\zh{爱,关心。} \textcolor{Sepia}{\selectlanguage{english}To love.} \textcolor{PineGreen}{\selectlanguage{french}Aimer d'amour.}  ¶ \textcolor{darkblue}{\textbf{\ipa{mɤ˧-qo˧}}} \zh{不爱} \textcolor{Sepia}{\selectlanguage{english}\mytextsc{neg}} \textcolor{PineGreen}{\selectlanguage{french}\mytextsc{neg}}  
 ¶ \textcolor{darkblue}{\textbf{\ipa{zo˧mv̩˥zo˩ qo˩}}} \zh{爱孩子} \textcolor{Sepia}{\selectlanguage{english}to love (one's) children} \textcolor{PineGreen}{\selectlanguage{french}aimer (ses) enfants}  
 ¶ \textcolor{darkblue}{\textbf{\ipa{õ˧-hĩ˥ qo˩}}} \zh{爱自己家人} \textcolor{Sepia}{\selectlanguage{english}to love one's family} \textcolor{PineGreen}{\selectlanguage{french}aimer sa famille}  

\lhead{\firstmark}
\rhead{\botmark}

\subsection{\hspace{-0.5cm} {\Large \textcolor{darkblue}{\textbf{\ipa{-qo˧}}}}\hspace{0.5cm}[\kern2pt{\textcolor{darkblue}{\textbf{\ipa{qo˥}}}}\kern2pt]} \hypertarget{-qo\string_M1}{}
\markboth{\textcolor{darkblue}{\textbf{\ipa{-qo˧}}}}{}
\textcolor{teal}{\zh{后置词}} \hspace{4pt} \zh{声调类:} M.
\zh{里。} \textcolor{Sepia}{\selectlanguage{english}In, inside.} \textcolor{PineGreen}{\selectlanguage{french}Dans.} \zh{~【参考】~} \hyperlink{}{\textcolor{darkblue}{\textbf{\ipa{-qo˧lo˩}}}} 
\lhead{\firstmark}
\rhead{\botmark}

\subsection{\hspace{-0.5cm} {\Large \textcolor{darkblue}{\textbf{\ipa{-qo˧lo˩}}}}\hspace{0.5cm}[\kern2pt{\textcolor{darkblue}{\textbf{\ipa{qo˧lo˩}}}}\kern2pt]} \hypertarget{-qo\string_Mlo\string_B1}{}
\markboth{\textcolor{darkblue}{\textbf{\ipa{-qo˧lo˩}}}}{}
\textcolor{teal}{\zh{后置词}} \hspace{4pt} \zh{声调类:} L\#.
\zh{里面。} \textcolor{Sepia}{\selectlanguage{english}In.} \textcolor{PineGreen}{\selectlanguage{french}Dans.}  ¶ \textcolor{darkblue}{\textbf{\ipa{ʈʂʰɯ˧ | ɑ˩ʁo˧-qo˧lo˩ dʑo˩}}} \zh{他在家里。} \textcolor{Sepia}{\selectlanguage{english}(S)he is in the house. / (S)he is inside.} \textcolor{PineGreen}{\selectlanguage{french}Il/elle est à la maison/dans la maison.}  
\zh{~【参考】~} \hyperlink{}{\textcolor{darkblue}{\textbf{\ipa{qo˧lo˩}}}} 
\lhead{\firstmark}
\rhead{\botmark}

\subsection{\hspace{-0.5cm} {\Large \textcolor{darkblue}{\textbf{\ipa{qo˧lo˩}}}}\hspace{0.5cm}[\kern2pt{\textcolor{darkblue}{\textbf{\ipa{qo˧lo˩}}}}\kern2pt]} \hypertarget{qo\string_Mlo\string_B1}{}
\markboth{\textcolor{darkblue}{\textbf{\ipa{qo˧lo˩}}}}{}
\textcolor{teal}{\zh{助词}} \hspace{4pt} \zh{声调类:} L\#.
\zh{里面。} \textcolor{Sepia}{\selectlanguage{english}Inside, within.} \textcolor{PineGreen}{\selectlanguage{french}Dedans, à l'intérieur de, dans.} \zh{~【参考】~} \hyperlink{}{\textcolor{darkblue}{\textbf{\ipa{-qo˧lo˩}}}} 
\lhead{\firstmark}
\rhead{\botmark}

\subsection{\hspace{-0.5cm} {\Large \textcolor{darkblue}{\textbf{\ipa{qo˧pv̩˩}}}}\hspace{0.5cm}[\kern2pt{\textcolor{darkblue}{\textbf{\ipa{qo˧pv̩˩}}}}\kern2pt]} \hypertarget{qo\string_Mpv\string_=\string_B1}{}
\markboth{\textcolor{darkblue}{\textbf{\ipa{qo˧pv̩˩}}}}{}
\textcolor{teal}{\zh{名词}} \hspace{4pt} \zh{声调类:} L\#.
\zh{布谷鸟。} \textcolor{Sepia}{\selectlanguage{english}Cuckoo.} \textcolor{PineGreen}{\selectlanguage{french}Coucou.}  ¶ \textcolor{darkblue}{\textbf{\ipa{qo˧pv̩˩-ɻwæ˩ | ɖɯ˧-ɲi˥}}} \zh{清明节。直译:“布谷鸟叫的那天”} \textcolor{Sepia}{\selectlanguage{english}Ancestors' Day, Tomb-Sweeping Day, on the first day of the fifth month; literally: 'the day when the cuckoo sings'} \textcolor{PineGreen}{\selectlanguage{french}Le Jour des Ancêtres, au 1er jour du 5e mois. Littéralement: “le jour où chante le coucou”.}  
 \zh{量词}: \textcolor{darkblue}{\textbf{\ipa{mi˩}}} 
\lhead{\firstmark}
\rhead{\botmark}

\subsection{\hspace{-0.5cm} {\Large \textcolor{darkblue}{\textbf{\ipa{qo˧pv̩˩-ʐwæ˩ɖʐæ˩}}}}\hspace{0.5cm}[\kern2pt{\textcolor{darkblue}{\textbf{\ipa{qo˩pv̩˧ʐwæ˧ɖʐæ˧}}}}\kern2pt]} \hypertarget{qo\string_Mpv\string_=\string_B-z`w\{\string_Bd`z`\{\string_B1}{}
\markboth{\textcolor{darkblue}{\textbf{\ipa{qo˧pv̩˩-ʐwæ˩ɖʐæ˩}}}}{}
\textcolor{teal}{\zh{名词}} \hspace{4pt} \zh{声调类:} LM-.
\zh{松鸦。} \textcolor{Sepia}{\selectlanguage{english}Jay, \textit{Garrulus glandarius sinensis}.} \textcolor{PineGreen}{\selectlanguage{french}Geai, \textit{Garrulus glandarius sinensis}.} 
\lhead{\firstmark}
\rhead{\botmark}

\subsection{\hspace{-0.5cm} {\Large \textcolor{darkblue}{\textbf{\ipa{qo˧tv̩˩}}}}\hspace{0.5cm}[\kern2pt{\textcolor{darkblue}{\textbf{\ipa{qo˧tv̩˩}}}}\kern2pt]} \hypertarget{qo\string_Mtv\string_=\string_B1}{}
\markboth{\textcolor{darkblue}{\textbf{\ipa{qo˧tv̩˩}}}}{}
\textcolor{teal}{\zh{名词}} \hspace{4pt} \zh{声调类:} L\#/LM.
\zh{果核。} \textcolor{Sepia}{\selectlanguage{english}Kernel, fruit stone, pit.} \textcolor{PineGreen}{\selectlanguage{french}Noyau (aussi pour: graines de tournesol; et pour: bobines de fil).}  ¶ \textcolor{darkblue}{\textbf{\ipa{dʑi˧ʁo˩-qo˩tv̩˩}}} \zh{桃子果核} \textcolor{Sepia}{\selectlanguage{english}peach kernel} \textcolor{PineGreen}{\selectlanguage{french}noyau de pêche}  
 \zh{量词}: \textcolor{darkblue}{\textbf{\ipa{ɭɯ˧}}} 
\lhead{\firstmark}
\rhead{\botmark}

\subsection{\hspace{-0.5cm} {\Large \textcolor{darkblue}{\textbf{\ipa{qo˩\textsubscript{a}}}}}\hspace{0.5cm}[\kern2pt{\textcolor{darkblue}{\textbf{\ipa{qo˩˥}}}}\kern2pt]} \hypertarget{qo\string_Ba1}{}
\markboth{\textcolor{darkblue}{\textbf{\ipa{qo˩\textsubscript{a}}}}}{}
\textcolor{teal}{\zh{动词}} \hspace{4pt} \zh{声调类:} L\textsubscript{a}.
\zh{放、储存。} \textcolor{Sepia}{\selectlanguage{english}To put away, to preserve (e.g. to put leftovers in a box so flies won't land on it).} \textcolor{PineGreen}{\selectlanguage{french}Garder, serrer, ranger (de la nourriture dans un récipient à l'abri des mouches).} 
\lhead{\firstmark}
\rhead{\botmark}

\subsection{\hspace{-0.5cm} {\Large \textcolor{darkblue}{\textbf{\ipa{qo˩ho˧˥}}}}\hspace{0.5cm}[\kern2pt{\textcolor{darkblue}{\textbf{\ipa{qo˩ho˧˥}}}}\kern2pt]} \hypertarget{qo\string_Bho\string_M\string_T1}{}
\markboth{\textcolor{darkblue}{\textbf{\ipa{qo˩ho˧˥}}}}{}
\textcolor{teal}{\zh{名词}} \hspace{4pt} \zh{声调类:} LM+MH\#.
\zh{礼盒。} \textcolor{Sepia}{\selectlanguage{english}Round wicker/bamboo box used to carry gifts.} \textcolor{PineGreen}{\selectlanguage{french}Boîte en vannerie ronde, dans laquelle on place les cadeaux qu’on vient offrir; est formée de deux parties qui s’emboîtent; on la porte lorsqu'on se rend chez quelqu'un dans le cadre d'un événement social important. Cf récit F4. Une photo de cet objet est présente dans le rapport d'enquête de terrain publié en 1986 en 3 volumes (永宁纳西族……调查).}  \zh{量词}: \textcolor{darkblue}{\textbf{\ipa{ɭɯ˧}}} 
\lhead{\firstmark}
\rhead{\botmark}

\subsection{\hspace{-0.5cm} {\Large \textcolor{darkblue}{\textbf{\ipa{qo˩qɑ˩}}}}\hspace{0.5cm}[\kern2pt{\textcolor{darkblue}{\textbf{\ipa{qo˩qɑ˩˥}}}}\kern2pt]} \hypertarget{qo\string_BqA\string_B1}{}
\markboth{\textcolor{darkblue}{\textbf{\ipa{qo˩qɑ˩}}}}{}
\textcolor{teal}{\zh{名词}} \hspace{4pt} \zh{声调类:} L.
\zh{垭口。} \textcolor{Sepia}{\selectlanguage{english}Mountain pass.} \textcolor{PineGreen}{\selectlanguage{french}Col (de montagne).}  \zh{量词}: \textcolor{darkblue}{\textbf{\ipa{ɭɯ˧}}} 
\lhead{\firstmark}
\rhead{\botmark}

\subsection{\hspace{-0.5cm} {\Large \textcolor{darkblue}{\textbf{\ipa{qo˩tv̩˩-lv̩˥}}}}\hspace{0.5cm}[\kern2pt{\textcolor{darkblue}{\textbf{\ipa{xxxx non-correspondance entre le nombre de morphèmes et le nombre de tons de morphèmes}}}}\kern2pt]} \hypertarget{qo\string_Btv\string_=\string_B-lv\string_=\string_T1}{}
\markboth{\textcolor{darkblue}{\textbf{\ipa{qo˩tv̩˩-lv̩˥}}}}{}
\textcolor{teal}{\zh{名词}} \hspace{4pt} \zh{声调类:} L+H\#.
\zh{团。} \textcolor{Sepia}{\selectlanguage{english}Ball, lump.} \textcolor{PineGreen}{\selectlanguage{french}Boule.}  ¶ \textcolor{darkblue}{\textbf{\ipa{li˩-qo˩tv̩˥-lv̩˩}}} \zh{沱茶} \textcolor{Sepia}{\selectlanguage{english}tea leaves compressed in bowl shape} \textcolor{PineGreen}{\selectlanguage{french}thé comprimé en boule}  
 ¶ \textcolor{darkblue}{\textbf{\ipa{li˩-qo˩tv̩˥-lv̩˩ | ɖɯ˧-qʰwɤ˧ tɕɤ˥}}} \zh{煮一碗沱茶} \textcolor{Sepia}{\selectlanguage{english}to make a bowl of tea, using tea leaves compressed in bowl shape} \textcolor{PineGreen}{\selectlanguage{french}préparer un bol de thé avec du thé comprimé en boule}  
 \zh{量词}: \textcolor{darkblue}{\textbf{\ipa{ɭɯ˧}}} 
\lhead{\firstmark}
\rhead{\botmark}

\subsection{\hspace{-0.5cm} {\Large \textcolor{darkblue}{\textbf{\ipa{qv̩˩˥}}}}\hspace{0.5cm}[\kern2pt{\textcolor{darkblue}{\textbf{\ipa{qv̩˩˥}}}}\kern2pt]} \hypertarget{qv\string_=\string_B\string_T1}{}
\markboth{\textcolor{darkblue}{\textbf{\ipa{qv̩˩˥}}}}{}
\textcolor{teal}{\zh{名词}} \hspace{4pt} \zh{声调类:} LH.
\zh{把手。} \textcolor{Sepia}{\selectlanguage{english}Handle.} \textcolor{PineGreen}{\selectlanguage{french}Poignée, manche (d'une valise, d'une bouteille thermos, d'une louche...).}  \zh{量词}: \textcolor{darkblue}{\textbf{\ipa{kʰwɤ˥}}} 
\lhead{\firstmark}
\rhead{\botmark}

\subsection{\hspace{-0.5cm} {\Large \textcolor{darkblue}{\textbf{\ipa{qv̩˧˥}}}}\hspace{0.5cm}[\kern2pt{\textcolor{darkblue}{\textbf{\ipa{qv̩˧˥}}}}\kern2pt]} \hypertarget{qv\string_=\string_M\string_T1}{}
\markboth{\textcolor{darkblue}{\textbf{\ipa{qv̩˧˥}}}}{}
\textcolor{teal}{\zh{动词}} \hspace{4pt} \zh{声调类:} MH.
\zh{吓(吓唬)。} \textcolor{Sepia}{\selectlanguage{english}To frighten.} \textcolor{PineGreen}{\selectlanguage{french}Faire peur, effrayer.}  ¶ \textcolor{darkblue}{\textbf{\ipa{hĩ˧ qv̩˩}}} \zh{吓人} \textcolor{Sepia}{\selectlanguage{english}to frighten people} \textcolor{PineGreen}{\selectlanguage{french}faire peur aux gens}  
 ¶ \textcolor{darkblue}{\textbf{\ipa{no˧ | hĩ˧ qv̩˩-zo˩! / ʈʂʰɯ˧-ɳɯ˧ | hĩ˧ qv̩˩-zo˩!}}} \zh{你吓人! / 他吓人!} \textcolor{Sepia}{\selectlanguage{english}You frighten people! / He frightens people!} \textcolor{PineGreen}{\selectlanguage{french}tu fais peur aux gens! Il fait peur aux gens!}  
 ¶ \textcolor{darkblue}{\textbf{\ipa{ʈʂʰɯ˧ | njæ˩ qv̩˩-tsʰɯ˩˥!}}} \zh{他吓人!} \textcolor{Sepia}{\selectlanguage{english}He frightens people!} \textcolor{PineGreen}{\selectlanguage{french}il fait peur!}  
 ¶ \textcolor{darkblue}{\textbf{\ipa{njɤ˧ɳɯ˧ | ʈʂʰɯ˧ qv̩˩-bi˩!}}} \zh{我要吓唬他一下!} \textcolor{Sepia}{\selectlanguage{english}I am going to frighten her/him!} \textcolor{PineGreen}{\selectlanguage{french}Je vais lui faire peur!}  
 ¶ \textcolor{darkblue}{\textbf{\ipa{tʰɑ˧-qv̩˧˥!}}} \zh{别吓唬(人家)!} \textcolor{Sepia}{\selectlanguage{english}\mytextsc{prohib}} \textcolor{PineGreen}{\selectlanguage{french}\mytextsc{prohib}}  

\lhead{\firstmark}
\rhead{\botmark}

\subsection{\hspace{-0.5cm} {\Large \textcolor{darkblue}{\textbf{\ipa{qv̩˩\textsubscript{a}}}}}\hspace{0.5cm}[\kern2pt{\textcolor{darkblue}{\textbf{\ipa{qv̩˩˥}}}}\kern2pt]} \hypertarget{qv\string_=\string_Ba1}{}
\markboth{\textcolor{darkblue}{\textbf{\ipa{qv̩˩\textsubscript{a}}}}}{}
\textcolor{teal}{\zh{动词}} \hspace{4pt} \zh{声调类:} L\textsubscript{a}.
\zh{冲走。} \textcolor{Sepia}{\selectlanguage{english}To wash (something) along (of water); to be carried (by water) (heavy objects, e.g. rocks are carried by a stream; the verb cannot be used for light objects, such as leaves).} \textcolor{PineGreen}{\selectlanguage{french}Faire rouler, emporter, charrier (un objet lourd: par exemple, le courant emporte des cailloux, les charriant au loin; le verbe ne peut s'employer pour des objets légers, par exemple des feuilles).}  ¶ \textcolor{darkblue}{\textbf{\ipa{le˧-qv̩˩ | le˧-po˧-tsʰɯ˧˥}}} \zh{冲到某个地方} \textcolor{Sepia}{\selectlanguage{english}to carry to a certain place, to wash along all the way to a certain place} \textcolor{PineGreen}{\selectlanguage{french}charrier, amener en faisant rouler: un torrent en crue charrie des cailloux jusque dans la plaine}  
 ¶ \textcolor{darkblue}{\textbf{\ipa{lv̩˧mi˧ | ɬi˧dʑɯ˩-ɳɯ˩ | qv̩˩˥.}}} \zh{石头被永宁河水冲(到坝子)} \textcolor{Sepia}{\selectlanguage{english}The stones are carried (down into the plain) by (the strong current of) the river of Yongning.} \textcolor{PineGreen}{\selectlanguage{french}les pierres sont amenées par (le courant de) la rivière de Yongning}  
 ¶ \textcolor{darkblue}{\textbf{\ipa{dʑɯ˧-ɳɯ˧ | le˧-qv̩˩ | le˧-po˧-tsʰɯ˧-hĩ˥ | lv̩˧mi˧}}} \zh{水流冲下来的石头} \textcolor{Sepia}{\selectlanguage{english}stones carried over (to this place) by the stream} \textcolor{PineGreen}{\selectlanguage{french}pierres amenées par la rivière, pierres charriées (jusqu'ici) par la rivière}  

\lhead{\firstmark}
\rhead{\botmark}

\subsection{\hspace{-0.5cm} {\Large \textcolor{darkblue}{\textbf{\ipa{qv̩˧dzi˩}}}}\hspace{0.5cm}[\kern2pt{\textcolor{darkblue}{\textbf{\ipa{qv̩˧dzi˩}}}}\kern2pt]} \hypertarget{qv\string_=\string_Mdzi\string_B1}{}
\markboth{\textcolor{darkblue}{\textbf{\ipa{qv̩˧dzi˩}}}}{}
\textcolor{teal}{\zh{名词}} \hspace{4pt} \zh{声调类:} L\#.
\zh{马尾松。} \textcolor{Sepia}{\selectlanguage{english}\textit{Pinus massoniana D.Don in Lamb.}, Masson's pine, Chinese red pine, horsetail pine. Its seeds are not edible (the fish eat them, but they are poisonous for humans).} \textcolor{PineGreen}{\selectlanguage{french}\textit{Pinus massoniana D.Don in Lamb.}, conifère de la famille des \textit{Pinaceae}. Ses pignes ne sont pas comestibles: les poissons les mangent, mais pour les hommes elles sont vénéneuses.} \zh{当地汉语方言:}\zh{马松树。} ¶ \textcolor{darkblue}{\textbf{\ipa{qv̩˧dzi˩-lv̩˩\textasciitilde{}lv̩˩, | dzɯ˧ mɤ˧-ɖo˧!}}} \zh{马松树的果子,不要吃!(有毒)} \textcolor{Sepia}{\selectlanguage{english}One must not eat the seeds of Masson's pine! (It is poisonous)} \textcolor{PineGreen}{\selectlanguage{french}Il ne faut pas manger les pignes du pin de Masson! (Elles sont vénéneuses.)}  
 \zh{量词}: \textcolor{darkblue}{\textbf{\ipa{dzi˩, ʝi˧}}} 
\lhead{\firstmark}
\rhead{\botmark}

\subsection{\hspace{-0.5cm} {\Large \textcolor{darkblue}{\textbf{\ipa{qv̩˧ɻ\#˥}}}}\hspace{0.5cm}[\kern2pt{\textcolor{darkblue}{\textbf{\ipa{qv̩˧ɻ˧}}}}\kern2pt]} \hypertarget{qv\string_=\string_Mr£`\#\string_T1}{}
\markboth{\textcolor{darkblue}{\textbf{\ipa{qv̩˧ɻ\#˥}}}}{}
\textcolor{teal}{\zh{名词}} \hspace{4pt} \zh{声调类:} \#H.
\zh{永宁的一座山。} \textcolor{Sepia}{\selectlanguage{english}Name of a mountain in Yongning.} \textcolor{PineGreen}{\selectlanguage{french}Une montagne de Yongning.}  ¶ \textcolor{darkblue}{\textbf{\ipa{qv̩˧ɻ̍˧-ʁo˧-qʰwɤ˥}}} \zh{\textcolor{darkblue}{\textbf{\ipa{/qv̩˧ɻ̍˧/}}}山的山顶} \textcolor{Sepia}{\selectlanguage{english}the top of the \textcolor{darkblue}{\textbf{\ipa{/qv̩˧ɻ̍˧/}}} mountain} \textcolor{PineGreen}{\selectlanguage{french}le sommet de la montagne \textcolor{darkblue}{\textbf{\ipa{/qv̩˧ɻ̍˧/}}}}  

\lhead{\firstmark}
\rhead{\botmark}

\subsection{\hspace{-0.5cm} {\Large \textcolor{darkblue}{\textbf{\ipa{qv̩˧tɕi˥}}}}\hspace{0.5cm}[\kern2pt{\textcolor{darkblue}{\textbf{\ipa{qv̩˧tɕi˥}}}}\kern2pt]} \hypertarget{qv\string_=\string_Mts£i\string_T1}{}
\markboth{\textcolor{darkblue}{\textbf{\ipa{qv̩˧tɕi˥}}}}{}
\textcolor{teal}{\zh{名词}} \hspace{4pt} \zh{声调类:} H\#.
\zh{痰。} \textcolor{Sepia}{\selectlanguage{english}Spittle, phlegm, sputum.} \textcolor{PineGreen}{\selectlanguage{french}Crachat, mucus.} 
\lhead{\firstmark}
\rhead{\botmark}

\subsection{\hspace{-0.5cm} {\Large \textcolor{darkblue}{\textbf{\ipa{qv̩˧ʈʂæ˧˥}}}}\hspace{0.5cm}[\kern2pt{\textcolor{darkblue}{\textbf{\ipa{qv̩˧ʈʂæ˧˥}}}}\kern2pt]} \hypertarget{qv\string_=\string_Mt`s`\{\string_M\string_T1}{}
\markboth{\textcolor{darkblue}{\textbf{\ipa{qv̩˧ʈʂæ˧˥}}}}{}
\textcolor{teal}{\zh{名词}} \hspace{4pt} \zh{声调类:} MH\#.
\ding{202} \zh{喉咙。} \textcolor{Sepia}{\selectlanguage{english}Throat.} \textcolor{PineGreen}{\selectlanguage{french}Gorge.}  \zh{量词}: \textcolor{darkblue}{\textbf{\ipa{ɭɯ˧}}} \ding{203} \zh{声音。} \textcolor{Sepia}{\selectlanguage{english}Voice.} \textcolor{PineGreen}{\selectlanguage{french}Voix.}  ¶ \textcolor{darkblue}{\textbf{\ipa{ʈʂʰɯ˧ | qv̩˧ʈʂæ˧ dʑɤ˥!}}} \zh{他嗓子好。} \textcolor{Sepia}{\selectlanguage{english}(S)he has a beautiful voice.} \textcolor{PineGreen}{\selectlanguage{french}Elle/il a une belle voix.}  
 ¶ \textcolor{darkblue}{\textbf{\ipa{ʈʂʰɯ˧ | qv̩˧ʈʂæ˧˥ | ɖwæ˧˥ | dʑɤ˩˥!}}} \zh{他嗓子很好。} \textcolor{Sepia}{\selectlanguage{english}(S)he has a really beautiful voice.} \textcolor{PineGreen}{\selectlanguage{french}Elle/il a une très belle voix.}  

\lhead{\firstmark}
\rhead{\botmark}

\subsection{\hspace{-0.5cm} {\Large \textcolor{darkblue}{\textbf{\ipa{qwɑ˧mæ\#˥}}}}\hspace{0.5cm}[\kern2pt{\textcolor{darkblue}{\textbf{\ipa{qwɑ˧mæ˧}}}}\kern2pt]} \hypertarget{qwA\string_Mm\{\#\string_T1}{}
\markboth{\textcolor{darkblue}{\textbf{\ipa{qwɑ˧mæ\#˥}}}}{}
\textcolor{teal}{\zh{名词}} \hspace{4pt} \zh{声调类:} \#H.
\zh{主屋的中庭:在主屋上半部分与门之间的空间。} \textcolor{Sepia}{\selectlanguage{english}Middle part of the main room.} \textcolor{PineGreen}{\selectlanguage{french}Partie médiane du foyer: sur la partie surélevée, mais “côté cuisine”, pas la partie la plus noble de l'espace où on prend les repas.}  \zh{量词}: \textcolor{darkblue}{\textbf{\ipa{kʰwɤ˥}}} 
\lhead{\firstmark}
\rhead{\botmark}

\subsection{\hspace{-0.5cm} {\Large \textcolor{darkblue}{\textbf{\ipa{qwæ˧}}}}\hspace{0.5cm}[\kern2pt{\textcolor{darkblue}{\textbf{\ipa{qwæ˥}}}}\kern2pt]} \hypertarget{qw\{\string_M1}{}
\markboth{\textcolor{darkblue}{\textbf{\ipa{qwæ˧}}}}{}
\textcolor{teal}{\zh{名词}} \hspace{4pt} \zh{声调类:} M.
\zh{床垫子。} \textcolor{Sepia}{\selectlanguage{english}Mat, bed mat.} \textcolor{PineGreen}{\selectlanguage{french}Sommier (de lit); banc large.}  ¶ \textcolor{darkblue}{\textbf{\ipa{qwæ˧mi\#˥}}} \zh{大床垫子} \textcolor{Sepia}{\selectlanguage{english}large mat} \textcolor{PineGreen}{\selectlanguage{french}grand sommier}  
 \zh{量词}: \textcolor{darkblue}{\textbf{\ipa{nɑ˧}}} 
\lhead{\firstmark}
\rhead{\botmark}

\subsection{\hspace{-0.5cm} {\Large \textcolor{darkblue}{\textbf{\ipa{qwæ˧lo˧˥}}}}\hspace{0.5cm}[\kern2pt{\textcolor{darkblue}{\textbf{\ipa{qwæ˧lo˧}}}}\kern2pt]} \hypertarget{qw\{\string_Mlo\string_M\string_T1}{}
\markboth{\textcolor{darkblue}{\textbf{\ipa{qwæ˧lo˧˥}}}}{}
\textcolor{teal}{\zh{名词}} \hspace{4pt} \zh{声调类:} MH\#.
\zh{过道、小道。} \textcolor{Sepia}{\selectlanguage{english}Passageway, small lane, small path.} \textcolor{PineGreen}{\selectlanguage{french}Petit passage, petit sentier.}  ¶ \textcolor{darkblue}{\textbf{\ipa{qwæ˧lo˧-qo˥ | gɤ˩tɕo˧ le˧-jo˩}}} \zh{抄小道} \textcolor{Sepia}{\selectlanguage{english}to come by the small lane} \textcolor{PineGreen}{\selectlanguage{french}venir par le petit chemin/la sente (contexte: on demande à l'enquêteur si, pour se rendre de la maison de la consultante à son hameau natal, tout proche, il est passé par la rue principale de Yongning, ou a emprunté le petit chemin de derrière, parmi les champs)}  
 \zh{量词}: \textcolor{darkblue}{\textbf{\ipa{kʰɯ˩}}} 
\lhead{\firstmark}
\rhead{\botmark}

\subsection{\hspace{-0.5cm} {\Large \textcolor{darkblue}{\textbf{\ipa{qwæ˧ʁo\#˥}}}}\hspace{0.5cm}[\kern2pt{\textcolor{darkblue}{\textbf{\ipa{qwæ˧ʁo˧}}}}\kern2pt]} \hypertarget{qw\{\string_MRo\#\string_T1}{}
\markboth{\textcolor{darkblue}{\textbf{\ipa{qwæ˧ʁo\#˥}}}}{}
\textcolor{teal}{\zh{名词}} \hspace{4pt} \zh{声调类:} \#H.
\zh{主屋里面的长凳:客人和老人坐的地方。} \textcolor{Sepia}{\selectlanguage{english}The bench of the main room, close to the hearth, where guests are seated.} \textcolor{PineGreen}{\selectlanguage{french}Banc de la pièce principale, proche du foyer, où s'asseyent les hôtes.}  \zh{量词}: \textcolor{darkblue}{\textbf{\ipa{ɭɯ˧}}} \zh{~【同音词】~} \hyperlink{}{\textcolor{darkblue}{\textbf{\ipa{qwæ˧˥}}} \textsubscript{3}} 
\lhead{\firstmark}
\rhead{\botmark}

\subsection{\hspace{-0.5cm} {\Large \textcolor{darkblue}{\textbf{\ipa{qwæ˧ʂe\#˥}}}}\hspace{0.5cm}[\kern2pt{\textcolor{darkblue}{\textbf{\ipa{qwæ˧ʂe˥}}}}\kern2pt]} \hypertarget{qw\{\string_Ms`e\#\string_T1}{}
\markboth{\textcolor{darkblue}{\textbf{\ipa{qwæ˧ʂe\#˥}}}}{}
\textcolor{teal}{\zh{名词}} \hspace{4pt} \zh{声调类:} \#H.
\zh{臭虫。} \textcolor{Sepia}{\selectlanguage{english}Bedbug.} \textcolor{PineGreen}{\selectlanguage{french}Punaise.}  \zh{量词}: \textcolor{darkblue}{\textbf{\ipa{mi˩}}} 
\lhead{\firstmark}
\rhead{\botmark}

\subsection{\hspace{-0.5cm} {\Large \textcolor{darkblue}{\textbf{\ipa{qwæ˧ʂe˧lɑ˧bv̩˥}}}}\hspace{0.5cm}[\kern2pt{\textcolor{darkblue}{\textbf{\ipa{qwæ˧ʂe˧lɑ˧bv̩˧}}}}\kern2pt]} \hypertarget{qw\{\string_Ms`e\string_MlA\string_Mbv\string_=\string_T1}{}
\markboth{\textcolor{darkblue}{\textbf{\ipa{qwæ˧ʂe˧lɑ˧bv̩˥}}}}{}
\textcolor{teal}{\zh{名词}} \hspace{4pt} \zh{声调类:} H\#.
\zh{一种蠕虫。} \textcolor{Sepia}{\selectlanguage{english}A species of worm.} \textcolor{PineGreen}{\selectlanguage{french}Sorte de ver.}  \zh{量词}: \textcolor{darkblue}{\textbf{\ipa{mi˩}}} 
\lhead{\firstmark}
\rhead{\botmark}

\subsection{\hspace{-0.5cm} {\Large \textcolor{darkblue}{\textbf{\ipa{qwæ˧zo˧zo˩}}}}\hspace{0.5cm}[\kern2pt{\textcolor{darkblue}{\textbf{\ipa{qwæ˧zo˧zo˥}}}}\kern2pt]} \hypertarget{qw\{\string_Mzo\string_Mzo\string_B1}{}
\markboth{\textcolor{darkblue}{\textbf{\ipa{qwæ˧zo˧zo˩}}}}{}
\textcolor{teal}{\zh{名词}} \hspace{4pt} \zh{声调类:} L\#.
\zh{主屋的长凳,离火塘近。这是客人的尊座。} \textcolor{Sepia}{\selectlanguage{english}The bench of the main room, close to the hearth, where guests are seated.} \textcolor{PineGreen}{\selectlanguage{french}Banc de la pièce principale, proche du foyer, où s'asseyent les hôtes.}  \zh{量词}: \textcolor{darkblue}{\textbf{\ipa{pɤ˩}}} 
\lhead{\firstmark}
\rhead{\botmark}

\subsection{\hspace{-0.5cm} {\Large \textcolor{darkblue}{\textbf{\ipa{qwæ˩ɖʐæ˩}}}}\hspace{0.5cm}[\kern2pt{\textcolor{darkblue}{\textbf{\ipa{qwæ˧ɖʐæ˧˥}}}}\kern2pt]} \hypertarget{qw\{\string_Bd`z`\{\string_B1}{}
\markboth{\textcolor{darkblue}{\textbf{\ipa{qwæ˩ɖʐæ˩}}}}{}
\textcolor{teal}{\zh{名词}} \hspace{4pt} \zh{声调类:} L.
\zh{颚、嘴、嘴巴、口。} \textcolor{Sepia}{\selectlanguage{english}Jaw; mouth.} \textcolor{PineGreen}{\selectlanguage{french}Mâchoire; bouche.}  ¶ \textcolor{darkblue}{\textbf{\ipa{qwæ˩ɖʐæ˩-qo˥-ɳɯ˩ | ʈʰæ˧˥}}} \zh{咬在嘴里} \textcolor{Sepia}{\selectlanguage{english}to masticate, to gnaw} \textcolor{PineGreen}{\selectlanguage{french}mastiquer, ronger}  
 \zh{量词}: \textcolor{darkblue}{\textbf{\ipa{ɭɯ˧}}} 
\lhead{\firstmark}
\rhead{\botmark}

\subsection{\hspace{-0.5cm} {\Large \textcolor{darkblue}{\textbf{\ipa{qwæ˩\textasciitilde{}qwæ˧˥}}}}\hspace{0.5cm}[\kern2pt{\textcolor{darkblue}{\textbf{\ipa{qwæ˧qwæ˩}}}}\kern2pt]} \hypertarget{qw\{\string_B~qw\{\string_M\string_T1}{}
\markboth{\textcolor{darkblue}{\textbf{\ipa{qwæ˩\textasciitilde{}qwæ˧˥}}}}{}
\textcolor{teal}{\zh{动词}} \hspace{4pt} \zh{声调类:} .
\zh{抠痒。} \textcolor{Sepia}{\selectlanguage{english}To scratch.} \textcolor{PineGreen}{\selectlanguage{french}Se gratter; gratter, gratouiller.}  ¶ \textcolor{darkblue}{\textbf{\ipa{le˧-qwæ˧\textasciitilde{}qwæ˩-ze˩}}} \zh{\mytextsc{accomp} \string_ \mytextsc{red} \mytextsc{pfv}} \textcolor{Sepia}{\selectlanguage{english}\mytextsc{accomp} \string_ \mytextsc{red} \mytextsc{pfv}} \textcolor{PineGreen}{\selectlanguage{french}\mytextsc{accomp} \string_ \mytextsc{red} \mytextsc{pfv}}  

\lhead{\firstmark}
\rhead{\botmark}

\subsection{\hspace{-0.5cm} {\Large \textcolor{darkblue}{\textbf{\ipa{qwæ˩ʂv̩˧˥}}}}\hspace{0.5cm}[\kern2pt{\textcolor{darkblue}{\textbf{\ipa{qwæ˧ʂv̩˧}}}}\kern2pt]} \hypertarget{qw\{\string_Bs`v\string_=\string_M\string_T1}{}
\markboth{\textcolor{darkblue}{\textbf{\ipa{qwæ˩ʂv̩˧˥}}}}{}
\textcolor{teal}{\zh{名词}} \hspace{4pt} \zh{声调类:} LM+MH\#.
\zh{马嚼子。} \textcolor{Sepia}{\selectlanguage{english}Bit (of a bridle).} \textcolor{PineGreen}{\selectlanguage{french}Mors.}  ¶ \textcolor{darkblue}{\textbf{\ipa{ʐwæ˧-qwæ˥ʂv̩˩}}} \zh{马嚼子} \textcolor{Sepia}{\selectlanguage{english}bit of a horse's bridle} \textcolor{PineGreen}{\selectlanguage{french}mors de cheval}  
 \zh{量词}: \textcolor{darkblue}{\textbf{\ipa{nɑ˧}}} 
\lhead{\firstmark}
\rhead{\botmark}

\subsection{\hspace{-0.5cm} {\Large \textcolor{darkblue}{\textbf{\ipa{qwæ˧˥}}} \textsubscript{1}}\hspace{0.5cm}[\kern2pt{\textcolor{darkblue}{\textbf{\ipa{qwæ˧˥}}}}\kern2pt]} \hypertarget{qw\{\string_M\string_T1}{}
\markboth{\textcolor{darkblue}{\textbf{\ipa{qwæ˧˥}}} \textsubscript{1}}{}
\textcolor{teal}{\zh{动词}} \hspace{4pt} \zh{声调类:} MH.
\ding{202} \zh{挖(土)。} \textcolor{Sepia}{\selectlanguage{english}To dig.} \textcolor{PineGreen}{\selectlanguage{french}Creuser, piocher (dans la terre meuble).}  ¶ \textcolor{darkblue}{\textbf{\ipa{tv̩˧qʰv̩˧ qwæ˧˥}}} \zh{挖洞} \textcolor{Sepia}{\selectlanguage{english}to dig a hole} \textcolor{PineGreen}{\selectlanguage{french}creuser un trou}  
 ¶ \textcolor{darkblue}{\textbf{\ipa{ʈʂe˧ qwæ˩}}} \zh{挖土} \textcolor{Sepia}{\selectlanguage{english}to dig out the soil} \textcolor{PineGreen}{\selectlanguage{french}creuser la terre}  
 ¶ \textcolor{darkblue}{\textbf{\ipa{qʰæ˧lo˧ qwæ˥}}} \zh{挖水沟} \textcolor{Sepia}{\selectlanguage{english}to dig a ditch} \textcolor{PineGreen}{\selectlanguage{french}dégager une rigole}  
 ¶ \textcolor{darkblue}{\textbf{\ipa{jɤ˩jo˥ qwæ˩}}} \zh{挖洋芋} \textcolor{Sepia}{\selectlanguage{english}to dig out potatoes, to harvest potatoes} \textcolor{PineGreen}{\selectlanguage{french}déterrer des pommes de terre, récolter des pommes de terre}  
\ding{203} \zh{舀(水)。} \textcolor{Sepia}{\selectlanguage{english}To scoop (water).} \textcolor{PineGreen}{\selectlanguage{french}Puiser (de l'eau).}  ¶ \textcolor{darkblue}{\textbf{\ipa{dʑɯ˩ qwæ˩˥}}} \zh{舀水} \textcolor{Sepia}{\selectlanguage{english}to scoop water} \textcolor{PineGreen}{\selectlanguage{french}puiser de l'eau}  

\lhead{\firstmark}
\rhead{\botmark}

\subsection{\hspace{-0.5cm} {\Large \textcolor{darkblue}{\textbf{\ipa{qwæ˧˥}}} \textsubscript{2}}\hspace{0.5cm}[\kern2pt{\textcolor{darkblue}{\textbf{\ipa{qwæ˧˥}}}}\kern2pt]} \hypertarget{qw\{\string_M\string_T2}{}
\markboth{\textcolor{darkblue}{\textbf{\ipa{qwæ˧˥}}} \textsubscript{2}}{}
\textcolor{teal}{\zh{动词}} \hspace{4pt} \zh{声调类:} MH.
\zh{雕刻。} \textcolor{Sepia}{\selectlanguage{english}To engrave.} \textcolor{PineGreen}{\selectlanguage{french}Graver.}  ¶ \textcolor{darkblue}{\textbf{\ipa{bæ˩bæ˩ qwæ˥}}} \zh{刻花} \textcolor{Sepia}{\selectlanguage{english}to engrave a flower} \textcolor{PineGreen}{\selectlanguage{french}graver une fleur}  
 ¶ \textcolor{darkblue}{\textbf{\ipa{qwæ˩\textasciitilde{}qwæ˧˥}}} \zh{\mytextsc{重叠}} \textcolor{Sepia}{\selectlanguage{english}\mytextsc{red}} \textcolor{PineGreen}{\selectlanguage{french}\mytextsc{red}}  
 ¶ \textcolor{darkblue}{\textbf{\ipa{bæ˩bæ˩ qwæ˥\textasciitilde{}qwæ˩}}} \zh{刻一朵花} \textcolor{Sepia}{\selectlanguage{english}to engrave flowers} \textcolor{PineGreen}{\selectlanguage{french}graver des fleurs}  

\lhead{\firstmark}
\rhead{\botmark}

\subsection{\hspace{-0.5cm} {\Large \textcolor{darkblue}{\textbf{\ipa{qwæ˧˥}}} \textsubscript{3}}\hspace{0.5cm}[\kern2pt{\textcolor{darkblue}{\textbf{\ipa{qwæ˧˥}}}}\kern2pt]} \hypertarget{qw\{\string_M\string_T3}{}
\markboth{\textcolor{darkblue}{\textbf{\ipa{qwæ˧˥}}} \textsubscript{3}}{}
\textcolor{teal}{\zh{名词}} \hspace{4pt} \zh{声调类:} \#H.
\zh{主屋里面的长凳:客人和老人坐的地方。} \textcolor{Sepia}{\selectlanguage{english}The bench of the main room, close to the hearth, where guests are seated.} \textcolor{PineGreen}{\selectlanguage{french}Banc de la pièce principale, proche du foyer, où s'asseyent les hôtes.}  \zh{量词}: \textcolor{darkblue}{\textbf{\ipa{ɭɯ˧}}} \zh{~【同音词】~} \hyperlink{}{\textcolor{darkblue}{\textbf{\ipa{qwæ˧ʁo\#˥}}}} 
\lhead{\firstmark}
\rhead{\botmark}

\subsection{\hspace{-0.5cm} {\Large \textcolor{darkblue}{\textbf{\ipa{qwæ˩˥}}}}\hspace{0.5cm}[\kern2pt{\textcolor{darkblue}{\textbf{\ipa{qwæ˩˥}}}}\kern2pt]} \hypertarget{qw\{\string_B\string_T1}{}
\markboth{\textcolor{darkblue}{\textbf{\ipa{qwæ˩˥}}}}{}
\textcolor{teal}{\zh{名词}} \hspace{4pt} \zh{声调类:} LH.
\zh{嘴巴(单音节)。} \textcolor{Sepia}{\selectlanguage{english}Jaw (monosyllable).} \textcolor{PineGreen}{\selectlanguage{french}Mâchoire (monosyllabe).}  \zh{量词}: \textcolor{darkblue}{\textbf{\ipa{ɭɯ˧}}} 
\lhead{\firstmark}
\rhead{\botmark}

\subsection{\hspace{-0.5cm} {\Large \textcolor{darkblue}{\textbf{\ipa{qwɤ˧}}}}\hspace{0.5cm}[\kern2pt{\textcolor{darkblue}{\textbf{\ipa{qwɤ˥}}}}\kern2pt]} \hypertarget{qw7\string_M1}{}
\markboth{\textcolor{darkblue}{\textbf{\ipa{qwɤ˧}}}}{}
\textcolor{teal}{\zh{名词}} \hspace{4pt} \zh{声调类:} M.
\zh{火塘。} \textcolor{Sepia}{\selectlanguage{english}Fire pit.} \textcolor{PineGreen}{\selectlanguage{french}Foyer, âtre, lieu où on fait du feu dans la maison.}  ¶ \textcolor{darkblue}{\textbf{\ipa{qwɤ˧, | mv̩˧ kʰɯ˩-di˩!}}} \zh{火塘,就是升火的地方!} \textcolor{Sepia}{\selectlanguage{english}The fire pit is the place where one puts fire / where one does a fire!} \textcolor{PineGreen}{\selectlanguage{french}Le foyer, c'est là où on allume le feu!}  
 \zh{量词}: \textcolor{darkblue}{\textbf{\ipa{ɭɯ˧}}} 
\lhead{\firstmark}
\rhead{\botmark}

\subsection{\hspace{-0.5cm} {\Large \textcolor{darkblue}{\textbf{\ipa{qwɤ˧\textsubscript{a}}}}}\hspace{0.5cm}[\kern2pt{\textcolor{darkblue}{\textbf{\ipa{qwɤ˥}}}}\kern2pt]} \hypertarget{qw7\string_Ma1}{}
\markboth{\textcolor{darkblue}{\textbf{\ipa{qwɤ˧\textsubscript{a}}}}}{}
\textcolor{teal}{\zh{动词}} \hspace{4pt} \zh{声调类:} M\textsubscript{a}.
\zh{告状。} \textcolor{Sepia}{\selectlanguage{english}To accuse, to denounce.} \textcolor{PineGreen}{\selectlanguage{french}Accuser.}  ¶ \textcolor{darkblue}{\textbf{\ipa{mɤ˧-qwɤ˧}}} \zh{不告状} \textcolor{Sepia}{\selectlanguage{english}\mytextsc{neg}} \textcolor{PineGreen}{\selectlanguage{french}\mytextsc{neg}}  
 ¶ \textcolor{darkblue}{\textbf{\ipa{hĩ˧ qwɤ˩}}} \zh{告一个人} \textcolor{Sepia}{\selectlanguage{english}to accuse someone, to denounce someone} \textcolor{PineGreen}{\selectlanguage{french}dénoncer quelqu'un}  
 ¶ \textcolor{darkblue}{\textbf{\ipa{njɤ˧-ɳɯ˧ | qwɤ˧-bi˧!}}} \zh{我要告状!} \textcolor{Sepia}{\selectlanguage{english}I am going to denounce/accuse} \textcolor{PineGreen}{\selectlanguage{french}je vais (te) dénoncer!}  
 ¶ \textcolor{darkblue}{\textbf{\ipa{no˧ | le˧-qwɤ˧-hõ˧!}}} \zh{你去告状吧!} \textcolor{Sepia}{\selectlanguage{english}Go and denounce (him/her)!} \textcolor{PineGreen}{\selectlanguage{french}va (le) dénoncer!}  
 ¶ \textcolor{darkblue}{\textbf{\ipa{qwɤ˧\textasciitilde{}qwɤ˩}}} \zh{\mytextsc{重叠}} \textcolor{Sepia}{\selectlanguage{english}\mytextsc{red}} \textcolor{PineGreen}{\selectlanguage{french}\mytextsc{red}}  

\lhead{\firstmark}
\rhead{\botmark}

\subsection{\hspace{-0.5cm} {\Large \textcolor{darkblue}{\textbf{\ipa{qwɤ˧ɭɯ\#˥}}}}\hspace{0.5cm}[\kern2pt{\textcolor{darkblue}{\textbf{\ipa{qwɤ˧ɭɯ˧}}}}\kern2pt]} \hypertarget{qw7\string_Ml\string_RM\#\string_T1}{}
\markboth{\textcolor{darkblue}{\textbf{\ipa{qwɤ˧ɭɯ\#˥}}}}{}
\textcolor{teal}{\zh{名词}} \hspace{4pt} \zh{声调类:} \#H.
\zh{营火、篝火。} \textcolor{Sepia}{\selectlanguage{english}Campfire.} \textcolor{PineGreen}{\selectlanguage{french}Feu de camp: foyer bâti à l'extérieur, provisoirement, lorsqu'on campe sur la montagne.}  ¶ \textcolor{darkblue}{\textbf{\ipa{qwɤ˧ɭɯ˧-pʰɤ˧bɤ˥}}} \zh{敬给祖先的礼物:即使在山上升起篝火野餐,还是要像在家里一样,用餐前先敬给祖先一些饭。} \textcolor{Sepia}{\selectlanguage{english}the gifts offered to the ancestors, at the fireplace: even when building a campfire for one day only on the mountain, one offers a little food to the ancestors before beginning the meal (in the same way as is done at home)} \textcolor{PineGreen}{\selectlanguage{french}les cadeaux offerts aux ancêtres: même lorsqu'il ne s'agit que d'un foyer provisoire, bâti pour une seule journée dans un campement en montagne, on pratique l'offrande d'un peu de nourriture}  

\lhead{\firstmark}
\rhead{\botmark}

\subsection{\hspace{-0.5cm} {\Large \textcolor{darkblue}{\textbf{\ipa{qwɤ˩\textsubscript{a}}}}}\hspace{0.5cm}[\kern2pt{\textcolor{darkblue}{\textbf{\ipa{qwɤ˩˥}}}}\kern2pt]} \hypertarget{qw7\string_Ba1}{}
\markboth{\textcolor{darkblue}{\textbf{\ipa{qwɤ˩\textsubscript{a}}}}}{}
\textcolor{teal}{\zh{动词}} \hspace{4pt} \zh{声调类:} L\textsubscript{a}.
\zh{生长、长。} \textcolor{Sepia}{\selectlanguage{english}To grow.} \textcolor{PineGreen}{\selectlanguage{french}Pousser, grandir.}  ¶ \textcolor{darkblue}{\textbf{\ipa{gɤ˩-qwɤ˥}}} \zh{长大,生长} \textcolor{Sepia}{\selectlanguage{english}to grow} \textcolor{PineGreen}{\selectlanguage{french}grandir, pousser}  
 ¶ \textcolor{darkblue}{\textbf{\ipa{ʈʂʰɯ˧ | gɤ˩-qwɤ˥-ze˩!}}} \zh{他长大了!(关于一个小孩)} \textcolor{Sepia}{\selectlanguage{english}(S)he has grown up! / (S)he has grown a lot! (About a child)} \textcolor{PineGreen}{\selectlanguage{french}Il/elle a grandi! (Au sujet d'un enfant qu'on revoit après un certain temps)}  

\lhead{\firstmark}
\rhead{\botmark}

\subsection{\hspace{-0.5cm} {\Large \textcolor{darkblue}{\textbf{\ipa{qwɤ˩pi˩}}}}\hspace{0.5cm}[\kern2pt{\textcolor{darkblue}{\textbf{\ipa{qwɤ˩pi˩˥}}}}\kern2pt]} \hypertarget{qw7\string_Bpi\string_B1}{}
\markboth{\textcolor{darkblue}{\textbf{\ipa{qwɤ˩pi˩}}}}{}
\textcolor{teal}{\zh{名词}} \hspace{4pt} \zh{声调类:} L.
\zh{嘴巴。} \textcolor{Sepia}{\selectlanguage{english}Mouth.} \textcolor{PineGreen}{\selectlanguage{french}Bouche.}  ¶ \textcolor{darkblue}{\textbf{\ipa{qwɤ˩pi˩-qo˩lo˥}}} \zh{嘴巴里} \textcolor{Sepia}{\selectlanguage{english}inside the mouth} \textcolor{PineGreen}{\selectlanguage{french}à l'intérieur de la bouche}  
 ¶ \textcolor{darkblue}{\textbf{\ipa{[F5] ko˩pi˩-ko˩lo˧}}} \zh{嘴巴里} \textcolor{Sepia}{\selectlanguage{english}inside the mouth} \textcolor{PineGreen}{\selectlanguage{french}dans la bouche, à l'intérieur de la bouche}  
 \zh{量词}: \textcolor{darkblue}{\textbf{\ipa{ɭɯ˧}}} 
\lhead{\firstmark}
\rhead{\botmark}

\newpage
\section*{\centering- \textcolor{darkblue}{\textbf{\ipa{qʰ}}} -}
\subsection{\hspace{-0.5cm} {\Large \textcolor{darkblue}{\textbf{\ipa{qʰɑ˥}}}}\hspace{0.5cm}[\kern2pt{\textcolor{darkblue}{\textbf{\ipa{qʰɑ˥}}}}\kern2pt]} \hypertarget{q\string_hA\string_T1}{}
\markboth{\textcolor{darkblue}{\textbf{\ipa{qʰɑ˥}}}}{}
\textcolor{teal}{\zh{形容词}} \hspace{4pt} \zh{声调类:} H.
\zh{苦。} \textcolor{Sepia}{\selectlanguage{english}Bitter.} \textcolor{PineGreen}{\selectlanguage{french}Amer.} 
\lhead{\firstmark}
\rhead{\botmark}

\subsection{\hspace{-0.5cm} {\Large \textcolor{darkblue}{\textbf{\ipa{qʰɑ˧-}}}}\hspace{0.5cm}[\kern2pt{\textcolor{darkblue}{\textbf{\ipa{qʰɑ˥}}}}\kern2pt]} \hypertarget{q\string_hA\string_M-1}{}
\markboth{\textcolor{darkblue}{\textbf{\ipa{qʰɑ˧-}}}}{}
\textcolor{teal}{\zh{助词}} \hspace{4pt} \zh{声调类:} .
\zh{多么、非常。} \textcolor{Sepia}{\selectlanguage{english}Very, extremely.} \textcolor{PineGreen}{\selectlanguage{french}Particulièrement, très.}  ¶ \textcolor{darkblue}{\textbf{\ipa{qʰɑ˧-ɖɯ˧-hĩ˧}}} \zh{非常大} \textcolor{Sepia}{\selectlanguage{english}extremely big} \textcolor{PineGreen}{\selectlanguage{french}extrêmement gros}  
 ¶ \textcolor{darkblue}{\textbf{\ipa{qʰɑ˧-ɖɯ˧-gv̩˧}}} \zh{非常大} \textcolor{Sepia}{\selectlanguage{english}extremely large; how large!} \textcolor{PineGreen}{\selectlanguage{french}particulièrement grand}  
 ¶ \textcolor{darkblue}{\textbf{\ipa{qʰɑ˧-ʂwæ˧-gv̩˧}}} \zh{很高、非常高} \textcolor{Sepia}{\selectlanguage{english}extremely tall; how tall!} \textcolor{PineGreen}{\selectlanguage{french}particulièrement grand, de très grande taille}  
 ¶ \textcolor{darkblue}{\textbf{\ipa{qʰɑ˧-ʂwæ˧-mi˧zo˥}}} \zh{很高} \textcolor{Sepia}{\selectlanguage{english}extremely tall} \textcolor{PineGreen}{\selectlanguage{french}très grand, de très haute taille}  
 ¶ \textcolor{darkblue}{\textbf{\ipa{qʰɑ˧-ɖɯ˧-mi˧zo˥}}} \zh{很大} \textcolor{Sepia}{\selectlanguage{english}extremely big} \textcolor{PineGreen}{\selectlanguage{french}très gros, de très grande envergure}  
\zh{~【参考】~} \hyperlink{}{\textcolor{darkblue}{\textbf{\ipa{qʰɑ˧}}} \textsubscript{1}} 
\lhead{\firstmark}
\rhead{\botmark}

\subsection{\hspace{-0.5cm} {\Large \textcolor{darkblue}{\textbf{\ipa{qʰɑ˧}}} \textsubscript{1}}\hspace{0.5cm}[\kern2pt{\textcolor{darkblue}{\textbf{\ipa{qʰɑ˥}}}}\kern2pt]} \hypertarget{q\string_hA\string_M1}{}
\markboth{\textcolor{darkblue}{\textbf{\ipa{qʰɑ˧}}} \textsubscript{1}}{}
\textcolor{teal}{\zh{代词}} \hspace{4pt} \zh{声调类:} M.
\zh{几、多少。} \textcolor{Sepia}{\selectlanguage{english}How many (small number).} \textcolor{PineGreen}{\selectlanguage{french}Combien.}  ¶ \textcolor{darkblue}{\textbf{\ipa{hĩ˧ | qʰɑ˧-kv̩˧˥?}}} \zh{几个人?} \textcolor{Sepia}{\selectlanguage{english}how many people?} \textcolor{PineGreen}{\selectlanguage{french}combien de gens?}  
 ¶ \textcolor{darkblue}{\textbf{\ipa{bæ˩bæ˩˥ | qʰɑ˧-bæ˩?}}} \zh{几朵花?} \textcolor{Sepia}{\selectlanguage{english}how many flowers?} \textcolor{PineGreen}{\selectlanguage{french}combien de fleurs?}  
 ¶ \textcolor{darkblue}{\textbf{\ipa{qʰɑ˧-ʑi˩?}}} \zh{几家?} \textcolor{Sepia}{\selectlanguage{english}how many families?} \textcolor{PineGreen}{\selectlanguage{french}combien de familles?}  
 ¶ \textcolor{darkblue}{\textbf{\ipa{hɑ˧ | qʰɑ˧-tɕʰi˩?}}} \zh{几顿饭?} \textcolor{Sepia}{\selectlanguage{english}how many meals?} \textcolor{PineGreen}{\selectlanguage{french}combien de repas?}  
 ¶ \textcolor{darkblue}{\textbf{\ipa{qʰɑ˧-ɲi˧?}}} \zh{几天?} \textcolor{Sepia}{\selectlanguage{english}how many days?} \textcolor{PineGreen}{\selectlanguage{french}combien de jours?}  
 ¶ \textcolor{darkblue}{\textbf{\ipa{qʰɑ˧-kʰv̩˧ gv̩˧-ze˩?}}} \zh{几岁了?} \textcolor{Sepia}{\selectlanguage{english}How old are you / is (s)he?} \textcolor{PineGreen}{\selectlanguage{french}quel âge avez-(vous)?}  
 ¶ \textcolor{darkblue}{\textbf{\ipa{qʰɑ˧-kʰv̩˧˥?}}} \zh{几年?} \textcolor{Sepia}{\selectlanguage{english}how many years?} \textcolor{PineGreen}{\selectlanguage{french}combien d'années?}  
 ¶ \textcolor{darkblue}{\textbf{\ipa{qʰɑ˧-kʰwɤ˧˥?}}} \zh{几块?} \textcolor{Sepia}{\selectlanguage{english}how many pieces?} \textcolor{PineGreen}{\selectlanguage{french}combien de morceaux?}  
 ¶ \textcolor{darkblue}{\textbf{\ipa{qʰɑ˧-nɑ˧?}}} \zh{几把?} \textcolor{Sepia}{\selectlanguage{english}how many (tools...)?} \textcolor{PineGreen}{\selectlanguage{french}combien (d'outils...)?}  
 ¶ \textcolor{darkblue}{\textbf{\ipa{sɯ˩tʰi˩˥ | qʰɑ˧-nɑ˧ dʑo˧?}}} \zh{有几把刀?} \textcolor{Sepia}{\selectlanguage{english}How many knives are there?} \textcolor{PineGreen}{\selectlanguage{french}Combien y a-t-il de couteaux?}  
 ¶ \textcolor{darkblue}{\textbf{\ipa{qʰɑ˧-kʰɯ˩}}} \zh{几条} \textcolor{Sepia}{\selectlanguage{english}how many (long objects)} \textcolor{PineGreen}{\selectlanguage{french}combien (d'objets longs)}  
 ¶ \textcolor{darkblue}{\textbf{\ipa{qʰɑ˧-kʰɯ˩ dʑo˩?}}} \zh{有几条?} \textcolor{Sepia}{\selectlanguage{english}How many (long objects) are there?} \textcolor{PineGreen}{\selectlanguage{french}Combien y a-t-il (d'objets longs)?}  
 ¶ \textcolor{darkblue}{\textbf{\ipa{qʰɑ˧-mæ˩ dʑo˩?}}} \zh{有几块(钱)?} \textcolor{Sepia}{\selectlanguage{english}How much money do (you) have?} \textcolor{PineGreen}{\selectlanguage{french}Combien (tu) as d'argent?}  
 ¶ \textcolor{darkblue}{\textbf{\ipa{si˧dzi˩ | qʰɑ˧-dzi˩?}}} \zh{几棵树?} \textcolor{Sepia}{\selectlanguage{english}how many trees?} \textcolor{PineGreen}{\selectlanguage{french}combien d'arbres?}  
 ¶ \textcolor{darkblue}{\textbf{\ipa{si˧kɤ˧˥ | qʰɑ˧-kɤ˧˥?}}} \zh{几枝树枝?} \textcolor{Sepia}{\selectlanguage{english}how many branches?} \textcolor{PineGreen}{\selectlanguage{french}combien de branches?}  
 ¶ \textcolor{darkblue}{\textbf{\ipa{qʰɑ˧-kʰɤ˧˥?}}} \zh{几筐?} \textcolor{Sepia}{\selectlanguage{english}how many baskets?} \textcolor{PineGreen}{\selectlanguage{french}combien de cageots?}  
\zh{~【参考】~} \hyperlink{}{\textcolor{darkblue}{\textbf{\ipa{qʰɑ˧}}} \textsubscript{2}} 
\lhead{\firstmark}
\rhead{\botmark}

\subsection{\hspace{-0.5cm} {\Large \textcolor{darkblue}{\textbf{\ipa{qʰɑ˧}}} \textsubscript{2}}\hspace{0.5cm}[\kern2pt{\textcolor{darkblue}{\textbf{\ipa{qʰɑ˥}}}}\kern2pt]} \hypertarget{q\string_hA\string_M2}{}
\markboth{\textcolor{darkblue}{\textbf{\ipa{qʰɑ˧}}} \textsubscript{2}}{}
\textcolor{teal}{\zh{助词}} \hspace{4pt} \zh{声调类:} M.
\zh{几(如:十几个)。} \textcolor{Sepia}{\selectlanguage{english}A few; several; some.} \textcolor{PineGreen}{\selectlanguage{french}Quelques, plusieurs.}  ¶ \textcolor{darkblue}{\textbf{\ipa{tsʰe˩-qʰɑ˩˥}}} \zh{十几个、十来个} \textcolor{Sepia}{\selectlanguage{english}ten and a few more (i.e. between ten and twenty)} \textcolor{PineGreen}{\selectlanguage{french}dix et plus (entre dix et vingt)}  
 ¶ \textcolor{darkblue}{\textbf{\ipa{tsʰe˩-qʰɑ˩-kv̩˩˥}}} \zh{十几个、十来个} \textcolor{Sepia}{\selectlanguage{english}ten and a few more (i.e. between ten and twenty)} \textcolor{PineGreen}{\selectlanguage{french}dix et plus (entre dix et vingt)}  
\zh{~【参考】~} \hyperlink{}{\textcolor{darkblue}{\textbf{\ipa{qʰɑ˧}}} \textsubscript{1}} 
\lhead{\firstmark}
\rhead{\botmark}

\subsection{\hspace{-0.5cm} {\Large \textcolor{darkblue}{\textbf{\ipa{qʰɑ˧dze˧}}}}\hspace{0.5cm}[\kern2pt{\textcolor{darkblue}{\textbf{\ipa{qʰɑ˧dze˧}}}}\kern2pt]} \hypertarget{q\string_hA\string_Mdze\string_M1}{}
\markboth{\textcolor{darkblue}{\textbf{\ipa{qʰɑ˧dze˧}}}}{}
\textcolor{teal}{\zh{名词}} \hspace{4pt} \zh{声调类:} M.
\zh{玉米、包谷。} \textcolor{Sepia}{\selectlanguage{english}Sweet corn; maize; Indian corn.} \textcolor{PineGreen}{\selectlanguage{french}Maïs.}  ¶ \textcolor{darkblue}{\textbf{\ipa{qʰɑ˧dze˧-kʰɯ˩ʈɯ˩}}} \zh{玉米的根} \textcolor{Sepia}{\selectlanguage{english}the roots of the sweetcorn plant} \textcolor{PineGreen}{\selectlanguage{french}racines des plants de maïs}  
 ¶ \textcolor{darkblue}{\textbf{\ipa{qʰɑ˧dze˧ qʰæ˩}}} \zh{采玉米:折断玉米棒子} \textcolor{Sepia}{\selectlanguage{english}to cut ears of sweetcorn, to snap off ears of sweetcorn} \textcolor{PineGreen}{\selectlanguage{french}cueillir le maïs: arracher les épis de maïs}  
 ¶ \textcolor{darkblue}{\textbf{\ipa{qʰɑ˧dze˧ ɖʐɤ˧˥}}} \zh{采玉米} \textcolor{Sepia}{\selectlanguage{english}to harvest sweetcorn, to pick sweetcorn} \textcolor{PineGreen}{\selectlanguage{french}cueillir le maïs}  
 ¶ \textcolor{darkblue}{\textbf{\ipa{qʰɑ˧dze˧-tsɑ˩bɤ˩ | ɖɯ˧-mɤ˩}}} \zh{一点玉米粉} \textcolor{Sepia}{\selectlanguage{english}a little sweetcorn flour} \textcolor{PineGreen}{\selectlanguage{french}un peu de farine de maïs}  
 ¶ \textcolor{darkblue}{\textbf{\ipa{qʰɑ˧dze˧-hɑ˧bɤ˥, | qʰɑ˧dze˧-hɑ˧ɭɯ\#˥, | qʰɑ˧dze˧-tsɑ˩bɤ˩}}} \zh{玉米的三种形态:玉米棒子,玉米粒,玉米粉} \textcolor{Sepia}{\selectlanguage{english}three forms of sweetcorn: sweetcorn ear; sweetcorn grains; sweetcorn flour} \textcolor{PineGreen}{\selectlanguage{french}le maïs sous trois formes: épis de maïs; maïs en grains; farine de maïs}  
 \zh{量词}: \textcolor{darkblue}{\textbf{\ipa{kɤ˧˥}}} 
\lhead{\firstmark}
\rhead{\botmark}

\subsection{\hspace{-0.5cm} {\Large \textcolor{darkblue}{\textbf{\ipa{qʰɑ˧dze˧-hwæ˩-di˩}}}}\hspace{0.5cm}[\kern2pt{\textcolor{darkblue}{\textbf{\ipa{xxxx non-correspondance entre le nombre de morphèmes et le nombre de tons de morphèmes}}}}\kern2pt]} \hypertarget{q\string_hA\string_Mdze\string_M-hw\{\string_B-di\string_B1}{}
\markboth{\textcolor{darkblue}{\textbf{\ipa{qʰɑ˧dze˧-hwæ˩-di˩}}}}{}
\textcolor{teal}{\zh{名词}} \hspace{4pt} \zh{声调类:} \mytextsc{L}.
\zh{粮架的横梁。} \textcolor{Sepia}{\selectlanguage{english}Horizontal beams of the rack for drying grain: 'the place to hang sweetcorn'.} \textcolor{PineGreen}{\selectlanguage{french}Poutrelle d'espalier à sécher le maïs: partie horizontale de la structure en bois. Périphrase: 'endroit où on accroche le maïs'.}  \zh{量词}: \textcolor{darkblue}{\textbf{\ipa{kɤ˧˥}}} 
\lhead{\firstmark}
\rhead{\botmark}

\subsection{\hspace{-0.5cm} {\Large \textcolor{darkblue}{\textbf{\ipa{qʰɑ˧dze˧-lv̩˧}}}}\hspace{0.5cm}[\kern2pt{\textcolor{darkblue}{\textbf{\ipa{xxxx non-correspondance entre le nombre de morphèmes et le nombre de tons de morphèmes}}}}\kern2pt]} \hypertarget{q\string_hA\string_Mdze\string_M-lv\string_=\string_M1}{}
\markboth{\textcolor{darkblue}{\textbf{\ipa{qʰɑ˧dze˧-lv̩˧}}}}{}
\textcolor{teal}{\zh{名词}} \hspace{4pt} \zh{声调类:} M.
\zh{包谷田、玉米田。} \textcolor{Sepia}{\selectlanguage{english}Maize field.} \textcolor{PineGreen}{\selectlanguage{french}Champ de maïs.}  \zh{量词}: \textcolor{darkblue}{\textbf{\ipa{pʰv̩˩}}} 
\lhead{\firstmark}
\rhead{\botmark}

\subsection{\hspace{-0.5cm} {\Large \textcolor{darkblue}{\textbf{\ipa{qʰɑ˧tɑ˧}}}}\hspace{0.5cm}[\kern2pt{\textcolor{darkblue}{\textbf{\ipa{qʰɑ˧tɑ˧}}}}\kern2pt]} \hypertarget{q\string_hA\string_MtA\string_M1}{}
\markboth{\textcolor{darkblue}{\textbf{\ipa{qʰɑ˧tɑ˧}}}}{}
\textcolor{teal}{\zh{代词}} \hspace{4pt} \zh{声调类:} M.
\zh{什么时候。} \textcolor{Sepia}{\selectlanguage{english}When.} \textcolor{PineGreen}{\selectlanguage{french}Quand.}  ¶ \textcolor{darkblue}{\textbf{\ipa{qʰɑ˧tɑ˧ bi˧?}}} \zh{你什么时候去?} \textcolor{Sepia}{\selectlanguage{english}When will you go?} \textcolor{PineGreen}{\selectlanguage{french}quand (y) vas(-tu)?}  

\lhead{\firstmark}
\rhead{\botmark}

\subsection{\hspace{-0.5cm} {\Large \textcolor{darkblue}{\textbf{\ipa{qʰɑ˩jɤ˩}}}}\hspace{0.5cm}[\kern2pt{\textcolor{darkblue}{\textbf{\ipa{qʰɑ˩jɤ˩˥}}}}\kern2pt]} \hypertarget{q\string_hA\string_Bj7\string_B1}{}
\markboth{\textcolor{darkblue}{\textbf{\ipa{qʰɑ˩jɤ˩}}}}{}
\textcolor{teal}{\zh{代词}} \hspace{4pt} \zh{声调类:} L.
\zh{多少。} \textcolor{Sepia}{\selectlanguage{english}How many.} \textcolor{PineGreen}{\selectlanguage{french}Combien.}  ¶ \textcolor{darkblue}{\textbf{\ipa{qʰɑ˩jɤ˩ tʰi˥-ki˩?}}} \zh{要给多少? = 多少钱?} \textcolor{Sepia}{\selectlanguage{english}How much does it cost?} \textcolor{PineGreen}{\selectlanguage{french}combien donner? / c'est combien?}  
 ¶ \textcolor{darkblue}{\textbf{\ipa{qʰɑ˩jɤ˩ ɲi˧?}}} \zh{要多少?} \textcolor{Sepia}{\selectlanguage{english}How much do (you) need?} \textcolor{PineGreen}{\selectlanguage{french}combien (t'en) faut-il?}  
 ¶ \textcolor{darkblue}{\textbf{\ipa{ɖʐe˧ | qʰɑ˩jɤ˩ ɲi˧?}}} \zh{要多少钱?} \textcolor{Sepia}{\selectlanguage{english}How much money does it cost?} \textcolor{PineGreen}{\selectlanguage{french}Combien d'argent ça coûte?}  

\lhead{\firstmark}
\rhead{\botmark}

\subsection{\hspace{-0.5cm} {\Large \textcolor{darkblue}{\textbf{\ipa{qʰɑ˩ne˩}}}}\hspace{0.5cm}[\kern2pt{\textcolor{darkblue}{\textbf{\ipa{qʰɑ˩ne˩˥}}}}\kern2pt]} \hypertarget{q\string_hA\string_Bne\string_B1}{}
\markboth{\textcolor{darkblue}{\textbf{\ipa{qʰɑ˩ne˩}}}}{}
\textcolor{teal}{\zh{代词}} \hspace{4pt} \zh{声调类:} L.
\zh{怎么样。} \textcolor{Sepia}{\selectlanguage{english}How.} \textcolor{PineGreen}{\selectlanguage{french}Pronom interrogatif: comment?}  ¶ \textcolor{darkblue}{\textbf{\ipa{qʰɑ˩ne˩ ʝi˥?}}} \zh{怎么做?} \textcolor{Sepia}{\selectlanguage{english}how to do / how is it done?} \textcolor{PineGreen}{\selectlanguage{french}comment faire?}  
 ¶ \textcolor{darkblue}{\textbf{\ipa{qʰɑ˩ne˩ ʝi˥-tso˩-ɲi˩?}}} \zh{要怎么做?} \textcolor{Sepia}{\selectlanguage{english}how must one do / how is it done?} \textcolor{PineGreen}{\selectlanguage{french}comment faut-il faire?}  
 ¶ \textcolor{darkblue}{\textbf{\ipa{qʰɑ˩ne˩ gv̩˩˥?}}} \zh{怎么做?} \textcolor{Sepia}{\selectlanguage{english}how to do / how is it done?} \textcolor{PineGreen}{\selectlanguage{french}comment faire?}  
 ¶ \textcolor{darkblue}{\textbf{\ipa{qʰɑ˩ne˩ gv̩˩-ho˥-ze˩?}}} \zh{怎么样了?发展到什么程度?} \textcolor{Sepia}{\selectlanguage{english}What happened?} \textcolor{PineGreen}{\selectlanguage{french}que s'est-il passé?}  

\lhead{\firstmark}
\rhead{\botmark}

\subsection{\hspace{-0.5cm} {\Large \textcolor{darkblue}{\textbf{\ipa{qʰæ˥}}}}\hspace{0.5cm}[\kern2pt{\textcolor{darkblue}{\textbf{\ipa{qʰæ˥}}}}\kern2pt]} \hypertarget{q\string_h\{\string_T1}{}
\markboth{\textcolor{darkblue}{\textbf{\ipa{qʰæ˥}}}}{}
\textcolor{teal}{\zh{名词}} \hspace{4pt} \zh{声调类:} \#H.
\ding{202} \zh{屎、垃圾、 肥料。} \textcolor{Sepia}{\selectlanguage{english}Excrements, dung, dropping.} \textcolor{PineGreen}{\selectlanguage{french}Excréments, fèces.}  ¶ \textcolor{darkblue}{\textbf{\ipa{qʰæ˧ ɖɯ˧-pɤ˧ ʂe˧˥}}} \zh{拉一泡屎} \textcolor{Sepia}{\selectlanguage{english}to defecate} \textcolor{PineGreen}{\selectlanguage{french}faire une crotte}  
 ¶ \textcolor{darkblue}{\textbf{\ipa{qʰæ˧ kv̩˥}}} \zh{捡(马……)屎} \textcolor{Sepia}{\selectlanguage{english}to pick up dung} \textcolor{PineGreen}{\selectlanguage{french}ramasser du crottin}  
 ¶ \textcolor{darkblue}{\textbf{\ipa{qʰæ˧-pi˩ kv̩˩}}} \zh{捡一点(马……)屎} \textcolor{Sepia}{\selectlanguage{english}to pick up a little dung (to fertilize the fields)} \textcolor{PineGreen}{\selectlanguage{french}ramasser un peu de crottin}  
 ¶ \textcolor{darkblue}{\textbf{\ipa{qʰæ˧ ɖɯ˧-pi˧ kv̩˥}}} \zh{同上:捡一点(马……)屎} \textcolor{Sepia}{\selectlanguage{english}as above: to pick up a little dung (to fertilize the fields)} \textcolor{PineGreen}{\selectlanguage{french}même sens: ramasser un peu de crottin}  
 \zh{量词}: \textcolor{darkblue}{\textbf{\ipa{pɤ˥}}} \ding{203} \zh{屁。} \textcolor{Sepia}{\selectlanguage{english}Flatulence, fart.} \textcolor{PineGreen}{\selectlanguage{french}Pet.}  ¶ \textcolor{darkblue}{\textbf{\ipa{qʰæ˧ kʰɯ˩}}} \zh{放屁} \textcolor{Sepia}{\selectlanguage{english}to fart} \textcolor{PineGreen}{\selectlanguage{french}péter}  
 ¶ \textcolor{darkblue}{\textbf{\ipa{qʰæ˧ | ɖɯ˧-pɤ˥ kʰɯ˩}}} \zh{放一个屁} \textcolor{Sepia}{\selectlanguage{english}to fart, to make a fart} \textcolor{PineGreen}{\selectlanguage{french}faire un pet}  
 \zh{量词}: \textcolor{darkblue}{\textbf{\ipa{pɤ˥}}} \ding{204} \zh{垃圾。} \textcolor{Sepia}{\selectlanguage{english}Refuse, garbage.} \textcolor{PineGreen}{\selectlanguage{french}Ordure, détritus.} 
\lhead{\firstmark}
\rhead{\botmark}

\subsection{\hspace{-0.5cm} {\Large \textcolor{darkblue}{\textbf{\ipa{qʰæ˥}}} \textsubscript{1}}\hspace{0.5cm}[\kern2pt{\textcolor{darkblue}{\textbf{\ipa{qʰæ˥}}}}\kern2pt]} \hypertarget{q\string_h\{\string_T1}{}
\markboth{\textcolor{darkblue}{\textbf{\ipa{qʰæ˥}}} \textsubscript{1}}{}
\textcolor{teal}{\zh{动词}} \hspace{4pt} \zh{声调类:} H.
\zh{啃(啃骨头)。} \textcolor{Sepia}{\selectlanguage{english}To gnaw, to nibble.} \textcolor{PineGreen}{\selectlanguage{french}Ronger.} 
\lhead{\firstmark}
\rhead{\botmark}

\subsection{\hspace{-0.5cm} {\Large \textcolor{darkblue}{\textbf{\ipa{qʰæ˥}}} \textsubscript{2}}\hspace{0.5cm}[\kern2pt{\textcolor{darkblue}{\textbf{\ipa{qʰæ˥}}}}\kern2pt]} \hypertarget{q\string_h\{\string_T2}{}
\markboth{\textcolor{darkblue}{\textbf{\ipa{qʰæ˥}}} \textsubscript{2}}{}
\textcolor{teal}{\zh{形容词}} \hspace{4pt} \zh{声调类:} H.
\zh{冷(水)。} \textcolor{Sepia}{\selectlanguage{english}Cold (water).} \textcolor{PineGreen}{\selectlanguage{french}Froide (eau).}  ¶ \textcolor{darkblue}{\textbf{\ipa{dʑɯ˩qʰæ˩}}} \zh{凉水} \textcolor{Sepia}{\selectlanguage{english}cold water} \textcolor{PineGreen}{\selectlanguage{french}eau froide}  
 ¶ \textcolor{darkblue}{\textbf{\ipa{qʰæ˧-ɕjæ˧-gv̩˧}}} \zh{冷得很} \textcolor{Sepia}{\selectlanguage{english}very cold} \textcolor{PineGreen}{\selectlanguage{french}très froid}  

\lhead{\firstmark}
\rhead{\botmark}

\subsection{\hspace{-0.5cm} {\Large \textcolor{darkblue}{\textbf{\ipa{qʰæ˥}}} \textsubscript{3}}\hspace{0.5cm}[\kern2pt{\textcolor{darkblue}{\textbf{\ipa{qʰæ˥}}}}\kern2pt]} \hypertarget{q\string_h\{\string_T3}{}
\markboth{\textcolor{darkblue}{\textbf{\ipa{qʰæ˥}}} \textsubscript{3}}{}
\textcolor{teal}{\zh{名词}} \hspace{4pt} \zh{声调类:} \#H.
\zh{水沟(单音节)。} \textcolor{Sepia}{\selectlanguage{english}Trench (monosyllable).} \textcolor{PineGreen}{\selectlanguage{french}Canal, rigole.}  \zh{量词}: \textcolor{darkblue}{\textbf{\ipa{kʰɯ˩}}} 
\lhead{\firstmark}
\rhead{\botmark}

\subsection{\hspace{-0.5cm} {\Large \textcolor{darkblue}{\textbf{\ipa{qʰæ˧\textsubscript{b}}}}}\hspace{0.5cm}[\kern2pt{\textcolor{darkblue}{\textbf{\ipa{qʰæ˩˥}}}}\kern2pt]} \hypertarget{q\string_h\{\string_Mb1}{}
\markboth{\textcolor{darkblue}{\textbf{\ipa{qʰæ˧\textsubscript{b}}}}}{}
\textcolor{teal}{\zh{动词}} \hspace{4pt} \zh{声调类:} M\textsubscript{b}.
\zh{断,破(棍子,竹竿)。} \textcolor{Sepia}{\selectlanguage{english}To break (a stick breaks).} \textcolor{PineGreen}{\selectlanguage{french}(se) briser, (se) casser (ex.: un bâton).}  ¶ \textcolor{darkblue}{\textbf{\ipa{le˧-qʰæ˧-ze˧}}} \zh{断了} \textcolor{Sepia}{\selectlanguage{english}\mytextsc{accomp} \string_ \mytextsc{pfv}} \textcolor{PineGreen}{\selectlanguage{french}\mytextsc{accomp} \string_ \mytextsc{pfv}}  
 ¶ \textcolor{darkblue}{\textbf{\ipa{si˧ qʰæ˩}}} \zh{砸木头} \textcolor{Sepia}{\selectlanguage{english}to break wood} \textcolor{PineGreen}{\selectlanguage{french}briser du bois}  

\lhead{\firstmark}
\rhead{\botmark}

\subsection{\hspace{-0.5cm} {\Large \textcolor{darkblue}{\textbf{\ipa{qʰæ˧kʰwɤ\#˥}}}}\hspace{0.5cm}[\kern2pt{\textcolor{darkblue}{\textbf{\ipa{qʰæ˧kʰwɤ˧}}}}\kern2pt]} \hypertarget{q\string_h\{\string_Mk\string_hw7\#\string_T1}{}
\markboth{\textcolor{darkblue}{\textbf{\ipa{qʰæ˧kʰwɤ\#˥}}}}{}
\textcolor{teal}{\zh{名词}} \hspace{4pt} \zh{声调类:} \#H.
\zh{小水坝,来堵塞田地里的小水渠。} \textcolor{Sepia}{\selectlanguage{english}Small dam (in canal; made of stones and earth).} \textcolor{PineGreen}{\selectlanguage{french}Petit barrage pour bloquer un canal d'irrigation, fait de pierres et de terre. Pour irriguer, on l'ouvre à coups de houe.}  ¶ \textcolor{darkblue}{\textbf{\ipa{qʰæ˧kʰwɤ˧ ɖɯ˧-ɭɯ˧}}} \zh{一个小水坝} \textcolor{Sepia}{\selectlanguage{english}a small dam} \textcolor{PineGreen}{\selectlanguage{french}un petit canal}  
 \zh{量词}: \textcolor{darkblue}{\textbf{\ipa{ɭɯ˧}}} 
\lhead{\firstmark}
\rhead{\botmark}

\subsection{\hspace{-0.5cm} {\Large \textcolor{darkblue}{\textbf{\ipa{qʰæ˧lo˧˥}}}}\hspace{0.5cm}[\kern2pt{\textcolor{darkblue}{\textbf{\ipa{qʰæ˧lo˧˥}}}}\kern2pt]} \hypertarget{q\string_h\{\string_Mlo\string_M\string_T1}{}
\markboth{\textcolor{darkblue}{\textbf{\ipa{qʰæ˧lo˧˥}}}}{}
\textcolor{teal}{\zh{名词}} \hspace{4pt} \zh{声调类:} MH\#.
\zh{小水渠。} \textcolor{Sepia}{\selectlanguage{english}Small gulley, small trench.} \textcolor{PineGreen}{\selectlanguage{french}Petite rigole, petit canal.}  \zh{量词}: \textcolor{darkblue}{\textbf{\ipa{kʰɯ˩}}} \zh{~【参考】~} \hyperlink{}{\textcolor{darkblue}{\textbf{\ipa{qʰæ˧zo\#˥}}}} 
\lhead{\firstmark}
\rhead{\botmark}

\subsection{\hspace{-0.5cm} {\Large \textcolor{darkblue}{\textbf{\ipa{qʰæ˧mi˧}}}}\hspace{0.5cm}[\kern2pt{\textcolor{darkblue}{\textbf{\ipa{qʰæ˧mi˧}}}}\kern2pt]} \hypertarget{q\string_h\{\string_Mmi\string_M1}{}
\markboth{\textcolor{darkblue}{\textbf{\ipa{qʰæ˧mi˧}}}}{}
\textcolor{teal}{\zh{名词}} \hspace{4pt} \zh{声调类:} M.
\zh{大水渠。} \textcolor{Sepia}{\selectlanguage{english}Large trench, canal.} \textcolor{PineGreen}{\selectlanguage{french}Grand canal.}  \zh{量词}: \textcolor{darkblue}{\textbf{\ipa{kʰɯ˩}}} 
\lhead{\firstmark}
\rhead{\botmark}

\subsection{\hspace{-0.5cm} {\Large \textcolor{darkblue}{\textbf{\ipa{qʰæ˧mo˩}}}}\hspace{0.5cm}[\kern2pt{\textcolor{darkblue}{\textbf{\ipa{qʰæ˧mo˩}}}}\kern2pt]} \hypertarget{q\string_h\{\string_Mmo\string_B1}{}
\markboth{\textcolor{darkblue}{\textbf{\ipa{qʰæ˧mo˩}}}}{}
\textcolor{teal}{\zh{名词}} \hspace{4pt} \zh{声调类:} L\#.
\zh{有毒的一种菌子。} \textcolor{Sepia}{\selectlanguage{english}A poisonous mushroom.} \textcolor{PineGreen}{\selectlanguage{french}Un champignon vénéneux.} 
\lhead{\firstmark}
\rhead{\botmark}

\subsection{\hspace{-0.5cm} {\Large \textcolor{darkblue}{\textbf{\ipa{qʰæ˧tv̩˧}}}}\hspace{0.5cm}[\kern2pt{\textcolor{darkblue}{\textbf{\ipa{qʰæ˧tv̩˧}}}}\kern2pt]} \hypertarget{q\string_h\{\string_Mtv\string_=\string_M1}{}
\markboth{\textcolor{darkblue}{\textbf{\ipa{qʰæ˧tv̩˧}}}}{}
\textcolor{teal}{\zh{名词}} \hspace{4pt} \zh{声调类:} M.
\zh{肛门。} \textcolor{Sepia}{\selectlanguage{english}Anus.} \textcolor{PineGreen}{\selectlanguage{french}Anus.}  \zh{量词}: \textcolor{darkblue}{\textbf{\ipa{ɭɯ˧}}} 
\lhead{\firstmark}
\rhead{\botmark}

\subsection{\hspace{-0.5cm} {\Large \textcolor{darkblue}{\textbf{\ipa{qʰæ˧tɕʰi˧}}}}\hspace{0.5cm}[\kern2pt{\textcolor{darkblue}{\textbf{\ipa{qʰæ˧tɕʰi˧}}}}\kern2pt]} \hypertarget{q\string_h\{\string_Mts£\string_hi\string_M1}{}
\markboth{\textcolor{darkblue}{\textbf{\ipa{qʰæ˧tɕʰi˧}}}}{}
\textcolor{teal}{\zh{名词}} \hspace{4pt} \zh{声调类:} M.
\zh{开基(永宁的一个村落)。} \textcolor{Sepia}{\selectlanguage{english}A village of Yongning; Chinese name: Kaiji.} \textcolor{PineGreen}{\selectlanguage{french}Un village de Yongning; nom chinois: Kaiji.}  ¶ \textcolor{darkblue}{\textbf{\ipa{ʈʂʰɯ˧ | qʰæ˧tɕʰi˧-hĩ˧ ɲi˥!}}} \zh{他是开基村人!} \textcolor{Sepia}{\selectlanguage{english}(S)he is from the village of Kaiji!} \textcolor{PineGreen}{\selectlanguage{french}c'est quelqu'un de Kaiji!}  
 ¶ \textcolor{darkblue}{\textbf{\ipa{dʑɤ˩bv̩˧kɤ˧-sɑ˥ʁwɤ˩, | hi˩ʁwɤ˩-lo˥, | æ˩mi˧-ʁwɤ\#˥, | lɑ˧lo˧-ʁwɤ˥, | lɑ˧ŋwɤ˧, | bɤ˧tsʰo˧gv̩˥, | ə˧lɑ˧-ʁwɤ\#˥, | gæ˧ɻæ˩, | qʰæ˧tɕʰi˧, | tʰo˧ʈɯ\#˥}}} \zh{摩梭传统地理概念中,属于永宁的十个村落} \textcolor{Sepia}{\selectlanguage{english}the ten villages traditionally considered as part of Yongning} \textcolor{PineGreen}{\selectlanguage{french}les dix villages comptant traditionnellement comme faisant partie de Yongning}  

\lhead{\firstmark}
\rhead{\botmark}

\subsection{\hspace{-0.5cm} {\Large \textcolor{darkblue}{\textbf{\ipa{qʰæ˧ʈæ˧˥}}}}\hspace{0.5cm}[\kern2pt{\textcolor{darkblue}{\textbf{\ipa{qʰæ˧ʈæ˧˥}}}}\kern2pt]} \hypertarget{q\string_h\{\string_Mt`\{\string_M\string_T1}{}
\markboth{\textcolor{darkblue}{\textbf{\ipa{qʰæ˧ʈæ˧˥}}}}{}
\textcolor{teal}{\zh{形容词}} \hspace{4pt} \zh{声调类:} MH\#.
\zh{安静。} \textcolor{Sepia}{\selectlanguage{english}Quiet, at peace.} \textcolor{PineGreen}{\selectlanguage{french}Paisible, tranquille (personnalité, trait de caractère).}  ¶ \textcolor{darkblue}{\textbf{\ipa{qʰæ˧ʈæ˧˥ | tʰi˧-dzi˩}}} \zh{安静地坐着} \textcolor{Sepia}{\selectlanguage{english}to sit quietly, free from toil and care} \textcolor{PineGreen}{\selectlanguage{french}être assis tranquillement, être tranquille, avoir l'esprit libre}  
 ¶ \textcolor{darkblue}{\textbf{\ipa{qʰæ˧ʈæ˧˥ | tʰi˧-ʝi˧}}} \zh{安静地工作} \textcolor{Sepia}{\selectlanguage{english}to work quietly} \textcolor{PineGreen}{\selectlanguage{french}travailler paisiblement}  

\lhead{\firstmark}
\rhead{\botmark}

\subsection{\hspace{-0.5cm} {\Large \textcolor{darkblue}{\textbf{\ipa{qʰæ˧zo\#˥}}}}\hspace{0.5cm}[\kern2pt{\textcolor{darkblue}{\textbf{\ipa{qʰæ˧zo˧}}}}\kern2pt]} \hypertarget{q\string_h\{\string_Mzo\#\string_T1}{}
\markboth{\textcolor{darkblue}{\textbf{\ipa{qʰæ˧zo\#˥}}}}{}
\textcolor{teal}{\zh{名词}} \hspace{4pt} \zh{声调类:} \#H.
\zh{小水渠。} \textcolor{Sepia}{\selectlanguage{english}Small trench/canal.} \textcolor{PineGreen}{\selectlanguage{french}Petit canal.}  \zh{量词}: \textcolor{darkblue}{\textbf{\ipa{kʰɯ˩}}} \zh{~【参考】~} \hyperlink{}{\textcolor{darkblue}{\textbf{\ipa{qʰæ˧lo˧˥}}}} 
\lhead{\firstmark}
\rhead{\botmark}

\subsection{\hspace{-0.5cm} {\Large \textcolor{darkblue}{\textbf{\ipa{qʰæ˩}}}}\hspace{0.5cm}[\kern2pt{\textcolor{darkblue}{\textbf{\ipa{qʰæ˩˥}}}}\kern2pt]} \hypertarget{q\string_h\{\string_B1}{}
\markboth{\textcolor{darkblue}{\textbf{\ipa{qʰæ˩}}}}{}
\textcolor{teal}{\zh{动词}} \hspace{4pt} \zh{声调类:} L\textsubscript{a}.
\zh{折断。} \textcolor{Sepia}{\selectlanguage{english}To crack, to snap off.} \textcolor{PineGreen}{\selectlanguage{french}Casser; ex.: casser une branche, récolter un épi de maïs en le cassant du plant de maïs.}  ¶ \textcolor{darkblue}{\textbf{\ipa{qʰɑ˧dze˧ qʰæ˩}}} \zh{采玉米} \textcolor{Sepia}{\selectlanguage{english}to harvest sweet corn (literally: to snap off ears of sweet corn)} \textcolor{PineGreen}{\selectlanguage{french}cueillir du maïs}  
 ¶ \textcolor{darkblue}{\textbf{\ipa{qʰɑ˧dze˧ | le˧-qʰæ˩-ze˩}}} \zh{玉米收好了。} \textcolor{Sepia}{\selectlanguage{english}The sweetcorn has been harvested.} \textcolor{PineGreen}{\selectlanguage{french}le maïs est cueilli}  
 ¶ \textcolor{darkblue}{\textbf{\ipa{qʰɑ˧dze˧ | ɖɯ˧-qʰæ˧\textasciitilde{}qʰæ˥-ɻ̍˩}}} \zh{去采些玉米} \textcolor{Sepia}{\selectlanguage{english}to harvest some sweet corn} \textcolor{PineGreen}{\selectlanguage{french}cueillir un peu de maïs}  

\lhead{\firstmark}
\rhead{\botmark}

\subsection{\hspace{-0.5cm} {\Large \textcolor{darkblue}{\textbf{\ipa{qʰæ˩\textsubscript{a}}}} \textsubscript{1}}\hspace{0.5cm}[\kern2pt{\textcolor{darkblue}{\textbf{\ipa{qʰæ˩˥}}}}\kern2pt]} \hypertarget{q\string_h\{\string_Ba1}{}
\markboth{\textcolor{darkblue}{\textbf{\ipa{qʰæ˩\textsubscript{a}}}} \textsubscript{1}}{}
\textcolor{teal}{\zh{形容词}} \hspace{4pt} \zh{声调类:} L\textsubscript{a}.
\zh{假。} \textcolor{Sepia}{\selectlanguage{english}False, fake.} \textcolor{PineGreen}{\selectlanguage{french}Faux, mensonger.}  ¶ \textcolor{darkblue}{\textbf{\ipa{qʰæ˩-hĩ˩˥, | tʰɑ˧-ʐwɤ˩!}}} \zh{假话,不要说! =不要撒谎!} \textcolor{Sepia}{\selectlanguage{english}Do not tell lies! / Do not tell things that are false!} \textcolor{PineGreen}{\selectlanguage{french}ne dis pas de mensonges!}  
 ¶ \textcolor{darkblue}{\textbf{\ipa{qʰæ˧ ʐwɤ˧}}} \zh{撒谎、说谎} \textcolor{Sepia}{\selectlanguage{english}to tell lies} \textcolor{PineGreen}{\selectlanguage{french}dire des mensonges}  

\lhead{\firstmark}
\rhead{\botmark}

\subsection{\hspace{-0.5cm} {\Large \textcolor{darkblue}{\textbf{\ipa{qʰæ˩\textsubscript{a}}}} \textsubscript{2}}\hspace{0.5cm}[\kern2pt{\textcolor{darkblue}{\textbf{\ipa{qʰæ˩˥}}}}\kern2pt]} \hypertarget{q\string_h\{\string_Ba2}{}
\markboth{\textcolor{darkblue}{\textbf{\ipa{qʰæ˩\textsubscript{a}}}} \textsubscript{2}}{}
\textcolor{teal}{\zh{形容词}} \hspace{4pt} \zh{声调类:} L\textsubscript{a}.
\ding{202} \zh{平静、安静,安乐、(身体)健康。} \textcolor{Sepia}{\selectlanguage{english}Well (to feel well); quiet.} \textcolor{PineGreen}{\selectlanguage{french}En paix, tranquille, paisible (époque); en bonne santé (corps).}  ¶ \textcolor{darkblue}{\textbf{\ipa{hĩ˧ | ə˩-qʰæ˩˥?}}} \zh{你好吗? / 一切好吗?} \textcolor{Sepia}{\selectlanguage{english}How are you?} \textcolor{PineGreen}{\selectlanguage{french}est-ce que ça va bien?/tu vas bien? (Formule équivalente du \textcolor{darkblue}{\textbf{\ipa{/ə˥-lɑ˧~lɑ˩/}}} de Lijiang 'Est-ce que vous êtes en bonne santé?', qui à Yongning évoque malencontreusement le rédupliqué \textcolor{darkblue}{\textbf{\ipa{/ə˩-lɑ˩~lɑ˧˥/}}} 'est-ce que [vous] vous disputez?')}  
 ¶ \textcolor{darkblue}{\textbf{\ipa{njɤ˧ | mɤ˧-qʰæ˩.}}} \zh{我不舒服。} \textcolor{Sepia}{\selectlanguage{english}I don't feel well.} \textcolor{PineGreen}{\selectlanguage{french}je ne me sens pas bien.}  
\ding{203} \zh{轻松。} \textcolor{Sepia}{\selectlanguage{english}Light, easy (work).} \textcolor{PineGreen}{\selectlanguage{french}Léger, peu fatigant (travail).}  ¶ \textcolor{darkblue}{\textbf{\ipa{qʰæ˩-hĩ˩˥}}} \zh{轻松的} \textcolor{Sepia}{\selectlanguage{english}\mytextsc{rel}} \textcolor{PineGreen}{\selectlanguage{french}\mytextsc{rel}}  

\lhead{\firstmark}
\rhead{\botmark}

\subsection{\hspace{-0.5cm} {\Large \textcolor{darkblue}{\textbf{\ipa{qʰæ˩bæ˩}}}}\hspace{0.5cm}[\kern2pt{\textcolor{darkblue}{\textbf{\ipa{qʰæ˧bæ˧}}}}\kern2pt]} \hypertarget{q\string_h\{\string_Bb\{\string_B1}{}
\markboth{\textcolor{darkblue}{\textbf{\ipa{qʰæ˩bæ˩}}}}{}
\textcolor{teal}{\zh{名词}} \hspace{4pt} \zh{声调类:} L.
\zh{调羹。} \textcolor{Sepia}{\selectlanguage{english}Spoon, used for salt, tsamba... It corresponds to European teaspoons and tablespoons.} \textcolor{PineGreen}{\selectlanguage{french}Cuillère de petite taille: pour le sel, le tsamba... Elle correspond aux cuillères à café et cuillères à soupe du paradigme européen.}  \zh{量词}: \textcolor{darkblue}{\textbf{\ipa{nɑ˧}}} 
\lhead{\firstmark}
\rhead{\botmark}

\subsection{\hspace{-0.5cm} {\Large \textcolor{darkblue}{\textbf{\ipa{qʰæ˩ʈv̩˩ɻæ˥}}}}\hspace{0.5cm}[\kern2pt{\textcolor{darkblue}{\textbf{\ipa{qʰæ˩ʈv̩˩ɻæ˥}}}}\kern2pt]} \hypertarget{q\string_h\{\string_Bt`v\string_=\string_Br£`\{\string_T1}{}
\markboth{\textcolor{darkblue}{\textbf{\ipa{qʰæ˩ʈv̩˩ɻæ˥}}}}{}
\textcolor{teal}{\zh{形容词}} \hspace{4pt} \zh{声调类:} L+H\#.
\zh{安宁。} \textcolor{Sepia}{\selectlanguage{english}Quiet, peaceful.} \textcolor{PineGreen}{\selectlanguage{french}Serein.}  ¶ \textcolor{darkblue}{\textbf{\ipa{qʰæ˩ʈv̩˩ɻæ˥ | ɖɯ˧-dzi˩-ɻ̍˩}}} \zh{安静地坐一会} \textcolor{Sepia}{\selectlanguage{english}to sit quietly for a while} \textcolor{PineGreen}{\selectlanguage{french}être assis tranquille, dans le calme}  
 ¶ \textcolor{darkblue}{\textbf{\ipa{qʰæ˩ʈv̩˩ɻæ˥-gv̩˩}}} \zh{安宁地} \textcolor{Sepia}{\selectlanguage{english}peacefully} \textcolor{PineGreen}{\selectlanguage{french}tranquillement}  

\lhead{\firstmark}
\rhead{\botmark}

\subsection{\hspace{-0.5cm} {\Large \textcolor{darkblue}{\textbf{\ipa{qʰæ˧˥}}} \textsubscript{1}}\hspace{0.5cm}[\kern2pt{\textcolor{darkblue}{\textbf{\ipa{qʰæ˧˥}}}}\kern2pt]} \hypertarget{q\string_h\{\string_M\string_T1}{}
\markboth{\textcolor{darkblue}{\textbf{\ipa{qʰæ˧˥}}} \textsubscript{1}}{}
\textcolor{teal}{\zh{动词}} \hspace{4pt} \zh{声调类:} MH.
\zh{出来(月亮,太阳)。} \textcolor{Sepia}{\selectlanguage{english}To come out (moon, sun).} \textcolor{PineGreen}{\selectlanguage{french}Paraître, se lever (lune, soleil).}  ¶ \textcolor{darkblue}{\textbf{\ipa{tʰi˧-qʰæ˧-ze˥}}} \zh{\mytextsc{dur} \string_ \mytextsc{pfv}} \textcolor{Sepia}{\selectlanguage{english}\mytextsc{dur} \string_ \mytextsc{pfv}} \textcolor{PineGreen}{\selectlanguage{french}\mytextsc{dur} \string_ \mytextsc{pfv}}  

\lhead{\firstmark}
\rhead{\botmark}

\subsection{\hspace{-0.5cm} {\Large \textcolor{darkblue}{\textbf{\ipa{qʰæ˧˥}}} \textsubscript{2}}\hspace{0.5cm}[\kern2pt{\textcolor{darkblue}{\textbf{\ipa{qʰæ˧˥}}}}\kern2pt]} \hypertarget{q\string_h\{\string_M\string_T2}{}
\markboth{\textcolor{darkblue}{\textbf{\ipa{qʰæ˧˥}}} \textsubscript{2}}{}
\textcolor{teal}{\zh{动词}} \hspace{4pt} \zh{声调类:} MH.
\zh{拆。} \textcolor{Sepia}{\selectlanguage{english}To pull down, to dismantle.} \textcolor{PineGreen}{\selectlanguage{french}Démolir.}  ¶ \textcolor{darkblue}{\textbf{\ipa{ʑi˧qʰwɤ˧ qʰæ˧˥}}} \zh{拆房子} \textcolor{Sepia}{\selectlanguage{english}to demolish a house} \textcolor{PineGreen}{\selectlanguage{french}démolir une maison}  

\lhead{\firstmark}
\rhead{\botmark}

\subsection{\hspace{-0.5cm} {\Large \textcolor{darkblue}{\textbf{\ipa{qʰæ˧˥}}} \textsubscript{3}}\hspace{0.5cm}[\kern2pt{\textcolor{darkblue}{\textbf{\ipa{qʰæ˧˥}}}}\kern2pt]} \hypertarget{q\string_h\{\string_M\string_T3}{}
\markboth{\textcolor{darkblue}{\textbf{\ipa{qʰæ˧˥}}} \textsubscript{3}}{}
\textcolor{teal}{\zh{动词}} \hspace{4pt} \zh{声调类:} MH.
\zh{分东西、(大家)平分东西。} \textcolor{Sepia}{\selectlanguage{english}To share: several people share something among themselves; someone shares out something.} \textcolor{PineGreen}{\selectlanguage{french}Partager, répartir.} 
\lhead{\firstmark}
\rhead{\botmark}

\subsection{\hspace{-0.5cm} {\Large \textcolor{darkblue}{\textbf{\ipa{qʰæ˧˥}}} \textsubscript{4}}\hspace{0.5cm}[\kern2pt{\textcolor{darkblue}{\textbf{\ipa{qʰæ˧˥}}}}\kern2pt]} \hypertarget{q\string_h\{\string_M\string_T4}{}
\markboth{\textcolor{darkblue}{\textbf{\ipa{qʰæ˧˥}}} \textsubscript{4}}{}
\textcolor{teal}{\zh{动词}} \hspace{4pt} \zh{声调类:} MH.
\zh{开枪。} \textcolor{Sepia}{\selectlanguage{english}To shoot (with a gun).} \textcolor{PineGreen}{\selectlanguage{french}Tirer (avec une arme à feu, une arbalète; aussi avec un arc: tirer une flèche).}  ¶ \textcolor{darkblue}{\textbf{\ipa{le˧-qʰæ˧-ze˥}}} \zh{开枪了} \textcolor{Sepia}{\selectlanguage{english}\mytextsc{accomp} \string_ \mytextsc{pfv}} \textcolor{PineGreen}{\selectlanguage{french}\mytextsc{accomp} \string_ \mytextsc{pfv}}  
 ¶ \textcolor{darkblue}{\textbf{\ipa{mv̩˧ʐe˧ qʰæ˩(-ze˩)}}} \zh{开枪} \textcolor{Sepia}{\selectlanguage{english}to shoot with a gun} \textcolor{PineGreen}{\selectlanguage{french}tirer avec une arme à feu}  

\lhead{\firstmark}
\rhead{\botmark}

\subsection{\hspace{-0.5cm} {\Large \textcolor{darkblue}{\textbf{\ipa{qʰæ˧˥}}} \textsubscript{5}}\hspace{0.5cm}[\kern2pt{\textcolor{darkblue}{\textbf{\ipa{qʰæ˧˥}}}}\kern2pt]} \hypertarget{q\string_h\{\string_M\string_T5}{}
\markboth{\textcolor{darkblue}{\textbf{\ipa{qʰæ˧˥}}} \textsubscript{5}}{}
\textcolor{teal}{\zh{形容词}} \hspace{4pt} \zh{声调类:} MH.
\zh{幸福,安逸,平安。} \textcolor{Sepia}{\selectlanguage{english}Happy, content, peaceful, at peace.} \textcolor{PineGreen}{\selectlanguage{french}Heureux.}  ¶ \textcolor{darkblue}{\textbf{\ipa{le˧-qʰæ˧-ze˥}}} \zh{\mytextsc{accomp} \string_ \mytextsc{pfv}} \textcolor{Sepia}{\selectlanguage{english}\mytextsc{accomp} \string_ \mytextsc{pfv}} \textcolor{PineGreen}{\selectlanguage{french}\mytextsc{accomp} \string_ \mytextsc{pfv}}  
 ¶ \textcolor{darkblue}{\textbf{\ipa{lo˧ qʰæ˩}}} \zh{轻松工作} \textcolor{Sepia}{\selectlanguage{english}to work in a quiet, relaxed manner} \textcolor{PineGreen}{\selectlanguage{french}travailler de façon tranquille, détendue, paisible}  

\lhead{\firstmark}
\rhead{\botmark}

\subsection{\hspace{-0.5cm} {\Large \textcolor{darkblue}{\textbf{\ipa{qʰæ˧˥}}} \textsubscript{6}}\hspace{0.5cm}[\kern2pt{\textcolor{darkblue}{\textbf{\ipa{qʰæ˧˥}}}}\kern2pt]} \hypertarget{q\string_h\{\string_M\string_T6}{}
\markboth{\textcolor{darkblue}{\textbf{\ipa{qʰæ˧˥}}} \textsubscript{6}}{}
\textcolor{teal}{\zh{动词}} \hspace{4pt} \zh{声调类:} MH.
\zh{糊、变黑(高温让油、食物变黑,变糊了)。} \textcolor{Sepia}{\selectlanguage{english}To burn, to go brown: food or oil gets close to burning point (but remains edible).} \textcolor{PineGreen}{\selectlanguage{french}Brûler, griller, noircir: la graisse chauffée à feu vif noircit, fume et donne de l'acroléine; le riz devient sec, très/trop cuit. La nourriture reste comestible.}  ¶ \textcolor{darkblue}{\textbf{\ipa{le˧-qʰæ˧-ze˥}}} \zh{\mytextsc{accomp} \string_ \mytextsc{pfv}} \textcolor{Sepia}{\selectlanguage{english}\mytextsc{accomp} \string_ \mytextsc{pfv}} \textcolor{PineGreen}{\selectlanguage{french}\mytextsc{accomp} \string_ \mytextsc{pfv}}  
 ¶ \textcolor{darkblue}{\textbf{\ipa{mɤ˧ | le˧-qʰæ˧-ze˥}}} \zh{油焦了!} \textcolor{Sepia}{\selectlanguage{english}The oil has burned / has reached boiling point / has gone black!} \textcolor{PineGreen}{\selectlanguage{french}L'huile a noirci / l'huile est parvenue à une très haute température.}  
 ¶ \textcolor{darkblue}{\textbf{\ipa{hɑ˧ | le˧-qʰæ˧-ze˥}}} \zh{饭糊了。} \textcolor{Sepia}{\selectlanguage{english}The rice has burned / is overcooked.} \textcolor{PineGreen}{\selectlanguage{french}Le riz a brûlé. / Le riz est trop cuit.}  
 ¶ \textcolor{darkblue}{\textbf{\ipa{v̩˩tsʰɤ˩˥ | hṽ˧\textasciitilde{}hṽ˧ F | le˧-qʰæ˧-ze˥!}}} \zh{菜都炒糊了!} \textcolor{Sepia}{\selectlanguage{english}The vegetables are going brown / are overcooked / are getting burnt from frying!} \textcolor{PineGreen}{\selectlanguage{french}Les légumes, à force de frire, les voilà brûlés! / les voilà trop cuits!}  
 ¶ \textcolor{darkblue}{\textbf{\ipa{ʂe˧ | hṽ˧\textasciitilde{}hṽ˧ F | le˧-qʰæ˧-ze˥!}}} \zh{肉都炒焦了!} \textcolor{Sepia}{\selectlanguage{english}The meat is going brown / is overcooked / is getting burnt from frying!} \textcolor{PineGreen}{\selectlanguage{french}La viande, à force de frire, la voilà brûlée! / la voilà trop cuite!}  

\lhead{\firstmark}
\rhead{\botmark}

\subsection{\hspace{-0.5cm} {\Large \textcolor{darkblue}{\textbf{\ipa{qʰo˧\textsubscript{a}}}}}\hspace{0.5cm}[\kern2pt{\textcolor{darkblue}{\textbf{\ipa{qʰo˥}}}}\kern2pt]} \hypertarget{q\string_ho\string_Ma1}{}
\markboth{\textcolor{darkblue}{\textbf{\ipa{qʰo˧\textsubscript{a}}}}}{}
\textcolor{teal}{\zh{动词}} \hspace{4pt} \zh{声调类:} M\textsubscript{a}.
\zh{堆起来。} \textcolor{Sepia}{\selectlanguage{english}To pile up (e.g. stones).} \textcolor{PineGreen}{\selectlanguage{french}Empiler (par exemple des pierres).}  ¶ \textcolor{darkblue}{\textbf{\ipa{lv̩˧mi˧ tʰi˧-qʰo˧}}} \zh{石头堆起来} \textcolor{Sepia}{\selectlanguage{english}to pile up stones} \textcolor{PineGreen}{\selectlanguage{french}empiler des pierres}  

\lhead{\firstmark}
\rhead{\botmark}

\subsection{\hspace{-0.5cm} {\Large \textcolor{darkblue}{\textbf{\ipa{qʰo˧lo˧}}}}\hspace{0.5cm}[\kern2pt{\textcolor{darkblue}{\textbf{\ipa{qʰo˧lo˧}}}}\kern2pt]} \hypertarget{q\string_ho\string_Mlo\string_M1}{}
\markboth{\textcolor{darkblue}{\textbf{\ipa{qʰo˧lo˧}}}}{}
\textcolor{teal}{\zh{名词}} \hspace{4pt} \zh{声调类:} M.
\zh{轮子。} \textcolor{Sepia}{\selectlanguage{english}Wheel.} \textcolor{PineGreen}{\selectlanguage{french}Roue.}  \zh{量词}: \textcolor{darkblue}{\textbf{\ipa{ɭɯ˧}}} 
\lhead{\firstmark}
\rhead{\botmark}

\subsection{\hspace{-0.5cm} {\Large \textcolor{darkblue}{\textbf{\ipa{qʰo˧mo˥}}}}\hspace{0.5cm}[\kern2pt{\textcolor{darkblue}{\textbf{\ipa{qʰo˧mo˥}}}}\kern2pt]} \hypertarget{q\string_ho\string_Mmo\string_T1}{}
\markboth{\textcolor{darkblue}{\textbf{\ipa{qʰo˧mo˥}}}}{}
\textcolor{teal}{\zh{名词}} \hspace{4pt} \zh{声调类:} H\#.
\zh{老牛(不产奶了)。} \textcolor{Sepia}{\selectlanguage{english}Old cow (which does not give milk anymore).} \textcolor{PineGreen}{\selectlanguage{french}Vieille vache (qui n'a plus de lait).}  \zh{量词}: \textcolor{darkblue}{\textbf{\ipa{pʰo˧˥}}} 
\lhead{\firstmark}
\rhead{\botmark}

\subsection{\hspace{-0.5cm} {\Large \textcolor{darkblue}{\textbf{\ipa{qʰo˩\textsubscript{b}}}}}\hspace{0.5cm}[\kern2pt{\textcolor{darkblue}{\textbf{\ipa{qʰo˩˥}}}}\kern2pt]} \hypertarget{q\string_ho\string_Bb1}{}
\markboth{\textcolor{darkblue}{\textbf{\ipa{qʰo˩\textsubscript{b}}}}}{}
\textcolor{teal}{\zh{动词}} \hspace{4pt} \zh{声调类:} L\textsubscript{b}.
\zh{邀请、请。} \textcolor{Sepia}{\selectlanguage{english}To invite, to treat.} \textcolor{PineGreen}{\selectlanguage{french}Inviter.}  ¶ \textcolor{darkblue}{\textbf{\ipa{hĩ˧ qʰo˧˥}}} \zh{邀请人} \textcolor{Sepia}{\selectlanguage{english}to invite someone} \textcolor{PineGreen}{\selectlanguage{french}inviter quelqu'un}  
 ¶ \textcolor{darkblue}{\textbf{\ipa{hĩ˧bæ˧ qʰo˧˥}}} \zh{邀请客人} \textcolor{Sepia}{\selectlanguage{english}to invite a guest} \textcolor{PineGreen}{\selectlanguage{french}inviter un hôte, convier un invité}  
 ¶ \textcolor{darkblue}{\textbf{\ipa{hĩ˧bæ˧ | qʰo˧-zo˧-ho˥}}} \zh{需要请一下客人!} \textcolor{Sepia}{\selectlanguage{english}We should invite guests!} \textcolor{PineGreen}{\selectlanguage{french}Il va falloir inviter des hôtes!}  
 ¶ \textcolor{darkblue}{\textbf{\ipa{hĩ˧bæ˧ qʰo˧-di˧˥}}} \zh{‘待客的东西’(老鼠药的委婉语。如果说出来要买老鼠药,老鼠会知道,就不会吃的。)} \textcolor{Sepia}{\selectlanguage{english}Euphemism for 'rat poison'. This phrase is intended not to attract the mice's attention to these preparations.} \textcolor{PineGreen}{\selectlanguage{french}Euphémisme pour désigner la mort-aux-rats. La croyance veut que si on expose clairement le projet, les rats vont se méfier et ne prendront pas cette nourriture empoisonnée.}  
 ¶ \textcolor{darkblue}{\textbf{\ipa{ɖɯ˧-qʰo˥\textasciitilde{}qʰo˩-ɻ̍˩}}} \zh{\mytextsc{delimitative} \mytextsc{red} \mytextsc{inceptive:请一下}} \textcolor{Sepia}{\selectlanguage{english}\mytextsc{delimitative} \mytextsc{red} \mytextsc{inceptive}} \textcolor{PineGreen}{\selectlanguage{french}\mytextsc{délimitative} \mytextsc{red} \mytextsc{inchoatif}}  
 ¶ \textcolor{darkblue}{\textbf{\ipa{qʰo˩-mɤ˥-qʰo˩}}} \zh{请不请} \textcolor{Sepia}{\selectlanguage{english}to invite or not} \textcolor{PineGreen}{\selectlanguage{french}inviter ou ne pas inviter}  
 ¶ \textcolor{darkblue}{\textbf{\ipa{qʰo˩-mɤ˩-ho˥}}} \zh{不请了 / 不要请了} \textcolor{Sepia}{\selectlanguage{english}...will not invite} \textcolor{PineGreen}{\selectlanguage{french}... ne vais pas inviter}  

\lhead{\firstmark}
\rhead{\botmark}

\subsection{\hspace{-0.5cm} {\Large \textcolor{darkblue}{\textbf{\ipa{qʰo˩dv̩˧˥}}}}\hspace{0.5cm}[\kern2pt{\textcolor{darkblue}{\textbf{\ipa{qʰo˩dv̩˧˥}}}}\kern2pt]} \hypertarget{q\string_ho\string_Bdv\string_=\string_M\string_T1}{}
\markboth{\textcolor{darkblue}{\textbf{\ipa{qʰo˩dv̩˧˥}}}}{}
\textcolor{teal}{\zh{名词}} \hspace{4pt} \zh{声调类:} LM+MH\#.
\zh{大锤子。} \textcolor{Sepia}{\selectlanguage{english}Hammer; typically a large wood hammer.} \textcolor{PineGreen}{\selectlanguage{french}Masse, marteau de grande taille; typiquement: masse en bois utilisée pour défoncer les grosses mottes après les labours.}  ¶ \textcolor{darkblue}{\textbf{\ipa{ʂe˩-qʰo˩dv̩˧˥}}} \zh{铁锤子} \textcolor{Sepia}{\selectlanguage{english}iron hammer} \textcolor{PineGreen}{\selectlanguage{french}masse en fer}  
 \zh{量词}: \textcolor{darkblue}{\textbf{\ipa{ɭɯ˧}}} 
\lhead{\firstmark}
\rhead{\botmark}

\subsection{\hspace{-0.5cm} {\Large \textcolor{darkblue}{\textbf{\ipa{qʰo˩mv̩˩}}}}\hspace{0.5cm}[\kern2pt{\textcolor{darkblue}{\textbf{\ipa{qʰo˩mv̩˩˥}}}}\kern2pt]} \hypertarget{q\string_ho\string_Bmv\string_=\string_B1}{}
\markboth{\textcolor{darkblue}{\textbf{\ipa{qʰo˩mv̩˩}}}}{}
\textcolor{teal}{\zh{名词}} \hspace{4pt} \zh{声调类:} L.
\zh{斗笠。} \textcolor{Sepia}{\selectlanguage{english}Straw hat.} \textcolor{PineGreen}{\selectlanguage{french}Chapeau de paille.}  \zh{量词}: \textcolor{darkblue}{\textbf{\ipa{ɭɯ˧}}} 
\lhead{\firstmark}
\rhead{\botmark}

\subsection{\hspace{-0.5cm} {\Large \textcolor{darkblue}{\textbf{\ipa{qʰo˩tv̩˧˥}}}}\hspace{0.5cm}[\kern2pt{\textcolor{darkblue}{\textbf{\ipa{qʰo˩tv̩˧˥}}}}\kern2pt]} \hypertarget{q\string_ho\string_Btv\string_=\string_M\string_T1}{}
\markboth{\textcolor{darkblue}{\textbf{\ipa{qʰo˩tv̩˧˥}}}}{}
\textcolor{teal}{\zh{名词}} \hspace{4pt} \zh{声调类:} LM+MH\#.
\zh{树墩、树桩。} \textcolor{Sepia}{\selectlanguage{english}Tree stump.} \textcolor{PineGreen}{\selectlanguage{french}Souche.}  \zh{量词}: \textcolor{darkblue}{\textbf{\ipa{ɭɯ˧}}} 
\lhead{\firstmark}
\rhead{\botmark}

\subsection{\hspace{-0.5cm} {\Large \textcolor{darkblue}{\textbf{\ipa{qʰo˧˥}}} \textsubscript{1}}\hspace{0.5cm}[\kern2pt{\textcolor{darkblue}{\textbf{\ipa{qʰo˧˥}}}}\kern2pt]} \hypertarget{q\string_ho\string_M\string_T1}{}
\markboth{\textcolor{darkblue}{\textbf{\ipa{qʰo˧˥}}} \textsubscript{1}}{}
\textcolor{teal}{\zh{动词}} \hspace{4pt} \zh{声调类:} MH.
\zh{啄。} \textcolor{Sepia}{\selectlanguage{english}To peck.} \textcolor{PineGreen}{\selectlanguage{french}Picorer.}  ¶ \textcolor{darkblue}{\textbf{\ipa{hɑ˧ qʰo˩(-ze˩)}}} \zh{啄粮食} \textcolor{Sepia}{\selectlanguage{english}to peck cereals} \textcolor{PineGreen}{\selectlanguage{french}picorer des céréales}  
 ¶ \textcolor{darkblue}{\textbf{\ipa{hɑ˧ qʰo˥\textasciitilde{}qʰo˩ (-dʑo˩)}}} \zh{啄粮食} \textcolor{Sepia}{\selectlanguage{english}to peck cereals} \textcolor{PineGreen}{\selectlanguage{french}picorer des céréales}  
 ¶ \textcolor{darkblue}{\textbf{\ipa{æ˩-ɳɯ˥ | hɑ˧ qʰo˩}}} \zh{鸡在啄粮食} \textcolor{Sepia}{\selectlanguage{english}the chicken is pecking cereals} \textcolor{PineGreen}{\selectlanguage{french}la poule picore}  

\lhead{\firstmark}
\rhead{\botmark}

\subsection{\hspace{-0.5cm} {\Large \textcolor{darkblue}{\textbf{\ipa{qʰo˧˥}}} \textsubscript{2}}\hspace{0.5cm}[\kern2pt{\textcolor{darkblue}{\textbf{\ipa{qʰo˧˥}}}}\kern2pt]} \hypertarget{q\string_ho\string_M\string_T2}{}
\markboth{\textcolor{darkblue}{\textbf{\ipa{qʰo˧˥}}} \textsubscript{2}}{}
\textcolor{teal}{\zh{动词}} \hspace{4pt} \zh{声调类:} MH.
\zh{杀,宰牲畜。} \textcolor{Sepia}{\selectlanguage{english}To kill; to slaughter (an animal).} \textcolor{PineGreen}{\selectlanguage{french}Tuer; abattre un animal.}  ¶ \textcolor{darkblue}{\textbf{\ipa{bo˩ qʰo˧˥ / bo˩ qʰo˧-ze˥}}} \zh{杀猪} \textcolor{Sepia}{\selectlanguage{english}to slaughter a pig} \textcolor{PineGreen}{\selectlanguage{french}tuer le cochon}  
 ¶ \textcolor{darkblue}{\textbf{\ipa{bo˩˥ | le˧-qʰo˧-ze˥}}} \zh{杀了猪} \textcolor{Sepia}{\selectlanguage{english}the pig has been slaughtered} \textcolor{PineGreen}{\selectlanguage{french}le cochon a été abattu}  
 ¶ \textcolor{darkblue}{\textbf{\ipa{æ˩ qʰo˧˥}}} \zh{杀鸡} \textcolor{Sepia}{\selectlanguage{english}to kill a chicken} \textcolor{PineGreen}{\selectlanguage{french}tuer un poulet}  
 ¶ \textcolor{darkblue}{\textbf{\ipa{ʝi˧ qʰo˩}}} \zh{杀牛} \textcolor{Sepia}{\selectlanguage{english}to kill a cow} \textcolor{PineGreen}{\selectlanguage{french}tuer une vache}  

\lhead{\firstmark}
\rhead{\botmark}

\subsection{\hspace{-0.5cm} {\Large \textcolor{darkblue}{\textbf{\ipa{qʰv̩˧}}} \textsubscript{1}}\hspace{0.5cm}[\kern2pt{\textcolor{darkblue}{\textbf{\ipa{qʰv̩˥}}}}\kern2pt]} \hypertarget{q\string_hv\string_=\string_M1}{}
\markboth{\textcolor{darkblue}{\textbf{\ipa{qʰv̩˧}}} \textsubscript{1}}{}
\textcolor{teal}{\zh{名词}} \hspace{4pt} \zh{声调类:} M.
\ding{202} \zh{洞。} \textcolor{Sepia}{\selectlanguage{english}Hole.} \textcolor{PineGreen}{\selectlanguage{french}Trou.}  \zh{量词}: \textcolor{darkblue}{\textbf{\ipa{ɭɯ˧}}} \ding{203} \zh{野兽的洞穴、野兽的窝。} \textcolor{Sepia}{\selectlanguage{english}Burrow.} \textcolor{PineGreen}{\selectlanguage{french}Terrier.}  ¶ \textcolor{darkblue}{\textbf{\ipa{ɖɤ˧-qʰv̩˧}}} \zh{狐狸的窝} \zh{tone: M} \textcolor{Sepia}{\selectlanguage{english}fox burrow} \textcolor{PineGreen}{\selectlanguage{french}terrier de renard}  
 ¶ \textcolor{darkblue}{\textbf{\ipa{ʂwæ˧ qʰv̩˧}}} \zh{水獭的窝} \zh{tone: M} \textcolor{Sepia}{\selectlanguage{english}otter's burrow} \textcolor{PineGreen}{\selectlanguage{french}terrier de loutre}  

\lhead{\firstmark}
\rhead{\botmark}

\subsection{\hspace{-0.5cm} {\Large \textcolor{darkblue}{\textbf{\ipa{qʰv̩˧}}} \textsubscript{2}}\hspace{0.5cm}[\kern2pt{\textcolor{darkblue}{\textbf{\ipa{qʰv̩˥}}}}\kern2pt]} \hypertarget{q\string_hv\string_=\string_M2}{}
\markboth{\textcolor{darkblue}{\textbf{\ipa{qʰv̩˧}}} \textsubscript{2}}{}
\textcolor{teal}{\zh{名词}} \hspace{4pt} \zh{声调类:} M.
\zh{犄角(锯下来的)。} \textcolor{Sepia}{\selectlanguage{english}Horn.} \textcolor{PineGreen}{\selectlanguage{french}Corne.}  ¶ \textcolor{darkblue}{\textbf{\ipa{ʝi˧-qʰv̩\#˥}}} \zh{牛角(过去,用牛角来当饮料容器)} \textcolor{Sepia}{\selectlanguage{english}Ox horn. Ox horns are used as containers for drinking.} \textcolor{PineGreen}{\selectlanguage{french}Corne de boeuf. La corne de boeuf était autrefois utilisée comme récipient pour boissons.}  
 ¶ \textcolor{darkblue}{\textbf{\ipa{ʈʂʰæ˧-qʰv̩˥}}} \zh{鹿角} \textcolor{Sepia}{\selectlanguage{english}stag horn} \textcolor{PineGreen}{\selectlanguage{french}corne de cerf}  
 \zh{量词}: \textcolor{darkblue}{\textbf{\ipa{ɭɯ˧}}} \textcolor{darkblue}{\textbf{\ipa{dze˩}}} 
\lhead{\firstmark}
\rhead{\botmark}

\subsection{\hspace{-0.5cm} {\Large \textcolor{darkblue}{\textbf{\ipa{qʰv̩˧˥}}} \textsubscript{1}}\hspace{0.5cm}[\kern2pt{\textcolor{darkblue}{\textbf{\ipa{qʰv̩˧˥}}}}\kern2pt]} \hypertarget{q\string_hv\string_=\string_M\string_T1}{}
\markboth{\textcolor{darkblue}{\textbf{\ipa{qʰv̩˧˥}}} \textsubscript{1}}{}
\textcolor{teal}{\zh{动词}} \hspace{4pt} \zh{声调类:} MH.
\zh{蜷缩。} \textcolor{Sepia}{\selectlanguage{english}To huddle up, to curl up.} \textcolor{PineGreen}{\selectlanguage{french}Se recroqueviller.}  ¶ \textcolor{darkblue}{\textbf{\ipa{ɲi˧-qʰv̩˧˥ | tʰi˧-dzi˩}}} \zh{坐着身体缩成一团} \textcolor{Sepia}{\selectlanguage{english}to be seated, leaning forward, torso bent towards the thighs} \textcolor{PineGreen}{\selectlanguage{french}être assis penché en avant, le torse penché vers les cuisses}  
 ¶ \textcolor{darkblue}{\textbf{\ipa{[M23] ɲi˧-qʰv̩˧-ʝi˥ | tʰi˧-dzi˩}}} \zh{坐着身体缩成一团} \textcolor{Sepia}{\selectlanguage{english}to be seated, leaning forward, torso bent towards the thighs} \textcolor{PineGreen}{\selectlanguage{french}être assis penché en avant, le torse penché vers les cuisses}  

\lhead{\firstmark}
\rhead{\botmark}

\subsection{\hspace{-0.5cm} {\Large \textcolor{darkblue}{\textbf{\ipa{qʰv̩˧˥}}} \textsubscript{2}}\hspace{0.5cm}[\kern2pt{\textcolor{darkblue}{\textbf{\ipa{qʰv̩˧˥}}}}\kern2pt]} \hypertarget{q\string_hv\string_=\string_M\string_T2}{}
\markboth{\textcolor{darkblue}{\textbf{\ipa{qʰv̩˧˥}}} \textsubscript{2}}{}
\textcolor{teal}{\zh{数词}} \hspace{4pt} \zh{声调类:} MH.
\zh{6。} \textcolor{Sepia}{\selectlanguage{english}6.} \textcolor{PineGreen}{\selectlanguage{french}6.} 
\lhead{\firstmark}
\rhead{\botmark}

\subsection{\hspace{-0.5cm} {\Large \textcolor{darkblue}{\textbf{\ipa{qʰv̩˥}}}}\hspace{0.5cm}[\kern2pt{\textcolor{darkblue}{\textbf{\ipa{qʰv̩˥}}}}\kern2pt]} \hypertarget{q\string_hv\string_=\string_T1}{}
\markboth{\textcolor{darkblue}{\textbf{\ipa{qʰv̩˥}}}}{}
\textcolor{teal}{\zh{名词}} \hspace{4pt} \zh{声调类:} \#H.
\zh{声音。} \textcolor{Sepia}{\selectlanguage{english}Noise, sound.} \textcolor{PineGreen}{\selectlanguage{french}Son, bruit.}  ¶ \textcolor{darkblue}{\textbf{\ipa{ʈʂʰɯ˧ | ə˧tso˧ qʰv̩˧ ɲi˥?}}} \zh{这是什么声音?} \textcolor{Sepia}{\selectlanguage{english}What is this sound?} \textcolor{PineGreen}{\selectlanguage{french}c'est quoi ce bruit?}  
 \zh{量词}: \textcolor{darkblue}{\textbf{\ipa{kʰwɤ˥}}} 
\lhead{\firstmark}
\rhead{\botmark}

\subsection{\hspace{-0.5cm} {\Large \textcolor{darkblue}{\textbf{\ipa{qʰv̩˥\textsubscript{a}}}}}\hspace{0.5cm}[\kern2pt{\textcolor{darkblue}{\textbf{\ipa{qʰv̩˥}}}}\kern2pt]} \hypertarget{q\string_hv\string_=\string_Ta1}{}
\markboth{\textcolor{darkblue}{\textbf{\ipa{qʰv̩˥\textsubscript{a}}}}}{}
\textcolor{teal}{\zh{量词}} \hspace{4pt} \zh{声调类:} H\textsubscript{a}.
\zh{量词:村落。} \textcolor{Sepia}{\selectlanguage{english}Classifier for hamlets / small villages.} \textcolor{PineGreen}{\selectlanguage{french}Classificateur des hameaux.}  ¶ \textcolor{darkblue}{\textbf{\ipa{ŋwɤ˧-qʰv̩˧, | tsʰe˧ɲi˧-ʑi˩}}} \zh{五个村落,十二个家庭!(描写阿拉瓦村的情况)} \textcolor{Sepia}{\selectlanguage{english}Five hamlets, twelve families! (This formula summarizes the statistics of the village of /ə˧lɑ˧-ʁwɤ\#˥/)} \textcolor{PineGreen}{\selectlanguage{french}Cinq hameaux, douze familles! (Formule résumant la statistique du village de /ə˧lɑ˧-ʁwɤ\#˥/)}  

\lhead{\firstmark}
\rhead{\botmark}

\subsection{\hspace{-0.5cm} {\Large \textcolor{darkblue}{\textbf{\ipa{qʰv̩˩ɖɯ˩}}}}\hspace{0.5cm}[\kern2pt{\textcolor{darkblue}{\textbf{\ipa{qʰv̩˩ɖɯ˩˥}}}}\kern2pt]} \hypertarget{q\string_hv\string_=\string_Bd`M\string_B1}{}
\markboth{\textcolor{darkblue}{\textbf{\ipa{qʰv̩˩ɖɯ˩}}}}{}
\textcolor{teal}{\zh{名词}} \hspace{4pt} \zh{声调类:} L.
\zh{关心。} \textcolor{Sepia}{\selectlanguage{english}Attachment (to someone): found in the phrase 'to be attached to someone, to care for someone'.} \textcolor{PineGreen}{\selectlanguage{french}Attachement envers quelqu'un; observé seulement dans l'expression “être attaché à, faire cas de, attacher du prix à”.}  ¶ \textcolor{darkblue}{\textbf{\ipa{qʰv̩˩ɖɯ˩ pʰv̩˥}}} \zh{关心(一个人),重视(如:孩子重视父母)} \textcolor{Sepia}{\selectlanguage{english}to care for someone, to respect, to feel attachment to someone} \textcolor{PineGreen}{\selectlanguage{french}attacher de l'importance à, respecter, être attaché à (ex.: relation des enfants à leurs parents)}  

\lhead{\firstmark}
\rhead{\botmark}

\subsection{\hspace{-0.5cm} {\Large \textcolor{darkblue}{\textbf{\ipa{qʰv̩˩ɖʐæ˩}}}}\hspace{0.5cm}[\kern2pt{\textcolor{darkblue}{\textbf{\ipa{qʰv̩˩ɖʐæ˩˥}}}}\kern2pt]} \hypertarget{q\string_hv\string_=\string_Bd`z`\{\string_B1}{}
\markboth{\textcolor{darkblue}{\textbf{\ipa{qʰv̩˩ɖʐæ˩}}}}{}
\textcolor{teal}{\zh{名词}} \hspace{4pt} \zh{声调类:} L.
\zh{小绳子,细的绳子。} \textcolor{Sepia}{\selectlanguage{english}String; small rope.} \textcolor{PineGreen}{\selectlanguage{french}Cordelette, ficelle.}  ¶ \textcolor{darkblue}{\textbf{\ipa{qʰv̩˩ɖʐæ˩ ʈʂʰɯ˩-kʰɯ˥}}} \zh{一条细的绳子} \textcolor{Sepia}{\selectlanguage{english}\mytextsc{n}+\mytextsc{dem}+\mytextsc{clf}} \textcolor{PineGreen}{\selectlanguage{french}\mytextsc{n}+\mytextsc{dem}+\mytextsc{clf}}  
 \zh{量词}: \textcolor{darkblue}{\textbf{\ipa{kʰɯ˩}}} 
\lhead{\firstmark}
\rhead{\botmark}

\subsection{\hspace{-0.5cm} {\Large \textcolor{darkblue}{\textbf{\ipa{qʰv̩˧dʑɯ˥\$}}}}\hspace{0.5cm}[\kern2pt{\textcolor{darkblue}{\textbf{\ipa{qʰv̩˧dʑɯ˥}}}}\kern2pt]} \hypertarget{q\string_hv\string_=\string_Mdz£M\string_T\$1}{}
\markboth{\textcolor{darkblue}{\textbf{\ipa{qʰv̩˧dʑɯ˥\$}}}}{}
\textcolor{teal}{\zh{名词}} \hspace{4pt} \zh{声调类:} H\$.
\zh{窟窿。} \textcolor{Sepia}{\selectlanguage{english}Hole, cavity (e.g. mouse hole, or trap to catch large animals).} \textcolor{PineGreen}{\selectlanguage{french}Cavité, trou (ex.: trou de souris; ou piège où on fait tomber les animaux sauvages).}  ¶ \textcolor{darkblue}{\textbf{\ipa{hwæ˧tsɯ˥-qʰv̩˩dʑi˩}}} \zh{耗子洞} \textcolor{Sepia}{\selectlanguage{english}mousehole} \textcolor{PineGreen}{\selectlanguage{french}trou de souris}  
 ¶ \textcolor{darkblue}{\textbf{\ipa{qʰv̩˧dʑɯ˧ tsʰi˧ (-ze˩)}}} \zh{挖一个洞} \textcolor{Sepia}{\selectlanguage{english}to bore a hole} \textcolor{PineGreen}{\selectlanguage{french}percer un trou}  
 \zh{量词}: \textcolor{darkblue}{\textbf{\ipa{ɭɯ˧}}} 
\lhead{\firstmark}
\rhead{\botmark}

\subsection{\hspace{-0.5cm} {\Large \textcolor{darkblue}{\textbf{\ipa{qʰv̩˧ɬi˧mi\#˥}}}}\hspace{0.5cm}[\kern2pt{\textcolor{darkblue}{\textbf{\ipa{qʰv̩˧ɬi˧mi˧}}}}\kern2pt]} \hypertarget{q\string_hv\string_=\string_MKi\string_Mmi\#\string_T1}{}
\markboth{\textcolor{darkblue}{\textbf{\ipa{qʰv̩˧ɬi˧mi\#˥}}}}{}
\textcolor{teal}{\zh{名词}} \hspace{4pt} \zh{声调类:} \#H.
\zh{六月。} \textcolor{Sepia}{\selectlanguage{english}6th month.} \textcolor{PineGreen}{\selectlanguage{french}6e mois.} 
\lhead{\firstmark}
\rhead{\botmark}

\subsection{\hspace{-0.5cm} {\Large \textcolor{darkblue}{\textbf{\ipa{qʰv̩˩tsʰi˧˥}}}}\hspace{0.5cm}[\kern2pt{\textcolor{darkblue}{\textbf{\ipa{qʰv̩˩tsʰi˧˥}}}}\kern2pt]} \hypertarget{q\string_hv\string_=\string_Bts\string_hi\string_M\string_T1}{}
\markboth{\textcolor{darkblue}{\textbf{\ipa{qʰv̩˩tsʰi˧˥}}}}{}
\textcolor{teal}{\zh{数词}} \hspace{4pt} \zh{声调类:} LM+MH\#.
\zh{60。} \textcolor{Sepia}{\selectlanguage{english}60.} \textcolor{PineGreen}{\selectlanguage{french}60.} 
\lhead{\firstmark}
\rhead{\botmark}

\subsection{\hspace{-0.5cm} {\Large \textcolor{darkblue}{\textbf{\ipa{qʰv̩˧tʰv̩\#˥}}}}\hspace{0.5cm}[\kern2pt{\textcolor{darkblue}{\textbf{\ipa{qʰv̩˧tʰv̩˧}}}}\kern2pt]} \hypertarget{q\string_hv\string_=\string_Mt\string_hv\string_=\#\string_T1}{}
\markboth{\textcolor{darkblue}{\textbf{\ipa{qʰv̩˧tʰv̩\#˥}}}}{}
\textcolor{teal}{\zh{量词}} \hspace{4pt} \zh{声调类:} \#H.
\zh{量词:一个牛角的容量。} \textcolor{Sepia}{\selectlanguage{english}Classifier: a hornful. The quantity of liquid (or powder) that can be contained in an ox's horn. Ox horns used to serve as containers for water.} \textcolor{PineGreen}{\selectlanguage{french}Classificateur: quantité de liquide (ou de poudre) que tient une corne de vache.}  ¶ \textcolor{darkblue}{\textbf{\ipa{ɖɯ˧-qʰv̩˧tʰv̩\#˥, | ɲi˧-qʰv̩˧tʰv̩\#˥, | so˩-qʰv̩˩tʰv̩˩˥, | ʐv̩˧-qʰv̩˧tʰv̩\#˥, | ŋwɤ˧-qʰv̩˧tʰv̩\#˥, | qʰv̩˧-qʰv̩˧tʰv̩\#˥, | ʂɯ˧-qʰv̩˧tʰv̩\#˥, | hõ˧-qʰv̩˧tʰv̩\#˥, | gv̩˧-qʰv̩˧tʰv̩\#˥, | tsʰe˩-qʰv̩˩tʰv̩˩˥}}} \zh{与数词结合,一至十} \textcolor{Sepia}{\selectlanguage{english}association with numerals from 1 to 10} \textcolor{PineGreen}{\selectlanguage{french}association avec des numéraux, de 1 à 10}  
\zh{~【参考】~} \hyperlink{}{\textcolor{darkblue}{\textbf{\ipa{qʰv̩˧tʰv˥\$}}}} 
\lhead{\firstmark}
\rhead{\botmark}

\subsection{\hspace{-0.5cm} {\Large \textcolor{darkblue}{\textbf{\ipa{qʰv̩˧tʰv˥\$}}}}\hspace{0.5cm}[\kern2pt{\textcolor{darkblue}{\textbf{\ipa{qʰv̩˧tʰv˥}}}}\kern2pt]} \hypertarget{q\string_hv\string_=\string_Mt\string_hv\string_T\$1}{}
\markboth{\textcolor{darkblue}{\textbf{\ipa{qʰv̩˧tʰv˥\$}}}}{}
\textcolor{teal}{\zh{名词}} \hspace{4pt} \zh{声调类:} H\$.
\zh{(牛)角。} \textcolor{Sepia}{\selectlanguage{english}Horn.} \textcolor{PineGreen}{\selectlanguage{french}Corne (de vache).}  ¶ \textcolor{darkblue}{\textbf{\ipa{qʰv˧tʰv˥ | ɖɯ˧-ɭɯ˧}}} \zh{一个角} \textcolor{Sepia}{\selectlanguage{english}a horn} \textcolor{PineGreen}{\selectlanguage{french}une corne}  
 ¶ \textcolor{darkblue}{\textbf{\ipa{qʰv˧tʰv˧ ɲi˥}}} \zh{是(牛)角。} \textcolor{Sepia}{\selectlanguage{english}It's a horn.} \textcolor{PineGreen}{\selectlanguage{french}C'est une corne.}  
 \zh{量词}: \textcolor{darkblue}{\textbf{\ipa{ɭɯ˧}}} \zh{~【参考】~} \hyperlink{}{\textcolor{darkblue}{\textbf{\ipa{qʰv̩˧tʰv̩\#˥}}}} 
\lhead{\firstmark}
\rhead{\botmark}

\subsection{\hspace{-0.5cm} {\Large \textcolor{darkblue}{\textbf{\ipa{qʰv̩˩\textasciitilde{}qʰv̩˧˥}}}}\hspace{0.5cm}[\kern2pt{\textcolor{darkblue}{\textbf{\ipa{qʰv̩˧qʰv̩˧˥}}}}\kern2pt]} \hypertarget{q\string_hv\string_=\string_B~q\string_hv\string_=\string_M\string_T1}{}
\markboth{\textcolor{darkblue}{\textbf{\ipa{qʰv̩˩\textasciitilde{}qʰv̩˧˥}}}}{}
\textcolor{teal}{\zh{动词}} \hspace{4pt} \zh{声调类:} MH.
\zh{折叠、裹起来。} \textcolor{Sepia}{\selectlanguage{english}To fold (clothes).} \textcolor{PineGreen}{\selectlanguage{french}Plier (vêtements).}  ¶ \textcolor{darkblue}{\textbf{\ipa{qʰv̩˩\textasciitilde{}qʰv̩˧-ze˥}}} \zh{折起来了} \textcolor{Sepia}{\selectlanguage{english}\mytextsc{pfv}} \textcolor{PineGreen}{\selectlanguage{french}\mytextsc{pfv}}  
 ¶ \textcolor{darkblue}{\textbf{\ipa{le˧-qʰv̩˩\textasciitilde{}qʰv̩˩}}} \zh{\mytextsc{accomp}} \textcolor{Sepia}{\selectlanguage{english}\mytextsc{accomp}} \textcolor{PineGreen}{\selectlanguage{french}\mytextsc{accomp}}  

\lhead{\firstmark}
\rhead{\botmark}

\subsection{\hspace{-0.5cm} {\Large \textcolor{darkblue}{\textbf{\ipa{qʰwæ˧}}}}\hspace{0.5cm}[\kern2pt{\textcolor{darkblue}{\textbf{\ipa{qʰwæ˥}}}}\kern2pt]} \hypertarget{q\string_hw\{\string_M1}{}
\markboth{\textcolor{darkblue}{\textbf{\ipa{qʰwæ˧}}}}{}
\textcolor{teal}{\zh{名词}} \hspace{4pt} \zh{声调类:} M.
\zh{信息,信。} \textcolor{Sepia}{\selectlanguage{english}Message (monosyllable).} \textcolor{PineGreen}{\selectlanguage{french}Lettre, message, parole/récit.}  ¶ \textcolor{darkblue}{\textbf{\ipa{qʰwæ˧ po˧˥}}} \zh{带信息、传信息,传一封信} \textcolor{Sepia}{\selectlanguage{english}to carry a letter; to convey a message} \textcolor{PineGreen}{\selectlanguage{french}envoyer une lettre}  
 ¶ \textcolor{darkblue}{\textbf{\ipa{qʰwæ˧ kʰwɤ˧˥}}} \zh{互相通信息、有联系(两个人互相通信息)} \textcolor{Sepia}{\selectlanguage{english}to be in touch (with someone)} \textcolor{PineGreen}{\selectlanguage{french}être en contact, être en correspondance; être en relation}  
 ¶ \textcolor{darkblue}{\textbf{\ipa{dɑ˧pɤ˧-qʰwæ\#˥}}} \zh{达巴的故事} \textcolor{Sepia}{\selectlanguage{english}the tales of the \textcolor{darkblue}{\textbf{\ipa{/dɑ˧pɤ˧/}}} priests} \textcolor{PineGreen}{\selectlanguage{french}les récits des prêtres \textcolor{darkblue}{\textbf{\ipa{/dɑ˧pɤ˧/}}}}  
 \zh{量词}: \textcolor{darkblue}{\textbf{\ipa{kʰwɤ˥}}} 
\lhead{\firstmark}
\rhead{\botmark}

\subsection{\hspace{-0.5cm} {\Large \textcolor{darkblue}{\textbf{\ipa{qʰwæ˧kʰwɤ\#˥}}}}\hspace{0.5cm}[\kern2pt{\textcolor{darkblue}{\textbf{\ipa{qʰwæ˩kʰwɤ˩˥}}}}\kern2pt]} \hypertarget{q\string_hw\{\string_Mk\string_hw7\#\string_T1}{}
\markboth{\textcolor{darkblue}{\textbf{\ipa{qʰwæ˧kʰwɤ\#˥}}}}{}
\textcolor{teal}{\zh{名词}} \hspace{4pt} \zh{声调类:} \#H.
\zh{闲话、流言、蜚语、闲言碎语、八卦。} \textcolor{Sepia}{\selectlanguage{english}Gossip, idle chatter.} \textcolor{PineGreen}{\selectlanguage{french}Récit, racontar, ragot, histoire.}  ¶ \textcolor{darkblue}{\textbf{\ipa{ɖɯ˧-zɯ˧ qʰwæ˧kʰwɤ˧}}} \zh{讲一点八卦} \textcolor{Sepia}{\selectlanguage{english}to tell a piece of gossip} \textcolor{PineGreen}{\selectlanguage{french}raconter un petit racontar, rapporter un petit ragot}  
 \zh{量词}: \textcolor{darkblue}{\textbf{\ipa{kʰwɤ˥}}} 
\lhead{\firstmark}
\rhead{\botmark}

\subsection{\hspace{-0.5cm} {\Large \textcolor{darkblue}{\textbf{\ipa{qʰwæ˧ɭɯ˧}}}}\hspace{0.5cm}[\kern2pt{\textcolor{darkblue}{\textbf{\ipa{qʰwæ˧ɭɯ˧˥}}}}\kern2pt]} \hypertarget{q\string_hw\{\string_Ml\string_RM\string_M1}{}
\markboth{\textcolor{darkblue}{\textbf{\ipa{qʰwæ˧ɭɯ˧}}}}{}
\textcolor{teal}{\zh{名词}} \hspace{4pt} \zh{声调类:} M.
\zh{菜园。} \textcolor{Sepia}{\selectlanguage{english}Vegetable garden.} \textcolor{PineGreen}{\selectlanguage{french}Potager.}  \zh{量词}: \textcolor{darkblue}{\textbf{\ipa{kɤ˧˥}}} 
\lhead{\firstmark}
\rhead{\botmark}

\subsection{\hspace{-0.5cm} {\Large \textcolor{darkblue}{\textbf{\ipa{qʰwæ˧mi\#˥}}}}\hspace{0.5cm}[\kern2pt{\textcolor{darkblue}{\textbf{\ipa{qʰwæ˧mi˧}}}}\kern2pt]} \hypertarget{q\string_hw\{\string_Mmi\#\string_T1}{}
\markboth{\textcolor{darkblue}{\textbf{\ipa{qʰwæ˧mi\#˥}}}}{}
\textcolor{teal}{\zh{名词}} \hspace{4pt} \zh{声调类:} \#H.
\zh{口信, 信息。} \textcolor{Sepia}{\selectlanguage{english}Message, information (extended meaning: letter).} \textcolor{PineGreen}{\selectlanguage{french}Message, information (d'où: lettre).}  ¶ \textcolor{darkblue}{\textbf{\ipa{qʰwæ˧mi˧ ʝi˧}}} \zh{带一个口信} \textcolor{Sepia}{\selectlanguage{english}to carry a message} \textcolor{PineGreen}{\selectlanguage{french}porter un message}  
 \zh{量词}: \textcolor{darkblue}{\textbf{\ipa{kʰwɤ˥}}} 
\lhead{\firstmark}
\rhead{\botmark}

\subsection{\hspace{-0.5cm} {\Large \textcolor{darkblue}{\textbf{\ipa{qʰwæ˧ʈɯ˥}}}}\hspace{0.5cm}[\kern2pt{\textcolor{darkblue}{\textbf{\ipa{qʰwæ˩ʈɯ˧˥}}}}\kern2pt]} \hypertarget{q\string_hw\{\string_Mt`M\string_T1}{}
\markboth{\textcolor{darkblue}{\textbf{\ipa{qʰwæ˧ʈɯ˥}}}}{}
\textcolor{teal}{\zh{名词}} \hspace{4pt} \zh{声调类:} H\#.
\zh{头帕。} \textcolor{Sepia}{\selectlanguage{english}Scarf, kerchief.} \textcolor{PineGreen}{\selectlanguage{french}Fichu (tissu qu'on porte sur la tête).}  \zh{量词}: \textcolor{darkblue}{\textbf{\ipa{bɤ˧˥}}} 
\lhead{\firstmark}
\rhead{\botmark}

\subsection{\hspace{-0.5cm} {\Large \textcolor{darkblue}{\textbf{\ipa{qʰwæ˩}}}}\hspace{0.5cm}[\kern2pt{\textcolor{darkblue}{\textbf{\ipa{qʰwæ˥}}}}\kern2pt]} \hypertarget{q\string_hw\{\string_B1}{}
\markboth{\textcolor{darkblue}{\textbf{\ipa{qʰwæ˩}}}}{}
\textcolor{teal}{\zh{名词}} \hspace{4pt} \zh{声调类:} L.
\zh{篱笆。} \textcolor{Sepia}{\selectlanguage{english}Fence, made of bamboo or of thorny shrub branches.} \textcolor{PineGreen}{\selectlanguage{french}Haie, faite de bambou ou de broussailles épineuses.}  \zh{量词}: \textcolor{darkblue}{\textbf{\ipa{kɤ˧˥}}} 
\lhead{\firstmark}
\rhead{\botmark}

\subsection{\hspace{-0.5cm} {\Large \textcolor{darkblue}{\textbf{\ipa{qʰwæ˩\textsubscript{a}}}}}\hspace{0.5cm}[\kern2pt{\textcolor{darkblue}{\textbf{\ipa{qʰwæ˧˥}}}}\kern2pt]} \hypertarget{q\string_hw\{\string_Ba1}{}
\markboth{\textcolor{darkblue}{\textbf{\ipa{qʰwæ˩\textsubscript{a}}}}}{}
\textcolor{teal}{\zh{动词}} \hspace{4pt} \zh{声调类:} L\textsubscript{a}.
\zh{挡住。} \textcolor{Sepia}{\selectlanguage{english}To block.} \textcolor{PineGreen}{\selectlanguage{french}Bloquer.} 
\lhead{\firstmark}
\rhead{\botmark}

\subsection{\hspace{-0.5cm} {\Large \textcolor{darkblue}{\textbf{\ipa{qʰwæ˩kɤ˩}}}}\hspace{0.5cm}[\kern2pt{\textcolor{darkblue}{\textbf{\ipa{qʰwæ˩kɤ˩˥}}}}\kern2pt]} \hypertarget{q\string_hw\{\string_Bk7\string_B1}{}
\markboth{\textcolor{darkblue}{\textbf{\ipa{qʰwæ˩kɤ˩}}}}{}
\textcolor{teal}{\zh{名词}} \hspace{4pt} \zh{声调类:} L.
\zh{一种灌木,1.5至2米高,可以当篱笆用。} \textcolor{Sepia}{\selectlanguage{english}A sort of shrub, reaching 1.5 to 2 meters in height.} \textcolor{PineGreen}{\selectlanguage{french}Une sorte d'arbuste d'environ 1 mètre 50 à 2 mètres de haut.}  ¶ \textcolor{darkblue}{\textbf{\ipa{qʰwæ˩kɤ˩-dzi˩˥}}} \zh{同上} \textcolor{Sepia}{\selectlanguage{english}same meaning} \textcolor{PineGreen}{\selectlanguage{french}même sens}  
 \zh{量词}: \textcolor{darkblue}{\textbf{\ipa{dzi˩, ʝi˧}}} 
\lhead{\firstmark}
\rhead{\botmark}

\subsection{\hspace{-0.5cm} {\Large \textcolor{darkblue}{\textbf{\ipa{qʰwæ˧˥}}} \textsubscript{1}}\hspace{0.5cm}[\kern2pt{\textcolor{darkblue}{\textbf{\ipa{qʰwæ˥}}}}\kern2pt]} \hypertarget{q\string_hw\{\string_M\string_T1}{}
\markboth{\textcolor{darkblue}{\textbf{\ipa{qʰwæ˧˥}}} \textsubscript{1}}{}
\textcolor{teal}{\zh{动词}} \hspace{4pt} \zh{声调类:} MH.
\zh{弄碎。} \textcolor{Sepia}{\selectlanguage{english}To break (bowl, jar), to crack (nuts).} \textcolor{PineGreen}{\selectlanguage{french}Briser (verre, vaisselle…), faire éclater; casser (des noix).}  ¶ \textcolor{darkblue}{\textbf{\ipa{ʁo˧do˧ qʰwæ˧˥}}} \zh{敲开坚果(在永宁,不用夹子:用锤子敲开)} \textcolor{Sepia}{\selectlanguage{english}to crack walnuts} \textcolor{PineGreen}{\selectlanguage{french}casser des noix}  

\lhead{\firstmark}
\rhead{\botmark}

\subsection{\hspace{-0.5cm} {\Large \textcolor{darkblue}{\textbf{\ipa{qʰwæ˧˥}}} \textsubscript{2}}\hspace{0.5cm}[\kern2pt{\textcolor{darkblue}{\textbf{\ipa{qʰwæ˧˥}}}}\kern2pt]} \hypertarget{q\string_hw\{\string_M\string_T2}{}
\markboth{\textcolor{darkblue}{\textbf{\ipa{qʰwæ˧˥}}} \textsubscript{2}}{}
\textcolor{teal}{\zh{动词}} \hspace{4pt} \zh{声调类:} MH.
\zh{掴、打。} \textcolor{Sepia}{\selectlanguage{english}To slap.} \textcolor{PineGreen}{\selectlanguage{french}Gifler.}  ¶ \textcolor{darkblue}{\textbf{\ipa{le˧-qʰwæ˧-ze˥}}} \zh{掴了} \textcolor{Sepia}{\selectlanguage{english}\mytextsc{accomp} \string_ \mytextsc{pfv}} \textcolor{PineGreen}{\selectlanguage{french}\mytextsc{accomp} \string_ \mytextsc{pfv}}  
 ¶ \textcolor{darkblue}{\textbf{\ipa{zɯ˧ɻ̍˧ qʰwæ˩}}} \zh{打嘴巴} \textcolor{Sepia}{\selectlanguage{english}to slap/smack someone's cheek} \textcolor{PineGreen}{\selectlanguage{french}gifler}  
 ¶ \textcolor{darkblue}{\textbf{\ipa{zɯ˧ɻ̍˧ | ɖɯ˧-ɭɯ˧ | tʰi˧-qʰwæ˧-bi˥!}}} \zh{我要打嘴巴了!(对孩子说)} \textcolor{Sepia}{\selectlanguage{english}I'm going to slap your cheek! (Said by an adult to a child)} \textcolor{PineGreen}{\selectlanguage{french}Je vais te gifler! / Je vais te flanquer une gifle! (A un enfant)}  

\lhead{\firstmark}
\rhead{\botmark}

\subsection{\hspace{-0.5cm} {\Large \textcolor{darkblue}{\textbf{\ipa{qʰwæ˧˥\textsubscript{a}}}}}\hspace{0.5cm}[\kern2pt{\textcolor{darkblue}{\textbf{\ipa{qʰwæ˩˥}}}}\kern2pt]} \hypertarget{q\string_hw\{\string_M\string_Ta1}{}
\markboth{\textcolor{darkblue}{\textbf{\ipa{qʰwæ˧˥\textsubscript{a}}}}}{}
\textcolor{teal}{\zh{量词}} \hspace{4pt} \zh{声调类:} MH\textsubscript{a}.
\zh{量词:丝,如纺之前的麻丝(一根)。} \textcolor{Sepia}{\selectlanguage{english}Classifier for filaments of hemp before spinning.} \textcolor{PineGreen}{\selectlanguage{french}Classificateur des filaments de chanvre avant filage.} 
\lhead{\firstmark}
\rhead{\botmark}

\subsection{\hspace{-0.5cm} {\Large \textcolor{darkblue}{\textbf{\ipa{qʰwɤ˧}}}}\hspace{0.5cm}[\kern2pt{\textcolor{darkblue}{\textbf{\ipa{qʰwɤ˥}}}}\kern2pt]} \hypertarget{q\string_hw7\string_M1}{}
\markboth{\textcolor{darkblue}{\textbf{\ipa{qʰwɤ˧}}}}{}
\textcolor{teal}{\zh{名词}} \hspace{4pt} \zh{声调类:} M.
\zh{痕迹。} \textcolor{Sepia}{\selectlanguage{english}Traces, track (left by an animal).} \textcolor{PineGreen}{\selectlanguage{french}Traces, piste (d'un animal; lorsqu'on chasse, on suit la piste d'un animal, on le suit à la trace).}  \zh{量词}: \textcolor{darkblue}{\textbf{\ipa{pʰo˧˥}}} 
\lhead{\firstmark}
\rhead{\botmark}

\subsection{\hspace{-0.5cm} {\Large \textcolor{darkblue}{\textbf{\ipa{qʰwɤ˧\textsubscript{a}}}}}\hspace{0.5cm}[\kern2pt{\textcolor{darkblue}{\textbf{\ipa{qʰwɤ˥}}}}\kern2pt]} \hypertarget{q\string_hw7\string_Ma1}{}
\markboth{\textcolor{darkblue}{\textbf{\ipa{qʰwɤ˧\textsubscript{a}}}}}{}
\textcolor{teal}{\zh{动词}} \hspace{4pt} \zh{声调类:} M\textsubscript{a}.
\zh{治好(骨折、病)。} \textcolor{Sepia}{\selectlanguage{english}To heal (wound, disease, broken bone...).} \textcolor{PineGreen}{\selectlanguage{french}Se guérir (blessure, maladie); se rétablir (une fracture).}  ¶ \textcolor{darkblue}{\textbf{\ipa{le˧-qʰwɤ˧-ɲi˥!}}} \zh{治好了!} \textcolor{Sepia}{\selectlanguage{english}It is healed! / It has healed!} \textcolor{PineGreen}{\selectlanguage{french}C'est guéri! / La fracture est rétablie!}  

\lhead{\firstmark}
\rhead{\botmark}

\subsection{\hspace{-0.5cm} {\Large \textcolor{darkblue}{\textbf{\ipa{qʰwɤ˧bi˩}}}}\hspace{0.5cm}[\kern2pt{\textcolor{darkblue}{\textbf{\ipa{qʰwɤ˧bi˩}}}}\kern2pt]} \hypertarget{q\string_hw7\string_Mbi\string_B1}{}
\markboth{\textcolor{darkblue}{\textbf{\ipa{qʰwɤ˧bi˩}}}}{}
\textcolor{teal}{\zh{名词}} \hspace{4pt} \zh{声调类:} L\#.
\ding{202} \zh{马蹄、马的脚。} \textcolor{Sepia}{\selectlanguage{english}Hoof (of horse); foot (of dog).} \textcolor{PineGreen}{\selectlanguage{french}Sabot, patte.}  ¶ \textcolor{darkblue}{\textbf{\ipa{ʐwæ˧-qʰwɤ˧bi˥\#}}} \zh{马蹄、(马、狗……的)脚} \textcolor{Sepia}{\selectlanguage{english}horse hoof} \textcolor{PineGreen}{\selectlanguage{french}sabot de cheval}  
 ¶ \textcolor{darkblue}{\textbf{\ipa{kʰv̩˩-qʰwɤ˩bi˥\#}}} \zh{狗脚} \textcolor{Sepia}{\selectlanguage{english}dog's foot} \textcolor{PineGreen}{\selectlanguage{french}patte de chien}  
 \zh{量词}: \textcolor{darkblue}{\textbf{\ipa{bi˩}}} \textcolor{darkblue}{\textbf{\ipa{tʰv̩˧˥}}} \ding{203} \zh{动物脚的痕迹、行径。} \textcolor{Sepia}{\selectlanguage{english}Track, trail, spoor, footprints (of an animal).} \textcolor{PineGreen}{\selectlanguage{french}Traces, piste (d'un animal).} 
\lhead{\firstmark}
\rhead{\botmark}

\subsection{\hspace{-0.5cm} {\Large \textcolor{darkblue}{\textbf{\ipa{qʰwɤ˧mi˥\$}}}}\hspace{0.5cm}[\kern2pt{\textcolor{darkblue}{\textbf{\ipa{qʰwɤ˧mi˥}}}}\kern2pt]} \hypertarget{q\string_hw7\string_Mmi\string_T\$1}{}
\markboth{\textcolor{darkblue}{\textbf{\ipa{qʰwɤ˧mi˥\$}}}}{}
\textcolor{teal}{\zh{名词}} \hspace{4pt} \zh{声调类:} H\$.
\zh{大碗(以前碗是用木头做的)。} \textcolor{Sepia}{\selectlanguage{english}Large bowl; it used to be made of wood.} \textcolor{PineGreen}{\selectlanguage{french}Grand bol (autrefois, les bols étaient en bois).} \zh{~【参考】~} \hyperlink{}{\textcolor{darkblue}{\textbf{\ipa{qʰwɤ˧pɤ˥\$}}}} 
\lhead{\firstmark}
\rhead{\botmark}

\subsection{\hspace{-0.5cm} {\Large \textcolor{darkblue}{\textbf{\ipa{qʰwɤ˧pɤ˥\$}}}}\hspace{0.5cm}[\kern2pt{\textcolor{darkblue}{\textbf{\ipa{qʰwɤ˧pɤ˥}}}}\kern2pt]} \hypertarget{q\string_hw7\string_Mp7\string_T\$1}{}
\markboth{\textcolor{darkblue}{\textbf{\ipa{qʰwɤ˧pɤ˥\$}}}}{}
\textcolor{teal}{\zh{名词}} \hspace{4pt} \zh{声调类:} H\$.
\zh{大碗。} \textcolor{Sepia}{\selectlanguage{english}Large bowl.} \textcolor{PineGreen}{\selectlanguage{french}Grand bol.} \zh{~【参考】~} \hyperlink{}{\textcolor{darkblue}{\textbf{\ipa{qʰwɤ˧mi˥\$}}}} 
\lhead{\firstmark}
\rhead{\botmark}

\subsection{\hspace{-0.5cm} {\Large \textcolor{darkblue}{\textbf{\ipa{qʰwɤ˧ʂe˩}}}}\hspace{0.5cm}[\kern2pt{\textcolor{darkblue}{\textbf{\ipa{qʰwɤ˧ʂe˩}}}}\kern2pt]} \hypertarget{q\string_hw7\string_Ms`e\string_B1}{}
\markboth{\textcolor{darkblue}{\textbf{\ipa{qʰwɤ˧ʂe˩}}}}{}
\textcolor{teal}{\zh{名词}} \hspace{4pt} \zh{声调类:} L\#.
\zh{马蹄铁。} \textcolor{Sepia}{\selectlanguage{english}Horseshoe.} \textcolor{PineGreen}{\selectlanguage{french}Fer à cheval.}  ¶ \textcolor{darkblue}{\textbf{\ipa{ʐwæ˧-qʰwɤ˧ʂe˥ (+ɲi˩)}}} \zh{马蹄铁} \textcolor{Sepia}{\selectlanguage{english}horseshoe} \textcolor{PineGreen}{\selectlanguage{french}fer à cheval}  
 \zh{量词}: \textcolor{darkblue}{\textbf{\ipa{nɑ˧}}} \textcolor{darkblue}{\textbf{\ipa{pʰo˧˥}}} 
\lhead{\firstmark}
\rhead{\botmark}

\subsection{\hspace{-0.5cm} {\Large \textcolor{darkblue}{\textbf{\ipa{qʰwɤ˧to˩}}}}\hspace{0.5cm}[\kern2pt{\textcolor{darkblue}{\textbf{\ipa{qʰwɤ˧to˩}}}}\kern2pt]} \hypertarget{q\string_hw7\string_Mto\string_B1}{}
\markboth{\textcolor{darkblue}{\textbf{\ipa{qʰwɤ˧to˩}}}}{}
\textcolor{teal}{\zh{名词}} \hspace{4pt} \zh{声调类:} L\#.
\zh{肩膀的末端。} \textcolor{Sepia}{\selectlanguage{english}Tip of the shoulder.} \textcolor{PineGreen}{\selectlanguage{french}Épaule (extrémité de l'épaule, bout de l'épaule).}  ¶ \textcolor{darkblue}{\textbf{\ipa{hĩ˧ ʈʂʰɯ˧-v̩˧, | qʰwɤ˧to˩ | ɖɯ˧-pi˧˥ | ʂwæ˧-hṽ˩-di˩!}}} \zh{这个人的肩膀不正,一高一低!} \textcolor{Sepia}{\selectlanguage{english}This person's shoulders are not quite straight! / His/her shoulders don't align!} \textcolor{PineGreen}{\selectlanguage{french}Ce type, il a les épaules un peu de travers! / Il a une épaule plus haute que l'autre!}  
 \zh{量词}: \textcolor{darkblue}{\textbf{\ipa{ɭɯ˧}}} 
\lhead{\firstmark}
\rhead{\botmark}

\subsection{\hspace{-0.5cm} {\Large \textcolor{darkblue}{\textbf{\ipa{qʰwɤ˧tʰv̩\#˥}}}}\hspace{0.5cm}[\kern2pt{\textcolor{darkblue}{\textbf{\ipa{qʰwɤ˧tʰv̩˧}}}}\kern2pt]} \hypertarget{q\string_hw7\string_Mt\string_hv\string_=\#\string_T1}{}
\markboth{\textcolor{darkblue}{\textbf{\ipa{qʰwɤ˧tʰv̩\#˥}}}}{}
\textcolor{teal}{\zh{名词}} \hspace{4pt} \zh{声调类:} \#H.
\zh{竹篓。} \textcolor{Sepia}{\selectlanguage{english}Bamboo basket to carry water (on back).} \textcolor{PineGreen}{\selectlanguage{french}Hotte en bambou pour porter de l'eau.}  \zh{量词}: \textcolor{darkblue}{\textbf{\ipa{ɭɯ˧}}} 
\lhead{\firstmark}
\rhead{\botmark}

\subsection{\hspace{-0.5cm} {\Large \textcolor{darkblue}{\textbf{\ipa{qʰwɤ˧tsʰi˩}}}}\hspace{0.5cm}[\kern2pt{\textcolor{darkblue}{\textbf{\ipa{qʰwɤ˧tsʰi˩}}}}\kern2pt]} \hypertarget{q\string_hw7\string_Mts\string_hi\string_B1}{}
\markboth{\textcolor{darkblue}{\textbf{\ipa{qʰwɤ˧tsʰi˩}}}}{}
\textcolor{teal}{\zh{名词}} \hspace{4pt} \zh{声调类:} L\#.
\zh{肩膀。} \textcolor{Sepia}{\selectlanguage{english}Shoulder.} \textcolor{PineGreen}{\selectlanguage{french}Épaule.}  ¶ \textcolor{darkblue}{\textbf{\ipa{qʰwɤ˧tsʰi˩-ʁo˩ | hwæ˧pʰæ˩ | ɖɯ˧-nɑ˧-ʈʂʰɯ˧ gɤ˧˥}}} \zh{肩上扛一把锄头} \textcolor{Sepia}{\selectlanguage{english}to carry a hoe on the shoulder} \textcolor{PineGreen}{\selectlanguage{french}porter une houe à l'épaule}  
 \zh{量词}: \textcolor{darkblue}{\textbf{\ipa{ɭɯ˧}}} 
\lhead{\firstmark}
\rhead{\botmark}

\subsection{\hspace{-0.5cm} {\Large \textcolor{darkblue}{\textbf{\ipa{qʰwɤ˧zo˥\$}}}}\hspace{0.5cm}[\kern2pt{\textcolor{darkblue}{\textbf{\ipa{qʰwɤ˧zo˥}}}}\kern2pt]} \hypertarget{q\string_hw7\string_Mzo\string_T\$1}{}
\markboth{\textcolor{darkblue}{\textbf{\ipa{qʰwɤ˧zo˥\$}}}}{}
\textcolor{teal}{\zh{名词}} \hspace{4pt} \zh{声调类:} H\$.
\zh{小碗。} \textcolor{Sepia}{\selectlanguage{english}Small bowl.} \textcolor{PineGreen}{\selectlanguage{french}Petit bol.} 
\lhead{\firstmark}
\rhead{\botmark}

\subsection{\hspace{-0.5cm} {\Large \textcolor{darkblue}{\textbf{\ipa{qʰwɤ˩\textsubscript{a}}}} \textsubscript{1}}\hspace{0.5cm}[\kern2pt{\textcolor{darkblue}{\textbf{\ipa{qʰwɤ˩˥}}}}\kern2pt]} \hypertarget{q\string_hw7\string_Ba1}{}
\markboth{\textcolor{darkblue}{\textbf{\ipa{qʰwɤ˩\textsubscript{a}}}} \textsubscript{1}}{}
\textcolor{teal}{\zh{形容词}} \hspace{4pt} \zh{声调类:} L\textsubscript{a}.
\zh{聪明。} \textcolor{Sepia}{\selectlanguage{english}Intelligent.} \textcolor{PineGreen}{\selectlanguage{french}Intelligent.}  ¶ \textcolor{darkblue}{\textbf{\ipa{qʰwɤ˩-hĩ˩˥}}} \zh{聪明的} \textcolor{Sepia}{\selectlanguage{english}\mytextsc{rel}/\mytextsc{nmlz}} \textcolor{PineGreen}{\selectlanguage{french}\mytextsc{rel}/\mytextsc{nmlz}}  
 ¶ \textcolor{darkblue}{\textbf{\ipa{qʰwɤ˩-le˥!}}} \zh{很聪明! / 太聪明了!} \textcolor{Sepia}{\selectlanguage{english}(You are/(s)he is) clever! (A comment when someone says/does something clever)} \textcolor{PineGreen}{\selectlanguage{french}(il est/tu es) intelligent! (Commentaire lorsque quelqu'un dit ou fait quelque chose d'astucieux)}  
 ¶ \textcolor{darkblue}{\textbf{\ipa{ɖwæ˧˥ | qʰwɤ˩˥!}}} \zh{很聪明!} \textcolor{Sepia}{\selectlanguage{english}\mytextsc{intensive}.very \string_} \textcolor{PineGreen}{\selectlanguage{french}\mytextsc{intensif}.très \string_: très intelligent}  

\lhead{\firstmark}
\rhead{\botmark}

\subsection{\hspace{-0.5cm} {\Large \textcolor{darkblue}{\textbf{\ipa{qʰwɤ˩\textsubscript{a}}}} \textsubscript{2}}\hspace{0.5cm}[\kern2pt{\textcolor{darkblue}{\textbf{\ipa{qʰwɤ˩˥}}}}\kern2pt]} \hypertarget{q\string_hw7\string_Ba2}{}
\markboth{\textcolor{darkblue}{\textbf{\ipa{qʰwɤ˩\textsubscript{a}}}} \textsubscript{2}}{}
\textcolor{teal}{\zh{形容词}} \hspace{4pt} \zh{声调类:} L\textsubscript{a}.
\zh{坏。} \textcolor{Sepia}{\selectlanguage{english}Bad.} \textcolor{PineGreen}{\selectlanguage{french}Mauvais.}  ¶ \textcolor{darkblue}{\textbf{\ipa{kʰv̩˧ | qʰwɤ˩-hĩ˩˥}}} \zh{(收成)不好的一年} \textcolor{Sepia}{\selectlanguage{english}a bad year (a year when crops are not good)} \textcolor{PineGreen}{\selectlanguage{french}une mauvaise année}  
 ¶ \textcolor{darkblue}{\textbf{\ipa{kʰv̩˧ qʰwɤ˧˥}}} \zh{(收成)不好的一年} \textcolor{Sepia}{\selectlanguage{english}a bad year (a year when crops are not good)} \textcolor{PineGreen}{\selectlanguage{french}une mauvaise année}  
 ¶ \textcolor{darkblue}{\textbf{\ipa{tsʰi˧-ʝi˧, | kʰv̩˧qʰwɤ˧ tʰv̩˧˥!}}} \zh{今年,年景不好!(收成不好)} \textcolor{Sepia}{\selectlanguage{english}This year is a bad year! (=a year when crops are not good)} \textcolor{PineGreen}{\selectlanguage{french}cette année, c'est une mauvaise année (les récoltes sont mauvaises)!}  
 ¶ \textcolor{darkblue}{\textbf{\ipa{ʈʂʰɯ˧ | nv̩˩mi˩˥ | ɖwæ˧˥ | qʰwɤ˩˥!}}} \zh{他心很坏!} \textcolor{Sepia}{\selectlanguage{english}He has a really bad heart! / He is a really bad man!} \textcolor{PineGreen}{\selectlanguage{french}Il a l'âme bien noire!}  
 ¶ \textcolor{darkblue}{\textbf{\ipa{qʰwɤ˩-ʝi˩˥}}} \zh{干坏事:损坏东西,干扰人家……} \textcolor{Sepia}{\selectlanguage{english}to do bad things: to damage stuff; to annoy people...} \textcolor{PineGreen}{\selectlanguage{french}faire des bêtises, faire du mal: abîmer des choses, faire des misères aux gens...}  
 ¶ \textcolor{darkblue}{\textbf{\ipa{ʈʂʰɯ˧-ɳɯ˧ | njɤ˧-bv̩˧ tso˧\textasciitilde{}tso˧ | le˧-qʰwɤ˩-ʝi˩-ze˩!}}} \zh{他弄坏了我的东西!} \textcolor{Sepia}{\selectlanguage{english}(S)he has damaged my stuff!} \textcolor{PineGreen}{\selectlanguage{french}Il a abîmé mes affaires!}  
 ¶ \textcolor{darkblue}{\textbf{\ipa{hĩ˧ qʰwɤ˧-ʝi˥}}} \zh{干扰人家、麻烦人} \textcolor{Sepia}{\selectlanguage{english}to annoy people} \textcolor{PineGreen}{\selectlanguage{french}embêter les gens, faire des misères aux gens}  
 ¶ \textcolor{darkblue}{\textbf{\ipa{ʈʂʰɯ˧ | to˩to˧mi˥ hĩ˩ qʰwɤ˩-ʝi˩!}}} \zh{他故意麻烦人!} \textcolor{Sepia}{\selectlanguage{english}(S)he purposedly annoys people! / (S)he annoys people on purpose!} \textcolor{PineGreen}{\selectlanguage{french}Il/elle fait exprès d'embêter les gens!}  

\lhead{\firstmark}
\rhead{\botmark}

\subsection{\hspace{-0.5cm} {\Large \textcolor{darkblue}{\textbf{\ipa{qʰwɤ˩ɖɯ˩}}}}\hspace{0.5cm}[\kern2pt{\textcolor{darkblue}{\textbf{\ipa{qʰwɤ˩ɖɯ˩˥}}}}\kern2pt]} \hypertarget{q\string_hw7\string_Bd`M\string_B1}{}
\markboth{\textcolor{darkblue}{\textbf{\ipa{qʰwɤ˩ɖɯ˩}}}}{}
\textcolor{teal}{\zh{名词}} \hspace{4pt} \zh{声调类:} L.
\zh{亲戚。} \textcolor{Sepia}{\selectlanguage{english}Relatives, members of the family.} \textcolor{PineGreen}{\selectlanguage{french}Membres de la famille (étendue).}  ¶ \textcolor{darkblue}{\textbf{\ipa{ə˧zɯ˩ | qʰwɤ˩ɖɯ˩ ɲi˥}}} \zh{咱们两个是一家人。} \textcolor{Sepia}{\selectlanguage{english}We two belong to the same family.} \textcolor{PineGreen}{\selectlanguage{french}Tous deux, on est de la même famille}  
 ¶ \textcolor{darkblue}{\textbf{\ipa{qʰwɤ˩ɖɯ˩˥, | v̩˩dze˩˥}}} \zh{亲人:泛指亲戚与亲密朋友们} \textcolor{Sepia}{\selectlanguage{english}family and friends, extended family circle} \textcolor{PineGreen}{\selectlanguage{french}famille et amis, cercle familial élargi}  
 ¶ \textcolor{darkblue}{\textbf{\ipa{qʰwɤ˩ɖɯ˩ to˥}}} \zh{建立起两个家庭之间的联系(通过婚姻)} \textcolor{Sepia}{\selectlanguage{english}to establish family ties between two families (through marriage)} \textcolor{PineGreen}{\selectlanguage{french}établir des liens familiaux, unir deux familles (par un mariage)}  
 \zh{量词}: \textcolor{darkblue}{\textbf{\ipa{v̩˧}}} 
\lhead{\firstmark}
\rhead{\botmark}

\subsection{\hspace{-0.5cm} {\Large \textcolor{darkblue}{\textbf{\ipa{qʰwɤ˧˥}}} \textsubscript{1}}\hspace{0.5cm}[\kern2pt{\textcolor{darkblue}{\textbf{\ipa{qʰwɤ˧˥}}}}\kern2pt]} \hypertarget{q\string_hw7\string_M\string_T1}{}
\markboth{\textcolor{darkblue}{\textbf{\ipa{qʰwɤ˧˥}}} \textsubscript{1}}{}
\textcolor{teal}{\zh{名词}} \hspace{4pt} \zh{声调类:} MH.
\zh{碗。} \textcolor{Sepia}{\selectlanguage{english}Bowl.} \textcolor{PineGreen}{\selectlanguage{french}Bol.}  \zh{量词}: \textcolor{darkblue}{\textbf{\ipa{ɭɯ˧}}} 
\lhead{\firstmark}
\rhead{\botmark}

\subsection{\hspace{-0.5cm} {\Large \textcolor{darkblue}{\textbf{\ipa{qʰwɤ˧˥}}} \textsubscript{2}}\hspace{0.5cm}[\kern2pt{\textcolor{darkblue}{\textbf{\ipa{qʰwɤ˧˥}}}}\kern2pt]} \hypertarget{q\string_hw7\string_M\string_T2}{}
\markboth{\textcolor{darkblue}{\textbf{\ipa{qʰwɤ˧˥}}} \textsubscript{2}}{}
\textcolor{teal}{\zh{名词}} \hspace{4pt} \zh{声调类:} MH.
\zh{故事。} \textcolor{Sepia}{\selectlanguage{english}Tale, story, yarn.} \textcolor{PineGreen}{\selectlanguage{french}Histoire, récit.}  ¶ \textcolor{darkblue}{\textbf{\ipa{æ˧ʂæ˧-qʰwɤ˧˥}}} \zh{老故事} \textcolor{Sepia}{\selectlanguage{english}tale, folk tale} \textcolor{PineGreen}{\selectlanguage{french}récit d'autrefois, conte}  
 \zh{量词}: \textcolor{darkblue}{\textbf{\ipa{kʰwɤ˥}}} 
\lhead{\firstmark}
\rhead{\botmark}

\subsection{\hspace{-0.5cm} {\Large \textcolor{darkblue}{\textbf{\ipa{qʰwɤ˧˥\textsubscript{a}}}}}\hspace{0.5cm}[\kern2pt{\textcolor{darkblue}{\textbf{\ipa{qʰwɤ˧˥}}}}\kern2pt]} \hypertarget{q\string_hw7\string_M\string_Ta1}{}
\markboth{\textcolor{darkblue}{\textbf{\ipa{qʰwɤ˧˥\textsubscript{a}}}}}{}
\textcolor{teal}{\zh{量词}} \hspace{4pt} \zh{声调类:} MH\textsubscript{a}.
\zh{量词:碗。} \textcolor{Sepia}{\selectlanguage{english}A bowl(ful) of.} \textcolor{PineGreen}{\selectlanguage{french}Classificateur des bols (utilisés comme quantité de mesure du non dénombrable).} 
\lhead{\firstmark}
\rhead{\botmark}

\newpage
\section*{\centering- \textcolor{darkblue}{\textbf{\ipa{ɻ}}} -}
\subsection{\hspace{-0.5cm} {\Large \textcolor{darkblue}{\textbf{\ipa{ɻ̍˩}}}}\hspace{0.5cm}[\kern2pt{\textcolor{darkblue}{\textbf{\ipa{ɻ̍˥}}}}\kern2pt]} \hypertarget{r£`̍\string_B1}{}
\markboth{\textcolor{darkblue}{\textbf{\ipa{ɻ̍˩}}}}{}
\textcolor{teal}{\zh{名词}} \hspace{4pt} \zh{声调类:} L.
\zh{面(一个四方形物品的四面)。} \textcolor{Sepia}{\selectlanguage{english}Side, direction.} \textcolor{PineGreen}{\selectlanguage{french}Direction, côté.}  ¶ \textcolor{darkblue}{\textbf{\ipa{[“Housebuilding”] ʐv̩˧-ɻ̍˥}}} \zh{四面} \textcolor{Sepia}{\selectlanguage{english}the four directions, the four sides (e.g. of a house)} \textcolor{PineGreen}{\selectlanguage{french}les quatre directions, les quatre côtés (d'une maison)}  

\lhead{\firstmark}
\rhead{\botmark}

\subsection{\hspace{-0.5cm} {\Large \textcolor{darkblue}{\textbf{\ipa{ɻ̍˩\textsubscript{b}}}}}\hspace{0.5cm}[\kern2pt{\textcolor{darkblue}{\textbf{\ipa{ɻ̍˩˥}}}}\kern2pt]} \hypertarget{r£`̍\string_Bb1}{}
\markboth{\textcolor{darkblue}{\textbf{\ipa{ɻ̍˩\textsubscript{b}}}}}{}
\textcolor{teal}{\zh{动词}} \hspace{4pt} \zh{声调类:} L\textsubscript{b}.
\zh{对着。} \textcolor{Sepia}{\selectlanguage{english}To face, to turn toward.} \textcolor{PineGreen}{\selectlanguage{french}Regarder, se tourner vers, faire face à.}  ¶ \textcolor{darkblue}{\textbf{\ipa{mɤ˧-ɻ̍˩}}} \zh{\mytextsc{neg}} \textcolor{Sepia}{\selectlanguage{english}\mytextsc{neg}} \textcolor{PineGreen}{\selectlanguage{french}\mytextsc{neg}}  
 ¶ \textcolor{darkblue}{\textbf{\ipa{ɖɯ˧-ɻ̍˧\textasciitilde{}ɻ̍˩}}} \zh{\mytextsc{delimitative} \string_ \mytextsc{red}} \textcolor{Sepia}{\selectlanguage{english}\mytextsc{delimitative} \string_ \mytextsc{red}} \textcolor{PineGreen}{\selectlanguage{french}\mytextsc{délimitatif} \string_ \mytextsc{red}}  
 ¶ \textcolor{darkblue}{\textbf{\ipa{ze˩gi˧ ɻ̍˥?}}} \zh{(我要)往哪边转?} \textcolor{Sepia}{\selectlanguage{english}In which direction should (I) look? / Which direction should I turn to?} \textcolor{PineGreen}{\selectlanguage{french}dans quelle direction regarder?}  
 ¶ \textcolor{darkblue}{\textbf{\ipa{no˧ | ʈʂʰɯ˧tɕo˧ ɻ̍˩!}}} \zh{你往这里转/往这里看!} \textcolor{Sepia}{\selectlanguage{english}Turn this way! / Turn towards this direction!} \textcolor{PineGreen}{\selectlanguage{french}Tourne-toi par ici! / Regarde par ici!}  
 ¶ \textcolor{darkblue}{\textbf{\ipa{gɤ˩-ɻ̍˥ mv̩˩-ɻ̍˩, | ə˧tso˧ li˧?}}} \zh{你左转右转,(到底)在看什么?} \textcolor{Sepia}{\selectlanguage{english}You turn in all directions; what are you looking for/at? / What are you looking for in all directions?} \textcolor{PineGreen}{\selectlanguage{french}Qu’as-tu à regarder de toutes parts ? / Tu regardes de toutes parts, que cherches-tu ?}  

\lhead{\firstmark}
\rhead{\botmark}

\subsection{\hspace{-0.5cm} {\Large \textcolor{darkblue}{\textbf{\ipa{ɻ̍˧bɤ˧}}}}\hspace{0.5cm}[\kern2pt{\textcolor{darkblue}{\textbf{\ipa{ɻ̍˧bɤ˧}}}}\kern2pt]} \hypertarget{r£`̍\string_Mb7\string_M1}{}
\markboth{\textcolor{darkblue}{\textbf{\ipa{ɻ̍˧bɤ˧}}}}{}
\textcolor{teal}{\zh{名词}} \hspace{4pt} \zh{声调类:} M.
\zh{实情,真理。} \textcolor{Sepia}{\selectlanguage{english}The truth; the facts.} \textcolor{PineGreen}{\selectlanguage{french}La vérité; le vrai et le faux; les faits authentiques.}  ¶ \textcolor{darkblue}{\textbf{\ipa{njɤ˧-ɳɯ˧ | ɻ̍˧bɤ˧ | ʐwɤ˩-bi˩˥!}}} \zh{我要把实情说出来!} \textcolor{Sepia}{\selectlanguage{english}I am going to tell the truth!} \textcolor{PineGreen}{\selectlanguage{french}je vais dire toute la vérité/je vais faire la lumière!}  
 \zh{量词}: \textcolor{darkblue}{\textbf{\ipa{kʰwɤ˥}}} 
\lhead{\firstmark}
\rhead{\botmark}

\subsection{\hspace{-0.5cm} {\Large \textcolor{darkblue}{\textbf{\ipa{ɻ̍˧qʰv̩˧}}}}\hspace{0.5cm}[\kern2pt{\textcolor{darkblue}{\textbf{\ipa{ɻ̍˧qʰv̩˧}}}}\kern2pt]} \hypertarget{r£`̍\string_Mq\string_hv\string_=\string_M1}{}
\markboth{\textcolor{darkblue}{\textbf{\ipa{ɻ̍˧qʰv̩˧}}}}{}
\textcolor{teal}{\zh{名词}} \hspace{4pt} \zh{声调类:} M.
\zh{温泉。} \textcolor{Sepia}{\selectlanguage{english}Warm springs.} \textcolor{PineGreen}{\selectlanguage{french}Source chaude.}  ¶ \textcolor{darkblue}{\textbf{\ipa{ɻ̍˧qʰv̩˧-dʑɯ˩}}} \zh{温泉水(不可饮用)} \textcolor{Sepia}{\selectlanguage{english}warm spring water (not drinkable)} \textcolor{PineGreen}{\selectlanguage{french}eau de source chaude (non potable)}  
 \zh{量词}: \textcolor{darkblue}{\textbf{\ipa{ɭɯ˧}}} 
\lhead{\firstmark}
\rhead{\botmark}

\subsection{\hspace{-0.5cm} {\Large \textcolor{darkblue}{\textbf{\ipa{ɻ̍˩ɻ̍˧-lo˩}}}}\hspace{0.5cm}[\kern2pt{\textcolor{darkblue}{\textbf{\ipa{ɻ̍˩ɻ̍˧lo˧}}}}\kern2pt]} \hypertarget{r£`̍\string_Br£`̍\string_M-lo\string_B1}{}
\markboth{\textcolor{darkblue}{\textbf{\ipa{ɻ̍˩ɻ̍˧-lo˩}}}}{}
\textcolor{teal}{\zh{名词}} \hspace{4pt} \zh{声调类:} LM-L.
\zh{马帮中的第二匹马。} \textcolor{Sepia}{\selectlanguage{english}The horse walking in second position in a caravan.} \textcolor{PineGreen}{\selectlanguage{french}Le cheval qui marche en second (derrière le cheval de tête), dans une caravan.} 
\lhead{\firstmark}
\rhead{\botmark}

\subsection{\hspace{-0.5cm} {\Large \textcolor{darkblue}{\textbf{\ipa{ɻ̍˧tɑ˧}}}}\hspace{0.5cm}[\kern2pt{\textcolor{darkblue}{\textbf{\ipa{ɻ̍˧tɑ˧}}}}\kern2pt]} \hypertarget{r£`̍\string_MtA\string_M1}{}
\markboth{\textcolor{darkblue}{\textbf{\ipa{ɻ̍˧tɑ˧}}}}{}
\textcolor{teal}{\zh{名词}} \hspace{4pt} \zh{声调类:} M.
\zh{淋巴结。} \textcolor{Sepia}{\selectlanguage{english}Lymph nodes, glands.} \textcolor{PineGreen}{\selectlanguage{french}Ganglions.}  \zh{量词}: \textcolor{darkblue}{\textbf{\ipa{ɭɯ˧}}} 
\lhead{\firstmark}
\rhead{\botmark}

\subsection{\hspace{-0.5cm} {\Large \textcolor{darkblue}{\textbf{\ipa{ɻ̍˩ʈʂʰe˧-ɖɯ˩mɑ˩}}}}\hspace{0.5cm}[\kern2pt{\textcolor{darkblue}{\textbf{\ipa{ɻ̍˩ʈʂʰe˧ɖɯ˩mɑ˩}}}}\kern2pt]} \hypertarget{r£`̍\string_Bt`s`\string_he\string_M-d`M\string_BmA\string_B1}{}
\markboth{\textcolor{darkblue}{\textbf{\ipa{ɻ̍˩ʈʂʰe˧-ɖɯ˩mɑ˩}}}}{}
\textcolor{teal}{\zh{名词}} \hspace{4pt} \zh{声调类:} LM-L.
\zh{女性名字。} \textcolor{Sepia}{\selectlanguage{english}Feminine given name.} \textcolor{PineGreen}{\selectlanguage{french}Prénom féminin.} 
\lhead{\firstmark}
\rhead{\botmark}

\subsection{\hspace{-0.5cm} {\Large \textcolor{darkblue}{\textbf{\ipa{ɻ̍˩ʈʂʰe˧-tsʰɯ˩ɻ̍˩}}}}\hspace{0.5cm}[\kern2pt{\textcolor{darkblue}{\textbf{\ipa{ɻ̍˩ʈʂʰe˧tsʰɯ˩ɻ̍˩}}}}\kern2pt]} \hypertarget{r£`̍\string_Bt`s`\string_he\string_M-ts\string_hM\string_Br£`̍\string_B1}{}
\markboth{\textcolor{darkblue}{\textbf{\ipa{ɻ̍˩ʈʂʰe˧-tsʰɯ˩ɻ̍˩}}}}{}
\textcolor{teal}{\zh{名词}} \hspace{4pt} \zh{声调类:} LM-L.
\zh{男性名字。} \textcolor{Sepia}{\selectlanguage{english}Masculine given name.} \textcolor{PineGreen}{\selectlanguage{french}Prénom masculin.} 
\lhead{\firstmark}
\rhead{\botmark}

\subsection{\hspace{-0.5cm} {\Large \textcolor{darkblue}{\textbf{\ipa{ɻ̍˩ʈʂʰe\#˥}}}}\hspace{0.5cm}[\kern2pt{\textcolor{darkblue}{\textbf{\ipa{ɻ̍˩ʈʂʰe˥}}}}\kern2pt]} \hypertarget{r£`̍\string_Bt`s`\string_he\#\string_T1}{}
\markboth{\textcolor{darkblue}{\textbf{\ipa{ɻ̍˩ʈʂʰe\#˥}}}}{}
\textcolor{teal}{\zh{名词}} \hspace{4pt} \zh{声调类:} LM+\#H.
\zh{男性名字。} \textcolor{Sepia}{\selectlanguage{english}Masculine given name.} \textcolor{PineGreen}{\selectlanguage{french}Prénom masculin.} 
\lhead{\firstmark}
\rhead{\botmark}

\subsection{\hspace{-0.5cm} {\Large \textcolor{darkblue}{\textbf{\ipa{‑ɻ̍˩}}}}\hspace{0.5cm}[\kern2pt{\textcolor{darkblue}{\textbf{\ipa{ɻ̍˩˥}}}}\kern2pt]} \hypertarget{‑r£`̍\string_B1}{}
\markboth{\textcolor{darkblue}{\textbf{\ipa{‑ɻ̍˩}}}}{}
\textcolor{teal}{\zh{后缀}} \hspace{4pt} \zh{声调类:} L.
\zh{\mytextsc{发端。}} \textcolor{Sepia}{\selectlanguage{english}\mytextsc{inceptive} (\mytextsc{inchoative}).} \textcolor{PineGreen}{\selectlanguage{french}\mytextsc{inchoatif}.} 
\lhead{\firstmark}
\rhead{\botmark}

\subsection{\hspace{-0.5cm} {\Large \textcolor{darkblue}{\textbf{\ipa{=ɻ̍˩}}}}\hspace{0.5cm}[\kern2pt{\textcolor{darkblue}{\textbf{\ipa{ɻ̍˩˥}}}}\kern2pt]} \hypertarget{=r£`̍\string_B1}{}
\markboth{\textcolor{darkblue}{\textbf{\ipa{=ɻ̍˩}}}}{}
\textcolor{teal}{\zh{附着词}} \hspace{4pt} \zh{声调类:} L.
\zh{联想复数:一家人、一族人、一辈人……。} \textcolor{Sepia}{\selectlanguage{english}Associative plural.} \textcolor{PineGreen}{\selectlanguage{french}Pluriel associatif, ou pluriel d'accompagnement, couramment utilisé avec les termes de parenté, les noms de clans...}  ¶ \textcolor{darkblue}{\textbf{\ipa{ʈʂʰɯ˧-ʑi˧=ɻ̍˥}}} \zh{这家的人} \textcolor{Sepia}{\selectlanguage{english}this household; the people of this family} \textcolor{PineGreen}{\selectlanguage{french}les gens de cette famille; cette maisonnée-ci}  

\lhead{\firstmark}
\rhead{\botmark}

\subsection{\hspace{-0.5cm} {\Large \textcolor{darkblue}{\textbf{\ipa{=ɻæ˩}}}}\hspace{0.5cm}[\kern2pt{\textcolor{darkblue}{\textbf{\ipa{ɻæ˩˥}}}}\kern2pt]} \hypertarget{=r£`\{\string_B1}{}
\markboth{\textcolor{darkblue}{\textbf{\ipa{=ɻæ˩}}}}{}
\textcolor{teal}{\zh{附着词}} \hspace{4pt} \zh{声调类:} L.
\zh{多数。} \textcolor{Sepia}{\selectlanguage{english}Plural.} \textcolor{PineGreen}{\selectlanguage{french}Pluriel.}  ¶ \textcolor{darkblue}{\textbf{\ipa{ʈʂʰɯ˧-ɻæ˥\$}}} \zh{这类的东西,……之类} \textcolor{Sepia}{\selectlanguage{english}these things, this sort of things} \textcolor{PineGreen}{\selectlanguage{french}ces choses-ci, cette sorte de choses}  

\lhead{\firstmark}
\rhead{\botmark}

\subsection{\hspace{-0.5cm} {\Large \textcolor{darkblue}{\textbf{\ipa{ɻæ˩\textsubscript{a}}}}}\hspace{0.5cm}[\kern2pt{\textcolor{darkblue}{\textbf{\ipa{ɻæ˩˥}}}}\kern2pt]} \hypertarget{r£`\{\string_Ba1}{}
\markboth{\textcolor{darkblue}{\textbf{\ipa{ɻæ˩\textsubscript{a}}}}}{}
\textcolor{teal}{\zh{形容词}} \hspace{4pt} \zh{声调类:} L\textsubscript{a}.
\zh{瘪。} \textcolor{Sepia}{\selectlanguage{english}Shrivelled, flat, shrunken.} \textcolor{PineGreen}{\selectlanguage{french}Plat, de forme plate, aplati.}  ¶ \textcolor{darkblue}{\textbf{\ipa{ɻæ˩-hĩ˩˥}}} \zh{瘪的} \textcolor{Sepia}{\selectlanguage{english}\mytextsc{rel}/\mytextsc{nmlz}} \textcolor{PineGreen}{\selectlanguage{french}\mytextsc{rel}/\mytextsc{nmlz}}  
 ¶ \textcolor{darkblue}{\textbf{\ipa{ɻæ˩ti˩ɻæ˥ (-gv̩˩)}}} \zh{瘪瘪的} \textcolor{Sepia}{\selectlanguage{english}shrivelled, flat, shrunken} \textcolor{PineGreen}{\selectlanguage{french}raplapla, ratatiné}  

\lhead{\firstmark}
\rhead{\botmark}

\subsection{\hspace{-0.5cm} {\Large \textcolor{darkblue}{\textbf{\ipa{ɻæ˩˥}}} \textsubscript{1}}\hspace{0.5cm}[\kern2pt{\textcolor{darkblue}{\textbf{\ipa{ɻæ˩˥}}}}\kern2pt]} \hypertarget{r£`\{\string_B\string_T1}{}
\markboth{\textcolor{darkblue}{\textbf{\ipa{ɻæ˩˥}}} \textsubscript{1}}{}
\textcolor{teal}{\zh{名词}} \hspace{4pt} \zh{声调类:} LH.
\zh{种子。} \textcolor{Sepia}{\selectlanguage{english}Seed.} \textcolor{PineGreen}{\selectlanguage{french}Graine.}  \zh{量词}: \textcolor{darkblue}{\textbf{\ipa{ɭɯ˧}}} 
\lhead{\firstmark}
\rhead{\botmark}

\subsection{\hspace{-0.5cm} {\Large \textcolor{darkblue}{\textbf{\ipa{ɻæ˩˥}}} \textsubscript{2}}\hspace{0.5cm}[\kern2pt{\textcolor{darkblue}{\textbf{\ipa{ɻæ˩˥}}}}\kern2pt]} \hypertarget{r£`\{\string_B\string_T2}{}
\markboth{\textcolor{darkblue}{\textbf{\ipa{ɻæ˩˥}}} \textsubscript{2}}{}
\textcolor{teal}{\zh{名词}} \hspace{4pt} \zh{声调类:} LH.
\zh{牛轭(单行或双行)。} \textcolor{Sepia}{\selectlanguage{english}Yoke (for one or two animals).} \textcolor{PineGreen}{\selectlanguage{french}Joug (le terme est le même pour un ou deux animaux).}  ¶ \textcolor{darkblue}{\textbf{\ipa{ʝi˧-ɻæ˥}}} \zh{牛轭} \textcolor{Sepia}{\selectlanguage{english}same meaning as the monosyllabic form: yoke (literally 'ox yoke')} \textcolor{PineGreen}{\selectlanguage{french}même sens que la forme monosyllabique: joug; littéralement 'joug de bœuf/buffle'}  
 ¶ \textcolor{darkblue}{\textbf{\ipa{ɻæ˩ ʈʂʰɯ˩-ɭɯ˥ / ɻæ˩ ʈʂʰɯ˩-ɭɯ˩˥}}} \zh{这个牛轭} \textcolor{Sepia}{\selectlanguage{english}\mytextsc{n}+\mytextsc{dem}+\mytextsc{clf;} allows two variants} \textcolor{PineGreen}{\selectlanguage{french}\mytextsc{n}+\mytextsc{dem}+\mytextsc{clf;} cette expression possède deux variantes tonales}  
 \zh{量词}: \textcolor{darkblue}{\textbf{\ipa{ɭɯ˧}}} 
\lhead{\firstmark}
\rhead{\botmark}

\subsection{\hspace{-0.5cm} {\Large \textcolor{darkblue}{\textbf{\ipa{ɻwæ˥}}}}\hspace{0.5cm}[\kern2pt{\textcolor{darkblue}{\textbf{\ipa{ɻwæ˥}}}}\kern2pt]} \hypertarget{r£`w\{\string_T1}{}
\markboth{\textcolor{darkblue}{\textbf{\ipa{ɻwæ˥}}}}{}
\textcolor{teal}{\zh{动词}} \hspace{4pt} \zh{声调类:} H.
\ding{202} \zh{喊、吼、叫(人、猫、牛、猪、羊、狼、驴、狮子、老虎、豺狼……)。} \textcolor{Sepia}{\selectlanguage{english}To cry (man, and animals: cat, cow, horse, donkey, chicken, lion, wolf…); to call out.} \textcolor{PineGreen}{\selectlanguage{french}Crier, hurler; miauler; braire; hennir; rugir (chat, bœuf, cochon, mouton, loup, lion).}  ¶ \textcolor{darkblue}{\textbf{\ipa{mɤ˧-ɻwæ˥}}} \zh{不叫} \textcolor{Sepia}{\selectlanguage{english}\mytextsc{neg}} \textcolor{PineGreen}{\selectlanguage{french}\mytextsc{neg}}  
 ¶ \textcolor{darkblue}{\textbf{\ipa{ɻwæ˧\textasciitilde{}ɻwæ˧}}} \zh{\mytextsc{重叠}} \textcolor{Sepia}{\selectlanguage{english}\mytextsc{red}} \textcolor{PineGreen}{\selectlanguage{french}\mytextsc{red}}  
 ¶ \textcolor{darkblue}{\textbf{\ipa{hĩ˧ ɻwæ˧-dʑo˩}}} \zh{有人在叫。} \textcolor{Sepia}{\selectlanguage{english}Someone is shouting} \textcolor{PineGreen}{\selectlanguage{french}Il y a quelqu'un qui est est en train d'appeler/de crier}  
 ¶ \textcolor{darkblue}{\textbf{\ipa{ɖɯ˧-ɻwæ˧-ɻ̍˥}}} \zh{叫一声} \textcolor{Sepia}{\selectlanguage{english}to call out} \textcolor{PineGreen}{\selectlanguage{french}appeler, lancer un appel}  
 ¶ \textcolor{darkblue}{\textbf{\ipa{hwɤ˧li˧ ɻwæ˥-dʑo˩}}} \zh{猫在叫} \textcolor{Sepia}{\selectlanguage{english}the cat is calling/crying} \textcolor{PineGreen}{\selectlanguage{french}le chat miaule}  
 ¶ \textcolor{darkblue}{\textbf{\ipa{æ̃˩ ɻwæ˥}}} \zh{鸡在叫} \textcolor{Sepia}{\selectlanguage{english}the chicken is cackling} \textcolor{PineGreen}{\selectlanguage{french}la poule caquette}  
 ¶ \textcolor{darkblue}{\textbf{\ipa{ʐwæ˧pʰæ˧di˧˥ | tʰi˧-ɻwæ˥-dʑo˩}}} \zh{驴在叫} \textcolor{Sepia}{\selectlanguage{english}the donkey is braying} \textcolor{PineGreen}{\selectlanguage{french}l'âne brait}  
 ¶ \textcolor{darkblue}{\textbf{\ipa{[F5] ʐwæ˧ | tʰi˧-ɻwæ˥-dʑo˩}}} \zh{马在嘶} \textcolor{Sepia}{\selectlanguage{english}the horse is whinnying} \textcolor{PineGreen}{\selectlanguage{french}le cheval est en train de hennir}  
 ¶ \textcolor{darkblue}{\textbf{\ipa{ʐwæ˧ ɻwæ˧-dʑo˩!}}} \zh{马在嘶} \textcolor{Sepia}{\selectlanguage{english}the horse is whinnying} \textcolor{PineGreen}{\selectlanguage{french}le cheval est en train de hennir}  
\ding{203} \zh{请、叫(来)。} \textcolor{Sepia}{\selectlanguage{english}To invite, to call over.} \textcolor{PineGreen}{\selectlanguage{french}Inviter, faire venir.}  ¶ \textcolor{darkblue}{\textbf{\ipa{ɖɯ˧-ɻwæ˧-ɻ̍˥}}} \zh{请来一下} \textcolor{Sepia}{\selectlanguage{english}\mytextsc{delimitative} \string_ \mytextsc{inceptive}} \textcolor{PineGreen}{\selectlanguage{french}\mytextsc{délimitatif} \string_ \mytextsc{inchoatif}}  
 ¶ \textcolor{darkblue}{\textbf{\ipa{tʰɑ˧-ɻwæ˥!}}} \zh{不要请!} \textcolor{Sepia}{\selectlanguage{english}\mytextsc{prohib}} \textcolor{PineGreen}{\selectlanguage{french}\mytextsc{prohib}}  
 ¶ \textcolor{darkblue}{\textbf{\ipa{ɻwæ˧-mɤ˧-bi˧!}}} \zh{不请他!} \textcolor{Sepia}{\selectlanguage{english}(We)'re not inviting (him/her)!} \textcolor{PineGreen}{\selectlanguage{french}(on) ne l'invite pas!}  

\lhead{\firstmark}
\rhead{\botmark}

\subsection{\hspace{-0.5cm} {\Large \textcolor{darkblue}{\textbf{\ipa{ɻwæ˥\textsubscript{b}}}}}\hspace{0.5cm}[\kern2pt{\textcolor{darkblue}{\textbf{\ipa{ɻwæ˥}}}}\kern2pt]} \hypertarget{r£`w\{\string_Tb1}{}
\markboth{\textcolor{darkblue}{\textbf{\ipa{ɻwæ˥\textsubscript{b}}}}}{}
\textcolor{teal}{\zh{量词}} \hspace{4pt} \zh{声调类:} H\textsubscript{b}.
\zh{量词:地方(一处)。} \textcolor{Sepia}{\selectlanguage{english}Classifier for places.} \textcolor{PineGreen}{\selectlanguage{french}Classificateur des lieux, des endroits.}  ¶ \textcolor{darkblue}{\textbf{\ipa{tʰv̩˧-ɻwæ˧-qo˥ | mɤ˧-tʰv̩˧-sɯ˩!}}} \zh{还没到这些地方} \textcolor{Sepia}{\selectlanguage{english}...has never been to those places} \textcolor{PineGreen}{\selectlanguage{french}...n'est jamais allé dans ces lieux-là}  

\lhead{\firstmark}
\rhead{\botmark}

\newpage
\section*{\centering- \textcolor{darkblue}{\textbf{\ipa{ɻ̃}}} -}
\subsection{\hspace{-0.5cm} {\Large \textcolor{darkblue}{\textbf{\ipa{ɻ̃˥}}}}\hspace{0.5cm}[\kern2pt{\textcolor{darkblue}{\textbf{\ipa{ɻ̃˥}}}}\kern2pt]} \hypertarget{r£`\string_~\string_T1}{}
\markboth{\textcolor{darkblue}{\textbf{\ipa{ɻ̃˥}}}}{}
\textcolor{teal}{\zh{名词}} \hspace{4pt} \zh{声调类:} \#H.
\zh{骨头。} \textcolor{Sepia}{\selectlanguage{english}Bone.} \textcolor{PineGreen}{\selectlanguage{french}Os, ossement.}  \zh{量词}: \textcolor{darkblue}{\textbf{\ipa{kɤ˧˥}}} 
\lhead{\firstmark}
\rhead{\botmark}

\subsection{\hspace{-0.5cm} {\Large \textcolor{darkblue}{\textbf{\ipa{ɻ̃˥}}}}\hspace{0.5cm}[\kern2pt{\textcolor{darkblue}{\textbf{\ipa{ɻ̃˥}}}}\kern2pt]} \hypertarget{r£`\string_~\string_T1}{}
\markboth{\textcolor{darkblue}{\textbf{\ipa{ɻ̃˥}}}}{}
\textcolor{teal}{\zh{形容词}} \hspace{4pt} \zh{声调类:} H.
\zh{困难、贫穷。} \textcolor{Sepia}{\selectlanguage{english}Destitute, impoverished, poor; troubled, helpless.} \textcolor{PineGreen}{\selectlanguage{french}Démuni, en mauvaise passe.}  ¶ \textcolor{darkblue}{\textbf{\ipa{le˧-ɻ̃˥-ze˩!}}} \zh{(他)真的很穷苦!} \textcolor{Sepia}{\selectlanguage{english}[(S)he] is really poor/helpless!} \textcolor{PineGreen}{\selectlanguage{french}(il) est en mauvaise passe!/il est à la rue!}  
 ¶ \textcolor{darkblue}{\textbf{\ipa{le˧-ɻ̃˧-bi˧}}} \zh{\mytextsc{accomp} \string_ \mytextsc{fut}\string_imm} \textcolor{Sepia}{\selectlanguage{english}\mytextsc{accomp} \string_ \mytextsc{fut}\string_imm} \textcolor{PineGreen}{\selectlanguage{french}\mytextsc{accomp} \string_ \mytextsc{fut}\string_imm}  
 ¶ \textcolor{darkblue}{\textbf{\ipa{mɤ˧-ɻ̃˥}}} \zh{\mytextsc{neg}} \textcolor{Sepia}{\selectlanguage{english}\mytextsc{neg}} \textcolor{PineGreen}{\selectlanguage{french}\mytextsc{neg}}  
 ¶ \textcolor{darkblue}{\textbf{\ipa{le˧-ɻ̃˧-zo˥, | ɻ̃˧-lɑ˩ bi˩-mɤ˩-dʑɯ˩!}}} \zh{很困难,也还没有到饿死的程度啊! / 再困难,也还没饿死!(直译:“再困难,也没有露出骨头!”这个成语,来安慰认为自己太可怜的人。)} \textcolor{Sepia}{\selectlanguage{english}“Sure, we're in poverty/we're hungry, but not to the point where bones are bare!” Play on words on 'poor, destitute' and 'bone', which are homophonous. The proverb is used to relativize people's perceived degree of misfortune.} \textcolor{PineGreen}{\selectlanguage{french}“Pour démuni/mal nourri/affamé qu'on soit, on n'en est pas encore maigre au point d'avoir les os à découvert!” Jeu de mots sur 'démuni' et 'ossement', qui sont homophones. Le proverbe sert à relativiser le malheur ressenti par quelqu'un.}  
 ¶ \textcolor{darkblue}{\textbf{\ipa{ɻ̃˧-ʐwɤ˧˥}}} \zh{诉苦、抱怨} \textcolor{Sepia}{\selectlanguage{english}to complain} \textcolor{PineGreen}{\selectlanguage{french}se plaindre}  
 ¶ \textcolor{darkblue}{\textbf{\ipa{ɻ̃˧-ʐwɤ˧ | dɑ˧-ʐwɤ˧-ɻ̍˥}}} \zh{诉苦、讲自己的不幸} \textcolor{Sepia}{\selectlanguage{english}to tell one's miseries, to complain about one's fate} \textcolor{PineGreen}{\selectlanguage{french}raconter ses malheurs; se plaindre}  
 ¶ \textcolor{darkblue}{\textbf{\ipa{ʈʂʰɯ˧ | mɑ˧dɑ˩-qʰwɤ˩, | ɻ̃˧-ʐwɤ˧ | dɑ˧-ʐwɤ˧-ɻ̍˥!}}} \zh{他不幸福,一直在讲自己怎么可怜!} \textcolor{Sepia}{\selectlanguage{english}He is unhappy; he spends his time complaining / he is always complaining!} \textcolor{PineGreen}{\selectlanguage{french}Il est malheureux; il passe son temps à se plaindre!}  

\lhead{\firstmark}
\rhead{\botmark}

\subsection{\hspace{-0.5cm} {\Large \textcolor{darkblue}{\textbf{\ipa{ɻ̃˧}}}}\hspace{0.5cm}[\kern2pt{\textcolor{darkblue}{\textbf{\ipa{ɻ̃˥}}}}\kern2pt]} \hypertarget{r£`\string_~\string_M1}{}
\markboth{\textcolor{darkblue}{\textbf{\ipa{ɻ̃˧}}}}{}
\textcolor{teal}{\zh{名词}} \hspace{4pt} \zh{声调类:} M.
\zh{家族。} \textcolor{Sepia}{\selectlanguage{english}Clan.} \textcolor{PineGreen}{\selectlanguage{french}Clan: ensemble de familles.}  ¶ \textcolor{darkblue}{\textbf{\ipa{ɻ̃˧ ɖɯ˧-ɻ̃˧}}} \zh{一个家族} \textcolor{Sepia}{\selectlanguage{english}one clan} \textcolor{PineGreen}{\selectlanguage{french}un clan}  
 \zh{量词}: \textcolor{darkblue}{\textbf{\ipa{ɻ̃˧}}} 
\lhead{\firstmark}
\rhead{\botmark}

\subsection{\hspace{-0.5cm} {\Large \textcolor{darkblue}{\textbf{\ipa{ɻ̃˧\textsubscript{b}}}}}\hspace{0.5cm}[\kern2pt{\textcolor{darkblue}{\textbf{\ipa{ɻ̃˥}}}}\kern2pt]} \hypertarget{r£`\string_~\string_Mb1}{}
\markboth{\textcolor{darkblue}{\textbf{\ipa{ɻ̃˧\textsubscript{b}}}}}{}
\textcolor{teal}{\zh{量词}} \hspace{4pt} \zh{声调类:} M\textsubscript{b}.
\zh{量词:家族。} \textcolor{Sepia}{\selectlanguage{english}Classifier for clans / extended families; literally 'one bone'. This unit is located one level higher up than the 'family community' in Fu Maoji's (1983) terminology.} \textcolor{PineGreen}{\selectlanguage{french}Classificateur des clans / groupes de familles: littéralement 'un os'. Echelon supérieur à celui de la 'communauté familiale' dans la terminologie de Fu Maoji (1983).} 
\lhead{\firstmark}
\rhead{\botmark}

\subsection{\hspace{-0.5cm} {\Large \textcolor{darkblue}{\textbf{\ipa{ɻ̃˧hæ˩}}}}\hspace{0.5cm}[\kern2pt{\textcolor{darkblue}{\textbf{\ipa{ɻ̃˧hæ˩}}}}\kern2pt]} \hypertarget{r£`\string_~\string_Mh\{\string_B1}{}
\markboth{\textcolor{darkblue}{\textbf{\ipa{ɻ̃˧hæ˩}}}}{}
\textcolor{teal}{\zh{名词}} \hspace{4pt} \zh{声调类:} L\#.
\zh{软骨。} \textcolor{Sepia}{\selectlanguage{english}Cartilage.} \textcolor{PineGreen}{\selectlanguage{french}Cartilage.}  \zh{量词}: \textcolor{darkblue}{\textbf{\ipa{ɭɯ˧}}} \textcolor{darkblue}{\textbf{\ipa{kɤ˧˥}}} 
\lhead{\firstmark}
\rhead{\botmark}

\subsection{\hspace{-0.5cm} {\Large \textcolor{darkblue}{\textbf{\ipa{ɻ̃˧kɤ˩}}}}\hspace{0.5cm}[\kern2pt{\textcolor{darkblue}{\textbf{\ipa{ɻ̃˧kɤ˩}}}}\kern2pt]} \hypertarget{r£`\string_~\string_Mk7\string_B1}{}
\markboth{\textcolor{darkblue}{\textbf{\ipa{ɻ̃˧kɤ˩}}}}{}
\textcolor{teal}{\zh{名词}} \hspace{4pt} \zh{声调类:} L\#.
\zh{脊椎骨。} \textcolor{Sepia}{\selectlanguage{english}Backbone.} \textcolor{PineGreen}{\selectlanguage{french}Colonne vertébrale.}  \zh{量词}: \textcolor{darkblue}{\textbf{\ipa{kɤ˧˥}}} 
\lhead{\firstmark}
\rhead{\botmark}

\subsection{\hspace{-0.5cm} {\Large \textcolor{darkblue}{\textbf{\ipa{ɻ̃˧ko˩}}}}\hspace{0.5cm}[\kern2pt{\textcolor{darkblue}{\textbf{\ipa{ɻ̃˧ko˩}}}}\kern2pt]} \hypertarget{r£`\string_~\string_Mko\string_B1}{}
\markboth{\textcolor{darkblue}{\textbf{\ipa{ɻ̃˧ko˩}}}}{}
\textcolor{teal}{\zh{名词}} \hspace{4pt} \zh{声调类:} L\#.
\zh{胫骨。} \textcolor{Sepia}{\selectlanguage{english}Shinbone, tibia.} \textcolor{PineGreen}{\selectlanguage{french}Tibia.}  ¶ \textcolor{darkblue}{\textbf{\ipa{hĩ˧-dzɑ˧ | ɖʐe˧ tʰɑ˧-ʝi˥, | ɻ̃˧ko˩ mi˩ tʰɑ˩-tʰv̩˩. |}}} \zh{“穷人莫借钱,胫骨莫受伤!”} \textcolor{Sepia}{\selectlanguage{english}The poor must not borrow money; the shinbone must not receive wounds! (Proverb, to explain that one must avoid hitting weak/sensitive spots.)} \textcolor{PineGreen}{\selectlanguage{french}“Le pauvre ne doit pas emprunter d'argent; le tibia ne doit pas recevoir de blessure!” (Ce proverbe enseigne qu'il ne faut pas toucher les points les plus sensibles, les plus fragiles.)}  
 \zh{量词}: \textcolor{darkblue}{\textbf{\ipa{kɤ˧˥}}} 
\lhead{\firstmark}
\rhead{\botmark}

\subsection{\hspace{-0.5cm} {\Large \textcolor{darkblue}{\textbf{\ipa{ɻ̃˧mi˧}}}}\hspace{0.5cm}[\kern2pt{\textcolor{darkblue}{\textbf{\ipa{ɻ̃˧mi˧}}}}\kern2pt]} \hypertarget{r£`\string_~\string_Mmi\string_M1}{}
\markboth{\textcolor{darkblue}{\textbf{\ipa{ɻ̃˧mi˧}}}}{}
\textcolor{teal}{\zh{名词}} \hspace{4pt} \zh{声调类:} M.
\zh{树干。} \textcolor{Sepia}{\selectlanguage{english}Tree trunk.} \textcolor{PineGreen}{\selectlanguage{french}Tronc.}  ¶ \textcolor{darkblue}{\textbf{\ipa{si˧dzi˩-ɻ̃˩mi˩}}} \zh{树干} \textcolor{Sepia}{\selectlanguage{english}tree trunk} \textcolor{PineGreen}{\selectlanguage{french}tronc d'arbre}  
 \zh{量词}: \textcolor{darkblue}{\textbf{\ipa{kɤ˧˥}}} 
\lhead{\firstmark}
\rhead{\botmark}

\subsection{\hspace{-0.5cm} {\Large \textcolor{darkblue}{\textbf{\ipa{ɻ̃˧ʈʂæ˩}}}}\hspace{0.5cm}[\kern2pt{\textcolor{darkblue}{\textbf{\ipa{ɻ̃˧ʈʂæ˩}}}}\kern2pt]} \hypertarget{r£`\string_~\string_Mt`s`\{\string_B1}{}
\markboth{\textcolor{darkblue}{\textbf{\ipa{ɻ̃˧ʈʂæ˩}}}}{}
\textcolor{teal}{\zh{名词}} \hspace{4pt} \zh{声调类:} L\#.
\zh{关节部位,关节。} \textcolor{Sepia}{\selectlanguage{english}Ankle, joint (between the foot and the leg, the arm and the hand…).} \textcolor{PineGreen}{\selectlanguage{french}Articulations (de la jambe: la cheville, le genou…; du bras: le poignet, le coude…).}  \zh{量词}: \textcolor{darkblue}{\textbf{\ipa{ʈʂæ˧˥}}} 
\lhead{\firstmark}
\rhead{\botmark}

\subsection{\hspace{-0.5cm} {\Large \textcolor{darkblue}{\textbf{\ipa{ɻ̃˧ʈʂwæ˩}}}}\hspace{0.5cm}[\kern2pt{\textcolor{darkblue}{\textbf{\ipa{ɻ̃˧ʈʂwæ˩}}}}\kern2pt]} \hypertarget{r£`\string_~\string_Mt`s`w\{\string_B1}{}
\markboth{\textcolor{darkblue}{\textbf{\ipa{ɻ̃˧ʈʂwæ˩}}}}{}
\textcolor{teal}{\zh{名词}} \hspace{4pt} \zh{声调类:} L\#.
\zh{接骨丹。} \textcolor{Sepia}{\selectlanguage{english}\textit{Toricellia angulata Oliv.}.} \textcolor{PineGreen}{\selectlanguage{french}\textit{Toricellia angulata Oliv.}.}  ¶ \textcolor{darkblue}{\textbf{\ipa{ɻ̃˧ʈʂwæ˩-si˩}}} \zh{同上} \textcolor{Sepia}{\selectlanguage{english}same meaning} \textcolor{PineGreen}{\selectlanguage{french}même sens}  

\lhead{\firstmark}
\rhead{\botmark}

\newpage
\section*{\centering- \textcolor{darkblue}{\textbf{\ipa{ʁ}}} -}
\subsection{\hspace{-0.5cm} {\Large \textcolor{darkblue}{\textbf{\ipa{ʁɑ˥}}}}\hspace{0.5cm}[\kern2pt{\textcolor{darkblue}{\textbf{\ipa{ʁɑ˥}}}}\kern2pt]} \hypertarget{RA\string_T1}{}
\markboth{\textcolor{darkblue}{\textbf{\ipa{ʁɑ˥}}}}{}
\textcolor{teal}{\zh{名词}} \hspace{4pt} \zh{声调类:} \#H.
\zh{力气。} \textcolor{Sepia}{\selectlanguage{english}Strength.} \textcolor{PineGreen}{\selectlanguage{french}Force.}  ¶ \textcolor{darkblue}{\textbf{\ipa{ʁɑ˧ ʑi˧}}} \zh{有力量} \textcolor{Sepia}{\selectlanguage{english}to have strength} \textcolor{PineGreen}{\selectlanguage{french}avoir de la force}  
 ¶ \textcolor{darkblue}{\textbf{\ipa{no˧ɻ̍˩ | hĩ˧tɕʰi˧ ʁɑ˧ ʑi˧!}}} \zh{你们家族很强大!} \textcolor{Sepia}{\selectlanguage{english}Your family/clan is powerful!} \textcolor{PineGreen}{\selectlanguage{french}votre famille/lignée/tribu est puissante!}  
 ¶ \textcolor{darkblue}{\textbf{\ipa{ʁɑ˧ tʰv̩˧ (+ze˩)}}} \zh{尽力} \textcolor{Sepia}{\selectlanguage{english}to exert oneself, to make efforts} \textcolor{PineGreen}{\selectlanguage{french}faire des efforts, donner toutes ses forces, s'impliquer (dans une tâche)}  

\lhead{\firstmark}
\rhead{\botmark}

\subsection{\hspace{-0.5cm} {\Large \textcolor{darkblue}{\textbf{\ipa{ʁɑ˥}}} \textsubscript{1}}\hspace{0.5cm}[\kern2pt{\textcolor{darkblue}{\textbf{\ipa{ʁɑ˥}}}}\kern2pt]} \hypertarget{RA\string_T1}{}
\markboth{\textcolor{darkblue}{\textbf{\ipa{ʁɑ˥}}} \textsubscript{1}}{}
\textcolor{teal}{\zh{动词}} \hspace{4pt} \zh{声调类:} H.
\zh{请。} \textcolor{Sepia}{\selectlanguage{english}To invite, to call over.} \textcolor{PineGreen}{\selectlanguage{french}Convier, faire venir, inviter.} 
\lhead{\firstmark}
\rhead{\botmark}

\subsection{\hspace{-0.5cm} {\Large \textcolor{darkblue}{\textbf{\ipa{ʁɑ˥}}} \textsubscript{2}}\hspace{0.5cm}[\kern2pt{\textcolor{darkblue}{\textbf{\ipa{ʁɑ˥}}}}\kern2pt]} \hypertarget{RA\string_T2}{}
\markboth{\textcolor{darkblue}{\textbf{\ipa{ʁɑ˥}}} \textsubscript{2}}{}
\textcolor{teal}{\zh{动词}} \hspace{4pt} \zh{声调类:} H.
\zh{赢。} \textcolor{Sepia}{\selectlanguage{english}To win, to succeed.} \textcolor{PineGreen}{\selectlanguage{french}Gagner.}  ¶ \textcolor{darkblue}{\textbf{\ipa{le˧-ʁɑ˥-ze˩}}} \zh{赢了} \textcolor{Sepia}{\selectlanguage{english}\mytextsc{accomp} \string_ \mytextsc{pfv}} \textcolor{PineGreen}{\selectlanguage{french}\mytextsc{accomp} \string_ \mytextsc{pfv}}  

\lhead{\firstmark}
\rhead{\botmark}

\subsection{\hspace{-0.5cm} {\Large \textcolor{darkblue}{\textbf{\ipa{ʁɑ˧}}} \textsubscript{1}}\hspace{0.5cm}[\kern2pt{\textcolor{darkblue}{\textbf{\ipa{ʁɑ˥}}}}\kern2pt]} \hypertarget{RA\string_M1}{}
\markboth{\textcolor{darkblue}{\textbf{\ipa{ʁɑ˧}}} \textsubscript{1}}{}
\textcolor{teal}{\zh{形容词}} \hspace{4pt} \zh{声调类:} M.
\zh{好(质量好,品质好,脾气好)。} \textcolor{Sepia}{\selectlanguage{english}Good (of good quality).} \textcolor{PineGreen}{\selectlanguage{french}Bon, fiable: objet de bonne qualité; travail de bonne tenue; personne ayant bon caractère.}  ¶ \textcolor{darkblue}{\textbf{\ipa{mɤ˧-ʁɑ˧-hĩ˧ ʂe˧}}} \zh{不好的肉(质量不好)} \textcolor{Sepia}{\selectlanguage{english}bad meat, meat of poor quality} \textcolor{PineGreen}{\selectlanguage{french}de la mauvaise viande}  
 ¶ \textcolor{darkblue}{\textbf{\ipa{mɤ˧-ʁɑ˧-hĩ˧ ʂe˧-kʰwɤ˧ ki˩}}} \zh{给一块不好的肉} \textcolor{Sepia}{\selectlanguage{english}to give a piece of bad meat} \textcolor{PineGreen}{\selectlanguage{french}donner un morceau de mauvaise viande}  
 ¶ \textcolor{darkblue}{\textbf{\ipa{pʰi˩ko˧ | mɤ˧-ʁɑ˧-ze˧!}}} \zh{苹果不好了! / 苹果不新鲜了!(三月份,上一季收获的苹果已经不好吃的了,或者烂了,或者变酸)} \textcolor{Sepia}{\selectlanguage{english}The apples are not good anymore! (Context: in March, apples from the previous harvest are not good anymore: they are either rotten or sour.)} \textcolor{PineGreen}{\selectlanguage{french}Les pommes ne sont plus bonnes! (Contexte: au mois de mars, les pommes de la récolte précédente ne sont plus bonnes, elles sont fripées ou pourries.)}  
 ¶ \textcolor{darkblue}{\textbf{\ipa{hĩ˧ ɖɯ˧-v̩˧ | ʁɑ˧-mɤ˧-ʑi˧-hĩ˧ ʐwɤ˧˥!}}} \zh{有人在乱说话!} \textcolor{Sepia}{\selectlanguage{english}Someone is talking nonsense!} \textcolor{PineGreen}{\selectlanguage{french}quelqu'un dit n'importe quoi}  

\lhead{\firstmark}
\rhead{\botmark}

\subsection{\hspace{-0.5cm} {\Large \textcolor{darkblue}{\textbf{\ipa{ʁɑ˧}}} \textsubscript{2}}\hspace{0.5cm}[\kern2pt{\textcolor{darkblue}{\textbf{\ipa{ʁɑ˥}}}}\kern2pt]} \hypertarget{RA\string_M2}{}
\markboth{\textcolor{darkblue}{\textbf{\ipa{ʁɑ˧}}} \textsubscript{2}}{}
\textcolor{teal}{\zh{动词}} \hspace{4pt} \zh{声调类:} M.
\textit{\zh{古语}} [\zh{古语}] \zh{道歉、请人家原谅。} \textcolor{Sepia}{\selectlanguage{english}To ask for forgiveness, to apologize.} \textcolor{PineGreen}{\selectlanguage{french}Présenter ses excuses, demander pardon.}  ¶ \textcolor{darkblue}{\textbf{\ipa{ʁɑ˧-ze˧!}}} \zh{抱歉! / 请原谅!(对地位比自己高的人说)} \textcolor{Sepia}{\selectlanguage{english}Please accept my apologies! (To a person of high rank)} \textcolor{PineGreen}{\selectlanguage{french}Pardon! (adressé à une personne de haut rang)}  

\lhead{\firstmark}
\rhead{\botmark}

\subsection{\hspace{-0.5cm} {\Large \textcolor{darkblue}{\textbf{\ipa{ʁɑ˧\textsubscript{b}}}}}\hspace{0.5cm}[\kern2pt{\textcolor{darkblue}{\textbf{\ipa{ʁɑ˥}}}}\kern2pt]} \hypertarget{RA\string_Mb1}{}
\markboth{\textcolor{darkblue}{\textbf{\ipa{ʁɑ˧\textsubscript{b}}}}}{}
\textcolor{teal}{\zh{动词}} \hspace{4pt} \zh{声调类:} M\textsubscript{b}.
\zh{跨(跨过小沟)。} \textcolor{Sepia}{\selectlanguage{english}To stride over (an obstacle), to straddle, to go beyond.} \textcolor{PineGreen}{\selectlanguage{french}Enjamber (un ruisseau...).}  ¶ \textcolor{darkblue}{\textbf{\ipa{le˧-ʁɑ˧-ze˧}}} \zh{跨过了} \textcolor{Sepia}{\selectlanguage{english}\mytextsc{accomp} \string_ \mytextsc{pfv}} \textcolor{PineGreen}{\selectlanguage{french}\mytextsc{accomp} \string_ \mytextsc{pfv}}  

\lhead{\firstmark}
\rhead{\botmark}

\subsection{\hspace{-0.5cm} {\Large \textcolor{darkblue}{\textbf{\ipa{ʁɑ˧dzi˧}}}}\hspace{0.5cm}[\kern2pt{\textcolor{darkblue}{\textbf{\ipa{ʁɑ˧dzi˧}}}}\kern2pt]} \hypertarget{RA\string_Mdzi\string_M1}{}
\markboth{\textcolor{darkblue}{\textbf{\ipa{ʁɑ˧dzi˧}}}}{}
\textcolor{teal}{\zh{名词}} \hspace{4pt} \zh{声调类:} M.
\zh{杨树。} \textcolor{Sepia}{\selectlanguage{english}Poplar.} \textcolor{PineGreen}{\selectlanguage{french}Peuplier.}  \zh{量词}: \textcolor{darkblue}{\textbf{\ipa{dzi˩}}} 
\lhead{\firstmark}
\rhead{\botmark}

\subsection{\hspace{-0.5cm} {\Large \textcolor{darkblue}{\textbf{\ipa{ʁɑ˧ɭɯ\#˥}}}}\hspace{0.5cm}[\kern2pt{\textcolor{darkblue}{\textbf{\ipa{ʁɑ˧ɭɯ˧}}}}\kern2pt]} \hypertarget{RA\string_Ml\string_RM\#\string_T1}{}
\markboth{\textcolor{darkblue}{\textbf{\ipa{ʁɑ˧ɭɯ\#˥}}}}{}
\textcolor{teal}{\zh{名词}} \hspace{4pt} \zh{声调类:} \#H.
\zh{石堆。} \textcolor{Sepia}{\selectlanguage{english}Cairn: a human-made pile of stones, used as trail marker. The stones of the cairn are not engraved.} \textcolor{PineGreen}{\selectlanguage{french}Cairn: tas de pierres qui aide à repérer un sentier.}  ¶ \textcolor{darkblue}{\textbf{\ipa{qo˩qɑ˩-ʁɑ˥ɭɯ˩}}} \zh{垭口石堆:垭口上的石堆} \textcolor{Sepia}{\selectlanguage{english}mountain pass cairn: a cairn at a mountain pass} \textcolor{PineGreen}{\selectlanguage{french}cairn situé à un col}  
 \zh{量词}: \textcolor{darkblue}{\textbf{\ipa{ɭɯ˧}}} 
\lhead{\firstmark}
\rhead{\botmark}

\subsection{\hspace{-0.5cm} {\Large \textcolor{darkblue}{\textbf{\ipa{ʁɑ˧pv̩˧}}}}\hspace{0.5cm}[\kern2pt{\textcolor{darkblue}{\textbf{\ipa{ʁɑ˧pv̩˧}}}}\kern2pt]} \hypertarget{RA\string_Mpv\string_=\string_M1}{}
\markboth{\textcolor{darkblue}{\textbf{\ipa{ʁɑ˧pv̩˧}}}}{}
\textcolor{teal}{\zh{名词}} \hspace{4pt} \zh{声调类:} M.
\zh{胸脯、胸膛。} \textcolor{Sepia}{\selectlanguage{english}Chest.} \textcolor{PineGreen}{\selectlanguage{french}Poitrine.}  \zh{量词}: \textcolor{darkblue}{\textbf{\ipa{ʈv̩˩}}} 
\lhead{\firstmark}
\rhead{\botmark}

\subsection{\hspace{-0.5cm} {\Large \textcolor{darkblue}{\textbf{\ipa{ʁɑ˧pv̩˧-ɻ̃\#˥}}}}\hspace{0.5cm}[\kern2pt{\textcolor{darkblue}{\textbf{\ipa{xxxx non-correspondance entre le nombre de morphèmes et le nombre de tons de morphèmes}}}}\kern2pt]} \hypertarget{RA\string_Mpv\string_=\string_M-r£`\string_~\#\string_T1}{}
\markboth{\textcolor{darkblue}{\textbf{\ipa{ʁɑ˧pv̩˧-ɻ̃\#˥}}}}{}
\textcolor{teal}{\zh{名词}} \hspace{4pt} \zh{声调类:} \#H.
\zh{锁骨。} \textcolor{Sepia}{\selectlanguage{english}Clavicle; collarbone.} \textcolor{PineGreen}{\selectlanguage{french}Clavicule.}  \zh{量词}: \textcolor{darkblue}{\textbf{\ipa{pʰæ˧˥}}} 
\lhead{\firstmark}
\rhead{\botmark}

\subsection{\hspace{-0.5cm} {\Large \textcolor{darkblue}{\textbf{\ipa{ʁɑ˧pʰv̩\#˥}}}}\hspace{0.5cm}[\kern2pt{\textcolor{darkblue}{\textbf{\ipa{ʁɑ˧pʰv̩˧}}}}\kern2pt]} \hypertarget{RA\string_Mp\string_hv\string_=\#\string_T1}{}
\markboth{\textcolor{darkblue}{\textbf{\ipa{ʁɑ˧pʰv̩\#˥}}}}{}
\textcolor{teal}{\zh{名词}} \hspace{4pt} \zh{声调类:} \#H.
\zh{工资, 工钱。} \textcolor{Sepia}{\selectlanguage{english}Salary, price paid for the work done by a worker.} \textcolor{PineGreen}{\selectlanguage{french}Salaire, littéralement “prix du travail”.}  \zh{量词}: \textcolor{darkblue}{\textbf{\ipa{kʰwɤ˥}}} 
\lhead{\firstmark}
\rhead{\botmark}

\subsection{\hspace{-0.5cm} {\Large \textcolor{darkblue}{\textbf{\ipa{ʁɑ˧-ʐwɤ˧˥}}}}\hspace{0.5cm}[\kern2pt{\textcolor{darkblue}{\textbf{\ipa{xxxx non-correspondance entre le nombre de morphèmes et le nombre de tons de morphèmes}}}}\kern2pt]} \hypertarget{RA\string_M-z`w7\string_M\string_T1}{}
\markboth{\textcolor{darkblue}{\textbf{\ipa{ʁɑ˧-ʐwɤ˧˥}}}}{}
\textcolor{teal}{\zh{动词}} \hspace{4pt} \zh{声调类:} MH\#.
\zh{欺负。} \textcolor{Sepia}{\selectlanguage{english}To browbeat; to take advantage of; to pick on.} \textcolor{PineGreen}{\selectlanguage{french}Provoquer/humilier.}  ¶ \textcolor{darkblue}{\textbf{\ipa{ʈʂʰɯ˧-v̩˧ | hĩ˧-ki˧ ʁɑ˧-ʐwɤ˧-ʝi˥!}}} \zh{他欺负人、他对人发脾气} \textcolor{Sepia}{\selectlanguage{english}(s)he is picking on someone} \textcolor{PineGreen}{\selectlanguage{french}elle/il provoque/humilie quelqu'un d'autre}  
 ¶ \textcolor{darkblue}{\textbf{\ipa{ʁɑ˧ ʐwɤ˧-ɻ̍˥}}} \zh{同上} \textcolor{Sepia}{\selectlanguage{english}as above} \textcolor{PineGreen}{\selectlanguage{french}même sens}  
 ¶ \textcolor{darkblue}{\textbf{\ipa{no˧ | ʁɑ˧ ʐwɤ˧-tʰɑ˧-ɻ̍˥!}}} \zh{你不要欺负人!} \textcolor{Sepia}{\selectlanguage{english}Don't browbeat people!} \textcolor{PineGreen}{\selectlanguage{french}Ne provoque pas les gens!}  

\lhead{\firstmark}
\rhead{\botmark}

\subsection{\hspace{-0.5cm} {\Large \textcolor{darkblue}{\textbf{\ipa{ʁɑ˩mi˥}}}}\hspace{0.5cm}[\kern2pt{\textcolor{darkblue}{\textbf{\ipa{ʁɑ˩mi˥}}}}\kern2pt]} \hypertarget{RA\string_Bmi\string_T1}{}
\markboth{\textcolor{darkblue}{\textbf{\ipa{ʁɑ˩mi˥}}}}{}
\textcolor{teal}{\zh{动词}} \hspace{4pt} \zh{声调类:} LH.
\zh{道歉。} \textcolor{Sepia}{\selectlanguage{english}To apologize.} \textcolor{PineGreen}{\selectlanguage{french}Demander pardon; formule de requête, et de remerciement. Le spectre des significations rappelle l'étymologie de “merci”: de “crier merci” (implorer la vie sauve) à un emploi comme formule de politesse courante.}  ¶ \textcolor{darkblue}{\textbf{\ipa{ʁɑ˩mi˥-ze˩!}}} \zh{谢谢!} \textcolor{Sepia}{\selectlanguage{english}Thank you!} \textcolor{PineGreen}{\selectlanguage{french}Merci!}  

\lhead{\firstmark}
\rhead{\botmark}

\subsection{\hspace{-0.5cm} {\Large \textcolor{darkblue}{\textbf{\ipa{ʁɑ˩ʂɯ˧}}}}\hspace{0.5cm}[\kern2pt{\textcolor{darkblue}{\textbf{\ipa{ʁɑ˩ʂɯ˥}}}}\kern2pt]} \hypertarget{RA\string_Bs`M\string_M1}{}
\markboth{\textcolor{darkblue}{\textbf{\ipa{ʁɑ˩ʂɯ˧}}}}{}
\textcolor{teal}{\zh{助词}} \hspace{4pt} \zh{声调类:} LM.
\zh{其实、事实上。} \textcolor{Sepia}{\selectlanguage{english}In fact.} \textcolor{PineGreen}{\selectlanguage{french}En fait, en réalité.} 
\lhead{\firstmark}
\rhead{\botmark}

\subsection{\hspace{-0.5cm} {\Large \textcolor{darkblue}{\textbf{\ipa{ʁæ˥}}}}\hspace{0.5cm}[\kern2pt{\textcolor{darkblue}{\textbf{\ipa{ʁæ˥}}}}\kern2pt]} \hypertarget{R\{\string_T1}{}
\markboth{\textcolor{darkblue}{\textbf{\ipa{ʁæ˥}}}}{}
\textcolor{teal}{\zh{名词}} \hspace{4pt} \zh{声调类:} \#H.
\zh{脖子(单音节)。} \textcolor{Sepia}{\selectlanguage{english}Neck (monosyllable).} \textcolor{PineGreen}{\selectlanguage{french}Cou (monosyllabe; moins usité que le disyllabe).}  \zh{量词}: \textcolor{darkblue}{\textbf{\ipa{ɭɯ˧}}} \zh{~【参考】~} \hyperlink{}{\textcolor{darkblue}{\textbf{\ipa{ʁæ˧ŋv̩˥}}}} 
\lhead{\firstmark}
\rhead{\botmark}

\subsection{\hspace{-0.5cm} {\Large \textcolor{darkblue}{\textbf{\ipa{ʁæ˧}}}}\hspace{0.5cm}[\kern2pt{\textcolor{darkblue}{\textbf{\ipa{ʁæ˥}}}}\kern2pt]} \hypertarget{R\{\string_M1}{}
\markboth{\textcolor{darkblue}{\textbf{\ipa{ʁæ˧}}}}{}
\textcolor{teal}{\zh{形容词}} \hspace{4pt} \zh{声调类:} M.
\zh{富。} \textcolor{Sepia}{\selectlanguage{english}Rich.} \textcolor{PineGreen}{\selectlanguage{french}Riche.} 
\lhead{\firstmark}
\rhead{\botmark}

\subsection{\hspace{-0.5cm} {\Large \textcolor{darkblue}{\textbf{\ipa{ʁæ˧bæ˧}}}}\hspace{0.5cm}[\kern2pt{\textcolor{darkblue}{\textbf{\ipa{ʁæ˧bæ˧}}}}\kern2pt]} \hypertarget{R\{\string_Mb\{\string_M1}{}
\markboth{\textcolor{darkblue}{\textbf{\ipa{ʁæ˧bæ˧}}}}{}
\textcolor{teal}{\zh{名词}} \hspace{4pt} \zh{声调类:} M.
\zh{盘子。} \textcolor{Sepia}{\selectlanguage{english}Dish, plate.} \textcolor{PineGreen}{\selectlanguage{french}Assiette.}  \zh{量词}: \textcolor{darkblue}{\textbf{\ipa{ɭɯ˧}}} 
\lhead{\firstmark}
\rhead{\botmark}

\subsection{\hspace{-0.5cm} {\Large \textcolor{darkblue}{\textbf{\ipa{ʁæ˧ɭɯ˥}}}}\hspace{0.5cm}[\kern2pt{\textcolor{darkblue}{\textbf{\ipa{ʁæ˧ɭɯ˥}}}}\kern2pt]} \hypertarget{R\{\string_Ml\string_RM\string_T1}{}
\markboth{\textcolor{darkblue}{\textbf{\ipa{ʁæ˧ɭɯ˥}}}}{}
\textcolor{teal}{\zh{名词}} \hspace{4pt} \zh{声调类:} H\#.
\zh{枷锁(木头做的)。} \textcolor{Sepia}{\selectlanguage{english}Fetters (wooden fetters, around the neck); yoke.} \textcolor{PineGreen}{\selectlanguage{french}Carcan (était en bois); littéralement “[objet dans lequel] on met le cou”.}  ¶ \textcolor{darkblue}{\textbf{\ipa{ʁæ˧ɭɯ˥ | ɖɯ˧-ɭɯ˧ kʰɯ˧˥}}} \zh{套上一个枷锁(在一个人的脖子上)} \textcolor{Sepia}{\selectlanguage{english}to put fetters (on someone's neck)} \textcolor{PineGreen}{\selectlanguage{french}mettre un joug (à quelqu'un)}  
 ¶ \textcolor{darkblue}{\textbf{\ipa{ʁæ˧ɭɯ˥ kʰɯ˩}}} \zh{套上枷锁(在一个人的脖子上)} \textcolor{Sepia}{\selectlanguage{english}to put fetters (on someone's neck)} \textcolor{PineGreen}{\selectlanguage{french}mettre le joug (à quelqu'un)}  

\lhead{\firstmark}
\rhead{\botmark}

\subsection{\hspace{-0.5cm} {\Large \textcolor{darkblue}{\textbf{\ipa{ʁæ˧mi˧}}}}\hspace{0.5cm}[\kern2pt{\textcolor{darkblue}{\textbf{\ipa{ʁæ˧mi˧}}}}\kern2pt]} \hypertarget{R\{\string_Mmi\string_M1}{}
\markboth{\textcolor{darkblue}{\textbf{\ipa{ʁæ˧mi˧}}}}{}
\textcolor{teal}{\zh{名词}} \hspace{4pt} \zh{声调类:} M.
\zh{剑。} \textcolor{Sepia}{\selectlanguage{english}Sword.} \textcolor{PineGreen}{\selectlanguage{french}Épée.}  \zh{量词}: \textcolor{darkblue}{\textbf{\ipa{nɑ˧}}} 
\lhead{\firstmark}
\rhead{\botmark}

\subsection{\hspace{-0.5cm} {\Large \textcolor{darkblue}{\textbf{\ipa{ʁæ˧ŋv̩˥}}}}\hspace{0.5cm}[\kern2pt{\textcolor{darkblue}{\textbf{\ipa{ʁæ˧ŋv̩˥}}}}\kern2pt]} \hypertarget{R\{\string_MNv\string_=\string_T1}{}
\markboth{\textcolor{darkblue}{\textbf{\ipa{ʁæ˧ŋv̩˥}}}}{}
\textcolor{teal}{\zh{名词}} \hspace{4pt} \zh{声调类:} H\#.
\zh{银衣领。} \textcolor{Sepia}{\selectlanguage{english}Silver-embellished collar (a precious part of the dress, with silver thread).} \textcolor{PineGreen}{\selectlanguage{french}Col (précieux, avec des fils d'argent).}  \zh{量词}: \textcolor{darkblue}{\textbf{\ipa{ɭɯ˧}}} 
\lhead{\firstmark}
\rhead{\botmark}

\subsection{\hspace{-0.5cm} {\Large \textcolor{darkblue}{\textbf{\ipa{ʁæ˧ɻ̍˥}}}}\hspace{0.5cm}[\kern2pt{\textcolor{darkblue}{\textbf{\ipa{ʁæ˧ɻ̍˥}}}}\kern2pt]} \hypertarget{R\{\string_Mr£`̍\string_T1}{}
\markboth{\textcolor{darkblue}{\textbf{\ipa{ʁæ˧ɻ̍˥}}}}{}
\textcolor{teal}{\zh{名词}} \hspace{4pt} \zh{声调类:} H\#.
\zh{脖子。} \textcolor{Sepia}{\selectlanguage{english}Neck.} \textcolor{PineGreen}{\selectlanguage{french}Cou.}  \zh{量词}: \textcolor{darkblue}{\textbf{\ipa{ɭɯ˧}}} \zh{~【参考】~} \hyperlink{}{\textcolor{darkblue}{\textbf{\ipa{ʁæ˧ʈv̩˥}}}} 
\lhead{\firstmark}
\rhead{\botmark}

\subsection{\hspace{-0.5cm} {\Large \textcolor{darkblue}{\textbf{\ipa{ʁæ˧tɑ˩}}}}\hspace{0.5cm}[\kern2pt{\textcolor{darkblue}{\textbf{\ipa{ʁæ˧tɑ˩}}}}\kern2pt]} \hypertarget{R\{\string_MtA\string_B1}{}
\markboth{\textcolor{darkblue}{\textbf{\ipa{ʁæ˧tɑ˩}}}}{}
\textcolor{teal}{\zh{名词}} \hspace{4pt} \zh{声调类:} L\#.
\zh{肩隆。} \textcolor{Sepia}{\selectlanguage{english}Withers: part of the ox's body on which the yoke rests.} \textcolor{PineGreen}{\selectlanguage{french}Garrot: partie du corps de l'animal sur lequel repose le joug.}  ¶ \textcolor{darkblue}{\textbf{\ipa{ʝi˧-ʁæ˧tɑ˥}}} \zh{牛肩隆} \textcolor{Sepia}{\selectlanguage{english}ox's withers} \textcolor{PineGreen}{\selectlanguage{french}garrot de vache}  
 ¶ \textcolor{darkblue}{\textbf{\ipa{ʁæ˧tɑ˩ tʰv̩˩-ɭɯ˩}}} \zh{这只肩隆} \textcolor{Sepia}{\selectlanguage{english}\mytextsc{n}+\mytextsc{dem}+\mytextsc{clf}} \textcolor{PineGreen}{\selectlanguage{french}\mytextsc{n}+\mytextsc{dem}+\mytextsc{clf}}  
 \zh{量词}: \textcolor{darkblue}{\textbf{\ipa{ɭɯ˧}}} 
\lhead{\firstmark}
\rhead{\botmark}

\subsection{\hspace{-0.5cm} {\Large \textcolor{darkblue}{\textbf{\ipa{ʁæ˧ʈv̩˥}}}}\hspace{0.5cm}[\kern2pt{\textcolor{darkblue}{\textbf{\ipa{ʁæ˧ʈv̩˥}}}}\kern2pt]} \hypertarget{R\{\string_Mt`v\string_=\string_T1}{}
\markboth{\textcolor{darkblue}{\textbf{\ipa{ʁæ˧ʈv̩˥}}}}{}
\textcolor{teal}{\zh{名词}} \hspace{4pt} \zh{声调类:} H\#.
\zh{脖子。} \textcolor{Sepia}{\selectlanguage{english}Neck.} \textcolor{PineGreen}{\selectlanguage{french}Cou.}  \zh{量词}: \textcolor{darkblue}{\textbf{\ipa{ɭɯ˧}}} \zh{~【参考】~} \hyperlink{}{\textcolor{darkblue}{\textbf{\ipa{ʁæ˧ɻ̍˥}}}} 
\lhead{\firstmark}
\rhead{\botmark}

\subsection{\hspace{-0.5cm} {\Large \textcolor{darkblue}{\textbf{\ipa{ʁæ˧zo\#˥}}}}\hspace{0.5cm}[\kern2pt{\textcolor{darkblue}{\textbf{\ipa{ʁæ˧zo˧}}}}\kern2pt]} \hypertarget{R\{\string_Mzo\#\string_T1}{}
\markboth{\textcolor{darkblue}{\textbf{\ipa{ʁæ˧zo\#˥}}}}{}
\textcolor{teal}{\zh{名词}} \hspace{4pt} \zh{声调类:} \#H.
\zh{短剑。} \textcolor{Sepia}{\selectlanguage{english}Short sword.} \textcolor{PineGreen}{\selectlanguage{french}Petite épée.} 
\lhead{\firstmark}
\rhead{\botmark}

\subsection{\hspace{-0.5cm} {\Large \textcolor{darkblue}{\textbf{\ipa{ʁæ˧ʑi˧}}}}\hspace{0.5cm}[\kern2pt{\textcolor{darkblue}{\textbf{\ipa{ʁæ˧ʑi˧}}}}\kern2pt]} \hypertarget{R\{\string_Mz£i\string_M1}{}
\markboth{\textcolor{darkblue}{\textbf{\ipa{ʁæ˧ʑi˧}}}}{}
\textcolor{teal}{\zh{动词}} \hspace{4pt} \zh{声调类:} M.
\zh{考虑。} \textcolor{Sepia}{\selectlanguage{english}To mind something; to take into account; to take into consideration; to care about.} \textcolor{PineGreen}{\selectlanguage{french}S'occuper de; se mêler de; prendre en considération.}  ¶ \textcolor{darkblue}{\textbf{\ipa{hĩ˧ | qʰɑ˧-kv̩˧ dʑo˧˥ | mɤ˧-ʁæ˧ʑi˧, | njɤ˧-ɳɯ˧ qʰæ˧˥! |}}} \zh{无论有多少个人,我都会去帮助!(情景:合作人描写她在永宁有大事时怎么去帮其它家庭的忙,不考虑活多么累,只考虑怎么能给予帮助)} \textcolor{Sepia}{\selectlanguage{english}No matter how many people (guests) there are, I (go to participate and) help! (Context: the consultant explains how, following Na traditions, she volunteers her time to help on important occasions, such as funerals, to help other families.)} \textcolor{PineGreen}{\selectlanguage{french}Moi, j'aide, sans m'inquiéter de savoir combien il y a d'invités (littéralement “de gens”)! (contexte: F4 explique comment on se dévouait autrefois pour aider les amis, non membres de la famille, lors des grandes occasions, telles que les funérailles)}  
 ¶ \textcolor{darkblue}{\textbf{\ipa{no˧ | mɤ˧-ʁæ˧ʑi˧!}}} \zh{别管我了! / 请让我安静! / 请不要打扰我了!} \textcolor{Sepia}{\selectlanguage{english}Leave me alone! / Leave me in peace! / Mind your own business!} \textcolor{PineGreen}{\selectlanguage{french}Fiche-moi la paix! / Laisse-moi tranquille! / Mêle-toi de tes affaires!}  

\lhead{\firstmark}
\rhead{\botmark}

\subsection{\hspace{-0.5cm} {\Large \textcolor{darkblue}{\textbf{\ipa{ʁæ˩\textsubscript{a}}}} \textsubscript{1}}\hspace{0.5cm}[\kern2pt{\textcolor{darkblue}{\textbf{\ipa{ʁæ˩˥}}}}\kern2pt]} \hypertarget{R\{\string_Ba1}{}
\markboth{\textcolor{darkblue}{\textbf{\ipa{ʁæ˩\textsubscript{a}}}} \textsubscript{1}}{}
\textcolor{teal}{\zh{动词}} \hspace{4pt} \zh{声调类:} L\textsubscript{a}.
\zh{散、散开,化,溶化(一块土在水里面散开)。} \textcolor{Sepia}{\selectlanguage{english}To fall apart, to scatter, to melt (e.g. clods of dry earth melting in water when a field is irrigated after ploughing).} \textcolor{PineGreen}{\selectlanguage{french}Se défaire, fondre, se dissoudre: une motte de terre plongée dans l'eau se défait.}  ¶ \textcolor{darkblue}{\textbf{\ipa{le˧-ʁæ˩-ze˩}}} \zh{\mytextsc{accomp} \string_ \mytextsc{pfv}} \textcolor{Sepia}{\selectlanguage{english}\mytextsc{accomp} \string_ \mytextsc{pfv}} \textcolor{PineGreen}{\selectlanguage{french}\mytextsc{accomp} \string_ \mytextsc{pfv}}  
 ¶ \textcolor{darkblue}{\textbf{\ipa{le˧-ʁæ˧\textasciitilde{}ʁæ˥ (-ze˩ / -bi˩)}}} \zh{\mytextsc{重叠}} \textcolor{Sepia}{\selectlanguage{english}\mytextsc{red}} \textcolor{PineGreen}{\selectlanguage{french}\mytextsc{red}}  
 ¶ \textcolor{darkblue}{\textbf{\ipa{ɖɯ˧-kʰwɤ˧ ʁæ˥}}} \zh{一块(土)散开} \textcolor{Sepia}{\selectlanguage{english}a lump (of earth) melts} \textcolor{PineGreen}{\selectlanguage{french}un morceau (de terre) se défait}  
 ¶ \textcolor{darkblue}{\textbf{\ipa{ʈʂe˧ʈv̩˥ | le˧-ʁæ˩-ze˩}}} \zh{土块散开在了(耕田后灌溉,土块散在水里)} \textcolor{Sepia}{\selectlanguage{english}Clods of earth fall apart (after ploughing, the fields are irrigated; clods of earth melt into the water)} \textcolor{PineGreen}{\selectlanguage{french}les mottes de terre se défond, se dissolvent (dans l'eau dont on inonde les champs après les labours)}  

\lhead{\firstmark}
\rhead{\botmark}

\subsection{\hspace{-0.5cm} {\Large \textcolor{darkblue}{\textbf{\ipa{ʁæ˩\textsubscript{a}}}} \textsubscript{2}}\hspace{0.5cm}[\kern2pt{\textcolor{darkblue}{\textbf{\ipa{ʁæ˩˥}}}}\kern2pt]} \hypertarget{R\{\string_Ba2}{}
\markboth{\textcolor{darkblue}{\textbf{\ipa{ʁæ˩\textsubscript{a}}}} \textsubscript{2}}{}
\textcolor{teal}{\zh{形容词}} \hspace{4pt} \zh{声调类:} L\textsubscript{a}.
\zh{醉。} \textcolor{Sepia}{\selectlanguage{english}Drunk.} \textcolor{PineGreen}{\selectlanguage{french}Ivre, saoul.}  ¶ \textcolor{darkblue}{\textbf{\ipa{ʐɯ˧ ʁæ˩(-ze˩)}}} \zh{醉酒} \textcolor{Sepia}{\selectlanguage{english}drunk} \textcolor{PineGreen}{\selectlanguage{french}ivre d'alcool}  

\lhead{\firstmark}
\rhead{\botmark}

\subsection{\hspace{-0.5cm} {\Large \textcolor{darkblue}{\textbf{\ipa{ʁæ˩\textsubscript{a}}}} \textsubscript{3}}\hspace{0.5cm}[\kern2pt{\textcolor{darkblue}{\textbf{\ipa{ʁæ˩˥}}}}\kern2pt]} \hypertarget{R\{\string_Ba3}{}
\markboth{\textcolor{darkblue}{\textbf{\ipa{ʁæ˩\textsubscript{a}}}} \textsubscript{3}}{}
\textcolor{teal}{\zh{形容词}} \hspace{4pt} \zh{声调类:} L\textsubscript{a}.
\zh{合适,吉利。} \textcolor{Sepia}{\selectlanguage{english}Appropriate; auspitious.} \textcolor{PineGreen}{\selectlanguage{french}Approprié; propice, favorable.}  ¶ \textcolor{darkblue}{\textbf{\ipa{ʁæ˧ mɤ˧-ʑi˧}}} \zh{不吉利、不合适} \textcolor{Sepia}{\selectlanguage{english}not propicious / not favourable} \textcolor{PineGreen}{\selectlanguage{french}ce n'est pas propice/favorable}  
 ¶ \textcolor{darkblue}{\textbf{\ipa{ʁæ˧ mɤ˧-ʑi˧, | ʝi˧ mɤ˧-tʰɑ˩! / ʝi˧-mɤ˧-ɖo˧!}}} \zh{不吉利 / 不合适(的事情),不能做!/ 不要做!(警告)} \textcolor{Sepia}{\selectlanguage{english}It is not appropriate / the situation is not propitious; it must not / should not be done! (A phrase to caution others against doing something)} \textcolor{PineGreen}{\selectlanguage{french}les circonstances ne sont pas propices / ce n'est pas une bonne idée, il ne faut pas le faire! (Mise en garde)}  
 ¶ \textcolor{darkblue}{\textbf{\ipa{ʁæ˧ mɤ˧-ʑi˧, | ʐwɤ˩ mɤ˩-tʰɑ˥! / ʁæ˧ mɤ˧-ʑi˧, | ʐwɤ˩ mɤ˩-ɖo˩˥!}}} \zh{不合适(的话),不能说! / 不合适(的话),不要说!(警告)} \textcolor{Sepia}{\selectlanguage{english}It's not appropriate; one must not talk about it! / One should not talk nonsense! (A phrase to caution others against being carelessly talkative)} \textcolor{PineGreen}{\selectlanguage{french}Ca ne convient pas; on ne peut pas le dire / on ne doit pas le dire! / Il ne faut pas tenir de propos inappropriés! / Il faut faire attention à ce qu'on dit! (Mise en garde)}  
 ¶ \textcolor{darkblue}{\textbf{\ipa{ʁæ˧-mɤ˧-ʑi˧, | tɕi˩-mɤ˩-ɖo˩˥!}}} \zh{乱七八糟的,不要记录! / 不好的,不要记录!(情景:选择一个故事来做记音翻译等。合作人提出,要考虑好记录哪些、选择好的资料,不能什么都记录。)} \textcolor{Sepia}{\selectlanguage{english}(You) must not transcribe the bad ones! / You must not transcribe the messy ones! (Context: the investigator was explaining his wish to choose, among the wealth of recorded narratives, those that are the most interesting and successful, to do a transcription and complete translation and annotation. By her answer, the consultant indicates her approval, at the same time as she shows her understanding of the idea: any materials that may be inappropriate in any way should be left out, and not put to writing.)} \textcolor{PineGreen}{\selectlanguage{french}Il ne faut pas transcrire ceux qui sont pas bons! Il ne faut pas écrire n'importe quoi! (Contexte: cette phrase récapitule le principe qui préside au choix des récits à transcrire et traduire. J'expliquais de mon mieux mon souhait de choisir, parmi les récits enregistrés --relativement nombreux--, ceux qui sont les plus intéressants, et les plus réussis. Par cet énoncé, la locutrice apporte son assentiment, en même temps qu'elle indique qu'elle comprend l'idée: il faut transcrire les récits qui sont bons; il faut bien choisir, et écarter ceux qui ne seraient pas appropriés en quoi que ce soit. Par le même énoncé, la locutrice témoigne en outre de sa modestie: elle ne défend pas l'idée selon laquelle tous ses récits sont d'égale qualité, et accepte de bonne grâce l'idée que certains sont plus réussis que d'autres, ou plus adéquats pour le propos du linguiste.)}  
 ¶ \textcolor{darkblue}{\textbf{\ipa{ʈʂʰɯ˧ | lo˧ | ʁæ˧-mɤ˧-ʑi˧ ʝi˧!}}} \zh{他工作做得乱七八糟!} \textcolor{Sepia}{\selectlanguage{english}He does a bad job of it! / He makes a mess of his work!} \textcolor{PineGreen}{\selectlanguage{french}Il ne fait pas attention dans son travail! il ne travaille pas avec soin! il fait n'importe quoi!}  

\lhead{\firstmark}
\rhead{\botmark}

\subsection{\hspace{-0.5cm} {\Large \textcolor{darkblue}{\textbf{\ipa{ʁæ˩\textsubscript{a}}}} \textsubscript{4}}\hspace{0.5cm}[\kern2pt{\textcolor{darkblue}{\textbf{\ipa{ʁæ˩˥}}}}\kern2pt]} \hypertarget{R\{\string_Ba4}{}
\markboth{\textcolor{darkblue}{\textbf{\ipa{ʁæ˩\textsubscript{a}}}} \textsubscript{4}}{}
\textcolor{teal}{\zh{形容词}} \hspace{4pt} \zh{声调类:} L\textsubscript{a}.
\zh{丑陋。} \textcolor{Sepia}{\selectlanguage{english}Ugly.} \textcolor{PineGreen}{\selectlanguage{french}Laid, vilain.} \zh{~【参考】~} \hyperlink{}{\textcolor{darkblue}{\textbf{\ipa{ɖʐv̩˩\textsubscript{a}}}} \textsubscript{1}} 
\lhead{\firstmark}
\rhead{\botmark}

\subsection{\hspace{-0.5cm} {\Large \textcolor{darkblue}{\textbf{\ipa{ʁæ˧˥}}}}\hspace{0.5cm}[\kern2pt{\textcolor{darkblue}{\textbf{\ipa{ʁæ˧˥}}}}\kern2pt]} \hypertarget{R\{\string_M\string_T1}{}
\markboth{\textcolor{darkblue}{\textbf{\ipa{ʁæ˧˥}}}}{}
\textcolor{teal}{\zh{形容词}} \hspace{4pt} \zh{声调类:} MH.
\zh{不好吃,恶心。} \textcolor{Sepia}{\selectlanguage{english}Nauseous, disgusting.} \textcolor{PineGreen}{\selectlanguage{french}Écoeurant, dégoûtant, pas bon au goût (pas forcément à cause d'un excès de graisse: par exemple, selon les critères gastronomiques locaux, mes flocons d'avoine entrent dans cette catégorie).} 
\lhead{\firstmark}
\rhead{\botmark}

\subsection{\hspace{-0.5cm} {\Large \textcolor{darkblue}{\textbf{\ipa{ʁæ˩˥}}}}\hspace{0.5cm}[\kern2pt{\textcolor{darkblue}{\textbf{\ipa{ʁæ˩˥}}}}\kern2pt]} \hypertarget{R\{\string_B\string_T1}{}
\markboth{\textcolor{darkblue}{\textbf{\ipa{ʁæ˩˥}}}}{}
\textcolor{teal}{\zh{名词}} \hspace{4pt} \zh{声调类:} LH.
\zh{树液。} \textcolor{Sepia}{\selectlanguage{english}Sap.} \textcolor{PineGreen}{\selectlanguage{french}Sève, résine.}  ¶ \textcolor{darkblue}{\textbf{\ipa{tʰo˩ʁæ˩˥}}} \zh{同上} \textcolor{Sepia}{\selectlanguage{english}same meaning} \textcolor{PineGreen}{\selectlanguage{french}même sens}  

\lhead{\firstmark}
\rhead{\botmark}

\subsection{\hspace{-0.5cm} {\Large \textcolor{darkblue}{\textbf{\ipa{ʁo˥}}}}\hspace{0.5cm}[\kern2pt{\textcolor{darkblue}{\textbf{\ipa{ʁo˥}}}}\kern2pt]} \hypertarget{Ro\string_T1}{}
\markboth{\textcolor{darkblue}{\textbf{\ipa{ʁo˥}}}}{}
\textcolor{teal}{\zh{名词}} \hspace{4pt} \zh{声调类:} \#H.
\ding{202} \zh{头(单音节)。} \textcolor{Sepia}{\selectlanguage{english}Head (monosyllable).} \textcolor{PineGreen}{\selectlanguage{french}Tête (monosyllabique).}  \zh{量词}: \textcolor{darkblue}{\textbf{\ipa{ɭɯ˧}}} \ding{203} \zh{开头。} \textcolor{Sepia}{\selectlanguage{english}Beginning.} \textcolor{PineGreen}{\selectlanguage{french}Début.}  ¶ \textcolor{darkblue}{\textbf{\ipa{ɬi˧-ʁo\#˥}}} \zh{月初} \textcolor{Sepia}{\selectlanguage{english}the beginning of the month, the first days of the month} \textcolor{PineGreen}{\selectlanguage{french}le début du mois}  
 ¶ \textcolor{darkblue}{\textbf{\ipa{kʰv̩˧-ʁo˥\$}}} \zh{年初} \textcolor{Sepia}{\selectlanguage{english}the beginning of the year} \textcolor{PineGreen}{\selectlanguage{french}le début de l'année}  
 ¶ \textcolor{darkblue}{\textbf{\ipa{*ɲi˧-ʁo˩}}} \zh{*天初} \textcolor{Sepia}{\selectlanguage{english}*the beginning of the day} \textcolor{PineGreen}{\selectlanguage{french}*le début de la journée}  

\lhead{\firstmark}
\rhead{\botmark}

\subsection{\hspace{-0.5cm} {\Large \textcolor{darkblue}{\textbf{\ipa{ʁo˥-ʐv̩˩}}}}\hspace{0.5cm}[\kern2pt{\textcolor{darkblue}{\textbf{\ipa{xxxx non-correspondance entre le nombre de morphèmes et le nombre de tons de morphèmes}}}}\kern2pt]} \hypertarget{Ro\string_T-z`v\string_=\string_B1}{}
\markboth{\textcolor{darkblue}{\textbf{\ipa{ʁo˥-ʐv̩˩}}}}{}
\textcolor{teal}{\zh{动词}} \hspace{4pt} \zh{声调类:} .
\zh{保佑。} \textcolor{Sepia}{\selectlanguage{english}Bless and protect.} \textcolor{PineGreen}{\selectlanguage{french}Bénir et protéger.}  ¶ \textcolor{darkblue}{\textbf{\ipa{mɤ˧-ʁo˥ʐv̩˩}}} \zh{\mytextsc{neg}} \textcolor{Sepia}{\selectlanguage{english}\mytextsc{neg}} \textcolor{PineGreen}{\selectlanguage{french}\mytextsc{neg}}  
 ¶ \textcolor{darkblue}{\textbf{\ipa{gɤ˧lɑ˧ | ɖɯ˧-ʁo˥ʐv̩˩-ɻ̍˩!}}} \zh{菩萨保佑!} \textcolor{Sepia}{\selectlanguage{english}May the gods bless (you/us)!} \textcolor{PineGreen}{\selectlanguage{french}Que les esprits [te/nous] bénissent!}  

\lhead{\firstmark}
\rhead{\botmark}

\subsection{\hspace{-0.5cm} {\Large \textcolor{darkblue}{\textbf{\ipa{ʁo˧}}} \textsubscript{1}}\hspace{0.5cm}[\kern2pt{\textcolor{darkblue}{\textbf{\ipa{ʁo˥}}}}\kern2pt]} \hypertarget{Ro\string_M1}{}
\markboth{\textcolor{darkblue}{\textbf{\ipa{ʁo˧}}} \textsubscript{1}}{}
\textcolor{teal}{\zh{动词}} \hspace{4pt} \zh{声调类:} M intrans.
\zh{下蛋。} \textcolor{Sepia}{\selectlanguage{english}To lay eggs.} \textcolor{PineGreen}{\selectlanguage{french}Pondre.}  ¶ \textcolor{darkblue}{\textbf{\ipa{æ˩ ʁo˥}}} \zh{下蛋} \textcolor{Sepia}{\selectlanguage{english}to lay eggs} \textcolor{PineGreen}{\selectlanguage{french}pondre des œufs}  
 ¶ \textcolor{darkblue}{\textbf{\ipa{æ˩mi˧ tʰi˧-ʁo˧-dʑo˧!}}} \zh{母鸡在下蛋!} \textcolor{Sepia}{\selectlanguage{english}The hen is laying eggs!} \textcolor{PineGreen}{\selectlanguage{french}la poule est en train de pondre!}  
 ¶ \textcolor{darkblue}{\textbf{\ipa{æ˩mi˧ | æ˩ ʁo˧-ze˩!}}} \zh{母鸡下蛋了!} \textcolor{Sepia}{\selectlanguage{english}The hen has laid eggs!} \textcolor{PineGreen}{\selectlanguage{french}la poule a pondu!}  

\lhead{\firstmark}
\rhead{\botmark}

\subsection{\hspace{-0.5cm} {\Large \textcolor{darkblue}{\textbf{\ipa{ʁo˧}}} \textsubscript{2}}\hspace{0.5cm}[\kern2pt{\textcolor{darkblue}{\textbf{\ipa{ʁo˥}}}}\kern2pt]} \hypertarget{Ro\string_M2}{}
\markboth{\textcolor{darkblue}{\textbf{\ipa{ʁo˧}}} \textsubscript{2}}{}
\textcolor{teal}{\zh{动词}} \hspace{4pt} \zh{声调类:} M intrans.
\zh{能……、有能力做。} \textcolor{Sepia}{\selectlanguage{english}To be able to, to manage to.} \textcolor{PineGreen}{\selectlanguage{french}Arriver à, parvenir à.}  ¶ \textcolor{darkblue}{\textbf{\ipa{njɤ˧ | tɕi˩-mɤ˩-ʁo˩˥!}}} \zh{我写不出来! / 我不会写!} \textcolor{Sepia}{\selectlanguage{english}I can't write! / I am not able to write! (Said by someone who has not learnt to write)} \textcolor{PineGreen}{\selectlanguage{french}je ne parviens pas à écrire/je ne sais pas écrire!}  
 ¶ \textcolor{darkblue}{\textbf{\ipa{njɤ˧ | tɕi˩-ʁo˩˥!}}} \zh{我会写! / 我写得出来!} \textcolor{Sepia}{\selectlanguage{english}I can write! / I am able to write! / I know how to write!} \textcolor{PineGreen}{\selectlanguage{french}je parviens à écrire/je sais écrire!}  

\lhead{\firstmark}
\rhead{\botmark}

\subsection{\hspace{-0.5cm} {\Large \textcolor{darkblue}{\textbf{\ipa{ʁo˧bv̩˧}}}}\hspace{0.5cm}[\kern2pt{\textcolor{darkblue}{\textbf{\ipa{ʁo˩bv̩˩˥}}}}\kern2pt]} \hypertarget{Ro\string_Mbv\string_=\string_M1}{}
\markboth{\textcolor{darkblue}{\textbf{\ipa{ʁo˧bv̩˧}}}}{}
\textcolor{teal}{\zh{名词}} \hspace{4pt} \zh{声调类:} M.
\zh{树的萌芽、新发出来的叶子。} \textcolor{Sepia}{\selectlanguage{english}Sprout, bud.} \textcolor{PineGreen}{\selectlanguage{french}Pousse d'arbre.}  ¶ \textcolor{darkblue}{\textbf{\ipa{tʰo˧-ʁo˧bv˥}}} \zh{小松树尖} \textcolor{Sepia}{\selectlanguage{english}bud of pine tree} \textcolor{PineGreen}{\selectlanguage{french}pousse de sapin}  
 ¶ \textcolor{darkblue}{\textbf{\ipa{tʰo˩ʂv˩-ʁo˥bv˩}}} \zh{小松树尖} \textcolor{Sepia}{\selectlanguage{english}bud of pine tree} \textcolor{PineGreen}{\selectlanguage{french}pousse de sapin; littéralement “pousses d'aiguilles de sapin”}  
 \zh{量词}: \textcolor{darkblue}{\textbf{\ipa{kʰwɤ˥}}} 
\lhead{\firstmark}
\rhead{\botmark}

\subsection{\hspace{-0.5cm} {\Large \textcolor{darkblue}{\textbf{\ipa{ʁo˧dɑ˧}}}}\hspace{0.5cm}[\kern2pt{\textcolor{darkblue}{\textbf{\ipa{ʁo˧dɑ˧}}}}\kern2pt]} \hypertarget{Ro\string_MdA\string_M1}{}
\markboth{\textcolor{darkblue}{\textbf{\ipa{ʁo˧dɑ˧}}}}{}
\textcolor{teal}{\zh{助词}} \hspace{4pt} \zh{声调类:} M.
\zh{前面,之前。} \textcolor{Sepia}{\selectlanguage{english}In front of.} \textcolor{PineGreen}{\selectlanguage{french}Devant, avant, auparavant.}  ¶ \textcolor{darkblue}{\textbf{\ipa{ʂɯ˧-kʰv̩˧-ʁo˧dɑ˧}}} \zh{七年前} \textcolor{Sepia}{\selectlanguage{english}seven years ago} \textcolor{PineGreen}{\selectlanguage{french}il y a sept ans}  
 ¶ \textcolor{darkblue}{\textbf{\ipa{ʁo˧dɑ˧ ɖɯ˧-so˩ ɲi˩}}} \zh{前几天} \textcolor{Sepia}{\selectlanguage{english}the past few days} \textcolor{PineGreen}{\selectlanguage{french}ces derniers jours, les quelques jours passés, il y a quelques jours}  

\lhead{\firstmark}
\rhead{\botmark}

\subsection{\hspace{-0.5cm} {\Large \textcolor{darkblue}{\textbf{\ipa{ʁo˧do˧}}} \textsubscript{1}}\hspace{0.5cm}[\kern2pt{\textcolor{darkblue}{\textbf{\ipa{ʁo˩do˥}}}}\kern2pt]} \hypertarget{Ro\string_Mdo\string_M1}{}
\markboth{\textcolor{darkblue}{\textbf{\ipa{ʁo˧do˧}}} \textsubscript{1}}{}
\textcolor{teal}{\zh{名词}} \hspace{4pt} \zh{声调类:} M.
\zh{核桃。} \textcolor{Sepia}{\selectlanguage{english}Walnut.} \textcolor{PineGreen}{\selectlanguage{french}Noix.}  ¶ \textcolor{darkblue}{\textbf{\ipa{ʁo˧do˧ qʰwæ˧˥}}} \zh{开核桃} \textcolor{Sepia}{\selectlanguage{english}to crack walnuts} \textcolor{PineGreen}{\selectlanguage{french}casser des noix}  
 ¶ \textcolor{darkblue}{\textbf{\ipa{ʁo˧do˧ ʐwæ˧}}} \zh{称核桃} \textcolor{Sepia}{\selectlanguage{english}to weigh walnuts} \textcolor{PineGreen}{\selectlanguage{french}peser des noix}  
 \zh{量词}: \textcolor{darkblue}{\textbf{\ipa{ɭɯ˧}}} 
\lhead{\firstmark}
\rhead{\botmark}

\subsection{\hspace{-0.5cm} {\Large \textcolor{darkblue}{\textbf{\ipa{ʁo˧do˧}}} \textsubscript{2}}\hspace{0.5cm}[\kern2pt{\textcolor{darkblue}{\textbf{\ipa{ʁo˧do˧}}}}\kern2pt]} \hypertarget{Ro\string_Mdo\string_M2}{}
\markboth{\textcolor{darkblue}{\textbf{\ipa{ʁo˧do˧}}} \textsubscript{2}}{}
\textcolor{teal}{\zh{名词}} \hspace{4pt} \zh{声调类:} M.
\zh{利息。} \textcolor{Sepia}{\selectlanguage{english}Interests.} \textcolor{PineGreen}{\selectlanguage{french}Intérêts.}  \zh{量词}: \textcolor{darkblue}{\textbf{\ipa{kʰwɤ˥}}} 
\lhead{\firstmark}
\rhead{\botmark}

\subsection{\hspace{-0.5cm} {\Large \textcolor{darkblue}{\textbf{\ipa{ʁo˧dzi˥}}}}\hspace{0.5cm}[\kern2pt{\textcolor{darkblue}{\textbf{\ipa{ʁo˧dzi˥}}}}\kern2pt]} \hypertarget{Ro\string_Mdzi\string_T1}{}
\markboth{\textcolor{darkblue}{\textbf{\ipa{ʁo˧dzi˥}}}}{}
\textcolor{teal}{\zh{动词}} \hspace{4pt} \zh{声调类:} H\#.
\zh{碰撞。} \textcolor{Sepia}{\selectlanguage{english}To collide, to run into.} \textcolor{PineGreen}{\selectlanguage{french}Heurter.}  ¶ \textcolor{darkblue}{\textbf{\ipa{le˧-ʁo˧dzi˥}}} \zh{\mytextsc{accomp}} \textcolor{Sepia}{\selectlanguage{english}\mytextsc{accomp}} \textcolor{PineGreen}{\selectlanguage{french}\mytextsc{accomp}}  
 ¶ \textcolor{darkblue}{\textbf{\ipa{hĩ˧ | tʰi˧-ʁo˧dzi˥ tsʰɯ˩(-ze˩)}}} \zh{人们(互相)碰撞} \textcolor{Sepia}{\selectlanguage{english}People have ran into one another} \textcolor{PineGreen}{\selectlanguage{french}deux personnes se sont heurtées/sont entrées en collision}  

\lhead{\firstmark}
\rhead{\botmark}

\subsection{\hspace{-0.5cm} {\Large \textcolor{darkblue}{\textbf{\ipa{ʁo˧dzi˩}}}}\hspace{0.5cm}[\kern2pt{\textcolor{darkblue}{\textbf{\ipa{ʁo˧dzi˩}}}}\kern2pt]} \hypertarget{Ro\string_Mdzi\string_B1}{}
\markboth{\textcolor{darkblue}{\textbf{\ipa{ʁo˧dzi˩}}}}{}
\textcolor{teal}{\zh{名词}} \hspace{4pt} \zh{声调类:} L\#.
\zh{藏族。} \textcolor{Sepia}{\selectlanguage{english}Tibetan.} \textcolor{PineGreen}{\selectlanguage{french}Tibétain.}  \zh{量词}: \textcolor{darkblue}{\textbf{\ipa{v̩˧}}} 
\lhead{\firstmark}
\rhead{\botmark}

\subsection{\hspace{-0.5cm} {\Large \textcolor{darkblue}{\textbf{\ipa{ʁo˧dzi˩-di˩}}}}\hspace{0.5cm}[\kern2pt{\textcolor{darkblue}{\textbf{\ipa{ʁo˧dzi˩di˧}}}}\kern2pt]} \hypertarget{Ro\string_Mdzi\string_B-di\string_B1}{}
\markboth{\textcolor{darkblue}{\textbf{\ipa{ʁo˧dzi˩-di˩}}}}{}
\textcolor{teal}{\zh{名词}} \hspace{4pt} \zh{声调类:} L\#-.
\zh{西藏。} \textcolor{Sepia}{\selectlanguage{english}Tibet (literally: 'the Tibetan land').} \textcolor{PineGreen}{\selectlanguage{french}Le Tibet (littéralement: 'la contrée des Tibétains').} 
\lhead{\firstmark}
\rhead{\botmark}

\subsection{\hspace{-0.5cm} {\Large \textcolor{darkblue}{\textbf{\ipa{ʁo˧dzi˩-tʰæ˩ɻæ˩}}}}\hspace{0.5cm}[\kern2pt{\textcolor{darkblue}{\textbf{\ipa{ʁo˧dzi˩tʰæ˧ɻæ˧}}}}\kern2pt]} \hypertarget{Ro\string_Mdzi\string_B-t\string_h\{\string_Br£`\{\string_B1}{}
\markboth{\textcolor{darkblue}{\textbf{\ipa{ʁo˧dzi˩-tʰæ˩ɻæ˩}}}}{}
\textcolor{teal}{\zh{名词}} \hspace{4pt} \zh{声调类:} L\#-.
\zh{旗子。} \textcolor{Sepia}{\selectlanguage{english}Flag, banner, pennant (literally: Tibetan writings).} \textcolor{PineGreen}{\selectlanguage{french}Drapeau, fanion (littéralement “écritures tibétaines”).}  \zh{量词}: \textcolor{darkblue}{\textbf{\ipa{pʰæ˧˥}}} 
\lhead{\firstmark}
\rhead{\botmark}

\subsection{\hspace{-0.5cm} {\Large \textcolor{darkblue}{\textbf{\ipa{ʁo˧dʑɯ˧}}}}\hspace{0.5cm}[\kern2pt{\textcolor{darkblue}{\textbf{\ipa{ʁo˧dʑɯ˧}}}}\kern2pt]} \hypertarget{Ro\string_Mdz£M\string_M1}{}
\markboth{\textcolor{darkblue}{\textbf{\ipa{ʁo˧dʑɯ˧}}}}{}
\textcolor{teal}{\zh{名词}} \hspace{4pt} \zh{声调类:} M.
\zh{马笼头。} \textcolor{Sepia}{\selectlanguage{english}Bridle; halter.} \textcolor{PineGreen}{\selectlanguage{french}Œillères.}  ¶ \textcolor{darkblue}{\textbf{\ipa{ʐwæ˧-ʁo˧dʑɯ˥ (ʈʂʰɯ˧ | ʐwæ˧-ʁo˧dʑɯ˥ ɲi˩)}}} \zh{马笼头} \textcolor{Sepia}{\selectlanguage{english}horse's halter} \textcolor{PineGreen}{\selectlanguage{french}œillères de cheval}  
 \zh{量词}: \textcolor{darkblue}{\textbf{\ipa{nɑ˧}}} \textcolor{darkblue}{\textbf{\ipa{pɤ˩}}} 
\lhead{\firstmark}
\rhead{\botmark}

\subsection{\hspace{-0.5cm} {\Large \textcolor{darkblue}{\textbf{\ipa{ʁo˧ɖɯ˧˥}}}}\hspace{0.5cm}[\kern2pt{\textcolor{darkblue}{\textbf{\ipa{ʁo˧ɖɯ˧}}}}\kern2pt]} \hypertarget{Ro\string_Md`M\string_M\string_T1}{}
\markboth{\textcolor{darkblue}{\textbf{\ipa{ʁo˧ɖɯ˧˥}}}}{}
\textcolor{teal}{\zh{名词}} \hspace{4pt} \zh{声调类:} MH\#.
\zh{蝌蚪。} \textcolor{Sepia}{\selectlanguage{english}Tadpole.} \textcolor{PineGreen}{\selectlanguage{french}Têtard.} 
\lhead{\firstmark}
\rhead{\botmark}

\subsection{\hspace{-0.5cm} {\Large \textcolor{darkblue}{\textbf{\ipa{ʁo˧gv̩\#˥}}}}\hspace{0.5cm}[\kern2pt{\textcolor{darkblue}{\textbf{\ipa{ʁo˧gv̩˧}}}}\kern2pt]} \hypertarget{Ro\string_Mgv\string_=\#\string_T1}{}
\markboth{\textcolor{darkblue}{\textbf{\ipa{ʁo˧gv̩\#˥}}}}{}
\textcolor{teal}{\zh{名词}} \hspace{4pt} \zh{声调类:} \#H.
\zh{枕头。} \textcolor{Sepia}{\selectlanguage{english}Pillow.} \textcolor{PineGreen}{\selectlanguage{french}Oreiller.}  \zh{量词}: \textcolor{darkblue}{\textbf{\ipa{ɭɯ˧}}} 
\lhead{\firstmark}
\rhead{\botmark}

\subsection{\hspace{-0.5cm} {\Large \textcolor{darkblue}{\textbf{\ipa{ʁo˧hṽ˧˥}}}}\hspace{0.5cm}[\kern2pt{\textcolor{darkblue}{\textbf{\ipa{ʁo˧hṽ˧˥}}}}\kern2pt]} \hypertarget{Ro\string_Mhv\string_~\string_M\string_T1}{}
\markboth{\textcolor{darkblue}{\textbf{\ipa{ʁo˧hṽ˧˥}}}}{}
\textcolor{teal}{\zh{名词}} \hspace{4pt} \zh{声调类:} MH\#.
\zh{头发。} \textcolor{Sepia}{\selectlanguage{english}Hair (of the head).} \textcolor{PineGreen}{\selectlanguage{french}Cheveux.}  \zh{量词}: \textcolor{darkblue}{\textbf{\ipa{kʰɯ˩}}} 
\lhead{\firstmark}
\rhead{\botmark}

\subsection{\hspace{-0.5cm} {\Large \textcolor{darkblue}{\textbf{\ipa{ʁo˧ʝi˧}}}}\hspace{0.5cm}[\kern2pt{\textcolor{darkblue}{\textbf{\ipa{ʁo˧ʝi˧}}}}\kern2pt]} \hypertarget{Ro\string_Mj££i\string_M1}{}
\markboth{\textcolor{darkblue}{\textbf{\ipa{ʁo˧ʝi˧}}}}{}
\textcolor{teal}{\zh{助词}} \hspace{4pt} \zh{声调类:} M.
\zh{后年。} \textcolor{Sepia}{\selectlanguage{english}The year after next.} \textcolor{PineGreen}{\selectlanguage{french}Dans deux ans.}  ¶ \textcolor{darkblue}{\textbf{\ipa{ʁo˧ʝi˧ ɖɯ˧-kʰv̩˧˥}}} \zh{后年} \textcolor{Sepia}{\selectlanguage{english}the year after next} \textcolor{PineGreen}{\selectlanguage{french}l'année dans deux ans}  

\lhead{\firstmark}
\rhead{\botmark}

\subsection{\hspace{-0.5cm} {\Large \textcolor{darkblue}{\textbf{\ipa{ʁo˧kɤ˩}}}}\hspace{0.5cm}[\kern2pt{\textcolor{darkblue}{\textbf{\ipa{ʁo˧kɤ˩}}}}\kern2pt]} \hypertarget{Ro\string_Mk7\string_B1}{}
\markboth{\textcolor{darkblue}{\textbf{\ipa{ʁo˧kɤ˩}}}}{}
\textcolor{teal}{\zh{名词}} \hspace{4pt} \zh{声调类:} L\#.
\zh{用来将长辫缠成盘头的黑色丝头饰(已经有孩子的女人戴的)。还没有孩子的青年女人,也戴这种头饰,但称作\textcolor{darkblue}{\textbf{\ipa{/ʁo˧ni˥/}}})。} \textcolor{Sepia}{\selectlanguage{english}Woven headdress for women who already have children; for young women who do not yet have children, this same item is called \textcolor{darkblue}{\textbf{\ipa{/ʁo˧ni˥/}}}.} \textcolor{PineGreen}{\selectlanguage{french}Coiffe en fils tressés; chez une jeune femme (qui a atteint ses 13 ans, mais n'a pas encore d'enfants), ce même objet est désigné comme \textcolor{darkblue}{\textbf{\ipa{/ʁo˧ni˥/}}}.}  \zh{量词}: \textcolor{darkblue}{\textbf{\ipa{kɤ˧˥}}} 
\lhead{\firstmark}
\rhead{\botmark}

\subsection{\hspace{-0.5cm} {\Large \textcolor{darkblue}{\textbf{\ipa{ʁo˧lv̩˧}}}}\hspace{0.5cm}[\kern2pt{\textcolor{darkblue}{\textbf{\ipa{ʁo˧lv̩˧}}}}\kern2pt]} \hypertarget{Ro\string_Mlv\string_=\string_M1}{}
\markboth{\textcolor{darkblue}{\textbf{\ipa{ʁo˧lv̩˧}}}}{}
\textcolor{teal}{\zh{动词}} \hspace{4pt} \zh{声调类:} M.
\zh{迷路。} \textcolor{Sepia}{\selectlanguage{english}To lose one's way, to become lost.} \textcolor{PineGreen}{\selectlanguage{french}Se perdre, perdre son chemin.}  ¶ \textcolor{darkblue}{\textbf{\ipa{le˧-ʁo˧lv̩˧}}} \zh{\mytextsc{accomp}} \textcolor{Sepia}{\selectlanguage{english}\mytextsc{accomp}} \textcolor{PineGreen}{\selectlanguage{french}\mytextsc{accomp}}  

\lhead{\firstmark}
\rhead{\botmark}

\subsection{\hspace{-0.5cm} {\Large \textcolor{darkblue}{\textbf{\ipa{ʁo˧ɬi˥}}}}\hspace{0.5cm}[\kern2pt{\textcolor{darkblue}{\textbf{\ipa{ʁo˧ɬi˥}}}}\kern2pt]} \hypertarget{Ro\string_MKi\string_T1}{}
\markboth{\textcolor{darkblue}{\textbf{\ipa{ʁo˧ɬi˥}}}}{}
\textcolor{teal}{\zh{名词}} \hspace{4pt} \zh{声调类:} H\#.
\zh{大粗针,用来缝琵琶肉。} \textcolor{Sepia}{\selectlanguage{english}Large needle with which animal hide can be sewn.} \textcolor{PineGreen}{\selectlanguage{french}Grosse aiguille avec laquelle on coud le paquet de viande de cochon au salpêtre qui se conserve une décennie; contrairement à ce que dit M21, n'est pas utilisé pour coudre les peaux d'animaux (mouton, bœuf, yak…): pour cela, il faut utiliser un poinçon.}  ¶ \textcolor{darkblue}{\textbf{\ipa{ʁo˧ɬi˥, | bo˩ʈʂʰæ˧ ʐv̩˩-di˩ ɲi˩.}}} \zh{大针,是用来缝琵琶肉的。} \textcolor{Sepia}{\selectlanguage{english}The large needle is used to sew pipa meat.} \textcolor{PineGreen}{\selectlanguage{french}La grosse aiguille, ça sert à coudre le cochon-conservé-entier (viande séchée “pipa”).}  
 \zh{量词}: \textcolor{darkblue}{\textbf{\ipa{ɭɯ˧}}} 
\lhead{\firstmark}
\rhead{\botmark}

\subsection{\hspace{-0.5cm} {\Large \textcolor{darkblue}{\textbf{\ipa{ʁo˧mi˥\$}}}}\hspace{0.5cm}[\kern2pt{\textcolor{darkblue}{\textbf{\ipa{ʁo˧mi˥}}}}\kern2pt]} \hypertarget{Ro\string_Mmi\string_T\$1}{}
\markboth{\textcolor{darkblue}{\textbf{\ipa{ʁo˧mi˥\$}}}}{}
\textcolor{teal}{\zh{名词}} \hspace{4pt} \zh{声调类:} H\$.
\zh{大针。} \textcolor{Sepia}{\selectlanguage{english}Large needle.} \textcolor{PineGreen}{\selectlanguage{french}Grosse aiguille.}  \zh{量词}: \textcolor{darkblue}{\textbf{\ipa{ɭɯ˧}}} 
\lhead{\firstmark}
\rhead{\botmark}

\subsection{\hspace{-0.5cm} {\Large \textcolor{darkblue}{\textbf{\ipa{ʁo˧mi˧}}}}\hspace{0.5cm}[\kern2pt{\textcolor{darkblue}{\textbf{\ipa{ʁo˧mi˧}}}}\kern2pt]} \hypertarget{Ro\string_Mmi\string_M1}{}
\markboth{\textcolor{darkblue}{\textbf{\ipa{ʁo˧mi˧}}}}{}
\textcolor{teal}{\zh{名词}} \hspace{4pt} \zh{声调类:} M.
\zh{国王、大臣、头领。} \textcolor{Sepia}{\selectlanguage{english}King; high official; chief.} \textcolor{PineGreen}{\selectlanguage{french}Roi; haut dignitaire, grand mandarin; chef.}  ¶ \textcolor{darkblue}{\textbf{\ipa{ʁo˧mi˧ ʝi˧-hĩ˧ hĩ˧}}} \zh{当国王、土司、大臣、头领……的人} \textcolor{Sepia}{\selectlanguage{english}person who has a role as king/high official/chief} \textcolor{PineGreen}{\selectlanguage{french}personne qui a fonction de dignitaire/chef}  
 ¶ \textcolor{darkblue}{\textbf{\ipa{kʰv̩˧mæ˧-ʁo˧mi˧}}} \zh{土匪的头领} \textcolor{Sepia}{\selectlanguage{english}head of (a band of) robbers} \textcolor{PineGreen}{\selectlanguage{french}chef des brigands, capitaine d'une troupe de brigands}  
 \zh{量词}: \textcolor{darkblue}{\textbf{\ipa{v̩˧}}} 
\lhead{\firstmark}
\rhead{\botmark}

\subsection{\hspace{-0.5cm} {\Large \textcolor{darkblue}{\textbf{\ipa{ʁo˧ni˥}}}}\hspace{0.5cm}[\kern2pt{\textcolor{darkblue}{\textbf{\ipa{ʁo˧ni˥}}}}\kern2pt]} \hypertarget{Ro\string_Mni\string_T1}{}
\markboth{\textcolor{darkblue}{\textbf{\ipa{ʁo˧ni˥}}}}{}
\textcolor{teal}{\zh{名词}} \hspace{4pt} \zh{声调类:} H\#.
\zh{用来将长辫缠成盘头的黑色丝头饰(还没有孩子的青年女人戴的)。已经有孩子的女人,也戴这种头饰,但称作\textcolor{darkblue}{\textbf{\ipa{/ʁo˧kɤ˩/}}})。} \textcolor{Sepia}{\selectlanguage{english}Woven headdress for young women who do not yet have children; for women who already have children, this same item is called \textcolor{darkblue}{\textbf{\ipa{/ʁo˧kɤ˩/}}}.} \textcolor{PineGreen}{\selectlanguage{french}Coiffe en fils tressés des jeunes femmes qui n'ont pas encore d'enfant. Les femmes qui ont des enfants portent également cette pièce de costume, mais elle est alors désignée comme \textcolor{darkblue}{\textbf{\ipa{/ʁo˧kɤ˩/}}}.}  \zh{量词}: \textcolor{darkblue}{\textbf{\ipa{bo˩}}} 
\lhead{\firstmark}
\rhead{\botmark}

\subsection{\hspace{-0.5cm} {\Large \textcolor{darkblue}{\textbf{\ipa{ʁo˧pʰɤ˩-ʁo˩dv̩˩lv̩˩}}}}\hspace{0.5cm}[\kern2pt{\textcolor{darkblue}{\textbf{\ipa{xxxx non-correspondance entre le nombre de morphèmes et le nombre de tons de morphèmes}}}}\kern2pt]} \hypertarget{Ro\string_Mp\string_h7\string_B-Ro\string_Bdv\string_=\string_Blv\string_=\string_B1}{}
\markboth{\textcolor{darkblue}{\textbf{\ipa{ʁo˧pʰɤ˩-ʁo˩dv̩˩lv̩˩}}}}{}
\textcolor{teal}{\zh{名词}} \hspace{4pt} \zh{声调类:} L\#.
\zh{短葶飞蓬。} \textcolor{Sepia}{\selectlanguage{english}\textit{Eugeron breviscapus} (a type of daisy).} \textcolor{PineGreen}{\selectlanguage{french}\textit{Eugeron breviscapus} (une sorte de pâquerette).} 
\lhead{\firstmark}
\rhead{\botmark}

\subsection{\hspace{-0.5cm} {\Large \textcolor{darkblue}{\textbf{\ipa{ʁo˧qɑ˥}}}}\hspace{0.5cm}[\kern2pt{\textcolor{darkblue}{\textbf{\ipa{ʁo˧qɑ˥}}}}\kern2pt]} \hypertarget{Ro\string_MqA\string_T1}{}
\markboth{\textcolor{darkblue}{\textbf{\ipa{ʁo˧qɑ˥}}}}{}
\textcolor{teal}{\zh{名词}} \hspace{4pt} \zh{声调类:} H\#.
\zh{锅盖、盖子。} \textcolor{Sepia}{\selectlanguage{english}Lid.} \textcolor{PineGreen}{\selectlanguage{french}Couvercle.}  \zh{量词}: \textcolor{darkblue}{\textbf{\ipa{ɭɯ˧}}} 
\lhead{\firstmark}
\rhead{\botmark}

\subsection{\hspace{-0.5cm} {\Large \textcolor{darkblue}{\textbf{\ipa{ʁo˧qʰwɤ˩}}}}\hspace{0.5cm}[\kern2pt{\textcolor{darkblue}{\textbf{\ipa{ʁo˧qʰwɤ˩}}}}\kern2pt]} \hypertarget{Ro\string_Mq\string_hw7\string_B1}{}
\markboth{\textcolor{darkblue}{\textbf{\ipa{ʁo˧qʰwɤ˩}}}}{}
\textcolor{teal}{\zh{名词}} \hspace{4pt} \zh{声调类:} L\#.
\ding{202} \zh{头,上面部分。} \textcolor{Sepia}{\selectlanguage{english}Head.} \textcolor{PineGreen}{\selectlanguage{french}Tête.}  ¶ \textcolor{darkblue}{\textbf{\ipa{ʁo˧qʰwɤ˩ dzi˩}}} \zh{坐在贵宾的位置上} \textcolor{Sepia}{\selectlanguage{english}to sit in a place of honour} \textcolor{PineGreen}{\selectlanguage{french}être assis à une place d'honneur}  
 ¶ \textcolor{darkblue}{\textbf{\ipa{õ˧-ʁo˥qʰwɤ˩}}} \zh{自己的头} \textcolor{Sepia}{\selectlanguage{english}one's own head} \textcolor{PineGreen}{\selectlanguage{french}sa propre tête}  
 ¶ \textcolor{darkblue}{\textbf{\ipa{õ˧-ʁo˥qʰwɤ˩ lɑ˩}}} \zh{打自己的头(情景:一个小孩用小棍子敲打自己的头)} \textcolor{Sepia}{\selectlanguage{english}to hit one's own head (context: a child hits its own head rhythmically with a stick)} \textcolor{PineGreen}{\selectlanguage{french}se taper sur la tête (contexte: un enfant tape en rythme sur sa propre tête avec une baguette)}  
 \zh{量词}: \textcolor{darkblue}{\textbf{\ipa{ɭɯ˧}}} \ding{203} \zh{上面部分。} \textcolor{Sepia}{\selectlanguage{english}Top part, upper part.} \textcolor{PineGreen}{\selectlanguage{french}Partie supérieure de.} 
\lhead{\firstmark}
\rhead{\botmark}

\subsection{\hspace{-0.5cm} {\Large \textcolor{darkblue}{\textbf{\ipa{ʁo˧so˩}}}}\hspace{0.5cm}[\kern2pt{\textcolor{darkblue}{\textbf{\ipa{ʁo˧so˩}}}}\kern2pt]} \hypertarget{Ro\string_Mso\string_B1}{}
\markboth{\textcolor{darkblue}{\textbf{\ipa{ʁo˧so˩}}}}{}
\textcolor{teal}{\zh{助词}} \hspace{4pt} \zh{声调类:} L\#.
\zh{后天。} \textcolor{Sepia}{\selectlanguage{english}The day after tomorrow.} \textcolor{PineGreen}{\selectlanguage{french}Après-demain.}  ¶ \textcolor{darkblue}{\textbf{\ipa{ʁo˧so˩ | -ɖɯ˧ɲi˥}}} \zh{后天} \textcolor{Sepia}{\selectlanguage{english}the day after tomorrow} \textcolor{PineGreen}{\selectlanguage{french}la journée d'après-demain}  

\lhead{\firstmark}
\rhead{\botmark}

\subsection{\hspace{-0.5cm} {\Large \textcolor{darkblue}{\textbf{\ipa{ʁo˧ʂv̩˧}}}}\hspace{0.5cm}[\kern2pt{\textcolor{darkblue}{\textbf{\ipa{ʁo˧ʂv̩˧}}}}\kern2pt]} \hypertarget{Ro\string_Ms`v\string_=\string_M1}{}
\markboth{\textcolor{darkblue}{\textbf{\ipa{ʁo˧ʂv̩˧}}}}{}
\textcolor{teal}{\zh{动词}} \hspace{4pt} \zh{声调类:} M.
\zh{带头、带路。} \textcolor{Sepia}{\selectlanguage{english}To guide, to show the way.} \textcolor{PineGreen}{\selectlanguage{french}Conduire, guider.}  ¶ \textcolor{darkblue}{\textbf{\ipa{ʐɤ˩mi˩ ʁo˩ʂv̩˩˥}}} \zh{带路} \textcolor{Sepia}{\selectlanguage{english}to show the way} \textcolor{PineGreen}{\selectlanguage{french}montrer le chemin}  
 ¶ \textcolor{darkblue}{\textbf{\ipa{ɖɯ˧-ʑi˩-ɳɯ˩ | ʁo˧ʂv̩˧}}} \zh{有一家带头:例如收庄稼时,一个家先开始收割,于是其它家庭也跟着开始收割。} \textcolor{Sepia}{\selectlanguage{english}a family shows the way/ sets an example (which other families follow): for instance, one family begins to harvest rice, and others follow their example} \textcolor{PineGreen}{\selectlanguage{french}une famille montre l'exemple: par exemple, une famille commence à récolter le riz, et les autres suivent son exemple}  
 ¶ \textcolor{darkblue}{\textbf{\ipa{ʁo˧ʂv̩˧-ze˧}}} \zh{带了路} \textcolor{Sepia}{\selectlanguage{english}\mytextsc{pfv}} \textcolor{PineGreen}{\selectlanguage{french}\mytextsc{pfv}}  
 ¶ \textcolor{darkblue}{\textbf{\ipa{njɤ˧=ɻ̍˩-ɳɯ˩ | ʁo˧ʂv̩˧!}}} \zh{是我们带头的!(其他家庭是跟着我们来的!)(情景:农业活动,如:收庄稼,是一个家庭先开始的,然后其他家庭也跟着来。)} \textcolor{Sepia}{\selectlanguage{english}We are showing the way! / We are setting an example for others! (Context: for agricultural activities, one household started first, and the others followed suit.)} \textcolor{PineGreen}{\selectlanguage{french}C'est nous qui lançons le mouvement!/ C'est nous qui donnons l'exemple aux autres! (explication: pour les travaux des champs, une maisonnée s'y attelait en premier, et les autres suivaient)}  

\lhead{\firstmark}
\rhead{\botmark}

\subsection{\hspace{-0.5cm} {\Large \textcolor{darkblue}{\textbf{\ipa{-ʁo˧to˩}}}}\hspace{0.5cm}[\kern2pt{\textcolor{darkblue}{\textbf{\ipa{ʁo˧to˩}}}}\kern2pt]} \hypertarget{-Ro\string_Mto\string_B1}{}
\markboth{\textcolor{darkblue}{\textbf{\ipa{-ʁo˧to˩}}}}{}
\textcolor{teal}{\zh{后置词}} \hspace{4pt} \zh{声调类:} L\#.
\ding{202} \zh{……之上。} \textcolor{Sepia}{\selectlanguage{english}On top of.} \textcolor{PineGreen}{\selectlanguage{french}Sur.}  ¶ \textcolor{darkblue}{\textbf{\ipa{qo˩qɑ˩-ʁo˩to˥}}} \zh{垭口上} \textcolor{Sepia}{\selectlanguage{english}at the top of the mountain pass, at the mountain pass} \textcolor{PineGreen}{\selectlanguage{french}en haut du col}  
 ¶ \textcolor{darkblue}{\textbf{\ipa{ʁo˧qʰwɤ˩-ʁo˩to˩}}} \zh{头上} \textcolor{Sepia}{\selectlanguage{english}on the head, on top of the head} \textcolor{PineGreen}{\selectlanguage{french}sur la tête, sur le sommet du crâne}  
 ¶ \textcolor{darkblue}{\textbf{\ipa{ʑi˧qʰwɤ˧-ʁo˧to˩}}} \zh{房子上面:例如:有鸟窝在房顶上} \textcolor{Sepia}{\selectlanguage{english}on the house; e.g. there is a bird's nest on the top of the house} \textcolor{PineGreen}{\selectlanguage{french}sur la maison; ex.: il y a un nid d’oiseaux sur la maison}  
\ding{203} \zh{……的时候。} \textcolor{Sepia}{\selectlanguage{english}While, at the time that, during the time that.} \textcolor{PineGreen}{\selectlanguage{french}Pendant, au moment de.}  ¶ \textcolor{darkblue}{\textbf{\ipa{hɑ˧dzɯ˧-ʁo˧to˩, | ʈʂʰɯ˧-ɳɯ˧ | mɤ˧-fv̩˧-ʝi˧.}}} \zh{吃饭的时候,他不高兴了/生气了。} \textcolor{Sepia}{\selectlanguage{english}During the meal, he felt displeased/he got angry.} \textcolor{PineGreen}{\selectlanguage{french}Au cours du repas, il se mit en colère/ devint triste.}  
\ding{204} \zh{向、往。} \textcolor{Sepia}{\selectlanguage{english}To, at, towards.} \textcolor{PineGreen}{\selectlanguage{french}À l'endroit de, à l'égard de, en direction de.} \ding{205} \zh{跟……相比。} \textcolor{Sepia}{\selectlanguage{english}Compared to.} \textcolor{PineGreen}{\selectlanguage{french}En comparaison de.} 
\lhead{\firstmark}
\rhead{\botmark}

\subsection{\hspace{-0.5cm} {\Large \textcolor{darkblue}{\textbf{\ipa{-ʁo˧tʰo˩}}}}\hspace{0.5cm}[\kern2pt{\textcolor{darkblue}{\textbf{\ipa{ʁo˧tʰo˩}}}}\kern2pt]} \hypertarget{-Ro\string_Mt\string_ho\string_B1}{}
\markboth{\textcolor{darkblue}{\textbf{\ipa{-ʁo˧tʰo˩}}}}{}
\textcolor{teal}{\zh{后置词}} \hspace{4pt} \zh{声调类:} L\#.
\zh{后面,自从。} \textcolor{Sepia}{\selectlanguage{english}Behind; since.} \textcolor{PineGreen}{\selectlanguage{french}Derrière; depuis.}  ¶ \textcolor{darkblue}{\textbf{\ipa{ʑi˧-tʰo˩}}} \zh{家后院(=菜园的地方)} \textcolor{Sepia}{\selectlanguage{english}behind the house (=the place where there is a vegetable garden)} \textcolor{PineGreen}{\selectlanguage{french}l'arrière de la maison (où il y a le potager)}  
 ¶ \textcolor{darkblue}{\textbf{\ipa{ʑi˧-ʁo˥tʰo˩}}} \zh{同上:家后院} \textcolor{Sepia}{\selectlanguage{english}as above: behind the house} \textcolor{PineGreen}{\selectlanguage{french}idem: derrière la maison, l'arrière de la maison}  
 ¶ \textcolor{darkblue}{\textbf{\ipa{ʑi˧qʰwɤ˧-ʁo˧tʰo˩}}} \zh{同上:家后院} \textcolor{Sepia}{\selectlanguage{english}as above: behind the house} \textcolor{PineGreen}{\selectlanguage{french}idem: derrière la maison, l'arrière de la maison}  

\lhead{\firstmark}
\rhead{\botmark}

\subsection{\hspace{-0.5cm} {\Large \textcolor{darkblue}{\textbf{\ipa{ʁo˧tɕʰɤ\#˥}}}}\hspace{0.5cm}[\kern2pt{\textcolor{darkblue}{\textbf{\ipa{ʁo˧tɕʰɤ˧}}}}\kern2pt]} \hypertarget{Ro\string_Mts£\string_h7\#\string_T1}{}
\markboth{\textcolor{darkblue}{\textbf{\ipa{ʁo˧tɕʰɤ\#˥}}}}{}
\textcolor{teal}{\zh{形容词}} \hspace{4pt} \zh{声调类:} \#H.
\zh{尖。} \textcolor{Sepia}{\selectlanguage{english}Sharp, pointed.} \textcolor{PineGreen}{\selectlanguage{french}Pointu.}  ¶ \textcolor{darkblue}{\textbf{\ipa{[F5] ʁo˧tɕʰɤ˧\textasciitilde{}tɕʰɤ˧-gv̩˧}}} \zh{尖} \textcolor{Sepia}{\selectlanguage{english}sharp} \textcolor{PineGreen}{\selectlanguage{french}pointu}  

\lhead{\firstmark}
\rhead{\botmark}

\subsection{\hspace{-0.5cm} {\Large \textcolor{darkblue}{\textbf{\ipa{ʁo˧tsʰe˧ʁo\#˥}}}}\hspace{0.5cm}[\kern2pt{\textcolor{darkblue}{\textbf{\ipa{ʁo˧tsʰe˧ʁo˩}}}}\kern2pt]} \hypertarget{Ro\string_Mts\string_he\string_MRo\#\string_T1}{}
\markboth{\textcolor{darkblue}{\textbf{\ipa{ʁo˧tsʰe˧ʁo\#˥}}}}{}
\textcolor{teal}{\zh{名词}} \hspace{4pt} \zh{声调类:} \#H.
\zh{顶上,如:山顶。} \textcolor{Sepia}{\selectlanguage{english}Top (e.g. mountain top).} \textcolor{PineGreen}{\selectlanguage{french}Sommet de, en haut de.}  ¶ \textcolor{darkblue}{\textbf{\ipa{ʁwɤ˧-bv̩˧ | ʁo˧tsʰe˧ʁo˧}}} \zh{山的顶,山顶} \textcolor{Sepia}{\selectlanguage{english}the top of the mountain, the mountain top} \textcolor{PineGreen}{\selectlanguage{french}le sommet de la montagne}  
 ¶ \textcolor{darkblue}{\textbf{\ipa{ʁo˧qʰwɤ˩-ʁo˩tsʰe˩}}} \zh{头顶} \textcolor{Sepia}{\selectlanguage{english}the top of the head} \textcolor{PineGreen}{\selectlanguage{french}le sommet de la tête}  

\lhead{\firstmark}
\rhead{\botmark}

\subsection{\hspace{-0.5cm} {\Large \textcolor{darkblue}{\textbf{\ipa{ʁo˧ʈv̩˧ʈv̩˥}}}}\hspace{0.5cm}[\kern2pt{\textcolor{darkblue}{\textbf{\ipa{ʁo˧ʈv̩˧ʈv̩˧}}}}\kern2pt]} \hypertarget{Ro\string_Mt`v\string_=\string_Mt`v\string_=\string_T1}{}
\markboth{\textcolor{darkblue}{\textbf{\ipa{ʁo˧ʈv̩˧ʈv̩˥}}}}{}
\textcolor{teal}{\zh{名词}} \hspace{4pt} \zh{声调类:} H\#.
\zh{彝族(带偏见的说法:“乱糟糟的头发”)。} \textcolor{Sepia}{\selectlanguage{english}Yi (derogatory term: “ungroomed heads”, “messy heads”).} \textcolor{PineGreen}{\selectlanguage{french}Yi (groupe ethnique): terme péjoratif: “les hirsutes”, “les ébouriffés”.}  \zh{量词}: \textcolor{darkblue}{\textbf{\ipa{v̩˧}}} 
\lhead{\firstmark}
\rhead{\botmark}

\subsection{\hspace{-0.5cm} {\Large \textcolor{darkblue}{\textbf{\ipa{ʁo˧ʈʂe˩}}}}\hspace{0.5cm}[\kern2pt{\textcolor{darkblue}{\textbf{\ipa{ʁo˧ʈʂe˩}}}}\kern2pt]} \hypertarget{Ro\string_Mt`s`e\string_B1}{}
\markboth{\textcolor{darkblue}{\textbf{\ipa{ʁo˧ʈʂe˩}}}}{}
\textcolor{teal}{\zh{名词}} \hspace{4pt} \zh{声调类:} L\#.
\zh{癣。} \textcolor{Sepia}{\selectlanguage{english}Tinea, ringworm.} \textcolor{PineGreen}{\selectlanguage{french}Teigne.}  ¶ \textcolor{darkblue}{\textbf{\ipa{[M23] ʁu˧ʈʂɯ˩ ɖwæ˧˥ tʰi˧ di˩!}}} \zh{他长了很多癣!} \textcolor{Sepia}{\selectlanguage{english}(She/he) has a big patch of tinea!} \textcolor{PineGreen}{\selectlanguage{french}(il) a vraiment la teigne à la tête!}  
 \zh{量词}: \textcolor{darkblue}{\textbf{\ipa{pʰæ˧˥}}} 
\lhead{\firstmark}
\rhead{\botmark}

\subsection{\hspace{-0.5cm} {\Large \textcolor{darkblue}{\textbf{\ipa{ʁo˧zo\#˥}}}}\hspace{0.5cm}[\kern2pt{\textcolor{darkblue}{\textbf{\ipa{ʁo˧zo˧˥}}}}\kern2pt]} \hypertarget{Ro\string_Mzo\#\string_T1}{}
\markboth{\textcolor{darkblue}{\textbf{\ipa{ʁo˧zo\#˥}}}}{}
\textcolor{teal}{\zh{名词}} \hspace{4pt} \zh{声调类:} \#H.
\zh{小针。} \textcolor{Sepia}{\selectlanguage{english}Small needle.} \textcolor{PineGreen}{\selectlanguage{french}Petite aiguille.}  \zh{量词}: \textcolor{darkblue}{\textbf{\ipa{ɭɯ˧}}} 
\lhead{\firstmark}
\rhead{\botmark}

\subsection{\hspace{-0.5cm} {\Large \textcolor{darkblue}{\textbf{\ipa{ʁo˧ʑi˧˥}}}}\hspace{0.5cm}[\kern2pt{\textcolor{darkblue}{\textbf{\ipa{ʁo˧ʑi˥}}}}\kern2pt]} \hypertarget{Ro\string_Mz£i\string_M\string_T1}{}
\markboth{\textcolor{darkblue}{\textbf{\ipa{ʁo˧ʑi˧˥}}}}{}
\textcolor{teal}{\zh{助词}} \hspace{4pt} \zh{声调类:} MH\#.
\zh{从……开始。} \textcolor{Sepia}{\selectlanguage{english}As from…, starting….} \textcolor{PineGreen}{\selectlanguage{french}À partir de.}  ¶ \textcolor{darkblue}{\textbf{\ipa{ɖɯ˧ɬi˧mi˧-ʁo˧ʑi˧˥}}} \zh{一月份开始} \textcolor{Sepia}{\selectlanguage{english}from the first month} \textcolor{PineGreen}{\selectlanguage{french}à partir du premier mois}  
 ¶ \textcolor{darkblue}{\textbf{\ipa{tsʰi˧ɲi˧-ʁo˧ʑi˧˥}}} \zh{今天开始} \textcolor{Sepia}{\selectlanguage{english}as from today} \textcolor{PineGreen}{\selectlanguage{french}à partir d'aujourd'hui}  
 ¶ \textcolor{darkblue}{\textbf{\ipa{tsʰi˧ʝi˧ ɖɯ˧-kʰv̩˧˥-ʁo˧ʑi˧˥}}} \zh{今年开始} \textcolor{Sepia}{\selectlanguage{english}from this year on, as from this year} \textcolor{PineGreen}{\selectlanguage{french}à partir de cette année}  
 ¶ \textcolor{darkblue}{\textbf{\ipa{gv̩˩ɬi˩mi˩-ʁo˩ʑi˩˥}}} \zh{九月份开始} \textcolor{Sepia}{\selectlanguage{english}as from the 9th month, starting from the 9th month} \textcolor{PineGreen}{\selectlanguage{french}à partir du 9e mois}  
 ¶ \textcolor{darkblue}{\textbf{\ipa{ʐe˧ʈæ˥ɬi˩-ʁo˩ʑi˩}}} \zh{十一月份开始} \textcolor{Sepia}{\selectlanguage{english}as from the 11th month, starting from the 11th month} \textcolor{PineGreen}{\selectlanguage{french}à partir du 11e mois}  

\lhead{\firstmark}
\rhead{\botmark}

\subsection{\hspace{-0.5cm} {\Large \textcolor{darkblue}{\textbf{\ipa{ʁo˩}}}}\hspace{0.5cm}[\kern2pt{\textcolor{darkblue}{\textbf{\ipa{ʁo˩˥}}}}\kern2pt]} \hypertarget{Ro\string_B1}{}
\markboth{\textcolor{darkblue}{\textbf{\ipa{ʁo˩}}}}{}
\textcolor{teal}{\zh{动词}} \hspace{4pt} \zh{声调类:} L.
\zh{掉入、沉下去。} \textcolor{Sepia}{\selectlanguage{english}To sink (e.g. a boat slowly sinking down into a lake).} \textcolor{PineGreen}{\selectlanguage{french}Tomber, sombrer (ex.: quelqu'un coule dans l'eau; un bateau sombre peu à peu dans le lac).}  ¶ \textcolor{darkblue}{\textbf{\ipa{mv̩˩tɕo˥ ʁo˩}}} \zh{掉入} \textcolor{Sepia}{\selectlanguage{english}to sink} \textcolor{PineGreen}{\selectlanguage{french}s'enfoncer, être englouti (par l'eau…)}  

\lhead{\firstmark}
\rhead{\botmark}

\subsection{\hspace{-0.5cm} {\Large \textcolor{darkblue}{\textbf{\ipa{ʁo˩\textsubscript{b}}}}}\hspace{0.5cm}[\kern2pt{\textcolor{darkblue}{\textbf{\ipa{ʁo˩˥}}}}\kern2pt]} \hypertarget{Ro\string_Bb1}{}
\markboth{\textcolor{darkblue}{\textbf{\ipa{ʁo˩\textsubscript{b}}}}}{}
\textcolor{teal}{\zh{量词}} \hspace{4pt} \zh{声调类:} L\textsubscript{b}.
\zh{量词:种。} \textcolor{Sepia}{\selectlanguage{english}A sort of.} \textcolor{PineGreen}{\selectlanguage{french}Classificateur des variétés/sortes de choses.}  ¶ \textcolor{darkblue}{\textbf{\ipa{ɖɯ˧-ʁo˩}}} \zh{一种(衣服、食物……)} \textcolor{Sepia}{\selectlanguage{english}one type (of clothing, food...)} \textcolor{PineGreen}{\selectlanguage{french}une sorte (de vêtement, de nourriture...)}  
 ¶ \textcolor{darkblue}{\textbf{\ipa{ʈʂʰɯ˧-ʁo˥}}} \zh{这种(衣服、食物……)} \textcolor{Sepia}{\selectlanguage{english}this type (of clothing, food...)} \textcolor{PineGreen}{\selectlanguage{french}cette sorte (de vêtement, de nourriture...)}  

\lhead{\firstmark}
\rhead{\botmark}

\subsection{\hspace{-0.5cm} {\Large \textcolor{darkblue}{\textbf{\ipa{ʁo˩\textsubscript{b}}}}}\hspace{0.5cm}[\kern2pt{\textcolor{darkblue}{\textbf{\ipa{ʁo˥}}}}\kern2pt]} \hypertarget{Ro\string_Bb1}{}
\markboth{\textcolor{darkblue}{\textbf{\ipa{ʁo˩\textsubscript{b}}}}}{}
\textcolor{teal}{\zh{动词}} \hspace{4pt} \zh{声调类:} L\textsubscript{b}.
\zh{出现、形成(如:出了茧子)。} \textcolor{Sepia}{\selectlanguage{english}To form, to appear: e.g. a callus has formed.} \textcolor{PineGreen}{\selectlanguage{french}Se former, apparaître (un cor se forme, un durillon se forme). Ce verbe n'a été observé qu'en association avec le mot 'durillon, cor'.}  ¶ \textcolor{darkblue}{\textbf{\ipa{sɯ˧ʈv̩˥ ʁo˩-ze˩! |}}} \zh{磨出了茧子!} \textcolor{Sepia}{\selectlanguage{english}A callus has formed!} \textcolor{PineGreen}{\selectlanguage{french}Un durillon s'est formé!}  
 ¶ \textcolor{darkblue}{\textbf{\ipa{ʁo˩-mɤ˩-ho˥}}} \zh{不会出(茧子)} \textcolor{Sepia}{\selectlanguage{english}\string_ \mytextsc{neg} \mytextsc{desiderative}} \textcolor{PineGreen}{\selectlanguage{french}\string_ \mytextsc{neg} \mytextsc{désidératif}}  

\lhead{\firstmark}
\rhead{\botmark}

\subsection{\hspace{-0.5cm} {\Large \textcolor{darkblue}{\textbf{\ipa{ʁo˩di˥}}}}\hspace{0.5cm}[\kern2pt{\textcolor{darkblue}{\textbf{\ipa{ʁo˧di˧}}}}\kern2pt]} \hypertarget{Ro\string_Bdi\string_T1}{}
\markboth{\textcolor{darkblue}{\textbf{\ipa{ʁo˩di˥}}}}{}
\textcolor{teal}{\zh{名词}} \hspace{4pt} \zh{声调类:} LH.
\zh{疯子。} \textcolor{Sepia}{\selectlanguage{english}Mad person.} \textcolor{PineGreen}{\selectlanguage{french}Fou, aliéné.}  \zh{量词}: \textcolor{darkblue}{\textbf{\ipa{v̩˧}}} 
\lhead{\firstmark}
\rhead{\botmark}

\subsection{\hspace{-0.5cm} {\Large \textcolor{darkblue}{\textbf{\ipa{ʁo˩ɖɯ˩so˧}}}}\hspace{0.5cm}[\kern2pt{\textcolor{darkblue}{\textbf{\ipa{ʁo˧ɖɯ˧so˧˥}}}}\kern2pt]} \hypertarget{Ro\string_Bd`M\string_Bso\string_M1}{}
\markboth{\textcolor{darkblue}{\textbf{\ipa{ʁo˩ɖɯ˩so˧}}}}{}
\textcolor{teal}{\zh{助词}} \hspace{4pt} \zh{声调类:} .
\zh{大后天。} \textcolor{Sepia}{\selectlanguage{english}In three days.} \textcolor{PineGreen}{\selectlanguage{french}Dans trois jours.} 
\lhead{\firstmark}
\rhead{\botmark}

\subsection{\hspace{-0.5cm} {\Large \textcolor{darkblue}{\textbf{\ipa{ʁo˩hi˩}}}}\hspace{0.5cm}[\kern2pt{\textcolor{darkblue}{\textbf{\ipa{ʁo˩hi˩˥}}}}\kern2pt]} \hypertarget{Ro\string_Bhi\string_B1}{}
\markboth{\textcolor{darkblue}{\textbf{\ipa{ʁo˩hi˩}}}}{}
\textcolor{teal}{\zh{名词}} \hspace{4pt} \zh{声调类:} L.
\zh{臼齿+后臼齿。} \textcolor{Sepia}{\selectlanguage{english}Molars and premolars.} \textcolor{PineGreen}{\selectlanguage{french}Molaires et prémolaires.}  \zh{量词}: \textcolor{darkblue}{\textbf{\ipa{ɭɯ˧}}} 
\lhead{\firstmark}
\rhead{\botmark}

\subsection{\hspace{-0.5cm} {\Large \textcolor{darkblue}{\textbf{\ipa{ʁo˩kʰv̩˩}}}}\hspace{0.5cm}[\kern2pt{\textcolor{darkblue}{\textbf{\ipa{ʁo˩kʰv̩˩˥}}}}\kern2pt]} \hypertarget{Ro\string_Bk\string_hv\string_=\string_B1}{}
\markboth{\textcolor{darkblue}{\textbf{\ipa{ʁo˩kʰv̩˩}}}}{}
\textcolor{teal}{\zh{名词}} \hspace{4pt} \zh{声调类:} L.
\zh{香木。} \textcolor{Sepia}{\selectlanguage{english}Sandalwood, sandlewood.} \textcolor{PineGreen}{\selectlanguage{french}Arbre à épice, arbre à encens de petite taille, qui pousse en montagne, dans les espaces ombragés.} \zh{当地汉语方言:}\zh{柏香。} ¶ \textcolor{darkblue}{\textbf{\ipa{ʁo˩kʰv̩˩-si˩}}} \zh{同上} \textcolor{Sepia}{\selectlanguage{english}same meaning} \textcolor{PineGreen}{\selectlanguage{french}même sens}  
\zh{~【参考】~} \hyperlink{}{\textcolor{darkblue}{\textbf{\ipa{tsɤ˧di˧}}}} 
\lhead{\firstmark}
\rhead{\botmark}

\subsection{\hspace{-0.5cm} {\Large \textcolor{darkblue}{\textbf{\ipa{ʁo˧˥}}}}\hspace{0.5cm}[\kern2pt{\textcolor{darkblue}{\textbf{\ipa{ʁo˧˥}}}}\kern2pt]} \hypertarget{Ro\string_M\string_T1}{}
\markboth{\textcolor{darkblue}{\textbf{\ipa{ʁo˧˥}}}}{}
\textcolor{teal}{\zh{名词}} \hspace{4pt} \zh{声调类:} MH.
\zh{针。} \textcolor{Sepia}{\selectlanguage{english}Needle.} \textcolor{PineGreen}{\selectlanguage{french}Aiguille.}  \zh{量词}: \textcolor{darkblue}{\textbf{\ipa{ɭɯ˧}}} \zh{~【参考】~} \textcolor{darkblue}{\textbf{\ipa{ʈʂe˥}}} 
\lhead{\firstmark}
\rhead{\botmark}

\subsection{\hspace{-0.5cm} {\Large \textcolor{darkblue}{\textbf{\ipa{ʁv̩˧˥}}}}\hspace{0.5cm}[\kern2pt{\textcolor{darkblue}{\textbf{\ipa{ʁv̩˧˥}}}}\kern2pt]} \hypertarget{Rv\string_=\string_M\string_T1}{}
\markboth{\textcolor{darkblue}{\textbf{\ipa{ʁv̩˧˥}}}}{}
\textcolor{teal}{\zh{名词}} \hspace{4pt} \zh{声调类:} MH.
\zh{黑颈鹤(候鸟)。} \textcolor{Sepia}{\selectlanguage{english}Crane (a migratory bird).} \textcolor{PineGreen}{\selectlanguage{french}Grue, Grus nigricollis Przew et autres espèces similaires. Il s'agit d'un oiseau migrateur.}  ¶ \textcolor{darkblue}{\textbf{\ipa{ʁv̩˧nɑ˥mi˩}}} \zh{同上:黑颈鹤} \textcolor{Sepia}{\selectlanguage{english}same meaning: crane} \textcolor{PineGreen}{\selectlanguage{french}même sens: grue}  
 ¶ \textcolor{darkblue}{\textbf{\ipa{ʁv̩˧ dzɯ˥-ze˩}}} \zh{吃了黑颈鹤} \textcolor{Sepia}{\selectlanguage{english}...ate the crane} \textcolor{PineGreen}{\selectlanguage{french}...a mangé une grue}  
 ¶ \textcolor{darkblue}{\textbf{\ipa{ʁv̩˧ hwæ˥-ze˩}}} \zh{买了黑颈鹤} \textcolor{Sepia}{\selectlanguage{english}...bought (a/the) crane} \textcolor{PineGreen}{\selectlanguage{french}...a acheté une grue}  
 \zh{量词}: \textcolor{darkblue}{\textbf{\ipa{mi˩}}} 
\lhead{\firstmark}
\rhead{\botmark}

\subsection{\hspace{-0.5cm} {\Large \textcolor{darkblue}{\textbf{\ipa{ʁv̩˥}}}}\hspace{0.5cm}[\kern2pt{\textcolor{darkblue}{\textbf{\ipa{ʁv̩˥}}}}\kern2pt]} \hypertarget{Rv\string_=\string_T1}{}
\markboth{\textcolor{darkblue}{\textbf{\ipa{ʁv̩˥}}}}{}
\textcolor{teal}{\zh{动词}} \hspace{4pt} \zh{声调类:} H.
\zh{吞,咽。} \textcolor{Sepia}{\selectlanguage{english}To swallow.} \textcolor{PineGreen}{\selectlanguage{french}Avaler, déglutir.}  ¶ \textcolor{darkblue}{\textbf{\ipa{le˧-ʁv̩˥}}} \zh{\mytextsc{accomp}} \textcolor{Sepia}{\selectlanguage{english}\mytextsc{accomp}} \textcolor{PineGreen}{\selectlanguage{french}\mytextsc{accomp}}  

\lhead{\firstmark}
\rhead{\botmark}

\subsection{\hspace{-0.5cm} {\Large \textcolor{darkblue}{\textbf{\ipa{ʁv̩˧mi˥\$}}}}\hspace{0.5cm}[\kern2pt{\textcolor{darkblue}{\textbf{\ipa{ʁv̩˧mi˥}}}}\kern2pt]} \hypertarget{Rv\string_=\string_Mmi\string_T\$1}{}
\markboth{\textcolor{darkblue}{\textbf{\ipa{ʁv̩˧mi˥\$}}}}{}
\textcolor{teal}{\zh{名词}} \hspace{4pt} \zh{声调类:} H\$.
\zh{母鹤。} \textcolor{Sepia}{\selectlanguage{english}Female crane.} \textcolor{PineGreen}{\selectlanguage{french}Grue femelle.}  \zh{量词}: \textcolor{darkblue}{\textbf{\ipa{mi˩}}} 
\lhead{\firstmark}
\rhead{\botmark}

\subsection{\hspace{-0.5cm} {\Large \textcolor{darkblue}{\textbf{\ipa{ʁv̩˧pʰv̩\#˥}}}}\hspace{0.5cm}[\kern2pt{\textcolor{darkblue}{\textbf{\ipa{ʁv̩˧pʰv̩˧}}}}\kern2pt]} \hypertarget{Rv\string_=\string_Mp\string_hv\string_=\#\string_T1}{}
\markboth{\textcolor{darkblue}{\textbf{\ipa{ʁv̩˧pʰv̩\#˥}}}}{}
\textcolor{teal}{\zh{名词}} \hspace{4pt} \zh{声调类:} \#H.
\zh{公鹤。} \textcolor{Sepia}{\selectlanguage{english}Male crane.} \textcolor{PineGreen}{\selectlanguage{french}Grue mâle.}  ¶ \textcolor{darkblue}{\textbf{\ipa{ʁv̩˧pʰv̩˧-ʁv̩˧mi\#˥}}} \zh{公鹤与母鹤} \textcolor{Sepia}{\selectlanguage{english}male crane and female crane} \textcolor{PineGreen}{\selectlanguage{french}grue mâle et grue femelle}  
 \zh{量词}: \textcolor{darkblue}{\textbf{\ipa{mi˩}}} 
\lhead{\firstmark}
\rhead{\botmark}

\subsection{\hspace{-0.5cm} {\Large \textcolor{darkblue}{\textbf{\ipa{ʁv̩˧zo\#˥}}}}\hspace{0.5cm}[\kern2pt{\textcolor{darkblue}{\textbf{\ipa{ʁv̩˧zo˧}}}}\kern2pt]} \hypertarget{Rv\string_=\string_Mzo\#\string_T1}{}
\markboth{\textcolor{darkblue}{\textbf{\ipa{ʁv̩˧zo\#˥}}}}{}
\textcolor{teal}{\zh{名词}} \hspace{4pt} \zh{声调类:} \#H.
\zh{小鹤。} \textcolor{Sepia}{\selectlanguage{english}Baby crane.} \textcolor{PineGreen}{\selectlanguage{french}Enfant grue.}  \zh{量词}: \textcolor{darkblue}{\textbf{\ipa{ɭɯ˧}}} 
\lhead{\firstmark}
\rhead{\botmark}

\subsection{\hspace{-0.5cm} {\Large \textcolor{darkblue}{\textbf{\ipa{ʁwæ˥}}}}\hspace{0.5cm}[\kern2pt{\textcolor{darkblue}{\textbf{\ipa{ʁwæ˥}}}}\kern2pt]} \hypertarget{Rw\{\string_T1}{}
\markboth{\textcolor{darkblue}{\textbf{\ipa{ʁwæ˥}}}}{}
\textcolor{teal}{\zh{名词}} \hspace{4pt} \zh{声调类:} \#H.
\zh{左边(单音节)。} \textcolor{Sepia}{\selectlanguage{english}Left (monosyllable).} \textcolor{PineGreen}{\selectlanguage{french}Gauche (monosyllabe).} 
\lhead{\firstmark}
\rhead{\botmark}

\subsection{\hspace{-0.5cm} {\Large \textcolor{darkblue}{\textbf{\ipa{ʁwæ˧gi\#˥}}}}\hspace{0.5cm}[\kern2pt{\textcolor{darkblue}{\textbf{\ipa{ʁwæ˧gi˧}}}}\kern2pt]} \hypertarget{Rw\{\string_Mgi\#\string_T1}{}
\markboth{\textcolor{darkblue}{\textbf{\ipa{ʁwæ˧gi\#˥}}}}{}
\textcolor{teal}{\zh{名词}} \hspace{4pt} \zh{声调类:} \#H.
\zh{左边。} \textcolor{Sepia}{\selectlanguage{english}Left side, left.} \textcolor{PineGreen}{\selectlanguage{french}Gauche, côté gauche.} 
\lhead{\firstmark}
\rhead{\botmark}

\subsection{\hspace{-0.5cm} {\Large \textcolor{darkblue}{\textbf{\ipa{ʁwæ˧gi˧dzɤ\#˥}}}}\hspace{0.5cm}[\kern2pt{\textcolor{darkblue}{\textbf{\ipa{ʁwæ˧gi˧dzɤ˧}}}}\kern2pt]} \hypertarget{Rw\{\string_Mgi\string_Mdz7\#\string_T1}{}
\markboth{\textcolor{darkblue}{\textbf{\ipa{ʁwæ˧gi˧dzɤ\#˥}}}}{}
\textcolor{teal}{\zh{名词}} \hspace{4pt} \zh{声调类:} \#H.
\zh{左、左边。} \textcolor{Sepia}{\selectlanguage{english}Left, left side, left direction.} \textcolor{PineGreen}{\selectlanguage{french}Gauche, côté gauche, direction de gauche.} 
\lhead{\firstmark}
\rhead{\botmark}

\subsection{\hspace{-0.5cm} {\Large \textcolor{darkblue}{\textbf{\ipa{ʁwæ˧lo˥}}}}\hspace{0.5cm}[\kern2pt{\textcolor{darkblue}{\textbf{\ipa{ʁwæ˧lo˥}}}}\kern2pt]} \hypertarget{Rw\{\string_Mlo\string_T1}{}
\markboth{\textcolor{darkblue}{\textbf{\ipa{ʁwæ˧lo˥}}}}{}
\textcolor{teal}{\zh{名词}} \hspace{4pt} \zh{声调类:} H\#.
\zh{左边,左手。} \textcolor{Sepia}{\selectlanguage{english}Left side, left direction.} \textcolor{PineGreen}{\selectlanguage{french}Gauche; direction de gauche.} 
\lhead{\firstmark}
\rhead{\botmark}

\subsection{\hspace{-0.5cm} {\Large \textcolor{darkblue}{\textbf{\ipa{ʁwæ˧tsɯ˥}}}}\hspace{0.5cm}[\kern2pt{\textcolor{darkblue}{\textbf{\ipa{ʁwæ˧tsɯ˥}}}}\kern2pt]} \hypertarget{Rw\{\string_MtsM\string_T1}{}
\markboth{\textcolor{darkblue}{\textbf{\ipa{ʁwæ˧tsɯ˥}}}}{}
\textcolor{teal}{\zh{名词}} \hspace{4pt} \zh{声调类:} H\#.
\zh{袜子。} \textcolor{Sepia}{\selectlanguage{english}Socks.} \textcolor{PineGreen}{\selectlanguage{french}Chaussettes.}  \zh{【借词】} \zh{袜子}

\lhead{\firstmark}
\rhead{\botmark}

\subsection{\hspace{-0.5cm} {\Large \textcolor{darkblue}{\textbf{\ipa{ʁwæ˧ʈʂʰe˩}}}}\hspace{0.5cm}[\kern2pt{\textcolor{darkblue}{\textbf{\ipa{ʁwæ˧ʈʂʰe˩}}}}\kern2pt]} \hypertarget{Rw\{\string_Mt`s`\string_he\string_B1}{}
\markboth{\textcolor{darkblue}{\textbf{\ipa{ʁwæ˧ʈʂʰe˩}}}}{}
\textcolor{teal}{\zh{动词}} \hspace{4pt} \zh{声调类:} L\#.
\zh{完成(汉语借词)。} \textcolor{Sepia}{\selectlanguage{english}To accomplish, to complete.} \textcolor{PineGreen}{\selectlanguage{french}Achever, mener à terme.}  \zh{【借词】} \zh{完成}
 ¶ \textcolor{darkblue}{\textbf{\ipa{le˧-ʁwæ˧ʈʂʰe˩-ze˩!}}} \zh{完成了!} \textcolor{Sepia}{\selectlanguage{english}It's complete! / It's finished!} \textcolor{PineGreen}{\selectlanguage{french}C'est achevé!}  

\lhead{\firstmark}
\rhead{\botmark}

\subsection{\hspace{-0.5cm} {\Large \textcolor{darkblue}{\textbf{\ipa{ʁwɤ˧}}} \textsubscript{1}}\hspace{0.5cm}[\kern2pt{\textcolor{darkblue}{\textbf{\ipa{ʁwɤ˥}}}}\kern2pt]} \hypertarget{Rw7\string_M1}{}
\markboth{\textcolor{darkblue}{\textbf{\ipa{ʁwɤ˧}}} \textsubscript{1}}{}
\textcolor{teal}{\zh{名词}} \hspace{4pt} \zh{声调类:} M.
\zh{山。} \textcolor{Sepia}{\selectlanguage{english}Mountain.} \textcolor{PineGreen}{\selectlanguage{french}Montagne.}  ¶ \textcolor{darkblue}{\textbf{\ipa{ʁo˧-ʂwæ˧}}} \zh{高山} \textcolor{Sepia}{\selectlanguage{english}high mountain} \textcolor{PineGreen}{\selectlanguage{french}haute montagne}  
 \zh{量词}: \textcolor{darkblue}{\textbf{\ipa{ɭɯ˧}}} 
\lhead{\firstmark}
\rhead{\botmark}

\subsection{\hspace{-0.5cm} {\Large \textcolor{darkblue}{\textbf{\ipa{ʁwɤ˧}}} \textsubscript{2}}\hspace{0.5cm}[\kern2pt{\textcolor{darkblue}{\textbf{\ipa{ʁwɤ˥}}}}\kern2pt]} \hypertarget{Rw7\string_M2}{}
\markboth{\textcolor{darkblue}{\textbf{\ipa{ʁwɤ˧}}} \textsubscript{2}}{}
\textcolor{teal}{\zh{名词}} \hspace{4pt} \zh{声调类:} M.
\zh{村寨,村落。} \textcolor{Sepia}{\selectlanguage{english}Village, hamlet.} \textcolor{PineGreen}{\selectlanguage{french}Village, hameau.}  ¶ \textcolor{darkblue}{\textbf{\ipa{ʁwɤ˧-qo˧}}} \zh{村子里} \textcolor{Sepia}{\selectlanguage{english}in the village} \textcolor{PineGreen}{\selectlanguage{french}dans le village}  
 ¶ \textcolor{darkblue}{\textbf{\ipa{[M23] ɖɯ˧-ʁwɤ˧ mɤ˧-ɲi˩: | ʈʂʰɯ˧-ʁwɤ˧... | ʈʂʰɯ˧-ʁwɤ˧…}}} \zh{它们不属于一个村落:这边,是……村,而那边,是……村。} \textcolor{Sepia}{\selectlanguage{english}These do not belong to the same village: here, it is the village named...; over there, it is the village named...} \textcolor{PineGreen}{\selectlanguage{french}Ce n'est pas le même endroit (littéralement: ce n'est pas le même village): ici, c'est (le village de)…; là, c'est (le village de)…}  
 \zh{量词}: \textcolor{darkblue}{\textbf{\ipa{ʁwɤ˧}}} 
\lhead{\firstmark}
\rhead{\botmark}

\subsection{\hspace{-0.5cm} {\Large \textcolor{darkblue}{\textbf{\ipa{ʁwɤ˧}}} \textsubscript{3}}\hspace{0.5cm}[\kern2pt{\textcolor{darkblue}{\textbf{\ipa{ʁwɤ˥}}}}\kern2pt]} \hypertarget{Rw7\string_M3}{}
\markboth{\textcolor{darkblue}{\textbf{\ipa{ʁwɤ˧}}} \textsubscript{3}}{}
\textcolor{teal}{\zh{名词}} \hspace{4pt} \zh{声调类:} M.
\zh{钱。} \textcolor{Sepia}{\selectlanguage{english}Money.} \textcolor{PineGreen}{\selectlanguage{french}Argent (non pas le métal, mais la monnaie).}  ¶ \textcolor{darkblue}{\textbf{\ipa{ɖʐe˧-ʁwɤ˧}}} \zh{钱} \textcolor{Sepia}{\selectlanguage{english}money} \textcolor{PineGreen}{\selectlanguage{french}argent}  

\lhead{\firstmark}
\rhead{\botmark}

\subsection{\hspace{-0.5cm} {\Large \textcolor{darkblue}{\textbf{\ipa{ʁwɤ˧\textsubscript{a}}}}}\hspace{0.5cm}[\kern2pt{\textcolor{darkblue}{\textbf{\ipa{ʁwɤ˥}}}}\kern2pt]} \hypertarget{Rw7\string_Ma1}{}
\markboth{\textcolor{darkblue}{\textbf{\ipa{ʁwɤ˧\textsubscript{a}}}}}{}
\textcolor{teal}{\zh{量词}} \hspace{4pt} \zh{声调类:} M\textsubscript{a}.
\zh{量词:堆(一堆粮食、一堆柴……)。} \textcolor{Sepia}{\selectlanguage{english}A heap (e.g. of grains, of cut wood); literally: 'a mountain of'.} \textcolor{PineGreen}{\selectlanguage{french}Classificateur des tas / amoncellements (de céréales, de bois coupé...); littéralement: 'une montagne de'.} 
\lhead{\firstmark}
\rhead{\botmark}

\subsection{\hspace{-0.5cm} {\Large \textcolor{darkblue}{\textbf{\ipa{ʁwɤ˧\textsubscript{a}}}}}\hspace{0.5cm}[\kern2pt{\textcolor{darkblue}{\textbf{\ipa{ʁwɤ˥}}}}\kern2pt]} \hypertarget{Rw7\string_Ma1}{}
\markboth{\textcolor{darkblue}{\textbf{\ipa{ʁwɤ˧\textsubscript{a}}}}}{}
\textcolor{teal}{\zh{动词}} \hspace{4pt} \zh{声调类:} M\textsubscript{a}.
\zh{堆 (例如:堆积泥土)。} \textcolor{Sepia}{\selectlanguage{english}To make a heap of (e.g. cereals), to pile up.} \textcolor{PineGreen}{\selectlanguage{french}Amasser, entasser.}  ¶ \textcolor{darkblue}{\textbf{\ipa{ɖɯ˧-ʁwɤ˧ tʰi˧-ʁwɤ˧}}} \zh{堆在一起} \textcolor{Sepia}{\selectlanguage{english}to make a heap, to heap together} \textcolor{PineGreen}{\selectlanguage{french}faire un tas, amasser en tas}  
 ¶ \textcolor{darkblue}{\textbf{\ipa{ɖɯ˧-ʁwɤ˧ tʰi˧-tɕɯ˥}}} \zh{收拾成一堆} \textcolor{Sepia}{\selectlanguage{english}to arrange into a heap} \textcolor{PineGreen}{\selectlanguage{french}ranger en tas}  
 ¶ \textcolor{darkblue}{\textbf{\ipa{tso˧\textasciitilde{}tso˧ | gɤ˩-ʁwɤ˥ lv̩˩}}} \zh{东西堆起来} \textcolor{Sepia}{\selectlanguage{english}to pile up objects} \textcolor{PineGreen}{\selectlanguage{french}entasser des objets}  
 ¶ \textcolor{darkblue}{\textbf{\ipa{ɖɯ˧-ʁwɤ˥-lv̩˩}}} \zh{收拾成一堆(如:有果子散在地上,把它们堆在一起)} \textcolor{Sepia}{\selectlanguage{english}to make into a heap (e.g. nuts, fruit... scattered around)} \textcolor{PineGreen}{\selectlanguage{french}rassembler en un tas (ex.: des piments épars, des noix, des fruits…)}  
 ¶ \textcolor{darkblue}{\textbf{\ipa{tso˧\textasciitilde{}tso˧ ʁwɤ˩}}} \zh{东西堆在一起} \textcolor{Sepia}{\selectlanguage{english}to pile up things} \textcolor{PineGreen}{\selectlanguage{french}entasser des choses}  

\lhead{\firstmark}
\rhead{\botmark}

\subsection{\hspace{-0.5cm} {\Large \textcolor{darkblue}{\textbf{\ipa{ʁwɤ˧lɑ˩-bi˩}}}}\hspace{0.5cm}[\kern2pt{\textcolor{darkblue}{\textbf{\ipa{ʁwɤ˧lɑ˩bi˧}}}}\kern2pt]} \hypertarget{Rw7\string_MlA\string_B-bi\string_B1}{}
\markboth{\textcolor{darkblue}{\textbf{\ipa{ʁwɤ˧lɑ˩-bi˩}}}}{}
\textcolor{teal}{\zh{名词}} \hspace{4pt} \zh{声调类:} L\#-.
\zh{瓦拉别(永宁的一个村落)。} \textcolor{Sepia}{\selectlanguage{english}Walabie, a village of the Yongning plain. It is inhabited by both Na and Pumi.} \textcolor{PineGreen}{\selectlanguage{french}Walabie, un village de la plaine de Yongning. Il est peuplé de Na et de Pumi.}  ¶ \textcolor{darkblue}{\textbf{\ipa{ə˧go˧-ʁwɤ˧, | ʁwɤ˧lɑ˩-bi˩, | bæ˧ʁwɤ˧, | tʰo˧tsʰe\#˥, | pi˧tsʰe˩-di˩, | pɤ˧dʑɤ˩-di˩, | ʁwɤ˧tv̩˧}}} \zh{永宁背向泸沽湖方向经过的村落。前两个村落拥有相当大的摩梭人口比例,第三个村落是摩梭村,最后一个是普米村。} \textcolor{Sepia}{\selectlanguage{english}Villages that one encounters as one leaves the plain of Yongning (away from the Lake); the first two are perceived as villages with a high proportion of Na members, and the third as a mostly Na village, whereas the next ones are Pumi (Prinmi).} \textcolor{PineGreen}{\selectlanguage{french}Villages au sortir de la plaine de Yongning; les deux premiers comportent une population na; le troisième est un village na; les suivants sont essentiellement des villages pumi/prinmi.}  

\lhead{\firstmark}
\rhead{\botmark}

\subsection{\hspace{-0.5cm} {\Large \textcolor{darkblue}{\textbf{\ipa{ʁwɤ˧qʰv̩˧}}}}\hspace{0.5cm}[\kern2pt{\textcolor{darkblue}{\textbf{\ipa{ʁwɤ˧qʰv̩˧}}}}\kern2pt]} \hypertarget{Rw7\string_Mq\string_hv\string_=\string_M1}{}
\markboth{\textcolor{darkblue}{\textbf{\ipa{ʁwɤ˧qʰv̩˧}}}}{}
\textcolor{teal}{\zh{名词}} \hspace{4pt} \zh{声调类:} M.
\zh{山洞。} \textcolor{Sepia}{\selectlanguage{english}Cave, cavern.} \textcolor{PineGreen}{\selectlanguage{french}Grotte, caverne.}  \zh{量词}: \textcolor{darkblue}{\textbf{\ipa{ɭɯ˧}}} \zh{~【参考】~} \hyperlink{}{\textcolor{darkblue}{\textbf{\ipa{ʁwɤ˧qʰv̩˧dʑɯ\#˥}}}} 
\lhead{\firstmark}
\rhead{\botmark}

\subsection{\hspace{-0.5cm} {\Large \textcolor{darkblue}{\textbf{\ipa{ʁwɤ˧qʰv̩˧dʑɯ\#˥}}}}\hspace{0.5cm}[\kern2pt{\textcolor{darkblue}{\textbf{\ipa{ʁwɤ˧qʰv̩˧dʑɯ˧}}}}\kern2pt]} \hypertarget{Rw7\string_Mq\string_hv\string_=\string_Mdz£M\#\string_T1}{}
\markboth{\textcolor{darkblue}{\textbf{\ipa{ʁwɤ˧qʰv̩˧dʑɯ\#˥}}}}{}
\textcolor{teal}{\zh{名词}} \hspace{4pt} \zh{声调类:} \#H.
\zh{山洞。} \textcolor{Sepia}{\selectlanguage{english}Cave, cavern.} \textcolor{PineGreen}{\selectlanguage{french}Grotte, caverne (où il est facile d'entrer).}  \zh{量词}: \textcolor{darkblue}{\textbf{\ipa{ɭɯ˧}}} \zh{~【参考】~} \hyperlink{}{\textcolor{darkblue}{\textbf{\ipa{ʁwɤ˧qʰv̩˧}}}} 
\lhead{\firstmark}
\rhead{\botmark}

\subsection{\hspace{-0.5cm} {\Large \textcolor{darkblue}{\textbf{\ipa{ʁwɤ˧\textasciitilde{}ʁwɤ˥\textsubscript{a}}}}}\hspace{0.5cm}[\kern2pt{\textcolor{darkblue}{\textbf{\ipa{ʁwɤ˩ʁwɤ˩˥}}}}\kern2pt]} \hypertarget{Rw7\string_M~Rw7\string_Ta1}{}
\markboth{\textcolor{darkblue}{\textbf{\ipa{ʁwɤ˧\textasciitilde{}ʁwɤ˥\textsubscript{a}}}}}{}
\textcolor{teal}{\zh{动词}} \hspace{4pt} \zh{声调类:} L\textsubscript{a}.
\zh{商量。} \textcolor{Sepia}{\selectlanguage{english}To discuss, to negociate.} \textcolor{PineGreen}{\selectlanguage{french}Discuter, négocier.}  ¶ \textcolor{darkblue}{\textbf{\ipa{ɖɯ˧-ʁwɤ˧\textasciitilde{}ʁwɤ˥-ɻ̍˩}}} \zh{商量商量} \textcolor{Sepia}{\selectlanguage{english}\mytextsc{delimitative} \string_ \mytextsc{red} \mytextsc{inceptive}} \textcolor{PineGreen}{\selectlanguage{french}\mytextsc{délimitatif} \string_ \mytextsc{red} \mytextsc{inchoatif}}  

\lhead{\firstmark}
\rhead{\botmark}

\subsection{\hspace{-0.5cm} {\Large \textcolor{darkblue}{\textbf{\ipa{ʁwɤ˧tv̩˧}}}}\hspace{0.5cm}[\kern2pt{\textcolor{darkblue}{\textbf{\ipa{ʁwɤ˧tv̩˧}}}}\kern2pt]} \hypertarget{Rw7\string_Mtv\string_=\string_M1}{}
\markboth{\textcolor{darkblue}{\textbf{\ipa{ʁwɤ˧tv̩˧}}}}{}
\textcolor{teal}{\zh{名词}} \hspace{4pt} \zh{声调类:} M.
\zh{温泉乡的一个村落。} \textcolor{Sepia}{\selectlanguage{english}A village near the Hot Springs.} \textcolor{PineGreen}{\selectlanguage{french}Un village proche de Wenquan.}  ¶ \textcolor{darkblue}{\textbf{\ipa{ʁwɤ˧tv̩˧-ʁwɤ˧}}} \zh{同上} \textcolor{Sepia}{\selectlanguage{english}same meaning} \textcolor{PineGreen}{\selectlanguage{french}même sens}  
 ¶ \textcolor{darkblue}{\textbf{\ipa{ə˧go˧-ʁwɤ˧, | ʁwɤ˧lɑ˩-bi˩, | bæ˧ʁwɤ˧, | tʰo˧tsʰe\#˥, | pi˧tsʰe˩-di˩, | pɤ˧dʑɤ˩-di˩, | ʁwɤ˧tv̩˧}}} \zh{永宁背向泸沽湖方向经过的村落。前两个村落拥有相当大的摩梭人口比例,第三个村落是摩梭村,最后一个是普米村。} \textcolor{Sepia}{\selectlanguage{english}Villages that one encounters as one leaves the plain of Yongning (away from the Lake); the first two are perceived as villages with a high proportion of Na members, and the third as a mostly Na village, whereas the next ones are Pumi (Prinmi).} \textcolor{PineGreen}{\selectlanguage{french}Villages au sortir de la plaine de Yongning; les deux premiers comportent une population na; le troisième est un village na; les suivants sont essentiellement des villages pumi/prinmi.}  
 ¶ \textcolor{darkblue}{\textbf{\ipa{ʁwɤ˧tv̩˧: | bɤ˧!}}} \zh{fv:/ʁwɤ˧tv̩˧/是一个普米族村落!} \textcolor{Sepia}{\selectlanguage{english}\textcolor{darkblue}{\textbf{\ipa{/ʁwɤ˧tv̩˧/}}} is a Pumi village!} \textcolor{PineGreen}{\selectlanguage{french}\textcolor{darkblue}{\textbf{\ipa{/ʁwɤ˧tv̩˧/}}}, c'est un village pumi!}  

\lhead{\firstmark}
\rhead{\botmark}

\subsection{\hspace{-0.5cm} {\Large \textcolor{darkblue}{\textbf{\ipa{ʁwɤ˧ʐv̩\#˥}}}}\hspace{0.5cm}[\kern2pt{\textcolor{darkblue}{\textbf{\ipa{ʁwɤ˧ʐv̩˧}}}}\kern2pt]} \hypertarget{Rw7\string_Mz`v\string_=\#\string_T1}{}
\markboth{\textcolor{darkblue}{\textbf{\ipa{ʁwɤ˧ʐv̩\#˥}}}}{}
\textcolor{teal}{\zh{名词}} \hspace{4pt} \zh{声调类:} \#H.
\zh{前所。} \textcolor{Sepia}{\selectlanguage{english}The village of Qiansuo.} \textcolor{PineGreen}{\selectlanguage{french}Qiansuo (localité perçue par F4 comme comportant beaucoup de Yi, et des Chinois/Han, en plus des Na, d'où des contacts linguistiques/emprunts/mélanges).}  ¶ \textcolor{darkblue}{\textbf{\ipa{ʁwɤ˧ʐv̩˧-lo˩mæ˩}}} \zh{同上} \textcolor{Sepia}{\selectlanguage{english}same meaning} \textcolor{PineGreen}{\selectlanguage{french}même sens}  
 ¶ \textcolor{darkblue}{\textbf{\ipa{ʁwɤ˧ʐv̩˧, | jɤ˧qʰɑ˧ dʑɤ˥; | hwɤ˧li˧-hɑ˧ mɤ˧-dʑo˧˥!}}} \zh{俗语:“前所,苦荞(庄稼)很好。猫,没得吃!”(说明:猫不吃苦荞。)} \textcolor{Sepia}{\selectlanguage{english}Adage: “In Qiansuo, bitter buckwheat grows beautifully; there's nothing for cats to eat!” Explanation: cats do not eat bitter buckwheat.} \textcolor{PineGreen}{\selectlanguage{french}dicton ancien: “A Qiansuo, le sarrasin amer pousse à merveille; les chats n'y ont rien à manger!” Explication: les chats ne mangent pas de sarrasin. / le sarrasin est bon, le chat n'aura rien à manger! (le chat passe en miaulant; mais il n'y a rien pour lui!) yyyy}  
 ¶ \textcolor{darkblue}{\textbf{\ipa{ʁwɤ˧ʐv˧, | jɤ˧qʰɑ˧ dʑɤ˥, | hwɤ˧li˧˥ | hɑ˧ mɤ˧-dʑo˧!}}} \zh{同上} \textcolor{Sepia}{\selectlanguage{english}as above} \textcolor{PineGreen}{\selectlanguage{french}comme ci-dessus}  

\lhead{\firstmark}
\rhead{\botmark}

\subsection{\hspace{-0.5cm} {\Large \textcolor{darkblue}{\textbf{\ipa{*ʁwɤ˩\textsubscript{a}}}}}\hspace{0.5cm}[\kern2pt{\textcolor{darkblue}{\textbf{\ipa{ʁwɤ˩˥}}}}\kern2pt]} \hypertarget{*Rw7\string_Ba1}{}
\markboth{\textcolor{darkblue}{\textbf{\ipa{*ʁwɤ˩\textsubscript{a}}}}}{}
\textcolor{teal}{\zh{动词}} \hspace{4pt} \zh{声调类:} L\textsubscript{a}.
\zh{商量(单音节)。} \textcolor{Sepia}{\selectlanguage{english}To negociate (monosyllabic root extracted from the reduplicated form).} \textcolor{PineGreen}{\selectlanguage{french}Discuter, négocier (racine extraite de la forme rédupliquée).} 
\lhead{\firstmark}
\rhead{\botmark}

\subsection{\hspace{-0.5cm} {\Large \textcolor{darkblue}{\textbf{\ipa{ʁwɤ˩ʁo˩}}}}\hspace{0.5cm}[\kern2pt{\textcolor{darkblue}{\textbf{\ipa{ʁwɤ˩ʁo˩˥}}}}\kern2pt]} \hypertarget{Rw7\string_BRo\string_B1}{}
\markboth{\textcolor{darkblue}{\textbf{\ipa{ʁwɤ˩ʁo˩}}}}{}
\textcolor{teal}{\zh{名词}} \hspace{4pt} \zh{声调类:} L.
\zh{山坡。} \textcolor{Sepia}{\selectlanguage{english}Hillside.} \textcolor{PineGreen}{\selectlanguage{french}Collines, versant des montagnes (pas forcément très escarpé, plutôt collines que très fortes pentes).}  ¶ \textcolor{darkblue}{\textbf{\ipa{ʁwɤ˩ʁo˩ dʑɤ˥bv̩˩ ə˩bi˩?}}} \zh{去山上玩,好吗?} \textcolor{Sepia}{\selectlanguage{english}You want to come and have fun on the mountain? Would you like to go and take a stroll on the mountain?} \textcolor{PineGreen}{\selectlanguage{french}tu viens te détendre sur la montagne?}  
 \zh{量词}: \textcolor{darkblue}{\textbf{\ipa{ʁwɤ˧}}} 
\lhead{\firstmark}
\rhead{\botmark}

\newpage
\section*{\centering- \textcolor{darkblue}{\textbf{\ipa{s}}} -}
\subsection{\hspace{-0.5cm} {\Large \textcolor{darkblue}{\textbf{\ipa{sɑ˥}}} \textsubscript{1}}\hspace{0.5cm}[\kern2pt{\textcolor{darkblue}{\textbf{\ipa{sɑ˥}}}}\kern2pt]} \hypertarget{sA\string_T1}{}
\markboth{\textcolor{darkblue}{\textbf{\ipa{sɑ˥}}} \textsubscript{1}}{}
\textcolor{teal}{\zh{名词}} \hspace{4pt} \zh{声调类:} \#H.
\ding{202} \zh{亚麻。} \textcolor{Sepia}{\selectlanguage{english}Flax, \textit{Linum usitatissimum}.} \textcolor{PineGreen}{\selectlanguage{french}Lin, \textit{Linum usitatissimum}, plante textile et oléagineuse.}  \zh{量词}: \textcolor{darkblue}{\textbf{\ipa{qʰwæ˧˥}}} \ding{203} \zh{火麻、胡麻。} \textcolor{Sepia}{\selectlanguage{english}Hemp, \textit{Cannabis sativa}.} \textcolor{PineGreen}{\selectlanguage{french}Chanvre, \textit{Cannabis sativa}, plante textile.}  \zh{量词}: \textcolor{darkblue}{\textbf{\ipa{qʰwæ˧˥}}} 
\lhead{\firstmark}
\rhead{\botmark}

\subsection{\hspace{-0.5cm} {\Large \textcolor{darkblue}{\textbf{\ipa{sɑ˥}}} \textsubscript{2}}\hspace{0.5cm}[\kern2pt{\textcolor{darkblue}{\textbf{\ipa{sɑ˥}}}}\kern2pt]} \hypertarget{sA\string_T2}{}
\markboth{\textcolor{darkblue}{\textbf{\ipa{sɑ˥}}} \textsubscript{2}}{}
\textcolor{teal}{\zh{量词}} \hspace{4pt} \zh{声调类:} H*.
\zh{量词:样,如:‘一样都没有’。} \textcolor{Sepia}{\selectlanguage{english}A thing (no plural; only used in the negative construction “there is not a thing”).} \textcolor{PineGreen}{\selectlanguage{french}Classificateur des choses/objets, utilisé seulement en tournure négative: 'quoi que ce soit'.}  ¶ \textcolor{darkblue}{\textbf{\ipa{ɖɯ˧-sɑ˥ | mɤ˧-dʑo˧!}}} \zh{一样也没有! / 没什么东西!(请客时的礼貌、自我贬低说法:请客人原谅菜不够丰盛)} \textcolor{Sepia}{\selectlanguage{english}There is simply nothing at all! (A polite statement made by the host when welcoming a guest for a meal, apologizing, in self-deprecation, for not offering a meal commensurate to one's wishes.)} \textcolor{PineGreen}{\selectlanguage{french}Il n’y a rien du tout [à manger]! (phrase polie qd on invite quelqu'un à manger: on prie le convive d’excuser la pauvreté des mets proposés)}  
\zh{~【参考】~} \hyperlink{}{\textcolor{darkblue}{\textbf{\ipa{so˥}}} \textsubscript{2}} 
\lhead{\firstmark}
\rhead{\botmark}

\subsection{\hspace{-0.5cm} {\Large \textcolor{darkblue}{\textbf{\ipa{sɑ˧bo\#˥}}}}\hspace{0.5cm}[\kern2pt{\textcolor{darkblue}{\textbf{\ipa{sɑ˧bo˧˥}}}}\kern2pt]} \hypertarget{sA\string_Mbo\#\string_T1}{}
\markboth{\textcolor{darkblue}{\textbf{\ipa{sɑ˧bo\#˥}}}}{}
\textcolor{teal}{\zh{名词}} \hspace{4pt} \zh{声调类:} \#H.
\zh{卷线杆、拉线棒。} \textcolor{Sepia}{\selectlanguage{english}Distaff.} \textcolor{PineGreen}{\selectlanguage{french}Quenouille: instrument en bois pour enrouler le fil, pour filer le chanvre.}  ¶ \textcolor{darkblue}{\textbf{\ipa{sɑ˧bo˧-di˧˥}}} \zh{同上} \textcolor{Sepia}{\selectlanguage{english}same meaning} \textcolor{PineGreen}{\selectlanguage{french}même sens}  
 \zh{量词}: \textcolor{darkblue}{\textbf{\ipa{nɑ˧}}} 
\lhead{\firstmark}
\rhead{\botmark}

\subsection{\hspace{-0.5cm} {\Large \textcolor{darkblue}{\textbf{\ipa{sɑ˧pʰv̩˧˥}}}}\hspace{0.5cm}[\kern2pt{\textcolor{darkblue}{\textbf{\ipa{sɑ˧pʰv̩˧}}}}\kern2pt]} \hypertarget{sA\string_Mp\string_hv\string_=\string_M\string_T1}{}
\markboth{\textcolor{darkblue}{\textbf{\ipa{sɑ˧pʰv̩˧˥}}}}{}
\textcolor{teal}{\zh{名词}} \hspace{4pt} \zh{声调类:} MH\#.
\zh{麻线。} \textcolor{Sepia}{\selectlanguage{english}Thread of linen, \textit{Cannabis sativa}.} \textcolor{PineGreen}{\selectlanguage{french}Fil de lin, \textit{Cannabis sativa}.}  ¶ \textcolor{darkblue}{\textbf{\ipa{sɑ˧pʰv̩˧-sɑ˧jɤ˥}}} \zh{麻线} \textcolor{Sepia}{\selectlanguage{english}linen thread} \textcolor{PineGreen}{\selectlanguage{french}fil de lin}  
 \zh{量词}: \textcolor{darkblue}{\textbf{\ipa{ɭɯ˧}}} 
\lhead{\firstmark}
\rhead{\botmark}

\subsection{\hspace{-0.5cm} {\Large \textcolor{darkblue}{\textbf{\ipa{sɑ˧tɕɯ˧}}}}\hspace{0.5cm}[\kern2pt{\textcolor{darkblue}{\textbf{\ipa{sɑ˧tɕɯ˩}}}}\kern2pt]} \hypertarget{sA\string_Mts£M\string_M1}{}
\markboth{\textcolor{darkblue}{\textbf{\ipa{sɑ˧tɕɯ˧}}}}{}
\textcolor{teal}{\zh{名词}} \hspace{4pt} \zh{声调类:} M.
\zh{女生殖器。} \textcolor{Sepia}{\selectlanguage{english}Vagina.} \textcolor{PineGreen}{\selectlanguage{french}Organe sexuel féminin, vagin (mot tabou).}  \zh{量词}: \textcolor{darkblue}{\textbf{\ipa{ɭɯ˧}}} 
\lhead{\firstmark}
\rhead{\botmark}

\subsection{\hspace{-0.5cm} {\Large \textcolor{darkblue}{\textbf{\ipa{sɑ˧tsʰv̩˩}}}}\hspace{0.5cm}[\kern2pt{\textcolor{darkblue}{\textbf{\ipa{sɑ˧tsʰv̩˧}}}}\kern2pt]} \hypertarget{sA\string_Mts\string_hv\string_=\string_B1}{}
\markboth{\textcolor{darkblue}{\textbf{\ipa{sɑ˧tsʰv̩˩}}}}{}
\textcolor{teal}{\zh{名词}} \hspace{4pt} \zh{声调类:} L\#.
\zh{酸醋(汉语借词)。} \textcolor{Sepia}{\selectlanguage{english}Vinegar.} \textcolor{PineGreen}{\selectlanguage{french}Vinaigre.}  \zh{【借词】} \zh{酸醋}
\zh{~【参考】~} \hyperlink{}{\textcolor{darkblue}{\textbf{\ipa{tsʰv̩˩˥}}}} 
\lhead{\firstmark}
\rhead{\botmark}

\subsection{\hspace{-0.5cm} {\Large \textcolor{darkblue}{\textbf{\ipa{sɑ˩mi˩}}}}\hspace{0.5cm}[\kern2pt{\textcolor{darkblue}{\textbf{\ipa{xxxx non-correspondance entre le nombre de morphèmes et le nombre de tons de morphèmes}}}}\kern2pt]} \hypertarget{sA\string_Bmi\string_B1}{}
\markboth{\textcolor{darkblue}{\textbf{\ipa{sɑ˩mi˩}}}}{}
\textcolor{teal}{\zh{名词}} \hspace{4pt} \zh{声调类:} L.
\zh{大麻。} \textcolor{Sepia}{\selectlanguage{english}Marijuana, cannabis, \textit{Cannabis indica}.} \textcolor{PineGreen}{\selectlanguage{french}Chanvre indien, kanja, marijuana, \textit{Cannabis indica} (plante euphorisante/psychotrope, qui est également comestible: les Na en tiraient de l'huile).}  ¶ \textcolor{darkblue}{\textbf{\ipa{sɑ˩mi˩-mæ˩ɻæ˥, | dzɯ˧-kv̩˩!}}} \zh{大麻油,是可以吃的!} \textcolor{Sepia}{\selectlanguage{english}Cannabis oil is edible!} \textcolor{PineGreen}{\selectlanguage{french}L'huile de lin, c'est comestible/ça se mange!}  
 \zh{量词}: \textcolor{darkblue}{\textbf{\ipa{kɤ˧˥}}} 
\lhead{\firstmark}
\rhead{\botmark}

\subsection{\hspace{-0.5cm} {\Large \textcolor{darkblue}{\textbf{\ipa{sɑ˩tsʰi˩}}} \textsubscript{1}}\hspace{0.5cm}[\kern2pt{\textcolor{darkblue}{\textbf{\ipa{sɑ˧tsʰi˧}}}}\kern2pt]} \hypertarget{sA\string_Bts\string_hi\string_B1}{}
\markboth{\textcolor{darkblue}{\textbf{\ipa{sɑ˩tsʰi˩}}} \textsubscript{1}}{}
\textcolor{teal}{\zh{名词}} \hspace{4pt} \zh{声调类:} L.
\zh{桨。} \textcolor{Sepia}{\selectlanguage{english}Oar.} \textcolor{PineGreen}{\selectlanguage{french}Rame.}  \zh{量词}: \textcolor{darkblue}{\textbf{\ipa{nɑ˧}}} \zh{~【参考】~} \hyperlink{}{\textcolor{darkblue}{\textbf{\ipa{sɑ˩tsʰi˩}}} \textsubscript{2}} 
\lhead{\firstmark}
\rhead{\botmark}

\subsection{\hspace{-0.5cm} {\Large \textcolor{darkblue}{\textbf{\ipa{sɑ˩tsʰi˩}}} \textsubscript{2}}\hspace{0.5cm}[\kern2pt{\textcolor{darkblue}{\textbf{\ipa{sɑ˩tsʰi˩˥}}}}\kern2pt]} \hypertarget{sA\string_Bts\string_hi\string_B2}{}
\markboth{\textcolor{darkblue}{\textbf{\ipa{sɑ˩tsʰi˩}}} \textsubscript{2}}{}
\textcolor{teal}{\zh{名词}} \hspace{4pt} \zh{声调类:} L.
\zh{像桨的木头工具,来搅拌猪食。} \textcolor{Sepia}{\selectlanguage{english}Wooden instrument resembling an oar, used to stir pigswill.} \textcolor{PineGreen}{\selectlanguage{french}Pale en bois utilisée pour touiller la pâtée des cochons; ressemble à une rame: l'ustensile de cuisine est de taille nettement plus petite que la rame des bateaux, mais de forme similaire.} \zh{~【参考】~} \hyperlink{}{\textcolor{darkblue}{\textbf{\ipa{sɑ˩tsʰi˩}}} \textsubscript{1}} 
\lhead{\firstmark}
\rhead{\botmark}

\subsection{\hspace{-0.5cm} {\Large \textcolor{darkblue}{\textbf{\ipa{sɑ˧˥}}}}\hspace{0.5cm}[\kern2pt{\textcolor{darkblue}{\textbf{\ipa{sɑ˧˥}}}}\kern2pt]} \hypertarget{sA\string_M\string_T1}{}
\markboth{\textcolor{darkblue}{\textbf{\ipa{sɑ˧˥}}}}{}
\textcolor{teal}{\zh{动词}} \hspace{4pt} \zh{声调类:} MH.
\zh{运送(货到目的地)。} \textcolor{Sepia}{\selectlanguage{english}To deliver.} \textcolor{PineGreen}{\selectlanguage{french}Livrer (à destination).}  ¶ \textcolor{darkblue}{\textbf{\ipa{le˧-sɑ˧-tʰi˥-ki˩}}} \zh{送(东西到人家里)} \textcolor{Sepia}{\selectlanguage{english}to deliver (to someone's place)} \textcolor{PineGreen}{\selectlanguage{french}même sens: livrer (un objet, une marchandise)}  

\lhead{\firstmark}
\rhead{\botmark}

\subsection{\hspace{-0.5cm} {\Large \textcolor{darkblue}{\textbf{\ipa{sɑ˧˥\textsubscript{a}}}}}\hspace{0.5cm}[\kern2pt{\textcolor{darkblue}{\textbf{\ipa{sɑ˧˥}}}}\kern2pt]} \hypertarget{sA\string_M\string_Ta1}{}
\markboth{\textcolor{darkblue}{\textbf{\ipa{sɑ˧˥\textsubscript{a}}}}}{}
\textcolor{teal}{\zh{量词}} \hspace{4pt} \zh{声调类:} MH\textsubscript{a}.
\zh{量词:腊猪脚(烟熏腊猪蹄子)(一只)。} \textcolor{Sepia}{\selectlanguage{english}Classifier for salted, smoked hog legs.} \textcolor{PineGreen}{\selectlanguage{french}Classificateur des pattes de cochon conservées.}  ¶ \textcolor{darkblue}{\textbf{\ipa{ʂe˧sɑ˩ | ɖɯ˧-sɑ˧˥}}} \zh{一只腊猪脚} \textcolor{Sepia}{\selectlanguage{english}a salted, smoked hog leg} \textcolor{PineGreen}{\selectlanguage{french}une patte de cochon conservée (viande des membres du cochon, conservée --séchée-- avec l'os)}  

\lhead{\firstmark}
\rhead{\botmark}

\subsection{\hspace{-0.5cm} {\Large \textcolor{darkblue}{\textbf{\ipa{sæ˧tsʰɤ˩}}}}\hspace{0.5cm}[\kern2pt{\textcolor{darkblue}{\textbf{\ipa{sæ˩tsʰɤ˩˥}}}}\kern2pt]} \hypertarget{s\{\string_Mts\string_h7\string_B1}{}
\markboth{\textcolor{darkblue}{\textbf{\ipa{sæ˧tsʰɤ˩}}}}{}
\textcolor{teal}{\zh{名词}} \hspace{4pt} \zh{声调类:} L\#.
\zh{酸菜(汉语借词)、泡菜。} \textcolor{Sepia}{\selectlanguage{english}Pickled vegetables.} \textcolor{PineGreen}{\selectlanguage{french}Légumes en saumure.}  \zh{【借词】} \zh{酸菜}

\lhead{\firstmark}
\rhead{\botmark}

\subsection{\hspace{-0.5cm} {\Large \textcolor{darkblue}{\textbf{\ipa{se˥}}}}\hspace{0.5cm}[\kern2pt{\textcolor{darkblue}{\textbf{\ipa{se˥}}}}\kern2pt]} \hypertarget{se\string_T1}{}
\markboth{\textcolor{darkblue}{\textbf{\ipa{se˥}}}}{}
\textcolor{teal}{\zh{动词}} \hspace{4pt} \zh{声调类:} H.
\zh{走、走路。} \textcolor{Sepia}{\selectlanguage{english}To walk.} \textcolor{PineGreen}{\selectlanguage{french}Marcher.}  ¶ \textcolor{darkblue}{\textbf{\ipa{le˧-se˥-ze˩}}} \zh{走了} \textcolor{Sepia}{\selectlanguage{english}\mytextsc{accomp} \string_ \mytextsc{pfv}} \textcolor{PineGreen}{\selectlanguage{french}\mytextsc{accomp} \string_ \mytextsc{pfv}}  
 ¶ \textcolor{darkblue}{\textbf{\ipa{se˧-ho˥-ze˩!}}} \zh{(婴儿)很快就学会走路了!} \textcolor{Sepia}{\selectlanguage{english}[The baby] will soon walk / will soon be able to walk!} \textcolor{PineGreen}{\selectlanguage{french}[Le bébé] va bientôt marcher / va bientôt savoir marcher!}  
 ¶ \textcolor{darkblue}{\textbf{\ipa{ʐɤ˩mi˩-qo˥ | so˩-hɑ̃˩ se˩˥}}} \zh{走在路上三天时间、走三天} \textcolor{Sepia}{\selectlanguage{english}to spend three days on the road, to make a trip that lasts three days} \textcolor{PineGreen}{\selectlanguage{french}passer trois nuits en route / faire un voyage qui va durer trois jours (au sujet d'un trajet de trois nuits de Lijiang à Hanoi: un train de nuit; un car de nuit le lendemain; et un second train de nuit le troisième jour)}  

\lhead{\firstmark}
\rhead{\botmark}

\subsection{\hspace{-0.5cm} {\Large \textcolor{darkblue}{\textbf{\ipa{se˧}}}}\hspace{0.5cm}[\kern2pt{\textcolor{darkblue}{\textbf{\ipa{se˩˥}}}}\kern2pt]} \hypertarget{se\string_M1}{}
\markboth{\textcolor{darkblue}{\textbf{\ipa{se˧}}}}{}
\textcolor{teal}{\zh{名词}} \hspace{4pt} \zh{声调类:} M.
\zh{岩羊。} \textcolor{Sepia}{\selectlanguage{english}Himalayan goral (\textit{Naemorhedus goral}), blue sheep.} \textcolor{PineGreen}{\selectlanguage{french}\textit{Naemorhedus goral}. Le même terme est employé par la locutrice pour décrire des photos de \textit{Pseudois schaeferi}, sorte de bouquetin.}  \zh{量词}: \textcolor{darkblue}{\textbf{\ipa{pʰo˧˥}}} 
\lhead{\firstmark}
\rhead{\botmark}

\subsection{\hspace{-0.5cm} {\Large \textcolor{darkblue}{\textbf{\ipa{se˧gi\#˥}}}}\hspace{0.5cm}[\kern2pt{\textcolor{darkblue}{\textbf{\ipa{se˧gi˧}}}}\kern2pt]} \hypertarget{se\string_Mgi\#\string_T1}{}
\markboth{\textcolor{darkblue}{\textbf{\ipa{se˧gi\#˥}}}}{}
\textcolor{teal}{\zh{名词}} \hspace{4pt} \zh{声调类:} \#H.
\zh{格姆山的藏语名字。} \textcolor{Sepia}{\selectlanguage{english}The Tibetan name of the mountain \textcolor{darkblue}{\textbf{\ipa{/kɤ˧mv̩˧˥/}}} (Chinese name: Gemu).} \textcolor{PineGreen}{\selectlanguage{french}Nom anciennement donné par les Tibétains à la montagne \textcolor{darkblue}{\textbf{\ipa{/kɤ˧mv̩˧˥/}}} (nom chinois: Gemu).}  ¶ \textcolor{darkblue}{\textbf{\ipa{se˧gi˧-kɤ˩mv̩˩}}} \zh{同上} \textcolor{Sepia}{\selectlanguage{english}same meaning} \textcolor{PineGreen}{\selectlanguage{french}même sens}  

\lhead{\firstmark}
\rhead{\botmark}

\subsection{\hspace{-0.5cm} {\Large \textcolor{darkblue}{\textbf{\ipa{se˧kʰɯ˩}}}}\hspace{0.5cm}[\kern2pt{\textcolor{darkblue}{\textbf{\ipa{se˩kʰɯ˥}}}}\kern2pt]} \hypertarget{se\string_Mk\string_hM\string_B1}{}
\markboth{\textcolor{darkblue}{\textbf{\ipa{se˧kʰɯ˩}}}}{}
\textcolor{teal}{\zh{名词}} \hspace{4pt} \zh{声调类:} L\#.
\zh{缎子。} \textcolor{Sepia}{\selectlanguage{english}Satin.} \textcolor{PineGreen}{\selectlanguage{french}Satin.}  ¶ \textcolor{darkblue}{\textbf{\ipa{se˧kʰɯ˩-ʁo˩ni˩}}} \zh{缎子发带} \textcolor{Sepia}{\selectlanguage{english}satin headdressxxxx} \textcolor{PineGreen}{\selectlanguage{french}coiffe en satin}  
 \zh{量词}: \textcolor{darkblue}{\textbf{\ipa{kʰɯ˩}}} 
\lhead{\firstmark}
\rhead{\botmark}

\subsection{\hspace{-0.5cm} {\Large \textcolor{darkblue}{\textbf{\ipa{se˧mi\#˥}}}}\hspace{0.5cm}[\kern2pt{\textcolor{darkblue}{\textbf{\ipa{se˧mi˧}}}}\kern2pt]} \hypertarget{se\string_Mmi\#\string_T1}{}
\markboth{\textcolor{darkblue}{\textbf{\ipa{se˧mi\#˥}}}}{}
\textcolor{teal}{\zh{名词}} \hspace{4pt} \zh{声调类:} \#H.
\zh{母岩羊。} \textcolor{Sepia}{\selectlanguage{english}Female goral (\textit{Naemorhedus goral}), female blue sheep.} \textcolor{PineGreen}{\selectlanguage{french}\textit{Naemorhedus goral} femelle.}  \zh{量词}: \textcolor{darkblue}{\textbf{\ipa{mi˩}}} 
\lhead{\firstmark}
\rhead{\botmark}

\subsection{\hspace{-0.5cm} {\Large \textcolor{darkblue}{\textbf{\ipa{se˧nɑ\#˥}}}}\hspace{0.5cm}[\kern2pt{\textcolor{darkblue}{\textbf{\ipa{se˧nɑ˩}}}}\kern2pt]} \hypertarget{se\string_MnA\#\string_T1}{}
\markboth{\textcolor{darkblue}{\textbf{\ipa{se˧nɑ\#˥}}}}{}
\textcolor{teal}{\zh{形容词}} \hspace{4pt} \zh{声调类:} \#H.
\zh{吝啬。} \textcolor{Sepia}{\selectlanguage{english}Stingy, miserly.} \textcolor{PineGreen}{\selectlanguage{french}Avare.}  ¶ \textcolor{darkblue}{\textbf{\ipa{ʈʂʰɯ˧ | se˧nɑ˧-hĩ˧ ɖɯ˧-v̩˧ ɲi˩!}}} \zh{他是一个吝啬的人!} \textcolor{Sepia}{\selectlanguage{english}It's a stingy person!} \textcolor{PineGreen}{\selectlanguage{french}C'est quelqu'un d'avare!}  
 ¶ \textcolor{darkblue}{\textbf{\ipa{ʈʂʰɯ˧ | ə˧-se˧nɑ˧? - se˧nɑ˧ | ʐwæ˩˥!}}} \zh{他吝啬吗? - 非常吝啬!} \textcolor{Sepia}{\selectlanguage{english}Is he stingy? - Oh yes, very much so!} \textcolor{PineGreen}{\selectlanguage{french}Est-il avare? - Oui, très avare!}  

\lhead{\firstmark}
\rhead{\botmark}

\subsection{\hspace{-0.5cm} {\Large \textcolor{darkblue}{\textbf{\ipa{se˧pʰɤ˧}}}}\hspace{0.5cm}[\kern2pt{\textcolor{darkblue}{\textbf{\ipa{se˧pʰɤ˩}}}}\kern2pt]} \hypertarget{se\string_Mp\string_h7\string_M1}{}
\markboth{\textcolor{darkblue}{\textbf{\ipa{se˧pʰɤ˧}}}}{}
\textcolor{teal}{\zh{名词}} \hspace{4pt} \zh{声调类:} M.
\zh{大惊小怪,麻烦。} \textcolor{Sepia}{\selectlanguage{english}Fuss.} \textcolor{PineGreen}{\selectlanguage{french}Complications.}  ¶ \textcolor{darkblue}{\textbf{\ipa{se˧pʰɤ˧ ʝi˧}}} \zh{小事大作} \textcolor{Sepia}{\selectlanguage{english}to make a big fuss about something} \textcolor{PineGreen}{\selectlanguage{french}se faire toute une affaire de quelque chose, s’en faire au point de porter comme une pierre dans le cœur}  
 \zh{量词}: \textcolor{darkblue}{\textbf{\ipa{kʰwɤ˥}}} 
\lhead{\firstmark}
\rhead{\botmark}

\subsection{\hspace{-0.5cm} {\Large \textcolor{darkblue}{\textbf{\ipa{se˧pʰv̩\#˥}}}}\hspace{0.5cm}[\kern2pt{\textcolor{darkblue}{\textbf{\ipa{se˧pʰv̩˩}}}}\kern2pt]} \hypertarget{se\string_Mp\string_hv\string_=\#\string_T1}{}
\markboth{\textcolor{darkblue}{\textbf{\ipa{se˧pʰv̩\#˥}}}}{}
\textcolor{teal}{\zh{名词}} \hspace{4pt} \zh{声调类:} \#H.
\zh{公岩羊。} \textcolor{Sepia}{\selectlanguage{english}Male goral (\textit{Naemorhedus goral}), male blue sheep.} \textcolor{PineGreen}{\selectlanguage{french}\textit{Naemorhedus goral} mâle.}  \zh{量词}: \textcolor{darkblue}{\textbf{\ipa{mi˩}}} 
\lhead{\firstmark}
\rhead{\botmark}

\subsection{\hspace{-0.5cm} {\Large \textcolor{darkblue}{\textbf{\ipa{se˧ʂɯ˩}}}}\hspace{0.5cm}[\kern2pt{\textcolor{darkblue}{\textbf{\ipa{se˧ʂɯ˧}}}}\kern2pt]} \hypertarget{se\string_Ms`M\string_B1}{}
\markboth{\textcolor{darkblue}{\textbf{\ipa{se˧ʂɯ˩}}}}{}
\textcolor{teal}{\zh{动词}} \hspace{4pt} \zh{声调类:} L\#.
\zh{浪费。} \textcolor{Sepia}{\selectlanguage{english}To waste.} \textcolor{PineGreen}{\selectlanguage{french}Gaspiller.}  ¶ \textcolor{darkblue}{\textbf{\ipa{ɖwæ˧˥ | se˧ʂɯ˩!}}} \zh{很浪费! / 太浪费了!} \textcolor{Sepia}{\selectlanguage{english}It's a waste!} \textcolor{PineGreen}{\selectlanguage{french}C'est du gaspillage!}  

\lhead{\firstmark}
\rhead{\botmark}

\subsection{\hspace{-0.5cm} {\Large \textcolor{darkblue}{\textbf{\ipa{se˧tʰo˩}}}}\hspace{0.5cm}[\kern2pt{\textcolor{darkblue}{\textbf{\ipa{se˩tʰo˩˥}}}}\kern2pt]} \hypertarget{se\string_Mt\string_ho\string_B1}{}
\markboth{\textcolor{darkblue}{\textbf{\ipa{se˧tʰo˩}}}}{}
\textcolor{teal}{\zh{名词}} \hspace{4pt} \zh{声调类:} L.
\zh{榫头(汉语借词)。} \textcolor{Sepia}{\selectlanguage{english}Tenon.} \textcolor{PineGreen}{\selectlanguage{french}Tenon.}  \zh{【借词】} \zh{榫头}
 ¶ \textcolor{darkblue}{\textbf{\ipa{se˧tʰo˩ | ɖɯ˧-ɭɯ˧}}} \zh{一个榫头} \textcolor{Sepia}{\selectlanguage{english}a tenon} \textcolor{PineGreen}{\selectlanguage{french}un tenon}  
 \zh{量词}: \textcolor{darkblue}{\textbf{\ipa{ɭɯ˧}}} 
\lhead{\firstmark}
\rhead{\botmark}

\subsection{\hspace{-0.5cm} {\Large \textcolor{darkblue}{\textbf{\ipa{se˧zo\#˥}}}}\hspace{0.5cm}[\kern2pt{\textcolor{darkblue}{\textbf{\ipa{se˧zo˧}}}}\kern2pt]} \hypertarget{se\string_Mzo\#\string_T1}{}
\markboth{\textcolor{darkblue}{\textbf{\ipa{se˧zo\#˥}}}}{}
\textcolor{teal}{\zh{名词}} \hspace{4pt} \zh{声调类:} \#H.
\zh{小岩羊。} \textcolor{Sepia}{\selectlanguage{english}Baby goral, baby blue sheep.} \textcolor{PineGreen}{\selectlanguage{french}Petit de \textit{Naemorhedus goral}.}  \zh{量词}: \textcolor{darkblue}{\textbf{\ipa{ɭɯ˧}}} 
\lhead{\firstmark}
\rhead{\botmark}

\subsection{\hspace{-0.5cm} {\Large \textcolor{darkblue}{\textbf{\ipa{se˧ʐɯ˩}}}}\hspace{0.5cm}[\kern2pt{\textcolor{darkblue}{\textbf{\ipa{se˧ʐɯ˧}}}}\kern2pt]} \hypertarget{se\string_Mz`M\string_B1}{}
\markboth{\textcolor{darkblue}{\textbf{\ipa{se˧ʐɯ˩}}}}{}
\textcolor{teal}{\zh{名词}} \hspace{4pt} \zh{声调类:} L\#.
\zh{生日(汉语借词)。} \textcolor{Sepia}{\selectlanguage{english}Birthday.} \textcolor{PineGreen}{\selectlanguage{french}Anniversaire.}  \zh{【借词】} \zh{生日}
 ¶ \textcolor{darkblue}{\textbf{\ipa{se˧ʐɯ˩ ko˩}}} \zh{过生日} \textcolor{Sepia}{\selectlanguage{english}to celebrate a birthday} \textcolor{PineGreen}{\selectlanguage{french}fêter un anniversaire}  

\lhead{\firstmark}
\rhead{\botmark}

\subsection{\hspace{-0.5cm} {\Large \textcolor{darkblue}{\textbf{\ipa{‑se˩}}}}\hspace{0.5cm}[\kern2pt{\textcolor{darkblue}{\textbf{\ipa{se˩˥}}}}\kern2pt]} \hypertarget{‑se\string_B1}{}
\markboth{\textcolor{darkblue}{\textbf{\ipa{‑se˩}}}}{}
\textcolor{teal}{\zh{后缀}} \hspace{4pt} \zh{声调类:} L.
\zh{\mytextsc{完成。}} \textcolor{Sepia}{\selectlanguage{english}Suffix indicating the completion of an action: the action has reached its end.} \textcolor{PineGreen}{\selectlanguage{french}Suffixe indiquant l'achèvement d'une action: l'action est conduite à son terme.}  ¶ \textcolor{darkblue}{\textbf{\ipa{se˩-ze˥!}}} \zh{完了!} \textcolor{Sepia}{\selectlanguage{english}It's finished! / It's completed!} \textcolor{PineGreen}{\selectlanguage{french}C'est fini!}  
 ¶ \textcolor{darkblue}{\textbf{\ipa{no˧ | tʰi˧-dzi˩-kʰɯ˩-se˩-dʑo˩, | dʑɯ˩-tsʰi˧ ɖɯ˧-qʰwɤ˧ pʰv̩˥ | tʰi˧-ki˧!}}} \zh{让(你)坐下以后,(我)给你倒一杯开水。} \textcolor{Sepia}{\selectlanguage{english}After you have been seated, (I) pour out a bowl of hot water (for you).} \textcolor{PineGreen}{\selectlanguage{french}Après que tu te sois assis, je te verse un verre d’eau chaude.}  

\lhead{\firstmark}
\rhead{\botmark}

\subsection{\hspace{-0.5cm} {\Large \textcolor{darkblue}{\textbf{\ipa{se˩\textsubscript{a}}}}}\hspace{0.5cm}[\kern2pt{\textcolor{darkblue}{\textbf{\ipa{se˥}}}}\kern2pt]} \hypertarget{se\string_Ba1}{}
\markboth{\textcolor{darkblue}{\textbf{\ipa{se˩\textsubscript{a}}}}}{}
\textcolor{teal}{\zh{动词}} \hspace{4pt} \zh{声调类:} L\textsubscript{a}.
\zh{完成。} \textcolor{Sepia}{\selectlanguage{english}To finish, to complete.} \textcolor{PineGreen}{\selectlanguage{french}Achever.}  ¶ \textcolor{darkblue}{\textbf{\ipa{le˧-ʝi˥ | le˧-se˩-ze˩!}}} \zh{做完了! / 完成了!} \textcolor{Sepia}{\selectlanguage{english}It's done and finished! / (I) have finished (the job)!} \textcolor{PineGreen}{\selectlanguage{french}(je) l'ai fait, j'ai fini!}  
 ¶ \textcolor{darkblue}{\textbf{\ipa{le˧-se˧\textasciitilde{}se˥-ze˩!}}} \zh{完成了!} \textcolor{Sepia}{\selectlanguage{english}It's finished, it's completed! / It's now over and done with!} \textcolor{PineGreen}{\selectlanguage{french}C'est fini, c'est achevé!}  
 ¶ \textcolor{darkblue}{\textbf{\ipa{mɤ˧-se˩}}} \zh{没有完!} \textcolor{Sepia}{\selectlanguage{english}It's not finished!} \textcolor{PineGreen}{\selectlanguage{french}Ce n'est pas fini!}  
 ¶ \textcolor{darkblue}{\textbf{\ipa{se˩˥ | -dʑo˩, | se˩-mɤ˩-tʰɑ˩˥! | dʑɤ˩˥ | -dʑo˩, | dʑɤ˩-kʰɯ˧ tʰɑ˥!}}} \zh{(想做)完,但是没办法做完!不过最后还是可以做得很好!(情景:谈及收集语言的工作)} \textcolor{Sepia}{\selectlanguage{english}A comment about linguistic documentation, summarizing an explanation provided by a student: “One cannot complete the task (=one cannot collect everything: every single word, every single turn of phrase, every single story...); but one can do nice things (=collect stories which constitute complete, interesting documents)”.} \textcolor{PineGreen}{\selectlanguage{french}Parenthèse au sujet de la documentation linguistique: on ne peut pas en voir le bout (tout collecter de façon exhaustive); mais on peut réaliser de belles choses!}  

\lhead{\firstmark}
\rhead{\botmark}

\subsection{\hspace{-0.5cm} {\Large \textcolor{darkblue}{\textbf{\ipa{se˩di˩}}}}\hspace{0.5cm}[\kern2pt{\textcolor{darkblue}{\textbf{\ipa{se˧di˧}}}}\kern2pt]} \hypertarget{se\string_Bdi\string_B1}{}
\markboth{\textcolor{darkblue}{\textbf{\ipa{se˩di˩}}}}{}
\textcolor{teal}{\zh{名词}} \hspace{4pt} \zh{声调类:} L.
\zh{锯。} \textcolor{Sepia}{\selectlanguage{english}Saw.} \textcolor{PineGreen}{\selectlanguage{french}Scie.}  ¶ \textcolor{darkblue}{\textbf{\ipa{se˩di˩˥ | ɖɯ˩-hĩ˩˥ | ɖɯ˧-nɑ˧}}} \zh{一把大锯} \textcolor{Sepia}{\selectlanguage{english}a large saw} \textcolor{PineGreen}{\selectlanguage{french}une grande scie}  
 ¶ \textcolor{darkblue}{\textbf{\ipa{se˩di˩˥ | tɕi˩-hĩ˩˥ | ɖɯ˧-nɑ˧}}} \zh{一把小锯} \textcolor{Sepia}{\selectlanguage{english}a small saw} \textcolor{PineGreen}{\selectlanguage{french}une petite scie}  
 ¶ \textcolor{darkblue}{\textbf{\ipa{se˩di˩˥ | ɬi˧-hĩ˧ | ɖɯ˧-nɑ˩}}} \zh{中间大小的锯子} \textcolor{Sepia}{\selectlanguage{english}a medium-sized saw} \textcolor{PineGreen}{\selectlanguage{french}une scie de taille moyenne}  
 \zh{量词}: \textcolor{darkblue}{\textbf{\ipa{nɑ˧}}} 
\lhead{\firstmark}
\rhead{\botmark}

\subsection{\hspace{-0.5cm} {\Large \textcolor{darkblue}{\textbf{\ipa{se˩gwɤ˩mi˥}}}}\hspace{0.5cm}[\kern2pt{\textcolor{darkblue}{\textbf{\ipa{se˧gwɤ˧mi˧}}}}\kern2pt]} \hypertarget{se\string_Bgw7\string_Bmi\string_T1}{}
\markboth{\textcolor{darkblue}{\textbf{\ipa{se˩gwɤ˩mi˥}}}}{}
\textcolor{teal}{\zh{名词}} \hspace{4pt} \zh{声调类:} L+H\#.
\zh{雕(不仅来指母雕)。} \textcolor{Sepia}{\selectlanguage{english}Vulture. This term is not restricted to female vultures, and hence does not provide an indication on sex.} \textcolor{PineGreen}{\selectlanguage{french}Vautour. Le terme n'est pas restreint aux vautours femelles; dans l'état actuel de la langue, il ne fournit pas d'indication de sexe.}  ¶ \textcolor{darkblue}{\textbf{\ipa{se˩gwɤ˩mi˥-pʰv̩˩}}} \zh{公雕} \textcolor{Sepia}{\selectlanguage{english}male vulture} \textcolor{PineGreen}{\selectlanguage{french}vautour mâle}  
 ¶ \textcolor{darkblue}{\textbf{\ipa{se˩gwɤ˩mi˥-zo˩}}} \zh{小雕} \textcolor{Sepia}{\selectlanguage{english}baby vulture} \textcolor{PineGreen}{\selectlanguage{french}petit vautour, bébé vautour}  
 ¶ \textcolor{darkblue}{\textbf{\ipa{se˩gwɤ˩mi˥-ʈʂʰɯ˩, | mi˩ ɲi˥!}}} \zh{这只雕是母的!} \textcolor{Sepia}{\selectlanguage{english}This vulture is a female!} \textcolor{PineGreen}{\selectlanguage{french}Ce vautour, c'est une femelle!}  
 \zh{量词}: \textcolor{darkblue}{\textbf{\ipa{mi˩}}} 
\lhead{\firstmark}
\rhead{\botmark}

\subsection{\hspace{-0.5cm} {\Large \textcolor{darkblue}{\textbf{\ipa{sɤ˥}}}}\hspace{0.5cm}[\kern2pt{\textcolor{darkblue}{\textbf{\ipa{sɤ˥}}}}\kern2pt]} \hypertarget{s7\string_T1}{}
\markboth{\textcolor{darkblue}{\textbf{\ipa{sɤ˥}}}}{}
\textcolor{teal}{\zh{名词}} \hspace{4pt} \zh{声调类:} \#H.
\zh{血。} \textcolor{Sepia}{\selectlanguage{english}Blood.} \textcolor{PineGreen}{\selectlanguage{french}Sang.}  \zh{量词}: \textcolor{darkblue}{\textbf{\ipa{ʈʰɤ˥}}} 
\lhead{\firstmark}
\rhead{\botmark}

\subsection{\hspace{-0.5cm} {\Large \textcolor{darkblue}{\textbf{\ipa{sɤ˧ɭɯ˩}}}}\hspace{0.5cm}[\kern2pt{\textcolor{darkblue}{\textbf{\ipa{sɤ˧ɭɯ˧}}}}\kern2pt]} \hypertarget{s7\string_Ml\string_RM\string_B1}{}
\markboth{\textcolor{darkblue}{\textbf{\ipa{sɤ˧ɭɯ˩}}}}{}
\textcolor{teal}{\zh{名词}} \hspace{4pt} \zh{声调类:} L\#.
\zh{梨子。} \textcolor{Sepia}{\selectlanguage{english}Pear.} \textcolor{PineGreen}{\selectlanguage{french}Poire.}  \zh{量词}: \textcolor{darkblue}{\textbf{\ipa{kʰɤ˧˥}}} \textcolor{darkblue}{\textbf{\ipa{ɭɯ˧}}} 
\lhead{\firstmark}
\rhead{\botmark}

\subsection{\hspace{-0.5cm} {\Large \textcolor{darkblue}{\textbf{\ipa{sɤ˧sɤ˧˥}}}}\hspace{0.5cm}[\kern2pt{\textcolor{darkblue}{\textbf{\ipa{sɤ˧sɤ˧˥}}}}\kern2pt]} \hypertarget{s7\string_Ms7\string_M\string_T1}{}
\markboth{\textcolor{darkblue}{\textbf{\ipa{sɤ˧sɤ˧˥}}}}{}
\textcolor{teal}{\zh{形容词}} \hspace{4pt} \zh{声调类:} MH\#.
\zh{舒畅。} \textcolor{Sepia}{\selectlanguage{english}Pleasant (circumstances).} \textcolor{PineGreen}{\selectlanguage{french}Agréable, plaisant (circonstances).}  ¶ \textcolor{darkblue}{\textbf{\ipa{si˧dzi˩-ʈʰæ˩qo˩dzi˩, | sɤ˧sɤ˧˥ | ʐwæ˩˥!}}} \zh{在树下坐着,感到很舒畅!} \textcolor{Sepia}{\selectlanguage{english}Being seated under this tree is especially pleasant!} \textcolor{PineGreen}{\selectlanguage{french}assis sous cet arbre, c'est le bonheur!}  
 ¶ \textcolor{darkblue}{\textbf{\ipa{ʈʂʰɯ˧-ɳɯ˧ | ɖɯ˧-ɖʐɯ˩ gwɤ˩-dʑo˩, | sɤ˧sɤ˧˥ | ʐwæ˩˥!}}} \zh{他唱了一会,真舒畅!} \textcolor{Sepia}{\selectlanguage{english}He has sung for a while; it was really pleasant!} \textcolor{PineGreen}{\selectlanguage{french}Il a chanté un moment; c'était vraiment plaisant!}  

\lhead{\firstmark}
\rhead{\botmark}

\subsection{\hspace{-0.5cm} {\Large \textcolor{darkblue}{\textbf{\ipa{sɤ˧tʰo˧˥}}}}\hspace{0.5cm}[\kern2pt{\textcolor{darkblue}{\textbf{\ipa{sɤ˧tʰo˧˥}}}}\kern2pt]} \hypertarget{s7\string_Mt\string_ho\string_M\string_T1}{}
\markboth{\textcolor{darkblue}{\textbf{\ipa{sɤ˧tʰo˧˥}}}}{}
\textcolor{teal}{\zh{名词}} \hspace{4pt} \zh{声调类:} MH\#.
\zh{一种松树。} \textcolor{Sepia}{\selectlanguage{english}A type of pine.} \textcolor{PineGreen}{\selectlanguage{french}Sorte de pin.} \zh{当地汉语方言:}\zh{阔松。} ¶ \textcolor{darkblue}{\textbf{\ipa{sɤ˧tʰo˧-dzi˧˥}}} \zh{同上} \textcolor{Sepia}{\selectlanguage{english}same meaning} \textcolor{PineGreen}{\selectlanguage{french}même sens (désigne une espèce de pin)}  
 \zh{量词}: \textcolor{darkblue}{\textbf{\ipa{dzi˩}}} 
\lhead{\firstmark}
\rhead{\botmark}

\subsection{\hspace{-0.5cm} {\Large \textcolor{darkblue}{\textbf{\ipa{sɤ˧tsi˥}}}}\hspace{0.5cm}[\kern2pt{\textcolor{darkblue}{\textbf{\ipa{sɤ˧tsi˥}}}}\kern2pt]} \hypertarget{s7\string_Mtsi\string_T1}{}
\markboth{\textcolor{darkblue}{\textbf{\ipa{sɤ˧tsi˥}}}}{}
\textcolor{teal}{\zh{名词}} \hspace{4pt} \zh{声调类:} H\#.
\zh{血管。} \textcolor{Sepia}{\selectlanguage{english}Vein.} \textcolor{PineGreen}{\selectlanguage{french}Veines.}  \zh{量词}: \textcolor{darkblue}{\textbf{\ipa{kʰɯ˩}}} 
\lhead{\firstmark}
\rhead{\botmark}

\subsection{\hspace{-0.5cm} {\Large \textcolor{darkblue}{\textbf{\ipa{sɤ˩˥}}}}\hspace{0.5cm}[\kern2pt{\textcolor{darkblue}{\textbf{\ipa{sɤ˩˥}}}}\kern2pt]} \hypertarget{s7\string_B\string_T1}{}
\markboth{\textcolor{darkblue}{\textbf{\ipa{sɤ˩˥}}}}{}
\textcolor{teal}{\zh{名词}} \hspace{4pt} \zh{声调类:} LH.
\zh{黑痣。} \textcolor{Sepia}{\selectlanguage{english}Mole; pigmented naevus.} \textcolor{PineGreen}{\selectlanguage{french}Grain de beauté.}  \zh{量词}: \textcolor{darkblue}{\textbf{\ipa{ɭɯ˧}}} 
\lhead{\firstmark}
\rhead{\botmark}

\subsection{\hspace{-0.5cm} {\Large \textcolor{darkblue}{\textbf{\ipa{si˥}}}}\hspace{0.5cm}[\kern2pt{\textcolor{darkblue}{\textbf{\ipa{si˧˥}}}}\kern2pt]} \hypertarget{si\string_T1}{}
\markboth{\textcolor{darkblue}{\textbf{\ipa{si˥}}}}{}
\textcolor{teal}{\zh{名词}} \hspace{4pt} \zh{声调类:} \#H.
\zh{木头。} \textcolor{Sepia}{\selectlanguage{english}Wood.} \textcolor{PineGreen}{\selectlanguage{french}Bois.}  ¶ \textcolor{darkblue}{\textbf{\ipa{si˧-mo˩}}} \zh{枯木} \textcolor{Sepia}{\selectlanguage{english}dead wood} \textcolor{PineGreen}{\selectlanguage{french}bois mort}  
 \zh{量词}: \textcolor{darkblue}{\textbf{\ipa{kɤ˧˥}}} 
\lhead{\firstmark}
\rhead{\botmark}

\subsection{\hspace{-0.5cm} {\Large \textcolor{darkblue}{\textbf{\ipa{si˧\textsubscript{a}}}}}\hspace{0.5cm}[\kern2pt{\textcolor{darkblue}{\textbf{\ipa{si˥}}}}\kern2pt]} \hypertarget{si\string_Ma1}{}
\markboth{\textcolor{darkblue}{\textbf{\ipa{si˧\textsubscript{a}}}}}{}
\textcolor{teal}{\zh{动词}} \hspace{4pt} \zh{声调类:} M\textsubscript{a}.
\zh{挑选。} \textcolor{Sepia}{\selectlanguage{english}To choose, to select.} \textcolor{PineGreen}{\selectlanguage{french}Choisir.}  ¶ \textcolor{darkblue}{\textbf{\ipa{le˧-si˧-ze˧}}} \zh{选了} \textcolor{Sepia}{\selectlanguage{english}\mytextsc{accomp} \string_ \mytextsc{pfv}} \textcolor{PineGreen}{\selectlanguage{french}\mytextsc{accomp} \string_ \mytextsc{pfv}}  
 ¶ \textcolor{darkblue}{\textbf{\ipa{no˧ si˧-bi˧!}}} \zh{你要选!} \textcolor{Sepia}{\selectlanguage{english}You choose! / Go ahead and choose!} \textcolor{PineGreen}{\selectlanguage{french}Tu choisis! / A toi le choix!}  
 ¶ \textcolor{darkblue}{\textbf{\ipa{njɤ˧-ɳɯ˧ si˧-bi˧!}}} \zh{是我来选!} \textcolor{Sepia}{\selectlanguage{english}I choose! / Let me choose!} \textcolor{PineGreen}{\selectlanguage{french}C'est moi qui choisis!}  
 ¶ \textcolor{darkblue}{\textbf{\ipa{le˧-si˥\textasciitilde{}si˩}}} \zh{\mytextsc{accomp} \string_ \mytextsc{red}} \textcolor{Sepia}{\selectlanguage{english}\mytextsc{accomp} \string_ \mytextsc{red}} \textcolor{PineGreen}{\selectlanguage{french}\mytextsc{accomp} \string_ \mytextsc{red}}  
 ¶ \textcolor{darkblue}{\textbf{\ipa{tso˧\textasciitilde{}tso˧ si˩}}} \zh{选东西} \textcolor{Sepia}{\selectlanguage{english}to choose things} \textcolor{PineGreen}{\selectlanguage{french}choisir des choses}  
 ¶ \textcolor{darkblue}{\textbf{\ipa{tso˧\textasciitilde{}tso˧ si˧\textasciitilde{}si˥}}} \zh{选选东西} \textcolor{Sepia}{\selectlanguage{english}to choose things} \textcolor{PineGreen}{\selectlanguage{french}choisir des choses}  
 ¶ \textcolor{darkblue}{\textbf{\ipa{dʑɤ˩-hĩ˥ | si˧}}} \zh{挑好的} \textcolor{Sepia}{\selectlanguage{english}to choose good ones} \textcolor{PineGreen}{\selectlanguage{french}choisir les plus beaux; en choisir de beaux (par exemple: sur la montagne, lorsqu'on choisit des arbres à abattre pour donner du bois du charpente)}  

\lhead{\firstmark}
\rhead{\botmark}

\subsection{\hspace{-0.5cm} {\Large \textcolor{darkblue}{\textbf{\ipa{si˧bv̩˧}}}}\hspace{0.5cm}[\kern2pt{\textcolor{darkblue}{\textbf{\ipa{si˧bv̩˧}}}}\kern2pt]} \hypertarget{si\string_Mbv\string_=\string_M1}{}
\markboth{\textcolor{darkblue}{\textbf{\ipa{si˧bv̩˧}}}}{}
\textcolor{teal}{\zh{名词}} \hspace{4pt} \zh{声调类:} M.
\zh{鬼。} \textcolor{Sepia}{\selectlanguage{english}Evil spirit.} \textcolor{PineGreen}{\selectlanguage{french}Démon (forme obtenue par élicitation; nettement moins courante que la forme féminine).}  \zh{量词}: \textcolor{darkblue}{\textbf{\ipa{v̩˧}}} 
\lhead{\firstmark}
\rhead{\botmark}

\subsection{\hspace{-0.5cm} {\Large \textcolor{darkblue}{\textbf{\ipa{si˧bv̩˧-mi\#˥}}}}\hspace{0.5cm}[\kern2pt{\textcolor{darkblue}{\textbf{\ipa{xxxx non-correspondance entre le nombre de morphèmes et le nombre de tons de morphèmes}}}}\kern2pt]} \hypertarget{si\string_Mbv\string_=\string_M-mi\#\string_T1}{}
\markboth{\textcolor{darkblue}{\textbf{\ipa{si˧bv̩˧-mi\#˥}}}}{}
\textcolor{teal}{\zh{名词}} \hspace{4pt} \zh{声调类:} \#H.
\zh{妖精。} \textcolor{Sepia}{\selectlanguage{english}Evil spirit (female).} \textcolor{PineGreen}{\selectlanguage{french}Démone.}  \zh{量词}: \textcolor{darkblue}{\textbf{\ipa{v̩˧}}} 
\lhead{\firstmark}
\rhead{\botmark}

\subsection{\hspace{-0.5cm} {\Large \textcolor{darkblue}{\textbf{\ipa{si˧bv̩˧-zo\#˥}}}}\hspace{0.5cm}[\kern2pt{\textcolor{darkblue}{\textbf{\ipa{xxxx non-correspondance entre le nombre de morphèmes et le nombre de tons de morphèmes}}}}\kern2pt]} \hypertarget{si\string_Mbv\string_=\string_M-zo\#\string_T1}{}
\markboth{\textcolor{darkblue}{\textbf{\ipa{si˧bv̩˧-zo\#˥}}}}{}
\textcolor{teal}{\zh{名词}} \hspace{4pt} \zh{声调类:} \#H.
\zh{鬼。} \textcolor{Sepia}{\selectlanguage{english}Evil spirit (masculine).} \textcolor{PineGreen}{\selectlanguage{french}Démon masculin (forme élicitée, sur la base de la forme féminine; est un mot qui existe, mais peu courant).}  \zh{量词}: \textcolor{darkblue}{\textbf{\ipa{v̩˧}}} 
\lhead{\firstmark}
\rhead{\botmark}

\subsection{\hspace{-0.5cm} {\Large \textcolor{darkblue}{\textbf{\ipa{si˧ɕi˧˥}}}}\hspace{0.5cm}[\kern2pt{\textcolor{darkblue}{\textbf{\ipa{si˧ɕi˧}}}}\kern2pt]} \hypertarget{si\string_Ms£i\string_M\string_T1}{}
\markboth{\textcolor{darkblue}{\textbf{\ipa{si˧ɕi˧˥}}}}{}
\textcolor{teal}{\zh{名词}} \hspace{4pt} \zh{声调类:} MH\#.
\zh{森林。} \textcolor{Sepia}{\selectlanguage{english}Forest.} \textcolor{PineGreen}{\selectlanguage{french}Forêt (clairsemée).}  ¶ \textcolor{darkblue}{\textbf{\ipa{[F5] tʰo˧ɕi˧˥}}} \zh{松树森林} \textcolor{Sepia}{\selectlanguage{english}pine forest} \textcolor{PineGreen}{\selectlanguage{french}forêt de pins}  
 \zh{量词}: \textcolor{darkblue}{\textbf{\ipa{pʰæ˧˥}}} 
\lhead{\firstmark}
\rhead{\botmark}

\subsection{\hspace{-0.5cm} {\Large \textcolor{darkblue}{\textbf{\ipa{si˧dzi˩}}}}\hspace{0.5cm}[\kern2pt{\textcolor{darkblue}{\textbf{\ipa{si˧dzi˧˥}}}}\kern2pt]} \hypertarget{si\string_Mdzi\string_B1}{}
\markboth{\textcolor{darkblue}{\textbf{\ipa{si˧dzi˩}}}}{}
\textcolor{teal}{\zh{名词}} \hspace{4pt} \zh{声调类:} L\#.
\zh{树。} \textcolor{Sepia}{\selectlanguage{english}Tree.} \textcolor{PineGreen}{\selectlanguage{french}Arbre.}  \zh{量词}: \textcolor{darkblue}{\textbf{\ipa{dzi˩, ʝi˧}}} 
\lhead{\firstmark}
\rhead{\botmark}

\subsection{\hspace{-0.5cm} {\Large \textcolor{darkblue}{\textbf{\ipa{si˧dzi˩-mv̩˩tsɯ˩}}}}\hspace{0.5cm}[\kern2pt{\textcolor{darkblue}{\textbf{\ipa{xxxx non-correspondance entre le nombre de morphèmes et le nombre de tons de morphèmes}}}}\kern2pt]} \hypertarget{si\string_Mdzi\string_B-mv\string_=\string_BtsM\string_B1}{}
\markboth{\textcolor{darkblue}{\textbf{\ipa{si˧dzi˩-mv̩˩tsɯ˩}}}}{}
\textcolor{teal}{\zh{名词}} \hspace{4pt} \zh{声调类:} \#L-L.
\zh{胚根。} \textcolor{Sepia}{\selectlanguage{english}Radicel, rootlet, small root.} \textcolor{PineGreen}{\selectlanguage{french}Radicelle, petites racine.} 
\lhead{\firstmark}
\rhead{\botmark}

\subsection{\hspace{-0.5cm} {\Large \textcolor{darkblue}{\textbf{\ipa{si˧dʑɯ˥}}}}\hspace{0.5cm}[\kern2pt{\textcolor{darkblue}{\textbf{\ipa{xxxx non-correspondance entre le nombre de morphèmes et le nombre de tons de morphèmes}}}}\kern2pt]} \hypertarget{si\string_Mdz£M\string_T1}{}
\markboth{\textcolor{darkblue}{\textbf{\ipa{si˧dʑɯ˥}}}}{}
\textcolor{teal}{\zh{名词}} \hspace{4pt} \zh{声调类:} H\#.
\zh{火煤、火捻、火种、劈柴、引柴。} \textcolor{Sepia}{\selectlanguage{english}Kindling.} \textcolor{PineGreen}{\selectlanguage{french}Petit bois, pour faire démarrer le feu; à Yongning, ce qu'on utilise: des morceaux de pin gorgés de résine, utilisés spécialement à cet effet.}  \zh{量词}: \textcolor{darkblue}{\textbf{\ipa{kʰwɤ˥}}} 
\lhead{\firstmark}
\rhead{\botmark}

\subsection{\hspace{-0.5cm} {\Large \textcolor{darkblue}{\textbf{\ipa{si˧gɯ˧}}}}\hspace{0.5cm}[\kern2pt{\textcolor{darkblue}{\textbf{\ipa{si˧gɯ˥}}}}\kern2pt]} \hypertarget{si\string_MgM\string_M1}{}
\markboth{\textcolor{darkblue}{\textbf{\ipa{si˧gɯ˧}}}}{}
\textcolor{teal}{\zh{名词}} \hspace{4pt} \zh{声调类:} M.
\zh{狮子。} \textcolor{Sepia}{\selectlanguage{english}Lion.} \textcolor{PineGreen}{\selectlanguage{french}Lion.}  \zh{【借词】}\zh{藏语} seng ge སེང་གེ
 \zh{量词}: \textcolor{darkblue}{\textbf{\ipa{mi˩}}} 
\lhead{\firstmark}
\rhead{\botmark}

\subsection{\hspace{-0.5cm} {\Large \textcolor{darkblue}{\textbf{\ipa{si˧gɯ˧-mi˩}}}}\hspace{0.5cm}[\kern2pt{\textcolor{darkblue}{\textbf{\ipa{xxxx non-correspondance entre le nombre de morphèmes et le nombre de tons de morphèmes}}}}\kern2pt]} \hypertarget{si\string_MgM\string_M-mi\string_B1}{}
\markboth{\textcolor{darkblue}{\textbf{\ipa{si˧gɯ˧-mi˩}}}}{}
\textcolor{teal}{\zh{名词}} \hspace{4pt} \zh{声调类:} \mytextsc{L}.
\zh{母狮。} \textcolor{Sepia}{\selectlanguage{english}Lioness.} \textcolor{PineGreen}{\selectlanguage{french}Lionne.}  \zh{量词}: \textcolor{darkblue}{\textbf{\ipa{mi˩}}} 
\lhead{\firstmark}
\rhead{\botmark}

\subsection{\hspace{-0.5cm} {\Large \textcolor{darkblue}{\textbf{\ipa{si˧gɯ˧-pʰv̩\#˥}}}}\hspace{0.5cm}[\kern2pt{\textcolor{darkblue}{\textbf{\ipa{xxxx non-correspondance entre le nombre de morphèmes et le nombre de tons de morphèmes}}}}\kern2pt]} \hypertarget{si\string_MgM\string_M-p\string_hv\string_=\#\string_T1}{}
\markboth{\textcolor{darkblue}{\textbf{\ipa{si˧gɯ˧-pʰv̩\#˥}}}}{}
\textcolor{teal}{\zh{名词}} \hspace{4pt} \zh{声调类:} \#H.
\zh{公狮子。} \textcolor{Sepia}{\selectlanguage{english}Male lion.} \textcolor{PineGreen}{\selectlanguage{french}Lion (mâle).}  \zh{量词}: \textcolor{darkblue}{\textbf{\ipa{mi˩}}} 
\lhead{\firstmark}
\rhead{\botmark}

\subsection{\hspace{-0.5cm} {\Large \textcolor{darkblue}{\textbf{\ipa{si˧gɯ˧-tsʰo\#˥}}}}\hspace{0.5cm}[\kern2pt{\textcolor{darkblue}{\textbf{\ipa{xxxx non-correspondance entre le nombre de morphèmes et le nombre de tons de morphèmes}}}}\kern2pt]} \hypertarget{si\string_MgM\string_M-ts\string_ho\#\string_T1}{}
\markboth{\textcolor{darkblue}{\textbf{\ipa{si˧gɯ˧-tsʰo\#˥}}}}{}
\textcolor{teal}{\zh{名词}} \hspace{4pt} \zh{声调类:} \#H.
\zh{狮子舞:土司准备的礼仪性表演。土司也亲自参与舞蹈。} \textcolor{Sepia}{\selectlanguage{english}Lion Dance: a show organized for the feudal lord.} \textcolor{PineGreen}{\selectlanguage{french}Danse du Lion: spectacle masqué, commandité par le seigneur féodal, qui participait lui-même à certaines des danses.} 
\lhead{\firstmark}
\rhead{\botmark}

\subsection{\hspace{-0.5cm} {\Large \textcolor{darkblue}{\textbf{\ipa{si˧gɯ˧-zo\#˥}}}}\hspace{0.5cm}[\kern2pt{\textcolor{darkblue}{\textbf{\ipa{xxxx non-correspondance entre le nombre de morphèmes et le nombre de tons de morphèmes}}}}\kern2pt]} \hypertarget{si\string_MgM\string_M-zo\#\string_T1}{}
\markboth{\textcolor{darkblue}{\textbf{\ipa{si˧gɯ˧-zo\#˥}}}}{}
\textcolor{teal}{\zh{名词}} \hspace{4pt} \zh{声调类:} \#H.
\zh{小狮子。} \textcolor{Sepia}{\selectlanguage{english}Lion cub.} \textcolor{PineGreen}{\selectlanguage{french}Lionceau, petit lion.}  \zh{量词}: \textcolor{darkblue}{\textbf{\ipa{ɭɯ˧}}} \textcolor{darkblue}{\textbf{\ipa{mi˩}}} 
\lhead{\firstmark}
\rhead{\botmark}

\subsection{\hspace{-0.5cm} {\Large \textcolor{darkblue}{\textbf{\ipa{si˧kɤ˧˥}}}}\hspace{0.5cm}[\kern2pt{\textcolor{darkblue}{\textbf{\ipa{si˧kɤ˧˥}}}}\kern2pt]} \hypertarget{si\string_Mk7\string_M\string_T1}{}
\markboth{\textcolor{darkblue}{\textbf{\ipa{si˧kɤ˧˥}}}}{}
\textcolor{teal}{\zh{名词}} \hspace{4pt} \zh{声调类:} MH\#.
\zh{树枝、小树枝,棍子。} \textcolor{Sepia}{\selectlanguage{english}Branch; rod, stick.} \textcolor{PineGreen}{\selectlanguage{french}Branche; petite branche; bâton, gourdin, canne pour marcher.}  \zh{量词}: \textcolor{darkblue}{\textbf{\ipa{kɤ˧˥}}} 
\lhead{\firstmark}
\rhead{\botmark}

\subsection{\hspace{-0.5cm} {\Large \textcolor{darkblue}{\textbf{\ipa{si˧kwɤ˩}}}}\hspace{0.5cm}[\kern2pt{\textcolor{darkblue}{\textbf{\ipa{xxxx ton non trouvé, à faire manuellement...}}}}\kern2pt]} \hypertarget{si\string_Mkw7\string_B1}{}
\markboth{\textcolor{darkblue}{\textbf{\ipa{si˧kwɤ˩}}}}{}
\textcolor{teal}{\zh{名词}} \hspace{4pt} \zh{声调类:} \#L.
\zh{木头框架,如:房子的木头框架。} \textcolor{Sepia}{\selectlanguage{english}Wooden structure (of a house), carpentry.} \textcolor{PineGreen}{\selectlanguage{french}Structure, charpente, gros œuvre en bois (d'une maison).}  ¶ \textcolor{darkblue}{\textbf{\ipa{ʑi˧mi˧-si˧kwɤ˩}}} \zh{房子的木头框架} \textcolor{Sepia}{\selectlanguage{english}a house's carpentry, a house's wooden structure} \textcolor{PineGreen}{\selectlanguage{french}la charpente d'une maison}  
 \zh{量词}: \textcolor{darkblue}{\textbf{\ipa{kwɤ˩}}} 
\lhead{\firstmark}
\rhead{\botmark}

\subsection{\hspace{-0.5cm} {\Large \textcolor{darkblue}{\textbf{\ipa{si˧kʰɯ\#˥}}}}\hspace{0.5cm}[\kern2pt{\textcolor{darkblue}{\textbf{\ipa{si˧kʰɯ˥}}}}\kern2pt]} \hypertarget{si\string_Mk\string_hM\#\string_T1}{}
\markboth{\textcolor{darkblue}{\textbf{\ipa{si˧kʰɯ\#˥}}}}{}
\textcolor{teal}{\zh{名词}} \hspace{4pt} \zh{声调类:} H\#.
\zh{色疙瘩。} \textcolor{PineGreen}{\selectlanguage{french}Yyyy.} \zh{当地汉语方言:}\zh{根三香。} ¶ \textcolor{darkblue}{\textbf{\ipa{si˧kʰɯ˧-bæ˥bæ˩}}} \zh{色疙瘩花} \textcolor{Sepia}{\selectlanguage{english}flower of...} \textcolor{PineGreen}{\selectlanguage{french}fleur de...}  
\zh{~【参考】~} \textcolor{darkblue}{\textbf{\ipa{si˧kʰɯ˧ɭɯ˧bv̩˥}}} 
\lhead{\firstmark}
\rhead{\botmark}

\subsection{\hspace{-0.5cm} {\Large \textcolor{darkblue}{\textbf{\ipa{si˧kʰɯ˧-ɭɯ˧bv̩˥}}}}\hspace{0.5cm}[\kern2pt{\textcolor{darkblue}{\textbf{\ipa{xxxx non-correspondance entre le nombre de morphèmes et le nombre de tons de morphèmes}}}}\kern2pt]} \hypertarget{si\string_Mk\string_hM\string_M-l\string_RM\string_Mbv\string_=\string_T1}{}
\markboth{\textcolor{darkblue}{\textbf{\ipa{si˧kʰɯ˧-ɭɯ˧bv̩˥}}}}{}
\textcolor{teal}{\zh{名词}} \hspace{4pt} \zh{声调类:} H\#.
\zh{白芍药。} \textcolor{Sepia}{\selectlanguage{english}White Chinese herbaceous peony, \textit{Paeonia lactiflora}.} \textcolor{PineGreen}{\selectlanguage{french}Pivoine blanche de Chine, \textit{Paeonia lactiflora}.}  \zh{量词}: \textcolor{darkblue}{\textbf{\ipa{kɤ˧˥}}} \zh{~【参考】~} \hyperlink{}{\textcolor{darkblue}{\textbf{\ipa{si˧kʰɯ\#˥}}}} 
\lhead{\firstmark}
\rhead{\botmark}

\subsection{\hspace{-0.5cm} {\Large \textcolor{darkblue}{\textbf{\ipa{si˧nɑ˥}}}}\hspace{0.5cm}[\kern2pt{\textcolor{darkblue}{\textbf{\ipa{si˧nɑ˥}}}}\kern2pt]} \hypertarget{si\string_MnA\string_T1}{}
\markboth{\textcolor{darkblue}{\textbf{\ipa{si˧nɑ˥}}}}{}
\textcolor{teal}{\zh{名词}} \hspace{4pt} \zh{声调类:} H\#.
\zh{森林深处(难走路)。} \textcolor{Sepia}{\selectlanguage{english}Deep forest.} \textcolor{PineGreen}{\selectlanguage{french}Forêt épaisse.}  \zh{量词}: \textcolor{darkblue}{\textbf{\ipa{pʰæ˧˥}}} 
\lhead{\firstmark}
\rhead{\botmark}

\subsection{\hspace{-0.5cm} {\Large \textcolor{darkblue}{\textbf{\ipa{si˧-ʁæ˧bæ˥}}}}\hspace{0.5cm}[\kern2pt{\textcolor{darkblue}{\textbf{\ipa{xxxx non-correspondance entre le nombre de morphèmes et le nombre de tons de morphèmes}}}}\kern2pt]} \hypertarget{si\string_M-R\{\string_Mb\{\string_T1}{}
\markboth{\textcolor{darkblue}{\textbf{\ipa{si˧-ʁæ˧bæ˥}}}}{}
\textcolor{teal}{\zh{名词}} \hspace{4pt} \zh{声调类:} H\#.
\zh{木盘子。} \textcolor{Sepia}{\selectlanguage{english}Wooden plate.} \textcolor{PineGreen}{\selectlanguage{french}Assiette en bois.}  \zh{量词}: \textcolor{darkblue}{\textbf{\ipa{ɭɯ˧}}} 
\lhead{\firstmark}
\rhead{\botmark}

\subsection{\hspace{-0.5cm} {\Large \textcolor{darkblue}{\textbf{\ipa{si˧ʁo\#˥}}}}\hspace{0.5cm}[\kern2pt{\textcolor{darkblue}{\textbf{\ipa{si˧ʁo˧}}}}\kern2pt]} \hypertarget{si\string_MRo\#\string_T1}{}
\markboth{\textcolor{darkblue}{\textbf{\ipa{si˧ʁo\#˥}}}}{}
\textcolor{teal}{\zh{名词}} \hspace{4pt} \zh{声调类:} \#H.
\zh{果树。} \textcolor{Sepia}{\selectlanguage{english}Fruit tree.} \textcolor{PineGreen}{\selectlanguage{french}Arbre fruitier.}  \zh{量词}: \textcolor{darkblue}{\textbf{\ipa{dzi˩}}} 
\lhead{\firstmark}
\rhead{\botmark}

\subsection{\hspace{-0.5cm} {\Large \textcolor{darkblue}{\textbf{\ipa{si˧ʁo˧si˧ɭɯ\#˥}}}}\hspace{0.5cm}[\kern2pt{\textcolor{darkblue}{\textbf{\ipa{si˧ʁo˧si˧ɭɯ˧}}}}\kern2pt]} \hypertarget{si\string_MRo\string_Msi\string_Ml\string_RM\#\string_T1}{}
\markboth{\textcolor{darkblue}{\textbf{\ipa{si˧ʁo˧si˧ɭɯ\#˥}}}}{}
\textcolor{teal}{\zh{名词}} \hspace{4pt} \zh{声调类:} \#H.
\zh{水果。} \textcolor{Sepia}{\selectlanguage{english}Fruit.} \textcolor{PineGreen}{\selectlanguage{french}Fruit.}  ¶ \textcolor{darkblue}{\textbf{\ipa{si˧ʁo˧si˧ɭɯ˧ ɲi˥}}} \zh{是水果。} \textcolor{Sepia}{\selectlanguage{english}\mytextsc{cop}} \textcolor{PineGreen}{\selectlanguage{french}\mytextsc{cop}}  
 \zh{量词}: \textcolor{darkblue}{\textbf{\ipa{ɭɯ˧}}} 
\lhead{\firstmark}
\rhead{\botmark}

\subsection{\hspace{-0.5cm} {\Large \textcolor{darkblue}{\textbf{\ipa{si˧-sæ˥qʰv̩˩}}}}\hspace{0.5cm}[\kern2pt{\textcolor{darkblue}{\textbf{\ipa{xxxx non-correspondance entre le nombre de morphèmes et le nombre de tons de morphèmes}}}}\kern2pt]} \hypertarget{si\string_M-s\{\string_Tq\string_hv\string_=\string_B1}{}
\markboth{\textcolor{darkblue}{\textbf{\ipa{si˧-sæ˥qʰv̩˩}}}}{}
\textcolor{teal}{\zh{名词}} \hspace{4pt} \zh{声调类:} \#H.
\zh{四川桦树,白桦树。} \textcolor{Sepia}{\selectlanguage{english}Birch, \textit{Betula szechuanica (Betula Pendula var. szechuanica)}.} \textcolor{PineGreen}{\selectlanguage{french}Bouleau, \textit{Betula szechuanica (Betula Pendula var. szechuanica)}.} 
\lhead{\firstmark}
\rhead{\botmark}

\subsection{\hspace{-0.5cm} {\Large \textcolor{darkblue}{\textbf{\ipa{si˧tʰv̩\#˥}}}}\hspace{0.5cm}[\kern2pt{\textcolor{darkblue}{\textbf{\ipa{si˧tʰv̩˧}}}}\kern2pt]} \hypertarget{si\string_Mt\string_hv\string_=\#\string_T1}{}
\markboth{\textcolor{darkblue}{\textbf{\ipa{si˧tʰv̩\#˥}}}}{}
\textcolor{teal}{\zh{名词}} \hspace{4pt} \zh{声调类:} \#H.
\zh{供桌:主屋里面的一个家具,是祖先的象征性住所。} \textcolor{Sepia}{\selectlanguage{english}A piece of furniture of the main room, which constitutes the symbolic dwelling of ancestors, and serves as an altar; on the New Year, some candles are lighted on it.} \textcolor{PineGreen}{\selectlanguage{french}Meuble-autel des ancêtres, dans la pièce principale, qui constitue le lieu symbolique où résident les ancêtres; on y met des bougies au Nouvel An.}  ¶ \textcolor{darkblue}{\textbf{\ipa{ʑi˧dv̩˧-nv̩˩mi˩, | si˧tʰv̩˧!}}} \zh{屋子的中心,就是祖先的供桌!} \textcolor{Sepia}{\selectlanguage{english}The heart of the house is the altar to the ancestors!} \textcolor{PineGreen}{\selectlanguage{french}le cœur de la maison, c'est le meuble-autel des ancêtres!}  

\lhead{\firstmark}
\rhead{\botmark}

\subsection{\hspace{-0.5cm} {\Large \textcolor{darkblue}{\textbf{\ipa{si˩qʰɑ˩}}}}\hspace{0.5cm}[\kern2pt{\textcolor{darkblue}{\textbf{\ipa{si˩qʰɑ˩˥}}}}\kern2pt]} \hypertarget{si\string_Bq\string_hA\string_B1}{}
\markboth{\textcolor{darkblue}{\textbf{\ipa{si˩qʰɑ˩}}}}{}
\textcolor{teal}{\zh{名词}} \hspace{4pt} \zh{声调类:} L.
\zh{梅子。} \textcolor{Sepia}{\selectlanguage{english}Plum tree, prune tree.} \textcolor{PineGreen}{\selectlanguage{french}Abricotier du Japon (essence bien représentée à Yongning).}  ¶ \textcolor{darkblue}{\textbf{\ipa{si˩qʰɑ˩-dʑɯ˩}}} \zh{用梅子做的一种汁,用法类似于醋。过去,永宁没有醋,醋是从内地买来的。} \textcolor{Sepia}{\selectlanguage{english}a liquid prepared from plums, which served as an equivalent of vinegar (vinegar was introduced late: it was bought in Chinese areas)} \textcolor{PineGreen}{\selectlanguage{french}un liquide préparé à base de prunelles d'abricotier du Japon, servant d'équivalent de vinaigre (le vinaigre a été introduit tardivement; il était acheté en pays chinois)}  

\lhead{\firstmark}
\rhead{\botmark}

\subsection{\hspace{-0.5cm} {\Large \textcolor{darkblue}{\textbf{\ipa{si˩tsʰɤ˩}}}}\hspace{0.5cm}[\kern2pt{\textcolor{darkblue}{\textbf{\ipa{si˩tsʰɤ˩˥}}}}\kern2pt]} \hypertarget{si\string_Bts\string_h7\string_B1}{}
\markboth{\textcolor{darkblue}{\textbf{\ipa{si˩tsʰɤ˩}}}}{}
\textcolor{teal}{\zh{名词}} \hspace{4pt} \zh{声调类:} L.
\ding{202} \zh{叶子。} \textcolor{Sepia}{\selectlanguage{english}Leaf.} \textcolor{PineGreen}{\selectlanguage{french}Feuille.}  ¶ \textcolor{darkblue}{\textbf{\ipa{si˧dzi˩-si˩tsʰɤ˩}}} \zh{树叶} \textcolor{Sepia}{\selectlanguage{english}tree leaf} \textcolor{PineGreen}{\selectlanguage{french}feuilles d'arbre}  
 \zh{量词}: \textcolor{darkblue}{\textbf{\ipa{tsʰɤ˧˥}}} \ding{203} \zh{鸡冠。} \textcolor{Sepia}{\selectlanguage{english}Cock's comb.} \textcolor{PineGreen}{\selectlanguage{french}Crête (du coq, d'un oiseau).}  ¶ \textcolor{darkblue}{\textbf{\ipa{æ̃˧ʂwæ˥-si˩tsʰɤ˩}}} \zh{公鸡冠} \textcolor{Sepia}{\selectlanguage{english}comb of (a) cock} \textcolor{PineGreen}{\selectlanguage{french}crête de coq}  

\lhead{\firstmark}
\rhead{\botmark}

\subsection{\hspace{-0.5cm} {\Large \textcolor{darkblue}{\textbf{\ipa{si˧˥}}} \textsubscript{1}}\hspace{0.5cm}[\kern2pt{\textcolor{darkblue}{\textbf{\ipa{si˩˥}}}}\kern2pt]} \hypertarget{si\string_M\string_T1}{}
\markboth{\textcolor{darkblue}{\textbf{\ipa{si˧˥}}} \textsubscript{1}}{}
\textcolor{teal}{\zh{动词}} \hspace{4pt} \zh{声调类:} MH.
\zh{剔,刮。} \textcolor{Sepia}{\selectlanguage{english}To shave (the beard or the head); to scrub (e.g. to scrub earth off vegetables).} \textcolor{PineGreen}{\selectlanguage{french}Raser (la barbe); gratter (la terre collée à un champignon).}  ¶ \textcolor{darkblue}{\textbf{\ipa{mo˧ si˥}}} \zh{刮菌子(刮掉污垢)} \textcolor{Sepia}{\selectlanguage{english}to scrub mushrooms (to take off the earth, moss...)} \textcolor{PineGreen}{\selectlanguage{french}gratter des champignons, pour en retirer la terre, les aiguilles de pin... Cela se fait souvent à sec.}  
 ¶ \textcolor{darkblue}{\textbf{\ipa{mv̩˧tsɯ˧ si˥}}} \zh{刮胡子} \textcolor{Sepia}{\selectlanguage{english}to shave (one's) beard} \textcolor{PineGreen}{\selectlanguage{french}raser la barbe}  
 ¶ \textcolor{darkblue}{\textbf{\ipa{ʁo˧qʰwɤ˩ si˩}}} \zh{剃头} \textcolor{Sepia}{\selectlanguage{english}to shave one's head} \textcolor{PineGreen}{\selectlanguage{french}raser le crâne, raser la tête}  
 ¶ \textcolor{darkblue}{\textbf{\ipa{ʁo˧qʰwɤ˩ si˩-di˩}}} \zh{理发刮刀} \textcolor{Sepia}{\selectlanguage{english}Razor: object used to shave the head or the beard. (In the main consultant's youth, not every family had a razor. One would call someone to the house to shave the head or the beard. It was mostly monks and elderly people who had their heads and beards shaved.)} \textcolor{PineGreen}{\selectlanguage{french}Rasoir, objet utilisé pour raser le crâne, mais aussi pour raser la barbe. Dans la jeunesse de F4, il existait quelques rasoirs; chaque famille n'en possédait pas. On faisait venir une personne sachant manier l'instrument. Ce sont les moines et les vieilles personnes qui faisaient le plus fréquemment appel à ces services.}  

\lhead{\firstmark}
\rhead{\botmark}

\subsection{\hspace{-0.5cm} {\Large \textcolor{darkblue}{\textbf{\ipa{si˧˥}}} \textsubscript{2}}\hspace{0.5cm}[\kern2pt{\textcolor{darkblue}{\textbf{\ipa{si˧˥}}}}\kern2pt]} \hypertarget{si\string_M\string_T2}{}
\markboth{\textcolor{darkblue}{\textbf{\ipa{si˧˥}}} \textsubscript{2}}{}
\textcolor{teal}{\zh{动词}} \hspace{4pt} \zh{声调类:} MH.
\zh{杀(人)。} \textcolor{Sepia}{\selectlanguage{english}To murder, to kill (a human being).} \textcolor{PineGreen}{\selectlanguage{french}Assassiner, tuer (un homme).}  ¶ \textcolor{darkblue}{\textbf{\ipa{hĩ˧ si˩}}} \zh{杀人} \textcolor{Sepia}{\selectlanguage{english}to kill someone} \textcolor{PineGreen}{\selectlanguage{french}tuer quelqu'un, assassiner quelqu'un}  

\lhead{\firstmark}
\rhead{\botmark}

\subsection{\hspace{-0.5cm} {\Large \textcolor{darkblue}{\textbf{\ipa{si˩˥}}}}\hspace{0.5cm}[\kern2pt{\textcolor{darkblue}{\textbf{\ipa{si˩˥}}}}\kern2pt]} \hypertarget{si\string_B\string_T1}{}
\markboth{\textcolor{darkblue}{\textbf{\ipa{si˩˥}}}}{}
\textcolor{teal}{\zh{名词}} \hspace{4pt} \zh{声调类:} LH.
\zh{肝。} \textcolor{Sepia}{\selectlanguage{english}Liver.} \textcolor{PineGreen}{\selectlanguage{french}Foie.}  \zh{量词}: \textcolor{darkblue}{\textbf{\ipa{ɭɯ˧}}} 
\lhead{\firstmark}
\rhead{\botmark}

\subsection{\hspace{-0.5cm} {\Large \textcolor{darkblue}{\textbf{\ipa{so˥}}} \textsubscript{1}}\hspace{0.5cm}[\kern2pt{\textcolor{darkblue}{\textbf{\ipa{so˥}}}}\kern2pt]} \hypertarget{so\string_T1}{}
\markboth{\textcolor{darkblue}{\textbf{\ipa{so˥}}} \textsubscript{1}}{}
\textcolor{teal}{\zh{名词}} \hspace{4pt} \zh{声调类:} \#H.
\zh{早上献给神的食物(含茶、酥油、面、蜂蜜),扔进松针火里烧。} \textcolor{Sepia}{\selectlanguage{english}Offering to the gods, given to them in the morning; it comprises tea, butter, flour, and honey; it is burnt over a fire of pine needles.} \textcolor{PineGreen}{\selectlanguage{french}Offrande aux esprits: repas qu'on leur offre le matin; on y met du thé, du beurre, de la farine, et du miel (et éventuellement des fleurs: \textcolor{darkblue}{\textbf{\ipa{/so˧dze˧-bæ˩bæ˩/}}}); on le fait brûler sur un feu d'épines de pin.}  ¶ \textcolor{darkblue}{\textbf{\ipa{so˧ dze˧ tʰi˧-qæ˩}}} \zh{烧蜂蜜献给神} \textcolor{Sepia}{\selectlanguage{english}to burn honey as an offering} \textcolor{PineGreen}{\selectlanguage{french}faire brûler du miel en offrande}  
 ¶ \textcolor{darkblue}{\textbf{\ipa{[M23] so˧ qæ˩}}} \zh{烧献给神(食物,……)} \textcolor{Sepia}{\selectlanguage{english}to burn an offering} \textcolor{PineGreen}{\selectlanguage{french}brûler une offrande; traditionnellement, du pin gorgé de résine.}  

\lhead{\firstmark}
\rhead{\botmark}

\subsection{\hspace{-0.5cm} {\Large \textcolor{darkblue}{\textbf{\ipa{so˥}}} \textsubscript{2}}\hspace{0.5cm}[\kern2pt{\textcolor{darkblue}{\textbf{\ipa{so˥}}}}\kern2pt]} \hypertarget{so\string_T2}{}
\markboth{\textcolor{darkblue}{\textbf{\ipa{so˥}}} \textsubscript{2}}{}
\textcolor{teal}{\zh{量词}} \hspace{4pt} \zh{声调类:} H*.
\zh{量词:样东西,如:‘一样东西都没有’。} \textcolor{Sepia}{\selectlanguage{english}A thing (no plural; only used in the negative construction “there is not a thing”).} \textcolor{PineGreen}{\selectlanguage{french}Classificateur des choses/objets, utilisé seulement en tournure négative: 'quoi que ce soit'.}  ¶ \textcolor{darkblue}{\textbf{\ipa{ɖɯ˧-so˥ | mɤ˧-dʑo˧!}}} \zh{一样也没有! / 没什么东西!(请客时的礼貌、自我贬低说法:请客人原谅菜不够丰盛)} \textcolor{Sepia}{\selectlanguage{english}There is simply nothing at all! (A polite statement made by the host when welcoming a guest for a meal, apologizing, in self-deprecation, for not offering a meal commensurate to one's wishes.)} \textcolor{PineGreen}{\selectlanguage{french}Il n’y a rien du tout [à manger]! (phrase polie qd on invite quelqu'un à manger: on prie le convive d’excuser la pauvreté des mets proposés)}  
\zh{~【参考】~} \textcolor{darkblue}{\textbf{\ipa{sɑ˥}}} 
\lhead{\firstmark}
\rhead{\botmark}

\subsection{\hspace{-0.5cm} {\Large \textcolor{darkblue}{\textbf{\ipa{so˧\textsubscript{a}}}}}\hspace{0.5cm}[\kern2pt{\textcolor{darkblue}{\textbf{\ipa{so˩˥}}}}\kern2pt]} \hypertarget{so\string_Ma1}{}
\markboth{\textcolor{darkblue}{\textbf{\ipa{so˧\textsubscript{a}}}}}{}
\textcolor{teal}{\zh{量词}} \hspace{4pt} \zh{声调类:} M\textsubscript{a}.
\zh{量词:早晨(一个)。} \textcolor{Sepia}{\selectlanguage{english}Classifier for mornings.} \textcolor{PineGreen}{\selectlanguage{french}Classificateur des matinées. Il existe trois expressions pour compter les journées: on peut dire: un jour; une matinée; ou une nuit.}  ¶ \textcolor{darkblue}{\textbf{\ipa{mv̩˩si˧-njɤ˧˥ | ɖɯ˧-so˧, | njɤ˧le˧gv̩˧ | ɖɯ˧-ɲi˥, | mv̩˧kʰv̩˥ | ɖɯ˧-hɑ̃˧˥!}}} \zh{一个早晨,一个白天,(或者说)一个晚上!(这句话,总结数日子的三个方式:‘一天’,可以说成‘一个早晨’、‘一个白天’、或‘一个晚上’。)} \textcolor{Sepia}{\selectlanguage{english}One morning; one day; [or] one night! (A sentence that summarizes the three ways to count days: a day can be referred to as “one morning”, “one day”, or “one night”.)} \textcolor{PineGreen}{\selectlanguage{french}Une matinée; une journée; [ou] une nuit! (Expression didactique résumant les trois façons de compter les jours: on peut compter les matinées, les journées, ou les soirées.)}  
 ¶ \textcolor{darkblue}{\textbf{\ipa{tʰv̩˧-so˩}}} \zh{那天早上} \textcolor{Sepia}{\selectlanguage{english}that morning} \textcolor{PineGreen}{\selectlanguage{french}ce matin-là}  

\lhead{\firstmark}
\rhead{\botmark}

\subsection{\hspace{-0.5cm} {\Large \textcolor{darkblue}{\textbf{\ipa{so˧dʑɯ\#˥}}}}\hspace{0.5cm}[\kern2pt{\textcolor{darkblue}{\textbf{\ipa{so˧dʑɯ˧˥}}}}\kern2pt]} \hypertarget{so\string_Mdz£M\#\string_T1}{}
\markboth{\textcolor{darkblue}{\textbf{\ipa{so˧dʑɯ\#˥}}}}{}
\textcolor{teal}{\zh{名词}} \hspace{4pt} \zh{声调类:} \#H.
\zh{陷阱。} \textcolor{Sepia}{\selectlanguage{english}Pitfall, pit, trap.} \textcolor{PineGreen}{\selectlanguage{french}Piège.}  ¶ \textcolor{darkblue}{\textbf{\ipa{so˧dʑɯ˧ | ɖɯ˧-ɭɯ˧ | qwæ˧˥}}} \zh{挖一个陷阱} \textcolor{Sepia}{\selectlanguage{english}to dig a pitfall, to dig a trap} \textcolor{PineGreen}{\selectlanguage{french}creuser une fosse, un piège}  

\lhead{\firstmark}
\rhead{\botmark}

\subsection{\hspace{-0.5cm} {\Large \textcolor{darkblue}{\textbf{\ipa{so˧hɑ̃˩}}}}\hspace{0.5cm}[\kern2pt{\textcolor{darkblue}{\textbf{\ipa{so˧hɑ̃˧}}}}\kern2pt]} \hypertarget{so\string_MhA\string_~\string_B1}{}
\markboth{\textcolor{darkblue}{\textbf{\ipa{so˧hɑ̃˩}}}}{}
\textcolor{teal}{\zh{助词}} \hspace{4pt} \zh{声调类:} L\#.
\zh{明晚。} \textcolor{Sepia}{\selectlanguage{english}Tomorrow evening.} \textcolor{PineGreen}{\selectlanguage{french}Demain soir.}  ¶ \textcolor{darkblue}{\textbf{\ipa{so˧hɑ̃˩ | -ɖɯ˩hɑ̃˩˥}}} \zh{明天晚上} \textcolor{Sepia}{\selectlanguage{english}tomorrow evening} \textcolor{PineGreen}{\selectlanguage{french}demain soir}  

\lhead{\firstmark}
\rhead{\botmark}

\subsection{\hspace{-0.5cm} {\Large \textcolor{darkblue}{\textbf{\ipa{so˧hwɤ˩}}}}\hspace{0.5cm}[\kern2pt{\textcolor{darkblue}{\textbf{\ipa{so˧hwɤ˩}}}}\kern2pt]} \hypertarget{so\string_Mhw7\string_B1}{}
\markboth{\textcolor{darkblue}{\textbf{\ipa{so˧hwɤ˩}}}}{}
\textcolor{teal}{\zh{助词}} \hspace{4pt} \zh{声调类:} L\#.
\zh{后来、以后,从此以后。} \textcolor{Sepia}{\selectlanguage{english}Afterwards; later; from now on.} \textcolor{PineGreen}{\selectlanguage{french}Ensuite; par la suite; à partir de maintenant, désormais.} 
\lhead{\firstmark}
\rhead{\botmark}

\subsection{\hspace{-0.5cm} {\Large \textcolor{darkblue}{\textbf{\ipa{so˧ʝi˥\$}}}}\hspace{0.5cm}[\kern2pt{\textcolor{darkblue}{\textbf{\ipa{so˧ʝi˩}}}}\kern2pt]} \hypertarget{so\string_Mj££i\string_T\$1}{}
\markboth{\textcolor{darkblue}{\textbf{\ipa{so˧ʝi˥\$}}}}{}
\textcolor{teal}{\zh{助词}} \hspace{4pt} \zh{声调类:} H\$.
\zh{明年。} \textcolor{Sepia}{\selectlanguage{english}Next year.} \textcolor{PineGreen}{\selectlanguage{french}L'année prochaine, l'an prochain.} 
\lhead{\firstmark}
\rhead{\botmark}

\subsection{\hspace{-0.5cm} {\Large \textcolor{darkblue}{\textbf{\ipa{so˧lo˧}}}}\hspace{0.5cm}[\kern2pt{\textcolor{darkblue}{\textbf{\ipa{so˧lo˧}}}}\kern2pt]} \hypertarget{so\string_Mlo\string_M1}{}
\markboth{\textcolor{darkblue}{\textbf{\ipa{so˧lo˧}}}}{}
\textcolor{teal}{\zh{名词}} \hspace{4pt} \zh{声调类:} M.
\zh{影响,榜样。} \textcolor{Sepia}{\selectlanguage{english}Influence, example (in education).} \textcolor{PineGreen}{\selectlanguage{french}Influence; exemple (dans l'éducation de quelqu'un).}  ¶ \textcolor{darkblue}{\textbf{\ipa{so˧lo˧ dzɑ˧! | mɤ˧-dʑɤ˩-hĩ˩ | ɖɯ˧-ʑi˩ ɲi˩!}}} \zh{他(对周围的人)有一个不好的影响!(他的家庭)是个不好的家庭!} \textcolor{Sepia}{\selectlanguage{english}He/she has a bad influence / he/she gives a bad example! (His/her family) is a bad family!} \textcolor{PineGreen}{\selectlanguage{french}Il/elle exerce une mauvaise influence / il/elle donne un mauvais exemple! (Sa famille,) c'est une mauvaise famille!}  
 ¶ \textcolor{darkblue}{\textbf{\ipa{so˧lo˧ mɤ˧-dʑɤ˩!}}} \zh{同上:(他对别人的)影响不好。} \textcolor{Sepia}{\selectlanguage{english}Same meaning as previous example: His/her example/influence is not good.} \textcolor{PineGreen}{\selectlanguage{french}Même sens que l'exemple précédent: Son exemple n'est pas bon / son influence n'est pas bonne.}  
 ¶ \textcolor{darkblue}{\textbf{\ipa{so˧lo˧ dʑɤ˩}}} \zh{好榜样、好例子、好教育} \textcolor{Sepia}{\selectlanguage{english}good influence; good example; good education} \textcolor{PineGreen}{\selectlanguage{french}bonne influence; bon exemple; bonne éducation}  
 \zh{量词}: \textcolor{darkblue}{\textbf{\ipa{kʰwɤ˥}}} 
\lhead{\firstmark}
\rhead{\botmark}

\subsection{\hspace{-0.5cm} {\Large \textcolor{darkblue}{\textbf{\ipa{so˧ɬi˧mi˧}}}}\hspace{0.5cm}[\kern2pt{\textcolor{darkblue}{\textbf{\ipa{so˧ɬi˧mi˥}}}}\kern2pt]} \hypertarget{so\string_MKi\string_Mmi\string_M1}{}
\markboth{\textcolor{darkblue}{\textbf{\ipa{so˧ɬi˧mi˧}}}}{}
\textcolor{teal}{\zh{名词}} \hspace{4pt} \zh{声调类:} M.
\zh{三月。} \textcolor{Sepia}{\selectlanguage{english}Third month.} \textcolor{PineGreen}{\selectlanguage{french}3e mois.} 
\lhead{\firstmark}
\rhead{\botmark}

\subsection{\hspace{-0.5cm} {\Large \textcolor{darkblue}{\textbf{\ipa{so˧ɲi˥}}}}\hspace{0.5cm}[\kern2pt{\textcolor{darkblue}{\textbf{\ipa{so˧ɲi˩}}}}\kern2pt]} \hypertarget{so\string_MJi\string_T1}{}
\markboth{\textcolor{darkblue}{\textbf{\ipa{so˧ɲi˥}}}}{}
\textcolor{teal}{\zh{助词}} \hspace{4pt} \zh{声调类:} .
\zh{明天、第二天。} \textcolor{Sepia}{\selectlanguage{english}Tomorrow.} \textcolor{PineGreen}{\selectlanguage{french}Demain, le lendemain.} 
\lhead{\firstmark}
\rhead{\botmark}

\subsection{\hspace{-0.5cm} {\Large \textcolor{darkblue}{\textbf{\ipa{so˩}}}}\hspace{0.5cm}[\kern2pt{\textcolor{darkblue}{\textbf{\ipa{so˩˥}}}}\kern2pt]} \hypertarget{so\string_B1}{}
\markboth{\textcolor{darkblue}{\textbf{\ipa{so˩}}}}{}
\textcolor{teal}{\zh{数词}} \hspace{4pt} \zh{声调类:} L.
\zh{3。} \textcolor{Sepia}{\selectlanguage{english}3.} \textcolor{PineGreen}{\selectlanguage{french}3.} 
\lhead{\firstmark}
\rhead{\botmark}

\subsection{\hspace{-0.5cm} {\Large \textcolor{darkblue}{\textbf{\ipa{so˩\textsubscript{a}}}} \textsubscript{1}}\hspace{0.5cm}[\kern2pt{\textcolor{darkblue}{\textbf{\ipa{so˩˥}}}}\kern2pt]} \hypertarget{so\string_Ba1}{}
\markboth{\textcolor{darkblue}{\textbf{\ipa{so˩\textsubscript{a}}}} \textsubscript{1}}{}
\textcolor{teal}{\zh{形容词}} \hspace{4pt} \zh{声调类:} L\textsubscript{a}.
\zh{香(吃得香,气味香)。} \textcolor{Sepia}{\selectlanguage{english}Good, pleasant to the taste or smell.} \textcolor{PineGreen}{\selectlanguage{french}Agréable, bon (goût, odeur).} 
\lhead{\firstmark}
\rhead{\botmark}

\subsection{\hspace{-0.5cm} {\Large \textcolor{darkblue}{\textbf{\ipa{so˩\textsubscript{a}}}} \textsubscript{2}}\hspace{0.5cm}[\kern2pt{\textcolor{darkblue}{\textbf{\ipa{so˩˥}}}}\kern2pt]} \hypertarget{so\string_Ba2}{}
\markboth{\textcolor{darkblue}{\textbf{\ipa{so˩\textsubscript{a}}}} \textsubscript{2}}{}
\textcolor{teal}{\zh{动词}} \hspace{4pt} \zh{声调类:} L\textsubscript{a}.
\ding{202} \zh{学习。} \textcolor{Sepia}{\selectlanguage{english}To study.} \textcolor{PineGreen}{\selectlanguage{french}Étudier.}  ¶ \textcolor{darkblue}{\textbf{\ipa{tʰæ˧ɻæ˩ so˩}}} \zh{读书、学习} \textcolor{Sepia}{\selectlanguage{english}to study (books)} \textcolor{PineGreen}{\selectlanguage{french}étudier (des livres)}  
 ¶ \textcolor{darkblue}{\textbf{\ipa{so˩ mɤ˩-se˥!}}} \zh{学不完!(关于语言学家的工作:做不完,不像做手工可以有一个明确的终点。)} \textcolor{Sepia}{\selectlanguage{english}There's no end of it! / One is never done with studying! (A comment about the linguist's endeavour to study a language: unlike manual work, it is never really finished.)} \textcolor{PineGreen}{\selectlanguage{french}c'est sans fin! / tu n'as jamais fini d'étudier! (au sujet du travail du linguiste, étudier une langue: à la différence des travaux manuels, ce n'est jamais fini, on n'en voit jamais le bout)}  
 ¶ \textcolor{darkblue}{\textbf{\ipa{ɖɯ˧-so˧\textasciitilde{}so˥-ɻ̍˩}}} \zh{学一学} \textcolor{Sepia}{\selectlanguage{english}to study a little} \textcolor{PineGreen}{\selectlanguage{french}étudier un peu}  
\ding{203} \zh{学一个人、模仿一个人。} \textcolor{Sepia}{\selectlanguage{english}To follow the example of someone, to imitate someone.} \textcolor{PineGreen}{\selectlanguage{french}Imiter.}  ¶ \textcolor{darkblue}{\textbf{\ipa{tʰv̩˧ tʰɑ˧-so˧˥!}}} \zh{别学他! / 别做得像他一样!} \textcolor{Sepia}{\selectlanguage{english}Don't follow his example! / Don't do like him!} \textcolor{PineGreen}{\selectlanguage{french}Ne t'avise pas de suivre son exemple!/Ne va pas faire comme lui!/Ne va pas prendre exemple sur lui!}  
\ding{204} \zh{教。} \textcolor{Sepia}{\selectlanguage{english}To teach.} \textcolor{PineGreen}{\selectlanguage{french}Enseigner.}  ¶ \textcolor{darkblue}{\textbf{\ipa{tʰæ˧ɻæ˩ so˩}}} \zh{教书} \textcolor{Sepia}{\selectlanguage{english}to teach} \textcolor{PineGreen}{\selectlanguage{french}enseigner}  
 ¶ \textcolor{darkblue}{\textbf{\ipa{njɤ˧-ɳɯ˧ | no˧ so˧-bi˧!}}} \textcolor{Sepia}{\selectlanguage{english}I'm going to teach you! / Let me teach you!} \textcolor{PineGreen}{\selectlanguage{french}Je vais t'enseigner/t'apprendre!}  

\lhead{\firstmark}
\rhead{\botmark}

\subsection{\hspace{-0.5cm} {\Large \textcolor{darkblue}{\textbf{\ipa{so˩\textasciitilde{}so˧˥}}}}\hspace{0.5cm}[\kern2pt{\textcolor{darkblue}{\textbf{\ipa{so˧so˧˥}}}}\kern2pt]} \hypertarget{so\string_B~so\string_M\string_T1}{}
\markboth{\textcolor{darkblue}{\textbf{\ipa{so˩\textasciitilde{}so˧˥}}}}{}
\textcolor{teal}{\zh{动词}} \hspace{4pt} \zh{声调类:} MH.
\zh{揉在手里。} \textcolor{Sepia}{\selectlanguage{english}To rub in one's hands.} \textcolor{PineGreen}{\selectlanguage{french}Frotter dans ses mains.}  ¶ \textcolor{darkblue}{\textbf{\ipa{le˧-so˩\textasciitilde{}so˩}}} \zh{揉来揉去} \textcolor{Sepia}{\selectlanguage{english}\mytextsc{accomp} \string_ \mytextsc{red}} \textcolor{PineGreen}{\selectlanguage{french}\mytextsc{accomp} \string_ \mytextsc{red}}  

\lhead{\firstmark}
\rhead{\botmark}

\subsection{\hspace{-0.5cm} {\Large \textcolor{darkblue}{\textbf{\ipa{so˧˥}}}}\hspace{0.5cm}[\kern2pt{\textcolor{darkblue}{\textbf{\ipa{so˧˥}}}}\kern2pt]} \hypertarget{so\string_M\string_T1}{}
\markboth{\textcolor{darkblue}{\textbf{\ipa{so˧˥}}}}{}
\textcolor{teal}{\zh{名词}} \hspace{4pt} \zh{声调类:} MH.
\ding{202} \zh{(一口)气。} \textcolor{Sepia}{\selectlanguage{english}Breath.} \textcolor{PineGreen}{\selectlanguage{french}Souffle.}  \zh{量词}: \textcolor{darkblue}{\textbf{\ipa{kʰɯ˩}}} \ding{203} \zh{蒸汽。} \textcolor{Sepia}{\selectlanguage{english}Vapour.} \textcolor{PineGreen}{\selectlanguage{french}Vapeur.}  ¶ \textcolor{darkblue}{\textbf{\ipa{so˧ tʰv̩˥-ze˩}}} \zh{热气冒出来了。} \textcolor{Sepia}{\selectlanguage{english}Vapour is coming out.} \textcolor{PineGreen}{\selectlanguage{french}il y a de la vapeur qui sort, ça fait de la vapeur}  

\lhead{\firstmark}
\rhead{\botmark}

\subsection{\hspace{-0.5cm} {\Large \textcolor{darkblue}{\textbf{\ipa{sɯ˥}}} \textsubscript{1}}\hspace{0.5cm}[\kern2pt{\textcolor{darkblue}{\textbf{\ipa{sɯ˥}}}}\kern2pt]} \hypertarget{sM\string_T1}{}
\markboth{\textcolor{darkblue}{\textbf{\ipa{sɯ˥}}} \textsubscript{1}}{}
\textcolor{teal}{\zh{动词}} \hspace{4pt} \zh{声调类:} H.
\zh{磨(刀)。} \textcolor{Sepia}{\selectlanguage{english}To whet.} \textcolor{PineGreen}{\selectlanguage{french}Aiguiser.}  ¶ \textcolor{darkblue}{\textbf{\ipa{ɖɯ˧-sɯ˧\textasciitilde{}sɯ˧-ɻ̍˥}}} \zh{磨一磨} \textcolor{Sepia}{\selectlanguage{english}to whet a little} \textcolor{PineGreen}{\selectlanguage{french}aiguiser un peu}  
 ¶ \textcolor{darkblue}{\textbf{\ipa{sɯ˩tʰi˩ sɯ˩˥}}} \zh{磨刀} \textcolor{Sepia}{\selectlanguage{english}to whet a knife} \textcolor{PineGreen}{\selectlanguage{french}aiguiser un couteau}  

\lhead{\firstmark}
\rhead{\botmark}

\subsection{\hspace{-0.5cm} {\Large \textcolor{darkblue}{\textbf{\ipa{sɯ˥}}} \textsubscript{2}}\hspace{0.5cm}[\kern2pt{\textcolor{darkblue}{\textbf{\ipa{sɯ˥}}}}\kern2pt]} \hypertarget{sM\string_T2}{}
\markboth{\textcolor{darkblue}{\textbf{\ipa{sɯ˥}}} \textsubscript{2}}{}
\textcolor{teal}{\zh{动词}} \hspace{4pt} \zh{声调类:} H.
\zh{知道。} \textcolor{Sepia}{\selectlanguage{english}To know.} \textcolor{PineGreen}{\selectlanguage{french}Savoir.}  ¶ \textcolor{darkblue}{\textbf{\ipa{mɤ˧-sɯ˥}}} \zh{不知道} \textcolor{Sepia}{\selectlanguage{english}\mytextsc{neg}} \textcolor{PineGreen}{\selectlanguage{french}\mytextsc{neg}}  

\lhead{\firstmark}
\rhead{\botmark}

\subsection{\hspace{-0.5cm} {\Large \textcolor{darkblue}{\textbf{\ipa{‑sɯ˧}}}}\hspace{0.5cm}[\kern2pt{\textcolor{darkblue}{\textbf{\ipa{sɯ˥}}}}\kern2pt]} \hypertarget{‑sM\string_M1}{}
\markboth{\textcolor{darkblue}{\textbf{\ipa{‑sɯ˧}}}}{}
\textcolor{teal}{\zh{后缀}} \hspace{4pt} \zh{声调类:} M.
\zh{首先、先。} \textcolor{Sepia}{\selectlanguage{english}First, at first, in the first place; anymore (in “not anymore”).} \textcolor{PineGreen}{\selectlanguage{french}D'abord; encore (dans la tournure: pas encore).}  ¶ \textcolor{darkblue}{\textbf{\ipa{njɤ˧ ʈʂʰɯ˧-sɯ˩ | dzɯ˧-bi˧!}}} \zh{我要先吃这个!} \textcolor{Sepia}{\selectlanguage{english}Let me eat this one first! / I'll eat this one first!} \textcolor{PineGreen}{\selectlanguage{french}je vais d'abord manger celui-ci!}  
 ¶ \textcolor{darkblue}{\textbf{\ipa{njɤ˧ | ʈʂʰɯ˧-sɯ˩ | li˧-bi˧!}}} \zh{我要先读这本!} \textcolor{Sepia}{\selectlanguage{english}I'll read this one first! (Context: examining two books, and deciding which one to read first)} \textcolor{PineGreen}{\selectlanguage{french}je vais d'abord lire celui-ci! (au sujet de deux livres)}  
 ¶ \textcolor{darkblue}{\textbf{\ipa{ʈʂʰɯ˧ sɯ˩ | hwæ˧-bi˧!}}} \zh{先买这个吧!} \textcolor{Sepia}{\selectlanguage{english}Let's buy this one first!} \textcolor{PineGreen}{\selectlanguage{french}(je) vais d'abord acheter celui-ci!}  
 ¶ \textcolor{darkblue}{\textbf{\ipa{ʈʂʰɯ˧ sɯ˩ | tɕʰi˧-bi˧!}}} \zh{先卖这个吧!} \textcolor{Sepia}{\selectlanguage{english}Let's sell this one first!} \textcolor{PineGreen}{\selectlanguage{french}(je) vais d'abord vendre celui-ci!}  
 ¶ \textcolor{darkblue}{\textbf{\ipa{ʈʂʰɯ˧ sɯ˩ | dzɯ˧-bi˧!}}} \zh{先吃这个吧!} \textcolor{Sepia}{\selectlanguage{english}Let's eat this one first!} \textcolor{PineGreen}{\selectlanguage{french}(je) vais d'abord manger celui-ci!}  
 ¶ \textcolor{darkblue}{\textbf{\ipa{ʈʂʰɯ˧ sɯ˩ | ʑi˩-bi˩˥}}} \zh{先拿这个吧!} \textcolor{Sepia}{\selectlanguage{english}Let's pick up this one first!} \textcolor{PineGreen}{\selectlanguage{french}(je) vais d'abord prendre celui-ci!}  
 ¶ \textcolor{darkblue}{\textbf{\ipa{ʈʂʰɯ˧ sɯ˩ | ʈʰɯ˩-bi˩˥}}} \zh{先喝这个吧!} \textcolor{Sepia}{\selectlanguage{english}Let's drink this one first!} \textcolor{PineGreen}{\selectlanguage{french}(je) vais d'abord boire celui-ci!}  
 ¶ \textcolor{darkblue}{\textbf{\ipa{ʈʂʰɯ˧ sɯ˩ | lɑ˧-bi˥}}} \zh{先打这个吧!} \textcolor{Sepia}{\selectlanguage{english}Let's beat this one first!} \textcolor{PineGreen}{\selectlanguage{french}(je) vais d'abord battre celui-ci!}  
 ¶ \textcolor{darkblue}{\textbf{\ipa{tv̩˧tv̩˥ sɯ˩ | tʰi˧-tsʰi˥!}}} \zh{你先戴上帽子!(情景:出门前,让孩子戴上帽子)} \textcolor{Sepia}{\selectlanguage{english}Put on your hat first! (Injunction to a little child before an outing)} \textcolor{PineGreen}{\selectlanguage{french}Mets d'abord ton chapeau! (injonction à un petit enfant, avant une sortie)}  
 ¶ \textcolor{darkblue}{\textbf{\ipa{no˧ | le˧-sɯ˧ gv̩˧\textasciitilde{}gv̩˥!}}} \zh{你先自己工作(一会)吧!(情景:调查者早上到合作人的家,但她忙着,而她知道调查者有不同类型的工作要做,其中有一些可以自己做,比如重新核对记录过的长篇语料。她说:“你先忙自己的一会吧!”)} \textcolor{Sepia}{\selectlanguage{english}Do your own work first! / Please work on your own for a start! (Context: when I arrive for a morning class, the consultant is busy; she knows that I have various tasks to do, some of which I can do on my own, such as verifying texts that have already been transcribed; she tells me: “Please work on your own for a start!”)} \textcolor{PineGreen}{\selectlanguage{french}Commence par travailler tout seul! (Consigne de la locutrice quand j'arrive pour ma leçon du matin. Elle est occupée; et elle sait que j'ai de quoi m'occuper seul en l'attendant: toiletter des textes déjà transcrits, etc. Elle me dit “Commence par travailler tout seul! / Commence par les tâches que tu peux faire tout seul!”}  

\lhead{\firstmark}
\rhead{\botmark}

\subsection{\hspace{-0.5cm} {\Large \textcolor{darkblue}{\textbf{\ipa{sɯ˧\textsubscript{a}}}}}\hspace{0.5cm}[\kern2pt{\textcolor{darkblue}{\textbf{\ipa{sɯ˥}}}}\kern2pt]} \hypertarget{sM\string_Ma1}{}
\markboth{\textcolor{darkblue}{\textbf{\ipa{sɯ˧\textsubscript{a}}}}}{}
\textcolor{teal}{\zh{动词}} \hspace{4pt} \zh{声调类:} M\textsubscript{a}.
\ding{202} \zh{串(珠)。} \textcolor{Sepia}{\selectlanguage{english}To string (beads).} \textcolor{PineGreen}{\selectlanguage{french}Enfiler (des perles).}  ¶ \textcolor{darkblue}{\textbf{\ipa{sɯ˧ɻ̍˧ sɯ˧}}} \zh{串珠} \textcolor{Sepia}{\selectlanguage{english}to string beads} \textcolor{PineGreen}{\selectlanguage{french}enfiler des perles}  
 ¶ \textcolor{darkblue}{\textbf{\ipa{le˧-sɯ˧-se˩-ze˩}}} \zh{串完了!} \textcolor{Sepia}{\selectlanguage{english}(I) have finished to string (beads)} \textcolor{PineGreen}{\selectlanguage{french}(j'ai) fini d'enfiler}  
 ¶ \textcolor{darkblue}{\textbf{\ipa{tso˧\textasciitilde{}tso˧ sɯ˩}}} \zh{串东西} \textcolor{Sepia}{\selectlanguage{english}to string things} \textcolor{PineGreen}{\selectlanguage{french}enfiler des choses}  
\ding{203} \zh{穿(裙子)。} \textcolor{Sepia}{\selectlanguage{english}To put on (a skirt).} \textcolor{PineGreen}{\selectlanguage{french}Enfiler (une jupe).}  ¶ \textcolor{darkblue}{\textbf{\ipa{ʈʰæ˧qʰwɤ˧ sɯ˧}}} \zh{穿裙子} \textcolor{Sepia}{\selectlanguage{english}to put on a skirt} \textcolor{PineGreen}{\selectlanguage{french}enfiler une robe}  

\lhead{\firstmark}
\rhead{\botmark}

\subsection{\hspace{-0.5cm} {\Large \textcolor{darkblue}{\textbf{\ipa{sɯ˧gv̩\#˥}}}}\hspace{0.5cm}[\kern2pt{\textcolor{darkblue}{\textbf{\ipa{sɯ˧gv̩˧}}}}\kern2pt]} \hypertarget{sM\string_Mgv\string_=\#\string_T1}{}
\markboth{\textcolor{darkblue}{\textbf{\ipa{sɯ˧gv̩\#˥}}}}{}
\textcolor{teal}{\zh{名词}} \hspace{4pt} \zh{声调类:} \#H.
\zh{箱子,柜子。} \textcolor{Sepia}{\selectlanguage{english}Box, case.} \textcolor{PineGreen}{\selectlanguage{french}Caisse, coffre; par extension: armoire.}  \zh{量词}: \textcolor{darkblue}{\textbf{\ipa{ɭɯ˧}}} 
\lhead{\firstmark}
\rhead{\botmark}

\subsection{\hspace{-0.5cm} {\Large \textcolor{darkblue}{\textbf{\ipa{sɯ˧kʰɯ˩}}}}\hspace{0.5cm}[\kern2pt{\textcolor{darkblue}{\textbf{\ipa{sɯ˩kʰɯ˥}}}}\kern2pt]} \hypertarget{sM\string_Mk\string_hM\string_B1}{}
\markboth{\textcolor{darkblue}{\textbf{\ipa{sɯ˧kʰɯ˩}}}}{}
\textcolor{teal}{\zh{名词}} \hspace{4pt} \zh{声调类:} L\#.
\zh{斯克:嫁到外边的女人去世时进行的仪式。} \textcolor{Sepia}{\selectlanguage{english}Ritual performed for the death of a female relative who left her maternal home to marry.} \textcolor{PineGreen}{\selectlanguage{french}Rituel lors du décès d'une femme qui a quitté la maison de sa mère pour se marier.} 
\lhead{\firstmark}
\rhead{\botmark}

\subsection{\hspace{-0.5cm} {\Large \textcolor{darkblue}{\textbf{\ipa{sɯ˧ljɤ˧}}}}\hspace{0.5cm}[\kern2pt{\textcolor{darkblue}{\textbf{\ipa{sɯ˧ljɤ˧}}}}\kern2pt]} \hypertarget{sM\string_Mlj7\string_M1}{}
\markboth{\textcolor{darkblue}{\textbf{\ipa{sɯ˧ljɤ˧}}}}{}
\textcolor{teal}{\zh{名词}} \hspace{4pt} \zh{声调类:} M.
\zh{塑料(汉语借词)。} \textcolor{Sepia}{\selectlanguage{english}Plastic.} \textcolor{PineGreen}{\selectlanguage{french}Plastique.}  \zh{【借词】} \zh{塑料}

\lhead{\firstmark}
\rhead{\botmark}

\subsection{\hspace{-0.5cm} {\Large \textcolor{darkblue}{\textbf{\ipa{sɯ˧ljɤ˧tʰo˧˥}}}}\hspace{0.5cm}[\kern2pt{\textcolor{darkblue}{\textbf{\ipa{sɯ˧ljɤ˧tʰo˧}}}}\kern2pt]} \hypertarget{sM\string_Mlj7\string_Mt\string_ho\string_M\string_T1}{}
\markboth{\textcolor{darkblue}{\textbf{\ipa{sɯ˧ljɤ˧tʰo˧˥}}}}{}
\textcolor{teal}{\zh{名词}} \hspace{4pt} \zh{声调类:} MH\#.
\zh{塑料桶(汉语借词)。} \textcolor{Sepia}{\selectlanguage{english}Plastic jerrican; used to store and transport drinking water.} \textcolor{PineGreen}{\selectlanguage{french}Container pour liquides, en matière plastique.}  \zh{量词}: \textcolor{darkblue}{\textbf{\ipa{ɭɯ˧}}} 
\lhead{\firstmark}
\rhead{\botmark}

\subsection{\hspace{-0.5cm} {\Large \textcolor{darkblue}{\textbf{\ipa{sɯ˧mɤ˩}}}}\hspace{0.5cm}[\kern2pt{\textcolor{darkblue}{\textbf{\ipa{sɯ˧mɤ˧˥}}}}\kern2pt]} \hypertarget{sM\string_Mm7\string_B1}{}
\markboth{\textcolor{darkblue}{\textbf{\ipa{sɯ˧mɤ˩}}}}{}
\textcolor{teal}{\zh{名词}} \hspace{4pt} \zh{声调类:} L\#.
\zh{紫苏。} \textcolor{Sepia}{\selectlanguage{english}Purple perilla, \textit{Perilla frutescens}, akajiso.} \textcolor{PineGreen}{\selectlanguage{french}Shiso, \textit{Perilla frutescens}, akajiso: plante alimentaire, aromatique, médicinale et ornementale de la famille des Lamiacées. Ses petites graines noires ressemblent au sésame. Elle a été introduite à Yongning récemment (années 1980?) On se sert des graines pour confectionner des confiseries; et on en extrait de l'huile, qui se mange.}  ¶ \textcolor{darkblue}{\textbf{\ipa{sɯ˧mɤ˩, | ɬi˧di˩ | tv̩˧-kv̩˧˥!}}} \zh{在永宁,有紫苏!/ 有人种紫苏!} \textcolor{Sepia}{\selectlanguage{english}Purple perilla is cultivated in Yongning! / Purple perilla is among the crops that are grown in Yongning!} \textcolor{PineGreen}{\selectlanguage{french}Le shiso, on en cultive à Yongning!}  
 ¶ \textcolor{darkblue}{\textbf{\ipa{sɯ˧mɤ˩-dze˩}}} \zh{含紫苏的糖果} \textcolor{Sepia}{\selectlanguage{english}candy containing purple perilla seeds} \textcolor{PineGreen}{\selectlanguage{french}friandise contenant des graines de shiso}  
 ¶ \textcolor{darkblue}{\textbf{\ipa{sɯ˧mɤ˩-mæ˩ɻæ˩}}} \zh{紫苏油} \textcolor{Sepia}{\selectlanguage{english}purple perilla oil} \textcolor{PineGreen}{\selectlanguage{french}huile de shiso}  

\lhead{\firstmark}
\rhead{\botmark}

\subsection{\hspace{-0.5cm} {\Large \textcolor{darkblue}{\textbf{\ipa{sɯ˧pv̩˩}}}}\hspace{0.5cm}[\kern2pt{\textcolor{darkblue}{\textbf{\ipa{sɯ˧pv̩˩}}}}\kern2pt]} \hypertarget{sM\string_Mpv\string_=\string_B1}{}
\markboth{\textcolor{darkblue}{\textbf{\ipa{sɯ˧pv̩˩}}}}{}
\textcolor{teal}{\zh{名词}} \hspace{4pt} \zh{声调类:} L\#.
\zh{膀胱。} \textcolor{Sepia}{\selectlanguage{english}Urinary bladder.} \textcolor{PineGreen}{\selectlanguage{french}Vessie.}  \zh{量词}: \textcolor{darkblue}{\textbf{\ipa{kɤ˧˥}}} 
\lhead{\firstmark}
\rhead{\botmark}

\subsection{\hspace{-0.5cm} {\Large \textcolor{darkblue}{\textbf{\ipa{sɯ˧pv̩˩-ni˩gv̩˩}}}}\hspace{0.5cm}[\kern2pt{\textcolor{darkblue}{\textbf{\ipa{sɯ˧pv̩˩ni˧gv̩˧}}}}\kern2pt]} \hypertarget{sM\string_Mpv\string_=\string_B-ni\string_Bgv\string_=\string_B1}{}
\markboth{\textcolor{darkblue}{\textbf{\ipa{sɯ˧pv̩˩-ni˩gv̩˩}}}}{}
\textcolor{teal}{\zh{形容词}} \hspace{4pt} \zh{声调类:} L\#-.
\zh{膀肿。} \textcolor{Sepia}{\selectlanguage{english}Swollen: literally: 'like a bladder'.} \textcolor{PineGreen}{\selectlanguage{french}Gonflé, galbé, rond: littéralement “comme une vessie”. Une vessie de porc que l'on gonfle prend une forme toute ronde.} 
\lhead{\firstmark}
\rhead{\botmark}

\subsection{\hspace{-0.5cm} {\Large \textcolor{darkblue}{\textbf{\ipa{sɯ˧pv̩˧-sɯ˥nɑ˩}}}}\hspace{0.5cm}[\kern2pt{\textcolor{darkblue}{\textbf{\ipa{sɯ˧pv̩˧sɯ˥nɑ˩}}}}\kern2pt]} \hypertarget{sM\string_Mpv\string_=\string_M-sM\string_TnA\string_B1}{}
\markboth{\textcolor{darkblue}{\textbf{\ipa{sɯ˧pv̩˧-sɯ˥nɑ˩}}}}{}
\textcolor{teal}{\zh{名词}} \hspace{4pt} \zh{声调类:} \#H-.
\zh{毛虫。} \textcolor{Sepia}{\selectlanguage{english}Caterpillar.} \textcolor{PineGreen}{\selectlanguage{french}Chenille.}  \zh{量词}: \textcolor{darkblue}{\textbf{\ipa{mi˩}}} 
\lhead{\firstmark}
\rhead{\botmark}

\subsection{\hspace{-0.5cm} {\Large \textcolor{darkblue}{\textbf{\ipa{sɯ˧pʰi˧}}}}\hspace{0.5cm}[\kern2pt{\textcolor{darkblue}{\textbf{\ipa{sɯ˧pʰi˥}}}}\kern2pt]} \hypertarget{sM\string_Mp\string_hi\string_M1}{}
\markboth{\textcolor{darkblue}{\textbf{\ipa{sɯ˧pʰi˧}}}}{}
\textcolor{teal}{\zh{名词}} \hspace{4pt} \zh{声调类:} M.
\zh{贵族,土司,奴隶主,官。音译:“司沛”。} \textcolor{Sepia}{\selectlanguage{english}Chieftain, nobleman, lord: the highest of the three castes (ranks) in feudal society.} \textcolor{PineGreen}{\selectlanguage{french}Noble, seigneur, chef: la plus haute des 3 castes de la société ancienne.}  ¶ \textcolor{darkblue}{\textbf{\ipa{ɖæ˩mi˧-sɯ˩pʰi˩}}} \zh{大寺贵族} \textcolor{Sepia}{\selectlanguage{english}the noblemen at the monastery} \textcolor{PineGreen}{\selectlanguage{french}les nobles du monastère}  
 ¶ \textcolor{darkblue}{\textbf{\ipa{sɯ˧pʰi˧ hĩ˩}}} \zh{贵族的臣子、贵族手下的人} \textcolor{Sepia}{\selectlanguage{english}the nobleman's subjects, the nobleman's people} \textcolor{PineGreen}{\selectlanguage{french}les sujets du seigneur, les gens du seigneur}  
 \zh{量词}: \textcolor{darkblue}{\textbf{\ipa{v̩˧}}} 
\lhead{\firstmark}
\rhead{\botmark}

\subsection{\hspace{-0.5cm} {\Large \textcolor{darkblue}{\textbf{\ipa{sɯ˧pʰi˧-zo˧}}}}\hspace{0.5cm}[\kern2pt{\textcolor{darkblue}{\textbf{\ipa{xxxx non-correspondance entre le nombre de morphèmes et le nombre de tons de morphèmes}}}}\kern2pt]} \hypertarget{sM\string_Mp\string_hi\string_M-zo\string_M1}{}
\markboth{\textcolor{darkblue}{\textbf{\ipa{sɯ˧pʰi˧-zo˧}}}}{}
\textcolor{teal}{\zh{名词}} \hspace{4pt} \zh{声调类:} H\#.
\zh{少爷。} \textcolor{Sepia}{\selectlanguage{english}Young man of the nobility.} \textcolor{PineGreen}{\selectlanguage{french}Jeune homme de la noblesse, fils de mandarin.}  \zh{量词}: \textcolor{darkblue}{\textbf{\ipa{v̩˧}}} 
\lhead{\firstmark}
\rhead{\botmark}

\subsection{\hspace{-0.5cm} {\Large \textcolor{darkblue}{\textbf{\ipa{sɯ˧ɻ̍˧}}}}\hspace{0.5cm}[\kern2pt{\textcolor{darkblue}{\textbf{\ipa{sɯ˧ɻ̍˧}}}}\kern2pt]} \hypertarget{sM\string_Mr£`̍\string_M1}{}
\markboth{\textcolor{darkblue}{\textbf{\ipa{sɯ˧ɻ̍˧}}}}{}
\textcolor{teal}{\zh{名词}} \hspace{4pt} \zh{声调类:} M.
\zh{珠,珠子,珍珠。} \textcolor{Sepia}{\selectlanguage{english}Bead, pearl.} \textcolor{PineGreen}{\selectlanguage{french}Perle.}  \zh{量词}: \textcolor{darkblue}{\textbf{\ipa{ɭɯ˧}}} 
\lhead{\firstmark}
\rhead{\botmark}

\subsection{\hspace{-0.5cm} {\Large \textcolor{darkblue}{\textbf{\ipa{sɯ˧ɻæ˧}}}}\hspace{0.5cm}[\kern2pt{\textcolor{darkblue}{\textbf{\ipa{sɯ˧ɻæ˧}}}}\kern2pt]} \hypertarget{sM\string_Mr£`\{\string_M1}{}
\markboth{\textcolor{darkblue}{\textbf{\ipa{sɯ˧ɻæ˧}}}}{}
\textcolor{teal}{\zh{名词}} \hspace{4pt} \zh{声调类:} M.
\zh{桌子。} \textcolor{Sepia}{\selectlanguage{english}Table.} \textcolor{PineGreen}{\selectlanguage{french}Table (à Yongning, vers le milieu du XXe siècle, elles étaient basses et carrées).}  \zh{量词}: \textcolor{darkblue}{\textbf{\ipa{pɤ˩}}} \zh{~【参考】~} \hyperlink{}{\textcolor{darkblue}{\textbf{\ipa{ʈʂo˧tsɯ˥}}}} 
\lhead{\firstmark}
\rhead{\botmark}

\subsection{\hspace{-0.5cm} {\Large \textcolor{darkblue}{\textbf{\ipa{sɯ˧ɻ̃\#˥}}}}\hspace{0.5cm}[\kern2pt{\textcolor{darkblue}{\textbf{\ipa{sɯ˧ɻ̃˧}}}}\kern2pt]} \hypertarget{sM\string_Mr£`\string_~\#\string_T1}{}
\markboth{\textcolor{darkblue}{\textbf{\ipa{sɯ˧ɻ̃\#˥}}}}{}
\textcolor{teal}{\zh{名词}} \hspace{4pt} \zh{声调类:} \#H.
\zh{树干。} \textcolor{Sepia}{\selectlanguage{english}Tree trunk.} \textcolor{PineGreen}{\selectlanguage{french}Tronc.}  ¶ \textcolor{darkblue}{\textbf{\ipa{si˧dzi˩ tʰv̩˩-dzi˩, | sɯ˧ɻ̃˧ dʑɤ˥!}}} \zh{这是棵好树!(可以用做木料)} \textcolor{Sepia}{\selectlanguage{english}This tree has a good trunk! (i.e. it is suitable for use in carpentry, making furniture...)} \textcolor{PineGreen}{\selectlanguage{french}Cet arbre, il a un beau tronc! (c'est-à-dire qu'il est utilisable en menuiserie)}  
 \zh{量词}: \textcolor{darkblue}{\textbf{\ipa{lo˩}}} 
\lhead{\firstmark}
\rhead{\botmark}

\subsection{\hspace{-0.5cm} {\Large \textcolor{darkblue}{\textbf{\ipa{sɯ˧ɻ̃˧mi\#˥}}}}\hspace{0.5cm}[\kern2pt{\textcolor{darkblue}{\textbf{\ipa{sɯ˧ɻ̃˧mi˧}}}}\kern2pt]} \hypertarget{sM\string_Mr£`\string_~\string_Mmi\#\string_T1}{}
\markboth{\textcolor{darkblue}{\textbf{\ipa{sɯ˧ɻ̃˧mi\#˥}}}}{}
\textcolor{teal}{\zh{名词}} \hspace{4pt} \zh{声调类:} \#H.
\zh{脊椎骨。} \textcolor{Sepia}{\selectlanguage{english}Backbone.} \textcolor{PineGreen}{\selectlanguage{french}Colonne vertébrale.}  \zh{量词}: \textcolor{darkblue}{\textbf{\ipa{dzi˩}}} 
\lhead{\firstmark}
\rhead{\botmark}

\subsection{\hspace{-0.5cm} {\Large \textcolor{darkblue}{\textbf{\ipa{sɯ˧sɯ˩}}}}\hspace{0.5cm}[\kern2pt{\textcolor{darkblue}{\textbf{\ipa{sɯ˧sɯ˩}}}}\kern2pt]} \hypertarget{sM\string_MsM\string_B1}{}
\markboth{\textcolor{darkblue}{\textbf{\ipa{sɯ˧sɯ˩}}}}{}
\textcolor{teal}{\zh{形容词}} \hspace{4pt} \zh{声调类:} L\#.
\zh{生(不熟)。} \textcolor{Sepia}{\selectlanguage{english}Raw.} \textcolor{PineGreen}{\selectlanguage{french}Cru.}  ¶ \textcolor{darkblue}{\textbf{\ipa{ʂe˧ sɯ˧\textasciitilde{}sɯ˥}}} \zh{生肉} \textcolor{Sepia}{\selectlanguage{english}raw meat} \textcolor{PineGreen}{\selectlanguage{french}viande crue}  
 ¶ \textcolor{darkblue}{\textbf{\ipa{ʈʂe˧ sɯ˧\textasciitilde{}sɯ˥}}} \zh{‘生土’:没有经过加工(加肥料等等)的土,还不适合种农作物} \textcolor{Sepia}{\selectlanguage{english}'raw earth': immature soil, earth that has not been prepared for agriculture by adding manure, etc} \textcolor{PineGreen}{\selectlanguage{french}'terre crue': terre qui n'a pas été préparée pour l'agriculture par l'ajout de fumier, etc}  

\lhead{\firstmark}
\rhead{\botmark}

\subsection{\hspace{-0.5cm} {\Large \textcolor{darkblue}{\textbf{\ipa{sɯ˧tsɯ˧}}}}\hspace{0.5cm}[\kern2pt{\textcolor{darkblue}{\textbf{\ipa{sɯ˧tsɯ˧}}}}\kern2pt]} \hypertarget{sM\string_MtsM\string_M1}{}
\markboth{\textcolor{darkblue}{\textbf{\ipa{sɯ˧tsɯ˧}}}}{}
\textcolor{teal}{\zh{名词}} \hspace{4pt} \zh{声调类:} M.
\zh{狮子(汉语借词)。} \textcolor{Sepia}{\selectlanguage{english}Lion.} \textcolor{PineGreen}{\selectlanguage{french}Lion.}  \zh{【借词】} \zh{狮子}
 \zh{量词}: \textcolor{darkblue}{\textbf{\ipa{mi˩}}} 
\lhead{\firstmark}
\rhead{\botmark}

\subsection{\hspace{-0.5cm} {\Large \textcolor{darkblue}{\textbf{\ipa{sɯ˧tsɯ˩}}}}\hspace{0.5cm}[\kern2pt{\textcolor{darkblue}{\textbf{\ipa{sɯ˧tsɯ˩}}}}\kern2pt]} \hypertarget{sM\string_MtsM\string_B1}{}
\markboth{\textcolor{darkblue}{\textbf{\ipa{sɯ˧tsɯ˩}}}}{}
\textcolor{teal}{\zh{名词}} \hspace{4pt} \zh{声调类:} L\#.
\zh{樟。} \textcolor{Sepia}{\selectlanguage{english}Camphor.} \textcolor{PineGreen}{\selectlanguage{french}Camphre (arbre).}  ¶ \textcolor{darkblue}{\textbf{\ipa{sɯ˧tsɯ˩-dzi˩}}} \zh{樟树} \textcolor{Sepia}{\selectlanguage{english}camphor tree} \textcolor{PineGreen}{\selectlanguage{french}arbre à camphre}  

\lhead{\firstmark}
\rhead{\botmark}

\subsection{\hspace{-0.5cm} {\Large \textcolor{darkblue}{\textbf{\ipa{sɯ˧ʈv̩˥}}}}\hspace{0.5cm}[\kern2pt{\textcolor{darkblue}{\textbf{\ipa{sɯ˧ʈv̩˥}}}}\kern2pt]} \hypertarget{sM\string_Mt`v\string_=\string_T1}{}
\markboth{\textcolor{darkblue}{\textbf{\ipa{sɯ˧ʈv̩˥}}}}{}
\textcolor{teal}{\zh{名词}} \hspace{4pt} \zh{声调类:} H\#.
\zh{茧子。} \textcolor{Sepia}{\selectlanguage{english}Callus.} \textcolor{PineGreen}{\selectlanguage{french}Cor, durillon.}  ¶ \textcolor{darkblue}{\textbf{\ipa{hĩ˧ ʈʂʰɯ˧-v̩˧-bv̩˧ | mv̩˧ɲi˧, | sɯ˧ʈv̩˥ ʁo˩!}}} \zh{这个人的拇指有茧子!} \textcolor{Sepia}{\selectlanguage{english}This person's thumb has a callus / developed a callus!} \textcolor{PineGreen}{\selectlanguage{french}le pouce de cette personne a un cor/un durillon!}  
 ¶ \textcolor{darkblue}{\textbf{\ipa{sɯ˧ʈv̩˥ | mɤ˧-ʁo˩-ze˩!}}} \zh{没有茧子了!} \textcolor{Sepia}{\selectlanguage{english}The callus is gone! / There is no callus anymore!} \textcolor{PineGreen}{\selectlanguage{french}(il n'y a/je n'ai) plus de durillon!}  
 ¶ \textcolor{darkblue}{\textbf{\ipa{sɯ˧ʈv̩˥ ʁo˩-ze˩!}}} \zh{磨出了茧子!} \textcolor{Sepia}{\selectlanguage{english}A callus has formed!} \textcolor{PineGreen}{\selectlanguage{french}Un durillon s'est formé!}  
 \zh{量词}: \textcolor{darkblue}{\textbf{\ipa{ɭɯ˧}}} 
\lhead{\firstmark}
\rhead{\botmark}

\subsection{\hspace{-0.5cm} {\Large \textcolor{darkblue}{\textbf{\ipa{sɯ˧zɯ\#˥}}}}\hspace{0.5cm}[\kern2pt{\textcolor{darkblue}{\textbf{\ipa{sɯ˧zɯ˧}}}}\kern2pt]} \hypertarget{sM\string_MzM\#\string_T1}{}
\markboth{\textcolor{darkblue}{\textbf{\ipa{sɯ˧zɯ\#˥}}}}{}
\textcolor{teal}{\zh{名词}} \hspace{4pt} \zh{声调类:} \#H.
\zh{家族、支系。} \textcolor{Sepia}{\selectlanguage{english}Family community.} \textcolor{PineGreen}{\selectlanguage{french}Communauté familiale; échelon inférieur à “os”.}  ¶ \textcolor{darkblue}{\textbf{\ipa{sɯ˧zɯ˧ ɖɯ˧-lo˩}}} \zh{一个支系,一条线} \textcolor{Sepia}{\selectlanguage{english}one family community} \textcolor{PineGreen}{\selectlanguage{french}une communauté familiale}  
 ¶ \textcolor{darkblue}{\textbf{\ipa{sɯ˧zɯ˧ ɖɯ˧-ʁwɤ˧}}} \zh{一个支系,一条线} \textcolor{Sepia}{\selectlanguage{english}one family community} \textcolor{PineGreen}{\selectlanguage{french}une communauté familiale}  
 ¶ \textcolor{darkblue}{\textbf{\ipa{sɯ˧zɯ˧ ə˩-dʑo˩?}}} \zh{家族齐全吗?/ 家族,人多吗?(谈婚姻前的题目之一:男方家族人多不多。以人多为好。)} \textcolor{Sepia}{\selectlanguage{english}Is there a (complete) family community? / Is the family large? (Question asked as part of discussions preliminary to marriage: Will the bride have a large family around her, be surrounded by a large family? A small family is considered much less attractive than a large one.)} \textcolor{PineGreen}{\selectlanguage{french}est-ce qu'il a une grande famille/est-ce que sa famille est nombreuse? =est-ce que la mariée sera bien entourée, intégrée dans une grande famille? (Question que l'on pose lors des discussions préliminaires aux mariages: on s'inquiète des qualités de la maisonnée que la jeune femme va rejoindre)}  
 \zh{量词}: \textcolor{darkblue}{\textbf{\ipa{lo˩, ʁwɤ˧}}} 
\lhead{\firstmark}
\rhead{\botmark}

\subsection{\hspace{-0.5cm} {\Large \textcolor{darkblue}{\textbf{\ipa{sɯ˩\textsubscript{a}}}}}\hspace{0.5cm}[\kern2pt{\textcolor{darkblue}{\textbf{\ipa{sɯ˩˥}}}}\kern2pt]} \hypertarget{sM\string_Ba1}{}
\markboth{\textcolor{darkblue}{\textbf{\ipa{sɯ˩\textsubscript{a}}}}}{}
\textcolor{teal}{\zh{动词}} \hspace{4pt} \zh{声调类:} L\textsubscript{a}.
\zh{活。} \textcolor{Sepia}{\selectlanguage{english}To live, to be alive.} \textcolor{PineGreen}{\selectlanguage{french}Vivre, être vivant.}  ¶ \textcolor{darkblue}{\textbf{\ipa{ʈʂʰɯ˧ tʰi˧-sɯ˩-dʑo˩!}}} \zh{他活着!} \textcolor{Sepia}{\selectlanguage{english}(S)he is alive!} \textcolor{PineGreen}{\selectlanguage{french}Elle/il est vivant(e)!}  
 ¶ \textcolor{darkblue}{\textbf{\ipa{ʈʂʰɯ˧ | mɤ˧-ʂɯ˧! | tʰi˧-sɯ˩-dʑo˩!}}} \zh{它没死,还活着!(一个植物、动物)} \textcolor{Sepia}{\selectlanguage{english}It's not dead! It's alive! (About a plant or animal that looked dead)} \textcolor{PineGreen}{\selectlanguage{french}ce n'est pas mort! c'est encore vivant! (au sujet d'une plante/d'un animal qui paraissait mort(e))}  

\lhead{\firstmark}
\rhead{\botmark}

\subsection{\hspace{-0.5cm} {\Large \textcolor{darkblue}{\textbf{\ipa{sɯ˩pv̩˩}}}}\hspace{0.5cm}[\kern2pt{\textcolor{darkblue}{\textbf{\ipa{sɯ˧pv̩˥}}}}\kern2pt]} \hypertarget{sM\string_Bpv\string_=\string_B1}{}
\markboth{\textcolor{darkblue}{\textbf{\ipa{sɯ˩pv̩˩}}}}{}
\textcolor{teal}{\zh{名词}} \hspace{4pt} \zh{声调类:} .
\zh{水泡(例如:开水烫了手,会形成水泡)。} \textcolor{Sepia}{\selectlanguage{english}Raised spot, blister.} \textcolor{PineGreen}{\selectlanguage{french}Cloque (par exemple, après qu'on se soit ébouillanté).}  ¶ \textcolor{darkblue}{\textbf{\ipa{sɯ˩pv̩˩ qʰwæ˥-ze˩!}}} \zh{起了水泡!} \textcolor{Sepia}{\selectlanguage{english}a raised spot has formed!} \textcolor{PineGreen}{\selectlanguage{french}une cloque s'est formée!}  
 \zh{量词}: \textcolor{darkblue}{\textbf{\ipa{ɭɯ˧}}} 
\lhead{\firstmark}
\rhead{\botmark}

\subsection{\hspace{-0.5cm} {\Large \textcolor{darkblue}{\textbf{\ipa{sɯ˩ɻ̍˩}}}}\hspace{0.5cm}[\kern2pt{\textcolor{darkblue}{\textbf{\ipa{sɯ˩ɻ̍˩˥}}}}\kern2pt]} \hypertarget{sM\string_Br£`̍\string_B1}{}
\markboth{\textcolor{darkblue}{\textbf{\ipa{sɯ˩ɻ̍˩}}}}{}
\textcolor{teal}{\zh{名词}} \hspace{4pt} \zh{声调类:} L.
\zh{磨刀石。} \textcolor{Sepia}{\selectlanguage{english}Whetting-stone.} \textcolor{PineGreen}{\selectlanguage{french}Pierre à aiguiser, “fusil”.}  \zh{量词}: \textcolor{darkblue}{\textbf{\ipa{ɭɯ˧}}} 
\lhead{\firstmark}
\rhead{\botmark}

\subsection{\hspace{-0.5cm} {\Large \textcolor{darkblue}{\textbf{\ipa{sɯ˩tʰi˩}}}}\hspace{0.5cm}[\kern2pt{\textcolor{darkblue}{\textbf{\ipa{sɯ˩tʰi˩˥}}}}\kern2pt]} \hypertarget{sM\string_Bt\string_hi\string_B1}{}
\markboth{\textcolor{darkblue}{\textbf{\ipa{sɯ˩tʰi˩}}}}{}
\textcolor{teal}{\zh{名词}} \hspace{4pt} \zh{声调类:} L.
\zh{刀。} \textcolor{Sepia}{\selectlanguage{english}Knife.} \textcolor{PineGreen}{\selectlanguage{french}Couteau.}  \zh{量词}: \textcolor{darkblue}{\textbf{\ipa{nɑ˧}}} 
\lhead{\firstmark}
\rhead{\botmark}

\subsection{\hspace{-0.5cm} {\Large \textcolor{darkblue}{\textbf{\ipa{sɯ˩tʰi˩-kʰɯ˥ʑi˩}}}}\hspace{0.5cm}[\kern2pt{\textcolor{darkblue}{\textbf{\ipa{xxxx ton non trouvé, à faire manuellement...}}}}\kern2pt]} \hypertarget{sM\string_Bt\string_hi\string_B-k\string_hM\string_Tz£i\string_B1}{}
\markboth{\textcolor{darkblue}{\textbf{\ipa{sɯ˩tʰi˩-kʰɯ˥ʑi˩}}}}{}
\textcolor{teal}{\zh{名词}} \hspace{4pt} \zh{声调类:} L+\#H-.
\zh{刀鞘。} \textcolor{Sepia}{\selectlanguage{english}Knife sheath.} \textcolor{PineGreen}{\selectlanguage{french}Fourreau du couteau, gaine du couteau.}  \zh{量词}: \textcolor{darkblue}{\textbf{\ipa{ɭɯ˧}}} 
\lhead{\firstmark}
\rhead{\botmark}

\newpage
\section*{\centering- \textcolor{darkblue}{\textbf{\ipa{ʂ}}} -}
\subsection{\hspace{-0.5cm} {\Large \textcolor{darkblue}{\textbf{\ipa{ʂæ˥-ljɤ˩}}}}\hspace{0.5cm}[\kern2pt{\textcolor{darkblue}{\textbf{\ipa{xxxx non-correspondance entre le nombre de morphèmes et le nombre de tons de morphèmes}}}}\kern2pt]} \hypertarget{s`\{\string_T-lj7\string_B1}{}
\markboth{\textcolor{darkblue}{\textbf{\ipa{ʂæ˥-ljɤ˩}}}}{}
\textcolor{teal}{\zh{动词}} \hspace{4pt} \zh{声调类:} H-.
\zh{商量(汉语借词。} \textcolor{Sepia}{\selectlanguage{english}To discuss.} \textcolor{PineGreen}{\selectlanguage{french}Discuter.}  \zh{【借词】} \zh{商量}

\lhead{\firstmark}
\rhead{\botmark}

\subsection{\hspace{-0.5cm} {\Large \textcolor{darkblue}{\textbf{\ipa{ʂæ˧}}}}\hspace{0.5cm}[\kern2pt{\textcolor{darkblue}{\textbf{\ipa{ʂæ˥}}}}\kern2pt]} \hypertarget{s`\{\string_M1}{}
\markboth{\textcolor{darkblue}{\textbf{\ipa{ʂæ˧}}}}{}
\textcolor{teal}{\zh{形容词}} \hspace{4pt} \zh{声调类:} M.
\ding{202} \zh{长。} \textcolor{Sepia}{\selectlanguage{english}Long.} \textcolor{PineGreen}{\selectlanguage{french}Long.}  ¶ \textcolor{darkblue}{\textbf{\ipa{qʰɑ˧-ʂæ˧-gv̩˧}}} \zh{非常长} \textcolor{Sepia}{\selectlanguage{english}extremely long} \textcolor{PineGreen}{\selectlanguage{french}très long}  
 ¶ \textcolor{darkblue}{\textbf{\ipa{le˧-ʈɤ˧-le˧-ʂæ˧ (+kʰɯ˧˥)}}} \zh{拉长} \textcolor{Sepia}{\selectlanguage{english}to lengthen} \textcolor{PineGreen}{\selectlanguage{french}allonger, étirer}  
\ding{203} \zh{远。} \textcolor{Sepia}{\selectlanguage{english}Distant, far.} \textcolor{PineGreen}{\selectlanguage{french}Lointain, distant, éloigné.} 
\lhead{\firstmark}
\rhead{\botmark}

\subsection{\hspace{-0.5cm} {\Large \textcolor{darkblue}{\textbf{\ipa{ʂæ˧ɖæ\#˥}}}}\hspace{0.5cm}[\kern2pt{\textcolor{darkblue}{\textbf{\ipa{ʂæ˧ɖæ˧}}}}\kern2pt]} \hypertarget{s`\{\string_Md`\{\#\string_T1}{}
\markboth{\textcolor{darkblue}{\textbf{\ipa{ʂæ˧ɖæ\#˥}}}}{}
\textcolor{teal}{\zh{名词}} \hspace{4pt} \zh{声调类:} \#H.
\zh{长度区别。} \textcolor{Sepia}{\selectlanguage{english}Difference in length.} \textcolor{PineGreen}{\selectlanguage{french}Différence de longueur.}  ¶ \textcolor{darkblue}{\textbf{\ipa{ʂæ˧ɖæ˧ di˥, | mɤ˧-dʑɤ˩!}}} \zh{如果长短不一,不好!/不行!(情景:解释砍树时如何选择合适的树)} \textcolor{Sepia}{\selectlanguage{english}If there are differences in length, it's not good / it won't do! (Context: explaining which trees to fell when in need of timber for housebuilding; the trees need to be about the same size.)} \textcolor{PineGreen}{\selectlanguage{french}S'il y a des différences de longueur, c'est vilain/ça ne convient pas! (Contexte: explication au sujet du choix d'arbres à abattre pour obtenir du bois de charpente.)}  
 ¶ \textcolor{darkblue}{\textbf{\ipa{ʂæ˧ɖæ˧ | mɤ˧-di˩!}}} \zh{没有长度区别,都一样齐!(等于是好的建房木料)(情景:同上)} \textcolor{Sepia}{\selectlanguage{english}There are no differences in length! (i.e. the timber is suitable for use in construction; same context as previous example)} \textcolor{PineGreen}{\selectlanguage{french}il n'y a pas de différences de longueur (=c'est très bien)! (Même contexte que ci-dessus: choix d'arbres à abattre pour obtenir du bois de charpente)}  
 \zh{量词}: \textcolor{darkblue}{\textbf{\ipa{kʰwɤ˥}}} 
\lhead{\firstmark}
\rhead{\botmark}

\subsection{\hspace{-0.5cm} {\Large \textcolor{darkblue}{\textbf{\ipa{ʂæ˧-lo˩pv˩}}}}\hspace{0.5cm}[\kern2pt{\textcolor{darkblue}{\textbf{\ipa{ʂæ˧lo˧pv˧}}}}\kern2pt]} \hypertarget{s`\{\string_M-lo\string_Bpv\string_B1}{}
\markboth{\textcolor{darkblue}{\textbf{\ipa{ʂæ˧-lo˩pv˩}}}}{}
\textcolor{teal}{\zh{名词}} \hspace{4pt} \zh{声调类:} -L.
\zh{山萝卜。} \textcolor{Sepia}{\selectlanguage{english}Scabious.} \textcolor{PineGreen}{\selectlanguage{french}Yyyy.}  \zh{【借词】} \zh{山萝卜}
\zh{~【参考】~} \hyperlink{}{\textcolor{darkblue}{\textbf{\ipa{hwɤ˧li˧-hwæ˧qʰæ\#˥}}}} 
\lhead{\firstmark}
\rhead{\botmark}

\subsection{\hspace{-0.5cm} {\Large \textcolor{darkblue}{\textbf{\ipa{ʂæ˧pʰi˧}}}}\hspace{0.5cm}[\kern2pt{\textcolor{darkblue}{\textbf{\ipa{ʂæ˩pʰi˩˥}}}}\kern2pt]} \hypertarget{s`\{\string_Mp\string_hi\string_M1}{}
\markboth{\textcolor{darkblue}{\textbf{\ipa{ʂæ˧pʰi˧}}}}{}
\textcolor{teal}{\zh{名词}} \hspace{4pt} \zh{声调类:} M.
\zh{商品。} \textcolor{Sepia}{\selectlanguage{english}Commodity, goods, merchandise.} \textcolor{PineGreen}{\selectlanguage{french}Marchandise, objet qui peut se vendre au marché.}  \zh{【借词】} \zh{商品}

\lhead{\firstmark}
\rhead{\botmark}

\subsection{\hspace{-0.5cm} {\Large \textcolor{darkblue}{\textbf{\ipa{ʂæ˧ʁwɤ˩}}}}\hspace{0.5cm}[\kern2pt{\textcolor{darkblue}{\textbf{\ipa{ʂæ˩ʁwɤ˩˥}}}}\kern2pt]} \hypertarget{s`\{\string_MRw7\string_B1}{}
\markboth{\textcolor{darkblue}{\textbf{\ipa{ʂæ˧ʁwɤ˩}}}}{}
\textcolor{teal}{\zh{名词}} \hspace{4pt} \zh{声调类:} L\#.
\zh{束河(旧称:龙泉):丽江坝子里的一个村落。由于束河商人多,经常有束河人到永宁等地,使得相当多的永宁人熟悉那个村落名。} \textcolor{Sepia}{\selectlanguage{english}Shuhe: the name of a village in the Lijiang plain.} \textcolor{PineGreen}{\selectlanguage{french}Shuhe: nom d'un village de la plaine de Lijiang (anciennement Longquan). Les terres de ce village étaient médiocres, et beaucoup de ses habitants se tournaient vers le commerce et voyageaient dans toute la région, d'où le fait que le nom de ce village soit connu à Yongning.}  \zh{【借词】}Naxi \textcolor{darkblue}{\textbf{\ipa{/sɑ˥wɤ˧/}}}

\lhead{\firstmark}
\rhead{\botmark}

\subsection{\hspace{-0.5cm} {\Large \textcolor{darkblue}{\textbf{\ipa{ʂæ˧tsɯ˧}}}}\hspace{0.5cm}[\kern2pt{\textcolor{darkblue}{\textbf{\ipa{ʂæ˧tsɯ˩}}}}\kern2pt]} \hypertarget{s`\{\string_MtsM\string_M1}{}
\markboth{\textcolor{darkblue}{\textbf{\ipa{ʂæ˧tsɯ˧}}}}{}
\textcolor{teal}{\zh{名词}} \hspace{4pt} \zh{声调类:} M.
\zh{裋、卡夫坦长衣:成年前男女小孩均穿的裋,成年男人也穿。} \textcolor{Sepia}{\selectlanguage{english}Kaftan: clothing that children used to wear before they came of age: a loose robe (the same for girls and boys); also worn by adult men in earlier times.} \textcolor{PineGreen}{\selectlanguage{french}Caftan: vêtement que portaient les enfants avant leurs treize ans: robe ample (la même pour les filles et les garçons); anciennement, les hommes aussi portaient ce type de vêtement.}  \zh{量词}: \textcolor{darkblue}{\textbf{\ipa{ɭɯ˧˥}}} 
\lhead{\firstmark}
\rhead{\botmark}

\subsection{\hspace{-0.5cm} {\Large \textcolor{darkblue}{\textbf{\ipa{ʂæ˩ɻ̃˩}}}}\hspace{0.5cm}[\kern2pt{\textcolor{darkblue}{\textbf{\ipa{ʂæ˧ɻ̃˧˥}}}}\kern2pt]} \hypertarget{s`\{\string_Br£`\string_~\string_B1}{}
\markboth{\textcolor{darkblue}{\textbf{\ipa{ʂæ˩ɻ̃˩}}}}{}
\textcolor{teal}{\zh{名词}} \hspace{4pt} \zh{声调类:} L.
\zh{骨头。} \textcolor{Sepia}{\selectlanguage{english}Bone.} \textcolor{PineGreen}{\selectlanguage{french}Os, ossement.}  \zh{量词}: \textcolor{darkblue}{\textbf{\ipa{kɤ˧˥}}} 
\lhead{\firstmark}
\rhead{\botmark}

\subsection{\hspace{-0.5cm} {\Large \textcolor{darkblue}{\textbf{\ipa{ʂæ˧˥}}} \textsubscript{1}}\hspace{0.5cm}[\kern2pt{\textcolor{darkblue}{\textbf{\ipa{ʂæ˥}}}}\kern2pt]} \hypertarget{s`\{\string_M\string_T1}{}
\markboth{\textcolor{darkblue}{\textbf{\ipa{ʂæ˧˥}}} \textsubscript{1}}{}
\textcolor{teal}{\zh{动词}} \hspace{4pt} \zh{声调类:} MH.
\zh{牵(牵着牛)。} \textcolor{Sepia}{\selectlanguage{english}To lead along (by hand, halter…).} \textcolor{PineGreen}{\selectlanguage{french}Tenir un chien en laisse, mener un chien; mener, guider, amener (les vaches aux pâturages, etc).}  ¶ \textcolor{darkblue}{\textbf{\ipa{kʰv̩˧ ʂæ˧˥}}} \zh{遛狗,狩猎} \textcolor{Sepia}{\selectlanguage{english}to lead a dog; to hunt} \textcolor{PineGreen}{\selectlanguage{french}mener un chien; chasser}  
 ¶ \textcolor{darkblue}{\textbf{\ipa{kʰv̩˧ʂæ˧ hɯ˧˥}}} \zh{狩猎去了} \textcolor{Sepia}{\selectlanguage{english}gone hunting, out hunting} \textcolor{PineGreen}{\selectlanguage{french}parti chasser, parti à la chasse}  

\lhead{\firstmark}
\rhead{\botmark}

\subsection{\hspace{-0.5cm} {\Large \textcolor{darkblue}{\textbf{\ipa{ʂæ˧˥}}} \textsubscript{2}}\hspace{0.5cm}[\kern2pt{\textcolor{darkblue}{\textbf{\ipa{ʂæ˧˥}}}}\kern2pt]} \hypertarget{s`\{\string_M\string_T2}{}
\markboth{\textcolor{darkblue}{\textbf{\ipa{ʂæ˧˥}}} \textsubscript{2}}{}
\textcolor{teal}{\zh{动词}} \hspace{4pt} \zh{声调类:} MH.
\ding{202} \zh{捆成一包。} \textcolor{Sepia}{\selectlanguage{english}To tie into bundles.} \textcolor{PineGreen}{\selectlanguage{french}Attacher, nouer en bottes.}  ¶ \textcolor{darkblue}{\textbf{\ipa{le˧-ʂæ˧˥}}} \zh{\mytextsc{accomp}} \textcolor{Sepia}{\selectlanguage{english}\mytextsc{accomp}} \textcolor{PineGreen}{\selectlanguage{french}\mytextsc{accomp}}  
 ¶ \textcolor{darkblue}{\textbf{\ipa{hɑ˧ ʂæ˩}}} \zh{刚收割的稻子,捆成捆} \textcolor{Sepia}{\selectlanguage{english}to tie freshly cut rice into bundles} \textcolor{PineGreen}{\selectlanguage{french}nouer le riz coupé en bottes}  
\ding{203} \zh{包。} \textcolor{Sepia}{\selectlanguage{english}To wrap, to pack.} \textcolor{PineGreen}{\selectlanguage{french}Envelopper, emballer (monosyllabique).}  ¶ \textcolor{darkblue}{\textbf{\ipa{ʂæ˩\textasciitilde{}ʂæ˧˥}}} \zh{\mytextsc{重叠:包一包}} \textcolor{Sepia}{\selectlanguage{english}\mytextsc{red}: to wrap, to pack} \textcolor{PineGreen}{\selectlanguage{french}\mytextsc{red}: emballer, envelopper}  
 ¶ \textcolor{darkblue}{\textbf{\ipa{ʂæ˩\textasciitilde{}ʂæ˧-ze˥}}} \zh{\mytextsc{red} \mytextsc{accomp}} \textcolor{Sepia}{\selectlanguage{english}\mytextsc{red} \mytextsc{accomp}} \textcolor{PineGreen}{\selectlanguage{french}\mytextsc{red} \mytextsc{accomp}}  
 ¶ \textcolor{darkblue}{\textbf{\ipa{tso˧\textasciitilde{}tso˧ ʂæ˥\textasciitilde{}ʂæ˩}}} \zh{包一包东西} \textcolor{Sepia}{\selectlanguage{english}to wrap things} \textcolor{PineGreen}{\selectlanguage{french}emballer des choses}  

\lhead{\firstmark}
\rhead{\botmark}

\subsection{\hspace{-0.5cm} {\Large \textcolor{darkblue}{\textbf{\ipa{ʂæ˧˥\textsubscript{a}}}}}\hspace{0.5cm}[\kern2pt{\textcolor{darkblue}{\textbf{\ipa{ʂæ˧˥}}}}\kern2pt]} \hypertarget{s`\{\string_M\string_Ta1}{}
\markboth{\textcolor{darkblue}{\textbf{\ipa{ʂæ˧˥\textsubscript{a}}}}}{}
\textcolor{teal}{\zh{量词}} \hspace{4pt} \zh{声调类:} MH\textsubscript{a}.
\zh{量词:捆。} \textcolor{Sepia}{\selectlanguage{english}A sheaf of cut rice or other crop (the amount cut at one go with a sickle and immediately tied together with one sprig).} \textcolor{PineGreen}{\selectlanguage{french}Classificateur des gerbes: ce qu'on coupe en un coup de faucille et attache d'un brin.}  ¶ \textcolor{darkblue}{\textbf{\ipa{zɯ˧ | ɖɯ˧-ʂæ˧˥}}} \zh{一捆草} \textcolor{Sepia}{\selectlanguage{english}a sheaf of grass} \textcolor{PineGreen}{\selectlanguage{french}une gerbe d'herbe (nouée ensemble par un brin)}  
 ¶ \textcolor{darkblue}{\textbf{\ipa{ɕi˧ɭɯ˧ | ɖɯ˧-ʂæ˧˥}}} \zh{一捆稻谷} \textcolor{Sepia}{\selectlanguage{english}a sheaf of rice} \textcolor{PineGreen}{\selectlanguage{french}une gerbe de riz (nouée par un brin)}  

\lhead{\firstmark}
\rhead{\botmark}

\subsection{\hspace{-0.5cm} {\Large \textcolor{darkblue}{\textbf{\ipa{ʂe˥}}} \textsubscript{1}}\hspace{0.5cm}[\kern2pt{\textcolor{darkblue}{\textbf{\ipa{ʂe˥}}}}\kern2pt]} \hypertarget{s`e\string_T1}{}
\markboth{\textcolor{darkblue}{\textbf{\ipa{ʂe˥}}} \textsubscript{1}}{}
\textcolor{teal}{\zh{名词}} \hspace{4pt} \zh{声调类:} \#H.
\zh{肉,肌肉。} \textcolor{Sepia}{\selectlanguage{english}Meat, flesh.} \textcolor{PineGreen}{\selectlanguage{french}Viande, chair.} 
\lhead{\firstmark}
\rhead{\botmark}

\subsection{\hspace{-0.5cm} {\Large \textcolor{darkblue}{\textbf{\ipa{ʂe˥}}} \textsubscript{2}}\hspace{0.5cm}[\kern2pt{\textcolor{darkblue}{\textbf{\ipa{ʂe˥}}}}\kern2pt]} \hypertarget{s`e\string_T2}{}
\markboth{\textcolor{darkblue}{\textbf{\ipa{ʂe˥}}} \textsubscript{2}}{}
\textcolor{teal}{\zh{名词}} \hspace{4pt} \zh{声调类:} \#H.
\zh{未熟粮食。} \textcolor{Sepia}{\selectlanguage{english}Unripe cereals.} \textcolor{PineGreen}{\selectlanguage{french}Céréales pas encore mûres: céréales en herbe, dont on voit déjà l'épi mais dont l'épi ne s'est pas encore incliné sous le poids du grain.}  ¶ \textcolor{darkblue}{\textbf{\ipa{ʂe˧do˧˥}}} \zh{未熟粮食} \textcolor{Sepia}{\selectlanguage{english}unripe cereals} \textcolor{PineGreen}{\selectlanguage{french}même sens}  

\lhead{\firstmark}
\rhead{\botmark}

\subsection{\hspace{-0.5cm} {\Large \textcolor{darkblue}{\textbf{\ipa{ʂe˧\textsubscript{a}}}}}\hspace{0.5cm}[\kern2pt{\textcolor{darkblue}{\textbf{\ipa{ʂe˩˥}}}}\kern2pt]} \hypertarget{s`e\string_Ma1}{}
\markboth{\textcolor{darkblue}{\textbf{\ipa{ʂe˧\textsubscript{a}}}}}{}
\textcolor{teal}{\zh{动词}} \hspace{4pt} \zh{声调类:} M\textsubscript{a}.
\zh{寻找。} \textcolor{Sepia}{\selectlanguage{english}To look for, to search for; to procure, to get.} \textcolor{PineGreen}{\selectlanguage{french}Chercher; se procurer.}  ¶ \textcolor{darkblue}{\textbf{\ipa{le˧-ʂe˧ le˧-ɖɯ˧-ze˧!}}} \zh{(我)找了……就找到了! / 找到了!} \textcolor{Sepia}{\selectlanguage{english}(I) looked for something, and I found it!} \textcolor{PineGreen}{\selectlanguage{french}(j'ai) cherché et (j'ai) trouvé!}  
 ¶ \textcolor{darkblue}{\textbf{\ipa{hĩ˧ ɖɯ˧-v̩˧ ʂe˧}}} \zh{直译:‘找一个人’。实际含义:去访问异性的人(一般是男人去访问女人)} \textcolor{Sepia}{\selectlanguage{english}literally 'to look for someone'; meaning: to visit someone of the opposite sex, to frequent someone of the opposite sex (this is typically a masculine activity)} \textcolor{PineGreen}{\selectlanguage{french}littéralement 'chercher quelqu'un'; sens: fréquenter quelqu'un du sexe opposé, rendre visite à quelqu'un du sexe opposé (généralement: se dit d'un homme)}  
 ¶ \textcolor{darkblue}{\textbf{\ipa{hĩ˧ ʂe˩}}} \zh{娶媳妇} \textcolor{Sepia}{\selectlanguage{english}to take a wife, to marry a wife} \textcolor{PineGreen}{\selectlanguage{french}prendre femme, épouser une femme}  
 ¶ \textcolor{darkblue}{\textbf{\ipa{tso˧\textasciitilde{}tso˧ ʂe˩}}} \zh{找东西} \textcolor{Sepia}{\selectlanguage{english}to look for things} \textcolor{PineGreen}{\selectlanguage{french}chercher quelque chose}  
 ¶ \textcolor{darkblue}{\textbf{\ipa{lo˧ mɤ˧-dʑo˧, | lo˧ ʂe˧!}}} \zh{没事找事!} \textcolor{Sepia}{\selectlanguage{english}[(S)he] looks for complications / creates unnecessary complications! (Literally: 'to look for work when there isn't any'.)} \textcolor{PineGreen}{\selectlanguage{french}[Il/elle] se crée des complications / se donner du travail!}  
 ¶ \textcolor{darkblue}{\textbf{\ipa{le˧-ʂe˧ tʰi˧-tɕɯ˥}}} \zh{准备(做饭的材料、旅途用品……)} \textcolor{Sepia}{\selectlanguage{english}to prepare (e.g. ingredients for a recipe, luggage for travel), to get (something) ready} \textcolor{PineGreen}{\selectlanguage{french}préparer (des ingrédients pour une recette, ses bagages...)}  

\lhead{\firstmark}
\rhead{\botmark}

\subsection{\hspace{-0.5cm} {\Large \textcolor{darkblue}{\textbf{\ipa{ʂe˧bæ˧}}}}\hspace{0.5cm}[\kern2pt{\textcolor{darkblue}{\textbf{\ipa{ʂe˩bæ˩˥}}}}\kern2pt]} \hypertarget{s`e\string_Mb\{\string_M1}{}
\markboth{\textcolor{darkblue}{\textbf{\ipa{ʂe˧bæ˧}}}}{}
\textcolor{teal}{\zh{名词}} \hspace{4pt} \zh{声调类:} M.
\zh{项圈、项链,锁链。} \textcolor{Sepia}{\selectlanguage{english}Necklace; chain.} \textcolor{PineGreen}{\selectlanguage{french}Collier; chaîne.}  ¶ \textcolor{darkblue}{\textbf{\ipa{ŋv̩˩-ʂe˩bæ˥}}} \zh{银项链} \textcolor{Sepia}{\selectlanguage{english}silver necklace} \textcolor{PineGreen}{\selectlanguage{french}collier en argent}  
 ¶ \textcolor{darkblue}{\textbf{\ipa{hæ̃˩-ʂe˩bæ˥}}} \zh{金项链} \textcolor{Sepia}{\selectlanguage{english}gold necklace} \textcolor{PineGreen}{\selectlanguage{french}collier en or}  
 ¶ \textcolor{darkblue}{\textbf{\ipa{ʂe˧mo˧ʂe˧bæ˧, | kʰv̩˩mi˩ pʰæ˩˥!}}} \zh{铁链,是来用拴狗的!} \textcolor{Sepia}{\selectlanguage{english}The iron necklace is used to tie the dog!} \textcolor{PineGreen}{\selectlanguage{french}Le collier de fer, c'est pour attacher le chien!}  
 ¶ \textcolor{darkblue}{\textbf{\ipa{kʰi˧-ʂe˧bæ˥, | ʂe˧mo˧ po˧-ɳɯ˧ | gv̩˩˥!}}} \zh{铁链,是来用拴狗的!} \textcolor{Sepia}{\selectlanguage{english}The door's chain (the chain used to lock the door) is made of iron!}  
 \zh{量词}: \textcolor{darkblue}{\textbf{\ipa{kʰɯ˩}}} 
\lhead{\firstmark}
\rhead{\botmark}

\subsection{\hspace{-0.5cm} {\Large \textcolor{darkblue}{\textbf{\ipa{ʂe˧bv̩\#˥}}}}\hspace{0.5cm}[\kern2pt{\textcolor{darkblue}{\textbf{\ipa{ʂe˧bv̩˧}}}}\kern2pt]} \hypertarget{s`e\string_Mbv\string_=\#\string_T1}{}
\markboth{\textcolor{darkblue}{\textbf{\ipa{ʂe˧bv̩\#˥}}}}{}
\textcolor{teal}{\zh{名词}} \hspace{4pt} \zh{声调类:} \#H.
\zh{香肠,把瘦肉装在肠子里。} \textcolor{Sepia}{\selectlanguage{english}Sausage, dried meat preserved in intestines.} \textcolor{PineGreen}{\selectlanguage{french}Saucisse; viande séchée conservée dans les intestins.} 
\lhead{\firstmark}
\rhead{\botmark}

\subsection{\hspace{-0.5cm} {\Large \textcolor{darkblue}{\textbf{\ipa{ʂe˧di˩}}}}\hspace{0.5cm}[\kern2pt{\textcolor{darkblue}{\textbf{\ipa{ʂe˩di˩˥}}}}\kern2pt]} \hypertarget{s`e\string_Mdi\string_B1}{}
\markboth{\textcolor{darkblue}{\textbf{\ipa{ʂe˧di˩}}}}{}
\textcolor{teal}{\zh{形容词}} \hspace{4pt} \zh{声调类:} L\#.
\zh{胖。} \textcolor{Sepia}{\selectlanguage{english}Fat (person).} \textcolor{PineGreen}{\selectlanguage{french}Gros.}  ¶ \textcolor{darkblue}{\textbf{\ipa{ʂe˧ di˩-ze˩!}}} \zh{胖了!} \textcolor{Sepia}{\selectlanguage{english}(He/she) has put on weight!} \textcolor{PineGreen}{\selectlanguage{french}(il/elle) a grossi!}  

\lhead{\firstmark}
\rhead{\botmark}

\subsection{\hspace{-0.5cm} {\Large \textcolor{darkblue}{\textbf{\ipa{ʂe˧dzo\#˥}}}}\hspace{0.5cm}[\kern2pt{\textcolor{darkblue}{\textbf{\ipa{ʂe˧dzo˩}}}}\kern2pt]} \hypertarget{s`e\string_Mdzo\#\string_T1}{}
\markboth{\textcolor{darkblue}{\textbf{\ipa{ʂe˧dzo\#˥}}}}{}
\textcolor{teal}{\zh{名词}} \hspace{4pt} \zh{声调类:} \#H.
\zh{放案板的家具。} \textcolor{Sepia}{\selectlanguage{english}Cooking table: a wooden piece of furniture on which one places the chopping board.} \textcolor{PineGreen}{\selectlanguage{french}Meuble de cuisine: structure en bois sur laquelle on fait la cuisine: on y pose la planche à découper, les ustensiles….}  \zh{量词}: \textcolor{darkblue}{\textbf{\ipa{pɤ˩}}} 
\lhead{\firstmark}
\rhead{\botmark}

\subsection{\hspace{-0.5cm} {\Large \textcolor{darkblue}{\textbf{\ipa{ʂe˧kʰɯ˧}}}}\hspace{0.5cm}[\kern2pt{\textcolor{darkblue}{\textbf{\ipa{ʂe˧kʰɯ˩}}}}\kern2pt]} \hypertarget{s`e\string_Mk\string_hM\string_M1}{}
\markboth{\textcolor{darkblue}{\textbf{\ipa{ʂe˧kʰɯ˧}}}}{}
\textcolor{teal}{\zh{名词}} \hspace{4pt} \zh{声调类:} M.
\zh{三脚架。} \textcolor{Sepia}{\selectlanguage{english}Tripod.} \textcolor{PineGreen}{\selectlanguage{french}Trépied de fer (dans le foyer, sur lequel on pose une casserole, une poële, une bouilloire…).}  \zh{量词}: \textcolor{darkblue}{\textbf{\ipa{nɑ˧}}} 
\lhead{\firstmark}
\rhead{\botmark}

\subsection{\hspace{-0.5cm} {\Large \textcolor{darkblue}{\textbf{\ipa{ʂe˧mi˧}}}}\hspace{0.5cm}[\kern2pt{\textcolor{darkblue}{\textbf{\ipa{ʂe˩mi˥}}}}\kern2pt]} \hypertarget{s`e\string_Mmi\string_M1}{}
\markboth{\textcolor{darkblue}{\textbf{\ipa{ʂe˧mi˧}}}}{}
\textcolor{teal}{\zh{名词}} \hspace{4pt} \zh{声调类:} M.
\zh{虱子。} \textcolor{Sepia}{\selectlanguage{english}Louse.} \textcolor{PineGreen}{\selectlanguage{french}Pou.}  \zh{量词}: \textcolor{darkblue}{\textbf{\ipa{mi˩}}} 
\lhead{\firstmark}
\rhead{\botmark}

\subsection{\hspace{-0.5cm} {\Large \textcolor{darkblue}{\textbf{\ipa{ʂe˧mo˧}}}}\hspace{0.5cm}[\kern2pt{\textcolor{darkblue}{\textbf{\ipa{ʂe˩mo˩˥}}}}\kern2pt]} \hypertarget{s`e\string_Mmo\string_M1}{}
\markboth{\textcolor{darkblue}{\textbf{\ipa{ʂe˧mo˧}}}}{}
\textcolor{teal}{\zh{名词}} \hspace{4pt} \zh{声调类:} M.
\zh{铁(双音节)。} \textcolor{Sepia}{\selectlanguage{english}Iron (disyllable).} \textcolor{PineGreen}{\selectlanguage{french}Fer (disyllabe).} \zh{~【参考】~} \hyperlink{}{\textcolor{darkblue}{\textbf{\ipa{ʂe˩}}}} 
\lhead{\firstmark}
\rhead{\botmark}

\subsection{\hspace{-0.5cm} {\Large \textcolor{darkblue}{\textbf{\ipa{ʂe˧nɑ˩}}}}\hspace{0.5cm}[\kern2pt{\textcolor{darkblue}{\textbf{\ipa{ʂe˧nɑ˧}}}}\kern2pt]} \hypertarget{s`e\string_MnA\string_B1}{}
\markboth{\textcolor{darkblue}{\textbf{\ipa{ʂe˧nɑ˩}}}}{}
\textcolor{teal}{\zh{名词}} \hspace{4pt} \zh{声调类:} L\#.
\zh{瘦肉。} \textcolor{Sepia}{\selectlanguage{english}Lean meat.} \textcolor{PineGreen}{\selectlanguage{french}Viande maigre.} 
\lhead{\firstmark}
\rhead{\botmark}

\subsection{\hspace{-0.5cm} {\Large \textcolor{darkblue}{\textbf{\ipa{ʂe˧ɲi˩}}}}\hspace{0.5cm}[\kern2pt{\textcolor{darkblue}{\textbf{\ipa{ʂe˧ɲi˧}}}}\kern2pt]} \hypertarget{s`e\string_MJi\string_B1}{}
\markboth{\textcolor{darkblue}{\textbf{\ipa{ʂe˧ɲi˩}}}}{}
\textcolor{teal}{\zh{名词}} \hspace{4pt} \zh{声调类:} L\#.
\zh{火钳。} \textcolor{Sepia}{\selectlanguage{english}Fire tongs.} \textcolor{PineGreen}{\selectlanguage{french}Pince à braises.}  \zh{量词}: \textcolor{darkblue}{\textbf{\ipa{nɑ˧}}} 
\lhead{\firstmark}
\rhead{\botmark}

\subsection{\hspace{-0.5cm} {\Large \textcolor{darkblue}{\textbf{\ipa{ʂe˧pv̩˩}}}}\hspace{0.5cm}[\kern2pt{\textcolor{darkblue}{\textbf{\ipa{ʂe˧pv̩˧}}}}\kern2pt]} \hypertarget{s`e\string_Mpv\string_=\string_B1}{}
\markboth{\textcolor{darkblue}{\textbf{\ipa{ʂe˧pv̩˩}}}}{}
\textcolor{teal}{\zh{名词}} \hspace{4pt} \zh{声调类:} L\#.
\zh{腊肉。} \textcolor{Sepia}{\selectlanguage{english}Cured meat; bacon.} \textcolor{PineGreen}{\selectlanguage{french}Viande séchée.} 
\lhead{\firstmark}
\rhead{\botmark}

\subsection{\hspace{-0.5cm} {\Large \textcolor{darkblue}{\textbf{\ipa{ʂe˧qʰv̩˧}}}}\hspace{0.5cm}[\kern2pt{\textcolor{darkblue}{\textbf{\ipa{ʂe˧qʰv̩˧}}}}\kern2pt]} \hypertarget{s`e\string_Mq\string_hv\string_=\string_M1}{}
\markboth{\textcolor{darkblue}{\textbf{\ipa{ʂe˧qʰv̩˧}}}}{}
\textcolor{teal}{\zh{名词}} \hspace{4pt} \zh{声调类:} M.
\zh{铁钉,钉子。} \textcolor{Sepia}{\selectlanguage{english}Iron nail; nail.} \textcolor{PineGreen}{\selectlanguage{french}Clou en fer.}  ¶ \textcolor{darkblue}{\textbf{\ipa{ʂe˧qʰv̩˧ lɑ˧˥}}} \zh{钉钉子} \textcolor{Sepia}{\selectlanguage{english}to hammer in a nail, to hit a nail} \textcolor{PineGreen}{\selectlanguage{french}enfoncer un clou, planter un clou}  
 \zh{量词}: \textcolor{darkblue}{\textbf{\ipa{ɭɯ˧}}} 
\lhead{\firstmark}
\rhead{\botmark}

\subsection{\hspace{-0.5cm} {\Large \textcolor{darkblue}{\textbf{\ipa{ʂe˧sɑ˩}}}}\hspace{0.5cm}[\kern2pt{\textcolor{darkblue}{\textbf{\ipa{ʂe˧sɑ˧}}}}\kern2pt]} \hypertarget{s`e\string_MsA\string_B1}{}
\markboth{\textcolor{darkblue}{\textbf{\ipa{ʂe˧sɑ˩}}}}{}
\textcolor{teal}{\zh{名词}} \hspace{4pt} \zh{声调类:} L\#.
\zh{猪腿肉。} \textcolor{Sepia}{\selectlanguage{english}Meat of the limbs of pig. This includes the four limbs; it usually refers to preserved meat, but can also be used to refer to the limbs of the living animal.} \textcolor{PineGreen}{\selectlanguage{french}Viande des membres du cochon: les membres postérieurs aussi bien que les membres antérieurs. Le terme s'emploie pour la pièce de boucherie: de la viande conservée (séchée) avec l'os; mais le même terme peut également s'employer pour désigner les membres de la bête vivante.}  \zh{量词}: \textcolor{darkblue}{\textbf{\ipa{sɑ˧˥}}} 
\lhead{\firstmark}
\rhead{\botmark}

\subsection{\hspace{-0.5cm} {\Large \textcolor{darkblue}{\textbf{\ipa{ʂe˧-sɯ˧sɯ˥}}}}\hspace{0.5cm}[\kern2pt{\textcolor{darkblue}{\textbf{\ipa{xxxx non-correspondance entre le nombre de morphèmes et le nombre de tons de morphèmes}}}}\kern2pt]} \hypertarget{s`e\string_M-sM\string_MsM\string_T1}{}
\markboth{\textcolor{darkblue}{\textbf{\ipa{ʂe˧-sɯ˧sɯ˥}}}}{}
\textcolor{teal}{\zh{名词}} \hspace{4pt} \zh{声调类:} H\#.
\zh{生肉。} \textcolor{Sepia}{\selectlanguage{english}Raw meat.} \textcolor{PineGreen}{\selectlanguage{french}Viande crue.} 
\lhead{\firstmark}
\rhead{\botmark}

\subsection{\hspace{-0.5cm} {\Large \textcolor{darkblue}{\textbf{\ipa{ʂe˧ʂe˧}}}}\hspace{0.5cm}[\kern2pt{\textcolor{darkblue}{\textbf{\ipa{ʂe˧ʂe˩}}}}\kern2pt]} \hypertarget{s`e\string_Ms`e\string_M1}{}
\markboth{\textcolor{darkblue}{\textbf{\ipa{ʂe˧ʂe˧}}}}{}
\textcolor{teal}{\zh{动词}} \hspace{4pt} \zh{声调类:} M.
\zh{着凉。} \textcolor{Sepia}{\selectlanguage{english}To catch a cold.} \textcolor{PineGreen}{\selectlanguage{french}Prendre froid, attraper un rhume, attraper froid.}  ¶ \textcolor{darkblue}{\textbf{\ipa{ʂe˧ʂe˧-ze˩}}} \zh{着凉了} \textcolor{Sepia}{\selectlanguage{english}\mytextsc{pfv}} \textcolor{PineGreen}{\selectlanguage{french}\mytextsc{pfv}}  

\lhead{\firstmark}
\rhead{\botmark}

\subsection{\hspace{-0.5cm} {\Large \textcolor{darkblue}{\textbf{\ipa{ʂe˧ʈʂe˩}}}}\hspace{0.5cm}[\kern2pt{\textcolor{darkblue}{\textbf{\ipa{ʂe˩ʈʂe˩˥}}}}\kern2pt]} \hypertarget{s`e\string_Mt`s`e\string_B1}{}
\markboth{\textcolor{darkblue}{\textbf{\ipa{ʂe˧ʈʂe˩}}}}{}
\textcolor{teal}{\zh{名词}} \hspace{4pt} \zh{声调类:} L\#.
\zh{棉布,布料。} \textcolor{Sepia}{\selectlanguage{english}Cotton fabric, cloth.} \textcolor{PineGreen}{\selectlanguage{french}Tissu de coton.}  \zh{量词}: \textcolor{darkblue}{\textbf{\ipa{pʰæ˧˥}}} \textcolor{darkblue}{\textbf{\ipa{kʰɤ˥}}} 
\lhead{\firstmark}
\rhead{\botmark}

\subsection{\hspace{-0.5cm} {\Large \textcolor{darkblue}{\textbf{\ipa{ʂe˧ʐe\#˥}}}}\hspace{0.5cm}[\kern2pt{\textcolor{darkblue}{\textbf{\ipa{ʂe˧ʐe˩}}}}\kern2pt]} \hypertarget{s`e\string_Mz`e\#\string_T1}{}
\markboth{\textcolor{darkblue}{\textbf{\ipa{ʂe˧ʐe\#˥}}}}{}
\textcolor{teal}{\zh{名词}} \hspace{4pt} \zh{声调类:} \#H.
\zh{腊肉,包括不同几类的腊肉,如火腿等。} \textcolor{Sepia}{\selectlanguage{english}Preserved pork meat.} \textcolor{PineGreen}{\selectlanguage{french}Viande de cochon préservée. Le terme recouvre diverses pièces de boucherie, dont le jambon.}  \zh{量词}: \textcolor{darkblue}{\textbf{\ipa{ʐe˥}}} 
\lhead{\firstmark}
\rhead{\botmark}

\subsection{\hspace{-0.5cm} {\Large \textcolor{darkblue}{\textbf{\ipa{ʂe˩}}}}\hspace{0.5cm}[\kern2pt{\textcolor{darkblue}{\textbf{\ipa{ʂe˥}}}}\kern2pt]} \hypertarget{s`e\string_B1}{}
\markboth{\textcolor{darkblue}{\textbf{\ipa{ʂe˩}}}}{}
\textcolor{teal}{\zh{名词}} \hspace{4pt} \zh{声调类:} L.
\zh{铁(单音节)。} \textcolor{Sepia}{\selectlanguage{english}Iron (monosyllable).} \textcolor{PineGreen}{\selectlanguage{french}Fer (monosyllabe).} 
\lhead{\firstmark}
\rhead{\botmark}

\subsection{\hspace{-0.5cm} {\Large \textcolor{darkblue}{\textbf{\ipa{ʂe˩\textsubscript{b}}}}}\hspace{0.5cm}[\kern2pt{\textcolor{darkblue}{\textbf{\ipa{ʂe˥}}}}\kern2pt]} \hypertarget{s`e\string_Bb1}{}
\markboth{\textcolor{darkblue}{\textbf{\ipa{ʂe˩\textsubscript{b}}}}}{}
\textcolor{teal}{\zh{动词}} \hspace{4pt} \zh{声调类:} L\textsubscript{b}.
\zh{小便,尿; 屙尿; 解溲; 拉(屎)。} \textcolor{Sepia}{\selectlanguage{english}To urinate.} \textcolor{PineGreen}{\selectlanguage{french}Uriner, pisser, faire pipi; déféquer.}  ¶ \textcolor{darkblue}{\textbf{\ipa{dʑi˧ ʂe˧˥}}} \zh{屙尿} \textcolor{Sepia}{\selectlanguage{english}to urinate} \textcolor{PineGreen}{\selectlanguage{french}pisser}  
 ¶ \textcolor{darkblue}{\textbf{\ipa{qʰæ˧ ʂe˧˥}}} \zh{拉屎} \textcolor{Sepia}{\selectlanguage{english}to defecate} \textcolor{PineGreen}{\selectlanguage{french}déféquer}  
 ¶ \textcolor{darkblue}{\textbf{\ipa{le˧-ʂe˩-ze˩}}} \zh{尿了} \textcolor{Sepia}{\selectlanguage{english}\mytextsc{accomp} \string_ \mytextsc{pfv}} \textcolor{PineGreen}{\selectlanguage{french}\mytextsc{accomp} \string_ \mytextsc{pfv}}  
 ¶ \textcolor{darkblue}{\textbf{\ipa{ɖɯ˧-ʈʰɤ˧ ʂe˧˥}}} \zh{尿一滴尿} \textcolor{Sepia}{\selectlanguage{english}to urinate a drop} \textcolor{PineGreen}{\selectlanguage{french}pisser une goutte}  

\lhead{\firstmark}
\rhead{\botmark}

\subsection{\hspace{-0.5cm} {\Large \textcolor{darkblue}{\textbf{\ipa{ʂe˩lɑ˩}}}}\hspace{0.5cm}[\kern2pt{\textcolor{darkblue}{\textbf{\ipa{ʂe˧lɑ˧}}}}\kern2pt]} \hypertarget{s`e\string_BlA\string_B1}{}
\markboth{\textcolor{darkblue}{\textbf{\ipa{ʂe˩lɑ˩}}}}{}
\textcolor{teal}{\zh{名词}} \hspace{4pt} \zh{声调类:} L.
\zh{打铁。} \textcolor{Sepia}{\selectlanguage{english}To forge.} \textcolor{PineGreen}{\selectlanguage{french}Forger, battre le fer.} 
\lhead{\firstmark}
\rhead{\botmark}

\subsection{\hspace{-0.5cm} {\Large \textcolor{darkblue}{\textbf{\ipa{ʂe˩-lɑ˩-hĩ˥}}}}\hspace{0.5cm}[\kern2pt{\textcolor{darkblue}{\textbf{\ipa{xxxx non-correspondance entre le nombre de morphèmes et le nombre de tons de morphèmes}}}}\kern2pt]} \hypertarget{s`e\string_B-lA\string_B-hi\string_~\string_T1}{}
\markboth{\textcolor{darkblue}{\textbf{\ipa{ʂe˩-lɑ˩-hĩ˥}}}}{}
\textcolor{teal}{\zh{名词}} \hspace{4pt} \zh{声调类:} L+H\#.
\zh{铁匠。} \textcolor{Sepia}{\selectlanguage{english}Blacksmith.} \textcolor{PineGreen}{\selectlanguage{french}Forgeron.}  ¶ \textcolor{darkblue}{\textbf{\ipa{ʂe˩lɑ˩-hĩ˥ hĩ˩}}} \zh{铁匠} \textcolor{Sepia}{\selectlanguage{english}blacksmith} \textcolor{PineGreen}{\selectlanguage{french}forgeron}  
 \zh{量词}: \textcolor{darkblue}{\textbf{\ipa{v̩˧}}} 
\lhead{\firstmark}
\rhead{\botmark}

\subsection{\hspace{-0.5cm} {\Large \textcolor{darkblue}{\textbf{\ipa{ʂe˩mɤ˩}}}}\hspace{0.5cm}[\kern2pt{\textcolor{darkblue}{\textbf{\ipa{ʂe˩mɤ˧˥}}}}\kern2pt]} \hypertarget{s`e\string_Bm7\string_B1}{}
\markboth{\textcolor{darkblue}{\textbf{\ipa{ʂe˩mɤ˩}}}}{}
\textcolor{teal}{\zh{名词}} \hspace{4pt} \zh{声调类:} L.
\zh{肥肉。} \textcolor{Sepia}{\selectlanguage{english}Fat meat.} \textcolor{PineGreen}{\selectlanguage{french}Viande grasse.} 
\lhead{\firstmark}
\rhead{\botmark}

\subsection{\hspace{-0.5cm} {\Large \textcolor{darkblue}{\textbf{\ipa{ʂe˩-mo˧˥}}}}\hspace{0.5cm}[\kern2pt{\textcolor{darkblue}{\textbf{\ipa{xxxx non-correspondance entre le nombre de morphèmes et le nombre de tons de morphèmes}}}}\kern2pt]} \hypertarget{s`e\string_B-mo\string_M\string_T1}{}
\markboth{\textcolor{darkblue}{\textbf{\ipa{ʂe˩-mo˧˥}}}}{}
\textcolor{teal}{\zh{名词}} \hspace{4pt} \zh{声调类:} LM+MH\#.
\zh{松茸。} \textcolor{Sepia}{\selectlanguage{english}Pine mushroom, matsutake, \textit{Tricholoma matsutake}.} \textcolor{PineGreen}{\selectlanguage{french}Champignon des pins, matsutake, \textit{Tricholoma matsutake}: un champignon comestible, rare et très apprécié.} 
\lhead{\firstmark}
\rhead{\botmark}

\subsection{\hspace{-0.5cm} {\Large \textcolor{darkblue}{\textbf{\ipa{ʂe˩ʂv̩˩}}}}\hspace{0.5cm}[\kern2pt{\textcolor{darkblue}{\textbf{\ipa{ʂe˧ʂv̩˥}}}}\kern2pt]} \hypertarget{s`e\string_Bs`v\string_=\string_B1}{}
\markboth{\textcolor{darkblue}{\textbf{\ipa{ʂe˩ʂv̩˩}}}}{}
\textcolor{teal}{\zh{名词}} \hspace{4pt} \zh{声调类:} L.
\zh{虮子。} \textcolor{Sepia}{\selectlanguage{english}Nit, egg of louse.} \textcolor{PineGreen}{\selectlanguage{french}Lente, oeuf de pou.}  \zh{量词}: \textcolor{darkblue}{\textbf{\ipa{ɭɯ˧}}} 
\lhead{\firstmark}
\rhead{\botmark}

\subsection{\hspace{-0.5cm} {\Large \textcolor{darkblue}{\textbf{\ipa{ʂɤ˧do˧˥}}}}\hspace{0.5cm}[\kern2pt{\textcolor{darkblue}{\textbf{\ipa{ʂɤ˩do˩˥}}}}\kern2pt]} \hypertarget{s`7\string_Mdo\string_M\string_T1}{}
\markboth{\textcolor{darkblue}{\textbf{\ipa{ʂɤ˧do˧˥}}}}{}
\textcolor{teal}{\zh{形容词}} \hspace{4pt} \zh{声调类:} MH\#.
\ding{202} \zh{害羞。} \textcolor{Sepia}{\selectlanguage{english}Ashamed.} \textcolor{PineGreen}{\selectlanguage{french}Honteux.}  ¶ \textcolor{darkblue}{\textbf{\ipa{ʂɤ˧do˧ mɤ˧-sɯ˥!}}} \zh{不知羞耻!} \textcolor{Sepia}{\selectlanguage{english}[(S)he] is sullen / impudent / has no sense of shame} \textcolor{PineGreen}{\selectlanguage{french}(il/elle) est effronté(e), ne connaît pas la politesse, est malpoli}  
\ding{203} \zh{娴静、礼貌。} \textcolor{Sepia}{\selectlanguage{english}Modest, demure, discreet, polite.} \textcolor{PineGreen}{\selectlanguage{french}Pudique; poli.}  ¶ \textcolor{darkblue}{\textbf{\ipa{ʈʂʰɯ˧ ʂɤ˧do˧-zo˥! / ʂɤ˧do˧ ʝi˥!}}} \zh{他/她很娴静 / 很持重!} \textcolor{Sepia}{\selectlanguage{english}(S)he is very modest/discreet/polite!} \textcolor{PineGreen}{\selectlanguage{french}Cette personne a de la pudeur!}  

\lhead{\firstmark}
\rhead{\botmark}

\subsection{\hspace{-0.5cm} {\Large \textcolor{darkblue}{\textbf{\ipa{ʂɤ˧ɲi\#˥}}}}\hspace{0.5cm}[\kern2pt{\textcolor{darkblue}{\textbf{\ipa{ʂɤ˧ɲi˧}}}}\kern2pt]} \hypertarget{s`7\string_MJi\#\string_T1}{}
\markboth{\textcolor{darkblue}{\textbf{\ipa{ʂɤ˧ɲi\#˥}}}}{}
\textcolor{teal}{\zh{名词}} \hspace{4pt} \zh{声调类:} \#H.
\zh{建议、意见。} \textcolor{Sepia}{\selectlanguage{english}Advice, suggestion, recommendation.} \textcolor{PineGreen}{\selectlanguage{french}Conseil, avis.}  ¶ \textcolor{darkblue}{\textbf{\ipa{ʂɤ˧ɲi˧ ʑi˧˥}}} \zh{请求意见,求教} \textcolor{Sepia}{\selectlanguage{english}to ask for advice / to ask for counsel} \textcolor{PineGreen}{\selectlanguage{french}demander un conseil / prendre le conseil (de quelqu'un)}  
 ¶ \textcolor{darkblue}{\textbf{\ipa{no˧ | hĩ˧-ki˧ | ʂɤ˧ɲi˧ mɤ˧-ʑi˧-zo˥!}}} \zh{你不要问人家的意见!} \textcolor{Sepia}{\selectlanguage{english}You shouldn't ask around for advice! / There is no need for you to ask for anyone's advice! (=You can make a decision by yourself.)} \textcolor{PineGreen}{\selectlanguage{french}Tu n'as pas à prendre son conseil! / Tu n'as pas à prendre le conseil d'autrui [à ce sujet: à toi de décider]!}  
 ¶ \textcolor{darkblue}{\textbf{\ipa{ə˧tse˧ʝi˧-zo˥ | ʂɤ˧ɲi˧ ʑi˧-tso˧-ɲi˥?}}} \zh{你为什么要问(他的)意见!} \textcolor{Sepia}{\selectlanguage{english}Why would you want to ask for (his/her) advice?} \textcolor{PineGreen}{\selectlanguage{french}Pourquoi donc lui demandes-tu conseil?}  
 \zh{量词}: \textcolor{darkblue}{\textbf{\ipa{kʰwɤ˥}}} 
\lhead{\firstmark}
\rhead{\botmark}

\subsection{\hspace{-0.5cm} {\Large \textcolor{darkblue}{\textbf{\ipa{ʂɤ˩\textsubscript{a}}}}}\hspace{0.5cm}[\kern2pt{\textcolor{darkblue}{\textbf{\ipa{ʂɤ˩˥}}}}\kern2pt]} \hypertarget{s`7\string_Ba1}{}
\markboth{\textcolor{darkblue}{\textbf{\ipa{ʂɤ˩\textsubscript{a}}}}}{}
\textcolor{teal}{\zh{动词}} \hspace{4pt} \zh{声调类:} L\textsubscript{a}.
\zh{撕(纸……)。} \textcolor{Sepia}{\selectlanguage{english}To tear, to rip.} \textcolor{PineGreen}{\selectlanguage{french}Déchirer (ex.: du papier).}  ¶ \textcolor{darkblue}{\textbf{\ipa{tso˧\textasciitilde{}tso˧ ʂɤ˥}}} \zh{撕东西} \textcolor{Sepia}{\selectlanguage{english}to tear things} \textcolor{PineGreen}{\selectlanguage{french}déchirer des choses}  
 ¶ \textcolor{darkblue}{\textbf{\ipa{tso˧\textasciitilde{}tso˧ ʂɤ˧\textasciitilde{}ʂɤ˥ (+ze˩)}}} \zh{撕东西} \textcolor{Sepia}{\selectlanguage{english}to tear things} \textcolor{PineGreen}{\selectlanguage{french}déchirer des choses}  
 ¶ \textcolor{darkblue}{\textbf{\ipa{le˧-ʂɤ˧\textasciitilde{}ʂɤ˥+ze˩}}} \zh{撕了} \textcolor{Sepia}{\selectlanguage{english}\mytextsc{accomp} \string_ \mytextsc{red} \mytextsc{pfv}} \textcolor{PineGreen}{\selectlanguage{french}\mytextsc{accomp} \string_ \mytextsc{red} \mytextsc{pfv}}  

\lhead{\firstmark}
\rhead{\botmark}

\subsection{\hspace{-0.5cm} {\Large \textcolor{darkblue}{\textbf{\ipa{ʂɤ˩ŋɤ\#˥}}}}\hspace{0.5cm}[\kern2pt{\textcolor{darkblue}{\textbf{\ipa{ʂɤ˩ŋɤ˥}}}}\kern2pt]} \hypertarget{s`7\string_BN7\#\string_T1}{}
\markboth{\textcolor{darkblue}{\textbf{\ipa{ʂɤ˩ŋɤ\#˥}}}}{}
\textcolor{teal}{\zh{名词}} \hspace{4pt} \zh{声调类:} LM+\#H.
\zh{锣。} \textcolor{Sepia}{\selectlanguage{english}Gong.} \textcolor{PineGreen}{\selectlanguage{french}Gong.}  ¶ \textcolor{darkblue}{\textbf{\ipa{ʂɤ˩ŋɤ˧ lɑ˩}}} \zh{打锣} \textcolor{Sepia}{\selectlanguage{english}to play the gong} \textcolor{PineGreen}{\selectlanguage{french}jouer du gong}  
 \zh{量词}: \textcolor{darkblue}{\textbf{\ipa{ɭɯ˧}}} 
\lhead{\firstmark}
\rhead{\botmark}

\subsection{\hspace{-0.5cm} {\Large \textcolor{darkblue}{\textbf{\ipa{ʂo˥}}}}\hspace{0.5cm}[\kern2pt{\textcolor{darkblue}{\textbf{\ipa{ʂo˥}}}}\kern2pt]} \hypertarget{s`o\string_T1}{}
\markboth{\textcolor{darkblue}{\textbf{\ipa{ʂo˥}}}}{}
\textcolor{teal}{\zh{动词}} \hspace{4pt} \zh{声调类:} H.
\zh{收割。} \textcolor{Sepia}{\selectlanguage{english}To reap, to gather in.} \textcolor{PineGreen}{\selectlanguage{french}Récolter.}  ¶ \textcolor{darkblue}{\textbf{\ipa{le˧-ʂo˥-ze˩}}} \zh{收割了} \textcolor{Sepia}{\selectlanguage{english}\mytextsc{accomp} \string_ \mytextsc{pfv}} \textcolor{PineGreen}{\selectlanguage{french}\mytextsc{accomp} \string_ \mytextsc{pfv}}  
 ¶ \textcolor{darkblue}{\textbf{\ipa{ɖɯ˧-kʰv̩˥ ɖɯ˧-ʂɯ˩ | gɤ˩-ʂo˥-ze˩!}}} \zh{每年收一次稻谷!} \textcolor{Sepia}{\selectlanguage{english}We have one harvest (of rice) every year!} \textcolor{PineGreen}{\selectlanguage{french}on récolte (le riz) une fois par an!}  

\lhead{\firstmark}
\rhead{\botmark}

\subsection{\hspace{-0.5cm} {\Large \textcolor{darkblue}{\textbf{\ipa{ʂo˧}}}}\hspace{0.5cm}[\kern2pt{\textcolor{darkblue}{\textbf{\ipa{ʂo˥}}}}\kern2pt]} \hypertarget{s`o\string_M1}{}
\markboth{\textcolor{darkblue}{\textbf{\ipa{ʂo˧}}}}{}
\textcolor{teal}{\zh{动词}} \hspace{4pt} \zh{声调类:} M.
\zh{收集。} \textcolor{Sepia}{\selectlanguage{english}To gather.} \textcolor{PineGreen}{\selectlanguage{french}Rassembler, assembler, accumuler.}  ¶ \textcolor{darkblue}{\textbf{\ipa{le˧-ʂo˧\textasciitilde{}ʂo˧}}} \zh{\mytextsc{accomp} \string_ \mytextsc{red}} \textcolor{Sepia}{\selectlanguage{english}\mytextsc{accomp} \string_ \mytextsc{red}} \textcolor{PineGreen}{\selectlanguage{french}\mytextsc{accomp} \string_ \mytextsc{red}}  
 ¶ \textcolor{darkblue}{\textbf{\ipa{ʂo˧\textasciitilde{}ʂo˧-zo˧-ho˩-ze˩}}} \zh{该收集一些了。} \textcolor{Sepia}{\selectlanguage{english}We are going to have to gather (things)} \textcolor{PineGreen}{\selectlanguage{french}Il va falloir rassembler/assembler.}  

\lhead{\firstmark}
\rhead{\botmark}

\subsection{\hspace{-0.5cm} {\Large \textcolor{darkblue}{\textbf{\ipa{ʂo˧}}}}\hspace{0.5cm}[\kern2pt{\textcolor{darkblue}{\textbf{\ipa{ʂo˥}}}}\kern2pt]} \hypertarget{s`o\string_M1}{}
\markboth{\textcolor{darkblue}{\textbf{\ipa{ʂo˧}}}}{}
\textcolor{teal}{\zh{感叹词}} \hspace{4pt} \zh{声调类:} M.
\zh{赶猪用的叹词:走!走!。} \textcolor{Sepia}{\selectlanguage{english}Interjection to get pigs to move forward.} \textcolor{PineGreen}{\selectlanguage{french}Interjection employée pour faire avancer les cochons, lorsqu'on les guide sur le chemin du pâturage: “Zou! / Allez!”.}  ¶ \textcolor{darkblue}{\textbf{\ipa{ʂo˧! / ʂo˧bɤ˩!}}} \zh{赶猪用的叹词} \textcolor{Sepia}{\selectlanguage{english}interjection to get pigs to move forward} \textcolor{PineGreen}{\selectlanguage{french}Interjection employée pour faire avancer les cochons, lorsqu'on les guide sur le chemin du pâturage: “Zou! / Allez!”}  

\lhead{\firstmark}
\rhead{\botmark}

\subsection{\hspace{-0.5cm} {\Large \textcolor{darkblue}{\textbf{\ipa{ʂo˩\textsubscript{a}}}}}\hspace{0.5cm}[\kern2pt{\textcolor{darkblue}{\textbf{\ipa{ʂo˥}}}}\kern2pt]} \hypertarget{s`o\string_Ba1}{}
\markboth{\textcolor{darkblue}{\textbf{\ipa{ʂo˩\textsubscript{a}}}}}{}
\textcolor{teal}{\zh{形容词}} \hspace{4pt} \zh{声调类:} L\textsubscript{a}.
\zh{干净、整洁,本质干净,清(水)。} \textcolor{Sepia}{\selectlanguage{english}Clean; clear (water).} \textcolor{PineGreen}{\selectlanguage{french}Propre (sens propre ou figuré); claire (eau).}  ¶ \textcolor{darkblue}{\textbf{\ipa{ʂo˩-hĩ˩˥}}} \zh{干净的} \textcolor{Sepia}{\selectlanguage{english}\mytextsc{nmlz}} \textcolor{PineGreen}{\selectlanguage{french}\mytextsc{nmlz}}  
 ¶ \textcolor{darkblue}{\textbf{\ipa{mɤ˧-ʂo˩}}} \zh{不干净、脏} \textcolor{Sepia}{\selectlanguage{english}not clean} \textcolor{PineGreen}{\selectlanguage{french}sale, malpropre}  
 ¶ \textcolor{darkblue}{\textbf{\ipa{ʈʂʰɯ˧ | ʂo˩-hĩ˩ ɲi˥. |}}} \zh{这是干净的。} \textcolor{Sepia}{\selectlanguage{english}This is clean.} \textcolor{PineGreen}{\selectlanguage{french}C'est propre.}  
 ¶ \textcolor{darkblue}{\textbf{\ipa{dʑɯ˧ ʂo˧}}} \zh{清水、干净的水} \textcolor{Sepia}{\selectlanguage{english}clean water, clear water} \textcolor{PineGreen}{\selectlanguage{french}de l'eau claire, de l'eau propre}  

\lhead{\firstmark}
\rhead{\botmark}

\subsection{\hspace{-0.5cm} {\Large \textcolor{darkblue}{\textbf{\ipa{ʂo˩qæ˩}}}}\hspace{0.5cm}[\kern2pt{\textcolor{darkblue}{\textbf{\ipa{ʂo˩qæ˩˥}}}}\kern2pt]} \hypertarget{s`o\string_Bq\{\string_B1}{}
\markboth{\textcolor{darkblue}{\textbf{\ipa{ʂo˩qæ˩}}}}{}
\textcolor{teal}{\zh{形容词}} \hspace{4pt} \zh{声调类:} L.
\zh{很干净。} \textcolor{Sepia}{\selectlanguage{english}Very clean.} \textcolor{PineGreen}{\selectlanguage{french}Tout propre.}  ¶ \textcolor{darkblue}{\textbf{\ipa{ʂo˩qæ˩˥ | -gv̩˩}}} \zh{很干净} \textcolor{Sepia}{\selectlanguage{english}very clean} \textcolor{PineGreen}{\selectlanguage{french}tout propre}  
 ¶ \textcolor{darkblue}{\textbf{\ipa{ɑ˩ʁo˧ | le˧-gv̩˧\textasciitilde{}gv̩˥ | ʂo˩qæ˩˥ | -gv̩˩}}} \zh{家收拾得干干净净} \textcolor{Sepia}{\selectlanguage{english}to put the house in order, that it be very clean} \textcolor{PineGreen}{\selectlanguage{french}ranger la maison, qu'elle soit bien propre}  

\lhead{\firstmark}
\rhead{\botmark}

\subsection{\hspace{-0.5cm} {\Large \textcolor{darkblue}{\textbf{\ipa{ʂo˧˥}}} \textsubscript{1}}\hspace{0.5cm}[\kern2pt{\textcolor{darkblue}{\textbf{\ipa{ʂo˧˥}}}}\kern2pt]} \hypertarget{s`o\string_M\string_T1}{}
\markboth{\textcolor{darkblue}{\textbf{\ipa{ʂo˧˥}}} \textsubscript{1}}{}
\textcolor{teal}{\zh{动词}} \hspace{4pt} \zh{声调类:} MH.
\zh{滑,光滑(路……)。} \textcolor{Sepia}{\selectlanguage{english}To slip, to slide.} \textcolor{PineGreen}{\selectlanguage{french}Glisser.}  ¶ \textcolor{darkblue}{\textbf{\ipa{mv̩˩tɕo˧ ʂo˧˥}}} \zh{滑下、滑倒} \textcolor{Sepia}{\selectlanguage{english}to slide down, to slip to the floor} \textcolor{PineGreen}{\selectlanguage{french}glisser vers le bas, glisser par terre}  
 ¶ \textcolor{darkblue}{\textbf{\ipa{ʈʂʰɯ˧ | le˧-ʂo˧˥, | tʰi˧-ʈwæ˧-ze˥}}} \zh{他滑了一跤} \textcolor{Sepia}{\selectlanguage{english}(S)he slipped and fell down} \textcolor{PineGreen}{\selectlanguage{french}il a glissé et il est tombé}  
 ¶ \textcolor{darkblue}{\textbf{\ipa{ʂo˩\textasciitilde{}ʂo˧˥}}} \zh{\mytextsc{重叠}} \textcolor{Sepia}{\selectlanguage{english}\mytextsc{red}} \textcolor{PineGreen}{\selectlanguage{french}\mytextsc{red}}  
 ¶ \textcolor{darkblue}{\textbf{\ipa{ɖæ˩ʂo˩˥ / ɖæ˩ʂo˩-ze˥}}} \zh{往下滑} \textcolor{Sepia}{\selectlanguage{english}to slide down} \textcolor{PineGreen}{\selectlanguage{french}glisser; dévaler une pente en glissant}  
 ¶ \textcolor{darkblue}{\textbf{\ipa{no˧ | ɖæ˩ʂo˩\textasciitilde{}ɖæ˥ʂo˩! |}}} \zh{你真滑头!} \textcolor{Sepia}{\selectlanguage{english}You are really cunning! (A criticism of someone who is not direct, not honest, who does not have a proper attitude: giving a slimy impression.)} \textcolor{PineGreen}{\selectlanguage{french}Tu es bien malhonnête! (Critique de quelqu'un qui n'est pas franc et direct, qui est faux jeton, qui n'a pas une bonne attitude, donnant une impression huileuse: qui s'esquive et se dérobe, comme un objet glissant qui se dérobe à la prise.)}  
\zh{~【参考】~} \hyperlink{}{\textcolor{darkblue}{\textbf{\ipa{ʂo˧˥}}} \textsubscript{2}} 
\lhead{\firstmark}
\rhead{\botmark}

\subsection{\hspace{-0.5cm} {\Large \textcolor{darkblue}{\textbf{\ipa{ʂo˧˥}}} \textsubscript{2}}\hspace{0.5cm}[\kern2pt{\textcolor{darkblue}{\textbf{\ipa{ʂo˧˥}}}}\kern2pt]} \hypertarget{s`o\string_M\string_T2}{}
\markboth{\textcolor{darkblue}{\textbf{\ipa{ʂo˧˥}}} \textsubscript{2}}{}
\textcolor{teal}{\zh{形容词}} \hspace{4pt} \zh{声调类:} MH.
\zh{光滑(路……)。} \textcolor{Sepia}{\selectlanguage{english}Slippery.} \textcolor{PineGreen}{\selectlanguage{french}Lisse, glissant.}  ¶ \textcolor{darkblue}{\textbf{\ipa{mɤ˩ ʂo˩-ʂo˥ |}}} \zh{油腻腻、滑腻} \textcolor{Sepia}{\selectlanguage{english}slippery with grease} \textcolor{PineGreen}{\selectlanguage{french}huileux, tout poisseux de graisse}  
 ¶ \textcolor{darkblue}{\textbf{\ipa{ɲi˧to˧ ɖɯ˧-ɭɯ˧ | dze˧-ʂo˧\textasciitilde{}ʂo˥}}} \zh{他嘴巴被糖粘得黏黏的} \textcolor{Sepia}{\selectlanguage{english}(her/his) whole mouth was slippery with sugar} \textcolor{PineGreen}{\selectlanguage{french}toute (sa) bouche est/était pleine de sucre / toute poisseuse à force de sucre}  
\zh{~【参考】~} \hyperlink{}{\textcolor{darkblue}{\textbf{\ipa{ʂo˧˥}}} \textsubscript{1}} 
\lhead{\firstmark}
\rhead{\botmark}

\subsection{\hspace{-0.5cm} {\Large \textcolor{darkblue}{\textbf{\ipa{ʂɻ̍˧˥}}}}\hspace{0.5cm}[\kern2pt{\textcolor{darkblue}{\textbf{\ipa{ʂɻ̍˧˥}}}}\kern2pt]} \hypertarget{s`r£`̍\string_M\string_T1}{}
\markboth{\textcolor{darkblue}{\textbf{\ipa{ʂɻ̍˧˥}}}}{}
\textcolor{teal}{\zh{形容词}} \hspace{4pt} \zh{声调类:} MH.
\zh{满。} \textcolor{Sepia}{\selectlanguage{english}Full.} \textcolor{PineGreen}{\selectlanguage{french}Rempli, plein.}  ¶ \textcolor{darkblue}{\textbf{\ipa{le˧-ʂɻ̍˧-ze˥}}} \zh{满了} \textcolor{Sepia}{\selectlanguage{english}\mytextsc{accomp} \string_ \mytextsc{pfv}} \textcolor{PineGreen}{\selectlanguage{french}\mytextsc{accomp} \string_ \mytextsc{pfv}}  

\lhead{\firstmark}
\rhead{\botmark}

\subsection{\hspace{-0.5cm} {\Large \textcolor{darkblue}{\textbf{\ipa{ʂɯ˧}}}}\hspace{0.5cm}[\kern2pt{\textcolor{darkblue}{\textbf{\ipa{ʂɯ˥}}}}\kern2pt]} \hypertarget{s`M\string_M1}{}
\markboth{\textcolor{darkblue}{\textbf{\ipa{ʂɯ˧}}}}{}
\textcolor{teal}{\zh{数词}} \hspace{4pt} \zh{声调类:} M? H\#? (pas L).
\zh{7。} \textcolor{Sepia}{\selectlanguage{english}7.} \textcolor{PineGreen}{\selectlanguage{french}7.} 
\lhead{\firstmark}
\rhead{\botmark}

\subsection{\hspace{-0.5cm} {\Large \textcolor{darkblue}{\textbf{\ipa{ʂɯ˧\textsubscript{a}}}} \textsubscript{1}}\hspace{0.5cm}[\kern2pt{\textcolor{darkblue}{\textbf{\ipa{ʂɯ˥}}}}\kern2pt]} \hypertarget{s`M\string_Ma1}{}
\markboth{\textcolor{darkblue}{\textbf{\ipa{ʂɯ˧\textsubscript{a}}}} \textsubscript{1}}{}
\textcolor{teal}{\zh{动词}} \hspace{4pt} \zh{声调类:} M\textsubscript{a}.
\zh{漏。} \textcolor{Sepia}{\selectlanguage{english}To leak.} \textcolor{PineGreen}{\selectlanguage{french}Fuir, s'écouler, se répandre, se vider.}  ¶ \textcolor{darkblue}{\textbf{\ipa{tʰi˧-ʂɯ˥\textasciitilde{}ʂɯ˩(-ze˩)}}} \zh{漏了!} \textcolor{Sepia}{\selectlanguage{english}it is leaking} \textcolor{PineGreen}{\selectlanguage{french}Ca fuit / ça se vide!}  
 ¶ \textcolor{darkblue}{\textbf{\ipa{mɤ˧-ʂɯ˥\textasciitilde{}ʂɯ˩! | mɤ˧-ʑi˧!}}} \zh{没漏,没流出去!} \textcolor{Sepia}{\selectlanguage{english}It does not leak; it does not flow out!} \textcolor{PineGreen}{\selectlanguage{french}Ca ne s'écoule pas, ça ne fuit pas! (ʑi˧: “couler”)}  

\lhead{\firstmark}
\rhead{\botmark}

\subsection{\hspace{-0.5cm} {\Large \textcolor{darkblue}{\textbf{\ipa{ʂɯ˧\textsubscript{a}}}} \textsubscript{2}}\hspace{0.5cm}[\kern2pt{\textcolor{darkblue}{\textbf{\ipa{ʂɯ˥}}}}\kern2pt]} \hypertarget{s`M\string_Ma2}{}
\markboth{\textcolor{darkblue}{\textbf{\ipa{ʂɯ˧\textsubscript{a}}}} \textsubscript{2}}{}
\textcolor{teal}{\zh{动词}} \hspace{4pt} \zh{声调类:} M\textsubscript{a}.
\zh{死。} \textcolor{Sepia}{\selectlanguage{english}To die.} \textcolor{PineGreen}{\selectlanguage{french}Mourir, décéder.}  ¶ \textcolor{darkblue}{\textbf{\ipa{le˧-ʂɯ˧-ho˩-ze˩}}} \zh{快要死了!(病了的植物、动物)} \textcolor{Sepia}{\selectlanguage{english}It's going to die! (About a sick plant or animal)} \textcolor{PineGreen}{\selectlanguage{french}ça va mourir! (au sujet d'une plante ou d'un animal malade)}  
 ¶ \textcolor{darkblue}{\textbf{\ipa{mɤ˧-ʂɯ˧-sɯ˩!}}} \zh{还没死!} \textcolor{Sepia}{\selectlanguage{english}(He/she/it) is not dead yet!} \textcolor{PineGreen}{\selectlanguage{french}(Il/elle/ce) n'est pas encore mort!}  
 ¶ \textcolor{darkblue}{\textbf{\ipa{no˧ | le˧-ʂɯ˧-bi˧-tsæ˧-ɲi˧-ze˩!}}} \zh{你去死吧!} \textcolor{Sepia}{\selectlanguage{english}Go and die! / May you die! (Imprecation)} \textcolor{PineGreen}{\selectlanguage{french}Crève donc! / Crève, charogne! (imprécation/malédiction, qu'on lance sous le coup de la colère)}  

\lhead{\firstmark}
\rhead{\botmark}

\subsection{\hspace{-0.5cm} {\Large \textcolor{darkblue}{\textbf{\ipa{ʂɯ˧dʑi˧}}}}\hspace{0.5cm}[\kern2pt{\textcolor{darkblue}{\textbf{\ipa{ʂɯ˩dʑi˩˥}}}}\kern2pt]} \hypertarget{s`M\string_Mdz£i\string_M1}{}
\markboth{\textcolor{darkblue}{\textbf{\ipa{ʂɯ˧dʑi˧}}}}{}
\textcolor{teal}{\zh{名词}} \hspace{4pt} \zh{声调类:} M.
\zh{寿衣。} \textcolor{Sepia}{\selectlanguage{english}Shroud, burial suit.} \textcolor{PineGreen}{\selectlanguage{french}Linceul, suaire, vêtement mortuaire (de: “mourir” et “habit”).}  ¶ \textcolor{darkblue}{\textbf{\ipa{ʂɯ˧dʑi˧ ʐv̩˥}}} \zh{缝寿衣} \textcolor{Sepia}{\selectlanguage{english}to sew the burial suit, to sew the shroud} \textcolor{PineGreen}{\selectlanguage{french}coudre les vêtements mortuaires, coudre le linceul}  

\lhead{\firstmark}
\rhead{\botmark}

\subsection{\hspace{-0.5cm} {\Large \textcolor{darkblue}{\textbf{\ipa{ʂɯ˧-ɬi˧mi˧}}}}\hspace{0.5cm}[\kern2pt{\textcolor{darkblue}{\textbf{\ipa{xxxx non-correspondance entre le nombre de morphèmes et le nombre de tons de morphèmes}}}}\kern2pt]} \hypertarget{s`M\string_M-Ki\string_Mmi\string_M1}{}
\markboth{\textcolor{darkblue}{\textbf{\ipa{ʂɯ˧-ɬi˧mi˧}}}}{}
\textcolor{teal}{\zh{名词}} \hspace{4pt} \zh{声调类:} M.
\zh{七月。} \textcolor{Sepia}{\selectlanguage{english}7th month.} \textcolor{PineGreen}{\selectlanguage{french}7e mois.} 
\lhead{\firstmark}
\rhead{\botmark}

\subsection{\hspace{-0.5cm} {\Large \textcolor{darkblue}{\textbf{\ipa{ʂɯ˧ɲi˥}}}}\hspace{0.5cm}[\kern2pt{\textcolor{darkblue}{\textbf{\ipa{ʂɯ˧ɲi˩}}}}\kern2pt]} \hypertarget{s`M\string_MJi\string_T1}{}
\markboth{\textcolor{darkblue}{\textbf{\ipa{ʂɯ˧ɲi˥}}}}{}
\textcolor{teal}{\zh{助词}} \hspace{4pt} \zh{声调类:} H\#.
\zh{前天。} \textcolor{Sepia}{\selectlanguage{english}The day before yesterday.} \textcolor{PineGreen}{\selectlanguage{french}Avant-hier.}  ¶ \textcolor{darkblue}{\textbf{\ipa{ʂɯ˧ɲi˥ | -ɖɯ˧ɲi˥}}} \zh{前天} \textcolor{Sepia}{\selectlanguage{english}the day before yesterday} \textcolor{PineGreen}{\selectlanguage{french}avant-hier}  

\lhead{\firstmark}
\rhead{\botmark}

\subsection{\hspace{-0.5cm} {\Large \textcolor{darkblue}{\textbf{\ipa{ʂɯ˧ʂɯ˧-dzi˩}}}}\hspace{0.5cm}[\kern2pt{\textcolor{darkblue}{\textbf{\ipa{ʂɯ˧ʂɯ˧dzi˧}}}}\kern2pt]} \hypertarget{s`M\string_Ms`M\string_M-dzi\string_B1}{}
\markboth{\textcolor{darkblue}{\textbf{\ipa{ʂɯ˧ʂɯ˧-dzi˩}}}}{}
\textcolor{teal}{\zh{名词}} \hspace{4pt} \zh{声调类:} -L.
\zh{三颗针。} \textcolor{Sepia}{\selectlanguage{english}Yyyy.} \textcolor{PineGreen}{\selectlanguage{french}Yyyy.}  \zh{量词}: \textcolor{darkblue}{\textbf{\ipa{dzi˩}}} 
\lhead{\firstmark}
\rhead{\botmark}

\subsection{\hspace{-0.5cm} {\Large \textcolor{darkblue}{\textbf{\ipa{ʂɯ˧tɤ˧ɻ\#˥}}}}\hspace{0.5cm}[\kern2pt{\textcolor{darkblue}{\textbf{\ipa{ʂɯ˧tɤ˧ɻ˧}}}}\kern2pt]} \hypertarget{s`M\string_Mt7\string_Mr£`\#\string_T1}{}
\markboth{\textcolor{darkblue}{\textbf{\ipa{ʂɯ˧tɤ˧ɻ\#˥}}}}{}
\textcolor{teal}{\zh{形容词}} \hspace{4pt} \zh{声调类:} \#H.
\zh{平滑。} \textcolor{Sepia}{\selectlanguage{english}Smooth (e.g. carefully planed wood).} \textcolor{PineGreen}{\selectlanguage{french}Lisse; par exemple: un pilier en bois, qui devient bien lisse par le travail du menuisier.}  ¶ \textcolor{darkblue}{\textbf{\ipa{ʂɯ˧tɤ˧ɻ̍˧-zo˥}}} \zh{很平滑} \textcolor{Sepia}{\selectlanguage{english}very smooth} \textcolor{PineGreen}{\selectlanguage{french}bien lisse}  
 ¶ \textcolor{darkblue}{\textbf{\ipa{ʂɯ˧tɤ˧ɻ̍˧ gv̩˧-ze˩}}} \zh{弄得平滑了} \textcolor{Sepia}{\selectlanguage{english}(it) was made nice and smooth} \textcolor{PineGreen}{\selectlanguage{french}on l'a bien lissé, on l'a rendu bien lisse}  

\lhead{\firstmark}
\rhead{\botmark}

\subsection{\hspace{-0.5cm} {\Large \textcolor{darkblue}{\textbf{\ipa{ʂɯ˧tsʰi˩}}}}\hspace{0.5cm}[\kern2pt{\textcolor{darkblue}{\textbf{\ipa{ʂɯ˧tsʰi˩}}}}\kern2pt]} \hypertarget{s`M\string_Mts\string_hi\string_B1}{}
\markboth{\textcolor{darkblue}{\textbf{\ipa{ʂɯ˧tsʰi˩}}}}{}
\textcolor{teal}{\zh{数词}} \hspace{4pt} \zh{声调类:} L\#.
\zh{70。} \textcolor{Sepia}{\selectlanguage{english}70.} \textcolor{PineGreen}{\selectlanguage{french}70.} 
\lhead{\firstmark}
\rhead{\botmark}

\subsection{\hspace{-0.5cm} {\Large \textcolor{darkblue}{\textbf{\ipa{ʂɯ˩\textsubscript{b}}}}}\hspace{0.5cm}[\kern2pt{\textcolor{darkblue}{\textbf{\ipa{ʂɯ˥}}}}\kern2pt]} \hypertarget{s`M\string_Bb1}{}
\markboth{\textcolor{darkblue}{\textbf{\ipa{ʂɯ˩\textsubscript{b}}}}}{}
\textcolor{teal}{\zh{量词}} \hspace{4pt} \zh{声调类:} L\textsubscript{b}.
\zh{量词:次数。} \textcolor{Sepia}{\selectlanguage{english}Times (repeating an action: doing something n times).} \textcolor{PineGreen}{\selectlanguage{french}Classificateur des fois (répétitions d'une action).} 
\lhead{\firstmark}
\rhead{\botmark}

\subsection{\hspace{-0.5cm} {\Large \textcolor{darkblue}{\textbf{\ipa{ʂɯ˩ʝi\#˥}}}}\hspace{0.5cm}[\kern2pt{\textcolor{darkblue}{\textbf{\ipa{ʂɯ˧ʝi˧}}}}\kern2pt]} \hypertarget{s`M\string_Bj££i\#\string_T1}{}
\markboth{\textcolor{darkblue}{\textbf{\ipa{ʂɯ˩ʝi\#˥}}}}{}
\textcolor{teal}{\zh{助词}} \hspace{4pt} \zh{声调类:} LM+\#H.
\zh{前年。} \textcolor{Sepia}{\selectlanguage{english}Two years ago.} \textcolor{PineGreen}{\selectlanguage{french}Il y a deux ans.}  ¶ \textcolor{darkblue}{\textbf{\ipa{ʂɯ˩ʝi˥ | ɖɯ˧-kʰv̩˧˥}}} \zh{前年} \textcolor{Sepia}{\selectlanguage{english}two years ago} \textcolor{PineGreen}{\selectlanguage{french}il y a deux ans, l'année il y a deux ans}  

\lhead{\firstmark}
\rhead{\botmark}

\subsection{\hspace{-0.5cm} {\Large \textcolor{darkblue}{\textbf{\ipa{ʂɯ˩kwæ˩ɻæ˥}}}}\hspace{0.5cm}[\kern2pt{\textcolor{darkblue}{\textbf{\ipa{ʂɯ˧kwæ˧ɻæ˩}}}}\kern2pt]} \hypertarget{s`M\string_Bkw\{\string_Br£`\{\string_T1}{}
\markboth{\textcolor{darkblue}{\textbf{\ipa{ʂɯ˩kwæ˩ɻæ˥}}}}{}
\textcolor{teal}{\zh{形容词}} \hspace{4pt} \zh{声调类:} L+H\#.
\zh{黄。} \textcolor{Sepia}{\selectlanguage{english}Yellow.} \textcolor{PineGreen}{\selectlanguage{french}Jaune.}  ¶ \textcolor{darkblue}{\textbf{\ipa{ʂɯ˩kwæ˩ɻæ˥-hĩ˩ gv̩˩-ze˩}}} \zh{[书]变黄了!} \textcolor{Sepia}{\selectlanguage{english}[the book] has turned yellow!} \textcolor{PineGreen}{\selectlanguage{french}[le livre] a jauni!}  
 ¶ \textcolor{darkblue}{\textbf{\ipa{[F5] ʂɯ˩kwæ˩˥ | ʂɯ˩kwæ˩˥ | gv̩˩}}} \zh{深黄} \textcolor{Sepia}{\selectlanguage{english}very yellow} \textcolor{PineGreen}{\selectlanguage{french}tout jaune}  

\lhead{\firstmark}
\rhead{\botmark}

\subsection{\hspace{-0.5cm} {\Large \textcolor{darkblue}{\textbf{\ipa{ʂɯ˩tsɯ˧}}}}\hspace{0.5cm}[\kern2pt{\textcolor{darkblue}{\textbf{\ipa{ʂɯ˩tsɯ˥}}}}\kern2pt]} \hypertarget{s`M\string_BtsM\string_M1}{}
\markboth{\textcolor{darkblue}{\textbf{\ipa{ʂɯ˩tsɯ˧}}}}{}
\textcolor{teal}{\zh{名词}} \hspace{4pt} \zh{声调类:} LM.
\zh{柿子(汉语借词)。} \textcolor{Sepia}{\selectlanguage{english}Persimmon.} \textcolor{PineGreen}{\selectlanguage{french}Kaki.}  \zh{【借词】} \zh{柿子}
 ¶ \textcolor{darkblue}{\textbf{\ipa{ʂɯ˩tsɯ˧ | ɖɯ˧-so˩-ɭɯ˩ hwæ˩-bi˩!}}} \zh{买一些柿子吧!} \textcolor{Sepia}{\selectlanguage{english}Let's buy a few persimmons!} \textcolor{PineGreen}{\selectlanguage{french}(Je) vais acheter quelques kakis!}  

\lhead{\firstmark}
\rhead{\botmark}

\subsection{\hspace{-0.5cm} {\Large \textcolor{darkblue}{\textbf{\ipa{ʂɯ˩tsɯ˧}}}}\hspace{0.5cm}[\kern2pt{\textcolor{darkblue}{\textbf{\ipa{ʂɯ˩tsɯ˥}}}}\kern2pt]} \hypertarget{s`M\string_BtsM\string_M1}{}
\markboth{\textcolor{darkblue}{\textbf{\ipa{ʂɯ˩tsɯ˧}}}}{}
\textcolor{teal}{\zh{名词}} \hspace{4pt} \zh{声调类:} LM.
\zh{手枪。} \textcolor{Sepia}{\selectlanguage{english}Pistol.} \textcolor{PineGreen}{\selectlanguage{french}Pistolet.}  ¶ \textcolor{darkblue}{\textbf{\ipa{ʂɯ˩tsɯ˧ | ɖɯ˧-nɑ˧ | tʰi˧-pɤ˥\textasciitilde{}pɤ˩}}} \zh{带手枪} \textcolor{Sepia}{\selectlanguage{english}to carry a pistol} \textcolor{PineGreen}{\selectlanguage{french}porter un pistolet}  

\lhead{\firstmark}
\rhead{\botmark}

\subsection{\hspace{-0.5cm} {\Large \textcolor{darkblue}{\textbf{\ipa{ʂɯ˧˥}}} \textsubscript{1}}\hspace{0.5cm}[\kern2pt{\textcolor{darkblue}{\textbf{\ipa{ʂɯ˧˥}}}}\kern2pt]} \hypertarget{s`M\string_M\string_T1}{}
\markboth{\textcolor{darkblue}{\textbf{\ipa{ʂɯ˧˥}}} \textsubscript{1}}{}
\textcolor{teal}{\zh{动词}} \hspace{4pt} \zh{声调类:} MH.
\zh{削(用刀)。} \textcolor{Sepia}{\selectlanguage{english}To peel (with a knife).} \textcolor{PineGreen}{\selectlanguage{french}Éplucher, peler, décortiquer (avec un instrument).}  ¶ \textcolor{darkblue}{\textbf{\ipa{ɣɯ˩ ʂɯ˧˥ / ɣɯ˩ʂɯ˧ ze˥}}} \zh{削皮} \textcolor{Sepia}{\selectlanguage{english}to peel, to peel off the skin} \textcolor{PineGreen}{\selectlanguage{french}éplucher la peau}  
 ¶ \textcolor{darkblue}{\textbf{\ipa{ɣɯ˩kɯ˧ ʂɯ˥}}} \zh{削皮} \textcolor{Sepia}{\selectlanguage{english}to peel, to peel off the skin} \textcolor{PineGreen}{\selectlanguage{french}éplucher la peau}  
 ¶ \textcolor{darkblue}{\textbf{\ipa{jɤ˩jo˧ ɣɯ˥ʂɯ˩}}} \zh{削洋芋皮} \textcolor{Sepia}{\selectlanguage{english}to peel potatoes} \textcolor{PineGreen}{\selectlanguage{french}peler des patates}  
 ¶ \textcolor{darkblue}{\textbf{\ipa{[F5] tso˧tso˧ ɣɯ˥ʂɯ˩}}} \zh{削东西} \textcolor{Sepia}{\selectlanguage{english}to peel things} \textcolor{PineGreen}{\selectlanguage{french}éplucher des choses}  

\lhead{\firstmark}
\rhead{\botmark}

\subsection{\hspace{-0.5cm} {\Large \textcolor{darkblue}{\textbf{\ipa{ʂɯ˧˥}}} \textsubscript{2}}\hspace{0.5cm}[\kern2pt{\textcolor{darkblue}{\textbf{\ipa{ʂɯ˧˥}}}}\kern2pt]} \hypertarget{s`M\string_M\string_T2}{}
\markboth{\textcolor{darkblue}{\textbf{\ipa{ʂɯ˧˥}}} \textsubscript{2}}{}
\textcolor{teal}{\zh{形容词}} \hspace{4pt} \zh{声调类:} MH.
\zh{新。} \textcolor{Sepia}{\selectlanguage{english}New, fresh.} \textcolor{PineGreen}{\selectlanguage{french}Nouveau, neuf, frais.}  ¶ \textcolor{darkblue}{\textbf{\ipa{ʂɯ˧-hĩ˧ ɲi˥!}}} \zh{是新的!} \textcolor{Sepia}{\selectlanguage{english}It's new!} \textcolor{PineGreen}{\selectlanguage{french}c'est neuf!}  
 ¶ \textcolor{darkblue}{\textbf{\ipa{ʂe˧ ʂɯ˩}}} \zh{新鲜的肉} \textcolor{Sepia}{\selectlanguage{english}fresh meat} \textcolor{PineGreen}{\selectlanguage{french}de la viande fraîche}  

\lhead{\firstmark}
\rhead{\botmark}

\subsection{\hspace{-0.5cm} {\Large \textcolor{darkblue}{\textbf{\ipa{ʂv̩˧˥}}}}\hspace{0.5cm}[\kern2pt{\textcolor{darkblue}{\textbf{\ipa{ʂv̩˧˥}}}}\kern2pt]} \hypertarget{s`v\string_=\string_M\string_T1}{}
\markboth{\textcolor{darkblue}{\textbf{\ipa{ʂv̩˧˥}}}}{}
\textcolor{teal}{\zh{动词}} \hspace{4pt} \zh{声调类:} MH.
\zh{拧(拧毛巾)。} \textcolor{Sepia}{\selectlanguage{english}To twist, to wring.} \textcolor{PineGreen}{\selectlanguage{french}Tordre, essorer (vêtement).}  ¶ \textcolor{darkblue}{\textbf{\ipa{le˧-ʂv̩˧-ze˥}}} \zh{拧了} \textcolor{Sepia}{\selectlanguage{english}\mytextsc{accomp} \string_ \mytextsc{pfv}} \textcolor{PineGreen}{\selectlanguage{french}\mytextsc{accomp} \string_ \mytextsc{pfv}}  
 ¶ \textcolor{darkblue}{\textbf{\ipa{dʑi˧hṽ˧ ʂv̩˩}}} \zh{拧衣服} \textcolor{Sepia}{\selectlanguage{english}to wring out clothes} \textcolor{PineGreen}{\selectlanguage{french}essorer des vêtements}  

\lhead{\firstmark}
\rhead{\botmark}

\subsection{\hspace{-0.5cm} {\Large \textcolor{darkblue}{\textbf{\ipa{ʂv̩˥}}}}\hspace{0.5cm}[\kern2pt{\textcolor{darkblue}{\textbf{\ipa{ʂv̩˥}}}}\kern2pt]} \hypertarget{s`v\string_=\string_T1}{}
\markboth{\textcolor{darkblue}{\textbf{\ipa{ʂv̩˥}}}}{}
\textcolor{teal}{\zh{名词}} \hspace{4pt} \zh{声调类:} H.
\zh{骰子。} \textcolor{Sepia}{\selectlanguage{english}Dice.} \textcolor{PineGreen}{\selectlanguage{french}Dé.}  ¶ \textcolor{darkblue}{\textbf{\ipa{ʂv̩˧ | ʐv̩˩-ɭɯ˩˥}}} \zh{四个骰子} \textcolor{Sepia}{\selectlanguage{english}four dice (dice came in pairs)} \textcolor{PineGreen}{\selectlanguage{french}quatre dés (les dés allaient par quatre)}  
 \zh{量词}: \textcolor{darkblue}{\textbf{\ipa{ɭɯ˧}}} 
\lhead{\firstmark}
\rhead{\botmark}

\subsection{\hspace{-0.5cm} {\Large \textcolor{darkblue}{\textbf{\ipa{ʂv̩˧\textsubscript{b}}}}}\hspace{0.5cm}[\kern2pt{\textcolor{darkblue}{\textbf{\ipa{ʂv̩˥}}}}\kern2pt]} \hypertarget{s`v\string_=\string_Mb1}{}
\markboth{\textcolor{darkblue}{\textbf{\ipa{ʂv̩˧\textsubscript{b}}}}}{}
\textcolor{teal}{\zh{动词}} \hspace{4pt} \zh{声调类:} M\textsubscript{b}.
\ding{202} \zh{带(孩子……)。} \textcolor{Sepia}{\selectlanguage{english}To look after, to take care of (children).} \textcolor{PineGreen}{\selectlanguage{french}S'occuper de, surveiller.}  ¶ \textcolor{darkblue}{\textbf{\ipa{zo˧mv̩˥ | ɖɯ˧-ɭɯ˧ ʂv̩˧}}} \zh{带个孩子} \textcolor{Sepia}{\selectlanguage{english}to take care of a child, to look after a child} \textcolor{PineGreen}{\selectlanguage{french}garder un enfant, s'occuper d'un enfant}  
 ¶ \textcolor{darkblue}{\textbf{\ipa{zo˧mv̩˥ ʂv̩˩}}} \zh{带孩子} \textcolor{Sepia}{\selectlanguage{english}to take care of a child, to look after a child} \textcolor{PineGreen}{\selectlanguage{french}garder un enfant, s'occuper d'un enfant}  
 ¶ \textcolor{darkblue}{\textbf{\ipa{le˧-ʂv̩˧ tʰi˧-kʰɯ˧˥ | tʰæ˧ɻ̍˩ so˩}}} \zh{让他学习、要求他学习(家长管孩子,让他学习)} \textcolor{Sepia}{\selectlanguage{english}to oblige to study (a mother obliges a child to study)} \textcolor{PineGreen}{\selectlanguage{french}obliger à étudier (une mère oblige son enfant à faire ses devoirs)}  
\ding{203} \zh{带(路)。} \textcolor{Sepia}{\selectlanguage{english}To lead (the way).} \textcolor{PineGreen}{\selectlanguage{french}Mener, guider.}  ¶ \textcolor{darkblue}{\textbf{\ipa{ʈæ˧ʂɯ˧ | ɖʐv̩˧ ʂv̩˧-po˧-bi˧-ho˥!}}} \zh{达石要管朋友(带他们去永宁旅游)} \textcolor{Sepia}{\selectlanguage{english}Dashi is going to take care of his friends [taking them on a tourist trip to Yongning]} \textcolor{PineGreen}{\selectlanguage{french}Dashi va accompagner ses amis [=les emmener en excursion à Yongning]!}  
 ¶ \textcolor{darkblue}{\textbf{\ipa{ʈæ˧ʂɯ˧ | ɖʐv̩˧ ʂv̩˧-bi˧-ho˩!}}} \zh{达石要管朋友(带他们去永宁旅游)} \textcolor{Sepia}{\selectlanguage{english}Dashi is going to take care of his friends [taking them on a tourist trip to Yongning]} \textcolor{PineGreen}{\selectlanguage{french}Dashi va accompagner ses amis [=les emmener en excursion à Yongning]!}  

\lhead{\firstmark}
\rhead{\botmark}

\subsection{\hspace{-0.5cm} {\Large \textcolor{darkblue}{\textbf{\ipa{ʂv̩˧ɖv̩˧}}}}\hspace{0.5cm}[\kern2pt{\textcolor{darkblue}{\textbf{\ipa{ʂv̩˧ɖv̩˧}}}}\kern2pt]} \hypertarget{s`v\string_=\string_Md`v\string_=\string_M1}{}
\markboth{\textcolor{darkblue}{\textbf{\ipa{ʂv̩˧ɖv̩˧}}}}{}
\textcolor{teal}{\zh{动词}} \hspace{4pt} \zh{声调类:} M.
\ding{202} \zh{想。} \textcolor{Sepia}{\selectlanguage{english}To think.} \textcolor{PineGreen}{\selectlanguage{french}Penser, réfléchir.}  ¶ \textcolor{darkblue}{\textbf{\ipa{ə˧tso˧ ʂv̩˧ɖv̩˧?}}} \zh{在想什么?} \textcolor{Sepia}{\selectlanguage{english}What are you thinking about? / Where's your mind?} \textcolor{PineGreen}{\selectlanguage{french}A quoi [tu] penses?}  
 ¶ \textcolor{darkblue}{\textbf{\ipa{njɤ˧ | ɖɯ˧ bæ˧ ʂv̩˧dv̩˧}}} \zh{我在想一件事情。} \textcolor{Sepia}{\selectlanguage{english}I'm thinking about something.} \textcolor{PineGreen}{\selectlanguage{french}je pense à quelque chose}  
 ¶ \textcolor{darkblue}{\textbf{\ipa{ʂv̩˧ɖv̩˧ tʰv̩˧}}} \zh{明白,想起} \textcolor{Sepia}{\selectlanguage{english}to understand} \textcolor{PineGreen}{\selectlanguage{french}comprendre; se souvenir}  
 ¶ \textcolor{darkblue}{\textbf{\ipa{njɤ˧ | ʂv̩˧ɖv̩˧ tʰv̩˧}}} \zh{我明白。} \textcolor{Sepia}{\selectlanguage{english}I understand.} \textcolor{PineGreen}{\selectlanguage{french}Je comprends.}  
 ¶ \textcolor{darkblue}{\textbf{\ipa{ʈʂʰɯ˧ | le˧-ʂv̩˧ɖv̩˧-le˧-tʰv̩˧-ze˧}}} \zh{他明白了。} \textcolor{Sepia}{\selectlanguage{english}He has understood.} \textcolor{PineGreen}{\selectlanguage{french}Il a compris.}  
\ding{203} \zh{想起、回忆。} \textcolor{Sepia}{\selectlanguage{english}To remember, to recollect, to recall.} \textcolor{PineGreen}{\selectlanguage{french}Se souvenir.}  ¶ \textcolor{darkblue}{\textbf{\ipa{ʂv̩˧ɖv̩˧ tʰv̩˧}}} \zh{想起} \textcolor{Sepia}{\selectlanguage{english}to remember} \textcolor{PineGreen}{\selectlanguage{french}se souvenir}  
 ¶ \textcolor{darkblue}{\textbf{\ipa{njɤ˧ | ʂv̩˧ɖv̩˧ tʰv̩˧}}} \zh{我想起} \textcolor{Sepia}{\selectlanguage{english}I remember} \textcolor{PineGreen}{\selectlanguage{french}je me souviens}  
 ¶ \textcolor{darkblue}{\textbf{\ipa{ʈʂʰɯ˧ | le˧-ʂv̩˧ɖv̩˧-le˧-tʰv̩˧-ze˧}}} \zh{他想起来了} \textcolor{Sepia}{\selectlanguage{english}He remembers, he recollects} \textcolor{PineGreen}{\selectlanguage{french}il se rappelle, ça lui revient}  
\ding{204} \zh{想念、感到悲哀。} \textcolor{Sepia}{\selectlanguage{english}To miss, to long for; to feel sorrowful, sad, grieved.} \textcolor{PineGreen}{\selectlanguage{french}Avoir la nostalgie de.}  ¶ \textcolor{darkblue}{\textbf{\ipa{ʂv̩˧ɖv̩˧ tʰv̩˧ | ʐwæ˩˥}}} \zh{特别想念} \textcolor{Sepia}{\selectlanguage{english}to be full of nostalgia} \textcolor{PineGreen}{\selectlanguage{french}être plongé dans le chagrin}  
 ¶ \textcolor{darkblue}{\textbf{\ipa{njɤ˧ | no˩ ʂv̩˩ɖv̩˩˥}}} \zh{我想你!} \textcolor{Sepia}{\selectlanguage{english}I miss you!} \textcolor{PineGreen}{\selectlanguage{french}tu me manques!}  
 ¶ \textcolor{darkblue}{\textbf{\ipa{ʂv̩˧ɖv̩˧ qʰwɤ˧ tʰv̩˧˥}}} \zh{想念} \textcolor{Sepia}{\selectlanguage{english}to have a fit of nostalgia} \textcolor{PineGreen}{\selectlanguage{french}être nostalgique, avoir une crise de nostalgie}  
 ¶ \textcolor{darkblue}{\textbf{\ipa{[F5] ʂv̩˧ɖv̩˧ mɤ˧-zo˧}}} \zh{不用发愁} \textcolor{Sepia}{\selectlanguage{english}There's no need to worry / feel unhappy} \textcolor{PineGreen}{\selectlanguage{french}on n'a pas à se faire de souci, il n'y a pas lieu de se morfondre}  

\lhead{\firstmark}
\rhead{\botmark}

\subsection{\hspace{-0.5cm} {\Large \textcolor{darkblue}{\textbf{\ipa{ʂv̩˩gv̩˩}}}}\hspace{0.5cm}[\kern2pt{\textcolor{darkblue}{\textbf{\ipa{ʂv̩˧gv̩˧}}}}\kern2pt]} \hypertarget{s`v\string_=\string_Bgv\string_=\string_B1}{}
\markboth{\textcolor{darkblue}{\textbf{\ipa{ʂv̩˩gv̩˩}}}}{}
\textcolor{teal}{\zh{名词}} \hspace{4pt} \zh{声调类:} L.
\zh{镰刀。} \textcolor{Sepia}{\selectlanguage{english}Sickle.} \textcolor{PineGreen}{\selectlanguage{french}Faucille.}  \zh{量词}: \textcolor{darkblue}{\textbf{\ipa{nɑ˧}}} 
\lhead{\firstmark}
\rhead{\botmark}

\subsection{\hspace{-0.5cm} {\Large \textcolor{darkblue}{\textbf{\ipa{ʂv̩˧kʰɯ˩}}}}\hspace{0.5cm}[\kern2pt{\textcolor{darkblue}{\textbf{\ipa{ʂv̩˩kʰɯ˩˥}}}}\kern2pt]} \hypertarget{s`v\string_=\string_Mk\string_hM\string_B1}{}
\markboth{\textcolor{darkblue}{\textbf{\ipa{ʂv̩˧kʰɯ˩}}}}{}
\textcolor{teal}{\zh{动词}} \hspace{4pt} \zh{声调类:} .
\zh{赌博。} \textcolor{Sepia}{\selectlanguage{english}To bet.} \textcolor{PineGreen}{\selectlanguage{french}Parier.}  ¶ \textcolor{darkblue}{\textbf{\ipa{ʂv̩˧kʰɯ˩ | -jɤ˩po˧}}} \zh{同上} \textcolor{Sepia}{\selectlanguage{english}same meaning} \textcolor{PineGreen}{\selectlanguage{french}même sens}  

\lhead{\firstmark}
\rhead{\botmark}

\subsection{\hspace{-0.5cm} {\Large \textcolor{darkblue}{\textbf{\ipa{ʂv̩˩njɤ˥}}}}\hspace{0.5cm}[\kern2pt{\textcolor{darkblue}{\textbf{\ipa{ʂv̩˩njɤ˥}}}}\kern2pt]} \hypertarget{s`v\string_=\string_Bnj7\string_T1}{}
\markboth{\textcolor{darkblue}{\textbf{\ipa{ʂv̩˩njɤ˥}}}}{}
\textcolor{teal}{\zh{名词}} \hspace{4pt} \zh{声调类:} LH.
\zh{树瘤。} \textcolor{Sepia}{\selectlanguage{english}Tree bur; burl.} \textcolor{PineGreen}{\selectlanguage{french}Nœud (sur un arbre).}  ¶ \textcolor{darkblue}{\textbf{\ipa{ʂv̩˩njɤ˥ ɲi˩}}} \zh{是树瘤。} \textcolor{Sepia}{\selectlanguage{english}\mytextsc{cop}} \textcolor{PineGreen}{\selectlanguage{french}\mytextsc{cop}}  
 \zh{量词}: \textcolor{darkblue}{\textbf{\ipa{ɭɯ˧}}} 
\lhead{\firstmark}
\rhead{\botmark}

\subsection{\hspace{-0.5cm} {\Large \textcolor{darkblue}{\textbf{\ipa{ʂv̩˧ʂv̩˧˥}}}}\hspace{0.5cm}[\kern2pt{\textcolor{darkblue}{\textbf{\ipa{ʂv̩˧ʂv̩˧˥}}}}\kern2pt]} \hypertarget{s`v\string_=\string_Ms`v\string_=\string_M\string_T1}{}
\markboth{\textcolor{darkblue}{\textbf{\ipa{ʂv̩˧ʂv̩˧˥}}}}{}
\textcolor{teal}{\zh{名词}} \hspace{4pt} \zh{声调类:} MH\#.
\zh{纸。} \textcolor{Sepia}{\selectlanguage{english}Paper.} \textcolor{PineGreen}{\selectlanguage{french}Papier.}  ¶ \textcolor{darkblue}{\textbf{\ipa{ʂv̩˧ʂv̩˧˥ | ɖɯ˧-pʰæ˧˥}}} \zh{一张纸} \textcolor{Sepia}{\selectlanguage{english}a sheet of paper} \textcolor{PineGreen}{\selectlanguage{french}une feuille de papier}  
 ¶ \textcolor{darkblue}{\textbf{\ipa{[F5] ʂv̩˧ʂv̩˧ ɖɯ˧ pʰæ˧˥}}} \zh{一张纸} \textcolor{Sepia}{\selectlanguage{english}a sheet of paper} \textcolor{PineGreen}{\selectlanguage{french}une feuille de papier}  
 \zh{量词}: \textcolor{darkblue}{\textbf{\ipa{pʰæ˧˥}}} 
\lhead{\firstmark}
\rhead{\botmark}

\subsection{\hspace{-0.5cm} {\Large \textcolor{darkblue}{\textbf{\ipa{ʂwæ˧}}}}\hspace{0.5cm}[\kern2pt{\textcolor{darkblue}{\textbf{\ipa{ʂwæ˥}}}}\kern2pt]} \hypertarget{s`w\{\string_M1}{}
\markboth{\textcolor{darkblue}{\textbf{\ipa{ʂwæ˧}}}}{}
\textcolor{teal}{\zh{名词}} \hspace{4pt} \zh{声调类:} M.
\zh{水獭。} \textcolor{Sepia}{\selectlanguage{english}Otter.} \textcolor{PineGreen}{\selectlanguage{french}Loutre.} \zh{当地汉语方言:}\zh{水潭猫。} ¶ \textcolor{darkblue}{\textbf{\ipa{ʂwæ˧-ɣɯ˩}}} \zh{水獭皮} \textcolor{Sepia}{\selectlanguage{english}otter skin} \textcolor{PineGreen}{\selectlanguage{french}peau de loutre}  

\lhead{\firstmark}
\rhead{\botmark}

\subsection{\hspace{-0.5cm} {\Large \textcolor{darkblue}{\textbf{\ipa{ʂwæ˧}}}}\hspace{0.5cm}[\kern2pt{\textcolor{darkblue}{\textbf{\ipa{ʂwæ˥}}}}\kern2pt]} \hypertarget{s`w\{\string_M1}{}
\markboth{\textcolor{darkblue}{\textbf{\ipa{ʂwæ˧}}}}{}
\textcolor{teal}{\zh{形容词}} \hspace{4pt} \zh{声调类:} M.
\zh{高。} \textcolor{Sepia}{\selectlanguage{english}Tall.} \textcolor{PineGreen}{\selectlanguage{french}Haut, de haute taille, grand.}  ¶ \textcolor{darkblue}{\textbf{\ipa{qʰɑ˧-ʂwæ˧-gv̩˧}}} \zh{非常高} \textcolor{Sepia}{\selectlanguage{english}very tall} \textcolor{PineGreen}{\selectlanguage{french}très grand}  
 ¶ \textcolor{darkblue}{\textbf{\ipa{ʈʂʰɯ˧ | ə˧pɤ˧ | -ʂwæ˩-gv̩˩˥!}}} \zh{他非常高!} \textcolor{Sepia}{\selectlanguage{english}(S)he is extremely tall!} \textcolor{PineGreen}{\selectlanguage{french}Elle/il est très grand(e)!}  
 ¶ \textcolor{darkblue}{\textbf{\ipa{gv̩˧mi˧ ʂwæ˧}}} \zh{高、身材高} \textcolor{Sepia}{\selectlanguage{english}tall; literally 'with a tall body'} \textcolor{PineGreen}{\selectlanguage{french}de grande taille; littéralement '(qui a un) grand corps'}  

\lhead{\firstmark}
\rhead{\botmark}

\subsection{\hspace{-0.5cm} {\Large \textcolor{darkblue}{\textbf{\ipa{ʂwæ˧\textsubscript{a}}}}}\hspace{0.5cm}[\kern2pt{\textcolor{darkblue}{\textbf{\ipa{ʂwæ˥}}}}\kern2pt]} \hypertarget{s`w\{\string_Ma1}{}
\markboth{\textcolor{darkblue}{\textbf{\ipa{ʂwæ˧\textsubscript{a}}}}}{}
\textcolor{teal}{\zh{动词}} \hspace{4pt} \zh{声调类:} M\textsubscript{a}.
\zh{搅拌。} \textcolor{Sepia}{\selectlanguage{english}To stir.} \textcolor{PineGreen}{\selectlanguage{french}Remuer (monosyllabe).}  ¶ \textcolor{darkblue}{\textbf{\ipa{le˧-ʂwæ˧}}} \zh{\mytextsc{accomp}} \textcolor{Sepia}{\selectlanguage{english}\mytextsc{accomp}} \textcolor{PineGreen}{\selectlanguage{french}\mytextsc{accomp}}  
 ¶ \textcolor{darkblue}{\textbf{\ipa{mɤ˧-ʂwæ˧}}} \zh{不搅拌} \textcolor{Sepia}{\selectlanguage{english}\mytextsc{neg}} \textcolor{PineGreen}{\selectlanguage{french}\mytextsc{neg}}  
 ¶ \textcolor{darkblue}{\textbf{\ipa{tso˧\textasciitilde{}tso˧ ʂwæ˩}}} \zh{搅拌东西} \textcolor{Sepia}{\selectlanguage{english}to stir things} \textcolor{PineGreen}{\selectlanguage{french}remuer des choses, touiller des choses}  

\lhead{\firstmark}
\rhead{\botmark}

\subsection{\hspace{-0.5cm} {\Large \textcolor{darkblue}{\textbf{\ipa{ʂwæ˧bæ˩}}}}\hspace{0.5cm}[\kern2pt{\textcolor{darkblue}{\textbf{\ipa{ʂwæ˧bæ˩}}}}\kern2pt]} \hypertarget{s`w\{\string_Mb\{\string_B1}{}
\markboth{\textcolor{darkblue}{\textbf{\ipa{ʂwæ˧bæ˩}}}}{}
\textcolor{teal}{\zh{名词}} \hspace{4pt} \zh{声调类:} L\#.
\zh{映山红。} \textcolor{Sepia}{\selectlanguage{english}Camellia flower.} \textcolor{PineGreen}{\selectlanguage{french}Camélia.} \zh{当地汉语方言:}\zh{山茶花。} ¶ \textcolor{darkblue}{\textbf{\ipa{ʂwæ˧bæ˩ bæ˩ |}}} \zh{山茶花开了。} \textcolor{Sepia}{\selectlanguage{english}The camellia flowers are in bloom.} \textcolor{PineGreen}{\selectlanguage{french}Les camélias sont en fleurs.}  
 ¶ \textcolor{darkblue}{\textbf{\ipa{so˧-ɬi˧mi˧, | ʂwæ˧bæ˩ bæ˩! |}}} \zh{山茶花是在三月份开花的!} \textcolor{Sepia}{\selectlanguage{english}Camellia flowers bloom in the third month!} \textcolor{PineGreen}{\selectlanguage{french}Les camélias fleurissent au troisième mois!}  
 ¶ \textcolor{darkblue}{\textbf{\ipa{ʂwæ˧bæ˩-si˩}}} \zh{山茶树} \textcolor{Sepia}{\selectlanguage{english}camellia tree} \textcolor{PineGreen}{\selectlanguage{french}camélia (arbre); littéralement “arbre des fleurs de camélia”}  

\lhead{\firstmark}
\rhead{\botmark}

\subsection{\hspace{-0.5cm} {\Large \textcolor{darkblue}{\textbf{\ipa{ʂwæ˧gv̩\#˥}}}}\hspace{0.5cm}[\kern2pt{\textcolor{darkblue}{\textbf{\ipa{ʂwæ˧gv̩˧}}}}\kern2pt]} \hypertarget{s`w\{\string_Mgv\string_=\#\string_T1}{}
\markboth{\textcolor{darkblue}{\textbf{\ipa{ʂwæ˧gv̩\#˥}}}}{}
\textcolor{teal}{\zh{名词}} \hspace{4pt} \zh{声调类:} \#H.
\zh{加泽大山(位于永宁西北的一座山)。} \textcolor{Sepia}{\selectlanguage{english}A mountain to the North-West of Yongning, called “Jiaze Mountain” in Chinese.} \textcolor{PineGreen}{\selectlanguage{french}Une montagne au nord-ouest de Yongning.}  ¶ \textcolor{darkblue}{\textbf{\ipa{kɤ˧mv̩˧˥, | æ˧ʂæ˧, | ŋwɤ˧hɑ̃˩, | ʂwæ˧gv̩\#˥, | nɑ˩tsʰi˩˥ | -tɕʰɤ˧pɤ˧mi\#˥, | qv̩˧ɻ̍˧-ʈʂʰɑ˧nɑ˥ |}}} \zh{永宁地区有固定名字的六座山。其它的山,因为没有重要的象征意义,因此没有取名。} \textcolor{Sepia}{\selectlanguage{english}The six mountains of Yongning that carry a name and have a definite symbolic value. The other mountains do not have comparable symbolic value, and fewer people use specific names for them.} \textcolor{PineGreen}{\selectlanguage{french}Les six montagnes de Yongning qui portent un nom. Les autres sommets du voisinage n'ont pas une valeur symbolique comparable, et ne portent pas de nom communément utilisé.}  

\lhead{\firstmark}
\rhead{\botmark}

\subsection{\hspace{-0.5cm} {\Large \textcolor{darkblue}{\textbf{\ipa{ʂwæ˧si\#˥}}}}\hspace{0.5cm}[\kern2pt{\textcolor{darkblue}{\textbf{\ipa{ʂwæ˧si˧}}}}\kern2pt]} \hypertarget{s`w\{\string_Msi\#\string_T1}{}
\markboth{\textcolor{darkblue}{\textbf{\ipa{ʂwæ˧si\#˥}}}}{}
\textcolor{teal}{\zh{名词}} \hspace{4pt} \zh{声调类:} \#H.
\zh{山茶树。} \textcolor{Sepia}{\selectlanguage{english}Camellia tree.} \textcolor{PineGreen}{\selectlanguage{french}Camélia (arbre).} 
\lhead{\firstmark}
\rhead{\botmark}

\subsection{\hspace{-0.5cm} {\Large \textcolor{darkblue}{\textbf{\ipa{ʂwæ˧tsɯ˥}}}}\hspace{0.5cm}[\kern2pt{\textcolor{darkblue}{\textbf{\ipa{ʂwæ˧tsɯ˥}}}}\kern2pt]} \hypertarget{s`w\{\string_MtsM\string_T1}{}
\markboth{\textcolor{darkblue}{\textbf{\ipa{ʂwæ˧tsɯ˥}}}}{}
\textcolor{teal}{\zh{名词}} \hspace{4pt} \zh{声调类:} H\#.
\zh{刷子。} \textcolor{Sepia}{\selectlanguage{english}Brush.} \textcolor{PineGreen}{\selectlanguage{french}Brosse.}  \zh{【借词】} \zh{刷子}
 \zh{量词}: \textcolor{darkblue}{\textbf{\ipa{nɑ˧}}} 
\lhead{\firstmark}
\rhead{\botmark}

\subsection{\hspace{-0.5cm} {\Large \textcolor{darkblue}{\textbf{\ipa{ʂwæ˩\textsubscript{a}}}}}\hspace{0.5cm}[\kern2pt{\textcolor{darkblue}{\textbf{\ipa{ʂwæ˩˥}}}}\kern2pt]} \hypertarget{s`w\{\string_Ba1}{}
\markboth{\textcolor{darkblue}{\textbf{\ipa{ʂwæ˩\textsubscript{a}}}}}{}
\textcolor{teal}{\zh{动词}} \hspace{4pt} \zh{声调类:} L\textsubscript{a}.
\zh{熏。} \textcolor{Sepia}{\selectlanguage{english}To cure (meat etc) with smoke.} \textcolor{PineGreen}{\selectlanguage{french}Fumer (aliment).}  ¶ \textcolor{darkblue}{\textbf{\ipa{ʂe˧ ʂwæ˥}}} \zh{熏肉} \textcolor{Sepia}{\selectlanguage{english}to cure meat with smoke} \textcolor{PineGreen}{\selectlanguage{french}fumer de la viande}  

\lhead{\firstmark}
\rhead{\botmark}

\subsection{\hspace{-0.5cm} {\Large \textcolor{darkblue}{\textbf{\ipa{ʂwæ˩gv̩˩}}}}\hspace{0.5cm}[\kern2pt{\textcolor{darkblue}{\textbf{\ipa{ʂwæ˩gv̩˩˥}}}}\kern2pt]} \hypertarget{s`w\{\string_Bgv\string_=\string_B1}{}
\markboth{\textcolor{darkblue}{\textbf{\ipa{ʂwæ˩gv̩˩}}}}{}
\textcolor{teal}{\zh{名词}} \hspace{4pt} \zh{声调类:} L.
\zh{柜子。} \textcolor{Sepia}{\selectlanguage{english}Kitchen cabinet, kitchen dresser.} \textcolor{PineGreen}{\selectlanguage{french}Buffet (où est rangée la vaisselle).}  \zh{量词}: \textcolor{darkblue}{\textbf{\ipa{ɭɯ˧}}} 
\lhead{\firstmark}
\rhead{\botmark}

\subsection{\hspace{-0.5cm} {\Large \textcolor{darkblue}{\textbf{\ipa{ʂwæ˧˥}}}}\hspace{0.5cm}[\kern2pt{\textcolor{darkblue}{\textbf{\ipa{ʂwæ˧˥}}}}\kern2pt]} \hypertarget{s`w\{\string_M\string_T1}{}
\markboth{\textcolor{darkblue}{\textbf{\ipa{ʂwæ˧˥}}}}{}
\textcolor{teal}{\zh{动词}} \hspace{4pt} \zh{声调类:} MH.
\zh{拉(屎)。} \textcolor{Sepia}{\selectlanguage{english}To defecate.} \textcolor{PineGreen}{\selectlanguage{french}Déféquer, faire caca.}  ¶ \textcolor{darkblue}{\textbf{\ipa{qʰæ˧ ʂwæ˩}}} \zh{拉屎} \textcolor{Sepia}{\selectlanguage{english}to defecate} \textcolor{PineGreen}{\selectlanguage{french}déféquer}  

\lhead{\firstmark}
\rhead{\botmark}

\subsection{\hspace{-0.5cm} {\Large \textcolor{darkblue}{\textbf{\ipa{ʂwæ˩˥}}}}\hspace{0.5cm}[\kern2pt{\textcolor{darkblue}{\textbf{\ipa{ʂwæ˩˥}}}}\kern2pt]} \hypertarget{s`w\{\string_B\string_T1}{}
\markboth{\textcolor{darkblue}{\textbf{\ipa{ʂwæ˩˥}}}}{}
\textcolor{teal}{\zh{名词}} \hspace{4pt} \zh{声调类:} LH.
\zh{楔子。} \textcolor{Sepia}{\selectlanguage{english}Wedge.} \textcolor{PineGreen}{\selectlanguage{french}Coin.}  ¶ \textcolor{darkblue}{\textbf{\ipa{ʂwæ˩ lɑ˧˥ / ʂwæ˩ lɑ˧-ze˥}}} \zh{打一个楔子} \textcolor{Sepia}{\selectlanguage{english}to strike a wedge} \textcolor{PineGreen}{\selectlanguage{french}frapper un coin}  
 ¶ \textcolor{darkblue}{\textbf{\ipa{ʂwæ˩ hwæ˥-ze˩}}} \zh{买了楔子} \textcolor{Sepia}{\selectlanguage{english}...bought a wedge} \textcolor{PineGreen}{\selectlanguage{french}...a acheté un coin}  
 ¶ \textcolor{darkblue}{\textbf{\ipa{ʂwæ˩ tʰv̩˩-ɭɯ˩˥ / ʂwæ˩ tʰv̩˩-ɭɯ˥}}} \zh{这个楔子} \textcolor{Sepia}{\selectlanguage{english}\string_ \mytextsc{dem}+\mytextsc{clf}} \textcolor{PineGreen}{\selectlanguage{french}\string_ \mytextsc{dem}+\mytextsc{clf}.générique}  
 ¶ \textcolor{darkblue}{\textbf{\ipa{ʂwæ˩ tʰv̩˩-kʰwɤ˩˥}}} \zh{这个楔子} \textcolor{Sepia}{\selectlanguage{english}\string_ \mytextsc{dem}+\mytextsc{clf}} \textcolor{PineGreen}{\selectlanguage{french}\mytextsc{n}+\mytextsc{dem}+\mytextsc{clf}.morceaux}  
 ¶ \textcolor{darkblue}{\textbf{\ipa{[F5] ʂwæ˩ kʰɯ˥}}} \zh{放一个楔子} \textcolor{Sepia}{\selectlanguage{english}to place a wedge, to put a wedge} \textcolor{PineGreen}{\selectlanguage{french}mettre un coin}  
 \zh{量词}: \textcolor{darkblue}{\textbf{\ipa{kʰwɤ˥ / ɭɯ˧}}} 
\lhead{\firstmark}
\rhead{\botmark}

\newpage
\section*{\centering- \textcolor{darkblue}{\textbf{\ipa{t}}} -}
\subsection{\hspace{-0.5cm} {\Large \textcolor{darkblue}{\textbf{\ipa{tɑ˥}}}}\hspace{0.5cm}[\kern2pt{\textcolor{darkblue}{\textbf{\ipa{tɑ˥}}}}\kern2pt]} \hypertarget{tA\string_T1}{}
\markboth{\textcolor{darkblue}{\textbf{\ipa{tɑ˥}}}}{}
\textcolor{teal}{\zh{形容词}} \hspace{4pt} \zh{声调类:} H.
\zh{可靠。} \textcolor{Sepia}{\selectlanguage{english}Reliable, trustworthy.} \textcolor{PineGreen}{\selectlanguage{french}Fiable.}  ¶ \textcolor{darkblue}{\textbf{\ipa{le˧-tɑ˥ (| ʐwæ˩˥)}}} \zh{很靠谱} \textcolor{Sepia}{\selectlanguage{english}very reliable} \textcolor{PineGreen}{\selectlanguage{french}très fiable}  
 ¶ \textcolor{darkblue}{\textbf{\ipa{le˧ mɤ˧-tɑ˥ (| ʐwæ˩˥)}}} \zh{不靠谱} \textcolor{Sepia}{\selectlanguage{english}not reliable at all} \textcolor{PineGreen}{\selectlanguage{french}pas fiable}  
 ¶ \textcolor{darkblue}{\textbf{\ipa{no˧ | le˧-mɤ˧-tɑ˥-hĩ˩ ɖɯ˧-v̩˧ ɲi˩!}}} \zh{你是不靠谱的人!} \textcolor{Sepia}{\selectlanguage{english}You are an irresponsible person! / You are not a reliable person!} \textcolor{PineGreen}{\selectlanguage{french}Tu es quelqu'un de pas fiable/pas responsable!}  

\lhead{\firstmark}
\rhead{\botmark}

\subsection{\hspace{-0.5cm} {\Large \textcolor{darkblue}{\textbf{\ipa{tɑ˥mo˩}}}}\hspace{0.5cm}[\kern2pt{\textcolor{darkblue}{\textbf{\ipa{xxxx ton non trouvé, à faire manuellement...}}}}\kern2pt]} \hypertarget{tA\string_Tmo\string_B1}{}
\markboth{\textcolor{darkblue}{\textbf{\ipa{tɑ˥mo˩}}}}{}
\textcolor{teal}{\zh{动词}} \hspace{4pt} \zh{声调类:} HL.
\zh{萎、萎蔫。} \textcolor{Sepia}{\selectlanguage{english}To wilt, to wither (flower...).} \textcolor{PineGreen}{\selectlanguage{french}Faner.}  ¶ \textcolor{darkblue}{\textbf{\ipa{lə˧-tɑ˥mo˩-ze˩!}}} \zh{萎蔫了!} \textcolor{Sepia}{\selectlanguage{english}It has wilted!} \textcolor{PineGreen}{\selectlanguage{french}Ca a fané! (Exemple: une fleur coupée, une fleur abîmée par le vent ou par un soleil trop ardent.)}  

\lhead{\firstmark}
\rhead{\botmark}

\subsection{\hspace{-0.5cm} {\Large \textcolor{darkblue}{\textbf{\ipa{tɑ˧\textsubscript{a}}}}}\hspace{0.5cm}[\kern2pt{\textcolor{darkblue}{\textbf{\ipa{tɑ˩˥}}}}\kern2pt]} \hypertarget{tA\string_Ma1}{}
\markboth{\textcolor{darkblue}{\textbf{\ipa{tɑ˧\textsubscript{a}}}}}{}
\textcolor{teal}{\zh{动词}} \hspace{4pt} \zh{声调类:} M\textsubscript{a}.
\zh{烘干。} \textcolor{Sepia}{\selectlanguage{english}To dry beside or over a fire.} \textcolor{PineGreen}{\selectlanguage{french}Chauffer au feu.}  ¶ \textcolor{darkblue}{\textbf{\ipa{kwɤ˧-kʰɯ˧ tʰi˧-tɑ˧}}} \zh{放在火炉旁边热一下(饭)} \textcolor{Sepia}{\selectlanguage{english}to warm up (food...) beside a fire} \textcolor{PineGreen}{\selectlanguage{french}réchauffer (la nourriture...) auprès du feu}  

\lhead{\firstmark}
\rhead{\botmark}

\subsection{\hspace{-0.5cm} {\Large \textcolor{darkblue}{\textbf{\ipa{tɑ˧dzi˩}}}}\hspace{0.5cm}[\kern2pt{\textcolor{darkblue}{\textbf{\ipa{tɑ˩dzi˧˥}}}}\kern2pt]} \hypertarget{tA\string_Mdzi\string_B1}{}
\markboth{\textcolor{darkblue}{\textbf{\ipa{tɑ˧dzi˩}}}}{}
\textcolor{teal}{\zh{名词}} \hspace{4pt} \zh{声调类:} L\#.
\zh{村落名。} \textcolor{Sepia}{\selectlanguage{english}A Na village down below Nhissei, upward from Lataddi.} \textcolor{PineGreen}{\selectlanguage{french}Village na en contrebas de Nhissei, en contrehaut de Lataddi (\textcolor{darkblue}{\textbf{\ipa{/lɑ˧tʰɑ˧-di˧˥/}}}).}  ¶ \textcolor{darkblue}{\textbf{\ipa{ɬi˧ki˧, | ɲi˧se˩, | tɑ˧dzi˩, | mv̩˧qʰwæ˩, | lɑ˧tʰɑ˧-di˧˥}}} \zh{永宁到泸沽湖所经过的村落,依次是:里格、尼赛、大祖、木垮,然后到拉塔地(拉塔地指的是泸沽湖周边的摩梭地区,包括左所、洛水村等)} \textcolor{Sepia}{\selectlanguage{english}Villages that one passes when moving away from the Yongning plain, towards Lugu lake. These villages do not count as part of Yongning proper. The last, \textcolor{darkblue}{\textbf{\ipa{/lɑ˧tʰɑ˧-di˧˥/}}}, is not a village name like the preceding four: it refers to the entire Na area beyond the fourth village.} \textcolor{PineGreen}{\selectlanguage{french}Villages dans l'ordre, après la plaine de Yongning, ne comptant pas comme faisant partie de Yongning. Le dernier, \textcolor{darkblue}{\textbf{\ipa{/lɑ˧tʰɑ˧-di˧˥/}}}, désigne toute la région na au-delà du quatrième village.}  

\lhead{\firstmark}
\rhead{\botmark}

\subsection{\hspace{-0.5cm} {\Large \textcolor{darkblue}{\textbf{\ipa{tɑ˧gɤ˩}}}}\hspace{0.5cm}[\kern2pt{\textcolor{darkblue}{\textbf{\ipa{tɑ˩gɤ˥}}}}\kern2pt]} \hypertarget{tA\string_Mg7\string_B1}{}
\markboth{\textcolor{darkblue}{\textbf{\ipa{tɑ˧gɤ˩}}}}{}
\textcolor{teal}{\zh{形容词}} \hspace{4pt} \zh{声调类:} L\#.
\zh{瘦弱、枯瘦。} \textcolor{Sepia}{\selectlanguage{english}Gaunt, emaciated.} \textcolor{PineGreen}{\selectlanguage{french}Maigre (personne maigre).} 
\lhead{\firstmark}
\rhead{\botmark}

\subsection{\hspace{-0.5cm} {\Large \textcolor{darkblue}{\textbf{\ipa{tɑ˧ho˧}}}}\hspace{0.5cm}[\kern2pt{\textcolor{darkblue}{\textbf{\ipa{tɑ˧ho˩}}}}\kern2pt]} \hypertarget{tA\string_Mho\string_M1}{}
\markboth{\textcolor{darkblue}{\textbf{\ipa{tɑ˧ho˧}}}}{}
\textcolor{teal}{\zh{助词}} \hspace{4pt} \zh{声调类:} M.
\zh{一起。} \textcolor{Sepia}{\selectlanguage{english}Together.} \textcolor{PineGreen}{\selectlanguage{french}Ensemble.}  ¶ \textcolor{darkblue}{\textbf{\ipa{ɖɯ˧-ʁwɤ˧ tɑ˧ho˧ kʰi˧˥}}} \zh{全村一起去了。} \textcolor{Sepia}{\selectlanguage{english}The whole village went together.} \textcolor{PineGreen}{\selectlanguage{french}Tout le village y est allé ensemble.}  
 ¶ \textcolor{darkblue}{\textbf{\ipa{tɑ˧ho˧ ʝi˧}}} \zh{一起工作} \textcolor{Sepia}{\selectlanguage{english}to work together} \textcolor{PineGreen}{\selectlanguage{french}travailler ensemble}  
 ¶ \textcolor{darkblue}{\textbf{\ipa{tɑ˧ho˧ tsʰo˧}}} \zh{一起跳舞} \textcolor{Sepia}{\selectlanguage{english}to dance together} \textcolor{PineGreen}{\selectlanguage{french}danser ensemble}  

\lhead{\firstmark}
\rhead{\botmark}

\subsection{\hspace{-0.5cm} {\Large \textcolor{darkblue}{\textbf{\ipa{tɑ˧ko˧}}}}\hspace{0.5cm}[\kern2pt{\textcolor{darkblue}{\textbf{\ipa{tɑ˧ko˧}}}}\kern2pt]} \hypertarget{tA\string_Mko\string_M1}{}
\markboth{\textcolor{darkblue}{\textbf{\ipa{tɑ˧ko˧}}}}{}
\textcolor{teal}{\zh{动词}} \hspace{4pt} \zh{声调类:} M.
\zh{打工(汉语借词)。} \textcolor{Sepia}{\selectlanguage{english}To do manual work, to get a job, to do odd jobs to make some money.} \textcolor{PineGreen}{\selectlanguage{french}Trouver du travail non qualifié (sur un chantier...), gagner de l'argent en faisant des petits boulots.}  \zh{【借词】} \zh{打工}
 ¶ \textcolor{darkblue}{\textbf{\ipa{tɑ˧ko˧ hɯ˧-ze˩!}}} \zh{(他)打工去了!} \textcolor{Sepia}{\selectlanguage{english}(S)he has gone (to the city, to another place...) to do odd jobs to make some money!} \textcolor{PineGreen}{\selectlanguage{french}(Elle/il) est parti(e) gagner de l'argent en faisant des petits boulots!}  

\lhead{\firstmark}
\rhead{\botmark}

\subsection{\hspace{-0.5cm} {\Large \textcolor{darkblue}{\textbf{\ipa{tɑ˧ko˩}}}}\hspace{0.5cm}[\kern2pt{\textcolor{darkblue}{\textbf{\ipa{tɑ˧ko˩}}}}\kern2pt]} \hypertarget{tA\string_Mko\string_B1}{}
\markboth{\textcolor{darkblue}{\textbf{\ipa{tɑ˧ko˩}}}}{}
\textcolor{teal}{\zh{动词}} \hspace{4pt} \zh{声调类:} L\#.
\zh{耽误。} \textcolor{Sepia}{\selectlanguage{english}To delay, to hold up.} \textcolor{PineGreen}{\selectlanguage{french}Retarder.}  ¶ \textcolor{darkblue}{\textbf{\ipa{hĩ˧ tɑ˧ko˥}}} \zh{耽误人家} \textcolor{Sepia}{\selectlanguage{english}to delay people} \textcolor{PineGreen}{\selectlanguage{french}retarder les gens}  
 ¶ \textcolor{darkblue}{\textbf{\ipa{ʈʂʰɯ˧ hĩ˧ tɑ˧ko˥ | ʐwæ˩˥!}}} \zh{他耽误大家很多!} \textcolor{Sepia}{\selectlanguage{english}(S)he delays people a lot!} \textcolor{PineGreen}{\selectlanguage{french}Il/elle retarde tout le monde!}  

\lhead{\firstmark}
\rhead{\botmark}

\subsection{\hspace{-0.5cm} {\Large \textcolor{darkblue}{\textbf{\ipa{tɑ˧nɑ˩}}}}\hspace{0.5cm}[\kern2pt{\textcolor{darkblue}{\textbf{\ipa{tɑ˧nɑ˩}}}}\kern2pt]} \hypertarget{tA\string_MnA\string_B1}{}
\markboth{\textcolor{darkblue}{\textbf{\ipa{tɑ˧nɑ˩}}}}{}
\textcolor{teal}{\zh{名词}} \hspace{4pt} \zh{声调类:} L\#.
\zh{弩弓。} \textcolor{Sepia}{\selectlanguage{english}Crossbow.} \textcolor{PineGreen}{\selectlanguage{french}Arbalète.}  \zh{量词}: \textcolor{darkblue}{\textbf{\ipa{pɤ˩}}} \textcolor{darkblue}{\textbf{\ipa{nɑ˧}}} 
\lhead{\firstmark}
\rhead{\botmark}

\subsection{\hspace{-0.5cm} {\Large \textcolor{darkblue}{\textbf{\ipa{tɑ˧pi˧}}}}\hspace{0.5cm}[\kern2pt{\textcolor{darkblue}{\textbf{\ipa{tɑ˧pi˧}}}}\kern2pt]} \hypertarget{tA\string_Mpi\string_M1}{}
\markboth{\textcolor{darkblue}{\textbf{\ipa{tɑ˧pi˧}}}}{}
\textcolor{teal}{\zh{形容词}} \hspace{4pt} \zh{声调类:} M.
\zh{如、像、像……那样。} \textcolor{Sepia}{\selectlanguage{english}Identical to, like, to the likeness of.} \textcolor{PineGreen}{\selectlanguage{french}Identique à, pareil à, semblable à, à l'exemple de.}  ¶ \textcolor{darkblue}{\textbf{\ipa{no˧-bi˧ tɑ˩pi˩, ...}}} \zh{像你} \textcolor{Sepia}{\selectlanguage{english}like you; following your example} \textcolor{PineGreen}{\selectlanguage{french}selon ton exemple; comme toi; identique à toi}  
 ¶ \textcolor{darkblue}{\textbf{\ipa{njɤ˧-bi˧ tɑ˩pi˩…}}} \zh{像我} \textcolor{Sepia}{\selectlanguage{english}like me; following my example} \textcolor{PineGreen}{\selectlanguage{french}comme moi; à mon exemple}  
 ¶ \textcolor{darkblue}{\textbf{\ipa{no˧=ɻ̍˩-bv̩˩, | njɤ˧=ɻ̍˩-bv̩˩, | tɑ˧pi˧!}}} \zh{你家的房子,我家的房子,都是一样的!(如:一个村子里的房子,都是按同一个模式建设的。)} \textcolor{Sepia}{\selectlanguage{english}Yours and ours are built on the same pattern / are identical! (Context: discussing the farms of the village: they are all built on the same model, in the same way, and thus identical.)} \textcolor{PineGreen}{\selectlanguage{french}Le mien et le tien, ils sont faits sur le même exemple =ils sont pareils! (au sujet de bâtiments, par exemple: les maisons d'un même village sont bâties sur le même modèle)}  
 ¶ \textcolor{darkblue}{\textbf{\ipa{no˧-ɳɯ˧ gv̩˩, | njɤ˧-ɳɯ˧-gv̩˩, | tɑ˧pi˧!}}} \zh{无论是谁来盖房,盖出来的都一样!} \textcolor{Sepia}{\selectlanguage{english}Whether it's you or me who's building [the house], it's the same / the result is the same!} \textcolor{PineGreen}{\selectlanguage{french}que ce soit toi ou moi qui construise [une maison], c'est pareil! / c'est sur le même modèle!}  
 ¶ \textcolor{darkblue}{\textbf{\ipa{ʈʂʰɯ˧-bi˩ | tɑ˧pi˧, | njɤ˧-ɳɯ˧ dɑ˧-bi˥-ze˩!}}} \zh{我要盖跟这一样的房子!} \textcolor{Sepia}{\selectlanguage{english}I am going to build [a house] like that one! / I am going to build [a house] that will be identical to his!} \textcolor{PineGreen}{\selectlanguage{french}je vais construire [une maison] comme celle-là/ je vais imiter cette maison-là! / Je vais construire une maison qui sera pareille à la sienne!}  

\lhead{\firstmark}
\rhead{\botmark}

\subsection{\hspace{-0.5cm} {\Large \textcolor{darkblue}{\textbf{\ipa{tɑ˧pi˧}}}}\hspace{0.5cm}[\kern2pt{\textcolor{darkblue}{\textbf{\ipa{tɑ˧pi˧}}}}\kern2pt]} \hypertarget{tA\string_Mpi\string_M1}{}
\markboth{\textcolor{darkblue}{\textbf{\ipa{tɑ˧pi˧}}}}{}
\textcolor{teal}{\zh{动词}} \hspace{4pt} \zh{声调类:} M.
\zh{打比方(汉语借词:当地汉语方言‘打比’)。} \textcolor{Sepia}{\selectlanguage{english}To take as an example, to draw an analog.} \textcolor{PineGreen}{\selectlanguage{french}Prendre pour exemple.} \zh{当地汉语方言:}\zh{打比。} \zh{【借词】} \zh{打比}
 ¶ \textcolor{darkblue}{\textbf{\ipa{tɑ˧pi˧-ze˩}}} \zh{打比方} \textcolor{Sepia}{\selectlanguage{english}\mytextsc{pfv}} \textcolor{PineGreen}{\selectlanguage{french}\mytextsc{pfv}}  

\lhead{\firstmark}
\rhead{\botmark}

\subsection{\hspace{-0.5cm} {\Large \textcolor{darkblue}{\textbf{\ipa{tɑ˧pv̩˩}}}}\hspace{0.5cm}[\kern2pt{\textcolor{darkblue}{\textbf{\ipa{tɑ˧pv̩˩}}}}\kern2pt]} \hypertarget{tA\string_Mpv\string_=\string_B1}{}
\markboth{\textcolor{darkblue}{\textbf{\ipa{tɑ˧pv̩˩}}}}{}
\textcolor{teal}{\zh{形容词}} \hspace{4pt} \zh{声调类:} L\#.
\zh{晒干的(水果、蔬菜……)。} \textcolor{Sepia}{\selectlanguage{english}Dry (fruit, vegetables), dried in the sun.} \textcolor{PineGreen}{\selectlanguage{french}Séché (au soleil) (ex.: légumes).}  ¶ \textcolor{darkblue}{\textbf{\ipa{v̩˩tsʰɤ˧-tɑ˧pv̩˥}}} \zh{晒干的蔬菜} \textcolor{Sepia}{\selectlanguage{english}dry vegetables, vegetables dried in the sun} \textcolor{PineGreen}{\selectlanguage{french}légumes séchés au soleil}  

\lhead{\firstmark}
\rhead{\botmark}

\subsection{\hspace{-0.5cm} {\Large \textcolor{darkblue}{\textbf{\ipa{tɑ˧pʰi˩}}}}\hspace{0.5cm}[\kern2pt{\textcolor{darkblue}{\textbf{\ipa{tɑ˧pʰi˩}}}}\kern2pt]} \hypertarget{tA\string_Mp\string_hi\string_B1}{}
\markboth{\textcolor{darkblue}{\textbf{\ipa{tɑ˧pʰi˩}}}}{}
\textcolor{teal}{\zh{名词}} \hspace{4pt} \zh{声调类:} L\#.
\zh{艾、艾蒿。} \textcolor{Sepia}{\selectlanguage{english}Chinese mugwort, \textit{Artemisia argyi}.} \textcolor{PineGreen}{\selectlanguage{french}Armoise de Chine, \textit{Artemisia argyi}.}  \zh{量词}: \textcolor{darkblue}{\textbf{\ipa{dzi˩}}} 
\lhead{\firstmark}
\rhead{\botmark}

\subsection{\hspace{-0.5cm} {\Large \textcolor{darkblue}{\textbf{\ipa{tɑ˧\textasciitilde{}tɑ˧}}} \textsubscript{1}}\hspace{0.5cm}[\kern2pt{\textcolor{darkblue}{\textbf{\ipa{tɑ˧tɑ˧}}}}\kern2pt]} \hypertarget{tA\string_M~tA\string_M1}{}
\markboth{\textcolor{darkblue}{\textbf{\ipa{tɑ˧\textasciitilde{}tɑ˧}}} \textsubscript{1}}{}
\textcolor{teal}{\zh{形容词}} \hspace{4pt} \zh{声调类:} M.
\zh{严肃认真、细心、细致,(看得)清楚、清晰。} \textcolor{Sepia}{\selectlanguage{english}Serious, reliable, careful; clear (to see clearly).} \textcolor{PineGreen}{\selectlanguage{french}Sérieux, attentif, soigneux; exact, précis (une chaussure convient précisément à un pied; quelqu'un observe avec précision/exactitude).}  ¶ \textcolor{darkblue}{\textbf{\ipa{mɤ˧-tɑ˧\textasciitilde{}tɑ˧}}} \zh{邋遢、草率、潦草} \textcolor{Sepia}{\selectlanguage{english}sloppy, not careful} \textcolor{PineGreen}{\selectlanguage{french}pas sérieux, négligé (au sujet d'un travail)}  
 ¶ \textcolor{darkblue}{\textbf{\ipa{lo˧ ʝi˧ mɤ˧-tɑ˧\textasciitilde{}tɑ˧}}} \zh{工作草率} \textcolor{Sepia}{\selectlanguage{english}to do sloppy work} \textcolor{PineGreen}{\selectlanguage{french}travailler sans soin, de façon négligée}  
 ¶ \textcolor{darkblue}{\textbf{\ipa{hĩ˧ ʈʂʰɯ˧-v̩˧, | tɑ˧\textasciitilde{}tɑ˧!}}} \zh{他很认真!} \textcolor{Sepia}{\selectlanguage{english}(S)he works carefully!} \textcolor{PineGreen}{\selectlanguage{french}lui, il est soigneux!}  
\zh{~【参考】~} \hyperlink{}{\textcolor{darkblue}{\textbf{\ipa{tɑ˧\textasciitilde{}tɑ˧}}} \textsubscript{2}} 
\lhead{\firstmark}
\rhead{\botmark}

\subsection{\hspace{-0.5cm} {\Large \textcolor{darkblue}{\textbf{\ipa{tɑ˧\textasciitilde{}tɑ˧}}} \textsubscript{2}}\hspace{0.5cm}[\kern2pt{\textcolor{darkblue}{\textbf{\ipa{tɑ˧tɑ˧}}}}\kern2pt]} \hypertarget{tA\string_M~tA\string_M2}{}
\markboth{\textcolor{darkblue}{\textbf{\ipa{tɑ˧\textasciitilde{}tɑ˧}}} \textsubscript{2}}{}
\textcolor{teal}{\zh{助词}} \hspace{4pt} \zh{声调类:} M.
\zh{刚(好)、正(好)。} \textcolor{Sepia}{\selectlanguage{english}Exactly (right), just (right).} \textcolor{PineGreen}{\selectlanguage{french}Précisément, justement, juste à point nommé.}  ¶ \textcolor{darkblue}{\textbf{\ipa{tɑ˧\textasciitilde{}tɑ˧ | ho˩˥! |}}} \zh{刚刚好!(如:一双鞋刚好合适)} \textcolor{Sepia}{\selectlanguage{english}Just right, exactly right. (Example: a pair of shoes fits perfectly.)} \textcolor{PineGreen}{\selectlanguage{french}Ca convient exactement/ça convient précisément (ex.: au sujet d'une paire de chaussures qu'on vient de vous offrir)}  
 ¶ \textcolor{darkblue}{\textbf{\ipa{le˧-li˧ tɑ˧\textasciitilde{}tɑ˧}}} \zh{看清楚} \textcolor{Sepia}{\selectlanguage{english}to see clearly} \textcolor{PineGreen}{\selectlanguage{french}voir clairement}  
\zh{~【参考】~} \hyperlink{}{\textcolor{darkblue}{\textbf{\ipa{tɑ˧\textasciitilde{}tɑ˧}}} \textsubscript{1}} 
\lhead{\firstmark}
\rhead{\botmark}

\subsection{\hspace{-0.5cm} {\Large \textcolor{darkblue}{\textbf{\ipa{tɑ˩\textsubscript{a}}}}}\hspace{0.5cm}[\kern2pt{\textcolor{darkblue}{\textbf{\ipa{tɑ˧˥}}}}\kern2pt]} \hypertarget{tA\string_Ba1}{}
\markboth{\textcolor{darkblue}{\textbf{\ipa{tɑ˩\textsubscript{a}}}}}{}
\textcolor{teal}{\zh{量词}} \hspace{4pt} \zh{声调类:} L\textsubscript{a}.
\zh{量词:钱(一笔)。} \textcolor{Sepia}{\selectlanguage{english}Classifier for sums of money.} \textcolor{PineGreen}{\selectlanguage{french}Classificateur des fortes sommes d'argent.}  ¶ \textcolor{darkblue}{\textbf{\ipa{ɖʐe˧ | ɖɯ˧-tɑ˩}}} \zh{一笔钱} \textcolor{Sepia}{\selectlanguage{english}a (big) sum of money} \textcolor{PineGreen}{\selectlanguage{french}un paquet d'argent, une liasse de billets…}  

\lhead{\firstmark}
\rhead{\botmark}

\subsection{\hspace{-0.5cm} {\Large \textcolor{darkblue}{\textbf{\ipa{tɑ˩dv̩˧˥}}}}\hspace{0.5cm}[\kern2pt{\textcolor{darkblue}{\textbf{\ipa{tɑ˩dv̩˥}}}}\kern2pt]} \hypertarget{tA\string_Bdv\string_=\string_M\string_T1}{}
\markboth{\textcolor{darkblue}{\textbf{\ipa{tɑ˩dv̩˧˥}}}}{}
\textcolor{teal}{\zh{名词}} \hspace{4pt} \zh{声调类:} LM+MH\#.
\zh{口袋、衣袋、兜子。} \textcolor{Sepia}{\selectlanguage{english}Pocket.} \textcolor{PineGreen}{\selectlanguage{french}Poche.}  ¶ \textcolor{darkblue}{\textbf{\ipa{njɤ˧ | tɑ˩dv̩˧-qo˥ | tsʰe˩mæ˩-tɑ˥kɤ˩-lɑ˩ dʑo˩!}}} \zh{我兜子里只有十元钱!} \textcolor{Sepia}{\selectlanguage{english}I only have ten yuan in my pocket!} \textcolor{PineGreen}{\selectlanguage{french}Je n’ai que dix yuan en poche!}  
 \zh{量词}: \textcolor{darkblue}{\textbf{\ipa{ɭɯ˧}}} 
\lhead{\firstmark}
\rhead{\botmark}

\subsection{\hspace{-0.5cm} {\Large \textcolor{darkblue}{\textbf{\ipa{tɑ˩dʑɤ\#˥}}}}\hspace{0.5cm}[\kern2pt{\textcolor{darkblue}{\textbf{\ipa{tɑ˩dʑɤ˧˥}}}}\kern2pt]} \hypertarget{tA\string_Bdz£7\#\string_T1}{}
\markboth{\textcolor{darkblue}{\textbf{\ipa{tɑ˩dʑɤ\#˥}}}}{}
\textcolor{teal}{\zh{名词}} \hspace{4pt} \zh{声调类:} LM+\#H.
\zh{男性名字,双胞胎中老二的名字。} \textcolor{Sepia}{\selectlanguage{english}Masculine given name given to the second among twins.} \textcolor{PineGreen}{\selectlanguage{french}Prénom masculin employé pour le second des jumeaux.} 
\lhead{\firstmark}
\rhead{\botmark}

\subsection{\hspace{-0.5cm} {\Large \textcolor{darkblue}{\textbf{\ipa{tɑ˩ɖʐo˧dzi˧˥}}}}\hspace{0.5cm}[\kern2pt{\textcolor{darkblue}{\textbf{\ipa{tɑ˧ɖʐo˧dzi˩}}}}\kern2pt]} \hypertarget{tA\string_Bd`z`o\string_Mdzi\string_M\string_T1}{}
\markboth{\textcolor{darkblue}{\textbf{\ipa{tɑ˩ɖʐo˧dzi˧˥}}}}{}
\textcolor{teal}{\zh{名词}} \hspace{4pt} \zh{声调类:} LM+MH\#.
\zh{小经幡。} \textcolor{Sepia}{\selectlanguage{english}Small prayer flag.} \textcolor{PineGreen}{\selectlanguage{french}Petit drapeau de prière.}  \zh{量词}: \textcolor{darkblue}{\textbf{\ipa{dzi˩}}} 
\lhead{\firstmark}
\rhead{\botmark}

\subsection{\hspace{-0.5cm} {\Large \textcolor{darkblue}{\textbf{\ipa{tɑ˩hwɤ˩}}}}\hspace{0.5cm}[\kern2pt{\textcolor{darkblue}{\textbf{\ipa{tɑ˧hwɤ˧}}}}\kern2pt]} \hypertarget{tA\string_Bhw7\string_B1}{}
\markboth{\textcolor{darkblue}{\textbf{\ipa{tɑ˩hwɤ˩}}}}{}
\textcolor{teal}{\zh{动词}} \hspace{4pt} \zh{声调类:} L.
 \zh{【借词】} \zh{打发?}
\ding{202} \zh{送礼(给家里以外的人)。} \textcolor{Sepia}{\selectlanguage{english}To offer gifts outside the family circle.} \textcolor{PineGreen}{\selectlanguage{french}Offrir des présents à des gens extérieurs à la famille.}  ¶ \textcolor{darkblue}{\textbf{\ipa{hĩ˧-ki˧ | ɖɯ˧-kʰwɤ˧ tɑ˥hwɤ˩-zo˩-ʝi˩!}}} \zh{应该给人家送礼了!(例如,人家为孩子进行成年礼时,要送礼。)} \textcolor{Sepia}{\selectlanguage{english}We shall have to make a present to people! / It's going to be an occasion to make a present to people! (For instance, when a child goes through the “Coming of age” rite.)} \textcolor{PineGreen}{\selectlanguage{french}Il va falloir faire un présent aux gens! / Ca va être l'occasion de faire un présent aux gens! (par exemple à l'occasion d'un rituel de passage à l'âge adulte)}  
 ¶ \textcolor{darkblue}{\textbf{\ipa{zo˧mv̩˥-ki˩, | tɑ˩hwɤ˩ mɤ˥-kv̩˩!}}} \zh{不会专门给孩子送(大)礼的!(说明:送礼,是送给家里的主人)} \textcolor{Sepia}{\selectlanguage{english}Presents are not for the kids! / We don't give big presents to children!} \textcolor{PineGreen}{\selectlanguage{french}On ne fait pas de présents aux enfants! (Explication: le présent donné par une famille à une autre de façon ritualisée est offert aux aînés, pas aux enfants; c'est d'une nature différente des petits cadeaux qu'on peut leur faire au quotidien.)}  
 ¶ \textcolor{darkblue}{\textbf{\ipa{ʐɯ˧ tɑ˩hwɤ˩}}} \zh{送酒(作为礼物)} \textcolor{Sepia}{\selectlanguage{english}to offer wine as a present} \textcolor{PineGreen}{\selectlanguage{french}offrir du vin comme présent}  
 ¶ \textcolor{darkblue}{\textbf{\ipa{li˩ tɑ˥hwɤ˩}}} \zh{送茶(作为礼物)} \textcolor{Sepia}{\selectlanguage{english}to offer tea as a present} \textcolor{PineGreen}{\selectlanguage{french}offrir du thé comme présent}  
 ¶ \textcolor{darkblue}{\textbf{\ipa{dze˧ tɑ˥hwɤ˩}}} \zh{送糖(作为礼物)} \textcolor{Sepia}{\selectlanguage{english}to offer sweets as a present} \textcolor{PineGreen}{\selectlanguage{french}offrir des sucreries/des bonbons comme présent}  
\ding{203} \zh{送陪嫁(嫁妆、陪奁)。} \textcolor{Sepia}{\selectlanguage{english}To give a dowry: to give goods to a young woman when she goes to her new home after her wedding. The dowry used to be brought on horseback, in two wood boxes: gifts must come in pairs, and the dowry is no exception.} \textcolor{PineGreen}{\selectlanguage{french}Fournir la dot: donner des biens à une jeune femme lorsqu'elle rejoint sa nouvelle famille lors du mariage. (Note: la dot est apportée à dos de cheval; elle est rangée dans deux caisses en bois: comme en d'autres circonstances, les présents doivent aller par deux.).}  ¶ \textcolor{darkblue}{\textbf{\ipa{ə˧tso˧ tɑ˩hwɤ˩-ʝi˩? | ə˧-sɯ˩kv̩˩ (-dʑo˩), | ɖɯ˧-li˧-ɻ̍˩-bi˩!}}} \zh{给的是什么嫁妆?咱们去看一看吧!(结婚的时候,陪嫁展示在大家眼前,显示女方家的大方程度)} \textcolor{Sepia}{\selectlanguage{english}What did they give as a dowry? Let's go and have a look! (At a wedding, the gifts given as a dowry are put on public display, for everyone to appreciate the parents' generosity.)} \textcolor{PineGreen}{\selectlanguage{french}Qu'est-ce qu'ils ont donné comme dot? Regardons un peu! / Allons voir! (Ce que disent les villageois invités à un mariage; les biens offerts en dot sont alors exposés, de façon à ce que chacun puisse apprécier la générosité des parents.)}  
 ¶ \textcolor{darkblue}{\textbf{\ipa{ti˧tsɯ˥ | qʰɑ˧-ɭɯ˧ tɑ˩hwɤ˩? - ti˧tsɯ˥ | ɲi˧-ɭɯ˧ tɑ˩hwɤ˩!}}} \zh{陪嫁有几个木箱? - 陪嫁有两个(木箱)!} \textcolor{Sepia}{\selectlanguage{english}How many boxes are there in the dowry? - The dowry consists of two boxes!} \textcolor{PineGreen}{\selectlanguage{french}Combien de caisses sont offertes en dot/ de combien de caisses la dot se compose-t-elle? - De deux caisses!}  

\lhead{\firstmark}
\rhead{\botmark}

\subsection{\hspace{-0.5cm} {\Large \textcolor{darkblue}{\textbf{\ipa{tɑ˩kɤ˧}}}}\hspace{0.5cm}[\kern2pt{\textcolor{darkblue}{\textbf{\ipa{tɑ˩kɤ˩˥}}}}\kern2pt]} \hypertarget{tA\string_Bk7\string_M1}{}
\markboth{\textcolor{darkblue}{\textbf{\ipa{tɑ˩kɤ˧}}}}{}
\textcolor{teal}{\zh{动词}} \hspace{4pt} \zh{声调类:} .
\zh{逗弄(动作)。} \textcolor{Sepia}{\selectlanguage{english}To tease (by gestures).} \textcolor{PineGreen}{\selectlanguage{french}Taquiner.} 
\lhead{\firstmark}
\rhead{\botmark}

\subsection{\hspace{-0.5cm} {\Large \textcolor{darkblue}{\textbf{\ipa{tɑ˩li˥}}}}\hspace{0.5cm}[\kern2pt{\textcolor{darkblue}{\textbf{\ipa{tɑ˩li˥}}}}\kern2pt]} \hypertarget{tA\string_Bli\string_T1}{}
\markboth{\textcolor{darkblue}{\textbf{\ipa{tɑ˩li˥}}}}{}
\textcolor{teal}{\zh{名词}} \hspace{4pt} \zh{声调类:} LH.
\zh{大理(汉语借词)。} \textcolor{Sepia}{\selectlanguage{english}Dali (city name).} \textcolor{PineGreen}{\selectlanguage{french}Dali (nom de ville).}  \zh{【借词】} \zh{大理}

\lhead{\firstmark}
\rhead{\botmark}

\subsection{\hspace{-0.5cm} {\Large \textcolor{darkblue}{\textbf{\ipa{tɑ˩mv̩˩}}}}\hspace{0.5cm}[\kern2pt{\textcolor{darkblue}{\textbf{\ipa{tɑ˩mv̩˩˥}}}}\kern2pt]} \hypertarget{tA\string_Bmv\string_=\string_B1}{}
\markboth{\textcolor{darkblue}{\textbf{\ipa{tɑ˩mv̩˩}}}}{}
\textcolor{teal}{\zh{动词}} \hspace{4pt} \zh{声调类:} L.
\zh{谚语。} \textcolor{Sepia}{\selectlanguage{english}Proverb.} \textcolor{PineGreen}{\selectlanguage{french}Proverbe.}  ¶ \textcolor{darkblue}{\textbf{\ipa{æ˧ʂæ˧-tɑ˩mv̩˩}}} \zh{同上:谚语(直译:‘从前的老话’)} \textcolor{Sepia}{\selectlanguage{english}same meaning: proverb (literally 'proverb of yore')} \textcolor{PineGreen}{\selectlanguage{french}même sens: proverbe (littéralement: 'proverbe ancien')}  
 ¶ \textcolor{darkblue}{\textbf{\ipa{[F5] æ˧ʂæ˧-tɑ˥mv̩˩}}} \zh{谚语、传统故事} \textcolor{Sepia}{\selectlanguage{english}proverb; traditional story} \textcolor{PineGreen}{\selectlanguage{french}proverbe; histoire ancienne}  

\lhead{\firstmark}
\rhead{\botmark}

\subsection{\hspace{-0.5cm} {\Large \textcolor{darkblue}{\textbf{\ipa{tɑ˩so˩kʰo˥}}}}\hspace{0.5cm}[\kern2pt{\textcolor{darkblue}{\textbf{\ipa{tɑ˩so˩kʰo˥}}}}\kern2pt]} \hypertarget{tA\string_Bso\string_Bk\string_ho\string_T1}{}
\markboth{\textcolor{darkblue}{\textbf{\ipa{tɑ˩so˩kʰo˥}}}}{}
\textcolor{teal}{\zh{助词}} \hspace{4pt} \zh{声调类:} L+H\#.
\zh{盘腿(而坐)。} \textcolor{Sepia}{\selectlanguage{english}Crosslegged (bodily posture).} \textcolor{PineGreen}{\selectlanguage{french}En tailleur (posture assise).}  ¶ \textcolor{darkblue}{\textbf{\ipa{tɑ˩so˩kʰo˥ | tʰi˧-dzi˩}}} \zh{打坐、盘腿而坐(和尚的坐姿)} \textcolor{Sepia}{\selectlanguage{english}To sit crosslegged. (This is the usual posture for monks, and also a usual posture for commoners.)} \textcolor{PineGreen}{\selectlanguage{french}être assis en tailleur (posture assise des moines, et posture également courante chez les gens du commun)}  

\lhead{\firstmark}
\rhead{\botmark}

\subsection{\hspace{-0.5cm} {\Large \textcolor{darkblue}{\textbf{\ipa{tɑ˧˥}}}}\hspace{0.5cm}[\kern2pt{\textcolor{darkblue}{\textbf{\ipa{tɑ˧˥}}}}\kern2pt]} \hypertarget{tA\string_M\string_T1}{}
\markboth{\textcolor{darkblue}{\textbf{\ipa{tɑ˧˥}}}}{}
\textcolor{teal}{\zh{动词}} \hspace{4pt} \zh{声调类:} MH.
\zh{退后。} \textcolor{Sepia}{\selectlanguage{english}To give way, to fall backward.} \textcolor{PineGreen}{\selectlanguage{french}Reculer, se retirer.}  ¶ \textcolor{darkblue}{\textbf{\ipa{ʁo˧tʰo˩ tɑ˩}}} \zh{往后退} \textcolor{Sepia}{\selectlanguage{english}to give way, to fall backward} \textcolor{PineGreen}{\selectlanguage{french}reculer, se retirer vers l'arrière}  

\lhead{\firstmark}
\rhead{\botmark}

\subsection{\hspace{-0.5cm} {\Large \textcolor{darkblue}{\textbf{\ipa{tɑ˧˥\textsubscript{a}}}}}\hspace{0.5cm}[\kern2pt{\textcolor{darkblue}{\textbf{\ipa{tɑ˥}}}}\kern2pt]} \hypertarget{tA\string_M\string_Ta1}{}
\markboth{\textcolor{darkblue}{\textbf{\ipa{tɑ˧˥\textsubscript{a}}}}}{}
\textcolor{teal}{\zh{量词}} \hspace{4pt} \zh{声调类:} MH\textsubscript{a}.
\zh{量词:全部、一切,大家。} \textcolor{Sepia}{\selectlanguage{english}Entirely, all; everyone.} \textcolor{PineGreen}{\selectlanguage{french}Entièrement, tout, tout le monde.}  ¶ \textcolor{darkblue}{\textbf{\ipa{ɖɯ˧-tɑ˧˥}}} \zh{全部、一切,大家} \textcolor{Sepia}{\selectlanguage{english}entirely, all; everyone} \textcolor{PineGreen}{\selectlanguage{french}entièrement, tout, tout le monde}  
 ¶ \textcolor{darkblue}{\textbf{\ipa{ɖɯ˧-tɑ˧=ɻæ˥}}} \zh{全部、一切,大家(同上,加上多数词素)} \textcolor{Sepia}{\selectlanguage{english}entirely, all; everyone (same meaning as above, with a plural morpheme)} \textcolor{PineGreen}{\selectlanguage{french}entièrement, tout, tout le monde (même sens que ci-dessus, avec le morphème de pluriel)}  

\lhead{\firstmark}
\rhead{\botmark}

\subsection{\hspace{-0.5cm} {\Large \textcolor{darkblue}{\textbf{\ipa{tɑ˩˥fv˩˥}}}}\hspace{0.5cm}[\kern2pt{\textcolor{darkblue}{\textbf{\ipa{tɑ˩fv˥}}}}\kern2pt]} \hypertarget{tA\string_B\string_Tfv\string_B\string_T1}{}
\markboth{\textcolor{darkblue}{\textbf{\ipa{tɑ˩˥fv˩˥}}}}{}
\textcolor{teal}{\zh{名词}} \hspace{4pt} \zh{声调类:} LH.LH.
\zh{大粪(汉语借词)。} \textcolor{Sepia}{\selectlanguage{english}Excrement.} \textcolor{PineGreen}{\selectlanguage{french}Excrément.}  \zh{【借词】} \zh{大粪}

\lhead{\firstmark}
\rhead{\botmark}

\subsection{\hspace{-0.5cm} {\Large \textcolor{darkblue}{\textbf{\ipa{tæ˧pv̩˩}}}}\hspace{0.5cm}[\kern2pt{\textcolor{darkblue}{\textbf{\ipa{tæ˧pv̩˩}}}}\kern2pt]} \hypertarget{t\{\string_Mpv\string_=\string_B1}{}
\markboth{\textcolor{darkblue}{\textbf{\ipa{tæ˧pv̩˩}}}}{}
\textcolor{teal}{\zh{形容词}} \hspace{4pt} \zh{声调类:} L\#.
\zh{瘦(人很瘦)。} \textcolor{Sepia}{\selectlanguage{english}Skinny, thin (person).} \textcolor{PineGreen}{\selectlanguage{french}Maigre, sec (personne maigre, au corps sec).}  ¶ \textcolor{darkblue}{\textbf{\ipa{v˩tsʰɤ˧˥ | le˧-tæ˥pv˩ kʰɯ˩}}} \zh{将蔬菜弄干(晒干)} \textcolor{Sepia}{\selectlanguage{english}to dry vegetables} \textcolor{PineGreen}{\selectlanguage{french}faire sécher des légumes}  
 ¶ \textcolor{darkblue}{\textbf{\ipa{v˩tsʰɤ˧˥ | tæ˧pv˩ gv˩}}} \zh{将蔬菜弄干(晒干)} \textcolor{Sepia}{\selectlanguage{english}to dry vegetables} \textcolor{PineGreen}{\selectlanguage{french}faire sécher des légumes}  
\zh{~【参考】~} \hyperlink{}{\textcolor{darkblue}{\textbf{\ipa{tɑ˧gɤ˩}}}} 
\lhead{\firstmark}
\rhead{\botmark}

\subsection{\hspace{-0.5cm} {\Large \textcolor{darkblue}{\textbf{\ipa{tæ˧ɻæ˩}}}}\hspace{0.5cm}[\kern2pt{\textcolor{darkblue}{\textbf{\ipa{tæ˧ɻæ˩}}}}\kern2pt]} \hypertarget{t\{\string_Mr£`\{\string_B1}{}
\markboth{\textcolor{darkblue}{\textbf{\ipa{tæ˧ɻæ˩}}}}{}
\textcolor{teal}{\zh{名词}} \hspace{4pt} \zh{声调类:} L\#.
\zh{喉管、喉结。} \textcolor{Sepia}{\selectlanguage{english}Oesophagus; Adam's apple.} \textcolor{PineGreen}{\selectlanguage{french}Pomme d'Adam; larynx, gorge, oesophage.}  \zh{量词}: \textcolor{darkblue}{\textbf{\ipa{ɭɯ˧}}} 
\lhead{\firstmark}
\rhead{\botmark}

\subsection{\hspace{-0.5cm} {\Large \textcolor{darkblue}{\textbf{\ipa{ti˧}}}}\hspace{0.5cm}[\kern2pt{\textcolor{darkblue}{\textbf{\ipa{ti˥}}}}\kern2pt]} \hypertarget{ti\string_M1}{}
\markboth{\textcolor{darkblue}{\textbf{\ipa{ti˧}}}}{}
\textcolor{teal}{\zh{动词}} \hspace{4pt} \zh{声调类:} M.
\zh{成熟(人成熟)。} \textcolor{Sepia}{\selectlanguage{english}To become mature, to become an adult.} \textcolor{PineGreen}{\selectlanguage{french}Mûrir, devenir adulte (d'une personne).}  ¶ \textcolor{darkblue}{\textbf{\ipa{hĩ˧ tʰv̩˧-v̩˧, | gɤ˩-ti˧-ze˧!}}} \zh{这个人,成熟了! / 是大人了!} \textcolor{Sepia}{\selectlanguage{english}This person has become an adult! / This person has grown up/has become mature!} \textcolor{PineGreen}{\selectlanguage{french}Cette personne a grandi / est devenue adulte / a mûri!}  
 ¶ \textcolor{darkblue}{\textbf{\ipa{hĩ˧ tʰv̩˧-v̩˧, | mɤ˧-ti˧-sɯ˩!}}} \zh{这个人,还不成熟!} \textcolor{Sepia}{\selectlanguage{english}This person is not mature yet! / This person is not an adult yet!} \textcolor{PineGreen}{\selectlanguage{french}Cette personne n'est pas encore adulte!}  
 ¶ \textcolor{darkblue}{\textbf{\ipa{zo˩mv̩˧ | gɤ˩-ti˧, | lo˧ hɑ˧!}}} \zh{孩子长成熟(的过程),还是挺难的!} \textcolor{Sepia}{\selectlanguage{english}For a child to grow/to become an adult is no easy business! (Refers to difficulty both for the child and for the family)} \textcolor{PineGreen}{\selectlanguage{french}La croissance d'un enfant (jusqu'à l'âge adulte), c'est pas facile! (La difficulté est pour les parents, et aussi pour l'enfant)}  

\lhead{\firstmark}
\rhead{\botmark}

\subsection{\hspace{-0.5cm} {\Large \textcolor{darkblue}{\textbf{\ipa{ti˧ɖo˥}}}}\hspace{0.5cm}[\kern2pt{\textcolor{darkblue}{\textbf{\ipa{ti˩ɖo˩˥}}}}\kern2pt]} \hypertarget{ti\string_Md`o\string_T1}{}
\markboth{\textcolor{darkblue}{\textbf{\ipa{ti˧ɖo˥}}}}{}
\textcolor{teal}{\zh{名词}} \hspace{4pt} \zh{声调类:} H\#.
\zh{男性名字。} \textcolor{Sepia}{\selectlanguage{english}Masculine given name.} \textcolor{PineGreen}{\selectlanguage{french}Prénom masculin.} 
\lhead{\firstmark}
\rhead{\botmark}

\subsection{\hspace{-0.5cm} {\Large \textcolor{darkblue}{\textbf{\ipa{ti˧pʰv̩\#˥}}}}\hspace{0.5cm}[\kern2pt{\textcolor{darkblue}{\textbf{\ipa{ti˩pʰv̩˩˥}}}}\kern2pt]} \hypertarget{ti\string_Mp\string_hv\string_=\#\string_T1}{}
\markboth{\textcolor{darkblue}{\textbf{\ipa{ti˧pʰv̩\#˥}}}}{}
\textcolor{teal}{\zh{名词}} \hspace{4pt} \zh{声调类:} \#H.
\zh{铜杯盏,做仪式用的。} \textcolor{Sepia}{\selectlanguage{english}Copper cup for offerings.} \textcolor{PineGreen}{\selectlanguage{french}Coupe de cuivre pour les offrandes; elle est évasée, et de la taille d'un petit gobelet à alcool.}  \zh{量词}: \textcolor{darkblue}{\textbf{\ipa{fv̩˩}}} 
\lhead{\firstmark}
\rhead{\botmark}

\subsection{\hspace{-0.5cm} {\Large \textcolor{darkblue}{\textbf{\ipa{ti˧tsɯ˥}}}}\hspace{0.5cm}[\kern2pt{\textcolor{darkblue}{\textbf{\ipa{ti˩tsɯ˥}}}}\kern2pt]} \hypertarget{ti\string_MtsM\string_T1}{}
\markboth{\textcolor{darkblue}{\textbf{\ipa{ti˧tsɯ˥}}}}{}
\textcolor{teal}{\zh{名词}} \hspace{4pt} \zh{声调类:} H\#.
\zh{竹箱。} \textcolor{Sepia}{\selectlanguage{english}Box (woven out of bamboo or wicker).} \textcolor{PineGreen}{\selectlanguage{french}Boîte en vannerie (objet qui n'est plus en usage aujourd'hui).}  \zh{量词}: \textcolor{darkblue}{\textbf{\ipa{ɭɯ˧}}} 
\lhead{\firstmark}
\rhead{\botmark}

\subsection{\hspace{-0.5cm} {\Large \textcolor{darkblue}{\textbf{\ipa{ti˧ʈʂʰɯ˩}}}}\hspace{0.5cm}[\kern2pt{\textcolor{darkblue}{\textbf{\ipa{ti˧ʈʂʰɯ˥}}}}\kern2pt]} \hypertarget{ti\string_Mt`s`\string_hM\string_B1}{}
\markboth{\textcolor{darkblue}{\textbf{\ipa{ti˧ʈʂʰɯ˩}}}}{}
\textcolor{teal}{\zh{名词}} \hspace{4pt} \zh{声调类:} L\#.
\zh{铁锤。} \textcolor{Sepia}{\selectlanguage{english}Hammer.} \textcolor{PineGreen}{\selectlanguage{french}Marteau.}  \zh{量词}: \textcolor{darkblue}{\textbf{\ipa{ɭɯ˧}}} 
\lhead{\firstmark}
\rhead{\botmark}

\subsection{\hspace{-0.5cm} {\Large \textcolor{darkblue}{\textbf{\ipa{ti˩\textsubscript{a}}}}}\hspace{0.5cm}[\kern2pt{\textcolor{darkblue}{\textbf{\ipa{ti˩˥}}}}\kern2pt]} \hypertarget{ti\string_Ba1}{}
\markboth{\textcolor{darkblue}{\textbf{\ipa{ti˩\textsubscript{a}}}}}{}
\textcolor{teal}{\zh{动词}} \hspace{4pt} \zh{声调类:} L\textsubscript{a}.
\ding{202} \zh{捣(花椒、大蒜……)。} \textcolor{Sepia}{\selectlanguage{english}To pound, e.g. pounding Szechuan pepper with a small metal pestle, or pounding earth to build a wall of earth.} \textcolor{PineGreen}{\selectlanguage{french}Piler: réduire quelque chose en poudre dans un mortier, par des coups répétés; comprimer de la terre pour former un mur de terre.}  ¶ \textcolor{darkblue}{\textbf{\ipa{læ˧tsɯ˥ ti˩}}} \zh{捣辣椒} \textcolor{Sepia}{\selectlanguage{english}to pound hot peppers} \textcolor{PineGreen}{\selectlanguage{french}piler le piment, réduire le piment en poudre}  
 ¶ \textcolor{darkblue}{\textbf{\ipa{tsʰo˧ko˧ ti˩}}} \zh{捣草果} \textcolor{Sepia}{\selectlanguage{english}to pound cardamom} \textcolor{PineGreen}{\selectlanguage{french}piler la cardamome, réduire de cardamome en poudre}  
 ¶ \textcolor{darkblue}{\textbf{\ipa{dze˩ ti˥}}} \zh{捣花椒} \textcolor{Sepia}{\selectlanguage{english}to pound Szechuan pepper} \textcolor{PineGreen}{\selectlanguage{french}piler le xanthoxyle, réduire le xanthoxyle en poudre}  
 ¶ \textcolor{darkblue}{\textbf{\ipa{ʈʂo˩bo˩ ti˥}}} \zh{垒土墙} \textcolor{Sepia}{\selectlanguage{english}to build a wall of earth, by pounding the earth} \textcolor{PineGreen}{\selectlanguage{french}construire un mur en terre en comprimant la terre à coups de masse}  
\ding{203} \zh{拍打。} \textcolor{Sepia}{\selectlanguage{english}To hit, to strike lightly.} \textcolor{PineGreen}{\selectlanguage{french}Donner une tape, tapoter, frapper quelqu'un légèrement.}  ¶ \textcolor{darkblue}{\textbf{\ipa{hĩ˧ ti˥}}} \zh{拍打人} \textcolor{Sepia}{\selectlanguage{english}to slap someone, to hit someone mildly} \textcolor{PineGreen}{\selectlanguage{french}donner une tape à quelqu'un}  
 ¶ \textcolor{darkblue}{\textbf{\ipa{hĩ˧ | ɖɯ˧-v̩˧ ti˩-ze˩}}} \zh{(他)拍打了某人。} \textcolor{Sepia}{\selectlanguage{english}(She/he) has slapped someone.} \textcolor{PineGreen}{\selectlanguage{french}(Elle/il) a donné une tape à quelqu'un.}  

\lhead{\firstmark}
\rhead{\botmark}

\subsection{\hspace{-0.5cm} {\Large \textcolor{darkblue}{\textbf{\ipa{ti˩pʰo˩}}}}\hspace{0.5cm}[\kern2pt{\textcolor{darkblue}{\textbf{\ipa{ti˩pʰo˥}}}}\kern2pt]} \hypertarget{ti\string_Bp\string_ho\string_B1}{}
\markboth{\textcolor{darkblue}{\textbf{\ipa{ti˩pʰo˩}}}}{}
\textcolor{teal}{\zh{名词}} \hspace{4pt} \zh{声调类:} L.
\zh{天花板。} \textcolor{Sepia}{\selectlanguage{english}Ceiling.} \textcolor{PineGreen}{\selectlanguage{french}Plafond.}  \zh{量词}: \textcolor{darkblue}{\textbf{\ipa{nɑ˧}}} 
\lhead{\firstmark}
\rhead{\botmark}

\subsection{\hspace{-0.5cm} {\Large \textcolor{darkblue}{\textbf{\ipa{ti˩tje˧}}}}\hspace{0.5cm}[\kern2pt{\textcolor{darkblue}{\textbf{\ipa{ti˧tje˧}}}}\kern2pt]} \hypertarget{ti\string_Btje\string_M1}{}
\markboth{\textcolor{darkblue}{\textbf{\ipa{ti˩tje˧}}}}{}
\textcolor{teal}{\zh{动词}} \hspace{4pt} \zh{声调类:} LM.
\zh{对待(汉语借词)。} \textcolor{Sepia}{\selectlanguage{english}To treat, to handle (someone).} \textcolor{PineGreen}{\selectlanguage{french}Traiter (quelqu'un).}  \zh{【借词】} \zh{对待}

\lhead{\firstmark}
\rhead{\botmark}

\subsection{\hspace{-0.5cm} {\Large \textcolor{darkblue}{\textbf{\ipa{ti˧˥}}}}\hspace{0.5cm}[\kern2pt{\textcolor{darkblue}{\textbf{\ipa{ti˧˥}}}}\kern2pt]} \hypertarget{ti\string_M\string_T1}{}
\markboth{\textcolor{darkblue}{\textbf{\ipa{ti˧˥}}}}{}
\textcolor{teal}{\zh{动词}} \hspace{4pt} \zh{声调类:} MH.
\zh{决定(汉语借词:定)。} \textcolor{Sepia}{\selectlanguage{english}To settle, to decide (Chinese borrowing).} \textcolor{PineGreen}{\selectlanguage{french}Décider, fixer, arrêter.}  \zh{【借词】} \zh{定}

\lhead{\firstmark}
\rhead{\botmark}

\subsection{\hspace{-0.5cm} {\Large \textcolor{darkblue}{\textbf{\ipa{ti˧˥\textsubscript{a}}}}}\hspace{0.5cm}[\kern2pt{\textcolor{darkblue}{\textbf{\ipa{ti˩˥}}}}\kern2pt]} \hypertarget{ti\string_M\string_Ta1}{}
\markboth{\textcolor{darkblue}{\textbf{\ipa{ti˧˥\textsubscript{a}}}}}{}
\textcolor{teal}{\zh{量词}} \hspace{4pt} \zh{声调类:} MH\textsubscript{a}.
\zh{量词:层(一层灰、一层木板……)。} \textcolor{Sepia}{\selectlanguage{english}Classifier for layers (of dust, of boards…).} \textcolor{PineGreen}{\selectlanguage{french}Couche (de poussière; de planches constituant un plancher; de tissu…).}  ¶ \textcolor{darkblue}{\textbf{\ipa{dʑɯ˩-nɑ˩mi˩˥ | gv̩˧-ti˩-qo˩ tʰv̩˩}}} \zh{到深山老林的最深处。直译:‘到深山老林的第九层’。这里的‘九’作为最高的数字,表示‘极深’的意思:不能说‘深山老林的第一层’、‘第二层’等。} \textcolor{Sepia}{\selectlanguage{english}To arrive at the heart of the heart of the alpine forest. Literally 'in the ninth layer of alpine forest'; 'ninth' here serves as the highest numeral, to refer to an extreme; there is no such thing as a 'first layer', a 'second layer' and so on.} \textcolor{PineGreen}{\selectlanguage{french}se retrouver au plus profond de la forêt: littéralement “dans la neuvième couche de forêt/d'alpage” (=au plus profond; on ne compte pas au-delà de la 9e “couche”; ce décompte est métaphorique, il ne correspond pas à un décompte en étapes à pied, par exemple.}  

\lhead{\firstmark}
\rhead{\botmark}

\subsection{\hspace{-0.5cm} {\Large \textcolor{darkblue}{\textbf{\ipa{tjɤ˧hwɑ˧˥}}}}\hspace{0.5cm}[\kern2pt{\textcolor{darkblue}{\textbf{\ipa{tjɤ˧hwɑ˧}}}}\kern2pt]} \hypertarget{tj7\string_MhwA\string_M\string_T1}{}
\markboth{\textcolor{darkblue}{\textbf{\ipa{tjɤ˧hwɑ˧˥}}}}{}
\textcolor{teal}{\zh{名词}} \hspace{4pt} \zh{声调类:} MH\#.
\zh{电话(汉语借词)。} \textcolor{Sepia}{\selectlanguage{english}Telephone.} \textcolor{PineGreen}{\selectlanguage{french}Téléphone.}  \zh{【借词】} \zh{电话}
 \zh{量词}: \textcolor{darkblue}{\textbf{\ipa{ɭɯ˧}}} 
\lhead{\firstmark}
\rhead{\botmark}

\subsection{\hspace{-0.5cm} {\Large \textcolor{darkblue}{\textbf{\ipa{tjɤ˧po˧}}}}\hspace{0.5cm}[\kern2pt{\textcolor{darkblue}{\textbf{\ipa{tjɤ˧po˧˥}}}}\kern2pt]} \hypertarget{tj7\string_Mpo\string_M1}{}
\markboth{\textcolor{darkblue}{\textbf{\ipa{tjɤ˧po˧}}}}{}
\textcolor{teal}{\zh{名词}} \hspace{4pt} \zh{声调类:} M.
\zh{碉堡(汉语借词)。} \textcolor{Sepia}{\selectlanguage{english}Pillbox; blockhouse.} \textcolor{PineGreen}{\selectlanguage{french}Fortin, forteresse.}  \zh{【借词】} \zh{碉堡}
 \zh{量词}: \textcolor{darkblue}{\textbf{\ipa{ɭɯ˧}}} 
\lhead{\firstmark}
\rhead{\botmark}

\subsection{\hspace{-0.5cm} {\Large \textcolor{darkblue}{\textbf{\ipa{tjɤ˩˥ʂɯ˧}}}}\hspace{0.5cm}[\kern2pt{\textcolor{darkblue}{\textbf{\ipa{tjɤ˧ʂɯ˧}}}}\kern2pt]} \hypertarget{tj7\string_B\string_Ts`M\string_M1}{}
\markboth{\textcolor{darkblue}{\textbf{\ipa{tjɤ˩˥ʂɯ˧}}}}{}
\textcolor{teal}{\zh{名词}} \hspace{4pt} \zh{声调类:} LH.M.
\zh{电视(汉语借词)。} \textcolor{Sepia}{\selectlanguage{english}Television.} \textcolor{PineGreen}{\selectlanguage{french}Télévision.}  \zh{【借词】} \zh{电视}
 ¶ \textcolor{darkblue}{\textbf{\ipa{tjɤ˩ʂɯ˧ li˥}}} \zh{看电视} \textcolor{Sepia}{\selectlanguage{english}to watch television} \textcolor{PineGreen}{\selectlanguage{french}regarder la télévision}  
 ¶ \textcolor{darkblue}{\textbf{\ipa{tjɤ˩ʂɯ˧-qo˥}}} \zh{电视上} \textcolor{Sepia}{\selectlanguage{english}on television} \textcolor{PineGreen}{\selectlanguage{french}à la télévision}  

\lhead{\firstmark}
\rhead{\botmark}

\subsection{\hspace{-0.5cm} {\Large \textcolor{darkblue}{\textbf{\ipa{to˥\textsubscript{a}}}}}\hspace{0.5cm}[\kern2pt{\textcolor{darkblue}{\textbf{\ipa{to˩˥}}}}\kern2pt]} \hypertarget{to\string_Ta1}{}
\markboth{\textcolor{darkblue}{\textbf{\ipa{to˥\textsubscript{a}}}}}{}
\textcolor{teal}{\zh{量词}} \hspace{4pt} \zh{声调类:} H\textsubscript{a}.
\zh{量词:抱。} \textcolor{Sepia}{\selectlanguage{english}An armful of.} \textcolor{PineGreen}{\selectlanguage{french}Une grande brassée, ce qu'on peut prendre dans les bras: par exemple lors de la récolte: une brassée de riz coupé.}  ¶ \textcolor{darkblue}{\textbf{\ipa{qʰv̩˩ɖʐæ˩˥ | ɖɯ˧-to˥}}} \zh{一抱绳子} \textcolor{Sepia}{\selectlanguage{english}an armful of string (i.e. a huge quantity of string; see the narrative TraderAndHisSon)} \textcolor{PineGreen}{\selectlanguage{french}toute une brassée de ficelle/cordelette (voir le récit TraderAndHisSon)}  

\lhead{\firstmark}
\rhead{\botmark}

\subsection{\hspace{-0.5cm} {\Large \textcolor{darkblue}{\textbf{\ipa{to˧bɤ\#˥}}}}\hspace{0.5cm}[\kern2pt{\textcolor{darkblue}{\textbf{\ipa{to˩bɤ˥}}}}\kern2pt]} \hypertarget{to\string_Mb7\#\string_T1}{}
\markboth{\textcolor{darkblue}{\textbf{\ipa{to˧bɤ\#˥}}}}{}
\textcolor{teal}{\zh{形容词}} \hspace{4pt} \zh{声调类:} \#H.
\zh{空。} \textcolor{Sepia}{\selectlanguage{english}Empty.} \textcolor{PineGreen}{\selectlanguage{french}Vide.}  ¶ \textcolor{darkblue}{\textbf{\ipa{to˧bɤ˧-ze˩}}} \zh{空了} \textcolor{Sepia}{\selectlanguage{english}\mytextsc{pfv}: it's empty, there is nothing left (e.g. a bowl is entirely emptied)} \textcolor{PineGreen}{\selectlanguage{french}il n'y a plus rien (ex.: un bol est complètement vidé)}  
 ¶ \textcolor{darkblue}{\textbf{\ipa{to˧bɤ˧ ɲi˥}}} \zh{是空的} \textcolor{Sepia}{\selectlanguage{english}\string_ \mytextsc{cop}: it's empty} \textcolor{PineGreen}{\selectlanguage{french}\string_ \mytextsc{cop}: c'est vide}  

\lhead{\firstmark}
\rhead{\botmark}

\subsection{\hspace{-0.5cm} {\Large \textcolor{darkblue}{\textbf{\ipa{to˧kɤ\#˥}}}}\hspace{0.5cm}[\kern2pt{\textcolor{darkblue}{\textbf{\ipa{to˧kɤ˧}}}}\kern2pt]} \hypertarget{to\string_Mk7\#\string_T1}{}
\markboth{\textcolor{darkblue}{\textbf{\ipa{to˧kɤ\#˥}}}}{}
\textcolor{teal}{\zh{名词}} \hspace{4pt} \zh{声调类:} \#H.
\ding{202} \zh{额头。} \textcolor{Sepia}{\selectlanguage{english}Forehead.} \textcolor{PineGreen}{\selectlanguage{french}Front.}  \zh{量词}: \textcolor{darkblue}{\textbf{\ipa{ʈv̩˩}}} \ding{203} \zh{运气。} \textcolor{Sepia}{\selectlanguage{english}Luck, good fortune.} \textcolor{PineGreen}{\selectlanguage{french}Chance, bonne fortune.}  ¶ \textcolor{darkblue}{\textbf{\ipa{to˧kɤ˧ dʑɤ˥}}} \zh{好运气,运气好} \textcolor{Sepia}{\selectlanguage{english}to be lucky; to have a good karma} \textcolor{PineGreen}{\selectlanguage{french}avoir de la chance; avoir un bon karma}  
 ¶ \textcolor{darkblue}{\textbf{\ipa{njɤ˧ | tsʰi˧ʝi˧ | to˧kɤ˧ dʑjɤ˥ (+ | ʐwæ˩˥)}}} \zh{我今年运气好!} \textcolor{Sepia}{\selectlanguage{english}This year, I am lucky! / This is an auspicious year for me!} \textcolor{PineGreen}{\selectlanguage{french}Cette année, j'ai de la chance!}  

\lhead{\firstmark}
\rhead{\botmark}

\subsection{\hspace{-0.5cm} {\Large \textcolor{darkblue}{\textbf{\ipa{to˧kɤ˧qʰæ˩di˩ | -bæ˩bæ˩˥}}}}\hspace{0.5cm}[\kern2pt{\textcolor{darkblue}{\textbf{\ipa{xxxx non-correspondance entre le nombre de groupes tonals et le nombre de tons}}}}\kern2pt]} \hypertarget{to\string_Mk7\string_Mq\string_h\{\string_Bdi\string_B | -b\{\string_Bb\{\string_B\string_T1}{}
\markboth{\textcolor{darkblue}{\textbf{\ipa{to˧kɤ˧qʰæ˩di˩ | -bæ˩bæ˩˥}}}}{}
\textcolor{teal}{\zh{名词}} \hspace{4pt} \zh{声调类:} .
\zh{永宁的一种植物。} \textcolor{Sepia}{\selectlanguage{english}Plant with long filaments.} \textcolor{PineGreen}{\selectlanguage{french}Plante à longs filaments.}  \zh{量词}: \textcolor{darkblue}{\textbf{\ipa{bæ˩}}} 
\lhead{\firstmark}
\rhead{\botmark}

\subsection{\hspace{-0.5cm} {\Large \textcolor{darkblue}{\textbf{\ipa{to˧qɑ˧}}}}\hspace{0.5cm}[\kern2pt{\textcolor{darkblue}{\textbf{\ipa{to˧qɑ˧}}}}\kern2pt]} \hypertarget{to\string_MqA\string_M1}{}
\markboth{\textcolor{darkblue}{\textbf{\ipa{to˧qɑ˧}}}}{}
\textcolor{teal}{\zh{名词}} \hspace{4pt} \zh{声调类:} M.
\zh{羔羊。} \textcolor{Sepia}{\selectlanguage{english}Kid.} \textcolor{PineGreen}{\selectlanguage{french}Chevreau.}  \zh{量词}: \textcolor{darkblue}{\textbf{\ipa{ɭɯ˧}}} 
\lhead{\firstmark}
\rhead{\botmark}

\subsection{\hspace{-0.5cm} {\Large \textcolor{darkblue}{\textbf{\ipa{to˧\textasciitilde{}to˧\textsubscript{b}}}}}\hspace{0.5cm}[\kern2pt{\textcolor{darkblue}{\textbf{\ipa{to˧to˥}}}}\kern2pt]} \hypertarget{to\string_M~to\string_Mb1}{}
\markboth{\textcolor{darkblue}{\textbf{\ipa{to˧\textasciitilde{}to˧\textsubscript{b}}}}}{}
\textcolor{teal}{\zh{动词}} \hspace{4pt} \zh{声调类:} M\textsubscript{b}.
\zh{抱小孩子、搂,互相拥抱。} \textcolor{Sepia}{\selectlanguage{english}To hold a child in one's arms; to hug.} \textcolor{PineGreen}{\selectlanguage{french}Prendre un enfant dans ses bras.}  ¶ \textcolor{darkblue}{\textbf{\ipa{zo˧mv̩˥ to˩\textasciitilde{}to˩}}} \zh{抱小孩子} \textcolor{Sepia}{\selectlanguage{english}to hold a child in one's arms, to hug a child} \textcolor{PineGreen}{\selectlanguage{french}porter un enfant dans ses bras}  

\lhead{\firstmark}
\rhead{\botmark}

\subsection{\hspace{-0.5cm} {\Large \textcolor{darkblue}{\textbf{\ipa{to˩\textsubscript{a}}}} \textsubscript{1}}\hspace{0.5cm}[\kern2pt{\textcolor{darkblue}{\textbf{\ipa{to˧˥}}}}\kern2pt]} \hypertarget{to\string_Ba1}{}
\markboth{\textcolor{darkblue}{\textbf{\ipa{to˩\textsubscript{a}}}} \textsubscript{1}}{}
\textcolor{teal}{\zh{动词}} \hspace{4pt} \zh{声调类:} L\textsubscript{a}.
\zh{摔交。} \textcolor{Sepia}{\selectlanguage{english}To wrestle.} \textcolor{PineGreen}{\selectlanguage{french}Lutter, faire de la lutte.}  ¶ \textcolor{darkblue}{\textbf{\ipa{le˧-to˩-ze˩}}} \zh{\mytextsc{accomp} \string_ \mytextsc{pfv}} \textcolor{Sepia}{\selectlanguage{english}\mytextsc{accomp} \string_ \mytextsc{pfv}} \textcolor{PineGreen}{\selectlanguage{french}\mytextsc{accomp} \string_ \mytextsc{pfv}}  
 ¶ \textcolor{darkblue}{\textbf{\ipa{ɖʐæ˧\textasciitilde{}ɖʐæ˧ to˩}}} \zh{摔交} \textcolor{Sepia}{\selectlanguage{english}to wrestle} \textcolor{PineGreen}{\selectlanguage{french}lutter, faire de la lutte}  

\lhead{\firstmark}
\rhead{\botmark}

\subsection{\hspace{-0.5cm} {\Large \textcolor{darkblue}{\textbf{\ipa{to˩\textsubscript{a}}}} \textsubscript{2}}\hspace{0.5cm}[\kern2pt{\textcolor{darkblue}{\textbf{\ipa{to˩˥}}}}\kern2pt]} \hypertarget{to\string_Ba2}{}
\markboth{\textcolor{darkblue}{\textbf{\ipa{to˩\textsubscript{a}}}} \textsubscript{2}}{}
\textcolor{teal}{\zh{动词}} \hspace{4pt} \zh{声调类:} L\textsubscript{a}.
\zh{躺下。} \textcolor{Sepia}{\selectlanguage{english}To lie down.} \textcolor{PineGreen}{\selectlanguage{french}S'allonger.}  ¶ \textcolor{darkblue}{\textbf{\ipa{tʰi˧-to˩-ɕjɤ˩}}} \zh{躺着休息} \textcolor{Sepia}{\selectlanguage{english}to lie down and rest} \textcolor{PineGreen}{\selectlanguage{french}se reposer en position allongée, s'allonger pour prendre un peu de repos}  

\lhead{\firstmark}
\rhead{\botmark}

\subsection{\hspace{-0.5cm} {\Large \textcolor{darkblue}{\textbf{\ipa{to˩\textsubscript{a}}}} \textsubscript{3}}\hspace{0.5cm}[\kern2pt{\textcolor{darkblue}{\textbf{\ipa{to˩˥}}}}\kern2pt]} \hypertarget{to\string_Ba3}{}
\markboth{\textcolor{darkblue}{\textbf{\ipa{to˩\textsubscript{a}}}} \textsubscript{3}}{}
\textcolor{teal}{\zh{动词}} \hspace{4pt} \zh{声调类:} L\textsubscript{a}.
\zh{有亲属关系。} \textcolor{Sepia}{\selectlanguage{english}To stand in a family relationship, to have family ties.} \textcolor{PineGreen}{\selectlanguage{french}Être en relation, entretenir un lien de parenté.}  ¶ \textcolor{darkblue}{\textbf{\ipa{le˧-to˧\textasciitilde{}to˥}}} \zh{\mytextsc{accomp} \string_ \mytextsc{red}} \textcolor{Sepia}{\selectlanguage{english}\mytextsc{accomp} \string_ \mytextsc{red}} \textcolor{PineGreen}{\selectlanguage{french}\mytextsc{accomp} \string_ \mytextsc{red}}  
 ¶ \textcolor{darkblue}{\textbf{\ipa{qʰwɤ˩ɖɯ˩˥ | le˧-to˩-ze˩}}} \zh{我们成了亲戚!(通过领养、婚姻……)} \textcolor{Sepia}{\selectlanguage{english}We have acquired a family tie! (through adoption, marriage...)} \textcolor{PineGreen}{\selectlanguage{french}(nous) avons acquis un lien de parenté! (par adoption, mariage...)}  

\lhead{\firstmark}
\rhead{\botmark}

\subsection{\hspace{-0.5cm} {\Large \textcolor{darkblue}{\textbf{\ipa{to˩bi\#˥}}}}\hspace{0.5cm}[\kern2pt{\textcolor{darkblue}{\textbf{\ipa{to˩bi˩˥}}}}\kern2pt]} \hypertarget{to\string_Bbi\#\string_T1}{}
\markboth{\textcolor{darkblue}{\textbf{\ipa{to˩bi\#˥}}}}{}
\textcolor{teal}{\zh{名词}} \hspace{4pt} \zh{声调类:} LM+\#H.
\zh{瓶子。} \textcolor{Sepia}{\selectlanguage{english}Bottle.} \textcolor{PineGreen}{\selectlanguage{french}Bouteille.}  \zh{量词}: \textcolor{darkblue}{\textbf{\ipa{ɭɯ˧}}} 
\lhead{\firstmark}
\rhead{\botmark}

\subsection{\hspace{-0.5cm} {\Large \textcolor{darkblue}{\textbf{\ipa{to˩bi˩}}}}\hspace{0.5cm}[\kern2pt{\textcolor{darkblue}{\textbf{\ipa{to˥}}}}\kern2pt]} \hypertarget{to\string_Bbi\string_B1}{}
\markboth{\textcolor{darkblue}{\textbf{\ipa{to˩bi˩}}}}{}
\textcolor{teal}{\zh{量词}} \hspace{4pt} \zh{声调类:} L.
\zh{量词:瓶。} \textcolor{Sepia}{\selectlanguage{english}Self-classifier for bottles.} \textcolor{PineGreen}{\selectlanguage{french}Auto-classificateur des bouteilles.}  ¶ \textcolor{darkblue}{\textbf{\ipa{ɖɯ˧-to˩bi˩, so˩-to˩bi˩˥, ʐv̩˧-to˥bi˩, qʰv̩˧-to˥bi˩, ʂɯ˧-to˩bi˩, gv̩˧-to˥bi˩, tsʰe˩-to˩bi˩˥}}} \zh{与数词结合,一至十} \textcolor{Sepia}{\selectlanguage{english}association with numerals from 1 to 10} \textcolor{PineGreen}{\selectlanguage{french}association avec des numéraux, de 1 à 10. Comportement tonal identique pour 1 et 2, 4 et 5, 6 et 8.}  

\lhead{\firstmark}
\rhead{\botmark}

\subsection{\hspace{-0.5cm} {\Large \textcolor{darkblue}{\textbf{\ipa{to˩kʰv̩˩mi˥}}}}\hspace{0.5cm}[\kern2pt{\textcolor{darkblue}{\textbf{\ipa{to˩kʰv̩˩mi˥}}}}\kern2pt]} \hypertarget{to\string_Bk\string_hv\string_=\string_Bmi\string_T1}{}
\markboth{\textcolor{darkblue}{\textbf{\ipa{to˩kʰv̩˩mi˥}}}}{}
\textcolor{teal}{\zh{名词}} \hspace{4pt} \zh{声调类:} L+H\#.
\zh{公狗。} \textcolor{Sepia}{\selectlanguage{english}Male dog.} \textcolor{PineGreen}{\selectlanguage{french}Chien (animal mâle).}  \zh{量词}: \textcolor{darkblue}{\textbf{\ipa{mi˩}}} \textcolor{darkblue}{\textbf{\ipa{pʰo˧˥}}} 
\lhead{\firstmark}
\rhead{\botmark}

\subsection{\hspace{-0.5cm} {\Large \textcolor{darkblue}{\textbf{\ipa{to˩mi˩}}} \textsubscript{1}}\hspace{0.5cm}[\kern2pt{\textcolor{darkblue}{\textbf{\ipa{to˩mi˩˥}}}}\kern2pt]} \hypertarget{to\string_Bmi\string_B1}{}
\markboth{\textcolor{darkblue}{\textbf{\ipa{to˩mi˩}}} \textsubscript{1}}{}
\textcolor{teal}{\zh{名词}} \hspace{4pt} \zh{声调类:} L.
\zh{柱子。} \textcolor{Sepia}{\selectlanguage{english}Pillar.} \textcolor{PineGreen}{\selectlanguage{french}Pilier.}  ¶ \textcolor{darkblue}{\textbf{\ipa{hæ̃˧ʂɯ˩-to˩mi˩}}} \zh{‘黄金柱’、‘宝贵柱’:对主屋两个柱子的庄严称呼} \textcolor{Sepia}{\selectlanguage{english}the Precious Pillars, the Golden Pillars: a solemn designation for the two pillars of the main building} \textcolor{PineGreen}{\selectlanguage{french}les Piliers d'Or, les Précieux Piliers: appellation solennelle pour les deux piliers de la maison}  
 \zh{量词}: \textcolor{darkblue}{\textbf{\ipa{ɭɯ˧}}} 
\lhead{\firstmark}
\rhead{\botmark}

\subsection{\hspace{-0.5cm} {\Large \textcolor{darkblue}{\textbf{\ipa{to˩mi˩}}} \textsubscript{2}}\hspace{0.5cm}[\kern2pt{\textcolor{darkblue}{\textbf{\ipa{to˩mi˩˥}}}}\kern2pt]} \hypertarget{to\string_Bmi\string_B2}{}
\markboth{\textcolor{darkblue}{\textbf{\ipa{to˩mi˩}}} \textsubscript{2}}{}
\textcolor{teal}{\zh{名词}} \hspace{4pt} \zh{声调类:} L.
\zh{大山坡。} \textcolor{Sepia}{\selectlanguage{english}Large slope.} \textcolor{PineGreen}{\selectlanguage{french}Grande pente (mot attesté, pas combinaison artificielle).} 
\lhead{\firstmark}
\rhead{\botmark}

\subsection{\hspace{-0.5cm} {\Large \textcolor{darkblue}{\textbf{\ipa{to˩pi˩}}}}\hspace{0.5cm}[\kern2pt{\textcolor{darkblue}{\textbf{\ipa{to˩pi˩˥}}}}\kern2pt]} \hypertarget{to\string_Bpi\string_B1}{}
\markboth{\textcolor{darkblue}{\textbf{\ipa{to˩pi˩}}}}{}
\textcolor{teal}{\zh{量词}} \hspace{4pt} \zh{声调类:} L.
\zh{量词:倍(多几倍、少几倍等等)。} \textcolor{Sepia}{\selectlanguage{english}Classifier for times: n times as many/much as….} \textcolor{PineGreen}{\selectlanguage{french}Fois, multiple de.}  ¶ \textcolor{darkblue}{\textbf{\ipa{ɖɯ˧-to˩pi˩, ɲi˧-to˩pi˩, so˩-to˩pi˩˥, ʐv̩˧-to˥pi˩, qʰv̩˧-to˥pi˩, ʂɯ˧-to˩pi˩, gv̩˧-to˥pi˩, tsʰe˩-to˩pi˩˥}}} \zh{与数词结合,一至十} \textcolor{Sepia}{\selectlanguage{english}association with numerals from 1 to 10} \textcolor{PineGreen}{\selectlanguage{french}association avec des numéraux, de 1 à 10. Comportement tonal identique pour 1 et 2, 4 et 5, 6 et 8.}  

\lhead{\firstmark}
\rhead{\botmark}

\subsection{\hspace{-0.5cm} {\Large \textcolor{darkblue}{\textbf{\ipa{to˩pv̩˧}}}}\hspace{0.5cm}[\kern2pt{\textcolor{darkblue}{\textbf{\ipa{to˩pv̩˥}}}}\kern2pt]} \hypertarget{to\string_Bpv\string_=\string_M1}{}
\markboth{\textcolor{darkblue}{\textbf{\ipa{to˩pv̩˧}}}}{}
\textcolor{teal}{\zh{助词}} \hspace{4pt} \zh{声调类:} LM.
\zh{最初。} \textcolor{Sepia}{\selectlanguage{english}To begin with, at first, in the first place.} \textcolor{PineGreen}{\selectlanguage{french}Au début, pour commencer.}  \zh{【借词】}\zh{藏语}?

\lhead{\firstmark}
\rhead{\botmark}

\subsection{\hspace{-0.5cm} {\Large \textcolor{darkblue}{\textbf{\ipa{to˩qo˩lv̩˥}}}}\hspace{0.5cm}[\kern2pt{\textcolor{darkblue}{\textbf{\ipa{to˩qo˩lv̩˥}}}}\kern2pt]} \hypertarget{to\string_Bqo\string_Blv\string_=\string_T1}{}
\markboth{\textcolor{darkblue}{\textbf{\ipa{to˩qo˩lv̩˥}}}}{}
\textcolor{teal}{\zh{形容词}} \hspace{4pt} \zh{声调类:} L+H\#.
\zh{圆形(球很圆)。} \textcolor{Sepia}{\selectlanguage{english}Round in shape.} \textcolor{PineGreen}{\selectlanguage{french}Rond.}  ¶ \textcolor{darkblue}{\textbf{\ipa{to˩qo˩lv̩˥-gv̩˩}}} \zh{圆形} \textcolor{Sepia}{\selectlanguage{english}round in shape} \textcolor{PineGreen}{\selectlanguage{french}rond}  

\lhead{\firstmark}
\rhead{\botmark}

\subsection{\hspace{-0.5cm} {\Large \textcolor{darkblue}{\textbf{\ipa{to˩qo˧˥}}}}\hspace{0.5cm}[\kern2pt{\textcolor{darkblue}{\textbf{\ipa{xxxx ton non trouvé, à faire manuellement...}}}}\kern2pt]} \hypertarget{to\string_Bqo\string_M\string_T1}{}
\markboth{\textcolor{darkblue}{\textbf{\ipa{to˩qo˧˥}}}}{}
\textcolor{teal}{\zh{动词}} \hspace{4pt} \zh{声调类:} L+MH\#.
\zh{倒过来、倒放倒置。} \textcolor{Sepia}{\selectlanguage{english}To turn upside down.} \textcolor{PineGreen}{\selectlanguage{french}Renverser, verser; à l'envers (ex: renverser le contenu d'une boîte sur la table).}  ¶ \textcolor{darkblue}{\textbf{\ipa{to˩qo˧˥ | tɕɯ˧}}} \zh{倒过来放} \textcolor{Sepia}{\selectlanguage{english}to put upside down} \textcolor{PineGreen}{\selectlanguage{french}mettre à l'envers, renverser}  
 ¶ \textcolor{darkblue}{\textbf{\ipa{njɤ˧-ɳɯ˧ | to˩qo˧-bi˧!}}} \zh{我要(将这个东西)倒过来放!} \textcolor{Sepia}{\selectlanguage{english}I am going to turn (this object) upside down!} \textcolor{PineGreen}{\selectlanguage{french}je vais renverser (ce pot, cette assiette…)}  
 ¶ \textcolor{darkblue}{\textbf{\ipa{to˩qo˧-ze˥}}} \zh{倒过来了} \textcolor{Sepia}{\selectlanguage{english}\mytextsc{pfv}} \textcolor{PineGreen}{\selectlanguage{french}\mytextsc{pfv}}  

\lhead{\firstmark}
\rhead{\botmark}

\subsection{\hspace{-0.5cm} {\Large \textcolor{darkblue}{\textbf{\ipa{to˩to˧mi˥}}}}\hspace{0.5cm}[\kern2pt{\textcolor{darkblue}{\textbf{\ipa{to˩to˧mi˥}}}}\kern2pt]} \hypertarget{to\string_Bto\string_Mmi\string_T1}{}
\markboth{\textcolor{darkblue}{\textbf{\ipa{to˩to˧mi˥}}}}{}
\textcolor{teal}{\zh{助词}} \hspace{4pt} \zh{声调类:} LM+H\#.
\ding{202} \zh{认真地。} \textcolor{Sepia}{\selectlanguage{english}Carefully.} \textcolor{PineGreen}{\selectlanguage{french}Soigneusement, attentivement.}  ¶ \textcolor{darkblue}{\textbf{\ipa{njɤ˧-ɳɯ˧ | to˩to˧ mi˥ | ʐwɤ˩-bi˩˥! |}}} \zh{我要认真地讲!} \textcolor{Sepia}{\selectlanguage{english}I will explain carefully! / I will explain very clearly, step by step!} \textcolor{PineGreen}{\selectlanguage{french}je vais parler soigneusement/je vais bien expliquer!}  
 ¶ \textcolor{darkblue}{\textbf{\ipa{to˩to˧-mi˥ | so˩˥}}} \zh{认真地学习} \textcolor{Sepia}{\selectlanguage{english}to study with great care} \textcolor{PineGreen}{\selectlanguage{french}étudier attentivement}  
\ding{203} \zh{故意地。} \textcolor{Sepia}{\selectlanguage{english}Intentionally, purposedly, on purpose.} \textcolor{PineGreen}{\selectlanguage{french}Volontairement, délibérément, de propos délibéré: quelqu'un fait exprès de faire quelque chose.} 
\lhead{\firstmark}
\rhead{\botmark}

\subsection{\hspace{-0.5cm} {\Large \textcolor{darkblue}{\textbf{\ipa{to˩ʈɯ˩}}}}\hspace{0.5cm}[\kern2pt{\textcolor{darkblue}{\textbf{\ipa{to˧ʈɯ˧}}}}\kern2pt]} \hypertarget{to\string_Bt`M\string_B1}{}
\markboth{\textcolor{darkblue}{\textbf{\ipa{to˩ʈɯ˩}}}}{}
\textcolor{teal}{\zh{形容词}} \hspace{4pt} \zh{声调类:} L.
\zh{矮。} \textcolor{Sepia}{\selectlanguage{english}Short (of person).} \textcolor{PineGreen}{\selectlanguage{french}Petit (d'un homme).}  ¶ \textcolor{darkblue}{\textbf{\ipa{to˩ʈɯ˩\textasciitilde{}ʈɯ˥}}} \zh{矮} \textcolor{Sepia}{\selectlanguage{english}short} \textcolor{PineGreen}{\selectlanguage{french}petit, de petite taille}  

\lhead{\firstmark}
\rhead{\botmark}

\subsection{\hspace{-0.5cm} {\Large \textcolor{darkblue}{\textbf{\ipa{to˩zo˩}}}}\hspace{0.5cm}[\kern2pt{\textcolor{darkblue}{\textbf{\ipa{to˧zo˧}}}}\kern2pt]} \hypertarget{to\string_Bzo\string_B1}{}
\markboth{\textcolor{darkblue}{\textbf{\ipa{to˩zo˩}}}}{}
\textcolor{teal}{\zh{名词}} \hspace{4pt} \zh{声调类:} L.
\zh{小山坡。} \textcolor{Sepia}{\selectlanguage{english}Small slope.} \textcolor{PineGreen}{\selectlanguage{french}Petite pente (mot attesté, pas combinaison artificielle).} 
\lhead{\firstmark}
\rhead{\botmark}

\subsection{\hspace{-0.5cm} {\Large \textcolor{darkblue}{\textbf{\ipa{to˩˥}}}}\hspace{0.5cm}[\kern2pt{\textcolor{darkblue}{\textbf{\ipa{to˩˥}}}}\kern2pt]} \hypertarget{to\string_B\string_T1}{}
\markboth{\textcolor{darkblue}{\textbf{\ipa{to˩˥}}}}{}
\textcolor{teal}{\zh{名词}} \hspace{4pt} \zh{声调类:} LH.
\zh{山坡,岗。} \textcolor{Sepia}{\selectlanguage{english}Mountain slope, hillside.} \textcolor{PineGreen}{\selectlanguage{french}Pente, versant escarpé de montagne/colline.}  ¶ \textcolor{darkblue}{\textbf{\ipa{to˩ do˩˥}}} \zh{爬山坡} \textcolor{Sepia}{\selectlanguage{english}to climb a hillside} \textcolor{PineGreen}{\selectlanguage{french}grimper une pente}  
 ¶ \textcolor{darkblue}{\textbf{\ipa{ʁwɤ˧-to˩}}} \zh{山坡} \textcolor{Sepia}{\selectlanguage{english}mountain slope} \textcolor{PineGreen}{\selectlanguage{french}pente de montagne}  
 \zh{量词}: \textcolor{darkblue}{\textbf{\ipa{ɭɯ˧}}} 
\lhead{\firstmark}
\rhead{\botmark}

\subsection{\hspace{-0.5cm} {\Large \textcolor{darkblue}{\textbf{\ipa{tv̩˧˥}}} \textsubscript{1}}\hspace{0.5cm}[\kern2pt{\textcolor{darkblue}{\textbf{\ipa{tv̩˥}}}}\kern2pt]} \hypertarget{tv\string_=\string_M\string_T1}{}
\markboth{\textcolor{darkblue}{\textbf{\ipa{tv̩˧˥}}} \textsubscript{1}}{}
\textcolor{teal}{\zh{动词}} \hspace{4pt} \zh{声调类:} MH.
\zh{搀扶、撑住、稳住。} \textcolor{Sepia}{\selectlanguage{english}To support, to stabilize, to consolidate.} \textcolor{PineGreen}{\selectlanguage{french}Soutenir.} 
\lhead{\firstmark}
\rhead{\botmark}

\subsection{\hspace{-0.5cm} {\Large \textcolor{darkblue}{\textbf{\ipa{tv̩˧˥}}} \textsubscript{2}}\hspace{0.5cm}[\kern2pt{\textcolor{darkblue}{\textbf{\ipa{tv̩˧˥}}}}\kern2pt]} \hypertarget{tv\string_=\string_M\string_T2}{}
\markboth{\textcolor{darkblue}{\textbf{\ipa{tv̩˧˥}}} \textsubscript{2}}{}
\textcolor{teal}{\zh{动词}} \hspace{4pt} \zh{声调类:} MH.
\zh{喂,喂到嘴里。} \textcolor{Sepia}{\selectlanguage{english}To pour (a liquid) into someone's mouth, to pour down someone's throat (e.g. pouring medicines into the throat of a sick person).} \textcolor{PineGreen}{\selectlanguage{french}Verser (un liquide) dans la bouche de quelqu'un, faire boire à quelqu'un.}  ¶ \textcolor{darkblue}{\textbf{\ipa{ʈʂʰæ˧ɣɯ˧ | tʰi˧-tv̩˧˥}}} \zh{喂药} \textcolor{Sepia}{\selectlanguage{english}to give medicines, to pour medicines into the throat of a sick person} \textcolor{PineGreen}{\selectlanguage{french}verser un médicament dans la bouche de quelqu'un, faire boire un médicament à quelqu'un}  

\lhead{\firstmark}
\rhead{\botmark}

\subsection{\hspace{-0.5cm} {\Large \textcolor{darkblue}{\textbf{\ipa{tv̩˧\textsubscript{a}}}}}\hspace{0.5cm}[\kern2pt{\textcolor{darkblue}{\textbf{\ipa{tv̩˩˥}}}}\kern2pt]} \hypertarget{tv\string_=\string_Ma1}{}
\markboth{\textcolor{darkblue}{\textbf{\ipa{tv̩˧\textsubscript{a}}}}}{}
\textcolor{teal}{\zh{动词}} \hspace{4pt} \zh{声调类:} M\textsubscript{a}.
\zh{耕种、插秧。} \textcolor{Sepia}{\selectlanguage{english}To plant, to bed out (rice).} \textcolor{PineGreen}{\selectlanguage{french}Planter; aussi: repiquer (le riz).}  ¶ \textcolor{darkblue}{\textbf{\ipa{ɕi˧ tv̩˧}}} \zh{插秧} \textcolor{Sepia}{\selectlanguage{english}to bed out rice} \textcolor{PineGreen}{\selectlanguage{french}repiquer le riz}  
 ¶ \textcolor{darkblue}{\textbf{\ipa{le˧-tv̩˧-ze˧}}} \textcolor{PineGreen}{\selectlanguage{french}\mytextsc{accomp} \string_ \mytextsc{pfv}}  
 ¶ \textcolor{darkblue}{\textbf{\ipa{le˧-tv̩˥-tv̩˩-ze˩}}} \textcolor{PineGreen}{\selectlanguage{french}\mytextsc{red}}  

\lhead{\firstmark}
\rhead{\botmark}

\subsection{\hspace{-0.5cm} {\Large \textcolor{darkblue}{\textbf{\ipa{tv̩˧\textsubscript{a}}}} \textsubscript{1}}\hspace{0.5cm}[\kern2pt{\textcolor{darkblue}{\textbf{\ipa{tv̩˥}}}}\kern2pt]} \hypertarget{tv\string_=\string_Ma1}{}
\markboth{\textcolor{darkblue}{\textbf{\ipa{tv̩˧\textsubscript{a}}}} \textsubscript{1}}{}
\textcolor{teal}{\zh{量词}} \hspace{4pt} \zh{声调类:} M\textsubscript{a}.
\zh{千(数词充当量词)。} \textcolor{Sepia}{\selectlanguage{english}1,000.} \textcolor{PineGreen}{\selectlanguage{french}1.000.}  ¶ \textcolor{darkblue}{\textbf{\ipa{ɖɯ˧-tv̩˧}}} \zh{一千} \textcolor{Sepia}{\selectlanguage{english}one thousand} \textcolor{PineGreen}{\selectlanguage{french}mille}  
 ¶ \textcolor{darkblue}{\textbf{\ipa{ɖɯ˧-tv̩˧ tv̩˧}}} \zh{一千千,等于一百万} \textcolor{Sepia}{\selectlanguage{english}one thousand thousands = one million} \textcolor{PineGreen}{\selectlanguage{french}mille milliers = un million}  
 ¶ \textcolor{darkblue}{\textbf{\ipa{tsʰe˩-tv̩˩ mæ˥}}} \zh{十千万,等于一亿} \textcolor{Sepia}{\selectlanguage{english}ten thousand times 10,000, i.e. one hundred million} \textcolor{PineGreen}{\selectlanguage{french}dix mille fois 10.000, soit cent millions}  

\lhead{\firstmark}
\rhead{\botmark}

\subsection{\hspace{-0.5cm} {\Large \textcolor{darkblue}{\textbf{\ipa{tv̩˧\textsubscript{a}}}} \textsubscript{2}}\hspace{0.5cm}[\kern2pt{\textcolor{darkblue}{\textbf{\ipa{tv̩˥}}}}\kern2pt]} \hypertarget{tv\string_=\string_Ma2}{}
\markboth{\textcolor{darkblue}{\textbf{\ipa{tv̩˧\textsubscript{a}}}} \textsubscript{2}}{}
\textcolor{teal}{\zh{量词}} \hspace{4pt} \zh{声调类:} M\textsubscript{a}.
\zh{量词:角(钱),一元的十分之一。} \textcolor{Sepia}{\selectlanguage{english}Classifier: a dime, i.e. one tenth of the monetary unit.} \textcolor{PineGreen}{\selectlanguage{french}Dixième d'unité monétaire.} 
\lhead{\firstmark}
\rhead{\botmark}

\subsection{\hspace{-0.5cm} {\Large \textcolor{darkblue}{\textbf{\ipa{tv̩˧ɕi˩}}}}\hspace{0.5cm}[\kern2pt{\textcolor{darkblue}{\textbf{\ipa{tv̩˩ɕi˩˥}}}}\kern2pt]} \hypertarget{tv\string_=\string_Ms£i\string_B1}{}
\markboth{\textcolor{darkblue}{\textbf{\ipa{tv̩˧ɕi˩}}}}{}
\textcolor{teal}{\zh{名词}} \hspace{4pt} \zh{声调类:} L\#.
\zh{蜈蚣。} \textcolor{Sepia}{\selectlanguage{english}Centipede.} \textcolor{PineGreen}{\selectlanguage{french}Millepattes.}  \zh{量词}: \textcolor{darkblue}{\textbf{\ipa{mi˩}}} 
\lhead{\firstmark}
\rhead{\botmark}

\subsection{\hspace{-0.5cm} {\Large \textcolor{darkblue}{\textbf{\ipa{tv̩˩ɭɯ˧˥}}}}\hspace{0.5cm}[\kern2pt{\textcolor{darkblue}{\textbf{\ipa{tv̩˩ɭɯ˧˥}}}}\kern2pt]} \hypertarget{tv\string_=\string_Bl\string_RM\string_M\string_T1}{}
\markboth{\textcolor{darkblue}{\textbf{\ipa{tv̩˩ɭɯ˧˥}}}}{}
\textcolor{teal}{\zh{名词}} \hspace{4pt} \zh{声调类:} LM+MH\#.
\zh{高级的背篓,过去用它放礼物。} \textcolor{Sepia}{\selectlanguage{english}Fine, high-quality basket carried on the back; it had a trough shape. Not in use anymore at the time of fieldwork.} \textcolor{PineGreen}{\selectlanguage{french}Hotte de grande qualité, dans laquelle on offrait des cadeaux; n'existe plus actuellement; était resserrée au milieu: de forme concave, pas convexe.}  \zh{量词}: \textcolor{darkblue}{\textbf{\ipa{ɭɯ˧}}} 
\lhead{\firstmark}
\rhead{\botmark}

\subsection{\hspace{-0.5cm} {\Large \textcolor{darkblue}{\textbf{\ipa{tv̩˧po˩}}}}\hspace{0.5cm}[\kern2pt{\textcolor{darkblue}{\textbf{\ipa{tv̩˧po˩}}}}\kern2pt]} \hypertarget{tv\string_=\string_Mpo\string_B1}{}
\markboth{\textcolor{darkblue}{\textbf{\ipa{tv̩˧po˩}}}}{}
\textcolor{teal}{\zh{动词}} \hspace{4pt} \zh{声调类:} L\#.
\zh{赌博(汉语借词)。} \textcolor{Sepia}{\selectlanguage{english}To gamble, to bet, to wager.} \textcolor{PineGreen}{\selectlanguage{french}Parier, jouer à des jeux d'argent.}  \zh{【借词】} \zh{赌博}

\lhead{\firstmark}
\rhead{\botmark}

\subsection{\hspace{-0.5cm} {\Large \textcolor{darkblue}{\textbf{\ipa{tv̩˧qʰv̩˧}}}}\hspace{0.5cm}[\kern2pt{\textcolor{darkblue}{\textbf{\ipa{tv̩˧qʰv̩˧}}}}\kern2pt]} \hypertarget{tv\string_=\string_Mq\string_hv\string_=\string_M1}{}
\markboth{\textcolor{darkblue}{\textbf{\ipa{tv̩˧qʰv̩˧}}}}{}
\textcolor{teal}{\zh{名词}} \hspace{4pt} \zh{声调类:} M.
\zh{临时坟墓。} \textcolor{Sepia}{\selectlanguage{english}Temporary tomb, where the body is placed prior to cremation.} \textcolor{PineGreen}{\selectlanguage{french}Tombe provisoire, où on place le corps du défunt avant la crémation.} 
\lhead{\firstmark}
\rhead{\botmark}

\subsection{\hspace{-0.5cm} {\Large \textcolor{darkblue}{\textbf{\ipa{tv̩˧tsʰɯ˧}}}}\hspace{0.5cm}[\kern2pt{\textcolor{darkblue}{\textbf{\ipa{tv̩˧tsʰɯ˧}}}}\kern2pt]} \hypertarget{tv\string_=\string_Mts\string_hM\string_M1}{}
\markboth{\textcolor{darkblue}{\textbf{\ipa{tv̩˧tsʰɯ˧}}}}{}
\textcolor{teal}{\zh{名词}} \hspace{4pt} \zh{声调类:} M.
\ding{202} \zh{时间。} \textcolor{Sepia}{\selectlanguage{english}Time.} \textcolor{PineGreen}{\selectlanguage{french}Temps.}  ¶ \textcolor{darkblue}{\textbf{\ipa{[F5] njɤ˧ | tv̩˧tsʰɯ˧ mɤ˧-dʑo˧.}}} \zh{我没时间。} \textcolor{Sepia}{\selectlanguage{english}I don't have the time.} \textcolor{PineGreen}{\selectlanguage{french}Je n'ai pas le temps.}  
 ¶ \textcolor{darkblue}{\textbf{\ipa{[F5] njɤ˧ | tv̩˧tsʰɯ˧ dʑo˧.}}} \zh{我有时间。} \textcolor{Sepia}{\selectlanguage{english}I have time. / I have some free time. / I have the time.} \textcolor{PineGreen}{\selectlanguage{french}J'ai du temps libre. / J'ai le temps.}  
 \zh{量词}: \textcolor{darkblue}{\textbf{\ipa{ɭɯ˧}}} \ding{203} \zh{时间段、小时。} \textcolor{Sepia}{\selectlanguage{english}Spell of time; hour.} \textcolor{PineGreen}{\selectlanguage{french}Période de temps, heure.}  ¶ \textcolor{darkblue}{\textbf{\ipa{tv̩˧tsʰɯ˧ | ɖɯ˧-ɭɯ˧}}} \zh{一个小时} \textcolor{Sepia}{\selectlanguage{english}one hour} \textcolor{PineGreen}{\selectlanguage{french}une heure}  
 ¶ \textcolor{darkblue}{\textbf{\ipa{tv̩˧tsʰɯ˧ ɖɯ˧-ɭɯ˧ gv̩˧-ze˧!}}} \zh{一个小时过去了。} \textcolor{Sepia}{\selectlanguage{english}One hour has gone by.} \textcolor{PineGreen}{\selectlanguage{french}Une heure a passé.}  
 ¶ \textcolor{darkblue}{\textbf{\ipa{tv̩˧tsʰɯ˧ ɖɯ˧-ɭɯ˧ le˧-hɯ˩-ze˩.}}} \zh{一个小时过去了。} \textcolor{Sepia}{\selectlanguage{english}One hour has gone by.} \textcolor{PineGreen}{\selectlanguage{french}Une heure s'est écoulée.}  
 ¶ \textcolor{darkblue}{\textbf{\ipa{[F5] tv̩˧tsʰɯ˧ qʰɑ˧-ɭɯ˧?}}} \zh{几点了?} \textcolor{Sepia}{\selectlanguage{english}What time is it? (Literally: “How many hours?”)} \textcolor{PineGreen}{\selectlanguage{french}Quelle heure est-il?}  

\lhead{\firstmark}
\rhead{\botmark}

\subsection{\hspace{-0.5cm} {\Large \textcolor{darkblue}{\textbf{\ipa{tv̩˩tv̩˩}}}}\hspace{0.5cm}[\kern2pt{\textcolor{darkblue}{\textbf{\ipa{tv̩˩tv̩˩˥}}}}\kern2pt]} \hypertarget{tv\string_=\string_Btv\string_=\string_B1}{}
\markboth{\textcolor{darkblue}{\textbf{\ipa{tv̩˩tv̩˩}}}}{}
\textcolor{teal}{\zh{形容词}} \hspace{4pt} \zh{声调类:} L.
\ding{202} \zh{直,笔直的(如:站直)。} \textcolor{Sepia}{\selectlanguage{english}Upright.} \textcolor{PineGreen}{\selectlanguage{french}Droit, bien d'aplomb.} \ding{203} \zh{耿直。} \textcolor{Sepia}{\selectlanguage{english}Upright, righteous, honest.} \textcolor{PineGreen}{\selectlanguage{french}Droit, intègre, honnête.} 
\lhead{\firstmark}
\rhead{\botmark}

\subsection{\hspace{-0.5cm} {\Large \textcolor{darkblue}{\textbf{\ipa{tv̩˧tv̩˥}}}}\hspace{0.5cm}[\kern2pt{\textcolor{darkblue}{\textbf{\ipa{tv̩˩tv̩˩˥}}}}\kern2pt]} \hypertarget{tv\string_=\string_Mtv\string_=\string_T1}{}
\markboth{\textcolor{darkblue}{\textbf{\ipa{tv̩˧tv̩˥}}}}{}
\textcolor{teal}{\zh{名词}} \hspace{4pt} \zh{声调类:} H\#.
\zh{帽子。} \textcolor{Sepia}{\selectlanguage{english}Hat.} \textcolor{PineGreen}{\selectlanguage{french}Chapeau.}  \zh{量词}: \textcolor{darkblue}{\textbf{\ipa{ɭɯ˧}}} 
\lhead{\firstmark}
\rhead{\botmark}

\newpage
\section*{\centering- \textcolor{darkblue}{\textbf{\ipa{tʰ}}} -}
\subsection{\hspace{-0.5cm} {\Large \textcolor{darkblue}{\textbf{\ipa{tʰɑ˧‑}}}}\hspace{0.5cm}[\kern2pt{\textcolor{darkblue}{\textbf{\ipa{tʰɑ˥}}}}\kern2pt]} \hypertarget{t\string_hA\string_M‑1}{}
\markboth{\textcolor{darkblue}{\textbf{\ipa{tʰɑ˧‑}}}}{}
\textcolor{teal}{\zh{前缀}} \hspace{4pt} \zh{声调类:} M/0.
\zh{不要、别\mytextsc{禁止式。}} \textcolor{Sepia}{\selectlanguage{english}Prohibitive.} \textcolor{PineGreen}{\selectlanguage{french}Prohibitif.}  ¶ \textcolor{darkblue}{\textbf{\ipa{tʰɑ˧-lɑ˩\textasciitilde{}lɑ˩-ze˩!}}} \zh{别吵架了!} \textcolor{Sepia}{\selectlanguage{english}Don't quarrel!} \textcolor{PineGreen}{\selectlanguage{french}Arrêtez de vous disputer! / Ne vous disputez pas!}  
 ¶ \textcolor{darkblue}{\textbf{\ipa{tʰɑ˧-dzo˧\textasciitilde{}dzo˥!}}} \zh{不要动来动去! / 不要碰来碰去!} \textcolor{Sepia}{\selectlanguage{english}Don't touch!} \textcolor{PineGreen}{\selectlanguage{french}Ne pas toucher! / Ne touchez pas!}  

\lhead{\firstmark}
\rhead{\botmark}

\subsection{\hspace{-0.5cm} {\Large \textcolor{darkblue}{\textbf{\ipa{tʰɑ˧v̩˥}}}}\hspace{0.5cm}[\kern2pt{\textcolor{darkblue}{\textbf{\ipa{tʰɑ˧v̩˥}}}}\kern2pt]} \hypertarget{t\string_hA\string_Mv\string_=\string_T1}{}
\markboth{\textcolor{darkblue}{\textbf{\ipa{tʰɑ˧v̩˥}}}}{}
\textcolor{teal}{\zh{名词}} \hspace{4pt} \zh{声调类:} H\#.
\zh{堂屋(汉语借词),来指客房。} \textcolor{Sepia}{\selectlanguage{english}Room for guests (local Chinese word; meaning in standard Chinese: central room, main hall). There is no direct equivalent in Na because the traditional house did not have a room for guests.} \textcolor{PineGreen}{\selectlanguage{french}Chambre des invités (emprunt au chinois local; sens en chinois standard: pièce centrale, salle de séjour). Il n'y a pas d'équivalent direct dans la maison na traditionnelle: c'est dans la resserre qu'on pouvait improviser une chambre supplémentaire.}  \zh{量词}: \textcolor{darkblue}{\textbf{\ipa{ɭɯ˧}}} 
\lhead{\firstmark}
\rhead{\botmark}

\subsection{\hspace{-0.5cm} {\Large \textcolor{darkblue}{\textbf{\ipa{tʰɑ˩lo˧}}}}\hspace{0.5cm}[\kern2pt{\textcolor{darkblue}{\textbf{\ipa{tʰɑ˩lo˥}}}}\kern2pt]} \hypertarget{t\string_hA\string_Blo\string_M1}{}
\markboth{\textcolor{darkblue}{\textbf{\ipa{tʰɑ˩lo˧}}}}{}
\textcolor{teal}{\zh{名词}} \hspace{4pt} \zh{声调类:} LM.
\zh{永宁的藏语名称。} \textcolor{Sepia}{\selectlanguage{english}The name given to the plain of Yongning by the Tibetans.} \textcolor{PineGreen}{\selectlanguage{french}Prononciation par les Na de thar lam, nom anciennement donné par les Tibétains à la plaine de Yongning.}  \zh{【借词】}\zh{藏语} thar lam
 ¶ \textcolor{darkblue}{\textbf{\ipa{tʰɑ˩lo˧-go˧bɤ˩}}} \zh{永宁大寺} \textcolor{Sepia}{\selectlanguage{english}the temple of Thar Lam} \textcolor{PineGreen}{\selectlanguage{french}le temple de Thar lam =le temple de Yongning, tel que l'appellent les Tibétains}  
 ¶ \textcolor{darkblue}{\textbf{\ipa{tʰɑ˩lo˧ se˧gi˧ kɤ˩mv̩˩}}} \zh{永宁格姆山} \textcolor{Sepia}{\selectlanguage{english}mount Gemu, in Yongning} \textcolor{PineGreen}{\selectlanguage{french}la montagne Gemu de Yongning}  

\lhead{\firstmark}
\rhead{\botmark}

\subsection{\hspace{-0.5cm} {\Large \textcolor{darkblue}{\textbf{\ipa{tʰɑ˩mi\#˥}}}}\hspace{0.5cm}[\kern2pt{\textcolor{darkblue}{\textbf{\ipa{tʰɑ˩mi˥}}}}\kern2pt]} \hypertarget{t\string_hA\string_Bmi\#\string_T1}{}
\markboth{\textcolor{darkblue}{\textbf{\ipa{tʰɑ˩mi\#˥}}}}{}
\textcolor{teal}{\zh{名词}} \hspace{4pt} \zh{声调类:} LM+\#H.
\zh{母水牛。} \textcolor{Sepia}{\selectlanguage{english}Female water buffalo.} \textcolor{PineGreen}{\selectlanguage{french}Buffle femelle.}  ¶ \textcolor{darkblue}{\textbf{\ipa{dʑi˧mi˧-tʰɑ˩mi˩}}} \zh{母水牛} \textcolor{Sepia}{\selectlanguage{english}same meaning: female water buffalo} \textcolor{PineGreen}{\selectlanguage{french}même sens: buffle femelle}  
 ¶ \textcolor{darkblue}{\textbf{\ipa{dʑi˧mi˧ ʈʂʰɯ˧-pʰo˩ dʑo˩, | tʰɑ˩mi˧ ɲi˥!}}} \zh{这头水牛是母的!} \textcolor{Sepia}{\selectlanguage{english}This buffalo is a female!} \textcolor{PineGreen}{\selectlanguage{french}ce buffle, c'est une femelle!}  
 \zh{量词}: \textcolor{darkblue}{\textbf{\ipa{pʰo˧˥}}} 
\lhead{\firstmark}
\rhead{\botmark}

\subsection{\hspace{-0.5cm} {\Large \textcolor{darkblue}{\textbf{\ipa{tʰɑ˩pʰv̩\#˥}}}}\hspace{0.5cm}[\kern2pt{\textcolor{darkblue}{\textbf{\ipa{tʰɑ˩pʰv̩˥}}}}\kern2pt]} \hypertarget{t\string_hA\string_Bp\string_hv\string_=\#\string_T1}{}
\markboth{\textcolor{darkblue}{\textbf{\ipa{tʰɑ˩pʰv̩\#˥}}}}{}
\textcolor{teal}{\zh{名词}} \hspace{4pt} \zh{声调类:} LM+\#H.
\zh{公水牛。} \textcolor{Sepia}{\selectlanguage{english}Male water buffalo.} \textcolor{PineGreen}{\selectlanguage{french}Buffle mâle.}  \zh{量词}: \textcolor{darkblue}{\textbf{\ipa{pʰo˧˥}}} 
\lhead{\firstmark}
\rhead{\botmark}

\subsection{\hspace{-0.5cm} {\Large \textcolor{darkblue}{\textbf{\ipa{tʰɑ˩tʰɑ˩}}}}\hspace{0.5cm}[\kern2pt{\textcolor{darkblue}{\textbf{\ipa{tʰɑ˩tʰɑ˩˥}}}}\kern2pt]} \hypertarget{t\string_hA\string_Bt\string_hA\string_B1}{}
\markboth{\textcolor{darkblue}{\textbf{\ipa{tʰɑ˩tʰɑ˩}}}}{}
\textcolor{teal}{\zh{名词}} \hspace{4pt} \zh{声调类:} L.
\zh{采野生植物如菌子等的好地方。} \textcolor{Sepia}{\selectlanguage{english}A good spot, a good place to find a certain species of plant: for instance, mushrooms that will grow there every year.} \textcolor{PineGreen}{\selectlanguage{french}Bon coin pour la cueillette de champignons, de plantes sauvages...}  ¶ \textcolor{darkblue}{\textbf{\ipa{tʰɑ˩tʰɑ˩˥ | ɖɯ˧-kʰwɤ˥}}} \zh{一个好地方} \textcolor{Sepia}{\selectlanguage{english}a good spot (for hunting a certain kind of wild plant)} \textcolor{PineGreen}{\selectlanguage{french}un bon coin (pour la cueillette)}  
 \zh{量词}: \textcolor{darkblue}{\textbf{\ipa{kʰwɤ˥}}} 
\lhead{\firstmark}
\rhead{\botmark}

\subsection{\hspace{-0.5cm} {\Large \textcolor{darkblue}{\textbf{\ipa{tʰɑ˩zo\#˥}}}}\hspace{0.5cm}[\kern2pt{\textcolor{darkblue}{\textbf{\ipa{tʰɑ˩zo˥}}}}\kern2pt]} \hypertarget{t\string_hA\string_Bzo\#\string_T1}{}
\markboth{\textcolor{darkblue}{\textbf{\ipa{tʰɑ˩zo\#˥}}}}{}
\textcolor{teal}{\zh{名词}} \hspace{4pt} \zh{声调类:} LM+\#H.
\zh{小水牛。} \textcolor{Sepia}{\selectlanguage{english}Baby water buffalo.} \textcolor{PineGreen}{\selectlanguage{french}Petit buffle, enfant du buffle.}  \zh{量词}: \textcolor{darkblue}{\textbf{\ipa{mi˩}}} 
\lhead{\firstmark}
\rhead{\botmark}

\subsection{\hspace{-0.5cm} {\Large \textcolor{darkblue}{\textbf{\ipa{tʰɑ˩-ʐwæ˧mi˧}}}}\hspace{0.5cm}[\kern2pt{\textcolor{darkblue}{\textbf{\ipa{tʰɑ˧ʐwæ˧mi˧}}}}\kern2pt]} \hypertarget{t\string_hA\string_B-z`w\{\string_Mmi\string_M1}{}
\markboth{\textcolor{darkblue}{\textbf{\ipa{tʰɑ˩-ʐwæ˧mi˧}}}}{}
\textcolor{teal}{\zh{名词}} \hspace{4pt} \zh{声调类:} L-.
\zh{驴子。} \textcolor{Sepia}{\selectlanguage{english}Donkey, ass (either jack or jenny or foal).} \textcolor{PineGreen}{\selectlanguage{french}Âne (mâle ou femelle).}  \zh{量词}: \textcolor{darkblue}{\textbf{\ipa{pʰo˧˥}}} 
\lhead{\firstmark}
\rhead{\botmark}

\subsection{\hspace{-0.5cm} {\Large \textcolor{darkblue}{\textbf{\ipa{tʰɑ˧˥}}} \textsubscript{1}}\hspace{0.5cm}[\kern2pt{\textcolor{darkblue}{\textbf{\ipa{tʰɑ˧˥}}}}\kern2pt]} \hypertarget{t\string_hA\string_M\string_T1}{}
\markboth{\textcolor{darkblue}{\textbf{\ipa{tʰɑ˧˥}}} \textsubscript{1}}{}
\textcolor{teal}{\zh{形容词}} \hspace{4pt} \zh{声调类:} MH.
\zh{锋利。} \textcolor{Sepia}{\selectlanguage{english}Sharp, keen.} \textcolor{PineGreen}{\selectlanguage{french}Aiguisé, qui coupe bien, affûté.} 
\lhead{\firstmark}
\rhead{\botmark}

\subsection{\hspace{-0.5cm} {\Large \textcolor{darkblue}{\textbf{\ipa{tʰɑ˧˥}}} \textsubscript{2}}\hspace{0.5cm}[\kern2pt{\textcolor{darkblue}{\textbf{\ipa{tʰɑ˧˥}}}}\kern2pt]} \hypertarget{t\string_hA\string_M\string_T2}{}
\markboth{\textcolor{darkblue}{\textbf{\ipa{tʰɑ˧˥}}} \textsubscript{2}}{}
\textcolor{teal}{\zh{动词}} \hspace{4pt} \zh{声调类:} MH.
\zh{可以,允许。} \textcolor{Sepia}{\selectlanguage{english}To be possible, to be allowed: \mytextsc{permissive}.} \textcolor{PineGreen}{\selectlanguage{french}Être possible, être autorisé: \mytextsc{permissif}.}  ¶ \textcolor{darkblue}{\textbf{\ipa{mɤ˧-tʰɑ˧˥ | mɤ˧-ʐv̩˩! | njɤ˧ | dzɯ˧-bi˧ni˧-mɤ˧-gv̩˧˥!}}} \zh{不怎么好吃!我不喜欢吃!} \textcolor{Sepia}{\selectlanguage{english}It's not that good! I don't like to eat that!} \textcolor{PineGreen}{\selectlanguage{french}Ce n'est pas vraiment bon! Je n'aime pas en manger!}  
 ¶ \textcolor{darkblue}{\textbf{\ipa{[Mushrooms] ə˩ljɤ˩hæ̃˩ʂɯ˥-mo˩-ʈʂʰɯ˩-dʑo˩, | hĩ˧ | mɤ˧-tʰɑ˧˥ | dv̩˩-mɤ˧-kv̩˧˥!}}} \zh{黄蜡伞不怎么会让人中毒!/ 毒性不太大!} \textcolor{Sepia}{\selectlanguage{english}The Golden Mushroom is not all that poisonous! / The Golden Mushroom is not really dangerous!} \textcolor{PineGreen}{\selectlanguage{french}Le Champignon Doré, il n'est pas si vénéneux que ça!}  

\lhead{\firstmark}
\rhead{\botmark}

\subsection{\hspace{-0.5cm} {\Large \textcolor{darkblue}{\textbf{\ipa{tʰɑ˩˥}}}}\hspace{0.5cm}[\kern2pt{\textcolor{darkblue}{\textbf{\ipa{tʰɑ˩˥}}}}\kern2pt]} \hypertarget{t\string_hA\string_B\string_T1}{}
\markboth{\textcolor{darkblue}{\textbf{\ipa{tʰɑ˩˥}}}}{}
\textcolor{teal}{\zh{名词}} \hspace{4pt} \zh{声调类:} LH.
\zh{水牛。} \textcolor{Sepia}{\selectlanguage{english}Water buffalo (monosyllabic form, extracted from disyllables such as \zh{/tʰɑ˩mi\#˥/} 'female buffalo').} \textcolor{PineGreen}{\selectlanguage{french}Buffle; forme monosyllabique, qui n'est pas utilisée telle quelle, seulement dans des formes telles que \zh{/tʰɑ˩mi\#˥/} 'buffle femelle'.}  \zh{量词}: \textcolor{darkblue}{\textbf{\ipa{pʰo˧˥}}} 
\lhead{\firstmark}
\rhead{\botmark}

\subsection{\hspace{-0.5cm} {\Large \textcolor{darkblue}{\textbf{\ipa{tʰæ˧ɻæ˩}}}}\hspace{0.5cm}[\kern2pt{\textcolor{darkblue}{\textbf{\ipa{tʰæ˧ɻæ˩}}}}\kern2pt]} \hypertarget{t\string_h\{\string_Mr£`\{\string_B1}{}
\markboth{\textcolor{darkblue}{\textbf{\ipa{tʰæ˧ɻæ˩}}}}{}
\textcolor{teal}{\zh{名词}} \hspace{4pt} \zh{声调类:} L\#.
\zh{书。} \textcolor{Sepia}{\selectlanguage{english}Book.} \textcolor{PineGreen}{\selectlanguage{french}Livre.}  \zh{量词}: \textcolor{darkblue}{\textbf{\ipa{pɤ˩}}} 
\lhead{\firstmark}
\rhead{\botmark}

\subsection{\hspace{-0.5cm} {\Large \textcolor{darkblue}{\textbf{\ipa{tʰæ˩tsɯ˧}}}}\hspace{0.5cm}[\kern2pt{\textcolor{darkblue}{\textbf{\ipa{tʰæ˩tsɯ˥}}}}\kern2pt]} \hypertarget{t\string_h\{\string_BtsM\string_M1}{}
\markboth{\textcolor{darkblue}{\textbf{\ipa{tʰæ˩tsɯ˧}}}}{}
\textcolor{teal}{\zh{名词}} \hspace{4pt} \zh{声调类:} LM.
\zh{坛子(汉语借词)。} \textcolor{Sepia}{\selectlanguage{english}Jar.} \textcolor{PineGreen}{\selectlanguage{french}Jarre.}  \zh{【借词】} \zh{坛子}

\lhead{\firstmark}
\rhead{\botmark}

\subsection{\hspace{-0.5cm} {\Large \textcolor{darkblue}{\textbf{\ipa{tʰi˧‑}}}}\hspace{0.5cm}[\kern2pt{\textcolor{darkblue}{\textbf{\ipa{tʰi˩˥}}}}\kern2pt]} \hypertarget{t\string_hi\string_M‑1}{}
\markboth{\textcolor{darkblue}{\textbf{\ipa{tʰi˧‑}}}}{}
\textcolor{teal}{\zh{前缀}} \hspace{4pt} \zh{声调类:} M/0.
\zh{持续体。} \textcolor{Sepia}{\selectlanguage{english}Durative (\mytextsc{dur}).} \textcolor{PineGreen}{\selectlanguage{french}Duratif (\mytextsc{dur}).}  ¶ \textcolor{darkblue}{\textbf{\ipa{tʰi˧-dzɯ˥-dʑo˩!}}} \zh{她在吃东西!} \textcolor{Sepia}{\selectlanguage{english}(She) is eating!} \textcolor{PineGreen}{\selectlanguage{french}(Elle) est en train de manger! / Elle mange! (Contexte: on constate avec joie qu'un enfant qui ne mangeait plus depuis deux jours est en train de ronger à belles dents un épi de maïs.)}  
 ¶ \textcolor{darkblue}{\textbf{\ipa{tʰi˧-mɤ˧-ɲi˥}}} \zh{否则、要不然} \textcolor{Sepia}{\selectlanguage{english}otherwise, or else} \textcolor{PineGreen}{\selectlanguage{french}faute de quoi}  

\lhead{\firstmark}
\rhead{\botmark}

\subsection{\hspace{-0.5cm} {\Large \textcolor{darkblue}{\textbf{\ipa{tʰi˧}}}}\hspace{0.5cm}[\kern2pt{\textcolor{darkblue}{\textbf{\ipa{tʰi˥}}}}\kern2pt]} \hypertarget{t\string_hi\string_M1}{}
\markboth{\textcolor{darkblue}{\textbf{\ipa{tʰi˧}}}}{}
\textcolor{teal}{\zh{形容词}} \hspace{4pt} \zh{声调类:} M.
\zh{能干。} \textcolor{Sepia}{\selectlanguage{english}Able; capable; competent; clever.} \textcolor{PineGreen}{\selectlanguage{french}Compétent, habile.}  ¶ \textcolor{darkblue}{\textbf{\ipa{ɖwæ˧˥ | tʰi˧}}} \zh{很能干} \textcolor{Sepia}{\selectlanguage{english}\mytextsc{intensive}.very} \textcolor{PineGreen}{\selectlanguage{french}\mytextsc{intensif}.très: très habile}  
 ¶ \textcolor{darkblue}{\textbf{\ipa{mv̩˩tʰi˩ tʰv̩˩-v̩˩˥}}} \zh{那个聪明女人} \textcolor{Sepia}{\selectlanguage{english}that clever woman} \textcolor{PineGreen}{\selectlanguage{french}cette femme intelligente}  
 ¶ \textcolor{darkblue}{\textbf{\ipa{zo˧tʰi˧}}} \zh{聪明男人} \textcolor{Sepia}{\selectlanguage{english}clever man} \textcolor{PineGreen}{\selectlanguage{french}homme intelligent}  

\lhead{\firstmark}
\rhead{\botmark}

\subsection{\hspace{-0.5cm} {\Large \textcolor{darkblue}{\textbf{\ipa{tʰi˩\textsubscript{a}}}}}\hspace{0.5cm}[\kern2pt{\textcolor{darkblue}{\textbf{\ipa{tʰi˧˥}}}}\kern2pt]} \hypertarget{t\string_hi\string_Ba1}{}
\markboth{\textcolor{darkblue}{\textbf{\ipa{tʰi˩\textsubscript{a}}}}}{}
\textcolor{teal}{\zh{动词}} \hspace{4pt} \zh{声调类:} L\textsubscript{a}.
\zh{刨。} \textcolor{Sepia}{\selectlanguage{english}To plane (wood flat).} \textcolor{PineGreen}{\selectlanguage{french}Raboter.}  ¶ \textcolor{darkblue}{\textbf{\ipa{tso˧\textasciitilde{}tso˧ tʰi˥(-ze˩)}}} \zh{刨东西} \textcolor{Sepia}{\selectlanguage{english}to plane things} \textcolor{PineGreen}{\selectlanguage{french}raboter quelque chose}  
 ¶ \textcolor{darkblue}{\textbf{\ipa{le˧-tʰi˩-ze˩}}} \zh{刨了} \textcolor{Sepia}{\selectlanguage{english}\mytextsc{accomp} \string_ \mytextsc{pfv}} \textcolor{PineGreen}{\selectlanguage{french}\mytextsc{accomp} \string_ \mytextsc{pfv}}  
 ¶ \textcolor{darkblue}{\textbf{\ipa{tso˧\textasciitilde{}tso˧ | le˧-tʰi˩(-ze˩)}}} \zh{刨东西} \textcolor{Sepia}{\selectlanguage{english}to plane things} \textcolor{PineGreen}{\selectlanguage{french}raboter quelque chose}  
 ¶ \textcolor{darkblue}{\textbf{\ipa{pæ˩pʰæ˧ tʰi˥}}} \zh{刨木板} \textcolor{Sepia}{\selectlanguage{english}to plane a plank} \textcolor{PineGreen}{\selectlanguage{french}raboter une planche}  

\lhead{\firstmark}
\rhead{\botmark}

\subsection{\hspace{-0.5cm} {\Large \textcolor{darkblue}{\textbf{\ipa{tʰi˩mi\#˥}}}}\hspace{0.5cm}[\kern2pt{\textcolor{darkblue}{\textbf{\ipa{tʰi˧mi˥}}}}\kern2pt]} \hypertarget{t\string_hi\string_Bmi\#\string_T1}{}
\markboth{\textcolor{darkblue}{\textbf{\ipa{tʰi˩mi\#˥}}}}{}
\textcolor{teal}{\zh{名词}} \hspace{4pt} \zh{声调类:} LM+\#H.
\zh{大刨。} \textcolor{Sepia}{\selectlanguage{english}Large plane.} \textcolor{PineGreen}{\selectlanguage{french}Grand rabot.}  \zh{量词}: \textcolor{darkblue}{\textbf{\ipa{nɑ˧}}} 
\lhead{\firstmark}
\rhead{\botmark}

\subsection{\hspace{-0.5cm} {\Large \textcolor{darkblue}{\textbf{\ipa{tʰi˩zo\#˥}}}}\hspace{0.5cm}[\kern2pt{\textcolor{darkblue}{\textbf{\ipa{tʰi˧zo˩}}}}\kern2pt]} \hypertarget{t\string_hi\string_Bzo\#\string_T1}{}
\markboth{\textcolor{darkblue}{\textbf{\ipa{tʰi˩zo\#˥}}}}{}
\textcolor{teal}{\zh{名词}} \hspace{4pt} \zh{声调类:} LM+\#H.
\zh{小刨。} \textcolor{Sepia}{\selectlanguage{english}Small plane.} \textcolor{PineGreen}{\selectlanguage{french}Petit rabot.}  \zh{量词}: \textcolor{darkblue}{\textbf{\ipa{nɑ˧}}} 
\lhead{\firstmark}
\rhead{\botmark}

\subsection{\hspace{-0.5cm} {\Large \textcolor{darkblue}{\textbf{\ipa{tʰi˩˥}}} \textsubscript{1}}\hspace{0.5cm}[\kern2pt{\textcolor{darkblue}{\textbf{\ipa{tʰi˩˥}}}}\kern2pt]} \hypertarget{t\string_hi\string_B\string_T1}{}
\markboth{\textcolor{darkblue}{\textbf{\ipa{tʰi˩˥}}} \textsubscript{1}}{}
\textcolor{teal}{\zh{名词}} \hspace{4pt} \zh{声调类:} LH.
\zh{刨。} \textcolor{Sepia}{\selectlanguage{english}Plane.} \textcolor{PineGreen}{\selectlanguage{french}Rabot.}  \zh{量词}: \textcolor{darkblue}{\textbf{\ipa{nɑ˧}}} 
\lhead{\firstmark}
\rhead{\botmark}

\subsection{\hspace{-0.5cm} {\Large \textcolor{darkblue}{\textbf{\ipa{tʰi˩˥}}} \textsubscript{2}}\hspace{0.5cm}[\kern2pt{\textcolor{darkblue}{\textbf{\ipa{tʰi˩˥}}}}\kern2pt]} \hypertarget{t\string_hi\string_B\string_T2}{}
\markboth{\textcolor{darkblue}{\textbf{\ipa{tʰi˩˥}}} \textsubscript{2}}{}
\textcolor{teal}{\zh{语气助词}} \hspace{4pt} \zh{声调类:} LM? LH?.
\zh{然后。} \textcolor{Sepia}{\selectlanguage{english}Discourse particle: so, then, and then.} \textcolor{PineGreen}{\selectlanguage{french}Particule de discours: alors, donc, après.} 
\lhead{\firstmark}
\rhead{\botmark}

\subsection{\hspace{-0.5cm} {\Large \textcolor{darkblue}{\textbf{\ipa{tʰo˥\textsubscript{a}}}}}\hspace{0.5cm}[\kern2pt{\textcolor{darkblue}{\textbf{\ipa{tʰo˥}}}}\kern2pt]} \hypertarget{t\string_ho\string_Ta1}{}
\markboth{\textcolor{darkblue}{\textbf{\ipa{tʰo˥\textsubscript{a}}}}}{}
\textcolor{teal}{\zh{量词}} \hspace{4pt} \zh{声调类:} H\textsubscript{a}.
\zh{量词:办法,解决的方法(一个)。} \textcolor{Sepia}{\selectlanguage{english}Classifier for solutions.} \textcolor{PineGreen}{\selectlanguage{french}Classificateur des solutions / issues heureuses.}  ¶ \textcolor{darkblue}{\textbf{\ipa{ə˧tso˧ tʰo˧ dʑo˧-kv̩˩?}}} \zh{有什么办法?} \textcolor{Sepia}{\selectlanguage{english}What can we do about it? / What can be done about it?} \textcolor{PineGreen}{\selectlanguage{french}Qu'est-ce qu'on y peut? / Qu'est-ce qu'on peut y faire?}  

\lhead{\firstmark}
\rhead{\botmark}

\subsection{\hspace{-0.5cm} {\Large \textcolor{darkblue}{\textbf{\ipa{tʰo˥\textsubscript{a}}}}}\hspace{0.5cm}[\kern2pt{\textcolor{darkblue}{\textbf{\ipa{tʰo˩˥}}}}\kern2pt]} \hypertarget{t\string_ho\string_Ta1}{}
\markboth{\textcolor{darkblue}{\textbf{\ipa{tʰo˥\textsubscript{a}}}}}{}
\textcolor{teal}{\zh{量词}} \hspace{4pt} \zh{声调类:} H\textsubscript{a}.
\zh{量词:套(汉语借词)。} \textcolor{Sepia}{\selectlanguage{english}Classifier for sets.} \textcolor{PineGreen}{\selectlanguage{french}Classificateur des ensembles, des lots.}  \zh{【借词】} \zh{套}

\lhead{\firstmark}
\rhead{\botmark}

\subsection{\hspace{-0.5cm} {\Large \textcolor{darkblue}{\textbf{\ipa{tʰo˧ɕi˩}}}}\hspace{0.5cm}[\kern2pt{\textcolor{darkblue}{\textbf{\ipa{tʰo˧ɕi˧}}}}\kern2pt]} \hypertarget{t\string_ho\string_Ms£i\string_B1}{}
\markboth{\textcolor{darkblue}{\textbf{\ipa{tʰo˧ɕi˩}}}}{}
\textcolor{teal}{\zh{名词}} \hspace{4pt} \zh{声调类:} L\#.
\zh{通信员(汉语借词)。} \textcolor{Sepia}{\selectlanguage{english}Messenger.} \textcolor{PineGreen}{\selectlanguage{french}Messager.}  \zh{【借词】} \zh{通信}
 \zh{量词}: \textcolor{darkblue}{\textbf{\ipa{v˧}}} 
\lhead{\firstmark}
\rhead{\botmark}

\subsection{\hspace{-0.5cm} {\Large \textcolor{darkblue}{\textbf{\ipa{tʰo˧ɕi˧˥}}}}\hspace{0.5cm}[\kern2pt{\textcolor{darkblue}{\textbf{\ipa{tʰo˧ɕi˩}}}}\kern2pt]} \hypertarget{t\string_ho\string_Ms£i\string_M\string_T1}{}
\markboth{\textcolor{darkblue}{\textbf{\ipa{tʰo˧ɕi˧˥}}}}{}
\textcolor{teal}{\zh{名词}} \hspace{4pt} \zh{声调类:} MH\#.
\zh{松树林。} \textcolor{Sepia}{\selectlanguage{english}Forest of conifers.} \textcolor{PineGreen}{\selectlanguage{french}Forêt de conifères.}  \zh{量词}: \textcolor{darkblue}{\textbf{\ipa{pʰæ˧˥}}} 
\lhead{\firstmark}
\rhead{\botmark}

\subsection{\hspace{-0.5cm} {\Large \textcolor{darkblue}{\textbf{\ipa{tʰo˧dzi˩}}}}\hspace{0.5cm}[\kern2pt{\textcolor{darkblue}{\textbf{\ipa{tʰo˧dzi˧˥}}}}\kern2pt]} \hypertarget{t\string_ho\string_Mdzi\string_B1}{}
\markboth{\textcolor{darkblue}{\textbf{\ipa{tʰo˧dzi˩}}}}{}
\textcolor{teal}{\zh{名词}} \hspace{4pt} \zh{声调类:} L\#.
\zh{松树。} \textcolor{Sepia}{\selectlanguage{english}Pine tree.} \textcolor{PineGreen}{\selectlanguage{french}Pin.}  \zh{量词}: \textcolor{darkblue}{\textbf{\ipa{dzi˩}}} 
\lhead{\firstmark}
\rhead{\botmark}

\subsection{\hspace{-0.5cm} {\Large \textcolor{darkblue}{\textbf{\ipa{tʰo˧dzi˩-hwæ˩tsɯ˩}}}}\hspace{0.5cm}[\kern2pt{\textcolor{darkblue}{\textbf{\ipa{xxxx non-correspondance entre le nombre de morphèmes et le nombre de tons de morphèmes}}}}\kern2pt]} \hypertarget{t\string_ho\string_Mdzi\string_B-hw\{\string_BtsM\string_B1}{}
\markboth{\textcolor{darkblue}{\textbf{\ipa{tʰo˧dzi˩-hwæ˩tsɯ˩}}}}{}
\textcolor{teal}{\zh{名词}} \hspace{4pt} \zh{声调类:} LM-.
\zh{刺猬。} \textcolor{Sepia}{\selectlanguage{english}Hedgehog.} \textcolor{PineGreen}{\selectlanguage{french}Hérisson; littéralement “souris des pins”.} 
\lhead{\firstmark}
\rhead{\botmark}

\subsection{\hspace{-0.5cm} {\Large \textcolor{darkblue}{\textbf{\ipa{tʰo˧fv̩˧}}}}\hspace{0.5cm}[\kern2pt{\textcolor{darkblue}{\textbf{\ipa{xxxx non-correspondance entre le nombre de morphèmes et le nombre de tons de morphèmes}}}}\kern2pt]} \hypertarget{t\string_ho\string_Mfv\string_=\string_M1}{}
\markboth{\textcolor{darkblue}{\textbf{\ipa{tʰo˧fv̩˧}}}}{}
\textcolor{teal}{\zh{名词}} \hspace{4pt} \zh{声调类:} M.
\zh{土匪(汉语借词)。} \textcolor{Sepia}{\selectlanguage{english}Bandit, brigand.} \textcolor{PineGreen}{\selectlanguage{french}Bandit, maraudeur.}  \zh{【借词】} \zh{土匪}

\lhead{\firstmark}
\rhead{\botmark}

\subsection{\hspace{-0.5cm} {\Large \textcolor{darkblue}{\textbf{\ipa{tʰo˧lɑ˧tɕi˧}}}}\hspace{0.5cm}[\kern2pt{\textcolor{darkblue}{\textbf{\ipa{tʰo˧lɑ˧tɕi˧}}}}\kern2pt]} \hypertarget{t\string_ho\string_MlA\string_Mts£i\string_M1}{}
\markboth{\textcolor{darkblue}{\textbf{\ipa{tʰo˧lɑ˧tɕi˧}}}}{}
\textcolor{teal}{\zh{名词}} \hspace{4pt} \zh{声调类:} M.
\zh{拖拉机(汉语借词)。} \textcolor{Sepia}{\selectlanguage{english}Tractor.} \textcolor{PineGreen}{\selectlanguage{french}Tracteur.}  \zh{【借词】} \zh{洋火}
 ¶ \textcolor{darkblue}{\textbf{\ipa{bo˩mi˧-tʰo˧lɑ˧tɕi˧}}} \zh{‘母猪拖拉机’:小型拖拉机} \textcolor{Sepia}{\selectlanguage{english}'sow-tractor': a small tractor (the first type that was introduced into Yongning)} \textcolor{PineGreen}{\selectlanguage{french}'tracteur-truie': petit tracteur (le premier modèle introduit à Yongning)}  
 \zh{量词}: \textcolor{darkblue}{\textbf{\ipa{yyyy}}} 
\lhead{\firstmark}
\rhead{\botmark}

\subsection{\hspace{-0.5cm} {\Large \textcolor{darkblue}{\textbf{\ipa{tʰo˧li˧}}}}\hspace{0.5cm}[\kern2pt{\textcolor{darkblue}{\textbf{\ipa{tʰo˧li˧}}}}\kern2pt]} \hypertarget{t\string_ho\string_Mli\string_M1}{}
\markboth{\textcolor{darkblue}{\textbf{\ipa{tʰo˧li˧}}}}{}
\textcolor{teal}{\zh{名词}} \hspace{4pt} \zh{声调类:} M.
\zh{兔子。} \textcolor{Sepia}{\selectlanguage{english}Rabbit.} \textcolor{PineGreen}{\selectlanguage{french}Lapin.}  \zh{量词}: \textcolor{darkblue}{\textbf{\ipa{mi˩}}} 
\lhead{\firstmark}
\rhead{\botmark}

\subsection{\hspace{-0.5cm} {\Large \textcolor{darkblue}{\textbf{\ipa{tʰo˧li˧kʰv̩˧˥}}}}\hspace{0.5cm}[\kern2pt{\textcolor{darkblue}{\textbf{\ipa{tʰo˧li˧kʰv̩˧˥}}}}\kern2pt]} \hypertarget{t\string_ho\string_Mli\string_Mk\string_hv\string_=\string_M\string_T1}{}
\markboth{\textcolor{darkblue}{\textbf{\ipa{tʰo˧li˧kʰv̩˧˥}}}}{}
\textcolor{teal}{\zh{名词}} \hspace{4pt} \zh{声调类:} MH\#.
\zh{兔年。} \textcolor{Sepia}{\selectlanguage{english}Year of the rabbit.} \textcolor{PineGreen}{\selectlanguage{french}Année du Lapin.} 
\lhead{\firstmark}
\rhead{\botmark}

\subsection{\hspace{-0.5cm} {\Large \textcolor{darkblue}{\textbf{\ipa{tʰo˧li˧-pʰv̩\#˥}}}}\hspace{0.5cm}[\kern2pt{\textcolor{darkblue}{\textbf{\ipa{xxxx non-correspondance entre le nombre de morphèmes et le nombre de tons de morphèmes}}}}\kern2pt]} \hypertarget{t\string_ho\string_Mli\string_M-p\string_hv\string_=\#\string_T1}{}
\markboth{\textcolor{darkblue}{\textbf{\ipa{tʰo˧li˧-pʰv̩\#˥}}}}{}
\textcolor{teal}{\zh{名词}} \hspace{4pt} \zh{声调类:} \#H.
\zh{公兔。} \textcolor{Sepia}{\selectlanguage{english}Male rabbit.} \textcolor{PineGreen}{\selectlanguage{french}Lapin mâle.}  \zh{量词}: \textcolor{darkblue}{\textbf{\ipa{mi˩}}} 
\lhead{\firstmark}
\rhead{\botmark}

\subsection{\hspace{-0.5cm} {\Large \textcolor{darkblue}{\textbf{\ipa{tʰo˧li˧-zo\#˥}}}}\hspace{0.5cm}[\kern2pt{\textcolor{darkblue}{\textbf{\ipa{xxxx non-correspondance entre le nombre de morphèmes et le nombre de tons de morphèmes}}}}\kern2pt]} \hypertarget{t\string_ho\string_Mli\string_M-zo\#\string_T1}{}
\markboth{\textcolor{darkblue}{\textbf{\ipa{tʰo˧li˧-zo\#˥}}}}{}
\textcolor{teal}{\zh{名词}} \hspace{4pt} \zh{声调类:} \#H.
\zh{小兔。} \textcolor{Sepia}{\selectlanguage{english}Baby rabbit.} \textcolor{PineGreen}{\selectlanguage{french}Petit lapin, bébé lapin.}  \zh{量词}: \textcolor{darkblue}{\textbf{\ipa{ɭɯ˧}}} 
\lhead{\firstmark}
\rhead{\botmark}

\subsection{\hspace{-0.5cm} {\Large \textcolor{darkblue}{\textbf{\ipa{tʰo˧-mo˩}}}}\hspace{0.5cm}[\kern2pt{\textcolor{darkblue}{\textbf{\ipa{xxxx non-correspondance entre le nombre de morphèmes et le nombre de tons de morphèmes}}}}\kern2pt]} \hypertarget{t\string_ho\string_M-mo\string_B1}{}
\markboth{\textcolor{darkblue}{\textbf{\ipa{tʰo˧-mo˩}}}}{}
\textcolor{teal}{\zh{名词}} \hspace{4pt} \zh{声调类:} L\#.
\zh{“松树菌”:一种菌子。} \textcolor{Sepia}{\selectlanguage{english}“Pine-tree mushroom”: an edible mushroom often found close to pine trees.} \textcolor{PineGreen}{\selectlanguage{french}“champignon des sapins”: champignon comestible, ainsi nommé parce qu'il pousse au pied des sapins.} 
\lhead{\firstmark}
\rhead{\botmark}

\subsection{\hspace{-0.5cm} {\Large \textcolor{darkblue}{\textbf{\ipa{tʰo˧ɻæ˥}}}}\hspace{0.5cm}[\kern2pt{\textcolor{darkblue}{\textbf{\ipa{tʰo˧ɻæ˥}}}}\kern2pt]} \hypertarget{t\string_ho\string_Mr£`\{\string_T1}{}
\markboth{\textcolor{darkblue}{\textbf{\ipa{tʰo˧ɻæ˥}}}}{}
\textcolor{teal}{\zh{名词}} \hspace{4pt} \zh{声调类:} H\#.
\zh{松子。} \textcolor{Sepia}{\selectlanguage{english}Pine-nut kernel.} \textcolor{PineGreen}{\selectlanguage{french}Pignon de pin (graine comestible).}  \zh{量词}: \textcolor{darkblue}{\textbf{\ipa{ʈʂwɤ˧}}} 
\lhead{\firstmark}
\rhead{\botmark}

\subsection{\hspace{-0.5cm} {\Large \textcolor{darkblue}{\textbf{\ipa{tʰo˧tsʰe˧-ʁwɤ\#˥}}}}\hspace{0.5cm}[\kern2pt{\textcolor{darkblue}{\textbf{\ipa{xxxx non-correspondance entre le nombre de morphèmes et le nombre de tons de morphèmes}}}}\kern2pt]} \hypertarget{t\string_ho\string_Mts\string_he\string_M-Rw7\#\string_T1}{}
\markboth{\textcolor{darkblue}{\textbf{\ipa{tʰo˧tsʰe˧-ʁwɤ\#˥}}}}{}
\textcolor{teal}{\zh{名词}} \hspace{4pt} \zh{声调类:} \#H.
\zh{温泉乡的一个村落。} \textcolor{Sepia}{\selectlanguage{english}A village close to the Hot Springs.} \textcolor{PineGreen}{\selectlanguage{french}Un village proche des Sources Chaudes.}  ¶ \textcolor{darkblue}{\textbf{\ipa{tʰo˧tsʰe\#˥}}} \zh{同上} \textcolor{Sepia}{\selectlanguage{english}same meaning} \textcolor{PineGreen}{\selectlanguage{french}même sens}  
 ¶ \textcolor{darkblue}{\textbf{\ipa{ə˧go˧-ʁwɤ˧, | ʁwɤ˧lɑ˩-bi˩, | bæ˧ʁwɤ˧, | tʰo˧tsʰe\#˥, | pi˧tsʰe˩-di˩, | pɤ˧dʑɤ˩-di˩, | ʁwɤ˧tv̩˧}}} \zh{永宁背向泸沽湖方向经过的村落。前两个村落拥有相当大的摩梭人口比例,第三个村落是摩梭村,最后一个是普米村。} \textcolor{Sepia}{\selectlanguage{english}Villages that one encounters as one leaves the plain of Yongning (away from the Lake); the first two are perceived as villages with a high proportion of Na members, and the third as a mostly Na village, whereas the next ones are Pumi (Prinmi).} \textcolor{PineGreen}{\selectlanguage{french}Villages au sortir de la plaine de Yongning; les deux premiers comportent une population na; le troisième est un village na; les suivants sont essentiellement des villages pumi/prinmi.}  
 ¶ \textcolor{darkblue}{\textbf{\ipa{tʰo˧tsʰe˧: | bɤ˧!}}} \zh{fv:/tʰo˧tsʰe˧/是一个普米族村落!} \textcolor{Sepia}{\selectlanguage{english}\textcolor{darkblue}{\textbf{\ipa{/tʰo˧tsʰe˧/}}} is a Pumi village!} \textcolor{PineGreen}{\selectlanguage{french}\textcolor{darkblue}{\textbf{\ipa{/tʰo˧tsʰe˧/}}}, c'est un village pumi!}  

\lhead{\firstmark}
\rhead{\botmark}

\subsection{\hspace{-0.5cm} {\Large \textcolor{darkblue}{\textbf{\ipa{tʰo˧ʈɯ\#˥}}}}\hspace{0.5cm}[\kern2pt{\textcolor{darkblue}{\textbf{\ipa{tʰo˩ʈɯ˩˥}}}}\kern2pt]} \hypertarget{t\string_ho\string_Mt`M\#\string_T1}{}
\markboth{\textcolor{darkblue}{\textbf{\ipa{tʰo˧ʈɯ\#˥}}}}{}
\textcolor{teal}{\zh{名词}} \hspace{4pt} \zh{声调类:} \#H.
\zh{拖支开基村(永宁的一个村落)。} \textcolor{Sepia}{\selectlanguage{english}A village in Yongning; Chinese: Tuozhikaiji.} \textcolor{PineGreen}{\selectlanguage{french}Un village de Yongning: Tuozhikaiji.}  ¶ \textcolor{darkblue}{\textbf{\ipa{dʑɤ˩bv̩˧kɤ˧-sɑ˥ʁwɤ˩, | hi˩ʁwɤ˩-lo˥, | æ˩mi˧-ʁwɤ\#˥, | lɑ˧lo˧-ʁwɤ˥, | lɑ˧ŋwɤ˧, | bɤ˧tsʰo˧gv̩˥, | ə˧lɑ˧-ʁwɤ\#˥, | gæ˧ɻæ˩, | qʰæ˧tɕʰi˧, | tʰo˧ʈɯ\#˥}}} \zh{摩梭传统地理概念中,属于永宁的十个村落} \textcolor{Sepia}{\selectlanguage{english}the ten villages traditionally considered as part of Yongning} \textcolor{PineGreen}{\selectlanguage{french}les dix villages comptant traditionnellement comme faisant partie de Yongning}  

\lhead{\firstmark}
\rhead{\botmark}

\subsection{\hspace{-0.5cm} {\Large \textcolor{darkblue}{\textbf{\ipa{tʰo˧ʐv̩˥}}}}\hspace{0.5cm}[\kern2pt{\textcolor{darkblue}{\textbf{\ipa{tʰo˩ʐv̩˩˥}}}}\kern2pt]} \hypertarget{t\string_ho\string_Mz`v\string_=\string_T1}{}
\markboth{\textcolor{darkblue}{\textbf{\ipa{tʰo˧ʐv̩˥}}}}{}
\textcolor{teal}{\zh{名词}} \hspace{4pt} \zh{声调类:} H\#.
\zh{鸽子。} \textcolor{Sepia}{\selectlanguage{english}Pigeon.} \textcolor{PineGreen}{\selectlanguage{french}Pigeon.}  ¶ \textcolor{darkblue}{\textbf{\ipa{tʰo˧ʐv̩˥-mi˩}}} \zh{母鸽子} \textcolor{Sepia}{\selectlanguage{english}female pigeon} \textcolor{PineGreen}{\selectlanguage{french}pigeon femelle}  
 ¶ \textcolor{darkblue}{\textbf{\ipa{tʰo˧ʐv̩˥-pʰv̩˩}}} \zh{公鸽子} \textcolor{Sepia}{\selectlanguage{english}male pigeon} \textcolor{PineGreen}{\selectlanguage{french}pigeon mâle}  
 ¶ \textcolor{darkblue}{\textbf{\ipa{tʰo˧ʐv̩˥-zo˩}}} \zh{小鸽子} \textcolor{Sepia}{\selectlanguage{english}baby pigeon} \textcolor{PineGreen}{\selectlanguage{french}petit pigeon}  
 \zh{量词}: \textcolor{darkblue}{\textbf{\ipa{mi˩}}} 
\lhead{\firstmark}
\rhead{\botmark}

\subsection{\hspace{-0.5cm} {\Large \textcolor{darkblue}{\textbf{\ipa{tʰo˩\textsubscript{a}}}}}\hspace{0.5cm}[\kern2pt{\textcolor{darkblue}{\textbf{\ipa{tʰo˥}}}}\kern2pt]} \hypertarget{t\string_ho\string_Ba1}{}
\markboth{\textcolor{darkblue}{\textbf{\ipa{tʰo˩\textsubscript{a}}}}}{}
\textcolor{teal}{\zh{动词}} \hspace{4pt} \zh{声调类:} L\textsubscript{a}.
\zh{靠。} \textcolor{Sepia}{\selectlanguage{english}To lean on.} \textcolor{PineGreen}{\selectlanguage{french}S’adosser à, s'appuyer.}  ¶ \textcolor{darkblue}{\textbf{\ipa{tʰi˧-tʰo˩}}} \zh{\mytextsc{dur}} \textcolor{Sepia}{\selectlanguage{english}\mytextsc{dur}} \textcolor{PineGreen}{\selectlanguage{french}\mytextsc{dur}}  
 ¶ \textcolor{darkblue}{\textbf{\ipa{tʰi˧-tʰo˩-ɻ̍˩}}} \zh{\mytextsc{dur} \string_ \mytextsc{inceptive}} \textcolor{Sepia}{\selectlanguage{english}\mytextsc{dur} \string_ \mytextsc{inceptive}} \textcolor{PineGreen}{\selectlanguage{french}\mytextsc{dur} \string_ \mytextsc{inchoatif}}  
 ¶ \textcolor{darkblue}{\textbf{\ipa{ɖɯ˧-tʰo˩-ɻ̍˩}}} \zh{\mytextsc{delimitative} \string_ \mytextsc{inceptive}} \textcolor{Sepia}{\selectlanguage{english}\mytextsc{delimitative} \string_ \mytextsc{inceptive}} \textcolor{PineGreen}{\selectlanguage{french}\mytextsc{délimitatif} \string_ \mytextsc{inchoatif}}  

\lhead{\firstmark}
\rhead{\botmark}

\subsection{\hspace{-0.5cm} {\Large \textcolor{darkblue}{\textbf{\ipa{tʰo˩lo˧}}}}\hspace{0.5cm}[\kern2pt{\textcolor{darkblue}{\textbf{\ipa{tʰo˩lo˥}}}}\kern2pt]} \hypertarget{t\string_ho\string_Blo\string_M1}{}
\markboth{\textcolor{darkblue}{\textbf{\ipa{tʰo˩lo˧}}}}{}
\textcolor{teal}{\zh{名词}} \hspace{4pt} \zh{声调类:} LM.
\zh{头马:马帮里走在最前面的那匹马。} \textcolor{Sepia}{\selectlanguage{english}The horse walking in front in a caravan.} \textcolor{PineGreen}{\selectlanguage{french}Cheval de tête, dans une caravane.} 
\lhead{\firstmark}
\rhead{\botmark}

\subsection{\hspace{-0.5cm} {\Large \textcolor{darkblue}{\textbf{\ipa{tʰo˩pʰv̩˧tɕʰɤ˧}}}}\hspace{0.5cm}[\kern2pt{\textcolor{darkblue}{\textbf{\ipa{tʰo˩pʰv̩˧tɕʰɤ˧}}}}\kern2pt]} \hypertarget{t\string_ho\string_Bp\string_hv\string_=\string_Mts£\string_h7\string_M1}{}
\markboth{\textcolor{darkblue}{\textbf{\ipa{tʰo˩pʰv̩˧tɕʰɤ˧}}}}{}
\textcolor{teal}{\zh{名词}} \hspace{4pt} \zh{声调类:} LM.
\zh{枪,明火枪。} \textcolor{Sepia}{\selectlanguage{english}Gun; firelock; rifle.} \textcolor{PineGreen}{\selectlanguage{french}Arme à feu, fusil; arquebuse.}  \zh{【借词】}?
 \zh{量词}: \textcolor{darkblue}{\textbf{\ipa{kʰɯ˩}}} 
\lhead{\firstmark}
\rhead{\botmark}

\subsection{\hspace{-0.5cm} {\Large \textcolor{darkblue}{\textbf{\ipa{tʰo˩ʁæ˩}}}}\hspace{0.5cm}[\kern2pt{\textcolor{darkblue}{\textbf{\ipa{tʰo˩ʁæ˩˥}}}}\kern2pt]} \hypertarget{t\string_ho\string_BR\{\string_B1}{}
\markboth{\textcolor{darkblue}{\textbf{\ipa{tʰo˩ʁæ˩}}}}{}
\textcolor{teal}{\zh{名词}} \hspace{4pt} \zh{声调类:} L.
\zh{松香。} \textcolor{Sepia}{\selectlanguage{english}Pine resin; colophony.} \textcolor{PineGreen}{\selectlanguage{french}Résine de pin.}  \zh{量词}: \textcolor{darkblue}{\textbf{\ipa{ʈʰɤ˥}}} 
\lhead{\firstmark}
\rhead{\botmark}

\subsection{\hspace{-0.5cm} {\Large \textcolor{darkblue}{\textbf{\ipa{tʰo˩ʂv̩˩}}}}\hspace{0.5cm}[\kern2pt{\textcolor{darkblue}{\textbf{\ipa{tʰo˩ʂv̩˩˥}}}}\kern2pt]} \hypertarget{t\string_ho\string_Bs`v\string_=\string_B1}{}
\markboth{\textcolor{darkblue}{\textbf{\ipa{tʰo˩ʂv̩˩}}}}{}
\textcolor{teal}{\zh{名词}} \hspace{4pt} \zh{声调类:} L.
\zh{树针。} \textcolor{Sepia}{\selectlanguage{english}Pine needles.} \textcolor{PineGreen}{\selectlanguage{french}Aiguilles de pin.}  \zh{量词}: \textcolor{darkblue}{\textbf{\ipa{qɑ˩}}} 
\lhead{\firstmark}
\rhead{\botmark}

\subsection{\hspace{-0.5cm} {\Large \textcolor{darkblue}{\textbf{\ipa{tʰo˩tɕi˧˥}}}}\hspace{0.5cm}[\kern2pt{\textcolor{darkblue}{\textbf{\ipa{tʰo˩tɕi˧˥}}}}\kern2pt]} \hypertarget{t\string_ho\string_Bts£i\string_M\string_T1}{}
\markboth{\textcolor{darkblue}{\textbf{\ipa{tʰo˩tɕi˧˥}}}}{}
\textcolor{teal}{\zh{名词}} \hspace{4pt} \zh{声调类:} LM+MH\#.
\zh{砖。} \textcolor{Sepia}{\selectlanguage{english}Brick.} \textcolor{PineGreen}{\selectlanguage{french}Brique à l'ancienne: brique crue.}  \zh{量词}: \textcolor{darkblue}{\textbf{\ipa{ɭɯ˧}}} 
\lhead{\firstmark}
\rhead{\botmark}

\subsection{\hspace{-0.5cm} {\Large \textcolor{darkblue}{\textbf{\ipa{tʰv̩˧˥}}} \textsubscript{1}}\hspace{0.5cm}[\kern2pt{\textcolor{darkblue}{\textbf{\ipa{tʰv̩˧˥}}}}\kern2pt]} \hypertarget{t\string_hv\string_=\string_M\string_T1}{}
\markboth{\textcolor{darkblue}{\textbf{\ipa{tʰv̩˧˥}}} \textsubscript{1}}{}
\textcolor{teal}{\zh{动词}} \hspace{4pt} \zh{声调类:} MH.
\zh{踩。} \textcolor{Sepia}{\selectlanguage{english}To step on, to tread on, to trample.} \textcolor{PineGreen}{\selectlanguage{french}Fouler du pied, marcher sur, écraser.}  ¶ \textcolor{darkblue}{\textbf{\ipa{tʰv̩˩\textasciitilde{}tʰv̩˧˥}}} \zh{\mytextsc{重叠}} \textcolor{Sepia}{\selectlanguage{english}\mytextsc{red}} \textcolor{PineGreen}{\selectlanguage{french}\mytextsc{red}}  
 ¶ \textcolor{darkblue}{\textbf{\ipa{ɖɯ˧-tʰv̩˧ tʰi˥-tʰv̩˩}}} \zh{踢一脚} \textcolor{Sepia}{\selectlanguage{english}to give a kick, to stamp the ground} \textcolor{PineGreen}{\selectlanguage{french}donner un coup de pied par terre, fouler le sol du pied}  
 ¶ \textcolor{darkblue}{\textbf{\ipa{kʰɯ˧tsʰɤ˧ tʰv̩˥-tsʰɯ˩}}} \zh{踢一脚} \textcolor{Sepia}{\selectlanguage{english}to give a kick, to stamp the ground} \textcolor{PineGreen}{\selectlanguage{french}donner un coup de pied par terre, fouler le sol du pied}  
 ¶ \textcolor{darkblue}{\textbf{\ipa{kʰɯ˧tsʰɤ˧ tʰv̩˥\textasciitilde{}tʰv̩˩}}} \zh{踢一脚} \textcolor{Sepia}{\selectlanguage{english}to give a kick, to stamp the ground} \textcolor{PineGreen}{\selectlanguage{french}donner un coup de pied par terre, fouler le sol du pied}  
 ¶ \textcolor{darkblue}{\textbf{\ipa{kʰɯ˧tsʰɤ˧ tʰɑ˧-tʰv̩˧˥!}}} \zh{别踢!} \textcolor{Sepia}{\selectlanguage{english}Do not stamp the ground! / Do not kick/tread on something!} \textcolor{PineGreen}{\selectlanguage{french}Ne donne pas de coup de pied!}  

\lhead{\firstmark}
\rhead{\botmark}

\subsection{\hspace{-0.5cm} {\Large \textcolor{darkblue}{\textbf{\ipa{tʰv̩˧˥}}} \textsubscript{2}}\hspace{0.5cm}[\kern2pt{\textcolor{darkblue}{\textbf{\ipa{tʰv̩˧˥}}}}\kern2pt]} \hypertarget{t\string_hv\string_=\string_M\string_T2}{}
\markboth{\textcolor{darkblue}{\textbf{\ipa{tʰv̩˧˥}}} \textsubscript{2}}{}
\textcolor{teal}{\zh{动词}} \hspace{4pt} \zh{声调类:} MH.
\zh{负担(某个活动的费用,如:请全村人吃饭)。} \textcolor{Sepia}{\selectlanguage{english}To take charge of, to foot the bill (e.g. someone invites the whole village to a feast; that person provides the food, but does not necessarily do the cooking).} \textcolor{PineGreen}{\selectlanguage{french}Se charger de, préparer, offrir (quelqu'un se charge d'offrir un repas aux gens du village; c'est lui qui paie, pas forcément qui fait la cuisine).} 
\lhead{\firstmark}
\rhead{\botmark}

\subsection{\hspace{-0.5cm} {\Large \textcolor{darkblue}{\textbf{\ipa{tʰv̩˥}}} \textsubscript{1}}\hspace{0.5cm}[\kern2pt{\textcolor{darkblue}{\textbf{\ipa{tʰv̩˧˥}}}}\kern2pt]} \hypertarget{t\string_hv\string_=\string_T1}{}
\markboth{\textcolor{darkblue}{\textbf{\ipa{tʰv̩˥}}} \textsubscript{1}}{}
\textcolor{teal}{\zh{代词}} \hspace{4pt} \zh{声调类:} \#H.
\zh{那\mytextsc{指示}.远指。} \textcolor{Sepia}{\selectlanguage{english}That; distal demonstrative.} \textcolor{PineGreen}{\selectlanguage{french}Démonstratif distal, qui forme un couple avec le démonstratif proximal.}  ¶ \textcolor{darkblue}{\textbf{\ipa{tʰv̩˧ ɲi˥!}}} \zh{是那个!} \textcolor{Sepia}{\selectlanguage{english}It's that one! / That's the one!} \textcolor{PineGreen}{\selectlanguage{french}c'est celui-là!}  
 ¶ \textcolor{darkblue}{\textbf{\ipa{tʰv̩˧-v̩\#˥}}} \zh{那个} \textcolor{Sepia}{\selectlanguage{english}that one} \textcolor{PineGreen}{\selectlanguage{french}celui-là (\mytextsc{dem}.dist-\mytextsc{clf}.individu)}  
\zh{~【参考】~} \hyperlink{}{\textcolor{darkblue}{\textbf{\ipa{tʰv̩˥}}} \textsubscript{2}} \zh{~【参考】~} \hyperlink{}{\textcolor{darkblue}{\textbf{\ipa{tʰv̩˥}}} \textsubscript{3}} 
\lhead{\firstmark}
\rhead{\botmark}

\subsection{\hspace{-0.5cm} {\Large \textcolor{darkblue}{\textbf{\ipa{tʰv̩˥}}} \textsubscript{2}}\hspace{0.5cm}[\kern2pt{\textcolor{darkblue}{\textbf{\ipa{tʰv̩˥}}}}\kern2pt]} \hypertarget{t\string_hv\string_=\string_T2}{}
\markboth{\textcolor{darkblue}{\textbf{\ipa{tʰv̩˥}}} \textsubscript{2}}{}
\textcolor{teal}{\zh{代词}} \hspace{4pt} \zh{声调类:} \#H.
\zh{他。} \textcolor{Sepia}{\selectlanguage{english}3rd person singular.} \textcolor{PineGreen}{\selectlanguage{french}Pronom de troisième personne du singulier; provient du démonstratif distal.}  ¶ \textcolor{darkblue}{\textbf{\ipa{tʰv̩˧=ɻ̍˩}}} \zh{他家、他家族、他的人} \textcolor{Sepia}{\selectlanguage{english}his family, his household, his clan, his kin} \textcolor{PineGreen}{\selectlanguage{french}sa famille, sa maisonnée, les siens}  
\zh{~【参考】~} \hyperlink{}{\textcolor{darkblue}{\textbf{\ipa{tʰv̩˥}}} \textsubscript{1}} \zh{~【参考】~} \hyperlink{}{\textcolor{darkblue}{\textbf{\ipa{tʰv̩˥}}} \textsubscript{3}} 
\lhead{\firstmark}
\rhead{\botmark}

\subsection{\hspace{-0.5cm} {\Large \textcolor{darkblue}{\textbf{\ipa{tʰv̩˥}}} \textsubscript{3}}\hspace{0.5cm}[\kern2pt{\textcolor{darkblue}{\textbf{\ipa{tʰv̩˥}}}}\kern2pt]} \hypertarget{t\string_hv\string_=\string_T3}{}
\markboth{\textcolor{darkblue}{\textbf{\ipa{tʰv̩˥}}} \textsubscript{3}}{}
\textcolor{teal}{\zh{后缀}} \hspace{4pt} \zh{声调类:} \#H.
\zh{\mytextsc{主题(°指示}.远指)。} \textcolor{Sepia}{\selectlanguage{english}Topic marker; grammaticalized from the distal demonstrative.} \textcolor{PineGreen}{\selectlanguage{french}Focalisateur; grammaticalisé à partir du démonstratif distal.} \zh{~【参考】~} \hyperlink{}{\textcolor{darkblue}{\textbf{\ipa{tʰv̩˥}}} \textsubscript{1}} \zh{~【参考】~} \hyperlink{}{\textcolor{darkblue}{\textbf{\ipa{tʰv̩˥}}} \textsubscript{2}} 
\lhead{\firstmark}
\rhead{\botmark}

\subsection{\hspace{-0.5cm} {\Large \textcolor{darkblue}{\textbf{\ipa{tʰv̩˧\textsubscript{a}}}}}\hspace{0.5cm}[\kern2pt{\textcolor{darkblue}{\textbf{\ipa{tʰv̩˥}}}}\kern2pt]} \hypertarget{t\string_hv\string_=\string_Ma1}{}
\markboth{\textcolor{darkblue}{\textbf{\ipa{tʰv̩˧\textsubscript{a}}}}}{}
\textcolor{teal}{\zh{动词}} \hspace{4pt} \zh{声调类:} M\textsubscript{a}.
\ding{202} \zh{出来。} \textcolor{Sepia}{\selectlanguage{english}To come out.} \textcolor{PineGreen}{\selectlanguage{french}Sortir.}  ¶ \textcolor{darkblue}{\textbf{\ipa{ɑ˩pʰo˩ tʰv̩˩˥}}} \zh{出来,如:动物从地洞里爬出来} \textcolor{Sepia}{\selectlanguage{english}to come out: e.g. an animal comes out of its burrow} \textcolor{PineGreen}{\selectlanguage{french}sortir, ex.: un animal sort de son terrier}  
 ¶ \textcolor{darkblue}{\textbf{\ipa{ɲi˧mi˧ tʰv̩˧}}} \zh{太阳出来} \textcolor{Sepia}{\selectlanguage{english}the sun comes out} \textcolor{PineGreen}{\selectlanguage{french}le soleil paraît}  
\ding{203} \zh{刮(风)。} \textcolor{Sepia}{\selectlanguage{english}To rise (wind).} \textcolor{PineGreen}{\selectlanguage{french}Souffler (vent).} \ding{204} \zh{发芽、抽芽。} \textcolor{Sepia}{\selectlanguage{english}To bud, to sprout (a tree sprouts).} \textcolor{PineGreen}{\selectlanguage{french}Germer, bourgeonner, donner des bourgeons.}  ¶ \textcolor{darkblue}{\textbf{\ipa{si˧dzi˩ | ʁo˧bv̩˧ tʰv̩˧}}} \zh{树抽芽} \textcolor{Sepia}{\selectlanguage{english}the tree buds} \textcolor{PineGreen}{\selectlanguage{french}l'arbre fait des bourgeons}  
\ding{205} \zh{出现。} \textcolor{Sepia}{\selectlanguage{english}To appear, to happen, to get (a wound).} \textcolor{PineGreen}{\selectlanguage{french}Apparaître, se faire: une blessure apparaît, on reçoit une blessure.}  ¶ \textcolor{darkblue}{\textbf{\ipa{mi˧ tʰv̩˧}}} \zh{受伤} \textcolor{Sepia}{\selectlanguage{english}to get wounded} \textcolor{PineGreen}{\selectlanguage{french}se faire une blessure/avoir une blessure/se blesser}  
 ¶ \textcolor{darkblue}{\textbf{\ipa{ɖɯ˧-v̩˧ mi˧ tʰv̩˧-ze˧!}}} \zh{有人受伤了!} \textcolor{Sepia}{\selectlanguage{english}someone has got wounded!} \textcolor{PineGreen}{\selectlanguage{french}quelqu'un s'est blessé!}  
\ding{206} \zh{建立、创造、制造出来。} \textcolor{Sepia}{\selectlanguage{english}To create; to found.} \textcolor{PineGreen}{\selectlanguage{french}Créer, fonder; se trouver, se fabriquer.}  ¶ \textcolor{darkblue}{\textbf{\ipa{ʑi˩ tʰv̩˩}}} \zh{分家、建立新家} \textcolor{Sepia}{\selectlanguage{english}to found a new home} \textcolor{PineGreen}{\selectlanguage{french}créer une nouvelle maison, fonder une nouvelle maison; traduit en chinois par \zh{分家}, concept en fait assez différent dans la mesure où \textcolor{darkblue}{\textbf{\ipa{ʑi˩ tʰv̩˩}}} évoque un essaimage, plutôt qu'une séparation.}  
 ¶ \textcolor{darkblue}{\textbf{\ipa{ʈʂʰɯ˧ | ʑi˩ tʰv̩˩-ze˥!}}} \zh{他建了新家!} \textcolor{Sepia}{\selectlanguage{english}(S)he founded a new home!} \textcolor{PineGreen}{\selectlanguage{french}Il/elle a fondé sa propre maisonnée!}  
 ¶ \textcolor{darkblue}{\textbf{\ipa{ʈʂʰɯ˧ | ʑi˩ tʰv̩˩-bi˩˥!}}} \zh{他要建个新家!} \textcolor{Sepia}{\selectlanguage{english}(S)he is going to found a new home!} \textcolor{PineGreen}{\selectlanguage{french}Il/elle va fonder sa propre maisonnée!}  
 ¶ \textcolor{darkblue}{\textbf{\ipa{lo˧ mɤ˧-dʑo˧, | lo˧ tʰv̩˧˥! / no˧ | lo˧ mɤ˧-dʑo˧, | lo˧ tʰv̩˧-ɲi˥!}}} \zh{自找麻烦!(这句,除贬义用法,还能用来表扬,如表扬一位当官的人努力去做好事,给自己找有意义的事情干。)} \textcolor{Sepia}{\selectlanguage{english}[(S)he] has no obligations, and yet (s)he works a lot / (s)he finds tasks to do! (A compliment to a civil servant who could be content to pocket a salary every month but who sets goals for her/himself and looks for useful tasks to accomplish. The sentence can also be used negatively, to criticize someone who takes up unnecessary tasks instead of keeping quiet.)} \textcolor{PineGreen}{\selectlanguage{french}Il n'a pas d'obligations, et pourtant il travaille! (Compliment à l'endroit d'un fonctionnaire qui pourrait se contenter de percevoir son salaire, mais qui se donne à lui-même des objectifs et des tâches à accomplir. La phrase peut également être employée de façon négative, pour critiquer quelqu'un qui déploie une activité inutile au lieu de se tenir tranquille.)}  

\lhead{\firstmark}
\rhead{\botmark}

\subsection{\hspace{-0.5cm} {\Large \textcolor{darkblue}{\textbf{\ipa{tʰv̩˧˥\textsubscript{a}}}}}\hspace{0.5cm}[\kern2pt{\textcolor{darkblue}{\textbf{\ipa{tʰv̩˥}}}}\kern2pt]} \hypertarget{t\string_hv\string_=\string_M\string_Ta1}{}
\markboth{\textcolor{darkblue}{\textbf{\ipa{tʰv̩˧˥\textsubscript{a}}}}}{}
\textcolor{teal}{\zh{量词}} \hspace{4pt} \zh{声调类:} MH\textsubscript{a}.
\zh{量词:步。} \textcolor{Sepia}{\selectlanguage{english}Classifier for steps (in walking).} \textcolor{PineGreen}{\selectlanguage{french}Pas, enjambée.}  ¶ \textcolor{darkblue}{\textbf{\ipa{ɖɯ˧-tʰv̩˧\textasciitilde{}ɖɯ˥-tʰv̩˩}}} \zh{一步一步} \textcolor{Sepia}{\selectlanguage{english}step by step, one step after the other} \textcolor{PineGreen}{\selectlanguage{french}pas à pas}  
 ¶ \textcolor{darkblue}{\textbf{\ipa{ɖɯ˧-tʰv̩˧˥, | ɖɯ˧-tʰv̩˧˥}}} \zh{一步又一步} \textcolor{Sepia}{\selectlanguage{english}step by step, one step after the other; same as above, but detaching the two parts of the phrase; this is closer to repetition than to reduplication} \textcolor{PineGreen}{\selectlanguage{french}idem, détachant les deux parties; cette forme est plus proche d'une répétition que d'une réduplication}  

\lhead{\firstmark}
\rhead{\botmark}

\subsection{\hspace{-0.5cm} {\Large \textcolor{darkblue}{\textbf{\ipa{tʰv̩˩\textsubscript{b}}}}}\hspace{0.5cm}[\kern2pt{\textcolor{darkblue}{\textbf{\ipa{tʰv̩˩˥}}}}\kern2pt]} \hypertarget{t\string_hv\string_=\string_Bb1}{}
\markboth{\textcolor{darkblue}{\textbf{\ipa{tʰv̩˩\textsubscript{b}}}}}{}
\textcolor{teal}{\zh{量词}} \hspace{4pt} \zh{声调类:} L\textsubscript{b}.
\zh{量词:一套(有十个)。更早的意思是八个。} \textcolor{Sepia}{\selectlanguage{english}Classifier for sets of tens. The term used to refer to sets of eight. The word remains in use, but its meaning has shifted towards the meaning of 'sets of ten', following the generalized use of decimal numeration.} \textcolor{PineGreen}{\selectlanguage{french}Classificateur des dizaines. Autrefois, le terme servait à compter par ensembles de 8. Le terme a demeuré, mais son sens s'est déplacé vers le sens de 'dizaine', suivant la généralisation du système de numération à base dix.}  ¶ \textcolor{darkblue}{\textbf{\ipa{qʰwɤ˩˥ | ɖɯ˧-tʰv̩˩}}} \zh{一套十个碗} \textcolor{Sepia}{\selectlanguage{english}a set of ten bowls} \textcolor{PineGreen}{\selectlanguage{french}un lot de dix bols}  
 ¶ \textcolor{darkblue}{\textbf{\ipa{ɖʐɯ˧ʂɯ˥ | ɖɯ˧-tʰv̩˩}}} \zh{一套十(双)筷子} \textcolor{Sepia}{\selectlanguage{english}a set of ten (pairs of) chopsticks} \textcolor{PineGreen}{\selectlanguage{french}un paquet de dix (paires de) baguettes}  

\lhead{\firstmark}
\rhead{\botmark}

\subsection{\hspace{-0.5cm} {\Large \textcolor{darkblue}{\textbf{\ipa{tʰv̩˧\textsubscript{b}}}}}\hspace{0.5cm}[\kern2pt{\textcolor{darkblue}{\textbf{\ipa{tʰv̩˩˥}}}}\kern2pt]} \hypertarget{t\string_hv\string_=\string_Mb1}{}
\markboth{\textcolor{darkblue}{\textbf{\ipa{tʰv̩˧\textsubscript{b}}}}}{}
\textcolor{teal}{\zh{动词}} \hspace{4pt} \zh{声调类:} M\textsubscript{b}.
\zh{借给人。} \textcolor{Sepia}{\selectlanguage{english}To lend.} \textcolor{PineGreen}{\selectlanguage{french}Prêter (un objet).}  ¶ \textcolor{darkblue}{\textbf{\ipa{tso˧\textasciitilde{}tso˧ tʰv̩˧}}} \zh{借东西(给人)} \textcolor{Sepia}{\selectlanguage{english}to lend something} \textcolor{PineGreen}{\selectlanguage{french}prêter quelque chose}  

\lhead{\firstmark}
\rhead{\botmark}

\subsection{\hspace{-0.5cm} {\Large \textcolor{darkblue}{\textbf{\ipa{tʰv̩˧gi˧}}}}\hspace{0.5cm}[\kern2pt{\textcolor{darkblue}{\textbf{\ipa{tʰv̩˧gi˩}}}}\kern2pt]} \hypertarget{t\string_hv\string_=\string_Mgi\string_M1}{}
\markboth{\textcolor{darkblue}{\textbf{\ipa{tʰv̩˧gi˧}}}}{}
\textcolor{teal}{\zh{助词}} \hspace{4pt} \zh{声调类:} .
\zh{那边。} \textcolor{Sepia}{\selectlanguage{english}In that direction.} \textcolor{PineGreen}{\selectlanguage{french}Là-bas, de ce côté-là.} 
\lhead{\firstmark}
\rhead{\botmark}

\subsection{\hspace{-0.5cm} {\Large \textcolor{darkblue}{\textbf{\ipa{tʰv̩˧ne-ʝi˥}}}}\hspace{0.5cm}[\kern2pt{\textcolor{darkblue}{\textbf{\ipa{xxxx non-correspondance entre le nombre de morphèmes et le nombre de tons de morphèmes}}}}\kern2pt]} \hypertarget{t\string_hv\string_=\string_Mne-j££i\string_T1}{}
\markboth{\textcolor{darkblue}{\textbf{\ipa{tʰv̩˧ne-ʝi˥}}}}{}
\textcolor{teal}{\zh{助词}} \hspace{4pt} \zh{声调类:} MH\#.
\zh{那样。} \textcolor{Sepia}{\selectlanguage{english}In that way.} \textcolor{PineGreen}{\selectlanguage{french}Ainsi, de cette façon (adverbe de manière), contenant le démonstratif distal.} 
\lhead{\firstmark}
\rhead{\botmark}

\subsection{\hspace{-0.5cm} {\Large \textcolor{darkblue}{\textbf{\ipa{tʰv̩˧ɲi\#˥}}}}\hspace{0.5cm}[\kern2pt{\textcolor{darkblue}{\textbf{\ipa{tʰv̩˧ɲi˧}}}}\kern2pt]} \hypertarget{t\string_hv\string_=\string_MJi\#\string_T1}{}
\markboth{\textcolor{darkblue}{\textbf{\ipa{tʰv̩˧ɲi\#˥}}}}{}
\textcolor{teal}{\zh{助词}} \hspace{4pt} \zh{声调类:} \#H.
\zh{那天。} \textcolor{Sepia}{\selectlanguage{english}That day.} \textcolor{PineGreen}{\selectlanguage{french}Ce jour-là (déictique lointain).} 
\lhead{\firstmark}
\rhead{\botmark}

\subsection{\hspace{-0.5cm} {\Large \textcolor{darkblue}{\textbf{\ipa{tʰv̩˧qo˧}}}}\hspace{0.5cm}[\kern2pt{\textcolor{darkblue}{\textbf{\ipa{tʰv̩˧qo˧}}}}\kern2pt]} \hypertarget{t\string_hv\string_=\string_Mqo\string_M1}{}
\markboth{\textcolor{darkblue}{\textbf{\ipa{tʰv̩˧qo˧}}}}{}
\textcolor{teal}{\zh{代词}} \hspace{4pt} \zh{声调类:} M.
\zh{那里、那个地方。} \textcolor{Sepia}{\selectlanguage{english}There; that place.} \textcolor{PineGreen}{\selectlanguage{french}Là-bas; cet endroit-là.} 
\lhead{\firstmark}
\rhead{\botmark}

\subsection{\hspace{-0.5cm} {\Large \textcolor{darkblue}{\textbf{\ipa{tʰv̩˧-si˥}}}}\hspace{0.5cm}[\kern2pt{\textcolor{darkblue}{\textbf{\ipa{xxxx non-correspondance entre le nombre de morphèmes et le nombre de tons de morphèmes}}}}\kern2pt]} \hypertarget{t\string_hv\string_=\string_M-si\string_T1}{}
\markboth{\textcolor{darkblue}{\textbf{\ipa{tʰv̩˧-si˥}}}}{}
\textcolor{teal}{\zh{助词}} \hspace{4pt} \zh{声调类:} H\#.
\zh{多。} \textcolor{Sepia}{\selectlanguage{english}Numerous.} \textcolor{PineGreen}{\selectlanguage{french}Nombreux.}  ¶ \textcolor{darkblue}{\textbf{\ipa{mv̩˧ʁo˧=ɻ̍˥-dʑo˩, | ɻæ˩˥ | tʰv̩˧-si˥ | tʰv̩˩-jɤ˩ dʑo˩˥!}}} \zh{天上的人,有许多许多种子!(故事:“种子”)} \textcolor{Sepia}{\selectlanguage{english}The People of the Sky had seeds in profusion! (from the narrative “Seeds”)} \textcolor{PineGreen}{\selectlanguage{french}Les gens du Ciel, ils avaient des semences en abondance! (récit “Seeds”)}  

\lhead{\firstmark}
\rhead{\botmark}

\subsection{\hspace{-0.5cm} {\Large \textcolor{darkblue}{\textbf{\ipa{‑tʰv̩˧}}} \textsubscript{1}}\hspace{0.5cm}[\kern2pt{\textcolor{darkblue}{\textbf{\ipa{tʰv̩˥}}}}\kern2pt]} \hypertarget{‑t\string_hv\string_=\string_M1}{}
\markboth{\textcolor{darkblue}{\textbf{\ipa{‑tʰv̩˧}}} \textsubscript{1}}{}
\textcolor{teal}{\zh{后置词}} \hspace{4pt} \zh{声调类:} M.
\zh{到……为止。} \textcolor{Sepia}{\selectlanguage{english}Temporal postposition: up to, up until.} \textcolor{PineGreen}{\selectlanguage{french}Postposition temporelle: jusqu'à.} 
\lhead{\firstmark}
\rhead{\botmark}

\subsection{\hspace{-0.5cm} {\Large \textcolor{darkblue}{\textbf{\ipa{‑tʰv̩˧}}} \textsubscript{2}}\hspace{0.5cm}[\kern2pt{\textcolor{darkblue}{\textbf{\ipa{tʰv̩˥}}}}\kern2pt]} \hypertarget{‑t\string_hv\string_=\string_M2}{}
\markboth{\textcolor{darkblue}{\textbf{\ipa{‑tʰv̩˧}}} \textsubscript{2}}{}
\textcolor{teal}{\zh{后缀}} \hspace{4pt} \zh{声调类:} M.
\zh{……成。} \textcolor{Sepia}{\selectlanguage{english}To achieve, to attain (a goal), to complete successfully (an action); grammaticalized from the verb 'to come out'.} \textcolor{PineGreen}{\selectlanguage{french}Parvenir à, réussir à, réaliser avec succès; grammaticalisé à partir du verbe 'sortir'.}  ¶ \textcolor{darkblue}{\textbf{\ipa{lo˧ ʝi˧-mɤ˧-tʰv̩˧}}} \zh{活做不出来、活做不成(比如:一个人经常被打扰,所以不能集中工作,没有效率,要做的事做不成)} \textcolor{Sepia}{\selectlanguage{english}not to be able to complete one's task, to be unable to do one's work fully (example: someone is constantly being disturbed, and consequently can't achieve what they wanted to/can't work in a focused way)} \textcolor{PineGreen}{\selectlanguage{french}ne pas parvenir à venir à bien d'une tâche; ex.: une personne est constamment dérangée et ne parvient pas à travailler de façon concentrée}  

\lhead{\firstmark}
\rhead{\botmark}

\newpage
\section*{\centering- \textcolor{darkblue}{\textbf{\ipa{tɕ}}} -}
\subsection{\hspace{-0.5cm} {\Large \textcolor{darkblue}{\textbf{\ipa{tɕæ˧hæ˩}}}}\hspace{0.5cm}[\kern2pt{\textcolor{darkblue}{\textbf{\ipa{tɕæ˧hæ˩}}}}\kern2pt]} \hypertarget{ts£\{\string_Mh\{\string_B1}{}
\markboth{\textcolor{darkblue}{\textbf{\ipa{tɕæ˧hæ˩}}}}{}
\textcolor{teal}{\zh{名词}} \hspace{4pt} \zh{声调类:} L\#.
\zh{橡胶(汉语借词。第二个音节:未确定。)。} \textcolor{Sepia}{\selectlanguage{english}Rubber.} \textcolor{PineGreen}{\selectlanguage{french}Caoutchouc.}  \zh{【借词】} \zh{胶} +?
 ¶ \textcolor{darkblue}{\textbf{\ipa{tɕæ˧hæ˩-dzɑ˩qʰwɤ˩}}} \zh{橡胶鞋、橡胶底鞋} \textcolor{Sepia}{\selectlanguage{english}rubber shoe, shoe with a rubber sole, sports shoe} \textcolor{PineGreen}{\selectlanguage{french}chaussures à semelle en gomme/en caoutchouc; baskets}  
 \zh{量词}: \textcolor{darkblue}{\textbf{\ipa{dzi˧}}} 
\lhead{\firstmark}
\rhead{\botmark}

\subsection{\hspace{-0.5cm} {\Large \textcolor{darkblue}{\textbf{\ipa{tɕæ˧pʰv̩˩}}}}\hspace{0.5cm}[\kern2pt{\textcolor{darkblue}{\textbf{\ipa{tɕæ˧pʰv̩˩}}}}\kern2pt]} \hypertarget{ts£\{\string_Mp\string_hv\string_=\string_B1}{}
\markboth{\textcolor{darkblue}{\textbf{\ipa{tɕæ˧pʰv̩˩}}}}{}
\textcolor{teal}{\zh{形容词}} \hspace{4pt} \zh{声调类:} L\#.
\zh{白(脸、衣服)。} \textcolor{Sepia}{\selectlanguage{english}White.} \textcolor{PineGreen}{\selectlanguage{french}Blanc (visage, habits, cheveux...).}  ¶ \textcolor{darkblue}{\textbf{\ipa{tɕæ˧pʰv̩˩-bɑ˩lɑ˩}}} \zh{白的衣服} \textcolor{Sepia}{\selectlanguage{english}white clothes} \textcolor{PineGreen}{\selectlanguage{french}vêtement blanc}  
 ¶ \textcolor{darkblue}{\textbf{\ipa{tɕæ˧pʰv̩˩-ʈæ˩qʰwɤ˩}}} \zh{白色裙子} \textcolor{Sepia}{\selectlanguage{english}white skirt} \textcolor{PineGreen}{\selectlanguage{french}robe blanche}  

\lhead{\firstmark}
\rhead{\botmark}

\subsection{\hspace{-0.5cm} {\Large \textcolor{darkblue}{\textbf{\ipa{tɕæ˧ɻæ˩}}}}\hspace{0.5cm}[\kern2pt{\textcolor{darkblue}{\textbf{\ipa{tɕæ˧ɻæ˩}}}}\kern2pt]} \hypertarget{ts£\{\string_Mr£`\{\string_B1}{}
\markboth{\textcolor{darkblue}{\textbf{\ipa{tɕæ˧ɻæ˩}}}}{}
\textcolor{teal}{\zh{名词}} \hspace{4pt} \zh{声调类:} L\#.
\zh{酸菜、泡菜。} \textcolor{Sepia}{\selectlanguage{english}Pickled vegetables.} \textcolor{PineGreen}{\selectlanguage{french}Légumes en saumure. On en mangeait une sorte chaque jour pendant la saison d'hiver: un jour navet en saumure, etc.}  ¶ \textcolor{darkblue}{\textbf{\ipa{wo˩-tɕæ˩ɻæ˥}}} \zh{圆根叶子酸菜} \textcolor{Sepia}{\selectlanguage{english}pickled turnip leaves} \textcolor{PineGreen}{\selectlanguage{french}feuilles de navet conservées dans la saumure}  
 ¶ \textcolor{darkblue}{\textbf{\ipa{tsʰɑ˧-tɕæ˧ɻæ˥}}}  
 ¶ \textcolor{darkblue}{\textbf{\ipa{ɬi˩bi˩-tɕæ˩ɻæ˥}}} \zh{圆根酸菜} \textcolor{Sepia}{\selectlanguage{english}pickled turnip} \textcolor{PineGreen}{\selectlanguage{french}navet conservé dans la saumure}  
 ¶ \textcolor{darkblue}{\textbf{\ipa{pɤ˧pɤ˧tsʰɯ˧-tɕæ˧ɻæ˥}}} \zh{圆白菜酸菜} \textcolor{Sepia}{\selectlanguage{english}picked Chinese cabbage} \textcolor{PineGreen}{\selectlanguage{french}chou chinois en saumure}  

\lhead{\firstmark}
\rhead{\botmark}

\subsection{\hspace{-0.5cm} {\Large \textcolor{darkblue}{\textbf{\ipa{tɕɤ}}}}\hspace{0.5cm}[\kern2pt{\textcolor{darkblue}{\textbf{\ipa{[]}}}}\kern2pt]} \hypertarget{ts£71}{}
\markboth{\textcolor{darkblue}{\textbf{\ipa{tɕɤ}}}}{}
\textcolor{teal}{\zh{感叹词}} \hspace{4pt} \zh{声调类:} 0.
\zh{感叹词:嘿!。} \textcolor{Sepia}{\selectlanguage{english}Interjection: hey!} \textcolor{PineGreen}{\selectlanguage{french}Interjection: tiens! eh!} 
\lhead{\firstmark}
\rhead{\botmark}

\subsection{\hspace{-0.5cm} {\Large \textcolor{darkblue}{\textbf{\ipa{tɕɤ˥}}}}\hspace{0.5cm}[\kern2pt{\textcolor{darkblue}{\textbf{\ipa{tɕɤ˥}}}}\kern2pt]} \hypertarget{ts£7\string_T1}{}
\markboth{\textcolor{darkblue}{\textbf{\ipa{tɕɤ˥}}}}{}
\textcolor{teal}{\zh{动词}} \hspace{4pt} \zh{声调类:} H.
\zh{褪色。} \textcolor{Sepia}{\selectlanguage{english}To fade (of colours).} \textcolor{PineGreen}{\selectlanguage{french}S'effacer (couleur).}  ¶ \textcolor{darkblue}{\textbf{\ipa{le˧-tɕɤ˥-ze˩}}} \zh{褪色了} \textcolor{Sepia}{\selectlanguage{english}\mytextsc{accomp} \string_ \mytextsc{pfv}} \textcolor{PineGreen}{\selectlanguage{french}\mytextsc{accomp} \string_ \mytextsc{pfv}}  

\lhead{\firstmark}
\rhead{\botmark}

\subsection{\hspace{-0.5cm} {\Large \textcolor{darkblue}{\textbf{\ipa{tɕɤ˧fv̩˩}}}}\hspace{0.5cm}[\kern2pt{\textcolor{darkblue}{\textbf{\ipa{tɕɤ˧fv̩˩}}}}\kern2pt]} \hypertarget{ts£7\string_Mfv\string_=\string_B1}{}
\markboth{\textcolor{darkblue}{\textbf{\ipa{tɕɤ˧fv̩˩}}}}{}
\textcolor{teal}{\zh{名词}} \hspace{4pt} \zh{声调类:} L\#.
\zh{塑料桶等存水用的容器。} \textcolor{Sepia}{\selectlanguage{english}Container for liquids, such as plastic jerricans; used to store and transport drinking water.} \textcolor{PineGreen}{\selectlanguage{french}Container pour liquides; s'emploie pour désigner les containers en matière plastique.}  \zh{量词}: \textcolor{darkblue}{\textbf{\ipa{ɭɯ˧}}} 
\lhead{\firstmark}
\rhead{\botmark}

\subsection{\hspace{-0.5cm} {\Large \textcolor{darkblue}{\textbf{\ipa{tɕɤ˧ho˩pæ˧}}}}\hspace{0.5cm}[\kern2pt{\textcolor{darkblue}{\textbf{\ipa{xxxx ton non trouvé, à faire manuellement...}}}}\kern2pt]} \hypertarget{ts£7\string_Mho\string_Bp\{\string_M1}{}
\markboth{\textcolor{darkblue}{\textbf{\ipa{tɕɤ˧ho˩pæ˧}}}}{}
\textcolor{teal}{\zh{名词}} \hspace{4pt} \zh{声调类:} MLM.
\zh{胶合板(汉语借词)。} \textcolor{Sepia}{\selectlanguage{english}Plywood, veneer board.} \textcolor{PineGreen}{\selectlanguage{french}Contreplaqué, panneau en contreplaqué.}  \zh{【借词】} \zh{胶合板}

\lhead{\firstmark}
\rhead{\botmark}

\subsection{\hspace{-0.5cm} {\Large \textcolor{darkblue}{\textbf{\ipa{tɕɤ˧qʰɑ\#˥}}}}\hspace{0.5cm}[\kern2pt{\textcolor{darkblue}{\textbf{\ipa{tɕɤ˧qʰɑ˧}}}}\kern2pt]} \hypertarget{ts£7\string_Mq\string_hA\#\string_T1}{}
\markboth{\textcolor{darkblue}{\textbf{\ipa{tɕɤ˧qʰɑ\#˥}}}}{}
\textcolor{teal}{\zh{名词}} \hspace{4pt} \zh{声调类:} \#H.
\zh{蒿、青蒿。} \textcolor{Sepia}{\selectlanguage{english}Mugwort, wormwood, \textit{Artemisia vulgaris}.} \textcolor{PineGreen}{\selectlanguage{french}Armoise, \textit{Artemisia vulgaris}.} \zh{当地汉语方言:}\zh{蒿草、蒿枝。} ¶ \textcolor{darkblue}{\textbf{\ipa{tɕɤ˧qʰɑ˧-mo˩}}} \zh{一种可以吃的菌子,长在蒿附近} \textcolor{Sepia}{\selectlanguage{english}a type of edible mushroom, called 'mugwort mushroom' because it grows close to mugwort} \textcolor{PineGreen}{\selectlanguage{french}un champignon comestible, nommé 'champignon de l'armoise' parce qu'il croît à proximité de l'armoise}  
\zh{~【参考】~} \hyperlink{}{\textcolor{darkblue}{\textbf{\ipa{ho˧ʈʂɯ˧}}}} 
\lhead{\firstmark}
\rhead{\botmark}

\subsection{\hspace{-0.5cm} {\Large \textcolor{darkblue}{\textbf{\ipa{tɕɤ˧tɑ˧}}}}\hspace{0.5cm}[\kern2pt{\textcolor{darkblue}{\textbf{\ipa{tɕɤ˧tɑ˧}}}}\kern2pt]} \hypertarget{ts£7\string_MtA\string_M1}{}
\markboth{\textcolor{darkblue}{\textbf{\ipa{tɕɤ˧tɑ˧}}}}{}
\textcolor{teal}{\zh{名词}} \hspace{4pt} \zh{声调类:} M.
\zh{牛轭(单行)(汉语借词)。} \textcolor{Sepia}{\selectlanguage{english}Yoke.} \textcolor{PineGreen}{\selectlanguage{french}Joug.} \zh{当地汉语方言:}\zh{牛夹担、牛枷档、牛拴。} \zh{【借词】} \zh{夹担}
 ¶ \textcolor{darkblue}{\textbf{\ipa{tɕɤ˧tɑ˧ tʰv̩˧-ɭɯ˧}}} \zh{这个牛轭} \textcolor{Sepia}{\selectlanguage{english}\mytextsc{n}+\mytextsc{dem}+\mytextsc{clf}} \textcolor{PineGreen}{\selectlanguage{french}\mytextsc{n}+\mytextsc{dem}+\mytextsc{clf}}  
 \zh{量词}: \textcolor{darkblue}{\textbf{\ipa{ɭɯ˧}}} 
\lhead{\firstmark}
\rhead{\botmark}

\subsection{\hspace{-0.5cm} {\Large \textcolor{darkblue}{\textbf{\ipa{tɕɤ˧\textasciitilde{}tɕɤ˧}}}}\hspace{0.5cm}[\kern2pt{\textcolor{darkblue}{\textbf{\ipa{tɕɤ˧tɕɤ˧}}}}\kern2pt]} \hypertarget{ts£7\string_M~ts£7\string_M1}{}
\markboth{\textcolor{darkblue}{\textbf{\ipa{tɕɤ˧\textasciitilde{}tɕɤ˧}}}}{}
\textcolor{teal}{\zh{助词}} \hspace{4pt} \zh{声调类:} M.
\zh{将将(汉语借词)、刚刚。} \textcolor{Sepia}{\selectlanguage{english}Just; exactly.} \textcolor{PineGreen}{\selectlanguage{french}Précisément, exactement (ex.: au moment précis où, juste au moment où).} \zh{当地汉语方言:}\zh{将将。} \zh{【借词】} \zh{将将}

\lhead{\firstmark}
\rhead{\botmark}

\subsection{\hspace{-0.5cm} {\Large \textcolor{darkblue}{\textbf{\ipa{tɕɤ˩}}}}\hspace{0.5cm}[\kern2pt{\textcolor{darkblue}{\textbf{\ipa{xxxx groupe tonal entier sans aucun ton}}}}\kern2pt]} \hypertarget{ts£7\string_B1}{}
\markboth{\textcolor{darkblue}{\textbf{\ipa{tɕɤ˩}}}}{}
\textcolor{teal}{\zh{动词}} \hspace{4pt} \zh{声调类:} .
\zh{打结、系上。} \textcolor{Sepia}{\selectlanguage{english}To bind together.} \textcolor{PineGreen}{\selectlanguage{french}Attacher (ex.: un joug sur une vache; des troncs…).}  \zh{【借词】} \zh{架?}
 ¶ \textcolor{darkblue}{\textbf{\ipa{ʁæ˧ɻ̍˥ | tʰi˧-tɕɤ˩}}} \zh{系上牛轭} \textcolor{Sepia}{\selectlanguage{english}to attach the yoke (to a buffalo)} \textcolor{PineGreen}{\selectlanguage{french}fixer (un joug sur un buffle)}  

\lhead{\firstmark}
\rhead{\botmark}

\subsection{\hspace{-0.5cm} {\Large \textcolor{darkblue}{\textbf{\ipa{tɕɤ˩ho˩tsɯ˥}}}}\hspace{0.5cm}[\kern2pt{\textcolor{darkblue}{\textbf{\ipa{tɕɤ˩ho˩tsɯ˥}}}}\kern2pt]} \hypertarget{ts£7\string_Bho\string_BtsM\string_T1}{}
\markboth{\textcolor{darkblue}{\textbf{\ipa{tɕɤ˩ho˩tsɯ˥}}}}{}
\textcolor{teal}{\zh{名词}} \hspace{4pt} \zh{声调类:} L+H\#.
\zh{骗子。} \textcolor{Sepia}{\selectlanguage{english}Swindler, cheat.} \textcolor{PineGreen}{\selectlanguage{french}Escroc.}  ¶ \textcolor{darkblue}{\textbf{\ipa{ʈʂʰɯ˧ | hĩ˧ ʈʂʰɯ˧-v̩˧ | tɕɤ˩ho˩tsɯ˥ ɲi˩.}}} \zh{这个人是骗子!} \textcolor{Sepia}{\selectlanguage{english}This man is a swindler!} \textcolor{PineGreen}{\selectlanguage{french}Cet homme, c'est un escroc!}  
 \zh{量词}: \textcolor{darkblue}{\textbf{\ipa{v̩˧}}} 
\lhead{\firstmark}
\rhead{\botmark}

\subsection{\hspace{-0.5cm} {\Large \textcolor{darkblue}{\textbf{\ipa{tɕɤ˧˥}}}}\hspace{0.5cm}[\kern2pt{\textcolor{darkblue}{\textbf{\ipa{tɕɤ˧˥}}}}\kern2pt]} \hypertarget{ts£7\string_M\string_T1}{}
\markboth{\textcolor{darkblue}{\textbf{\ipa{tɕɤ˧˥}}}}{}
\textcolor{teal}{\zh{动词}} \hspace{4pt} \zh{声调类:} MH.
\zh{煮。} \textcolor{Sepia}{\selectlanguage{english}To boil, to cook thoroughly; to cook in a pot.} \textcolor{PineGreen}{\selectlanguage{french}Bouillir; cuire en faisant bouillir; cuire dans une casserole.}  ¶ \textcolor{darkblue}{\textbf{\ipa{ʂe˧ tɕɤ˩}}} \zh{煮肉} \textcolor{Sepia}{\selectlanguage{english}to boil meat} \textcolor{PineGreen}{\selectlanguage{french}faire bouillir de la viande, faire cuire de la viande à l'eau}  
 ¶ \textcolor{darkblue}{\textbf{\ipa{bo˩-hɑ˧ tɕɤ˩}}} \zh{煮猪食} \textcolor{Sepia}{\selectlanguage{english}to boil pigswill, to cook pigswill} \textcolor{PineGreen}{\selectlanguage{french}faire bouillir la pâtée des cochons}  
 ¶ \textcolor{darkblue}{\textbf{\ipa{ho˧ tɕɤ˩}}} \zh{煮粥} \textcolor{Sepia}{\selectlanguage{english}to cook stew} \textcolor{PineGreen}{\selectlanguage{french}faire du ragoût}  
 ¶ \textcolor{darkblue}{\textbf{\ipa{dʑɯ˩ʁo˩˥, | mo˧-no˥, | mo˧ tɕɤ˥-hĩ˩ lɑ˩-ɲi˩-mæ˩! |}}} \zh{在山上,菌子,就是简单煮一下而已!(放在锅里,加油、加盐。用菌子自身的水分)} \textcolor{Sepia}{\selectlanguage{english}Up on the mountain, to cook mushrooms, (we) simply cook them in a pot! (This does not refer to boiling in the sense of 'cooking in hot water': the mushrooms are put in a pot; one adds grease and salt, and the mushrooms cook in their own juice.)} \textcolor{PineGreen}{\selectlanguage{french}(Quand on se trouve sur) la montagne, les champignons, on les fait simplement cuire dans une casserole! (Littéralement: “on se contente de les faire bouillir”.) (On mettait simplement les champignons dans une casserole, sans eau, avec du sel et de la graisse; les champignons cuisaient alors dans leur propre eau.)}  

\lhead{\firstmark}
\rhead{\botmark}

\subsection{\hspace{-0.5cm} {\Large \textcolor{darkblue}{\textbf{\ipa{tɕi˥}}}}\hspace{0.5cm}[\kern2pt{\textcolor{darkblue}{\textbf{\ipa{tɕi˥}}}}\kern2pt]} \hypertarget{ts£i\string_T1}{}
\markboth{\textcolor{darkblue}{\textbf{\ipa{tɕi˥}}}}{}
\textcolor{teal}{\zh{动词}} \hspace{4pt} \zh{声调类:} H.
\zh{抖、抖动,摇动。} \textcolor{Sepia}{\selectlanguage{english}To shake (e.g. clothes after washing; to shake one's head).} \textcolor{PineGreen}{\selectlanguage{french}Secouer (ex.: pour défroisser des vêtements après lavage; aussi: secouer la tête).}  ¶ \textcolor{darkblue}{\textbf{\ipa{le˧-tɕi˧\textasciitilde{}tɕi˧-ze˩}}} \zh{\mytextsc{accomp} \string_ \mytextsc{pfv}} \textcolor{Sepia}{\selectlanguage{english}\mytextsc{accomp} \string_ \mytextsc{pfv}} \textcolor{PineGreen}{\selectlanguage{french}\mytextsc{accomp} \string_ \mytextsc{pfv}}  
 ¶ \textcolor{darkblue}{\textbf{\ipa{tʰi˧-tɕi˧\textasciitilde{}tɕi˧+ze˩}}} \zh{\mytextsc{dur} \string_ \mytextsc{pfv}} \textcolor{Sepia}{\selectlanguage{english}\mytextsc{dur} \string_ \mytextsc{pfv}} \textcolor{PineGreen}{\selectlanguage{french}\mytextsc{dur} \string_ \mytextsc{pfv}}  
 ¶ \textcolor{darkblue}{\textbf{\ipa{ʁo˧qʰwɤ˩ tɕi˩\textasciitilde{}tɕi˩}}} \zh{摇头} \textcolor{Sepia}{\selectlanguage{english}to shake one's head} \textcolor{PineGreen}{\selectlanguage{french}agiter la tête, secouer la tête}  
 ¶ \textcolor{darkblue}{\textbf{\ipa{ɖɯ˧-tɕi˧\textasciitilde{}tɕi˧-ɻ̍˥}}} \zh{摇一摇} \textcolor{Sepia}{\selectlanguage{english}\mytextsc{demilitative} \mytextsc{red} \mytextsc{inceptive}} \textcolor{PineGreen}{\selectlanguage{french}\mytextsc{délimitatif} \string_ \mytextsc{red} \mytextsc{inchoatif}}  

\lhead{\firstmark}
\rhead{\botmark}

\subsection{\hspace{-0.5cm} {\Large \textcolor{darkblue}{\textbf{\ipa{tɕi˧}}} \textsubscript{1}}\hspace{0.5cm}[\kern2pt{\textcolor{darkblue}{\textbf{\ipa{tɕi˥}}}}\kern2pt]} \hypertarget{ts£i\string_M1}{}
\markboth{\textcolor{darkblue}{\textbf{\ipa{tɕi˧}}} \textsubscript{1}}{}
\textcolor{teal}{\zh{形容词}} \hspace{4pt} \zh{声调类:} M.
\ding{202} \zh{酸。} \textcolor{Sepia}{\selectlanguage{english}Acid.} \textcolor{PineGreen}{\selectlanguage{french}Acide.}  ¶ \textcolor{darkblue}{\textbf{\ipa{tɕʰɯ˩-hĩ˩˥}}}  
 ¶ \textcolor{darkblue}{\textbf{\ipa{[M18] tɕi˧-hĩ˧ pʰi˩}}} \zh{吐酸水} \textcolor{Sepia}{\selectlanguage{english}to have acid reflux} \textcolor{PineGreen}{\selectlanguage{french}avoir des remontées acides}  
\ding{203} \zh{(通过发酵的)酸。} \textcolor{Sepia}{\selectlanguage{english}Sour, fermented.} \textcolor{PineGreen}{\selectlanguage{french}Fermenté.} 
\lhead{\firstmark}
\rhead{\botmark}

\subsection{\hspace{-0.5cm} {\Large \textcolor{darkblue}{\textbf{\ipa{tɕi˧}}} \textsubscript{2}}\hspace{0.5cm}[\kern2pt{\textcolor{darkblue}{\textbf{\ipa{tɕi˥}}}}\kern2pt]} \hypertarget{ts£i\string_M2}{}
\markboth{\textcolor{darkblue}{\textbf{\ipa{tɕi˧}}} \textsubscript{2}}{}
\textcolor{teal}{\zh{名词}} \hspace{4pt} \zh{声调类:} M.
\zh{圈套。} \textcolor{Sepia}{\selectlanguage{english}Snare, trap, trick.} \textcolor{PineGreen}{\selectlanguage{french}Piège.}  ¶ \textcolor{darkblue}{\textbf{\ipa{tɕi˧ kʰɯ˧˥}}} \zh{设下圈套} \textcolor{Sepia}{\selectlanguage{english}to set a trap} \textcolor{PineGreen}{\selectlanguage{french}poser un piège}  
 \zh{量词}: \textcolor{darkblue}{\textbf{\ipa{ɭɯ˧}}} 
\lhead{\firstmark}
\rhead{\botmark}

\subsection{\hspace{-0.5cm} {\Large \textcolor{darkblue}{\textbf{\ipa{tɕi˧\textsubscript{a}}}}}\hspace{0.5cm}[\kern2pt{\textcolor{darkblue}{\textbf{\ipa{tɕi˥}}}}\kern2pt]} \hypertarget{ts£i\string_Ma1}{}
\markboth{\textcolor{darkblue}{\textbf{\ipa{tɕi˧\textsubscript{a}}}}}{}
\textcolor{teal}{\zh{量词}} \hspace{4pt} \zh{声调类:} M\textsubscript{a}.
\zh{量词:一些。} \textcolor{Sepia}{\selectlanguage{english}Some, a few.} \textcolor{PineGreen}{\selectlanguage{french}Quelques-uns, certains, une partie.}  ¶ \textcolor{darkblue}{\textbf{\ipa{ɖɯ˧-tɕi˧}}} \zh{一些} \textcolor{Sepia}{\selectlanguage{english}some, a few} \textcolor{PineGreen}{\selectlanguage{french}quelques-uns, certains}  
 ¶ \textcolor{darkblue}{\textbf{\ipa{ʈʂʰɯ˧-tɕi˩}}} \zh{这些} \textcolor{Sepia}{\selectlanguage{english}these few} \textcolor{PineGreen}{\selectlanguage{french}ceux-ci}  

\lhead{\firstmark}
\rhead{\botmark}

\subsection{\hspace{-0.5cm} {\Large \textcolor{darkblue}{\textbf{\ipa{tɕi˧do˩}}}}\hspace{0.5cm}[\kern2pt{\textcolor{darkblue}{\textbf{\ipa{tɕi˧do˩}}}}\kern2pt]} \hypertarget{ts£i\string_Mdo\string_B1}{}
\markboth{\textcolor{darkblue}{\textbf{\ipa{tɕi˧do˩}}}}{}
\textcolor{teal}{\zh{名词}} \hspace{4pt} \zh{声调类:} L\#.
\zh{橘子。} \textcolor{Sepia}{\selectlanguage{english}Tangerine.} \textcolor{PineGreen}{\selectlanguage{french}Mandarine.} \zh{当地汉语方言:}\zh{黄果。} \zh{量词}: \textcolor{darkblue}{\textbf{\ipa{ɭɯ˧}}} 
\lhead{\firstmark}
\rhead{\botmark}

\subsection{\hspace{-0.5cm} {\Large \textcolor{darkblue}{\textbf{\ipa{tɕi˧-dʑɯ˩}}}}\hspace{0.5cm}[\kern2pt{\textcolor{darkblue}{\textbf{\ipa{xxxx non-correspondance entre le nombre de morphèmes et le nombre de tons de morphèmes}}}}\kern2pt]} \hypertarget{ts£i\string_M-dz£M\string_B1}{}
\markboth{\textcolor{darkblue}{\textbf{\ipa{tɕi˧-dʑɯ˩}}}}{}
\textcolor{teal}{\zh{名词}} \hspace{4pt} \zh{声调类:} L\#.
\ding{202} \zh{用梅子等野生果子做出来的一种药品(酸水),食物中毒的情况下给病人和这种酸水让他呕吐。} \textcolor{Sepia}{\selectlanguage{english}Acid potion: a preparation from sour plums or wild berries, used to make people vomit when they had food poisoning (e.g. from eating poisonous mushrooms).} \textcolor{PineGreen}{\selectlanguage{french}Potion acide: une préparation à base de prunelles acides ou baies sauvages, utilisée pour faire vomir les personnes victimes d'un empoisonnement alimentaire (par exemple par des champignons vénéneux).} \ding{203} \zh{醋。} \textcolor{Sepia}{\selectlanguage{english}Vinegar.} \textcolor{PineGreen}{\selectlanguage{french}Vinaigre.} 
\lhead{\firstmark}
\rhead{\botmark}

\subsection{\hspace{-0.5cm} {\Large \textcolor{darkblue}{\textbf{\ipa{tɕi˧kwɤ˧}}}}\hspace{0.5cm}[\kern2pt{\textcolor{darkblue}{\textbf{\ipa{tɕi˧kwɤ˩}}}}\kern2pt]} \hypertarget{ts£i\string_Mkw7\string_M1}{}
\markboth{\textcolor{darkblue}{\textbf{\ipa{tɕi˧kwɤ˧}}}}{}
\textcolor{teal}{\zh{名词}} \hspace{4pt} \zh{声调类:} M.
\zh{瓜。} \textcolor{Sepia}{\selectlanguage{english}Melon, gourd.} \textcolor{PineGreen}{\selectlanguage{french}Courge (inclut les courgettes).}  ¶ \textcolor{darkblue}{\textbf{\ipa{tɕi˧kwɤ˧ bv̩˧-ɻ̍˧ (+ɲi˩)}}} \zh{小瓜} \textcolor{Sepia}{\selectlanguage{english}small melon} \textcolor{PineGreen}{\selectlanguage{french}petite courge}  
 ¶ \textcolor{darkblue}{\textbf{\ipa{tɕi˧kwɤ˧ kwɤ˧mo˩}}} \zh{大瓜} \textcolor{Sepia}{\selectlanguage{english}large melon} \textcolor{PineGreen}{\selectlanguage{french}grosse courge}  
 \zh{量词}: \textcolor{darkblue}{\textbf{\ipa{ɭɯ˧}}} 
\lhead{\firstmark}
\rhead{\botmark}

\subsection{\hspace{-0.5cm} {\Large \textcolor{darkblue}{\textbf{\ipa{tɕi˧sɯ˧pɤ˧}}}}\hspace{0.5cm}[\kern2pt{\textcolor{darkblue}{\textbf{\ipa{tɕi˩sɯ˩pɤ˥}}}}\kern2pt]} \hypertarget{ts£i\string_MsM\string_Mp7\string_M1}{}
\markboth{\textcolor{darkblue}{\textbf{\ipa{tɕi˧sɯ˧pɤ˧}}}}{}
\textcolor{teal}{\zh{名词}} \hspace{4pt} \zh{声调类:} M.
\zh{牦牛奶酪。} \textcolor{Sepia}{\selectlanguage{english}Cheese made of yak milk. First, the milk is creamed, then boiled again, with an additive to make it curdle; finally, the preparation is left to dry and harden. It is used in cooking (some of it can be added to gruel), and also as a treatment for diaorrhea. It can keep for a long time.} \textcolor{PineGreen}{\selectlanguage{french}Fromage au lait de yak. On commençait par écrémer le lait, puis on faisait bouillir, avec un additif pour le faire cailler; enfin la préparation se solidifiait. Cette préparation était utilisée dans l'alimentation: on en mettait dans les bouillies de céréales. Elle pouvait se conserver. Ce fromage, acide et dur, était recommandé aux personnes ayant des soucis digestifs (comme remède à la diarrhée), et aux personnes âgées.}  ¶ \textcolor{darkblue}{\textbf{\ipa{mv̩˧ɭɯ˩-pʰɤ˩bɤ˩, | tɕi˧sɯ˧pɤ˧!}}} \zh{木里的礼物:牦牛的奶酪! / 牦牛奶酪,是木里的特产!} \textcolor{Sepia}{\selectlanguage{english}The gift from Muli is yak cheese! / The gift that people usually bring back from their trips to Muli is yak cheese! / Yak cheese is a specialty of Muli! (Yak cheese used to be one of the delicacies that young men offered to young ladies when coming back from caravan journeys.)} \textcolor{PineGreen}{\selectlanguage{french}Le cadeau (qu'on ramène de Muli), c'est le fromage de yak! / La spécialité de Muli, c'est le fromage de yak! (Autrefois, c'était un des cadeaux que les jeunes gens offraient aux jeunes filles au retour de leurs voyages.)}  
 ¶ \textcolor{darkblue}{\textbf{\ipa{mv̩˧ɭɯ˩ pʰɤ˩bɤ˩, | tɕi˧sɯ˧pɤ˧! | ə˧ɖo˧ ʁo˧ ɖʐɯ˥\textasciitilde{}ɖʐɯ˩ ʝi˩-ze˩!}}} \zh{(从)木里(带回来)的礼物,就是牦牛奶酪!亲爱的(=收礼物的那个人),会摇头的!(吃、喝的时候会摇头,是因为牦牛奶奶酪比较酸)} \textcolor{Sepia}{\selectlanguage{english}The gift from Muli is yak cheese! (My) beloved will shake her head (when tasting the delightfully acid cheese)! (Words from a song that used to be sung when travelling, imagining the return to Yongning.)} \textcolor{PineGreen}{\selectlanguage{french}Le cadeau (qu'on ramène de Muli), c'est le fromage de yak! Ma bien-aimée va secouer la tête (lorsqu'elle goûtera à ce fromage, très acide)! (Paroles d'une chanson qu'on chantait en chemin, en imaginant le retour.)}  
 ¶ \textcolor{darkblue}{\textbf{\ipa{tɕi˧sɯ˧pɤ˧, | ɖɯ˧-tɑ˧˥ | gv̩˧-mɤ˧-kv̩˥! | ʝi˧-kʰv̩˥-lɑ˩ gv̩˩-kv̩˩!}}} \zh{不是每个人都会做牦牛奶酪!只有少数(人)才会做!} \textcolor{Sepia}{\selectlanguage{english}Not everyone knew how to make yak cheese! Only a few had this know-how!} \textcolor{PineGreen}{\selectlanguage{french}Ce n'est pas tout le monde qui savait faire du fromage de yak! Il n'y a que certaines (personnes/familles) qui savaient le faire!}  
 ¶ \textcolor{darkblue}{\textbf{\ipa{tɕi˧sɯ˧pɤ˧-dʑɯ˩}}} \zh{一种饮料:将牦牛奶酪溶化在水里} \textcolor{Sepia}{\selectlanguage{english}water in which one has diluted some yak cheese; it has medicinal properties} \textcolor{PineGreen}{\selectlanguage{french}eau dans laquelle on a dilué du fromage de yak; elle a des propriétés médicinales}  
 ¶ \textcolor{darkblue}{\textbf{\ipa{tɕi˧sɯ˧pɤ˧ ʈʰɯ˩}}} \zh{喝溶化在水里的牦牛奶酪(直译:喝牦牛奶酪)} \textcolor{Sepia}{\selectlanguage{english}to drink water in which one has diluted some yak cheese; literally: 'to drink yak cheese'} \textcolor{PineGreen}{\selectlanguage{french}boire de l'eau dans laquelle on a dilué du fromage de yak; littéralement: 'boire du fromage de yak'}  

\lhead{\firstmark}
\rhead{\botmark}

\subsection{\hspace{-0.5cm} {\Large \textcolor{darkblue}{\textbf{\ipa{tɕi˧tɕi˧ | læ˩sæ˧-dzi˩}}}}\hspace{0.5cm}[\kern2pt{\textcolor{darkblue}{\textbf{\ipa{xxxx non-correspondance entre le nombre de groupes tonals et le nombre de tons}}}}\kern2pt]} \hypertarget{ts£i\string_Mts£i\string_M | l\{\string_Bs\{\string_M-dzi\string_B1}{}
\markboth{\textcolor{darkblue}{\textbf{\ipa{tɕi˧tɕi˧ | læ˩sæ˧-dzi˩}}}}{}
\textcolor{teal}{\zh{名词}} \hspace{4pt} \zh{声调类:} M-LH-L.
\zh{一种树,木质很硬。} \textcolor{Sepia}{\selectlanguage{english}A type of hardwood (not identified yet).} \textcolor{PineGreen}{\selectlanguage{french}Un arbre au bois très dur.} 
\lhead{\firstmark}
\rhead{\botmark}

\subsection{\hspace{-0.5cm} {\Large \textcolor{darkblue}{\textbf{\ipa{tɕi˩\textsubscript{a}}}}}\hspace{0.5cm}[\kern2pt{\textcolor{darkblue}{\textbf{\ipa{tɕi˩˥}}}}\kern2pt]} \hypertarget{ts£i\string_Ba1}{}
\markboth{\textcolor{darkblue}{\textbf{\ipa{tɕi˩\textsubscript{a}}}}}{}
\textcolor{teal}{\zh{形容词}} \hspace{4pt} \zh{声调类:} L\textsubscript{a}.
\zh{矮,低,小。} \textcolor{Sepia}{\selectlanguage{english}Small; short (not tall).} \textcolor{PineGreen}{\selectlanguage{french}Petit.}  ¶ \textcolor{darkblue}{\textbf{\ipa{tɕi˩-hĩ˩˥}}} \zh{矮的} \textcolor{Sepia}{\selectlanguage{english}\mytextsc{nmlz}} \textcolor{PineGreen}{\selectlanguage{french}(qui est) petit}  
 ¶ \textcolor{darkblue}{\textbf{\ipa{gv̩˧mi˧ tɕi˩}}} \zh{矮} \textcolor{Sepia}{\selectlanguage{english}short (not tall)} \textcolor{PineGreen}{\selectlanguage{french}de petite taille}  

\lhead{\firstmark}
\rhead{\botmark}

\subsection{\hspace{-0.5cm} {\Large \textcolor{darkblue}{\textbf{\ipa{tɕi˩nv̩˧˥}}}}\hspace{0.5cm}[\kern2pt{\textcolor{darkblue}{\textbf{\ipa{tɕi˧nv̩˥}}}}\kern2pt]} \hypertarget{ts£i\string_Bnv\string_=\string_M\string_T1}{}
\markboth{\textcolor{darkblue}{\textbf{\ipa{tɕi˩nv̩˧˥}}}}{}
\textcolor{teal}{\zh{名词}} \hspace{4pt} \zh{声调类:} LM+MH\#.
\zh{马鞍下面的毯子。} \textcolor{Sepia}{\selectlanguage{english}Saddle mat.} \textcolor{PineGreen}{\selectlanguage{french}Tapis de selle.}  ¶ \textcolor{darkblue}{\textbf{\ipa{ʐwæ˧-tɕi˥nv̩˩}}} \zh{马鞍毯子} \textcolor{Sepia}{\selectlanguage{english}horse saddle mat} \textcolor{PineGreen}{\selectlanguage{french}tapis de selle de cheval}  
 \zh{量词}: \textcolor{darkblue}{\textbf{\ipa{pɤ˩}}} 
\lhead{\firstmark}
\rhead{\botmark}

\subsection{\hspace{-0.5cm} {\Large \textcolor{darkblue}{\textbf{\ipa{tɕi˩qɑ˥}}}}\hspace{0.5cm}[\kern2pt{\textcolor{darkblue}{\textbf{\ipa{tɕi˩qɑ˧˥}}}}\kern2pt]} \hypertarget{ts£i\string_BqA\string_T1}{}
\markboth{\textcolor{darkblue}{\textbf{\ipa{tɕi˩qɑ˥}}}}{}
\textcolor{teal}{\zh{名词}} \hspace{4pt} \zh{声调类:} LH.
\zh{毯子。} \textcolor{Sepia}{\selectlanguage{english}Carpet.} \textcolor{PineGreen}{\selectlanguage{french}Tapis.}  \zh{量词}: \textcolor{darkblue}{\textbf{\ipa{ɭɯ˧}}} 
\lhead{\firstmark}
\rhead{\botmark}

\subsection{\hspace{-0.5cm} {\Large \textcolor{darkblue}{\textbf{\ipa{tɕi˩˥}}}}\hspace{0.5cm}[\kern2pt{\textcolor{darkblue}{\textbf{\ipa{tɕi˩˥}}}}\kern2pt]} \hypertarget{ts£i\string_B\string_T1}{}
\markboth{\textcolor{darkblue}{\textbf{\ipa{tɕi˩˥}}}}{}
\textcolor{teal}{\zh{名词}} \hspace{4pt} \zh{声调类:} LH.
\zh{马鞍。} \textcolor{Sepia}{\selectlanguage{english}Saddle.} \textcolor{PineGreen}{\selectlanguage{french}Selle.}  ¶ \textcolor{darkblue}{\textbf{\ipa{ʐwæ˧-tɕi˥}}} \zh{马鞍} \textcolor{Sepia}{\selectlanguage{english}horse saddle} \textcolor{PineGreen}{\selectlanguage{french}selle de cheval}  
 \zh{量词}: \textcolor{darkblue}{\textbf{\ipa{pɤ˩}}} 
\lhead{\firstmark}
\rhead{\botmark}

\subsection{\hspace{-0.5cm} {\Large \textcolor{darkblue}{\textbf{\ipa{tɕo˥}}}}\hspace{0.5cm}[\kern2pt{\textcolor{darkblue}{\textbf{\ipa{tɕo˩˥}}}}\kern2pt]} \hypertarget{ts£o\string_T1}{}
\markboth{\textcolor{darkblue}{\textbf{\ipa{tɕo˥}}}}{}
\textcolor{teal}{\zh{名词}} \hspace{4pt} \zh{声调类:} H.
\zh{方向。} \textcolor{Sepia}{\selectlanguage{english}Direction.} \textcolor{PineGreen}{\selectlanguage{french}Sens, direction.}  ¶ \textcolor{darkblue}{\textbf{\ipa{ʈʂʰɯ˧-tɕo˧}}} \zh{这个方向,向这里} \textcolor{Sepia}{\selectlanguage{english}this way} \textcolor{PineGreen}{\selectlanguage{french}dans cette direction-ci}  
 ¶ \textcolor{darkblue}{\textbf{\ipa{ɖɯ˧-tɕo˥}}} \zh{一边} \textcolor{Sepia}{\selectlanguage{english}one side, in one direction} \textcolor{PineGreen}{\selectlanguage{french}d'un côté, dans une direction}  
 ¶ \textcolor{darkblue}{\textbf{\ipa{gɤ˩-tɕo˧}}} \zh{向上,往上} \textcolor{Sepia}{\selectlanguage{english}upward, towards the top} \textcolor{PineGreen}{\selectlanguage{french}vers le haut}  
 ¶ \textcolor{darkblue}{\textbf{\ipa{dv̩˩tɕo˧}}} \zh{那边} \textcolor{Sepia}{\selectlanguage{english}that way} \textcolor{PineGreen}{\selectlanguage{french}dans cette direction-là}  

\lhead{\firstmark}
\rhead{\botmark}

\subsection{\hspace{-0.5cm} {\Large \textcolor{darkblue}{\textbf{\ipa{tɕo˩ɕjo˧}}}}\hspace{0.5cm}[\kern2pt{\textcolor{darkblue}{\textbf{\ipa{tɕo˩ɕjo˥}}}}\kern2pt]} \hypertarget{ts£o\string_Bs£jo\string_M1}{}
\markboth{\textcolor{darkblue}{\textbf{\ipa{tɕo˩ɕjo˧}}}}{}
\textcolor{teal}{\zh{名词}} \hspace{4pt} \zh{声调类:} LM.
\zh{口哨。} \textcolor{Sepia}{\selectlanguage{english}Whistle, whistling noise.} \textcolor{PineGreen}{\selectlanguage{french}Sifflement.}  ¶ \textcolor{darkblue}{\textbf{\ipa{tɕo˩ɕjo˧ | ɖɯ˧-ɖʐo˩ kʰɯ˩}}} \zh{吹口哨、吹一声口哨} \textcolor{Sepia}{\selectlanguage{english}to whistle a little, to whistle a few notes} \textcolor{PineGreen}{\selectlanguage{french}siffler un air, siffler un coup}  

\lhead{\firstmark}
\rhead{\botmark}

\subsection{\hspace{-0.5cm} {\Large \textcolor{darkblue}{\textbf{\ipa{tɕo˩mv̩˧}}}}\hspace{0.5cm}[\kern2pt{\textcolor{darkblue}{\textbf{\ipa{tɕo˩mv̩˥}}}}\kern2pt]} \hypertarget{ts£o\string_Bmv\string_=\string_M1}{}
\markboth{\textcolor{darkblue}{\textbf{\ipa{tɕo˩mv̩˧}}}}{}
\textcolor{teal}{\zh{名词}} \hspace{4pt} \zh{声调类:} LM.
\zh{舅妈(舅:汉语借词,妈:摩梭话“女人”)。} \textcolor{Sepia}{\selectlanguage{english}Wife of maternal uncle. The word consists of a Chinese borrowing, \zh{舅} 'maternal uncle', to which is added the Na word for 'woman'.} \textcolor{PineGreen}{\selectlanguage{french}Femme de l'oncle maternel; constitué d'un emprunt chinois, \zh{舅} 'oncle maternel', et d'un mot na: 'femme'.}  \zh{【借词】} \zh{舅}
 \zh{量词}: \textcolor{darkblue}{\textbf{\ipa{v̩˧}}} 
\lhead{\firstmark}
\rhead{\botmark}

\subsection{\hspace{-0.5cm} {\Large \textcolor{darkblue}{\textbf{\ipa{tɕɯ˥}}}}\hspace{0.5cm}[\kern2pt{\textcolor{darkblue}{\textbf{\ipa{tɕɯ˥}}}}\kern2pt]} \hypertarget{ts£M\string_T1}{}
\markboth{\textcolor{darkblue}{\textbf{\ipa{tɕɯ˥}}}}{}
\textcolor{teal}{\zh{动词}} \hspace{4pt} \zh{声调类:} H.
\ding{202} \zh{放置。} \textcolor{Sepia}{\selectlanguage{english}To put, to lay up.} \textcolor{PineGreen}{\selectlanguage{french}Poser, ranger, mettre, placer.}  ¶ \textcolor{darkblue}{\textbf{\ipa{tʰi˧-tɕɯ˥}}} \zh{\mytextsc{dur}} \textcolor{Sepia}{\selectlanguage{english}\mytextsc{dur}} \textcolor{PineGreen}{\selectlanguage{french}\mytextsc{dur}}  
 ¶ \textcolor{darkblue}{\textbf{\ipa{[F5] ɖɯ˩hĩ˧ | ɖɯ˩˧ | tʰi˧-tɕɯ˥, | tɕi˩hĩ˧ | tɕi˩˧ | tʰi˧-tɕɯ˥}}} \zh{大小归类} \textcolor{Sepia}{\selectlanguage{english}to put big ones with big ones, small ones with small ones} \textcolor{PineGreen}{\selectlanguage{french}mettre les grands avec les grands, les petits avec les petits}  
\ding{203} \zh{决定、定下来。} \textcolor{Sepia}{\selectlanguage{english}To settle, to decide.} \textcolor{PineGreen}{\selectlanguage{french}Fixer, décider (ex.: les puissances suprêmes fixent la durée de la vie humaine).}  ¶ \textcolor{darkblue}{\textbf{\ipa{le˧-ʐwɤ˩ | tʰi˧-tɕɯ˥}}} \zh{说好、决定} \textcolor{Sepia}{\selectlanguage{english}to settle} \textcolor{PineGreen}{\selectlanguage{french}fixer; arrêter; décider que}  
 ¶ \textcolor{darkblue}{\textbf{\ipa{le˧-ʐwɤ˩ | tʰi˧-tɕɯ˧-ɲi˥-tsɯ˩!}}} \zh{说好了! / 决定好了!} \textcolor{Sepia}{\selectlanguage{english}It's settled!} \textcolor{PineGreen}{\selectlanguage{french}C'est fixé/c'est décidé/c'est arrêté!}  

\lhead{\firstmark}
\rhead{\botmark}

\subsection{\hspace{-0.5cm} {\Large \textcolor{darkblue}{\textbf{\ipa{tɕɯ˧}}}}\hspace{0.5cm}[\kern2pt{\textcolor{darkblue}{\textbf{\ipa{tɕɯ˥}}}}\kern2pt]} \hypertarget{ts£M\string_M1}{}
\markboth{\textcolor{darkblue}{\textbf{\ipa{tɕɯ˧}}}}{}
\textcolor{teal}{\zh{名词}} \hspace{4pt} \zh{声调类:} M.
\zh{云。} \textcolor{Sepia}{\selectlanguage{english}Cloud.} \textcolor{PineGreen}{\selectlanguage{french}Nuage.}  ¶ \textcolor{darkblue}{\textbf{\ipa{mv̩˧tɕɯ˥}}} \zh{天上多云} \textcolor{Sepia}{\selectlanguage{english}the weather is cloudy} \textcolor{PineGreen}{\selectlanguage{french}il y a des nuages, le temps est nuageux}  
 ¶ \textcolor{darkblue}{\textbf{\ipa{mv̩˧ʁo˥, | tɕɯ˧!}}} \zh{天上有云!} \textcolor{Sepia}{\selectlanguage{english}The sky is cloudy!} \textcolor{PineGreen}{\selectlanguage{french}le ciel est nuageux!}  
 ¶ \textcolor{darkblue}{\textbf{\ipa{mv̩˧ʁo˥ tɕɯ˩ pʰv̩˩ |}}} \zh{天上有云!} \textcolor{Sepia}{\selectlanguage{english}The sky is cloudy!} \textcolor{PineGreen}{\selectlanguage{french}le ciel est nuageux!}  
 ¶ \textcolor{darkblue}{\textbf{\ipa{tɕɯ˧pʰv̩˩; tɕɯ˧ | pʰv̩˩tɕæ˩˥ | -gv̩˩}}} \zh{白云、白色的云} \textcolor{Sepia}{\selectlanguage{english}white cloud} \textcolor{PineGreen}{\selectlanguage{french}nuage blanc}  
 ¶ \textcolor{darkblue}{\textbf{\ipa{mv̩˧nɑ˥-tɕɯ˩nɑ˩-ɻ̍˩!}}} \zh{天很黑,有很多乌云} \textcolor{Sepia}{\selectlanguage{english}the sky is dark / the sky is very cloudy} \textcolor{PineGreen}{\selectlanguage{french}il fait sombre/ le ciel est très nuageux!}  
 \zh{量词}: \textcolor{darkblue}{\textbf{\ipa{kʰwɤ˥}}} 
\lhead{\firstmark}
\rhead{\botmark}

\subsection{\hspace{-0.5cm} {\Large \textcolor{darkblue}{\textbf{\ipa{tɕɯ˧\textsubscript{b}}}}}\hspace{0.5cm}[\kern2pt{\textcolor{darkblue}{\textbf{\ipa{tɕɯ˥}}}}\kern2pt]} \hypertarget{ts£M\string_Mb1}{}
\markboth{\textcolor{darkblue}{\textbf{\ipa{tɕɯ˧\textsubscript{b}}}}}{}
\textcolor{teal}{\zh{动词}} \hspace{4pt} \zh{声调类:} M\textsubscript{b}.
\zh{摇晃。} \textcolor{Sepia}{\selectlanguage{english}To shake.} \textcolor{PineGreen}{\selectlanguage{french}Secouer (monosyllabe).}  ¶ \textcolor{darkblue}{\textbf{\ipa{tso˧\textasciitilde{}tso˧ tɕɯ˧}}} \zh{摇东西} \textcolor{Sepia}{\selectlanguage{english}to shake things} \textcolor{PineGreen}{\selectlanguage{french}secouer des choses}  

\lhead{\firstmark}
\rhead{\botmark}

\subsection{\hspace{-0.5cm} {\Large \textcolor{darkblue}{\textbf{\ipa{tɕɯ˧ɭɯ˧}}}}\hspace{0.5cm}[\kern2pt{\textcolor{darkblue}{\textbf{\ipa{tɕɯ˩ɭɯ˩˥}}}}\kern2pt]} \hypertarget{ts£M\string_Ml\string_RM\string_M1}{}
\markboth{\textcolor{darkblue}{\textbf{\ipa{tɕɯ˧ɭɯ˧}}}}{}
\textcolor{teal}{\zh{动词}} \hspace{4pt} \zh{声调类:} M.
\zh{缠绕。} \textcolor{Sepia}{\selectlanguage{english}To roll, to spool, to reel.} \textcolor{PineGreen}{\selectlanguage{french}Enrouler, embobiner.}  ¶ \textcolor{darkblue}{\textbf{\ipa{njɤ˧-ɳɯ˧ | tɕɯ˧ɭɯ˧-bi˧!}}} \zh{让我来缠吧!} \textcolor{Sepia}{\selectlanguage{english}Let me reel! / Let me do the reeling!} \textcolor{PineGreen}{\selectlanguage{french}Je me charge d'enrouler! / C'est moi qui vais enrouler!}  

\lhead{\firstmark}
\rhead{\botmark}

\subsection{\hspace{-0.5cm} {\Large \textcolor{darkblue}{\textbf{\ipa{tɕɯ˧mi˥\$}}}}\hspace{0.5cm}[\kern2pt{\textcolor{darkblue}{\textbf{\ipa{tɕɯ˩mi˥}}}}\kern2pt]} \hypertarget{ts£M\string_Mmi\string_T\$1}{}
\markboth{\textcolor{darkblue}{\textbf{\ipa{tɕɯ˧mi˥\$}}}}{}
\textcolor{teal}{\zh{名词}} \hspace{4pt} \zh{声调类:} H\$.
\zh{大称。} \textcolor{Sepia}{\selectlanguage{english}Large scale.} \textcolor{PineGreen}{\selectlanguage{french}Grande balance.} 
\lhead{\firstmark}
\rhead{\botmark}

\subsection{\hspace{-0.5cm} {\Large \textcolor{darkblue}{\textbf{\ipa{tɕɯ˧pv̩˧}}}}\hspace{0.5cm}[\kern2pt{\textcolor{darkblue}{\textbf{\ipa{tɕɯ˩pv̩˥}}}}\kern2pt]} \hypertarget{ts£M\string_Mpv\string_=\string_M1}{}
\markboth{\textcolor{darkblue}{\textbf{\ipa{tɕɯ˧pv̩˧}}}}{}
\textcolor{teal}{\zh{形容词}} \hspace{4pt} \zh{声调类:} .
\zh{轻松快乐、舒畅。} \textcolor{Sepia}{\selectlanguage{english}At ease.} \textcolor{PineGreen}{\selectlanguage{french}À l'aise, peinard.}  ¶ \textcolor{darkblue}{\textbf{\ipa{ʈʂʰɯ˧qo˧ | tɕɯ˧pv̩˧-ʂe˧\textasciitilde{}ʂe˧ | ɖɯ˧-dzi˩-zo˩-ho˩!}}} \zh{在这边舒畅地坐一会吧!} \textcolor{Sepia}{\selectlanguage{english}Have a seat here, happy and relaxed!} \textcolor{PineGreen}{\selectlanguage{french}assieds-toi ici, bien peinard!}  
 ¶ \textcolor{darkblue}{\textbf{\ipa{ʈʂʰɯ˧qo˧ | tɕɯ˧pv̩˧-ʂe˧\textasciitilde{}ʂe˧-zo˥ | ɖɯ˧-dzi˩-bi˩-ɻ̍˩!}}} \zh{在这边舒畅地坐一会吧!} \textcolor{Sepia}{\selectlanguage{english}Let's have a seat here, happy and relaxed!} \textcolor{PineGreen}{\selectlanguage{french}asseyons-nous donc ici, bien peinards!}  

\lhead{\firstmark}
\rhead{\botmark}

\subsection{\hspace{-0.5cm} {\Large \textcolor{darkblue}{\textbf{\ipa{tɕɯ˧sɯ˧˥}}}}\hspace{0.5cm}[\kern2pt{\textcolor{darkblue}{\textbf{\ipa{tɕɯ˧sɯ˧˥}}}}\kern2pt]} \hypertarget{ts£M\string_MsM\string_M\string_T1}{}
\markboth{\textcolor{darkblue}{\textbf{\ipa{tɕɯ˧sɯ˧˥}}}}{}
\textcolor{teal}{\zh{名词}} \hspace{4pt} \zh{声调类:} MH\#.
\zh{雾。} \textcolor{Sepia}{\selectlanguage{english}Mist, fog.} \textcolor{PineGreen}{\selectlanguage{french}Brume.}  ¶ \textcolor{darkblue}{\textbf{\ipa{tɕɯ˧sɯ˧mv̩˥}}} \zh{有雾} \textcolor{Sepia}{\selectlanguage{english}there is some fog, there is some mist} \textcolor{PineGreen}{\selectlanguage{french}il y a de la brume}  
 \zh{量词}: \textcolor{darkblue}{\textbf{\ipa{ti˧˥}}} 
\lhead{\firstmark}
\rhead{\botmark}

\subsection{\hspace{-0.5cm} {\Large \textcolor{darkblue}{\textbf{\ipa{tɕɯ˧wɤ˧}}}}\hspace{0.5cm}[\kern2pt{\textcolor{darkblue}{\textbf{\ipa{tɕɯ˧wɤ˧}}}}\kern2pt]} \hypertarget{ts£M\string_Mw7\string_M1}{}
\markboth{\textcolor{darkblue}{\textbf{\ipa{tɕɯ˧wɤ˧}}}}{}
\textcolor{teal}{\zh{动词}} \hspace{4pt} \zh{声调类:} M.
\zh{转生、转世。} \textcolor{Sepia}{\selectlanguage{english}To reincarnate.} \textcolor{PineGreen}{\selectlanguage{french}Se réincarner.}  ¶ \textcolor{darkblue}{\textbf{\ipa{le˧-tɕɯ˧wɤ˧-ho˥!}}} \zh{他要转生了!} \textcolor{Sepia}{\selectlanguage{english}(She/he) is going to get reincarnated! (About a deceased person)} \textcolor{PineGreen}{\selectlanguage{french}(La défunte / le défunt) va se réincarner!}  

\lhead{\firstmark}
\rhead{\botmark}

\subsection{\hspace{-0.5cm} {\Large \textcolor{darkblue}{\textbf{\ipa{tɕɯ˧zo˥\$}}}}\hspace{0.5cm}[\kern2pt{\textcolor{darkblue}{\textbf{\ipa{tɕɯ˧zo˥}}}}\kern2pt]} \hypertarget{ts£M\string_Mzo\string_T\$1}{}
\markboth{\textcolor{darkblue}{\textbf{\ipa{tɕɯ˧zo˥\$}}}}{}
\textcolor{teal}{\zh{名词}} \hspace{4pt} \zh{声调类:} H\$.
\zh{小称。} \textcolor{Sepia}{\selectlanguage{english}Small scale.} \textcolor{PineGreen}{\selectlanguage{french}Petite balance.} 
\lhead{\firstmark}
\rhead{\botmark}

\subsection{\hspace{-0.5cm} {\Large \textcolor{darkblue}{\textbf{\ipa{tɕɯ˩\textsubscript{a}}}}}\hspace{0.5cm}[\kern2pt{\textcolor{darkblue}{\textbf{\ipa{tɕɯ˥}}}}\kern2pt]} \hypertarget{ts£M\string_Ba1}{}
\markboth{\textcolor{darkblue}{\textbf{\ipa{tɕɯ˩\textsubscript{a}}}}}{}
\textcolor{teal}{\zh{动词}} \hspace{4pt} \zh{声调类:} L\textsubscript{a}.
\zh{写。} \textcolor{Sepia}{\selectlanguage{english}To write.} \textcolor{PineGreen}{\selectlanguage{french}Écrire.}  ¶ \textcolor{darkblue}{\textbf{\ipa{le˧-tɕɯ˩-ze˩}}} \zh{写了} \textcolor{Sepia}{\selectlanguage{english}\mytextsc{accomp}+\mytextsc{pfv}} \textcolor{PineGreen}{\selectlanguage{french}\mytextsc{accomp}+\mytextsc{pfv}}  
 ¶ \textcolor{darkblue}{\textbf{\ipa{tʰæ˧ɻæ˩ tɕɯ˩}}} \zh{写、写书} \textcolor{Sepia}{\selectlanguage{english}to write, to write a text, to write a book} \textcolor{PineGreen}{\selectlanguage{french}écrire quelque chose/écrire du texte/écrire un livre}  
 ¶ \textcolor{darkblue}{\textbf{\ipa{ɖɯ˧-kʰv̩˥ | tsʰe˧-ɲi˧ ɬi˧, | njɤ˧ | tsʰe˧-ɲi˧ bæ˧ tɕɯ˩-bi˩-ʂv̩˩ɖv̩˩!}}} \zh{一年有十二个月,我就想(一年之内)记十二个故事!(情景:我两个月内完成了两个故事的记录工作。发音合作人举这个例句,鼓励我坚持这种速度,一年内再记十二个故事。)} \textcolor{Sepia}{\selectlanguage{english}There are twelve months in one year; I would like to transcribe twelve stories (in the coming year)! (Context: the consultant notices that I completed the transcription of two texts in two months; by providing this example sentence, she suggests to me the project of keeping up the same rhythm, transcribing twelve stories in the coming year.)} \textcolor{PineGreen}{\selectlanguage{french}Dans une année il y a douze mois; je voudrais transcrire douze histoires (au cours de l'année qui vient)! (contexte: en septembre 2011, Ama remarque que j'ai transcrit deux contes en deux mois; en m'offrant cet exemple, elle me souffle le projet de garder le rythme et de transcrire une histoire par mois soit douze pendant l'année qui vient)}  
 ¶ \textcolor{darkblue}{\textbf{\ipa{ɖɯ˧-tɕɯ˧\textasciitilde{}tɕɯ˥-ɻ̍˩}}} \zh{\mytextsc{delimitative} \string_ \mytextsc{red} \mytextsc{inceptive}} \textcolor{Sepia}{\selectlanguage{english}\mytextsc{delimitative} \string_ \mytextsc{red} \mytextsc{inceptive}} \textcolor{PineGreen}{\selectlanguage{french}\mytextsc{délimitatif} \string_ \mytextsc{red} \mytextsc{inchoatif}}  
 ¶ \textcolor{darkblue}{\textbf{\ipa{tɕɯ˩-di˩˥}}} \zh{笔。直译:‘(用来)书写的(东西)’} \textcolor{Sepia}{\selectlanguage{english}brush, pen; literally 'thing to write'} \textcolor{PineGreen}{\selectlanguage{french}pinceau; littéralement 'chose pour écrire'}  
 ¶ \textcolor{darkblue}{\textbf{\ipa{tʰæ˧ɻæ˩-tɕɯ˩-di˩}}} \zh{笔。直译:‘(用来)写书的(东西)’} \textcolor{Sepia}{\selectlanguage{english}brush, pen; literally 'thing to write books'} \textcolor{PineGreen}{\selectlanguage{french}pinceau; littéralement 'chose pour écrire des livres'}  
 ¶ \textcolor{darkblue}{\textbf{\ipa{ʈʂʰɯ˧ | tʰi˧-tɕɯ˧\textasciitilde{}tɕɯ˥ dʑo˩}}} \zh{他正在写写东西。} \textcolor{Sepia}{\selectlanguage{english}(S)he is writing} \textcolor{PineGreen}{\selectlanguage{french}Elle/il est en train d'écrire}  

\lhead{\firstmark}
\rhead{\botmark}

\subsection{\hspace{-0.5cm} {\Large \textcolor{darkblue}{\textbf{\ipa{tɕɯ˩lv̩˩ho˥}}}}\hspace{0.5cm}[\kern2pt{\textcolor{darkblue}{\textbf{\ipa{xxxx non-correspondance entre le nombre de morphèmes et le nombre de tons de morphèmes}}}}\kern2pt]} \hypertarget{ts£M\string_Blv\string_=\string_Bho\string_T1}{}
\markboth{\textcolor{darkblue}{\textbf{\ipa{tɕɯ˩lv̩˩ho˥}}}}{}
\textcolor{teal}{\zh{名词}} \hspace{4pt} \zh{声调类:} L+H\#.
\zh{弹弓。} \textcolor{Sepia}{\selectlanguage{english}Sling.} \textcolor{PineGreen}{\selectlanguage{french}Fronde (bande de tissu permettant de catapulter un objet).}  \zh{量词}: \textcolor{darkblue}{\textbf{\ipa{ɭɯ˧}}} 
\lhead{\firstmark}
\rhead{\botmark}

\subsection{\hspace{-0.5cm} {\Large \textcolor{darkblue}{\textbf{\ipa{tɕɯ˩ɭɯ˩}}}}\hspace{0.5cm}[\kern2pt{\textcolor{darkblue}{\textbf{\ipa{tɕɯ˧ɭɯ˧}}}}\kern2pt]} \hypertarget{ts£M\string_Bl\string_RM\string_B1}{}
\markboth{\textcolor{darkblue}{\textbf{\ipa{tɕɯ˩ɭɯ˩}}}}{}
\textcolor{teal}{\zh{名词}} \hspace{4pt} \zh{声调类:} L.
\zh{伯劳鸟。} \textcolor{Sepia}{\selectlanguage{english}Shrike, \textit{Lanius tephronotus}.} \textcolor{PineGreen}{\selectlanguage{french}Pie grièche du Tibet, \textit{Lanius tephronotus}.}  \zh{量词}: \textcolor{darkblue}{\textbf{\ipa{mi˩}}} 
\lhead{\firstmark}
\rhead{\botmark}

\subsection{\hspace{-0.5cm} {\Large \textcolor{darkblue}{\textbf{\ipa{tɕɯ˩ɭɯ˩-qʰæ˥bæ˩}}}}\hspace{0.5cm}[\kern2pt{\textcolor{darkblue}{\textbf{\ipa{xxxx non-correspondance entre le nombre de morphèmes et le nombre de tons de morphèmes}}}}\kern2pt]} \hypertarget{ts£M\string_Bl\string_RM\string_B-q\string_h\{\string_Tb\{\string_B1}{}
\markboth{\textcolor{darkblue}{\textbf{\ipa{tɕɯ˩ɭɯ˩-qʰæ˥bæ˩}}}}{}
\textcolor{teal}{\zh{名词}} \hspace{4pt} \zh{声调类:} L+\#H-.
\zh{终石藤。} \textcolor{PineGreen}{\selectlanguage{french}Littéralement: “cuillère de l'oiseau yyyy shrike”; mettre des renvois.}  \zh{量词}: \textcolor{darkblue}{\textbf{\ipa{dzi˩}}} 
\lhead{\firstmark}
\rhead{\botmark}

\subsection{\hspace{-0.5cm} {\Large \textcolor{darkblue}{\textbf{\ipa{tɕɯ˩mi˥}}}}\hspace{0.5cm}[\kern2pt{\textcolor{darkblue}{\textbf{\ipa{tɕɯ˩mi˥}}}}\kern2pt]} \hypertarget{ts£M\string_Bmi\string_T1}{}
\markboth{\textcolor{darkblue}{\textbf{\ipa{tɕɯ˩mi˥}}}}{}
\textcolor{teal}{\zh{名词}} \hspace{4pt} \zh{声调类:} LH.
\zh{画眉鸟。} \textcolor{Sepia}{\selectlanguage{english}Chinese Hwamei or Melodious Laughingthrush (\textit{Leucodioptron canorum}).} \textcolor{PineGreen}{\selectlanguage{french}Passereau de la famille des Leiothrichidae: (\textit{Leucodioptron canorum}). Le nom “hwamei” signifie «sourcils peints» en référence à la marque distinctive autour des yeux de l'oiseau.} \zh{当地汉语方言:}\zh{画眉鸟。} ¶ \textcolor{darkblue}{\textbf{\ipa{tɕɯ˩mi˥ | ə˧mi˧ ɲi˩!}}} \zh{是一个画眉鸟妈妈!(=是母的画眉鸟)} \textcolor{Sepia}{\selectlanguage{english}It's a mummy hwamei! (=a female)} \textcolor{PineGreen}{\selectlanguage{french}c'est une maman hwamei! (=une femelle)}  
 ¶ \textcolor{darkblue}{\textbf{\ipa{tɕɯ˩mi˥ | zo˧ ɲi˥!}}} \zh{是一个小画眉鸟!} \textcolor{Sepia}{\selectlanguage{english}It's a baby hwamei!} \textcolor{PineGreen}{\selectlanguage{french}c'est un petit hwamei! (=un enfant/bébé)}  
 ¶ \textcolor{darkblue}{\textbf{\ipa{tɕɯ˩mi˥ | pʰv̩˧ ɲi˩!}}} \zh{是公的画眉鸟!} \textcolor{Sepia}{\selectlanguage{english}It's a male hwamei!} \textcolor{PineGreen}{\selectlanguage{french}C'est un hwamei mâle!}  
 \zh{量词}: \textcolor{darkblue}{\textbf{\ipa{mi˩}}} 
\lhead{\firstmark}
\rhead{\botmark}

\subsection{\hspace{-0.5cm} {\Large \textcolor{darkblue}{\textbf{\ipa{tɕɯ˧˥}}} \textsubscript{1}}\hspace{0.5cm}[\kern2pt{\textcolor{darkblue}{\textbf{\ipa{tɕɯ˧˥}}}}\kern2pt]} \hypertarget{ts£M\string_M\string_T1}{}
\markboth{\textcolor{darkblue}{\textbf{\ipa{tɕɯ˧˥}}} \textsubscript{1}}{}
\textcolor{teal}{\zh{动词}} \hspace{4pt} \zh{声调类:} MH.
\zh{驮运。} \textcolor{Sepia}{\selectlanguage{english}To pack-transport.} \textcolor{PineGreen}{\selectlanguage{french}Transporter à dos d'animaux.}  ¶ \textcolor{darkblue}{\textbf{\ipa{ʐwæ˧ tɕɯ˩}}} \zh{用马驮运、做马帮} \textcolor{Sepia}{\selectlanguage{english}to pack-transport, to transport on horseback} \textcolor{PineGreen}{\selectlanguage{french}faire du commerce par caravanes, transporter en caravane, organiser une caravane}  
 ¶ \textcolor{darkblue}{\textbf{\ipa{ʐwæ˧ʁo˧ tʰi˧-tɕɯ˧˥}}} \zh{用马驮运} \textcolor{Sepia}{\selectlanguage{english}to transport on horseback} \textcolor{PineGreen}{\selectlanguage{french}transporter à dos de cheval}  
 ¶ \textcolor{darkblue}{\textbf{\ipa{ʐwæ˧-tɕɯ˩-zo˩}}} \zh{加入马帮的男人} \textcolor{Sepia}{\selectlanguage{english}person who takes part in a caravan, who works in a caravan} \textcolor{PineGreen}{\selectlanguage{french}caravanier, personne qui va avec les caravanes}  

\lhead{\firstmark}
\rhead{\botmark}

\subsection{\hspace{-0.5cm} {\Large \textcolor{darkblue}{\textbf{\ipa{tɕɯ˧˥}}} \textsubscript{2}}\hspace{0.5cm}[\kern2pt{\textcolor{darkblue}{\textbf{\ipa{tɕɯ˧˥}}}}\kern2pt]} \hypertarget{ts£M\string_M\string_T2}{}
\markboth{\textcolor{darkblue}{\textbf{\ipa{tɕɯ˧˥}}} \textsubscript{2}}{}
\textcolor{teal}{\zh{名词}} \hspace{4pt} \zh{声调类:} MH.
\zh{水蛭、蚂蟥。} \textcolor{Sepia}{\selectlanguage{english}Leech.} \textcolor{PineGreen}{\selectlanguage{french}Sangsue.} \zh{当地汉语方言:}\zh{蚂蟥。} \zh{量词}: \textcolor{darkblue}{\textbf{\ipa{mi˩}}} 
\lhead{\firstmark}
\rhead{\botmark}

\subsection{\hspace{-0.5cm} {\Large \textcolor{darkblue}{\textbf{\ipa{tɕɯ˧˥}}} \textsubscript{3}}\hspace{0.5cm}[\kern2pt{\textcolor{darkblue}{\textbf{\ipa{tɕɯ˧˥}}}}\kern2pt]} \hypertarget{ts£M\string_M\string_T3}{}
\markboth{\textcolor{darkblue}{\textbf{\ipa{tɕɯ˧˥}}} \textsubscript{3}}{}
\textcolor{teal}{\zh{名词}} \hspace{4pt} \zh{声调类:} MH.
\zh{马蜂 (黄蜂)。} \textcolor{Sepia}{\selectlanguage{english}Wasp.} \textcolor{PineGreen}{\selectlanguage{french}Guêpe.}  ¶ \textcolor{darkblue}{\textbf{\ipa{tɕɯ˧mi˥\$}}} \zh{母蚂蜂(人工的词)} \textcolor{Sepia}{\selectlanguage{english}female wasp (elicited combination)} \textcolor{PineGreen}{\selectlanguage{french}guêpe femelle (élicité pour le propos de l'étude tonale)}  
 ¶ \textcolor{darkblue}{\textbf{\ipa{tɕɯ˧pʰv̩\#˥}}} \zh{公马蜂} \textcolor{Sepia}{\selectlanguage{english}male wasp (elicited combination)} \textcolor{PineGreen}{\selectlanguage{french}guêpe mâle (élicité pour le propos de l'étude tonale)}  
 ¶ \textcolor{darkblue}{\textbf{\ipa{tɕɯ˧zo\#˥}}} \zh{小马蜂} \textcolor{Sepia}{\selectlanguage{english}baby wasp (elicited combination)} \textcolor{PineGreen}{\selectlanguage{french}bébé guêpe (élicité pour le propos de l'étude tonale)}  
 \zh{量词}: \textcolor{darkblue}{\textbf{\ipa{mi˩}}} 
\lhead{\firstmark}
\rhead{\botmark}

\subsection{\hspace{-0.5cm} {\Large \textcolor{darkblue}{\textbf{\ipa{tɕɯ˧˥}}} \textsubscript{4}}\hspace{0.5cm}[\kern2pt{\textcolor{darkblue}{\textbf{\ipa{tɕɯ˧˥}}}}\kern2pt]} \hypertarget{ts£M\string_M\string_T4}{}
\markboth{\textcolor{darkblue}{\textbf{\ipa{tɕɯ˧˥}}} \textsubscript{4}}{}
\textcolor{teal}{\zh{名词}} \hspace{4pt} \zh{声调类:} MH.
\zh{称。} \textcolor{Sepia}{\selectlanguage{english}Scale.} \textcolor{PineGreen}{\selectlanguage{french}Balance.}  \zh{量词}: \textcolor{darkblue}{\textbf{\ipa{nɑ˧}}} 
\lhead{\firstmark}
\rhead{\botmark}

\subsection{\hspace{-0.5cm} {\Large \textcolor{darkblue}{\textbf{\ipa{tɕɯ˧˥\textsubscript{a}}}} \textsubscript{1}}\hspace{0.5cm}[\kern2pt{\textcolor{darkblue}{\textbf{\ipa{tɕɯ˩˥}}}}\kern2pt]} \hypertarget{ts£M\string_M\string_Ta1}{}
\markboth{\textcolor{darkblue}{\textbf{\ipa{tɕɯ˧˥\textsubscript{a}}}} \textsubscript{1}}{}
\textcolor{teal}{\zh{量词}} \hspace{4pt} \zh{声调类:} MH\textsubscript{a}.
\zh{量词:驮子(一匹)。} \textcolor{Sepia}{\selectlanguage{english}Classifier for loads carried by a pack-animal.} \textcolor{PineGreen}{\selectlanguage{french}Classificateur des charges sur une bête de somme.} 
\lhead{\firstmark}
\rhead{\botmark}

\subsection{\hspace{-0.5cm} {\Large \textcolor{darkblue}{\textbf{\ipa{tɕɯ˧˥\textsubscript{a}}}} \textsubscript{2}}\hspace{0.5cm}[\kern2pt{\textcolor{darkblue}{\textbf{\ipa{tɕɯ˧˥}}}}\kern2pt]} \hypertarget{ts£M\string_M\string_Ta2}{}
\markboth{\textcolor{darkblue}{\textbf{\ipa{tɕɯ˧˥\textsubscript{a}}}} \textsubscript{2}}{}
\textcolor{teal}{\zh{量词}} \hspace{4pt} \zh{声调类:} MH\textsubscript{a}.
\zh{量词:斤(用于固体,也用于液体)(汉语借词)。} \textcolor{Sepia}{\selectlanguage{english}Classifier: a pound of.} \textcolor{PineGreen}{\selectlanguage{french}Livre (aussi pour les liquides: pinte).}  \zh{【借词】} \zh{斤}
 ¶ \textcolor{darkblue}{\textbf{\ipa{ʐɯ˧ | ɖɯ˧-tɕɯ˧˥}}} \zh{一斤酒} \textcolor{Sepia}{\selectlanguage{english}a pint of wine} \textcolor{PineGreen}{\selectlanguage{french}une pinte de vin}  

\lhead{\firstmark}
\rhead{\botmark}

\subsection{\hspace{-0.5cm} {\Large \textcolor{darkblue}{\textbf{\ipa{tɕɯ˩˥}}}}\hspace{0.5cm}[\kern2pt{\textcolor{darkblue}{\textbf{\ipa{tɕɯ˩˥}}}}\kern2pt]} \hypertarget{ts£M\string_B\string_T1}{}
\markboth{\textcolor{darkblue}{\textbf{\ipa{tɕɯ˩˥}}}}{}
\textcolor{teal}{\zh{名词}} \hspace{4pt} \zh{声调类:} LH.
\zh{口水、唾、唾沫、唾液。} \textcolor{Sepia}{\selectlanguage{english}Saliva.} \textcolor{PineGreen}{\selectlanguage{french}Salive.} 
\lhead{\firstmark}
\rhead{\botmark}

\subsection{\hspace{-0.5cm} {\Large \textcolor{darkblue}{\textbf{\ipa{tɕʰɤ˧pɤ˧-mi\#˥}}}}\hspace{0.5cm}[\kern2pt{\textcolor{darkblue}{\textbf{\ipa{xxxx non-correspondance entre le nombre de morphèmes et le nombre de tons de morphèmes}}}}\kern2pt]} \hypertarget{ts£\string_h7\string_Mp7\string_M-mi\#\string_T1}{}
\markboth{\textcolor{darkblue}{\textbf{\ipa{tɕʰɤ˧pɤ˧-mi\#˥}}}}{}
\textcolor{teal}{\zh{名词}} \hspace{4pt} \zh{声调类:} \#H.
\zh{一处神泉。} \textcolor{Sepia}{\selectlanguage{english}The name of a sacred spring located in a cave on mount \textcolor{darkblue}{\textbf{\ipa{/nɑ˩tsʰi˩/}}}.} \textcolor{PineGreen}{\selectlanguage{french}Nom d'une source sacrée, située dans une grotte, sur la montagne \textcolor{darkblue}{\textbf{\ipa{/nɑ˩tsʰi˩/}}}.}  ¶ \textcolor{darkblue}{\textbf{\ipa{nɑ˩tsʰi˩˥ | tɕʰɤ˧pɤ˧-mi\#˥}}} \zh{神泉所在山的全称} \textcolor{Sepia}{\selectlanguage{english}full name of the mountain where the spring is located} \textcolor{PineGreen}{\selectlanguage{french}nom complet de la montagne où se trouve la source sacrée}  

\lhead{\firstmark}
\rhead{\botmark}

\subsection{\hspace{-0.5cm} {\Large \textcolor{darkblue}{\textbf{\ipa{tɕʰɤ˧ʂo\#˥}}}}\hspace{0.5cm}[\kern2pt{\textcolor{darkblue}{\textbf{\ipa{tɕʰɤ˧ʂo˧}}}}\kern2pt]} \hypertarget{ts£\string_h7\string_Ms`o\#\string_T1}{}
\markboth{\textcolor{darkblue}{\textbf{\ipa{tɕʰɤ˧ʂo\#˥}}}}{}
\textcolor{teal}{\zh{名词}} \hspace{4pt} \zh{声调类:} \#H.
\zh{祭坛。} \textcolor{Sepia}{\selectlanguage{english}Altar.} \textcolor{PineGreen}{\selectlanguage{french}Autel, lieu où on brûle de l'encens (dans la maison: l'autel principal est à l'étage, dans le bâtiment face à la porte de la ferme).}  \zh{量词}: \textcolor{darkblue}{\textbf{\ipa{nɑ˧}}} 
\lhead{\firstmark}
\rhead{\botmark}

\subsection{\hspace{-0.5cm} {\Large \textcolor{darkblue}{\textbf{\ipa{tɕʰɤ˧ti\#˥}}}}\hspace{0.5cm}[\kern2pt{\textcolor{darkblue}{\textbf{\ipa{tɕʰɤ˧ti˧}}}}\kern2pt]} \hypertarget{ts£\string_h7\string_Mti\#\string_T1}{}
\markboth{\textcolor{darkblue}{\textbf{\ipa{tɕʰɤ˧ti\#˥}}}}{}
\textcolor{teal}{\zh{名词}} \hspace{4pt} \zh{声调类:} \#H.
\zh{塔。} \textcolor{Sepia}{\selectlanguage{english}Stupa, tower.} \textcolor{PineGreen}{\selectlanguage{french}Stupa, tour.}  \zh{量词}: \textcolor{darkblue}{\textbf{\ipa{ɭɯ˧}}} 
\lhead{\firstmark}
\rhead{\botmark}

\subsection{\hspace{-0.5cm} {\Large \textcolor{darkblue}{\textbf{\ipa{tɕʰɤ˧tɕo˩}}}}\hspace{0.5cm}[\kern2pt{\textcolor{darkblue}{\textbf{\ipa{tɕʰɤ˧tɕo˩}}}}\kern2pt]} \hypertarget{ts£\string_h7\string_Mts£o\string_B1}{}
\markboth{\textcolor{darkblue}{\textbf{\ipa{tɕʰɤ˧tɕo˩}}}}{}
\textcolor{teal}{\zh{名词}} \hspace{4pt} \zh{声调类:} L\#.
\zh{双人锯:以前用于把圆木截成板材的大的双人锯(汉语借词)。} \textcolor{Sepia}{\selectlanguage{english}Two-man saw: a saw designed for use by two sawyers.} \textcolor{PineGreen}{\selectlanguage{french}Scie passe-partout: grande scie avec une poignée à chaque extrémité, maniée par deux bûcherons.}  \zh{【借词】} \zh{??}

\lhead{\firstmark}
\rhead{\botmark}

\subsection{\hspace{-0.5cm} {\Large \textcolor{darkblue}{\textbf{\ipa{tɕʰɤ˧tɕʰɤ˧˥}}}}\hspace{0.5cm}[\kern2pt{\textcolor{darkblue}{\textbf{\ipa{tɕʰɤ˧tɕʰɤ˧˥}}}}\kern2pt]} \hypertarget{ts£\string_h7\string_Mts£\string_h7\string_M\string_T1}{}
\markboth{\textcolor{darkblue}{\textbf{\ipa{tɕʰɤ˧tɕʰɤ˧˥}}}}{}
\textcolor{teal}{\zh{助词}} \hspace{4pt} \zh{声调类:} MH\#.
\zh{彻底。} \textcolor{Sepia}{\selectlanguage{english}Entirely, completely, totally.} \textcolor{PineGreen}{\selectlanguage{french}Entièrement, tout à fait, complètement.} 
\lhead{\firstmark}
\rhead{\botmark}

\subsection{\hspace{-0.5cm} {\Large \textcolor{darkblue}{\textbf{\ipa{tɕʰɤ˩lv̩˩}}}}\hspace{0.5cm}[\kern2pt{\textcolor{darkblue}{\textbf{\ipa{tɕʰɤ˩lv̩˩˥}}}}\kern2pt]} \hypertarget{ts£\string_h7\string_Blv\string_=\string_B1}{}
\markboth{\textcolor{darkblue}{\textbf{\ipa{tɕʰɤ˩lv̩˩}}}}{}
\textcolor{teal}{\zh{名词}} \hspace{4pt} \zh{声调类:} L.
\zh{百合。} \textcolor{Sepia}{\selectlanguage{english}Lily, lily buds.} \textcolor{PineGreen}{\selectlanguage{french}Lis.}  ¶ \textcolor{darkblue}{\textbf{\ipa{tɕʰɤ˩lv̩˩-hṽ˩hṽ˩˥}}} \zh{炒百合} \textcolor{Sepia}{\selectlanguage{english}stir-fried lily buds} \textcolor{PineGreen}{\selectlanguage{french}lis cuits au wok}  
 ¶ \textcolor{darkblue}{\textbf{\ipa{tɕʰɤ˩lv̩˩˥, | kv̩˧-pʰæ˧di˥!}}} \zh{百合,像大蒜!} \textcolor{Sepia}{\selectlanguage{english}Lily buds look like garlic!} \textcolor{PineGreen}{\selectlanguage{french}Le lis, ça ressemble à de l'ail!}  
 ¶ \textcolor{darkblue}{\textbf{\ipa{tɕʰɤ˩lv̩˩˥, | dʑɯ˩-nɑ˩mi˩-ʁo˥ dʑɯ˩-nɑ˩mi˩-ʁo˥ di˩-kv̩˩!}}} \zh{百合长在高山上!} \textcolor{Sepia}{\selectlanguage{english}Lilies grow high up on the mountain!} \textcolor{PineGreen}{\selectlanguage{french}Le lis, ça pousse dans la montagne/en haute montagne!}  
 ¶ \textcolor{darkblue}{\textbf{\ipa{tɕʰɤ˩lv̩˩˥, | dʑɯ˩-nɑ˩mi˩-ʁo˥ | di˩-kv̩˩˥! |}}} \zh{百合长在高山上!} \textcolor{Sepia}{\selectlanguage{english}Lilies grow high up on the mountain!} \textcolor{PineGreen}{\selectlanguage{french}Le lis, ça pousse dans la montagne/en haute montagne!}  

\lhead{\firstmark}
\rhead{\botmark}

\subsection{\hspace{-0.5cm} {\Large \textcolor{darkblue}{\textbf{\ipa{tɕʰɤ˩ʈʂv̩˧}}}}\hspace{0.5cm}[\kern2pt{\textcolor{darkblue}{\textbf{\ipa{tɕʰɤ˩ʈʂv̩˥}}}}\kern2pt]} \hypertarget{ts£\string_h7\string_Bt`s`v\string_=\string_M1}{}
\markboth{\textcolor{darkblue}{\textbf{\ipa{tɕʰɤ˩ʈʂv̩˧}}}}{}
\textcolor{teal}{\zh{名词}} \hspace{4pt} \zh{声调类:} LM.
\zh{酒杯。} \textcolor{Sepia}{\selectlanguage{english}Glass used for wine.} \textcolor{PineGreen}{\selectlanguage{french}Verre pour le vin ou autres liquides (sans anse); en verre ou autre matériau.}  \zh{量词}: \textcolor{darkblue}{\textbf{\ipa{ɭɯ˧}}} 
\lhead{\firstmark}
\rhead{\botmark}

\subsection{\hspace{-0.5cm} {\Large \textcolor{darkblue}{\textbf{\ipa{tɕʰɤ˩ʈʂv˧}}}}\hspace{0.5cm}[\kern2pt{\textcolor{darkblue}{\textbf{\ipa{tɕʰɤ˩ʈʂv˥}}}}\kern2pt]} \hypertarget{ts£\string_h7\string_Bt`s`v\string_M1}{}
\markboth{\textcolor{darkblue}{\textbf{\ipa{tɕʰɤ˩ʈʂv˧}}}}{}
\textcolor{teal}{\zh{名词}} \hspace{4pt} \zh{声调类:} LM.
\zh{杯子。} \textcolor{Sepia}{\selectlanguage{english}Drinking glass, goblet.} \textcolor{PineGreen}{\selectlanguage{french}Verre, gobelet.}  ¶ \textcolor{darkblue}{\textbf{\ipa{bo˧ʐæ˧-tɕʰɤ˩ʈʂv˩}}} \zh{玻璃茶杯} \textcolor{Sepia}{\selectlanguage{english}goblet for drinking tea (made of glass)} \textcolor{PineGreen}{\selectlanguage{french}gobelet à thé en verre}  

\lhead{\firstmark}
\rhead{\botmark}

\subsection{\hspace{-0.5cm} {\Large \textcolor{darkblue}{\textbf{\ipa{tɕʰɤ˧˥}}}}\hspace{0.5cm}[\kern2pt{\textcolor{darkblue}{\textbf{\ipa{tɕʰɤ˧˥}}}}\kern2pt]} \hypertarget{ts£\string_h7\string_M\string_T1}{}
\markboth{\textcolor{darkblue}{\textbf{\ipa{tɕʰɤ˧˥}}}}{}
\textcolor{teal}{\zh{动词}} \hspace{4pt} \zh{声调类:} MH.
\zh{欺骗。} \textcolor{Sepia}{\selectlanguage{english}To cheat on someone, to deceive.} \textcolor{PineGreen}{\selectlanguage{french}Tromper.}  ¶ \textcolor{darkblue}{\textbf{\ipa{le˧-tɕʰɤ˧-ze˥}}} \zh{欺骗了} \textcolor{Sepia}{\selectlanguage{english}\mytextsc{accomp} \string_ \mytextsc{pfv}} \textcolor{PineGreen}{\selectlanguage{french}\mytextsc{accomp} \string_ \mytextsc{pfv}}  
 ¶ \textcolor{darkblue}{\textbf{\ipa{hĩ˧ tɕʰɤ˧(-ze˩)}}} \zh{骗人} \textcolor{Sepia}{\selectlanguage{english}to cheat on people, to deceive people} \textcolor{PineGreen}{\selectlanguage{french}tromper les gens}  
 ¶ \textcolor{darkblue}{\textbf{\ipa{no˧ | hĩ˧ tɕʰɤ˧!}}} \zh{你骗人!} \textcolor{Sepia}{\selectlanguage{english}You cheat people! / You deceive people!} \textcolor{PineGreen}{\selectlanguage{french}vous trompez les gens!}  
 ¶ \textcolor{darkblue}{\textbf{\ipa{(hĩ˧ |) no˩ tɕʰɤ˩˥!}}} \zh{人家骗你!} \textcolor{Sepia}{\selectlanguage{english}People cheat you!} \textcolor{PineGreen}{\selectlanguage{french}les gens vous trompent!}  
 ¶ \textcolor{darkblue}{\textbf{\ipa{(no˧ |) njɤ˩ tɕʰɤ˩˥!}}} \zh{你骗我!} \textcolor{Sepia}{\selectlanguage{english}You cheat on me!} \textcolor{PineGreen}{\selectlanguage{french}Vous me trompez!}  

\lhead{\firstmark}
\rhead{\botmark}

\subsection{\hspace{-0.5cm} {\Large \textcolor{darkblue}{\textbf{\ipa{tɕʰi˥}}}}\hspace{0.5cm}[\kern2pt{\textcolor{darkblue}{\textbf{\ipa{tɕʰi˥}}}}\kern2pt]} \hypertarget{ts£\string_hi\string_T1}{}
\markboth{\textcolor{darkblue}{\textbf{\ipa{tɕʰi˥}}}}{}
\textcolor{teal}{\zh{名词}} \hspace{4pt} \zh{声调类:} \#H.
\zh{刺。} \textcolor{Sepia}{\selectlanguage{english}Thorn.} \textcolor{PineGreen}{\selectlanguage{french}Épine.}  \zh{量词}: \textcolor{darkblue}{\textbf{\ipa{kɤ˧˥}}} 
\lhead{\firstmark}
\rhead{\botmark}

\subsection{\hspace{-0.5cm} {\Large \textcolor{darkblue}{\textbf{\ipa{tɕʰi˧\textsubscript{b}}}} \textsubscript{1}}\hspace{0.5cm}[\kern2pt{\textcolor{darkblue}{\textbf{\ipa{tɕʰi˥}}}}\kern2pt]} \hypertarget{ts£\string_hi\string_Mb1}{}
\markboth{\textcolor{darkblue}{\textbf{\ipa{tɕʰi˧\textsubscript{b}}}} \textsubscript{1}}{}
\textcolor{teal}{\zh{动词}} \hspace{4pt} \zh{声调类:} M\textsubscript{b}.
\zh{守卫。} \textcolor{Sepia}{\selectlanguage{english}To guard, to defend (e.g. guard a house).} \textcolor{PineGreen}{\selectlanguage{french}Garder, surveiller (ex.: garder la maison).}  ¶ \textcolor{darkblue}{\textbf{\ipa{ɑ˩ʁo˧ tɕʰi˧}}} \zh{守护家} \textcolor{Sepia}{\selectlanguage{english}to watch over the house, to guard the house} \textcolor{PineGreen}{\selectlanguage{french}garder la maison}  
 ¶ \textcolor{darkblue}{\textbf{\ipa{ɑ˩ʁo˧ tʰi˧-tɕʰi˧-dʑo˧}}} \zh{守着家} \textcolor{Sepia}{\selectlanguage{english}watching over the house} \textcolor{PineGreen}{\selectlanguage{french}en train de surveiller la maison}  
 ¶ \textcolor{darkblue}{\textbf{\ipa{tso˧\textasciitilde{}tso˧ tɕʰi˧}}} \zh{守着东西} \textcolor{Sepia}{\selectlanguage{english}to watch over things} \textcolor{PineGreen}{\selectlanguage{french}surveiller des objets}  

\lhead{\firstmark}
\rhead{\botmark}

\subsection{\hspace{-0.5cm} {\Large \textcolor{darkblue}{\textbf{\ipa{tɕʰi˧\textsubscript{b}}}} \textsubscript{2}}\hspace{0.5cm}[\kern2pt{\textcolor{darkblue}{\textbf{\ipa{tɕʰi˥}}}}\kern2pt]} \hypertarget{ts£\string_hi\string_Mb2}{}
\markboth{\textcolor{darkblue}{\textbf{\ipa{tɕʰi˧\textsubscript{b}}}} \textsubscript{2}}{}
\textcolor{teal}{\zh{动词}} \hspace{4pt} \zh{声调类:} M\textsubscript{b}.
\zh{卖。} \textcolor{Sepia}{\selectlanguage{english}To sell.} \textcolor{PineGreen}{\selectlanguage{french}Vendre.} 
\lhead{\firstmark}
\rhead{\botmark}

\subsection{\hspace{-0.5cm} {\Large \textcolor{darkblue}{\textbf{\ipa{tɕʰi˧ɖv̩\#˥}}}}\hspace{0.5cm}[\kern2pt{\textcolor{darkblue}{\textbf{\ipa{tɕʰi˧ɖv̩˧}}}}\kern2pt]} \hypertarget{ts£\string_hi\string_Md`v\string_=\#\string_T1}{}
\markboth{\textcolor{darkblue}{\textbf{\ipa{tɕʰi˧ɖv̩\#˥}}}}{}
\textcolor{teal}{\zh{名词}} \hspace{4pt} \zh{声调类:} \#H.
\zh{女性名字。} \textcolor{Sepia}{\selectlanguage{english}Feminine given name.} \textcolor{PineGreen}{\selectlanguage{french}Prénom féminin.} 
\lhead{\firstmark}
\rhead{\botmark}

\subsection{\hspace{-0.5cm} {\Large \textcolor{darkblue}{\textbf{\ipa{tɕʰi˧nɑ˥}}}}\hspace{0.5cm}[\kern2pt{\textcolor{darkblue}{\textbf{\ipa{tɕʰi˧nɑ˧}}}}\kern2pt]} \hypertarget{ts£\string_hi\string_MnA\string_T1}{}
\markboth{\textcolor{darkblue}{\textbf{\ipa{tɕʰi˧nɑ˥}}}}{}
\textcolor{teal}{\zh{名词}} \hspace{4pt} \zh{声调类:} H\#.
\zh{青刺果、青刺尖、阿娜斯果。} \textcolor{Sepia}{\selectlanguage{english}Prinsepia, \textit{Prinsepia utilis Royle}; its seeds yield a highly valued oil, for both cooking and massaging on people's bodies.} \textcolor{PineGreen}{\selectlanguage{french}Prinsepia, \textit{Prinsepia utilis Royle}; végétal qui sert pour les haies, à grosses épines, petites fleurs jaunes, et tige verte vernissée. On tire de ses graines une huile de grand prix, utilisée dans des préparations alimentaires et comme cosmétique/huile de massage.}  ¶ \textcolor{darkblue}{\textbf{\ipa{tɕʰi˧nɑ˥-dzi˩}}} \zh{青刺尖} \textcolor{Sepia}{\selectlanguage{english}prinsepia plant} \textcolor{PineGreen}{\selectlanguage{french}prinsepia (la plante)}  
 ¶ \textcolor{darkblue}{\textbf{\ipa{tɕʰi˧nɑ˥-bæ˩bæ˩}}} \zh{青刺果花} \textcolor{Sepia}{\selectlanguage{english}prinsepia flower} \textcolor{PineGreen}{\selectlanguage{french}fleur de prinsepia}  

\lhead{\firstmark}
\rhead{\botmark}

\subsection{\hspace{-0.5cm} {\Large \textcolor{darkblue}{\textbf{\ipa{tɕʰi˧ʈʂʰɤ˥}}}}\hspace{0.5cm}[\kern2pt{\textcolor{darkblue}{\textbf{\ipa{xxxx non-correspondance entre le nombre de morphèmes et le nombre de tons de morphèmes}}}}\kern2pt]} \hypertarget{ts£\string_hi\string_Mt`s`\string_h7\string_T1}{}
\markboth{\textcolor{darkblue}{\textbf{\ipa{tɕʰi˧ʈʂʰɤ˥}}}}{}
\textcolor{teal}{\zh{名词}} \hspace{4pt} \zh{声调类:} H\#.
\zh{汽车(汉语借词)。} \textcolor{Sepia}{\selectlanguage{english}Car.} \textcolor{PineGreen}{\selectlanguage{french}Voiture, automobile.}  \zh{【借词】} \zh{汽车}
 \zh{量词}: \textcolor{darkblue}{\textbf{\ipa{nɑ˧}}} 
\lhead{\firstmark}
\rhead{\botmark}

\subsection{\hspace{-0.5cm} {\Large \textcolor{darkblue}{\textbf{\ipa{tɕʰi˩\textsubscript{b}}}}}\hspace{0.5cm}[\kern2pt{\textcolor{darkblue}{\textbf{\ipa{tɕʰi˩˥}}}}\kern2pt]} \hypertarget{ts£\string_hi\string_Bb1}{}
\markboth{\textcolor{darkblue}{\textbf{\ipa{tɕʰi˩\textsubscript{b}}}}}{}
\textcolor{teal}{\zh{量词}} \hspace{4pt} \zh{声调类:} L\textsubscript{b}.
\zh{量词:饭(一顿)。} \textcolor{Sepia}{\selectlanguage{english}Classifier for meals.} \textcolor{PineGreen}{\selectlanguage{french}Classificateur des repas.}  ¶ \textcolor{darkblue}{\textbf{\ipa{ɖɯ˧-tɕʰi˩ dzɯ˩}}} \zh{吃一顿} \textcolor{Sepia}{\selectlanguage{english}to have a meal, to eat a meal} \textcolor{PineGreen}{\selectlanguage{french}prendre un repas}  
 ¶ \textcolor{darkblue}{\textbf{\ipa{gv̩˧-tɕʰi˥}}} \zh{就顿(饭)} \textcolor{Sepia}{\selectlanguage{english}nine meals} \textcolor{PineGreen}{\selectlanguage{french}neuf repas}  
 ¶ \textcolor{darkblue}{\textbf{\ipa{tɕʰi˩ tʰv̩˩˥}}} \zh{带饭,“出(一)顿(饭)”:被请参加守孝时,要给那家主人带上饭)} \textcolor{Sepia}{\selectlanguage{english}to contribute food for the meals during a funeral ceremony: when one is invited to a funeral, one brings food as a contribution to the funeral} \textcolor{PineGreen}{\selectlanguage{french}apporter de la nourriture, apporter un repas: lorsqu'on est invité à participer à des cérémonies funéraires, on apporte à manger, pour contribuer aux repas collectifs}  
 ¶ \textcolor{darkblue}{\textbf{\ipa{tɕʰi˩tʰv̩˩-hĩ˥}}} \zh{给大家供饭的那个人(不一定是主人)} \textcolor{Sepia}{\selectlanguage{english}the person who provides the meal at a wake (following a funeral); it is generally someone who is not from the household.} \textcolor{PineGreen}{\selectlanguage{french}la personne qui se charge du repas / qui nourrit tous les participants (lors d'un repas de veillée funéraire)}  

\lhead{\firstmark}
\rhead{\botmark}

\subsection{\hspace{-0.5cm} {\Large \textcolor{darkblue}{\textbf{\ipa{tɕʰi˩tsɯ˧}}}}\hspace{0.5cm}[\kern2pt{\textcolor{darkblue}{\textbf{\ipa{tɕʰi˧tsɯ˥}}}}\kern2pt]} \hypertarget{ts£\string_hi\string_BtsM\string_M1}{}
\markboth{\textcolor{darkblue}{\textbf{\ipa{tɕʰi˩tsɯ˧}}}}{}
\textcolor{teal}{\zh{名词}} \hspace{4pt} \zh{声调类:} LM.
\zh{茄子。} \textcolor{Sepia}{\selectlanguage{english}Eggplant.} \textcolor{PineGreen}{\selectlanguage{french}Aubergine.}  \zh{【借词】} \zh{茄子}

\lhead{\firstmark}
\rhead{\botmark}

\subsection{\hspace{-0.5cm} {\Large \textcolor{darkblue}{\textbf{\ipa{tɕʰo˩}}}}\hspace{0.5cm}[\kern2pt{\textcolor{darkblue}{\textbf{\ipa{tɕʰo˩˥}}}}\kern2pt]} \hypertarget{ts£\string_ho\string_B1}{}
\markboth{\textcolor{darkblue}{\textbf{\ipa{tɕʰo˩}}}}{}
\textcolor{teal}{\zh{量词}} \hspace{4pt} \zh{声调类:} L *.
\zh{量词:一起。} \textcolor{Sepia}{\selectlanguage{english}Classifier: in combination with 'one', means 'together'; no plural form.} \textcolor{PineGreen}{\selectlanguage{french}Ensemble.}  ¶ \textcolor{darkblue}{\textbf{\ipa{ɖɯ˧-tɕʰo˩}}} \zh{一起} \textcolor{Sepia}{\selectlanguage{english}together} \textcolor{PineGreen}{\selectlanguage{french}ensemble}  
 ¶ \textcolor{darkblue}{\textbf{\ipa{le˧-tɕʰo˥\textasciitilde{}tɕʰo˩}}} \zh{同上:一起} \textcolor{Sepia}{\selectlanguage{english}same meaning as above: together} \textcolor{PineGreen}{\selectlanguage{french}même sens que ci-dessus: ensemble}  

\lhead{\firstmark}
\rhead{\botmark}

\subsection{\hspace{-0.5cm} {\Large \textcolor{darkblue}{\textbf{\ipa{tɕʰo˩\textsubscript{a}}}}}\hspace{0.5cm}[\kern2pt{\textcolor{darkblue}{\textbf{\ipa{tɕʰo˩˥}}}}\kern2pt]} \hypertarget{ts£\string_ho\string_Ba1}{}
\markboth{\textcolor{darkblue}{\textbf{\ipa{tɕʰo˩\textsubscript{a}}}}}{}
\textcolor{teal}{\zh{动词}} \hspace{4pt} \zh{声调类:} L\textsubscript{a}.
\zh{陪伴、一起去、跟着。} \textcolor{Sepia}{\selectlanguage{english}To accompany someone, to go along with someone.} \textcolor{PineGreen}{\selectlanguage{french}Accompagner, suivre (quelqu'un lors d'un voyage, par exemple); aller avec.}  ¶ \textcolor{darkblue}{\textbf{\ipa{hĩ˧ tɕʰo˥}}} \zh{陪伴某人} \textcolor{Sepia}{\selectlanguage{english}to accompany someone} \textcolor{PineGreen}{\selectlanguage{french}suivre quelqu'un}  
 ¶ \textcolor{darkblue}{\textbf{\ipa{ɖɯ˧-tɕʰo˩ tʰi˩-tɕʰo˩ |}}} \zh{陪伴某人} \textcolor{Sepia}{\selectlanguage{english}to make up a set, to go with each other/one another: for instance, in the main room, the thangka above the hearth and the paintings on the cupboard that hosts the altar to the ancestors make up a set, they go with each other} \textcolor{PineGreen}{\selectlanguage{french}aller ensemble, former un ensemble: par exemple, dans la pièce principale de la maison, le thangka au-dessus du foyer et les peintures sur le buffet-autel des ancêtres forment un tout, elles vont ensemble}  

\lhead{\firstmark}
\rhead{\botmark}

\subsection{\hspace{-0.5cm} {\Large \textcolor{darkblue}{\textbf{\ipa{tɕʰo˩mi\#˥}}}}\hspace{0.5cm}[\kern2pt{\textcolor{darkblue}{\textbf{\ipa{tɕʰo˩mi˥}}}}\kern2pt]} \hypertarget{ts£\string_ho\string_Bmi\#\string_T1}{}
\markboth{\textcolor{darkblue}{\textbf{\ipa{tɕʰo˩mi\#˥}}}}{}
\textcolor{teal}{\zh{名词}} \hspace{4pt} \zh{声调类:} LM+\#H.
\zh{大瓢。} \textcolor{Sepia}{\selectlanguage{english}Large ladle.} \textcolor{PineGreen}{\selectlanguage{french}Grande louche.}  \zh{量词}: \textcolor{darkblue}{\textbf{\ipa{nɑ˧}}} 
\lhead{\firstmark}
\rhead{\botmark}

\subsection{\hspace{-0.5cm} {\Large \textcolor{darkblue}{\textbf{\ipa{tɕʰo˩qʰwɤ˧}}}}\hspace{0.5cm}[\kern2pt{\textcolor{darkblue}{\textbf{\ipa{tɕʰo˩qʰwɤ˥}}}}\kern2pt]} \hypertarget{ts£\string_ho\string_Bq\string_hw7\string_M1}{}
\markboth{\textcolor{darkblue}{\textbf{\ipa{tɕʰo˩qʰwɤ˧}}}}{}
\textcolor{teal}{\zh{名词}} \hspace{4pt} \zh{声调类:} LM.
\zh{用来煮猪食的勺子。} \textcolor{Sepia}{\selectlanguage{english}Ladle used for pigswill.} \textcolor{PineGreen}{\selectlanguage{french}Louche utilisée pour les aliments des animaux; à la date de l'enquête, c'était un objet en aluminium, tandis que celui utilisé pour puiser l'eau, que l'on peut porter à la bouche, est en étain.}  \zh{量词}: \textcolor{darkblue}{\textbf{\ipa{nɑ˧}}} 
\lhead{\firstmark}
\rhead{\botmark}

\subsection{\hspace{-0.5cm} {\Large \textcolor{darkblue}{\textbf{\ipa{tɕʰo˩zo\#˥}}}}\hspace{0.5cm}[\kern2pt{\textcolor{darkblue}{\textbf{\ipa{tɕʰo˩zo˥}}}}\kern2pt]} \hypertarget{ts£\string_ho\string_Bzo\#\string_T1}{}
\markboth{\textcolor{darkblue}{\textbf{\ipa{tɕʰo˩zo\#˥}}}}{}
\textcolor{teal}{\zh{名词}} \hspace{4pt} \zh{声调类:} LM+\#H.
\zh{小瓢。} \textcolor{Sepia}{\selectlanguage{english}Small ladle.} \textcolor{PineGreen}{\selectlanguage{french}Petite louche.}  \zh{量词}: \textcolor{darkblue}{\textbf{\ipa{nɑ˧}}} 
\lhead{\firstmark}
\rhead{\botmark}

\subsection{\hspace{-0.5cm} {\Large \textcolor{darkblue}{\textbf{\ipa{tɕʰo˧˥}}}}\hspace{0.5cm}[\kern2pt{\textcolor{darkblue}{\textbf{\ipa{tɕʰo˧˥}}}}\kern2pt]} \hypertarget{ts£\string_ho\string_M\string_T1}{}
\markboth{\textcolor{darkblue}{\textbf{\ipa{tɕʰo˧˥}}}}{}
\textcolor{teal}{\zh{动词}} \hspace{4pt} \zh{声调类:} MH.
\zh{(将木料)砍成方形。} \textcolor{Sepia}{\selectlanguage{english}To square (off).} \textcolor{PineGreen}{\selectlanguage{french}Équarrir (une grosse pièce de bois de charpente).}  ¶ \textcolor{darkblue}{\textbf{\ipa{bi˩mi˩-ɳɯ˥ | tɕʰo˧˥}}} \zh{用斧头砍成方形} \textcolor{Sepia}{\selectlanguage{english}to square off with an axe} \textcolor{PineGreen}{\selectlanguage{french}équarrir à la hache}  

\lhead{\firstmark}
\rhead{\botmark}

\subsection{\hspace{-0.5cm} {\Large \textcolor{darkblue}{\textbf{\ipa{tɕʰo˩˧}}}}\hspace{0.5cm}[\kern2pt{\textcolor{darkblue}{\textbf{\ipa{tɕʰo˩˥}}}}\kern2pt]} \hypertarget{ts£\string_ho\string_B\string_M1}{}
\markboth{\textcolor{darkblue}{\textbf{\ipa{tɕʰo˩˧}}}}{}
\textcolor{teal}{\zh{名词}} \hspace{4pt} \zh{声调类:} LM.
\zh{勺子、瓢。} \textcolor{Sepia}{\selectlanguage{english}Ladle, scoop used for water.} \textcolor{PineGreen}{\selectlanguage{french}Louche (de grande taille: grosse louche pour puiser l'eau; tient plus d'un litre).}  \zh{量词}: \textcolor{darkblue}{\textbf{\ipa{nɑ˧}}} 
\lhead{\firstmark}
\rhead{\botmark}

\subsection{\hspace{-0.5cm} {\Large \textcolor{darkblue}{\textbf{\ipa{tɕʰɯ˥}}}}\hspace{0.5cm}[\kern2pt{\textcolor{darkblue}{\textbf{\ipa{tɕʰɯ˧˥}}}}\kern2pt]} \hypertarget{ts£\string_hM\string_T1}{}
\markboth{\textcolor{darkblue}{\textbf{\ipa{tɕʰɯ˥}}}}{}
\textcolor{teal}{\zh{动词}} \hspace{4pt} \zh{声调类:} H.
\zh{穿刺、 刺破。} \textcolor{Sepia}{\selectlanguage{english}To pierce (e.g. a cow's nose).} \textcolor{PineGreen}{\selectlanguage{french}Percer, transpercer.}  ¶ \textcolor{darkblue}{\textbf{\ipa{ʝi˧ ʈʂʰɯ˧-pʰo˩, | ɲi˧ tɕʰi˧-ze˩!}}} \zh{这头牛的鼻子被穿刺(为了安一个牛鼻圈)} \textcolor{Sepia}{\selectlanguage{english}This ox's nose was pierced (to put a ring)} \textcolor{PineGreen}{\selectlanguage{french}Ce bœuf, on lui a percé le museau (pour y placer un anneau)!}  

\lhead{\firstmark}
\rhead{\botmark}

\subsection{\hspace{-0.5cm} {\Large \textcolor{darkblue}{\textbf{\ipa{tɕʰɯ˧\textsubscript{a}}}} \textsubscript{1}}\hspace{0.5cm}[\kern2pt{\textcolor{darkblue}{\textbf{\ipa{tɕʰɯ˩˥}}}}\kern2pt]} \hypertarget{ts£\string_hM\string_Ma1}{}
\markboth{\textcolor{darkblue}{\textbf{\ipa{tɕʰɯ˧\textsubscript{a}}}} \textsubscript{1}}{}
\textcolor{teal}{\zh{动词}} \hspace{4pt} \zh{声调类:} M\textsubscript{a}.
\zh{举、抬(胳膊)。} \textcolor{Sepia}{\selectlanguage{english}To raise (one's arm).} \textcolor{PineGreen}{\selectlanguage{french}Lever (le bras…).}  ¶ \textcolor{darkblue}{\textbf{\ipa{lo˩qʰwɤ˥ | gɤ˩-tɕʰɯ˧}}} \zh{举手、抬胳膊} \textcolor{Sepia}{\selectlanguage{english}to raise one's arm} \textcolor{PineGreen}{\selectlanguage{french}lever le bras}  
 ¶ \textcolor{darkblue}{\textbf{\ipa{kʰɯ˧tsʰɤ˧˥ | gɤ˩-tɕʰɯ˧}}} \zh{抬脚} \textcolor{Sepia}{\selectlanguage{english}to raise one's leg} \textcolor{PineGreen}{\selectlanguage{french}lever la jambe}  
 ¶ \textcolor{darkblue}{\textbf{\ipa{gɤ˩-mɤ˧-tɕʰɯ˧}}} \zh{不抬起来} \textcolor{Sepia}{\selectlanguage{english}not to raise} \textcolor{PineGreen}{\selectlanguage{french}ne pas lever}  

\lhead{\firstmark}
\rhead{\botmark}

\subsection{\hspace{-0.5cm} {\Large \textcolor{darkblue}{\textbf{\ipa{tɕʰɯ˧\textsubscript{a}}}} \textsubscript{2}}\hspace{0.5cm}[\kern2pt{\textcolor{darkblue}{\textbf{\ipa{tɕʰɯ˥}}}}\kern2pt]} \hypertarget{ts£\string_hM\string_Ma2}{}
\markboth{\textcolor{darkblue}{\textbf{\ipa{tɕʰɯ˧\textsubscript{a}}}} \textsubscript{2}}{}
\textcolor{teal}{\zh{动词}} \hspace{4pt} \zh{声调类:} M\textsubscript{a}.
\ding{202} \zh{守护。} \textcolor{Sepia}{\selectlanguage{english}To guard, to keep guard.} \textcolor{PineGreen}{\selectlanguage{french}Garder, faire la garde.} \ding{203} \zh{居丧、守灵。} \textcolor{Sepia}{\selectlanguage{english}To keep a deathwatch, to sit with others at a funeral wake.} \textcolor{PineGreen}{\selectlanguage{french}Veiller un défunt, lors d'une veillée funèbre.}  ¶ \textcolor{darkblue}{\textbf{\ipa{hĩ˧ tɕʰɯ˧}}} \zh{同上:守灵} \textcolor{Sepia}{\selectlanguage{english}same meaning: to keep a deathwatch for a deceased person} \textcolor{PineGreen}{\selectlanguage{french}même sens: veiller un défunt}  

\lhead{\firstmark}
\rhead{\botmark}

\subsection{\hspace{-0.5cm} {\Large \textcolor{darkblue}{\textbf{\ipa{tɕʰɯ˧bo˧˥}}}}\hspace{0.5cm}[\kern2pt{\textcolor{darkblue}{\textbf{\ipa{tɕʰɯ˧bo˧}}}}\kern2pt]} \hypertarget{ts£\string_hM\string_Mbo\string_M\string_T1}{}
\markboth{\textcolor{darkblue}{\textbf{\ipa{tɕʰɯ˧bo˧˥}}}}{}
\textcolor{teal}{\zh{形容词}} \hspace{4pt} \zh{声调类:} MH\#.
\zh{凉快。} \textcolor{Sepia}{\selectlanguage{english}Fresh, cool.} \textcolor{PineGreen}{\selectlanguage{french}Frais.} 
\lhead{\firstmark}
\rhead{\botmark}

\subsection{\hspace{-0.5cm} {\Large \textcolor{darkblue}{\textbf{\ipa{tɕʰɯ˧lo\#˥}}}}\hspace{0.5cm}[\kern2pt{\textcolor{darkblue}{\textbf{\ipa{tɕʰɯ˩lo˩˥}}}}\kern2pt]} \hypertarget{ts£\string_hM\string_Mlo\#\string_T1}{}
\markboth{\textcolor{darkblue}{\textbf{\ipa{tɕʰɯ˧lo\#˥}}}}{}
\textcolor{teal}{\zh{名词}} \hspace{4pt} \zh{声调类:} \#H.
\zh{大盘子。} \textcolor{Sepia}{\selectlanguage{english}Large plate.} \textcolor{PineGreen}{\selectlanguage{french}Grande assiette.}  \zh{量词}: \textcolor{darkblue}{\textbf{\ipa{ɭɯ˧}}} 
\lhead{\firstmark}
\rhead{\botmark}

\subsection{\hspace{-0.5cm} {\Large \textcolor{darkblue}{\textbf{\ipa{tɕʰɯ˧si˩-dʑɤ˩pv̩˩}}}}\hspace{0.5cm}[\kern2pt{\textcolor{darkblue}{\textbf{\ipa{tɕʰɯ˧si˩dʑɤ˧pv̩˧}}}}\kern2pt]} \hypertarget{ts£\string_hM\string_Msi\string_B-dz£7\string_Bpv\string_=\string_B1}{}
\markboth{\textcolor{darkblue}{\textbf{\ipa{tɕʰɯ˧si˩-dʑɤ˩pv̩˩}}}}{}
\textcolor{teal}{\zh{名词}} \hspace{4pt} \zh{声调类:} L\#-.
\zh{妖怪。} \textcolor{Sepia}{\selectlanguage{english}Monster, demon.} \textcolor{PineGreen}{\selectlanguage{french}Monstre, revenant.}  ¶ \textcolor{darkblue}{\textbf{\ipa{no˧ | tɕʰɯ˧si˩-dʑɤ˩pv̩˩-ki˩ | le˧-hɯ˩-ɲi˩-ze˩!}}} \zh{你已经到妖怪的世界那边(就恳求你不要回来了)!(对鬼说的话)} \textcolor{Sepia}{\selectlanguage{english}You have gone away to the world of monstres (and should not come back to trouble the living)! (Speech addressed to a ghost that one beseeches should not come back)} \textcolor{PineGreen}{\selectlanguage{french}Tu es parti rejoindre les monstres! (propos tenus à un revenant qu'on enjoint de ne plus revenir hanter les vivants)}  

\lhead{\firstmark}
\rhead{\botmark}

\subsection{\hspace{-0.5cm} {\Large \textcolor{darkblue}{\textbf{\ipa{tɕʰɯ˧sɯ˥}}}}\hspace{0.5cm}[\kern2pt{\textcolor{darkblue}{\textbf{\ipa{tɕʰɯ˧sɯ˥}}}}\kern2pt]} \hypertarget{ts£\string_hM\string_MsM\string_T1}{}
\markboth{\textcolor{darkblue}{\textbf{\ipa{tɕʰɯ˧sɯ˥}}}}{}
\textcolor{teal}{\zh{形容词}} \hspace{4pt} \zh{声调类:} H\#.
\zh{悲哀、伤心。} \textcolor{Sepia}{\selectlanguage{english}Sad, grieved.} \textcolor{PineGreen}{\selectlanguage{french}Triste, dans l'affliction, plongé dans le chagrin.} 
\lhead{\firstmark}
\rhead{\botmark}

\subsection{\hspace{-0.5cm} {\Large \textcolor{darkblue}{\textbf{\ipa{tɕʰɯ˩\textsubscript{a}}}}}\hspace{0.5cm}[\kern2pt{\textcolor{darkblue}{\textbf{\ipa{tɕʰɯ˧˥}}}}\kern2pt]} \hypertarget{ts£\string_hM\string_Ba1}{}
\markboth{\textcolor{darkblue}{\textbf{\ipa{tɕʰɯ˩\textsubscript{a}}}}}{}
\textcolor{teal}{\zh{形容词}} \hspace{4pt} \zh{声调类:} L\textsubscript{a}.
\zh{甜。} \textcolor{Sepia}{\selectlanguage{english}Sweet.} \textcolor{PineGreen}{\selectlanguage{french}Sucré.} 
\lhead{\firstmark}
\rhead{\botmark}

\subsection{\hspace{-0.5cm} {\Large \textcolor{darkblue}{\textbf{\ipa{tɕʰɯ˩di˩}}}}\hspace{0.5cm}[\kern2pt{\textcolor{darkblue}{\textbf{\ipa{tɕʰɯ˧di˧˥}}}}\kern2pt]} \hypertarget{ts£\string_hM\string_Bdi\string_B1}{}
\markboth{\textcolor{darkblue}{\textbf{\ipa{tɕʰɯ˩di˩}}}}{}
\textcolor{teal}{\zh{动词}} \hspace{4pt} \zh{声调类:} L.
\zh{狩猎。} \textcolor{Sepia}{\selectlanguage{english}To hunt.} \textcolor{PineGreen}{\selectlanguage{french}Chasser.} \zh{~【参考】~} \textcolor{darkblue}{\textbf{\ipa{tɕʰɯ˩˥, di˧˥1}}} 
\lhead{\firstmark}
\rhead{\botmark}

\subsection{\hspace{-0.5cm} {\Large \textcolor{darkblue}{\textbf{\ipa{tɕʰɯ˩di˩kʰv̩˩}}}}\hspace{0.5cm}[\kern2pt{\textcolor{darkblue}{\textbf{\ipa{tɕʰɯ˩di˩kʰv̩˩˥}}}}\kern2pt]} \hypertarget{ts£\string_hM\string_Bdi\string_Bk\string_hv\string_=\string_B1}{}
\markboth{\textcolor{darkblue}{\textbf{\ipa{tɕʰɯ˩di˩kʰv̩˩}}}}{}
\textcolor{teal}{\zh{名词}} \hspace{4pt} \zh{声调类:} L.
\zh{猎狗。} \textcolor{Sepia}{\selectlanguage{english}Hunting dog, hound.} \textcolor{PineGreen}{\selectlanguage{french}Chien de chasse.}  ¶ \textcolor{darkblue}{\textbf{\ipa{tɕʰɯ˩di˩-kʰv̩˥mi˩}}} \zh{猎狗} \textcolor{Sepia}{\selectlanguage{english}same meaning} \textcolor{PineGreen}{\selectlanguage{french}même sens}  
 \zh{量词}: \textcolor{darkblue}{\textbf{\ipa{mi˩}}} 
\lhead{\firstmark}
\rhead{\botmark}

\subsection{\hspace{-0.5cm} {\Large \textcolor{darkblue}{\textbf{\ipa{tɕʰɯ˩mi\#˥}}}}\hspace{0.5cm}[\kern2pt{\textcolor{darkblue}{\textbf{\ipa{tɕʰɯ˧mi˥}}}}\kern2pt]} \hypertarget{ts£\string_hM\string_Bmi\#\string_T1}{}
\markboth{\textcolor{darkblue}{\textbf{\ipa{tɕʰɯ˩mi\#˥}}}}{}
\textcolor{teal}{\zh{名词}} \hspace{4pt} \zh{声调类:} LM+\#H / L.
\zh{母麂子。} \textcolor{Sepia}{\selectlanguage{english}Female muntjac.} \textcolor{PineGreen}{\selectlanguage{french}Muntjac femelle.}  \zh{量词}: \textcolor{darkblue}{\textbf{\ipa{mi˩}}} 
\lhead{\firstmark}
\rhead{\botmark}

\subsection{\hspace{-0.5cm} {\Large \textcolor{darkblue}{\textbf{\ipa{tɕʰɯ˩pʰv̩\#˥}}}}\hspace{0.5cm}[\kern2pt{\textcolor{darkblue}{\textbf{\ipa{tɕʰɯ˩pʰv̩˥}}}}\kern2pt]} \hypertarget{ts£\string_hM\string_Bp\string_hv\string_=\#\string_T1}{}
\markboth{\textcolor{darkblue}{\textbf{\ipa{tɕʰɯ˩pʰv̩\#˥}}}}{}
\textcolor{teal}{\zh{名词}} \hspace{4pt} \zh{声调类:} LM+\#H / L.
\zh{公麂子。} \textcolor{Sepia}{\selectlanguage{english}Male muntjac.} \textcolor{PineGreen}{\selectlanguage{french}Muntjac mâle.}  \zh{量词}: \textcolor{darkblue}{\textbf{\ipa{mi˩}}} 
\lhead{\firstmark}
\rhead{\botmark}

\subsection{\hspace{-0.5cm} {\Large \textcolor{darkblue}{\textbf{\ipa{tɕʰɯ˩-ʁo˩-tɕʰɯ˥!}}}}\hspace{0.5cm}[\kern2pt{\textcolor{darkblue}{\textbf{\ipa{xxxx non-correspondance entre le nombre de morphèmes et le nombre de tons de morphèmes}}}}\kern2pt]} \hypertarget{ts£\string_hM\string_B-Ro\string_B-ts£\string_hM\string_T!1}{}
\markboth{\textcolor{darkblue}{\textbf{\ipa{tɕʰɯ˩-ʁo˩-tɕʰɯ˥!}}}}{}
\textcolor{teal}{\zh{助词}} \hspace{4pt} \zh{声调类:} L+H\#.
\zh{旁边的人打嚏喷时说的祝愿话。} \textcolor{Sepia}{\selectlanguage{english}Bless you! (what one says when someone sneezes).} \textcolor{PineGreen}{\selectlanguage{french}A vos souhaits! (formule que l'on dit lorsque quelqu'un éternue).} 
\lhead{\firstmark}
\rhead{\botmark}

\subsection{\hspace{-0.5cm} {\Large \textcolor{darkblue}{\textbf{\ipa{tɕʰɯ˩\textasciitilde{}tɕʰɯ˧˥}}}}\hspace{0.5cm}[\kern2pt{\textcolor{darkblue}{\textbf{\ipa{tɕʰɯ˧tɕʰɯ˧˥}}}}\kern2pt]} \hypertarget{ts£\string_hM\string_B~ts£\string_hM\string_M\string_T1}{}
\markboth{\textcolor{darkblue}{\textbf{\ipa{tɕʰɯ˩\textasciitilde{}tɕʰɯ˧˥}}}}{}
\textcolor{teal}{\zh{动词}} \hspace{4pt} \zh{声调类:} MH.
\zh{吸吮。} \textcolor{Sepia}{\selectlanguage{english}To suck.} \textcolor{PineGreen}{\selectlanguage{french}Sucer.}  ¶ \textcolor{darkblue}{\textbf{\ipa{lo˩mi˧ tɕʰi˩\textasciitilde{}tɕʰi˩}}} \zh{吮拇指} \textcolor{Sepia}{\selectlanguage{english}to suck one's thumb} \textcolor{PineGreen}{\selectlanguage{french}sucer son pouce}  

\lhead{\firstmark}
\rhead{\botmark}

\subsection{\hspace{-0.5cm} {\Large \textcolor{darkblue}{\textbf{\ipa{tɕʰɯ˩zo\#˥}}}}\hspace{0.5cm}[\kern2pt{\textcolor{darkblue}{\textbf{\ipa{tɕʰɯ˩zo˥}}}}\kern2pt]} \hypertarget{ts£\string_hM\string_Bzo\#\string_T1}{}
\markboth{\textcolor{darkblue}{\textbf{\ipa{tɕʰɯ˩zo\#˥}}}}{}
\textcolor{teal}{\zh{名词}} \hspace{4pt} \zh{声调类:} LM+\#H / L.
\zh{麂子崽子。} \textcolor{Sepia}{\selectlanguage{english}Baby muntjac.} \textcolor{PineGreen}{\selectlanguage{french}Petit muntjac.}  \zh{量词}: \textcolor{darkblue}{\textbf{\ipa{ɭɯ˧}}} 
\lhead{\firstmark}
\rhead{\botmark}

\subsection{\hspace{-0.5cm} {\Large \textcolor{darkblue}{\textbf{\ipa{tɕʰɯ˧˥}}}}\hspace{0.5cm}[\kern2pt{\textcolor{darkblue}{\textbf{\ipa{tɕʰɯ˩˥}}}}\kern2pt]} \hypertarget{ts£\string_hM\string_M\string_T1}{}
\markboth{\textcolor{darkblue}{\textbf{\ipa{tɕʰɯ˧˥}}}}{}
\textcolor{teal}{\zh{名词}} \hspace{4pt} \zh{声调类:} MH.
\zh{漆。} \textcolor{Sepia}{\selectlanguage{english}Lacquer, paint.} \textcolor{PineGreen}{\selectlanguage{french}Peinture, laque.}  \zh{【借词】} \zh{漆}
 ¶ \textcolor{darkblue}{\textbf{\ipa{tɕʰɯ˧ jɤ˥-zo˩-ho˩!}}} \zh{该刷漆了!} \textcolor{Sepia}{\selectlanguage{english}It's time to paint (the room, the house...)! / We're going to have to paint (the room, the house...)!} \textcolor{PineGreen}{\selectlanguage{french}Il va falloir (re)peindre}  

\lhead{\firstmark}
\rhead{\botmark}

\subsection{\hspace{-0.5cm} {\Large \textcolor{darkblue}{\textbf{\ipa{tɕʰɯ˧˥}}} \textsubscript{1}}\hspace{0.5cm}[\kern2pt{\textcolor{darkblue}{\textbf{\ipa{tɕʰɯ˧˥}}}}\kern2pt]} \hypertarget{ts£\string_hM\string_M\string_T1}{}
\markboth{\textcolor{darkblue}{\textbf{\ipa{tɕʰɯ˧˥}}} \textsubscript{1}}{}
\textcolor{teal}{\zh{动词}} \hspace{4pt} \zh{声调类:} MH.
\zh{扔(垃圾)。} \textcolor{Sepia}{\selectlanguage{english}To throw away.} \textcolor{PineGreen}{\selectlanguage{french}Jeter, se débarrasser de (poubelles, détritus…); abandonner.}  ¶ \textcolor{darkblue}{\textbf{\ipa{ɖæ˩˥ | tʰi˧-tɕʰɯ˧˥}}} \zh{扔垃圾} \textcolor{Sepia}{\selectlanguage{english}to throw garbage} \textcolor{PineGreen}{\selectlanguage{french}jeter des détritus}  

\lhead{\firstmark}
\rhead{\botmark}

\subsection{\hspace{-0.5cm} {\Large \textcolor{darkblue}{\textbf{\ipa{tɕʰɯ˧˥}}} \textsubscript{2}}\hspace{0.5cm}[\kern2pt{\textcolor{darkblue}{\textbf{\ipa{tɕʰɯ˧˥}}}}\kern2pt]} \hypertarget{ts£\string_hM\string_M\string_T2}{}
\markboth{\textcolor{darkblue}{\textbf{\ipa{tɕʰɯ˧˥}}} \textsubscript{2}}{}
\textcolor{teal}{\zh{动词}} \hspace{4pt} \zh{声调类:} MH.
\zh{吐(吐口水)。} \textcolor{Sepia}{\selectlanguage{english}To spit.} \textcolor{PineGreen}{\selectlanguage{french}Cracher.} 
\lhead{\firstmark}
\rhead{\botmark}

\subsection{\hspace{-0.5cm} {\Large \textcolor{darkblue}{\textbf{\ipa{tɕʰɯ˧˥}}} \textsubscript{3}}\hspace{0.5cm}[\kern2pt{\textcolor{darkblue}{\textbf{\ipa{tɕʰɯ˧˥}}}}\kern2pt]} \hypertarget{ts£\string_hM\string_M\string_T3}{}
\markboth{\textcolor{darkblue}{\textbf{\ipa{tɕʰɯ˧˥}}} \textsubscript{3}}{}
\textcolor{teal}{\zh{动词}} \hspace{4pt} \zh{声调类:} MH.
\zh{丢失、弄丢。} \textcolor{Sepia}{\selectlanguage{english}To lose, to misplace.} \textcolor{PineGreen}{\selectlanguage{french}Perdre.}  ¶ \textcolor{darkblue}{\textbf{\ipa{le˧-tɕʰɯ˧-ze˥}}} \zh{丢了} \textcolor{Sepia}{\selectlanguage{english}\mytextsc{accomp} \string_ \mytextsc{pfv}} \textcolor{PineGreen}{\selectlanguage{french}\mytextsc{accomp} \string_ \mytextsc{pfv}}  
 ¶ \textcolor{darkblue}{\textbf{\ipa{le˧-tɕʰɯ˧-hɯ˥-ze˩!}}} \zh{丢掉了!} \textcolor{Sepia}{\selectlanguage{english}It's lost!} \textcolor{PineGreen}{\selectlanguage{french}c'est perdu!}  

\lhead{\firstmark}
\rhead{\botmark}

\subsection{\hspace{-0.5cm} {\Large \textcolor{darkblue}{\textbf{\ipa{tɕʰɯ˧˥}}} \textsubscript{4}}\hspace{0.5cm}[\kern2pt{\textcolor{darkblue}{\textbf{\ipa{tɕʰɯ˧˥}}}}\kern2pt]} \hypertarget{ts£\string_hM\string_M\string_T4}{}
\markboth{\textcolor{darkblue}{\textbf{\ipa{tɕʰɯ˧˥}}} \textsubscript{4}}{}
\textcolor{teal}{\zh{形容词}} \hspace{4pt} \zh{声调类:} MH.
\zh{担心。} \textcolor{Sepia}{\selectlanguage{english}Anxious, worried.} \textcolor{PineGreen}{\selectlanguage{french}Inquiet, angoissé, tourmenté, oppressé.}  ¶ \textcolor{darkblue}{\textbf{\ipa{nv̩˩mi˩ tɕʰɯ˥}}} \zh{担心} \textcolor{Sepia}{\selectlanguage{english}anxious, worried} \textcolor{PineGreen}{\selectlanguage{french}inquiet}  

\lhead{\firstmark}
\rhead{\botmark}

\subsection{\hspace{-0.5cm} {\Large \textcolor{darkblue}{\textbf{\ipa{tɕʰɯ˧˥}}} \textsubscript{5}}\hspace{0.5cm}[\kern2pt{\textcolor{darkblue}{\textbf{\ipa{tɕʰɯ˧˥}}}}\kern2pt]} \hypertarget{ts£\string_hM\string_M\string_T5}{}
\markboth{\textcolor{darkblue}{\textbf{\ipa{tɕʰɯ˧˥}}} \textsubscript{5}}{}
\textcolor{teal}{\zh{形容词}} \hspace{4pt} \zh{声调类:} MH.
\textit{\zh{古语}} [\zh{古语}] \zh{舒服。} \textcolor{Sepia}{\selectlanguage{english}At ease, comfortable.} \textcolor{PineGreen}{\selectlanguage{french}Observé seulement en tournure négative: ne pas avoir (de quoi vivre); être démuni. On peut imaginer comme sens ancien “bien doté, à l'aise”.}  ¶ \textcolor{darkblue}{\textbf{\ipa{mɤ˧-tɕʰɯ˧-bi˥ / mɤ˧-tɕʰɯ˧˥ |-bi˩}}} \zh{虽然很贫穷,……} \textcolor{Sepia}{\selectlanguage{english}even if one is in need, ...} \textcolor{PineGreen}{\selectlanguage{french}même si on est dans le besoin/quoi qu'on soit dans le besoin, ...}  
 ¶ \textcolor{darkblue}{\textbf{\ipa{mɤ˧-dʑo˧ mɤ˧-tɕʰɯ˧-ɻ̍˧-bi˥, | ɖwæ˩ mɤ˧-zo˧!}}} \zh{虽然穷,莫担心!(因为菩萨会救好人)} \textcolor{Sepia}{\selectlanguage{english}Even if one is in need, one should not worry! (because the Gods will do something to save us)} \textcolor{PineGreen}{\selectlanguage{french}“Même si on est sans rien, dans le besoin, il ne faut pas s'inquiéter!” (car le Ciel vient en aide aux gens qui font de leur mieux)}  
 ¶ \textcolor{darkblue}{\textbf{\ipa{hĩ˧ ʈʂʰɯ˧-v̩˧-dʑo˩, | ɖwæ˧˥ | mɤ˧-tɕʰɯ˧˥!}}} \zh{这个人,真的很穷!} \textcolor{Sepia}{\selectlanguage{english}This person is really in need!} \textcolor{PineGreen}{\selectlanguage{french}Il est vraiment dans le besoin/nécessiteux!}  

\lhead{\firstmark}
\rhead{\botmark}

\subsection{\hspace{-0.5cm} {\Large \textcolor{darkblue}{\textbf{\ipa{tɕʰɯ˩˥}}}}\hspace{0.5cm}[\kern2pt{\textcolor{darkblue}{\textbf{\ipa{tɕʰɯ˥}}}}\kern2pt]} \hypertarget{ts£\string_hM\string_B\string_T1}{}
\markboth{\textcolor{darkblue}{\textbf{\ipa{tɕʰɯ˩˥}}}}{}
\textcolor{teal}{\zh{名词}} \hspace{4pt} \zh{声调类:} LH.
\zh{麂子。} \textcolor{Sepia}{\selectlanguage{english}Muntjac.} \textcolor{PineGreen}{\selectlanguage{french}Muntjac.}  ¶ \textcolor{darkblue}{\textbf{\ipa{tɕʰɯ˩ hwæ˧-ze˩}}} \zh{买麂子} \textcolor{Sepia}{\selectlanguage{english}...has bought (a/the) muntjac} \textcolor{PineGreen}{\selectlanguage{french}...a acheté un muntjac}  
 ¶ \textcolor{darkblue}{\textbf{\ipa{tɕʰɯ˩ dzɯ˩-ze˥}}} \zh{吃了麂子} \textcolor{Sepia}{\selectlanguage{english}...has eaten muntjac} \textcolor{PineGreen}{\selectlanguage{french}...a mangé un muntjac}  
 \zh{量词}: \textcolor{darkblue}{\textbf{\ipa{pʰo˧˥}}} 
\lhead{\firstmark}
\rhead{\botmark}

\newpage
\section*{\centering- \textcolor{darkblue}{\textbf{\ipa{ts}}} -}
\subsection{\hspace{-0.5cm} {\Large \textcolor{darkblue}{\textbf{\ipa{tsɑ˧}}}}\hspace{0.5cm}[\kern2pt{\textcolor{darkblue}{\textbf{\ipa{tsɑ˩˥}}}}\kern2pt]} \hypertarget{tsA\string_M1}{}
\markboth{\textcolor{darkblue}{\textbf{\ipa{tsɑ˧}}}}{}
\textcolor{teal}{\zh{形容词}} \hspace{4pt} \zh{声调类:} M.
\zh{忙。} \textcolor{Sepia}{\selectlanguage{english}Busy.} \textcolor{PineGreen}{\selectlanguage{french}Occupé, affairé, pressé.}  ¶ \textcolor{darkblue}{\textbf{\ipa{ɖwæ˧˥ | tsɑ˧}}} \zh{很忙} \textcolor{Sepia}{\selectlanguage{english}\mytextsc{intensive}.very: very busy} \textcolor{PineGreen}{\selectlanguage{french}\mytextsc{intensif}.très: très occupé}  
 ¶ \textcolor{darkblue}{\textbf{\ipa{tsɑ˧ | ʐwæ˩˥}}} \zh{非常忙} \textcolor{Sepia}{\selectlanguage{english}extremely busy} \textcolor{PineGreen}{\selectlanguage{french}extrêmement occupé}  

\lhead{\firstmark}
\rhead{\botmark}

\subsection{\hspace{-0.5cm} {\Large \textcolor{darkblue}{\textbf{\ipa{tsɑ˧bɤ˧}}}}\hspace{0.5cm}[\kern2pt{\textcolor{darkblue}{\textbf{\ipa{tsɑ˧bɤ˧˥}}}}\kern2pt]} \hypertarget{tsA\string_Mb7\string_M1}{}
\markboth{\textcolor{darkblue}{\textbf{\ipa{tsɑ˧bɤ˧}}}}{}
\textcolor{teal}{\zh{名词}} \hspace{4pt} \zh{声调类:} M.
\zh{糌粑、面粉、粉、粉末。} \textcolor{Sepia}{\selectlanguage{english}Powder; flour.} \textcolor{PineGreen}{\selectlanguage{french}Poudre; farine.}  \zh{【借词】}\zh{藏语} rtsam pa
 ¶ \textcolor{darkblue}{\textbf{\ipa{qʰɑ˧dze˧-tsɑ˩bɤ˩}}} \zh{玉米粉} \textcolor{Sepia}{\selectlanguage{english}sweetcorn flour} \textcolor{PineGreen}{\selectlanguage{french}farine de maïs}  
 ¶ \textcolor{darkblue}{\textbf{\ipa{dze˧ɭɯ˧-tsɑ˩bɤ˩}}} \zh{小麦面} \textcolor{Sepia}{\selectlanguage{english}wheat flour} \textcolor{PineGreen}{\selectlanguage{french}farine de blé}  
 ¶ \textcolor{darkblue}{\textbf{\ipa{lv̩˧mi˧-tsɑ˩bɤ˩}}} \zh{石头粉、被磨成粉的石头} \textcolor{Sepia}{\selectlanguage{english}stone powder, powdered stone} \textcolor{PineGreen}{\selectlanguage{french}poudre de pierre, pierre pulvérisée}  
 ¶ \textcolor{darkblue}{\textbf{\ipa{tsʰi˧zi˧-tsɑ˧bɤ˥}}} \zh{青稞面粉} \textcolor{Sepia}{\selectlanguage{english}highland barley flour} \textcolor{PineGreen}{\selectlanguage{french}farine d'orge}  
 ¶ \textcolor{darkblue}{\textbf{\ipa{mv̩˩zɯ˩-tsɑ˩bɤ˥}}} \zh{燕麦面粉} \textcolor{Sepia}{\selectlanguage{english}oatmeal flour} \textcolor{PineGreen}{\selectlanguage{french}farine d'avoine}  
 ¶ \textcolor{darkblue}{\textbf{\ipa{jɤ˩jo˧-tsɑ˧bɤ˥}}} \zh{洋芋面粉} \textcolor{Sepia}{\selectlanguage{english}potato flour (elicited combination)} \textcolor{PineGreen}{\selectlanguage{french}farine de pommes de terre}  
 ¶ \textcolor{darkblue}{\textbf{\ipa{nv̩˩ɭɯ˧-tsɑ˩bɤ˩}}} \zh{黄豆面粉} \textcolor{Sepia}{\selectlanguage{english}soy flour} \textcolor{PineGreen}{\selectlanguage{french}farine de soja}  
 ¶ \textcolor{darkblue}{\textbf{\ipa{læ˧tsɯ˥-tsɑ˩bɤ˩}}} \zh{辣椒粉} \textcolor{Sepia}{\selectlanguage{english}chili powder} \textcolor{PineGreen}{\selectlanguage{french}piment en poudre}  
 ¶ \textcolor{darkblue}{\textbf{\ipa{ʈʂʰæ˧ɣɯ˧-tsɑ˧bɤ˥}}} \zh{药粉,粉状药品。如:“云南白药”消毒粉。} \textcolor{Sepia}{\selectlanguage{english}Medicine powder, medicine in powder form. For instance: the disinfectant commonly used in Yongning at the time of fieldwork, of the brand \zh{云南白药}} \textcolor{PineGreen}{\selectlanguage{french}Médicament en poudre. (Exemple: le désinfectant en poudre actuellement utilisé, de la marque \zh{云南白药}.)}  
 ¶ \textcolor{darkblue}{\textbf{\ipa{ʂæ˩ɻ̃˩-tsɑ˩bɤ˥}}} \zh{骨头粉} \textcolor{Sepia}{\selectlanguage{english}bone powder} \textcolor{PineGreen}{\selectlanguage{french}poudre d'os}  
 ¶ \textcolor{darkblue}{\textbf{\ipa{jɤ˧-tsɑ˧bɤ˧}}} \zh{烟草粉} \textcolor{Sepia}{\selectlanguage{english}tobacco powder} \textcolor{PineGreen}{\selectlanguage{french}tabac en poudre}  
 ¶ \textcolor{darkblue}{\textbf{\ipa{jɤ˧ɻ̃˧-tsɑ˧bɤ˥}}} \zh{烟草粉} \textcolor{Sepia}{\selectlanguage{english}tobacco powder} \textcolor{PineGreen}{\selectlanguage{french}poudre de tabac, tabac en poudre}  
 ¶ \textcolor{darkblue}{\textbf{\ipa{dze˩-tsɑ˩bɤ˥}}} \zh{花椒粉} \textcolor{Sepia}{\selectlanguage{english}Szechuan pepper powder} \textcolor{PineGreen}{\selectlanguage{french}xanthoxyle en poudre}  
 ¶ \textcolor{darkblue}{\textbf{\ipa{dze˧-tsɑ˧bɤ˥}}} \zh{砂糖} \textcolor{Sepia}{\selectlanguage{english}powdered sugar, granulated sugar} \textcolor{PineGreen}{\selectlanguage{french}sucre en poudre}  
 ¶ \textcolor{darkblue}{\textbf{\ipa{tsɑ˧bɤ˧ mɤ˩}}} \textcolor{Sepia}{\selectlanguage{english}to eat dry flour (made of grilled barley)} \textcolor{PineGreen}{\selectlanguage{french}manger du tsamba sec}  
 ¶ \textcolor{darkblue}{\textbf{\ipa{tsɑ˧bɤ˧ gv̩˩}}} \zh{炒糌粑,制作糌粑} \textcolor{Sepia}{\selectlanguage{english}to cook tsamba (grilled flour)} \textcolor{PineGreen}{\selectlanguage{french}préparer du tsamba/de la farine grillée}  

\lhead{\firstmark}
\rhead{\botmark}

\subsection{\hspace{-0.5cm} {\Large \textcolor{darkblue}{\textbf{\ipa{tsɑ˧ʐo˩}}}}\hspace{0.5cm}[\kern2pt{\textcolor{darkblue}{\textbf{\ipa{tsɑ˧ʐo˩}}}}\kern2pt]} \hypertarget{tsA\string_Mz`o\string_B1}{}
\markboth{\textcolor{darkblue}{\textbf{\ipa{tsɑ˧ʐo˩}}}}{}
\textcolor{teal}{\zh{形容词}} \hspace{4pt} \zh{声调类:} L\#.
\zh{勤快。} \textcolor{Sepia}{\selectlanguage{english}Diligent, conscientious.} \textcolor{PineGreen}{\selectlanguage{french}Zélé, assidu.} 
\lhead{\firstmark}
\rhead{\botmark}

\subsection{\hspace{-0.5cm} {\Large \textcolor{darkblue}{\textbf{\ipa{tsɑ˩}}}}\hspace{0.5cm}[\kern2pt{\textcolor{darkblue}{\textbf{\ipa{tsɑ˥}}}}\kern2pt]} \hypertarget{tsA\string_B1}{}
\markboth{\textcolor{darkblue}{\textbf{\ipa{tsɑ˩}}}}{}
\textcolor{teal}{\zh{动词}} \hspace{4pt} \zh{声调类:} L.
\zh{眨眼。} \textcolor{Sepia}{\selectlanguage{english}To wink (as a discreet sign of mutual understanding).} \textcolor{PineGreen}{\selectlanguage{french}Faire un clin d'oeil (discret signe d'intelligence).}  ¶ \textcolor{darkblue}{\textbf{\ipa{ʈʂʰɯ˧ | njɤ˩ɭɯ˧ tsɑ˩\textasciitilde{}tsɑ˩-dʑo˩!}}} \zh{\mytextsc{重叠:他在眨眨眼!}} \textcolor{Sepia}{\selectlanguage{english}\mytextsc{red}: (S)he is winking!} \textcolor{PineGreen}{\selectlanguage{french}\mytextsc{red}: Elle/il est en train de faire un clin d'oeil!}  
 ¶ \textcolor{darkblue}{\textbf{\ipa{ʈʂʰɯ˧ | njɤ˩ɭɯ˧ tsɑ˩-dʑo˩!}}} \zh{他在眨眼!} \textcolor{Sepia}{\selectlanguage{english}(S)he is winking!} \textcolor{PineGreen}{\selectlanguage{french}Elle/il est en train de faire un clin d'oeil!}  
 ¶ \textcolor{darkblue}{\textbf{\ipa{tsɑ˩\textasciitilde{}tsɑ˧˥}}} \zh{\mytextsc{重叠}} \textcolor{Sepia}{\selectlanguage{english}\mytextsc{red}} \textcolor{PineGreen}{\selectlanguage{french}\mytextsc{red}}  
 ¶ \textcolor{darkblue}{\textbf{\ipa{mɤ˧-tsɑ˩\textasciitilde{}tsɑ˩}}} \zh{不眨眼} \textcolor{Sepia}{\selectlanguage{english}\mytextsc{neg} \mytextsc{red}} \textcolor{PineGreen}{\selectlanguage{french}\mytextsc{neg} \mytextsc{red}}  
\zh{~【参考】~} \hyperlink{}{\textcolor{darkblue}{\textbf{\ipa{tsɯ˩pʰɤ˩}}}} 
\lhead{\firstmark}
\rhead{\botmark}

\subsection{\hspace{-0.5cm} {\Large \textcolor{darkblue}{\textbf{\ipa{tsɑ˩tɕi˩}}}}\hspace{0.5cm}[\kern2pt{\textcolor{darkblue}{\textbf{\ipa{tsɑ˩tɕi˩˥}}}}\kern2pt]} \hypertarget{tsA\string_Bts£i\string_B1}{}
\markboth{\textcolor{darkblue}{\textbf{\ipa{tsɑ˩tɕi˩}}}}{}
\textcolor{teal}{\zh{名词}} \hspace{4pt} \zh{声调类:} L.
\zh{杂菌(汉语借词)。} \textcolor{Sepia}{\selectlanguage{english}Various mushrooms, mixed mushrooms.} \textcolor{PineGreen}{\selectlanguage{french}Champignons divers, champignons variés.}  \zh{【借词】} \zh{杂菌}
 \zh{量词}: \textcolor{darkblue}{\textbf{\ipa{ɭɯ˧}}} \textcolor{darkblue}{\textbf{\ipa{mo˧˥}}} 
\lhead{\firstmark}
\rhead{\botmark}

\subsection{\hspace{-0.5cm} {\Large \textcolor{darkblue}{\textbf{\ipa{tsɑ˧˥}}} \textsubscript{1}}\hspace{0.5cm}[\kern2pt{\textcolor{darkblue}{\textbf{\ipa{tsɑ˥}}}}\kern2pt]} \hypertarget{tsA\string_M\string_T1}{}
\markboth{\textcolor{darkblue}{\textbf{\ipa{tsɑ˧˥}}} \textsubscript{1}}{}
\textcolor{teal}{\zh{动词}} \hspace{4pt} \zh{声调类:} MH.
\ding{202} \zh{打碎(坷拉),踢(一脚)。} \textcolor{Sepia}{\selectlanguage{english}To kick, to smash (clods of earth).} \textcolor{PineGreen}{\selectlanguage{french}Donner un coup de pied; briser (les mottes de terre, après le labour, avec une bêche, ou une masse en bois).}  ¶ \textcolor{darkblue}{\textbf{\ipa{le˧-tsɑ˧-ze˥}}} \zh{\mytextsc{accomp}+\mytextsc{pfv}} \textcolor{Sepia}{\selectlanguage{english}\mytextsc{accomp}+\mytextsc{pfv}} \textcolor{PineGreen}{\selectlanguage{french}\mytextsc{accomp}+\mytextsc{pfv}}  
 ¶ \textcolor{darkblue}{\textbf{\ipa{ʈʂe˧ tsɑ˩}}} \zh{打碎土坷垃} \textcolor{Sepia}{\selectlanguage{english}to smash clods of earth, after plowing (with a hand instrument, such as a hoe)} \textcolor{PineGreen}{\selectlanguage{french}briser les mottes de terre après le labour (avec un instrument manuel: houe, bêche)}  
 ¶ \textcolor{darkblue}{\textbf{\ipa{ɖɯ˧-tsɑ˧ tʰi˥-tsɑ˩}}} \zh{踢了又踢} \textcolor{Sepia}{\selectlanguage{english}to kick repeatedly, to give one kick after the other} \textcolor{PineGreen}{\selectlanguage{french}donner une succession de coups de pied}  
\ding{203} \zh{划(船)。} \textcolor{Sepia}{\selectlanguage{english}To row (a boat).} \textcolor{PineGreen}{\selectlanguage{french}Ramer (=geste comparable à celui de briser les mottes: geste répétitif, exerçant la force coup après coup, sur l'eau, comme on le ferait sur des mottes de terre).}  ¶ \textcolor{darkblue}{\textbf{\ipa{tsɑ˧-hɯ˥-tsɑ˩-ɻ̍˩}}} \zh{用力地划船、一直划船} \textcolor{Sepia}{\selectlanguage{english}to row in a sustained way, to row with great vigour} \textcolor{PineGreen}{\selectlanguage{french}ramer de façon soutenue}  

\lhead{\firstmark}
\rhead{\botmark}

\subsection{\hspace{-0.5cm} {\Large \textcolor{darkblue}{\textbf{\ipa{tsɑ˧˥}}} \textsubscript{2}}\hspace{0.5cm}[\kern2pt{\textcolor{darkblue}{\textbf{\ipa{tsɑ˧˥}}}}\kern2pt]} \hypertarget{tsA\string_M\string_T2}{}
\markboth{\textcolor{darkblue}{\textbf{\ipa{tsɑ˧˥}}} \textsubscript{2}}{}
\textcolor{teal}{\zh{动词}} \hspace{4pt} \zh{声调类:} MH.
\zh{放置、放下。} \textcolor{Sepia}{\selectlanguage{english}To lay (up/down), to place.} \textcolor{PineGreen}{\selectlanguage{french}Déposer, poser.}  ¶ \textcolor{darkblue}{\textbf{\ipa{mv̩˩tɕo˧ tsɑ˧˥}}} \zh{放下、放在地上} \textcolor{Sepia}{\selectlanguage{english}to lay down, to put down on the floor} \textcolor{PineGreen}{\selectlanguage{french}poser à terre}  

\lhead{\firstmark}
\rhead{\botmark}

\subsection{\hspace{-0.5cm} {\Large \textcolor{darkblue}{\textbf{\ipa{‑tsæ˧}}}}\hspace{0.5cm}[\kern2pt{\textcolor{darkblue}{\textbf{\ipa{tsæ˥}}}}\kern2pt]} \hypertarget{‑ts\{\string_M1}{}
\markboth{\textcolor{darkblue}{\textbf{\ipa{‑tsæ˧}}}}{}
\textcolor{teal}{\zh{后缀}} \hspace{4pt} \zh{声调类:} M.
\zh{\mytextsc{使动:让。}} \textcolor{Sepia}{\selectlanguage{english}Causative.} \textcolor{PineGreen}{\selectlanguage{french}Causatif.}  ¶ \textcolor{darkblue}{\textbf{\ipa{tʰi˧-dzɯ˥-kʰɯ˩-tsæ˩-ɲi˩!}}} \zh{必须让她吃!(情景:一个小女孩拒绝吃饭,家人就说这句。)} \textcolor{Sepia}{\selectlanguage{english}(We) have to get her to eat! (Context: comment made by a family member when a young child refused to have a meal)} \textcolor{PineGreen}{\selectlanguage{french}il faut l'obliger à manger/il faut la faire manger! (Commentaire d'un membre de la famille au sujet d'une petite fille qui refuse un repas)}  
 ¶ \textcolor{darkblue}{\textbf{\ipa{tʰi˧-ʐwɤ˩-kʰɯ˩-tsæ˩-ɲi˩!}}} \zh{必须让他说!(在以上例子的基础上编的句子)} \textcolor{Sepia}{\selectlanguage{english}(We) have to get (him/her) to talk! (Variant based on the preceding example)} \textcolor{PineGreen}{\selectlanguage{french}il faut le faire parler/il faut l'obliger à parler! (Variante créée par analogie avec l'exemple précédent)}  

\lhead{\firstmark}
\rhead{\botmark}

\subsection{\hspace{-0.5cm} {\Large \textcolor{darkblue}{\textbf{\ipa{tsæ˧qæ˥}}}}\hspace{0.5cm}[\kern2pt{\textcolor{darkblue}{\textbf{\ipa{tsæ˧qæ˥}}}}\kern2pt]} \hypertarget{ts\{\string_Mq\{\string_T1}{}
\markboth{\textcolor{darkblue}{\textbf{\ipa{tsæ˧qæ˥}}}}{}
\textcolor{teal}{\zh{名词}} \hspace{4pt} \zh{声调类:} H\#.
\ding{202} \zh{钩子。} \textcolor{Sepia}{\selectlanguage{english}Hook.} \textcolor{PineGreen}{\selectlanguage{french}Crochet.}  \zh{量词}: \textcolor{darkblue}{\textbf{\ipa{nɑ˧}}} \textcolor{darkblue}{\textbf{\ipa{ɭɯ˧}}} \ding{203} \zh{撞针。} \textcolor{Sepia}{\selectlanguage{english}Firing pin (of a gun).} \textcolor{PineGreen}{\selectlanguage{french}Percuteur (de fusil).} 
\lhead{\firstmark}
\rhead{\botmark}

\subsection{\hspace{-0.5cm} {\Large \textcolor{darkblue}{\textbf{\ipa{tse˧bæ˥}}}}\hspace{0.5cm}[\kern2pt{\textcolor{darkblue}{\textbf{\ipa{tse˧bæ˥}}}}\kern2pt]} \hypertarget{tse\string_Mb\{\string_T1}{}
\markboth{\textcolor{darkblue}{\textbf{\ipa{tse˧bæ˥}}}}{}
\textcolor{teal}{\zh{名词}} \hspace{4pt} \zh{声调类:} H\#.
\zh{火绒。} \textcolor{Sepia}{\selectlanguage{english}Tinder.} \textcolor{PineGreen}{\selectlanguage{french}Amadou.}  \zh{量词}: \textcolor{darkblue}{\textbf{\ipa{kʰɯ˩}}} 
\lhead{\firstmark}
\rhead{\botmark}

\subsection{\hspace{-0.5cm} {\Large \textcolor{darkblue}{\textbf{\ipa{tse˧bo\#˥}}}}\hspace{0.5cm}[\kern2pt{\textcolor{darkblue}{\textbf{\ipa{tse˧bo˧}}}}\kern2pt]} \hypertarget{tse\string_Mbo\#\string_T1}{}
\markboth{\textcolor{darkblue}{\textbf{\ipa{tse˧bo\#˥}}}}{}
\textcolor{teal}{\zh{名词}} \hspace{4pt} \zh{声调类:} \#H.
\zh{铃铛。} \textcolor{Sepia}{\selectlanguage{english}Small bell.} \textcolor{PineGreen}{\selectlanguage{french}Clochette (portée par le bétail: chevaux, parfois chiens).}  \zh{量词}: \textcolor{darkblue}{\textbf{\ipa{ɭɯ˧}}} 
\lhead{\firstmark}
\rhead{\botmark}

\subsection{\hspace{-0.5cm} {\Large \textcolor{darkblue}{\textbf{\ipa{tse˧di\#˥}}}}\hspace{0.5cm}[\kern2pt{\textcolor{darkblue}{\textbf{\ipa{tse˧di˧}}}}\kern2pt]} \hypertarget{tse\string_Mdi\#\string_T1}{}
\markboth{\textcolor{darkblue}{\textbf{\ipa{tse˧di\#˥}}}}{}
\textcolor{teal}{\zh{名词}} \hspace{4pt} \zh{声调类:} \#H.
\zh{檀香木、檀香、檀木。} \textcolor{Sepia}{\selectlanguage{english}Sandalwood, sandlewood.} \textcolor{PineGreen}{\selectlanguage{french}Bois de santal arbre à épice, arbre à encens.}  ¶ \textcolor{darkblue}{\textbf{\ipa{tse˧di˧-si\#˥}}} \zh{同上} \textcolor{Sepia}{\selectlanguage{english}same meaning} \textcolor{PineGreen}{\selectlanguage{french}même sens}  

\lhead{\firstmark}
\rhead{\botmark}

\subsection{\hspace{-0.5cm} {\Large \textcolor{darkblue}{\textbf{\ipa{tse˧kʰo˩}}}}\hspace{0.5cm}[\kern2pt{\textcolor{darkblue}{\textbf{\ipa{tse˧kʰo˩}}}}\kern2pt]} \hypertarget{tse\string_Mk\string_ho\string_B1}{}
\markboth{\textcolor{darkblue}{\textbf{\ipa{tse˧kʰo˩}}}}{}
\textcolor{teal}{\zh{名词}} \hspace{4pt} \zh{声调类:} L\#.
\zh{佛龛。} \textcolor{Sepia}{\selectlanguage{english}Sanctuary (small sanctuary on the mountain).} \textcolor{PineGreen}{\selectlanguage{french}Sanctuaire (petit sanctuaire sur la montagne; n'est pas habitable).}  \zh{量词}: \textcolor{darkblue}{\textbf{\ipa{ɭɯ˧}}} 
\lhead{\firstmark}
\rhead{\botmark}

\subsection{\hspace{-0.5cm} {\Large \textcolor{darkblue}{\textbf{\ipa{tse˧lv̩˥}}}}\hspace{0.5cm}[\kern2pt{\textcolor{darkblue}{\textbf{\ipa{tse˧lv̩˥}}}}\kern2pt]} \hypertarget{tse\string_Mlv\string_=\string_T1}{}
\markboth{\textcolor{darkblue}{\textbf{\ipa{tse˧lv̩˥}}}}{}
\textcolor{teal}{\zh{名词}} \hspace{4pt} \zh{声调类:} H\#.
\zh{燧石。} \textcolor{Sepia}{\selectlanguage{english}Flint.} \textcolor{PineGreen}{\selectlanguage{french}Silex.}  \zh{量词}: \textcolor{darkblue}{\textbf{\ipa{ɭɯ˧}}} 
\lhead{\firstmark}
\rhead{\botmark}

\subsection{\hspace{-0.5cm} {\Large \textcolor{darkblue}{\textbf{\ipa{tse˧mi˥}}}}\hspace{0.5cm}[\kern2pt{\textcolor{darkblue}{\textbf{\ipa{tse˧mi˥}}}}\kern2pt]} \hypertarget{tse\string_Mmi\string_T1}{}
\markboth{\textcolor{darkblue}{\textbf{\ipa{tse˧mi˥}}}}{}
\textcolor{teal}{\zh{名词}} \hspace{4pt} \zh{声调类:} H\#.
\zh{火镰。} \textcolor{Sepia}{\selectlanguage{english}Lighter.} \textcolor{PineGreen}{\selectlanguage{french}Briquet.}  \zh{量词}: \textcolor{darkblue}{\textbf{\ipa{nɑ˧}}} 
\lhead{\firstmark}
\rhead{\botmark}

\subsection{\hspace{-0.5cm} {\Large \textcolor{darkblue}{\textbf{\ipa{tse˧mi˥-dʑɯ˩ʁo˩}}}}\hspace{0.5cm}[\kern2pt{\textcolor{darkblue}{\textbf{\ipa{tse˧mi˥dʑɯ˩ʁo˩}}}}\kern2pt]} \hypertarget{tse\string_Mmi\string_T-dz£M\string_BRo\string_B1}{}
\markboth{\textcolor{darkblue}{\textbf{\ipa{tse˧mi˥-dʑɯ˩ʁo˩}}}}{}
\textcolor{teal}{\zh{名词}} \hspace{4pt} \zh{声调类:} H\#-L.
\zh{温泉乡的主要村落。} \textcolor{Sepia}{\selectlanguage{english}The village of Wenquan, in the plain of Yongning, where hot springs are located, hence the Chinese name Wenquan, 'hot springs'.} \textcolor{PineGreen}{\selectlanguage{french}Le village de Wenquan (possède des sources chaudes).} 
\lhead{\firstmark}
\rhead{\botmark}

\subsection{\hspace{-0.5cm} {\Large \textcolor{darkblue}{\textbf{\ipa{tse˩\textsubscript{a}}}} \textsubscript{1}}\hspace{0.5cm}[\kern2pt{\textcolor{darkblue}{\textbf{\ipa{tse˩˥}}}}\kern2pt]} \hypertarget{tse\string_Ba1}{}
\markboth{\textcolor{darkblue}{\textbf{\ipa{tse˩\textsubscript{a}}}} \textsubscript{1}}{}
\textcolor{teal}{\zh{动词}} \hspace{4pt} \zh{声调类:} L\textsubscript{a}.
\zh{追赶。} \textcolor{Sepia}{\selectlanguage{english}To chase after; to pursue.} \textcolor{PineGreen}{\selectlanguage{french}Suivre à la trace, poursuivre, pister.}  ¶ \textcolor{darkblue}{\textbf{\ipa{hĩ˧ tse˥}}} \zh{追赶某人} \textcolor{Sepia}{\selectlanguage{english}to chase after someone} \textcolor{PineGreen}{\selectlanguage{french}suivre quelqu'un à la trace}  

\lhead{\firstmark}
\rhead{\botmark}

\subsection{\hspace{-0.5cm} {\Large \textcolor{darkblue}{\textbf{\ipa{tse˩\textsubscript{a}}}} \textsubscript{2}}\hspace{0.5cm}[\kern2pt{\textcolor{darkblue}{\textbf{\ipa{tse˩˥}}}}\kern2pt]} \hypertarget{tse\string_Ba2}{}
\markboth{\textcolor{darkblue}{\textbf{\ipa{tse˩\textsubscript{a}}}} \textsubscript{2}}{}
\textcolor{teal}{\zh{动词}} \hspace{4pt} \zh{声调类:} L\textsubscript{a}.
\zh{漂浮 (浮在水上)。} \textcolor{Sepia}{\selectlanguage{english}To float.} \textcolor{PineGreen}{\selectlanguage{french}Flotter.}  ¶ \textcolor{darkblue}{\textbf{\ipa{gɤ˩tse˧}}} \zh{同上:漂浮 (浮在水上)} \textcolor{Sepia}{\selectlanguage{english}as above: to float} \textcolor{PineGreen}{\selectlanguage{french}même sens: flotter}  
 ¶ \textcolor{darkblue}{\textbf{\ipa{ɖɯ˧-tse˧\textasciitilde{}tse˥-ɻ̍˩}}} \zh{\mytextsc{delimitative} \string_ \mytextsc{red} \mytextsc{inceptive}} \textcolor{Sepia}{\selectlanguage{english}\mytextsc{delimitative} \string_ \mytextsc{red} \mytextsc{inceptive}} \textcolor{PineGreen}{\selectlanguage{french}\mytextsc{délimitatif} \string_ \mytextsc{red} \mytextsc{inchoatif}}  
 ¶ \textcolor{darkblue}{\textbf{\ipa{dʑɯ˩ʁo˩˥ | tʰi˧-tse˩ (-dʑo˩)}}} \zh{让木头漂到下游} \textcolor{Sepia}{\selectlanguage{english}to float (in a torrent) on the mountain (e.g. timber is floated downstream)} \textcolor{PineGreen}{\selectlanguage{french}faire flotter (dans un torrent), en montagne (ex.: des troncs qu'on ramène du lieu d'abattage jusqu'à la plaine)}  
 ¶ \textcolor{darkblue}{\textbf{\ipa{dʑɯ˩ʁo˩ tse˧}}} \zh{同上:让木头漂到下游} \textcolor{Sepia}{\selectlanguage{english}as above: to float down from the mountain} \textcolor{PineGreen}{\selectlanguage{french}même sens: ramener de la montagne en faisant descendre la rivière}  

\lhead{\firstmark}
\rhead{\botmark}

\subsection{\hspace{-0.5cm} {\Large \textcolor{darkblue}{\textbf{\ipa{tse˩\textsubscript{a}}}} \textsubscript{3}}\hspace{0.5cm}[\kern2pt{\textcolor{darkblue}{\textbf{\ipa{tse˩˥}}}}\kern2pt]} \hypertarget{tse\string_Ba3}{}
\markboth{\textcolor{darkblue}{\textbf{\ipa{tse˩\textsubscript{a}}}} \textsubscript{3}}{}
\textcolor{teal}{\zh{动词}} \hspace{4pt} \zh{声调类:} L\textsubscript{a}.
\zh{锁门。} \textcolor{Sepia}{\selectlanguage{english}To lock.} \textcolor{PineGreen}{\selectlanguage{french}Fermer à clef.}  ¶ \textcolor{darkblue}{\textbf{\ipa{kʰi˧ tse˥(-ze˩) / kʰi˧ tʰi˧-tse˩}}} \zh{锁门} \textcolor{Sepia}{\selectlanguage{english}to lock the door} \textcolor{PineGreen}{\selectlanguage{french}verrouiller la porte}  

\lhead{\firstmark}
\rhead{\botmark}

\subsection{\hspace{-0.5cm} {\Large \textcolor{darkblue}{\textbf{\ipa{tse˩pʰæ˧˥}}}}\hspace{0.5cm}[\kern2pt{\textcolor{darkblue}{\textbf{\ipa{tse˩pʰæ˧˥}}}}\kern2pt]} \hypertarget{tse\string_Bp\string_h\{\string_M\string_T1}{}
\markboth{\textcolor{darkblue}{\textbf{\ipa{tse˩pʰæ˧˥}}}}{}
\textcolor{teal}{\zh{名词}} \hspace{4pt} \zh{声调类:} LM+MH\#.
\zh{民国之前的货币。} \textcolor{Sepia}{\selectlanguage{english}Coins of the imperial times.} \textcolor{PineGreen}{\selectlanguage{french}Pièces de l'époque impériale.}  ¶ \textcolor{darkblue}{\textbf{\ipa{æ˧-tse˥pʰæ˩}}} \zh{民国之前的铜币} \textcolor{Sepia}{\selectlanguage{english}bronze coins of the imperial times} \textcolor{PineGreen}{\selectlanguage{french}pièce en bronze de l'époque impériale}  
 \zh{量词}: \textcolor{darkblue}{\textbf{\ipa{pʰæ˧˥}}} 
\lhead{\firstmark}
\rhead{\botmark}

\subsection{\hspace{-0.5cm} {\Large \textcolor{darkblue}{\textbf{\ipa{tse˩qwæ˧˥}}}}\hspace{0.5cm}[\kern2pt{\textcolor{darkblue}{\textbf{\ipa{tse˩qwæ˧˥}}}}\kern2pt]} \hypertarget{tse\string_Bqw\{\string_M\string_T1}{}
\markboth{\textcolor{darkblue}{\textbf{\ipa{tse˩qwæ˧˥}}}}{}
\textcolor{teal}{\zh{名词}} \hspace{4pt} \zh{声调类:} LM+MH\#.
\zh{钥匙。} \textcolor{Sepia}{\selectlanguage{english}Key.} \textcolor{PineGreen}{\selectlanguage{french}Clef.}  \zh{量词}: \textcolor{darkblue}{\textbf{\ipa{ɭɯ˧}}} 
\lhead{\firstmark}
\rhead{\botmark}

\subsection{\hspace{-0.5cm} {\Large \textcolor{darkblue}{\textbf{\ipa{tse˩tɑ˧˥}}}}\hspace{0.5cm}[\kern2pt{\textcolor{darkblue}{\textbf{\ipa{tse˩tɑ˧˥}}}}\kern2pt]} \hypertarget{tse\string_BtA\string_M\string_T1}{}
\markboth{\textcolor{darkblue}{\textbf{\ipa{tse˩tɑ˧˥}}}}{}
\textcolor{teal}{\zh{名词}} \hspace{4pt} \zh{声调类:} LM+MH\#.
\zh{剪刀。} \textcolor{Sepia}{\selectlanguage{english}Scissors.} \textcolor{PineGreen}{\selectlanguage{french}Ciseaux.}  \zh{量词}: \textcolor{darkblue}{\textbf{\ipa{nɑ˧}}} 
\lhead{\firstmark}
\rhead{\botmark}

\subsection{\hspace{-0.5cm} {\Large \textcolor{darkblue}{\textbf{\ipa{tse˩ʈʂʰv̩˩}}}}\hspace{0.5cm}[\kern2pt{\textcolor{darkblue}{\textbf{\ipa{tse˩ʈʂʰv̩˩˥}}}}\kern2pt]} \hypertarget{tse\string_Bt`s`\string_hv\string_=\string_B1}{}
\markboth{\textcolor{darkblue}{\textbf{\ipa{tse˩ʈʂʰv̩˩}}}}{}
\textcolor{teal}{\zh{名词}} \hspace{4pt} \zh{声调类:} L.
\zh{骂狗的话。} \textcolor{Sepia}{\selectlanguage{english}Derogatory term of address for a dog.} \textcolor{PineGreen}{\selectlanguage{french}Sac à puces: terme d'insulte pour un chien.} \zh{~【参考】~} \hyperlink{}{\textcolor{darkblue}{\textbf{\ipa{tse˩ʈʂʰv̩˩-kʰv̩˥}}}} 
\lhead{\firstmark}
\rhead{\botmark}

\subsection{\hspace{-0.5cm} {\Large \textcolor{darkblue}{\textbf{\ipa{tse˩ʈʂʰv̩˩-kʰv̩˥}}}}\hspace{0.5cm}[\kern2pt{\textcolor{darkblue}{\textbf{\ipa{xxxx non-correspondance entre le nombre de morphèmes et le nombre de tons de morphèmes}}}}\kern2pt]} \hypertarget{tse\string_Bt`s`\string_hv\string_=\string_B-k\string_hv\string_=\string_T1}{}
\markboth{\textcolor{darkblue}{\textbf{\ipa{tse˩ʈʂʰv̩˩-kʰv̩˥}}}}{}
\textcolor{teal}{\zh{名词}} \hspace{4pt} \zh{声调类:} L+H\#.
\zh{骂狗的话。} \textcolor{Sepia}{\selectlanguage{english}Derogatory term of address for a dog.} \textcolor{PineGreen}{\selectlanguage{french}Sac à puces: terme d'insulte pour un chien.}  ¶ \textcolor{darkblue}{\textbf{\ipa{tse˩ʈʂʰv̩˩-kʰv̩˧ ! | mv̩˩tɕo˧ se˥ !}}} \zh{你这坏狗,下去!} \textcolor{Sepia}{\selectlanguage{english}Come down, you damn dog!} \textcolor{PineGreen}{\selectlanguage{french}Descends, sac à puces! (Injonction adressée à un chien qui s'aventurait dans la partie haute de la salle à manger)}  
\zh{~【参考】~} \hyperlink{}{\textcolor{darkblue}{\textbf{\ipa{tse˩ʈʂʰv̩˩}}}} 
\lhead{\firstmark}
\rhead{\botmark}

\subsection{\hspace{-0.5cm} {\Large \textcolor{darkblue}{\textbf{\ipa{tse˩˥}}}}\hspace{0.5cm}[\kern2pt{\textcolor{darkblue}{\textbf{\ipa{tse˩˥}}}}\kern2pt]} \hypertarget{tse\string_B\string_T1}{}
\markboth{\textcolor{darkblue}{\textbf{\ipa{tse˩˥}}}}{}
\textcolor{teal}{\zh{名词}} \hspace{4pt} \zh{声调类:} LH.
\zh{锁。} \textcolor{Sepia}{\selectlanguage{english}Lock.} \textcolor{PineGreen}{\selectlanguage{french}Serrure, verrou.}  ¶ \textcolor{darkblue}{\textbf{\ipa{æ˧tse˥}}} \zh{铜锁} \textcolor{Sepia}{\selectlanguage{english}bronze lock} \textcolor{PineGreen}{\selectlanguage{french}verrou en bronze}  
 \zh{量词}: \textcolor{darkblue}{\textbf{\ipa{nɑ˧}}} 
\lhead{\firstmark}
\rhead{\botmark}

\subsection{\hspace{-0.5cm} {\Large \textcolor{darkblue}{\textbf{\ipa{tsɤ˧}}} \textsubscript{1}}\hspace{0.5cm}[\kern2pt{\textcolor{darkblue}{\textbf{\ipa{tsɤ˥}}}}\kern2pt]} \hypertarget{ts7\string_M1}{}
\markboth{\textcolor{darkblue}{\textbf{\ipa{tsɤ˧}}} \textsubscript{1}}{}
\textcolor{teal}{\zh{动词}} \hspace{4pt} \zh{声调类:} M intrans.
\zh{形成,变成。} \textcolor{Sepia}{\selectlanguage{english}To become, tu turn into; to be.} \textcolor{PineGreen}{\selectlanguage{french}Se transformer, créer, devenir; être.}  ¶ \textcolor{darkblue}{\textbf{\ipa{sɯ˧pv̩˧-sɯ˥nɑ˩-ʈʂʰɯ˩ | ə˧dzɤ˧\textasciitilde{}dzɤ˥-zo˩ | pʰi˧li˩ tsɤ˩-ɲi˩-kv̩˩-tsɯ˩ | -mv̩˩!}}} \zh{毛虫能慢慢变成蝴蝶,不是吗?} \textcolor{Sepia}{\selectlanguage{english}The caterpillar gradually becomes a butterfly, doesn't it!} \textcolor{PineGreen}{\selectlanguage{french}la chenille devient peu à peu papillon!}  
 ¶ \textcolor{darkblue}{\textbf{\ipa{ɖɯ˧-bæ˧ mɤ˧-tsɤ˧}}} \zh{有区别、不一样} \textcolor{Sepia}{\selectlanguage{english}It's not the same} \textcolor{PineGreen}{\selectlanguage{french}ce n'est pas la même chose, ce n'est pas pareil}  

\lhead{\firstmark}
\rhead{\botmark}

\subsection{\hspace{-0.5cm} {\Large \textcolor{darkblue}{\textbf{\ipa{tsɤ˧}}} \textsubscript{2}}\hspace{0.5cm}[\kern2pt{\textcolor{darkblue}{\textbf{\ipa{tsɤ˥}}}}\kern2pt]} \hypertarget{ts7\string_M2}{}
\markboth{\textcolor{darkblue}{\textbf{\ipa{tsɤ˧}}} \textsubscript{2}}{}
\textcolor{teal}{\zh{形容词}} \hspace{4pt} \zh{声调类:} M.
\zh{对,合适,成。} \textcolor{Sepia}{\selectlanguage{english}Suitable, correct.} \textcolor{PineGreen}{\selectlanguage{french}Correct, qui va bien.}  ¶ \textcolor{darkblue}{\textbf{\ipa{(le˧-)tsɤ˧-ze˧!}}} \zh{好了! / 弄好了! / 成!} \textcolor{Sepia}{\selectlanguage{english}It's okay! / It's arranged! / Things have been made good!} \textcolor{PineGreen}{\selectlanguage{french}c'est bon, c'est arrangé!}  
 ¶ \textcolor{darkblue}{\textbf{\ipa{tsɤ˧-ʝi˧!}}} \zh{行! / 好的!(表示同意或接受命令)} \textcolor{Sepia}{\selectlanguage{english}Okay, fine! (Indication of acquiescence to an instruction)} \textcolor{PineGreen}{\selectlanguage{french}OK! C'est bon! (formule très courante, pour indiquer son acquiescement à une instruction reçue)}  
 ¶ \textcolor{darkblue}{\textbf{\ipa{tsɤ˧ ɲi˥!}}} \zh{好的!} \textcolor{Sepia}{\selectlanguage{english}That's fine!} \textcolor{PineGreen}{\selectlanguage{french}c'est bon!}  
 ¶ \textcolor{darkblue}{\textbf{\ipa{no˧ | mɤ˧-bi˧ mɤ˧-tsɤ˧!}}} \zh{你如果不去,就不对! =你不能不去!} \textcolor{Sepia}{\selectlanguage{english}It wouldn't be right for you not to go!} \textcolor{PineGreen}{\selectlanguage{french}Tu ne peux pas ne pas y aller!, littéralement “que tu n'y ailles pas, ça ne va pas!”}  
 ¶ \textcolor{darkblue}{\textbf{\ipa{ʈʂʰɯ˧ | ɖɯ˧-pi˧˥ | mɤ˧-tsɤ˧!}}} \zh{他有一点不对劲吧!} \textcolor{Sepia}{\selectlanguage{english}He is not quite OK! / There's something wrong with him!} \textcolor{PineGreen}{\selectlanguage{french}Lui, il est pas très net! / Y'a quelque chose qui va pas chez lui!}  
 ¶ \textcolor{darkblue}{\textbf{\ipa{tsɤ˧ mɤ˧-ʝi˧-ze˧!}}} \zh{不好了!/不行了!} \textcolor{Sepia}{\selectlanguage{english}It won't do! / It won't work! / It's no good!} \textcolor{PineGreen}{\selectlanguage{french}Ca ne va plus!}  
 ¶ \textcolor{darkblue}{\textbf{\ipa{hĩ˧-ɳɯ˩ | le˧-so˩, | tsɤ˧!}}} \zh{人家教,是好事! / 人家教,是要珍惜的! / 有人愿意教你,是件好事!} \textcolor{Sepia}{\selectlanguage{english}When people teach you something, it's fortunate / it's good / it's an opportunity to seize! (Context: discussing the behaviour of someone who would not listen to good advice.)} \textcolor{PineGreen}{\selectlanguage{french}Quand on t'apprend quelque chose, c'est une chance à saisir ! / Quand il se trouve quelqu'un qui est disposé à t'apprendre quelque chose, c'est une chance à saisir! / Si tu écoutes les bons conseils, tout ira bien! (Contexte: on évoque quelqu'un qui n'est pas enclin à écouter les bons conseils: qui se braque quand on lui fournit d'utiles conseils.)}  
 ¶ \textcolor{darkblue}{\textbf{\ipa{hĩ˧-ɳɯ˩ | le˧-so˩, | tsɤ˧-kv˧˥!}}} \zh{同上} \textcolor{Sepia}{\selectlanguage{english}as above} \textcolor{PineGreen}{\selectlanguage{french}même sens}  

\lhead{\firstmark}
\rhead{\botmark}

\subsection{\hspace{-0.5cm} {\Large \textcolor{darkblue}{\textbf{\ipa{tsɤ˧}}} \textsubscript{3}}\hspace{0.5cm}[\kern2pt{\textcolor{darkblue}{\textbf{\ipa{tsɤ˥}}}}\kern2pt]} \hypertarget{ts7\string_M3}{}
\markboth{\textcolor{darkblue}{\textbf{\ipa{tsɤ˧}}} \textsubscript{3}}{}
\textcolor{teal}{\zh{形容词}} \hspace{4pt} \zh{声调类:} M.
\zh{细(粉状)。} \textcolor{Sepia}{\selectlanguage{english}Fine (powder).} \textcolor{PineGreen}{\selectlanguage{french}Fine (poudre).}  ¶ \textcolor{darkblue}{\textbf{\ipa{tsɑ˧bɤ˧ tsɤ\#˥}}} \zh{细粮} \textcolor{Sepia}{\selectlanguage{english}fine flour} \textcolor{PineGreen}{\selectlanguage{french}farine fine}  

\lhead{\firstmark}
\rhead{\botmark}

\subsection{\hspace{-0.5cm} {\Large \textcolor{darkblue}{\textbf{\ipa{tsɤ˧}}} \textsubscript{4}}\hspace{0.5cm}[\kern2pt{\textcolor{darkblue}{\textbf{\ipa{tsɤ˥}}}}\kern2pt]} \hypertarget{ts7\string_M4}{}
\markboth{\textcolor{darkblue}{\textbf{\ipa{tsɤ˧}}} \textsubscript{4}}{}
\textcolor{teal}{\zh{形容词}} \hspace{4pt} \zh{声调类:} M.
\zh{嘴馋。} \textcolor{Sepia}{\selectlanguage{english}Greedy.} \textcolor{PineGreen}{\selectlanguage{french}Gourmand.} \zh{~【参考】~} \hyperlink{}{\textcolor{darkblue}{\textbf{\ipa{tsɤ˧ʁo˧-tsʰi˧ʁo˥}}}} 
\lhead{\firstmark}
\rhead{\botmark}

\subsection{\hspace{-0.5cm} {\Large \textcolor{darkblue}{\textbf{\ipa{tsɤ˧di˧}}}}\hspace{0.5cm}[\kern2pt{\textcolor{darkblue}{\textbf{\ipa{tsɤ˩di˩˥}}}}\kern2pt]} \hypertarget{ts7\string_Mdi\string_M1}{}
\markboth{\textcolor{darkblue}{\textbf{\ipa{tsɤ˧di˧}}}}{}
\textcolor{teal}{\zh{名词}} \hspace{4pt} \zh{声调类:} M.
\zh{香木。} \textcolor{Sepia}{\selectlanguage{english}Sandalwood, sandlewood; a tall tree, not a shrub.} \textcolor{PineGreen}{\selectlanguage{french}Arbre à épice, arbre à encens de grande taille.} \zh{当地汉语方言:}\zh{柏香。} ¶ \textcolor{darkblue}{\textbf{\ipa{tsɤ˧di˧-dzi˩}}} \zh{同上} \textcolor{Sepia}{\selectlanguage{english}same meaning} \textcolor{PineGreen}{\selectlanguage{french}même sens}  
\zh{~【参考】~} \textcolor{darkblue}{\textbf{\ipa{ʁo˩kʰv˩}}} 
\lhead{\firstmark}
\rhead{\botmark}

\subsection{\hspace{-0.5cm} {\Large \textcolor{darkblue}{\textbf{\ipa{tsɤ˧ɖɯ˧}}}}\hspace{0.5cm}[\kern2pt{\textcolor{darkblue}{\textbf{\ipa{tsɤ˧ɖɯ˩}}}}\kern2pt]} \hypertarget{ts7\string_Md`M\string_M1}{}
\markboth{\textcolor{darkblue}{\textbf{\ipa{tsɤ˧ɖɯ˧}}}}{}
\textcolor{teal}{\zh{动词}} \hspace{4pt} \zh{声调类:} .
\zh{生崽子(牛类)。} \textcolor{Sepia}{\selectlanguage{english}To give birth (cattle).} \textcolor{PineGreen}{\selectlanguage{french}Mettre bas (bovidé).}  ¶ \textcolor{darkblue}{\textbf{\ipa{tsɤ˧ɖɯ˧-ze˩}}} \zh{生崽子了} \textcolor{Sepia}{\selectlanguage{english}\mytextsc{pfv}} \textcolor{PineGreen}{\selectlanguage{french}\mytextsc{pfv}}  
 ¶ \textcolor{darkblue}{\textbf{\ipa{(dʑi˧mi˧) tsɤ˧ɖɯ˧-ze˩}}} \zh{水牛生崽子了。} \textcolor{Sepia}{\selectlanguage{english}(the water buffalo) has given birth.} \textcolor{PineGreen}{\selectlanguage{french}(le buffle) a enfanté.}  

\lhead{\firstmark}
\rhead{\botmark}

\subsection{\hspace{-0.5cm} {\Large \textcolor{darkblue}{\textbf{\ipa{tsɤ˧ʁo˧-tsʰi˧ʁo˥}}}}\hspace{0.5cm}[\kern2pt{\textcolor{darkblue}{\textbf{\ipa{xxxx non-correspondance entre le nombre de morphèmes et le nombre de tons de morphèmes}}}}\kern2pt]} \hypertarget{ts7\string_MRo\string_M-ts\string_hi\string_MRo\string_T1}{}
\markboth{\textcolor{darkblue}{\textbf{\ipa{tsɤ˧ʁo˧-tsʰi˧ʁo˥}}}}{}
\textcolor{teal}{\zh{形容词}} \hspace{4pt} \zh{声调类:} H\#.
\zh{馋。} \textcolor{Sepia}{\selectlanguage{english}Greedy.} \textcolor{PineGreen}{\selectlanguage{french}Gourmand.}  ¶ \textcolor{darkblue}{\textbf{\ipa{tsɤ˧ʁo˧-tsʰi˧ʁo˥ tsʰi˩}}} \zh{馋} \textcolor{Sepia}{\selectlanguage{english}to be greedy} \textcolor{PineGreen}{\selectlanguage{french}être gourmand}  
\zh{~【参考】~} \hyperlink{}{\textcolor{darkblue}{\textbf{\ipa{tsɤ˧}}} \textsubscript{4}} 
\lhead{\firstmark}
\rhead{\botmark}

\subsection{\hspace{-0.5cm} {\Large \textcolor{darkblue}{\textbf{\ipa{tsi˥}}}}\hspace{0.5cm}[\kern2pt{\textcolor{darkblue}{\textbf{\ipa{tsi˥}}}}\kern2pt]} \hypertarget{tsi\string_T1}{}
\markboth{\textcolor{darkblue}{\textbf{\ipa{tsi˥}}}}{}
\textcolor{teal}{\zh{名词}} \hspace{4pt} \zh{声调类:} \#H.
\ding{202} \zh{裂缝、缝隙。} \textcolor{Sepia}{\selectlanguage{english}Crack.} \textcolor{PineGreen}{\selectlanguage{french}Fissure, interstice.}  ¶ \textcolor{darkblue}{\textbf{\ipa{tsi˧ qʰwæ˧-ze˥!}}} \zh{有了裂缝!} \textcolor{Sepia}{\selectlanguage{english}A crack has appeared!} \textcolor{PineGreen}{\selectlanguage{french}il y a une fissure qui s'est faite! / ça s'est fendu! to crack; to develop a chink/crack/fissure}  
 ¶ \textcolor{darkblue}{\textbf{\ipa{tsi˧ hɯ˧-ze˧!}}} \zh{有了裂缝!} \textcolor{Sepia}{\selectlanguage{english}A crack has appeared!} \textcolor{PineGreen}{\selectlanguage{french}ça s'est fissuré!}  
 \zh{量词}: \textcolor{darkblue}{\textbf{\ipa{pʰæ˧˥}}} \ding{203} \zh{针脚。} \textcolor{Sepia}{\selectlanguage{english}Stitch.} \textcolor{PineGreen}{\selectlanguage{french}Couture.} 
\lhead{\firstmark}
\rhead{\botmark}

\subsection{\hspace{-0.5cm} {\Large \textcolor{darkblue}{\textbf{\ipa{tsi˧\textsubscript{a}}}}}\hspace{0.5cm}[\kern2pt{\textcolor{darkblue}{\textbf{\ipa{tsi˥}}}}\kern2pt]} \hypertarget{tsi\string_Ma1}{}
\markboth{\textcolor{darkblue}{\textbf{\ipa{tsi˧\textsubscript{a}}}}}{}
\textcolor{teal}{\zh{形容词}} \hspace{4pt} \zh{声调类:} M\textsubscript{a}.
\zh{辣。} \textcolor{Sepia}{\selectlanguage{english}Spicy.} \textcolor{PineGreen}{\selectlanguage{french}Piquant, pimenté.}  ¶ \textcolor{darkblue}{\textbf{\ipa{ʈʂʰɯ˧ tsi˧-hĩ˧ ɲi˥!}}} \zh{这是辣的!} \textcolor{Sepia}{\selectlanguage{english}It's spicy!} \textcolor{PineGreen}{\selectlanguage{french}c'est piquant/c'est pimenté!}  

\lhead{\firstmark}
\rhead{\botmark}

\subsection{\hspace{-0.5cm} {\Large \textcolor{darkblue}{\textbf{\ipa{tsi˧\textsubscript{b}}}}}\hspace{0.5cm}[\kern2pt{\textcolor{darkblue}{\textbf{\ipa{tsi˥}}}}\kern2pt]} \hypertarget{tsi\string_Mb1}{}
\markboth{\textcolor{darkblue}{\textbf{\ipa{tsi˧\textsubscript{b}}}}}{}
\textcolor{teal}{\zh{动词}} \hspace{4pt} \zh{声调类:} M\textsubscript{b}.
\zh{安装。} \textcolor{Sepia}{\selectlanguage{english}To set up, to install.} \textcolor{PineGreen}{\selectlanguage{french}Fixer, installer, mettre en place (ex.: un pilier, dans une maison en construction).}  ¶ \textcolor{darkblue}{\textbf{\ipa{tso˧\textasciitilde{}tso˧ tsi˧}}} \zh{安装东西} \textcolor{Sepia}{\selectlanguage{english}to set up something} \textcolor{PineGreen}{\selectlanguage{french}installer quelque chose}  
 ¶ \textcolor{darkblue}{\textbf{\ipa{tso˧\textasciitilde{}tso˧ | gɤ˩-tsi˧-ɻ̍˥}}} \zh{安装东西} \textcolor{Sepia}{\selectlanguage{english}to set up something} \textcolor{PineGreen}{\selectlanguage{french}installer quelque chose, mettre quelque chose en place}  
 ¶ \textcolor{darkblue}{\textbf{\ipa{gɤ˩-tsi˧ tʰi˧-tɕɯ˥}}} \zh{立起来} \textcolor{Sepia}{\selectlanguage{english}to set up (vertically)} \textcolor{PineGreen}{\selectlanguage{french}(re)lever, (re)mettre d'aplomb}  

\lhead{\firstmark}
\rhead{\botmark}

\subsection{\hspace{-0.5cm} {\Large \textcolor{darkblue}{\textbf{\ipa{tsi˧gi˥\$}}}}\hspace{0.5cm}[\kern2pt{\textcolor{darkblue}{\textbf{\ipa{tsi˧gi˥}}}}\kern2pt]} \hypertarget{tsi\string_Mgi\string_T\$1}{}
\markboth{\textcolor{darkblue}{\textbf{\ipa{tsi˧gi˥\$}}}}{}
\textcolor{teal}{\zh{名词}} \hspace{4pt} \zh{声调类:} H\$.
\zh{缝隙,例如:墙上的。} \textcolor{Sepia}{\selectlanguage{english}Crack, fissure.} \textcolor{PineGreen}{\selectlanguage{french}Fissure.}  ¶ \textcolor{darkblue}{\textbf{\ipa{tsi˧gi˥ | ɖɯ˧-kʰwɤ˥}}} \zh{一个缝隙} \textcolor{Sepia}{\selectlanguage{english}a fissure} \textcolor{PineGreen}{\selectlanguage{french}une fissure}  
 ¶ \textcolor{darkblue}{\textbf{\ipa{tsi˧gi˥ | ɖɯ˧-kʰwɤ˧ tʰi˧-di˥}}} \zh{有一个缝隙} \textcolor{Sepia}{\selectlanguage{english}there is a fissure} \textcolor{PineGreen}{\selectlanguage{french}il y a une fissure}  
 \zh{量词}: \textcolor{darkblue}{\textbf{\ipa{ɭɯ˧, kʰwɤ˥}}} 
\lhead{\firstmark}
\rhead{\botmark}

\subsection{\hspace{-0.5cm} {\Large \textcolor{darkblue}{\textbf{\ipa{tsi˩\textsubscript{a}}}}}\hspace{0.5cm}[\kern2pt{\textcolor{darkblue}{\textbf{\ipa{tsi˩˥}}}}\kern2pt]} \hypertarget{tsi\string_Ba1}{}
\markboth{\textcolor{darkblue}{\textbf{\ipa{tsi˩\textsubscript{a}}}}}{}
\textcolor{teal}{\zh{动词}} \hspace{4pt} \zh{声调类:} L\textsubscript{a}.
\zh{烧开。} \textcolor{Sepia}{\selectlanguage{english}To boil, to bring to a boil.} \textcolor{PineGreen}{\selectlanguage{french}Faire bouillir.}  ¶ \textcolor{darkblue}{\textbf{\ipa{dʑɯ˧ | le˧-tsi˩-tʰv̩˩-ze˩!}}} \zh{水开了!} \textcolor{Sepia}{\selectlanguage{english}The water is boiling!} \textcolor{PineGreen}{\selectlanguage{french}L'eau bout!}  
 ¶ \textcolor{darkblue}{\textbf{\ipa{dʑɯ˩ tsi˩-tʰv̩˩-ze˥!}}} \zh{水开了!} \textcolor{Sepia}{\selectlanguage{english}The water is boiling!} \textcolor{PineGreen}{\selectlanguage{french}L'eau bout!}  
 ¶ \textcolor{darkblue}{\textbf{\ipa{mɤ˧-tsi˩-tʰv̩˩-sɯ˩!}}} \zh{还没有烧开!} \textcolor{Sepia}{\selectlanguage{english}It is not boiling yet!} \textcolor{PineGreen}{\selectlanguage{french}Ca ne bout pas encore!}  
 ¶ \textcolor{darkblue}{\textbf{\ipa{ɖɯ˧-tsi˩-tʰv̩˩-ɻ̍˩-kʰɯ˩}}} \zh{煮一会儿} \textcolor{Sepia}{\selectlanguage{english}to leave to boil for a while} \textcolor{PineGreen}{\selectlanguage{french}laisser bouillir un moment}  
 ¶ \textcolor{darkblue}{\textbf{\ipa{ɖɯ˧-tsi˧\textasciitilde{}tsi˥-ɻ̍˩ kʰɯ˩}}} \zh{煮一会儿} \textcolor{Sepia}{\selectlanguage{english}to boil a while} \textcolor{PineGreen}{\selectlanguage{french}faire bouillir un moment}  

\lhead{\firstmark}
\rhead{\botmark}

\subsection{\hspace{-0.5cm} {\Large \textcolor{darkblue}{\textbf{\ipa{tsi˩ɭɯ˩}}}}\hspace{0.5cm}[\kern2pt{\textcolor{darkblue}{\textbf{\ipa{tsi˩ɭɯ˩˥}}}}\kern2pt]} \hypertarget{tsi\string_Bl\string_RM\string_B1}{}
\markboth{\textcolor{darkblue}{\textbf{\ipa{tsi˩ɭɯ˩}}}}{}
\textcolor{teal}{\zh{名词}} \hspace{4pt} \zh{声调类:} L.
\zh{一种小鸟。} \textcolor{Sepia}{\selectlanguage{english}A species of small bird.} \textcolor{PineGreen}{\selectlanguage{french}Petit oiseau de couleur bleue/verte.}  \zh{量词}: \textcolor{darkblue}{\textbf{\ipa{mi˩}}} 
\lhead{\firstmark}
\rhead{\botmark}

\subsection{\hspace{-0.5cm} {\Large \textcolor{darkblue}{\textbf{\ipa{‑tso˧}}}}\hspace{0.5cm}[\kern2pt{\textcolor{darkblue}{\textbf{\ipa{tso˥}}}}\kern2pt]} \hypertarget{‑tso\string_M1}{}
\markboth{\textcolor{darkblue}{\textbf{\ipa{‑tso˧}}}}{}
\textcolor{teal}{\zh{后缀}} \hspace{4pt} \zh{声调类:} M.
\zh{\mytextsc{意志。}} \textcolor{Sepia}{\selectlanguage{english}\mytextsc{volitive}.} \textcolor{PineGreen}{\selectlanguage{french}\mytextsc{volitif}.}  ¶ \textcolor{darkblue}{\textbf{\ipa{dʑɤ˩bv̩˥-tso˩-ɲi˩-mæ˩!}}} \zh{玩一玩吧!} \textcolor{Sepia}{\selectlanguage{english}Let's go and play!} \textcolor{PineGreen}{\selectlanguage{french}On joue, d'accord? / Allez, on va jouer!}  
 ¶ \textcolor{darkblue}{\textbf{\ipa{ə˧tso˧ tv̩˧-tso˧-ɲi˥ ?}}} \zh{要种什么东西?} \textcolor{Sepia}{\selectlanguage{english}What do you plan to plant? / Which crop are you going to plant?} \textcolor{PineGreen}{\selectlanguage{french}Qu'est-ce que vous comptez planter?}  
 ¶ \textcolor{darkblue}{\textbf{\ipa{ə˧tso˧ ʝi˧-bi˧-tso˧-ɲi˥?}}} \zh{要做什么了?} \textcolor{Sepia}{\selectlanguage{english}What do you plan to do now?} \textcolor{PineGreen}{\selectlanguage{french}Qu'est-ce que vous comptez faire maintenant?}  

\lhead{\firstmark}
\rhead{\botmark}

\subsection{\hspace{-0.5cm} {\Large \textcolor{darkblue}{\textbf{\ipa{tso˧kʰwɤ\#˥}}}}\hspace{0.5cm}[\kern2pt{\textcolor{darkblue}{\textbf{\ipa{tso˧kʰwɤ˧}}}}\kern2pt]} \hypertarget{tso\string_Mk\string_hw7\#\string_T1}{}
\markboth{\textcolor{darkblue}{\textbf{\ipa{tso˧kʰwɤ\#˥}}}}{}
\textcolor{teal}{\zh{名词}} \hspace{4pt} \zh{声调类:} \#H.
\zh{袋子。} \textcolor{Sepia}{\selectlanguage{english}Bag (of fabric or leather).} \textcolor{PineGreen}{\selectlanguage{french}Sac (était fait en toile ou en cuir).}  \zh{量词}: \textcolor{darkblue}{\textbf{\ipa{ɭɯ˧}}} 
\lhead{\firstmark}
\rhead{\botmark}

\subsection{\hspace{-0.5cm} {\Large \textcolor{darkblue}{\textbf{\ipa{tso˧lo˧-mv̩˥tso˩}}}}\hspace{0.5cm}[\kern2pt{\textcolor{darkblue}{\textbf{\ipa{tso˧lo˧mv̩˥tso˩}}}}\kern2pt]} \hypertarget{tso\string_Mlo\string_M-mv\string_=\string_Ttso\string_B1}{}
\markboth{\textcolor{darkblue}{\textbf{\ipa{tso˧lo˧-mv̩˥tso˩}}}}{}
\textcolor{teal}{\zh{名词}} \hspace{4pt} \zh{声调类:} \#H-.
\zh{东西,工具。} \textcolor{Sepia}{\selectlanguage{english}Tool; thing, object.} \textcolor{PineGreen}{\selectlanguage{french}Outil; chose, objet, truc.}  \zh{量词}: \textcolor{darkblue}{\textbf{\ipa{nɑ˧, ɭɯ˧}}} 
\lhead{\firstmark}
\rhead{\botmark}

\subsection{\hspace{-0.5cm} {\Large \textcolor{darkblue}{\textbf{\ipa{tso˧qwɤ˧}}}}\hspace{0.5cm}[\kern2pt{\textcolor{darkblue}{\textbf{\ipa{tso˧qwɤ˧}}}}\kern2pt]} \hypertarget{tso\string_Mqw7\string_M1}{}
\markboth{\textcolor{darkblue}{\textbf{\ipa{tso˧qwɤ˧}}}}{}
\textcolor{teal}{\zh{名词}} \hspace{4pt} \zh{声调类:} M.
\zh{小床角:主屋里面的一个角落,有床垫子。用餐、招待客人的时候,会有人在上面坐。刚出生的婴儿也在此处睡觉。人去世后,尸体先放在那个地方。} \textcolor{Sepia}{\selectlanguage{english}Sleeping corner: a part of the main room where there is bedding; some people can sit there during meals or family reunions. Newborn babies sleep there. After a decease, corpses are placed on that bed.} \textcolor{PineGreen}{\selectlanguage{french}Chambrette: partie de la pièce principale dans laquelle se trouve un couchage; on y place provisoirement les nouveaux-nés, et les défunts.}  \zh{量词}: \textcolor{darkblue}{\textbf{\ipa{ɭɯ˧}}} 
\lhead{\firstmark}
\rhead{\botmark}

\subsection{\hspace{-0.5cm} {\Large \textcolor{darkblue}{\textbf{\ipa{tso˧tso\#˥}}}}\hspace{0.5cm}[\kern2pt{\textcolor{darkblue}{\textbf{\ipa{tso˧tso˧}}}}\kern2pt]} \hypertarget{tso\string_Mtso\#\string_T1}{}
\markboth{\textcolor{darkblue}{\textbf{\ipa{tso˧tso\#˥}}}}{}
\textcolor{teal}{\zh{名词}} \hspace{4pt} \zh{声调类:} \#H.
\zh{东西。} \textcolor{Sepia}{\selectlanguage{english}Thing, thingummy, stuff.} \textcolor{PineGreen}{\selectlanguage{french}Chose, truc, bidule, objet, machin.}  ¶ \textcolor{darkblue}{\textbf{\ipa{tso˧\textasciitilde{}tso˧-zo˧\textasciitilde{}zo˧-mv̩˧\textasciitilde{}mv̩˥}}} \zh{各种东西、各种各样乱七八糟东西} \textcolor{Sepia}{\selectlanguage{english}thingummy, stuff} \textcolor{PineGreen}{\selectlanguage{french}bidule}  
 ¶ \textcolor{darkblue}{\textbf{\ipa{tso˧\textasciitilde{}tso˧ hwæ˩}}} \zh{买东西} \textcolor{Sepia}{\selectlanguage{english}to buy things} \textcolor{PineGreen}{\selectlanguage{french}acheter des choses}  
 ¶ \textcolor{darkblue}{\textbf{\ipa{tso˧\textasciitilde{}tso˧ tɕʰi˧(-ze˩)}}} \zh{卖东西} \textcolor{Sepia}{\selectlanguage{english}to sell things} \textcolor{PineGreen}{\selectlanguage{french}vendre des choses}  
 ¶ \textcolor{darkblue}{\textbf{\ipa{tso˧\textasciitilde{}tso˧ dzɯ˧(-ze˩)}}} \zh{吃东西} \textcolor{Sepia}{\selectlanguage{english}to eat things} \textcolor{PineGreen}{\selectlanguage{french}manger des choses}  
 ¶ \textcolor{darkblue}{\textbf{\ipa{tso˧\textasciitilde{}tso˧ dze˥}}} \zh{切东西} \textcolor{Sepia}{\selectlanguage{english}to cut things} \textcolor{PineGreen}{\selectlanguage{french}couper des choses}  
 ¶ \textcolor{darkblue}{\textbf{\ipa{tso˧\textasciitilde{}tso˧ ʈʰɯ˩}}} \zh{喝东西} \textcolor{Sepia}{\selectlanguage{english}to drink things} \textcolor{PineGreen}{\selectlanguage{french}boire des choses}  
 ¶ \textcolor{darkblue}{\textbf{\ipa{tso˧\textasciitilde{}tso˧ lɑ˩}}} \zh{打东西} \textcolor{Sepia}{\selectlanguage{english}to beat things} \textcolor{PineGreen}{\selectlanguage{french}frapper des choses}  
 \zh{量词}: \textcolor{darkblue}{\textbf{\ipa{nɑ˧, ɭɯ˧}}} 
\lhead{\firstmark}
\rhead{\botmark}

\subsection{\hspace{-0.5cm} {\Large \textcolor{darkblue}{\textbf{\ipa{tso˩\textsubscript{c}}}}}\hspace{0.5cm}[\kern2pt{\textcolor{darkblue}{\textbf{\ipa{tso˩˥}}}}\kern2pt]} \hypertarget{tso\string_Bc1}{}
\markboth{\textcolor{darkblue}{\textbf{\ipa{tso˩\textsubscript{c}}}}}{}
\textcolor{teal}{\zh{量词}} \hspace{4pt} \zh{声调类:} L\textsubscript{c}.
\zh{量词:间(房间,分隔间,包间)。} \textcolor{Sepia}{\selectlanguage{english}Classifier for rooms.} \textcolor{PineGreen}{\selectlanguage{french}Classificateur des pièces (dans la maison), des compartiments (dans un grenier).}  ¶ \textcolor{darkblue}{\textbf{\ipa{ʈʂʰɯ˧-tso˥}}} \zh{这间} \textcolor{Sepia}{\selectlanguage{english}this room} \textcolor{PineGreen}{\selectlanguage{french}cette pièce}  

\lhead{\firstmark}
\rhead{\botmark}

\subsection{\hspace{-0.5cm} {\Large \textcolor{darkblue}{\textbf{\ipa{tso˩\textsubscript{a}}}}}\hspace{0.5cm}[\kern2pt{\textcolor{darkblue}{\textbf{\ipa{tso˩˥}}}}\kern2pt]} \hypertarget{tso\string_Ba1}{}
\markboth{\textcolor{darkblue}{\textbf{\ipa{tso˩\textsubscript{a}}}}}{}
\textcolor{teal}{\zh{动词}} \hspace{4pt} \zh{声调类:} L\textsubscript{a}.
\zh{砌(墙)、建(桥梁)。} \textcolor{Sepia}{\selectlanguage{english}To build a wall, a bridge….} \textcolor{PineGreen}{\selectlanguage{french}Construire un mur, un pont….}  ¶ \textcolor{darkblue}{\textbf{\ipa{ɖʐɤ˩ tso˩}}} \zh{修一座楼梯} \textcolor{Sepia}{\selectlanguage{english}to build stairs} \textcolor{PineGreen}{\selectlanguage{french}construire un escalier}  
 ¶ \textcolor{darkblue}{\textbf{\ipa{dzo˩ tso˩}}} \zh{修一座桥} \textcolor{Sepia}{\selectlanguage{english}to build a bridge} \textcolor{PineGreen}{\selectlanguage{french}construire un pont}  
 ¶ \textcolor{darkblue}{\textbf{\ipa{ɖɯ˧-tso˧\textasciitilde{}tso˥-ɻ̍˩}}} \zh{修东西} \textcolor{Sepia}{\selectlanguage{english}to build something} \textcolor{PineGreen}{\selectlanguage{french}construire quelque chose}  

\lhead{\firstmark}
\rhead{\botmark}

\subsection{\hspace{-0.5cm} {\Large \textcolor{darkblue}{\textbf{\ipa{tso˩qʰv̩˩ɻ̍˥}}}}\hspace{0.5cm}[\kern2pt{\textcolor{darkblue}{\textbf{\ipa{tso˧qʰv̩˧ɻ̍˥}}}}\kern2pt]} \hypertarget{tso\string_Bq\string_hv\string_=\string_Br£`̍\string_T1}{}
\markboth{\textcolor{darkblue}{\textbf{\ipa{tso˩qʰv̩˩ɻ̍˥}}}}{}
\textcolor{teal}{\zh{名词}} \hspace{4pt} \zh{声调类:} H\#.
\zh{玄关、门厅。} \textcolor{Sepia}{\selectlanguage{english}Porch, enclosed porch, vestibule: a narrow area between the door and the courtyard, covered by the roof (and hence sheltered from rain), and, in some houses, shut off from the coutyard by a wall with one door approximately in the middle. This porch is the area that one reaches when crossing the threshold, coming out from the main room. In the main consultant's house, where the porch is not enclosed, it is exposed to sunshine until the afternoon, and tasks such as chopping vegetables are carried out sitting in this area.} \textcolor{PineGreen}{\selectlanguage{french}Porche, vestibule: espace situé entre la cour et la pièce principale, c'est-à-dire l'espace, protégé de la pluie par la toiture, où l'on parvient lorsqu'on passe le seuil en sortant de la pièce principale. Dans certaines maisons, ce porche est séparé de la cour par une cloison de bois, percée d'une porte à peu près au milieu de sa longueur.}  \zh{量词}: \textcolor{darkblue}{\textbf{\ipa{kʰwɤ˥}}} 
\lhead{\firstmark}
\rhead{\botmark}

\subsection{\hspace{-0.5cm} {\Large \textcolor{darkblue}{\textbf{\ipa{tso˩\textasciitilde{}tso˧˥}}}}\hspace{0.5cm}[\kern2pt{\textcolor{darkblue}{\textbf{\ipa{tso˩tso˧˥}}}}\kern2pt]} \hypertarget{tso\string_B~tso\string_M\string_T1}{}
\markboth{\textcolor{darkblue}{\textbf{\ipa{tso˩\textasciitilde{}tso˧˥}}}}{}
\textcolor{teal}{\zh{动词}} \hspace{4pt} \zh{声调类:} LM+MH\#.
\zh{拌好狗食。} \textcolor{Sepia}{\selectlanguage{english}To mix, to prepare (dog food).} \textcolor{PineGreen}{\selectlanguage{french}Touiller, faire la pâtée (du chien…).} 
\lhead{\firstmark}
\rhead{\botmark}

\subsection{\hspace{-0.5cm} {\Large \textcolor{darkblue}{\textbf{\ipa{tsɯ˥}}} \textsubscript{1}}\hspace{0.5cm}[\kern2pt{\textcolor{darkblue}{\textbf{\ipa{tsɯ˥}}}}\kern2pt]} \hypertarget{tsM\string_T1}{}
\markboth{\textcolor{darkblue}{\textbf{\ipa{tsɯ˥}}} \textsubscript{1}}{}
\textcolor{teal}{\zh{动词}} \hspace{4pt} \zh{声调类:} H.
\ding{202} \zh{绑、捆、栓。} \textcolor{Sepia}{\selectlanguage{english}To tie (with a rope).} \textcolor{PineGreen}{\selectlanguage{french}Attacher.}  ¶ \textcolor{darkblue}{\textbf{\ipa{dʑi˧mi˧ tʰi˧-tsɯ˥}}} \zh{栓水牛} \textcolor{Sepia}{\selectlanguage{english}to tie a water buffalo} \textcolor{PineGreen}{\selectlanguage{french}attacher le buffle}  
 ¶ \textcolor{darkblue}{\textbf{\ipa{tsɯ˧\textasciitilde{}tsɯ˧}}} \zh{\mytextsc{重叠}} \textcolor{Sepia}{\selectlanguage{english}\mytextsc{red}} \textcolor{PineGreen}{\selectlanguage{french}\mytextsc{red}}  
 ¶ \textcolor{darkblue}{\textbf{\ipa{le˧-tsɯ˧\textasciitilde{}tsɯ˧}}} \zh{\mytextsc{accomp} \mytextsc{red}} \textcolor{Sepia}{\selectlanguage{english}\mytextsc{accomp} \mytextsc{red}} \textcolor{PineGreen}{\selectlanguage{french}\mytextsc{accomp} \mytextsc{red}}  
 ¶ \textcolor{darkblue}{\textbf{\ipa{tʰi˧-tsɯ˧\textasciitilde{}tsɯ˧}}} \zh{\mytextsc{dur} \mytextsc{red}} \textcolor{Sepia}{\selectlanguage{english}\mytextsc{dur} \mytextsc{red}} \textcolor{PineGreen}{\selectlanguage{french}\mytextsc{dur} \mytextsc{red}}  
\ding{203} \zh{上吊自杀、缢。} \textcolor{Sepia}{\selectlanguage{english}To hang oneself.} \textcolor{PineGreen}{\selectlanguage{french}Se pendre.}  ¶ \textcolor{darkblue}{\textbf{\ipa{ʁæ˧tsɯ˧ le˧-ʂɯ˧ +ze˧}}} \zh{上吊自杀、缢} \textcolor{Sepia}{\selectlanguage{english}to hang oneself} \textcolor{PineGreen}{\selectlanguage{french}se pendre}  

\lhead{\firstmark}
\rhead{\botmark}

\subsection{\hspace{-0.5cm} {\Large \textcolor{darkblue}{\textbf{\ipa{tsɯ˥}}} \textsubscript{2}}\hspace{0.5cm}[\kern2pt{\textcolor{darkblue}{\textbf{\ipa{tsɯ˥}}}}\kern2pt]} \hypertarget{tsM\string_T2}{}
\markboth{\textcolor{darkblue}{\textbf{\ipa{tsɯ˥}}} \textsubscript{2}}{}
\textcolor{teal}{\zh{动词}} \hspace{4pt} \zh{声调类:} H.
\zh{打捞。} \textcolor{Sepia}{\selectlanguage{english}Dredge for, fish for, scoop up out of water.} \textcolor{PineGreen}{\selectlanguage{french}Prendre avec une écumoire, récupérer dans l'eau.} 
\lhead{\firstmark}
\rhead{\botmark}

\subsection{\hspace{-0.5cm} {\Large \textcolor{darkblue}{\textbf{\ipa{tsɯ˧}}}}\hspace{0.5cm}[\kern2pt{\textcolor{darkblue}{\textbf{\ipa{tsɯ˥}}}}\kern2pt]} \hypertarget{tsM\string_M1}{}
\markboth{\textcolor{darkblue}{\textbf{\ipa{tsɯ˧}}}}{}
\textcolor{teal}{\zh{名词}} \hspace{4pt} \zh{声调类:} M.
\zh{字。} \textcolor{Sepia}{\selectlanguage{english}Letter, Chinese character.} \textcolor{PineGreen}{\selectlanguage{french}Lettre, caractère chinois.}  \zh{【借词】} \zh{字}

\lhead{\firstmark}
\rhead{\botmark}

\subsection{\hspace{-0.5cm} {\Large \textcolor{darkblue}{\textbf{\ipa{tsɯ˩\textsubscript{a}}}}}\hspace{0.5cm}[\kern2pt{\textcolor{darkblue}{\textbf{\ipa{tsɯ˥}}}}\kern2pt]} \hypertarget{tsM\string_Ba1}{}
\markboth{\textcolor{darkblue}{\textbf{\ipa{tsɯ˩\textsubscript{a}}}}}{}
\textcolor{teal}{\zh{动词}} \hspace{4pt} \zh{声调类:} L\textsubscript{a}.
\zh{堵塞、塞住洞口。} \textcolor{Sepia}{\selectlanguage{english}To block up.} \textcolor{PineGreen}{\selectlanguage{french}Boucher/être bouché; obstruer (ex.: obstruer l'entrée d'un trou).} 
\lhead{\firstmark}
\rhead{\botmark}

\subsection{\hspace{-0.5cm} {\Large \textcolor{darkblue}{\textbf{\ipa{tsɯ˩pʰɤ˩}}}}\hspace{0.5cm}[\kern2pt{\textcolor{darkblue}{\textbf{\ipa{tsɯ˧pʰɤ˧˥}}}}\kern2pt]} \hypertarget{tsM\string_Bp\string_h7\string_B1}{}
\markboth{\textcolor{darkblue}{\textbf{\ipa{tsɯ˩pʰɤ˩}}}}{}
\textcolor{teal}{\zh{动词}} \hspace{4pt} \zh{声调类:} L.
\ding{202} \zh{眨眼。} \textcolor{Sepia}{\selectlanguage{english}To blink.} \textcolor{PineGreen}{\selectlanguage{french}Cligner des yeux.}  ¶ \textcolor{darkblue}{\textbf{\ipa{mɤ˧-tsɯ˩pʰɤ˩}}} \zh{不眨眼} \textcolor{Sepia}{\selectlanguage{english}\mytextsc{neg}} \textcolor{PineGreen}{\selectlanguage{french}\mytextsc{neg}}  
 ¶ \textcolor{darkblue}{\textbf{\ipa{ɖɯ˧-tsɯ˧\textasciitilde{}tsɯ˥-ɻ̍˩}}} \zh{\mytextsc{delimitative} \mytextsc{red} \mytextsc{inceptive}} \textcolor{Sepia}{\selectlanguage{english}\mytextsc{delimitative} \mytextsc{red} \mytextsc{inceptive}} \textcolor{PineGreen}{\selectlanguage{french}\mytextsc{délimitatif} \mytextsc{red} \mytextsc{inchoatif}}  
 ¶ \textcolor{darkblue}{\textbf{\ipa{njɤ˩ɭɯ˧ tsɯ˩pʰɤ˩}}} \zh{眨眼} \textcolor{Sepia}{\selectlanguage{english}to blink} \textcolor{PineGreen}{\selectlanguage{french}cligner des yeux}  
\ding{203} \zh{眨眼。} \textcolor{Sepia}{\selectlanguage{english}To wink (as a discreet sign of mutual understanding).} \textcolor{PineGreen}{\selectlanguage{french}Faire un clin d'oeil (discret signe d'intelligence).}  ¶ \textcolor{darkblue}{\textbf{\ipa{ʈʂʰɯ˧ | njɤ˩ɭɯ˧ tsɯ˩pʰɤ˩-dʑo˩!}}} \zh{他在眨眼!} \textcolor{Sepia}{\selectlanguage{english}(S)he is winking!} \textcolor{PineGreen}{\selectlanguage{french}Elle/il est en train de faire un clin d'oeil!}  

\lhead{\firstmark}
\rhead{\botmark}

\subsection{\hspace{-0.5cm} {\Large \textcolor{darkblue}{\textbf{\ipa{tsɯ˧˥}}}}\hspace{0.5cm}[\kern2pt{\textcolor{darkblue}{\textbf{\ipa{tsɯ˧˥}}}}\kern2pt]} \hypertarget{tsM\string_M\string_T1}{}
\markboth{\textcolor{darkblue}{\textbf{\ipa{tsɯ˧˥}}}}{}
\textcolor{teal}{\zh{后缀}} \hspace{4pt} \zh{声调类:} MH.
\zh{据说\mytextsc{传闻据素。}} \textcolor{Sepia}{\selectlanguage{english}Reported/hearsay evidential: the speaker indicates that they are not in a position to vouch personally for what they are saying, that it is based on indirect knowledge, from hearsay.} \textcolor{PineGreen}{\selectlanguage{french}Particule d'évidentialité rapportée: elle indique une connaissance indirecte, par ouï-dire, et non par connaissance directe.} 
\lhead{\firstmark}
\rhead{\botmark}

\subsection{\hspace{-0.5cm} {\Large \textcolor{darkblue}{\textbf{\ipa{tsɯ˧˥}}}}\hspace{0.5cm}[\kern2pt{\textcolor{darkblue}{\textbf{\ipa{tsɯ˧˥}}}}\kern2pt]} \hypertarget{tsM\string_M\string_T1}{}
\markboth{\textcolor{darkblue}{\textbf{\ipa{tsɯ˧˥}}}}{}
\textcolor{teal}{\zh{动词}} \hspace{4pt} \zh{声调类:} MH.
\zh{叫、叫做。} \textcolor{Sepia}{\selectlanguage{english}To call, to give the name of….} \textcolor{PineGreen}{\selectlanguage{french}Appeler, nommer, désigner.}  ¶ \textcolor{darkblue}{\textbf{\ipa{ʈæ˧ʂɯ˧-ɳɯ˧ | no˧-ki˥ | jɤ˩-ʐe˧ ɲi˩-tsɯ˩-mɤ˩-tsɯ˩!}}} \zh{达石把你叫作“老外”!} \textcolor{Sepia}{\selectlanguage{english}\textcolor{darkblue}{\textbf{\ipa{/ʈæ˧ʂɯ˧/}}} calls you “foreigner”!} \textcolor{PineGreen}{\selectlanguage{french}\textcolor{darkblue}{\textbf{\ipa{/ʈæ˧ʂɯ˧/}}} t'appelle “l'étranger” / il te traite d'étranger!}  

\lhead{\firstmark}
\rhead{\botmark}

\subsection{\hspace{-0.5cm} {\Large \textcolor{darkblue}{\textbf{\ipa{tsʰɑ˧bo\#˥}}}}\hspace{0.5cm}[\kern2pt{\textcolor{darkblue}{\textbf{\ipa{tsʰɑ˧bo˧}}}}\kern2pt]} \hypertarget{ts\string_hA\string_Mbo\#\string_T1}{}
\markboth{\textcolor{darkblue}{\textbf{\ipa{tsʰɑ˧bo\#˥}}}}{}
\textcolor{teal}{\zh{名词}} \hspace{4pt} \zh{声调类:} \#H.
\zh{厨师。} \textcolor{Sepia}{\selectlanguage{english}Cook.} \textcolor{PineGreen}{\selectlanguage{french}Cuisinier.}  ¶ \textcolor{darkblue}{\textbf{\ipa{tsʰɑ˧bo˧ lɑ˩}}} \zh{当厨师} \textcolor{Sepia}{\selectlanguage{english}to be a cook, to work as a cook, to get employed as a cook} \textcolor{PineGreen}{\selectlanguage{french}être cuisinier, s'engager comme cuisinier, faire le travail de cuisinier}  
 ¶ \textcolor{darkblue}{\textbf{\ipa{tsʰɑ˧bo˧ ʝi˧}}} \zh{当厨师} \textcolor{Sepia}{\selectlanguage{english}to be a cook, to work as a cook, to get employed as a cook} \textcolor{PineGreen}{\selectlanguage{french}être cuisinier, s'engager comme cuisinier, faire le travail de cuisinier}  

\lhead{\firstmark}
\rhead{\botmark}

\subsection{\hspace{-0.5cm} {\Large \textcolor{darkblue}{\textbf{\ipa{tsʰɑ˧kv̩˩}}}}\hspace{0.5cm}[\kern2pt{\textcolor{darkblue}{\textbf{\ipa{tsʰɑ˧kv̩˩}}}}\kern2pt]} \hypertarget{ts\string_hA\string_Mkv\string_=\string_B1}{}
\markboth{\textcolor{darkblue}{\textbf{\ipa{tsʰɑ˧kv̩˩}}}}{}
\textcolor{teal}{\zh{名词}} \hspace{4pt} \zh{声调类:} L\#.
\zh{仓库(汉语借词)。} \textcolor{Sepia}{\selectlanguage{english}Warehouse, storehouse.} \textcolor{PineGreen}{\selectlanguage{french}Réserve, magasin.}  \zh{【借词】} \zh{仓库}

\lhead{\firstmark}
\rhead{\botmark}

\subsection{\hspace{-0.5cm} {\Large \textcolor{darkblue}{\textbf{\ipa{tsʰɑ˧tɕɤ˧˥}}}}\hspace{0.5cm}[\kern2pt{\textcolor{darkblue}{\textbf{\ipa{tsʰɑ˧tɕɤ˧˥}}}}\kern2pt]} \hypertarget{ts\string_hA\string_Mts£7\string_M\string_T1}{}
\markboth{\textcolor{darkblue}{\textbf{\ipa{tsʰɑ˧tɕɤ˧˥}}}}{}
\textcolor{teal}{\zh{名词}} \hspace{4pt} \zh{声调类:} MH.
\zh{青菜幼苗。} \textcolor{Sepia}{\selectlanguage{english}Seedlings.} \textcolor{PineGreen}{\selectlanguage{french}Jeunes pousses, petites pousses qu'on récolte pour les manger.} 
\lhead{\firstmark}
\rhead{\botmark}

\subsection{\hspace{-0.5cm} {\Large \textcolor{darkblue}{\textbf{\ipa{tsʰɑ˩pʰɑ˩lɑ˥}}}}\hspace{0.5cm}[\kern2pt{\textcolor{darkblue}{\textbf{\ipa{tsʰɑ˩pʰɑ˩lɑ˥}}}}\kern2pt]} \hypertarget{ts\string_hA\string_Bp\string_hA\string_BlA\string_T1}{}
\markboth{\textcolor{darkblue}{\textbf{\ipa{tsʰɑ˩pʰɑ˩lɑ˥}}}}{}
\textcolor{teal}{\zh{名词}} \hspace{4pt} \zh{声调类:} L+H\#.
\zh{苞谷叶(玉米穰子的叶子)。} \textcolor{Sepia}{\selectlanguage{english}Husk of sweet corn (maize) cobs.} \textcolor{PineGreen}{\selectlanguage{french}Feuilles d'épi de maïs: les feuilles qui entourent l'épi de maïs.}  ¶ \textcolor{darkblue}{\textbf{\ipa{qʰɑ˧dze˧-tsʰɑ˩pʰɑ˩lɑ˩}}} \zh{同上} \textcolor{Sepia}{\selectlanguage{english}same meaning} \textcolor{PineGreen}{\selectlanguage{french}même sens}  

\lhead{\firstmark}
\rhead{\botmark}

\subsection{\hspace{-0.5cm} {\Large \textcolor{darkblue}{\textbf{\ipa{tsʰæ˧pʰv˧˥}}}}\hspace{0.5cm}[\kern2pt{\textcolor{darkblue}{\textbf{\ipa{tsʰæ˧pʰv˧˥}}}}\kern2pt]} \hypertarget{ts\string_h\{\string_Mp\string_hv\string_M\string_T1}{}
\markboth{\textcolor{darkblue}{\textbf{\ipa{tsʰæ˧pʰv˧˥}}}}{}
\textcolor{teal}{\zh{名词}} \hspace{4pt} \zh{声调类:} MH\#.
\zh{白菜(借汉语‘白菜’的第二个音节来充当这个名词的第一个音节:按摩梭话句法,形容词在名词后面,跟汉语相反)。} \textcolor{Sepia}{\selectlanguage{english}Chinese cabbage. This is a calque from the Chinese 'white vegetable', using the Chinese word for 'vegetable' in association with the Na word for 'white'.} \textcolor{PineGreen}{\selectlanguage{french}Chou chinois. Il s'agit d'un calque du chinois 'légume blanc', employant le nom chinois pour 'légume' associé à l'adjectif na pour 'blanc'.}  \zh{【借词】} \zh{菜}
 \zh{量词}: \textcolor{darkblue}{\textbf{\ipa{po˧}}} \zh{~【参考】~} \hyperlink{}{\textcolor{darkblue}{\textbf{\ipa{v̩˩tsʰɤ˧-pʰv̩˥}}}} 
\lhead{\firstmark}
\rhead{\botmark}

\subsection{\hspace{-0.5cm} {\Large \textcolor{darkblue}{\textbf{\ipa{tsʰe\#˥}}}}\hspace{0.5cm}[\kern2pt{\textcolor{darkblue}{\textbf{\ipa{tsʰe˥}}}}\kern2pt]} \hypertarget{ts\string_he\#\string_T1}{}
\markboth{\textcolor{darkblue}{\textbf{\ipa{tsʰe\#˥}}}}{}
\textcolor{teal}{\zh{名词}} \hspace{4pt} \zh{声调类:} \#H.
\zh{盐。} \textcolor{Sepia}{\selectlanguage{english}Salt.} \textcolor{PineGreen}{\selectlanguage{french}Sel.} 
\lhead{\firstmark}
\rhead{\botmark}

\subsection{\hspace{-0.5cm} {\Large \textcolor{darkblue}{\textbf{\ipa{tsʰe˧}}}}\hspace{0.5cm}[\kern2pt{\textcolor{darkblue}{\textbf{\ipa{tsʰe˩˥}}}}\kern2pt]} \hypertarget{ts\string_he\string_M1}{}
\markboth{\textcolor{darkblue}{\textbf{\ipa{tsʰe˧}}}}{}
\textcolor{teal}{\zh{数词}} \hspace{4pt} \zh{声调类:} L.
\zh{10。} \textcolor{Sepia}{\selectlanguage{english}10.} \textcolor{PineGreen}{\selectlanguage{french}10.} 
\lhead{\firstmark}
\rhead{\botmark}

\subsection{\hspace{-0.5cm} {\Large \textcolor{darkblue}{\textbf{\ipa{tsʰe˧do˧˥}}}}\hspace{0.5cm}[\kern2pt{\textcolor{darkblue}{\textbf{\ipa{tsʰe˧do˧˥}}}}\kern2pt]} \hypertarget{ts\string_he\string_Mdo\string_M\string_T1}{}
\markboth{\textcolor{darkblue}{\textbf{\ipa{tsʰe˧do˧˥}}}}{}
\textcolor{teal}{\zh{助词}} \hspace{4pt} \zh{声调类:} MH\# | L.
\zh{月初。} \textcolor{Sepia}{\selectlanguage{english}The beginning of the month.} \textcolor{PineGreen}{\selectlanguage{french}Début du mois.}  ¶ \textcolor{darkblue}{\textbf{\ipa{tsʰe˧do˧-ɖɯ˧ɲi\#˥ / tsʰe˧do˧-ɖɯ˧ɲi˥}}} \zh{初一} \textcolor{Sepia}{\selectlanguage{english}the 1st day of the month} \textcolor{PineGreen}{\selectlanguage{french}le 1er du mois}  
 ¶ \textcolor{darkblue}{\textbf{\ipa{tsʰe˧do˧-ɲi˧ɲi\#˥ / tsʰe˧do˧-ɲi˧ɲi˥}}} \zh{初二} \textcolor{Sepia}{\selectlanguage{english}the second day of the month} \textcolor{PineGreen}{\selectlanguage{french}le deuxième jour du mois}  
 ¶ \textcolor{darkblue}{\textbf{\ipa{tsʰe˧do˧˥ | -so˩ɲi˩˥}}} \zh{初三} \textcolor{Sepia}{\selectlanguage{english}the third day of the month} \textcolor{PineGreen}{\selectlanguage{french}le 3e du mois}  
 ¶ \textcolor{darkblue}{\textbf{\ipa{tsʰe˧do˧-ŋwɤ˥ɲi˩}}} \zh{初五} \textcolor{Sepia}{\selectlanguage{english}the 5th day of the month} \textcolor{PineGreen}{\selectlanguage{french}le 5e jour du mois}  
 ¶ \textcolor{darkblue}{\textbf{\ipa{tsʰe˧do˧-hõ˥ɲi˩}}} \zh{初八} \textcolor{Sepia}{\selectlanguage{english}the 8th day of the month} \textcolor{PineGreen}{\selectlanguage{french}le 8e jour du mois}  
 ¶ \textcolor{darkblue}{\textbf{\ipa{tsʰe˧do˧˥ | -tsʰe˩ɲi˩˥}}} \zh{初十} \textcolor{Sepia}{\selectlanguage{english}the 10th day of the month} \textcolor{PineGreen}{\selectlanguage{french}le 10e jour du mois}  
 ¶ \textcolor{darkblue}{\textbf{\ipa{tsʰe˧do˧˥ | -tsʰe˩ɖɯ˩ɲi˩˥}}} \zh{十一日} \textcolor{Sepia}{\selectlanguage{english}the 11th day of the month} \textcolor{PineGreen}{\selectlanguage{french}le 11e jour du mois}  

\lhead{\firstmark}
\rhead{\botmark}

\subsection{\hspace{-0.5cm} {\Large \textcolor{darkblue}{\textbf{\ipa{tsʰe˧hṽ˧˥}}}}\hspace{0.5cm}[\kern2pt{\textcolor{darkblue}{\textbf{\ipa{tsʰe˧hṽ˧˥}}}}\kern2pt]} \hypertarget{ts\string_he\string_Mhv\string_~\string_M\string_T1}{}
\markboth{\textcolor{darkblue}{\textbf{\ipa{tsʰe˧hṽ˧˥}}}}{}
\textcolor{teal}{\zh{名词}} \hspace{4pt} \zh{声调类:} MH\#.
\zh{万年青。} \textcolor{Sepia}{\selectlanguage{english}Chinese evergreen.} \textcolor{PineGreen}{\selectlanguage{french}Aspidistra.}  ¶ \textcolor{darkblue}{\textbf{\ipa{tsʰe˧hṽ˧-dzi˧˥}}} \zh{万年青树} \textcolor{Sepia}{\selectlanguage{english}Chinese evergreen tree} \textcolor{PineGreen}{\selectlanguage{french}arbre/plant d'aspidistra}  
 ¶ \textcolor{darkblue}{\textbf{\ipa{tsʰe˧hṽ˧-bæ˥bæ˩}}} \zh{万年青花} \textcolor{Sepia}{\selectlanguage{english}flowers of Chinese evergreen} \textcolor{PineGreen}{\selectlanguage{french}fleurs d'aspidistra}  
 \zh{量词}: \textcolor{darkblue}{\textbf{\ipa{dzi˩}}} 
\lhead{\firstmark}
\rhead{\botmark}

\subsection{\hspace{-0.5cm} {\Large \textcolor{darkblue}{\textbf{\ipa{tsʰe˧jɤ˧mi˥}}}}\hspace{0.5cm}[\kern2pt{\textcolor{darkblue}{\textbf{\ipa{tsʰe˧jɤ˧mi˥}}}}\kern2pt]} \hypertarget{ts\string_he\string_Mj7\string_Mmi\string_T1}{}
\markboth{\textcolor{darkblue}{\textbf{\ipa{tsʰe˧jɤ˧mi˥}}}}{}
\textcolor{teal}{\zh{名词}} \hspace{4pt} \zh{声调类:} H\#.
\zh{沼泽。} \textcolor{Sepia}{\selectlanguage{english}Marsh, bog, swamp.} \textcolor{PineGreen}{\selectlanguage{french}Marécage.}  ¶ \textcolor{darkblue}{\textbf{\ipa{tsʰe˧jɤ˧mi˥-qo˩, | ʈʰæ˧tɕi˧ɭɯ˥ | tʰi˧-di˩!}}} \zh{沼泽里,只长野草!} \textcolor{Sepia}{\selectlanguage{english}In marshes, there are clumps of wild herbs!} \textcolor{PineGreen}{\selectlanguage{french}sur les terres marécageuses, il ne pousse que des petites touffes d'herbe!}  
 \zh{量词}: \textcolor{darkblue}{\textbf{\ipa{pʰæ˧˥}}} 
\lhead{\firstmark}
\rhead{\botmark}

\subsection{\hspace{-0.5cm} {\Large \textcolor{darkblue}{\textbf{\ipa{tsʰe˧ɬi˧mi˧}}}}\hspace{0.5cm}[\kern2pt{\textcolor{darkblue}{\textbf{\ipa{tsʰe˧ɬi˧mi˧}}}}\kern2pt]} \hypertarget{ts\string_he\string_MKi\string_Mmi\string_M1}{}
\markboth{\textcolor{darkblue}{\textbf{\ipa{tsʰe˧ɬi˧mi˧}}}}{}
\textcolor{teal}{\zh{名词}} \hspace{4pt} \zh{声调类:} M.
\zh{十月。} \textcolor{Sepia}{\selectlanguage{english}10th month.} \textcolor{PineGreen}{\selectlanguage{french}10e mois.} 
\lhead{\firstmark}
\rhead{\botmark}

\subsection{\hspace{-0.5cm} {\Large \textcolor{darkblue}{\textbf{\ipa{tsʰe˧qʰɑ˩}}}}\hspace{0.5cm}[\kern2pt{\textcolor{darkblue}{\textbf{\ipa{tsʰe˧qʰɑ˩}}}}\kern2pt]} \hypertarget{ts\string_he\string_Mq\string_hA\string_B1}{}
\markboth{\textcolor{darkblue}{\textbf{\ipa{tsʰe˧qʰɑ˩}}}}{}
\textcolor{teal}{\zh{形容词}} \hspace{4pt} \zh{声调类:} L\#.
\zh{太咸。} \textcolor{Sepia}{\selectlanguage{english}Too salty.} \textcolor{PineGreen}{\selectlanguage{french}Trop salé.} \zh{~【参考】~} \hyperlink{}{\textcolor{darkblue}{\textbf{\ipa{tsʰe˧so˧˥}}}} 
\lhead{\firstmark}
\rhead{\botmark}

\subsection{\hspace{-0.5cm} {\Large \textcolor{darkblue}{\textbf{\ipa{tsʰe˧so˧˥}}}}\hspace{0.5cm}[\kern2pt{\textcolor{darkblue}{\textbf{\ipa{tsʰe˧so˧˥}}}}\kern2pt]} \hypertarget{ts\string_he\string_Mso\string_M\string_T1}{}
\markboth{\textcolor{darkblue}{\textbf{\ipa{tsʰe˧so˧˥}}}}{}
\textcolor{teal}{\zh{形容词}} \hspace{4pt} \zh{声调类:} MH\#.
\zh{咸。} \textcolor{Sepia}{\selectlanguage{english}Salty (pleasantly salty).} \textcolor{PineGreen}{\selectlanguage{french}Salé (agréablement: salé à point).} \zh{~【参考】~} \hyperlink{}{\textcolor{darkblue}{\textbf{\ipa{tsʰe˧qʰɑ˩}}}} 
\lhead{\firstmark}
\rhead{\botmark}

\subsection{\hspace{-0.5cm} {\Large \textcolor{darkblue}{\textbf{\ipa{tsʰe˧tʰv̩\#˥}}}}\hspace{0.5cm}[\kern2pt{\textcolor{darkblue}{\textbf{\ipa{tsʰe˧tʰv̩˧}}}}\kern2pt]} \hypertarget{ts\string_he\string_Mt\string_hv\string_=\#\string_T1}{}
\markboth{\textcolor{darkblue}{\textbf{\ipa{tsʰe˧tʰv̩\#˥}}}}{}
\textcolor{teal}{\zh{名词}} \hspace{4pt} \zh{声调类:} \#H.
\ding{202} \zh{天花。} \textcolor{Sepia}{\selectlanguage{english}Smallpox.} \textcolor{PineGreen}{\selectlanguage{french}Variole, petite vérole.}  ¶ \textcolor{darkblue}{\textbf{\ipa{tsʰe˧tʰv̩˧ | bæ˩bæ˩ bæ˥-ze˩}}} \zh{天花/麻疹犯了。} \textcolor{Sepia}{\selectlanguage{english}Smallpox has broken out.} \textcolor{PineGreen}{\selectlanguage{french}La variole s'est déclarée.}  
 \zh{量词}: \textcolor{darkblue}{\textbf{\ipa{ʂɯ˩}}} \ding{203} \zh{麻疹,疹子。} \textcolor{Sepia}{\selectlanguage{english}Measles.} \textcolor{PineGreen}{\selectlanguage{french}Rougeole.}  ¶ \textcolor{darkblue}{\textbf{\ipa{gv̩˧-kʰv̩˩ mɤ˩-gv̩˩, | tsʰe˧ mɤ˧-tʰv̩˧, | hĩ˧ ʈʂɤ˧-mɤ˧-kv̩˩!}}} \zh{九岁前不得麻疹,不能成人! / 得麻疹,就是小孩生长过程中必须要的一件事情!} \textcolor{Sepia}{\selectlanguage{english}If one does not catch the measles before age nine, one cannot become an adult (literally 'a person')!} \textcolor{PineGreen}{\selectlanguage{french}Si, avant l'âge de neuf ans, on ne contracte pas la rougeole, on ne peut pas devenir adulte! / Attraper la rougeole, ça fait partie du processus de croissance vers l'âge adulte!}  

\lhead{\firstmark}
\rhead{\botmark}

\subsection{\hspace{-0.5cm} {\Large \textcolor{darkblue}{\textbf{\ipa{tsʰe˧ʈʂæ˧}}}}\hspace{0.5cm}[\kern2pt{\textcolor{darkblue}{\textbf{\ipa{tsʰe˧ʈʂæ˧}}}}\kern2pt]} \hypertarget{ts\string_he\string_Mt`s`\{\string_M1}{}
\markboth{\textcolor{darkblue}{\textbf{\ipa{tsʰe˧ʈʂæ˧}}}}{}
\textcolor{teal}{\zh{名词}} \hspace{4pt} \zh{声调类:} M.
\zh{村长。} \textcolor{Sepia}{\selectlanguage{english}Village head, small-ranking official.} \textcolor{PineGreen}{\selectlanguage{french}Chef de village, petit officiel.}  \zh{【借词】} \zh{村长}

\lhead{\firstmark}
\rhead{\botmark}

\subsection{\hspace{-0.5cm} {\Large \textcolor{darkblue}{\textbf{\ipa{tsʰe˩\textsubscript{b}}}}}\hspace{0.5cm}[\kern2pt{\textcolor{darkblue}{\textbf{\ipa{tsʰe˩˥}}}}\kern2pt]} \hypertarget{ts\string_he\string_Bb1}{}
\markboth{\textcolor{darkblue}{\textbf{\ipa{tsʰe˩\textsubscript{b}}}}}{}
\textcolor{teal}{\zh{量词}} \hspace{4pt} \zh{声调类:} L\textsubscript{b}.
\zh{量词:寸(汉语借词)。} \textcolor{Sepia}{\selectlanguage{english}Classifier: an inch (1/3 decimeter).} \textcolor{PineGreen}{\selectlanguage{french}Pouce (cette unité de mesure n'était pas en usage chez les Na avant son introduction par emprunt au chinois).}  \zh{【借词】} \zh{寸}

\lhead{\firstmark}
\rhead{\botmark}

\subsection{\hspace{-0.5cm} {\Large \textcolor{darkblue}{\textbf{\ipa{tsʰe˩\textsubscript{b}}}}}\hspace{0.5cm}[\kern2pt{\textcolor{darkblue}{\textbf{\ipa{tsʰe˩˥}}}}\kern2pt]} \hypertarget{ts\string_he\string_Bb1}{}
\markboth{\textcolor{darkblue}{\textbf{\ipa{tsʰe˩\textsubscript{b}}}}}{}
\textcolor{teal}{\zh{量词}} \hspace{4pt} \zh{声调类:} L\textsubscript{b}.
\zh{量词:数辫子的节(一节)。} \textcolor{Sepia}{\selectlanguage{english}Classifier for knots, e.g. in a braid.} \textcolor{PineGreen}{\selectlanguage{french}Classificateur des nœuds dans une tresse.} 
\lhead{\firstmark}
\rhead{\botmark}

\subsection{\hspace{-0.5cm} {\Large \textcolor{darkblue}{\textbf{\ipa{tsʰe˩gv̩˩}}}}\hspace{0.5cm}[\kern2pt{\textcolor{darkblue}{\textbf{\ipa{tsʰe˩gv̩˩˥}}}}\kern2pt]} \hypertarget{ts\string_he\string_Bgv\string_=\string_B1}{}
\markboth{\textcolor{darkblue}{\textbf{\ipa{tsʰe˩gv̩˩}}}}{}
\textcolor{teal}{\zh{数词}} \hspace{4pt} \zh{声调类:} L.
\zh{19。} \textcolor{Sepia}{\selectlanguage{english}19.} \textcolor{PineGreen}{\selectlanguage{french}19.} 
\lhead{\firstmark}
\rhead{\botmark}

\subsection{\hspace{-0.5cm} {\Large \textcolor{darkblue}{\textbf{\ipa{tsʰe˩hõ˩}}}}\hspace{0.5cm}[\kern2pt{\textcolor{darkblue}{\textbf{\ipa{tsʰe˩hõ˩˥}}}}\kern2pt]} \hypertarget{ts\string_he\string_Bho\string_~\string_B1}{}
\markboth{\textcolor{darkblue}{\textbf{\ipa{tsʰe˩hõ˩}}}}{}
\textcolor{teal}{\zh{数词}} \hspace{4pt} \zh{声调类:} L.
\zh{18。} \textcolor{Sepia}{\selectlanguage{english}18.} \textcolor{PineGreen}{\selectlanguage{french}18.} 
\lhead{\firstmark}
\rhead{\botmark}

\subsection{\hspace{-0.5cm} {\Large \textcolor{darkblue}{\textbf{\ipa{tsʰe˩ŋwɤ˩}}}}\hspace{0.5cm}[\kern2pt{\textcolor{darkblue}{\textbf{\ipa{tsʰe˩ŋwɤ˩˥}}}}\kern2pt]} \hypertarget{ts\string_he\string_BNw7\string_B1}{}
\markboth{\textcolor{darkblue}{\textbf{\ipa{tsʰe˩ŋwɤ˩}}}}{}
\textcolor{teal}{\zh{数词}} \hspace{4pt} \zh{声调类:} L.
\zh{15。} \textcolor{Sepia}{\selectlanguage{english}15.} \textcolor{PineGreen}{\selectlanguage{french}15.} 
\lhead{\firstmark}
\rhead{\botmark}

\subsection{\hspace{-0.5cm} {\Large \textcolor{darkblue}{\textbf{\ipa{tsʰe˩qʰv̩˩}}}}\hspace{0.5cm}[\kern2pt{\textcolor{darkblue}{\textbf{\ipa{tsʰe˩qʰv̩˩˥}}}}\kern2pt]} \hypertarget{ts\string_he\string_Bq\string_hv\string_=\string_B1}{}
\markboth{\textcolor{darkblue}{\textbf{\ipa{tsʰe˩qʰv̩˩}}}}{}
\textcolor{teal}{\zh{数词}} \hspace{4pt} \zh{声调类:} L.
\zh{16。} \textcolor{Sepia}{\selectlanguage{english}16.} \textcolor{PineGreen}{\selectlanguage{french}16.} 
\lhead{\firstmark}
\rhead{\botmark}

\subsection{\hspace{-0.5cm} {\Large \textcolor{darkblue}{\textbf{\ipa{tsʰe˩ʐv̩˩}}}}\hspace{0.5cm}[\kern2pt{\textcolor{darkblue}{\textbf{\ipa{tsʰe˩ʐv̩˩˥}}}}\kern2pt]} \hypertarget{ts\string_he\string_Bz`v\string_=\string_B1}{}
\markboth{\textcolor{darkblue}{\textbf{\ipa{tsʰe˩ʐv̩˩}}}}{}
\textcolor{teal}{\zh{数词}} \hspace{4pt} \zh{声调类:} L.
\zh{14。} \textcolor{Sepia}{\selectlanguage{english}14.} \textcolor{PineGreen}{\selectlanguage{french}14.} 
\lhead{\firstmark}
\rhead{\botmark}

\subsection{\hspace{-0.5cm} {\Large \textcolor{darkblue}{\textbf{\ipa{tsʰɤ˩\textsubscript{a}}}}}\hspace{0.5cm}[\kern2pt{\textcolor{darkblue}{\textbf{\ipa{tsʰɤ˩˥}}}}\kern2pt]} \hypertarget{ts\string_h7\string_Ba1}{}
\markboth{\textcolor{darkblue}{\textbf{\ipa{tsʰɤ˩\textsubscript{a}}}}}{}
\textcolor{teal}{\zh{动词}} \hspace{4pt} \zh{声调类:} L\textsubscript{a}.
\zh{编(头发,线)。} \textcolor{Sepia}{\selectlanguage{english}To plait, to weave (hair, thread).} \textcolor{PineGreen}{\selectlanguage{french}Tresser (les cheveux, fils).}  ¶ \textcolor{darkblue}{\textbf{\ipa{ʁo˧qʰwɤ˩ tsʰɤ˩}}} \zh{编辫子} \textcolor{Sepia}{\selectlanguage{english}to weave the hair} \textcolor{PineGreen}{\selectlanguage{french}tresser les cheveux, littéralement “tresser la tête”}  
 ¶ \textcolor{darkblue}{\textbf{\ipa{hæ̃˧pɤ˧ le˧-tsʰɤ˩}}} \zh{梳一条辫子} \textcolor{Sepia}{\selectlanguage{english}to plait hair} \textcolor{PineGreen}{\selectlanguage{french}faire une tresse}  
 ¶ \textcolor{darkblue}{\textbf{\ipa{ɖɯ˧-tsʰɤ˧\textasciitilde{}tsʰɤ˥-ɻ̍˩}}} \zh{\mytextsc{delimitative} \string_ \mytextsc{red} \mytextsc{inceptive}} \textcolor{Sepia}{\selectlanguage{english}\mytextsc{delimitative} \string_ \mytextsc{red} \mytextsc{inceptive}} \textcolor{PineGreen}{\selectlanguage{french}\mytextsc{délimitatif} \string_ \mytextsc{red} \mytextsc{inchoatif}}  

\lhead{\firstmark}
\rhead{\botmark}

\subsection{\hspace{-0.5cm} {\Large \textcolor{darkblue}{\textbf{\ipa{tsʰɤ˧˥}}} \textsubscript{1}}\hspace{0.5cm}[\kern2pt{\textcolor{darkblue}{\textbf{\ipa{tsʰɤ˧˥}}}}\kern2pt]} \hypertarget{ts\string_h7\string_M\string_T1}{}
\markboth{\textcolor{darkblue}{\textbf{\ipa{tsʰɤ˧˥}}} \textsubscript{1}}{}
\textcolor{teal}{\zh{动词}} \hspace{4pt} \zh{声调类:} MH.
\zh{挤奶。} \textcolor{Sepia}{\selectlanguage{english}To milk.} \textcolor{PineGreen}{\selectlanguage{french}Traire (vache, brebis).}  ¶ \textcolor{darkblue}{\textbf{\ipa{tso˧\textasciitilde{}tso˧ tsʰɤ˩ + ze˩}}} \zh{挤出东西} \textcolor{Sepia}{\selectlanguage{english}to milk things} \textcolor{PineGreen}{\selectlanguage{french}traire des choses}  
 ¶ \textcolor{darkblue}{\textbf{\ipa{ʝi˧-bv̩˧ | ɳæ˧ tsʰɤ˩}}} \zh{挤牛奶} \textcolor{Sepia}{\selectlanguage{english}to milk a cow} \textcolor{PineGreen}{\selectlanguage{french}traire (le lait de) la vache}  

\lhead{\firstmark}
\rhead{\botmark}

\subsection{\hspace{-0.5cm} {\Large \textcolor{darkblue}{\textbf{\ipa{tsʰɤ˧˥}}} \textsubscript{2}}\hspace{0.5cm}[\kern2pt{\textcolor{darkblue}{\textbf{\ipa{tsʰɤ˧˥}}}}\kern2pt]} \hypertarget{ts\string_h7\string_M\string_T2}{}
\markboth{\textcolor{darkblue}{\textbf{\ipa{tsʰɤ˧˥}}} \textsubscript{2}}{}
\textcolor{teal}{\zh{动词}} \hspace{4pt} \zh{声调类:} MH.
\zh{摩擦。} \textcolor{Sepia}{\selectlanguage{english}To rub (e.g. rough fabric rubs against the skin).} \textcolor{PineGreen}{\selectlanguage{french}Frotter, faire une friction (ex.: un tissu grossier frotte sur la peau, et l'irrite).} 
\lhead{\firstmark}
\rhead{\botmark}

\subsection{\hspace{-0.5cm} {\Large \textcolor{darkblue}{\textbf{\ipa{tsʰɤ˧˥}}} \textsubscript{3}}\hspace{0.5cm}[\kern2pt{\textcolor{darkblue}{\textbf{\ipa{tsʰɤ˧˥}}}}\kern2pt]} \hypertarget{ts\string_h7\string_M\string_T3}{}
\markboth{\textcolor{darkblue}{\textbf{\ipa{tsʰɤ˧˥}}} \textsubscript{3}}{}
\textcolor{teal}{\zh{动词}} \hspace{4pt} \zh{声调类:} MH.
\zh{抢。} \textcolor{Sepia}{\selectlanguage{english}To attack, to pillage (e.g. bandits attack a caravan).} \textcolor{PineGreen}{\selectlanguage{french}Attaquer, piller, s'en prendre à (des brigands attaquent un convoi).} 
\lhead{\firstmark}
\rhead{\botmark}

\subsection{\hspace{-0.5cm} {\Large \textcolor{darkblue}{\textbf{\ipa{tsʰɤ˧˥}}} \textsubscript{4}}\hspace{0.5cm}[\kern2pt{\textcolor{darkblue}{\textbf{\ipa{tsʰɤ˧˥}}}}\kern2pt]} \hypertarget{ts\string_h7\string_M\string_T4}{}
\markboth{\textcolor{darkblue}{\textbf{\ipa{tsʰɤ˧˥}}} \textsubscript{4}}{}
\textcolor{teal}{\zh{动词}} \hspace{4pt} \zh{声调类:} MH.
\zh{还(东西)。} \textcolor{Sepia}{\selectlanguage{english}To give back, to return.} \textcolor{PineGreen}{\selectlanguage{french}Rendre.} 
\lhead{\firstmark}
\rhead{\botmark}

\subsection{\hspace{-0.5cm} {\Large \textcolor{darkblue}{\textbf{\ipa{tsʰɤ˧˥\textsubscript{a}}}}}\hspace{0.5cm}[\kern2pt{\textcolor{darkblue}{\textbf{\ipa{tsʰɤ˧˥}}}}\kern2pt]} \hypertarget{ts\string_h7\string_M\string_Ta1}{}
\markboth{\textcolor{darkblue}{\textbf{\ipa{tsʰɤ˧˥\textsubscript{a}}}}}{}
\textcolor{teal}{\zh{量词}} \hspace{4pt} \zh{声调类:} MH\textsubscript{a}.
\zh{量词:凸凹的物品,如:鸡冠(一顶)、叶子(一片)、蒜(一头)。} \textcolor{Sepia}{\selectlanguage{english}Classifier for dented or bumpy objects: cockscombs, leaves, and bulbs of garlic.} \textcolor{PineGreen}{\selectlanguage{french}Classificateur des objets bosselés: feuilles, crêtes de coq, têtes d'ail (une tête d'ail se divise en gousses un peu comme on effeuille une branche ou une fleur).}  ¶ \textcolor{darkblue}{\textbf{\ipa{bæ˩bæ˩˥ | ɖɯ˧-tsʰɤ˧˥}}} \zh{一朵花} \textcolor{Sepia}{\selectlanguage{english}a flower} \textcolor{PineGreen}{\selectlanguage{french}une fleur}  

\lhead{\firstmark}
\rhead{\botmark}

\subsection{\hspace{-0.5cm} {\Large \textcolor{darkblue}{\textbf{\ipa{tsʰi\#˥}}}}\hspace{0.5cm}[\kern2pt{\textcolor{darkblue}{\textbf{\ipa{tsʰi˥}}}}\kern2pt]} \hypertarget{ts\string_hi\#\string_T1}{}
\markboth{\textcolor{darkblue}{\textbf{\ipa{tsʰi\#˥}}}}{}
\textcolor{teal}{\zh{名词}} \hspace{4pt} \zh{声调类:} \#H.
\zh{旱季(冬天至春天:农历九月到来年二月)。} \textcolor{Sepia}{\selectlanguage{english}Dry season (winter and spring: from the 9th lunar month to the 2nd lunar month).} \textcolor{PineGreen}{\selectlanguage{french}Saison sèche (hiver et printemps; du 9e mois au 2e mois du calendrier lunaire compris).} 
\lhead{\firstmark}
\rhead{\botmark}

\subsection{\hspace{-0.5cm} {\Large \textcolor{darkblue}{\textbf{\ipa{tsʰi˥\textsubscript{a}}}}}\hspace{0.5cm}[\kern2pt{\textcolor{darkblue}{\textbf{\ipa{tsʰi˥}}}}\kern2pt]} \hypertarget{ts\string_hi\string_Ta1}{}
\markboth{\textcolor{darkblue}{\textbf{\ipa{tsʰi˥\textsubscript{a}}}}}{}
\textcolor{teal}{\zh{量词}} \hspace{4pt} \zh{声调类:} H\textsubscript{a}.
\zh{量词:动物皮(一张),布料(一块)。} \textcolor{Sepia}{\selectlanguage{english}Classifier for pelts / hides (animal skins), and for pieces of fabric.} \textcolor{PineGreen}{\selectlanguage{french}Classificateur des peaux d'animaux, et des pièces de tissu.}  ¶ \textcolor{darkblue}{\textbf{\ipa{ɖɯ˧-tsʰi˥}}} \zh{一张动物皮} \textcolor{Sepia}{\selectlanguage{english}one pelt} \textcolor{PineGreen}{\selectlanguage{french}une peau}  
 ¶ \textcolor{darkblue}{\textbf{\ipa{ɖɯ˧-tsʰi˧ ɲi˥}}} \zh{这是一张(动物皮)} \textcolor{Sepia}{\selectlanguage{english}It's a pelt.} \textcolor{PineGreen}{\selectlanguage{french}c'est une peau}  

\lhead{\firstmark}
\rhead{\botmark}

\subsection{\hspace{-0.5cm} {\Large \textcolor{darkblue}{\textbf{\ipa{tsʰi˧}}} \textsubscript{1}}\hspace{0.5cm}[\kern2pt{\textcolor{darkblue}{\textbf{\ipa{tsʰi˥}}}}\kern2pt]} \hypertarget{ts\string_hi\string_M1}{}
\markboth{\textcolor{darkblue}{\textbf{\ipa{tsʰi˧}}} \textsubscript{1}}{}
\textcolor{teal}{\zh{形容词}} \hspace{4pt} \zh{声调类:} M.
\zh{热,烫。} \textcolor{Sepia}{\selectlanguage{english}Hot; scalding.} \textcolor{PineGreen}{\selectlanguage{french}Chaud.}  ¶ \textcolor{darkblue}{\textbf{\ipa{tsʰi˧-zo˧ mɤ˧-tʰɑ˧˥!}}} \zh{热得受不了!} \textcolor{Sepia}{\selectlanguage{english}The heat is unbearable! / The weather is unbearably hot!} \textcolor{PineGreen}{\selectlanguage{french}il fait une chaleur insupportable!}  
\zh{~【参考】~} \hyperlink{}{\textcolor{darkblue}{\textbf{\ipa{tsʰi˧}}} \textsubscript{2}} 
\lhead{\firstmark}
\rhead{\botmark}

\subsection{\hspace{-0.5cm} {\Large \textcolor{darkblue}{\textbf{\ipa{tsʰi˧}}} \textsubscript{2}}\hspace{0.5cm}[\kern2pt{\textcolor{darkblue}{\textbf{\ipa{tsʰi˥}}}}\kern2pt]} \hypertarget{ts\string_hi\string_M2}{}
\markboth{\textcolor{darkblue}{\textbf{\ipa{tsʰi˧}}} \textsubscript{2}}{}
\textcolor{teal}{\zh{形容词}} \hspace{4pt} \zh{声调类:} M.
\zh{明亮。} \textcolor{Sepia}{\selectlanguage{english}Bright.} \textcolor{PineGreen}{\selectlanguage{french}Brillant, lumineux, ardent.}  ¶ \textcolor{darkblue}{\textbf{\ipa{ɲi˧mi˧ tsʰi˧}}} \zh{太阳很晒} \textcolor{Sepia}{\selectlanguage{english}the sun shines, the sunlight is strong} \textcolor{PineGreen}{\selectlanguage{french}le soleil est très fort/le soleil tape dur}  
 ¶ \textcolor{darkblue}{\textbf{\ipa{ɬi˧mi˧ tsʰi˧}}} \zh{月亮很亮、月光很明亮} \textcolor{Sepia}{\selectlanguage{english}the moon shines, the moonlight is strong} \textcolor{PineGreen}{\selectlanguage{french}la lune brille, la lune luit, on y voit clair à la lumière de la lune}  
\zh{~【参考】~} \hyperlink{}{\textcolor{darkblue}{\textbf{\ipa{tsʰi˧}}} \textsubscript{1}} 
\lhead{\firstmark}
\rhead{\botmark}

\subsection{\hspace{-0.5cm} {\Large \textcolor{darkblue}{\textbf{\ipa{tsʰi˧\textsubscript{b}}}}}\hspace{0.5cm}[\kern2pt{\textcolor{darkblue}{\textbf{\ipa{tsʰi˥}}}}\kern2pt]} \hypertarget{ts\string_hi\string_Mb1}{}
\markboth{\textcolor{darkblue}{\textbf{\ipa{tsʰi˧\textsubscript{b}}}}}{}
\textcolor{teal}{\zh{动词}} \hspace{4pt} \zh{声调类:} M\textsubscript{b}.
\zh{戴帽子。} \textcolor{Sepia}{\selectlanguage{english}To wear (a hat).} \textcolor{PineGreen}{\selectlanguage{french}Porter (un chapeau).}  ¶ \textcolor{darkblue}{\textbf{\ipa{tv̩˧tv̩˥ tsʰi˩}}} \zh{戴上帽子} \textcolor{Sepia}{\selectlanguage{english}to put on a hat} \textcolor{PineGreen}{\selectlanguage{french}mettre un chapeau}  

\lhead{\firstmark}
\rhead{\botmark}

\subsection{\hspace{-0.5cm} {\Large \textcolor{darkblue}{\textbf{\ipa{tsʰi˧bv̩˩}}}}\hspace{0.5cm}[\kern2pt{\textcolor{darkblue}{\textbf{\ipa{tsʰi˧bv̩˩}}}}\kern2pt]} \hypertarget{ts\string_hi\string_Mbv\string_=\string_B1}{}
\markboth{\textcolor{darkblue}{\textbf{\ipa{tsʰi˧bv̩˩}}}}{}
\textcolor{teal}{\zh{形容词}} \hspace{4pt} \zh{声调类:} L\#.
\zh{闷热。} \textcolor{Sepia}{\selectlanguage{english}Muggy, sultry, oppressively hot.} \textcolor{PineGreen}{\selectlanguage{french}Étouffant.} 
\lhead{\firstmark}
\rhead{\botmark}

\subsection{\hspace{-0.5cm} {\Large \textcolor{darkblue}{\textbf{\ipa{tsʰi˧ʝi\#˥}}}}\hspace{0.5cm}[\kern2pt{\textcolor{darkblue}{\textbf{\ipa{tsʰi˧ʝi˧}}}}\kern2pt]} \hypertarget{ts\string_hi\string_Mj££i\#\string_T1}{}
\markboth{\textcolor{darkblue}{\textbf{\ipa{tsʰi˧ʝi\#˥}}}}{}
\textcolor{teal}{\zh{助词}} \hspace{4pt} \zh{声调类:} \#H.
\zh{今年。} \textcolor{Sepia}{\selectlanguage{english}This year.} \textcolor{PineGreen}{\selectlanguage{french}Cette année.}  ¶ \textcolor{darkblue}{\textbf{\ipa{tsʰi˧ʝi˧-se˥, | …}}} \zh{到了今年,……} \textcolor{Sepia}{\selectlanguage{english}Until this year, ...} \textcolor{PineGreen}{\selectlanguage{french}Jusqu'à cette année, ...}  

\lhead{\firstmark}
\rhead{\botmark}

\subsection{\hspace{-0.5cm} {\Large \textcolor{darkblue}{\textbf{\ipa{tsʰi˧ɲi\#˥}}}}\hspace{0.5cm}[\kern2pt{\textcolor{darkblue}{\textbf{\ipa{tsʰi˧ɲi˧}}}}\kern2pt]} \hypertarget{ts\string_hi\string_MJi\#\string_T1}{}
\markboth{\textcolor{darkblue}{\textbf{\ipa{tsʰi˧ɲi\#˥}}}}{}
\textcolor{teal}{\zh{助词}} \hspace{4pt} \zh{声调类:} \#H.
\zh{今天。} \textcolor{Sepia}{\selectlanguage{english}Today.} \textcolor{PineGreen}{\selectlanguage{french}Aujourd'hui.}  ¶ \textcolor{darkblue}{\textbf{\ipa{tsʰi˧ɲi˧-ʁo˧dɑ˧}}} \zh{今天之前} \textcolor{Sepia}{\selectlanguage{english}before today; previously} \textcolor{PineGreen}{\selectlanguage{french}avant ce jour, avant aujourd'hui; précédemment}  

\lhead{\firstmark}
\rhead{\botmark}

\subsection{\hspace{-0.5cm} {\Large \textcolor{darkblue}{\textbf{\ipa{tsʰi˧qʰæ˧˥}}}}\hspace{0.5cm}[\kern2pt{\textcolor{darkblue}{\textbf{\ipa{tsʰi˧qʰæ˧˥}}}}\kern2pt]} \hypertarget{ts\string_hi\string_Mq\string_h\{\string_M\string_T1}{}
\markboth{\textcolor{darkblue}{\textbf{\ipa{tsʰi˧qʰæ˧˥}}}}{}
\textcolor{teal}{\zh{助词}} \hspace{4pt} \zh{声调类:} MH.
\zh{现在。} \textcolor{Sepia}{\selectlanguage{english}Now, currently, these days.} \textcolor{PineGreen}{\selectlanguage{french}En ce moment, actuellement, maintenant.} 
\lhead{\firstmark}
\rhead{\botmark}

\subsection{\hspace{-0.5cm} {\Large \textcolor{darkblue}{\textbf{\ipa{tsʰi˧si˩-dʑɤ˩pv̩˩}}}}\hspace{0.5cm}[\kern2pt{\textcolor{darkblue}{\textbf{\ipa{tsʰi˧si˩dʑɤ˧pv̩˧}}}}\kern2pt]} \hypertarget{ts\string_hi\string_Msi\string_B-dz£7\string_Bpv\string_=\string_B1}{}
\markboth{\textcolor{darkblue}{\textbf{\ipa{tsʰi˧si˩-dʑɤ˩pv̩˩}}}}{}
\textcolor{teal}{\zh{名词}} \hspace{4pt} \zh{声调类:} L\#-.
\zh{神灵的世界、死人的世界。} \textcolor{Sepia}{\selectlanguage{english}The world of spirits, the world of the dead.} \textcolor{PineGreen}{\selectlanguage{french}Le monde des esprits, le monde des morts.} 
\lhead{\firstmark}
\rhead{\botmark}

\subsection{\hspace{-0.5cm} {\Large \textcolor{darkblue}{\textbf{\ipa{tsʰi˧ti\#˥}}}}\hspace{0.5cm}[\kern2pt{\textcolor{darkblue}{\textbf{\ipa{tsʰi˧ti˧}}}}\kern2pt]} \hypertarget{ts\string_hi\string_Mti\#\string_T1}{}
\markboth{\textcolor{darkblue}{\textbf{\ipa{tsʰi˧ti\#˥}}}}{}
\textcolor{teal}{\zh{名词}} \hspace{4pt} \zh{声调类:} \#H.
\zh{男性名字。} \textcolor{Sepia}{\selectlanguage{english}Masculine given name.} \textcolor{PineGreen}{\selectlanguage{french}Prénom masculin.} 
\lhead{\firstmark}
\rhead{\botmark}

\subsection{\hspace{-0.5cm} {\Large \textcolor{darkblue}{\textbf{\ipa{tsʰi˧zi\#˥}}}}\hspace{0.5cm}[\kern2pt{\textcolor{darkblue}{\textbf{\ipa{tsʰi˧zi˧}}}}\kern2pt]} \hypertarget{ts\string_hi\string_Mzi\#\string_T1}{}
\markboth{\textcolor{darkblue}{\textbf{\ipa{tsʰi˧zi\#˥}}}}{}
\textcolor{teal}{\zh{名词}} \hspace{4pt} \zh{声调类:} \#H.
\zh{青稞。} \textcolor{Sepia}{\selectlanguage{english}Highland barley, \textit{Hordeum vulgare var. nudum Hook. f.}.} \textcolor{PineGreen}{\selectlanguage{french}Orge d'altitude, \textit{Hordeum vulgare var. nudum Hook. f.}.}  ¶ \textcolor{darkblue}{\textbf{\ipa{tsʰi˧zi˧ | nɑ˩-hĩ˩˥}}} \zh{黑青稞} \textcolor{Sepia}{\selectlanguage{english}black barley} \textcolor{PineGreen}{\selectlanguage{french}orge noir}  
 ¶ \textcolor{darkblue}{\textbf{\ipa{tsʰi˧zi˧ | pʰv̩˩-hĩ˩˥}}} \zh{白青稞} \textcolor{Sepia}{\selectlanguage{english}white barley} \textcolor{PineGreen}{\selectlanguage{french}orge blanc}  
 \zh{量词}: \textcolor{darkblue}{\textbf{\ipa{kɤ˧˥}}} 
\lhead{\firstmark}
\rhead{\botmark}

\subsection{\hspace{-0.5cm} {\Large \textcolor{darkblue}{\textbf{\ipa{tsʰi˧zi˧-ɻ̃\#˥}}}}\hspace{0.5cm}[\kern2pt{\textcolor{darkblue}{\textbf{\ipa{xxxx non-correspondance entre le nombre de morphèmes et le nombre de tons de morphèmes}}}}\kern2pt]} \hypertarget{ts\string_hi\string_Mzi\string_M-r£`\string_~\#\string_T1}{}
\markboth{\textcolor{darkblue}{\textbf{\ipa{tsʰi˧zi˧-ɻ̃\#˥}}}}{}
\textcolor{teal}{\zh{名词}} \hspace{4pt} \zh{声调类:} \#H.
\zh{青稞杆。} \textcolor{Sepia}{\selectlanguage{english}Highland barley straw.} \textcolor{PineGreen}{\selectlanguage{french}Paille d'orge.}  \zh{量词}: \textcolor{darkblue}{\textbf{\ipa{kɤ˧˥}}} 
\lhead{\firstmark}
\rhead{\botmark}

\subsection{\hspace{-0.5cm} {\Large \textcolor{darkblue}{\textbf{\ipa{tsʰi˩\textsubscript{a}}}}}\hspace{0.5cm}[\kern2pt{\textcolor{darkblue}{\textbf{\ipa{tsʰi˩˥}}}}\kern2pt]} \hypertarget{ts\string_hi\string_Ba1}{}
\markboth{\textcolor{darkblue}{\textbf{\ipa{tsʰi˩\textsubscript{a}}}}}{}
\textcolor{teal}{\zh{形容词}} \hspace{4pt} \zh{声调类:} L\textsubscript{a}.
\zh{细(树、体型细小)。} \textcolor{Sepia}{\selectlanguage{english}Fine, thin.} \textcolor{PineGreen}{\selectlanguage{french}Fin (objet).}  ¶ \textcolor{darkblue}{\textbf{\ipa{tsʰi˩-hĩ˩˥}}} \zh{细的} \textcolor{Sepia}{\selectlanguage{english}\mytextsc{nmlz}} \textcolor{PineGreen}{\selectlanguage{french}\mytextsc{nmlz}}  
 ¶ \textcolor{darkblue}{\textbf{\ipa{qʰɑ˧-tsʰi˧-gv̩˧}}} \zh{非常细} \textcolor{Sepia}{\selectlanguage{english}very thin} \textcolor{PineGreen}{\selectlanguage{french}très fin}  
 ¶ \textcolor{darkblue}{\textbf{\ipa{dʑɤ˧˥ | tsʰi˩-njæ˩˥ | -gv̩˩!}}} \zh{真细!} \textcolor{Sepia}{\selectlanguage{english}It is really thin!} \textcolor{PineGreen}{\selectlanguage{french}C'est vraiment fin!}  

\lhead{\firstmark}
\rhead{\botmark}

\subsection{\hspace{-0.5cm} {\Large \textcolor{darkblue}{\textbf{\ipa{tsʰi˩mv̩˩tʰv̩˩}}}}\hspace{0.5cm}[\kern2pt{\textcolor{darkblue}{\textbf{\ipa{tsʰi˩mv̩˩tʰv̩˩˥}}}}\kern2pt]} \hypertarget{ts\string_hi\string_Bmv\string_=\string_Bt\string_hv\string_=\string_B1}{}
\markboth{\textcolor{darkblue}{\textbf{\ipa{tsʰi˩mv̩˩tʰv̩˩}}}}{}
\textcolor{teal}{\zh{名词}} \hspace{4pt} \zh{声调类:} L.
\zh{跳着的鬼。} \textcolor{Sepia}{\selectlanguage{english}Dancing demon.} \textcolor{PineGreen}{\selectlanguage{french}Démon qui danse.} 
\lhead{\firstmark}
\rhead{\botmark}

\subsection{\hspace{-0.5cm} {\Large \textcolor{darkblue}{\textbf{\ipa{tsʰi˩tv̩˩}}}}\hspace{0.5cm}[\kern2pt{\textcolor{darkblue}{\textbf{\ipa{tsʰi˩tv̩˩˥}}}}\kern2pt]} \hypertarget{ts\string_hi\string_Btv\string_=\string_B1}{}
\markboth{\textcolor{darkblue}{\textbf{\ipa{tsʰi˩tv̩˩}}}}{}
\textcolor{teal}{\zh{名词}} \hspace{4pt} \zh{声调类:} L.
\zh{骨髓。} \textcolor{Sepia}{\selectlanguage{english}Bone marrow.} \textcolor{PineGreen}{\selectlanguage{french}Moëlle.}  \zh{量词}: \textcolor{darkblue}{\textbf{\ipa{kʰwɤ˥}}} 
\lhead{\firstmark}
\rhead{\botmark}

\subsection{\hspace{-0.5cm} {\Large \textcolor{darkblue}{\textbf{\ipa{tsʰi˩tsʰi˩}}}}\hspace{0.5cm}[\kern2pt{\textcolor{darkblue}{\textbf{\ipa{tsʰi˩tsʰi˩˥}}}}\kern2pt]} \hypertarget{ts\string_hi\string_Bts\string_hi\string_B1}{}
\markboth{\textcolor{darkblue}{\textbf{\ipa{tsʰi˩tsʰi˩}}}}{}
\textcolor{teal}{\zh{名词}} \hspace{4pt} \zh{声调类:} L.
\zh{豌豆。} \textcolor{Sepia}{\selectlanguage{english}Peas, garden peas.} \textcolor{PineGreen}{\selectlanguage{french}Pois, petits pois.}  \zh{量词}: \textcolor{darkblue}{\textbf{\ipa{kɤ˧˥}}} 
\lhead{\firstmark}
\rhead{\botmark}

\subsection{\hspace{-0.5cm} {\Large \textcolor{darkblue}{\textbf{\ipa{tsʰi˧˥}}} \textsubscript{1}}\hspace{0.5cm}[\kern2pt{\textcolor{darkblue}{\textbf{\ipa{tsʰi˧˥}}}}\kern2pt]} \hypertarget{ts\string_hi\string_M\string_T1}{}
\markboth{\textcolor{darkblue}{\textbf{\ipa{tsʰi˧˥}}} \textsubscript{1}}{}
\textcolor{teal}{\zh{动词}} \hspace{4pt} \zh{声调类:} MH.
\zh{盖,建 (房子)。} \textcolor{Sepia}{\selectlanguage{english}To construct a house, to build a house.} \textcolor{PineGreen}{\selectlanguage{french}Construire.}  ¶ \textcolor{darkblue}{\textbf{\ipa{ʑi˧qʰwɤ˧ tsʰi˧˥}}} \zh{建 房子} \textcolor{Sepia}{\selectlanguage{english}to build a house} \textcolor{PineGreen}{\selectlanguage{french}construire un bâtiment}  

\lhead{\firstmark}
\rhead{\botmark}

\subsection{\hspace{-0.5cm} {\Large \textcolor{darkblue}{\textbf{\ipa{tsʰi˧˥}}} \textsubscript{2}}\hspace{0.5cm}[\kern2pt{\textcolor{darkblue}{\textbf{\ipa{tsʰi˧˥}}}}\kern2pt]} \hypertarget{ts\string_hi\string_M\string_T2}{}
\markboth{\textcolor{darkblue}{\textbf{\ipa{tsʰi˧˥}}} \textsubscript{2}}{}
\textcolor{teal}{\zh{动词}} \hspace{4pt} \zh{声调类:} MH.
\zh{穿一个洞。} \textcolor{Sepia}{\selectlanguage{english}To bore a hole, to punch a hole.} \textcolor{PineGreen}{\selectlanguage{french}Percer un trou, faire un trou (ex.: dans un tissu, un mur...).} 
\lhead{\firstmark}
\rhead{\botmark}

\subsection{\hspace{-0.5cm} {\Large \textcolor{darkblue}{\textbf{\ipa{tsʰi˧˥}}} \textsubscript{3}}\hspace{0.5cm}[\kern2pt{\textcolor{darkblue}{\textbf{\ipa{tsʰi˧˥}}}}\kern2pt]} \hypertarget{ts\string_hi\string_M\string_T3}{}
\markboth{\textcolor{darkblue}{\textbf{\ipa{tsʰi˧˥}}} \textsubscript{3}}{}
\textcolor{teal}{\zh{动词}} \hspace{4pt} \zh{声调类:} MH.
\zh{点(火)。} \textcolor{Sepia}{\selectlanguage{english}To start (a fire).} \textcolor{PineGreen}{\selectlanguage{french}Allumer (un feu).}  ¶ \textcolor{darkblue}{\textbf{\ipa{mv̩˧ tsʰi˧˥}}} \zh{点火} \textcolor{Sepia}{\selectlanguage{english}to start a fire} \textcolor{PineGreen}{\selectlanguage{french}allumer un feu}  
 ¶ \textcolor{darkblue}{\textbf{\ipa{njɤ˧-ɳɯ˧ | mv̩˧tsʰi˧-bi˥}}} \zh{我要点个火} \textcolor{Sepia}{\selectlanguage{english}I am going to start a fire} \textcolor{PineGreen}{\selectlanguage{french}je vais allumer le feu}  
 ¶ \textcolor{darkblue}{\textbf{\ipa{mv̩˩tsʰo˩ tsʰi˧}}} \zh{用含很多树脂的木头来引火} \textcolor{Sepia}{\selectlanguage{english}to put fire to a piece of wood full of sap (to start a fire)} \textcolor{PineGreen}{\selectlanguage{french}mettre le feu à un bout de bois plein de sève (pour faire partir le feu)}  

\lhead{\firstmark}
\rhead{\botmark}

\subsection{\hspace{-0.5cm} {\Large \textcolor{darkblue}{\textbf{\ipa{tsʰi˧˥}}} \textsubscript{4}}\hspace{0.5cm}[\kern2pt{\textcolor{darkblue}{\textbf{\ipa{tsʰi˧˥}}}}\kern2pt]} \hypertarget{ts\string_hi\string_M\string_T4}{}
\markboth{\textcolor{darkblue}{\textbf{\ipa{tsʰi˧˥}}} \textsubscript{4}}{}
\textcolor{teal}{\zh{动词}} \hspace{4pt} \zh{声调类:} MH.
\zh{蹲。} \textcolor{Sepia}{\selectlanguage{english}To squat.} \textcolor{PineGreen}{\selectlanguage{french}S'accroupir.}  ¶ \textcolor{darkblue}{\textbf{\ipa{le˧-tsʰi˩\textasciitilde{}tsʰi˩ | tʰi˧-dzi˩}}} \zh{盘腿坐} \textcolor{Sepia}{\selectlanguage{english}to sit cross-legged} \textcolor{PineGreen}{\selectlanguage{french}être accroupi (être assis avec les genoux regroupés sur la poitrine)}  
 ¶ \textcolor{darkblue}{\textbf{\ipa{gɤ˩-tsʰi˧\textasciitilde{}tsʰi˩ tʰi˧-dzi˩}}} \zh{盘腿坐} \textcolor{Sepia}{\selectlanguage{english}as above} \textcolor{PineGreen}{\selectlanguage{french}même sens}  

\lhead{\firstmark}
\rhead{\botmark}

\subsection{\hspace{-0.5cm} {\Large \textcolor{darkblue}{\textbf{\ipa{tsʰi˧˥}}} \textsubscript{5}}\hspace{0.5cm}[\kern2pt{\textcolor{darkblue}{\textbf{\ipa{tsʰi˧˥}}}}\kern2pt]} \hypertarget{ts\string_hi\string_M\string_T5}{}
\markboth{\textcolor{darkblue}{\textbf{\ipa{tsʰi˧˥}}} \textsubscript{5}}{}
\textcolor{teal}{\zh{形容词}} \hspace{4pt} \zh{声调类:} MH.
\textit{\zh{古语}} [\zh{古语}] \zh{病。} \textcolor{Sepia}{\selectlanguage{english}Sick, ill.} \textcolor{PineGreen}{\selectlanguage{french}Malade, souffrant.}  ¶ \textcolor{darkblue}{\textbf{\ipa{mɤ˧-go˩ mɤ˩-tsʰi˩-ɻ̍˩ |}}} \zh{健康:不病、不痛} \textcolor{Sepia}{\selectlanguage{english}to be in good health: not sick, not suffering} \textcolor{PineGreen}{\selectlanguage{french}être bien portant, ne pas être malade}  

\lhead{\firstmark}
\rhead{\botmark}

\subsection{\hspace{-0.5cm} {\Large \textcolor{darkblue}{\textbf{\ipa{tsʰo˥}}}}\hspace{0.5cm}[\kern2pt{\textcolor{darkblue}{\textbf{\ipa{tsʰo˥}}}}\kern2pt]} \hypertarget{ts\string_ho\string_T1}{}
\markboth{\textcolor{darkblue}{\textbf{\ipa{tsʰo˥}}}}{}
\textcolor{teal}{\zh{形容词}} \hspace{4pt} \zh{声调类:} H.
\zh{齐全。} \textcolor{Sepia}{\selectlanguage{english}Complete, all in readiness.} \textcolor{PineGreen}{\selectlanguage{french}Complet, au grand complet.}  ¶ \textcolor{darkblue}{\textbf{\ipa{ə˧tso˧-mɤ˧-ɲi˩, | tʰi˧-tsʰo˥-ze˩!}}} \zh{什么都准备得很齐全!} \textcolor{Sepia}{\selectlanguage{english}All is in readiness! Everything is now ready! (Context: preparation for a feast, a meal...)} \textcolor{PineGreen}{\selectlanguage{french}Tout y est! Tout est prêt! (Au sujet de préparatifs pour une fête, un repas...)}  
 ¶ \textcolor{darkblue}{\textbf{\ipa{mɤ˧-tsʰo˧-sɯ˥! | wɤ˩˥ | ɲi˧-bæ˧ hwæ˧-zo˧-ho˩!}}} \zh{还不算齐全! / 还没有装饰齐全!(情景:朋友们表扬新装修的丽江房子,主人谦虚回答:‘还不算齐全!’)} \textcolor{Sepia}{\selectlanguage{english}[Decoration] is not complete yet! [I] still need to purchase a few items! (Context: visitors admire a newly furbished apartment in town; the landlord answers their compliments by saying 'The work is not finished yet!')} \textcolor{PineGreen}{\selectlanguage{french}On n'y est pas encore tout à fait / ce n'est pas encore tout à fait prêt! Il reste deux ou trois trucs à acheter! (Contexte: on achève la décoration d'un appartement à la ville; aux compliments des visiteurs, l'heureux propriétaire répond: 'Ce n'est pas encore terminé!')}  
 ¶ \textcolor{darkblue}{\textbf{\ipa{tʰi˧-tsʰo˥-kʰɯ˩}}} \zh{\mytextsc{dur} \string_ \mytextsc{caus:完成、弄齐全}} \textcolor{Sepia}{\selectlanguage{english}\mytextsc{dur} \string_ \mytextsc{caus}: to complete, to bring to complete readiness} \textcolor{PineGreen}{\selectlanguage{french}\mytextsc{dur} \string_ \mytextsc{caus}: porter à son point d'achèvement, porter au grand complet}  

\lhead{\firstmark}
\rhead{\botmark}

\subsection{\hspace{-0.5cm} {\Large \textcolor{darkblue}{\textbf{\ipa{tsʰo˧\textsubscript{b}}}}}\hspace{0.5cm}[\kern2pt{\textcolor{darkblue}{\textbf{\ipa{tsʰo˩˥}}}}\kern2pt]} \hypertarget{ts\string_ho\string_Mb1}{}
\markboth{\textcolor{darkblue}{\textbf{\ipa{tsʰo˧\textsubscript{b}}}}}{}
\textcolor{teal}{\zh{动词}} \hspace{4pt} \zh{声调类:} M\textsubscript{b}.
\zh{跳。} \textcolor{Sepia}{\selectlanguage{english}To jump.} \textcolor{PineGreen}{\selectlanguage{french}Sauter.}  ¶ \textcolor{darkblue}{\textbf{\ipa{bæ˧ tsʰo˧}}} \zh{跳绳} \textcolor{Sepia}{\selectlanguage{english}to skip rope} \textcolor{PineGreen}{\selectlanguage{french}sauter à la corde}  
 ¶ \textcolor{darkblue}{\textbf{\ipa{tsʰo˧\textasciitilde{}tsʰo˧}}} \zh{\mytextsc{重叠}} \textcolor{Sepia}{\selectlanguage{english}\mytextsc{red}} \textcolor{PineGreen}{\selectlanguage{french}forme rédupliquée: trépigner, sautiller ici et là}  

\lhead{\firstmark}
\rhead{\botmark}

\subsection{\hspace{-0.5cm} {\Large \textcolor{darkblue}{\textbf{\ipa{tsʰo˧ɖɯ˩}}}}\hspace{0.5cm}[\kern2pt{\textcolor{darkblue}{\textbf{\ipa{tsʰo˧ɖɯ˩}}}}\kern2pt]} \hypertarget{ts\string_ho\string_Md`M\string_B1}{}
\markboth{\textcolor{darkblue}{\textbf{\ipa{tsʰo˧ɖɯ˩}}}}{}
\textcolor{teal}{\zh{名词}} \hspace{4pt} \zh{声调类:} L\#.
\zh{集体舞。} \textcolor{Sepia}{\selectlanguage{english}Group dance.} \textcolor{PineGreen}{\selectlanguage{french}Danse en groupe: parfois dix personnes, parfois jusqu'à cent (un village entier).}  ¶ \textcolor{darkblue}{\textbf{\ipa{tsʰo˧ɖɯ˩ tsʰo˩}}} \zh{跳一个集体舞} \textcolor{Sepia}{\selectlanguage{english}to perform a group dance} \textcolor{PineGreen}{\selectlanguage{french}faire une grande danse collective}  

\lhead{\firstmark}
\rhead{\botmark}

\subsection{\hspace{-0.5cm} {\Large \textcolor{darkblue}{\textbf{\ipa{tsʰo˧ɖwæ\#˥}}}}\hspace{0.5cm}[\kern2pt{\textcolor{darkblue}{\textbf{\ipa{tsʰo˧ɖwæ˧}}}}\kern2pt]} \hypertarget{ts\string_ho\string_Md`w\{\#\string_T1}{}
\markboth{\textcolor{darkblue}{\textbf{\ipa{tsʰo˧ɖwæ\#˥}}}}{}
\textcolor{teal}{\zh{名词}} \hspace{4pt} \zh{声调类:} \#H.
\zh{石头台阶。} \textcolor{Sepia}{\selectlanguage{english}Stone step.} \textcolor{PineGreen}{\selectlanguage{french}Marche en pierre.}  \zh{量词}: \textcolor{darkblue}{\textbf{\ipa{ɖwæ˥}}} 
\lhead{\firstmark}
\rhead{\botmark}

\subsection{\hspace{-0.5cm} {\Large \textcolor{darkblue}{\textbf{\ipa{tsʰo˧ko˧}}}}\hspace{0.5cm}[\kern2pt{\textcolor{darkblue}{\textbf{\ipa{tsʰo˧ko˧}}}}\kern2pt]} \hypertarget{ts\string_ho\string_Mko\string_M1}{}
\markboth{\textcolor{darkblue}{\textbf{\ipa{tsʰo˧ko˧}}}}{}
\textcolor{teal}{\zh{名词}} \hspace{4pt} \zh{声调类:} M.
\zh{小豆蔻。} \textcolor{Sepia}{\selectlanguage{english}Cardamom, \textit{Elletaria cardamomum}.} \textcolor{PineGreen}{\selectlanguage{french}Cardamome, \textit{Elletaria cardamomum}.} \zh{当地汉语方言:}\zh{草果。} \zh{【借词】}\zh{草果}
 \zh{量词}: \textcolor{darkblue}{\textbf{\ipa{ɭɯ˧}}} 
\lhead{\firstmark}
\rhead{\botmark}

\subsection{\hspace{-0.5cm} {\Large \textcolor{darkblue}{\textbf{\ipa{tsʰo˧pæ\#˥}}}}\hspace{0.5cm}[\kern2pt{\textcolor{darkblue}{\textbf{\ipa{tsʰo˧pæ˧}}}}\kern2pt]} \hypertarget{ts\string_ho\string_Mp\{\#\string_T1}{}
\markboth{\textcolor{darkblue}{\textbf{\ipa{tsʰo˧pæ\#˥}}}}{}
\textcolor{teal}{\zh{名词}} \hspace{4pt} \zh{声调类:} \#H.
\zh{马帮头领。} \textcolor{Sepia}{\selectlanguage{english}Caravan chief.} \textcolor{PineGreen}{\selectlanguage{french}Chef de caravane.}  \zh{【借词】}\zh{藏语} tshong.pa “merchant"

\lhead{\firstmark}
\rhead{\botmark}

\subsection{\hspace{-0.5cm} {\Large \textcolor{darkblue}{\textbf{\ipa{tsʰo˧pjɤ˧}}}}\hspace{0.5cm}[\kern2pt{\textcolor{darkblue}{\textbf{\ipa{tsʰo˧pjɤ˧}}}}\kern2pt]} \hypertarget{ts\string_ho\string_Mpj7\string_M1}{}
\markboth{\textcolor{darkblue}{\textbf{\ipa{tsʰo˧pjɤ˧}}}}{}
\textcolor{teal}{\zh{名词}} \hspace{4pt} \zh{声调类:} M.
\zh{肥皂。} \textcolor{Sepia}{\selectlanguage{english}Soap. Presumably borrowed from a language of Burma: cp. Nung /tshɑ³¹ pi⁵⁵ iɔ⁵⁵/ [Dai 1992], Luxi Achang and Lianghe Achang /tshɑu⁵⁵ pjɑu⁵⁵/ [Dai 1985], Longchuan Achang /tshau³¹ piau³¹/ [Dai 1992]. Culturally, it is not unlikely that soap was first introduced through contact/commerce with ethnic groups of Burma.} \textcolor{PineGreen}{\selectlanguage{french}Savon. Sans doute mot emprunté à une langue de birmanie: cp. nung: tshɑ³¹ pi⁵⁵ iɔ⁵⁵ [Dai 1992]; achang de Luxi et Lianghe: tshɑu⁵⁵ pjɑu⁵⁵ [Dai 1985]; achang de Longchuan: tshau³¹ piau³¹ [Dai 1992]. Culturellement, il est plausible que le savon ait été introduit par le contact/commerce avec des groupes ethniques de Birmanie.}  \zh{量词}: \textcolor{darkblue}{\textbf{\ipa{ɭɯ˧}}} 
\lhead{\firstmark}
\rhead{\botmark}

\subsection{\hspace{-0.5cm} {\Large \textcolor{darkblue}{\textbf{\ipa{tsʰo˧qʰwɤ˧mi\#˥}}}}\hspace{0.5cm}[\kern2pt{\textcolor{darkblue}{\textbf{\ipa{tsʰo˧qʰwɤ˧mi˧}}}}\kern2pt]} \hypertarget{ts\string_ho\string_Mq\string_hw7\string_Mmi\#\string_T1}{}
\markboth{\textcolor{darkblue}{\textbf{\ipa{tsʰo˧qʰwɤ˧mi\#˥}}}}{}
\textcolor{teal}{\zh{名词}} \hspace{4pt} \zh{声调类:} \#H.
\zh{鬼。} \textcolor{Sepia}{\selectlanguage{english}Demon, ghost.} \textcolor{PineGreen}{\selectlanguage{french}Démon, fantôme.}  \zh{量词}: \textcolor{darkblue}{\textbf{\ipa{v̩˧}}} \zh{~【参考】~} \hyperlink{}{\textcolor{darkblue}{\textbf{\ipa{tsʰo˧qʰwɤ˧zo\#˥}}}} 
\lhead{\firstmark}
\rhead{\botmark}

\subsection{\hspace{-0.5cm} {\Large \textcolor{darkblue}{\textbf{\ipa{tsʰo˧qʰwɤ˧mi˧-bæ˥bæ˩}}}}\hspace{0.5cm}[\kern2pt{\textcolor{darkblue}{\textbf{\ipa{tsʰo˧qʰwɤ˧mi˧bæ˥bæ˩}}}}\kern2pt]} \hypertarget{ts\string_ho\string_Mq\string_hw7\string_Mmi\string_M-b\{\string_Tb\{\string_B1}{}
\markboth{\textcolor{darkblue}{\textbf{\ipa{tsʰo˧qʰwɤ˧mi˧-bæ˥bæ˩}}}}{}
\textcolor{teal}{\zh{名词}} \hspace{4pt} \zh{声调类:} \#H-.
\zh{翠雀花。} \textcolor{Sepia}{\selectlanguage{english}A blue flower, \textit{Delphinium grandiflorum}.} \textcolor{PineGreen}{\selectlanguage{french}Une fleur bleue, \textit{Delphinium grandiflorum}.} \zh{~【参考】~} \hyperlink{}{\textcolor{darkblue}{\textbf{\ipa{tsʰo˧qʰwɤ˧mi\#˥}}}} 
\lhead{\firstmark}
\rhead{\botmark}

\subsection{\hspace{-0.5cm} {\Large \textcolor{darkblue}{\textbf{\ipa{tsʰo˧qʰwɤ˧zo\#˥}}}}\hspace{0.5cm}[\kern2pt{\textcolor{darkblue}{\textbf{\ipa{tsʰo˧qʰwɤ˧zo˧}}}}\kern2pt]} \hypertarget{ts\string_ho\string_Mq\string_hw7\string_Mzo\#\string_T1}{}
\markboth{\textcolor{darkblue}{\textbf{\ipa{tsʰo˧qʰwɤ˧zo\#˥}}}}{}
\textcolor{teal}{\zh{名词}} \hspace{4pt} \zh{声调类:} \#H.
\zh{鬼。} \textcolor{Sepia}{\selectlanguage{english}Demon, ghost (this word is less common than that with a feminine suffix).} \textcolor{PineGreen}{\selectlanguage{french}Démon, fantôme (forme moins courante que celle comportant un suffixe féminin).} \zh{~【参考】~} \hyperlink{}{\textcolor{darkblue}{\textbf{\ipa{tsʰo˧qʰwɤ˧mi\#˥}}}} 
\lhead{\firstmark}
\rhead{\botmark}

\subsection{\hspace{-0.5cm} {\Large \textcolor{darkblue}{\textbf{\ipa{tsʰo˧qʰwɤ˩}}}}\hspace{0.5cm}[\kern2pt{\textcolor{darkblue}{\textbf{\ipa{tsʰo˧qʰwɤ˩}}}}\kern2pt]} \hypertarget{ts\string_ho\string_Mq\string_hw7\string_B1}{}
\markboth{\textcolor{darkblue}{\textbf{\ipa{tsʰo˧qʰwɤ˩}}}}{}
\textcolor{teal}{\zh{助词}} \hspace{4pt} \zh{声调类:} L\#.
\zh{今晚。} \textcolor{Sepia}{\selectlanguage{english}Tonight.} \textcolor{PineGreen}{\selectlanguage{french}Ce soir.}  ¶ \textcolor{darkblue}{\textbf{\ipa{tsʰo˧qʰwɤ˩ | mv̩˩kʰv̩˧˥}}} \zh{同上:今晚} \textcolor{Sepia}{\selectlanguage{english}same meaning: tonight} \textcolor{PineGreen}{\selectlanguage{french}même sens: ce soir}  

\lhead{\firstmark}
\rhead{\botmark}

\subsection{\hspace{-0.5cm} {\Large \textcolor{darkblue}{\textbf{\ipa{tsʰo˧ʁo\#˥}}}}\hspace{0.5cm}[\kern2pt{\textcolor{darkblue}{\textbf{\ipa{tsʰo˧ʁo˧}}}}\kern2pt]} \hypertarget{ts\string_ho\string_MRo\#\string_T1}{}
\markboth{\textcolor{darkblue}{\textbf{\ipa{tsʰo˧ʁo\#˥}}}}{}
\textcolor{teal}{\zh{名词}} \hspace{4pt} \zh{声调类:} \#H.
\zh{牲畜圈:家门口的那栋楼,下为畜厩,上存饲料或另辟为房间。} \textcolor{Sepia}{\selectlanguage{english}Stable: the building at the entrance of the farm, through which one comes when entering the farm. It is made of wood. On the ground floor, there are stables; hay and straw are stored on the first floor.} \textcolor{PineGreen}{\selectlanguage{french}Étable: bâtiment à l'entrée de la ferme, que l'on traverse en entrant dans la ferme. Construit en bois. Au rez-de-chaussée se trouvent les étables des porcs; à l'étage un grenier à foin.}  \zh{量词}: \textcolor{darkblue}{\textbf{\ipa{ɭɯ˧}}} 
\lhead{\firstmark}
\rhead{\botmark}

\subsection{\hspace{-0.5cm} {\Large \textcolor{darkblue}{\textbf{\ipa{tsʰo˧tsɯ˧}}}}\hspace{0.5cm}[\kern2pt{\textcolor{darkblue}{\textbf{\ipa{tsʰo˧tsɯ˧}}}}\kern2pt]} \hypertarget{ts\string_ho\string_MtsM\string_M1}{}
\markboth{\textcolor{darkblue}{\textbf{\ipa{tsʰo˧tsɯ˧}}}}{}
\textcolor{teal}{\zh{名词}} \hspace{4pt} \zh{声调类:} M.
\zh{葱,韭葱。} \textcolor{Sepia}{\selectlanguage{english}Onion; leek.} \textcolor{PineGreen}{\selectlanguage{french}Poireau, oignon.}  \zh{【借词】} \zh{葱子}
 \zh{量词}: \textcolor{darkblue}{\textbf{\ipa{po˧}}} 
\lhead{\firstmark}
\rhead{\botmark}

\subsection{\hspace{-0.5cm} {\Large \textcolor{darkblue}{\textbf{\ipa{tsʰo˩}}}}\hspace{0.5cm}[\kern2pt{\textcolor{darkblue}{\textbf{\ipa{tsʰo˥}}}}\kern2pt]} \hypertarget{ts\string_ho\string_B1}{}
\markboth{\textcolor{darkblue}{\textbf{\ipa{tsʰo˩}}}}{}
\textcolor{teal}{\zh{名词}} \hspace{4pt} \zh{声调类:} L.
\zh{人类。} \textcolor{Sepia}{\selectlanguage{english}Human beings; the human species.} \textcolor{PineGreen}{\selectlanguage{french}Espèce humaine, êtres humains; terme ancien apparaissant dans certains proverbes.} 
\lhead{\firstmark}
\rhead{\botmark}

\subsection{\hspace{-0.5cm} {\Large \textcolor{darkblue}{\textbf{\ipa{tsʰo˩mo˩}}}}\hspace{0.5cm}[\kern2pt{\textcolor{darkblue}{\textbf{\ipa{tsʰo˩mo˩˥}}}}\kern2pt]} \hypertarget{ts\string_ho\string_Bmo\string_B1}{}
\markboth{\textcolor{darkblue}{\textbf{\ipa{tsʰo˩mo˩}}}}{}
\textcolor{teal}{\zh{名词}} \hspace{4pt} \zh{声调类:} L.
\zh{老头。} \textcolor{Sepia}{\selectlanguage{english}Old man.} \textcolor{PineGreen}{\selectlanguage{french}Vieil homme, vieillard.} 
\lhead{\firstmark}
\rhead{\botmark}

\subsection{\hspace{-0.5cm} {\Large \textcolor{darkblue}{\textbf{\ipa{tsʰo˩tsɯ˧}}}}\hspace{0.5cm}[\kern2pt{\textcolor{darkblue}{\textbf{\ipa{tsʰo˩tsɯ˥}}}}\kern2pt]} \hypertarget{ts\string_ho\string_BtsM\string_M1}{}
\markboth{\textcolor{darkblue}{\textbf{\ipa{tsʰo˩tsɯ˧}}}}{}
\textcolor{teal}{\zh{名词}} \hspace{4pt} \zh{声调类:} LM.
\zh{锉刀。} \textcolor{Sepia}{\selectlanguage{english}File (tool).} \textcolor{PineGreen}{\selectlanguage{french}Lime.}  \zh{【借词】} \zh{锉子}
 \zh{量词}: \textcolor{darkblue}{\textbf{\ipa{nɑ˧}}} 
\lhead{\firstmark}
\rhead{\botmark}

\subsection{\hspace{-0.5cm} {\Large \textcolor{darkblue}{\textbf{\ipa{tsʰo˧˥}}}}\hspace{0.5cm}[\kern2pt{\textcolor{darkblue}{\textbf{\ipa{tsʰo˧˥}}}}\kern2pt]} \hypertarget{ts\string_ho\string_M\string_T1}{}
\markboth{\textcolor{darkblue}{\textbf{\ipa{tsʰo˧˥}}}}{}
\textcolor{teal}{\zh{名词}} \hspace{4pt} \zh{声调类:} MH.
\zh{重视、关心、恭敬。} \textcolor{Sepia}{\selectlanguage{english}Respect, attention, esteem.} \textcolor{PineGreen}{\selectlanguage{french}Respect, attention, estime.}  ¶ \textcolor{darkblue}{\textbf{\ipa{ʈʂʰɯ˧-ɳɯ˧ | njɤ˧-ki˧ | ɖwæ˧˥ | tsʰo˧ ʝi˥!}}} \zh{他很重视我 / 他对我很尊敬、很关心。} \textcolor{Sepia}{\selectlanguage{english}He/she treats me with great respect/attention.} \textcolor{PineGreen}{\selectlanguage{french}Il/elle me traite avec les plus grands égards / est aux petits soins pour moi!}  

\lhead{\firstmark}
\rhead{\botmark}

\subsection{\hspace{-0.5cm} {\Large \textcolor{darkblue}{\textbf{\ipa{tsʰɯ˧hṽ˥\$}}}}\hspace{0.5cm}[\kern2pt{\textcolor{darkblue}{\textbf{\ipa{tsʰɯ˧hṽ˧}}}}\kern2pt]} \hypertarget{ts\string_hM\string_Mhv\string_~\string_T\$1}{}
\markboth{\textcolor{darkblue}{\textbf{\ipa{tsʰɯ˧hṽ˥\$}}}}{}
\textcolor{teal}{\zh{名词}} \hspace{4pt} \zh{声调类:} H\$.
\zh{羊毛。} \textcolor{Sepia}{\selectlanguage{english}Wool.} \textcolor{PineGreen}{\selectlanguage{french}Laine.}  \zh{量词}: \textcolor{darkblue}{\textbf{\ipa{kʰwɤ˥}}} 
\lhead{\firstmark}
\rhead{\botmark}

\subsection{\hspace{-0.5cm} {\Large \textcolor{darkblue}{\textbf{\ipa{tsʰɯ˧mi˥\$}}}}\hspace{0.5cm}[\kern2pt{\textcolor{darkblue}{\textbf{\ipa{tsʰɯ˧mi˥}}}}\kern2pt]} \hypertarget{ts\string_hM\string_Mmi\string_T\$1}{}
\markboth{\textcolor{darkblue}{\textbf{\ipa{tsʰɯ˧mi˥\$}}}}{}
\textcolor{teal}{\zh{名词}} \hspace{4pt} \zh{声调类:} H\$.
\zh{母山羊。} \textcolor{Sepia}{\selectlanguage{english}Nanny goat.} \textcolor{PineGreen}{\selectlanguage{french}Chèvre.}  ¶ \textcolor{darkblue}{\textbf{\ipa{tsʰɯ˧mi˧-po˧lo˥}}} \zh{母山羊与公山羊} \textcolor{Sepia}{\selectlanguage{english}nanny goat and billy goat} \textcolor{PineGreen}{\selectlanguage{french}chèvre et bouc}  
 \zh{量词}: \textcolor{darkblue}{\textbf{\ipa{pʰo˧˥}}} 
\lhead{\firstmark}
\rhead{\botmark}

\subsection{\hspace{-0.5cm} {\Large \textcolor{darkblue}{\textbf{\ipa{tsʰɯ˧mi˧-to˧qɑ˥\$}}}}\hspace{0.5cm}[\kern2pt{\textcolor{darkblue}{\textbf{\ipa{xxxx non-correspondance entre le nombre de morphèmes et le nombre de tons de morphèmes}}}}\kern2pt]} \hypertarget{ts\string_hM\string_Mmi\string_M-to\string_MqA\string_T\$1}{}
\markboth{\textcolor{darkblue}{\textbf{\ipa{tsʰɯ˧mi˧-to˧qɑ˥\$}}}}{}
\textcolor{teal}{\zh{名词}} \hspace{4pt} \zh{声调类:} H\$.
\zh{公山羊(包括公山羊羔)(可以来指所有的山羊,包括母的和公的)。} \textcolor{Sepia}{\selectlanguage{english}Male goat; also used to refer to a young male goat, or even to goats in general, male and female.} \textcolor{PineGreen}{\selectlanguage{french}Bouc; s'emploie aussi pour un chevreau (cabri), ou même pour toute l'espèce, y compris les chèvres.}  \zh{量词}: \textcolor{darkblue}{\textbf{\ipa{pʰo˧˥}}} 
\lhead{\firstmark}
\rhead{\botmark}

\subsection{\hspace{-0.5cm} {\Large \textcolor{darkblue}{\textbf{\ipa{tsʰɯ˧pʰv̩\#˥}}}}\hspace{0.5cm}[\kern2pt{\textcolor{darkblue}{\textbf{\ipa{tsʰɯ˩pʰv̩˩˥}}}}\kern2pt]} \hypertarget{ts\string_hM\string_Mp\string_hv\string_=\#\string_T1}{}
\markboth{\textcolor{darkblue}{\textbf{\ipa{tsʰɯ˧pʰv̩\#˥}}}}{}
\textcolor{teal}{\zh{名词}} \hspace{4pt} \zh{声调类:} \#H.
\zh{公山羊。} \textcolor{Sepia}{\selectlanguage{english}He-goat.} \textcolor{PineGreen}{\selectlanguage{french}Bouc (terme élicité; plus courant: \textcolor{darkblue}{\textbf{\ipa{/po˧lo˧/}}}).}  \zh{量词}: \textcolor{darkblue}{\textbf{\ipa{pʰo˧˥}}} 
\lhead{\firstmark}
\rhead{\botmark}

\subsection{\hspace{-0.5cm} {\Large \textcolor{darkblue}{\textbf{\ipa{tsʰɯ˧ɻ̍\#˥}}}}\hspace{0.5cm}[\kern2pt{\textcolor{darkblue}{\textbf{\ipa{tsʰɯ˧ɻ̍˧}}}}\kern2pt]} \hypertarget{ts\string_hM\string_Mr£`̍\#\string_T1}{}
\markboth{\textcolor{darkblue}{\textbf{\ipa{tsʰɯ˧ɻ̍\#˥}}}}{}
\textcolor{teal}{\zh{名词}} \hspace{4pt} \zh{声调类:} \#H.
\zh{男女通用名。} \textcolor{Sepia}{\selectlanguage{english}A unixex given name: a given name used for both men and women.} \textcolor{PineGreen}{\selectlanguage{french}Prénom unisexe: prénom utilisé pour les deux sexes.} 
\lhead{\firstmark}
\rhead{\botmark}

\subsection{\hspace{-0.5cm} {\Large \textcolor{darkblue}{\textbf{\ipa{tsʰɯ˧ʂwæ˥}}}}\hspace{0.5cm}[\kern2pt{\textcolor{darkblue}{\textbf{\ipa{tsʰɯ˧ʂwæ˥}}}}\kern2pt]} \hypertarget{ts\string_hM\string_Ms`w\{\string_T1}{}
\markboth{\textcolor{darkblue}{\textbf{\ipa{tsʰɯ˧ʂwæ˥}}}}{}
\textcolor{teal}{\zh{名词}} \hspace{4pt} \zh{声调类:} H\#.
\zh{阉山羊。} \textcolor{Sepia}{\selectlanguage{english}Wether (castrated goat, neutered goat).} \textcolor{PineGreen}{\selectlanguage{french}Bouc castré.}  \zh{量词}: \textcolor{darkblue}{\textbf{\ipa{pʰo˧˥}}} 
\lhead{\firstmark}
\rhead{\botmark}

\subsection{\hspace{-0.5cm} {\Large \textcolor{darkblue}{\textbf{\ipa{tsʰɯ˧-to˧qɑ˥}}}}\hspace{0.5cm}[\kern2pt{\textcolor{darkblue}{\textbf{\ipa{xxxx non-correspondance entre le nombre de morphèmes et le nombre de tons de morphèmes}}}}\kern2pt]} \hypertarget{ts\string_hM\string_M-to\string_MqA\string_T1}{}
\markboth{\textcolor{darkblue}{\textbf{\ipa{tsʰɯ˧-to˧qɑ˥}}}}{}
\textcolor{teal}{\zh{名词}} \hspace{4pt} \zh{声调类:} H\#.
\zh{羔羊、羔子。} \textcolor{Sepia}{\selectlanguage{english}Kid (child of the goat).} \textcolor{PineGreen}{\selectlanguage{french}Chevreau, cabri.}  \zh{量词}: \textcolor{darkblue}{\textbf{\ipa{pʰo˧˥}}} \zh{~【参考】~} \hyperlink{}{\textcolor{darkblue}{\textbf{\ipa{tsʰɯ˧zo˥\$}}}} 
\lhead{\firstmark}
\rhead{\botmark}

\subsection{\hspace{-0.5cm} {\Large \textcolor{darkblue}{\textbf{\ipa{tsʰɯ˧zo\#˥}}}}\hspace{0.5cm}[\kern2pt{\textcolor{darkblue}{\textbf{\ipa{tsʰɯ˧zo˧}}}}\kern2pt]} \hypertarget{ts\string_hM\string_Mzo\#\string_T1}{}
\markboth{\textcolor{darkblue}{\textbf{\ipa{tsʰɯ˧zo\#˥}}}}{}
\textcolor{teal}{\zh{名词}} \hspace{4pt} \zh{声调类:} \#H.
\zh{母山羊羔。} \textcolor{Sepia}{\selectlanguage{english}Young nanny goat.} \textcolor{PineGreen}{\selectlanguage{french}Chevrette.}  \zh{量词}: \textcolor{darkblue}{\textbf{\ipa{ɭɯ˧}}} 
\lhead{\firstmark}
\rhead{\botmark}

\subsection{\hspace{-0.5cm} {\Large \textcolor{darkblue}{\textbf{\ipa{tsʰɯ˧zo˥\$}}}}\hspace{0.5cm}[\kern2pt{\textcolor{darkblue}{\textbf{\ipa{tsʰɯ˧zo˥}}}}\kern2pt]} \hypertarget{ts\string_hM\string_Mzo\string_T\$1}{}
\markboth{\textcolor{darkblue}{\textbf{\ipa{tsʰɯ˧zo˥\$}}}}{}
\textcolor{teal}{\zh{名词}} \hspace{4pt} \zh{声调类:} H\$.
\zh{山羊羔。} \textcolor{Sepia}{\selectlanguage{english}Kid (child of the goat).} \textcolor{PineGreen}{\selectlanguage{french}Chevreau, cabri.}  ¶ \textcolor{darkblue}{\textbf{\ipa{tsʰɯ˧zo˧-to˧qɑ˥}}} \zh{母山羊羔与公山羊羔} \textcolor{Sepia}{\selectlanguage{english}young nanny goat(s) and young kid(s)} \textcolor{PineGreen}{\selectlanguage{french}chevrettes et chevreaux}  
 \zh{量词}: \textcolor{darkblue}{\textbf{\ipa{ɭɯ˧}}} \zh{~【参考】~} \hyperlink{}{\textcolor{darkblue}{\textbf{\ipa{tsʰɯ˧-to˧qɑ˥}}}} 
\lhead{\firstmark}
\rhead{\botmark}

\subsection{\hspace{-0.5cm} {\Large \textcolor{darkblue}{\textbf{\ipa{tsʰɯ˩\textsubscript{a}}}}}\hspace{0.5cm}[\kern2pt{\textcolor{darkblue}{\textbf{\ipa{tsʰɯ˩˥}}}}\kern2pt]} \hypertarget{ts\string_hM\string_Ba1}{}
\markboth{\textcolor{darkblue}{\textbf{\ipa{tsʰɯ˩\textsubscript{a}}}}}{}
\textcolor{teal}{\zh{动词}} \hspace{4pt} \zh{声调类:} L\textsubscript{a}.
\zh{来(过去式)。} \textcolor{Sepia}{\selectlanguage{english}To come (\mytextsc{pst}).} \textcolor{PineGreen}{\selectlanguage{french}Venir (\mytextsc{pst}).}  ¶ \textcolor{darkblue}{\textbf{\ipa{le˧-gwɤ˩\textasciitilde{}gwɤ˩ | le˧-tsʰɯ˩-ze˩}}} \zh{散步回来} \textcolor{Sepia}{\selectlanguage{english}to be back from a stroll} \textcolor{PineGreen}{\selectlanguage{french}revenir de promenade}  
 ¶ \textcolor{darkblue}{\textbf{\ipa{le˧-tsʰɯ˩-ze˩}}} \zh{回来了} \textcolor{Sepia}{\selectlanguage{english}to be back} \textcolor{PineGreen}{\selectlanguage{french}être de retour}  
 ¶ \textcolor{darkblue}{\textbf{\ipa{ɖɯ˧-ʝi˧-ɳɯ˧ tsʰɯ˧˥, | ɖɯ˧-ki˧ tʰv̩˧!}}} \zh{“我们都来自不同的地方,但现在在一起了!”} \textcolor{Sepia}{\selectlanguage{english}“We have come from different places, and now we arrive in the same place / we come together!” This turn of phrase is not intelligible without prior learning, as it literally means “Coming from one place; arriving in one place”.} \textcolor{PineGreen}{\selectlanguage{french}“Venus de différents endroits, nous voici réunis en ce lieu!” L'expression est obscure pour qui ne l'a pas apprise (par exemple pour des locuteurs du bord du Lac): son sens littéral n'est pas particulièrement parlant: “On vient d'un endroit; on arrive à un endroit!”}  

\lhead{\firstmark}
\rhead{\botmark}

\subsection{\hspace{-0.5cm} {\Large \textcolor{darkblue}{\textbf{\ipa{tsʰɯ˩tsʰɯ˩ɻ̃˩}}}}\hspace{0.5cm}[\kern2pt{\textcolor{darkblue}{\textbf{\ipa{tsʰɯ˩tsʰɯ˩ɻ̃˩˥}}}}\kern2pt]} \hypertarget{ts\string_hM\string_Bts\string_hM\string_Br£`\string_~\string_B1}{}
\markboth{\textcolor{darkblue}{\textbf{\ipa{tsʰɯ˩tsʰɯ˩ɻ̃˩}}}}{}
\textcolor{teal}{\zh{名词}} \hspace{4pt} \zh{声调类:} L.
\zh{豌豆干草。} \textcolor{Sepia}{\selectlanguage{english}Dry plant of garden peas, garden peas hay.} \textcolor{PineGreen}{\selectlanguage{french}Paille de petits pois.}  ¶ \textcolor{darkblue}{\textbf{\ipa{ʈʂʰɯ˧ | tsʰɯ˩tsʰɯ˩ɻ̃˩ ɲi˥.}}} \zh{这是豌豆干草。} \textcolor{Sepia}{\selectlanguage{english}This is garden pea hay.} \textcolor{PineGreen}{\selectlanguage{french}C'est de la paille de haricots.}  
 \zh{量词}: \textcolor{darkblue}{\textbf{\ipa{kɤ˧˥}}} 
\lhead{\firstmark}
\rhead{\botmark}

\subsection{\hspace{-0.5cm} {\Large \textcolor{darkblue}{\textbf{\ipa{tsʰɯ˧˥}}} \textsubscript{1}}\hspace{0.5cm}[\kern2pt{\textcolor{darkblue}{\textbf{\ipa{tsʰɯ˥}}}}\kern2pt]} \hypertarget{ts\string_hM\string_M\string_T1}{}
\markboth{\textcolor{darkblue}{\textbf{\ipa{tsʰɯ˧˥}}} \textsubscript{1}}{}
\textcolor{teal}{\zh{动词}} \hspace{4pt} \zh{声调类:} MH.
\zh{剪成片。} \textcolor{Sepia}{\selectlanguage{english}To cut to pieces (e.g. to cut away at a piece of clothing with scissors).} \textcolor{PineGreen}{\selectlanguage{french}Taillader (ex.: taillader un vêtement, le découper avec des ciseaux; n'est pas: tailler du tissu pour faire des vêtements).}  ¶ \textcolor{darkblue}{\textbf{\ipa{tʰɑ˧-tsʰɯ˧˥!}}} \zh{\mytextsc{prohib}} \textcolor{Sepia}{\selectlanguage{english}\mytextsc{prohib}} \textcolor{PineGreen}{\selectlanguage{french}\mytextsc{prohib}}  
 ¶ \textcolor{darkblue}{\textbf{\ipa{dʑi˧hṽ˧ tsʰɯ˩}}} \zh{把衣服剪成片} \textcolor{Sepia}{\selectlanguage{english}to cut clothes to pieces} \textcolor{PineGreen}{\selectlanguage{french}couper des vêtements en morceaux}  

\lhead{\firstmark}
\rhead{\botmark}

\subsection{\hspace{-0.5cm} {\Large \textcolor{darkblue}{\textbf{\ipa{tsʰɯ˧˥}}} \textsubscript{2}}\hspace{0.5cm}[\kern2pt{\textcolor{darkblue}{\textbf{\ipa{tsʰɯ˧˥}}}}\kern2pt]} \hypertarget{ts\string_hM\string_M\string_T2}{}
\markboth{\textcolor{darkblue}{\textbf{\ipa{tsʰɯ˧˥}}} \textsubscript{2}}{}
\textcolor{teal}{\zh{名词}} \hspace{4pt} \zh{声调类:} MH.
\zh{山羊。} \textcolor{Sepia}{\selectlanguage{english}Goat (male or female).} \textcolor{PineGreen}{\selectlanguage{french}Chèvre/bouc.}  \zh{量词}: \textcolor{darkblue}{\textbf{\ipa{pʰo˧˥}}} 
\lhead{\firstmark}
\rhead{\botmark}

\subsection{\hspace{-0.5cm} {\Large \textcolor{darkblue}{\textbf{\ipa{tsʰv̩˩˥}}}}\hspace{0.5cm}[\kern2pt{\textcolor{darkblue}{\textbf{\ipa{tsʰv̩˩˥}}}}\kern2pt]} \hypertarget{ts\string_hv\string_=\string_B\string_T1}{}
\markboth{\textcolor{darkblue}{\textbf{\ipa{tsʰv̩˩˥}}}}{}
\textcolor{teal}{\zh{名词}} \hspace{4pt} \zh{声调类:} LH.
\zh{醋(汉语借词)。} \textcolor{Sepia}{\selectlanguage{english}Vinegar.} \textcolor{PineGreen}{\selectlanguage{french}Vinaigre.}  \zh{【借词】} \zh{醋}
\zh{~【参考】~} \textcolor{darkblue}{\textbf{\ipa{sɑ˧tsʰv̩˩, tɕi˧-dʑɯ˩}}} 
\lhead{\firstmark}
\rhead{\botmark}

\newpage
\section*{\centering- \textcolor{darkblue}{\textbf{\ipa{ʈ}}} -}
\subsection{\hspace{-0.5cm} {\Large \textcolor{darkblue}{\textbf{\ipa{ʈæ˧bɤ˧}}}}\hspace{0.5cm}[\kern2pt{\textcolor{darkblue}{\textbf{\ipa{ʈæ˩bɤ˩˥}}}}\kern2pt]} \hypertarget{t`\{\string_Mb7\string_M1}{}
\markboth{\textcolor{darkblue}{\textbf{\ipa{ʈæ˧bɤ˧}}}}{}
\textcolor{teal}{\zh{名词}} \hspace{4pt} \zh{声调类:} M.
\zh{和尚,尼姑。} \textcolor{Sepia}{\selectlanguage{english}Monk, nun.} \textcolor{PineGreen}{\selectlanguage{french}Moine, nonne.}  ¶ \textcolor{darkblue}{\textbf{\ipa{ʈæ˧bɤ˧ʈʂʰo˧}}} \zh{同上} \textcolor{Sepia}{\selectlanguage{english}same meaning} \textcolor{PineGreen}{\selectlanguage{french}même sens}  
 ¶ \textcolor{darkblue}{\textbf{\ipa{ʈæ˧bɤ˧ ʝi˧-hĩ˧-hĩ˧}}} \zh{当和尚的人} \textcolor{Sepia}{\selectlanguage{english}person who is a monk} \textcolor{PineGreen}{\selectlanguage{french}même sens}  
 ¶ \textcolor{darkblue}{\textbf{\ipa{hæ˧ʈæ˩bɤ˩}}} \zh{汉人和尚} \textcolor{Sepia}{\selectlanguage{english}Chinese monk} \textcolor{PineGreen}{\selectlanguage{french}moine chinois}  
 \zh{量词}: \textcolor{darkblue}{\textbf{\ipa{v̩˧}}} 
\lhead{\firstmark}
\rhead{\botmark}

\subsection{\hspace{-0.5cm} {\Large \textcolor{darkblue}{\textbf{\ipa{ʈæ˧kwæ˧˥}}}}\hspace{0.5cm}[\kern2pt{\textcolor{darkblue}{\textbf{\ipa{ʈæ˧kwæ˧˥}}}}\kern2pt]} \hypertarget{t`\{\string_Mkw\{\string_M\string_T1}{}
\markboth{\textcolor{darkblue}{\textbf{\ipa{ʈæ˧kwæ˧˥}}}}{}
\textcolor{teal}{\zh{形容词}} \hspace{4pt} \zh{声调类:} MH\#.
\zh{爱浪费。} \textcolor{Sepia}{\selectlanguage{english}Prodigal, wasteful.} \textcolor{PineGreen}{\selectlanguage{french}Prodigue, qui dépense tout.}  ¶ \textcolor{darkblue}{\textbf{\ipa{ʈʂʰɯ˧ | ʈæ˧kwæ˧-hĩ˥ | ɖɯ˧-v̩˧ ɲi˩.}}} \zh{他是爱浪费的人。} \textcolor{Sepia}{\selectlanguage{english}(S)he is a prodigal person.} \textcolor{PineGreen}{\selectlanguage{french}C'est un prodigue/quelqu'un qui dépense tout/qui mène la maison à la ruine.}  

\lhead{\firstmark}
\rhead{\botmark}

\subsection{\hspace{-0.5cm} {\Large \textcolor{darkblue}{\textbf{\ipa{ʈæ˧qo˧}}}}\hspace{0.5cm}[\kern2pt{\textcolor{darkblue}{\textbf{\ipa{ʈæ˧qo˧}}}}\kern2pt]} \hypertarget{t`\{\string_Mqo\string_M1}{}
\markboth{\textcolor{darkblue}{\textbf{\ipa{ʈæ˧qo˧}}}}{}
\textcolor{teal}{\zh{助词}} \hspace{4pt} \zh{声调类:} M.
\zh{底下。} \textcolor{Sepia}{\selectlanguage{english}At bottom, at the bottom of.} \textcolor{PineGreen}{\selectlanguage{french}Au fond de.}  ¶ \textcolor{darkblue}{\textbf{\ipa{hi˩nɑ˧mi˧-ʈæ˧qo˥}}} \zh{在湖底下} \textcolor{Sepia}{\selectlanguage{english}at the bottom of the Lake} \textcolor{PineGreen}{\selectlanguage{french}au fond du Lac}  
 ¶ \textcolor{darkblue}{\textbf{\ipa{ʈæ˧qo˧ tɕɯ˧}}} \zh{放在底下} \textcolor{Sepia}{\selectlanguage{english}to place at the bottom of...} \textcolor{PineGreen}{\selectlanguage{french}mettre au fond de...}  
 ¶ \textcolor{darkblue}{\textbf{\ipa{hi˩nɑ˧mi˧, | ʈæ˧ mɤ˧-do˩; | hĩ˧-nv̩˥mi˩, | ɳv̩˧ mɤ˧-tʰɑ˩.}}} \zh{“人的心,湖底藏:看不清,摸不透!” 直译:“湖,(我们)看不到(它的)底下。人的心,是知道不了的!”(情歌里的一个谚语)} \textcolor{Sepia}{\selectlanguage{english}“One can't see to the bottom of the Lake; one can't know the heart of men.” (Proverb that comes up in courting songs.)} \textcolor{PineGreen}{\selectlanguage{french}“On ne voit pas le fond du lac; on ne connaît pas le cœur des hommes!” (Proverbe qui apparaît dans les chansons que se chantaient les jeunes gens se faisant la cour.)}  

\lhead{\firstmark}
\rhead{\botmark}

\subsection{\hspace{-0.5cm} {\Large \textcolor{darkblue}{\textbf{\ipa{ʈæ˧ʂɯ˧}}}}\hspace{0.5cm}[\kern2pt{\textcolor{darkblue}{\textbf{\ipa{ʈæ˧ʂɯ˧}}}}\kern2pt]} \hypertarget{t`\{\string_Ms`M\string_M1}{}
\markboth{\textcolor{darkblue}{\textbf{\ipa{ʈæ˧ʂɯ˧}}}}{}
\textcolor{teal}{\zh{名词}} \hspace{4pt} \zh{声调类:} M.
\zh{男性名字。} \textcolor{Sepia}{\selectlanguage{english}Masculine given name.} \textcolor{PineGreen}{\selectlanguage{french}Prénom masculin.} 
\lhead{\firstmark}
\rhead{\botmark}

\subsection{\hspace{-0.5cm} {\Large \textcolor{darkblue}{\textbf{\ipa{ʈæ˩\textsubscript{a}}}}}\hspace{0.5cm}[\kern2pt{\textcolor{darkblue}{\textbf{\ipa{ʈæ˧˥}}}}\kern2pt]} \hypertarget{t`\{\string_Ba1}{}
\markboth{\textcolor{darkblue}{\textbf{\ipa{ʈæ˩\textsubscript{a}}}}}{}
\textcolor{teal}{\zh{动词}} \hspace{4pt} \zh{声调类:} L\textsubscript{a}.
\ding{202} \zh{关(门、羊)。} \textcolor{Sepia}{\selectlanguage{english}To lock up (animals), to close (a door).} \textcolor{PineGreen}{\selectlanguage{french}Fermer, refermer; enfermer (ex.: des moutons); aussi: fermer une route.}  ¶ \textcolor{darkblue}{\textbf{\ipa{bv̩˩qo˩ ʈæ˥}}} \zh{关牛圈} \textcolor{Sepia}{\selectlanguage{english}to close the stable} \textcolor{PineGreen}{\selectlanguage{french}fermer l'étable}  
 ¶ \textcolor{darkblue}{\textbf{\ipa{tʰi˧-ʈæ˩}}} \zh{关门} \textcolor{Sepia}{\selectlanguage{english}\mytextsc{dur}: to close} \textcolor{PineGreen}{\selectlanguage{french}\mytextsc{dur}: fermer}  
 ¶ \textcolor{darkblue}{\textbf{\ipa{kʰi˧ ʈæ˥}}} \zh{关门} \textcolor{Sepia}{\selectlanguage{english}to close the door} \textcolor{PineGreen}{\selectlanguage{french}fermer la porte}  
\ding{203} \zh{扣(扣子)、系、结。} \textcolor{Sepia}{\selectlanguage{english}To tie (a knot).} \textcolor{PineGreen}{\selectlanguage{french}Nouer (un noeud).} 
\lhead{\firstmark}
\rhead{\botmark}

\subsection{\hspace{-0.5cm} {\Large \textcolor{darkblue}{\textbf{\ipa{ʈæ˩ɖɯ˧}}}}\hspace{0.5cm}[\kern2pt{\textcolor{darkblue}{\textbf{\ipa{ʈæ˧ɖɯ˧}}}}\kern2pt]} \hypertarget{t`\{\string_Bd`M\string_M1}{}
\markboth{\textcolor{darkblue}{\textbf{\ipa{ʈæ˩ɖɯ˧}}}}{}
\textcolor{teal}{\zh{形容词}} \hspace{4pt} \zh{声调类:} LM.
\zh{安乐。} \textcolor{Sepia}{\selectlanguage{english}Satisfied, quiet.} \textcolor{PineGreen}{\selectlanguage{french}Satisfait, tranquille.}  ¶ \textcolor{darkblue}{\textbf{\ipa{mɤ˧-ʈæ˩ɖɯ˩}}} \zh{不高兴、不安} \textcolor{Sepia}{\selectlanguage{english}dissatisfied, restless} \textcolor{PineGreen}{\selectlanguage{french}mécontent, furieux}  
 ¶ \textcolor{darkblue}{\textbf{\ipa{ə˧mɑ˧ | tsʰi˧-ɲi˧ | ʈæ˩ɖɯ˧ tʰi˧-dzi˩-dʑo˩!}}} \zh{今天,阿妈安乐地坐着。} \textcolor{Sepia}{\selectlanguage{english}Today, Ama is sitting quietly!} \textcolor{PineGreen}{\selectlanguage{french}Aujourd'hui, Ama est assise bien tranquille!}  
 ¶ \textcolor{darkblue}{\textbf{\ipa{ʈʂʰɯ˧-ɳɯ˧ | njɤ˧-ki˧ | mɤ˧-ʈæ˩ɖɯ˩-hĩ˩ ʐwɤ˩!}}} \zh{他跟我说了一些让我不安的(事情)! / 他跟我说的,让我生气!} \textcolor{Sepia}{\selectlanguage{english}He told me unpleasant things! / He told me vexing things! / He told me some things that make me frustrated/dissatisfied!} \textcolor{PineGreen}{\selectlanguage{french}il m'a dit des choses qui fâchent! (=il m'a vexé, il m'a dit des choses désobligeantes)}  

\lhead{\firstmark}
\rhead{\botmark}

\subsection{\hspace{-0.5cm} {\Large \textcolor{darkblue}{\textbf{\ipa{ʈæ˩tsʰo\#˥}}} \textsubscript{1}}\hspace{0.5cm}[\kern2pt{\textcolor{darkblue}{\textbf{\ipa{ʈæ˩tsʰo˥}}}}\kern2pt]} \hypertarget{t`\{\string_Bts\string_ho\#\string_T1}{}
\markboth{\textcolor{darkblue}{\textbf{\ipa{ʈæ˩tsʰo\#˥}}} \textsubscript{1}}{}
\textcolor{teal}{\zh{名词}} \hspace{4pt} \zh{声调类:} LM+\#H.
\zh{班、小组。} \textcolor{Sepia}{\selectlanguage{english}Class, group, set (of monks).} \textcolor{PineGreen}{\selectlanguage{french}Classe, groupe, ensemble (de prêtres).}  ¶ \textcolor{darkblue}{\textbf{\ipa{ʈæ˩tsʰo˧ | ɖɯ˧-ɭɯ˧}}} \zh{一个小组、一帮(和尚)} \textcolor{Sepia}{\selectlanguage{english}a group (of priests)} \textcolor{PineGreen}{\selectlanguage{french}une classe, un groupe (de prêtres)}  
 \zh{量词}: \textcolor{darkblue}{\textbf{\ipa{ɭɯ˧}}} \zh{~【参考】~} \hyperlink{}{\textcolor{darkblue}{\textbf{\ipa{ʈæ˩tsʰo\#˥}}} \textsubscript{2}} 
\lhead{\firstmark}
\rhead{\botmark}

\subsection{\hspace{-0.5cm} {\Large \textcolor{darkblue}{\textbf{\ipa{ʈæ˩tsʰo\#˥}}} \textsubscript{2}}\hspace{0.5cm}[\kern2pt{\textcolor{darkblue}{\textbf{\ipa{ʈæ˩tsʰo˩˥}}}}\kern2pt]} \hypertarget{t`\{\string_Bts\string_ho\#\string_T2}{}
\markboth{\textcolor{darkblue}{\textbf{\ipa{ʈæ˩tsʰo\#˥}}} \textsubscript{2}}{}
\textcolor{teal}{\zh{量词}} \hspace{4pt} \zh{声调类:} L.
\zh{量词:和尚(一帮、一班)。} \textcolor{Sepia}{\selectlanguage{english}Daeco.} \textcolor{PineGreen}{\selectlanguage{french}Auto-classificateur des classes/groupes (de prêtres).}  ¶ \textcolor{darkblue}{\textbf{\ipa{ɖɯ˧-ʈæ˩tsʰo˩}}} \zh{一班(和尚)} \textcolor{Sepia}{\selectlanguage{english}a group (of monks)} \textcolor{PineGreen}{\selectlanguage{french}un groupe (de prêtres)}  
\zh{~【参考】~} \hyperlink{}{\textcolor{darkblue}{\textbf{\ipa{ʈæ˩tsʰo\#˥}}} \textsubscript{1}} 
\lhead{\firstmark}
\rhead{\botmark}

\subsection{\hspace{-0.5cm} {\Large \textcolor{darkblue}{\textbf{\ipa{ʈæ˩ʈv̩\#˥}}}}\hspace{0.5cm}[\kern2pt{\textcolor{darkblue}{\textbf{\ipa{ʈæ˩ʈv̩˥}}}}\kern2pt]} \hypertarget{t`\{\string_Bt`v\string_=\#\string_T1}{}
\markboth{\textcolor{darkblue}{\textbf{\ipa{ʈæ˩ʈv̩\#˥}}}}{}
\textcolor{teal}{\zh{名词}} \hspace{4pt} \zh{声调类:} LM+\#H.
\zh{男性名字。} \textcolor{Sepia}{\selectlanguage{english}Masculine given name.} \textcolor{PineGreen}{\selectlanguage{french}Prénom masculin.} 
\lhead{\firstmark}
\rhead{\botmark}

\subsection{\hspace{-0.5cm} {\Large \textcolor{darkblue}{\textbf{\ipa{ʈɤ˧\textsubscript{a}}}}}\hspace{0.5cm}[\kern2pt{\textcolor{darkblue}{\textbf{\ipa{ʈɤ˥}}}}\kern2pt]} \hypertarget{t`7\string_Ma1}{}
\markboth{\textcolor{darkblue}{\textbf{\ipa{ʈɤ˧\textsubscript{a}}}}}{}
\textcolor{teal}{\zh{动词}} \hspace{4pt} \zh{声调类:} M\textsubscript{a}.
\zh{拉、拽。} \textcolor{Sepia}{\selectlanguage{english}To pull.} \textcolor{PineGreen}{\selectlanguage{french}Tirer.}  ¶ \textcolor{darkblue}{\textbf{\ipa{tso˧\textasciitilde{}tso˧ ʈɤ˩(-ze˩)}}} \zh{拉拽东西} \textcolor{Sepia}{\selectlanguage{english}to pull something} \textcolor{PineGreen}{\selectlanguage{french}tirer quelque chose}  
 ¶ \textcolor{darkblue}{\textbf{\ipa{mv̩˧ʐe˧ qʰæ˩ | le˧-wo˧-ʈɤ˥-di˩}}} \zh{扳机} \textcolor{Sepia}{\selectlanguage{english}periphrase to refer to the trigger of a gun: the part that one pulls to shoot} \textcolor{PineGreen}{\selectlanguage{french}périphrase pour désigner la gâchette d'un pistolet: ce qu'on tire vers soi pour faire feu}  

\lhead{\firstmark}
\rhead{\botmark}

\subsection{\hspace{-0.5cm} {\Large \textcolor{darkblue}{\textbf{\ipa{ʈi˥\textsubscript{a}}}}}\hspace{0.5cm}[\kern2pt{\textcolor{darkblue}{\textbf{\ipa{ʈi˩˥}}}}\kern2pt]} \hypertarget{t`i\string_Ta1}{}
\markboth{\textcolor{darkblue}{\textbf{\ipa{ʈi˥\textsubscript{a}}}}}{}
\textcolor{teal}{\zh{量词}} \hspace{4pt} \zh{声调类:} H\textsubscript{a}.
\zh{量词:拃(大拇指和食指之间的距离。一般不用大拇指和中指之间的距离。)。} \textcolor{Sepia}{\selectlanguage{english}A handspan (between the thumb and index). The distance between the thumb and the middle finger is not commonly used.} \textcolor{PineGreen}{\selectlanguage{french}Empan: distance entre le pouce et l'index écartés. D'ordinaire, on n'emploie pas la distance entre pouce et majeur.} 
\lhead{\firstmark}
\rhead{\botmark}

\subsection{\hspace{-0.5cm} {\Large \textcolor{darkblue}{\textbf{\ipa{ʈi˩\textsubscript{a}}}}}\hspace{0.5cm}[\kern2pt{\textcolor{darkblue}{\textbf{\ipa{ʈi˩˥}}}}\kern2pt]} \hypertarget{t`i\string_Ba1}{}
\markboth{\textcolor{darkblue}{\textbf{\ipa{ʈi˩\textsubscript{a}}}}}{}
\textcolor{teal}{\zh{动词}} \hspace{4pt} \zh{声调类:} L\textsubscript{a}.
\zh{起(如:起来,起床)。} \textcolor{Sepia}{\selectlanguage{english}To get up.} \textcolor{PineGreen}{\selectlanguage{french}Se lever.}  ¶ \textcolor{darkblue}{\textbf{\ipa{gɤ˩-ʈi˧}}} \zh{起来} \textcolor{Sepia}{\selectlanguage{english}to get up} \textcolor{PineGreen}{\selectlanguage{french}se lever}  
 ¶ \textcolor{darkblue}{\textbf{\ipa{ʑi˧ ʈi˥}}} \zh{醒来} \textcolor{Sepia}{\selectlanguage{english}to wake up} \textcolor{PineGreen}{\selectlanguage{french}se réveiller}  
 ¶ \textcolor{darkblue}{\textbf{\ipa{ʑi˧ gɤ˧-ʈi˩}}} \zh{醒来} \textcolor{Sepia}{\selectlanguage{english}to wake up} \textcolor{PineGreen}{\selectlanguage{french}se réveiller}  
 ¶ \textcolor{darkblue}{\textbf{\ipa{gɤ˩ mɤ˥-ʈi˩}}} \zh{不起床} \textcolor{Sepia}{\selectlanguage{english}not to get up} \textcolor{PineGreen}{\selectlanguage{french}ne pas se lever}  
 ¶ \textcolor{darkblue}{\textbf{\ipa{mɤ˧-ʈi˩-sɯ˩!}}} \zh{还没起床!} \textcolor{Sepia}{\selectlanguage{english}(She/he) has not got up yet / is not up yet!} \textcolor{PineGreen}{\selectlanguage{french}(Il/elle) n'est pas encore levé(e)!}  
 ¶ \textcolor{darkblue}{\textbf{\ipa{le˧-ʈi˩-ze˩!}}} \zh{起床了!} \textcolor{Sepia}{\selectlanguage{english}(She/he) has got up!} \textcolor{PineGreen}{\selectlanguage{french}(il/elle) s'est levé(e)!}  
 ¶ \textcolor{darkblue}{\textbf{\ipa{ɖɯ˧-ʈi˧\textasciitilde{}ʈi˥-ɻ̍˩}}} \zh{起来一下} \textcolor{Sepia}{\selectlanguage{english}\mytextsc{delimitative} \mytextsc{red} \mytextsc{inceptive}} \textcolor{PineGreen}{\selectlanguage{french}\mytextsc{délimitatif} \mytextsc{red} \mytextsc{inchoatif}}  

\lhead{\firstmark}
\rhead{\botmark}

\subsection{\hspace{-0.5cm} {\Large \textcolor{darkblue}{\textbf{\ipa{ʈɯ˧\textsubscript{a}}}}}\hspace{0.5cm}[\kern2pt{\textcolor{darkblue}{\textbf{\ipa{ʈɯ˧˥}}}}\kern2pt]} \hypertarget{t`M\string_Ma1}{}
\markboth{\textcolor{darkblue}{\textbf{\ipa{ʈɯ˧\textsubscript{a}}}}}{}
\textcolor{teal}{\zh{动词}} \hspace{4pt} \zh{声调类:} M\textsubscript{a}.
\zh{安装、摆好。} \textcolor{Sepia}{\selectlanguage{english}To set in place.} \textcolor{PineGreen}{\selectlanguage{french}Mettre en place, installer à sa juste place.}  ¶ \textcolor{darkblue}{\textbf{\ipa{ʂe˧kʰɯ˧ tʰi˧-ʈɯ˧˥, | v̩˧ | tʰi˧-ʈɯ˧}}} \zh{(建完新房后)安装三脚架、把锅摆好(在三脚架上)} \textcolor{Sepia}{\selectlanguage{english}to set the tripod in place (in the hearth); to set the large pot in place (as part of the final steps in setting up the new home, after a new house has been built)} \textcolor{PineGreen}{\selectlanguage{french}mettre en place le trépied de fer dans le foyer, mettre en place la grande casserole (sur le trépied) (Contexte: description de la “pendaison de crémaillère”, dans une nouvelle maison)}  
 ¶ \textcolor{darkblue}{\textbf{\ipa{tsʰo˩-ɻ̃˩˥ | dʑɯ˩ mɤ˩-ʈɯ˩˥, | lɑ˧-ʂe˧ | kʰv̩˧ tʰɑ˩-ki˩!}}} \zh{“人骨头,莫碰水!老虎肉,莫给狗!”(这个谚语,来强调摩梭与藏族的一些不同习惯:摩梭禁止让尸体或骨灰沾水。)} \textcolor{Sepia}{\selectlanguage{english}“Human bones must not be put in water; tiger's flesh must not be given to the dog!” (Explanation: corpses were not buried in water, unlike in certain Tibetan customs. Neither the body, nor the ashes of cremation, must be put in water.)} \textcolor{PineGreen}{\selectlanguage{french}“Les ossements humains, on ne les met pas à l'eau! La chair du tigre, on ne la donne pas au chien!” Sens: on n'enterre pas les gens dans l'eau (à la différence de certaines coutumes tibétaines); on prenait soin de n'immerger dans l'eau, ni le corps, ni les cendres après la crémation.}  

\lhead{\firstmark}
\rhead{\botmark}

\subsection{\hspace{-0.5cm} {\Large \textcolor{darkblue}{\textbf{\ipa{ʈɯ˧ʈʰæ\#˥}}}}\hspace{0.5cm}[\kern2pt{\textcolor{darkblue}{\textbf{\ipa{ʈɯ˩ʈʰæ˩˥}}}}\kern2pt]} \hypertarget{t`M\string_Mt`\string_h\{\#\string_T1}{}
\markboth{\textcolor{darkblue}{\textbf{\ipa{ʈɯ˧ʈʰæ\#˥}}}}{}
\textcolor{teal}{\zh{名词}} \hspace{4pt} \zh{声调类:} \#H.
\zh{家底、财产(贵重物品)。} \textcolor{Sepia}{\selectlanguage{english}Patrimony, family wealth, property.} \textcolor{PineGreen}{\selectlanguage{french}Patrimoine.}  ¶ \textcolor{darkblue}{\textbf{\ipa{ɑ˩ʁo˧ ʈɯ˧ʈʰæ˧!}}} \zh{祝你们家发财!} \textcolor{Sepia}{\selectlanguage{english}Prosperity to the family!} \textcolor{PineGreen}{\selectlanguage{french}Prospérité à la famille!}  
 ¶ \textcolor{darkblue}{\textbf{\ipa{ʈʂʰɯ˧ | ʈɯ˧ʈʰæ˧ | ɖwæ˧˥ | dʑo˧-ʝi˧!}}} \zh{他家底很好! / 他家有钱!} \textcolor{Sepia}{\selectlanguage{english}(S)he has a large patrimony / His/her family is rich!} \textcolor{PineGreen}{\selectlanguage{french}Il/elle est riche! / Sa famille est riche!}  
 \zh{量词}: \textcolor{darkblue}{\textbf{\ipa{kʰwɤ˥}}} 
\lhead{\firstmark}
\rhead{\botmark}

\subsection{\hspace{-0.5cm} {\Large \textcolor{darkblue}{\textbf{\ipa{ʈɯ˧˥}}}}\hspace{0.5cm}[\kern2pt{\textcolor{darkblue}{\textbf{\ipa{ʈɯ˥}}}}\kern2pt]} \hypertarget{t`M\string_M\string_T1}{}
\markboth{\textcolor{darkblue}{\textbf{\ipa{ʈɯ˧˥}}}}{}
\textcolor{teal}{\zh{动词}} \hspace{4pt} \zh{声调类:} MH.
\zh{以滚水将蔬菜或亚麻灼过。} \textcolor{Sepia}{\selectlanguage{english}To blanch: to scald with boiling water, as a preliminary stage in cooking (e.g. for dried vegetables) or in making thread (from linen).} \textcolor{PineGreen}{\selectlanguage{french}Blanchir à l'eau bouillante: du lin pour préparer du fil pour le tissage, des légumes séchés avant de les utiliser pour la cuisine...}  ¶ \textcolor{darkblue}{\textbf{\ipa{tʰi˧-ʈɯ˧˥}}} \zh{\mytextsc{dur}} \textcolor{Sepia}{\selectlanguage{english}\mytextsc{dur}} \textcolor{PineGreen}{\selectlanguage{french}\mytextsc{dur}}  
 ¶ \textcolor{darkblue}{\textbf{\ipa{dʑɯ˩tsʰi˩-qo˥ | tʰi˧-ʈɯ˧˥ / dʑɯ˩tsʰi˩-qo˥ | ʈɯ˧˥}}} \zh{以滚水灼过} \textcolor{Sepia}{\selectlanguage{english}to blanch with boiling water} \textcolor{PineGreen}{\selectlanguage{french}blanchir à l'eau bouillante}  
 ¶ \textcolor{darkblue}{\textbf{\ipa{dʑɯ˧-qo˧ | ʈɯ˧˥}}} \zh{以水灼过} \textcolor{Sepia}{\selectlanguage{english}to blanch with water} \textcolor{PineGreen}{\selectlanguage{french}blanchir à l'eau}  
 ¶ \textcolor{darkblue}{\textbf{\ipa{v˩tsʰɤ˧ ʈɯ˥}}} \zh{灼蔬菜} \textcolor{Sepia}{\selectlanguage{english}to blanch vegetables} \textcolor{PineGreen}{\selectlanguage{french}blanchir des légumes}  
 ¶ \textcolor{darkblue}{\textbf{\ipa{sɑ˧, | ʈɯ˧-kv˥!}}} \zh{亚麻,要灼过!} \textcolor{Sepia}{\selectlanguage{english}Linen needs to be blanched!} \textcolor{PineGreen}{\selectlanguage{french}Le chanvre, ça se blanchit! (Au cours de la préparation du chanvre pour en faire du fil, il faut le blanchir.)}  

\lhead{\firstmark}
\rhead{\botmark}

\subsection{\hspace{-0.5cm} {\Large \textcolor{darkblue}{\textbf{\ipa{ʈv̩˩}}}}\hspace{0.5cm}[\kern2pt{\textcolor{darkblue}{\textbf{\ipa{ʈv̩˥}}}}\kern2pt]} \hypertarget{t`v\string_=\string_B1}{}
\markboth{\textcolor{darkblue}{\textbf{\ipa{ʈv̩˩}}}}{}
\textcolor{teal}{\zh{名词}} \hspace{4pt} \zh{声调类:} L.
\zh{死扣、死结。} \textcolor{Sepia}{\selectlanguage{english}Knot.} \textcolor{PineGreen}{\selectlanguage{french}Noeud.}  ¶ \textcolor{darkblue}{\textbf{\ipa{ɖɯ˧-ʈv̩˩}}} \zh{一个死结} \textcolor{Sepia}{\selectlanguage{english}a knot} \textcolor{PineGreen}{\selectlanguage{french}un noeud}  
 ¶ \textcolor{darkblue}{\textbf{\ipa{ɖɯ˧-ʈv̩˩ | tʰi˧-ʈv̩˩}}} \zh{打一个死结} \textcolor{Sepia}{\selectlanguage{english}to tie a knot} \textcolor{PineGreen}{\selectlanguage{french}faire un noeud}  

\lhead{\firstmark}
\rhead{\botmark}

\subsection{\hspace{-0.5cm} {\Large \textcolor{darkblue}{\textbf{\ipa{ʈv̩˩\textsubscript{a}}}} \textsubscript{1}}\hspace{0.5cm}[\kern2pt{\textcolor{darkblue}{\textbf{\ipa{ʈv̩˧˥}}}}\kern2pt]} \hypertarget{t`v\string_=\string_Ba1}{}
\markboth{\textcolor{darkblue}{\textbf{\ipa{ʈv̩˩\textsubscript{a}}}} \textsubscript{1}}{}
\textcolor{teal}{\zh{动词}} \hspace{4pt} \zh{声调类:} L\textsubscript{a}.
\zh{编(竹子)。} \textcolor{Sepia}{\selectlanguage{english}To weave (bamboo).} \textcolor{PineGreen}{\selectlanguage{french}Tresser (vannerie).}  ¶ \textcolor{darkblue}{\textbf{\ipa{qʰwɤ˧tʰv̩˧ ʈv̩˥}}} \zh{编背水的背篓} \textcolor{Sepia}{\selectlanguage{english}to weave a basket for carrying water} \textcolor{PineGreen}{\selectlanguage{french}tresser une hotte dorsale (en bambou) pour porter l'eau}  
 ¶ \textcolor{darkblue}{\textbf{\ipa{mi˩ɬi˩ ʈv̩˥}}} \zh{编竹子} \textcolor{Sepia}{\selectlanguage{english}to weave bamboo} \textcolor{PineGreen}{\selectlanguage{french}tresser du bambou}  
 ¶ \textcolor{darkblue}{\textbf{\ipa{tso˧\textasciitilde{}tso˧ ʈv̩˥}}} \zh{编东西} \textcolor{Sepia}{\selectlanguage{english}to weave things} \textcolor{PineGreen}{\selectlanguage{french}tresser des choses}  

\lhead{\firstmark}
\rhead{\botmark}

\subsection{\hspace{-0.5cm} {\Large \textcolor{darkblue}{\textbf{\ipa{ʈv̩˩\textsubscript{a}}}} \textsubscript{2}}\hspace{0.5cm}[\kern2pt{\textcolor{darkblue}{\textbf{\ipa{ʈv̩˩˥}}}}\kern2pt]} \hypertarget{t`v\string_=\string_Ba2}{}
\markboth{\textcolor{darkblue}{\textbf{\ipa{ʈv̩˩\textsubscript{a}}}} \textsubscript{2}}{}
\textcolor{teal}{\zh{动词}} \hspace{4pt} \zh{声调类:} L\textsubscript{a}.
\zh{掷(掷石头)。} \textcolor{Sepia}{\selectlanguage{english}To throw (a stone at someone).} \textcolor{PineGreen}{\selectlanguage{french}Lancer (une pierre sur quelqu'un).}  ¶ \textcolor{darkblue}{\textbf{\ipa{mɤ˧-ʈv̩˩}}} \zh{\mytextsc{neg}} \textcolor{Sepia}{\selectlanguage{english}\mytextsc{neg}} \textcolor{PineGreen}{\selectlanguage{french}\mytextsc{neg}}  
 ¶ \textcolor{darkblue}{\textbf{\ipa{lv̩˧mi˧ ʈv̩˩}}} \zh{掷石头} \textcolor{Sepia}{\selectlanguage{english}to throw a stone} \textcolor{PineGreen}{\selectlanguage{french}lancer une pierre}  
 ¶ \textcolor{darkblue}{\textbf{\ipa{tso˧\textasciitilde{}tso˧ ʈv̩˥}}} \zh{掷东西} \textcolor{Sepia}{\selectlanguage{english}to throw things} \textcolor{PineGreen}{\selectlanguage{french}jeter des choses}  

\lhead{\firstmark}
\rhead{\botmark}

\subsection{\hspace{-0.5cm} {\Large \textcolor{darkblue}{\textbf{\ipa{ʈv̩˩\textsubscript{b}}}}}\hspace{0.5cm}[\kern2pt{\textcolor{darkblue}{\textbf{\ipa{ʈv̩˥}}}}\kern2pt]} \hypertarget{t`v\string_=\string_Bb1}{}
\markboth{\textcolor{darkblue}{\textbf{\ipa{ʈv̩˩\textsubscript{b}}}}}{}
\textcolor{teal}{\zh{量词}} \hspace{4pt} \zh{声调类:} L\textsubscript{b}.
\zh{量词:大块,如:一块肉,从一个人的份到几公斤的重量。} \textcolor{Sepia}{\selectlanguage{english}Classifier: a large chunk of: a piece larger than a mouthful. The size can range from a chunk of meat corresponding to one serving for one guest, to a quarter of meat weighing several kilos.} \textcolor{PineGreen}{\selectlanguage{french}Classificateur des quartiers/pièces: morceaux de taille supérieure à une bouchée. Il peut s'agir d'un morceau de viande qu'on donne à un convive, et qui se mange en plusieurs bouchées, mais aussi d'un gros quartier de viande (plusieurs kilos).} 
\lhead{\firstmark}
\rhead{\botmark}

\subsection{\hspace{-0.5cm} {\Large \textcolor{darkblue}{\textbf{\ipa{ʈv̩˩qʰv̩˩}}}}\hspace{0.5cm}[\kern2pt{\textcolor{darkblue}{\textbf{\ipa{ʈv̩˩qʰv̩˩˥}}}}\kern2pt]} \hypertarget{t`v\string_=\string_Bq\string_hv\string_=\string_B1}{}
\markboth{\textcolor{darkblue}{\textbf{\ipa{ʈv̩˩qʰv̩˩}}}}{}
\textcolor{teal}{\zh{名词}} \hspace{4pt} \zh{声调类:} L.
\zh{活扣。} \textcolor{Sepia}{\selectlanguage{english}Slipknot.} \textcolor{PineGreen}{\selectlanguage{french}Noeud coulant.}  ¶ \textcolor{darkblue}{\textbf{\ipa{ʈv̩˩qʰv̩˩˥ | tʰi˧-ʈv̩˩}}} \zh{打活扣} \textcolor{Sepia}{\selectlanguage{english}to tie a slipknot} \textcolor{PineGreen}{\selectlanguage{french}faire un noeud coulant}  
 ¶ \textcolor{darkblue}{\textbf{\ipa{ʈv̩˩qʰv̩˩˥ | ɖɯ˧-ɭɯ˧ | tʰi˧-ʈv̩˩}}} \zh{打一个活扣} \textcolor{Sepia}{\selectlanguage{english}to tie a slipknot} \textcolor{PineGreen}{\selectlanguage{french}faire un noeud coulant}  

\lhead{\firstmark}
\rhead{\botmark}

\subsection{\hspace{-0.5cm} {\Large \textcolor{darkblue}{\textbf{\ipa{ʈwæ˩\textsubscript{a}}}}}\hspace{0.5cm}[\kern2pt{\textcolor{darkblue}{\textbf{\ipa{ʈwæ˧˥}}}}\kern2pt]} \hypertarget{t`w\{\string_Ba1}{}
\markboth{\textcolor{darkblue}{\textbf{\ipa{ʈwæ˩\textsubscript{a}}}}}{}
\textcolor{teal}{\zh{动词}} \hspace{4pt} \zh{声调类:} L\textsubscript{a}.
\zh{冻。} \textcolor{Sepia}{\selectlanguage{english}To freeze, to solidify.} \textcolor{PineGreen}{\selectlanguage{french}Geler, se figer.}  ¶ \textcolor{darkblue}{\textbf{\ipa{dʑɯ˩ ʈwæ˩˥}}} \zh{水冻成冰} \textcolor{Sepia}{\selectlanguage{english}water freezes} \textcolor{PineGreen}{\selectlanguage{french}l'eau gèle, il gèle}  
 ¶ \textcolor{darkblue}{\textbf{\ipa{dʑɯ˩pʰæ˩ ʈwæ˧-ze˩}}} \zh{水冻成冰了。} \textcolor{Sepia}{\selectlanguage{english}ice has formed} \textcolor{PineGreen}{\selectlanguage{french}l'eau a gelé, de la glace s'est formée}  
 ¶ \textcolor{darkblue}{\textbf{\ipa{ɖɯ˧-ʈwæ˧\textasciitilde{}ʈwæ˥-ɻ̍˩ kʰɯ˩}}} \zh{冷冻,放在冷箱} \textcolor{Sepia}{\selectlanguage{english}to put to freeze, to put in the deep freeze} \textcolor{PineGreen}{\selectlanguage{french}faire geler, mettre à congeler}  

\lhead{\firstmark}
\rhead{\botmark}

\subsection{\hspace{-0.5cm} {\Large \textcolor{darkblue}{\textbf{\ipa{ʈwæ˧˥}}}}\hspace{0.5cm}[\kern2pt{\textcolor{darkblue}{\textbf{\ipa{ʈwæ˥}}}}\kern2pt]} \hypertarget{t`w\{\string_M\string_T1}{}
\markboth{\textcolor{darkblue}{\textbf{\ipa{ʈwæ˧˥}}}}{}
\textcolor{teal}{\zh{动词}} \hspace{4pt} \zh{声调类:} MH.
\zh{跌倒(路很滑)。} \textcolor{Sepia}{\selectlanguage{english}To fall down (on a slippery road).} \textcolor{PineGreen}{\selectlanguage{french}Tomber (en glissant).} 
\lhead{\firstmark}
\rhead{\botmark}

\subsection{\hspace{-0.5cm} {\Large \textcolor{darkblue}{\textbf{\ipa{ʈwɤ˧\textsubscript{a}}}}}\hspace{0.5cm}[\kern2pt{\textcolor{darkblue}{\textbf{\ipa{ʈwɤ˩˥}}}}\kern2pt]} \hypertarget{t`w7\string_Ma1}{}
\markboth{\textcolor{darkblue}{\textbf{\ipa{ʈwɤ˧\textsubscript{a}}}}}{}
\textcolor{teal}{\zh{动词}} \hspace{4pt} \zh{声调类:} M\textsubscript{a}.
\zh{啼,鸡叫。} \textcolor{Sepia}{\selectlanguage{english}To sing (of bird), to cock-a-doodle-doo (cock).} \textcolor{PineGreen}{\selectlanguage{french}Le coq chante/fait cocorico; un oiseau chante.}  ¶ \textcolor{darkblue}{\textbf{\ipa{æ̃˩ ʈwɤ˧ (+ze˧)}}} \zh{鸡叫。} \textcolor{Sepia}{\selectlanguage{english}The cock sings.} \textcolor{PineGreen}{\selectlanguage{french}Le coq chante/fait cocorico.}  

\lhead{\firstmark}
\rhead{\botmark}

\newpage
\section*{\centering- \textcolor{darkblue}{\textbf{\ipa{ʈʰ}}} -}
\subsection{\hspace{-0.5cm} {\Large \textcolor{darkblue}{\textbf{\ipa{ʈʰæ˧-mɤ˧-ʝi\#˥}}}}\hspace{0.5cm}[\kern2pt{\textcolor{darkblue}{\textbf{\ipa{xxxx non-correspondance entre le nombre de morphèmes et le nombre de tons de morphèmes}}}}\kern2pt]} \hypertarget{t`\string_h\{\string_M-m7\string_M-j££i\#\string_T1}{}
\markboth{\textcolor{darkblue}{\textbf{\ipa{ʈʰæ˧-mɤ˧-ʝi\#˥}}}}{}
\textcolor{teal}{\zh{形容词}} \hspace{4pt} \zh{声调类:} \#H.
\zh{乱。} \textcolor{Sepia}{\selectlanguage{english}Disorderly.} \textcolor{PineGreen}{\selectlanguage{french}Désordonné.}  ¶ \textcolor{darkblue}{\textbf{\ipa{ɑ˩ʁo˧ | ʈʰæ˧-mɤ˧-ʝi˧! |}}} \zh{家很乱!} \textcolor{Sepia}{\selectlanguage{english}The house is in a mess!} \textcolor{PineGreen}{\selectlanguage{french}la maison est en grand désordre!}  
 ¶ \textcolor{darkblue}{\textbf{\ipa{ʈʰæ˧-mɤ˧-ʝi˧ ɲi˥! |}}} \zh{真乱!} \textcolor{Sepia}{\selectlanguage{english}It's a real mess!} \textcolor{PineGreen}{\selectlanguage{french}C'est vraiment le désordre!}  

\lhead{\firstmark}
\rhead{\botmark}

\subsection{\hspace{-0.5cm} {\Large \textcolor{darkblue}{\textbf{\ipa{ʈʰæ˧mi˧-ɳɯ˩}}}}\hspace{0.5cm}[\kern2pt{\textcolor{darkblue}{\textbf{\ipa{xxxx non-correspondance entre le nombre de morphèmes et le nombre de tons de morphèmes}}}}\kern2pt]} \hypertarget{t`\string_h\{\string_Mmi\string_M-n`M\string_B1}{}
\markboth{\textcolor{darkblue}{\textbf{\ipa{ʈʰæ˧mi˧-ɳɯ˩}}}}{}
\textcolor{teal}{\zh{助词}} \hspace{4pt} \zh{声调类:} L\#.
\zh{真的。} \textcolor{Sepia}{\selectlanguage{english}Really, actually.} \textcolor{PineGreen}{\selectlanguage{french}Vraiment, réellement.}  ¶ \textcolor{darkblue}{\textbf{\ipa{ʈʂʰɯ˧ | ʈʰæ˧mi˧-ɳɯ˩ | go˩˥!}}} \zh{他真的病了!} \textcolor{Sepia}{\selectlanguage{english}(S)he is really ill!} \textcolor{PineGreen}{\selectlanguage{french}Il/elle est vraiment malade!}  

\lhead{\firstmark}
\rhead{\botmark}

\subsection{\hspace{-0.5cm} {\Large \textcolor{darkblue}{\textbf{\ipa{-ʈʰæ˧qo˩}}}}\hspace{0.5cm}[\kern2pt{\textcolor{darkblue}{\textbf{\ipa{ʈʰæ˧qo˩}}}}\kern2pt]} \hypertarget{-t`\string_h\{\string_Mqo\string_B1}{}
\markboth{\textcolor{darkblue}{\textbf{\ipa{-ʈʰæ˧qo˩}}}}{}
\textcolor{teal}{\zh{后置词}} \hspace{4pt} \zh{声调类:} L\#.
\zh{……之下、下面。} \textcolor{Sepia}{\selectlanguage{english}Under.} \textcolor{PineGreen}{\selectlanguage{french}Sous (sous le ciel; sous la tente); au pied (d'une montagne).} 
\lhead{\firstmark}
\rhead{\botmark}

\subsection{\hspace{-0.5cm} {\Large \textcolor{darkblue}{\textbf{\ipa{ʈʰæ˧qʰwɤ˧}}}}\hspace{0.5cm}[\kern2pt{\textcolor{darkblue}{\textbf{\ipa{ʈʰæ˧qʰwɤ˧}}}}\kern2pt]} \hypertarget{t`\string_h\{\string_Mq\string_hw7\string_M1}{}
\markboth{\textcolor{darkblue}{\textbf{\ipa{ʈʰæ˧qʰwɤ˧}}}}{}
\textcolor{teal}{\zh{名词}} \hspace{4pt} \zh{声调类:} M.
\zh{裙子。} \textcolor{Sepia}{\selectlanguage{english}Skirt.} \textcolor{PineGreen}{\selectlanguage{french}Jupe.}  \zh{量词}: \textcolor{darkblue}{\textbf{\ipa{ɭɯ˧˥}}} 
\lhead{\firstmark}
\rhead{\botmark}

\subsection{\hspace{-0.5cm} {\Large \textcolor{darkblue}{\textbf{\ipa{*ʈʰæ˩}}}}\hspace{0.5cm}[\kern2pt{\textcolor{darkblue}{\textbf{\ipa{ʈʰæ˥}}}}\kern2pt]} \hypertarget{*t`\string_h\{\string_B1}{}
\markboth{\textcolor{darkblue}{\textbf{\ipa{*ʈʰæ˩}}}}{}
\textcolor{teal}{\zh{名词}} \hspace{4pt} \zh{声调类:} L.
\zh{裙子(单音节)。} \textcolor{Sepia}{\selectlanguage{english}Skirt (monosyllabic form extracted from the set phrase 'to wear the skirt').} \textcolor{PineGreen}{\selectlanguage{french}Jupe; monosyllabe extrait d'après le comportement dans l'expression figée /ʈʰæ˩ ki˩˥/ 'enfiler la jupe', avec verbe au ton La (nom du rituel de passage à l'âge adulte). Le monosyllabe n'est pas usité hors de cette expression. Par exemple, */ʈʰæ˩ ɲi˩˥/ 'c'est une jupe' est catégoriquement refusé par F4.}  ¶ \textcolor{darkblue}{\textbf{\ipa{ʈʰæ˧ | le˧-ki˩}}} \zh{穿上裙子} \textcolor{Sepia}{\selectlanguage{english}to put on the skirt (\mytextsc{accomp})} \textcolor{PineGreen}{\selectlanguage{french}enfiler la jupe (\mytextsc{accomp})}  
 ¶ \textcolor{darkblue}{\textbf{\ipa{ʈʰæ˩ ki˩˥}}} \zh{女人成年的礼仪:“穿裙子”} \textcolor{Sepia}{\selectlanguage{english}the ritual of coming of age for women: “putting on the skirt”} \textcolor{PineGreen}{\selectlanguage{french}rituel de passage à l'âge adulte, pour les jeunes femmes: “enfiler la jupe”}  

\lhead{\firstmark}
\rhead{\botmark}

\subsection{\hspace{-0.5cm} {\Large \textcolor{darkblue}{\textbf{\ipa{ʈʰæ˩ki˩}}}}\hspace{0.5cm}[\kern2pt{\textcolor{darkblue}{\textbf{\ipa{ʈʰæ˩ki˩˥}}}}\kern2pt]} \hypertarget{t`\string_h\{\string_Bki\string_B1}{}
\markboth{\textcolor{darkblue}{\textbf{\ipa{ʈʰæ˩ki˩}}}}{}
\textcolor{teal}{\zh{动词}} \hspace{4pt} \zh{声调类:} L.
\zh{举行女孩的成年礼。} \textcolor{Sepia}{\selectlanguage{english}To perform the ceremony for a person's coming of age.} \textcolor{PineGreen}{\selectlanguage{french}Réaliser la cérémonie de passage à l'âge adulte des femmes.}  ¶ \textcolor{darkblue}{\textbf{\ipa{ʈʰæ˩ki˩-ze˥!}}} \zh{穿裙了! / 行过穿裙礼了! / 她成年了!} \textcolor{Sepia}{\selectlanguage{english}She has come of age! / The ceremony for her coming of age has been performed!} \textcolor{PineGreen}{\selectlanguage{french}Elle est adulte maintenant! / La cérémonie de passage à l'âge adulte a été réalisée!}  

\lhead{\firstmark}
\rhead{\botmark}

\subsection{\hspace{-0.5cm} {\Large \textcolor{darkblue}{\textbf{\ipa{ʈʰæ˩\textasciitilde{}ʈʰæ˧˥}}}}\hspace{0.5cm}[\kern2pt{\textcolor{darkblue}{\textbf{\ipa{ʈʰæ˧ʈʰæ˧˥}}}}\kern2pt]} \hypertarget{t`\string_h\{\string_B~t`\string_h\{\string_M\string_T1}{}
\markboth{\textcolor{darkblue}{\textbf{\ipa{ʈʰæ˩\textasciitilde{}ʈʰæ˧˥}}}}{}
\textcolor{teal}{\zh{动词}} \hspace{4pt} \zh{声调类:} MH.
\zh{痒。} \textcolor{Sepia}{\selectlanguage{english}To itch.} \textcolor{PineGreen}{\selectlanguage{french}Démanger.}  ¶ \textcolor{darkblue}{\textbf{\ipa{le˧-ʈʰæ˩\textasciitilde{}ʈʰæ˩-ze˩}}} \zh{\mytextsc{accomp} \mytextsc{red} \mytextsc{pfv}} \textcolor{Sepia}{\selectlanguage{english}\mytextsc{accomp} \mytextsc{red} \mytextsc{pfv}} \textcolor{PineGreen}{\selectlanguage{french}\mytextsc{accomp} \mytextsc{red} \mytextsc{pfv}}  

\lhead{\firstmark}
\rhead{\botmark}

\subsection{\hspace{-0.5cm} {\Large \textcolor{darkblue}{\textbf{\ipa{ʈʰæ˧˥}}}}\hspace{0.5cm}[\kern2pt{\textcolor{darkblue}{\textbf{\ipa{ʈʰæ˧˥}}}}\kern2pt]} \hypertarget{t`\string_h\{\string_M\string_T1}{}
\markboth{\textcolor{darkblue}{\textbf{\ipa{ʈʰæ˧˥}}}}{}
\textcolor{teal}{\zh{动词}} \hspace{4pt} \zh{声调类:} MH.
\ding{202} \zh{咬、叮。} \textcolor{Sepia}{\selectlanguage{english}To bite; to sting.} \textcolor{PineGreen}{\selectlanguage{french}Mordre (mordre à belles dents dans quelque chose); piquer (une abeille pique quelqu'un).}  ¶ \textcolor{darkblue}{\textbf{\ipa{tso˧\textasciitilde{}tso˧ ʈʰæ˩(-ze˩)}}} \zh{咬东西} \textcolor{Sepia}{\selectlanguage{english}to bite things} \textcolor{PineGreen}{\selectlanguage{french}mordre quelque chose}  
 ¶ \textcolor{darkblue}{\textbf{\ipa{hĩ˧ ʈʰæ˩}}} \zh{咬人} \textcolor{Sepia}{\selectlanguage{english}to bite someone (e.g. a dog bites a stranger)} \textcolor{PineGreen}{\selectlanguage{french}mordre quelqu'un (ex.: un chien mord un inconnu de passage)}  
\ding{203} \zh{对号、合适、相配:建房时,两块木材调剂地刚好合适,好像互相“咬紧”的样子。} \textcolor{Sepia}{\selectlanguage{english}To fit, to adjust, to match: e.g. when building a house, to pieces of carpentry fit each other exactly, and 'bite' into each other to perfection.} \textcolor{PineGreen}{\selectlanguage{french}S'emboîter, s'ajuster (au sujet de pièces de charpenterie); emploi figuré de “mordre”: les pièces s'ajustent comme si elles mordaient les unes dans les autres.} 
\lhead{\firstmark}
\rhead{\botmark}

\subsection{\hspace{-0.5cm} {\Large \textcolor{darkblue}{\textbf{\ipa{ʈʰɤ˥\textsubscript{a}}}}}\hspace{0.5cm}[\kern2pt{\textcolor{darkblue}{\textbf{\ipa{ʈʰɤ˥}}}}\kern2pt]} \hypertarget{t`\string_h7\string_Ta1}{}
\markboth{\textcolor{darkblue}{\textbf{\ipa{ʈʰɤ˥\textsubscript{a}}}}}{}
\textcolor{teal}{\zh{量词}} \hspace{4pt} \zh{声调类:} H\textsubscript{a}.
\zh{量词:滴。} \textcolor{Sepia}{\selectlanguage{english}A drop (of liquid).} \textcolor{PineGreen}{\selectlanguage{french}Goutte (une goutte de liquide).} 
\lhead{\firstmark}
\rhead{\botmark}

\subsection{\hspace{-0.5cm} {\Large \textcolor{darkblue}{\textbf{\ipa{ʈʰɤ˧˥}}}}\hspace{0.5cm}[\kern2pt{\textcolor{darkblue}{\textbf{\ipa{ʈʰɤ˩˥}}}}\kern2pt]} \hypertarget{t`\string_h7\string_M\string_T1}{}
\markboth{\textcolor{darkblue}{\textbf{\ipa{ʈʰɤ˧˥}}}}{}
\textcolor{teal}{\zh{动词}} \hspace{4pt} \zh{声调类:} MH.
\zh{滴(水往下滴)。} \textcolor{Sepia}{\selectlanguage{english}To drip, to dribble.} \textcolor{PineGreen}{\selectlanguage{french}Goutter, dégouliner, couler goutte à goutte.}  ¶ \textcolor{darkblue}{\textbf{\ipa{tʰi˧-ʈʰɤ˩\textasciitilde{}ʈʰɤ˩}}} \zh{滴着滴着} \textcolor{Sepia}{\selectlanguage{english}\mytextsc{dur} \mytextsc{red}} \textcolor{PineGreen}{\selectlanguage{french}\mytextsc{dur} \mytextsc{red}}  

\lhead{\firstmark}
\rhead{\botmark}

\subsection{\hspace{-0.5cm} {\Large \textcolor{darkblue}{\textbf{\ipa{ʈʰi˩\textsubscript{a}}}}}\hspace{0.5cm}[\kern2pt{\textcolor{darkblue}{\textbf{\ipa{ʈʰi˥}}}}\kern2pt]} \hypertarget{t`\string_hi\string_Ba1}{}
\markboth{\textcolor{darkblue}{\textbf{\ipa{ʈʰi˩\textsubscript{a}}}}}{}
\textcolor{teal}{\zh{形容词}} \hspace{4pt} \zh{声调类:} L\textsubscript{a}.
\zh{累、疲倦、精疲力竭。} \textcolor{Sepia}{\selectlanguage{english}Tired, weary.} \textcolor{PineGreen}{\selectlanguage{french}Fatigué.}  ¶ \textcolor{darkblue}{\textbf{\ipa{le˧-ʈʰi˩-ze˩}}} \zh{累了} \textcolor{Sepia}{\selectlanguage{english}\mytextsc{accomp} \string_ \mytextsc{pfv}} \textcolor{PineGreen}{\selectlanguage{french}\mytextsc{accomp} \string_ \mytextsc{pfv}}  
 ¶ \textcolor{darkblue}{\textbf{\ipa{njɤ˧ | ʈʰi˩˥!}}} \zh{我累了!} \textcolor{Sepia}{\selectlanguage{english}I am tired!} \textcolor{PineGreen}{\selectlanguage{french}je suis fatigué!}  
 ¶ \textcolor{darkblue}{\textbf{\ipa{njɤ˧ | ʈʰi˩-ze˥!}}} \zh{我累了!} \textcolor{Sepia}{\selectlanguage{english}I am tired!} \textcolor{PineGreen}{\selectlanguage{french}je suis fatigué!}  

\lhead{\firstmark}
\rhead{\botmark}

\subsection{\hspace{-0.5cm} {\Large \textcolor{darkblue}{\textbf{\ipa{ʈʰɯ˩\textsubscript{a}}}}}\hspace{0.5cm}[\kern2pt{\textcolor{darkblue}{\textbf{\ipa{ʈʰɯ˥}}}}\kern2pt]} \hypertarget{t`\string_hM\string_Ba1}{}
\markboth{\textcolor{darkblue}{\textbf{\ipa{ʈʰɯ˩\textsubscript{a}}}}}{}
\textcolor{teal}{\zh{动词}} \hspace{4pt} \zh{声调类:} L\textsubscript{a}.
\zh{打喷嚏。} \textcolor{Sepia}{\selectlanguage{english}To sneeze.} \textcolor{PineGreen}{\selectlanguage{french}Éternuer.}  ¶ \textcolor{darkblue}{\textbf{\ipa{ɖɯ˧-ʈʰɯ˧\textasciitilde{}ʈʰɯ˥}}} \textcolor{Sepia}{\selectlanguage{english}\mytextsc{inceptive} \mytextsc{red}} \textcolor{PineGreen}{\selectlanguage{french}\mytextsc{inchoatif} \mytextsc{red}}  

\lhead{\firstmark}
\rhead{\botmark}

\subsection{\hspace{-0.5cm} {\Large \textcolor{darkblue}{\textbf{\ipa{ʈʰɯ˩\textsubscript{b}}}}}\hspace{0.5cm}[\kern2pt{\textcolor{darkblue}{\textbf{\ipa{ʈʰɯ˩˥}}}}\kern2pt]} \hypertarget{t`\string_hM\string_Bb1}{}
\markboth{\textcolor{darkblue}{\textbf{\ipa{ʈʰɯ˩\textsubscript{b}}}}}{}
\textcolor{teal}{\zh{动词}} \hspace{4pt} \zh{声调类:} L\textsubscript{b}.
\zh{喝。} \textcolor{Sepia}{\selectlanguage{english}To drink.} \textcolor{PineGreen}{\selectlanguage{french}Boire.}  ¶ \textcolor{darkblue}{\textbf{\ipa{njɤ˧ | mɤ˧-ʈʰɯ˩}}} \zh{我不喝} \textcolor{Sepia}{\selectlanguage{english}I don't drink} \textcolor{PineGreen}{\selectlanguage{french}je ne bois pas}  
 ¶ \textcolor{darkblue}{\textbf{\ipa{ʈʰɯ˩-ze˥}}} \zh{喝了} \textcolor{Sepia}{\selectlanguage{english}\mytextsc{pfv}} \textcolor{PineGreen}{\selectlanguage{french}\mytextsc{pfv}}  
 ¶ \textcolor{darkblue}{\textbf{\ipa{le˧-ʈʰɯ˩-ze˩}}} \zh{\mytextsc{accomp} \string_ \mytextsc{pfv}} \textcolor{Sepia}{\selectlanguage{english}\mytextsc{accomp} \string_ \mytextsc{pfv}} \textcolor{PineGreen}{\selectlanguage{french}\mytextsc{accomp} \string_ \mytextsc{pfv}}  
 ¶ \textcolor{darkblue}{\textbf{\ipa{ʐɯ˧ ʈʰɯ˩}}} \zh{喝酒} \textcolor{Sepia}{\selectlanguage{english}to drink wine} \textcolor{PineGreen}{\selectlanguage{french}boire du vin}  
 ¶ \textcolor{darkblue}{\textbf{\ipa{jɤ˧ ʈʰɯ˩}}} \zh{抽烟} \textcolor{Sepia}{\selectlanguage{english}to smoke (tobacco)} \textcolor{PineGreen}{\selectlanguage{french}fumer (du tabac)}  
 ¶ \textcolor{darkblue}{\textbf{\ipa{dʑɯ˩qʰæ˩ ʈʰɯ˩˥}}} \zh{喝凉水} \textcolor{Sepia}{\selectlanguage{english}to drink cold water} \textcolor{PineGreen}{\selectlanguage{french}boire de l'eau froide}  
 ¶ \textcolor{darkblue}{\textbf{\ipa{dʑɯ˩tsʰi˩ ʈʰɯ˩˥}}} \zh{喝热水} \textcolor{Sepia}{\selectlanguage{english}to drink hot water} \textcolor{PineGreen}{\selectlanguage{french}boire de l'eau chaude}  
 ¶ \textcolor{darkblue}{\textbf{\ipa{li˩ ʈʰɯ˩}}} \zh{喝茶} \textcolor{Sepia}{\selectlanguage{english}to drink tea} \textcolor{PineGreen}{\selectlanguage{french}boire du thé}  
 ¶ \textcolor{darkblue}{\textbf{\ipa{v̩˩dʑɯ˩ ʈʰɯ˩˥}}} \zh{喝汤} \textcolor{Sepia}{\selectlanguage{english}to drink soup} \textcolor{PineGreen}{\selectlanguage{french}boire de la soupe}  
 ¶ \textcolor{darkblue}{\textbf{\ipa{dʑɯ˧ ʈʰɯ˧}}} \zh{喝水} \textcolor{Sepia}{\selectlanguage{english}to drink water} \textcolor{PineGreen}{\selectlanguage{french}boire de l'eau}  
 ¶ \textcolor{darkblue}{\textbf{\ipa{njɤ˧ | dʑɯ˧ ʈʰɯ˧-ze˧}}} \zh{我喝了水} \textcolor{Sepia}{\selectlanguage{english}I have drunk some water} \textcolor{PineGreen}{\selectlanguage{french}j'ai bu de l'eau}  
 ¶ \textcolor{darkblue}{\textbf{\ipa{njɤ˧ | dʑɯ˧ ʈʰɯ˧-zo˧-ho˩}}} \zh{我应该喝水了。} \textcolor{Sepia}{\selectlanguage{english}I'm going to have to drink water.} \textcolor{PineGreen}{\selectlanguage{french}Il va falloir que je boive de l'eau.}  

\lhead{\firstmark}
\rhead{\botmark}

\newpage
\section*{\centering- \textcolor{darkblue}{\textbf{\ipa{ʈʂ}}} -}
\subsection{\hspace{-0.5cm} {\Large \textcolor{darkblue}{\textbf{\ipa{ʈʂɑ˧tɑ˥}}}}\hspace{0.5cm}[\kern2pt{\textcolor{darkblue}{\textbf{\ipa{ʈʂɑ˧tɑ˥}}}}\kern2pt]} \hypertarget{t`s`A\string_MtA\string_T1}{}
\markboth{\textcolor{darkblue}{\textbf{\ipa{ʈʂɑ˧tɑ˥}}}}{}
\textcolor{teal}{\zh{名词}} \hspace{4pt} \zh{声调类:} H\#.
\zh{记号。} \textcolor{Sepia}{\selectlanguage{english}Sign.} \textcolor{PineGreen}{\selectlanguage{french}Signe.}  ¶ \textcolor{darkblue}{\textbf{\ipa{ʈʂɑ˧tɑ˥ ʝi˩}}} \zh{写一个符号、画一个符号} \textcolor{Sepia}{\selectlanguage{english}to make a mark, to write a sign} \textcolor{PineGreen}{\selectlanguage{french}faire une marque, inscrire un signe}  
 ¶ \textcolor{darkblue}{\textbf{\ipa{ʈʂɑ˧tɑ˥ tɕi˩}}} \zh{写符号、画符号} \textcolor{Sepia}{\selectlanguage{english}to write signs, to make marks} \textcolor{PineGreen}{\selectlanguage{french}écrire des signes, faire des marques (pas pour un unique signe: se rapproche de l'écriture d'un message/texte)}  
 \zh{量词}: \textcolor{darkblue}{\textbf{\ipa{kʰwɤ˥}}} 
\lhead{\firstmark}
\rhead{\botmark}

\subsection{\hspace{-0.5cm} {\Large \textcolor{darkblue}{\textbf{\ipa{ʈʂæ˧mo\#˥}}}}\hspace{0.5cm}[\kern2pt{\textcolor{darkblue}{\textbf{\ipa{ʈʂæ˧mo˧}}}}\kern2pt]} \hypertarget{t`s`\{\string_Mmo\#\string_T1}{}
\markboth{\textcolor{darkblue}{\textbf{\ipa{ʈʂæ˧mo\#˥}}}}{}
\textcolor{teal}{\zh{名词}} \hspace{4pt} \zh{声调类:} \#H.
\zh{一种有毒的菌子。} \textcolor{Sepia}{\selectlanguage{english}A poisonous mushroom.} \textcolor{PineGreen}{\selectlanguage{french}Un champignon vénéneux.}  ¶ \textcolor{darkblue}{\textbf{\ipa{ʈʂæ˧mo˧-kʰi˧tɕʰɯ˩-mo˩ / kʰi˧tɕʰɯ˩-mo˩}}} \zh{同上} \textcolor{Sepia}{\selectlanguage{english}same meaning} \textcolor{PineGreen}{\selectlanguage{french}même sens}  
\zh{~【同义词】~} \hyperlink{}{\textcolor{darkblue}{\textbf{\ipa{kʰi˧tɕʰɯ˩-mo˩}}}}. 
\lhead{\firstmark}
\rhead{\botmark}

\subsection{\hspace{-0.5cm} {\Large \textcolor{darkblue}{\textbf{\ipa{ʈʂæ˧ʈʂɯ˧}}}}\hspace{0.5cm}[\kern2pt{\textcolor{darkblue}{\textbf{\ipa{ʈʂæ˧ʈʂɯ˧}}}}\kern2pt]} \hypertarget{t`s`\{\string_Mt`s`M\string_M1}{}
\markboth{\textcolor{darkblue}{\textbf{\ipa{ʈʂæ˧ʈʂɯ˧}}}}{}
\textcolor{teal}{\zh{助词}} \hspace{4pt} \zh{声调类:} M.
\zh{确切、真的。} \textcolor{Sepia}{\selectlanguage{english}Truthfully, accurately, really.} \textcolor{PineGreen}{\selectlanguage{french}Véritablement, vraiment, pour de vrai.} 
\lhead{\firstmark}
\rhead{\botmark}

\subsection{\hspace{-0.5cm} {\Large \textcolor{darkblue}{\textbf{\ipa{ʈʂæ˧wɤ˩}}}}\hspace{0.5cm}[\kern2pt{\textcolor{darkblue}{\textbf{\ipa{ʈʂæ˧wɤ˩}}}}\kern2pt]} \hypertarget{t`s`\{\string_Mw7\string_B1}{}
\markboth{\textcolor{darkblue}{\textbf{\ipa{ʈʂæ˧wɤ˩}}}}{}
\textcolor{teal}{\zh{名词}} \hspace{4pt} \zh{声调类:} L\#.
\zh{仆人,佣人。} \textcolor{Sepia}{\selectlanguage{english}Servant.} \textcolor{PineGreen}{\selectlanguage{french}Serviteur.}  \zh{量词}: \textcolor{darkblue}{\textbf{\ipa{v̩˧}}} 
\lhead{\firstmark}
\rhead{\botmark}

\subsection{\hspace{-0.5cm} {\Large \textcolor{darkblue}{\textbf{\ipa{ʈʂæ˩do\#˥}}}}\hspace{0.5cm}[\kern2pt{\textcolor{darkblue}{\textbf{\ipa{ʈʂæ˧do˧}}}}\kern2pt]} \hypertarget{t`s`\{\string_Bdo\#\string_T1}{}
\markboth{\textcolor{darkblue}{\textbf{\ipa{ʈʂæ˩do\#˥}}}}{}
\textcolor{teal}{\zh{名词}} \hspace{4pt} \zh{声调类:} LM+\#H.
\zh{打酥油茶的罐、酥油茶搅拌器,黄油搅乳器。} \textcolor{Sepia}{\selectlanguage{english}Container in which butter-tea is mixed; also: butter churn.} \textcolor{PineGreen}{\selectlanguage{french}Récipient dans lequel on bat le thé au beurre (tube-baratte en bois); aussi: grande baratte pour baratter le beurre.}  \zh{量词}: \textcolor{darkblue}{\textbf{\ipa{ɭɯ˧}}} 
\lhead{\firstmark}
\rhead{\botmark}

\subsection{\hspace{-0.5cm} {\Large \textcolor{darkblue}{\textbf{\ipa{ʈʂæ˧˥}}} \textsubscript{1}}\hspace{0.5cm}[\kern2pt{\textcolor{darkblue}{\textbf{\ipa{ʈʂæ˧˥}}}}\kern2pt]} \hypertarget{t`s`\{\string_M\string_T1}{}
\markboth{\textcolor{darkblue}{\textbf{\ipa{ʈʂæ˧˥}}} \textsubscript{1}}{}
\textcolor{teal}{\zh{动词}} \hspace{4pt} \zh{声调类:} MH.
\zh{抢劫、抢。} \textcolor{Sepia}{\selectlanguage{english}To rob, to steal.} \textcolor{PineGreen}{\selectlanguage{french}Voler, s'emparer de, extorquer, arracher.}  ¶ \textcolor{darkblue}{\textbf{\ipa{le˧-ʈʂæ˧-ze˥}}} \zh{抢了} \textcolor{Sepia}{\selectlanguage{english}\mytextsc{accomp} \string_ \mytextsc{pfv}} \textcolor{PineGreen}{\selectlanguage{french}\mytextsc{accomp} \string_ \mytextsc{pfv}}  
 ¶ \textcolor{darkblue}{\textbf{\ipa{tso˧\textasciitilde{}tso˧ ʈʂæ˩}}} \zh{抢东西} \textcolor{Sepia}{\selectlanguage{english}to steal things} \textcolor{PineGreen}{\selectlanguage{french}voler des choses}  
 ¶ \textcolor{darkblue}{\textbf{\ipa{le˧-ʈʂæ˧-po˥-hɯ˩(-ze˩)}}} \zh{把东西抢走了} \textcolor{Sepia}{\selectlanguage{english}[(S)he] stole away (something)} \textcolor{PineGreen}{\selectlanguage{french}(il) a extorqué quelque chose et est parti avec}  
 ¶ \textcolor{darkblue}{\textbf{\ipa{hĩ˧ ʈʂæ˩}}} \zh{抢劫} \textcolor{Sepia}{\selectlanguage{english}to rob people, to steal from people} \textcolor{PineGreen}{\selectlanguage{french}voler les gens, extorquer des choses aux gens}  

\lhead{\firstmark}
\rhead{\botmark}

\subsection{\hspace{-0.5cm} {\Large \textcolor{darkblue}{\textbf{\ipa{ʈʂæ˧˥}}} \textsubscript{2}}\hspace{0.5cm}[\kern2pt{\textcolor{darkblue}{\textbf{\ipa{ʈʂæ˧˥}}}}\kern2pt]} \hypertarget{t`s`\{\string_M\string_T2}{}
\markboth{\textcolor{darkblue}{\textbf{\ipa{ʈʂæ˧˥}}} \textsubscript{2}}{}
\textcolor{teal}{\zh{动词}} \hspace{4pt} \zh{声调类:} MH.
\zh{派人。} \textcolor{Sepia}{\selectlanguage{english}To send someone.} \textcolor{PineGreen}{\selectlanguage{french}Envoyer qqun.}  ¶ \textcolor{darkblue}{\textbf{\ipa{ɖɯ˧-v̩˧ ʈʂæ˧˥}}} \zh{派一个人} \textcolor{Sepia}{\selectlanguage{english}to send someone} \textcolor{PineGreen}{\selectlanguage{french}envoyer quelqu'un}  
 ¶ \textcolor{darkblue}{\textbf{\ipa{hĩ˧ ʈʂæ˩}}} \zh{同上} \textcolor{Sepia}{\selectlanguage{english}as above} \textcolor{PineGreen}{\selectlanguage{french}idem}  

\lhead{\firstmark}
\rhead{\botmark}

\subsection{\hspace{-0.5cm} {\Large \textcolor{darkblue}{\textbf{\ipa{ʈʂæ˧˥}}} \textsubscript{3}}\hspace{0.5cm}[\kern2pt{\textcolor{darkblue}{\textbf{\ipa{ʈʂæ˧˥}}}}\kern2pt]} \hypertarget{t`s`\{\string_M\string_T3}{}
\markboth{\textcolor{darkblue}{\textbf{\ipa{ʈʂæ˧˥}}} \textsubscript{3}}{}
\textcolor{teal}{\zh{动词}} \hspace{4pt} \zh{声调类:} MH.
\zh{安上(如:缝扣子、安上马鞍)。} \textcolor{Sepia}{\selectlanguage{english}To set, to attach (e.g. to sew a button, to put a saddle on a horse).} \textcolor{PineGreen}{\selectlanguage{french}Fixer, accrocher (ex.: coudre un bouton sur un vêtement; attacher la selle sur un cheval).}  ¶ \textcolor{darkblue}{\textbf{\ipa{pv̩˩ɭɯ˥ ʈʂæ˩}}} \zh{缝扣子} \textcolor{Sepia}{\selectlanguage{english}to sew a button} \textcolor{PineGreen}{\selectlanguage{french}coudre un bouton (sur un vêtement)}  
 ¶ \textcolor{darkblue}{\textbf{\ipa{ʐwæ˧tɕi˥ ʈʂæ˩}}} \zh{备鞍} \textcolor{Sepia}{\selectlanguage{english}to put a saddle on a horse, to saddle a horse} \textcolor{PineGreen}{\selectlanguage{french}attacher la selle d'un cheval; seller un cheval}  
 ¶ \textcolor{darkblue}{\textbf{\ipa{ɖɯ˧-ɲi˥, | so˧-ʂɯ˧ ʈʂæ˧˥!}}} \zh{(走马帮,)一天备鞍三次!} \textcolor{Sepia}{\selectlanguage{english}In one day [of caravan journey], one saddles (horses) three times!} \textcolor{PineGreen}{\selectlanguage{french}Au cours d'une journée, on selle trois fois (les chevaux, lorsqu'on est parti en caravane)!}  

\lhead{\firstmark}
\rhead{\botmark}

\subsection{\hspace{-0.5cm} {\Large \textcolor{darkblue}{\textbf{\ipa{ʈʂæ˧˥}}} \textsubscript{4}}\hspace{0.5cm}[\kern2pt{\textcolor{darkblue}{\textbf{\ipa{ʈʂæ˧˥}}}}\kern2pt]} \hypertarget{t`s`\{\string_M\string_T4}{}
\markboth{\textcolor{darkblue}{\textbf{\ipa{ʈʂæ˧˥}}} \textsubscript{4}}{}
\textcolor{teal}{\zh{名词}} \hspace{4pt} \zh{声调类:} MH.
\ding{202} \zh{关节。} \textcolor{Sepia}{\selectlanguage{english}Articulation.} \textcolor{PineGreen}{\selectlanguage{french}Articulation.}  \zh{量词}: \textcolor{darkblue}{\textbf{\ipa{ʈʂæ˧˥}}} \ding{203} \zh{段(时间)、时代。} \textcolor{Sepia}{\selectlanguage{english}Period, epoch, age, era, span of time.} \textcolor{PineGreen}{\selectlanguage{french}Période, époque; segment de temps.} 
\lhead{\firstmark}
\rhead{\botmark}

\subsection{\hspace{-0.5cm} {\Large \textcolor{darkblue}{\textbf{\ipa{ʈʂæ˧˥\textsubscript{a}}}}}\hspace{0.5cm}[\kern2pt{\textcolor{darkblue}{\textbf{\ipa{ʈʂæ˥}}}}\kern2pt]} \hypertarget{t`s`\{\string_M\string_Ta1}{}
\markboth{\textcolor{darkblue}{\textbf{\ipa{ʈʂæ˧˥\textsubscript{a}}}}}{}
\textcolor{teal}{\zh{量词}} \hspace{4pt} \zh{声调类:} MH\textsubscript{a}.
\zh{量词.玉米(一棒)。} \textcolor{Sepia}{\selectlanguage{english}Classifier for ears (of sweet corn).} \textcolor{PineGreen}{\selectlanguage{french}Classificateur des épis de maïs (mûrs).}  ¶ \textcolor{darkblue}{\textbf{\ipa{qʰɑ˧dze˧ | ɖɯ˧-ʈʂæ˧˥}}} \zh{一棒玉米} \textcolor{Sepia}{\selectlanguage{english}an ear of sweet corn} \textcolor{PineGreen}{\selectlanguage{french}un épi de maïs}  
 ¶ \textcolor{darkblue}{\textbf{\ipa{qʰɑ˧dze˧ | ɖɯ˧-ʈʂæ˧ ɖʐɤ˥}}} \zh{掰一棒玉米} \textcolor{Sepia}{\selectlanguage{english}to pick an ear of sweet corn} \textcolor{PineGreen}{\selectlanguage{french}arracher un épi de maïs, récolter un épi de maïs}  

\lhead{\firstmark}
\rhead{\botmark}

\subsection{\hspace{-0.5cm} {\Large \textcolor{darkblue}{\textbf{\ipa{ʈʂe˥}}} \textsubscript{1}}\hspace{0.5cm}[\kern2pt{\textcolor{darkblue}{\textbf{\ipa{ʈʂe˥}}}}\kern2pt]} \hypertarget{t`s`e\string_T1}{}
\markboth{\textcolor{darkblue}{\textbf{\ipa{ʈʂe˥}}} \textsubscript{1}}{}
\textcolor{teal}{\zh{名词}} \hspace{4pt} \zh{声调类:} \#H.
\zh{土壤。} \textcolor{Sepia}{\selectlanguage{english}Earth.} \textcolor{PineGreen}{\selectlanguage{french}Terre.}  ¶ \textcolor{darkblue}{\textbf{\ipa{ʈʂe˧pv̩˩}}} \zh{干土} \textcolor{Sepia}{\selectlanguage{english}dry earth} \textcolor{PineGreen}{\selectlanguage{french}terre sèche}  
 ¶ \textcolor{darkblue}{\textbf{\ipa{ʈʂe˧ sɯ˧\textasciitilde{}sɯ˥}}} \zh{‘生土’:没有经过加工(加肥料等等)的土,还不适合种农作物} \textcolor{Sepia}{\selectlanguage{english}'raw earth': immature soil, earth that has not been prepared for agriculture by adding manure, etc} \textcolor{PineGreen}{\selectlanguage{french}'terre crue': terre qui n'a pas été préparée pour l'agriculture par l'ajout de fumier, etc}  

\lhead{\firstmark}
\rhead{\botmark}

\subsection{\hspace{-0.5cm} {\Large \textcolor{darkblue}{\textbf{\ipa{ʈʂe˥}}} \textsubscript{2}}\hspace{0.5cm}[\kern2pt{\textcolor{darkblue}{\textbf{\ipa{ʈʂe˥}}}}\kern2pt]} \hypertarget{t`s`e\string_T2}{}
\markboth{\textcolor{darkblue}{\textbf{\ipa{ʈʂe˥}}} \textsubscript{2}}{}
\textcolor{teal}{\zh{名词}} \hspace{4pt} \zh{声调类:} \#H.
\zh{针(汉语借词)。} \textcolor{Sepia}{\selectlanguage{english}Needle.} \textcolor{PineGreen}{\selectlanguage{french}Aiguille.}  \zh{量词}: \textcolor{darkblue}{\textbf{\ipa{ɭɯ˧}}} \zh{~【参考】~} \hyperlink{}{\textcolor{darkblue}{\textbf{\ipa{ʁo˧˥}}}} 
\lhead{\firstmark}
\rhead{\botmark}

\subsection{\hspace{-0.5cm} {\Large \textcolor{darkblue}{\textbf{\ipa{ʈʂe˧dɑ˥}}}}\hspace{0.5cm}[\kern2pt{\textcolor{darkblue}{\textbf{\ipa{ʈʂe˧dɑ˥}}}}\kern2pt]} \hypertarget{t`s`e\string_MdA\string_T1}{}
\markboth{\textcolor{darkblue}{\textbf{\ipa{ʈʂe˧dɑ˥}}}}{}
\textcolor{teal}{\zh{名词}} \hspace{4pt} \zh{声调类:} H\#.
\zh{隔板。} \textcolor{Sepia}{\selectlanguage{english}Partition.} \textcolor{PineGreen}{\selectlanguage{french}Cloison.}  \zh{量词}: \textcolor{darkblue}{\textbf{\ipa{do˥}}} 
\lhead{\firstmark}
\rhead{\botmark}

\subsection{\hspace{-0.5cm} {\Large \textcolor{darkblue}{\textbf{\ipa{ʈʂe˧gi˥\$}}}}\hspace{0.5cm}[\kern2pt{\textcolor{darkblue}{\textbf{\ipa{ʈʂe˧gi˥}}}}\kern2pt]} \hypertarget{t`s`e\string_Mgi\string_T\$1}{}
\markboth{\textcolor{darkblue}{\textbf{\ipa{ʈʂe˧gi˥\$}}}}{}
\textcolor{teal}{\zh{助词}} \hspace{4pt} \zh{声调类:} H\$.
\zh{中间、之间、间。} \textcolor{Sepia}{\selectlanguage{english}In-between, in the middle of.} \textcolor{PineGreen}{\selectlanguage{french}Entre, au milieu de.}  ¶ \textcolor{darkblue}{\textbf{\ipa{ə˧-sɯ˩kv̩˩-ʈʂe˩gi˩}}} \zh{在咱们之间(的空间)} \textcolor{Sepia}{\selectlanguage{english}(in the space) between us, in the space that separates us} \textcolor{PineGreen}{\selectlanguage{french}entre nous, dans l'espace qui nous sépare}  

\lhead{\firstmark}
\rhead{\botmark}

\subsection{\hspace{-0.5cm} {\Large \textcolor{darkblue}{\textbf{\ipa{ʈʂe˩\textsubscript{a}}}}}\hspace{0.5cm}[\kern2pt{\textcolor{darkblue}{\textbf{\ipa{ʈʂe˩˥}}}}\kern2pt]} \hypertarget{t`s`e\string_Ba1}{}
\markboth{\textcolor{darkblue}{\textbf{\ipa{ʈʂe˩\textsubscript{a}}}}}{}
\textcolor{teal}{\zh{动词}} \hspace{4pt} \zh{声调类:} L\textsubscript{a}.
\zh{刺痛。} \textcolor{Sepia}{\selectlanguage{english}To sting, to pierce (e.g. a thorn).} \textcolor{PineGreen}{\selectlanguage{french}Percer, transpercer (une écharde, un piquant de plante...).}  ¶ \textcolor{darkblue}{\textbf{\ipa{le˧-ʈʂe˩-ze˩}}} \zh{\mytextsc{accomp} \string_ \mytextsc{pfv}} \textcolor{Sepia}{\selectlanguage{english}\mytextsc{accomp} \string_ \mytextsc{pfv}} \textcolor{PineGreen}{\selectlanguage{french}\mytextsc{accomp} \string_ \mytextsc{pfv}}  
 ¶ \textcolor{darkblue}{\textbf{\ipa{tɕʰi˧-ɳɯ˧ ʈʂe˩-ze˩}}} \zh{被刺所刺痛} \textcolor{Sepia}{\selectlanguage{english}to be pierced by a thorn, to catch a thorn} \textcolor{PineGreen}{\selectlanguage{french}être piqué par une épine, se prendre une épine}  
 ¶ \textcolor{darkblue}{\textbf{\ipa{tso˧\textasciitilde{}tso˧ ʈʂe˥}}} \zh{刺到一个东西} \textcolor{Sepia}{\selectlanguage{english}to pierce something} \textcolor{PineGreen}{\selectlanguage{french}percer quelque chose}  

\lhead{\firstmark}
\rhead{\botmark}

\subsection{\hspace{-0.5cm} {\Large \textcolor{darkblue}{\textbf{\ipa{ʈʂe˩kʰɯ˩}}}}\hspace{0.5cm}[\kern2pt{\textcolor{darkblue}{\textbf{\ipa{ʈʂe˩kʰɯ˩˥}}}}\kern2pt]} \hypertarget{t`s`e\string_Bk\string_hM\string_B1}{}
\markboth{\textcolor{darkblue}{\textbf{\ipa{ʈʂe˩kʰɯ˩}}}}{}
\textcolor{teal}{\zh{名词}} \hspace{4pt} \zh{声调类:} L.
\zh{缝。} \textcolor{Sepia}{\selectlanguage{english}Seam.} \textcolor{PineGreen}{\selectlanguage{french}Couture (d'un vêtement).}  \zh{量词}: \textcolor{darkblue}{\textbf{\ipa{ɭɯ˧}}} 
\lhead{\firstmark}
\rhead{\botmark}

\subsection{\hspace{-0.5cm} {\Large \textcolor{darkblue}{\textbf{\ipa{ʈʂe˩ʂwæ˧˥}}}}\hspace{0.5cm}[\kern2pt{\textcolor{darkblue}{\textbf{\ipa{ʈʂe˩ʂwæ˧˥}}}}\kern2pt]} \hypertarget{t`s`e\string_Bs`w\{\string_M\string_T1}{}
\markboth{\textcolor{darkblue}{\textbf{\ipa{ʈʂe˩ʂwæ˧˥}}}}{}
\textcolor{teal}{\zh{名词}} \hspace{4pt} \zh{声调类:} LM+MH\#.
\zh{砾石。} \textcolor{Sepia}{\selectlanguage{english}Grit, gravel.} \textcolor{PineGreen}{\selectlanguage{french}Gravier, sable grossier.} 
\lhead{\firstmark}
\rhead{\botmark}

\subsection{\hspace{-0.5cm} {\Large \textcolor{darkblue}{\textbf{\ipa{ʈʂɤ˧\textsubscript{a}}}}}\hspace{0.5cm}[\kern2pt{\textcolor{darkblue}{\textbf{\ipa{ʈʂɤ˥}}}}\kern2pt]} \hypertarget{t`s`7\string_Ma1}{}
\markboth{\textcolor{darkblue}{\textbf{\ipa{ʈʂɤ˧\textsubscript{a}}}}}{}
\textcolor{teal}{\zh{动词}} \hspace{4pt} \zh{声调类:} M\textsubscript{a}.
\ding{202} \zh{数、算。} \textcolor{Sepia}{\selectlanguage{english}To count; to calculate.} \textcolor{PineGreen}{\selectlanguage{french}Compter; calculer.}  ¶ \textcolor{darkblue}{\textbf{\ipa{ʈʂɤ˧\textasciitilde{}ʈʂɤ˩}}} \zh{\mytextsc{重叠:算一算}} \textcolor{Sepia}{\selectlanguage{english}\mytextsc{red}} \textcolor{PineGreen}{\selectlanguage{french}\mytextsc{red}}  
 ¶ \textcolor{darkblue}{\textbf{\ipa{ɖɯ˧-ʈʂɤ˥\textasciitilde{}ʈʂɤ˩-ɻ̍˩}}} \zh{算一下} \textcolor{Sepia}{\selectlanguage{english}to do some counting, to take a count} \textcolor{PineGreen}{\selectlanguage{french}faire quelques calculs}  
 ¶ \textcolor{darkblue}{\textbf{\ipa{tso˧\textasciitilde{}tso˧ ʈʂɤ˩}}} \zh{数东西} \textcolor{Sepia}{\selectlanguage{english}to count things} \textcolor{PineGreen}{\selectlanguage{french}compter des choses}  
 ¶ \textcolor{darkblue}{\textbf{\ipa{hĩ˧ ʈʂɤ˩}}} \zh{数人} \textcolor{Sepia}{\selectlanguage{english}to count people} \textcolor{PineGreen}{\selectlanguage{french}compter les gens}  
 ¶ \textcolor{darkblue}{\textbf{\ipa{bo˩ ʈʂɤ˧}}} \zh{数猪} \textcolor{Sepia}{\selectlanguage{english}to count pigs} \textcolor{PineGreen}{\selectlanguage{french}compter les porcs}  
 ¶ \textcolor{darkblue}{\textbf{\ipa{le˧-ʈʂɤ˧-ze˧}}} \zh{数了} \textcolor{Sepia}{\selectlanguage{english}\mytextsc{accomp} \string_ \mytextsc{pfv}} \textcolor{PineGreen}{\selectlanguage{french}\mytextsc{accomp} \string_ \mytextsc{pfv}}  
\ding{203} \zh{算命。} \textcolor{Sepia}{\selectlanguage{english}To do divination, to do fortune-telling.} \textcolor{PineGreen}{\selectlanguage{french}Dire la bonne aventure, pratiquer la divination.}  ¶ \textcolor{darkblue}{\textbf{\ipa{le˧-ʈʂɤ˥\textasciitilde{}ʈʂɤ˩}}} \zh{算命} \textcolor{Sepia}{\selectlanguage{english}to do divination, to do fortune-telling} \textcolor{PineGreen}{\selectlanguage{french}dire la bonne aventure, pratiquer la divination}  
 ¶ \textcolor{darkblue}{\textbf{\ipa{ɖɯ˧-ʈʂɤ˥\textasciitilde{}ʈʂɤ˩-ɻ̍˩}}} \zh{算一下命} \textcolor{Sepia}{\selectlanguage{english}to do some fortune-telling} \textcolor{PineGreen}{\selectlanguage{french}dire la bonne aventure}  
 ¶ \textcolor{darkblue}{\textbf{\ipa{ɲi˧ŋwɤ˩hɑ̃˩tʰɑ˩ | ɖɯ˧-ɭɯ˧ | ʈʂɤ˧-bi˧!}}} \zh{要掐算一下日子} \textcolor{Sepia}{\selectlanguage{english}(We are) going to look for an auspicious day (for an event such as a wedding or the building of a house)} \textcolor{PineGreen}{\selectlanguage{french}On va chercher un jour propice! (pour un événement tel qu'un mariage ou la construction d'une maison)}  
 ¶ \textcolor{darkblue}{\textbf{\ipa{kɯ˧ ʈʂɤ˧, | hɑ̃˧ ʈʂɤ˧}}} \zh{掐算一下。直译:“算星星,算日子”。} \textcolor{Sepia}{\selectlanguage{english}to look for an auspicious day for an important event; literally: “to count stars, to count days”} \textcolor{PineGreen}{\selectlanguage{french}chercher un jour propice pour un événement, tel que le début de la construction d'une maison; littéralement: “compter les étoiles, compter les jours”}  
\ding{204} \zh{算是,当作。} \textcolor{Sepia}{\selectlanguage{english}To count as.} \textcolor{PineGreen}{\selectlanguage{french}Compter comme, être, avoir fonction de, avoir rôle de.}  ¶ \textcolor{darkblue}{\textbf{\ipa{hĩ˧ ɖɯ˧-v̩˧ ʈʂɤ˧-ze˧!}}} \zh{变成大人了!(十三岁成年礼时常用的一句话)} \textcolor{Sepia}{\selectlanguage{english}(She/he) now counts as a (grown-up) person! / (She/he) can now be considered an adult! (A comment typically made when a child reaches adulthood, at age 13.)} \textcolor{PineGreen}{\selectlanguage{french}(Elle/il) compte maintenant comme une grande personne! / C'est un(e) adulte, maintenant! (Ce qu'on dit d'un enfant qui atteint l'âge adulte: 13 ans.)}  
 ¶ \textcolor{darkblue}{\textbf{\ipa{dʑɤ˩ ʈʂɤ˧}}} \zh{算是很好的} \textcolor{Sepia}{\selectlanguage{english}to be good, to count as good, to be considered as good} \textcolor{PineGreen}{\selectlanguage{french}être très bien}  
 ¶ \textcolor{darkblue}{\textbf{\ipa{ʈʂʰɯ˧ | õ˧-bv̩˥-õ˩ | dʑɤ˩ʈʂɤ˧ (+ | ʐwæ˩˥)}}} \zh{他觉得自己很了不起!} \textcolor{Sepia}{\selectlanguage{english}He thinks highly of himself! / He is proud of himself / conceited!} \textcolor{PineGreen}{\selectlanguage{french}Il a une haute idée de lui-même! / Il est orgueilleux!}  
 ¶ \textcolor{darkblue}{\textbf{\ipa{hɤ˩ ʈʂɤ˩˥}}} \zh{算很了不起的,算很能干的} \textcolor{Sepia}{\selectlanguage{english}to count as remarkable, to be considered as remarkable} \textcolor{PineGreen}{\selectlanguage{french}être habile / admirable / remarquable, être considéré comme habile, compter comme (quelqu'un d')habile}  
 ¶ \textcolor{darkblue}{\textbf{\ipa{ɖwæ˧˥ | hɤ˩ ʈʂɤ˩˥}}} \zh{同上} \textcolor{Sepia}{\selectlanguage{english}same meaning} \textcolor{PineGreen}{\selectlanguage{french}même sens}  
 ¶ \textcolor{darkblue}{\textbf{\ipa{ʈʂʰɯ˧ | gi˧zɯ˧ ʈʂɤ˧-tso˧-ɲi˥.}}} \zh{他是做弟弟的!(强调该人的社会角色)} \textcolor{Sepia}{\selectlanguage{english}He has the status of little brother! (Comment made to emphasize someone's role in the family.)} \textcolor{PineGreen}{\selectlanguage{french}C'est le petit frère / il a le statut de petit frère! (Commentaire qui rappelle le statut familial de la personne concernée.)}  
 ¶ \textcolor{darkblue}{\textbf{\ipa{ʈʂʰɯ˧ | gi˧zɯ˧ ʈʂɤ˧-ɲi˥!}}} \zh{他是做弟弟的!(强调该人的社会角色)} \textcolor{Sepia}{\selectlanguage{english}He has the status of little brother! (Comment made to emphasize someone's role in the family.)} \textcolor{PineGreen}{\selectlanguage{french}C'est le petit frère / il a le statut de petit frère! (Commentaire qui rappelle le statut familial de la personne concernée.)}  
 ¶ \textcolor{darkblue}{\textbf{\ipa{ʈʂʰɯ˧ | bɤ˧ ʈʂɤ˧-tso˧-ɲi˥!}}} \zh{他是普米族!(强调该人的民族)} \textcolor{Sepia}{\selectlanguage{english}He/she is Pumi! (Comment made to emphasize this aspect of someone's identity.)} \textcolor{PineGreen}{\selectlanguage{french}Il/elle est pumi! (Commentaire qui rappelle un élément de l'identité de la personne concernée.)}  
 ¶ \textcolor{darkblue}{\textbf{\ipa{ʈʂʰɯ˧ | nɑ˩ ʈʂɤ˧-tso˧-ɲi˥!}}} \zh{他是摩梭人!(强调该人的民族身份)} \textcolor{Sepia}{\selectlanguage{english}He/she is Na! (Comment made to emphasize this aspect of someone's identity.)} \textcolor{PineGreen}{\selectlanguage{french}Il/elle est na! (Commentaire qui rappelle un élément de l'identité de la personne concernée)}  
 ¶ \textcolor{darkblue}{\textbf{\ipa{ʈʂʰɯ˧ | æ˧mv̩˩ ʈʂɤ˩-ɲi˩!}}} \zh{她是做姐姐的! / 他是做哥哥的!(强调该人的社会角色)} \textcolor{Sepia}{\selectlanguage{english}She has the status of elder sister! / He has the status of elder brother! (Comment made to emphasize someone's role in the family.)} \textcolor{PineGreen}{\selectlanguage{french}C'est elle la grande soeur! / C'est lui le grand frère! (Commentaire qui rappelle le statut familial de la personne concernée.)}  
 ¶ \textcolor{darkblue}{\textbf{\ipa{ʈʂʰɯ˧ | gi˧zɯ˧-go˩mi˩ ʈʂɤ˩-ɲi˩!}}} \zh{他们是(兄)弟(姐)妹!} \textcolor{Sepia}{\selectlanguage{english}They are brothers and sisters!} \textcolor{PineGreen}{\selectlanguage{french}Ils sont frère et sœur!}  
 ¶ \textcolor{darkblue}{\textbf{\ipa{ʈʂʰɯ˧ | æ˧mv̩˧-go˧mi˥ | ʈʂɤ˧-tso˧ mɤ˧-ɲi˥! | mɤ˧-ʈʂɤ˧!}}} \zh{他们不算是兄弟姐妹!} \textcolor{Sepia}{\selectlanguage{english}They can't be considered as brothers and sisters! / They are not actually brothers and sisters!} \textcolor{PineGreen}{\selectlanguage{french}Ils ne sont pas frères et sœurs! / Ils n'ont pas cette relation-là!}  
 ¶ \textcolor{darkblue}{\textbf{\ipa{ʐwæ˩ ʈʂɤ˩}}} \zh{了不起} \textcolor{Sepia}{\selectlanguage{english}remarkable, great, exceptional, outstanding} \textcolor{PineGreen}{\selectlanguage{french}remarquable, extraordinaire, exceptionnel}  
 ¶ \textcolor{darkblue}{\textbf{\ipa{ʈʂʰɯ˧ | ə˧tso˧ ʐwæ˩ ʈʂɤ˩-tso˩ dʑo˩?}}} \zh{他有什么了不起的?} \textcolor{Sepia}{\selectlanguage{english}What's so remarkable about her/him?} \textcolor{PineGreen}{\selectlanguage{french}Qu'est-ce qu'elle/il a de si remarquable? / Quelles qualités exceptionnelles a-t-il/elle (que je doive prendre son conseil/suivre son avis)?}  
 ¶ \textcolor{darkblue}{\textbf{\ipa{ʈʂʰɯ˧ | ʐwæ˩ ʈʂɤ˩˥ | ʐwæ˩˥!}}} \zh{他非常了不起!} \textcolor{Sepia}{\selectlanguage{english}It's really an outstanding person!} \textcolor{PineGreen}{\selectlanguage{french}C'est quelqu'un de vraiment remarquable/extraordinaire!}  

\lhead{\firstmark}
\rhead{\botmark}

\subsection{\hspace{-0.5cm} {\Large \textcolor{darkblue}{\textbf{\ipa{ʈʂo˧kʰɯ˩}}}}\hspace{0.5cm}[\kern2pt{\textcolor{darkblue}{\textbf{\ipa{ʈʂo˧kʰɯ˩}}}}\kern2pt]} \hypertarget{t`s`o\string_Mk\string_hM\string_B1}{}
\markboth{\textcolor{darkblue}{\textbf{\ipa{ʈʂo˧kʰɯ˩}}}}{}
\textcolor{teal}{\zh{名词}} \hspace{4pt} \zh{声调类:} L\#.
\zh{忠克:亲人去世时举行的仪式。} \textcolor{Sepia}{\selectlanguage{english}Ritual performed on the occasion of the death of a family member.} \textcolor{PineGreen}{\selectlanguage{french}Rituel pour la mort d'une personne de sa famille.} 
\lhead{\firstmark}
\rhead{\botmark}

\subsection{\hspace{-0.5cm} {\Large \textcolor{darkblue}{\textbf{\ipa{ʈʂo˧ɭɯ\#˥}}}}\hspace{0.5cm}[\kern2pt{\textcolor{darkblue}{\textbf{\ipa{ʈʂo˧ɭɯ˧}}}}\kern2pt]} \hypertarget{t`s`o\string_Ml\string_RM\#\string_T1}{}
\markboth{\textcolor{darkblue}{\textbf{\ipa{ʈʂo˧ɭɯ\#˥}}}}{}
\textcolor{teal}{\zh{名词}} \hspace{4pt} \zh{声调类:} \#H.
\zh{手推磨。} \textcolor{Sepia}{\selectlanguage{english}Hand-operated grindstone.} \textcolor{PineGreen}{\selectlanguage{french}Moulin à main.}  ¶ \textcolor{darkblue}{\textbf{\ipa{ʈʂo˧ɭɯ˧-nv̩˥mi˩}}} \zh{手推磨的轴心} \textcolor{Sepia}{\selectlanguage{english}the axis of the grindstone (literally: its hear)} \textcolor{PineGreen}{\selectlanguage{french}l'axe du moulin (littéralement: son cœur)}  
 \zh{量词}: \textcolor{darkblue}{\textbf{\ipa{nɑ˧}}} 
\lhead{\firstmark}
\rhead{\botmark}

\subsection{\hspace{-0.5cm} {\Large \textcolor{darkblue}{\textbf{\ipa{ʈʂo˧ɭɯ˧ʈʂo˧˥}}}}\hspace{0.5cm}[\kern2pt{\textcolor{darkblue}{\textbf{\ipa{ʈʂo˧ɭɯ˧ʈʂo˧˥}}}}\kern2pt]} \hypertarget{t`s`o\string_Ml\string_RM\string_Mt`s`o\string_M\string_T1}{}
\markboth{\textcolor{darkblue}{\textbf{\ipa{ʈʂo˧ɭɯ˧ʈʂo˧˥}}}}{}
\textcolor{teal}{\zh{名词}} \hspace{4pt} \zh{声调类:} MH\#.
\zh{一种水虫。} \textcolor{Sepia}{\selectlanguage{english}A water insect.} \textcolor{PineGreen}{\selectlanguage{french}Insecte aquatique.}  \zh{量词}: \textcolor{darkblue}{\textbf{\ipa{mi˩}}} 
\lhead{\firstmark}
\rhead{\botmark}

\subsection{\hspace{-0.5cm} {\Large \textcolor{darkblue}{\textbf{\ipa{ʈʂo˧ʂɯ\#˥}}}}\hspace{0.5cm}[\kern2pt{\textcolor{darkblue}{\textbf{\ipa{ʈʂo˧ʂɯ˧}}}}\kern2pt]} \hypertarget{t`s`o\string_Ms`M\#\string_T1}{}
\markboth{\textcolor{darkblue}{\textbf{\ipa{ʈʂo˧ʂɯ\#˥}}}}{}
\textcolor{teal}{\zh{名词}} \hspace{4pt} \zh{声调类:} \#H.
\zh{忠实(永宁的一个村落)。} \textcolor{Sepia}{\selectlanguage{english}Name of a village: Zhongshi.} \textcolor{PineGreen}{\selectlanguage{french}Village de Zhongshi.}  ¶ \textcolor{darkblue}{\textbf{\ipa{ɖæ˩ʂɯ\#˥, | ʈʂo˧ʂɯ\#˥, | bɤ˩tɕʰɯ˩˥, | dɑ˧pʰo˥, | bɤ˧dzi˩, | dze˧bo˧}}} \zh{永宁坝的六个村落,按传统排序:从距离泸沽湖最近的村落说起。} \textcolor{Sepia}{\selectlanguage{english}the six villages of the plain of Yongning, in traditional order: by order of increasing distance from the Lake} \textcolor{PineGreen}{\selectlanguage{french}les six villages de la plaine de Yongning, dans l'ordre, qui prend comme point d'origine le village le plus proche du Lac}  

\lhead{\firstmark}
\rhead{\botmark}

\subsection{\hspace{-0.5cm} {\Large \textcolor{darkblue}{\textbf{\ipa{ʈʂo˧tsɯ˥}}}}\hspace{0.5cm}[\kern2pt{\textcolor{darkblue}{\textbf{\ipa{ʈʂo˧tsɯ˥}}}}\kern2pt]} \hypertarget{t`s`o\string_MtsM\string_T1}{}
\markboth{\textcolor{darkblue}{\textbf{\ipa{ʈʂo˧tsɯ˥}}}}{}
\textcolor{teal}{\zh{名词}} \hspace{4pt} \zh{声调类:} H\#.
\zh{桌子(汉语借词)。} \textcolor{Sepia}{\selectlanguage{english}Table.} \textcolor{PineGreen}{\selectlanguage{french}Table.}  \zh{【借词】} \zh{桌子}
 \zh{量词}: \textcolor{darkblue}{\textbf{\ipa{pɤ˩}}} \zh{~【参考】~} \hyperlink{}{\textcolor{darkblue}{\textbf{\ipa{sɯ˧ɻæ˧}}}} 
\lhead{\firstmark}
\rhead{\botmark}

\subsection{\hspace{-0.5cm} {\Large \textcolor{darkblue}{\textbf{\ipa{ʈʂo˩}}}}\hspace{0.5cm}[\kern2pt{\textcolor{darkblue}{\textbf{\ipa{ʈʂo˩˥}}}}\kern2pt]} \hypertarget{t`s`o\string_B1}{}
\markboth{\textcolor{darkblue}{\textbf{\ipa{ʈʂo˩}}}}{}
\textcolor{teal}{\zh{量词}} \hspace{4pt} \zh{声调类:} L.
\zh{量词:饭(一顿)。} \textcolor{Sepia}{\selectlanguage{english}Classifier for meals.} \textcolor{PineGreen}{\selectlanguage{french}Classificateur des repas.}  ¶ \textcolor{darkblue}{\textbf{\ipa{ɖɯ˧-ʈʂo˩ tʰi˩-pæ˩ |}}} \zh{摆饭,摆饭桌} \textcolor{Sepia}{\selectlanguage{english}to serve a meal, to set a meal} \textcolor{PineGreen}{\selectlanguage{french}servir un repas}  
 ¶ \textcolor{darkblue}{\textbf{\ipa{ʐo˩˥, | njɤ˧ ɖɯ˧-ʈʂo˩ pæ˩-bi˩!}}} \zh{我来管午饭这一顿!} \textcolor{Sepia}{\selectlanguage{english}For lunch, I will serve a (whole) meal! / I'm taking charge of lunch!} \textcolor{PineGreen}{\selectlanguage{french}au déjeuner, je vais (vous) servir (tout le) repas!}  
 ¶ \textcolor{darkblue}{\textbf{\ipa{hɑ˧ ɖɯ˧-ʈʂo˩}}} \zh{一顿饭} \textcolor{Sepia}{\selectlanguage{english}a meal} \textcolor{PineGreen}{\selectlanguage{french}un repas}  

\lhead{\firstmark}
\rhead{\botmark}

\subsection{\hspace{-0.5cm} {\Large \textcolor{darkblue}{\textbf{\ipa{ʈʂo˩bo˩}}}}\hspace{0.5cm}[\kern2pt{\textcolor{darkblue}{\textbf{\ipa{ʈʂo˧bo˧}}}}\kern2pt]} \hypertarget{t`s`o\string_Bbo\string_B1}{}
\markboth{\textcolor{darkblue}{\textbf{\ipa{ʈʂo˩bo˩}}}}{}
\textcolor{teal}{\zh{名词}} \hspace{4pt} \zh{声调类:} L.
\zh{土墙。} \textcolor{Sepia}{\selectlanguage{english}Earthen wall.} \textcolor{PineGreen}{\selectlanguage{french}Mur en terre.}  ¶ \textcolor{darkblue}{\textbf{\ipa{ʈʂo˩bo˩ ti˥}}} \zh{垒墙} \textcolor{Sepia}{\selectlanguage{english}to build a wall of earth, by pounding the earth} \textcolor{PineGreen}{\selectlanguage{french}construire un mur en terre (en comprimant la terre)}  
 \zh{量词}: \textcolor{darkblue}{\textbf{\ipa{do˥}}} 
\lhead{\firstmark}
\rhead{\botmark}

\subsection{\hspace{-0.5cm} {\Large \textcolor{darkblue}{\textbf{\ipa{ʈʂo˩mv̩˩}}}}\hspace{0.5cm}[\kern2pt{\textcolor{darkblue}{\textbf{\ipa{ʈʂo˩mv̩˩˥}}}}\kern2pt]} \hypertarget{t`s`o\string_Bmv\string_=\string_B1}{}
\markboth{\textcolor{darkblue}{\textbf{\ipa{ʈʂo˩mv̩˩}}}}{}
\textcolor{teal}{\zh{名词}} \hspace{4pt} \zh{声调类:} L.
\zh{沙子。} \textcolor{Sepia}{\selectlanguage{english}Fine sand.} \textcolor{PineGreen}{\selectlanguage{french}Sable fin.} 
\lhead{\firstmark}
\rhead{\botmark}

\subsection{\hspace{-0.5cm} {\Large \textcolor{darkblue}{\textbf{\ipa{ʈʂɻ̍˥}}}}\hspace{0.5cm}[\kern2pt{\textcolor{darkblue}{\textbf{\ipa{ʈʂɻ̍˥}}}}\kern2pt]} \hypertarget{t`s`r£`̍\string_T1}{}
\markboth{\textcolor{darkblue}{\textbf{\ipa{ʈʂɻ̍˥}}}}{}
\textcolor{teal}{\zh{动词}} \hspace{4pt} \zh{声调类:} H.
\zh{咳嗽。} \textcolor{Sepia}{\selectlanguage{english}To cough.} \textcolor{PineGreen}{\selectlanguage{french}Tousser.}  ¶ \textcolor{darkblue}{\textbf{\ipa{ʈʂʰɯ˧ | tʰi˧-ʈʂɻ̍˥-dʑo˩}}} \zh{他在咳嗽} \textcolor{Sepia}{\selectlanguage{english}(S)he is coughing.} \textcolor{PineGreen}{\selectlanguage{french}il tousse}  

\lhead{\firstmark}
\rhead{\botmark}

\subsection{\hspace{-0.5cm} {\Large \textcolor{darkblue}{\textbf{\ipa{ʈʂɯ˧}}}}\hspace{0.5cm}[\kern2pt{\textcolor{darkblue}{\textbf{\ipa{ʈʂɯ˧˥}}}}\kern2pt]} \hypertarget{t`s`M\string_M1}{}
\markboth{\textcolor{darkblue}{\textbf{\ipa{ʈʂɯ˧}}}}{}
\textcolor{teal}{\zh{名词}} \hspace{4pt} \zh{声调类:} M.
\zh{爪子。} \textcolor{Sepia}{\selectlanguage{english}Claws.} \textcolor{PineGreen}{\selectlanguage{french}Griffes (d'un animal); serres (d'un oiseau).}  \zh{量词}: \textcolor{darkblue}{\textbf{\ipa{ɭɯ˧}}} \zh{~【参考】~} \hyperlink{}{\textcolor{darkblue}{\textbf{\ipa{kv̩˧ʈʂɯ˧˥}}}} 
\lhead{\firstmark}
\rhead{\botmark}

\subsection{\hspace{-0.5cm} {\Large \textcolor{darkblue}{\textbf{\ipa{ʈʂɯ˧dzi˩}}}}\hspace{0.5cm}[\kern2pt{\textcolor{darkblue}{\textbf{\ipa{ʈʂɯ˩dzi˩˥}}}}\kern2pt]} \hypertarget{t`s`M\string_Mdzi\string_B1}{}
\markboth{\textcolor{darkblue}{\textbf{\ipa{ʈʂɯ˧dzi˩}}}}{}
\textcolor{teal}{\zh{名词}} \hspace{4pt} \zh{声调类:} L\#.
\zh{漆树。} \textcolor{Sepia}{\selectlanguage{english}Lacquer tree, varnish tree.} \textcolor{PineGreen}{\selectlanguage{french}Arbre à vernis.} 
\lhead{\firstmark}
\rhead{\botmark}

\subsection{\hspace{-0.5cm} {\Large \textcolor{darkblue}{\textbf{\ipa{ʈʂɯ˧fv̩\#˥}}}}\hspace{0.5cm}[\kern2pt{\textcolor{darkblue}{\textbf{\ipa{ʈʂɯ˧fv̩˩}}}}\kern2pt]} \hypertarget{t`s`M\string_Mfv\string_=\#\string_T1}{}
\markboth{\textcolor{darkblue}{\textbf{\ipa{ʈʂɯ˧fv̩\#˥}}}}{}
\textcolor{teal}{\zh{名词}} \hspace{4pt} \zh{声调类:} \#H.
\zh{(土)知府,如:永宁知府(汉语借词)。} \textcolor{Sepia}{\selectlanguage{english}Local government.} \textcolor{PineGreen}{\selectlanguage{french}Gouvernement local, pouvoir local.}  \zh{【借词】} \zh{知府}
 ¶ \textcolor{darkblue}{\textbf{\ipa{no˧ | ɬi˧di˩-ʈʂɯ˩fv̩˩-ni˩-zo˩!}}} \zh{你像永宁土知府! / 你是永宁土知府吧!(批评独断的人、一手包办的人)} \textcolor{Sepia}{\selectlanguage{english}You think you're the government, hey?! (A criticism to people who keep telling others how they should behave, as if they lorded it over everyone else.)} \textcolor{PineGreen}{\selectlanguage{french}Tu te prends pour le seigneur! (critique qu'on s'adressait aux gens qui se mêlaient de dicter leur conduite aux autres, comme s'ils étaient les maîtres des lieux)}  
 ¶ \textcolor{darkblue}{\textbf{\ipa{no˧ | ʈʂɯ˧fv̩˧-mi˧-ni˧˥ | -zo˩!}}} \zh{你好像是永宁大公主! / 你好像是永宁知府女主人!(批评一个独断的女人)} \textcolor{Sepia}{\selectlanguage{english}You are the Princess of Yongning, hey?! (Criticism addressed to an overbearing woman)} \textcolor{PineGreen}{\selectlanguage{french}Tu joues les princesses! (littéralement “les femmes du pouvoir”) Critique adressée à une femme qui prend de grands airs.}  

\lhead{\firstmark}
\rhead{\botmark}

\subsection{\hspace{-0.5cm} {\Large \textcolor{darkblue}{\textbf{\ipa{ʈʂɯ˧mɤ˩}}}}\hspace{0.5cm}[\kern2pt{\textcolor{darkblue}{\textbf{\ipa{ʈʂɯ˧mɤ˥}}}}\kern2pt]} \hypertarget{t`s`M\string_Mm7\string_B1}{}
\markboth{\textcolor{darkblue}{\textbf{\ipa{ʈʂɯ˧mɤ˩}}}}{}
\textcolor{teal}{\zh{名词}} \hspace{4pt} \zh{声调类:} L\#.
\zh{芝麻(汉语借词)。} \textcolor{Sepia}{\selectlanguage{english}Sesame.} \textcolor{PineGreen}{\selectlanguage{french}Sésame.}  \zh{【借词】} \zh{芝麻}
 ¶ \textcolor{darkblue}{\textbf{\ipa{ʈʂɯ˧mɤ˩, | ɬi˧di˩ | mɤ˧-tʰv̩˧-ɲi˥!}}} \zh{永宁不产芝麻!} \textcolor{Sepia}{\selectlanguage{english}Sesame does not grow in Yongning! / Sesame isn't grown in Yongning!} \textcolor{PineGreen}{\selectlanguage{french}Le sésame ne pousse pas à Yongning!}  

\lhead{\firstmark}
\rhead{\botmark}

\subsection{\hspace{-0.5cm} {\Large \textcolor{darkblue}{\textbf{\ipa{ʈʂɯ˧˥}}}}\hspace{0.5cm}[\kern2pt{\textcolor{darkblue}{\textbf{\ipa{ʈʂɯ˥}}}}\kern2pt]} \hypertarget{t`s`M\string_M\string_T1}{}
\markboth{\textcolor{darkblue}{\textbf{\ipa{ʈʂɯ˧˥}}}}{}
\textcolor{teal}{\zh{动词}} \hspace{4pt} \zh{声调类:} MH.
\zh{筛。} \textcolor{Sepia}{\selectlanguage{english}To sift.} \textcolor{PineGreen}{\selectlanguage{french}Tamiser.}  ¶ \textcolor{darkblue}{\textbf{\ipa{le˧-ʈʂɯ˧-ze˥}}} \zh{\mytextsc{accomp} \string_ \mytextsc{pfv}} \textcolor{Sepia}{\selectlanguage{english}\mytextsc{accomp} \string_ \mytextsc{pfv}} \textcolor{PineGreen}{\selectlanguage{french}\mytextsc{accomp} \string_ \mytextsc{pfv}}  
 ¶ \textcolor{darkblue}{\textbf{\ipa{ɖɯ˧-ʈʂɯ˧-ɻ̍˥}}} \zh{\mytextsc{delimitative} \string_ \mytextsc{inceptive}} \textcolor{Sepia}{\selectlanguage{english}\mytextsc{delimitative} \string_ \mytextsc{inceptive}} \textcolor{PineGreen}{\selectlanguage{french}\mytextsc{délimitatif} \string_ \mytextsc{inchoatif}}  

\lhead{\firstmark}
\rhead{\botmark}

\subsection{\hspace{-0.5cm} {\Large \textcolor{darkblue}{\textbf{\ipa{ʈʂv̩˩}}}}\hspace{0.5cm}[\kern2pt{\textcolor{darkblue}{\textbf{\ipa{ʈʂv̩˩˥}}}}\kern2pt]} \hypertarget{t`s`v\string_=\string_B1}{}
\markboth{\textcolor{darkblue}{\textbf{\ipa{ʈʂv̩˩}}}}{}
\textcolor{teal}{\zh{形容词}} \hspace{4pt} \zh{声调类:} L.
\zh{平和的(生肖)。} \textcolor{Sepia}{\selectlanguage{english}Peaceful, soft (astrological sign).} \textcolor{PineGreen}{\selectlanguage{french}Paisible, aimable, pacifique, doux (signe astrologique); s'emploie au sujet des signes astrologiques: certains sont considérés comme 'paisibles', comme le Boeuf, le Lapin et la Chèvre, ce qui rend les personnes nées cette année-là appropriées pour certains rites/certaines tâches (ex.: lors du rite de passage à l'âge adulte), et au contraire non appropriées pour d'autres.}  ¶ \textcolor{darkblue}{\textbf{\ipa{kʰv̩˧ ʈʂv̩˧˥}}} \zh{平和的生肖,如牛、兔、羊} \textcolor{Sepia}{\selectlanguage{english}a peaceful, soft astrological sign, such as the Ox, the Rabbit and the Goat} \textcolor{PineGreen}{\selectlanguage{french}signe pacifique, calme, non belliqueux}  

\lhead{\firstmark}
\rhead{\botmark}

\subsection{\hspace{-0.5cm} {\Large \textcolor{darkblue}{\textbf{\ipa{ʈʂv̩˩˥}}}}\hspace{0.5cm}[\kern2pt{\textcolor{darkblue}{\textbf{\ipa{ʈʂv̩˩˥}}}}\kern2pt]} \hypertarget{t`s`v\string_=\string_B\string_T1}{}
\markboth{\textcolor{darkblue}{\textbf{\ipa{ʈʂv̩˩˥}}}}{}
\textcolor{teal}{\zh{名词}} \hspace{4pt} \zh{声调类:} LH.
\zh{汗(单音节)。} \textcolor{Sepia}{\selectlanguage{english}Sweat (monosyllable).} \textcolor{PineGreen}{\selectlanguage{french}Sueur (monosyllabe).}  ¶ \textcolor{darkblue}{\textbf{\ipa{ʈʂv̩˧ bv̩˧nv̩˩}}} \zh{有汗(臭)的味道} \textcolor{Sepia}{\selectlanguage{english}smelling of sweat, reeking of sweat} \textcolor{PineGreen}{\selectlanguage{french}qui sent la sueur, malodorant}  

\lhead{\firstmark}
\rhead{\botmark}

\subsection{\hspace{-0.5cm} {\Large \textcolor{darkblue}{\textbf{\ipa{ʈʂv̩˩\textsubscript{a}}}} \textsubscript{1}}\hspace{0.5cm}[\kern2pt{\textcolor{darkblue}{\textbf{\ipa{ʈʂv̩˩˥}}}}\kern2pt]} \hypertarget{t`s`v\string_=\string_Ba1}{}
\markboth{\textcolor{darkblue}{\textbf{\ipa{ʈʂv̩˩\textsubscript{a}}}} \textsubscript{1}}{}
\textcolor{teal}{\zh{动词}} \hspace{4pt} \zh{声调类:} L\textsubscript{a}.
\zh{传染。} \textcolor{Sepia}{\selectlanguage{english}To contaminate, to infect.} \textcolor{PineGreen}{\selectlanguage{french}Contaminer, infecter.}  ¶ \textcolor{darkblue}{\textbf{\ipa{hĩ˧ ʈʂv̩˥-ho˩}}} \zh{(病毒)会传染人的} \textcolor{Sepia}{\selectlanguage{english}(the disease) is going to contaminate someone / is going to contaminate people} \textcolor{PineGreen}{\selectlanguage{french}(la maladie) va contaminer quelqu'un}  
 ¶ \textcolor{darkblue}{\textbf{\ipa{ʈʂv̩˧\textasciitilde{}ʈʂv̩˥}}} \zh{\mytextsc{重叠}} \textcolor{Sepia}{\selectlanguage{english}\mytextsc{red}} \textcolor{PineGreen}{\selectlanguage{french}\mytextsc{red}}  
 ¶ \textcolor{darkblue}{\textbf{\ipa{ʈʂv̩˧\textasciitilde{}ʈʂv̩˥-ɻ̍˩ ho˩}}} \zh{(病毒)会传染的。} \textcolor{Sepia}{\selectlanguage{english}(the disease) is going to contaminate (people)} \textcolor{PineGreen}{\selectlanguage{french}(la maladie) va contaminer (des gens)}  
 ¶ \textcolor{darkblue}{\textbf{\ipa{njɤ˧-ɳɯ˧ | no˧ ʈʂv̩˧-ʝi˥!}}} \zh{(要小心:)我会传染你的!} \textcolor{Sepia}{\selectlanguage{english}(Be careful,) I may contaminate you / pass on my cold to you!} \textcolor{PineGreen}{\selectlanguage{french}(Attention,) je vais te contaminer/te passer (mon rhume)!}  

\lhead{\firstmark}
\rhead{\botmark}

\subsection{\hspace{-0.5cm} {\Large \textcolor{darkblue}{\textbf{\ipa{ʈʂv̩˩\textsubscript{a}}}} \textsubscript{2}}\hspace{0.5cm}[\kern2pt{\textcolor{darkblue}{\textbf{\ipa{ʈʂv̩˩˥}}}}\kern2pt]} \hypertarget{t`s`v\string_=\string_Ba2}{}
\markboth{\textcolor{darkblue}{\textbf{\ipa{ʈʂv̩˩\textsubscript{a}}}} \textsubscript{2}}{}
\textcolor{teal}{\zh{动词}} \hspace{4pt} \zh{声调类:} L\textsubscript{a}.
\zh{点(蜡烛……)。} \textcolor{Sepia}{\selectlanguage{english}To light (a candle).} \textcolor{PineGreen}{\selectlanguage{french}Allumer (une bougie).} 
\lhead{\firstmark}
\rhead{\botmark}

\subsection{\hspace{-0.5cm} {\Large \textcolor{darkblue}{\textbf{\ipa{ʈʂv̩˧di˧˥}}}}\hspace{0.5cm}[\kern2pt{\textcolor{darkblue}{\textbf{\ipa{ʈʂv̩˧di˧˥}}}}\kern2pt]} \hypertarget{t`s`v\string_=\string_Mdi\string_M\string_T1}{}
\markboth{\textcolor{darkblue}{\textbf{\ipa{ʈʂv̩˧di˧˥}}}}{}
\textcolor{teal}{\zh{名词}} \hspace{4pt} \zh{声调类:} MH\#.
\zh{村落名。} \textcolor{Sepia}{\selectlanguage{english}Name of a village outside the plain of Lijiang, in the vicinity of the Lake, close to \textcolor{darkblue}{\textbf{\ipa{/lɑ˧tʰɑ˧-di˧˥/}}}.} \textcolor{PineGreen}{\selectlanguage{french}Village na hors de la plaine de Yongning, vers le Lac, non loin de \textcolor{darkblue}{\textbf{\ipa{/lɑ˧tʰɑ˧-di˧˥/}}}.} 
\lhead{\firstmark}
\rhead{\botmark}

\subsection{\hspace{-0.5cm} {\Large \textcolor{darkblue}{\textbf{\ipa{ʈʂv̩˩dʑɯ˥}}}}\hspace{0.5cm}[\kern2pt{\textcolor{darkblue}{\textbf{\ipa{ʈʂv̩˩dʑɯ˥}}}}\kern2pt]} \hypertarget{t`s`v\string_=\string_Bdz£M\string_T1}{}
\markboth{\textcolor{darkblue}{\textbf{\ipa{ʈʂv̩˩dʑɯ˥}}}}{}
\textcolor{teal}{\zh{名词}} \hspace{4pt} \zh{声调类:} LH.
\zh{汗。} \textcolor{Sepia}{\selectlanguage{english}Sweat.} \textcolor{PineGreen}{\selectlanguage{french}Sueur.} 
\lhead{\firstmark}
\rhead{\botmark}

\subsection{\hspace{-0.5cm} {\Large \textcolor{darkblue}{\textbf{\ipa{ʈʂv̩˧pɤ˩}}}}\hspace{0.5cm}[\kern2pt{\textcolor{darkblue}{\textbf{\ipa{ʈʂv̩˧pɤ˩}}}}\kern2pt]} \hypertarget{t`s`v\string_=\string_Mp7\string_B1}{}
\markboth{\textcolor{darkblue}{\textbf{\ipa{ʈʂv̩˧pɤ˩}}}}{}
\textcolor{teal}{\zh{名词}} \hspace{4pt} \zh{声调类:} L\#.
\zh{菜板、俎。} \textcolor{Sepia}{\selectlanguage{english}Cutting-board; vessel or cutting board for meat.} \textcolor{PineGreen}{\selectlanguage{french}Planche à découper.}  \zh{量词}: \textcolor{darkblue}{\textbf{\ipa{nɑ˧}}} 
\lhead{\firstmark}
\rhead{\botmark}

\subsection{\hspace{-0.5cm} {\Large \textcolor{darkblue}{\textbf{\ipa{ʈʂv̩˧tɕɯ˥}}}}\hspace{0.5cm}[\kern2pt{\textcolor{darkblue}{\textbf{\ipa{ʈʂv̩˧tɕɯ˥}}}}\kern2pt]} \hypertarget{t`s`v\string_=\string_Mts£M\string_T1}{}
\markboth{\textcolor{darkblue}{\textbf{\ipa{ʈʂv̩˧tɕɯ˥}}}}{}
\textcolor{teal}{\zh{名词}} \hspace{4pt} \zh{声调类:} H\#.
\zh{痰。} \textcolor{Sepia}{\selectlanguage{english}Spittle, phlegm, sputum.} \textcolor{PineGreen}{\selectlanguage{french}Crachat.} 
\lhead{\firstmark}
\rhead{\botmark}

\subsection{\hspace{-0.5cm} {\Large \textcolor{darkblue}{\textbf{\ipa{ʈʂwɑ˧\textasciitilde{}ʈʂwɑ˧-nɑ˧\textasciitilde{}nɑ\#˥}}}}\hspace{0.5cm}[\kern2pt{\textcolor{darkblue}{\textbf{\ipa{xxxx non-correspondance entre le nombre de morphèmes et le nombre de tons de morphèmes}}}}\kern2pt]} \hypertarget{t`s`wA\string_M~t`s`wA\string_M-nA\string_M~nA\#\string_T1}{}
\markboth{\textcolor{darkblue}{\textbf{\ipa{ʈʂwɑ˧\textasciitilde{}ʈʂwɑ˧-nɑ˧\textasciitilde{}nɑ\#˥}}}}{}
\textcolor{teal}{\zh{形容词}} \hspace{4pt} \zh{声调类:} \#H.
\zh{杂、混杂。} \textcolor{Sepia}{\selectlanguage{english}Mixed; diverse, heterogeneous; messy.} \textcolor{PineGreen}{\selectlanguage{french}Divers, varié, désordonné.}  ¶ \textcolor{darkblue}{\textbf{\ipa{ʈʂwɑ˧\textasciitilde{}ʈʂwɑ˧-nɑ˧\textasciitilde{}nɑ˧-hĩ˥}}} \zh{混杂的} \textcolor{Sepia}{\selectlanguage{english}\mytextsc{rel}/nmlz} \textcolor{PineGreen}{\selectlanguage{french}\mytextsc{rel}/nmlz}  
 ¶ \textcolor{darkblue}{\textbf{\ipa{ʈʂwɑ˧\textasciitilde{}ʈʂwɑ˧-nɑ˧\textasciitilde{}nɑ˧-ɻ̍˥}}} \zh{混杂} \textcolor{Sepia}{\selectlanguage{english}messy} \textcolor{PineGreen}{\selectlanguage{french}désordonné}  

\lhead{\firstmark}
\rhead{\botmark}

\subsection{\hspace{-0.5cm} {\Large \textcolor{darkblue}{\textbf{\ipa{ʈʂwæ˥\textsubscript{a}}}}}\hspace{0.5cm}[\kern2pt{\textcolor{darkblue}{\textbf{\ipa{ʈʂwæ˧˥}}}}\kern2pt]} \hypertarget{t`s`w\{\string_Ta1}{}
\markboth{\textcolor{darkblue}{\textbf{\ipa{ʈʂwæ˥\textsubscript{a}}}}}{}
\textcolor{teal}{\zh{量词}} \hspace{4pt} \zh{声调类:} H\textsubscript{a}.
\zh{量词:征途、路程、路途、征程,趟。} \textcolor{Sepia}{\selectlanguage{english}Classifier for journeys.} \textcolor{PineGreen}{\selectlanguage{french}Classificateur des trajets.}  ¶ \textcolor{darkblue}{\textbf{\ipa{ɖɯ˧-ɲi˥ | ɖɯ˧-ʈʂwæ˧ bi˧}}} \zh{一天去一趟} \textcolor{Sepia}{\selectlanguage{english}to go once a day, to go one time each day} \textcolor{PineGreen}{\selectlanguage{french}(y) aller une fois par jour}  

\lhead{\firstmark}
\rhead{\botmark}

\subsection{\hspace{-0.5cm} {\Large \textcolor{darkblue}{\textbf{\ipa{ʈʂwæ˧tʰo˩}}}}\hspace{0.5cm}[\kern2pt{\textcolor{darkblue}{\textbf{\ipa{ʈʂwæ˧tʰo˧}}}}\kern2pt]} \hypertarget{t`s`w\{\string_Mt\string_ho\string_B1}{}
\markboth{\textcolor{darkblue}{\textbf{\ipa{ʈʂwæ˧tʰo˩}}}}{}
\textcolor{teal}{\zh{名词}} \hspace{4pt} \zh{声调类:} L\#.
\zh{砖头(汉语借词)。} \textcolor{Sepia}{\selectlanguage{english}Brick.} \textcolor{PineGreen}{\selectlanguage{french}Brique.}  \zh{【借词】} \zh{砖头}
\zh{~【参考】~} \hyperlink{}{\textcolor{darkblue}{\textbf{\ipa{tʰo˩tɕi˧˥}}}} 
\lhead{\firstmark}
\rhead{\botmark}

\subsection{\hspace{-0.5cm} {\Large \textcolor{darkblue}{\textbf{\ipa{ʈʂwæ˧\textasciitilde{}ʈʂwæ˧}}}}\hspace{0.5cm}[\kern2pt{\textcolor{darkblue}{\textbf{\ipa{ʈʂwæ˧ʈʂwæ˧˥}}}}\kern2pt]} \hypertarget{t`s`w\{\string_M~t`s`w\{\string_M1}{}
\markboth{\textcolor{darkblue}{\textbf{\ipa{ʈʂwæ˧\textasciitilde{}ʈʂwæ˧}}}}{}
\textcolor{teal}{\zh{动词}} \hspace{4pt} \zh{声调类:} M.
\zh{搅拌、使混合。} \textcolor{Sepia}{\selectlanguage{english}To mix.} \textcolor{PineGreen}{\selectlanguage{french}Mélanger.} 
\lhead{\firstmark}
\rhead{\botmark}

\subsection{\hspace{-0.5cm} {\Large \textcolor{darkblue}{\textbf{\ipa{ʈʂwæ˩ho˧ɻ̍˧}}}}\hspace{0.5cm}[\kern2pt{\textcolor{darkblue}{\textbf{\ipa{ʈʂwæ˧ho˧ɻ̍˧}}}}\kern2pt]} \hypertarget{t`s`w\{\string_Bho\string_Mr£`̍\string_M1}{}
\markboth{\textcolor{darkblue}{\textbf{\ipa{ʈʂwæ˩ho˧ɻ̍˧}}}}{}
\textcolor{teal}{\zh{名词}} \hspace{4pt} \zh{声调类:} LM.
\zh{钻子。} \textcolor{Sepia}{\selectlanguage{english}Drill.} \textcolor{PineGreen}{\selectlanguage{french}Perceuse.}  \zh{【借词】} \zh{钻}

\lhead{\firstmark}
\rhead{\botmark}

\subsection{\hspace{-0.5cm} {\Large \textcolor{darkblue}{\textbf{\ipa{ʈʂwæ˩\textasciitilde{}ʈʂwæ˧˥}}}}\hspace{0.5cm}[\kern2pt{\textcolor{darkblue}{\textbf{\ipa{ʈʂwæ˧ʈʂwæ˧}}}}\kern2pt]} \hypertarget{t`s`w\{\string_B~t`s`w\{\string_M\string_T1}{}
\markboth{\textcolor{darkblue}{\textbf{\ipa{ʈʂwæ˩\textasciitilde{}ʈʂwæ˧˥}}}}{}
\textcolor{teal}{\zh{动词}} \hspace{4pt} \zh{声调类:} MH.
\zh{手拉手。} \textcolor{Sepia}{\selectlanguage{english}To hold by the hand, to hold each other's hands.} \textcolor{PineGreen}{\selectlanguage{french}Se tenir par la main, se tenir la main.}  ¶ \textcolor{darkblue}{\textbf{\ipa{le˧-ʈʂwæ˧\textasciitilde{}ʈʂwæ˧-ze˧!}}} \zh{\mytextsc{accomp} \mytextsc{red} \mytextsc{pfv}} \textcolor{Sepia}{\selectlanguage{english}\mytextsc{accomp} \mytextsc{red} \mytextsc{pfv}} \textcolor{PineGreen}{\selectlanguage{french}\mytextsc{accomp} \mytextsc{red} \mytextsc{pfv}}  
 ¶ \textcolor{darkblue}{\textbf{\ipa{ʈʂwæ˩\textasciitilde{}ʈʂwæ˧-ɻ̍˥}}} \zh{\mytextsc{red} \mytextsc{inceptive}} \textcolor{Sepia}{\selectlanguage{english}\mytextsc{red} \mytextsc{inceptive}} \textcolor{PineGreen}{\selectlanguage{french}\mytextsc{red} \mytextsc{inchoatif;} même sens: se tenir par la main}  

\lhead{\firstmark}
\rhead{\botmark}

\subsection{\hspace{-0.5cm} {\Large \textcolor{darkblue}{\textbf{\ipa{ʈʂwæ˧˥}}} \textsubscript{1}}\hspace{0.5cm}[\kern2pt{\textcolor{darkblue}{\textbf{\ipa{ʈʂwæ˧˥}}}}\kern2pt]} \hypertarget{t`s`w\{\string_M\string_T1}{}
\markboth{\textcolor{darkblue}{\textbf{\ipa{ʈʂwæ˧˥}}} \textsubscript{1}}{}
\textcolor{teal}{\zh{动词}} \hspace{4pt} \zh{声调类:} MH.
 \zh{【借词】} \zh{装?}
\ding{202} \zh{安装。} \textcolor{Sepia}{\selectlanguage{english}To set up, to install.} \textcolor{PineGreen}{\selectlanguage{french}Installer.}  ¶ \textcolor{darkblue}{\textbf{\ipa{tjɤ˧hwɑ˧ ʈʂwæ˥}}} \zh{安装电话(座机)} \textcolor{Sepia}{\selectlanguage{english}to set up the telephone, to put up a telephone line (in a house that did not have it before)} \textcolor{PineGreen}{\selectlanguage{french}installer le téléphone (dans une demeure qui n'y était pas reliée précédemment)}  
 ¶ \textcolor{darkblue}{\textbf{\ipa{le˧-ʈʂwæ˧˥ le˧-tse˧-ze˧!}}} \zh{装好了!} \textcolor{Sepia}{\selectlanguage{english}It's installed! / It is now well installed!} \textcolor{PineGreen}{\selectlanguage{french}C'est bien installé!}  
\ding{203} \zh{补(牙)、修好(坏牙)。} \textcolor{Sepia}{\selectlanguage{english}To repair, to cure (a tooth).} \textcolor{PineGreen}{\selectlanguage{french}Soigner, réparer (une dent).}  ¶ \textcolor{darkblue}{\textbf{\ipa{hi˧ ʈʂwæ˩}}} \zh{补牙、修好坏牙} \textcolor{Sepia}{\selectlanguage{english}to cure a tooth} \textcolor{PineGreen}{\selectlanguage{french}soigner une dent; littéralement “remettre une dent”}  
 ¶ \textcolor{darkblue}{\textbf{\ipa{hi˧ | le˧-ʈʂwæ˧-ze˥!}}} \zh{牙补好了!} \textcolor{Sepia}{\selectlanguage{english}The tooth is cured!} \textcolor{PineGreen}{\selectlanguage{french}La dent est soignée! / La dent est réparée!}  
\ding{204} \zh{结线。} \textcolor{Sepia}{\selectlanguage{english}To tie (a string), to make a knot to tie two pieces of thread together.} \textcolor{PineGreen}{\selectlanguage{french}Nouer (des fils).}  ¶ \textcolor{darkblue}{\textbf{\ipa{kʰɯ˩ ʈʂwæ˩˥}}} \zh{结线} \textcolor{Sepia}{\selectlanguage{english}to tie pieces of thread together} \textcolor{PineGreen}{\selectlanguage{french}attacher des brins de fil ensemble, nouer des fils (par ex. lorsqu'on prépare le métier à tisser)}  

\lhead{\firstmark}
\rhead{\botmark}

\subsection{\hspace{-0.5cm} {\Large \textcolor{darkblue}{\textbf{\ipa{ʈʂwæ˧˥}}} \textsubscript{2}}\hspace{0.5cm}[\kern2pt{\textcolor{darkblue}{\textbf{\ipa{ʈʂwæ˧˥}}}}\kern2pt]} \hypertarget{t`s`w\{\string_M\string_T2}{}
\markboth{\textcolor{darkblue}{\textbf{\ipa{ʈʂwæ˧˥}}} \textsubscript{2}}{}
\textcolor{teal}{\zh{动词}} \hspace{4pt} \zh{声调类:} MH.
\zh{欣赏、品尝(饮食、音乐……)。} \textcolor{Sepia}{\selectlanguage{english}To savour, to enjoy, to relish.} \textcolor{PineGreen}{\selectlanguage{french}Savourer, déguster, siroter (nourriture ou boisson).}  ¶ \textcolor{darkblue}{\textbf{\ipa{no˧ | li˩ ʈʂwæ˧-ɻ̍˥! |}}} \zh{请您品一点茶!(礼貌说法)} \textcolor{Sepia}{\selectlanguage{english}Please enjoy a cup of tea! (A polite invitation)} \textcolor{PineGreen}{\selectlanguage{french}Veuillez prendre un peu de thé! (Invitation polie)}  
 ¶ \textcolor{darkblue}{\textbf{\ipa{ʐɯ˧ F | ʈʂwæ˧˥! | li˩˥ F | ʈʂwæ˧˥! hɑ˧ F | ʈʂwæ˧˥!}}} \zh{酒,是可以品尝的!茶,是可以品尝的!饭,是可以品尝的!(关于‘品尝’这个动词的说明)} \textcolor{Sepia}{\selectlanguage{english}Wine is something to be savoured! Tea is something to be savoured! Food is something to be savoured! (An explanation about the use of the verb.)} \textcolor{PineGreen}{\selectlanguage{french}L'alcool, ça se savoure; le thé, ça se savoure! (Explication au sujet des emplois du verbe)}  
 ¶ \textcolor{darkblue}{\textbf{\ipa{hɑ˧ ʈʂwæ˩}}} \zh{品尝食物} \textcolor{Sepia}{\selectlanguage{english}to savour food} \textcolor{PineGreen}{\selectlanguage{french}savourer de la nourriture}  
 ¶ \textcolor{darkblue}{\textbf{\ipa{li˩ ʈʂwæ˧˥}}} \zh{品茶} \textcolor{Sepia}{\selectlanguage{english}to savour tea} \textcolor{PineGreen}{\selectlanguage{french}savourer du thé}  
 ¶ \textcolor{darkblue}{\textbf{\ipa{ʐɯ˧ ʈʂwæ˧˥}}} \zh{品酒} \textcolor{Sepia}{\selectlanguage{english}to savour wine} \textcolor{PineGreen}{\selectlanguage{french}savourer de l'alcool}  
 ¶ \textcolor{darkblue}{\textbf{\ipa{ə˩kʰɯ˩ ʈʂwæ˥}}} \zh{尝尝圆根(玩笑话,因为圆根没有什么滋味)} \textcolor{Sepia}{\selectlanguage{english}to savour turnip (an ironic but fully acceptable statement)} \textcolor{PineGreen}{\selectlanguage{french}savourer des feuilles de navet (formulation ironique mais tout à fait acceptable)}  

\lhead{\firstmark}
\rhead{\botmark}

\subsection{\hspace{-0.5cm} {\Large \textcolor{darkblue}{\textbf{\ipa{ʈʂwɤ˧\textsubscript{a}}}}}\hspace{0.5cm}[\kern2pt{\textcolor{darkblue}{\textbf{\ipa{ʈʂwɤ˥}}}}\kern2pt]} \hypertarget{t`s`w7\string_Ma1}{}
\markboth{\textcolor{darkblue}{\textbf{\ipa{ʈʂwɤ˧\textsubscript{a}}}}}{}
\textcolor{teal}{\zh{量词}} \hspace{4pt} \zh{声调类:} M\textsubscript{a}.
\zh{量词:捧。} \textcolor{Sepia}{\selectlanguage{english}A handful (with one single hand).} \textcolor{PineGreen}{\selectlanguage{french}Classificateur des poignées: ce que l'on peut prendre dans une seule main.} 
\lhead{\firstmark}
\rhead{\botmark}

\subsection{\hspace{-0.5cm} {\Large \textcolor{darkblue}{\textbf{\ipa{ʈʂwɤ˧\textsubscript{a}}}}}\hspace{0.5cm}[\kern2pt{\textcolor{darkblue}{\textbf{\ipa{ʈʂwɤ˩˥}}}}\kern2pt]} \hypertarget{t`s`w7\string_Ma1}{}
\markboth{\textcolor{darkblue}{\textbf{\ipa{ʈʂwɤ˧\textsubscript{a}}}}}{}
\textcolor{teal}{\zh{动词}} \hspace{4pt} \zh{声调类:} M\textsubscript{a}.
\zh{抓(用爪子抓)。} \textcolor{Sepia}{\selectlanguage{english}To scratch (with claws, e.g. of tiger).} \textcolor{PineGreen}{\selectlanguage{french}Griffer (ex.: un tigre griffe).}  ¶ \textcolor{darkblue}{\textbf{\ipa{tso˧\textasciitilde{}tso˧ ʈʂwɤ˩}}} \zh{抓东西} \textcolor{Sepia}{\selectlanguage{english}to scratch objects} \textcolor{PineGreen}{\selectlanguage{french}griffer des objets}  

\lhead{\firstmark}
\rhead{\botmark}

\subsection{\hspace{-0.5cm} {\Large \textcolor{darkblue}{\textbf{\ipa{ʈʂwɤ˧\textasciitilde{}ʈʂwɤ˩}}}}\hspace{0.5cm}[\kern2pt{\textcolor{darkblue}{\textbf{\ipa{ʈʂwɤ˧ʈʂwɤ˧˥}}}}\kern2pt]} \hypertarget{t`s`w7\string_M~t`s`w7\string_B1}{}
\markboth{\textcolor{darkblue}{\textbf{\ipa{ʈʂwɤ˧\textasciitilde{}ʈʂwɤ˩}}}}{}
\textcolor{teal}{\zh{动词}} \hspace{4pt} \zh{声调类:} M.
\zh{触碰。} \textcolor{Sepia}{\selectlanguage{english}To touch.} \textcolor{PineGreen}{\selectlanguage{french}Toucher.}  ¶ \textcolor{darkblue}{\textbf{\ipa{ə˧tso˧ mɤ˧-ɲi˩ ʈʂwɤ˧\textasciitilde{}ʈʂwɤ˩!}}} \zh{你什么都碰,是吗!(小孩爬在桌子上,试着拿每个东西)} \textcolor{Sepia}{\selectlanguage{english}You really touch all and everything, don't you! (Mildly scolding a baby that crawls around on a table and grabs every object in turn)} \textcolor{PineGreen}{\selectlanguage{french}(tu) touches vraiment à tout! (doux reproche adressé à un bébé qui se promène sur une table et s'empare de tout ce qui s'y trouve)}  

\lhead{\firstmark}
\rhead{\botmark}

\subsection{\hspace{-0.5cm} {\Large \textcolor{darkblue}{\textbf{\ipa{ʈʂʰɑ˧lɑ˧}}}}\hspace{0.5cm}[\kern2pt{\textcolor{darkblue}{\textbf{\ipa{ʈʂʰɑ˧lɑ˧}}}}\kern2pt]} \hypertarget{t`s`\string_hA\string_MlA\string_M1}{}
\markboth{\textcolor{darkblue}{\textbf{\ipa{ʈʂʰɑ˧lɑ˧}}}}{}
\textcolor{teal}{\zh{动词}} \hspace{4pt} \zh{声调类:} M.
\zh{商量、交谈、谈天、聊天。} \textcolor{Sepia}{\selectlanguage{english}To discuss, to have a talk, to chat.} \textcolor{PineGreen}{\selectlanguage{french}Discuter, bavarder.}  ¶ \textcolor{darkblue}{\textbf{\ipa{hĩ˧-qɑ˩ ʈʂʰɑ˩lɑ˩}}} \zh{跟人聊天} \textcolor{Sepia}{\selectlanguage{english}to have a chat with someone} \textcolor{PineGreen}{\selectlanguage{french}bavarder avec quelqu'un}  
 ¶ \textcolor{darkblue}{\textbf{\ipa{ɖɯ˧-kʰwɤ˧ ʈʂʰɑ˧lɑ˥}}} \zh{聊聊天} \textcolor{Sepia}{\selectlanguage{english}to have a small chat} \textcolor{PineGreen}{\selectlanguage{french}bavarder un peu, avoir une causerie avec}  
 ¶ \textcolor{darkblue}{\textbf{\ipa{njɤ˧ | no˧-qɑ˧ ʈʂʰɑ˧lɑ˥}}} \zh{我给你讲、我跟你聊聊天} \textcolor{Sepia}{\selectlanguage{english}I tell you, I narrate to you} \textcolor{PineGreen}{\selectlanguage{french}je te raconte, je te dis}  

\lhead{\firstmark}
\rhead{\botmark}

\subsection{\hspace{-0.5cm} {\Large \textcolor{darkblue}{\textbf{\ipa{ʈʂʰɑ˧lɑ˧-mv̩˧lɑ˩}}}}\hspace{0.5cm}[\kern2pt{\textcolor{darkblue}{\textbf{\ipa{xxxx non-correspondance entre le nombre de morphèmes et le nombre de tons de morphèmes}}}}\kern2pt]} \hypertarget{t`s`\string_hA\string_MlA\string_M-mv\string_=\string_MlA\string_B1}{}
\markboth{\textcolor{darkblue}{\textbf{\ipa{ʈʂʰɑ˧lɑ˧-mv̩˧lɑ˩}}}}{}
\textcolor{teal}{\zh{动词}} \hspace{4pt} \zh{声调类:} M.
\zh{商量、交谈、谈天、聊天。} \textcolor{Sepia}{\selectlanguage{english}To discuss, to have a talk, to chat.} \textcolor{PineGreen}{\selectlanguage{french}Discuter, bavarder.}  ¶ \textcolor{darkblue}{\textbf{\ipa{ʈʂʰɑ˧lɑ˧-mv̩˧lɑ˩-ɻ̍˩}}} \zh{聊聊天} \textcolor{Sepia}{\selectlanguage{english}to have a chat} \textcolor{PineGreen}{\selectlanguage{french}bavarder un peu, avoir une causerie}  

\lhead{\firstmark}
\rhead{\botmark}

\subsection{\hspace{-0.5cm} {\Large \textcolor{darkblue}{\textbf{\ipa{ʈʂʰɑ˧nɑ˥}}}}\hspace{0.5cm}[\kern2pt{\textcolor{darkblue}{\textbf{\ipa{ʈʂʰɑ˧nɑ˥}}}}\kern2pt]} \hypertarget{t`s`\string_hA\string_MnA\string_T1}{}
\markboth{\textcolor{darkblue}{\textbf{\ipa{ʈʂʰɑ˧nɑ˥}}}}{}
\textcolor{teal}{\zh{名词}} \hspace{4pt} \zh{声调类:} H\#.
\zh{一眼山泉的名字。} \textcolor{Sepia}{\selectlanguage{english}The name of a sacred spring, at the foot of a cliff, on the mountain \textcolor{darkblue}{\textbf{\ipa{/qv̩˧ɻ\#˥/}}}.} \textcolor{PineGreen}{\selectlanguage{french}Nom d'une source sacrée, située au pied d'une falaise, sur la montagne \textcolor{darkblue}{\textbf{\ipa{/qv̩˧ɻ\#˥/;}}} on disait que son eau sortait du ventre de la montagne. Le récit DumbChildren raconte comment son eau était utilisée comme remède de fertilité.}  ¶ \textcolor{darkblue}{\textbf{\ipa{qv̩˧ɻ̍˧-ʈʂʰɑ˧nɑ˥\#}}} \zh{山的全名,包括水泉名} \textcolor{Sepia}{\selectlanguage{english}the full name of the mountain} \textcolor{PineGreen}{\selectlanguage{french}nom complet de la montagne}  
 ¶ \textcolor{darkblue}{\textbf{\ipa{kɤ˧mv̩˧˥, | æ˧ʂæ˧, | ŋwɤ˧hɑ̃˩, | ʂwæ˧gv̩\#˥, | nɑ˩tsʰi˩˥ | -tɕʰɤ˧pɤ˧mi\#˥, | qv̩˧ɻ̍˧-ʈʂʰɑ˧nɑ˥ |}}} \zh{永宁地区有固定名字的六座山。其它的山,因为没有重要的象征意义,因此没有取名。} \textcolor{Sepia}{\selectlanguage{english}The six mountains of Yongning that carry a name and have a definite symbolic value. The other mountains do not have comparable symbolic value, and fewer people use specific names for them.} \textcolor{PineGreen}{\selectlanguage{french}Les six montagnes de Yongning qui portent un nom. Les autres sommets du voisinage n'ont pas une valeur symbolique comparable, et ne portent pas de nom communément utilisé.}  

\lhead{\firstmark}
\rhead{\botmark}

\subsection{\hspace{-0.5cm} {\Large \textcolor{darkblue}{\textbf{\ipa{ʈʂʰæ˥}}}}\hspace{0.5cm}[\kern2pt{\textcolor{darkblue}{\textbf{\ipa{ʈʂʰæ˧˥}}}}\kern2pt]} \hypertarget{t`s`\string_h\{\string_T1}{}
\markboth{\textcolor{darkblue}{\textbf{\ipa{ʈʂʰæ˥}}}}{}
\textcolor{teal}{\zh{动词}} \hspace{4pt} \zh{声调类:} H.
\zh{洗(洗衣服,洗澡……)。} \textcolor{Sepia}{\selectlanguage{english}To wash (clothes, oneself…).} \textcolor{PineGreen}{\selectlanguage{french}Laver (les habits, la vaisselle…), rincer (le riz…).}  ¶ \textcolor{darkblue}{\textbf{\ipa{dʑi˧hṽ˧ ʈʂʰæ˧}}} \zh{洗衣服} \textcolor{Sepia}{\selectlanguage{english}to wash clothes} \textcolor{PineGreen}{\selectlanguage{french}laver des vêtements}  
 ¶ \textcolor{darkblue}{\textbf{\ipa{bɑ˩lɑ˩ ʈʂʰæ˩˥}}} \zh{洗上衣} \textcolor{Sepia}{\selectlanguage{english}to wash shirts} \textcolor{PineGreen}{\selectlanguage{french}laver des chemises}  
 ¶ \textcolor{darkblue}{\textbf{\ipa{ɬi˧qʰwɤ˩ ʈʂʰæ˩}}} \zh{洗裤子} \textcolor{Sepia}{\selectlanguage{english}to wash trousers} \textcolor{PineGreen}{\selectlanguage{french}laver des pantalons}  
 ¶ \textcolor{darkblue}{\textbf{\ipa{gv̩˧mi˧ ʈʂʰæ˧}}} \zh{洗澡} \textcolor{Sepia}{\selectlanguage{english}to wash oneself, to take a bath/shower} \textcolor{PineGreen}{\selectlanguage{french}se laver, prendre un bain/une douche}  
 ¶ \textcolor{darkblue}{\textbf{\ipa{gv̩˧mi˧ ʈʂʰæ˧\textasciitilde{}ʈʂʰæ˧}}} \zh{洗一下身体} \textcolor{Sepia}{\selectlanguage{english}to wash oneself a bit, to do a quick clean-up} \textcolor{PineGreen}{\selectlanguage{french}se laver un coup}  
 ¶ \textcolor{darkblue}{\textbf{\ipa{hɑ˧ ʈʂʰæ˧}}} \zh{淘洗粮食} \textcolor{Sepia}{\selectlanguage{english}to rinse cereals (before cooking)} \textcolor{PineGreen}{\selectlanguage{french}rincer une céréale (avant de la cuire)}  
 ¶ \textcolor{darkblue}{\textbf{\ipa{ɕi˧ʈʂʰwæ˧ ʈʂʰæ˧(-ze˩)}}} \zh{淘米} \textcolor{Sepia}{\selectlanguage{english}to rinse rice (before cooking)} \textcolor{PineGreen}{\selectlanguage{french}rincer le riz (avant de le cuire)}  

\lhead{\firstmark}
\rhead{\botmark}

\subsection{\hspace{-0.5cm} {\Large \textcolor{darkblue}{\textbf{\ipa{ʈʂʰæ˧ɣɯ\#˥}}}}\hspace{0.5cm}[\kern2pt{\textcolor{darkblue}{\textbf{\ipa{ʈʂʰæ˧ɣɯ˧}}}}\kern2pt]} \hypertarget{t`s`\string_h\{\string_MGM\#\string_T1}{}
\markboth{\textcolor{darkblue}{\textbf{\ipa{ʈʂʰæ˧ɣɯ\#˥}}}}{}
\textcolor{teal}{\zh{名词}} \hspace{4pt} \zh{声调类:} \#H.
\zh{药。} \textcolor{Sepia}{\selectlanguage{english}Medicine.} \textcolor{PineGreen}{\selectlanguage{french}Médicament.}  ¶ \textcolor{darkblue}{\textbf{\ipa{ʈʂʰæ˧ɣɯ˧ ʈʰɯ˧˥}}} \zh{吃药(直译:“喝药”)} \textcolor{Sepia}{\selectlanguage{english}to take a medicine; literally “to drink a medicine”} \textcolor{PineGreen}{\selectlanguage{french}Prendre un médicament. Littéralement: “boire un médicament” (collocation différente du chinois \zh{吃药} “manger un médicament”)}  
 ¶ \textcolor{darkblue}{\textbf{\ipa{ʈʂʰæ˧ɣɯ˧ lɑ˩}}} \zh{打农药} \textcolor{Sepia}{\selectlanguage{english}to spread pesticides (in an orchard, a vegetable garden or a field)} \textcolor{PineGreen}{\selectlanguage{french}répandre des pesticides, traiter (un verger, un potager, un champ…)}  

\lhead{\firstmark}
\rhead{\botmark}

\subsection{\hspace{-0.5cm} {\Large \textcolor{darkblue}{\textbf{\ipa{ʈʂʰæ˧ɣɯ˧-ki˩-hĩ˩-hĩ˩}}}}\hspace{0.5cm}[\kern2pt{\textcolor{darkblue}{\textbf{\ipa{xxxx non-correspondance entre le nombre de morphèmes et le nombre de tons de morphèmes}}}}\kern2pt]} \hypertarget{t`s`\string_h\{\string_MGM\string_M-ki\string_B-hi\string_~\string_B-hi\string_~\string_B1}{}
\markboth{\textcolor{darkblue}{\textbf{\ipa{ʈʂʰæ˧ɣɯ˧-ki˩-hĩ˩-hĩ˩}}}}{}
\textcolor{teal}{\zh{名词}} \hspace{4pt} \zh{声调类:} \mytextsc{L}.
\zh{医生。} \textcolor{Sepia}{\selectlanguage{english}Doctor; literally: “person who gives medicines”.} \textcolor{PineGreen}{\selectlanguage{french}Médecin, docteur; littéralement: “personne qui donne des médicaments”.}  \zh{量词}: \textcolor{darkblue}{\textbf{\ipa{v̩˧}}} 
\lhead{\firstmark}
\rhead{\botmark}

\subsection{\hspace{-0.5cm} {\Large \textcolor{darkblue}{\textbf{\ipa{ʈʂʰæ˧mi˥\$}}}}\hspace{0.5cm}[\kern2pt{\textcolor{darkblue}{\textbf{\ipa{ʈʂʰæ˧mi˥}}}}\kern2pt]} \hypertarget{t`s`\string_h\{\string_Mmi\string_T\$1}{}
\markboth{\textcolor{darkblue}{\textbf{\ipa{ʈʂʰæ˧mi˥\$}}}}{}
\textcolor{teal}{\zh{名词}} \hspace{4pt} \zh{声调类:} H\$.
\zh{母马鹿。} \textcolor{Sepia}{\selectlanguage{english}Doe, hind.} \textcolor{PineGreen}{\selectlanguage{french}Biche.}  \zh{量词}: \textcolor{darkblue}{\textbf{\ipa{pʰo˧˥}}} 
\lhead{\firstmark}
\rhead{\botmark}

\subsection{\hspace{-0.5cm} {\Large \textcolor{darkblue}{\textbf{\ipa{ʈʂʰæ˧nɑ˥}}}}\hspace{0.5cm}[\kern2pt{\textcolor{darkblue}{\textbf{\ipa{ʈʂʰæ˧nɑ˥}}}}\kern2pt]} \hypertarget{t`s`\string_h\{\string_MnA\string_T1}{}
\markboth{\textcolor{darkblue}{\textbf{\ipa{ʈʂʰæ˧nɑ˥}}}}{}
\textcolor{teal}{\zh{名词}} \hspace{4pt} \zh{声调类:} H\#.
\zh{黑鹿。} \textcolor{Sepia}{\selectlanguage{english}Black stag: a legendary species, which only spirits are able to hunt down.} \textcolor{PineGreen}{\selectlanguage{french}Cerf noir: espèce légendaire, que seuls les esprits sont à même de chasser et abattre.}  \zh{量词}: \textcolor{darkblue}{\textbf{\ipa{pʰo˧˥}}} 
\lhead{\firstmark}
\rhead{\botmark}

\subsection{\hspace{-0.5cm} {\Large \textcolor{darkblue}{\textbf{\ipa{ʈʂʰæ˧pʰv̩\#˥}}}}\hspace{0.5cm}[\kern2pt{\textcolor{darkblue}{\textbf{\ipa{ʈʂʰæ˧pʰv̩˧}}}}\kern2pt]} \hypertarget{t`s`\string_h\{\string_Mp\string_hv\string_=\#\string_T1}{}
\markboth{\textcolor{darkblue}{\textbf{\ipa{ʈʂʰæ˧pʰv̩\#˥}}}}{}
\textcolor{teal}{\zh{名词}} \hspace{4pt} \zh{声调类:} \#H.
\zh{公马鹿。} \textcolor{Sepia}{\selectlanguage{english}Male deer.} \textcolor{PineGreen}{\selectlanguage{french}Cerf (mâle).}  \zh{量词}: \textcolor{darkblue}{\textbf{\ipa{pʰo˧˥}}} 
\lhead{\firstmark}
\rhead{\botmark}

\subsection{\hspace{-0.5cm} {\Large \textcolor{darkblue}{\textbf{\ipa{ʈʂʰæ˧qʰv̩˥\$}}}}\hspace{0.5cm}[\kern2pt{\textcolor{darkblue}{\textbf{\ipa{ʈʂʰæ˧qʰv̩˥}}}}\kern2pt]} \hypertarget{t`s`\string_h\{\string_Mq\string_hv\string_=\string_T\$1}{}
\markboth{\textcolor{darkblue}{\textbf{\ipa{ʈʂʰæ˧qʰv̩˥\$}}}}{}
\textcolor{teal}{\zh{名词}} \hspace{4pt} \zh{声调类:} H\$.
\zh{鹿角,鹿茸。} \textcolor{Sepia}{\selectlanguage{english}Antlers; pilose antler (of young stags).} \textcolor{PineGreen}{\selectlanguage{french}Bois d'un cerf (même mot pour les bois d'un jeune cerf, utilisés comme aphrodisiaque en médecine traditionnelle).}  \zh{量词}: \textcolor{darkblue}{\textbf{\ipa{ɭɯ˧}}} 
\lhead{\firstmark}
\rhead{\botmark}

\subsection{\hspace{-0.5cm} {\Large \textcolor{darkblue}{\textbf{\ipa{ʈʂʰæ˧\textasciitilde{}ʈʂʰæ˧}}}}\hspace{0.5cm}[\kern2pt{\textcolor{darkblue}{\textbf{\ipa{ʈʂʰæ˧ʈʂʰæ˧}}}}\kern2pt]} \hypertarget{t`s`\string_h\{\string_M~t`s`\string_h\{\string_M1}{}
\markboth{\textcolor{darkblue}{\textbf{\ipa{ʈʂʰæ˧\textasciitilde{}ʈʂʰæ˧}}}}{}
\textcolor{teal}{\zh{形容词}} \hspace{4pt} \zh{声调类:} M.
\zh{结实、质量好,(东西)耐用,(人)可靠。} \textcolor{Sepia}{\selectlanguage{english}Solid, of good quality.} \textcolor{PineGreen}{\selectlanguage{french}Solide, de bonne qualité, résistant (vêtement, outil, objet...).} 
\lhead{\firstmark}
\rhead{\botmark}

\subsection{\hspace{-0.5cm} {\Large \textcolor{darkblue}{\textbf{\ipa{ʈʂʰæ˧zo\#˥}}}}\hspace{0.5cm}[\kern2pt{\textcolor{darkblue}{\textbf{\ipa{ʈʂʰæ˧zo˧}}}}\kern2pt]} \hypertarget{t`s`\string_h\{\string_Mzo\#\string_T1}{}
\markboth{\textcolor{darkblue}{\textbf{\ipa{ʈʂʰæ˧zo\#˥}}}}{}
\textcolor{teal}{\zh{名词}} \hspace{4pt} \zh{声调类:} \#H.
\zh{小鹿。} \textcolor{Sepia}{\selectlanguage{english}Baby deer.} \textcolor{PineGreen}{\selectlanguage{french}Faon.}  \zh{量词}: \textcolor{darkblue}{\textbf{\ipa{ɭɯ˧}}} 
\lhead{\firstmark}
\rhead{\botmark}

\subsection{\hspace{-0.5cm} {\Large \textcolor{darkblue}{\textbf{\ipa{ʈʂʰæ˧˥}}} \textsubscript{1}}\hspace{0.5cm}[\kern2pt{\textcolor{darkblue}{\textbf{\ipa{ʈʂʰæ˧˥}}}}\kern2pt]} \hypertarget{t`s`\string_h\{\string_M\string_T1}{}
\markboth{\textcolor{darkblue}{\textbf{\ipa{ʈʂʰæ˧˥}}} \textsubscript{1}}{}
\textcolor{teal}{\zh{名词}} \hspace{4pt} \zh{声调类:} MH.
\zh{马鹿。} \textcolor{Sepia}{\selectlanguage{english}Deer, red deer, \textit{Cervus elaphus kansuensis}.} \textcolor{PineGreen}{\selectlanguage{french}Cerf, \textit{Cervus elaphus kansuensis}.}  \zh{量词}: \textcolor{darkblue}{\textbf{\ipa{pʰo˧˥}}} 
\lhead{\firstmark}
\rhead{\botmark}

\subsection{\hspace{-0.5cm} {\Large \textcolor{darkblue}{\textbf{\ipa{ʈʂʰæ˧˥}}} \textsubscript{2}}\hspace{0.5cm}[\kern2pt{\textcolor{darkblue}{\textbf{\ipa{ʈʂʰæ˧˥}}}}\kern2pt]} \hypertarget{t`s`\string_h\{\string_M\string_T2}{}
\markboth{\textcolor{darkblue}{\textbf{\ipa{ʈʂʰæ˧˥}}} \textsubscript{2}}{}
\textcolor{teal}{\zh{量词}} \hspace{4pt} \zh{声调类:} MH.
\zh{量词:代、世、辈、世代。} \textcolor{Sepia}{\selectlanguage{english}Classifier for generations.} \textcolor{PineGreen}{\selectlanguage{french}Classificateur des générations.} 
\lhead{\firstmark}
\rhead{\botmark}

\subsection{\hspace{-0.5cm} {\Large \textcolor{darkblue}{\textbf{\ipa{ʈʂʰe˧\textsubscript{b}}}}}\hspace{0.5cm}[\kern2pt{\textcolor{darkblue}{\textbf{\ipa{ʈʂʰe˥}}}}\kern2pt]} \hypertarget{t`s`\string_he\string_Mb1}{}
\markboth{\textcolor{darkblue}{\textbf{\ipa{ʈʂʰe˧\textsubscript{b}}}}}{}
\textcolor{teal}{\zh{动词}} \hspace{4pt} \zh{声调类:} M\textsubscript{b}.
\zh{伸(伸手)。} \textcolor{Sepia}{\selectlanguage{english}To stretch (one's hand...).} \textcolor{PineGreen}{\selectlanguage{french}Tendre, étendre (la main).}  ¶ \textcolor{darkblue}{\textbf{\ipa{le˧-ʈʂʰe˧-ze˧}}} \zh{\mytextsc{accomp} \string_ \mytextsc{pfv}} \textcolor{Sepia}{\selectlanguage{english}\mytextsc{accomp} \string_ \mytextsc{pfv}} \textcolor{PineGreen}{\selectlanguage{french}\mytextsc{accomp} \string_ \mytextsc{pfv}}  
 ¶ \textcolor{darkblue}{\textbf{\ipa{mv̩˩tɕo˧ ʈʂʰe˧}}} \zh{向下伸展} \textcolor{Sepia}{\selectlanguage{english}to stretch down} \textcolor{PineGreen}{\selectlanguage{french}étendre vers le bas}  
 ¶ \textcolor{darkblue}{\textbf{\ipa{lo˩qʰwɤ˧ | ə˩pʰo˩ ʈʂʰe˩˥}}} \zh{手伸到外边} \textcolor{Sepia}{\selectlanguage{english}to strech one's hand outside (e.g. out the window)} \textcolor{PineGreen}{\selectlanguage{french}étendre la main à l'extérieur (par une fenêtre)}  
 ¶ \textcolor{darkblue}{\textbf{\ipa{tso˧\textasciitilde{}tso˧ ʈʂʰe˧}}} \zh{伸出一个东西,如:从车窗里伸出一个棍子} \textcolor{Sepia}{\selectlanguage{english}to extend something, to stick out something (e.g. to extend a cane out the window of a car)} \textcolor{PineGreen}{\selectlanguage{french}étendre quelque chose: par exemple, faire sortir un bâton par une fenêtre}  

\lhead{\firstmark}
\rhead{\botmark}

\subsection{\hspace{-0.5cm} {\Large \textcolor{darkblue}{\textbf{\ipa{ʈʂʰe˧\textasciitilde{}ʈʂʰe˧}}}}\hspace{0.5cm}[\kern2pt{\textcolor{darkblue}{\textbf{\ipa{ʈʂʰe˧ʈʂʰe˧}}}}\kern2pt]} \hypertarget{t`s`\string_he\string_M~t`s`\string_he\string_M1}{}
\markboth{\textcolor{darkblue}{\textbf{\ipa{ʈʂʰe˧\textasciitilde{}ʈʂʰe˧}}}}{}
\textcolor{teal}{\zh{量词}} \hspace{4pt} \zh{声调类:} M.
\zh{量词:一面(墙)。} \textcolor{Sepia}{\selectlanguage{english}Classifiers for walls, i.e. the width of a room: for instance, a cupboard can be described as extending over an entire wall, i.e. occupying the entire width of a room.} \textcolor{PineGreen}{\selectlanguage{french}Classificateur pour les murs, et donc pour la largeur de toute une pièce: un buffet/placard occupe toute la largeur d'une pièce, par exemple.} 
\lhead{\firstmark}
\rhead{\botmark}

\subsection{\hspace{-0.5cm} {\Large \textcolor{darkblue}{\textbf{\ipa{ʈʂʰe˩ko˧}}}}\hspace{0.5cm}[\kern2pt{\textcolor{darkblue}{\textbf{\ipa{ʈʂʰe˩ko˥}}}}\kern2pt]} \hypertarget{t`s`\string_he\string_Bko\string_M1}{}
\markboth{\textcolor{darkblue}{\textbf{\ipa{ʈʂʰe˩ko˧}}}}{}
\textcolor{teal}{\zh{动词}} \hspace{4pt} \zh{声调类:} LM.
\zh{成功(汉语借词)。} \textcolor{Sepia}{\selectlanguage{english}To succeed.} \textcolor{PineGreen}{\selectlanguage{french}Réussir.}  \zh{【借词】} \zh{成功}

\lhead{\firstmark}
\rhead{\botmark}

\subsection{\hspace{-0.5cm} {\Large \textcolor{darkblue}{\textbf{\ipa{ʈʂʰɤ˧tsɯ˧}}}}\hspace{0.5cm}[\kern2pt{\textcolor{darkblue}{\textbf{\ipa{ʈʂʰɤ˧tsɯ˧}}}}\kern2pt]} \hypertarget{t`s`\string_h7\string_MtsM\string_M1}{}
\markboth{\textcolor{darkblue}{\textbf{\ipa{ʈʂʰɤ˧tsɯ˧}}}}{}
\textcolor{teal}{\zh{名词}} \hspace{4pt} \zh{声调类:} M.
\zh{车子(汉语借词)。} \textcolor{Sepia}{\selectlanguage{english}Car, bus.} \textcolor{PineGreen}{\selectlanguage{french}Voiture, automobile, car.}  \zh{【借词】} \zh{车子}

\lhead{\firstmark}
\rhead{\botmark}

\subsection{\hspace{-0.5cm} {\Large \textcolor{darkblue}{\textbf{\ipa{ʈʂʰɤ˧zo˥-ʈʂʰɤ˩mv̩˩}}}}\hspace{0.5cm}[\kern2pt{\textcolor{darkblue}{\textbf{\ipa{ʈʂʰɤ˧zo˥ʈʂʰɤ˩mv̩˩}}}}\kern2pt]} \hypertarget{t`s`\string_h7\string_Mzo\string_T-t`s`\string_h7\string_Bmv\string_=\string_B1}{}
\markboth{\textcolor{darkblue}{\textbf{\ipa{ʈʂʰɤ˧zo˥-ʈʂʰɤ˩mv̩˩}}}}{}
\textcolor{teal}{\zh{名词}} \hspace{4pt} \zh{声调类:} H\#-L.
\zh{私生子:没有名分的孩子、不明来路。} \textcolor{Sepia}{\selectlanguage{english}Love child.} \textcolor{PineGreen}{\selectlanguage{french}Enfant naturel.}  ¶ \textcolor{darkblue}{\textbf{\ipa{ə˧dɑ˥ | ɲi˩-ɲi˥ | mɤ˧-sɯ˥ | ʈʂʰɯ˧-v̩˧, | ʈʂʰɤ˧zo˥-ʈʂʰɤ˩mv̩˩ mv̩˩ʈʂæ˩.}}} \zh{一个人不知道他父亲是谁,就称作“私生子”。} \textcolor{Sepia}{\selectlanguage{english}Someone who does not know who his father is, is called a “love child”.} \textcolor{PineGreen}{\selectlanguage{french}Celui qui ne sait pas qui est son père, on l'appelle “enfant naturel”.}  
 ¶ \textcolor{darkblue}{\textbf{\ipa{ə˧ʝi˧-ʂɯ˥ʝi˩, | ʈʂʰɤ˧zo˥-ʈʂʰɤ˩mv̩˩ ʐɤ˩-hĩ˩-lɑ˩ ɲi˩!}}} \zh{过去,大家会公开把“私生子”养大,不会大惊小怪的!} \textcolor{Sepia}{\selectlanguage{english}In the past, one used to bring up love children, and that was that! / In the past, one used to bring up love children without making any fuss!} \textcolor{PineGreen}{\selectlanguage{french}Autrefois, les enfants naturels, on les élevait et voilà tout!/on les élevait tout simplement, sans faire de difficultés!}  

\lhead{\firstmark}
\rhead{\botmark}

\subsection{\hspace{-0.5cm} {\Large \textcolor{darkblue}{\textbf{\ipa{ʈʂʰɤ˩\textsubscript{a}}}}}\hspace{0.5cm}[\kern2pt{\textcolor{darkblue}{\textbf{\ipa{ʈʂʰɤ˥}}}}\kern2pt]} \hypertarget{t`s`\string_h7\string_Ba1}{}
\markboth{\textcolor{darkblue}{\textbf{\ipa{ʈʂʰɤ˩\textsubscript{a}}}}}{}
\textcolor{teal}{\zh{动词}} \hspace{4pt} \zh{声调类:} L\textsubscript{a}.
\zh{分。} \textcolor{Sepia}{\selectlanguage{english}To share (several people share something among themselves; someone shares out something).} \textcolor{PineGreen}{\selectlanguage{french}Répartir, diviser.}  ¶ \textcolor{darkblue}{\textbf{\ipa{ɖɯ˧-v̩˧ ɖɯ˧-kʰwɤ˥ | le˧-ʈʂʰɤ˧\textasciitilde{}ʈʂʰɤ˥}}} \zh{平分} \textcolor{Sepia}{\selectlanguage{english}to share: one piece for each person} \textcolor{PineGreen}{\selectlanguage{french}répartir un morceau par personne}  

\lhead{\firstmark}
\rhead{\botmark}

\subsection{\hspace{-0.5cm} {\Large \textcolor{darkblue}{\textbf{\ipa{ʈʂʰɤ˩ho˥}}}}\hspace{0.5cm}[\kern2pt{\textcolor{darkblue}{\textbf{\ipa{ʈʂʰɤ˩ho˥}}}}\kern2pt]} \hypertarget{t`s`\string_h7\string_Bho\string_T1}{}
\markboth{\textcolor{darkblue}{\textbf{\ipa{ʈʂʰɤ˩ho˥}}}}{}
\textcolor{teal}{\zh{名词}} \hspace{4pt} \zh{声调类:} LH.
\zh{水壶(汉语借词:茶壶)。} \textcolor{Sepia}{\selectlanguage{english}Kettle.} \textcolor{PineGreen}{\selectlanguage{french}Bouilloire.}  \zh{【借词】}\zh{茶壶}
 \zh{量词}: \textcolor{darkblue}{\textbf{\ipa{ɭɯ˧}}} 
\lhead{\firstmark}
\rhead{\botmark}

\subsection{\hspace{-0.5cm} {\Large \textcolor{darkblue}{\textbf{\ipa{ʈʂʰɤ˩kɤ˧}}}}\hspace{0.5cm}[\kern2pt{\textcolor{darkblue}{\textbf{\ipa{ʈʂʰɤ˩kɤ˥}}}}\kern2pt]} \hypertarget{t`s`\string_h7\string_Bk7\string_M1}{}
\markboth{\textcolor{darkblue}{\textbf{\ipa{ʈʂʰɤ˩kɤ˧}}}}{}
\textcolor{teal}{\zh{名词}} \hspace{4pt} \zh{声调类:} LM.
\zh{缸子,杯子。} \textcolor{Sepia}{\selectlanguage{english}Goblet.} \textcolor{PineGreen}{\selectlanguage{french}Gobelet (avec anse); en métal ou poterie.}  \zh{量词}: \textcolor{darkblue}{\textbf{\ipa{ɭɯ˧}}} 
\lhead{\firstmark}
\rhead{\botmark}

\subsection{\hspace{-0.5cm} {\Large \textcolor{darkblue}{\textbf{\ipa{ʈʂʰɤ˩qo˧}}}}\hspace{0.5cm}[\kern2pt{\textcolor{darkblue}{\textbf{\ipa{ʈʂʰɤ˩qo˥}}}}\kern2pt]} \hypertarget{t`s`\string_h7\string_Bqo\string_M1}{}
\markboth{\textcolor{darkblue}{\textbf{\ipa{ʈʂʰɤ˩qo˧}}}}{}
\textcolor{teal}{\zh{名词}} \hspace{4pt} \zh{声调类:} LM.
\zh{关注、关心。} \textcolor{Sepia}{\selectlanguage{english}Attention, interest, care.} \textcolor{PineGreen}{\selectlanguage{french}Attention, intérêt.}  ¶ \textcolor{darkblue}{\textbf{\ipa{ʈʂʰɤ˩qo˧ kʰɯ˧˥}}} \zh{关心、关注} \textcolor{Sepia}{\selectlanguage{english}to pay attention to, to care for} \textcolor{PineGreen}{\selectlanguage{french}se soucier de, prêter attention à}  
 ¶ \textcolor{darkblue}{\textbf{\ipa{ʈʂʰɤ˩qo˧ | ɖwæ˧˥ | tʰi˧-kʰɯ˧˥}}} \zh{很关心、很关注} \textcolor{Sepia}{\selectlanguage{english}to pay great attention to, to care greatly for (e.g. a grandmother paying great attention to a little child's feeding)} \textcolor{PineGreen}{\selectlanguage{french}prêter une grande attention à, être très attentif à (ex.: une grand-mère très attentive à l'alimentation d'un petit enfant)}  
 ¶ \textcolor{darkblue}{\textbf{\ipa{ʈʂʰɤ˩qo˧ | mɤ˧-kʰɯ˧˥}}} \zh{不关心、不关注} \textcolor{Sepia}{\selectlanguage{english}to pay little attention to, not to care for} \textcolor{PineGreen}{\selectlanguage{french}être insensible à, ne pas prêter attention à}  
 \zh{量词}: \textcolor{darkblue}{\textbf{\ipa{kʰwɤ˥}}} 
\lhead{\firstmark}
\rhead{\botmark}

\subsection{\hspace{-0.5cm} {\Large \textcolor{darkblue}{\textbf{\ipa{ʈʂʰɤ˩tɕʰɯ˩}}}}\hspace{0.5cm}[\kern2pt{\textcolor{darkblue}{\textbf{\ipa{ʈʂʰɤ˩tɕʰɯ˩˥}}}}\kern2pt]} \hypertarget{t`s`\string_h7\string_Bts£\string_hM\string_B1}{}
\markboth{\textcolor{darkblue}{\textbf{\ipa{ʈʂʰɤ˩tɕʰɯ˩}}}}{}
\textcolor{teal}{\zh{形容词}} \hspace{4pt} \zh{声调类:} L.
\zh{利害,值得崇拜。} \textcolor{Sepia}{\selectlanguage{english}Admirable, with high qualities.} \textcolor{PineGreen}{\selectlanguage{french}Très doué, très calé, possédant des qualités admirables (par ex.: personne très savante).} 
\lhead{\firstmark}
\rhead{\botmark}

\subsection{\hspace{-0.5cm} {\Large \textcolor{darkblue}{\textbf{\ipa{ʈʂʰɤ˩\textasciitilde{}ʈʂʰɤ˧˥}}}}\hspace{0.5cm}[\kern2pt{\textcolor{darkblue}{\textbf{\ipa{ʈʂʰɤ˧ʈʂʰɤ˧˥}}}}\kern2pt]} \hypertarget{t`s`\string_h7\string_B~t`s`\string_h7\string_M\string_T1}{}
\markboth{\textcolor{darkblue}{\textbf{\ipa{ʈʂʰɤ˩\textasciitilde{}ʈʂʰɤ˧˥}}}}{}
\textcolor{teal}{\zh{动词}} \hspace{4pt} \zh{声调类:} MH.
\zh{抚摸。} \textcolor{Sepia}{\selectlanguage{english}To feel, to touch, to stroke.} \textcolor{PineGreen}{\selectlanguage{french}Toucher.}  ¶ \textcolor{darkblue}{\textbf{\ipa{ʈʂʰɤ˩ʈʂʰɤ˧ mɤ˥-tʰɑ˩!}}} \zh{禁止触碰!} \textcolor{Sepia}{\selectlanguage{english}One must not touch!} \textcolor{PineGreen}{\selectlanguage{french}il ne faut pas toucher!}  
 ¶ \textcolor{darkblue}{\textbf{\ipa{tʰɑ˧-ʈʂʰɤ˩ʈʂʰɤ˩!}}} \zh{别碰!} \textcolor{Sepia}{\selectlanguage{english}Do not touch!} \textcolor{PineGreen}{\selectlanguage{french}ne touchez pas!}  
 ¶ \textcolor{darkblue}{\textbf{\ipa{tso˧\textasciitilde{}tso˧ ʈʂʰɤ˥ʈʂʰɤ˩}}} \zh{抚摸东西} \textcolor{Sepia}{\selectlanguage{english}to touch something} \textcolor{PineGreen}{\selectlanguage{french}toucher quelque chose}  

\lhead{\firstmark}
\rhead{\botmark}

\subsection{\hspace{-0.5cm} {\Large \textcolor{darkblue}{\textbf{\ipa{ʈʂʰo˥}}}}\hspace{0.5cm}[\kern2pt{\textcolor{darkblue}{\textbf{\ipa{ʈʂʰo˥}}}}\kern2pt]} \hypertarget{t`s`\string_ho\string_T1}{}
\markboth{\textcolor{darkblue}{\textbf{\ipa{ʈʂʰo˥}}}}{}
\textcolor{teal}{\zh{动词}} \hspace{4pt} \zh{声调类:} H.
\zh{拜(神)。} \textcolor{Sepia}{\selectlanguage{english}To pray (to a god): to recite prayers, to chant prayers.} \textcolor{PineGreen}{\selectlanguage{french}Prier (une divinité): réciter des prières, psalmodier des prières.}  ¶ \textcolor{darkblue}{\textbf{\ipa{ʈʂʰo˧do˩ ʈʂʰo˩}}} \zh{祭祀祖先} \textcolor{Sepia}{\selectlanguage{english}to pray to the spirit of the home} \textcolor{PineGreen}{\selectlanguage{french}prier l'esprit du foyer}  
 ¶ \textcolor{darkblue}{\textbf{\ipa{hĩ˧-mo˥, | zo˩qo˧ ʂɯ˧, | zo˩qo˧-ɳɯ˧ ʈʂʰo˧-zo˧!}}} \zh{要在家人去世地点进行祭拜!} \textcolor{Sepia}{\selectlanguage{english}Deceased members of the family are honoured at the place where they passed away!} \textcolor{PineGreen}{\selectlanguage{french}Les membres décédés de la famille, c'est à l'endroit où ils sont morts qu'on leur rend hommage!}  

\lhead{\firstmark}
\rhead{\botmark}

\subsection{\hspace{-0.5cm} {\Large \textcolor{darkblue}{\textbf{\ipa{ʈʂʰo˧\textsubscript{b}}}}}\hspace{0.5cm}[\kern2pt{\textcolor{darkblue}{\textbf{\ipa{ʈʂʰo˥}}}}\kern2pt]} \hypertarget{t`s`\string_ho\string_Mb1}{}
\markboth{\textcolor{darkblue}{\textbf{\ipa{ʈʂʰo˧\textsubscript{b}}}}}{}
\textcolor{teal}{\zh{动词}} \hspace{4pt} \zh{声调类:} M\textsubscript{b}.
\zh{朗读。} \textcolor{Sepia}{\selectlanguage{english}To read aloud.} \textcolor{PineGreen}{\selectlanguage{french}Lire à haute voix.}  ¶ \textcolor{darkblue}{\textbf{\ipa{le˧-ʈʂʰo˧-ze˧}}} \zh{朗读了} \textcolor{Sepia}{\selectlanguage{english}\mytextsc{accomp} \string_ \mytextsc{pfv}} \textcolor{PineGreen}{\selectlanguage{french}\mytextsc{accomp} \string_ \mytextsc{pfv}}  
 ¶ \textcolor{darkblue}{\textbf{\ipa{le˧-ʈʂʰo˧-le˧-se˩}}} \zh{朗读完了。} \textcolor{Sepia}{\selectlanguage{english}(I) have finished reading aloud} \textcolor{PineGreen}{\selectlanguage{french}(j'ai) fini de lire}  
 ¶ \textcolor{darkblue}{\textbf{\ipa{tʰæ˧ɻæ˩ ʈʂʰo˩}}} \zh{朗读一本书} \textcolor{Sepia}{\selectlanguage{english}to read a book aloud} \textcolor{PineGreen}{\selectlanguage{french}lire un livre}  
 ¶ \textcolor{darkblue}{\textbf{\ipa{ʈʂʰo˧\textasciitilde{}ʈʂʰo˧}}} \zh{\mytextsc{重叠}} \textcolor{Sepia}{\selectlanguage{english}\mytextsc{red}} \textcolor{PineGreen}{\selectlanguage{french}\mytextsc{red}}  

\lhead{\firstmark}
\rhead{\botmark}

\subsection{\hspace{-0.5cm} {\Large \textcolor{darkblue}{\textbf{\ipa{ʈʂʰo˧bɤ\#˥}}}}\hspace{0.5cm}[\kern2pt{\textcolor{darkblue}{\textbf{\ipa{ʈʂʰo˩bɤ˩˥}}}}\kern2pt]} \hypertarget{t`s`\string_ho\string_Mb7\#\string_T1}{}
\markboth{\textcolor{darkblue}{\textbf{\ipa{ʈʂʰo˧bɤ\#˥}}}}{}
\textcolor{teal}{\zh{名词}} \hspace{4pt} \zh{声调类:} \#H.
\zh{男上衣。} \textcolor{Sepia}{\selectlanguage{english}Masculine clothing worn on special occasions.} \textcolor{PineGreen}{\selectlanguage{french}Vêtement masculin, que les hommes portaient à partir de 13 ans: sorte de veste serrée à la ceinture, qu'on portait sur la chemise lors des grandes occasions: mariage, invitations….}  \zh{量词}: \textcolor{darkblue}{\textbf{\ipa{ɭɯ˧˥}}} 
\lhead{\firstmark}
\rhead{\botmark}

\subsection{\hspace{-0.5cm} {\Large \textcolor{darkblue}{\textbf{\ipa{ʈʂʰo˧bv̩˩}}}}\hspace{0.5cm}[\kern2pt{\textcolor{darkblue}{\textbf{\ipa{ʈʂʰo˧bv̩˧}}}}\kern2pt]} \hypertarget{t`s`\string_ho\string_Mbv\string_=\string_B1}{}
\markboth{\textcolor{darkblue}{\textbf{\ipa{ʈʂʰo˧bv̩˩}}}}{}
\textcolor{teal}{\zh{名词}} \hspace{4pt} \zh{声调类:} L\#.
\zh{菖蒲。} \textcolor{Sepia}{\selectlanguage{english}Calamus, sweet flag, bitterroot, \textit{Acorus calamus} (a tall wetland plant).} \textcolor{PineGreen}{\selectlanguage{french}Acore odorant, jonc odorant, \textit{Acorus calamus}: plante herbacée aquatique, pérenne, rhizomateuse.}  \zh{量词}: \textcolor{darkblue}{\textbf{\ipa{dzi˩}}} 
\lhead{\firstmark}
\rhead{\botmark}

\subsection{\hspace{-0.5cm} {\Large \textcolor{darkblue}{\textbf{\ipa{ʈʂʰo˧do˩}}}}\hspace{0.5cm}[\kern2pt{\textcolor{darkblue}{\textbf{\ipa{ʈʂʰo˧do˧}}}}\kern2pt]} \hypertarget{t`s`\string_ho\string_Mdo\string_B1}{}
\markboth{\textcolor{darkblue}{\textbf{\ipa{ʈʂʰo˧do˩}}}}{}
\textcolor{teal}{\zh{名词}} \hspace{4pt} \zh{声调类:} L\#.
\zh{火塘上面祖先灵位。} \textcolor{Sepia}{\selectlanguage{english}Small eminence next to the hearth, symbolising the ancestors, on top of which some food is offered at the beginning of each meal.} \textcolor{PineGreen}{\selectlanguage{french}Petite éminence juste à côté du foyer, en contrebas de l'autel où on offre des cadeaux aux ancêtres; c'est sur cette petite éminence qu'on dépose un peu de nourriture au début de chaque repas, en offrande aux ancêtres.}  \zh{量词}: \textcolor{darkblue}{\textbf{\ipa{ɭɯ˧}}} 
\lhead{\firstmark}
\rhead{\botmark}

\subsection{\hspace{-0.5cm} {\Large \textcolor{darkblue}{\textbf{\ipa{ʈʂʰo˧lo\#˥}}}}\hspace{0.5cm}[\kern2pt{\textcolor{darkblue}{\textbf{\ipa{ʈʂʰo˧lo˧}}}}\kern2pt]} \hypertarget{t`s`\string_ho\string_Mlo\#\string_T1}{}
\markboth{\textcolor{darkblue}{\textbf{\ipa{ʈʂʰo˧lo\#˥}}}}{}
\textcolor{teal}{\zh{名词}} \hspace{4pt} \zh{声调类:} \#H.
\zh{平底大锅(直径大概半米),用来煎洋芋饼等等。} \textcolor{Sepia}{\selectlanguage{english}Frying pan (large, with flat bottom).} \textcolor{PineGreen}{\selectlanguage{french}Grande poêle à fond plat, diamètre un peu supérieur à 50 cm, pour frire des aliments (galettes de pomme de terre, fèves).}  \zh{量词}: \textcolor{darkblue}{\textbf{\ipa{ɭɯ˧}}} 
\lhead{\firstmark}
\rhead{\botmark}

\subsection{\hspace{-0.5cm} {\Large \textcolor{darkblue}{\textbf{\ipa{ʈʂʰɻ̍˧}}}}\hspace{0.5cm}[\kern2pt{\textcolor{darkblue}{\textbf{\ipa{ʈʂʰɻ̍˥}}}}\kern2pt]} \hypertarget{t`s`\string_hr£`̍\string_M1}{}
\markboth{\textcolor{darkblue}{\textbf{\ipa{ʈʂʰɻ̍˧}}}}{}
\textcolor{teal}{\zh{名词}} \hspace{4pt} \zh{声调类:} M.
\zh{铧头,犁铧。} \textcolor{Sepia}{\selectlanguage{english}Ploughshare.} \textcolor{PineGreen}{\selectlanguage{french}Soc de l'araire.}  ¶ \textcolor{darkblue}{\textbf{\ipa{ʈʂʰɻ̍˧ ʈʂʰɯ˧-ɭɯ˧}}} \zh{这把铧头} \textcolor{Sepia}{\selectlanguage{english}\mytextsc{n}+\mytextsc{dem}+\mytextsc{clf}} \textcolor{PineGreen}{\selectlanguage{french}\mytextsc{n}+\mytextsc{dem}+\mytextsc{clf}}  
 \zh{量词}: \textcolor{darkblue}{\textbf{\ipa{ɭɯ˧}}} 
\lhead{\firstmark}
\rhead{\botmark}

\subsection{\hspace{-0.5cm} {\Large \textcolor{darkblue}{\textbf{\ipa{ʈʂʰɻ̍˧˥}}} \textsubscript{1}}\hspace{0.5cm}[\kern2pt{\textcolor{darkblue}{\textbf{\ipa{ʈʂʰɻ̍˧˥}}}}\kern2pt]} \hypertarget{t`s`\string_hr£`̍\string_M\string_T1}{}
\markboth{\textcolor{darkblue}{\textbf{\ipa{ʈʂʰɻ̍˧˥}}} \textsubscript{1}}{}
\textcolor{teal}{\zh{动词}} \hspace{4pt} \zh{声调类:} MH.
\zh{握 (握刀把)。} \textcolor{Sepia}{\selectlanguage{english}To grasp (e.g. a sword hilt).} \textcolor{PineGreen}{\selectlanguage{french}Empoigner, prendre en main, saisir, tenir fermement (ex.: couteau); serrer, crisper (le poing).}  ¶ \textcolor{darkblue}{\textbf{\ipa{sɯ˩tʰi˩˥ | (ɖɯ˧)-nɑ˧ | tʰi˧-ʈʂʰɻ̍˧˥ (+dʑo˩)}}} \zh{手里握刀} \textcolor{Sepia}{\selectlanguage{english}to grasp a knife} \textcolor{PineGreen}{\selectlanguage{french}empoigner un couteau}  
 ¶ \textcolor{darkblue}{\textbf{\ipa{ʈʂʰɻ̍˧ mɤ˧-bi˧!}}} \zh{我不要拿(刀)!} \textcolor{Sepia}{\selectlanguage{english}I won't grasp (this knife, ...)} \textcolor{PineGreen}{\selectlanguage{french}je ne veux pas empoigner/ pas question que j'empoigne (ce couteau,…)}  
 ¶ \textcolor{darkblue}{\textbf{\ipa{lo˩qʰwɤ˧ ʈʂʰɻ̍˩\textasciitilde{}ʈʂʰɻ̍˩ |}}} \zh{攥紧拳头} \textcolor{Sepia}{\selectlanguage{english}to tighten the fist} \textcolor{PineGreen}{\selectlanguage{french}serrer le poing}  

\lhead{\firstmark}
\rhead{\botmark}

\subsection{\hspace{-0.5cm} {\Large \textcolor{darkblue}{\textbf{\ipa{ʈʂʰɻ̍˧˥}}} \textsubscript{2}}\hspace{0.5cm}[\kern2pt{\textcolor{darkblue}{\textbf{\ipa{ʈʂʰɻ̍˧˥}}}}\kern2pt]} \hypertarget{t`s`\string_hr£`̍\string_M\string_T2}{}
\markboth{\textcolor{darkblue}{\textbf{\ipa{ʈʂʰɻ̍˧˥}}} \textsubscript{2}}{}
\textcolor{teal}{\zh{名词}} \hspace{4pt} \zh{声调类:} MH.
\zh{肺。} \textcolor{Sepia}{\selectlanguage{english}Lung.} \textcolor{PineGreen}{\selectlanguage{french}Poumon.}  \zh{量词}: \textcolor{darkblue}{\textbf{\ipa{ɭɯ˧}}} 
\lhead{\firstmark}
\rhead{\botmark}

\subsection{\hspace{-0.5cm} {\Large \textcolor{darkblue}{\textbf{\ipa{ʈʂʰɻ̍˧˥\textsubscript{a}}}}}\hspace{0.5cm}[\kern2pt{\textcolor{darkblue}{\textbf{\ipa{ʈʂʰɻ̍˧˥}}}}\kern2pt]} \hypertarget{t`s`\string_hr£`̍\string_M\string_Ta1}{}
\markboth{\textcolor{darkblue}{\textbf{\ipa{ʈʂʰɻ̍˧˥\textsubscript{a}}}}}{}
\textcolor{teal}{\zh{量词}} \hspace{4pt} \zh{声调类:} MH\textsubscript{a}.
\zh{量词:团,掐。指的是一只手里能拿的量,压成团,如:手里拿煮熟的粮食,压成饭团。} \textcolor{Sepia}{\selectlanguage{english}Classifier for handfuls / balls: loose substance shaped into ball form by compressing it in the hand, for instance a handful of cooked cereals.} \textcolor{PineGreen}{\selectlanguage{french}Classificateur des boules/poignées: la quantité que l'on compacte en la serrant dans une main, par exemple une poignée de céréale cuite qu'on compresse en boule.} 
\lhead{\firstmark}
\rhead{\botmark}

\subsection{\hspace{-0.5cm} {\Large \textcolor{darkblue}{\textbf{\ipa{ʈʂʰɯ˥}}} \textsubscript{1}}\hspace{0.5cm}[\kern2pt{\textcolor{darkblue}{\textbf{\ipa{ʈʂʰɯ˧˥}}}}\kern2pt]} \hypertarget{t`s`\string_hM\string_T1}{}
\markboth{\textcolor{darkblue}{\textbf{\ipa{ʈʂʰɯ˥}}} \textsubscript{1}}{}
\textcolor{teal}{\zh{代词}} \hspace{4pt} \zh{声调类:} \#H.
\zh{这\mytextsc{指示}.近指。} \textcolor{Sepia}{\selectlanguage{english}This; proximal demonstrative.} \textcolor{PineGreen}{\selectlanguage{french}Démonstratif proximal, qui forme un couple avec le démonstratif distal.}  ¶ \textcolor{darkblue}{\textbf{\ipa{ʈʂʰɯ˧ ɲi˥!}}} \zh{是这个! / 对的!} \textcolor{Sepia}{\selectlanguage{english}This is it! / That's it! / That's right!} \textcolor{PineGreen}{\selectlanguage{french}C'est ça!}  
 ¶ \textcolor{darkblue}{\textbf{\ipa{ʈʂʰɯ˧-v̩\#˥}}} \zh{这个} \textcolor{Sepia}{\selectlanguage{english}this one} \textcolor{PineGreen}{\selectlanguage{french}celui-ci (\mytextsc{dem}\string_prox-\mytextsc{clf}.individu)}  
 ¶ \textcolor{darkblue}{\textbf{\ipa{ʈʂʰɯ˧=ɻæ˥}}} \zh{这些} \textcolor{Sepia}{\selectlanguage{english}these things} \textcolor{PineGreen}{\selectlanguage{french}\mytextsc{pl}: ces choses-là}  
\zh{~【参考】~} \hyperlink{}{\textcolor{darkblue}{\textbf{\ipa{ʈʂʰɯ˥}}} \textsubscript{2}} \zh{~【参考】~} \textcolor{darkblue}{\textbf{\ipa{-ʈʂʰɯ˥}}} 
\lhead{\firstmark}
\rhead{\botmark}

\subsection{\hspace{-0.5cm} {\Large \textcolor{darkblue}{\textbf{\ipa{ʈʂʰɯ˥}}} \textsubscript{2}}\hspace{0.5cm}[\kern2pt{\textcolor{darkblue}{\textbf{\ipa{ʈʂʰɯ˥}}}}\kern2pt]} \hypertarget{t`s`\string_hM\string_T2}{}
\markboth{\textcolor{darkblue}{\textbf{\ipa{ʈʂʰɯ˥}}} \textsubscript{2}}{}
\textcolor{teal}{\zh{代词}} \hspace{4pt} \zh{声调类:} \#H.
\zh{他。} \textcolor{Sepia}{\selectlanguage{english}3rd person singular pronoun.} \textcolor{PineGreen}{\selectlanguage{french}Pronom de troisième personne du singulier.}  ¶ \textcolor{darkblue}{\textbf{\ipa{ʈʂʰɯ˧ ɲi˥!}}} \zh{是他!} \textcolor{Sepia}{\selectlanguage{english}That's her/him!} \textcolor{PineGreen}{\selectlanguage{french}C'est elle/lui!}  
\zh{~【参考】~} \hyperlink{}{\textcolor{darkblue}{\textbf{\ipa{ʈʂʰɯ˥}}} \textsubscript{1}} \zh{~【参考】~} \textcolor{darkblue}{\textbf{\ipa{-ʈʂʰɯ˥}}} 
\lhead{\firstmark}
\rhead{\botmark}

\subsection{\hspace{-0.5cm} {\Large \textcolor{darkblue}{\textbf{\ipa{ʈʂʰɯ˧}}}}\hspace{0.5cm}[\kern2pt{\textcolor{darkblue}{\textbf{\ipa{ʈʂʰɯ˥}}}}\kern2pt]} \hypertarget{t`s`\string_hM\string_M1}{}
\markboth{\textcolor{darkblue}{\textbf{\ipa{ʈʂʰɯ˧}}}}{}
\textcolor{teal}{\zh{后缀}} \hspace{4pt} \zh{声调类:} M.
\zh{\mytextsc{主题(°指示}.近指)。} \textcolor{Sepia}{\selectlanguage{english}Topic marker; grammaticalized from the proximal demonstrative.} \textcolor{PineGreen}{\selectlanguage{french}Focalisateur; grammaticalisé à partir du démonstratif proximal.} \zh{~【参考】~} \hyperlink{}{\textcolor{darkblue}{\textbf{\ipa{ʈʂʰɯ˥}}} \textsubscript{1}} \zh{~【参考】~} \hyperlink{}{\textcolor{darkblue}{\textbf{\ipa{ʈʂʰɯ˥}}} \textsubscript{2}} 
\lhead{\firstmark}
\rhead{\botmark}

\subsection{\hspace{-0.5cm} {\Large \textcolor{darkblue}{\textbf{\ipa{ʈʂʰɯ˧-gɤ˧}}}}\hspace{0.5cm}[\kern2pt{\textcolor{darkblue}{\textbf{\ipa{xxxx non-correspondance entre le nombre de morphèmes et le nombre de tons de morphèmes}}}}\kern2pt]} \hypertarget{t`s`\string_hM\string_M-g7\string_M1}{}
\markboth{\textcolor{darkblue}{\textbf{\ipa{ʈʂʰɯ˧-gɤ˧}}}}{}
\textcolor{teal}{\zh{助词}} \hspace{4pt} \zh{声调类:} M.
\zh{这里。} \textcolor{Sepia}{\selectlanguage{english}Here.} \textcolor{PineGreen}{\selectlanguage{french}Ici, à cet endroit-ci.} \zh{~【参考】~} \hyperlink{}{\textcolor{darkblue}{\textbf{\ipa{ʈʂʰɯ˧gi\#˥}}}} 
\lhead{\firstmark}
\rhead{\botmark}

\subsection{\hspace{-0.5cm} {\Large \textcolor{darkblue}{\textbf{\ipa{ʈʂʰɯ˧gi\#˥}}}}\hspace{0.5cm}[\kern2pt{\textcolor{darkblue}{\textbf{\ipa{ʈʂʰɯ˧gi˧}}}}\kern2pt]} \hypertarget{t`s`\string_hM\string_Mgi\#\string_T1}{}
\markboth{\textcolor{darkblue}{\textbf{\ipa{ʈʂʰɯ˧gi\#˥}}}}{}
\textcolor{teal}{\zh{助词}} \hspace{4pt} \zh{声调类:} \#H.
\zh{这边。} \textcolor{Sepia}{\selectlanguage{english}Here.} \textcolor{PineGreen}{\selectlanguage{french}Ici, à cet endroit-ci.} \zh{~【参考】~} \hyperlink{}{\textcolor{darkblue}{\textbf{\ipa{ʈʂʰɯ˧-gɤ˧}}}} 
\lhead{\firstmark}
\rhead{\botmark}

\subsection{\hspace{-0.5cm} {\Large \textcolor{darkblue}{\textbf{\ipa{ʈʂʰɯ˧ne˧-ʝi˥}}}}\hspace{0.5cm}[\kern2pt{\textcolor{darkblue}{\textbf{\ipa{xxxx non-correspondance entre le nombre de morphèmes et le nombre de tons de morphèmes}}}}\kern2pt]} \hypertarget{t`s`\string_hM\string_Mne\string_M-j££i\string_T1}{}
\markboth{\textcolor{darkblue}{\textbf{\ipa{ʈʂʰɯ˧ne˧-ʝi˥}}}}{}
\textcolor{teal}{\zh{助词}} \hspace{4pt} \zh{声调类:} MH\#.
\zh{这样,这么。} \textcolor{Sepia}{\selectlanguage{english}Thus, in this way.} \textcolor{PineGreen}{\selectlanguage{french}Ainsi, de cette façon (adverbe de manière).}  ¶ \textcolor{darkblue}{\textbf{\ipa{ʈʂʰɯ˧ne˧-ʝi˥ | le˧-ʐwɤ˩!}}} \zh{是这样讲的!} \textcolor{Sepia}{\selectlanguage{english}This is how it's said!} \textcolor{PineGreen}{\selectlanguage{french}c'est comme ça qu'on dit!}  
 ¶ \textcolor{darkblue}{\textbf{\ipa{ʈʂʰɯ˧ne˧-ʝi˥ | le˧-pi˥!}}} \zh{是这样说的!} \textcolor{Sepia}{\selectlanguage{english}This is how it's said!} \textcolor{PineGreen}{\selectlanguage{french}c'est comme ça qu'on parle!}  
 ¶ \textcolor{darkblue}{\textbf{\ipa{ʈʂʰɯ˧ne˧-ʝi˥ | le˧-ʝi˥!}}} \zh{是这样做的!} \textcolor{Sepia}{\selectlanguage{english}This is how it's done!} \textcolor{PineGreen}{\selectlanguage{french}c'est comme ça qu'on fait!}  

\lhead{\firstmark}
\rhead{\botmark}

\subsection{\hspace{-0.5cm} {\Large \textcolor{darkblue}{\textbf{\ipa{ʈʂʰɯ˧qɑ˧}}}}\hspace{0.5cm}[\kern2pt{\textcolor{darkblue}{\textbf{\ipa{ʈʂʰɯ˧qɑ˧}}}}\kern2pt]} \hypertarget{t`s`\string_hM\string_MqA\string_M1}{}
\markboth{\textcolor{darkblue}{\textbf{\ipa{ʈʂʰɯ˧qɑ˧}}}}{}
\textcolor{teal}{\zh{助词}} \hspace{4pt} \zh{声调类:} .
\zh{一起。} \textcolor{Sepia}{\selectlanguage{english}Together.} \textcolor{PineGreen}{\selectlanguage{french}Ensemble.} 
\lhead{\firstmark}
\rhead{\botmark}

\subsection{\hspace{-0.5cm} {\Large \textcolor{darkblue}{\textbf{\ipa{ʈʂʰɯ˧-qo˧}}}}\hspace{0.5cm}[\kern2pt{\textcolor{darkblue}{\textbf{\ipa{xxxx non-correspondance entre le nombre de morphèmes et le nombre de tons de morphèmes}}}}\kern2pt]} \hypertarget{t`s`\string_hM\string_M-qo\string_M1}{}
\markboth{\textcolor{darkblue}{\textbf{\ipa{ʈʂʰɯ˧-qo˧}}}}{}
\textcolor{teal}{\zh{助词}} \hspace{4pt} \zh{声调类:} M.
\zh{这里。} \textcolor{Sepia}{\selectlanguage{english}Here.} \textcolor{PineGreen}{\selectlanguage{french}Ici, à cet endroit-ci.} 
\lhead{\firstmark}
\rhead{\botmark}

\subsection{\hspace{-0.5cm} {\Large \textcolor{darkblue}{\textbf{\ipa{ʈʂʰɯ˧tɕi˩}}}}\hspace{0.5cm}[\kern2pt{\textcolor{darkblue}{\textbf{\ipa{ʈʂʰɯ˧tɕi˩}}}}\kern2pt]} \hypertarget{t`s`\string_hM\string_Mts£i\string_B1}{}
\markboth{\textcolor{darkblue}{\textbf{\ipa{ʈʂʰɯ˧tɕi˩}}}}{}
\textcolor{teal}{\zh{代词}} \hspace{4pt} \zh{声调类:} L\#.
\zh{他们。} \textcolor{Sepia}{\selectlanguage{english}Third-person plural pronoun.} \textcolor{PineGreen}{\selectlanguage{french}Pronom de troisième personne pluriel.} 
\lhead{\firstmark}
\rhead{\botmark}

\subsection{\hspace{-0.5cm} {\Large \textcolor{darkblue}{\textbf{\ipa{ʈʂʰɯ˧=zɯ˩}}}}\hspace{0.5cm}[\kern2pt{\textcolor{darkblue}{\textbf{\ipa{ʈʂʰɯ˧zɯ˩}}}}\kern2pt]} \hypertarget{t`s`\string_hM\string_M=zM\string_B1}{}
\markboth{\textcolor{darkblue}{\textbf{\ipa{ʈʂʰɯ˧=zɯ˩}}}}{}
\textcolor{teal}{\zh{代词}} \hspace{4pt} \zh{声调类:} L\#.
\zh{他们两个。} \textcolor{Sepia}{\selectlanguage{english}Dual third-person pronoun: the two of them.} \textcolor{PineGreen}{\selectlanguage{french}Pronom de troisième personne duel: eux deux.} 
\lhead{\firstmark}
\rhead{\botmark}

\subsection{\hspace{-0.5cm} {\Large \textcolor{darkblue}{\textbf{\ipa{ʈʂʰv̩˩}}} \textsubscript{1}}\hspace{0.5cm}[\kern2pt{\textcolor{darkblue}{\textbf{\ipa{ʈʂʰv̩˧˥}}}}\kern2pt]} \hypertarget{t`s`\string_hv\string_=\string_B1}{}
\markboth{\textcolor{darkblue}{\textbf{\ipa{ʈʂʰv̩˩}}} \textsubscript{1}}{}
\textcolor{teal}{\zh{动词}} \hspace{4pt} \zh{声调类:} MH.
\zh{完成。} \textcolor{Sepia}{\selectlanguage{english}To complete, to finish.} \textcolor{PineGreen}{\selectlanguage{french}Achever, terminer.}  ¶ \textcolor{darkblue}{\textbf{\ipa{le˧-ʈʂʰv̩˩-se˩}}} \zh{完成了} \textcolor{Sepia}{\selectlanguage{english}\mytextsc{accomp} \string_ \mytextsc{cmpl}} \textcolor{PineGreen}{\selectlanguage{french}\mytextsc{accomp} \string_ \mytextsc{cmpl}}  
 ¶ \textcolor{darkblue}{\textbf{\ipa{tsʰi˧-ɲi˧-bv̩˧ | lo˧ | le˧-ʈʂʰv̩˩! | le˧-se˩-ze˩!}}} \zh{今天的工作完成了!就算完工了吧!} \textcolor{Sepia}{\selectlanguage{english}Today's work is completed! It's finished!} \textcolor{PineGreen}{\selectlanguage{french}Le travail d'aujourd'hui... on tourne la page! Il est fini!}  

\lhead{\firstmark}
\rhead{\botmark}

\subsection{\hspace{-0.5cm} {\Large \textcolor{darkblue}{\textbf{\ipa{ʈʂʰv̩˩}}} \textsubscript{2}}\hspace{0.5cm}[\kern2pt{\textcolor{darkblue}{\textbf{\ipa{ʈʂʰv̩˩˥}}}}\kern2pt]} \hypertarget{t`s`\string_hv\string_=\string_B2}{}
\markboth{\textcolor{darkblue}{\textbf{\ipa{ʈʂʰv̩˩}}} \textsubscript{2}}{}
\textcolor{teal}{\zh{动词}} \hspace{4pt} \zh{声调类:} L.
\zh{除开。} \textcolor{Sepia}{\selectlanguage{english}To set aside, to set apart, to distinguish.} \textcolor{PineGreen}{\selectlanguage{french}Mettre à part.}  ¶ \textcolor{darkblue}{\textbf{\ipa{gɤ˩-ʈʂʰv̩˧, | mv̩˩-ʈʂʰv̩˧-tsæ˩-ɲi˩}}} \zh{不算在里面、不算在一起} \textcolor{Sepia}{\selectlanguage{english}to set aside, to distinguish, not to mix} \textcolor{PineGreen}{\selectlanguage{french}laisser à part, distinguer, ne pas mettre ensemble, ne pas fourrer dans le même sac}  
 ¶ \textcolor{darkblue}{\textbf{\ipa{no˧-bv̩˧ | gɤ˩-ʈʂʰv̩˧! | njɤ˧-bv̩˧, | mv̩˩-ʈʂʰv̩˧!}}} \zh{你的算你的,我的算我的!} \textcolor{Sepia}{\selectlanguage{english}Your stuff belongs to you; and mine belongs to me!} \textcolor{PineGreen}{\selectlanguage{french}Ce qui est à toi est à toi; ce qui est à moi est à moi!}  

\lhead{\firstmark}
\rhead{\botmark}

\subsection{\hspace{-0.5cm} {\Large \textcolor{darkblue}{\textbf{\ipa{ʈʂʰv̩˧}}}}\hspace{0.5cm}[\kern2pt{\textcolor{darkblue}{\textbf{\ipa{ʈʂʰv̩˥}}}}\kern2pt]} \hypertarget{t`s`\string_hv\string_=\string_M1}{}
\markboth{\textcolor{darkblue}{\textbf{\ipa{ʈʂʰv̩˧}}}}{}
\textcolor{teal}{\zh{名词}} \hspace{4pt} \zh{声调类:} M.
\zh{早饭。} \textcolor{Sepia}{\selectlanguage{english}Breakfast.} \textcolor{PineGreen}{\selectlanguage{french}Repas du matin/ petit déjeuner.}  ¶ \textcolor{darkblue}{\textbf{\ipa{ʈʂʰv̩˧ dzɯ˧(-ze˩)}}} \zh{吃早饭} \textcolor{Sepia}{\selectlanguage{english}to have breakfast, to eat breakfast} \textcolor{PineGreen}{\selectlanguage{french}prendre le petit déjeuner}  
 ¶ \textcolor{darkblue}{\textbf{\ipa{bæ˧qʰæ˧ ʈʂʰv̩\#˥}}} \zh{丧礼早餐:参加火葬仪式的人留在去世的人家,一起吃一点早饭再回家。} \textcolor{Sepia}{\selectlanguage{english}The breakfast shared when coming back from the cremation ceremony. Guests stop at the house of the deceased, where they are offered breakfast before they set home.} \textcolor{PineGreen}{\selectlanguage{french}le petit déjeuner qu'on prend au retour de la cérémonie de crémation: revenant du lieu où a eu lieu la crémation, les invités, en nombre relativement restreint, font une pause dans la maison du défunt, où on leur offre une collation avant qu'ils ne s'en retournent.}  

\lhead{\firstmark}
\rhead{\botmark}

\subsection{\hspace{-0.5cm} {\Large \textcolor{darkblue}{\textbf{\ipa{ʈʂʰv̩˧˥}}}}\hspace{0.5cm}[\kern2pt{\textcolor{darkblue}{\textbf{\ipa{ʈʂʰv̩˧˥}}}}\kern2pt]} \hypertarget{t`s`\string_hv\string_=\string_M\string_T1}{}
\markboth{\textcolor{darkblue}{\textbf{\ipa{ʈʂʰv̩˧˥}}}}{}
\textcolor{teal}{\zh{动词}} \hspace{4pt} \zh{声调类:} MH.
\zh{掺和。} \textcolor{Sepia}{\selectlanguage{english}To add water, to pour extra water.} \textcolor{PineGreen}{\selectlanguage{french}Ajouter de l’eau, verser de l'eau.}  ¶ \textcolor{darkblue}{\textbf{\ipa{le˧-ʈʂʰv̩˧-ze˥}}} \zh{\mytextsc{accomp} \string_ \mytextsc{pfv}} \textcolor{Sepia}{\selectlanguage{english}\mytextsc{accomp} \string_ \mytextsc{pfv}} \textcolor{PineGreen}{\selectlanguage{french}\mytextsc{accomp} \string_ \mytextsc{pfv}}  
 ¶ \textcolor{darkblue}{\textbf{\ipa{dʑɯ˩ ʈʂʰv̩˩˥}}} \zh{加水(如:往锅里添加水)} \textcolor{Sepia}{\selectlanguage{english}to add water (e.g. in a pot)} \textcolor{PineGreen}{\selectlanguage{french}ajouter de l'eau (dans une marmite, ...)}  

\lhead{\firstmark}
\rhead{\botmark}

\subsection{\hspace{-0.5cm} {\Large \textcolor{darkblue}{\textbf{\ipa{ʈʂʰv̩˩\textsubscript{a}}}}}\hspace{0.5cm}[\kern2pt{\textcolor{darkblue}{\textbf{\ipa{ʈʂʰv̩˩˥}}}}\kern2pt]} \hypertarget{t`s`\string_hv\string_=\string_Ba1}{}
\markboth{\textcolor{darkblue}{\textbf{\ipa{ʈʂʰv̩˩\textsubscript{a}}}}}{}
\textcolor{teal}{\zh{动词}} \hspace{4pt} \zh{声调类:} L\textsubscript{a}.
\zh{染。} \textcolor{Sepia}{\selectlanguage{english}To dye.} \textcolor{PineGreen}{\selectlanguage{french}Teindre.}  ¶ \textcolor{darkblue}{\textbf{\ipa{mɤ˧-ʈʂʰv̩˩}}} \zh{\mytextsc{neg}} \textcolor{Sepia}{\selectlanguage{english}\mytextsc{neg}} \textcolor{PineGreen}{\selectlanguage{french}\mytextsc{neg}}  
 ¶ \textcolor{darkblue}{\textbf{\ipa{ʈʂʰv̩˩ mɤ˩-bi˩˥!}}} \zh{\string_ \mytextsc{neg} \mytextsc{fut}\string_imm} \textcolor{Sepia}{\selectlanguage{english}\string_ \mytextsc{neg} \mytextsc{fut}\string_imm} \textcolor{PineGreen}{\selectlanguage{french}\string_ \mytextsc{neg} \mytextsc{fut}\string_imm}  
 ¶ \textcolor{darkblue}{\textbf{\ipa{tso˧\textasciitilde{}tso˧ ʈʂʰv̩˥}}} \zh{染东西} \textcolor{Sepia}{\selectlanguage{english}to dye things} \textcolor{PineGreen}{\selectlanguage{french}teindre des choses}  

\lhead{\firstmark}
\rhead{\botmark}

\subsection{\hspace{-0.5cm} {\Large \textcolor{darkblue}{\textbf{\ipa{ʈʂʰv̩˧dʑɯ˧}}}}\hspace{0.5cm}[\kern2pt{\textcolor{darkblue}{\textbf{\ipa{ʈʂʰv̩˧dʑɯ˧}}}}\kern2pt]} \hypertarget{t`s`\string_hv\string_=\string_Mdz£M\string_M1}{}
\markboth{\textcolor{darkblue}{\textbf{\ipa{ʈʂʰv̩˧dʑɯ˧}}}}{}
\textcolor{teal}{\zh{名词}} \hspace{4pt} \zh{声调类:} M.
\zh{染料。} \textcolor{Sepia}{\selectlanguage{english}Dye, dyestuff.} \textcolor{PineGreen}{\selectlanguage{french}Teinture.}  ¶ \textcolor{darkblue}{\textbf{\ipa{dʑi˧hṽ˧-ʈʂʰv̩˧dʑɯ˧}}} \zh{衣服染料} \textcolor{Sepia}{\selectlanguage{english}dye for clothes} \textcolor{PineGreen}{\selectlanguage{french}teinture pour vêtements}  
 ¶ \textcolor{darkblue}{\textbf{\ipa{ʈʂʰv̩˧dʑɯ˧ | hṽ˩-hĩ˩˥}}} \zh{红色的染料} \textcolor{Sepia}{\selectlanguage{english}red dye} \textcolor{PineGreen}{\selectlanguage{french}teinture rouge}  
 \zh{量词}: \textcolor{darkblue}{\textbf{\ipa{kʰwɤ˥}}} 
\lhead{\firstmark}
\rhead{\botmark}

\subsection{\hspace{-0.5cm} {\Large \textcolor{darkblue}{\textbf{\ipa{ʈʂʰv̩˧mi˧}}}}\hspace{0.5cm}[\kern2pt{\textcolor{darkblue}{\textbf{\ipa{ʈʂʰv̩˧mi˧}}}}\kern2pt]} \hypertarget{t`s`\string_hv\string_=\string_Mmi\string_M1}{}
\markboth{\textcolor{darkblue}{\textbf{\ipa{ʈʂʰv̩˧mi˧}}}}{}
\textcolor{teal}{\zh{名词}} \hspace{4pt} \zh{声调类:} M.
\zh{太太、老婆、媳妇。} \textcolor{Sepia}{\selectlanguage{english}Wife.} \textcolor{PineGreen}{\selectlanguage{french}Épouse, femme.}  \zh{量词}: \textcolor{darkblue}{\textbf{\ipa{v̩˧}}} 
\lhead{\firstmark}
\rhead{\botmark}

\subsection{\hspace{-0.5cm} {\Large \textcolor{darkblue}{\textbf{\ipa{ʈʂʰv̩˧ɻ̍˧qʰv̩\#˥}}}}\hspace{0.5cm}[\kern2pt{\textcolor{darkblue}{\textbf{\ipa{ʈʂʰv̩˧ɻ̍˧qʰv̩˧}}}}\kern2pt]} \hypertarget{t`s`\string_hv\string_=\string_Mr£`̍\string_Mq\string_hv\string_=\#\string_T1}{}
\markboth{\textcolor{darkblue}{\textbf{\ipa{ʈʂʰv̩˧ɻ̍˧qʰv̩\#˥}}}}{}
\textcolor{teal}{\zh{名词}} \hspace{4pt} \zh{声调类:} \#H.
\zh{蚂蚁巢。} \textcolor{Sepia}{\selectlanguage{english}Ant nest.} \textcolor{PineGreen}{\selectlanguage{french}Fourmilière.}  ¶ \textcolor{darkblue}{\textbf{\ipa{ʈʂʰv̩˧ɻ̍˧qʰv̩˧ ɲi˥!}}} \zh{是蚂蚁巢!} \textcolor{Sepia}{\selectlanguage{english}It's an ant nest!} \textcolor{PineGreen}{\selectlanguage{french}c'est une fourmilière!}  
 \zh{量词}: \textcolor{darkblue}{\textbf{\ipa{ɭɯ˧}}} \zh{~【参考】~} \hyperlink{}{\textcolor{darkblue}{\textbf{\ipa{ʈʂʰv̩˧ɻ̍˥\$}}}} 
\lhead{\firstmark}
\rhead{\botmark}

\subsection{\hspace{-0.5cm} {\Large \textcolor{darkblue}{\textbf{\ipa{ʈʂʰv̩˧ɻ̍˥\$}}}}\hspace{0.5cm}[\kern2pt{\textcolor{darkblue}{\textbf{\ipa{ʈʂʰv̩˧ɻ̍˥}}}}\kern2pt]} \hypertarget{t`s`\string_hv\string_=\string_Mr£`̍\string_T\$1}{}
\markboth{\textcolor{darkblue}{\textbf{\ipa{ʈʂʰv̩˧ɻ̍˥\$}}}}{}
\textcolor{teal}{\zh{名词}} \hspace{4pt} \zh{声调类:} H\$.
\zh{蚂蚁。} \textcolor{Sepia}{\selectlanguage{english}Ant.} \textcolor{PineGreen}{\selectlanguage{french}Fourmi.}  ¶ \textcolor{darkblue}{\textbf{\ipa{ʈʂʰv̩˧ɻ̍˧ | tɕi˩-hĩ˩˥}}} \zh{小蚂蚁} \textcolor{Sepia}{\selectlanguage{english}small ant} \textcolor{PineGreen}{\selectlanguage{french}petite fourmi}  
 \zh{量词}: \textcolor{darkblue}{\textbf{\ipa{mi˩}}} 
\lhead{\firstmark}
\rhead{\botmark}

\subsection{\hspace{-0.5cm} {\Large \textcolor{darkblue}{\textbf{\ipa{ʈʂʰwæ˧\textsubscript{a}}}} \textsubscript{1}}\hspace{0.5cm}[\kern2pt{\textcolor{darkblue}{\textbf{\ipa{ʈʂʰwæ˩˥}}}}\kern2pt]} \hypertarget{t`s`\string_hw\{\string_Ma1}{}
\markboth{\textcolor{darkblue}{\textbf{\ipa{ʈʂʰwæ˧\textsubscript{a}}}} \textsubscript{1}}{}
\textcolor{teal}{\zh{动词}} \hspace{4pt} \zh{声调类:} M\textsubscript{a}.
\zh{腐烂。} \textcolor{Sepia}{\selectlanguage{english}To rot.} \textcolor{PineGreen}{\selectlanguage{french}Pourrir.}  ¶ \textcolor{darkblue}{\textbf{\ipa{ʈʂʰwæ˧-ze˧}}} \zh{烂了} \textcolor{Sepia}{\selectlanguage{english}\mytextsc{pfv}} \textcolor{PineGreen}{\selectlanguage{french}\mytextsc{pfv}}  
 ¶ \textcolor{darkblue}{\textbf{\ipa{le˧-ʈʂʰwæ˧-ze˧}}} \zh{\mytextsc{accomp} \string_ \mytextsc{pfv}} \textcolor{Sepia}{\selectlanguage{english}\mytextsc{accomp} \string_ \mytextsc{pfv}} \textcolor{PineGreen}{\selectlanguage{french}\mytextsc{accomp} \string_ \mytextsc{pfv}}  
 ¶ \textcolor{darkblue}{\textbf{\ipa{hĩ˧-ɳɯ˩ | mɤ˧-dzɯ˥, | le˧-ʈʂʰwæ˧-ze˧! |}}} \zh{没人吃,就烂了!(一个西瓜被忘记在橱柜里,就腐烂了)} \textcolor{Sepia}{\selectlanguage{english}No one ate it, and now it's rotten! (About a water melon that was forgotten in the cupboard.)} \textcolor{PineGreen}{\selectlanguage{french}On a oublié de la manger, et maintenant c'est pourri! (au sujet d'une pastèque qui a traîné dans le garde-manger et est maintenant incomestible)}  

\lhead{\firstmark}
\rhead{\botmark}

\subsection{\hspace{-0.5cm} {\Large \textcolor{darkblue}{\textbf{\ipa{ʈʂʰwæ˧\textsubscript{a}}}} \textsubscript{2}}\hspace{0.5cm}[\kern2pt{\textcolor{darkblue}{\textbf{\ipa{ʈʂʰwæ˥}}}}\kern2pt]} \hypertarget{t`s`\string_hw\{\string_Ma2}{}
\markboth{\textcolor{darkblue}{\textbf{\ipa{ʈʂʰwæ˧\textsubscript{a}}}} \textsubscript{2}}{}
\textcolor{teal}{\zh{动词}} \hspace{4pt} \zh{声调类:} M\textsubscript{a}.
\zh{醒来。} \textcolor{Sepia}{\selectlanguage{english}To wake up.} \textcolor{PineGreen}{\selectlanguage{french}Se réveiller.}  ¶ \textcolor{darkblue}{\textbf{\ipa{le˧-ʈʂʰwæ˧-ze˧}}} \zh{\mytextsc{accomp} \string_ \mytextsc{pfv}} \textcolor{Sepia}{\selectlanguage{english}\mytextsc{accomp} \string_ \mytextsc{pfv}} \textcolor{PineGreen}{\selectlanguage{french}\mytextsc{accomp} \string_ \mytextsc{pfv}}  
 ¶ \textcolor{darkblue}{\textbf{\ipa{gɤ˩ʈʂʰwæ˧}}} \zh{醒来} \textcolor{Sepia}{\selectlanguage{english}to wake up} \textcolor{PineGreen}{\selectlanguage{french}se réveiller}  
 ¶ \textcolor{darkblue}{\textbf{\ipa{gɤ˩ʈʂʰwæ˧-ze˧!}}} \zh{醒来了!} \textcolor{Sepia}{\selectlanguage{english}[(S)he] has woken up!} \textcolor{PineGreen}{\selectlanguage{french}(il) s'est réveillé!}  

\lhead{\firstmark}
\rhead{\botmark}

\subsection{\hspace{-0.5cm} {\Large \textcolor{darkblue}{\textbf{\ipa{ʈʂʰwæ˧-bv̩˧nv̩\#˥}}}}\hspace{0.5cm}[\kern2pt{\textcolor{darkblue}{\textbf{\ipa{xxxx non-correspondance entre le nombre de morphèmes et le nombre de tons de morphèmes}}}}\kern2pt]} \hypertarget{t`s`\string_hw\{\string_M-bv\string_=\string_Mnv\string_=\#\string_T1}{}
\markboth{\textcolor{darkblue}{\textbf{\ipa{ʈʂʰwæ˧-bv̩˧nv̩\#˥}}}}{}
\textcolor{teal}{\zh{形容词}} \hspace{4pt} \zh{声调类:} \#H.
\zh{食物变味,有臭味道了。} \textcolor{Sepia}{\selectlanguage{english}Bad, spoilt, rotten.} \textcolor{PineGreen}{\selectlanguage{french}Puant, à l'odeur de pourriture.}  ¶ \textcolor{darkblue}{\textbf{\ipa{ʈʂʰwæ˧-bv̩˧nv̩˧ ɲi˥!}}} \zh{臭了、有臭味道了} \textcolor{Sepia}{\selectlanguage{english}It stinks! / It smells rotten!} \textcolor{PineGreen}{\selectlanguage{french}Ca sent le pourri! / Ca pue la pourriture! / C'est vraiment malodorant!}  

\lhead{\firstmark}
\rhead{\botmark}

\subsection{\hspace{-0.5cm} {\Large \textcolor{darkblue}{\textbf{\ipa{ʈʂʰwæ˧kɯ˧}}}}\hspace{0.5cm}[\kern2pt{\textcolor{darkblue}{\textbf{\ipa{ʈʂʰwæ˩kɯ˥}}}}\kern2pt]} \hypertarget{t`s`\string_hw\{\string_MkM\string_M1}{}
\markboth{\textcolor{darkblue}{\textbf{\ipa{ʈʂʰwæ˧kɯ˧}}}}{}
\textcolor{teal}{\zh{名词}} \hspace{4pt} \zh{声调类:} M.
\zh{网。} \textcolor{Sepia}{\selectlanguage{english}Net.} \textcolor{PineGreen}{\selectlanguage{french}Filet.}  ¶ \textcolor{darkblue}{\textbf{\ipa{ʈʂʰwæ˧kɯ˧ tʰv̩˧-nɑ˩}}} \zh{这个网} \textcolor{Sepia}{\selectlanguage{english}\mytextsc{n}+\mytextsc{dem}+\mytextsc{clf}} \textcolor{PineGreen}{\selectlanguage{french}\mytextsc{n}+\mytextsc{dem}+\mytextsc{clf}}  
 \zh{量词}: \textcolor{darkblue}{\textbf{\ipa{nɑ˧}}} 
\lhead{\firstmark}
\rhead{\botmark}

\subsection{\hspace{-0.5cm} {\Large \textcolor{darkblue}{\textbf{\ipa{ʈʂʰwæ˧tsɯ˧}}}}\hspace{0.5cm}[\kern2pt{\textcolor{darkblue}{\textbf{\ipa{ʈʂʰwæ˩tsɯ˩˥}}}}\kern2pt]} \hypertarget{t`s`\string_hw\{\string_MtsM\string_M1}{}
\markboth{\textcolor{darkblue}{\textbf{\ipa{ʈʂʰwæ˧tsɯ˧}}}}{}
\textcolor{teal}{\zh{名词}} \hspace{4pt} \zh{声调类:} M.
\zh{窗户。} \textcolor{Sepia}{\selectlanguage{english}Window.} \textcolor{PineGreen}{\selectlanguage{french}Fenêtre.} \zh{当地汉语方言:}\zh{窗子。} \zh{【借词】} \zh{窗子}
 \zh{量词}: \textcolor{darkblue}{\textbf{\ipa{nɑ˧}}} 
\lhead{\firstmark}
\rhead{\botmark}

\subsection{\hspace{-0.5cm} {\Large \textcolor{darkblue}{\textbf{\ipa{ʈʂʰwæ˧ʈʂʰwæ˧}}}}\hspace{0.5cm}[\kern2pt{\textcolor{darkblue}{\textbf{\ipa{ʈʂʰwæ˧ʈʂʰwæ˧}}}}\kern2pt]} \hypertarget{t`s`\string_hw\{\string_Mt`s`\string_hw\{\string_M1}{}
\markboth{\textcolor{darkblue}{\textbf{\ipa{ʈʂʰwæ˧ʈʂʰwæ˧}}}}{}
\textcolor{teal}{\zh{名词}} \hspace{4pt} \zh{声调类:} M.
\zh{钹。} \textcolor{Sepia}{\selectlanguage{english}Cymbals (Chinese borrowing).} \textcolor{PineGreen}{\selectlanguage{french}Cymbales (mot emprunté au chinois).}  \zh{量词}: \textcolor{darkblue}{\textbf{\ipa{nɑ˧}}} 
\lhead{\firstmark}
\rhead{\botmark}

\subsection{\hspace{-0.5cm} {\Large \textcolor{darkblue}{\textbf{\ipa{ʈʂʰwæ˩\textsubscript{a}}}}}\hspace{0.5cm}[\kern2pt{\textcolor{darkblue}{\textbf{\ipa{ʈʂʰwæ˥}}}}\kern2pt]} \hypertarget{t`s`\string_hw\{\string_Ba1}{}
\markboth{\textcolor{darkblue}{\textbf{\ipa{ʈʂʰwæ˩\textsubscript{a}}}}}{}
\textcolor{teal}{\zh{形容词}} \hspace{4pt} \zh{声调类:} L\textsubscript{a}.
\zh{快(动作快,跑得快)。} \textcolor{Sepia}{\selectlanguage{english}Rapid, fast.} \textcolor{PineGreen}{\selectlanguage{french}Rapide.}  ¶ \textcolor{darkblue}{\textbf{\ipa{ʈʂʰwæ˩-hĩ˩˥}}} \zh{快的} \textcolor{Sepia}{\selectlanguage{english}\mytextsc{rel}/\mytextsc{nmlz}} \textcolor{PineGreen}{\selectlanguage{french}\mytextsc{rel}/\mytextsc{nmlz}}  
 ¶ \textcolor{darkblue}{\textbf{\ipa{ɲi˧to˧ ʈʂʰwæ˩}}} \zh{嘴快} \textcolor{Sepia}{\selectlanguage{english}who has a loose tongue, to talks too much} \textcolor{PineGreen}{\selectlanguage{french}bavard, qui parle sans réfléchir suffisamment}  

\lhead{\firstmark}
\rhead{\botmark}

\subsection{\hspace{-0.5cm} {\Large \textcolor{darkblue}{\textbf{\ipa{ʈʂʰwæ˩tsʰɯ˩}}}}\hspace{0.5cm}[\kern2pt{\textcolor{darkblue}{\textbf{\ipa{ʈʂʰwæ˧tsʰɯ˩}}}}\kern2pt]} \hypertarget{t`s`\string_hw\{\string_Bts\string_hM\string_B1}{}
\markboth{\textcolor{darkblue}{\textbf{\ipa{ʈʂʰwæ˩tsʰɯ˩}}}}{}
\textcolor{teal}{\zh{动词}} \hspace{4pt} \zh{声调类:} L.
\zh{创造(汉语借词)。} \textcolor{Sepia}{\selectlanguage{english}To create.} \textcolor{PineGreen}{\selectlanguage{french}Créer.}  \zh{【借词】} \zh{创造}

\lhead{\firstmark}
\rhead{\botmark}

\subsection{\hspace{-0.5cm} {\Large \textcolor{darkblue}{\textbf{\ipa{ʈʂʰwæ˧˥}}} \textsubscript{1}}\hspace{0.5cm}[\kern2pt{\textcolor{darkblue}{\textbf{\ipa{ʈʂʰwæ˩˥}}}}\kern2pt]} \hypertarget{t`s`\string_hw\{\string_M\string_T1}{}
\markboth{\textcolor{darkblue}{\textbf{\ipa{ʈʂʰwæ˧˥}}} \textsubscript{1}}{}
\textcolor{teal}{\zh{动词}} \hspace{4pt} \zh{声调类:} MH.
\zh{藏(东西)。} \textcolor{Sepia}{\selectlanguage{english}To hide (an object).} \textcolor{PineGreen}{\selectlanguage{french}Cacher: cacher un objet.} 
\lhead{\firstmark}
\rhead{\botmark}

\subsection{\hspace{-0.5cm} {\Large \textcolor{darkblue}{\textbf{\ipa{ʈʂʰwæ˧˥}}} \textsubscript{2}}\hspace{0.5cm}[\kern2pt{\textcolor{darkblue}{\textbf{\ipa{ʈʂʰwæ˧˥}}}}\kern2pt]} \hypertarget{t`s`\string_hw\{\string_M\string_T2}{}
\markboth{\textcolor{darkblue}{\textbf{\ipa{ʈʂʰwæ˧˥}}} \textsubscript{2}}{}
\textcolor{teal}{\zh{动词}} \hspace{4pt} \zh{声调类:} MH.
\zh{插、戳。} \textcolor{Sepia}{\selectlanguage{english}To stab.} \textcolor{PineGreen}{\selectlanguage{french}Enfoncer (un couteau) d'un geste brutal: poignarder, de haut en bas, avec force; insérer, planter, ficher (ex.: un couteau, une aiguille...).} 
\lhead{\firstmark}
\rhead{\botmark}

\subsection{\hspace{-0.5cm} {\Large \textcolor{darkblue}{\textbf{\ipa{ʈʂʰwæ˩˧}}}}\hspace{0.5cm}[\kern2pt{\textcolor{darkblue}{\textbf{\ipa{ʈʂʰwæ˧˥}}}}\kern2pt]} \hypertarget{t`s`\string_hw\{\string_B\string_M1}{}
\markboth{\textcolor{darkblue}{\textbf{\ipa{ʈʂʰwæ˩˧}}}}{}
\textcolor{teal}{\zh{名词}} \hspace{4pt} \zh{声调类:} LM.
\zh{船(汉语借词)。} \textcolor{Sepia}{\selectlanguage{english}Boat.} \textcolor{PineGreen}{\selectlanguage{french}Bateau (emprunt chinois ancien); désormais, c'est le terme employé pour les grands bateaux, par ex. les barges pour passer le Yangtze; ce sont des Chinois et des Hmong qui auraient installé le bateau permettant de passer le Yangtze, d'où l'utilisation d'un mot chinois.}  \zh{【借词】} \zh{船}
 \zh{量词}: \textcolor{darkblue}{\textbf{\ipa{nɑ˧}}} 
\lhead{\firstmark}
\rhead{\botmark}

\subsection{\hspace{-0.5cm} {\Large \textcolor{darkblue}{\textbf{\ipa{ʈʂʰwɤ˧tsʰi˧˥}}}}\hspace{0.5cm}[\kern2pt{\textcolor{darkblue}{\textbf{\ipa{ʈʂʰwɤ˧tsʰi˧}}}}\kern2pt]} \hypertarget{t`s`\string_hw7\string_Mts\string_hi\string_M\string_T1}{}
\markboth{\textcolor{darkblue}{\textbf{\ipa{ʈʂʰwɤ˧tsʰi˧˥}}}}{}
\textcolor{teal}{\zh{形容词}} \hspace{4pt} \zh{声调类:} MH\#.
\zh{窄。} \textcolor{Sepia}{\selectlanguage{english}Narrow.} \textcolor{PineGreen}{\selectlanguage{french}Étroit.} 
\lhead{\firstmark}
\rhead{\botmark}

\subsection{\hspace{-0.5cm} {\Large \textcolor{darkblue}{\textbf{\ipa{ʈʂʰwɤ˩}}}}\hspace{0.5cm}[\kern2pt{\textcolor{darkblue}{\textbf{\ipa{ʈʂʰwɤ˥}}}}\kern2pt]} \hypertarget{t`s`\string_hw7\string_B1}{}
\markboth{\textcolor{darkblue}{\textbf{\ipa{ʈʂʰwɤ˩}}}}{}
\textcolor{teal}{\zh{名词}} \hspace{4pt} \zh{声调类:} L.
\zh{晚饭。} \textcolor{Sepia}{\selectlanguage{english}Dinner.} \textcolor{PineGreen}{\selectlanguage{french}Repas du soir, dîner.}  ¶ \textcolor{darkblue}{\textbf{\ipa{ʈʂʰwɤ˩ gv̩˩˥}}} \zh{做晚饭} \textcolor{Sepia}{\selectlanguage{english}to cook dinner} \textcolor{PineGreen}{\selectlanguage{french}cuisiner le dîner}  
 ¶ \textcolor{darkblue}{\textbf{\ipa{ʈʂʰwɤ˩ tʰv̩˩˥}}} \zh{请吃晚饭,提供晚餐(不一定自己做:意思是提供原料)} \textcolor{Sepia}{\selectlanguage{english}to take charge of dinner, to prepare dinner, to provide dinner (this can refer to providing the ingredients for the meal, not necessarily preparing it oneself)} \textcolor{PineGreen}{\selectlanguage{french}offrir à dîner, se charger du dîner (personne qui invite, pas nécessairement qui fait la cuisine elle-même)}  
 ¶ \textcolor{darkblue}{\textbf{\ipa{ʈʂʰwɤ˩ dzɯ˩˥}}} \zh{吃晚饭} \textcolor{Sepia}{\selectlanguage{english}to eat dinner} \textcolor{PineGreen}{\selectlanguage{french}prendre le repas du soir}  

\lhead{\firstmark}
\rhead{\botmark}

\newpage
\section*{\centering- \textcolor{darkblue}{\textbf{\ipa{u}}} -}
\subsection{\hspace{-0.5cm} {\Large \textcolor{darkblue}{\textbf{\ipa{u˧}}}}\hspace{0.5cm}[\kern2pt{\textcolor{darkblue}{\textbf{\ipa{u˥}}}}\kern2pt]} \hypertarget{u\string_M1}{}
\markboth{\textcolor{darkblue}{\textbf{\ipa{u˧}}}}{}
\textcolor{teal}{\zh{代词}} \hspace{4pt} \zh{声调类:} M.
\zh{我家人、我家族。} \textcolor{Sepia}{\selectlanguage{english}First person, associative: my family/my people. This root is only attested together with a plural or associative clitic.} \textcolor{PineGreen}{\selectlanguage{french}Pronom de 1e personne, associatif: les miens. Cette racine n'apparaît qu'en combinaison avec un clitique pluriel ou associatif.}  ¶ \textcolor{darkblue}{\textbf{\ipa{u˧=ɻ˩, ʈʂʰɯ˧=ɻ˩}}} \textcolor{Sepia}{\selectlanguage{english}My clan, his clan: two terms that stand in a relation of opposition} \textcolor{PineGreen}{\selectlanguage{french}mon clan, son clan : deux termes qui forment une opposition}  
 ¶ \textcolor{darkblue}{\textbf{\ipa{u˧ɻ̍˩ | ə˧si˧}}} \zh{我家祖母} \textcolor{Sepia}{\selectlanguage{english}my great-grandmother} \textcolor{PineGreen}{\selectlanguage{french}mon arrière-grand-mère}  
 ¶ \textcolor{darkblue}{\textbf{\ipa{u˧=ɻæ˩, ʈʂʰɯ˧=ɻæ˩}}} \textcolor{Sepia}{\selectlanguage{english}Us, them: two terms that stand in a relation of opposition} \textcolor{PineGreen}{\selectlanguage{french}Nous autres, eux : deux termes qui forment une opposition.}  

\lhead{\firstmark}
\rhead{\botmark}

\newpage
\section*{\centering- \textcolor{darkblue}{\textbf{\ipa{v}}} \textcolor{darkblue}{\textbf{\ipa{ṽ}}} -}
\subsection{\hspace{-0.5cm} {\Large \textcolor{darkblue}{\textbf{\ipa{v̩˩}}}}\hspace{0.5cm}[\kern2pt{\textcolor{darkblue}{\textbf{\ipa{v̩˥}}}}\kern2pt]} \hypertarget{v\string_=\string_B1}{}
\markboth{\textcolor{darkblue}{\textbf{\ipa{v̩˩}}}}{}
\textcolor{teal}{\zh{动词}} \hspace{4pt} \zh{声调类:} L\textsubscript{a}.
\zh{搂(人的脖子)。} \textcolor{Sepia}{\selectlanguage{english}To hug, to embrace.} \textcolor{PineGreen}{\selectlanguage{french}Embrasser, prendre dans ses bras (monosyllabe extrait de la forme \textcolor{darkblue}{\textbf{\ipa{/le˧-v̩˧~v̩˥/}}}).}  ¶ \textcolor{darkblue}{\textbf{\ipa{ʁæ˧ | le˧-v̩˧\textasciitilde{}v̩˥}}} \zh{搂人的脖子} \textcolor{Sepia}{\selectlanguage{english}to embrace someone's neck} \textcolor{PineGreen}{\selectlanguage{french}prendre le cou (de quelqu'un) dans son bras}  
 ¶ \textcolor{darkblue}{\textbf{\ipa{ʁæ˧ | le˧-v̩˩}}} \zh{同上} \textcolor{Sepia}{\selectlanguage{english}as above} \textcolor{PineGreen}{\selectlanguage{french}même sens}  
 ¶ \textcolor{darkblue}{\textbf{\ipa{ʁæ˧ v̩˥ se˩}}} \zh{互相搂着走} \textcolor{Sepia}{\selectlanguage{english}to walk together with someone, arm curled around the neck} \textcolor{PineGreen}{\selectlanguage{french}marcher/se promener en tenant le cou de quelqu'un/enlacé avec quelqu'un}  

\lhead{\firstmark}
\rhead{\botmark}

\subsection{\hspace{-0.5cm} {\Large \textcolor{darkblue}{\textbf{\ipa{v̩˧}}}}\hspace{0.5cm}[\kern2pt{\textcolor{darkblue}{\textbf{\ipa{v̩˩˥}}}}\kern2pt]} \hypertarget{v\string_=\string_M1}{}
\markboth{\textcolor{darkblue}{\textbf{\ipa{v̩˧}}}}{}
\textcolor{teal}{\zh{量词}} \hspace{4pt} \zh{声调类:} M *.
\zh{量词:人(一个人)。只能用于单数。} \textcolor{Sepia}{\selectlanguage{english}Classifier for one individual (human); can only be used after the numeral 'one', i.e. either in the singular or in association with numbers whose last figure is 'one'.} \textcolor{PineGreen}{\selectlanguage{french}Classificateur pour un individu humain; ne peut s'employer que pour le chiffre 'un', autrement dit soit au singulier, soit avec des nombres dont le dernier chiffre est 'un'.}  ¶ \textcolor{darkblue}{\textbf{\ipa{ɖɯ˧-v̩\#˥; ɖɯ˧-v̩˧ ɲi˥}}} \zh{一个人,是一个人(为了确认声调而问的短语)} \textcolor{Sepia}{\selectlanguage{english}one person; it is one person (elicited to verify tone)} \textcolor{PineGreen}{\selectlanguage{french}1 personne; c'est 1 personne (élicité pour vérifier le ton)}  
 ¶ \textcolor{darkblue}{\textbf{\ipa{tsʰe˧ɖɯ˧-v̩˧}}} \zh{十一个人} \textcolor{Sepia}{\selectlanguage{english}11 persons} \textcolor{PineGreen}{\selectlanguage{french}11 personnes}  
 ¶ \textcolor{darkblue}{\textbf{\ipa{ɲi˧tsi˧ɖɯ˧-v̩˧}}} \zh{二十一个人} \textcolor{Sepia}{\selectlanguage{english}21 persons} \textcolor{PineGreen}{\selectlanguage{french}21 personnes}  
 ¶ \textcolor{darkblue}{\textbf{\ipa{so˧tsʰi˧ɖɯ˧-v̩˧}}} \zh{三十一个人} \textcolor{Sepia}{\selectlanguage{english}31 persons} \textcolor{PineGreen}{\selectlanguage{french}31 personnes}  
 ¶ \textcolor{darkblue}{\textbf{\ipa{ʐv̩˧tsʰi˩ɖɯ˩-v̩˩}}} \zh{四十一个人} \textcolor{Sepia}{\selectlanguage{english}41 persons} \textcolor{PineGreen}{\selectlanguage{french}41 personnes}  
 ¶ \textcolor{darkblue}{\textbf{\ipa{ŋwɤ˧tsʰi˩ɖɯ˩-v̩˩}}} \zh{五十一个人} \textcolor{Sepia}{\selectlanguage{english}51 persons} \textcolor{PineGreen}{\selectlanguage{french}51 personnes}  
 ¶ \textcolor{darkblue}{\textbf{\ipa{qʰv̩˧tsʰi˧ɖɯ˧-v̩˥}}} \zh{六十一个人} \textcolor{Sepia}{\selectlanguage{english}61 persons} \textcolor{PineGreen}{\selectlanguage{french}61 personnes}  
 ¶ \textcolor{darkblue}{\textbf{\ipa{ʂɯ˧tsʰi˩ɖɯ˩-v̩˩}}} \zh{七十一个人} \textcolor{Sepia}{\selectlanguage{english}71 persons} \textcolor{PineGreen}{\selectlanguage{french}71 personnes}  
 ¶ \textcolor{darkblue}{\textbf{\ipa{hõ˧tsʰi˩ɖɯ˩-v̩˩}}} \zh{八十一个人} \textcolor{Sepia}{\selectlanguage{english}81 persons} \textcolor{PineGreen}{\selectlanguage{french}81 personnes}  
 ¶ \textcolor{darkblue}{\textbf{\ipa{gv̩˧tsʰi˩ɖɯ˩-v̩˩}}} \zh{九十一个人} \textcolor{Sepia}{\selectlanguage{english}91 persons} \textcolor{PineGreen}{\selectlanguage{french}91 personnes}  

\lhead{\firstmark}
\rhead{\botmark}

\subsection{\hspace{-0.5cm} {\Large \textcolor{darkblue}{\textbf{\ipa{v̩˥}}}}\hspace{0.5cm}[\kern2pt{\textcolor{darkblue}{\textbf{\ipa{v̩˥}}}}\kern2pt]} \hypertarget{v\string_=\string_T1}{}
\markboth{\textcolor{darkblue}{\textbf{\ipa{v̩˥}}}}{}
\textcolor{teal}{\zh{名词}} \hspace{4pt} \zh{声调类:} \#H.
\zh{锅。} \textcolor{Sepia}{\selectlanguage{english}Pot.} \textcolor{PineGreen}{\selectlanguage{french}Casserole (terme générique).}  \zh{量词}: \textcolor{darkblue}{\textbf{\ipa{ɭɯ˧}}} 
\lhead{\firstmark}
\rhead{\botmark}

\subsection{\hspace{-0.5cm} {\Large \textcolor{darkblue}{\textbf{\ipa{v̩˥\textsubscript{b}}}}}\hspace{0.5cm}[\kern2pt{\textcolor{darkblue}{\textbf{\ipa{v̩˥}}}}\kern2pt]} \hypertarget{v\string_=\string_Tb1}{}
\markboth{\textcolor{darkblue}{\textbf{\ipa{v̩˥\textsubscript{b}}}}}{}
\textcolor{teal}{\zh{量词}} \hspace{4pt} \zh{声调类:} H\textsubscript{b}.
\zh{量词:锅(一口),或锅的容量。} \textcolor{Sepia}{\selectlanguage{english}Self-classifier for pots; classifier for potfuls (of food, liquid…).} \textcolor{PineGreen}{\selectlanguage{french}Auto-classificateur des casseroles; et classificateur des casserolées (utilisant la casserole comme mesure de quantité de nourriture, liquide ou solide).} 
\lhead{\firstmark}
\rhead{\botmark}

\subsection{\hspace{-0.5cm} {\Large \textcolor{darkblue}{\textbf{\ipa{v̩˩dze˩}}}}\hspace{0.5cm}[\kern2pt{\textcolor{darkblue}{\textbf{\ipa{v̩˥}}}}\kern2pt]} \hypertarget{v\string_=\string_Bdze\string_B1}{}
\markboth{\textcolor{darkblue}{\textbf{\ipa{v̩˩dze˩}}}}{}
\textcolor{teal}{\zh{名词}} \hspace{4pt} \zh{声调类:} L.
\zh{鸟。} \textcolor{Sepia}{\selectlanguage{english}Bird.} \textcolor{PineGreen}{\selectlanguage{french}Oiseau.}  ¶ \textcolor{darkblue}{\textbf{\ipa{v̩˩dze˩-bi˥ | hṽ˧ ʑi˥}}} \zh{鸟(身)上有(羽)毛。} \textcolor{Sepia}{\selectlanguage{english}There are feathers on the bird.} \textcolor{PineGreen}{\selectlanguage{french}Sur l'oiseau, il y a des plumes.}  
 ¶ \textcolor{darkblue}{\textbf{\ipa{v̩˩dze˩-mi˩}}} \zh{母鸟} \textcolor{Sepia}{\selectlanguage{english}female bird} \textcolor{PineGreen}{\selectlanguage{french}oiseau femelle}  
 ¶ \textcolor{darkblue}{\textbf{\ipa{v̩˩dze˩-pʰv̩˩}}} \zh{公鸟} \textcolor{Sepia}{\selectlanguage{english}male bird} \textcolor{PineGreen}{\selectlanguage{french}oiseau mâle}  
 ¶ \textcolor{darkblue}{\textbf{\ipa{v̩˩dze˩-zo˩}}} \zh{小鸟} \textcolor{Sepia}{\selectlanguage{english}baby bird} \textcolor{PineGreen}{\selectlanguage{french}petit oiseau}  
 \zh{量词}: \textcolor{darkblue}{\textbf{\ipa{mi˩}}} 
\lhead{\firstmark}
\rhead{\botmark}

\subsection{\hspace{-0.5cm} {\Large \textcolor{darkblue}{\textbf{\ipa{v̩˩dze˩-kʰv̩˩}}}}\hspace{0.5cm}[\kern2pt{\textcolor{darkblue}{\textbf{\ipa{xxxx non-correspondance entre le nombre de morphèmes et le nombre de tons de morphèmes}}}}\kern2pt]} \hypertarget{v\string_=\string_Bdze\string_B-k\string_hv\string_=\string_B1}{}
\markboth{\textcolor{darkblue}{\textbf{\ipa{v̩˩dze˩-kʰv̩˩}}}}{}
\textcolor{teal}{\zh{名词}} \hspace{4pt} \zh{声调类:} L.
\zh{鸟窝,鸟巢。} \textcolor{Sepia}{\selectlanguage{english}Nest.} \textcolor{PineGreen}{\selectlanguage{french}Nid.}  ¶ \textcolor{darkblue}{\textbf{\ipa{v̩˩dze˩kʰv̩˩ ɲi˥.}}} \zh{是鸟窝} \textcolor{Sepia}{\selectlanguage{english}\string_ \mytextsc{cop}} \textcolor{PineGreen}{\selectlanguage{french}\string_ \mytextsc{cop}}  
 \zh{量词}: \textcolor{darkblue}{\textbf{\ipa{ɭɯ˧}}} 
\lhead{\firstmark}
\rhead{\botmark}

\subsection{\hspace{-0.5cm} {\Large \textcolor{darkblue}{\textbf{\ipa{v̩˧dʑo\#˥}}}}\hspace{0.5cm}[\kern2pt{\textcolor{darkblue}{\textbf{\ipa{v̩˩dʑo˩˥}}}}\kern2pt]} \hypertarget{v\string_=\string_Mdz£o\#\string_T1}{}
\markboth{\textcolor{darkblue}{\textbf{\ipa{v̩˧dʑo\#˥}}}}{}
\textcolor{teal}{\zh{名词}} \hspace{4pt} \zh{声调类:} \#H.
\zh{屋脚(村落名)。} \textcolor{Sepia}{\selectlanguage{english}Wujiao township.} \textcolor{PineGreen}{\selectlanguage{french}Wujiao (nom de village).} 
\lhead{\firstmark}
\rhead{\botmark}

\subsection{\hspace{-0.5cm} {\Large \textcolor{darkblue}{\textbf{\ipa{v̩˩dʑɯ˩}}}}\hspace{0.5cm}[\kern2pt{\textcolor{darkblue}{\textbf{\ipa{v̩˧dʑɯ˧}}}}\kern2pt]} \hypertarget{v\string_=\string_Bdz£M\string_B1}{}
\markboth{\textcolor{darkblue}{\textbf{\ipa{v̩˩dʑɯ˩}}}}{}
\textcolor{teal}{\zh{名词}} \hspace{4pt} \zh{声调类:} L.
\zh{汤。} \textcolor{Sepia}{\selectlanguage{english}Soup.} \textcolor{PineGreen}{\selectlanguage{french}Soupe.}  ¶ \textcolor{darkblue}{\textbf{\ipa{æ˩ʂe˧-v̩˥dʑɯ˩}}} \zh{鸡汤} \textcolor{Sepia}{\selectlanguage{english}chicken soup} \textcolor{PineGreen}{\selectlanguage{french}soupe de poulet}  

\lhead{\firstmark}
\rhead{\botmark}

\subsection{\hspace{-0.5cm} {\Large \textcolor{darkblue}{\textbf{\ipa{v̩˧ko˧}}}}\hspace{0.5cm}[\kern2pt{\textcolor{darkblue}{\textbf{\ipa{v̩˩ko˩˥}}}}\kern2pt]} \hypertarget{v\string_=\string_Mko\string_M1}{}
\markboth{\textcolor{darkblue}{\textbf{\ipa{v̩˧ko˧}}}}{}
\textcolor{teal}{\zh{名词}} \hspace{4pt} \zh{声调类:} M.
\zh{乌龟(汉语借词)。} \textcolor{Sepia}{\selectlanguage{english}Tortoise.} \textcolor{PineGreen}{\selectlanguage{french}Tortue.} 
\lhead{\firstmark}
\rhead{\botmark}

\subsection{\hspace{-0.5cm} {\Large \textcolor{darkblue}{\textbf{\ipa{v̩˧lɑ˩-ʝi˩}}}}\hspace{0.5cm}[\kern2pt{\textcolor{darkblue}{\textbf{\ipa{xxxx non-correspondance entre le nombre de morphèmes et le nombre de tons de morphèmes}}}}\kern2pt]} \hypertarget{v\string_=\string_MlA\string_B-j££i\string_B1}{}
\markboth{\textcolor{darkblue}{\textbf{\ipa{v̩˧lɑ˩-ʝi˩}}}}{}
\textcolor{teal}{\zh{动词}} \hspace{4pt} \zh{声调类:} L\#-.
\zh{做生意。} \textcolor{Sepia}{\selectlanguage{english}To trade, to do business.} \textcolor{PineGreen}{\selectlanguage{french}Faire du commerce.} 
\lhead{\firstmark}
\rhead{\botmark}

\subsection{\hspace{-0.5cm} {\Large \textcolor{darkblue}{\textbf{\ipa{v̩˧lɑ˩-ʝi˩-hĩ˩-hĩ˩}}}}\hspace{0.5cm}[\kern2pt{\textcolor{darkblue}{\textbf{\ipa{xxxx non-correspondance entre le nombre de morphèmes et le nombre de tons de morphèmes}}}}\kern2pt]} \hypertarget{v\string_=\string_MlA\string_B-j££i\string_B-hi\string_~\string_B-hi\string_~\string_B1}{}
\markboth{\textcolor{darkblue}{\textbf{\ipa{v̩˧lɑ˩-ʝi˩-hĩ˩-hĩ˩}}}}{}
\textcolor{teal}{\zh{名词}} \hspace{4pt} \zh{声调类:} L\#-.
\zh{商人。} \textcolor{Sepia}{\selectlanguage{english}Merchant.} \textcolor{PineGreen}{\selectlanguage{french}Marchand.}  ¶ \textcolor{darkblue}{\textbf{\ipa{v̩˧lɑ˩-ʝi˩-hĩ˩}}} \zh{商人} \textcolor{Sepia}{\selectlanguage{english}merchant} \textcolor{PineGreen}{\selectlanguage{french}marchand}  
 \zh{量词}: \textcolor{darkblue}{\textbf{\ipa{v̩˧}}} 
\lhead{\firstmark}
\rhead{\botmark}

\subsection{\hspace{-0.5cm} {\Large \textcolor{darkblue}{\textbf{\ipa{v̩˧mi\#˥}}}}\hspace{0.5cm}[\kern2pt{\textcolor{darkblue}{\textbf{\ipa{xxxx non-correspondance entre le nombre de morphèmes et le nombre de tons de morphèmes}}}}\kern2pt]} \hypertarget{v\string_=\string_Mmi\#\string_T1}{}
\markboth{\textcolor{darkblue}{\textbf{\ipa{v̩˧mi\#˥}}}}{}
\textcolor{teal}{\zh{名词}} \hspace{4pt} \zh{声调类:} \#H.
\zh{大锅。} \textcolor{Sepia}{\selectlanguage{english}Large cooking pot.} \textcolor{PineGreen}{\selectlanguage{french}Grande casserole.} 
\lhead{\firstmark}
\rhead{\botmark}

\subsection{\hspace{-0.5cm} {\Large \textcolor{darkblue}{\textbf{\ipa{v̩˩tsʰɤ˧˥}}}}\hspace{0.5cm}[\kern2pt{\textcolor{darkblue}{\textbf{\ipa{v̩˧tsʰɤ˧}}}}\kern2pt]} \hypertarget{v\string_=\string_Bts\string_h7\string_M\string_T1}{}
\markboth{\textcolor{darkblue}{\textbf{\ipa{v̩˩tsʰɤ˧˥}}}}{}
\textcolor{teal}{\zh{名词}} \hspace{4pt} \zh{声调类:} LM+MH\#.
\zh{蔬菜。} \textcolor{Sepia}{\selectlanguage{english}Vegetables (in a broad sense including fresh vegetables and picked vegetables).} \textcolor{PineGreen}{\selectlanguage{french}Légumes.}  ¶ \textcolor{darkblue}{\textbf{\ipa{[Housebuilding2] v˩tsʰɤ˧-tsʰɑ˧nɑ˥}}} \zh{新鲜蔬菜。直译:‘绿油油的青菜’。指的不是某种具体的青菜,而是任何新鲜蔬菜,分别于酸菜。制造酸菜的过程中,蔬菜(萝卜等等)褪色:失去原来的深色。} \textcolor{Sepia}{\selectlanguage{english}fresh vegetables. Literally 'dark vegetables'. This does not refer to one species in particular, but to all sorts of fresh vegetables, as opposed to pickled vegetables. In the process of preserving vegetables, their original darker colours tend to fade away.} \textcolor{PineGreen}{\selectlanguage{french}légumes frais. Littéralement 'légume de couleur sombre. L'expression ne renvoie pas à une espèce en particulier, mais désigne globalement les légumes verts, par opposition aux légumes conservés, qui perdaient de leur couleur au cours du processus de fermentation.}  
 \zh{量词}: \textcolor{darkblue}{\textbf{\ipa{po˧}}} 
\lhead{\firstmark}
\rhead{\botmark}

\subsection{\hspace{-0.5cm} {\Large \textcolor{darkblue}{\textbf{\ipa{v̩˩tsʰɤ˧-bv̩\#˥}}}}\hspace{0.5cm}[\kern2pt{\textcolor{darkblue}{\textbf{\ipa{xxxx non-correspondance entre le nombre de morphèmes et le nombre de tons de morphèmes}}}}\kern2pt]} \hypertarget{v\string_=\string_Bts\string_h7\string_M-bv\string_=\#\string_T1}{}
\markboth{\textcolor{darkblue}{\textbf{\ipa{v̩˩tsʰɤ˧-bv̩\#˥}}}}{}
\textcolor{teal}{\zh{名词}} \hspace{4pt} \zh{声调类:} LM+\#H.
\zh{瓢虫。} \textcolor{Sepia}{\selectlanguage{english}Ladybug, ladybird.} \textcolor{PineGreen}{\selectlanguage{french}Coccinelle.}  \zh{量词}: \textcolor{darkblue}{\textbf{\ipa{mi˩}}} 
\lhead{\firstmark}
\rhead{\botmark}

\subsection{\hspace{-0.5cm} {\Large \textcolor{darkblue}{\textbf{\ipa{v̩˩tsʰɤ˧-pʰv̩˥}}}}\hspace{0.5cm}[\kern2pt{\textcolor{darkblue}{\textbf{\ipa{xxxx non-correspondance entre le nombre de morphèmes et le nombre de tons de morphèmes}}}}\kern2pt]} \hypertarget{v\string_=\string_Bts\string_h7\string_M-p\string_hv\string_=\string_T1}{}
\markboth{\textcolor{darkblue}{\textbf{\ipa{v̩˩tsʰɤ˧-pʰv̩˥}}}}{}
\textcolor{teal}{\zh{名词}} \hspace{4pt} \zh{声调类:} LM+H\#.
\zh{白菜。} \textcolor{Sepia}{\selectlanguage{english}Chinese cabbage.} \textcolor{PineGreen}{\selectlanguage{french}Chou chinois.}  \zh{量词}: \textcolor{darkblue}{\textbf{\ipa{po˧}}} \zh{~【同音词】~} \hyperlink{}{\textcolor{darkblue}{\textbf{\ipa{tsʰæ˧pʰv˧˥}}}} 
\lhead{\firstmark}
\rhead{\botmark}

\subsection{\hspace{-0.5cm} {\Large \textcolor{darkblue}{\textbf{\ipa{v̩˩tsʰɤ˧-v̩˥ɲi˩}}}}\hspace{0.5cm}[\kern2pt{\textcolor{darkblue}{\textbf{\ipa{xxxx non-correspondance entre le nombre de morphèmes et le nombre de tons de morphèmes}}}}\kern2pt]} \hypertarget{v\string_=\string_Bts\string_h7\string_M-v\string_=\string_TJi\string_B1}{}
\markboth{\textcolor{darkblue}{\textbf{\ipa{v̩˩tsʰɤ˧-v̩˥ɲi˩}}}}{}
\textcolor{teal}{\zh{名词}} \hspace{4pt} \zh{声调类:} LM+\#H-.
\zh{蔬菜。} \textcolor{Sepia}{\selectlanguage{english}Vegetables.} \textcolor{PineGreen}{\selectlanguage{french}Légumes.}  \zh{量词}: \textcolor{darkblue}{\textbf{\ipa{qɑ˩}}} 
\lhead{\firstmark}
\rhead{\botmark}

\subsection{\hspace{-0.5cm} {\Large \textcolor{darkblue}{\textbf{\ipa{v̩˧zo\#˥}}}}\hspace{0.5cm}[\kern2pt{\textcolor{darkblue}{\textbf{\ipa{xxxx non-correspondance entre le nombre de morphèmes et le nombre de tons de morphèmes}}}}\kern2pt]} \hypertarget{v\string_=\string_Mzo\#\string_T1}{}
\markboth{\textcolor{darkblue}{\textbf{\ipa{v̩˧zo\#˥}}}}{}
\textcolor{teal}{\zh{名词}} \hspace{4pt} \zh{声调类:} \#H.
\zh{小锅。} \textcolor{Sepia}{\selectlanguage{english}Small cooking pot.} \textcolor{PineGreen}{\selectlanguage{french}Petite casserole.} 
\lhead{\firstmark}
\rhead{\botmark}

\subsection{\hspace{-0.5cm} {\Large \textcolor{darkblue}{\textbf{\ipa{v̩˧\textasciitilde{}v̩˧\textsubscript{a}}}}}\hspace{0.5cm}[\kern2pt{\textcolor{darkblue}{\textbf{\ipa{v̩˧v̩˧}}}}\kern2pt]} \hypertarget{v\string_=\string_M~v\string_=\string_Ma1}{}
\markboth{\textcolor{darkblue}{\textbf{\ipa{v̩˧\textasciitilde{}v̩˧\textsubscript{a}}}}}{}
\textcolor{teal}{\zh{动词}} \hspace{4pt} \zh{声调类:} M\textsubscript{a}.
\zh{嚼。} \textcolor{Sepia}{\selectlanguage{english}To chew; to chew the cud.} \textcolor{PineGreen}{\selectlanguage{french}Mâcher.}  ¶ \textcolor{darkblue}{\textbf{\ipa{le˧-v̩˧\textasciitilde{}v̩˧ +ze˩}}} \zh{\mytextsc{accomp}} \textcolor{Sepia}{\selectlanguage{english}\mytextsc{accomp}} \textcolor{PineGreen}{\selectlanguage{french}\mytextsc{accomp}}  
 ¶ \textcolor{darkblue}{\textbf{\ipa{le˧-wo˧ v̩˧\textasciitilde{}v̩˧}}} \zh{反刍} \textcolor{Sepia}{\selectlanguage{english}to chew the cud} \textcolor{PineGreen}{\selectlanguage{french}ruminer (la vache rumine)}  

\lhead{\firstmark}
\rhead{\botmark}

\newpage
\section*{\centering- \textcolor{darkblue}{\textbf{\ipa{w}}} \textcolor{darkblue}{\textbf{\ipa{wɑ}}} \textcolor{darkblue}{\textbf{\ipa{wæ}}} \textcolor{darkblue}{\textbf{\ipa{wɤ}}} \textcolor{darkblue}{\textbf{\ipa{wo}}} \textcolor{darkblue}{\textbf{\ipa{wɤ̃}}} -}
\subsection{\hspace{-0.5cm} {\Large \textcolor{darkblue}{\textbf{\ipa{wɤ˧}}} \textsubscript{1}}\hspace{0.5cm}[\kern2pt{\textcolor{darkblue}{\textbf{\ipa{wɤ˩˥}}}}\kern2pt]} \hypertarget{w7\string_M1}{}
\markboth{\textcolor{darkblue}{\textbf{\ipa{wɤ˧}}} \textsubscript{1}}{}
\textcolor{teal}{\zh{名词}} \hspace{4pt} \zh{声调类:} M.
\zh{奴隶,农奴。音译:“俄”。} \textcolor{Sepia}{\selectlanguage{english}Serf, slave (lowest of the 3 ranks in feudal society).} \textcolor{PineGreen}{\selectlanguage{french}Serf, esclave (la plus basse caste de la société ancienne).}  \zh{量词}: \textcolor{darkblue}{\textbf{\ipa{v̩˧}}} 
\lhead{\firstmark}
\rhead{\botmark}

\subsection{\hspace{-0.5cm} {\Large \textcolor{darkblue}{\textbf{\ipa{wɤ˧}}} \textsubscript{2}}\hspace{0.5cm}[\kern2pt{\textcolor{darkblue}{\textbf{\ipa{wɤ˥}}}}\kern2pt]} \hypertarget{w7\string_M2}{}
\markboth{\textcolor{darkblue}{\textbf{\ipa{wɤ˧}}} \textsubscript{2}}{}
\textcolor{teal}{\zh{语气助词}} \hspace{4pt} \zh{声调类:} M.
\zh{句尾助词:吧、呗。} \textcolor{Sepia}{\selectlanguage{english}Final particle conveying exclamation, with a nuance of obviousness.} \textcolor{PineGreen}{\selectlanguage{french}Particule finale exclamative, avec une nuance d'évidence.} 
\lhead{\firstmark}
\rhead{\botmark}

\subsection{\hspace{-0.5cm} {\Large \textcolor{darkblue}{\textbf{\ipa{wɤ˩\textsubscript{a}}}}}\hspace{0.5cm}[\kern2pt{\textcolor{darkblue}{\textbf{\ipa{wɤ˥}}}}\kern2pt]} \hypertarget{w7\string_Ba1}{}
\markboth{\textcolor{darkblue}{\textbf{\ipa{wɤ˩\textsubscript{a}}}}}{}
\textcolor{teal}{\zh{动词}} \hspace{4pt} \zh{声调类:} L\textsubscript{a}.
\zh{依赖。} \textcolor{Sepia}{\selectlanguage{english}To depend on.} \textcolor{PineGreen}{\selectlanguage{french}Dépendre de, se reposer sur.}  ¶ \textcolor{darkblue}{\textbf{\ipa{hĩ˧-bi˥ | wɤ˩-mɤ˩-bi˩˥!}}} \zh{不要依赖别人!} \textcolor{Sepia}{\selectlanguage{english}One should not depend on others!} \textcolor{PineGreen}{\selectlanguage{french}Il ne faut pas dépendre des autres!}  
 ¶ \textcolor{darkblue}{\textbf{\ipa{hĩ˧-bi˥ | wɤ˩-v̩˩-tʰv̩˩˥!}}} \zh{(无论如何)人都会依靠别人的!(意思是:人不能完全独立,人活在人间就会或多或少需要依靠别人。)} \textcolor{Sepia}{\selectlanguage{english}(Whether one wants or not) one depends on others (in some respect or other)!} \textcolor{PineGreen}{\selectlanguage{french}On se trouve dépendre des autres!}  

\lhead{\firstmark}
\rhead{\botmark}

\subsection{\hspace{-0.5cm} {\Large \textcolor{darkblue}{\textbf{\ipa{wɤ˩\textsubscript{b}}}}}\hspace{0.5cm}[\kern2pt{\textcolor{darkblue}{\textbf{\ipa{wɤ˩˥}}}}\kern2pt]} \hypertarget{w7\string_Bb1}{}
\markboth{\textcolor{darkblue}{\textbf{\ipa{wɤ˩\textsubscript{b}}}}}{}
\textcolor{teal}{\zh{量词}} \hspace{4pt} \zh{声调类:} L\textsubscript{b}.
\zh{量词:担,负荷。} \textcolor{Sepia}{\selectlanguage{english}Load, charge, weight.} \textcolor{PineGreen}{\selectlanguage{french}Classificateur des charges/fardeaux qu'une personne peut porter.}  ¶ \textcolor{darkblue}{\textbf{\ipa{ɖɯ˧-wɤ˩ pɤ˩\textasciitilde{}pɤ˩ |}}} \zh{背一担} \textcolor{Sepia}{\selectlanguage{english}to carry a load} \textcolor{PineGreen}{\selectlanguage{french}porter une charge}  
 ¶ \textcolor{darkblue}{\textbf{\ipa{ɖɯ˧-wɤ˩, | ɖɯ˧-wɤ˩ | le˧-kʰɯ˩\textasciitilde{}kʰɯ˩ | tʰi˧-tɕɯ˥ |}}} \zh{将驮的大包堆起来} \textcolor{Sepia}{\selectlanguage{english}to pile up loads, one after the other} \textcolor{PineGreen}{\selectlanguage{french}entasser les charges, l'une après l'autre}  

\lhead{\firstmark}
\rhead{\botmark}

\subsection{\hspace{-0.5cm} {\Large \textcolor{darkblue}{\textbf{\ipa{wɤ˩\textasciitilde{}wɤ˩}}}}\hspace{0.5cm}[\kern2pt{\textcolor{darkblue}{\textbf{\ipa{wɤ˩wɤ˥}}}}\kern2pt]} \hypertarget{w7\string_B~w7\string_B1}{}
\markboth{\textcolor{darkblue}{\textbf{\ipa{wɤ˩\textasciitilde{}wɤ˩}}}}{}
\textcolor{teal}{\zh{动词}} \hspace{4pt} \zh{声调类:} L.
\zh{绕过。} \textcolor{Sepia}{\selectlanguage{english}To detour past, to bypass.} \textcolor{PineGreen}{\selectlanguage{french}Contourner.}  ¶ \textcolor{darkblue}{\textbf{\ipa{le˧-wɤ˩-ze˩}}} \zh{绕了} \textcolor{Sepia}{\selectlanguage{english}\mytextsc{accomp} \string_ \mytextsc{pfv}} \textcolor{PineGreen}{\selectlanguage{french}\mytextsc{accomp} \string_ pfv}  
 ¶ \textcolor{darkblue}{\textbf{\ipa{ɖɯ˧-wɤ˩\textasciitilde{}wɤ˩-ɻ̍˩}}} \zh{绕一绕} \textcolor{Sepia}{\selectlanguage{english}\mytextsc{delimitative} \string_ \mytextsc{red} \mytextsc{inceptive}} \textcolor{PineGreen}{\selectlanguage{french}\mytextsc{délimitatif} \string_ \mytextsc{red} \mytextsc{inchoatif}}  
 ¶ \textcolor{darkblue}{\textbf{\ipa{[PHONO] wɤ˩\textasciitilde{}wɤ˩ bi˩˥}}} \zh{\mytextsc{imm}\string_fut} \textcolor{Sepia}{\selectlanguage{english}\mytextsc{imm}\string_fut} \textcolor{PineGreen}{\selectlanguage{french}\mytextsc{fut}\string_imm}  
 ¶ \textcolor{darkblue}{\textbf{\ipa{[PHONO] wɤ˩\textasciitilde{}wɤ˩-ze˥}}} \zh{绕了} \textcolor{Sepia}{\selectlanguage{english}\mytextsc{pfv}} \textcolor{PineGreen}{\selectlanguage{french}\mytextsc{pfv}}  
 ¶ \textcolor{darkblue}{\textbf{\ipa{le˧-wɤ˩\textasciitilde{}wɤ˩ | le˧-se˥}}} \zh{走路绕过} \textcolor{Sepia}{\selectlanguage{english}to bypass on foot; to walk past, bypassing (a certain place)} \textcolor{PineGreen}{\selectlanguage{french}contourner à pied}  

\lhead{\firstmark}
\rhead{\botmark}

\subsection{\hspace{-0.5cm} {\Large \textcolor{darkblue}{\textbf{\ipa{wɤ˩˥}}}}\hspace{0.5cm}[\kern2pt{\textcolor{darkblue}{\textbf{\ipa{wɤ˥}}}}\kern2pt]} \hypertarget{w7\string_B\string_T1}{}
\markboth{\textcolor{darkblue}{\textbf{\ipa{wɤ˩˥}}}}{}
\textcolor{teal}{\zh{助词}} \hspace{4pt} \zh{声调类:} LM? LH?.
\zh{又,再。} \textcolor{Sepia}{\selectlanguage{english}Again; also.} \textcolor{PineGreen}{\selectlanguage{french}À nouveau, encore; aussi.}  ¶ \textcolor{darkblue}{\textbf{\ipa{wɤ˩˥ | ɖɯ˧-ʂɯ˩}}} \zh{再一次、又一次} \textcolor{Sepia}{\selectlanguage{english}once again, once more, one more time} \textcolor{PineGreen}{\selectlanguage{french}une nouvelle fois, une fois de plus}  

\lhead{\firstmark}
\rhead{\botmark}

\subsection{\hspace{-0.5cm} {\Large \textcolor{darkblue}{\textbf{\ipa{wo˥}}}}\hspace{0.5cm}[\kern2pt{\textcolor{darkblue}{\textbf{\ipa{wo˧˥}}}}\kern2pt]} \hypertarget{wo\string_T1}{}
\markboth{\textcolor{darkblue}{\textbf{\ipa{wo˥}}}}{}
\textcolor{teal}{\zh{形容词}} \hspace{4pt} \zh{声调类:} H.
\zh{硬,坚硬,结实。} \textcolor{Sepia}{\selectlanguage{english}Hard, solid, resilient.} \textcolor{PineGreen}{\selectlanguage{french}Dur, solide, résistant.}  ¶ \textcolor{darkblue}{\textbf{\ipa{le˧-wo˥-ze˩}}} \zh{硬了} \textcolor{Sepia}{\selectlanguage{english}\mytextsc{accomp} \string_ \mytextsc{pfv}: it hardened} \textcolor{PineGreen}{\selectlanguage{french}\mytextsc{accomp} \string_ \mytextsc{pfv}: ça a durci, c'est devenu dur}  

\lhead{\firstmark}
\rhead{\botmark}

\subsection{\hspace{-0.5cm} {\Large \textcolor{darkblue}{\textbf{\ipa{wo˩\textsubscript{b}}}}}\hspace{0.5cm}[\kern2pt{\textcolor{darkblue}{\textbf{\ipa{wo˩˥}}}}\kern2pt]} \hypertarget{wo\string_Bb1}{}
\markboth{\textcolor{darkblue}{\textbf{\ipa{wo˩\textsubscript{b}}}}}{}
\textcolor{teal}{\zh{量词}} \hspace{4pt} \zh{声调类:} L\textsubscript{b}.
\zh{量词:牛(一架)。} \textcolor{Sepia}{\selectlanguage{english}Classifier for teams of oxen. In Yongning, the ard is drawn by two oxen, or two small water buffaloes, or one strong water buffalo.} \textcolor{PineGreen}{\selectlanguage{french}Classificateur des paires de bœufs; attelage de bœufs pour tirer l'araire. A Yongning, l'attelage comporte deux bœufs, ou deux petits buffles, ou un seul buffle vigoureux.}  ¶ \textcolor{darkblue}{\textbf{\ipa{dʑi˧mi˧ | ɲi˧-pʰo˧˥, | ɖɯ˧-wo˩!}}} \zh{两头水牛,等于一架!} \textcolor{Sepia}{\selectlanguage{english}Two water buffaloes make up one team!} \textcolor{PineGreen}{\selectlanguage{french}Deux buffles, cela forme un attelage!}  

\lhead{\firstmark}
\rhead{\botmark}

\subsection{\hspace{-0.5cm} {\Large \textcolor{darkblue}{\textbf{\ipa{wo˩kɤ\#˥}}}}\hspace{0.5cm}[\kern2pt{\textcolor{darkblue}{\textbf{\ipa{wo˩kɤ˩˥}}}}\kern2pt]} \hypertarget{wo\string_Bk7\#\string_T1}{}
\markboth{\textcolor{darkblue}{\textbf{\ipa{wo˩kɤ\#˥}}}}{}
\textcolor{teal}{\zh{名词}} \hspace{4pt} \zh{声调类:} LM+\#H.
\zh{秋千(鞦韆)。} \textcolor{Sepia}{\selectlanguage{english}Swing.} \textcolor{PineGreen}{\selectlanguage{french}Balançoire.}  ¶ \textcolor{darkblue}{\textbf{\ipa{wo˩kɤ˧-tsɑ˧-di˧˥}}} \zh{同上:秋千} \textcolor{Sepia}{\selectlanguage{english}same meaning: swing} \textcolor{PineGreen}{\selectlanguage{french}même sens: balançoire}  
 ¶ \textcolor{darkblue}{\textbf{\ipa{wo˩kɤ˧ tsɑ˧˥}}} \zh{同上:秋千} \textcolor{Sepia}{\selectlanguage{english}same meaning: swing} \textcolor{PineGreen}{\selectlanguage{french}même sens: balançoire}  
 \zh{量词}: \textcolor{darkblue}{\textbf{\ipa{nɑ˧}}} 
\lhead{\firstmark}
\rhead{\botmark}

\subsection{\hspace{-0.5cm} {\Large \textcolor{darkblue}{\textbf{\ipa{wo˧˥}}}}\hspace{0.5cm}[\kern2pt{\textcolor{darkblue}{\textbf{\ipa{wo˩˥}}}}\kern2pt]} \hypertarget{wo\string_M\string_T1}{}
\markboth{\textcolor{darkblue}{\textbf{\ipa{wo˧˥}}}}{}
\textcolor{teal}{\zh{动词}} \hspace{4pt} \zh{声调类:} MH.
\zh{重新做、再来做。} \textcolor{Sepia}{\selectlanguage{english}To do (something) over again.} \textcolor{PineGreen}{\selectlanguage{french}Se retourner (quelqu'un est assis et se retourne: mouvement du torse).}  ¶ \textcolor{darkblue}{\textbf{\ipa{le˧-wo˧ ʐwɤ˧˥}}} \zh{回答} \textcolor{Sepia}{\selectlanguage{english}to answer} \textcolor{PineGreen}{\selectlanguage{french}répondre, donner une réponse}  
 ¶ \textcolor{darkblue}{\textbf{\ipa{le˧-wo˧-ɻ̍˥}}} \zh{转身} \textcolor{Sepia}{\selectlanguage{english}to turn around (e.g. in order to look back)} \textcolor{PineGreen}{\selectlanguage{french}se retourner}  
 ¶ \textcolor{darkblue}{\textbf{\ipa{le˧-wo˧ li˥}}} \zh{往后看} \textcolor{Sepia}{\selectlanguage{english}to look back} \textcolor{PineGreen}{\selectlanguage{french}regarder derrière soi}  
 ¶ \textcolor{darkblue}{\textbf{\ipa{lə-˧wo˧ tʰo˥-tɕo˩}}} \zh{转身} \textcolor{Sepia}{\selectlanguage{english}to turn around (e.g. in order to look back)} \textcolor{PineGreen}{\selectlanguage{french}se retourner}  
 ¶ \textcolor{darkblue}{\textbf{\ipa{le˧-wo˧-tɕo˥!}}} \zh{转身!(婴儿爬下床,头朝下。奶奶告诉她:要先转身)} \textcolor{Sepia}{\selectlanguage{english}Turn around! (Said to a baby who is about to get down a bed head first)} \textcolor{PineGreen}{\selectlanguage{french}retourne-toi! (adressé à un bébé qui s'apprête à descendre d'un lit la tête la première)}  
 ¶ \textcolor{darkblue}{\textbf{\ipa{le˧-wo˧˥ | le˧-hɯ˩}}} \zh{回去了} \textcolor{Sepia}{\selectlanguage{english}has gone back, went back} \textcolor{PineGreen}{\selectlanguage{french}...est reparti}  

\lhead{\firstmark}
\rhead{\botmark}

\subsection{\hspace{-0.5cm} {\Large \textcolor{darkblue}{\textbf{\ipa{wo˩˥}}}}\hspace{0.5cm}[\kern2pt{\textcolor{darkblue}{\textbf{\ipa{wo˥}}}}\kern2pt]} \hypertarget{wo\string_B\string_T1}{}
\markboth{\textcolor{darkblue}{\textbf{\ipa{wo˩˥}}}}{}
\textcolor{teal}{\zh{名词}} \hspace{4pt} \zh{声调类:} LH.
\zh{圆根的叶子。} \textcolor{Sepia}{\selectlanguage{english}Turnip leaves; they used to be eaten as a vegetable.} \textcolor{PineGreen}{\selectlanguage{french}Feuilles du navet.}  ¶ \textcolor{darkblue}{\textbf{\ipa{wo˩bɤ˧˥}}} \zh{同上:圆根叶子} \textcolor{Sepia}{\selectlanguage{english}same meaning: turnip leaves} \textcolor{PineGreen}{\selectlanguage{french}même sens: feuilles du navet}  
 ¶ \textcolor{darkblue}{\textbf{\ipa{[Housebuilding2] wo˩-v˥tsʰɤ˩}}} \zh{同上:圆根叶子} \textcolor{Sepia}{\selectlanguage{english}same meaning: turnip leaves; literally 'turnip leaves vegetable', emphasizing the fact that they are used as a vegetable: as an ingredient in a recipe} \textcolor{PineGreen}{\selectlanguage{french}même sens: feuilles du navet; littéralement 'légume-feuilles du navet'; l'expression souligne qu'il s'agit d'une variété de légume: d'un ingrédient pour la cuisine}  
 ¶ \textcolor{darkblue}{\textbf{\ipa{[Housebuilding2] wo˩-tɕæ˩ɻæ˥}}} \zh{圆根叶子酸菜} \textcolor{Sepia}{\selectlanguage{english}pickled turnip leaves} \textcolor{PineGreen}{\selectlanguage{french}feuilles du navet conservées dans la saumure}  

\lhead{\firstmark}
\rhead{\botmark}

\newpage
\section*{\centering- \textcolor{darkblue}{\textbf{\ipa{w̃}}} \textcolor{darkblue}{\textbf{\ipa{w̃æ}}} -}
\subsection{\hspace{-0.5cm} {\Large \textcolor{darkblue}{\textbf{\ipa{w̃æ˧}}}}\hspace{0.5cm}[\kern2pt{\textcolor{darkblue}{\textbf{\ipa{w̃æ˥}}}}\kern2pt]} \hypertarget{w\string_~\{\string_M1}{}
\markboth{\textcolor{darkblue}{\textbf{\ipa{w̃æ˧}}}}{}
\textcolor{teal}{\zh{动词}} \hspace{4pt} \zh{声调类:} M intrans.
\zh{肿,膨胀,(肚子)胀。} \textcolor{Sepia}{\selectlanguage{english}To swell, to inflate (e.g. the belly is swollen).} \textcolor{PineGreen}{\selectlanguage{french}Se gonfler, enfler (ventre).}  ¶ \textcolor{darkblue}{\textbf{\ipa{ɻ̍˧tɑ˧ w̃æ˧ (-ze˧)}}} \zh{淋巴结肿了} \textcolor{Sepia}{\selectlanguage{english}glands are swollen} \textcolor{PineGreen}{\selectlanguage{french}les ganglions sont enflés}  
 ¶ \textcolor{darkblue}{\textbf{\ipa{tso˧\textasciitilde{}tso˧ w̃æ˩}}} \zh{东西膨胀了} \textcolor{Sepia}{\selectlanguage{english}something has swollen} \textcolor{PineGreen}{\selectlanguage{french}quelque chose a enflé}  

\lhead{\firstmark}
\rhead{\botmark}

\newpage
\section*{\centering- \textcolor{darkblue}{\textbf{\ipa{z}}} -}
\subsection{\hspace{-0.5cm} {\Large \textcolor{darkblue}{\textbf{\ipa{zɑ˥}}}}\hspace{0.5cm}[\kern2pt{\textcolor{darkblue}{\textbf{\ipa{xxxx groupe tonal entier sans aucun ton}}}}\kern2pt]} \hypertarget{zA\string_T1}{}
\markboth{\textcolor{darkblue}{\textbf{\ipa{zɑ˥}}}}{}
\textcolor{teal}{\zh{形容词}} \hspace{4pt} \zh{声调类:} H.
\zh{仅仅。} \textcolor{Sepia}{\selectlanguage{english}Restricted to, limited to.} \textcolor{PineGreen}{\selectlanguage{french}Limité à, restreint à (en tournure négative).}  ¶ \textcolor{darkblue}{\textbf{\ipa{ʁwɤ˧-qo˧-ɳɯ˧-lɑ˧ mɤ˧-zɑ˥ (…)}}} \zh{不仅有村子里的人} \textcolor{Sepia}{\selectlanguage{english}not only the people from the village} \textcolor{PineGreen}{\selectlanguage{french}Il n'y avait pas que les gens du village (…) (récit “Funeral”)}  

\lhead{\firstmark}
\rhead{\botmark}

\subsection{\hspace{-0.5cm} {\Large \textcolor{darkblue}{\textbf{\ipa{zɑ˧ɭɯ˧}}}}\hspace{0.5cm}[\kern2pt{\textcolor{darkblue}{\textbf{\ipa{xxxx non-correspondance entre le nombre de morphèmes et le nombre de tons de morphèmes}}}}\kern2pt]} \hypertarget{zA\string_Ml\string_RM\string_M1}{}
\markboth{\textcolor{darkblue}{\textbf{\ipa{zɑ˧ɭɯ˧}}}}{}
\textcolor{teal}{\zh{名词}} \hspace{4pt} \zh{声调类:} M.
\zh{阉猪。} \textcolor{Sepia}{\selectlanguage{english}Barrow, castrated male pig, neutered pig.} \textcolor{PineGreen}{\selectlanguage{french}Porc castré.}  \zh{量词}: \textcolor{darkblue}{\textbf{\ipa{pʰo˧˥}}} \textcolor{darkblue}{\textbf{\ipa{v̩˧}}} 
\lhead{\firstmark}
\rhead{\botmark}

\subsection{\hspace{-0.5cm} {\Large \textcolor{darkblue}{\textbf{\ipa{zɑ˧zɑ˧}}}}\hspace{0.5cm}[\kern2pt{\textcolor{darkblue}{\textbf{\ipa{zɑ˧zɑ˧}}}}\kern2pt]} \hypertarget{zA\string_MzA\string_M1}{}
\markboth{\textcolor{darkblue}{\textbf{\ipa{zɑ˧zɑ˧}}}}{}
\textcolor{teal}{\zh{形容词}} \hspace{4pt} \zh{声调类:} M.
\zh{细心、细致。} \textcolor{Sepia}{\selectlanguage{english}Careful.} \textcolor{PineGreen}{\selectlanguage{french}Attentif, soigneux.} 
\lhead{\firstmark}
\rhead{\botmark}

\subsection{\hspace{-0.5cm} {\Large \textcolor{darkblue}{\textbf{\ipa{zɑ˩\textsubscript{a}}}}}\hspace{0.5cm}[\kern2pt{\textcolor{darkblue}{\textbf{\ipa{zɑ˥}}}}\kern2pt]} \hypertarget{zA\string_Ba1}{}
\markboth{\textcolor{darkblue}{\textbf{\ipa{zɑ˩\textsubscript{a}}}}}{}
\textcolor{teal}{\zh{动词}} \hspace{4pt} \zh{声调类:} L\textsubscript{a}.
\zh{下(山……)。} \textcolor{Sepia}{\selectlanguage{english}To go downward (a mountain), to descend.} \textcolor{PineGreen}{\selectlanguage{french}Descendre (redescendre de la montagne).}  ¶ \textcolor{darkblue}{\textbf{\ipa{ʁwɤ˩ zɑ˩˥}}} \zh{下山} \textcolor{Sepia}{\selectlanguage{english}to go down the mountain} \textcolor{PineGreen}{\selectlanguage{french}descendre de la montagne}  
 ¶ \textcolor{darkblue}{\textbf{\ipa{mɤ˧-zɑ˩-sɯ˩}}} \zh{还没下来} \textcolor{Sepia}{\selectlanguage{english}not to go down yet} \textcolor{PineGreen}{\selectlanguage{french}ne pas descendre encore}  
 ¶ \textcolor{darkblue}{\textbf{\ipa{ɖɯ˧-zɑ˧\textasciitilde{}zɑ˥-ɻ̍˩}}} \zh{下来一下} \textcolor{Sepia}{\selectlanguage{english}\mytextsc{delimitative} \string_ \mytextsc{red} \mytextsc{inceptive}} \textcolor{PineGreen}{\selectlanguage{french}\mytextsc{délimitatif} \string_ \mytextsc{red} \mytextsc{inchoatif}}  

\lhead{\firstmark}
\rhead{\botmark}

\subsection{\hspace{-0.5cm} {\Large \textcolor{darkblue}{\textbf{\ipa{zɑ˩-bɑ˧lɑ˩}}}}\hspace{0.5cm}[\kern2pt{\textcolor{darkblue}{\textbf{\ipa{xxxx non-correspondance entre le nombre de morphèmes et le nombre de tons de morphèmes}}}}\kern2pt]} \hypertarget{zA\string_B-bA\string_MlA\string_B1}{}
\markboth{\textcolor{darkblue}{\textbf{\ipa{zɑ˩-bɑ˧lɑ˩}}}}{}
\textcolor{teal}{\zh{名词}} \hspace{4pt} \zh{声调类:} L-L\#.
\ding{202} \zh{火塘旁边墙上的壁画(唐卡:内容来自藏传佛教)。} \textcolor{Sepia}{\selectlanguage{english}Religious painting (thangka) on wood, on the wall next to the hearth.} \textcolor{PineGreen}{\selectlanguage{french}Peinture religieuse (thangka) sur bois, dans une niche sur le mur, au-dessus du foyer.} \ding{203} \zh{火,火塘与家的神。} \textcolor{Sepia}{\selectlanguage{english}Divinity of fire, of the hearth, and of the house.} \textcolor{PineGreen}{\selectlanguage{french}Divinité du feu, du foyer et de la maison.} 
\lhead{\firstmark}
\rhead{\botmark}

\subsection{\hspace{-0.5cm} {\Large \textcolor{darkblue}{\textbf{\ipa{zɑ˩ɲi˥-ʂɤ˩}}}}\hspace{0.5cm}[\kern2pt{\textcolor{darkblue}{\textbf{\ipa{xxxx non-correspondance entre le nombre de morphèmes et le nombre de tons de morphèmes}}}}\kern2pt]} \hypertarget{zA\string_BJi\string_T-s`7\string_B1}{}
\markboth{\textcolor{darkblue}{\textbf{\ipa{zɑ˩ɲi˥-ʂɤ˩}}}}{}
\textcolor{teal}{\zh{名词}} \hspace{4pt} \zh{声调类:} LH-.
\zh{吸血鬼。} \textcolor{Sepia}{\selectlanguage{english}Vampire: a demon of human shape (the size of a large person), who feeds on blood.} \textcolor{PineGreen}{\selectlanguage{french}Vampire; démon malfaisant de forme humaine (de la taille d'un humain de grande taille), qui ne mange pas de viande, et se nourrit de sang.} 
\lhead{\firstmark}
\rhead{\botmark}

\subsection{\hspace{-0.5cm} {\Large \textcolor{darkblue}{\textbf{\ipa{‑ze˧}}}}\hspace{0.5cm}[\kern2pt{\textcolor{darkblue}{\textbf{\ipa{ze˥}}}}\kern2pt]} \hypertarget{‑ze\string_M1}{}
\markboth{\textcolor{darkblue}{\textbf{\ipa{‑ze˧}}}}{}
\textcolor{teal}{\zh{后缀}} \hspace{4pt} \zh{声调类:} M.
\zh{\mytextsc{整体体。}} \textcolor{Sepia}{\selectlanguage{english}Perfective, \mytextsc{pfv}.} \textcolor{PineGreen}{\selectlanguage{french}Perfectif, \mytextsc{pfv}.} 
\lhead{\firstmark}
\rhead{\botmark}

\subsection{\hspace{-0.5cm} {\Large \textcolor{darkblue}{\textbf{\ipa{ze˩}}}}\hspace{0.5cm}[\kern2pt{\textcolor{darkblue}{\textbf{\ipa{ze˩˥}}}}\kern2pt]} \hypertarget{ze\string_B1}{}
\markboth{\textcolor{darkblue}{\textbf{\ipa{ze˩}}}}{}
\textcolor{teal}{\zh{代词}} \hspace{4pt} \zh{声调类:} L.
\zh{哪。} \textcolor{Sepia}{\selectlanguage{english}Which.} \textcolor{PineGreen}{\selectlanguage{french}Quel, lequel.} 
\lhead{\firstmark}
\rhead{\botmark}

\subsection{\hspace{-0.5cm} {\Large \textcolor{darkblue}{\textbf{\ipa{ze˩bæ˧}}}}\hspace{0.5cm}[\kern2pt{\textcolor{darkblue}{\textbf{\ipa{ze˩bæ˥}}}}\kern2pt]} \hypertarget{ze\string_Bb\{\string_M1}{}
\markboth{\textcolor{darkblue}{\textbf{\ipa{ze˩bæ˧}}}}{}
\textcolor{teal}{\zh{代词}} \hspace{4pt} \zh{声调类:} LM.
\zh{哪,哪个 (哪个碗),哪一种。} \textcolor{Sepia}{\selectlanguage{english}Which; which kind.} \textcolor{PineGreen}{\selectlanguage{french}Quelle sorte de, lequel.}  ¶ \textcolor{darkblue}{\textbf{\ipa{ze˩bæ˧ ɲi˥?}}} \zh{是哪个?是哪一样?} \textcolor{Sepia}{\selectlanguage{english}Which one is it? / Which kind is it?} \textcolor{PineGreen}{\selectlanguage{french}c'est lequel?/c'est de quelle sorte?}  

\lhead{\firstmark}
\rhead{\botmark}

\subsection{\hspace{-0.5cm} {\Large \textcolor{darkblue}{\textbf{\ipa{ze˩bæ˩}}}}\hspace{0.5cm}[\kern2pt{\textcolor{darkblue}{\textbf{\ipa{ze˩bæ˩˥}}}}\kern2pt]} \hypertarget{ze\string_Bb\{\string_B1}{}
\markboth{\textcolor{darkblue}{\textbf{\ipa{ze˩bæ˩}}}}{}
\textcolor{teal}{\zh{名词}} \hspace{4pt} \zh{声调类:} L.
\zh{闪电、打闪电、霹雷。} \textcolor{Sepia}{\selectlanguage{english}Flash of lightning, thunderbolt.} \textcolor{PineGreen}{\selectlanguage{french}Éclair.}  ¶ \textcolor{darkblue}{\textbf{\ipa{ze˩bæ˩-ze˥!}}} \zh{打闪电了!} \textcolor{Sepia}{\selectlanguage{english}There has been a flash of lightning!} \textcolor{PineGreen}{\selectlanguage{french}il y a eu un éclair!}  
 ¶ \textcolor{darkblue}{\textbf{\ipa{ze˩bæ˩˥ | -dʑo˩!}}} \zh{打着闪电!} \textcolor{Sepia}{\selectlanguage{english}There are flashes of lightning!} \textcolor{PineGreen}{\selectlanguage{french}il y a des éclairs!}  
 \zh{量词}: \textcolor{darkblue}{\textbf{\ipa{bæ˩}}} 
\lhead{\firstmark}
\rhead{\botmark}

\subsection{\hspace{-0.5cm} {\Large \textcolor{darkblue}{\textbf{\ipa{ze˩gɤ˧}}}}\hspace{0.5cm}[\kern2pt{\textcolor{darkblue}{\textbf{\ipa{ze˩gɤ˥}}}}\kern2pt]} \hypertarget{ze\string_Bg7\string_M1}{}
\markboth{\textcolor{darkblue}{\textbf{\ipa{ze˩gɤ˧}}}}{}
\textcolor{teal}{\zh{代词}} \hspace{4pt} \zh{声调类:} LM.
\zh{哪里,什么地方。} \textcolor{Sepia}{\selectlanguage{english}At which place, where.} \textcolor{PineGreen}{\selectlanguage{french}Où, à quel endroit.} 
\lhead{\firstmark}
\rhead{\botmark}

\subsection{\hspace{-0.5cm} {\Large \textcolor{darkblue}{\textbf{\ipa{ze˩mi˩}}}}\hspace{0.5cm}[\kern2pt{\textcolor{darkblue}{\textbf{\ipa{ze˩mi˩˥}}}}\kern2pt]} \hypertarget{ze\string_Bmi\string_B1}{}
\markboth{\textcolor{darkblue}{\textbf{\ipa{ze˩mi˩}}}}{}
\textcolor{teal}{\zh{名词}} \hspace{4pt} \zh{声调类:} L.
\zh{甥女(姐妹的女儿)。} \textcolor{Sepia}{\selectlanguage{english}Niece.} \textcolor{PineGreen}{\selectlanguage{french}Nièce (enfant d'une soeur).}  \zh{量词}: \textcolor{darkblue}{\textbf{\ipa{v̩˧}}} 
\lhead{\firstmark}
\rhead{\botmark}

\subsection{\hspace{-0.5cm} {\Large \textcolor{darkblue}{\textbf{\ipa{ze˩v̩˩}}}}\hspace{0.5cm}[\kern2pt{\textcolor{darkblue}{\textbf{\ipa{ze˩v̩˩˥}}}}\kern2pt]} \hypertarget{ze\string_Bv\string_=\string_B1}{}
\markboth{\textcolor{darkblue}{\textbf{\ipa{ze˩v̩˩}}}}{}
\textcolor{teal}{\zh{名词}} \hspace{4pt} \zh{声调类:} L.
\zh{外甥(姐妹的儿子)。} \textcolor{Sepia}{\selectlanguage{english}Nephew (son of one's sister).} \textcolor{PineGreen}{\selectlanguage{french}Neveu (fils d'une sœur).}  \zh{量词}: \textcolor{darkblue}{\textbf{\ipa{v̩˧}}} 
\lhead{\firstmark}
\rhead{\botmark}

\subsection{\hspace{-0.5cm} {\Large \textcolor{darkblue}{\textbf{\ipa{ze˩v̩˩-ze˧mi˩}}}}\hspace{0.5cm}[\kern2pt{\textcolor{darkblue}{\textbf{\ipa{ze˩v̩˩ze˧mi˩}}}}\kern2pt]} \hypertarget{ze\string_Bv\string_=\string_B-ze\string_Mmi\string_B1}{}
\markboth{\textcolor{darkblue}{\textbf{\ipa{ze˩v̩˩-ze˧mi˩}}}}{}
\textcolor{teal}{\zh{名词}} \hspace{4pt} \zh{声调类:} L-L\#.
\zh{外甥甥女(姐妹的儿女)。} \textcolor{Sepia}{\selectlanguage{english}Nephews and nieces.} \textcolor{PineGreen}{\selectlanguage{french}Neveux et nièces (du côté des sœurs: enfants des sœurs).} 
\lhead{\firstmark}
\rhead{\botmark}

\subsection{\hspace{-0.5cm} {\Large \textcolor{darkblue}{\textbf{\ipa{zo˥}}}}\hspace{0.5cm}[\kern2pt{\textcolor{darkblue}{\textbf{\ipa{zo˩˥}}}}\kern2pt]} \hypertarget{zo\string_T1}{}
\markboth{\textcolor{darkblue}{\textbf{\ipa{zo˥}}}}{}
\textcolor{teal}{\zh{名词}} \hspace{4pt} \zh{声调类:} \#H.
\ding{202} \zh{儿子。} \textcolor{Sepia}{\selectlanguage{english}Son.} \textcolor{PineGreen}{\selectlanguage{french}Fils.}  ¶ \textcolor{darkblue}{\textbf{\ipa{zo˧ ɲi˥-kv̩˩}}} \zh{两个儿子} \textcolor{Sepia}{\selectlanguage{english}two sons} \textcolor{PineGreen}{\selectlanguage{french}deux fils}  
 \zh{量词}: \textcolor{darkblue}{\textbf{\ipa{v̩˧}}} \ding{203} \zh{男人。} \textcolor{Sepia}{\selectlanguage{english}Man, \textit{Vir}.} \textcolor{PineGreen}{\selectlanguage{french}Homme, \textit{Vir}.} 
\lhead{\firstmark}
\rhead{\botmark}

\subsection{\hspace{-0.5cm} {\Large \textcolor{darkblue}{\textbf{\ipa{‑zo˧}}}}\hspace{0.5cm}[\kern2pt{\textcolor{darkblue}{\textbf{\ipa{zo˥}}}}\kern2pt]} \hypertarget{‑zo\string_M1}{}
\markboth{\textcolor{darkblue}{\textbf{\ipa{‑zo˧}}}}{}
\textcolor{teal}{\zh{后缀}} \hspace{4pt} \zh{声调类:} M.
\zh{应该、必须。} \textcolor{Sepia}{\selectlanguage{english}Obligative.} \textcolor{PineGreen}{\selectlanguage{french}Obligatif.} 
\lhead{\firstmark}
\rhead{\botmark}

\subsection{\hspace{-0.5cm} {\Large \textcolor{darkblue}{\textbf{\ipa{zo˧\textsubscript{a}}}}}\hspace{0.5cm}[\kern2pt{\textcolor{darkblue}{\textbf{\ipa{zo˩˥}}}}\kern2pt]} \hypertarget{zo\string_Ma1}{}
\markboth{\textcolor{darkblue}{\textbf{\ipa{zo˧\textsubscript{a}}}}}{}
\textcolor{teal}{\zh{动词}} \hspace{4pt} \zh{声调类:} M\textsubscript{a}.
\zh{要,应该。} \textcolor{Sepia}{\selectlanguage{english}To have to, to be necessary.} \textcolor{PineGreen}{\selectlanguage{french}Devoir.}  ¶ \textcolor{darkblue}{\textbf{\ipa{mɤ˧-zo˧ (-ze˧)! | tʰi˧-kwɤ˩-kʰɯ˩!}}} \zh{不用了!算了吧!} \textcolor{Sepia}{\selectlanguage{english}It's not necessary! Forget it!} \textcolor{PineGreen}{\selectlanguage{french}ce n'est pas la peine! laisse tomber!}  
 ¶ \textcolor{darkblue}{\textbf{\ipa{ʈʂʰɯ˧ne˧-ʝi˥ | ʝi˧-zo˧-ho˥-ɲi˩!}}} \zh{是应该这样做的!} \textcolor{Sepia}{\selectlanguage{english}That's how one must do! / That's how it's done!} \textcolor{PineGreen}{\selectlanguage{french}Il faut faire comme ça!}  

\lhead{\firstmark}
\rhead{\botmark}

\subsection{\hspace{-0.5cm} {\Large \textcolor{darkblue}{\textbf{\ipa{zo˧bæ˩}}}}\hspace{0.5cm}[\kern2pt{\textcolor{darkblue}{\textbf{\ipa{zo˧bæ˧}}}}\kern2pt]} \hypertarget{zo\string_Mb\{\string_B1}{}
\markboth{\textcolor{darkblue}{\textbf{\ipa{zo˧bæ˩}}}}{}
\textcolor{teal}{\zh{名词}} \hspace{4pt} \zh{声调类:} L\#.
\zh{笨人、傻瓜。} \textcolor{Sepia}{\selectlanguage{english}Fool, idiot.} \textcolor{PineGreen}{\selectlanguage{french}Imbécile, idiot.}  ¶ \textcolor{darkblue}{\textbf{\ipa{mɤ˧-zo˧bæ˩!}}} \zh{(你)不是笨蛋!(情景:一个人批评自己是笨蛋,人家安慰他。)} \textcolor{Sepia}{\selectlanguage{english}No, (you) are not an idiot! (A reassuring answer to someone who deprecates himself as an idiot.)} \textcolor{PineGreen}{\selectlanguage{french}(Non, tu n'es) pas idiot(e)! (Propos rassurant adressé à un interlocuteur accablé par ses propres maladresses.)}  
 ¶ \textcolor{darkblue}{\textbf{\ipa{zo˧bæ˩-mv̩˩bæ˩}}} \zh{傻瓜们(不分男女)} \textcolor{Sepia}{\selectlanguage{english}silly people, idiots (of both sexes)} \textcolor{PineGreen}{\selectlanguage{french}idiots, imbéciles (sans distinction de sexe)}  
 \zh{量词}: \textcolor{darkblue}{\textbf{\ipa{v̩˧}}} 
\lhead{\firstmark}
\rhead{\botmark}

\subsection{\hspace{-0.5cm} {\Large \textcolor{darkblue}{\textbf{\ipa{zo˧ɖɯ\#˥}}}}\hspace{0.5cm}[\kern2pt{\textcolor{darkblue}{\textbf{\ipa{zo˧ɖɯ˧}}}}\kern2pt]} \hypertarget{zo\string_Md`M\#\string_T1}{}
\markboth{\textcolor{darkblue}{\textbf{\ipa{zo˧ɖɯ\#˥}}}}{}
\textcolor{teal}{\zh{名词}} \hspace{4pt} \zh{声调类:} \#H.
\zh{大儿子。} \textcolor{Sepia}{\selectlanguage{english}Eldest son.} \textcolor{PineGreen}{\selectlanguage{french}Fils aîné.}  ¶ \textcolor{darkblue}{\textbf{\ipa{zo˧ɖɯ˧-mv̩˥ɖɯ˩}}} \zh{大儿子与大女儿} \textcolor{Sepia}{\selectlanguage{english}eldest son and eldest daughter} \textcolor{PineGreen}{\selectlanguage{french}fils et fille aînés}  

\lhead{\firstmark}
\rhead{\botmark}

\subsection{\hspace{-0.5cm} {\Large \textcolor{darkblue}{\textbf{\ipa{zo˧hṽ˧-mv̩˥zo˩}}}}\hspace{0.5cm}[\kern2pt{\textcolor{darkblue}{\textbf{\ipa{zo˧hṽ˧˥mv̩˧zo˧}}}}\kern2pt]} \hypertarget{zo\string_Mhv\string_~\string_M-mv\string_=\string_Tzo\string_B1}{}
\markboth{\textcolor{darkblue}{\textbf{\ipa{zo˧hṽ˧-mv̩˥zo˩}}}}{}
\textcolor{teal}{\zh{名词}} \hspace{4pt} \zh{声调类:} MH\#-.
\zh{后代。} \textcolor{Sepia}{\selectlanguage{english}Descendants.} \textcolor{PineGreen}{\selectlanguage{french}Les descendants.}  ¶ \textcolor{darkblue}{\textbf{\ipa{zo˧hṽ˧mv̩˥zo˩=ɻæ˩}}} \zh{\string_ \mytextsc{联想复数}} \textcolor{Sepia}{\selectlanguage{english}\string_ \mytextsc{associative}} \textcolor{PineGreen}{\selectlanguage{french}\string_ \mytextsc{associatif}}  

\lhead{\firstmark}
\rhead{\botmark}

\subsection{\hspace{-0.5cm} {\Large \textcolor{darkblue}{\textbf{\ipa{zo˧hṽ˧˥}}}}\hspace{0.5cm}[\kern2pt{\textcolor{darkblue}{\textbf{\ipa{zo˧hṽ˧˥}}}}\kern2pt]} \hypertarget{zo\string_Mhv\string_~\string_M\string_T1}{}
\markboth{\textcolor{darkblue}{\textbf{\ipa{zo˧hṽ˧˥}}}}{}
\textcolor{teal}{\zh{名词}} \hspace{4pt} \zh{声调类:} MH\#.
\ding{202} \zh{儿子。} \textcolor{Sepia}{\selectlanguage{english}Son.} \textcolor{PineGreen}{\selectlanguage{french}Fils.}  ¶ \textcolor{darkblue}{\textbf{\ipa{zo˧hṽ˧=ɻæ˥}}} \zh{儿子们} \textcolor{Sepia}{\selectlanguage{english}the sons} \textcolor{PineGreen}{\selectlanguage{french}les fils}  
 \zh{量词}: \textcolor{darkblue}{\textbf{\ipa{v̩˧}}} \ding{203} \zh{小伙子、 青年男子。} \textcolor{Sepia}{\selectlanguage{english}Young chap, young lad, young man.} \textcolor{PineGreen}{\selectlanguage{french}Jeune homme, petit gars.} 
\lhead{\firstmark}
\rhead{\botmark}

\subsection{\hspace{-0.5cm} {\Large \textcolor{darkblue}{\textbf{\ipa{zo˧mv̩˥}}}}\hspace{0.5cm}[\kern2pt{\textcolor{darkblue}{\textbf{\ipa{zo˧mv̩˥}}}}\kern2pt]} \hypertarget{zo\string_Mmv\string_=\string_T1}{}
\markboth{\textcolor{darkblue}{\textbf{\ipa{zo˧mv̩˥}}}}{}
\textcolor{teal}{\zh{名词}} \hspace{4pt} \zh{声调类:} H\#.
\zh{孩子。} \textcolor{Sepia}{\selectlanguage{english}Child.} \textcolor{PineGreen}{\selectlanguage{french}Enfant.}  ¶ \textcolor{darkblue}{\textbf{\ipa{zo˧mv̩˥ | æ˧mv̩˥tɕi˩-hĩ˩}}} \zh{新生婴儿} \textcolor{Sepia}{\selectlanguage{english}newborn baby, infant} \textcolor{PineGreen}{\selectlanguage{french}nouveau-né, nourrisson}  
 \zh{量词}: \textcolor{darkblue}{\textbf{\ipa{ɭɯ˧}}} 
\lhead{\firstmark}
\rhead{\botmark}

\subsection{\hspace{-0.5cm} {\Large \textcolor{darkblue}{\textbf{\ipa{zo˧tv̩˧-mv̩˥tv̩˩}}}}\hspace{0.5cm}[\kern2pt{\textcolor{darkblue}{\textbf{\ipa{zo˧tv̩˧mv̩˥tv̩˩}}}}\kern2pt]} \hypertarget{zo\string_Mtv\string_=\string_M-mv\string_=\string_Ttv\string_=\string_B1}{}
\markboth{\textcolor{darkblue}{\textbf{\ipa{zo˧tv̩˧-mv̩˥tv̩˩}}}}{}
\textcolor{teal}{\zh{名词}} \hspace{4pt} \zh{声调类:} \#H-.
\zh{独生子(男女通用)。} \textcolor{Sepia}{\selectlanguage{english}Only child (boy or girl).} \textcolor{PineGreen}{\selectlanguage{french}Enfant unique (fils unique ou fille unique).} 
\lhead{\firstmark}
\rhead{\botmark}

\subsection{\hspace{-0.5cm} {\Large \textcolor{darkblue}{\textbf{\ipa{zo˧tv̩\#˥}}}}\hspace{0.5cm}[\kern2pt{\textcolor{darkblue}{\textbf{\ipa{zo˧tv̩˧}}}}\kern2pt]} \hypertarget{zo\string_Mtv\string_=\#\string_T1}{}
\markboth{\textcolor{darkblue}{\textbf{\ipa{zo˧tv̩\#˥}}}}{}
\textcolor{teal}{\zh{名词}} \hspace{4pt} \zh{声调类:} \#H.
\zh{独生子,独生男孩。} \textcolor{Sepia}{\selectlanguage{english}Only son.} \textcolor{PineGreen}{\selectlanguage{french}Fils unique.}  ¶ \textcolor{darkblue}{\textbf{\ipa{zo˧tv̩˧ ɖɯ˧-v̩˧-lɑ˧ dʑo˧˥!}}} \zh{(她)只有一个独生男孩子!} \textcolor{Sepia}{\selectlanguage{english}(She) just has an only son!} \textcolor{PineGreen}{\selectlanguage{french}(elle) n'a qu'un fils unique!}  
 ¶ \textcolor{darkblue}{\textbf{\ipa{ʂɯ˧-ɬi˧mi˧, | zo˧tv̩˧ ʐɤ˥-tʰɑ˩-se˩!}}} \zh{“七月份,独生子不要上路!”(七月份是大雨季,摩梭人认为七月份的路最不安全:有生命危险)} \textcolor{Sepia}{\selectlanguage{english}“In the seventh month, let not an only son take the road!” (The seventh month is the peak of the rainy season; it was considered as a wrong time for long travels.)} \textcolor{PineGreen}{\selectlanguage{french}“Au septième mois, un fils unique ne doit pas aller par les chemins!” (Le septième mois, au plus fort des grandes pluies, était considéré comme un mois défavorable pour voyager; les voyageurs risquaient d'y rester.)}  

\lhead{\firstmark}
\rhead{\botmark}

\subsection{\hspace{-0.5cm} {\Large \textcolor{darkblue}{\textbf{\ipa{zo˧tʰi˧}}}}\hspace{0.5cm}[\kern2pt{\textcolor{darkblue}{\textbf{\ipa{zo˧tʰi˧}}}}\kern2pt]} \hypertarget{zo\string_Mt\string_hi\string_M1}{}
\markboth{\textcolor{darkblue}{\textbf{\ipa{zo˧tʰi˧}}}}{}
\textcolor{teal}{\zh{名词}} \hspace{4pt} \zh{声调类:} M.
\zh{聪明的人。} \textcolor{Sepia}{\selectlanguage{english}Intelligent person.} \textcolor{PineGreen}{\selectlanguage{french}Personne intelligente.}  ¶ \textcolor{darkblue}{\textbf{\ipa{zo˧tʰi˧ ɖɯ˧-v̩˧}}} \zh{一个聪明的人} \textcolor{Sepia}{\selectlanguage{english}an intelligent person} \textcolor{PineGreen}{\selectlanguage{french}une personne intelligente}  

\lhead{\firstmark}
\rhead{\botmark}

\subsection{\hspace{-0.5cm} {\Large \textcolor{darkblue}{\textbf{\ipa{zo˧tʰi˧}}}}\hspace{0.5cm}[\kern2pt{\textcolor{darkblue}{\textbf{\ipa{zo˧tʰi˧}}}}\kern2pt]} \hypertarget{zo\string_Mt\string_hi\string_M1}{}
\markboth{\textcolor{darkblue}{\textbf{\ipa{zo˧tʰi˧}}}}{}
\textcolor{teal}{\zh{形容词}} \hspace{4pt} \zh{声调类:} M.
\zh{聪明。} \textcolor{Sepia}{\selectlanguage{english}Intelligent.} \textcolor{PineGreen}{\selectlanguage{french}Intelligent.}  ¶ \textcolor{darkblue}{\textbf{\ipa{ʈʂʰɯ˧ | zo˧tʰi˧ | ʐwæ˩˥!}}} \zh{他很聪明!} \textcolor{Sepia}{\selectlanguage{english}He is very clever!} \textcolor{PineGreen}{\selectlanguage{french}il est très intelligent!}  
 ¶ \textcolor{darkblue}{\textbf{\ipa{ʈʂʰɯ˧ | mɤ˧-tʰi˧!}}} \zh{他不聪明!} \textcolor{Sepia}{\selectlanguage{english}He is not clever!} \textcolor{PineGreen}{\selectlanguage{french}il n'est pas intelligent/il n'est pas bien malin! (on ne peut dire: /*mɤ˧-zo˧tʰi˧/)}  

\lhead{\firstmark}
\rhead{\botmark}

\subsection{\hspace{-0.5cm} {\Large \textcolor{darkblue}{\textbf{\ipa{zo˧tɕi˥}}}}\hspace{0.5cm}[\kern2pt{\textcolor{darkblue}{\textbf{\ipa{zo˧tɕi˥}}}}\kern2pt]} \hypertarget{zo\string_Mts£i\string_T1}{}
\markboth{\textcolor{darkblue}{\textbf{\ipa{zo˧tɕi˥}}}}{}
\textcolor{teal}{\zh{名词}} \hspace{4pt} \zh{声调类:} H\#.
\zh{最小的儿子。} \textcolor{Sepia}{\selectlanguage{english}Youngest son.} \textcolor{PineGreen}{\selectlanguage{french}Fils dernier-né, benjamin.}  ¶ \textcolor{darkblue}{\textbf{\ipa{zo˧tɕi˥-mv̩˩tɕi˩}}} \zh{最小的儿子与女儿} \textcolor{Sepia}{\selectlanguage{english}youngest son and youngest daughter} \textcolor{PineGreen}{\selectlanguage{french}le benjamin et la benjamine: les plus jeunes enfants}  

\lhead{\firstmark}
\rhead{\botmark}

\subsection{\hspace{-0.5cm} {\Large \textcolor{darkblue}{\textbf{\ipa{zo˧zo˧-mv̩˧mv̩˥}}}}\hspace{0.5cm}[\kern2pt{\textcolor{darkblue}{\textbf{\ipa{xxxx non-correspondance entre le nombre de morphèmes et le nombre de tons de morphèmes}}}}\kern2pt]} \hypertarget{zo\string_Mzo\string_M-mv\string_=\string_Mmv\string_=\string_T1}{}
\markboth{\textcolor{darkblue}{\textbf{\ipa{zo˧zo˧-mv̩˧mv̩˥}}}}{}
\textcolor{teal}{\zh{名词}} \hspace{4pt} \zh{声调类:} H\#.
\zh{东西。} \textcolor{Sepia}{\selectlanguage{english}Thing, thingummy.} \textcolor{PineGreen}{\selectlanguage{french}Truc, bidule.}  \zh{量词}: \textcolor{darkblue}{\textbf{\ipa{kʰwɤ˥}}} 
\lhead{\firstmark}
\rhead{\botmark}

\subsection{\hspace{-0.5cm} {\Large \textcolor{darkblue}{\textbf{\ipa{zo˧ʐɤ\#˥}}}}\hspace{0.5cm}[\kern2pt{\textcolor{darkblue}{\textbf{\ipa{zo˧ʐɤ˧}}}}\kern2pt]} \hypertarget{zo\string_Mz`7\#\string_T1}{}
\markboth{\textcolor{darkblue}{\textbf{\ipa{zo˧ʐɤ\#˥}}}}{}
\textcolor{teal}{\zh{名词}} \hspace{4pt} \zh{声调类:} \#H.
\zh{义子。} \textcolor{Sepia}{\selectlanguage{english}Adoptive son, foster son.} \textcolor{PineGreen}{\selectlanguage{french}Fils adoptif.} 
\lhead{\firstmark}
\rhead{\botmark}

\subsection{\hspace{-0.5cm} {\Large \textcolor{darkblue}{\textbf{\ipa{zo˩bv̩˥li˩}}}}\hspace{0.5cm}[\kern2pt{\textcolor{darkblue}{\textbf{\ipa{zo˧bv̩˧li˩}}}}\kern2pt]} \hypertarget{zo\string_Bbv\string_=\string_Tli\string_B1}{}
\markboth{\textcolor{darkblue}{\textbf{\ipa{zo˩bv̩˥li˩}}}}{}
\textcolor{teal}{\zh{名词}} \hspace{4pt} \zh{声调类:} .
\zh{宇宙。} \textcolor{Sepia}{\selectlanguage{english}Universe.} \textcolor{PineGreen}{\selectlanguage{french}Univers.}  \zh{【借词】}\zh{藏语}? (Lidz 2010: 108)
 ¶ \textcolor{darkblue}{\textbf{\ipa{sɑ˧ | -zo˩bv̩˥-li˩}}} \zh{宇宙} \textcolor{Sepia}{\selectlanguage{english}the universe} \textcolor{PineGreen}{\selectlanguage{french}l'univers}  

\lhead{\firstmark}
\rhead{\botmark}

\subsection{\hspace{-0.5cm} {\Large \textcolor{darkblue}{\textbf{\ipa{zo˩no˧}}}}\hspace{0.5cm}[\kern2pt{\textcolor{darkblue}{\textbf{\ipa{zo˩no˥}}}}\kern2pt]} \hypertarget{zo\string_Bno\string_M1}{}
\markboth{\textcolor{darkblue}{\textbf{\ipa{zo˩no˧}}}}{}
\textcolor{teal}{\zh{助词}} \hspace{4pt} \zh{声调类:} LM.
\zh{现在。} \textcolor{Sepia}{\selectlanguage{english}Now.} \textcolor{PineGreen}{\selectlanguage{french}Maintenant, actuellement: désigne le moment présent (heure de la journée), comme la période présente (époque contemporaine, par opposition à d'autres époques); également employé comme élément phatique, “gap-filler”: “alors…”; “eh bien…”.}  ¶ \textcolor{darkblue}{\textbf{\ipa{zo˩no˥ | gɤ˩-ʈi˧!}}} \zh{刚起床! / 刚才才起床!} \textcolor{Sepia}{\selectlanguage{english}She only just woke up! (Context: someone walks into the house in the afternoon, sees a little child playing, and notes: “She has got up!” The child's grandmother answers: “She only just woke up!”)} \textcolor{PineGreen}{\selectlanguage{french}Elle vient de se réveiller/de se lever! / Elle s'est réveillée à l'instant! (contexte: quelqu'un entre dans la maison, voit un petit enfant en train de jouer et constate: “Elle est réveillée!” Sa grand-mère répond: “Elle vient de se réveiller!”)}  

\lhead{\firstmark}
\rhead{\botmark}

\subsection{\hspace{-0.5cm} {\Large \textcolor{darkblue}{\textbf{\ipa{zo˩qo˧}}}}\hspace{0.5cm}[\kern2pt{\textcolor{darkblue}{\textbf{\ipa{zo˩qo˥}}}}\kern2pt]} \hypertarget{zo\string_Bqo\string_M1}{}
\markboth{\textcolor{darkblue}{\textbf{\ipa{zo˩qo˧}}}}{}
\textcolor{teal}{\zh{代词}} \hspace{4pt} \zh{声调类:} LM.
\zh{哪里。} \textcolor{Sepia}{\selectlanguage{english}Where.} \textcolor{PineGreen}{\selectlanguage{french}Où.}  ¶ \textcolor{darkblue}{\textbf{\ipa{no˧ | zo˩qo˧ bi˧?}}} \zh{你去哪里?} \textcolor{Sepia}{\selectlanguage{english}Where are you going?} \textcolor{PineGreen}{\selectlanguage{french}Où tu vas?}  
 ¶ \textcolor{darkblue}{\textbf{\ipa{zo˩qo˧-ɳɯ˧ | tsʰɯ˩˥?}}} \zh{从哪里来?} \textcolor{Sepia}{\selectlanguage{english}Where (are you) coming from?} \textcolor{PineGreen}{\selectlanguage{french}D'où (tu) viens?}  
 ¶ \textcolor{darkblue}{\textbf{\ipa{no˧ | hɑ˧ | zo˩qo˧ dzɯ˧-bi˧-pi˧, | ɖɯ˧-bæ˧ lɑ˧ ɲi˥!}}} \zh{无论你到哪里去吃,都一样!(情景:新开的饭馆)} \textcolor{Sepia}{\selectlanguage{english}No matter which one you choose: they're all the same! (Context: discussing the restaurant scene in Yongning; in the speaker's view, the newly opened restaurants all share the same qualities and shortcomings, for instance concerning hygiene.)} \textcolor{PineGreen}{\selectlanguage{french}Peu importe où tu vas manger, c'est partout pareil! (contexte: au sujet des restaurants récemment ouverts à Yongning, qui partagent les mêmes qualités et défauts dont des problèmes d'hygiène)}  
 ¶ \textcolor{darkblue}{\textbf{\ipa{zo˩qo˧ tʰv̩˧?}}} \zh{你到哪里了?(打手机)} \textcolor{Sepia}{\selectlanguage{english}Where are you? (Typical question when calling someone on their mobile phone)} \textcolor{PineGreen}{\selectlanguage{french}Où tu es? (question typique quand on appelle quelqu'un sur son téléphone portable)}  

\lhead{\firstmark}
\rhead{\botmark}

\subsection{\hspace{-0.5cm} {\Large \textcolor{darkblue}{\textbf{\ipa{zɯ˥}}}}\hspace{0.5cm}[\kern2pt{\textcolor{darkblue}{\textbf{\ipa{zɯ˥}}}}\kern2pt]} \hypertarget{zM\string_T1}{}
\markboth{\textcolor{darkblue}{\textbf{\ipa{zɯ˥}}}}{}
\textcolor{teal}{\zh{名词}} \hspace{4pt} \zh{声调类:} \#H.
\zh{草。} \textcolor{Sepia}{\selectlanguage{english}Grass.} \textcolor{PineGreen}{\selectlanguage{french}Herbe.}  \zh{量词}: \textcolor{darkblue}{\textbf{\ipa{kʰɤ˧˥}}} 
\lhead{\firstmark}
\rhead{\botmark}

\subsection{\hspace{-0.5cm} {\Large \textcolor{darkblue}{\textbf{\ipa{zɯ˧}}}}\hspace{0.5cm}[\kern2pt{\textcolor{darkblue}{\textbf{\ipa{zɯ˥}}}}\kern2pt]} \hypertarget{zM\string_M1}{}
\markboth{\textcolor{darkblue}{\textbf{\ipa{zɯ˧}}}}{}
\textcolor{teal}{\zh{名词}} \hspace{4pt} \zh{声调类:} M.
\zh{生命。} \textcolor{Sepia}{\selectlanguage{english}Life, existence.} \textcolor{PineGreen}{\selectlanguage{french}Vie, existence.}  ¶ \textcolor{darkblue}{\textbf{\ipa{zɯ˧ʂæ\#˥}}} \zh{长命、长的人生} \textcolor{Sepia}{\selectlanguage{english}long life} \textcolor{PineGreen}{\selectlanguage{french}longue vie}  
 ¶ \textcolor{darkblue}{\textbf{\ipa{zɯ˧ ʂæ˧ | hɑ̃˧-ʝi˧-kʰɯ˩!}}} \zh{祝你长寿!} \textcolor{Sepia}{\selectlanguage{english}May you have a long life!} \textcolor{PineGreen}{\selectlanguage{french}Puisses-tu avoir longue vie! (bénédiction)}  
 ¶ \textcolor{darkblue}{\textbf{\ipa{zɯ˧ɖæ\#˥}}} \zh{短命} \textcolor{Sepia}{\selectlanguage{english}short life} \textcolor{PineGreen}{\selectlanguage{french}courte vie}  
\zh{~【参考】~} \hyperlink{}{\textcolor{darkblue}{\textbf{\ipa{zɯ˧\textsubscript{b}}}}} 
\lhead{\firstmark}
\rhead{\botmark}

\subsection{\hspace{-0.5cm} {\Large \textcolor{darkblue}{\textbf{\ipa{zɯ˧\textsubscript{b}}}}}\hspace{0.5cm}[\kern2pt{\textcolor{darkblue}{\textbf{\ipa{zɯ˥}}}}\kern2pt]} \hypertarget{zM\string_Mb1}{}
\markboth{\textcolor{darkblue}{\textbf{\ipa{zɯ˧\textsubscript{b}}}}}{}
\textcolor{teal}{\zh{量词}} \hspace{4pt} \zh{声调类:} M\textsubscript{b}.
\zh{量词:辈子。} \textcolor{Sepia}{\selectlanguage{english}Self-classifier for life, existence.} \textcolor{PineGreen}{\selectlanguage{french}Auto-classificateur de la vie, de l'existence entière.}  ¶ \textcolor{darkblue}{\textbf{\ipa{ɖɯ˧-zɯ˧}}} \zh{一辈子(的时间)} \textcolor{Sepia}{\selectlanguage{english}all of (one's) life, a lifetime} \textcolor{PineGreen}{\selectlanguage{french}toute la vie}  
\zh{~【参考】~} \hyperlink{}{\textcolor{darkblue}{\textbf{\ipa{zɯ˧}}}} 
\lhead{\firstmark}
\rhead{\botmark}

\subsection{\hspace{-0.5cm} {\Large \textcolor{darkblue}{\textbf{\ipa{zɯ˧hṽ˩}}}}\hspace{0.5cm}[\kern2pt{\textcolor{darkblue}{\textbf{\ipa{zɯ˧hṽ˩}}}}\kern2pt]} \hypertarget{zM\string_Mhv\string_~\string_B1}{}
\markboth{\textcolor{darkblue}{\textbf{\ipa{zɯ˧hṽ˩}}}}{}
\textcolor{teal}{\zh{形容词}} \hspace{4pt} \zh{声调类:} L\#.
\zh{绿(布料、线)。} \textcolor{Sepia}{\selectlanguage{english}Green.} \textcolor{PineGreen}{\selectlanguage{french}Vert.}  ¶ \textcolor{darkblue}{\textbf{\ipa{zɯ˧hṽ˩-ni˩gv̩˩}}} \zh{绿} \textcolor{Sepia}{\selectlanguage{english}green} \textcolor{PineGreen}{\selectlanguage{french}de couleur verte}  
 ¶ \textcolor{darkblue}{\textbf{\ipa{[F5] zɯ˧hṽ˩ | \textasciitilde{}zɯ˧hṽ˩-ni˩gv̩˩}}} \zh{全绿} \textcolor{Sepia}{\selectlanguage{english}all green, green all over} \textcolor{PineGreen}{\selectlanguage{french}tout vert}  

\lhead{\firstmark}
\rhead{\botmark}

\subsection{\hspace{-0.5cm} {\Large \textcolor{darkblue}{\textbf{\ipa{zɯ˧pv̩˩}}}}\hspace{0.5cm}[\kern2pt{\textcolor{darkblue}{\textbf{\ipa{zɯ˧pv̩˩}}}}\kern2pt]} \hypertarget{zM\string_Mpv\string_=\string_B1}{}
\markboth{\textcolor{darkblue}{\textbf{\ipa{zɯ˧pv̩˩}}}}{}
\textcolor{teal}{\zh{名词}} \hspace{4pt} \zh{声调类:} L\#.
\zh{干草。} \textcolor{Sepia}{\selectlanguage{english}Hay, dry grass.} \textcolor{PineGreen}{\selectlanguage{french}Foin; s'emploie aussi parfois pour désigner la paille: dans la maison, on ne stocke que de la paille de riz, pas de foin; l'herbe cueillie verte puis séchée (foin) n'est pas entreposée, mais aussitôt donnée aux animaux.}  \zh{量词}: \textcolor{darkblue}{\textbf{\ipa{kʰɤ˧˥}}} 
\lhead{\firstmark}
\rhead{\botmark}

\subsection{\hspace{-0.5cm} {\Large \textcolor{darkblue}{\textbf{\ipa{zɯ˧-qʰɑ˧mi\#˥}}}}\hspace{0.5cm}[\kern2pt{\textcolor{darkblue}{\textbf{\ipa{xxxx non-correspondance entre le nombre de morphèmes et le nombre de tons de morphèmes}}}}\kern2pt]} \hypertarget{zM\string_M-q\string_hA\string_Mmi\#\string_T1}{}
\markboth{\textcolor{darkblue}{\textbf{\ipa{zɯ˧-qʰɑ˧mi\#˥}}}}{}
\textcolor{teal}{\zh{名词}} \hspace{4pt} \zh{声调类:} \#H.
\zh{蓑草、山草、山草根、龙须草、山茅草、羊草、拟金茅。} \textcolor{Sepia}{\selectlanguage{english}Sabai grass, \textit{Eulaliopsis binata (Retz.) C. E. Hubb.}.} \textcolor{PineGreen}{\selectlanguage{french}\textit{Eulaliopsis binata (Retz.) C. E. Hubb.}, herbe sauvage. L'herbe n'est pas prisée du bétail, non plus que ses racines, et n'est jamais consommée par les humains. Les racines de cette herbe sont utilisées dans les rituels: elles ont une odeur forte à la combustion.} \zh{当地汉语方言:}\zh{狗尾巴草。} \zh{量词}: \textcolor{darkblue}{\textbf{\ipa{po˧}}} 
\lhead{\firstmark}
\rhead{\botmark}

\subsection{\hspace{-0.5cm} {\Large \textcolor{darkblue}{\textbf{\ipa{zɯ˧ɻ\#˥}}}}\hspace{0.5cm}[\kern2pt{\textcolor{darkblue}{\textbf{\ipa{zɯ˧ɻ˧}}}}\kern2pt]} \hypertarget{zM\string_Mr£`\#\string_T1}{}
\markboth{\textcolor{darkblue}{\textbf{\ipa{zɯ˧ɻ\#˥}}}}{}
\textcolor{teal}{\zh{名词}} \hspace{4pt} \zh{声调类:} \#H.
\zh{腮、腮帮子。} \textcolor{Sepia}{\selectlanguage{english}Cheek.} \textcolor{PineGreen}{\selectlanguage{french}Joue (partie basse, en-dessous des pommettes; vers l'articulation des deux mâchoires).}  ¶ \textcolor{darkblue}{\textbf{\ipa{zɯ˧ɻ̍˧ qʰwæ˩}}} \zh{掌掴、打嘴巴} \textcolor{Sepia}{\selectlanguage{english}to slap/smack someone's cheek} \textcolor{PineGreen}{\selectlanguage{french}gifler}  
 \zh{量词}: \textcolor{darkblue}{\textbf{\ipa{ɭɯ˧}}} 
\lhead{\firstmark}
\rhead{\botmark}

\subsection{\hspace{-0.5cm} {\Large \textcolor{darkblue}{\textbf{\ipa{zɯ˧\textasciitilde{}zɯ˧}}}}\hspace{0.5cm}[\kern2pt{\textcolor{darkblue}{\textbf{\ipa{zɯ˧zɯ˧}}}}\kern2pt]} \hypertarget{zM\string_M~zM\string_M1}{}
\markboth{\textcolor{darkblue}{\textbf{\ipa{zɯ˧\textasciitilde{}zɯ˧}}}}{}
\textcolor{teal}{\zh{名词}} \hspace{4pt} \zh{声调类:} M.
\zh{生命。} \textcolor{Sepia}{\selectlanguage{english}Life, existence.} \textcolor{PineGreen}{\selectlanguage{french}Vie, existence.}  ¶ \textcolor{darkblue}{\textbf{\ipa{hĩ˧-zɯ˧\textasciitilde{}zɯ˥\$}}} \zh{人生} \textcolor{Sepia}{\selectlanguage{english}human life} \textcolor{PineGreen}{\selectlanguage{french}la vie humaine}  
 ¶ \textcolor{darkblue}{\textbf{\ipa{hĩ˧ zɯ˧ | ʂæ˧ | ʐwæ˩˥}}} \zh{很长的人生 / 人生很长} \textcolor{Sepia}{\selectlanguage{english}a very long life / life is very long} \textcolor{PineGreen}{\selectlanguage{french}une très longue vie / la vie est longue}  
 \zh{量词}: \textcolor{darkblue}{\textbf{\ipa{ljɤ˩}}} 
\lhead{\firstmark}
\rhead{\botmark}

\subsection{\hspace{-0.5cm} {\Large \textcolor{darkblue}{\textbf{\ipa{zɯ˩\textasciitilde{}zɯ˩}}}}\hspace{0.5cm}[\kern2pt{\textcolor{darkblue}{\textbf{\ipa{zɯ˩zɯ˩˥}}}}\kern2pt]} \hypertarget{zM\string_B~zM\string_B1}{}
\markboth{\textcolor{darkblue}{\textbf{\ipa{zɯ˩\textasciitilde{}zɯ˩}}}}{}
\textcolor{teal}{\zh{动词}} \hspace{4pt} \zh{声调类:} L.
\zh{麻木。} \textcolor{Sepia}{\selectlanguage{english}To be numb.} \textcolor{PineGreen}{\selectlanguage{french}Être engourdi.}  ¶ \textcolor{darkblue}{\textbf{\ipa{gv̩˧dv̩˧gv̩˧mi˧ | zɯ˩\textasciitilde{}zɯ˩˥}}} \zh{身体麻木、全身麻木} \textcolor{Sepia}{\selectlanguage{english}to be numb (whole body)} \textcolor{PineGreen}{\selectlanguage{french}avoir le corps engourdi}  
 ¶ \textcolor{darkblue}{\textbf{\ipa{gv̩˧mi˧ | zɯ˩\textasciitilde{}zɯ˩˥}}} \zh{身体麻木、全身麻木} \textcolor{Sepia}{\selectlanguage{english}to be numb (whole body)} \textcolor{PineGreen}{\selectlanguage{french}avoir le corps engourdi}  
 ¶ \textcolor{darkblue}{\textbf{\ipa{tʰi˧-zɯ˩\textasciitilde{}zɯ˩}}} \zh{\mytextsc{dur} \mytextsc{red}} \textcolor{Sepia}{\selectlanguage{english}\mytextsc{dur} \mytextsc{red}} \textcolor{PineGreen}{\selectlanguage{french}\mytextsc{dur} \mytextsc{red}}  

\lhead{\firstmark}
\rhead{\botmark}

\newpage
\section*{\centering- \textcolor{darkblue}{\textbf{\ipa{ʐ}}} -}
\subsection{\hspace{-0.5cm} {\Large \textcolor{darkblue}{\textbf{\ipa{ʐ}}}}\hspace{0.5cm}[\kern2pt{\textcolor{darkblue}{\textbf{\ipa{[]}}}}\kern2pt]} \hypertarget{z`1}{}
\markboth{\textcolor{darkblue}{\textbf{\ipa{ʐ}}}}{}
\textcolor{teal}{\zh{状貌词}} \hspace{4pt} \zh{声调类:} 0.
\zh{形声词:轰隆隆!(拉很重的物品在地板上的隆隆声,卡车的隆隆声)。} \textcolor{Sepia}{\selectlanguage{english}Rumbling sound of heavy loads carried over a wooden floor, of lorries... Brroom!} \textcolor{PineGreen}{\selectlanguage{french}Bruit de grondement des grosses charges qu'on traîne sur le sol, des moteurs de camions: Brrroum!} 
\lhead{\firstmark}
\rhead{\botmark}

\subsection{\hspace{-0.5cm} {\Large \textcolor{darkblue}{\textbf{\ipa{ʐæ˧}}}}\hspace{0.5cm}[\kern2pt{\textcolor{darkblue}{\textbf{\ipa{ʐæ˩˥}}}}\kern2pt]} \hypertarget{z`\{\string_M1}{}
\markboth{\textcolor{darkblue}{\textbf{\ipa{ʐæ˧}}}}{}
\textcolor{teal}{\zh{形容词}} \hspace{4pt} \zh{声调类:} M.
\zh{高大。} \textcolor{Sepia}{\selectlanguage{english}Tall and big; great; impressive.} \textcolor{PineGreen}{\selectlanguage{french}Grand et fort, massif, baraqué.}  ¶ \textcolor{darkblue}{\textbf{\ipa{ʐæ˧-ni˩gv̩˩}}} \zh{高大} \textcolor{Sepia}{\selectlanguage{english}tall and big; great; impressive} \textcolor{PineGreen}{\selectlanguage{french}grand et fort}  
 ¶ \textcolor{darkblue}{\textbf{\ipa{hĩ˧ | ʈʂʰɯ˧-v̩˧, | ʐæ˧-ni˩gv̩˩!}}} \zh{这人很高大!} \textcolor{Sepia}{\selectlanguage{english}This person looks impressive / tall and big!} \textcolor{PineGreen}{\selectlanguage{french}Elle/il est grand(e) et fort(e) / impressionnant(e)!}  
 ¶ \textcolor{darkblue}{\textbf{\ipa{ʐæ˧ni˩ | mɤ˧-gv̩˧}}} \zh{个子不高} \textcolor{Sepia}{\selectlanguage{english}not tall, not impressive, not great-looking} \textcolor{PineGreen}{\selectlanguage{french}pas bien grand (en taille), pas bien impressionnant}  
 ¶ \textcolor{darkblue}{\textbf{\ipa{ʐæ˧ | ʐwæ˩˥}}} \zh{很高大} \textcolor{Sepia}{\selectlanguage{english}very tall and big} \textcolor{PineGreen}{\selectlanguage{french}très grand et fort}  

\lhead{\firstmark}
\rhead{\botmark}

\subsection{\hspace{-0.5cm} {\Large \textcolor{darkblue}{\textbf{\ipa{ʐæ˧\textsubscript{a}}}}}\hspace{0.5cm}[\kern2pt{\textcolor{darkblue}{\textbf{\ipa{ʐæ˩˥}}}}\kern2pt]} \hypertarget{z`\{\string_Ma1}{}
\markboth{\textcolor{darkblue}{\textbf{\ipa{ʐæ˧\textsubscript{a}}}}}{}
\textcolor{teal}{\zh{动词}} \hspace{4pt} \zh{声调类:} M\textsubscript{a}.
\ding{202} \zh{笑。} \textcolor{Sepia}{\selectlanguage{english}To laugh.} \textcolor{PineGreen}{\selectlanguage{french}Rire.}  ¶ \textcolor{darkblue}{\textbf{\ipa{zo˧hṽ˥ | hĩ˧ ʐæ˧\textasciitilde{}ʐæ˥-kʰɯ˩}}} \zh{孩子们把大家逗笑了。} \textcolor{Sepia}{\selectlanguage{english}The kids make people laugh} \textcolor{PineGreen}{\selectlanguage{french}les enfants taquinent les gens, les font rire}  
 ¶ \textcolor{darkblue}{\textbf{\ipa{hĩ˧ | ʐæ˧\textasciitilde{}ʐæ˥ kʰɯ˩}}} \zh{让大家笑一笑} \textcolor{Sepia}{\selectlanguage{english}to make people laugh, to amuse people} \textcolor{PineGreen}{\selectlanguage{french}faire rire les gens, amuser les gens, faire rire le public}  
 ¶ \textcolor{darkblue}{\textbf{\ipa{ʐæ˧\textasciitilde{}ʐæ˩-di˩}}} \zh{笑话,好笑的话} \textcolor{Sepia}{\selectlanguage{english}jokes, funny talk} \textcolor{PineGreen}{\selectlanguage{french}plaisanteries, blagues}  
\ding{203} \zh{嘲笑别人、出言不逊。} \textcolor{Sepia}{\selectlanguage{english}To laugh at; to be impertinent; to deride, to make fun of (people).} \textcolor{PineGreen}{\selectlanguage{french}Être impertinent, déranger, se moquer du monde.}  ¶ \textcolor{darkblue}{\textbf{\ipa{le˧-ʐæ˧-ze˧}}} \zh{出言不逊了} \textcolor{Sepia}{\selectlanguage{english}\mytextsc{accomp} \string_ \mytextsc{pfv}} \textcolor{PineGreen}{\selectlanguage{french}\mytextsc{accomp} \string_ \mytextsc{pfv}}  
 ¶ \textcolor{darkblue}{\textbf{\ipa{le˧-ʐæ˥\textasciitilde{}ʐæ˩}}} \zh{笑一笑(别人)} \textcolor{Sepia}{\selectlanguage{english}\mytextsc{red}} \textcolor{PineGreen}{\selectlanguage{french}\mytextsc{red}}  
 ¶ \textcolor{darkblue}{\textbf{\ipa{hĩ˧ ʐæ˩}}} \zh{嘲笑人家} \textcolor{Sepia}{\selectlanguage{english}to make fun of other people} \textcolor{PineGreen}{\selectlanguage{french}être impertinent avec les gens, déranger les gens}  
 ¶ \textcolor{darkblue}{\textbf{\ipa{le˧-ʐæ˥\textasciitilde{}ʐæ˩-ze˩}}} \zh{嘲笑了} \textcolor{Sepia}{\selectlanguage{english}\mytextsc{red} \mytextsc{pfv}} \textcolor{PineGreen}{\selectlanguage{french}\mytextsc{red} \mytextsc{pfv}}  

\lhead{\firstmark}
\rhead{\botmark}

\subsection{\hspace{-0.5cm} {\Large \textcolor{darkblue}{\textbf{\ipa{ʐæ˧v̩˩-tʰv̩˩}}}}\hspace{0.5cm}[\kern2pt{\textcolor{darkblue}{\textbf{\ipa{ʐæ˧v̩˩tʰv̩˧}}}}\kern2pt]} \hypertarget{z`\{\string_Mv\string_=\string_B-t\string_hv\string_=\string_B1}{}
\markboth{\textcolor{darkblue}{\textbf{\ipa{ʐæ˧v̩˩-tʰv̩˩}}}}{}
\textcolor{teal}{\zh{动词}} \hspace{4pt} \zh{声调类:} L\#-.
\zh{开玩笑。} \textcolor{Sepia}{\selectlanguage{english}To joke, to crack a joke.} \textcolor{PineGreen}{\selectlanguage{french}Blaguer, faire une blague, faire une plaisanterie.}  ¶ \textcolor{darkblue}{\textbf{\ipa{ʐæ˧v̩˩-tʰv̩˩ | ʐwæ˩˥}}} \zh{开很多玩笑、一直开玩笑} \textcolor{Sepia}{\selectlanguage{english}to crack jokes all the time, to make lots of jokes} \textcolor{PineGreen}{\selectlanguage{french}plaisanter follement, rire beaucoup}  
 ¶ \textcolor{darkblue}{\textbf{\ipa{ʐæ˧v̩˩-tʰv̩˩-hĩ˩ ʐwɤ˩}}} \zh{开个玩笑} \textcolor{Sepia}{\selectlanguage{english}to crack a joke} \textcolor{PineGreen}{\selectlanguage{french}lancer une blague, dire une plaisanterie}  

\lhead{\firstmark}
\rhead{\botmark}

\subsection{\hspace{-0.5cm} {\Large \textcolor{darkblue}{\textbf{\ipa{ʐæ˩\textsubscript{b}}}}}\hspace{0.5cm}[\kern2pt{\textcolor{darkblue}{\textbf{\ipa{ʐæ˥}}}}\kern2pt]} \hypertarget{z`\{\string_Bb1}{}
\markboth{\textcolor{darkblue}{\textbf{\ipa{ʐæ˩\textsubscript{b}}}}}{}
\textcolor{teal}{\zh{动词}} \hspace{4pt} \zh{声调类:} L\textsubscript{b}.
\zh{搅拌合混。} \textcolor{Sepia}{\selectlanguage{english}To mix.} \textcolor{PineGreen}{\selectlanguage{french}Mélanger, tourner (un mélange, une préparation).}  ¶ \textcolor{darkblue}{\textbf{\ipa{le˧-ʐæ˧\textasciitilde{}ʐæ˥}}} \zh{搅拌} \textcolor{Sepia}{\selectlanguage{english}\mytextsc{accomp} \string_ \mytextsc{red}} \textcolor{PineGreen}{\selectlanguage{french}\mytextsc{accomp} \string_ \mytextsc{red}}  

\lhead{\firstmark}
\rhead{\botmark}

\subsection{\hspace{-0.5cm} {\Large \textcolor{darkblue}{\textbf{\ipa{ʐæ˩mi\#˥}}}}\hspace{0.5cm}[\kern2pt{\textcolor{darkblue}{\textbf{\ipa{ʐæ˧mi˧}}}}\kern2pt]} \hypertarget{z`\{\string_Bmi\#\string_T1}{}
\markboth{\textcolor{darkblue}{\textbf{\ipa{ʐæ˩mi\#˥}}}}{}
\textcolor{teal}{\zh{名词}} \hspace{4pt} \zh{声调类:} LM+\#H.
\zh{母豹子。} \textcolor{Sepia}{\selectlanguage{english}Female leopard.} \textcolor{PineGreen}{\selectlanguage{french}Léopard femelle.}  ¶ \textcolor{darkblue}{\textbf{\ipa{ʐæ˩mi˧-ʐæ˥zo˩}}} \zh{母豹子与公豹子} \textcolor{Sepia}{\selectlanguage{english}female leopard and male leopard} \textcolor{PineGreen}{\selectlanguage{french}léopard femelle et léopard mâle}  
 \zh{量词}: \textcolor{darkblue}{\textbf{\ipa{pʰo˧˥}}} 
\lhead{\firstmark}
\rhead{\botmark}

\subsection{\hspace{-0.5cm} {\Large \textcolor{darkblue}{\textbf{\ipa{ʐæ˩pʰv̩˧}}}}\hspace{0.5cm}[\kern2pt{\textcolor{darkblue}{\textbf{\ipa{xxxx non-correspondance entre le nombre de morphèmes et le nombre de tons de morphèmes}}}}\kern2pt]} \hypertarget{z`\{\string_Bp\string_hv\string_=\string_M1}{}
\markboth{\textcolor{darkblue}{\textbf{\ipa{ʐæ˩pʰv̩˧}}}}{}
\textcolor{teal}{\zh{名词}} \hspace{4pt} \zh{声调类:} LM.
\zh{公豹子。} \textcolor{Sepia}{\selectlanguage{english}Male leopard.} \textcolor{PineGreen}{\selectlanguage{french}Léopard mâle.}  ¶ \textcolor{darkblue}{\textbf{\ipa{ʐæ˩pʰv̩˧-ʐæ˩mi˩}}} \zh{公豹子与母豹子} \textcolor{Sepia}{\selectlanguage{english}male leopard and female leopard} \textcolor{PineGreen}{\selectlanguage{french}léopard mâle et léopard femelle}  
 \zh{量词}: \textcolor{darkblue}{\textbf{\ipa{pʰo˧˥}}} 
\lhead{\firstmark}
\rhead{\botmark}

\subsection{\hspace{-0.5cm} {\Large \textcolor{darkblue}{\textbf{\ipa{ʐæ˩sɯ˩}}}}\hspace{0.5cm}[\kern2pt{\textcolor{darkblue}{\textbf{\ipa{ʐæ˩sɯ˥}}}}\kern2pt]} \hypertarget{z`\{\string_BsM\string_B1}{}
\markboth{\textcolor{darkblue}{\textbf{\ipa{ʐæ˩sɯ˩}}}}{}
\textcolor{teal}{\zh{名词}} \hspace{4pt} \zh{声调类:} L.
\zh{披毡。} \textcolor{Sepia}{\selectlanguage{english}Rough felt made only of sheep wool. One drapes it over one's shoulders as an outdoor protection from the cold.} \textcolor{PineGreen}{\selectlanguage{french}Feutre grossier, fait uniquement de laine de mouton, dont on se drape en extérieur pour se protéger du froid.}  \zh{量词}: \textcolor{darkblue}{\textbf{\ipa{ɭɯ˧˥}}} 
\lhead{\firstmark}
\rhead{\botmark}

\subsection{\hspace{-0.5cm} {\Large \textcolor{darkblue}{\textbf{\ipa{ʐæ˩sɯ˩-kʰwæ˩ɻæ˧}}}}\hspace{0.5cm}[\kern2pt{\textcolor{darkblue}{\textbf{\ipa{xxxx non-correspondance entre le nombre de morphèmes et le nombre de tons de morphèmes}}}}\kern2pt]} \hypertarget{z`\{\string_BsM\string_B-k\string_hw\{\string_Br£`\{\string_M1}{}
\markboth{\textcolor{darkblue}{\textbf{\ipa{ʐæ˩sɯ˩-kʰwæ˩ɻæ˧}}}}{}
\textcolor{teal}{\zh{名词}} \hspace{4pt} \zh{声调类:} .
\zh{毡子(真正的毡子)做的垫子。} \textcolor{Sepia}{\selectlanguage{english}Felt mat.} \textcolor{PineGreen}{\selectlanguage{french}Natte en feutre. Le terme désigne spécifiquement les nattes/matelas/tissus en feutre véritable, par opposition avec le sens étendu que peut avoir \textcolor{darkblue}{\textbf{\ipa{/kʰwæ˧ɻæ\#˥/}}}.}  \zh{量词}: \textcolor{darkblue}{\textbf{\ipa{tsʰi˥}}} 
\lhead{\firstmark}
\rhead{\botmark}

\subsection{\hspace{-0.5cm} {\Large \textcolor{darkblue}{\textbf{\ipa{ʐæ˩ʂæ˧}}}}\hspace{0.5cm}[\kern2pt{\textcolor{darkblue}{\textbf{\ipa{ʐæ˩ʂæ˥}}}}\kern2pt]} \hypertarget{z`\{\string_Bs`\{\string_M1}{}
\markboth{\textcolor{darkblue}{\textbf{\ipa{ʐæ˩ʂæ˧}}}}{}
\textcolor{teal}{\zh{形容词}} \hspace{4pt} \zh{声调类:} LM.
\zh{远。} \textcolor{Sepia}{\selectlanguage{english}Far, distant.} \textcolor{PineGreen}{\selectlanguage{french}Loin, lointain.}  ¶ \textcolor{darkblue}{\textbf{\ipa{no˧ | ʈʂʰɯ˧ | ə˩-ʐæ˥ʂæ˩? | dʑɤ˩kʰwɤ˧ ə˩-di˩? | - dʑɤ˩˥ | dʑɤ˩kʰwɤ˧ mɤ˧-di˥! | mɤ˧-ʐæ˩ʂæ˩!}}} \zh{你们很熟吗? - 不很熟!} \textcolor{Sepia}{\selectlanguage{english}Are you distant from him? Is there distance (between you)? - There is not much distance to speak of! We are not distant! (=we are close friends)} \textcolor{PineGreen}{\selectlanguage{french}tu es loin de lui? Y a-t-il de la distance entre vous? (=vous êtes proches/intimes, ou pas?) - Non, il n'y a guère de distance! Nous ne somme pas éloignés!}  

\lhead{\firstmark}
\rhead{\botmark}

\subsection{\hspace{-0.5cm} {\Large \textcolor{darkblue}{\textbf{\ipa{ʐæ˩tsɯ˧˥}}}}\hspace{0.5cm}[\kern2pt{\textcolor{darkblue}{\textbf{\ipa{ʐæ˩tsɯ˧˥}}}}\kern2pt]} \hypertarget{z`\{\string_BtsM\string_M\string_T1}{}
\markboth{\textcolor{darkblue}{\textbf{\ipa{ʐæ˩tsɯ˧˥}}}}{}
\textcolor{teal}{\zh{名词}} \hspace{4pt} \zh{声调类:} LM+MH\#.
\zh{小路、径道。} \textcolor{Sepia}{\selectlanguage{english}Path, trail.} \textcolor{PineGreen}{\selectlanguage{french}Sentier, petit chemin.}  ¶ \textcolor{darkblue}{\textbf{\ipa{ʐæ˩tsɯ˧-ʐɤ˥mi˩}}} \zh{径道} \textcolor{Sepia}{\selectlanguage{english}small trail} \textcolor{PineGreen}{\selectlanguage{french}chemin de traverse, raccourci}  
 \zh{量词}: \textcolor{darkblue}{\textbf{\ipa{kʰɯ˩}}} 
\lhead{\firstmark}
\rhead{\botmark}

\subsection{\hspace{-0.5cm} {\Large \textcolor{darkblue}{\textbf{\ipa{ʐæ˩zo\#˥}}}}\hspace{0.5cm}[\kern2pt{\textcolor{darkblue}{\textbf{\ipa{ʐæ˩zo˥}}}}\kern2pt]} \hypertarget{z`\{\string_Bzo\#\string_T1}{}
\markboth{\textcolor{darkblue}{\textbf{\ipa{ʐæ˩zo\#˥}}}}{}
\textcolor{teal}{\zh{名词}} \hspace{4pt} \zh{声调类:} LM+\#H.
\zh{小豹子。} \textcolor{Sepia}{\selectlanguage{english}Little leopard, baby leopard.} \textcolor{PineGreen}{\selectlanguage{french}Bébé léopard, petit léopard.}  ¶ \textcolor{darkblue}{\textbf{\ipa{ʐæ˩zo˧-ʐæ˥mi˩}}} \zh{小豹子与母豹子} \textcolor{Sepia}{\selectlanguage{english}baby leopard and female leopard} \textcolor{PineGreen}{\selectlanguage{french}petit léopard et léopard femelle}  
 \zh{量词}: \textcolor{darkblue}{\textbf{\ipa{ɭɯ˧}}} 
\lhead{\firstmark}
\rhead{\botmark}

\subsection{\hspace{-0.5cm} {\Large \textcolor{darkblue}{\textbf{\ipa{ʐæ˩˥}}}}\hspace{0.5cm}[\kern2pt{\textcolor{darkblue}{\textbf{\ipa{ʐæ˥}}}}\kern2pt]} \hypertarget{z`\{\string_B\string_T1}{}
\markboth{\textcolor{darkblue}{\textbf{\ipa{ʐæ˩˥}}}}{}
\textcolor{teal}{\zh{名词}} \hspace{4pt} \zh{声调类:} LH.
\zh{豹子。} \textcolor{Sepia}{\selectlanguage{english}Leopard, panther (note: these two terms are homonymous in English).} \textcolor{PineGreen}{\selectlanguage{french}Léopard, panthère (note: ces deux termes sont homonymes en français).}  ¶ \textcolor{darkblue}{\textbf{\ipa{ʐæ˩ dzɯ˧-ze˩}}} \zh{吃了豹子} \textcolor{Sepia}{\selectlanguage{english}...ate (a/the) leopard} \textcolor{PineGreen}{\selectlanguage{french}...a mangé (un/le) léopard}  
 ¶ \textcolor{darkblue}{\textbf{\ipa{ʐæ˩ hwæ˧-ze˩}}} \zh{买了豹子} \textcolor{Sepia}{\selectlanguage{english}...bought (a/the) leopard} \textcolor{PineGreen}{\selectlanguage{french}...a acheté (un/le) léopard}  
 \zh{量词}: \textcolor{darkblue}{\textbf{\ipa{pʰo˧˥}}} 
\lhead{\firstmark}
\rhead{\botmark}

\subsection{\hspace{-0.5cm} {\Large \textcolor{darkblue}{\textbf{\ipa{ʐe˥}}}}\hspace{0.5cm}[\kern2pt{\textcolor{darkblue}{\textbf{\ipa{ʐe˥}}}}\kern2pt]} \hypertarget{z`e\string_T1}{}
\markboth{\textcolor{darkblue}{\textbf{\ipa{ʐe˥}}}}{}
\textcolor{teal}{\zh{量词}} \hspace{4pt} \zh{声调类:} H\textsubscript{a}.
\zh{量词:熏肉(一块)。} \textcolor{Sepia}{\selectlanguage{english}Classifier for quarters of preserved meat.} \textcolor{PineGreen}{\selectlanguage{french}Classificateur des morceaux de viande conservée.} 
\lhead{\firstmark}
\rhead{\botmark}

\subsection{\hspace{-0.5cm} {\Large \textcolor{darkblue}{\textbf{\ipa{ʐe˥}}} \textsubscript{1}}\hspace{0.5cm}[\kern2pt{\textcolor{darkblue}{\textbf{\ipa{ʐe˥}}}}\kern2pt]} \hypertarget{z`e\string_T1}{}
\markboth{\textcolor{darkblue}{\textbf{\ipa{ʐe˥}}} \textsubscript{1}}{}
\textcolor{teal}{\zh{名词}} \hspace{4pt} \zh{声调类:} \#H.
\zh{箭。} \textcolor{Sepia}{\selectlanguage{english}Arrow.} \textcolor{PineGreen}{\selectlanguage{french}Flèche.}  ¶ \textcolor{darkblue}{\textbf{\ipa{ʐe˧ɻ̃˧ | ɖɯ˧-kʰɯ˩}}} \zh{一枝箭。也来指一个家庭} \textcolor{Sepia}{\selectlanguage{english}an arrow; also, metaphorically: a family, a lineage} \textcolor{PineGreen}{\selectlanguage{french}une flèche; désigne aussi, de façon métaphorique, une lignée/une famille}  
 \zh{量词}: \textcolor{darkblue}{\textbf{\ipa{kʰɯ˩}}} 
\lhead{\firstmark}
\rhead{\botmark}

\subsection{\hspace{-0.5cm} {\Large \textcolor{darkblue}{\textbf{\ipa{ʐe˥}}} \textsubscript{2}}\hspace{0.5cm}[\kern2pt{\textcolor{darkblue}{\textbf{\ipa{ʐe˥}}}}\kern2pt]} \hypertarget{z`e\string_T2}{}
\markboth{\textcolor{darkblue}{\textbf{\ipa{ʐe˥}}} \textsubscript{2}}{}
\textcolor{teal}{\zh{名词}} \hspace{4pt} \zh{声调类:} \#H.
\zh{雨季(夏天至秋天:三月份至八月份)。} \textcolor{Sepia}{\selectlanguage{english}Rainy season (summer and autumn: from the 3rd to the 8th month of the lunar calendar).} \textcolor{PineGreen}{\selectlanguage{french}Saison des pluies (été et automne: du 3e au 8e mois du calendrier lunaire).} 
\lhead{\firstmark}
\rhead{\botmark}

\subsection{\hspace{-0.5cm} {\Large \textcolor{darkblue}{\textbf{\ipa{ʐe˧ʈæ˥-ɬi˩}}}}\hspace{0.5cm}[\kern2pt{\textcolor{darkblue}{\textbf{\ipa{ʐe˧ʈæ˥ɬi˩}}}}\kern2pt]} \hypertarget{z`e\string_Mt`\{\string_T-Ki\string_B1}{}
\markboth{\textcolor{darkblue}{\textbf{\ipa{ʐe˧ʈæ˥-ɬi˩}}}}{}
\textcolor{teal}{\zh{名词}} \hspace{4pt} \zh{声调类:} H\#-L.
\zh{十一月。} \textcolor{Sepia}{\selectlanguage{english}11th month.} \textcolor{PineGreen}{\selectlanguage{french}11e mois.} 
\lhead{\firstmark}
\rhead{\botmark}

\subsection{\hspace{-0.5cm} {\Large \textcolor{darkblue}{\textbf{\ipa{ʐe˧v̩\#˥}}}}\hspace{0.5cm}[\kern2pt{\textcolor{darkblue}{\textbf{\ipa{ʐe˧v̩˧}}}}\kern2pt]} \hypertarget{z`e\string_Mv\string_=\#\string_T1}{}
\markboth{\textcolor{darkblue}{\textbf{\ipa{ʐe˧v̩\#˥}}}}{}
\textcolor{teal}{\zh{名词}} \hspace{4pt} \zh{声调类:} \#H.
\zh{阉牛。} \textcolor{Sepia}{\selectlanguage{english}Castrated ox, neutered ox, steer.} \textcolor{PineGreen}{\selectlanguage{french}Taureau castré.}  \zh{量词}: \textcolor{darkblue}{\textbf{\ipa{pʰo˧˥}}} 
\lhead{\firstmark}
\rhead{\botmark}

\subsection{\hspace{-0.5cm} {\Large \textcolor{darkblue}{\textbf{\ipa{ʐe˧zo\#˥}}}}\hspace{0.5cm}[\kern2pt{\textcolor{darkblue}{\textbf{\ipa{ʐe˧zo˧}}}}\kern2pt]} \hypertarget{z`e\string_Mzo\#\string_T1}{}
\markboth{\textcolor{darkblue}{\textbf{\ipa{ʐe˧zo\#˥}}}}{}
\textcolor{teal}{\zh{名词}} \hspace{4pt} \zh{声调类:} \#H.
\zh{箭。} \textcolor{Sepia}{\selectlanguage{english}Arrow.} \textcolor{PineGreen}{\selectlanguage{french}Flèche.}  ¶ \textcolor{darkblue}{\textbf{\ipa{ʐe˧zo˧ | ɖɯ˧-kʰɯ˩}}} \zh{一枝箭} \textcolor{Sepia}{\selectlanguage{english}one arrow} \textcolor{PineGreen}{\selectlanguage{french}une flèche}  

\lhead{\firstmark}
\rhead{\botmark}

\subsection{\hspace{-0.5cm} {\Large \textcolor{darkblue}{\textbf{\ipa{ʐe˩ʐe˧-bæ˩bæ˩}}}}\hspace{0.5cm}[\kern2pt{\textcolor{darkblue}{\textbf{\ipa{ʐe˩ʐe˧bæ˩bæ˩}}}}\kern2pt]} \hypertarget{z`e\string_Bz`e\string_M-b\{\string_Bb\{\string_B1}{}
\markboth{\textcolor{darkblue}{\textbf{\ipa{ʐe˩ʐe˧-bæ˩bæ˩}}}}{}
\textcolor{teal}{\zh{名词}} \hspace{4pt} \zh{声调类:} LM-L.
\zh{野棉花(直译:‘洋人花’)。} \textcolor{Sepia}{\selectlanguage{english}Wild cotton (literally: “Westerners' flower”).} \textcolor{PineGreen}{\selectlanguage{french}Coton sauvage;; littéralement “la fleur des Occidentaux”.} \zh{~【参考】~} \hyperlink{}{\textcolor{darkblue}{\textbf{\ipa{je˩ʐe˧}}}} 
\lhead{\firstmark}
\rhead{\botmark}

\subsection{\hspace{-0.5cm} {\Large \textcolor{darkblue}{\textbf{\ipa{ʐe˩ʐe˧-læ˧tsɯ˥}}}}\hspace{0.5cm}[\kern2pt{\textcolor{darkblue}{\textbf{\ipa{ʐe˩ʐe˧læ˧tsɯ˥}}}}\kern2pt]} \hypertarget{z`e\string_Bz`e\string_M-l\{\string_MtsM\string_T1}{}
\markboth{\textcolor{darkblue}{\textbf{\ipa{ʐe˩ʐe˧-læ˧tsɯ˥}}}}{}
\textcolor{teal}{\zh{名词}} \hspace{4pt} \zh{声调类:} LM-H\#.
\zh{喂猪的牧草。} \textcolor{Sepia}{\selectlanguage{english}One of the three main types of plants used for pig fodder.} \textcolor{PineGreen}{\selectlanguage{french}Sorte de fourrage pour les cochons (il y en a trois en tout).} 
\lhead{\firstmark}
\rhead{\botmark}

\subsection{\hspace{-0.5cm} {\Large \textcolor{darkblue}{\textbf{\ipa{ʐɤ˧\textsubscript{b}}}}}\hspace{0.5cm}[\kern2pt{\textcolor{darkblue}{\textbf{\ipa{ʐɤ˩˥}}}}\kern2pt]} \hypertarget{z`7\string_Mb1}{}
\markboth{\textcolor{darkblue}{\textbf{\ipa{ʐɤ˧\textsubscript{b}}}}}{}
\textcolor{teal}{\zh{动词}} \hspace{4pt} \zh{声调类:} M\textsubscript{b}.
\zh{饲养(动物)、养(孩子)、管(老人)。} \textcolor{Sepia}{\selectlanguage{english}To raise (animals, or children); to care for (the elderly).} \textcolor{PineGreen}{\selectlanguage{french}Élever (des enfants ou des animaux); s'occuper de (personnes âgées).}  ¶ \textcolor{darkblue}{\textbf{\ipa{bo˩ ʐɤ˧}}} \zh{养猪} \textcolor{Sepia}{\selectlanguage{english}to raise pigs} \textcolor{PineGreen}{\selectlanguage{french}élever des cochons}  
 ¶ \textcolor{darkblue}{\textbf{\ipa{ʐwæ˧zo˧ ʐɤ˧}}} \zh{养小马} \textcolor{Sepia}{\selectlanguage{english}to raise colts} \textcolor{PineGreen}{\selectlanguage{french}élever des poulains}  

\lhead{\firstmark}
\rhead{\botmark}

\subsection{\hspace{-0.5cm} {\Large \textcolor{darkblue}{\textbf{\ipa{ʐɤ˩\textsubscript{c}}}}}\hspace{0.5cm}[\kern2pt{\textcolor{darkblue}{\textbf{\ipa{ʐɤ˩˥}}}}\kern2pt]} \hypertarget{z`7\string_Bc1}{}
\markboth{\textcolor{darkblue}{\textbf{\ipa{ʐɤ˩\textsubscript{c}}}}}{}
\textcolor{teal}{\zh{量词}} \hspace{4pt} \zh{声调类:} L\textsubscript{c}.
\zh{量词:图案(画画或织布)(一个)。} \textcolor{Sepia}{\selectlanguage{english}Classifier for lines/patterns (in weaving, drawing, painting…).} \textcolor{PineGreen}{\selectlanguage{french}Classificateur des motifs, tracés, lignes, dans les dessins, peintures et tissages.} 
\lhead{\firstmark}
\rhead{\botmark}

\subsection{\hspace{-0.5cm} {\Large \textcolor{darkblue}{\textbf{\ipa{ʐɤ˩\textsubscript{a}}}}}\hspace{0.5cm}[\kern2pt{\textcolor{darkblue}{\textbf{\ipa{ʐɤ˩˥}}}}\kern2pt]} \hypertarget{z`7\string_Ba1}{}
\markboth{\textcolor{darkblue}{\textbf{\ipa{ʐɤ˩\textsubscript{a}}}}}{}
\textcolor{teal}{\zh{形容词}} \hspace{4pt} \zh{声调类:} L\textsubscript{a}.
\zh{干净。} \textcolor{Sepia}{\selectlanguage{english}Clean.} \textcolor{PineGreen}{\selectlanguage{french}Propre.}  ¶ \textcolor{darkblue}{\textbf{\ipa{ʈʂʰɯ˧ | ʐɤ˩-hĩ˩ ɲi˥}}} \zh{这是干净的} \textcolor{Sepia}{\selectlanguage{english}It is clean} \textcolor{PineGreen}{\selectlanguage{french}c'est propre}  
 ¶ \textcolor{darkblue}{\textbf{\ipa{mɤ˧-ʐɤ˩}}} \zh{不干净} \textcolor{Sepia}{\selectlanguage{english}not clean, dirty} \textcolor{PineGreen}{\selectlanguage{french}crasseux, dégoûtant (vêtements, nourriture…)}  

\lhead{\firstmark}
\rhead{\botmark}

\subsection{\hspace{-0.5cm} {\Large \textcolor{darkblue}{\textbf{\ipa{ʐɤ˩mi˩}}}}\hspace{0.5cm}[\kern2pt{\textcolor{darkblue}{\textbf{\ipa{ʐɤ˩mi˩˥}}}}\kern2pt]} \hypertarget{z`7\string_Bmi\string_B1}{}
\markboth{\textcolor{darkblue}{\textbf{\ipa{ʐɤ˩mi˩}}}}{}
\textcolor{teal}{\zh{名词}} \hspace{4pt} \zh{声调类:} L.
\zh{路。} \textcolor{Sepia}{\selectlanguage{english}Road.} \textcolor{PineGreen}{\selectlanguage{french}Route.}  ¶ \textcolor{darkblue}{\textbf{\ipa{hĩ˧ | ɖɯ˧-v̩˧\textasciitilde{}ɖɯ˧-v̩˧ | le˧-se˥, | ʐɤ˩mi˩ tsɤ˩˥!}}} \zh{路是人走出来的!} \textcolor{Sepia}{\selectlanguage{english}People walk, one after the other, and they create a path! (Context: when people go to fell trees on the mountain, where there was no path before, their passage open a new path, whose traces remain visible and may become a customary path.)} \textcolor{PineGreen}{\selectlanguage{french}Contexte: on va couper du bois en montagne, à un endroit où il n'y a pas de chemin. Les gens se succèdent, et cela finit par ouvrir un chemin/former une sorte de chemin}  
 \zh{量词}: \textcolor{darkblue}{\textbf{\ipa{kʰɯ˩}}} 
\lhead{\firstmark}
\rhead{\botmark}

\subsection{\hspace{-0.5cm} {\Large \textcolor{darkblue}{\textbf{\ipa{ʐɤ˩ni˩}}}}\hspace{0.5cm}[\kern2pt{\textcolor{darkblue}{\textbf{\ipa{ʐɤ˩ni˩˥}}}}\kern2pt]} \hypertarget{z`7\string_Bni\string_B1}{}
\markboth{\textcolor{darkblue}{\textbf{\ipa{ʐɤ˩ni˩}}}}{}
\textcolor{teal}{\zh{形容词}} \hspace{4pt} \zh{声调类:} L.
\zh{近。} \textcolor{Sepia}{\selectlanguage{english}Near.} \textcolor{PineGreen}{\selectlanguage{french}Proche.} 
\lhead{\firstmark}
\rhead{\botmark}

\subsection{\hspace{-0.5cm} {\Large \textcolor{darkblue}{\textbf{\ipa{ʐɤ˩qo˩}}}}\hspace{0.5cm}[\kern2pt{\textcolor{darkblue}{\textbf{\ipa{ʐɤ˩qo˩˥}}}}\kern2pt]} \hypertarget{z`7\string_Bqo\string_B1}{}
\markboth{\textcolor{darkblue}{\textbf{\ipa{ʐɤ˩qo˩}}}}{}
\textcolor{teal}{\zh{名词}} \hspace{4pt} \zh{声调类:} L.
\ding{202} \zh{小牛。} \textcolor{Sepia}{\selectlanguage{english}Calf.} \textcolor{PineGreen}{\selectlanguage{french}Veau.}  \zh{量词}: \textcolor{darkblue}{\textbf{\ipa{ɭɯ˧}}} \ding{203} \zh{公犏牛。} \textcolor{Sepia}{\selectlanguage{english}Male pianniu (hybrid of yak and cattle).} \textcolor{PineGreen}{\selectlanguage{french}Pianniu, pienniu, dzo, zopiok (mâle).} 
\lhead{\firstmark}
\rhead{\botmark}

\subsection{\hspace{-0.5cm} {\Large \textcolor{darkblue}{\textbf{\ipa{ʐɤ˩ʐɤ˧˥}}}}\hspace{0.5cm}[\kern2pt{\textcolor{darkblue}{\textbf{\ipa{ʐɤ˩ʐɤ˧˥}}}}\kern2pt]} \hypertarget{z`7\string_Bz`7\string_M\string_T1}{}
\markboth{\textcolor{darkblue}{\textbf{\ipa{ʐɤ˩ʐɤ˧˥}}}}{}
\textcolor{teal}{\zh{名词}} \hspace{4pt} \zh{声调类:} LM+MH\#.
\zh{花纹、图案。} \textcolor{Sepia}{\selectlanguage{english}Lines, pattern.} \textcolor{PineGreen}{\selectlanguage{french}Motif.}  ¶ \textcolor{darkblue}{\textbf{\ipa{ʐɤ˩ʐɤ˧ tʰi˧-di˥}}} \zh{有花纹} \textcolor{Sepia}{\selectlanguage{english}with lines/patterns / there are lines/patterns} \textcolor{PineGreen}{\selectlanguage{french}qui a des motifs, des dessins (ex.: un tissu)}  
 ¶ \textcolor{darkblue}{\textbf{\ipa{[F5] bɑ˩lɑ˩˥ | ʈʰɯ˧-ɭɯ˥-bi˩ | ʐɤ˩ʐɤ˧ tʰi˧-di˥}}} \zh{这衣服上面有花纹。} \textcolor{Sepia}{\selectlanguage{english}There are lines/patterns on this piece of clothing.} \textcolor{PineGreen}{\selectlanguage{french}sur ce vêtement il y a un motif}  
 \zh{量词}: \textcolor{darkblue}{\textbf{\ipa{ʐɤ˩}}} 
\lhead{\firstmark}
\rhead{\botmark}

\subsection{\hspace{-0.5cm} {\Large \textcolor{darkblue}{\textbf{\ipa{ʐɤ˩˧}}}}\hspace{0.5cm}[\kern2pt{\textcolor{darkblue}{\textbf{\ipa{ʐɤ˥}}}}\kern2pt]} \hypertarget{z`7\string_B\string_M1}{}
\markboth{\textcolor{darkblue}{\textbf{\ipa{ʐɤ˩˧}}}}{}
\textcolor{teal}{\zh{名词}} \hspace{4pt} \zh{声调类:} LM.
\zh{路(单音节)。} \textcolor{Sepia}{\selectlanguage{english}Road (monosyllable).} \textcolor{PineGreen}{\selectlanguage{french}Route (monosyllabe).}  ¶ \textcolor{darkblue}{\textbf{\ipa{ʐɤ˩mi˩-qo˥}}} \zh{路上} \textcolor{Sepia}{\selectlanguage{english}on the road, on the way} \textcolor{PineGreen}{\selectlanguage{french}sur le chemin}  
 ¶ \textcolor{darkblue}{\textbf{\ipa{ʐɤ˩mi˩-qo˥, | hĩ˧ se˧! |}}} \zh{路上有人走!} \textcolor{Sepia}{\selectlanguage{english}People are walking on the road/path!} \textcolor{PineGreen}{\selectlanguage{french}il y a des gens qui passent sur le chemin!}  
 ¶ \textcolor{darkblue}{\textbf{\ipa{ʐɤ˩ se˩-zo˩˥}}} \zh{旅人,特别指走马帮的商人} \textcolor{Sepia}{\selectlanguage{english}traveller, person who travels; specifically: person who does commerce by caravans} \textcolor{PineGreen}{\selectlanguage{french}voyageur, homme qui voyage; spécifiquement: personne partant faire du commerce en caravane}  
 \zh{量词}: \textcolor{darkblue}{\textbf{\ipa{kʰɯ˩}}} 
\lhead{\firstmark}
\rhead{\botmark}

\subsection{\hspace{-0.5cm} {\Large \textcolor{darkblue}{\textbf{\ipa{ʐo˩}}}}\hspace{0.5cm}[\kern2pt{\textcolor{darkblue}{\textbf{\ipa{ʐo˥}}}}\kern2pt]} \hypertarget{z`o\string_B1}{}
\markboth{\textcolor{darkblue}{\textbf{\ipa{ʐo˩}}}}{}
\textcolor{teal}{\zh{名词}} \hspace{4pt} \zh{声调类:} L.
\zh{中午。} \textcolor{Sepia}{\selectlanguage{english}Noon; lunch.} \textcolor{PineGreen}{\selectlanguage{french}Midi; repas de midi/déjeuner.}  ¶ \textcolor{darkblue}{\textbf{\ipa{ʐo˩ dzɯ˩˥}}} \zh{吃午饭} \textcolor{Sepia}{\selectlanguage{english}to have lunch} \textcolor{PineGreen}{\selectlanguage{french}prendre son déjeuner}  

\lhead{\firstmark}
\rhead{\botmark}

\subsection{\hspace{-0.5cm} {\Large \textcolor{darkblue}{\textbf{\ipa{ʐo˩\textsubscript{a}}}} \textsubscript{1}}\hspace{0.5cm}[\kern2pt{\textcolor{darkblue}{\textbf{\ipa{ʐo˥}}}}\kern2pt]} \hypertarget{z`o\string_Ba1}{}
\markboth{\textcolor{darkblue}{\textbf{\ipa{ʐo˩\textsubscript{a}}}} \textsubscript{1}}{}
\textcolor{teal}{\zh{动词}} \hspace{4pt} \zh{声调类:} L\textsubscript{a}.
\zh{甩来甩去。} \textcolor{Sepia}{\selectlanguage{english}To swing back and forth.} \textcolor{PineGreen}{\selectlanguage{french}Se balancer.}  ¶ \textcolor{darkblue}{\textbf{\ipa{ɖɯ˧-ʐo˩-ɻ̍˩}}} \zh{甩来甩去} \textcolor{Sepia}{\selectlanguage{english}to swing back and forth} \textcolor{PineGreen}{\selectlanguage{french}se balancer un peu}  
 ¶ \textcolor{darkblue}{\textbf{\ipa{ʐo˩\textasciitilde{}ʐo˧-ze˥}}} \zh{\mytextsc{red} \mytextsc{pfv}} \textcolor{Sepia}{\selectlanguage{english}\mytextsc{red} \mytextsc{pfv}} \textcolor{PineGreen}{\selectlanguage{french}\mytextsc{red} \mytextsc{pfv}}  
 ¶ \textcolor{darkblue}{\textbf{\ipa{[PHONO] le˧-ʐo˩\textasciitilde{}ʐo˩}}} \zh{\mytextsc{accomp} \mytextsc{red}} \textcolor{Sepia}{\selectlanguage{english}\mytextsc{accomp} \mytextsc{red}} \textcolor{PineGreen}{\selectlanguage{french}\mytextsc{accomp} \mytextsc{red}}  

\lhead{\firstmark}
\rhead{\botmark}

\subsection{\hspace{-0.5cm} {\Large \textcolor{darkblue}{\textbf{\ipa{ʐo˩\textsubscript{a}}}} \textsubscript{2}}\hspace{0.5cm}[\kern2pt{\textcolor{darkblue}{\textbf{\ipa{ʐo˩˥}}}}\kern2pt]} \hypertarget{z`o\string_Ba2}{}
\markboth{\textcolor{darkblue}{\textbf{\ipa{ʐo˩\textsubscript{a}}}} \textsubscript{2}}{}
\textcolor{teal}{\zh{形容词}} \hspace{4pt} \zh{声调类:} L\textsubscript{a}.
\zh{轻。} \textcolor{Sepia}{\selectlanguage{english}Light.} \textcolor{PineGreen}{\selectlanguage{french}Léger.} 
\lhead{\firstmark}
\rhead{\botmark}

\subsection{\hspace{-0.5cm} {\Large \textcolor{darkblue}{\textbf{\ipa{ʐo˩dzɯ˩}}}}\hspace{0.5cm}[\kern2pt{\textcolor{darkblue}{\textbf{\ipa{ʐo˩dzɯ˩˥}}}}\kern2pt]} \hypertarget{z`o\string_BdzM\string_B1}{}
\markboth{\textcolor{darkblue}{\textbf{\ipa{ʐo˩dzɯ˩}}}}{}
\textcolor{teal}{\zh{动词}} \hspace{4pt} \zh{声调类:} L.
\zh{吃午饭。} \textcolor{Sepia}{\selectlanguage{english}To eat lunch.} \textcolor{PineGreen}{\selectlanguage{french}Déjeuner, prendre le repas de midi.}  ¶ \textcolor{darkblue}{\textbf{\ipa{ʐo˩ dzɯ˩˥}}} \zh{吃午饭} \textcolor{Sepia}{\selectlanguage{english}to have lunch} \textcolor{PineGreen}{\selectlanguage{french}déjeuner (verbe), prendre le déjeuner}  
 ¶ \textcolor{darkblue}{\textbf{\ipa{ʐo˩ dzɯ˩-se˥}}} \zh{下午} \textcolor{Sepia}{\selectlanguage{english}afternoon} \textcolor{PineGreen}{\selectlanguage{french}l'après-midi}  

\lhead{\firstmark}
\rhead{\botmark}

\subsection{\hspace{-0.5cm} {\Large \textcolor{darkblue}{\textbf{\ipa{ʐo˩\textasciitilde{}ʐo˧˥}}}}\hspace{0.5cm}[\kern2pt{\textcolor{darkblue}{\textbf{\ipa{ʐo˧ʐo˧˥}}}}\kern2pt]} \hypertarget{z`o\string_B~z`o\string_M\string_T1}{}
\markboth{\textcolor{darkblue}{\textbf{\ipa{ʐo˩\textasciitilde{}ʐo˧˥}}}}{}
\textcolor{teal}{\zh{动词}} \hspace{4pt} \zh{声调类:} MH.
\zh{摔、摇摆。} \textcolor{Sepia}{\selectlanguage{english}To swing.} \textcolor{PineGreen}{\selectlanguage{french}Se balancer.} 
\lhead{\firstmark}
\rhead{\botmark}

\subsection{\hspace{-0.5cm} {\Large \textcolor{darkblue}{\textbf{\ipa{ʐɯ˥}}}}\hspace{0.5cm}[\kern2pt{\textcolor{darkblue}{\textbf{\ipa{ʐɯ˥}}}}\kern2pt]} \hypertarget{z`M\string_T1}{}
\markboth{\textcolor{darkblue}{\textbf{\ipa{ʐɯ˥}}}}{}
\textcolor{teal}{\zh{形容词}} \hspace{4pt} \zh{声调类:} H.
\zh{重。} \textcolor{Sepia}{\selectlanguage{english}Heavy.} \textcolor{PineGreen}{\selectlanguage{french}Lourd.} 
\lhead{\firstmark}
\rhead{\botmark}

\subsection{\hspace{-0.5cm} {\Large \textcolor{darkblue}{\textbf{\ipa{ʐɯ˧}}}}\hspace{0.5cm}[\kern2pt{\textcolor{darkblue}{\textbf{\ipa{ʐɯ˥}}}}\kern2pt]} \hypertarget{z`M\string_M1}{}
\markboth{\textcolor{darkblue}{\textbf{\ipa{ʐɯ˧}}}}{}
\textcolor{teal}{\zh{名词}} \hspace{4pt} \zh{声调类:} M.
\zh{酒。} \textcolor{Sepia}{\selectlanguage{english}Fermented alcohol, wine.} \textcolor{PineGreen}{\selectlanguage{french}Alcool fermenté, chang, vin.}  ¶ \textcolor{darkblue}{\textbf{\ipa{ʐɯ˧ pʰv̩˧˥}}} \zh{斟酒} \textcolor{Sepia}{\selectlanguage{english}to pour wine} \textcolor{PineGreen}{\selectlanguage{french}verser à boire}  

\lhead{\firstmark}
\rhead{\botmark}

\subsection{\hspace{-0.5cm} {\Large \textcolor{darkblue}{\textbf{\ipa{ʐɯ˧ɭɯ˧}}}}\hspace{0.5cm}[\kern2pt{\textcolor{darkblue}{\textbf{\ipa{ʐɯ˧ɭɯ˧}}}}\kern2pt]} \hypertarget{z`M\string_Ml\string_RM\string_M1}{}
\markboth{\textcolor{darkblue}{\textbf{\ipa{ʐɯ˧ɭɯ˧}}}}{}
\textcolor{teal}{\zh{动词}} \hspace{4pt} \zh{声调类:} M.
\zh{地震。} \textcolor{Sepia}{\selectlanguage{english}To shake (of earth), earthquake.} \textcolor{PineGreen}{\selectlanguage{french}Tremblement de terre/la terre tremble.}  ¶ \textcolor{darkblue}{\textbf{\ipa{ʐɯ˧ɭɯ˧-ze˧!}}} \zh{地震了!} \textcolor{Sepia}{\selectlanguage{english}There is an earthquake!} \textcolor{PineGreen}{\selectlanguage{french}Il y a un tremblement de terre! / La terre tremble!}  

\lhead{\firstmark}
\rhead{\botmark}

\subsection{\hspace{-0.5cm} {\Large \textcolor{darkblue}{\textbf{\ipa{ʐɯ˧nɑ˩}}}}\hspace{0.5cm}[\kern2pt{\textcolor{darkblue}{\textbf{\ipa{ʐɯ˧nɑ˩}}}}\kern2pt]} \hypertarget{z`M\string_MnA\string_B1}{}
\markboth{\textcolor{darkblue}{\textbf{\ipa{ʐɯ˧nɑ˩}}}}{}
\textcolor{teal}{\zh{名词}} \hspace{4pt} \zh{声调类:} L\#.
\zh{醇酒,好酒。} \textcolor{Sepia}{\selectlanguage{english}Strong alcohol, high-quality alcohol.} \textcolor{PineGreen}{\selectlanguage{french}Alcool fort; alcool de qualité supérieure.} 
\lhead{\firstmark}
\rhead{\botmark}

\subsection{\hspace{-0.5cm} {\Large \textcolor{darkblue}{\textbf{\ipa{ʐɯ˩dzi˥}}}}\hspace{0.5cm}[\kern2pt{\textcolor{darkblue}{\textbf{\ipa{ʐɯ˩dzi˥}}}}\kern2pt]} \hypertarget{z`M\string_Bdzi\string_T1}{}
\markboth{\textcolor{darkblue}{\textbf{\ipa{ʐɯ˩dzi˥}}}}{}
\textcolor{teal}{\zh{名词}} \hspace{4pt} \zh{声调类:} LH.
\zh{杉树。} \textcolor{Sepia}{\selectlanguage{english}Cedar.} \textcolor{PineGreen}{\selectlanguage{french}Cèdre.}  \zh{量词}: \textcolor{darkblue}{\textbf{\ipa{dzi˩}}} 
\lhead{\firstmark}
\rhead{\botmark}

\subsection{\hspace{-0.5cm} {\Large \textcolor{darkblue}{\textbf{\ipa{ʐɯ˩gv̩˩}}}}\hspace{0.5cm}[\kern2pt{\textcolor{darkblue}{\textbf{\ipa{ʐɯ˩gv̩˩˥}}}}\kern2pt]} \hypertarget{z`M\string_Bgv\string_=\string_B1}{}
\markboth{\textcolor{darkblue}{\textbf{\ipa{ʐɯ˩gv̩˩}}}}{}
\textcolor{teal}{\zh{名词}} \hspace{4pt} \zh{声调类:} L.
\zh{船。} \textcolor{Sepia}{\selectlanguage{english}Boat.} \textcolor{PineGreen}{\selectlanguage{french}Canot, bateau (utilisé uniquement pour les barques circulant sur le Lac, pas pour les autres bateaux).}  ¶ \textcolor{darkblue}{\textbf{\ipa{ʐɯ˩gv̩˩ dzi˩˥}}} \zh{坐船} \textcolor{Sepia}{\selectlanguage{english}to sit in a boat} \textcolor{PineGreen}{\selectlanguage{french}être assis dans un bateau, être à bord d'un bateau}  
 \zh{量词}: \textcolor{darkblue}{\textbf{\ipa{ɭɯ˧}}} \textcolor{darkblue}{\textbf{\ipa{nɑ˧}}} 
\lhead{\firstmark}
\rhead{\botmark}

\subsection{\hspace{-0.5cm} {\Large \textcolor{darkblue}{\textbf{\ipa{ʐɯ˩-mo˧˥}}}}\hspace{0.5cm}[\kern2pt{\textcolor{darkblue}{\textbf{\ipa{xxxx non-correspondance entre le nombre de morphèmes et le nombre de tons de morphèmes}}}}\kern2pt]} \hypertarget{z`M\string_B-mo\string_M\string_T1}{}
\markboth{\textcolor{darkblue}{\textbf{\ipa{ʐɯ˩-mo˧˥}}}}{}
\textcolor{teal}{\zh{名词}} \hspace{4pt} \zh{声调类:} LM+MH\#.
\zh{“杉树菌”:一种菌子。} \textcolor{Sepia}{\selectlanguage{english}“mushroom of the cedar tree”: a sort of mushroom often found close to cedar trees.} \textcolor{PineGreen}{\selectlanguage{french}“champignon des cèdres”; champignon comestible, de la même famille que le “champignon des sapins”, \textcolor{darkblue}{\textbf{\ipa{/tʰo˧-mo˩/}}}.}  ¶ \textcolor{darkblue}{\textbf{\ipa{tʰo˧mo˩-ʐɯ˩mo˩}}} \zh{松树菌与杉树菌} \textcolor{Sepia}{\selectlanguage{english}Pine-tree mushroom and cedar-tree mushroom} \textcolor{PineGreen}{\selectlanguage{french}champignon des sapins et champignon des cèdres}  
\zh{~【参考】~} \hyperlink{}{\textcolor{darkblue}{\textbf{\ipa{ʐɯ˩dzi˥}}}} 
\lhead{\firstmark}
\rhead{\botmark}

\subsection{\hspace{-0.5cm} {\Large \textcolor{darkblue}{\textbf{\ipa{ʐɯ˩tse˧}}}}\hspace{0.5cm}[\kern2pt{\textcolor{darkblue}{\textbf{\ipa{ʐɯ˩tse˥}}}}\kern2pt]} \hypertarget{z`M\string_Btse\string_M1}{}
\markboth{\textcolor{darkblue}{\textbf{\ipa{ʐɯ˩tse˧}}}}{}
\textcolor{teal}{\zh{名词}} \hspace{4pt} \zh{声调类:} LM.
\zh{山神。} \textcolor{Sepia}{\selectlanguage{english}Mountain spirit.} \textcolor{PineGreen}{\selectlanguage{french}Esprit de la montagne.}  \zh{量词}: \textcolor{darkblue}{\textbf{\ipa{v̩˧}}} 
\lhead{\firstmark}
\rhead{\botmark}

\subsection{\hspace{-0.5cm} {\Large \textcolor{darkblue}{\textbf{\ipa{ʐɯ˩tse˧-mæ˧ʂæ˩}}}}\hspace{0.5cm}[\kern2pt{\textcolor{darkblue}{\textbf{\ipa{ʐɯ˩tse˧mæ˧ʂæ˩}}}}\kern2pt]} \hypertarget{z`M\string_Btse\string_M-m\{\string_Ms`\{\string_B1}{}
\markboth{\textcolor{darkblue}{\textbf{\ipa{ʐɯ˩tse˧-mæ˧ʂæ˩}}}}{}
\textcolor{teal}{\zh{名词}} \hspace{4pt} \zh{声调类:} LM-L\#.
\zh{锦鸡。} \textcolor{Sepia}{\selectlanguage{english}Golden pheasant.} \textcolor{PineGreen}{\selectlanguage{french}Faisan doré.} \zh{当地汉语方言:}\zh{山扎拉。}\zh{~【参考】~} \hyperlink{}{\textcolor{darkblue}{\textbf{\ipa{ʐɯ˩tse˧}}}} 
\lhead{\firstmark}
\rhead{\botmark}

\subsection{\hspace{-0.5cm} {\Large \textcolor{darkblue}{\textbf{\ipa{ʐɯ˩tsɯ˧}}}}\hspace{0.5cm}[\kern2pt{\textcolor{darkblue}{\textbf{\ipa{ʐɯ˩tsɯ˥}}}}\kern2pt]} \hypertarget{z`M\string_BtsM\string_M1}{}
\markboth{\textcolor{darkblue}{\textbf{\ipa{ʐɯ˩tsɯ˧}}}}{}
\textcolor{teal}{\zh{名词}} \hspace{4pt} \zh{声调类:} LM.
\zh{日子(汉语借词)。} \textcolor{Sepia}{\selectlanguage{english}Days; life; time.} \textcolor{PineGreen}{\selectlanguage{french}Jours, temps.}  \zh{【借词】} \zh{日子}
 ¶ \textcolor{darkblue}{\textbf{\ipa{ʐɯ˩tsɯ˧ ʈʂɤ˧}}} \zh{算日子(为了选择吉利的一天)} \textcolor{Sepia}{\selectlanguage{english}to look for an auspicious date (for building a house or other important project)} \textcolor{PineGreen}{\selectlanguage{french}rechercher une date propice (pour la construction d'une maison ou autre projet important)}  

\lhead{\firstmark}
\rhead{\botmark}

\subsection{\hspace{-0.5cm} {\Large \textcolor{darkblue}{\textbf{\ipa{ʐɯ˩tsɯ˧mɤ˩ʈʂʰɤ˩}}}}\hspace{0.5cm}[\kern2pt{\textcolor{darkblue}{\textbf{\ipa{xxxx non-correspondance entre le nombre de morphèmes et le nombre de tons de morphèmes}}}}\kern2pt]} \hypertarget{z`M\string_BtsM\string_Mm7\string_Bt`s`\string_h7\string_B1}{}
\markboth{\textcolor{darkblue}{\textbf{\ipa{ʐɯ˩tsɯ˧mɤ˩ʈʂʰɤ˩}}}}{}
\textcolor{teal}{\zh{名词}} \hspace{4pt} \zh{声调类:} LM-L.
\zh{褥子。} \textcolor{Sepia}{\selectlanguage{english}Mattress.} \textcolor{PineGreen}{\selectlanguage{french}Matelas.}  \zh{量词}: \textcolor{darkblue}{\textbf{\ipa{tsʰi˥}}} 
\lhead{\firstmark}
\rhead{\botmark}

\subsection{\hspace{-0.5cm} {\Large \textcolor{darkblue}{\textbf{\ipa{ʐv̩˧}}}}\hspace{0.5cm}[\kern2pt{\textcolor{darkblue}{\textbf{\ipa{ʐv̩˥}}}}\kern2pt]} \hypertarget{z`v\string_=\string_M1}{}
\markboth{\textcolor{darkblue}{\textbf{\ipa{ʐv̩˧}}}}{}
\textcolor{teal}{\zh{数词}} \hspace{4pt} \zh{声调类:} M? H\#?.
\zh{4。} \textcolor{Sepia}{\selectlanguage{english}4.} \textcolor{PineGreen}{\selectlanguage{french}4.} 
\lhead{\firstmark}
\rhead{\botmark}

\subsection{\hspace{-0.5cm} {\Large \textcolor{darkblue}{\textbf{\ipa{ʐv̩˧˥}}}}\hspace{0.5cm}[\kern2pt{\textcolor{darkblue}{\textbf{\ipa{ʐv̩˧˥}}}}\kern2pt]} \hypertarget{z`v\string_=\string_M\string_T1}{}
\markboth{\textcolor{darkblue}{\textbf{\ipa{ʐv̩˧˥}}}}{}
\textcolor{teal}{\zh{动词}} \hspace{4pt} \zh{声调类:} MH.
\zh{缝。} \textcolor{Sepia}{\selectlanguage{english}To sew.} \textcolor{PineGreen}{\selectlanguage{french}Coudre.} 
\lhead{\firstmark}
\rhead{\botmark}

\subsection{\hspace{-0.5cm} {\Large \textcolor{darkblue}{\textbf{\ipa{ʐv̩˩\textsubscript{a}}}} \textsubscript{1}}\hspace{0.5cm}[\kern2pt{\textcolor{darkblue}{\textbf{\ipa{ʐv̩˩˥}}}}\kern2pt]} \hypertarget{z`v\string_=\string_Ba1}{}
\markboth{\textcolor{darkblue}{\textbf{\ipa{ʐv̩˩\textsubscript{a}}}} \textsubscript{1}}{}
\textcolor{teal}{\zh{动词}} \hspace{4pt} \zh{声调类:} L\textsubscript{a}.
\ding{202} \zh{揉(面)。} \textcolor{Sepia}{\selectlanguage{english}To knead (dough).} \textcolor{PineGreen}{\selectlanguage{french}Pétrir (la pâte), malaxer.}  ¶ \textcolor{darkblue}{\textbf{\ipa{pɤ˩jɤ˧ ʐv̩˥}}} \zh{揉面} \textcolor{Sepia}{\selectlanguage{english}to knead dough} \textcolor{PineGreen}{\selectlanguage{french}pétrir la pâte}  
 ¶ \textcolor{darkblue}{\textbf{\ipa{ʐv̩˧\textasciitilde{}ʐv̩˥}}} \zh{\mytextsc{重叠}} \textcolor{Sepia}{\selectlanguage{english}\mytextsc{red}} \textcolor{PineGreen}{\selectlanguage{french}\mytextsc{red}}  
 ¶ \textcolor{darkblue}{\textbf{\ipa{ɖɯ˧-kʰwɤ˧ ʐv̩˥}}} \zh{揉一块(面团)} \textcolor{Sepia}{\selectlanguage{english}to knead a piece (of dough)} \textcolor{PineGreen}{\selectlanguage{french}pétrir un morceau}  
\ding{203} \zh{皱(衣服)。} \textcolor{Sepia}{\selectlanguage{english}To crease, to crumple, to wrinkle.} \textcolor{PineGreen}{\selectlanguage{french}Froisser, plisser.}  ¶ \textcolor{darkblue}{\textbf{\ipa{bɑ˩lɑ˩ ʐv̩˥(-ze˩)}}} \zh{衣服皱了} \textcolor{Sepia}{\selectlanguage{english}to crease clothes; the clothes have been creased; the clothes are creased} \textcolor{PineGreen}{\selectlanguage{french}les vêtements sont froissés, les vêtements ont été froissés}  
 ¶ \textcolor{darkblue}{\textbf{\ipa{ʐv̩˧\textasciitilde{}ʐv̩˥}}} \zh{\mytextsc{重叠}} \textcolor{Sepia}{\selectlanguage{english}\mytextsc{red}} \textcolor{PineGreen}{\selectlanguage{french}\mytextsc{red}}  
 ¶ \textcolor{darkblue}{\textbf{\ipa{le˧-ʐv̩˧\textasciitilde{}ʐv̩˥-ze˩}}} \zh{\mytextsc{accomp} \string_ \mytextsc{red} \mytextsc{pfv}} \textcolor{Sepia}{\selectlanguage{english}\mytextsc{accomp} \string_ \mytextsc{red} \mytextsc{pfv}} \textcolor{PineGreen}{\selectlanguage{french}\mytextsc{accomp} \string_ \mytextsc{red} \mytextsc{pfv}}  

\lhead{\firstmark}
\rhead{\botmark}

\subsection{\hspace{-0.5cm} {\Large \textcolor{darkblue}{\textbf{\ipa{ʐv̩˩\textsubscript{a}}}} \textsubscript{2}}\hspace{0.5cm}[\kern2pt{\textcolor{darkblue}{\textbf{\ipa{ʐv̩˩˥}}}}\kern2pt]} \hypertarget{z`v\string_=\string_Ba2}{}
\markboth{\textcolor{darkblue}{\textbf{\ipa{ʐv̩˩\textsubscript{a}}}} \textsubscript{2}}{}
\textcolor{teal}{\zh{形容词}} \hspace{4pt} \zh{声调类:} L\textsubscript{a}.
\zh{好吃。} \textcolor{Sepia}{\selectlanguage{english}Delicious, good (to the taste).} \textcolor{PineGreen}{\selectlanguage{french}Bon (au goût).} 
\lhead{\firstmark}
\rhead{\botmark}

\subsection{\hspace{-0.5cm} {\Large \textcolor{darkblue}{\textbf{\ipa{ʐv̩˧bæ˧}}}}\hspace{0.5cm}[\kern2pt{\textcolor{darkblue}{\textbf{\ipa{ʐv̩˧bæ˧}}}}\kern2pt]} \hypertarget{z`v\string_=\string_Mb\{\string_M1}{}
\markboth{\textcolor{darkblue}{\textbf{\ipa{ʐv̩˧bæ˧}}}}{}
\textcolor{teal}{\zh{名词}} \hspace{4pt} \zh{声调类:} M.
\zh{蛇。} \textcolor{Sepia}{\selectlanguage{english}Snake, serpent.} \textcolor{PineGreen}{\selectlanguage{french}Serpent.}  ¶ \textcolor{darkblue}{\textbf{\ipa{ʐv̩˧bæ˧ ɣɯ˩ pʰv̩˩}}} \zh{蛇蜕皮} \textcolor{Sepia}{\selectlanguage{english}The snake sheds skin / exuviates} \textcolor{PineGreen}{\selectlanguage{french}Le serpent mue}  
 \zh{量词}: \textcolor{darkblue}{\textbf{\ipa{mi˩}}} 
\lhead{\firstmark}
\rhead{\botmark}

\subsection{\hspace{-0.5cm} {\Large \textcolor{darkblue}{\textbf{\ipa{ʐv̩˧bæ˧-bv̩˧-hɑ\#˥}}}}\hspace{0.5cm}[\kern2pt{\textcolor{darkblue}{\textbf{\ipa{xxxx non-correspondance entre le nombre de morphèmes et le nombre de tons de morphèmes}}}}\kern2pt]} \hypertarget{z`v\string_=\string_Mb\{\string_M-bv\string_=\string_M-hA\#\string_T1}{}
\markboth{\textcolor{darkblue}{\textbf{\ipa{ʐv̩˧bæ˧-bv̩˧-hɑ\#˥}}}}{}
\textcolor{teal}{\zh{名词}} \hspace{4pt} \zh{声调类:} \#H.
\zh{能喂给猪的三种草之一。} \textcolor{Sepia}{\selectlanguage{english}One of the three types of pig fodder.} \textcolor{PineGreen}{\selectlanguage{french}L'une des trois sortes de fourrage que l'on donne aux cochons; M18 propose comme étymologie populaire “serpent en colère”, du fait que cette plante a des feuilles grasses qui ressemblent à un serpent, et qui sont entortillées.}  \zh{量词}: \textcolor{darkblue}{\textbf{\ipa{qɑ˩}}} 
\lhead{\firstmark}
\rhead{\botmark}

\subsection{\hspace{-0.5cm} {\Large \textcolor{darkblue}{\textbf{\ipa{ʐv̩˧bæ˧-pʰv̩\#˥}}}}\hspace{0.5cm}[\kern2pt{\textcolor{darkblue}{\textbf{\ipa{xxxx non-correspondance entre le nombre de morphèmes et le nombre de tons de morphèmes}}}}\kern2pt]} \hypertarget{z`v\string_=\string_Mb\{\string_M-p\string_hv\string_=\#\string_T1}{}
\markboth{\textcolor{darkblue}{\textbf{\ipa{ʐv̩˧bæ˧-pʰv̩\#˥}}}}{}
\textcolor{teal}{\zh{名词}} \hspace{4pt} \zh{声调类:} \#H.
\zh{公蛇。} \textcolor{Sepia}{\selectlanguage{english}Male snake.} \textcolor{PineGreen}{\selectlanguage{french}Serpent mâle.}  \zh{量词}: \textcolor{darkblue}{\textbf{\ipa{mi˩}}} 
\lhead{\firstmark}
\rhead{\botmark}

\subsection{\hspace{-0.5cm} {\Large \textcolor{darkblue}{\textbf{\ipa{ʐv̩˧bæ˧-zo\#˥}}}}\hspace{0.5cm}[\kern2pt{\textcolor{darkblue}{\textbf{\ipa{xxxx non-correspondance entre le nombre de morphèmes et le nombre de tons de morphèmes}}}}\kern2pt]} \hypertarget{z`v\string_=\string_Mb\{\string_M-zo\#\string_T1}{}
\markboth{\textcolor{darkblue}{\textbf{\ipa{ʐv̩˧bæ˧-zo\#˥}}}}{}
\textcolor{teal}{\zh{名词}} \hspace{4pt} \zh{声调类:} \#H.
\zh{小蛇。} \textcolor{Sepia}{\selectlanguage{english}Baby snake.} \textcolor{PineGreen}{\selectlanguage{french}Petit serpent.}  \zh{量词}: \textcolor{darkblue}{\textbf{\ipa{ɭɯ˧}}} 
\lhead{\firstmark}
\rhead{\botmark}

\subsection{\hspace{-0.5cm} {\Large \textcolor{darkblue}{\textbf{\ipa{ʐv̩˧bɤ\#˥}}}}\hspace{0.5cm}[\kern2pt{\textcolor{darkblue}{\textbf{\ipa{ʐv̩˧bɤ˧}}}}\kern2pt]} \hypertarget{z`v\string_=\string_Mb7\#\string_T1}{}
\markboth{\textcolor{darkblue}{\textbf{\ipa{ʐv̩˧bɤ\#˥}}}}{}
\textcolor{teal}{\zh{名词}} \hspace{4pt} \zh{声调类:} \#H.
\zh{高山普米族(永宁以北地区:木里等)。} \textcolor{Sepia}{\selectlanguage{english}The Pumi (Prinmi) people of the mountains.} \textcolor{PineGreen}{\selectlanguage{french}Les Pumi des montagnes, du côté de Muli et Jiaze.}  \zh{量词}: \textcolor{darkblue}{\textbf{\ipa{v̩˧}}} 
\lhead{\firstmark}
\rhead{\botmark}

\subsection{\hspace{-0.5cm} {\Large \textcolor{darkblue}{\textbf{\ipa{ʐv̩˧di˧˥}}}}\hspace{0.5cm}[\kern2pt{\textcolor{darkblue}{\textbf{\ipa{ʐv̩˧di˧˥}}}}\kern2pt]} \hypertarget{z`v\string_=\string_Mdi\string_M\string_T1}{}
\markboth{\textcolor{darkblue}{\textbf{\ipa{ʐv̩˧di˧˥}}}}{}
\textcolor{teal}{\zh{名词}} \hspace{4pt} \zh{声调类:} MH\#.
\zh{金沙江边的地方(气候热)。} \textcolor{Sepia}{\selectlanguage{english}The warm area on the banks of the Yangtze river: Fengke, Labai….} \textcolor{PineGreen}{\selectlanguage{french}Les rives du Yangtze; le climat y est chaud et humide. Ces régions sont perçus par les Na de Yongning comme peuplées de Pumi; ils imaginent que les habitants de Fengke et Labai seraient des descendants des Pumi. (Source: consultants F4, F5, M21.).} 
\lhead{\firstmark}
\rhead{\botmark}

\subsection{\hspace{-0.5cm} {\Large \textcolor{darkblue}{\textbf{\ipa{ʐv̩˧dzi˩}}}}\hspace{0.5cm}[\kern2pt{\textcolor{darkblue}{\textbf{\ipa{ʐv̩˧dzi˩}}}}\kern2pt]} \hypertarget{z`v\string_=\string_Mdzi\string_B1}{}
\markboth{\textcolor{darkblue}{\textbf{\ipa{ʐv̩˧dzi˩}}}}{}
\textcolor{teal}{\zh{名词}} \hspace{4pt} \zh{声调类:} L\#.
\zh{柳树,杨柳。} \textcolor{Sepia}{\selectlanguage{english}Willow tree.} \textcolor{PineGreen}{\selectlanguage{french}Saule.}  \zh{量词}: \textcolor{darkblue}{\textbf{\ipa{dzi˩}}} 
\lhead{\firstmark}
\rhead{\botmark}

\subsection{\hspace{-0.5cm} {\Large \textcolor{darkblue}{\textbf{\ipa{ʐv̩˧hĩ\#˥}}}}\hspace{0.5cm}[\kern2pt{\textcolor{darkblue}{\textbf{\ipa{ʐv̩˧hĩ˧}}}}\kern2pt]} \hypertarget{z`v\string_=\string_Mhi\string_~\#\string_T1}{}
\markboth{\textcolor{darkblue}{\textbf{\ipa{ʐv̩˧hĩ\#˥}}}}{}
\textcolor{teal}{\zh{名词}} \hspace{4pt} \zh{声调类:} \#H.
\zh{普米族。} \textcolor{Sepia}{\selectlanguage{english}One of the designations of the Pumi (ethnic group).} \textcolor{PineGreen}{\selectlanguage{french}Désignation des Pumi.}  \zh{量词}: \textcolor{darkblue}{\textbf{\ipa{v̩˧}}} 
\lhead{\firstmark}
\rhead{\botmark}

\subsection{\hspace{-0.5cm} {\Large \textcolor{darkblue}{\textbf{\ipa{ʐv̩˩-ɬi˩mi˩}}}}\hspace{0.5cm}[\kern2pt{\textcolor{darkblue}{\textbf{\ipa{xxxx non-correspondance entre le nombre de morphèmes et le nombre de tons de morphèmes}}}}\kern2pt]} \hypertarget{z`v\string_=\string_B-Ki\string_Bmi\string_B1}{}
\markboth{\textcolor{darkblue}{\textbf{\ipa{ʐv̩˩-ɬi˩mi˩}}}}{}
\textcolor{teal}{\zh{名词}} \hspace{4pt} \zh{声调类:} L.
\zh{四月。} \textcolor{Sepia}{\selectlanguage{english}4th month.} \textcolor{PineGreen}{\selectlanguage{french}4e mois.} 
\lhead{\firstmark}
\rhead{\botmark}

\subsection{\hspace{-0.5cm} {\Large \textcolor{darkblue}{\textbf{\ipa{ʐv̩˩ɭɯ˥}}}}\hspace{0.5cm}[\kern2pt{\textcolor{darkblue}{\textbf{\ipa{ʐv̩˩ɭɯ˥}}}}\kern2pt]} \hypertarget{z`v\string_=\string_Bl\string_RM\string_T1}{}
\markboth{\textcolor{darkblue}{\textbf{\ipa{ʐv̩˩ɭɯ˥}}}}{}
\textcolor{teal}{\zh{名词}} \hspace{4pt} \zh{声调类:} LH.
\zh{支撑顶板的梁。} \textcolor{Sepia}{\selectlanguage{english}Beam.} \textcolor{PineGreen}{\selectlanguage{french}Poutre soutenant la toiture, posée horizontalement, dans le sens de la longueur du bâtiment. Sur elle reposent les poutrelles courtes posées inclinées dans le sens de la largeur du bâtiment, /hæ̃˧kʰɤ˧˥/.}  \zh{量词}: \textcolor{darkblue}{\textbf{\ipa{ɭɯ˧}}} 
\lhead{\firstmark}
\rhead{\botmark}

\subsection{\hspace{-0.5cm} {\Large \textcolor{darkblue}{\textbf{\ipa{ʐv̩˩mi˩}}}}\hspace{0.5cm}[\kern2pt{\textcolor{darkblue}{\textbf{\ipa{ʐv̩˩mi˩˥}}}}\kern2pt]} \hypertarget{z`v\string_=\string_Bmi\string_B1}{}
\markboth{\textcolor{darkblue}{\textbf{\ipa{ʐv̩˩mi˩}}}}{}
\textcolor{teal}{\zh{名词}} \hspace{4pt} \zh{声调类:} L.
\zh{弓。} \textcolor{Sepia}{\selectlanguage{english}Bow (archery bow).} \textcolor{PineGreen}{\selectlanguage{french}Arc.}  \zh{量词}: \textcolor{darkblue}{\textbf{\ipa{nɑ˧}}} 
\lhead{\firstmark}
\rhead{\botmark}

\subsection{\hspace{-0.5cm} {\Large \textcolor{darkblue}{\textbf{\ipa{ʐv̩˧mi\#˥}}}}\hspace{0.5cm}[\kern2pt{\textcolor{darkblue}{\textbf{\ipa{ʐv̩˧mi˧}}}}\kern2pt]} \hypertarget{z`v\string_=\string_Mmi\#\string_T1}{}
\markboth{\textcolor{darkblue}{\textbf{\ipa{ʐv̩˧mi\#˥}}}}{}
\textcolor{teal}{\zh{名词}} \hspace{4pt} \zh{声调类:} \#H.
\zh{孙女。} \textcolor{Sepia}{\selectlanguage{english}Granddaughter.} \textcolor{PineGreen}{\selectlanguage{french}Petite-fille.}  ¶ \textcolor{darkblue}{\textbf{\ipa{njɤ˧ | ʐv̩˧mi˧ | ɖɯ˧-ɭɯ˧ dʑo˧}}} \zh{我有一个孙女。} \textcolor{Sepia}{\selectlanguage{english}I have a granddaughter.} \textcolor{PineGreen}{\selectlanguage{french}J'ai une petite-fille.}  
 \zh{量词}: \textcolor{darkblue}{\textbf{\ipa{ɭɯ˧}}} 
\lhead{\firstmark}
\rhead{\botmark}

\subsection{\hspace{-0.5cm} {\Large \textcolor{darkblue}{\textbf{\ipa{ʐv̩˧mv̩˧lɑ˧di˧˥}}}}\hspace{0.5cm}[\kern2pt{\textcolor{darkblue}{\textbf{\ipa{ʐv̩˧mv̩˧lɑ˧di˧˥}}}}\kern2pt]} \hypertarget{z`v\string_=\string_Mmv\string_=\string_MlA\string_Mdi\string_M\string_T1}{}
\markboth{\textcolor{darkblue}{\textbf{\ipa{ʐv̩˧mv̩˧lɑ˧di˧˥}}}}{}
\textcolor{teal}{\zh{名词}} \hspace{4pt} \zh{声调类:} MH\#.
\zh{江边普米族地区(带偏见的说法)。} \textcolor{Sepia}{\selectlanguage{english}The territory of the Pumi people on the banks of the Yangtze river. This area is perceived as less central and pleasant than Yongning.} \textcolor{PineGreen}{\selectlanguage{french}Les territoires des Pumi, au bord du fleuve Yangtze. Le terme est connoté péjorativement: cette région est perçue comme périphérique et moins plaisante que la plaine de Yongning.}  ¶ \textcolor{darkblue}{\textbf{\ipa{ʐv̩˧mv̩˧lɑ˧di˧-hĩ˥}}} \zh{普米族地区的人们} \textcolor{Sepia}{\selectlanguage{english}people from the Pumi territories} \textcolor{PineGreen}{\selectlanguage{french}habitants des territoires pumi des bords du fleuve; personnes pumi}  
 \zh{量词}: \textcolor{darkblue}{\textbf{\ipa{v̩˧}}} 
\lhead{\firstmark}
\rhead{\botmark}

\subsection{\hspace{-0.5cm} {\Large \textcolor{darkblue}{\textbf{\ipa{ʐv̩˧-ɲi˧-ʁo˧tʰo˥}}}}\hspace{0.5cm}[\kern2pt{\textcolor{darkblue}{\textbf{\ipa{xxxx non-correspondance entre le nombre de morphèmes et le nombre de tons de morphèmes}}}}\kern2pt]} \hypertarget{z`v\string_=\string_M-Ji\string_M-Ro\string_Mt\string_ho\string_T1}{}
\markboth{\textcolor{darkblue}{\textbf{\ipa{ʐv̩˧-ɲi˧-ʁo˧tʰo˥}}}}{}
\textcolor{teal}{\zh{助词}} \hspace{4pt} \zh{声调类:} H\#.
\zh{四天以后。} \textcolor{Sepia}{\selectlanguage{english}In four days.} \textcolor{PineGreen}{\selectlanguage{french}Dans quatre jours.} 
\lhead{\firstmark}
\rhead{\botmark}

\subsection{\hspace{-0.5cm} {\Large \textcolor{darkblue}{\textbf{\ipa{ʐv̩˧ɻ̍˥}}}}\hspace{0.5cm}[\kern2pt{\textcolor{darkblue}{\textbf{\ipa{ʐv̩˧ɻ̍˥}}}}\kern2pt]} \hypertarget{z`v\string_=\string_Mr£`̍\string_T1}{}
\markboth{\textcolor{darkblue}{\textbf{\ipa{ʐv̩˧ɻ̍˥}}}}{}
\textcolor{teal}{\zh{形容词}} \hspace{4pt} \zh{声调类:} H\#.
\zh{正方形。} \textcolor{Sepia}{\selectlanguage{english}Square.} \textcolor{PineGreen}{\selectlanguage{french}Carré.}  ¶ \textcolor{darkblue}{\textbf{\ipa{ʐv̩˩-hĩ˩˥}}} \zh{方形的} \textcolor{Sepia}{\selectlanguage{english}\mytextsc{nmlz}} \textcolor{PineGreen}{\selectlanguage{french}\mytextsc{nmlz}}  
 ¶ \textcolor{darkblue}{\textbf{\ipa{ʐv̩˧ɻ̍˥-gv̩˩}}} \zh{方形的} \textcolor{Sepia}{\selectlanguage{english}square} \textcolor{PineGreen}{\selectlanguage{french}carré}  

\lhead{\firstmark}
\rhead{\botmark}

\subsection{\hspace{-0.5cm} {\Large \textcolor{darkblue}{\textbf{\ipa{ʐv̩˧-tsʰi˩}}}}\hspace{0.5cm}[\kern2pt{\textcolor{darkblue}{\textbf{\ipa{xxxx non-correspondance entre le nombre de morphèmes et le nombre de tons de morphèmes}}}}\kern2pt]} \hypertarget{z`v\string_=\string_M-ts\string_hi\string_B1}{}
\markboth{\textcolor{darkblue}{\textbf{\ipa{ʐv̩˧-tsʰi˩}}}}{}
\textcolor{teal}{\zh{数词}} \hspace{4pt} \zh{声调类:} L\#.
\zh{40。} \textcolor{Sepia}{\selectlanguage{english}40.} \textcolor{PineGreen}{\selectlanguage{french}40.} 
\lhead{\firstmark}
\rhead{\botmark}

\subsection{\hspace{-0.5cm} {\Large \textcolor{darkblue}{\textbf{\ipa{ʐv̩˧v̩˥-ʐv̩˩mi˩}}}}\hspace{0.5cm}[\kern2pt{\textcolor{darkblue}{\textbf{\ipa{ʐv̩˧v̩˥ʐv̩˩mi˩}}}}\kern2pt]} \hypertarget{z`v\string_=\string_Mv\string_=\string_T-z`v\string_=\string_Bmi\string_B1}{}
\markboth{\textcolor{darkblue}{\textbf{\ipa{ʐv̩˧v̩˥-ʐv̩˩mi˩}}}}{}
\textcolor{teal}{\zh{名词}} \hspace{4pt} \zh{声调类:} H\#-.
\zh{孙子孙女。} \textcolor{Sepia}{\selectlanguage{english}Grandchildren.} \textcolor{PineGreen}{\selectlanguage{french}Petits-enfants.} 
\lhead{\firstmark}
\rhead{\botmark}

\subsection{\hspace{-0.5cm} {\Large \textcolor{darkblue}{\textbf{\ipa{ʐv̩˧v̩\#˥}}}}\hspace{0.5cm}[\kern2pt{\textcolor{darkblue}{\textbf{\ipa{ʐv̩˧v̩˧}}}}\kern2pt]} \hypertarget{z`v\string_=\string_Mv\string_=\#\string_T1}{}
\markboth{\textcolor{darkblue}{\textbf{\ipa{ʐv̩˧v̩\#˥}}}}{}
\textcolor{teal}{\zh{名词}} \hspace{4pt} \zh{声调类:} \#H.
\zh{孙子。} \textcolor{Sepia}{\selectlanguage{english}Grandson.} \textcolor{PineGreen}{\selectlanguage{french}Petit-fils.}  ¶ \textcolor{darkblue}{\textbf{\ipa{njɤ˧ | ʐv̩˧v̩˧ ɖɯ˧-ɭɯ˧ dʑo˧.}}} \zh{我有一个孙子。} \textcolor{Sepia}{\selectlanguage{english}I have a grandson.} \textcolor{PineGreen}{\selectlanguage{french}j'ai un petit-fils}  
 \zh{量词}: \textcolor{darkblue}{\textbf{\ipa{ɭɯ˧}}} 
\lhead{\firstmark}
\rhead{\botmark}

\subsection{\hspace{-0.5cm} {\Large \textcolor{darkblue}{\textbf{\ipa{ʐv̩˧-zo\#˥}}}}\hspace{0.5cm}[\kern2pt{\textcolor{darkblue}{\textbf{\ipa{xxxx non-correspondance entre le nombre de morphèmes et le nombre de tons de morphèmes}}}}\kern2pt]} \hypertarget{z`v\string_=\string_M-zo\#\string_T1}{}
\markboth{\textcolor{darkblue}{\textbf{\ipa{ʐv̩˧-zo\#˥}}}}{}
\textcolor{teal}{\zh{名词}} \hspace{4pt} \zh{声调类:} \#H.
\zh{小弓。} \textcolor{Sepia}{\selectlanguage{english}Small bow (archery bow).} \textcolor{PineGreen}{\selectlanguage{french}Petit arc.}  \zh{量词}: \textcolor{darkblue}{\textbf{\ipa{nɑ˧}}} 
\lhead{\firstmark}
\rhead{\botmark}

\subsection{\hspace{-0.5cm} {\Large \textcolor{darkblue}{\textbf{\ipa{ʐwæ˥}}}}\hspace{0.5cm}[\kern2pt{\textcolor{darkblue}{\textbf{\ipa{ʐwæ˥}}}}\kern2pt]} \hypertarget{z`w\{\string_T1}{}
\markboth{\textcolor{darkblue}{\textbf{\ipa{ʐwæ˥}}}}{}
\textcolor{teal}{\zh{名词}} \hspace{4pt} \zh{声调类:} \#H.
\zh{马。} \textcolor{Sepia}{\selectlanguage{english}Horse.} \textcolor{PineGreen}{\selectlanguage{french}Cheval.}  ¶ \textcolor{darkblue}{\textbf{\ipa{dʑɯ˩ʁo˩-ʐwæ˩}}} \zh{野马} \textcolor{Sepia}{\selectlanguage{english}wild horse} \textcolor{PineGreen}{\selectlanguage{french}cheval sauvage, non domestiqué}  
 ¶ \textcolor{darkblue}{\textbf{\ipa{o-ho-ho! ʐwæ˧-ɳɯ˩ | dzɯ˧-po˧-hɯ˥-ze˩!}}} \zh{啊呀嚒!马把饲料都吃光了!} \textcolor{Sepia}{\selectlanguage{english}Oops! The horse scoffed the lot!} \textcolor{PineGreen}{\selectlanguage{french}Houlàlà! Le cheval nous l'a mangé! (Contexte: on laisse dans la cour des céréales, ou du fourrage, et pendant qu'on a le dos tourné, le cheval chaparde cette nourriture.)}  
 \zh{量词}: \textcolor{darkblue}{\textbf{\ipa{v̩˧}}} 
\lhead{\firstmark}
\rhead{\botmark}

\subsection{\hspace{-0.5cm} {\Large \textcolor{darkblue}{\textbf{\ipa{ʐwæ˧\textsubscript{a}}}}}\hspace{0.5cm}[\kern2pt{\textcolor{darkblue}{\textbf{\ipa{ʐwæ˥}}}}\kern2pt]} \hypertarget{z`w\{\string_Ma1}{}
\markboth{\textcolor{darkblue}{\textbf{\ipa{ʐwæ˧\textsubscript{a}}}}}{}
\textcolor{teal}{\zh{动词}} \hspace{4pt} \zh{声调类:} M\textsubscript{a}.
\zh{称。} \textcolor{Sepia}{\selectlanguage{english}To weigh (with scales).} \textcolor{PineGreen}{\selectlanguage{french}Peser (à l'aide d'une balance).}  ¶ \textcolor{darkblue}{\textbf{\ipa{mɤ˧-ʐwæ˧}}} \zh{不称} \textcolor{Sepia}{\selectlanguage{english}\mytextsc{neg}} \textcolor{PineGreen}{\selectlanguage{french}\mytextsc{neg}}  
 ¶ \textcolor{darkblue}{\textbf{\ipa{le˧-ʐwæ˧-ze˧}}} \zh{称了} \textcolor{Sepia}{\selectlanguage{english}\mytextsc{accomp} \string_ \mytextsc{pfv}} \textcolor{PineGreen}{\selectlanguage{french}\mytextsc{accomp} \string_ \mytextsc{pfv}}  
 ¶ \textcolor{darkblue}{\textbf{\ipa{tso˧\textasciitilde{}tso˧ ʐwæ˩}}} \zh{称东西} \textcolor{Sepia}{\selectlanguage{english}to weigh things} \textcolor{PineGreen}{\selectlanguage{french}peser des choses}  
 ¶ \textcolor{darkblue}{\textbf{\ipa{ʁo˧do˧ ʐwæ˧}}} \zh{称核桃} \textcolor{Sepia}{\selectlanguage{english}to weigh walnuts} \textcolor{PineGreen}{\selectlanguage{french}peser des noix}  

\lhead{\firstmark}
\rhead{\botmark}

\subsection{\hspace{-0.5cm} {\Large \textcolor{darkblue}{\textbf{\ipa{ʐwæ˧bv̩˧˥}}}}\hspace{0.5cm}[\kern2pt{\textcolor{darkblue}{\textbf{\ipa{ʐwæ˧bv̩˧˥}}}}\kern2pt]} \hypertarget{z`w\{\string_Mbv\string_=\string_M\string_T1}{}
\markboth{\textcolor{darkblue}{\textbf{\ipa{ʐwæ˧bv̩˧˥}}}}{}
\textcolor{teal}{\zh{名词}} \hspace{4pt} \zh{声调类:} MH\#.
\zh{马圈。} \textcolor{Sepia}{\selectlanguage{english}Horse's stable.} \textcolor{PineGreen}{\selectlanguage{french}Enclos des chevaux.}  \zh{量词}: \textcolor{darkblue}{\textbf{\ipa{ɭɯ˧}}} 
\lhead{\firstmark}
\rhead{\botmark}

\subsection{\hspace{-0.5cm} {\Large \textcolor{darkblue}{\textbf{\ipa{ʐwæ˧-hɑ\#˥}}}}\hspace{0.5cm}[\kern2pt{\textcolor{darkblue}{\textbf{\ipa{xxxx non-correspondance entre le nombre de morphèmes et le nombre de tons de morphèmes}}}}\kern2pt]} \hypertarget{z`w\{\string_M-hA\#\string_T1}{}
\markboth{\textcolor{darkblue}{\textbf{\ipa{ʐwæ˧-hɑ\#˥}}}}{}
\textcolor{teal}{\zh{名词}} \hspace{4pt} \zh{声调类:} \#H.
\zh{马料、马饲料。} \textcolor{Sepia}{\selectlanguage{english}Horse feed.} \textcolor{PineGreen}{\selectlanguage{french}Nourriture pour cheval.} 
\lhead{\firstmark}
\rhead{\botmark}

\subsection{\hspace{-0.5cm} {\Large \textcolor{darkblue}{\textbf{\ipa{ʐwæ˧-kʰv̩˩}}}}\hspace{0.5cm}[\kern2pt{\textcolor{darkblue}{\textbf{\ipa{xxxx non-correspondance entre le nombre de morphèmes et le nombre de tons de morphèmes}}}}\kern2pt]} \hypertarget{z`w\{\string_M-k\string_hv\string_=\string_B1}{}
\markboth{\textcolor{darkblue}{\textbf{\ipa{ʐwæ˧-kʰv̩˩}}}}{}
\textcolor{teal}{\zh{名词}} \hspace{4pt} \zh{声调类:} L\#.
\zh{马年。} \textcolor{Sepia}{\selectlanguage{english}Year of the Horse.} \textcolor{PineGreen}{\selectlanguage{french}Année du cheval.} 
\lhead{\firstmark}
\rhead{\botmark}

\subsection{\hspace{-0.5cm} {\Large \textcolor{darkblue}{\textbf{\ipa{ʐwæ˧-ɭɯ\#˥}}}}\hspace{0.5cm}[\kern2pt{\textcolor{darkblue}{\textbf{\ipa{xxxx non-correspondance entre le nombre de morphèmes et le nombre de tons de morphèmes}}}}\kern2pt]} \hypertarget{z`w\{\string_M-l\string_RM\#\string_T1}{}
\markboth{\textcolor{darkblue}{\textbf{\ipa{ʐwæ˧-ɭɯ\#˥}}}}{}
\textcolor{teal}{\zh{名词}} \hspace{4pt} \zh{声调类:} \#H.
\zh{马料(粮食)。} \textcolor{Sepia}{\selectlanguage{english}Cereals for the horse, horse fodder.} \textcolor{PineGreen}{\selectlanguage{french}Nourriture pour cheval (céréales).} 
\lhead{\firstmark}
\rhead{\botmark}

\subsection{\hspace{-0.5cm} {\Large \textcolor{darkblue}{\textbf{\ipa{ʐwæ˧pʰæ˧di˧˥}}}}\hspace{0.5cm}[\kern2pt{\textcolor{darkblue}{\textbf{\ipa{ʐwæ˧pʰæ˧di˧˥}}}}\kern2pt]} \hypertarget{z`w\{\string_Mp\string_h\{\string_Mdi\string_M\string_T1}{}
\markboth{\textcolor{darkblue}{\textbf{\ipa{ʐwæ˧pʰæ˧di˧˥}}}}{}
\textcolor{teal}{\zh{名词}} \hspace{4pt} \zh{声调类:} MH\#.
\zh{拉马链子。} \textcolor{Sepia}{\selectlanguage{english}Lunge, tether (for a horse).} \textcolor{PineGreen}{\selectlanguage{french}Longe, objet pour accrocher le cheval.}  \zh{量词}: \textcolor{darkblue}{\textbf{\ipa{ɭɯ˧}}} 
\lhead{\firstmark}
\rhead{\botmark}

\subsection{\hspace{-0.5cm} {\Large \textcolor{darkblue}{\textbf{\ipa{ʐwæ˧-qʰæ\#˥}}}}\hspace{0.5cm}[\kern2pt{\textcolor{darkblue}{\textbf{\ipa{xxxx non-correspondance entre le nombre de morphèmes et le nombre de tons de morphèmes}}}}\kern2pt]} \hypertarget{z`w\{\string_M-q\string_h\{\#\string_T1}{}
\markboth{\textcolor{darkblue}{\textbf{\ipa{ʐwæ˧-qʰæ\#˥}}}}{}
\textcolor{teal}{\zh{名词}} \hspace{4pt} \zh{声调类:} \#H.
\zh{马粪。} \textcolor{Sepia}{\selectlanguage{english}Horse manure.} \textcolor{PineGreen}{\selectlanguage{french}Crottin de cheval.}  \zh{量词}: \textcolor{darkblue}{\textbf{\ipa{ʁwɤ˧}}} 
\lhead{\firstmark}
\rhead{\botmark}

\subsection{\hspace{-0.5cm} {\Large \textcolor{darkblue}{\textbf{\ipa{ʐwæ˧ʁo˩}}}}\hspace{0.5cm}[\kern2pt{\textcolor{darkblue}{\textbf{\ipa{ʐwæ˧ʁo˩}}}}\kern2pt]} \hypertarget{z`w\{\string_MRo\string_B1}{}
\markboth{\textcolor{darkblue}{\textbf{\ipa{ʐwæ˧ʁo˩}}}}{}
\textcolor{teal}{\zh{名词}} \hspace{4pt} \zh{声调类:} L\#.
\zh{骟马。} \textcolor{Sepia}{\selectlanguage{english}Castrated horse, gelding, neutered horse.} \textcolor{PineGreen}{\selectlanguage{french}Cheval castré.}  \zh{量词}: \textcolor{darkblue}{\textbf{\ipa{v̩˧}}} 
\lhead{\firstmark}
\rhead{\botmark}

\subsection{\hspace{-0.5cm} {\Large \textcolor{darkblue}{\textbf{\ipa{ʐwæ˧sɯ˩}}}}\hspace{0.5cm}[\kern2pt{\textcolor{darkblue}{\textbf{\ipa{ʐwæ˧sɯ˩}}}}\kern2pt]} \hypertarget{z`w\{\string_MsM\string_B1}{}
\markboth{\textcolor{darkblue}{\textbf{\ipa{ʐwæ˧sɯ˩}}}}{}
\textcolor{teal}{\zh{名词}} \hspace{4pt} \zh{声调类:} L\#.
\zh{公马。} \textcolor{Sepia}{\selectlanguage{english}Stallion.} \textcolor{PineGreen}{\selectlanguage{french}Étalon.}  ¶ \textcolor{darkblue}{\textbf{\ipa{ʐwæ˧sɯ˩-ʐwæ˩mi˩}}} \zh{公马与母马} \textcolor{Sepia}{\selectlanguage{english}stallion and mare} \textcolor{PineGreen}{\selectlanguage{french}étalon et jument}  
 ¶ \textcolor{darkblue}{\textbf{\ipa{ʐwæ˧sɯ˩-ʐwæ˩zo˩}}} \zh{公马与小马} \textcolor{Sepia}{\selectlanguage{english}stallion and colt} \textcolor{PineGreen}{\selectlanguage{french}étalon et poulain}  
 \zh{量词}: \textcolor{darkblue}{\textbf{\ipa{mi˩}}} 
\lhead{\firstmark}
\rhead{\botmark}

\subsection{\hspace{-0.5cm} {\Large \textcolor{darkblue}{\textbf{\ipa{ʐwæ˧zo\#˥}}}}\hspace{0.5cm}[\kern2pt{\textcolor{darkblue}{\textbf{\ipa{ʐwæ˧zo˧}}}}\kern2pt]} \hypertarget{z`w\{\string_Mzo\#\string_T1}{}
\markboth{\textcolor{darkblue}{\textbf{\ipa{ʐwæ˧zo\#˥}}}}{}
\textcolor{teal}{\zh{名词}} \hspace{4pt} \zh{声调类:} \#H.
\zh{马驹子。} \textcolor{Sepia}{\selectlanguage{english}Colt, pony, filly, foal.} \textcolor{PineGreen}{\selectlanguage{french}Poulain.}  ¶ \textcolor{darkblue}{\textbf{\ipa{ʐwæ˧zo˧-ʐwæ˥mi˩}}} \zh{马驹子与母马} \textcolor{Sepia}{\selectlanguage{english}colt and mare} \textcolor{PineGreen}{\selectlanguage{french}poulain et jument}  
 \zh{量词}: \textcolor{darkblue}{\textbf{\ipa{ɭɯ˧}}} 
\lhead{\firstmark}
\rhead{\botmark}

\subsection{\hspace{-0.5cm} {\Large \textcolor{darkblue}{\textbf{\ipa{ʐwæ˧-zɯ\#˥}}}}\hspace{0.5cm}[\kern2pt{\textcolor{darkblue}{\textbf{\ipa{xxxx non-correspondance entre le nombre de morphèmes et le nombre de tons de morphèmes}}}}\kern2pt]} \hypertarget{z`w\{\string_M-zM\#\string_T1}{}
\markboth{\textcolor{darkblue}{\textbf{\ipa{ʐwæ˧-zɯ\#˥}}}}{}
\textcolor{teal}{\zh{名词}} \hspace{4pt} \zh{声调类:} \#H.
\zh{喂马的草。} \textcolor{Sepia}{\selectlanguage{english}Hay for horses, horse hay.} \textcolor{PineGreen}{\selectlanguage{french}Fourrage pour le cheval, foin pour cheval.} 
\lhead{\firstmark}
\rhead{\botmark}

\subsection{\hspace{-0.5cm} {\Large \textcolor{darkblue}{\textbf{\ipa{ʐwæ˧\textasciitilde{}ʐwæ˧}}}}\hspace{0.5cm}[\kern2pt{\textcolor{darkblue}{\textbf{\ipa{ʐwæ˧ʐwæ˧}}}}\kern2pt]} \hypertarget{z`w\{\string_M~z`w\{\string_M1}{}
\markboth{\textcolor{darkblue}{\textbf{\ipa{ʐwæ˧\textasciitilde{}ʐwæ˧}}}}{}
\textcolor{teal}{\zh{动词}} \hspace{4pt} \zh{声调类:} M.
\ding{202} \zh{收拾。} \textcolor{Sepia}{\selectlanguage{english}To put (things) in order.} \textcolor{PineGreen}{\selectlanguage{french}Ranger (des objets).}  ¶ \textcolor{darkblue}{\textbf{\ipa{tso˧\textasciitilde{}tso˧ ʐwæ˧\textasciitilde{}ʐwæ˧(-ze˩)}}} \zh{收拾东西} \textcolor{Sepia}{\selectlanguage{english}to put things in order} \textcolor{PineGreen}{\selectlanguage{french}ranger des choses}  
 ¶ \textcolor{darkblue}{\textbf{\ipa{le˧-ʐwæ˧\textasciitilde{}ʐwæ˧ ɖɯ˧-ʝi˧-tɕɯ˥}}} \zh{把东西收拾在一起} \textcolor{Sepia}{\selectlanguage{english}to put things in order in one place, to arrange things together in one place} \textcolor{PineGreen}{\selectlanguage{french}ranger des choses et les mettre à leur place, ranger des choses ensemble}  
\ding{203} \zh{聚集。} \textcolor{Sepia}{\selectlanguage{english}To gather (people).} \textcolor{PineGreen}{\selectlanguage{french}Rassembler (des gens).} 
\lhead{\firstmark}
\rhead{\botmark}

\subsection{\hspace{-0.5cm} {\Large \textcolor{darkblue}{\textbf{\ipa{ʐwæ˩}}}}\hspace{0.5cm}[\kern2pt{\textcolor{darkblue}{\textbf{\ipa{ʐwæ˩˥}}}}\kern2pt]} \hypertarget{z`w\{\string_B1}{}
\markboth{\textcolor{darkblue}{\textbf{\ipa{ʐwæ˩}}}}{}
\textcolor{teal}{\zh{助词}} \hspace{4pt} \zh{声调类:} L.
\zh{很、极。} \textcolor{Sepia}{\selectlanguage{english}Extremely.} \textcolor{PineGreen}{\selectlanguage{french}Extrêmement.}  ¶ \textcolor{darkblue}{\textbf{\ipa{ʐwæ˩-ze˥!}}} \zh{太多了!} \textcolor{Sepia}{\selectlanguage{english}That's too much! / There's too much!} \textcolor{PineGreen}{\selectlanguage{french}Il y en a trop! / ça fait trop!}  

\lhead{\firstmark}
\rhead{\botmark}

\subsection{\hspace{-0.5cm} {\Large \textcolor{darkblue}{\textbf{\ipa{ʐwæ˩\textsubscript{a}}}} \textsubscript{1}}\hspace{0.5cm}[\kern2pt{\textcolor{darkblue}{\textbf{\ipa{ʐwæ˩˥}}}}\kern2pt]} \hypertarget{z`w\{\string_Ba1}{}
\markboth{\textcolor{darkblue}{\textbf{\ipa{ʐwæ˩\textsubscript{a}}}} \textsubscript{1}}{}
\textcolor{teal}{\zh{动词}} \hspace{4pt} \zh{声调类:} L\textsubscript{a}.
\zh{昏,昏厥。} \textcolor{Sepia}{\selectlanguage{english}To swoon.} \textcolor{PineGreen}{\selectlanguage{french}S'évanouir.}  ¶ \textcolor{darkblue}{\textbf{\ipa{le˧-ʈʰi˩ | le˧-ʐwæ˩-ze˩}}} \zh{累得都昏倒了} \textcolor{Sepia}{\selectlanguage{english}to be so tired as to fall into a swoon, to swoon from exhaustion} \textcolor{PineGreen}{\selectlanguage{french}s'évanouir à force de fatigue, s'évanouir d'épuisement}  

\lhead{\firstmark}
\rhead{\botmark}

\subsection{\hspace{-0.5cm} {\Large \textcolor{darkblue}{\textbf{\ipa{ʐwæ˩\textsubscript{a}}}} \textsubscript{2}}\hspace{0.5cm}[\kern2pt{\textcolor{darkblue}{\textbf{\ipa{ʐwæ˩˥}}}}\kern2pt]} \hypertarget{z`w\{\string_Ba2}{}
\markboth{\textcolor{darkblue}{\textbf{\ipa{ʐwæ˩\textsubscript{a}}}} \textsubscript{2}}{}
\textcolor{teal}{\zh{形容词}} \hspace{4pt} \zh{声调类:} L\textsubscript{a}.
\zh{好,能干。} \textcolor{Sepia}{\selectlanguage{english}Good, well (working well, strongly).} \textcolor{PineGreen}{\selectlanguage{french}Habile, bon, capable.}  ¶ \textcolor{darkblue}{\textbf{\ipa{ʐwæ˩-hĩ˩˥}}} \zh{能干的} \textcolor{Sepia}{\selectlanguage{english}\mytextsc{nmlz}} \textcolor{PineGreen}{\selectlanguage{french}\mytextsc{nmlz}}  
 ¶ \textcolor{darkblue}{\textbf{\ipa{ʈʂʰɯ˧ ɖwæ˧˥ | ʐwæ˩˥!}}} \zh{他很能干!} \textcolor{Sepia}{\selectlanguage{english}He is very capable!} \textcolor{PineGreen}{\selectlanguage{french}Il est très habile! / Il est formidable!}  

\lhead{\firstmark}
\rhead{\botmark}

\subsection{\hspace{-0.5cm} {\Large \textcolor{darkblue}{\textbf{\ipa{ʐwæ˩mi˩}}}}\hspace{0.5cm}[\kern2pt{\textcolor{darkblue}{\textbf{\ipa{ʐwæ˩mi˩˥}}}}\kern2pt]} \hypertarget{z`w\{\string_Bmi\string_B1}{}
\markboth{\textcolor{darkblue}{\textbf{\ipa{ʐwæ˩mi˩}}}}{}
\textcolor{teal}{\zh{名词}} \hspace{4pt} \zh{声调类:} L.
\zh{母马。} \textcolor{Sepia}{\selectlanguage{english}Mare.} \textcolor{PineGreen}{\selectlanguage{french}Jument; également employé pour une jeune jument: pouliche, “bébé cheval” de sexe féminin.}  ¶ \textcolor{darkblue}{\textbf{\ipa{ʂe˩-ʐwæ˩mi˥}}} \zh{自行车(“铁马”)} \textcolor{Sepia}{\selectlanguage{english}bicycle} \textcolor{PineGreen}{\selectlanguage{french}vélo; néologisme introduit par F4 d'après le taïwanais tiěmǎ \zh{铁马} que j'essayais de traduire en na.}  
 ¶ \textcolor{darkblue}{\textbf{\ipa{ʐwæ˩mi˩-ʐwæ˩zo˩}}} \zh{母马与马驹子} \textcolor{Sepia}{\selectlanguage{english}mare and colt} \textcolor{PineGreen}{\selectlanguage{french}jument et poulain}  
 \zh{量词}: \textcolor{darkblue}{\textbf{\ipa{v̩˧}}} \textcolor{darkblue}{\textbf{\ipa{jɤ˧˥}}} 
\lhead{\firstmark}
\rhead{\botmark}

\subsection{\hspace{-0.5cm} {\Large \textcolor{darkblue}{\textbf{\ipa{ʐwæ˧˥}}}}\hspace{0.5cm}[\kern2pt{\textcolor{darkblue}{\textbf{\ipa{ʐwæ˧˥}}}}\kern2pt]} \hypertarget{z`w\{\string_M\string_T1}{}
\markboth{\textcolor{darkblue}{\textbf{\ipa{ʐwæ˧˥}}}}{}
\textcolor{teal}{\zh{动词}} \hspace{4pt} \zh{声调类:} MH.
\zh{薅锄、锄草。} \textcolor{Sepia}{\selectlanguage{english}To hoe weeds.} \textcolor{PineGreen}{\selectlanguage{french}Sarcler, biner.}  ¶ \textcolor{darkblue}{\textbf{\ipa{ʐwæ˩\textasciitilde{}ʐwæ˧˥}}} \zh{\mytextsc{重叠}} \textcolor{Sepia}{\selectlanguage{english}\mytextsc{red}} \textcolor{PineGreen}{\selectlanguage{french}\mytextsc{red}}  
 ¶ \textcolor{darkblue}{\textbf{\ipa{jɤ˩jo˥ ʐwæ˩}}} \zh{洋芋地里锄草} \textcolor{Sepia}{\selectlanguage{english}to hoe potatoes, to weed a potato field} \textcolor{PineGreen}{\selectlanguage{french}sarcler des pommes de terre}  
 ¶ \textcolor{darkblue}{\textbf{\ipa{jɤ˩jo˧ ʐwæ˧\textasciitilde{}ʐwæ˥}}} \zh{洋芋地里锄草} \textcolor{Sepia}{\selectlanguage{english}to hoe potatoes, to weed a potato field} \textcolor{PineGreen}{\selectlanguage{french}sarcler des pommes de terre}  
 ¶ \textcolor{darkblue}{\textbf{\ipa{qʰɑ˧dze˧ ʐwæ˧˥}}} \zh{苞谷地里锄草} \textcolor{Sepia}{\selectlanguage{english}to hoe sweetcorn, to weed a sweetcorn field} \textcolor{PineGreen}{\selectlanguage{french}sarcler du maïs}  
 ¶ \textcolor{darkblue}{\textbf{\ipa{qʰɑ˧dze˧ ʐwæ˧\textasciitilde{}ʐwæ˥}}} \zh{苞谷地里锄草} \textcolor{Sepia}{\selectlanguage{english}to hoe sweetcorn, to weed a sweetcorn field} \textcolor{PineGreen}{\selectlanguage{french}sarcler du maïs}  

\lhead{\firstmark}
\rhead{\botmark}

\subsection{\hspace{-0.5cm} {\Large \textcolor{darkblue}{\textbf{\ipa{ʐwɤ˧}}}}\hspace{0.5cm}[\kern2pt{\textcolor{darkblue}{\textbf{\ipa{ʐwɤ˥}}}}\kern2pt]} \hypertarget{z`w7\string_M1}{}
\markboth{\textcolor{darkblue}{\textbf{\ipa{ʐwɤ˧}}}}{}
\textcolor{teal}{\zh{形容词}} \hspace{4pt} \zh{声调类:} M.
\zh{饿。} \textcolor{Sepia}{\selectlanguage{english}Hungry (monosyllable).} \textcolor{PineGreen}{\selectlanguage{french}Qui a faim (forme monosyllabique). Se combine en disyllabe avec le mot “nourriture”.}  ¶ \textcolor{darkblue}{\textbf{\ipa{hɑ˧-ʐwɤ˩}}} \zh{饿} \textcolor{Sepia}{\selectlanguage{english}to be hungry} \textcolor{PineGreen}{\selectlanguage{french}avoir faim}  

\lhead{\firstmark}
\rhead{\botmark}

\subsection{\hspace{-0.5cm} {\Large \textcolor{darkblue}{\textbf{\ipa{ʐwɤ˧mv̩˧}}}}\hspace{0.5cm}[\kern2pt{\textcolor{darkblue}{\textbf{\ipa{ʐwɤ˩mv̩˩˥}}}}\kern2pt]} \hypertarget{z`w7\string_Mmv\string_=\string_M1}{}
\markboth{\textcolor{darkblue}{\textbf{\ipa{ʐwɤ˧mv̩˧}}}}{}
\textcolor{teal}{\zh{名词}} \hspace{4pt} \zh{声调类:} M.
\zh{惯用语、习惯语、习语。} \textcolor{Sepia}{\selectlanguage{english}Idiom, set phrase, fixed expression.} \textcolor{PineGreen}{\selectlanguage{french}Formule toute faite, expression toute faite, expression idiomatique.}  ¶ \textcolor{darkblue}{\textbf{\ipa{ʐwɤ˧mv̩˧ dʑo˧-kv̩˧˥ !}}} \zh{有这么一句老话! / 有这么一个说法!} \textcolor{Sepia}{\selectlanguage{english}This is how they say! / There's such a set phrase!} \textcolor{PineGreen}{\selectlanguage{french}C'est comme ça qu'on dit! / Il y a une expression comme ça!}  
 ¶ \textcolor{darkblue}{\textbf{\ipa{æ˧ʂæ˧-ʐwɤ˧mv̩˧ | ɖɯ˧-kʰwɤ˥}}} \zh{一句老话、一个传统的说法} \textcolor{Sepia}{\selectlanguage{english}a saying, a set phrase of the old times} \textcolor{PineGreen}{\selectlanguage{french}une expression toute faite du temps jadis, un dicton}  
 \zh{量词}: \textcolor{darkblue}{\textbf{\ipa{kʰwɤ˥}}} 
\lhead{\firstmark}
\rhead{\botmark}

\subsection{\hspace{-0.5cm} {\Large \textcolor{darkblue}{\textbf{\ipa{ʐwɤ˩\textsubscript{b}}}}}\hspace{0.5cm}[\kern2pt{\textcolor{darkblue}{\textbf{\ipa{ʐwɤ˥}}}}\kern2pt]} \hypertarget{z`w7\string_Bb1}{}
\markboth{\textcolor{darkblue}{\textbf{\ipa{ʐwɤ˩\textsubscript{b}}}}}{}
\textcolor{teal}{\zh{动词}} \hspace{4pt} \zh{声调类:} L\textsubscript{b}.
\zh{讲话。} \textcolor{Sepia}{\selectlanguage{english}To speak.} \textcolor{PineGreen}{\selectlanguage{french}Parler.}  ¶ \textcolor{darkblue}{\textbf{\ipa{ʐwɤ˧\textasciitilde{}ʐwɤ˩ mɤ˩-hĩ˩}}} \zh{哑巴、不会讲话的人} \textcolor{Sepia}{\selectlanguage{english}dumb person, person who is not able to speak} \textcolor{PineGreen}{\selectlanguage{french}muet, personne muette, personne qui ne parle pas}  
 ¶ \textcolor{darkblue}{\textbf{\ipa{ʐwɤ˧\textasciitilde{}ʐwɤ˩ mɤ˩-hĩ˩, | ʈʂʰɯ˧-v̩˧!}}} \zh{不会讲话,这个人! / 这个人,不会讲话!} \textcolor{Sepia}{\selectlanguage{english}(S)he is not able to speak! / (S)he won't speak!} \textcolor{PineGreen}{\selectlanguage{french}Elle/il ne sait pas parler, elle/lui!}  
 ¶ \textcolor{darkblue}{\textbf{\ipa{le˧-ʐwɤ˩-ze˩}}} \zh{讲了} \textcolor{Sepia}{\selectlanguage{english}\mytextsc{accomp} \string_ \mytextsc{pfv}} \textcolor{PineGreen}{\selectlanguage{french}\mytextsc{accomp} \string_ \mytextsc{pfv}}  
 ¶ \textcolor{darkblue}{\textbf{\ipa{no˧ | ə˧tso˧ ʐwɤ˩-ɲi˩?}}} \zh{你说什么?} \textcolor{Sepia}{\selectlanguage{english}What are you saying? / What do you mean?} \textcolor{PineGreen}{\selectlanguage{french}Que dis-tu? / Qu'est-ce que tu veux dire?}  
 ¶ \textcolor{darkblue}{\textbf{\ipa{ʐwɤ˧\textasciitilde{}ʐwɤ˩}}} \zh{\mytextsc{重叠}} \textcolor{Sepia}{\selectlanguage{english}\mytextsc{red}} \textcolor{PineGreen}{\selectlanguage{french}\mytextsc{red}} \textcolor{PineGreen}{\selectlanguage{french}PHONO}  
 ¶ \textcolor{darkblue}{\textbf{\ipa{le˧-ʐwɤ˩}}} \zh{回答} \textcolor{Sepia}{\selectlanguage{english}to answer} \textcolor{PineGreen}{\selectlanguage{french}répondre, donner une réponse}  
 ¶ \textcolor{darkblue}{\textbf{\ipa{le˧-wo˧ ʐwɤ˧˥}}} \zh{回答} \textcolor{Sepia}{\selectlanguage{english}to answer} \textcolor{PineGreen}{\selectlanguage{french}répondre, donner une réponse}  
 ¶ \textcolor{darkblue}{\textbf{\ipa{ʈʂʰɯ˧ | le˧-ʐwɤ˩-bi˩-dʑo˩...}}} \zh{依照他的说法……} \textcolor{Sepia}{\selectlanguage{english}According to her/him... / From her/his point of view...} \textcolor{PineGreen}{\selectlanguage{french}A ce qu'elle/il dit...}  
 ¶ \textcolor{darkblue}{\textbf{\ipa{hĩ˧-qɑ˧ ʐwɤ˧\textasciitilde{}ʐwɤ˥}}} \zh{对人家讲} \textcolor{Sepia}{\selectlanguage{english}to speak to people} \textcolor{PineGreen}{\selectlanguage{french}parler aux gens}  
 ¶ \textcolor{darkblue}{\textbf{\ipa{le˧-ʐwɤ˧\textasciitilde{}ʐwɤ˥-ze˩}}} \zh{\mytextsc{accomp} \mytextsc{red} \mytextsc{pfv}} \textcolor{Sepia}{\selectlanguage{english}\mytextsc{accomp} \mytextsc{red} \mytextsc{pfv}} \textcolor{PineGreen}{\selectlanguage{french}\mytextsc{accomp} \mytextsc{red} \mytextsc{pfv}}  

\lhead{\firstmark}
\rhead{\botmark}

\newpage
\section*{\centering- \textcolor{darkblue}{\textbf{\ipa{ʑ}}} -}
\subsection{\hspace{-0.5cm} {\Large \textcolor{darkblue}{\textbf{\ipa{ʑi˥}}}}\hspace{0.5cm}[\kern2pt{\textcolor{darkblue}{\textbf{\ipa{ʑi˥}}}}\kern2pt]} \hypertarget{z£i\string_T1}{}
\markboth{\textcolor{darkblue}{\textbf{\ipa{ʑi˥}}}}{}
\textcolor{teal}{\zh{动词}} \hspace{4pt} \zh{声调类:} H.
\zh{有,拥有(抽象:有力量,有勇气)。} \textcolor{Sepia}{\selectlanguage{english}To be present: abstract entity (courage, strength) or concrete entity (beard).} \textcolor{PineGreen}{\selectlanguage{french}Être présent, y avoir; propriété du corps, de l'âme, d'un objet… Ex.: avoir de la force; avoir de la barbe; il y a une resserre dans la maison.}  ¶ \textcolor{darkblue}{\textbf{\ipa{mɤ˧-ʑi˥}}} \zh{没有} \textcolor{Sepia}{\selectlanguage{english}\mytextsc{neg}} \textcolor{PineGreen}{\selectlanguage{french}\mytextsc{neg}}  

\lhead{\firstmark}
\rhead{\botmark}

\subsection{\hspace{-0.5cm} {\Large \textcolor{darkblue}{\textbf{\ipa{ʑi˧\textsubscript{a}}}}}\hspace{0.5cm}[\kern2pt{\textcolor{darkblue}{\textbf{\ipa{ʑi˥}}}}\kern2pt]} \hypertarget{z£i\string_Ma1}{}
\markboth{\textcolor{darkblue}{\textbf{\ipa{ʑi˧\textsubscript{a}}}}}{}
\textcolor{teal}{\zh{动词}} \hspace{4pt} \zh{声调类:} M\textsubscript{a}.
\ding{202} \zh{漏(水)。} \textcolor{Sepia}{\selectlanguage{english}To leak.} \textcolor{PineGreen}{\selectlanguage{french}Couler, avoir une fuite; s'écouler (fleuve).}  ¶ \textcolor{darkblue}{\textbf{\ipa{mv̩˩tɕo˧ ʑi˧}}} \zh{(水)往下漏} \textcolor{Sepia}{\selectlanguage{english}to leak, to drip down} \textcolor{PineGreen}{\selectlanguage{french}fuir, avoir une fuite}  
\ding{203} \zh{流(河水流着)。} \textcolor{Sepia}{\selectlanguage{english}To flow (a river flows).} \textcolor{PineGreen}{\selectlanguage{french}S'écouler, couler (rivière).}  ¶ \textcolor{darkblue}{\textbf{\ipa{mv̩˩tɕo˧ ʑi˧}}} \zh{(河)往下游流} \textcolor{Sepia}{\selectlanguage{english}to flow down (the water of a brook flows down)} \textcolor{PineGreen}{\selectlanguage{french}couler (rivière)}  

\lhead{\firstmark}
\rhead{\botmark}

\subsection{\hspace{-0.5cm} {\Large \textcolor{darkblue}{\textbf{\ipa{ʑi˧dv̩˧}}}}\hspace{0.5cm}[\kern2pt{\textcolor{darkblue}{\textbf{\ipa{ʑi˩dv̩˩˥}}}}\kern2pt]} \hypertarget{z£i\string_Mdv\string_=\string_M1}{}
\markboth{\textcolor{darkblue}{\textbf{\ipa{ʑi˧dv̩˧}}}}{}
\textcolor{teal}{\zh{名词}} \hspace{4pt} \zh{声调类:} M.
\ding{202} \zh{家。} \textcolor{Sepia}{\selectlanguage{english}The household.} \textcolor{PineGreen}{\selectlanguage{french}Maisonnée.}  ¶ \textcolor{darkblue}{\textbf{\ipa{ɖɯ˧-ʑi˩dv̩˩}}} \zh{一家人,包括所有成员} \textcolor{Sepia}{\selectlanguage{english}a household} \textcolor{PineGreen}{\selectlanguage{french}une maisonnée/ toute la maison}  
 ¶ \textcolor{darkblue}{\textbf{\ipa{ɑ˩ʁo˧-ʑi˧dv̩˧ ʝi˧}}} \zh{管家} \textcolor{Sepia}{\selectlanguage{english}to take care of the house, to look after the household, to be in charge of the household} \textcolor{PineGreen}{\selectlanguage{french}s'occuper de la maison, veiller au bon fonctionnement de la maison}  
 \zh{量词}: \textcolor{darkblue}{\textbf{\ipa{ɭɯ˧}}} \ding{203} \zh{农舍,包括院子、人住的楼、动物住的楼等。} \textcolor{Sepia}{\selectlanguage{english}The entire farmhouse, comprising the main house and the other buildings.} \textcolor{PineGreen}{\selectlanguage{french}L'ensemble de la ferme, comprenant plusieurs bâtiments, le bétail et les gens.}  \zh{量词}: \textcolor{darkblue}{\textbf{\ipa{ɭɯ˧}}} 
\lhead{\firstmark}
\rhead{\botmark}

\subsection{\hspace{-0.5cm} {\Large \textcolor{darkblue}{\textbf{\ipa{ʑi˧dv̩˧ʝi˧-hĩ\#˥}}}}\hspace{0.5cm}[\kern2pt{\textcolor{darkblue}{\textbf{\ipa{xxxx non-correspondance entre le nombre de morphèmes et le nombre de tons de morphèmes}}}}\kern2pt]} \hypertarget{z£i\string_Mdv\string_=\string_Mj££i\string_M-hi\string_~\#\string_T1}{}
\markboth{\textcolor{darkblue}{\textbf{\ipa{ʑi˧dv̩˧ʝi˧-hĩ\#˥}}}}{}
\textcolor{teal}{\zh{名词}} \hspace{4pt} \zh{声调类:} H\#.
\zh{一家之主、家长。} \textcolor{Sepia}{\selectlanguage{english}The person in charge of the house, the master/mistress of the house.} \textcolor{PineGreen}{\selectlanguage{french}La personne qui s'occupe de la maison, le maître/la maîtresse de céans.} 
\lhead{\firstmark}
\rhead{\botmark}

\subsection{\hspace{-0.5cm} {\Large \textcolor{darkblue}{\textbf{\ipa{ʑi˧kv̩˧wo˧}}}}\hspace{0.5cm}[\kern2pt{\textcolor{darkblue}{\textbf{\ipa{ʑi˩kv̩˧wo˧˥}}}}\kern2pt]} \hypertarget{z£i\string_Mkv\string_=\string_Mwo\string_M1}{}
\markboth{\textcolor{darkblue}{\textbf{\ipa{ʑi˧kv̩˧wo˧}}}}{}
\textcolor{teal}{\zh{名词}} \hspace{4pt} \zh{声调类:} M.
\zh{房顶。} \textcolor{Sepia}{\selectlanguage{english}Roof.} \textcolor{PineGreen}{\selectlanguage{french}Toit.}  \zh{量词}: \textcolor{darkblue}{\textbf{\ipa{tsʰi˩}}} 
\lhead{\firstmark}
\rhead{\botmark}

\subsection{\hspace{-0.5cm} {\Large \textcolor{darkblue}{\textbf{\ipa{ʑi˧mi˧}}}}\hspace{0.5cm}[\kern2pt{\textcolor{darkblue}{\textbf{\ipa{ʑi˧mi˧}}}}\kern2pt]} \hypertarget{z£i\string_Mmi\string_M1}{}
\markboth{\textcolor{darkblue}{\textbf{\ipa{ʑi˧mi˧}}}}{}
\textcolor{teal}{\zh{名词}} \hspace{4pt} \zh{声调类:} M.
\ding{202} \zh{家里有火塘的那个房子(“祖母房”)。} \textcolor{Sepia}{\selectlanguage{english}The main building of the house/farm: the building where the hearth is located.} \textcolor{PineGreen}{\selectlanguage{french}Le bâtiment de la maison où se trouve le foyer.}  \zh{量词}: \textcolor{darkblue}{\textbf{\ipa{ɭɯ˧}}} \ding{203} \zh{整个家园。} \textcolor{Sepia}{\selectlanguage{english}The entire house, the entire farm.} \textcolor{PineGreen}{\selectlanguage{french}L'ensemble de la maison; l'ensemble de la ferme.} 
\lhead{\firstmark}
\rhead{\botmark}

\subsection{\hspace{-0.5cm} {\Large \textcolor{darkblue}{\textbf{\ipa{ʑi˧mv̩˧˥}}}}\hspace{0.5cm}[\kern2pt{\textcolor{darkblue}{\textbf{\ipa{ʑi˩mv̩˥}}}}\kern2pt]} \hypertarget{z£i\string_Mmv\string_=\string_M\string_T1}{}
\markboth{\textcolor{darkblue}{\textbf{\ipa{ʑi˧mv̩˧˥}}}}{}
\textcolor{teal}{\zh{名词}} \hspace{4pt} \zh{声调类:} MH\#.
\zh{梦。} \textcolor{Sepia}{\selectlanguage{english}Dream.} \textcolor{PineGreen}{\selectlanguage{french}Rêve.}  ¶ \textcolor{darkblue}{\textbf{\ipa{ʑi˧mv̩˧ qʰwɤ˧˥}}} \zh{做梦} \textcolor{Sepia}{\selectlanguage{english}to have a dream} \textcolor{PineGreen}{\selectlanguage{french}faire un rêve}  
 ¶ \textcolor{darkblue}{\textbf{\ipa{ʑi˧mv̩˧ sɯ˧}}} \zh{梦游,梦呓} \textcolor{Sepia}{\selectlanguage{english}to sleep-walk (somnambulism); also: to speak in one's dreams} \textcolor{PineGreen}{\selectlanguage{french}être somnambule; parler dans son sommeil}  
 ¶ \textcolor{darkblue}{\textbf{\ipa{njɤ˧ | ə˧hwɤ˧ | ʑi˧mv̩˥ | mɤ˧-dʑɤ˩!}}} \zh{我昨天做了恶梦!} \textcolor{Sepia}{\selectlanguage{english}I have not had good dreams yesterday night! / I had a nightmare yesterday night!} \textcolor{PineGreen}{\selectlanguage{french}J'ai fait un cauchemar hier!}  
 \zh{量词}: \textcolor{darkblue}{\textbf{\ipa{kʰwɤ˥}}} 
\lhead{\firstmark}
\rhead{\botmark}

\subsection{\hspace{-0.5cm} {\Large \textcolor{darkblue}{\textbf{\ipa{ʑi˧ŋɤ˥}}}}\hspace{0.5cm}[\kern2pt{\textcolor{darkblue}{\textbf{\ipa{ʑi˧ŋɤ˧˥}}}}\kern2pt]} \hypertarget{z£i\string_MN7\string_T1}{}
\markboth{\textcolor{darkblue}{\textbf{\ipa{ʑi˧ŋɤ˥}}}}{}
\textcolor{teal}{\zh{动词}} \hspace{4pt} \zh{声调类:} H\#.
\zh{打瞌睡。} \textcolor{Sepia}{\selectlanguage{english}To doze off, to nod.} \textcolor{PineGreen}{\selectlanguage{french}Somnoler.}  ¶ \textcolor{darkblue}{\textbf{\ipa{ʑi˧ŋɤ˥-ze˩}}} \zh{\mytextsc{pfv}} \textcolor{Sepia}{\selectlanguage{english}\mytextsc{pfv}} \textcolor{PineGreen}{\selectlanguage{french}\mytextsc{pfv}}  

\lhead{\firstmark}
\rhead{\botmark}

\subsection{\hspace{-0.5cm} {\Large \textcolor{darkblue}{\textbf{\ipa{ʑi˧ŋv̩˥}}}}\hspace{0.5cm}[\kern2pt{\textcolor{darkblue}{\textbf{\ipa{ʑi˧ŋv̩˥}}}}\kern2pt]} \hypertarget{z£i\string_MNv\string_=\string_T1}{}
\markboth{\textcolor{darkblue}{\textbf{\ipa{ʑi˧ŋv̩˥}}}}{}
\textcolor{teal}{\zh{动词}} \hspace{4pt} \zh{声调类:} H\#.
\zh{睡觉。} \textcolor{Sepia}{\selectlanguage{english}To sleep.} \textcolor{PineGreen}{\selectlanguage{french}Dormir.}  ¶ \textcolor{darkblue}{\textbf{\ipa{le˧-ʑi˧ŋv̩˥}}} \zh{\mytextsc{accomp}} \textcolor{Sepia}{\selectlanguage{english}\mytextsc{accomp}} \textcolor{PineGreen}{\selectlanguage{french}\mytextsc{accomp}}  
 ¶ \textcolor{darkblue}{\textbf{\ipa{ʑi˧ŋv̩˥-ho˩}}} \zh{要睡了} \textcolor{Sepia}{\selectlanguage{english}is going to sleep} \textcolor{PineGreen}{\selectlanguage{french}qui va s'endormir, qui est ensommeillé}  

\lhead{\firstmark}
\rhead{\botmark}

\subsection{\hspace{-0.5cm} {\Large \textcolor{darkblue}{\textbf{\ipa{ʑi˧qʰwɤ˧}}}}\hspace{0.5cm}[\kern2pt{\textcolor{darkblue}{\textbf{\ipa{ʑi˩qʰwɤ˥}}}}\kern2pt]} \hypertarget{z£i\string_Mq\string_hw7\string_M1}{}
\markboth{\textcolor{darkblue}{\textbf{\ipa{ʑi˧qʰwɤ˧}}}}{}
\textcolor{teal}{\zh{名词}} \hspace{4pt} \zh{声调类:} M.
\zh{房屋。} \textcolor{Sepia}{\selectlanguage{english}Building; houses.} \textcolor{PineGreen}{\selectlanguage{french}Bâtiment, bâtiment d'habitation, pièce d'habitation; en naxi, a aussi le sens de “maisonnée”.}  ¶ \textcolor{darkblue}{\textbf{\ipa{ʑi˧qʰwɤ˧ gv̩˩}}} \zh{建房} \textcolor{Sepia}{\selectlanguage{english}to build a building} \textcolor{PineGreen}{\selectlanguage{french}bâtir une maison}  
 ¶ \textcolor{darkblue}{\textbf{\ipa{ʑi˧qʰwɤ˧-lɑ˧ do˥!}}} \zh{只看到房子! / 能看见的只有房子!(合作者说,丽江市区都是房子,看不到田。这一点,不像永宁坝:二十世纪的永宁,只有一些小村落分散在一大片田地中。)} \textcolor{Sepia}{\selectlanguage{english}One can only see buildings! (A comment by the consultant about the city of Lijiang: the plain is now thoroughly covered by buildings, and one cannot see fields anymore, unlike in Yongning, where until recently there were only a few hamlets scattered among a landscape of groves, pastures, and cultivated fields.)} \textcolor{PineGreen}{\selectlanguage{french}on ne voit que des maisons/des bâtiments! (commentaires au sujet de la ville de Lijiang, où on ne voit pas les champs, à la différence de la plaine de Yongning: campagne où il y avait peu de maisons et de grands espaces cultivés.}  
 ¶ \textcolor{darkblue}{\textbf{\ipa{ɕjo˩ɕjɤ˩-ʑi˩qʰwɤ˥}}} \zh{学校的楼(‘学校’:汉语借词)} \textcolor{Sepia}{\selectlanguage{english}the building(s) of the school, the school buildings} \textcolor{PineGreen}{\selectlanguage{french}les bâtiments de l'école (du chinois \zh{学校} “école”)}  
 ¶ \textcolor{darkblue}{\textbf{\ipa{ʑi˧qʰwɤ˧ ʈʂʰv̩˩}}} \zh{给房子刷颜色(直译:‘染房’)} \textcolor{Sepia}{\selectlanguage{english}to paint a house} \textcolor{PineGreen}{\selectlanguage{french}peindre une maison; littéralement: “teindre une maison”}  
 ¶ \textcolor{darkblue}{\textbf{\ipa{ʑi˧qʰwɤ˧ tɕʰi˧-hĩ˧ kʰv̩˥mi˩}}} \zh{看门狗} \textcolor{Sepia}{\selectlanguage{english}watchdog} \textcolor{PineGreen}{\selectlanguage{french}chien de garde, chien qui garde la maison}  
 ¶ \textcolor{darkblue}{\textbf{\ipa{ʑi˧qʰwɤ˧ tɕʰi˧-hĩ˧ kʰv̩˥}}} \zh{看门狗} \textcolor{Sepia}{\selectlanguage{english}watchdog} \textcolor{PineGreen}{\selectlanguage{french}chien de garde, chien qui garde la maison}  
 \zh{量词}: \textcolor{darkblue}{\textbf{\ipa{ɭɯ˧}}} 
\lhead{\firstmark}
\rhead{\botmark}

\subsection{\hspace{-0.5cm} {\Large \textcolor{darkblue}{\textbf{\ipa{ʑi˧ʁæ˥\$}}}}\hspace{0.5cm}[\kern2pt{\textcolor{darkblue}{\textbf{\ipa{ʑi˧ʁæ˧}}}}\kern2pt]} \hypertarget{z£i\string_MR\{\string_T\$1}{}
\markboth{\textcolor{darkblue}{\textbf{\ipa{ʑi˧ʁæ˥\$}}}}{}
\textcolor{teal}{\zh{名词}} \hspace{4pt} \zh{声调类:} H\$.
\zh{房屋的上后方。} \textcolor{Sepia}{\selectlanguage{english}The back of the house, the space behind the house.} \textcolor{PineGreen}{\selectlanguage{french}L'espace situé derrière la maison: entre le bâtiment et les murs de la ferme.}  ¶ \textcolor{darkblue}{\textbf{\ipa{ɲi˧ʈʂæ˧-ʑi˧-ʁo˧tʰo˥, | ʑi˧ʁæ˧ ɲi˥ mæ˩!}}} \zh{两层楼房后面(这块地方)叫做“房屋的上后方”! / 房屋背后(这块地方)叫做“房屋的上后方”!} \textcolor{Sepia}{\selectlanguage{english}The place behind the two-storey building is called 'ʑi˧ʁæ˥\$'!} \textcolor{PineGreen}{\selectlanguage{french}Derrière le bâtiment à deux étages, c'est 'ʑi˧ʁæ˥\$'! / Derrière le bâtiment à deux étages, il y a ce qu'on appelle 'l'espace derrière la maison'!}  
 \zh{量词}: \textcolor{darkblue}{\textbf{\ipa{kʰwɤ˥}}} 
\lhead{\firstmark}
\rhead{\botmark}

\subsection{\hspace{-0.5cm} {\Large \textcolor{darkblue}{\textbf{\ipa{ʑi˧ʁo˥\$}}}}\hspace{0.5cm}[\kern2pt{\textcolor{darkblue}{\textbf{\ipa{ʑi˧ʁo˥}}}}\kern2pt]} \hypertarget{z£i\string_MRo\string_T\$1}{}
\markboth{\textcolor{darkblue}{\textbf{\ipa{ʑi˧ʁo˥\$}}}}{}
\textcolor{teal}{\zh{名词}} \hspace{4pt} \zh{声调类:} H\$.
\zh{床。} \textcolor{Sepia}{\selectlanguage{english}Bed.} \textcolor{PineGreen}{\selectlanguage{french}Lit (le couchage entier).}  \zh{量词}: \textcolor{darkblue}{\textbf{\ipa{ɭɯ˧˥}}} 
\lhead{\firstmark}
\rhead{\botmark}

\subsection{\hspace{-0.5cm} {\Large \textcolor{darkblue}{\textbf{\ipa{ʑi˩}}} \textsubscript{2}}\hspace{0.5cm}[\kern2pt{\textcolor{darkblue}{\textbf{\ipa{xxxx ton non trouvé, à faire manuellement...}}}}\kern2pt]} \hypertarget{z£i\string_B2}{}
\markboth{\textcolor{darkblue}{\textbf{\ipa{ʑi˩}}} \textsubscript{2}}{}
\textcolor{teal}{\zh{量词}} \hspace{4pt} \zh{声调类:} L\textsubscript{2}.
\zh{家庭(一户人)。} \textcolor{Sepia}{\selectlanguage{english}Family.} \textcolor{PineGreen}{\selectlanguage{french}Famille.}  ¶ \textcolor{darkblue}{\textbf{\ipa{hĩ˧ | ɖɯ˧-ʑi˩}}} \zh{一家人} \textcolor{Sepia}{\selectlanguage{english}a family} \textcolor{PineGreen}{\selectlanguage{french}une famille}  
 ¶ \textcolor{darkblue}{\textbf{\ipa{ʈʂʰɯ˧-ʑi˥}}} \zh{这家} \textcolor{Sepia}{\selectlanguage{english}this family} \textcolor{PineGreen}{\selectlanguage{french}cette famille-ci}  
 ¶ \textcolor{darkblue}{\textbf{\ipa{ŋwæ˧-qʰv̩˧, | tsʰe˧ɲi˧-ʑi˩}}} \zh{五个村落,十二个家庭!(阿拉瓦村的人口简介)} \textcolor{Sepia}{\selectlanguage{english}Five hamlets, twelve families! (A summary of the statistics of the village of \textcolor{darkblue}{\textbf{\ipa{/ə˧lɑ˧-ʁwɤ\#˥/}}}.)} \textcolor{PineGreen}{\selectlanguage{french}Cinq hameaux, douze familles! (Formule résumant la statistique du village de \textcolor{darkblue}{\textbf{\ipa{/ə˧lɑ˧-ʁwɤ\#˥/}}}.)}  

\lhead{\firstmark}
\rhead{\botmark}

\subsection{\hspace{-0.5cm} {\Large \textcolor{darkblue}{\textbf{\ipa{ʑi˩\textsubscript{b}}}}}\hspace{0.5cm}[\kern2pt{\textcolor{darkblue}{\textbf{\ipa{ʑi˥}}}}\kern2pt]} \hypertarget{z£i\string_Bb1}{}
\markboth{\textcolor{darkblue}{\textbf{\ipa{ʑi˩\textsubscript{b}}}}}{}
\textcolor{teal}{\zh{动词}} \hspace{4pt} \zh{声调类:} L\textsubscript{b}.
\ding{202} \zh{拿,捉 (捉鸡)。} \textcolor{Sepia}{\selectlanguage{english}To clutch, to grasp, to grab.} \textcolor{PineGreen}{\selectlanguage{french}Attraper, saisir, prendre (ex.: un animal récalcitrant).}  ¶ \textcolor{darkblue}{\textbf{\ipa{æ˩ ʑi˧}}} \zh{捉鸡} \textcolor{Sepia}{\selectlanguage{english}to catch a chicken} \textcolor{PineGreen}{\selectlanguage{french}attraper (un/le) poulet}  
 ¶ \textcolor{darkblue}{\textbf{\ipa{æ˩˥ | le˧-ʑi˩}}} \zh{捉鸡} \textcolor{Sepia}{\selectlanguage{english}to catch a chicken} \textcolor{PineGreen}{\selectlanguage{french}attraper (un/le) poulet}  
 ¶ \textcolor{darkblue}{\textbf{\ipa{hĩ˧ ʑi˧˥}}} \zh{抓人} \textcolor{Sepia}{\selectlanguage{english}to grab someone} \textcolor{PineGreen}{\selectlanguage{french}attraper quelqu'un}  
 ¶ \textcolor{darkblue}{\textbf{\ipa{ʁæ˧ ʑi˧}}} \zh{搂(用胳膊搂脖子)} \textcolor{Sepia}{\selectlanguage{english}to hold sb in one's arms (arm over the neck)} \textcolor{PineGreen}{\selectlanguage{french}passer le bras autour du cou de quelqu'un}  
\ding{203} \zh{带、拿过来。} \textcolor{Sepia}{\selectlanguage{english}To take along, to bring along.} \textcolor{PineGreen}{\selectlanguage{french}Amener, prendre avec soi.}  ¶ \textcolor{darkblue}{\textbf{\ipa{tso˧\textasciitilde{}tso˧ ʑi˧˥}}} \zh{带东西过来} \textcolor{Sepia}{\selectlanguage{english}to bring something} \textcolor{PineGreen}{\selectlanguage{french}attraper quelque chose}  

\lhead{\firstmark}
\rhead{\botmark}

\subsection{\hspace{-0.5cm} {\Large \textcolor{darkblue}{\textbf{\ipa{ʑi˩hṽ\#˥}}}}\hspace{0.5cm}[\kern2pt{\textcolor{darkblue}{\textbf{\ipa{ʑi˧hṽ˥}}}}\kern2pt]} \hypertarget{z£i\string_Bhv\string_~\#\string_T1}{}
\markboth{\textcolor{darkblue}{\textbf{\ipa{ʑi˩hṽ\#˥}}}}{}
\textcolor{teal}{\zh{名词}} \hspace{4pt} \zh{声调类:} LM+\#H.
\zh{人身上的毛。} \textcolor{Sepia}{\selectlanguage{english}Body hair (of humans).} \textcolor{PineGreen}{\selectlanguage{french}Poils corporels.}  \zh{量词}: \textcolor{darkblue}{\textbf{\ipa{kʰɯ˩}}} 
\lhead{\firstmark}
\rhead{\botmark}

\subsection{\hspace{-0.5cm} {\Large \textcolor{darkblue}{\textbf{\ipa{ʑi˩-kʰv̩˧˥}}}}\hspace{0.5cm}[\kern2pt{\textcolor{darkblue}{\textbf{\ipa{xxxx non-correspondance entre le nombre de morphèmes et le nombre de tons de morphèmes}}}}\kern2pt]} \hypertarget{z£i\string_B-k\string_hv\string_=\string_M\string_T1}{}
\markboth{\textcolor{darkblue}{\textbf{\ipa{ʑi˩-kʰv̩˧˥}}}}{}
\textcolor{teal}{\zh{名词}} \hspace{4pt} \zh{声调类:} LM+MH\#.
\zh{猴年。} \textcolor{Sepia}{\selectlanguage{english}Year of the monkey.} \textcolor{PineGreen}{\selectlanguage{french}Année du singe.} 
\lhead{\firstmark}
\rhead{\botmark}

\subsection{\hspace{-0.5cm} {\Large \textcolor{darkblue}{\textbf{\ipa{ʑi˩mi\#˥}}}}\hspace{0.5cm}[\kern2pt{\textcolor{darkblue}{\textbf{\ipa{ʑi˧mi˧}}}}\kern2pt]} \hypertarget{z£i\string_Bmi\#\string_T1}{}
\markboth{\textcolor{darkblue}{\textbf{\ipa{ʑi˩mi\#˥}}}}{}
\textcolor{teal}{\zh{名词}} \hspace{4pt} \zh{声调类:} LM+\#H.
\zh{母猴。} \textcolor{Sepia}{\selectlanguage{english}Female monkey.} \textcolor{PineGreen}{\selectlanguage{french}Singe femelle.}  \zh{量词}: \textcolor{darkblue}{\textbf{\ipa{mi˩}}} 
\lhead{\firstmark}
\rhead{\botmark}

\subsection{\hspace{-0.5cm} {\Large \textcolor{darkblue}{\textbf{\ipa{ʑi˩pʰv̩\#˥}}}}\hspace{0.5cm}[\kern2pt{\textcolor{darkblue}{\textbf{\ipa{ʑi˧pʰv̩˥}}}}\kern2pt]} \hypertarget{z£i\string_Bp\string_hv\string_=\#\string_T1}{}
\markboth{\textcolor{darkblue}{\textbf{\ipa{ʑi˩pʰv̩\#˥}}}}{}
\textcolor{teal}{\zh{名词}} \hspace{4pt} \zh{声调类:} LM+\#H.
\zh{公猴。} \textcolor{Sepia}{\selectlanguage{english}Male monkey.} \textcolor{PineGreen}{\selectlanguage{french}Singe mâle.}  \zh{量词}: \textcolor{darkblue}{\textbf{\ipa{mi˩}}} 
\lhead{\firstmark}
\rhead{\botmark}

\subsection{\hspace{-0.5cm} {\Large \textcolor{darkblue}{\textbf{\ipa{ʑi˩zo\#˥}}}}\hspace{0.5cm}[\kern2pt{\textcolor{darkblue}{\textbf{\ipa{ʑi˧zo˥}}}}\kern2pt]} \hypertarget{z£i\string_Bzo\#\string_T1}{}
\markboth{\textcolor{darkblue}{\textbf{\ipa{ʑi˩zo\#˥}}}}{}
\textcolor{teal}{\zh{名词}} \hspace{4pt} \zh{声调类:} LM+\#H.
\zh{小猴子。} \textcolor{Sepia}{\selectlanguage{english}Infant monkey/ape.} \textcolor{PineGreen}{\selectlanguage{french}Petit singe.}  \zh{量词}: \textcolor{darkblue}{\textbf{\ipa{ɭɯ˧}}} 
\lhead{\firstmark}
\rhead{\botmark}

\subsection{\hspace{-0.5cm} {\Large \textcolor{darkblue}{\textbf{\ipa{ʑi˧˥}}} \textsubscript{1}}\hspace{0.5cm}[\kern2pt{\textcolor{darkblue}{\textbf{\ipa{ʑi˧˥}}}}\kern2pt]} \hypertarget{z£i\string_M\string_T1}{}
\markboth{\textcolor{darkblue}{\textbf{\ipa{ʑi˧˥}}} \textsubscript{1}}{}
\textcolor{teal}{\zh{动词}} \hspace{4pt} \zh{声调类:} MH.
\zh{睡觉。} \textcolor{Sepia}{\selectlanguage{english}To sleep.} \textcolor{PineGreen}{\selectlanguage{french}Dormir.}  ¶ \textcolor{darkblue}{\textbf{\ipa{le˧-ʑi˧˥}}} \zh{\mytextsc{accomp}} \textcolor{Sepia}{\selectlanguage{english}\mytextsc{accomp}} \textcolor{PineGreen}{\selectlanguage{french}\mytextsc{accomp}}  
 ¶ \textcolor{darkblue}{\textbf{\ipa{le˧-ʑi˧-ze˥}}} \zh{\mytextsc{accomp}  \mytextsc{pfv}} \textcolor{Sepia}{\selectlanguage{english}\mytextsc{accomp}  \mytextsc{pfv}} \textcolor{PineGreen}{\selectlanguage{french}\mytextsc{accomp}  \mytextsc{pfv}}  
 ¶ \textcolor{darkblue}{\textbf{\ipa{æ˩ ʑi˧-ze˥}}} \zh{鸡睡觉了} \textcolor{Sepia}{\selectlanguage{english}the chicken has gone asleep} \textcolor{PineGreen}{\selectlanguage{french}la poule s'est endormie}  
 ¶ \textcolor{darkblue}{\textbf{\ipa{le˧-ʑi˧-bi˧-ze˩!}}} \zh{要睡觉了!} \textcolor{Sepia}{\selectlanguage{english}I'm going to sleep!} \textcolor{PineGreen}{\selectlanguage{french}( je) vais dormir!/(je) vais me coucher!}  
 ¶ \textcolor{darkblue}{\textbf{\ipa{le˧-ʑi˧˥, | ʑi˧-mɤ˥-tʰɑ˩!}}} \zh{想睡,但睡不了!} \textcolor{Sepia}{\selectlanguage{english}Were I to try to sleep, I would not be able to! / I would like to sleep, but I can't!} \textcolor{PineGreen}{\selectlanguage{french}j'essaierais de dormir, que je n'y arriverais pas! / Dormir, il ne faut pas y penser/je n'y arriverais pas! (contexte: une personne âgée se plaint de maux de tête; on lui suggère d'aller se reposer/faire une sieste)}  
 ¶ \textcolor{darkblue}{\textbf{\ipa{pʰæ˧tɕi˥-zo˩-ɳɯ˩ | mv̩˩zo˩-qɑ˥ ʑi˩}}} \zh{小伙子跟年轻女人睡!(庸俗说法)} \textcolor{Sepia}{\selectlanguage{english}The young man sleeps with the young woman. (This phrasing refers rather bluntly to sexual intercourse, and is considered extremely rude.)} \textcolor{PineGreen}{\selectlanguage{french}Le jeune homme couche avec la jeune femme. (Formulation considérée comme vulgaire; ce thème est tabou.)}  

\lhead{\firstmark}
\rhead{\botmark}

\subsection{\hspace{-0.5cm} {\Large \textcolor{darkblue}{\textbf{\ipa{ʑi˩˥}}}}\hspace{0.5cm}[\kern2pt{\textcolor{darkblue}{\textbf{\ipa{ʑi˩˥}}}}\kern2pt]} \hypertarget{z£i\string_B\string_T1}{}
\markboth{\textcolor{darkblue}{\textbf{\ipa{ʑi˩˥}}}}{}
\textcolor{teal}{\zh{名词}} \hspace{4pt} \zh{声调类:} LH.
\zh{猴子。} \textcolor{Sepia}{\selectlanguage{english}Monkey, ape.} \textcolor{PineGreen}{\selectlanguage{french}Singe.}  ¶ \textcolor{darkblue}{\textbf{\ipa{ʑi˩ dzɯ˧-ze˩}}} \zh{吃了猴子} \textcolor{Sepia}{\selectlanguage{english}...has eaten (a/the) monkey} \textcolor{PineGreen}{\selectlanguage{french}(sujet non spécifié: un tigre...) a mangé (le) singe}  
 ¶ \textcolor{darkblue}{\textbf{\ipa{ʑi˩ hwæ˧-ze˩}}} \zh{买了猴子} \textcolor{Sepia}{\selectlanguage{english}...has bought (a/the) monkey} \textcolor{PineGreen}{\selectlanguage{french}(sujet non spécifié: quelqu'un) a acheté (le) singe}  
 \zh{量词}: \textcolor{darkblue}{\textbf{\ipa{mi˩}}} 
\lhead{\firstmark}
\rhead{\botmark}

\subsection{\hspace{-0.5cm} {\Large \textcolor{darkblue}{\textbf{\ipa{*ʑi˩˧}}}}\hspace{0.5cm}[\kern2pt{\textcolor{darkblue}{\textbf{\ipa{ʑi˩˥}}}}\kern2pt]} \hypertarget{*z£i\string_B\string_M1}{}
\markboth{\textcolor{darkblue}{\textbf{\ipa{*ʑi˩˧}}}}{}
\textcolor{teal}{\zh{名词}} \hspace{4pt} \zh{声调类:} LM? LH?.
\zh{房屋。} \textcolor{Sepia}{\selectlanguage{english}Building; houses.} \textcolor{PineGreen}{\selectlanguage{french}Maison, bâtiment d'habitation; monosyllabe extrait de l'expression /ʑi˩ tsʰi˩˥, | æ̃˩ tsʰi˧/ 'bâtir une demeure': le schéma tonal, avec un verbe au ton MH, peut provenir d'un nom au ton LM ou LH.}  ¶ \textcolor{darkblue}{\textbf{\ipa{ʑi˩ tsʰi˧˥, | æ̃˩ tsʰi˥}}} \zh{建房立家(固定词语)} \textcolor{Sepia}{\selectlanguage{english}to build a house, to build a home (set phrase)} \textcolor{PineGreen}{\selectlanguage{french}Bâtir une demeure, bâtir un foyer (expression proverbiale)}  
 ¶ \textcolor{darkblue}{\textbf{\ipa{ʑi˩ tʰv̩˩˥}}} \zh{分家,建立自己的新房屋比如:孩子多,一个孩子建自己的房子)} \textcolor{Sepia}{\selectlanguage{english}to found a new home, to build a new house} \textcolor{PineGreen}{\selectlanguage{french}Fonder une nouvelle demeure: lorsque la famille est nombreuse, un de ses membres peut bâtir sa propre demeure}  
 ¶ \textcolor{darkblue}{\textbf{\ipa{ʑi˩ qʰæ˧˥}}} \zh{拆房子(这个例子是调查者构造的,F4确定是可以说的。造这个例子的目的有两个:看单音节词根“家”能不能跟其它动词结合,也试着确定它的调类。)} \textcolor{Sepia}{\selectlanguage{english}to demolish a house} \textcolor{PineGreen}{\selectlanguage{french}démolir une demeure}  
 ¶ \textcolor{darkblue}{\textbf{\ipa{*ʑi˩ hwæ˧}}} \zh{*买房(这个例子是调查者构造的,F4确定是不可以说的。造这个例子的目的有两个:看单音节词根“家”能不能跟其它动词结合,也试着确定它的调类。)} \textcolor{Sepia}{\selectlanguage{english}*to buy a house (example coined by the investigator, to investigate the monosyllable's potential to combine with other verbs, and its tone category; this example was refused by speaker F4)} \textcolor{PineGreen}{\selectlanguage{french}*acheter une demeure; forme non correcte, proposée pour voir dans quelle mesure le monosyllabe peut se combiner avec divers verbes.}  

\lhead{\firstmark}
\rhead{\botmark}

\lhead{\firstmark}
\rhead{\botmark}

\lhead{\firstmark}
\rhead{\botmark}

\lhead{\firstmark}
\rhead{\botmark}

\lhead{\firstmark}
\rhead{\botmark}

\lhead{\firstmark}
\rhead{\botmark}

\lhead{\firstmark}
\rhead{\botmark}

\lhead{\firstmark}
\rhead{\botmark}

\lhead{\firstmark}
\rhead{\botmark}

\lhead{\firstmark}
\rhead{\botmark}

\lhead{\firstmark}
\rhead{\botmark}

\lhead{\firstmark}
\rhead{\botmark}

\lhead{\firstmark}
\rhead{\botmark}

\lhead{\firstmark}
\rhead{\botmark}

\lhead{\firstmark}
\rhead{\botmark}

\lhead{\firstmark}
\rhead{\botmark}

\lhead{\firstmark}
\rhead{\botmark}

\lhead{\firstmark}
\rhead{\botmark}

\lhead{\firstmark}
\rhead{\botmark}

\lhead{\firstmark}
\rhead{\botmark}

\lhead{\firstmark}
\rhead{\botmark}

\lhead{\firstmark}
\rhead{\botmark}

\end{multicols}

\newpage
\section*{\centering 没有发现对应摩挲固有词}

\begin{center}
\begin{longtable}{r|l}
 & \textcolor{brown}{\zh{猪獾}} \\
 & \textcolor{brown}{\zh{猞猁}} \\
 & \textcolor{brown}{\zh{穿山甲}} \\
 & \textcolor{brown}{\zh{耳垂}} \\
 & \textcolor{brown}{\zh{太阳穴}} \\
 & \textcolor{brown}{\zh{乳扇}} \\
 & \textcolor{brown}{\zh{猪鬃毛牙刷}} \\
 & \textcolor{brown}{\zh{抛石机}} \\
 & \textcolor{brown}{\zh{绷弓子}} \\
 & \textcolor{brown}{\zh{当归}} \\
 & \textcolor{brown}{\zh{山茛菪}} \\
 & \textcolor{brown}{\zh{冬瓜}} \\
 & \textcolor{brown}{\zh{芹菜}} \\
 & \textcolor{brown}{\zh{草麻黄}} \\
 & \textcolor{brown}{\zh{红豆杉}} \\
 & \textcolor{brown}{\zh{山紫菀}} \\
 & \textcolor{brown}{\zh{卷柏}} \\
 & \textcolor{brown}{\zh{野人}} \\
 & \textcolor{brown}{\zh{双胞胎}} \\
 & \textcolor{brown}{\zh{用吸管喝}} \\
 & \textcolor{brown}{\zh{后悔}} \\
\end{longtable}\end{center}
\end{document}
