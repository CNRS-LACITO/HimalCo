

\begin{table}[H]
\label{ɛ.vi} \centering
\caption{अकर्मक क्रिया  घ्र्यन्य  "बल्नु"  }
\begin{tabular}{l|l|l|l|l|l|l|l|l|l|l|l|l}  \toprule
&अभूत & भूत & आज्ञार्थक \\ 
उ:ङ्‌ &घ्र्यङ &घ्र्यङत \\ 
इऽचि &घ्र्ययि &घ्र्या:इति   \\ 
ओऽचु &घ्र्ययु &घ्र्या:इतु   \\ 
इक् &घ्र्यकि &घ्र्याक्‌तिकि   \\ 
ओक् &घ्र्यक &घ्र्याक्‌तक   \\ 
इन् & इघ्र्य & इघ्रात्य &घ्राये  \\ 
एऽचि & इघ्र्ययि & इघ्र्या:इति &घ्र्या:इये    \\ 
ए:न् & इघ्र्यनि  & इघ्रात्‍नु &घ्रा:न्ये  \\ 
अम् & घ्र्य & घ्रात्य   \\ 
अम्सु & घ्र्ययि & घ्र्या:इति     \\ 
अम्ह्‍याम् & घ्र्यनु  & घ्रात्‍नु \\ 
\bottomrule
\end{tabular}
\end{table}


\begin{table}[H]
\label{ɛp.vi} \centering
\caption{अकर्मक क्रिया  ख्या:म्‍न्य  "हुनै लागेको "  }
\begin{tabular}{l|l|l|l|l|l|l|l|l|l|l|l|l}  \toprule
&अभूत & भूत & आज्ञार्थक \\ 
उ:ङ्‌ &ख्या:म्ङ &ख्यपत \\ 
इऽचि &ख्यपि &ख्यपिति   \\ 
ओऽचु &ख्यपु &ख्यपुतु   \\ 
इक् &ख्याप्कि &ख्याप्‍तिकि   \\ 
ओक् &ख्याप्क &ख्याप्‍तक   \\ 
इन् & इख्याऽप् & इख्याप्‍त्य &ख्याप्ये  \\ 
एऽचि & इख्यपि & इख्यपिति &ख्यपिये    \\ 
ए:न् & इख्या:म्‍नि  & इख्याप्‍त्यनु &ख्याप्‍नुये  \\ 
अम् & ख्याऽप् & ख्याप्‍त्य   \\ 
अम्सु & ख्यपि & ख्यपिति   \\ 
अम्ह्‍याम् & ख्या:म्‍नु  & ख्याप्‍त्यनु \\ 
\bottomrule
\end{tabular}
\end{table}


\begin{table}[H]
\label{ɛt.vi} \centering
\caption{अकर्मक क्रिया  अ्या:न्‍न्य  "भन्नु"  }
\begin{tabular}{l|l|l|l|l|l|l|l|l|l|l|l|l}  \toprule
&अभूत & भूत & आज्ञार्थक \\ 
उ:ङ्‌ &अ्या:इङ &अ्यास्त \\ 
इऽचि &अ्यचि &अ्यास्ति   \\ 
ओऽचु &अ्यचु &अ्यास्तु   \\ 
इक् &अ्याह्इकि &अ्याह्इतिकि   \\ 
ओक् &अ्याह्इक &अ्याह्इतक   \\ 
इन् & इअ्या:इ & इअ्यास्त्य &अ्यच्‍चे  \\ 
एऽचि & इअ्यचि & इअ्यास्ति &अ्यचिये    \\ 
ए:न् & इअ्या:न्‍नि  & इअ्यास्त्यनु &अ्यास्‍नुये  \\ 
अम् & अ्या:इ & अ्यास्त्य   \\ 
अम्सु & अ्यचि & अ्यास्ति   \\ 
अम्ह्‍याम् & अ्या:न्‍नु  & अ्यास्त्यनु \\ 
\bottomrule
\end{tabular}
\end{table}


\begin{table}[H]
\label{ɛm.vi} \centering
\caption{अकर्मक क्रिया  झ्याम्‍न्य  "हराउनु"  }
\begin{tabular}{l|l|l|l|l|l|l|l|l|l|l|l|l}  \toprule
&अभूत & भूत & आज्ञार्थक \\ 
उ:ङ्‌ &झ्याम्ङ &झ्यमत \\ 
इऽचि &झ्यमि &झ्यमिति   \\ 
ओऽचु &झ्यमु &झ्यमुतु   \\ 
इक् &झ्याम्कि &झ्याम्तिकि   \\ 
ओक् &झ्याम्क &झ्याम्तक   \\ 
इन् & इझ्याम् & इझ्या:म्त्य &झ्या:म्ये  \\ 
एऽचि & इझ्यमि & इझ्यमिति &झ्यमिये    \\ 
ए:न् & इझ्याम्‍नि  & इझ्या:म्त्यनु &झ्या:म्‍नुये  \\ 
अम् & झ्याम् & झ्या:म्त्य   \\ 
अम्सु & झ्यमि & झ्यमिति   \\ 
अम्ह्‍याम् & झ्याम्‍नु  & झ्या:म्त्यनु \\ 
\bottomrule
\end{tabular}
\end{table}


\begin{table}[H]
\label{ɛn.vi} \centering
\caption{अकर्मक क्रिया  ङ्‌याइन्य  "बस्नु"  }
\begin{tabular}{l|l|l|l|l|l|l|l|l|l|l|l|l}  \toprule
&अभूत & भूत & आज्ञार्थक \\ 
उ:ङ्‌ &ङ्‌याइङ &ङ्‌याऽस्त \\ 
इऽचि &ङ्‌याचि &ङ्‌याऽस्ति   \\ 
ओऽचु &ङ्‌याचु &ङ्‌याऽस्तु   \\ 
इक् &ङ्‌याइकि &ङ्‌याइतिकि   \\ 
ओक् &ङ्‌याइक &ङ्‌याइतक   \\ 
इन् & इङ्‌याइ & इङ्‌याऽस्त्य &ङ्‌याचे  \\ 
एऽचि & इङ्‌याचि & इङ्‌याऽस्ति &ङ्‌याचिये    \\ 
ए:न् & इङ्‌याइनि  & इङ्‌याऽस्त्यनु &ङ्‌याऽस्‍नुये  \\ 
अम् & ङ्‌याइ & ङ्‌याऽस्त्य   \\ 
अम्सु & ङ्‌याचि & ङ्‌याऽस्ति   \\ 
अम्ह्‍याम् & ङ्‌याइनु  & ङ्‌याऽस्त्यनु \\ 
\bottomrule
\end{tabular}
\end{table}


\begin{table}[H]
\label{aŋ.vi} \centering
\caption{अकर्मक क्रिया  घान्य  "मान्नु"  }
\begin{tabular}{l|l|l|l|l|l|l|l|l|l|l|l|l}  \toprule
&अभूत & भूत & आज्ञार्थक \\ 
उ:ङ्‌ &घाङ &घङत \\ 
इऽचि &घङि &घङिति   \\ 
ओऽचु &घङु &घङुतु   \\ 
इक् &घाङ्‌कि &घाङ्‌तिकि   \\ 
ओक् &घाङ्‌क &घाङ्‌तक   \\ 
इन् & इघाङ्‌ & इघा:ङ्‌त्य &घा:ङ्‌ये  \\ 
एऽचि & इघङि & इघङिति &घङिये    \\ 
ए:न् & इघानि  & इघा:ङ्‌त्यनु &घा:ङ्‌नुये  \\ 
अम् & घाङ्‌ & घा:ङ्‌त्य   \\ 
अम्सु & घङि & घङिति   \\ 
अम्ह्‍याम् & घानु  & घा:ङ्‌त्यनु \\ 
\bottomrule
\end{tabular}
\end{table}


\begin{table}[H]
\label{ɛr.vi} \centering
\caption{अकर्मक क्रिया  घ्यार्न्य  "नीद नलाग्नु"  }
\begin{tabular}{l|l|l|l|l|l|l|l|l|l|l|l|l}  \toprule
&अभूत & भूत & आज्ञार्थक \\ 
उ:ङ्‌ &घ्यार्ङ &घ्यरत \\ 
इऽचि &घ्यरि &घ्यरिति   \\ 
ओऽचु &घ्यरु &घ्यरुतु   \\ 
इक् &घ्यार्कि &घ्यार्तिकि   \\ 
ओक् &घ्यार्क &घ्यार्तक   \\ 
इन् & इघ्यार् & इघ्या:र्त्य &घ्या:र्‍ये  \\ 
एऽचि & इघ्यरि & इघ्यरिति &घ्यरिये    \\ 
ए:न् & इघ्यार्नि  & इघ्या:र्त्यनु &घ्या:र्नुये  \\ 
अम् & घ्यार् & घ्या:र्त्य   \\ 
अम्सु & घ्यरि & घ्यरिति   \\ 
अम्ह्‍याम् & घ्यार्नु  & घ्या:र्त्यनु \\ 
\bottomrule
\end{tabular}
\end{table}


\begin{table}[H]
\label{ɛl.vi} \centering
\caption{अकर्मक क्रिया  ङ्‌याल्न्य  "फटाहा गर्नु"  }
\begin{tabular}{l|l|l|l|l|l|l|l|l|l|l|l|l}  \toprule
&अभूत & भूत & आज्ञार्थक \\ 
उ:ङ्‌ &ङ्‌याल्ङ &ङ्‌यलत \\ 
इऽचि &ङ्‌यलि &ङ्‌यलिति   \\ 
ओऽचु &ङ्‌यलु &ङ्‌यलुतु   \\ 
इक् &ङ्‌याल्कि &ङ्‌याल्तिकि   \\ 
ओक् &ङ्‌याल्क &ङ्‌याल्तक   \\ 
इन् & इङ्‌याल् & इङ्‌या:ल्त्य &ङ्‌या:ल्ये  \\ 
एऽचि & इङ्‌यलि & इङ्‌यलिति &ङ्‌यलिये    \\ 
ए:न् & इङ्‌याल्नि  & इङ्‌या:ल्त्यनु &ङ्‌या:ल्नुये  \\ 
अम् & ङ्‌याल् & ङ्‌या:ल्त्य   \\ 
अम्सु & ङ्‌यलि & ङ्‌यलिति   \\ 
अम्ह्‍याम् & ङ्‌याल्नु  & ङ्‌या:ल्त्यनु \\ 
\bottomrule
\end{tabular}
\end{table}


\begin{table}[H]
\label{a.vt} \centering
\caption{सकर्मक क्रिया  क्योन्य  "खानु"  }
\begin{tabular}{l|l|l|l|l|l|l|l|l|l|l|l|l}  \toprule
&अभूत & भूत & आज्ञार्थक \\ 
उङ &कङ &कु:ङ्‌त \\ 
इऽचिअ्य &क्योयि &क्यो:इति   \\ 
ओऽचुअ &क्योयु &क्यो:इतु   \\ 
इक्अ्य &क्योकि &क्योक्‌तिकि   \\ 
ओक्अ &क्योक &क्योक्‌तक   \\ 
इन्य & इक्य & इक्युत्य &क्युये  \\ 
एऽचिअ्य & इक्योयि & इक्यो:इति &क्यो:इये    \\ 
ए:न्अ्य & इक्योनि  & इकोत्‍नु &को:न्ये  \\ 
अम्अ्य & क्य & क्युत्य   \\ 
अम्सुअ & क्यसु & क्युत्सु     \\ 
अम्ह्‍याम्अ्य & क्यनु  & क्युत्‍नु \\ 
\midrule
इन्य/अम्अ्य उ:ङ्‌&इक्योङ &इक्योङत &क्योङये \\ 
एऽचिअ्य/अम्सुअ उ:ङ्‌ &इक्योङसु &इक्योङतसु &क्योङसुये \\ 
ए:न्अ्य/अम्ह्‍याम्अ्य उ:ङ्‌ &इक्योङनु &इक्योङतनु &क्योङनुये \\ 
इन्य/अम्अ्य इऽचि &इक्योयि &इक्यो:इति    \\ 
इन्य/अम्अ्य ओऽचु &इक्योयु &इक्यो:इतु  &क्योइये  \\ 
इन्य/अम्अ्य इक् &इक्योकि &इक्योक्‌तिकि   \\ 
इन्य/अम्अ्य ओक् &इक्योक &इक्योक्‌तक  &क्योकये  \\ 
अम्अ्य इन् & इक्यो & इकोऽत्य   \\ 
अम्अ्य एऽचि & इक्योयि & इक्यो:इति     \\ 
अम्अ्य ए:न् & इक्योनि  & इकोत्‍नु  \\ 
\midrule
उङ इन् & क्योन्य  & क्यो:न्त्यनि  \\ 
उङ एऽचि & क्यो:न्सु  & क्यो:न्त्यान्सु   \\ 
उङ ए:न्& क्यो:न्‍नु  & क्यो:न्त्यान्‍नु   \\ 
\bottomrule
\end{tabular}
\end{table}


\begin{table}[H]
\label{ɛp.vt} \centering
\caption{सकर्मक क्रिया  ख्या:म्‍न्य  "हुनै लागेको "  }
\begin{tabular}{l|l|l|l|l|l|l|l|l|l|l|l|l}  \toprule
&अभूत & भूत & आज्ञार्थक \\ 
उङ &ख्यबु &ख्यबुत \\ 
उङ अम्सु &ख्यबुसु &ख्यबुतसु \\ 
उङ अम्ह्‍याम् &ख्यबुनु &ख्यबुतनु \\ 
इऽचिअ्य &ख्यपि &ख्यपिति   \\ 
ओऽचुअ &ख्यपु &ख्यपुतु   \\ 
इक्अ्य &ख्याप्कि &ख्याप्‍तिकि   \\ 
ओक्अ &ख्याप्क &ख्याप्‍तक   \\ 
इन्य अम् & इख्याब्यु  & इख्याऽप्‍त्य &ख्याबे  \\ 
इन्य अम्सु & इख्याऽप्सु  & इख्याऽप्‍त्यसु   \\ 
इन्य अम्ह्‍याम् & इख्याऽप्‍नु  & इख्याऽप्‍त्यनु   \\ 
एऽचिअ्य & इख्यपि & इख्यपिति &ख्यपिये    \\ 
ए:न्अ्य & इख्या:म्‍नि  & इख्याप्‍त्यनु &ख्याप्‍नुये  \\ 
अम्अ्य & ख्याब्यु  & ख्याऽप्‍त्य  \\ 
अम्सुअ & ख्याऽप्सु & ख्याऽप्‍त्यसु  \\ 
अम्ह्‍याम्अ्य & ख्याऽप्‍नु  & ख्याऽप्‍त्यनु \\ 
\midrule
इन्य/अम्अ्य उ:ङ्‌&इख्या:म्ङ & इख्यपत &ख्यपये \\ 
एऽचिअ्य/अम्सुअ उ:ङ्‌ &इख्या:म्ङसु & इख्यपतसु &ख्यपसुये \\ 
ए:न्अ्य/अम्ह्‍याम्अ्य उ:ङ्‌ &इख्या:म्ङनु & इख्यपतनु &ख्यपनुये \\ 
इन्य/अम्अ्य इऽचि & इख्यपि & इख्यपिति    \\ 
इन्य/अम्अ्य ओऽचु & इख्यपु & इख्यपुतु  &ख्यपुये  \\ 
इन्य/अम्अ्य इक् & इख्याप्कि & इख्याप्‍तिकि   \\ 
इन्य/अम्अ्य ओक् & इख्याप्क & इख्याप्‍तक  &ख्याप्कये  \\ 
अम्अ्य इन् & इख्याऽप् & इख्याप्‍त्य   \\ 
अम्अ्य एऽचि & इख्यपि & इख्यपिति    \\ 
अम्अ्य ए:न् & इख्या:म्‍नि  & इख्याप्‍त्यनु  \\ 
\midrule
उङ इन् & ख्या:म्‍न्य  & ख्या:म्त्यनि  \\ 
उङ एऽचि & ख्या:म्सु  & ख्या:म्त्यान्सु   \\ 
उङ ए:न्& ख्या:म्‍नु  & ख्या:म्त्यान्‍नु   \\ 
\bottomrule
\end{tabular}
\end{table}


\begin{table}[H]
\label{ɛt.vt} \centering
\caption{सकर्मक क्रिया  क्या:न्‍न्य  "टोक्नु"  }
\begin{tabular}{l|l|l|l|l|l|l|l|l|l|l|l|l}  \toprule
&अभूत & भूत & आज्ञार्थक \\ 
उङ &क्यदु &क्या:त \\ 
उङ अम्सु &क्यदुसु &क्या:तसु \\ 
उङ अम्ह्‍याम् &क्यदुनु &क्या:तनु \\ 
इऽचिअ्य &क्यचि &क्यास्ति   \\ 
ओऽचुअ &क्यचु &क्यास्तु   \\ 
इक्अ्य &क्याह्इकि &क्याह्इतिकि   \\ 
ओक्अ &क्याह्इक &क्याह्इतक   \\ 
इन्य अम् & इक्याद्‌यु  & इक्या:त्य &क्यादे  \\ 
इन्य अम्सु & इक्याऽत्सु  & इक्या:त्यसु   \\ 
इन्य अम्ह्‍याम् & इक्याऽत्‍नु  & इक्या:त्यनु   \\ 
एऽचिअ्य & इक्यचि & इक्यास्ति &क्यचिये    \\ 
ए:न्अ्य & इक्या:न्‍नि  & इक्यास्त्यनु &क्यास्‍नुये  \\ 
अम्अ्य & क्याद्‌यु  & क्या:त्य  \\ 
अम्सुअ & क्याऽत्सु & क्या:त्यसु  \\ 
अम्ह्‍याम्अ्य & क्याऽत्‍नु  & क्या:त्यनु \\ 
\midrule
इन्य/अम्अ्य उ:ङ्‌&इक्या:इङ & इक्यास्त &क्यचये \\ 
एऽचिअ्य/अम्सुअ उ:ङ्‌ &इक्या:इङसु & इक्यास्तसु &क्यचसुये \\ 
ए:न्अ्य/अम्ह्‍याम्अ्य उ:ङ्‌ &इक्या:इङनु & इक्यास्तनु &क्यचनुये \\ 
इन्य/अम्अ्य इऽचि & इक्यचि & इक्यास्ति    \\ 
इन्य/अम्अ्य ओऽचु & इक्यचु & इक्यास्तु  &क्यचुये  \\ 
इन्य/अम्अ्य इक् & इक्याह्इकि & इक्याह्इतिकि   \\ 
इन्य/अम्अ्य ओक् & इक्याह्इक & इक्याह्इतक  &क्याह्इकये  \\ 
अम्अ्य इन् & इक्या:इ & इक्यास्त्य   \\ 
अम्अ्य एऽचि & इक्यचि & इक्यास्ति    \\ 
अम्अ्य ए:न् & इक्या:न्‍नि  & इक्यास्त्यनु  \\ 
\midrule
उङ इन् & क्या:न्‍न्य  & क्या:न्त्यनि  \\ 
उङ एऽचि & क्या:न्सु  & क्या:न्त्यान्सु   \\ 
उङ ए:न्& क्या:न्‍नु  & क्या:न्त्यान्‍नु   \\ 
\bottomrule
\end{tabular}
\end{table}


\begin{table}[H]
\label{ak.vt} \centering
\caption{सकर्मक क्रिया  सा:न्य  "छान्नु"  }
\begin{tabular}{l|l|l|l|l|l|l|l|l|l|l|l|l}  \toprule
&अभूत & भूत & आज्ञार्थक \\ 
उङ &सगु &सगुत \\ 
उङ अम्सु &सगुसु &सगुतसु \\ 
उङ अम्ह्‍याम् &सगुनु &सगुतनु \\ 
इऽचिअ्य &सकि &सकिति   \\ 
ओऽचुअ &सकु &सकुतु   \\ 
इक्अ्य &साक्‌कि &साक्‌तिकि   \\ 
ओक्अ &साक्‌क &साक्‌तक   \\ 
इन्य अम् & इसाग्यु  & इसाक्‌त्य &सागे  \\ 
इन्य अम्सु & इसाक्सु  & इसाक्‌त्यसु   \\ 
इन्य अम्ह्‍याम् & इसाक्‍नु  & इसाक्‌त्यनु   \\ 
एऽचिअ्य & इसकि & इसकिति &सकिये    \\ 
ए:न्अ्य & इसा:नि  & इसाक्‌त्यनु &साक्‍नुये  \\ 
अम्अ्य & साग्यु  & साक्‌त्य  \\ 
अम्सुअ & साक्सु & साक्‌त्यसु  \\ 
अम्ह्‍याम्अ्य & साक्‍नु  & साक्‌त्यनु \\ 
\midrule
इन्य/अम्अ्य उ:ङ्‌&इसा:ङ & इसकत &सकये \\ 
एऽचिअ्य/अम्सुअ उ:ङ्‌ &इसा:ङसु & इसकतसु &सकसुये \\ 
ए:न्अ्य/अम्ह्‍याम्अ्य उ:ङ्‌ &इसा:ङनु & इसकतनु &सकनुये \\ 
इन्य/अम्अ्य इऽचि & इसकि & इसकिति    \\ 
इन्य/अम्अ्य ओऽचु & इसकु & इसकुतु  &सकुये  \\ 
इन्य/अम्अ्य इक् & इसाक्‌कि & इसाक्‌तिकि   \\ 
इन्य/अम्अ्य ओक् & इसाक्‌क & इसाक्‌तक  &साक्‌कये  \\ 
अम्अ्य इन् & इसा: & इसाक्‌त्य   \\ 
अम्अ्य एऽचि & इसकि & इसकिति    \\ 
अम्अ्य ए:न् & इसा:नि  & इसाक्‌त्यनु  \\ 
\midrule
उङ इन् & सा:न्य  & सा:न्त्यनि  \\ 
उङ एऽचि & सा:न्सु  & सा:न्त्यान्सु   \\ 
उङ ए:न्& सा:न्‍नु  & सा:न्त्यान्‍नु   \\ 
\bottomrule
\end{tabular}
\end{table}


\begin{table}[H]
\label{ɛm.vt} \centering
\caption{सकर्मक क्रिया  ख्ल्याम्‍न्य  "खानु (बोक्सी)"  }
\begin{tabular}{l|l|l|l|l|l|l|l|l|l|l|l|l}  \toprule
&अभूत & भूत & आज्ञार्थक \\ 
उङ &ख्ल्यमु &ख्ल्यमुत \\ 
उङ अम्सु &ख्ल्यमुसु &ख्ल्यमुतसु \\ 
उङ अम्ह्‍याम् &ख्ल्यमुनु &ख्ल्यमुतनु \\ 
इऽचिअ्य &ख्ल्यमि &ख्ल्यमिति   \\ 
ओऽचुअ &ख्ल्यमु &ख्ल्यमुतु   \\ 
इक्अ्य &ख्ल्याम्कि &ख्ल्याम्तिकि   \\ 
ओक्अ &ख्ल्याम्क &ख्ल्याम्तक   \\ 
इन्य अम् & इख्ल्याम्यु  & इख्ल्या:म्त्य &ख्ल्यामे  \\ 
इन्य अम्सु & इख्ल्या:म्सु  & इख्ल्या:म्त्यसु   \\ 
इन्य अम्ह्‍याम् & इख्ल्या:म्‍नु  & इख्ल्या:म्त्यनु   \\ 
एऽचिअ्य & इख्ल्यमि & इख्ल्यमिति &ख्ल्यमिये    \\ 
ए:न्अ्य & इख्ल्याम्‍नि  & इख्ल्या:म्त्यनु &ख्ल्या:म्‍नुये  \\ 
अम्अ्य & ख्ल्याम्यु  & ख्ल्या:म्त्य  \\ 
अम्सुअ & ख्ल्या:म्सु & ख्ल्या:म्त्यसु  \\ 
अम्ह्‍याम्अ्य & ख्ल्या:म्‍नु  & ख्ल्या:म्त्यनु \\ 
\midrule
इन्य/अम्अ्य उ:ङ्‌&इख्ल्याम्ङ & इख्ल्यमत &ख्ल्यमये \\ 
एऽचिअ्य/अम्सुअ उ:ङ्‌ &इख्ल्याम्ङसु & इख्ल्यमतसु &ख्ल्यमसुये \\ 
ए:न्अ्य/अम्ह्‍याम्अ्य उ:ङ्‌ &इख्ल्याम्ङनु & इख्ल्यमतनु &ख्ल्यमनुये \\ 
इन्य/अम्अ्य इऽचि & इख्ल्यमि & इख्ल्यमिति    \\ 
इन्य/अम्अ्य ओऽचु & इख्ल्यमु & इख्ल्यमुतु  &ख्ल्यमुये  \\ 
इन्य/अम्अ्य इक् & इख्ल्याम्कि & इख्ल्याम्तिकि   \\ 
इन्य/अम्अ्य ओक् & इख्ल्याम्क & इख्ल्याम्तक  &ख्ल्याम्कये  \\ 
अम्अ्य इन् & इख्ल्याम् & इख्ल्या:म्त्य   \\ 
अम्अ्य एऽचि & इख्ल्यमि & इख्ल्यमिति    \\ 
अम्अ्य ए:न् & इख्ल्याम्‍नि  & इख्ल्या:म्त्यनु  \\ 
\midrule
उङ इन् & ख्ल्याम्‍न्य  & ख्ल्याम्त्यनि  \\ 
उङ एऽचि & ख्ल्या:म्सु  & ख्ल्याम्त्यान्सु   \\ 
उङ ए:न्& ख्ल्या:म्‍नु  & ख्ल्याम्त्यान्‍नु   \\ 
\bottomrule
\end{tabular}
\end{table}


\begin{table}[H]
\label{aŋ.vt} \centering
\caption{सकर्मक क्रिया  यान्य  "चाल्नु, खुसुक्क चोर्नु, खुसुखुसु चोर्नु या चोरी गर्नु"  }
\begin{tabular}{l|l|l|l|l|l|l|l|l|l|l|l|l}  \toprule
&अभूत & भूत & आज्ञार्थक \\ 
उङ म्य &यङु &यङुत \\ 
उङ म्यसु &यङुसु &यङुतसु \\ 
उङ म्यह्‍याम् &यङुनु &यङुतनु \\ 
इऽचिअ्य  &यङि &यङिति   \\ 
ओऽचुअ &यङु &यङुतु   \\ 
इक्अ्य &याङ्‌कि &याङ्‌तिकि   \\ 
ओक्अ &याङ्‌क &याङ्‌तक   \\ 
इन्य म्य& इयाङ्‌यु  & इया:ङ्‌त्य &याङे  \\ 
इन्य म्यसु & इया:ङ्‌सु  & इया:ङ्‌त्यसु   \\ 
इन्य म्यह्‍याम् & इया:ङ्‌नु  & इया:ङ्‌त्यनु   \\ 
एऽचिअ्य & इयङि & इयङिति &यङिये    \\ 
ए:न्अ्य & इयानि  & इया:ङ्‌त्यनु &या:ङ्‌नुये  \\ 
अम्अ्य & याङ्‌यु  & या:ङ्‌त्य  \\ 
अम्सुअ & या:ङ्‌सु & या:ङ्‌त्यसु  \\ 
अम्ह्‍याम्अ्य & या:ङ्‌नु  & या:ङ्‌त्यनु \\ 
\bottomrule
\end{tabular}
\end{table}


\begin{table}[H]
\label{ɛr.vt} \centering
\caption{सकर्मक क्रिया  व्यार्न्य  "दु:ख मान्नु"  }
\begin{tabular}{l|l|l|l|l|l|l|l|l|l|l|l|l}  \toprule
&अभूत & भूत & आज्ञार्थक \\ 
उङ म्य &व्यरु &व्यरुत \\ 
उङ म्यसु &व्यरुसु &व्यरुतसु \\ 
उङ म्यह्‍याम् &व्यरुनु &व्यरुतनु \\ 
इऽचिअ्य  &व्यरि &व्यरिति   \\ 
ओऽचुअ &व्यरु &व्यरुतु   \\ 
इक्अ्य &व्यार्कि &व्यार्तिकि   \\ 
ओक्अ &व्यार्क &व्यार्तक   \\ 
इन्य म्य& इव्यार्‍यु  & इव्या:र्त्य &व्यारे  \\ 
इन्य म्यसु & इव्या:र्सु  & इव्या:र्त्यसु   \\ 
इन्य म्यह्‍याम् & इव्या:र्नु  & इव्या:र्त्यनु   \\ 
एऽचिअ्य & इव्यरि & इव्यरिति &व्यरिये    \\ 
ए:न्अ्य & इव्यार्नि  & इव्या:र्त्यनु &व्या:र्नुये  \\ 
अम्अ्य & व्यार्‍यु  & व्या:र्त्य  \\ 
अम्सुअ & व्या:र्सु & व्या:र्त्यसु  \\ 
अम्ह्‍याम्अ्य & व्या:र्नु  & व्या:र्त्यनु \\ 
\bottomrule
\end{tabular}
\end{table}


\begin{table}[H]
\label{al.vt} \centering
\caption{सकर्मक क्रिया  याल्न्य  "कुट्नु"  }
\begin{tabular}{l|l|l|l|l|l|l|l|l|l|l|l|l}  \toprule
&अभूत & भूत & आज्ञार्थक \\ 
उङ &यलु &यलुत \\ 
उङ अम्सु &यलुसु &यलुतसु \\ 
उङ अम्ह्‍याम् &यलुनु &यलुतनु \\ 
इऽचिअ्य &यलि &यलिति   \\ 
ओऽचुअ &यलु &यलुतु   \\ 
इक्अ्य &याल्कि &याल्तिकि   \\ 
ओक्अ &याल्क &याल्तक   \\ 
इन्य अम् & इयाल्यु  & इया:ल्त्य &याले  \\ 
इन्य अम्सु & इया:ल्सु  & इया:ल्त्यसु   \\ 
इन्य अम्ह्‍याम् & इया:ल्नु  & इया:ल्त्यनु   \\ 
एऽचिअ्य & इयलि & इयलिति &यलिये    \\ 
ए:न्अ्य & इयाल्नि  & इया:ल्त्यनु &या:ल्नुये  \\ 
अम्अ्य & याल्यु  & या:ल्त्य  \\ 
अम्सुअ & या:ल्सु & या:ल्त्यसु  \\ 
अम्ह्‍याम्अ्य & या:ल्नु  & या:ल्त्यनु \\ 
\midrule
इन्य/अम्अ्य उ:ङ्‌&इयाल्ङ & इयलत &यलये \\ 
एऽचिअ्य/अम्सुअ उ:ङ्‌ &इयाल्ङसु & इयलतसु &यलसुये \\ 
ए:न्अ्य/अम्ह्‍याम्अ्य उ:ङ्‌ &इयाल्ङनु & इयलतनु &यलनुये \\ 
इन्य/अम्अ्य इऽचि & इयलि & इयलिति    \\ 
इन्य/अम्अ्य ओऽचु & इयलु & इयलुतु  &यलुये  \\ 
इन्य/अम्अ्य इक् & इयाल्कि & इयाल्तिकि   \\ 
इन्य/अम्अ्य ओक् & इयाल्क & इयाल्तक  &याल्कये  \\ 
अम्अ्य इन् & इयाल् & इया:ल्त्य   \\ 
अम्अ्य एऽचि & इयलि & इयलिति    \\ 
अम्अ्य ए:न् & इयाल्नि  & इया:ल्त्यनु  \\ 
\midrule
उङ इन् & याल्न्य  & याल्त्यनि  \\ 
उङ एऽचि & या:ल्सु  & याल्त्यान्सु   \\ 
उङ ए:न्& या:ल्नु  & याल्त्यान्‍नु   \\ 
\bottomrule
\end{tabular}
\end{table}


\begin{table}[H]
\label{ɛl.vt} \centering
\caption{सकर्मक क्रिया  फ्याल्न्य  "बिग्रनु"  }
\begin{tabular}{l|l|l|l|l|l|l|l|l|l|l|l|l}  \toprule
&अभूत & भूत & आज्ञार्थक \\ 
उङ &फ्यलु &फ्यलुत \\ 
उङ अम्सु &फ्यलुसु &फ्यलुतसु \\ 
उङ अम्ह्‍याम् &फ्यलुनु &फ्यलुतनु \\ 
इऽचिअ्य &फ्यलि &फ्यलिति   \\ 
ओऽचुअ &फ्यलु &फ्यलुतु   \\ 
इक्अ्य &फ्याल्कि &फ्याल्तिकि   \\ 
ओक्अ &फ्याल्क &फ्याल्तक   \\ 
इन्य अम् & इफ्याल्यु  & इफ्या:ल्त्य &फ्याले  \\ 
इन्य अम्सु & इफ्या:ल्सु  & इफ्या:ल्त्यसु   \\ 
इन्य अम्ह्‍याम् & इफ्या:ल्नु  & इफ्या:ल्त्यनु   \\ 
एऽचिअ्य & इफ्यलि & इफ्यलिति &फ्यलिये    \\ 
ए:न्अ्य & इफ्याल्नि  & इफ्या:ल्त्यनु &फ्या:ल्नुये  \\ 
अम्अ्य & फ्याल्यु  & फ्या:ल्त्य  \\ 
अम्सुअ & फ्या:ल्सु & फ्या:ल्त्यसु  \\ 
अम्ह्‍याम्अ्य & फ्या:ल्नु  & फ्या:ल्त्यनु \\ 
\midrule
इन्य/अम्अ्य उ:ङ्‌&इफ्याल्ङ & इफ्यलत &फ्यलये \\ 
एऽचिअ्य/अम्सुअ उ:ङ्‌ &इफ्याल्ङसु & इफ्यलतसु &फ्यलसुये \\ 
ए:न्अ्य/अम्ह्‍याम्अ्य उ:ङ्‌ &इफ्याल्ङनु & इफ्यलतनु &फ्यलनुये \\ 
इन्य/अम्अ्य इऽचि & इफ्यलि & इफ्यलिति    \\ 
इन्य/अम्अ्य ओऽचु & इफ्यलु & इफ्यलुतु  &फ्यलुये  \\ 
इन्य/अम्अ्य इक् & इफ्याल्कि & इफ्याल्तिकि   \\ 
इन्य/अम्अ्य ओक् & इफ्याल्क & इफ्याल्तक  &फ्याल्कये  \\ 
अम्अ्य इन् & इफ्याल् & इफ्या:ल्त्य   \\ 
अम्अ्य एऽचि & इफ्यलि & इफ्यलिति    \\ 
अम्अ्य ए:न् & इफ्याल्नि  & इफ्या:ल्त्यनु  \\ 
\midrule
उङ इन् & फ्याल्न्य  & फ्याल्त्यनि  \\ 
उङ एऽचि & फ्या:ल्सु  & फ्याल्त्यान्सु   \\ 
उङ ए:न्& फ्या:ल्नु  & फ्याल्त्यान्‍नु   \\ 
\bottomrule
\end{tabular}
\end{table}


\begin{table}[H]
\label{ɛpt.vt} \centering
\caption{सकर्मक क्रिया  फ्र्या:म्‍न्य  "कोट्याउनु"  }
\begin{tabular}{l|l|l|l|l|l|l|l|l|l|l|l|l}  \toprule
&अभूत & भूत & आज्ञार्थक \\ 
उङ &फ्र्याप्‍तु &फ्र्याप्‍त \\ 
उङ अम्सु&फ्र्याप्‍तुसु &फ्र्याप्‍तसु \\ 
उङ अम्ह्‍याम्&फ्र्याप्‍तुनु &फ्र्याप्‍तनु \\ 
इऽचिअ्य &फ्र्यपि &फ्र्यपिति   \\ 
ओऽचुअ        &फ्र्यपु &फ्र्यपुतु   \\ 
इक्अ्य&फ्र्याप्कि &फ्र्याप्‍तिकि   \\ 
ओक्अ &फ्र्याप्क &फ्र्याप्‍तक   \\ 
इन्य & इफ्र्याप्‍त्यु  & इफ्र्याप्‍त्य &फ्र्याप्‍ते  \\ 
इन्य अम्सु& इफ्र्याप्सु  & इफ्र्याप्‍त्यसु   \\ 
इन्य अम्ह्‍याम्& इफ्र्याप्‍नु  & इफ्र्याप्‍त्यनु   \\ 
एऽचि & इफ्र्यपि & इफ्र्यपिति &फ्र्यपिये    \\ 
ए:न् & इफ्र्या:म्‍नि  & इफ्र्याप्‍त्यनु &फ्र्याप्‍नुये  \\ 
अम्अ्य & फ्र्याप्‍त्यु  & फ्र्याप्‍त्य  \\ 
अम्सुअ्य & फ्र्याप्सु & फ्र्याप्‍त्यसु  \\ 
अम्ह्‍याम्अ्य & फ्र्याप्‍नु  & फ्र्याप्‍त्यनु \\ 
\midrule
इन्य, अम्अ्य उ:ङ्‌ &इफ्र्या:म्ङ &इफ्र्यपत &फ्र्यपये \\ 
एऽचिअ्य/अम्सुअ उ:ङ्‌ &इफ्र्या:म्ङसु &इफ्र्यपतसु &फ्र्यपसुये \\ 
ए:न्अ्य/अम्ह्‍याम्अ्य उ:ङ्‌ &इफ्र्या:म्ङनु &इफ्र्यपतनु &फ्र्यपनुये \\ 
इन्य/अम्अ्य इऽचि &इफ्र्यपि &इफ्र्यपिति    \\ 
इन्य/अम्अ्य ओऽचु &इफ्र्यपु &इफ्र्यपुतु  &फ्र्यपुये  \\ 
इन्य/अम्अ्य इक् &इफ्र्याप्कि &इफ्र्याप्‍तिकि   \\ 
इन्य/अम्अ्य ओक् &इफ्र्याप्क &इफ्र्याप्‍तक  &फ्र्याप्कये  \\ 
अम्अ्य इन् & इफ्र्याऽप् & इफ्र्याप्‍त्य   \\ 
अम्अ्य एऽचि & इफ्र्यपि & इफ्र्यपिति    \\ 
अम्अ्य ए:न् & इफ्र्या:म्‍नि  & इफ्र्याप्‍त्यनु  \\ 
\midrule
उङ इन् & फ्र्या:म्‍न्य  & फ्र्या:म्त्यनि  \\ 
उङ एऽचि & फ्र्या:म्सु  & फ्र्या:म्त्यान्सु   \\ 
उङ ए:न्& फ्र्या:म्‍नु  & फ्र्या:म्त्यान्‍नु   \\ 
\bottomrule
\end{tabular}
\end{table}


\begin{table}[H]
\label{ɛtt.vt} \centering
\caption{सकर्मक क्रिया  र्‍या:न्‍न्य  "रोज्नु"  }
\begin{tabular}{l|l|l|l|l|l|l|l|l|l|l|l|l}  \toprule
&अभूत & भूत & आज्ञार्थक \\ 
उङ &र्‍यात्तु &र्‍यात्त \\ 
उङ अम्सु&र्‍यात्तुसु &र्‍यात्तसु \\ 
उङ अम्ह्‍याम्&र्‍यात्तुनु &र्‍यात्तनु \\ 
इऽचिअ्य &र्‍यचि &र्‍यास्ति   \\ 
ओऽचुअ        &र्‍यचु &र्‍यास्तु   \\ 
इक्अ्य&र्‍याह्इकि &र्‍याह्इतिकि   \\ 
ओक्अ &र्‍याह्इक &र्‍याह्इतक   \\ 
इन्य & इर्‍यात्त्यु  & इर्‍यात्त्य &र्‍यात्ते  \\ 
इन्य अम्सु& इर्‍यात्सु  & इर्‍यात्त्यसु   \\ 
इन्य अम्ह्‍याम्& इर्‍यात्‍नु  & इर्‍यात्त्यनु   \\ 
एऽचि & इर्‍यचि & इर्‍यास्ति &र्‍यचिये    \\ 
ए:न् & इर्‍या:न्‍नि  & इर्‍यास्त्यनु &र्‍यास्‍नुये  \\ 
अम्अ्य & र्‍यात्त्यु  & र्‍यात्त्य  \\ 
अम्सुअ्य & र्‍यात्सु & र्‍यात्त्यसु  \\ 
अम्ह्‍याम्अ्य & र्‍यात्‍नु  & र्‍यात्त्यनु \\ 
\midrule
इन्य, अम्अ्य उ:ङ्‌ &इर्‍या:इङ &इर्‍यास्त &र्‍यचये \\ 
एऽचिअ्य/अम्सुअ उ:ङ्‌ &इर्‍या:इङसु &इर्‍यास्तसु &र्‍यचसुये \\ 
ए:न्अ्य/अम्ह्‍याम्अ्य उ:ङ्‌ &इर्‍या:इङनु &इर्‍यास्तनु &र्‍यचनुये \\ 
इन्य/अम्अ्य इऽचि &इर्‍यचि &इर्‍यास्ति    \\ 
इन्य/अम्अ्य ओऽचु &इर्‍यचु &इर्‍यास्तु  &र्‍यचुये  \\ 
इन्य/अम्अ्य इक् &इर्‍याह्इकि &इर्‍याह्इतिकि   \\ 
इन्य/अम्अ्य ओक् &इर्‍याह्इक &इर्‍याह्इतक  &र्‍याह्इकये  \\ 
अम्अ्य इन् & इर्‍या:इ & इर्‍यास्त्य   \\ 
अम्अ्य एऽचि & इर्‍यचि & इर्‍यास्ति    \\ 
अम्अ्य ए:न् & इर्‍या:न्‍नि  & इर्‍यास्त्यनु  \\ 
\midrule
उङ इन् & र्‍या:न्‍न्य  & र्‍या:न्त्यनि  \\ 
उङ एऽचि & र्‍या:न्सु  & र्‍या:न्त्यान्सु   \\ 
उङ ए:न्& र्‍या:न्‍नु  & र्‍या:न्त्यान्‍नु   \\ 
\bottomrule
\end{tabular}
\end{table}


\begin{table}[H]
\label{akt.vt} \centering
\caption{सकर्मक क्रिया  क्रा:न्य  "बान बसाल्नु"  }
\begin{tabular}{l|l|l|l|l|l|l|l|l|l|l|l|l}  \toprule
&अभूत & भूत & आज्ञार्थक \\ 
उङ &क्राक्‌तु &क्राक्‌त \\ 
उङ अम्सु&क्राक्‌तुसु &क्राक्‌तसु \\ 
उङ अम्ह्‍याम्&क्राक्‌तुनु &क्राक्‌तनु \\ 
इऽचिअ्य &क्रकि &क्रकिति   \\ 
ओऽचुअ        &क्रकु &क्रकुतु   \\ 
इक्अ्य&क्राक्‌कि &क्राक्‌तिकि   \\ 
ओक्अ &क्राक्‌क &क्राक्‌तक   \\ 
इन्य & इक्राक्‌त्यु  & इक्राक्‌त्य &क्राक्‌ते  \\ 
इन्य अम्सु& इक्राक्सु  & इक्राक्‌त्यसु   \\ 
इन्य अम्ह्‍याम्& इक्राक्‍नु  & इक्राक्‌त्यनु   \\ 
एऽचि & इक्रकि & इक्रकिति &क्रकिये    \\ 
ए:न् & इक्रा:नि  & इक्राक्‌त्यनु &क्राक्‍नुये  \\ 
अम्अ्य & क्राक्‌त्यु  & क्राक्‌त्य  \\ 
अम्सुअ्य & क्राक्सु & क्राक्‌त्यसु  \\ 
अम्ह्‍याम्अ्य & क्राक्‍नु  & क्राक्‌त्यनु \\ 
\midrule
इन्य, अम्अ्य उ:ङ्‌ &इक्रा:ङ &इक्रकत &क्रकये \\ 
एऽचिअ्य/अम्सुअ उ:ङ्‌ &इक्रा:ङसु &इक्रकतसु &क्रकसुये \\ 
ए:न्अ्य/अम्ह्‍याम्अ्य उ:ङ्‌ &इक्रा:ङनु &इक्रकतनु &क्रकनुये \\ 
इन्य/अम्अ्य इऽचि &इक्रकि &इक्रकिति    \\ 
इन्य/अम्अ्य ओऽचु &इक्रकु &इक्रकुतु  &क्रकुये  \\ 
इन्य/अम्अ्य इक् &इक्राक्‌कि &इक्राक्‌तिकि   \\ 
इन्य/अम्अ्य ओक् &इक्राक्‌क &इक्राक्‌तक  &क्राक्‌कये  \\ 
अम्अ्य इन् & इक्रा: & इक्राक्‌त्य   \\ 
अम्अ्य एऽचि & इक्रकि & इक्रकिति    \\ 
अम्अ्य ए:न् & इक्रा:नि  & इक्राक्‌त्यनु  \\ 
\midrule
उङ इन् & क्रा:न्य  & क्रा:न्त्यनि  \\ 
उङ एऽचि & क्रा:न्सु  & क्रा:न्त्यान्सु   \\ 
उङ ए:न्& क्रा:न्‍नु  & क्रा:न्त्यान्‍नु   \\ 
\bottomrule
\end{tabular}
\end{table}


\begin{table}[H]
\label{ɛmt.vt} \centering
\caption{सकर्मक क्रिया  घ्र्याम्‍न्य  "घिनाउनु"  }
\begin{tabular}{l|l|l|l|l|l|l|l|l|l|l|l|l}  \toprule
&अभूत & भूत & आज्ञार्थक \\ 
उङ &घ्र्याम्दु &घ्र्या:म्त \\ 
उङ अम्सु&घ्र्याम्दुसु &घ्र्या:म्तसु \\ 
उङ अम्ह्‍याम्&घ्र्याम्दुनु &घ्र्या:म्तनु \\ 
इऽचिअ्य &घ्र्यमि &घ्र्यमिति   \\ 
ओऽचुअ        &घ्र्यमु &घ्र्यमुतु   \\ 
इक्अ्य&घ्र्याम्कि &घ्र्याम्तिकि   \\ 
ओक्अ &घ्र्याम्क &घ्र्याम्तक   \\ 
इन्य & इघ्र्याम्द्‌यु  & इघ्र्या:म्त्य &घ्र्याम्दे  \\ 
इन्य अम्सु& इघ्र्या:म्सु  & इघ्र्या:म्त्यसु   \\ 
इन्य अम्ह्‍याम्& इघ्र्या:म्‍नु  & इघ्र्या:म्त्यनु   \\ 
एऽचि & इघ्र्यमि & इघ्र्यमिति &घ्र्यमिये    \\ 
ए:न् & इघ्र्याम्‍नि  & इघ्र्या:म्त्यनु &घ्र्या:म्‍नुये  \\ 
अम्अ्य & घ्र्याम्द्‌यु  & घ्र्या:म्त्य  \\ 
अम्सुअ्य & घ्र्या:म्सु & घ्र्या:म्त्यसु  \\ 
अम्ह्‍याम्अ्य & घ्र्या:म्‍नु  & घ्र्या:म्त्यनु \\ 
\midrule
इन्य, अम्अ्य उ:ङ्‌ &इघ्र्याम्ङ &इघ्र्यमत &घ्र्यमये \\ 
एऽचिअ्य/अम्सुअ उ:ङ्‌ &इघ्र्याम्ङसु &इघ्र्यमतसु &घ्र्यमसुये \\ 
ए:न्अ्य/अम्ह्‍याम्अ्य उ:ङ्‌ &इघ्र्याम्ङनु &इघ्र्यमतनु &घ्र्यमनुये \\ 
इन्य/अम्अ्य इऽचि &इघ्र्यमि &इघ्र्यमिति    \\ 
इन्य/अम्अ्य ओऽचु &इघ्र्यमु &इघ्र्यमुतु  &घ्र्यमुये  \\ 
इन्य/अम्अ्य इक् &इघ्र्याम्कि &इघ्र्याम्तिकि   \\ 
इन्य/अम्अ्य ओक् &इघ्र्याम्क &इघ्र्याम्तक  &घ्र्याम्कये  \\ 
अम्अ्य इन् & इघ्र्याम् & इघ्र्या:म्त्य   \\ 
अम्अ्य एऽचि & इघ्र्यमि & इघ्र्यमिति    \\ 
अम्अ्य ए:न् & इघ्र्याम्‍नि  & इघ्र्या:म्त्यनु  \\ 
\midrule
उङ इन् & घ्र्याम्‍न्य  & घ्र्याम्त्यनि  \\ 
उङ एऽचि & घ्र्या:म्सु  & घ्र्याम्त्यान्सु   \\ 
उङ ए:न्& घ्र्या:म्‍नु  & घ्र्याम्त्यान्‍नु   \\ 
\bottomrule
\end{tabular}
\end{table}


\begin{table}[H]
\label{ɛnt.vt} \centering
\caption{सकर्मक क्रिया  प्याइन्य  "झम्टनु"  }
\begin{tabular}{l|l|l|l|l|l|l|l|l|l|l|l|l}  \toprule
&अभूत & भूत & आज्ञार्थक \\ 
उङ &प्यान्दु &प्या:न्त \\ 
उङ अम्सु&प्यान्दुसु &प्या:न्तसु \\ 
उङ अम्ह्‍याम्&प्यान्दुनु &प्या:न्तनु \\ 
इऽचिअ्य &प्याचि &प्याऽस्ति   \\ 
ओऽचुअ        &प्याचु &प्याऽस्तु   \\ 
इक्अ्य&प्याइकि &प्याइतिकि   \\ 
ओक्अ &प्याइक &प्याइतक   \\ 
इन्य & इप्यान्द्‌यु  & इप्या:न्त्य &प्यान्दे  \\ 
इन्य अम्सु& इप्या:न्सु  & इप्या:न्त्यसु   \\ 
इन्य अम्ह्‍याम्& इप्या:न्‍नु  & इप्या:न्त्यनु   \\ 
एऽचि & इप्याचि & इप्याऽस्ति &प्याचिये    \\ 
ए:न् & इप्याइनि  & इप्याऽस्त्यनु &प्याऽस्‍नुये  \\ 
अम्अ्य & प्यान्द्‌यु  & प्या:न्त्य  \\ 
अम्सुअ्य & प्या:न्सु & प्या:न्त्यसु  \\ 
अम्ह्‍याम्अ्य & प्या:न्‍नु  & प्या:न्त्यनु \\ 
\midrule
इन्य, अम्अ्य उ:ङ्‌ &इप्याइङ &इप्याऽस्त &प्याचये \\ 
एऽचिअ्य/अम्सुअ उ:ङ्‌ &इप्याइङसु &इप्याऽस्तसु &प्याचसुये \\ 
ए:न्अ्य/अम्ह्‍याम्अ्य उ:ङ्‌ &इप्याइङनु &इप्याऽस्तनु &प्याचनुये \\ 
इन्य/अम्अ्य इऽचि &इप्याचि &इप्याऽस्ति    \\ 
इन्य/अम्अ्य ओऽचु &इप्याचु &इप्याऽस्तु  &प्याचुये  \\ 
इन्य/अम्अ्य इक् &इप्याइकि &इप्याइतिकि   \\ 
इन्य/अम्अ्य ओक् &इप्याइक &इप्याइतक  &प्याइकये  \\ 
अम्अ्य इन् & इप्याइ & इप्याऽस्त्य   \\ 
अम्अ्य एऽचि & इप्याचि & इप्याऽस्ति    \\ 
अम्अ्य ए:न् & इप्याइनि  & इप्याऽस्त्यनु  \\ 
\midrule
उङ इन् & प्याइन्य  & प्याइत्यनि  \\ 
उङ एऽचि & प्या:इसु  & प्याइत्यान्सु   \\ 
उङ ए:न्& प्या:इनु  & प्याइत्यान्‍नु   \\ 
\bottomrule
\end{tabular}
\end{table}


\begin{table}[H]
\label{aŋt.vt} \centering
\caption{सकर्मक क्रिया  कान्य  "बसाल्नु"  }
\begin{tabular}{l|l|l|l|l|l|l|l|l|l|l|l|l}  \toprule
&अभूत & भूत & आज्ञार्थक \\ 
उङ &कान्दु &का:न्त \\ 
उङ म्यसु &कान्दुसु &का:न्तसु \\ 
उङ म्यह्‍याम् &कान्दुनु &का:न्तनु \\ 
इऽचिअ्य &कङि &कङिति   \\ 
ओऽचुअ &कङु &कङुतु   \\ 
इक्अ्य &काङ्‌कि &काङ्‌तिकि   \\ 
ओक्अ &काङ्‌क &काङ्‌तक   \\ 
इन्य अम् & इकान्द्‌यु  & इका:न्त्य &कान्दे  \\ 
इन्य म्यसु & इका:न्सु  & इका:न्त्यसु   \\ 
इन्य म्यह्‍याम् & इका:न्‍नु  & इका:न्त्यनु   \\ 
एऽचिअ्य & इकङि & इकङिति &कङिये    \\ 
ए:न्अ्य & इकानि  & इका:ङ्‌त्यनु &का:ङ्‌नुये  \\ 
अम्अ्य & कान्द्‌यु  & का:न्त्य  \\ 
अम्सुअ & का:न्सु & का:न्त्यसु  \\ 
अम्ह्‍याम्अ्य & का:न्‍नु  & का:न्त्यनु \\ 
\bottomrule
\end{tabular}
\end{table}


\begin{table}[H]
\label{e.vi} \centering
\caption{अकर्मक क्रिया  जेन्य  "बोल्नु"  }
\begin{tabular}{l|l|l|l|l|l|l|l|l|l|l|l|l}  \toprule
&अभूत & भूत & आज्ञार्थक \\ 
उ:ङ्‌ &जेङ &जेङत \\ 
इऽचि &जेयि &जे:इति   \\ 
ओऽचु &जेयु &जे:इतु   \\ 
इक् &जेकि &जेक्‌तिकि   \\ 
ओक् &जेक &जेक्‌तक   \\ 
इन् & इजे & इजेऽत्य &जेऽये  \\ 
एऽचि & इजेयि & इजे:इति &जे:इये    \\ 
ए:न् & इजेनि  & इजेत्‍नु &जे:न्ये  \\ 
अम् & जे & जेऽत्य   \\ 
अम्सु & जेयि & जे:इति     \\ 
अम्ह्‍याम् & जेनु  & जेत्‍नु \\ 
\bottomrule
\end{tabular}
\end{table}


\begin{table}[H]
\label{ep.vi} \centering
\caption{अकर्मक क्रिया  रे:म्‍न्य  "उभिनु"  }
\begin{tabular}{l|l|l|l|l|l|l|l|l|l|l|l|l}  \toprule
&अभूत & भूत & आज्ञार्थक \\ 
उ:ङ्‌ &रे:म्ङ &रेपत \\ 
इऽचि &रेपि &रेपिति   \\ 
ओऽचु &रेपु &रेपुतु   \\ 
इक् &रेप्कि &रेप्‍तिकि   \\ 
ओक् &रेप्क &रेप्‍तक   \\ 
इन् & इरेऽप् & इरेप्‍त्य &रेप्ये  \\ 
एऽचि & इरेपि & इरेपिति &रेपिये    \\ 
ए:न् & इरे:म्‍नि  & इरेप्‍त्यनु &रेप्‍नुये  \\ 
अम् & रेऽप् & रेप्‍त्य   \\ 
अम्सु & रेपि & रेपिति   \\ 
अम्ह्‍याम् & रे:म्‍नु  & रेप्‍त्यनु \\ 
\bottomrule
\end{tabular}
\end{table}


\begin{table}[H]
\label{et.vi} \centering
\caption{अकर्मक क्रिया  ङे:न्‍न्य  "दुख्‍नु"  }
\begin{tabular}{l|l|l|l|l|l|l|l|l|l|l|l|l}  \toprule
&अभूत & भूत & आज्ञार्थक \\ 
उ:ङ्‌ &ङे:इङ &ङेस्त \\ 
इऽचि &ङेचि &ङेस्ति   \\ 
ओऽचु &ङेचु &ङेस्तु   \\ 
इक् &ङेह्इकि &ङेह्इतिकि   \\ 
ओक् &ङेह्इक &ङेह्इतक   \\ 
इन् & इङे:इ & इङेस्त्य &ङेच्‍चे  \\ 
एऽचि & इङेचि & इङेस्ति &ङेचिये    \\ 
ए:न् & इङे:न्‍नि  & इङेस्त्यनु &ङेस्‍नुये  \\ 
अम् & ङे:इ & ङेस्त्य   \\ 
अम्सु & ङेचि & ङेस्ति   \\ 
अम्ह्‍याम् & ङे:न्‍नु  & ङेस्त्यनु \\ 
\bottomrule
\end{tabular}
\end{table}


\begin{table}[H]
\label{ek.vi} \centering
\caption{अकर्मक क्रिया  चे:न्य  "चिच्चा गर्नु"  }
\begin{tabular}{l|l|l|l|l|l|l|l|l|l|l|l|l}  \toprule
&अभूत & भूत & आज्ञार्थक \\ 
उ:ङ्‌ &चे:ङ &चेकत \\ 
इऽचि &चेकि &चेकिति   \\ 
ओऽचु &चेकु &चेकुतु   \\ 
इक् &चेक्‌कि &चेक्‌तिकि   \\ 
ओक् &चेक्‌क &चेक्‌तक   \\ 
इन् & इचे: & इचेक्‌त्य &चेक्ये  \\ 
एऽचि & इचेकि & इचेकिति &चेकिये    \\ 
ए:न् & इचे:नि  & इचेक्‌त्यनु &चेक्‍नुये  \\ 
अम् & चे: & चेक्‌त्य   \\ 
अम्सु & चेकि & चेकिति   \\ 
अम्ह्‍याम् & चे:नु  & चेक्‌त्यनु \\ 
\bottomrule
\end{tabular}
\end{table}


\begin{table}[H]
\label{em.vi} \centering
\caption{अकर्मक क्रिया  येम्‍न्य  "बाली उठाउँन ढीलो हुनु"  }
\begin{tabular}{l|l|l|l|l|l|l|l|l|l|l|l|l}  \toprule
&अभूत & भूत   \\ 
म्य & येम् & ये:म्त्य   \\ 
\bottomrule
\end{tabular}
\end{table}


\begin{table}[H]
\label{en.vi} \centering
\caption{अकर्मक क्रिया  लेइन्य  "निस्कनु"  }
\begin{tabular}{l|l|l|l|l|l|l|l|l|l|l|l|l}  \toprule
&अभूत & भूत & आज्ञार्थक \\ 
उ:ङ्‌ &लेइङ &लेऽस्त \\ 
इऽचि &लेऽचि &लेऽस्ति   \\ 
ओऽचु &लेऽचु &लेऽस्तु   \\ 
इक् &लेइकि &लेइतिकि   \\ 
ओक् &लेइक &लेइतक   \\ 
इन् & इलेइ & इलेऽस्त्य &लेऽचे  \\ 
एऽचि & इलेऽचि & इलेऽस्ति &लेऽचिये    \\ 
ए:न् & इलेइनि  & इलेऽस्त्यनु &लेऽस्‍नुये  \\ 
अम् & लेइ & लेऽस्त्य   \\ 
अम्सु & लेऽचि & लेऽस्ति   \\ 
अम्ह्‍याम् & लेइनु  & लेऽस्त्यनु \\ 
\bottomrule
\end{tabular}
\end{table}


\begin{table}[H]
\label{eŋ.vi} \centering
\caption{अकर्मक क्रिया  भ्रेऽन्य  "अल्छी गर्नु"  }
\begin{tabular}{l|l|l|l|l|l|l|l|l|l|l|l|l}  \toprule
&अभूत & भूत & आज्ञार्थक \\ 
उ:ङ्‌ &भ्रेऽङ &भ्रेङत \\ 
इऽचि &भ्रेङि &भ्रेङिति   \\ 
ओऽचु &भ्रेङु &भ्रेङुतु   \\ 
इक् &भ्रेङ्‌कि &भ्रेङ्‌तिकि   \\ 
ओक् &भ्रेङ्‌क &भ्रेङ्‌तक   \\ 
इन् & इभ्रेङ्‌ & इभ्रे:ङ्‌त्य &भ्रे:ङ्‌ये  \\ 
एऽचि & इभ्रेङि & इभ्रेङिति &भ्रेङिये    \\ 
ए:न् & इभ्रेऽनि  & इभ्रे:ङ्‌त्यनु &भ्रे:ङ्‌नुये  \\ 
अम् & भ्रेङ्‌ & भ्रे:ङ्‌त्य   \\ 
अम्सु & भ्रेङि & भ्रेङिति   \\ 
अम्ह्‍याम् & भ्रेऽनु  & भ्रे:ङ्‌त्यनु \\ 
\bottomrule
\end{tabular}
\end{table}


\begin{table}[H]
\label{er.vi} \centering
\caption{अकर्मक क्रिया  भेर्न्य  "उड्नु"  }
\begin{tabular}{l|l|l|l|l|l|l|l|l|l|l|l|l}  \toprule
&अभूत & भूत & आज्ञार्थक \\ 
उ:ङ्‌ &भेर्ङ &भेरत \\ 
इऽचि &भेरि &भेरिति   \\ 
ओऽचु &भेरु &भेरुतु   \\ 
इक् &भेर्कि &भेर्तिकि   \\ 
ओक् &भेर्क &भेर्तक   \\ 
इन् & इभेर् & इभे:र्त्य &भे:र्‍ये  \\ 
एऽचि & इभेरि & इभेरिति &भेरिये    \\ 
ए:न् & इभेर्नि  & इभे:र्त्यनु &भे:र्नुये  \\ 
अम् & भेर् & भे:र्त्य   \\ 
अम्सु & भेरि & भेरिति   \\ 
अम्ह्‍याम् & भेर्नु  & भे:र्त्यनु \\ 
\bottomrule
\end{tabular}
\end{table}


\begin{table}[H]
\label{el.vi} \centering
\caption{अकर्मक क्रिया  एल्न्य  "पोख्‍नु"  }
\begin{tabular}{l|l|l|l|l|l|l|l|l|l|l|l|l}  \toprule
&अभूत & भूत   \\ 
म्य & एल् & ए:ल्त्य   \\ 
\bottomrule
\end{tabular}
\end{table}


\begin{table}[H]
\label{ep.vt} \centering
\caption{सकर्मक क्रिया  ख्ले:म्‍न्य  "ताछ्नु"  }
\begin{tabular}{l|l|l|l|l|l|l|l|l|l|l|l|l}  \toprule
&अभूत & भूत & आज्ञार्थक \\ 
उङ म्य &ख्लेबु &ख्लेबुत \\ 
उङ म्यसु &ख्लेबुसु &ख्लेबुतसु \\ 
उङ म्यह्‍याम् &ख्लेबुनु &ख्लेबुतनु \\ 
इऽचिअ्य  &ख्लेपि &ख्लेपिति   \\ 
ओऽचुअ &ख्लेपु &ख्लेपुतु   \\ 
इक्अ्य &ख्लेप्कि &ख्लेप्‍तिकि   \\ 
ओक्अ &ख्लेप्क &ख्लेप्‍तक   \\ 
इन्य म्य& इख्लेऽब्यु  & इख्लेऽप्‍त्य &ख्लेऽबे  \\ 
इन्य म्यसु & इख्लेऽप्सु  & इख्लेऽप्‍त्यसु   \\ 
इन्य म्यह्‍याम् & इख्लेऽप्‍नु  & इख्लेऽप्‍त्यनु   \\ 
एऽचिअ्य & इख्लेपि & इख्लेपिति &ख्लेपिये    \\ 
ए:न्अ्य & इख्ले:म्‍नि  & इख्लेप्‍त्यनु &ख्लेप्‍नुये  \\ 
अम्अ्य & ख्लेऽब्यु  & ख्लेऽप्‍त्य  \\ 
अम्सुअ & ख्लेऽप्सु & ख्लेऽप्‍त्यसु  \\ 
अम्ह्‍याम्अ्य & ख्लेऽप्‍नु  & ख्लेऽप्‍त्यनु \\ 
\bottomrule
\end{tabular}
\end{table}


\begin{table}[H]
\label{et.vt} \centering
\caption{सकर्मक क्रिया  से:न्‍न्य  "मार्नु"  }
\begin{tabular}{l|l|l|l|l|l|l|l|l|l|l|l|l}  \toprule
&अभूत & भूत & आज्ञार्थक \\ 
उङ &सेदु &से:त \\ 
उङ अम्सु &सेदुसु &से:तसु \\ 
उङ अम्ह्‍याम् &सेदुनु &से:तनु \\ 
इऽचिअ्य &सेचि &सेस्ति   \\ 
ओऽचुअ &सेचु &सेस्तु   \\ 
इक्अ्य &सेह्इकि &सेह्इतिकि   \\ 
ओक्अ &सेह्इक &सेह्इतक   \\ 
इन्य अम् & इसेऽद्‌यु  & इसे:त्य &सेऽदे  \\ 
इन्य अम्सु & इसेऽत्सु  & इसे:त्यसु   \\ 
इन्य अम्ह्‍याम् & इसेऽत्‍नु  & इसे:त्यनु   \\ 
एऽचिअ्य & इसेचि & इसेस्ति &सेचिये    \\ 
ए:न्अ्य & इसे:न्‍नि  & इसेस्त्यनु &सेस्‍नुये  \\ 
अम्अ्य & सेऽद्‌यु  & से:त्य  \\ 
अम्सुअ & सेऽत्सु & से:त्यसु  \\ 
अम्ह्‍याम्अ्य & सेऽत्‍नु  & से:त्यनु \\ 
\midrule
इन्य/अम्अ्य उ:ङ्‌&इसे:इङ & इसेस्त &सेचये \\ 
एऽचिअ्य/अम्सुअ उ:ङ्‌ &इसे:इङसु & इसेस्तसु &सेचसुये \\ 
ए:न्अ्य/अम्ह्‍याम्अ्य उ:ङ्‌ &इसे:इङनु & इसेस्तनु &सेचनुये \\ 
इन्य/अम्अ्य इऽचि & इसेचि & इसेस्ति    \\ 
इन्य/अम्अ्य ओऽचु & इसेचु & इसेस्तु  &सेचुये  \\ 
इन्य/अम्अ्य इक् & इसेह्इकि & इसेह्इतिकि   \\ 
इन्य/अम्अ्य ओक् & इसेह्इक & इसेह्इतक  &सेह्इकये  \\ 
अम्अ्य इन् & इसे:इ & इसेस्त्य   \\ 
अम्अ्य एऽचि & इसेचि & इसेस्ति    \\ 
अम्अ्य ए:न् & इसे:न्‍नि  & इसेस्त्यनु  \\ 
\midrule
उङ इन् & से:न्‍न्य  & से:न्त्यनि  \\ 
उङ एऽचि & से:न्सु  & से:न्त्यान्सु   \\ 
उङ ए:न्& से:न्‍नु  & से:न्त्यान्‍नु   \\ 
\bottomrule
\end{tabular}
\end{table}


\begin{table}[H]
\label{ek.vt} \centering
\caption{सकर्मक क्रिया  धे:न्य  "पुछ्नु"  }
\begin{tabular}{l|l|l|l|l|l|l|l|l|l|l|l|l}  \toprule
&अभूत & भूत & आज्ञार्थक \\ 
उङ म्य &धेगु &धेगुत \\ 
उङ म्यसु &धेगुसु &धेगुतसु \\ 
उङ म्यह्‍याम् &धेगुनु &धेगुतनु \\ 
इऽचिअ्य  &धेकि &धेकिति   \\ 
ओऽचुअ &धेकु &धेकुतु   \\ 
इक्अ्य &धेक्‌कि &धेक्‌तिकि   \\ 
ओक्अ &धेक्‌क &धेक्‌तक   \\ 
इन्य म्य& इधेऽग्यु  & इधेऽक्‌त्य &धेऽगे  \\ 
इन्य म्यसु & इधेऽक्सु  & इधेऽक्‌त्यसु   \\ 
इन्य म्यह्‍याम् & इधेऽक्‍नु  & इधेऽक्‌त्यनु   \\ 
एऽचिअ्य & इधेकि & इधेकिति &धेकिये    \\ 
ए:न्अ्य & इधे:नि  & इधेक्‌त्यनु &धेक्‍नुये  \\ 
अम्अ्य & धेऽग्यु  & धेऽक्‌त्य  \\ 
अम्सुअ & धेऽक्सु & धेऽक्‌त्यसु  \\ 
अम्ह्‍याम्अ्य & धेऽक्‍नु  & धेऽक्‌त्यनु \\ 
\bottomrule
\end{tabular}
\end{table}


\begin{table}[H]
\label{em.vt} \centering
\caption{सकर्मक क्रिया  फ्लेम्‍न्य  "कुल्चनु"  }
\begin{tabular}{l|l|l|l|l|l|l|l|l|l|l|l|l}  \toprule
&अभूत & भूत & आज्ञार्थक \\ 
उङ &फ्लेमु &फ्लेमुत \\ 
उङ अम्सु &फ्लेमुसु &फ्लेमुतसु \\ 
उङ अम्ह्‍याम् &फ्लेमुनु &फ्लेमुतनु \\ 
इऽचिअ्य &फ्लेमि &फ्लेमिति   \\ 
ओऽचुअ &फ्लेमु &फ्लेमुतु   \\ 
इक्अ्य &फ्लेम्कि &फ्लेम्तिकि   \\ 
ओक्अ &फ्लेम्क &फ्लेम्तक   \\ 
इन्य अम् & इफ्लेऽम्यु  & इफ्ले:म्त्य &फ्लेऽमे  \\ 
इन्य अम्सु & इफ्ले:म्सु  & इफ्ले:म्त्यसु   \\ 
इन्य अम्ह्‍याम् & इफ्ले:म्‍नु  & इफ्ले:म्त्यनु   \\ 
एऽचिअ्य & इफ्लेमि & इफ्लेमिति &फ्लेमिये    \\ 
ए:न्अ्य & इफ्लेम्‍नि  & इफ्ले:म्त्यनु &फ्ले:म्‍नुये  \\ 
अम्अ्य & फ्लेऽम्यु  & फ्ले:म्त्य  \\ 
अम्सुअ & फ्ले:म्सु & फ्ले:म्त्यसु  \\ 
अम्ह्‍याम्अ्य & फ्ले:म्‍नु  & फ्ले:म्त्यनु \\ 
\midrule
इन्य/अम्अ्य उ:ङ्‌&इफ्लेम्ङ & इफ्लेमत &फ्लेमये \\ 
एऽचिअ्य/अम्सुअ उ:ङ्‌ &इफ्लेम्ङसु & इफ्लेमतसु &फ्लेमसुये \\ 
ए:न्अ्य/अम्ह्‍याम्अ्य उ:ङ्‌ &इफ्लेम्ङनु & इफ्लेमतनु &फ्लेमनुये \\ 
इन्य/अम्अ्य इऽचि & इफ्लेमि & इफ्लेमिति    \\ 
इन्य/अम्अ्य ओऽचु & इफ्लेमु & इफ्लेमुतु  &फ्लेमुये  \\ 
इन्य/अम्अ्य इक् & इफ्लेम्कि & इफ्लेम्तिकि   \\ 
इन्य/अम्अ्य ओक् & इफ्लेम्क & इफ्लेम्तक  &फ्लेम्कये  \\ 
अम्अ्य इन् & इफ्लेम् & इफ्ले:म्त्य   \\ 
अम्अ्य एऽचि & इफ्लेमि & इफ्लेमिति    \\ 
अम्अ्य ए:न् & इफ्लेम्‍नि  & इफ्ले:म्त्यनु  \\ 
\midrule
उङ इन् & फ्लेम्‍न्य  & फ्लेम्त्यनि  \\ 
उङ एऽचि & फ्ले:म्सु  & फ्लेम्त्यान्सु   \\ 
उङ ए:न्& फ्ले:म्‍नु  & फ्लेम्त्यान्‍नु   \\ 
\bottomrule
\end{tabular}
\end{table}


\begin{table}[H]
\label{e.vt} \centering
\caption{सकर्मक क्रिया  घ्लेन्य  "पूज्नु"  }
\begin{tabular}{l|l|l|l|l|l|l|l|l|l|l|l|l}  \toprule
&अभूत & भूत & आज्ञार्थक \\ 
उ:ङ्‌ &घ्लेङ &घ्ले:ङ्‌त \\ 
इऽचि &घ्लेयि &घ्ले:इति   \\ 
ओऽचु &घ्लेयु &घ्ले:इतु   \\ 
इक् &घ्लेकि &घ्लेक्‌तिकि   \\ 
ओक् &घ्लेक &घ्लेक्‌तक   \\ 
इन् & इघ्ले & इघ्लेत्य &घ्लेये  \\ 
एऽचि & इघ्लेयि & इघ्ले:इति &घ्ले:इये    \\ 
ए:न् & इघ्लेनि  & इघ्लेत्‍नु &घ्ले:न्ये  \\ 
अम् & घ्ले & घ्लेत्य   \\ 
अम्सु & घ्लेसु & घ्लेत्सु     \\ 
अम्ह्‍याम् & घ्लेनु  & घ्लेत्‍नु \\ 
\bottomrule
\end{tabular}
\end{table}


\begin{table}[H]
\label{eŋ.vt} \centering
\caption{सकर्मक क्रिया  छेऽन्य  "घृणा गर्नु"  }
\begin{tabular}{l|l|l|l|l|l|l|l|l|l|l|l|l}  \toprule
&अभूत & भूत & आज्ञार्थक \\ 
उङ &छेङु &छेङुत \\ 
उङ अम्सु &छेङुसु &छेङुतसु \\ 
उङ अम्ह्‍याम् &छेङुनु &छेङुतनु \\ 
इऽचिअ्य &छेङि &छेङिति   \\ 
ओऽचुअ &छेङु &छेङुतु   \\ 
इक्अ्य &छेङ्‌कि &छेङ्‌तिकि   \\ 
ओक्अ &छेङ्‌क &छेङ्‌तक   \\ 
इन्य अम् & इछेऽङ्‌यु  & इछे:ङ्‌त्य &छेऽङे  \\ 
इन्य अम्सु & इछे:ङ्‌सु  & इछे:ङ्‌त्यसु   \\ 
इन्य अम्ह्‍याम् & इछे:ङ्‌नु  & इछे:ङ्‌त्यनु   \\ 
एऽचिअ्य & इछेङि & इछेङिति &छेङिये    \\ 
ए:न्अ्य & इछेऽनि  & इछे:ङ्‌त्यनु &छे:ङ्‌नुये  \\ 
अम्अ्य & छेऽङ्‌यु  & छे:ङ्‌त्य  \\ 
अम्सुअ & छे:ङ्‌सु & छे:ङ्‌त्यसु  \\ 
अम्ह्‍याम्अ्य & छे:ङ्‌नु  & छे:ङ्‌त्यनु \\ 
\midrule
इन्य/अम्अ्य उ:ङ्‌&इछेऽङ & इछेङत &छेङये \\ 
एऽचिअ्य/अम्सुअ उ:ङ्‌ &इछेऽङसु & इछेङतसु &छेङसुये \\ 
ए:न्अ्य/अम्ह्‍याम्अ्य उ:ङ्‌ &इछेऽङनु & इछेङतनु &छेङनुये \\ 
इन्य/अम्अ्य इऽचि & इछेङि & इछेङिति    \\ 
इन्य/अम्अ्य ओऽचु & इछेङु & इछेङुतु  &छेङुये  \\ 
इन्य/अम्अ्य इक् & इछेङ्‌कि & इछेङ्‌तिकि   \\ 
इन्य/अम्अ्य ओक् & इछेङ्‌क & इछेङ्‌तक  &छेङ्‌कये  \\ 
अम्अ्य इन् & इछेङ्‌ & इछे:ङ्‌त्य   \\ 
अम्अ्य एऽचि & इछेङि & इछेङिति    \\ 
अम्अ्य ए:न् & इछेऽनि  & इछे:ङ्‌त्यनु  \\ 
\midrule
उङ इन् & छेऽन्य  & छेन्त्यनि  \\ 
उङ एऽचि & छे:न्सु  & छेन्त्यान्सु   \\ 
उङ ए:न्& छे:न्‍नु  & छेन्त्यान्‍नु   \\ 
\bottomrule
\end{tabular}
\end{table}


\begin{table}[H]
\label{er.vt} \centering
\caption{सकर्मक क्रिया  नेर्न्य  "सिद्धयाउनु"  }
\begin{tabular}{l|l|l|l|l|l|l|l|l|l|l|l|l}  \toprule
&अभूत & भूत & आज्ञार्थक \\ 
उङ &नेरु &नेरुत \\ 
उङ अम्सु &नेरुसु &नेरुतसु \\ 
उङ अम्ह्‍याम् &नेरुनु &नेरुतनु \\ 
इऽचिअ्य &नेरि &नेरिति   \\ 
ओऽचुअ &नेरु &नेरुतु   \\ 
इक्अ्य &नेर्कि &नेर्तिकि   \\ 
ओक्अ &नेर्क &नेर्तक   \\ 
इन्य अम् & इनेऽर्‍यु  & इने:र्त्य &नेऽरे  \\ 
इन्य अम्सु & इने:र्सु  & इने:र्त्यसु   \\ 
इन्य अम्ह्‍याम् & इने:र्नु  & इने:र्त्यनु   \\ 
एऽचिअ्य & इनेरि & इनेरिति &नेरिये    \\ 
ए:न्अ्य & इनेर्नि  & इने:र्त्यनु &ने:र्नुये  \\ 
अम्अ्य & नेऽर्‍यु  & ने:र्त्य  \\ 
अम्सुअ & ने:र्सु & ने:र्त्यसु  \\ 
अम्ह्‍याम्अ्य & ने:र्नु  & ने:र्त्यनु \\ 
\midrule
इन्य/अम्अ्य उ:ङ्‌&इनेर्ङ & इनेरत &नेरये \\ 
एऽचिअ्य/अम्सुअ उ:ङ्‌ &इनेर्ङसु & इनेरतसु &नेरसुये \\ 
ए:न्अ्य/अम्ह्‍याम्अ्य उ:ङ्‌ &इनेर्ङनु & इनेरतनु &नेरनुये \\ 
इन्य/अम्अ्य इऽचि & इनेरि & इनेरिति    \\ 
इन्य/अम्अ्य ओऽचु & इनेरु & इनेरुतु  &नेरुये  \\ 
इन्य/अम्अ्य इक् & इनेर्कि & इनेर्तिकि   \\ 
इन्य/अम्अ्य ओक् & इनेर्क & इनेर्तक  &नेर्कये  \\ 
अम्अ्य इन् & इनेर् & इने:र्त्य   \\ 
अम्अ्य एऽचि & इनेरि & इनेरिति    \\ 
अम्अ्य ए:न् & इनेर्नि  & इने:र्त्यनु  \\ 
\midrule
उङ इन् & नेर्न्य  & नेर्त्यनि  \\ 
उङ एऽचि & ने:र्सु  & नेर्त्यान्सु   \\ 
उङ ए:न्& ने:र्नु  & नेर्त्यान्‍नु   \\ 
\bottomrule
\end{tabular}
\end{table}


\begin{table}[H]
\label{el.vt} \centering
\caption{सकर्मक क्रिया  थेल्न्य  "उप्किनु"  }
\begin{tabular}{l|l|l|l|l|l|l|l|l|l|l|l|l}  \toprule
&अभूत & भूत & आज्ञार्थक \\ 
उङ म्य &थेलु &थेलुत \\ 
उङ म्यसु &थेलुसु &थेलुतसु \\ 
उङ म्यह्‍याम् &थेलुनु &थेलुतनु \\ 
इऽचिअ्य  &थेलि &थेलिति   \\ 
ओऽचुअ &थेलु &थेलुतु   \\ 
इक्अ्य &थेल्कि &थेल्तिकि   \\ 
ओक्अ &थेल्क &थेल्तक   \\ 
इन्य म्य& इथेऽल्यु  & इथे:ल्त्य &थेऽले  \\ 
इन्य म्यसु & इथे:ल्सु  & इथे:ल्त्यसु   \\ 
इन्य म्यह्‍याम् & इथे:ल्नु  & इथे:ल्त्यनु   \\ 
एऽचिअ्य & इथेलि & इथेलिति &थेलिये    \\ 
ए:न्अ्य & इथेल्नि  & इथे:ल्त्यनु &थे:ल्नुये  \\ 
अम्अ्य & थेऽल्यु  & थे:ल्त्य  \\ 
अम्सुअ & थे:ल्सु & थे:ल्त्यसु  \\ 
अम्ह्‍याम्अ्य & थे:ल्नु  & थे:ल्त्यनु \\ 
\bottomrule
\end{tabular}
\end{table}


\begin{table}[H]
\label{ept.vt} \centering
\caption{सकर्मक क्रिया  के:म्‍न्य  "चढ्नु, चिल्नु"  }
\begin{tabular}{l|l|l|l|l|l|l|l|l|l|l|l|l}  \toprule
&अभूत & भूत & आज्ञार्थक \\ 
उङ &केप्‍तु &केप्‍त \\ 
उङ अम्सु&केप्‍तुसु &केप्‍तसु \\ 
उङ अम्ह्‍याम्&केप्‍तुनु &केप्‍तनु \\ 
इऽचिअ्य &केपि &केपिति   \\ 
ओऽचुअ        &केपु &केपुतु   \\ 
इक्अ्य&केप्कि &केप्‍तिकि   \\ 
ओक्अ &केप्क &केप्‍तक   \\ 
इन्य & इकेप्‍त्यु  & इकेप्‍त्य &केप्‍ते  \\ 
इन्य अम्सु& इकेप्सु  & इकेप्‍त्यसु   \\ 
इन्य अम्ह्‍याम्& इकेप्‍नु  & इकेप्‍त्यनु   \\ 
एऽचि & इकेपि & इकेपिति &केपिये    \\ 
ए:न् & इके:म्‍नि  & इकेप्‍त्यनु &केप्‍नुये  \\ 
अम्अ्य & केप्‍त्यु  & केप्‍त्य  \\ 
अम्सुअ्य & केप्सु & केप्‍त्यसु  \\ 
अम्ह्‍याम्अ्य & केप्‍नु  & केप्‍त्यनु \\ 
\midrule
इन्य, अम्अ्य उ:ङ्‌ &इके:म्ङ &इकेपत &केपये \\ 
एऽचिअ्य/अम्सुअ उ:ङ्‌ &इके:म्ङसु &इकेपतसु &केपसुये \\ 
ए:न्अ्य/अम्ह्‍याम्अ्य उ:ङ्‌ &इके:म्ङनु &इकेपतनु &केपनुये \\ 
इन्य/अम्अ्य इऽचि &इकेपि &इकेपिति    \\ 
इन्य/अम्अ्य ओऽचु &इकेपु &इकेपुतु  &केपुये  \\ 
इन्य/अम्अ्य इक् &इकेप्कि &इकेप्‍तिकि   \\ 
इन्य/अम्अ्य ओक् &इकेप्क &इकेप्‍तक  &केप्कये  \\ 
अम्अ्य इन् & इकेऽप् & इकेप्‍त्य   \\ 
अम्अ्य एऽचि & इकेपि & इकेपिति    \\ 
अम्अ्य ए:न् & इके:म्‍नि  & इकेप्‍त्यनु  \\ 
\midrule
उङ इन् & के:म्‍न्य  & के:म्त्यनि  \\ 
उङ एऽचि & के:म्सु  & के:म्त्यान्सु   \\ 
उङ ए:न्& के:म्‍नु  & के:म्त्यान्‍नु   \\ 
\bottomrule
\end{tabular}
\end{table}


\begin{table}[H]
\label{ett.vt} \centering
\caption{सकर्मक क्रिया  रे:न्‍न्य  "हास्नु"  }
\begin{tabular}{l|l|l|l|l|l|l|l|l|l|l|l|l}  \toprule
&अभूत & भूत & आज्ञार्थक \\ 
उङ &रेत्तु &रेत्त \\ 
उङ अम्सु&रेत्तुसु &रेत्तसु \\ 
उङ अम्ह्‍याम्&रेत्तुनु &रेत्तनु \\ 
इऽचिअ्य &रेचि &रेस्ति   \\ 
ओऽचुअ        &रेचु &रेस्तु   \\ 
इक्अ्य&रेह्इकि &रेह्इतिकि   \\ 
ओक्अ &रेह्इक &रेह्इतक   \\ 
इन्य & इरेत्त्यु  & इरेत्त्य &रेत्ते  \\ 
इन्य अम्सु& इरेत्सु  & इरेत्त्यसु   \\ 
इन्य अम्ह्‍याम्& इरेत्‍नु  & इरेत्त्यनु   \\ 
एऽचि & इरेचि & इरेस्ति &रेचिये    \\ 
ए:न् & इरे:न्‍नि  & इरेस्त्यनु &रेस्‍नुये  \\ 
अम्अ्य & रेत्त्यु  & रेत्त्य  \\ 
अम्सुअ्य & रेत्सु & रेत्त्यसु  \\ 
अम्ह्‍याम्अ्य & रेत्‍नु  & रेत्त्यनु \\ 
\midrule
इन्य, अम्अ्य उ:ङ्‌ &इरे:इङ &इरेस्त &रेचये \\ 
एऽचिअ्य/अम्सुअ उ:ङ्‌ &इरे:इङसु &इरेस्तसु &रेचसुये \\ 
ए:न्अ्य/अम्ह्‍याम्अ्य उ:ङ्‌ &इरे:इङनु &इरेस्तनु &रेचनुये \\ 
इन्य/अम्अ्य इऽचि &इरेचि &इरेस्ति    \\ 
इन्य/अम्अ्य ओऽचु &इरेचु &इरेस्तु  &रेचुये  \\ 
इन्य/अम्अ्य इक् &इरेह्इकि &इरेह्इतिकि   \\ 
इन्य/अम्अ्य ओक् &इरेह्इक &इरेह्इतक  &रेह्इकये  \\ 
अम्अ्य इन् & इरे:इ & इरेस्त्य   \\ 
अम्अ्य एऽचि & इरेचि & इरेस्ति    \\ 
अम्अ्य ए:न् & इरे:न्‍नि  & इरेस्त्यनु  \\ 
\midrule
उङ इन् & रे:न्‍न्य  & रे:न्त्यनि  \\ 
उङ एऽचि & रे:न्सु  & रे:न्त्यान्सु   \\ 
उङ ए:न्& रे:न्‍नु  & रे:न्त्यान्‍नु   \\ 
\bottomrule
\end{tabular}
\end{table}


\begin{table}[H]
\label{ekt.vt} \centering
\caption{सकर्मक क्रिया  रे:न्य  "लेख्‍नु"  }
\begin{tabular}{l|l|l|l|l|l|l|l|l|l|l|l|l}  \toprule
&अभूत & भूत & आज्ञार्थक \\ 
उङ &रेक्‌तु &रेक्‌त \\ 
उङ अम्सु&रेक्‌तुसु &रेक्‌तसु \\ 
उङ अम्ह्‍याम्&रेक्‌तुनु &रेक्‌तनु \\ 
इऽचिअ्य &रेकि &रेकिति   \\ 
ओऽचुअ        &रेकु &रेकुतु   \\ 
इक्अ्य&रेक्‌कि &रेक्‌तिकि   \\ 
ओक्अ &रेक्‌क &रेक्‌तक   \\ 
इन्य & इरेक्‌त्यु  & इरेक्‌त्य &रेक्‌ते  \\ 
इन्य अम्सु& इरेक्सु  & इरेक्‌त्यसु   \\ 
इन्य अम्ह्‍याम्& इरेक्‍नु  & इरेक्‌त्यनु   \\ 
एऽचि & इरेकि & इरेकिति &रेकिये    \\ 
ए:न् & इरे:नि  & इरेक्‌त्यनु &रेक्‍नुये  \\ 
अम्अ्य & रेक्‌त्यु  & रेक्‌त्य  \\ 
अम्सुअ्य & रेक्सु & रेक्‌त्यसु  \\ 
अम्ह्‍याम्अ्य & रेक्‍नु  & रेक्‌त्यनु \\ 
\midrule
इन्य, अम्अ्य उ:ङ्‌ &इरे:ङ &इरेकत &रेकये \\ 
एऽचिअ्य/अम्सुअ उ:ङ्‌ &इरे:ङसु &इरेकतसु &रेकसुये \\ 
ए:न्अ्य/अम्ह्‍याम्अ्य उ:ङ्‌ &इरे:ङनु &इरेकतनु &रेकनुये \\ 
इन्य/अम्अ्य इऽचि &इरेकि &इरेकिति    \\ 
इन्य/अम्अ्य ओऽचु &इरेकु &इरेकुतु  &रेकुये  \\ 
इन्य/अम्अ्य इक् &इरेक्‌कि &इरेक्‌तिकि   \\ 
इन्य/अम्अ्य ओक् &इरेक्‌क &इरेक्‌तक  &रेक्‌कये  \\ 
अम्अ्य इन् & इरे: & इरेक्‌त्य   \\ 
अम्अ्य एऽचि & इरेकि & इरेकिति    \\ 
अम्अ्य ए:न् & इरे:नि  & इरेक्‌त्यनु  \\ 
\midrule
उङ इन् & रे:न्य  & रे:न्त्यनि  \\ 
उङ एऽचि & रे:न्सु  & रे:न्त्यान्सु   \\ 
उङ ए:न्& रे:न्‍नु  & रे:न्त्यान्‍नु   \\ 
\bottomrule
\end{tabular}
\end{table}


\begin{table}[H]
\label{emt.vt} \centering
\caption{सकर्मक क्रिया  लेम्‍न्य  "फकाउनु"  }
\begin{tabular}{l|l|l|l|l|l|l|l|l|l|l|l|l}  \toprule
&अभूत & भूत & आज्ञार्थक \\ 
उङ &लेम्दु &ले:म्त \\ 
उङ अम्सु&लेम्दुसु &ले:म्तसु \\ 
उङ अम्ह्‍याम्&लेम्दुनु &ले:म्तनु \\ 
इऽचिअ्य &लेमि &लेमिति   \\ 
ओऽचुअ        &लेमु &लेमुतु   \\ 
इक्अ्य&लेम्कि &लेम्तिकि   \\ 
ओक्अ &लेम्क &लेम्तक   \\ 
इन्य & इलेम्द्‌यु  & इले:म्त्य &लेम्दे  \\ 
इन्य अम्सु& इले:म्सु  & इले:म्त्यसु   \\ 
इन्य अम्ह्‍याम्& इले:म्‍नु  & इले:म्त्यनु   \\ 
एऽचि & इलेमि & इलेमिति &लेमिये    \\ 
ए:न् & इलेम्‍नि  & इले:म्त्यनु &ले:म्‍नुये  \\ 
अम्अ्य & लेम्द्‌यु  & ले:म्त्य  \\ 
अम्सुअ्य & ले:म्सु & ले:म्त्यसु  \\ 
अम्ह्‍याम्अ्य & ले:म्‍नु  & ले:म्त्यनु \\ 
\midrule
इन्य, अम्अ्य उ:ङ्‌ &इलेम्ङ &इलेमत &लेमये \\ 
एऽचिअ्य/अम्सुअ उ:ङ्‌ &इलेम्ङसु &इलेमतसु &लेमसुये \\ 
ए:न्अ्य/अम्ह्‍याम्अ्य उ:ङ्‌ &इलेम्ङनु &इलेमतनु &लेमनुये \\ 
इन्य/अम्अ्य इऽचि &इलेमि &इलेमिति    \\ 
इन्य/अम्अ्य ओऽचु &इलेमु &इलेमुतु  &लेमुये  \\ 
इन्य/अम्अ्य इक् &इलेम्कि &इलेम्तिकि   \\ 
इन्य/अम्अ्य ओक् &इलेम्क &इलेम्तक  &लेम्कये  \\ 
अम्अ्य इन् & इलेम् & इले:म्त्य   \\ 
अम्अ्य एऽचि & इलेमि & इलेमिति    \\ 
अम्अ्य ए:न् & इलेम्‍नि  & इले:म्त्यनु  \\ 
\midrule
उङ इन् & लेम्‍न्य  & लेम्त्यनि  \\ 
उङ एऽचि & ले:म्सु  & लेम्त्यान्सु   \\ 
उङ ए:न्& ले:म्‍नु  & लेम्त्यान्‍नु   \\ 
\bottomrule
\end{tabular}
\end{table}


\begin{table}[H]
\label{ent.vt} \centering
\caption{सकर्मक क्रिया  छेइन्य  "चिन्नु"  }
\begin{tabular}{l|l|l|l|l|l|l|l|l|l|l|l|l}  \toprule
&अभूत & भूत & आज्ञार्थक \\ 
उङ &छेन्दु &छे:न्त \\ 
उङ अम्सु&छेन्दुसु &छे:न्तसु \\ 
उङ अम्ह्‍याम्&छेन्दुनु &छे:न्तनु \\ 
इऽचिअ्य &छेऽचि &छेऽस्ति   \\ 
ओऽचुअ        &छेऽचु &छेऽस्तु   \\ 
इक्अ्य&छेइकि &छेइतिकि   \\ 
ओक्अ &छेइक &छेइतक   \\ 
इन्य & इछेन्द्‌यु  & इछे:न्त्य &छेन्दे  \\ 
इन्य अम्सु& इछे:न्सु  & इछे:न्त्यसु   \\ 
इन्य अम्ह्‍याम्& इछे:न्‍नु  & इछे:न्त्यनु   \\ 
एऽचि & इछेऽचि & इछेऽस्ति &छेऽचिये    \\ 
ए:न् & इछेइनि  & इछेऽस्त्यनु &छेऽस्‍नुये  \\ 
अम्अ्य & छेन्द्‌यु  & छे:न्त्य  \\ 
अम्सुअ्य & छे:न्सु & छे:न्त्यसु  \\ 
अम्ह्‍याम्अ्य & छे:न्‍नु  & छे:न्त्यनु \\ 
\midrule
इन्य, अम्अ्य उ:ङ्‌ &इछेइङ &इछेऽस्त &छेऽचये \\ 
एऽचिअ्य/अम्सुअ उ:ङ्‌ &इछेइङसु &इछेऽस्तसु &छेऽचसुये \\ 
ए:न्अ्य/अम्ह्‍याम्अ्य उ:ङ्‌ &इछेइङनु &इछेऽस्तनु &छेऽचनुये \\ 
इन्य/अम्अ्य इऽचि &इछेऽचि &इछेऽस्ति    \\ 
इन्य/अम्अ्य ओऽचु &इछेऽचु &इछेऽस्तु  &छेऽचुये  \\ 
इन्य/अम्अ्य इक् &इछेइकि &इछेइतिकि   \\ 
इन्य/अम्अ्य ओक् &इछेइक &इछेइतक  &छेइकये  \\ 
अम्अ्य इन् & इछेइ & इछेऽस्त्य   \\ 
अम्अ्य एऽचि & इछेऽचि & इछेऽस्ति    \\ 
अम्अ्य ए:न् & इछेइनि  & इछेऽस्त्यनु  \\ 
\midrule
उङ इन् & छेइन्य  & छेइत्यनि  \\ 
उङ एऽचि & छे:इसु  & छेइत्यान्सु   \\ 
उङ ए:न्& छे:इनु  & छेइत्यान्‍नु   \\ 
\bottomrule
\end{tabular}
\end{table}


\begin{table}[H]
\label{eŋt.vt} \centering
\caption{सकर्मक क्रिया  रेऽन्य  "दु:ख दिनु"  }
\begin{tabular}{l|l|l|l|l|l|l|l|l|l|l|l|l}  \toprule
&अभूत & भूत & आज्ञार्थक \\ 
उङ &रेन्दु &रे:न्त \\ 
उङ अम्सु&रेन्दुसु &रे:न्तसु \\ 
उङ अम्ह्‍याम्&रेन्दुनु &रे:न्तनु \\ 
इऽचिअ्य &रेङि &रेङिति   \\ 
ओऽचुअ        &रेङु &रेङुतु   \\ 
इक्अ्य&रेङ्‌कि &रेङ्‌तिकि   \\ 
ओक्अ &रेङ्‌क &रेङ्‌तक   \\ 
इन्य & इरेन्द्‌यु  & इरे:न्त्य &रेन्दे  \\ 
इन्य अम्सु& इरे:न्सु  & इरे:न्त्यसु   \\ 
इन्य अम्ह्‍याम्& इरे:न्‍नु  & इरे:न्त्यनु   \\ 
एऽचि & इरेङि & इरेङिति &रेङिये    \\ 
ए:न् & इरेऽनि  & इरे:ङ्‌त्यनु &रे:ङ्‌नुये  \\ 
अम्अ्य & रेन्द्‌यु  & रे:न्त्य  \\ 
अम्सुअ्य & रे:न्सु & रे:न्त्यसु  \\ 
अम्ह्‍याम्अ्य & रे:न्‍नु  & रे:न्त्यनु \\ 
\midrule
इन्य, अम्अ्य उ:ङ्‌ &इरेऽङ &इरेङत &रेङये \\ 
एऽचिअ्य/अम्सुअ उ:ङ्‌ &इरेऽङसु &इरेङतसु &रेङसुये \\ 
ए:न्अ्य/अम्ह्‍याम्अ्य उ:ङ्‌ &इरेऽङनु &इरेङतनु &रेङनुये \\ 
इन्य/अम्अ्य इऽचि &इरेङि &इरेङिति    \\ 
इन्य/अम्अ्य ओऽचु &इरेङु &इरेङुतु  &रेङुये  \\ 
इन्य/अम्अ्य इक् &इरेङ्‌कि &इरेङ्‌तिकि   \\ 
इन्य/अम्अ्य ओक् &इरेङ्‌क &इरेङ्‌तक  &रेङ्‌कये  \\ 
अम्अ्य इन् & इरेङ्‌ & इरे:ङ्‌त्य   \\ 
अम्अ्य एऽचि & इरेङि & इरेङिति    \\ 
अम्अ्य ए:न् & इरेऽनि  & इरे:ङ्‌त्यनु  \\ 
\midrule
उङ इन् & रेऽन्य  & रेन्त्यनि  \\ 
उङ एऽचि & रे:न्सु  & रेन्त्यान्सु   \\ 
उङ ए:न्& रे:न्‍नु  & रेन्त्यान्‍नु   \\ 
\bottomrule
\end{tabular}
\end{table}


\begin{table}[H]
\label{ert.vt} \centering
\caption{सकर्मक क्रिया  धेर्न्य  "सुहाउनु"  }
\begin{tabular}{l|l|l|l|l|l|l|l|l|l|l|l|l}  \toprule
&अभूत & भूत & आज्ञार्थक \\ 
उङ &धेर्दु &धे:र्त \\ 
उङ अम्सु&धेर्दुसु &धे:र्तसु \\ 
उङ अम्ह्‍याम्&धेर्दुनु &धे:र्तनु \\ 
इऽचिअ्य &धेरि &धेरिति   \\ 
ओऽचुअ        &धेरु &धेरुतु   \\ 
इक्अ्य&धेर्कि &धेर्तिकि   \\ 
ओक्अ &धेर्क &धेर्तक   \\ 
इन्य & इधेर्द्‌यु  & इधे:र्त्य &धेर्दे  \\ 
इन्य अम्सु& इधे:र्सु  & इधे:र्त्यसु   \\ 
इन्य अम्ह्‍याम्& इधे:र्नु  & इधे:र्त्यनु   \\ 
एऽचि & इधेरि & इधेरिति &धेरिये    \\ 
ए:न् & इधेर्नि  & इधे:र्त्यनु &धे:र्नुये  \\ 
अम्अ्य & धेर्द्‌यु  & धे:र्त्य  \\ 
अम्सुअ्य & धे:र्सु & धे:र्त्यसु  \\ 
अम्ह्‍याम्अ्य & धे:र्नु  & धे:र्त्यनु \\ 
\midrule
इन्य, अम्अ्य उ:ङ्‌ &इधेर्ङ &इधेरत &धेरये \\ 
एऽचिअ्य/अम्सुअ उ:ङ्‌ &इधेर्ङसु &इधेरतसु &धेरसुये \\ 
ए:न्अ्य/अम्ह्‍याम्अ्य उ:ङ्‌ &इधेर्ङनु &इधेरतनु &धेरनुये \\ 
इन्य/अम्अ्य इऽचि &इधेरि &इधेरिति    \\ 
इन्य/अम्अ्य ओऽचु &इधेरु &इधेरुतु  &धेरुये  \\ 
इन्य/अम्अ्य इक् &इधेर्कि &इधेर्तिकि   \\ 
इन्य/अम्अ्य ओक् &इधेर्क &इधेर्तक  &धेर्कये  \\ 
अम्अ्य इन् & इधेर् & इधे:र्त्य   \\ 
अम्अ्य एऽचि & इधेरि & इधेरिति    \\ 
अम्अ्य ए:न् & इधेर्नि  & इधे:र्त्यनु  \\ 
\midrule
उङ इन् & धेर्न्य  & धेर्त्यनि  \\ 
उङ एऽचि & धे:र्सु  & धेर्त्यान्सु   \\ 
उङ ए:न्& धे:र्नु  & धेर्त्यान्‍नु   \\ 
\bottomrule
\end{tabular}
\end{table}


\begin{table}[H]
\label{elt.vt} \centering
\caption{अकर्मक क्रिया  छेल्न्य  "चम्किलो हुनु"  }
\begin{tabular}{l|l|l|l|l|l|l|l|l|l|l|l|l}  \toprule
&अभूत & भूत   \\ 
म्य & छेल्द्‌यु  & छे:ल्त्य  \\ 
\bottomrule
\end{tabular}
\end{table}


\begin{table}[H]
\label{i.vi} \centering
\caption{अकर्मक क्रिया  पिन्य  "परबाट आउनु"  }
\begin{tabular}{l|l|l|l|l|l|l|l|l|l|l|l|l}  \toprule
&अभूत & भूत & आज्ञार्थक \\ 
उ:ङ्‌ &पिङ &पिङत \\ 
इऽचि &पियि &पि:इति   \\ 
ओऽचु &पियु &पि:इतु   \\ 
इक् &पिकि &पिक्‌तिकि   \\ 
ओक् &पिक &पिक्‌तक   \\ 
इन् & इपि & इपुऽत्य &पुऽये  \\ 
एऽचि & इपियि & इपि:इति &पि:इये    \\ 
ए:न् & इपिनि  & इपुत्‍नु &पु:न्ये  \\ 
अम् & पि & पुऽत्य   \\ 
अम्सु & पियि & पि:इति     \\ 
अम्ह्‍याम् & पिनु  & पुत्‍नु \\ 
\bottomrule
\end{tabular}
\end{table}


\begin{table}[H]
\label{ip.vi} \centering
\caption{अकर्मक क्रिया  झ:म्‍न्य  "कुहुनु"  }
\begin{tabular}{l|l|l|l|l|l|l|l|l|l|l|l|l}  \toprule
&अभूत & भूत   \\ 
म्य & झप् & झिप्‍त्य   \\ 
\bottomrule
\end{tabular}
\end{table}


\begin{table}[H]
\label{it.vi} \centering
\caption{अकर्मक क्रिया  झ:न्‍न्य  "भिज्नु"  }
\begin{tabular}{l|l|l|l|l|l|l|l|l|l|l|l|l}  \toprule
&अभूत & भूत & आज्ञार्थक \\ 
उ:ङ्‌ &झै:ङ &झिस्त \\ 
इऽचि &झिचि &झिस्ति   \\ 
ओऽचु &झिचु &झिस्तु   \\ 
इक् &झह्इकि &झह्इतिकि   \\ 
ओक् &झह्इक &झह्इतक   \\ 
इन् & इझै: & इझिस्त्य &झिच्‍चे  \\ 
एऽचि & इझिचि & इझिस्ति &झिचिये    \\ 
ए:न् & इझ:न्‍नि  & इझिस्त्यनु &झिस्‍नुये  \\ 
अम् & झै: & झिस्त्य   \\ 
अम्सु & झिचि & झिस्ति   \\ 
अम्ह्‍याम् & झ:न्‍नु  & झिस्त्यनु \\ 
\bottomrule
\end{tabular}
\end{table}


\begin{table}[H]
\label{ik.vi} \centering
\caption{अकर्मक क्रिया  घु:न्य  "डोरिमा झुण्डिनु"  }
\begin{tabular}{l|l|l|l|l|l|l|l|l|l|l|l|l}  \toprule
&अभूत & भूत   \\ 
म्य & घु: & घिक्‌त्य   \\ 
\bottomrule
\end{tabular}
\end{table}


\begin{table}[H]
\label{im.vi} \centering
\caption{अकर्मक क्रिया  लम्‍न्य  "पलाउनु"  }
\begin{tabular}{l|l|l|l|l|l|l|l|l|l|l|l|l}  \toprule
&अभूत & भूत   \\ 
म्य & लम् & लि:म्त्य   \\ 
\bottomrule
\end{tabular}
\end{table}


\begin{table}[H]
\label{in.vi} \centering
\caption{अकर्मक क्रिया  ङैन्य  "डराउनु"  }
\begin{tabular}{l|l|l|l|l|l|l|l|l|l|l|l|l}  \toprule
&अभूत & भूत & आज्ञार्थक \\ 
उ:ङ्‌ &ङैङ &ङिऽस्त \\ 
इऽचि &ङिऽचि &ङिऽस्ति   \\ 
ओऽचु &ङिऽचु &ङिऽस्तु   \\ 
इक् &ङैकि &ङैतिकि   \\ 
ओक् &ङैक &ङैतक   \\ 
इन् & इङै & इङिऽस्त्य &ङिऽचे  \\ 
एऽचि & इङिऽचि & इङिऽस्ति &ङिऽचिये    \\ 
ए:न् & इङैनि  & इङिऽस्त्यनु &ङिऽस्‍नुये  \\ 
अम् & ङै & ङिऽस्त्य   \\ 
अम्सु & ङिऽचि & ङिऽस्ति   \\ 
अम्ह्‍याम् & ङैनु  & ङिऽस्त्यनु \\ 
\bottomrule
\end{tabular}
\end{table}


\begin{table}[H]
\label{iŋ.vi} \centering
\caption{अकर्मक क्रिया  बुऽन्य  "राम्रो देखिनु"  }
\begin{tabular}{l|l|l|l|l|l|l|l|l|l|l|l|l}  \toprule
&अभूत & भूत & आज्ञार्थक \\ 
उ:ङ्‌ &बुऽङ &बिङत \\ 
इऽचि &बिङि &बिङिति   \\ 
ओऽचु &बिङु &बिङुतु   \\ 
इक् &बङ्‌कि &बङ्‌तिकि   \\ 
ओक् &बङ्‌क &बङ्‌तक   \\ 
इन् & इबङ्‌ & इबि:ङ्‌त्य &बि:ङ्‌ये  \\ 
एऽचि & इबिङि & इबिङिति &बिङिये    \\ 
ए:न् & इबुऽनि  & इबि:ङ्‌त्यनु &बि:ङ्‌नुये  \\ 
अम् & बङ्‌ & बि:ङ्‌त्य   \\ 
अम्सु & बिङि & बिङिति   \\ 
अम्ह्‍याम् & बुऽनु  & बि:ङ्‌त्यनु \\ 
\bottomrule
\end{tabular}
\end{table}




\begin{table}[H]
\label{i.vt} \centering
\caption{सकर्मक क्रिया  बिन्य  "दिनु"  }
\begin{tabular}{l|l|l|l|l|l|l|l|l|l|l|l|l}  \toprule
&अभूत & भूत & आज्ञार्थक \\ 
उङ &बिङ &बि:ङ्‌त \\ 
इऽचिअ्य &बियि &बि:इति   \\ 
ओऽचुअ &बियु &बि:इतु   \\ 
इक्अ्य &बिकि &बिक्‌तिकि   \\ 
ओक्अ &बिक &बिक्‌तक   \\ 
इन्य & इबि & इबित्य &बिये  \\ 
एऽचिअ्य & इबियि & इबि:इति &बि:इये    \\ 
ए:न्अ्य & इबिनि  & इबुत्‍नु &बु:न्ये  \\ 
अम्अ्य & बि & बित्य   \\ 
अम्सुअ & बिसु & बित्सु     \\ 
अम्ह्‍याम्अ्य & बिनु  & बित्‍नु \\ 
\midrule
इन्य/अम्अ्य उ:ङ्‌&इबिङ &इबिङत &बिङये \\ 
एऽचिअ्य/अम्सुअ उ:ङ्‌ &इबिङसु &इबिङतसु &बिङसुये \\ 
ए:न्अ्य/अम्ह्‍याम्अ्य उ:ङ्‌ &इबिङनु &इबिङतनु &बिङनुये \\ 
इन्य/अम्अ्य इऽचि &इबियि &इबि:इति    \\ 
इन्य/अम्अ्य ओऽचु &इबियु &इबि:इतु  &बिइये  \\ 
इन्य/अम्अ्य इक् &इबिकि &इबिक्‌तिकि   \\ 
इन्य/अम्अ्य ओक् &इबिक &इबिक्‌तक  &बिकये  \\ 
अम्अ्य इन् & इबि & इबुऽत्य   \\ 
अम्अ्य एऽचि & इबियि & इबि:इति     \\ 
अम्अ्य ए:न् & इबिनि  & इबुत्‍नु  \\ 
\midrule
उङ इन् & बिन्य  & बि:न्त्यनि  \\ 
उङ एऽचि & बि:न्सु  & बि:न्त्यान्सु   \\ 
उङ ए:न्& बि:न्‍नु  & बि:न्त्यान्‍नु   \\ 
\bottomrule
\end{tabular}
\end{table}


\begin{table}[H]
\label{ip.vt} \centering
\caption{सकर्मक क्रिया  ख:म्‍न्य  "पकाउनु"  }
\begin{tabular}{l|l|l|l|l|l|l|l|l|l|l|l|l}  \toprule
&अभूत & भूत & आज्ञार्थक \\ 
उङ &खिबु &खिबुत \\ 
उङ अम्सु &खिबुसु &खिबुतसु \\ 
उङ अम्ह्‍याम् &खिबुनु &खिबुतनु \\ 
इऽचिअ्य &खिपि &खिपिति   \\ 
ओऽचुअ &खिपु &खिपुतु   \\ 
इक्अ्य &खप्कि &खप्‍तिकि   \\ 
ओक्अ &खप्क &खप्‍तक   \\ 
इन्य अम् & इखिऽब्यु  & इखिऽप्‍त्य &खिऽबे  \\ 
इन्य अम्सु & इखिऽप्सु  & इखिऽप्‍त्यसु   \\ 
इन्य अम्ह्‍याम् & इखिऽप्‍नु  & इखिऽप्‍त्यनु   \\ 
एऽचिअ्य & इखिपि & इखिपिति &खिपिये    \\ 
ए:न्अ्य & इख:म्‍नि  & इखिप्‍त्यनु &खिप्‍नुये  \\ 
अम्अ्य & खिऽब्यु  & खिऽप्‍त्य  \\ 
अम्सुअ & खिऽप्सु & खिऽप्‍त्यसु  \\ 
अम्ह्‍याम्अ्य & खिऽप्‍नु  & खिऽप्‍त्यनु \\ 
\midrule
इन्य/अम्अ्य उ:ङ्‌&इख:म्ङ & इखिपत &खिपये \\ 
एऽचिअ्य/अम्सुअ उ:ङ्‌ &इख:म्ङसु & इखिपतसु &खिपसुये \\ 
ए:न्अ्य/अम्ह्‍याम्अ्य उ:ङ्‌ &इख:म्ङनु & इखिपतनु &खिपनुये \\ 
इन्य/अम्अ्य इऽचि & इखिपि & इखिपिति    \\ 
इन्य/अम्अ्य ओऽचु & इखिपु & इखिपुतु  &खिपुये  \\ 
इन्य/अम्अ्य इक् & इखप्कि & इखप्‍तिकि   \\ 
इन्य/अम्अ्य ओक् & इखप्क & इखप्‍तक  &खप्कये  \\ 
अम्अ्य इन् & इखप् & इखिप्‍त्य   \\ 
अम्अ्य एऽचि & इखिपि & इखिपिति    \\ 
अम्अ्य ए:न् & इख:म्‍नि  & इखिप्‍त्यनु  \\ 
\midrule
उङ इन् & ख:म्‍न्य  & ख:म्त्यनि  \\ 
उङ एऽचि & ख:म्सु  & ख:म्त्यान्सु   \\ 
उङ ए:न्& ख:म्‍नु  & ख:म्त्यान्‍नु   \\ 
\bottomrule
\end{tabular}
\end{table}


\begin{table}[H]
\label{it.vt} \centering
\caption{सकर्मक क्रिया  स:न्‍न्य  "छोड्नु"  }
\begin{tabular}{l|l|l|l|l|l|l|l|l|l|l|l|l}  \toprule
&अभूत & भूत & आज्ञार्थक \\ 
उङ &सिदु &सि:त \\ 
उङ अम्सु &सिदुसु &सि:तसु \\ 
उङ अम्ह्‍याम् &सिदुनु &सि:तनु \\ 
इऽचिअ्य &सिचि &सिस्ति   \\ 
ओऽचुअ &सिचु &सिस्तु   \\ 
इक्अ्य &सह्इकि &सह्इतिकि   \\ 
ओक्अ &सह्इक &सह्इतक   \\ 
इन्य अम् & इसिऽद्‌यु  & इसि:त्य &सिऽदे  \\ 
इन्य अम्सु & इसिऽत्सु  & इसि:त्यसु   \\ 
इन्य अम्ह्‍याम् & इसिऽत्‍नु  & इसि:त्यनु   \\ 
एऽचिअ्य & इसिचि & इसिस्ति &सिचिये    \\ 
ए:न्अ्य & इस:न्‍नि  & इसिस्त्यनु &सिस्‍नुये  \\ 
अम्अ्य & सिऽद्‌यु  & सि:त्य  \\ 
अम्सुअ & सिऽत्सु & सि:त्यसु  \\ 
अम्ह्‍याम्अ्य & सिऽत्‍नु  & सि:त्यनु \\ 
\midrule
इन्य/अम्अ्य उ:ङ्‌&इसै:ङ & इसिस्त &सिचये \\ 
एऽचिअ्य/अम्सुअ उ:ङ्‌ &इसै:ङसु & इसिस्तसु &सिचसुये \\ 
ए:न्अ्य/अम्ह्‍याम्अ्य उ:ङ्‌ &इसै:ङनु & इसिस्तनु &सिचनुये \\ 
इन्य/अम्अ्य इऽचि & इसिचि & इसिस्ति    \\ 
इन्य/अम्अ्य ओऽचु & इसिचु & इसिस्तु  &सिचुये  \\ 
इन्य/अम्अ्य इक् & इसह्इकि & इसह्इतिकि   \\ 
इन्य/अम्अ्य ओक् & इसह्इक & इसह्इतक  &सह्इकये  \\ 
अम्अ्य इन् & इसै: & इसिस्त्य   \\ 
अम्अ्य एऽचि & इसिचि & इसिस्ति    \\ 
अम्अ्य ए:न् & इस:न्‍नि  & इसिस्त्यनु  \\ 
\midrule
उङ इन् & स:न्‍न्य  & स:न्त्यनि  \\ 
उङ एऽचि & स:न्सु  & स:न्त्यान्सु   \\ 
उङ ए:न्& स:न्‍नु  & स:न्त्यान्‍नु   \\ 
\bottomrule
\end{tabular}
\end{table}


\begin{table}[H]
\label{ik.vt} \centering
\caption{सकर्मक क्रिया  कु:न्य  "गाँठो पार्नु"  }
\begin{tabular}{l|l|l|l|l|l|l|l|l|l|l|l|l}  \toprule
&अभूत & भूत & आज्ञार्थक \\ 
उङ &किगु &किगुत \\ 
उङ अम्सु &किगुसु &किगुतसु \\ 
उङ अम्ह्‍याम् &किगुनु &किगुतनु \\ 
इऽचिअ्य &किकि &किकिति   \\ 
ओऽचुअ &किकु &किकुतु   \\ 
इक्अ्य &कक्‌कि &कक्‌तिकि   \\ 
ओक्अ &कक्‌क &कक्‌तक   \\ 
इन्य अम् & इकिऽग्यु  & इकिऽक्‌त्य &किऽगे  \\ 
इन्य अम्सु & इकिऽक्सु  & इकिऽक्‌त्यसु   \\ 
इन्य अम्ह्‍याम् & इकिऽक्‍नु  & इकिऽक्‌त्यनु   \\ 
एऽचिअ्य & इकिकि & इकिकिति &किकिये    \\ 
ए:न्अ्य & इकु:नि  & इकिक्‌त्यनु &किक्‍नुये  \\ 
अम्अ्य & किऽग्यु  & किऽक्‌त्य  \\ 
अम्सुअ & किऽक्सु & किऽक्‌त्यसु  \\ 
अम्ह्‍याम्अ्य & किऽक्‍नु  & किऽक्‌त्यनु \\ 
\midrule
इन्य/अम्अ्य उ:ङ्‌&इकु:ङ & इकिकत &किकये \\ 
एऽचिअ्य/अम्सुअ उ:ङ्‌ &इकु:ङसु & इकिकतसु &किकसुये \\ 
ए:न्अ्य/अम्ह्‍याम्अ्य उ:ङ्‌ &इकु:ङनु & इकिकतनु &किकनुये \\ 
इन्य/अम्अ्य इऽचि & इकिकि & इकिकिति    \\ 
इन्य/अम्अ्य ओऽचु & इकिकु & इकिकुतु  &किकुये  \\ 
इन्य/अम्अ्य इक् & इकक्‌कि & इकक्‌तिकि   \\ 
इन्य/अम्अ्य ओक् & इकक्‌क & इकक्‌तक  &कक्‌कये  \\ 
अम्अ्य इन् & इकु: & इकिक्‌त्य   \\ 
अम्अ्य एऽचि & इकिकि & इकिकिति    \\ 
अम्अ्य ए:न् & इकु:नि  & इकिक्‌त्यनु  \\ 
\midrule
उङ इन् & कु:न्य  & क:न्त्यनि  \\ 
उङ एऽचि & क:न्सु  & क:न्त्यान्सु   \\ 
उङ ए:न्& क:न्‍नु  & क:न्त्यान्‍नु   \\ 
\bottomrule
\end{tabular}
\end{table}


\begin{table}[H]
\label{im.vt} \centering
\caption{सकर्मक क्रिया  छम्‍न्य  "मिच्नु"  }
\begin{tabular}{l|l|l|l|l|l|l|l|l|l|l|l|l}  \toprule
&अभूत & भूत & आज्ञार्थक \\ 
उङ &छिमु &छिमुत \\ 
उङ अम्सु &छिमुसु &छिमुतसु \\ 
उङ अम्ह्‍याम् &छिमुनु &छिमुतनु \\ 
इऽचिअ्य &छिमि &छिमिति   \\ 
ओऽचुअ &छिमु &छिमुतु   \\ 
इक्अ्य &छम्कि &छम्तिकि   \\ 
ओक्अ &छम्क &छम्तक   \\ 
इन्य अम् & इछिऽम्यु  & इछि:म्त्य &छिऽमे  \\ 
इन्य अम्सु & इछि:म्सु  & इछि:म्त्यसु   \\ 
इन्य अम्ह्‍याम् & इछि:म्‍नु  & इछि:म्त्यनु   \\ 
एऽचिअ्य & इछिमि & इछिमिति &छिमिये    \\ 
ए:न्अ्य & इछम्‍नि  & इछि:म्त्यनु &छि:म्‍नुये  \\ 
अम्अ्य & छिऽम्यु  & छि:म्त्य  \\ 
अम्सुअ & छि:म्सु & छि:म्त्यसु  \\ 
अम्ह्‍याम्अ्य & छि:म्‍नु  & छि:म्त्यनु \\ 
\midrule
इन्य/अम्अ्य उ:ङ्‌&इछम्ङ & इछिमत &छिमये \\ 
एऽचिअ्य/अम्सुअ उ:ङ्‌ &इछम्ङसु & इछिमतसु &छिमसुये \\ 
ए:न्अ्य/अम्ह्‍याम्अ्य उ:ङ्‌ &इछम्ङनु & इछिमतनु &छिमनुये \\ 
इन्य/अम्अ्य इऽचि & इछिमि & इछिमिति    \\ 
इन्य/अम्अ्य ओऽचु & इछिमु & इछिमुतु  &छिमुये  \\ 
इन्य/अम्अ्य इक् & इछम्कि & इछम्तिकि   \\ 
इन्य/अम्अ्य ओक् & इछम्क & इछम्तक  &छम्कये  \\ 
अम्अ्य इन् & इछम् & इछि:म्त्य   \\ 
अम्अ्य एऽचि & इछिमि & इछिमिति    \\ 
अम्अ्य ए:न् & इछम्‍नि  & इछि:म्त्यनु  \\ 
\midrule
उङ इन् & छम्‍न्य  & छम्त्यनि  \\ 
उङ एऽचि & छ:म्सु  & छम्त्यान्सु   \\ 
उङ ए:न्& छ:म्‍नु  & छम्त्यान्‍नु   \\ 
\bottomrule
\end{tabular}
\end{table}


\begin{table}[H]
\label{iŋ.vt} \centering
\caption{सकर्मक क्रिया  सुऽन्य  "सोध्नु"  }
\begin{tabular}{l|l|l|l|l|l|l|l|l|l|l|l|l}  \toprule
&अभूत & भूत & आज्ञार्थक \\ 
उङ &सिङु &सिङुत \\ 
उङ अम्सु &सिङुसु &सिङुतसु \\ 
उङ अम्ह्‍याम् &सिङुनु &सिङुतनु \\ 
इऽचिअ्य &सिङि &सिङिति   \\ 
ओऽचुअ &सिङु &सिङुतु   \\ 
इक्अ्य &सङ्‌कि &सङ्‌तिकि   \\ 
ओक्अ &सङ्‌क &सङ्‌तक   \\ 
इन्य अम् & इसिऽङ्‌यु  & इसि:ङ्‌त्य &सिऽङे  \\ 
इन्य अम्सु & इसि:ङ्‌सु  & इसि:ङ्‌त्यसु   \\ 
इन्य अम्ह्‍याम् & इसि:ङ्‌नु  & इसि:ङ्‌त्यनु   \\ 
एऽचिअ्य & इसिङि & इसिङिति &सिङिये    \\ 
ए:न्अ्य & इसुऽनि  & इसि:ङ्‌त्यनु &सि:ङ्‌नुये  \\ 
अम्अ्य & सिऽङ्‌यु  & सि:ङ्‌त्य  \\ 
अम्सुअ & सि:ङ्‌सु & सि:ङ्‌त्यसु  \\ 
अम्ह्‍याम्अ्य & सि:ङ्‌नु  & सि:ङ्‌त्यनु \\ 
\midrule
इन्य/अम्अ्य उ:ङ्‌&इसुऽङ & इसिङत &सिङये \\ 
एऽचिअ्य/अम्सुअ उ:ङ्‌ &इसुऽङसु & इसिङतसु &सिङसुये \\ 
ए:न्अ्य/अम्ह्‍याम्अ्य उ:ङ्‌ &इसुऽङनु & इसिङतनु &सिङनुये \\ 
इन्य/अम्अ्य इऽचि & इसिङि & इसिङिति    \\ 
इन्य/अम्अ्य ओऽचु & इसिङु & इसिङुतु  &सिङुये  \\ 
इन्य/अम्अ्य इक् & इसङ्‌कि & इसङ्‌तिकि   \\ 
इन्य/अम्अ्य ओक् & इसङ्‌क & इसङ्‌तक  &सङ्‌कये  \\ 
अम्अ्य इन् & इसङ्‌ & इसि:ङ्‌त्य   \\ 
अम्अ्य एऽचि & इसिङि & इसिङिति    \\ 
अम्अ्य ए:न् & इसुऽनि  & इसि:ङ्‌त्यनु  \\ 
\midrule
उङ इन् & सुऽन्य  & सन्त्यनि  \\ 
उङ एऽचि & स:न्सु  & सन्त्यान्सु   \\ 
उङ ए:न्& स:न्‍नु  & सन्त्यान्‍नु   \\ 
\bottomrule
\end{tabular}
\end{table}


\begin{table}[H]
\label{ir.vt} \centering
\caption{सकर्मक क्रिया  चर्न्य  "चिन्नु"  }
\begin{tabular}{l|l|l|l|l|l|l|l|l|l|l|l|l}  \toprule
&अभूत & भूत & आज्ञार्थक \\ 
उङ &चिरु &चिरुत \\ 
उङ अम्सु &चिरुसु &चिरुतसु \\ 
उङ अम्ह्‍याम् &चिरुनु &चिरुतनु \\ 
इऽचिअ्य &चिरि &चिरिति   \\ 
ओऽचुअ &चिरु &चिरुतु   \\ 
इक्अ्य &चर्कि &चर्तिकि   \\ 
ओक्अ &चर्क &चर्तक   \\ 
इन्य अम् & इचिऽर्‍यु  & इचि:र्त्य &चिऽरे  \\ 
इन्य अम्सु & इचि:र्सु  & इचि:र्त्यसु   \\ 
इन्य अम्ह्‍याम् & इचि:र्नु  & इचि:र्त्यनु   \\ 
एऽचिअ्य & इचिरि & इचिरिति &चिरिये    \\ 
ए:न्अ्य & इचर्नि  & इचि:र्त्यनु &चि:र्नुये  \\ 
अम्अ्य & चिऽर्‍यु  & चि:र्त्य  \\ 
अम्सुअ & चि:र्सु & चि:र्त्यसु  \\ 
अम्ह्‍याम्अ्य & चि:र्नु  & चि:र्त्यनु \\ 
\midrule
इन्य/अम्अ्य उ:ङ्‌&इचर्ङ & इचिरत &चिरये \\ 
एऽचिअ्य/अम्सुअ उ:ङ्‌ &इचर्ङसु & इचिरतसु &चिरसुये \\ 
ए:न्अ्य/अम्ह्‍याम्अ्य उ:ङ्‌ &इचर्ङनु & इचिरतनु &चिरनुये \\ 
इन्य/अम्अ्य इऽचि & इचिरि & इचिरिति    \\ 
इन्य/अम्अ्य ओऽचु & इचिरु & इचिरुतु  &चिरुये  \\ 
इन्य/अम्अ्य इक् & इचर्कि & इचर्तिकि   \\ 
इन्य/अम्अ्य ओक् & इचर्क & इचर्तक  &चर्कये  \\ 
अम्अ्य इन् & इचर् & इचि:र्त्य   \\ 
अम्अ्य एऽचि & इचिरि & इचिरिति    \\ 
अम्अ्य ए:न् & इचर्नि  & इचि:र्त्यनु  \\ 
\midrule
उङ इन् & चर्न्य  & चर्त्यनि  \\ 
उङ एऽचि & च:र्सु  & चर्त्यान्सु   \\ 
उङ ए:न्& च:र्नु  & चर्त्यान्‍नु   \\ 
\bottomrule
\end{tabular}
\end{table}


\begin{table}[H]
\label{il.vt} \centering
\caption{सकर्मक क्रिया  हल्न्य  "पिल्स्याउनु"  }
\begin{tabular}{l|l|l|l|l|l|l|l|l|l|l|l|l}  \toprule
&अभूत & भूत & आज्ञार्थक \\ 
उङ &हिलु &हिलुत \\ 
उङ अम्सु &हिलुसु &हिलुतसु \\ 
उङ अम्ह्‍याम् &हिलुनु &हिलुतनु \\ 
इऽचिअ्य &हिलि &हिलिति   \\ 
ओऽचुअ &हिलु &हिलुतु   \\ 
इक्अ्य &हल्कि &हल्तिकि   \\ 
ओक्अ &हल्क &हल्तक   \\ 
इन्य अम् & इहिऽल्यु  & इहि:ल्त्य &हिऽले  \\ 
इन्य अम्सु & इहि:ल्सु  & इहि:ल्त्यसु   \\ 
इन्य अम्ह्‍याम् & इहि:ल्नु  & इहि:ल्त्यनु   \\ 
एऽचिअ्य & इहिलि & इहिलिति &हिलिये    \\ 
ए:न्अ्य & इहल्नि  & इहि:ल्त्यनु &हि:ल्नुये  \\ 
अम्अ्य & हिऽल्यु  & हि:ल्त्य  \\ 
अम्सुअ & हि:ल्सु & हि:ल्त्यसु  \\ 
अम्ह्‍याम्अ्य & हि:ल्नु  & हि:ल्त्यनु \\ 
\midrule
इन्य/अम्अ्य उ:ङ्‌&इहल्ङ & इहिलत &हिलये \\ 
एऽचिअ्य/अम्सुअ उ:ङ्‌ &इहल्ङसु & इहिलतसु &हिलसुये \\ 
ए:न्अ्य/अम्ह्‍याम्अ्य उ:ङ्‌ &इहल्ङनु & इहिलतनु &हिलनुये \\ 
इन्य/अम्अ्य इऽचि & इहिलि & इहिलिति    \\ 
इन्य/अम्अ्य ओऽचु & इहिलु & इहिलुतु  &हिलुये  \\ 
इन्य/अम्अ्य इक् & इहल्कि & इहल्तिकि   \\ 
इन्य/अम्अ्य ओक् & इहल्क & इहल्तक  &हल्कये  \\ 
अम्अ्य इन् & इहल् & इहि:ल्त्य   \\ 
अम्अ्य एऽचि & इहिलि & इहिलिति    \\ 
अम्अ्य ए:न् & इहल्नि  & इहि:ल्त्यनु  \\ 
\midrule
उङ इन् & हल्न्य  & हल्त्यनि  \\ 
उङ एऽचि & ह:ल्सु  & हल्त्यान्सु   \\ 
उङ ए:न्& ह:ल्नु  & हल्त्यान्‍नु   \\ 
\bottomrule
\end{tabular}
\end{table}


\begin{table}[H]
\label{ipt.vt} \centering
\caption{सकर्मक क्रिया  अ:म्‍न्य  "सुताउनु"  }
\begin{tabular}{l|l|l|l|l|l|l|l|l|l|l|l|l}  \toprule
&अभूत & भूत & आज्ञार्थक \\ 
उङ &अप्‍तु &अप्‍त \\ 
उङ अम्सु&अप्‍तुसु &अप्‍तसु \\ 
उङ अम्ह्‍याम्&अप्‍तुनु &अप्‍तनु \\ 
इऽचिअ्य &इपि &इपिति   \\ 
ओऽचुअ        &इपु &इपुतु   \\ 
इक्अ्य&अप्कि &अप्‍तिकि   \\ 
ओक्अ &अप्क &अप्‍तक   \\ 
इन्य & इअप्‍त्यु  & इअप्‍त्य &अप्‍ते  \\ 
इन्य अम्सु& इअप्सु  & इअप्‍त्यसु   \\ 
इन्य अम्ह्‍याम्& इअप्‍नु  & इअप्‍त्यनु   \\ 
एऽचि & इइपि & इइपिति &इपिये    \\ 
ए:न् & इअ:म्‍नि  & इइप्‍त्यनु &इप्‍नुये  \\ 
अम्अ्य & अप्‍त्यु  & अप्‍त्य  \\ 
अम्सुअ्य & अप्सु & अप्‍त्यसु  \\ 
अम्ह्‍याम्अ्य & अप्‍नु  & अप्‍त्यनु \\ 
\midrule
इन्य, अम्अ्य उ:ङ्‌ &इअ:म्ङ &इइपत &इपये \\ 
एऽचिअ्य/अम्सुअ उ:ङ्‌ &इअ:म्ङसु &इइपतसु &इपसुये \\ 
ए:न्अ्य/अम्ह्‍याम्अ्य उ:ङ्‌ &इअ:म्ङनु &इइपतनु &इपनुये \\ 
इन्य/अम्अ्य इऽचि &इइपि &इइपिति    \\ 
इन्य/अम्अ्य ओऽचु &इइपु &इइपुतु  &इपुये  \\ 
इन्य/अम्अ्य इक् &इअप्कि &इअप्‍तिकि   \\ 
इन्य/अम्अ्य ओक् &इअप्क &इअप्‍तक  &अप्कये  \\ 
अम्अ्य इन् & इअप् & इइप्‍त्य   \\ 
अम्अ्य एऽचि & इइपि & इइपिति    \\ 
अम्अ्य ए:न् & इअ:म्‍नि  & इइप्‍त्यनु  \\ 
\midrule
उङ इन् & अ:म्‍न्य  & अ:म्त्यनि  \\ 
उङ एऽचि & अ:म्सु  & अ:म्त्यान्सु   \\ 
उङ ए:न्& अ:म्‍नु  & अ:म्त्यान्‍नु   \\ 
\bottomrule
\end{tabular}
\end{table}


\begin{table}[H]
\label{itt.vt} \centering
\caption{सकर्मक क्रिया  स:न्‍न्य  "धामीले फुक्नु"  }
\begin{tabular}{l|l|l|l|l|l|l|l|l|l|l|l|l}  \toprule
&अभूत & भूत & आज्ञार्थक \\ 
उङ &सत्तु &सत्त \\ 
उङ अम्सु&सत्तुसु &सत्तसु \\ 
उङ अम्ह्‍याम्&सत्तुनु &सत्तनु \\ 
इऽचिअ्य &सिचि &सिस्ति   \\ 
ओऽचुअ        &सिचु &सिस्तु   \\ 
इक्अ्य&सह्इकि &सह्इतिकि   \\ 
ओक्अ &सह्इक &सह्इतक   \\ 
इन्य & इसत्त्यु  & इसत्त्य &सत्ते  \\ 
इन्य अम्सु& इसत्सु  & इसत्त्यसु   \\ 
इन्य अम्ह्‍याम्& इसत्‍नु  & इसत्त्यनु   \\ 
एऽचि & इसिचि & इसिस्ति &सिचिये    \\ 
ए:न् & इस:न्‍नि  & इसिस्त्यनु &सिस्‍नुये  \\ 
अम्अ्य & सत्त्यु  & सत्त्य  \\ 
अम्सुअ्य & सत्सु & सत्त्यसु  \\ 
अम्ह्‍याम्अ्य & सत्‍नु  & सत्त्यनु \\ 
\midrule
इन्य, अम्अ्य उ:ङ्‌ &इसै:ङ &इसिस्त &सिचये \\ 
एऽचिअ्य/अम्सुअ उ:ङ्‌ &इसै:ङसु &इसिस्तसु &सिचसुये \\ 
ए:न्अ्य/अम्ह्‍याम्अ्य उ:ङ्‌ &इसै:ङनु &इसिस्तनु &सिचनुये \\ 
इन्य/अम्अ्य इऽचि &इसिचि &इसिस्ति    \\ 
इन्य/अम्अ्य ओऽचु &इसिचु &इसिस्तु  &सिचुये  \\ 
इन्य/अम्अ्य इक् &इसह्इकि &इसह्इतिकि   \\ 
इन्य/अम्अ्य ओक् &इसह्इक &इसह्इतक  &सह्इकये  \\ 
अम्अ्य इन् & इसै: & इसिस्त्य   \\ 
अम्अ्य एऽचि & इसिचि & इसिस्ति    \\ 
अम्अ्य ए:न् & इस:न्‍नि  & इसिस्त्यनु  \\ 
\midrule
उङ इन् & स:न्‍न्य  & स:न्त्यनि  \\ 
उङ एऽचि & स:न्सु  & स:न्त्यान्सु   \\ 
उङ ए:न्& स:न्‍नु  & स:न्त्यान्‍नु   \\ 
\bottomrule
\end{tabular}
\end{table}


\begin{table}[H]
\label{ikt.vt} \centering
\caption{सकर्मक क्रिया  घ्रु:न्य  "समात्‍नु"  }
\begin{tabular}{l|l|l|l|l|l|l|l|l|l|l|l|l}  \toprule
&अभूत & भूत & आज्ञार्थक \\ 
उङ &घ्रक्‌तु &घ्रक्‌त \\ 
उङ अम्सु&घ्रक्‌तुसु &घ्रक्‌तसु \\ 
उङ अम्ह्‍याम्&घ्रक्‌तुनु &घ्रक्‌तनु \\ 
इऽचिअ्य &घ्रिकि &घ्रिकिति   \\ 
ओऽचुअ        &घ्रिकु &घ्रिकुतु   \\ 
इक्अ्य&घ्रक्‌कि &घ्रक्‌तिकि   \\ 
ओक्अ &घ्रक्‌क &घ्रक्‌तक   \\ 
इन्य & इघ्रक्‌त्यु  & इघ्रक्‌त्य &घ्रक्‌ते  \\ 
इन्य अम्सु& इघ्रक्सु  & इघ्रक्‌त्यसु   \\ 
इन्य अम्ह्‍याम्& इघ्रक्‍नु  & इघ्रक्‌त्यनु   \\ 
एऽचि & इघ्रिकि & इघ्रिकिति &घ्रिकिये    \\ 
ए:न् & इघ्रु:नि  & इघ्रिक्‌त्यनु &घ्रिक्‍नुये  \\ 
अम्अ्य & घ्रक्‌त्यु  & घ्रक्‌त्य  \\ 
अम्सुअ्य & घ्रक्सु & घ्रक्‌त्यसु  \\ 
अम्ह्‍याम्अ्य & घ्रक्‍नु  & घ्रक्‌त्यनु \\ 
\midrule
इन्य, अम्अ्य उ:ङ्‌ &इघ्रु:ङ &इघ्रिकत &घ्रिकये \\ 
एऽचिअ्य/अम्सुअ उ:ङ्‌ &इघ्रु:ङसु &इघ्रिकतसु &घ्रिकसुये \\ 
ए:न्अ्य/अम्ह्‍याम्अ्य उ:ङ्‌ &इघ्रु:ङनु &इघ्रिकतनु &घ्रिकनुये \\ 
इन्य/अम्अ्य इऽचि &इघ्रिकि &इघ्रिकिति    \\ 
इन्य/अम्अ्य ओऽचु &इघ्रिकु &इघ्रिकुतु  &घ्रिकुये  \\ 
इन्य/अम्अ्य इक् &इघ्रक्‌कि &इघ्रक्‌तिकि   \\ 
इन्य/अम्अ्य ओक् &इघ्रक्‌क &इघ्रक्‌तक  &घ्रक्‌कये  \\ 
अम्अ्य इन् & इघ्रु: & इघ्रिक्‌त्य   \\ 
अम्अ्य एऽचि & इघ्रिकि & इघ्रिकिति    \\ 
अम्अ्य ए:न् & इघ्रु:नि  & इघ्रिक्‌त्यनु  \\ 
\midrule
उङ इन् & घ्रु:न्य  & घ्र:न्त्यनि  \\ 
उङ एऽचि & घ्र:न्सु  & घ्र:न्त्यान्सु   \\ 
उङ ए:न्& घ्र:न्‍नु  & घ्र:न्त्यान्‍नु   \\ 
\bottomrule
\end{tabular}
\end{table}


\begin{table}[H]
\label{imt.vt} \centering
\caption{सकर्मक क्रिया  मम्‍न्य  "सोच्नु"  }
\begin{tabular}{l|l|l|l|l|l|l|l|l|l|l|l|l}  \toprule
&अभूत & भूत & आज्ञार्थक \\ 
उङ &मम्दु &म:म्त \\ 
उङ अम्सु&मम्दुसु &म:म्तसु \\ 
उङ अम्ह्‍याम्&मम्दुनु &म:म्तनु \\ 
इऽचिअ्य &मिमि &मिमिति   \\ 
ओऽचुअ        &मिमु &मिमुतु   \\ 
इक्अ्य&मम्कि &मम्तिकि   \\ 
ओक्अ &मम्क &मम्तक   \\ 
इन्य & इमम्द्‌यु  & इम:म्त्य &मम्दे  \\ 
इन्य अम्सु& इम:म्सु  & इम:म्त्यसु   \\ 
इन्य अम्ह्‍याम्& इम:म्‍नु  & इम:म्त्यनु   \\ 
एऽचि & इमिमि & इमिमिति &मिमिये    \\ 
ए:न् & इमम्‍नि  & इमि:म्त्यनु &मि:म्‍नुये  \\ 
अम्अ्य & मम्द्‌यु  & म:म्त्य  \\ 
अम्सुअ्य & म:म्सु & म:म्त्यसु  \\ 
अम्ह्‍याम्अ्य & म:म्‍नु  & म:म्त्यनु \\ 
\midrule
इन्य, अम्अ्य उ:ङ्‌ &इमम्ङ &इमिमत &मिमये \\ 
एऽचिअ्य/अम्सुअ उ:ङ्‌ &इमम्ङसु &इमिमतसु &मिमसुये \\ 
ए:न्अ्य/अम्ह्‍याम्अ्य उ:ङ्‌ &इमम्ङनु &इमिमतनु &मिमनुये \\ 
इन्य/अम्अ्य इऽचि &इमिमि &इमिमिति    \\ 
इन्य/अम्अ्य ओऽचु &इमिमु &इमिमुतु  &मिमुये  \\ 
इन्य/अम्अ्य इक् &इमम्कि &इमम्तिकि   \\ 
इन्य/अम्अ्य ओक् &इमम्क &इमम्तक  &मम्कये  \\ 
अम्अ्य इन् & इमम् & इमि:म्त्य   \\ 
अम्अ्य एऽचि & इमिमि & इमिमिति    \\ 
अम्अ्य ए:न् & इमम्‍नि  & इमि:म्त्यनु  \\ 
\midrule
उङ इन् & मम्‍न्य  & मम्त्यनि  \\ 
उङ एऽचि & म:म्सु  & मम्त्यान्सु   \\ 
उङ ए:न्& म:म्‍नु  & मम्त्यान्‍नु   \\ 
\bottomrule
\end{tabular}
\end{table}


\begin{table}[H]
\label{int.vt} \centering
\caption{सकर्मक क्रिया  ङैन्य  "डराउनु"  }
\begin{tabular}{l|l|l|l|l|l|l|l|l|l|l|l|l}  \toprule
&अभूत & भूत & आज्ञार्थक \\ 
उङ &ङन्दु &ङ:न्त \\ 
उङ अम्सु&ङन्दुसु &ङ:न्तसु \\ 
उङ अम्ह्‍याम्&ङन्दुनु &ङ:न्तनु \\ 
इऽचिअ्य &ङिऽचि &ङिऽस्ति   \\ 
ओऽचुअ        &ङिऽचु &ङिऽस्तु   \\ 
इक्अ्य&ङैकि &ङैतिकि   \\ 
ओक्अ &ङैक &ङैतक   \\ 
इन्य & इङन्द्‌यु  & इङ:न्त्य &ङन्दे  \\ 
इन्य अम्सु& इङ:न्सु  & इङ:न्त्यसु   \\ 
इन्य अम्ह्‍याम्& इङ:न्‍नु  & इङ:न्त्यनु   \\ 
एऽचि & इङिऽचि & इङिऽस्ति &ङिऽचिये    \\ 
ए:न् & इङैनि  & इङिऽस्त्यनु &ङिऽस्‍नुये  \\ 
अम्अ्य & ङन्द्‌यु  & ङ:न्त्य  \\ 
अम्सुअ्य & ङ:न्सु & ङ:न्त्यसु  \\ 
अम्ह्‍याम्अ्य & ङ:न्‍नु  & ङ:न्त्यनु \\ 
\midrule
इन्य, अम्अ्य उ:ङ्‌ &इङैङ &इङिऽस्त &ङिऽचये \\ 
एऽचिअ्य/अम्सुअ उ:ङ्‌ &इङैङसु &इङिऽस्तसु &ङिऽचसुये \\ 
ए:न्अ्य/अम्ह्‍याम्अ्य उ:ङ्‌ &इङैङनु &इङिऽस्तनु &ङिऽचनुये \\ 
इन्य/अम्अ्य इऽचि &इङिऽचि &इङिऽस्ति    \\ 
इन्य/अम्अ्य ओऽचु &इङिऽचु &इङिऽस्तु  &ङिऽचुये  \\ 
इन्य/अम्अ्य इक् &इङैकि &इङैतिकि   \\ 
इन्य/अम्अ्य ओक् &इङैक &इङैतक  &ङैकये  \\ 
अम्अ्य इन् & इङै & इङिऽस्त्य   \\ 
अम्अ्य एऽचि & इङिऽचि & इङिऽस्ति    \\ 
अम्अ्य ए:न् & इङैनि  & इङिऽस्त्यनु  \\ 
\midrule
उङ इन् & ङैन्य  & ङैत्यनि  \\ 
उङ एऽचि & ङै:सु  & ङैत्यान्सु   \\ 
उङ ए:न्& ङै:नु  & ङैत्यान्‍नु   \\ 
\bottomrule
\end{tabular}
\end{table}


\begin{table}[H]
\label{ilt.vt} \centering
\caption{सकर्मक क्रिया  छल्न्य  "असन्तुष्ट हुनु"  }
\begin{tabular}{l|l|l|l|l|l|l|l|l|l|l|l|l}  \toprule
&अभूत & भूत & आज्ञार्थक \\ 
उङ &छल्दु &छ:ल्त \\ 
उङ अम्सु&छल्दुसु &छ:ल्तसु \\ 
उङ अम्ह्‍याम्&छल्दुनु &छ:ल्तनु \\ 
इऽचिअ्य &छिलि &छिलिति   \\ 
ओऽचुअ        &छिलु &छिलुतु   \\ 
इक्अ्य&छल्कि &छल्तिकि   \\ 
ओक्अ &छल्क &छल्तक   \\ 
इन्य & इछल्द्‌यु  & इछ:ल्त्य &छल्दे  \\ 
इन्य अम्सु& इछ:ल्सु  & इछ:ल्त्यसु   \\ 
इन्य अम्ह्‍याम्& इछ:ल्नु  & इछ:ल्त्यनु   \\ 
एऽचि & इछिलि & इछिलिति &छिलिये    \\ 
ए:न् & इछल्नि  & इछि:ल्त्यनु &छि:ल्नुये  \\ 
अम्अ्य & छल्द्‌यु  & छ:ल्त्य  \\ 
अम्सुअ्य & छ:ल्सु & छ:ल्त्यसु  \\ 
अम्ह्‍याम्अ्य & छ:ल्नु  & छ:ल्त्यनु \\ 
\midrule
इन्य, अम्अ्य उ:ङ्‌ &इछल्ङ &इछिलत &छिलये \\ 
एऽचिअ्य/अम्सुअ उ:ङ्‌ &इछल्ङसु &इछिलतसु &छिलसुये \\ 
ए:न्अ्य/अम्ह्‍याम्अ्य उ:ङ्‌ &इछल्ङनु &इछिलतनु &छिलनुये \\ 
इन्य/अम्अ्य इऽचि &इछिलि &इछिलिति    \\ 
इन्य/अम्अ्य ओऽचु &इछिलु &इछिलुतु  &छिलुये  \\ 
इन्य/अम्अ्य इक् &इछल्कि &इछल्तिकि   \\ 
इन्य/अम्अ्य ओक् &इछल्क &इछल्तक  &छल्कये  \\ 
अम्अ्य इन् & इछल् & इछि:ल्त्य   \\ 
अम्अ्य एऽचि & इछिलि & इछिलिति    \\ 
अम्अ्य ए:न् & इछल्नि  & इछि:ल्त्यनु  \\ 
\midrule
उङ इन् & छल्न्य  & छल्त्यनि  \\ 
उङ एऽचि & छ:ल्सु  & छल्त्यान्सु   \\ 
उङ ए:न्& छ:ल्नु  & छल्त्यान्‍नु   \\ 
\bottomrule
\end{tabular}
\end{table}


\begin{table}[H]
\label{o.vi} \centering
\caption{अकर्मक क्रिया  ह्‍योन्य  "आउनु"  }
\begin{tabular}{l|l|l|l|l|l|l|l|l|l|l|l|l}  \toprule
&अभूत & भूत & आज्ञार्थक \\ 
उ:ङ्‌ &ह्‍योङ &ह्‍योङत \\ 
इऽचि &ह्‍योयि &ह्‍यो:इति   \\ 
ओऽचु &ह्‍योयु &ह्‍यो:इतु   \\ 
इक् &ह्‍योकि &ह्‍योक्‌तिकि   \\ 
ओक् &ह्‍योक &ह्‍योक्‌तक   \\ 
इन् & इह्‍यो & इहोऽत्य &होऽये  \\ 
एऽचि & इह्‍योयि & इह्‍यो:इति &ह्‍यो:इये    \\ 
ए:न् & इह्‍योनि  & इहोत्‍नु &हो:न्ये  \\ 
अम् & ह्‍यो & होऽत्य   \\ 
अम्सु & ह्‍योयि & ह्‍यो:इति     \\ 
अम्ह्‍याम् & ह्‍योनु  & होत्‍नु \\ 
\bottomrule
\end{tabular}
\end{table}


\begin{table}[H]
\label{op.vi} \centering
\caption{अकर्मक क्रिया  स्वा:म्‍न्य  "अघाउनु"  }
\begin{tabular}{l|l|l|l|l|l|l|l|l|l|l|l|l}  \toprule
&अभूत & भूत & आज्ञार्थक \\ 
उ:ङ्‌ &स्वा:म्ङ &स्योपत \\ 
इऽचि &स्योपि &स्योपिति   \\ 
ओऽचु &स्योपु &स्योपुतु   \\ 
इक् &स्वाप्कि &स्वाप्‍तिकि   \\ 
ओक् &स्वाप्क &स्वाप्‍तक   \\ 
इन् & इस्वाप् & इस्योप्‍त्य &स्योप्ये  \\ 
एऽचि & इस्योपि & इस्योपिति &स्योपिये    \\ 
ए:न् & इस्वा:म्‍नि  & इस्योप्‍त्यनु &स्योप्‍नुये  \\ 
अम् & स्वाप् & स्योप्‍त्य   \\ 
अम्सु & स्योपि & स्योपिति   \\ 
अम्ह्‍याम् & स्वा:म्‍नु  & स्योप्‍त्यनु \\ 
\bottomrule
\end{tabular}
\end{table}


\begin{table}[H]
\label{ot.vi} \centering
\caption{अकर्मक क्रिया  य्वा:न्‍न्य  "कस्तो देखिनु"  }
\begin{tabular}{l|l|l|l|l|l|l|l|l|l|l|l|l}  \toprule
&अभूत & भूत & आज्ञार्थक \\ 
उ:ङ्‌ &य्वा:इङ &य्योस्त \\ 
इऽचि &य्योचि &य्योस्ति   \\ 
ओऽचु &य्योचु &य्योस्तु   \\ 
इक् &य्वाह्इकि &य्वाह्इतिकि   \\ 
ओक् &य्वाह्इक &य्वाह्इतक   \\ 
इन् & इय्वा:इ & इय्योस्त्य &य्योच्‍चे  \\ 
एऽचि & इय्योचि & इय्योस्ति &य्योचिये    \\ 
ए:न् & इय्वा:न्‍नि  & इय्योस्त्यनु &य्योस्‍नुये  \\ 
अम् & य्वा:इ & य्योस्त्य   \\ 
अम्सु & य्योचि & य्योस्ति   \\ 
अम्ह्‍याम् & य्वा:न्‍नु  & य्योस्त्यनु \\ 
\bottomrule
\end{tabular}
\end{table}


\begin{table}[H]
\label{ok.vi} \centering
\caption{अकर्मक क्रिया  ओ:न्य  "बास्नु"  }
\begin{tabular}{l|l|l|l|l|l|l|l|l|l|l|l|l}  \toprule
&अभूत & भूत   \\ 
म्य & ओ: & अ्योक्‌त्य   \\ 
\bottomrule
\end{tabular}
\end{table}


\begin{table}[H]
\label{om.vi} \centering
\caption{अकर्मक क्रिया  छ्वाम्‍न्य  "नाच्नु"  }
\begin{tabular}{l|l|l|l|l|l|l|l|l|l|l|l|l}  \toprule
&अभूत & भूत & आज्ञार्थक \\ 
उ:ङ्‌ &छ्वाम्ङ &छ्योमत \\ 
इऽचि &छ्योमि &छ्योमिति   \\ 
ओऽचु &छ्योमु &छ्योमुतु   \\ 
इक् &छ्वाम्कि &छ्वाम्तिकि   \\ 
ओक् &छ्वाम्क &छ्वाम्तक   \\ 
इन् & इछ्वाम् & इछ्यो:म्त्य &छ्यो:म्ये  \\ 
एऽचि & इछ्योमि & इछ्योमिति &छ्योमिये    \\ 
ए:न् & इछ्वाम्‍नि  & इछ्यो:म्त्यनु &छ्यो:म्‍नुये  \\ 
अम् & छ्वाम् & छ्यो:म्त्य   \\ 
अम्सु & छ्योमि & छ्योमिति   \\ 
अम्ह्‍याम् & छ्वाम्‍नु  & छ्यो:म्त्यनु \\ 
\bottomrule
\end{tabular}
\end{table}


\begin{table}[H]
\label{on.vi} \centering
\caption{अकर्मक क्रिया  च्‍वाइन्य  "उर्लनु"  }
\begin{tabular}{l|l|l|l|l|l|l|l|l|l|l|l|l}  \toprule
&अभूत & भूत & आज्ञार्थक \\ 
उ:ङ्‌ &च्‍वाइङ &च्योऽस्त \\ 
इऽचि &च्योऽचि &च्योऽस्ति   \\ 
ओऽचु &च्योऽचु &च्योऽस्तु   \\ 
इक् &च्‍वाइकि &च्‍वाइतिकि   \\ 
ओक् &च्‍वाइक &च्‍वाइतक   \\ 
इन् & इच्‍वाइ & इच्योऽस्त्य &च्योऽचे  \\ 
एऽचि & इच्योऽचि & इच्योऽस्ति &च्योऽचिये    \\ 
ए:न् & इच्‍वाइनि  & इच्योऽस्त्यनु &च्योऽस्‍नुये  \\ 
अम् & च्‍वाइ & च्योऽस्त्य   \\ 
अम्सु & च्योऽचि & च्योऽस्ति   \\ 
अम्ह्‍याम् & च्‍वाइनु  & च्योऽस्त्यनु \\ 
\bottomrule
\end{tabular}
\end{table}


\begin{table}[H]
\label{oŋ.vi} \centering
\caption{अकर्मक क्रिया  खोऽन्य  "माथी आउनु"  }
\begin{tabular}{l|l|l|l|l|l|l|l|l|l|l|l|l}  \toprule
&अभूत & भूत & आज्ञार्थक \\ 
उ:ङ्‌ &खोऽङ &ख्योङत \\ 
इऽचि &ख्योङि &ख्योङिति   \\ 
ओऽचु &ख्योङु &ख्योङुतु   \\ 
इक् &खोङ्‌कि &खोङ्‌तिकि   \\ 
ओक् &खोङ्‌क &खोङ्‌तक   \\ 
इन् & इखोङ्‌ & इख्यो:ङ्‌त्य &ख्यो:ङ्‌ये  \\ 
एऽचि & इख्योङि & इख्योङिति &ख्योङिये    \\ 
ए:न् & इखोऽनि  & इख्यो:ङ्‌त्यनु &ख्यो:ङ्‌नुये  \\ 
अम् & खोङ्‌ & ख्यो:ङ्‌त्य   \\ 
अम्सु & ख्योङि & ख्योङिति   \\ 
अम्ह्‍याम् & खोऽनु  & ख्यो:ङ्‌त्यनु \\ 
\bottomrule
\end{tabular}
\end{table}


\begin{table}[H]
\label{or.vi} \centering
\caption{अकर्मक क्रिया  भ्वार्न्य  "बढ्नु"  }
\begin{tabular}{l|l|l|l|l|l|l|l|l|l|l|l|l}  \toprule
&अभूत & भूत & आज्ञार्थक \\ 
उ:ङ्‌ &भ्वार्ङ &भ्योरत \\ 
इऽचि &भ्योरि &भ्योरिति   \\ 
ओऽचु &भ्योरु &भ्योरुतु   \\ 
इक् &भ्वार्कि &भ्वार्तिकि   \\ 
ओक् &भ्वार्क &भ्वार्तक   \\ 
इन् & इभ्वार् & इभ्यो:र्त्य &भ्यो:र्‍ये  \\ 
एऽचि & इभ्योरि & इभ्योरिति &भ्योरिये    \\ 
ए:न् & इभ्वार्नि  & इभ्यो:र्त्यनु &भ्यो:र्नुये  \\ 
अम् & भ्वार् & भ्यो:र्त्य   \\ 
अम्सु & भ्योरि & भ्योरिति   \\ 
अम्ह्‍याम् & भ्वार्नु  & भ्यो:र्त्यनु \\ 
\bottomrule
\end{tabular}
\end{table}


\begin{table}[H]
\label{ol.vi} \centering
\caption{अकर्मक क्रिया  घ्वाल्न्य  "बढ्नु"  }
\begin{tabular}{l|l|l|l|l|l|l|l|l|l|l|l|l}  \toprule
&अभूत & भूत   \\ 
म्य & घ्वाल् & घ्यो:ल्त्य   \\ 
\bottomrule
\end{tabular}
\end{table}


\begin{table}[H]
\label{o.vt} \centering
\caption{सकर्मक क्रिया  फ्ल्योन्य  "सघाउनु"  }
\begin{tabular}{l|l|l|l|l|l|l|l|l|l|l|l|l}  \toprule
&अभूत & भूत & आज्ञार्थक \\ 
उङ &फ्ल्योङ &फ्ल्यो:ङ्‌त \\ 
इऽचिअ्य &फ्ल्योयि &फ्ल्यो:इति   \\ 
ओऽचुअ &फ्ल्योयु &फ्ल्यो:इतु   \\ 
इक्अ्य &फ्ल्योकि &फ्ल्योक्‌तिकि   \\ 
ओक्अ &फ्ल्योक &फ्ल्योक्‌तक   \\ 
इन्य & इफ्ल्यो & इफ्ल्योत्य &फ्ल्योये  \\ 
एऽचिअ्य & इफ्ल्योयि & इफ्ल्यो:इति &फ्ल्यो:इये    \\ 
ए:न्अ्य & इफ्ल्योनि  & इफ्लोत्‍नु &फ्लो:न्ये  \\ 
अम्अ्य & फ्ल्यो & फ्ल्योत्य   \\ 
अम्सुअ & फ्ल्योसु & फ्ल्योत्सु     \\ 
अम्ह्‍याम्अ्य & फ्ल्योनु  & फ्ल्योत्‍नु \\ 
\midrule
इन्य/अम्अ्य उ:ङ्‌&इफ्ल्योङ &इफ्ल्योङत &फ्ल्योङये \\ 
एऽचिअ्य/अम्सुअ उ:ङ्‌ &इफ्ल्योङसु &इफ्ल्योङतसु &फ्ल्योङसुये \\ 
ए:न्अ्य/अम्ह्‍याम्अ्य उ:ङ्‌ &इफ्ल्योङनु &इफ्ल्योङतनु &फ्ल्योङनुये \\ 
इन्य/अम्अ्य इऽचि &इफ्ल्योयि &इफ्ल्यो:इति    \\ 
इन्य/अम्अ्य ओऽचु &इफ्ल्योयु &इफ्ल्यो:इतु  &फ्ल्योइये  \\ 
इन्य/अम्अ्य इक् &इफ्ल्योकि &इफ्ल्योक्‌तिकि   \\ 
इन्य/अम्अ्य ओक् &इफ्ल्योक &इफ्ल्योक्‌तक  &फ्ल्योकये  \\ 
अम्अ्य इन् & इफ्ल्यो & इफ्लोऽत्य   \\ 
अम्अ्य एऽचि & इफ्ल्योयि & इफ्ल्यो:इति     \\ 
अम्अ्य ए:न् & इफ्ल्योनि  & इफ्लोत्‍नु  \\ 
\midrule
उङ इन् & फ्ल्योन्य  & फ्ल्यो:न्त्यनि  \\ 
उङ एऽचि & फ्ल्यो:न्सु  & फ्ल्यो:न्त्यान्सु   \\ 
उङ ए:न्& फ्ल्यो:न्‍नु  & फ्ल्यो:न्त्यान्‍नु   \\ 
\bottomrule
\end{tabular}
\end{table}


\begin{table}[H]
\label{op.vt} \centering
\caption{सकर्मक क्रिया  अ्वा:म्‍न्य  "हान्नु"  }
\begin{tabular}{l|l|l|l|l|l|l|l|l|l|l|l|l}  \toprule
&अभूत & भूत & आज्ञार्थक \\ 
उङ &ओबु &ओबुत \\ 
उङ अम्सु &ओबुसु &ओबुतसु \\ 
उङ अम्ह्‍याम् &ओबुनु &ओबुतनु \\ 
इऽचिअ्य &अ्योपि &अ्योपिति   \\ 
ओऽचुअ &अ्योपु &अ्योपुतु   \\ 
इक्अ्य &अ्वाप्कि &अ्वाप्‍तिकि   \\ 
ओक्अ &अ्वाप्क &अ्वाप्‍तक   \\ 
इन्य अम् & इअ्योऽब्यु  & इअ्योऽप्‍त्य &अ्योऽबे  \\ 
इन्य अम्सु & इअ्योऽप्सु  & इअ्योऽप्‍त्यसु   \\ 
इन्य अम्ह्‍याम् & इअ्योऽप्‍नु  & इअ्योऽप्‍त्यनु   \\ 
एऽचिअ्य & इअ्योपि & इअ्योपिति &अ्योपिये    \\ 
ए:न्अ्य & इअ्वा:म्‍नि  & इअ्योप्‍त्यनु &अ्योप्‍नुये  \\ 
अम्अ्य & अ्योऽब्यु  & अ्योऽप्‍त्य  \\ 
अम्सुअ & अ्योऽप्सु & अ्योऽप्‍त्यसु  \\ 
अम्ह्‍याम्अ्य & अ्योऽप्‍नु  & अ्योऽप्‍त्यनु \\ 
\midrule
इन्य/अम्अ्य उ:ङ्‌&इअ्वा:म्ङ & इअ्योपत &अ्योपये \\ 
एऽचिअ्य/अम्सुअ उ:ङ्‌ &इअ्वा:म्ङसु & इअ्योपतसु &अ्योपसुये \\ 
ए:न्अ्य/अम्ह्‍याम्अ्य उ:ङ्‌ &इअ्वा:म्ङनु & इअ्योपतनु &अ्योपनुये \\ 
इन्य/अम्अ्य इऽचि & इअ्योपि & इअ्योपिति    \\ 
इन्य/अम्अ्य ओऽचु & इअ्योपु & इअ्योपुतु  &अ्योपुये  \\ 
इन्य/अम्अ्य इक् & इअ्वाप्कि & इअ्वाप्‍तिकि   \\ 
इन्य/अम्अ्य ओक् & इअ्वाप्क & इअ्वाप्‍तक  &अ्वाप्कये  \\ 
अम्अ्य इन् & इअ्वाप् & इअ्योप्‍त्य   \\ 
अम्अ्य एऽचि & इअ्योपि & इअ्योपिति    \\ 
अम्अ्य ए:न् & इअ्वा:म्‍नि  & इअ्योप्‍त्यनु  \\ 
\midrule
उङ इन् & अ्वा:म्‍न्य  & अ्वा:म्त्यनि  \\ 
उङ एऽचि & अ्वा:म्सु  & अ्वा:म्त्यान्सु   \\ 
उङ ए:न्& अ्वा:म्‍नु  & अ्वा:म्त्यान्‍नु   \\ 
\bottomrule
\end{tabular}
\end{table}


\begin{table}[H]
\label{ot.vt} \centering
\caption{सकर्मक क्रिया  र्‍वा:न्‍न्य  "पुग्नु"  }
\begin{tabular}{l|l|l|l|l|l|l|l|l|l|l|l|l}  \toprule
&अभूत & भूत & आज्ञार्थक \\ 
उङ &रोदु &रो:त \\ 
उङ अम्सु &रोदुसु &रो:तसु \\ 
उङ अम्ह्‍याम् &रोदुनु &रो:तनु \\ 
इऽचिअ्य &र्‍योचि &र्‍योस्ति   \\ 
ओऽचुअ &र्‍योचु &र्‍योस्तु   \\ 
इक्अ्य &र्‍वाह्इकि &र्‍वाह्इतिकि   \\ 
ओक्अ &र्‍वाह्इक &र्‍वाह्इतक   \\ 
इन्य अम् & इर्‍योऽद्‌यु  & इर्‍यो:त्य &र्‍योऽदे  \\ 
इन्य अम्सु & इर्‍योऽत्सु  & इर्‍यो:त्यसु   \\ 
इन्य अम्ह्‍याम् & इर्‍योऽत्‍नु  & इर्‍यो:त्यनु   \\ 
एऽचिअ्य & इर्‍योचि & इर्‍योस्ति &र्‍योचिये    \\ 
ए:न्अ्य & इर्‍वा:न्‍नि  & इर्‍योस्त्यनु &र्‍योस्‍नुये  \\ 
अम्अ्य & र्‍योऽद्‌यु  & र्‍यो:त्य  \\ 
अम्सुअ & र्‍योऽत्सु & र्‍यो:त्यसु  \\ 
अम्ह्‍याम्अ्य & र्‍योऽत्‍नु  & र्‍यो:त्यनु \\ 
\midrule
इन्य/अम्अ्य उ:ङ्‌&इर्‍वा:इङ & इर्‍योस्त &र्‍योचये \\ 
एऽचिअ्य/अम्सुअ उ:ङ्‌ &इर्‍वा:इङसु & इर्‍योस्तसु &र्‍योचसुये \\ 
ए:न्अ्य/अम्ह्‍याम्अ्य उ:ङ्‌ &इर्‍वा:इङनु & इर्‍योस्तनु &र्‍योचनुये \\ 
इन्य/अम्अ्य इऽचि & इर्‍योचि & इर्‍योस्ति    \\ 
इन्य/अम्अ्य ओऽचु & इर्‍योचु & इर्‍योस्तु  &र्‍योचुये  \\ 
इन्य/अम्अ्य इक् & इर्‍वाह्इकि & इर्‍वाह्इतिकि   \\ 
इन्य/अम्अ्य ओक् & इर्‍वाह्इक & इर्‍वाह्इतक  &र्‍वाह्इकये  \\ 
अम्अ्य इन् & इर्‍वा:इ & इर्‍योस्त्य   \\ 
अम्अ्य एऽचि & इर्‍योचि & इर्‍योस्ति    \\ 
अम्अ्य ए:न् & इर्‍वा:न्‍नि  & इर्‍योस्त्यनु  \\ 
\midrule
उङ इन् & र्‍वा:न्‍न्य  & र्‍वा:न्त्यनि  \\ 
उङ एऽचि & र्‍वा:न्सु  & र्‍वा:न्त्यान्सु   \\ 
उङ ए:न्& र्‍वा:न्‍नु  & र्‍वा:न्त्यान्‍नु   \\ 
\bottomrule
\end{tabular}
\end{table}


\begin{table}[H]
\label{ok.vt} \centering
\caption{सकर्मक क्रिया  फ्रो:न्य  "फुकाउनु"  }
\begin{tabular}{l|l|l|l|l|l|l|l|l|l|l|l|l}  \toprule
&अभूत & भूत & आज्ञार्थक \\ 
उङ &फ्रोगु &फ्रोगुत \\ 
उङ अम्सु &फ्रोगुसु &फ्रोगुतसु \\ 
उङ अम्ह्‍याम् &फ्रोगुनु &फ्रोगुतनु \\ 
इऽचिअ्य &फ्र्योकि &फ्र्योकिति   \\ 
ओऽचुअ &फ्र्योकु &फ्र्योकुतु   \\ 
इक्अ्य &फ्रोक्‌कि &फ्रोक्‌तिकि   \\ 
ओक्अ &फ्रोक्‌क &फ्रोक्‌तक   \\ 
इन्य अम् & इफ्र्योऽग्यु  & इफ्र्योऽक्‌त्य &फ्र्योऽगे  \\ 
इन्य अम्सु & इफ्र्योऽक्सु  & इफ्र्योऽक्‌त्यसु   \\ 
इन्य अम्ह्‍याम् & इफ्र्योऽक्‍नु  & इफ्र्योऽक्‌त्यनु   \\ 
एऽचिअ्य & इफ्र्योकि & इफ्र्योकिति &फ्र्योकिये    \\ 
ए:न्अ्य & इफ्रो:नि  & इफ्र्योक्‌त्यनु &फ्र्योक्‍नुये  \\ 
अम्अ्य & फ्र्योऽग्यु  & फ्र्योऽक्‌त्य  \\ 
अम्सुअ & फ्र्योऽक्सु & फ्र्योऽक्‌त्यसु  \\ 
अम्ह्‍याम्अ्य & फ्र्योऽक्‍नु  & फ्र्योऽक्‌त्यनु \\ 
\midrule
इन्य/अम्अ्य उ:ङ्‌&इफ्रो:ङ & इफ्र्योकत &फ्र्योकये \\ 
एऽचिअ्य/अम्सुअ उ:ङ्‌ &इफ्रो:ङसु & इफ्र्योकतसु &फ्र्योकसुये \\ 
ए:न्अ्य/अम्ह्‍याम्अ्य उ:ङ्‌ &इफ्रो:ङनु & इफ्र्योकतनु &फ्र्योकनुये \\ 
इन्य/अम्अ्य इऽचि & इफ्र्योकि & इफ्र्योकिति    \\ 
इन्य/अम्अ्य ओऽचु & इफ्र्योकु & इफ्र्योकुतु  &फ्र्योकुये  \\ 
इन्य/अम्अ्य इक् & इफ्रोक्‌कि & इफ्रोक्‌तिकि   \\ 
इन्य/अम्अ्य ओक् & इफ्रोक्‌क & इफ्रोक्‌तक  &फ्रोक्‌कये  \\ 
अम्अ्य इन् & इफ्रो: & इफ्र्योक्‌त्य   \\ 
अम्अ्य एऽचि & इफ्र्योकि & इफ्र्योकिति    \\ 
अम्अ्य ए:न् & इफ्रो:नि  & इफ्र्योक्‌त्यनु  \\ 
\midrule
उङ इन् & फ्रो:न्य  & फ्रो:न्त्यनि  \\ 
उङ एऽचि & फ्रो:न्सु  & फ्रो:न्त्यान्सु   \\ 
उङ ए:न्& फ्रो:न्‍नु  & फ्रो:न्त्यान्‍नु   \\ 
\bottomrule
\end{tabular}
\end{table}


\begin{table}[H]
\label{om.vt} \centering
\caption{सकर्मक क्रिया  ल्वाम्‍न्य  "खोज्नु"  }
\begin{tabular}{l|l|l|l|l|l|l|l|l|l|l|l|l}  \toprule
&अभूत & भूत & आज्ञार्थक \\ 
उङ &लोमु &लोमुत \\ 
उङ अम्सु &लोमुसु &लोमुतसु \\ 
उङ अम्ह्‍याम् &लोमुनु &लोमुतनु \\ 
इऽचिअ्य &ल्योमि &ल्योमिति   \\ 
ओऽचुअ &ल्योमु &ल्योमुतु   \\ 
इक्अ्य &ल्वाम्कि &ल्वाम्तिकि   \\ 
ओक्अ &ल्वाम्क &ल्वाम्तक   \\ 
इन्य अम् & इल्योऽम्यु  & इल्यो:म्त्य &ल्योऽमे  \\ 
इन्य अम्सु & इल्यो:म्सु  & इल्यो:म्त्यसु   \\ 
इन्य अम्ह्‍याम् & इल्यो:म्‍नु  & इल्यो:म्त्यनु   \\ 
एऽचिअ्य & इल्योमि & इल्योमिति &ल्योमिये    \\ 
ए:न्अ्य & इल्वाम्‍नि  & इल्यो:म्त्यनु &ल्यो:म्‍नुये  \\ 
अम्अ्य & ल्योऽम्यु  & ल्यो:म्त्य  \\ 
अम्सुअ & ल्यो:म्सु & ल्यो:म्त्यसु  \\ 
अम्ह्‍याम्अ्य & ल्यो:म्‍नु  & ल्यो:म्त्यनु \\ 
\midrule
इन्य/अम्अ्य उ:ङ्‌&इल्वाम्ङ & इल्योमत &ल्योमये \\ 
एऽचिअ्य/अम्सुअ उ:ङ्‌ &इल्वाम्ङसु & इल्योमतसु &ल्योमसुये \\ 
ए:न्अ्य/अम्ह्‍याम्अ्य उ:ङ्‌ &इल्वाम्ङनु & इल्योमतनु &ल्योमनुये \\ 
इन्य/अम्अ्य इऽचि & इल्योमि & इल्योमिति    \\ 
इन्य/अम्अ्य ओऽचु & इल्योमु & इल्योमुतु  &ल्योमुये  \\ 
इन्य/अम्अ्य इक् & इल्वाम्कि & इल्वाम्तिकि   \\ 
इन्य/अम्अ्य ओक् & इल्वाम्क & इल्वाम्तक  &ल्वाम्कये  \\ 
अम्अ्य इन् & इल्वाम् & इल्यो:म्त्य   \\ 
अम्अ्य एऽचि & इल्योमि & इल्योमिति    \\ 
अम्अ्य ए:न् & इल्वाम्‍नि  & इल्यो:म्त्यनु  \\ 
\midrule
उङ इन् & ल्वाम्‍न्य  & ल्वाम्त्यनि  \\ 
उङ एऽचि & ल्वा:म्सु  & ल्वाम्त्यान्सु   \\ 
उङ ए:न्& ल्वा:म्‍नु  & ल्वाम्त्यान्‍नु   \\ 
\bottomrule
\end{tabular}
\end{table}


\begin{table}[H]
\label{oŋ.vt} \centering
\caption{सकर्मक क्रिया  चोऽन्य  "चुली पार्नु"  }
\begin{tabular}{l|l|l|l|l|l|l|l|l|l|l|l|l}  \toprule
&अभूत & भूत & आज्ञार्थक \\ 
उङ &चोङु &चोङुत \\ 
उङ अम्सु &चोङुसु &चोङुतसु \\ 
उङ अम्ह्‍याम् &चोङुनु &चोङुतनु \\ 
इऽचिअ्य &च्योङि &च्योङिति   \\ 
ओऽचुअ &च्योङु &च्योङुतु   \\ 
इक्अ्य &चोङ्‌कि &चोङ्‌तिकि   \\ 
ओक्अ &चोङ्‌क &चोङ्‌तक   \\ 
इन्य अम् & इच्योऽङ्‌यु  & इच्यो:ङ्‌त्य &च्योऽङे  \\ 
इन्य अम्सु & इच्यो:ङ्‌सु  & इच्यो:ङ्‌त्यसु   \\ 
इन्य अम्ह्‍याम् & इच्यो:ङ्‌नु  & इच्यो:ङ्‌त्यनु   \\ 
एऽचिअ्य & इच्योङि & इच्योङिति &च्योङिये    \\ 
ए:न्अ्य & इचोऽनि  & इच्यो:ङ्‌त्यनु &च्यो:ङ्‌नुये  \\ 
अम्अ्य & च्योऽङ्‌यु  & च्यो:ङ्‌त्य  \\ 
अम्सुअ & च्यो:ङ्‌सु & च्यो:ङ्‌त्यसु  \\ 
अम्ह्‍याम्अ्य & च्यो:ङ्‌नु  & च्यो:ङ्‌त्यनु \\ 
\midrule
इन्य/अम्अ्य उ:ङ्‌&इचोऽङ & इच्योङत &च्योङये \\ 
एऽचिअ्य/अम्सुअ उ:ङ्‌ &इचोऽङसु & इच्योङतसु &च्योङसुये \\ 
ए:न्अ्य/अम्ह्‍याम्अ्य उ:ङ्‌ &इचोऽङनु & इच्योङतनु &च्योङनुये \\ 
इन्य/अम्अ्य इऽचि & इच्योङि & इच्योङिति    \\ 
इन्य/अम्अ्य ओऽचु & इच्योङु & इच्योङुतु  &च्योङुये  \\ 
इन्य/अम्अ्य इक् & इचोङ्‌कि & इचोङ्‌तिकि   \\ 
इन्य/अम्अ्य ओक् & इचोङ्‌क & इचोङ्‌तक  &चोङ्‌कये  \\ 
अम्अ्य इन् & इचोङ्‌ & इच्यो:ङ्‌त्य   \\ 
अम्अ्य एऽचि & इच्योङि & इच्योङिति    \\ 
अम्अ्य ए:न् & इचोऽनि  & इच्यो:ङ्‌त्यनु  \\ 
\midrule
उङ इन् & चोऽन्य  & चोन्त्यनि  \\ 
उङ एऽचि & चो:न्सु  & चोन्त्यान्सु   \\ 
उङ ए:न्& चो:न्‍नु  & चोन्त्यान्‍नु   \\ 
\bottomrule
\end{tabular}
\end{table}


\begin{table}[H]
\label{or.vt} \centering
\caption{सकर्मक क्रिया  ख्‍वार्न्य  "भुट्नु"  }
\begin{tabular}{l|l|l|l|l|l|l|l|l|l|l|l|l}  \toprule
&अभूत & भूत & आज्ञार्थक \\ 
उङ &खोरु &खोरुत \\ 
उङ अम्सु &खोरुसु &खोरुतसु \\ 
उङ अम्ह्‍याम् &खोरुनु &खोरुतनु \\ 
इऽचिअ्य &ख्योरि &ख्योरिति   \\ 
ओऽचुअ &ख्योरु &ख्योरुतु   \\ 
इक्अ्य &ख्‍वार्कि &ख्‍वार्तिकि   \\ 
ओक्अ &ख्‍वार्क &ख्‍वार्तक   \\ 
इन्य अम् & इख्योऽर्‍यु  & इख्यो:र्त्य &ख्योऽरे  \\ 
इन्य अम्सु & इख्यो:र्सु  & इख्यो:र्त्यसु   \\ 
इन्य अम्ह्‍याम् & इख्यो:र्नु  & इख्यो:र्त्यनु   \\ 
एऽचिअ्य & इख्योरि & इख्योरिति &ख्योरिये    \\ 
ए:न्अ्य & इख्‍वार्नि  & इख्यो:र्त्यनु &ख्यो:र्नुये  \\ 
अम्अ्य & ख्योऽर्‍यु  & ख्यो:र्त्य  \\ 
अम्सुअ & ख्यो:र्सु & ख्यो:र्त्यसु  \\ 
अम्ह्‍याम्अ्य & ख्यो:र्नु  & ख्यो:र्त्यनु \\ 
\midrule
इन्य/अम्अ्य उ:ङ्‌&इख्‍वार्ङ & इख्योरत &ख्योरये \\ 
एऽचिअ्य/अम्सुअ उ:ङ्‌ &इख्‍वार्ङसु & इख्योरतसु &ख्योरसुये \\ 
ए:न्अ्य/अम्ह्‍याम्अ्य उ:ङ्‌ &इख्‍वार्ङनु & इख्योरतनु &ख्योरनुये \\ 
इन्य/अम्अ्य इऽचि & इख्योरि & इख्योरिति    \\ 
इन्य/अम्अ्य ओऽचु & इख्योरु & इख्योरुतु  &ख्योरुये  \\ 
इन्य/अम्अ्य इक् & इख्‍वार्कि & इख्‍वार्तिकि   \\ 
इन्य/अम्अ्य ओक् & इख्‍वार्क & इख्‍वार्तक  &ख्‍वार्कये  \\ 
अम्अ्य इन् & इख्‍वार् & इख्यो:र्त्य   \\ 
अम्अ्य एऽचि & इख्योरि & इख्योरिति    \\ 
अम्अ्य ए:न् & इख्‍वार्नि  & इख्यो:र्त्यनु  \\ 
\midrule
उङ इन् & ख्‍वार्न्य  & ख्‍वार्त्यनि  \\ 
उङ एऽचि & ख्‍वा:र्सु  & ख्‍वार्त्यान्सु   \\ 
उङ ए:न्& ख्‍वा:र्नु  & ख्‍वार्त्यान्‍नु   \\ 
\bottomrule
\end{tabular}
\end{table}


\begin{table}[H]
\label{ol.vt} \centering
\caption{सकर्मक क्रिया  त्वाल्न्य  "दलादल गर्नु"  }
\begin{tabular}{l|l|l|l|l|l|l|l|l|l|l|l|l}  \toprule
&अभूत & भूत & आज्ञार्थक \\ 
उङ &तोलु &तोलुत \\ 
उङ अम्सु &तोलुसु &तोलुतसु \\ 
उङ अम्ह्‍याम् &तोलुनु &तोलुतनु \\ 
इऽचिअ्य &त्योलि &त्योलिति   \\ 
ओऽचुअ &त्योलु &त्योलुतु   \\ 
इक्अ्य &त्वाल्कि &त्वाल्तिकि   \\ 
ओक्अ &त्वाल्क &त्वाल्तक   \\ 
इन्य अम् & इत्योऽल्यु  & इत्यो:ल्त्य &त्योऽले  \\ 
इन्य अम्सु & इत्यो:ल्सु  & इत्यो:ल्त्यसु   \\ 
इन्य अम्ह्‍याम् & इत्यो:ल्नु  & इत्यो:ल्त्यनु   \\ 
एऽचिअ्य & इत्योलि & इत्योलिति &त्योलिये    \\ 
ए:न्अ्य & इत्वाल्नि  & इत्यो:ल्त्यनु &त्यो:ल्नुये  \\ 
अम्अ्य & त्योऽल्यु  & त्यो:ल्त्य  \\ 
अम्सुअ & त्यो:ल्सु & त्यो:ल्त्यसु  \\ 
अम्ह्‍याम्अ्य & त्यो:ल्नु  & त्यो:ल्त्यनु \\ 
\midrule
इन्य/अम्अ्य उ:ङ्‌&इत्वाल्ङ & इत्योलत &त्योलये \\ 
एऽचिअ्य/अम्सुअ उ:ङ्‌ &इत्वाल्ङसु & इत्योलतसु &त्योलसुये \\ 
ए:न्अ्य/अम्ह्‍याम्अ्य उ:ङ्‌ &इत्वाल्ङनु & इत्योलतनु &त्योलनुये \\ 
इन्य/अम्अ्य इऽचि & इत्योलि & इत्योलिति    \\ 
इन्य/अम्अ्य ओऽचु & इत्योलु & इत्योलुतु  &त्योलुये  \\ 
इन्य/अम्अ्य इक् & इत्वाल्कि & इत्वाल्तिकि   \\ 
इन्य/अम्अ्य ओक् & इत्वाल्क & इत्वाल्तक  &त्वाल्कये  \\ 
अम्अ्य इन् & इत्वाल् & इत्यो:ल्त्य   \\ 
अम्अ्य एऽचि & इत्योलि & इत्योलिति    \\ 
अम्अ्य ए:न् & इत्वाल्नि  & इत्यो:ल्त्यनु  \\ 
\midrule
उङ इन् & त्वाल्न्य  & त्वाल्त्यनि  \\ 
उङ एऽचि & त्वा:ल्सु  & त्वाल्त्यान्सु   \\ 
उङ ए:न्& त्वा:ल्नु  & त्वाल्त्यान्‍नु   \\ 
\bottomrule
\end{tabular}
\end{table}


\begin{table}[H]
\label{opt.vt} \centering
\caption{सकर्मक क्रिया  स्वा:म्‍न्य  "जोड्ले पिट्नु"  }
\begin{tabular}{l|l|l|l|l|l|l|l|l|l|l|l|l}  \toprule
&अभूत & भूत & आज्ञार्थक \\ 
उङ &स्वाप्‍तु &स्वाप्‍त \\ 
उङ अम्सु&स्वाप्‍तुसु &स्वाप्‍तसु \\ 
उङ अम्ह्‍याम्&स्वाप्‍तुनु &स्वाप्‍तनु \\ 
इऽचिअ्य &स्योपि &स्योपिति   \\ 
ओऽचुअ        &स्योपु &स्योपुतु   \\ 
इक्अ्य&स्वाप्कि &स्वाप्‍तिकि   \\ 
ओक्अ &स्वाप्क &स्वाप्‍तक   \\ 
इन्य & इस्वाप्‍त्यु  & इस्वाप्‍त्य &स्वाप्‍ते  \\ 
इन्य अम्सु& इस्वाप्सु  & इस्वाप्‍त्यसु   \\ 
इन्य अम्ह्‍याम्& इस्वाप्‍नु  & इस्वाप्‍त्यनु   \\ 
एऽचि & इस्योपि & इस्योपिति &स्योपिये    \\ 
ए:न् & इस्वा:म्‍नि  & इस्योप्‍त्यनु &स्योप्‍नुये  \\ 
अम्अ्य & स्वाप्‍त्यु  & स्वाप्‍त्य  \\ 
अम्सुअ्य & स्वाप्सु & स्वाप्‍त्यसु  \\ 
अम्ह्‍याम्अ्य & स्वाप्‍नु  & स्वाप्‍त्यनु \\ 
\midrule
इन्य, अम्अ्य उ:ङ्‌ &इस्वा:म्ङ &इस्योपत &स्योपये \\ 
एऽचिअ्य/अम्सुअ उ:ङ्‌ &इस्वा:म्ङसु &इस्योपतसु &स्योपसुये \\ 
ए:न्अ्य/अम्ह्‍याम्अ्य उ:ङ्‌ &इस्वा:म्ङनु &इस्योपतनु &स्योपनुये \\ 
इन्य/अम्अ्य इऽचि &इस्योपि &इस्योपिति    \\ 
इन्य/अम्अ्य ओऽचु &इस्योपु &इस्योपुतु  &स्योपुये  \\ 
इन्य/अम्अ्य इक् &इस्वाप्कि &इस्वाप्‍तिकि   \\ 
इन्य/अम्अ्य ओक् &इस्वाप्क &इस्वाप्‍तक  &स्वाप्कये  \\ 
अम्अ्य इन् & इस्वाप् & इस्योप्‍त्य   \\ 
अम्अ्य एऽचि & इस्योपि & इस्योपिति    \\ 
अम्अ्य ए:न् & इस्वा:म्‍नि  & इस्योप्‍त्यनु  \\ 
\midrule
उङ इन् & स्वा:म्‍न्य  & स्वा:म्त्यनि  \\ 
उङ एऽचि & स्वा:म्सु  & स्वा:म्त्यान्सु   \\ 
उङ ए:न्& स्वा:म्‍नु  & स्वा:म्त्यान्‍नु   \\ 
\bottomrule
\end{tabular}
\end{table}


\begin{table}[H]
\label{ott.vt} \centering
\caption{सकर्मक क्रिया  भ्र्वा:न्‍न्य  "बोलाउनु"  }
\begin{tabular}{l|l|l|l|l|l|l|l|l|l|l|l|l}  \toprule
&अभूत & भूत & आज्ञार्थक \\ 
उङ &भ्र्वात्तु &भ्र्वात्त \\ 
उङ अम्सु&भ्र्वात्तुसु &भ्र्वात्तसु \\ 
उङ अम्ह्‍याम्&भ्र्वात्तुनु &भ्र्वात्तनु \\ 
इऽचिअ्य &भ्र्योचि &भ्र्योस्ति   \\ 
ओऽचुअ        &भ्र्योचु &भ्र्योस्तु   \\ 
इक्अ्य&भ्र्वाह्इकि &भ्र्वाह्इतिकि   \\ 
ओक्अ &भ्र्वाह्इक &भ्र्वाह्इतक   \\ 
इन्य & इभ्र्वात्त्यु  & इभ्र्वात्त्य &भ्र्वात्ते  \\ 
इन्य अम्सु& इभ्र्वात्सु  & इभ्र्वात्त्यसु   \\ 
इन्य अम्ह्‍याम्& इभ्र्वात्‍नु  & इभ्र्वात्त्यनु   \\ 
एऽचि & इभ्र्योचि & इभ्र्योस्ति &भ्र्योचिये    \\ 
ए:न् & इभ्र्वा:न्‍नि  & इभ्र्योस्त्यनु &भ्र्योस्‍नुये  \\ 
अम्अ्य & भ्र्वात्त्यु  & भ्र्वात्त्य  \\ 
अम्सुअ्य & भ्र्वात्सु & भ्र्वात्त्यसु  \\ 
अम्ह्‍याम्अ्य & भ्र्वात्‍नु  & भ्र्वात्त्यनु \\ 
\midrule
इन्य, अम्अ्य उ:ङ्‌ &इभ्र्वा:इङ &इभ्र्योस्त &भ्र्योचये \\ 
एऽचिअ्य/अम्सुअ उ:ङ्‌ &इभ्र्वा:इङसु &इभ्र्योस्तसु &भ्र्योचसुये \\ 
ए:न्अ्य/अम्ह्‍याम्अ्य उ:ङ्‌ &इभ्र्वा:इङनु &इभ्र्योस्तनु &भ्र्योचनुये \\ 
इन्य/अम्अ्य इऽचि &इभ्र्योचि &इभ्र्योस्ति    \\ 
इन्य/अम्अ्य ओऽचु &इभ्र्योचु &इभ्र्योस्तु  &भ्र्योचुये  \\ 
इन्य/अम्अ्य इक् &इभ्र्वाह्इकि &इभ्र्वाह्इतिकि   \\ 
इन्य/अम्अ्य ओक् &इभ्र्वाह्इक &इभ्र्वाह्इतक  &भ्र्वाह्इकये  \\ 
अम्अ्य इन् & इभ्र्वा:इ & इभ्र्योस्त्य   \\ 
अम्अ्य एऽचि & इभ्र्योचि & इभ्र्योस्ति    \\ 
अम्अ्य ए:न् & इभ्र्वा:न्‍नि  & इभ्र्योस्त्यनु  \\ 
\midrule
उङ इन् & भ्र्वा:न्‍न्य  & भ्र्वा:न्त्यनि  \\ 
उङ एऽचि & भ्र्वा:न्सु  & भ्र्वा:न्त्यान्सु   \\ 
उङ ए:न्& भ्र्वा:न्‍नु  & भ्र्वा:न्त्यान्‍नु   \\ 
\bottomrule
\end{tabular}
\end{table}


\begin{table}[H]
\label{okt.vt} \centering
\caption{सकर्मक क्रिया  हो:न्य  "हकार्नु"  }
\begin{tabular}{l|l|l|l|l|l|l|l|l|l|l|l|l}  \toprule
&अभूत & भूत & आज्ञार्थक \\ 
उङ &होक्‌तु &होक्‌त \\ 
उङ अम्सु&होक्‌तुसु &होक्‌तसु \\ 
उङ अम्ह्‍याम्&होक्‌तुनु &होक्‌तनु \\ 
इऽचिअ्य &ह्‍योकि &ह्‍योकिति   \\ 
ओऽचुअ        &ह्‍योकु &ह्‍योकुतु   \\ 
इक्अ्य&होक्‌कि &होक्‌तिकि   \\ 
ओक्अ &होक्‌क &होक्‌तक   \\ 
इन्य & इहोक्‌त्यु  & इहोक्‌त्य &होक्‌ते  \\ 
इन्य अम्सु& इहोक्सु  & इहोक्‌त्यसु   \\ 
इन्य अम्ह्‍याम्& इहोक्‍नु  & इहोक्‌त्यनु   \\ 
एऽचि & इह्‍योकि & इह्‍योकिति &ह्‍योकिये    \\ 
ए:न् & इहो:नि  & इह्‍योक्‌त्यनु &ह्‍योक्‍नुये  \\ 
अम्अ्य & होक्‌त्यु  & होक्‌त्य  \\ 
अम्सुअ्य & होक्सु & होक्‌त्यसु  \\ 
अम्ह्‍याम्अ्य & होक्‍नु  & होक्‌त्यनु \\ 
\midrule
इन्य, अम्अ्य उ:ङ्‌ &इहो:ङ &इह्‍योकत &ह्‍योकये \\ 
एऽचिअ्य/अम्सुअ उ:ङ्‌ &इहो:ङसु &इह्‍योकतसु &ह्‍योकसुये \\ 
ए:न्अ्य/अम्ह्‍याम्अ्य उ:ङ्‌ &इहो:ङनु &इह्‍योकतनु &ह्‍योकनुये \\ 
इन्य/अम्अ्य इऽचि &इह्‍योकि &इह्‍योकिति    \\ 
इन्य/अम्अ्य ओऽचु &इह्‍योकु &इह्‍योकुतु  &ह्‍योकुये  \\ 
इन्य/अम्अ्य इक् &इहोक्‌कि &इहोक्‌तिकि   \\ 
इन्य/अम्अ्य ओक् &इहोक्‌क &इहोक्‌तक  &होक्‌कये  \\ 
अम्अ्य इन् & इहो: & इह्‍योक्‌त्य   \\ 
अम्अ्य एऽचि & इह्‍योकि & इह्‍योकिति    \\ 
अम्अ्य ए:न् & इहो:नि  & इह्‍योक्‌त्यनु  \\ 
\midrule
उङ इन् & हो:न्य  & हो:न्त्यनि  \\ 
उङ एऽचि & हो:न्सु  & हो:न्त्यान्सु   \\ 
उङ ए:न्& हो:न्‍नु  & हो:न्त्यान्‍नु   \\ 
\bottomrule
\end{tabular}
\end{table}


\begin{table}[H]
\label{omt.vt} \centering
\caption{अकर्मक क्रिया  अ्वाम्‍न्य  "पाक्नु"  }
\begin{tabular}{l|l|l|l|l|l|l|l|l|l|l|l|l}  \toprule
&अभूत & भूत   \\ 
म्य & अ्वाम्द्‌यु  & अ्वा:म्त्य  \\ 
\bottomrule
\end{tabular}
\end{table}


\begin{table}[H]
\label{ont.vt} \centering
\caption{सकर्मक क्रिया  ब्वाइन्य  "छुनु"  }
\begin{tabular}{l|l|l|l|l|l|l|l|l|l|l|l|l}  \toprule
&अभूत & भूत & आज्ञार्थक \\ 
उङ &ब्वान्दु &ब्वा:न्त \\ 
उङ अम्सु&ब्वान्दुसु &ब्वा:न्तसु \\ 
उङ अम्ह्‍याम्&ब्वान्दुनु &ब्वा:न्तनु \\ 
इऽचिअ्य &ब्योऽचि &ब्योऽस्ति   \\ 
ओऽचुअ        &ब्योऽचु &ब्योऽस्तु   \\ 
इक्अ्य&ब्वाइकि &ब्वाइतिकि   \\ 
ओक्अ &ब्वाइक &ब्वाइतक   \\ 
इन्य & इब्वान्द्‌यु  & इब्वा:न्त्य &ब्वान्दे  \\ 
इन्य अम्सु& इब्वा:न्सु  & इब्वा:न्त्यसु   \\ 
इन्य अम्ह्‍याम्& इब्वा:न्‍नु  & इब्वा:न्त्यनु   \\ 
एऽचि & इब्योऽचि & इब्योऽस्ति &ब्योऽचिये    \\ 
ए:न् & इब्वाइनि  & इब्योऽस्त्यनु &ब्योऽस्‍नुये  \\ 
अम्अ्य & ब्वान्द्‌यु  & ब्वा:न्त्य  \\ 
अम्सुअ्य & ब्वा:न्सु & ब्वा:न्त्यसु  \\ 
अम्ह्‍याम्अ्य & ब्वा:न्‍नु  & ब्वा:न्त्यनु \\ 
\midrule
इन्य, अम्अ्य उ:ङ्‌ &इब्वाइङ &इब्योऽस्त &ब्योऽचये \\ 
एऽचिअ्य/अम्सुअ उ:ङ्‌ &इब्वाइङसु &इब्योऽस्तसु &ब्योऽचसुये \\ 
ए:न्अ्य/अम्ह्‍याम्अ्य उ:ङ्‌ &इब्वाइङनु &इब्योऽस्तनु &ब्योऽचनुये \\ 
इन्य/अम्अ्य इऽचि &इब्योऽचि &इब्योऽस्ति    \\ 
इन्य/अम्अ्य ओऽचु &इब्योऽचु &इब्योऽस्तु  &ब्योऽचुये  \\ 
इन्य/अम्अ्य इक् &इब्वाइकि &इब्वाइतिकि   \\ 
इन्य/अम्अ्य ओक् &इब्वाइक &इब्वाइतक  &ब्वाइकये  \\ 
अम्अ्य इन् & इब्वाइ & इब्योऽस्त्य   \\ 
अम्अ्य एऽचि & इब्योऽचि & इब्योऽस्ति    \\ 
अम्अ्य ए:न् & इब्वाइनि  & इब्योऽस्त्यनु  \\ 
\midrule
उङ इन् & ब्वाइन्य  & ब्वाइत्यनि  \\ 
उङ एऽचि & ब्वा:इसु  & ब्वाइत्यान्सु   \\ 
उङ ए:न्& ब्वा:इनु  & ब्वाइत्यान्‍नु   \\ 
\bottomrule
\end{tabular}
\end{table}


\begin{table}[H]
\label{oŋt.vt} \centering
\caption{सकर्मक क्रिया  नोऽन्य  "दोष लगाउनु"  }
\begin{tabular}{l|l|l|l|l|l|l|l|l|l|l|l|l}  \toprule
&अभूत & भूत & आज्ञार्थक \\ 
उङ &नोन्दु &नो:न्त \\ 
उङ अम्सु&नोन्दुसु &नो:न्तसु \\ 
उङ अम्ह्‍याम्&नोन्दुनु &नो:न्तनु \\ 
इऽचिअ्य &न्योङि &न्योङिति   \\ 
ओऽचुअ        &न्योङु &न्योङुतु   \\ 
इक्अ्य&नोङ्‌कि &नोङ्‌तिकि   \\ 
ओक्अ &नोङ्‌क &नोङ्‌तक   \\ 
इन्य & इनोन्द्‌यु  & इनो:न्त्य &नोन्दे  \\ 
इन्य अम्सु& इनो:न्सु  & इनो:न्त्यसु   \\ 
इन्य अम्ह्‍याम्& इनो:न्‍नु  & इनो:न्त्यनु   \\ 
एऽचि & इन्योङि & इन्योङिति &न्योङिये    \\ 
ए:न् & इनोऽनि  & इन्यो:ङ्‌त्यनु &न्यो:ङ्‌नुये  \\ 
अम्अ्य & नोन्द्‌यु  & नो:न्त्य  \\ 
अम्सुअ्य & नो:न्सु & नो:न्त्यसु  \\ 
अम्ह्‍याम्अ्य & नो:न्‍नु  & नो:न्त्यनु \\ 
\midrule
इन्य, अम्अ्य उ:ङ्‌ &इनोऽङ &इन्योङत &न्योङये \\ 
एऽचिअ्य/अम्सुअ उ:ङ्‌ &इनोऽङसु &इन्योङतसु &न्योङसुये \\ 
ए:न्अ्य/अम्ह्‍याम्अ्य उ:ङ्‌ &इनोऽङनु &इन्योङतनु &न्योङनुये \\ 
इन्य/अम्अ्य इऽचि &इन्योङि &इन्योङिति    \\ 
इन्य/अम्अ्य ओऽचु &इन्योङु &इन्योङुतु  &न्योङुये  \\ 
इन्य/अम्अ्य इक् &इनोङ्‌कि &इनोङ्‌तिकि   \\ 
इन्य/अम्अ्य ओक् &इनोङ्‌क &इनोङ्‌तक  &नोङ्‌कये  \\ 
अम्अ्य इन् & इनोङ्‌ & इन्यो:ङ्‌त्य   \\ 
अम्अ्य एऽचि & इन्योङि & इन्योङिति    \\ 
अम्अ्य ए:न् & इनोऽनि  & इन्यो:ङ्‌त्यनु  \\ 
\midrule
उङ इन् & नोऽन्य  & नोन्त्यनि  \\ 
उङ एऽचि & नो:न्सु  & नोन्त्यान्सु   \\ 
उङ ए:न्& नो:न्‍नु  & नोन्त्यान्‍नु   \\ 
\bottomrule
\end{tabular}
\end{table}


\begin{table}[H]
\label{ort.vt} \centering
\caption{सकर्मक क्रिया  ध्वार्न्य  "लाभ् हुनु, फाइदा हुनु"  }
\begin{tabular}{l|l|l|l|l|l|l|l|l|l|l|l|l}  \toprule
&अभूत & भूत & आज्ञार्थक \\ 
उङ &ध्वार्दु &ध्वा:र्त \\ 
उङ अम्सु&ध्वार्दुसु &ध्वा:र्तसु \\ 
उङ अम्ह्‍याम्&ध्वार्दुनु &ध्वा:र्तनु \\ 
इऽचिअ्य &ध्योरि &ध्योरिति   \\ 
ओऽचुअ        &ध्योरु &ध्योरुतु   \\ 
इक्अ्य&ध्वार्कि &ध्वार्तिकि   \\ 
ओक्अ &ध्वार्क &ध्वार्तक   \\ 
इन्य & इध्वार्द्‌यु  & इध्वा:र्त्य &ध्वार्दे  \\ 
इन्य अम्सु& इध्वा:र्सु  & इध्वा:र्त्यसु   \\ 
इन्य अम्ह्‍याम्& इध्वा:र्नु  & इध्वा:र्त्यनु   \\ 
एऽचि & इध्योरि & इध्योरिति &ध्योरिये    \\ 
ए:न् & इध्वार्नि  & इध्यो:र्त्यनु &ध्यो:र्नुये  \\ 
अम्अ्य & ध्वार्द्‌यु  & ध्वा:र्त्य  \\ 
अम्सुअ्य & ध्वा:र्सु & ध्वा:र्त्यसु  \\ 
अम्ह्‍याम्अ्य & ध्वा:र्नु  & ध्वा:र्त्यनु \\ 
\midrule
इन्य, अम्अ्य उ:ङ्‌ &इध्वार्ङ &इध्योरत &ध्योरये \\ 
एऽचिअ्य/अम्सुअ उ:ङ्‌ &इध्वार्ङसु &इध्योरतसु &ध्योरसुये \\ 
ए:न्अ्य/अम्ह्‍याम्अ्य उ:ङ्‌ &इध्वार्ङनु &इध्योरतनु &ध्योरनुये \\ 
इन्य/अम्अ्य इऽचि &इध्योरि &इध्योरिति    \\ 
इन्य/अम्अ्य ओऽचु &इध्योरु &इध्योरुतु  &ध्योरुये  \\ 
इन्य/अम्अ्य इक् &इध्वार्कि &इध्वार्तिकि   \\ 
इन्य/अम्अ्य ओक् &इध्वार्क &इध्वार्तक  &ध्वार्कये  \\ 
अम्अ्य इन् & इध्वार् & इध्यो:र्त्य   \\ 
अम्अ्य एऽचि & इध्योरि & इध्योरिति    \\ 
अम्अ्य ए:न् & इध्वार्नि  & इध्यो:र्त्यनु  \\ 
\midrule
उङ इन् & ध्वार्न्य  & ध्वार्त्यनि  \\ 
उङ एऽचि & ध्वा:र्सु  & ध्वार्त्यान्सु   \\ 
उङ ए:न्& ध्वा:र्नु  & ध्वार्त्यान्‍नु   \\ 
\bottomrule
\end{tabular}
\end{table}


\begin{table}[H]
\label{olt.vt} \centering
\caption{सकर्मक क्रिया  क्‍वाल्न्य  "खेद्‍नु"  }
\begin{tabular}{l|l|l|l|l|l|l|l|l|l|l|l|l}  \toprule
&अभूत & भूत & आज्ञार्थक \\ 
उङ &क्‍वाल्दु &क्‍वा:ल्त \\ 
उङ अम्सु&क्‍वाल्दुसु &क्‍वा:ल्तसु \\ 
उङ अम्ह्‍याम्&क्‍वाल्दुनु &क्‍वा:ल्तनु \\ 
इऽचिअ्य &क्योलि &क्योलिति   \\ 
ओऽचुअ        &क्योलु &क्योलुतु   \\ 
इक्अ्य&क्‍वाल्कि &क्‍वाल्तिकि   \\ 
ओक्अ &क्‍वाल्क &क्‍वाल्तक   \\ 
इन्य & इक्‍वाल्द्‌यु  & इक्‍वा:ल्त्य &क्‍वाल्दे  \\ 
इन्य अम्सु& इक्‍वा:ल्सु  & इक्‍वा:ल्त्यसु   \\ 
इन्य अम्ह्‍याम्& इक्‍वा:ल्नु  & इक्‍वा:ल्त्यनु   \\ 
एऽचि & इक्योलि & इक्योलिति &क्योलिये    \\ 
ए:न् & इक्‍वाल्नि  & इक्यो:ल्त्यनु &क्यो:ल्नुये  \\ 
अम्अ्य & क्‍वाल्द्‌यु  & क्‍वा:ल्त्य  \\ 
अम्सुअ्य & क्‍वा:ल्सु & क्‍वा:ल्त्यसु  \\ 
अम्ह्‍याम्अ्य & क्‍वा:ल्नु  & क्‍वा:ल्त्यनु \\ 
\midrule
इन्य, अम्अ्य उ:ङ्‌ &इक्‍वाल्ङ &इक्योलत &क्योलये \\ 
एऽचिअ्य/अम्सुअ उ:ङ्‌ &इक्‍वाल्ङसु &इक्योलतसु &क्योलसुये \\ 
ए:न्अ्य/अम्ह्‍याम्अ्य उ:ङ्‌ &इक्‍वाल्ङनु &इक्योलतनु &क्योलनुये \\ 
इन्य/अम्अ्य इऽचि &इक्योलि &इक्योलिति    \\ 
इन्य/अम्अ्य ओऽचु &इक्योलु &इक्योलुतु  &क्योलुये  \\ 
इन्य/अम्अ्य इक् &इक्‍वाल्कि &इक्‍वाल्तिकि   \\ 
इन्य/अम्अ्य ओक् &इक्‍वाल्क &इक्‍वाल्तक  &क्‍वाल्कये  \\ 
अम्अ्य इन् & इक्‍वाल् & इक्यो:ल्त्य   \\ 
अम्अ्य एऽचि & इक्योलि & इक्योलिति    \\ 
अम्अ्य ए:न् & इक्‍वाल्नि  & इक्यो:ल्त्यनु  \\ 
\midrule
उङ इन् & क्‍वाल्न्य  & क्‍वाल्त्यनि  \\ 
उङ एऽचि & क्‍वा:ल्सु  & क्‍वाल्त्यान्सु   \\ 
उङ ए:न्& क्‍वा:ल्नु  & क्‍वाल्त्यान्‍नु   \\ 
\bottomrule
\end{tabular}
\end{table}


\begin{table}[H]
\label{u.vi} \centering
\caption{अकर्मक क्रिया  स्युन्य  "चिलाउनु"  }
\begin{tabular}{l|l|l|l|l|l|l|l|l|l|l|l|l}  \toprule
&अभूत & भूत & आज्ञार्थक \\ 
उ:ङ्‌ &स्युङ &स्युङत \\ 
इऽचि &स्युयि &स्यु:इति   \\ 
ओऽचु &स्युयु &स्यु:इतु   \\ 
इक् &स्युकि &स्युक्‌तिकि   \\ 
ओक् &स्युक &स्युक्‌तक   \\ 
इन् & इस्यु & इसुऽत्य &सुऽये  \\ 
एऽचि & इस्युयि & इस्यु:इति &स्यु:इये    \\ 
ए:न् & इस्युनि  & इसुत्‍नु &सु:न्ये  \\ 
अम् & स्यु & सुऽत्य   \\ 
अम्सु & स्युयि & स्यु:इति     \\ 
अम्ह्‍याम् & स्युनु  & सुत्‍नु \\ 
\bottomrule
\end{tabular}
\end{table}


\begin{table}[H]
\label{ut.vi} \centering
\caption{अकर्मक क्रिया  ह:न्‍न्य  "डढ्नु"  }
\begin{tabular}{l|l|l|l|l|l|l|l|l|l|l|l|l}  \toprule
&अभूत & भूत   \\ 
म्य & है: & ह्‍युस्त्य   \\ 
\bottomrule
\end{tabular}
\end{table}


\begin{table}[H]
\label{uk.vi} \centering
\caption{अकर्मक क्रिया  झु:न्य  "भाग्नु"  }
\begin{tabular}{l|l|l|l|l|l|l|l|l|l|l|l|l}  \toprule
&अभूत & भूत & आज्ञार्थक \\ 
उ:ङ्‌ &झु:ङ &झ्युकत \\ 
इऽचि &झ्युकि &झ्युकिति   \\ 
ओऽचु &झ्युकु &झ्युकुतु   \\ 
इक् &झुक्‌कि &झुक्‌तिकि   \\ 
ओक् &झुक्‌क &झुक्‌तक   \\ 
इन् & इझु: & इझ्युक्‌त्य &झ्युक्ये  \\ 
एऽचि & इझ्युकि & इझ्युकिति &झ्युकिये    \\ 
ए:न् & इझु:नि  & इझ्युक्‌त्यनु &झ्युक्‍नुये  \\ 
अम् & झु: & झ्युक्‌त्य   \\ 
अम्सु & झ्युकि & झ्युकिति   \\ 
अम्ह्‍याम् & झु:नु  & झ्युक्‌त्यनु \\ 
\bottomrule
\end{tabular}
\end{table}


\begin{table}[H]
\label{um.vi} \centering
\caption{अकर्मक क्रिया  घ्रम्‍न्य  "थाक्नु"  }
\begin{tabular}{l|l|l|l|l|l|l|l|l|l|l|l|l}  \toprule
&अभूत & भूत & आज्ञार्थक \\ 
उ:ङ्‌ &घ्रम्ङ &घ्र्युमत \\ 
इऽचि &घ्र्युमि &घ्र्युमिति   \\ 
ओऽचु &घ्र्युमु &घ्र्युमुतु   \\ 
इक् &घ्रम्कि &घ्रम्तिकि   \\ 
ओक् &घ्रम्क &घ्रम्तक   \\ 
इन् & इघ्रम् & इघ्र्यु:म्त्य &घ्र्यु:म्ये  \\ 
एऽचि & इघ्र्युमि & इघ्र्युमिति &घ्र्युमिये    \\ 
ए:न् & इघ्रम्‍नि  & इघ्र्यु:म्त्यनु &घ्र्यु:म्‍नुये  \\ 
अम् & घ्रम् & घ्र्यु:म्त्य   \\ 
अम्सु & घ्र्युमि & घ्र्युमिति   \\ 
अम्ह्‍याम् & घ्रम्‍नु  & घ्र्यु:म्त्यनु \\ 
\bottomrule
\end{tabular}
\end{table}


\begin{table}[H]
\label{un.vi} \centering
\caption{अकर्मक क्रिया  खैन्य  "निद्रले झुल्नु"  }
\begin{tabular}{l|l|l|l|l|l|l|l|l|l|l|l|l}  \toprule
&अभूत & भूत & आज्ञार्थक \\ 
उ:ङ्‌ &खैङ &ख्युऽस्त \\ 
इऽचि &ख्युऽचि &ख्युऽस्ति   \\ 
ओऽचु &ख्युऽचु &ख्युऽस्तु   \\ 
इक् &खैकि &खैतिकि   \\ 
ओक् &खैक &खैतक   \\ 
इन् & इखै & इख्युऽस्त्य &ख्युऽचे  \\ 
एऽचि & इख्युऽचि & इख्युऽस्ति &ख्युऽचिये    \\ 
ए:न् & इखैनि  & इख्युऽस्त्यनु &ख्युऽस्‍नुये  \\ 
अम् & खै & ख्युऽस्त्य   \\ 
अम्सु & ख्युऽचि & ख्युऽस्ति   \\ 
अम्ह्‍याम् & खैनु  & ख्युऽस्त्यनु \\ 
\bottomrule
\end{tabular}
\end{table}


\begin{table}[H]
\label{ur.vi} \centering
\caption{अकर्मक क्रिया  घर्न्य  "दौडनु"  }
\begin{tabular}{l|l|l|l|l|l|l|l|l|l|l|l|l}  \toprule
&अभूत & भूत & आज्ञार्थक \\ 
उ:ङ्‌ &घर्ङ &घ्युरत \\ 
इऽचि &घ्युरि &घ्युरिति   \\ 
ओऽचु &घ्युरु &घ्युरुतु   \\ 
इक् &घर्कि &घर्तिकि   \\ 
ओक् &घर्क &घर्तक   \\ 
इन् & इघर् & इघ्यु:र्त्य &घ्यु:र्‍ये  \\ 
एऽचि & इघ्युरि & इघ्युरिति &घ्युरिये    \\ 
ए:न् & इघर्नि  & इघ्यु:र्त्यनु &घ्यु:र्नुये  \\ 
अम् & घर् & घ्यु:र्त्य   \\ 
अम्सु & घ्युरि & घ्युरिति   \\ 
अम्ह्‍याम् & घर्नु  & घ्यु:र्त्यनु \\ 
\bottomrule
\end{tabular}
\end{table}


\begin{table}[H]
\label{u.vt} \centering
\caption{सकर्मक क्रिया  अ्युन्य  "बौराउनु"  }
\begin{tabular}{l|l|l|l|l|l|l|l|l|l|l|l|l}  \toprule
&अभूत & भूत & आज्ञार्थक \\ 
उङ &अ्युङ &अ्यु:ङ्‌त \\ 
इऽचिअ्य &अ्युयि &अ्यु:इति   \\ 
ओऽचुअ &अ्युयु &अ्यु:इतु   \\ 
इक्अ्य &अ्युकि &अ्युक्‌तिकि   \\ 
ओक्अ &अ्युक &अ्युक्‌तक   \\ 
इन्य & इअ्यु & इअ्युत्य &अ्युये  \\ 
एऽचिअ्य & इअ्युयि & इअ्यु:इति &अ्यु:इये    \\ 
ए:न्अ्य & इअ्युनि  & इउत्‍नु &उ:न्ये  \\ 
अम्अ्य & अ्यु & अ्युत्य   \\ 
अम्सुअ & अ्युसु & अ्युत्सु     \\ 
अम्ह्‍याम्अ्य & अ्युनु  & अ्युत्‍नु \\ 
\midrule
इन्य/अम्अ्य उ:ङ्‌&इअ्युङ &इअ्युङत &अ्युङये \\ 
एऽचिअ्य/अम्सुअ उ:ङ्‌ &इअ्युङसु &इअ्युङतसु &अ्युङसुये \\ 
ए:न्अ्य/अम्ह्‍याम्अ्य उ:ङ्‌ &इअ्युङनु &इअ्युङतनु &अ्युङनुये \\ 
इन्य/अम्अ्य इऽचि &इअ्युयि &इअ्यु:इति    \\ 
इन्य/अम्अ्य ओऽचु &इअ्युयु &इअ्यु:इतु  &अ्युइये  \\ 
इन्य/अम्अ्य इक् &इअ्युकि &इअ्युक्‌तिकि   \\ 
इन्य/अम्अ्य ओक् &इअ्युक &इअ्युक्‌तक  &अ्युकये  \\ 
अम्अ्य इन् & इअ्यु & इऊत्य   \\ 
अम्अ्य एऽचि & इअ्युयि & इअ्यु:इति     \\ 
अम्अ्य ए:न् & इअ्युनि  & इउत्‍नु  \\ 
\midrule
उङ इन् & अ्युन्य  & अ्यु:न्त्यनि  \\ 
उङ एऽचि & अ्यु:न्सु  & अ्यु:न्त्यान्सु   \\ 
उङ ए:न्& अ्यु:न्‍नु  & अ्यु:न्त्यान्‍नु   \\ 
\bottomrule
\end{tabular}
\end{table}


\begin{table}[H]
\label{up.vt} \centering
\caption{सकर्मक क्रिया  ग:म्‍न्य  "लुकाउनु"  }
\begin{tabular}{l|l|l|l|l|l|l|l|l|l|l|l|l}  \toprule
&अभूत & भूत & आज्ञार्थक \\ 
उङ &गुबु &गुबुत \\ 
उङ अम्सु &गुबुसु &गुबुतसु \\ 
उङ अम्ह्‍याम् &गुबुनु &गुबुतनु \\ 
इऽचिअ्य &ग्युपि &ग्युपिति   \\ 
ओऽचुअ &ग्युपु &ग्युपुतु   \\ 
इक्अ्य &गप्कि &गप्‍तिकि   \\ 
ओक्अ &गप्क &गप्‍तक   \\ 
इन्य अम् & इग्युऽब्यु  & इग्युऽप्‍त्य &ग्युऽबे  \\ 
इन्य अम्सु & इग्युऽप्सु  & इग्युऽप्‍त्यसु   \\ 
इन्य अम्ह्‍याम् & इग्युऽप्‍नु  & इग्युऽप्‍त्यनु   \\ 
एऽचिअ्य & इग्युपि & इग्युपिति &ग्युपिये    \\ 
ए:न्अ्य & इग:म्‍नि  & इग्युप्‍त्यनु &ग्युप्‍नुये  \\ 
अम्अ्य & ग्युऽब्यु  & ग्युऽप्‍त्य  \\ 
अम्सुअ & ग्युऽप्सु & ग्युऽप्‍त्यसु  \\ 
अम्ह्‍याम्अ्य & ग्युऽप्‍नु  & ग्युऽप्‍त्यनु \\ 
\midrule
इन्य/अम्अ्य उ:ङ्‌&इग:म्ङ & इग्युपत &ग्युपये \\ 
एऽचिअ्य/अम्सुअ उ:ङ्‌ &इग:म्ङसु & इग्युपतसु &ग्युपसुये \\ 
ए:न्अ्य/अम्ह्‍याम्अ्य उ:ङ्‌ &इग:म्ङनु & इग्युपतनु &ग्युपनुये \\ 
इन्य/अम्अ्य इऽचि & इग्युपि & इग्युपिति    \\ 
इन्य/अम्अ्य ओऽचु & इग्युपु & इग्युपुतु  &ग्युपुये  \\ 
इन्य/अम्अ्य इक् & इगप्कि & इगप्‍तिकि   \\ 
इन्य/अम्अ्य ओक् & इगप्क & इगप्‍तक  &गप्कये  \\ 
अम्अ्य इन् & इगप् & इग्युप्‍त्य   \\ 
अम्अ्य एऽचि & इग्युपि & इग्युपिति    \\ 
अम्अ्य ए:न् & इग:म्‍नि  & इग्युप्‍त्यनु  \\ 
\midrule
उङ इन् & ग:म्‍न्य  & ग:म्त्यनि  \\ 
उङ एऽचि & ग:म्सु  & ग:म्त्यान्सु   \\ 
उङ ए:न्& ग:म्‍नु  & ग:म्त्यान्‍नु   \\ 
\bottomrule
\end{tabular}
\end{table}


\begin{table}[H]
\label{ut.vt} \centering
\caption{सकर्मक क्रिया  ल:न्‍न्य  "भन्नु"  }
\begin{tabular}{l|l|l|l|l|l|l|l|l|l|l|l|l}  \toprule
&अभूत & भूत & आज्ञार्थक \\ 
उङ &लुदु &लु:त \\ 
उङ अम्सु &लुदुसु &लु:तसु \\ 
उङ अम्ह्‍याम् &लुदुनु &लु:तनु \\ 
इऽचिअ्य &ल्युचि &ल्युस्ति   \\ 
ओऽचुअ &ल्युचु &ल्युस्तु   \\ 
इक्अ्य &लह्इकि &लह्इतिकि   \\ 
ओक्अ &लह्इक &लह्इतक   \\ 
इन्य अम् & इल्युऽद्‌यु  & इल्यु:त्य &ल्युऽदे  \\ 
इन्य अम्सु & इल्युऽत्सु  & इल्यु:त्यसु   \\ 
इन्य अम्ह्‍याम् & इल्युऽत्‍नु  & इल्यु:त्यनु   \\ 
एऽचिअ्य & इल्युचि & इल्युस्ति &ल्युचिये    \\ 
ए:न्अ्य & इल:न्‍नि  & इल्युस्त्यनु &ल्युस्‍नुये  \\ 
अम्अ्य & ल्युऽद्‌यु  & ल्यु:त्य  \\ 
अम्सुअ & ल्युऽत्सु & ल्यु:त्यसु  \\ 
अम्ह्‍याम्अ्य & ल्युऽत्‍नु  & ल्यु:त्यनु \\ 
\midrule
इन्य/अम्अ्य उ:ङ्‌&इलै:ङ & इल्युस्त &ल्युचये \\ 
एऽचिअ्य/अम्सुअ उ:ङ्‌ &इलै:ङसु & इल्युस्तसु &ल्युचसुये \\ 
ए:न्अ्य/अम्ह्‍याम्अ्य उ:ङ्‌ &इलै:ङनु & इल्युस्तनु &ल्युचनुये \\ 
इन्य/अम्अ्य इऽचि & इल्युचि & इल्युस्ति    \\ 
इन्य/अम्अ्य ओऽचु & इल्युचु & इल्युस्तु  &ल्युचुये  \\ 
इन्य/अम्अ्य इक् & इलह्इकि & इलह्इतिकि   \\ 
इन्य/अम्अ्य ओक् & इलह्इक & इलह्इतक  &लह्इकये  \\ 
अम्अ्य इन् & इलै: & इल्युस्त्य   \\ 
अम्अ्य एऽचि & इल्युचि & इल्युस्ति    \\ 
अम्अ्य ए:न् & इल:न्‍नि  & इल्युस्त्यनु  \\ 
\midrule
उङ इन् & ल:न्‍न्य  & ल:न्त्यनि  \\ 
उङ एऽचि & ल:न्सु  & ल:न्त्यान्सु   \\ 
उङ ए:न्& ल:न्‍नु  & ल:न्त्यान्‍नु   \\ 
\bottomrule
\end{tabular}
\end{table}


\begin{table}[H]
\label{uk.vt} \centering
\caption{सकर्मक क्रिया  फु:न्य  "उठाउनु"  }
\begin{tabular}{l|l|l|l|l|l|l|l|l|l|l|l|l}  \toprule
&अभूत & भूत & आज्ञार्थक \\ 
उङ &फुगु &फुगुत \\ 
उङ अम्सु &फुगुसु &फुगुतसु \\ 
उङ अम्ह्‍याम् &फुगुनु &फुगुतनु \\ 
इऽचिअ्य &फ्युकि &फ्युकिति   \\ 
ओऽचुअ &फ्युकु &फ्युकुतु   \\ 
इक्अ्य &फुक्‌कि &फुक्‌तिकि   \\ 
ओक्अ &फुक्‌क &फुक्‌तक   \\ 
इन्य अम् & इफ्युऽग्यु  & इफ्युऽक्‌त्य &फ्युऽगे  \\ 
इन्य अम्सु & इफ्युऽक्सु  & इफ्युऽक्‌त्यसु   \\ 
इन्य अम्ह्‍याम् & इफ्युऽक्‍नु  & इफ्युऽक्‌त्यनु   \\ 
एऽचिअ्य & इफ्युकि & इफ्युकिति &फ्युकिये    \\ 
ए:न्अ्य & इफु:नि  & इफ्युक्‌त्यनु &फ्युक्‍नुये  \\ 
अम्अ्य & फ्युऽग्यु  & फ्युऽक्‌त्य  \\ 
अम्सुअ & फ्युऽक्सु & फ्युऽक्‌त्यसु  \\ 
अम्ह्‍याम्अ्य & फ्युऽक्‍नु  & फ्युऽक्‌त्यनु \\ 
\midrule
इन्य/अम्अ्य उ:ङ्‌&इफु:ङ & इफ्युकत &फ्युकये \\ 
एऽचिअ्य/अम्सुअ उ:ङ्‌ &इफु:ङसु & इफ्युकतसु &फ्युकसुये \\ 
ए:न्अ्य/अम्ह्‍याम्अ्य उ:ङ्‌ &इफु:ङनु & इफ्युकतनु &फ्युकनुये \\ 
इन्य/अम्अ्य इऽचि & इफ्युकि & इफ्युकिति    \\ 
इन्य/अम्अ्य ओऽचु & इफ्युकु & इफ्युकुतु  &फ्युकुये  \\ 
इन्य/अम्अ्य इक् & इफुक्‌कि & इफुक्‌तिकि   \\ 
इन्य/अम्अ्य ओक् & इफुक्‌क & इफुक्‌तक  &फुक्‌कये  \\ 
अम्अ्य इन् & इफु: & इफ्युक्‌त्य   \\ 
अम्अ्य एऽचि & इफ्युकि & इफ्युकिति    \\ 
अम्अ्य ए:न् & इफु:नि  & इफ्युक्‌त्यनु  \\ 
\midrule
उङ इन् & फु:न्य  & फु:न्त्यनि  \\ 
उङ एऽचि & फु:न्सु  & फु:न्त्यान्सु   \\ 
उङ ए:न्& फु:न्‍नु  & फु:न्त्यान्‍नु   \\ 
\bottomrule
\end{tabular}
\end{table}


\begin{table}[H]
\label{um.vt} \centering
\caption{सकर्मक क्रिया  दम्‍न्य  "भेट्नु"  }
\begin{tabular}{l|l|l|l|l|l|l|l|l|l|l|l|l}  \toprule
&अभूत & भूत & आज्ञार्थक \\ 
उङ &दुमु &दुमुत \\ 
उङ अम्सु &दुमुसु &दुमुतसु \\ 
उङ अम्ह्‍याम् &दुमुनु &दुमुतनु \\ 
इऽचिअ्य &द्‌युमि &द्‌युमिति   \\ 
ओऽचुअ &द्‌युमु &द्‌युमुतु   \\ 
इक्अ्य &दम्कि &दम्तिकि   \\ 
ओक्अ &दम्क &दम्तक   \\ 
इन्य अम् & इद्‌युऽम्यु  & इद्‌यु:म्त्य &द्‌युऽमे  \\ 
इन्य अम्सु & इद्‌यु:म्सु  & इद्‌यु:म्त्यसु   \\ 
इन्य अम्ह्‍याम् & इद्‌यु:म्‍नु  & इद्‌यु:म्त्यनु   \\ 
एऽचिअ्य & इद्‌युमि & इद्‌युमिति &द्‌युमिये    \\ 
ए:न्अ्य & इदम्‍नि  & इद्‌यु:म्त्यनु &द्‌यु:म्‍नुये  \\ 
अम्अ्य & द्‌युऽम्यु  & द्‌यु:म्त्य  \\ 
अम्सुअ & द्‌यु:म्सु & द्‌यु:म्त्यसु  \\ 
अम्ह्‍याम्अ्य & द्‌यु:म्‍नु  & द्‌यु:म्त्यनु \\ 
\midrule
इन्य/अम्अ्य उ:ङ्‌&इदम्ङ & इद्‌युमत &द्‌युमये \\ 
एऽचिअ्य/अम्सुअ उ:ङ्‌ &इदम्ङसु & इद्‌युमतसु &द्‌युमसुये \\ 
ए:न्अ्य/अम्ह्‍याम्अ्य उ:ङ्‌ &इदम्ङनु & इद्‌युमतनु &द्‌युमनुये \\ 
इन्य/अम्अ्य इऽचि & इद्‌युमि & इद्‌युमिति    \\ 
इन्य/अम्अ्य ओऽचु & इद्‌युमु & इद्‌युमुतु  &द्‌युमुये  \\ 
इन्य/अम्अ्य इक् & इदम्कि & इदम्तिकि   \\ 
इन्य/अम्अ्य ओक् & इदम्क & इदम्तक  &दम्कये  \\ 
अम्अ्य इन् & इदम् & इद्‌यु:म्त्य   \\ 
अम्अ्य एऽचि & इद्‌युमि & इद्‌युमिति    \\ 
अम्अ्य ए:न् & इदम्‍नि  & इद्‌यु:म्त्यनु  \\ 
\midrule
उङ इन् & दम्‍न्य  & दम्त्यनि  \\ 
उङ एऽचि & द:म्सु  & दम्त्यान्सु   \\ 
उङ ए:न्& द:म्‍नु  & दम्त्यान्‍नु   \\ 
\bottomrule
\end{tabular}
\end{table}


\begin{table}[H]
\label{uŋ.vt} \centering
\caption{सकर्मक क्रिया  हुऽन्य  "पर्खनु"  }
\begin{tabular}{l|l|l|l|l|l|l|l|l|l|l|l|l}  \toprule
&अभूत & भूत & आज्ञार्थक \\ 
उङ &हुङु &हुङुत \\ 
उङ अम्सु &हुङुसु &हुङुतसु \\ 
उङ अम्ह्‍याम् &हुङुनु &हुङुतनु \\ 
इऽचिअ्य &ह्‍युङि &ह्‍युङिति   \\ 
ओऽचुअ &ह्‍युङु &ह्‍युङुतु   \\ 
इक्अ्य &हुङ्‌कि &हुङ्‌तिकि   \\ 
ओक्अ &हुङ्‌क &हुङ्‌तक   \\ 
इन्य अम् & इह्‍युऽङ्‌यु  & इह्‍यु:ङ्‌त्य &ह्‍युऽङे  \\ 
इन्य अम्सु & इह्‍यु:ङ्‌सु  & इह्‍यु:ङ्‌त्यसु   \\ 
इन्य अम्ह्‍याम् & इह्‍यु:ङ्‌नु  & इह्‍यु:ङ्‌त्यनु   \\ 
एऽचिअ्य & इह्‍युङि & इह्‍युङिति &ह्‍युङिये    \\ 
ए:न्अ्य & इहुऽनि  & इह्‍यु:ङ्‌त्यनु &ह्‍यु:ङ्‌नुये  \\ 
अम्अ्य & ह्‍युऽङ्‌यु  & ह्‍यु:ङ्‌त्य  \\ 
अम्सुअ & ह्‍यु:ङ्‌सु & ह्‍यु:ङ्‌त्यसु  \\ 
अम्ह्‍याम्अ्य & ह्‍यु:ङ्‌नु  & ह्‍यु:ङ्‌त्यनु \\ 
\midrule
इन्य/अम्अ्य उ:ङ्‌&इहुऽङ & इह्‍युङत &ह्‍युङये \\ 
एऽचिअ्य/अम्सुअ उ:ङ्‌ &इहुऽङसु & इह्‍युङतसु &ह्‍युङसुये \\ 
ए:न्अ्य/अम्ह्‍याम्अ्य उ:ङ्‌ &इहुऽङनु & इह्‍युङतनु &ह्‍युङनुये \\ 
इन्य/अम्अ्य इऽचि & इह्‍युङि & इह्‍युङिति    \\ 
इन्य/अम्अ्य ओऽचु & इह्‍युङु & इह्‍युङुतु  &ह्‍युङुये  \\ 
इन्य/अम्अ्य इक् & इहुङ्‌कि & इहुङ्‌तिकि   \\ 
इन्य/अम्अ्य ओक् & इहुङ्‌क & इहुङ्‌तक  &हुङ्‌कये  \\ 
अम्अ्य इन् & इहुङ्‌ & इह्‍यु:ङ्‌त्य   \\ 
अम्अ्य एऽचि & इह्‍युङि & इह्‍युङिति    \\ 
अम्अ्य ए:न् & इहुऽनि  & इह्‍यु:ङ्‌त्यनु  \\ 
\midrule
उङ इन् & हुऽन्य  & हुन्त्यनि  \\ 
उङ एऽचि & हु:न्सु  & हुन्त्यान्सु   \\ 
उङ ए:न्& हु:न्‍नु  & हुन्त्यान्‍नु   \\ 
\bottomrule
\end{tabular}
\end{table}


\begin{table}[H]
\label{ur.vt} \centering
\caption{सकर्मक क्रिया  सर्न्य  "धुनु"  }
\begin{tabular}{l|l|l|l|l|l|l|l|l|l|l|l|l}  \toprule
&अभूत & भूत & आज्ञार्थक \\ 
उङ &सुरु &सुरुत \\ 
उङ अम्सु &सुरुसु &सुरुतसु \\ 
उङ अम्ह्‍याम् &सुरुनु &सुरुतनु \\ 
इऽचिअ्य &स्युरि &स्युरिति   \\ 
ओऽचुअ &स्युरु &स्युरुतु   \\ 
इक्अ्य &सर्कि &सर्तिकि   \\ 
ओक्अ &सर्क &सर्तक   \\ 
इन्य अम् & इस्युऽर्‍यु  & इस्यु:र्त्य &स्युऽरे  \\ 
इन्य अम्सु & इस्यु:र्सु  & इस्यु:र्त्यसु   \\ 
इन्य अम्ह्‍याम् & इस्यु:र्नु  & इस्यु:र्त्यनु   \\ 
एऽचिअ्य & इस्युरि & इस्युरिति &स्युरिये    \\ 
ए:न्अ्य & इसर्नि  & इस्यु:र्त्यनु &स्यु:र्नुये  \\ 
अम्अ्य & स्युऽर्‍यु  & स्यु:र्त्य  \\ 
अम्सुअ & स्यु:र्सु & स्यु:र्त्यसु  \\ 
अम्ह्‍याम्अ्य & स्यु:र्नु  & स्यु:र्त्यनु \\ 
\midrule
इन्य/अम्अ्य उ:ङ्‌&इसर्ङ & इस्युरत &स्युरये \\ 
एऽचिअ्य/अम्सुअ उ:ङ्‌ &इसर्ङसु & इस्युरतसु &स्युरसुये \\ 
ए:न्अ्य/अम्ह्‍याम्अ्य उ:ङ्‌ &इसर्ङनु & इस्युरतनु &स्युरनुये \\ 
इन्य/अम्अ्य इऽचि & इस्युरि & इस्युरिति    \\ 
इन्य/अम्अ्य ओऽचु & इस्युरु & इस्युरुतु  &स्युरुये  \\ 
इन्य/अम्अ्य इक् & इसर्कि & इसर्तिकि   \\ 
इन्य/अम्अ्य ओक् & इसर्क & इसर्तक  &सर्कये  \\ 
अम्अ्य इन् & इसर् & इस्यु:र्त्य   \\ 
अम्अ्य एऽचि & इस्युरि & इस्युरिति    \\ 
अम्अ्य ए:न् & इसर्नि  & इस्यु:र्त्यनु  \\ 
\midrule
उङ इन् & सर्न्य  & सर्त्यनि  \\ 
उङ एऽचि & स:र्सु  & सर्त्यान्सु   \\ 
उङ ए:न्& स:र्नु  & सर्त्यान्‍नु   \\ 
\bottomrule
\end{tabular}
\end{table}


\begin{table}[H]
\label{ul.vt} \centering
\caption{सकर्मक क्रिया  गल्न्य  "ढाक्नु"  }
\begin{tabular}{l|l|l|l|l|l|l|l|l|l|l|l|l}  \toprule
&अभूत & भूत & आज्ञार्थक \\ 
उङ &गुलु &गुलुत \\ 
उङ अम्सु &गुलुसु &गुलुतसु \\ 
उङ अम्ह्‍याम् &गुलुनु &गुलुतनु \\ 
इऽचिअ्य &ग्युलि &ग्युलिति   \\ 
ओऽचुअ &ग्युलु &ग्युलुतु   \\ 
इक्अ्य &गल्कि &गल्तिकि   \\ 
ओक्अ &गल्क &गल्तक   \\ 
इन्य अम् & इग्युऽल्यु  & इग्यु:ल्त्य &ग्युऽले  \\ 
इन्य अम्सु & इग्यु:ल्सु  & इग्यु:ल्त्यसु   \\ 
इन्य अम्ह्‍याम् & इग्यु:ल्नु  & इग्यु:ल्त्यनु   \\ 
एऽचिअ्य & इग्युलि & इग्युलिति &ग्युलिये    \\ 
ए:न्अ्य & इगल्नि  & इग्यु:ल्त्यनु &ग्यु:ल्नुये  \\ 
अम्अ्य & ग्युऽल्यु  & ग्यु:ल्त्य  \\ 
अम्सुअ & ग्यु:ल्सु & ग्यु:ल्त्यसु  \\ 
अम्ह्‍याम्अ्य & ग्यु:ल्नु  & ग्यु:ल्त्यनु \\ 
\midrule
इन्य/अम्अ्य उ:ङ्‌&इगल्ङ & इग्युलत &ग्युलये \\ 
एऽचिअ्य/अम्सुअ उ:ङ्‌ &इगल्ङसु & इग्युलतसु &ग्युलसुये \\ 
ए:न्अ्य/अम्ह्‍याम्अ्य उ:ङ्‌ &इगल्ङनु & इग्युलतनु &ग्युलनुये \\ 
इन्य/अम्अ्य इऽचि & इग्युलि & इग्युलिति    \\ 
इन्य/अम्अ्य ओऽचु & इग्युलु & इग्युलुतु  &ग्युलुये  \\ 
इन्य/अम्अ्य इक् & इगल्कि & इगल्तिकि   \\ 
इन्य/अम्अ्य ओक् & इगल्क & इगल्तक  &गल्कये  \\ 
अम्अ्य इन् & इगल् & इग्यु:ल्त्य   \\ 
अम्अ्य एऽचि & इग्युलि & इग्युलिति    \\ 
अम्अ्य ए:न् & इगल्नि  & इग्यु:ल्त्यनु  \\ 
\midrule
उङ इन् & गल्न्य  & गल्त्यनि  \\ 
उङ एऽचि & ग:ल्सु  & गल्त्यान्सु   \\ 
उङ ए:न्& ग:ल्नु  & गल्त्यान्‍नु   \\ 
\bottomrule
\end{tabular}
\end{table}


\begin{table}[H]
\label{utt.vt} \centering
\caption{सकर्मक क्रिया  ह:न्‍न्य  "पोल्नु"  }
\begin{tabular}{l|l|l|l|l|l|l|l|l|l|l|l|l}  \toprule
&अभूत & भूत & आज्ञार्थक \\ 
उङ &हत्तु &हत्त \\ 
उङ अम्सु&हत्तुसु &हत्तसु \\ 
उङ अम्ह्‍याम्&हत्तुनु &हत्तनु \\ 
इऽचिअ्य &ह्‍युचि &ह्‍युस्ति   \\ 
ओऽचुअ        &ह्‍युचु &ह्‍युस्तु   \\ 
इक्अ्य&हह्इकि &हह्इतिकि   \\ 
ओक्अ &हह्इक &हह्इतक   \\ 
इन्य & इहत्त्यु  & इहत्त्य &हत्ते  \\ 
इन्य अम्सु& इहत्सु  & इहत्त्यसु   \\ 
इन्य अम्ह्‍याम्& इहत्‍नु  & इहत्त्यनु   \\ 
एऽचि & इह्‍युचि & इह्‍युस्ति &ह्‍युचिये    \\ 
ए:न् & इह:न्‍नि  & इह्‍युस्त्यनु &ह्‍युस्‍नुये  \\ 
अम्अ्य & हत्त्यु  & हत्त्य  \\ 
अम्सुअ्य & हत्सु & हत्त्यसु  \\ 
अम्ह्‍याम्अ्य & हत्‍नु  & हत्त्यनु \\ 
\midrule
इन्य, अम्अ्य उ:ङ्‌ &इहै:ङ &इह्‍युस्त &ह्‍युचये \\ 
एऽचिअ्य/अम्सुअ उ:ङ्‌ &इहै:ङसु &इह्‍युस्तसु &ह्‍युचसुये \\ 
ए:न्अ्य/अम्ह्‍याम्अ्य उ:ङ्‌ &इहै:ङनु &इह्‍युस्तनु &ह्‍युचनुये \\ 
इन्य/अम्अ्य इऽचि &इह्‍युचि &इह्‍युस्ति    \\ 
इन्य/अम्अ्य ओऽचु &इह्‍युचु &इह्‍युस्तु  &ह्‍युचुये  \\ 
इन्य/अम्अ्य इक् &इहह्इकि &इहह्इतिकि   \\ 
इन्य/अम्अ्य ओक् &इहह्इक &इहह्इतक  &हह्इकये  \\ 
अम्अ्य इन् & इहै: & इह्‍युस्त्य   \\ 
अम्अ्य एऽचि & इह्‍युचि & इह्‍युस्ति    \\ 
अम्अ्य ए:न् & इह:न्‍नि  & इह्‍युस्त्यनु  \\ 
\midrule
उङ इन् & ह:न्‍न्य  & ह:न्त्यनि  \\ 
उङ एऽचि & ह:न्सु  & ह:न्त्यान्सु   \\ 
उङ ए:न्& ह:न्‍नु  & ह:न्त्यान्‍नु   \\ 
\bottomrule
\end{tabular}
\end{table}


\begin{table}[H]
\label{ukt.vt} \centering
\caption{सकर्मक क्रिया  कु:न्य  "नुहाउनु"  }
\begin{tabular}{l|l|l|l|l|l|l|l|l|l|l|l|l}  \toprule
&अभूत & भूत & आज्ञार्थक \\ 
उङ &कुक्‌तु &कुक्‌त \\ 
उङ अम्सु&कुक्‌तुसु &कुक्‌तसु \\ 
उङ अम्ह्‍याम्&कुक्‌तुनु &कुक्‌तनु \\ 
इऽचिअ्य &क्युकि &क्युकिति   \\ 
ओऽचुअ        &क्युकु &क्युकुतु   \\ 
इक्अ्य&कुक्‌कि &कुक्‌तिकि   \\ 
ओक्अ &कुक्‌क &कुक्‌तक   \\ 
इन्य & इकुक्‌त्यु  & इकुक्‌त्य &कुक्‌ते  \\ 
इन्य अम्सु& इकुक्सु  & इकुक्‌त्यसु   \\ 
इन्य अम्ह्‍याम्& इकुक्‍नु  & इकुक्‌त्यनु   \\ 
एऽचि & इक्युकि & इक्युकिति &क्युकिये    \\ 
ए:न् & इकु:नि  & इक्युक्‌त्यनु &क्युक्‍नुये  \\ 
अम्अ्य & कुक्‌त्यु  & कुक्‌त्य  \\ 
अम्सुअ्य & कुक्सु & कुक्‌त्यसु  \\ 
अम्ह्‍याम्अ्य & कुक्‍नु  & कुक्‌त्यनु \\ 
\midrule
इन्य, अम्अ्य उ:ङ्‌ &इकु:ङ &इक्युकत &क्युकये \\ 
एऽचिअ्य/अम्सुअ उ:ङ्‌ &इकु:ङसु &इक्युकतसु &क्युकसुये \\ 
ए:न्अ्य/अम्ह्‍याम्अ्य उ:ङ्‌ &इकु:ङनु &इक्युकतनु &क्युकनुये \\ 
इन्य/अम्अ्य इऽचि &इक्युकि &इक्युकिति    \\ 
इन्य/अम्अ्य ओऽचु &इक्युकु &इक्युकुतु  &क्युकुये  \\ 
इन्य/अम्अ्य इक् &इकुक्‌कि &इकुक्‌तिकि   \\ 
इन्य/अम्अ्य ओक् &इकुक्‌क &इकुक्‌तक  &कुक्‌कये  \\ 
अम्अ्य इन् & इकु: & इक्युक्‌त्य   \\ 
अम्अ्य एऽचि & इक्युकि & इक्युकिति    \\ 
अम्अ्य ए:न् & इकु:नि  & इक्युक्‌त्यनु  \\ 
\midrule
उङ इन् & कु:न्य  & कु:न्त्यनि  \\ 
उङ एऽचि & कु:न्सु  & कु:न्त्यान्सु   \\ 
उङ ए:न्& कु:न्‍नु  & कु:न्त्यान्‍नु   \\ 
\bottomrule
\end{tabular}
\end{table}


\begin{table}[H]
\label{umt.vt} \centering
\caption{सकर्मक क्रिया  तम्‍न्य  "खोज्नु"  }
\begin{tabular}{l|l|l|l|l|l|l|l|l|l|l|l|l}  \toprule
&अभूत & भूत & आज्ञार्थक \\ 
उङ &तम्दु &त:म्त \\ 
उङ अम्सु&तम्दुसु &त:म्तसु \\ 
उङ अम्ह्‍याम्&तम्दुनु &त:म्तनु \\ 
इऽचिअ्य &त्युमि &त्युमिति   \\ 
ओऽचुअ        &त्युमु &त्युमुतु   \\ 
इक्अ्य&तम्कि &तम्तिकि   \\ 
ओक्अ &तम्क &तम्तक   \\ 
इन्य & इतम्द्‌यु  & इत:म्त्य &तम्दे  \\ 
इन्य अम्सु& इत:म्सु  & इत:म्त्यसु   \\ 
इन्य अम्ह्‍याम्& इत:म्‍नु  & इत:म्त्यनु   \\ 
एऽचि & इत्युमि & इत्युमिति &त्युमिये    \\ 
ए:न् & इतम्‍नि  & इत्यु:म्त्यनु &त्यु:म्‍नुये  \\ 
अम्अ्य & तम्द्‌यु  & त:म्त्य  \\ 
अम्सुअ्य & त:म्सु & त:म्त्यसु  \\ 
अम्ह्‍याम्अ्य & त:म्‍नु  & त:म्त्यनु \\ 
\midrule
इन्य, अम्अ्य उ:ङ्‌ &इतम्ङ &इत्युमत &त्युमये \\ 
एऽचिअ्य/अम्सुअ उ:ङ्‌ &इतम्ङसु &इत्युमतसु &त्युमसुये \\ 
ए:न्अ्य/अम्ह्‍याम्अ्य उ:ङ्‌ &इतम्ङनु &इत्युमतनु &त्युमनुये \\ 
इन्य/अम्अ्य इऽचि &इत्युमि &इत्युमिति    \\ 
इन्य/अम्अ्य ओऽचु &इत्युमु &इत्युमुतु  &त्युमुये  \\ 
इन्य/अम्अ्य इक् &इतम्कि &इतम्तिकि   \\ 
इन्य/अम्अ्य ओक् &इतम्क &इतम्तक  &तम्कये  \\ 
अम्अ्य इन् & इतम् & इत्यु:म्त्य   \\ 
अम्अ्य एऽचि & इत्युमि & इत्युमिति    \\ 
अम्अ्य ए:न् & इतम्‍नि  & इत्यु:म्त्यनु  \\ 
\midrule
उङ इन् & तम्‍न्य  & तम्त्यनि  \\ 
उङ एऽचि & त:म्सु  & तम्त्यान्सु   \\ 
उङ ए:न्& त:म्‍नु  & तम्त्यान्‍नु   \\ 
\bottomrule
\end{tabular}
\end{table}


\begin{table}[H]
\label{unt.vt} \centering
\caption{सकर्मक क्रिया  फ्लैन्य  "फुकाल्नु"  }
\begin{tabular}{l|l|l|l|l|l|l|l|l|l|l|l|l}  \toprule
&अभूत & भूत & आज्ञार्थक \\ 
उङ &फ्लन्दु &फ्ल:न्त \\ 
उङ म्यसु &फ्लन्दुसु &फ्ल:न्तसु \\ 
उङ म्यह्‍याम् &फ्लन्दुनु &फ्ल:न्तनु \\ 
इऽचिअ्य &फ्ल्युऽचि &फ्ल्युऽस्ति   \\ 
ओऽचुअ &फ्ल्युऽचु &फ्ल्युऽस्तु   \\ 
इक्अ्य &फ्लैकि &फ्लैतिकि   \\ 
ओक्अ &फ्लैक &फ्लैतक   \\ 
इन्य अम् & इफ्लन्द्‌यु  & इफ्ल:न्त्य &फ्लन्दे  \\ 
इन्य म्यसु & इफ्ल:न्सु  & इफ्ल:न्त्यसु   \\ 
इन्य म्यह्‍याम् & इफ्ल:न्‍नु  & इफ्ल:न्त्यनु   \\ 
एऽचिअ्य & इफ्ल्युऽचि & इफ्ल्युऽस्ति &फ्ल्युऽचिये    \\ 
ए:न्अ्य & इफ्लैनि  & इफ्ल्युऽस्त्यनु &फ्ल्युऽस्‍नुये  \\ 
अम्अ्य & फ्लन्द्‌यु  & फ्ल:न्त्य  \\ 
अम्सुअ & फ्ल:न्सु & फ्ल:न्त्यसु  \\ 
अम्ह्‍याम्अ्य & फ्ल:न्‍नु  & फ्ल:न्त्यनु \\ 
\bottomrule
\end{tabular}
\end{table}


\begin{table}[H]
\label{uŋt.vt} \centering
\caption{सकर्मक क्रिया  थुऽन्य  "घोच्नु"  }
\begin{tabular}{l|l|l|l|l|l|l|l|l|l|l|l|l}  \toprule
&अभूत & भूत & आज्ञार्थक \\ 
उङ &थुन्दु &थु:न्त \\ 
उङ अम्सु&थुन्दुसु &थु:न्तसु \\ 
उङ अम्ह्‍याम्&थुन्दुनु &थु:न्तनु \\ 
इऽचिअ्य &थ्युङि &थ्युङिति   \\ 
ओऽचुअ        &थ्युङु &थ्युङुतु   \\ 
इक्अ्य&थुङ्‌कि &थुङ्‌तिकि   \\ 
ओक्अ &थुङ्‌क &थुङ्‌तक   \\ 
इन्य & इथुन्द्‌यु  & इथु:न्त्य &थुन्दे  \\ 
इन्य अम्सु& इथु:न्सु  & इथु:न्त्यसु   \\ 
इन्य अम्ह्‍याम्& इथु:न्‍नु  & इथु:न्त्यनु   \\ 
एऽचि & इथ्युङि & इथ्युङिति &थ्युङिये    \\ 
ए:न् & इथुऽनि  & इथ्यु:ङ्‌त्यनु &थ्यु:ङ्‌नुये  \\ 
अम्अ्य & थुन्द्‌यु  & थु:न्त्य  \\ 
अम्सुअ्य & थु:न्सु & थु:न्त्यसु  \\ 
अम्ह्‍याम्अ्य & थु:न्‍नु  & थु:न्त्यनु \\ 
\midrule
इन्य, अम्अ्य उ:ङ्‌ &इथुऽङ &इथ्युङत &थ्युङये \\ 
एऽचिअ्य/अम्सुअ उ:ङ्‌ &इथुऽङसु &इथ्युङतसु &थ्युङसुये \\ 
ए:न्अ्य/अम्ह्‍याम्अ्य उ:ङ्‌ &इथुऽङनु &इथ्युङतनु &थ्युङनुये \\ 
इन्य/अम्अ्य इऽचि &इथ्युङि &इथ्युङिति    \\ 
इन्य/अम्अ्य ओऽचु &इथ्युङु &इथ्युङुतु  &थ्युङुये  \\ 
इन्य/अम्अ्य इक् &इथुङ्‌कि &इथुङ्‌तिकि   \\ 
इन्य/अम्अ्य ओक् &इथुङ्‌क &इथुङ्‌तक  &थुङ्‌कये  \\ 
अम्अ्य इन् & इथुङ्‌ & इथ्यु:ङ्‌त्य   \\ 
अम्अ्य एऽचि & इथ्युङि & इथ्युङिति    \\ 
अम्अ्य ए:न् & इथुऽनि  & इथ्यु:ङ्‌त्यनु  \\ 
\midrule
उङ इन् & थुऽन्य  & थुन्त्यनि  \\ 
उङ एऽचि & थु:न्सु  & थुन्त्यान्सु   \\ 
उङ ए:न्& थु:न्‍नु  & थुन्त्यान्‍नु   \\ 
\bottomrule
\end{tabular}
\end{table}


\begin{table}[H]
\label{urt.vt} \centering
\caption{सकर्मक क्रिया  कर्न्य  "ल्याईदिनु"  }
\begin{tabular}{l|l|l|l|l|l|l|l|l|l|l|l|l}  \toprule
&अभूत & भूत & आज्ञार्थक \\ 
उङ &कर्दु &क:र्त \\ 
उङ अम्सु&कर्दुसु &क:र्तसु \\ 
उङ अम्ह्‍याम्&कर्दुनु &क:र्तनु \\ 
इऽचिअ्य &क्युरि &क्युरिति   \\ 
ओऽचुअ        &क्युरु &क्युरुतु   \\ 
इक्अ्य&कर्कि &कर्तिकि   \\ 
ओक्अ &कर्क &कर्तक   \\ 
इन्य & इकर्द्‌यु  & इक:र्त्य &कर्दे  \\ 
इन्य अम्सु& इक:र्सु  & इक:र्त्यसु   \\ 
इन्य अम्ह्‍याम्& इक:र्नु  & इक:र्त्यनु   \\ 
एऽचि & इक्युरि & इक्युरिति &क्युरिये    \\ 
ए:न् & इकर्नि  & इक्यु:र्त्यनु &क्यु:र्नुये  \\ 
अम्अ्य & कर्द्‌यु  & क:र्त्य  \\ 
अम्सुअ्य & क:र्सु & क:र्त्यसु  \\ 
अम्ह्‍याम्अ्य & क:र्नु  & क:र्त्यनु \\ 
\midrule
इन्य, अम्अ्य उ:ङ्‌ &इकर्ङ &इक्युरत &क्युरये \\ 
एऽचिअ्य/अम्सुअ उ:ङ्‌ &इकर्ङसु &इक्युरतसु &क्युरसुये \\ 
ए:न्अ्य/अम्ह्‍याम्अ्य उ:ङ्‌ &इकर्ङनु &इक्युरतनु &क्युरनुये \\ 
इन्य/अम्अ्य इऽचि &इक्युरि &इक्युरिति    \\ 
इन्य/अम्अ्य ओऽचु &इक्युरु &इक्युरुतु  &क्युरुये  \\ 
इन्य/अम्अ्य इक् &इकर्कि &इकर्तिकि   \\ 
इन्य/अम्अ्य ओक् &इकर्क &इकर्तक  &कर्कये  \\ 
अम्अ्य इन् & इकर् & इक्यु:र्त्य   \\ 
अम्अ्य एऽचि & इक्युरि & इक्युरिति    \\ 
अम्अ्य ए:न् & इकर्नि  & इक्यु:र्त्यनु  \\ 
\midrule
उङ इन् & कर्न्य  & कर्त्यनि  \\ 
उङ एऽचि & क:र्सु  & कर्त्यान्सु   \\ 
उङ ए:न्& क:र्नु  & कर्त्यान्‍नु   \\ 
\bottomrule
\end{tabular}
\end{table}
