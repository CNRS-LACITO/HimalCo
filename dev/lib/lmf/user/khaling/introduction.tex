\section{Why a verb dictionary?}

While it might seem unusual to have a dictionary made up entirely of verbs, we feel that this is the most impactful contribution we can make to Khaling lexical studies.

Traditionally, dictionaries list verbs according to their infinitive form -- in Khaling, the infinitive is the form that ends in -न्य \ipa{-nɛ}.  The complexity of Khaling verbs is such that the infinitive of a verb does not provide sufficient information  to be able to accurately conjugate the verb.

Take for example the verb कर्न्य \ipa{kʌ̄rnɛ}: it has two meanings, `carry' and `bring for someone'.Their infinitives are the same, yet when conjugated, for example with a 1st person singular subject in the non-past, the forms are different:

\begin{itemize}
\item उङ कुरु  \ipa{ʔuŋʌ kuru}, `I carry'
\item उङ कर्दु \ipa{ʔuŋʌ kʌ̄rdu}, `I bring for someone'
\end{itemize}

The reason for this is that the verb roots are different -- one is \ipa{kur}, the other is \ipa{kurt} -- even though their infinitives are pronounced exactly the same way.

This verb dictionary's mission is to provide complete information about all verbs, allowing them to be conjugated correctly. Each verb is thus listed by its root form, alongside the more familiar infinitive, and we also provide verb conjugation tables for each type of verb, so that any verb can be conjugated in any of its forms.
 
\section{How to use this dictionary and its conjugation tables}
Verbs are listed by infinitive form followed by a form in parentheses, which is the verb root.

The link at the bottom of the entry refers to the correct conjugation table.

You will need to determine what the nature of the subject (also called agent, if it carries out the action) and of the object (also called patient), if present.  You will also need to determine whether you want a non-past (=present or future), past, or imperative (=command) form of the verb.

\begin{table}[H]
\caption{Personal pronouns in Khaling} \label{tab:pro} \centering
\begin{tabular}{llllllllllllllllllll}
\toprule
1S  & 	\ipa{ʔûŋ}  & उ:ङ्& 	or  & 	\ipa{ʔuŋʌ}  &  उङ \\
1DI  & 	\ipa{ʔīːtsi}  & ईचि& 	or  & 	\ipa{ʔīːtsiʔɛ}  &ईचिअ्य \\
1DE  & 	\ipa{ʔōːtsu}  &ओऽचु & 	or  & 	\ipa{ʔōːtsuʔɛ}  &ओऽचुअ्य \\
1PI  & 	\ipa{ʔik}  &इक् & 	or  & 	\ipa{ʔikʔɛ}  &इक्अ्य  \\
1PE  & 	\ipa{ʔok}  & ओक् & 	or  & 	\ipa{ʔokʔɛ}  &ओक्अ्य  \\
2S  & 	\ipa{ʔin}  & इन्& 	or  & 	\ipa{ʔinɛ}  &इन्यअ्य \\
2D  & 	\ipa{ʔēːtsi}  & एचि & 	or  & 	\ipa{ʔēːtsiʔɛ}  &एचिअ्य \\
2P  & 	\ipa{ʔên}  &ए:न् & 	or  & 	\ipa{ʔênʔɛ}  & ए:न्अ्य \\
3S  & 	\ipa{ʔʌ̄m}  &अम् & 	or  & 	\ipa{ʔʌ̄mʔɛ}  &अम्अ्य \\
3D  & 	\ipa{ʔʌ̄msu}  & अम्‌सु& 	or  & 	\ipa{ʔʌ̄msuʔɛ}  & अम्‌सुअ्य\\
3P  & 	\ipa{ʔʌ̄mɦɛm}  & अम्‌ह्‌याम् & 	or  & 	\ipa{ʔʌ̄mɦɛmʔɛ}  &म्‌ह्‌याम्अ्य \\
\bottomrule
\end{tabular}
\end{table}

X>Y	: X is the subject/ agent, Y is the object/patient.

Once you determine the correct combination of subject and object, and whether you need a non-past, past or imperative form, you will need to look at the example verb which is in the corresponding slot of the verb conjugation table.

The example verb will have the same ending but probably not the same beginning as the verb you wish to conjugate. You will need to substitute the correct beginning consonant(s) so they match the verb you are conjugating.

For instance, you want to conjugate कर्न्य \ipa{kʌ̄rnɛ}, 'carry', with a \textsc{1sg} agent (उङ \ipa{ʔuŋʌ}) and a  \textsc{3sg}  patient (अम् \ipa{ʔʌ̄m}, or any singular noun), in the non-past.

The verb dictionary will give you the verb root \ipa{kur} as corresponding to 'carry'. This is a transitive verb, and its root ends in the VC combination \ipa{-ur}, which corresponds to Tables \ref{ur.vt.eng} (in English and IPA) and \ref{ur.vt} (in Nepali).

The \textsc{1s$\rightarrow$3s} non-past cell of the conjugation table gives you: सुरु \ipa{suru} (because the example verb is सर्न्य \ipa{sʌ̄rnɛ}, `wash'.

You must change the first letter from स \ipa{s} to क \ipa{k} (सर्न्य \ipa{sʌ̄rnɛ} > कर्न्य \ipa{kʌ̄rnɛ}) to obtain the correct form: सुरु \ipa{suru} > कुरु \ipa{kuru}.

\section{खालिङ् क्रियापद शब्दकोष बारेमा}
विश्व मानचित्रमा नेपाल सानो भए पनि बहुजातिहरुको थलो हो । बहुजाति, बहुभाषा अनि बहुसंस्कृति नेपालको परिचय हो । जनगणना २०६८ अनुसार नेपालमा १२३ भाषाहरु छन् । यी सबै भाषाहरुलाई नेपाल सरकारले राष्ट्रभाषाको मान्यता दिएको छ । नेपालका आदिवासी किरातीहरुले बोल्‍ने भाषाहरु (SINO-TIBETAN) चिनीय-भोट भाषा परिवारको भोटबर्मेली शाखाभित्र पर्दछन् । यीमध्ये खालिङ भाषा पनि एक हो । खालिङ भाषा बोल्‍ने किरात खालिङ जातिको मुख्य थलो सर्वोच्‍च शिखर सगरमाथा/माउण्ट एभेरेष्ट/चोमोलुङ्‍माको  काखमा अवस्थित सोलुखुम्बु जिल्‍लामा रहेका जुबिङ्, वाकु, पावै, काँकु, बासा अनि टाक्सिन्दु गा‍‍विसहरु हुन् । यी गाविसका खालिङ्‍हरु नै खालिङ् भाषाका मुख्य बक्ताहरु हुन् । सोलुखुम्बुका अन्य गाविसहरुमा पनि खालिङ् जातिको बसोबास रहेका छन् । खोटाङ्, संखुवासभा, भोजपुर, काठमाण्डौ, ललितपुर, भक्तपुर, पोखरा, सुनसरी, मोरङ्, झापा, पाँचथर, ईलाम आदि विभिन्‍न जिल्‍लाहरु र विदेशमा समेत खालिङ् भाषीहरुको बसोबास रहेको छ । 
अन्य भाषाभन्दा रोचक विशेषताहरु भएको खालिङ समुदायमा बोलिने लयात्मक, सरस अनि सरल भाषा हो “खालिङ भाषा” । सोलुखुम्बु जिल्‍लाका ज्युब्यु, वाकु, फल्‍ले, राप्छा, बस, ख्यास्त्याप्, काङ्‍कु गाँउहरुमा बोलिने भाषा र लवजलाई आधार मानी यो शब्दकोष तयार पारिएको हो । यो भाषालाई बढी वैज्ञानिक र व्यापक बनाउनको लागि देवनागरि लिपिमा लेखिएको छ । सकेसम्म सहज रुपमा सबैले बुझ्‍न सकुन् र खालिङ भाषाको ध्वनिलाई दुरुस्त उच्‍चारण गर्न सकियोस भन्‍नेमा हाम्रो प्रयत्‍न रहेको छ । 

यो पुस्तक सिङ्‍गो खालिङ समुदायको लागि उपयोगी हुनेछ । साथै अन्य समुदायका भाषाप्रेमी स्वदेशी तथा विदेशीहरुको लागि पनि यो पुस्तक उपयोगी हुनेछ भन्‍ने आशा छ । 

यो शब्दकोष तयार गर्दा सहयोग गरिदिनु हुने- धनमाया (मलिम्), एकप्रसाद, पबिधन, नौबहादुर (ऐनस्लाल्) धर्मेन्द्र, सौकुमार, भीम, जानकी, भैराज, पूर्ण (संगीत प्रभाकार), तत्कालिन संघको केन्द्रीय अध्यक्ष राम रतन लगयत तत्कालिन किरात खालिङ उत्थान संघका सम्पूर्ण केन्द्रीय कार्य समिति, संघको जिल्‍ला अध्यक्ष महेश लगयत किरात खालिङ उत्थान संघका सम्पूर्ण सोलुखुम्बु जिल्‍ला कार्य समिति, स्थानीयबासी आशवीर, रण, देवीकुमारी, पुर्ख पूर्णबहादुर (जिरिसे), पुर्ख खबरे, पुर्ख बिजमे, पुर्ख छत्रमान, पुर्ख देउनस्, बिजम्‌फुल्, मज्जति (कबिमाया), तम्रमी रासधन, पुर्ख धन, उत्रबहादुर (फोङ्‍साइला) आदि खालिङ पुर्खहरु धन्यवादका पात्रा हुन् ।
लामो समय सम्मको प्रयास पश्चात “खालिङ् क्रियापद शब्दकोष” पाठक सामु प्रस्तुत गर्न पाउदा हर्षितछौं । अन्तमा तपाईको अमूल्य सुझावहरुको पनि अपेक्षा गर्दछौं, जसले भाषाको निरन्तर कार्यको लागि सूनमा सुगन्ध थपिनेछ ।
अलस्यो, धन्यवाद ।
-धन खालिङ, हान्दुङ् 





यो शब्दकोष प्रयोग गर्ने तरिकाको सन्दर्भमा:-

१)	देवनागरि लिपि प्रयोग गरिएकोले यसैको नियम अनुसार लेख्‍ने प्रयास गरिएको छ ।

२)	खालिङ् भाषाका हलन्त नभएको अक्षर पूरै उच्‍चारण गर्नु पर्नेछ । 
\begin{itemize}
\item जस्तै:- उङ यि:उ जङ – म भात खान्छु । पकतम्स – पुर्खपखानी । तेऽर – गोगुन ।
\item  तु:ब – एउटा । मेखि मक – होनि तब । सुऽत – काँचो, नपाकेको । लुऽज- कोदो । 
\end{itemize}
खालिङ भाषाको विशेष उदत्त र स्वरित ध्वनि छन् । लेख्‍ने तरिका उस्तै भए पनि उदत्त र स्वरित ध्वनिको कारण अर्थमा धेरै अन्तर हुन्छ ।

३) अवग्रह:- सँस्कृत, हिन्दी र बर्मी भाषामा (Long) Level tone को लागि अवग्रह  ' ऽ ' उदत्त ध्वनिको प्रयोग भएको पाइन्छ । खालिङ भाषाको कुनै शब्दहरु उच्‍चारण गर्दा लामो स्वर तान्‍नु पर्ने हुन्छ । यस्तो अवस्थामा उदत्त ध्वनिको लागि अवग्रह प्रयोग to indicate long vowel गरिएको छ । जस्तै:-
\begin{itemize}
\item काऽन्य – भान बसाल्‍नु ।
\item केऽन्य – सेलाउनु ।
\item चेऽन्य – फल् चेऽन्य - हल्क कम्पन्कासाथ नाङ्लोमा चामल या च्याख्ल छिन्क्याउनु ।
\item प्रोऽन्य – ठिक हुनु, राम्रो हुनु ।
\item मिऽस्त्य – पाक्यो । 
\item रेऽन्य – हेला गर्नु ।
\item हाऽन्य – कोक्याउनु ।
\end{itemize}

४) विस\reph{}ग :-	 स्वरित ध्वनि Falling tone
बर्मी भाषा र चिनिया भाषाका स्वरित ध्वनिको संकेत गर्न विस\reph{}ग प्रयोग गरेको पाइन्छ । खालिङ भाषामा पनि स्वरित ध्वनि भएको स्थानमा विस\reph{}ग ( : ) प्रयोग गरिएको छ । विस\reph{}गको प्रयोगले खालिङ शब्दको वास्तविक उच्‍चारण गर्न सकिने छ । 
जस्तै:-
\begin{itemize}
\item  का:न्य – केलाउनु, खन्‍नु, ठुङ्‍नु, डस्‍नु ।
\item  के:न्य – केलाउनु ।
\item  चे:न्य – चिमोट्‍नु, (साग सिस्‍नु आदि) टिप्‍नु, थुन्‍नु ।
\item  प्रो:न्य – उफ्रनु । 
\item  मिस्त्य –  मर्‍यो ।
\item  रे:न्य – लेख्‍नु । 
हा:न्य – उघार्नु, खोल्‍नु ।
\end{itemize}

सेइबी ख्योस्त्य –  हेर्न गयो । से:इबी ख्योस्त्य –  मार्न गयो ।
र्याीम्‍न्य – आगोको रापले डढ्‍नु, रापले खैरो हुनु वा पार्नु । र्या :म्‍न्य – पिट्‍नु । 
तिकी: – कुनै आवाजको संकेत । तिकि –  हामी (दुई) बिसाऔं ।

५) अभिश्रुति:- खालिङ भाषाका “अ‌्य, अ्या, अ्यो, अ्यु, अ्वा” अभिश्रुतिहरु हुन् । 
' अ्य ' र ' अ्या ' को प्रयोग:- अम्अ्य अ्यास्त्य – उसले भन्यो ।
' अ्यो ',  ' य्वा '  र  ' अ्वा ' को प्रयोग -  
अ्वार्म्यपुङ् अ्योअ्यो य्वाह्इप्य मत्त्यु – सुनाखरीको फूल पहेलो हुन्छ ।
' अ्यु ' को प्रयोग – 
हातिअ्य लेङाक्सी अ्यु:ल्त्य – हात्तिले केरा उखाल्यो ।
झुङ् के:ल्त्यलो सङ् अ्युऽस्त्य – हुरी चल्दा रुख ढल्यो ।

देवनागरि लिपिमा खालिङ भाषाका वर्णहरु
\begin{itemize}
\item स्वर वर्ण:- 

अ आ इ उ ए ओ 

\item अभिश्रुति:- 

अ्य, अ्या, अ्यो, अ्यु, अ्यू, अ्वा 

\item व्याञ्‍जन:-

क ख ग घ ङ    

च छ ज झ  

(ट ठ ड ढ) यो व\reph{}गबाट धेरै कम शब्दहरु पाईएका छन् । जस्तै:- टाक्टव्य, डग्डव्य, ठोकचुरा, ढ्वाङ् आदि ।
 
त थ द ध न 

प फ ब भ म    

य र ल व    

स  ह

य व - अर्ध स्वरवर्ण ।

\end{itemize}
