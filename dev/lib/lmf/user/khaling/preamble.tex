\documentclass[oldfontcommands,oneside,a4paper,11pt]{article}
\title{Khaling dictionary}
\author{Guillaume Jacques, Aimée Lahaussois, Dhan Bahadur Rai, Yadav Kumar}
\usepackage{pifont}
\usepackage{fontspec}
\usepackage{booktabs}
\usepackage{xltxtra}
\usepackage{polyglossia}
\usepackage[table]{xcolor}
\usepackage{float}
\usepackage{memhfixc}
\usepackage{amssymb}
\usepackage{multicol}
\setlength{\columnseprule}{1pt}
\setlength{\columnsep}{1.5cm}
\setmainfont[Script=Devanagari]{Sahadeva}
%\setmainfont[Script=Devanagari]{Sanskrit 2003}
\newfontfamily\english[]{CharisSIL}
\newcommand{\eng}[1]{{\english #1}}
\newfontfamily\phon[Mapping=tex-text,Ligatures=Common,Scale=MatchLowercase,FakeSlant=0.3]{CharisSIL}
\newcommand{\ipa}[1]{{\phon #1}} % API en italique
\newfontfamily\cn[Mapping=tex-text,Ligatures=Common,Scale=MatchUppercase]{SimSun} % pour le chinois
\newcommand{\zh}[1]{{\cn #1}}
\newfontfamily\mx[Mapping=tex-text,Ligatures=Common,Scale=MatchUppercase]{ArialUnicodeMS} % pour les questions
\newcommand{\nq}[1]{{\mx #1}}
%\newfontfamily\sktfont[Script=Devanagari]{Sanskrit 2003}
\newfontfamily\sktfont{Sanskrit 2003}
\newcommand{\skt}[1]{{\sktfont #1}}
\newfontfamily\mangalfont{Mangal}
\newcommand{\mgl}[1]{{\mangalfont#1}}
\newcommand{\reph}{\vphantom{}र्}
\XeTeXlinebreaklocale "zh" % 使用中文换行
\XeTeXlinebreakskip = 0pt plus 1pt
\usepackage{fancyhdr}
\pagestyle{fancy}
\fancyheadoffset{3.4em}
\usepackage[dvipdfmx,xetex,bigfiles,final,activate=onclick,deactivate=onclick,transparent,passcontext]{media9}
\usepackage{graphicx}
\usepackage[bookmarks=true,colorlinks,linkcolor=blue]{hyperref}
\usepackage{gb4e}
\usepackage{vmargin}
% {marge gauche}{marge en haut}{marge droite}{marge en bas}{hauteur de l'entête}{distance entre l'entête et le texte}{hauteur du pied de page}{distance entre le texte et le pied de page}
\setmarginsrb{2cm}{1cm}{1.5cm}{1cm}{0.5cm}{1cm}{0.5cm}{1cm}
