\documentclass[oldfontcommands,oneside,a4paper,11pt]{article}
\title{Japhug dictionary}
\author{Guillaume Jacques}
\usepackage{fontspec}
\usepackage{booktabs}
\usepackage{xltxtra}
\usepackage{polyglossia}
\usepackage[table]{xcolor}
\usepackage{float}
\usepackage{memhfixc}
\usepackage{multicol}
\setmainfont[Mapping=tex-text,Numbers=OldStyle,Ligatures=Common]{CharisSIL}
\newfontfamily\phon[Mapping=tex-text,Ligatures=Common,Scale=MatchLowercase,FakeSlant=0.3]{CharisSIL}
\newcommand{\ipa}[1]{{\phon #1}} % API en italique
\newfontfamily\cn[Mapping=tex-text,Ligatures=Common,Scale=MatchUppercase]{SimSun} % pour le chinois
\newcommand{\zh}[1]{{\cn #1}}
\newfontfamily\mx[Mapping=tex-text,Ligatures=Common,Scale=MatchUppercase]{ArialUnicodeMS} % pour les questions
\newcommand{\nq}[1]{{\mx #1}}
\XeTeXlinebreaklocale "zh" % 使用中文换行
\XeTeXlinebreakskip = 0pt plus 1pt
\usepackage[dvipdfmx,xetex,bigfiles,final,activate=onclick,deactivate=onclick,windowed=false,transparent,passcontext]{media9}
\usepackage{graphicx}
\usepackage[colorlinks,linkcolor=blue]{hyperref}
\usepackage{gb4e}

\addmediapath{/Users/celine/Work/CNRS/workspace/HimalCo/dict/japhug/data/audio/mp3/}
\addmediapath{/Users/celine/Work/CNRS/workspace/HimalCo/dict/japhug/data/audio/wav/}
\graphicspath{{/Users/celine/Work/CNRS/workspace/HimalCo/dev/lib/lmf/src/output/img/}}
