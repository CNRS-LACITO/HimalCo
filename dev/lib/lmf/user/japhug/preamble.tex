\documentclass[oldfontcommands,oneside,a4paper,11pt]{article} 
\title{Dictionnaire Japhug-chinois-français}
\author{Guillaume Jacques}
\usepackage{pifont}
\usepackage{fontspec}
\usepackage{natbib}
\usepackage{booktabs}
\usepackage{xltxtra} 
\usepackage{longtable}
\usepackage{polyglossia}
\setdefaultlanguage{french}
\usepackage[table]{xcolor}
\usepackage{multicol}
\setlength{\columnseprule}{1pt}
\setlength{\columnsep}{1.5cm}
\usepackage{float}
\usepackage{lineno}
\usepackage{textcomp}
\usepackage[bookmarks=true,colorlinks,linkcolor=blue]{hyperref}
\hypersetup{bookmarks=false,bookmarksnumbered,bookmarksopenlevel=5,bookmarksdepth=5,xetex,colorlinks=true,linkcolor=blue,citecolor=blue}
\usepackage[all]{hypcap}
\usepackage{memhfixc}
\usepackage{amssymb}
\usepackage{lscape}
%\setmainfont[Mapping=tex-text,Numbers=OldStyle,Ligatures=Common]{Charis SIL} 
\newfontfamily\phon[Mapping=tex-text,Ligatures=Common,Scale=MatchLowercase,FakeSlant=0.3]{Charis SIL} 
\newcommand{\ipa}[1]{{\phon #1}} % API toujours en italique
\newfontfamily\cn[Mapping=tex-text,Ligatures=Common,Scale=MatchUppercase]{MingLiU} % (ou SimSun) pour le chinois
\newcommand{\zh}[1]{{\cn #1}}
\newfontfamily\mx[Mapping=tex-text,Ligatures=Common,Scale=MatchUppercase]{ArialUnicodeMS} % pour les questions
\newcommand{\nq}[1]{{\mx #1}}
\newcommand{\grise}[1]{\cellcolor{lightgray}\textbf{#1}}
\newcommand{\bleute}[1]{\cellcolor{green}\textbf{#1}}
\newcommand{\rouge}[1]{\cellcolor{red}\textbf{#1}}
\newcommand{\topic}{\textsc{dem}}
\newcommand{\tete}{\textsuperscript{\textsc{head}}}
\newcommand{\rc}{\textsubscript{\textsc{rc}}}
\XeTeXlinebreaklocale 'zh' % 使用中文换行
\XeTeXlinebreakskip = 0pt plus 1pt % CIRCG
 \usepackage{fancyhdr}
\pagestyle{fancy}
\fancyheadoffset{3.4em}
\usepackage[dvipdfmx,xetex,bigfiles,final,activate=onclick,deactivate=onclick,transparent,passcontext]{media9}
\usepackage{graphicx}
\usepackage{gb4e}
\usepackage{vmargin}
% {marge gauche}{marge en haut}{marge droite}{marge en bas}{hauteur de l'entête}{distance entre l'entête et le texte}{hauteur du pied de page}{distance entre le texte et le pied de page}
\setmarginsrb{2cm}{1cm}{1.5cm}{1cm}{0.5cm}{1cm}{0.5cm}{1cm}
