\pagenumbering{roman}
	{\LARGE \textbf{Introduction}}
	\section{About the language} \label{sec:language}

This dictionary documents the lexicon of the Na language (\ipa{nɑ˩-ʐwɤ˥}) as spoken in and around the plain of Yongning, located in Southwestern China, at the border between Yunnan and Sichuan, at a latitude of 27°50’ N and a longitude of 100°41’ E. This language is known locally as ‘Mosuo'. 

	\section{Chronology and method} \label{sec:method}

The author's fieldwork on Yongning Na began in October 2006, with tone as its main focus (lexical tone, and tonal morphology). This required examining as many lexical items as possible to ensure that no tone category was overlooked, but lexicographic work was not in itself a priority. A list of words was begun through elicitation, and gradually expanded and corrected as narratives were recorded and transcribed; addition of new words was therefore a slow process. An advantage of placing the emphasis on text collection is that a context is available to help clarify the meaning of newly encountered words, also offering a basis for further discussion of their usage. But systematic elicitation of large amounts of vocabulary was not carried out, hence the limited number of entries: currently slightly under 3,000. 

Unless otherwise stated, all the data are from one language consultant, Mrs. Latami Dashilame (\ipa{lɑ˧tʰɑ˧mi˥ ʈæ˧ʂɯ˧-lɑ˩mv˩}; Chinese: \zh{拉它米打史拉么}). She was born in 1950 in the hamlet called \ipa{ə˧lɑ˧-ʁwɤ\#˥} in Na, close to the monastery of Yongning. The administrative coordinates of this village are: Yúnnán province, Lìjiāng municipality, Nínglàng Yí autonomous county, Yǒngníng district, Ālāwǎ village (\zh{云南省丽江市宁蒗彝族自治县永宁乡阿拉瓦村}). The choice to work in one location only, and essentially with one consultant, is, again, based on the investigator's focus on the tone system. There is considerable dialectal diversity within the Na area (much more so than in the Naxi-speaking area); the tone systems of different villages are conspicuously different, and this geographical diversity combines with dramatic differences across social groups, and across generations. The obvious thing to do seemed to be an in-depth description and analysis of the language as spoken by one person (simultaneously making a few forays into other idiolects and dialects). Data from other speakers are indicated using their codes in the author's database of speakers of Naish languages. Table \ref{tab:consul} provides the speaker codes.

\begin{table}[H]
	\caption{Language consultant codes}
	\centering \label{tab:consul}
	\begin{tabular}{lllllll}
		\toprule
		speaker code &   name &  year of birth \\
		\midrule
F4 (main consultant) & \ipa{lɑ˧tʰɑ˧mi˥ ʈæ˧ʂɯ˧-lɑ˩mv˩} & 1950 \\ F5 &  \ipa{ki˧zo˧} & 1973  \\ F6 &  \ipa{tɕʰi˧ɖv\#˥} & 1987 \\ M18 &  \ipa{lɑ˧tʰɑ˧mi˥ ʈæ˧ʂɯ˧-ʈæ˩ʈv˩} & 1972 \\ M21 & \ipa{ho˧dʑɤ˧tsʰe˥} & 1942 \\ M23 & \ipa{ɖɯ˩ɖʐɯ˧} & 1974 \\
		\bottomrule
	\end{tabular}
\end{table}

The list of words as of 2011 was deposited in \href{http://stedt.berkeley.edu/}{the STEDT database}. The same year, under the impetus of Guillaume Jacques and Aimée Lahaussois, plans were made to bring the word list closer to the standards of a full-fledged dictionary. A project was deposited with the Agence Nationale de la Recherche, accepted in 2012, and begun in 2013: \href{http://himalco.huma-num.fr/}{the HimalCo project} (ANR-12-CORP-0006). Céline Buret, a computing engineer, worked with the project team for two years (Nov. 2014-Oct. 2015). She converted the data to the format of the Field Linguist's Toolbox (MDF), then produced scripts for conversion to a XML format complying with the LMF standard, allowing for automatic conversion to an online format as well as to LaTeX files (with PDF as the final output for circulation). In 2015, version 1.0 of the online and PDF versions of the dictionary were produced and published online, along with the source document in MDF (Toolbox) format.

\section{Guide to using the dictionary} \label{sec:howto}

	\subsection{Formats: trilingual Na-Chinese-English or Na-Chinese French} \label{sec:versions}

Entries and examples have translations into English, Chinese and French. Two language settings are offered for the PDF and online dictionary: either Na-Chinese-English, or Na-Chinese-French. The English and the French are not typeset alongside each other in the same document because distinguishing them visually is not obvious, even with the help of typographic devices such as using different fonts and colours. In the author's own experience, it was found that the presence of four languages alongside one another made consultation more difficult; specifically, English translations tended to be a distraction slowing down access to the French translations, and English users may similarly find that French clutters the layout. On the other hand, Chinese characters are visually well-distinct from Latin-based scripts, and so it did not appear necessary to separate the Chinese and produce a Na-Chinese version. Moreover, Chinese translations are often a useful complement to the translation in English (or French), as there are often closer equivalents: for instance \ipa{gɤ˧˥} translates straightforwardly as Chinese \zh{扛} whereas the English translation is more roundabout: ‘to carry on the shoulder'. Users who wish to have access to all the information can download the original file in Toolbox (MDF) format. 

	\subsection{Format of entries} \label{sec:entries}

Each entry contains
\begin{itemize}
	\item \textit{phonological transcription:} the form of the word in phonetic alphabet; tone is indicated in terms of phonological categories
	\item \textit{part of speech:} an indication of the part of speech, using a simple set of labels
	\item \textit{tone:} the tone category of the word. This information is already present in the phonological transcription; having it repeated on its own facilitates searches
	\item \textit{definitions} in Chinese and English
	\item \textit{examples} with translations
	\item \textit{links} to related words, such as synonyms, or constituent parts of complex words 
	\item \textit{classifier:} for nouns, an indication on the more commonly associated classifiers
\end{itemize}

Among examples, those elicited to verify the output of certain combinations of tones are marked as ‘PHONO': examples elicited for the purpose of the phonological study. Proverbs and sayings are marked as ‘PROVERB'.

Some pieces of information are not shown in the PDF and online versions. These are:
\begin{itemize}
	\item An indication of \textit{semantic domain}: ‘society', ‘house', ‘body', ‘plant', ‘animal'... No attempt was made to use a fine-grained classification of the sort found in the WordNet database of English, where nouns, verbs, adjectives and adverbs are grouped into sets of cognitive synonym \citep{Fellbaum 2005}. This is simply a rough division into subsets for convenient sorting; the labelling relies partly on form, and partly on semantic contents. As for other aspects in the dictionary, choices made reflect the investigator's research priorities: for instance, the entries for ‘day’, ‘night’, ‘month’, ‘year’ were tagged as “classifiers", along with all other nouns that can appear immediately after a numeral. This allowed easy extraction of all classifiers for the purpose of a study of the tone patterns of classifiers \citep{Michaud2013}. These lexical items could just as well have been tagged as 'time', in view of their semantic field. The numbers ‘100’, ‘1,000’ and ‘10,000’ were likewise labelled as “classifiers" rather than numerals.
	\item \textit{Notes on past notations:} information tracing the history of notations, from the first fieldwork to the current version. For instance, the entry \ipa{ŋwɤ˧pʰæ˧˥} ‘tile' has a note that indicates that it was initially written with a M.H tone pattern, and with vowel \ipa{æ} in both syllables: *\ipa{ŋwæ˧pʰæ˥}. The note explains that the perception of \ipa{æ} in the first syllable is due to a phonetic tendency towards regressive vowel harmony. Verifications are also consigned in this field. About half the entries have information of this type.
	\item \textit{glosses:} glosses in English, Chinese and French, intended for the glossing of texts. The dictionary adopts the abbreviations recommended in the Leipzig Glossing Rules \citep{Comrie}; all other terms are provided in full. Glosses mostly follow the choices made by \citep{Lidz2010}.
\end{itemize}

	\subsection{Part-of-speech labelling} \label{sec:pos}
	
Dictionary entries carry a part-of-speech label. A rough-and-ready typology has been followed: see the table below. Needless to say, this system has limitations: a refined typology would require subcategories, e.g. defining classifiers as a subset among nouns; and categories such as ‘adverbs' raise greater difficulties, lacking a clear definition.
\begin{table}
	\caption{Parts of speech}
	\centering \label{tab:consul}
	\begin{tabular}{lll}
		\toprule
		label & meaning & Leipzig Glossing Rules? y/n \\
		\midrule
		adj & adjective & y \\
		clf & classifier & y \\
		clitic & (same) & n \\
		cnj & conjunction & n \\
		ideophone & (same) & n \\
		disc.PTCL & discourse particle & n \\
		intj & interjection & n \\
		lnk & linker & n \\
		n & noun & y \\
		num & numeral & n \\
		pref & prefix & n \\
		postp & postposition & n \\
		pro & pronoun & n \\
		suff & suffix & n \\
		v & verb & y \\
		\bottomrule
	\end{tabular}
\end{table}

No attempt was made at including expressive noises in the dictionary, such as the sound \ipa{ɬː}. The meaning of this sound in Na can be characterized in the same way as that of words in the dictionary: the full definition would be that  \textit{it expresses enjoyment of food or drink (‘Yummy!'), and is also used to express admiration of a beautiful object, scene, or prospect}. But a reason for leaving it out is that, unlike interjections,  \ipa{ɬː} is not pronounced on expiratory airflow, but on inspired airflow. The air flows through the sides of the mouth, which is where saliva flows when one's mouth waters. Observations about such sounds (including clicks), like that of gestures, appeared to fall outside the scope of the dictionary.

	\subsection{Loanwords} \label{sec:loan}

Borrowings from Chinese and Tibetan are indicated as such in cases where identification seems straightforward. No efforts at systematic elicitation of borrowings from either language were made, but all loanwords occurring in texts were added to the dictionary. The information provided includes: donor language; form in the donor language; and explanations. When the number of syllables in the borrowed word is the same as in the donor language, the glosses in English (and French) start by the original word followed by two colons and a translation: e.g. ‘\zh{办法}::solution' for \ipa{pæ˧˥hwɤ˧}. 

	\section{Planned improvements and mid-term perspectives} \label{sec:improv}
	
This dictionary is conceived of as work-in-progress: successive versions will be released, probably every two years or so, (i) as an online dictionary in HTML format, (ii) as PDF documents, and (iii) in database format (native Toolbox/MDF format, then, in due course, the successors of this format). 

Planned improvements for future versions include the addition of
\begin{itemize}
	\item \textit{a phonetic transcription of tone as it surfaces on the item pronounced in isolation:} a surface-phonological transcription of tone, in addition to the indication of the underlying tone category
	\item \textit{audio files for each head word:} this function has successfully been tested, but the editing of audio files still needs to be conducted
	\item \textit{links to the entire set of online recordings}: listing all textual occurrences in the lexicon entry, with links to the audio file and its aligned transcription. Textual occurrences ultimately constitute the best resource to document a word's usage. The examples currently presented in the dictionary are few in number, compared to the occurrences in texts; and their context of use may not be clear, despite efforts at clarifying their nature (singling out examples elicited for the purpose of morpho-phonological investigation by the mention ‘PHONO') and at providing contextual information for examples jotted down during fieldwork.
	\item \textit{more cross-references} between entries, pointing to synonyms, etc.
\end{itemize}

Collaborations are welcome for the following improvements :
\begin{itemize}
	\item \textit{the vocabulary of religion:} the field of religion remains mostly unexplored; the main consultant and I both lack the command of Tibetan that would be essential for this part of the investigation, and involvement of consultants from the Yongning monastery did not prove feasible in view of current restrictions on contacts with foreigners
	\item \textit{plants and animals:} as a dweller of the Yongning plain, the main consultant does not have extensive knowledge of wild plants and animals; the number of entries recorded so far remains small, and some definitions are currently limited to general indications such as ‘a type of pine'. To arrive at exact identification, and at more extensive lexicographic coverage, would require collaboration with other consultants, and with botanists.
\end{itemize}

Last but not least, Roselle Dobbs began adding a proposed orthographic representation for each head word, using a transcription that she developed with Na consultants, with a view to use within the Na community. Until this task is completed and orthography added to the online dictionary, requests for further information about orthographic developments should be sent directly to: rosellemay@hotmail.com 

	\section{Other resources about Yongning Na} \label{sec:resources}
	
	In the classical tradition of linguistic fieldwork \citep{Dixon2007}, a language description should include a dictionary, a grammar, and a collection of texts. 
	
	\begin{itemize}
		\item \textit{A set of Na recordings with time-aligned transcriptions} is available from the Pangloss Collection \citep{Michailovsky2014}; the current web address is lacito.vjf.cnrs.fr/pangloss/languages/Na\_en.htm 
		\item \textit{The grammar} is still in its early stages of preparation. A preliminary draft of a book-length study of Na morpho-tonology can be found online: https://halshs.archives-ouvertes.fr/halshs-01094049/document It also contains detailed information on the phonemic analysis.
	\end{itemize}
	
A review of the literature about Na and the other languages of the Naish  group is provided (in Chinese) by \citet{Li2015}. For an English-language introduction, see \citet{Michaud2015b}.

I would gratefully receive any comments or notifications of errors that the reader may wish to bring to my attention: please send e-mail to michaud.cnrs@gmail.com 


	\section{Acknowledgments} \label{sec:ackno}

Many thanks to Picus Ding for putting me in touch with the Mosuo scholar Latami Dashi. Special thanks to Latami Dashi for supporting and encouraging my work with his mother Latami Dashilame over the years. Many thanks to the main consultant, Latami Dashilame, and to all family members. 

Many thanks to Céline Buret and Séverine Guillaume for their much-appreciated computational expertise, and to Guillaume Jacques for suggestions all along the way. Many thanks to connoisseurs of the Na culture and language for useful exchanges: Lamu Gatusa \zh{拉木嘎吐萨} (Chinese pen-name: \zh{石高峰}), Liberty Lidz, Christine Mathieu, Pascale-Marie Milan and He Sana \zh{何梭娜}. Special thanks to Roselle Dobbs for extensive discussions and vigorous proof-reading over the years. Many thanks to A Hui \zh{阿慧} (to my knowledge the first speaker of Mosuo to read a M.A. degree in language and linguistics) for suggesting corrections. Remaining errors are my own responsibility. 

I am grateful for the opportunity allowed me by my home institution, Centre National de la Recherche Scientifique, of staying in China in 2011-2012 for extensive fieldwork, through a temporary affiliation with the CNRS’s research centre in China: CEFC (Centre d’Etudes Français sur la Chine contemporaine). From November 2012 to June 2016, I was based at the international research institute MICA, in Hanoi, in an exceptionally stimulating environment allowing for close collaboration with colleagues from Asia and elsewhere. Special thanks to the heads of the institute, Phạm Thị Ngọc Yến (succeeded in 2015 by Nguyễn Việt Sơn) and Eric Castelli, for their support and encouragement.

I am grateful to the Dongba Culture Research Institute (\zh{丽江市东巴文化研究院}) in Lijiang and the Horse-Tea Road Culture Research Centre (\zh{云南大学茶马古道文化研究所}) in Kunming for inviting me to become an Adjunct member (\zh{外聘研究员}), and for facilitating administrative and practical matters; special thanks to Li Dejing \zh{李德静} and to Mu Jihong\zh{木霁弘}. At Yunnan University, many thanks are due to Duan Bingchang \zh{段炳昌}, Wang Weidong \zh{王卫东}, Zhao Yanzhen \zh{赵燕珍} and Yang Liquan \zh{杨立权} for their careful and sensitive management of fieldwork-related administrative matters.
	
So many people have supported this project that I must apologize for those names that should be here but were inadvertently left off the list.

This work was supported financially by the ANR project HimalCo (ANR-12-CORP-0006), and constitutes a contribution to the LabEx “Empirical Foundations of Linguistics" project (ANR-10-LABX-0083).

\begin{thebibliography}{7}
	\providecommand{\natexlab}[1]{#1}
	\providecommand{\url}[1]{#1}
	\providecommand{\urlprefix}{}
	\expandafter\ifx\csname urlstyle\endcsname\relax
	\providecommand{\doi}[1]{doi:\discretionary{}{}{}#1}\else
	\providecommand{\doi}{doi:\discretionary{}{}{}\begingroup
		\urlstyle{rm}\Url}\fi
	
	\bibitem[{Li(2015)}]{Li2015}
	Li Zihe [\zh{李子鹤}]. 2015.
	\newblock
	\zh{纳西语言研究回顾------兼论语言在文化研究中的基础地位}
	[{A} review of {Naxi} language studies, with a discussion of the fundamental
	role of cultural studies for linguistic research].
	\newblock \zh{茶马古道研究期刊} 4. 125--131.
	
	\bibitem[{Comrie et~al.()Comrie, Haspelmath \& Bickel}]{Comrie}
	Comrie, Bernard, Martin Haspelmath \& Balthasar Bickel. 2008.
	\newblock Leipzig {Glossing Rules}.
	\newblock
	\urlprefix\url{http://www.eva.mpg.de/lingua/resources/glossing-rules.php}.
	
	\bibitem[{Dixon(2007)}]{Dixon2007}
	Dixon, Robert~M. 2007.
	\newblock Field linguistics: a minor manual.
	\newblock \emph{Sprachtypologie und Universalienforschung} 60(1). 12--31.
	
	\bibitem[{Lidz(2010)}]{Lidz2010}
	Lidz, Liberty. 2010.
	\newblock \emph{A descriptive grammar of {Yongning Na} ({Mosuo})}.
	\newblock Austin: University of Texas, Department of linguistics dissertation.
	\newblock
	\urlprefix\url{https://repositories.lib.utexas.edu/bitstream/handle/2152/ETD-UT-2010-12-2643/LIDZ-DISSERTATION.pdf}.
	\newblock Ph. D.
	
	\bibitem[{Michailovsky et~al.(2014)Michailovsky, Mazaudon, Michaud, Guillaume,
		Fran{\c{c}}ois \& Adamou}]{Michailovsky2014}
	Michailovsky, Boyd, Martine Mazaudon, Alexis Michaud, S{\'{e}}verine Guillaume,
	Alexandre Fran{\c{c}}ois \& Evangelia Adamou. 2014.
	\newblock Documenting and researching endangered languages: the {Pangloss
		Collection}.
	\newblock \emph{Language Documentation and Conservation} 8. 119--135.
	\newblock \urlprefix\url{http://hdl.handle.net/10125/4621}.
	
	\bibitem[{Michaud(2013)}]{Michaud2013}
	Michaud, Alexis. 2013.
	\newblock The tone patterns of numeral-plus-classifier phrases in {Yongning
		Na}: a synchronic description and analysis.
	\newblock In Nathan Hill \& Tom Owen-Smith (eds.), \emph{Transhimalayan
		{Linguistics}. {Historical} and {Descriptive} {Linguistics} of the
		{Himalayan} {Area}} (Trends in {Linguistics}. {Studies} and {Monographs}
	[{TiLSM}] 266), 275--311. Berlin: De Gruyter Mouton.
	
	\bibitem[{Michaud et~al.(2015)Michaud, Limin \& Yaoping}]{Michaud2015b}
	Michaud, Alexis, He~Limin \& Zhong Yaoping. 2015.
	\newblock Naxi / {Naish}.
	\newblock In Rint Sybesma, Wolfgang Behr, Zev Handel \& C.T.~James Huang
	(eds.), \emph{Encyclopedia of {Chinese} {Language} and {Linguistics}},
	Leiden: Brill.
	
\end{thebibliography}

\cleardoublepage


	{\LARGE \textbf{\zh{前言}}}
	\section{ \zh{缘起}} \label{sec:language}

 \zh{ 在北纬27.50度,东经100.41度的交界点上生活着一个知名的族群——摩梭人。它奇特的民族风俗使它名扬天下,但是,它的语言研究还远远跟不上它的名气。作为一名语音学家找到了摩梭语言就如同进入了阿拉丁的宝库:摩梭语音的丰富多彩,尤其是声调在整个语法系统里所起的核心作用,独特而又迷人。
 	本词典的全部信息来自于一位充满独特个人魅力的摩梭阿妈——拉它米打史拉么。1950年出生的这位阿妈,差不多与新中国同年龄,她身上的故事就是摩梭人今天与昨天的缩影。在云南省丽江市宁蒗县永宁乡平静村独自养育了四个儿女,其中一位就是摩梭著名学者拉他咪达石(拉他咪王勇)。有幸结识拉他咪达石先生与他亲爱的母亲是这本词典得以与世人见面的关键,在此向他们表示最深切的感谢。}

	\section{ \zh{词典}} \label{sec:method}

 \zh{2006年在纳西语语音研究告一段落之后,我开始了对摩梭语音的分析。一开始的初衷只是进行音系分析,但却惊奇地发现摩梭声调不仅是单纯的音系问题,同时与语法有着千丝万缕的联系。这一发现,促使我开始对摩梭话进行全方位的研究。研究方法以搜集来的长篇语料为蓝本,在搜集语料的同时进行音系实验(系统的机械问答),并以长篇语料中的词汇为基准,汇集成为本部词典。词典目前的条目数量不到三千,但我的初衷并不致力于单纯的词汇收集而是希望能在有限的条目中最大限度地呈现摩梭语音的独特性。

本词典以国际音标为注音基础。声调标注系统由于比较复杂,因此启用了一套专门符号,该符号的用法及说明见}Michaud 2015\zh{。

本词典有三种呈现格式,在线词典、PDF文本、与Toolbox资料库。内容将陆续进行更正及删补。同时欢迎读者的指正与批评。来信请寄}alexis.michaud@vjf.cnrs.fr \zh{。}

\section{ \zh{其它}} \label{sec:other}

 \zh{著名语言学家孙天心提出,语言学家有三件宝:长篇语料、词典与参考语法。以这个标准来看,长篇语料就是我囊中最大的宝贝,历时十年搜集而来的近二十个小时的资料及逐句标注翻译,已经全部在线公开。在线地址:“泛语资料库”(}Pangloss Collection \zh{),}lacito.vjf.cnrs.fr/pangloss/languages/Na\_en.htm \zh{。下一步的相关工作,是完成一本《摩梭话声调研究》,初稿也已上线,地址:}https://halshs.archives-ouvertes.fr/halshs-01094049/document \zh{。}

	\section{ \zh{致谢}} \label{sec:thks}
	
\zh{拉它米打史拉么阿妈及全家人、拉他咪达石(摩梭知名学者,出版有多部摩梭文化研究专著)、李德静(丽江市东巴文化研究院院长)、黄行(中央社会科学院民族学研究所所长)、段炳昌(云南大学人文学院院长)、木霁弘(云南大学茶马古道文化研究所所长)、王卫东(云南大学人文学院中文系系主任)、和学光(云南党校图书馆馆长)、赵燕珍(云南大学人文学院中文系教授)、杨立权(云南大学人文学院中文系教授)、丁思之教授(普米语研究专家)、拉木嘎吐萨(摩梭知名学者)、阿慧(摩梭人研究生)、}Céline Buret\zh{(工程师)、}Séverine Guillaume\zh{(工程师)、向柏霖(}Guillaume Jacques\zh{,语言学专家)、
李力(}Liberty Lidz\zh{(,摩梭话研究专家)、杜玫瑰(}Roselle Dobbs\zh{,摩梭研究者)、}Pascale-Marie Milan\zh{(人类学家,摩梭文化研究专家)、}Christine Mathieu\zh{(人类学家,摩梭文化研究专家):感谢你们的大力支持。

在此也向成书过程中曾给予帮助的许多朋友和专家们一并致谢!}




	\section{ \zh{参考书目}} \label{sec:refs}
	\begin{itemize}
		\item \zh{李子鹤. 纳西语言研究回顾——兼论语言在文化研究中的基础地位[J]. 茶马古道研究期刊, 2015, 4: 125–131.}
		\item \zh{孙天心. 藏缅语的调查[J]. 语言学论丛, 2007, 36: 98–107.}
		\item LIDZ L. A descriptive grammar of Yongning Na (Mosuo) [D]. Austin: University of Texas, Department of linguistics, 2010.
		\zh{(下载地址:}https://repositories.lib.utexas.edu/bitstream/handle/2152/ETD-UT-2010-12-2643/LIDZ-DISSERTATION.pdf\zh{)}
		\item MICHAUD A. Phrasing, prominence, and morphotonology: How utterances are divided into tone groups in Yongning Na [J]. Bulletin of Chinese Linguistics, 2015. \zh{(下载出版前版本的地址:}https://halshs.archives-ouvertes.fr/halshs-01162331\zh{)}
	\end{itemize}

\cleardoublepage
\pagenumbering{arabic}
\setcounter{page}{1}
 

	

