\pagenumbering{roman}

\cn
\section*{\zh{前言}}

这部《嘉绒-汉-法词典》是茶堡嘉绒语的首部词典,收录\total{entrycounter}个词条。茶堡话分布在马尔康县龙尔甲、沙尔宗、大藏三个乡。虽然按照当地人“茶堡”(\ipa{tɕɤpʰɯ})这一地名不包括龙尔甲乡,但是由于龙尔甲乡的方言与沙尔宗乡和大藏乡区别不大,交流对话没有任何障碍,所以仍然用“茶堡话”作为三个乡的共同语言的统称。

本部词典的编辑工作是笔者的语言描写计划的其中一个项目,此外笔者还对茶堡话的语法进行了深入的研究(参考向柏霖 \citeyear{jacques08}以及\citealt{jacques12incorp, jacques13tropative, jacques14antipassive,  jacques15spontaneous}, \citeyear{jacques15causative, jacques16relatives}等)并收集了近80个小时的长篇故事,这些故事的国际音标转写以及语音文件已经发表在PANGLOSS语料库的网站上(\citealt{michailovsky14pangloss})。笔者在语法研究中借鉴了前辈学者林向荣(\citeyear{linxr93jiarong})、孙天心(\citeyear{jackson00sidaba, jackson04zhuangmaoci, jackson06paisheng, jackson14morpho}等)、林幼菁和罗尔武(\citeyear{linluo03})等的研究成果,并从他们的著作和论文中得到了很多启发。


本词典以龙尔甲乡干木鸟村(\ipa{kɤmɲɯ})的口音为标准,是笔者与陈珍和柏尔青两位老师自2002年以来至今长期合作的共同研究成果。

词条的一大部分(特别是动词和状貌词的词条)都包含有代表性的例句,很多这些例句是从对话和传统故事中选出来的,并附有语音文件。

每一项词条包含法语和汉语的定义,并注明词类,茶堡话的词类包括下列14种:

\begin{itemize}
\item \ital{adv} 副词
\item \ital{clf} 量词
\item \ital{idph} 状貌词
\item \ital{intj} 感叹词
\item \ital{n} 名词
\item \ital{np} 被领属名词
\item \ital{postp} 介词
\item \ital{pro} 人称代词
\item \ital{vi} 不及物动词
\item \ital{vinh} 不及物动词(无人主语)
\item \ital{vi-t} 半及物动词
\item \ital{vs} 静态动词
\item \ital{vt} 及物动词
\item \ital{pc(x,y)} 复合谓语
\end{itemize}

复合谓语是由两个单词组成的,其中第一个单词 \ital{x} 可以是名词或者动词,第二个单词 \ital{y}必须是动词。缩写符号\ital{pc}后面的两个符号分别表示第一和第二成分的词类,例如 \ipa{tɯ-ʑi,loʁ} \ital{pc(np,vs)} “感到恶心”表明,第一个成分 \ipa{tɯ-ʑi} 是被领属名词,而第二成分\ipa{loʁ} 是静态动词。

附在 \ital{idph} 后的数字表示状貌词形态模式(根据 \citealt{japhug14ideophones}的分类)。

在动词的词条中 \ital{dir} 标注动词所搭配的趋向前缀(参考 \citealt[267-9]{jacques14linking}), 下划线字符 \_ 表示该动词可以和任意趋向连用(用于移位或具体动作动词)。不规则动词(如 \ipa{ɕe} “去”或  \ipa{ɣɤʑu}  “有”)的词条中附有所有无法预测的形式(特别是词干交替,第二人称/泛称等形式)。

 以下缩写词表示茶堡话动词的派生形态:

\begin{itemize}
\item \caps{acaus} 反使动 
\item \caps{apass} 反被动态
\item \caps{appl} 施用态
\item \caps{autoben} 为己态
\item \caps{caus} 使动态
\item \caps{comp} 复合动词
\item \caps{deexp} 感受者泛化
\item \caps{deidph} 动词化状貌词
\item \caps{denom} 动词化名词
\item \caps{facil} 便利态
\item \caps{incorp} 名词并入
\item \caps{n.orient} 无定方向
\item \caps{pass} 被动态
\item \caps{recip} 互动态
\item \caps{refl} 反身态
\item \caps{trop} 意动态
\item \caps{vert} 回归
\end{itemize}

本词典得到了许多同行业者的修改意见和建议,笔者特别感谢龚勋、彭国珍、沙加尔、帅彦辰和章舒娅对初稿的仔细审阅。
\newpage
\normalfont
\section*{Introduction}

Ce  dictionnaire décrit le lexique de la langue japhug (\ipa{kɯrɯ skɤt}), parlée dans la région de Japhug (\zh{茶堡}, \ipa{tɕɤpʰɯ}) au district de Mbarkhams (\zh{马尔康县}), préfecture de Rngaba (\zh{阿坝州}) au Sichuan en Chine, dans les cantons de Gdongbrgyad (\zh{龙尔甲乡}, \ipa{ʁdɯrɟɤt}), Gsarrdzong (\zh{沙尔宗乡},   \ipa{sarndzu}) de Datshang (\zh{大藏乡}, \ipa{tatshi}).  

Seul le dialecte de Kamnyu (\zh{干木鸟村}, \ipa{kɤmɲɯ}) est représenté dans ce dictionnaire. Cette langue a déjà fait l'objet d'une courte description grammaticale (\citealt{jacques08}) ainsi que d'un recueil d'histoires traditionnelles (\citealt{jacques10gesar}). Un corpus de texte plus important est en cours de publication sur l'archive Pangloss (\citealt{michailovsky14pangloss}). Ces recherches ont bénéficié des travaux précédents sur d'autres langues gyalrong, en particulier la grammaire de \citet{linxr93jiarong} et les articles de Jackson Sun et de Shidanluo (\citealt{jackson00sidaba, jackson04zhuangmaoci, jackson06paisheng, jackson14morpho} etc), ainsi que du travail de Lin Youjing et Norbu sur le dialecte de Datshang (\citealt{linluo03}).

Ce dictionnaire est basé sur les matériaux recueillis à Mbarkhams par l'auteur auprès de Tshendzin (Chenzhen \zh{陈珍}) et Dpalcan (Baierqing \zh{柏尔青}) depuis juillet 2002. Une grande partie des mots, en particulier les verbes et les idéophones, sont illustrés par des exemples enregistrés représentatifs, dont certains proviennent de conversations ou d'histoires traditionnelles.

Chaque entrée du dictionnaire contient une définition en français et en chinois ainsi que la partie du discours du mot, parmi les suivantes:

\begin{itemize}
\item \textit{adv} adverbe
\item \textit{clf} classificateur
\item \textit{idph} idéophone
\item \textit{intj} interjection
\item \textit{n} nom
\item \textit{np} nom inaliénablement possédé 
\item \textit{postp} postposition
\item \textit{pro} pronom
\item \textit{vi} verbe intransitif
\item \textit{vinh} verbe intransitif sans sujet humain
\item \textit{vi-t} verbe semi-transitif
\item \textit{vs} verbe statif
\item \textit{vt} verbe transitif
\item \textit{pc(x,y)} prédicat complexe 
\end{itemize}

Les parties du discours des premiers et deuxièmes éléments des prédicats complexes sont respectivement \ipa{x} et \ipa{y}. Par exemple \ipa{loʁ,tɯ-ʑi} \textit{pc(vs,np)} `avoir la nausée' signifie que l'élément \ipa{loʁ} est morphologiquement un verbe statif, et \ipa{tɯ-ʑi} un nom possédé.

Le numéro qui suit \textit{idph} correspond au patron idéophonique (selon la classification décrite dans \citealt{japhug14ideophones}).

Les verbes contiennent après \textit{dir} le ou les préfixes directionnels utilisés pour former les tiroirs verbaux (décrits dans \citealt[267-9]{jacques14linking}). Le symbole \_  est utilisé pour les verbes de mouvement, de manipulation ou d'action concrète compatibles avec les sept séries de préfixes. Pour les verbes irréguliers (tels que \ipa{ɕe} `aller' ou \ipa{ɣɤʑu}  `exister'), les formes non-prévisibles (thème du passé, seconde personne ou générique) sont toutes indiquées. 

Les dérivations verbales sont indiquées par les abréviations suivantes (voir \citealt{jacques12incorp, jacques13tropative, jacques14antipassive,  jacques15spontaneous}, \citeyear{jacques15causative, jacques16relatives}):

\begin{itemize}
\item \textsc{acaus} anticausatif 
\item \textsc{apass} antipassif
\item \textsc{appl} applicatif
\item \textsc{autoben} autobénéfactif-spontané
\item \textsc{caus} causatif 
\item \textsc{comp} composé
\item \textsc{deexp} dé-expérienceur
\item \textsc{deidph} déidéophonique
\item \textsc{denom} dénominal
\item \textsc{facil} facilitatif
\item \textsc{incorp} incorporation
\item \textsc{n.orient} action non-orientée
\item \textsc{pass} passif
\item \textsc{recip} réciproque
\item \textsc{refl} réfléchi
\item \textsc{trop} tropatif
\item \textsc{vert} vertitif
\end{itemize}

Ce dictionnaire a bénéficié des corrections de nombreux collègues et étudiants, en particulier Gong Xun, Peng Guozhen, Laurent Sagart, Shuai Yanchen et Zhang Shuya. Je remercie également Rémy Bonnet, Céline Buret, Séverine Guillaume et Thomas Pellard pour leur aide avec les scripts de conversions du format MDF vers \LaTeX.

Ce travail a été financé par le projet ANR HimalCo  (ANR-12-CORP-0006) et est en relation avec le projet de recherche LR-4.11 ‘‘Automatic Paradigm Generation and Language Description’’ du Labex EFL (fondé par l'ANR/CGI).


 
 

%\bibliographystyle{unified}
%\bibliography{bibliogj}
\newpage
\begin{thebibliography}{17}
\providecommand{\natexlab}[1]{#1}
\providecommand{\url}[1]{#1}
\providecommand{\urlprefix}{}
\expandafter\ifx\csname urlstyle\endcsname\relax
  \providecommand{\doi}[1]{doi:\discretionary{}{}{}#1}\else
  \providecommand{\doi}{doi:\discretionary{}{}{}\begingroup
  \urlstyle{rm}\Url}\fi

\bibitem[{Jacques(2008)}]{jacques08}
Jacques, Guillaume. 2008.
\newblock \emph{\zh{嘉絨語研究} {J}iāróngy\v{u} yánjiū ({S}tudy on the
  {R}gyalrong language)}.
\newblock Beijing: Minzu chubanshe.

\bibitem[{Jacques(2012)}]{jacques12incorp}
Jacques, Guillaume. 2012.
\newblock {F}rom denominal derivation to incorporation.
\newblock \emph{Lingua} 122.11. 1207--1231.

\bibitem[{Jacques(2013{\natexlab{a}})}]{jacques13tropative}
Jacques, Guillaume. 2013{\natexlab{a}}.
\newblock {A}pplicative and tropative derivations in {J}aphug {R}gyalrong.
\newblock \emph{Linguistics of the Tibeto-Burman Area} 36.2. 1--13.

\bibitem[{Jacques(2013{\natexlab{b}})}]{japhug14ideophones}
Jacques, Guillaume. 2013{\natexlab{b}}.
\newblock {I}deophones in {J}aphug {R}gyalrong.
\newblock \emph{Anthropological Linguistics} 55.3. 256--287.

\bibitem[{Jacques(2014{\natexlab{a}})}]{jacques14linking}
Jacques, Guillaume. 2014{\natexlab{a}}.
\newblock {C}lause linking in {J}aphug {R}gyalrong.
\newblock \emph{Linguistics of the Tibeto-Burman Area} 37.2. 263--327.

\bibitem[{Jacques(2014{\natexlab{b}})}]{jacques14antipassive}
Jacques, Guillaume. 2014{\natexlab{b}}.
\newblock {D}enominal affixes as sources of antipassive markers in {J}aphug
  {R}gyalrong.
\newblock \emph{Lingua} 138. 1--22.

\bibitem[{Jacques(to appear{(\natexlab{a})})}]{jacques16relatives}
Jacques, Guillaume. to appear{(\natexlab{a})}.
\newblock {S}ubjects, objects and relativization in {J}aphug.
\newblock \emph{Journal of Chinese Linguistics} .

\bibitem[{Jacques(to appear{(\natexlab{b})})}]{jacques15causative}
Jacques, Guillaume. to appear{(\natexlab{b})}.
\newblock {T}he origin of the causative prefix in {R}gyalrong languages and its
  implication for proto-{S}ino-{T}ibetan reconstruction.
\newblock \emph{Folia Linguistica Historica} .

\bibitem[{Jacques(to appear{(\natexlab{c})})}]{jacques15spontaneous}
Jacques, Guillaume. to appear{(\natexlab{c})}.
\newblock {T}he spontaneous-autobenefactive prefix in {J}aphug {R}gyalrong.
\newblock \emph{Linguistics of the Tibeto Burman Area} .

\bibitem[{Jacques \& Chen(2010)}]{jacques10gesar}
Jacques, Guillaume \& Zhen Chen. 2010.
\newblock \emph{{U}ne version rgyalrong de l'épopée de {G}esar}.
\newblock Osaka: National Museum of Ethnology.

\bibitem[{Lín(1993)}]{linxr93jiarong}
Lín, Xiàngróng. 1993.
\newblock \emph{\zh{嘉戎語研究} {J}iāróngy\v{u} yánjiū ({A} study on
  the {R}gyalrong language)}.
\newblock \zh{成都:四川民族出版社} Chéngdū: Sìchuān mínzú
  chūb\v{a}nshè.
\newblock (\zh{林向榮}).

\bibitem[{Lín \& Luó\v{e}rw\v{u}(2003)}]{linluo03}
Lín, Yòujīng \& Luó\v{e}rw\v{u}. 2003.
\newblock \zh{茶堡嘉戎語大藏話的趨向前綴及動詞詞幹變化}
  {C}hábǎo jiāróngyǔ {D}àzànghuà de qūxiàng qiánzhuì
  jí dòngcí cígàn biànhuà ({T}he directional prefixes and verb
  stem alternations in the {D}atshang dialect of {J}aphug).
\newblock \emph{\zh{民族語文} Mínzú y\v{u}wén} 4. 19--29.

\bibitem[{Michailovsky et~al.(2014)Michailovsky, Mazaudon, Michaud, Guillaume,
  François \& Adamou}]{michailovsky14pangloss}
Michailovsky, Boyd, Martine Mazaudon, Alexis Michaud, Séverine Guillaume,
  Alexandre François \& Evangelia Adamou. 2014.
\newblock {D}ocumenting and researching endangered languages: the {P}angloss
  {C}ollection.
\newblock \emph{Language Documentation and Conservation} 8. 119–135.

\bibitem[{Sun(2000)}]{jackson00sidaba}
Sun, Jackson T.-S. 2000.
\newblock {P}arallelisms in the {V}erb {M}orphology of {S}idaba r{G}yalrong and
  {L}avrung in r{G}yalrongic.
\newblock \emph{Language and Linguistics} 1.1. 161--190.

\bibitem[{Sun(2006)}]{jackson06paisheng}
Sun, Jackson T.-S. 2006.
\newblock \zh{嘉戎語動詞的派生形態} {J}iāróngyǔ dòngcí de
  pàishēng xíngtài ({D}erivational morphology in the {R}gyalrong verb).
\newblock \emph{Minzu yuwen \zh{民族語文}} 4.3. 3--14.

\bibitem[{Sun(2014)}]{jackson14morpho}
Sun, Jackson T.-S. 2014.
\newblock {S}ino-{T}ibetan: {R}gyalrong.
\newblock In Rochelle Lieber \& Pavol Štekauer (eds.), \emph{{T}he {O}xford
  {H}andbook of {D}erivational {M}orphology}, 630--650. Oxford: Oxford
  University Press.

\bibitem[{Sun \& Shidanluo(2004)}]{jackson04zhuangmaoci}
Sun, Jackson T.-S. \& Shidanluo. 2004.
\newblock \zh{草登嘉戎語的狀貌詞} {C}ǎodēng {J}iāróngyǔ de
  zhuàngmàocí ({T}he ideophones in {T}shobdun {R}gyalrong).
\newblock \emph{Mínzú y\v{u}wén \zh{民族語文}} 5. 1--11.

\end{thebibliography}

\cleardoublepage
\pagenumbering{arabic}
\setmainfont[Mapping=tex-text,Numbers=OldStyle,Ligatures=Common]{Charis SIL} 