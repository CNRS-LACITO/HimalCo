\documentclass[twoside,11pt]{book}
\title{Dictionnaire Japhug-chinois-français\\\zh{嘉绒-汉-法词典\\ Version 1.11}}
\author{Guillaume Jacques\\\zh{向柏霖}}

\usepackage{pifont}
\usepackage{fontspec}
\usepackage{natbib}
\usepackage{booktabs}
\usepackage{xltxtra} 
\usepackage{polyglossia}
\setdefaultlanguage{french}
\usepackage[table]{xcolor}
\usepackage{multicol}
\setlength{\columnseprule}{1pt}
\setlength{\columnsep}{1.5cm}
\usepackage{float}
\usepackage{textcomp}
\usepackage[bookmarks=true,colorlinks,linkcolor=blue]{hyperref}
\hypersetup{bookmarks=false,bookmarksnumbered,bookmarksopenlevel=5,bookmarksdepth=5,xetex,colorlinks=true,linkcolor=blue,citecolor=blue}
\usepackage[all]{hypcap}
\usepackage{memhfixc}
\usepackage{amssymb}
\usepackage{lscape}
%\setmainfont[Mapping=tex-text,Numbers=OldStyle,Ligatures=Common]{Charis SIL} 
\newfontfamily\phon[Mapping=tex-text,Ligatures=Common,Scale=MatchLowercase]{Charis SIL} 
\newcommand{\ipa}[1]{{\phon #1}}  
\newfontfamily\cn[Mapping=tex-text,Ligatures=Common,Scale=MatchUppercase]{SimSun} % (ou SimSun) pour le chinois
\newcommand{\zh}[1]{{\cn #1}}
\newfontfamily\mx[Mapping=tex-text,Ligatures=Common,Scale=MatchUppercase]{ArialUnicodeMS} % pour les questions
\newcommand{\nq}[1]{{\mx #1}}
\newcommand{\grise}[1]{\cellcolor{lightgray}\textbf{#1}}
\newcommand{\bleute}[1]{\cellcolor{green}\textbf{#1}}
\newcommand{\rouge}[1]{\cellcolor{red}\textbf{#1}}
\newcommand{\topic}{\textsc{dem}}
\newcommand{\tete}{\textsuperscript{\textsc{head}}}
\newcommand{\rc}{\textsubscript{\textsc{rc}}}
\newcommand{\ital}[1]{{\normalfont\textit{#1}}}
\newcommand{\caps}[1]{{\normalfont\textsc{#1}}}
\XeTeXlinebreaklocale 'zh' % 使用中文换行
\XeTeXlinebreakskip = 0pt plus 1pt % CIRCG
\usepackage{fancyhdr}
\pagestyle{fancy}
\fancyheadoffset{3.4em}
\fancyhead[LE,LO]{\rightmark}
\fancyhead[RE,RO]{\leftmark}
\usepackage[dvipdfmx,xetex,bigfiles,final,activate=onclick,deactivate=onclick,transparent,passcontext]{media9}
\usepackage{graphicx}
\usepackage{gb4e}
\usepackage{vmargin}
% {marge gauche}{marge en haut}{marge droite}{marge en bas}{hauteur de l'entête}{distance entre l'entête et le texte}{hauteur du pied de page}{distance entre le texte et le pied de page}
\setmarginsrb{3cm}{2cm}{2cm}{2cm}{1cm}{2cm}{1cm}{2cm}

% \usepackage[x11names]{xcolor}
% \colorlet{headcolor}{blue}% couleur des têtes d'entrée
\usepackage{hanging}
\newcommand{\lx}[1]{{\Large\bfseries#1}}
%%%%% Environnememt pour chaque entrée %%%%%
\usepackage{totcount}
\newcounter{entrycounter}\setcounter{entrycounter}{0}\regtotcounter{entrycounter}%Compteur du nombre d'entrées
\usepackage{hanging}
\newenvironment{entry}[1]{% l'environnement à 1 argument, la tête d'entrée
\stepcounter{entrycounter}% on incrémente le compteur
\hangpara{1em}{1}% après la ligne 1, chaque ligne est indentée d'1em, ce que fait que la première ligne apparaît avec un retrait
\markboth{#1}{#1}% on marque les entrées pour pouvoir les insérer dans les en-têtes de page
\addcontentsline{toc}{subsection}{#1}% chaque entrée est ajoutée aux signets pdf comme étant une sous-section
\lx{#1}\hspace{.5em}% on imprime la tête d'entrée avec la mise en forme lx et on ajoute de l'espace à sa gauche
}{\par\smallskip}% on change de paragraphe et on laisse un peu d'espace avant l'entrée suivante

\newenvironment{subentry}[1]{% l'environnement à 1 argument, la tête d'entrée
%\stepcounter{entrycounter}% on incrémente le compteur
\hangpara{1em}{1}% après la ligne 1, chaque ligne est indentée d'1em, ce que fait que la première ligne apparaît avec un retrait
%\markboth{#1}{#1}% on marque les entrées pour pouvoir les insérer dans les en-têtes de page
\addcontentsline{toc}{subsubsection}{#1}% chaque entrée est ajoutée aux signets pdf comme étant une sous-section
\hspace{.5em}$\blacksquare$#1\hspace{.5em}% on imprime la tête d'entrée avec la mise en forme lx et on ajoute de l'espace à sa gauche
}{\par\smallskip}% on change de paragraphe et on laisse un peu d'espace avant l'entrée suivante
\newcommand{\refentry}{\zh{【参考】}}
\newcommand{\etymology}{\zh{【借词】}} 
\newcommand{\antonym}{\zh{【反义词】}} 
\newcommand{\synonym}{\zh{【同义词】}} 
\newcommand{\usage}{\zh{【用法】}} 
\newenvironment{bottompar}{\par\vspace*{\fill}}{\clearpage}
 