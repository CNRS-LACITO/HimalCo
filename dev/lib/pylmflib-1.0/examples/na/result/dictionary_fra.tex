\documentclass[twoside]{article}
\author{Alexis Michaud}
% ajouté en novembre 2016, en remplacement de vmargin
\usepackage[paperwidth=185mm,paperheight=260mm,margin=30mm]{geometry}
\usepackage{pifont}
\usepackage{fontspec}
\usepackage{natbib}
\usepackage{booktabs}
\usepackage{xltxtra}
\usepackage{polyglossia}
\usepackage[table]{xcolor}
\usepackage{longtable}
\definecolor{darkblue}{rgb}{0,0,0.75}
\usepackage{float}
\usepackage{lineno}
\usepackage{textcomp}
\usepackage{memhfixc}
\usepackage{lscape}
\usepackage{amssymb}
\usepackage{multicol}
\setlength{\columnseprule}{1pt}
\setlength{\columnsep}{1.5cm}
%\setmainfont[Mapping=tex-text,Numbers=OldStyle,Ligatures=Common]{Charis SIL} 
\setmainfont[Mapping=tex-text,Numbers=OldStyle,Ligatures=Common,ItalicFont=*,ItalicFeatures=FakeSlant]{DoulosSIL}
\newfontfamily\phon[Mapping=tex-text,Ligatures=Common,Scale=MatchLowercase]{Charis SIL} % ou DoulosSIL
\newcommand{\ipa}[1]{{\phon #1}} % API jamais en italique
\newcommand{\grise}[1]{\cellcolor{lightgray}\textbf{#1}}
\newcommand{\bleute}[1]{\cellcolor{green}\textbf{#1}}
\newcommand{\rouge}[1]{\cellcolor{red}\textbf{#1}}
\newfontfamily\cn[Mapping=tex-text,Ligatures=Common,Scale=MatchUppercase]{SimSun} % pour le chinois
\newcommand{\zh}[1]{{\cn #1}}
\newcommand{\topic}{\textsc{dem}}
\newcommand{\tete}{\textsuperscript{\textsc{head}}}
\newcommand{\rc}{\textsubscript{\textsc{rc}}}
\XeTeXlinebreaklocale "zh" % 使用中文换行
\XeTeXlinebreakskip = 0pt plus 1pt  % CIRCG
\usepackage{fancyhdr}
\pagestyle{fancy}
\fancyheadoffset{3.4em}
\usepackage[dvipdfmx,xetex,bigfiles,final,activate=onclick,deactivate=onclick,transparent,passcontext]{media9}
\usepackage{graphicx}
\usepackage[bookmarks=true,colorlinks,linkcolor=blue]{hyperref}
\hypersetup{bookmarks=false,bookmarksnumbered,bookmarksopenlevel=5,bookmarksdepth=5,xetex,colorlinks=true,linkcolor=blue,citecolor=blue}
\usepackage[all]{hypcap}
\usepackage{gb4e}
%%% retirés en novembre 2016 : pas utiles. Remplacé par: geometry.
%%\usepackage{vmargin}
%%% {marge gauche}{marge en haut}{marge droite}{marge en bas}{hauteur de l'entête}{distance entre l'entête et le texte}{hauteur du pied de page}{distance entre le texte et le pied de page}
%%\setmarginsrb{2cm}{1cm}{1.5cm}{1cm}{0.5cm}{1cm}{0.5cm}{1cm}

\def\mytextsc{\bgroup\obeyspaces\mytextscaux}
\def\mytextscaux#1{\mytextscauxii #1\relax\relax\egroup}
\def\mytextscauxii#1{%
\ifx\relax#1\else \ifcat#1\@sptoken{} \expandafter\expandafter\expandafter\mytextscauxii\else
\ifnum`#1=\uccode`#1 {\normalsize #1}\else {\footnotesize \uppercase{#1}}\fi \expandafter\expandafter\expandafter\mytextscauxii\expandafter\fi\fi}
% Ne pas afficher la numérotation des sections, sous-sections etc
\setcounter{secnumdepth}{0}\title{Dictionnaire na-chinois-français en ligne (version 1.0)}
\setdefaultlanguage{french}

\begin{document}
\maketitle
\newpage
\markboth{INTRODUCTION}{}

	{\LARGE \textbf{Note introductive}}
	
La base de données lexicographique réalisée pour le na de Yongning est quadrilingue: les mots et exemples sont traduits en chinois, français et anglais. Pour le dictionnaire en ligne (HTML) et au format PDF, deux versions sont proposées: le présent document, na-chinois-français; et une version na-chinois-anglais. Ce choix paraissait préférable à la réalisation d'une unique version quadrilingue, car la présence du français et de l'anglais côte à côte, même distingués par la typographie, paraissait gêner la lecture.

La version française est pour l'heure essentiellement utilisée par son auteur; en conséquence, il ne paraisssait pas urgent de la doter d'une introduction en français, et on s'est contenté d'une introduction en anglais (pour la version na-chinois-anglais), à laquelle est renvoyé le lecteur souhaitant obtenir des précisions au sujet de ce dictionnaire. 

La présente version, première version publique, est numérotée 1.0. Il est prévu des mises à jour tous les ans ou tous les deux ans environ, à mesure de l'enrichissement du dictionnaire. Les lecteurs sont vivement encouragés à prendre contact (michaud.cnrs@gmail.com) pour signaler des erreurs et proposer des améliorations.



\def\mytextsc{\bgroup\obeyspaces\mytextscaux}
\def\mytextscaux#1{\mytextscauxii #1\relax\relax\egroup}
\def\mytextscauxii#1{%
\ifx\relax#1\else \ifcat#1\@sptoken{} \expandafter\expandafter\expandafter\mytextscauxii\else
\ifnum`#1=\uccode`#1 {\normalsize #1}\else {\footnotesize \uppercase{#1}}\fi \expandafter\expandafter\expandafter\mytextscauxii\expandafter\fi\fi}

\setlength\parindent{0cm}

\addmediapath{C:/Dropbox/GitHub/HimalCo/dev/lib/pylmflib-1.0/examples/na/mp3}
\addmediapath{C:/Dropbox/GitHub/HimalCo/dev/lib/pylmflib-1.0/examples/na/mp3/mp3}
\addmediapath{C:/Dropbox/GitHub/HimalCo/dev/lib/pylmflib-1.0/examples/na/mp3/wav}
\graphicspath{{C:/Dropbox/GitHub/HimalCo/dev/lib/pylmflib-1.0/pylmflib/output/img/}}

\newpage
\begin{multicols}{2}

\newpage
\section*{\centering- \textcolor{darkblue}{\textbf{\ipa{ɑ}}} -}
\paragraph{\hspace{-0.5cm} {\Large \textcolor{darkblue}{\textbf{\ipa{ɑ˩mi\#˥}}}}} \hypertarget{A\string_Bmi\#\string_T1}{}
\markboth{\textcolor{darkblue}{\textbf{\ipa{ɑ˩mi\#˥}}}}{}
\textcolor{teal}{\mytextsc{nom}} \hspace{4pt} Ton~: LM+\#H.

Oie (femelle).
\textcolor{brown}{\zh{母鹅。}}


\begin{exe}
\sn \textcolor{darkblue}{\textbf{\ipa{ɑ˩mi˧-ɑ˥pʰv̩˩}}}
\trans oie et jars
\trans \textit{\textcolor{brown}{\zh{公鹅与母鹅}}}
\end{exe}
 \mytextsc{clf}~: \textcolor{darkblue}{\textbf{\ipa{mi˩}}} 

\paragraph{\hspace{-0.5cm} {\Large \textcolor{darkblue}{\textbf{\ipa{ɑ˩pʰo˩}}}}} \hypertarget{A\string_Bp\string_ho\string_B1}{}
\markboth{\textcolor{darkblue}{\textbf{\ipa{ɑ˩pʰo˩}}}}{}
\textcolor{teal}{\mytextsc{adverbe}} \hspace{4pt} Ton~: L.

Dehors, à l'extérieur.
\textcolor{brown}{\zh{外面。}}


\begin{exe}
\sn \textcolor{darkblue}{\textbf{\ipa{ɑ˩pʰo˩ bi˩˥}}}
\trans aller dehors; aller faire ses besoins
\trans \textit{\textcolor{brown}{\zh{出去,解手}}}
\end{exe}
\begin{exe}
\sn \textcolor{darkblue}{\textbf{\ipa{ɑ˩pʰo˩ bi˩-ze˥!}}}
\trans On sort! / [Je] vais faire mes besoins!
\trans \textit{\textcolor{brown}{\zh{出去了! / 出去解手!}}}
\end{exe}
\begin{exe}
\sn \textcolor{darkblue}{\textbf{\ipa{ɑ˩pʰo˩ bi˩-ʂv̩˥ɖv̩˩!}}}
\trans [Elle] a envie de sortir! (Contexte: par une journée ensoleillée, un membre de la famille sent qu'un enfant aurait envie d'aller jouer dehors.)
\trans \textit{\textcolor{brown}{\zh{(她)想出去!(情景:婴儿看外边,觉得她好像想出去。)}}}
\end{exe}
\begin{exe}
\sn \textcolor{darkblue}{\textbf{\ipa{ə˧dɑ˥ | ə˩pʰo˩ hɯ˩-ze˥!}}}
\trans Papa est sorti! (Adressé à une petite fille qui vient de se réveiller de sa sieste et cherche son père.)
\trans \textit{\textcolor{brown}{\zh{爸爸出去了!}}}
\end{exe}
\begin{exe}
\sn \textcolor{darkblue}{\textbf{\ipa{ə˩pʰo˩-bv̩˥ | lo˧ ʝi˧}}}
\trans travailler à l'extérieur, aider d'autres familles (par exemple pour la récolte); aussi : aller chercher du travail à la ville
\trans \textit{\textcolor{brown}{\zh{在外边工作:去帮别人家的忙(特别是收庄稼的时候),或者到城市里面打工}}}
\end{exe}


\paragraph{\hspace{-0.5cm} {\Large \textcolor{darkblue}{\textbf{\ipa{ɑ˩pʰo˩-hĩ˩}}}}} \hypertarget{A\string_Bp\string_ho\string_B-hi\string_~\string_B1}{}
\markboth{\textcolor{darkblue}{\textbf{\ipa{ɑ˩pʰo˩-hĩ˩}}}}{}
\textcolor{teal}{\mytextsc{nom}} \hspace{4pt} Ton~: L.

Quelqu'un d'autre; une personne extérieure à la famille.
\textcolor{brown}{\zh{外人,别人。农村人称呼所有城里人为外面的人。}}




\paragraph{\hspace{-0.5cm} {\Large \textcolor{darkblue}{\textbf{\ipa{ɑ˩pʰv̩˧˥}}}}} \hypertarget{A\string_Bp\string_hv\string_=\string_M\string_T1}{}
\markboth{\textcolor{darkblue}{\textbf{\ipa{ɑ˩pʰv̩˧˥}}}}{}
\textcolor{teal}{\mytextsc{verbe}} \hspace{4pt} Ton~: LM+MH\#.

Roter.
\textcolor{brown}{\zh{打饱嗝儿。}}




\paragraph{\hspace{-0.5cm} {\Large \textcolor{darkblue}{\textbf{\ipa{ɑ˩pʰv̩\#˥}}}}} \hypertarget{A\string_Bp\string_hv\string_=\#\string_T1}{}
\markboth{\textcolor{darkblue}{\textbf{\ipa{ɑ˩pʰv̩\#˥}}}}{}
\textcolor{teal}{\mytextsc{nom}} \hspace{4pt} Ton~: LM+\#H.

Jars: mâle de l'oie.
\textcolor{brown}{\zh{公鹅。}}


 \mytextsc{clf}~: \textcolor{darkblue}{\textbf{\ipa{mi˩}}} 

\paragraph{\hspace{-0.5cm} {\Large \textcolor{darkblue}{\textbf{\ipa{ɑ˩ʁo˧}}}}} \hypertarget{A\string_BRo\string_M1}{}
\markboth{\textcolor{darkblue}{\textbf{\ipa{ɑ˩ʁo˧}}}}{}
\textcolor{teal}{\mytextsc{nom}} \hspace{4pt} Ton~: LM.

Le foyer, la pièce principale; la maison.
\textcolor{brown}{\zh{家、家里。}}


\begin{exe}
\sn \textcolor{darkblue}{\textbf{\ipa{ɑ˩ʁo˧-hĩ\#˥}}}
\trans la maisonnée, les gens de la famille (proche: ceux qui habitent sous le même toit), la lignée
\trans \textit{\textcolor{brown}{\zh{家人(住在一起的家人)}}}
\end{exe}
\begin{exe}
\sn \textcolor{darkblue}{\textbf{\ipa{ɑ˩ʁo˧=ɻæ˩}}}
\trans la maisonnée, les gens de la famille
\trans \textit{\textcolor{brown}{\zh{家人、家族(住在一起的家人)}}}
\end{exe}
\begin{exe}
\sn \textcolor{darkblue}{\textbf{\ipa{njɤ˧ | ɑ˩ʁo˧}}}
\trans mon foyer, ma maison
\trans \textit{\textcolor{brown}{\zh{我家}}}
\end{exe}
\begin{exe}
\sn \textcolor{darkblue}{\textbf{\ipa{njɤ˧ | ɑ˩ʁo˧=ɻ̍˩}}}
\trans ma famille, ma lignée
\trans \textit{\textcolor{brown}{\zh{我的家族}}}
\end{exe}
\begin{exe}
\sn \textcolor{darkblue}{\textbf{\ipa{no˧ | ɑ˩ʁo˧}}}
\trans ton foyer, ta maison
\trans \textit{\textcolor{brown}{\zh{你家}}}
\end{exe}
\begin{exe}
\sn \textcolor{darkblue}{\textbf{\ipa{ɑ˩ʁo˧ ʝi˧}}}
\trans gérer la maisonnée, s'occuper de la famille (tâche de la personne qui répartit les travaux à accomplir, veille aux approvisionnements...)
\trans \textit{\textcolor{brown}{\zh{管家里的事情(如:分配工作、家务等)}}}
\end{exe}
 \mytextsc{clf}~: \textcolor{darkblue}{\textbf{\ipa{ɭɯ˧}}} 

\paragraph{\hspace{-0.5cm} {\Large \textcolor{darkblue}{\textbf{\ipa{ɑ˩zo\#˥}}}}} \hypertarget{A\string_Bzo\#\string_T1}{}
\markboth{\textcolor{darkblue}{\textbf{\ipa{ɑ˩zo\#˥}}}}{}
\textcolor{teal}{\mytextsc{nom}} \hspace{4pt} Ton~: LM+\#H.

Oison, petit de l'oie.
\textcolor{brown}{\zh{小鹅。}}




\paragraph{\hspace{-0.5cm} {\Large \textcolor{darkblue}{\textbf{\ipa{ɑ˩˧}}}}} \hypertarget{A\string_B\string_M1}{}
\markboth{\textcolor{darkblue}{\textbf{\ipa{ɑ˩˧}}}}{}
\textcolor{teal}{\mytextsc{nom}} \hspace{4pt} Ton~: LM.

Oie.
\textcolor{brown}{\zh{鹅。}}


\begin{exe}
\sn \textcolor{darkblue}{\textbf{\ipa{ɑ˩ dzɯ˥-ze˩}}}
\trans ...a mangé (une) oie
\trans \textit{\textcolor{brown}{\zh{吃了鹅}}}
\end{exe}
\begin{exe}
\sn \textcolor{darkblue}{\textbf{\ipa{ɑ˩ hwæ˧-ze˧}}}
\trans ...a acheté (une) oie
\trans \textit{\textcolor{brown}{\zh{买了鹅}}}
\end{exe}
 \mytextsc{clf}~: \textcolor{darkblue}{\textbf{\ipa{mi˩}}} 

\newpage
\section*{\centering- \textcolor{darkblue}{\textbf{\ipa{æ}}} \textcolor{darkblue}{\textbf{\ipa{æ̃}}} -}
\paragraph{\hspace{-0.5cm} {\Large \textcolor{darkblue}{\textbf{\ipa{æ˧bæ˧}}}}} \hypertarget{\{\string_Mb\{\string_M1}{}
\markboth{\textcolor{darkblue}{\textbf{\ipa{æ˧bæ˧}}}}{}
\textcolor{teal}{\mytextsc{nom}} \hspace{4pt} Ton~: M.

Goître.
\textcolor{brown}{\zh{甲状腺肿瘤。}}


 \mytextsc{clf}~: \textcolor{darkblue}{\textbf{\ipa{ɭɯ˧}}} 

\paragraph{\hspace{-0.5cm} {\Large \textcolor{darkblue}{\textbf{\ipa{æ˧bæ˧-ʈʂʰæ˧ɣɯ\#˥}}}}} \hypertarget{\{\string_Mb\{\string_M-t`s`\string_h\{\string_MGM\#\string_T1}{}
\markboth{\textcolor{darkblue}{\textbf{\ipa{æ˧bæ˧-ʈʂʰæ˧ɣɯ\#˥}}}}{}
\textcolor{teal}{\mytextsc{nom}} \hspace{4pt} Ton~: \#H.

Algue; littéralement ``médicament contre le goître" car tel était le motif de la diffusion à Yongning de l'algue, qui contient de l'iode.
\textcolor{brown}{\zh{海带。}}




\paragraph{\hspace{-0.5cm} {\Large \textcolor{darkblue}{\textbf{\ipa{æ˧-hi˩hi˩}}}}} \hypertarget{\{\string_M-hi\string_Bhi\string_B1}{}
\markboth{\textcolor{darkblue}{\textbf{\ipa{æ˧-hi˩hi˩}}}}{}
\textcolor{teal}{\mytextsc{interjection}} \hspace{4pt} Ton~: \mytextsc{L}.

Cri qui exprime défiance ou provocation, mais également triomphe ou exultation.
\textcolor{brown}{\zh{啊嘿嘿!(蔑视、挑衅或欢呼的叫声)。}}


\begin{exe}
\sn \textcolor{darkblue}{\textbf{\ipa{æ˧-hi˩hi˩ kʰɯ˩}}}
\trans lancer des cris de défiance
\trans \textit{\textcolor{brown}{\zh{喊出蔑视、挑衅或欢呼的叫声}}}
\end{exe}


\paragraph{\hspace{-0.5cm} {\Large \textcolor{darkblue}{\textbf{\ipa{æ˧ʝi˩}}}}} \hypertarget{\{\string_Mj££i\string_B1}{}
\markboth{\textcolor{darkblue}{\textbf{\ipa{æ˧ʝi˩}}}}{}
\textcolor{teal}{\mytextsc{nom}} \hspace{4pt} Ton~: L\#.

Cri.
\textcolor{brown}{\zh{叫声。}}


\begin{exe}
\sn \textcolor{darkblue}{\textbf{\ipa{æ˧ʝi˩ kʰɯ˩}}}
\trans crier
\trans \textit{\textcolor{brown}{\zh{叫}}}
\end{exe}
\begin{exe}
\sn \textcolor{darkblue}{\textbf{\ipa{no˧ | æ˧ʝi˩ tʰɑ˩-kʰɯ˩! | no˧ se˧dʑæ˩ɻæ˩-gv˩! |}}}
\trans Ne fais pas tant de bruit! (Contexte: réprimande qu'on adressait aux gens qui parlaient trop fort, qui élevaient la voix.)
\trans \textit{\textcolor{brown}{\zh{别那么大声!}}}
\end{exe}


\paragraph{\hspace{-0.5cm} {\Large \textcolor{darkblue}{\textbf{\ipa{æ˧ɲi\#˥}}}}} \hypertarget{\{\string_MJi\#\string_T1}{}
\markboth{\textcolor{darkblue}{\textbf{\ipa{æ˧ɲi\#˥}}}}{}
\textcolor{teal}{\mytextsc{nom}} \hspace{4pt} Ton~: \#H.

Clarinette.
\textcolor{brown}{\zh{唢呐。}}


 \mytextsc{clf}~: \textcolor{darkblue}{\textbf{\ipa{ɭɯ˧}}} 

\paragraph{\hspace{-0.5cm} {\Large \textcolor{darkblue}{\textbf{\ipa{æ˧ʁwæ˧}}}}} \hypertarget{\{\string_MRw\{\string_M1}{}
\markboth{\textcolor{darkblue}{\textbf{\ipa{æ˧ʁwæ˧}}}}{}
\textcolor{teal}{\mytextsc{nom}} \hspace{4pt} Ton~: M.

Abricot.
\textcolor{brown}{\zh{杏。}}


\begin{exe}
\sn \textcolor{darkblue}{\textbf{\ipa{æ˧ʁwæ˧ | ɖɯ˧-ɭɯ˧}}}
\trans un abricot
\trans \textit{\textcolor{brown}{\zh{一颗杏}}}
\end{exe}
 \mytextsc{clf}~: \textcolor{darkblue}{\textbf{\ipa{ɭɯ˧}}} 

\paragraph{\hspace{-0.5cm} {\Large \textcolor{darkblue}{\textbf{\ipa{æ˧ʂæ\#˥}}}}} \hypertarget{\{\string_Ms`\{\#\string_T1}{}
\markboth{\textcolor{darkblue}{\textbf{\ipa{æ˧ʂæ\#˥}}}}{}
\textcolor{teal}{\mytextsc{adverbe}} \hspace{4pt} Ton~: \#H.

Jadis, dans le passé.
\textcolor{brown}{\zh{从前。}}


\begin{exe}
\sn \textcolor{darkblue}{\textbf{\ipa{æ˧ʂæ˧-kɤ˥ʈʂɯ˩}}}
\trans Dires du temps jadis, traditions orales
\trans \textit{\textcolor{brown}{\zh{过去的说法,过去的口传文化}}}
\end{exe}


\paragraph{\hspace{-0.5cm} {\Large \textcolor{darkblue}{\textbf{\ipa{æ˧ʂæ˧}}}}} \hypertarget{\{\string_Ms`\{\string_M1}{}
\markboth{\textcolor{darkblue}{\textbf{\ipa{æ˧ʂæ˧}}}}{}
\textcolor{teal}{\mytextsc{nom}} \hspace{4pt} Ton~: M.

Nom d'une montagne: l'une des deux principales montagnes autour de la plaine de Yongning, la montagne masculine (``le jeune homme": \textcolor{darkblue}{\textbf{\ipa{/pʰæ˧tɕi˥/}}}); l'autre étant la montagne \textcolor{darkblue}{\textbf{\ipa{/kɤ˧mv̩˧˥/}}}, montagne féminine (``la jeune femme": \textcolor{darkblue}{\textbf{\ipa{mi˩zɯ˩˥/}}}).
\textcolor{brown}{\zh{一座山的名字。}}


\begin{exe}
\sn \textcolor{darkblue}{\textbf{\ipa{kɤ˧mv̩˧˥, | æ˧ʂæ˧, | ŋwɤ˧hɑ̃˩, | ʂwæ˧gv̩\#˥, | nɑ˩tsʰi˩˥ | -tɕʰɤ˧pɤ˧mi\#˥, | qv̩˧ɻ̍˧-ʈʂʰɑ˧nɑ˥ |}}}
\trans Les six montagnes de Yongning qui portent un nom. Les autres sommets du voisinage n'ont pas une valeur symbolique comparable, et ne portent pas de nom communément utilisé.
\trans \textit{\textcolor{brown}{\zh{永宁地区有固定名字的六座山。其它的山,因为没有重要的象征意义,因此没有取名。}}}
\end{exe}


\paragraph{\hspace{-0.5cm} {\Large \textcolor{darkblue}{\textbf{\ipa{æ˧ʂæ˧-pi˧mv̩˧˥}}}}} \hypertarget{\{\string_Ms`\{\string_M-pi\string_Mmv\string_=\string_M\string_T1}{}
\markboth{\textcolor{darkblue}{\textbf{\ipa{æ˧ʂæ˧-pi˧mv̩˧˥}}}}{}
\textcolor{teal}{\mytextsc{nom}} \hspace{4pt} Ton~: MH\#.
\textit{De:} \textbf{æ˧ʂæ\#˥ et pi˧mv̩˥\$} 
Conte, récit du temps jadis; terme plus familier que \textcolor{darkblue}{\textbf{\ipa{/æ˧ʂæ˧-tɑ˩mv̩˩/}}}.
\textcolor{brown}{\zh{传统故事。}}


 \mytextsc{clf}~: \textcolor{darkblue}{\textbf{\ipa{kʰwɤ˥}}} 

\paragraph{\hspace{-0.5cm} {\Large \textcolor{darkblue}{\textbf{\ipa{æ˧ʂæ˧-qʰwæ\#˥}}}}} \hypertarget{\{\string_Ms`\{\string_M-q\string_hw\{\#\string_T1}{}
\markboth{\textcolor{darkblue}{\textbf{\ipa{æ˧ʂæ˧-qʰwæ\#˥}}}}{}
\textcolor{teal}{\mytextsc{nom}} \hspace{4pt} Ton~: \#H.
\textit{De:} \textbf{æ˧ʂæ\#˥ et qʰwæ˧} 
Tradition orale; littéralement: ``messages du temps jadis".
\textcolor{brown}{\zh{口传文化。直译:“(来自)古时候的寓意”。}}


\begin{exe}
\sn \textcolor{darkblue}{\textbf{\ipa{æ˧ʂæ˧-qʰwæ˧-ɳɯ˥ | dʑo˩-ɲi˥!}}}
\trans ``(C'est pas moi qui invente ça:) c'est un dicton d'autrefois/ c'est quelque chose qui existe dans la tradition!" (Commentaire de quelqu'un qui cite un proverbe/dicton, et souligne qu'il ne s'agit pas de paroles en l'air, mais de vérités.)
\trans \textit{\textcolor{brown}{\zh{“(这些道理,不是我个人的意见:)传统中是这样讲的! / 咱们的口传文化中就是这么讲的!”(情景:一个人提到一个谚语,也强调这些不是空话,而是重要的一个道理。)}}}
\end{exe}


\paragraph{\hspace{-0.5cm} {\Large \textcolor{darkblue}{\textbf{\ipa{æ˧ʂæ˧-qʰwɤ˧˥}}}}} \hypertarget{\{\string_Ms`\{\string_M-q\string_hw7\string_M\string_T1}{}
\markboth{\textcolor{darkblue}{\textbf{\ipa{æ˧ʂæ˧-qʰwɤ˧˥}}}}{}
\textcolor{teal}{\mytextsc{nom}} \hspace{4pt} Ton~: MH\#.

Histoire, conte, récit traditionnel.
\textcolor{brown}{\zh{故事。}}


\begin{exe}
\sn \textcolor{darkblue}{\textbf{\ipa{æ˧ʂæ˧qʰwɤ˧ ʐwɤ˧˥}}}
\trans raconter une histoire
\trans \textit{\textcolor{brown}{\zh{讲故事}}}
\end{exe}
\begin{exe}
\sn \textcolor{darkblue}{\textbf{\ipa{ə˧ʝi˧-ʂɯ˥ʝi˩, | æ˧ʂæ˧qʰwɤ˧ ʐwɤ˧-kv̩˥!}}}
\trans dans le temps, (on) racontait des histoires!
\trans \textit{\textcolor{brown}{\zh{在过去,大家经常讲故事!}}}
\end{exe}
 \mytextsc{clf}~: \textcolor{darkblue}{\textbf{\ipa{kʰwɤ˥}}} 

\paragraph{\hspace{-0.5cm} {\Large \textcolor{darkblue}{\textbf{\ipa{æ˧ʂæ˧-tɑ˩mv̩˩}}}}} \hypertarget{\{\string_Ms`\{\string_M-tA\string_Bmv\string_=\string_B1}{}
\markboth{\textcolor{darkblue}{\textbf{\ipa{æ˧ʂæ˧-tɑ˩mv̩˩}}}}{}
\textcolor{teal}{\mytextsc{nom}} \hspace{4pt} Ton~: \mytextsc{L}.
\textit{De:} \textbf{æ˧ʂæ\#˥ et tɑ˩mv̩˩} 
Proverbe, dicton.
\textcolor{brown}{\zh{谚语、习语。}}


\begin{exe}
\sn \textcolor{darkblue}{\textbf{\ipa{æ˧ʂæ˧-tɑ˩mv̩˩ tʰv̩˩-kʰwɤ˩}}}
\trans ce dicton
\trans \textit{\textcolor{brown}{\zh{那个习语}}}
\end{exe}
 \mytextsc{clf}~: \textcolor{darkblue}{\textbf{\ipa{kʰwɤ˥}}} 

\paragraph{\hspace{-0.5cm} {\Large \textcolor{darkblue}{\textbf{\ipa{æ˧tse˥-pʰæ˩}}}}} \hypertarget{\{\string_Mtse\string_T-p\string_h\{\string_B1}{}
\markboth{\textcolor{darkblue}{\textbf{\ipa{æ˧tse˥-pʰæ˩}}}}{}
\textcolor{teal}{\mytextsc{nom}} \hspace{4pt} Ton~: H\#-L.

Os du genou.
\textcolor{brown}{\zh{膝盖骨。}}


 \mytextsc{clf}~: \textcolor{darkblue}{\textbf{\ipa{ɭɯ˧}}} 

\paragraph{\hspace{-0.5cm} {\Large \textcolor{darkblue}{\textbf{\ipa{æ˧tsɯ˥-pɤ˩lv̩˩}}}}} \hypertarget{\{\string_MtsM\string_T-p7\string_Blv\string_=\string_B1}{}
\markboth{\textcolor{darkblue}{\textbf{\ipa{æ˧tsɯ˥-pɤ˩lv̩˩}}}}{}
\textcolor{teal}{\mytextsc{nom}} \hspace{4pt} Ton~: H\#-L.

Nuque.
\textcolor{brown}{\zh{项背 、项、脖颈儿。}}


 \mytextsc{clf}~: \textcolor{darkblue}{\textbf{\ipa{ɭɯ˧}}} 

\paragraph{\hspace{-0.5cm} {\Large \textcolor{darkblue}{\textbf{\ipa{æ˧ʈwɤ˩}}}}} \hypertarget{\{\string_Mt`w7\string_B1}{}
\markboth{\textcolor{darkblue}{\textbf{\ipa{æ˧ʈwɤ˩}}}}{}
\textcolor{teal}{\mytextsc{nom}} \hspace{4pt} Ton~: L\#.

Le petit matin.
\textcolor{brown}{\zh{清晨、一大早(鸡叫的时候)。}}




\paragraph{\hspace{-0.5cm} {\Large \textcolor{darkblue}{\textbf{\ipa{æ˩gv̩˩}}}}} \hypertarget{\{\string_Bgv\string_=\string_B1}{}
\markboth{\textcolor{darkblue}{\textbf{\ipa{æ˩gv̩˩}}}}{}
\textcolor{teal}{\mytextsc{nom}} \hspace{4pt} Ton~: L.

Araire. Il n'existe pas deux termes distincts, l'un pour l'araire et l'autre pour la charrue; à la date de l'enquête, seule l'araire était en usage.
\textcolor{brown}{\zh{犁头。}}


\begin{exe}
\sn \textcolor{darkblue}{\textbf{\ipa{æ˩gv̩˩ tʰv̩˩-nɑ˥}}}
\trans \mytextsc{n}+\mytextsc{dem}+\mytextsc{clf}
\trans \textit{\textcolor{brown}{\zh{这把犁头}}}
\end{exe}
\begin{exe}
\sn \textcolor{darkblue}{\textbf{\ipa{æ˩mo˥}}}
\trans araire usagée, vieille araire (hors d'usage du fait de l'usure)
\trans \textit{\textcolor{brown}{\zh{陈旧的犁头(不能再用了)}}}
\end{exe}
\begin{exe}
\sn \textcolor{darkblue}{\textbf{\ipa{æ˩mo˥ tʰv̩˩-nɑ˩}}}
\trans \mytextsc{n}+\mytextsc{dem}+\mytextsc{clf}
\trans \textit{\textcolor{brown}{\zh{这个旧犁杆}}}
\end{exe}
\begin{exe}
\sn \textcolor{darkblue}{\textbf{\ipa{æ˩-ʂɯ˩˥}}}
\trans araire neuve
\trans \textit{\textcolor{brown}{\zh{新的犁头}}}
\end{exe}
 \mytextsc{clf}~: \textcolor{darkblue}{\textbf{\ipa{nɑ˧}}} 

\paragraph{\hspace{-0.5cm} {\Large \textcolor{darkblue}{\textbf{\ipa{æ˩gv̩˩-mæ˩qo˥}}}}} \hypertarget{\{\string_Bgv\string_=\string_B-m\{\string_Bqo\string_T1}{}
\markboth{\textcolor{darkblue}{\textbf{\ipa{æ˩gv̩˩-mæ˩qo˥}}}}{}
\textcolor{teal}{\mytextsc{nom}} \hspace{4pt} Ton~: L+H\#.

Mancheron de l'araire, manche de l'araire.
\textcolor{brown}{\zh{犁把。}}


\begin{exe}
\sn \textcolor{darkblue}{\textbf{\ipa{æ̃˩gv̩˩-mæ˩ ʑi˩-hĩ˥}}}
\trans la personne qui tient le mancheron de la charrue
\trans \textit{\textcolor{brown}{\zh{抓着犁把的人}}}
\end{exe}
\begin{exe}
\sn \textcolor{darkblue}{\textbf{\ipa{æ̃˩gv̩˩-mæ˩qo˥ tʰv̩˩-nɑ˩}}}
\trans \mytextsc{n}+\mytextsc{dem}+\mytextsc{clf}
\trans \textit{\textcolor{brown}{\zh{这个犁把}}}
\end{exe}
 \mytextsc{clf}~: \textcolor{darkblue}{\textbf{\ipa{nɑ˧}}} 

\paragraph{\hspace{-0.5cm} {\Large \textcolor{darkblue}{\textbf{\ipa{æ˩mi˧-mv̩˧ʈv̩˥}}}}} \hypertarget{\{\string_Bmi\string_M-mv\string_=\string_Mt`v\string_=\string_T1}{}
\markboth{\textcolor{darkblue}{\textbf{\ipa{æ˩mi˧-mv̩˧ʈv̩˥}}}}{}
\textcolor{teal}{\mytextsc{nom}} \hspace{4pt} Ton~: LM+H\#.

Astragale (os du dessus du pied; os ``en boule" sur les deux côtés du pied, au-dessous de la cheville).
\textcolor{brown}{\zh{踝骨。}}


 \mytextsc{clf}~: \textcolor{darkblue}{\textbf{\ipa{ɭɯ˧}}} 

\paragraph{\hspace{-0.5cm} {\Large \textcolor{darkblue}{\textbf{\ipa{æ˩mi˧-ʁwɤ\#˥}}}}} \hypertarget{\{\string_Bmi\string_M-Rw7\#\string_T1}{}
\markboth{\textcolor{darkblue}{\textbf{\ipa{æ˩mi˧-ʁwɤ\#˥}}}}{}
\textcolor{teal}{\mytextsc{nom}} \hspace{4pt} Ton~: LM+\#H.

Amiwa, premier village que l'on rencontre sur la route entre \textcolor{darkblue}{\textbf{\ipa{/qʰæ˧tɕʰi˧/}}} et \textcolor{darkblue}{\textbf{\ipa{/ʈʂo˧ʂɯ\#˥/}}}. Dans la géographie traditionnelle na, qui prend comme point d'origine le lac Lugu, Amiwa est le troisième village de la plaine de Yongning.
\textcolor{brown}{\zh{阿咪瓦、阿米瓦(永宁的一个村落)。}}


\begin{exe}
\sn \textcolor{darkblue}{\textbf{\ipa{dʑɤ˩bv̩˧kɤ˧-sɑ˥ʁwɤ˩, | hi˩ʁwɤ˩-lo˥, | æ˩mi˧-ʁwɤ\#˥, | lɑ˧lo˧-ʁwɤ˥, | lɑ˧ŋwɤ˧, | bɤ˧tsʰo˧gv̩˥, | ə˧lɑ˧-ʁwɤ\#˥, | gæ˧ɻæ˩, | qʰæ˧tɕʰi˧, | tʰo˧ʈɯ\#˥}}}
\trans les dix villages comptant traditionnellement comme faisant partie de Yongning
\trans \textit{\textcolor{brown}{\zh{摩梭传统地理概念中,属于永宁的十个村落}}}
\end{exe}


\paragraph{\hspace{-0.5cm} {\Large \textcolor{darkblue}{\textbf{\ipa{æ˩mo˧}}}}} \hypertarget{\{\string_Bmo\string_M1}{}
\markboth{\textcolor{darkblue}{\textbf{\ipa{æ˩mo˧}}}}{}
\textcolor{teal}{\mytextsc{nom}} \hspace{4pt} Ton~: LM.

Timon (âge, haie) de l'araire.
\textcolor{brown}{\zh{犁杆。}}


\begin{exe}
\sn \textcolor{darkblue}{\textbf{\ipa{æ˩gv̩˩-mo˥}}}
\trans même sens
\trans \textit{\textcolor{brown}{\zh{同上}}}
\end{exe}
 \mytextsc{clf}~: \textcolor{darkblue}{\textbf{\ipa{nɑ˧}}} \textit{Voir~:} \hyperlink{\{\string_Bgv\string_=\string_B1}{\textcolor{darkblue}{\textbf{\ipa{æ˩gv̩˩}}}} 

\paragraph{\hspace{-0.5cm} {\Large \textcolor{darkblue}{\textbf{\ipa{æ˩pʰæ˧˥}}}}} \hypertarget{\{\string_Bp\string_h\{\string_M\string_T1}{}
\markboth{\textcolor{darkblue}{\textbf{\ipa{æ˩pʰæ˧˥}}}}{}
\textcolor{teal}{\mytextsc{nom}} \hspace{4pt} Ton~: LM+MH\#.

Falaise. Le terme désigne spécifiquement le dos d'une falaise: l'espace relativement plat en bord de précipice. Pour faire référence à la paroi (face verticale de la falaise), on ajoute \textcolor{darkblue}{\textbf{\ipa{/lɑ˧bi˧/}}} 'escarpement'.
\textcolor{brown}{\zh{悬崖、崖山、崖壁。}}


\begin{exe}
\sn \textcolor{darkblue}{\textbf{\ipa{æ˩pʰæ˧-lɑ˧bi˥}}}
\trans même sens
\trans \textit{\textcolor{brown}{\zh{同上}}}
\end{exe}
 \mytextsc{clf}~: \textcolor{darkblue}{\textbf{\ipa{pʰæ˧˥}}} 

\paragraph{\hspace{-0.5cm} {\Large \textcolor{darkblue}{\textbf{\ipa{æ˩qʰv̩˥}}}}} \hypertarget{\{\string_Bq\string_hv\string_=\string_T1}{}
\markboth{\textcolor{darkblue}{\textbf{\ipa{æ˩qʰv̩˥}}}}{}
\textcolor{teal}{\mytextsc{nom}} \hspace{4pt} Ton~: LH.

Crevasse, petite grotte (où il est difficile de pénétrer).
\textcolor{brown}{\zh{小山洞(难进去,或者钻不进去的山洞)。}}


 \mytextsc{clf}~: \textcolor{darkblue}{\textbf{\ipa{ɭɯ˧}}} 

\paragraph{\hspace{-0.5cm} {\Large \textcolor{darkblue}{\textbf{\ipa{æ˩ʈv̩˥}}}}} \hypertarget{\{\string_Bt`v\string_=\string_T1}{}
\markboth{\textcolor{darkblue}{\textbf{\ipa{æ˩ʈv̩˥}}}}{}
\textcolor{teal}{\mytextsc{nom}} \hspace{4pt} Ton~: LH.

Gros rocher, roc.
\textcolor{brown}{\zh{大岩石。}}


 \mytextsc{clf}~: \textcolor{darkblue}{\textbf{\ipa{ʈv̩˩}}} 

\paragraph{\hspace{-0.5cm} {\Large \textcolor{darkblue}{\textbf{\ipa{æ̃˥}}}}} \hypertarget{\{\string_~\string_T1}{}
\markboth{\textcolor{darkblue}{\textbf{\ipa{æ̃˥}}}}{}
\textcolor{teal}{\mytextsc{nom}} \hspace{4pt} Ton~: \#H.

Cuivre; bronze.
\textcolor{brown}{\zh{铜,包括黄铜、红铜、青铜。}}


\begin{exe}
\sn \textcolor{darkblue}{\textbf{\ipa{æ̃˧tso˧-æ̃˧mo˩}}}
\trans instruments en cuivre, objets en cuivre
\trans \textit{\textcolor{brown}{\zh{铜做的工具、物品}}}
\end{exe}
\begin{exe}
\sn \textcolor{darkblue}{\textbf{\ipa{æ̃˧ lɑ˩-zo˩-ɳɯ˩, | ʂe˧ mɤ˧-lɑ˧˥!}}}
\trans ``Celui qui travaille le cuivre, il ne doit pas travailler le fer/on ne doit pas lui confier de tâches de forgeron (=travail du fer)!" Ces deux spécialités demandent des qualités différentes: de la force physique pour le travail du fer, et du soin pour le travail du cuivre. Le dicton s'emploie pour souligner que chacun a son domaine de compétence.
\trans \textit{\textcolor{brown}{\zh{“打铜的人,不打铁!”这两种工作需要不同的能力:打铁需要体力,打铜需要细致。这个谚语意指:每个人有他的专业,不能随便跨越到其它领域去。}}}
\end{exe}


\paragraph{\hspace{-0.5cm} {\Large \textcolor{darkblue}{\textbf{\ipa{æ̃˧qæ˩}}} \textsubscript{1}}} \hypertarget{\{\string_~\string_Mq\{\string_B1}{}
\markboth{\textcolor{darkblue}{\textbf{\ipa{æ̃˧qæ˩}}} \textsubscript{1}}{}
\textcolor{teal}{\mytextsc{nom}} \hspace{4pt} Ton~: L\#.

Perroquet.
\textcolor{brown}{\zh{鹦鹉。}}


 \mytextsc{clf}~: \textcolor{darkblue}{\textbf{\ipa{mi˩}}} \textit{Voir~:} \hyperlink{\{\string_~\string_Mq\{\string_B2}{\textcolor{darkblue}{\textbf{\ipa{æ̃˧qæ˩}}} \textsubscript{2}} 

\paragraph{\hspace{-0.5cm} {\Large \textcolor{darkblue}{\textbf{\ipa{æ̃˧qæ˩}}} \textsubscript{2}}} \hypertarget{\{\string_~\string_Mq\{\string_B2}{}
\markboth{\textcolor{darkblue}{\textbf{\ipa{æ̃˧qæ˩}}} \textsubscript{2}}{}
\textcolor{teal}{\mytextsc{adjectif}} \hspace{4pt} Ton~: L\#.

De couleur bleue/verte; couleur un peu plus légère que le vert de la plaine; équivalent du chinois \textcolor{brown}{\zh{青}}. Littéralement: '[couleur] perroquet'.
\textcolor{brown}{\zh{像鹦鹉的颜色:青色、蓝色、绿色。}}


\begin{exe}
\sn \textcolor{darkblue}{\textbf{\ipa{æ̃˧qæ˩-ni˩gv̩˩}}}
\trans couleur perroquet; littéralement 'comme un perroquet'
\trans \textit{\textcolor{brown}{\zh{像鹦鹉的颜色:青、蓝色、绿色}}}
\end{exe}
\begin{exe}
\sn \textcolor{darkblue}{\textbf{\ipa{[F5] æ̃˧qæ˩-bɑ˩lɑ˩}}}
\trans vêtement bleu; littéralement vêtement 'couleur perroquet'
\trans \textit{\textcolor{brown}{\zh{青、蓝色、绿色衣服}}}
\end{exe}
\begin{exe}
\sn \textcolor{darkblue}{\textbf{\ipa{[F5] æ̃˧qæ˩ni˩\textasciitilde{}æ̃˧qæ˩ni˩gv̩˩}}}
\trans \mytextsc{red;} même sens: bleu-vert
\trans \textit{\textcolor{brown}{\zh{\mytextsc{重叠。同上:青色}}}}
\end{exe}
\textit{Voir~:} \hyperlink{\{\string_~\string_Mq\{\string_B1}{\textcolor{darkblue}{\textbf{\ipa{æ̃˧qæ˩}}} \textsubscript{1}} 

\paragraph{\hspace{-0.5cm} {\Large \textcolor{darkblue}{\textbf{\ipa{æ̃˧ʂwæ˥}}}}} \hypertarget{\{\string_~\string_Ms`w\{\string_T1}{}
\markboth{\textcolor{darkblue}{\textbf{\ipa{æ̃˧ʂwæ˥}}}}{}
\textcolor{teal}{\mytextsc{nom}} \hspace{4pt} Ton~: H\#.

Coq.
\textcolor{brown}{\zh{公鸡。}}


\begin{exe}
\sn \textcolor{darkblue}{\textbf{\ipa{æ̃˧ʂwæ˥-æ̃˩mi˩}}}
\trans coq et poule
\trans \textit{\textcolor{brown}{\zh{公鸡与母鸡}}}
\end{exe}
 \mytextsc{clf}~: \textcolor{darkblue}{\textbf{\ipa{mi˩}}} 

\paragraph{\hspace{-0.5cm} {\Large \textcolor{darkblue}{\textbf{\ipa{æ̃˧tsɯ˥}}}}} \hypertarget{\{\string_~\string_MtsM\string_T1}{}
\markboth{\textcolor{darkblue}{\textbf{\ipa{æ̃˧tsɯ˥}}}}{}
\textcolor{teal}{\mytextsc{nom}} \hspace{4pt} Ton~: H\#.

Poussin.
\textcolor{brown}{\zh{雏鸡、稚鸡。}}


 \mytextsc{clf}~: \textcolor{darkblue}{\textbf{\ipa{ɭɯ˧}}} 

\paragraph{\hspace{-0.5cm} {\Large \textcolor{darkblue}{\textbf{\ipa{æ̃˧tsɯ˥-kʰɯ˩ʈʂɤ˩-mo˩}}}}} \hypertarget{\{\string_~\string_MtsM\string_T-k\string_hM\string_Bt`s`7\string_B-mo\string_B1}{}
\markboth{\textcolor{darkblue}{\textbf{\ipa{æ̃˧tsɯ˥-kʰɯ˩ʈʂɤ˩-mo˩}}}}{}
\textcolor{teal}{\mytextsc{nom}} \hspace{4pt} Ton~: H\#-.

``champignon griffes-de-poulet": champignon comestible.
\textcolor{brown}{\zh{扫把菌,扫帚菌(一种菌子)。}}




\paragraph{\hspace{-0.5cm} {\Large \textcolor{darkblue}{\textbf{\ipa{æ̃˧ʈwɤ˩-mv̩˩kʰv̩˩}}}}} \hypertarget{\{\string_~\string_Mt`w7\string_B-mv\string_=\string_Bk\string_hv\string_=\string_B1}{}
\markboth{\textcolor{darkblue}{\textbf{\ipa{æ̃˧ʈwɤ˩-mv̩˩kʰv̩˩}}}}{}
\textcolor{teal}{\mytextsc{adverbe}} \hspace{4pt} Ton~: L\#-L.

Du matin au soir, constamment.
\textcolor{brown}{\zh{一直不停地,从早到晚。直译:‘(从)早上(到)晚上’。}}




\paragraph{\hspace{-0.5cm} {\Large \textcolor{darkblue}{\textbf{\ipa{æ̃˧-v̩\#˥}}}}} \hypertarget{\{\string_~\string_M-v\string_=\#\string_T1}{}
\markboth{\textcolor{darkblue}{\textbf{\ipa{æ̃˧-v̩\#˥}}}}{}
\textcolor{teal}{\mytextsc{nom}} \hspace{4pt} Ton~: \#H.

Casserole en cuivre.
\textcolor{brown}{\zh{铜锅。}}


 \mytextsc{clf}~: \textcolor{darkblue}{\textbf{\ipa{ɭɯ˧}}} 

\paragraph{\hspace{-0.5cm} {\Large \textcolor{darkblue}{\textbf{\ipa{æ̃˩\textsubscript{a}}}}}} \hypertarget{\{\string_~\string_Ba1}{}
\markboth{\textcolor{darkblue}{\textbf{\ipa{æ̃˩\textsubscript{a}}}}}{}
\textcolor{teal}{\mytextsc{classificateur}} \hspace{4pt} Ton~: L\textsubscript{a}.

Classificateur des feux.
\textcolor{brown}{\zh{量词:火(一团)。}}


\begin{exe}
\sn \textcolor{darkblue}{\textbf{\ipa{mv̩˧ | ʈʂʰɯ˧-æ̃˥}}}
\trans ce feu (ton: H\# / H\$)
\trans \textit{\textcolor{brown}{\zh{这团火}}}
\end{exe}


\paragraph{\hspace{-0.5cm} {\Large \textcolor{darkblue}{\textbf{\ipa{æ̃˩\textsubscript{a}}}} \textsubscript{1}}} \hypertarget{\{\string_~\string_Ba1}{}
\markboth{\textcolor{darkblue}{\textbf{\ipa{æ̃˩\textsubscript{a}}}} \textsubscript{1}}{}
\textcolor{teal}{\mytextsc{verbe}} \hspace{4pt} Ton~: L\textsubscript{a}.

Réfléchir, renvoyer (un miroir renvoie la lumière; une cloison/toiture étanche renvoie la pluie =est étanche à la pluie).
\textcolor{brown}{\zh{反射、辉映。}}




\paragraph{\hspace{-0.5cm} {\Large \textcolor{darkblue}{\textbf{\ipa{æ̃˩\textsubscript{a}}}} \textsubscript{2}}} \hypertarget{\{\string_~\string_Ba2}{}
\markboth{\textcolor{darkblue}{\textbf{\ipa{æ̃˩\textsubscript{a}}}} \textsubscript{2}}{}
\textcolor{teal}{\mytextsc{verbe}} \hspace{4pt} Ton~: L\textsubscript{a}.

S'enliser; se coincer, se bloquer.
\textcolor{brown}{\zh{堵塞、塞。}}


\begin{exe}
\sn \textcolor{darkblue}{\textbf{\ipa{ʝi˩mi˩˥ | ɖʐæ˩qʰæ˧-qo˩ æ̃˩!}}}
\trans La vache est enlisée dans la boue.
\trans \textit{\textcolor{brown}{\zh{牛陷在泥巴里。}}}
\end{exe}


\paragraph{\hspace{-0.5cm} {\Large \textcolor{darkblue}{\textbf{\ipa{æ̃˩bi˩}}}}} \hypertarget{\{\string_~\string_Bbi\string_B1}{}
\markboth{\textcolor{darkblue}{\textbf{\ipa{æ̃˩bi˩}}}}{}
\textcolor{teal}{\mytextsc{nom}} \hspace{4pt} Ton~: L.

Abi: village sur le chemin de Qiansuo.
\textcolor{brown}{\zh{从阿拉瓦村到前所路上经过的一个村落。}}


\begin{exe}
\sn \textcolor{darkblue}{\textbf{\ipa{æ̃˩bi˩-ʁwɤ˥}}}
\trans même sens: le village de \textcolor{darkblue}{\textbf{\ipa{/æ̃˩bi˩/}}}
\trans \textit{\textcolor{brown}{\zh{同上}}}
\end{exe}
\begin{exe}
\sn \textcolor{darkblue}{\textbf{\ipa{æ̃˩bi˩-hĩ˥ ɲi˩!}}}
\trans C'est quelqu'un du village de \textcolor{darkblue}{\textbf{\ipa{/æ̃˩bi˩/!}}}
\trans \textit{\textcolor{brown}{\zh{是\textcolor{darkblue}{\textbf{\ipa{/æ̃˩bi˩/}}}村的人!}}}
\end{exe}


\paragraph{\hspace{-0.5cm} {\Large \textcolor{darkblue}{\textbf{\ipa{æ̃˩bv̩˥}}}}} \hypertarget{\{\string_~\string_Bbv\string_=\string_T1}{}
\markboth{\textcolor{darkblue}{\textbf{\ipa{æ̃˩bv̩˥}}}}{}
\textcolor{teal}{\mytextsc{nom}} \hspace{4pt} Ton~: LH.

Poulailler.
\textcolor{brown}{\zh{鸡圈。}}


 \mytextsc{clf}~: \textcolor{darkblue}{\textbf{\ipa{ɭɯ˧}}} 

\paragraph{\hspace{-0.5cm} {\Large \textcolor{darkblue}{\textbf{\ipa{æ̃˩-kʰv̩˧˥}}}}} \hypertarget{\{\string_~\string_B-k\string_hv\string_=\string_M\string_T1}{}
\markboth{\textcolor{darkblue}{\textbf{\ipa{æ̃˩-kʰv̩˧˥}}}}{}
\textcolor{teal}{\mytextsc{nom}} \hspace{4pt} Ton~: LM+MH\#.

Année du coq.
\textcolor{brown}{\zh{鸡年。}}




\paragraph{\hspace{-0.5cm} {\Large \textcolor{darkblue}{\textbf{\ipa{æ̃˩li˧pʰæ˥}}}}} \hypertarget{\{\string_~\string_Bli\string_Mp\string_h\{\string_T1}{}
\markboth{\textcolor{darkblue}{\textbf{\ipa{æ̃˩li˧pʰæ˥}}}}{}
\textcolor{teal}{\mytextsc{nom}} \hspace{4pt} Ton~: LM+H\#.

Miroir.
\textcolor{brown}{\zh{镜子。}}


 \mytextsc{clf}~: \textcolor{darkblue}{\textbf{\ipa{pʰæ˧˥}}} 

\paragraph{\hspace{-0.5cm} {\Large \textcolor{darkblue}{\textbf{\ipa{æ̃˩ɬi\#˥}}}}} \hypertarget{\{\string_~\string_BKi\#\string_T1}{}
\markboth{\textcolor{darkblue}{\textbf{\ipa{æ̃˩ɬi\#˥}}}}{}
\textcolor{teal}{\mytextsc{nom}} \hspace{4pt} Ton~: LM+\#H.

Âme.
\textcolor{brown}{\zh{灵魂、魂魄。}}

 Emprunt~: tibétain bla(olderform:brla)

 \mytextsc{clf}~: \textcolor{darkblue}{\textbf{\ipa{v̩˧}}} individu

\paragraph{\hspace{-0.5cm} {\Large \textcolor{darkblue}{\textbf{\ipa{æ̃˩mi˧}}}}} \hypertarget{\{\string_~\string_Bmi\string_M1}{}
\markboth{\textcolor{darkblue}{\textbf{\ipa{æ̃˩mi˧}}}}{}
\textcolor{teal}{\mytextsc{nom}} \hspace{4pt} Ton~: LM.

Poule.
\textcolor{brown}{\zh{母鸡。}}


\begin{exe}
\sn \textcolor{darkblue}{\textbf{\ipa{æ̃˩mi˧-æ̃˧ʂwæ˥\#}}}
\trans poule et coq
\trans \textit{\textcolor{brown}{\zh{母鸡与公鸡}}}
\end{exe}
\begin{exe}
\sn \textcolor{darkblue}{\textbf{\ipa{æ̃˩mi˧-æ̃˧tsɯ˥\#}}}
\trans poule et poussins
\trans \textit{\textcolor{brown}{\zh{母鸡与稚鸡}}}
\end{exe}
 \mytextsc{clf}~: \textcolor{darkblue}{\textbf{\ipa{mi˩}}} 

\paragraph{\hspace{-0.5cm} {\Large \textcolor{darkblue}{\textbf{\ipa{æ̃˩ʁv̩˩}}}}} \hypertarget{\{\string_~\string_BRv\string_=\string_B1}{}
\markboth{\textcolor{darkblue}{\textbf{\ipa{æ̃˩ʁv̩˩}}}}{}
\textcolor{teal}{\mytextsc{nom}} \hspace{4pt} Ton~: L.

Œuf.
\textcolor{brown}{\zh{蛋。}}


\begin{exe}
\sn \textcolor{darkblue}{\textbf{\ipa{bæ˧mi˧-æ̃˩ʁv̩˩}}}
\trans œuf de cane
\trans \textit{\textcolor{brown}{\zh{鸭子蛋}}}
\end{exe}
\begin{exe}
\sn \textcolor{darkblue}{\textbf{\ipa{æ̃˩ʁv̩˩ dzɯ˩˥}}}
\trans manger des œufs
\trans \textit{\textcolor{brown}{\zh{吃蛋}}}
\end{exe}
 \mytextsc{clf}~: \textcolor{darkblue}{\textbf{\ipa{ɭɯ˧}}} 

\paragraph{\hspace{-0.5cm} {\Large \textcolor{darkblue}{\textbf{\ipa{æ̃˩ʂe˧li˥-mo˩}}}}} \hypertarget{\{\string_~\string_Bs`e\string_Mli\string_T-mo\string_B1}{}
\markboth{\textcolor{darkblue}{\textbf{\ipa{æ̃˩ʂe˧li˥-mo˩}}}}{}
\textcolor{teal}{\mytextsc{nom}} \hspace{4pt} Ton~: LM+H\#-.

``champignon viande-de-poulet": un champignon comestible, \textit{Amanita spissa}.
\textcolor{brown}{\zh{麻母鸡菌:一种可以吃的菌子,块鳞灰毒鹅膏菌。}}




\paragraph{\hspace{-0.5cm} {\Large \textcolor{darkblue}{\textbf{\ipa{æ̃˩ʂe˩}}}}} \hypertarget{\{\string_~\string_Bs`e\string_B1}{}
\markboth{\textcolor{darkblue}{\textbf{\ipa{æ̃˩ʂe˩}}}}{}
\textcolor{teal}{\mytextsc{nom}} \hspace{4pt} Ton~: L.

Muscle.
\textcolor{brown}{\zh{肌肉。}}


\begin{exe}
\sn \textcolor{darkblue}{\textbf{\ipa{æ̃˩ʂe˩ tsʰi˩˥}}}
\trans avoir la fièvre
\trans \textit{\textcolor{brown}{\zh{发烧}}}
\end{exe}
 \mytextsc{clf}~: \textcolor{darkblue}{\textbf{\ipa{kʰwɤ˥}}} 

\paragraph{\hspace{-0.5cm} {\Large \textcolor{darkblue}{\textbf{\ipa{æ̃˩zɯ˩}}}}} \hypertarget{\{\string_~\string_BzM\string_B1}{}
\markboth{\textcolor{darkblue}{\textbf{\ipa{æ̃˩zɯ˩}}}}{}
\textcolor{teal}{\mytextsc{nom}} \hspace{4pt} Ton~: L.

Agate. Des perles d'agate de diverses couleurs sont utilisées en orfèvrerie. Elles sont de la taille d'un oeuf de caille, les plus gros approchent la taille d'un oeuf de poule. Les perles d'agate étaient intégrées aux bijoux et vêtements, dans une tradition d'inspiration tibétaine.
\textcolor{brown}{\zh{玛瑙。}}


\begin{exe}
\sn \textcolor{darkblue}{\textbf{\ipa{sɯ˧ɻ̍˧-æ̃˩zɯ˩}}}
\trans agate en forme de perle
\trans \textit{\textcolor{brown}{\zh{珠子形状的玛瑙}}}
\end{exe}
\begin{exe}
\sn \textcolor{darkblue}{\textbf{\ipa{æ̃˩zɯ˩-ʂo˩\textasciitilde{}ʂo˥}}}
\trans tout plein d'agate, plein de morceaux d'agate (d'un vêtement)
\trans \textit{\textcolor{brown}{\zh{(衣服上)都镶嵌着玛瑙}}}
\end{exe}
 \mytextsc{clf}~: \textcolor{darkblue}{\textbf{\ipa{ɭɯ˧}}} 

\paragraph{\hspace{-0.5cm} {\Large \textcolor{darkblue}{\textbf{\ipa{æ̃˩˥}}}}} \hypertarget{\{\string_~\string_B\string_T1}{}
\markboth{\textcolor{darkblue}{\textbf{\ipa{æ̃˩˥}}}}{}
\textcolor{teal}{\mytextsc{nom}} \hspace{4pt} Ton~: LH.

Âme (monosyllabe).
\textcolor{brown}{\zh{灵魂。}}


 \mytextsc{clf}~: \textcolor{darkblue}{\textbf{\ipa{v̩˧}}} 

\paragraph{\hspace{-0.5cm} {\Large \textcolor{darkblue}{\textbf{\ipa{æ̃˩˧}}}}} \hypertarget{\{\string_~\string_B\string_M1}{}
\markboth{\textcolor{darkblue}{\textbf{\ipa{æ̃˩˧}}}}{}
\textcolor{teal}{\mytextsc{nom}} \hspace{4pt} Ton~: LM.

Poulet, poule.
\textcolor{brown}{\zh{鸡。}}


\begin{exe}
\sn \textcolor{darkblue}{\textbf{\ipa{æ̃˩ dzɯ˥-ze˩}}}
\trans ...a mangé (un/du) poulet
\trans \textit{\textcolor{brown}{\zh{吃了鸡}}}
\end{exe}
\begin{exe}
\sn \textcolor{darkblue}{\textbf{\ipa{æ̃˩ hwæ˧-ze˧}}}
\trans ...a acheté (un/du) poulet
\trans \textit{\textcolor{brown}{\zh{买了鸡}}}
\end{exe}
\begin{exe}
\sn \textcolor{darkblue}{\textbf{\ipa{æ̃˩˥, | kʰv̩˧, | bo˩˥, | hwɤ˧˥, | ʝi˧, | lɑ˧, | tʰo˧li˧, | mv̩˧gv̩˧, | bv̩˧ʐv̩˧, | ʐwæ˧, | jo˧, | ʑi˩˥}}}
\trans les douze signes astrologiques
\trans \textit{\textcolor{brown}{\zh{十二个生肖}}}
\end{exe}
\begin{exe}
\sn \textcolor{darkblue}{\textbf{\ipa{æ̃˩-mɤ˥}}}
\trans graisse de poulet
\trans \textit{\textcolor{brown}{\zh{鸡油}}}
\end{exe}
\begin{exe}
\sn \textcolor{darkblue}{\textbf{\ipa{æ̃˩-mɤ˥ dzɯ˩}}}
\trans manger de la graisse de poulet
\trans \textit{\textcolor{brown}{\zh{吃鸡油}}}
\end{exe}
 \mytextsc{clf}~: \textcolor{darkblue}{\textbf{\ipa{mi˩}}} 

\newpage
\section*{\centering- \textcolor{darkblue}{\textbf{\ipa{b}}} -}
\paragraph{\hspace{-0.5cm} {\Large \textcolor{darkblue}{\textbf{\ipa{bɑ˧lɑ˧kʰɯ˧tsʰɤ˧}}}}} \hypertarget{bA\string_MlA\string_Mk\string_hM\string_Mts\string_h7\string_M1}{}
\markboth{\textcolor{darkblue}{\textbf{\ipa{bɑ˧lɑ˧kʰɯ˧tsʰɤ˧}}}}{}
\textcolor{teal}{\mytextsc{nom}} \hspace{4pt} Ton~: M.

Araignée.
\textcolor{brown}{\zh{蜘蛛。}}


 \mytextsc{clf}~: \textcolor{darkblue}{\textbf{\ipa{kʰɯ˩}}} 

\paragraph{\hspace{-0.5cm} {\Large \textcolor{darkblue}{\textbf{\ipa{bɑ˩lɑ˩}}}}} \hypertarget{bA\string_BlA\string_B1}{}
\markboth{\textcolor{darkblue}{\textbf{\ipa{bɑ˩lɑ˩}}}}{}
\textcolor{teal}{\mytextsc{nom}} \hspace{4pt} Ton~: L.
\ding{202} 
Chemise, veste; vêtement.
\textcolor{brown}{\zh{上衣,衣服。}}


\begin{exe}
\sn \textcolor{darkblue}{\textbf{\ipa{ɣɯ˩-bɑ˩lɑ˥ (+ɲi˩)}}}
\trans veste de cuir
\trans \textit{\textcolor{brown}{\zh{皮衣}}}
\end{exe}
\ding{203} 
Placenta.
\textcolor{brown}{\zh{胎盘、衣胞。}}


 \mytextsc{clf}~: \textcolor{darkblue}{\textbf{\ipa{ɭɯ˧}}} \textcolor{darkblue}{\textbf{\ipa{ɭɯ˧}}} 

\paragraph{\hspace{-0.5cm} {\Large \textcolor{darkblue}{\textbf{\ipa{bɑ˩˥}}}}} \hypertarget{bA\string_B\string_T1}{}
\markboth{\textcolor{darkblue}{\textbf{\ipa{bɑ˩˥}}}}{}
\textcolor{teal}{\mytextsc{particule}} \textcolor{teal}{\mytextsc{de}} \textcolor{teal}{\mytextsc{discours}} \hspace{4pt} Ton~: L?.

Particule finale affirmative: “...n'est-ce pas”.
\textcolor{brown}{\zh{句尾助词,表示肯定:“……是吧。”。}}




\paragraph{\hspace{-0.5cm} {\Large \textcolor{darkblue}{\textbf{\ipa{bæ˧}}} \textsubscript{1}}} \hypertarget{b\{\string_M1}{}
\markboth{\textcolor{darkblue}{\textbf{\ipa{bæ˧}}} \textsubscript{1}}{}
\textcolor{teal}{\mytextsc{adjectif}} \hspace{4pt} Ton~: M.

Stupide, sot, idiot.
\textcolor{brown}{\zh{傻、笨、蠢。}}


\begin{exe}
\sn \textcolor{darkblue}{\textbf{\ipa{bæ˧-hĩ˧}}}
\trans \mytextsc{rel}
\trans \textit{\textcolor{brown}{\zh{傻的}}}
\end{exe}


\paragraph{\hspace{-0.5cm} {\Large \textcolor{darkblue}{\textbf{\ipa{bæ˧}}} \textsubscript{2}}} \hypertarget{b\{\string_M2}{}
\markboth{\textcolor{darkblue}{\textbf{\ipa{bæ˧}}} \textsubscript{2}}{}
\textcolor{teal}{\mytextsc{verbe}} \hspace{4pt} Ton~: M intrans.

Laisser tomber, abandonner, ne pas s'entêter.
\textcolor{brown}{\zh{放弃。}}


\begin{exe}
\sn \textcolor{darkblue}{\textbf{\ipa{no˧ | bæ˧-ze˩!}}}
\trans Tu dois être bien déçu; allez, laisse tomber! (Ce qu'on dit à quelqu'un qui a échoué après de multiples tentatives.)
\trans \textit{\textcolor{brown}{\zh{你算了吧!(感叹)}}}
\end{exe}
\begin{exe}
\sn \textcolor{darkblue}{\textbf{\ipa{bæ˧-ze˩ mæ˩!}}}
\trans Laisse tomber, allez! (nuance d'évidence)
\trans \textit{\textcolor{brown}{\zh{算了嘛!(感叹)}}}
\end{exe}


\paragraph{\hspace{-0.5cm} {\Large \textcolor{darkblue}{\textbf{\ipa{bæ˧\textsubscript{a}}}}}} \hypertarget{b\{\string_Ma1}{}
\markboth{\textcolor{darkblue}{\textbf{\ipa{bæ˧\textsubscript{a}}}}}{}
\textcolor{teal}{\mytextsc{classificateur}} \hspace{4pt} Ton~: M\textsubscript{a}.

Classificateur des espèces/sortes de choses. Proche de \textcolor{darkblue}{\textbf{\ipa{/ʁo˩b/}}} ``sorte, variété". S'emploie dans la construction ``c'est la même chose".
\textcolor{brown}{\zh{量词:东西(一样)。}}


\begin{exe}
\sn \textcolor{darkblue}{\textbf{\ipa{ɖɯ˧-bæ˧-lɑ˧ ɲi˥!}}}
\trans c'est pareil!/c'est la même chose!
\trans \textit{\textcolor{brown}{\zh{是一样的!}}}
\end{exe}
\begin{exe}
\sn \textcolor{darkblue}{\textbf{\ipa{ʝi˧kʰv̩˥-dʑo˩, | ɲi˧-bæ˧ | ʐwɤ˩-tʰɑ˩˥! | ʝi˧kʰv̩˥-dʑo˩, | ɖɯ˧-bæ˧-lɑ˧ ʐwɤ˧-tʰɑ˥!}}}
\trans Il y en a certaines (=des expressions/des combinaisons de mots), on peut les prononcer de deux façons/elles ont deux schémas tonals différents! Il y en a certaines, il n'y a qu'une façon de les dire/il n'y a qu'une sorte (de réalisation tonale possible)! (commentaire au sujet d'expressions qui ont deux variantes tonales)
\trans \textit{\textcolor{brown}{\zh{有些(词组)有两种说法,有些只有一种说法!(情景:讨论的是一些有两种不同变调发音的词组,发音合作人确定:确实有些有两种不同的变调,而有些只有一种声调模型。)}}}
\end{exe}
\begin{exe}
\sn \textcolor{darkblue}{\textbf{\ipa{ɲi˧-bæ˧-ɳɯ˧ | ɖɯ˧-bæ˧ ʝi˧}}}
\trans confondre deux choses (ex.: confondre deux sons, et les noter de la même façon, alors qu'ils s'opposent entre eux)
\trans \textit{\textcolor{brown}{\zh{两者混淆,例如把两个音(两个不同的音位)写成一样,混淆两者}}}
\end{exe}


\paragraph{\hspace{-0.5cm} {\Large \textcolor{darkblue}{\textbf{\ipa{bæ˧bv̩˥}}}}} \hypertarget{b\{\string_Mbv\string_=\string_T1}{}
\markboth{\textcolor{darkblue}{\textbf{\ipa{bæ˧bv̩˥}}}}{}
\textcolor{teal}{\mytextsc{nom}} \hspace{4pt} Ton~: H\#.

Goret, porcelet, cochonnet, petit cochon.
\textcolor{brown}{\zh{猪崽。}}


\begin{exe}
\sn \textcolor{darkblue}{\textbf{\ipa{bæ˧bv̩˥-zo˩}}}
\trans même sens: goret
\trans \textit{\textcolor{brown}{\zh{猪崽}}}
\end{exe}
 \mytextsc{clf}~: \textcolor{darkblue}{\textbf{\ipa{ɭɯ˧}}} 

\paragraph{\hspace{-0.5cm} {\Large \textcolor{darkblue}{\textbf{\ipa{bæ˧ɖæ˧}}}}} \hypertarget{b\{\string_Md`\{\string_M1}{}
\markboth{\textcolor{darkblue}{\textbf{\ipa{bæ˧ɖæ˧}}}}{}
\textcolor{teal}{\mytextsc{nom}} \hspace{4pt} Ton~: M.

Cordelette.
\textcolor{brown}{\zh{细绳。}}


 \mytextsc{clf}~: \textcolor{darkblue}{\textbf{\ipa{kʰɯ˩}}} 

\paragraph{\hspace{-0.5cm} {\Large \textcolor{darkblue}{\textbf{\ipa{bæ˧mi˧}}} \textsubscript{1}}} \hypertarget{b\{\string_Mmi\string_M1}{}
\markboth{\textcolor{darkblue}{\textbf{\ipa{bæ˧mi˧}}} \textsubscript{1}}{}
\textcolor{teal}{\mytextsc{nom}} \hspace{4pt} Ton~: M.
\ding{202} 
Canard (sans préciser le sexe: canard ou cane).
\textcolor{brown}{\zh{鸭子。}}


\ding{203} 
Cane.
\textcolor{brown}{\zh{母鸭子。}}


\begin{exe}
\sn \textcolor{darkblue}{\textbf{\ipa{bæ˧mi˧-bæ˧pʰv̩\#˥}}}
\trans cane et canard
\trans \textit{\textcolor{brown}{\zh{母鸭子与公鸭子}}}
\end{exe}
\begin{exe}
\sn \textcolor{darkblue}{\textbf{\ipa{bæ˧mi˧-bæ˧zo\#˥}}}
\trans cane et caneton
\trans \textit{\textcolor{brown}{\zh{母鸭与小鸭子}}}
\end{exe}
 \mytextsc{clf}~: \textcolor{darkblue}{\textbf{\ipa{mi˩}}} 

\paragraph{\hspace{-0.5cm} {\Large \textcolor{darkblue}{\textbf{\ipa{bæ˧mi˧}}} \textsubscript{2}}} \hypertarget{b\{\string_Mmi\string_M2}{}
\markboth{\textcolor{darkblue}{\textbf{\ipa{bæ˧mi˧}}} \textsubscript{2}}{}
\textcolor{teal}{\mytextsc{nom}} \hspace{4pt} Ton~: M.

Grosse corde.
\textcolor{brown}{\zh{粗绳索。}}


 \mytextsc{clf}~: \textcolor{darkblue}{\textbf{\ipa{kʰɯ˩}}} 

\paragraph{\hspace{-0.5cm} {\Large \textcolor{darkblue}{\textbf{\ipa{bæ˧mi˧-pʰv̩\#˥}}}}} \hypertarget{b\{\string_Mmi\string_M-p\string_hv\string_=\#\string_T1}{}
\markboth{\textcolor{darkblue}{\textbf{\ipa{bæ˧mi˧-pʰv̩\#˥}}}}{}
\textcolor{teal}{\mytextsc{nom}} \hspace{4pt} Ton~: \#H.

Canard (mâle).
\textcolor{brown}{\zh{公鸭子。}}


\begin{exe}
\sn \textcolor{darkblue}{\textbf{\ipa{bæ˧mi˧-pʰv̩˧ tʰv̩˧-mi˧˥}}}
\trans \mytextsc{n}+\mytextsc{dem}+\mytextsc{clf}
\trans \textit{\textcolor{brown}{\zh{这只公鸭子}}}
\end{exe}
 \mytextsc{clf}~: \textcolor{darkblue}{\textbf{\ipa{mi˩}}} \textit{Voir~:} \hyperlink{b\{\string_Mp\string_hv\string_=\#\string_T1}{\textcolor{darkblue}{\textbf{\ipa{bæ˧pʰv̩\#˥}}}} 

\paragraph{\hspace{-0.5cm} {\Large \textcolor{darkblue}{\textbf{\ipa{bæ˧pʰv̩\#˥}}}}} \hypertarget{b\{\string_Mp\string_hv\string_=\#\string_T1}{}
\markboth{\textcolor{darkblue}{\textbf{\ipa{bæ˧pʰv̩\#˥}}}}{}
\textcolor{teal}{\mytextsc{nom}} \hspace{4pt} Ton~: \#H.

Canard (mâle).
\textcolor{brown}{\zh{公鸭子。}}


\begin{exe}
\sn \textcolor{darkblue}{\textbf{\ipa{bæ˧pʰv̩˧ tʰv̩˧-mi˧˥ / bæ˧pʰv̩˧ tʰv̩˧-mi˥\#}}}
\trans \mytextsc{n}+\mytextsc{dem}+\mytextsc{clf}
\trans \textit{\textcolor{brown}{\zh{这个公鸭子}}}
\end{exe}
\begin{exe}
\sn \textcolor{darkblue}{\textbf{\ipa{bæ˧pʰv̩˧-bæ˧mi\#˥}}}
\trans canard et cane
\trans \textit{\textcolor{brown}{\zh{公鸭子与母鸭子}}}
\end{exe}
 \mytextsc{clf}~: \textcolor{darkblue}{\textbf{\ipa{mi˩}}} \textit{Voir~:} \hyperlink{b\{\string_Mmi\string_M-p\string_hv\string_=\#\string_T1}{\textcolor{darkblue}{\textbf{\ipa{bæ˧mi˧-pʰv̩\#˥}}}} 

\paragraph{\hspace{-0.5cm} {\Large \textcolor{darkblue}{\textbf{\ipa{bæ˧ʁwɤ˧}}}}} \hypertarget{b\{\string_MRw7\string_M1}{}
\markboth{\textcolor{darkblue}{\textbf{\ipa{bæ˧ʁwɤ˧}}}}{}
\textcolor{teal}{\mytextsc{nom}} \hspace{4pt} Ton~: M.

Un village proche des Source Chaudes.
\textcolor{brown}{\zh{温泉乡的一个村落。}}


\begin{exe}
\sn \textcolor{darkblue}{\textbf{\ipa{bæ˧ʁwɤ˧-ʁwɤ˧}}}
\trans même sens: le village de \textcolor{darkblue}{\textbf{\ipa{/bæ˧ʁwɤ˧/}}}
\trans \textit{\textcolor{brown}{\zh{同上:\textcolor{darkblue}{\textbf{\ipa{/bæ˧ʁwɤ˧/}}}村}}}
\end{exe}
\begin{exe}
\sn \textcolor{darkblue}{\textbf{\ipa{ə˧go˧-ʁwɤ˧, | ʁwɤ˧lɑ˩-bi˩, | bæ˧ʁwɤ˧, | tʰo˧tsʰe\#˥, | pi˧tsʰe˩-di˩, | pɤ˧dʑɤ˩-di˩, | ʁwɤ˧tv̩˧}}}
\trans Villages au sortir de la plaine de Yongning; les deux premiers comportent une population na; le troisième est un village na; les suivants sont essentiellement des villages pumi/prinmi.
\trans \textit{\textcolor{brown}{\zh{永宁背向泸沽湖方向经过的村落。前两个村落拥有相当大的摩梭人口比例,第三个村落是摩梭村,最后一个是普米村。}}}
\end{exe}
\begin{exe}
\sn \textcolor{darkblue}{\textbf{\ipa{bæ˧ʁwɤ˧: | nɑ˩˥!}}}
\trans \textcolor{darkblue}{\textbf{\ipa{/bæ˧ʁwɤ˧/}}}, c'est un village na!
\trans \textit{\textcolor{brown}{\zh{\textcolor{darkblue}{\textbf{\ipa{/bæ˧ʁwɤ˧/}}}是一个摩梭人村落!}}}
\end{exe}


\paragraph{\hspace{-0.5cm} {\Large \textcolor{darkblue}{\textbf{\ipa{bæ˧zo\#˥}}}}} \hypertarget{b\{\string_Mzo\#\string_T1}{}
\markboth{\textcolor{darkblue}{\textbf{\ipa{bæ˧zo\#˥}}}}{}
\textcolor{teal}{\mytextsc{nom}} \hspace{4pt} Ton~: \#H.

Caneton, petit canard.
\textcolor{brown}{\zh{小鸭子。}}


\begin{exe}
\sn \textcolor{darkblue}{\textbf{\ipa{bæ˧zo˧ tʰv̩˧-ɭɯ\#˥}}}
\trans \mytextsc{n}+\mytextsc{dem}+\mytextsc{clf}
\trans \textit{\textcolor{brown}{\zh{这只小鸭子}}}
\end{exe}
\begin{exe}
\sn \textcolor{darkblue}{\textbf{\ipa{bæ˧zo˧-bæ˧mi\#˥}}}
\trans caneton et cane
\trans \textit{\textcolor{brown}{\zh{小鸭子与母鸭}}}
\end{exe}
 \mytextsc{clf}~: \textcolor{darkblue}{\textbf{\ipa{ɭɯ˧}}} 

\paragraph{\hspace{-0.5cm} {\Large \textcolor{darkblue}{\textbf{\ipa{bæ˩}}} \textsubscript{1}}} \hypertarget{b\{\string_B1}{}
\markboth{\textcolor{darkblue}{\textbf{\ipa{bæ˩}}} \textsubscript{1}}{}
\textcolor{teal}{\mytextsc{nom}} \hspace{4pt} Ton~: L.

Corde.
\textcolor{brown}{\zh{绳子。}}


\begin{exe}
\sn \textcolor{darkblue}{\textbf{\ipa{bæ˩ ʈʂʰɯ˩-kʰɯ˥}}}
\trans \mytextsc{n}+\mytextsc{dem}+\mytextsc{clf}
\trans \textit{\textcolor{brown}{\zh{这条绳子}}}
\end{exe}
 \mytextsc{clf}~: \textcolor{darkblue}{\textbf{\ipa{ʈʰɤ˥}}} \textcolor{darkblue}{\textbf{\ipa{ɖæ˩}}} \textcolor{darkblue}{\textbf{\ipa{kʰɯ˩}}} 

\paragraph{\hspace{-0.5cm} {\Large \textcolor{darkblue}{\textbf{\ipa{bæ˩}}} \textsubscript{2}}} \hypertarget{b\{\string_B2}{}
\markboth{\textcolor{darkblue}{\textbf{\ipa{bæ˩}}} \textsubscript{2}}{}
\textcolor{teal}{\mytextsc{verbe}} \hspace{4pt} Ton~: L.

Pus.
\textcolor{brown}{\zh{脓。}}


\begin{exe}
\sn \textcolor{darkblue}{\textbf{\ipa{bæ˩ bæ˧-ze˩}}}
\trans la blessure donne du pus, il y a du pus
\trans \textit{\textcolor{brown}{\zh{伤口在化脓}}}
\end{exe}
\begin{exe}
\sn \textcolor{darkblue}{\textbf{\ipa{bæ˩˥ | le˧-bæ˩-ze˩}}}
\trans la blessure donne du pus, il y a du pus
\trans \textit{\textcolor{brown}{\zh{伤口在化脓}}}
\end{exe}
\textit{Voir~:} \hyperlink{b\{\string_B\string_T1}{\textcolor{darkblue}{\textbf{\ipa{bæ˩˥}}}} 

\paragraph{\hspace{-0.5cm} {\Large \textcolor{darkblue}{\textbf{\ipa{bæ˩\textsubscript{a}}}} \textsubscript{1}}} \hypertarget{b\{\string_Ba1}{}
\markboth{\textcolor{darkblue}{\textbf{\ipa{bæ˩\textsubscript{a}}}} \textsubscript{1}}{}
\textcolor{teal}{\mytextsc{verbe}} \hspace{4pt} Ton~: L\textsubscript{a}.

Balayer.
\textcolor{brown}{\zh{扫。}}


\begin{exe}
\sn \textcolor{darkblue}{\textbf{\ipa{ɖæ˩ bæ˧}}}
\trans balayer les saletés, balayer le sol
\trans \textit{\textcolor{brown}{\zh{扫地}}}
\end{exe}
\begin{exe}
\sn \textcolor{darkblue}{\textbf{\ipa{le˧-bæ˧\textasciitilde{}bæ˥}}}
\trans \mytextsc{accomp} \mytextsc{red}
\trans \textit{\textcolor{brown}{\zh{扫一扫}}}
\end{exe}
\begin{exe}
\sn \textcolor{darkblue}{\textbf{\ipa{ɖʐɤ˩ bæ˩˥}}}
\trans balayer l'escalier
\trans \textit{\textcolor{brown}{\zh{扫楼梯}}}
\end{exe}
\begin{exe}
\sn \textcolor{darkblue}{\textbf{\ipa{njɤ˧ | ɖʐɤ˩ bæ˩-zo˩-ho˥.}}}
\trans Il va falloir que je balaie l'escalier!
\trans \textit{\textcolor{brown}{\zh{我要扫楼梯了!}}}
\end{exe}
\begin{exe}
\sn \textcolor{darkblue}{\textbf{\ipa{gi˩ bæ˩˥}}}
\trans balayer le grenier à céréales
\trans \textit{\textcolor{brown}{\zh{扫仓廪}}}
\end{exe}
\begin{exe}
\sn \textcolor{darkblue}{\textbf{\ipa{njɤ˧ | gi˩ bæ˩-zo˩-ho˥.}}}
\trans Il va falloir que je balaie le grenier à céréales!
\trans \textit{\textcolor{brown}{\zh{我要扫仓廪了!}}}
\end{exe}


\paragraph{\hspace{-0.5cm} {\Large \textcolor{darkblue}{\textbf{\ipa{bæ˩\textsubscript{a}}}} \textsubscript{2}}} \hypertarget{b\{\string_Ba2}{}
\markboth{\textcolor{darkblue}{\textbf{\ipa{bæ˩\textsubscript{a}}}} \textsubscript{2}}{}
\textcolor{teal}{\mytextsc{verbe}} \hspace{4pt} Ton~: L\textsubscript{a}.

S'ouvrir (fleur), fleurir.
\textcolor{brown}{\zh{开花。}}


\begin{exe}
\sn \textcolor{darkblue}{\textbf{\ipa{bæ˩bæ˩ bæ˥-ze˩}}}
\trans La fleur a fleuri.
\trans \textit{\textcolor{brown}{\zh{花开了。}}}
\end{exe}


\paragraph{\hspace{-0.5cm} {\Large \textcolor{darkblue}{\textbf{\ipa{bæ˩\textsubscript{a}}}} \textsubscript{3}}} \hypertarget{b\{\string_Ba3}{}
\markboth{\textcolor{darkblue}{\textbf{\ipa{bæ˩\textsubscript{a}}}} \textsubscript{3}}{}
\textcolor{teal}{\mytextsc{classificateur}} \hspace{4pt} Ton~: L\textsubscript{a}.

Auto-classificateur des fleurs.
\textcolor{brown}{\zh{量词:花(一朵)。}}


\begin{exe}
\sn \textcolor{darkblue}{\textbf{\ipa{tʰv̩˧-bæ˥}}}
\trans \mytextsc{dem} \string_ (ton: H\# / H\$)
\trans \textit{\textcolor{brown}{\zh{\mytextsc{指示代词} \string_ :这朵(花)}}}
\end{exe}


\paragraph{\hspace{-0.5cm} {\Large \textcolor{darkblue}{\textbf{\ipa{bæ˩bæ˩}}} \textsubscript{1}}} \hypertarget{b\{\string_Bb\{\string_B1}{}
\markboth{\textcolor{darkblue}{\textbf{\ipa{bæ˩bæ˩}}} \textsubscript{1}}{}
\textcolor{teal}{\mytextsc{nom}} \hspace{4pt} Ton~: L.

Fleur.
\textcolor{brown}{\zh{花。}}


 \mytextsc{clf}~: \textcolor{darkblue}{\textbf{\ipa{bæ˩}}} \textit{Voir~:} \hyperlink{b\{\string_Bb\{\string_B2}{\textcolor{darkblue}{\textbf{\ipa{bæ˩bæ˩}}} \textsubscript{2}} 

\paragraph{\hspace{-0.5cm} {\Large \textcolor{darkblue}{\textbf{\ipa{bæ˩bæ˩}}} \textsubscript{2}}} \hypertarget{b\{\string_Bb\{\string_B2}{}
\markboth{\textcolor{darkblue}{\textbf{\ipa{bæ˩bæ˩}}} \textsubscript{2}}{}
\textcolor{teal}{\mytextsc{adjectif}} \hspace{4pt} Ton~: L.

Bariolé, tacheté, moucheté (ex.: un oeuf moucheté, un oiseau au pelage moucheté, une pierre ayant plusieurs couleurs).
\textcolor{brown}{\zh{花的(蛋、石头、鸟)。}}


\begin{exe}
\sn \textcolor{darkblue}{\textbf{\ipa{bæ˩bæ˩ tʰi˩-di˥}}}
\trans même sens: bariolé, tacheté (par ex.: oeuf, oiseau, pierre)
\trans \textit{\textcolor{brown}{\zh{花的,有花纹}}}
\end{exe}
\textit{Voir~:} \hyperlink{b\{\string_Bb\{\string_B1}{\textcolor{darkblue}{\textbf{\ipa{bæ˩bæ˩}}} \textsubscript{1}} 

\paragraph{\hspace{-0.5cm} {\Large \textcolor{darkblue}{\textbf{\ipa{bæ˩dʑɯ˥}}}}} \hypertarget{b\{\string_Bdz£M\string_T1}{}
\markboth{\textcolor{darkblue}{\textbf{\ipa{bæ˩dʑɯ˥}}}}{}
\textcolor{teal}{\mytextsc{nom}} \hspace{4pt} Ton~: LH.

Récolte; plantes que l'on a semées.
\textcolor{brown}{\zh{庄稼。}}


\begin{exe}
\sn \textcolor{darkblue}{\textbf{\ipa{bæ˩dʑɯ˥ | mɤ˧-dʑɤ˩!}}}
\trans La récolte n'est pas bonne!
\trans \textit{\textcolor{brown}{\zh{收成不好!}}}
\end{exe}
\begin{exe}
\sn \textcolor{darkblue}{\textbf{\ipa{bæ˩dʑɯ˧ | tv̩˧-bæ˩ le˩-mv̩˩-kʰɯ˩!}}}
\trans Puissent mille récoltes venir à maturité! (Bénédiction qu'on dit aux aînés lors de cérémonies: par exemple lors du rite de passage à l'âge adulte)
\trans \textit{\textcolor{brown}{\zh{祝:一千棵庄稼成熟!(成年礼、过年等节庆时的祝福用语,晚辈对长辈的祝福)}}}
\end{exe}
 \mytextsc{clf}~: \textcolor{darkblue}{\textbf{\ipa{bæ˩}}} 

\paragraph{\hspace{-0.5cm} {\Large \textcolor{darkblue}{\textbf{\ipa{bæ˩-lɑ˩\textasciitilde{}lɑ˥}}}}} \hypertarget{b\{\string_B-lA\string_B~lA\string_T1}{}
\markboth{\textcolor{darkblue}{\textbf{\ipa{bæ˩-lɑ˩\textasciitilde{}lɑ˥}}}}{}
\textcolor{teal}{\mytextsc{adjectif}} \hspace{4pt} Ton~: L.

Flasque, sans consistance.
\textcolor{brown}{\zh{软,柔软、软塌塌。}}




\paragraph{\hspace{-0.5cm} {\Large \textcolor{darkblue}{\textbf{\ipa{bæ˩-ljɤ˧\textasciitilde{}ljɤ˧}}}}} \hypertarget{b\{\string_B-lj7\string_M~lj7\string_M1}{}
\markboth{\textcolor{darkblue}{\textbf{\ipa{bæ˩-ljɤ˧\textasciitilde{}ljɤ˧}}}}{}
\textcolor{teal}{\mytextsc{nom}} \hspace{4pt} Ton~: L-.

Pomme de pin, fruit du sapin.
\textcolor{brown}{\zh{杉树果。}}


 \mytextsc{clf}~: \textcolor{darkblue}{\textbf{\ipa{ɭɯ˧}}} 

\paragraph{\hspace{-0.5cm} {\Large \textcolor{darkblue}{\textbf{\ipa{bæ˩pʰv̩˥}}}}} \hypertarget{b\{\string_Bp\string_hv\string_=\string_T1}{}
\markboth{\textcolor{darkblue}{\textbf{\ipa{bæ˩pʰv̩˥}}}}{}
\textcolor{teal}{\mytextsc{nom}} \hspace{4pt} Ton~: L+H\#.

Chrysanthème couronné, chrysanthème des jardins, chrysanthème comestible ou chrysanthème à couronnes, \textit{Glebionis coronaria}.
\textcolor{brown}{\zh{茼蒿。}}


\begin{exe}
\sn \textcolor{darkblue}{\textbf{\ipa{bæ˩pʰv̩˥-bv̩˩ | bæ˩bæ˩˥}}}
\trans fleur de chrysanthème couronné
\trans \textit{\textcolor{brown}{\zh{茼蒿的顶花}}}
\end{exe}
\begin{exe}
\sn \textcolor{darkblue}{\textbf{\ipa{bæ˩pʰv̩˥-bæ˩bæ˩}}}
\trans fleur de chrysanthème couronné
\trans \textit{\textcolor{brown}{\zh{茼蒿顶花}}}
\end{exe}
 \mytextsc{clf}~: \textcolor{darkblue}{\textbf{\ipa{po˧}}} 

\paragraph{\hspace{-0.5cm} {\Large \textcolor{darkblue}{\textbf{\ipa{bæ˩-ʁwæ˩\textasciitilde{}ʁwæ˥}}}}} \hypertarget{b\{\string_B-Rw\{\string_B~Rw\{\string_T1}{}
\markboth{\textcolor{darkblue}{\textbf{\ipa{bæ˩-ʁwæ˩\textasciitilde{}ʁwæ˥}}}}{}
\textcolor{teal}{\mytextsc{adjectif}} \hspace{4pt} Ton~: L.

Relâché.
\textcolor{brown}{\zh{松。}}


\begin{exe}
\sn \textcolor{darkblue}{\textbf{\ipa{ʈʂʰɯ˧ | ɖwæ˧˥ | bæ˩ʁwæ˩\textasciitilde{}ʁwæ˥-ʝi˩!}}}
\trans C'est tout relâché, ce n'est pas bien serré! (Au sujet d'une charge sur le dos d'un mulet)
\trans \textit{\textcolor{brown}{\zh{(驮在马上面的货物没系好)松动了!}}}
\end{exe}


\paragraph{\hspace{-0.5cm} {\Large \textcolor{darkblue}{\textbf{\ipa{bæ˩ʈʂo˥}}}}} \hypertarget{b\{\string_Bt`s`o\string_T1}{}
\markboth{\textcolor{darkblue}{\textbf{\ipa{bæ˩ʈʂo˥}}}}{}
\textcolor{teal}{\mytextsc{nom}} \hspace{4pt} Ton~: LH.

Balai.
\textcolor{brown}{\zh{扫帚。}}


 \mytextsc{clf}~: \textcolor{darkblue}{\textbf{\ipa{nɑ˧}}} 

\paragraph{\hspace{-0.5cm} {\Large \textcolor{darkblue}{\textbf{\ipa{bæ˩ʈʂwæ˩}}}}} \hypertarget{b\{\string_Bt`s`w\{\string_B1}{}
\markboth{\textcolor{darkblue}{\textbf{\ipa{bæ˩ʈʂwæ˩}}}}{}
\textcolor{teal}{\mytextsc{nom}} \hspace{4pt} Ton~: L.

Rênes.
\textcolor{brown}{\zh{缰绳。}}


\begin{exe}
\sn \textcolor{darkblue}{\textbf{\ipa{ʐwæ˧-bæ˥ʈʂwæ˩}}}
\trans rênes du cheval
\trans \textit{\textcolor{brown}{\zh{马缰绳}}}
\end{exe}
 \mytextsc{clf}~: \textcolor{darkblue}{\textbf{\ipa{kʰɯ˩}}} 

\paragraph{\hspace{-0.5cm} {\Large \textcolor{darkblue}{\textbf{\ipa{bæ˧˥}}}}} \hypertarget{b\{\string_M\string_T1}{}
\markboth{\textcolor{darkblue}{\textbf{\ipa{bæ˧˥}}}}{}
\textcolor{teal}{\mytextsc{verbe}} \hspace{4pt} Ton~: MH.

Courir.
\textcolor{brown}{\zh{跑。}}


\begin{exe}
\sn \textcolor{darkblue}{\textbf{\ipa{le˧-bæ˧-ze˥}}}
\trans \mytextsc{accomp} \string_ \mytextsc{pfv}
\trans \textit{\textcolor{brown}{\zh{跑了}}}
\end{exe}


\paragraph{\hspace{-0.5cm} {\Large \textcolor{darkblue}{\textbf{\ipa{bæ˩˥}}}}} \hypertarget{b\{\string_B\string_T1}{}
\markboth{\textcolor{darkblue}{\textbf{\ipa{bæ˩˥}}}}{}
\textcolor{teal}{\mytextsc{nom}} \hspace{4pt} Ton~: LH.

Pus.
\textcolor{brown}{\zh{脓。}}


\begin{exe}
\sn \textcolor{darkblue}{\textbf{\ipa{bæ˩ bæ˧-ze˩}}}
\trans la blessure donne du pus, il y a du pus
\trans \textit{\textcolor{brown}{\zh{伤口在化脓}}}
\end{exe}
\begin{exe}
\sn \textcolor{darkblue}{\textbf{\ipa{bæ˩˥ | le˧-bæ˩-ze˩}}}
\trans la blessure donne du pus, il y a du pus
\trans \textit{\textcolor{brown}{\zh{伤口在化脓}}}
\end{exe}
\begin{exe}
\sn \textcolor{darkblue}{\textbf{\ipa{bæ˩˥ | le˧-bæ˩-ze˩}}}
\trans yyyy reporter : verbe, ton ˩
\end{exe}
 \mytextsc{clf}~: \textcolor{darkblue}{\textbf{\ipa{ʈʰɤ˥}}} \textit{Voir~:} \hyperlink{b\{\string_B2}{\textcolor{darkblue}{\textbf{\ipa{bæ˩}}} \textsubscript{2}} 

\paragraph{\hspace{-0.5cm} {\Large \textcolor{darkblue}{\textbf{\ipa{bæ˩˧}}}}} \hypertarget{b\{\string_B\string_M1}{}
\markboth{\textcolor{darkblue}{\textbf{\ipa{bæ˩˧}}}}{}
\textcolor{teal}{\mytextsc{nom}} \hspace{4pt} Ton~: LM.

Récolte; plantes que l'on a semées.
\textcolor{brown}{\zh{庄稼。}}


\begin{exe}
\sn \textcolor{darkblue}{\textbf{\ipa{bæ˩ ɲi˧}}}
\trans \mytextsc{cop}
\trans \textit{\textcolor{brown}{\zh{是庄稼}}}
\end{exe}
\begin{exe}
\sn \textcolor{darkblue}{\textbf{\ipa{ɖɯ˧-kʰv̩˧ ʈv̩˧-bæ˥ mv̩˩, | ɕi˧-kʰv̩˧ | le˧-mɤ˧-dzɯ˧!}}}
\trans ``Quand bien même on aurait fait une récolte fabuleuse, ça ne nous durerait pas éternellement: ça se rejoue chaque année!" Littéralement: ``si, une année, mille récoltes parvenaient à maturité, on n['en] mangerait pas [pour autant pendant] cent ans =on n'aurait pas à manger pour cent ans!" Le proverbe sert à se consoler d'une mauvaise récolte, qui va obliger à une année frugale: ``Si belle soit la récolte, elle n'aurait de toute façon pas duré éternellement; tout est à recommencer l'année suivante, voyons donc de l'avant!"
\trans \textit{\textcolor{brown}{\zh{“一年收千棵,不够吃百年!”(这个谚语,来慰藉收成不好的年份。)}}}
\end{exe}
 \mytextsc{clf}~: \textcolor{darkblue}{\textbf{\ipa{bæ˩}}} 

\paragraph{\hspace{-0.5cm} {\Large \textcolor{darkblue}{\textbf{\ipa{bɤ˥}}}}} \hypertarget{b7\string_T1}{}
\markboth{\textcolor{darkblue}{\textbf{\ipa{bɤ˥}}}}{}
\textcolor{teal}{\mytextsc{nom}} \hspace{4pt} Ton~: \#H.

Pumi (Prinmi) (groupe ethnique).
\textcolor{brown}{\zh{普米族。}}


\begin{exe}
\sn \textcolor{darkblue}{\textbf{\ipa{bɤ˧-ʐwɤ˧ so˥}}}
\trans apprendre la langue prinmi
\trans \textit{\textcolor{brown}{\zh{学普米语}}}
\end{exe}
 \mytextsc{clf}~: \textcolor{darkblue}{\textbf{\ipa{v̩˧}}} 

\paragraph{\hspace{-0.5cm} {\Large \textcolor{darkblue}{\textbf{\ipa{bɤ˧dzi˩}}}}} \hypertarget{b7\string_Mdzi\string_B1}{}
\markboth{\textcolor{darkblue}{\textbf{\ipa{bɤ˧dzi˩}}}}{}
\textcolor{teal}{\mytextsc{nom}} \hspace{4pt} Ton~: L\#.

Un des villages de la plaine de Yongning.
\textcolor{brown}{\zh{八珠(永宁的一个村落)。}}


\begin{exe}
\sn \textcolor{darkblue}{\textbf{\ipa{ɖæ˩ʂɯ\#˥, | ʈʂo˧ʂɯ\#˥, | bɤ˩tɕʰɯ˩˥, | dɑ˧pʰo˥, | bɤ˧dzi˩, | dze˧bo˧}}}
\trans les six villages de la plaine de Yongning, dans l'ordre, qui prend comme point d'origine le village le plus proche du Lac
\trans \textit{\textcolor{brown}{\zh{永宁坝的六个村落,按传统排序:从距离泸沽湖最近的村落说起。}}}
\end{exe}


\paragraph{\hspace{-0.5cm} {\Large \textcolor{darkblue}{\textbf{\ipa{bɤ˧kɯ˧}}}}} \hypertarget{b7\string_MkM\string_M1}{}
\markboth{\textcolor{darkblue}{\textbf{\ipa{bɤ˧kɯ˧}}}}{}
\textcolor{teal}{\mytextsc{nom}} \hspace{4pt} Ton~: M.

Vannerie: tamis, crible, en forme de gourde; on y met des légumes, des choses à porter. Cette vannerie est commode à porter.
\textcolor{brown}{\zh{筛子。}}


 \mytextsc{clf}~: \textcolor{darkblue}{\textbf{\ipa{nɑ˧}}} 

\paragraph{\hspace{-0.5cm} {\Large \textcolor{darkblue}{\textbf{\ipa{bɤ˧mi\#˥}}}}} \hypertarget{b7\string_Mmi\#\string_T1}{}
\markboth{\textcolor{darkblue}{\textbf{\ipa{bɤ˧mi\#˥}}}}{}
\textcolor{teal}{\mytextsc{nom}} \hspace{4pt} Ton~: \#H.

Femme pumi.
\textcolor{brown}{\zh{普米族女人。}}


 \mytextsc{clf}~: \textcolor{darkblue}{\textbf{\ipa{v̩˧}}} 

\paragraph{\hspace{-0.5cm} {\Large \textcolor{darkblue}{\textbf{\ipa{bɤ˧mi˥-ʂe˩}}}}} \hypertarget{b7\string_Mmi\string_T-s`e\string_B1}{}
\markboth{\textcolor{darkblue}{\textbf{\ipa{bɤ˧mi˥-ʂe˩}}}}{}
\textcolor{teal}{\mytextsc{nom}} \hspace{4pt} Ton~: H\#-.

Cupronickel: alliage cuivre-nickel.
\textcolor{brown}{\zh{白铜。}}




\paragraph{\hspace{-0.5cm} {\Large \textcolor{darkblue}{\textbf{\ipa{bɤ˧ʂɯ˩}}}}} \hypertarget{b7\string_Ms`M\string_B1}{}
\markboth{\textcolor{darkblue}{\textbf{\ipa{bɤ˧ʂɯ˩}}}}{}
\textcolor{teal}{\mytextsc{nom}} \hspace{4pt} Ton~: L\#.

Nom d'un village de la plaine de Lijiang, d'où venaient de nombreux marchands, d'où le fait que son nom soit connu à Yongning. En naxi: \textcolor{darkblue}{\textbf{\ipa{/bɤ˧ʂɯ˩/}}}.
\textcolor{brown}{\zh{白沙(丽江坝子里的一个村落)。}}




\paragraph{\hspace{-0.5cm} {\Large \textcolor{darkblue}{\textbf{\ipa{bɤ˧tʰv̩˩}}}}} \hypertarget{b7\string_Mt\string_hv\string_=\string_B1}{}
\markboth{\textcolor{darkblue}{\textbf{\ipa{bɤ˧tʰv̩˩}}}}{}
\textcolor{teal}{\mytextsc{nom}} \hspace{4pt} Ton~: L\#.

Empreintes, traces de pas, traces de pied.
\textcolor{brown}{\zh{脚印。}}


\begin{exe}
\sn \textcolor{darkblue}{\textbf{\ipa{hĩ˧-bɤ˧tʰv̩˥}}}
\trans empreintes (de pied) d'homme
\trans \textit{\textcolor{brown}{\zh{人的脚印}}}
\end{exe}
\begin{exe}
\sn \textcolor{darkblue}{\textbf{\ipa{kʰv̩˩mi˩-bɤ˩tʰv̩˥}}}
\trans empreintes de (pattes de) chien
\trans \textit{\textcolor{brown}{\zh{狗爪印}}}
\end{exe}
 \mytextsc{clf}~: \textcolor{darkblue}{\textbf{\ipa{tʰv̩˧˥}}} 

\paragraph{\hspace{-0.5cm} {\Large \textcolor{darkblue}{\textbf{\ipa{bɤ˧tsʰo˧gv̩˥}}}}} \hypertarget{b7\string_Mts\string_ho\string_Mgv\string_=\string_T1}{}
\markboth{\textcolor{darkblue}{\textbf{\ipa{bɤ˧tsʰo˧gv̩˥}}}}{}
\textcolor{teal}{\mytextsc{nom}} \hspace{4pt} Ton~: H\#.

Un des villages de la plaine de Yongning; lieu de l'actuel marché; terme également employé pour désigner le lieu d'habitation des artisans Naxi).
\textcolor{brown}{\zh{巴搓古(永宁的一个村落)。}}


\begin{exe}
\sn \textcolor{darkblue}{\textbf{\ipa{bɤ˧tsʰo˧gv̩˥-hĩ˩}}}
\trans quelqu'un de Bacuogu
\trans \textit{\textcolor{brown}{\zh{从巴搓古来的一个人}}}
\end{exe}
\begin{exe}
\sn \textcolor{darkblue}{\textbf{\ipa{dʑɤ˩bv̩˧kɤ˧-sɑ˥ʁwɤ˩, | hi˩ʁwɤ˩-lo˥, | æ˩mi˧-ʁwɤ\#˥, | lɑ˧lo˧-ʁwɤ˥, | lɑ˧ŋwɤ˧, | bɤ˧tsʰo˧gv̩˥, | ə˧lɑ˧-ʁwɤ\#˥, | gæ˧ɻæ˩, | qʰæ˧tɕʰi˧, | tʰo˧ʈɯ\#˥}}}
\trans les dix villages comptant traditionnellement comme faisant partie de Yongning
\trans \textit{\textcolor{brown}{\zh{摩梭传统地理概念中,属于永宁的十个村落}}}
\end{exe}


\paragraph{\hspace{-0.5cm} {\Large \textcolor{darkblue}{\textbf{\ipa{bɤ˧zo\#˥}}}}} \hypertarget{b7\string_Mzo\#\string_T1}{}
\markboth{\textcolor{darkblue}{\textbf{\ipa{bɤ˧zo\#˥}}}}{}
\textcolor{teal}{\mytextsc{nom}} \hspace{4pt} Ton~: \#H.

Homme pumi.
\textcolor{brown}{\zh{普米族男人。}}


 \mytextsc{clf}~: \textcolor{darkblue}{\textbf{\ipa{v̩˧}}} 

\paragraph{\hspace{-0.5cm} {\Large \textcolor{darkblue}{\textbf{\ipa{bɤ˩\textsubscript{a}}}} \textsubscript{1}}} \hypertarget{b7\string_Ba1}{}
\markboth{\textcolor{darkblue}{\textbf{\ipa{bɤ˩\textsubscript{a}}}} \textsubscript{1}}{}
\textcolor{teal}{\mytextsc{classificateur}} \hspace{4pt} Ton~: L\textsubscript{a}.

Classificateur des épis de maïs.
\textcolor{brown}{\zh{量词:玉米棒子(一根)。}}


\begin{exe}
\sn \textcolor{darkblue}{\textbf{\ipa{hɑ˧bɤ˥ | ɖɯ˧-bɤ˩}}}
\trans un épi de maïs
\trans \textit{\textcolor{brown}{\zh{一根玉米棒子}}}
\end{exe}
\begin{exe}
\sn \textcolor{darkblue}{\textbf{\ipa{tʰv̩˧-bɤ˥}}}
\trans \mytextsc{dem} \string_ (ton: H\# / H\$)
\trans \textit{\textcolor{brown}{\zh{\mytextsc{指示代词} \string_:那根(玉米棒子)}}}
\end{exe}


\paragraph{\hspace{-0.5cm} {\Large \textcolor{darkblue}{\textbf{\ipa{bɤ˩\textsubscript{a}}}} \textsubscript{2}}} \hypertarget{b7\string_Ba2}{}
\markboth{\textcolor{darkblue}{\textbf{\ipa{bɤ˩\textsubscript{a}}}} \textsubscript{2}}{}
\textcolor{teal}{\mytextsc{classificateur}} \hspace{4pt} Ton~: L\textsubscript{a}.

Classificateur des moitiés.
\textcolor{brown}{\zh{量词:半。}}


\begin{exe}
\sn \textcolor{darkblue}{\textbf{\ipa{ɖɯ˧-bɤ˩-lɑ˩ tʰv̩˩-sɯ˩! | ɖɯ˧-hu˧-ɻ̍˥!}}}
\trans Je n'ai fait que la moitié! Attends un peu! (Contexte: quelqu'un est en train de trier les vêtements, et signale qu'il n'a pas fini)
\trans \textit{\textcolor{brown}{\zh{我才干了一半!稍等!(情景:一个在收拾衣服,告诉对方:工作没完,还需要时间。)}}}
\end{exe}
\begin{exe}
\sn \textcolor{darkblue}{\textbf{\ipa{ɖɯ˧-bɤ˩-lɑ˩ tʰv̩˩-ze˩!}}}
\trans Tu es à mi-chemin! / Tu n'as parcouru que la moitié du chemin! (Réflexion de la consultante principale au sujet de mon apprentissage de la langue na. Cette formulation souligne le chemin qui reste à parcourir, mais l'emploi du perfectif apporte une note d'encouragement, là où l'exemple précédant insisterait essentiellement sur le chemin restant à parcourir: \textcolor{darkblue}{\textbf{\ipa{ɖɯ˧-bɤ˩-lɑ˩ tʰv̩˩-sɯ˩}}}.)
\trans \textit{\textcolor{brown}{\zh{你才到了一半!(合作人对调查者学摩梭话的评定)}}}
\end{exe}
\begin{exe}
\sn \textcolor{darkblue}{\textbf{\ipa{ʐæ˩ʂæ˥ | ʐwæ˩˥! | le˧-se˥, | ɖɯ˧-bɤ˩-qo˩-lɑ˩ tʰv̩˩-sɯ˩!}}}
\trans C'est bien loin! On a marché, et on n'est encore parvenu qu'à mi-chemin!
\trans \textit{\textcolor{brown}{\zh{真远!走啊走,才走了一半的路!}}}
\end{exe}
\begin{exe}
\sn \textcolor{darkblue}{\textbf{\ipa{ʐɤ˩mi˩˥ | ɖɯ˧-bɤ˩}}}
\trans la moitié du chemin
\trans \textit{\textcolor{brown}{\zh{半路}}}
\end{exe}
\begin{exe}
\sn \textcolor{darkblue}{\textbf{\ipa{ə˧mi˧! | wɤ˩˥ | ɖɯ˧-bɤ˩ dʑo˩-sɯ˩-wɤ˩!}}}
\trans Houlàà! (Que c'est loin!) Il reste encore la moitié du chemin (à parcourir)!
\trans \textit{\textcolor{brown}{\zh{啊呀嚒!还剩一半的路啊!}}}
\end{exe}


\paragraph{\hspace{-0.5cm} {\Large \textcolor{darkblue}{\textbf{\ipa{bɤ˩tɕʰɯ˩}}}}} \hypertarget{b7\string_Bts£\string_hM\string_B1}{}
\markboth{\textcolor{darkblue}{\textbf{\ipa{bɤ˩tɕʰɯ˩}}}}{}
\textcolor{teal}{\mytextsc{nom}} \hspace{4pt} Ton~: L.

Un des villages de la plaine de Yongning.
\textcolor{brown}{\zh{八七(永宁的一个村落)。}}


\begin{exe}
\sn \textcolor{darkblue}{\textbf{\ipa{ɖæ˩ʂɯ\#˥, | ʈʂo˧ʂɯ\#˥, | bɤ˩tɕʰɯ˩˥, | dɑ˧pʰo˥, | bɤ˧dzi˩, | dze˧bo˧}}}
\trans les six villages de la plaine de Yongning, dans l'ordre, qui prend comme point d'origine le village le plus proche du Lac
\trans \textit{\textcolor{brown}{\zh{永宁坝的六个村落,按传统排序:从距离泸沽湖最近的村落说起。}}}
\end{exe}


\paragraph{\hspace{-0.5cm} {\Large \textcolor{darkblue}{\textbf{\ipa{bɤ˧˥\textsubscript{a}}}}}} \hypertarget{b7\string_M\string_Ta1}{}
\markboth{\textcolor{darkblue}{\textbf{\ipa{bɤ˧˥\textsubscript{a}}}}}{}
\textcolor{teal}{\mytextsc{classificateur}} \hspace{4pt} Ton~: MH\textsubscript{a}.

Classificateur des fichus et foulards.
\textcolor{brown}{\zh{量词:头帕(一条)。}}




\paragraph{\hspace{-0.5cm} {\Large \textcolor{darkblue}{\textbf{\ipa{‑bi}}}}} \hypertarget{‑bi1}{}
\markboth{\textcolor{darkblue}{\textbf{\ipa{‑bi}}}}{}
\textcolor{teal}{\mytextsc{conjonction}} \hspace{4pt} Ton~: 0.

\mytextsc{adversatif}: bien que, même si.
\textcolor{brown}{\zh{虽然……。}}


\begin{exe}
\sn \textcolor{darkblue}{\textbf{\ipa{ʈʂʰɯ˧ | nɑ˩ ɲi˥-pi˩-bi˩-bi˩, | nɑ˩-ʐwɤ˧ | mɤ˧-kv̩˧˥!}}}
\trans Bien qu'elle/il soit Na, elle/il ne sait pas parler la langue na!
\trans \textit{\textcolor{brown}{\zh{他虽然是摩梭人但不会讲摩梭话。}}}
\end{exe}
\begin{exe}
\sn \textcolor{darkblue}{\textbf{\ipa{*ʈʂʰɯ˧ | nɑ˩ ɲi˥-bi˩, … / *ʈʂʰɯ˧ | nɑ˩ ɲi˥-bi˩-bi˩}}}
\trans phrase non acceptable; l'intention était de dire 'bien qu'il soit Na…'
\trans \textit{\textcolor{brown}{\zh{病句:不能这样说“他虽然是摩梭人……”}}}
\end{exe}


\paragraph{\hspace{-0.5cm} {\Large \textcolor{darkblue}{\textbf{\ipa{bi˥}}}}} \hypertarget{bi\string_T1}{}
\markboth{\textcolor{darkblue}{\textbf{\ipa{bi˥}}}}{}
\textcolor{teal}{\mytextsc{nom}} \hspace{4pt} Ton~: \#H.

Neige.
\textcolor{brown}{\zh{雪。}}


\begin{exe}
\sn \textcolor{darkblue}{\textbf{\ipa{bi˧ gi˧-ze˩}}}
\trans il neige
\trans \textit{\textcolor{brown}{\zh{下雪了}}}
\end{exe}
 \mytextsc{clf}~: \textcolor{darkblue}{\textbf{\ipa{ʁwɤ˧, etc}}} 

\paragraph{\hspace{-0.5cm} {\Large \textcolor{darkblue}{\textbf{\ipa{bi˥}}}}} \hypertarget{bi\string_T1}{}
\markboth{\textcolor{darkblue}{\textbf{\ipa{bi˥}}}}{}
\textcolor{teal}{\mytextsc{adjectif}} \hspace{4pt} Ton~: H.

Mince; peu profond.
\textcolor{brown}{\zh{薄,浅(水浅)。}}


\begin{exe}
\sn \textcolor{darkblue}{\textbf{\ipa{bi˧ | ʐwæ˩˥!}}}
\trans C'est très peu profond!
\trans \textit{\textcolor{brown}{\zh{很浅!}}}
\end{exe}
\begin{exe}
\sn \textcolor{darkblue}{\textbf{\ipa{dʑɯ˧ | ɖɯ˧-pi˧ bi˧˥}}}
\trans L'eau est assez peu profonde.
\trans \textit{\textcolor{brown}{\zh{水有点浅。}}}
\end{exe}
\begin{exe}
\sn \textcolor{darkblue}{\textbf{\ipa{dʑɯ˧ bi˧-hĩ˧, | mɤ˧-ɖwæ˩!}}}
\trans L'eau pas profonde, ça fait pas peur! / L'eau n'est pas profonde; tu vas quand même pas avoir peur!
\trans \textit{\textcolor{brown}{\zh{水很浅,不用怕!}}}
\end{exe}


\paragraph{\hspace{-0.5cm} {\Large \textcolor{darkblue}{\textbf{\ipa{‑bi˧}}}}} \hypertarget{‑bi\string_M1}{}
\markboth{\textcolor{darkblue}{\textbf{\ipa{‑bi˧}}}}{}
\textcolor{teal}{\mytextsc{suffixe}} \hspace{4pt} Ton~: M.

Futur immédiat.
\textcolor{brown}{\zh{要\mytextsc{近将来。}}}




\paragraph{\hspace{-0.5cm} {\Large \textcolor{darkblue}{\textbf{\ipa{bi˧}}} \textsubscript{2}}} \hypertarget{bi\string_M2}{}
\markboth{\textcolor{darkblue}{\textbf{\ipa{bi˧}}} \textsubscript{2}}{}
\textcolor{teal}{\mytextsc{nom}} \hspace{4pt} Ton~: M.

Le village; les gens du village, les voisins.
\textcolor{brown}{\zh{村落,邻居、村里的人们。}}




\paragraph{\hspace{-0.5cm} {\Large \textcolor{darkblue}{\textbf{\ipa{bi˧}}} \textsubscript{3}}} \hypertarget{bi\string_M3}{}
\markboth{\textcolor{darkblue}{\textbf{\ipa{bi˧}}} \textsubscript{3}}{}
\textcolor{teal}{\mytextsc{verbe}} \hspace{4pt} Ton~: M.

Oser.
\textcolor{brown}{\zh{敢。}}


\begin{exe}
\sn \textcolor{darkblue}{\textbf{\ipa{ʝi˧-mɤ˧-bi˧}}}
\trans ne pas oser faire
\trans \textit{\textcolor{brown}{\zh{不敢做}}}
\end{exe}


\paragraph{\hspace{-0.5cm} {\Large \textcolor{darkblue}{\textbf{\ipa{bi˧\textsubscript{c}}}} \textsubscript{1}}} \hypertarget{bi\string_Mc1}{}
\markboth{\textcolor{darkblue}{\textbf{\ipa{bi˧\textsubscript{c}}}} \textsubscript{1}}{}
\textcolor{teal}{\mytextsc{verbe}} \hspace{4pt} Ton~: M\textsubscript{c}.

Aller.
\textcolor{brown}{\zh{去。}}


\begin{exe}
\sn \textcolor{darkblue}{\textbf{\ipa{bi˧-tʰɑ˧!}}}
\trans \mytextsc{abilitive}: On peut y aller!
\trans \textit{\textcolor{brown}{\zh{可以去!}}}
\end{exe}
\begin{exe}
\sn \textcolor{darkblue}{\textbf{\ipa{bi˧-tʰɑ˧-ze˥!}}}
\trans \mytextsc{abilitive}+\mytextsc{pfv}: Ca y est, on peut y aller!
\trans \textit{\textcolor{brown}{\zh{可以去了!}}}
\end{exe}
\begin{exe}
\sn \textcolor{darkblue}{\textbf{\ipa{le˧-bi˩}}}
\trans retourner; s'en retourner
\trans \textit{\textcolor{brown}{\zh{回去,返回}}}
\end{exe}


\paragraph{\hspace{-0.5cm} {\Large \textcolor{darkblue}{\textbf{\ipa{bi˧bv̩˥}}}}} \hypertarget{bi\string_Mbv\string_=\string_T1}{}
\markboth{\textcolor{darkblue}{\textbf{\ipa{bi˧bv̩˥}}}}{}
\textcolor{teal}{\mytextsc{verbe}} \hspace{4pt} Ton~: H\#.

Couler à flots, couler en trombe.
\textcolor{brown}{\zh{流淌,冲下去,下泻,很快地流。}}


\begin{exe}
\sn \textcolor{darkblue}{\textbf{\ipa{tʰi˧-ʈwæ˧˥, | sɤ˧ | bi˧bv̩˥-ze˩!}}}
\trans [il] est tombé [et s'est blessé]; le sang a coulé à flots!
\trans \textit{\textcolor{brown}{\zh{(他)摔倒了,流了很多血}}}
\end{exe}
\begin{exe}
\sn \textcolor{darkblue}{\textbf{\ipa{dʑɯ˧ | bi˧bv̩˥-ze˩!}}}
\trans L'eau coule à flots!
\trans \textit{\textcolor{brown}{\zh{水流如注!}}}
\end{exe}
\begin{exe}
\sn \textcolor{darkblue}{\textbf{\ipa{dʑɯ˩nɑ˩mi˩ bi˩bv̩˥-ze˩-pʰæ˩di˩!}}}
\trans On dirait qu'il y a une coulée de boue / un glissement de terrain sur la montagne!
\trans \textit{\textcolor{brown}{\zh{山上好像有了泥石流!}}}
\end{exe}


\paragraph{\hspace{-0.5cm} {\Large \textcolor{darkblue}{\textbf{\ipa{bi˧ɕi˧kv̩˥}}}}} \hypertarget{bi\string_Ms£i\string_Mkv\string_=\string_T1}{}
\markboth{\textcolor{darkblue}{\textbf{\ipa{bi˧ɕi˧kv̩˥}}}}{}
\textcolor{teal}{\mytextsc{nom}} \hspace{4pt} Ton~: H\#.

Joue.
\textcolor{brown}{\zh{腮,腮帮子。}}


 \mytextsc{clf}~: \textcolor{darkblue}{\textbf{\ipa{ɭɯ˧}}} 

\paragraph{\hspace{-0.5cm} {\Large \textcolor{darkblue}{\textbf{\ipa{bi˧hæ˧˥}}}}} \hypertarget{bi\string_Mh\{\string_M\string_T1}{}
\markboth{\textcolor{darkblue}{\textbf{\ipa{bi˧hæ˧˥}}}}{}
\textcolor{teal}{\mytextsc{nom}} \hspace{4pt} Ton~: MH\#.

Sangle ventrale.
\textcolor{brown}{\zh{马肚带。}}


\begin{exe}
\sn \textcolor{darkblue}{\textbf{\ipa{ʐwæ˧-bi˥hæ˩}}}
\trans sangle de cheval
\trans \textit{\textcolor{brown}{\zh{马肚带}}}
\end{exe}
 \mytextsc{clf}~: \textcolor{darkblue}{\textbf{\ipa{kʰɯ˩}}} 

\paragraph{\hspace{-0.5cm} {\Large \textcolor{darkblue}{\textbf{\ipa{bi˧-lv̩˧\textasciitilde{}lv̩˥}}}}} \hypertarget{bi\string_M-lv\string_=\string_M~lv\string_=\string_T1}{}
\markboth{\textcolor{darkblue}{\textbf{\ipa{bi˧-lv̩˧\textasciitilde{}lv̩˥}}}}{}
\textcolor{teal}{\mytextsc{nom}} \hspace{4pt} Ton~: H\#.

Flocons de neige.
\textcolor{brown}{\zh{雪花。}}


 \mytextsc{clf}~: \textcolor{darkblue}{\textbf{\ipa{ɭɯ˧}}} 

\paragraph{\hspace{-0.5cm} {\Large \textcolor{darkblue}{\textbf{\ipa{bi˧mi˧}}}}} \hypertarget{bi\string_Mmi\string_M1}{}
\markboth{\textcolor{darkblue}{\textbf{\ipa{bi˧mi˧}}}}{}
\textcolor{teal}{\mytextsc{nom}} \hspace{4pt} Ton~: M.

Ventre.
\textcolor{brown}{\zh{肚子。}}


\begin{exe}
\sn \textcolor{darkblue}{\textbf{\ipa{bi˧mi˧-ɖɯ˩}}}
\trans avide, glouton; littéralement ``qui a un gros ventre/gros estomac"
\trans \textit{\textcolor{brown}{\zh{贪心不足,贪吃}}}
\end{exe}
 \mytextsc{clf}~: \textcolor{darkblue}{\textbf{\ipa{ɭɯ˧}}} 

\paragraph{\hspace{-0.5cm} {\Large \textcolor{darkblue}{\textbf{\ipa{bi˧tɑ˧}}}}} \hypertarget{bi\string_MtA\string_M1}{}
\markboth{\textcolor{darkblue}{\textbf{\ipa{bi˧tɑ˧}}}}{}
\textcolor{teal}{\mytextsc{nom}} \hspace{4pt} Ton~: M.

Ceinture large, en tissu.
\textcolor{brown}{\zh{宽腰带。}}


 \mytextsc{clf}~: \textcolor{darkblue}{\textbf{\ipa{tsʰi˥}}} 

\paragraph{\hspace{-0.5cm} {\Large \textcolor{darkblue}{\textbf{\ipa{bi˧tɕɤ˩}}}}} \hypertarget{bi\string_Mts£7\string_B1}{}
\markboth{\textcolor{darkblue}{\textbf{\ipa{bi˧tɕɤ˩}}}}{}
\textcolor{teal}{\mytextsc{nom}} \hspace{4pt} Ton~: L\#.

Nombril.
\textcolor{brown}{\zh{肚脐。}}


 \mytextsc{clf}~: \textcolor{darkblue}{\textbf{\ipa{kʰwɤ˥}}} 

\paragraph{\hspace{-0.5cm} {\Large \textcolor{darkblue}{\textbf{\ipa{bi˧tɕo˧}}}}} \hypertarget{bi\string_Mts£o\string_M1}{}
\markboth{\textcolor{darkblue}{\textbf{\ipa{bi˧tɕo˧}}}}{}
\textcolor{teal}{\mytextsc{nom}} \hspace{4pt} Ton~: M.

Les villages environnants; par exemple, pour la consultante principale, le village, \textcolor{darkblue}{\textbf{\ipa{/ʁwɤ˧qo˧/}}}, c'est \textcolor{darkblue}{\textbf{\ipa{/ə˧lɑ˧-ʁwɤ˧/;}}} les villages environnants constituent le voisinage, \textcolor{darkblue}{\textbf{\ipa{/bi˧tɕo˧/}}}.
\textcolor{brown}{\zh{周围的村落。}}




\paragraph{\hspace{-0.5cm} {\Large \textcolor{darkblue}{\textbf{\ipa{bi˧zɯ˧}}}}} \hypertarget{bi\string_MzM\string_M1}{}
\markboth{\textcolor{darkblue}{\textbf{\ipa{bi˧zɯ˧}}}}{}
\textcolor{teal}{\mytextsc{nom}} \hspace{4pt} Ton~: M.

Bas-ventre.
\textcolor{brown}{\zh{小肚子。}}


 \mytextsc{clf}~: \textcolor{darkblue}{\textbf{\ipa{ɭɯ˧}}} 

\paragraph{\hspace{-0.5cm} {\Large \textcolor{darkblue}{\textbf{\ipa{bi˩}}}}} \hypertarget{bi\string_B1}{}
\markboth{\textcolor{darkblue}{\textbf{\ipa{bi˩}}}}{}
\textcolor{teal}{\mytextsc{postposition}} \hspace{4pt} Ton~: L.

Sur; vers.
\textcolor{brown}{\zh{向、至、往。}}


\begin{exe}
\sn \textcolor{darkblue}{\textbf{\ipa{gv̩˧mi˧-bi˩}}}
\trans sur le corps
\trans \textit{\textcolor{brown}{\zh{身上}}}
\end{exe}
\begin{exe}
\sn \textcolor{darkblue}{\textbf{\ipa{kʰɯ˧tsʰɤ˧-bi˥}}}
\trans sur les pieds
\trans \textit{\textcolor{brown}{\zh{脚上}}}
\end{exe}
\begin{exe}
\sn \textcolor{darkblue}{\textbf{\ipa{ʐæ˩sɯ˩-bi˥ | tʰi˧-ʈʂʰv̩˧˥}}}
\trans [elle] a agrippé son vêtement
\trans \textit{\textcolor{brown}{\zh{抓住毡子}}}
\end{exe}
\begin{exe}
\sn \textcolor{darkblue}{\textbf{\ipa{lo˩qʰwɤ˧ bi˩}}}
\trans sur la main
\trans \textit{\textcolor{brown}{\zh{手上}}}
\end{exe}
\begin{exe}
\sn \textcolor{darkblue}{\textbf{\ipa{pʰæ˧qʰwɤ˩ bi˩, | mɤ˩ tʰi˩-jɤ˩˥.}}}
\trans s’étaler de l’huile sur le visage, se mettre de la crème solaire sur le visage
\trans \textit{\textcolor{brown}{\zh{抹防晒霜}}}
\end{exe}


\paragraph{\hspace{-0.5cm} {\Large \textcolor{darkblue}{\textbf{\ipa{bi˩\textsubscript{c}}}}}} \hypertarget{bi\string_Bc1}{}
\markboth{\textcolor{darkblue}{\textbf{\ipa{bi˩\textsubscript{c}}}}}{}
\textcolor{teal}{\mytextsc{classificateur}} \hspace{4pt} Ton~: L\textsubscript{c}.

Auto-classificateur des sabots d'animaux, et des traces qu'ils laissent sur le sol.
\textcolor{brown}{\zh{量词:动物的脚或脚印(一只)。}}




\paragraph{\hspace{-0.5cm} {\Large \textcolor{darkblue}{\textbf{\ipa{bi˩bi˧}}}}} \hypertarget{bi\string_Bbi\string_M1}{}
\markboth{\textcolor{darkblue}{\textbf{\ipa{bi˩bi˧}}}}{}
\textcolor{teal}{\mytextsc{nom}} \hspace{4pt} Ton~: LM.

Cosse de haricot.
\textcolor{brown}{\zh{豆荚。}}


\begin{exe}
\sn \textcolor{darkblue}{\textbf{\ipa{bi˩bi˧ ɲi˩}}}
\trans \mytextsc{cop}
\trans \textit{\textcolor{brown}{\zh{是豆荚}}}
\end{exe}
\begin{exe}
\sn \textcolor{darkblue}{\textbf{\ipa{nv̩˩ɭɯ˧-bi˩bi˩}}}
\trans cosses de soja
\trans \textit{\textcolor{brown}{\zh{黄豆荚}}}
\end{exe}
 \mytextsc{clf}~: \textcolor{darkblue}{\textbf{\ipa{kʰwɤ˥}}} 

\paragraph{\hspace{-0.5cm} {\Large \textcolor{darkblue}{\textbf{\ipa{bi˩mi˩}}}}} \hypertarget{bi\string_Bmi\string_B1}{}
\markboth{\textcolor{darkblue}{\textbf{\ipa{bi˩mi˩}}}}{}
\textcolor{teal}{\mytextsc{nom}} \hspace{4pt} Ton~: L.

Hache.
\textcolor{brown}{\zh{斧头。}}


 \mytextsc{clf}~: \textcolor{darkblue}{\textbf{\ipa{nɑ˧}}} 

\paragraph{\hspace{-0.5cm} {\Large \textcolor{darkblue}{\textbf{\ipa{bi˩pʰv̩˧˥}}}}} \hypertarget{bi\string_Bp\string_hv\string_=\string_M\string_T1}{}
\markboth{\textcolor{darkblue}{\textbf{\ipa{bi˩pʰv̩˧˥}}}}{}
\textcolor{teal}{\mytextsc{nom}} \hspace{4pt} Ton~: LM+MH\#.

Gourde (cucurbitacée); son fruit est la calebasse, qui devient dure en séchant.
\textcolor{brown}{\zh{葫芦。}}




\paragraph{\hspace{-0.5cm} {\Large \textcolor{darkblue}{\textbf{\ipa{bi˩pʰv̩˧-dʑɯ˧˥}}}}} \hypertarget{bi\string_Bp\string_hv\string_=\string_M-dz£M\string_M\string_T1}{}
\markboth{\textcolor{darkblue}{\textbf{\ipa{bi˩pʰv̩˧-dʑɯ˧˥}}}}{}
\textcolor{teal}{\mytextsc{nom}} \hspace{4pt} Ton~: LM+MH\#.

Inondation.
\textcolor{brown}{\zh{洪水。}}




\paragraph{\hspace{-0.5cm} {\Large \textcolor{darkblue}{\textbf{\ipa{bi˩ʁo˥}}}}} \hypertarget{bi\string_BRo\string_T1}{}
\markboth{\textcolor{darkblue}{\textbf{\ipa{bi˩ʁo˥}}}}{}
\textcolor{teal}{\mytextsc{nom}} \hspace{4pt} Ton~: LH.

Bourse. La bourse était autrefois attachée à la ceinture, sur sa face interne.
\textcolor{brown}{\zh{钱包(过去:是系在腰带上的)。}}


\begin{exe}
\sn \textcolor{darkblue}{\textbf{\ipa{bi˩ʁo˧ tʰi˧-ʂæ˧˥}}}
\trans attacher sa bourse (à sa ceinture)
\trans \textit{\textcolor{brown}{\zh{系上钱包(在腰带上)}}}
\end{exe}
 \mytextsc{clf}~: \textcolor{darkblue}{\textbf{\ipa{ɭɯ˧}}} 

\paragraph{\hspace{-0.5cm} {\Large \textcolor{darkblue}{\textbf{\ipa{bi˩tɑ˩}}}}} \hypertarget{bi\string_BtA\string_B1}{}
\markboth{\textcolor{darkblue}{\textbf{\ipa{bi˩tɑ˩}}}}{}
\textcolor{teal}{\mytextsc{verbe}} \hspace{4pt} Ton~: L.

Tirer.
\textcolor{brown}{\zh{拖。}}


\begin{exe}
\sn \textcolor{darkblue}{\textbf{\ipa{bi˩tɑ˩-ze˥}}}
\trans \mytextsc{pfv}
\trans \textit{\textcolor{brown}{\zh{拖了}}}
\end{exe}


\paragraph{\hspace{-0.5cm} {\Large \textcolor{darkblue}{\textbf{\ipa{bi˩-tsɯ˧tsɯ˧}}}}} \hypertarget{bi\string_B-tsM\string_MtsM\string_M1}{}
\markboth{\textcolor{darkblue}{\textbf{\ipa{bi˩-tsɯ˧tsɯ˧}}}}{}
\textcolor{teal}{\mytextsc{nom}} \hspace{4pt} Ton~: L-.

Fraise sauvage, \textit{Fragaria vesca}.
\textcolor{brown}{\zh{野草莓。}}




\paragraph{\hspace{-0.5cm} {\Large \textcolor{darkblue}{\textbf{\ipa{bi˩ʈʂʰɤ\#˥}}}}} \hypertarget{bi\string_Bt`s`\string_h7\#\string_T1}{}
\markboth{\textcolor{darkblue}{\textbf{\ipa{bi˩ʈʂʰɤ\#˥}}}}{}
\textcolor{teal}{\mytextsc{nom}} \hspace{4pt} Ton~: LM+\#H.

Favoris, rouflaquettes.
\textcolor{brown}{\zh{胡须。}}


 \mytextsc{clf}~: \textcolor{darkblue}{\textbf{\ipa{kʰwɤ˥}}} 

\paragraph{\hspace{-0.5cm} {\Large \textcolor{darkblue}{\textbf{\ipa{bi˩wɤ˧}}}}} \hypertarget{bi\string_Bw7\string_M1}{}
\markboth{\textcolor{darkblue}{\textbf{\ipa{bi˩wɤ˧}}}}{}
\textcolor{teal}{\mytextsc{nom}} \hspace{4pt} Ton~: LM.

Services (ou argent) donnés en récompense à un moine.
\textcolor{brown}{\zh{酬劳。}}


 \mytextsc{clf}~: \textcolor{darkblue}{\textbf{\ipa{ɭɯ˧}}} 

\paragraph{\hspace{-0.5cm} {\Large \textcolor{darkblue}{\textbf{\ipa{bo}}}}} \hypertarget{bo1}{}
\markboth{\textcolor{darkblue}{\textbf{\ipa{bo}}}}{}
\textcolor{teal}{\mytextsc{particule}} \textcolor{teal}{\mytextsc{de}} \textcolor{teal}{\mytextsc{discours}} \hspace{4pt} Ton~: 0.

Particule finale empruntée au chinois, exprimant une affirmation vigoureuse: ah que si!
\textcolor{brown}{\zh{句尾助词:啵(汉语借词)。}}

 Emprunt~: chinois \textcolor{brown}{\zh{啵}}



\paragraph{\hspace{-0.5cm} {\Large \textcolor{darkblue}{\textbf{\ipa{bo˧}}} \textsubscript{1}}} \hypertarget{bo\string_M1}{}
\markboth{\textcolor{darkblue}{\textbf{\ipa{bo˧}}} \textsubscript{1}}{}
\textcolor{teal}{\mytextsc{nom}} \hspace{4pt} Ton~: M.

Diguette en bord de champ, sur laquelle on peut marcher.
\textcolor{brown}{\zh{田埂、小堤坝。}}


\begin{exe}
\sn \textcolor{darkblue}{\textbf{\ipa{ʈʂʰɯ˧-qo˧ | bo˧ ɖɯ˧-ɭɯ˧ tʰi˧-di˩.}}}
\trans ici, il y a une diguette.
\trans \textit{\textcolor{brown}{\zh{这里有一个小堤坝。}}}
\end{exe}
\begin{exe}
\sn \textcolor{darkblue}{\textbf{\ipa{bo˧-kʰi˧}}}
\trans le bord de la diguette
\trans \textit{\textcolor{brown}{\zh{小堤坝的边沿}}}
\end{exe}
\begin{exe}
\sn \textcolor{darkblue}{\textbf{\ipa{[F5] bo˧ ɖɯ˧-pʰæ˧˥}}}
\trans une diguette
\trans \textit{\textcolor{brown}{\zh{一个小堤坝}}}
\end{exe}
 \mytextsc{clf}~: \textcolor{darkblue}{\textbf{\ipa{ɭɯ˧}}} objets ronds

\paragraph{\hspace{-0.5cm} {\Large \textcolor{darkblue}{\textbf{\ipa{bo˧}}} \textsubscript{2}}} \hypertarget{bo\string_M2}{}
\markboth{\textcolor{darkblue}{\textbf{\ipa{bo˧}}} \textsubscript{2}}{}
\textcolor{teal}{\mytextsc{adjectif}} \hspace{4pt} Ton~: M.

Lumineux.
\textcolor{brown}{\zh{亮,光明。}}


\begin{exe}
\sn \textcolor{darkblue}{\textbf{\ipa{tʰi˧-bo˧-dʑo˧}}}
\trans ça éclaire, c'est lumineux (définition d'une lampe)
\trans \textit{\textcolor{brown}{\zh{(它)在发光。(描写灯)}}}
\end{exe}
\begin{exe}
\sn \textcolor{darkblue}{\textbf{\ipa{bo˧-hĩ˧}}}
\trans \mytextsc{rel}
\trans \textit{\textcolor{brown}{\zh{发亮的}}}
\end{exe}


\paragraph{\hspace{-0.5cm} {\Large \textcolor{darkblue}{\textbf{\ipa{bo˧\textsubscript{b}}}}}} \hypertarget{bo\string_Mb1}{}
\markboth{\textcolor{darkblue}{\textbf{\ipa{bo˧\textsubscript{b}}}}}{}
\textcolor{teal}{\mytextsc{verbe}} \hspace{4pt} Ton~: M\textsubscript{b}.

Faire tourner (la quenouille, pour filer le fil de lin).
\textcolor{brown}{\zh{纺(麻线),使旋转。}}


\begin{exe}
\sn \textcolor{darkblue}{\textbf{\ipa{tso˧\textasciitilde{}tso˧ bo˧}}}
\trans faire tourner quelque chose
\trans \textit{\textcolor{brown}{\zh{使东西旋转}}}
\end{exe}
\textit{Voir~:} \hyperlink{sA\string_Mbo\#\string_T1}{\textcolor{darkblue}{\textbf{\ipa{sɑ˧bo\#˥}}}} 

\paragraph{\hspace{-0.5cm} {\Large \textcolor{darkblue}{\textbf{\ipa{bo˧tsi˩}}}}} \hypertarget{bo\string_Mtsi\string_B1}{}
\markboth{\textcolor{darkblue}{\textbf{\ipa{bo˧tsi˩}}}}{}
\textcolor{teal}{\mytextsc{nom}} \hspace{4pt} Ton~: L\#.

Crinière.
\textcolor{brown}{\zh{(马)鬃。}}


\begin{exe}
\sn \textcolor{darkblue}{\textbf{\ipa{ʐwæ˧-bo˧tsi˥\#}}}
\trans crinière du cheval
\trans \textit{\textcolor{brown}{\zh{马鬃}}}
\end{exe}
\begin{exe}
\sn \textcolor{darkblue}{\textbf{\ipa{bo˩-bo˧tsi˩}}}
\trans soies de porc ou de sanglier
\trans \textit{\textcolor{brown}{\zh{猪鬃}}}
\end{exe}
 \mytextsc{clf}~: \textcolor{darkblue}{\textbf{\ipa{kʰwɤ˥}}} \textcolor{darkblue}{\textbf{\ipa{qɑ˩}}} morceaux

\paragraph{\hspace{-0.5cm} {\Large \textcolor{darkblue}{\textbf{\ipa{bo˧ʐæ˧}}}}} \hypertarget{bo\string_Mz`\{\string_M1}{}
\markboth{\textcolor{darkblue}{\textbf{\ipa{bo˧ʐæ˧}}}}{}
\textcolor{teal}{\mytextsc{nom}} \hspace{4pt} Ton~: M.

Verre (matière).
\textcolor{brown}{\zh{玻璃。}}


\begin{exe}
\sn \textcolor{darkblue}{\textbf{\ipa{bo˧ʐæ˧-tɕʰɤ˩ʈʂv˩}}}
\trans gobelet à thé en verre
\trans \textit{\textcolor{brown}{\zh{玻璃茶杯}}}
\end{exe}


\paragraph{\hspace{-0.5cm} {\Large \textcolor{darkblue}{\textbf{\ipa{bo˧ʐæ˧ʈæ˧qʰv̩\#˥}}}}} \hypertarget{bo\string_Mz`\{\string_Mt`\{\string_Mq\string_hv\string_=\#\string_T1}{}
\markboth{\textcolor{darkblue}{\textbf{\ipa{bo˧ʐæ˧ʈæ˧qʰv̩\#˥}}}}{}
\textcolor{teal}{\mytextsc{nom}} \hspace{4pt} Ton~: \#H.

Écho (en certains endroits des montagnes, il y a un écho).
\textcolor{brown}{\zh{回音。}}


\begin{exe}
\sn \textcolor{darkblue}{\textbf{\ipa{bo˧ʐæ˧ʈæ˧qʰv̩˧ | tʰi˧-ɖʐɯ˩\textasciitilde{}ɖʐɯ˩!}}}
\trans Il y a de l'écho!
\trans \textit{\textcolor{brown}{\zh{有回音}}}
\end{exe}


\paragraph{\hspace{-0.5cm} {\Large \textcolor{darkblue}{\textbf{\ipa{bo˩\textsubscript{a}}}}}} \hypertarget{bo\string_Ba1}{}
\markboth{\textcolor{darkblue}{\textbf{\ipa{bo˩\textsubscript{a}}}}}{}
\textcolor{teal}{\mytextsc{verbe}} \hspace{4pt} Ton~: L\textsubscript{a}.

Embrasser.
\textcolor{brown}{\zh{亲吻。}}


\begin{exe}
\sn \textcolor{darkblue}{\textbf{\ipa{ɖɯ˧-bo˩-ɻ̍˩}}}
\trans \mytextsc{délimitatif} \string_ \mytextsc{inchoatif}
\trans \textit{\textcolor{brown}{\zh{吻一下}}}
\end{exe}
\begin{exe}
\sn \textcolor{darkblue}{\textbf{\ipa{ɖɯ˧-bo˧\textasciitilde{}bo˥-ɻ̍˩}}}
\trans \mytextsc{délimitatif} \string_ \mytextsc{red} \mytextsc{inchoatif}
\trans \textit{\textcolor{brown}{\zh{吻一下}}}
\end{exe}


\paragraph{\hspace{-0.5cm} {\Large \textcolor{darkblue}{\textbf{\ipa{bo˩\textsubscript{b}}}}}} \hypertarget{bo\string_Bb1}{}
\markboth{\textcolor{darkblue}{\textbf{\ipa{bo˩\textsubscript{b}}}}}{}
\textcolor{teal}{\mytextsc{classificateur}} \hspace{4pt} Ton~: L\textsubscript{b}.

Classificateur des coiffes (parures pour la chevelure des femmes).
\textcolor{brown}{\zh{量词:缎子发带(一条)。}}




\paragraph{\hspace{-0.5cm} {\Large \textcolor{darkblue}{\textbf{\ipa{bo˩-bi˧mi˧}}}}} \hypertarget{bo\string_B-bi\string_Mmi\string_M1}{}
\markboth{\textcolor{darkblue}{\textbf{\ipa{bo˩-bi˧mi˧}}}}{}
\textcolor{teal}{\mytextsc{nom}} \hspace{4pt} Ton~: L-.

Viande du ventre du cochon.
\textcolor{brown}{\zh{猪肚肉。}}


 \mytextsc{clf}~: \textcolor{darkblue}{\textbf{\ipa{ɭɯ˧}}} 

\paragraph{\hspace{-0.5cm} {\Large \textcolor{darkblue}{\textbf{\ipa{bo˩-bv̩˥}}}}} \hypertarget{bo\string_B-bv\string_=\string_T1}{}
\markboth{\textcolor{darkblue}{\textbf{\ipa{bo˩-bv̩˥}}}}{}
\textcolor{teal}{\mytextsc{nom}} \hspace{4pt} Ton~: LH.

Enclos des porcs.
\textcolor{brown}{\zh{猪圈。}}


 \mytextsc{clf}~: \textcolor{darkblue}{\textbf{\ipa{ɭɯ˧}}} 

\paragraph{\hspace{-0.5cm} {\Large \textcolor{darkblue}{\textbf{\ipa{bo˩dze˧}}}}} \hypertarget{bo\string_Bdze\string_M1}{}
\markboth{\textcolor{darkblue}{\textbf{\ipa{bo˩dze˧}}}}{}
\textcolor{teal}{\mytextsc{nom}} \hspace{4pt} Ton~: LM.

Alouette.
\textcolor{brown}{\zh{百灵鸟。}}


\textit{Voir~:} \hyperlink{bo\string_Bdze\string_M-ko\string_Bdze\string_B1}{\textcolor{darkblue}{\textbf{\ipa{bo˩dze˧-ko˩dze˩}}}} 

\paragraph{\hspace{-0.5cm} {\Large \textcolor{darkblue}{\textbf{\ipa{bo˩dze˧-ko˩dze˩}}}}} \hypertarget{bo\string_Bdze\string_M-ko\string_Bdze\string_B1}{}
\markboth{\textcolor{darkblue}{\textbf{\ipa{bo˩dze˧-ko˩dze˩}}}}{}
\textcolor{teal}{\mytextsc{nom}} \hspace{4pt} Ton~: LM-L.

Alouette.
\textcolor{brown}{\zh{百灵鸟。}}


\textit{Voir~:} \hyperlink{bo\string_Bdze\string_M1}{\textcolor{darkblue}{\textbf{\ipa{bo˩dze˧}}}} 

\paragraph{\hspace{-0.5cm} {\Large \textcolor{darkblue}{\textbf{\ipa{bo˩-ɣɯ˥}}}}} \hypertarget{bo\string_B-GM\string_T1}{}
\markboth{\textcolor{darkblue}{\textbf{\ipa{bo˩-ɣɯ˥}}}}{}
\textcolor{teal}{\mytextsc{nom}} \hspace{4pt} Ton~: LH.

Couenne.
\textcolor{brown}{\zh{猪皮。}}


\begin{exe}
\sn \textcolor{darkblue}{\textbf{\ipa{bo˩-ɣɯ˧kɯ˩}}}
\trans même sens: couenne
\trans \textit{\textcolor{brown}{\zh{同上:猪皮}}}
\end{exe}
 \mytextsc{clf}~: \textcolor{darkblue}{\textbf{\ipa{tsʰi˥}}} 

\paragraph{\hspace{-0.5cm} {\Large \textcolor{darkblue}{\textbf{\ipa{bo˩-hɑ\#˥}}}}} \hypertarget{bo\string_B-hA\#\string_T1}{}
\markboth{\textcolor{darkblue}{\textbf{\ipa{bo˩-hɑ\#˥}}}}{}
\textcolor{teal}{\mytextsc{nom}} \hspace{4pt} Ton~: LM+\#H.

Nourriture du cochon/ pâtée des porcs.
\textcolor{brown}{\zh{猪食。}}


 \mytextsc{clf}~: \textcolor{darkblue}{\textbf{\ipa{kʰɤ˧˥}}} 

\paragraph{\hspace{-0.5cm} {\Large \textcolor{darkblue}{\textbf{\ipa{bo˩-kʰɯ˧}}}}} \hypertarget{bo\string_B-k\string_hM\string_M1}{}
\markboth{\textcolor{darkblue}{\textbf{\ipa{bo˩-kʰɯ˧}}}}{}
\textcolor{teal}{\mytextsc{nom}} \hspace{4pt} Ton~: LM.

Pieds de porc (pièce de boucherie): viande séchée conservée dans la peau du pied de cochon.
\textcolor{brown}{\zh{猪脚腊肉:把猪腿的皮剥下来,缝成筒形,塞满瘦肉。}}


 \mytextsc{clf}~: \textcolor{darkblue}{\textbf{\ipa{ɭɯ˧}}} 

\paragraph{\hspace{-0.5cm} {\Large \textcolor{darkblue}{\textbf{\ipa{bo˩-kʰv̩˧˥}}}}} \hypertarget{bo\string_B-k\string_hv\string_=\string_M\string_T1}{}
\markboth{\textcolor{darkblue}{\textbf{\ipa{bo˩-kʰv̩˧˥}}}}{}
\textcolor{teal}{\mytextsc{nom}} \hspace{4pt} Ton~: LM+MH\#.

Année du porc.
\textcolor{brown}{\zh{猪年。}}




\paragraph{\hspace{-0.5cm} {\Large \textcolor{darkblue}{\textbf{\ipa{bo˩lo˧}}}}} \hypertarget{bo\string_Blo\string_M1}{}
\markboth{\textcolor{darkblue}{\textbf{\ipa{bo˩lo˧}}}}{}
\textcolor{teal}{\mytextsc{nom}} \hspace{4pt} Ton~: LM.

Mortaise.
\textcolor{brown}{\zh{榫眼。}}


\begin{exe}
\sn \textcolor{darkblue}{\textbf{\ipa{bo˩lo˧ | ɖɯ˧-ɭɯ˧}}}
\trans une mortaise
\trans \textit{\textcolor{brown}{\zh{一个榫眼}}}
\end{exe}
 \mytextsc{clf}~: \textcolor{darkblue}{\textbf{\ipa{ɭɯ˧}}} 

\paragraph{\hspace{-0.5cm} {\Large \textcolor{darkblue}{\textbf{\ipa{bo˩ɬɑ˥}}}}} \hypertarget{bo\string_BKA\string_T1}{}
\markboth{\textcolor{darkblue}{\textbf{\ipa{bo˩ɬɑ˥}}}}{}
\textcolor{teal}{\mytextsc{nom}} \hspace{4pt} Ton~: LH.

Verrat.
\textcolor{brown}{\zh{种公猪。}}


 \mytextsc{clf}~: \textcolor{darkblue}{\textbf{\ipa{v̩˧}}} 

\paragraph{\hspace{-0.5cm} {\Large \textcolor{darkblue}{\textbf{\ipa{bo˩-ɬo˥}}}}} \hypertarget{bo\string_B-Ko\string_T1}{}
\markboth{\textcolor{darkblue}{\textbf{\ipa{bo˩-ɬo˥}}}}{}
\textcolor{teal}{\mytextsc{nom}} \hspace{4pt} Ton~: LH.

Côtes de porc.
\textcolor{brown}{\zh{猪肋骨。}}


\begin{exe}
\sn \textcolor{darkblue}{\textbf{\ipa{bo˩ɬo˥ | ɖɯ˧-do˥}}}
\trans Un quartier de côtes de porc; désigne aussi, métaphoriquement, une famille, dans laquelle chaque individu est solidaire des autres
\trans \textit{\textcolor{brown}{\zh{一大块猪肋骨。也来比喻一个家庭,强调家所有成员之间的密切关系。}}}
\end{exe}
 \mytextsc{clf}~: \textcolor{darkblue}{\textbf{\ipa{ɭɯ˧}}} 

\paragraph{\hspace{-0.5cm} {\Large \textcolor{darkblue}{\textbf{\ipa{bo˩-mæ˧qv̩˩}}}}} \hypertarget{bo\string_B-m\{\string_Mqv\string_=\string_B1}{}
\markboth{\textcolor{darkblue}{\textbf{\ipa{bo˩-mæ˧qv̩˩}}}}{}
\textcolor{teal}{\mytextsc{nom}} \hspace{4pt} Ton~: L-L\#.

Queue du cochon.
\textcolor{brown}{\zh{猪尾巴。}}


 \mytextsc{clf}~: \textcolor{darkblue}{\textbf{\ipa{ɭɯ˧}}} 

\paragraph{\hspace{-0.5cm} {\Large \textcolor{darkblue}{\textbf{\ipa{bo˩-mɤ˥}}}}} \hypertarget{bo\string_B-m7\string_T1}{}
\markboth{\textcolor{darkblue}{\textbf{\ipa{bo˩-mɤ˥}}}}{}
\textcolor{teal}{\mytextsc{nom}} \hspace{4pt} Ton~: LH.

Saindoux (gras de porc).
\textcolor{brown}{\zh{猪油。}}




\paragraph{\hspace{-0.5cm} {\Large \textcolor{darkblue}{\textbf{\ipa{bo˩mi˧}}}}} \hypertarget{bo\string_Bmi\string_M1}{}
\markboth{\textcolor{darkblue}{\textbf{\ipa{bo˩mi˧}}}}{}
\textcolor{teal}{\mytextsc{nom}} \hspace{4pt} Ton~: LM.

Truie.
\textcolor{brown}{\zh{母猪。}}


\begin{exe}
\sn \textcolor{darkblue}{\textbf{\ipa{bo˩mi˧ ʑi˩}}}
\trans attraper une truie
\trans \textit{\textcolor{brown}{\zh{抓母猪}}}
\end{exe}
\begin{exe}
\sn \textcolor{darkblue}{\textbf{\ipa{bo˩mi˧ do˧ (+ze˩)}}}
\trans ...a vu (une/la) truie
\trans \textit{\textcolor{brown}{\zh{见了母猪}}}
\end{exe}
\begin{exe}
\sn \textcolor{darkblue}{\textbf{\ipa{bo˩mi˧-bæ˧bv̩˥}}}
\trans truie et porcelets
\trans \textit{\textcolor{brown}{\zh{母猪与猪仔}}}
\end{exe}
 \mytextsc{clf}~: \textcolor{darkblue}{\textbf{\ipa{mi˩}}} \textcolor{darkblue}{\textbf{\ipa{v̩˧}}} 

\paragraph{\hspace{-0.5cm} {\Large \textcolor{darkblue}{\textbf{\ipa{bo˩mi˧-dʑɯ˩pv̩˩}}}}} \hypertarget{bo\string_Bmi\string_M-dz£M\string_Bpv\string_=\string_B1}{}
\markboth{\textcolor{darkblue}{\textbf{\ipa{bo˩mi˧-dʑɯ˩pv̩˩}}}}{}
\textcolor{teal}{\mytextsc{nom}} \hspace{4pt} Ton~: LM-L.

Dytique, \textit{(Dytiscus}.
\textcolor{brown}{\zh{龙虱。}}




\paragraph{\hspace{-0.5cm} {\Large \textcolor{darkblue}{\textbf{\ipa{bo˩mi˧-dʑɯ˩pʰv̩˩}}}}} \hypertarget{bo\string_Bmi\string_M-dz£M\string_Bp\string_hv\string_=\string_B1}{}
\markboth{\textcolor{darkblue}{\textbf{\ipa{bo˩mi˧-dʑɯ˩pʰv̩˩}}}}{}
\textcolor{teal}{\mytextsc{nom}} \hspace{4pt} Ton~: LM-L.

Charançon.
\textcolor{brown}{\zh{象鼻虫,米象。}}




\paragraph{\hspace{-0.5cm} {\Large \textcolor{darkblue}{\textbf{\ipa{bo˩mi˧-ɳæ˧tɕʰɯ˩}}}}} \hypertarget{bo\string_Bmi\string_M-n`\{\string_Mts£\string_hM\string_B1}{}
\markboth{\textcolor{darkblue}{\textbf{\ipa{bo˩mi˧-ɳæ˧tɕʰɯ˩}}}}{}
\textcolor{teal}{\mytextsc{nom}} \hspace{4pt} Ton~: LM-L\#.

Pissenlit.
\textcolor{brown}{\zh{蒲公英。}}


 \mytextsc{clf}~: \textcolor{darkblue}{\textbf{\ipa{po˧}}} 

\paragraph{\hspace{-0.5cm} {\Large \textcolor{darkblue}{\textbf{\ipa{bo˩mi˧-ʁo˩do˩}}}}} \hypertarget{bo\string_Bmi\string_M-Ro\string_Bdo\string_B1}{}
\markboth{\textcolor{darkblue}{\textbf{\ipa{bo˩mi˧-ʁo˩do˩}}}}{}
\textcolor{teal}{\mytextsc{nom}} \hspace{4pt} Ton~: LM-L.

Yyyy littéralement ``noix des truies"; était employé autrefois pour application sur les plaies qui suppuraient: on en extrayait le jus.
\textcolor{brown}{\zh{曼陀罗。}}


 \mytextsc{clf}~: \textcolor{darkblue}{\textbf{\ipa{dzi˩}}} 

\paragraph{\hspace{-0.5cm} {\Large \textcolor{darkblue}{\textbf{\ipa{bo˩pʰv̩˧}}}}} \hypertarget{bo\string_Bp\string_hv\string_=\string_M1}{}
\markboth{\textcolor{darkblue}{\textbf{\ipa{bo˩pʰv̩˧}}}}{}
\textcolor{teal}{\mytextsc{nom}} \hspace{4pt} Ton~: LM.

Verrat.
\textcolor{brown}{\zh{种公猪。}}


 \mytextsc{clf}~: \textcolor{darkblue}{\textbf{\ipa{v̩˧}}} 

\paragraph{\hspace{-0.5cm} {\Large \textcolor{darkblue}{\textbf{\ipa{bo˩qʰæ˧-pv̩˧ʈɤ˥-ɻ̍˩}}}}} \hypertarget{bo\string_Bq\string_h\{\string_M-pv\string_=\string_Mt`7\string_T-r£`̍\string_B1}{}
\markboth{\textcolor{darkblue}{\textbf{\ipa{bo˩qʰæ˧-pv̩˧ʈɤ˥-ɻ̍˩}}}}{}
\textcolor{teal}{\mytextsc{nom}} \hspace{4pt} Ton~: LM - H\# -.

Bousier: sorte de scarabée, qui prolifère dans les étables lorsqu'il fait chaud.
\textcolor{brown}{\zh{蜣螂。}}




\paragraph{\hspace{-0.5cm} {\Large \textcolor{darkblue}{\textbf{\ipa{bo˩tv̩\#˥}}}}} \hypertarget{bo\string_Btv\string_=\#\string_T1}{}
\markboth{\textcolor{darkblue}{\textbf{\ipa{bo˩tv̩\#˥}}}}{}
\textcolor{teal}{\mytextsc{nom}} \hspace{4pt} Ton~: LM+\#H.

Sanglier.
\textcolor{brown}{\zh{野猪。}}


\begin{exe}
\sn \textcolor{darkblue}{\textbf{\ipa{bo˩tv̩˧ hwæ˥}}}
\trans acheter un sanglier
\trans \textit{\textcolor{brown}{\zh{买野猪}}}
\end{exe}
 \mytextsc{clf}~: \textcolor{darkblue}{\textbf{\ipa{mi˩}}} 

\paragraph{\hspace{-0.5cm} {\Large \textcolor{darkblue}{\textbf{\ipa{bo˩ʈʂʰæ˥}}}}} \hypertarget{bo\string_Bt`s`\string_h\{\string_T1}{}
\markboth{\textcolor{darkblue}{\textbf{\ipa{bo˩ʈʂʰæ˥}}}}{}
\textcolor{teal}{\mytextsc{nom}} \hspace{4pt} Ton~: LH.

Lard; le même terme est employé pour désigner le cochon entier désossé et conservé dans sa peau (au moyen de salpêtre et de sel), qui se conserve une décennie, appelé ``viande pipa" en chinois. La glose adoptée dans les textes est: cochon-conservé-entier.
\textcolor{brown}{\zh{猪膘,琵琶肉。}}




\paragraph{\hspace{-0.5cm} {\Large \textcolor{darkblue}{\textbf{\ipa{bo˩zɑ˧mi\#˥}}}}} \hypertarget{bo\string_BzA\string_Mmi\#\string_T1}{}
\markboth{\textcolor{darkblue}{\textbf{\ipa{bo˩zɑ˧mi\#˥}}}}{}
\textcolor{teal}{\mytextsc{nom}} \hspace{4pt} Ton~: LM+\#H.

``Petite Truie": nom employé pour les petites filles pendant leurs premiers mois, avant qu'on leur donne un vrai nom. Le vilain nom dont on l'affuble vise à éviter que le nourrisson ne soit repéré par de mauvais esprits. (Actuellement, l'état-civil nécessite qu'un nom soit donné dès la naissance; mais celui-ci ne commence à être employé dans les conversations familiales qu'une fois passés les premiers mois.).
\textcolor{brown}{\zh{猪崽子(给刚出生的女孩起的名字,让鬼对她不感兴趣,不会来害小孩)。}}




\paragraph{\hspace{-0.5cm} {\Large \textcolor{darkblue}{\textbf{\ipa{bo˩˧}}}}} \hypertarget{bo\string_B\string_M1}{}
\markboth{\textcolor{darkblue}{\textbf{\ipa{bo˩˧}}}}{}
\textcolor{teal}{\mytextsc{nom}} \hspace{4pt} Ton~: LM.

Porc, cochon.
\textcolor{brown}{\zh{猪。}}


\begin{exe}
\sn \textcolor{darkblue}{\textbf{\ipa{bo˩ hwæ˧-ze˧}}}
\trans ...a acheté (du/un) porc
\trans \textit{\textcolor{brown}{\zh{买了猪}}}
\end{exe}
\begin{exe}
\sn \textcolor{darkblue}{\textbf{\ipa{bo˩ dzɯ˥-ze˩}}}
\trans ...a mangé (du/un) porc
\trans \textit{\textcolor{brown}{\zh{吃了猪}}}
\end{exe}
 \mytextsc{clf}~: \textcolor{darkblue}{\textbf{\ipa{mi˩}}} \textcolor{darkblue}{\textbf{\ipa{v̩˧}}} 

\paragraph{\hspace{-0.5cm} {\Large \textcolor{darkblue}{\textbf{\ipa{bõ}}}}} \hypertarget{bo\string_~1}{}
\markboth{\textcolor{darkblue}{\textbf{\ipa{bõ}}}}{}
\textcolor{teal}{\mytextsc{idéophone}} \hspace{4pt} Ton~: 0.

Bruit d'un choc entre deux objets durs, par exemple un coup de hache sur un tronc: Boum!
\textcolor{brown}{\zh{形声词:斧头砍树。砰! / 啪!。}}




\paragraph{\hspace{-0.5cm} {\Large \textcolor{darkblue}{\textbf{\ipa{bv̩˩}}}}} \hypertarget{bv\string_=\string_B1}{}
\markboth{\textcolor{darkblue}{\textbf{\ipa{bv̩˩}}}}{}
\textcolor{teal}{\mytextsc{nom}} \hspace{4pt} Ton~: L.

Enclos (monosyllabe).
\textcolor{brown}{\zh{牲畜圈(单音节)。}}


\begin{exe}
\sn \textcolor{darkblue}{\textbf{\ipa{bv̩˩-qo˩}}}
\trans dans l'enclos
\trans \textit{\textcolor{brown}{\zh{牲畜圈里面}}}
\end{exe}
\begin{exe}
\sn \textcolor{darkblue}{\textbf{\ipa{bv̩˩qo˩ ʈæ˧}}}
\trans enfermer dans l'étable
\trans \textit{\textcolor{brown}{\zh{关在圈里}}}
\end{exe}
 \mytextsc{clf}~: \textcolor{darkblue}{\textbf{\ipa{ɭɯ˧}}} 

\paragraph{\hspace{-0.5cm} {\Large \textcolor{darkblue}{\textbf{\ipa{bv̩˩˧}}} \textsubscript{1}}} \hypertarget{bv\string_=\string_B\string_M1}{}
\markboth{\textcolor{darkblue}{\textbf{\ipa{bv̩˩˧}}} \textsubscript{1}}{}
\textcolor{teal}{\mytextsc{nom}} \hspace{4pt} Ton~: LM.

Yak, Bos grunniens (sauvage ou domestiqué).
\textcolor{brown}{\zh{牦牛/野牦牛。}}


\begin{exe}
\sn \textcolor{darkblue}{\textbf{\ipa{bv̩˩-hṽ˩˥}}}
\trans poil de yak
\trans \textit{\textcolor{brown}{\zh{牦牛毛}}}
\end{exe}
\begin{exe}
\sn \textcolor{darkblue}{\textbf{\ipa{bv̩˩ dzɯ˧-ze˩}}}
\trans ...a mangé (du) yak
\trans \textit{\textcolor{brown}{\zh{吃了牦牛}}}
\end{exe}
\begin{exe}
\sn \textcolor{darkblue}{\textbf{\ipa{bv̩˩ hwæ˧-ze˧}}}
\trans ...a acheté (du) yak
\trans \textit{\textcolor{brown}{\zh{买了牦牛}}}
\end{exe}
 \mytextsc{clf}~: \textcolor{darkblue}{\textbf{\ipa{pʰo˧˥}}} 

\paragraph{\hspace{-0.5cm} {\Large \textcolor{darkblue}{\textbf{\ipa{bv̩˩˧}}} \textsubscript{2}}} \hypertarget{bv\string_=\string_B\string_M2}{}
\markboth{\textcolor{darkblue}{\textbf{\ipa{bv̩˩˧}}} \textsubscript{2}}{}
\textcolor{teal}{\mytextsc{nom}} \hspace{4pt} Ton~: LM.

Étuve.
\textcolor{brown}{\zh{蒸笼。}}


 \mytextsc{clf}~: \textcolor{darkblue}{\textbf{\ipa{mi˩}}} \textit{Voir~:} \hyperlink{bv\string_=\string_Bdi\string_B1}{\textcolor{darkblue}{\textbf{\ipa{bv̩˩di˩}}}} 

\paragraph{\hspace{-0.5cm} {\Large \textcolor{darkblue}{\textbf{\ipa{bv̩˧}}}}} \hypertarget{bv\string_=\string_M1}{}
\markboth{\textcolor{darkblue}{\textbf{\ipa{bv̩˧}}}}{}
\textcolor{teal}{\mytextsc{nom}} \hspace{4pt} Ton~: M.

Intestin.
\textcolor{brown}{\zh{肠子。}}


 \mytextsc{clf}~: \textcolor{darkblue}{\textbf{\ipa{kʰɯ˩}}} 

\paragraph{\hspace{-0.5cm} {\Large \textcolor{darkblue}{\textbf{\ipa{bv̩˥}}}}} \hypertarget{bv\string_=\string_T1}{}
\markboth{\textcolor{darkblue}{\textbf{\ipa{bv̩˥}}}}{}
\textcolor{teal}{\mytextsc{nom}} \hspace{4pt} Ton~: \#H.

Ver; insecte.
\textcolor{brown}{\zh{虫。}}


\begin{exe}
\sn \textcolor{darkblue}{\textbf{\ipa{bv̩˧ tʰv̩˧-mi˧˥ / bv̩˧ tʰv̩˧-mi˥\#}}}
\trans \mytextsc{n}+\mytextsc{dem}+\mytextsc{clf}
\trans \textit{\textcolor{brown}{\zh{这只虫}}}
\end{exe}
 \mytextsc{clf}~: \textcolor{darkblue}{\textbf{\ipa{mi˩}}} 

\paragraph{\hspace{-0.5cm} {\Large \textcolor{darkblue}{\textbf{\ipa{bv̩˥}}} \textsubscript{1}}} \hypertarget{bv\string_=\string_T1}{}
\markboth{\textcolor{darkblue}{\textbf{\ipa{bv̩˥}}} \textsubscript{1}}{}
\textcolor{teal}{\mytextsc{adjectif}} \hspace{4pt} Ton~: H.

Épais (tronc); grossier (farine, poudre).
\textcolor{brown}{\zh{粗(树粗大,粉末不精细……)。}}


\begin{exe}
\sn \textcolor{darkblue}{\textbf{\ipa{qʰɑ˧-bv̩˧-gv̩˧}}}
\trans très épais
\trans \textit{\textcolor{brown}{\zh{很粗}}}
\end{exe}
\begin{exe}
\sn \textcolor{darkblue}{\textbf{\ipa{qʰɑ˧bv̩˧\textasciitilde{}bv̩˧-gv̩˧}}}
\trans très épais (idem ci-dessus)
\trans \textit{\textcolor{brown}{\zh{很粗(同上)}}}
\end{exe}


\paragraph{\hspace{-0.5cm} {\Large \textcolor{darkblue}{\textbf{\ipa{bv̩˥}}} \textsubscript{2}}} \hypertarget{bv\string_=\string_T2}{}
\markboth{\textcolor{darkblue}{\textbf{\ipa{bv̩˥}}} \textsubscript{2}}{}
\textcolor{teal}{\mytextsc{verbe}} \hspace{4pt} Ton~: H.

Partager, distribuer, répartir, diviser (ancien mot pour ``donner").
\textcolor{brown}{\zh{分东西。}}


\begin{exe}
\sn \textcolor{darkblue}{\textbf{\ipa{ɖɯ˧-v̩˧ ɖɯ˧-kʰwɤ˥ | le˧-bv̩˧\textasciitilde{}bv̩˧}}}
\trans partager, un morceau par personne
\trans \textit{\textcolor{brown}{\zh{分给一人一块}}}
\end{exe}
\begin{exe}
\sn \textcolor{darkblue}{\textbf{\ipa{le˧-bv̩˧\textasciitilde{}bv̩˧ tʰi˧-kwɤ˩}}}
\trans littéralement ``séparer et poser"; sens: mettre en désordre, disposer en désordre
\trans \textit{\textcolor{brown}{\zh{弄乱,散开}}}
\end{exe}


\paragraph{\hspace{-0.5cm} {\Large \textcolor{darkblue}{\textbf{\ipa{bv̩˥}}} \textsubscript{3}}} \hypertarget{bv\string_=\string_T3}{}
\markboth{\textcolor{darkblue}{\textbf{\ipa{bv̩˥}}} \textsubscript{3}}{}
\textcolor{teal}{\mytextsc{verbe}} \hspace{4pt} Ton~: H.

Griller, faire griller.
\textcolor{brown}{\zh{烤,炙。}}


\begin{exe}
\sn \textcolor{darkblue}{\textbf{\ipa{hɑ˧ tʰi˧-bv̩˥}}}
\trans faire griller du riz
\trans \textit{\textcolor{brown}{\zh{烤饭}}}
\end{exe}


\paragraph{\hspace{-0.5cm} {\Large \textcolor{darkblue}{\textbf{\ipa{bv̩˩\textsubscript{a}}}} \textsubscript{1}}} \hypertarget{bv\string_=\string_Ba1}{}
\markboth{\textcolor{darkblue}{\textbf{\ipa{bv̩˩\textsubscript{a}}}} \textsubscript{1}}{}
\textcolor{teal}{\mytextsc{verbe}} \hspace{4pt} Ton~: L\textsubscript{a}.

Couver.
\textcolor{brown}{\zh{孵。}}


\begin{exe}
\sn \textcolor{darkblue}{\textbf{\ipa{æ˩mi˧ bv̩˩}}}
\trans La poule couve.
\trans \textit{\textcolor{brown}{\zh{母鸡孵蛋}}}
\end{exe}


\paragraph{\hspace{-0.5cm} {\Large \textcolor{darkblue}{\textbf{\ipa{bv̩˩\textsubscript{a}}}} \textsubscript{2}}} \hypertarget{bv\string_=\string_Ba2}{}
\markboth{\textcolor{darkblue}{\textbf{\ipa{bv̩˩\textsubscript{a}}}} \textsubscript{2}}{}
\textcolor{teal}{\mytextsc{verbe}} \hspace{4pt} Ton~: L\textsubscript{a}.

Cuire à la vapeur, étuver.
\textcolor{brown}{\zh{蒸。}}


\begin{exe}
\sn \textcolor{darkblue}{\textbf{\ipa{le˧-bv̩˩-ze˩}}}
\trans \mytextsc{accomp} \string_ \mytextsc{pfv}
\trans \textit{\textcolor{brown}{\zh{蒸了}}}
\end{exe}
\begin{exe}
\sn \textcolor{darkblue}{\textbf{\ipa{pɤ˩jɤ˧ bv̩˥}}}
\trans cuire de la pâte à pain à la vapeur, faire des petits pains à la vapeur
\trans \textit{\textcolor{brown}{\zh{蒸馒头}}}
\end{exe}
\begin{exe}
\sn \textcolor{darkblue}{\textbf{\ipa{hɑ˧ bv̩˥\textasciitilde{}bv̩˩}}}
\trans cuire du riz à la vapeur
\trans \textit{\textcolor{brown}{\zh{蒸米饭}}}
\end{exe}


\paragraph{\hspace{-0.5cm} {\Large \textcolor{darkblue}{\textbf{\ipa{bv̩˩\textsubscript{a}}}} \textsubscript{3}}} \hypertarget{bv\string_=\string_Ba3}{}
\markboth{\textcolor{darkblue}{\textbf{\ipa{bv̩˩\textsubscript{a}}}} \textsubscript{3}}{}
\textcolor{teal}{\mytextsc{verbe}} \hspace{4pt} Ton~: L\textsubscript{a}.

Vivre, couler des jours, vivre (sa vie).
\textcolor{brown}{\zh{过(日子)。}}


\begin{exe}
\sn \textcolor{darkblue}{\textbf{\ipa{zɯ˧ bv̩˩}}}
\trans vivre sa vie
\trans \textit{\textcolor{brown}{\zh{过日子}}}
\end{exe}
\begin{exe}
\sn \textcolor{darkblue}{\textbf{\ipa{hĩ˧-zɯ˧ bv̩˥, | lo˧hɑ˧!}}}
\trans passer sa vie/vivre la vie des hommes, c'est difficile! / la vie est dure!
\trans \textit{\textcolor{brown}{\zh{生活,是艰难的!}}}
\end{exe}
\begin{exe}
\sn \textcolor{darkblue}{\textbf{\ipa{hĩ˧-zɯ˧ | le˧-bv̩˩-ze˩}}}
\trans (Sa) vie a passé! / (Sa) vie est terminée! (Réflexion après le décès de quelqu'un.)
\trans \textit{\textcolor{brown}{\zh{他的日子,就结束了!(情景:一个人去世了,葬礼的时候,有人这样说。)}}}
\end{exe}
\begin{exe}
\sn \textcolor{darkblue}{\textbf{\ipa{qʰwɤ˧-ɭɯ˥, | ʈʂʰæ˧-mɤ˧-dʑɯ˧! | ʈʂʰɯ˧ ɖɯ˧-zɯ˧ bv̩˩-ze˩!}}}
\trans Il n'a jamais eu à faire la vaisselle (de sa vie)! Voilà comment s'est passée toute sa vie! (Commentaire au sujet de la vie d'un mandarin qui s'était entièrement soustrait aux tâches manuelles.)
\trans \textit{\textcolor{brown}{\zh{他从来没有洗过碗!他这辈子就是这么过来的!(关于一个官员,完全不用管家务、日常生活中的活儿:有人来管一切。)}}}
\end{exe}
\begin{exe}
\sn \textcolor{darkblue}{\textbf{\ipa{ɖɯ˧-ɲi˥\textasciitilde{}ɖɯ˩-ɲi˩ | bv̩˩ lo˩ fv̩˩˥!}}}
\trans Les journées passent bien vite! / Comme le temps passe!
\trans \textit{\textcolor{brown}{\zh{日子过得真快!}}}
\end{exe}


\paragraph{\hspace{-0.5cm} {\Large \textcolor{darkblue}{\textbf{\ipa{bv̩˩\textsubscript{a}}}} \textsubscript{4}}} \hypertarget{bv\string_=\string_Ba4}{}
\markboth{\textcolor{darkblue}{\textbf{\ipa{bv̩˩\textsubscript{a}}}} \textsubscript{4}}{}
\textcolor{teal}{\mytextsc{verbe}} \hspace{4pt} Ton~: L\textsubscript{a}.
\ding{202} 
Asperger; arroser.
\textcolor{brown}{\zh{泼水,浇(浇菜)。}}


\begin{exe}
\sn \textcolor{darkblue}{\textbf{\ipa{le˧-bv̩˩-ze˩}}}
\trans \mytextsc{accomp} \string_ \mytextsc{pfv}
\trans \textit{\textcolor{brown}{\zh{泼了}}}
\end{exe}
\begin{exe}
\sn \textcolor{darkblue}{\textbf{\ipa{dʑɯ˩ bv̩˩˥}}}
\trans asperger d'eau; arroser
\trans \textit{\textcolor{brown}{\zh{泼水}}}
\end{exe}
\begin{exe}
\sn \textcolor{darkblue}{\textbf{\ipa{ɖɯ˧-bv̩˧\textasciitilde{}bv̩˥-ɻ̍˩}}}
\trans \mytextsc{délimitatif} \mytextsc{red} \mytextsc{inchoatif}
\trans \textit{\textcolor{brown}{\zh{泼一泼}}}
\end{exe}
\begin{exe}
\sn \textcolor{darkblue}{\textbf{\ipa{le˧-bv̩˧\textasciitilde{}bv̩˥-ze˩}}}
\trans \mytextsc{accomp} \mytextsc{red} \mytextsc{pfv}
\trans \textit{\textcolor{brown}{\zh{泼了一点}}}
\end{exe}
\ding{203} 
Disperser, semer (ex.: des graines).
\textcolor{brown}{\zh{撒(种子)。}}


\begin{exe}
\sn \textcolor{darkblue}{\textbf{\ipa{ɻæ˩ bv̩˥}}}
\trans semer à la volée, répandre des graines (pour les semailles)
\trans \textit{\textcolor{brown}{\zh{撒种子}}}
\end{exe}
\begin{exe}
\sn \textcolor{darkblue}{\textbf{\ipa{tʰi˧-bv̩˩-ɻ̍˩}}}
\trans Sème donc!
\trans \textit{\textcolor{brown}{\zh{撒吧!}}}
\end{exe}
\begin{exe}
\sn \textcolor{darkblue}{\textbf{\ipa{tʰi˧-bv̩˩-qɑ˩!}}}
\trans Sème!
\trans \textit{\textcolor{brown}{\zh{撒吧!}}}
\end{exe}


\paragraph{\hspace{-0.5cm} {\Large \textcolor{darkblue}{\textbf{\ipa{bv̩˩\textsubscript{a}}}} \textsubscript{5}}} \hypertarget{bv\string_=\string_Ba5}{}
\markboth{\textcolor{darkblue}{\textbf{\ipa{bv̩˩\textsubscript{a}}}} \textsubscript{5}}{}
\textcolor{teal}{\mytextsc{adjectif}} \hspace{4pt} Ton~: L\textsubscript{a}.

Clairsemé, à nu.
\textcolor{brown}{\zh{细、薄。}}


\begin{exe}
\sn \textcolor{darkblue}{\textbf{\ipa{ʁo˧ bv̩˧˥}}}
\trans chauve (littéralement ``la tête est à nu")
\trans \textit{\textcolor{brown}{\zh{头秃、头发很少}}}
\end{exe}
\begin{exe}
\sn \textcolor{darkblue}{\textbf{\ipa{ʁo˧-bv̩˧-hĩ˥}}}
\trans un homme chauve
\trans \textit{\textcolor{brown}{\zh{秃子}}}
\end{exe}
\begin{exe}
\sn \textcolor{darkblue}{\textbf{\ipa{ʁo˧-bv̩˧-zo˥}}}
\trans idem
\trans \textit{\textcolor{brown}{\zh{同上}}}
\end{exe}
\begin{exe}
\sn \textcolor{darkblue}{\textbf{\ipa{ʈʂʰɯ˧ | ʁo˧ bv̩˧-ze˥}}}
\trans il a perdu ses cheveux, il est devenu chauve
\trans \textit{\textcolor{brown}{\zh{他秃头了,他头发掉了}}}
\end{exe}
\begin{exe}
\sn \textcolor{darkblue}{\textbf{\ipa{ʁo˧qʰwɤ˩ | le˧-bv̩˩-ze˩}}}
\trans (sa) tête s'est dégarnie, (sa) tête est devenue chauve
\trans \textit{\textcolor{brown}{\zh{(他)秃头了。}}}
\end{exe}
\begin{exe}
\sn \textcolor{darkblue}{\textbf{\ipa{bv̩˩-hĩ˩˥}}}
\trans \mytextsc{rel}
\trans \textit{\textcolor{brown}{\zh{秃的}}}
\end{exe}


\paragraph{\hspace{-0.5cm} {\Large \textcolor{darkblue}{\textbf{\ipa{bv̩˧ɖæ˧}}}}} \hypertarget{bv\string_=\string_Md`\{\string_M1}{}
\markboth{\textcolor{darkblue}{\textbf{\ipa{bv̩˧ɖæ˧}}}}{}
\textcolor{teal}{\mytextsc{adjectif}} \hspace{4pt} Ton~: M.
\textit{De:} \textbf{bv̩˧ et ɖæ˥} 
De mauvaise humeur, ayant mauvais caractère.
\textcolor{brown}{\zh{脾气很坏。}}


\begin{exe}
\sn \textcolor{darkblue}{\textbf{\ipa{ʈʂʰɯ˧ | bv̩˧ɖæ˧-ze˩!}}}
\trans Il est de mauvais poil/Il est de mauvaise humeur!
\trans \textit{\textcolor{brown}{\zh{他现在脾气很坏!/ 他生气了!}}}
\end{exe}


\paragraph{\hspace{-0.5cm} {\Large \textcolor{darkblue}{\textbf{\ipa{bv̩˩di˩}}}}} \hypertarget{bv\string_=\string_Bdi\string_B1}{}
\markboth{\textcolor{darkblue}{\textbf{\ipa{bv̩˩di˩}}}}{}
\textcolor{teal}{\mytextsc{nom}} \hspace{4pt} Ton~: L.

Étuve.
\textcolor{brown}{\zh{蒸笼。}}


 \mytextsc{clf}~: \textcolor{darkblue}{\textbf{\ipa{ɭɯ˧}}} \textit{Voir~:} \hyperlink{bv\string_=\string_B\string_M2}{\textcolor{darkblue}{\textbf{\ipa{bv̩˩˧}}} \textsubscript{2}} 

\paragraph{\hspace{-0.5cm} {\Large \textcolor{darkblue}{\textbf{\ipa{bv̩˩dze˩}}} \textsubscript{1}}} \hypertarget{bv\string_=\string_Bdze\string_B1}{}
\markboth{\textcolor{darkblue}{\textbf{\ipa{bv̩˩dze˩}}} \textsubscript{1}}{}
\textcolor{teal}{\mytextsc{nom}} \hspace{4pt} Ton~: L.

Grosse cuillère (pour servir le riz, etc.).
\textcolor{brown}{\zh{大调羹。}}


 \mytextsc{clf}~: \textcolor{darkblue}{\textbf{\ipa{nɑ˧}}} \textit{Voir~:} \hyperlink{bv\string_=\string_Bdze\string_B2}{\textcolor{darkblue}{\textbf{\ipa{bv̩˩dze˩}}} \textsubscript{2}} 

\paragraph{\hspace{-0.5cm} {\Large \textcolor{darkblue}{\textbf{\ipa{bv̩˩dze˩}}} \textsubscript{2}}} \hypertarget{bv\string_=\string_Bdze\string_B2}{}
\markboth{\textcolor{darkblue}{\textbf{\ipa{bv̩˩dze˩}}} \textsubscript{2}}{}
\textcolor{teal}{\mytextsc{classificateur}} \hspace{4pt} Ton~: L.

Classificateur des cuillerée.
\textcolor{brown}{\zh{量词:勺。}}


\begin{exe}
\sn \textcolor{darkblue}{\textbf{\ipa{ɖɯ˧-bv̩˩dze˩}}}
\trans une louchée, une louche de
\trans \textit{\textcolor{brown}{\zh{一勺}}}
\end{exe}
\textit{Voir~:} \hyperlink{bv\string_=\string_Bdze\string_B1}{\textcolor{darkblue}{\textbf{\ipa{bv̩˩dze˩}}} \textsubscript{1}} 

\paragraph{\hspace{-0.5cm} {\Large \textcolor{darkblue}{\textbf{\ipa{bv̩˧hu˧˥}}}}} \hypertarget{bv\string_=\string_Mhu\string_M\string_T1}{}
\markboth{\textcolor{darkblue}{\textbf{\ipa{bv̩˧hu˧˥}}}}{}
\textcolor{teal}{\mytextsc{nom}} \hspace{4pt} Ton~: MH\#.

Tube digestif: estomac+intestin.
\textcolor{brown}{\zh{胃与肠。}}


 \mytextsc{clf}~: \textcolor{darkblue}{\textbf{\ipa{kwɤ˩}}} 

\paragraph{\hspace{-0.5cm} {\Large \textcolor{darkblue}{\textbf{\ipa{bv̩˩hwɤ˩}}}}} \hypertarget{bv\string_=\string_Bhw7\string_B1}{}
\markboth{\textcolor{darkblue}{\textbf{\ipa{bv̩˩hwɤ˩}}}}{}
\textcolor{teal}{\mytextsc{nom}} \hspace{4pt} Ton~: L.

Cabane de berger, sur la montagne; n'est pas occupée à l'année, mais est assez solide pour être utilisée année après année, à la différence des cabanes provisoires construites lorsqu'on doit rester qq jours sur la montagne pour couper du bois.
\textcolor{brown}{\zh{山上放牧的人暂时住的木头小房。}}


 \mytextsc{clf}~: \textcolor{darkblue}{\textbf{\ipa{ɭɯ˧}}} 

\paragraph{\hspace{-0.5cm} {\Large \textcolor{darkblue}{\textbf{\ipa{bv̩˧kʰɯ˧˥}}}}} \hypertarget{bv\string_=\string_Mk\string_hM\string_M\string_T1}{}
\markboth{\textcolor{darkblue}{\textbf{\ipa{bv̩˧kʰɯ˧˥}}}}{}
\textcolor{teal}{\mytextsc{nom}} \hspace{4pt} Ton~: MH\#.

Ver à soie.
\textcolor{brown}{\zh{蚕。}}


 \mytextsc{clf}~: \textcolor{darkblue}{\textbf{\ipa{kʰɯ˩}}} 

\paragraph{\hspace{-0.5cm} {\Large \textcolor{darkblue}{\textbf{\ipa{bv̩˩ɭɯ˩}}}}} \hypertarget{bv\string_=\string_Bl\string_RM\string_B1}{}
\markboth{\textcolor{darkblue}{\textbf{\ipa{bv̩˩ɭɯ˩}}}}{}
\textcolor{teal}{\mytextsc{nom}} \hspace{4pt} Ton~: L.

Rein.
\textcolor{brown}{\zh{肾。}}


 \mytextsc{clf}~: \textcolor{darkblue}{\textbf{\ipa{ɭɯ˧}}} 

\paragraph{\hspace{-0.5cm} {\Large \textcolor{darkblue}{\textbf{\ipa{bv̩˧mi˧}}} \textsubscript{1}}} \hypertarget{bv\string_=\string_Mmi\string_M1}{}
\markboth{\textcolor{darkblue}{\textbf{\ipa{bv̩˧mi˧}}} \textsubscript{1}}{}
\textcolor{teal}{\mytextsc{nom}} \hspace{4pt} Ton~: M.

Yak femelle, dri, drimo, nak.
\textcolor{brown}{\zh{母牦牛。}}


\begin{exe}
\sn \textcolor{darkblue}{\textbf{\ipa{bv̩˧mi˧-bv̩˩ʂwæ˩}}}
\trans yak femelle et yak châtré
\trans \textit{\textcolor{brown}{\zh{母牦牛与阉割牦牛}}}
\end{exe}
\begin{exe}
\sn \textcolor{darkblue}{\textbf{\ipa{bv̩˧mi˧-bv̩˧zo\#˥}}}
\trans maman yack et petit yack
\trans \textit{\textcolor{brown}{\zh{母牦牛与小牦牛}}}
\end{exe}
 \mytextsc{clf}~: \textcolor{darkblue}{\textbf{\ipa{mi˩}}} 

\paragraph{\hspace{-0.5cm} {\Large \textcolor{darkblue}{\textbf{\ipa{bv̩˧mi˧}}} \textsubscript{2}}} \hypertarget{bv\string_=\string_Mmi\string_M2}{}
\markboth{\textcolor{darkblue}{\textbf{\ipa{bv̩˧mi˧}}} \textsubscript{2}}{}
\textcolor{teal}{\mytextsc{nom}} \hspace{4pt} Ton~: M.

Grande étuve.
\textcolor{brown}{\zh{大蒸笼。}}


 \mytextsc{clf}~: \textcolor{darkblue}{\textbf{\ipa{mi˩}}} 

\paragraph{\hspace{-0.5cm} {\Large \textcolor{darkblue}{\textbf{\ipa{bv̩˧-nɑ˥mi˩}}}}} \hypertarget{bv\string_=\string_M-nA\string_Tmi\string_B1}{}
\markboth{\textcolor{darkblue}{\textbf{\ipa{bv̩˧-nɑ˥mi˩}}}}{}
\textcolor{teal}{\mytextsc{nom}} \hspace{4pt} Ton~: \#H-.

\textit{Mythimna separata (Walker)}.
\textcolor{brown}{\zh{玉米黏虫。}}




\paragraph{\hspace{-0.5cm} {\Large \textcolor{darkblue}{\textbf{\ipa{bv̩˧nv̩˧}}} \textsubscript{1}}} \hypertarget{bv\string_=\string_Mnv\string_=\string_M1}{}
\markboth{\textcolor{darkblue}{\textbf{\ipa{bv̩˧nv̩˧}}} \textsubscript{1}}{}
\textcolor{teal}{\mytextsc{verbe}} \hspace{4pt} Ton~: M intrans.

Sentir (par l'odorat).
\textcolor{brown}{\zh{闻到(嗅觉)。}}


\begin{exe}
\sn \textcolor{darkblue}{\textbf{\ipa{no˧ | ɖɯ˧-bv̩˧nv̩˧-ɻ̍˩! | ɖwæ˩˥ | ɕjɤ˧!}}}
\trans Sens donc! ça sent très bon/c'est très odorant/parfumé!
\trans \textit{\textcolor{brown}{\zh{你闻一闻吧!好香!}}}
\end{exe}
\textit{Voir~:} \hyperlink{bv\string_=\string_Mnv\string_=\string_M2}{\textcolor{darkblue}{\textbf{\ipa{bv̩˧nv̩˧}}} \textsubscript{2}} 

\paragraph{\hspace{-0.5cm} {\Large \textcolor{darkblue}{\textbf{\ipa{bv̩˧nv̩˧}}} \textsubscript{2}}} \hypertarget{bv\string_=\string_Mnv\string_=\string_M2}{}
\markboth{\textcolor{darkblue}{\textbf{\ipa{bv̩˧nv̩˧}}} \textsubscript{2}}{}
\textcolor{teal}{\mytextsc{adjectif}} \hspace{4pt} Ton~: M intrans.

Malodorant, puant, qui a une mauvaise odeur.
\textcolor{brown}{\zh{臭。}}


\begin{exe}
\sn \textcolor{darkblue}{\textbf{\ipa{bv̩˧nv̩˧-ze˧}}}
\trans \mytextsc{pfv}
\trans \textit{\textcolor{brown}{\zh{臭了}}}
\end{exe}
\textit{Voir~:} \hyperlink{bv\string_=\string_Mnv\string_=\string_M1}{\textcolor{darkblue}{\textbf{\ipa{bv̩˧nv̩˧}}} \textsubscript{1}} 

\paragraph{\hspace{-0.5cm} {\Large \textcolor{darkblue}{\textbf{\ipa{bv̩˧pʰv̩˧}}}}} \hypertarget{bv\string_=\string_Mp\string_hv\string_=\string_M1}{}
\markboth{\textcolor{darkblue}{\textbf{\ipa{bv̩˧pʰv̩˧}}}}{}
\textcolor{teal}{\mytextsc{nom}} \hspace{4pt} Ton~: M.

Yak mâle. Ce mot est une forme élicitée; la forme usuelle est: \textcolor{darkblue}{\textbf{\ipa{/bv̩˩ʂwæ˩/}}}.
\textcolor{brown}{\zh{公牦牛。}}


 \mytextsc{clf}~: \textcolor{darkblue}{\textbf{\ipa{mi˩}}} 

\paragraph{\hspace{-0.5cm} {\Large \textcolor{darkblue}{\textbf{\ipa{bv̩˩-qʰæ˩}}}}} \hypertarget{bv\string_=\string_B-q\string_h\{\string_B1}{}
\markboth{\textcolor{darkblue}{\textbf{\ipa{bv̩˩-qʰæ˩}}}}{}
\textcolor{teal}{\mytextsc{nom}} \hspace{4pt} Ton~: L.

Fumier.
\textcolor{brown}{\zh{肥料、粪。}}


\begin{exe}
\sn \textcolor{darkblue}{\textbf{\ipa{bv̩˩qʰæ˩ tʰv̩˩-ʁwɤ˥}}}
\trans \mytextsc{n}+\mytextsc{dem}+\mytextsc{clf}
\trans \textit{\textcolor{brown}{\zh{这堆肥料}}}
\end{exe}
 \mytextsc{clf}~: \textcolor{darkblue}{\textbf{\ipa{ʁwɤ˧}}} tas

\paragraph{\hspace{-0.5cm} {\Large \textcolor{darkblue}{\textbf{\ipa{bv̩˩qo˩-bv̩˧qʰæ˩}}}}} \hypertarget{bv\string_=\string_Bqo\string_B-bv\string_=\string_Mq\string_h\{\string_B1}{}
\markboth{\textcolor{darkblue}{\textbf{\ipa{bv̩˩qo˩-bv̩˧qʰæ˩}}}}{}
\textcolor{teal}{\mytextsc{nom}} \hspace{4pt} Ton~: L-L\#.

Fumier.
\textcolor{brown}{\zh{肥料、粪。}}


 \mytextsc{clf}~: \textcolor{darkblue}{\textbf{\ipa{ʁwɤ˧}}} 

\paragraph{\hspace{-0.5cm} {\Large \textcolor{darkblue}{\textbf{\ipa{bv̩˩qo˩-qʰæ˩}}}}} \hypertarget{bv\string_=\string_Bqo\string_B-q\string_h\{\string_B1}{}
\markboth{\textcolor{darkblue}{\textbf{\ipa{bv̩˩qo˩-qʰæ˩}}}}{}
\textcolor{teal}{\mytextsc{nom}} \hspace{4pt} Ton~: L.

Fumier.
\textcolor{brown}{\zh{肥料 , 粪。}}


\begin{exe}
\sn \textcolor{darkblue}{\textbf{\ipa{bv̩˩qo˩-qʰæ˩ tʰv̩˩-ʁwɤ˥}}}
\trans \mytextsc{n}+\mytextsc{dem}+\mytextsc{clf}
\trans \textit{\textcolor{brown}{\zh{这堆肥料}}}
\end{exe}
 \mytextsc{clf}~: \textcolor{darkblue}{\textbf{\ipa{ʁwɤ˧}}} 

\paragraph{\hspace{-0.5cm} {\Large \textcolor{darkblue}{\textbf{\ipa{bv̩˩qʰv̩˩}}}}} \hypertarget{bv\string_=\string_Bq\string_hv\string_=\string_B1}{}
\markboth{\textcolor{darkblue}{\textbf{\ipa{bv̩˩qʰv̩˩}}}}{}
\textcolor{teal}{\mytextsc{nom}} \hspace{4pt} Ton~: L.
\ding{202} 
Conque, \textit{Turbinella pyrum L.}. On y souffle, comme dans une trompe, lors des cérémonies. Chaque famille en possède une paire.
\textcolor{brown}{\zh{法螺、海螺、螺号。}}


\ding{203} 
Lignes de la main (dont la forme en spirale évoque les conques, objet symbolique important dans la culture na).
\textcolor{brown}{\zh{掌纹。}}


\begin{exe}
\sn \textcolor{darkblue}{\textbf{\ipa{lo˩qʰwɤ˧-bv̩˧ | bv̩˩qʰv̩˩˥}}}
\trans les lignes de la main
\trans \textit{\textcolor{brown}{\zh{掌纹}}}
\end{exe}
 \mytextsc{clf}~: \textcolor{darkblue}{\textbf{\ipa{dze˩}}} paire; explication: va par paires, chaque maisonnée en a une paire

\paragraph{\hspace{-0.5cm} {\Large \textcolor{darkblue}{\textbf{\ipa{bv̩˧qʰv̩˧ʑi˩-hĩ˩}}}}} \hypertarget{bv\string_=\string_Mq\string_hv\string_=\string_Mz£i\string_B-hi\string_~\string_B1}{}
\markboth{\textcolor{darkblue}{\textbf{\ipa{bv̩˧qʰv̩˧ʑi˩-hĩ˩}}}}{}
\textcolor{teal}{\mytextsc{nom}} \hspace{4pt} Ton~: L\#.

Escargot.
\textcolor{brown}{\zh{蜗牛,螺蛳。}}


 \mytextsc{clf}~: \textcolor{darkblue}{\textbf{\ipa{mi˩}}} 

\paragraph{\hspace{-0.5cm} {\Large \textcolor{darkblue}{\textbf{\ipa{bv̩˧ɻ\#˥}}}}} \hypertarget{bv\string_=\string_Mr£`\#\string_T1}{}
\markboth{\textcolor{darkblue}{\textbf{\ipa{bv̩˧ɻ\#˥}}}}{}
\textcolor{teal}{\mytextsc{nom}} \hspace{4pt} Ton~: \#H.

Mouche.
\textcolor{brown}{\zh{苍蝇。}}


\begin{exe}
\sn \textcolor{darkblue}{\textbf{\ipa{bv̩˧ɻ̍˧ ʈʂʰɯ˧-mi˧˥ / bv̩˧ɻ̍˧ ʈʂʰɯ˧-mi˥\#}}}
\trans \mytextsc{n}+\mytextsc{dem}+\mytextsc{clf}
\trans \textit{\textcolor{brown}{\zh{这只苍蝇}}}
\end{exe}
 \mytextsc{clf}~: \textcolor{darkblue}{\textbf{\ipa{mi˩}}} 

\paragraph{\hspace{-0.5cm} {\Large \textcolor{darkblue}{\textbf{\ipa{bv̩˧ʂæ˧}}}}} \hypertarget{bv\string_=\string_Ms`\{\string_M1}{}
\markboth{\textcolor{darkblue}{\textbf{\ipa{bv̩˧ʂæ˧}}}}{}
\textcolor{teal}{\mytextsc{adjectif}} \hspace{4pt} Ton~: M.
\textit{De:} \textbf{bv̩˧ et ʂæ˧} 
De bonne humeur, ayant bon caractère.
\textcolor{brown}{\zh{脾气好。}}


\begin{exe}
\sn \textcolor{darkblue}{\textbf{\ipa{ʈʂʰɯ˧ | bv̩˧ʂæ˧-ze˩}}}
\trans il est de bonne humeur
\trans \textit{\textcolor{brown}{\zh{他现在脾气好。/ 他高兴了。}}}
\end{exe}
\begin{exe}
\sn \textcolor{darkblue}{\textbf{\ipa{bv̩˧ʂæ˧ | ʐwæ˩˥}}}
\trans de très bonne humeur
\trans \textit{\textcolor{brown}{\zh{脾气很好}}}
\end{exe}


\paragraph{\hspace{-0.5cm} {\Large \textcolor{darkblue}{\textbf{\ipa{bv̩˩ʂwæ˩}}}}} \hypertarget{bv\string_=\string_Bs`w\{\string_B1}{}
\markboth{\textcolor{darkblue}{\textbf{\ipa{bv̩˩ʂwæ˩}}}}{}
\textcolor{teal}{\mytextsc{nom}} \hspace{4pt} Ton~: L.

Yak châtré.
\textcolor{brown}{\zh{阉割过的牦牛。}}


\begin{exe}
\sn \textcolor{darkblue}{\textbf{\ipa{bv̩˩ʂwæ˩-bv̩˥mi˩}}}
\trans yak châtré et yak femelle
\trans \textit{\textcolor{brown}{\zh{阉割过的公牦牛与母牦牛}}}
\end{exe}
 \mytextsc{clf}~: \textcolor{darkblue}{\textbf{\ipa{pʰo˧˥}}} 

\paragraph{\hspace{-0.5cm} {\Large \textcolor{darkblue}{\textbf{\ipa{bv̩˧tɕi˧}}}}} \hypertarget{bv\string_=\string_Mts£i\string_M1}{}
\markboth{\textcolor{darkblue}{\textbf{\ipa{bv̩˧tɕi˧}}}}{}
\textcolor{teal}{\mytextsc{nom}} \hspace{4pt} Ton~: M.

Pêche sauvage (de petite taille).
\textcolor{brown}{\zh{毛桃。}}


 \mytextsc{clf}~: \textcolor{darkblue}{\textbf{\ipa{tɕi˧˥}}} \textcolor{darkblue}{\textbf{\ipa{ɭɯ˧}}} livre; unité

\paragraph{\hspace{-0.5cm} {\Large \textcolor{darkblue}{\textbf{\ipa{bv̩˧ʈʂɯ˥}}}}} \hypertarget{bv\string_=\string_Mt`s`M\string_T1}{}
\markboth{\textcolor{darkblue}{\textbf{\ipa{bv̩˧ʈʂɯ˥}}}}{}
\textcolor{teal}{\mytextsc{nom}} \hspace{4pt} Ton~: H\#.

Vanneries: tamis où l'on fait sécher les graines de courge et autres produits de la ferme.
\textcolor{brown}{\zh{筛子。}}


 \mytextsc{clf}~: \textcolor{darkblue}{\textbf{\ipa{nɑ˧}}} 

\paragraph{\hspace{-0.5cm} {\Large \textcolor{darkblue}{\textbf{\ipa{bv̩˧ʈʂʰv̩˧}}}}} \hypertarget{bv\string_=\string_Mt`s`\string_hv\string_=\string_M1}{}
\markboth{\textcolor{darkblue}{\textbf{\ipa{bv̩˧ʈʂʰv̩˧}}}}{}
\textcolor{teal}{\mytextsc{nom}} \hspace{4pt} Ton~: M.

Cymbales.
\textcolor{brown}{\zh{钹。}}


 \mytextsc{clf}~: \textcolor{darkblue}{\textbf{\ipa{nɑ˧}}} 

\paragraph{\hspace{-0.5cm} {\Large \textcolor{darkblue}{\textbf{\ipa{bv̩˩zo˩}}}}} \hypertarget{bv\string_=\string_Bzo\string_B1}{}
\markboth{\textcolor{darkblue}{\textbf{\ipa{bv̩˩zo˩}}}}{}
\textcolor{teal}{\mytextsc{nom}} \hspace{4pt} Ton~: L.

Petite étuve.
\textcolor{brown}{\zh{小蒸笼。}}


 \mytextsc{clf}~: \textcolor{darkblue}{\textbf{\ipa{ɭɯ˧}}} 

\paragraph{\hspace{-0.5cm} {\Large \textcolor{darkblue}{\textbf{\ipa{bv̩˧zo\#˥}}}}} \hypertarget{bv\string_=\string_Mzo\#\string_T1}{}
\markboth{\textcolor{darkblue}{\textbf{\ipa{bv̩˧zo\#˥}}}}{}
\textcolor{teal}{\mytextsc{nom}} \hspace{4pt} Ton~: \#H.

Petit du yak.
\textcolor{brown}{\zh{小牦牛。}}


\begin{exe}
\sn \textcolor{darkblue}{\textbf{\ipa{bv̩˧zo˧ tʰv̩˧-mi˧˥ / bv̩˧zo˧ tʰv̩˧-mi˥\#}}}
\trans \mytextsc{n}+\mytextsc{dem}+\mytextsc{clf}
\trans \textit{\textcolor{brown}{\zh{这头小牦牛}}}
\end{exe}
 \mytextsc{clf}~: \textcolor{darkblue}{\textbf{\ipa{mi˩}}} 

\paragraph{\hspace{-0.5cm} {\Large \textcolor{darkblue}{\textbf{\ipa{bv̩˩ʐv̩˩-dzi˩}}}}} \hypertarget{bv\string_=\string_Bz`v\string_=\string_B-dzi\string_B1}{}
\markboth{\textcolor{darkblue}{\textbf{\ipa{bv̩˩ʐv̩˩-dzi˩}}}}{}
\textcolor{teal}{\mytextsc{nom}} \hspace{4pt} Ton~: L.

Lierre.
\textcolor{brown}{\zh{常春藤。}}


 \mytextsc{clf}~: \textcolor{darkblue}{\textbf{\ipa{dzi˩}}} 

\paragraph{\hspace{-0.5cm} {\Large \textcolor{darkblue}{\textbf{\ipa{bv̩˧ʐv̩˧-kʰv̩˧˥}}}}} \hypertarget{bv\string_=\string_Mz`v\string_=\string_M-k\string_hv\string_=\string_M\string_T1}{}
\markboth{\textcolor{darkblue}{\textbf{\ipa{bv̩˧ʐv̩˧-kʰv̩˧˥}}}}{}
\textcolor{teal}{\mytextsc{nom}} \hspace{4pt} Ton~: MH\#.

Année du serpent.
\textcolor{brown}{\zh{蛇年。}}


\textit{Voir~:} \hyperlink{z`v\string_=\string_Mb\{\string_M1}{\textcolor{darkblue}{\textbf{\ipa{ʐv̩˧bæ˧}}}} 

\paragraph{\hspace{-0.5cm} {\Large \textcolor{darkblue}{\textbf{\ipa{‑bv˧}}}}} \hypertarget{‑bv\string_M1}{}
\markboth{\textcolor{darkblue}{\textbf{\ipa{‑bv˧}}}}{}
\textcolor{teal}{\mytextsc{suffixe}} \hspace{4pt} Ton~: 0.

Possessif.
\textcolor{brown}{\zh{属式:的。}}




\newpage
\section*{\centering- \textcolor{darkblue}{\textbf{\ipa{ɕ}}} -}
\paragraph{\hspace{-0.5cm} {\Large \textcolor{darkblue}{\textbf{\ipa{ɕi˥\textsubscript{a}}}}}} \hypertarget{s£i\string_Ta1}{}
\markboth{\textcolor{darkblue}{\textbf{\ipa{ɕi˥\textsubscript{a}}}}}{}
\textcolor{teal}{\mytextsc{classificateur}} \hspace{4pt} Ton~: H\textsubscript{a}.

100.
\textcolor{brown}{\zh{百。}}


\begin{exe}
\sn \textcolor{darkblue}{\textbf{\ipa{ɖɯ˧-ɕi˥}}}
\trans cent
\trans \textit{\textcolor{brown}{\zh{一百}}}
\end{exe}
\begin{exe}
\sn \textcolor{darkblue}{\textbf{\ipa{ɖɯ˧-ɕi˧ kʰv̩˧˥}}}
\trans cent ans, un siècle
\trans \textit{\textcolor{brown}{\zh{一百年}}}
\end{exe}
\begin{exe}
\sn \textcolor{darkblue}{\textbf{\ipa{ɖɯ˧-ɕi˧ kʰv̩˧\textasciitilde{}ɖɯ˥-ɕi˩ kʰv̩˩}}}
\trans siècle après siècle
\trans \textit{\textcolor{brown}{\zh{一百年又一百年}}}
\end{exe}
\begin{exe}
\sn \textcolor{darkblue}{\textbf{\ipa{ɕi˧-kʰv̩˧˥}}}
\trans cent ans (formulation abrégée)
\trans \textit{\textcolor{brown}{\zh{百年(“一百年”的省略说法)}}}
\end{exe}


\paragraph{\hspace{-0.5cm} {\Large \textcolor{darkblue}{\textbf{\ipa{ɕi˥\textsubscript{b}}}}}} \hypertarget{s£i\string_Tb1}{}
\markboth{\textcolor{darkblue}{\textbf{\ipa{ɕi˥\textsubscript{b}}}}}{}
\textcolor{teal}{\mytextsc{classificateur}} \hspace{4pt} Ton~: H\textsubscript{b}.

Centième d'unité monétaire.
\textcolor{brown}{\zh{量词:分(一分钱)。}}




\paragraph{\hspace{-0.5cm} {\Large \textcolor{darkblue}{\textbf{\ipa{ɕi˧}}}}} \hypertarget{s£i\string_M1}{}
\markboth{\textcolor{darkblue}{\textbf{\ipa{ɕi˧}}}}{}
\textcolor{teal}{\mytextsc{nom}} \hspace{4pt} Ton~: M.

Riz (monosyllabe).
\textcolor{brown}{\zh{米(单音节)。}}




\paragraph{\hspace{-0.5cm} {\Large \textcolor{darkblue}{\textbf{\ipa{ɕi˧ɕi˩-lo˩}}}}} \hypertarget{s£i\string_Ms£i\string_B-lo\string_B1}{}
\markboth{\textcolor{darkblue}{\textbf{\ipa{ɕi˧ɕi˩-lo˩}}}}{}
\textcolor{teal}{\mytextsc{nom}} \hspace{4pt} Ton~: L\#-.

Les plus petites des côtelettes.
\textcolor{brown}{\zh{最细的肋骨。}}




\paragraph{\hspace{-0.5cm} {\Large \textcolor{darkblue}{\textbf{\ipa{ɕi˧-ho˩ʂɯ˩}}}}} \hypertarget{s£i\string_M-ho\string_Bs`M\string_B1}{}
\markboth{\textcolor{darkblue}{\textbf{\ipa{ɕi˧-ho˩ʂɯ˩}}}}{}
\textcolor{teal}{\mytextsc{nom}} \hspace{4pt} Ton~: -L.

Tomate.
\textcolor{brown}{\zh{西红柿(汉语借词)。}}

 Emprunt~: chinois \textcolor{brown}{\zh{西红柿}}



\paragraph{\hspace{-0.5cm} {\Large \textcolor{darkblue}{\textbf{\ipa{ɕi˧lv̩˧}}}}} \hypertarget{s£i\string_Mlv\string_=\string_M1}{}
\markboth{\textcolor{darkblue}{\textbf{\ipa{ɕi˧lv̩˧}}}}{}
\textcolor{teal}{\mytextsc{nom}} \hspace{4pt} Ton~: M.

Champs de riz.
\textcolor{brown}{\zh{水田。}}


 \mytextsc{clf}~: \textcolor{darkblue}{\textbf{\ipa{pʰæ˧˥, kɤ˧˥}}} 

\paragraph{\hspace{-0.5cm} {\Large \textcolor{darkblue}{\textbf{\ipa{ɕi˧lv̩˧-mv̩˧di˧˥}}}}} \hypertarget{s£i\string_Mlv\string_=\string_M-mv\string_=\string_Mdi\string_M\string_T1}{}
\markboth{\textcolor{darkblue}{\textbf{\ipa{ɕi˧lv̩˧-mv̩˧di˧˥}}}}{}
\textcolor{teal}{\mytextsc{nom}} \hspace{4pt} Ton~: \mytextsc{MH}\#.

Champs de riz.
\textcolor{brown}{\zh{水田。}}


 \mytextsc{clf}~: \textcolor{darkblue}{\textbf{\ipa{pʰæ˧˥, kɤ˧˥}}} 

\paragraph{\hspace{-0.5cm} {\Large \textcolor{darkblue}{\textbf{\ipa{ɕi˧ɭɯ˧}}}}} \hypertarget{s£i\string_Ml\string_RM\string_M1}{}
\markboth{\textcolor{darkblue}{\textbf{\ipa{ɕi˧ɭɯ˧}}}}{}
\textcolor{teal}{\mytextsc{nom}} \hspace{4pt} Ton~: M.

Riz paddy; par extension: champ de riz.
\textcolor{brown}{\zh{稻子,稻田。}}


 \mytextsc{clf}~: \textcolor{darkblue}{\textbf{\ipa{kɤ˧˥ (pour le grain)}}} \textcolor{darkblue}{\textbf{\ipa{pʰæ˧˥ (pour un champ)}}} 

\paragraph{\hspace{-0.5cm} {\Large \textcolor{darkblue}{\textbf{\ipa{ɕi˧ɭɯ˧-lv̩˧pʰv̩˩}}}}} \hypertarget{s£i\string_Ml\string_RM\string_M-lv\string_=\string_Mp\string_hv\string_=\string_B1}{}
\markboth{\textcolor{darkblue}{\textbf{\ipa{ɕi˧ɭɯ˧-lv̩˧pʰv̩˩}}}}{}
\textcolor{teal}{\mytextsc{nom}} \hspace{4pt} Ton~: \mytextsc{L}\#.

Champs de riz.
\textcolor{brown}{\zh{水田。}}


 \mytextsc{clf}~: \textcolor{darkblue}{\textbf{\ipa{pʰæ˧˥, kɤ˧˥}}} 

\paragraph{\hspace{-0.5cm} {\Large \textcolor{darkblue}{\textbf{\ipa{ɕi˧tv̩˧di˩-lv̩˧}}}}} \hypertarget{s£i\string_Mtv\string_=\string_Mdi\string_B-lv\string_=\string_M1}{}
\markboth{\textcolor{darkblue}{\textbf{\ipa{ɕi˧tv̩˧di˩-lv̩˧}}}}{}
\textcolor{teal}{\mytextsc{nom}} \hspace{4pt} Ton~: \mytextsc{L}.

Champs de riz.
\textcolor{brown}{\zh{水田。}}


 \mytextsc{clf}~: \textcolor{darkblue}{\textbf{\ipa{pʰæ˧˥, kɤ˧˥}}} 

\paragraph{\hspace{-0.5cm} {\Large \textcolor{darkblue}{\textbf{\ipa{ɕi˧tɕʰi\#˥}}}}} \hypertarget{s£i\string_Mts£\string_hi\#\string_T1}{}
\markboth{\textcolor{darkblue}{\textbf{\ipa{ɕi˧tɕʰi\#˥}}}}{}
\textcolor{teal}{\mytextsc{nom}} \hspace{4pt} Ton~: \#H.

Son de riz.
\textcolor{brown}{\zh{米糠。}}


 \mytextsc{clf}~: \textcolor{darkblue}{\textbf{\ipa{mɤ˩}}} 

\paragraph{\hspace{-0.5cm} {\Large \textcolor{darkblue}{\textbf{\ipa{ɕi˧ʈʂʰwæ˧}}}}} \hypertarget{s£i\string_Mt`s`\string_hw\{\string_M1}{}
\markboth{\textcolor{darkblue}{\textbf{\ipa{ɕi˧ʈʂʰwæ˧}}}}{}
\textcolor{teal}{\mytextsc{nom}} \hspace{4pt} Ton~: M.

Riz décortiqué.
\textcolor{brown}{\zh{米。}}


\begin{exe}
\sn \textcolor{darkblue}{\textbf{\ipa{ɕi˧ʈʂʰwæ˧-hɑ˧}}}
\trans riz cuit; littéralement: ``nourriture-riz cuit"; formulation employée pour préciser le terme \textcolor{darkblue}{\textbf{\ipa{/hɑ˥/}}}, qui désigne toutes les nourritures.
\trans \textit{\textcolor{brown}{\zh{米饭}}}
\end{exe}


\paragraph{\hspace{-0.5cm} {\Large \textcolor{darkblue}{\textbf{\ipa{ɕi˩}}}}} \hypertarget{s£i\string_B1}{}
\markboth{\textcolor{darkblue}{\textbf{\ipa{ɕi˩}}}}{}
\textcolor{teal}{\mytextsc{verbe}} \hspace{4pt} Ton~: L\textsubscript{a}?.
\textit{Archaïque} [\textcolor{brown}{\zh{古语}}] 
Craindre, avoir peur de. Verbe qui paraît suranné; il ne se trouve que dans quelques expressions.
\textcolor{brown}{\zh{怕、害怕。}}


\begin{exe}
\sn \textcolor{darkblue}{\textbf{\ipa{njɤ˧ | no˩ ɕi˩ tʰɑ˥-mɤ˩-ʝi˩! | njɤ˧ | no˩ ɖwæ˩ tʰɑ˥-mɤ˩-ʝi˩!}}}
\trans Ne va pas croire que tu me fasses peur! / Si tu crois que j'ai peur de toi! Si tu crois que je te crains/que tu me fais peur! (Formule de défi.)
\trans \textit{\textcolor{brown}{\zh{不要以为我害怕你!(挑衅的话)}}}
\end{exe}
\begin{exe}
\sn \textcolor{darkblue}{\textbf{\ipa{njɤ˧ | no˩ ɕi˩-mɤ˩-ʝi˥!}}}
\trans Tu ne me fais pas peur!
\trans \textit{\textcolor{brown}{\zh{你不让我害怕 / 我不害怕你!}}}
\end{exe}
\begin{exe}
\sn \textcolor{darkblue}{\textbf{\ipa{njɤ˧ | tʰv̩˧ ɕi˩-mɤ˩-ʝi˩!}}}
\trans Il ne me fait pas peur!
\trans \textit{\textcolor{brown}{\zh{我不怕他!}}}
\end{exe}
\begin{exe}
\sn \textcolor{darkblue}{\textbf{\ipa{njɤ˧ | ʈʂʰɯ˧-v̩˧ do˧˥, | ʁo˧ ɕi˧˥ | ʐwæ˩˥!}}}
\trans Quand je le vois, j'ai peur!/Il me fait très peur!
\trans \textit{\textcolor{brown}{\zh{我见他,非常害怕!}}}
\end{exe}


\paragraph{\hspace{-0.5cm} {\Large \textcolor{darkblue}{\textbf{\ipa{ɕi˩dv̩˥}}}}} \hypertarget{s£i\string_Bdv\string_=\string_T1}{}
\markboth{\textcolor{darkblue}{\textbf{\ipa{ɕi˩dv̩˥}}}}{}
\textcolor{teal}{\mytextsc{nom}} \hspace{4pt} Ton~: LH.

Encens; bâtonnet d'encens.
\textcolor{brown}{\zh{香,香火。}}


\begin{exe}
\sn \textcolor{darkblue}{\textbf{\ipa{ɕi˩dv̩˥ qæ˩}}}
\trans brûler de l'encens
\trans \textit{\textcolor{brown}{\zh{烧香}}}
\end{exe}


\paragraph{\hspace{-0.5cm} {\Large \textcolor{darkblue}{\textbf{\ipa{ɕi˩dzi˥}}}}} \hypertarget{s£i\string_Bdzi\string_T1}{}
\markboth{\textcolor{darkblue}{\textbf{\ipa{ɕi˩dzi˥}}}}{}
\textcolor{teal}{\mytextsc{nom}} \hspace{4pt} Ton~: LH.

Genévrier; arbre dont des branchages sont employés lors des rituels (suivant la tradition tibétaine).
\textcolor{brown}{\zh{柏树。}}


 \mytextsc{clf}~: \textcolor{darkblue}{\textbf{\ipa{dzi˩}}} 

\paragraph{\hspace{-0.5cm} {\Large \textcolor{darkblue}{\textbf{\ipa{ɕi˩ʈʰæ˧˥}}} \textsubscript{1}}} \hypertarget{s£i\string_Bt`\string_h\{\string_M\string_T1}{}
\markboth{\textcolor{darkblue}{\textbf{\ipa{ɕi˩ʈʰæ˧˥}}} \textsubscript{1}}{}
\textcolor{teal}{\mytextsc{adjectif}} \hspace{4pt} Ton~: LM+MH\#.

Bègue, qui a un bégaiement.
\textcolor{brown}{\zh{结巴。}}


\begin{exe}
\sn \textcolor{darkblue}{\textbf{\ipa{ʈʂʰɯ˧ | ɖɯ˧-pi˧˥ | ɕi˩ʈʰæ˧˥}}}
\trans Il/elle est un peu bègue.
\trans \textit{\textcolor{brown}{\zh{他有一点结巴。}}}
\end{exe}
\begin{exe}
\sn \textcolor{darkblue}{\textbf{\ipa{ʈʂʰɯ˧ | ɕi˩ʈʰæ˧-zo˥.}}}
\trans Il est très bègue / il bégaie beaucoup / il a un fort bégaiement.
\trans \textit{\textcolor{brown}{\zh{他很结巴。}}}
\end{exe}
\textit{Voir~:} \hyperlink{s£i\string_Bt`\string_h\{\string_M\string_T2}{\textcolor{darkblue}{\textbf{\ipa{ɕi˩ʈʰæ˧˥}}} \textsubscript{2}} 

\paragraph{\hspace{-0.5cm} {\Large \textcolor{darkblue}{\textbf{\ipa{ɕi˩ʈʰæ˧˥}}} \textsubscript{2}}} \hypertarget{s£i\string_Bt`\string_h\{\string_M\string_T2}{}
\markboth{\textcolor{darkblue}{\textbf{\ipa{ɕi˩ʈʰæ˧˥}}} \textsubscript{2}}{}
\textcolor{teal}{\mytextsc{nom}} \hspace{4pt} Ton~: LM+MH\#.

Bègue.
\textcolor{brown}{\zh{结巴。}}


\begin{exe}
\sn \textcolor{darkblue}{\textbf{\ipa{ʈʂʰɯ˧ ɕi˩ʈʰæ˧ ɲi˥}}}
\trans Il/elle est bègue.
\trans \textit{\textcolor{brown}{\zh{他是结巴。}}}
\end{exe}
\textit{Voir~:} \hyperlink{s£i\string_Bt`\string_h\{\string_M\string_T1}{\textcolor{darkblue}{\textbf{\ipa{ɕi˩ʈʰæ˧˥}}} \textsubscript{1}} 

\paragraph{\hspace{-0.5cm} {\Large \textcolor{darkblue}{\textbf{\ipa{ɕi˩˥}}}}} \hypertarget{s£i\string_B\string_T1}{}
\markboth{\textcolor{darkblue}{\textbf{\ipa{ɕi˩˥}}}}{}
\textcolor{teal}{\mytextsc{nom}} \hspace{4pt} Ton~: LH.

Encens (monosyllabe).
\textcolor{brown}{\zh{香(单音节)。}}


\begin{exe}
\sn \textcolor{darkblue}{\textbf{\ipa{ɕi˩ qæ˧˥}}}
\trans brûler de l'encens
\trans \textit{\textcolor{brown}{\zh{烧香}}}
\end{exe}


\paragraph{\hspace{-0.5cm} {\Large \textcolor{darkblue}{\textbf{\ipa{ɕjɤ˥}}}}} \hypertarget{s£j7\string_T1}{}
\markboth{\textcolor{darkblue}{\textbf{\ipa{ɕjɤ˥}}}}{}
\textcolor{teal}{\mytextsc{verbe}} \hspace{4pt} Ton~: H.

Inventer, trouver.
\textcolor{brown}{\zh{发明、想出、找到(办法)。}}


\begin{exe}
\sn \textcolor{darkblue}{\textbf{\ipa{le˧-ɕjɤ˥}}}
\trans \mytextsc{accomp}
\trans \textit{\textcolor{brown}{\zh{想出了}}}
\end{exe}
\begin{exe}
\sn \textcolor{darkblue}{\textbf{\ipa{ʈʂʰɯ˧ | pæ˧˥hwɤ˧ | ɕjɤ˧ ɣɯ˧!}}}
\trans Il/elle a une solution à tout/ sait trouver une solution en toutes circonstances!
\trans \textit{\textcolor{brown}{\zh{他很会想办法的!}}}
\end{exe}


\paragraph{\hspace{-0.5cm} {\Large \textcolor{darkblue}{\textbf{\ipa{ɕjɤ˧-bv̩˧nv̩˧}}}}} \hypertarget{s£j7\string_M-bv\string_=\string_Mnv\string_=\string_M1}{}
\markboth{\textcolor{darkblue}{\textbf{\ipa{ɕjɤ˧-bv̩˧nv̩˧}}}}{}
\textcolor{teal}{\mytextsc{adjectif}} \hspace{4pt} Ton~: M.
\textit{De:} \textbf{bv̩˧nv̩˧} 
Bonne (odeur).
\textcolor{brown}{\zh{香(气味)。}}


\begin{exe}
\sn \textcolor{darkblue}{\textbf{\ipa{ʈʂʰɯ˧ ɕjɤ˧-bv̩˧nv̩˧ ɲi˩.}}}
\trans ça sent bon!
\trans \textit{\textcolor{brown}{\zh{这很香(气味香)。}}}
\end{exe}


\paragraph{\hspace{-0.5cm} {\Large \textcolor{darkblue}{\textbf{\ipa{ɕjɤ˩\textasciitilde{}ɕjɤ˩}}}}} \hypertarget{s£j7\string_B~s£j7\string_B1}{}
\markboth{\textcolor{darkblue}{\textbf{\ipa{ɕjɤ˩\textasciitilde{}ɕjɤ˩}}}}{}
\textcolor{teal}{\mytextsc{verbe}} \hspace{4pt} Ton~: L.

Maltraiter.
\textcolor{brown}{\zh{欺负。}}


\begin{exe}
\sn \textcolor{darkblue}{\textbf{\ipa{hĩ˧ ɕjɤ˥\textasciitilde{}ɕjɤ˩}}}
\trans maltraiter quelqu'un
\trans \textit{\textcolor{brown}{\zh{欺负人}}}
\end{exe}
\begin{exe}
\sn \textcolor{darkblue}{\textbf{\ipa{no˧ | njɤ˩ ɕjɤ˩\textasciitilde{}ɕjɤ˩-mv̩˩-zo˩˥! / no˧ | njɤ˩ ɕjɤ˩\textasciitilde{}ɕjɤ˩˥!}}}
\trans Vous me maltraitez!
\trans \textit{\textcolor{brown}{\zh{你对我不好!你欺负我!}}}
\end{exe}
\begin{exe}
\sn \textcolor{darkblue}{\textbf{\ipa{no˧ | njɤ˩ ɕjɤ˩\textasciitilde{}ɕjɤ˩-ze˥!}}}
\trans Vous m'avez maltraité!
\trans \textit{\textcolor{brown}{\zh{你欺负了我!}}}
\end{exe}


\paragraph{\hspace{-0.5cm} {\Large \textcolor{darkblue}{\textbf{\ipa{ɕjɤ˩jo˩}}}}} \hypertarget{s£j7\string_Bjo\string_B1}{}
\markboth{\textcolor{darkblue}{\textbf{\ipa{ɕjɤ˩jo˩}}}}{}
\textcolor{teal}{\mytextsc{nom}} \hspace{4pt} Ton~: L.

\textit{Fritillaria cirrhosa}.
\textcolor{brown}{\zh{贝母。}}


 \mytextsc{clf}~: \textcolor{darkblue}{\textbf{\ipa{ɭɯ˧}}} 

\paragraph{\hspace{-0.5cm} {\Large \textcolor{darkblue}{\textbf{\ipa{ɕjɤ˩tʰv̩˧˥}}}}} \hypertarget{s£j7\string_Bt\string_hv\string_=\string_M\string_T1}{}
\markboth{\textcolor{darkblue}{\textbf{\ipa{ɕjɤ˩tʰv̩˧˥}}}}{}
\textcolor{teal}{\mytextsc{verbe}} \hspace{4pt} Ton~: LM+MH\#.

Insulter, maudire, se moquer; réprimander, gronder.
\textcolor{brown}{\zh{骂,批评。}}


\begin{exe}
\sn \textcolor{darkblue}{\textbf{\ipa{hĩ˧ ɕjɤ˥tʰv̩˩}}}
\trans insulter quelqu'un/ critiquer quelqu'un
\trans \textit{\textcolor{brown}{\zh{骂人、批评人}}}
\end{exe}


\paragraph{\hspace{-0.5cm} {\Large \textcolor{darkblue}{\textbf{\ipa{ɕjɤ˧˥}}}}} \hypertarget{s£j7\string_M\string_T1}{}
\markboth{\textcolor{darkblue}{\textbf{\ipa{ɕjɤ˧˥}}}}{}
\textcolor{teal}{\mytextsc{verbe}} \hspace{4pt} Ton~: MH.

Essayer, goûter, expérimenter.
\textcolor{brown}{\zh{尝试、体会、经过。}}


\begin{exe}
\sn \textcolor{darkblue}{\textbf{\ipa{le˧-ɕjɤ˧-ze˥}}}
\trans \mytextsc{accomp} \string_ \mytextsc{pfv}
\trans \textit{\textcolor{brown}{\zh{试了}}}
\end{exe}
\begin{exe}
\sn \textcolor{darkblue}{\textbf{\ipa{tso˧\textasciitilde{}tso˧ ɕjɤ˩}}}
\trans goûter quelque chose
\trans \textit{\textcolor{brown}{\zh{尝一个东西}}}
\end{exe}
\begin{exe}
\sn \textcolor{darkblue}{\textbf{\ipa{no˧ ɖɯ˧-kʰwɤ˥ ɕjɤ˩!}}}
\trans goûte un peu! goûte un morceau!
\trans \textit{\textcolor{brown}{\zh{你尝一口吧!}}}
\end{exe}
\begin{exe}
\sn \textcolor{darkblue}{\textbf{\ipa{ɖɯ˧-ɕjɤ˧-ɻ̍˥!}}}
\trans Goûte voir! / Essaie voir!
\trans \textit{\textcolor{brown}{\zh{尝一尝吧! / 试一试吧!}}}
\end{exe}


\paragraph{\hspace{-0.5cm} {\Large \textcolor{darkblue}{\textbf{\ipa{ɕjo˩li\#˥}}}}} \hypertarget{s£jo\string_Bli\#\string_T1}{}
\markboth{\textcolor{darkblue}{\textbf{\ipa{ɕjo˩li\#˥}}}}{}
\textcolor{teal}{\mytextsc{nom}} \hspace{4pt} Ton~: LM+\#H.

Flûte (type ``flûte traversière" et non ``flûte à bec").
\textcolor{brown}{\zh{笛子。}}


 \mytextsc{clf}~: \textcolor{darkblue}{\textbf{\ipa{ɭɯ˧}}} 

\paragraph{\hspace{-0.5cm} {\Large \textcolor{darkblue}{\textbf{\ipa{ɕɯ˩\textsubscript{a}}}}}} \hypertarget{s£M\string_Ba1}{}
\markboth{\textcolor{darkblue}{\textbf{\ipa{ɕɯ˩\textsubscript{a}}}}}{}
\textcolor{teal}{\mytextsc{verbe}} \hspace{4pt} Ton~: L\textsubscript{a}.

Élever (terme plus relevé que \textcolor{brown}{\zh{ʐɤ˧}}).
\textcolor{brown}{\zh{养。}}


\begin{exe}
\sn \textcolor{darkblue}{\textbf{\ipa{ɕɯ˩zo\#˥}}}
\trans enfant adopté
\trans \textit{\textcolor{brown}{\zh{养儿}}}
\end{exe}
\begin{exe}
\sn \textcolor{darkblue}{\textbf{\ipa{ho˧zo˧-ɕɯ˧zo˥, | æ̃˩ mɤ˧-tsɤ˧! | hĩ˧-zo˧mv˥, | ʐɤ˧ tʰɑ˧-mɤ˧-ʝi˧!}}}
\trans Un bébé faisan qu'on élève chez soi ne devient pas un poulet (n'est pas domestiqué pour autant)! Il ne faut pas élever les enfants d'autrui! (Proverbe qui ne s'applique pas à l'adoption d'enfants qui ont perdu leurs attaches à leur famille biologique, mais à l'adoption d'enfants qui restent en contact avec leurs proches: quelque soin que l'on consacre à leur éducation, ils restent plus attachés à leur famille d'origine.)
\trans \textit{\textcolor{brown}{\zh{养的小雉,不会变成鸡!人家的孩子,不要养!(指的不是领养孤儿,而是养别人的孩子:无论多么关心孩子,他还是会更爱自己原来的家人。)}}}
\end{exe}


\newpage
\section*{\centering- \textcolor{darkblue}{\textbf{\ipa{d}}} -}
\paragraph{\hspace{-0.5cm} {\Large \textcolor{darkblue}{\textbf{\ipa{dɑ˧ʝi˩}}}}} \hypertarget{dA\string_Mj££i\string_B1}{}
\markboth{\textcolor{darkblue}{\textbf{\ipa{dɑ˧ʝi˩}}}}{}
\textcolor{teal}{\mytextsc{nom}} \hspace{4pt} Ton~: L\#.

Mule.
\textcolor{brown}{\zh{骡子。}}


\begin{exe}
\sn \textcolor{darkblue}{\textbf{\ipa{dɑ˧ʝi˩-dʑo˩, | ɖɯ˩mi˧ dʑo˧-kv̩˥-mæ˩! | ɖɯ˩zo˧ dʑo˧-kv̩˥-mæ˩!}}}
\trans Les mules, ça se répartit en mules mâles et mules femelles! / Il existe une distinction de sexe parmi les mules! (Explication fournie à un visiteur citadin peu au fait de l'élevage des animaux.)
\trans \textit{\textcolor{brown}{\zh{骡子呢,有母骡子!(也)有公骡子! / 骡子,分母的和公的!(这个说明是给一个不懂畜牧业的城里人听)}}}
\end{exe}
 \mytextsc{clf}~: \textcolor{darkblue}{\textbf{\ipa{v̩˧}}} 

\paragraph{\hspace{-0.5cm} {\Large \textcolor{darkblue}{\textbf{\ipa{dɑ˧pɤ˧}}}}} \hypertarget{dA\string_Mp7\string_M1}{}
\markboth{\textcolor{darkblue}{\textbf{\ipa{dɑ˧pɤ˧}}}}{}
\textcolor{teal}{\mytextsc{nom}} \hspace{4pt} Ton~: M.

Prêtre de la religion locale.
\textcolor{brown}{\zh{宗教礼师。音译:达巴。}}


\begin{exe}
\sn \textcolor{darkblue}{\textbf{\ipa{dɑ˧pɤ˧ ʝi˧-hĩ˧ hĩ˧}}}
\trans prêtre, personne qui joue le rôle de prêtre/qui est prêtre
\trans \textit{\textcolor{brown}{\zh{当达巴的人}}}
\end{exe}
 \mytextsc{clf}~: \textcolor{darkblue}{\textbf{\ipa{v̩˧}}} 

\paragraph{\hspace{-0.5cm} {\Large \textcolor{darkblue}{\textbf{\ipa{dɑ˧pv̩\#˥}}}}} \hypertarget{dA\string_Mpv\string_=\#\string_T1}{}
\markboth{\textcolor{darkblue}{\textbf{\ipa{dɑ˧pv̩\#˥}}}}{}
\textcolor{teal}{\mytextsc{nom}} \hspace{4pt} Ton~: \#H.

Maître de maison, hôte (personne qui accueille).
\textcolor{brown}{\zh{主人。}}


\begin{exe}
\sn \textcolor{darkblue}{\textbf{\ipa{ʑi˧dv̩˧ dɑ˧pv̩˧}}}
\trans l'hôte de la maison
\trans \textit{\textcolor{brown}{\zh{家的主人}}}
\end{exe}
\begin{exe}
\sn \textcolor{darkblue}{\textbf{\ipa{ʑi˧dv̩˧-ʝi˧-hĩ˧ dɑ˧pv̩˧}}}
\trans idem
\trans \textit{\textcolor{brown}{\zh{同上}}}
\end{exe}
 \mytextsc{clf}~: \textcolor{darkblue}{\textbf{\ipa{v̩˧}}} 

\paragraph{\hspace{-0.5cm} {\Large \textcolor{darkblue}{\textbf{\ipa{dɑ˧pʰo˥}}}}} \hypertarget{dA\string_Mp\string_ho\string_T1}{}
\markboth{\textcolor{darkblue}{\textbf{\ipa{dɑ˧pʰo˥}}}}{}
\textcolor{teal}{\mytextsc{nom}} \hspace{4pt} Ton~: LH.

Dapo (nom de village).
\textcolor{brown}{\zh{达坡(永宁的一个村落)。}}


\begin{exe}
\sn \textcolor{darkblue}{\textbf{\ipa{ɖæ˩ʂɯ\#˥, | ʈʂo˧ʂɯ\#˥, | bɤ˩tɕʰɯ˩˥, | dɑ˧pʰo˥, | bɤ˧dzi˩, | dze˧bo˧}}}
\trans les six villages de la plaine de Yongning, dans l'ordre, qui prend comme point d'origine le village le plus proche du Lac
\trans \textit{\textcolor{brown}{\zh{永宁坝的六个村落,按传统排序:从距离泸沽湖最近的村落说起。}}}
\end{exe}


\paragraph{\hspace{-0.5cm} {\Large \textcolor{darkblue}{\textbf{\ipa{dɑ˧ʁwɤ\#˥}}}}} \hypertarget{dA\string_MRw7\#\string_T1}{}
\markboth{\textcolor{darkblue}{\textbf{\ipa{dɑ˧ʁwɤ\#˥}}}}{}
\textcolor{teal}{\mytextsc{nom}} \hspace{4pt} Ton~: \#H.

Un village en aval de Qiansuo; la langue parlée là-bas serait relativement proche de celle de la plaine de Yongning.
\textcolor{brown}{\zh{一个村落,在前所的下游。据说那边的方言跟丽江坝比较接近。}}




\paragraph{\hspace{-0.5cm} {\Large \textcolor{darkblue}{\textbf{\ipa{dɑ˩}}}}} \hypertarget{dA\string_B1}{}
\markboth{\textcolor{darkblue}{\textbf{\ipa{dɑ˩}}}}{}
\textcolor{teal}{\mytextsc{adjectif}} \hspace{4pt} Ton~: L.
\textit{Archaïque} [\textcolor{brown}{\zh{古语}}] 
Heureux.
\textcolor{brown}{\zh{幸福、平安、安好。}}


\begin{exe}
\sn \textcolor{darkblue}{\textbf{\ipa{mɤ˧-dɑ˩-qʰwɤ˩}}}
\trans chanson mélancolique, récit des malheurs
\trans \textit{\textcolor{brown}{\zh{悲情歌,讲述自己日子痛苦}}}
\end{exe}
\begin{exe}
\sn \textcolor{darkblue}{\textbf{\ipa{mɤ˧-dɑ˩!}}}
\trans Formule employée en début de chanson mélancolique, et parfois au début d'un conte. (La même formule est en usage dans la langue lazé.)
\trans \textit{\textcolor{brown}{\zh{悲情歌的开头词}}}
\end{exe}
\begin{exe}
\sn \textcolor{darkblue}{\textbf{\ipa{mɤ˧-dɑ˩-mi˩}}}
\trans Comme ci-dessus: même sens que la forme sans suffixe \textcolor{brown}{\zh{/-mi/}}.
\trans \textit{\textcolor{brown}{\zh{同上}}}
\end{exe}
\begin{exe}
\sn \textcolor{darkblue}{\textbf{\ipa{ɖwæ˧˥ | hɤ˩-dɑ˥! | ɖwæ˧˥ | hɤ˩˥!}}}
\trans Bravo, bravo! (Contexte: compliment saluant l'exploit d'une petite fille parvenue à grimper sur un meuble.)
\trans \textit{\textcolor{brown}{\zh{很了不起啊!(情景:表扬一个小孩子成功地爬上了一个家具)}}}
\end{exe}
\begin{exe}
\sn \textcolor{darkblue}{\textbf{\ipa{ɖʐɯ˩dɑ˥-kʰɤ˩dɑ˩-ɻ̍˩}}}
\trans Tout va bien, tout est pour le mieux. (S'emploie par exemple pour décrire une période sans disette, ni tremblement de terre, ni épidémie, ni guerre ou autre catastrophe.)
\trans \textit{\textcolor{brown}{\zh{一切都安好。(如:来指一段时间没有饥荒、地震、流行病、战争等灾难)}}}
\end{exe}


\paragraph{\hspace{-0.5cm} {\Large \textcolor{darkblue}{\textbf{\ipa{dɑ˩\textsubscript{b}}}}}} \hypertarget{dA\string_Bb1}{}
\markboth{\textcolor{darkblue}{\textbf{\ipa{dɑ˩\textsubscript{b}}}}}{}
\textcolor{teal}{\mytextsc{verbe}} \hspace{4pt} Ton~: L\textsubscript{b}.

Tisser.
\textcolor{brown}{\zh{织。}}


\begin{exe}
\sn \textcolor{darkblue}{\textbf{\ipa{ɣɯ˧ dɑ˩}}}
\trans tisser du tissu
\trans \textit{\textcolor{brown}{\zh{织布}}}
\end{exe}
\begin{exe}
\sn \textcolor{darkblue}{\textbf{\ipa{ɣɯ˧ | le˧-dɑ˩}}}
\trans tisser du tissu
\trans \textit{\textcolor{brown}{\zh{织布}}}
\end{exe}
\begin{exe}
\sn \textcolor{darkblue}{\textbf{\ipa{tso˧\textasciitilde{}tso˧ dɑ˩}}}
\trans tisser des choses
\trans \textit{\textcolor{brown}{\zh{织东西}}}
\end{exe}
\begin{exe}
\sn \textcolor{darkblue}{\textbf{\ipa{ɖɯ˧-dɑ˧\textasciitilde{}dɑ˩-ɻ̍˩}}}
\trans \mytextsc{délimitatif} \string_ \mytextsc{red} \mytextsc{inchoatif}
\trans \textit{\textcolor{brown}{\zh{织一下}}}
\end{exe}


\paragraph{\hspace{-0.5cm} {\Large \textcolor{darkblue}{\textbf{\ipa{dɑ˩kʰɤ˩}}}}} \hypertarget{dA\string_Bk\string_h7\string_B1}{}
\markboth{\textcolor{darkblue}{\textbf{\ipa{dɑ˩kʰɤ˩}}}}{}
\textcolor{teal}{\mytextsc{nom}} \hspace{4pt} Ton~: L.

Tambour.
\textcolor{brown}{\zh{鼓。}}


\begin{exe}
\sn \textcolor{darkblue}{\textbf{\ipa{dɑ˩kʰɤ˩ lɑ˥(-ze˩)}}}
\trans jouer du tambour
\trans \textit{\textcolor{brown}{\zh{打鼓}}}
\end{exe}
 \mytextsc{clf}~: \textcolor{darkblue}{\textbf{\ipa{ɭɯ˧}}} 

\paragraph{\hspace{-0.5cm} {\Large \textcolor{darkblue}{\textbf{\ipa{dɑ˩to\#˥}}}}} \hypertarget{dA\string_Bto\#\string_T1}{}
\markboth{\textcolor{darkblue}{\textbf{\ipa{dɑ˩to\#˥}}}}{}
\textcolor{teal}{\mytextsc{adverbe}} \hspace{4pt} Ton~: LM+\#H.

Poliment.
\textcolor{brown}{\zh{客气地。}}


\begin{exe}
\sn \textcolor{darkblue}{\textbf{\ipa{dɑ˩to˧ ʐwɤ˧˥}}}
\trans faire des politesses
\trans \textit{\textcolor{brown}{\zh{说客气话}}}
\end{exe}
\begin{exe}
\sn \textcolor{darkblue}{\textbf{\ipa{dɑ˩to˧ ʐwɤ˧-hĩ˥-lɑ˩ ɲi˩!}}}
\trans C'est juste pour être poli! / C'est juste une façon de dire! (Commentaire de quelqu'un au sujet d'une invitation lancée par un voisin, qui est une simple politesse et pas une vraie invitation; il convient de la décliner.)
\trans \textit{\textcolor{brown}{\zh{这只是客气话而已!}}}
\end{exe}


\paragraph{\hspace{-0.5cm} {\Large \textcolor{darkblue}{\textbf{\ipa{dɑ˩to˩}}}}} \hypertarget{dA\string_Bto\string_B1}{}
\markboth{\textcolor{darkblue}{\textbf{\ipa{dɑ˩to˩}}}}{}
\textcolor{teal}{\mytextsc{adverbe}} \hspace{4pt} Ton~: L.

Au fond, en réalité, en définitive.
\textcolor{brown}{\zh{说到底,根本上,归根结底。}}




\paragraph{\hspace{-0.5cm} {\Large \textcolor{darkblue}{\textbf{\ipa{dɑ˧˥}}}}} \hypertarget{dA\string_M\string_T1}{}
\markboth{\textcolor{darkblue}{\textbf{\ipa{dɑ˧˥}}}}{}
\textcolor{teal}{\mytextsc{nom}} \hspace{4pt} Ton~: MH.

Infortune, malheur.
Dialecte chinois local~: \zh{苦。}


\begin{exe}
\sn \textcolor{darkblue}{\textbf{\ipa{dɑ˧-ʐwɤ˧˥}}}
\trans se plaindre de son infortune, gémir sur son sort
\trans \textit{\textcolor{brown}{\zh{诉苦}}}
\end{exe}
\begin{exe}
\sn \textcolor{darkblue}{\textbf{\ipa{ɻ̃˧-ʐwɤ˧ | dɑ˧-ʐwɤ˧-ɻ̍˥}}}
\trans raconter ses malheurs; se plaindre
\trans \textit{\textcolor{brown}{\zh{讲自己的不幸}}}
\end{exe}
\begin{exe}
\sn \textcolor{darkblue}{\textbf{\ipa{ʈʂʰɯ˧ | mɑ˧dɑ˩-qʰwɤ˩, | ɻ̃˧-ʐwɤ˧ | dɑ˧-ʐwɤ˧-ɻ̍˥!}}}
\trans Il est malheureux; il passe son temps à se plaindre!
\trans \textit{\textcolor{brown}{\zh{他不幸福,他一直在诉苦!}}}
\end{exe}


\paragraph{\hspace{-0.5cm} {\Large \textcolor{darkblue}{\textbf{\ipa{dɑ˧˥}}} \textsubscript{1}}} \hypertarget{dA\string_M\string_T1}{}
\markboth{\textcolor{darkblue}{\textbf{\ipa{dɑ˧˥}}} \textsubscript{1}}{}
\textcolor{teal}{\mytextsc{verbe}} \hspace{4pt} Ton~: MH.

Construire (une maison…).
\textcolor{brown}{\zh{建(房子)。}}


\begin{exe}
\sn \textcolor{darkblue}{\textbf{\ipa{ʑi˧mi˧ dɑ˧˥}}}
\trans construire une maison
\trans \textit{\textcolor{brown}{\zh{建房}}}
\end{exe}


\paragraph{\hspace{-0.5cm} {\Large \textcolor{darkblue}{\textbf{\ipa{dɑ˧˥}}} \textsubscript{2}}} \hypertarget{dA\string_M\string_T2}{}
\markboth{\textcolor{darkblue}{\textbf{\ipa{dɑ˧˥}}} \textsubscript{2}}{}
\textcolor{teal}{\mytextsc{verbe}} \hspace{4pt} Ton~: MH.

Couper un arbre, abattre un arbre; ouvrir une brèche (dans une digue).
\textcolor{brown}{\zh{砍(树),割(肉)。}}


\begin{exe}
\sn \textcolor{darkblue}{\textbf{\ipa{le˧-dɑ˧-ze˥}}}
\trans \mytextsc{accomp} \string_ \mytextsc{pfv}
\trans \textit{\textcolor{brown}{\zh{砍了(树),割了(肉)}}}
\end{exe}
\begin{exe}
\sn \textcolor{darkblue}{\textbf{\ipa{ɖɯ˧-dɑ˧ tʰi˥-dɑ˩}}}
\trans donner un coup
\trans \textit{\textcolor{brown}{\zh{砍一下}}}
\end{exe}
\begin{exe}
\sn \textcolor{darkblue}{\textbf{\ipa{dɑ˩\textasciitilde{}dɑ˧˥}}}
\trans \mytextsc{red}
\trans \textit{\textcolor{brown}{\zh{\mytextsc{重叠}}}}
\end{exe}
\begin{exe}
\sn \textcolor{darkblue}{\textbf{\ipa{le˧-dɑ˩\textasciitilde{}dɑ˩(-ze˩)}}}
\trans (j'ai) découpé (ex.: le poulet) en morceaux
\trans \textit{\textcolor{brown}{\zh{(我把一只鸡)割成块了}}}
\end{exe}
\begin{exe}
\sn \textcolor{darkblue}{\textbf{\ipa{ʂe˧ dɑ˥\textasciitilde{}dɑ˩}}}
\trans hacher de la viande, couper de la viande en morceaux
\trans \textit{\textcolor{brown}{\zh{把肉剁碎}}}
\end{exe}


\paragraph{\hspace{-0.5cm} {\Large \textcolor{darkblue}{\textbf{\ipa{dɑ˧˥\textsubscript{b}}}}}} \hypertarget{dA\string_M\string_Tb1}{}
\markboth{\textcolor{darkblue}{\textbf{\ipa{dɑ˧˥\textsubscript{b}}}}}{}
\textcolor{teal}{\mytextsc{classificateur}} \hspace{4pt} Ton~: MH\textsubscript{b}.

Auto-classificateur des coups.
\textcolor{brown}{\zh{量词:下(打一下)。}}


\begin{exe}
\sn \textcolor{darkblue}{\textbf{\ipa{ɖɯ˧-dɑ˧˥}}}
\trans un coup
\trans \textit{\textcolor{brown}{\zh{当头一棒}}}
\end{exe}
\begin{exe}
\sn \textcolor{darkblue}{\textbf{\ipa{ɖɯ˧-dɑ˧ tʰi˥-dɑ˩}}}
\trans donner un coup
\trans \textit{\textcolor{brown}{\zh{打一下}}}
\end{exe}


\paragraph{\hspace{-0.5cm} {\Large \textcolor{darkblue}{\textbf{\ipa{dɤ˧-qo˧}}}}} \hypertarget{d7\string_M-qo\string_M1}{}
\markboth{\textcolor{darkblue}{\textbf{\ipa{dɤ˧-qo˧}}}}{}
\textcolor{teal}{\mytextsc{adverbe}} \hspace{4pt} Ton~: M.

Par là-bas tout au loin, tout au loin là-bas.
\textcolor{brown}{\zh{那里(远指)。}}




\paragraph{\hspace{-0.5cm} {\Large \textcolor{darkblue}{\textbf{\ipa{dɤ˧-tʰv̩˧-gi\#˥}}}}} \hypertarget{d7\string_M-t\string_hv\string_=\string_M-gi\#\string_T1}{}
\markboth{\textcolor{darkblue}{\textbf{\ipa{dɤ˧-tʰv̩˧-gi\#˥}}}}{}
\textcolor{teal}{\mytextsc{adverbe}} \hspace{4pt} Ton~: \#H.

Au loin, de ce côté-là.
\textcolor{brown}{\zh{那边(远指)。}}




\paragraph{\hspace{-0.5cm} {\Large \textcolor{darkblue}{\textbf{\ipa{dɤ˧-tʰv̩˧qo˧}}}}} \hypertarget{d7\string_M-t\string_hv\string_=\string_Mqo\string_M1}{}
\markboth{\textcolor{darkblue}{\textbf{\ipa{dɤ˧-tʰv̩˧qo˧}}}}{}
\textcolor{teal}{\mytextsc{adverbe}} \hspace{4pt} Ton~: M.

Par là-bas tout au loin.
\textcolor{brown}{\zh{那边(远指)。}}


\textit{Voir~:} \hyperlink{d7\string_M-t`s`\string_hM\string_Mqo\string_M1}{\textcolor{darkblue}{\textbf{\ipa{dɤ˧-ʈʂʰɯ˧qo˧}}}} 

\paragraph{\hspace{-0.5cm} {\Large \textcolor{darkblue}{\textbf{\ipa{dɤ˧-ʈʂʰɯ˧qo˧}}}}} \hypertarget{d7\string_M-t`s`\string_hM\string_Mqo\string_M1}{}
\markboth{\textcolor{darkblue}{\textbf{\ipa{dɤ˧-ʈʂʰɯ˧qo˧}}}}{}
\textcolor{teal}{\mytextsc{adverbe}} \hspace{4pt} Ton~: M.

Par là-bas tout au loin.
\textcolor{brown}{\zh{那边(远指)。}}


\textit{Voir~:} \hyperlink{d7\string_M-t\string_hv\string_=\string_Mqo\string_M1}{\textcolor{darkblue}{\textbf{\ipa{dɤ˧-tʰv̩˧qo˧}}}} 

\paragraph{\hspace{-0.5cm} {\Large \textcolor{darkblue}{\textbf{\ipa{di˧mi˧}}}}} \hypertarget{di\string_Mmi\string_M1}{}
\markboth{\textcolor{darkblue}{\textbf{\ipa{di˧mi˧}}}}{}
\textcolor{teal}{\mytextsc{nom}} \hspace{4pt} Ton~: M.

Grande plaine.
\textcolor{brown}{\zh{平坝。}}


\begin{exe}
\sn \textcolor{darkblue}{\textbf{\ipa{ɬi˧di˩-di˩mi˩}}}
\trans la plaine de Yongning
\trans \textit{\textcolor{brown}{\zh{永宁坝}}}
\end{exe}
 \mytextsc{clf}~: \textcolor{darkblue}{\textbf{\ipa{di˩}}} 

\paragraph{\hspace{-0.5cm} {\Large \textcolor{darkblue}{\textbf{\ipa{di˧qo˧}}}}} \hypertarget{di\string_Mqo\string_M1}{}
\markboth{\textcolor{darkblue}{\textbf{\ipa{di˧qo˧}}}}{}
\textcolor{teal}{\mytextsc{nom}} \hspace{4pt} Ton~: M.

Plaine.
\textcolor{brown}{\zh{平坝。}}


 \mytextsc{clf}~: \textcolor{darkblue}{\textbf{\ipa{di˩}}} 

\paragraph{\hspace{-0.5cm} {\Large \textcolor{darkblue}{\textbf{\ipa{di˧ɻæ˧}}}}} \hypertarget{di\string_Mr£`\{\string_M1}{}
\markboth{\textcolor{darkblue}{\textbf{\ipa{di˧ɻæ˧}}}}{}
\textcolor{teal}{\mytextsc{nom}} \hspace{4pt} Ton~: M.

Plaine.
\textcolor{brown}{\zh{平坝。}}


 \mytextsc{clf}~: \textcolor{darkblue}{\textbf{\ipa{di˩}}} 

\paragraph{\hspace{-0.5cm} {\Large \textcolor{darkblue}{\textbf{\ipa{‑di˩}}}}} \hypertarget{‑di\string_B1}{}
\markboth{\textcolor{darkblue}{\textbf{\ipa{‑di˩}}}}{}
\textcolor{teal}{\mytextsc{suffixe}} \hspace{4pt} Ton~: L.

Nominalisateur; locatif; purposif.
\textcolor{brown}{\zh{\mytextsc{名物化}/\mytextsc{处所格}/\mytextsc{目的格。}}}


\begin{exe}
\sn \textcolor{darkblue}{\textbf{\ipa{tso˧\textasciitilde{}tso˧-tɕɯ˧-di˧˥}}}
\trans (meuble/objet) sur lequel on pose des choses
\trans \textit{\textcolor{brown}{\zh{可以摆东西的(家具)}}}
\end{exe}


\paragraph{\hspace{-0.5cm} {\Large \textcolor{darkblue}{\textbf{\ipa{di˩\textsubscript{c}}}}}} \hypertarget{di\string_Bc1}{}
\markboth{\textcolor{darkblue}{\textbf{\ipa{di˩\textsubscript{c}}}}}{}
\textcolor{teal}{\mytextsc{classificateur}} \hspace{4pt} Ton~: L\textsubscript{c}.

Classificateur pour les plaines, les étendues de terre, les lieux.
\textcolor{brown}{\zh{量词:坝子、地方(一个)。}}


\begin{exe}
\sn \textcolor{darkblue}{\textbf{\ipa{ɖɯ˧-v̩˧ | ɖɯ˧-di˩ hɯ˩}}}
\trans se séparer, partir chacun de son côté (par exemple: des frères se séparent et chacun va son chemin)
\trans \textit{\textcolor{brown}{\zh{分开,每个人去不同的地方}}}
\end{exe}


\paragraph{\hspace{-0.5cm} {\Large \textcolor{darkblue}{\textbf{\ipa{di˩\textsubscript{a}}}}}} \hypertarget{di\string_Ba1}{}
\markboth{\textcolor{darkblue}{\textbf{\ipa{di˩\textsubscript{a}}}}}{}
\textcolor{teal}{\mytextsc{verbe}} \hspace{4pt} Ton~: L\textsubscript{a}.

Existentiel; posséder. Possession non amovible aussi bien que transitoire: avoir une maison, aussi bien que: avoir une tache de graisse sur la joue; avoir une différence de longueur (deux objets ont une différence de longueur).
\textcolor{brown}{\zh{存在动词:有,拥有。例如:有家,有污垢在衣服上,有长短区别(两个物品有长短区别)。}}


\begin{exe}
\sn \textcolor{darkblue}{\textbf{\ipa{ʈʰɯ˧ | ɑ˩ʁo˧ mɤ˧-di˩-hĩ˩.}}}
\trans Elle/il n'a pas de maison, elle/il est sans domicile
\trans \textit{\textcolor{brown}{\zh{他没有家。}}}
\end{exe}
\begin{exe}
\sn \textcolor{darkblue}{\textbf{\ipa{mɤ˧ tʰi˧-di˩}}}
\trans il y a de la graisse (ex.: autour de la bouche de quelqu'un qui vient de croquer de la viande à belles dents)
\trans \textit{\textcolor{brown}{\zh{有油(例如:一个人的嘴巴周围油乎乎,有油)}}}
\end{exe}
\begin{exe}
\sn \textcolor{darkblue}{\textbf{\ipa{ɖɯ˧-kʰwɤ˧ di˥}}}
\trans il y a quelque chose
\trans \textit{\textcolor{brown}{\zh{有一块东西}}}
\end{exe}


\paragraph{\hspace{-0.5cm} {\Large \textcolor{darkblue}{\textbf{\ipa{di˩-gɤ˩lɑ˥}}}}} \hypertarget{di\string_B-g7\string_BlA\string_T1}{}
\markboth{\textcolor{darkblue}{\textbf{\ipa{di˩-gɤ˩lɑ˥}}}}{}
\textcolor{teal}{\mytextsc{nom}} \hspace{4pt} Ton~: L+H\#.

Esprit de la terre, Bodhisattva terrestre.
\textcolor{brown}{\zh{地菩萨。}}


 \mytextsc{clf}~: \textcolor{darkblue}{\textbf{\ipa{v̩˧}}} 

\paragraph{\hspace{-0.5cm} {\Large \textcolor{darkblue}{\textbf{\ipa{di˩li˩}}}}} \hypertarget{di\string_Bli\string_B1}{}
\markboth{\textcolor{darkblue}{\textbf{\ipa{di˩li˩}}}}{}
\textcolor{teal}{\mytextsc{nom}} \hspace{4pt} Ton~: L.

Ottélie: une plante verte comestible: \textit{Ottelia Acuminata, Boottia acuminata, Ottelia yunnanensis}.
\textcolor{brown}{\zh{海菜花。}}Dialecte chinois local~: \zh{龙爪菜。}


\begin{exe}
\sn \textcolor{darkblue}{\textbf{\ipa{di˩li˩-ʁo˩bv̩˥ (ton: L+H\#)}}}
\trans pousses d'ottélie
\trans \textit{\textcolor{brown}{\zh{海菜花的萌芽}}}
\end{exe}
\begin{exe}
\sn \textcolor{darkblue}{\textbf{\ipa{di˩li˩-ʁo˩bv̩˥ hwæ˩}}}
\trans acheter des pousses d'ottélie (verbe au ton M)
\trans \textit{\textcolor{brown}{\zh{买海菜花萌芽}}}
\end{exe}
\begin{exe}
\sn \textcolor{darkblue}{\textbf{\ipa{di˩li˩-ʁo˩bv̩˥ tɕʰi˩}}}
\trans vendre des pousses d'ottélie
\trans \textit{\textcolor{brown}{\zh{卖海菜花萌芽}}}
\end{exe}
\begin{exe}
\sn \textcolor{darkblue}{\textbf{\ipa{di˩li˩-ʁo˩bv̩˥ dzɯ˩}}}
\trans manger des pousses d'ottélie
\trans \textit{\textcolor{brown}{\zh{吃海菜花萌芽}}}
\end{exe}
\begin{exe}
\sn \textcolor{darkblue}{\textbf{\ipa{di˩li˩-ʁo˩bv̩˥ dze˩}}}
\trans couper des pousses d'ottélie
\trans \textit{\textcolor{brown}{\zh{割海菜花萌芽}}}
\end{exe}
\begin{exe}
\sn \textcolor{darkblue}{\textbf{\ipa{di˩li˩-ʁo˩bv̩˥ tɕɤ˩}}}
\trans faire bouillir des pousses d'ottélie
\trans \textit{\textcolor{brown}{\zh{煮海菜花萌芽}}}
\end{exe}
 \mytextsc{clf}~: \textcolor{darkblue}{\textbf{\ipa{qɑ˩}}} 

\paragraph{\hspace{-0.5cm} {\Large \textcolor{darkblue}{\textbf{\ipa{di˧˥}}} \textsubscript{1}}} \hypertarget{di\string_M\string_T1}{}
\markboth{\textcolor{darkblue}{\textbf{\ipa{di˧˥}}} \textsubscript{1}}{}
\textcolor{teal}{\mytextsc{verbe}} \hspace{4pt} Ton~: MH.

Poursuivre, chasser; disperser, repousser, faire déguerpir.
\textcolor{brown}{\zh{打散,驱赶,撵,赶,打猎。}}


\begin{exe}
\sn \textcolor{darkblue}{\textbf{\ipa{tɕʰɯ˩di˩˥}}}
\trans chasser le muntjac; chasser
\trans \textit{\textcolor{brown}{\zh{赶麂子,狩猎}}}
\end{exe}
\begin{exe}
\sn \textcolor{darkblue}{\textbf{\ipa{tɕʰɯ˩di˩-bi˩-ni˩gv̩˩˥}}}
\trans avoir l'habitude de chasser
\trans \textit{\textcolor{brown}{\zh{有打猎的习惯、喜欢打猎}}}
\end{exe}
\begin{exe}
\sn \textcolor{darkblue}{\textbf{\ipa{di˩\textasciitilde{}di˧˥ / di˩\textasciitilde{}di˧-ze˥}}}
\trans \mytextsc{red}: suivre à la trace, pister
\trans \textit{\textcolor{brown}{\zh{\mytextsc{重叠:跟着、追着}}}}
\end{exe}
\begin{exe}
\sn \textcolor{darkblue}{\textbf{\ipa{tʰi˧-di˩\textasciitilde{}di˩}}}
\trans \mytextsc{dur} \mytextsc{red}
\trans \textit{\textcolor{brown}{\zh{\mytextsc{dur} \mytextsc{red}}}}
\end{exe}


\paragraph{\hspace{-0.5cm} {\Large \textcolor{darkblue}{\textbf{\ipa{di˧˥}}} \textsubscript{2}}} \hypertarget{di\string_M\string_T2}{}
\markboth{\textcolor{darkblue}{\textbf{\ipa{di˧˥}}} \textsubscript{2}}{}
\textcolor{teal}{\mytextsc{verbe}} \hspace{4pt} Ton~: MH.

S'écouler, couler; avoir la courante = avoir la diarrhée.
\textcolor{brown}{\zh{拉(肚子)。}}


\begin{exe}
\sn \textcolor{darkblue}{\textbf{\ipa{bi˧mi˧ di˧˥}}}
\trans avoir la diarrhée
\trans \textit{\textcolor{brown}{\zh{拉肚子}}}
\end{exe}


\paragraph{\hspace{-0.5cm} {\Large \textcolor{darkblue}{\textbf{\ipa{di˩˥}}}}} \hypertarget{di\string_B\string_T1}{}
\markboth{\textcolor{darkblue}{\textbf{\ipa{di˩˥}}}}{}
\textcolor{teal}{\mytextsc{nom}} \hspace{4pt} Ton~: LH.

Terre (le ciel et la terre).
\textcolor{brown}{\zh{地(天地的地)。}}


\begin{exe}
\sn \textcolor{darkblue}{\textbf{\ipa{di˩ dv̩˩-ze˥}}}
\trans creuser la terre
\trans \textit{\textcolor{brown}{\zh{挖土}}}
\end{exe}
\begin{exe}
\sn \textcolor{darkblue}{\textbf{\ipa{di˩ hwæ˧-ze˩}}}
\trans a acheté de la terre
\trans \textit{\textcolor{brown}{\zh{买了土}}}
\end{exe}
\begin{exe}
\sn \textcolor{darkblue}{\textbf{\ipa{ɖɯ˧-di˩ ɖɯ˩-bæ˩!}}}
\trans C'est différent à chaque endroit/chaque endroit a ses choses propres (par exemple, en matière de langues, chaque village a sa prononciation, son dialecte)
\trans \textit{\textcolor{brown}{\zh{一个地方,一个样! = 每个地方有自己的特色(如:每个村落有自己的摩梭方言/土语)}}}
\end{exe}
 \mytextsc{clf}~: \textcolor{darkblue}{\textbf{\ipa{di˩}}} 

\paragraph{\hspace{-0.5cm} {\Large \textcolor{darkblue}{\textbf{\ipa{do˥}}} \textsubscript{1}}} \hypertarget{do\string_T1}{}
\markboth{\textcolor{darkblue}{\textbf{\ipa{do˥}}} \textsubscript{1}}{}
\textcolor{teal}{\mytextsc{verbe}} \hspace{4pt} Ton~: H.

Grimper, monter, escalader, gravir.
\textcolor{brown}{\zh{爬,上去,上山。}}


\begin{exe}
\sn \textcolor{darkblue}{\textbf{\ipa{ʈʂo˩bo˩ do˩˥}}}
\trans grimper un mur, gravir un mur
\trans \textit{\textcolor{brown}{\zh{爬墙}}}
\end{exe}
\begin{exe}
\sn \textcolor{darkblue}{\textbf{\ipa{gɤ˩-do˧}}}
\trans grimper, gravir
\trans \textit{\textcolor{brown}{\zh{爬上}}}
\end{exe}
\begin{exe}
\sn \textcolor{darkblue}{\textbf{\ipa{ʁwɤ˩ do˩˥}}}
\trans gravir une montagne, grimper la montagne, faire de la montagne
\trans \textit{\textcolor{brown}{\zh{爬山}}}
\end{exe}
\begin{exe}
\sn \textcolor{darkblue}{\textbf{\ipa{to˩ do˩˥}}}
\trans grimper la pente/ grimper une pente
\trans \textit{\textcolor{brown}{\zh{爬上一个山坡}}}
\end{exe}


\paragraph{\hspace{-0.5cm} {\Large \textcolor{darkblue}{\textbf{\ipa{do˥}}} \textsubscript{2}}} \hypertarget{do\string_T2}{}
\markboth{\textcolor{darkblue}{\textbf{\ipa{do˥}}} \textsubscript{2}}{}
\textcolor{teal}{\mytextsc{verbe}} \hspace{4pt} Ton~: H.

S'accoupler (animaux).
\textcolor{brown}{\zh{交配、交尾。}}


\begin{exe}
\sn \textcolor{darkblue}{\textbf{\ipa{bo˩ɬɑ˥ | bo˩mi˧ do˧}}}
\trans Le verrat s'accouple avec la truie.
\trans \textit{\textcolor{brown}{\zh{公猪与母猪交配。}}}
\end{exe}


\paragraph{\hspace{-0.5cm} {\Large \textcolor{darkblue}{\textbf{\ipa{do˥\textsubscript{a}}}}}} \hypertarget{do\string_Ta1}{}
\markboth{\textcolor{darkblue}{\textbf{\ipa{do˥\textsubscript{a}}}}}{}
\textcolor{teal}{\mytextsc{classificateur}} \hspace{4pt} Ton~: H\textsubscript{a}.

Classificateur des cloisons et murs.
\textcolor{brown}{\zh{\mytextsc{量词}.墙壁(一堵)。}}


\begin{exe}
\sn \textcolor{darkblue}{\textbf{\ipa{ʈʂʰɯ˧-do\#˥}}}
\trans cette cloison
\trans \textit{\textcolor{brown}{\zh{这堵(墙壁)}}}
\end{exe}


\paragraph{\hspace{-0.5cm} {\Large \textcolor{darkblue}{\textbf{\ipa{do˧}}} \textsubscript{1}}} \hypertarget{do\string_M1}{}
\markboth{\textcolor{darkblue}{\textbf{\ipa{do˧}}} \textsubscript{1}}{}
\textcolor{teal}{\mytextsc{adjectif}} \hspace{4pt} Ton~: M.

Bête, stupide.
\textcolor{brown}{\zh{笨、愚蠢。}}


\begin{exe}
\sn \textcolor{darkblue}{\textbf{\ipa{zo˩ do˩˥}}}
\trans un idiot, un fou du village; perçu comme: ``un homme qui n'a pas grandi", ``quelqu'un qui est resté enfant"
\trans \textit{\textcolor{brown}{\zh{傻瓜}}}
\end{exe}


\paragraph{\hspace{-0.5cm} {\Large \textcolor{darkblue}{\textbf{\ipa{do˧}}} \textsubscript{2}}} \hypertarget{do\string_M2}{}
\markboth{\textcolor{darkblue}{\textbf{\ipa{do˧}}} \textsubscript{2}}{}
\textcolor{teal}{\mytextsc{adjectif}} \hspace{4pt} Ton~: M.

Stérile.
\textcolor{brown}{\zh{不能生育。}}




\paragraph{\hspace{-0.5cm} {\Large \textcolor{darkblue}{\textbf{\ipa{do˧bæ˧}}}}} \hypertarget{do\string_Mb\{\string_M1}{}
\markboth{\textcolor{darkblue}{\textbf{\ipa{do˧bæ˧}}}}{}
\textcolor{teal}{\mytextsc{nom}} \hspace{4pt} Ton~: M.

Cuisse.
\textcolor{brown}{\zh{大腿。}}


\begin{exe}
\sn \textcolor{darkblue}{\textbf{\ipa{do˧bæ˧ | ɖɯ˩-hĩ˩˥}}}
\trans cuisse
\trans \textit{\textcolor{brown}{\zh{大腿}}}
\end{exe}
\begin{exe}
\sn \textcolor{darkblue}{\textbf{\ipa{do˧bæ˧ | tɕi˩-hĩ˩˥}}}
\trans mollet
\trans \textit{\textcolor{brown}{\zh{小腿}}}
\end{exe}
 \mytextsc{clf}~: \textcolor{darkblue}{\textbf{\ipa{ʈv̩˩}}} 

\paragraph{\hspace{-0.5cm} {\Large \textcolor{darkblue}{\textbf{\ipa{do˧bv̩˧}}}}} \hypertarget{do\string_Mbv\string_=\string_M1}{}
\markboth{\textcolor{darkblue}{\textbf{\ipa{do˧bv̩˧}}}}{}
\textcolor{teal}{\mytextsc{nom}} \hspace{4pt} Ton~: M.

Fesse.
\textcolor{brown}{\zh{屁股。}}


 \mytextsc{clf}~: \textcolor{darkblue}{\textbf{\ipa{ɭɯ˧}}} 

\paragraph{\hspace{-0.5cm} {\Large \textcolor{darkblue}{\textbf{\ipa{do˩}}}}} \hypertarget{do\string_B1}{}
\markboth{\textcolor{darkblue}{\textbf{\ipa{do˩}}}}{}
\textcolor{teal}{\mytextsc{adjectif}} \hspace{4pt} Ton~: L.
\textit{Archaïque} [\textcolor{brown}{\zh{古语}}] 
Immature.
\textcolor{brown}{\zh{不成熟、晚熟。}}


\begin{exe}
\sn \textcolor{darkblue}{\textbf{\ipa{ŋwɤ˩ɬi˩-mi˩˥, | ʂe˧ mɤ˧-mv̩˥, | ʂe˧ do˧˥! | tsʰe˩ŋwɤ˩ kʰv̩˥, | zo˧ mɤ˧-ti˩, | zo˧ do˧˥!}}}
\trans Au cinquième mois, une céréale qui ne mûrit pas/qui ne donne pas de grain, c'est une récolte stérile/qui reste en herbe! A quinze ans, le garçon qui n'a pas encore acquis de maturité (=qui ne fréquente pas encore les filles), c'est qu'il a un problème/c'est un attardé!
\trans \textit{\textcolor{brown}{\zh{五月份,谷物还是小草(还不出谷粒),算是晚熟!男人十五岁还不成熟(=还不见姑娘),算是晚熟!}}}
\end{exe}


\paragraph{\hspace{-0.5cm} {\Large \textcolor{darkblue}{\textbf{\ipa{do˩\textsubscript{b}}}}}} \hypertarget{do\string_Bb1}{}
\markboth{\textcolor{darkblue}{\textbf{\ipa{do˩\textsubscript{b}}}}}{}
\textcolor{teal}{\mytextsc{verbe}} \hspace{4pt} Ton~: L\textsubscript{b}.

Voir, apercevoir.
\textcolor{brown}{\zh{看见,遇见,见。}}


\begin{exe}
\sn \textcolor{darkblue}{\textbf{\ipa{ɖɯ˧-do˥\textasciitilde{}do˩-ɻ̍˩}}}
\trans \mytextsc{délimitatif} \string_ \mytextsc{red} \mytextsc{inchoatif}
\trans \textit{\textcolor{brown}{\zh{见一见}}}
\end{exe}
\begin{exe}
\sn \textcolor{darkblue}{\textbf{\ipa{ɖɯ˧-kʰwɤ˧ do˧˥}}}
\trans apercevoir un bout/un morceau
\trans \textit{\textcolor{brown}{\zh{看见一块(东西)}}}
\end{exe}
\begin{exe}
\sn \textcolor{darkblue}{\textbf{\ipa{tso˧\textasciitilde{}tso˧ do˧˥}}}
\trans apercevoir des choses/apercevoir quelque chose
\trans \textit{\textcolor{brown}{\zh{看见东西}}}
\end{exe}
\begin{exe}
\sn \textcolor{darkblue}{\textbf{\ipa{do˩-mɤ˩-ho˥}}}
\trans \string_ \mytextsc{neg} \mytextsc{désidératif}
\trans \textit{\textcolor{brown}{\zh{不许(看)见}}}
\end{exe}
\begin{exe}
\sn \textcolor{darkblue}{\textbf{\ipa{bo˩mi˧ do˩ (+ze˩)}}}
\trans ...a vu (une/la) truie
\trans \textit{\textcolor{brown}{\zh{看见了母猪}}}
\end{exe}


\paragraph{\hspace{-0.5cm} {\Large \textcolor{darkblue}{\textbf{\ipa{do˩bv̩\#˥}}}}} \hypertarget{do\string_Bbv\string_=\#\string_T1}{}
\markboth{\textcolor{darkblue}{\textbf{\ipa{do˩bv̩\#˥}}}}{}
\textcolor{teal}{\mytextsc{nom}} \hspace{4pt} Ton~: LM+\#H.

Mur de mani (le nom désigne l'ensemble du mur, pas seulement une des tablettes qui s'y trouvent). Le mur de mani est un mur de pierre sèche et de sable, comportant des tablettes de pierre sur lesquelles est gravé une inscription: le plus souvent Om Mani Padme Hum. Un mur de mani doit être contourné dans le sens des aiguilles d'une montre: le sens de rotation de l'univers, selon la doctrine bouddhiste.
\textcolor{brown}{\zh{嘛呢堆。}}


 \mytextsc{clf}~: \textcolor{darkblue}{\textbf{\ipa{ɭɯ˧}}} \textit{Voir~:} \textcolor{darkblue}{\textbf{\ipa{mɑ˩ɳɯ˧-do˥bv˩, mɑ˩ɳɯ\#˥}}} 

\paragraph{\hspace{-0.5cm} {\Large \textcolor{darkblue}{\textbf{\ipa{do˩kv̩\#˥}}}}} \hypertarget{do\string_Bkv\string_=\#\string_T1}{}
\markboth{\textcolor{darkblue}{\textbf{\ipa{do˩kv̩\#˥}}}}{}
\textcolor{teal}{\mytextsc{nom}} \hspace{4pt} Ton~: LM+\#H.

Poutrelles soutenant le plancher du premier étage.
\textcolor{brown}{\zh{小梁子,作为楼上(第二层)木地板的底。}}


 \mytextsc{clf}~: \textcolor{darkblue}{\textbf{\ipa{ɭɯ˧}}} 

\paragraph{\hspace{-0.5cm} {\Large \textcolor{darkblue}{\textbf{\ipa{dv̩˩}}} \textsubscript{1}}} \hypertarget{dv\string_=\string_B1}{}
\markboth{\textcolor{darkblue}{\textbf{\ipa{dv̩˩}}} \textsubscript{1}}{}
\textcolor{teal}{\mytextsc{classificateur}} \hspace{4pt} Ton~: L *.

Classificateur des troupeaux; ne s'utilise qu'au singulier.
\textcolor{brown}{\zh{量词:人、牲畜(一群、一队)。}}


\textit{Voir~:} \hyperlink{d7\string_M-t\string_hv\string_=\string_M-gi\#\string_T1}{\textcolor{darkblue}{\textbf{\ipa{dɤ˧-tʰv̩˧-gi\#˥}}}} 

\paragraph{\hspace{-0.5cm} {\Large \textcolor{darkblue}{\textbf{\ipa{dv̩˩}}} \textsubscript{2}}} \hypertarget{dv\string_=\string_B2}{}
\markboth{\textcolor{darkblue}{\textbf{\ipa{dv̩˩}}} \textsubscript{2}}{}
\textcolor{teal}{\mytextsc{pronom}} \hspace{4pt} Ton~: L?.

Démonstratif distal, qui apparaît dans l'indication de direction ``par ici, dans cette direction".
\textcolor{brown}{\zh{指示代词:那边(远指),从‘那个方向’这个短语提取出来的。}}


\begin{exe}
\sn \textcolor{darkblue}{\textbf{\ipa{dv̩˩-tɕo˧}}}
\trans cette direction-là
\trans \textit{\textcolor{brown}{\zh{那个方向}}}
\end{exe}
\begin{exe}
\sn \textcolor{darkblue}{\textbf{\ipa{dv̩˩tɕo˧ fæ˧}}}
\trans cette direction-là
\trans \textit{\textcolor{brown}{\zh{那个方向}}}
\end{exe}
\textit{Voir~:} \hyperlink{d7\string_M-t\string_hv\string_=\string_M-gi\#\string_T1}{\textcolor{darkblue}{\textbf{\ipa{dɤ˧-tʰv̩˧-gi\#˥}}}} 

\paragraph{\hspace{-0.5cm} {\Large \textcolor{darkblue}{\textbf{\ipa{dv̩˩˧}}} \textsubscript{1}}} \hypertarget{dv\string_=\string_B\string_M1}{}
\markboth{\textcolor{darkblue}{\textbf{\ipa{dv̩˩˧}}} \textsubscript{1}}{}
\textcolor{teal}{\mytextsc{nom}} \hspace{4pt} Ton~: LM.

Belette.
\textcolor{brown}{\zh{黄鼠狼,黄喉貂。}}


\begin{exe}
\sn \textcolor{darkblue}{\textbf{\ipa{dv̩˩ hwæ˧-ze˧}}}
\trans ...a acheté une belette
\trans \textit{\textcolor{brown}{\zh{买了黄鼠狼}}}
\end{exe}
\begin{exe}
\sn \textcolor{darkblue}{\textbf{\ipa{dv̩˩ dzɯ˧-ze˩}}}
\trans ...a mangé une belette
\trans \textit{\textcolor{brown}{\zh{吃了黄鼠狼}}}
\end{exe}
 \mytextsc{clf}~: \textcolor{darkblue}{\textbf{\ipa{mi˩}}} 

\paragraph{\hspace{-0.5cm} {\Large \textcolor{darkblue}{\textbf{\ipa{dv̩˩˧}}} \textsubscript{2}}} \hypertarget{dv\string_=\string_B\string_M2}{}
\markboth{\textcolor{darkblue}{\textbf{\ipa{dv̩˩˧}}} \textsubscript{2}}{}
\textcolor{teal}{\mytextsc{nom}} \hspace{4pt} Ton~: LM.

Poison.
\textcolor{brown}{\zh{毒。}}


\textit{Voir~:} \hyperlink{dv\string_=\string_Ba1}{\textcolor{darkblue}{\textbf{\ipa{dv̩˩a}}}} 

\paragraph{\hspace{-0.5cm} {\Large \textcolor{darkblue}{\textbf{\ipa{dv̩˥}}}}} \hypertarget{dv\string_=\string_T1}{}
\markboth{\textcolor{darkblue}{\textbf{\ipa{dv̩˥}}}}{}
\textcolor{teal}{\mytextsc{verbe}} \hspace{4pt} Ton~: H.

Creuser.
\textcolor{brown}{\zh{挖。}}


\begin{exe}
\sn \textcolor{darkblue}{\textbf{\ipa{ʈʂe˧ dv̩˧(-ze˩)}}}
\trans piocher la terre, creuser la terre
\trans \textit{\textcolor{brown}{\zh{挖土}}}
\end{exe}


\paragraph{\hspace{-0.5cm} {\Large \textcolor{darkblue}{\textbf{\ipa{dv̩˩\textsubscript{a}}}}}} \hypertarget{dv\string_=\string_Ba1}{}
\markboth{\textcolor{darkblue}{\textbf{\ipa{dv̩˩\textsubscript{a}}}}}{}
\textcolor{teal}{\mytextsc{verbe}} \hspace{4pt} Ton~: L\textsubscript{a}.
\ding{202} 
Empoisonner, rendre malade.
\textcolor{brown}{\zh{毒害、毒化。}}


\begin{exe}
\sn \textcolor{darkblue}{\textbf{\ipa{ʈʂʰɯ˧, | hĩ˧ dv̩˥-mɤ˩-kv̩˩! | ʈʂʰɯ˧, | hĩ˧ dv̩˥-kv̩˩!}}}
\trans Celui-ci, il n'est pas dangereux / il est comestible! Celui-là [en revanche], il est vénéneux / il est dangereux / il peut vous rendre malade / il peut vous empoisonner / il n'est pas comestible! (Au sujet de diverses sortes de champignons.)
\trans \textit{\textcolor{brown}{\zh{这个,不会让人中毒!那个(反倒)会让人中毒!(情景:谈不同菌子种类。)}}}
\end{exe}
\begin{exe}
\sn \textcolor{darkblue}{\textbf{\ipa{ʈʂʰɯ˧, | dv̩˩-mɤ˩-kv̩˥!}}}
\trans C'est inoffensif/comestible/pas vénéneux/pas dangereux! (Au sujet d'une sorte de champignon.)
\trans \textit{\textcolor{brown}{\zh{这个,不会让人中毒!(情景:谈不同菌子种类。)}}}
\end{exe}
\ding{203} 
Détester.
\textcolor{brown}{\zh{讨厌、恨。}}


\begin{exe}
\sn \textcolor{darkblue}{\textbf{\ipa{le˧-dv̩˩-ze˩}}}
\trans \mytextsc{accomp} \string_ \mytextsc{pfv}
\end{exe}
\begin{exe}
\sn \textcolor{darkblue}{\textbf{\ipa{njɤ˧ | ʈʂʰɯ˧ dv̩˥ | ʐwæ˩˥!}}}
\trans je le déteste!
\trans \textit{\textcolor{brown}{\zh{我很讨厌他!}}}
\end{exe}
\begin{exe}
\sn \textcolor{darkblue}{\textbf{\ipa{dv̩˩-zo˧-mɤ˧-tʰɑ˧˥}}}
\trans détester à mort
\trans \textit{\textcolor{brown}{\zh{讨厌得不行}}}
\end{exe}
\textit{Voir~:} \hyperlink{dv\string_=\string_B\string_M2}{\textcolor{darkblue}{\textbf{\ipa{dv̩˩˧}}} \textsubscript{2}} 

\paragraph{\hspace{-0.5cm} {\Large \textcolor{darkblue}{\textbf{\ipa{dv̩˩\textsubscript{b}}}}}} \hypertarget{dv\string_=\string_Bb1}{}
\markboth{\textcolor{darkblue}{\textbf{\ipa{dv̩˩\textsubscript{b}}}}}{}
\textcolor{teal}{\mytextsc{classificateur}} \hspace{4pt} Ton~: L\textsubscript{b}.

Classificateur des petits groupes (de personnes): quelques-uns (plus de 3).
\textcolor{brown}{\zh{量词:人(一些)。}}


\begin{exe}
\sn \textcolor{darkblue}{\textbf{\ipa{hĩ˧ ɖɯ˧-dv̩˩}}}
\trans quelques personnes, un groupe de personnes
\trans \textit{\textcolor{brown}{\zh{一些人}}}
\end{exe}
\begin{exe}
\sn \textcolor{darkblue}{\textbf{\ipa{hĩ˧ ʈʂʰɯ˧-dv̩˥}}}
\trans ce groupe de gens, ces qq personnes
\trans \textit{\textcolor{brown}{\zh{这些人}}}
\end{exe}


\paragraph{\hspace{-0.5cm} {\Large \textcolor{darkblue}{\textbf{\ipa{dv̩˩bi˩}}}}} \hypertarget{dv\string_=\string_Bbi\string_B1}{}
\markboth{\textcolor{darkblue}{\textbf{\ipa{dv̩˩bi˩}}}}{}
\textcolor{teal}{\mytextsc{adverbe}} \hspace{4pt} Ton~: L.

En face.
\textcolor{brown}{\zh{对面。}}




\paragraph{\hspace{-0.5cm} {\Large \textcolor{darkblue}{\textbf{\ipa{dv̩˩mi\#˥}}}}} \hypertarget{dv\string_=\string_Bmi\#\string_T1}{}
\markboth{\textcolor{darkblue}{\textbf{\ipa{dv̩˩mi\#˥}}}}{}
\textcolor{teal}{\mytextsc{nom}} \hspace{4pt} Ton~: LM+\#H.

Belette femelle.
\textcolor{brown}{\zh{母黄鼠狼。}}


\begin{exe}
\sn \textcolor{darkblue}{\textbf{\ipa{dv̩˩mi˧-dv̩˥pʰv̩˩}}}
\trans belette femelle et belette mâle
\trans \textit{\textcolor{brown}{\zh{母黄鼠狼与公黄鼠狼}}}
\end{exe}
 \mytextsc{clf}~: \textcolor{darkblue}{\textbf{\ipa{mi˩}}} 

\paragraph{\hspace{-0.5cm} {\Large \textcolor{darkblue}{\textbf{\ipa{dv̩˩pʰæ˧}}}}} \hypertarget{dv\string_=\string_Bp\string_h\{\string_M1}{}
\markboth{\textcolor{darkblue}{\textbf{\ipa{dv̩˩pʰæ˧}}}}{}
\textcolor{teal}{\mytextsc{nom}} \hspace{4pt} Ton~: LM.

Partie du bâtiment principal dans laquelle étaient conservées les céréales: le grenier à céréales.
\textcolor{brown}{\zh{仓廪。摩梭话音译:‘独帕’。}}


 \mytextsc{clf}~: \textcolor{darkblue}{\textbf{\ipa{ɭɯ˧}}} 

\paragraph{\hspace{-0.5cm} {\Large \textcolor{darkblue}{\textbf{\ipa{dv̩˩pʰv̩\#˥}}}}} \hypertarget{dv\string_=\string_Bp\string_hv\string_=\#\string_T1}{}
\markboth{\textcolor{darkblue}{\textbf{\ipa{dv̩˩pʰv̩\#˥}}}}{}
\textcolor{teal}{\mytextsc{nom}} \hspace{4pt} Ton~: LM+\#H / LM.

Belette mâle.
\textcolor{brown}{\zh{公黄鼠狼。}}


 \mytextsc{clf}~: \textcolor{darkblue}{\textbf{\ipa{mi˩}}} 

\paragraph{\hspace{-0.5cm} {\Large \textcolor{darkblue}{\textbf{\ipa{dv̩˩zo\#˥}}}}} \hypertarget{dv\string_=\string_Bzo\#\string_T1}{}
\markboth{\textcolor{darkblue}{\textbf{\ipa{dv̩˩zo\#˥}}}}{}
\textcolor{teal}{\mytextsc{nom}} \hspace{4pt} Ton~: LM+\#H / LM.

Bébé belette.
\textcolor{brown}{\zh{黄鼠狼的崽子。}}




\newpage
\section*{\centering- \textcolor{darkblue}{\textbf{\ipa{dz}}} -}
\paragraph{\hspace{-0.5cm} {\Large \textcolor{darkblue}{\textbf{\ipa{dzɑ˥}}}}} \hypertarget{dzA\string_T1}{}
\markboth{\textcolor{darkblue}{\textbf{\ipa{dzɑ˥}}}}{}
\textcolor{teal}{\mytextsc{adjectif}} \hspace{4pt} Ton~: H.





\ding{202} 
Mauvais (action…), inférieur, indigne.
\textcolor{brown}{\zh{坏、差、下级(行为……)。}}


\begin{exe}
\sn \textcolor{darkblue}{\textbf{\ipa{ʈʂʰɯ˧-ɳɯ˧ | njɤ˧-ki˧ | dzɑ˧-ʝi˧ | ʐwæ˩˥!}}}
\trans il me méprise vraiment!
\trans \textit{\textcolor{brown}{\zh{他很瞧不起我!}}}
\end{exe}
\begin{exe}
\sn \textcolor{darkblue}{\textbf{\ipa{hĩ˧ ʈʂʰɯ˧-v̩˧ dʑo˩, | õ˧-ki˥ | dzɑ˧-ʝi˧-ze˩!}}}
\trans cette personne ne se respecte pas!
\trans \textit{\textcolor{brown}{\zh{这个人,不尊重自己!}}}
\end{exe}
\begin{exe}
\sn \textcolor{darkblue}{\textbf{\ipa{mv̩˧ dzɑ˧.}}}
\trans Il fait mauvais temps.
\trans \textit{\textcolor{brown}{\zh{天气很坏。}}}
\end{exe}
\begin{exe}
\sn \textcolor{darkblue}{\textbf{\ipa{mv̩˧ dzɑ˧-ze˩}}}
\trans Le temps se met au mauvais, il commence à faire mauvais temps.
\trans \textit{\textcolor{brown}{\zh{天气变坏了。}}}
\end{exe}
\begin{exe}
\sn \textcolor{darkblue}{\textbf{\ipa{lo˧ dzɑ˧}}}
\trans bâclé, mal fait (travail)
\trans \textit{\textcolor{brown}{\zh{(工作)差}}}
\end{exe}
\ding{203} 
Indigent, pauvre (personne…).
\textcolor{brown}{\zh{穷(人)。}}


\begin{exe}
\sn \textcolor{darkblue}{\textbf{\ipa{dzɑ˧ | -ʐwæ˩-ze˥!}}}
\trans (il/elle est) très pauvre!
\trans \textit{\textcolor{brown}{\zh{他很穷!}}}
\end{exe}
\begin{exe}
\sn \textcolor{darkblue}{\textbf{\ipa{ɑ˩ʁo˧ | bo˩ʈʂʰæ˧ mɤ˧-dʑo˧, | dzɑ˧ ʈʂɤ˧-kv̩˩!}}}
\trans Ne pas avoir de cochon-entier-conservé à la maison, ça fait vraiment mauvais effet/ça fait vraiment indigent/c'est la honte!
\trans \textit{\textcolor{brown}{\zh{如果家里没有猪膘,会显得很穷!}}}
\end{exe}
\begin{exe}
\sn \textcolor{darkblue}{\textbf{\ipa{dzɑ˧ ʈʂɤ˧ | ʐwæ˩˥!}}}
\trans C'est vraiment la honte/on paraît vraiment à la rue! (Au sujet de situations stigmatisées socialement, comme de ne pas posséder les nourritures ou vêtement nécessaires aux principaux rituels.)
\trans \textit{\textcolor{brown}{\zh{真羞耻啊!}}}
\end{exe}


\paragraph{\hspace{-0.5cm} {\Large \textcolor{darkblue}{\textbf{\ipa{dzɑ˩qʰwɤ˩}}}}} \hypertarget{dzA\string_Bq\string_hw7\string_B1}{}
\markboth{\textcolor{darkblue}{\textbf{\ipa{dzɑ˩qʰwɤ˩}}}}{}
\textcolor{teal}{\mytextsc{nom}} \hspace{4pt} Ton~: L.

Chaussure.
\textcolor{brown}{\zh{鞋、鞋子。}}


 \mytextsc{clf}~: \textcolor{darkblue}{\textbf{\ipa{dzi˧}}} 

\paragraph{\hspace{-0.5cm} {\Large \textcolor{darkblue}{\textbf{\ipa{dze˥}}}}} \hypertarget{dze\string_T1}{}
\markboth{\textcolor{darkblue}{\textbf{\ipa{dze˥}}}}{}
\textcolor{teal}{\mytextsc{nom}} \hspace{4pt} Ton~: \#H.

Sucre.
\textcolor{brown}{\zh{糖。}}




\paragraph{\hspace{-0.5cm} {\Large \textcolor{darkblue}{\textbf{\ipa{dze˧bɤ˩}}}}} \hypertarget{dze\string_Mb7\string_B1}{}
\markboth{\textcolor{darkblue}{\textbf{\ipa{dze˧bɤ˩}}}}{}
\textcolor{teal}{\mytextsc{nom}} \hspace{4pt} Ton~: L\#.

Chauve-souris; s'emploie pour toutes les espèces, y compris le renard volant.
\textcolor{brown}{\zh{蝙蝠、飞鼠。}}


\begin{exe}
\sn \textcolor{darkblue}{\textbf{\ipa{dze˧bɤ˩-zo˩ | ɖɯ˧-ɭɯ˧}}}
\trans une petite chauve-souris, un bébé chauve-souris
\trans \textit{\textcolor{brown}{\zh{一只小蝙蝠}}}
\end{exe}
\begin{exe}
\sn \textcolor{darkblue}{\textbf{\ipa{dze˧bɤ˩-pʰv̩˩ | ɖɯ˧-mi˩}}}
\trans une chauve-souris mâle
\trans \textit{\textcolor{brown}{\zh{一只公蝙蝠}}}
\end{exe}
\begin{exe}
\sn \textcolor{darkblue}{\textbf{\ipa{dze˧bɤ˩-mi˩ | ɖɯ˧-mi˩}}}
\trans une chauve-souris femelle
\trans \textit{\textcolor{brown}{\zh{一只母蝙蝠}}}
\end{exe}
 \mytextsc{clf}~: \textcolor{darkblue}{\textbf{\ipa{mi˩}}} 

\paragraph{\hspace{-0.5cm} {\Large \textcolor{darkblue}{\textbf{\ipa{dze˧bo˧}}}}} \hypertarget{dze\string_Mbo\string_M1}{}
\markboth{\textcolor{darkblue}{\textbf{\ipa{dze˧bo˧}}}}{}
\textcolor{teal}{\mytextsc{nom}} \hspace{4pt} Ton~: M.
\ding{202} 
Nom de clan/famille étendue qui compte 3 familles. C'est également le nom d'un village de la plaine de Yongning.
\textcolor{brown}{\zh{者波(姓)。这个家族有三个家庭。}}


\begin{exe}
\sn \textcolor{darkblue}{\textbf{\ipa{dze˧bo˧=ɻ̍˩}}}
\trans le clan \textcolor{darkblue}{\textbf{\ipa{/dze˧bo˧/}}}, la famille \textcolor{darkblue}{\textbf{\ipa{/dze˧bo˧/}}}
\trans \textit{\textcolor{brown}{\zh{者波家族}}}
\end{exe}
\ding{203} 
Zhebo, un village de la plaine de Yongning. Il est divisé en deux parties, ``du haut" et ``du bas": \textcolor{darkblue}{\textbf{\ipa{/gɤ˩ʁwɤ˧/}}} et \textcolor{darkblue}{\textbf{\ipa{/mv̩˩ʁwɤ˧/}}}.
\textcolor{brown}{\zh{者波(永宁的一个村落)。村落有两个部分,\textcolor{darkblue}{\textbf{\ipa{/gɤ˩ʁwɤ˧/}}}‘上村’与\textcolor{darkblue}{\textbf{\ipa{/mv̩˩ʁwɤ˧/}}}‘下村’.}}


\begin{exe}
\sn \textcolor{darkblue}{\textbf{\ipa{ɖæ˩ʂɯ\#˥, | ʈʂo˧ʂɯ\#˥, | bɤ˩tɕʰɯ˩˥, | dɑ˧pʰo˥, | bɤ˧dzi˩, | dze˧bo˧}}}
\trans les six villages de la plaine de Yongning, dans l'ordre, qui prend comme point d'origine le village le plus proche du Lac
\trans \textit{\textcolor{brown}{\zh{永宁坝的六个村落,按传统排序:从距离泸沽湖最近的村落说起。}}}
\end{exe}


\paragraph{\hspace{-0.5cm} {\Large \textcolor{darkblue}{\textbf{\ipa{dze˧dv̩˩}}}}} \hypertarget{dze\string_Mdv\string_=\string_B1}{}
\markboth{\textcolor{darkblue}{\textbf{\ipa{dze˧dv̩˩}}}}{}
\textcolor{teal}{\mytextsc{nom}} \hspace{4pt} Ton~: L\#.

Galette de céréale (blé, avoine…), pain.
\textcolor{brown}{\zh{饼。}}


\begin{exe}
\sn \textcolor{darkblue}{\textbf{\ipa{dze˧dv̩˩-pɤ˩jɤ˩}}}
\trans galette de céréale (même sens)
\trans \textit{\textcolor{brown}{\zh{粮食饼}}}
\end{exe}


\paragraph{\hspace{-0.5cm} {\Large \textcolor{darkblue}{\textbf{\ipa{dze˧hi˧}}}}} \hypertarget{dze\string_Mhi\string_M1}{}
\markboth{\textcolor{darkblue}{\textbf{\ipa{dze˧hi˧}}}}{}
\textcolor{teal}{\mytextsc{nom}} \hspace{4pt} Ton~: M.

Beaux-parents.
\textcolor{brown}{\zh{丈人。}}


\begin{exe}
\sn \textcolor{darkblue}{\textbf{\ipa{njɤ˧ | dze˧hi˧-ki˩ bi˩!}}}
\trans Je vais chez mes beaux-parents!
\trans \textit{\textcolor{brown}{\zh{我去我丈人(那边)!}}}
\end{exe}
\begin{exe}
\sn \textcolor{darkblue}{\textbf{\ipa{no˧ | dze˧hi˧ | ə˩-to˩-ze˥? - le˧-to˩-ze˩!}}}
\trans Tu as une belle-famille? =Tu es marié(e)? -Oui!
\trans \textit{\textcolor{brown}{\zh{你有丈人吗?(=你结婚了吗?)-有的!(=结婚了!)}}}
\end{exe}
\begin{exe}
\sn \textcolor{darkblue}{\textbf{\ipa{no˧ | dze˧hi˧ to˩ ə˩-bi˩?}}}
\trans Tu comptes te marier? (Littéralement: Tu vas te lier avec une belle-famille?)
\trans \textit{\textcolor{brown}{\zh{你打算结婚吗?}}}
\end{exe}


\paragraph{\hspace{-0.5cm} {\Large \textcolor{darkblue}{\textbf{\ipa{dze˧kʰɤ˧˥}}}}} \hypertarget{dze\string_Mk\string_h7\string_M\string_T1}{}
\markboth{\textcolor{darkblue}{\textbf{\ipa{dze˧kʰɤ˧˥}}}}{}
\textcolor{teal}{\mytextsc{nom}} \hspace{4pt} Ton~: MH\#.

Roturier, la 2e des 3 castes de la société ancienne, intermédiaire entre la noblesse et les serfs.
\textcolor{brown}{\zh{百姓。音译:“责卡”。}}


 \mytextsc{clf}~: \textcolor{darkblue}{\textbf{\ipa{v̩˧}}} 

\paragraph{\hspace{-0.5cm} {\Large \textcolor{darkblue}{\textbf{\ipa{dze˧ɭɯ˧}}}}} \hypertarget{dze\string_Ml\string_RM\string_M1}{}
\markboth{\textcolor{darkblue}{\textbf{\ipa{dze˧ɭɯ˧}}}}{}
\textcolor{teal}{\mytextsc{nom}} \hspace{4pt} Ton~: M.

Blé, froment.
\textcolor{brown}{\zh{小麦。}}




\paragraph{\hspace{-0.5cm} {\Large \textcolor{darkblue}{\textbf{\ipa{dze˧ɭɯ˧-ɻ̃\#˥}}}}} \hypertarget{dze\string_Ml\string_RM\string_M-r£`\string_~\#\string_T1}{}
\markboth{\textcolor{darkblue}{\textbf{\ipa{dze˧ɭɯ˧-ɻ̃\#˥}}}}{}
\textcolor{teal}{\mytextsc{nom}} \hspace{4pt} Ton~: \#H.

Paille de blé.
\textcolor{brown}{\zh{麦杆。}}




\paragraph{\hspace{-0.5cm} {\Large \textcolor{darkblue}{\textbf{\ipa{dze˧-ɻ̃\#˥}}}}} \hypertarget{dze\string_M-r£`\string_~\#\string_T1}{}
\markboth{\textcolor{darkblue}{\textbf{\ipa{dze˧-ɻ̃\#˥}}}}{}
\textcolor{teal}{\mytextsc{nom}} \hspace{4pt} Ton~: \#H.

Paille de blé.
\textcolor{brown}{\zh{小麦秆。}}




\paragraph{\hspace{-0.5cm} {\Large \textcolor{darkblue}{\textbf{\ipa{dze˧-tɕʰi\#˥}}}}} \hypertarget{dze\string_M-ts£\string_hi\#\string_T1}{}
\markboth{\textcolor{darkblue}{\textbf{\ipa{dze˧-tɕʰi\#˥}}}}{}
\textcolor{teal}{\mytextsc{nom}} \hspace{4pt} Ton~: \#H.

Barbe de blé.
\textcolor{brown}{\zh{麦芒。}}




\paragraph{\hspace{-0.5cm} {\Large \textcolor{darkblue}{\textbf{\ipa{dze˧-ʈʂæ˥}}}}} \hypertarget{dze\string_M-t`s`\{\string_T1}{}
\markboth{\textcolor{darkblue}{\textbf{\ipa{dze˧-ʈʂæ˥}}}}{}
\textcolor{teal}{\mytextsc{nom}} \hspace{4pt} Ton~: H\#.

Dard de l'abeille.
\textcolor{brown}{\zh{蜜蜂的螫針。}}


 \mytextsc{clf}~: \textcolor{darkblue}{\textbf{\ipa{ɭɯ˧}}} 

\paragraph{\hspace{-0.5cm} {\Large \textcolor{darkblue}{\textbf{\ipa{dze˧ʈʂɯ˧}}}}} \hypertarget{dze\string_Mt`s`M\string_M1}{}
\markboth{\textcolor{darkblue}{\textbf{\ipa{dze˧ʈʂɯ˧}}}}{}
\textcolor{teal}{\mytextsc{nom}} \hspace{4pt} Ton~: M.

Vanneries: tamis, crible.
\textcolor{brown}{\zh{筛子。}}


 \mytextsc{clf}~: \textcolor{darkblue}{\textbf{\ipa{nɑ˧}}} 

\paragraph{\hspace{-0.5cm} {\Large \textcolor{darkblue}{\textbf{\ipa{}}}}} \hypertarget{dze\string_Mt`s`\string_h7\$\string_T1}{}
\markboth{\textcolor{darkblue}{\textbf{\ipa{}}}}{}
\textcolor{teal}{\mytextsc{nom}} \hspace{4pt} Ton~: H\$.

Céréales; la céréale traditionnelle était l'orge, mais le sens de l'expression tend actuellement à se confondre avec celui de l'expression chinoise \textcolor{brown}{\zh{五谷}} 'les cinq céréales': le riz; deux sortes de millet; le blé; et les fèves.
\textcolor{brown}{\zh{粮食。现在,这个词的含义受到汉语‘五谷’这个词的影响,用来指代‘五谷杂粮’,相当于所有粮食类,如:小米类、稻谷、麦子、玉米以及豆类与薯类。}}




\paragraph{\hspace{-0.5cm} {\Large \textcolor{darkblue}{\textbf{\ipa{dze˩}}}}} \hypertarget{dze\string_B1}{}
\markboth{\textcolor{darkblue}{\textbf{\ipa{dze˩}}}}{}
\textcolor{teal}{\mytextsc{verbe}} \hspace{4pt} Ton~: L.

Rester, être en trop, devenir un reste (nourriture, boisson).
\textcolor{brown}{\zh{剩下(饭或饮料)。}}


\begin{exe}
\sn \textcolor{darkblue}{\textbf{\ipa{dzɯ˧-dze˥-ze˩!}}}
\trans il y a des restes! / on n'a pas achevé de manger (un plat)!
\trans \textit{\textcolor{brown}{\zh{剩了一些饭!/ 剩了一些吃的!}}}
\end{exe}
\begin{exe}
\sn \textcolor{darkblue}{\textbf{\ipa{gɤ˩-dze˥ +ze˩!}}}
\trans Il en reste / il y a des restes!
\trans \textit{\textcolor{brown}{\zh{有剩下的!}}}
\end{exe}
\begin{exe}
\sn \textcolor{darkblue}{\textbf{\ipa{ʈʰɯ˩ dze˩-ze˥}}}
\trans Il en reste / on n'a pas achevé de boire (un verre…); ne pas être entièrement bu
\trans \textit{\textcolor{brown}{\zh{喝剩了、没喝完}}}
\end{exe}
\begin{exe}
\sn \textcolor{darkblue}{\textbf{\ipa{le˧-se˩-ze˩! | gɤ˩-mɤ˧-dze˩!}}}
\trans on a tout fini, il n'y a pas de restes! / tout a été mangé, bu..., il n'en reste plus !
\trans \textit{\textcolor{brown}{\zh{完了!(=全部吃/喝完了!)没有剩!}}}
\end{exe}


\paragraph{\hspace{-0.5cm} {\Large \textcolor{darkblue}{\textbf{\ipa{dze˩\textsubscript{a}}}}}} \hypertarget{dze\string_Ba1}{}
\markboth{\textcolor{darkblue}{\textbf{\ipa{dze˩\textsubscript{a}}}}}{}
\textcolor{teal}{\mytextsc{classificateur}} \hspace{4pt} Ton~: L\textsubscript{a}.

Classificateur des lots de deux objets non indissociables; par ex.: lot de 2 casseroles; paire de haut-parleurs… Pour les paires non dissociables (ex.: paire de chaussures), on utilise: /dzi˧/.
\textcolor{brown}{\zh{量词:瓶子、锅(一对)。}}


\begin{exe}
\sn \textcolor{darkblue}{\textbf{\ipa{zo˧mv̩˥ | ɖɯ˧-dze˩}}}
\trans des jumeaux (littéralement ``une paire d'enfants")
\trans \textit{\textcolor{brown}{\zh{双胞胎(直译:“一对孩子”)}}}
\end{exe}
\begin{exe}
\sn \textcolor{darkblue}{\textbf{\ipa{ʈʂʰɯ˧-dze˥}}}
\trans \mytextsc{dem} \string_ (ton: H\# / H\$)
\trans \textit{\textcolor{brown}{\zh{\mytextsc{指示代词} \string_}}}
\end{exe}


\paragraph{\hspace{-0.5cm} {\Large \textcolor{darkblue}{\textbf{\ipa{dze˩\textsubscript{a}}}} \textsubscript{1}}} \hypertarget{dze\string_Ba1}{}
\markboth{\textcolor{darkblue}{\textbf{\ipa{dze˩\textsubscript{a}}}} \textsubscript{1}}{}
\textcolor{teal}{\mytextsc{verbe}} \hspace{4pt} Ton~: L\textsubscript{a}.

Voler (dans les airs).
\textcolor{brown}{\zh{飞。}}


\begin{exe}
\sn \textcolor{darkblue}{\textbf{\ipa{le˧-dze˩-hɯ˩-ze˩}}}
\trans (L'oiseau) est parti à tire-d'aile.
\trans \textit{\textcolor{brown}{\zh{(鸟)飞走了。}}}
\end{exe}
\begin{exe}
\sn \textcolor{darkblue}{\textbf{\ipa{mv̩˧ʁo˧ dze˧˥}}}
\trans voler dans le ciel
\trans \textit{\textcolor{brown}{\zh{在天空中飞}}}
\end{exe}


\paragraph{\hspace{-0.5cm} {\Large \textcolor{darkblue}{\textbf{\ipa{dze˩\textsubscript{a}}}} \textsubscript{2}}} \hypertarget{dze\string_Ba2}{}
\markboth{\textcolor{darkblue}{\textbf{\ipa{dze˩\textsubscript{a}}}} \textsubscript{2}}{}
\textcolor{teal}{\mytextsc{verbe}} \hspace{4pt} Ton~: L\textsubscript{a}.

Couper (avec un couteau).
\textcolor{brown}{\zh{切(用刀)。}}


\begin{exe}
\sn \textcolor{darkblue}{\textbf{\ipa{le˧-dze˩}}}
\trans \mytextsc{accomp}
\trans \textit{\textcolor{brown}{\zh{\mytextsc{accomp}}}}
\end{exe}
\begin{exe}
\sn \textcolor{darkblue}{\textbf{\ipa{dze˧\textasciitilde{}dze˥}}}
\trans \mytextsc{red}
\trans \textit{\textcolor{brown}{\zh{\mytextsc{red}}}}
\end{exe}
\begin{exe}
\sn \textcolor{darkblue}{\textbf{\ipa{le˧-dze˧\textasciitilde{}dze˥}}}
\trans \mytextsc{accomp} \string_ \mytextsc{red}
\trans \textit{\textcolor{brown}{\zh{\mytextsc{accomp} \string_ \mytextsc{red}}}}
\end{exe}
\begin{exe}
\sn \textcolor{darkblue}{\textbf{\ipa{v̩˩tsʰɤ˧ dze˧\textasciitilde{}dze˥}}}
\trans découper des légumes
\trans \textit{\textcolor{brown}{\zh{切菜}}}
\end{exe}
\begin{exe}
\sn \textcolor{darkblue}{\textbf{\ipa{nv̩˩dʑɯ˥ dze˩\textasciitilde{}dze˩}}}
\trans découper du tofu
\trans \textit{\textcolor{brown}{\zh{切豆腐}}}
\end{exe}


\paragraph{\hspace{-0.5cm} {\Large \textcolor{darkblue}{\textbf{\ipa{dze˩dʑɯ˧˥}}}}} \hypertarget{dze\string_Bdz£M\string_M\string_T1}{}
\markboth{\textcolor{darkblue}{\textbf{\ipa{dze˩dʑɯ˧˥}}}}{}
\textcolor{teal}{\mytextsc{adjectif}} \hspace{4pt} Ton~: LM+MH\#.

Orgueilleux, arrogant.
\textcolor{brown}{\zh{骄傲,自以为好。}}


\begin{exe}
\sn \textcolor{darkblue}{\textbf{\ipa{ʈʂʰɯ˧ | hĩ˧-bi˥ | mɤ˧-li˧! | dze˩dʑɯ˧˥ | ʐwæ˧˥!}}}
\trans il méprise les autres! il est très orgueilleux!
\trans \textit{\textcolor{brown}{\zh{他看不起别人!他很骄傲!}}}
\end{exe}


\paragraph{\hspace{-0.5cm} {\Large \textcolor{darkblue}{\textbf{\ipa{dze˩mi˧}}}}} \hypertarget{dze\string_Bmi\string_M1}{}
\markboth{\textcolor{darkblue}{\textbf{\ipa{dze˩mi˧}}}}{}
\textcolor{teal}{\mytextsc{nom}} \hspace{4pt} Ton~: LM.

Abeille.
\textcolor{brown}{\zh{蜜蜂。}}


 \mytextsc{clf}~: \textcolor{darkblue}{\textbf{\ipa{mi˩}}} 

\paragraph{\hspace{-0.5cm} {\Large \textcolor{darkblue}{\textbf{\ipa{dze˩mi˧-bæ˩bæ˩}}}}} \hypertarget{dze\string_Bmi\string_M-b\{\string_Bb\{\string_B1}{}
\markboth{\textcolor{darkblue}{\textbf{\ipa{dze˩mi˧-bæ˩bæ˩}}}}{}
\textcolor{teal}{\mytextsc{nom}} \hspace{4pt} Ton~: L\#.

\textit{Artemisia suboligata}; littéralement ``la fleur aux abeilles".
\textcolor{brown}{\zh{茶绒蒿。}}


 \mytextsc{clf}~: \textcolor{darkblue}{\textbf{\ipa{bæ˩}}} 

\paragraph{\hspace{-0.5cm} {\Large \textcolor{darkblue}{\textbf{\ipa{dze˩mi˧-dze\#˥}}}}} \hypertarget{dze\string_Bmi\string_M-dze\#\string_T1}{}
\markboth{\textcolor{darkblue}{\textbf{\ipa{dze˩mi˧-dze\#˥}}}}{}
\textcolor{teal}{\mytextsc{nom}} \hspace{4pt} Ton~: LM+\#H.
\textit{De:} \textbf{dze˩mi˧ et dze˥} 
Miel.
\textcolor{brown}{\zh{蜂蜜。}}


\begin{exe}
\sn \textcolor{darkblue}{\textbf{\ipa{dze˩mi˧dze˧ dzɯ˧}}}
\trans manger du miel
\trans \textit{\textcolor{brown}{\zh{吃蜂蜜}}}
\end{exe}
 \mytextsc{clf}~: \textcolor{darkblue}{\textbf{\ipa{kʰwɤ˥}}} 

\paragraph{\hspace{-0.5cm} {\Large \textcolor{darkblue}{\textbf{\ipa{dze˩mi˧-kʰv̩˩}}}}} \hypertarget{dze\string_Bmi\string_M-k\string_hv\string_=\string_B1}{}
\markboth{\textcolor{darkblue}{\textbf{\ipa{dze˩mi˧-kʰv̩˩}}}}{}
\textcolor{teal}{\mytextsc{nom}} \hspace{4pt} Ton~: LM-L.

Ruche.
\textcolor{brown}{\zh{蜂窝。}}


 \mytextsc{clf}~: \textcolor{darkblue}{\textbf{\ipa{ɭɯ˧}}} 

\paragraph{\hspace{-0.5cm} {\Large \textcolor{darkblue}{\textbf{\ipa{dze˩mi˧-pv̩˥ɻ̍˩}}}}} \hypertarget{dze\string_Bmi\string_M-pv\string_=\string_Tr£`̍\string_B1}{}
\markboth{\textcolor{darkblue}{\textbf{\ipa{dze˩mi˧-pv̩˥ɻ̍˩}}}}{}
\textcolor{teal}{\mytextsc{nom}} \hspace{4pt} Ton~: LM+\#H-.

\textit{Adenophora sp.}.
\textcolor{brown}{\zh{沙参。}}Dialecte chinois local~: \zh{yyyy fusionner les 2 entrées; est un terme générique ; tʰi˧-pv˥ɻ˩ = tʰi˧-hɑ̃˧˥ se reposer qq part, se poser quelque part。}


\begin{exe}
\sn \textcolor{darkblue}{\textbf{\ipa{dze˩mi˧-pv̩˥ɻ̍˩-kʰɯ˩ʈɯ˩}}}
\trans racine d'\textit{Adenophora sp.}
\trans \textit{\textcolor{brown}{\zh{沙参根}}}
\end{exe}


\paragraph{\hspace{-0.5cm} {\Large \textcolor{darkblue}{\textbf{\ipa{dze˩mi˧-pv̩˥ɻ̍˩}}}}} \hypertarget{dze\string_Bmi\string_M-pv\string_=\string_Tr£`̍\string_B1}{}
\markboth{\textcolor{darkblue}{\textbf{\ipa{dze˩mi˧-pv̩˥ɻ̍˩}}}}{}
\textcolor{teal}{\mytextsc{nom}} \hspace{4pt} Ton~: LM+\#H-.

Gentiane.
\textcolor{brown}{\zh{秦艽。}}


\begin{exe}
\sn \textcolor{darkblue}{\textbf{\ipa{dʑɯ˧qʰɑ˧-bæ˩bæ˩}}}
\trans fleurs de gentiane
\trans \textit{\textcolor{brown}{\zh{秦艽花}}}
\end{exe}
 \mytextsc{clf}~: \textcolor{darkblue}{\textbf{\ipa{qɑ˩}}} 

\paragraph{\hspace{-0.5cm} {\Large \textcolor{darkblue}{\textbf{\ipa{*dze˩˧}}}}} \hypertarget{*dze\string_B\string_M1}{}
\markboth{\textcolor{darkblue}{\textbf{\ipa{*dze˩˧}}}}{}
\textcolor{teal}{\mytextsc{nom}} \hspace{4pt} Ton~: LM.

Abeille (racine déduite du disyllabe).
\textcolor{brown}{\zh{蜜蜂。}}




\paragraph{\hspace{-0.5cm} {\Large \textcolor{darkblue}{\textbf{\ipa{dze˩˧}}}}} \hypertarget{dze\string_B\string_M1}{}
\markboth{\textcolor{darkblue}{\textbf{\ipa{dze˩˧}}}}{}
\textcolor{teal}{\mytextsc{nom}} \hspace{4pt} Ton~: LM.

Xanthoxyle, poivre de Chine, poivre du Sichuan.
\textcolor{brown}{\zh{花椒。}}


 \mytextsc{clf}~: \textcolor{darkblue}{\textbf{\ipa{mɤ˩}}} un peu

\paragraph{\hspace{-0.5cm} {\Large \textcolor{darkblue}{\textbf{\ipa{dzɤ˥\textsubscript{b}}}}}} \hypertarget{dz7\string_Tb1}{}
\markboth{\textcolor{darkblue}{\textbf{\ipa{dzɤ˥\textsubscript{b}}}}}{}
\textcolor{teal}{\mytextsc{classificateur}} \hspace{4pt} Ton~: H\textsubscript{b}.

Côté.
\textcolor{brown}{\zh{量词:面。}}


\begin{exe}
\sn \textcolor{darkblue}{\textbf{\ipa{ʈʂʰɯ˧-dzɤ˧}}}
\trans ce côté-ci
\trans \textit{\textcolor{brown}{\zh{这面}}}
\end{exe}
\begin{exe}
\sn \textcolor{darkblue}{\textbf{\ipa{ɖɯ˧-dzɤ˥}}}
\trans un côté
\trans \textit{\textcolor{brown}{\zh{一面}}}
\end{exe}


\paragraph{\hspace{-0.5cm} {\Large \textcolor{darkblue}{\textbf{\ipa{dzɤ˩\textsubscript{a}}}}}} \hypertarget{dz7\string_Ba1}{}
\markboth{\textcolor{darkblue}{\textbf{\ipa{dzɤ˩\textsubscript{a}}}}}{}
\textcolor{teal}{\mytextsc{verbe}} \hspace{4pt} Ton~: L\textsubscript{a}.

S’écrouler, s'effondrer (mur); tomber (arbre), se renverser, s'abattre.
\textcolor{brown}{\zh{塌毁,倒塌 ,倒。}}


\begin{exe}
\sn \textcolor{darkblue}{\textbf{\ipa{mv̩˩tɕo˧ dzɤ˩}}}
\trans même sens: s'écrouler
\trans \textit{\textcolor{brown}{\zh{同上:塌毁}}}
\end{exe}
\begin{exe}
\sn \textcolor{darkblue}{\textbf{\ipa{le˧-dzɤ˩-ze˩}}}
\trans \mytextsc{accomp} \string_ \mytextsc{pfv}
\trans \textit{\textcolor{brown}{\zh{塌毁了}}}
\end{exe}


\paragraph{\hspace{-0.5cm} {\Large \textcolor{darkblue}{\textbf{\ipa{dzi˥}}}}} \hypertarget{dzi\string_T1}{}
\markboth{\textcolor{darkblue}{\textbf{\ipa{dzi˥}}}}{}
\textcolor{teal}{\mytextsc{nom}} \hspace{4pt} Ton~: \#H.

Burin, ciseau.
\textcolor{brown}{\zh{凿子。}}


 \mytextsc{clf}~: \textcolor{darkblue}{\textbf{\ipa{ɭɯ˧ (*nɑ˧)}}} 

\paragraph{\hspace{-0.5cm} {\Large \textcolor{darkblue}{\textbf{\ipa{dzi˧\textsubscript{b}}}}}} \hypertarget{dzi\string_Mb1}{}
\markboth{\textcolor{darkblue}{\textbf{\ipa{dzi˧\textsubscript{b}}}}}{}
\textcolor{teal}{\mytextsc{classificateur}} \hspace{4pt} Ton~: M\textsubscript{b}.

Paire d'objets qui constitue une unité: par exemple une paire de chaussures.
\textcolor{brown}{\zh{量词:鞋(一双)。}}


\begin{exe}
\sn \textcolor{darkblue}{\textbf{\ipa{ɣɯ˩-dzɑ˩qʰwɤ˥ | ɖɯ˧-dzi˧}}}
\trans une paire de chaussures en cuir
\trans \textit{\textcolor{brown}{\zh{一双皮鞋}}}
\end{exe}


\paragraph{\hspace{-0.5cm} {\Large \textcolor{darkblue}{\textbf{\ipa{dzi˧dzi˧}}}}} \hypertarget{dzi\string_Mdzi\string_M1}{}
\markboth{\textcolor{darkblue}{\textbf{\ipa{dzi˧dzi˧}}}}{}
\textcolor{teal}{\mytextsc{nom}} \hspace{4pt} Ton~: M.

Chêne blanc oriental.
\textcolor{brown}{\zh{青冈树、槲栎。}}


\begin{exe}
\sn \textcolor{darkblue}{\textbf{\ipa{dzi˧dzi˧, | si˧dzi˩-mv̩˩!}}}
\trans \textcolor{darkblue}{\textbf{\ipa{/dzi˧dzi˧/}}}, c'est un nom d'arbre! / C'est le nom d'un arbre!
\trans \textit{\textcolor{brown}{\zh{\textcolor{darkblue}{\textbf{\ipa{dzi˧dzi˧}}}是一种树的名字!}}}
\end{exe}
\textit{Syn~:} \hyperlink{dz£M\string_Bsi\string_B1}{\textcolor{darkblue}{\textbf{\ipa{dʑɯ˩si˩}}}}. 

\paragraph{\hspace{-0.5cm} {\Large \textcolor{darkblue}{\textbf{\ipa{dzi˧dzi˧-mo˧˥}}}}} \hypertarget{dzi\string_Mdzi\string_M-mo\string_M\string_T1}{}
\markboth{\textcolor{darkblue}{\textbf{\ipa{dzi˧dzi˧-mo˧˥}}}}{}
\textcolor{teal}{\mytextsc{nom}} \hspace{4pt} Ton~: MH\#.

Champignon comestible, qui ne pousse pas sur la terre, seulement sur les arbres tombés; est utilisé comme médicament pour les maux d'estomac.
\textcolor{brown}{\zh{一种可以吃的菌子,长在枯木上。}}




\paragraph{\hspace{-0.5cm} {\Large \textcolor{darkblue}{\textbf{\ipa{dzi˧ɖæ˧}}}}} \hypertarget{dzi\string_Md`\{\string_M1}{}
\markboth{\textcolor{darkblue}{\textbf{\ipa{dzi˧ɖæ˧}}}}{}
\textcolor{teal}{\mytextsc{nom}} \hspace{4pt} Ton~: M.

Emplacement, localisation (emploi typique: emplacement d'une maison).
\textcolor{brown}{\zh{位置、所在地。}}


 \mytextsc{clf}~: \textcolor{darkblue}{\textbf{\ipa{kʰwɤ˥}}} 

\paragraph{\hspace{-0.5cm} {\Large \textcolor{darkblue}{\textbf{\ipa{dzi˩}}} \textsubscript{1}}} \hypertarget{dzi\string_B1}{}
\markboth{\textcolor{darkblue}{\textbf{\ipa{dzi˩}}} \textsubscript{1}}{}
\textcolor{teal}{\mytextsc{verbe}} \hspace{4pt} Ton~: L.

Tomber, venir (la nuit tombe, la nuit vient).
\textcolor{brown}{\zh{来(晚上来了)。}}


\begin{exe}
\sn \textcolor{darkblue}{\textbf{\ipa{nɑ˩˥ | le˧-dzi˩-ze˩!}}}
\trans la nuit est tombée! Il fait noir!
\trans \textit{\textcolor{brown}{\zh{天黑了!}}}
\end{exe}
\begin{exe}
\sn \textcolor{darkblue}{\textbf{\ipa{nɑ˩˥ | le˧-dzi˩ | le˧-se˩-ze˩!}}}
\trans il fait tout à fait nuit!
\trans \textit{\textcolor{brown}{\zh{天完全黑了!}}}
\end{exe}


\paragraph{\hspace{-0.5cm} {\Large \textcolor{darkblue}{\textbf{\ipa{dzi˩}}} \textsubscript{2}}} \hypertarget{dzi\string_B2}{}
\markboth{\textcolor{darkblue}{\textbf{\ipa{dzi˩}}} \textsubscript{2}}{}
\textcolor{teal}{\mytextsc{classificateur}} \hspace{4pt} Ton~: L\textsubscript{c}.

Un costume entier: tous les vêtements qu'on porte.
\textcolor{brown}{\zh{量词:衣服(一套)。}}


\begin{exe}
\sn \textcolor{darkblue}{\textbf{\ipa{dʑi˧hṽ˥ | ɖɯ˧-dzi˩}}}
\trans un costume entier, un vêtement
\trans \textit{\textcolor{brown}{\zh{一套衣服}}}
\end{exe}
\begin{exe}
\sn \textcolor{darkblue}{\textbf{\ipa{dʑi˧hṽ˧ ɖɯ˧-dzi˩}}}
\trans un costume entier, un vêtement (même sens que l'exemple précédent; intégration en un seul groupe tonal)
\trans \textit{\textcolor{brown}{\zh{一套衣服(同上,但将短语合在一起,构成一个单一的声调短语)}}}
\end{exe}


\paragraph{\hspace{-0.5cm} {\Large \textcolor{darkblue}{\textbf{\ipa{dzi˩\textsubscript{a}}}} \textsubscript{1}}} \hypertarget{dzi\string_Ba1}{}
\markboth{\textcolor{darkblue}{\textbf{\ipa{dzi˩\textsubscript{a}}}} \textsubscript{1}}{}
\textcolor{teal}{\mytextsc{verbe}} \hspace{4pt} Ton~: L\textsubscript{a}.
\ding{202} 
S'asseoir, être assis.
\textcolor{brown}{\zh{坐。}}


\begin{exe}
\sn \textcolor{darkblue}{\textbf{\ipa{tʰi˧-dzi˩!}}}
\trans Asseyez-vous!
\trans \textit{\textcolor{brown}{\zh{坐下!}}}
\end{exe}
\begin{exe}
\sn \textcolor{darkblue}{\textbf{\ipa{hĩ˧bæ˧ ʈʂʰɯ˧-qo˧ dzi˩.}}}
\trans L'invité s'asseoit ici.
\trans \textit{\textcolor{brown}{\zh{客人是坐在这边的。}}}
\end{exe}
\begin{exe}
\sn \textcolor{darkblue}{\textbf{\ipa{(ʈʂʰɯ˧ | ) tʰi˧-dzi˩-kʰɯ˩-se˩.}}}
\trans Il est assis/installé, il a pris sa place.
\trans \textit{\textcolor{brown}{\zh{他坐下了。}}}
\end{exe}
\begin{exe}
\sn \textcolor{darkblue}{\textbf{\ipa{le˧-dzi˧\textasciitilde{}dzi˥}}}
\trans se tenir assis; s'emploie, par euphémisme, pour désigner la participation à une veillée funèbre
\trans \textit{\textcolor{brown}{\zh{坐一坐。来指:居丧、守灵(委婉语)}}}
\end{exe}
\ding{203} 
Demeurer, habiter.
\textcolor{brown}{\zh{住。}}


\begin{exe}
\sn \textcolor{darkblue}{\textbf{\ipa{dzi˩-bi˩-ni˩gv̩˩}}}
\trans s'habituer (à un environnement; à des habitudes de vie: nourriture, boisson...; à quelque chose; à quelqu'un)
\trans \textit{\textcolor{brown}{\zh{习惯(一个新的环境、一个地方的饮食……)}}}
\end{exe}
\begin{exe}
\sn \textcolor{darkblue}{\textbf{\ipa{dzi˩-bi˩-ni˩-mɤ˩-gv̩˩˥}}}
\trans je n'aime pas/je ne m'y fais pas/ça ne me plaît pas (ex.: quelqu'un ne veut pas rester quelque part, il s'y trouve mal et a la nostalgie d'ailleurs)
\trans \textit{\textcolor{brown}{\zh{不习惯}}}
\end{exe}
\begin{exe}
\sn \textcolor{darkblue}{\textbf{\ipa{njɤ˧ | ʈʂʰɯ˧-qo˧ | dzi˩-bi˩-ni˩-mɤ˩-gv̩˩˥}}}
\trans Je ne me fais pas à ici! / Je ne suis pas bien ici! (Paraphrase proposée par M23: ``Je ne veux pas rester ici!")
\trans \textit{\textcolor{brown}{\zh{我不习惯这里! / 我不喜欢这里! / 我不想待了!}}}
\end{exe}


\paragraph{\hspace{-0.5cm} {\Large \textcolor{darkblue}{\textbf{\ipa{dzi˩\textsubscript{a}}}} \textsubscript{2}}} \hypertarget{dzi\string_Ba2}{}
\markboth{\textcolor{darkblue}{\textbf{\ipa{dzi˩\textsubscript{a}}}} \textsubscript{2}}{}
\textcolor{teal}{\mytextsc{verbe}} \hspace{4pt} Ton~: L\textsubscript{a}.

Se rassembler (groupe de personnes).
\textcolor{brown}{\zh{聚集。}}


\begin{exe}
\sn \textcolor{darkblue}{\textbf{\ipa{ɖɯ˧-ʁwɤ˧ | le˧-dzi˧\textasciitilde{}dzi˥}}}
\trans tout le village se rassemble
\trans \textit{\textcolor{brown}{\zh{全村都聚集在一起}}}
\end{exe}
\begin{exe}
\sn \textcolor{darkblue}{\textbf{\ipa{hĩ˧ ɖɯ˧-v̩˧ | le˧-ʂɯ˧-ze˧! le˧-dzi˧\textasciitilde{}dzi˥ jo˩!}}}
\trans Quelqu'un est mort! Venez participer à la réunion/au rassemblement!
\trans \textit{\textcolor{brown}{\zh{一个人去世了!来参加丧礼吧!}}}
\end{exe}


\paragraph{\hspace{-0.5cm} {\Large \textcolor{darkblue}{\textbf{\ipa{dzi˩\textsubscript{a}}}} \textsubscript{3}}} \hypertarget{dzi\string_Ba3}{}
\markboth{\textcolor{darkblue}{\textbf{\ipa{dzi˩\textsubscript{a}}}} \textsubscript{3}}{}
\textcolor{teal}{\mytextsc{verbe}} \hspace{4pt} Ton~: L\textsubscript{a}.

Tomber, sombrer (ex.: un bateau qui sombre peu à peu dans le lac).
\textcolor{brown}{\zh{掉入、沉下去。}}


\begin{exe}
\sn \textcolor{darkblue}{\textbf{\ipa{mv̩˩tɕo˧ dzi˩}}}
\trans \mytextsc{directionnel} \string_ ; même sens: sombrer, couler
\trans \textit{\textcolor{brown}{\zh{往下掉入、沉下去}}}
\end{exe}


\paragraph{\hspace{-0.5cm} {\Large \textcolor{darkblue}{\textbf{\ipa{dzi˩\textsubscript{b}}}}}} \hypertarget{dzi\string_Bb1}{}
\markboth{\textcolor{darkblue}{\textbf{\ipa{dzi˩\textsubscript{b}}}}}{}
\textcolor{teal}{\mytextsc{classificateur}} \hspace{4pt} Ton~: L\textsubscript{b}.

Classificateur des arbres.
\textcolor{brown}{\zh{量词:树(一棵),竹子(一根)。}}


\begin{exe}
\sn \textcolor{darkblue}{\textbf{\ipa{si˧dzi˩ | ɖɯ˧-dzi˩}}}
\trans un arbre
\trans \textit{\textcolor{brown}{\zh{一棵树}}}
\end{exe}
\begin{exe}
\sn \textcolor{darkblue}{\textbf{\ipa{tʰv̩˧-dzi˧˥}}}
\trans cet arbre
\trans \textit{\textcolor{brown}{\zh{那棵树}}}
\end{exe}


\paragraph{\hspace{-0.5cm} {\Large \textcolor{darkblue}{\textbf{\ipa{dzi˩ʁo˩}}}}} \hypertarget{dzi\string_BRo\string_B1}{}
\markboth{\textcolor{darkblue}{\textbf{\ipa{dzi˩ʁo˩}}}}{}
\textcolor{teal}{\mytextsc{nom}} \hspace{4pt} Ton~: L.

Place, place assise.
\textcolor{brown}{\zh{座位。}}


 \mytextsc{clf}~: \textcolor{darkblue}{\textbf{\ipa{kʰwɤ˥}}} 

\paragraph{\hspace{-0.5cm} {\Large \textcolor{darkblue}{\textbf{\ipa{dzi˩tsʰɤ˩}}}}} \hypertarget{dzi\string_Bts\string_h7\string_B1}{}
\markboth{\textcolor{darkblue}{\textbf{\ipa{dzi˩tsʰɤ˩}}}}{}
\textcolor{teal}{\mytextsc{nom}} \hspace{4pt} Ton~: L.

Arbuste: sorte de houx, de grande taille.
\textcolor{brown}{\zh{永宁的一种灌木。}}


 \mytextsc{clf}~: \textcolor{darkblue}{\textbf{\ipa{dzi˩}}} 

\paragraph{\hspace{-0.5cm} {\Large \textcolor{darkblue}{\textbf{\ipa{dzi˧˥}}}}} \hypertarget{dzi\string_M\string_T1}{}
\markboth{\textcolor{darkblue}{\textbf{\ipa{dzi˧˥}}}}{}
\textcolor{teal}{\mytextsc{verbe}} \hspace{4pt} Ton~: MH.

Trembler.
\textcolor{brown}{\zh{颤抖、抖动。}}


\begin{exe}
\sn \textcolor{darkblue}{\textbf{\ipa{njɤ˩ dzi˧˥}}}
\trans la paupière tremble
\trans \textit{\textcolor{brown}{\zh{眼皮跳}}}
\end{exe}
\begin{exe}
\sn \textcolor{darkblue}{\textbf{\ipa{njɤ˩ dzi˧-ze˥}}}
\trans la paupière tremble
\trans \textit{\textcolor{brown}{\zh{眼皮跳}}}
\end{exe}
\begin{exe}
\sn \textcolor{darkblue}{\textbf{\ipa{njɤ˩ dzi˧˥ | ʐwæ˩˥}}}
\trans la paupière tremble très fort
\trans \textit{\textcolor{brown}{\zh{眼皮跳得很厉害}}}
\end{exe}
\begin{exe}
\sn \textcolor{darkblue}{\textbf{\ipa{njɤ˩˥ | le˧-dzi˧-ze˥}}}
\trans la paupière tremble
\trans \textit{\textcolor{brown}{\zh{眼皮跳}}}
\end{exe}
\begin{exe}
\sn \textcolor{darkblue}{\textbf{\ipa{njɤ˩˥ | mɤ˧-dzi˧˥}}}
\trans la paupière ne tremble pas
\trans \textit{\textcolor{brown}{\zh{眼皮不跳}}}
\end{exe}


\paragraph{\hspace{-0.5cm} {\Large \textcolor{darkblue}{\textbf{\ipa{dzo˥}}}}} \hypertarget{dzo\string_T1}{}
\markboth{\textcolor{darkblue}{\textbf{\ipa{dzo˥}}}}{}
\textcolor{teal}{\mytextsc{nom}} \hspace{4pt} Ton~: \#H.

Grêle.
\textcolor{brown}{\zh{冰雹。}}


\begin{exe}
\sn \textcolor{darkblue}{\textbf{\ipa{dzo˧ lɑ˩}}}
\trans il grêle
\trans \textit{\textcolor{brown}{\zh{下冰雹}}}
\end{exe}
\begin{exe}
\sn \textcolor{darkblue}{\textbf{\ipa{dzo˧ gi˧-ze˩}}}
\trans il tombe de la grêle
\trans \textit{\textcolor{brown}{\zh{下冰雹了}}}
\end{exe}


\paragraph{\hspace{-0.5cm} {\Large \textcolor{darkblue}{\textbf{\ipa{dzo˧-lv̩˧\textasciitilde{}lv̩˥}}}}} \hypertarget{dzo\string_M-lv\string_=\string_M~lv\string_=\string_T1}{}
\markboth{\textcolor{darkblue}{\textbf{\ipa{dzo˧-lv̩˧\textasciitilde{}lv̩˥}}}}{}
\textcolor{teal}{\mytextsc{nom}} \hspace{4pt} Ton~: H\#.

Grêlon.
\textcolor{brown}{\zh{冰雹。}}


 \mytextsc{clf}~: \textcolor{darkblue}{\textbf{\ipa{ɭɯ˧}}} 

\paragraph{\hspace{-0.5cm} {\Large \textcolor{darkblue}{\textbf{\ipa{dzo˧mi˧}}}}} \hypertarget{dzo\string_Mmi\string_M1}{}
\markboth{\textcolor{darkblue}{\textbf{\ipa{dzo˧mi˧}}}}{}
\textcolor{teal}{\mytextsc{nom}} \hspace{4pt} Ton~: M.

Grande cuve, grand tonneau (sens très proche du précédent, peuvent s'employer de façon interchangeable pour certains des tonneaux).
\textcolor{brown}{\zh{大桶。}}


 \mytextsc{clf}~: \textcolor{darkblue}{\textbf{\ipa{ɭɯ˧}}} 

\paragraph{\hspace{-0.5cm} {\Large \textcolor{darkblue}{\textbf{\ipa{dzo˧zo\#˥}}}}} \hypertarget{dzo\string_Mzo\#\string_T1}{}
\markboth{\textcolor{darkblue}{\textbf{\ipa{dzo˧zo\#˥}}}}{}
\textcolor{teal}{\mytextsc{nom}} \hspace{4pt} Ton~: \#H.

Petite cuve.
\textcolor{brown}{\zh{小桶。}}


 \mytextsc{clf}~: \textcolor{darkblue}{\textbf{\ipa{ɭɯ˧}}} 

\paragraph{\hspace{-0.5cm} {\Large \textcolor{darkblue}{\textbf{\ipa{dzo˩}}}}} \hypertarget{dzo\string_B1}{}
\markboth{\textcolor{darkblue}{\textbf{\ipa{dzo˩}}}}{}
\textcolor{teal}{\mytextsc{nom}} \hspace{4pt} Ton~: L.

Pont.
\textcolor{brown}{\zh{桥。}}


\begin{exe}
\sn \textcolor{darkblue}{\textbf{\ipa{dzo˧ | ɖɯ˧ pɤ˩}}}
\trans un pont
\trans \textit{\textcolor{brown}{\zh{一辆桥}}}
\end{exe}
\begin{exe}
\sn \textcolor{darkblue}{\textbf{\ipa{dzo˩ bæ˩˥}}}
\trans balayer le pont
\trans \textit{\textcolor{brown}{\zh{扫桥}}}
\end{exe}
\begin{exe}
\sn \textcolor{darkblue}{\textbf{\ipa{njɤ˧ | dzo˩ bæ˩-zo˩-ho˥.}}}
\trans Il va falloir que je balaie le pont.
\trans \textit{\textcolor{brown}{\zh{我要扫桥了。}}}
\end{exe}
\begin{exe}
\sn \textcolor{darkblue}{\textbf{\ipa{dzo˩ gv̩˩˥}}}
\trans construire (ou réparer) un pont
\trans \textit{\textcolor{brown}{\zh{建一辆桥}}}
\end{exe}
\begin{exe}
\sn \textcolor{darkblue}{\textbf{\ipa{njɤ˧ | dzo˩ gv̩˩-zo˩-ho˥.}}}
\trans Il va falloir que je construise (/répare) un pont.
\end{exe}
 \mytextsc{clf}~: \textcolor{darkblue}{\textbf{\ipa{pɤ˩}}} \textcolor{darkblue}{\textbf{\ipa{nɑ˧}}} 

\paragraph{\hspace{-0.5cm} {\Large \textcolor{darkblue}{\textbf{\ipa{dzo˩\textasciitilde{}dzo˧˥}}}}} \hypertarget{dzo\string_B~dzo\string_M\string_T1}{}
\markboth{\textcolor{darkblue}{\textbf{\ipa{dzo˩\textasciitilde{}dzo˧˥}}}}{}
\textcolor{teal}{\mytextsc{verbe}} \hspace{4pt} Ton~: MH.

Se moquer de, rire de, ridiculiser.
\textcolor{brown}{\zh{嘲笑、取笑。}}


\begin{exe}
\sn \textcolor{darkblue}{\textbf{\ipa{hĩ˧ dzo˧-dzo˥-ho˩-ni˩zo˩!}}}
\trans On dirait qu'(elle/il) va tourner quelqu'un en dérision
\trans \textit{\textcolor{brown}{\zh{他好像要开始取笑人家了!}}}
\end{exe}
\begin{exe}
\sn \textcolor{darkblue}{\textbf{\ipa{tʰɑ˧ dzo˩\textasciitilde{}dzo˩!}}}
\trans Ne vous moquez pas!/ Pas de sarcasmes!
\trans \textit{\textcolor{brown}{\zh{别嘲笑(人家)!}}}
\end{exe}


\paragraph{\hspace{-0.5cm} {\Large \textcolor{darkblue}{\textbf{\ipa{dzo˩mi\#˥}}}}} \hypertarget{dzo\string_Bmi\#\string_T1}{}
\markboth{\textcolor{darkblue}{\textbf{\ipa{dzo˩mi\#˥}}}}{}
\textcolor{teal}{\mytextsc{nom}} \hspace{4pt} Ton~: LM+\#H.

Lézard femelle.
\textcolor{brown}{\zh{母壁虎。}}


\begin{exe}
\sn \textcolor{darkblue}{\textbf{\ipa{dzo˧mi˧-dzo˩pʰv̩˩}}}
\trans lézard femelle et lézard mâle
\trans \textit{\textcolor{brown}{\zh{母壁虎与公壁虎}}}
\end{exe}
 \mytextsc{clf}~: \textcolor{darkblue}{\textbf{\ipa{mi˩}}} 

\paragraph{\hspace{-0.5cm} {\Large \textcolor{darkblue}{\textbf{\ipa{dzo˩pʰv̩˩}}}}} \hypertarget{dzo\string_Bp\string_hv\string_=\string_B1}{}
\markboth{\textcolor{darkblue}{\textbf{\ipa{dzo˩pʰv̩˩}}}}{}
\textcolor{teal}{\mytextsc{nom}} \hspace{4pt} Ton~: L.

Lézard mâle.
\textcolor{brown}{\zh{公壁虎。}}


\begin{exe}
\sn \textcolor{darkblue}{\textbf{\ipa{dzo˩pʰv̩˩-dzo˩mi˥}}}
\trans lézard mâle et lézard femelle
\trans \textit{\textcolor{brown}{\zh{公壁虎与母壁虎}}}
\end{exe}
 \mytextsc{clf}~: \textcolor{darkblue}{\textbf{\ipa{mi˩}}} \textcolor{darkblue}{\textbf{\ipa{ɭɯ˧}}} 

\paragraph{\hspace{-0.5cm} {\Large \textcolor{darkblue}{\textbf{\ipa{dzo˩zo\#˥}}}}} \hypertarget{dzo\string_Bzo\#\string_T1}{}
\markboth{\textcolor{darkblue}{\textbf{\ipa{dzo˩zo\#˥}}}}{}
\textcolor{teal}{\mytextsc{nom}} \hspace{4pt} Ton~: LM+\#H.

Bébé lézard.
\textcolor{brown}{\zh{小壁虎。}}


 \mytextsc{clf}~: \textcolor{darkblue}{\textbf{\ipa{mi˩}}} \textcolor{darkblue}{\textbf{\ipa{ɭɯ˧}}} 

\paragraph{\hspace{-0.5cm} {\Large \textcolor{darkblue}{\textbf{\ipa{dzo˩˧}}}}} \hypertarget{dzo\string_B\string_M1}{}
\markboth{\textcolor{darkblue}{\textbf{\ipa{dzo˩˧}}}}{}
\textcolor{teal}{\mytextsc{nom}} \hspace{4pt} Ton~: LM.

Lézard.
\textcolor{brown}{\zh{壁虎,蜥蜴,四脚蛇。}}


\begin{exe}
\sn \textcolor{darkblue}{\textbf{\ipa{dzo˩ hwæ˧-ze˧}}}
\trans ...a acheté (un) lézard
\trans \textit{\textcolor{brown}{\zh{买了壁虎}}}
\end{exe}
\begin{exe}
\sn \textcolor{darkblue}{\textbf{\ipa{dzo˩ dzɯ˧-ze˩}}}
\trans ...a mangé (un) lézard
\trans \textit{\textcolor{brown}{\zh{吃了壁虎}}}
\end{exe}
 \mytextsc{clf}~: \textcolor{darkblue}{\textbf{\ipa{mi˩}}} 

\paragraph{\hspace{-0.5cm} {\Large \textcolor{darkblue}{\textbf{\ipa{dzɯ˥}}}}} \hypertarget{dzM\string_T1}{}
\markboth{\textcolor{darkblue}{\textbf{\ipa{dzɯ˥}}}}{}
\textcolor{teal}{\mytextsc{verbe}} \hspace{4pt} Ton~: H.

Manger.
\textcolor{brown}{\zh{吃。}}


\begin{exe}
\sn \textcolor{darkblue}{\textbf{\ipa{le˧-dzɯ˥}}}
\trans \mytextsc{accomp}
\trans \textit{\textcolor{brown}{\zh{\mytextsc{实施}}}}
\end{exe}
\begin{exe}
\sn \textcolor{darkblue}{\textbf{\ipa{hɑ˧ dzɯ˧}}}
\trans prendre un repas, manger de la nourriture
\trans \textit{\textcolor{brown}{\zh{吃饭}}}
\end{exe}
\begin{exe}
\sn \textcolor{darkblue}{\textbf{\ipa{njɤ˧ | hɑ˧ le˧-dzɯ˥-ze˩}}}
\trans J'ai mangé. J'ai pris mon repas.
\trans \textit{\textcolor{brown}{\zh{我吃过饭了。}}}
\end{exe}
\begin{exe}
\sn \textcolor{darkblue}{\textbf{\ipa{dzɯ˧-di˧˥}}}
\trans nourriture, chose à manger
\trans \textit{\textcolor{brown}{\zh{吃的(东西)}}}
\end{exe}
\begin{exe}
\sn \textcolor{darkblue}{\textbf{\ipa{dzɯ˧-bi˩-ze˩!}}}
\trans (On) va manger! / C'est l'heure de manger!
\trans \textit{\textcolor{brown}{\zh{要吃饭了!}}}
\end{exe}


\paragraph{\hspace{-0.5cm} {\Large \textcolor{darkblue}{\textbf{\ipa{dzɯ˧tsɯ˧˥}}}}} \hypertarget{dzM\string_MtsM\string_M\string_T1}{}
\markboth{\textcolor{darkblue}{\textbf{\ipa{dzɯ˧tsɯ˧˥}}}}{}
\textcolor{teal}{\mytextsc{nom}} \hspace{4pt} Ton~: MH\#.

Sorte de houx, petites feuilles à piquants.
\textcolor{brown}{\zh{永宁的一种灌木。}}


 \mytextsc{clf}~: \textcolor{darkblue}{\textbf{\ipa{pʰæ˧˥}}} 

\newpage
\section*{\centering- \textcolor{darkblue}{\textbf{\ipa{dʑ}}} -}
\paragraph{\hspace{-0.5cm} {\Large \textcolor{darkblue}{\textbf{\ipa{dʑɤ˧bo˩}}}}} \hypertarget{dz£7\string_Mbo\string_B1}{}
\markboth{\textcolor{darkblue}{\textbf{\ipa{dʑɤ˧bo˩}}}}{}
\textcolor{teal}{\mytextsc{nom}} \hspace{4pt} Ton~: L\#.

Grenier à céréale; selon M23, n'existe plus: était, dans le temps, un bâtiment exclusivement consacré à la conservation des céréales: un grenier.
\textcolor{brown}{\zh{粮仓。}}


 \mytextsc{clf}~: \textcolor{darkblue}{\textbf{\ipa{ɭɯ˧}}} 

\paragraph{\hspace{-0.5cm} {\Large \textcolor{darkblue}{\textbf{\ipa{dʑɤ˧do˩}}}}} \hypertarget{dz£7\string_Mdo\string_B1}{}
\markboth{\textcolor{darkblue}{\textbf{\ipa{dʑɤ˧do˩}}}}{}
\textcolor{teal}{\mytextsc{nom}} \hspace{4pt} Ton~: L\#.

Zhongdian (nom de lieu).
\textcolor{brown}{\zh{中甸。}}


\begin{exe}
\sn \textcolor{darkblue}{\textbf{\ipa{dʑɤ˧do˩-bɤ˩}}}
\trans les Pumi de Zhongdian
\trans \textit{\textcolor{brown}{\zh{中甸普米族}}}
\end{exe}


\paragraph{\hspace{-0.5cm} {\Large \textcolor{darkblue}{\textbf{\ipa{dʑɤ˩\textsubscript{b}}}}}} \hypertarget{dz£7\string_Bb1}{}
\markboth{\textcolor{darkblue}{\textbf{\ipa{dʑɤ˩\textsubscript{b}}}}}{}
\textcolor{teal}{\mytextsc{adjectif}} \hspace{4pt} Ton~: L\textsubscript{b}.

Bon (bonne décision).
\textcolor{brown}{\zh{好。}}


\begin{exe}
\sn \textcolor{darkblue}{\textbf{\ipa{mɤ˧-dʑɤ˩}}}
\trans pas bien, mauvais
\trans \textit{\textcolor{brown}{\zh{坏}}}
\end{exe}
\begin{exe}
\sn \textcolor{darkblue}{\textbf{\ipa{dʑɤ˩-hĩ˥}}}
\trans \mytextsc{rel}
\trans \textit{\textcolor{brown}{\zh{\mytextsc{rel}}}}
\end{exe}
\begin{exe}
\sn \textcolor{darkblue}{\textbf{\ipa{(no˧) ɖwæ˧˥ | dʑɤ˩˥!}}}
\trans Tu es quelqu'un de bien!
\trans \textit{\textcolor{brown}{\zh{你很好!}}}
\end{exe}
\begin{exe}
\sn \textcolor{darkblue}{\textbf{\ipa{dʑɤ˩-kʰɯ˥!}}}
\trans bénédiction dite au Nouvel An: ``Bonnes (choses)!" =``Prospérité!" =``Bonne année!"
\trans \textit{\textcolor{brown}{\zh{新年祝福:“祝一切好! / 万事如意!”}}}
\end{exe}
\begin{exe}
\sn \textcolor{darkblue}{\textbf{\ipa{no˧ | le˧-ʝi˥ | dʑɤ˩˥, | hĩ˧-ɳɯ˩ | do˩˥! | ʈʂʰɯ˧ | le˧-ʝi˥ | mɤ˧-dʑɤ˩, | hĩ˧-ɳɯ˩ | ʐwɤ˩˥!}}}
\trans ``Si tu travailles bien, les gens s'en rendent compte! S'il travail mal, les gens le disent!" = ``Le bon travail est reconnu; le mauvais travail reçoit des critiques!" (Contexte: commentaire au sujet d'un mauvais livre. C'est ce qu'on disait autrefois: la belle ouvrage est reconnue; le mauvais travail vous attire des critiques.)
\trans \textit{\textcolor{brown}{\zh{你做得好,人家(会)发现!他做的不好,人家(会)说(他)!}}}
\end{exe}


\paragraph{\hspace{-0.5cm} {\Large \textcolor{darkblue}{\textbf{\ipa{dʑɤ˩bv̩˥}}}}} \hypertarget{dz£7\string_Bbv\string_=\string_T1}{}
\markboth{\textcolor{darkblue}{\textbf{\ipa{dʑɤ˩bv̩˥}}}}{}
\textcolor{teal}{\mytextsc{verbe}} \hspace{4pt} Ton~: LH.

Jouer.
\textcolor{brown}{\zh{玩,玩耍。}}


\begin{exe}
\sn \textcolor{darkblue}{\textbf{\ipa{dʑɤ˩bv̩˥ -bi˩/-ze˩}}}
\trans \string_ \mytextsc{fut}.imm/\mytextsc{pfv}
\trans \textit{\textcolor{brown}{\zh{要玩耍 / 玩耍了}}}
\end{exe}
\begin{exe}
\sn \textcolor{darkblue}{\textbf{\ipa{le˧-dʑɤ˩bv̩˩ +ze˩}}}
\trans \mytextsc{accomp} \string_ \mytextsc{pfv}
\trans \textit{\textcolor{brown}{\zh{玩耍了}}}
\end{exe}
\begin{exe}
\sn \textcolor{darkblue}{\textbf{\ipa{tʰi˧-dʑɤ˩bv̩˩, | tʰi˧-dʑɤ˩bv̩˩, | le˧-fv̩˧!}}}
\trans ils jouent, ils jouent… ils sont contents! (Au sujet d'enfants qui jouent ensemble)
\trans \textit{\textcolor{brown}{\zh{他们玩着玩着,很高兴!(情景:几个小孩子一起玩)}}}
\end{exe}


\paragraph{\hspace{-0.5cm} {\Large \textcolor{darkblue}{\textbf{\ipa{dʑɤ˩bv̩˥-di˩}}}}} \hypertarget{dz£7\string_Bbv\string_=\string_T-di\string_B1}{}
\markboth{\textcolor{darkblue}{\textbf{\ipa{dʑɤ˩bv̩˥-di˩}}}}{}
\textcolor{teal}{\mytextsc{nom}} \hspace{4pt} Ton~: LH-.

Jouet.
\textcolor{brown}{\zh{玩具。}}




\paragraph{\hspace{-0.5cm} {\Large \textcolor{darkblue}{\textbf{\ipa{dʑɤ˩bv̩˧kɤ˧-sɑ˥ʁwɤ˩}}}}} \hypertarget{dz£7\string_Bbv\string_=\string_Mk7\string_M-sA\string_TRw7\string_B1}{}
\markboth{\textcolor{darkblue}{\textbf{\ipa{dʑɤ˩bv̩˧kɤ˧-sɑ˥ʁwɤ˩}}}}{}
\textcolor{teal}{\mytextsc{nom}} \hspace{4pt} Ton~: LM+MH\#-.

Gaoming, un village au nord-est de Yongning.
\textcolor{brown}{\zh{高明 (摩梭话名称的汉译:嘎撒瓦)(永宁的一个村落)。}}


\begin{exe}
\sn \textcolor{darkblue}{\textbf{\ipa{dʑɤ˩bv̩˧kɤ˧-sɑ˥ʁwɤ˩, | hi˩ʁwɤ˩-lo˥, | æ˩mi˧-ʁwɤ\#˥, | lɑ˧lo˧-ʁwɤ˥, | lɑ˧ŋwɤ˧, | bɤ˧tsʰo˧gv̩˥, | ə˧lɑ˧-ʁwɤ\#˥, | gæ˧ɻæ˩, | qʰæ˧tɕʰi˧, | tʰo˧ʈɯ\#˥}}}
\trans les dix villages comptant traditionnellement comme faisant partie de Yongning
\trans \textit{\textcolor{brown}{\zh{摩梭传统地理概念中,属于永宁的十个村落}}}
\end{exe}


\paragraph{\hspace{-0.5cm} {\Large \textcolor{darkblue}{\textbf{\ipa{dʑɤ˩bv̩˥-ʁwɤ˩}}}}} \hypertarget{dz£7\string_Bbv\string_=\string_T-Rw7\string_B1}{}
\markboth{\textcolor{darkblue}{\textbf{\ipa{dʑɤ˩bv̩˥-ʁwɤ˩}}}}{}
\textcolor{teal}{\mytextsc{nom}} \hspace{4pt} Ton~: LH-.

Jiabowa (nom de village).
\textcolor{brown}{\zh{甲波瓦(永宁的一个村落)。}}




\paragraph{\hspace{-0.5cm} {\Large \textcolor{darkblue}{\textbf{\ipa{dʑɤ˩ɕjɤ˩}}}}} \hypertarget{dz£7\string_Bs£j7\string_B1}{}
\markboth{\textcolor{darkblue}{\textbf{\ipa{dʑɤ˩ɕjɤ˩}}}}{}
\textcolor{teal}{\mytextsc{nom}} \hspace{4pt} Ton~: L.

Semelle (en paille); chausson (en paille).
\textcolor{brown}{\zh{鞋垫。}}




\paragraph{\hspace{-0.5cm} {\Large \textcolor{darkblue}{\textbf{\ipa{dʑɤ˩kʰwɤ˧}}}}} \hypertarget{dz£7\string_Bk\string_hw7\string_M1}{}
\markboth{\textcolor{darkblue}{\textbf{\ipa{dʑɤ˩kʰwɤ˧}}}}{}
\textcolor{teal}{\mytextsc{nom}} \hspace{4pt} Ton~: LM.

Distance.
\textcolor{brown}{\zh{距离。}}


\begin{exe}
\sn \textcolor{darkblue}{\textbf{\ipa{no˧ | ʈʂʰɯ˧ | ə˩-ʐæ˥ʂæ˩? | dʑɤ˩kʰwɤ˧ ə˩-di˩? | - dʑɤ˩˥ | dʑɤ˩kʰwɤ˧ mɤ˧-di˥! | mɤ˧-ʐæ˩ʂæ˩!}}}
\trans tu es loin de lui? Y a-t-il de la distance entre vous? (=vous êtes proches/intimes, ou pas?) - Non, il n'y a guère de distance! Nous ne somme pas éloignés!
\trans \textit{\textcolor{brown}{\zh{你们很熟吗?/ 你们很亲吗? - 不很熟!/ 不很亲!}}}
\end{exe}


\paragraph{\hspace{-0.5cm} {\Large \textcolor{darkblue}{\textbf{\ipa{dʑɤ˩pi\#˥}}}}} \hypertarget{dz£7\string_Bpi\#\string_T1}{}
\markboth{\textcolor{darkblue}{\textbf{\ipa{dʑɤ˩pi\#˥}}}}{}
\textcolor{teal}{\mytextsc{adjectif}} \hspace{4pt} Ton~: LM+\#H.

Beaucoup.
\textcolor{brown}{\zh{多。}}


\begin{exe}
\sn \textcolor{darkblue}{\textbf{\ipa{dʑɤ˩pi˧ ʝi˧}}}
\trans C'est très utile
\trans \textit{\textcolor{brown}{\zh{很有用、有很多用处}}}
\end{exe}
\begin{exe}
\sn \textcolor{darkblue}{\textbf{\ipa{dʑɤ˩pi˧ dʑo˧!}}}
\trans J'en ai beaucoup! (possession)
\trans \textit{\textcolor{brown}{\zh{我有很多!}}}
\end{exe}
\begin{exe}
\sn \textcolor{darkblue}{\textbf{\ipa{dʑɤ˩pi˧ dʑo˧˥}}}
\trans Il en existe beaucoup, il y en a beaucoup. (Note: existentiel, et non possession)
\trans \textit{\textcolor{brown}{\zh{有很多。}}}
\end{exe}


\paragraph{\hspace{-0.5cm} {\Large \textcolor{darkblue}{\textbf{\ipa{dʑɤ˩qʰɑ˥}}}}} \hypertarget{dz£7\string_Bq\string_hA\string_T1}{}
\markboth{\textcolor{darkblue}{\textbf{\ipa{dʑɤ˩qʰɑ˥}}}}{}
\textcolor{teal}{\mytextsc{adverbe}} \hspace{4pt} Ton~: LH.

Continuellement, par un effort soutenu.
\textcolor{brown}{\zh{一直、一个劲地。}}


\begin{exe}
\sn \textcolor{darkblue}{\textbf{\ipa{dʑɤ˩qʰɑ˥ ʈɤ˩}}}
\trans tirer en un effort continu
\trans \textit{\textcolor{brown}{\zh{一个劲地拉}}}
\end{exe}
\begin{exe}
\sn \textcolor{darkblue}{\textbf{\ipa{dʑɤ˩qʰɑ˥ mi˩}}}
\trans appuyer en un effort continu
\trans \textit{\textcolor{brown}{\zh{一个劲地推}}}
\end{exe}
\begin{exe}
\sn \textcolor{darkblue}{\textbf{\ipa{dʑɤ˩qʰɑ˥ lɑ˩}}}
\trans frapper continuellement, en un effort continu
\trans \textit{\textcolor{brown}{\zh{一个劲地打}}}
\end{exe}


\paragraph{\hspace{-0.5cm} {\Large \textcolor{darkblue}{\textbf{\ipa{dʑɤ˩so˧}}}}} \hypertarget{dz£7\string_Bso\string_M1}{}
\markboth{\textcolor{darkblue}{\textbf{\ipa{dʑɤ˩so˧}}}}{}
\textcolor{teal}{\mytextsc{adjectif}} \hspace{4pt} Ton~: LM.

En abondance, beaucoup.
\textcolor{brown}{\zh{好几(个)。}}


\begin{exe}
\sn \textcolor{darkblue}{\textbf{\ipa{dʑɤ˩-so˧ ɲi˧}}}
\trans beaucoup de jours, longtemps
\trans \textit{\textcolor{brown}{\zh{好几天}}}
\end{exe}


\paragraph{\hspace{-0.5cm} {\Large \textcolor{darkblue}{\textbf{\ipa{dʑɤ˩tsʰi\#˥}}}}} \hypertarget{dz£7\string_Bts\string_hi\#\string_T1}{}
\markboth{\textcolor{darkblue}{\textbf{\ipa{dʑɤ˩tsʰi\#˥}}}}{}
\textcolor{teal}{\mytextsc{nom}} \hspace{4pt} Ton~: LM+\#H.

Prénom masculin.
\textcolor{brown}{\zh{男性名字。}}




\paragraph{\hspace{-0.5cm} {\Large \textcolor{darkblue}{\textbf{\ipa{dʑɤ˩tsʰi˧-ɖɯ˩mɑ˩}}}}} \hypertarget{dz£7\string_Bts\string_hi\string_M-d`M\string_BmA\string_B1}{}
\markboth{\textcolor{darkblue}{\textbf{\ipa{dʑɤ˩tsʰi˧-ɖɯ˩mɑ˩}}}}{}
\textcolor{teal}{\mytextsc{nom}} \hspace{4pt} Ton~: LM-L.

Prénom féminin.
\textcolor{brown}{\zh{女性名字。}}




\paragraph{\hspace{-0.5cm} {\Large \textcolor{darkblue}{\textbf{\ipa{dʑɤ˩tsʰi˧-tsi˩mv̩˩}}}}} \hypertarget{dz£7\string_Bts\string_hi\string_M-tsi\string_Bmv\string_=\string_B1}{}
\markboth{\textcolor{darkblue}{\textbf{\ipa{dʑɤ˩tsʰi˧-tsi˩mv̩˩}}}}{}
\textcolor{teal}{\mytextsc{nom}} \hspace{4pt} Ton~: LM-L.

Drapeau de prière; on en attache sur un arbre proche de la maison, ou sur le sommet de la maison.
\textcolor{brown}{\zh{经幡、风马旗(挂在家旁边的树上,或房顶上)。}}


 \mytextsc{clf}~: \textcolor{darkblue}{\textbf{\ipa{dzi˩}}} 

\paragraph{\hspace{-0.5cm} {\Large \textcolor{darkblue}{\textbf{\ipa{dʑɤ˩tsʰo˧}}}}} \hypertarget{dz£7\string_Bts\string_ho\string_M1}{}
\markboth{\textcolor{darkblue}{\textbf{\ipa{dʑɤ˩tsʰo˧}}}}{}
\textcolor{teal}{\mytextsc{verbe}} \hspace{4pt} Ton~: LM.

Danser.
\textcolor{brown}{\zh{跳舞。}}


\begin{exe}
\sn \textcolor{darkblue}{\textbf{\ipa{ʈʂʰɯ˧ | dʑɤ˩tsʰo˧ mɤ˧-dʑɤ˩!}}}
\trans il/elle ne danse pas bien!
\trans \textit{\textcolor{brown}{\zh{他舞跳得不好!}}}
\end{exe}
\begin{exe}
\sn \textcolor{darkblue}{\textbf{\ipa{dʑɤ˩tsʰo˧ | ɖɯ˧-hɑ̃˧ tsʰo˧}}}
\trans danser (toute) une soirée
\trans \textit{\textcolor{brown}{\zh{跳一整夜舞}}}
\end{exe}
\begin{exe}
\sn \textcolor{darkblue}{\textbf{\ipa{ʑi˧qʰwɤ˧-ʂɯ˧-qo˧ | dʑɤ˩tsʰo˧ ʁɑ˧ ʂe˩}}}
\trans organiser une pendaison de crémaillère dans une nouvelle maison
\trans \textit{\textcolor{brown}{\zh{在新房子举办乔迁宴会}}}
\end{exe}


\paragraph{\hspace{-0.5cm} {\Large \textcolor{darkblue}{\textbf{\ipa{dʑi˥}}} \textsubscript{1}}} \hypertarget{dz£i\string_T1}{}
\markboth{\textcolor{darkblue}{\textbf{\ipa{dʑi˥}}} \textsubscript{1}}{}
\textcolor{teal}{\mytextsc{nom}} \hspace{4pt} Ton~: \#H.

Urine.
\textcolor{brown}{\zh{尿。}}


\begin{exe}
\sn \textcolor{darkblue}{\textbf{\ipa{dʑi˧ bæ˥}}}
\trans balayer l'urine
\trans \textit{\textcolor{brown}{\zh{扫尿}}}
\end{exe}
\begin{exe}
\sn \textcolor{darkblue}{\textbf{\ipa{dʑi˧-lɑ˩ | qʰæ˧}}}
\trans excréments: urine et fèces
\trans \textit{\textcolor{brown}{\zh{大小便的统称}}}
\end{exe}


\paragraph{\hspace{-0.5cm} {\Large \textcolor{darkblue}{\textbf{\ipa{dʑi˥}}} \textsubscript{2}}} \hypertarget{dz£i\string_T2}{}
\markboth{\textcolor{darkblue}{\textbf{\ipa{dʑi˥}}} \textsubscript{2}}{}
\textcolor{teal}{\mytextsc{nom}} \hspace{4pt} Ton~: H.

Habit, vêtement (mot monosyllabique).
\textcolor{brown}{\zh{衣服。}}


\begin{exe}
\sn \textcolor{darkblue}{\textbf{\ipa{kʰv̩˧ʂɯ˥, | dʑi˧ qæ˧!}}}
\trans Au Nouvel An, on change de vêtements/ on porte des vêtements neufs!
\trans \textit{\textcolor{brown}{\zh{过年,换衣服! / 过年,要穿新衣服!}}}
\end{exe}
\begin{exe}
\sn \textcolor{darkblue}{\textbf{\ipa{dʑi˧ qæ˧-ze˩!}}}
\trans (il/elle) a changé de vêtements!
\trans \textit{\textcolor{brown}{\zh{换衣服了!}}}
\end{exe}
\begin{exe}
\sn \textcolor{darkblue}{\textbf{\ipa{nɑ˩-dʑi\#˥}}}
\trans le vêtement des Na
\trans \textit{\textcolor{brown}{\zh{摩梭服装}}}
\end{exe}
\begin{exe}
\sn \textcolor{darkblue}{\textbf{\ipa{hæ˧-dʑi\#˥}}}
\trans le vêtement des Chinois (Han)
\trans \textit{\textcolor{brown}{\zh{汉族服装}}}
\end{exe}
 \mytextsc{clf}~: \textcolor{darkblue}{\textbf{\ipa{ɭɯ˧}}} 

\paragraph{\hspace{-0.5cm} {\Large \textcolor{darkblue}{\textbf{\ipa{dʑi˧hṽ˥\$}}}}} \hypertarget{dz£i\string_Mhv\string_~\string_T\$1}{}
\markboth{\textcolor{darkblue}{\textbf{\ipa{dʑi˧hṽ˥\$}}}}{}
\textcolor{teal}{\mytextsc{nom}} \hspace{4pt} Ton~: H\$.

Habit, vêtement (terme générique).
\textcolor{brown}{\zh{衣服。}}


 \mytextsc{clf}~: \textcolor{darkblue}{\textbf{\ipa{ɭɯ˧}}} 

\paragraph{\hspace{-0.5cm} {\Large \textcolor{darkblue}{\textbf{\ipa{dʑi˧mi\#˥}}}}} \hypertarget{dz£i\string_Mmi\#\string_T1}{}
\markboth{\textcolor{darkblue}{\textbf{\ipa{dʑi˧mi\#˥}}}}{}
\textcolor{teal}{\mytextsc{nom}} \hspace{4pt} Ton~: \#H.

Buffle (femelle).
\textcolor{brown}{\zh{母水牛。}}


\begin{exe}
\sn \textcolor{darkblue}{\textbf{\ipa{dʑi˧mi˧ tʰv̩˧-pʰo˩}}}
\trans \mytextsc{n}+\mytextsc{dem}+\mytextsc{clf}
\trans \textit{\textcolor{brown}{\zh{这头母水牛}}}
\end{exe}
\begin{exe}
\sn \textcolor{darkblue}{\textbf{\ipa{dʑi˧mi˧-dʑi˧zo\#˥ / dʑi˧mi˧-dʑi˥zo˩}}}
\trans buffles femelle et mâle
\trans \textit{\textcolor{brown}{\zh{母水牛与公水牛}}}
\end{exe}
 \mytextsc{clf}~: \textcolor{darkblue}{\textbf{\ipa{pʰo˧˥}}} 

\paragraph{\hspace{-0.5cm} {\Large \textcolor{darkblue}{\textbf{\ipa{dʑi˧zo\#˥}}}}} \hypertarget{dz£i\string_Mzo\#\string_T1}{}
\markboth{\textcolor{darkblue}{\textbf{\ipa{dʑi˧zo\#˥}}}}{}
\textcolor{teal}{\mytextsc{nom}} \hspace{4pt} Ton~: \#H.

Buffle (enfant mâle).
\textcolor{brown}{\zh{小水牛(水牛崽子),一般指公的小水牛。}}


\begin{exe}
\sn \textcolor{darkblue}{\textbf{\ipa{dʑi˧zo˧ tʰv̩˧-ɭɯ\#˥}}}
\trans \mytextsc{n}+\mytextsc{dem}+\mytextsc{clf}
\trans \textit{\textcolor{brown}{\zh{这只水牛崽子}}}
\end{exe}
 \mytextsc{clf}~: \textcolor{darkblue}{\textbf{\ipa{ɭɯ˧}}} 

\paragraph{\hspace{-0.5cm} {\Large \textcolor{darkblue}{\textbf{\ipa{dʑi˩wɤ˩}}}}} \hypertarget{dz£i\string_Bw7\string_B1}{}
\markboth{\textcolor{darkblue}{\textbf{\ipa{dʑi˩wɤ˩}}}}{}
\textcolor{teal}{\mytextsc{nom}} \hspace{4pt} Ton~: L.

Étriers.
\textcolor{brown}{\zh{马镫。}}


 \mytextsc{clf}~: \textcolor{darkblue}{\textbf{\ipa{dze˩}}} 

\paragraph{\hspace{-0.5cm} {\Large \textcolor{darkblue}{\textbf{\ipa{‑dʑo˥}}}}} \hypertarget{‑dz£o\string_T1}{}
\markboth{\textcolor{darkblue}{\textbf{\ipa{‑dʑo˥}}}}{}
\textcolor{teal}{\mytextsc{suffixe}} \hspace{4pt} Ton~: H.

Marqueur de topic.
\textcolor{brown}{\zh{\mytextsc{主题。}}}




\paragraph{\hspace{-0.5cm} {\Large \textcolor{darkblue}{\textbf{\ipa{‑dʑo˧}}}}} \hypertarget{‑dz£o\string_M1}{}
\markboth{\textcolor{darkblue}{\textbf{\ipa{‑dʑo˧}}}}{}
\textcolor{teal}{\mytextsc{suffixe}} \hspace{4pt} Ton~: M.

Aspect progressif.
\textcolor{brown}{\zh{\mytextsc{进行式。}}}




\paragraph{\hspace{-0.5cm} {\Large \textcolor{darkblue}{\textbf{\ipa{dʑo˧\textsubscript{b}}}}}} \hypertarget{dz£o\string_Mb1}{}
\markboth{\textcolor{darkblue}{\textbf{\ipa{dʑo˧\textsubscript{b}}}}}{}
\textcolor{teal}{\mytextsc{verbe}} \hspace{4pt} Ton~: M\textsubscript{b}.

Posséder; y avoir; avoir un objet, avoir une pensée…; autrefois, il y avait une mère et sa fille...
\textcolor{brown}{\zh{有,拥有。}}


\begin{exe}
\sn \textcolor{darkblue}{\textbf{\ipa{mɤ˧-dʑo˧-ze˧!}}}
\trans Il n'y en a plus!
\trans \textit{\textcolor{brown}{\zh{没有了!}}}
\end{exe}
\begin{exe}
\sn \textcolor{darkblue}{\textbf{\ipa{le˧-dʑo˧-ze˧!}}}
\trans ça y est, il y en a!
\trans \textit{\textcolor{brown}{\zh{有了!}}}
\end{exe}
\begin{exe}
\sn \textcolor{darkblue}{\textbf{\ipa{ʈʂʰɯ˧ | ɑ˩ʁo˧ | ɖɯ˧-sɑ˥ | mɤ˧-dʑo˧!}}}
\trans Il n'y a rien chez lui = sa maison est dans l'indigence
\trans \textit{\textcolor{brown}{\zh{他家什么也没有 = 他家贫穷}}}
\end{exe}
\begin{exe}
\sn \textcolor{darkblue}{\textbf{\ipa{njɤ˧ | mv̩˩zɯ˩-ni˥mi˩ | ŋi˧-kv̩˧ dʑo˧˥!}}}
\trans j’ai deux frères et sœurs!
\trans \textit{\textcolor{brown}{\zh{我有两个兄弟姐妹!}}}
\end{exe}
\begin{exe}
\sn \textcolor{darkblue}{\textbf{\ipa{dʑo˧-tʰɑ˧˥!}}}
\trans Cela arrive / il peut y en avoir / c'est possible qu'il y en ait!
\trans \textit{\textcolor{brown}{\zh{会有的!}}}
\end{exe}
\begin{exe}
\sn \textcolor{darkblue}{\textbf{\ipa{tso˧\textasciitilde{}tso˧ dʑo˧}}}
\trans il y a des choses
\trans \textit{\textcolor{brown}{\zh{他有东西}}}
\end{exe}
\begin{exe}
\sn \textcolor{darkblue}{\textbf{\ipa{njɤ˧-ɻ̍˩, | ɖɯ˧-ɭɯ˧-lɑ˧ dʑo˥!}}}
\trans Nous, on n'en a qu'un(, d'enfant)!
\trans \textit{\textcolor{brown}{\zh{我们只有一个(孩子)!}}}
\end{exe}
\begin{exe}
\sn \textcolor{darkblue}{\textbf{\ipa{ɖwæ˧˥ | dʑo˧-ɲi˥!}}}
\trans Il y en a des quantités! (ex.: au sujet du bois de construction qu'on prépare en vue de la construction d'une maison)
\trans \textit{\textcolor{brown}{\zh{有很多!(如:准备建房,积累的木材有很多)}}}
\end{exe}


\paragraph{\hspace{-0.5cm} {\Large \textcolor{darkblue}{\textbf{\ipa{dʑo˩\textsubscript{b}}}}}} \hypertarget{dz£o\string_Bb1}{}
\markboth{\textcolor{darkblue}{\textbf{\ipa{dʑo˩\textsubscript{b}}}}}{}
\textcolor{teal}{\mytextsc{verbe}} \hspace{4pt} Ton~: L\textsubscript{b}.

Existentiel pour les êtres animés (dont les personnes).
\textcolor{brown}{\zh{存在动词:有,存在着。如:某人在家或家里有几口人。}}


\begin{exe}
\sn \textcolor{darkblue}{\textbf{\ipa{mɤ˧-dʑo˩}}}
\trans \mytextsc{neg}
\trans \textit{\textcolor{brown}{\zh{没有、不在}}}
\end{exe}
\begin{exe}
\sn \textcolor{darkblue}{\textbf{\ipa{ʈʂʰɯ˧ | ɑ˩ʁo˧ mɤ˧-dʑo˩!}}}
\trans Il/elle n'est pas à la maison!
\trans \textit{\textcolor{brown}{\zh{他不在家!}}}
\end{exe}


\paragraph{\hspace{-0.5cm} {\Large \textcolor{darkblue}{\textbf{\ipa{‑dʑɯ˧}}}}} \hypertarget{‑dz£M\string_M1}{}
\markboth{\textcolor{darkblue}{\textbf{\ipa{‑dʑɯ˧}}}}{}
\textcolor{teal}{\mytextsc{suffixe}} \hspace{4pt} Ton~: M.

\mytextsc{expérientiel}.
\textcolor{brown}{\zh{……过。}}




\paragraph{\hspace{-0.5cm} {\Large \textcolor{darkblue}{\textbf{\ipa{dʑɯ˧}}}}} \hypertarget{dz£M\string_M1}{}
\markboth{\textcolor{darkblue}{\textbf{\ipa{dʑɯ˧}}}}{}
\textcolor{teal}{\mytextsc{nom}} \hspace{4pt} Ton~: M.

Le moment (de), l'heure (de).
\textcolor{brown}{\zh{……的时间。}}


\begin{exe}
\sn \textcolor{darkblue}{\textbf{\ipa{ʈʂʰwɤ˩ dzɯ˩-bi˩-dʑɯ˩˥}}}
\trans l'heure du repas du soir
\trans \textit{\textcolor{brown}{\zh{吃晚餐的时间}}}
\end{exe}
\begin{exe}
\sn \textcolor{darkblue}{\textbf{\ipa{ɑ˩pʰo˩ bi˩-dʑɯ˩˥}}}
\trans l'heure d'aller dehors, le moment d'aller dehors
\trans \textit{\textcolor{brown}{\zh{出去的(合适)时间}}}
\end{exe}
\begin{exe}
\sn \textcolor{darkblue}{\textbf{\ipa{le˧-ʑi˧-bi˧-dʑɯ˧ tʰv̩˧-ze˩!}}}
\trans Il est l'heure d'aller dormir!
\trans \textit{\textcolor{brown}{\zh{睡觉的时间到了!}}}
\end{exe}
\begin{exe}
\sn \textcolor{darkblue}{\textbf{\ipa{ʐo˩ dzɯ˩-bi˩-dʑɯ˩˥}}}
\trans l'heure du déjeuner
\trans \textit{\textcolor{brown}{\zh{午饭的时间}}}
\end{exe}


\paragraph{\hspace{-0.5cm} {\Large \textcolor{darkblue}{\textbf{\ipa{dʑɯ˧dv̩˧}}}}} \hypertarget{dz£M\string_Mdv\string_=\string_M1}{}
\markboth{\textcolor{darkblue}{\textbf{\ipa{dʑɯ˧dv̩˧}}}}{}
\textcolor{teal}{\mytextsc{nom}} \hspace{4pt} Ton~: M.

Ver de terre.
\textcolor{brown}{\zh{蚯蚓。}}


\begin{exe}
\sn \textcolor{darkblue}{\textbf{\ipa{dʑɯ˧dv̩˧-mi˩, | ə˩-dʑo˩˥?}}}
\trans Les vers de terre femelle, ça existe? (Cette phrase permet d'éliciter une forme associant 'ver de terre' au suffixe 'femelle', dans l'idée d'étudier les règles tonales productives en synchronie pour les suffixes de genre.)
\trans \textit{\textcolor{brown}{\zh{有母蚯蚓吗?}}}
\end{exe}
\begin{exe}
\sn \textcolor{darkblue}{\textbf{\ipa{dʑɯ˧dv̩˧-pʰv̩˩, | ə˩-dʑo˩˥?}}}
\trans Les vers de terre mâle, ça existe? (Cette phrase permet d'éliciter une forme associant 'ver de terre' au suffixe 'mâle', dans l'idée d'étudier les règles tonales productives en synchronie pour les suffixes de genre.)
\trans \textit{\textcolor{brown}{\zh{有公蚯蚓吗?}}}
\end{exe}
 \mytextsc{clf}~: \textcolor{darkblue}{\textbf{\ipa{kʰɯ˩}}} 

\paragraph{\hspace{-0.5cm} {\Large \textcolor{darkblue}{\textbf{\ipa{dʑɯ˧dze˧mi\#˥}}}}} \hypertarget{dz£M\string_Mdze\string_Mmi\#\string_T1}{}
\markboth{\textcolor{darkblue}{\textbf{\ipa{dʑɯ˧dze˧mi\#˥}}}}{}
\textcolor{teal}{\mytextsc{nom}} \hspace{4pt} Ton~: \#H.

Cigale.
\textcolor{brown}{\zh{蝉。}}


\begin{exe}
\sn \textcolor{darkblue}{\textbf{\ipa{dʑɯ˧dze˧-mi˧ tʰv̩˧-mi˧˥ / dʑɯ˧dze˧-mi˧ tʰv̩˧-mi˥\#}}}
\trans \mytextsc{n}+\mytextsc{dem}+\mytextsc{clf}
\trans \textit{\textcolor{brown}{\zh{这只蝉}}}
\end{exe}
 \mytextsc{clf}~: \textcolor{darkblue}{\textbf{\ipa{mi˩}}} 

\paragraph{\hspace{-0.5cm} {\Large \textcolor{darkblue}{\textbf{\ipa{dʑɯ˧ki˥}}}}} \hypertarget{dz£M\string_Mki\string_T1}{}
\markboth{\textcolor{darkblue}{\textbf{\ipa{dʑɯ˧ki˥}}}}{}
\textcolor{teal}{\mytextsc{nom}} \hspace{4pt} Ton~: H\#.

Gaine: large ceinture en tissu, qui peut servir à porter un enfant; aussi: ceinture (terme générique).
\textcolor{brown}{\zh{布带子,用来背小孩的带子,腰带。}}


 \mytextsc{clf}~: \textcolor{darkblue}{\textbf{\ipa{kʰɯ˩}}} 

\paragraph{\hspace{-0.5cm} {\Large \textcolor{darkblue}{\textbf{\ipa{dʑɯ˧-li˧}}}}} \hypertarget{dz£M\string_M-li\string_M1}{}
\markboth{\textcolor{darkblue}{\textbf{\ipa{dʑɯ˧-li˧}}}}{}
\textcolor{teal}{\mytextsc{verbe}} \hspace{4pt} Ton~: M.

Irriguer.
\textcolor{brown}{\zh{灌溉。}}


\begin{exe}
\sn \textcolor{darkblue}{\textbf{\ipa{dʑɯ˧-li˧-ze˧}}}
\trans \mytextsc{pfv}
\trans \textit{\textcolor{brown}{\zh{灌溉了}}}
\end{exe}
\begin{exe}
\sn \textcolor{darkblue}{\textbf{\ipa{dʑɯ˧-mɤ˧-li˧-hĩ˧ lv̩˧}}}
\trans champ sec/pluvial: ``champ qu'on n'irrigue pas"
\trans \textit{\textcolor{brown}{\zh{旱田:不灌溉的田}}}
\end{exe}


\paragraph{\hspace{-0.5cm} {\Large \textcolor{darkblue}{\textbf{\ipa{dʑɯ˧ɭɯ˧}}}}} \hypertarget{dz£M\string_Ml\string_RM\string_M1}{}
\markboth{\textcolor{darkblue}{\textbf{\ipa{dʑɯ˧ɭɯ˧}}}}{}
\textcolor{teal}{\mytextsc{nom}} \hspace{4pt} Ton~: M.

Millet, \textit{Panicum miliaceum}.
\textcolor{brown}{\zh{黍,小米。}}


\begin{exe}
\sn \textcolor{darkblue}{\textbf{\ipa{dʑɯ˧ɭɯ˧-ho\#˥}}}
\trans gruau de millet
\trans \textit{\textcolor{brown}{\zh{小米粥}}}
\end{exe}
\textit{Voir~:} \textcolor{darkblue}{\textbf{\ipa{dʑɯ˧njɤ˧, dʑɯ˧ʈʂʰwæ\#˥}}} 

\paragraph{\hspace{-0.5cm} {\Large \textcolor{darkblue}{\textbf{\ipa{dʑɯ˧mi˧}}}}} \hypertarget{dz£M\string_Mmi\string_M1}{}
\markboth{\textcolor{darkblue}{\textbf{\ipa{dʑɯ˧mi˧}}}}{}
\textcolor{teal}{\mytextsc{nom}} \hspace{4pt} Ton~: M.

Grande rivière.
\textcolor{brown}{\zh{大河。}}


 \mytextsc{clf}~: \textcolor{darkblue}{\textbf{\ipa{kʰɯ˩}}} 

\paragraph{\hspace{-0.5cm} {\Large \textcolor{darkblue}{\textbf{\ipa{dʑɯ˧njɤ˧}}}}} \hypertarget{dz£M\string_Mnj7\string_M1}{}
\markboth{\textcolor{darkblue}{\textbf{\ipa{dʑɯ˧njɤ˧}}}}{}
\textcolor{teal}{\mytextsc{nom}} \hspace{4pt} Ton~: M.

Millet, \textit{Panicum miliaceum}.
\textcolor{brown}{\zh{黍,小米。}}


\begin{exe}
\sn \textcolor{darkblue}{\textbf{\ipa{dʑɯ˧njɤ˧, | ʐɯ˧ tɕɤ˧˥!}}}
\trans Le millet, on s'en sert pour faire du vin!
\trans \textit{\textcolor{brown}{\zh{小米,用来酿酒!}}}
\end{exe}
\begin{exe}
\sn \textcolor{darkblue}{\textbf{\ipa{dʑɯ˧njɤ˧-hɑ\#˥}}}
\trans millet cuit
\trans \textit{\textcolor{brown}{\zh{小米饭}}}
\end{exe}
\textit{Voir~:} \textcolor{darkblue}{\textbf{\ipa{dʑɯ˧ɭɯ˧, dʑɯ˧ʈʂʰwæ\#˥}}} 

\paragraph{\hspace{-0.5cm} {\Large \textcolor{darkblue}{\textbf{\ipa{dʑɯ˧qʰɑ˧}}}}} \hypertarget{dz£M\string_Mq\string_hA\string_M1}{}
\markboth{\textcolor{darkblue}{\textbf{\ipa{dʑɯ˧qʰɑ˧}}}}{}
\textcolor{teal}{\mytextsc{nom}} \hspace{4pt} Ton~: M.

Yyyy.
\textcolor{brown}{\zh{夏枯草。}}


\begin{exe}
\sn \textcolor{darkblue}{\textbf{\ipa{dʑɯ˧qʰɑ˧-bæ˩bæ˩}}}
\trans fleurs de yyyy
\trans \textit{\textcolor{brown}{\zh{夏枯草花}}}
\end{exe}
 \mytextsc{clf}~: \textcolor{darkblue}{\textbf{\ipa{qɑ˩}}} 

\paragraph{\hspace{-0.5cm} {\Large \textcolor{darkblue}{\textbf{\ipa{dʑɯ˧qʰv̩˩}}}}} \hypertarget{dz£M\string_Mq\string_hv\string_=\string_B1}{}
\markboth{\textcolor{darkblue}{\textbf{\ipa{dʑɯ˧qʰv̩˩}}}}{}
\textcolor{teal}{\mytextsc{nom}} \hspace{4pt} Ton~: L\#.

Plante sauvage dont les graines forment de grosses boules de graines.
\textcolor{brown}{\zh{永宁的一种植物。}}


\begin{exe}
\sn \textcolor{darkblue}{\textbf{\ipa{dʑɯ˧qʰv̩˩-lv̩˩lv̩˩}}}
\trans graines de la plante en question
\trans \textit{\textcolor{brown}{\zh{这种植物的种子}}}
\end{exe}
 \mytextsc{clf}~: \textcolor{darkblue}{\textbf{\ipa{ɭɯ˧}}} 

\paragraph{\hspace{-0.5cm} {\Large \textcolor{darkblue}{\textbf{\ipa{dʑɯ˧qʰv̩˧}}}}} \hypertarget{dz£M\string_Mq\string_hv\string_=\string_M1}{}
\markboth{\textcolor{darkblue}{\textbf{\ipa{dʑɯ˧qʰv̩˧}}}}{}
\textcolor{teal}{\mytextsc{nom}} \hspace{4pt} Ton~: M.

Puits.
\textcolor{brown}{\zh{井、水井。}}


\begin{exe}
\sn \textcolor{darkblue}{\textbf{\ipa{ɑ˩ʁo˥ | dʑɯ˧qʰv̩˧ tʰi˧-di˩.}}}
\trans il y a un puits à la maison/dans la ferme.
\trans \textit{\textcolor{brown}{\zh{家里有水井。}}}
\end{exe}
 \mytextsc{clf}~: \textcolor{darkblue}{\textbf{\ipa{ɭɯ˧}}} 

\paragraph{\hspace{-0.5cm} {\Large \textcolor{darkblue}{\textbf{\ipa{dʑɯ˧ʁo˩}}}}} \hypertarget{dz£M\string_MRo\string_B1}{}
\markboth{\textcolor{darkblue}{\textbf{\ipa{dʑɯ˧ʁo˩}}}}{}
\textcolor{teal}{\mytextsc{nom}} \hspace{4pt} Ton~: L\#.

Pêche.
\textcolor{brown}{\zh{桃子。}}




\paragraph{\hspace{-0.5cm} {\Large \textcolor{darkblue}{\textbf{\ipa{dʑɯ˧ʈʂʰwæ\#˥}}}}} \hypertarget{dz£M\string_Mt`s`\string_hw\{\#\string_T1}{}
\markboth{\textcolor{darkblue}{\textbf{\ipa{dʑɯ˧ʈʂʰwæ\#˥}}}}{}
\textcolor{teal}{\mytextsc{nom}} \hspace{4pt} Ton~: \#H.

Millet décortiqué, \textit{Panicum miliaceum}.
\textcolor{brown}{\zh{已碾的小米。}}


\textit{Voir~:} \textcolor{darkblue}{\textbf{\ipa{dʑɯ˧ɭɯ˧, dʑɯ˧njɤ˧}}} 

\paragraph{\hspace{-0.5cm} {\Large \textcolor{darkblue}{\textbf{\ipa{dʑɯ˩}}}}} \hypertarget{dz£M\string_B1}{}
\markboth{\textcolor{darkblue}{\textbf{\ipa{dʑɯ˩}}}}{}
\textcolor{teal}{\mytextsc{nom}} \hspace{4pt} Ton~: L.
\ding{202} 
Eau.
\textcolor{brown}{\zh{水。}}


\begin{exe}
\sn \textcolor{darkblue}{\textbf{\ipa{dʑɯ˧ ʈʰɯ˧}}}
\trans boire de l'eau
\trans \textit{\textcolor{brown}{\zh{喝水}}}
\end{exe}
\begin{exe}
\sn \textcolor{darkblue}{\textbf{\ipa{ʈʂʰɯ˧ dʑɯ˧ ʈʰɯ˧-dʑo˧!}}}
\trans il est en train de boire de l'eau!
\trans \textit{\textcolor{brown}{\zh{他在喝水}}}
\end{exe}
\begin{exe}
\sn \textcolor{darkblue}{\textbf{\ipa{dʑɯ˧ | ɖɯ˧-ʈʰɤ˧ ʈʰɯ˧˥}}}
\trans boire un peu d'eau (littéralement ``une goutte d'eau")
\trans \textit{\textcolor{brown}{\zh{喝一点水(直译:‘一滴水’)}}}
\end{exe}
\begin{exe}
\sn \textcolor{darkblue}{\textbf{\ipa{dʑɯ˩ kʰɯ˩˥}}}
\trans mettre de l'eau
\trans \textit{\textcolor{brown}{\zh{放水}}}
\end{exe}
\begin{exe}
\sn \textcolor{darkblue}{\textbf{\ipa{dʑɯ˩ mæ˩˥}}}
\trans irriguer, arroser, mettre de l’eau
\trans \textit{\textcolor{brown}{\zh{浇灌、灌溉}}}
\end{exe}
\begin{exe}
\sn \textcolor{darkblue}{\textbf{\ipa{dʑɯ˩ qæ˩, | hɑ˩ qæ˩˥ |}}}
\trans description du dépaysement que connaît le voyageur qui arrive en pays étranger et doit 'changer d'eau, changer de nourriture'. Ce dépaysement commande des stratégies de prévention de soucis de santé: en particulier, il était usuel de faire bouillir un peu de terre locale dans de l'eau, et de boire cette préparation de façon à s'accoutumer.
\trans \textit{\textcolor{brown}{\zh{‘换水换土’:这个短语描述旅人到他人乡的情况,带来水土不服的危险。为了预防这类不良反应,摩梭旅人习惯水煮一点当地的土,喝下去,为了适应当地的水土。}}}
\end{exe}
\begin{exe}
\sn \textcolor{darkblue}{\textbf{\ipa{[F5] dʑɯ˧ | mv̩˩tɕo˧ dɑ˧˥}}}
\trans l'eau coule vers le bas
\trans \textit{\textcolor{brown}{\zh{水往下流}}}
\end{exe}
\ding{203} 
Rivière.
\textcolor{brown}{\zh{河流。}}


 \mytextsc{clf}~: \textcolor{darkblue}{\textbf{\ipa{kʰɯ˩}}} 

\paragraph{\hspace{-0.5cm} {\Large \textcolor{darkblue}{\textbf{\ipa{dʑɯ˩\textsubscript{a}}}}}} \hypertarget{dz£M\string_Ba1}{}
\markboth{\textcolor{darkblue}{\textbf{\ipa{dʑɯ˩\textsubscript{a}}}}}{}
\textcolor{teal}{\mytextsc{verbe}} \hspace{4pt} Ton~: L\textsubscript{a}.

Rouler, tordre (par exemple: rouler des brins, pour en faire une corde; ne s'emploie pas pour les fibres très fines, pour lesquelles on dit: \textcolor{darkblue}{\textbf{\ipa{/ʈʂwæ˧˥/}}}).
\textcolor{brown}{\zh{搓(搓绳子)。}}


\begin{exe}
\sn \textcolor{darkblue}{\textbf{\ipa{le˧-dʑɯ˩-ze˩}}}
\trans \mytextsc{accomp} \string_ \mytextsc{pfv}
\trans \textit{\textcolor{brown}{\zh{搓了}}}
\end{exe}
\begin{exe}
\sn \textcolor{darkblue}{\textbf{\ipa{bæ˩ dʑɯ˩˥}}}
\trans tordre une corde
\trans \textit{\textcolor{brown}{\zh{搓绳子}}}
\end{exe}
\begin{exe}
\sn \textcolor{darkblue}{\textbf{\ipa{qʰv̩˩ɖʐæ˩ dʑɯ˥}}}
\trans faire une ficelle, une petite cordelette
\trans \textit{\textcolor{brown}{\zh{搓一根小绳子}}}
\end{exe}
\begin{exe}
\sn \textcolor{darkblue}{\textbf{\ipa{ɖɯ˧-kʰwɤ˧ dʑɯ˥}}}
\trans tordre un peu / tordre quelque chose
\trans \textit{\textcolor{brown}{\zh{搓一下}}}
\end{exe}


\paragraph{\hspace{-0.5cm} {\Large \textcolor{darkblue}{\textbf{\ipa{dʑɯ˩-æ̃˩tsɯ˧}}}}} \hypertarget{dz£M\string_B-\{\string_~\string_BtsM\string_M1}{}
\markboth{\textcolor{darkblue}{\textbf{\ipa{dʑɯ˩-æ̃˩tsɯ˧}}}}{}
\textcolor{teal}{\mytextsc{nom}} \hspace{4pt} Ton~: L-LM.

Gibier d'eau, sauvagine; employé pour divers oiseaux tels que: bécasseau, chevalier (\textit{Calidris}), avocette, marouette, et râle.
\textcolor{brown}{\zh{水禽,包括几种不同的小鸟,如:鹬。}}


 \mytextsc{clf}~: \textcolor{darkblue}{\textbf{\ipa{ɭɯ˧}}} 

\paragraph{\hspace{-0.5cm} {\Large \textcolor{darkblue}{\textbf{\ipa{dʑɯ˩dze˩}}}}} \hypertarget{dz£M\string_Bdze\string_B1}{}
\markboth{\textcolor{darkblue}{\textbf{\ipa{dʑɯ˩dze˩}}}}{}
\textcolor{teal}{\mytextsc{nom}} \hspace{4pt} Ton~: L.

Louche utilisée pour faire la cuisine, distribuer la soupe.
\textcolor{brown}{\zh{舀汤的勺子。}}


 \mytextsc{clf}~: \textcolor{darkblue}{\textbf{\ipa{nɑ˧}}} 

\paragraph{\hspace{-0.5cm} {\Large \textcolor{darkblue}{\textbf{\ipa{dʑɯ˩gɤ˩di˩}}}}} \hypertarget{dz£M\string_Bg7\string_Bdi\string_B1}{}
\markboth{\textcolor{darkblue}{\textbf{\ipa{dʑɯ˩gɤ˩di˩}}}}{}
\textcolor{teal}{\mytextsc{nom}} \hspace{4pt} Ton~: L.

Palanche.
\textcolor{brown}{\zh{扁担。}}


 \mytextsc{clf}~: \textcolor{darkblue}{\textbf{\ipa{nɑ˧}}} 

\paragraph{\hspace{-0.5cm} {\Large \textcolor{darkblue}{\textbf{\ipa{dʑɯ˩gv̩˩}}}}} \hypertarget{dz£M\string_Bgv\string_=\string_B1}{}
\markboth{\textcolor{darkblue}{\textbf{\ipa{dʑɯ˩gv̩˩}}}}{}
\textcolor{teal}{\mytextsc{nom}} \hspace{4pt} Ton~: L.

Cuve où l'on conserve l'eau potable, tonneau d'eau. A la date de l'enquête, il s'agissait d'un baril en fer.
\textcolor{brown}{\zh{大水桶,水槽。}}


\begin{exe}
\sn \textcolor{darkblue}{\textbf{\ipa{[F5] pv̩˩-dʑɯ˩gv̩˩˥}}}
\trans même sens
\end{exe}
 \mytextsc{clf}~: \textcolor{darkblue}{\textbf{\ipa{ɭɯ˧}}} 

\paragraph{\hspace{-0.5cm} {\Large \textcolor{darkblue}{\textbf{\ipa{dʑɯ˩gv̩˥}}}}} \hypertarget{dz£M\string_Bgv\string_=\string_T1}{}
\markboth{\textcolor{darkblue}{\textbf{\ipa{dʑɯ˩gv̩˥}}}}{}
\textcolor{teal}{\mytextsc{adjectif}} \hspace{4pt} Ton~: LH.

Voûté, qui a le dos rond, courbé.
\textcolor{brown}{\zh{驼背。}}




\paragraph{\hspace{-0.5cm} {\Large \textcolor{darkblue}{\textbf{\ipa{dʑɯ˩hṽ˧˥}}}}} \hypertarget{dz£M\string_Bhv\string_~\string_M\string_T1}{}
\markboth{\textcolor{darkblue}{\textbf{\ipa{dʑɯ˩hṽ˧˥}}}}{}
\textcolor{teal}{\mytextsc{nom}} \hspace{4pt} Ton~: LM+MH\#.

Mélange d'eau et de farine: par ex. de la pâte à pain, du tsamba avec de l'eau….
\textcolor{brown}{\zh{面和水和成的浆糊。}}




\paragraph{\hspace{-0.5cm} {\Large \textcolor{darkblue}{\textbf{\ipa{dʑɯ˩-hwæ˩tsɯ˥}}}}} \hypertarget{dz£M\string_B-hw\{\string_BtsM\string_T1}{}
\markboth{\textcolor{darkblue}{\textbf{\ipa{dʑɯ˩-hwæ˩tsɯ˥}}}}{}
\textcolor{teal}{\mytextsc{nom}} \hspace{4pt} Ton~: L+H\#.

Musaraigne; la locutrice emploie une périphrase: ``souris sauvage".
\textcolor{brown}{\zh{尖鼠、鼩鼱。}}




\paragraph{\hspace{-0.5cm} {\Large \textcolor{darkblue}{\textbf{\ipa{dʑɯ˩kʰi˩}}}}} \hypertarget{dz£M\string_Bk\string_hi\string_B1}{}
\markboth{\textcolor{darkblue}{\textbf{\ipa{dʑɯ˩kʰi˩}}}}{}
\textcolor{teal}{\mytextsc{nom}} \hspace{4pt} Ton~: L.

Bord de l'eau.
\textcolor{brown}{\zh{水边。}}




\paragraph{\hspace{-0.5cm} {\Large \textcolor{darkblue}{\textbf{\ipa{dʑɯ˩kʰv̩˩}}}}} \hypertarget{dz£M\string_Bk\string_hv\string_=\string_B1}{}
\markboth{\textcolor{darkblue}{\textbf{\ipa{dʑɯ˩kʰv̩˩}}}}{}
\textcolor{teal}{\mytextsc{nom}} \hspace{4pt} Ton~: L.

Mousse.
\textcolor{brown}{\zh{青苔。}}




\paragraph{\hspace{-0.5cm} {\Large \textcolor{darkblue}{\textbf{\ipa{dʑɯ˩nɑ˩hæ̃˩tʰɑ˩}}}}} \hypertarget{dz£M\string_BnA\string_Bh\{\string_~\string_Bt\string_hA\string_B1}{}
\markboth{\textcolor{darkblue}{\textbf{\ipa{dʑɯ˩nɑ˩hæ̃˩tʰɑ˩}}}}{}
\textcolor{teal}{\mytextsc{nom}} \hspace{4pt} Ton~: L.

Moulin à eau.
\textcolor{brown}{\zh{水磨。}}


 \mytextsc{clf}~: \textcolor{darkblue}{\textbf{\ipa{pɤ˩}}} 

\paragraph{\hspace{-0.5cm} {\Large \textcolor{darkblue}{\textbf{\ipa{dʑɯ˩nɑ˩mi˩}}}}} \hypertarget{dz£M\string_BnA\string_Bmi\string_B1}{}
\markboth{\textcolor{darkblue}{\textbf{\ipa{dʑɯ˩nɑ˩mi˩}}}}{}
\textcolor{teal}{\mytextsc{nom}} \hspace{4pt} Ton~: L.

Forêt d'altitude, régions sauvages en altitude.
\textcolor{brown}{\zh{深山老林、高山上的地方。}}


\textit{Voir~:} \textcolor{darkblue}{\textbf{\ipa{dʑɯ˩nɑ˩mi˩-ʁo˩, dʑɯ˩ʁo˩}}} 

\paragraph{\hspace{-0.5cm} {\Large \textcolor{darkblue}{\textbf{\ipa{dʑɯ˩nɑ˩mi˩-ʁo˩}}}}} \hypertarget{dz£M\string_BnA\string_Bmi\string_B-Ro\string_B1}{}
\markboth{\textcolor{darkblue}{\textbf{\ipa{dʑɯ˩nɑ˩mi˩-ʁo˩}}}}{}
\textcolor{teal}{\mytextsc{nom}} \hspace{4pt} Ton~: L.

Forêt d'altitude, régions sauvages en altitude.
\textcolor{brown}{\zh{深山老林、高山上的地方。}}


\textit{Voir~:} \textcolor{darkblue}{\textbf{\ipa{dʑɯ˩nɑ˩mi˩, dʑɯ˩ʁo˩}}} 

\paragraph{\hspace{-0.5cm} {\Large \textcolor{darkblue}{\textbf{\ipa{dʑɯ˩pɤ˩-kv̩˧hĩ˩}}}}} \hypertarget{dz£M\string_Bp7\string_B-kv\string_=\string_Mhi\string_~\string_B1}{}
\markboth{\textcolor{darkblue}{\textbf{\ipa{dʑɯ˩pɤ˩-kv̩˧hĩ˩}}}}{}
\textcolor{teal}{\mytextsc{nom}} \hspace{4pt} Ton~: L-L\#.

Source.
\textcolor{brown}{\zh{水泉、山泉。}}


\begin{exe}
\sn \textcolor{darkblue}{\textbf{\ipa{dʑɯ˩pɤ˩-kv̩˧hĩ˩ | tʰi˧-di˩}}}
\trans il y a une source
\trans \textit{\textcolor{brown}{\zh{有水泉}}}
\end{exe}
\textit{Voir~:} \textcolor{darkblue}{\textbf{\ipa{dʑɯ˩pɤ˩qʰv̩˩, dʑɯ˩pɤ˩tv̩˩qʰv̩˥}}} 

\paragraph{\hspace{-0.5cm} {\Large \textcolor{darkblue}{\textbf{\ipa{dʑɯ˩pɤ˩qʰv̩˩}}}}} \hypertarget{dz£M\string_Bp7\string_Bq\string_hv\string_=\string_B1}{}
\markboth{\textcolor{darkblue}{\textbf{\ipa{dʑɯ˩pɤ˩qʰv̩˩}}}}{}
\textcolor{teal}{\mytextsc{nom}} \hspace{4pt} Ton~: L.

Source.
\textcolor{brown}{\zh{水泉、山泉。}}


\textit{Voir~:} \textcolor{darkblue}{\textbf{\ipa{dʑɯ˩pɤ˩-kv̩˧hĩ˩, dʑɯ˩pɤ˩tv̩˩qʰv̩˥}}} 

\paragraph{\hspace{-0.5cm} {\Large \textcolor{darkblue}{\textbf{\ipa{dʑɯ˩pɤ˩tv̩˩qʰv̩˥}}}}} \hypertarget{dz£M\string_Bp7\string_Btv\string_=\string_Bq\string_hv\string_=\string_T1}{}
\markboth{\textcolor{darkblue}{\textbf{\ipa{dʑɯ˩pɤ˩tv̩˩qʰv̩˥}}}}{}
\textcolor{teal}{\mytextsc{nom}} \hspace{4pt} Ton~: L+H\#.

Source.
\textcolor{brown}{\zh{水泉、山泉。}}


\textit{Voir~:} \textcolor{darkblue}{\textbf{\ipa{dʑɯ˩pɤ˩qʰv̩˩, dʑɯ˩pɤ˩-kv̩˧hĩ˩}}} 

\paragraph{\hspace{-0.5cm} {\Large \textcolor{darkblue}{\textbf{\ipa{dʑɯ˩pʰæ˩}}}}} \hypertarget{dz£M\string_Bp\string_h\{\string_B1}{}
\markboth{\textcolor{darkblue}{\textbf{\ipa{dʑɯ˩pʰæ˩}}}}{}
\textcolor{teal}{\mytextsc{nom}} \hspace{4pt} Ton~: L.

Glace.
\textcolor{brown}{\zh{冰。}}


 \mytextsc{clf}~: \textcolor{darkblue}{\textbf{\ipa{pʰæ˧˥}}} 

\paragraph{\hspace{-0.5cm} {\Large \textcolor{darkblue}{\textbf{\ipa{dʑɯ˩qʰæ˩}}}}} \hypertarget{dz£M\string_Bq\string_h\{\string_B1}{}
\markboth{\textcolor{darkblue}{\textbf{\ipa{dʑɯ˩qʰæ˩}}}}{}
\textcolor{teal}{\mytextsc{nom}} \hspace{4pt} Ton~: L.

Eau froide.
\textcolor{brown}{\zh{凉水。}}




\paragraph{\hspace{-0.5cm} {\Large \textcolor{darkblue}{\textbf{\ipa{dʑɯ˩qʰwɤ˩-lv̩˩}}}}} \hypertarget{dz£M\string_Bq\string_hw7\string_B-lv\string_=\string_B1}{}
\markboth{\textcolor{darkblue}{\textbf{\ipa{dʑɯ˩qʰwɤ˩-lv̩˩}}}}{}
\textcolor{teal}{\mytextsc{nom}} \hspace{4pt} Ton~: L.

Marais (terre impropre à la culture).
\textcolor{brown}{\zh{沼泽、湿地。}}Dialecte chinois local~: \zh{潮地。}


 \mytextsc{clf}~: \textcolor{darkblue}{\textbf{\ipa{kɤ˧˥}}} 

\paragraph{\hspace{-0.5cm} {\Large \textcolor{darkblue}{\textbf{\ipa{dʑɯ˩ʁo˩}}}}} \hypertarget{dz£M\string_BRo\string_B1}{}
\markboth{\textcolor{darkblue}{\textbf{\ipa{dʑɯ˩ʁo˩}}}}{}
\textcolor{teal}{\mytextsc{nom}} \hspace{4pt} Ton~: L.

Forêt d'altitude, régions sauvages en altitude.
\textcolor{brown}{\zh{深山老林、高山上的地方。}}


\textit{Voir~:} \textcolor{darkblue}{\textbf{\ipa{dʑɯ˩nɑ˩mi˩, dʑɯ˩nɑ˩mi˩-ʁo˩}}} 

\paragraph{\hspace{-0.5cm} {\Large \textcolor{darkblue}{\textbf{\ipa{dʑɯ˩ʁo˩-æ̃˧}}}}} \hypertarget{dz£M\string_BRo\string_B-\{\string_~\string_M1}{}
\markboth{\textcolor{darkblue}{\textbf{\ipa{dʑɯ˩ʁo˩-æ̃˧}}}}{}
\textcolor{teal}{\mytextsc{nom}} \hspace{4pt} Ton~: L-M.

Caille, \textit{Coturnix}; terme employé pour divers oiseaux, dont des râles (\textit{Crex}).
\textcolor{brown}{\zh{鹌鹑。}}


 \mytextsc{clf}~: \textcolor{darkblue}{\textbf{\ipa{mi˩}}} 

\paragraph{\hspace{-0.5cm} {\Large \textcolor{darkblue}{\textbf{\ipa{dʑɯ˩ʁo˩-bo˧}}}}} \hypertarget{dz£M\string_BRo\string_B-bo\string_M1}{}
\markboth{\textcolor{darkblue}{\textbf{\ipa{dʑɯ˩ʁo˩-bo˧}}}}{}
\textcolor{teal}{\mytextsc{nom}} \hspace{4pt} Ton~: L-M.

Sanglier, porc sauvage.
\textcolor{brown}{\zh{野猪。}}


 \mytextsc{clf}~: \textcolor{darkblue}{\textbf{\ipa{mi˩}}} \textit{Syn~:} \hyperlink{bo\string_Btv\string_=\#\string_T1}{\textcolor{darkblue}{\textbf{\ipa{bo˩tv̩\#˥}}}}. 

\paragraph{\hspace{-0.5cm} {\Large \textcolor{darkblue}{\textbf{\ipa{dʑɯ˩ʁo˩-dze˧}}}}} \hypertarget{dz£M\string_BRo\string_B-dze\string_M1}{}
\markboth{\textcolor{darkblue}{\textbf{\ipa{dʑɯ˩ʁo˩-dze˧}}}}{}
\textcolor{teal}{\mytextsc{nom}} \hspace{4pt} Ton~: L-M.

Xanthoxyle sauvage.
\textcolor{brown}{\zh{野花椒。}}




\paragraph{\hspace{-0.5cm} {\Large \textcolor{darkblue}{\textbf{\ipa{dʑɯ˩ʁo˩-hwɤ˩li˧}}}}} \hypertarget{dz£M\string_BRo\string_B-hw7\string_Bli\string_M1}{}
\markboth{\textcolor{darkblue}{\textbf{\ipa{dʑɯ˩ʁo˩-hwɤ˩li˧}}}}{}
\textcolor{teal}{\mytextsc{nom}} \hspace{4pt} Ton~: L-LM.

Chat sauvage, \textit{Felis temincki}.
\textcolor{brown}{\zh{野猫。}}


 \mytextsc{clf}~: \textcolor{darkblue}{\textbf{\ipa{mi˩}}} 

\paragraph{\hspace{-0.5cm} {\Large \textcolor{darkblue}{\textbf{\ipa{dʑɯ˩ʁo˩-ɬi˩bi˧}}}}} \hypertarget{dz£M\string_BRo\string_B-Ki\string_Bbi\string_M1}{}
\markboth{\textcolor{darkblue}{\textbf{\ipa{dʑɯ˩ʁo˩-ɬi˩bi˧}}}}{}
\textcolor{teal}{\mytextsc{nom}} \hspace{4pt} Ton~: L-LM.

Igname de Chine (shan-yao).
\textcolor{brown}{\zh{山药。}}


 \mytextsc{clf}~: \textcolor{darkblue}{\textbf{\ipa{ɭɯ˧}}} 

\paragraph{\hspace{-0.5cm} {\Large \textcolor{darkblue}{\textbf{\ipa{dʑɯ˩ʁo˩-zɯ˩}}}}} \hypertarget{dz£M\string_BRo\string_B-zM\string_B1}{}
\markboth{\textcolor{darkblue}{\textbf{\ipa{dʑɯ˩ʁo˩-zɯ˩}}}}{}
\textcolor{teal}{\mytextsc{nom}} \hspace{4pt} Ton~: L.

Herbes de la montagne, herbes sauvages, foin poussant sur l'alpage.
\textcolor{brown}{\zh{野草。}}


\begin{exe}
\sn \textcolor{darkblue}{\textbf{\ipa{ʈʂʰɯ˧ | dʑɯ˩ʁo˩-zɯ˩ ɲi˥.}}}
\trans \mytextsc{dem} \string_ \mytextsc{cop}
\trans \textit{\textcolor{brown}{\zh{\mytextsc{指示代词} \string_ \mytextsc{系词}}}}
\end{exe}
 \mytextsc{clf}~: \textcolor{darkblue}{\textbf{\ipa{qɑ˩}}} \textcolor{darkblue}{\textbf{\ipa{po˧}}} botte; unité

\paragraph{\hspace{-0.5cm} {\Large \textcolor{darkblue}{\textbf{\ipa{dʑɯ˩si˩}}}}} \hypertarget{dz£M\string_Bsi\string_B1}{}
\markboth{\textcolor{darkblue}{\textbf{\ipa{dʑɯ˩si˩}}}}{}
\textcolor{teal}{\mytextsc{nom}} \hspace{4pt} Ton~: L.

Chêne blanc oriental.
\textcolor{brown}{\zh{青冈树、槲栎。}}


\textit{Syn~:} \hyperlink{dzi\string_Mdzi\string_M1}{\textcolor{darkblue}{\textbf{\ipa{dzi˧dzi˧}}}}. 

\paragraph{\hspace{-0.5cm} {\Large \textcolor{darkblue}{\textbf{\ipa{dʑɯ˩so˩}}}}} \hypertarget{dz£M\string_Bso\string_B1}{}
\markboth{\textcolor{darkblue}{\textbf{\ipa{dʑɯ˩so˩}}}}{}
\textcolor{teal}{\mytextsc{nom}} \hspace{4pt} Ton~: L.

Vague.
\textcolor{brown}{\zh{波浪。}}


\begin{exe}
\sn \textcolor{darkblue}{\textbf{\ipa{dʑɯ˩so˩ pʰv̩˩˥}}}
\trans il y a une vague, une vague déferle
\trans \textit{\textcolor{brown}{\zh{有波浪}}}
\end{exe}
 \mytextsc{clf}~: \textcolor{darkblue}{\textbf{\ipa{pʰæ˧˥}}} 

\paragraph{\hspace{-0.5cm} {\Large \textcolor{darkblue}{\textbf{\ipa{dʑɯ˩ʂo˥}}}}} \hypertarget{dz£M\string_Bs`o\string_T1}{}
\markboth{\textcolor{darkblue}{\textbf{\ipa{dʑɯ˩ʂo˥}}}}{}
\textcolor{teal}{\mytextsc{nom}} \hspace{4pt} Ton~: L+H\#.

Nom d'un rituel.
\textcolor{brown}{\zh{一项仪式。}}


\begin{exe}
\sn \textcolor{darkblue}{\textbf{\ipa{dʑɯ˩ʂo˥-tsɑ˩bɤ˩}}}
\trans farine (tsamba) pouvant servir aux cérémonies; elle ne doit pas contenir d'avoine. A la fin de la cérémonie, on la jette
\trans \textit{\textcolor{brown}{\zh{做仪式时所使用的面粉。这种面粉里不要含有燕麦。仪式结束后,面粉被扔掉。}}}
\end{exe}


\paragraph{\hspace{-0.5cm} {\Large \textcolor{darkblue}{\textbf{\ipa{dʑɯ˩ʂwæ˩}}}}} \hypertarget{dz£M\string_Bs`w\{\string_B1}{}
\markboth{\textcolor{darkblue}{\textbf{\ipa{dʑɯ˩ʂwæ˩}}}}{}
\textcolor{teal}{\mytextsc{nom}} \hspace{4pt} Ton~: L.

\textit{Lysimachia christinae Hance}.
\textcolor{brown}{\zh{过路黄。}}


\begin{exe}
\sn \textcolor{darkblue}{\textbf{\ipa{dʑɯ˩ʂwæ˩-bæ˥bæ˩}}}
\trans fleur de yyyyy
\trans \textit{\textcolor{brown}{\zh{过路黄花}}}
\end{exe}


\paragraph{\hspace{-0.5cm} {\Large \textcolor{darkblue}{\textbf{\ipa{dʑɯ˩tɤ˩ɻ̍˥}}}}} \hypertarget{dz£M\string_Bt7\string_Br£`̍\string_T1}{}
\markboth{\textcolor{darkblue}{\textbf{\ipa{dʑɯ˩tɤ˩ɻ̍˥}}}}{}
\textcolor{teal}{\mytextsc{adjectif}} \hspace{4pt} Ton~: L+H\#.

Mouillé, humide.
\textcolor{brown}{\zh{湿。}}


\begin{exe}
\sn \textcolor{darkblue}{\textbf{\ipa{dʑɯ˩tɤ˩ɻ̍˥ gv̩˩-ze˩}}}
\trans ça s'est mouillé
\trans \textit{\textcolor{brown}{\zh{湿了!}}}
\end{exe}


\paragraph{\hspace{-0.5cm} {\Large \textcolor{darkblue}{\textbf{\ipa{dʑɯ˩tɕʰɯ˩lɑ˩qʰɑ˥}}}}} \hypertarget{dz£M\string_Bts£\string_hM\string_BlA\string_Bq\string_hA\string_T1}{}
\markboth{\textcolor{darkblue}{\textbf{\ipa{dʑɯ˩tɕʰɯ˩lɑ˩qʰɑ˥}}}}{}
\textcolor{teal}{\mytextsc{nom}} \hspace{4pt} Ton~: L+H\#.

Sorte de prunelle très acide, qui pousse à l'état sauvage; utilisée en décoction, en association avec du gingembre et de petites pommes séchées, contre les maux de gorge.
\textcolor{brown}{\zh{一种梅子。}}




\paragraph{\hspace{-0.5cm} {\Large \textcolor{darkblue}{\textbf{\ipa{dʑɯ˩tɕʰɯ˩lɑ˩qʰæ˥}}}}} \hypertarget{dz£M\string_Bts£\string_hM\string_BlA\string_Bq\string_h\{\string_T1}{}
\markboth{\textcolor{darkblue}{\textbf{\ipa{dʑɯ˩tɕʰɯ˩lɑ˩qʰæ˥}}}}{}
\textcolor{teal}{\mytextsc{nom}} \hspace{4pt} Ton~: L+H\#.

Argousier, \textit{Hippophae rhamnoides Linn.}.
\textcolor{brown}{\zh{沙棘。}}




\paragraph{\hspace{-0.5cm} {\Large \textcolor{darkblue}{\textbf{\ipa{dʑɯ˩tsʰi˩}}}}} \hypertarget{dz£M\string_Bts\string_hi\string_B1}{}
\markboth{\textcolor{darkblue}{\textbf{\ipa{dʑɯ˩tsʰi˩}}}}{}
\textcolor{teal}{\mytextsc{nom}} \hspace{4pt} Ton~: L.

Eau bouillante, eau chaude.
\textcolor{brown}{\zh{开水,热水。}}




\paragraph{\hspace{-0.5cm} {\Large \textcolor{darkblue}{\textbf{\ipa{dʑɯ˩tsʰi˩ʈʰɯ˩di˩}}}}} \hypertarget{dz£M\string_Bts\string_hi\string_Bt`\string_hM\string_Bdi\string_B1}{}
\markboth{\textcolor{darkblue}{\textbf{\ipa{dʑɯ˩tsʰi˩ʈʰɯ˩di˩}}}}{}
\textcolor{teal}{\mytextsc{nom}} \hspace{4pt} Ton~: L.

Petite gourde thermos individuelle.
\textcolor{brown}{\zh{口杯。}}


 \mytextsc{clf}~: \textcolor{darkblue}{\textbf{\ipa{ɭɯ˧}}} 

\paragraph{\hspace{-0.5cm} {\Large \textcolor{darkblue}{\textbf{\ipa{dʑɯ˩ʈv̩˧}}}}} \hypertarget{dz£M\string_Bt`v\string_=\string_M1}{}
\markboth{\textcolor{darkblue}{\textbf{\ipa{dʑɯ˩ʈv̩˧}}}}{}
\textcolor{teal}{\mytextsc{adjectif}} \hspace{4pt} Ton~: LM.

Être gravement voûté, avoir une bosse.
\textcolor{brown}{\zh{驼背(严重的病)。}}


\begin{exe}
\sn \textcolor{darkblue}{\textbf{\ipa{dʑɯ˩ʈv̩˧-ze˩}}}
\trans \mytextsc{pfv}
\trans \textit{\textcolor{brown}{\zh{驼背了}}}
\end{exe}


\paragraph{\hspace{-0.5cm} {\Large \textcolor{darkblue}{\textbf{\ipa{dʑɯ˩zo˩}}}}} \hypertarget{dz£M\string_Bzo\string_B1}{}
\markboth{\textcolor{darkblue}{\textbf{\ipa{dʑɯ˩zo˩}}}}{}
\textcolor{teal}{\mytextsc{nom}} \hspace{4pt} Ton~: L.

Ruisseau.
\textcolor{brown}{\zh{溪流。}}


 \mytextsc{clf}~: \textcolor{darkblue}{\textbf{\ipa{kʰɯ˩}}} 

\paragraph{\hspace{-0.5cm} {\Large \textcolor{darkblue}{\textbf{\ipa{dʑɯ˩ʐv̩˩}}}}} \hypertarget{dz£M\string_Bz`v\string_=\string_B1}{}
\markboth{\textcolor{darkblue}{\textbf{\ipa{dʑɯ˩ʐv̩˩}}}}{}
\textcolor{teal}{\mytextsc{verbe}} \hspace{4pt} Ton~: L.

Nager.
\textcolor{brown}{\zh{游泳。}}




\paragraph{\hspace{-0.5cm} {\Large \textcolor{darkblue}{\textbf{\ipa{dʑɯ˧˥}}}}} \hypertarget{dz£M\string_M\string_T1}{}
\markboth{\textcolor{darkblue}{\textbf{\ipa{dʑɯ˧˥}}}}{}
\textcolor{teal}{\mytextsc{adjectif}} \hspace{4pt} Ton~: MH.

Nombreux, beaucoup (dénombrable).
\textcolor{brown}{\zh{多。}}


\begin{exe}
\sn \textcolor{darkblue}{\textbf{\ipa{hĩ˧ dʑɯ˩}}}
\trans les gens sont nombreux, il y a beaucoup de monde
\trans \textit{\textcolor{brown}{\zh{人多。}}}
\end{exe}


\newpage
\section*{\centering- \textcolor{darkblue}{\textbf{\ipa{ɖ}}} -}
\paragraph{\hspace{-0.5cm} {\Large \textcolor{darkblue}{\textbf{\ipa{ɖæ˥}}}}} \hypertarget{d`\{\string_T1}{}
\markboth{\textcolor{darkblue}{\textbf{\ipa{ɖæ˥}}}}{}
\textcolor{teal}{\mytextsc{adjectif}} \hspace{4pt} Ton~: H.

Court.
\textcolor{brown}{\zh{短。}}




\paragraph{\hspace{-0.5cm} {\Large \textcolor{darkblue}{\textbf{\ipa{ɖæ˧\textasciitilde{}ɖæ˩}}}}} \hypertarget{d`\{\string_M~d`\{\string_B1}{}
\markboth{\textcolor{darkblue}{\textbf{\ipa{ɖæ˧\textasciitilde{}ɖæ˩}}}}{}
\textcolor{teal}{\mytextsc{adjectif}} \hspace{4pt} Ton~: L\#.

Horizontal.
\textcolor{brown}{\zh{横着(横躺在路上)。}}


\begin{exe}
\sn \textcolor{darkblue}{\textbf{\ipa{ɖæ˧\textasciitilde{}ɖæ˩ | tʰi˧-tɕɯ˥}}}
\trans poser à plat
\trans \textit{\textcolor{brown}{\zh{横着放}}}
\end{exe}


\paragraph{\hspace{-0.5cm} {\Large \textcolor{darkblue}{\textbf{\ipa{ɖæ˩\textsubscript{a}}}}}} \hypertarget{d`\{\string_Ba1}{}
\markboth{\textcolor{darkblue}{\textbf{\ipa{ɖæ˩\textsubscript{a}}}}}{}
\textcolor{teal}{\mytextsc{classificateur}} \hspace{4pt} Ton~: L\textsubscript{a}.

Classificateur des sections, pour objets pouvant être divisés dans le sens de la longueur. Pour une route, cette unité correspond à 1/2 journée de marche; pour du tissu, elle correspond à une pièce.
\textcolor{brown}{\zh{量词:路(段)/布(匹)。}}


\begin{exe}
\sn \textcolor{darkblue}{\textbf{\ipa{ʐɤ˩mi˩˥ | ɖɯ˧-ɖæ˩}}}
\trans un bout de chemin, un bout de route
\trans \textit{\textcolor{brown}{\zh{一段路}}}
\end{exe}
\begin{exe}
\sn \textcolor{darkblue}{\textbf{\ipa{ɲi˧, ɲi˩-ɖæ˩! |}}}
\trans formule figée traditionnel: un jour, ça fait deux étapes! (si on peut parvenir quelque part avant le déjeuner, la distance est considérée comme une seule étape; si on y parvient dans l'après-midi, on compte deux étapes)
\trans \textit{\textcolor{brown}{\zh{一天两段路!(说明:早上出发,如果午饭前能到目的地,距离算是一段,如果下午晚上才到,算两段。)}}}
\end{exe}
\begin{exe}
\sn \textcolor{darkblue}{\textbf{\ipa{ʈʂʰɯ˧-ɖæ˥}}}
\trans \mytextsc{dem} \string_ (ton: H\# / H\$)
\trans \textit{\textcolor{brown}{\zh{\mytextsc{指示代词} \string_}}}
\end{exe}


\paragraph{\hspace{-0.5cm} {\Large \textcolor{darkblue}{\textbf{\ipa{ɖæ˩\textsubscript{a}}}}}} \hypertarget{d`\{\string_Ba1}{}
\markboth{\textcolor{darkblue}{\textbf{\ipa{ɖæ˩\textsubscript{a}}}}}{}
\textcolor{teal}{\mytextsc{verbe}} \hspace{4pt} Ton~: L\textsubscript{a}.

Passer, traverser (une rivière en bateau, une montagne…).
\textcolor{brown}{\zh{渡(坐船渡河……)。}}


\begin{exe}
\sn \textcolor{darkblue}{\textbf{\ipa{dʑɯ˩ ɖæ˩˥ / dʑɯ˩ ɖæ˩-ze˥}}}
\trans passer une rivière
\trans \textit{\textcolor{brown}{\zh{渡河}}}
\end{exe}
\begin{exe}
\sn \textcolor{darkblue}{\textbf{\ipa{dʑɯ˧ | ɖɯ˧-kʰɯ˩ ɖæ˩}}}
\trans idem
\trans \textit{\textcolor{brown}{\zh{同上}}}
\end{exe}


\paragraph{\hspace{-0.5cm} {\Large \textcolor{darkblue}{\textbf{\ipa{ɖæ˩dʑɯ˥}}}}} \hypertarget{d`\{\string_Bdz£M\string_T1}{}
\markboth{\textcolor{darkblue}{\textbf{\ipa{ɖæ˩dʑɯ˥}}}}{}
\textcolor{teal}{\mytextsc{nom}} \hspace{4pt} Ton~: LH.

Détritus, saletés, crasse.
\textcolor{brown}{\zh{污垢。}}


 \mytextsc{clf}~: \textcolor{darkblue}{\textbf{\ipa{ʁwɤ˧, etc}}} 

\paragraph{\hspace{-0.5cm} {\Large \textcolor{darkblue}{\textbf{\ipa{ɖæ˩-lɑ˧so˧}}}}} \hypertarget{d`\{\string_B-lA\string_Mso\string_M1}{}
\markboth{\textcolor{darkblue}{\textbf{\ipa{ɖæ˩-lɑ˧so˧}}}}{}
\textcolor{teal}{\mytextsc{nom}} \hspace{4pt} Ton~: L-.

Nom de cérémonie que les moines (un ou deux) pratiquent une fois par an (pendant la première quinzaine de l'année) au domicile de la personne qui les y invite: offrande de céréales (ou de fruits) aux divinités. L'objet de la cérémonie est d'assurer la prospérité financière de la maisonnée.
\textcolor{brown}{\zh{一种祈福仪式,和尚在过年时主持行礼。}}

 Emprunt~: tibétain 'braslhagsol



\paragraph{\hspace{-0.5cm} {\Large \textcolor{darkblue}{\textbf{\ipa{ɖæ˩mi˧}}}}} \hypertarget{d`\{\string_Bmi\string_M1}{}
\markboth{\textcolor{darkblue}{\textbf{\ipa{ɖæ˩mi˧}}}}{}
\textcolor{teal}{\mytextsc{nom}} \hspace{4pt} Ton~: LM.

Nom du temple de Yongning.
\textcolor{brown}{\zh{永宁大寺。}}

 Emprunt~: tibétain dgramed

\begin{exe}
\sn \textcolor{darkblue}{\textbf{\ipa{ɖæ˩mi˧-ʈæ˩bɤ˩}}}
\trans un prêtre du monastère
\trans \textit{\textcolor{brown}{\zh{永宁大寺的和尚}}}
\end{exe}


\paragraph{\hspace{-0.5cm} {\Large \textcolor{darkblue}{\textbf{\ipa{ɖæ˩mi˧-go˧bɤ˩}}}}} \hypertarget{d`\{\string_Bmi\string_M-go\string_Mb7\string_B1}{}
\markboth{\textcolor{darkblue}{\textbf{\ipa{ɖæ˩mi˧-go˧bɤ˩}}}}{}
\textcolor{teal}{\mytextsc{nom}} \hspace{4pt} Ton~: LM-L\#.

Le temple de Yongning.
\textcolor{brown}{\zh{永宁大寺。}}

 Emprunt~: tibétain dgrameddgonpa



\paragraph{\hspace{-0.5cm} {\Large \textcolor{darkblue}{\textbf{\ipa{ɖæ˩pʰv̩˥}}}}} \hypertarget{d`\{\string_Bp\string_hv\string_=\string_T1}{}
\markboth{\textcolor{darkblue}{\textbf{\ipa{ɖæ˩pʰv̩˥}}}}{}
\textcolor{teal}{\mytextsc{nom}} \hspace{4pt} Ton~: LH.

Poussière.
\textcolor{brown}{\zh{灰尘。}}


 \mytextsc{clf}~: \textcolor{darkblue}{\textbf{\ipa{ti˧˥}}} 

\paragraph{\hspace{-0.5cm} {\Large \textcolor{darkblue}{\textbf{\ipa{ɖæ˩ʂɯ\#˥}}}}} \hypertarget{d`\{\string_Bs`M\#\string_T1}{}
\markboth{\textcolor{darkblue}{\textbf{\ipa{ɖæ˩ʂɯ\#˥}}}}{}
\textcolor{teal}{\mytextsc{nom}} \hspace{4pt} Ton~: LM+\#H.

Nom de village.
\textcolor{brown}{\zh{扎实(永宁的一个村落)。}}


\begin{exe}
\sn \textcolor{darkblue}{\textbf{\ipa{ɖæ˩ʂɯ˧-ʁwɤ\#˥}}}
\trans même sens
\trans \textit{\textcolor{brown}{\zh{同上:扎实村}}}
\end{exe}
\begin{exe}
\sn \textcolor{darkblue}{\textbf{\ipa{ɖæ˩ʂɯ\#˥, | ʈʂo˧ʂɯ\#˥, | bɤ˩tɕʰɯ˩˥, | dɑ˧pʰo˥, | bɤ˧dzi˩, | dze˧bo˧}}}
\trans les six villages de la plaine de Yongning, dans l'ordre, qui prend comme point d'origine le village le plus proche du Lac
\trans \textit{\textcolor{brown}{\zh{永宁坝的六个村落,按传统排序:从距离泸沽湖最近的村落说起。}}}
\end{exe}


\paragraph{\hspace{-0.5cm} {\Large \textcolor{darkblue}{\textbf{\ipa{ɖæ˩˧}}}}} \hypertarget{d`\{\string_B\string_M1}{}
\markboth{\textcolor{darkblue}{\textbf{\ipa{ɖæ˩˧}}}}{}
\textcolor{teal}{\mytextsc{nom}} \hspace{4pt} Ton~: LM.





\ding{202} 
Poussière.
\textcolor{brown}{\zh{尘土。}}


\begin{exe}
\sn \textcolor{darkblue}{\textbf{\ipa{ɖæ˩˥ | ɖɯ˧-ti˧ tʰi˧-di˥}}}
\trans il y a une couche de poussière
\trans \textit{\textcolor{brown}{\zh{有一层灰}}}
\end{exe}
\begin{exe}
\sn \textcolor{darkblue}{\textbf{\ipa{ɖæ˩ bæ˧}}}
\trans balayer la poussière
\trans \textit{\textcolor{brown}{\zh{扫灰}}}
\end{exe}
\ding{203} 
Saletés.
\textcolor{brown}{\zh{污垢。}}


 \mytextsc{clf}~: \textcolor{darkblue}{\textbf{\ipa{ti˧˥}}} couche

\paragraph{\hspace{-0.5cm} {\Large \textcolor{darkblue}{\textbf{\ipa{ɖɤ˥}}}}} \hypertarget{d`7\string_T1}{}
\markboth{\textcolor{darkblue}{\textbf{\ipa{ɖɤ˥}}}}{}
\textcolor{teal}{\mytextsc{verbe}} \hspace{4pt} Ton~: H.

Ramper.
\textcolor{brown}{\zh{爬行,匍匐。}}


\begin{exe}
\sn \textcolor{darkblue}{\textbf{\ipa{ɖɤ˧\textasciitilde{}ɖɤ˧ (-ze˩)}}}
\trans \mytextsc{red}
\trans \textit{\textcolor{brown}{\zh{\mytextsc{重叠:爬一爬}}}}
\end{exe}
\begin{exe}
\sn \textcolor{darkblue}{\textbf{\ipa{ʈʂʰɯ˧ | ɖɤ˧\textasciitilde{}ɖɤ˧-ʁo˧-ze˩!}}}
\trans Elle sait ramper! (Au sujet d'un bébé qui se déplace en rampant.)
\trans \textit{\textcolor{brown}{\zh{她会爬了!}}}
\end{exe}


\paragraph{\hspace{-0.5cm} {\Large \textcolor{darkblue}{\textbf{\ipa{ɖɤ˧mi˧}}}}} \hypertarget{d`7\string_Mmi\string_M1}{}
\markboth{\textcolor{darkblue}{\textbf{\ipa{ɖɤ˧mi˧}}}}{}
\textcolor{teal}{\mytextsc{nom}} \hspace{4pt} Ton~: M.

Renard.
\textcolor{brown}{\zh{狐狸。}}


\begin{exe}
\sn \textcolor{darkblue}{\textbf{\ipa{ɖɤ˧mi˧-zo\#˥}}}
\trans petit renard, renardeau
\trans \textit{\textcolor{brown}{\zh{小狐狸}}}
\end{exe}
\begin{exe}
\sn \textcolor{darkblue}{\textbf{\ipa{ɖɤ˧mi˧-pʰv̩\#˥}}}
\trans renard mâle
\trans \textit{\textcolor{brown}{\zh{公狐狸}}}
\end{exe}
\begin{exe}
\sn \textcolor{darkblue}{\textbf{\ipa{ɖɤ˧mi˧, | mi˩ ɲi˥!}}}
\trans Ce renard, c'est une femelle!
\trans \textit{\textcolor{brown}{\zh{这只狐狸是母的!}}}
\end{exe}
 \mytextsc{clf}~: \textcolor{darkblue}{\textbf{\ipa{pʰo˧˥}}} 

\paragraph{\hspace{-0.5cm} {\Large \textcolor{darkblue}{\textbf{\ipa{ɖɤ˩\textsubscript{a}}}}}} \hypertarget{d`7\string_Ba1}{}
\markboth{\textcolor{darkblue}{\textbf{\ipa{ɖɤ˩\textsubscript{a}}}}}{}
\textcolor{teal}{\mytextsc{adjectif}} \hspace{4pt} Ton~: L\textsubscript{a}.

Brûlant, ardent (soleil), chaud (temps).
\textcolor{brown}{\zh{很热(天气),阳光强烈。}}


\begin{exe}
\sn \textcolor{darkblue}{\textbf{\ipa{ɲi˧mi˧ | ɖɤ˩-ze˥!}}}
\trans le soleil est torride, le soleil est très vif
\trans \textit{\textcolor{brown}{\zh{太阳很大、很强烈}}}
\end{exe}
\begin{exe}
\sn \textcolor{darkblue}{\textbf{\ipa{ɖɤ˩-hĩ˩˥}}}
\trans \mytextsc{rel}
\trans \textit{\textcolor{brown}{\zh{热的}}}
\end{exe}


\paragraph{\hspace{-0.5cm} {\Large \textcolor{darkblue}{\textbf{\ipa{ɖo˧}}}}} \hypertarget{d`o\string_M1}{}
\markboth{\textcolor{darkblue}{\textbf{\ipa{ɖo˧}}}}{}
\textcolor{teal}{\mytextsc{verbe}} \hspace{4pt} Ton~: M intrans.

Devoir, être obligé de; permettre, autoriser, accorder; ordonner, donner un ordre.
\textcolor{brown}{\zh{让,指使、使唤。}}


\begin{exe}
\sn \textcolor{darkblue}{\textbf{\ipa{po˧ mɤ˧-ɖo˧!}}}
\trans (Tu n'as) pas le droit de le prendre! (ex.: un enfant n'est pas autorisé à jouer avec un couteau)
\trans \textit{\textcolor{brown}{\zh{不许拿!}}}
\end{exe}
\begin{exe}
\sn \textcolor{darkblue}{\textbf{\ipa{ʈʂʰɯ˧, | po˧ ɖo˧!}}}
\trans Ca, (tu) peux le prendre / tu peux jouer avec! (Même contexte que ci-dessus: on indique à un enfant ce qu'on a le droit de prendre, et ce qu'on n'a pas le droit de prendre.)
\trans \textit{\textcolor{brown}{\zh{那个,是可以拿的! / 那个,是可以玩的!(情景同上:告诉一个小孩子什么东西可以拿,什么不可以拿。)}}}
\end{exe}
\begin{exe}
\sn \textcolor{darkblue}{\textbf{\ipa{gɤ˩ do˧ mɤ˧-ɖo˧!}}}
\trans (tu) n'as pas le droit de grimper/monter sur (une table…)
\trans \textit{\textcolor{brown}{\zh{不许爬上(桌子……)}}}
\end{exe}
\begin{exe}
\sn \textcolor{darkblue}{\textbf{\ipa{lɑ˧-kʰv̩˧˥, | ʑi˧qʰwɤ˧ tsʰi˧-mɤ˧-ɖo˧! | ʑi˩-kʰv̩˩˥, | ʑi˧qʰwɤ˧ tsʰi˧-mɤ˧-ɖo˧! |}}}
\trans L'année du Tigre, l'année du Singe, il ne faut pas construire de maison/il ne faut pas se lancer dans la construction d'une maison! (Ce sont des années trop ``dures", \textcolor{darkblue}{\textbf{\ipa{/wu˧/}}}, selon l'astrologie traditionnelle)
\trans \textit{\textcolor{brown}{\zh{虎年,不要建房!猴年,不要建房!(这样的年,被认为是太‘硬’的。)}}}
\end{exe}
\begin{exe}
\sn \textcolor{darkblue}{\textbf{\ipa{ʝi˧ mɤ˧-ɖo˧!}}}
\trans Il ne faut pas faire (ça)!
\trans \textit{\textcolor{brown}{\zh{不要做!}}}
\end{exe}


\paragraph{\hspace{-0.5cm} {\Large \textcolor{darkblue}{\textbf{\ipa{}}}}} \hypertarget{d`M\string_M-1}{}
\markboth{\textcolor{darkblue}{\textbf{\ipa{}}}}{}
\textcolor{teal}{\mytextsc{préposition}} \hspace{4pt} Ton~: M.

\mytextsc{délimitatif}.
\textcolor{brown}{\zh{\mytextsc{进行时态。}}}




\paragraph{\hspace{-0.5cm} {\Large \textcolor{darkblue}{\textbf{\ipa{ɖɯ˧\textsubscript{b}}}}}} \hypertarget{d`M\string_Mb1}{}
\markboth{\textcolor{darkblue}{\textbf{\ipa{ɖɯ˧\textsubscript{b}}}}}{}
\textcolor{teal}{\mytextsc{verbe}} \hspace{4pt} Ton~: M\textsubscript{b}.

Obtenir, trouver.
\textcolor{brown}{\zh{得到。}}


\begin{exe}
\sn \textcolor{darkblue}{\textbf{\ipa{le˧-ʂe˧ le˧-ɖɯ˧-ze˧!}}}
\trans j'ai cherché et je l'ai trouvé! j'ai trouvé (en cherchant)!
\trans \textit{\textcolor{brown}{\zh{找到了!}}}
\end{exe}
\begin{exe}
\sn \textcolor{darkblue}{\textbf{\ipa{ɖɯ˧-tʰɑ˧˥!}}}
\trans On peut obtenir!
\trans \textit{\textcolor{brown}{\zh{可以得到的!}}}
\end{exe}
\begin{exe}
\sn \textcolor{darkblue}{\textbf{\ipa{ɖɯ˧-tʰɑ˧-ze˥!}}}
\trans On a réussi à obtenir!
\trans \textit{\textcolor{brown}{\zh{(我们)成功地得到了!}}}
\end{exe}
\begin{exe}
\sn \textcolor{darkblue}{\textbf{\ipa{tso˧\textasciitilde{}tso˧ ɖɯ˧ (+ze˧)}}}
\trans obtenir quelque chose
\trans \textit{\textcolor{brown}{\zh{得到东西}}}
\end{exe}


\paragraph{\hspace{-0.5cm} {\Large \textcolor{darkblue}{\textbf{\ipa{ɖɯ˧-ɬi˧mi˧}}}}} \hypertarget{d`M\string_M-Ki\string_Mmi\string_M1}{}
\markboth{\textcolor{darkblue}{\textbf{\ipa{ɖɯ˧-ɬi˧mi˧}}}}{}
\textcolor{teal}{\mytextsc{nom}} \hspace{4pt} Ton~: M.

1er mois.
\textcolor{brown}{\zh{正月。}}




\paragraph{\hspace{-0.5cm} {\Large \textcolor{darkblue}{\textbf{\ipa{ɖɯ˧-njɤ˧}}}}} \hypertarget{d`M\string_M-nj7\string_M1}{}
\markboth{\textcolor{darkblue}{\textbf{\ipa{ɖɯ˧-njɤ˧}}}}{}
\textcolor{teal}{\mytextsc{adverbe}} \hspace{4pt} Ton~: M.

Sans cesse; sans arrêt; toujours.
\textcolor{brown}{\zh{一直、一直不停。}}


\begin{exe}
\sn \textcolor{darkblue}{\textbf{\ipa{ɖɯ˧-njɤ˧ | so˩˥}}}
\trans étudier sans arrêt
\trans \textit{\textcolor{brown}{\zh{一直不停地学习}}}
\end{exe}
\begin{exe}
\sn \textcolor{darkblue}{\textbf{\ipa{ɖɯ˧-njɤ˧ | lo˧ ʝi˧}}}
\trans travailler sans arrêt
\trans \textit{\textcolor{brown}{\zh{一直不停地工作}}}
\end{exe}
\begin{exe}
\sn \textcolor{darkblue}{\textbf{\ipa{ɖɯ˧-njɤ˧-zo˥}}}
\trans souvent
\trans \textit{\textcolor{brown}{\zh{经常、常}}}
\end{exe}
\begin{exe}
\sn \textcolor{darkblue}{\textbf{\ipa{ɖɯ˧-njɤ˧ hwæ˩; ɖɯ˧-njɤ˧ tɕʰi˧; ɖɯ˧-njɤ˧ dzɯ˧; ɖɯ˧-njɤ˧ dze˧˥; ɖɯ˧-njɤ˧ ʐwɤ˧˥; ɖɯ˧-njɤ˧ lɑ˧˥}}}
\trans avec des verbes aux six tons, pour étudier les tons : acheter; vendre; manger; couper; parler; frapper
\trans \textit{\textcolor{brown}{\zh{跟六个调类的动词结合:买,卖,吃,切,说,打}}}
\end{exe}
\begin{exe}
\sn \textcolor{darkblue}{\textbf{\ipa{ɖɯ˧-njɤ˧ | hwæ˧; ɖɯ˧-njɤ˧ | tɕʰi˧; ɖɯ˧-njɤ˧ | dzɯ˧; ɖɯ˧-njɤ˧ | dze˩˥; ɖɯ˧-njɤ˧ | ʐwɤ˩˥; ɖɯ˧-njɤ˧ | lɑ˧˥}}}
\trans avec des verbes aux six tons, pour étudier les tons : acheter; vendre; manger; couper; parler; frapper; en séparant en groupes tonals
\trans \textit{\textcolor{brown}{\zh{跟六个调类的动词结合:买,卖,吃,切,说,打}}}
\end{exe}


\paragraph{\hspace{-0.5cm} {\Large \textcolor{darkblue}{\textbf{\ipa{ɖɯ˧-ɲi˧-ɖɯ˥-hɑ̃˩}}}}} \hypertarget{d`M\string_M-Ji\string_M-d`M\string_T-hA\string_~\string_B1}{}
\markboth{\textcolor{darkblue}{\textbf{\ipa{ɖɯ˧-ɲi˧-ɖɯ˥-hɑ̃˩}}}}{}
\textcolor{teal}{\mytextsc{nom}} \hspace{4pt} Ton~: \#H-.

Un jour et une nuit.
\textcolor{brown}{\zh{一天一夜。}}




\paragraph{\hspace{-0.5cm} {\Large \textcolor{darkblue}{\textbf{\ipa{ɖɯ˧-so˩}}}}} \hypertarget{d`M\string_M-so\string_B1}{}
\markboth{\textcolor{darkblue}{\textbf{\ipa{ɖɯ˧-so˩}}}}{}
\textcolor{teal}{\mytextsc{nom}} \hspace{4pt} Ton~: L\textsubscript{a}.

Un petit nombre de, quelques-uns.
\textcolor{brown}{\zh{一些、两三个(直译:‘一三(个)’。}}


\begin{exe}
\sn \textcolor{darkblue}{\textbf{\ipa{hĩ˧ | ɖɯ˧-so˩ kv̩˩}}}
\trans quelques personnes (deux, trois...)
\trans \textit{\textcolor{brown}{\zh{几个人}}}
\end{exe}
\begin{exe}
\sn \textcolor{darkblue}{\textbf{\ipa{ɖɯ˧-so˩ ɲi˩}}}
\trans quelques jours
\trans \textit{\textcolor{brown}{\zh{几天}}}
\end{exe}


\paragraph{\hspace{-0.5cm} {\Large \textcolor{darkblue}{\textbf{\ipa{ɖɯ˧ʈæ˩}}}}} \hypertarget{d`M\string_Mt`\{\string_B1}{}
\markboth{\textcolor{darkblue}{\textbf{\ipa{ɖɯ˧ʈæ˩}}}}{}
\textcolor{teal}{\mytextsc{nom}} \hspace{4pt} Ton~: L\#.

Rite pratiqué par les moines du monastère après un décès.
\textcolor{brown}{\zh{一个葬礼仪式,由和尚主持。}}




\paragraph{\hspace{-0.5cm} {\Large \textcolor{darkblue}{\textbf{\ipa{ɖɯ˩\textsubscript{a}}}}}} \hypertarget{d`M\string_Ba1}{}
\markboth{\textcolor{darkblue}{\textbf{\ipa{ɖɯ˩\textsubscript{a}}}}}{}
\textcolor{teal}{\mytextsc{adjectif}} \hspace{4pt} Ton~: L\textsubscript{a}.

Grand.
\textcolor{brown}{\zh{大。}}


\begin{exe}
\sn \textcolor{darkblue}{\textbf{\ipa{mɤ˧-ɖɯ˩}}}
\trans \mytextsc{neg}
\trans \textit{\textcolor{brown}{\zh{\mytextsc{neg}}}}
\end{exe}
\begin{exe}
\sn \textcolor{darkblue}{\textbf{\ipa{ɖɯ˩-hĩ˩˥}}}
\trans \mytextsc{rel}
\trans \textit{\textcolor{brown}{\zh{\mytextsc{rel}}}}
\end{exe}
\begin{exe}
\sn \textcolor{darkblue}{\textbf{\ipa{le˧-ɖɯ˩(-ze˩)}}}
\trans \mytextsc{accomp}+\mytextsc{pfv}: ça a grandi!/ il/elle a grandi!
\trans \textit{\textcolor{brown}{\zh{\mytextsc{accomp}+\mytextsc{pfv}}}}
\end{exe}
\begin{exe}
\sn \textcolor{darkblue}{\textbf{\ipa{ə˧pɤ˥ɖɯ˩-gv̩˩}}}
\trans très grand
\trans \textit{\textcolor{brown}{\zh{好大、大得很}}}
\end{exe}


\paragraph{\hspace{-0.5cm} {\Large \textcolor{darkblue}{\textbf{\ipa{ɖɯ˩ɖʐɯ˧}}}}} \hypertarget{d`M\string_Bd`z`M\string_M1}{}
\markboth{\textcolor{darkblue}{\textbf{\ipa{ɖɯ˩ɖʐɯ˧}}}}{}
\textcolor{teal}{\mytextsc{nom}} \hspace{4pt} Ton~: LM.

Prénom masculin.
\textcolor{brown}{\zh{男性名字:独知。}}

 Emprunt~: tibétain



\paragraph{\hspace{-0.5cm} {\Large \textcolor{darkblue}{\textbf{\ipa{ɖɯ˩ɖʐɯ˧-tsʰɯ˩ɻ̍˩}}}}} \hypertarget{d`M\string_Bd`z`M\string_M-ts\string_hM\string_Br£`̍\string_B1}{}
\markboth{\textcolor{darkblue}{\textbf{\ipa{ɖɯ˩ɖʐɯ˧-tsʰɯ˩ɻ̍˩}}}}{}
\textcolor{teal}{\mytextsc{nom}} \hspace{4pt} Ton~: LM-L.

Prénom masculin.
\textcolor{brown}{\zh{男性名字。}}




\paragraph{\hspace{-0.5cm} {\Large \textcolor{darkblue}{\textbf{\ipa{ɖɯ˩hĩ˩}}}}} \hypertarget{d`M\string_Bhi\string_~\string_B1}{}
\markboth{\textcolor{darkblue}{\textbf{\ipa{ɖɯ˩hĩ˩}}}}{}
\textcolor{teal}{\mytextsc{nom}} \hspace{4pt} Ton~: L.

Gens importants.
\textcolor{brown}{\zh{大人、重要的人(包括长辈)。}}


 \mytextsc{clf}~: \textcolor{darkblue}{\textbf{\ipa{v̩˧}}} 

\paragraph{\hspace{-0.5cm} {\Large \textcolor{darkblue}{\textbf{\ipa{ɖɯ˩lo\#˥}}}}} \hypertarget{d`M\string_Blo\#\string_T1}{}
\markboth{\textcolor{darkblue}{\textbf{\ipa{ɖɯ˩lo\#˥}}}}{}
\textcolor{teal}{\mytextsc{nom}} \hspace{4pt} Ton~: LM+\#H.





\ding{202} 
Coutume, tradition.
\textcolor{brown}{\zh{传统。}}


\begin{exe}
\sn \textcolor{darkblue}{\textbf{\ipa{ɖɯ˩lo˧ ɖɯ˧-kʰwɤ˥ | tʰi˧-so˥-ɻ̍˩}}}
\trans enseigner une coutume
\trans \textit{\textcolor{brown}{\zh{教授一个传统、一个习俗}}}
\end{exe}
\ding{203} 
Savoir-vivre.
\textcolor{brown}{\zh{礼仪、礼貌。}}


\begin{exe}
\sn \textcolor{darkblue}{\textbf{\ipa{ʈʂʰɯ˧ | ɖɯ˩lo˧ dʑɤ˥!}}}
\trans Il/elle sait vivre/connaît les coutumes/a du savoir-vivre!
\trans \textit{\textcolor{brown}{\zh{他懂礼貌、他会做人}}}
\end{exe}
\ding{204} 
Ordre des choses.
\textcolor{brown}{\zh{道理。}}


 \mytextsc{clf}~: \textcolor{darkblue}{\textbf{\ipa{kʰwɤ˥}}} 

\paragraph{\hspace{-0.5cm} {\Large \textcolor{darkblue}{\textbf{\ipa{ɖɯ˩mɑ\#˥}}}}} \hypertarget{d`M\string_BmA\#\string_T1}{}
\markboth{\textcolor{darkblue}{\textbf{\ipa{ɖɯ˩mɑ\#˥}}}}{}
\textcolor{teal}{\mytextsc{nom}} \hspace{4pt} Ton~: LM+\#H.

Prénom féminin.
\textcolor{brown}{\zh{女性名字。}}




\paragraph{\hspace{-0.5cm} {\Large \textcolor{darkblue}{\textbf{\ipa{ɖɯ˩mɑ˧-ɬɑ˩tsʰo˩}}}}} \hypertarget{d`M\string_BmA\string_M-KA\string_Bts\string_ho\string_B1}{}
\markboth{\textcolor{darkblue}{\textbf{\ipa{ɖɯ˩mɑ˧-ɬɑ˩tsʰo˩}}}}{}
\textcolor{teal}{\mytextsc{nom}} \hspace{4pt} Ton~: LM-L.

Prénom féminin.
\textcolor{brown}{\zh{女性名字。}}




\paragraph{\hspace{-0.5cm} {\Large \textcolor{darkblue}{\textbf{\ipa{ɖɯ˩mɑ˧-pv̩˩ʈʰɯ˩}}}}} \hypertarget{d`M\string_BmA\string_M-pv\string_=\string_Bt`\string_hM\string_B1}{}
\markboth{\textcolor{darkblue}{\textbf{\ipa{ɖɯ˩mɑ˧-pv̩˩ʈʰɯ˩}}}}{}
\textcolor{teal}{\mytextsc{nom}} \hspace{4pt} Ton~: LM-L.

Prénom féminin.
\textcolor{brown}{\zh{女性名字。}}




\paragraph{\hspace{-0.5cm} {\Large \textcolor{darkblue}{\textbf{\ipa{ɖɯ˩mi\#˥}}}}} \hypertarget{d`M\string_Bmi\#\string_T1}{}
\markboth{\textcolor{darkblue}{\textbf{\ipa{ɖɯ˩mi\#˥}}}}{}
\textcolor{teal}{\mytextsc{nom}} \hspace{4pt} Ton~: LM+\#H.

Mule femelle. C'est un animal stérile. Il est docile et fort, et peut accomplir des tâches importantes comme d'être l'animal de tête dans une caravane. C'est donc un animal de grand prix.
\textcolor{brown}{\zh{母骡子、母马骡。}}


\begin{exe}
\sn \textcolor{darkblue}{\textbf{\ipa{ɖɯ˩mi˧-ɖɯ˥zo˩}}}
\trans mule femelle et mule mâle
\trans \textit{\textcolor{brown}{\zh{母骡子与公骡子}}}
\end{exe}
 \mytextsc{clf}~: \textcolor{darkblue}{\textbf{\ipa{mi˩}}} 

\paragraph{\hspace{-0.5cm} {\Large \textcolor{darkblue}{\textbf{\ipa{ɖɯ˩zo\#˥}}}}} \hypertarget{d`M\string_Bzo\#\string_T1}{}
\markboth{\textcolor{darkblue}{\textbf{\ipa{ɖɯ˩zo\#˥}}}}{}
\textcolor{teal}{\mytextsc{nom}} \hspace{4pt} Ton~: LM+\#H.

Mule mâle (animal moins prisé que la femelle).
\textcolor{brown}{\zh{公骡子。}}


\begin{exe}
\sn \textcolor{darkblue}{\textbf{\ipa{ɖɯ˩zo˧-ɖɯ˥mi˩}}}
\trans mule mâle et mule femelle
\trans \textit{\textcolor{brown}{\zh{公骡子与母骡子}}}
\end{exe}
 \mytextsc{clf}~: \textcolor{darkblue}{\textbf{\ipa{ɭɯ˧}}} 

\paragraph{\hspace{-0.5cm} {\Large \textcolor{darkblue}{\textbf{\ipa{ɖɯ˧˥}}}}} \hypertarget{d`M\string_M\string_T1}{}
\markboth{\textcolor{darkblue}{\textbf{\ipa{ɖɯ˧˥}}}}{}
\textcolor{teal}{\mytextsc{nombre}} \hspace{4pt} Ton~: MH.

1.
\textcolor{brown}{\zh{1。}}




\paragraph{\hspace{-0.5cm} {\Large \textcolor{darkblue}{\textbf{\ipa{ɖv̩˩}}}}} \hypertarget{d`v\string_=\string_B1}{}
\markboth{\textcolor{darkblue}{\textbf{\ipa{ɖv̩˩}}}}{}
\textcolor{teal}{\mytextsc{nom}} \hspace{4pt} Ton~: L.

Ailes (forme monosyllabique; la forme disyllabique est préférée).
\textcolor{brown}{\zh{翅膀。}}


 \mytextsc{clf}~: \textcolor{darkblue}{\textbf{\ipa{dze˩}}} paire

\paragraph{\hspace{-0.5cm} {\Large \textcolor{darkblue}{\textbf{\ipa{ɖv̩˧qæ˧}}}}} \hypertarget{d`v\string_=\string_Mq\{\string_M1}{}
\markboth{\textcolor{darkblue}{\textbf{\ipa{ɖv̩˧qæ˧}}}}{}
\textcolor{teal}{\mytextsc{nom}} \hspace{4pt} Ton~: M.

Ailes.
\textcolor{brown}{\zh{翅膀。}}


\begin{exe}
\sn \textcolor{darkblue}{\textbf{\ipa{kɤ˩nɑ˧mi˧-ɖv̩˧qæ˥}}}
\trans aile d'aigle
\trans \textit{\textcolor{brown}{\zh{老鹰翅膀}}}
\end{exe}
 \mytextsc{clf}~: \textcolor{darkblue}{\textbf{\ipa{dze˩}}} paire

\paragraph{\hspace{-0.5cm} {\Large \textcolor{darkblue}{\textbf{\ipa{ɖwæ˥}}}}} \hypertarget{d`w\{\string_T1}{}
\markboth{\textcolor{darkblue}{\textbf{\ipa{ɖwæ˥}}}}{}
\textcolor{teal}{\mytextsc{nom}} \hspace{4pt} Ton~: \#H.
\ding{202} 
Mare.
\textcolor{brown}{\zh{池塘。}}


\begin{exe}
\sn \textcolor{darkblue}{\textbf{\ipa{[F5] ɖwæ˩ɬo˩mi˧}}}
\trans grand étang
\trans \textit{\textcolor{brown}{\zh{大池塘}}}
\end{exe}
\ding{203} 
Réserve d'eau (artificielle).
\textcolor{brown}{\zh{水坑。}}


 \mytextsc{clf}~: \textcolor{darkblue}{\textbf{\ipa{ɭɯ˧}}} objets ronds

\paragraph{\hspace{-0.5cm} {\Large \textcolor{darkblue}{\textbf{\ipa{ɖwæ˥}}}}} \hypertarget{d`w\{\string_T1}{}
\markboth{\textcolor{darkblue}{\textbf{\ipa{ɖwæ˥}}}}{}
\textcolor{teal}{\mytextsc{adjectif}} \hspace{4pt} Ton~: H.

Trouble (le même terme est employé pour l'eau vive et pour l'eau stagnante).
\textcolor{brown}{\zh{浑浊 (水)。}}


\begin{exe}
\sn \textcolor{darkblue}{\textbf{\ipa{dʑɯ˧ ɖwæ\#˥}}}
\trans eau trouble
\trans \textit{\textcolor{brown}{\zh{浑浊的水}}}
\end{exe}
\begin{exe}
\sn \textcolor{darkblue}{\textbf{\ipa{dʑɯ˧ | ɖwæ˧-ze˩!}}}
\trans l'eau s'est troublée! l'eau est devenue trouble!
\trans \textit{\textcolor{brown}{\zh{水浑浊了。}}}
\end{exe}


\paragraph{\hspace{-0.5cm} {\Large \textcolor{darkblue}{\textbf{\ipa{ɖwæ˥\textsubscript{a}}}}}} \hypertarget{d`w\{\string_Ta1}{}
\markboth{\textcolor{darkblue}{\textbf{\ipa{ɖwæ˥\textsubscript{a}}}}}{}
\textcolor{teal}{\mytextsc{classificateur}} \hspace{4pt} Ton~: H\textsubscript{a}.

Classificateur des marches d'escalier.
\textcolor{brown}{\zh{量词:梯级、楼梯(一节)。}}


\begin{exe}
\sn \textcolor{darkblue}{\textbf{\ipa{ɖɯ˧-ɖwæ˧ ɲi˥}}}
\trans c'est une marche
\trans \textit{\textcolor{brown}{\zh{是一节/一节阶梯。(引出这句是为了了解这个词在不同语境的声调变化。)}}}
\end{exe}
\begin{exe}
\sn \textcolor{darkblue}{\textbf{\ipa{ʈʂʰɯ˧-ɖwæ\#˥}}}
\trans cette marche
\trans \textit{\textcolor{brown}{\zh{这节阶梯}}}
\end{exe}


\paragraph{\hspace{-0.5cm} {\Large \textcolor{darkblue}{\textbf{\ipa{ɖwæ˧-pɤ˧ɭɯ˥}}}}} \hypertarget{d`w\{\string_M-p7\string_Ml\string_RM\string_T1}{}
\markboth{\textcolor{darkblue}{\textbf{\ipa{ɖwæ˧-pɤ˧ɭɯ˥}}}}{}
\textcolor{teal}{\mytextsc{nom}} \hspace{4pt} Ton~: H\#.

Flaque (naturelle).
\textcolor{brown}{\zh{水潭。}}


\begin{exe}
\sn \textcolor{darkblue}{\textbf{\ipa{ɖwæ˧ tʰi˧-pɤ˥ɭɯ˩}}}
\trans une flaque/petite mare s'est formée, il y a une flaque
\trans \textit{\textcolor{brown}{\zh{有水潭}}}
\end{exe}
 \mytextsc{clf}~: \textcolor{darkblue}{\textbf{\ipa{ɭɯ˧}}} objets ronds

\paragraph{\hspace{-0.5cm} {\Large \textcolor{darkblue}{\textbf{\ipa{ɖwæ˩\textsubscript{a}}}}}} \hypertarget{d`w\{\string_Ba1}{}
\markboth{\textcolor{darkblue}{\textbf{\ipa{ɖwæ˩\textsubscript{a}}}}}{}
\textcolor{teal}{\mytextsc{verbe}} \hspace{4pt} Ton~: L\textsubscript{a}.

Avoir peur.
\textcolor{brown}{\zh{害怕。}}


\begin{exe}
\sn \textcolor{darkblue}{\textbf{\ipa{njɤ˧ | ɖwæ˩˥!}}}
\trans J'ai peur!
\trans \textit{\textcolor{brown}{\zh{我害怕!}}}
\end{exe}
\begin{exe}
\sn \textcolor{darkblue}{\textbf{\ipa{njɤ˧ | ʈʂʰɯ˧-v̩˧ | ɖwæ˩˥ | ʐwæ˩˥!}}}
\trans J'ai très peur de lui!
\trans \textit{\textcolor{brown}{\zh{我很害怕那个人!}}}
\end{exe}


\paragraph{\hspace{-0.5cm} {\Large \textcolor{darkblue}{\textbf{\ipa{ɖwæ˧˥}}} \textsubscript{1}}} \hypertarget{d`w\{\string_M\string_T1}{}
\markboth{\textcolor{darkblue}{\textbf{\ipa{ɖwæ˧˥}}} \textsubscript{1}}{}
\textcolor{teal}{\mytextsc{verbe}} \hspace{4pt} Ton~: MH.

Fouetter, donner des coups (ex.: un tigre fouette le sol avec sa queue).
\textcolor{brown}{\zh{鞭打、抽打、加鞭。}}


\begin{exe}
\sn \textcolor{darkblue}{\textbf{\ipa{mæ˧qv̩˩-po˩-ɳɯ˩ | ɖwæ˧˥}}}
\trans donner des coups de queue (ex.: le tigre fouette le sol de sa queue)
\trans \textit{\textcolor{brown}{\zh{用尾巴来抽打(如:老虎用尾巴来抽打地面)}}}
\end{exe}


\paragraph{\hspace{-0.5cm} {\Large \textcolor{darkblue}{\textbf{\ipa{ɖwæ˧˥}}} \textsubscript{2}}} \hypertarget{d`w\{\string_M\string_T2}{}
\markboth{\textcolor{darkblue}{\textbf{\ipa{ɖwæ˧˥}}} \textsubscript{2}}{}
\textcolor{teal}{\mytextsc{adverbe}} \hspace{4pt} Ton~: MH.

Intensif: très.
\textcolor{brown}{\zh{很、极。}}


\begin{exe}
\sn \textcolor{darkblue}{\textbf{\ipa{ʈʂʰɯ˧ | ɖwæ˧˥ | æ˧mv̩˩ fv̩˩!}}}
\trans elle aime beaucoup sa grande sœur!
\trans \textit{\textcolor{brown}{\zh{她很喜欢她姐姐!}}}
\end{exe}


\newpage
\section*{\centering- \textcolor{darkblue}{\textbf{\ipa{ɖʐ}}} -}
\paragraph{\hspace{-0.5cm} {\Large \textcolor{darkblue}{\textbf{\ipa{ɖʐæ˧\textsubscript{b}}}}}} \hypertarget{d`z`\{\string_Mb1}{}
\markboth{\textcolor{darkblue}{\textbf{\ipa{ɖʐæ˧\textsubscript{b}}}}}{}
\textcolor{teal}{\mytextsc{verbe}} \hspace{4pt} Ton~: M\textsubscript{b}.

Monter à cheval.
\textcolor{brown}{\zh{骑马。}}


\begin{exe}
\sn \textcolor{darkblue}{\textbf{\ipa{le˧-ɖʐæ˧-ze˧}}}
\trans \mytextsc{accomp} \string_ \mytextsc{pfv}
\trans \textit{\textcolor{brown}{\zh{\mytextsc{accomp} \string_ \mytextsc{pfv}}}}
\end{exe}
\begin{exe}
\sn \textcolor{darkblue}{\textbf{\ipa{ʐwæ˧ ɖʐæ˧}}}
\trans monter à cheval
\trans \textit{\textcolor{brown}{\zh{骑马}}}
\end{exe}
\begin{exe}
\sn \textcolor{darkblue}{\textbf{\ipa{ɖʐæ˧-tʰɑ˧˥!}}}
\trans On peut le monter!
\trans \textit{\textcolor{brown}{\zh{可以骑的!}}}
\end{exe}


\paragraph{\hspace{-0.5cm} {\Large \textcolor{darkblue}{\textbf{\ipa{ɖʐæ˧qʰæ˥\$}}}}} \hypertarget{d`z`\{\string_Mq\string_h\{\string_T\$1}{}
\markboth{\textcolor{darkblue}{\textbf{\ipa{ɖʐæ˧qʰæ˥\$}}}}{}
\textcolor{teal}{\mytextsc{nom}} \hspace{4pt} Ton~: H\$.

Boue.
\textcolor{brown}{\zh{泥巴。}}


\begin{exe}
\sn \textcolor{darkblue}{\textbf{\ipa{ɖʐæ˧qʰæ˧ ʐæ˥(-ze˩)}}}
\trans De la boue s'est formée; il y a de la boue, c'est tout boueux. (Littéralement: ``de la boue s'est mélangée".)
\trans \textit{\textcolor{brown}{\zh{有泥巴了。}}}
\end{exe}
\begin{exe}
\sn \textcolor{darkblue}{\textbf{\ipa{[F5] ɖʐæ˧qʰæ˧ ʐæ˧\textasciitilde{}ʐæ˥}}}
\trans De la boue s'est formée; il y a de la boue, c'est tout boueux. (Littéralement: ``de la boue s'est mélangée".)
\trans \textit{\textcolor{brown}{\zh{有泥巴了}}}
\end{exe}


\paragraph{\hspace{-0.5cm} {\Large \textcolor{darkblue}{\textbf{\ipa{ɖʐæ˩\textsubscript{a}}}}}} \hypertarget{d`z`\{\string_Ba1}{}
\markboth{\textcolor{darkblue}{\textbf{\ipa{ɖʐæ˩\textsubscript{a}}}}}{}
\textcolor{teal}{\mytextsc{verbe}} \hspace{4pt} Ton~: L\textsubscript{a}.

Fondre.
\textcolor{brown}{\zh{融化。}}


\begin{exe}
\sn \textcolor{darkblue}{\textbf{\ipa{mɤ˧ | le˧-ɖʐæ˩-ze˩}}}
\trans la graisse a fondu (ex.: du saindoux qui fond dans un chaudron)
\trans \textit{\textcolor{brown}{\zh{油融化了。}}}
\end{exe}
\begin{exe}
\sn \textcolor{darkblue}{\textbf{\ipa{dʑi˩pʰæ˩˥ | le˧-ɖʐæ˩-ze˩}}}
\trans La glace a fondu.
\trans \textit{\textcolor{brown}{\zh{冰融化了。}}}
\end{exe}


\paragraph{\hspace{-0.5cm} {\Large \textcolor{darkblue}{\textbf{\ipa{ɖʐæ˩bv˩}}}}} \hypertarget{d`z`\{\string_Bbv\string_B1}{}
\markboth{\textcolor{darkblue}{\textbf{\ipa{ɖʐæ˩bv˩}}}}{}
\textcolor{teal}{\mytextsc{nom}} \hspace{4pt} Ton~: L.

Sorcier.
\textcolor{brown}{\zh{法师。}}


\begin{exe}
\sn \textcolor{darkblue}{\textbf{\ipa{ə˧pʰv˧-ɖʐæ˩bv˩}}}
\trans 'Grand-père sorcier': terme d'adresse respectueux pour un sorcier d'âge avancé, ou considéré comme ayant des pouvoirs considérables
\trans \textit{\textcolor{brown}{\zh{‘法师爷爷’:对年龄高(或被认为本事很大)的法师的尊重称呼}}}
\end{exe}
\begin{exe}
\sn \textcolor{darkblue}{\textbf{\ipa{ə˧v˧-ɖʐæ˥bv˩}}}
\trans 'Oncle sorcier': terme d'adresse respectueux pour un sorcier
\trans \textit{\textcolor{brown}{\zh{‘法师舅舅’:对法师的尊重称呼}}}
\end{exe}
 \mytextsc{clf}~: \textcolor{darkblue}{\textbf{\ipa{v̩˧}}} 

\paragraph{\hspace{-0.5cm} {\Large \textcolor{darkblue}{\textbf{\ipa{ɖʐe˧}}}}} \hypertarget{d`z`e\string_M1}{}
\markboth{\textcolor{darkblue}{\textbf{\ipa{ɖʐe˧}}}}{}
\textcolor{teal}{\mytextsc{nom}} \hspace{4pt} Ton~: M.

Argent (avoir de l'argent).
\textcolor{brown}{\zh{钱。}}




\paragraph{\hspace{-0.5cm} {\Large \textcolor{darkblue}{\textbf{\ipa{ɖʐe˧gɯ˧}}}}} \hypertarget{d`z`e\string_MgM\string_M1}{}
\markboth{\textcolor{darkblue}{\textbf{\ipa{ɖʐe˧gɯ˧}}}}{}
\textcolor{teal}{\mytextsc{nom}} \hspace{4pt} Ton~: M.

Yongsheng (nom de comté).
\textcolor{brown}{\zh{永胜(地名)。}}


\begin{exe}
\sn \textcolor{darkblue}{\textbf{\ipa{ɖʐe˧gɯ˧-to˩mi˩}}}
\trans une haute montagne située à Yongsheng
\trans \textit{\textcolor{brown}{\zh{永胜的一座高山}}}
\end{exe}
\begin{exe}
\sn \textcolor{darkblue}{\textbf{\ipa{ɖʐe˧gɯ˧-hæ˧}}}
\trans Chinois de Yongsheng (note: le comté de Yongsheng était peuplé majoritairement de Chinois (Han).)
\trans \textit{\textcolor{brown}{\zh{永胜汉族}}}
\end{exe}
\begin{exe}
\sn \textcolor{darkblue}{\textbf{\ipa{ɖʐe˧gɯ˧-dʑo˧, | hæ˧-ʂo˧\textasciitilde{}ʂo˩!}}}
\trans A Yongsheng, il y a plein de Chinois (Han)!
\trans \textit{\textcolor{brown}{\zh{永胜,汉族群多!}}}
\end{exe}


\paragraph{\hspace{-0.5cm} {\Large \textcolor{darkblue}{\textbf{\ipa{ɖʐe˧ʁwɤ˧}}}}} \hypertarget{d`z`e\string_MRw7\string_M1}{}
\markboth{\textcolor{darkblue}{\textbf{\ipa{ɖʐe˧ʁwɤ˧}}}}{}
\textcolor{teal}{\mytextsc{nom}} \hspace{4pt} Ton~: M.

Argent (monnaie); richesse.
\textcolor{brown}{\zh{钱。}}




\paragraph{\hspace{-0.5cm} {\Large \textcolor{darkblue}{\textbf{\ipa{ɖʐɤ˧qʰwɤ˧}}}}} \hypertarget{d`z`7\string_Mq\string_hw7\string_M1}{}
\markboth{\textcolor{darkblue}{\textbf{\ipa{ɖʐɤ˧qʰwɤ˧}}}}{}
\textcolor{teal}{\mytextsc{nom}} \hspace{4pt} Ton~: M.

Rhume.
\textcolor{brown}{\zh{感冒。}}


\begin{exe}
\sn \textcolor{darkblue}{\textbf{\ipa{ɖʐɤ˧qʰwɤ˧ go˩}}}
\trans avoir un rhume, être enrhumé
\trans \textit{\textcolor{brown}{\zh{感冒}}}
\end{exe}
\begin{exe}
\sn \textcolor{darkblue}{\textbf{\ipa{ɖʐɤ˧qʰwɤ˧ mɤ˧-go˩}}}
\trans ...n'est pas enrhumé
\trans \textit{\textcolor{brown}{\zh{没感冒}}}
\end{exe}
 \mytextsc{clf}~: \textcolor{darkblue}{\textbf{\ipa{ʂɯ˩}}} fois

\paragraph{\hspace{-0.5cm} {\Large \textcolor{darkblue}{\textbf{\ipa{ɖʐɤ˧qʰwɤ˧ʈʂe\#˥}}}}} \hypertarget{d`z`7\string_Mq\string_hw7\string_Mt`s`e\#\string_T1}{}
\markboth{\textcolor{darkblue}{\textbf{\ipa{ɖʐɤ˧qʰwɤ˧ʈʂe\#˥}}}}{}
\textcolor{teal}{\mytextsc{nom}} \hspace{4pt} Ton~: \#H.

Poinçon, alène.
\textcolor{brown}{\zh{锥、锥子。}}


 \mytextsc{clf}~: \textcolor{darkblue}{\textbf{\ipa{ɭɯ˧}}} 

\paragraph{\hspace{-0.5cm} {\Large \textcolor{darkblue}{\textbf{\ipa{ɖʐɤ˩}}}}} \hypertarget{d`z`7\string_B1}{}
\markboth{\textcolor{darkblue}{\textbf{\ipa{ɖʐɤ˩}}}}{}
\textcolor{teal}{\mytextsc{nom}} \hspace{4pt} Ton~: L.
\ding{202} 
Échelle.
\textcolor{brown}{\zh{梯子。}}


\begin{exe}
\sn \textcolor{darkblue}{\textbf{\ipa{ɖʐɤ˩ do˧}}}
\trans gravir une échelle
\trans \textit{\textcolor{brown}{\zh{爬上一个梯子}}}
\end{exe}
\begin{exe}
\sn \textcolor{darkblue}{\textbf{\ipa{ɖʐɤ˧ | gɤ˩-do˧}}}
\trans même sens, avec ajout d'un directionnel: gravir une échelle
\trans \textit{\textcolor{brown}{\zh{爬上一个梯子}}}
\end{exe}
\ding{203} 
Escalier (en bois, sauf indication contraire).
\textcolor{brown}{\zh{楼梯。}}


\begin{exe}
\sn \textcolor{darkblue}{\textbf{\ipa{lv̩˧mi˧-ɖʐɤ˩ (+ɲi˩)}}}
\trans escalier en pierre
\trans \textit{\textcolor{brown}{\zh{石头楼梯}}}
\end{exe}
 \mytextsc{clf}~: \textcolor{darkblue}{\textbf{\ipa{pɤ˩}}} 

\paragraph{\hspace{-0.5cm} {\Large \textcolor{darkblue}{\textbf{\ipa{ɖʐɤ˩ɖwæ˩}}}}} \hypertarget{d`z`7\string_Bd`w\{\string_B1}{}
\markboth{\textcolor{darkblue}{\textbf{\ipa{ɖʐɤ˩ɖwæ˩}}}}{}
\textcolor{teal}{\mytextsc{nom}} \hspace{4pt} Ton~: L.

Marche d'escalier.
\textcolor{brown}{\zh{台阶。}}


\begin{exe}
\sn \textcolor{darkblue}{\textbf{\ipa{lv̩˧mi˧-ɖʐɤ˩ɖwæ˩}}}
\trans marche en pierre
\trans \textit{\textcolor{brown}{\zh{石头台阶}}}
\end{exe}
 \mytextsc{clf}~: \textcolor{darkblue}{\textbf{\ipa{ɖwæ˥}}} 

\paragraph{\hspace{-0.5cm} {\Large \textcolor{darkblue}{\textbf{\ipa{ɖʐɤ˩kɤ˥\$}}}}} \hypertarget{d`z`7\string_Bk7\string_T\$1}{}
\markboth{\textcolor{darkblue}{\textbf{\ipa{ɖʐɤ˩kɤ˥\$}}}}{}
\textcolor{teal}{\mytextsc{nom}} \hspace{4pt} Ton~: LM+H\$.

Nom de clan/famille étendue. Deux familles portent ce nom à Yongning.
\textcolor{brown}{\zh{一个姓。这个姓,永宁有两家。}}


\begin{exe}
\sn \textcolor{darkblue}{\textbf{\ipa{ɖʐɤ˩kɤ˧=ɻ̍˥\$}}}
\trans le clan \textcolor{darkblue}{\textbf{\ipa{/ɖʐɤ˩kɤ˥\$/}}}
\trans \textit{\textcolor{brown}{\zh{\textcolor{darkblue}{\textbf{\ipa{/ɖʐɤ˩kɤ˥\$/}}}家族}}}
\end{exe}


\paragraph{\hspace{-0.5cm} {\Large \textcolor{darkblue}{\textbf{\ipa{ɖʐɤ˧˥}}} \textsubscript{1}}} \hypertarget{d`z`7\string_M\string_T1}{}
\markboth{\textcolor{darkblue}{\textbf{\ipa{ɖʐɤ˧˥}}} \textsubscript{1}}{}
\textcolor{teal}{\mytextsc{verbe}} \hspace{4pt} Ton~: MH.
\ding{202} 
Cueillir (des fruits, des légumes); arracher (des mauvaises herbes).
\textcolor{brown}{\zh{摘(果子、蔬菜)、扯(草)。}}


\begin{exe}
\sn \textcolor{darkblue}{\textbf{\ipa{le˧-ɖʐɤ˧-po˥-jo˩!}}}
\trans cueille-m'en qq-unes!/cueilles-en et passe-les(-nous) par ici!
\trans \textit{\textcolor{brown}{\zh{(你)去给摘(一些)过来吧!}}}
\end{exe}
\begin{exe}
\sn \textcolor{darkblue}{\textbf{\ipa{v̩˩tsʰɤ˧ ɖʐɤ˥}}}
\trans cueillir des légumes
\trans \textit{\textcolor{brown}{\zh{摘蔬菜}}}
\end{exe}
\begin{exe}
\sn \textcolor{darkblue}{\textbf{\ipa{le˧-ɖʐɤ˧˥, | mv̩˩-tɕo˧ kwɤ˩}}}
\trans arracher et jeter (les mauvaises herbes)
\trans \textit{\textcolor{brown}{\zh{扯(荒草),扔掉}}}
\end{exe}
\ding{203} 
Déchirer, couper (fil); briser; broyer; détruire (une maison).
\textcolor{brown}{\zh{拆(线),拔,捣碎。}}


\begin{exe}
\sn \textcolor{darkblue}{\textbf{\ipa{le˧-ɖʐɤ˩\textasciitilde{}ɖʐɤ˩}}}
\trans \mytextsc{red}
\trans \textit{\textcolor{brown}{\zh{\mytextsc{重叠:拆拆}}}}
\end{exe}
\begin{exe}
\sn \textcolor{darkblue}{\textbf{\ipa{ʑi˧qʰwɤ˧ ɖʐɤ˧˥}}}
\trans détruire une maison, démolir une maison
\trans \textit{\textcolor{brown}{\zh{拆房子}}}
\end{exe}
\begin{exe}
\sn \textcolor{darkblue}{\textbf{\ipa{le˧-ɖʐɤ˧˥ | ɲi˧-gi˧ gv̩˧}}}
\trans déchirer en deux morceaux
\trans \textit{\textcolor{brown}{\zh{拆成两半}}}
\end{exe}


\paragraph{\hspace{-0.5cm} {\Large \textcolor{darkblue}{\textbf{\ipa{ɖʐɤ˧˥}}} \textsubscript{2}}} \hypertarget{d`z`7\string_M\string_T2}{}
\markboth{\textcolor{darkblue}{\textbf{\ipa{ɖʐɤ˧˥}}} \textsubscript{2}}{}
\textcolor{teal}{\mytextsc{verbe}} \hspace{4pt} Ton~: MH.

Déployer, ouvrir en soutenant; ex.: déployer la tente.
\textcolor{brown}{\zh{撑开(帐篷)。}}


\begin{exe}
\sn \textcolor{darkblue}{\textbf{\ipa{le˧-ɖʐɤ˩\textasciitilde{}ɖʐɤ˩}}}
\trans \mytextsc{red}
\trans \textit{\textcolor{brown}{\zh{\mytextsc{重叠}}}}
\end{exe}


\paragraph{\hspace{-0.5cm} {\Large \textcolor{darkblue}{\textbf{\ipa{ɖʐo˥}}}}} \hypertarget{d`z`o\string_T1}{}
\markboth{\textcolor{darkblue}{\textbf{\ipa{ɖʐo˥}}}}{}
\textcolor{teal}{\mytextsc{nom}} \hspace{4pt} Ton~: \#H.

Pièce de charpente carrée (côté: environ 18 cm), dans les parties du bâtiment qui n'ont pas de piliers: \textcolor{darkblue}{\textbf{\ipa{/gæ˩pʰæ˧/}}}, \textcolor{darkblue}{\textbf{\ipa{/mv̩˩pʰæ˧/}}}. Elles supportent la charpente. (M18 pense que ce terme désigne toute la structure du bâtiment.).
\textcolor{brown}{\zh{大梁。}}


 \mytextsc{clf}~: \textcolor{darkblue}{\textbf{\ipa{ɖʐo˥}}} 

\paragraph{\hspace{-0.5cm} {\Large \textcolor{darkblue}{\textbf{\ipa{ɖʐo˥}}}}} \hypertarget{d`z`o\string_T1}{}
\markboth{\textcolor{darkblue}{\textbf{\ipa{ɖʐo˥}}}}{}
\textcolor{teal}{\mytextsc{adjectif}} \hspace{4pt} Ton~: H.

Froid.
\textcolor{brown}{\zh{冷(天气……)。}}




\paragraph{\hspace{-0.5cm} {\Large \textcolor{darkblue}{\textbf{\ipa{ɖʐo˥\textsubscript{a}}}}}} \hypertarget{d`z`o\string_Ta1}{}
\markboth{\textcolor{darkblue}{\textbf{\ipa{ɖʐo˥\textsubscript{a}}}}}{}
\textcolor{teal}{\mytextsc{classificateur}} \hspace{4pt} Ton~: H\textsubscript{a}.

Classificateur des poutres.
\textcolor{brown}{\zh{量词:梁(一根)。}}




\paragraph{\hspace{-0.5cm} {\Large \textcolor{darkblue}{\textbf{\ipa{ɖʐo˩\textsubscript{b}}}}}} \hypertarget{d`z`o\string_Bb1}{}
\markboth{\textcolor{darkblue}{\textbf{\ipa{ɖʐo˩\textsubscript{b}}}}}{}
\textcolor{teal}{\mytextsc{verbe}} \hspace{4pt} Ton~: L\textsubscript{b}.

Écraser (au moulin; ou avec les dents).
\textcolor{brown}{\zh{弄碎(用牙齿、手磨)。}}


\begin{exe}
\sn \textcolor{darkblue}{\textbf{\ipa{ʈʂo˧ɭɯ˧ ɖʐo˧˥}}}
\trans écraser avec un moulin (/ʈʂu˧ɭɯ\#˥/: moulin)
\trans \textit{\textcolor{brown}{\zh{用手磨弄碎}}}
\end{exe}
\begin{exe}
\sn \textcolor{darkblue}{\textbf{\ipa{ɖɯ˧-kʰwɤ˧ ɖʐo˧˥}}}
\trans écraser un morceau (de quelque chose)
\trans \textit{\textcolor{brown}{\zh{弄碎一块}}}
\end{exe}
\begin{exe}
\sn \textcolor{darkblue}{\textbf{\ipa{ɖɯ˧-mɤ˩ ɖʐo˩}}}
\trans écraser un peu (de quelque chose)
\trans \textit{\textcolor{brown}{\zh{弄碎一点(东西)}}}
\end{exe}


\paragraph{\hspace{-0.5cm} {\Large \textcolor{darkblue}{\textbf{\ipa{ɖʐɯ˥}}}}} \hypertarget{d`z`M\string_T1}{}
\markboth{\textcolor{darkblue}{\textbf{\ipa{ɖʐɯ˥}}}}{}
\textcolor{teal}{\mytextsc{nom}} \hspace{4pt} Ton~: \#H.

Marché.
\textcolor{brown}{\zh{集市(圩场,街子)。}}


 \mytextsc{clf}~: \textcolor{darkblue}{\textbf{\ipa{ɖʐɯ˩}}} 

\paragraph{\hspace{-0.5cm} {\Large \textcolor{darkblue}{\textbf{\ipa{ɖʐɯ˥\textsubscript{a}}}}}} \hypertarget{d`z`M\string_Ta1}{}
\markboth{\textcolor{darkblue}{\textbf{\ipa{ɖʐɯ˥\textsubscript{a}}}}}{}
\textcolor{teal}{\mytextsc{classificateur}} \hspace{4pt} Ton~: H\textsubscript{a}.

Auto-classificateur des marchés/villes.
\textcolor{brown}{\zh{量词:市场(一个),城市(一个)。}}


\begin{exe}
\sn \textcolor{darkblue}{\textbf{\ipa{ɖʐɯ˧ | ɖɯ˧-ɖʐɯ˥}}}
\trans un marché
\trans \textit{\textcolor{brown}{\zh{一个市场}}}
\end{exe}


\paragraph{\hspace{-0.5cm} {\Large \textcolor{darkblue}{\textbf{\ipa{ɖʐɯ˥kʰɤ˩}}}}} \hypertarget{d`z`M\string_Tk\string_h7\string_B1}{}
\markboth{\textcolor{darkblue}{\textbf{\ipa{ɖʐɯ˥kʰɤ˩}}}}{}
\textcolor{teal}{\mytextsc{nom}} \hspace{4pt} Ton~: .

Un moment.
\textcolor{brown}{\zh{(一)会儿。}}


\begin{exe}
\sn \textcolor{darkblue}{\textbf{\ipa{ɖɯ˧-ɖʐɯ˥kʰɤ˩}}}
\trans un moment
\trans \textit{\textcolor{brown}{\zh{一会儿}}}
\end{exe}


\paragraph{\hspace{-0.5cm} {\Large \textcolor{darkblue}{\textbf{\ipa{ɖʐɯ˧\textasciitilde{}ɖʐɯ˥}}}}} \hypertarget{d`z`M\string_M~d`z`M\string_T1}{}
\markboth{\textcolor{darkblue}{\textbf{\ipa{ɖʐɯ˧\textasciitilde{}ɖʐɯ˥}}}}{}
\textcolor{teal}{\mytextsc{nom}} \hspace{4pt} Ton~: .

Moment, instant.
\textcolor{brown}{\zh{(一)会儿。}}


\begin{exe}
\sn \textcolor{darkblue}{\textbf{\ipa{ɖɯ˧-ɖʐɯ˧\textasciitilde{}ɖʐɯ˥}}}
\trans un moment
\trans \textit{\textcolor{brown}{\zh{一会儿}}}
\end{exe}
\begin{exe}
\sn \textcolor{darkblue}{\textbf{\ipa{ɖɯ˧-ɖʐɯ˧\textasciitilde{}ɖʐɯ˥ ʝi˩}}}
\trans travailler un moment
\trans \textit{\textcolor{brown}{\zh{工作一会儿}}}
\end{exe}


\paragraph{\hspace{-0.5cm} {\Large \textcolor{darkblue}{\textbf{\ipa{ɖʐɯ˧qo˩}}}}} \hypertarget{d`z`M\string_Mqo\string_B1}{}
\markboth{\textcolor{darkblue}{\textbf{\ipa{ɖʐɯ˧qo˩}}}}{}
\textcolor{teal}{\mytextsc{adverbe}} \hspace{4pt} Ton~: L\#.

En ville.
\textcolor{brown}{\zh{在城里、在市里。}}


\begin{exe}
\sn \textcolor{darkblue}{\textbf{\ipa{ɖʐɯ˧qo˩ kʰi˩}}}
\trans aller dans la rue, faire un tour en ville
\trans \textit{\textcolor{brown}{\zh{上街}}}
\end{exe}


\paragraph{\hspace{-0.5cm} {\Large \textcolor{darkblue}{\textbf{\ipa{ɖʐɯ˧ʂɯ˥}}}}} \hypertarget{d`z`M\string_Ms`M\string_T1}{}
\markboth{\textcolor{darkblue}{\textbf{\ipa{ɖʐɯ˧ʂɯ˥}}}}{}
\textcolor{teal}{\mytextsc{nom}} \hspace{4pt} Ton~: H\#.

Baguettes.
\textcolor{brown}{\zh{筷子。}}


 \mytextsc{clf}~: \textcolor{darkblue}{\textbf{\ipa{dzi˧}}} 

\paragraph{\hspace{-0.5cm} {\Large \textcolor{darkblue}{\textbf{\ipa{ɖʐɯ˧ʈʂɯ˥}}}}} \hypertarget{d`z`M\string_Mt`s`M\string_T1}{}
\markboth{\textcolor{darkblue}{\textbf{\ipa{ɖʐɯ˧ʈʂɯ˥}}}}{}
\textcolor{teal}{\mytextsc{nom}} \hspace{4pt} Ton~: H\#.

Vannerie.
\textcolor{brown}{\zh{筛子。}}


 \mytextsc{clf}~: \textcolor{darkblue}{\textbf{\ipa{nɑ˧}}} 

\paragraph{\hspace{-0.5cm} {\Large \textcolor{darkblue}{\textbf{\ipa{ɖʐɯ˩\textasciitilde{}ɖʐɯ˧˥}}}}} \hypertarget{d`z`M\string_B~d`z`M\string_M\string_T1}{}
\markboth{\textcolor{darkblue}{\textbf{\ipa{ɖʐɯ˩\textasciitilde{}ɖʐɯ˧˥}}}}{}
\textcolor{teal}{\mytextsc{verbe}} \hspace{4pt} Ton~: MH.

Secouer (la tête).
\textcolor{brown}{\zh{摇(头)。}}


\begin{exe}
\sn \textcolor{darkblue}{\textbf{\ipa{ʁo˧ ɖʐɯ˥\textasciitilde{}ɖʐɯ˩}}}
\trans secouer la tête
\trans \textit{\textcolor{brown}{\zh{摇头}}}
\end{exe}
\begin{exe}
\sn \textcolor{darkblue}{\textbf{\ipa{ʁo˧ | le˧-ɖʐɯ˩\textasciitilde{}ɖʐɯ˩-ze˩}}}
\trans a secoué la tête
\trans \textit{\textcolor{brown}{\zh{摇了头}}}
\end{exe}


\paragraph{\hspace{-0.5cm} {\Large \textcolor{darkblue}{\textbf{\ipa{ɖʐɯ˩kʰɤ˥}}}}} \hypertarget{d`z`M\string_Bk\string_h7\string_T1}{}
\markboth{\textcolor{darkblue}{\textbf{\ipa{ɖʐɯ˩kʰɤ˥}}}}{}
\textcolor{teal}{\mytextsc{nom}} \hspace{4pt} Ton~: LH.
\textit{De:} \textbf{ɖʐɯ˩a} 
Époque, ère, état de la société.
\textcolor{brown}{\zh{时代。}}


 \mytextsc{clf}~: \textcolor{darkblue}{\textbf{\ipa{ɖʐɯ˩}}} 

\paragraph{\hspace{-0.5cm} {\Large \textcolor{darkblue}{\textbf{\ipa{ɖʐɯ˩tso\#˥}}}}} \hypertarget{d`z`M\string_Btso\#\string_T1}{}
\markboth{\textcolor{darkblue}{\textbf{\ipa{ɖʐɯ˩tso\#˥}}}}{}
\textcolor{teal}{\mytextsc{nom}} \hspace{4pt} Ton~: LM+\#H.

Règles de conduite sociale, règles régissant la société (politique, société).
\textcolor{brown}{\zh{社会规矩。}}


\begin{exe}
\sn \textcolor{darkblue}{\textbf{\ipa{ɖʐɯ˩tso˥ | hĩ˧-qo˩-ɳɯ˩ | le˧-tsʰɯ˩-ɲi˩-tsɯ˩-mæ˩!}}}
\trans ces morales (les contes, les proverbes...) ça provient des hommes! / la morale (des histoires, ...) c'est le fruit de l'expérience des hommes!
\trans \textit{\textcolor{brown}{\zh{社会规矩,是通过人类的经验形成的! / 社会规矩,是人(按一代代的经验)创造的!}}}
\end{exe}
 \mytextsc{clf}~: \textcolor{darkblue}{\textbf{\ipa{kʰwɤ˥}}} 

\paragraph{\hspace{-0.5cm} {\Large \textcolor{darkblue}{\textbf{\ipa{ɖʐv̩˧}}} \textsubscript{1}}} \hypertarget{d`z`v\string_=\string_M1}{}
\markboth{\textcolor{darkblue}{\textbf{\ipa{ɖʐv̩˧}}} \textsubscript{1}}{}
\textcolor{teal}{\mytextsc{verbe}} \hspace{4pt} Ton~: M intrans.

Brûler; prendre feu.
\textcolor{brown}{\zh{燃烧。}}


\begin{exe}
\sn \textcolor{darkblue}{\textbf{\ipa{mv̩˧ ɖʐv̩˧-ze˩!}}}
\trans Ca a pris feu! / Au feu!
\trans \textit{\textcolor{brown}{\zh{着火了!}}}
\end{exe}
\begin{exe}
\sn \textcolor{darkblue}{\textbf{\ipa{mv̩˧ le˧-ɖʐv̩˧-ze˧!}}}
\trans Le feu a pris!
\trans \textit{\textcolor{brown}{\zh{开始着火了!}}}
\end{exe}
\begin{exe}
\sn \textcolor{darkblue}{\textbf{\ipa{tʰi˧-ɖʐv̩˧-dʑo˧!}}}
\trans C'est en train de brûler! / Le feu est en train de brûler!
\trans \textit{\textcolor{brown}{\zh{火在燃烧!}}}
\end{exe}


\paragraph{\hspace{-0.5cm} {\Large \textcolor{darkblue}{\textbf{\ipa{ɖʐv̩˧}}} \textsubscript{2}}} \hypertarget{d`z`v\string_=\string_M2}{}
\markboth{\textcolor{darkblue}{\textbf{\ipa{ɖʐv̩˧}}} \textsubscript{2}}{}
\textcolor{teal}{\mytextsc{nom}} \hspace{4pt} Ton~: M.

Ami/amie, compagnon/compagne.
\textcolor{brown}{\zh{朋友、伙伴、伴侣。}}


\begin{exe}
\sn \textcolor{darkblue}{\textbf{\ipa{njɤ˧ | ɖʐv̩˧ ɲi˩.}}}
\trans C'est mon ami(e).
\trans \textit{\textcolor{brown}{\zh{是我朋友。}}}
\end{exe}
\begin{exe}
\sn \textcolor{darkblue}{\textbf{\ipa{õ˧ ɖʐv̩˥, õ˩ li˩! |}}}
\trans 'On est influencé par ses amis!' (Littéralement: 'On observe ses amis!') Le proverbe souligne l'influence des amis, en bien ou en mal selon qu'on a ou non choisi judicieusement.
\trans \textit{\textcolor{brown}{\zh{“大家都容易受朋友的影响!”(直译:“自己的朋友,自己看(=自己爱学他们的习惯)”)}}}
\end{exe}
 \mytextsc{clf}~: \textcolor{darkblue}{\textbf{\ipa{v̩˧}}} 

\paragraph{\hspace{-0.5cm} {\Large \textcolor{darkblue}{\textbf{\ipa{ɖʐv̩˧}}} \textsubscript{3}}} \hypertarget{d`z`v\string_=\string_M3}{}
\markboth{\textcolor{darkblue}{\textbf{\ipa{ɖʐv̩˧}}} \textsubscript{3}}{}
\textcolor{teal}{\mytextsc{nom}} \hspace{4pt} Ton~: M.

Accident (grave).
\textcolor{brown}{\zh{事故,(不幸的)大事。}}


\begin{exe}
\sn \textcolor{darkblue}{\textbf{\ipa{ɖʐv̩˧ kʰɯ˧˥}}}
\trans causer un accident, commettre une faute; il se passe quelque chose de grave
\trans \textit{\textcolor{brown}{\zh{犯错误,出大事}}}
\end{exe}
\begin{exe}
\sn \textcolor{darkblue}{\textbf{\ipa{ɖʐv̩˧ kʰɯ˧-ze˥}}}
\trans Comme ci-dessus, avec ajout du \mytextsc{pfv}
\trans \textit{\textcolor{brown}{\zh{同上,加上\mytextsc{pfv语素}}}}
\end{exe}
\begin{exe}
\sn \textcolor{darkblue}{\textbf{\ipa{ɖʐv̩˧ ɖɯ˧-ɖʐv̩˧ | kʰɯ˧-ze˥!}}}
\trans il est arrivé un accident!
\trans \textit{\textcolor{brown}{\zh{出大事了!}}}
\end{exe}
 \mytextsc{clf}~: \textcolor{darkblue}{\textbf{\ipa{ɖʐv̩˧}}} 

\paragraph{\hspace{-0.5cm} {\Large \textcolor{darkblue}{\textbf{\ipa{ɖʐv̩˧}}} \textsubscript{4}}} \hypertarget{d`z`v\string_=\string_M4}{}
\markboth{\textcolor{darkblue}{\textbf{\ipa{ɖʐv̩˧}}} \textsubscript{4}}{}
\textcolor{teal}{\mytextsc{nom}} \hspace{4pt} Ton~: M.

Rosée.
\textcolor{brown}{\zh{露水。}}


\textit{Voir~:} \hyperlink{d`z`v\string_=\string_Mq\string_hA\string_M1}{\textcolor{darkblue}{\textbf{\ipa{ɖʐv̩˧qʰɑ˧}}}} 

\paragraph{\hspace{-0.5cm} {\Large \textcolor{darkblue}{\textbf{\ipa{ɖʐv̩˥}}}}} \hypertarget{d`z`v\string_=\string_T1}{}
\markboth{\textcolor{darkblue}{\textbf{\ipa{ɖʐv̩˥}}}}{}
\textcolor{teal}{\mytextsc{nom}} \hspace{4pt} Ton~: \#H.

Artère.
\textcolor{brown}{\zh{动脉。}}


 \mytextsc{clf}~: \textcolor{darkblue}{\textbf{\ipa{kʰɯ˩}}} \textit{Voir~:} \hyperlink{d`z`v\string_=\string_Mtsi\string_T1}{\textcolor{darkblue}{\textbf{\ipa{ɖʐv̩˧tsi˥}}}} 

\paragraph{\hspace{-0.5cm} {\Large \textcolor{darkblue}{\textbf{\ipa{ɖʐv̩˥}}}}} \hypertarget{d`z`v\string_=\string_T1}{}
\markboth{\textcolor{darkblue}{\textbf{\ipa{ɖʐv̩˥}}}}{}
\textcolor{teal}{\mytextsc{adjectif}} \hspace{4pt} Ton~: H.

Humide, mouillé.
\textcolor{brown}{\zh{湿。}}


\begin{exe}
\sn \textcolor{darkblue}{\textbf{\ipa{le˧-ɖʐv̩˥-ze˩}}}
\trans \mytextsc{accomp} \string_ \mytextsc{pfv}
\trans \textit{\textcolor{brown}{\zh{\mytextsc{accomp} \string_ \mytextsc{pfv}}}}
\end{exe}
\begin{exe}
\sn \textcolor{darkblue}{\textbf{\ipa{ɖʐv̩˧\textasciitilde{}ɖʐv̩˧}}}
\trans \mytextsc{red}
\trans \textit{\textcolor{brown}{\zh{\mytextsc{red}}}}
\end{exe}
\begin{exe}
\sn \textcolor{darkblue}{\textbf{\ipa{ʈʂe˧ ɖʐv̩˧-ze˩!}}}
\trans la terre est mouillée!
\trans \textit{\textcolor{brown}{\zh{土湿了。}}}
\end{exe}


\paragraph{\hspace{-0.5cm} {\Large \textcolor{darkblue}{\textbf{\ipa{ɖʐv̩˥}}}}} \hypertarget{d`z`v\string_=\string_T1}{}
\markboth{\textcolor{darkblue}{\textbf{\ipa{ɖʐv̩˥}}}}{}
\textcolor{teal}{\mytextsc{verbe}} \hspace{4pt} Ton~: H.

Augmenter.
\textcolor{brown}{\zh{涨。}}


\begin{exe}
\sn \textcolor{darkblue}{\textbf{\ipa{hĩ˧ ɖʐv̩˧}}}
\trans Les gens deviennent nombreux, se multiplient
\trans \textit{\textcolor{brown}{\zh{人变多}}}
\end{exe}
\begin{exe}
\sn \textcolor{darkblue}{\textbf{\ipa{mo˧ ɖʐv̩˥}}}
\trans les champignons se multiplient
\trans \textit{\textcolor{brown}{\zh{菌子长得多}}}
\end{exe}


\paragraph{\hspace{-0.5cm} {\Large \textcolor{darkblue}{\textbf{\ipa{ɖʐv̩˩\textsubscript{a}}}} \textsubscript{1}}} \hypertarget{d`z`v\string_=\string_Ba1}{}
\markboth{\textcolor{darkblue}{\textbf{\ipa{ɖʐv̩˩\textsubscript{a}}}} \textsubscript{1}}{}
\textcolor{teal}{\mytextsc{adjectif}} \hspace{4pt} Ton~: L\textsubscript{a}.

Laid, vilain.
\textcolor{brown}{\zh{丑陋。}}


\begin{exe}
\sn \textcolor{darkblue}{\textbf{\ipa{ɖʐv̩˩-hĩ˩˥}}}
\trans \mytextsc{rel}/\mytextsc{nmlz}
\trans \textit{\textcolor{brown}{\zh{丑的}}}
\end{exe}
\begin{exe}
\sn \textcolor{darkblue}{\textbf{\ipa{ʈʂʰɯ˧-v̩˧ | ɖwæ˧˥ | ɖʐv̩˩˥!}}}
\trans celui-là/celle-là est vraiment méchant/mauvais
\trans \textit{\textcolor{brown}{\zh{这个好丑!}}}
\end{exe}


\paragraph{\hspace{-0.5cm} {\Large \textcolor{darkblue}{\textbf{\ipa{ɖʐv̩˩\textsubscript{a}}}} \textsubscript{2}}} \hypertarget{d`z`v\string_=\string_Ba2}{}
\markboth{\textcolor{darkblue}{\textbf{\ipa{ɖʐv̩˩\textsubscript{a}}}} \textsubscript{2}}{}
\textcolor{teal}{\mytextsc{verbe}} \hspace{4pt} Ton~: L\textsubscript{a}.

Décider, choisir.
\textcolor{brown}{\zh{决定、选择、拿主意。}}


\begin{exe}
\sn \textcolor{darkblue}{\textbf{\ipa{njɤ˧-ɳɯ˧ | ɖʐv̩˧ ʝi˧-bi˧!}}}
\trans C'est moi qui vais décider!
\trans \textit{\textcolor{brown}{\zh{我来决定吧!}}}
\end{exe}


\paragraph{\hspace{-0.5cm} {\Large \textcolor{darkblue}{\textbf{\ipa{ɖʐv̩˧\textsubscript{b}}}}}} \hypertarget{d`z`v\string_=\string_Mb1}{}
\markboth{\textcolor{darkblue}{\textbf{\ipa{ɖʐv̩˧\textsubscript{b}}}}}{}
\textcolor{teal}{\mytextsc{classificateur}} \hspace{4pt} Ton~: M\textsubscript{b}.

Auto-classificateur des accidents.
\textcolor{brown}{\zh{量词:事故(一场)。}}




\paragraph{\hspace{-0.5cm} {\Large \textcolor{darkblue}{\textbf{\ipa{ɖʐv̩˧-nɑ˥mi˩}}}}} \hypertarget{d`z`v\string_=\string_M-nA\string_Tmi\string_B1}{}
\markboth{\textcolor{darkblue}{\textbf{\ipa{ɖʐv̩˧-nɑ˥mi˩}}}}{}
\textcolor{teal}{\mytextsc{nom}} \hspace{4pt} Ton~: \#H-.

Héron: oiseau échassier, non migrateur.
\textcolor{brown}{\zh{鹳。}}


 \mytextsc{clf}~: \textcolor{darkblue}{\textbf{\ipa{mi˩}}} 

\paragraph{\hspace{-0.5cm} {\Large \textcolor{darkblue}{\textbf{\ipa{ɖʐv̩˧qʰɑ˧}}}}} \hypertarget{d`z`v\string_=\string_Mq\string_hA\string_M1}{}
\markboth{\textcolor{darkblue}{\textbf{\ipa{ɖʐv̩˧qʰɑ˧}}}}{}
\textcolor{teal}{\mytextsc{nom}} \hspace{4pt} Ton~: M.

Rosée.
\textcolor{brown}{\zh{露水。}}


\textit{Voir~:} \hyperlink{d`z`v\string_=\string_M4}{\textcolor{darkblue}{\textbf{\ipa{ɖʐv̩˧}}} \textsubscript{4}} 

\paragraph{\hspace{-0.5cm} {\Large \textcolor{darkblue}{\textbf{\ipa{ɖʐv̩˩ti\#˥}}}}} \hypertarget{d`z`v\string_=\string_Bti\#\string_T1}{}
\markboth{\textcolor{darkblue}{\textbf{\ipa{ɖʐv̩˩ti\#˥}}}}{}
\textcolor{teal}{\mytextsc{nom}} \hspace{4pt} Ton~: LM+\#H.

Lance.
\textcolor{brown}{\zh{矛。}}




\paragraph{\hspace{-0.5cm} {\Large \textcolor{darkblue}{\textbf{\ipa{ɖʐv̩˧tsi˥}}}}} \hypertarget{d`z`v\string_=\string_Mtsi\string_T1}{}
\markboth{\textcolor{darkblue}{\textbf{\ipa{ɖʐv̩˧tsi˥}}}}{}
\textcolor{teal}{\mytextsc{nom}} \hspace{4pt} Ton~: H\#.
\ding{202} 
Artère (du corps humain).
\textcolor{brown}{\zh{动脉。}}


\ding{203} 
Tige (d'une plante).
\textcolor{brown}{\zh{茎。}}


 \mytextsc{clf}~: \textcolor{darkblue}{\textbf{\ipa{kʰɯ˩}}} \textit{Voir~:} \textcolor{darkblue}{\textbf{\ipa{ɖʐv̩˥}}} 

\paragraph{\hspace{-0.5cm} {\Large \textcolor{darkblue}{\textbf{\ipa{ɖʐv̩˧ʐv̩˧-ɖʐv̩˧mi\#˥}}}}} \hypertarget{d`z`v\string_=\string_Mz`v\string_=\string_M-d`z`v\string_=\string_Mmi\#\string_T1}{}
\markboth{\textcolor{darkblue}{\textbf{\ipa{ɖʐv̩˧ʐv̩˧-ɖʐv̩˧mi\#˥}}}}{}
\textcolor{teal}{\mytextsc{nom}} \hspace{4pt} Ton~: \#H.

Ami(e).
\textcolor{brown}{\zh{朋友、伙伴、伴侣。}}




\paragraph{\hspace{-0.5cm} {\Large \textcolor{darkblue}{\textbf{\ipa{ɖʐwæ˥}}}}} \hypertarget{d`z`w\{\string_T1}{}
\markboth{\textcolor{darkblue}{\textbf{\ipa{ɖʐwæ˥}}}}{}
\textcolor{teal}{\mytextsc{nom}} \hspace{4pt} Ton~: \#H.

Petite houe (plus petite que \textcolor{darkblue}{\textbf{\ipa{/hwæ˧pʰæ˩/}}}).
\textcolor{brown}{\zh{锄头。}}


 \mytextsc{clf}~: \textcolor{darkblue}{\textbf{\ipa{nɑ˧}}} 

\paragraph{\hspace{-0.5cm} {\Large \textcolor{darkblue}{\textbf{\ipa{ɖʐwæ˧lɑ˧-ʁo˧ɖɯ˧˥}}}}} \hypertarget{d`z`w\{\string_MlA\string_M-Ro\string_Md`M\string_M\string_T1}{}
\markboth{\textcolor{darkblue}{\textbf{\ipa{ɖʐwæ˧lɑ˧-ʁo˧ɖɯ˧˥}}}}{}
\textcolor{teal}{\mytextsc{nom}} \hspace{4pt} Ton~: MH\#.

Oiseau ressemblant à un moineau, au corps blanc et noir; M23 croît le reconnaître dans: Pericrocotus divaricatus, mais cette espèce n'existe que dans le nord de la Chine.
\textcolor{brown}{\zh{雀。}}


 \mytextsc{clf}~: \textcolor{darkblue}{\textbf{\ipa{mi˩}}} 

\paragraph{\hspace{-0.5cm} {\Large \textcolor{darkblue}{\textbf{\ipa{ɖʐwæ˧mi˧}}}}} \hypertarget{d`z`w\{\string_Mmi\string_M1}{}
\markboth{\textcolor{darkblue}{\textbf{\ipa{ɖʐwæ˧mi˧}}}}{}
\textcolor{teal}{\mytextsc{nom}} \hspace{4pt} Ton~: M.

Moineau.
\textcolor{brown}{\zh{麻雀。}}


 \mytextsc{clf}~: \textcolor{darkblue}{\textbf{\ipa{mi˩}}} 

\paragraph{\hspace{-0.5cm} {\Large \textcolor{darkblue}{\textbf{\ipa{ɖʐwæ˧pʰv̩\#˥}}}}} \hypertarget{d`z`w\{\string_Mp\string_hv\string_=\#\string_T1}{}
\markboth{\textcolor{darkblue}{\textbf{\ipa{ɖʐwæ˧pʰv̩\#˥}}}}{}
\textcolor{teal}{\mytextsc{nom}} \hspace{4pt} Ton~: \#H.

Moineau mâle.
\textcolor{brown}{\zh{公麻雀。}}


\begin{exe}
\sn \textcolor{darkblue}{\textbf{\ipa{ɖʐwæ˧pʰv̩˧ tʰv̩˧-mi˧˥ / ɖʐwæ˧pʰv̩˧ tʰv̩˧-mi˥\#}}}
\trans \mytextsc{n}+\mytextsc{dem}+\mytextsc{clf}
\trans \textit{\textcolor{brown}{\zh{这只公麻雀}}}
\end{exe}
 \mytextsc{clf}~: \textcolor{darkblue}{\textbf{\ipa{mi˩}}} 

\paragraph{\hspace{-0.5cm} {\Large \textcolor{darkblue}{\textbf{\ipa{ɖʐwæ˧zo\#˥}}}}} \hypertarget{d`z`w\{\string_Mzo\#\string_T1}{}
\markboth{\textcolor{darkblue}{\textbf{\ipa{ɖʐwæ˧zo\#˥}}}}{}
\textcolor{teal}{\mytextsc{nom}} \hspace{4pt} Ton~: \#H.

Moinillon, petit moineau, bébé moineau.
\textcolor{brown}{\zh{小麻雀。}}


 \mytextsc{clf}~: \textcolor{darkblue}{\textbf{\ipa{v̩˧, mi˩}}} 

\paragraph{\hspace{-0.5cm} {\Large \textcolor{darkblue}{\textbf{\ipa{ɖʐwæ˩\textsubscript{a}}}}}} \hypertarget{d`z`w\{\string_Ba1}{}
\markboth{\textcolor{darkblue}{\textbf{\ipa{ɖʐwæ˩\textsubscript{a}}}}}{}
\textcolor{teal}{\mytextsc{verbe}} \hspace{4pt} Ton~: L\textsubscript{a}.

Se disputer (monosyllabe).
\textcolor{brown}{\zh{吵架。}}


\begin{exe}
\sn \textcolor{darkblue}{\textbf{\ipa{ɖʐwæ˧\textasciitilde{}ɖʐwæ˥}}}
\trans \mytextsc{red}
\trans \textit{\textcolor{brown}{\zh{\mytextsc{重叠}}}}
\end{exe}


\paragraph{\hspace{-0.5cm} {\Large \textcolor{darkblue}{\textbf{\ipa{ɖʐwæ˩hi˩}}}}} \hypertarget{d`z`w\{\string_Bhi\string_B1}{}
\markboth{\textcolor{darkblue}{\textbf{\ipa{ɖʐwæ˩hi˩}}}}{}
\textcolor{teal}{\mytextsc{nom}} \hspace{4pt} Ton~: L.
\ding{202} 
Canine (dent), croc.
\textcolor{brown}{\zh{獠牙。}}


\ding{203} 
Crocs (de bête).
\textcolor{brown}{\zh{动物的牙(犬牙)。}}


 \mytextsc{clf}~: \textcolor{darkblue}{\textbf{\ipa{ɭɯ˧}}} 

\paragraph{\hspace{-0.5cm} {\Large \textcolor{darkblue}{\textbf{\ipa{ɖʐwæ˧˥}}}}} \hypertarget{d`z`w\{\string_M\string_T1}{}
\markboth{\textcolor{darkblue}{\textbf{\ipa{ɖʐwæ˧˥}}}}{}
\textcolor{teal}{\mytextsc{verbe}} \hspace{4pt} Ton~: MH.

Tomber; laisser tomber, lâcher (un objet qu'on tenait à la main).
\textcolor{brown}{\zh{掉下。}}


\begin{exe}
\sn \textcolor{darkblue}{\textbf{\ipa{mv̩˩tɕo˧ ɖʐwæ˧˥ / mv̩˩tɕo˧ ɖʐwæ˧-ze˥}}}
\trans tomber par terre; littéralement ``tomber vers le bas"
\trans \textit{\textcolor{brown}{\zh{掉下去(+了)}}}
\end{exe}


\paragraph{\hspace{-0.5cm} {\Large \textcolor{darkblue}{\textbf{\ipa{ɖʐwæ˩˧}}}}} \hypertarget{d`z`w\{\string_B\string_M1}{}
\markboth{\textcolor{darkblue}{\textbf{\ipa{ɖʐwæ˩˧}}}}{}
\textcolor{teal}{\mytextsc{nom}} \hspace{4pt} Ton~: LM.

Moineau (forme monosyllabique; n'est pas d'usage courant).
\textcolor{brown}{\zh{麻雀。}}


 \mytextsc{clf}~: \textcolor{darkblue}{\textbf{\ipa{mi˩}}} 

\newpage
\section*{\centering- \textcolor{darkblue}{\textbf{\ipa{ə}}} -}
\paragraph{\hspace{-0.5cm} {\Large \textcolor{darkblue}{\textbf{\ipa{ə}}}}} \hypertarget{@1}{}
\markboth{\textcolor{darkblue}{\textbf{\ipa{ə}}}}{}
\textcolor{teal}{\mytextsc{interjection}} \hspace{4pt} Ton~: .

Interjection.
\textcolor{brown}{\zh{感叹词。}}




\paragraph{\hspace{-0.5cm} {\Large \textcolor{darkblue}{\textbf{\ipa{ə˧bɑ˩-lɑ˩bɑ˩}}}}} \hypertarget{@\string_MbA\string_B-lA\string_BbA\string_B1}{}
\markboth{\textcolor{darkblue}{\textbf{\ipa{ə˧bɑ˩-lɑ˩bɑ˩}}}}{}
\textcolor{teal}{\mytextsc{nom}} \hspace{4pt} Ton~: L\#-.

Cactus.
\textcolor{brown}{\zh{仙人掌。}}


\begin{exe}
\sn \textcolor{darkblue}{\textbf{\ipa{ə˧bɑ˩-lɑ˩bɑ˩ | ɖɯ˧-dzi˩}}}
\trans un cactus
\trans \textit{\textcolor{brown}{\zh{一棵仙人掌}}}
\end{exe}


\paragraph{\hspace{-0.5cm} {\Large \textcolor{darkblue}{\textbf{\ipa{ə˧bo˥\$}}}}} \hypertarget{@\string_Mbo\string_T\$1}{}
\markboth{\textcolor{darkblue}{\textbf{\ipa{ə˧bo˥\$}}}}{}
\textcolor{teal}{\mytextsc{nom}} \hspace{4pt} Ton~: H\$.

Oncle paternel=frère du père (sens vérifié: renvoie à la famille du père).
\textcolor{brown}{\zh{父亲的兄弟。}}


\begin{exe}
\sn \textcolor{darkblue}{\textbf{\ipa{ə˧bo˧-ɖɯ˧˥}}}
\trans oncle paternel aîné du père
\trans \textit{\textcolor{brown}{\zh{伯父:父亲的哥哥}}}
\end{exe}
\begin{exe}
\sn \textcolor{darkblue}{\textbf{\ipa{ə˧bo˧-tɕi˥ (+ɲi˩)}}}
\trans oncle paternel cadet du père
\trans \textit{\textcolor{brown}{\zh{叔叔:父亲的弟弟}}}
\end{exe}
 \mytextsc{clf}~: \textcolor{darkblue}{\textbf{\ipa{v̩˧}}} 

\paragraph{\hspace{-0.5cm} {\Large \textcolor{darkblue}{\textbf{\ipa{ə˧bo˧tɕo˧li˧}}}}} \hypertarget{@\string_Mbo\string_Mts£o\string_Mli\string_M1}{}
\markboth{\textcolor{darkblue}{\textbf{\ipa{ə˧bo˧tɕo˧li˧}}}}{}
\textcolor{teal}{\mytextsc{nom}} \hspace{4pt} Ton~: M.

Criquet.
\textcolor{brown}{\zh{蟋蟀。}}


 \mytextsc{clf}~: \textcolor{darkblue}{\textbf{\ipa{mi˩}}} 

\paragraph{\hspace{-0.5cm} {\Large \textcolor{darkblue}{\textbf{\ipa{ə˧bv̩˩}}}}} \hypertarget{@\string_Mbv\string_=\string_B1}{}
\markboth{\textcolor{darkblue}{\textbf{\ipa{ə˧bv̩˩}}}}{}
\textcolor{teal}{\mytextsc{nom}} \hspace{4pt} Ton~: L\#.

Four où on cuit les briques, les objets en céramique….
\textcolor{brown}{\zh{烤砖、陶器等用的烤炉。}}


 \mytextsc{clf}~: \textcolor{darkblue}{\textbf{\ipa{ɭɯ˧}}} 

\paragraph{\hspace{-0.5cm} {\Large \textcolor{darkblue}{\textbf{\ipa{ə˧bv̩˧-ʁwɤ˧}}}}} \hypertarget{@\string_Mbv\string_=\string_M-Rw7\string_M1}{}
\markboth{\textcolor{darkblue}{\textbf{\ipa{ə˧bv̩˧-ʁwɤ˧}}}}{}
\textcolor{teal}{\mytextsc{nom}} \hspace{4pt} Ton~: M.

Abuwa (nom de village).
\textcolor{brown}{\zh{阿布瓦村。}}




\paragraph{\hspace{-0.5cm} {\Large \textcolor{darkblue}{\textbf{\ipa{ə˧ɕjɤ˩}}}}} \hypertarget{@\string_Ms£j7\string_B1}{}
\markboth{\textcolor{darkblue}{\textbf{\ipa{ə˧ɕjɤ˩}}}}{}
\textcolor{teal}{\mytextsc{nom}} \hspace{4pt} Ton~: L\#.

Petit(e) ami(e), amant(e).
\textcolor{brown}{\zh{情人。}}


 \mytextsc{clf}~: \textcolor{darkblue}{\textbf{\ipa{v̩˧}}} \textit{Voir~:} \hyperlink{@\string_Md`o\string_M1}{\textcolor{darkblue}{\textbf{\ipa{ə˧ɖo˧}}}} 

\paragraph{\hspace{-0.5cm} {\Large \textcolor{darkblue}{\textbf{\ipa{ə˧ɕjo˩}}}}} \hypertarget{@\string_Ms£jo\string_B1}{}
\markboth{\textcolor{darkblue}{\textbf{\ipa{ə˧ɕjo˩}}}}{}
\textcolor{teal}{\mytextsc{nom}} \hspace{4pt} Ton~: L\#.

Nom de clan/famille étendue. Deux familles portent ce nom à Yongning.
\textcolor{brown}{\zh{一个姓。这个姓,永宁有两家。}}


\begin{exe}
\sn \textcolor{darkblue}{\textbf{\ipa{ə˧ɕjo˩=ɻ̍˩}}}
\trans le clan \textcolor{darkblue}{\textbf{\ipa{/ə˧ɕjo˩/}}}
\trans \textit{\textcolor{brown}{\zh{\textcolor{darkblue}{\textbf{\ipa{/ə˧ɕjo˩/}}}家族}}}
\end{exe}


\paragraph{\hspace{-0.5cm} {\Large \textcolor{darkblue}{\textbf{\ipa{ə˧dɑ˥\$}}}}} \hypertarget{@\string_MdA\string_T\$1}{}
\markboth{\textcolor{darkblue}{\textbf{\ipa{ə˧dɑ˥\$}}}}{}
\textcolor{teal}{\mytextsc{nom}} \hspace{4pt} Ton~: H\$.

Père.
\textcolor{brown}{\zh{父亲。}}


 \mytextsc{clf}~: \textcolor{darkblue}{\textbf{\ipa{v̩˧}}} 

\paragraph{\hspace{-0.5cm} {\Large \textcolor{darkblue}{\textbf{\ipa{ə˧dɑ˧-ə˧mi\#˥}}}}} \hypertarget{@\string_MdA\string_M-@\string_Mmi\#\string_T1}{}
\markboth{\textcolor{darkblue}{\textbf{\ipa{ə˧dɑ˧-ə˧mi\#˥}}}}{}
\textcolor{teal}{\mytextsc{nom}} \hspace{4pt} Ton~: \#H.

Père et mère.
\textcolor{brown}{\zh{父母。}}


\begin{exe}
\sn \textcolor{darkblue}{\textbf{\ipa{ə˧dɑ˧-ə˧mi˧ ɲi˥-kv̩˩}}}
\trans le père et la mère, tous les deux; le couple formé du père et de la mère
\trans \textit{\textcolor{brown}{\zh{父母亲}}}
\end{exe}


\paragraph{\hspace{-0.5cm} {\Large \textcolor{darkblue}{\textbf{\ipa{ə˧dɑ˧-zo\#˥}}}}} \hypertarget{@\string_MdA\string_M-zo\#\string_T1}{}
\markboth{\textcolor{darkblue}{\textbf{\ipa{ə˧dɑ˧-zo\#˥}}}}{}
\textcolor{teal}{\mytextsc{nom}} \hspace{4pt} Ton~: \#H.

Père et fils.
\textcolor{brown}{\zh{父子。}}




\paragraph{\hspace{-0.5cm} {\Large \textcolor{darkblue}{\textbf{\ipa{ə˧dze˧}}}}} \hypertarget{@\string_Mdze\string_M1}{}
\markboth{\textcolor{darkblue}{\textbf{\ipa{ə˧dze˧}}}}{}
\textcolor{teal}{\mytextsc{nom}} \hspace{4pt} Ton~: M.

Grémil des teinturiers, \textit{Lithospermum erythrorhizon Sieb. et Zucc.}.
\textcolor{brown}{\zh{紫草。}}Dialecte chinois local~: \zh{紫红草。}


\begin{exe}
\sn \textcolor{darkblue}{\textbf{\ipa{ə˧dze˧-njɤ˩hṽ˩}}}
\trans même sens
\trans \textit{\textcolor{brown}{\zh{紫草}}}
\end{exe}
\begin{exe}
\sn \textcolor{darkblue}{\textbf{\ipa{ə˧dze˧-bæ˩bæ˩}}}
\trans fleurs de grémil
\trans \textit{\textcolor{brown}{\zh{紫草花}}}
\end{exe}


\paragraph{\hspace{-0.5cm} {\Large \textcolor{darkblue}{\textbf{\ipa{ə˧-dzɤ˥\$}}}}} \hypertarget{@\string_M-dz7\string_T\$1}{}
\markboth{\textcolor{darkblue}{\textbf{\ipa{ə˧-dzɤ˥\$}}}}{}
\textcolor{teal}{\mytextsc{adverbe}} \hspace{4pt} Ton~: H\$.

Lentement, doucement.
\textcolor{brown}{\zh{慢。}}


\begin{exe}
\sn \textcolor{darkblue}{\textbf{\ipa{ə˧-dzɤ˧ ʝi˧}}}
\trans travailler lentement, faire lentement
\trans \textit{\textcolor{brown}{\zh{慢慢做}}}
\end{exe}
\begin{exe}
\sn \textcolor{darkblue}{\textbf{\ipa{ə˧dzɤ˧ le˧-hõ˩! |}}}
\trans Au revoir! (Dit par l'hôte à la personne qui s'en va. Littéralement: ``Allez doucement!" / ``Prenez votre temps en chemin!")
\trans \textit{\textcolor{brown}{\zh{慢走!}}}
\end{exe}
\begin{exe}
\sn \textcolor{darkblue}{\textbf{\ipa{ə˧dzɤ˥ | le˧-hõ˩! |}}}
\trans Au revoir!
\trans \textit{\textcolor{brown}{\zh{慢走!}}}
\end{exe}
\begin{exe}
\sn \textcolor{darkblue}{\textbf{\ipa{ə˧dzɤ˧ le˧-dzi˩! |}}}
\trans Au revoir! (Dit par l'invité à son hôte. Littéralement: ``Restez assis doucement = tranquillement!"
\trans \textit{\textcolor{brown}{\zh{慢慢坐!}}}
\end{exe}
\textit{Voir~:} \textcolor{darkblue}{\textbf{\ipa{ə˧ze˧, ə˧-dzɤ˧\textasciitilde{}dzɤ˥}}} 

\paragraph{\hspace{-0.5cm} {\Large \textcolor{darkblue}{\textbf{\ipa{ə˧-dzɤ˧\textasciitilde{}dzɤ˥}}}}} \hypertarget{@\string_M-dz7\string_M~dz7\string_T1}{}
\markboth{\textcolor{darkblue}{\textbf{\ipa{ə˧-dzɤ˧\textasciitilde{}dzɤ˥}}}}{}
\textcolor{teal}{\mytextsc{adverbe}} \hspace{4pt} Ton~: H\#.
\textit{De:} \textbf{ə˧-dzɤ˥\$, ə˧ze˧} 
Lentement, doucement.
\textcolor{brown}{\zh{慢慢地。}}


\begin{exe}
\sn \textcolor{darkblue}{\textbf{\ipa{ʈʂʰɯ˧ | ɖwæ˧˥ | ə˧-dzɤ˧\textasciitilde{}dzɤ˥ ʝi˩-kv̩˩!}}}
\trans Il/elle travaille avec grand soin. (Le sens littéral est ``Il/elle travaille très lentement", mais cela n'est pas une critique: cela signifie qu'il/elle sait prendre le temps pour réaliser du bon travail.)
\trans \textit{\textcolor{brown}{\zh{他工作很细致。(直译:‘他工作很慢’,但不是批评:意味着那个人懂得慢慢来做,做得更仔细。)}}}
\end{exe}
\begin{exe}
\sn \textcolor{darkblue}{\textbf{\ipa{[F5] ə˧-dzɤ˧\textasciitilde{}dzɤ˥ ʝi˩}}}
\trans travailler lentement, faire lentement
\trans \textit{\textcolor{brown}{\zh{慢慢地做}}}
\end{exe}
\begin{exe}
\sn \textcolor{darkblue}{\textbf{\ipa{[M21] ə˧-zɤ˧\textasciitilde{}zɤ˥ ʝi˩}}}
\trans Travaille doucement! / Prends ton temps! / travailler lentement, faire lentement
\trans \textit{\textcolor{brown}{\zh{慢慢地做}}}
\end{exe}
\textit{Voir~:} \hyperlink{@\string_M-dz7\string_T\$1}{\textcolor{darkblue}{\textbf{\ipa{ə˧-dzɤ˥\$}}}} 

\paragraph{\hspace{-0.5cm} {\Large \textcolor{darkblue}{\textbf{\ipa{ə˧ɖo˧}}}}} \hypertarget{@\string_Md`o\string_M1}{}
\markboth{\textcolor{darkblue}{\textbf{\ipa{ə˧ɖo˧}}}}{}
\textcolor{teal}{\mytextsc{nom}} \hspace{4pt} Ton~: M.

Petit ami, petite amie, amant(e).
\textcolor{brown}{\zh{情人(音译:阿注)。}}


 \mytextsc{clf}~: \textcolor{darkblue}{\textbf{\ipa{v̩˧}}} \textit{Voir~:} \hyperlink{@\string_Ms£j7\string_B1}{\textcolor{darkblue}{\textbf{\ipa{ə˧ɕjɤ˩}}}} 

\paragraph{\hspace{-0.5cm} {\Large \textcolor{darkblue}{\textbf{\ipa{ə˧go˧}}}}} \hypertarget{@\string_Mgo\string_M1}{}
\markboth{\textcolor{darkblue}{\textbf{\ipa{ə˧go˧}}}}{}
\textcolor{teal}{\mytextsc{nom}} \hspace{4pt} Ton~: M.

Nom de clan/famille étendue. Trois familles portent ce nom à Yongning.
\textcolor{brown}{\zh{一个姓。这个姓,永宁有三家。}}


\begin{exe}
\sn \textcolor{darkblue}{\textbf{\ipa{ə˧go˧=ɻ̍˩}}}
\trans le clan \textcolor{darkblue}{\textbf{\ipa{/ə˧go˧/}}}
\trans \textit{\textcolor{brown}{\zh{\textcolor{darkblue}{\textbf{\ipa{/ə˧go˧/}}}家族}}}
\end{exe}
\begin{exe}
\sn \textcolor{darkblue}{\textbf{\ipa{ə˧go˧ | dʑɤ˩tsʰi˧}}}
\trans la personne prénommée \textcolor{darkblue}{\textbf{\ipa{/dʑɤ˩tsʰi\#˥/}}}, du clan \textcolor{darkblue}{\textbf{\ipa{/ə˧go˧/}}}
\trans \textit{\textcolor{brown}{\zh{\textcolor{darkblue}{\textbf{\ipa{/ə˧go˧/}}} 家族名叫\textcolor{darkblue}{\textbf{\ipa{/dʑɤ˩tsʰi\#˥/}}}那个人}}}
\end{exe}


\paragraph{\hspace{-0.5cm} {\Large \textcolor{darkblue}{\textbf{\ipa{ə˧go˧-ʁwɤ˧}}}}} \hypertarget{@\string_Mgo\string_M-Rw7\string_M1}{}
\markboth{\textcolor{darkblue}{\textbf{\ipa{ə˧go˧-ʁwɤ˧}}}}{}
\textcolor{teal}{\mytextsc{nom}} \hspace{4pt} Ton~: M.

Un village proche de Wenquan.
\textcolor{brown}{\zh{温泉乡的一个村落。}}


\begin{exe}
\sn \textcolor{darkblue}{\textbf{\ipa{ə˧go˧-ʁwɤ˧, | ʁwɤ˧lɑ˩-bi˩, | bæ˧ʁwɤ˧, | tʰo˧tsʰe\#˥, | pi˧tsʰe˩-di˩, | pɤ˧dʑɤ˩-di˩, | ʁwɤ˧tv̩˧}}}
\trans Villages au sortir de la plaine de Yongning; les deux premiers comportent une population na; le troisième est un village na; les suivants sont essentiellement des villages pumi/prinmi.
\trans \textit{\textcolor{brown}{\zh{永宁背向泸沽湖方向经过的村落。前两个村落拥有相当大的摩梭人口比例,第三个村落是摩梭村,最后一个是普米村。}}}
\end{exe}


\paragraph{\hspace{-0.5cm} {\Large \textcolor{darkblue}{\textbf{\ipa{ə˧gɯ˩}}}}} \hypertarget{@\string_MgM\string_B1}{}
\markboth{\textcolor{darkblue}{\textbf{\ipa{ə˧gɯ˩}}}}{}
\textcolor{teal}{\mytextsc{nom}} \hspace{4pt} Ton~: L\#.

Menthe.
\textcolor{brown}{\zh{薄荷。}}


 \mytextsc{clf}~: \textcolor{darkblue}{\textbf{\ipa{po˧}}} 

\paragraph{\hspace{-0.5cm} {\Large \textcolor{darkblue}{\textbf{\ipa{ə˧hɑ˩-bɑ˩lɑ˩}}}}} \hypertarget{@\string_MhA\string_B-bA\string_BlA\string_B1}{}
\markboth{\textcolor{darkblue}{\textbf{\ipa{ə˧hɑ˩-bɑ˩lɑ˩}}}}{}
\textcolor{teal}{\mytextsc{nom}} \hspace{4pt} Ton~: L\#-.

Chanson traditionnelle.
\textcolor{brown}{\zh{民歌。}}


\begin{exe}
\sn \textcolor{darkblue}{\textbf{\ipa{ə˧hɑ˩bɑ˩lɑ˩ | ɖɯ˧-ɖʐo˩ gwɤ˩}}}
\trans chanter une chanson
\trans \textit{\textcolor{brown}{\zh{唱一首摩梭歌}}}
\end{exe}
 \mytextsc{clf}~: \textcolor{darkblue}{\textbf{\ipa{ɖʐo˩}}} 

\paragraph{\hspace{-0.5cm} {\Large \textcolor{darkblue}{\textbf{\ipa{ə˧hĩ˥}}}}} \hypertarget{@\string_Mhi\string_~\string_T1}{}
\markboth{\textcolor{darkblue}{\textbf{\ipa{ə˧hĩ˥}}}}{}
\textcolor{teal}{\mytextsc{adverbe}} \hspace{4pt} Ton~: H\#.

Dans un moment.
\textcolor{brown}{\zh{一会儿、待会儿、等一下。}}


\begin{exe}
\sn \textcolor{darkblue}{\textbf{\ipa{ə˧hĩ˥-ɳɯ˩, | li˧-kʰɯ˧-bi˥!}}}
\trans Tout à l’heure je vais te montrer! / dans un moment, je te montrerai!
\trans \textit{\textcolor{brown}{\zh{待会儿,我给你看吧!}}}
\end{exe}


\paragraph{\hspace{-0.5cm} {\Large \textcolor{darkblue}{\textbf{\ipa{ə˧hwɤ˧}}}}} \hypertarget{@\string_Mhw7\string_M1}{}
\markboth{\textcolor{darkblue}{\textbf{\ipa{ə˧hwɤ˧}}}}{}
\textcolor{teal}{\mytextsc{adverbe}} \hspace{4pt} Ton~: M.

Hier soir.
\textcolor{brown}{\zh{昨晚。}}


\begin{exe}
\sn \textcolor{darkblue}{\textbf{\ipa{ə˧hwɤ˧ | mv̩˩kʰv̩˧˥}}}
\trans hier au soir, dans la nuit
\trans \textit{\textcolor{brown}{\zh{昨晚,夜里}}}
\end{exe}


\paragraph{\hspace{-0.5cm} {\Large \textcolor{darkblue}{\textbf{\ipa{ə˧hwɤ˧-zo˧hṽ˧˥}}}}} \hypertarget{@\string_Mhw7\string_M-zo\string_Mhv\string_~\string_M\string_T1}{}
\markboth{\textcolor{darkblue}{\textbf{\ipa{ə˧hwɤ˧-zo˧hṽ˧˥}}}}{}
\textcolor{teal}{\mytextsc{nom}} \hspace{4pt} Ton~: \mytextsc{MH}\#.

Bébé, nouveau-né, nourrisson.
\textcolor{brown}{\zh{婴儿。}}




\paragraph{\hspace{-0.5cm} {\Large \textcolor{darkblue}{\textbf{\ipa{ə˧jɤ˩}}}}} \hypertarget{@\string_Mj7\string_B1}{}
\markboth{\textcolor{darkblue}{\textbf{\ipa{ə˧jɤ˩}}}}{}
\textcolor{teal}{\mytextsc{nom}} \hspace{4pt} Ton~: L\#.

Tante maternelle (sœur aînée de la mère).
\textcolor{brown}{\zh{姨母 (比母亲大)。}}


 \mytextsc{clf}~: \textcolor{darkblue}{\textbf{\ipa{v̩˧}}} 

\paragraph{\hspace{-0.5cm} {\Large \textcolor{darkblue}{\textbf{\ipa{ə˧ʝi˥\$}}}}} \hypertarget{@\string_Mj££i\string_T\$1}{}
\markboth{\textcolor{darkblue}{\textbf{\ipa{ə˧ʝi˥\$}}}}{}
\textcolor{teal}{\mytextsc{adverbe}} \hspace{4pt} Ton~: H\$.

L'année dernière, l'année passée, l'an passé, l'an dernier.
\textcolor{brown}{\zh{去年。}}




\paragraph{\hspace{-0.5cm} {\Large \textcolor{darkblue}{\textbf{\ipa{ə˧ʝi˧-ʂɯ˥ʝi˩}}}}} \hypertarget{@\string_Mj££i\string_M-s`M\string_Tj££i\string_B1}{}
\markboth{\textcolor{darkblue}{\textbf{\ipa{ə˧ʝi˧-ʂɯ˥ʝi˩}}}}{}
\textcolor{teal}{\mytextsc{adverbe}} \hspace{4pt} Ton~: \#H-.

Jadis, aux temps anciens, il était une fois.
\textcolor{brown}{\zh{很久以前,古时候,传说古代。}}




\paragraph{\hspace{-0.5cm} {\Large \textcolor{darkblue}{\textbf{\ipa{ə˧ʝi˧-tsʰi˧ʝi\#˥}}}}} \hypertarget{@\string_Mj££i\string_M-ts\string_hi\string_Mj££i\#\string_T1}{}
\markboth{\textcolor{darkblue}{\textbf{\ipa{ə˧ʝi˧-tsʰi˧ʝi\#˥}}}}{}
\textcolor{teal}{\mytextsc{adverbe}} \hspace{4pt} Ton~: \#H.

Ces années-ci, actuellement.
\textcolor{brown}{\zh{这几年、现在这个时代。}}




\paragraph{\hspace{-0.5cm} {\Large \textcolor{darkblue}{\textbf{\ipa{ə˧lɑ˧}}}}} \hypertarget{@\string_MlA\string_M1}{}
\markboth{\textcolor{darkblue}{\textbf{\ipa{ə˧lɑ˧}}}}{}
\textcolor{teal}{\mytextsc{nom}} \hspace{4pt} Ton~: M.

Nom de clan/famille étendue. Trois familles portent ce nom à Yongning. C'est l'un des trois clans qui se sont établis les premiers dans le voisinage du monastère de Yongning, les deux autres étant \textcolor{darkblue}{\textbf{\ipa{/kɤ˧˥tʰɑ˩/}}} et \textcolor{darkblue}{\textbf{\ipa{/lɑ˧tʰɑ˧mi˥\$/}}}.
\textcolor{brown}{\zh{一个姓。这个姓,永宁有三家。}}


\begin{exe}
\sn \textcolor{darkblue}{\textbf{\ipa{ə˧lɑ˧=ɻ̍˩}}}
\trans le clan \textcolor{darkblue}{\textbf{\ipa{/ə˧lɑ˧/}}}
\trans \textit{\textcolor{brown}{\zh{\textcolor{darkblue}{\textbf{\ipa{/ə˧lɑ˧/}}}家族}}}
\end{exe}


\paragraph{\hspace{-0.5cm} {\Large \textcolor{darkblue}{\textbf{\ipa{ə˧lɑ˧-ʁwɤ\#˥}}}}} \hypertarget{@\string_MlA\string_M-Rw7\#\string_T1}{}
\markboth{\textcolor{darkblue}{\textbf{\ipa{ə˧lɑ˧-ʁwɤ\#˥}}}}{}
\textcolor{teal}{\mytextsc{nom}} \hspace{4pt} Ton~: \#H.

Un hameau de Yongning, proche du monastère (lieu de naissance de la consultante principale). Nom chinois: Alawa.
\textcolor{brown}{\zh{永宁寺旁边的村落(主合作人住的地方)。(音译:阿拉瓦,旧名:七家村,因为村落在1960年左右有七个家庭)。}}


\begin{exe}
\sn \textcolor{darkblue}{\textbf{\ipa{dʑɤ˩bv̩˧kɤ˧-sɑ˥ʁwɤ˩, | hi˩ʁwɤ˩-lo˥, | æ˩mi˧-ʁwɤ\#˥, | lɑ˧lo˧-ʁwɤ˥, | lɑ˧ŋwɤ˧, | bɤ˧tsʰo˧gv̩˥, | ə˧lɑ˧-ʁwɤ\#˥, | gæ˧ɻæ˩, | qʰæ˧tɕʰi˧, | tʰo˧ʈɯ\#˥}}}
\trans les dix villages comptant traditionnellement comme faisant partie de Yongning
\trans \textit{\textcolor{brown}{\zh{摩梭传统地理概念中,属于永宁的十个村落}}}
\end{exe}


\paragraph{\hspace{-0.5cm} {\Large \textcolor{darkblue}{\textbf{\ipa{ə˧mɑ˧}}}}} \hypertarget{@\string_MmA\string_M1}{}
\markboth{\textcolor{darkblue}{\textbf{\ipa{ə˧mɑ˧}}}}{}
\textcolor{teal}{\mytextsc{nom}} \hspace{4pt} Ton~: M.

Mère (terme d'adresse).
\textcolor{brown}{\zh{阿妈(孩子对母亲的称呼)。}}


 \mytextsc{clf}~: \textcolor{darkblue}{\textbf{\ipa{v̩˧}}} 

\paragraph{\hspace{-0.5cm} {\Large \textcolor{darkblue}{\textbf{\ipa{ə˧mi˧}}}}} \hypertarget{@\string_Mmi\string_M1}{}
\markboth{\textcolor{darkblue}{\textbf{\ipa{ə˧mi˧}}}}{}
\textcolor{teal}{\mytextsc{nom}} \hspace{4pt} Ton~: M.

Mère; le terme s'emploie aussi pour désigner les tantes.
\textcolor{brown}{\zh{母亲、姑母、姨母、伯母、叔母、大娘、婶、大妈、姨、伯母、舅母、大婶、大姨、阿姨、妗母、妗子、舅妈、婶母、婶娘、婶子、叔母、姨妈、姨母、姨娘。}}


\begin{exe}
\sn \textcolor{darkblue}{\textbf{\ipa{ə˧mi˧=ɻæ˩}}}
\trans \string_ \mytextsc{associatif}: les mères =les femmes de la génération supérieure
\trans \textit{\textcolor{brown}{\zh{母亲们 =长辈女性}}}
\end{exe}
 \mytextsc{clf}~: \textcolor{darkblue}{\textbf{\ipa{v̩˧}}} \textcolor{darkblue}{\textbf{\ipa{jɤ˧˥}}} 

\paragraph{\hspace{-0.5cm} {\Large \textcolor{darkblue}{\textbf{\ipa{ə˧mi˧-ɖɯ˩}}}}} \hypertarget{@\string_Mmi\string_M-d`M\string_B1}{}
\markboth{\textcolor{darkblue}{\textbf{\ipa{ə˧mi˧-ɖɯ˩}}}}{}
\textcolor{teal}{\mytextsc{nom}} \hspace{4pt} Ton~: L\#.

Tante maternelle (sœur aînée de la mère).
\textcolor{brown}{\zh{姨母 (比母亲大)。}}


 \mytextsc{clf}~: \textcolor{darkblue}{\textbf{\ipa{v̩˧}}} 

\paragraph{\hspace{-0.5cm} {\Large \textcolor{darkblue}{\textbf{\ipa{ə˧mi˧-mv̩˩}}}}} \hypertarget{@\string_Mmi\string_M-mv\string_=\string_B1}{}
\markboth{\textcolor{darkblue}{\textbf{\ipa{ə˧mi˧-mv̩˩}}}}{}
\textcolor{teal}{\mytextsc{nom}} \hspace{4pt} Ton~: \mytextsc{L}.

Mère et fille.
\textcolor{brown}{\zh{母女,母亲与女孩。}}




\paragraph{\hspace{-0.5cm} {\Large \textcolor{darkblue}{\textbf{\ipa{ə˧mi˧-tɕi˩}}}}} \hypertarget{@\string_Mmi\string_M-ts£i\string_B1}{}
\markboth{\textcolor{darkblue}{\textbf{\ipa{ə˧mi˧-tɕi˩}}}}{}
\textcolor{teal}{\mytextsc{nom}} \hspace{4pt} Ton~: L\#.

Tante (soeur cadette de la mère).
\textcolor{brown}{\zh{姨母 (比母亲小)。}}


 \mytextsc{clf}~: \textcolor{darkblue}{\textbf{\ipa{v̩˧}}} 

\paragraph{\hspace{-0.5cm} {\Large \textcolor{darkblue}{\textbf{\ipa{ə˧mi˧-ze˩mi˩}}}}} \hypertarget{@\string_Mmi\string_M-ze\string_Bmi\string_B1}{}
\markboth{\textcolor{darkblue}{\textbf{\ipa{ə˧mi˧-ze˩mi˩}}}}{}
\textcolor{teal}{\mytextsc{nom}} \hspace{4pt} Ton~: \mytextsc{L}.

Tante et nièce.
\textcolor{brown}{\zh{姑母、姨母、伯母、或叔母(=婶母)与侄女或外甥女。}}




\paragraph{\hspace{-0.5cm} {\Large \textcolor{darkblue}{\textbf{\ipa{ə˧mi˧-zo\#˥}}}}} \hypertarget{@\string_Mmi\string_M-zo\#\string_T1}{}
\markboth{\textcolor{darkblue}{\textbf{\ipa{ə˧mi˧-zo\#˥}}}}{}
\textcolor{teal}{\mytextsc{nom}} \hspace{4pt} Ton~: \#H.

Mère et fils.
\textcolor{brown}{\zh{母子。}}




\paragraph{\hspace{-0.5cm} {\Large \textcolor{darkblue}{\textbf{\ipa{ə˧mv̩˩}}}}} \hypertarget{@\string_Mmv\string_=\string_B1}{}
\markboth{\textcolor{darkblue}{\textbf{\ipa{ə˧mv̩˩}}}}{}
\textcolor{teal}{\mytextsc{nom}} \hspace{4pt} Ton~: L\#.

Aîné: grand frère, grande sœur (employé aussi entre cousins).
\textcolor{brown}{\zh{哥哥,姐姐(也指堂哥堂姐)。}}


\begin{exe}
\sn \textcolor{darkblue}{\textbf{\ipa{æ˧mv̩˩=ɻæ˩}}}
\trans \mytextsc{associatif}: les aînés dans la fratrie: sœurs et frères aînés
\trans \textit{\textcolor{brown}{\zh{联想复数:哥哥们、姐姐们}}}
\end{exe}
 \mytextsc{clf}~: \textcolor{darkblue}{\textbf{\ipa{v̩˧}}} 

\paragraph{\hspace{-0.5cm} {\Large \textcolor{darkblue}{\textbf{\ipa{ə˧mv̩˧-gi˥zɯ˩}}}}} \hypertarget{@\string_Mmv\string_=\string_M-gi\string_TzM\string_B1}{}
\markboth{\textcolor{darkblue}{\textbf{\ipa{ə˧mv̩˧-gi˥zɯ˩}}}}{}
\textcolor{teal}{\mytextsc{nom}} \hspace{4pt} Ton~: \#H-.

Frères, quel que soit leur âge (aînés ou cadets).
\textcolor{brown}{\zh{兄弟(哥哥们与弟弟们)。}}




\paragraph{\hspace{-0.5cm} {\Large \textcolor{darkblue}{\textbf{\ipa{ə˧mv̩˧-go˧mi˥}}}}} \hypertarget{@\string_Mmv\string_=\string_M-go\string_Mmi\string_T1}{}
\markboth{\textcolor{darkblue}{\textbf{\ipa{ə˧mv̩˧-go˧mi˥}}}}{}
\textcolor{teal}{\mytextsc{nom}} \hspace{4pt} Ton~: \mytextsc{H}\#.

Sœurs (aînées ou cadettes).
\textcolor{brown}{\zh{姐妹。}}




\paragraph{\hspace{-0.5cm} {\Large \textcolor{darkblue}{\textbf{\ipa{ə˧mv̩˥-tɕi˩}}}}} \hypertarget{@\string_Mmv\string_=\string_T-ts£i\string_B1}{}
\markboth{\textcolor{darkblue}{\textbf{\ipa{ə˧mv̩˥-tɕi˩}}}}{}
\textcolor{teal}{\mytextsc{adjectif}} \hspace{4pt} Ton~: H\#.

Petit, tout petit, riquiqui.
\textcolor{brown}{\zh{小、细小。}}


\begin{exe}
\sn \textcolor{darkblue}{\textbf{\ipa{ə˧mv̩˥-tɕi˩-gv̩˩}}}
\trans tout petit
\trans \textit{\textcolor{brown}{\zh{小、细小}}}
\end{exe}
\begin{exe}
\sn \textcolor{darkblue}{\textbf{\ipa{ə˧mv̩˥-tɕi˩-hĩ˩}}}
\trans tout petit
\trans \textit{\textcolor{brown}{\zh{细小的}}}
\end{exe}
\begin{exe}
\sn \textcolor{darkblue}{\textbf{\ipa{ə˧mv̩˥tɕi˩ | ɖɯ˧-kʰwɤ˥}}}
\trans un petit morceau
\trans \textit{\textcolor{brown}{\zh{一小块}}}
\end{exe}


\paragraph{\hspace{-0.5cm} {\Large \textcolor{darkblue}{\textbf{\ipa{ə˧ɲi˥\$}}}}} \hypertarget{@\string_MJi\string_T\$1}{}
\markboth{\textcolor{darkblue}{\textbf{\ipa{ə˧ɲi˥\$}}}}{}
\textcolor{teal}{\mytextsc{adverbe}} \hspace{4pt} Ton~: H\$.

Hier.
\textcolor{brown}{\zh{昨天。}}




\paragraph{\hspace{-0.5cm} {\Large \textcolor{darkblue}{\textbf{\ipa{ə˧ɲi˥-tsæ˩qæ˩}}}}} \hypertarget{@\string_MJi\string_T-ts\{\string_Bq\{\string_B1}{}
\markboth{\textcolor{darkblue}{\textbf{\ipa{ə˧ɲi˥-tsæ˩qæ˩}}}}{}
\textcolor{teal}{\mytextsc{nom}} \hspace{4pt} Ton~: H\#-L.

Auriculaire.
\textcolor{brown}{\zh{小指。}}


 \mytextsc{clf}~: \textcolor{darkblue}{\textbf{\ipa{ɭɯ˧}}} 

\paragraph{\hspace{-0.5cm} {\Large \textcolor{darkblue}{\textbf{\ipa{ə˧ɲi˧mɤ˩}}}}} \hypertarget{@\string_MJi\string_Mm7\string_B1}{}
\markboth{\textcolor{darkblue}{\textbf{\ipa{ə˧ɲi˧mɤ˩}}}}{}
\textcolor{teal}{\mytextsc{nom}} \hspace{4pt} Ton~: .

Nonne bouddhiste.
\textcolor{brown}{\zh{尼姑。}}


\begin{exe}
\sn \textcolor{darkblue}{\textbf{\ipa{[F5] mi˩zɯ˩-ʈæ˩bɤ˥}}}
\trans femme-prêtre
\trans \textit{\textcolor{brown}{\zh{尼姑}}}
\end{exe}


\paragraph{\hspace{-0.5cm} {\Large \textcolor{darkblue}{\textbf{\ipa{ə˧ɲi˧-tsʰi˧ɲi\#˥}}}}} \hypertarget{@\string_MJi\string_M-ts\string_hi\string_MJi\#\string_T1}{}
\markboth{\textcolor{darkblue}{\textbf{\ipa{ə˧ɲi˧-tsʰi˧ɲi\#˥}}}}{}
\textcolor{teal}{\mytextsc{adverbe}} \hspace{4pt} Ton~: \#H.

Ces temps-ci, ces jours-ci.
\textcolor{brown}{\zh{近来。}}




\paragraph{\hspace{-0.5cm} {\Large \textcolor{darkblue}{\textbf{\ipa{ə˧pʰv̩˧}}}}} \hypertarget{@\string_Mp\string_hv\string_=\string_M1}{}
\markboth{\textcolor{darkblue}{\textbf{\ipa{ə˧pʰv̩˧}}}}{}
\textcolor{teal}{\mytextsc{nom}} \hspace{4pt} Ton~: M.

Frère de la grand-mère = oncle de l'oncle maternel = oncle de la mère; sens étendu: personnage masculin important 2 générations au-dessus de soi.
\textcolor{brown}{\zh{舅姥爷:姥姥的哥哥或弟弟(也就是母亲的舅舅)。泛指:“祖父”。}}


 \mytextsc{clf}~: \textcolor{darkblue}{\textbf{\ipa{v̩˧}}} 

\paragraph{\hspace{-0.5cm} {\Large \textcolor{darkblue}{\textbf{\ipa{ə˧si˧}}}}} \hypertarget{@\string_Msi\string_M1}{}
\markboth{\textcolor{darkblue}{\textbf{\ipa{ə˧si˧}}}}{}
\textcolor{teal}{\mytextsc{nom}} \hspace{4pt} Ton~: M.

Arrière-grand-mère (bisaïeule); sens étendu: bisaïeule et ses frères et soeurs, c'est-à-dire les membres importants de la famille à la 3e génération.
\textcolor{brown}{\zh{祖母。泛指:祖母与其兄弟姐妹。}}


 \mytextsc{clf}~: \textcolor{darkblue}{\textbf{\ipa{v̩˧}}} 

\paragraph{\hspace{-0.5cm} {\Large \textcolor{darkblue}{\textbf{\ipa{ə˧si˧-ə˧pʰv̩\#˥}}}}} \hypertarget{@\string_Msi\string_M-@\string_Mp\string_hv\string_=\#\string_T1}{}
\markboth{\textcolor{darkblue}{\textbf{\ipa{ə˧si˧-ə˧pʰv̩\#˥}}}}{}
\textcolor{teal}{\mytextsc{nom}} \hspace{4pt} Ton~: \#H.

Ancêtres aux 3e et 4e générations.
\textcolor{brown}{\zh{祖宗(三、四代以前)。}}




\paragraph{\hspace{-0.5cm} {\Large \textcolor{darkblue}{\textbf{\ipa{ə˧so˧}}}}} \hypertarget{@\string_Mso\string_M1}{}
\markboth{\textcolor{darkblue}{\textbf{\ipa{ə˧so˧}}}}{}
\textcolor{teal}{\mytextsc{adverbe}} \hspace{4pt} Ton~: M.

Tout à l'heure, il y a un moment.
\textcolor{brown}{\zh{刚才。}}




\paragraph{\hspace{-0.5cm} {\Large \textcolor{darkblue}{\textbf{\ipa{ə˧-sɯ˩kv̩˩}}}}} \hypertarget{@\string_M-sM\string_Bkv\string_=\string_B1}{}
\markboth{\textcolor{darkblue}{\textbf{\ipa{ə˧-sɯ˩kv̩˩}}}}{}
\textcolor{teal}{\mytextsc{pronom}} \hspace{4pt} Ton~: \mytextsc{L} / LMH.

1e pers. pl., inclusive.
\textcolor{brown}{\zh{咱们。}}


\begin{exe}
\sn \textcolor{darkblue}{\textbf{\ipa{ə˧-sɯ˩kv̩˩ | ɖɯ˧-tɑ˧˥ | nɑ˩ ɲi˧!}}}
\trans On est tous Na!
\trans \textit{\textcolor{brown}{\zh{咱们都是摩梭人!}}}
\end{exe}
\begin{exe}
\sn \textcolor{darkblue}{\textbf{\ipa{ə˧-sɯ˩kv̩˩ lɑ˩ ɲi˩!}}}
\trans On est entre nous! (Contexte: une grand-mère encourage sa petite-fille d'un an à déféquer, alors qu'il y a du public, ce qui pourrait être intimidant; elle la rassure: ``C'est la famille/on est entre nous!")
\trans \textit{\textcolor{brown}{\zh{只有我们!(=没有人在看!)(情景:一位奶奶鼓励一岁的小孙女拉屎。周围有人,这可能会让小姑娘害羞,所以奶奶安慰她:“都是家人,可以轻松!”)}}}
\end{exe}


\paragraph{\hspace{-0.5cm} {\Large \textcolor{darkblue}{\textbf{\ipa{ə˧ʂɯ˧ɲi˥-ɖɯ˧ɲi˥}}}}} \hypertarget{@\string_Ms`M\string_MJi\string_T-d`M\string_MJi\string_T1}{}
\markboth{\textcolor{darkblue}{\textbf{\ipa{ə˧ʂɯ˧ɲi˥-ɖɯ˧ɲi˥}}}}{}
\textcolor{teal}{\mytextsc{adverbe}} \hspace{4pt} Ton~: H\#-H\$.

Ces derniers jours, les jours passés.
\textcolor{brown}{\zh{前几天。}}




\paragraph{\hspace{-0.5cm} {\Large \textcolor{darkblue}{\textbf{\ipa{ə˧ti˥-dzi˩}}}}} \hypertarget{@\string_Mti\string_T-dzi\string_B1}{}
\markboth{\textcolor{darkblue}{\textbf{\ipa{ə˧ti˥-dzi˩}}}}{}
\textcolor{teal}{\mytextsc{nom}} \hspace{4pt} Ton~: H\#-.

Weixi (localité du Yunnan).
\textcolor{brown}{\zh{维西。}}




\paragraph{\hspace{-0.5cm} {\Large \textcolor{darkblue}{\textbf{\ipa{ə˧tɕi˩}}}}} \hypertarget{@\string_Mts£i\string_B1}{}
\markboth{\textcolor{darkblue}{\textbf{\ipa{ə˧tɕi˩}}}}{}
\textcolor{teal}{\mytextsc{nom}} \hspace{4pt} Ton~: L\#.

Tante (soeur cadette de la mère).
\textcolor{brown}{\zh{姨母 (比母亲小)。}}


\begin{exe}
\sn \textcolor{darkblue}{\textbf{\ipa{ə˧tɕi˩=ɻæ˩}}}
\trans \mytextsc{associatif}: les tantes
\trans \textit{\textcolor{brown}{\zh{姨母们}}}
\end{exe}
 \mytextsc{clf}~: \textcolor{darkblue}{\textbf{\ipa{v̩˧}}} 

\paragraph{\hspace{-0.5cm} {\Large \textcolor{darkblue}{\textbf{\ipa{ə˧tse˥\$}}}}} \hypertarget{@\string_Mtse\string_T\$1}{}
\markboth{\textcolor{darkblue}{\textbf{\ipa{ə˧tse˥\$}}}}{}
\textcolor{teal}{\mytextsc{adverbe}} \hspace{4pt} Ton~: H\$.

Pourquoi.
\textcolor{brown}{\zh{为什么。}}


\begin{exe}
\sn \textcolor{darkblue}{\textbf{\ipa{ə˧tse˧-ʝi˥ / ə˧tse˧-ʝi˧}}}
\trans Pourquoi? (Deux variantes, même sens)
\trans \textit{\textcolor{brown}{\zh{为什么?(有两个变体,意思一致)}}}
\end{exe}
\begin{exe}
\sn \textcolor{darkblue}{\textbf{\ipa{no˧ | ə˧tse˧-ʝi˥ | mɤ˧-tsʰɯ˩ ɲi˩? / no˧ | ə˧tse˧-ʝi˧-zo˥ | mɤ˧-tsʰɯ˩ ɲi˩?}}}
\trans pourquoi tu ne viens pas/n'es pas venu?
\trans \textit{\textcolor{brown}{\zh{你为什么没有来?}}}
\end{exe}
\begin{exe}
\sn \textcolor{darkblue}{\textbf{\ipa{no˧ | ə˧tse˧-ʝi˥ | mɤ˧-dzɯ˥? = no˧ | ə˧tse˧-ʝi˥ | mɤ˧-dzɯ˧-ɲi˥?}}}
\trans pourquoi tu ne manges pas?
\trans \textit{\textcolor{brown}{\zh{你为什么不吃?}}}
\end{exe}
\begin{exe}
\sn \textcolor{darkblue}{\textbf{\ipa{ʈʂʰɯ˧ | ə˧tse˧-ɲi˥-hɯ˩?}}}
\trans qu'est-ce que ça veut dire?
\trans \textit{\textcolor{brown}{\zh{这是怎么一回事?}}}
\end{exe}


\paragraph{\hspace{-0.5cm} {\Large \textcolor{darkblue}{\textbf{\ipa{ə˧tso˧}}}}} \hypertarget{@\string_Mtso\string_M1}{}
\markboth{\textcolor{darkblue}{\textbf{\ipa{ə˧tso˧}}}}{}
\textcolor{teal}{\mytextsc{pronom}} \hspace{4pt} Ton~: M.

\mytextsc{interrog}.quoi (quoi, pronom interrogatif).
\textcolor{brown}{\zh{什么。}}


\begin{exe}
\sn \textcolor{darkblue}{\textbf{\ipa{ə˧tso˧ ɲi˩?}}}
\trans Qu'est-ce que c'est?
\trans \textit{\textcolor{brown}{\zh{是什么?}}}
\end{exe}
\begin{exe}
\sn \textcolor{darkblue}{\textbf{\ipa{no˧ | ə˧tso˧ ʝi˧ bi˧?}}}
\trans Qu'est-ce que tu vas faire? (Cette phrase peut se substituer à ``où tu vas?", \textcolor{darkblue}{\textbf{\ipa{/zo˩qo˧ bi˧?/}}}, comme salutation adressée à quelqu'un qui est en chemin)
\trans \textit{\textcolor{brown}{\zh{你要做什么?}}}
\end{exe}
\begin{exe}
\sn \textcolor{darkblue}{\textbf{\ipa{no˧ | ə˧tse˧ bi˧?}}}
\trans forme contractée de 2: dans \textcolor{darkblue}{\textbf{\ipa{[tse˧]}}}, \textcolor{darkblue}{\textbf{\ipa{/tso˧/}}} et le \textcolor{darkblue}{\textbf{\ipa{/ʝi˧/}}} suivant fusionnent en une seule syllabe.
\trans \textit{\textcolor{brown}{\zh{上面例子的缩短格式:\textcolor{darkblue}{\textbf{\ipa{/tso˧/}}}与\textcolor{darkblue}{\textbf{\ipa{/ʝi˧/}}}合成一个音节,\textcolor{darkblue}{\textbf{\ipa{[tse˧]}}}。}}}
\end{exe}


\paragraph{\hspace{-0.5cm} {\Large \textcolor{darkblue}{\textbf{\ipa{ə˧tso˧\textasciitilde{}ə˧tso˥}}}}} \hypertarget{@\string_Mtso\string_M~@\string_Mtso\string_T1}{}
\markboth{\textcolor{darkblue}{\textbf{\ipa{ə˧tso˧\textasciitilde{}ə˧tso˥}}}}{}
\textcolor{teal}{\mytextsc{pronom}} \hspace{4pt} Ton~: H\#.

\mytextsc{interrog}.quoi, rédupliqué.
\textcolor{brown}{\zh{什么(重叠)。}}




\paragraph{\hspace{-0.5cm} {\Large \textcolor{darkblue}{\textbf{\ipa{ə˧tso˧-mɤ˧-ɲi˩}}}}} \hypertarget{@\string_Mtso\string_M-m7\string_M-Ji\string_B1}{}
\markboth{\textcolor{darkblue}{\textbf{\ipa{ə˧tso˧-mɤ˧-ɲi˩}}}}{}
\textcolor{teal}{\mytextsc{pronom}} \hspace{4pt} Ton~: L\#.

Tout, toutes les sortes de.
\textcolor{brown}{\zh{各种。}}




\paragraph{\hspace{-0.5cm} {\Large \textcolor{darkblue}{\textbf{\ipa{ə˧v̩˧˥}}} \textsubscript{1}}} \hypertarget{@\string_Mv\string_=\string_M\string_T1}{}
\markboth{\textcolor{darkblue}{\textbf{\ipa{ə˧v̩˧˥}}} \textsubscript{1}}{}
\textcolor{teal}{\mytextsc{adjectif}} \hspace{4pt} Ton~: MH\#.

Beau, joli.
\textcolor{brown}{\zh{美,好看,美丽。}}


\begin{exe}
\sn \textcolor{darkblue}{\textbf{\ipa{ɖwæ˧˥ | ə˧v̩˧˥}}}
\trans \mytextsc{intensif}.très
\trans \textit{\textcolor{brown}{\zh{很好看!}}}
\end{exe}
\begin{exe}
\sn \textcolor{darkblue}{\textbf{\ipa{ə˧-mɤ˧-v̩˧˥}}}
\trans \mytextsc{neg}: vilain, laid
\trans \textit{\textcolor{brown}{\zh{丑陋}}}
\end{exe}


\paragraph{\hspace{-0.5cm} {\Large \textcolor{darkblue}{\textbf{\ipa{ə˧v̩˧˥}}} \textsubscript{2}}} \hypertarget{@\string_Mv\string_=\string_M\string_T2}{}
\markboth{\textcolor{darkblue}{\textbf{\ipa{ə˧v̩˧˥}}} \textsubscript{2}}{}
\textcolor{teal}{\mytextsc{nom}} \hspace{4pt} Ton~: MH\#.

Oncle maternel =frère de la mère (aîné ou cadet).
\textcolor{brown}{\zh{舅舅、舅父 (比母亲大或比母亲小不区分)。}}


\begin{exe}
\sn \textcolor{darkblue}{\textbf{\ipa{ə˧v̩˧-ɖɯ˧˥}}}
\trans oncle, aîné de la mère
\trans \textit{\textcolor{brown}{\zh{比母亲大的舅舅}}}
\end{exe}
\begin{exe}
\sn \textcolor{darkblue}{\textbf{\ipa{ə˧v̩˧-tɕi˥}}}
\trans oncle, cadet de la mère
\trans \textit{\textcolor{brown}{\zh{比母亲小的舅舅}}}
\end{exe}
\begin{exe}
\sn \textcolor{darkblue}{\textbf{\ipa{mv˧ʁo˥ | tʰi˧-dze˩, | kɤ˩-nɑ˧mi˧ ɖɯ˧˥ ! | di˧qo˧ ʈʰɯ˧-dʑo˩, | ə˧v˧ ɖɯ˧˥!}}}
\trans “Parmi tout ce qui vole dans le ciel, l'aigle est le plus grand; parmi tout ce qui marche sur la terre, l'oncle est le plus grand.”
\trans \textit{\textcolor{brown}{\zh{“天上飞的,是老鹰最大。天下走的,是舅舅最大。”}}}
\end{exe}
\begin{exe}
\sn \textcolor{darkblue}{\textbf{\ipa{mv˧ʁo˥ dze˩hĩ˩-dʑo˥, | kɤ˩-nɑ˧mi˧; | di˧qo˧ se˧-dʑo˩, | ə˧v˧˥!}}}
\trans “Parmi tout ce qui vole dans le ciel, l'aigle est le plus grand; parmi tout ce qui marche sur la terre, l'oncle est le plus grand.”
\trans \textit{\textcolor{brown}{\zh{“天上飞的,是老鹰最大。天下走的,是舅舅最大。”}}}
\end{exe}
\begin{exe}
\sn \textcolor{darkblue}{\textbf{\ipa{mv˧ʁo˥ dze˩hĩ˩˥ | -dʑo˥, | kɤ˩-nɑ˧mi˧; | di˧qo˧ se˧-dʑo˩, | ə˧v˧˥!}}}
\trans “Parmi tout ce qui vole dans le ciel, l'aigle est le plus grand; parmi tout ce qui marche sur la terre, l'oncle est le plus grand.”
\trans \textit{\textcolor{brown}{\zh{“天上飞的,是老鹰最大。天下走的,是舅舅最大。”}}}
\end{exe}
\begin{exe}
\sn \textcolor{darkblue}{\textbf{\ipa{mv˧ʁo˥ dze˩hĩ˩-dʑo˥, | kɤ˩-nɑ˧mi˧; | di˧qo˧-dʑo˧, | ə˧v˧˥!}}}
\trans “Parmi tout ce qui vole dans le ciel, l'aigle est le plus grand; parmi tout ce qui marche sur la terre, l'oncle est le plus grand.”
\trans \textit{\textcolor{brown}{\zh{“天上飞的,是老鹰最大。天下走的,是舅舅最大。”}}}
\end{exe}
\begin{exe}
\sn \textcolor{darkblue}{\textbf{\ipa{mv˧ʁo˥ | dze˩-hĩ˩-dʑo˥, | ɖɯ˩-hĩ˩-dʑo˥, | kɤ˩-nɑ˧mi˧! | mv˧di˧-qo˥ | ɖɯ˩-hĩ˩-dʑo˥, | ə˧v˧˥!}}}
\trans “Parmi tout ce qui vole dans le ciel, l'aigle est le plus grand; parmi tout ce qui marche sur la terre, l'oncle est le plus grand.”
\trans \textit{\textcolor{brown}{\zh{“天上飞的,是老鹰最大。天下走的,是舅舅最大。”}}}
\end{exe}
 \mytextsc{clf}~: \textcolor{darkblue}{\textbf{\ipa{v̩˧}}} 

\paragraph{\hspace{-0.5cm} {\Large \textcolor{darkblue}{\textbf{\ipa{ə˧v̩˧-ze˥v̩˩}}}}} \hypertarget{@\string_Mv\string_=\string_M-ze\string_Tv\string_=\string_B1}{}
\markboth{\textcolor{darkblue}{\textbf{\ipa{ə˧v̩˧-ze˥v̩˩}}}}{}
\textcolor{teal}{\mytextsc{nom}} \hspace{4pt} Ton~: H\#-.

Oncle et neveu.
\textcolor{brown}{\zh{叔叔侄子。}}




\paragraph{\hspace{-0.5cm} {\Large \textcolor{darkblue}{\textbf{\ipa{ə˧ze˧}}}}} \hypertarget{@\string_Mze\string_M1}{}
\markboth{\textcolor{darkblue}{\textbf{\ipa{ə˧ze˧}}}}{}
\textcolor{teal}{\mytextsc{adverbe}} \hspace{4pt} Ton~: H\#.

Lentement, doucement.
\textcolor{brown}{\zh{慢慢地。}}


\begin{exe}
\sn \textcolor{darkblue}{\textbf{\ipa{ə˧ze˧ le˧-hõ˩!}}}
\trans Salutations à quelqu'un qui s'en va: ``Au revoir!", littéralement ``Marche doucement!"
\trans \textit{\textcolor{brown}{\zh{慢走!}}}
\end{exe}
\begin{exe}
\sn \textcolor{darkblue}{\textbf{\ipa{ə˧ze˧ le˧-dzi˩!}}}
\trans Salutation lorsqu'on quitte quelqu'un: 'Au revoir!', littéralement 'Reste tranquillement assis!'
\trans \textit{\textcolor{brown}{\zh{慢慢坐!}}}
\end{exe}
\textit{Voir~:} \hyperlink{@\string_M-dz7\string_T\$1}{\textcolor{darkblue}{\textbf{\ipa{ə˧-dzɤ˥\$}}}} 

\paragraph{\hspace{-0.5cm} {\Large \textcolor{darkblue}{\textbf{\ipa{ə˧zo˩-ʁwɤ˩}}}}} \hypertarget{@\string_Mzo\string_B-Rw7\string_B1}{}
\markboth{\textcolor{darkblue}{\textbf{\ipa{ə˧zo˩-ʁwɤ˩}}}}{}
\textcolor{teal}{\mytextsc{nom}} \hspace{4pt} Ton~: L\#-.

Un village proche des Sources Chaudes.
\textcolor{brown}{\zh{温泉乡的一个村落。}}




\paragraph{\hspace{-0.5cm} {\Large \textcolor{darkblue}{\textbf{\ipa{ə˧=zɯ˩}}}}} \hypertarget{@\string_M=zM\string_B1}{}
\markboth{\textcolor{darkblue}{\textbf{\ipa{ə˧=zɯ˩}}}}{}
\textcolor{teal}{\mytextsc{pronom}} \hspace{4pt} Ton~: L\# / L.

Pronom duel inclusif: nous deux (le locuteur et l'interlocuteur).
\textcolor{brown}{\zh{咱们两个。}}




\paragraph{\hspace{-0.5cm} {\Large \textcolor{darkblue}{\textbf{\ipa{ə˧ʐv̩˩}}}}} \hypertarget{@\string_Mz`v\string_=\string_B1}{}
\markboth{\textcolor{darkblue}{\textbf{\ipa{ə˧ʐv̩˩}}}}{}
\textcolor{teal}{\mytextsc{adjectif}} \hspace{4pt} Ton~: L\#.

Ancien, usagé.
\textcolor{brown}{\zh{陈旧。}}


\begin{exe}
\sn \textcolor{darkblue}{\textbf{\ipa{ʂe˧ ʐv̩˥}}}
\trans de la vieille viande, de la viande pas fraîche
\trans \textit{\textcolor{brown}{\zh{陈肉、不新鲜的肉}}}
\end{exe}


\paragraph{\hspace{-0.5cm} {\Large \textcolor{darkblue}{\textbf{\ipa{ə˧ʑi˧-ə˧pʰv̩˧˥}}}}} \hypertarget{@\string_Mz£i\string_M-@\string_Mp\string_hv\string_=\string_M\string_T1}{}
\markboth{\textcolor{darkblue}{\textbf{\ipa{ə˧ʑi˧-ə˧pʰv̩˧˥}}}}{}
\textcolor{teal}{\mytextsc{nom}} \hspace{4pt} Ton~: MH\#.

La grand-mère et ses frères: les aînés 2 génération au-dessus de soi.
\textcolor{brown}{\zh{奶奶与她的兄弟。}}




\paragraph{\hspace{-0.5cm} {\Large \textcolor{darkblue}{\textbf{\ipa{ə˧ʑi˧-ʐv̩˥mi˩}}}}} \hypertarget{@\string_Mz£i\string_M-z`v\string_=\string_Tmi\string_B1}{}
\markboth{\textcolor{darkblue}{\textbf{\ipa{ə˧ʑi˧-ʐv̩˥mi˩}}}}{}
\textcolor{teal}{\mytextsc{nom}} \hspace{4pt} Ton~: H\#-.

Grand-mère et petite-fille.
\textcolor{brown}{\zh{奶奶与孙女。}}




\paragraph{\hspace{-0.5cm} {\Large \textcolor{darkblue}{\textbf{\ipa{ə˧ʑi˧˥}}}}} \hypertarget{@\string_Mz£i\string_M\string_T1}{}
\markboth{\textcolor{darkblue}{\textbf{\ipa{ə˧ʑi˧˥}}}}{}
\textcolor{teal}{\mytextsc{nom}} \hspace{4pt} Ton~: MH\#.

Grand-mère, aïeule; vieille femme.
\textcolor{brown}{\zh{祖母,姥姥,老妪。}}


\begin{exe}
\sn \textcolor{darkblue}{\textbf{\ipa{ə˧ʑi˧ ʝi˧ so˥-zo˩-ho˩-ze˩!}}}
\trans Il va falloir que j'apprenne à me comporter (sagement) comme une grand-mère! (Contexte: remarque teintée d'humour de la consultante principale face à l'âge qui vient et ses soucis: un médecin lui déconseille les sofas/assises molles et lui recommande une chaise haute en bois; elle se fait la réflexion qu'elle a vieilli et doit maintenant apprendre à prendre des précautions.)
\trans \textit{\textcolor{brown}{\zh{我要学习当老太太了!(情景:一位医生建议合作人不要坐在小凳子或者软沙发上了,而要坐更高的木头椅子。她幽默地说:“看来我是老年人了!”)}}}
\end{exe}
 \mytextsc{clf}~: \textcolor{darkblue}{\textbf{\ipa{v̩˧}}} 

\paragraph{\hspace{-0.5cm} {\Large \textcolor{darkblue}{\textbf{\ipa{}}}}} \hypertarget{@\string_B‑1}{}
\markboth{\textcolor{darkblue}{\textbf{\ipa{}}}}{}
\textcolor{teal}{\mytextsc{pronom}} \hspace{4pt} Ton~: L.

Interrogation totale.
\textcolor{brown}{\zh{……吗?。}}


\begin{exe}
\sn \textcolor{darkblue}{\textbf{\ipa{dʑɯ˧ | ə˩-dʑo˧?}}}
\trans est-ce qu’il y a de l’eau ?
\trans \textit{\textcolor{brown}{\zh{有谁吗?}}}
\end{exe}
\begin{exe}
\sn \textcolor{darkblue}{\textbf{\ipa{ə˩-ŋi˩˥ ?}}}
\trans Est-ce que c’est ça ?/ C'est bien ça? ... n’est-ce pas ?
\trans \textit{\textcolor{brown}{\zh{对吗? / 对吧?}}}
\end{exe}


\paragraph{\hspace{-0.5cm} {\Large \textcolor{darkblue}{\textbf{\ipa{ə˩kʰɯ˩}}}}} \hypertarget{@\string_Bk\string_hM\string_B1}{}
\markboth{\textcolor{darkblue}{\textbf{\ipa{ə˩kʰɯ˩}}}}{}
\textcolor{teal}{\mytextsc{nom}} \hspace{4pt} Ton~: L.

Navet, \textit{Brassica rapa}.
\textcolor{brown}{\zh{芜菁 、扁萝卜、大头菜、蔓菁。}}


\begin{exe}
\sn \textcolor{darkblue}{\textbf{\ipa{ə˩kʰɯ˩-bv̩˧ | kʰɯ˩ʈɯ˩˥}}}
\trans racine de navet
\trans \textit{\textcolor{brown}{\zh{芜菁的根}}}
\end{exe}
 \mytextsc{clf}~: \textcolor{darkblue}{\textbf{\ipa{ɭɯ˧}}} 

\paragraph{\hspace{-0.5cm} {\Large \textcolor{darkblue}{\textbf{\ipa{ə˩ljɤ˩hæ̃˩ʂɯ˥-mo˩}}}}} \hypertarget{@\string_Blj7\string_Bh\{\string_~\string_Bs`M\string_T-mo\string_B1}{}
\markboth{\textcolor{darkblue}{\textbf{\ipa{ə˩ljɤ˩hæ̃˩ʂɯ˥-mo˩}}}}{}
\textcolor{teal}{\mytextsc{nom}} \hspace{4pt} Ton~: L+H\#-.

Grand champignon jaune vif, comestible: \textit{Hygrophorus lucorum Kalc hbr.}. Littéralement ``champignon doré".
\textcolor{brown}{\zh{柠檬黄蜡伞(一种菌子)。}}Dialecte chinois local~: \zh{黄蜡伞。}




\paragraph{\hspace{-0.5cm} {\Large \textcolor{darkblue}{\textbf{\ipa{ə˩qo˥}}}}} \hypertarget{@\string_Bqo\string_T1}{}
\markboth{\textcolor{darkblue}{\textbf{\ipa{ə˩qo˥}}}}{}
\textcolor{teal}{\mytextsc{adverbe}} \hspace{4pt} Ton~: LH.

À l'intérieur, vers l'intérieur.
\textcolor{brown}{\zh{往里。}}




\newpage
\section*{\centering- \textcolor{darkblue}{\textbf{\ipa{f}}} -}
\paragraph{\hspace{-0.5cm} {\Large \textcolor{darkblue}{\textbf{\ipa{fɑ˧tɑ˧˥}}}}} \hypertarget{fA\string_MtA\string_M\string_T1}{}
\markboth{\textcolor{darkblue}{\textbf{\ipa{fɑ˧tɑ˧˥}}}}{}
\textcolor{teal}{\mytextsc{adjectif}} \hspace{4pt} Ton~: MH.

Développé, florissant.
\textcolor{brown}{\zh{发达。}}

 Emprunt~: chinois \textcolor{brown}{\zh{发达}}

\begin{exe}
\sn \textcolor{darkblue}{\textbf{\ipa{fɑ˧tɑ˧-ze˥}}}
\trans \mytextsc{pfv}
\trans \textit{\textcolor{brown}{\zh{很发达的了}}}
\end{exe}


\paragraph{\hspace{-0.5cm} {\Large \textcolor{darkblue}{\textbf{\ipa{fɑ˩\textsubscript{a}}}}}} \hypertarget{fA\string_Ba1}{}
\markboth{\textcolor{darkblue}{\textbf{\ipa{fɑ˩\textsubscript{a}}}}}{}
\textcolor{teal}{\mytextsc{verbe}} \hspace{4pt} Ton~: L\textsubscript{a}.

Fermenter.
\textcolor{brown}{\zh{发酵(汉语借词:发)。}}

 Emprunt~: chinois \textcolor{brown}{\zh{发(酵)}}

\begin{exe}
\sn \textcolor{darkblue}{\textbf{\ipa{tsɑ˧bɤ˧ ɖɯ˧-mɤ˩ | tʰi˧-fɑ˩}}}
\trans faire lever un peu de farine
\trans \textit{\textcolor{brown}{\zh{发一点面}}}
\end{exe}
\begin{exe}
\sn \textcolor{darkblue}{\textbf{\ipa{tsɑ˧bɤ˧ tʰi˧-fɑ˩! | pɤ˩jɤ˧ gv̩˥-bi˩!}}}
\trans Fais lever de la farine, on va faire des petits pains!
\trans \textit{\textcolor{brown}{\zh{你发一点面吧!要做馒头!}}}
\end{exe}


\paragraph{\hspace{-0.5cm} {\Large \textcolor{darkblue}{\textbf{\ipa{fæ˧}}}}} \hypertarget{f\{\string_M1}{}
\markboth{\textcolor{darkblue}{\textbf{\ipa{fæ˧}}}}{}
\textcolor{teal}{\mytextsc{nom}} \hspace{4pt} Ton~: M.

Direction.
\textcolor{brown}{\zh{方(方向的方)(汉语借词)。}}

 Emprunt~: chinois \textcolor{brown}{\zh{方}}

\begin{exe}
\sn \textcolor{darkblue}{\textbf{\ipa{dv̩˩tɕo˧ fæ˧}}}
\trans cette direction-là
\trans \textit{\textcolor{brown}{\zh{那个方向}}}
\end{exe}
\textit{Voir~:} \hyperlink{d7\string_M-t\string_hv\string_=\string_M-gi\#\string_T1}{\textcolor{darkblue}{\textbf{\ipa{dɤ˧-tʰv̩˧-gi\#˥}}}} 

\paragraph{\hspace{-0.5cm} {\Large \textcolor{darkblue}{\textbf{\ipa{fv̩˩˧}}}}} \hypertarget{fv\string_=\string_B\string_M1}{}
\markboth{\textcolor{darkblue}{\textbf{\ipa{fv̩˩˧}}}}{}
\textcolor{teal}{\mytextsc{nom}} \hspace{4pt} Ton~: LM.

Le voisinage, les voisins.
\textcolor{brown}{\zh{邻居,村里的人们。}}




\paragraph{\hspace{-0.5cm} {\Large \textcolor{darkblue}{\textbf{\ipa{fv̩˧}}}}} \hypertarget{fv\string_=\string_M1}{}
\markboth{\textcolor{darkblue}{\textbf{\ipa{fv̩˧}}}}{}
\textcolor{teal}{\mytextsc{adjectif}} \hspace{4pt} Ton~: M.

Content, joyeux; agréable; aimer, apprécier.
\textcolor{brown}{\zh{高兴、起劲,喜欢、爱、愿意。}}


\begin{exe}
\sn \textcolor{darkblue}{\textbf{\ipa{ɖwæ˧˥ | fv̩˧}}}
\trans \mytextsc{intensif}.très: très content, tout content
\trans \textit{\textcolor{brown}{\zh{很高兴}}}
\end{exe}
\begin{exe}
\sn \textcolor{darkblue}{\textbf{\ipa{dʑɤ˩˥ | fv̩˧}}}
\trans très content, tout content
\trans \textit{\textcolor{brown}{\zh{很高兴}}}
\end{exe}
\begin{exe}
\sn \textcolor{darkblue}{\textbf{\ipa{mɤ˧-fv̩˧ ʝi˧}}}
\trans se mettre en colère, s'énerver
\trans \textit{\textcolor{brown}{\zh{生气}}}
\end{exe}
\begin{exe}
\sn \textcolor{darkblue}{\textbf{\ipa{ʈʂʰɯ˧ mɤ˧-fv̩˧ ʝi˧!}}}
\trans Il/elle est mécontent(e) / en colère.
\trans \textit{\textcolor{brown}{\zh{他在生气。}}}
\end{exe}
\begin{exe}
\sn \textcolor{darkblue}{\textbf{\ipa{[F5] ɖwæ˧˥ | fv̩˧hĩ˧ ɖɯ˧-v̩˧ ɲi˩}}}
\trans c'est quelqu'un de très agréable
\trans \textit{\textcolor{brown}{\zh{他是很善良的人。}}}
\end{exe}


\paragraph{\hspace{-0.5cm} {\Large \textcolor{darkblue}{\textbf{\ipa{fv̩˩bi˩}}}}} \hypertarget{fv\string_=\string_Bbi\string_B1}{}
\markboth{\textcolor{darkblue}{\textbf{\ipa{fv̩˩bi˩}}}}{}
\textcolor{teal}{\mytextsc{nom}} \hspace{4pt} Ton~: L.

Contrée, voisinage, ensemble de villages où habitent des gens de la famille étendue.
\textcolor{brown}{\zh{邻里、邻村:大家族居住的那片地方,包括几个小村落。}}




\paragraph{\hspace{-0.5cm} {\Large \textcolor{darkblue}{\textbf{\ipa{fv̩˧kʰo˥}}}}} \hypertarget{fv\string_=\string_Mk\string_ho\string_T1}{}
\markboth{\textcolor{darkblue}{\textbf{\ipa{fv̩˧kʰo˥}}}}{}
\textcolor{teal}{\mytextsc{nom}} \hspace{4pt} Ton~: H\#.

Fengke: village situé au bord du Yang-tsé, sur la rive droite.
\textcolor{brown}{\zh{奉科(金沙江边的一个地区)。}}




\paragraph{\hspace{-0.5cm} {\Large \textcolor{darkblue}{\textbf{\ipa{fv̩˧ʂɯ˩}}}}} \hypertarget{fv\string_=\string_Ms`M\string_B1}{}
\markboth{\textcolor{darkblue}{\textbf{\ipa{fv̩˧ʂɯ˩}}}}{}
\textcolor{teal}{\mytextsc{nom}} \hspace{4pt} Ton~: L\#.

Rhumatismes.
\textcolor{brown}{\zh{风湿(汉语借词)。}}

 Emprunt~: chinois \textcolor{brown}{\zh{风湿}}

\begin{exe}
\sn \textcolor{darkblue}{\textbf{\ipa{fv̩˧ʂɯ˩ go˩}}}
\trans souffrir de rhumatismes, avoir des rhumatismes
\trans \textit{\textcolor{brown}{\zh{有风湿、得风湿}}}
\end{exe}


\newpage
\section*{\centering- \textcolor{darkblue}{\textbf{\ipa{g}}} -}
\paragraph{\hspace{-0.5cm} {\Large \textcolor{darkblue}{\textbf{\ipa{gæ˧ɻæ˩}}}}} \hypertarget{g\{\string_Mr£`\{\string_B1}{}
\markboth{\textcolor{darkblue}{\textbf{\ipa{gæ˧ɻæ˩}}}}{}
\textcolor{teal}{\mytextsc{nom}} \hspace{4pt} Ton~: L\#.

Village situé à environ 1,5 km à l'ouest de \textcolor{darkblue}{\textbf{\ipa{/ə˧lɑ˧-ʁwɤ\#˥/:}}} à main gauche en sortant de la vallée de Yongning, en direction de Eya. En chinois: Gaer.
\textcolor{brown}{\zh{嘎尔村。}}


\begin{exe}
\sn \textcolor{darkblue}{\textbf{\ipa{dʑɤ˩bv̩˧kɤ˧-sɑ˥ʁwɤ˩, | hi˩ʁwɤ˩-lo˥, | æ˩mi˧-ʁwɤ\#˥, | lɑ˧lo˧-ʁwɤ˥, | lɑ˧ŋwɤ˧, | bɤ˧tsʰo˧gv̩˥, | ə˧lɑ˧-ʁwɤ\#˥, | gæ˧ɻæ˩, | qʰæ˧tɕʰi˧, | tʰo˧ʈɯ\#˥}}}
\trans les dix villages comptant traditionnellement comme faisant partie de Yongning
\trans \textit{\textcolor{brown}{\zh{摩梭传统地理概念中,属于永宁的十个村落}}}
\end{exe}


\paragraph{\hspace{-0.5cm} {\Large \textcolor{darkblue}{\textbf{\ipa{gæ˩ɖæ˧}}}}} \hypertarget{g\{\string_Bd`\{\string_M1}{}
\markboth{\textcolor{darkblue}{\textbf{\ipa{gæ˩ɖæ˧}}}}{}
\textcolor{teal}{\mytextsc{nom}} \hspace{4pt} Ton~: LM.

Le haut du corps.
\textcolor{brown}{\zh{上半身。}}




\paragraph{\hspace{-0.5cm} {\Large \textcolor{darkblue}{\textbf{\ipa{gæ˩pʰæ˧}}}}} \hypertarget{g\{\string_Bp\string_h\{\string_M1}{}
\markboth{\textcolor{darkblue}{\textbf{\ipa{gæ˩pʰæ˧}}}}{}
\textcolor{teal}{\mytextsc{nom}} \hspace{4pt} Ton~: LM.

Resserre, pièce où on conserve certains produits: dans le même bâtiment que la cuisine-foyer-salle à manger, à sa gauche (vu depuis la cour).
\textcolor{brown}{\zh{储藏室、库房:存粮食、火腿的房间。}}


 \mytextsc{clf}~: \textcolor{darkblue}{\textbf{\ipa{tso˩}}} 

\paragraph{\hspace{-0.5cm} {\Large \textcolor{darkblue}{\textbf{\ipa{-gɤ˧}}}}} \hypertarget{-g7\string_M1}{}
\markboth{\textcolor{darkblue}{\textbf{\ipa{-gɤ˧}}}}{}
\textcolor{teal}{\mytextsc{nom}} \hspace{4pt} Ton~: M.
\ding{202} 
Lieu, endroit.
\textcolor{brown}{\zh{地方。}}


\begin{exe}
\sn \textcolor{darkblue}{\textbf{\ipa{njɤ˧ | ɖɯ˧-ʝi˧ (-gɤ˧) bi˧-zo˧-ho˩!}}}
\trans Je dois me rendre quelque part! / Je dois faire un voyage! / Je m'en vais! (Contexte: lorsqu'on se prépare réellement à un voyage; ou lors d'une dispute, lorsqu'on menace de quitter la maison.)
\trans \textit{\textcolor{brown}{\zh{我要去一个别的地方! / 我要换一个地方了! / 我要走了!}}}
\end{exe}
\begin{exe}
\sn \textcolor{darkblue}{\textbf{\ipa{ze˩ gɤ˧}}}
\trans quel endroit
\trans \textit{\textcolor{brown}{\zh{什么地方}}}
\end{exe}
\begin{exe}
\sn \textcolor{darkblue}{\textbf{\ipa{ʈʂʰɯ˧-gɤ˧}}}
\trans cet endroit-ci
\trans \textit{\textcolor{brown}{\zh{这个地方}}}
\end{exe}
\begin{exe}
\sn \textcolor{darkblue}{\textbf{\ipa{tʰv̩˧-gɤ˧}}}
\trans cet endroit-là
\trans \textit{\textcolor{brown}{\zh{那个地方}}}
\end{exe}
\ding{203} 
Moment.
\textcolor{brown}{\zh{时候。}}


\begin{exe}
\sn \textcolor{darkblue}{\textbf{\ipa{ʂɯ˧-ɬi˧mi˧-qo˧-gɤ˧ tʰv̩˧}}}
\trans quand est venu le septième mois, quand on en est au septième mois
\trans \textit{\textcolor{brown}{\zh{七月到了的时候}}}
\end{exe}
\begin{exe}
\sn \textcolor{darkblue}{\textbf{\ipa{ʂɯ˧-ɬi˧mi˧-qo˧-gɤ˧-dʑo˥}}}
\trans pendant le septième mois, au cours du septième mois
\trans \textit{\textcolor{brown}{\zh{七月的时候}}}
\end{exe}


\paragraph{\hspace{-0.5cm} {\Large \textcolor{darkblue}{\textbf{\ipa{gɤ˧\textsubscript{b}}}}}} \hypertarget{g7\string_Mb1}{}
\markboth{\textcolor{darkblue}{\textbf{\ipa{gɤ˧\textsubscript{b}}}}}{}
\textcolor{teal}{\mytextsc{verbe}} \hspace{4pt} Ton~: M\textsubscript{b}.

Manquer de.
\textcolor{brown}{\zh{缺乏。}}


\begin{exe}
\sn \textcolor{darkblue}{\textbf{\ipa{mɤ˧-gɤ˧}}}
\trans \mytextsc{neg}: ne pas manquer de
\trans \textit{\textcolor{brown}{\zh{不缺乏}}}
\end{exe}


\paragraph{\hspace{-0.5cm} {\Large \textcolor{darkblue}{\textbf{\ipa{gɤ˧bɤ˧}}}}} \hypertarget{g7\string_Mb7\string_M1}{}
\markboth{\textcolor{darkblue}{\textbf{\ipa{gɤ˧bɤ˧}}}}{}
\textcolor{teal}{\mytextsc{nom}} \hspace{4pt} Ton~: M.

Ombre.
\textcolor{brown}{\zh{影子。}}


\begin{exe}
\sn \textcolor{darkblue}{\textbf{\ipa{gɤ˧bɤ˧ li˧}}}
\trans regarder la télé (néologisme)
\trans \textit{\textcolor{brown}{\zh{看电视}}}
\end{exe}
 \mytextsc{clf}~: \textcolor{darkblue}{\textbf{\ipa{v̩˧}}} 

\paragraph{\hspace{-0.5cm} {\Large \textcolor{darkblue}{\textbf{\ipa{-gɤ˧bi\#˥}}}}} \hypertarget{-g7\string_Mbi\#\string_T1}{}
\markboth{\textcolor{darkblue}{\textbf{\ipa{-gɤ˧bi\#˥}}}}{}
\textcolor{teal}{\mytextsc{postposition}} \hspace{4pt} Ton~: \#H.

Sur, dessus.
\textcolor{brown}{\zh{上面。}}


\begin{exe}
\sn \textcolor{darkblue}{\textbf{\ipa{ʑi˧qʰwɤ˧-gɤ˧bi˧}}}
\trans sur la maison, sur le toit
\trans \textit{\textcolor{brown}{\zh{在房頂上}}}
\end{exe}


\paragraph{\hspace{-0.5cm} {\Large \textcolor{darkblue}{\textbf{\ipa{gɤ˧lɑ˧}}}}} \hypertarget{g7\string_MlA\string_M1}{}
\markboth{\textcolor{darkblue}{\textbf{\ipa{gɤ˧lɑ˧}}}}{}
\textcolor{teal}{\mytextsc{nom}} \hspace{4pt} Ton~: M.

Dieu, bouddha, bodhisattva.
\textcolor{brown}{\zh{神,菩萨,佛。}}

 Emprunt~: tibétain lha

 \mytextsc{clf}~: \textcolor{darkblue}{\textbf{\ipa{v̩˧}}} 

\paragraph{\hspace{-0.5cm} {\Large \textcolor{darkblue}{\textbf{\ipa{gɤ˧lɑ˧-pɤ\#˥}}}}} \hypertarget{g7\string_MlA\string_M-p7\#\string_T1}{}
\markboth{\textcolor{darkblue}{\textbf{\ipa{gɤ˧lɑ˧-pɤ\#˥}}}}{}
\textcolor{teal}{\mytextsc{nom}} \hspace{4pt} Ton~: \#H.

Image du bouddha.
\textcolor{brown}{\zh{佛像。}}


 \mytextsc{clf}~: \textcolor{darkblue}{\textbf{\ipa{pɤ˥}}} 

\paragraph{\hspace{-0.5cm} {\Large \textcolor{darkblue}{\textbf{\ipa{gɤ˧lɑ˧-ʑi˩}}}}} \hypertarget{g7\string_MlA\string_M-z£i\string_B1}{}
\markboth{\textcolor{darkblue}{\textbf{\ipa{gɤ˧lɑ˧-ʑi˩}}}}{}
\textcolor{teal}{\mytextsc{nom}} \hspace{4pt} Ton~: \mytextsc{L}.

Pièce du culte: pièce des esprits, pièce des ancêtres, où se trouve un autel. Un rituel y est effectué chaque matin. Le nom désigne par extension l'intégralité d'un des quatre bâtiments de la ferme traditionnelle na.
\textcolor{brown}{\zh{经堂(拜佛、拜祖先的房间)。}}


\begin{exe}
\sn \textcolor{darkblue}{\textbf{\ipa{gɤ˧lɑ˧-ʑi˩-di˩}}}
\trans même sens
\trans \textit{\textcolor{brown}{\zh{同上}}}
\end{exe}
 \mytextsc{clf}~: \textcolor{darkblue}{\textbf{\ipa{ɭɯ˧}}} 

\paragraph{\hspace{-0.5cm} {\Large \textcolor{darkblue}{\textbf{\ipa{gɤ˧qo˥}}}}} \hypertarget{g7\string_Mqo\string_T1}{}
\markboth{\textcolor{darkblue}{\textbf{\ipa{gɤ˧qo˥}}}}{}
\textcolor{teal}{\mytextsc{nom}} \hspace{4pt} Ton~: MH.

Haut du foyer: partie de la pièce où l’on prend les repas, autour du foyer; c'est une structure en bois, surélevée d’une vingtaine de centimètres par rapport au sol cimenté.
\textcolor{brown}{\zh{主屋的高处:人吃饭的地方。}}


 \mytextsc{clf}~: \textcolor{darkblue}{\textbf{\ipa{ɭɯ˧}}} 

\paragraph{\hspace{-0.5cm} {\Large \textcolor{darkblue}{\textbf{\ipa{}}}}} \hypertarget{g7\string_B‑1}{}
\markboth{\textcolor{darkblue}{\textbf{\ipa{}}}}{}
\textcolor{teal}{\mytextsc{adverbe}} \hspace{4pt} Ton~: L.

Préfixe directionnel: vers le haut.
\textcolor{brown}{\zh{向上、往上。}}


\textit{Voir~:} \textcolor{darkblue}{\textbf{\ipa{gɤ˩-qo˧, gɤ˩-tʰv̩˧qo˧, gɤ˩-ʈʂʰɯ˧qo˧}}} 

\paragraph{\hspace{-0.5cm} {\Large \textcolor{darkblue}{\textbf{\ipa{gɤ˩}}}}} \hypertarget{g7\string_B1}{}
\markboth{\textcolor{darkblue}{\textbf{\ipa{gɤ˩}}}}{}
\textcolor{teal}{\mytextsc{adjectif}} \hspace{4pt} Ton~: L.

Querelleur, belliqueux, batailleur. Ce terme s'emploie au sujet des signes astrologiques: certains sont considérés comme 'bagarreurs', comme le Tigre et le Singe, ce qui rend les personnes nées cette année-là peu appropriées pour certains rites (ex.: lors du rite de passage à l'âge adulte), et au contraire très prisés pour d'autres.
\textcolor{brown}{\zh{爱吵架。}}


\begin{exe}
\sn \textcolor{darkblue}{\textbf{\ipa{kʰv̩˧ gɤ˧˥}}}
\trans signe belliqueux (concept astrologique: certains signes confèrent aux gens nés l'année correspondante un caractère dur/belliqueux)
\trans \textit{\textcolor{brown}{\zh{爱打架的年份/生肖:十二个生肖中,虎、猴……被认为是爱打架的。}}}
\end{exe}
\begin{exe}
\sn \textcolor{darkblue}{\textbf{\ipa{kʰv̩˧ gɤ˧-hĩ˥}}}
\trans personne d'une année batailleuse
\trans \textit{\textcolor{brown}{\zh{属一个爱打架的年份/生肖的人。十二个生肖中,虎、猴……被认为是爱打架的。}}}
\end{exe}
\begin{exe}
\sn \textcolor{darkblue}{\textbf{\ipa{ʑi˩hṽ˥, | lɑ˧ : | kʰv̩˧ gɤ˧˥!}}}
\trans Les signes astrologiques du Singe et du Tigre sont des signes batailleurs!
\trans \textit{\textcolor{brown}{\zh{属猴和属虎的人很爱吵架!}}}
\end{exe}
\begin{exe}
\sn \textcolor{darkblue}{\textbf{\ipa{ʑi˩˥, | lɑ˧, | kʰv̩˧ gɤ˧˥!}}}
\trans Même sens que ci-dessus; formulation modernisée par l'enquêteur, utilisant le terme usuel pour 'singe'.
\trans \textit{\textcolor{brown}{\zh{同上}}}
\end{exe}


\paragraph{\hspace{-0.5cm} {\Large \textcolor{darkblue}{\textbf{\ipa{gɤ˩\textsubscript{a}}}} \textsubscript{1}}} \hypertarget{g7\string_Ba1}{}
\markboth{\textcolor{darkblue}{\textbf{\ipa{gɤ˩\textsubscript{a}}}} \textsubscript{1}}{}
\textcolor{teal}{\mytextsc{verbe}} \hspace{4pt} Ton~: L\textsubscript{a}.

S’éteindre.
\textcolor{brown}{\zh{灭,熄。}}


\begin{exe}
\sn \textcolor{darkblue}{\textbf{\ipa{mv̩˧ | le˧-gɤ˩(-ze˩)}}}
\trans Le feu s'est éteint.
\trans \textit{\textcolor{brown}{\zh{火灭了。}}}
\end{exe}


\paragraph{\hspace{-0.5cm} {\Large \textcolor{darkblue}{\textbf{\ipa{gɤ˩\textsubscript{a}}}} \textsubscript{2}}} \hypertarget{g7\string_Ba2}{}
\markboth{\textcolor{darkblue}{\textbf{\ipa{gɤ˩\textsubscript{a}}}} \textsubscript{2}}{}
\textcolor{teal}{\mytextsc{verbe}} \hspace{4pt} Ton~: L\textsubscript{a}.

Être satisfait, content (de son sort), heureux.
\textcolor{brown}{\zh{满意,幸福,甘心,服气。}}


\begin{exe}
\sn \textcolor{darkblue}{\textbf{\ipa{hɤ˩-zo˥, | le˧-gɤ˩-ze˩!}}}
\trans (il a réussi) habilement (ce qu'il voulait faire), il est content/satisfait!
\trans \textit{\textcolor{brown}{\zh{很成功,真高兴! / 他成功了,很满意!}}}
\end{exe}
\begin{exe}
\sn \textcolor{darkblue}{\textbf{\ipa{ʈʂʰɯ˧ | ɖwæ˧˥ | le˧-gɤ˩-ze˩!}}}
\trans il est très content!
\trans \textit{\textcolor{brown}{\zh{他很满意!}}}
\end{exe}
\begin{exe}
\sn \textcolor{darkblue}{\textbf{\ipa{no˩-se˥, | ɖwæ˧˥ | le˧-gɤ˩-ze˩: | zo˧mv̩˥ hɤ˩-zo˩!}}}
\trans Vous, vous avez bien de la chance/vous avez toutes raisons d'être satisfait(e)/vous avez des sujets de satisfaction: vos enfants sont brillants/habiles!
\trans \textit{\textcolor{brown}{\zh{你呢,(应该)很满意:(你的)孩子很成功!}}}
\end{exe}
\begin{exe}
\sn \textcolor{darkblue}{\textbf{\ipa{mɤ˧-gɤ˩}}}
\trans être mécontent, ne pas se résigner, être récalcitrant
\trans \textit{\textcolor{brown}{\zh{不满意、不甘心、不服气}}}
\end{exe}


\paragraph{\hspace{-0.5cm} {\Large \textcolor{darkblue}{\textbf{\ipa{gɤ˩\textsubscript{a}}}} \textsubscript{3}}} \hypertarget{g7\string_Ba3}{}
\markboth{\textcolor{darkblue}{\textbf{\ipa{gɤ˩\textsubscript{a}}}} \textsubscript{3}}{}
\textcolor{teal}{\mytextsc{adjectif}} \hspace{4pt} Ton~: L\textsubscript{a}.

Surpris, étonné, abasourdi; terrifié.
\textcolor{brown}{\zh{震惊。}}


\begin{exe}
\sn \textcolor{darkblue}{\textbf{\ipa{le˧-gɤ˩-ze˩}}}
\trans \mytextsc{accomp} \string_ \mytextsc{pfv}
\trans \textit{\textcolor{brown}{\zh{震惊了}}}
\end{exe}
\begin{exe}
\sn \textcolor{darkblue}{\textbf{\ipa{no˧ | hĩ˧ gɤ˧-kʰɯ˥!}}}
\trans Tu fais peur aux gens!
\trans \textit{\textcolor{brown}{\zh{你让人害怕!}}}
\end{exe}


\paragraph{\hspace{-0.5cm} {\Large \textcolor{darkblue}{\textbf{\ipa{gɤ˩bv̩˧}}}}} \hypertarget{g7\string_Bbv\string_=\string_M1}{}
\markboth{\textcolor{darkblue}{\textbf{\ipa{gɤ˩bv̩˧}}}}{}
\textcolor{teal}{\mytextsc{verbe}} \hspace{4pt} Ton~: LM.

Déborder.
\textcolor{brown}{\zh{溢出来。}}


\begin{exe}
\sn \textcolor{darkblue}{\textbf{\ipa{gɤ˩bv̩˧-ze˩}}}
\trans \mytextsc{pfv}
\trans \textit{\textcolor{brown}{\zh{溢出来了}}}
\end{exe}


\paragraph{\hspace{-0.5cm} {\Large \textcolor{darkblue}{\textbf{\ipa{gɤ˩dzɤ˧}}}}} \hypertarget{g7\string_Bdz7\string_M1}{}
\markboth{\textcolor{darkblue}{\textbf{\ipa{gɤ˩dzɤ˧}}}}{}
\textcolor{teal}{\mytextsc{adverbe}} \hspace{4pt} Ton~: LM.

Haut, partie supérieure, partie noble (d'une salle, d'une tablée…) (symboliquement: ``la tête").
\textcolor{brown}{\zh{在上部分,上座。}}


\begin{exe}
\sn \textcolor{darkblue}{\textbf{\ipa{gɤ˩dzɤ˧ dzi˧˥}}}
\trans être assis à une place d'honneur
\trans \textit{\textcolor{brown}{\zh{坐上座}}}
\end{exe}
\begin{exe}
\sn \textcolor{darkblue}{\textbf{\ipa{no˧ | gɤ˩dzɤ˧ dzi˧˥!}}}
\trans Veuillez vous installer à l'une des premières places! / Veuillez prendre l'une des places d'honneur!
\trans \textit{\textcolor{brown}{\zh{请您坐在上座!}}}
\end{exe}
\textit{Voir~:} \textcolor{darkblue}{\textbf{\ipa{gɤ˩-}}} 

\paragraph{\hspace{-0.5cm} {\Large \textcolor{darkblue}{\textbf{\ipa{gɤ˩-qo˧}}}}} \hypertarget{g7\string_B-qo\string_M1}{}
\markboth{\textcolor{darkblue}{\textbf{\ipa{gɤ˩-qo˧}}}}{}
\textcolor{teal}{\mytextsc{adverbe}} \hspace{4pt} Ton~: M.

Par là-bas tout en haut.
\textcolor{brown}{\zh{那上面(指高处)。}}


\textit{Voir~:} \textcolor{darkblue}{\textbf{\ipa{gɤ˩-, gɤ˩-ʈʂʰɯ˧qo˧, gɤ˩-tʰv̩˧qo˧,}}} 

\paragraph{\hspace{-0.5cm} {\Large \textcolor{darkblue}{\textbf{\ipa{gɤ˩qwɤ˧}}}}} \hypertarget{g7\string_Bqw7\string_M1}{}
\markboth{\textcolor{darkblue}{\textbf{\ipa{gɤ˩qwɤ˧}}}}{}
\textcolor{teal}{\mytextsc{nom}} \hspace{4pt} Ton~: LM.

Autel en contrehaut du foyer, où on dépose les présents qu'apportent les invités/les membres de la famille, les offrant ainsi aux ancêtres.
\textcolor{brown}{\zh{火炉旁边的祭坛(上面摆礼物等)。}}


 \mytextsc{clf}~: \textcolor{darkblue}{\textbf{\ipa{ɭɯ˧}}} 

\paragraph{\hspace{-0.5cm} {\Large \textcolor{darkblue}{\textbf{\ipa{gɤ˩ʁwɤ\#˥}}}}} \hypertarget{g7\string_BRw7\#\string_T1}{}
\markboth{\textcolor{darkblue}{\textbf{\ipa{gɤ˩ʁwɤ\#˥}}}}{}
\textcolor{teal}{\mytextsc{nom}} \hspace{4pt} Ton~: LM+\#H.

Nom de village; en chinois: Gewa.
\textcolor{brown}{\zh{格瓦村:永宁的一个村落。直译:上村。音译:格瓦。}}




\paragraph{\hspace{-0.5cm} {\Large \textcolor{darkblue}{\textbf{\ipa{gɤ˩ʁwɤ˧}}}}} \hypertarget{g7\string_BRw7\string_M1}{}
\markboth{\textcolor{darkblue}{\textbf{\ipa{gɤ˩ʁwɤ˧}}}}{}
\textcolor{teal}{\mytextsc{nom}} \hspace{4pt} Ton~: LM.

Cours supérieur (d'une rivière), amont.
\textcolor{brown}{\zh{上游。}}




\paragraph{\hspace{-0.5cm} {\Large \textcolor{darkblue}{\textbf{\ipa{gɤ˩-tʰv̩˧-gi\#˥}}}}} \hypertarget{g7\string_B-t\string_hv\string_=\string_M-gi\#\string_T1}{}
\markboth{\textcolor{darkblue}{\textbf{\ipa{gɤ˩-tʰv̩˧-gi\#˥}}}}{}
\textcolor{teal}{\mytextsc{adverbe}} \hspace{4pt} Ton~: L-\#H.

Au loin par là-haut, de ce côté tout là-haut.
\textcolor{brown}{\zh{那里(指高处)。}}


\textit{Voir~:} \textcolor{darkblue}{\textbf{\ipa{gɤ˩‑, gɤ˩-qo˧, gɤ˩-tʰv̩˧qo˧, gɤ˩-ʈʂʰɯ˧qo˧}}} 

\paragraph{\hspace{-0.5cm} {\Large \textcolor{darkblue}{\textbf{\ipa{gɤ˩-tʰv̩˧qo˧}}}}} \hypertarget{g7\string_B-t\string_hv\string_=\string_Mqo\string_M1}{}
\markboth{\textcolor{darkblue}{\textbf{\ipa{gɤ˩-tʰv̩˧qo˧}}}}{}
\textcolor{teal}{\mytextsc{adverbe}} \hspace{4pt} Ton~: L-\#H.

Au loin par là-haut, de ce côté tout là-haut.
\textcolor{brown}{\zh{那里(指高处)。}}


\textit{Voir~:} \textcolor{darkblue}{\textbf{\ipa{gɤ˩‑, gɤ˩-qo˧, gɤ˩-ʈʂʰɯ˧qo˧}}} 

\paragraph{\hspace{-0.5cm} {\Large \textcolor{darkblue}{\textbf{\ipa{gɤ˩ʈʂæ˧˥}}}}} \hypertarget{g7\string_Bt`s`\{\string_M\string_T1}{}
\markboth{\textcolor{darkblue}{\textbf{\ipa{gɤ˩ʈʂæ˧˥}}}}{}
\textcolor{teal}{\mytextsc{nom}} \hspace{4pt} Ton~: LM+MH\#.

Haut du corps, partie supérieure du corps.
\textcolor{brown}{\zh{上半(身)。}}




\paragraph{\hspace{-0.5cm} {\Large \textcolor{darkblue}{\textbf{\ipa{gɤ˩ʈʂʰæ˧-hĩ˧˥}}}}} \hypertarget{g7\string_Bt`s`\string_h\{\string_M-hi\string_~\string_M\string_T1}{}
\markboth{\textcolor{darkblue}{\textbf{\ipa{gɤ˩ʈʂʰæ˧-hĩ˧˥}}}}{}
\textcolor{teal}{\mytextsc{nom}} \hspace{4pt} Ton~: LM+MH\#.

Les générations passées, les ancêtres.
\textcolor{brown}{\zh{祖先。}}




\paragraph{\hspace{-0.5cm} {\Large \textcolor{darkblue}{\textbf{\ipa{gɤ˩-ʈʂʰɯ˧-gi\#˥}}}}} \hypertarget{g7\string_B-t`s`\string_hM\string_M-gi\#\string_T1}{}
\markboth{\textcolor{darkblue}{\textbf{\ipa{gɤ˩-ʈʂʰɯ˧-gi\#˥}}}}{}
\textcolor{teal}{\mytextsc{adverbe}} \hspace{4pt} Ton~: L-\#H.

Au loin par là-haut, de ce côté tout là-haut.
\textcolor{brown}{\zh{那里(指高处)。}}


\textit{Voir~:} \textcolor{darkblue}{\textbf{\ipa{gɤ˩-, gɤ˩-qo˧, gɤ˩-tʰv̩˧qo˧, gɤ˩-ʈʂʰɯ˧qo˧}}} 

\paragraph{\hspace{-0.5cm} {\Large \textcolor{darkblue}{\textbf{\ipa{gɤ˩-ʈʂʰɯ˧qo˧}}}}} \hypertarget{g7\string_B-t`s`\string_hM\string_Mqo\string_M1}{}
\markboth{\textcolor{darkblue}{\textbf{\ipa{gɤ˩-ʈʂʰɯ˧qo˧}}}}{}
\textcolor{teal}{\mytextsc{adverbe}} \hspace{4pt} Ton~: L-\#H.

Au loin par là-haut, de ce côté tout là-haut.
\textcolor{brown}{\zh{那里(指高处)。}}


\textit{Voir~:} \textcolor{darkblue}{\textbf{\ipa{gɤ˩-, gɤ˩-qo˧, gɤ˩-tʰv̩˧qo˧}}} 

\paragraph{\hspace{-0.5cm} {\Large \textcolor{darkblue}{\textbf{\ipa{gɤ˧˥}}}}} \hypertarget{g7\string_M\string_T1}{}
\markboth{\textcolor{darkblue}{\textbf{\ipa{gɤ˧˥}}}}{}
\textcolor{teal}{\mytextsc{verbe}} \hspace{4pt} Ton~: MH.

Porter à l’épaule; porter sur une palanche.
\textcolor{brown}{\zh{扛,担。}}


\begin{exe}
\sn \textcolor{darkblue}{\textbf{\ipa{tʰi˧-gɤ˧˥}}}
\trans \mytextsc{dur}
\trans \textit{\textcolor{brown}{\zh{\mytextsc{dur}}}}
\end{exe}
\begin{exe}
\sn \textcolor{darkblue}{\textbf{\ipa{tʰi˧-gɤ˧-ze˥}}}
\trans \mytextsc{dur} \string_ \mytextsc{pfv}
\trans \textit{\textcolor{brown}{\zh{\mytextsc{dur} \string_ \mytextsc{pfv}}}}
\end{exe}
\begin{exe}
\sn \textcolor{darkblue}{\textbf{\ipa{le˧-gɤ˧-ze˥}}}
\trans \mytextsc{accomp} \string_ \mytextsc{pfv}
\trans \textit{\textcolor{brown}{\zh{扛了}}}
\end{exe}
\begin{exe}
\sn \textcolor{darkblue}{\textbf{\ipa{tso˧\textasciitilde{}tso˧ gɤ˩}}}
\trans porter quelque chose à l'épaule
\trans \textit{\textcolor{brown}{\zh{扛东西}}}
\end{exe}
\begin{exe}
\sn \textcolor{darkblue}{\textbf{\ipa{njɤ˧(-ɳɯ˧) | gɤ˧-bi˥!}}}
\trans C'est moi qui porte!
\trans \textit{\textcolor{brown}{\zh{我来扛吧!}}}
\end{exe}


\paragraph{\hspace{-0.5cm} {\Large \textcolor{darkblue}{\textbf{\ipa{gi˥}}} \textsubscript{1}}} \hypertarget{gi\string_T1}{}
\markboth{\textcolor{darkblue}{\textbf{\ipa{gi˥}}} \textsubscript{1}}{}
\textcolor{teal}{\mytextsc{verbe}} \hspace{4pt} Ton~: H.

Tomber (neige, pluie), neiger, pleuvoir.
\textcolor{brown}{\zh{下(雨,雪)。}}


\begin{exe}
\sn \textcolor{darkblue}{\textbf{\ipa{bi˧ gi˧. / bi˧ gi˧-ze˩.}}}
\trans Il neige. / Il a neigé.
\trans \textit{\textcolor{brown}{\zh{下雪。 / 下雪了。}}}
\end{exe}
\begin{exe}
\sn \textcolor{darkblue}{\textbf{\ipa{hi˩ gi˩˥. / hi˩ gi˩-ze˥.}}}
\trans Il pleut. / Il a plu.
\trans \textit{\textcolor{brown}{\zh{下雨。 / 下雨了。}}}
\end{exe}
\begin{exe}
\sn \textcolor{darkblue}{\textbf{\ipa{tsʰi˧-ɲi˧-dʑo˩, | hi˩ gi˩-ze˥, | le˧-gɤ˩-ze˩!}}}
\trans Aujourd'hui, il s'est mis à pleuvoir / il a plu; c'est bien! (Commentaire au sujet de la pluie qui est venue, après une longue période de sécheresse.)
\trans \textit{\textcolor{brown}{\zh{今天,下雨了,真好!(情景:大旱灾过后,雨季终于来了,这对庄稼很好。)}}}
\end{exe}


\paragraph{\hspace{-0.5cm} {\Large \textcolor{darkblue}{\textbf{\ipa{gi˥}}} \textsubscript{2}}} \hypertarget{gi\string_T2}{}
\markboth{\textcolor{darkblue}{\textbf{\ipa{gi˥}}} \textsubscript{2}}{}
\textcolor{teal}{\mytextsc{verbe}} \hspace{4pt} Ton~: H.

Devoir de l'argent, avoir des dettes.
\textcolor{brown}{\zh{欠(钱)。}}


\begin{exe}
\sn \textcolor{darkblue}{\textbf{\ipa{ɖʐe˧ | tʰi˧-gi˥}}}
\trans devoir de l'argent
\trans \textit{\textcolor{brown}{\zh{欠钱}}}
\end{exe}


\paragraph{\hspace{-0.5cm} {\Large \textcolor{darkblue}{\textbf{\ipa{gi˥\textsubscript{a}}}}}} \hypertarget{gi\string_Ta1}{}
\markboth{\textcolor{darkblue}{\textbf{\ipa{gi˥\textsubscript{a}}}}}{}
\textcolor{teal}{\mytextsc{classificateur}} \hspace{4pt} Ton~: H\textsubscript{a}.
\ding{202} 
Une moitié, un demi.
\textcolor{brown}{\zh{量词:一半。}}


\begin{exe}
\sn \textcolor{darkblue}{\textbf{\ipa{ɖɯ˧-gi˥}}}
\trans une moitié
\trans \textit{\textcolor{brown}{\zh{一半}}}
\end{exe}
\begin{exe}
\sn \textcolor{darkblue}{\textbf{\ipa{tsʰe˩ʐv̩˩-gi˥}}}
\trans quatorze moitiés (combinaison permettant de déterminer la catégorie tonale de ce classificateur: elle établit que le ton est H1 et non H2)
\trans \textit{\textcolor{brown}{\zh{十四个半(注:这是为了确定调类而问的短语)}}}
\end{exe}
\begin{exe}
\sn \textcolor{darkblue}{\textbf{\ipa{tv̩˧tsʰɯ˧ | ɖɯ˧-gi˥}}}
\trans la moitié du temps, la moitié de la durée
\trans \textit{\textcolor{brown}{\zh{一半的时间}}}
\end{exe}
\ding{203} 
Un côté (d'une pièce, d'une maison…); une direction.
\textcolor{brown}{\zh{量词:一面(房屋的一面)。}}


\begin{exe}
\sn \textcolor{darkblue}{\textbf{\ipa{ɖɯ˧-gi˧ hõ˧}}}
\trans aller d'un certain côté, aller de son côté
\trans \textit{\textcolor{brown}{\zh{往一个方向走、走自己的方向}}}
\end{exe}
\begin{exe}
\sn \textcolor{darkblue}{\textbf{\ipa{ɖɯ˧-v̩˧ | ɖɯ˧-gi˧ hɯ˧}}}
\trans aller chacun de son côté; se séparer
\trans \textit{\textcolor{brown}{\zh{分开:每个人去自己的方向}}}
\end{exe}


\paragraph{\hspace{-0.5cm} {\Large \textcolor{darkblue}{\textbf{\ipa{gi˧dʑɯ˧}}}}} \hypertarget{gi\string_Mdz£M\string_M1}{}
\markboth{\textcolor{darkblue}{\textbf{\ipa{gi˧dʑɯ˧}}}}{}
\textcolor{teal}{\mytextsc{nom}} \hspace{4pt} Ton~: M.

Le fleuve Yangtze.
\textcolor{brown}{\zh{金沙江。}}


\begin{exe}
\sn \textcolor{darkblue}{\textbf{\ipa{gi˧dʑɯ˧-kʰi\#˥}}}
\trans le bord du fleuve Yangtze: Fengke, Labai...
\trans \textit{\textcolor{brown}{\zh{金沙江边:奉科,拉伯……}}}
\end{exe}
\begin{exe}
\sn \textcolor{darkblue}{\textbf{\ipa{gi˧dʑɯ˧-kʰi˧-hĩ\#˥}}}
\trans les riverains du Yangtze: gens de Labai, Fengke...
\trans \textit{\textcolor{brown}{\zh{金沙江边的人:奉科人,拉伯人……}}}
\end{exe}


\paragraph{\hspace{-0.5cm} {\Large \textcolor{darkblue}{\textbf{\ipa{gi˧-nɑ˧mi\#˥}}}}} \hypertarget{gi\string_M-nA\string_Mmi\#\string_T1}{}
\markboth{\textcolor{darkblue}{\textbf{\ipa{gi˧-nɑ˧mi\#˥}}}}{}
\textcolor{teal}{\mytextsc{nom}} \hspace{4pt} Ton~: \#H.

Ours (mâle ou femelle). Il n'existe pas de terme désignant de façon non ambiguë une ourse.
\textcolor{brown}{\zh{熊,母熊。}}


\begin{exe}
\sn \textcolor{darkblue}{\textbf{\ipa{gi˧-nɑ˧mi˧ tʰv̩˧-pʰo˩}}}
\trans \mytextsc{n}+\mytextsc{dem}+\mytextsc{clf}
\trans \textit{\textcolor{brown}{\zh{这只熊}}}
\end{exe}
 \mytextsc{clf}~: \textcolor{darkblue}{\textbf{\ipa{pʰo˧˥}}} 

\paragraph{\hspace{-0.5cm} {\Large \textcolor{darkblue}{\textbf{\ipa{gi˧-nɑ˧mi˧-pʰv̩\#˥}}}}} \hypertarget{gi\string_M-nA\string_Mmi\string_M-p\string_hv\string_=\#\string_T1}{}
\markboth{\textcolor{darkblue}{\textbf{\ipa{gi˧-nɑ˧mi˧-pʰv̩\#˥}}}}{}
\textcolor{teal}{\mytextsc{nom}} \hspace{4pt} Ton~: \#H.

Ours (mâle).
\textcolor{brown}{\zh{公熊。}}


\begin{exe}
\sn \textcolor{darkblue}{\textbf{\ipa{gi˧-nɑ˧mi˧-pʰv̩˧ tʰv̩˧-pʰo˩}}}
\trans \mytextsc{n}+\mytextsc{dem}+\mytextsc{clf}
\trans \textit{\textcolor{brown}{\zh{这只公熊}}}
\end{exe}
 \mytextsc{clf}~: \textcolor{darkblue}{\textbf{\ipa{pʰo˧˥}}} 

\paragraph{\hspace{-0.5cm} {\Large \textcolor{darkblue}{\textbf{\ipa{gi˧-nɑ˧mi˧-zo\#˥}}}}} \hypertarget{gi\string_M-nA\string_Mmi\string_M-zo\#\string_T1}{}
\markboth{\textcolor{darkblue}{\textbf{\ipa{gi˧-nɑ˧mi˧-zo\#˥}}}}{}
\textcolor{teal}{\mytextsc{nom}} \hspace{4pt} Ton~: \#H.

Ourson (de sexe masculin).
\textcolor{brown}{\zh{小熊。}}


\begin{exe}
\sn \textcolor{darkblue}{\textbf{\ipa{gi˧-nɑ˧mi˧-zo˧ tʰv̩˧-ɭɯ\#˥}}}
\trans \mytextsc{n}+\mytextsc{dem}+\mytextsc{clf}
\trans \textit{\textcolor{brown}{\zh{这只小熊}}}
\end{exe}
 \mytextsc{clf}~: \textcolor{darkblue}{\textbf{\ipa{ɭɯ˧}}} 

\paragraph{\hspace{-0.5cm} {\Large \textcolor{darkblue}{\textbf{\ipa{gi˧zɯ\#˥}}}}} \hypertarget{gi\string_MzM\#\string_T1}{}
\markboth{\textcolor{darkblue}{\textbf{\ipa{gi˧zɯ\#˥}}}}{}
\textcolor{teal}{\mytextsc{nom}} \hspace{4pt} Ton~: \#H.

Petit frère (employé aussi entre cousins).
\textcolor{brown}{\zh{弟弟(也可指更年轻的表弟)。}}


\begin{exe}
\sn \textcolor{darkblue}{\textbf{\ipa{gi˧zɯ˧=ɻæ˥}}}
\trans \mytextsc{associatif}: les petits frères
\trans \textit{\textcolor{brown}{\zh{联想复数:弟弟们,表弟们}}}
\end{exe}
 \mytextsc{clf}~: \textcolor{darkblue}{\textbf{\ipa{v̩˧}}} 

\paragraph{\hspace{-0.5cm} {\Large \textcolor{darkblue}{\textbf{\ipa{gi˧zɯ˧-go˧mi\#˥}}}}} \hypertarget{gi\string_MzM\string_M-go\string_Mmi\#\string_T1}{}
\markboth{\textcolor{darkblue}{\textbf{\ipa{gi˧zɯ˧-go˧mi\#˥}}}}{}
\textcolor{teal}{\mytextsc{nom}} \hspace{4pt} Ton~: \#H.

Cadets: petits frères+ petites sœurs.
\textcolor{brown}{\zh{弟弟妹妹。}}




\paragraph{\hspace{-0.5cm} {\Large \textcolor{darkblue}{\textbf{\ipa{gi˩}}}}} \hypertarget{gi\string_B1}{}
\markboth{\textcolor{darkblue}{\textbf{\ipa{gi˩}}}}{}
\textcolor{teal}{\mytextsc{nom}} \hspace{4pt} Ton~: L.

Ours.
\textcolor{brown}{\zh{大熊。}}




\paragraph{\hspace{-0.5cm} {\Large \textcolor{darkblue}{\textbf{\ipa{gi˩}}}}} \hypertarget{gi\string_B1}{}
\markboth{\textcolor{darkblue}{\textbf{\ipa{gi˩}}}}{}
\textcolor{teal}{\mytextsc{nom}} \hspace{4pt} Ton~: L.

Grenier à céréales; selon M23, est le lieu dans la maison où on stocke les céréales.
\textcolor{brown}{\zh{粮仓。}}


\begin{exe}
\sn \textcolor{darkblue}{\textbf{\ipa{gi˧mi˧}}}
\trans grand grenier
\trans \textit{\textcolor{brown}{\zh{大粮仓}}}
\end{exe}
\begin{exe}
\sn \textcolor{darkblue}{\textbf{\ipa{gi˩zo˩˥}}}
\trans petit grenier
\trans \textit{\textcolor{brown}{\zh{小粮仓}}}
\end{exe}
\begin{exe}
\sn \textcolor{darkblue}{\textbf{\ipa{njɤ˧ | gi˩ gv̩˩-zo˩-ho˥}}}
\trans il va falloir que je répare le grenier à céréales!
\trans \textit{\textcolor{brown}{\zh{我应该修粮仓!}}}
\end{exe}
 \mytextsc{clf}~: \textcolor{darkblue}{\textbf{\ipa{ɭɯ˧}}} 

\paragraph{\hspace{-0.5cm} {\Large \textcolor{darkblue}{\textbf{\ipa{gi˩\textsubscript{a}}}}}} \hypertarget{gi\string_Ba1}{}
\markboth{\textcolor{darkblue}{\textbf{\ipa{gi˩\textsubscript{a}}}}}{}
\textcolor{teal}{\mytextsc{adjectif}} \hspace{4pt} Ton~: L\textsubscript{a}.
\textit{De:} \textbf{/gɯ˩a 2/ et /ʝi˥/} 
Vrai, vraiment.
\textcolor{brown}{\zh{真,真的。}}


\begin{exe}
\sn \textcolor{darkblue}{\textbf{\ipa{mɤ˧-gi˩!}}}
\trans c'est pas vrai!
\trans \textit{\textcolor{brown}{\zh{不是的! / 不是真的!}}}
\end{exe}
\begin{exe}
\sn \textcolor{darkblue}{\textbf{\ipa{ə˩-gi˩˥?}}}
\trans c'est vrai? / n'est-ce pas?
\trans \textit{\textcolor{brown}{\zh{对吧? / 对吗?}}}
\end{exe}
\begin{exe}
\sn \textcolor{darkblue}{\textbf{\ipa{ə˩-gi˩˥ ? – gi˩˥!}}}
\trans N'est-ce pas? - Oui-da! (Le locuteur demande à son interlocuteur de confirmer qu’il adhère à son propos; l'autre donne son assentiment.)
\trans \textit{\textcolor{brown}{\zh{对吧? -对的!}}}
\end{exe}
\begin{exe}
\sn \textcolor{darkblue}{\textbf{\ipa{gi˩-hĩ˩ ʐwɤ˥}}}
\trans dire vrai
\trans \textit{\textcolor{brown}{\zh{说实话,老实说}}}
\end{exe}
\begin{exe}
\sn \textcolor{darkblue}{\textbf{\ipa{gi˩˥ | -gɯ˩˥}}}
\trans vraiment, véritablement
\trans \textit{\textcolor{brown}{\zh{真的,真正的}}}
\end{exe}


\paragraph{\hspace{-0.5cm} {\Large \textcolor{darkblue}{\textbf{\ipa{gi˩kɯ˩}}}}} \hypertarget{gi\string_BkM\string_B1}{}
\markboth{\textcolor{darkblue}{\textbf{\ipa{gi˩kɯ˩}}}}{}
\textcolor{teal}{\mytextsc{nom}} \hspace{4pt} Ton~: L.

Musc (littéralement: 'bile d'ours').
\textcolor{brown}{\zh{麝香(直译:大熊胆)。}}




\paragraph{\hspace{-0.5cm} {\Large \textcolor{darkblue}{\textbf{\ipa{‑gi˧˥}}}}} \hypertarget{‑gi\string_M\string_T1}{}
\markboth{\textcolor{darkblue}{\textbf{\ipa{‑gi˧˥}}}}{}
\textcolor{teal}{\mytextsc{postposition}} \hspace{4pt} Ton~: MH.

Derrière.
\textcolor{brown}{\zh{后面,(最)后。}}


\begin{exe}
\sn \textcolor{darkblue}{\textbf{\ipa{ə˧mɑ˧-gi˧˥}}}
\trans derrière maman
\trans \textit{\textcolor{brown}{\zh{妈妈后面}}}
\end{exe}
\begin{exe}
\sn \textcolor{darkblue}{\textbf{\ipa{lɑ˧-gi˧˥}}}
\trans derrière le tigre
\trans \textit{\textcolor{brown}{\zh{老虎后面}}}
\end{exe}
\begin{exe}
\sn \textcolor{darkblue}{\textbf{\ipa{bo˩-gi˥}}}
\trans derrière le cochon
\trans \textit{\textcolor{brown}{\zh{猪后面}}}
\end{exe}
\begin{exe}
\sn \textcolor{darkblue}{\textbf{\ipa{mv̩˩-gi˥}}}
\trans derrière la fille
\trans \textit{\textcolor{brown}{\zh{女儿后面}}}
\end{exe}
\begin{exe}
\sn \textcolor{darkblue}{\textbf{\ipa{ʐwæ˧-gi˥}}}
\trans derrière le cheval
\trans \textit{\textcolor{brown}{\zh{马后面}}}
\end{exe}
\begin{exe}
\sn \textcolor{darkblue}{\textbf{\ipa{ʈʂʰɯ˧-gi˥ | tʰi˧-tɕʰo˩}}}
\trans se cacher là-derrière
\trans \textit{\textcolor{brown}{\zh{藏那后面}}}
\end{exe}
\begin{exe}
\sn \textcolor{darkblue}{\textbf{\ipa{no˧-gi˧ njɤ˥ ʈʂwæ˩!}}}
\trans je te suis, je marche dans tes pas; je t'imite
\trans \textit{\textcolor{brown}{\zh{我跟你走! / 我都按你说的来做吧!}}}
\end{exe}
\begin{exe}
\sn \textcolor{darkblue}{\textbf{\ipa{ɖɯ˧-v̩˧-gi˧˥, | ɖɯ˧-v̩˧ hwæ˧!}}}
\trans en acheter un après l'autre (contexte: un caravanier achète des chevaux l'un après l'autre, afin de se constituer sa propre caravane)
\trans \textit{\textcolor{brown}{\zh{一个接着一个地买(情景:一个人接二连三地买马,最后组成自己的马帮队)}}}
\end{exe}
\begin{exe}
\sn \textcolor{darkblue}{\textbf{\ipa{[F5] gi˧˥ | ɖɯ˧-qɑ˩ gv̩˩-bi˩!}}}
\trans On va en faire une dernière botte! (contexte: on travaille le lin, botte après botte; vers la fin d'une longue séance de travail, quelqu'un annonce: ``On va en faire une dernière botte! / Une dernière botte, et on s'arrête!")
\trans \textit{\textcolor{brown}{\zh{再做一捆吧!(情景:女人们在纺麻线,工作了很久,一个人就说:“再做最后一捆(就收工吧)!”)}}}
\end{exe}
\begin{exe}
\sn \textcolor{darkblue}{\textbf{\ipa{gi˧-se˧}}}
\trans marcher derrière, suivre derrière
\trans \textit{\textcolor{brown}{\zh{在后面走,在后面跟着}}}
\end{exe}


\paragraph{\hspace{-0.5cm} {\Large \textcolor{darkblue}{\textbf{\ipa{go˧bɤ˩}}}}} \hypertarget{go\string_Mb7\string_B1}{}
\markboth{\textcolor{darkblue}{\textbf{\ipa{go˧bɤ˩}}}}{}
\textcolor{teal}{\mytextsc{nom}} \hspace{4pt} Ton~: L\#.

Temple, monastère.
\textcolor{brown}{\zh{庙,寺。}}

 Emprunt~: tibétain dgonpa

 \mytextsc{clf}~: \textcolor{darkblue}{\textbf{\ipa{ɭɯ˧}}} 

\paragraph{\hspace{-0.5cm} {\Large \textcolor{darkblue}{\textbf{\ipa{go˧mi˧}}}}} \hypertarget{go\string_Mmi\string_M1}{}
\markboth{\textcolor{darkblue}{\textbf{\ipa{go˧mi˧}}}}{}
\textcolor{teal}{\mytextsc{nom}} \hspace{4pt} Ton~: M.

Petite soeur (employé aussi pour les cousines plus jeunes).
\textcolor{brown}{\zh{妹妹。}}


\begin{exe}
\sn \textcolor{darkblue}{\textbf{\ipa{go˧mi˧=ɻæ˩}}}
\trans \mytextsc{associatif}: les petites sœurs, les jeunes cousines
\trans \textit{\textcolor{brown}{\zh{联想复数:妹妹们,表妹们}}}
\end{exe}
 \mytextsc{clf}~: \textcolor{darkblue}{\textbf{\ipa{v̩˧}}} 

\paragraph{\hspace{-0.5cm} {\Large \textcolor{darkblue}{\textbf{\ipa{go˩\textsubscript{a}}}}}} \hypertarget{go\string_Ba1}{}
\markboth{\textcolor{darkblue}{\textbf{\ipa{go˩\textsubscript{a}}}}}{}
\textcolor{teal}{\mytextsc{verbe}} \hspace{4pt} Ton~: L\textsubscript{a}.

Souffrir, avoir mal; être malade.
\textcolor{brown}{\zh{痛,病 (生病)。}}


\begin{exe}
\sn \textcolor{darkblue}{\textbf{\ipa{njɤ˧ | go˩˥!}}}
\trans J'ai mal!
\trans \textit{\textcolor{brown}{\zh{我痛!}}}
\end{exe}
\begin{exe}
\sn \textcolor{darkblue}{\textbf{\ipa{njɤ˧ | go˩˥ | ʐwæ˩˥!}}}
\trans J'ai très mal!
\trans \textit{\textcolor{brown}{\zh{我好疼!}}}
\end{exe}
\begin{exe}
\sn \textcolor{darkblue}{\textbf{\ipa{go˩-hĩ˩˥}}}
\trans \mytextsc{nmlz}: patient, malade
\trans \textit{\textcolor{brown}{\zh{病人,病的(人)}}}
\end{exe}
\begin{exe}
\sn \textcolor{darkblue}{\textbf{\ipa{hĩ˧ | go˩-hĩ˩˥}}}
\trans patient, personne malade, malade
\trans \textit{\textcolor{brown}{\zh{病人}}}
\end{exe}
\begin{exe}
\sn \textcolor{darkblue}{\textbf{\ipa{bi˧mi˧ go˩}}}
\trans avoir mal au ventre
\trans \textit{\textcolor{brown}{\zh{肚子疼}}}
\end{exe}


\paragraph{\hspace{-0.5cm} {\Large \textcolor{darkblue}{\textbf{\ipa{go˩bi˧}}}}} \hypertarget{go\string_Bbi\string_M1}{}
\markboth{\textcolor{darkblue}{\textbf{\ipa{go˩bi˧}}}}{}
\textcolor{teal}{\mytextsc{nom}} \hspace{4pt} Ton~: LM.

Lijiang (la ville).
\textcolor{brown}{\zh{丽江城。}}


\begin{exe}
\sn \textcolor{darkblue}{\textbf{\ipa{go˩bi˧-ɖʐɯ˧qo˩}}}
\trans la ville de Lijiang
\trans \textit{\textcolor{brown}{\zh{丽江城}}}
\end{exe}


\paragraph{\hspace{-0.5cm} {\Large \textcolor{darkblue}{\textbf{\ipa{go˩bo˥}}}}} \hypertarget{go\string_Bbo\string_T1}{}
\markboth{\textcolor{darkblue}{\textbf{\ipa{go˩bo˥}}}}{}
\textcolor{teal}{\mytextsc{nom}} \hspace{4pt} Ton~: LH.

Bétail, animaux domestiques.
\textcolor{brown}{\zh{牲畜。}}


 \mytextsc{clf}~: \textcolor{darkblue}{\textbf{\ipa{pʰo˧˥}}} 

\paragraph{\hspace{-0.5cm} {\Large \textcolor{darkblue}{\textbf{\ipa{gɯ˩\textsubscript{a}}}} \textsubscript{1}}} \hypertarget{gM\string_Ba1}{}
\markboth{\textcolor{darkblue}{\textbf{\ipa{gɯ˩\textsubscript{a}}}} \textsubscript{1}}{}
\textcolor{teal}{\mytextsc{verbe}} \hspace{4pt} Ton~: L\textsubscript{a}.

Croire.
\textcolor{brown}{\zh{相信。}}


\begin{exe}
\sn \textcolor{darkblue}{\textbf{\ipa{ʈʂʰɯ˧-ɳɯ˧ ʐwɤ˩-hĩ˩, | njɤ˧ | mɤ˧-gɯ˩!}}}
\trans Je ne crois pas ce qu'il dit! / Je ne le crois pas! / Je ne crois pas un mot de ce qu'il raconte!
\trans \textit{\textcolor{brown}{\zh{他说的话,我不相信!}}}
\end{exe}


\paragraph{\hspace{-0.5cm} {\Large \textcolor{darkblue}{\textbf{\ipa{gɯ˩\textsubscript{a}}}} \textsubscript{2}}} \hypertarget{gM\string_Ba2}{}
\markboth{\textcolor{darkblue}{\textbf{\ipa{gɯ˩\textsubscript{a}}}} \textsubscript{2}}{}
\textcolor{teal}{\mytextsc{adjectif}} \hspace{4pt} Ton~: L\textsubscript{a}.

Vrai, authentique, véritable.
\textcolor{brown}{\zh{真,真的。}}


\begin{exe}
\sn \textcolor{darkblue}{\textbf{\ipa{mɤ˧-gɯ˩}}}
\trans pas vrai
\trans \textit{\textcolor{brown}{\zh{不是真的}}}
\end{exe}
\begin{exe}
\sn \textcolor{darkblue}{\textbf{\ipa{gɯ˩-hĩ˩˥}}}
\trans \mytextsc{nmlz}
\trans \textit{\textcolor{brown}{\zh{真的}}}
\end{exe}
\begin{exe}
\sn \textcolor{darkblue}{\textbf{\ipa{ə˩-gɯ˩˥?}}}
\trans c'est vrai?
\trans \textit{\textcolor{brown}{\zh{真的吗?}}}
\end{exe}
\begin{exe}
\sn \textcolor{darkblue}{\textbf{\ipa{gɯ˩ wɤ˩-ɻ̍˥!}}}
\trans C'est bien ça! / C'est réellement ainsi!
\trans \textit{\textcolor{brown}{\zh{就是真的啊! / 的确是这样啊!}}}
\end{exe}
\begin{exe}
\sn \textcolor{darkblue}{\textbf{\ipa{gɯ˩-ʝi˥?}}}
\trans C'est vrai? Vraiment?
\trans \textit{\textcolor{brown}{\zh{原来是这样吗?}}}
\end{exe}
\begin{exe}
\sn \textcolor{darkblue}{\textbf{\ipa{gɯ˩ ʂv̩˩ɖv̩˩˥}}}
\trans croire (quelqu'un, quelque chose): littéralement ``penser que c'est vrai"
\trans \textit{\textcolor{brown}{\zh{相信}}}
\end{exe}
\begin{exe}
\sn \textcolor{darkblue}{\textbf{\ipa{gɯ˧ ʐwɤ˧}}}
\trans dire la vérité
\trans \textit{\textcolor{brown}{\zh{说实话}}}
\end{exe}


\paragraph{\hspace{-0.5cm} {\Large \textcolor{darkblue}{\textbf{\ipa{gɯ˩ɭɯ˧˥}}}}} \hypertarget{gM\string_Bl\string_RM\string_M\string_T1}{}
\markboth{\textcolor{darkblue}{\textbf{\ipa{gɯ˩ɭɯ˧˥}}}}{}
\textcolor{teal}{\mytextsc{verbe}} \hspace{4pt} Ton~: LM+MH\#.

Frotter (ex.: se frotter les yeux, frotter un vêtement).
\textcolor{brown}{\zh{揉。}}


\begin{exe}
\sn \textcolor{darkblue}{\textbf{\ipa{gɯ˩ɭɯ˧-ze˥}}}
\trans \mytextsc{pfv}
\trans \textit{\textcolor{brown}{\zh{揉了}}}
\end{exe}
\begin{exe}
\sn \textcolor{darkblue}{\textbf{\ipa{le˧-gɯ˩ɭɯ˩+ze˩}}}
\trans \mytextsc{accomp} \string_ \mytextsc{pfv}
\trans \textit{\textcolor{brown}{\zh{揉了}}}
\end{exe}
\begin{exe}
\sn \textcolor{darkblue}{\textbf{\ipa{le˧-gɯ˩ɭɯ˩\textasciitilde{}le˧-gɯ˩ɭɯ˩}}}
\trans \mytextsc{accomp} \mytextsc{red} \mytextsc{pfv}
\trans \textit{\textcolor{brown}{\zh{揉一揉}}}
\end{exe}


\paragraph{\hspace{-0.5cm} {\Large \textcolor{darkblue}{\textbf{\ipa{gv̩˧}}} \textsubscript{1}}} \hypertarget{gv\string_=\string_M1}{}
\markboth{\textcolor{darkblue}{\textbf{\ipa{gv̩˧}}} \textsubscript{1}}{}
\textcolor{teal}{\mytextsc{verbe}} \hspace{4pt} Ton~: M\textsubscript{c}.

S'écouler, passer (le temps passe); se passer (un événement).
\textcolor{brown}{\zh{过去 (时间)、过,发生。}}


\begin{exe}
\sn \textcolor{darkblue}{\textbf{\ipa{le˧-gv̩˩-ze˩}}}
\trans \mytextsc{accomp} \string_ \mytextsc{pfv}
\trans \textit{\textcolor{brown}{\zh{已经过去了}}}
\end{exe}
\begin{exe}
\sn \textcolor{darkblue}{\textbf{\ipa{ɖɯ˧-ɭɯ˧ gv̩˧}}}
\trans une heure se passe
\trans \textit{\textcolor{brown}{\zh{一个小时过去了}}}
\end{exe}
\begin{exe}
\sn \textcolor{darkblue}{\textbf{\ipa{tsʰe˩-ɲi˩ gv̩˩-ze˥!}}}
\trans Dix jours ont passé!
\trans \textit{\textcolor{brown}{\zh{十天过去了}}}
\end{exe}
\begin{exe}
\sn \textcolor{darkblue}{\textbf{\ipa{mɤ˧-gv̩˧-ze˧!}}}
\trans (ah là là,) ça ne va plus!
\trans \textit{\textcolor{brown}{\zh{不好了! / 不行了!}}}
\end{exe}
\begin{exe}
\sn \textcolor{darkblue}{\textbf{\ipa{ʈʂʰɯ˧ne˧-ʝi˥ | gv̩˧, -tsɯ˩-mv̩˩!}}}
\trans Ca c'est passé comme ça, à ce qu'on raconte!
\trans \textit{\textcolor{brown}{\zh{据说是这样发生的!}}}
\end{exe}


\paragraph{\hspace{-0.5cm} {\Large \textcolor{darkblue}{\textbf{\ipa{gv̩˧}}} \textsubscript{2}}} \hypertarget{gv\string_=\string_M2}{}
\markboth{\textcolor{darkblue}{\textbf{\ipa{gv̩˧}}} \textsubscript{2}}{}
\textcolor{teal}{\mytextsc{adjectif}} \hspace{4pt} Ton~: M.

Bon (bon cœur).
\textcolor{brown}{\zh{好(心好)。}}


\begin{exe}
\sn \textcolor{darkblue}{\textbf{\ipa{ɖwæ˧˥ | gv̩˧!}}}
\trans \mytextsc{intensif}.très
\trans \textit{\textcolor{brown}{\zh{很好!}}}
\end{exe}
\begin{exe}
\sn \textcolor{darkblue}{\textbf{\ipa{mɤ˧-gv̩˧!}}}
\trans \mytextsc{neg}
\trans \textit{\textcolor{brown}{\zh{不好}}}
\end{exe}


\paragraph{\hspace{-0.5cm} {\Large \textcolor{darkblue}{\textbf{\ipa{gv̩˧}}} \textsubscript{3}}} \hypertarget{gv\string_=\string_M3}{}
\markboth{\textcolor{darkblue}{\textbf{\ipa{gv̩˧}}} \textsubscript{3}}{}
\textcolor{teal}{\mytextsc{nombre}} \hspace{4pt} Ton~: M? H\#? (pas L).

9.
\textcolor{brown}{\zh{9。}}




\paragraph{\hspace{-0.5cm} {\Large \textcolor{darkblue}{\textbf{\ipa{gv̩˧}}} \textsubscript{4}}} \hypertarget{gv\string_=\string_M4}{}
\markboth{\textcolor{darkblue}{\textbf{\ipa{gv̩˧}}} \textsubscript{4}}{}
\textcolor{teal}{\mytextsc{verbe}} \hspace{4pt} Ton~: M.

Être, devenir (verbe statif).
\textcolor{brown}{\zh{系词。}}


\begin{exe}
\sn \textcolor{darkblue}{\textbf{\ipa{ʈʂʰɯ˧ | no˧ | ɲi˧gɤ˧ | ʂwæ˧-mɤ˧-gv̩˧!}}}
\trans elle a le nez moins droit que toi! (Au sujet d'une petite fille dont le nez ne ressemble pas au nez droit de son père)
\trans \textit{\textcolor{brown}{\zh{她的鼻子没有你的直!(关于一个鼻子比较扁的小女孩)}}}
\end{exe}
\begin{exe}
\sn \textcolor{darkblue}{\textbf{\ipa{ʐæ˧ni˩ | mɤ˧-gv̩˧}}}
\trans pas bien grand (en taille), pas bien impressionnant
\trans \textit{\textcolor{brown}{\zh{个子不高}}}
\end{exe}


\paragraph{\hspace{-0.5cm} {\Large \textcolor{darkblue}{\textbf{\ipa{gv̩˥}}}}} \hypertarget{gv\string_=\string_T1}{}
\markboth{\textcolor{darkblue}{\textbf{\ipa{gv̩˥}}}}{}
\textcolor{teal}{\mytextsc{verbe}} \hspace{4pt} Ton~: H.

Passer, traverser (un cours d'eau, un lac…).
\textcolor{brown}{\zh{过(一条河、一个湖……)。}}


\begin{exe}
\sn \textcolor{darkblue}{\textbf{\ipa{dʑɯ˩ gv̩˩˥}}}
\trans traverser l'eau/traverser la rivière
\trans \textit{\textcolor{brown}{\zh{过河}}}
\end{exe}


\paragraph{\hspace{-0.5cm} {\Large \textcolor{darkblue}{\textbf{\ipa{gv̩˥}}}}} \hypertarget{gv\string_=\string_T1}{}
\markboth{\textcolor{darkblue}{\textbf{\ipa{gv̩˥}}}}{}
\textcolor{teal}{\mytextsc{nom}} \hspace{4pt} Ton~: \#H.

Auge, mangeoire.
\textcolor{brown}{\zh{马槽。}}


\begin{exe}
\sn \textcolor{darkblue}{\textbf{\ipa{ʐwæ˧gv̩\#˥}}}
\trans auge du cheval
\trans \textit{\textcolor{brown}{\zh{马槽}}}
\end{exe}
 \mytextsc{clf}~: \textcolor{darkblue}{\textbf{\ipa{ɭɯ˧}}} 

\paragraph{\hspace{-0.5cm} {\Large \textcolor{darkblue}{\textbf{\ipa{gv̩˩\textsubscript{a}}}} \textsubscript{1}}} \hypertarget{gv\string_=\string_Ba1}{}
\markboth{\textcolor{darkblue}{\textbf{\ipa{gv̩˩\textsubscript{a}}}} \textsubscript{1}}{}
\textcolor{teal}{\mytextsc{verbe}} \hspace{4pt} Ton~: L\textsubscript{a}.
\ding{202} 
Cuisiner, préparer (un repas, de la nourriture).
\textcolor{brown}{\zh{做(饭)。}}


\begin{exe}
\sn \textcolor{darkblue}{\textbf{\ipa{hɑ˧ gv̩˥}}}
\trans faire la cuisine, cuisiner
\trans \textit{\textcolor{brown}{\zh{做饭}}}
\end{exe}
\begin{exe}
\sn \textcolor{darkblue}{\textbf{\ipa{le˧-gv̩˩-ze˩}}}
\trans \mytextsc{accomp} \string_ \mytextsc{pfv}
\trans \textit{\textcolor{brown}{\zh{做(饭)了}}}
\end{exe}
\begin{exe}
\sn \textcolor{darkblue}{\textbf{\ipa{njɤ˧ | hɑ˧ gv̩˥-bi˩!}}}
\trans je vais faire la cuisine!
\trans \textit{\textcolor{brown}{\zh{我来做饭吧!}}}
\end{exe}
\ding{203} 
Construire (une maison).
\textcolor{brown}{\zh{盖、建 (房子)。}}


\begin{exe}
\sn \textcolor{darkblue}{\textbf{\ipa{ʑi˧qʰwɤ˧ gv̩˩}}}
\trans construire un bâtiment
\trans \textit{\textcolor{brown}{\zh{建房}}}
\end{exe}
\ding{204} 
Fabriquer ou réparer (un outil).
\textcolor{brown}{\zh{修理、做出来(工具)。}}


\begin{exe}
\sn \textcolor{darkblue}{\textbf{\ipa{le˧-gv̩˧\textasciitilde{}gv̩˥}}}
\trans \mytextsc{red}: réparer
\trans \textit{\textcolor{brown}{\zh{\mytextsc{重叠:修理}}}}
\end{exe}
\begin{exe}
\sn \textcolor{darkblue}{\textbf{\ipa{le˧-gv̩˩ | le˧-tʰv̩˧-ze˧!}}}
\trans Ca y est, c'est réparé/c'est fabriqué/c'est fini!
\trans \textit{\textcolor{brown}{\zh{修理好了!/ 修理出来了!}}}
\end{exe}


\paragraph{\hspace{-0.5cm} {\Large \textcolor{darkblue}{\textbf{\ipa{gv̩˩\textsubscript{a}}}} \textsubscript{2}}} \hypertarget{gv\string_=\string_Ba2}{}
\markboth{\textcolor{darkblue}{\textbf{\ipa{gv̩˩\textsubscript{a}}}} \textsubscript{2}}{}
\textcolor{teal}{\mytextsc{verbe}} \hspace{4pt} Ton~: L\textsubscript{a}.

Ranger.
\textcolor{brown}{\zh{收拾。}}


\begin{exe}
\sn \textcolor{darkblue}{\textbf{\ipa{tʰi˧-gv̩˧\textasciitilde{}gv̩˥}}}
\trans \mytextsc{dur}
\trans \textit{\textcolor{brown}{\zh{\mytextsc{dur}}}}
\end{exe}
\begin{exe}
\sn \textcolor{darkblue}{\textbf{\ipa{ɖɯ˧-gv̩˧\textasciitilde{}gv̩˥-ɻ̍˩}}}
\trans ranger un peu
\trans \textit{\textcolor{brown}{\zh{收拾一下}}}
\end{exe}
\begin{exe}
\sn \textcolor{darkblue}{\textbf{\ipa{le˧-gv̩˧\textasciitilde{}gv̩˥ | tʰi˧-tɕɯ˥}}}
\trans ranger et bien mettre à sa place
\trans \textit{\textcolor{brown}{\zh{收拾,摆好}}}
\end{exe}


\paragraph{\hspace{-0.5cm} {\Large \textcolor{darkblue}{\textbf{\ipa{gv̩˩\textsubscript{a}}}} \textsubscript{3}}} \hypertarget{gv\string_=\string_Ba3}{}
\markboth{\textcolor{darkblue}{\textbf{\ipa{gv̩˩\textsubscript{a}}}} \textsubscript{3}}{}
\textcolor{teal}{\mytextsc{verbe}} \hspace{4pt} Ton~: L\textsubscript{a}.

Se coucher (le soleil se couche), décliner.
\textcolor{brown}{\zh{落下(太阳落山)。}}


\begin{exe}
\sn \textcolor{darkblue}{\textbf{\ipa{ɲi˧mi˧ gv̩˩-se˩}}}
\trans à la nuit tombée, une fois la nuit tombée, après le coucher du soleil
\trans \textit{\textcolor{brown}{\zh{在太阳落山之后,在太阳落山了以后}}}
\end{exe}
\begin{exe}
\sn \textcolor{darkblue}{\textbf{\ipa{ɲi˧mi˧ | le˧-gv̩˩-ze˩.}}}
\trans Le soleil s'est couché.
\trans \textit{\textcolor{brown}{\zh{太阳落山了。}}}
\end{exe}
\begin{exe}
\sn \textcolor{darkblue}{\textbf{\ipa{ɲi˧mi˧ | mɤ˧-gv̩˩-sɯ˩.}}}
\trans Le soleil ne s'est pas encore couché.
\trans \textit{\textcolor{brown}{\zh{太阳还没有落。}}}
\end{exe}


\paragraph{\hspace{-0.5cm} {\Large \textcolor{darkblue}{\textbf{\ipa{gv̩˧dv̩˧}}}}} \hypertarget{gv\string_=\string_Mdv\string_=\string_M1}{}
\markboth{\textcolor{darkblue}{\textbf{\ipa{gv̩˧dv̩˧}}}}{}
\textcolor{teal}{\mytextsc{nom}} \hspace{4pt} Ton~: M.

Dos.
\textcolor{brown}{\zh{脊背。}}


 \mytextsc{clf}~: \textcolor{darkblue}{\textbf{\ipa{ʈv̩˩}}} 

\paragraph{\hspace{-0.5cm} {\Large \textcolor{darkblue}{\textbf{\ipa{gv̩˧dv̩˧-gv̩˧mi˧}}}}} \hypertarget{gv\string_=\string_Mdv\string_=\string_M-gv\string_=\string_Mmi\string_M1}{}
\markboth{\textcolor{darkblue}{\textbf{\ipa{gv̩˧dv̩˧-gv̩˧mi˧}}}}{}
\textcolor{teal}{\mytextsc{nom}} \hspace{4pt} Ton~: M.
\textit{De:} \textbf{gv̩˧dv̩˧ et gv̩˧mi˧} 
Corps.
\textcolor{brown}{\zh{身体。}}


 \mytextsc{clf}~: \textcolor{darkblue}{\textbf{\ipa{ɭɯ˧}}} 

\paragraph{\hspace{-0.5cm} {\Large \textcolor{darkblue}{\textbf{\ipa{gv̩˩dʑɯ˩}}}}} \hypertarget{gv\string_=\string_Bdz£M\string_B1}{}
\markboth{\textcolor{darkblue}{\textbf{\ipa{gv̩˩dʑɯ˩}}}}{}
\textcolor{teal}{\mytextsc{adjectif}} \hspace{4pt} Ton~: L.

En colère, affligé.
\textcolor{brown}{\zh{生气。}}




\paragraph{\hspace{-0.5cm} {\Large \textcolor{darkblue}{\textbf{\ipa{gv̩˧kv̩˩}}}}} \hypertarget{gv\string_=\string_Mkv\string_=\string_B1}{}
\markboth{\textcolor{darkblue}{\textbf{\ipa{gv̩˧kv̩˩}}}}{}
\textcolor{teal}{\mytextsc{nom}} \hspace{4pt} Ton~: L\#.

Intonation, manière de s'exprimer; par extension, peut désigner les tons.
\textcolor{brown}{\zh{语调,声调。}}


\begin{exe}
\sn \textcolor{darkblue}{\textbf{\ipa{gv̩˧kv̩˩-gv̩˩li˩ | ʐwɤ˩˥}}}
\trans parler avec une élocution soignée/agréable
\trans \textit{\textcolor{brown}{\zh{说话说得好听、有口才、口若悬河、能言善辩}}}
\end{exe}


\paragraph{\hspace{-0.5cm} {\Large \textcolor{darkblue}{\textbf{\ipa{gv̩˩ɬi˩mi˩}}}}} \hypertarget{gv\string_=\string_BKi\string_Bmi\string_B1}{}
\markboth{\textcolor{darkblue}{\textbf{\ipa{gv̩˩ɬi˩mi˩}}}}{}
\textcolor{teal}{\mytextsc{nom}} \hspace{4pt} Ton~: L.

9e mois.
\textcolor{brown}{\zh{九月。}}




\paragraph{\hspace{-0.5cm} {\Large \textcolor{darkblue}{\textbf{\ipa{gv̩˧mɑ˧}}}}} \hypertarget{gv\string_=\string_MmA\string_M1}{}
\markboth{\textcolor{darkblue}{\textbf{\ipa{gv̩˧mɑ˧}}}}{}
\textcolor{teal}{\mytextsc{nom}} \hspace{4pt} Ton~: M.

Prénom masculin.
\textcolor{brown}{\zh{男性名字。}}


\begin{exe}
\sn \textcolor{darkblue}{\textbf{\ipa{hĩ˧ | ʈʂʰɯ˧-v̩˧, | gv̩˧mɑ˧ mv̩˧ʈʂæ˧˥!}}}
\trans Cette personne s'appelle \textcolor{darkblue}{\textbf{\ipa{/gv̩˧mɑ˧/}}} !
\trans \textit{\textcolor{brown}{\zh{这个人,名叫\textcolor{darkblue}{\textbf{\ipa{/gv̩˧mɑ˧/}}}!}}}
\end{exe}


\paragraph{\hspace{-0.5cm} {\Large \textcolor{darkblue}{\textbf{\ipa{gv̩˧mi˧}}}}} \hypertarget{gv\string_=\string_Mmi\string_M1}{}
\markboth{\textcolor{darkblue}{\textbf{\ipa{gv̩˧mi˧}}}}{}
\textcolor{teal}{\mytextsc{nom}} \hspace{4pt} Ton~: M.

Corps.
\textcolor{brown}{\zh{身体。}}


 \mytextsc{clf}~: \textcolor{darkblue}{\textbf{\ipa{ɭɯ˧}}} 

\paragraph{\hspace{-0.5cm} {\Large \textcolor{darkblue}{\textbf{\ipa{gv̩˩pʰæ˩}}}}} \hypertarget{gv\string_=\string_Bp\string_h\{\string_B1}{}
\markboth{\textcolor{darkblue}{\textbf{\ipa{gv̩˩pʰæ˩}}}}{}
\textcolor{teal}{\mytextsc{nom}} \hspace{4pt} Ton~: L.

Planche de bois fine: trois ou quatre centimètres.
\textcolor{brown}{\zh{相当薄的木板。}}


 \mytextsc{clf}~: \textcolor{darkblue}{\textbf{\ipa{pʰæ˧˥}}} 

\paragraph{\hspace{-0.5cm} {\Large \textcolor{darkblue}{\textbf{\ipa{gv̩˧sɯ˥-pv̩˩}}}}} \hypertarget{gv\string_=\string_MsM\string_T-pv\string_=\string_B1}{}
\markboth{\textcolor{darkblue}{\textbf{\ipa{gv̩˧sɯ˥-pv̩˩}}}}{}
\textcolor{teal}{\mytextsc{nom}} \hspace{4pt} Ton~: H\#-L.

Omoplate.
\textcolor{brown}{\zh{肩胛骨。}}


 \mytextsc{clf}~: \textcolor{darkblue}{\textbf{\ipa{kʰwɤ˥}}} 

\paragraph{\hspace{-0.5cm} {\Large \textcolor{darkblue}{\textbf{\ipa{gv̩˧tɕʰɯ˧˥}}}}} \hypertarget{gv\string_=\string_Mts£\string_hM\string_M\string_T1}{}
\markboth{\textcolor{darkblue}{\textbf{\ipa{gv̩˧tɕʰɯ˧˥}}}}{}
\textcolor{teal}{\mytextsc{verbe}} \hspace{4pt} Ton~: MH\#.

Prendre froid, attraper un rhume, attraper froid.
\textcolor{brown}{\zh{着凉。}}




\paragraph{\hspace{-0.5cm} {\Large \textcolor{darkblue}{\textbf{\ipa{gv̩˧tsʰi˩}}}}} \hypertarget{gv\string_=\string_Mts\string_hi\string_B1}{}
\markboth{\textcolor{darkblue}{\textbf{\ipa{gv̩˧tsʰi˩}}}}{}
\textcolor{teal}{\mytextsc{nombre}} \hspace{4pt} Ton~: L\#.

90.
\textcolor{brown}{\zh{90。}}




\paragraph{\hspace{-0.5cm} {\Large \textcolor{darkblue}{\textbf{\ipa{gwɤ˩\textsubscript{a}}}}}} \hypertarget{gw7\string_Ba1}{}
\markboth{\textcolor{darkblue}{\textbf{\ipa{gwɤ˩\textsubscript{a}}}}}{}
\textcolor{teal}{\mytextsc{verbe}} \hspace{4pt} Ton~: L\textsubscript{a}.

Chanter.
\textcolor{brown}{\zh{唱、唱歌。}}


\begin{exe}
\sn \textcolor{darkblue}{\textbf{\ipa{njɤ˧ | ɖɯ˧-ɖʐo˩ | gwɤ˩-ze˥!}}}
\trans j'ai chanté une chanson!
\trans \textit{\textcolor{brown}{\zh{我唱了一首歌!}}}
\end{exe}
\begin{exe}
\sn \textcolor{darkblue}{\textbf{\ipa{no˧ | ɖɯ˧-ɖʐo˩ gwɤ˩!}}}
\trans chante-nous une chanson!
\trans \textit{\textcolor{brown}{\zh{你唱一首吧!}}}
\end{exe}
\begin{exe}
\sn \textcolor{darkblue}{\textbf{\ipa{ɖɯ˧-kʰwɤ˧ gwɤ˥}}}
\trans chanter une chanson
\trans \textit{\textcolor{brown}{\zh{唱一下}}}
\end{exe}
\begin{exe}
\sn \textcolor{darkblue}{\textbf{\ipa{ɖɯ˧-kʰwɤ˧ gwɤ˥-ɻ̍˩}}}
\trans chanter une chanson
\trans \textit{\textcolor{brown}{\zh{唱一下}}}
\end{exe}
\begin{exe}
\sn \textcolor{darkblue}{\textbf{\ipa{nɑ˩-gwɤ˥}}}
\trans les chansons des Na
\trans \textit{\textcolor{brown}{\zh{摩梭民歌}}}
\end{exe}
\begin{exe}
\sn \textcolor{darkblue}{\textbf{\ipa{ʈʂʰɯ˧ | nɑ˩-gwɤ˥ F | kv̩˧˥! | hæ˧-gwɤ˩ F | kv̩˧˥! | ʁo˧dzi˩-gwɤ˩ F | kv̩˧-ʝi˥! |}}}
\trans Il sait chanter (toutes sortes de styles:) les chansons na! les chansons chinoises! les chansons tibétaines!
\trans \textit{\textcolor{brown}{\zh{他会唱很多种风格的歌曲:摩梭的,会唱!汉族的,会唱!藏族的,会唱!}}}
\end{exe}


\paragraph{\hspace{-0.5cm} {\Large \textcolor{darkblue}{\textbf{\ipa{gwɤ˩\textasciitilde{}gwɤ˧˥}}}}} \hypertarget{gw7\string_B~gw7\string_M\string_T1}{}
\markboth{\textcolor{darkblue}{\textbf{\ipa{gwɤ˩\textasciitilde{}gwɤ˧˥}}}}{}
\textcolor{teal}{\mytextsc{verbe}} \hspace{4pt} Ton~: L.

Se promener, se divertir, faire du tourisme.
\textcolor{brown}{\zh{逛,玩,游。}}


\begin{exe}
\sn \textcolor{darkblue}{\textbf{\ipa{le˧-gwɤ˩\textasciitilde{}gwɤ˩ | le˧-tsʰɯ˩-ze˩!}}}
\trans (tu) reviens de promenade!/ Alors comme ça te voilà revenu de ta promenade!
\trans \textit{\textcolor{brown}{\zh{你已经散步回来了!}}}
\end{exe}
\begin{exe}
\sn \textcolor{darkblue}{\textbf{\ipa{ʈʂʰɯ˧ | gwɤ˩\textasciitilde{}gwɤ˩-hɯ˩-ze˥!}}}
\trans Il/elle est parti(e) se promener!
\trans \textit{\textcolor{brown}{\zh{他散步去了!}}}
\end{exe}
\begin{exe}
\sn \textcolor{darkblue}{\textbf{\ipa{æ˧ʂæ˧ gwɤ˩; | qv̩˧ɻ̍˧ gwɤ˥; | nɑ˩tsʰi˩ gwɤ˥}}}
\trans ``faire le tour de la montagne \textcolor{darkblue}{\textbf{\ipa{/æ˧ʂæ˧/;}}} faire le tour de la montagne \textcolor{darkblue}{\textbf{\ipa{/qv̩˧ɻ\#˥/;}}} faire le tour de la montagne \textcolor{darkblue}{\textbf{\ipa{/nɑ˩tsʰi˩/":}}} façon de désigner les rites qu'on pratiquait sur ces montagnes: pour des enfants qui tardaient à apprendre à parler, etc.
\trans \textit{\textcolor{brown}{\zh{绕\textcolor{darkblue}{\textbf{\ipa{æ˧ʂæ˧}}}山,绕\textcolor{darkblue}{\textbf{\ipa{qv̩˧ɻ\#˥}}}山,绕\textcolor{darkblue}{\textbf{\ipa{nɑ˩tsʰi˩}}}山(做“绕山”仪式,为了求生子等)}}}
\end{exe}


\paragraph{\hspace{-0.5cm} {\Large \textcolor{darkblue}{\textbf{\ipa{gwɤ˩ʝi˧}}}}} \hypertarget{gw7\string_Bj££i\string_M1}{}
\markboth{\textcolor{darkblue}{\textbf{\ipa{gwɤ˩ʝi˧}}}}{}
\textcolor{teal}{\mytextsc{adverbe}} \hspace{4pt} Ton~: LM.

En ordre/rangé.
\textcolor{brown}{\zh{整齐。}}


\begin{exe}
\sn \textcolor{darkblue}{\textbf{\ipa{tso˧\textasciitilde{}tso˧ | gwɤ˩ʝi˧ tʰi˧-tɕɯ˥ |}}}
\trans mettre des choses en ordre, ranger des choses
\trans \textit{\textcolor{brown}{\zh{把东西摆整齐}}}
\end{exe}


\newpage
\section*{\centering- \textcolor{darkblue}{\textbf{\ipa{ɣ}}} -}
\paragraph{\hspace{-0.5cm} {\Large \textcolor{darkblue}{\textbf{\ipa{ɣɯ˥}}}}} \hypertarget{GM\string_T1}{}
\markboth{\textcolor{darkblue}{\textbf{\ipa{ɣɯ˥}}}}{}
\textcolor{teal}{\mytextsc{adjectif}} \hspace{4pt} Ton~: H.

Habile, compétent, bon.
\textcolor{brown}{\zh{能干、好(做事情做得好)。}}


\begin{exe}
\sn \textcolor{darkblue}{\textbf{\ipa{mɤ˧-ɣɯ˥}}}
\trans \mytextsc{neg}
\trans \textit{\textcolor{brown}{\zh{不能干}}}
\end{exe}
\begin{exe}
\sn \textcolor{darkblue}{\textbf{\ipa{ʈʂʰɯ˧-ɳɯ˧, | bɑ˩lɑ˩ hwæ˧ | ɣɯ˧!}}}
\trans Il/elle s'entend à acheter des vêtements! / Il/elle a du talent pour acheter des vêtements!
\trans \textit{\textcolor{brown}{\zh{他很会买衣服!}}}
\end{exe}


\paragraph{\hspace{-0.5cm} {\Large \textcolor{darkblue}{\textbf{\ipa{ɣɯ˧}}}}} \hypertarget{GM\string_M1}{}
\markboth{\textcolor{darkblue}{\textbf{\ipa{ɣɯ˧}}}}{}
\textcolor{teal}{\mytextsc{nom}} \hspace{4pt} Ton~: \#H.

Tissu.
\textcolor{brown}{\zh{布料。}}


\begin{exe}
\sn \textcolor{darkblue}{\textbf{\ipa{ɣɯ˧dzo˩, | ɣɯ˧ni˧˥, | ɣɯ˧, | ɖɯ˧-ʑi˩ ɲi˩-ze˩!}}}
\trans Le métier à tisser, la structure en bambou qui maintient les fils, le tissu, c'est de la même famille! / ça forme une famille! (Commentaire sémantique de la locutrice, au cours d'une séance où il était question de tissus et de tissage)
\trans \textit{\textcolor{brown}{\zh{织布机、竹子的框(让线不乱混)、布料,属于同一类!(直译:“都是一家的!”)}}}
\end{exe}
 \mytextsc{clf}~: \textcolor{darkblue}{\textbf{\ipa{bo˩}}} 

\paragraph{\hspace{-0.5cm} {\Large \textcolor{darkblue}{\textbf{\ipa{ɣɯ˧bo˩}}}}} \hypertarget{GM\string_Mbo\string_B1}{}
\markboth{\textcolor{darkblue}{\textbf{\ipa{ɣɯ˧bo˩}}}}{}
\textcolor{teal}{\mytextsc{nom}} \hspace{4pt} Ton~: MH\#.

Trame (lorsqu'on tisse, il y a du fil de trame, et du fil de chaîne); la trame désigne l'ensemble des fils de trame.
\textcolor{brown}{\zh{纬线、纬纱。}}


 \mytextsc{clf}~: \textcolor{darkblue}{\textbf{\ipa{bo˩}}} 

\paragraph{\hspace{-0.5cm} {\Large \textcolor{darkblue}{\textbf{\ipa{ɣɯ˧dzo˩}}}}} \hypertarget{GM\string_Mdzo\string_B1}{}
\markboth{\textcolor{darkblue}{\textbf{\ipa{ɣɯ˧dzo˩}}}}{}
\textcolor{teal}{\mytextsc{nom}} \hspace{4pt} Ton~: L\#.

Métier à tisser, appareil à tisser.
\textcolor{brown}{\zh{织布机。}}


 \mytextsc{clf}~: \textcolor{darkblue}{\textbf{\ipa{nɑ˧}}} 

\paragraph{\hspace{-0.5cm} {\Large \textcolor{darkblue}{\textbf{\ipa{ɣɯ˧ni˧˥}}}}} \hypertarget{GM\string_Mni\string_M\string_T1}{}
\markboth{\textcolor{darkblue}{\textbf{\ipa{ɣɯ˧ni˧˥}}}}{}
\textcolor{teal}{\mytextsc{nom}} \hspace{4pt} Ton~: MH\#.

Petite structure en bambou maintenant les fils du métier à tisser; ses fils (blancs) sont verticaux, et passent à travers la trame.
\textcolor{brown}{\zh{织布机的一部分:竹子的框,让线不乱混。}}


 \mytextsc{clf}~: \textcolor{darkblue}{\textbf{\ipa{dze˩}}} 

\paragraph{\hspace{-0.5cm} {\Large \textcolor{darkblue}{\textbf{\ipa{ɣɯ˩kɯ˧˥}}}}} \hypertarget{GM\string_BkM\string_M\string_T1}{}
\markboth{\textcolor{darkblue}{\textbf{\ipa{ɣɯ˩kɯ˧˥}}}}{}
\textcolor{teal}{\mytextsc{nom}} \hspace{4pt} Ton~: LM+MH\#.
\ding{202} 
Pelure de fruit ou de légume.
\textcolor{brown}{\zh{皮、鸡蛋壳、麦麸。}}


\begin{exe}
\sn \textcolor{darkblue}{\textbf{\ipa{pʰi˩ko˧-ɣɯ˩kɯ˩}}}
\trans pelure de pomme
\trans \textit{\textcolor{brown}{\zh{苹果皮}}}
\end{exe}
\begin{exe}
\sn \textcolor{darkblue}{\textbf{\ipa{jɤ˩jo˧-ɣɯ˥kɯ˩}}}
\trans pelure de pomme de terre
\trans \textit{\textcolor{brown}{\zh{洋芋皮}}}
\end{exe}
\ding{203} 
Fourrure, peau d'animal.
\textcolor{brown}{\zh{皮。}}


\begin{exe}
\sn \textcolor{darkblue}{\textbf{\ipa{ʂe˧-ɣɯ˥kɯ˩}}}
\trans peau de la viande (peau de poulet, couenne de porc...)
\trans \textit{\textcolor{brown}{\zh{肉皮:鸡皮、猪肉的皮……}}}
\end{exe}
\ding{204} 
Coquille (d'oeuf).
\textcolor{brown}{\zh{蛋壳。}}


\ding{205} 
Son (de céréale).
\textcolor{brown}{\zh{麸。}}


\begin{exe}
\sn \textcolor{darkblue}{\textbf{\ipa{dze˧ɭɯ˧-ɣɯ˩kɯ˩}}}
\trans son de blé
\trans \textit{\textcolor{brown}{\zh{小麦麸}}}
\end{exe}
 \mytextsc{clf}~: \textcolor{darkblue}{\textbf{\ipa{kʰwɤ˥}}} 

\paragraph{\hspace{-0.5cm} {\Large \textcolor{darkblue}{\textbf{\ipa{ɣɯ˩-nɑ˥mi˩}}}}} \hypertarget{GM\string_B-nA\string_Tmi\string_B1}{}
\markboth{\textcolor{darkblue}{\textbf{\ipa{ɣɯ˩-nɑ˥mi˩}}}}{}
\textcolor{teal}{\mytextsc{nom}} \hspace{4pt} Ton~: LH-.

Terme péjoratif pour désigner les Yi (groupe ethnique): ``les peaux-noires".
\textcolor{brown}{\zh{彝族(带偏见的说法)。}}


\begin{exe}
\sn \textcolor{darkblue}{\textbf{\ipa{ɣɯ˩-nɑ˥mi˩-zo˩}}}
\trans homme yi
\trans \textit{\textcolor{brown}{\zh{彝族男人}}}
\end{exe}
\begin{exe}
\sn \textcolor{darkblue}{\textbf{\ipa{ɣɯ˩-nɑ˥mi˩-mv̩˩}}}
\trans femme yi
\trans \textit{\textcolor{brown}{\zh{彝族女人}}}
\end{exe}
 \mytextsc{clf}~: \textcolor{darkblue}{\textbf{\ipa{v̩˧}}} 

\paragraph{\hspace{-0.5cm} {\Large \textcolor{darkblue}{\textbf{\ipa{ɣɯ˩˥}}}}} \hypertarget{GM\string_B\string_T1}{}
\markboth{\textcolor{darkblue}{\textbf{\ipa{ɣɯ˩˥}}}}{}
\textcolor{teal}{\mytextsc{nom}} \hspace{4pt} Ton~: LH.

Peau.
\textcolor{brown}{\zh{皮肤。}}


\begin{exe}
\sn \textcolor{darkblue}{\textbf{\ipa{ɣɯ˩ dzɯ˩˥}}}
\trans manger de la peau
\trans \textit{\textcolor{brown}{\zh{吃皮}}}
\end{exe}
\begin{exe}
\sn \textcolor{darkblue}{\textbf{\ipa{ɣɯ˩˥ | ɖɯ˧-ʂɯ˩ pʰv˩}}}
\trans littéralement 'perdre sa peau'; sens: être épuisé et blessé par un travail ardu (par exemple: abattre des arbres sur la montagne et ramener les grumes jusque dans la plaine)
\trans \textit{\textcolor{brown}{\zh{直译:‘脱皮一次’。意思:疲劳而受伤(因为做了很辛苦的工作,如:在深山老林砍树、扛树干回到坝子)}}}
\end{exe}
 \mytextsc{clf}~: \textcolor{darkblue}{\textbf{\ipa{tsʰi˥}}} cl des peaux d'animaux

\newpage
\section*{\centering- \textcolor{darkblue}{\textbf{\ipa{h}}} -}
\paragraph{\hspace{-0.5cm} {\Large \textcolor{darkblue}{\textbf{\ipa{hɑ˥}}}}} \hypertarget{hA\string_T1}{}
\markboth{\textcolor{darkblue}{\textbf{\ipa{hɑ˥}}}}{}
\textcolor{teal}{\mytextsc{nom}} \hspace{4pt} Ton~: \#H.

Nourriture.
\textcolor{brown}{\zh{饭,米饭。}}


\begin{exe}
\sn \textcolor{darkblue}{\textbf{\ipa{hɑ˧-ʈv̩˧\textasciitilde{}ʈv̩˥}}}
\trans boule de céréale (riz ou autre)
\trans \textit{\textcolor{brown}{\zh{饭坨坨、饭团}}}
\end{exe}
\begin{exe}
\sn \textcolor{darkblue}{\textbf{\ipa{hɑ˧ dzɯ˧}}}
\trans manger
\trans \textit{\textcolor{brown}{\zh{吃饭}}}
\end{exe}
\begin{exe}
\sn \textcolor{darkblue}{\textbf{\ipa{ʈʂʰɯ˧ | hɑ˧ dzɯ˧-dʑo˩!}}}
\trans Il/elle est en train de manger!
\trans \textit{\textcolor{brown}{\zh{他在吃饭!}}}
\end{exe}
\begin{exe}
\sn \textcolor{darkblue}{\textbf{\ipa{hɑ˧ʂɯ˩}}}
\trans céréales fraîches (juste après la cueillette; on en prépare des galettes particulièrement savoureuses)
\trans \textit{\textcolor{brown}{\zh{新鲜的粮食(可以用来烤很香的饼)}}}
\end{exe}


\paragraph{\hspace{-0.5cm} {\Large \textcolor{darkblue}{\textbf{\ipa{hɑ˧bɤ˥}}}}} \hypertarget{hA\string_Mb7\string_T1}{}
\markboth{\textcolor{darkblue}{\textbf{\ipa{hɑ˧bɤ˥}}}}{}
\textcolor{teal}{\mytextsc{nom}} \hspace{4pt} Ton~: H\#.

Épi de maïs.
\textcolor{brown}{\zh{玉米棒子。}}


\begin{exe}
\sn \textcolor{darkblue}{\textbf{\ipa{qʰɑ˧dze˧-hɑ˧bɤ˥}}}
\trans maïs en épi; épi de maïs
\trans \textit{\textcolor{brown}{\zh{玉米棒子}}}
\end{exe}
 \mytextsc{clf}~: \textcolor{darkblue}{\textbf{\ipa{bɤ˩}}} 

\paragraph{\hspace{-0.5cm} {\Large \textcolor{darkblue}{\textbf{\ipa{hɑ˧-bv̩˥\textasciitilde{}bv̩˩-di˩}}}}} \hypertarget{hA\string_M-bv\string_=\string_T~bv\string_=\string_B-di\string_B1}{}
\markboth{\textcolor{darkblue}{\textbf{\ipa{hɑ˧-bv̩˥\textasciitilde{}bv̩˩-di˩}}}}{}
\textcolor{teal}{\mytextsc{nom}} \hspace{4pt} Ton~: \#H--.

Étuve pour le riz.
\textcolor{brown}{\zh{甑。}}


 \mytextsc{clf}~: \textcolor{darkblue}{\textbf{\ipa{ɭɯ˧}}} 

\paragraph{\hspace{-0.5cm} {\Large \textcolor{darkblue}{\textbf{\ipa{hɑ˧-gv̩˥-di˩}}}}} \hypertarget{hA\string_M-gv\string_=\string_T-di\string_B1}{}
\markboth{\textcolor{darkblue}{\textbf{\ipa{hɑ˧-gv̩˥-di˩}}}}{}
\textcolor{teal}{\mytextsc{nom}} \hspace{4pt} Ton~: H\#-.

Fourneau.
\textcolor{brown}{\zh{炉子、灶头。}}




\paragraph{\hspace{-0.5cm} {\Large \textcolor{darkblue}{\textbf{\ipa{hɑ˧ɭɯ\#˥}}}}} \hypertarget{hA\string_Ml\string_RM\#\string_T1}{}
\markboth{\textcolor{darkblue}{\textbf{\ipa{hɑ˧ɭɯ\#˥}}}}{}
\textcolor{teal}{\mytextsc{nom}} \hspace{4pt} Ton~: \#H.

Céréales.
\textcolor{brown}{\zh{粮食。}}




\paragraph{\hspace{-0.5cm} {\Large \textcolor{darkblue}{\textbf{\ipa{hɑ˧mi˥}}}}} \hypertarget{hA\string_Mmi\string_T1}{}
\markboth{\textcolor{darkblue}{\textbf{\ipa{hɑ˧mi˥}}}}{}
\textcolor{teal}{\mytextsc{verbe}} \hspace{4pt} Ton~: H\#.

Mendier.
\textcolor{brown}{\zh{讨饭。}}


\begin{exe}
\sn \textcolor{darkblue}{\textbf{\ipa{hɑ˧mi˥-hĩ˩}}}
\trans \string_ \mytextsc{rel}: mendiant, [personne] qui mendie
\trans \textit{\textcolor{brown}{\zh{要饭的、乞丐}}}
\end{exe}
\textit{Voir~:} \hyperlink{mi\string_Ba1}{\textcolor{darkblue}{\textbf{\ipa{mi˩a}}}} 

\paragraph{\hspace{-0.5cm} {\Large \textcolor{darkblue}{\textbf{\ipa{hɑ˧pv̩˩}}}}} \hypertarget{hA\string_Mpv\string_=\string_B1}{}
\markboth{\textcolor{darkblue}{\textbf{\ipa{hɑ˧pv̩˩}}}}{}
\textcolor{teal}{\mytextsc{nom}} \hspace{4pt} Ton~: L\#.

Riz cuit 'sec': le riz tel qu'il est servi aux repas, par opposition avec le gruau de riz.
\textcolor{brown}{\zh{干的米饭(与稀饭不同)。}}




\paragraph{\hspace{-0.5cm} {\Large \textcolor{darkblue}{\textbf{\ipa{hɑ˧ʂɯ˥}}}}} \hypertarget{hA\string_Ms`M\string_T1}{}
\markboth{\textcolor{darkblue}{\textbf{\ipa{hɑ˧ʂɯ˥}}}}{}
\textcolor{teal}{\mytextsc{connecteur}} \hspace{4pt} Ton~: H\#.

Explétif, emprunté au chinois: ``quand même/aussi...".
\textcolor{brown}{\zh{还是(汉语借词)。}}

 Emprunt~: chinois \textcolor{brown}{\zh{还是}}



\paragraph{\hspace{-0.5cm} {\Large \textcolor{darkblue}{\textbf{\ipa{hɑ˧-ʐwɤ˩}}}}} \hypertarget{hA\string_M-z`w7\string_B1}{}
\markboth{\textcolor{darkblue}{\textbf{\ipa{hɑ˧-ʐwɤ˩}}}}{}
\textcolor{teal}{\mytextsc{adjectif}} \hspace{4pt} Ton~: L\#.

Avoir faim.
\textcolor{brown}{\zh{饿(饭)。}}


\textit{Voir~:} \hyperlink{z`w7\string_M1}{\textcolor{darkblue}{\textbf{\ipa{ʐwɤ˧}}}} 

\paragraph{\hspace{-0.5cm} {\Large \textcolor{darkblue}{\textbf{\ipa{hɑ˩\textsubscript{a}}}}}} \hypertarget{hA\string_Ba1}{}
\markboth{\textcolor{darkblue}{\textbf{\ipa{hɑ˩\textsubscript{a}}}}}{}
\textcolor{teal}{\mytextsc{verbe}} \hspace{4pt} Ton~: L\textsubscript{a}.

Ouvrir (les yeux); s'ouvrir (un sac).
\textcolor{brown}{\zh{睁开(眼睛)。}}


\begin{exe}
\sn \textcolor{darkblue}{\textbf{\ipa{tʰi˧-hɑ˩}}}
\trans \mytextsc{dur}
\trans \textit{\textcolor{brown}{\zh{\mytextsc{dur}}}}
\end{exe}
\begin{exe}
\sn \textcolor{darkblue}{\textbf{\ipa{njɤ˩ɭɯ˥ | gɤ˩-hɑ˥ |}}}
\trans ouvrir les yeux
\trans \textit{\textcolor{brown}{\zh{睁开眼睛}}}
\end{exe}
\begin{exe}
\sn \textcolor{darkblue}{\textbf{\ipa{njɤ˩ɭɯ˧ hɑ˩}}}
\trans ouvrir les yeux
\trans \textit{\textcolor{brown}{\zh{睁开眼睛}}}
\end{exe}


\paragraph{\hspace{-0.5cm} {\Large \textcolor{darkblue}{\textbf{\ipa{hɑ̃˧mo˥}}}}} \hypertarget{hA\string_~\string_Mmo\string_T1}{}
\markboth{\textcolor{darkblue}{\textbf{\ipa{hɑ̃˧mo˥}}}}{}
\textcolor{teal}{\mytextsc{adjectif}} \hspace{4pt} Ton~: H\#.

Âgé, vieux (personne humaine).
\textcolor{brown}{\zh{年老。}}


\begin{exe}
\sn \textcolor{darkblue}{\textbf{\ipa{hĩ˧ ʈʂʰɯ˧-v̩˧ | hɑ̃˧mo˥ | ʐwæ˩˥!}}}
\trans Cette personne est très âgée!
\trans \textit{\textcolor{brown}{\zh{这个人,年纪非常大!}}}
\end{exe}


\paragraph{\hspace{-0.5cm} {\Large \textcolor{darkblue}{\textbf{\ipa{hɑ̃˧˥}}} \textsubscript{1}}} \hypertarget{hA\string_~\string_M\string_T1}{}
\markboth{\textcolor{darkblue}{\textbf{\ipa{hɑ̃˧˥}}} \textsubscript{1}}{}
\textcolor{teal}{\mytextsc{verbe}} \hspace{4pt} Ton~: MH.

Éteindre (par exemple le feu du foyer).
\textcolor{brown}{\zh{把火炉灭了。}}


\begin{exe}
\sn \textcolor{darkblue}{\textbf{\ipa{mv̩˧ le˧-hɑ̃˧˥}}}
\trans éteindre le feu
\trans \textit{\textcolor{brown}{\zh{灭火}}}
\end{exe}


\paragraph{\hspace{-0.5cm} {\Large \textcolor{darkblue}{\textbf{\ipa{hɑ̃˧˥}}} \textsubscript{2}}} \hypertarget{hA\string_~\string_M\string_T2}{}
\markboth{\textcolor{darkblue}{\textbf{\ipa{hɑ̃˧˥}}} \textsubscript{2}}{}
\textcolor{teal}{\mytextsc{verbe}} \hspace{4pt} Ton~: MH.

Passer la nuit.
\textcolor{brown}{\zh{过夜。}}


\begin{exe}
\sn \textcolor{darkblue}{\textbf{\ipa{ɖɯ˧-hɑ̃˧ tʰi˥-hɑ̃˩ |}}}
\trans se reposer une soirée, passer une soirée/nuitée (qq part)
\trans \textit{\textcolor{brown}{\zh{过夜}}}
\end{exe}
\begin{exe}
\sn \textcolor{darkblue}{\textbf{\ipa{ʑi˧qʰwɤ˧ ɖɯ˧-ɭɯ˧-qo˧ hɑ̃˧˥}}}
\trans passer la nuit dans une maison
\trans \textit{\textcolor{brown}{\zh{在一个人家过夜}}}
\end{exe}


\paragraph{\hspace{-0.5cm} {\Large \textcolor{darkblue}{\textbf{\ipa{hɑ̃˧˥\textsubscript{a}}}}}} \hypertarget{hA\string_~\string_M\string_Ta1}{}
\markboth{\textcolor{darkblue}{\textbf{\ipa{hɑ̃˧˥\textsubscript{a}}}}}{}
\textcolor{teal}{\mytextsc{classificateur}} \hspace{4pt} Ton~: MH\textsubscript{a}.

Nuit; par extension: utilisé pour le décompte des jours.
\textcolor{brown}{\zh{量词:夜。}}


\begin{exe}
\sn \textcolor{darkblue}{\textbf{\ipa{ɖɯ˧-hɑ̃˧˥}}}
\trans une nuit
\trans \textit{\textcolor{brown}{\zh{一夜}}}
\end{exe}
\begin{exe}
\sn \textcolor{darkblue}{\textbf{\ipa{tsʰe˩-hɑ̃˩˥}}}
\trans dix soirées=10 jours
\trans \textit{\textcolor{brown}{\zh{十夜(等于十天)}}}
\end{exe}
\begin{exe}
\sn \textcolor{darkblue}{\textbf{\ipa{ɖɯ˧-hɑ̃˧ lɑ˥-dʑo˩!}}}
\trans Il ne reste qu'une soirée!
\trans \textit{\textcolor{brown}{\zh{只有一个晚上了!}}}
\end{exe}
\begin{exe}
\sn \textcolor{darkblue}{\textbf{\ipa{[F5] ɖɯ˧-hɑ̃˧-ɳɯ˥ | le˧-li˧-le˧-se˩-ze˩!}}}
\trans (Il) a tout lu en deux jours! (contexte imaginé: on offre un livre à quelqu'un; en deux jours il a tout lu)
\trans \textit{\textcolor{brown}{\zh{一个晚上就读完了!/一天之内都读完了!(情景:送一个人一本书,他马上全部读完)}}}
\end{exe}


\paragraph{\hspace{-0.5cm} {\Large \textcolor{darkblue}{\textbf{\ipa{hæ˧}}}}} \hypertarget{h\{\string_M1}{}
\markboth{\textcolor{darkblue}{\textbf{\ipa{hæ˧}}}}{}
\textcolor{teal}{\mytextsc{nom}} \hspace{4pt} Ton~: M.

Chinois (Han).
\textcolor{brown}{\zh{汉人。}}


\begin{exe}
\sn \textcolor{darkblue}{\textbf{\ipa{hæ˧-mi\#˥}}}
\trans une femme chinoise, une Chinoise (Han)
\trans \textit{\textcolor{brown}{\zh{汉族女人}}}
\end{exe}
\begin{exe}
\sn \textcolor{darkblue}{\textbf{\ipa{hæ˧-mv̩˧ hæ˧-di˧˥}}}
\trans le territoire des Chinois (Han): Chengdu, Kunming...
\trans \textit{\textcolor{brown}{\zh{汉族地区,包括成都、昆明等等}}}
\end{exe}
\begin{exe}
\sn \textcolor{darkblue}{\textbf{\ipa{hæ˧-di˩}}}
\trans le territoire des Chinois (Han): Chengdu, Kunming...; l'expression est employée pour désigner la direction du sud
\trans \textit{\textcolor{brown}{\zh{汉族地区,包括成都、昆明等等,来代指南方}}}
\end{exe}
\begin{exe}
\sn \textcolor{darkblue}{\textbf{\ipa{hæ˧-zo˧bæ˩}}}
\trans homme chinois han (terme péjoratif: littéralement 'idiot de Chinois')
\trans \textit{\textcolor{brown}{\zh{汉男人(带偏见的称呼)}}}
\end{exe}
 \mytextsc{clf}~: \textcolor{darkblue}{\textbf{\ipa{v̩˧}}} 

\paragraph{\hspace{-0.5cm} {\Large \textcolor{darkblue}{\textbf{\ipa{hæ˧di˩-ʈæ˩bɤ˩}}}}} \hypertarget{h\{\string_Mdi\string_B-t`\{\string_Bb7\string_B1}{}
\markboth{\textcolor{darkblue}{\textbf{\ipa{hæ˧di˩-ʈæ˩bɤ˩}}}}{}
\textcolor{teal}{\mytextsc{nom}} \hspace{4pt} Ton~: \mytextsc{L}.

Bhiksu, moine mendiant.
\textcolor{brown}{\zh{比丘、游僧。}}




\paragraph{\hspace{-0.5cm} {\Large \textcolor{darkblue}{\textbf{\ipa{hæ˧ɭɯ\#˥}}}}} \hypertarget{h\{\string_Ml\string_RM\#\string_T1}{}
\markboth{\textcolor{darkblue}{\textbf{\ipa{hæ˧ɭɯ\#˥}}}}{}
\textcolor{teal}{\mytextsc{nom}} \hspace{4pt} Ton~: \#H.

Sorgho, gaoliang; céréale dont on se sert pour faire du vin.
\textcolor{brown}{\zh{高粱。}}


\textit{Voir~:} \hyperlink{k7\string_Mlj7\string_B1}{\textcolor{darkblue}{\textbf{\ipa{kɤ˧ljɤ˩}}}} 

\paragraph{\hspace{-0.5cm} {\Large \textcolor{darkblue}{\textbf{\ipa{hæ˧se˧}}}}} \hypertarget{h\{\string_Mse\string_M1}{}
\markboth{\textcolor{darkblue}{\textbf{\ipa{hæ˧se˧}}}}{}
\textcolor{teal}{\mytextsc{nom}} \hspace{4pt} Ton~: M.

Agar (ressemble à une algue).
\textcolor{brown}{\zh{石花菜、海参。}}

 Emprunt~: chinois dialect\textcolor{brown}{\zh{海参}}



\paragraph{\hspace{-0.5cm} {\Large \textcolor{darkblue}{\textbf{\ipa{hæ˧ʐwɤ˩}}}}} \hypertarget{h\{\string_Mz`w7\string_B1}{}
\markboth{\textcolor{darkblue}{\textbf{\ipa{hæ˧ʐwɤ˩}}}}{}
\textcolor{teal}{\mytextsc{nom}} \hspace{4pt} Ton~: L\#.

La langue chinoise.
\textcolor{brown}{\zh{汉语。}}




\paragraph{\hspace{-0.5cm} {\Large \textcolor{darkblue}{\textbf{\ipa{hæ˩\textsubscript{a}}}}}} \hypertarget{h\{\string_Ba1}{}
\markboth{\textcolor{darkblue}{\textbf{\ipa{hæ˩\textsubscript{a}}}}}{}
\textcolor{teal}{\mytextsc{verbe}} \hspace{4pt} Ton~: L\textsubscript{a}.

Causer du tort à (note: ce mot n'est pas un emprunt, la ressemble avec le mandarin est une coïncidence).
\textcolor{brown}{\zh{祸害、害。}}


\begin{exe}
\sn \textcolor{darkblue}{\textbf{\ipa{hĩ˧ hæ˥}}}
\trans causer du tort aux gens
\trans \textit{\textcolor{brown}{\zh{害人}}}
\end{exe}
\begin{exe}
\sn \textcolor{darkblue}{\textbf{\ipa{hĩ˧ hæ˥-kv̩˩}}}
\trans qui est susceptible de causer du tort/d'être cruel
\trans \textit{\textcolor{brown}{\zh{会害人的、残忍、凶狠}}}
\end{exe}
\begin{exe}
\sn \textcolor{darkblue}{\textbf{\ipa{hĩ˧ hæ˥-zo˩}}}
\trans homme terrifiant
\trans \textit{\textcolor{brown}{\zh{可怕的人}}}
\end{exe}


\paragraph{\hspace{-0.5cm} {\Large \textcolor{darkblue}{\textbf{\ipa{hæ˧˥}}} \textsubscript{1}}} \hypertarget{h\{\string_M\string_T1}{}
\markboth{\textcolor{darkblue}{\textbf{\ipa{hæ˧˥}}} \textsubscript{1}}{}
\textcolor{teal}{\mytextsc{adjectif}} \hspace{4pt} Ton~: MH.
\ding{202} 
Souple, mou (branche…).
\textcolor{brown}{\zh{软、柔软(树枝……)。}}


\begin{exe}
\sn \textcolor{darkblue}{\textbf{\ipa{hæ˧njæ˧˥ | -gv̩˩}}}
\trans souple
\trans \textit{\textcolor{brown}{\zh{软、柔软(树枝……)}}}
\end{exe}
\ding{203} 
Léger, clair, délayé (gruau, soupe...).
\textcolor{brown}{\zh{稀(粥、汤)。}}




\paragraph{\hspace{-0.5cm} {\Large \textcolor{darkblue}{\textbf{\ipa{hæ˧˥}}} \textsubscript{2}}} \hypertarget{h\{\string_M\string_T2}{}
\markboth{\textcolor{darkblue}{\textbf{\ipa{hæ˧˥}}} \textsubscript{2}}{}
\textcolor{teal}{\mytextsc{nom}} \hspace{4pt} Ton~: MH.

Chaux.
\textcolor{brown}{\zh{石灰。}}


\begin{exe}
\sn \textcolor{darkblue}{\textbf{\ipa{hæ˧ hwæ˥}}}
\trans acheter de la chaux
\trans \textit{\textcolor{brown}{\zh{买石灰}}}
\end{exe}
\begin{exe}
\sn \textcolor{darkblue}{\textbf{\ipa{hæ˧ tɕʰi˥}}}
\trans vendre de la chaux
\trans \textit{\textcolor{brown}{\zh{卖石灰}}}
\end{exe}
\begin{exe}
\sn \textcolor{darkblue}{\textbf{\ipa{hæ˧ ki˥}}}
\trans donner de la chaux
\trans \textit{\textcolor{brown}{\zh{给石灰}}}
\end{exe}
\begin{exe}
\sn \textcolor{darkblue}{\textbf{\ipa{hæ˧ dv̩˥}}}
\trans piocher de la chaux
\trans \textit{\textcolor{brown}{\zh{挖石灰}}}
\end{exe}
\begin{exe}
\sn \textcolor{darkblue}{\textbf{\ipa{hæ˧ bæ˥}}}
\trans balayer de la chaux
\trans \textit{\textcolor{brown}{\zh{扫石灰}}}
\end{exe}
\begin{exe}
\sn \textcolor{darkblue}{\textbf{\ipa{hæ˧ gɤ˥}}}
\trans porter de la chaux
\trans \textit{\textcolor{brown}{\zh{扛石灰}}}
\end{exe}


\paragraph{\hspace{-0.5cm} {\Large \textcolor{darkblue}{\textbf{\ipa{hæ̃˧}}}}} \hypertarget{h\{\string_~\string_M1}{}
\markboth{\textcolor{darkblue}{\textbf{\ipa{hæ̃˧}}}}{}
\textcolor{teal}{\mytextsc{nom}} \hspace{4pt} Ton~: M.

Vent.
\textcolor{brown}{\zh{风。}}


\begin{exe}
\sn \textcolor{darkblue}{\textbf{\ipa{hæ̃˧ tʰv̩˧ / hæ̃˧ tʰv̩˧-ze˧}}}
\trans il y a du vent, le vent souffle
\trans \textit{\textcolor{brown}{\zh{刮风了}}}
\end{exe}
\begin{exe}
\sn \textcolor{darkblue}{\textbf{\ipa{wɤ˩˥ | hæ̃˧ tʰv̩˧-ho˩-ze˩!}}}
\trans Le vent va se lever à nouveau!/ On dirait que le vent va se remettre à souffler!
\trans \textit{\textcolor{brown}{\zh{风又要刮起来了!}}}
\end{exe}
 \mytextsc{clf}~: \textcolor{darkblue}{\textbf{\ipa{kʰwɤ˥}}} 

\paragraph{\hspace{-0.5cm} {\Large \textcolor{darkblue}{\textbf{\ipa{hæ̃˧\textsubscript{a}}}}}} \hypertarget{h\{\string_~\string_Ma1}{}
\markboth{\textcolor{darkblue}{\textbf{\ipa{hæ̃˧\textsubscript{a}}}}}{}
\textcolor{teal}{\mytextsc{verbe}} \hspace{4pt} Ton~: M\textsubscript{a}.

Vanner: verser doucement dans une vannerie; la balle s'envole à mesure, emportée par le vent.
\textcolor{brown}{\zh{扬(粮食)。}}


\begin{exe}
\sn \textcolor{darkblue}{\textbf{\ipa{hɑ˧ hæ̃˩}}}
\trans vanner du grain
\trans \textit{\textcolor{brown}{\zh{扬粮食}}}
\end{exe}
\begin{exe}
\sn \textcolor{darkblue}{\textbf{\ipa{tso˧\textasciitilde{}tso˧ hæ̃˩}}}
\trans vanner des choses
\trans \textit{\textcolor{brown}{\zh{扬东西}}}
\end{exe}


\paragraph{\hspace{-0.5cm} {\Large \textcolor{darkblue}{\textbf{\ipa{hæ̃˧do˧}}}}} \hypertarget{h\{\string_~\string_Mdo\string_M1}{}
\markboth{\textcolor{darkblue}{\textbf{\ipa{hæ̃˧do˧}}}}{}
\textcolor{teal}{\mytextsc{nom}} \hspace{4pt} Ton~: M.

Aire à battre le grain.
\textcolor{brown}{\zh{打场。}}


\begin{exe}
\sn \textcolor{darkblue}{\textbf{\ipa{hæ̃˧do˧ bæ˩}}}
\trans balayer l'aire à battre le grain
\trans \textit{\textcolor{brown}{\zh{清扫打场}}}
\end{exe}
 \mytextsc{clf}~: \textcolor{darkblue}{\textbf{\ipa{ɭɯ˧}}} 

\paragraph{\hspace{-0.5cm} {\Large \textcolor{darkblue}{\textbf{\ipa{hæ̃˧kʰɤ˧˥}}}}} \hypertarget{h\{\string_~\string_Mk\string_h7\string_M\string_T1}{}
\markboth{\textcolor{darkblue}{\textbf{\ipa{hæ̃˧kʰɤ˧˥}}}}{}
\textcolor{teal}{\mytextsc{nom}} \hspace{4pt} Ton~: MH\#.

Pièce de charpente: poutrelles de toiture: poutrelles courtes, installées en inclinaison, dans le sens de la largeur du bâtiment, sur les poutres horizontales, ʐv̩˩ɭɯ˥. Les tuiles (autrefois: les planches) reposent sur ces poutrelles.
\textcolor{brown}{\zh{椽子。}}


 \mytextsc{clf}~: \textcolor{darkblue}{\textbf{\ipa{ɭɯ˧}}} 

\paragraph{\hspace{-0.5cm} {\Large \textcolor{darkblue}{\textbf{\ipa{hæ̃˧kʰo˧}}}}} \hypertarget{h\{\string_~\string_Mk\string_ho\string_M1}{}
\markboth{\textcolor{darkblue}{\textbf{\ipa{hæ̃˧kʰo˧}}}}{}
\textcolor{teal}{\mytextsc{nom}} \hspace{4pt} Ton~: M.

Demoiselle de la noblesse.
\textcolor{brown}{\zh{小姐、公主。}}


\begin{exe}
\sn \textcolor{darkblue}{\textbf{\ipa{hæ̃˧kʰo˧-mi˧}}}
\trans même sens: demoiselle, princesse
\trans \textit{\textcolor{brown}{\zh{同上:小姐、公主}}}
\end{exe}
 \mytextsc{clf}~: \textcolor{darkblue}{\textbf{\ipa{v̩˧}}} 

\paragraph{\hspace{-0.5cm} {\Large \textcolor{darkblue}{\textbf{\ipa{hæ̃˧pɤ˧}}}}} \hypertarget{h\{\string_~\string_Mp7\string_M1}{}
\markboth{\textcolor{darkblue}{\textbf{\ipa{hæ̃˧pɤ˧}}}}{}
\textcolor{teal}{\mytextsc{nom}} \hspace{4pt} Ton~: M.

Tresse.
\textcolor{brown}{\zh{辫子。}}


 \mytextsc{clf}~: \textcolor{darkblue}{\textbf{\ipa{kʰɯ˩}}} 

\paragraph{\hspace{-0.5cm} {\Large \textcolor{darkblue}{\textbf{\ipa{hæ̃˧qʰv̩˥\$}}}}} \hypertarget{h\{\string_~\string_Mq\string_hv\string_=\string_T\$1}{}
\markboth{\textcolor{darkblue}{\textbf{\ipa{hæ̃˧qʰv̩˥\$}}}}{}
\textcolor{teal}{\mytextsc{adverbe}} \hspace{4pt} Ton~: H\$.

En pleine nuit, tard dans la nuit.
\textcolor{brown}{\zh{半夜。}}




\paragraph{\hspace{-0.5cm} {\Large \textcolor{darkblue}{\textbf{\ipa{}}}}} \hypertarget{h\{\string_~\string_Ms`M\string_B-1}{}
\markboth{\textcolor{darkblue}{\textbf{\ipa{}}}}{}
\textcolor{teal}{\mytextsc{nom}} \hspace{4pt} Ton~: L\#.

'précieux': préfixe ajouté à certains noms pour construire une appellation prestigieuse. ; n'est pas productif: on ne peut l'ajouter à: /ə˧mɑ˧/ 'mère', /ɑ˩ʁo˧/ 'maison', etc.



\begin{exe}
\sn \textcolor{darkblue}{\textbf{\ipa{hæ̃˧ʂɯ˩-to˩mi˩}}}
\trans les Piliers d'Or, les Précieux Piliers: appellation solennelle pour les deux piliers de la maison
\trans \textit{\textcolor{brown}{\zh{‘黄金柱’、‘宝贵柱’:对主屋两个柱子的庄严称呼}}}
\end{exe}
 \mytextsc{clf}~: \textcolor{darkblue}{\textbf{\ipa{nɑ˧}}} 

\paragraph{\hspace{-0.5cm} {\Large \textcolor{darkblue}{\textbf{\ipa{hæ̃˧ʂɯ˩-pæ˩pʰæ˩}}}}} \hypertarget{h\{\string_~\string_Ms`M\string_B-p\{\string_Bp\string_h\{\string_B1}{}
\markboth{\textcolor{darkblue}{\textbf{\ipa{hæ̃˧ʂɯ˩-pæ˩pʰæ˩}}}}{}
\textcolor{teal}{\mytextsc{nom}} \hspace{4pt} Ton~: L\#-.

Espalier en bois, dans la cour des fermes, pour faire sécher les épis de maïs avant égrenage.
\textcolor{brown}{\zh{粮架。}}


 \mytextsc{clf}~: \textcolor{darkblue}{\textbf{\ipa{pʰæ˧˥}}} 

\paragraph{\hspace{-0.5cm} {\Large \textcolor{darkblue}{\textbf{\ipa{hæ̃˧ʂv̩˧pɤ˥}}}}} \hypertarget{h\{\string_~\string_Ms`v\string_=\string_Mp7\string_T1}{}
\markboth{\textcolor{darkblue}{\textbf{\ipa{hæ̃˧ʂv̩˧pɤ˥}}}}{}
\textcolor{teal}{\mytextsc{nom}} \hspace{4pt} Ton~: H\#.

Mari.
\textcolor{brown}{\zh{丈夫。}}




\paragraph{\hspace{-0.5cm} {\Large \textcolor{darkblue}{\textbf{\ipa{hæ̃˧ʐɤ˥}}}}} \hypertarget{h\{\string_~\string_Mz`7\string_T1}{}
\markboth{\textcolor{darkblue}{\textbf{\ipa{hæ̃˧ʐɤ˥}}}}{}
\textcolor{teal}{\mytextsc{nom}} \hspace{4pt} Ton~: H\#.

Trace de découpe, marque de coupure.
\textcolor{brown}{\zh{切割的痕迹。}}


\begin{exe}
\sn \textcolor{darkblue}{\textbf{\ipa{hæ̃˧ʐɤ˥ tʰv̩˩-kʰwɤ˩}}}
\trans \mytextsc{n}+\mytextsc{dem}+\mytextsc{clf}: cette trace de découpe
\trans \textit{\textcolor{brown}{\zh{这道割痕}}}
\end{exe}
 \mytextsc{clf}~: \textcolor{darkblue}{\textbf{\ipa{kʰwɤ˥}}} 

\paragraph{\hspace{-0.5cm} {\Large \textcolor{darkblue}{\textbf{\ipa{hæ̃˩}}}}} \hypertarget{h\{\string_~\string_B1}{}
\markboth{\textcolor{darkblue}{\textbf{\ipa{hæ̃˩}}}}{}
\textcolor{teal}{\mytextsc{nom}} \hspace{4pt} Ton~: L.

Or (métal).
\textcolor{brown}{\zh{金子。}}


 \mytextsc{clf}~: \textcolor{darkblue}{\textbf{\ipa{ʈv̩˩}}} 

\paragraph{\hspace{-0.5cm} {\Large \textcolor{darkblue}{\textbf{\ipa{hæ̃˩-bɑ˧lɑ˩}}}}} \hypertarget{h\{\string_~\string_B-bA\string_MlA\string_B1}{}
\markboth{\textcolor{darkblue}{\textbf{\ipa{hæ̃˩-bɑ˧lɑ˩}}}}{}
\textcolor{teal}{\mytextsc{nom}} \hspace{4pt} Ton~: L-L\#.

Soie.
\textcolor{brown}{\zh{丝绸。}}


 \mytextsc{clf}~: \textcolor{darkblue}{\textbf{\ipa{ɭɯ˧}}} 

\paragraph{\hspace{-0.5cm} {\Large \textcolor{darkblue}{\textbf{\ipa{hæ̃˩bæ˩}}}}} \hypertarget{h\{\string_~\string_Bb\{\string_B1}{}
\markboth{\textcolor{darkblue}{\textbf{\ipa{hæ̃˩bæ˩}}}}{}
\textcolor{teal}{\mytextsc{verbe}} \hspace{4pt} Ton~: L.

Effectuer une danse rituelle.
\textcolor{brown}{\zh{跳大神。}}




\paragraph{\hspace{-0.5cm} {\Large \textcolor{darkblue}{\textbf{\ipa{hæ̃˩di˩}}}}} \hypertarget{h\{\string_~\string_Bdi\string_B1}{}
\markboth{\textcolor{darkblue}{\textbf{\ipa{hæ̃˩di˩}}}}{}
\textcolor{teal}{\mytextsc{nom}} \hspace{4pt} Ton~: L.

Règle.
\textcolor{brown}{\zh{尺。}}


 \mytextsc{clf}~: \textcolor{darkblue}{\textbf{\ipa{nɑ˧}}} 

\paragraph{\hspace{-0.5cm} {\Large \textcolor{darkblue}{\textbf{\ipa{hæ̃˩qʰwɤ˩}}}}} \hypertarget{h\{\string_~\string_Bq\string_hw7\string_B1}{}
\markboth{\textcolor{darkblue}{\textbf{\ipa{hæ̃˩qʰwɤ˩}}}}{}
\textcolor{teal}{\mytextsc{nom}} \hspace{4pt} Ton~: L.

Lavande.
\textcolor{brown}{\zh{薰衣草(永宁的一种植物)。}}


 \mytextsc{clf}~: \textcolor{darkblue}{\textbf{\ipa{ɭɯ˧}}} 

\paragraph{\hspace{-0.5cm} {\Large \textcolor{darkblue}{\textbf{\ipa{hæ̃˩sɤ˩}}}}} \hypertarget{h\{\string_~\string_Bs7\string_B1}{}
\markboth{\textcolor{darkblue}{\textbf{\ipa{hæ̃˩sɤ˩}}}}{}
\textcolor{teal}{\mytextsc{nom}} \hspace{4pt} Ton~: L.

Pie.
\textcolor{brown}{\zh{喜鹊。}}


 \mytextsc{clf}~: \textcolor{darkblue}{\textbf{\ipa{mi˩}}} 

\paragraph{\hspace{-0.5cm} {\Large \textcolor{darkblue}{\textbf{\ipa{hæ̃˧˥}}}}} \hypertarget{h\{\string_~\string_M\string_T1}{}
\markboth{\textcolor{darkblue}{\textbf{\ipa{hæ̃˧˥}}}}{}
\textcolor{teal}{\mytextsc{verbe}} \hspace{4pt} Ton~: MH.
\ding{202} 
Trancher, couper au moyen d'un instrument tranchant: couteau, épée…; ex.: tailler un vêtement.
\textcolor{brown}{\zh{切,裁。}}


\begin{exe}
\sn \textcolor{darkblue}{\textbf{\ipa{le˧-hæ̃˧-ze˥}}}
\trans \mytextsc{accomp} \string_ \mytextsc{pfv}
\trans \textit{\textcolor{brown}{\zh{切了}}}
\end{exe}
\begin{exe}
\sn \textcolor{darkblue}{\textbf{\ipa{tʰɑ˧-hæ̃˧˥!}}}
\trans \mytextsc{prohib}
\trans \textit{\textcolor{brown}{\zh{别切!}}}
\end{exe}
\begin{exe}
\sn \textcolor{darkblue}{\textbf{\ipa{bɑ˩lɑ˩˥ | le˧-hæ̃˧˥, | le˧-ʐv̩˧˥}}}
\trans tailler des vêtements et les coudre
\trans \textit{\textcolor{brown}{\zh{裁(布料来做)衣服,又缝(衣服) / 先裁布料,再缝衣服}}}
\end{exe}
\ding{203} 
Castrer, châtrer.
\textcolor{brown}{\zh{阉割。}}




\paragraph{\hspace{-0.5cm} {\Large \textcolor{darkblue}{\textbf{\ipa{hɤ˧}}}}} \hypertarget{h7\string_M1}{}
\markboth{\textcolor{darkblue}{\textbf{\ipa{hɤ˧}}}}{}
\textcolor{teal}{\mytextsc{nom}} \hspace{4pt} Ton~: M.

Tout.
\textcolor{brown}{\zh{全部。}}


\begin{exe}
\sn \textcolor{darkblue}{\textbf{\ipa{ɖɯ˧-hɤ˧ | mɤ˧-go˩}}}
\trans n'avoir aucune maladie, ne souffrir de rien
\trans \textit{\textcolor{brown}{\zh{一点也没病、没有任何痛苦}}}
\end{exe}
\begin{exe}
\sn \textcolor{darkblue}{\textbf{\ipa{ɖɯ˧-hɤ˧ | mɤ˧-sɯ˥}}}
\trans ne pas savoir quoi que ce soit, être ignorant de tout, ne rien savoir du tout
\trans \textit{\textcolor{brown}{\zh{什么也不知道}}}
\end{exe}
\begin{exe}
\sn \textcolor{darkblue}{\textbf{\ipa{ʈʂʰɯ˧ | ɖɯ˧-hɤ˧ hwæ˧}}}
\trans (il/elle) achète (le) tout
\trans \textit{\textcolor{brown}{\zh{他全部都买。/他什么都买。}}}
\end{exe}


\paragraph{\hspace{-0.5cm} {\Large \textcolor{darkblue}{\textbf{\ipa{hɤ˩\textsubscript{a}}}} \textsubscript{1}}} \hypertarget{h7\string_Ba1}{}
\markboth{\textcolor{darkblue}{\textbf{\ipa{hɤ˩\textsubscript{a}}}} \textsubscript{1}}{}
\textcolor{teal}{\mytextsc{verbe}} \hspace{4pt} Ton~: L\textsubscript{a}.

Chauffer au feu, sécher au feu.
\textcolor{brown}{\zh{烘干。}}


\begin{exe}
\sn \textcolor{darkblue}{\textbf{\ipa{tʰi˧-hɤ˩}}}
\trans \mytextsc{dur}
\trans \textit{\textcolor{brown}{\zh{\mytextsc{dur}}}}
\end{exe}
\begin{exe}
\sn \textcolor{darkblue}{\textbf{\ipa{le˧-hɤ˩}}}
\trans \mytextsc{accomp}
\trans \textit{\textcolor{brown}{\zh{\mytextsc{accomp}}}}
\end{exe}
\begin{exe}
\sn \textcolor{darkblue}{\textbf{\ipa{ɖɯ˧-hɤ˩-ɻ̍˩}}}
\trans \mytextsc{délimitatif} \string_ \mytextsc{inchoatif}: chauffer un coup, chauffer un peu
\trans \textit{\textcolor{brown}{\zh{烘干一下}}}
\end{exe}
\begin{exe}
\sn \textcolor{darkblue}{\textbf{\ipa{le˧-hɤ˩-ze˩, | le˧-pv̩˧-ze˧!}}}
\trans on l'a chauffé au feu, ça a séché
\trans \textit{\textcolor{brown}{\zh{烘干了,(现在)干了!}}}
\end{exe}
\begin{exe}
\sn \textcolor{darkblue}{\textbf{\ipa{ɖɯ˧-kʰwɤ˧ hɤ˥}}}
\trans chauffer quelque chose
\trans \textit{\textcolor{brown}{\zh{烘干一个东西}}}
\end{exe}


\paragraph{\hspace{-0.5cm} {\Large \textcolor{darkblue}{\textbf{\ipa{hɤ˩\textsubscript{a}}}} \textsubscript{2}}} \hypertarget{h7\string_Ba2}{}
\markboth{\textcolor{darkblue}{\textbf{\ipa{hɤ˩\textsubscript{a}}}} \textsubscript{2}}{}
\textcolor{teal}{\mytextsc{verbe}} \hspace{4pt} Ton~: L\textsubscript{a}.

Partir, forme passée perfective.
\textcolor{brown}{\zh{去,\mytextsc{过去式,°整体体。}}}




\paragraph{\hspace{-0.5cm} {\Large \textcolor{darkblue}{\textbf{\ipa{hɤ˩\textsubscript{a}}}} \textsubscript{3}}} \hypertarget{h7\string_Ba3}{}
\markboth{\textcolor{darkblue}{\textbf{\ipa{hɤ˩\textsubscript{a}}}} \textsubscript{3}}{}
\textcolor{teal}{\mytextsc{adjectif}} \hspace{4pt} Ton~: L\textsubscript{a}.

Bon, approprié, bien; capable (adjectif), habile à une technique.
\textcolor{brown}{\zh{好(技巧好),好(表扬一个人的行为)。}}


\begin{exe}
\sn \textcolor{darkblue}{\textbf{\ipa{ɖwæ˧˥ | hɤ˩˥!}}}
\trans \mytextsc{intensif}.très: c'est très bien!
\trans \textit{\textcolor{brown}{\zh{很好!}}}
\end{exe}
\begin{exe}
\sn \textcolor{darkblue}{\textbf{\ipa{mɤ˧-hɤ˩}}}
\trans \mytextsc{neg}
\trans \textit{\textcolor{brown}{\zh{不好}}}
\end{exe}
\begin{exe}
\sn \textcolor{darkblue}{\textbf{\ipa{hɤ˩-hĩ˩˥}}}
\trans \mytextsc{rel}/\mytextsc{nmlz}
\trans \textit{\textcolor{brown}{\zh{好的}}}
\end{exe}


\paragraph{\hspace{-0.5cm} {\Large \textcolor{darkblue}{\textbf{\ipa{hi˥}}}}} \hypertarget{hi\string_T1}{}
\markboth{\textcolor{darkblue}{\textbf{\ipa{hi˥}}}}{}
\textcolor{teal}{\mytextsc{nom}} \hspace{4pt} Ton~: \#H.

Dent.
\textcolor{brown}{\zh{牙齿。}}


\begin{exe}
\sn \textcolor{darkblue}{\textbf{\ipa{hi˧ go˧˥}}}
\trans avoir mal aux dents
\trans \textit{\textcolor{brown}{\zh{牙疼}}}
\end{exe}
 \mytextsc{clf}~: \textcolor{darkblue}{\textbf{\ipa{ɭɯ˧}}} 

\paragraph{\hspace{-0.5cm} {\Large \textcolor{darkblue}{\textbf{\ipa{*hi˧}}}}} \hypertarget{*hi\string_M1}{}
\markboth{\textcolor{darkblue}{\textbf{\ipa{*hi˧}}}}{}
\textcolor{teal}{\mytextsc{adjectif}} \hspace{4pt} Ton~: H?.
\textit{Archaïque} [\textcolor{brown}{\zh{古语}}] 
Rapide, rapidement (racine extraite de la forme disyllabique).
\textcolor{brown}{\zh{快。}}


\begin{exe}
\sn \textcolor{darkblue}{\textbf{\ipa{hi˧le˩ ʝi˩}}}
\trans faire rapidement
\trans \textit{\textcolor{brown}{\zh{快速做}}}
\end{exe}
\begin{exe}
\sn \textcolor{darkblue}{\textbf{\ipa{hi˧le˩ | le˧-jo˩!}}}
\trans viens vite!
\trans \textit{\textcolor{brown}{\zh{快来!}}}
\end{exe}
\begin{exe}
\sn \textcolor{darkblue}{\textbf{\ipa{ʈʂʰɯ˧ | ɖwæ˧˥ | hi˧le˩ | ʝi˧-kv̩˩!}}}
\trans Lui, il sait travailler vite!
\trans \textit{\textcolor{brown}{\zh{他做事很麻利!}}}
\end{exe}


\paragraph{\hspace{-0.5cm} {\Large \textcolor{darkblue}{\textbf{\ipa{hi˧dʑi˧}}}}} \hypertarget{hi\string_Mdz£i\string_M1}{}
\markboth{\textcolor{darkblue}{\textbf{\ipa{hi˧dʑi˧}}}}{}
\textcolor{teal}{\mytextsc{nom}} \hspace{4pt} Ton~: M.

Cape de pluie, vêtement qui protège de la pluie (en paille, écorce...).
\textcolor{brown}{\zh{蓑衣。}}


\begin{exe}
\sn \textcolor{darkblue}{\textbf{\ipa{hi˩ gi˩-ze˥, | hi˧dʑi˧ tʰi˧-mv̩˧.}}}
\trans Il pleut, mets une cape de pluie.
\trans \textit{\textcolor{brown}{\zh{下雨了,披蓑衣(雨衣)吧。}}}
\end{exe}
 \mytextsc{clf}~: \textcolor{darkblue}{\textbf{\ipa{ɭɯ˧}}} 

\paragraph{\hspace{-0.5cm} {\Large \textcolor{darkblue}{\textbf{\ipa{hi˧kʰɯ\#˥}}}}} \hypertarget{hi\string_Mk\string_hM\#\string_T1}{}
\markboth{\textcolor{darkblue}{\textbf{\ipa{hi˧kʰɯ\#˥}}}}{}
\textcolor{teal}{\mytextsc{nom}} \hspace{4pt} Ton~: \#H.

Gencive.
\textcolor{brown}{\zh{牙龈。}}


\begin{exe}
\sn \textcolor{darkblue}{\textbf{\ipa{hi˧kʰɯ˧ ʈʂʰæ˧}}}
\trans se brosser les dents; \textcolor{darkblue}{\textbf{\ipa{/hi˧kʰɯ˧/}}} peut désigner tout ce qu'on lave quand on se brosse les dents: gencives et dents.
\trans \textit{\textcolor{brown}{\zh{刷牙}}}
\end{exe}
\begin{exe}
\sn \textcolor{darkblue}{\textbf{\ipa{hi˧kʰɯ˧-ʈv̩˥}}}
\trans racine des dents
\trans \textit{\textcolor{brown}{\zh{牙根}}}
\end{exe}


\paragraph{\hspace{-0.5cm} {\Large \textcolor{darkblue}{\textbf{\ipa{hi˧qʰwɤ˩}}}}} \hypertarget{hi\string_Mq\string_hw7\string_B1}{}
\markboth{\textcolor{darkblue}{\textbf{\ipa{hi˧qʰwɤ˩}}}}{}
\textcolor{teal}{\mytextsc{nom}} \hspace{4pt} Ton~: L\#.

Dent gâtée, dent cariée, carie.
\textcolor{brown}{\zh{蛀牙。}}


 \mytextsc{clf}~: \textcolor{darkblue}{\textbf{\ipa{ɭɯ˧}}} 

\paragraph{\hspace{-0.5cm} {\Large \textcolor{darkblue}{\textbf{\ipa{hi˧tʰɑ˩}}}}} \hypertarget{hi\string_Mt\string_hA\string_B1}{}
\markboth{\textcolor{darkblue}{\textbf{\ipa{hi˧tʰɑ˩}}}}{}
\textcolor{teal}{\mytextsc{adjectif}} \hspace{4pt} Ton~: L\#.

Aiguisé, qui coupe bien, affûté.
\textcolor{brown}{\zh{锋利。}}




\paragraph{\hspace{-0.5cm} {\Large \textcolor{darkblue}{\textbf{\ipa{hi˧tʰo˧˥}}}}} \hypertarget{hi\string_Mt\string_ho\string_M\string_T1}{}
\markboth{\textcolor{darkblue}{\textbf{\ipa{hi˧tʰo˧˥}}}}{}
\textcolor{teal}{\mytextsc{nom}} \hspace{4pt} Ton~: MH\#.

Dent.
\textcolor{brown}{\zh{牙齿。}}


 \mytextsc{clf}~: \textcolor{darkblue}{\textbf{\ipa{ɭɯ˧}}} 

\paragraph{\hspace{-0.5cm} {\Large \textcolor{darkblue}{\textbf{\ipa{hi˧tsɯ˩}}}}} \hypertarget{hi\string_MtsM\string_B1}{}
\markboth{\textcolor{darkblue}{\textbf{\ipa{hi˧tsɯ˩}}}}{}
\textcolor{teal}{\mytextsc{nom}} \hspace{4pt} Ton~: L\#.

Incisives (dents).
\textcolor{brown}{\zh{门牙。}}


 \mytextsc{clf}~: \textcolor{darkblue}{\textbf{\ipa{ɭɯ˧}}} 

\paragraph{\hspace{-0.5cm} {\Large \textcolor{darkblue}{\textbf{\ipa{hi˩}}} \textsubscript{1}}} \hypertarget{hi\string_B1}{}
\markboth{\textcolor{darkblue}{\textbf{\ipa{hi˩}}} \textsubscript{1}}{}
\textcolor{teal}{\mytextsc{nom}} \hspace{4pt} Ton~: L.

Lac (monosyllabe).
\textcolor{brown}{\zh{湖、海(单音节)。}}


 \mytextsc{clf}~: \textcolor{darkblue}{\textbf{\ipa{ɭɯ˧}}} 

\paragraph{\hspace{-0.5cm} {\Large \textcolor{darkblue}{\textbf{\ipa{hi˩}}} \textsubscript{2}}} \hypertarget{hi\string_B2}{}
\markboth{\textcolor{darkblue}{\textbf{\ipa{hi˩}}} \textsubscript{2}}{}
\textcolor{teal}{\mytextsc{verbe}} \hspace{4pt} Ton~: L.

Exister, se trouver: verbe d'existence pour objets non mobiles, par exemple le Lac existe/se trouve à un endroit.
\textcolor{brown}{\zh{存在动词:固定不动的物体,如:泸沽湖。}}


\begin{exe}
\sn \textcolor{darkblue}{\textbf{\ipa{hi˩nɑ˧mi˧ | tʰi˧-hi˩}}}
\trans le Lac se trouve là/existe là/se trouve là, immuable
\trans \textit{\textcolor{brown}{\zh{有(泸沽)湖(在那儿)}}}
\end{exe}


\paragraph{\hspace{-0.5cm} {\Large \textcolor{darkblue}{\textbf{\ipa{hi˩dʑɯ˩}}}}} \hypertarget{hi\string_Bdz£M\string_B1}{}
\markboth{\textcolor{darkblue}{\textbf{\ipa{hi˩dʑɯ˩}}}}{}
\textcolor{teal}{\mytextsc{nom}} \hspace{4pt} Ton~: L.

Charbon de bois.
\textcolor{brown}{\zh{炭。}}


 \mytextsc{clf}~: \textcolor{darkblue}{\textbf{\ipa{kʰɤ˧˥}}} 

\paragraph{\hspace{-0.5cm} {\Large \textcolor{darkblue}{\textbf{\ipa{hi˩mi˩}}}}} \hypertarget{hi\string_Bmi\string_B1}{}
\markboth{\textcolor{darkblue}{\textbf{\ipa{hi˩mi˩}}}}{}
\textcolor{teal}{\mytextsc{nom}} \hspace{4pt} Ton~: L.

Langue.
\textcolor{brown}{\zh{舌头。}}


\begin{exe}
\sn \textcolor{darkblue}{\textbf{\ipa{hi˩mi˩˥, | ɻ̃˧ mɤ˧-ʑi˧! | ə˧tso˧ ʐwɤ˩-bi˩, | õ˧-lɑ˥ ɖʐv̩˩!}}}
\trans ``La langue n'a pas d'os! Ce qu'on dit, soi seul sait (si c'est la vérité)!"
\trans \textit{\textcolor{brown}{\zh{“舌头没有骨头。讲的是什么(=是否真的),只有自己才知道!”(谚语)}}}
\end{exe}
 \mytextsc{clf}~: \textcolor{darkblue}{\textbf{\ipa{ɭɯ˧}}} 

\paragraph{\hspace{-0.5cm} {\Large \textcolor{darkblue}{\textbf{\ipa{hi˩nɑ˧mi\#˥}}}}} \hypertarget{hi\string_BnA\string_Mmi\#\string_T1}{}
\markboth{\textcolor{darkblue}{\textbf{\ipa{hi˩nɑ˧mi\#˥}}}}{}
\textcolor{teal}{\mytextsc{nom}} \hspace{4pt} Ton~: LM+\#H.

Lac.
\textcolor{brown}{\zh{湖。}}


 \mytextsc{clf}~: \textcolor{darkblue}{\textbf{\ipa{ɭɯ˧}}} 

\paragraph{\hspace{-0.5cm} {\Large \textcolor{darkblue}{\textbf{\ipa{hi˩ɲi˩zo˩}}}}} \hypertarget{hi\string_BJi\string_Bzo\string_B1}{}
\markboth{\textcolor{darkblue}{\textbf{\ipa{hi˩ɲi˩zo˩}}}}{}
\textcolor{teal}{\mytextsc{nom}} \hspace{4pt} Ton~: L.

Salamandre.
\textcolor{brown}{\zh{娃娃鱼。}}


 \mytextsc{clf}~: \textcolor{darkblue}{\textbf{\ipa{mi˩}}} 

\paragraph{\hspace{-0.5cm} {\Large \textcolor{darkblue}{\textbf{\ipa{hi˩qʰɑ˩}}}}} \hypertarget{hi\string_Bq\string_hA\string_B1}{}
\markboth{\textcolor{darkblue}{\textbf{\ipa{hi˩qʰɑ˩}}}}{}
\textcolor{teal}{\mytextsc{nom}} \hspace{4pt} Ton~: L.

Orage.
\textcolor{brown}{\zh{暴雨。}}


\begin{exe}
\sn \textcolor{darkblue}{\textbf{\ipa{hi˩qʰɑ˩ lɑ˥(-ze˩)}}}
\trans l'orage éclate, il y a de l'orage
\trans \textit{\textcolor{brown}{\zh{下暴雨了}}}
\end{exe}
 \mytextsc{clf}~: \textcolor{darkblue}{\textbf{\ipa{ʂɯ˩}}} cl des fois

\paragraph{\hspace{-0.5cm} {\Large \textcolor{darkblue}{\textbf{\ipa{hi˩ʁwɤ˩-lo˧}}}}} \hypertarget{hi\string_BRw7\string_B-lo\string_M1}{}
\markboth{\textcolor{darkblue}{\textbf{\ipa{hi˩ʁwɤ˩-lo˧}}}}{}
\textcolor{teal}{\mytextsc{nom}} \hspace{4pt} Ton~: L-.

Un village de la plaine de Yongning.
\textcolor{brown}{\zh{永宁的一个村落。}}


\begin{exe}
\sn \textcolor{darkblue}{\textbf{\ipa{dʑɤ˩bv̩˧kɤ˧-sɑ˥ʁwɤ˩, | hi˩ʁwɤ˩-lo˥, | æ˩mi˧-ʁwɤ\#˥, | lɑ˧lo˧-ʁwɤ˥, | lɑ˧ŋwɤ˧, | bɤ˧tsʰo˧gv̩˥, | ə˧lɑ˧-ʁwɤ\#˥, | gæ˧ɻæ˩, | qʰæ˧tɕʰi˧, | tʰo˧ʈɯ\#˥}}}
\trans les dix villages comptant traditionnellement comme faisant partie de Yongning
\trans \textit{\textcolor{brown}{\zh{摩梭传统地理概念中,属于永宁的十个村落}}}
\end{exe}


\paragraph{\hspace{-0.5cm} {\Large \textcolor{darkblue}{\textbf{\ipa{hi˩ʐæ˥}}}}} \hypertarget{hi\string_Bz`\{\string_T1}{}
\markboth{\textcolor{darkblue}{\textbf{\ipa{hi˩ʐæ˥}}}}{}
\textcolor{teal}{\mytextsc{nom}} \hspace{4pt} Ton~: LH.
\ding{202} 
Luette.
\textcolor{brown}{\zh{小舌。}}


\begin{exe}
\sn \textcolor{darkblue}{\textbf{\ipa{qv̩˧ʈʂæ˧-bv̩˥ | hi˩ʐæ˧}}}
\trans la luette; la précision '...de la gorge' permet de lever l'ambiguïté lorsqu'il pourrait s'agir du tendon de la langue, lui aussi désigné comme \textcolor{darkblue}{\textbf{\ipa{/hi˩ʐæ˥/}}}.
\trans \textit{\textcolor{brown}{\zh{小舌}}}
\end{exe}
\ding{203} 
Tendon de la langue.
\textcolor{brown}{\zh{舌头的筋。}}


\begin{exe}
\sn \textcolor{darkblue}{\textbf{\ipa{hi˩mi˩-bv̩˧ | hi˩ʐæ˧}}}
\trans le tendon de la langue; la précision 'de la langue' permet de lever l'ambiguïté dans les cas où il pourrait aussi s'agir de la luette, elle aussi désignée comme \textcolor{darkblue}{\textbf{\ipa{/hi˩ʐæ˥/}}}.
\trans \textit{\textcolor{brown}{\zh{舌头的筋}}}
\end{exe}
 \mytextsc{clf}~: \textcolor{darkblue}{\textbf{\ipa{ɭɯ˧}}} 

\paragraph{\hspace{-0.5cm} {\Large \textcolor{darkblue}{\textbf{\ipa{hi˩˥}}}}} \hypertarget{hi\string_B\string_T1}{}
\markboth{\textcolor{darkblue}{\textbf{\ipa{hi˩˥}}}}{}
\textcolor{teal}{\mytextsc{nom}} \hspace{4pt} Ton~: LH.

Pluie.
\textcolor{brown}{\zh{雨。}}


\begin{exe}
\sn \textcolor{darkblue}{\textbf{\ipa{hi˩ gi˩˥ / hi˩ gi˩-ze˥}}}
\trans il pleut
\trans \textit{\textcolor{brown}{\zh{下雨了}}}
\end{exe}
 \mytextsc{clf}~: \textcolor{darkblue}{\textbf{\ipa{ʂɯ˩}}} fois

\paragraph{\hspace{-0.5cm} {\Large \textcolor{darkblue}{\textbf{\ipa{‑hĩ˥}}}}} \hypertarget{‑hi\string_~\string_T1}{}
\markboth{\textcolor{darkblue}{\textbf{\ipa{‑hĩ˥}}}}{}
\textcolor{teal}{\mytextsc{conjonction}} \hspace{4pt} Ton~: 0.

Relativisateur et nominalisateur.
\textcolor{brown}{\zh{关系从句/名词化。}}




\paragraph{\hspace{-0.5cm} {\Large \textcolor{darkblue}{\textbf{\ipa{hĩ˥}}}}} \hypertarget{hi\string_~\string_T1}{}
\markboth{\textcolor{darkblue}{\textbf{\ipa{hĩ˥}}}}{}
\textcolor{teal}{\mytextsc{nom}} \hspace{4pt} Ton~: \#H.

Personne; être humain; homme (sans indication de genre).
\textcolor{brown}{\zh{人。}}


\begin{exe}
\sn \textcolor{darkblue}{\textbf{\ipa{hĩ˧ | ɖɯ˧-v̩˧}}}
\trans une personne
\trans \textit{\textcolor{brown}{\zh{一个人}}}
\end{exe}
\begin{exe}
\sn \textcolor{darkblue}{\textbf{\ipa{hĩ˧-ɻ̃˧ | ɖɯ˧-lo˩}}}
\trans une lignée, une famille
\trans \textit{\textcolor{brown}{\zh{一个家族}}}
\end{exe}
\begin{exe}
\sn \textcolor{darkblue}{\textbf{\ipa{hĩ˧-mv˥ hĩ˩-di˩}}}
\trans les terres étrangères, les terres d'autres gens (par opposition avec sa propre terre natale)
\trans \textit{\textcolor{brown}{\zh{人家的地方,人家的故乡(不是自己的地方)}}}
\end{exe}
\begin{exe}
\sn \textcolor{darkblue}{\textbf{\ipa{hĩ˧-mv˥ hĩ˩-di˩ | qʰɑ˧-dʑɤ˥\textasciitilde{}dʑɤ˩, | õ˧-mv˥ õ˩-di˩ tsʰe˩ mɤ˩-gv˩!}}}
\trans Si belles soient les terres d'autrui, elles n'auront jamais la beauté de ses propres terres / de la terre natale !
\trans \textit{\textcolor{brown}{\zh{其他人的地方怎么好,也比不过自己的地方!}}}
\end{exe}
 \mytextsc{clf}~: \textcolor{darkblue}{\textbf{\ipa{v̩˧}}} 

\paragraph{\hspace{-0.5cm} {\Large \textcolor{darkblue}{\textbf{\ipa{hĩ˧bæ\#˥}}}}} \hypertarget{hi\string_~\string_Mb\{\#\string_T1}{}
\markboth{\textcolor{darkblue}{\textbf{\ipa{hĩ˧bæ\#˥}}}}{}
\textcolor{teal}{\mytextsc{nom}} \hspace{4pt} Ton~: \#H.

Invité, visiteur, hôte.
\textcolor{brown}{\zh{客人。}}


\begin{exe}
\sn \textcolor{darkblue}{\textbf{\ipa{hĩ˧bæ˧ ʝi˧}}}
\trans participer à une fête en tant qu'invité, se rendre à une fête/à une invitation
\trans \textit{\textcolor{brown}{\zh{做客}}}
\end{exe}
\begin{exe}
\sn \textcolor{darkblue}{\textbf{\ipa{hĩ˧bæ˧ tsʰɯ˧-ze˥ ! |}}}
\trans Un invité est arrivé!
\trans \textit{\textcolor{brown}{\zh{客人来了!}}}
\end{exe}
 \mytextsc{clf}~: \textcolor{darkblue}{\textbf{\ipa{v̩˧}}} 

\paragraph{\hspace{-0.5cm} {\Large \textcolor{darkblue}{\textbf{\ipa{hĩ˧hĩ\#˥}}}}} \hypertarget{hi\string_~\string_Mhi\string_~\#\string_T1}{}
\markboth{\textcolor{darkblue}{\textbf{\ipa{hĩ˧hĩ\#˥}}}}{}
\textcolor{teal}{\mytextsc{nom}} \hspace{4pt} Ton~: \#H.

Les gens extérieurs à la famille (s'oppose à: ``les gens de la famille").
\textcolor{brown}{\zh{外人。}}


 \mytextsc{clf}~: \textcolor{darkblue}{\textbf{\ipa{v̩˧}}} 

\paragraph{\hspace{-0.5cm} {\Large \textcolor{darkblue}{\textbf{\ipa{hĩ˧-lɑ˩-kv̩˩-hĩ˩}}}}} \hypertarget{hi\string_~\string_M-lA\string_B-kv\string_=\string_B-hi\string_~\string_B1}{}
\markboth{\textcolor{darkblue}{\textbf{\ipa{hĩ˧-lɑ˩-kv̩˩-hĩ˩}}}}{}
\textcolor{teal}{\mytextsc{nom}} \hspace{4pt} Ton~: -L--.

Personne dangereuse, ennemi, bandit; littéralement: ``personne susceptible de frapper les gens".
\textcolor{brown}{\zh{危险的人,仇人,敌人。}}




\paragraph{\hspace{-0.5cm} {\Large \textcolor{darkblue}{\textbf{\ipa{hĩ˧mɤ˧gɤ˥}}}}} \hypertarget{hi\string_~\string_Mm7\string_Mg7\string_T1}{}
\markboth{\textcolor{darkblue}{\textbf{\ipa{hĩ˧mɤ˧gɤ˥}}}}{}
\textcolor{teal}{\mytextsc{verbe}} \hspace{4pt} Ton~: .

Envier, être jaloux de.
\textcolor{brown}{\zh{妒忌。}}


\begin{exe}
\sn \textcolor{darkblue}{\textbf{\ipa{hĩ˧mɤ˧gɤ˥ ʝi˩}}}
\trans envier (quelqu'un), faire une crise de jalousie
\trans \textit{\textcolor{brown}{\zh{妒忌}}}
\end{exe}


\paragraph{\hspace{-0.5cm} {\Large \textcolor{darkblue}{\textbf{\ipa{hĩ˧mo˥}}}}} \hypertarget{hi\string_~\string_Mmo\string_T1}{}
\markboth{\textcolor{darkblue}{\textbf{\ipa{hĩ˧mo˥}}}}{}
\textcolor{teal}{\mytextsc{nom}} \hspace{4pt} Ton~: H\#.

Personne âgée, vieillard, vieillarde.
\textcolor{brown}{\zh{老人。}}


\begin{exe}
\sn \textcolor{darkblue}{\textbf{\ipa{hĩ˧mo˥-hĩ˩}}}
\trans \string_ \mytextsc{rel;} même sens
\trans \textit{\textcolor{brown}{\zh{老人、老的人}}}
\end{exe}
 \mytextsc{clf}~: \textcolor{darkblue}{\textbf{\ipa{v̩˧}}} 

\paragraph{\hspace{-0.5cm} {\Large \textcolor{darkblue}{\textbf{\ipa{hĩ˧mo˩}}}}} \hypertarget{hi\string_~\string_Mmo\string_B1}{}
\markboth{\textcolor{darkblue}{\textbf{\ipa{hĩ˧mo˩}}}}{}
\textcolor{teal}{\mytextsc{nom}} \hspace{4pt} Ton~: L\#.
\ding{202} 
Cadavre.
\textcolor{brown}{\zh{尸体。}}


\begin{exe}
\sn \textcolor{darkblue}{\textbf{\ipa{hĩ˧mo˩-kʰɯ˩-di˩}}}
\trans cercueil (périphrase: ``objet (dans lequel) on met le cadavre")
\trans \textit{\textcolor{brown}{\zh{棺材}}}
\end{exe}
\ding{203} 
Tombe, tombeau.
\textcolor{brown}{\zh{坟墓。}}


 \mytextsc{clf}~: \textcolor{darkblue}{\textbf{\ipa{mo˧}}} 

\paragraph{\hspace{-0.5cm} {\Large \textcolor{darkblue}{\textbf{\ipa{hĩ˧nv̩˥}}}}} \hypertarget{hi\string_~\string_Mnv\string_=\string_T1}{}
\markboth{\textcolor{darkblue}{\textbf{\ipa{hĩ˧nv̩˥}}}}{}
\textcolor{teal}{\mytextsc{verbe}} \hspace{4pt} Ton~: .

Être désobéissant, être espiègle, désobéir.
\textcolor{brown}{\zh{不听话、调皮。}}




\paragraph{\hspace{-0.5cm} {\Large \textcolor{darkblue}{\textbf{\ipa{hĩ˧-tɕʰɯ\#˥}}}}} \hypertarget{hi\string_~\string_M-ts£\string_hM\#\string_T1}{}
\markboth{\textcolor{darkblue}{\textbf{\ipa{hĩ˧-tɕʰɯ\#˥}}}}{}
\textcolor{teal}{\mytextsc{nom}} \hspace{4pt} Ton~: \#H.

Membre de la famille de même génération: frère, sœur, cousin(e) (du côté maternel).
\textcolor{brown}{\zh{同一辈的亲戚:兄弟姐妹、堂兄弟姐妹。}}


\begin{exe}
\sn \textcolor{darkblue}{\textbf{\ipa{hĩ˧-tɕʰɯ˧ - hĩ˧-ʈʂɤ\#˥}}}
\trans même sens: les gens de la même génération, dans la famille: frères, sœurs, mais aussi cousins du côté maternel
\trans \textit{\textcolor{brown}{\zh{同一辈的亲戚:兄弟姐妹、堂兄弟姐妹}}}
\end{exe}
\begin{exe}
\sn \textcolor{darkblue}{\textbf{\ipa{ʈʂʰɯ˧ | njɤ˧ | hĩ˧ tɕʰɯ˧ ɲi˥!}}}
\trans C'est mon cousin/ma cousine/quelqu'un de ma fratrie!
\trans \textit{\textcolor{brown}{\zh{他是跟我同一辈的亲戚!(=堂兄弟姐妹)}}}
\end{exe}
\begin{exe}
\sn \textcolor{darkblue}{\textbf{\ipa{hĩ˧-tɕʰɯ˧ mɤ˧-ɲi˥ F | hĩ˧-tɕʰɯ˧ ʝi˧ tʰɑ˩-kv̩˩!}}}
\trans ``Même si on n'est pas de la même famille (au départ), on peut le devenir!" Formule traditionnelle pour désigner les liens quasi-familiaux tissés entre amis, qui reviennent à des formes d'adoption au sein du cercle familial.
\trans \textit{\textcolor{brown}{\zh{“不是亲戚,也可以变成亲戚!”这个俗语来形容朋友之间的深情,变成像家人之间的感情。}}}
\end{exe}
\textit{Syn~:} \hyperlink{hi\string_~\string_M-t`s`7\#\string_T1}{\textcolor{darkblue}{\textbf{\ipa{hĩ˧-ʈʂɤ\#˥}}}}. 

\paragraph{\hspace{-0.5cm} {\Large \textcolor{darkblue}{\textbf{\ipa{hĩ˧-ʈʂɤ\#˥}}}}} \hypertarget{hi\string_~\string_M-t`s`7\#\string_T1}{}
\markboth{\textcolor{darkblue}{\textbf{\ipa{hĩ˧-ʈʂɤ\#˥}}}}{}
\textcolor{teal}{\mytextsc{nom}} \hspace{4pt} Ton~: \#H.

Membre de la famille de même génération: frère, sœur, cousin(e) (du côté maternel).
\textcolor{brown}{\zh{同一辈的亲戚:兄弟姐妹、堂兄弟姐妹。}}


\begin{exe}
\sn \textcolor{darkblue}{\textbf{\ipa{hĩ˧-tɕʰɯ˧ - hĩ˧-ʈʂɤ\#˥}}}
\trans même sens: les gens de la même génération, dans la famille: frères, sœurs, mais aussi cousins du côté maternel
\trans \textit{\textcolor{brown}{\zh{同一辈的亲戚:兄弟姐妹、堂兄弟姐妹}}}
\end{exe}
\textit{Syn~:} \hyperlink{hi\string_~\string_M-ts£\string_hM\#\string_T1}{\textcolor{darkblue}{\textbf{\ipa{hĩ˧-tɕʰɯ\#˥}}}}. 

\paragraph{\hspace{-0.5cm} {\Large \textcolor{darkblue}{\textbf{\ipa{hĩ˧˥}}} \textsubscript{1}}} \hypertarget{hi\string_~\string_M\string_T1}{}
\markboth{\textcolor{darkblue}{\textbf{\ipa{hĩ˧˥}}} \textsubscript{1}}{}
\textcolor{teal}{\mytextsc{verbe}} \hspace{4pt} Ton~: MH.

Être debout, se tenir debout.
\textcolor{brown}{\zh{站(站立)。}}




\paragraph{\hspace{-0.5cm} {\Large \textcolor{darkblue}{\textbf{\ipa{hĩ˧˥}}} \textsubscript{2}}} \hypertarget{hi\string_~\string_M\string_T2}{}
\markboth{\textcolor{darkblue}{\textbf{\ipa{hĩ˧˥}}} \textsubscript{2}}{}
\textcolor{teal}{\mytextsc{verbe}} \hspace{4pt} Ton~: MH.

Devoir, falloir.
\textcolor{brown}{\zh{应该。}}


\begin{exe}
\sn \textcolor{darkblue}{\textbf{\ipa{mɤ˧-hĩ˧}}}
\trans \mytextsc{neg}
\trans \textit{\textcolor{brown}{\zh{\mytextsc{否定}}}}
\end{exe}
\begin{exe}
\sn \textcolor{darkblue}{\textbf{\ipa{no˧ | ʝi˧-hĩ˧˥!}}}
\trans c'est à toi de le faire! / il faut que tu le fasses!
\trans \textit{\textcolor{brown}{\zh{你应该做!}}}
\end{exe}
\begin{exe}
\sn \textcolor{darkblue}{\textbf{\ipa{njɤ˧ | ʝi˧-mɤ˧-hĩ˧-hĩ˥ | (ɖɯ˧-pi˧) ʝi˧-ze˩! |}}}
\trans j'ai fait quelque chose que j'aurais pas dû!
\trans \textit{\textcolor{brown}{\zh{我做了一件不应该做的事!}}}
\end{exe}
\begin{exe}
\sn \textcolor{darkblue}{\textbf{\ipa{no˧ | lo˧ ʝi˧-hĩ˧!}}}
\trans Il faut que tu travailles!
\trans \textit{\textcolor{brown}{\zh{你应该工作啊!}}}
\end{exe}


\paragraph{\hspace{-0.5cm} {\Large \textcolor{darkblue}{\textbf{\ipa{ho˥}}} \textsubscript{1}}} \hypertarget{ho\string_T1}{}
\markboth{\textcolor{darkblue}{\textbf{\ipa{ho˥}}} \textsubscript{1}}{}
\textcolor{teal}{\mytextsc{nom}} \hspace{4pt} Ton~: \#H.

Faisan (utilisé aussi pour: cailles, et certaines poules sauvages).
\textcolor{brown}{\zh{雉。}}


\begin{exe}
\sn \textcolor{darkblue}{\textbf{\ipa{ho˧ tʰv̩˧-mi˧˥ / ho˧ tʰv̩˧-mi˥\#}}}
\trans \mytextsc{n}+\mytextsc{dem}+\mytextsc{clf}
\trans \textit{\textcolor{brown}{\zh{这只雉}}}
\end{exe}
 \mytextsc{clf}~: \textcolor{darkblue}{\textbf{\ipa{mi˩}}} 

\paragraph{\hspace{-0.5cm} {\Large \textcolor{darkblue}{\textbf{\ipa{ho˥}}} \textsubscript{2}}} \hypertarget{ho\string_T2}{}
\markboth{\textcolor{darkblue}{\textbf{\ipa{ho˥}}} \textsubscript{2}}{}
\textcolor{teal}{\mytextsc{nom}} \hspace{4pt} Ton~: \#H.

Gruau.
\textcolor{brown}{\zh{粥。}}


\begin{exe}
\sn \textcolor{darkblue}{\textbf{\ipa{ho˧ ʈʰɯ˧˥}}}
\trans boire du gruau
\trans \textit{\textcolor{brown}{\zh{喝粥}}}
\end{exe}


\paragraph{\hspace{-0.5cm} {\Large \textcolor{darkblue}{\textbf{\ipa{ho˧ɕjæ˩}}}}} \hypertarget{ho\string_Ms£j\{\string_B1}{}
\markboth{\textcolor{darkblue}{\textbf{\ipa{ho˧ɕjæ˩}}}}{}
\textcolor{teal}{\mytextsc{nom}} \hspace{4pt} Ton~: LM.

Mèche.
\textcolor{brown}{\zh{火绳,导火索。}}Dialecte chinois local~: \zh{火线。}

 Emprunt~: chinois \textcolor{brown}{\zh{火线}}



\paragraph{\hspace{-0.5cm} {\Large \textcolor{darkblue}{\textbf{\ipa{ho˧di˧}}}}} \hypertarget{ho\string_Mdi\string_M1}{}
\markboth{\textcolor{darkblue}{\textbf{\ipa{ho˧di˧}}}}{}
\textcolor{teal}{\mytextsc{nom}} \hspace{4pt} Ton~: M.

Régions chinoises (Han) du Sichuan: Yanyuan, Yanbian, Xichang...
\textcolor{brown}{\zh{四川(盐源、盐边、西昌……)。}}




\paragraph{\hspace{-0.5cm} {\Large \textcolor{darkblue}{\textbf{\ipa{ho˧dʑɯ˧tɤ˥ɻ̍˩}}}}} \hypertarget{ho\string_Mdz£M\string_Mt7\string_Tr£`̍\string_B1}{}
\markboth{\textcolor{darkblue}{\textbf{\ipa{ho˧dʑɯ˧tɤ˥ɻ̍˩}}}}{}
\textcolor{teal}{\mytextsc{nom}} \hspace{4pt} Ton~: \#H-.

Pâte, colle à base de farine, liquide visqueux.
\textcolor{brown}{\zh{浆糊,浆子。}}


\textit{Voir~:} \hyperlink{ho\string_Mdz£M\string_M\string_T1}{\textcolor{darkblue}{\textbf{\ipa{ho˧dʑɯ˧˥}}}} 

\paragraph{\hspace{-0.5cm} {\Large \textcolor{darkblue}{\textbf{\ipa{ho˧dʑɯ˧˥}}}}} \hypertarget{ho\string_Mdz£M\string_M\string_T1}{}
\markboth{\textcolor{darkblue}{\textbf{\ipa{ho˧dʑɯ˧˥}}}}{}
\textcolor{teal}{\mytextsc{nom}} \hspace{4pt} Ton~: MH\#.

Pâte, colle à base de farine, liquide visqueux.
\textcolor{brown}{\zh{浆糊,浆子。}}


\textit{Voir~:} \hyperlink{ho\string_Mdz£M\string_Mt7\string_Tr£`̍\string_B1}{\textcolor{darkblue}{\textbf{\ipa{ho˧dʑɯ˧tɤ˥ɻ̍˩}}}} 

\paragraph{\hspace{-0.5cm} {\Large \textcolor{darkblue}{\textbf{\ipa{ho˧ko˧}}}}} \hypertarget{ho\string_Mko\string_M1}{}
\markboth{\textcolor{darkblue}{\textbf{\ipa{ho˧ko˧}}}}{}
\textcolor{teal}{\mytextsc{nom}} \hspace{4pt} Ton~: M.

Grand récipient pour faire la fondue mongole.
\textcolor{brown}{\zh{火锅(汉语借词)。}}

 Emprunt~: chinois \textcolor{brown}{\zh{火锅}}

\begin{exe}
\sn \textcolor{darkblue}{\textbf{\ipa{æ̃˧-ho˧ko˥}}}
\trans récipient pour fondue en cuivre
\trans \textit{\textcolor{brown}{\zh{铜火锅}}}
\end{exe}


\paragraph{\hspace{-0.5cm} {\Large \textcolor{darkblue}{\textbf{\ipa{ho˧mi\#˥}}}}} \hypertarget{ho\string_Mmi\#\string_T1}{}
\markboth{\textcolor{darkblue}{\textbf{\ipa{ho˧mi\#˥}}}}{}
\textcolor{teal}{\mytextsc{nom}} \hspace{4pt} Ton~: \#H.

Faisan femelle.
\textcolor{brown}{\zh{母雉。}}


\begin{exe}
\sn \textcolor{darkblue}{\textbf{\ipa{ho˧mi˧ tʰv̩˧-mi˧˥ / ho˧mi˧ tʰv̩˧-mi˥\#}}}
\trans \mytextsc{n}+\mytextsc{dem}+\mytextsc{clf}
\trans \textit{\textcolor{brown}{\zh{这个母雉}}}
\end{exe}
\begin{exe}
\sn \textcolor{darkblue}{\textbf{\ipa{ho˧mi˧-ho˧pʰv̩˥ / ho˧mi˧-ho˥pʰv̩˩}}}
\trans faisan femelle et faisan mâle
\trans \textit{\textcolor{brown}{\zh{母雉与公雉}}}
\end{exe}
 \mytextsc{clf}~: \textcolor{darkblue}{\textbf{\ipa{mi˩}}} 

\paragraph{\hspace{-0.5cm} {\Large \textcolor{darkblue}{\textbf{\ipa{ho˧pʰv̩\#˥}}}}} \hypertarget{ho\string_Mp\string_hv\string_=\#\string_T1}{}
\markboth{\textcolor{darkblue}{\textbf{\ipa{ho˧pʰv̩\#˥}}}}{}
\textcolor{teal}{\mytextsc{nom}} \hspace{4pt} Ton~: \#H.

Faisan mâle.
\textcolor{brown}{\zh{公雉。}}


\begin{exe}
\sn \textcolor{darkblue}{\textbf{\ipa{ho˧pʰv̩˧ tʰv̩˧-mi˧˥ / ho˧pʰv̩˧ tʰv̩˧-mi˥\#}}}
\trans \mytextsc{n}+\mytextsc{dem}+\mytextsc{clf}
\trans \textit{\textcolor{brown}{\zh{这只公雉}}}
\end{exe}
 \mytextsc{clf}~: \textcolor{darkblue}{\textbf{\ipa{mi˩}}} 

\paragraph{\hspace{-0.5cm} {\Large \textcolor{darkblue}{\textbf{\ipa{ho˧ʈʂɯ˧}}}}} \hypertarget{ho\string_Mt`s`M\string_M1}{}
\markboth{\textcolor{darkblue}{\textbf{\ipa{ho˧ʈʂɯ˧}}}}{}
\textcolor{teal}{\mytextsc{nom}} \hspace{4pt} Ton~: M.

Armoise, \textit{Artemisia vulgaris}.
\textcolor{brown}{\zh{蒿(汉语借词:蒿枝)。}}Dialecte chinois local~: \zh{蒿草、蒿枝。}

 Emprunt~: chinois \textcolor{brown}{\zh{蒿枝}}

\textit{Voir~:} \hyperlink{ts£7\string_Mq\string_hA\#\string_T1}{\textcolor{darkblue}{\textbf{\ipa{tɕɤ˧qʰɑ\#˥}}}} 

\paragraph{\hspace{-0.5cm} {\Large \textcolor{darkblue}{\textbf{\ipa{ho˧zo\#˥}}}}} \hypertarget{ho\string_Mzo\#\string_T1}{}
\markboth{\textcolor{darkblue}{\textbf{\ipa{ho˧zo\#˥}}}}{}
\textcolor{teal}{\mytextsc{nom}} \hspace{4pt} Ton~: \#H.

Bébé faisan.
\textcolor{brown}{\zh{小雉。}}


\begin{exe}
\sn \textcolor{darkblue}{\textbf{\ipa{ho˧zo˧ tʰv̩˧-ɭɯ\#˥}}}
\trans \mytextsc{n}+\mytextsc{dem}+\mytextsc{clf}
\trans \textit{\textcolor{brown}{\zh{这只小雉}}}
\end{exe}
 \mytextsc{clf}~: \textcolor{darkblue}{\textbf{\ipa{ɭɯ˧}}} 

\paragraph{\hspace{-0.5cm} {\Large \textcolor{darkblue}{\textbf{\ipa{‑ho˩}}}}} \hypertarget{‑ho\string_B1}{}
\markboth{\textcolor{darkblue}{\textbf{\ipa{‑ho˩}}}}{}
\textcolor{teal}{\mytextsc{suffixe}} \hspace{4pt} Ton~: L.

Future / desiderative / conjecture.
\textcolor{brown}{\zh{\mytextsc{未来}\string_愿望。}}


\begin{exe}
\sn \textcolor{darkblue}{\textbf{\ipa{hi˩gi˩ ə˥-ho˩? - hi˩ gi˩ ho˥!}}}
\trans Va-t-il pleuvoir ? - Oui!
\trans \textit{\textcolor{brown}{\zh{要下雨了吗? - 是的,要下雨了!}}}
\end{exe}
\begin{exe}
\sn \textcolor{darkblue}{\textbf{\ipa{ʈʂʰɯ˧ | so˧ɲi˥ | le˧-jo˩ ho˩-hĩ˩ | ə˩-ɲi˩˥ ? - ʈʂʰɯ˧ | so˧ɲi˥ | le˧-jo˩-ho˩-ɲi˩-mæ˩.}}}
\trans Viendra-t-il demain ? - (Oui) je pense qu’il viendra demain. (Qd on est presque sûr)
\trans \textit{\textcolor{brown}{\zh{他明天要来买? - (是的,)他明天会来的。(回答表示:比较肯定。)}}}
\end{exe}
\begin{exe}
\sn \textcolor{darkblue}{\textbf{\ipa{so˧ɲi˥ | le˧-ɬi˥ | mɤ˧-ho˥!}}}
\trans Demain, ils ne seront plus en vacances! (contexte: on parle d'une crèche qui a été en vacances la semaine précédente à l'occasion de la Fête des enseignants)
\trans \textit{\textcolor{brown}{\zh{明天就不休息了!}}}
\end{exe}
\begin{exe}
\sn \textcolor{darkblue}{\textbf{\ipa{tɕʰi˧ ə˧-ho˩?}}}
\trans va(-t-il) vendre?
\trans \textit{\textcolor{brown}{\zh{要卖吗?/ 会卖吗?}}}
\end{exe}
\begin{exe}
\sn \textcolor{darkblue}{\textbf{\ipa{hwæ˧ ə˧-ho˥?}}}
\trans va(-t-il) acheter?
\trans \textit{\textcolor{brown}{\zh{要买吗?/ 会买马?}}}
\end{exe}


\paragraph{\hspace{-0.5cm} {\Large \textcolor{darkblue}{\textbf{\ipa{ho˩\textsubscript{a}}}}}} \hypertarget{ho\string_Ba1}{}
\markboth{\textcolor{darkblue}{\textbf{\ipa{ho˩\textsubscript{a}}}}}{}
\textcolor{teal}{\mytextsc{adjectif}} \hspace{4pt} Ton~: L\textsubscript{a}.

Exact, correct; adapté, convenable.
\textcolor{brown}{\zh{准确,合适。}}

 Emprunt~: chinois ancien:\textcolor{brown}{\zh{合?}}

\begin{exe}
\sn \textcolor{darkblue}{\textbf{\ipa{mɤ˧-ho˩}}}
\trans \mytextsc{neg}: faux, erroné, inapproprié
\trans \textit{\textcolor{brown}{\zh{不合适,不准,不对}}}
\end{exe}
\begin{exe}
\sn \textcolor{darkblue}{\textbf{\ipa{ho˩-ze˧!}}}
\trans \mytextsc{pfv}
\trans \textit{\textcolor{brown}{\zh{对了! / 准确!}}}
\end{exe}
\begin{exe}
\sn \textcolor{darkblue}{\textbf{\ipa{ho˩-hĩ˩˥}}}
\trans \mytextsc{rel}/\mytextsc{nmlz}
\trans \textit{\textcolor{brown}{\zh{准确的}}}
\end{exe}


\paragraph{\hspace{-0.5cm} {\Large \textcolor{darkblue}{\textbf{\ipa{ho˩ɕjæ˧}}}}} \hypertarget{ho\string_Bs£j\{\string_M1}{}
\markboth{\textcolor{darkblue}{\textbf{\ipa{ho˩ɕjæ˧}}}}{}
\textcolor{teal}{\mytextsc{nom}} \hspace{4pt} Ton~: LM.

Hysope, \textit{Elshotzia sp.}.
\textcolor{brown}{\zh{藿香。}}

 Emprunt~: chinois \textcolor{brown}{\zh{藿香}}



\paragraph{\hspace{-0.5cm} {\Large \textcolor{darkblue}{\textbf{\ipa{ho˩dʑɯ˩}}}}} \hypertarget{ho\string_Bdz£M\string_B1}{}
\markboth{\textcolor{darkblue}{\textbf{\ipa{ho˩dʑɯ˩}}}}{}
\textcolor{teal}{\mytextsc{adjectif}} \hspace{4pt} Ton~: L.

Indigent.
\textcolor{brown}{\zh{穷苦、凋敝、寒苦、竭蹶、穷乏。}}


\begin{exe}
\sn \textcolor{darkblue}{\textbf{\ipa{ho˩dʑɯ˩-ze˥}}}
\trans \mytextsc{pfv}: qui se retrouve à la rue, qui devient démuni
\trans \textit{\textcolor{brown}{\zh{变穷苦了}}}
\end{exe}


\paragraph{\hspace{-0.5cm} {\Large \textcolor{darkblue}{\textbf{\ipa{ho˩lo˧pv̩˥}}}}} \hypertarget{ho\string_Blo\string_Mpv\string_=\string_T1}{}
\markboth{\textcolor{darkblue}{\textbf{\ipa{ho˩lo˧pv̩˥}}}}{}
\textcolor{teal}{\mytextsc{nom}} \hspace{4pt} Ton~: LM+H\#.

Carotte.
\textcolor{brown}{\zh{胡萝卜。}}

 Emprunt~: chinois \textcolor{brown}{\zh{胡萝卜}}

 \mytextsc{clf}~: \textcolor{darkblue}{\textbf{\ipa{ɭɯ˧}}} 

\paragraph{\hspace{-0.5cm} {\Large \textcolor{darkblue}{\textbf{\ipa{ho˧˥}}}}} \hypertarget{ho\string_M\string_T1}{}
\markboth{\textcolor{darkblue}{\textbf{\ipa{ho˧˥}}}}{}
\textcolor{teal}{\mytextsc{verbe}} \hspace{4pt} Ton~: MH.

Siroter, boire à petites gorgées.
\textcolor{brown}{\zh{小口地喝。}}


\begin{exe}
\sn \textcolor{darkblue}{\textbf{\ipa{ʐɯ˧ ho˧˥}}}
\trans siroter du vin
\trans \textit{\textcolor{brown}{\zh{小口地喝酒}}}
\end{exe}
\begin{exe}
\sn \textcolor{darkblue}{\textbf{\ipa{ʐɯ˧ ho˧\textasciitilde{}ho˥}}}
\trans siroter du vin
\trans \textit{\textcolor{brown}{\zh{小口地喝酒}}}
\end{exe}
\begin{exe}
\sn \textcolor{darkblue}{\textbf{\ipa{ʐɯ˧ | ɖɯ˧-ho˧\textasciitilde{}ho˥}}}
\trans siroter du vin
\trans \textit{\textcolor{brown}{\zh{喝一小口酒}}}
\end{exe}


\paragraph{\hspace{-0.5cm} {\Large \textcolor{darkblue}{\textbf{\ipa{hõ˧}}}}} \hypertarget{ho\string_~\string_M1}{}
\markboth{\textcolor{darkblue}{\textbf{\ipa{hõ˧}}}}{}
\textcolor{teal}{\mytextsc{verbe}} \hspace{4pt} Ton~: M intrans.
\ding{202} 
Partir (impératif).
\textcolor{brown}{\zh{走(离开),\mytextsc{命令式。}}}


\begin{exe}
\sn \textcolor{darkblue}{\textbf{\ipa{no˧ hõ˧!}}}
\trans vas-y!
\trans \textit{\textcolor{brown}{\zh{你走吧!}}}
\end{exe}
\begin{exe}
\sn \textcolor{darkblue}{\textbf{\ipa{no˧ | le˧-hõ˧!}}}
\trans Marche!/Vas-y!
\trans \textit{\textcolor{brown}{\zh{你走吧!}}}
\end{exe}
\begin{exe}
\sn \textcolor{darkblue}{\textbf{\ipa{ə˧-ze˧\textasciitilde{}ze˥ hõ˩! / ə˧-dzɤ˥ | le˧-hõ˧!}}}
\trans salutation respectueuse à quelqu'un qui se met en chemin: Prends ton temps en chemin!
\trans \textit{\textcolor{brown}{\zh{慢走!}}}
\end{exe}
\begin{exe}
\sn \textcolor{darkblue}{\textbf{\ipa{ɑ˩pʰo˩ hõ˩˥!}}}
\trans Dehors! / Dégage!
\trans \textit{\textcolor{brown}{\zh{出去!走开!滚出去!}}}
\end{exe}
\ding{203} 
Impératif.
\textcolor{brown}{\zh{\mytextsc{命令式。}}}


\begin{exe}
\sn \textcolor{darkblue}{\textbf{\ipa{no˧ | dzɯ˧-hõ˧!}}}
\trans Mange!
\trans \textit{\textcolor{brown}{\zh{你吃吧!}}}
\end{exe}


\paragraph{\hspace{-0.5cm} {\Large \textcolor{darkblue}{\textbf{\ipa{hõ˧-ɬi˧mi\#˥}}}}} \hypertarget{ho\string_~\string_M-Ki\string_Mmi\#\string_T1}{}
\markboth{\textcolor{darkblue}{\textbf{\ipa{hõ˧-ɬi˧mi\#˥}}}}{}
\textcolor{teal}{\mytextsc{nom}} \hspace{4pt} Ton~: \#H.

8e mois.
\textcolor{brown}{\zh{八月。}}




\paragraph{\hspace{-0.5cm} {\Large \textcolor{darkblue}{\textbf{\ipa{hõ˩tsʰi˧˥}}}}} \hypertarget{ho\string_~\string_Bts\string_hi\string_M\string_T1}{}
\markboth{\textcolor{darkblue}{\textbf{\ipa{hõ˩tsʰi˧˥}}}}{}
\textcolor{teal}{\mytextsc{nombre}} \hspace{4pt} Ton~: LM+MH\#.

80.
\textcolor{brown}{\zh{80。}}




\paragraph{\hspace{-0.5cm} {\Large \textcolor{darkblue}{\textbf{\ipa{hõ˧˥}}}}} \hypertarget{ho\string_~\string_M\string_T1}{}
\markboth{\textcolor{darkblue}{\textbf{\ipa{hõ˧˥}}}}{}
\textcolor{teal}{\mytextsc{nombre}} \hspace{4pt} Ton~: MH.

8.
\textcolor{brown}{\zh{8。}}




\paragraph{\hspace{-0.5cm} {\Large \textcolor{darkblue}{\textbf{\ipa{hu˥}}}}} \hypertarget{hu\string_T1}{}
\markboth{\textcolor{darkblue}{\textbf{\ipa{hu˥}}}}{}
\textcolor{teal}{\mytextsc{verbe}} \hspace{4pt} Ton~: H.

Attendre.
\textcolor{brown}{\zh{等候。}}


\begin{exe}
\sn \textcolor{darkblue}{\textbf{\ipa{le˧-hu˥-ze˩}}}
\trans \mytextsc{accomp} \string_ \mytextsc{pfv}
\trans \textit{\textcolor{brown}{\zh{等了}}}
\end{exe}
\begin{exe}
\sn \textcolor{darkblue}{\textbf{\ipa{ɖɯ˧-hu˥ / ɖɯ˧-hu˧-ɻ̍˥}}}
\trans attendre un peu / Attends un peu!
\trans \textit{\textcolor{brown}{\zh{等一下 / 请等一下!}}}
\end{exe}
\begin{exe}
\sn \textcolor{darkblue}{\textbf{\ipa{hĩ˧ hu˧}}}
\trans attendre quelqu'un
\trans \textit{\textcolor{brown}{\zh{等人}}}
\end{exe}


\paragraph{\hspace{-0.5cm} {\Large \textcolor{darkblue}{\textbf{\ipa{hu˧mi˥\$}}}}} \hypertarget{hu\string_Mmi\string_T\$1}{}
\markboth{\textcolor{darkblue}{\textbf{\ipa{hu˧mi˥\$}}}}{}
\textcolor{teal}{\mytextsc{nom}} \hspace{4pt} Ton~: H\$.

Estomac.
\textcolor{brown}{\zh{胃。}}


 \mytextsc{clf}~: \textcolor{darkblue}{\textbf{\ipa{ɭɯ˧}}} 

\paragraph{\hspace{-0.5cm} {\Large \textcolor{darkblue}{\textbf{\ipa{hu˧˥}}} \textsubscript{1}}} \hypertarget{hu\string_M\string_T1}{}
\markboth{\textcolor{darkblue}{\textbf{\ipa{hu˧˥}}} \textsubscript{1}}{}
\textcolor{teal}{\mytextsc{verbe}} \hspace{4pt} Ton~: MH.

Avoir la nostalgie de.
\textcolor{brown}{\zh{想念。}}


\begin{exe}
\sn \textcolor{darkblue}{\textbf{\ipa{ə˧mi˧ hu˧˥}}}
\trans avoir la nostalgie de sa mère
\trans \textit{\textcolor{brown}{\zh{想念母亲}}}
\end{exe}


\paragraph{\hspace{-0.5cm} {\Large \textcolor{darkblue}{\textbf{\ipa{hu˧˥}}} \textsubscript{2}}} \hypertarget{hu\string_M\string_T2}{}
\markboth{\textcolor{darkblue}{\textbf{\ipa{hu˧˥}}} \textsubscript{2}}{}
\textcolor{teal}{\mytextsc{nom}} \hspace{4pt} Ton~: MH.

Estomac au sens large: entrailles, tripes (tout le système digestif).
\textcolor{brown}{\zh{内脏:胃、肠子等。}}


 \mytextsc{clf}~: \textcolor{darkblue}{\textbf{\ipa{ɭɯ˧}}} 

\paragraph{\hspace{-0.5cm} {\Large \textcolor{darkblue}{\textbf{\ipa{hɯ˧}}}}} \hypertarget{hM\string_M1}{}
\markboth{\textcolor{darkblue}{\textbf{\ipa{hɯ˧}}}}{}
\textcolor{teal}{\mytextsc{verbe}} \hspace{4pt} Ton~: M\textsubscript{c}.

Aller, forme passée.
\textcolor{brown}{\zh{走(过去式)。}}


\begin{exe}
\sn \textcolor{darkblue}{\textbf{\ipa{(ki˧zo˧) | lo˧ ʝi˧-hɯ˧(-ze˩)!}}}
\trans Kizo est partie travailler!
\trans \textit{\textcolor{brown}{\zh{给若(人名)干活去了!}}}
\end{exe}
\begin{exe}
\sn \textcolor{darkblue}{\textbf{\ipa{le˧-hɯ˩-hĩ˩ hĩ˩}}}
\trans personne décédée; littéralement "personne qui est partie
\trans \textit{\textcolor{brown}{\zh{委婉语:‘走了的人’=去世了的人}}}
\end{exe}


\paragraph{\hspace{-0.5cm} {\Large \textcolor{darkblue}{\textbf{\ipa{hɯ˧\textsubscript{b}}}}}} \hypertarget{hM\string_Mb1}{}
\markboth{\textcolor{darkblue}{\textbf{\ipa{hɯ˧\textsubscript{b}}}}}{}
\textcolor{teal}{\mytextsc{classificateur}} \hspace{4pt} Ton~: M\textsubscript{b}.

Quelques-uns, un peu, une petite quantité de.
\textcolor{brown}{\zh{量词:一点。}}


\begin{exe}
\sn \textcolor{darkblue}{\textbf{\ipa{ʈʂʰæ˧ɣɯ˧ ɖɯ˧-hɯ˧}}}
\trans quelques médicaments
\trans \textit{\textcolor{brown}{\zh{一些药}}}
\end{exe}


\paragraph{\hspace{-0.5cm} {\Large \textcolor{darkblue}{\textbf{\ipa{hṽ˥}}}}} \hypertarget{hv\string_~\string_T1}{}
\markboth{\textcolor{darkblue}{\textbf{\ipa{hṽ˥}}}}{}
\textcolor{teal}{\mytextsc{verbe}} \hspace{4pt} Ton~: H.

Frire (viande, légumes...), cuire au wok.
\textcolor{brown}{\zh{炒(肉、菜)。}}


\begin{exe}
\sn \textcolor{darkblue}{\textbf{\ipa{hṽ˧\textasciitilde{}hṽ˧}}}
\trans \mytextsc{red}
\trans \textit{\textcolor{brown}{\zh{重叠}}}
\end{exe}
\begin{exe}
\sn \textcolor{darkblue}{\textbf{\ipa{le˧-hṽ˧\textasciitilde{}hṽ˧}}}
\trans \mytextsc{accomp} \mytextsc{red}
\trans \textit{\textcolor{brown}{\zh{\mytextsc{accomp} \mytextsc{red}}}}
\end{exe}
\begin{exe}
\sn \textcolor{darkblue}{\textbf{\ipa{hṽ˧\textasciitilde{}hṽ˧-ze˩}}}
\trans \mytextsc{red} \mytextsc{pfv}
\trans \textit{\textcolor{brown}{\zh{炒了}}}
\end{exe}
\begin{exe}
\sn \textcolor{darkblue}{\textbf{\ipa{ʂe˧ hṽ˧\textasciitilde{}hṽ˧}}}
\trans frire de la viande
\trans \textit{\textcolor{brown}{\zh{炒肉}}}
\end{exe}
\begin{exe}
\sn \textcolor{darkblue}{\textbf{\ipa{v̩˩tsʰɤ˧ hṽ˧\textasciitilde{}hṽ˧}}}
\trans frire des légumes
\trans \textit{\textcolor{brown}{\zh{炒蔬菜}}}
\end{exe}
\begin{exe}
\sn \textcolor{darkblue}{\textbf{\ipa{læ˧tsɯ˥ hṽ˩\textasciitilde{}hṽ˩}}}
\trans frire des piments
\trans \textit{\textcolor{brown}{\zh{炒辣椒}}}
\end{exe}
\begin{exe}
\sn \textcolor{darkblue}{\textbf{\ipa{hɑ˧ hṽ˧\textasciitilde{}hṽ˧}}}
\trans frire un plat/faire la cuisine/frire du riz/de la nourriture
\trans \textit{\textcolor{brown}{\zh{炒饭}}}
\end{exe}


\paragraph{\hspace{-0.5cm} {\Large \textcolor{darkblue}{\textbf{\ipa{hṽ˥}}}}} \hypertarget{hv\string_~\string_T1}{}
\markboth{\textcolor{darkblue}{\textbf{\ipa{hṽ˥}}}}{}
\textcolor{teal}{\mytextsc{nom}} \hspace{4pt} Ton~: \#H.

Poils (pour les animaux, y compris les épines du hérisson; aussi pour les hommes).
\textcolor{brown}{\zh{毛、豪猪的翎。}}


 \mytextsc{clf}~: \textcolor{darkblue}{\textbf{\ipa{kʰɯ˩}}} 

\paragraph{\hspace{-0.5cm} {\Large \textcolor{darkblue}{\textbf{\ipa{hṽ˧dɤ˧ɻ\#˥}}}}} \hypertarget{hv\string_~\string_Md7\string_Mr£`\#\string_T1}{}
\markboth{\textcolor{darkblue}{\textbf{\ipa{hṽ˧dɤ˧ɻ\#˥}}}}{}
\textcolor{teal}{\mytextsc{adjectif}} \hspace{4pt} Ton~: \#H.

Maladroit, incapable.
\textcolor{brown}{\zh{笨拙,经常损坏东西。}}


\begin{exe}
\sn \textcolor{darkblue}{\textbf{\ipa{hṽ˩-hĩ˩˥}}}
\trans \mytextsc{rel}
\trans \textit{\textcolor{brown}{\zh{笨拙的}}}
\end{exe}
\begin{exe}
\sn \textcolor{darkblue}{\textbf{\ipa{hṽ˧dɤ˧ɻ̍˧\textasciitilde{}hṽ˧dɤ˧ɻ̍˧-zo˥}}}
\trans un maladroit
\trans \textit{\textcolor{brown}{\zh{笨手笨脚的(男)人}}}
\end{exe}
\begin{exe}
\sn \textcolor{darkblue}{\textbf{\ipa{[F5] hṽ˩ɖɻ̍˩\textasciitilde{}hṽ˧ɖɻ̍˧-gv̩˧}}}
\trans \mytextsc{red}
\trans \textit{\textcolor{brown}{\zh{重叠:笨笨的}}}
\end{exe}


\paragraph{\hspace{-0.5cm} {\Large \textcolor{darkblue}{\textbf{\ipa{hṽ˧\textasciitilde{}hṽ˩-ɖʐæ˩\textasciitilde{}ɖʐæ˩}}}}} \hypertarget{hv\string_~\string_M~hv\string_~\string_B-d`z`\{\string_B~d`z`\{\string_B1}{}
\markboth{\textcolor{darkblue}{\textbf{\ipa{hṽ˧\textasciitilde{}hṽ˩-ɖʐæ˩\textasciitilde{}ɖʐæ˩}}}}{}
\textcolor{teal}{\mytextsc{adjectif}} \hspace{4pt} Ton~: L\#-.
\textit{De:} \textbf{hṽ˩a 1} 
Tout rouge.
\textcolor{brown}{\zh{红红的。}}




\paragraph{\hspace{-0.5cm} {\Large \textcolor{darkblue}{\textbf{\ipa{hṽ˧nɑ˩}}}}} \hypertarget{hv\string_~\string_MnA\string_B1}{}
\markboth{\textcolor{darkblue}{\textbf{\ipa{hṽ˧nɑ˩}}}}{}
\textcolor{teal}{\mytextsc{nom}} \hspace{4pt} Ton~: L\#.

Bête sauvage.
\textcolor{brown}{\zh{野兽。}}


 \mytextsc{clf}~: \textcolor{darkblue}{\textbf{\ipa{mi˩}}} 

\paragraph{\hspace{-0.5cm} {\Large \textcolor{darkblue}{\textbf{\ipa{hṽ˩\textsubscript{a}}}} \textsubscript{1}}} \hypertarget{hv\string_~\string_Ba1}{}
\markboth{\textcolor{darkblue}{\textbf{\ipa{hṽ˩\textsubscript{a}}}} \textsubscript{1}}{}
\textcolor{teal}{\mytextsc{adjectif}} \hspace{4pt} Ton~: L\textsubscript{a}.

Rouge (ex.: vêtement rouge, sang rouge).
\textcolor{brown}{\zh{红色的。}}




\paragraph{\hspace{-0.5cm} {\Large \textcolor{darkblue}{\textbf{\ipa{hṽ˩\textsubscript{a}}}} \textsubscript{2}}} \hypertarget{hv\string_~\string_Ba2}{}
\markboth{\textcolor{darkblue}{\textbf{\ipa{hṽ˩\textsubscript{a}}}} \textsubscript{2}}{}
\textcolor{teal}{\mytextsc{adjectif}} \hspace{4pt} Ton~: L\textsubscript{a}.

De petite taille, de petite stature; bas.
\textcolor{brown}{\zh{矮、低。}}




\paragraph{\hspace{-0.5cm} {\Large \textcolor{darkblue}{\textbf{\ipa{hṽ˩-ɖʐæ˩ɻæ˥}}}}} \hypertarget{hv\string_~\string_B-d`z`\{\string_Br£`\{\string_T1}{}
\markboth{\textcolor{darkblue}{\textbf{\ipa{hṽ˩-ɖʐæ˩ɻæ˥}}}}{}
\textcolor{teal}{\mytextsc{adjectif}} \hspace{4pt} Ton~: L+H\#.
\textit{De:} \textbf{hṽ˩a 1} 
Tout rouge.
\textcolor{brown}{\zh{红红的。}}


\begin{exe}
\sn \textcolor{darkblue}{\textbf{\ipa{hṽ˩ɖʐæ˩ɻæ˥-gv̩˩}}}
\trans tout rouge
\trans \textit{\textcolor{brown}{\zh{红红的}}}
\end{exe}
\begin{exe}
\sn \textcolor{darkblue}{\textbf{\ipa{[F5] hṽ˩ɖʐæ˩˥ | hṽ˩ɖʐæ˩˥ gv̩˩}}}
\trans \mytextsc{red;} les deux premières syllabes ont une fréquence fondamentale nettement plus haute que les deux suivantes
\trans \textit{\textcolor{brown}{\zh{重叠}}}
\end{exe}


\paragraph{\hspace{-0.5cm} {\Large \textcolor{darkblue}{\textbf{\ipa{hṽ˩\textasciitilde{}hṽ˩}}}}} \hypertarget{hv\string_~\string_B~hv\string_~\string_B1}{}
\markboth{\textcolor{darkblue}{\textbf{\ipa{hṽ˩\textasciitilde{}hṽ˩}}}}{}
\textcolor{teal}{\mytextsc{verbe}} \hspace{4pt} Ton~: L.

Ennuyer, empêcher, faire obstruction à.
\textcolor{brown}{\zh{干扰、防碍。}}


\begin{exe}
\sn \textcolor{darkblue}{\textbf{\ipa{hĩ˧ hṽ˥\textasciitilde{}hṽ˩}}}
\trans ennuyer les gens
\trans \textit{\textcolor{brown}{\zh{干扰人家}}}
\end{exe}


\paragraph{\hspace{-0.5cm} {\Large \textcolor{darkblue}{\textbf{\ipa{hwɑ˩kwɤ˧}}}}} \hypertarget{hwA\string_Bkw7\string_M1}{}
\markboth{\textcolor{darkblue}{\textbf{\ipa{hwɑ˩kwɤ˧}}}}{}
\textcolor{teal}{\mytextsc{nom}} \hspace{4pt} Ton~: LM.

Concombre.
\textcolor{brown}{\zh{黄瓜(汉语借词)。}}

 Emprunt~: chinois \textcolor{brown}{\zh{黄瓜}}

 \mytextsc{clf}~: \textcolor{darkblue}{\textbf{\ipa{ɭɯ˧}}} 

\paragraph{\hspace{-0.5cm} {\Large \textcolor{darkblue}{\textbf{\ipa{hwæ˧\textsubscript{a}}}}}} \hypertarget{hw\{\string_Ma1}{}
\markboth{\textcolor{darkblue}{\textbf{\ipa{hwæ˧\textsubscript{a}}}}}{}
\textcolor{teal}{\mytextsc{verbe}} \hspace{4pt} Ton~: M\textsubscript{a}.

Acheter.
\textcolor{brown}{\zh{买。}}


\begin{exe}
\sn \textcolor{darkblue}{\textbf{\ipa{le˧-hwæ˧}}}
\trans \mytextsc{accomp}
\trans \textit{\textcolor{brown}{\zh{\mytextsc{accomp}}}}
\end{exe}
\begin{exe}
\sn \textcolor{darkblue}{\textbf{\ipa{tso˧\textasciitilde{}tso˧ hwæ˩}}}
\trans acheter des choses
\trans \textit{\textcolor{brown}{\zh{买东西}}}
\end{exe}
\begin{exe}
\sn \textcolor{darkblue}{\textbf{\ipa{ɖɯ˧-kʰwɤ˥ hwæ˩}}}
\trans acheter un morceau
\trans \textit{\textcolor{brown}{\zh{买一块}}}
\end{exe}
\begin{exe}
\sn \textcolor{darkblue}{\textbf{\ipa{hwæ˧\textasciitilde{}hwæ˩}}}
\trans \mytextsc{red}
\trans \textit{\textcolor{brown}{\zh{\mytextsc{重叠}}}}
\end{exe}


\paragraph{\hspace{-0.5cm} {\Large \textcolor{darkblue}{\textbf{\ipa{hwæ˧ɖʐæ˥}}} \textsubscript{1}}} \hypertarget{hw\{\string_Md`z`\{\string_T1}{}
\markboth{\textcolor{darkblue}{\textbf{\ipa{hwæ˧ɖʐæ˥}}} \textsubscript{1}}{}
\textcolor{teal}{\mytextsc{nom}} \hspace{4pt} Ton~: H\#.

Écureuil.
\textcolor{brown}{\zh{松鼠,灰鼠。}}


\begin{exe}
\sn \textcolor{darkblue}{\textbf{\ipa{hwæ˧ɖʐæ˥-pʰv̩˩}}}
\trans écureuil mâle
\trans \textit{\textcolor{brown}{\zh{公松鼠}}}
\end{exe}
\begin{exe}
\sn \textcolor{darkblue}{\textbf{\ipa{hwæ˧ɖʐæ˥-mi˩}}}
\trans écureuil femelle
\trans \textit{\textcolor{brown}{\zh{母松鼠}}}
\end{exe}
 \mytextsc{clf}~: \textcolor{darkblue}{\textbf{\ipa{mi˩}}} 

\paragraph{\hspace{-0.5cm} {\Large \textcolor{darkblue}{\textbf{\ipa{hwæ˧ɖʐæ˥}}} \textsubscript{2}}} \hypertarget{hw\{\string_Md`z`\{\string_T2}{}
\markboth{\textcolor{darkblue}{\textbf{\ipa{hwæ˧ɖʐæ˥}}} \textsubscript{2}}{}
\textcolor{teal}{\mytextsc{nom}} \hspace{4pt} Ton~: H\#.

Verrue.
\textcolor{brown}{\zh{瘊子、肉赘。}}


\begin{exe}
\sn \textcolor{darkblue}{\textbf{\ipa{hwæ˧ʈʂæ˥ tʰv̩˩}}}
\trans une verrue se forme; attraper une verrue
\trans \textit{\textcolor{brown}{\zh{长瘊子}}}
\end{exe}
\begin{exe}
\sn \textcolor{darkblue}{\textbf{\ipa{hwæ˧ʈʂæ˥ | le˧-tʰv̩˧-ze˧!}}}
\trans Une verrue s'est formée!
\trans \textit{\textcolor{brown}{\zh{长瘊子了!}}}
\end{exe}
 \mytextsc{clf}~: \textcolor{darkblue}{\textbf{\ipa{mi˩}}} 

\paragraph{\hspace{-0.5cm} {\Large \textcolor{darkblue}{\textbf{\ipa{hwæ˧pʰæ˥}}}}} \hypertarget{hw\{\string_Mp\string_h\{\string_T1}{}
\markboth{\textcolor{darkblue}{\textbf{\ipa{hwæ˧pʰæ˥}}}}{}
\textcolor{teal}{\mytextsc{nom}} \hspace{4pt} Ton~: H\#.

Pièce de tissu.
\textcolor{brown}{\zh{一块布。}}


 \mytextsc{clf}~: \textcolor{darkblue}{\textbf{\ipa{pʰæ˧˥}}} 

\paragraph{\hspace{-0.5cm} {\Large \textcolor{darkblue}{\textbf{\ipa{hwæ˧pʰæ˩}}}}} \hypertarget{hw\{\string_Mp\string_h\{\string_B1}{}
\markboth{\textcolor{darkblue}{\textbf{\ipa{hwæ˧pʰæ˩}}}}{}
\textcolor{teal}{\mytextsc{nom}} \hspace{4pt} Ton~: L\#.

Grosse houe.
\textcolor{brown}{\zh{大锄。}}Dialecte chinois local~: \zh{挖锄。}


\begin{exe}
\sn \textcolor{darkblue}{\textbf{\ipa{hwæ˧pʰæ˩ tʰv̩˩-nɑ˩}}}
\trans \mytextsc{n}+\mytextsc{dem}+\mytextsc{clf}
\trans \textit{\textcolor{brown}{\zh{这把大锄}}}
\end{exe}
 \mytextsc{clf}~: \textcolor{darkblue}{\textbf{\ipa{nɑ˧}}} 

\paragraph{\hspace{-0.5cm} {\Large \textcolor{darkblue}{\textbf{\ipa{hwæ˧pʰæ˩-gv̩˩-di˩}}}}} \hypertarget{hw\{\string_Mp\string_h\{\string_B-gv\string_=\string_B-di\string_B1}{}
\markboth{\textcolor{darkblue}{\textbf{\ipa{hwæ˧pʰæ˩-gv̩˩-di˩}}}}{}
\textcolor{teal}{\mytextsc{nom}} \hspace{4pt} Ton~: L\#--.

Métier à tisser.
\textcolor{brown}{\zh{织布机。}}


\begin{exe}
\sn \textcolor{darkblue}{\textbf{\ipa{hwæ˧pʰæ˩gv̩˩di˩-tɕi˩tɕʰi˧}}}
\trans métier à tisser industriel, machine à faire du tissu (formé de 'métier à tisser' + le mot chinois pour 'machine')
\trans \textit{\textcolor{brown}{\zh{工业织布机。直译:“织布机器”(在摩梭词后面加上汉语的“机器”)}}}
\end{exe}
 \mytextsc{clf}~: \textcolor{darkblue}{\textbf{\ipa{nɑ˧}}} 

\paragraph{\hspace{-0.5cm} {\Large \textcolor{darkblue}{\textbf{\ipa{hwæ˧tsɯ˥}}}}} \hypertarget{hw\{\string_MtsM\string_T1}{}
\markboth{\textcolor{darkblue}{\textbf{\ipa{hwæ˧tsɯ˥}}}}{}
\textcolor{teal}{\mytextsc{nom}} \hspace{4pt} Ton~: H\#.

Rat.
\textcolor{brown}{\zh{老鼠(汉语借词)。}}Dialecte chinois local~: \zh{耗子。}

 Emprunt~: chinois \textcolor{brown}{\zh{耗子}}

\begin{exe}
\sn \textcolor{darkblue}{\textbf{\ipa{hwæ˧tsɯ˥-pʰv̩˩}}}
\trans rat mâle
\trans \textit{\textcolor{brown}{\zh{公老鼠}}}
\end{exe}
\begin{exe}
\sn \textcolor{darkblue}{\textbf{\ipa{hwæ˧tsɯ˥-mi˩}}}
\trans rat femelle
\trans \textit{\textcolor{brown}{\zh{母老鼠}}}
\end{exe}
 \mytextsc{clf}~: \textcolor{darkblue}{\textbf{\ipa{mi˩}}} 

\paragraph{\hspace{-0.5cm} {\Large \textcolor{darkblue}{\textbf{\ipa{hwæ˧tsɯ˥-njɤ˩di˩}}}}} \hypertarget{hw\{\string_MtsM\string_T-nj7\string_Bdi\string_B1}{}
\markboth{\textcolor{darkblue}{\textbf{\ipa{hwæ˧tsɯ˥-njɤ˩di˩}}}}{}
\textcolor{teal}{\mytextsc{nom}} \hspace{4pt} Ton~: H\#-.

Chardon.
\textcolor{brown}{\zh{大蓟。}}


 \mytextsc{clf}~: \textcolor{darkblue}{\textbf{\ipa{dzi˩}}} 

\paragraph{\hspace{-0.5cm} {\Large \textcolor{darkblue}{\textbf{\ipa{}}}}} \hypertarget{hw\{\string_MtsM\string_T-nj7\string_Bdi\string_B-si\string_Bdzi\string_B1}{}
\markboth{\textcolor{darkblue}{\textbf{\ipa{}}}}{}
\textcolor{teal}{\mytextsc{nom}} \hspace{4pt} Ton~: H\#--.

Bardane: \textit{Arctium lappa}, plante dont les graines adhèrent à la laine et la queue des moutons. On en tire un médicament contre le rhume.
\textcolor{brown}{\zh{牛蒡。}}Dialecte chinois local~: \zh{牛蒡子。}




\paragraph{\hspace{-0.5cm} {\Large \textcolor{darkblue}{\textbf{\ipa{hwæ˩\textsubscript{a}}}} \textsubscript{1}}} \hypertarget{hw\{\string_Ba1}{}
\markboth{\textcolor{darkblue}{\textbf{\ipa{hwæ˩\textsubscript{a}}}} \textsubscript{1}}{}
\textcolor{teal}{\mytextsc{verbe}} \hspace{4pt} Ton~: L\textsubscript{a}.

Fermer (la porte).
\textcolor{brown}{\zh{关(出门,就关门)。}}


\begin{exe}
\sn \textcolor{darkblue}{\textbf{\ipa{kʰi˧ | tʰi˧-hwæ˩!}}}
\trans Ferme la porte!
\trans \textit{\textcolor{brown}{\zh{关门吧!}}}
\end{exe}
\begin{exe}
\sn \textcolor{darkblue}{\textbf{\ipa{ʂe˧bæ˧ | le˧-wo˧-hwæ˥}}}
\trans mettre la chaîne à la porte (quand on sort de la maison, on ferme la porte avec une chaîne de fer, et un verrou)
\end{exe}
\begin{exe}
\sn \textcolor{darkblue}{\textbf{\ipa{kʰi˧-bi˥ di˩-hĩ˩ ʂe˩bæ˩}}}
\end{exe}
\textit{Voir~:} \hyperlink{t`\{\string_Ba1}{\textcolor{darkblue}{\textbf{\ipa{ʈæ˩a}}}} 

\paragraph{\hspace{-0.5cm} {\Large \textcolor{darkblue}{\textbf{\ipa{hwæ˩\textsubscript{a}}}} \textsubscript{2}}} \hypertarget{hw\{\string_Ba2}{}
\markboth{\textcolor{darkblue}{\textbf{\ipa{hwæ˩\textsubscript{a}}}} \textsubscript{2}}{}
\textcolor{teal}{\mytextsc{verbe}} \hspace{4pt} Ton~: L\textsubscript{a}.

Accrocher, suspendre; être accroché, suspendu à; se tenir à.
\textcolor{brown}{\zh{悬挂、挂在墙上。}}


\begin{exe}
\sn \textcolor{darkblue}{\textbf{\ipa{tso˧\textasciitilde{}tso˧ | gɤ˧bi˧ hwæ˥}}}
\trans accrocher des choses en hauteur
\trans \textit{\textcolor{brown}{\zh{挂东西在上面}}}
\end{exe}
\begin{exe}
\sn \textcolor{darkblue}{\textbf{\ipa{tso˧\textasciitilde{}tso˧ hwæ˥}}}
\trans accrocher des choses
\trans \textit{\textcolor{brown}{\zh{挂东西}}}
\end{exe}
\begin{exe}
\sn \textcolor{darkblue}{\textbf{\ipa{ʂe˧ | tʰi˧-hwæ˩}}}
\trans accrocher de la viande (au-dessus du foyer, pour la fumer)
\trans \textit{\textcolor{brown}{\zh{挂肉(在火塘上,为了熏肉)}}}
\end{exe}
\begin{exe}
\sn \textcolor{darkblue}{\textbf{\ipa{tso˧\textasciitilde{}tso˧ | tʰi˧-hwæ˩}}}
\trans accrocher des choses
\trans \textit{\textcolor{brown}{\zh{挂东西}}}
\end{exe}
\begin{exe}
\sn \textcolor{darkblue}{\textbf{\ipa{ʂe˧-hwæ˥-di˩}}}
\trans Poutrelle de la pièce principale, juste en-dessous des poutres maîtresses, servant à suspendre de la viande qui sèche. Littéralement ``objet pour accrocher de la viande"
\trans \textit{\textcolor{brown}{\zh{主屋里面的小梁(大梁下面),用来挂肉,熏肉。直译:“挂肉的东西”。}}}
\end{exe}
\begin{exe}
\sn \textcolor{darkblue}{\textbf{\ipa{tso˧\textasciitilde{}tso˧-hwæ˥-di˩}}}
\trans objet servant à suspendre des choses; cette périphrase peut par exemple désigner la poutrelle servant à accrocher de la viande, dans la pièce principale
\trans \textit{\textcolor{brown}{\zh{挂(东西)用的(东西),如:钩子、用来挂肉的小梁……}}}
\end{exe}


\paragraph{\hspace{-0.5cm} {\Large \textcolor{darkblue}{\textbf{\ipa{hwɤ̃˩\textsubscript{a}}}}}} \hypertarget{hw7\string_~\string_Ba1}{}
\markboth{\textcolor{darkblue}{\textbf{\ipa{hwɤ̃˩\textsubscript{a}}}}}{}
\textcolor{teal}{\mytextsc{adjectif}} \hspace{4pt} Ton~: L\textsubscript{a}.

En retard.
\textcolor{brown}{\zh{迟,晚。}}


\begin{exe}
\sn \textcolor{darkblue}{\textbf{\ipa{hwɤ̃˩-hĩ˩˥}}}
\trans \mytextsc{rel}/\mytextsc{nmlz}
\trans \textit{\textcolor{brown}{\zh{迟的}}}
\end{exe}
\begin{exe}
\sn \textcolor{darkblue}{\textbf{\ipa{ʈʂʰɯ˧ ʑi˧-ʈi˥ hwɤ̃˩!}}}
\trans il se lève tard
\trans \textit{\textcolor{brown}{\zh{他起床起得晚!}}}
\end{exe}


\paragraph{\hspace{-0.5cm} {\Large \textcolor{darkblue}{\textbf{\ipa{hwɤ˥}}}}} \hypertarget{hw7\string_T1}{}
\markboth{\textcolor{darkblue}{\textbf{\ipa{hwɤ˥}}}}{}
\textcolor{teal}{\mytextsc{nom}} \hspace{4pt} Ton~: \#H.

Cadeau (souvent financier) qu'on offre à l'occasion des grands événements: décès, mariages.
\textcolor{brown}{\zh{在大事发生的时候送的礼物(近期一般给钱):婚礼、葬礼。}}


\begin{exe}
\sn \textcolor{darkblue}{\textbf{\ipa{hwɤ˧ | ɖɯ˧-kʰwɤ˥}}}
\trans un cadeau, une donation
\trans \textit{\textcolor{brown}{\zh{一份大礼物}}}
\end{exe}
 \mytextsc{clf}~: \textcolor{darkblue}{\textbf{\ipa{kʰwɤ˥}}} 

\paragraph{\hspace{-0.5cm} {\Large \textcolor{darkblue}{\textbf{\ipa{hwɤ˥}}}}} \hypertarget{hw7\string_T1}{}
\markboth{\textcolor{darkblue}{\textbf{\ipa{hwɤ˥}}}}{}
\textcolor{teal}{\mytextsc{verbe}} \hspace{4pt} Ton~: H.

Participer à une cérémonie funèbre; littéralement ``envoyer quelqu'un", c'est-à-dire accompagner quelqu'un vers l'au-delà.
\textcolor{brown}{\zh{执绋送丧。}}




\paragraph{\hspace{-0.5cm} {\Large \textcolor{darkblue}{\textbf{\ipa{hwɤ˧}}}}} \hypertarget{hw7\string_M1}{}
\markboth{\textcolor{darkblue}{\textbf{\ipa{hwɤ˧}}}}{}
\textcolor{teal}{\mytextsc{adjectif}} \hspace{4pt} Ton~: M.

Vaste (plaine), étendu, grand (pièce de tissu), de grande taille (objet, légume...).
\textcolor{brown}{\zh{宽,辽阔,宽敞。}}


\begin{exe}
\sn \textcolor{darkblue}{\textbf{\ipa{qʰɑ˧-hwɤ˧-gv̩˧}}}
\trans très vaste
\trans \textit{\textcolor{brown}{\zh{非常宽敞}}}
\end{exe}


\paragraph{\hspace{-0.5cm} {\Large \textcolor{darkblue}{\textbf{\ipa{hwɤ˧kʰv̩˥}}}}} \hypertarget{hw7\string_Mk\string_hv\string_=\string_T1}{}
\markboth{\textcolor{darkblue}{\textbf{\ipa{hwɤ˧kʰv̩˥}}}}{}
\textcolor{teal}{\mytextsc{nom}} \hspace{4pt} Ton~: H\#.

Année du Chat (correspondant à l'année chinoise du Rat).
\textcolor{brown}{\zh{鼠年(摩梭话称作“猫年”)。}}




\paragraph{\hspace{-0.5cm} {\Large \textcolor{darkblue}{\textbf{\ipa{hwɤ˧li˧-bv̩˥}}}}} \hypertarget{hw7\string_Mli\string_M-bv\string_=\string_T1}{}
\markboth{\textcolor{darkblue}{\textbf{\ipa{hwɤ˧li˧-bv̩˥}}}}{}
\textcolor{teal}{\mytextsc{nom}} \hspace{4pt} Ton~: H\#.

Mezzanine: espace de la pièce principale où un plancher est aménagé sous la charpente, formant comme une mezzanine, mais que les habitants humains n'utilisent pas: l'endroit étrant très enfumé, on n'y place qu'un plancher peu solide et sans rambarde; d'où le nom: ``(la pièce) du chat". On y laisse parfois des objets (vanneries par exemples), qui y sont relativement préservés des insectes par la fumée.
\textcolor{brown}{\zh{夹层:主屋的夹层。因为烟多,所以人不能将这个空间当卧室。只有一层薄的木地板。}}


 \mytextsc{clf}~: \textcolor{darkblue}{\textbf{\ipa{kɤ˧˥}}} \textit{Syn~:} \hyperlink{hw7\string_Mli\string_M-se\string_M-di\string_M\string_T1}{\textcolor{darkblue}{\textbf{\ipa{hwɤ˧li˧-se˧-di˧˥}}}}. 

\paragraph{\hspace{-0.5cm} {\Large \textcolor{darkblue}{\textbf{\ipa{hwɤ˧li˧-hwæ˧qʰæ\#˥}}}}} \hypertarget{hw7\string_Mli\string_M-hw\{\string_Mq\string_h\{\#\string_T1}{}
\markboth{\textcolor{darkblue}{\textbf{\ipa{hwɤ˧li˧-hwæ˧qʰæ\#˥}}}}{}
\textcolor{teal}{\mytextsc{nom}} \hspace{4pt} Ton~: \#H.

Yyyy.
\textcolor{brown}{\zh{山萝卜。}}




\paragraph{\hspace{-0.5cm} {\Large \textcolor{darkblue}{\textbf{\ipa{hwɤ˧li˧-se˧-di˧˥}}}}} \hypertarget{hw7\string_Mli\string_M-se\string_M-di\string_M\string_T1}{}
\markboth{\textcolor{darkblue}{\textbf{\ipa{hwɤ˧li˧-se˧-di˧˥}}}}{}
\textcolor{teal}{\mytextsc{nom}} \hspace{4pt} Ton~: MH\#.

Mezzanine: espace de la pièce principale où un plancher est aménagé sous la charpente, formant comme une mezzanine, mais que les habitants humains n'utilisent pas: l'endroit étrant très enfumé, on n'y place qu'un plancher peu solide et sans rambarde; d'où le nom: ``(la pièce) du chat". On y laisse parfois des objets (vanneries par exemples), qui y sont relativement préservés des insectes par la fumée.
\textcolor{brown}{\zh{夹层:主屋的夹层。因为烟多,所以人不能将这个空间当卧室。只有一层薄的木地板。}}


 \mytextsc{clf}~: \textcolor{darkblue}{\textbf{\ipa{kɤ˧˥}}} \textit{Syn~:} \hyperlink{hw7\string_Mli\string_M-bv\string_=\string_T1}{\textcolor{darkblue}{\textbf{\ipa{hwɤ˧li˧-bv̩˥}}}}. 

\paragraph{\hspace{-0.5cm} {\Large \textcolor{darkblue}{\textbf{\ipa{hwɤ˧li˧-ʂɯ˧mo˥}}}}} \hypertarget{hw7\string_Mli\string_M-s`M\string_Mmo\string_T1}{}
\markboth{\textcolor{darkblue}{\textbf{\ipa{hwɤ˧li˧-ʂɯ˧mo˥}}}}{}
\textcolor{teal}{\mytextsc{nom}} \hspace{4pt} Ton~: H\#.

Vieux chat, vieux matou (de l'un ou l'autre sexe).
\textcolor{brown}{\zh{老猫(不分公、母)。}}


 \mytextsc{clf}~: \textcolor{darkblue}{\textbf{\ipa{mi˩}}} 

\paragraph{\hspace{-0.5cm} {\Large \textcolor{darkblue}{\textbf{\ipa{hwɤ˧li˧-zo˧˥}}}}} \hypertarget{hw7\string_Mli\string_M-zo\string_M\string_T1}{}
\markboth{\textcolor{darkblue}{\textbf{\ipa{hwɤ˧li˧-zo˧˥}}}}{}
\textcolor{teal}{\mytextsc{nom}} \hspace{4pt} Ton~: MH\#.

Chaton.
\textcolor{brown}{\zh{小猫。}}


 \mytextsc{clf}~: \textcolor{darkblue}{\textbf{\ipa{ɭɯ˧}}} 

\paragraph{\hspace{-0.5cm} {\Large \textcolor{darkblue}{\textbf{\ipa{hwɤ˧li˧˥}}}}} \hypertarget{hw7\string_Mli\string_M\string_T1}{}
\markboth{\textcolor{darkblue}{\textbf{\ipa{hwɤ˧li˧˥}}}}{}
\textcolor{teal}{\mytextsc{nom}} \hspace{4pt} Ton~: MH\#.

Chat.
\textcolor{brown}{\zh{猫。}}


 \mytextsc{clf}~: \textcolor{darkblue}{\textbf{\ipa{mi˩}}} 

\paragraph{\hspace{-0.5cm} {\Large \textcolor{darkblue}{\textbf{\ipa{hwɤ˧mi˥\$}}}}} \hypertarget{hw7\string_Mmi\string_T\$1}{}
\markboth{\textcolor{darkblue}{\textbf{\ipa{hwɤ˧mi˥\$}}}}{}
\textcolor{teal}{\mytextsc{nom}} \hspace{4pt} Ton~: H\$.

Chatte.
\textcolor{brown}{\zh{母猫。}}


\begin{exe}
\sn \textcolor{darkblue}{\textbf{\ipa{hwɤ˧mi˧-hwɤ˥pʰv̩˩ / hwɤ˧mi˧-hwɤ˧pʰv̩˥\#}}}
\trans chatte et matou
\trans \textit{\textcolor{brown}{\zh{母猫与公猫}}}
\end{exe}
 \mytextsc{clf}~: \textcolor{darkblue}{\textbf{\ipa{mi˩}}} 

\paragraph{\hspace{-0.5cm} {\Large \textcolor{darkblue}{\textbf{\ipa{hwɤ˧pʰv̩\#˥}}}}} \hypertarget{hw7\string_Mp\string_hv\string_=\#\string_T1}{}
\markboth{\textcolor{darkblue}{\textbf{\ipa{hwɤ˧pʰv̩\#˥}}}}{}
\textcolor{teal}{\mytextsc{nom}} \hspace{4pt} Ton~: \#H.

Matou, chat mâle.
\textcolor{brown}{\zh{公猫。}}


\begin{exe}
\sn \textcolor{darkblue}{\textbf{\ipa{hwɤ˧pʰv̩˧ tʰv̩˧-mi˥\#}}}
\trans \mytextsc{n}+\mytextsc{dem}+\mytextsc{clf}
\trans \textit{\textcolor{brown}{\zh{那个公猫}}}
\end{exe}
\begin{exe}
\sn \textcolor{darkblue}{\textbf{\ipa{hwɤ˧pʰv̩˧-hwɤ˧mi˥}}}
\trans matou et chatte
\trans \textit{\textcolor{brown}{\zh{公猫与母猫}}}
\end{exe}
 \mytextsc{clf}~: \textcolor{darkblue}{\textbf{\ipa{mi˩}}} 

\paragraph{\hspace{-0.5cm} {\Large \textcolor{darkblue}{\textbf{\ipa{hwɤ˧se˧}}}}} \hypertarget{hw7\string_Mse\string_M1}{}
\markboth{\textcolor{darkblue}{\textbf{\ipa{hwɤ˧se˧}}}}{}
\textcolor{teal}{\mytextsc{nom}} \hspace{4pt} Ton~: M.

Cacahuètes.
\textcolor{brown}{\zh{花生。}}

 Emprunt~: chinois \textcolor{brown}{\zh{花生}}

\begin{exe}
\sn \textcolor{darkblue}{\textbf{\ipa{hwɤ˧se˧-qo˧tv̩˩}}}
\trans même sens; littéralement ``graines de cacahuètes"
\trans \textit{\textcolor{brown}{\zh{花生米}}}
\end{exe}


\paragraph{\hspace{-0.5cm} {\Large \textcolor{darkblue}{\textbf{\ipa{hwɤ˧tɕi˥}}}}} \hypertarget{hw7\string_Mts£i\string_T1}{}
\markboth{\textcolor{darkblue}{\textbf{\ipa{hwɤ˧tɕi˥}}}}{}
\textcolor{teal}{\mytextsc{nom}} \hspace{4pt} Ton~: H\#.

Parelle sauvage, oseille crépue, patience crépue, patience sauvage, \textit{Rumex crispus}. Cette plante constitue l'une des trois sortes de fourrage utilisées pour les cochons; elle est aussi consommée par les humains.
\textcolor{brown}{\zh{土大黄(学名:皱叶酸模)(喂猪的牧草)。}}


\begin{exe}
\sn \textcolor{darkblue}{\textbf{\ipa{hwɤ˧tɕi˥-bæ˩bæ˩}}}
\trans même sens
\trans \textit{\textcolor{brown}{\zh{同上}}}
\end{exe}
\begin{exe}
\sn \textcolor{darkblue}{\textbf{\ipa{hwɤ˧tɕʰi˥-ʁo˩bv̩˩}}}
\trans pousses de parelle sauvage
\trans \textit{\textcolor{brown}{\zh{土大黄的嫩芽}}}
\end{exe}
 \mytextsc{clf}~: \textcolor{darkblue}{\textbf{\ipa{po˧}}} 

\paragraph{\hspace{-0.5cm} {\Large \textcolor{darkblue}{\textbf{\ipa{hwɤ˧zo\#˥}}}}} \hypertarget{hw7\string_Mzo\#\string_T1}{}
\markboth{\textcolor{darkblue}{\textbf{\ipa{hwɤ˧zo\#˥}}}}{}
\textcolor{teal}{\mytextsc{nom}} \hspace{4pt} Ton~: \#H.

Chaton.
\textcolor{brown}{\zh{小猫。}}


\begin{exe}
\sn \textcolor{darkblue}{\textbf{\ipa{hwɤ˧zo˧ tʰv̩˧-ɭɯ\#˥}}}
\trans \mytextsc{n}+\mytextsc{dem}+\mytextsc{clf}
\trans \textit{\textcolor{brown}{\zh{那个小猫}}}
\end{exe}
\begin{exe}
\sn \textcolor{darkblue}{\textbf{\ipa{hwɤ˧zo˧-hwɤ˧mi˥}}}
\trans chats (toute la famille: chatons et parents)
\trans \textit{\textcolor{brown}{\zh{猫,包括小猫、母猫和公猫}}}
\end{exe}
 \mytextsc{clf}~: \textcolor{darkblue}{\textbf{\ipa{ɭɯ˧}}} 

\paragraph{\hspace{-0.5cm} {\Large \textcolor{darkblue}{\textbf{\ipa{hwɤ˩}}}}} \hypertarget{hw7\string_B1}{}
\markboth{\textcolor{darkblue}{\textbf{\ipa{hwɤ˩}}}}{}
\textcolor{teal}{\mytextsc{verbe}} \hspace{4pt} Ton~: L\textsubscript{a}.
\ding{202} 
Attacher, emballer, préparer un fardeau/une charge, constituer un ballot (avec de l'herbe, des objets…), mettre en botte, mettre en ballot.
\textcolor{brown}{\zh{捆(捆成捆儿)。}}


\begin{exe}
\sn \textcolor{darkblue}{\textbf{\ipa{zɯ˧-wɤ˧ hwɤ˥}}}
\trans faire un ballot d'herbes
\trans \textit{\textcolor{brown}{\zh{将草捆成一垛、捆一垛草}}}
\end{exe}
\begin{exe}
\sn \textcolor{darkblue}{\textbf{\ipa{si˧-wɤ˧ hwɤ˥}}}
\trans faire un ballot de bois
\trans \textit{\textcolor{brown}{\zh{将木头捆成一堆、捆一堆木头}}}
\end{exe}
\begin{exe}
\sn \textcolor{darkblue}{\textbf{\ipa{hɑ˧-wɤ˧ hwɤ˥}}}
\trans faire un ballot de céréales
\trans \textit{\textcolor{brown}{\zh{将粮食捆成一包、捆一包粮食}}}
\end{exe}
\begin{exe}
\sn \textcolor{darkblue}{\textbf{\ipa{wɤ˩ hwɤ˩˥}}}
\trans préparer un fardeau, mettre en ballot / mettre (des objets, des choses) en paquet, de façon à ce qu'une personne puisse le porter
\trans \textit{\textcolor{brown}{\zh{捆成一包}}}
\end{exe}
\begin{exe}
\sn \textcolor{darkblue}{\textbf{\ipa{wɤ˩˥ | tʰi˧-hwɤ˩}}}
\trans et après, on en fait un ballot / on en fait un paquet / on attache ça ensemble!
\trans \textit{\textcolor{brown}{\zh{然后,捆成一包!}}}
\end{exe}
\begin{exe}
\sn \textcolor{darkblue}{\textbf{\ipa{wɤ˩˥ | ɖɯ˧-wɤ˩ hwɤ˩}}}
\trans faire un ballot de plus
\trans \textit{\textcolor{brown}{\zh{又捆一包}}}
\end{exe}
\ding{203} 
Sens figuré: avoir des embarras, se créer des embarras sur les bras, s'empêtrer.
\textcolor{brown}{\zh{有困难、像把自己捆起来一样。}}


\begin{exe}
\sn \textcolor{darkblue}{\textbf{\ipa{hĩ˧, | wɤ˩ hwɤ˧ ʝi˧-ni˥gv̩˩!}}}
\trans Les gens, ils étaient accablés / ils étaient malheureux! (Image: les gens étaient comme ficelés, incapables de se mouvoir normalement, de vivre leur vie normalement.)
\trans \textit{\textcolor{brown}{\zh{人家难受,像被捆一样}}}
\end{exe}
\begin{exe}
\sn \textcolor{darkblue}{\textbf{\ipa{wɤ˩hwɤ˧ ʝi˧-ni˥gv̩˩-ɲi˩-ze˩!}}}
\trans (je me) suis mis des grosses complications sur les bras!
\trans \textit{\textcolor{brown}{\zh{我给自己找麻烦了!}}}
\end{exe}


\paragraph{\hspace{-0.5cm} {\Large \textcolor{darkblue}{\textbf{\ipa{hwɤ˩\textsubscript{a}}}}}} \hypertarget{hw7\string_Ba1}{}
\markboth{\textcolor{darkblue}{\textbf{\ipa{hwɤ˩\textsubscript{a}}}}}{}
\textcolor{teal}{\mytextsc{verbe}} \hspace{4pt} Ton~: L\textsubscript{a}.

Passer un objet, envoyer un objet (à quelqu'un).
\textcolor{brown}{\zh{递过去。}}


\begin{exe}
\sn \textcolor{darkblue}{\textbf{\ipa{hĩ˧-ki˧ | tso˧\textasciitilde{}tso˧ hwɤ˥}}}
\trans envoyer des choses à quelqu'un
\trans \textit{\textcolor{brown}{\zh{给人家寄东西}}}
\end{exe}


\paragraph{\hspace{-0.5cm} {\Large \textcolor{darkblue}{\textbf{\ipa{hwɤ˩dʑɯ˩}}}}} \hypertarget{hw7\string_Bdz£M\string_B1}{}
\markboth{\textcolor{darkblue}{\textbf{\ipa{hwɤ˩dʑɯ˩}}}}{}
\textcolor{teal}{\mytextsc{nom}} \hspace{4pt} Ton~: L.

Cabane, hutte (qu'on construit sur la montagne quand on doit y passer qq nuits, par exemple pour couper du bois).
\textcolor{brown}{\zh{山上过夜的小木房。}}


 \mytextsc{clf}~: \textcolor{darkblue}{\textbf{\ipa{ɭɯ˧}}} 

\paragraph{\hspace{-0.5cm} {\Large \textcolor{darkblue}{\textbf{\ipa{hwɤ˩kæ˧}}}}} \hypertarget{hw7\string_Bk\{\string_M1}{}
\markboth{\textcolor{darkblue}{\textbf{\ipa{hwɤ˩kæ˧}}}}{}
\textcolor{teal}{\mytextsc{nom}} \hspace{4pt} Ton~: LM.

Bouleau rouge; son bois est bon, on s'en sert pour fabriquer des araires.
\textcolor{brown}{\zh{红桦树。}}


\begin{exe}
\sn \textcolor{darkblue}{\textbf{\ipa{hwɤ˩kæ˧-si˧dzi˩}}}
\trans même sens: bouleau rouge
\trans \textit{\textcolor{brown}{\zh{同上:红桦树}}}
\end{exe}


\paragraph{\hspace{-0.5cm} {\Large \textcolor{darkblue}{\textbf{\ipa{hwɤ˩ʈi˥}}}}} \hypertarget{hw7\string_Bt`i\string_T1}{}
\markboth{\textcolor{darkblue}{\textbf{\ipa{hwɤ˩ʈi˥}}}}{}
\textcolor{teal}{\mytextsc{verbe}} \hspace{4pt} Ton~: LH.

Rouiller.
\textcolor{brown}{\zh{生锈。}}


\begin{exe}
\sn \textcolor{darkblue}{\textbf{\ipa{hwɤ˩ʈi˥-ze˩}}}
\trans \mytextsc{pfv}: ça a rouillé
\trans \textit{\textcolor{brown}{\zh{生锈了}}}
\end{exe}


\paragraph{\hspace{-0.5cm} {\Large \textcolor{darkblue}{\textbf{\ipa{hwɤ˧˥}}} \textsubscript{1}}} \hypertarget{hw7\string_M\string_T1}{}
\markboth{\textcolor{darkblue}{\textbf{\ipa{hwɤ˧˥}}} \textsubscript{1}}{}
\textcolor{teal}{\mytextsc{verbe}} \hspace{4pt} Ton~: MH.

Repriser, raccommoder (vêtement).
\textcolor{brown}{\zh{补。}}


\begin{exe}
\sn \textcolor{darkblue}{\textbf{\ipa{le˧-hwɤ˧˥}}}
\trans \mytextsc{accomp}
\trans \textit{\textcolor{brown}{\zh{\mytextsc{accomp}}}}
\end{exe}
\begin{exe}
\sn \textcolor{darkblue}{\textbf{\ipa{bɑ˩lɑ˩ hwɤ˥}}}
\trans réparer un vêtement, recoudre un vêtement, rapetasser un vêtement
\trans \textit{\textcolor{brown}{\zh{补衣服}}}
\end{exe}


\paragraph{\hspace{-0.5cm} {\Large \textcolor{darkblue}{\textbf{\ipa{hwɤ˧˥}}} \textsubscript{2}}} \hypertarget{hw7\string_M\string_T2}{}
\markboth{\textcolor{darkblue}{\textbf{\ipa{hwɤ˧˥}}} \textsubscript{2}}{}
\textcolor{teal}{\mytextsc{nom}} \hspace{4pt} Ton~: MH.

Chat (monosyllabique).
\textcolor{brown}{\zh{猫(单音节)。}}


 \mytextsc{clf}~: \textcolor{darkblue}{\textbf{\ipa{mi˩}}} 

\paragraph{\hspace{-0.5cm} {\Large \textcolor{darkblue}{\textbf{\ipa{hwɤ˧˥}}} \textsubscript{3}}} \hypertarget{hw7\string_M\string_T3}{}
\markboth{\textcolor{darkblue}{\textbf{\ipa{hwɤ˧˥}}} \textsubscript{3}}{}
\textcolor{teal}{\mytextsc{nom}} \hspace{4pt} Ton~: MH.

Rouille (monosyllabe).
\textcolor{brown}{\zh{锈(单音节)。}}




\newpage
\section*{\centering- \textcolor{darkblue}{\textbf{\ipa{ĩ}}} -}
\paragraph{\hspace{-0.5cm} {\Large \textcolor{darkblue}{\textbf{\ipa{ĩ˧}}}}} \hypertarget{i\string_~\string_M1}{}
\markboth{\textcolor{darkblue}{\textbf{\ipa{ĩ˧}}}}{}
\textcolor{teal}{\mytextsc{interjection}} \hspace{4pt} Ton~: M.

Oui, d'accord.
\textcolor{brown}{\zh{是的,好的。}}


\begin{exe}
\sn \textcolor{darkblue}{\textbf{\ipa{ʈʂʰɯ˧-ɳɯ˧ | ``ĩ˧! ĩ˧!" | pi˧. |}}}
\trans il a dit: ``Oui, oui!"
\trans \textit{\textcolor{brown}{\zh{他说:“是的,是的!”}}}
\end{exe}


\newpage
\section*{\centering- \textcolor{darkblue}{\textbf{\ipa{j}}} \textcolor{darkblue}{\textbf{\ipa{jæ}}} \textcolor{darkblue}{\textbf{\ipa{jɤ}}} \textcolor{darkblue}{\textbf{\ipa{jo}}} \textcolor{darkblue}{\textbf{\ipa{je}}} -}
\paragraph{\hspace{-0.5cm} {\Large \textcolor{darkblue}{\textbf{\ipa{jɤ˧}}} \textsubscript{1}}} \hypertarget{j7\string_M1}{}
\markboth{\textcolor{darkblue}{\textbf{\ipa{jɤ˧}}} \textsubscript{1}}{}
\textcolor{teal}{\mytextsc{adjectif}} \hspace{4pt} Ton~: M.

Bien (ne s'utilise qu'en tournure négative).
\textcolor{brown}{\zh{好(只出现在否定词后面)。}}


\begin{exe}
\sn \textcolor{darkblue}{\textbf{\ipa{mɤ˧-jɤ˧}}}
\trans \mytextsc{neg}: ce n'est pas bien! / c'est pas beau, ça! (Au sujet du comportement de quelqu'un)
\trans \textit{\textcolor{brown}{\zh{不好(形容一个人的行为)}}}
\end{exe}


\paragraph{\hspace{-0.5cm} {\Large \textcolor{darkblue}{\textbf{\ipa{jɤ˧}}} \textsubscript{2}}} \hypertarget{j7\string_M2}{}
\markboth{\textcolor{darkblue}{\textbf{\ipa{jɤ˧}}} \textsubscript{2}}{}
\textcolor{teal}{\mytextsc{adjectif}} \hspace{4pt} Ton~: M.

Plat.
\textcolor{brown}{\zh{平(土地)。}}


\begin{exe}
\sn \textcolor{darkblue}{\textbf{\ipa{mɤ˧-jɤ˧}}}
\trans \mytextsc{neg}: pas plat; inégal
\trans \textit{\textcolor{brown}{\zh{不平}}}
\end{exe}


\paragraph{\hspace{-0.5cm} {\Large \textcolor{darkblue}{\textbf{\ipa{jɤ˧}}} \textsubscript{3}}} \hypertarget{j7\string_M3}{}
\markboth{\textcolor{darkblue}{\textbf{\ipa{jɤ˧}}} \textsubscript{3}}{}
\textcolor{teal}{\mytextsc{nom}} \hspace{4pt} Ton~: M.

Tabac, cigarette.
\textcolor{brown}{\zh{烟。}}

 Emprunt~: chinois \textcolor{brown}{\zh{烟?}}

\begin{exe}
\sn \textcolor{darkblue}{\textbf{\ipa{jɤ˧ ʈʰɯ˩}}}
\trans fumer
\trans \textit{\textcolor{brown}{\zh{抽烟}}}
\end{exe}
 \mytextsc{clf}~: \textcolor{darkblue}{\textbf{\ipa{ko˧}}} 

\paragraph{\hspace{-0.5cm} {\Large \textcolor{darkblue}{\textbf{\ipa{jɤ˧gɯ˩}}}}} \hypertarget{j7\string_MgM\string_B1}{}
\markboth{\textcolor{darkblue}{\textbf{\ipa{jɤ˧gɯ˩}}}}{}
\textcolor{teal}{\mytextsc{nom}} \hspace{4pt} Ton~: L\#.

Sarrasin, blé noir, \textit{Fagopyrum esculentum}.
\textcolor{brown}{\zh{甜荞/荞麦/花荞。}}


 \mytextsc{clf}~: \textcolor{darkblue}{\textbf{\ipa{kɤ˧˥}}} \textit{Voir~:} \hyperlink{j7\string_Mq\string_hA\#\string_T1}{\textcolor{darkblue}{\textbf{\ipa{jɤ˧qʰɑ\#˥}}}} 

\paragraph{\hspace{-0.5cm} {\Large \textcolor{darkblue}{\textbf{\ipa{jɤ˧ŋɤ˧}}}}} \hypertarget{j7\string_MN7\string_M1}{}
\markboth{\textcolor{darkblue}{\textbf{\ipa{jɤ˧ŋɤ˧}}}}{}
\textcolor{teal}{\mytextsc{nom}} \hspace{4pt} Ton~: M.

La ville de Chengdu, dans le Sichuan.
\textcolor{brown}{\zh{成都。}}


\begin{exe}
\sn \textcolor{darkblue}{\textbf{\ipa{ho˧di˧-jɤ˧ŋɤ˧}}}
\trans même sens
\trans \textit{\textcolor{brown}{\zh{同上}}}
\end{exe}


\paragraph{\hspace{-0.5cm} {\Large \textcolor{darkblue}{\textbf{\ipa{jɤ˧qʰɑ\#˥}}}}} \hypertarget{j7\string_Mq\string_hA\#\string_T1}{}
\markboth{\textcolor{darkblue}{\textbf{\ipa{jɤ˧qʰɑ\#˥}}}}{}
\textcolor{teal}{\mytextsc{nom}} \hspace{4pt} Ton~: \#H.

Sarrasin amer, \textit{Fagopyrum tataricum Gaertn}.
\textcolor{brown}{\zh{苦荞。}}


 \mytextsc{clf}~: \textcolor{darkblue}{\textbf{\ipa{kɤ˧˥}}} \textit{Voir~:} \hyperlink{j7\string_MgM\string_B1}{\textcolor{darkblue}{\textbf{\ipa{jɤ˧gɯ˩}}}} 

\paragraph{\hspace{-0.5cm} {\Large \textcolor{darkblue}{\textbf{\ipa{jɤ˧qʰɑ˧-pɤ˥jɤ˩-mo˩}}}}} \hypertarget{j7\string_Mq\string_hA\string_M-p7\string_Tj7\string_B-mo\string_B1}{}
\markboth{\textcolor{darkblue}{\textbf{\ipa{jɤ˧qʰɑ˧-pɤ˥jɤ˩-mo˩}}}}{}
\textcolor{teal}{\mytextsc{nom}} \hspace{4pt} Ton~: \#H\mytextsc{}.

Bolet, cèpe, \textit{Boletus edulis}; littéralement ``champignon-galette de sarrasin", du fait de sa texture.
\textcolor{brown}{\zh{牛肝菌。}}


\textit{Voir~:} \hyperlink{njo\string_Bk\{\string_Mts£i\string_B\string_T1}{\textcolor{darkblue}{\textbf{\ipa{njo˩kæ˧tɕi˩˥}}}} 

\paragraph{\hspace{-0.5cm} {\Large \textcolor{darkblue}{\textbf{\ipa{jɤ˧wo˧˥}}}}} \hypertarget{j7\string_Mwo\string_M\string_T1}{}
\markboth{\textcolor{darkblue}{\textbf{\ipa{jɤ˧wo˧˥}}}}{}
\textcolor{teal}{\mytextsc{verbe}} \hspace{4pt} Ton~: MH\#.

Régresser (le contraire de: faire des progrès).
\textcolor{brown}{\zh{倒退、退步。}}


\begin{exe}
\sn \textcolor{darkblue}{\textbf{\ipa{no˧ | jɤ˧wo˧˥ | sɯ˧ɖʐæ˧! / no˧ | le˧-wo˥ | sɯ˧ɖʐæ˧!}}}
\trans tu régresses! (adressé à un petit enfant qui a fait caca dans sa culotte, alors que depuis plusieurs semaines il avait pris l'habitude du pot)
\trans \textit{\textcolor{brown}{\zh{你这是在退步!(情景:一个小孩已经几个礼拜有了上厕所的习惯,那天又把屎拉在裤头里)}}}
\end{exe}


\paragraph{\hspace{-0.5cm} {\Large \textcolor{darkblue}{\textbf{\ipa{jɤ˩\textsubscript{a}}}} \textsubscript{1}}} \hypertarget{j7\string_Ba1}{}
\markboth{\textcolor{darkblue}{\textbf{\ipa{jɤ˩\textsubscript{a}}}} \textsubscript{1}}{}
\textcolor{teal}{\mytextsc{verbe}} \hspace{4pt} Ton~: L\textsubscript{a}.

Enrouler (ex.: des fibres de lin).
\textcolor{brown}{\zh{盘、盘绕(线)。}}


\begin{exe}
\sn \textcolor{darkblue}{\textbf{\ipa{sɑ˧ jɤ˥}}}
\trans enrouler du fil de lin
\trans \textit{\textcolor{brown}{\zh{盘麻线}}}
\end{exe}
\begin{exe}
\sn \textcolor{darkblue}{\textbf{\ipa{sɑ˧ | le˧-jɤ˩}}}
\trans enrouler du fil de lin
\trans \textit{\textcolor{brown}{\zh{盘麻线}}}
\end{exe}
\textit{Voir~:} \hyperlink{ts£M\string_Ml\string_RM\string_M1}{\textcolor{darkblue}{\textbf{\ipa{tɕɯ˧ɭɯ˧}}}} 

\paragraph{\hspace{-0.5cm} {\Large \textcolor{darkblue}{\textbf{\ipa{jɤ˩\textsubscript{a}}}} \textsubscript{2}}} \hypertarget{j7\string_Ba2}{}
\markboth{\textcolor{darkblue}{\textbf{\ipa{jɤ˩\textsubscript{a}}}} \textsubscript{2}}{}
\textcolor{teal}{\mytextsc{adjectif}} \hspace{4pt} Ton~: L\textsubscript{a}.

Trop cuit, dissous, tout décomposé: ex.: pommes de terre qui éclatent, pois qui se décomposent en purée.
\textcolor{brown}{\zh{(煮)烂。}}


\begin{exe}
\sn \textcolor{darkblue}{\textbf{\ipa{le˧-tɕɤ˧˥ | le˧-jɤ˩-ze˩!}}}
\trans ça s'est décomposé à force de bouillir!
\trans \textit{\textcolor{brown}{\zh{煮烂了!}}}
\end{exe}
\begin{exe}
\sn \textcolor{darkblue}{\textbf{\ipa{jɤ˩-hĩ˩˥}}}
\trans \mytextsc{rel}/\mytextsc{nmlz}
\trans \textit{\textcolor{brown}{\zh{烂的}}}
\end{exe}


\paragraph{\hspace{-0.5cm} {\Large \textcolor{darkblue}{\textbf{\ipa{jɤ˩\textsubscript{b}}}} \textsubscript{1}}} \hypertarget{j7\string_Bb1}{}
\markboth{\textcolor{darkblue}{\textbf{\ipa{jɤ˩\textsubscript{b}}}} \textsubscript{1}}{}
\textcolor{teal}{\mytextsc{verbe}} \hspace{4pt} Ton~: L\textsubscript{b}.

Être fatigué, être sans entrain, être à la masse.
\textcolor{brown}{\zh{没精神。}}


\begin{exe}
\sn \textcolor{darkblue}{\textbf{\ipa{tʰi˧-jɤ˩-ho˩-ze˩!}}}
\trans (Il/elle) est à la masse!
\trans \textit{\textcolor{brown}{\zh{他没精神了!}}}
\end{exe}
\begin{exe}
\sn \textcolor{darkblue}{\textbf{\ipa{ɑ˩ʁo˧ ʂv̩˧ɖv̩˧ | tʰi˧-jɤ˩-ho˩-tsɯ˩!}}}
\trans Quand on a la nostalgie, on est sans entrain!
\trans \textit{\textcolor{brown}{\zh{想家的时候,没精神!}}}
\end{exe}
\begin{exe}
\sn \textcolor{darkblue}{\textbf{\ipa{ɑ˩ʁo˧ ʂv̩˧ɖv̩˧-zo˥, | tʰi˧-jɤ˩-ho˩!}}}
\trans Quand on a la nostalgie, on est sans entrain!
\trans \textit{\textcolor{brown}{\zh{想家的时候,没精神!}}}
\end{exe}
\begin{exe}
\sn \textcolor{darkblue}{\textbf{\ipa{ɲi˧mi˧ tsʰi˧-zo˩, | tʰi˧-jɤ˩-ho˩!}}}
\trans Quand il fait très chaud, on est sans entrain!
\trans \textit{\textcolor{brown}{\zh{天气很热,没精神!}}}
\end{exe}
\begin{exe}
\sn \textcolor{darkblue}{\textbf{\ipa{jɤ˩-mɤ˥-jɤ˩}}}
\trans \string_ \mytextsc{neg} \string_
\trans \textit{\textcolor{brown}{\zh{\string_ \mytextsc{neg} \string_}}}
\end{exe}


\paragraph{\hspace{-0.5cm} {\Large \textcolor{darkblue}{\textbf{\ipa{jɤ˩\textsubscript{b}}}} \textsubscript{2}}} \hypertarget{j7\string_Bb2}{}
\markboth{\textcolor{darkblue}{\textbf{\ipa{jɤ˩\textsubscript{b}}}} \textsubscript{2}}{}
\textcolor{teal}{\mytextsc{classificateur}} \hspace{4pt} Ton~: L\textsubscript{b}.

Classificateur des rangées de légumes (dans un potager, dans un champ).
\textcolor{brown}{\zh{量词:排(一排菜)。}}


\begin{exe}
\sn \textcolor{darkblue}{\textbf{\ipa{v˩tsʰɤ˧˥ | ɖɯ˧-jɤ˩ tʰi˩-pʰo˩}}}
\trans planter une rangée de légumes
\trans \textit{\textcolor{brown}{\zh{种一排菜}}}
\end{exe}


\paragraph{\hspace{-0.5cm} {\Large \textcolor{darkblue}{\textbf{\ipa{jɤ˩ɕjo˧-dzɑ˧qʰwɤ˩}}}}} \hypertarget{j7\string_Bs£jo\string_M-dzA\string_Mq\string_hw7\string_B1}{}
\markboth{\textcolor{darkblue}{\textbf{\ipa{jɤ˩ɕjo˧-dzɑ˧qʰwɤ˩}}}}{}
\textcolor{teal}{\mytextsc{nom}} \hspace{4pt} Ton~: LM-L\#.

Sandale.
\textcolor{brown}{\zh{凉鞋。}}


 \mytextsc{clf}~: \textcolor{darkblue}{\textbf{\ipa{dzi˧}}} paire

\paragraph{\hspace{-0.5cm} {\Large \textcolor{darkblue}{\textbf{\ipa{jɤ˩ho˧}}}}} \hypertarget{j7\string_Bho\string_M1}{}
\markboth{\textcolor{darkblue}{\textbf{\ipa{jɤ˩ho˧}}}}{}
\textcolor{teal}{\mytextsc{nom}} \hspace{4pt} Ton~: LM.

Allumette.
\textcolor{brown}{\zh{火柴(洋火)。}}

 Emprunt~: chinois \textcolor{brown}{\zh{洋火}}

 \mytextsc{clf}~: \textcolor{darkblue}{\textbf{\ipa{po˩}}} 

\paragraph{\hspace{-0.5cm} {\Large \textcolor{darkblue}{\textbf{\ipa{jɤ˩jo\#˥}}}}} \hypertarget{j7\string_Bjo\#\string_T1}{}
\markboth{\textcolor{darkblue}{\textbf{\ipa{jɤ˩jo\#˥}}}}{}
\textcolor{teal}{\mytextsc{nom}} \hspace{4pt} Ton~: LM+\#H.

Pomme de terre.
\textcolor{brown}{\zh{洋芋、土豆 、马铃薯(汉语借词)。}}

 Emprunt~: chinois \textcolor{brown}{\zh{洋芋}}

 \mytextsc{clf}~: \textcolor{darkblue}{\textbf{\ipa{kɤ˧˥}}} 

\paragraph{\hspace{-0.5cm} {\Large \textcolor{darkblue}{\textbf{\ipa{jɤ˩jo˧-bv̩\#˥}}}}} \hypertarget{j7\string_Bjo\string_M-bv\string_=\#\string_T1}{}
\markboth{\textcolor{darkblue}{\textbf{\ipa{jɤ˩jo˧-bv̩\#˥}}}}{}
\textcolor{teal}{\mytextsc{nom}} \hspace{4pt} Ton~: LM+\#H.
\textit{De:} \textbf{jɤ˩jo\#˥ et bv̩˥} 
Larve de taupin, ver fil de fer, \textit{Agriotes lineatus}: ver qui mange les tubercules.
\textcolor{brown}{\zh{蛴螬。}}


 \mytextsc{clf}~: \textcolor{darkblue}{\textbf{\ipa{mi˩}}} 

\paragraph{\hspace{-0.5cm} {\Large \textcolor{darkblue}{\textbf{\ipa{jɤ˩pæ˧sɯ˥\$}}}}} \hypertarget{j7\string_Bp\{\string_MsM\string_T\$1}{}
\markboth{\textcolor{darkblue}{\textbf{\ipa{jɤ˩pæ˧sɯ˥\$}}}}{}
\textcolor{teal}{\mytextsc{nom}} \hspace{4pt} Ton~: LM+H\$.

Petit Chef Yang: nom de famille constitué de l'expression chinoise \textcolor{brown}{\zh{杨把事}}, formé du patronyme \textcolor{brown}{\zh{杨}}, suivi du terme chinois renvoyant au plus bas degré de la hiérarchie féodale: le chef de hameau, \textcolor{brown}{\zh{把事}}. Ce nom est propre à une seule famille de Yongning.
\textcolor{brown}{\zh{杨把事。这个姓,由两部分组成的:‘杨’姓(汉语借词)与封建社会最小领导层次:‘把事’。}}


\begin{exe}
\sn \textcolor{darkblue}{\textbf{\ipa{jɤ˩pæ˧sɯ˧=ɻ̍˥\$}}}
\trans \string_ \mytextsc{associatif}: les gens de la famille Petit Chef Yang
\trans \textit{\textcolor{brown}{\zh{杨把事家族}}}
\end{exe}
\begin{exe}
\sn \textcolor{darkblue}{\textbf{\ipa{jɤ˩pɑ˧sɯ˥ | ʈæ˧ʂɯ˧}}}
\trans nom propre d'une personne de la famille Petit Chef Yang: 'Dashi de la famille Petit Chef Yang'.
\trans \textit{\textcolor{brown}{\zh{杨把事家的一个人的名字:杨把事•达石}}}
\end{exe}
\textit{Voir~:} \hyperlink{p\{\string_MsM\string_M1}{\textcolor{darkblue}{\textbf{\ipa{pæ˧sɯ˧}}}} 

\paragraph{\hspace{-0.5cm} {\Large \textcolor{darkblue}{\textbf{\ipa{jɤ˩po˧}}}}} \hypertarget{j7\string_Bpo\string_M1}{}
\markboth{\textcolor{darkblue}{\textbf{\ipa{jɤ˩po˧}}}}{}
\textcolor{teal}{\mytextsc{verbe}} \hspace{4pt} Ton~: LM.

Parier.Parier.
\textcolor{brown}{\zh{赌博、打赌。}}




\paragraph{\hspace{-0.5cm} {\Large \textcolor{darkblue}{\textbf{\ipa{jɤ˩tʰi˧-ʁæ˩bæ˩}}}}} \hypertarget{j7\string_Bt\string_hi\string_M-R\{\string_Bb\{\string_B1}{}
\markboth{\textcolor{darkblue}{\textbf{\ipa{jɤ˩tʰi˧-ʁæ˩bæ˩}}}}{}
\textcolor{teal}{\mytextsc{nom}} \hspace{4pt} Ton~: LM-L.

Assiette en faïence/porcelaine.
\textcolor{brown}{\zh{瓷盘。}}


 \mytextsc{clf}~: \textcolor{darkblue}{\textbf{\ipa{ɭɯ˧}}} 

\paragraph{\hspace{-0.5cm} {\Large \textcolor{darkblue}{\textbf{\ipa{jɤ˧˥}}} \textsubscript{1}}} \hypertarget{j7\string_M\string_T1}{}
\markboth{\textcolor{darkblue}{\textbf{\ipa{jɤ˧˥}}} \textsubscript{1}}{}
\textcolor{teal}{\mytextsc{verbe}} \hspace{4pt} Ton~: MH.

Lécher.
\textcolor{brown}{\zh{舔。}}


\begin{exe}
\sn \textcolor{darkblue}{\textbf{\ipa{tso˧\textasciitilde{}tso˧ jɤ˩}}}
\trans lécher quelque chose
\trans \textit{\textcolor{brown}{\zh{舔东西}}}
\end{exe}
\begin{exe}
\sn \textcolor{darkblue}{\textbf{\ipa{dzɯ˧-di˧ jɤ˥}}}
\trans lécher de la nourriture
\trans \textit{\textcolor{brown}{\zh{舔食品}}}
\end{exe}
\begin{exe}
\sn \textcolor{darkblue}{\textbf{\ipa{[F5] tso˧tso˧ ɖɯ˧-kʰwɤ˥ jɤ˩-ze˩}}}
\trans (elle/il) a léché quelque chose
\trans \textit{\textcolor{brown}{\zh{他舔了一个东西。}}}
\end{exe}


\paragraph{\hspace{-0.5cm} {\Large \textcolor{darkblue}{\textbf{\ipa{jɤ˧˥}}} \textsubscript{2}}} \hypertarget{j7\string_M\string_T2}{}
\markboth{\textcolor{darkblue}{\textbf{\ipa{jɤ˧˥}}} \textsubscript{2}}{}
\textcolor{teal}{\mytextsc{nom}} \hspace{4pt} Ton~: MH.

Radis sauvage qui pousse en montagne; on le consomme surtout au printemps, à une époque où il n'y a pas encore de légumes. Ce radis est récoltée par les Yi et vendu dans la plaine.
\textcolor{brown}{\zh{红萝卜菜:一种山上的野菜。春天的时候,菜园的蔬菜还没有成熟的时候,永宁的人吃红萝卜菜。彝族在高山上采下来,在永宁卖。}}Dialecte chinois local~: \zh{野山菜,\textcolor{darkblue}{\textbf{\ipa{/ʝi˧ʂæ˧tsʰɤ˩/}}}。}


\begin{exe}
\sn \textcolor{darkblue}{\textbf{\ipa{jɤ˧ dzɯ˧ | qʰɑ˧-sɯ˥\textasciitilde{}sɯ˩, | jɤ˧ ʈʂɤ˥ ŋv̩˩-ɭɯ˩\textasciitilde{}ɭɯ˩!}}}
\trans ``Le radis sauvage, ça a un goût amer quand on le mange, et ça vous fait pleurer pour le récolter!" (de: qʰɑ˧ 'amer'+expressif) / ``récolter le radis sauvage, ça fait pleurer!" Non pas à cause de la plante elle-même, pas comme un oignon qui piquerait les yeux: mais parce qu'on s'épuisait à aller le chercher en haute montagne.
\trans \textit{\textcolor{brown}{\zh{“红萝卜菜,味道苦,去摘也要流眼泪! / 红萝卜菜,吃起来苦,摘起来也苦!”摘红萝卜菜,需要爬高山,寻找时间长,永宁坝子的农民觉得这比较苦。}}}
\end{exe}
\textit{Voir~:} \hyperlink{j££i\string_Ms`\{\string_Mts\string_h7\string_B1}{\textcolor{darkblue}{\textbf{\ipa{ʝi˧ʂæ˧tsʰɤ˩}}}} 

\paragraph{\hspace{-0.5cm} {\Large \textcolor{darkblue}{\textbf{\ipa{jɤ˧˥}}} \textsubscript{3}}} \hypertarget{j7\string_M\string_T3}{}
\markboth{\textcolor{darkblue}{\textbf{\ipa{jɤ˧˥}}} \textsubscript{3}}{}
\textcolor{teal}{\mytextsc{verbe}} \hspace{4pt} Ton~: MH.

Étendre, appliquer, mettre (ex.: appliquer un onguent).
\textcolor{brown}{\zh{抹、涂抹。}}


\begin{exe}
\sn \textcolor{darkblue}{\textbf{\ipa{pʰv˧ʂɯ˧ jɤ˧˥}}}
\trans appliquer une crème de beauté ou de la crème solaire
\trans \textit{\textcolor{brown}{\zh{抹防晒霜}}}
\end{exe}
\begin{exe}
\sn \textcolor{darkblue}{\textbf{\ipa{mɤ˩ jɤ˩˥}}}
\trans appliquer de la graisse (ex.: sur une peau sèche)
\trans \textit{\textcolor{brown}{\zh{涂抹油}}}
\end{exe}
\begin{exe}
\sn \textcolor{darkblue}{\textbf{\ipa{tʰi˧-jɤ˧˥}}}
\trans \mytextsc{dur} \string_
\end{exe}


\paragraph{\hspace{-0.5cm} {\Large \textcolor{darkblue}{\textbf{\ipa{jɤ˧˥\textsubscript{a}}}} \textsubscript{1}}} \hypertarget{j7\string_M\string_Ta1}{}
\markboth{\textcolor{darkblue}{\textbf{\ipa{jɤ˧˥\textsubscript{a}}}} \textsubscript{1}}{}
\textcolor{teal}{\mytextsc{classificateur}} \hspace{4pt} Ton~: MH\textsubscript{a}.

Classificateur des créatures femelles; employé pour les personnes de sexe féminin (appellation qui ne marque pas de respect, mais n'est pas injurieuse), et pour certains animaux domestiques.
\textcolor{brown}{\zh{量词:母性、雌性(人或动物)(一个/一只)。}}




\paragraph{\hspace{-0.5cm} {\Large \textcolor{darkblue}{\textbf{\ipa{jɤ˧˥\textsubscript{a}}}} \textsubscript{2}}} \hypertarget{j7\string_M\string_Ta2}{}
\markboth{\textcolor{darkblue}{\textbf{\ipa{jɤ˧˥\textsubscript{a}}}} \textsubscript{2}}{}
\textcolor{teal}{\mytextsc{classificateur}} \hspace{4pt} Ton~: MH\textsubscript{a}.

Classificateur pour la pâte à pain: quantité de pâte aux œufs que l'on peut préparer avec un œuf. Ce classificateur est également utilisé pour le thé tassé en galettes.
\textcolor{brown}{\zh{量词:面(一团),茶饼(一个)等。(一团面,是和了一个鸡蛋的面团的量。)。}}


\begin{exe}
\sn \textcolor{darkblue}{\textbf{\ipa{æ˩ʁv̩˩-pɤ˥jɤ˩ | ɖɯ˧-jɤ˧˥}}}
\trans une boule de pâte à pain à l'oeuf
\trans \textit{\textcolor{brown}{\zh{一个鸡蛋面团}}}
\end{exe}
\begin{exe}
\sn \textcolor{darkblue}{\textbf{\ipa{ʝi˧-jɤ˧˥}}}
\trans une boule/galette
\trans \textit{\textcolor{brown}{\zh{一个团/并}}}
\end{exe}


\paragraph{\hspace{-0.5cm} {\Large \textcolor{darkblue}{\textbf{\ipa{jo˥}}}}} \hypertarget{jo\string_T1}{}
\markboth{\textcolor{darkblue}{\textbf{\ipa{jo˥}}}}{}
\textcolor{teal}{\mytextsc{nom}} \hspace{4pt} Ton~: \#H.

Jade (matière, pierre).
\textcolor{brown}{\zh{玉石。}}

 Emprunt~: chinois \textcolor{brown}{\zh{玉}}

 \mytextsc{clf}~: \textcolor{darkblue}{\textbf{\ipa{pʰo˧˥}}} 

\paragraph{\hspace{-0.5cm} {\Large \textcolor{darkblue}{\textbf{\ipa{jo˧}}}}} \hypertarget{jo\string_M1}{}
\markboth{\textcolor{darkblue}{\textbf{\ipa{jo˧}}}}{}
\textcolor{teal}{\mytextsc{verbe}} \hspace{4pt} Ton~: M intrans.

Venir, entrer.
\textcolor{brown}{\zh{来。}}


\begin{exe}
\sn \textcolor{darkblue}{\textbf{\ipa{le˧-jo˧-ze˧!}}}
\trans \mytextsc{accomp} \string_ \mytextsc{pfv}: (il/est) est arrivé(e) / est entré(e)!
\trans \textit{\textcolor{brown}{\zh{来了!}}}
\end{exe}


\paragraph{\hspace{-0.5cm} {\Large \textcolor{darkblue}{\textbf{\ipa{jo˧gv̩˧}}}}} \hypertarget{jo\string_Mgv\string_=\string_M1}{}
\markboth{\textcolor{darkblue}{\textbf{\ipa{jo˧gv̩˧}}}}{}
\textcolor{teal}{\mytextsc{nom}} \hspace{4pt} Ton~: M.

Lijiang (toute la région: la ville, et la plaine environnante).
\textcolor{brown}{\zh{丽江(包括丽江坝子)。}}




\paragraph{\hspace{-0.5cm} {\Large \textcolor{darkblue}{\textbf{\ipa{jo˧gv̩˧-ŋv̩˧lv̩˧}}}}} \hypertarget{jo\string_Mgv\string_=\string_M-Nv\string_=\string_Mlv\string_=\string_M1}{}
\markboth{\textcolor{darkblue}{\textbf{\ipa{jo˧gv̩˧-ŋv̩˧lv̩˧}}}}{}
\textcolor{teal}{\mytextsc{nom}} \hspace{4pt} Ton~: M.

La montagne Yulong: principale montagne de Lijiang.
\textcolor{brown}{\zh{玉龙雪山。}}




\paragraph{\hspace{-0.5cm} {\Large \textcolor{darkblue}{\textbf{\ipa{jo˧mi˧}}}}} \hypertarget{jo\string_Mmi\string_M1}{}
\markboth{\textcolor{darkblue}{\textbf{\ipa{jo˧mi˧}}}}{}
\textcolor{teal}{\mytextsc{nom}} \hspace{4pt} Ton~: M.

Brebis.
\textcolor{brown}{\zh{母绵羊。}}


\begin{exe}
\sn \textcolor{darkblue}{\textbf{\ipa{jo˧mi˧-po˧lo˧}}}
\trans brebis et bélier
\trans \textit{\textcolor{brown}{\zh{母绵羊与公羊}}}
\end{exe}
 \mytextsc{clf}~: \textcolor{darkblue}{\textbf{\ipa{pʰo˧˥}}} 

\paragraph{\hspace{-0.5cm} {\Large \textcolor{darkblue}{\textbf{\ipa{jo˧mi˧-ʁwɤ˧}}}}} \hypertarget{jo\string_Mmi\string_M-Rw7\string_M1}{}
\markboth{\textcolor{darkblue}{\textbf{\ipa{jo˧mi˧-ʁwɤ˧}}}}{}
\textcolor{teal}{\mytextsc{nom}} \hspace{4pt} Ton~: M.

Le second village que l'on rencontre sur le trajet entre \textcolor{darkblue}{\textbf{\ipa{/qʰæ˧tɕʰi˧/}}} et \textcolor{darkblue}{\textbf{\ipa{/ʈʂo˧ʂɯ\#˥/}}}.
\textcolor{brown}{\zh{有米瓦村。}}




\paragraph{\hspace{-0.5cm} {\Large \textcolor{darkblue}{\textbf{\ipa{jo˩}}}}} \hypertarget{jo\string_B1}{}
\markboth{\textcolor{darkblue}{\textbf{\ipa{jo˩}}}}{}
\textcolor{teal}{\mytextsc{nom}} \hspace{4pt} Ton~: L.

Mouton.
\textcolor{brown}{\zh{绵羊。}}


\begin{exe}
\sn \textcolor{darkblue}{\textbf{\ipa{jo˩-ɣɯ˩˥}}}
\trans peau de mouton
\trans \textit{\textcolor{brown}{\zh{羊皮}}}
\end{exe}
 \mytextsc{clf}~: \textcolor{darkblue}{\textbf{\ipa{pʰo˧˥}}} 

\paragraph{\hspace{-0.5cm} {\Large \textcolor{darkblue}{\textbf{\ipa{jo˩\textsubscript{b}}}}}} \hypertarget{jo\string_Bb1}{}
\markboth{\textcolor{darkblue}{\textbf{\ipa{jo˩\textsubscript{b}}}}}{}
\textcolor{teal}{\mytextsc{classificateur}} \hspace{4pt} Ton~: L\textsubscript{b}.

Unité de poids: once.
\textcolor{brown}{\zh{量词:两(一两)。}}




\paragraph{\hspace{-0.5cm} {\Large \textcolor{darkblue}{\textbf{\ipa{jo˩gi˩}}}}} \hypertarget{jo\string_Bgi\string_B1}{}
\markboth{\textcolor{darkblue}{\textbf{\ipa{jo˩gi˩}}}}{}
\textcolor{teal}{\mytextsc{nom}} \hspace{4pt} Ton~: L.

Droite (contraire de: gauche).
\textcolor{brown}{\zh{右边。}}


\begin{exe}
\sn \textcolor{darkblue}{\textbf{\ipa{jo˩gi˩dzɤ˩}}}
\trans du côté droit, à droite
\trans \textit{\textcolor{brown}{\zh{右边}}}
\end{exe}
\textit{Voir~:} \hyperlink{jo\string_B\string_M1}{\textcolor{darkblue}{\textbf{\ipa{jo˩˧}}}} 

\paragraph{\hspace{-0.5cm} {\Large \textcolor{darkblue}{\textbf{\ipa{jo˩kʰv̩˩}}}}} \hypertarget{jo\string_Bk\string_hv\string_=\string_B1}{}
\markboth{\textcolor{darkblue}{\textbf{\ipa{jo˩kʰv̩˩}}}}{}
\textcolor{teal}{\mytextsc{nom}} \hspace{4pt} Ton~: L.

Année du mouton.
\textcolor{brown}{\zh{羊年。}}




\paragraph{\hspace{-0.5cm} {\Large \textcolor{darkblue}{\textbf{\ipa{jo˩lo˩}}}}} \hypertarget{jo\string_Blo\string_B1}{}
\markboth{\textcolor{darkblue}{\textbf{\ipa{jo˩lo˩}}}}{}
\textcolor{teal}{\mytextsc{nom}} \hspace{4pt} Ton~: L.

Droite (contraire de: gauche).
\textcolor{brown}{\zh{右边。}}


\textit{Voir~:} \hyperlink{jo\string_B\string_M1}{\textcolor{darkblue}{\textbf{\ipa{jo˩˧}}}} 

\paragraph{\hspace{-0.5cm} {\Large \textcolor{darkblue}{\textbf{\ipa{jo˩pv̩˧}}}}} \hypertarget{jo\string_Bpv\string_=\string_M1}{}
\markboth{\textcolor{darkblue}{\textbf{\ipa{jo˩pv̩˧}}}}{}
\textcolor{teal}{\mytextsc{nom}} \hspace{4pt} Ton~: LM / LM+MH\#.

Toile cirée.
\textcolor{brown}{\zh{油布。}}

 Emprunt~: chinois \textcolor{brown}{\zh{油布}}

\begin{exe}
\sn \textcolor{darkblue}{\textbf{\ipa{jo˩pv̩˧˥}}}
\trans toile cirée (variante tonale)
\trans \textit{\textcolor{brown}{\zh{油布(声调变体)}}}
\end{exe}
 \mytextsc{clf}~: \textcolor{darkblue}{\textbf{\ipa{tsʰi˥}}} 

\paragraph{\hspace{-0.5cm} {\Large \textcolor{darkblue}{\textbf{\ipa{jo˩pʰv̩˩}}}}} \hypertarget{jo\string_Bp\string_hv\string_=\string_B1}{}
\markboth{\textcolor{darkblue}{\textbf{\ipa{jo˩pʰv̩˩}}}}{}
\textcolor{teal}{\mytextsc{nom}} \hspace{4pt} Ton~: L.

Bélier.
\textcolor{brown}{\zh{公绵羊。}}


\begin{exe}
\sn \textcolor{darkblue}{\textbf{\ipa{jo˧pʰv̩˧ tʰv̩˧-mi˥\#}}}
\trans \string_ \mytextsc{dem} \mytextsc{clf}: ce bélier
\trans \textit{\textcolor{brown}{\zh{这头公羊}}}
\end{exe}
 \mytextsc{clf}~: \textcolor{darkblue}{\textbf{\ipa{pʰo˧˥}}} \textcolor{darkblue}{\textbf{\ipa{mi˩}}} \textit{Voir~:} \hyperlink{po\string_Mlo\string_M1}{\textcolor{darkblue}{\textbf{\ipa{po˧lo˧}}}} 

\paragraph{\hspace{-0.5cm} {\Large \textcolor{darkblue}{\textbf{\ipa{jo˩ʂwæ˩}}}}} \hypertarget{jo\string_Bs`w\{\string_B1}{}
\markboth{\textcolor{darkblue}{\textbf{\ipa{jo˩ʂwæ˩}}}}{}
\textcolor{teal}{\mytextsc{nom}} \hspace{4pt} Ton~: L.

Bélier châtré.
\textcolor{brown}{\zh{阉羊。}}


 \mytextsc{clf}~: \textcolor{darkblue}{\textbf{\ipa{pʰo˧˥}}} 

\paragraph{\hspace{-0.5cm} {\Large \textcolor{darkblue}{\textbf{\ipa{jo˩zo˩}}}}} \hypertarget{jo\string_Bzo\string_B1}{}
\markboth{\textcolor{darkblue}{\textbf{\ipa{jo˩zo˩}}}}{}
\textcolor{teal}{\mytextsc{nom}} \hspace{4pt} Ton~: L.

Agneau.
\textcolor{brown}{\zh{绵羊羔。}}


 \mytextsc{clf}~: \textcolor{darkblue}{\textbf{\ipa{ɭɯ˧}}} 

\paragraph{\hspace{-0.5cm} {\Large \textcolor{darkblue}{\textbf{\ipa{jo˧˥}}}}} \hypertarget{jo\string_M\string_T1}{}
\markboth{\textcolor{darkblue}{\textbf{\ipa{jo˧˥}}}}{}
\textcolor{teal}{\mytextsc{verbe}} \hspace{4pt} Ton~: MH.

Offrir.
\textcolor{brown}{\zh{赠给。}}




\paragraph{\hspace{-0.5cm} {\Large \textcolor{darkblue}{\textbf{\ipa{jo˩˧}}}}} \hypertarget{jo\string_B\string_M1}{}
\markboth{\textcolor{darkblue}{\textbf{\ipa{jo˩˧}}}}{}
\textcolor{teal}{\mytextsc{nom}} \hspace{4pt} Ton~: LM.

Droite (opposé de: gauche).
\textcolor{brown}{\zh{右边。}}


\textit{Voir~:} \hyperlink{jo\string_Bgi\string_B1}{\textcolor{darkblue}{\textbf{\ipa{jo˩gi˩}}}} 

\paragraph{\hspace{-0.5cm} {\Large \textcolor{darkblue}{\textbf{\ipa{je˧pʰi˧-jɤ\#˥}}}}} \hypertarget{je\string_Mp\string_hi\string_M-j7\#\string_T1}{}
\markboth{\textcolor{darkblue}{\textbf{\ipa{je˧pʰi˧-jɤ\#˥}}}}{}
\textcolor{teal}{\mytextsc{nom}} \hspace{4pt} Ton~: \#H.
\textit{De:} \textbf{\textcolor{brown}{\zh{鸦片}} et jɤ˧} 
Opium.
\textcolor{brown}{\zh{鸦片(汉语借词)。}}

 Emprunt~: chinois \textcolor{brown}{\zh{鸦片}}



\paragraph{\hspace{-0.5cm} {\Large \textcolor{darkblue}{\textbf{\ipa{je˩ʐe˧}}}}} \hypertarget{je\string_Bz`e\string_M1}{}
\markboth{\textcolor{darkblue}{\textbf{\ipa{je˩ʐe˧}}}}{}
\textcolor{teal}{\mytextsc{nom}} \hspace{4pt} Ton~: LM.

Occidental.
\textcolor{brown}{\zh{西方人(“洋人”)(汉语借词)。}}Dialecte chinois local~: \zh{洋人。}

 Emprunt~: chinois \textcolor{brown}{\zh{洋人}}

 \mytextsc{clf}~: \textcolor{darkblue}{\textbf{\ipa{v̩˧}}} 

\newpage
\section*{\centering- \textcolor{darkblue}{\textbf{\ipa{ʝ}}} -}
\paragraph{\hspace{-0.5cm} {\Large \textcolor{darkblue}{\textbf{\ipa{ʝi˥}}} \textsubscript{1}}} \hypertarget{j££i\string_T1}{}
\markboth{\textcolor{darkblue}{\textbf{\ipa{ʝi˥}}} \textsubscript{1}}{}
\textcolor{teal}{\mytextsc{nom}} \hspace{4pt} Ton~: \#H.

Vache, boeuf.
\textcolor{brown}{\zh{牛。}}


\begin{exe}
\sn \textcolor{darkblue}{\textbf{\ipa{ʝi˧-ɣɯ˥}}}
\trans peau de vache
\trans \textit{\textcolor{brown}{\zh{牛皮}}}
\end{exe}
\begin{exe}
\sn \textcolor{darkblue}{\textbf{\ipa{ʝi˧ tʰv̩˧-pʰo˩}}}
\trans \mytextsc{n}+\mytextsc{dem}+\mytextsc{clf}
\trans \textit{\textcolor{brown}{\zh{那头牛}}}
\end{exe}
 \mytextsc{clf}~: \textcolor{darkblue}{\textbf{\ipa{pʰo˧˥}}} 

\paragraph{\hspace{-0.5cm} {\Large \textcolor{darkblue}{\textbf{\ipa{ʝi˥}}} \textsubscript{2}}} \hypertarget{j££i\string_T2}{}
\markboth{\textcolor{darkblue}{\textbf{\ipa{ʝi˥}}} \textsubscript{2}}{}
\textcolor{teal}{\mytextsc{verbe}} \hspace{4pt} Ton~: H.

Travailler, faire.
\textcolor{brown}{\zh{做,工作。}}


\begin{exe}
\sn \textcolor{darkblue}{\textbf{\ipa{ɖwæ˧˥ | lo˧ ʝi˧}}}
\trans travailleur, assidu, qui travaille beaucoup
\trans \textit{\textcolor{brown}{\zh{勤劳、努力}}}
\end{exe}
\begin{exe}
\sn \textcolor{darkblue}{\textbf{\ipa{ɖɯ˧-sɑ˥ | mɤ˧-ʝi˥}}}
\trans ne rien faire du tout
\trans \textit{\textcolor{brown}{\zh{什么也不干}}}
\end{exe}
\begin{exe}
\sn \textcolor{darkblue}{\textbf{\ipa{ə˧tso˧-mɤ˧-ɲi˩ | ʝi˧-bi˧-zo˧-ho˥!}}}
\trans Il faut tout faire! / On va devoir m'occuper de tout!
\trans \textit{\textcolor{brown}{\zh{什么都要做! / 我什么都要干(/管)!}}}
\end{exe}
\begin{exe}
\sn \textcolor{darkblue}{\textbf{\ipa{ʈʂʰɯ˧ne-ʝi˥ | ʝi˧-zo˧-ho˥-ɲi˩!}}}
\trans Voilà comment il faut faire! / C'est comme ça qu'on fait!
\trans \textit{\textcolor{brown}{\zh{应该这样做的!}}}
\end{exe}
\begin{exe}
\sn \textcolor{darkblue}{\textbf{\ipa{ɑ˩ʁo˧ ʝi˧}}}
\trans gérer la maisonnée, s'occuper de la famille (tâche de la personne qui répartit les travaux à accomplir, veille aux approvisionnements...)
\trans \textit{\textcolor{brown}{\zh{管理家里的大小事情(如:分配工作、家务等)}}}
\end{exe}


\paragraph{\hspace{-0.5cm} {\Large \textcolor{darkblue}{\textbf{\ipa{ʝi˥}}} \textsubscript{3}}} \hypertarget{j££i\string_T3}{}
\markboth{\textcolor{darkblue}{\textbf{\ipa{ʝi˥}}} \textsubscript{3}}{}
\textcolor{teal}{\mytextsc{verbe}} \hspace{4pt} Ton~: H.

Dessiner, tracer.
\textcolor{brown}{\zh{画。}}


\begin{exe}
\sn \textcolor{darkblue}{\textbf{\ipa{mɤ˧-ʝi˥}}}
\trans \mytextsc{neg}
\trans \textit{\textcolor{brown}{\zh{不画}}}
\end{exe}
\begin{exe}
\sn \textcolor{darkblue}{\textbf{\ipa{tʰɑ˧-ʝi˥!}}}
\trans \mytextsc{prohib}
\trans \textit{\textcolor{brown}{\zh{别画!}}}
\end{exe}
\begin{exe}
\sn \textcolor{darkblue}{\textbf{\ipa{ʈʂɑ˧tɑ˥ ʝi˩}}}
\trans tracer un signe, dessiner un signe
\trans \textit{\textcolor{brown}{\zh{画一个符号}}}
\end{exe}


\paragraph{\hspace{-0.5cm} {\Large \textcolor{darkblue}{\textbf{\ipa{ʝi˥}}} \textsubscript{4}}} \hypertarget{j££i\string_T4}{}
\markboth{\textcolor{darkblue}{\textbf{\ipa{ʝi˥}}} \textsubscript{4}}{}
\textcolor{teal}{\mytextsc{nom}} \hspace{4pt} Ton~: \#H.

Jarre en terre cuite.
\textcolor{brown}{\zh{坛子,罐子 (陶器)。}}


 \mytextsc{clf}~: \textcolor{darkblue}{\textbf{\ipa{ɭɯ˧}}} 

\paragraph{\hspace{-0.5cm} {\Large \textcolor{darkblue}{\textbf{\ipa{ʝi˥}}} \textsubscript{5}}} \hypertarget{j££i\string_T5}{}
\markboth{\textcolor{darkblue}{\textbf{\ipa{ʝi˥}}} \textsubscript{5}}{}
\textcolor{teal}{\mytextsc{verbe}} \hspace{4pt} Ton~: H.

Informer.
\textcolor{brown}{\zh{通知、告诉。}}


\begin{exe}
\sn \textcolor{darkblue}{\textbf{\ipa{le˧-ʝi˥-ze˩}}}
\trans \mytextsc{accomp} \string_ \mytextsc{pfv}
\trans \textit{\textcolor{brown}{\zh{通知了}}}
\end{exe}
\begin{exe}
\sn \textcolor{darkblue}{\textbf{\ipa{qʰwæ˧ mi˧ ʝi˧}}}
\trans donner une nouvelle
\trans \textit{\textcolor{brown}{\zh{告诉(一个)消息}}}
\end{exe}
\begin{exe}
\sn \textcolor{darkblue}{\textbf{\ipa{njɤ˧ | hĩ˧-ki˧ | qʰwæ˧mi˧ ʝi˧-ze˩}}}
\trans j'ai annoncé la nouvelle aux gens / j'ai annoncé une nouvelle à quelqu'un
\trans \textit{\textcolor{brown}{\zh{我告诉了人家(那个消息)。}}}
\end{exe}


\paragraph{\hspace{-0.5cm} {\Large \textcolor{darkblue}{\textbf{\ipa{ʝi˥}}} \textsubscript{6}}} \hypertarget{j££i\string_T6}{}
\markboth{\textcolor{darkblue}{\textbf{\ipa{ʝi˥}}} \textsubscript{6}}{}
\textcolor{teal}{\mytextsc{nom}} \hspace{4pt} Ton~: \#H.
\textit{Archaïque} [\textcolor{brown}{\zh{古语}}] 
Homme \textit{(vir)}.
\textcolor{brown}{\zh{男人。}}




\paragraph{\hspace{-0.5cm} {\Large \textcolor{darkblue}{\textbf{\ipa{ʝi˥}}} \textsubscript{7}}} \hypertarget{j££i\string_T7}{}
\markboth{\textcolor{darkblue}{\textbf{\ipa{ʝi˥}}} \textsubscript{7}}{}
\textcolor{teal}{\mytextsc{verbe}} \hspace{4pt} Ton~: H.

Verbe d'existence: choses amovibles.
\textcolor{brown}{\zh{存在动词:有(可移动物品)。}}


\begin{exe}
\sn \textcolor{darkblue}{\textbf{\ipa{ə˧tso˧-mɤ˧-ɲi˩, | le˧-ʂe˧, | le˧-ʝi˥!}}}
\trans (En vue d'un rituel, d'une fête…) on rassemble toutes sortes de choses; on en a (sous la main)/on a fait une provision! / On prépare tout par avance (pour le rituel/la fête)!
\trans \textit{\textcolor{brown}{\zh{所有(的东西都)找,(就)有了 = 所有的东西都备好了}}}
\end{exe}


\paragraph{\hspace{-0.5cm} {\Large \textcolor{darkblue}{\textbf{\ipa{ʝi˧}}} \textsubscript{1}}} \hypertarget{j££i\string_M1}{}
\markboth{\textcolor{darkblue}{\textbf{\ipa{ʝi˧}}} \textsubscript{1}}{}
\textcolor{teal}{\mytextsc{verbe}} \hspace{4pt} Ton~: M\textsubscript{c}.

Venir.
\textcolor{brown}{\zh{来。}}


\begin{exe}
\sn \textcolor{darkblue}{\textbf{\ipa{lɑ˧ ʝi˧-ze˧!}}}
\trans Voilà le tigre! / Un tigre arrive!
\trans \textit{\textcolor{brown}{\zh{老虎来了!}}}
\end{exe}
\begin{exe}
\sn \textcolor{darkblue}{\textbf{\ipa{lɑ˧ le˧-ʝi˩-ze˩!}}}
\trans Voilà le tigre qui revient! / Le tigre est revenu!/ Le tigre est de retour!
\trans \textit{\textcolor{brown}{\zh{老虎又来了!}}}
\end{exe}
\begin{exe}
\sn \textcolor{darkblue}{\textbf{\ipa{mɤ˧-ʝi˧-ze˧!}}}
\trans Ca ne va plus! / C'est la catastrophe!
\trans \textit{\textcolor{brown}{\zh{不好了!不行了!}}}
\end{exe}


\paragraph{\hspace{-0.5cm} {\Large \textcolor{darkblue}{\textbf{\ipa{ʝi˧}}} \textsubscript{2}}} \hypertarget{j££i\string_M2}{}
\markboth{\textcolor{darkblue}{\textbf{\ipa{ʝi˧}}} \textsubscript{2}}{}
\textcolor{teal}{\mytextsc{nom}} \hspace{4pt} Ton~: M.

Un (numéral, à emploi restreint; ne se combine qu'avec le classificateur /ɭɯ˧/).
\textcolor{brown}{\zh{一。}}


\begin{exe}
\sn \textcolor{darkblue}{\textbf{\ipa{zo˧mv̩˥ | ʝi˧-ɭɯ˧ ʂv̩˧}}}
\trans s'occuper d'un enfant
\trans \textit{\textcolor{brown}{\zh{管一个孩子}}}
\end{exe}


\paragraph{\hspace{-0.5cm} {\Large \textcolor{darkblue}{\textbf{\ipa{ʝi˧\textsubscript{b}}}}}} \hypertarget{j££i\string_Mb1}{}
\markboth{\textcolor{darkblue}{\textbf{\ipa{ʝi˧\textsubscript{b}}}}}{}
\textcolor{teal}{\mytextsc{classificateur}} \hspace{4pt} Ton~: M\textsubscript{b}.

Classificateur des lieux.
\textcolor{brown}{\zh{量词:地方(一个)。}}


\begin{exe}
\sn \textcolor{darkblue}{\textbf{\ipa{ɖɯ˧-ʝi˧}}}
\trans un endroit; qq part
\trans \textit{\textcolor{brown}{\zh{一个地方}}}
\end{exe}
\begin{exe}
\sn \textcolor{darkblue}{\textbf{\ipa{ɖɯ˧-ʝi˧ dzi˩}}}
\trans habiter quelque part; emménager quelque part/déménager vers quelque part
\trans \textit{\textcolor{brown}{\zh{住在一个地方,搬家到一个地方}}}
\end{exe}
\begin{exe}
\sn \textcolor{darkblue}{\textbf{\ipa{ɖɯ˧-v˧ | ɖɯ˧-ʝi˧ hɯ˧}}}
\trans chacun s'en va de son côté (contexte: les membres d'une famille vont habiter en des lieux différents pour raisons professionnelles)
\trans \textit{\textcolor{brown}{\zh{个去个的地方!/ 每个人去不同的地方!(情景:由于工作原因,一家的成员经常需要去不同的城市工作。)}}}
\end{exe}


\paragraph{\hspace{-0.5cm} {\Large \textcolor{darkblue}{\textbf{\ipa{ʝi˧-bv̩˧˥}}}}} \hypertarget{j££i\string_M-bv\string_=\string_M\string_T1}{}
\markboth{\textcolor{darkblue}{\textbf{\ipa{ʝi˧-bv̩˧˥}}}}{}
\textcolor{teal}{\mytextsc{nom}} \hspace{4pt} Ton~: MH\#.

Étable (des vaches).
\textcolor{brown}{\zh{牛圈。}}


 \mytextsc{clf}~: \textcolor{darkblue}{\textbf{\ipa{ɭɯ˧}}} 

\paragraph{\hspace{-0.5cm} {\Large \textcolor{darkblue}{\textbf{\ipa{ʝi˧kʰv̩˩}}}}} \hypertarget{j££i\string_Mk\string_hv\string_=\string_B1}{}
\markboth{\textcolor{darkblue}{\textbf{\ipa{ʝi˧kʰv̩˩}}}}{}
\textcolor{teal}{\mytextsc{nom}} \hspace{4pt} Ton~: L\#.

Année du bœuf / année du boeuf.
\textcolor{brown}{\zh{牛年。}}




\paragraph{\hspace{-0.5cm} {\Large \textcolor{darkblue}{\textbf{\ipa{ʝi˧kʰv̩˥}}}}} \hypertarget{j££i\string_Mk\string_hv\string_=\string_T1}{}
\markboth{\textcolor{darkblue}{\textbf{\ipa{ʝi˧kʰv̩˥}}}}{}
\textcolor{teal}{\mytextsc{pronom}} \hspace{4pt} Ton~: H\#.

Certains.
\textcolor{brown}{\zh{一些。}}


\begin{exe}
\sn \textcolor{darkblue}{\textbf{\ipa{hĩ˧ ʝi˧kʰv̩˥}}}
\trans certaines personnes, une partie des gens
\trans \textit{\textcolor{brown}{\zh{一些人}}}
\end{exe}


\paragraph{\hspace{-0.5cm} {\Large \textcolor{darkblue}{\textbf{\ipa{ʝi˧kʰwɤ˥\$}}}}} \hypertarget{j££i\string_Mk\string_hw7\string_T\$1}{}
\markboth{\textcolor{darkblue}{\textbf{\ipa{ʝi˧kʰwɤ˥\$}}}}{}
\textcolor{teal}{\mytextsc{pronom}} \hspace{4pt} Ton~: H\$.

Un peu.
\textcolor{brown}{\zh{一点。}}




\paragraph{\hspace{-0.5cm} {\Large \textcolor{darkblue}{\textbf{\ipa{ʝi˧lo\#˥}}}}} \hypertarget{j££i\string_Mlo\#\string_T1}{}
\markboth{\textcolor{darkblue}{\textbf{\ipa{ʝi˧lo\#˥}}}}{}
\textcolor{teal}{\mytextsc{nom}} \hspace{4pt} Ton~: \#H.

Traitement (d'autrui), attitude.
\textcolor{brown}{\zh{态度、对待的态度。}}


\begin{exe}
\sn \textcolor{darkblue}{\textbf{\ipa{ʝi˧lo˧ dʑɤ˥!}}}
\trans (Il / elle) a une attitude positive
\trans \textit{\textcolor{brown}{\zh{态度积极}}}
\end{exe}
\begin{exe}
\sn \textcolor{darkblue}{\textbf{\ipa{ʈʂʰɯ˧ | ʝi˧lo˧ | dʑɤ˩˥! | hĩ˧-ki˧ | dʑɤ˩-ʝi˥!}}}
\trans Il/elle traite bien les gens! Il/elle fait de bonnes actions!
\trans \textit{\textcolor{brown}{\zh{他(对人)态度好!对人好/做好事!}}}
\end{exe}
\begin{exe}
\sn \textcolor{darkblue}{\textbf{\ipa{ʝi˧lo˧ dzɑ˧}}}
\trans (avoir une) mauvaise attitude: paresseuse, dissipée…
\trans \textit{\textcolor{brown}{\zh{态度不好}}}
\end{exe}
\begin{exe}
\sn \textcolor{darkblue}{\textbf{\ipa{njɤ˧-ɳɯ˧ hɑ˧ gv̩˥, | ʝi˧lo˧ dzɑ˧!}}}
\trans Quand je fais la cuisine, je ne suis pas bien concentrée/je travaille n'importe comment!
\trans \textit{\textcolor{brown}{\zh{我做饭,集中不了精神 / 做的乱七八糟!}}}
\end{exe}
\begin{exe}
\sn \textcolor{darkblue}{\textbf{\ipa{ʈʂʰɯ˧ | ə˧tso˧ ʝi˧lo˧ ɲi˥?}}}
\trans Qu'est-ce que c'est que cette attitude? (critique adressée à quelqu'un qui fait n'importe quoi)
\trans \textit{\textcolor{brown}{\zh{这是什么态度啊?(批评一个人的态度)}}}
\end{exe}


\paragraph{\hspace{-0.5cm} {\Large \textcolor{darkblue}{\textbf{\ipa{ʝi˧mi˧}}}}} \hypertarget{j££i\string_Mmi\string_M1}{}
\markboth{\textcolor{darkblue}{\textbf{\ipa{ʝi˧mi˧}}}}{}
\textcolor{teal}{\mytextsc{nom}} \hspace{4pt} Ton~: M.

Jarre.
\textcolor{brown}{\zh{坛子,罐子 (陶器)。}}


 \mytextsc{clf}~: \textcolor{darkblue}{\textbf{\ipa{ɭɯ˧}}} 

\paragraph{\hspace{-0.5cm} {\Large \textcolor{darkblue}{\textbf{\ipa{ʝi˧pʰv̩\#˥}}}}} \hypertarget{j££i\string_Mp\string_hv\string_=\#\string_T1}{}
\markboth{\textcolor{darkblue}{\textbf{\ipa{ʝi˧pʰv̩\#˥}}}}{}
\textcolor{teal}{\mytextsc{nom}} \hspace{4pt} Ton~: \#H.

Taureau.
\textcolor{brown}{\zh{公牛。}}


\begin{exe}
\sn \textcolor{darkblue}{\textbf{\ipa{ʝi˧pʰv̩˧ tʰv̩˧-mi˥\#}}}
\trans \mytextsc{n}+\mytextsc{dem}+\mytextsc{clf}
\trans \textit{\textcolor{brown}{\zh{那头公牛}}}
\end{exe}
\begin{exe}
\sn \textcolor{darkblue}{\textbf{\ipa{ʝi˧pʰv̩˧ tʰv̩˧-ɭɯ\#˥}}}
\trans \mytextsc{n}+\mytextsc{dem}+\mytextsc{clf}.animaux
\trans \textit{\textcolor{brown}{\zh{那头公牛}}}
\end{exe}
 \mytextsc{clf}~: \textcolor{darkblue}{\textbf{\ipa{ɭɯ˧ / mi˩}}} 

\paragraph{\hspace{-0.5cm} {\Large \textcolor{darkblue}{\textbf{\ipa{ʝi˧qv̩˥}}}}} \hypertarget{j££i\string_Mqv\string_=\string_T1}{}
\markboth{\textcolor{darkblue}{\textbf{\ipa{ʝi˧qv̩˥}}}}{}
\textcolor{teal}{\mytextsc{nom}} \hspace{4pt} Ton~: H\#.

Collier: une partie du harnais utilisé pour les labours, qui maintient le joug en place; cette corde passe sous le cou de l'animal, et est fixée au joug.
\textcolor{brown}{\zh{轭的一个部分,将牛轭安在牛的脖子上。}}


 \mytextsc{clf}~: \textcolor{darkblue}{\textbf{\ipa{ɭɯ˧}}} 

\paragraph{\hspace{-0.5cm} {\Large \textcolor{darkblue}{\textbf{\ipa{ʝi˧ʁæ˥}}}}} \hypertarget{j££i\string_MR\{\string_T1}{}
\markboth{\textcolor{darkblue}{\textbf{\ipa{ʝi˧ʁæ˥}}}}{}
\textcolor{teal}{\mytextsc{nom}} \hspace{4pt} Ton~: H\#.

Vache, boeuf.
\textcolor{brown}{\zh{黄牛。}}


 \mytextsc{clf}~: \textcolor{darkblue}{\textbf{\ipa{pʰo˧˥}}} 

\paragraph{\hspace{-0.5cm} {\Large \textcolor{darkblue}{\textbf{\ipa{ʝi˧ʁo\#˥}}}}} \hypertarget{j££i\string_MRo\#\string_T1}{}
\markboth{\textcolor{darkblue}{\textbf{\ipa{ʝi˧ʁo\#˥}}}}{}
\textcolor{teal}{\mytextsc{adjectif}} \hspace{4pt} Ton~: \#H.
\textit{De:} \textbf{ʝi˥ 1 et ʁo˧ 2} 
Capable; littéralement ``qui sait faire".
\textcolor{brown}{\zh{能干、不缺劳力。}}


\begin{exe}
\sn \textcolor{darkblue}{\textbf{\ipa{ʈʂʰɯ˧ | ʝi˧ʁo˧-hĩ˧ | ɖɯ˧-v̩˧ ɲi˩}}}
\trans c'est quelqu'un d'habile/de capable
\trans \textit{\textcolor{brown}{\zh{他是一个能干/称职的人。}}}
\end{exe}
\begin{exe}
\sn \textcolor{darkblue}{\textbf{\ipa{ʝi˧ʁo˧-zo˥}}}
\trans un homme capable/habile, un gaillard compétent
\trans \textit{\textcolor{brown}{\zh{一个能干的男人}}}
\end{exe}
\begin{exe}
\sn \textcolor{darkblue}{\textbf{\ipa{ʝi˧ʁo˧ ɲi˥}}}
\trans \mytextsc{cop}
\trans \textit{\textcolor{brown}{\zh{\mytextsc{cop}}}}
\end{exe}


\paragraph{\hspace{-0.5cm} {\Large \textcolor{darkblue}{\textbf{\ipa{ʝi˧sɑ˧}}}}} \hypertarget{j££i\string_MsA\string_M1}{}
\markboth{\textcolor{darkblue}{\textbf{\ipa{ʝi˧sɑ˧}}}}{}
\textcolor{teal}{\mytextsc{nom}} \hspace{4pt} Ton~: M.

Parapluie (emprunt).
\textcolor{brown}{\zh{雨伞。}}

 Emprunt~: chinois \textcolor{brown}{\zh{雨伞}}

 \mytextsc{clf}~: \textcolor{darkblue}{\textbf{\ipa{nɑ˧}}} 

\paragraph{\hspace{-0.5cm} {\Large \textcolor{darkblue}{\textbf{\ipa{ʝi˧se˧}}} \textsubscript{1}}} \hypertarget{j££i\string_Mse\string_M1}{}
\markboth{\textcolor{darkblue}{\textbf{\ipa{ʝi˧se˧}}} \textsubscript{1}}{}
\textcolor{teal}{\mytextsc{nom}} \hspace{4pt} Ton~: M.

Médecin.
\textcolor{brown}{\zh{医生(汉语借词)。}}

 Emprunt~: chinois \textcolor{brown}{\zh{医生}}



\paragraph{\hspace{-0.5cm} {\Large \textcolor{darkblue}{\textbf{\ipa{ʝi˧se˧}}} \textsubscript{2}}} \hypertarget{j££i\string_Mse\string_M2}{}
\markboth{\textcolor{darkblue}{\textbf{\ipa{ʝi˧se˧}}} \textsubscript{2}}{}
\textcolor{teal}{\mytextsc{adjectif}} \hspace{4pt} Ton~: M.

Sauvage, spontané: plantes qui poussent spontanément (par opposition aux plantes cultivées), animaux sauvages.
\textcolor{brown}{\zh{野生(汉语借词)。}}

 Emprunt~: chinois \textcolor{brown}{\zh{野生}}

\begin{exe}
\sn \textcolor{darkblue}{\textbf{\ipa{ʝi˧se˧-hĩ˧}}}
\trans \string_ \mytextsc{rel}/\mytextsc{nmlz}
\trans \textit{\textcolor{brown}{\zh{野生的}}}
\end{exe}


\paragraph{\hspace{-0.5cm} {\Large \textcolor{darkblue}{\textbf{\ipa{ʝi˧sɯ˥}}}}} \hypertarget{j££i\string_MsM\string_T1}{}
\markboth{\textcolor{darkblue}{\textbf{\ipa{ʝi˧sɯ˥}}}}{}
\textcolor{teal}{\mytextsc{nom}} \hspace{4pt} Ton~: H\#.

Signification, sens.
\textcolor{brown}{\zh{意思(汉语借词)。}}

 Emprunt~: chinois \textcolor{brown}{\zh{意思}}



\paragraph{\hspace{-0.5cm} {\Large \textcolor{darkblue}{\textbf{\ipa{ʝi˧ʂæ˧tsʰɤ˩}}}}} \hypertarget{j££i\string_Ms`\{\string_Mts\string_h7\string_B1}{}
\markboth{\textcolor{darkblue}{\textbf{\ipa{ʝi˧ʂæ˧tsʰɤ˩}}}}{}
\textcolor{teal}{\mytextsc{nom}} \hspace{4pt} Ton~: L\#.

Radis sauvage qui pousse en montagne; on le consomme surtout au printemps, à une époque où il n'y a pas encore de légumes. Ce radis est récoltée par les Yi et vendu dans la plaine.
\textcolor{brown}{\zh{红萝卜菜(汉语借词:野山菜):一种山上的野菜。春天的时候,菜园的蔬菜还没有成熟的时候,永宁的人吃红萝卜菜。彝族人从高山上采下来,在永宁卖。}}Dialecte chinois local~: \zh{野山菜。}

 Emprunt~: chinois \textcolor{brown}{\zh{野山菜}}

\textit{Voir~:} \hyperlink{j7\string_M\string_T2}{\textcolor{darkblue}{\textbf{\ipa{jɤ˧˥}}} \textsubscript{2}} 

\paragraph{\hspace{-0.5cm} {\Large \textcolor{darkblue}{\textbf{\ipa{ʝi˧ʂɯ˥}}}}} \hypertarget{j££i\string_Ms`M\string_T1}{}
\markboth{\textcolor{darkblue}{\textbf{\ipa{ʝi˧ʂɯ˥}}}}{}
\textcolor{teal}{\mytextsc{nom}} \hspace{4pt} Ton~: H\#.

Prénom masculin.
\textcolor{brown}{\zh{男性名字。}}




\paragraph{\hspace{-0.5cm} {\Large \textcolor{darkblue}{\textbf{\ipa{ʝi˧tɕi˧}}}}} \hypertarget{j££i\string_Mts£i\string_M1}{}
\markboth{\textcolor{darkblue}{\textbf{\ipa{ʝi˧tɕi˧}}}}{}
\textcolor{teal}{\mytextsc{nom}} \hspace{4pt} Ton~: M.

Prénom féminin.
\textcolor{brown}{\zh{女性名字。}}




\paragraph{\hspace{-0.5cm} {\Large \textcolor{darkblue}{\textbf{\ipa{ʝi˧tɕi˧-ɖɯ˩mɑ˩}}}}} \hypertarget{j££i\string_Mts£i\string_M-d`M\string_BmA\string_B1}{}
\markboth{\textcolor{darkblue}{\textbf{\ipa{ʝi˧tɕi˧-ɖɯ˩mɑ˩}}}}{}
\textcolor{teal}{\mytextsc{nom}} \hspace{4pt} Ton~: \mytextsc{L}.

Prénom féminin.
\textcolor{brown}{\zh{女性名字。}}




\paragraph{\hspace{-0.5cm} {\Large \textcolor{darkblue}{\textbf{\ipa{ʝi˧tɕi˧-ɬɑ˩mv̩˩}}}}} \hypertarget{j££i\string_Mts£i\string_M-KA\string_Bmv\string_=\string_B1}{}
\markboth{\textcolor{darkblue}{\textbf{\ipa{ʝi˧tɕi˧-ɬɑ˩mv̩˩}}}}{}
\textcolor{teal}{\mytextsc{nom}} \hspace{4pt} Ton~: \mytextsc{L}.

Prénom féminin.
\textcolor{brown}{\zh{女性名字。}}




\paragraph{\hspace{-0.5cm} {\Large \textcolor{darkblue}{\textbf{\ipa{ʝi˧tsɯ˧}}}}} \hypertarget{j££i\string_MtsM\string_M1}{}
\markboth{\textcolor{darkblue}{\textbf{\ipa{ʝi˧tsɯ˧}}}}{}
\textcolor{teal}{\mytextsc{nom}} \hspace{4pt} Ton~: M.

Chaise (emprunt).
\textcolor{brown}{\zh{椅子。}}

 Emprunt~: chinois \textcolor{brown}{\zh{椅子}}

 \mytextsc{clf}~: \textcolor{darkblue}{\textbf{\ipa{nɑ˧}}} 

\paragraph{\hspace{-0.5cm} {\Large \textcolor{darkblue}{\textbf{\ipa{ʝi˧ʈʂʰe˥-mi˩}}}}} \hypertarget{j££i\string_Mt`s`\string_he\string_T-mi\string_B1}{}
\markboth{\textcolor{darkblue}{\textbf{\ipa{ʝi˧ʈʂʰe˥-mi˩}}}}{}
\textcolor{teal}{\mytextsc{nom}} \hspace{4pt} Ton~: H\#-.

Sud.
\textcolor{brown}{\zh{南方。}}


\begin{exe}
\sn \textcolor{darkblue}{\textbf{\ipa{ʝi˧ʈʂʰe˥mi˩-gi˩dzɤ˩ se˩}}}
\trans marcher en direction du sud
\trans \textit{\textcolor{brown}{\zh{往南方走}}}
\end{exe}


\paragraph{\hspace{-0.5cm} {\Large \textcolor{darkblue}{\textbf{\ipa{ʝi˧zo\#˥}}}}} \hypertarget{j££i\string_Mzo\#\string_T1}{}
\markboth{\textcolor{darkblue}{\textbf{\ipa{ʝi˧zo\#˥}}}}{}
\textcolor{teal}{\mytextsc{nom}} \hspace{4pt} Ton~: \#H.

Veau.
\textcolor{brown}{\zh{小牛。}}


\begin{exe}
\sn \textcolor{darkblue}{\textbf{\ipa{ʝi˧zo˧ tʰv̩˧-ɭɯ\#˥}}}
\trans \mytextsc{n}+\mytextsc{dem}+\mytextsc{clf}
\trans \textit{\textcolor{brown}{\zh{那头小牛}}}
\end{exe}
 \mytextsc{clf}~: \textcolor{darkblue}{\textbf{\ipa{pʰo˧˥ / ɭɯ˧}}} 

\paragraph{\hspace{-0.5cm} {\Large \textcolor{darkblue}{\textbf{\ipa{ʝi˩bv̩˩}}} \textsubscript{1}}} \hypertarget{j££i\string_Bbv\string_=\string_B1}{}
\markboth{\textcolor{darkblue}{\textbf{\ipa{ʝi˩bv̩˩}}} \textsubscript{1}}{}
\textcolor{teal}{\mytextsc{nom}} \hspace{4pt} Ton~: L.

Grêlé.
\textcolor{brown}{\zh{麻子。}}


\begin{exe}
\sn \textcolor{darkblue}{\textbf{\ipa{ʝi˩bv̩˩-ʝi˧ʈv̩˩ʈv̩˩}}}
\trans même sens
\trans \textit{\textcolor{brown}{\zh{同上}}}
\end{exe}
 \mytextsc{clf}~: \textcolor{darkblue}{\textbf{\ipa{v̩˧}}} 

\paragraph{\hspace{-0.5cm} {\Large \textcolor{darkblue}{\textbf{\ipa{ʝi˩bv̩˩}}} \textsubscript{2}}} \hypertarget{j££i\string_Bbv\string_=\string_B2}{}
\markboth{\textcolor{darkblue}{\textbf{\ipa{ʝi˩bv̩˩}}} \textsubscript{2}}{}
\textcolor{teal}{\mytextsc{nom}} \hspace{4pt} Ton~: L.

Taureau.
\textcolor{brown}{\zh{公牛。}}


 \mytextsc{clf}~: \textcolor{darkblue}{\textbf{\ipa{pʰo˧˥}}} 

\paragraph{\hspace{-0.5cm} {\Large \textcolor{darkblue}{\textbf{\ipa{ʝi˩di˩-mi˥}}}}} \hypertarget{j££i\string_Bdi\string_B-mi\string_T1}{}
\markboth{\textcolor{darkblue}{\textbf{\ipa{ʝi˩di˩-mi˥}}}}{}
\textcolor{teal}{\mytextsc{nom}} \hspace{4pt} Ton~: L+H\#.

Génisse; s'emploie pour une petite vache, aussi bien que pour le pianniu (tibétain: dzomo).
\textcolor{brown}{\zh{小牝牛(包括黄牛和小母犏牛)。}}


 \mytextsc{clf}~: \textcolor{darkblue}{\textbf{\ipa{ɭɯ˧}}} \textcolor{darkblue}{\textbf{\ipa{pʰo˧˥}}} 

\paragraph{\hspace{-0.5cm} {\Large \textcolor{darkblue}{\textbf{\ipa{ʝi˩mi˩}}}}} \hypertarget{j££i\string_Bmi\string_B1}{}
\markboth{\textcolor{darkblue}{\textbf{\ipa{ʝi˩mi˩}}}}{}
\textcolor{teal}{\mytextsc{nom}} \hspace{4pt} Ton~: L.

Vache.
\textcolor{brown}{\zh{母牛。}}


\begin{exe}
\sn \textcolor{darkblue}{\textbf{\ipa{ʝi˩mi˩-ʐɤ˥qo˩}}}
\trans vache et veau
\trans \textit{\textcolor{brown}{\zh{母牛与小牛}}}
\end{exe}
 \mytextsc{clf}~: \textcolor{darkblue}{\textbf{\ipa{pʰo˧˥}}} 

\paragraph{\hspace{-0.5cm} {\Large \textcolor{darkblue}{\textbf{\ipa{ʝi˩næ˩-se˧}}}}} \hypertarget{j££i\string_Bn\{\string_B-se\string_M1}{}
\markboth{\textcolor{darkblue}{\textbf{\ipa{ʝi˩næ˩-se˧}}}}{}
\textcolor{teal}{\mytextsc{nom}} \hspace{4pt} Ton~: L-M.

Kunming et la partie orientale du Yunnan, une fois passés Lijiang et Dali.
\textcolor{brown}{\zh{云南,昆明……。}}

 Emprunt~: chinois \textcolor{brown}{\zh{云南省}}

\begin{exe}
\sn \textcolor{darkblue}{\textbf{\ipa{sɯ˧pʰi˧ | ʝi˩næ˩-se˧-qo˧ hɯ˧-ɲi˥!}}}
\trans Le seigneur est parti à Kunming!
\trans \textit{\textcolor{brown}{\zh{土司到昆明去了!}}}
\end{exe}


\paragraph{\hspace{-0.5cm} {\Large \textcolor{darkblue}{\textbf{\ipa{ʝi˩ŋɤ˧˥}}}}} \hypertarget{j££i\string_BN7\string_M\string_T1}{}
\markboth{\textcolor{darkblue}{\textbf{\ipa{ʝi˩ŋɤ˧˥}}}}{}
\textcolor{teal}{\mytextsc{verbe}} \hspace{4pt} Ton~: LM+MH\#.

Se courber vers l’arrière.
\textcolor{brown}{\zh{往后仰。}}


\begin{exe}
\sn \textcolor{darkblue}{\textbf{\ipa{ʝi˩ŋɤ˧-ze˥}}}
\trans \mytextsc{pfv}
\trans \textit{\textcolor{brown}{\zh{往后仰了}}}
\end{exe}
\begin{exe}
\sn \textcolor{darkblue}{\textbf{\ipa{ʝi˩ŋɤ˧˥ | tʰi˧-dzi˩}}}
\trans être assis en s'inclinant vers l'arrière
\trans \textit{\textcolor{brown}{\zh{坐着往后仰}}}
\end{exe}


\paragraph{\hspace{-0.5cm} {\Large \textcolor{darkblue}{\textbf{\ipa{ʝi˩qʰv̩˩}}}}} \hypertarget{j££i\string_Bq\string_hv\string_=\string_B1}{}
\markboth{\textcolor{darkblue}{\textbf{\ipa{ʝi˩qʰv̩˩}}}}{}
\textcolor{teal}{\mytextsc{nom}} \hspace{4pt} Ton~: L.

Manche.
\textcolor{brown}{\zh{袖子。}}


 \mytextsc{clf}~: \textcolor{darkblue}{\textbf{\ipa{ɭɯ˧}}} 

\paragraph{\hspace{-0.5cm} {\Large \textcolor{darkblue}{\textbf{\ipa{ʝi˩ʈʂæ˧˥}}}}} \hypertarget{j££i\string_Bt`s`\{\string_M\string_T1}{}
\markboth{\textcolor{darkblue}{\textbf{\ipa{ʝi˩ʈʂæ˧˥}}}}{}
\textcolor{teal}{\mytextsc{nom}} \hspace{4pt} Ton~: LM+MH\#.

Taille.
\textcolor{brown}{\zh{腰。}}


 \mytextsc{clf}~: \textcolor{darkblue}{\textbf{\ipa{ɭɯ˧}}} 

\paragraph{\hspace{-0.5cm} {\Large \textcolor{darkblue}{\textbf{\ipa{ʝi˩˥}}}}} \hypertarget{j££i\string_B\string_T1}{}
\markboth{\textcolor{darkblue}{\textbf{\ipa{ʝi˩˥}}}}{}
\textcolor{teal}{\mytextsc{nom}} \hspace{4pt} Ton~: LH.

Bouton.
\textcolor{brown}{\zh{痘痘。}}


\begin{exe}
\sn \textcolor{darkblue}{\textbf{\ipa{ʝi˩ tʰv̩˩˥}}}
\trans avoir des boutons
\trans \textit{\textcolor{brown}{\zh{出痘痘}}}
\end{exe}
 \mytextsc{clf}~: \textcolor{darkblue}{\textbf{\ipa{ɭɯ˧}}} 

\newpage
\section*{\centering- \textcolor{darkblue}{\textbf{\ipa{k}}} -}
\paragraph{\hspace{-0.5cm} {\Large \textcolor{darkblue}{\textbf{\ipa{kæ˧ʈʂe˧}}}}} \hypertarget{k\{\string_Mt`s`e\string_M1}{}
\markboth{\textcolor{darkblue}{\textbf{\ipa{kæ˧ʈʂe˧}}}}{}
\textcolor{teal}{\mytextsc{nom}} \hspace{4pt} Ton~: M.

Acupuncture.
\textcolor{brown}{\zh{针灸(汉语借词:干针)。}}Dialecte chinois local~: \zh{干针。}

 Emprunt~: chinois \textcolor{brown}{\zh{干针}}

\begin{exe}
\sn \textcolor{darkblue}{\textbf{\ipa{kæ˧ʈʂe˧ lɑ˧˥}}}
\trans faire une séance d'acupuncture, placer des aiguilles d'acupuncture
\trans \textit{\textcolor{brown}{\zh{扎针灸}}}
\end{exe}


\paragraph{\hspace{-0.5cm} {\Large \textcolor{darkblue}{\textbf{\ipa{kæ˧ʈʂɯ˧}}}}} \hypertarget{k\{\string_Mt`s`M\string_M1}{}
\markboth{\textcolor{darkblue}{\textbf{\ipa{kæ˧ʈʂɯ˧}}}}{}
\textcolor{teal}{\mytextsc{nom}} \hspace{4pt} Ton~: M.

Canne à sucre.
\textcolor{brown}{\zh{甘蔗。}}

 Emprunt~: chinois \textcolor{brown}{\zh{甘蔗}}



\paragraph{\hspace{-0.5cm} {\Large \textcolor{darkblue}{\textbf{\ipa{kɤ˧dzi˧}}}}} \hypertarget{k7\string_Mdzi\string_M1}{}
\markboth{\textcolor{darkblue}{\textbf{\ipa{kɤ˧dzi˧}}}}{}
\textcolor{teal}{\mytextsc{verbe}} \hspace{4pt} Ton~: M.

Prendre place (lors d'un repas, d'une cérémonie...).
\textcolor{brown}{\zh{坐下(在饭桌)。}}


\begin{exe}
\sn \textcolor{darkblue}{\textbf{\ipa{ɑ˩ʁo˧-hĩ˧ | kɤ˧dzi˧-ze˧.}}}
\trans Les gens de la famille se sont assis/ont pris place.
\trans \textit{\textcolor{brown}{\zh{家人入座了。}}}
\end{exe}


\paragraph{\hspace{-0.5cm} {\Large \textcolor{darkblue}{\textbf{\ipa{kɤ˧kɤ˩}}}}} \hypertarget{k7\string_Mk7\string_B1}{}
\markboth{\textcolor{darkblue}{\textbf{\ipa{kɤ˧kɤ˩}}}}{}
\textcolor{teal}{\mytextsc{adverbe}} \hspace{4pt} Ton~: L\#.

Proche de, à côté de.
\textcolor{brown}{\zh{挨着(坐……)。}}


\begin{exe}
\sn \textcolor{darkblue}{\textbf{\ipa{(tso˧\textasciitilde{}tso˧) kɤ˧kɤ˩ | tʰi˧-tɕɯ˥}}}
\trans ranger des choses en bon ordre
\trans \textit{\textcolor{brown}{\zh{摆整齐、使均匀,如:一排排挨着}}}
\end{exe}
\begin{exe}
\sn \textcolor{darkblue}{\textbf{\ipa{[Tiger2] kɤ˧kɤ˩ | tʰi˧-se˥}}}
\trans marcher en file indienne
\trans \textit{\textcolor{brown}{\zh{并排走}}}
\end{exe}
\begin{exe}
\sn \textcolor{darkblue}{\textbf{\ipa{kɤ˧kɤ˩ | tʰi˧-dzi˩}}}
\trans être assis les uns à côté des autres, proches les uns des autres
\trans \textit{\textcolor{brown}{\zh{挨着坐}}}
\end{exe}


\paragraph{\hspace{-0.5cm} {\Large \textcolor{darkblue}{\textbf{\ipa{kɤ˧ljɤ˩}}}}} \hypertarget{k7\string_Mlj7\string_B1}{}
\markboth{\textcolor{darkblue}{\textbf{\ipa{kɤ˧ljɤ˩}}}}{}
\textcolor{teal}{\mytextsc{nom}} \hspace{4pt} Ton~: L\#.

Sorgho, gaoliang; céréale dont on se sert pour faire du vin.
\textcolor{brown}{\zh{高粱(汉语借词)。}}

 Emprunt~: chinois \textcolor{brown}{\zh{高粱}}

\textit{Voir~:} \hyperlink{h\{\string_Ml\string_RM\#\string_T1}{\textcolor{darkblue}{\textbf{\ipa{hæ˧ɭɯ\#˥}}}} 

\paragraph{\hspace{-0.5cm} {\Large \textcolor{darkblue}{\textbf{\ipa{kɤ˧mi˧}}} \textsubscript{1}}} \hypertarget{k7\string_Mmi\string_M1}{}
\markboth{\textcolor{darkblue}{\textbf{\ipa{kɤ˧mi˧}}} \textsubscript{1}}{}
\textcolor{teal}{\mytextsc{nom}} \hspace{4pt} Ton~: M.

Faucon femelle.
\textcolor{brown}{\zh{母隼。}}


\begin{exe}
\sn \textcolor{darkblue}{\textbf{\ipa{kɤ˩mi˩-kɤ˩pʰv̩˥}}}
\trans faucon femelle et faucon mâle
\trans \textit{\textcolor{brown}{\zh{母隼与公隼}}}
\end{exe}
 \mytextsc{clf}~: \textcolor{darkblue}{\textbf{\ipa{mi˩}}} 

\paragraph{\hspace{-0.5cm} {\Large \textcolor{darkblue}{\textbf{\ipa{kɤ˧mi˧}}} \textsubscript{2}}} \hypertarget{k7\string_Mmi\string_M2}{}
\markboth{\textcolor{darkblue}{\textbf{\ipa{kɤ˧mi˧}}} \textsubscript{2}}{}
\textcolor{teal}{\mytextsc{nom}} \hspace{4pt} Ton~: M.

Grande jarre; grande bouteille.
\textcolor{brown}{\zh{大坛子,大瓶。}}


 \mytextsc{clf}~: \textcolor{darkblue}{\textbf{\ipa{ɭɯ˧}}} 

\paragraph{\hspace{-0.5cm} {\Large \textcolor{darkblue}{\textbf{\ipa{kɤ˧mv̩˧˥}}}}} \hypertarget{k7\string_Mmv\string_=\string_M\string_T1}{}
\markboth{\textcolor{darkblue}{\textbf{\ipa{kɤ˧mv̩˧˥}}}}{}
\textcolor{teal}{\mytextsc{nom}} \hspace{4pt} Ton~: MH\#.

La montagne Gemu (Yongning).
\textcolor{brown}{\zh{格母山。}}


\begin{exe}
\sn \textcolor{darkblue}{\textbf{\ipa{ɬi˧di˩-kɤ˩mv̩˩}}}
\trans la montagne Gemu de Yongning
\trans \textit{\textcolor{brown}{\zh{永宁格姆山}}}
\end{exe}
\begin{exe}
\sn \textcolor{darkblue}{\textbf{\ipa{kɤ˧mv̩˧-hæ̃˧kʰo˥}}}
\trans ``la princesse Gemu"; autre nom de la montagne Gemu, considérée comme une divinité féminine
\trans \textit{\textcolor{brown}{\zh{格姆公主:格姆山别名(格姆山被看作女神)}}}
\end{exe}
\begin{exe}
\sn \textcolor{darkblue}{\textbf{\ipa{kɤ˧mv̩˧˥, | æ˧ʂæ˧, | ŋwɤ˧hɑ̃˩, | ʂwæ˧gv̩\#˥, | nɑ˩tsʰi˩˥ | -tɕʰɤ˧pɤ˧mi\#˥, | qv̩˧ɻ̍˧-ʈʂʰɑ˧nɑ˥ |}}}
\trans Les six montagnes de Yongning qui portent un nom. Les autres sommets du voisinage n'ont pas une valeur symbolique comparable, et ne portent pas de nom communément utilisé.
\trans \textit{\textcolor{brown}{\zh{永宁地区有固定名字的六座山。其它山,没有重要的象征意义,因此也没有固定名称。}}}
\end{exe}


\paragraph{\hspace{-0.5cm} {\Large \textcolor{darkblue}{\textbf{\ipa{kɤ˧ʈʂɯ˩}}} \textsubscript{1}}} \hypertarget{k7\string_Mt`s`M\string_B1}{}
\markboth{\textcolor{darkblue}{\textbf{\ipa{kɤ˧ʈʂɯ˩}}} \textsubscript{1}}{}
\textcolor{teal}{\mytextsc{verbe}} \hspace{4pt} Ton~: L\#.

Parler, raconter.
\textcolor{brown}{\zh{讲。}}


\begin{exe}
\sn \textcolor{darkblue}{\textbf{\ipa{hĩ˧-ki˧ | tʰɑ˧-kɤ˧ʈʂɯ˩!}}}
\trans il ne faut pas le dire aux gens! / c'est secret!
\trans \textit{\textcolor{brown}{\zh{不要告诉人家!}}}
\end{exe}
\begin{exe}
\sn \textcolor{darkblue}{\textbf{\ipa{kɤ˧-tʰɑ˥-ʈʂɯ˩!}}}
\trans il ne faut pas le dire! / c'est secret!
\trans \textit{\textcolor{brown}{\zh{不要告诉人家!}}}
\end{exe}
\begin{exe}
\sn \textcolor{darkblue}{\textbf{\ipa{njɤ˧-ɳɯ˧ | kɤ˧ʈʂɯ˩-bi˩!}}}
\trans je vais intervenir/je vais dire quelque chose!
\trans \textit{\textcolor{brown}{\zh{我要说一点事情!}}}
\end{exe}
\begin{exe}
\sn \textcolor{darkblue}{\textbf{\ipa{no˧ | kɤ˧ʈʂɯ˩ dʑo˩-ɲi˩!}}}
\trans il faut que tu dises quelque chose!
\trans \textit{\textcolor{brown}{\zh{你得说话啊!}}}
\end{exe}
\begin{exe}
\sn \textcolor{darkblue}{\textbf{\ipa{ʈʂʰɯ˧ | kɤ˧ʈʂɯ˩ | dʑɤ˩˥ | mɤ˧-mv̩˧-sɯ˥! / ʈʂʰɯ˧ | kɤ˧ʈʂɯ˩ dʑɤ˩˥ | mɤ˧-mv̩˧\textasciitilde{}mv̩˧-sɯ˥!}}}
\trans elle ne comprend pas encore grand'chose à ce qu'on dit! / elle ne sait pas encore comprendre ce qu'on dit! (au sujet d'une fillette de moins de 2 ans qui ne parle pas encore)
\trans \textit{\textcolor{brown}{\zh{她还不怎么听得懂话!(关于一个不会说话的两岁小孩)}}}
\end{exe}
\begin{exe}
\sn \textcolor{darkblue}{\textbf{\ipa{tʰɑ˧-kɤ˧ʈʂɯ˩! | hĩ˧ ɳv̩˧ tʰɑ˧-kʰɯ˩!}}}
\trans N'en parle pas! il ne faut pas que les gens le sachent!
\trans \textit{\textcolor{brown}{\zh{不要告诉(人家)!别让人家知道!}}}
\end{exe}
\begin{exe}
\sn \textcolor{darkblue}{\textbf{\ipa{hĩ˧-ki˧ | kɤ˧-mɤ˧-ʈʂɯ˩}}}
\trans (faire quelque chose) en cachette, sans le dire à personne
\trans \textit{\textcolor{brown}{\zh{不跟人家说(自己做什么事)}}}
\end{exe}
\begin{exe}
\sn \textcolor{darkblue}{\textbf{\ipa{kɤ˧ʈʂɯ˩ ɲi˩}}}
\trans sage (au sujet d'un enfant) (littéralement: qui écoute ce qu'on lui dit)
\trans \textit{\textcolor{brown}{\zh{听话,乖(来形容一个孩子)}}}
\end{exe}
\textit{Voir~:} \hyperlink{k7\string_Mt`s`M\string_B2}{\textcolor{darkblue}{\textbf{\ipa{kɤ˧ʈʂɯ˩}}} \textsubscript{2}} 

\paragraph{\hspace{-0.5cm} {\Large \textcolor{darkblue}{\textbf{\ipa{kɤ˧ʈʂɯ˩}}} \textsubscript{2}}} \hypertarget{k7\string_Mt`s`M\string_B2}{}
\markboth{\textcolor{darkblue}{\textbf{\ipa{kɤ˧ʈʂɯ˩}}} \textsubscript{2}}{}
\textcolor{teal}{\mytextsc{nom}} \hspace{4pt} Ton~: L\#.

Parole.
\textcolor{brown}{\zh{话。}}


\begin{exe}
\sn \textcolor{darkblue}{\textbf{\ipa{kɤ˧ʈʂɯ˩ ʝi˩}}}
\trans promettre; aussi: jurer ses grands dieux, prêter serment devant les Dieux: lorsque deux personnes avaient un différend qu'elles ne parvenaient pas à trancher, elles allaient raconter chacune sa version des faits devant les Dieux (au monastère); ceux-ci punissaient ensuite le coupable (par des calamités qui frappaient la famille du coupable).
\trans \textit{\textcolor{brown}{\zh{答应,誓、发誓。两个人发生矛盾的时候,如果无法协调,他们会去大寺,在神像前讲述他们各自的观点,发誓他们自己讲的是真的。神会惩罚说谎的人(他家会有祸害)。}}}
\end{exe}
\begin{exe}
\sn \textcolor{darkblue}{\textbf{\ipa{ʈʂʰɯ˧ | kɤ˧ʈʂɯ˩-ʝi˩}}}
\trans il/elle promet
\trans \textit{\textcolor{brown}{\zh{他答应}}}
\end{exe}
\begin{exe}
\sn \textcolor{darkblue}{\textbf{\ipa{ʈʂʰɯ˧ | kɤ˧ʈʂɯ˩ | mɤ˧-ʝi˥!}}}
\trans il n'a pas promis!
\trans \textit{\textcolor{brown}{\zh{他没有答应!}}}
\end{exe}
\begin{exe}
\sn \textcolor{darkblue}{\textbf{\ipa{hĩ˧-kɤ˧ʈʂɯ˥ ɲi˩}}}
\trans écouter les conseils d'autrui, prêter attention à la parole d'autrui, écouter les bons conseils (attitude jugée positive et souhaitable par la consultante)
\trans \textit{\textcolor{brown}{\zh{听别人的建议、把别人的话当回事}}}
\end{exe}
\begin{exe}
\sn \textcolor{darkblue}{\textbf{\ipa{hĩ˧-kɤ˧ʈʂɯ˥ | le˧-ɲi˥}}}
\trans même sens
\trans \textit{\textcolor{brown}{\zh{同上}}}
\end{exe}
\begin{exe}
\sn \textcolor{darkblue}{\textbf{\ipa{hĩ˧-kɤ˧ʈʂɯ˥ | mɤ˧-ɲi˥}}}
\trans ne pas écouter les bons conseils, ne pas prêter attention à ce qu'on vous dit
\trans \textit{\textcolor{brown}{\zh{听不进去别人的意见与建议}}}
\end{exe}
 \mytextsc{clf}~: \textcolor{darkblue}{\textbf{\ipa{kʰwɤ˥}}} \textit{Voir~:} \hyperlink{k7\string_Mt`s`M\string_B1}{\textcolor{darkblue}{\textbf{\ipa{kɤ˧ʈʂɯ˩}}} \textsubscript{1}} 

\paragraph{\hspace{-0.5cm} {\Large \textcolor{darkblue}{\textbf{\ipa{kɤ˧v̩\#˥}}}}} \hypertarget{k7\string_Mv\string_=\#\string_T1}{}
\markboth{\textcolor{darkblue}{\textbf{\ipa{kɤ˧v̩\#˥}}}}{}
\textcolor{teal}{\mytextsc{nom}} \hspace{4pt} Ton~: \#H.

Amulette.
\textcolor{brown}{\zh{护符,护身符。}}


 \mytextsc{clf}~: \textcolor{darkblue}{\textbf{\ipa{ɭɯ˧}}} 

\paragraph{\hspace{-0.5cm} {\Large \textcolor{darkblue}{\textbf{\ipa{kɤ˧wɤ\#˥}}}}} \hypertarget{k7\string_Mw7\#\string_T1}{}
\markboth{\textcolor{darkblue}{\textbf{\ipa{kɤ˧wɤ\#˥}}}}{}
\textcolor{teal}{\mytextsc{nom}} \hspace{4pt} Ton~: \#H.

Destinée, affinité prédestinée.
\textcolor{brown}{\zh{缘分、共同命运。}}


\begin{exe}
\sn \textcolor{darkblue}{\textbf{\ipa{kɤ˧wɤ˧-ljɤ˧˥}}}
\trans avoir une affinité prédestinée, avoir un destin commun
\trans \textit{\textcolor{brown}{\zh{有缘分、有共同命运}}}
\end{exe}


\paragraph{\hspace{-0.5cm} {\Large \textcolor{darkblue}{\textbf{\ipa{kɤ˧zo\#˥}}}}} \hypertarget{k7\string_Mzo\#\string_T1}{}
\markboth{\textcolor{darkblue}{\textbf{\ipa{kɤ˧zo\#˥}}}}{}
\textcolor{teal}{\mytextsc{nom}} \hspace{4pt} Ton~: \#H.

Prénom masculin.
\textcolor{brown}{\zh{男性名字。}}




\paragraph{\hspace{-0.5cm} {\Large \textcolor{darkblue}{\textbf{\ipa{kɤ˧zo˧-tsʰɯ˩ɻ̍˩}}}}} \hypertarget{k7\string_Mzo\string_M-ts\string_hM\string_Br£`̍\string_B1}{}
\markboth{\textcolor{darkblue}{\textbf{\ipa{kɤ˧zo˧-tsʰɯ˩ɻ̍˩}}}}{}
\textcolor{teal}{\mytextsc{nom}} \hspace{4pt} Ton~: \mytextsc{L}.

Prénom masculin.
\textcolor{brown}{\zh{男性名字。}}




\paragraph{\hspace{-0.5cm} {\Large \textcolor{darkblue}{\textbf{\ipa{kɤ˩}}}}} \hypertarget{k7\string_B1}{}
\markboth{\textcolor{darkblue}{\textbf{\ipa{kɤ˩}}}}{}
\textcolor{teal}{\mytextsc{nom}} \hspace{4pt} Ton~: L.

Bouteille.
\textcolor{brown}{\zh{瓶子。}}


\begin{exe}
\sn \textcolor{darkblue}{\textbf{\ipa{ʐɯ˧-kɤ˩}}}
\trans bouteille d'alcool
\trans \textit{\textcolor{brown}{\zh{酒瓶}}}
\end{exe}
\begin{exe}
\sn \textcolor{darkblue}{\textbf{\ipa{ʐɯ˧ ɖɯ˧-kɤ˩}}}
\trans une bouteille d'alcool
\trans \textit{\textcolor{brown}{\zh{一瓶酒}}}
\end{exe}
 \mytextsc{clf}~: \textcolor{darkblue}{\textbf{\ipa{ɭɯ˧}}} 

\paragraph{\hspace{-0.5cm} {\Large \textcolor{darkblue}{\textbf{\ipa{kɤ˩\textsubscript{a}}}}}} \hypertarget{k7\string_Ba1}{}
\markboth{\textcolor{darkblue}{\textbf{\ipa{kɤ˩\textsubscript{a}}}}}{}
\textcolor{teal}{\mytextsc{classificateur}} \hspace{4pt} Ton~: L\textsubscript{a}.

Classificateur des bouteilles.
\textcolor{brown}{\zh{量词:瓶。}}


\begin{exe}
\sn \textcolor{darkblue}{\textbf{\ipa{kɤ˩zo˩˥}}}
\trans petite bouteille
\trans \textit{\textcolor{brown}{\zh{一小瓶}}}
\end{exe}
\begin{exe}
\sn \textcolor{darkblue}{\textbf{\ipa{ʈʂʰɯ˧-kɤ˥}}}
\trans \mytextsc{dem} \string_ (tone: H\# / H\$)
\trans \textit{\textcolor{brown}{\zh{\mytextsc{指示代词} \string_}}}
\end{exe}


\paragraph{\hspace{-0.5cm} {\Large \textcolor{darkblue}{\textbf{\ipa{kɤ˩\textasciitilde{}kɤ˧˥}}}}} \hypertarget{k7\string_B~k7\string_M\string_T1}{}
\markboth{\textcolor{darkblue}{\textbf{\ipa{kɤ˩\textasciitilde{}kɤ˧˥}}}}{}
\textcolor{teal}{\mytextsc{verbe}} \hspace{4pt} Ton~: MH.

Tapoter.
\textcolor{brown}{\zh{敲、拍。}}


\begin{exe}
\sn \textcolor{darkblue}{\textbf{\ipa{kʰi˧ kɤ˥\textasciitilde{}kɤ˩}}}
\trans frapper à la porte
\trans \textit{\textcolor{brown}{\zh{敲门}}}
\end{exe}
\begin{exe}
\sn \textcolor{darkblue}{\textbf{\ipa{njɤ˧-ɳɯ˧ | no˧ | kɤ˩\textasciitilde{}kɤ˧-bi˥!}}}
\trans Je vais te donner une tape/une fessée! (Menace d'un adulte à un enfant)
\trans \textit{\textcolor{brown}{\zh{我要打你屁股了!(大人对孩子说)}}}
\end{exe}
\begin{exe}
\sn \textcolor{darkblue}{\textbf{\ipa{ʈʂo˧tsɯ˥ kɤ˩\textasciitilde{}kɤ˩ (-ze˩/-bi˩)}}}
\trans heurter la table, taper sur la table
\trans \textit{\textcolor{brown}{\zh{拍拍桌子}}}
\end{exe}
\begin{exe}
\sn \textcolor{darkblue}{\textbf{\ipa{gv̩˧dv̩˧ kɤ˧\textasciitilde{}kɤ˩}}}
\trans tapoter sur le dos de quelqu'un (pour soulager un mal de dos)
\trans \textit{\textcolor{brown}{\zh{敲敲背}}}
\end{exe}
\textit{Voir~:} \hyperlink{k7\string_M\string_T1}{\textcolor{darkblue}{\textbf{\ipa{kɤ˧˥}}}} 

\paragraph{\hspace{-0.5cm} {\Large \textcolor{darkblue}{\textbf{\ipa{kɤ˩lo˧˥}}}}} \hypertarget{k7\string_Blo\string_M\string_T1}{}
\markboth{\textcolor{darkblue}{\textbf{\ipa{kɤ˩lo˧˥}}}}{}
\textcolor{teal}{\mytextsc{nom}} \hspace{4pt} Ton~: LM+MH\#.

Branche.
\textcolor{brown}{\zh{树枝。}}


\begin{exe}
\sn \textcolor{darkblue}{\textbf{\ipa{si˧dzi˩-kɤ˩lo˩}}}
\trans branche d'arbre
\trans \textit{\textcolor{brown}{\zh{树枝}}}
\end{exe}
\begin{exe}
\sn \textcolor{darkblue}{\textbf{\ipa{si˧-kɤ˥lo˩}}}
\trans idem
\trans \textit{\textcolor{brown}{\zh{同上}}}
\end{exe}
 \mytextsc{clf}~: \textcolor{darkblue}{\textbf{\ipa{kɤ˧˥}}} 

\paragraph{\hspace{-0.5cm} {\Large \textcolor{darkblue}{\textbf{\ipa{kɤ˩-nɑ˧mi˧}}}}} \hypertarget{k7\string_B-nA\string_Mmi\string_M1}{}
\markboth{\textcolor{darkblue}{\textbf{\ipa{kɤ˩-nɑ˧mi˧}}}}{}
\textcolor{teal}{\mytextsc{nom}} \hspace{4pt} Ton~: L-.

Aigle.
\textcolor{brown}{\zh{老鹰。}}


 \mytextsc{clf}~: \textcolor{darkblue}{\textbf{\ipa{mi˩}}} 

\paragraph{\hspace{-0.5cm} {\Large \textcolor{darkblue}{\textbf{\ipa{kɤ˩pʰv̩˩}}}}} \hypertarget{k7\string_Bp\string_hv\string_=\string_B1}{}
\markboth{\textcolor{darkblue}{\textbf{\ipa{kɤ˩pʰv̩˩}}}}{}
\textcolor{teal}{\mytextsc{nom}} \hspace{4pt} Ton~: L.

Faucon mâle.
\textcolor{brown}{\zh{公隼。}}


\begin{exe}
\sn \textcolor{darkblue}{\textbf{\ipa{kɤ˩pʰv̩˩-kɤ˩mi˥}}}
\trans faucon mâle et faucon femelle
\trans \textit{\textcolor{brown}{\zh{公隼与母隼}}}
\end{exe}
 \mytextsc{clf}~: \textcolor{darkblue}{\textbf{\ipa{mi˩}}} 

\paragraph{\hspace{-0.5cm} {\Large \textcolor{darkblue}{\textbf{\ipa{kɤ˩-tjɤ˧ljɤ\#˥}}}}} \hypertarget{k7\string_B-tj7\string_Mlj7\#\string_T1}{}
\markboth{\textcolor{darkblue}{\textbf{\ipa{kɤ˩-tjɤ˧ljɤ\#˥}}}}{}
\textcolor{teal}{\mytextsc{nom}} \hspace{4pt} Ton~: L-\#H.

Clochette s'accrochant autour du cou (ex.: clochette d'un cheval).
\textcolor{brown}{\zh{铃铛。}}


 \mytextsc{clf}~: \textcolor{darkblue}{\textbf{\ipa{ɭɯ˧}}} 

\paragraph{\hspace{-0.5cm} {\Large \textcolor{darkblue}{\textbf{\ipa{kɤ˩zo˩}}} \textsubscript{1}}} \hypertarget{k7\string_Bzo\string_B1}{}
\markboth{\textcolor{darkblue}{\textbf{\ipa{kɤ˩zo˩}}} \textsubscript{1}}{}
\textcolor{teal}{\mytextsc{nom}} \hspace{4pt} Ton~: L.

Bébé faucon.
\textcolor{brown}{\zh{小隼。}}




\paragraph{\hspace{-0.5cm} {\Large \textcolor{darkblue}{\textbf{\ipa{kɤ˩zo˩}}} \textsubscript{2}}} \hypertarget{k7\string_Bzo\string_B2}{}
\markboth{\textcolor{darkblue}{\textbf{\ipa{kɤ˩zo˩}}} \textsubscript{2}}{}
\textcolor{teal}{\mytextsc{nom}} \hspace{4pt} Ton~: L.

Petite bouteille.
\textcolor{brown}{\zh{小瓶子。}}




\paragraph{\hspace{-0.5cm} {\Large \textcolor{darkblue}{\textbf{\ipa{}}}}} \hypertarget{k7\string_Bxxxx1}{}
\markboth{\textcolor{darkblue}{\textbf{\ipa{}}}}{}
\textcolor{teal}{\mytextsc{verbe}} \hspace{4pt} Ton~: xxxx vérifier : L\textsubscript{a}? L\textsubscript{b}?.

Scier (du bois).
\textcolor{brown}{\zh{锯(木头)。}}




\paragraph{\hspace{-0.5cm} {\Large \textcolor{darkblue}{\textbf{\ipa{kɤ˧˥}}}}} \hypertarget{k7\string_M\string_T1}{}
\markboth{\textcolor{darkblue}{\textbf{\ipa{kɤ˧˥}}}}{}
\textcolor{teal}{\mytextsc{verbe}} \hspace{4pt} Ton~: MH.

Frapper à la porte, heurter à la porte.
\textcolor{brown}{\zh{敲门。}}


\begin{exe}
\sn \textcolor{darkblue}{\textbf{\ipa{tʰi˧-kɤ˧˥}}}
\trans \mytextsc{dur}
\trans \textit{\textcolor{brown}{\zh{\mytextsc{dur}}}}
\end{exe}
\textit{Voir~:} \textcolor{darkblue}{\textbf{\ipa{kɤ˩kɤ˧˥}}} 

\paragraph{\hspace{-0.5cm} {\Large \textcolor{darkblue}{\textbf{\ipa{kɤ˧˥\textsubscript{a}}}} \textsubscript{1}}} \hypertarget{k7\string_M\string_Ta1}{}
\markboth{\textcolor{darkblue}{\textbf{\ipa{kɤ˧˥\textsubscript{a}}}} \textsubscript{1}}{}
\textcolor{teal}{\mytextsc{classificateur}} \hspace{4pt} Ton~: MH\textsubscript{a}.

Classificateur des bâtons.
\textcolor{brown}{\zh{量词:棍子、树枝(一根)。}}


\begin{exe}
\sn \textcolor{darkblue}{\textbf{\ipa{si˧-kɤ˧˥ | ɖɯ˧-kɤ˧˥}}}
\trans une branche (d'arbre)
\trans \textit{\textcolor{brown}{\zh{一根树枝}}}
\end{exe}


\paragraph{\hspace{-0.5cm} {\Large \textcolor{darkblue}{\textbf{\ipa{kɤ˧˥\textsubscript{a}}}} \textsubscript{2}}} \hypertarget{k7\string_M\string_Ta2}{}
\markboth{\textcolor{darkblue}{\textbf{\ipa{kɤ˧˥\textsubscript{a}}}} \textsubscript{2}}{}
\textcolor{teal}{\mytextsc{classificateur}} \hspace{4pt} Ton~: MH\textsubscript{a}.

Une étendue de terre.
\textcolor{brown}{\zh{量词:地(一片)。}}




\paragraph{\hspace{-0.5cm} {\Large \textcolor{darkblue}{\textbf{\ipa{kɤ˧˥tʰɑ˩}}}}} \hypertarget{k7\string_M\string_Tt\string_hA\string_B1}{}
\markboth{\textcolor{darkblue}{\textbf{\ipa{kɤ˧˥tʰɑ˩}}}}{}
\textcolor{teal}{\mytextsc{nom}} \hspace{4pt} Ton~: MH+L.

Nom de clan/famille étendue. Deux familles portent ce nom à Yongning. C'est l'un des trois premiers clans à s'être établis à proximité du monastère de Yongning, les deux autres étant \textcolor{darkblue}{\textbf{\ipa{/ə˧lɑ˧/}}} et \textcolor{darkblue}{\textbf{\ipa{/lɑ˧tʰɑ˧mi˥\$/}}}.
\textcolor{brown}{\zh{一个姓。这个姓,永宁有两家。}}


\begin{exe}
\sn \textcolor{darkblue}{\textbf{\ipa{kɤ˧˥tʰɑ˩=ɻ̍˩}}}
\trans le clan \textcolor{darkblue}{\textbf{\ipa{/kɤ˧˥tʰɑ˩/}}}
\trans \textit{\textcolor{brown}{\zh{\textcolor{darkblue}{\textbf{\ipa{/kɤ˧˥tʰɑ˩/}}}家族}}}
\end{exe}


\paragraph{\hspace{-0.5cm} {\Large \textcolor{darkblue}{\textbf{\ipa{kɤ˩˧}}}}} \hypertarget{k7\string_B\string_M1}{}
\markboth{\textcolor{darkblue}{\textbf{\ipa{kɤ˩˧}}}}{}
\textcolor{teal}{\mytextsc{nom}} \hspace{4pt} Ton~: LM.

Buse, faucon.
\textcolor{brown}{\zh{隼、“小鹰”。}}


\begin{exe}
\sn \textcolor{darkblue}{\textbf{\ipa{kɤ˩ hwæ˧-ze˧}}}
\trans ...a acheté (un) faucon
\trans \textit{\textcolor{brown}{\zh{买了隼}}}
\end{exe}
\begin{exe}
\sn \textcolor{darkblue}{\textbf{\ipa{kɤ˩ dzɯ˧-ze˩}}}
\trans ...a mangé (un) faucon
\trans \textit{\textcolor{brown}{\zh{吃了隼}}}
\end{exe}
 \mytextsc{clf}~: \textcolor{darkblue}{\textbf{\ipa{mi˩}}} 

\paragraph{\hspace{-0.5cm} {\Large \textcolor{darkblue}{\textbf{\ipa{ki˥\textsubscript{a}}}}}} \hypertarget{ki\string_Ta1}{}
\markboth{\textcolor{darkblue}{\textbf{\ipa{ki˥\textsubscript{a}}}}}{}
\textcolor{teal}{\mytextsc{classificateur}} \hspace{4pt} Ton~: H\textsubscript{a}.

En association avec le numéral 'un', ce classificateur signifie 'ensemble'.
\textcolor{brown}{\zh{量词:加上数词‘一’,这个量词表示‘一起’。}}


\begin{exe}
\sn \textcolor{darkblue}{\textbf{\ipa{ɖɯ˧-ki˥}}}
\trans ensemble
\trans \textit{\textcolor{brown}{\zh{一起(共事)}}}
\end{exe}
\begin{exe}
\sn \textcolor{darkblue}{\textbf{\ipa{ɖɯ˧-ki˧ tʰv̩˧}}}
\trans arriver ensemble/en même temps
\trans \textit{\textcolor{brown}{\zh{同时到达}}}
\end{exe}
\begin{exe}
\sn \textcolor{darkblue}{\textbf{\ipa{ɖɯ˧-ʝi˧-ɳɯ˧ tsʰɯ˧˥, | ɖɯ˧-ki˧ tʰv̩˧!}}}
\trans Venant du même endroit, (on) arrive ensemble!
\trans \textit{\textcolor{brown}{\zh{从一个地方,一起到!}}}
\end{exe}
\begin{exe}
\sn \textcolor{darkblue}{\textbf{\ipa{ɖɯ˧-ki˧ dzi˧˥}}}
\trans habiter ensemble
\trans \textit{\textcolor{brown}{\zh{住在一起}}}
\end{exe}


\paragraph{\hspace{-0.5cm} {\Large \textcolor{darkblue}{\textbf{\ipa{‑ki˧}}}}} \hypertarget{‑ki\string_M1}{}
\markboth{\textcolor{darkblue}{\textbf{\ipa{‑ki˧}}}}{}
\textcolor{teal}{\mytextsc{suffixe}} \hspace{4pt} Ton~: M.

Datif/allatif.
\textcolor{brown}{\zh{对\mytextsc{与格}/向格。}}


\begin{exe}
\sn \textcolor{darkblue}{\textbf{\ipa{ʈʂʰɯ˧-ki˧ ʐwɤ˧˥}}}
\trans lui dire, lui parler
\trans \textit{\textcolor{brown}{\zh{给他讲}}}
\end{exe}
\begin{exe}
\sn \textcolor{darkblue}{\textbf{\ipa{ʈʂʰɯ˧-ki˧ ʐwɤ˧-bi˥}}}
\trans idem+futur immédiat
\trans \textit{\textcolor{brown}{\zh{要给他讲}}}
\end{exe}
\begin{exe}
\sn \textcolor{darkblue}{\textbf{\ipa{ə˧mɑ˧-ɳɯ˧ | njɤ˧-ki˧ | nɑ˩-ʐwɤ˧ so˩!}}}
\trans Ama m'enseigne la langue na!
\trans \textit{\textcolor{brown}{\zh{阿妈教我摩梭话!}}}
\end{exe}


\paragraph{\hspace{-0.5cm} {\Large \textcolor{darkblue}{\textbf{\ipa{ki˧\textsubscript{a}}}}}} \hypertarget{ki\string_Ma1}{}
\markboth{\textcolor{darkblue}{\textbf{\ipa{ki˧\textsubscript{a}}}}}{}
\textcolor{teal}{\mytextsc{verbe}} \hspace{4pt} Ton~: M\textsubscript{a}.

Donner, passer, transmettre.
\textcolor{brown}{\zh{给、传、献给、发(工资),嫁给。}}


\begin{exe}
\sn \textcolor{darkblue}{\textbf{\ipa{ki˧\textasciitilde{}ki˩}}}
\trans \mytextsc{red}
\trans \textit{\textcolor{brown}{\zh{重叠}}}
\end{exe}
\begin{exe}
\sn \textcolor{darkblue}{\textbf{\ipa{tso˧\textasciitilde{}tso˧-ki˩}}}
\trans donner des choses
\trans \textit{\textcolor{brown}{\zh{给东西}}}
\end{exe}
\begin{exe}
\sn \textcolor{darkblue}{\textbf{\ipa{tso˧\textasciitilde{}tso˧ ki˧\textasciitilde{}ki˥}}}
\trans donner des choses
\trans \textit{\textcolor{brown}{\zh{给东西}}}
\end{exe}
\begin{exe}
\sn \textcolor{darkblue}{\textbf{\ipa{hĩ˧-ki˧ ki˩}}}
\trans 1. donner à quelqu'un. 2. se donner en mariage à quelqu'un (pour une femme)
\trans \textit{\textcolor{brown}{\zh{1.许配给人家。2.嫁给人}}}
\end{exe}
\begin{exe}
\sn \textcolor{darkblue}{\textbf{\ipa{hĩ˧-ki˧ | ɖwæ˧˥ | tʰi˧-ki˧}}}
\trans être généreux
\trans \textit{\textcolor{brown}{\zh{大方}}}
\end{exe}
\begin{exe}
\sn \textcolor{darkblue}{\textbf{\ipa{hĩ˧-ki˧ ki˩ fv̩˩}}}
\trans qui aime faire des cadeaux, qui aime donner des choses aux gens
\trans \textit{\textcolor{brown}{\zh{爱送礼,爱给别人送东西}}}
\end{exe}
\begin{exe}
\sn \textcolor{darkblue}{\textbf{\ipa{pʰɤ˧bɤ˧ ki˧ (-bi˧)}}}
\trans donner des cadeaux
\trans \textit{\textcolor{brown}{\zh{送礼物}}}
\end{exe}
\begin{exe}
\sn \textcolor{darkblue}{\textbf{\ipa{hɑ˧ ki˩}}}
\trans nourrir, donner à manger
\trans \textit{\textcolor{brown}{\zh{喂饭}}}
\end{exe}


\paragraph{\hspace{-0.5cm} {\Large \textcolor{darkblue}{\textbf{\ipa{ki˧li˥}}}}} \hypertarget{ki\string_Mli\string_T1}{}
\markboth{\textcolor{darkblue}{\textbf{\ipa{ki˧li˥}}}}{}
\textcolor{teal}{\mytextsc{adverbe}} \hspace{4pt} Ton~: H\#.

En désordre (formulation expressive, quasi-onomatopéique).
\textcolor{brown}{\zh{乱七八糟。}}




\paragraph{\hspace{-0.5cm} {\Large \textcolor{darkblue}{\textbf{\ipa{ki˧zo\#˥}}}}} \hypertarget{ki\string_Mzo\#\string_T1}{}
\markboth{\textcolor{darkblue}{\textbf{\ipa{ki˧zo\#˥}}}}{}
\textcolor{teal}{\mytextsc{nom}} \hspace{4pt} Ton~: \#H.

Prénom unisexe: prénom utilisé pour les deux sexes.
\textcolor{brown}{\zh{男女通用名。}}




\paragraph{\hspace{-0.5cm} {\Large \textcolor{darkblue}{\textbf{\ipa{ki˧zo˧-ɖɯ˩mɑ˩}}}}} \hypertarget{ki\string_Mzo\string_M-d`M\string_BmA\string_B1}{}
\markboth{\textcolor{darkblue}{\textbf{\ipa{ki˧zo˧-ɖɯ˩mɑ˩}}}}{}
\textcolor{teal}{\mytextsc{nom}} \hspace{4pt} Ton~: \mytextsc{L}.

Prénom féminin.
\textcolor{brown}{\zh{女性名字。}}




\paragraph{\hspace{-0.5cm} {\Large \textcolor{darkblue}{\textbf{\ipa{ki˧zo˧-ɬɑ˩mv̩˩}}}}} \hypertarget{ki\string_Mzo\string_M-KA\string_Bmv\string_=\string_B1}{}
\markboth{\textcolor{darkblue}{\textbf{\ipa{ki˧zo˧-ɬɑ˩mv̩˩}}}}{}
\textcolor{teal}{\mytextsc{nom}} \hspace{4pt} Ton~: \mytextsc{L}.

Prénom féminin.
\textcolor{brown}{\zh{女性名字。}}




\paragraph{\hspace{-0.5cm} {\Large \textcolor{darkblue}{\textbf{\ipa{ki˩\textsubscript{a}}}}}} \hypertarget{ki\string_Ba1}{}
\markboth{\textcolor{darkblue}{\textbf{\ipa{ki˩\textsubscript{a}}}}}{}
\textcolor{teal}{\mytextsc{verbe}} \hspace{4pt} Ton~: L\textsubscript{a}.

Enfiler, porter, mettre (une robe, un pantalon...).
\textcolor{brown}{\zh{穿上(裤子、袜子、鞋子)。}}


\begin{exe}
\sn \textcolor{darkblue}{\textbf{\ipa{ɬi˧qʰwɤ˩ | ɖɯ˧-ɭɯ˧ ki˩}}}
\trans enfiler un pantalon
\trans \textit{\textcolor{brown}{\zh{穿上裤子}}}
\end{exe}
\begin{exe}
\sn \textcolor{darkblue}{\textbf{\ipa{dzɑ˩qʰwɤ˩˥ | ɖɯ˧-dzi˧ ki˩}}}
\trans enfiler une paire de chaussures
\trans \textit{\textcolor{brown}{\zh{穿上一双鞋}}}
\end{exe}
\begin{exe}
\sn \textcolor{darkblue}{\textbf{\ipa{ʈʰæ˩ ki˩˥}}}
\trans porter la jupe; nom du rituel de passage à l'âge adulte (à treize ans révolus) pour les jeunes filles
\trans \textit{\textcolor{brown}{\zh{“穿裙”:这是成年礼的名称(穿裙礼:女孩满13岁,即为成人)}}}
\end{exe}
\begin{exe}
\sn \textcolor{darkblue}{\textbf{\ipa{ɬi˧ ki˥}}}
\trans porter le pantalon; nom du rituel de passage à l'âge adulte (à treize ans révolus) pour les jeunes gens
\trans \textit{\textcolor{brown}{\zh{“穿裤”:这是成年礼的名称(穿裤礼:男孩满了13岁,即为成人)}}}
\end{exe}


\paragraph{\hspace{-0.5cm} {\Large \textcolor{darkblue}{\textbf{\ipa{ki˩mi˧}}}}} \hypertarget{ki\string_Bmi\string_M1}{}
\markboth{\textcolor{darkblue}{\textbf{\ipa{ki˩mi˧}}}}{}
\textcolor{teal}{\mytextsc{nom}} \hspace{4pt} Ton~: LM.

Grosse mouche, dont les larves sont redoutées.
\textcolor{brown}{\zh{绿头苍蝇。}}


 \mytextsc{clf}~: \textcolor{darkblue}{\textbf{\ipa{mi˩}}} 

\paragraph{\hspace{-0.5cm} {\Large \textcolor{darkblue}{\textbf{\ipa{ki˩tɑ\#˥}}}}} \hypertarget{ki\string_BtA\#\string_T1}{}
\markboth{\textcolor{darkblue}{\textbf{\ipa{ki˩tɑ\#˥}}}}{}
\textcolor{teal}{\mytextsc{nom}} \hspace{4pt} Ton~: LM+\#H.

Sac de cuir et de lin, dans lequel on plaçait ce que possédait la maisonnée: or, argent… qu'on enterrait quelque part dans la maison, pour se prémunir contre les voleurs; les matières choisies, cuir et lin, se conservaient très longtemps; le sac avait quatre ou cinq épaisseurs de tissu, pour le rendre plus solide.
\textcolor{brown}{\zh{皮袋,来装家里的财物:金币、银币……这个皮袋,埋在房子里的一个保密地方,防贼。为了让它很结实,袋子有四、五层麻布内衬。可以保存很长时间。}}


\begin{exe}
\sn \textcolor{darkblue}{\textbf{\ipa{æ˧-tse˥pʰæ˩ | ɖɯ˧-ki˩tɑ˩}}}
\trans un sac de pièces de cuivre
\trans \textit{\textcolor{brown}{\zh{一袋铜币}}}
\end{exe}


\paragraph{\hspace{-0.5cm} {\Large \textcolor{darkblue}{\textbf{\ipa{ki˩ti\#˥}}}}} \hypertarget{ki\string_Bti\#\string_T1}{}
\markboth{\textcolor{darkblue}{\textbf{\ipa{ki˩ti\#˥}}}}{}
\textcolor{teal}{\mytextsc{nom}} \hspace{4pt} Ton~: LM+\#H.

Ceinture en cuir.
\textcolor{brown}{\zh{皮腰带。}}


 \mytextsc{clf}~: \textcolor{darkblue}{\textbf{\ipa{kʰɯ˩}}} 

\paragraph{\hspace{-0.5cm} {\Large \textcolor{darkblue}{\textbf{\ipa{ko˥}}}}} \hypertarget{ko\string_T1}{}
\markboth{\textcolor{darkblue}{\textbf{\ipa{ko˥}}}}{}
\textcolor{teal}{\mytextsc{nom}} \hspace{4pt} Ton~: \#H.

Colline, petite montagne.
\textcolor{brown}{\zh{小山。}}


 \mytextsc{clf}~: \textcolor{darkblue}{\textbf{\ipa{ɭɯ˧}}} 

\paragraph{\hspace{-0.5cm} {\Large \textcolor{darkblue}{\textbf{\ipa{ko˧\textsubscript{a}}}}}} \hypertarget{ko\string_Ma1}{}
\markboth{\textcolor{darkblue}{\textbf{\ipa{ko˧\textsubscript{a}}}}}{}
\textcolor{teal}{\mytextsc{classificateur}} \hspace{4pt} Ton~: M\textsubscript{a}.

Classificateur des petits objets, tels que des cigarettes.
\textcolor{brown}{\zh{量词:小东西,例如烟(一只)。}}




\paragraph{\hspace{-0.5cm} {\Large \textcolor{darkblue}{\textbf{\ipa{ko˧ɖæ\#˥}}}}} \hypertarget{ko\string_Md`\{\#\string_T1}{}
\markboth{\textcolor{darkblue}{\textbf{\ipa{ko˧ɖæ\#˥}}}}{}
\textcolor{teal}{\mytextsc{nom}} \hspace{4pt} Ton~: \#H.

Statue de bouddha.
\textcolor{brown}{\zh{佛像。}}

 Emprunt~: tibétain sku-vdra(sku-'dra)

\begin{exe}
\sn \textcolor{darkblue}{\textbf{\ipa{ko˧ɖæ˧-zo˧}}}
\trans statuette du Bouddha
\trans \textit{\textcolor{brown}{\zh{小佛像}}}
\end{exe}
 \mytextsc{clf}~: \textcolor{darkblue}{\textbf{\ipa{ɭɯ˧}}} 

\paragraph{\hspace{-0.5cm} {\Large \textcolor{darkblue}{\textbf{\ipa{ko˧li\#˥}}}}} \hypertarget{ko\string_Mli\#\string_T1}{}
\markboth{\textcolor{darkblue}{\textbf{\ipa{ko˧li\#˥}}}}{}
\textcolor{teal}{\mytextsc{nom}} \hspace{4pt} Ton~: \#H.

Soufflet à bouche: un tube dans lequel on souffle pour attiser le feu.
\textcolor{brown}{\zh{吹火筒,用来吹火的小管子。}}


 \mytextsc{clf}~: \textcolor{darkblue}{\textbf{\ipa{ɭɯ˧}}} 

\paragraph{\hspace{-0.5cm} {\Large \textcolor{darkblue}{\textbf{\ipa{ko˧no˧-ʁo\#˥}}}}} \hypertarget{ko\string_Mno\string_M-Ro\#\string_T1}{}
\markboth{\textcolor{darkblue}{\textbf{\ipa{ko˧no˧-ʁo\#˥}}}}{}
\textcolor{teal}{\mytextsc{nom}} \hspace{4pt} Ton~: \#H.

Crête, ligne de faîte (en montagne).
\textcolor{brown}{\zh{山梁。}}


 \mytextsc{clf}~: \textcolor{darkblue}{\textbf{\ipa{kʰwɤ˥}}} 

\paragraph{\hspace{-0.5cm} {\Large \textcolor{darkblue}{\textbf{\ipa{ko˧sɯ\#˥}}}}} \hypertarget{ko\string_MsM\#\string_T1}{}
\markboth{\textcolor{darkblue}{\textbf{\ipa{ko˧sɯ\#˥}}}}{}
\textcolor{teal}{\mytextsc{nom}} \hspace{4pt} Ton~: \#H.

Boutique.
\textcolor{brown}{\zh{商店、小卖部(汉语借词:公司)。}}

 Emprunt~: chinois \textcolor{brown}{\zh{公司}}



\paragraph{\hspace{-0.5cm} {\Large \textcolor{darkblue}{\textbf{\ipa{ko˩\textsubscript{a}}}}}} \hypertarget{ko\string_Ba1}{}
\markboth{\textcolor{darkblue}{\textbf{\ipa{ko˩\textsubscript{a}}}}}{}
\textcolor{teal}{\mytextsc{verbe}} \hspace{4pt} Ton~: L\textsubscript{a}.

Se chauffer au feu ou au soleil; prendre le soleil.
\textcolor{brown}{\zh{烤火取暖,晒太阳。}}


\begin{exe}
\sn \textcolor{darkblue}{\textbf{\ipa{mv̩˧ ko˥}}}
\trans se chauffer au feu
\trans \textit{\textcolor{brown}{\zh{烤火取暖}}}
\end{exe}
\begin{exe}
\sn \textcolor{darkblue}{\textbf{\ipa{le˧-ko˩-ze˩}}}
\trans \mytextsc{accomp} \string_ \mytextsc{pfv}
\trans \textit{\textcolor{brown}{\zh{烤火了}}}
\end{exe}
\begin{exe}
\sn \textcolor{darkblue}{\textbf{\ipa{ɲi˧mi˧ ko˩}}}
\trans se réchauffer au soleil
\trans \textit{\textcolor{brown}{\zh{晒太阳}}}
\end{exe}
\begin{exe}
\sn \textcolor{darkblue}{\textbf{\ipa{ɲi˧mi˧ ɖɯ˧-ko˩-ɻ̍˩}}}
\trans se réchauffer un moment au soleil
\trans \textit{\textcolor{brown}{\zh{晒晒太阳}}}
\end{exe}
\begin{exe}
\sn \textcolor{darkblue}{\textbf{\ipa{ɲi˧mi˧ ɖɯ˧-ko˧\textasciitilde{}ko˥-ɻ̍˩}}}
\trans se réchauffer un moment au soleil
\trans \textit{\textcolor{brown}{\zh{晒晒太阳}}}
\end{exe}


\paragraph{\hspace{-0.5cm} {\Large \textcolor{darkblue}{\textbf{\ipa{ko˩dze˧}}}}} \hypertarget{ko\string_Bdze\string_M1}{}
\markboth{\textcolor{darkblue}{\textbf{\ipa{ko˩dze˧}}}}{}
\textcolor{teal}{\mytextsc{nom}} \hspace{4pt} Ton~: LM.

Sorte de colombe.
\textcolor{brown}{\zh{一种鸽子。}}


 \mytextsc{clf}~: \textcolor{darkblue}{\textbf{\ipa{mi˩}}} 

\paragraph{\hspace{-0.5cm} {\Large \textcolor{darkblue}{\textbf{\ipa{ko˩dʑo˩}}}}} \hypertarget{ko\string_Bdz£o\string_B1}{}
\markboth{\textcolor{darkblue}{\textbf{\ipa{ko˩dʑo˩}}}}{}
\textcolor{teal}{\mytextsc{nom}} \hspace{4pt} Ton~: .

Paon.
\textcolor{brown}{\zh{孔雀。}}




\paragraph{\hspace{-0.5cm} {\Large \textcolor{darkblue}{\textbf{\ipa{ko˩ɖʐo˩}}}}} \hypertarget{ko\string_Bd`z`o\string_B1}{}
\markboth{\textcolor{darkblue}{\textbf{\ipa{ko˩ɖʐo˩}}}}{}
\textcolor{teal}{\mytextsc{nom}} \hspace{4pt} Ton~: L.

Fléau pour battre le grain.
\textcolor{brown}{\zh{连枷。}}


 \mytextsc{clf}~: \textcolor{darkblue}{\textbf{\ipa{nɑ˧}}} 

\paragraph{\hspace{-0.5cm} {\Large \textcolor{darkblue}{\textbf{\ipa{ko˩qʰɑ˧-dʑɯ\#˥}}}}} \hypertarget{ko\string_Bq\string_hA\string_M-dz£M\#\string_T1}{}
\markboth{\textcolor{darkblue}{\textbf{\ipa{ko˩qʰɑ˧-dʑɯ\#˥}}}}{}
\textcolor{teal}{\mytextsc{nom}} \hspace{4pt} Ton~: LM+\#H.

Yyy.
\textcolor{brown}{\zh{金梅花。}}


\begin{exe}
\sn \textcolor{darkblue}{\textbf{\ipa{ko˩qʰɑ˧-dʑɯ˧-bæ˥bæ˩}}}
\trans fleur de...
\trans \textit{\textcolor{brown}{\zh{金梅花的花}}}
\end{exe}


\paragraph{\hspace{-0.5cm} {\Large \textcolor{darkblue}{\textbf{\ipa{ko˧˥}}} \textsubscript{1}}} \hypertarget{ko\string_M\string_T1}{}
\markboth{\textcolor{darkblue}{\textbf{\ipa{ko˧˥}}} \textsubscript{1}}{}
\textcolor{teal}{\mytextsc{adverbe}} \hspace{4pt} Ton~: MH.

Trop, excessivement.
\textcolor{brown}{\zh{过于,太(汉语借词)。}}

 Emprunt~: chinois \textcolor{brown}{\zh{过}}



\paragraph{\hspace{-0.5cm} {\Large \textcolor{darkblue}{\textbf{\ipa{ko˧˥}}} \textsubscript{2}}} \hypertarget{ko\string_M\string_T2}{}
\markboth{\textcolor{darkblue}{\textbf{\ipa{ko˧˥}}} \textsubscript{2}}{}
\textcolor{teal}{\mytextsc{verbe}} \hspace{4pt} Ton~: MH.

Se passer, avoir lieu: les jours passent, la vie se passe; couler (des jours/des années).
\textcolor{brown}{\zh{过(汉语借词)。}}

 Emprunt~: chinois \textcolor{brown}{\zh{过}}

\begin{exe}
\sn \textcolor{darkblue}{\textbf{\ipa{se˧ʐɯ˩ ko˩}}}
\trans fêter un anniversaire
\trans \textit{\textcolor{brown}{\zh{过生日}}}
\end{exe}


\paragraph{\hspace{-0.5cm} {\Large \textcolor{darkblue}{\textbf{\ipa{kɯ˥}}}}} \hypertarget{kM\string_T1}{}
\markboth{\textcolor{darkblue}{\textbf{\ipa{kɯ˥}}}}{}
\textcolor{teal}{\mytextsc{nom}} \hspace{4pt} Ton~: \#H.
\ding{202} 
Vésicule biliaire.
\textcolor{brown}{\zh{胆。}}


\ding{203} 
Bile.
\textcolor{brown}{\zh{胆汁。}}


 \mytextsc{clf}~: \textcolor{darkblue}{\textbf{\ipa{ɭɯ˧}}} 

\paragraph{\hspace{-0.5cm} {\Large \textcolor{darkblue}{\textbf{\ipa{kɯ˥}}}}} \hypertarget{kM\string_T1}{}
\markboth{\textcolor{darkblue}{\textbf{\ipa{kɯ˥}}}}{}
\textcolor{teal}{\mytextsc{adjectif}} \hspace{4pt} Ton~: H.

Serré, tendu.
\textcolor{brown}{\zh{紧。}}


\begin{exe}
\sn \textcolor{darkblue}{\textbf{\ipa{le˧-tsɯ˥ | le˧-kɯ˥-kʰɯ˩}}}
\trans attacher très serré
\trans \textit{\textcolor{brown}{\zh{绑紧}}}
\end{exe}
\begin{exe}
\sn \textcolor{darkblue}{\textbf{\ipa{le˧-kɯ˥-se˩}}}
\trans \mytextsc{accomp} \string_ \mytextsc{pfv}
\trans \textit{\textcolor{brown}{\zh{紧了}}}
\end{exe}


\paragraph{\hspace{-0.5cm} {\Large \textcolor{darkblue}{\textbf{\ipa{kɯ˧}}}}} \hypertarget{kM\string_M1}{}
\markboth{\textcolor{darkblue}{\textbf{\ipa{kɯ˧}}}}{}
\textcolor{teal}{\mytextsc{nom}} \hspace{4pt} Ton~: M.

Étoile.
\textcolor{brown}{\zh{星星。}}


\begin{exe}
\sn \textcolor{darkblue}{\textbf{\ipa{mv̩˧ʁo˥ | kɯ˧}}}
\trans le ciel est étoilé, on voit les étoiles du ciel
\trans \textit{\textcolor{brown}{\zh{天上有星星、天上看得见星星}}}
\end{exe}
\begin{exe}
\sn \textcolor{darkblue}{\textbf{\ipa{nɑ˩-ʈʂʰɯ˥, | kɯ˧ mɤ˧-li˧! | di˧mi˧-lɑ˧ li˥!}}}
\trans Les Na, ils ne regardent pas les étoiles! Ils ne regardent que la plaine (=leur plaine: la plaine de Yongning)! (commentaire de F4 au sujet de son ignorance des noms d'étoiles et de constellations)
\trans \textit{\textcolor{brown}{\zh{摩梭呢,不看星星,只看平坝(=永宁坝子)!(合作人说明为什么她不知道星星、星座的名字:摩梭人本来对天文不太感兴趣。)}}}
\end{exe}
 \mytextsc{clf}~: \textcolor{darkblue}{\textbf{\ipa{ɭɯ˧, kɯ˧}}} objets ronds

\paragraph{\hspace{-0.5cm} {\Large \textcolor{darkblue}{\textbf{\ipa{kɯ˧\textsubscript{b}}}}}} \hypertarget{kM\string_Mb1}{}
\markboth{\textcolor{darkblue}{\textbf{\ipa{kɯ˧\textsubscript{b}}}}}{}
\textcolor{teal}{\mytextsc{classificateur}} \hspace{4pt} Ton~: M\textsubscript{b}.

Auto-classificateur des étoiles; classificateur des jours propices.
\textcolor{brown}{\zh{量词:星星(一个)。}}




\paragraph{\hspace{-0.5cm} {\Large \textcolor{darkblue}{\textbf{\ipa{kɯ˧ɭɯ˧}}}}} \hypertarget{kM\string_Ml\string_RM\string_M1}{}
\markboth{\textcolor{darkblue}{\textbf{\ipa{kɯ˧ɭɯ˧}}}}{}
\textcolor{teal}{\mytextsc{nom}} \hspace{4pt} Ton~: M.

Esprit bienfaisant.
\textcolor{brown}{\zh{神。}}


\begin{exe}
\sn \textcolor{darkblue}{\textbf{\ipa{kɯ˧ɭɯ˧ | ɖɯ˧-dze˩}}}
\trans deux esprits bienfaisants
\trans \textit{\textcolor{brown}{\zh{两个(好)神}}}
\end{exe}
 \mytextsc{clf}~: \textcolor{darkblue}{\textbf{\ipa{dze˩}}} 

\paragraph{\hspace{-0.5cm} {\Large \textcolor{darkblue}{\textbf{\ipa{kɯ˧qʰæ˧ʂe˧˥}}}}} \hypertarget{kM\string_Mq\string_h\{\string_Ms`e\string_M\string_T1}{}
\markboth{\textcolor{darkblue}{\textbf{\ipa{kɯ˧qʰæ˧ʂe˧˥}}}}{}
\textcolor{teal}{\mytextsc{nom}} \hspace{4pt} Ton~: MH\#.

Comète (littéralement ``les étoiles défèquent").
\textcolor{brown}{\zh{流星。}}


 \mytextsc{clf}~: \textcolor{darkblue}{\textbf{\ipa{ʂɯ˩}}} fois

\paragraph{\hspace{-0.5cm} {\Large \textcolor{darkblue}{\textbf{\ipa{kɯ˩\textsubscript{a}}}}}} \hypertarget{kM\string_Ba1}{}
\markboth{\textcolor{darkblue}{\textbf{\ipa{kɯ˩\textsubscript{a}}}}}{}
\textcolor{teal}{\mytextsc{verbe}} \hspace{4pt} Ton~: L\textsubscript{a}.

Laisser quelqu'un en rade, laisser quelqu'un seul au moment où il aurait besoin d'aide, faire mine d'ignorer quelqu'un qui aurait besoin d'aide, négliger d'assister quelqu'un. Il n'a pas été trouvé d'équivalent chinois simple pour ce terme, qui renvoie à la situation où un manque de réel sympathie pour quelqu'un se traduit par le fait qu'on n'est pas tenté de faire l'effort de l'aider lorsqu'il en aurait besoin: on fait alors comme si de rien n'était, comme si on n'était pas au courant de la situation de cette personne.
\textcolor{brown}{\zh{不理需要帮忙的人:知道一个人需要帮助,自己也有能力帮忙,但假装没看见、什么事没有。}}


\begin{exe}
\sn \textcolor{darkblue}{\textbf{\ipa{hĩ˧ kɯ˥}}}
\trans même sens
\trans \textit{\textcolor{brown}{\zh{同上}}}
\end{exe}
\begin{exe}
\sn \textcolor{darkblue}{\textbf{\ipa{hĩ˧-ɳɯ˩ | kɯ˩-kv̩˥!}}}
\trans Il arrive que les gens te laissent tomber! / Il arrive que les gens ne t'apportent pas leur aide quand tu en aurais besoin!
\trans \textit{\textcolor{brown}{\zh{人家在你需要帮忙的时候就会不理你的!(如果处不好关系,人家对你没有什么好感,到时候你需要帮忙他们就不理你了。)}}}
\end{exe}
\begin{exe}
\sn \textcolor{darkblue}{\textbf{\ipa{kɯ˩-mɤ˩-kv̩˥!}}}
\trans (ils/elles) n(e t)'aideront pas!
\trans \textit{\textcolor{brown}{\zh{人家在你需要帮忙的时候就会不理你的!}}}
\end{exe}
\begin{exe}
\sn \textcolor{darkblue}{\textbf{\ipa{hĩ˧-ɳɯ˩ | kɯ˩-tʰɑ˩-kʰɯ˥!}}}
\trans Fais en sorte que les gens ne te laissent pas en rade (quand tu auras besoin d'aide)! (Explication: il faut faire en sorte de gagner une estime et une sympathie réelles de la part des gens que l'on connaît; de la sorte, ils vous aideront spontanément lorsque vous en aurez besoin. Sinon, leur manque de réelle sympathie se traduira par le fait qu'ils ne feront pas l'effort de donner un coup de main lorsqu'on en aura besoin: ils feront alors comme si de rien n'était, comme s'ils n'étaient pas conscients de notre besoin.)
\trans \textit{\textcolor{brown}{\zh{别让人家(在你需要帮忙的时候)不理你!}}}
\end{exe}
\begin{exe}
\sn \textcolor{darkblue}{\textbf{\ipa{njɤ˧ | no˩ kɯ˩-hĩ˥ mɤ˩-ɲi˩! | njɤ˧ | no˧-ki˧ | dʑɤ˩-so˥-ɲi˩!}}}
\trans Je ne te néglige pas du tout! (Bien au contraire:) je t'enseigne bien / je fais de mon mieux pour t'apprendre des choses! (Contexte imaginé: un étudiant s'estime négligé par un enseignant; celui-ci se rend compte du mécontentement de l'étudiant, et lui dit qu'il interprète mal.)
\trans \textit{\textcolor{brown}{\zh{我不是不重视你!(刚好相反:)我是用心教你的 / 我努力教你最好的!(情景:一名学生认为老师忽视他,老师发现学生不高兴,就说明。)}}}
\end{exe}


\paragraph{\hspace{-0.5cm} {\Large \textcolor{darkblue}{\textbf{\ipa{kɯ˩ɻ̍˧}}}}} \hypertarget{kM\string_Br£`̍\string_M1}{}
\markboth{\textcolor{darkblue}{\textbf{\ipa{kɯ˩ɻ̍˧}}}}{}
\textcolor{teal}{\mytextsc{nom}} \hspace{4pt} Ton~: LM.

Violon à deux cordes, erhu.
\textcolor{brown}{\zh{胡琴,二胡。}}


\begin{exe}
\sn \textcolor{darkblue}{\textbf{\ipa{kɯ˩ɻ̍˧ ʈɤ˧}}}
\trans jouer du erhu
\trans \textit{\textcolor{brown}{\zh{拉二胡}}}
\end{exe}
 \mytextsc{clf}~: \textcolor{darkblue}{\textbf{\ipa{nɑ˧}}} 

\paragraph{\hspace{-0.5cm} {\Large \textcolor{darkblue}{\textbf{\ipa{kv̩˧˥}}}}} \hypertarget{kv\string_=\string_M\string_T1}{}
\markboth{\textcolor{darkblue}{\textbf{\ipa{kv̩˧˥}}}}{}
\textcolor{teal}{\mytextsc{verbe}} \hspace{4pt} Ton~: MH.

Pouvoir, être capable de, avoir la compétence pour (verbe de modalité épistémique).
\textcolor{brown}{\zh{会、有能力做。}}




\paragraph{\hspace{-0.5cm} {\Large \textcolor{darkblue}{\textbf{\ipa{kv̩˥}}}}} \hypertarget{kv\string_=\string_T1}{}
\markboth{\textcolor{darkblue}{\textbf{\ipa{kv̩˥}}}}{}
\textcolor{teal}{\mytextsc{nom}} \hspace{4pt} Ton~: \#H.

Ail, \textit{Allium sativum}.
\textcolor{brown}{\zh{大蒜。}}


 \mytextsc{clf}~: \textcolor{darkblue}{\textbf{\ipa{ɭɯ˧}}} \textcolor{darkblue}{\textbf{\ipa{tsʰɤ˧˥}}} gousse; tête

\paragraph{\hspace{-0.5cm} {\Large \textcolor{darkblue}{\textbf{\ipa{kv̩˩\textsubscript{a}}}} \textsubscript{1}}} \hypertarget{kv\string_=\string_Ba1}{}
\markboth{\textcolor{darkblue}{\textbf{\ipa{kv̩˩\textsubscript{a}}}} \textsubscript{1}}{}
\textcolor{teal}{\mytextsc{verbe}} \hspace{4pt} Ton~: L\textsubscript{a}.
\ding{202} 
Ramasser; cueillir (des baies, des choses qu'on se baisse pour cueillir).
\textcolor{brown}{\zh{捡起来,拾。}}


\begin{exe}
\sn \textcolor{darkblue}{\textbf{\ipa{kv̩˧\textasciitilde{}kv̩˥}}}
\trans \mytextsc{red}
\trans \textit{\textcolor{brown}{\zh{\mytextsc{重叠}}}}
\end{exe}
\begin{exe}
\sn \textcolor{darkblue}{\textbf{\ipa{gɤ˩-kv̩˧\textasciitilde{}kv̩˥}}}
\trans ramasser (quelque chose qui se trouvait à terre)
\trans \textit{\textcolor{brown}{\zh{捡起来(地上的东西)}}}
\end{exe}
\begin{exe}
\sn \textcolor{darkblue}{\textbf{\ipa{le˧-kv̩˧\textasciitilde{}kv̩˥}}}
\trans ramasser (quelque chose qui se trouvait à terre)
\trans \textit{\textcolor{brown}{\zh{捡起来(地上的东西)}}}
\end{exe}
\ding{203} 
Pêcher.
\textcolor{brown}{\zh{钓鱼。}}


\begin{exe}
\sn \textcolor{darkblue}{\textbf{\ipa{ɲi˧zo˧ kv̩˥}}}
\trans pêcher du poisson
\trans \textit{\textcolor{brown}{\zh{钓鱼}}}
\end{exe}


\paragraph{\hspace{-0.5cm} {\Large \textcolor{darkblue}{\textbf{\ipa{kv̩˩\textsubscript{a}}}} \textsubscript{2}}} \hypertarget{kv\string_=\string_Ba2}{}
\markboth{\textcolor{darkblue}{\textbf{\ipa{kv̩˩\textsubscript{a}}}} \textsubscript{2}}{}
\textcolor{teal}{\mytextsc{verbe}} \hspace{4pt} Ton~: L\textsubscript{a}.

Traverser.
\textcolor{brown}{\zh{过。}}


\begin{exe}
\sn \textcolor{darkblue}{\textbf{\ipa{ʈʂʰwæ˩ kv̩˥}}}
\trans traverser en bateau
\trans \textit{\textcolor{brown}{\zh{坐船过(河)}}}
\end{exe}


\paragraph{\hspace{-0.5cm} {\Large \textcolor{darkblue}{\textbf{\ipa{kv̩˧dʑɯ˧˥}}}}} \hypertarget{kv\string_=\string_Mdz£M\string_M\string_T1}{}
\markboth{\textcolor{darkblue}{\textbf{\ipa{kv̩˧dʑɯ˧˥}}}}{}
\textcolor{teal}{\mytextsc{nom}} \hspace{4pt} Ton~: MH.

Tente.
\textcolor{brown}{\zh{帐篷。}}


\begin{exe}
\sn \textcolor{darkblue}{\textbf{\ipa{kv̩˧dʑɯ˧ lɑ˥}}}
\trans déplier une tente, installer une tente
\trans \textit{\textcolor{brown}{\zh{安装帐篷、搭建帐篷}}}
\end{exe}
 \mytextsc{clf}~: \textcolor{darkblue}{\textbf{\ipa{nɑ˧}}} 

\paragraph{\hspace{-0.5cm} {\Large \textcolor{darkblue}{\textbf{\ipa{kv̩˧ʝi˥}}}}} \hypertarget{kv\string_=\string_Mj££i\string_T1}{}
\markboth{\textcolor{darkblue}{\textbf{\ipa{kv̩˧ʝi˥}}}}{}
\textcolor{teal}{\mytextsc{adverbe}} \hspace{4pt} Ton~: H\#.

Pour de vrai, réellement.
\textcolor{brown}{\zh{真的、的确、确实。}}




\paragraph{\hspace{-0.5cm} {\Large \textcolor{darkblue}{\textbf{\ipa{kv̩˧ʝi˥\$}}}}} \hypertarget{kv\string_=\string_Mj££i\string_T\$1}{}
\markboth{\textcolor{darkblue}{\textbf{\ipa{kv̩˧ʝi˥\$}}}}{}
\textcolor{teal}{\mytextsc{nom}} \hspace{4pt} Ton~: H\$.

Vie, existence.
\textcolor{brown}{\zh{生命。}}




\paragraph{\hspace{-0.5cm} {\Large \textcolor{darkblue}{\textbf{\ipa{kv̩˩kv̩˩}}}}} \hypertarget{kv\string_=\string_Bkv\string_=\string_B1}{}
\markboth{\textcolor{darkblue}{\textbf{\ipa{kv̩˩kv̩˩}}}}{}
\textcolor{teal}{\mytextsc{nom}} \hspace{4pt} Ton~: L.

Pommettes.
\textcolor{brown}{\zh{颧骨。}}


 \mytextsc{clf}~: \textcolor{darkblue}{\textbf{\ipa{ɭɯ˧}}} \textit{Voir~:} \hyperlink{nj7\string_Mkv\string_=\string_B1}{\textcolor{darkblue}{\textbf{\ipa{njɤ˧kv̩˩}}}} 

\paragraph{\hspace{-0.5cm} {\Large \textcolor{darkblue}{\textbf{\ipa{kv̩˧lv̩˧lv̩˥}}}}} \hypertarget{kv\string_=\string_Mlv\string_=\string_Mlv\string_=\string_T1}{}
\markboth{\textcolor{darkblue}{\textbf{\ipa{kv̩˧lv̩˧lv̩˥}}}}{}
\textcolor{teal}{\mytextsc{nom}} \hspace{4pt} Ton~: H\#.

Tresse d'ail, ail tressé.
\textcolor{brown}{\zh{蒜瓣。}}


 \mytextsc{clf}~: \textcolor{darkblue}{\textbf{\ipa{ɭɯ˧}}} 

\paragraph{\hspace{-0.5cm} {\Large \textcolor{darkblue}{\textbf{\ipa{kv̩˩nɑ˧˥}}}}} \hypertarget{kv\string_=\string_BnA\string_M\string_T1}{}
\markboth{\textcolor{darkblue}{\textbf{\ipa{kv̩˩nɑ˧˥}}}}{}
\textcolor{teal}{\mytextsc{nom}} \hspace{4pt} Ton~: LM+MH\#.

Soie.
\textcolor{brown}{\zh{丝绸。}}


\begin{exe}
\sn \textcolor{darkblue}{\textbf{\ipa{kv̩˩nɑ˧-bɑ˧lɑ˥}}}
\trans vêtement en soie
\trans \textit{\textcolor{brown}{\zh{丝绸衣服}}}
\end{exe}
 \mytextsc{clf}~: \textcolor{darkblue}{\textbf{\ipa{tsʰi˥}}} 

\paragraph{\hspace{-0.5cm} {\Large \textcolor{darkblue}{\textbf{\ipa{kv̩˧ɲi˥}}}}} \hypertarget{kv\string_=\string_MJi\string_T1}{}
\markboth{\textcolor{darkblue}{\textbf{\ipa{kv̩˧ɲi˥}}}}{}
\textcolor{teal}{\mytextsc{adjectif}} \hspace{4pt} Ton~: H\#.

Vide, sans rien.
\textcolor{brown}{\zh{空手,空。}}


\begin{exe}
\sn \textcolor{darkblue}{\textbf{\ipa{bi˩ʁo˧ | kv̩˧ɲi˥-kʰɯ˩}}}
\trans vider la bourse (de quelqu'un), c'est-à-dire lui prendre son argent)
\trans \textit{\textcolor{brown}{\zh{(把一个人的)钱包弄空}}}
\end{exe}
\begin{exe}
\sn \textcolor{darkblue}{\textbf{\ipa{tɕʰɯ˩ di˩-hɯ˩˥, | mɤ˧-ɖɯ˧, | kv̩˧ɲi˥ | le˧-tsʰɯ˩!}}}
\trans Il est parti chasser le muntjac, il n'en a pas tué, et est revenu bredouille!
\trans \textit{\textcolor{brown}{\zh{他去狩猎,没得(任何猎物),空手回来!}}}
\end{exe}


\paragraph{\hspace{-0.5cm} {\Large \textcolor{darkblue}{\textbf{\ipa{kv̩˧ʁo˧bv̩˥}}}}} \hypertarget{kv\string_=\string_MRo\string_Mbv\string_=\string_T1}{}
\markboth{\textcolor{darkblue}{\textbf{\ipa{kv̩˧ʁo˧bv̩˥}}}}{}
\textcolor{teal}{\mytextsc{nom}} \hspace{4pt} Ton~: H\#.

Pousses d'ail (aliment).
\textcolor{brown}{\zh{蒜苗。}}


 \mytextsc{clf}~: \textcolor{darkblue}{\textbf{\ipa{kɤ˧˥}}} 

\paragraph{\hspace{-0.5cm} {\Large \textcolor{darkblue}{\textbf{\ipa{kv̩˧ʂe˥\$}}}}} \hypertarget{kv\string_=\string_Ms`e\string_T\$1}{}
\markboth{\textcolor{darkblue}{\textbf{\ipa{kv̩˧ʂe˥\$}}}}{}
\textcolor{teal}{\mytextsc{nom}} \hspace{4pt} Ton~: H\$.

Puce.
\textcolor{brown}{\zh{跳蚤。}}


 \mytextsc{clf}~: \textcolor{darkblue}{\textbf{\ipa{mi˩}}} 

\paragraph{\hspace{-0.5cm} {\Large \textcolor{darkblue}{\textbf{\ipa{kv̩˩tɑ˩}}}}} \hypertarget{kv\string_=\string_BtA\string_B1}{}
\markboth{\textcolor{darkblue}{\textbf{\ipa{kv̩˩tɑ˩}}}}{}
\textcolor{teal}{\mytextsc{verbe}} \hspace{4pt} Ton~: L.

Regrouper, rassembler (ex.: des troncs, après leur abattage).
\textcolor{brown}{\zh{集中在一起(如:砍木材后,把木材堆在一起)。}}




\paragraph{\hspace{-0.5cm} {\Large \textcolor{darkblue}{\textbf{\ipa{kv̩˧tsʰɑ˥\$}}}}} \hypertarget{kv\string_=\string_Mts\string_hA\string_T\$1}{}
\markboth{\textcolor{darkblue}{\textbf{\ipa{kv̩˧tsʰɑ˥\$}}}}{}
\textcolor{teal}{\mytextsc{nom}} \hspace{4pt} Ton~: H\$.

Nom de clan/famille étendue. Ce nom était celui de la famille des seigneurs (pumi/prinmi) de Muli.
\textcolor{brown}{\zh{一个姓(木里土司,普米族,的姓)。}}


\begin{exe}
\sn \textcolor{darkblue}{\textbf{\ipa{kv̩˧tsʰɑ˧=ɻ̍˥\$}}}
\trans le clan \textcolor{darkblue}{\textbf{\ipa{/kv̩˧tsʰɑ˥\$/}}}, la famille \textcolor{darkblue}{\textbf{\ipa{/kv̩˧tsʰɑ˥\$/}}}
\trans \textit{\textcolor{brown}{\zh{\textcolor{darkblue}{\textbf{\ipa{/kv̩˧tsʰɑ˥\$/}}}家族}}}
\end{exe}
\begin{exe}
\sn \textcolor{darkblue}{\textbf{\ipa{kv̩˧tsʰɑ˧=ɻ̍˧ pi˥-zo˩!}}}
\trans On les appelait ``les \textcolor{darkblue}{\textbf{\ipa{/kv̩˧tsʰɑ˥\$/"!}}}
\trans \textit{\textcolor{brown}{\zh{人家把他们称作“\textcolor{darkblue}{\textbf{\ipa{/kv̩˧tsʰɑ˥\$/}}}家族”!}}}
\end{exe}


\paragraph{\hspace{-0.5cm} {\Large \textcolor{darkblue}{\textbf{\ipa{kv̩˧tsʰɤ˩}}}}} \hypertarget{kv\string_=\string_Mts\string_h7\string_B1}{}
\markboth{\textcolor{darkblue}{\textbf{\ipa{kv̩˧tsʰɤ˩}}}}{}
\textcolor{teal}{\mytextsc{nom}} \hspace{4pt} Ton~: L\#.

Tête d'ail.
\textcolor{brown}{\zh{蒜头。}}


 \mytextsc{clf}~: \textcolor{darkblue}{\textbf{\ipa{tsʰɤ˧˥}}} 

\paragraph{\hspace{-0.5cm} {\Large \textcolor{darkblue}{\textbf{\ipa{kv̩˧ʈʂɯ˧˥}}}}} \hypertarget{kv\string_=\string_Mt`s`M\string_M\string_T1}{}
\markboth{\textcolor{darkblue}{\textbf{\ipa{kv̩˧ʈʂɯ˧˥}}}}{}
\textcolor{teal}{\mytextsc{nom}} \hspace{4pt} Ton~: MH\#.

Ongle.
\textcolor{brown}{\zh{指甲。}}


 \mytextsc{clf}~: \textcolor{darkblue}{\textbf{\ipa{ɭɯ˧}}} 

\paragraph{\hspace{-0.5cm} {\Large \textcolor{darkblue}{\textbf{\ipa{‑kv̩˧˥}}}}} \hypertarget{‑kv\string_=\string_M\string_T1}{}
\markboth{\textcolor{darkblue}{\textbf{\ipa{‑kv̩˧˥}}}}{}
\textcolor{teal}{\mytextsc{suffixe}} \hspace{4pt} Ton~: MH.

\mytextsc{abilitive;} a aussi des emplois de futur.
\textcolor{brown}{\zh{能。}}




\paragraph{\hspace{-0.5cm} {\Large \textcolor{darkblue}{\textbf{\ipa{kwɑ˧fæ˩}}}}} \hypertarget{kwA\string_Mf\{\string_B1}{}
\markboth{\textcolor{darkblue}{\textbf{\ipa{kwɑ˧fæ˩}}}}{}
\textcolor{teal}{\mytextsc{nom}} \hspace{4pt} Ton~: L\#.

Nom d'un hôtel.
\textcolor{brown}{\zh{官房(汉语借词),酒店名称。}}

 Emprunt~: chinois \textcolor{brown}{\zh{官房}}

\begin{exe}
\sn \textcolor{darkblue}{\textbf{\ipa{kwɑ˧fæ˩}}}
\trans le nom abrégé d'un hôtel cinq étoiles où travaille l'une des filles de la consultante principale.
\trans \textit{\textcolor{brown}{\zh{丽江官房大酒店的简称。注:发音合作人的女儿在丽江官房大酒店工作。}}}
\end{exe}


\paragraph{\hspace{-0.5cm} {\Large \textcolor{darkblue}{\textbf{\ipa{kwɑ˧tsʰɑ˧}}}}} \hypertarget{kwA\string_Mts\string_hA\string_M1}{}
\markboth{\textcolor{darkblue}{\textbf{\ipa{kwɑ˧tsʰɑ˧}}}}{}
\textcolor{teal}{\mytextsc{nom}} \hspace{4pt} Ton~: M.

Cercueil.
\textcolor{brown}{\zh{棺材(汉语借词)。}}

 Emprunt~: chinois \textcolor{brown}{\zh{棺材}}

\begin{exe}
\sn \textcolor{darkblue}{\textbf{\ipa{kwɑ˧tsʰɑ˧, | hĩ˧-mo˩-kʰɯ˩-di˩ ɲi˩!}}}
\trans Le cercueil, c'est là où on met le cadavre! / Le cercueil, c'est l'objet qui accueille le cadavre!
\trans \textit{\textcolor{brown}{\zh{棺材,是装尸体的! / 棺材,是用来装尸体的!}}}
\end{exe}
 \mytextsc{clf}~: \textcolor{darkblue}{\textbf{\ipa{ɭɯ˧}}} 

\paragraph{\hspace{-0.5cm} {\Large \textcolor{darkblue}{\textbf{\ipa{kwæ˧}}}}} \hypertarget{kw\{\string_M1}{}
\markboth{\textcolor{darkblue}{\textbf{\ipa{kwæ˧}}}}{}
\textcolor{teal}{\mytextsc{verbe}} \hspace{4pt} Ton~: M.

S'occuper de, se charger de.
\textcolor{brown}{\zh{管(汉语借词)。}}

 Emprunt~: chinois \textcolor{brown}{\zh{管}}



\paragraph{\hspace{-0.5cm} {\Large \textcolor{darkblue}{\textbf{\ipa{kwæ˧fæ˥}}}}} \hypertarget{kw\{\string_Mf\{\string_T1}{}
\markboth{\textcolor{darkblue}{\textbf{\ipa{kwæ˧fæ˥}}}}{}
\textcolor{teal}{\mytextsc{nom}} \hspace{4pt} Ton~: H\#.

Poutre intermédiaire: pièce de charpente horizontale, posée sur une poutre maîtresse, et soutenant deux des poutres du toit, /ʐv̩˩ɭɯ˧/.
\textcolor{brown}{\zh{中等大小的梁。}}


 \mytextsc{clf}~: \textcolor{darkblue}{\textbf{\ipa{pʰæ˧˥}}} 

\paragraph{\hspace{-0.5cm} {\Large \textcolor{darkblue}{\textbf{\ipa{kwæ˧pæ˥}}}}} \hypertarget{kw\{\string_Mp\{\string_T1}{}
\markboth{\textcolor{darkblue}{\textbf{\ipa{kwæ˧pæ˥}}}}{}
\textcolor{teal}{\mytextsc{nom}} \hspace{4pt} Ton~: H\#.

Herse en bois (vraisemblablement un emprunt au chinois; terme emprunté: pas identifié avec certitude).
\textcolor{brown}{\zh{耙(可能是汉语借词。原来借来的词:刮板?? 刮耙??)。}}


 \mytextsc{clf}~: \textcolor{darkblue}{\textbf{\ipa{nɑ˧}}} 

\paragraph{\hspace{-0.5cm} {\Large \textcolor{darkblue}{\textbf{\ipa{kwæ˧tsɯ˧}}}}} \hypertarget{kw\{\string_MtsM\string_M1}{}
\markboth{\textcolor{darkblue}{\textbf{\ipa{kwæ˧tsɯ˧}}}}{}
\textcolor{teal}{\mytextsc{nom}} \hspace{4pt} Ton~: M.

Graine de tournesol.
\textcolor{brown}{\zh{葵花瓜籽(汉语借词)。}}

 Emprunt~: chinois \textcolor{brown}{\zh{瓜子}}



\paragraph{\hspace{-0.5cm} {\Large \textcolor{darkblue}{\textbf{\ipa{‑kwɤ}}}}} \hypertarget{‑kw71}{}
\markboth{\textcolor{darkblue}{\textbf{\ipa{‑kwɤ}}}}{}
\textcolor{teal}{\mytextsc{conjonction}} \hspace{4pt} Ton~: 0.

Lorsque; ne peut s'employer seul, mais apparaît dans la formule \textcolor{darkblue}{\textbf{\ipa{/kwɤ-tɕɯ-lɑ/}}}.
\textcolor{brown}{\zh{……的时候。}}




\paragraph{\hspace{-0.5cm} {\Large \textcolor{darkblue}{\textbf{\ipa{kwɤ˧ɭɯ˩}}}}} \hypertarget{kw7\string_Ml\string_RM\string_B1}{}
\markboth{\textcolor{darkblue}{\textbf{\ipa{kwɤ˧ɭɯ˩}}}}{}
\textcolor{teal}{\mytextsc{nom}} \hspace{4pt} Ton~: L\#.

Jarre; trésor, objet de grande valeur.
\textcolor{brown}{\zh{坛子、罐子 (陶器),宝贝。}}


\begin{exe}
\sn \textcolor{darkblue}{\textbf{\ipa{ʈʂʰɯ˧ | njɤ˧ kwɤ˧ɭɯ˩ ɲi˩!}}}
\trans C'est mon petit trésor! (dit au sujet d'un enfant)
\trans \textit{\textcolor{brown}{\zh{他是我宝贝!(母亲说孩子是她的宝贝)}}}
\end{exe}
 \mytextsc{clf}~: \textcolor{darkblue}{\textbf{\ipa{ɭɯ˧}}} 

\paragraph{\hspace{-0.5cm} {\Large \textcolor{darkblue}{\textbf{\ipa{kwɤ˧pɤ˧}}}}} \hypertarget{kw7\string_Mp7\string_M1}{}
\markboth{\textcolor{darkblue}{\textbf{\ipa{kwɤ˧pɤ˧}}}}{}
\textcolor{teal}{\mytextsc{nom}} \hspace{4pt} Ton~: M.

Explication, enseignement.
\textcolor{brown}{\zh{解释,教导、教诲。}}


\begin{exe}
\sn \textcolor{darkblue}{\textbf{\ipa{kwɤ˧pɤ˧ ɖɯ˧-kʰwɤ˥ lɑ˩}}}
\trans enseigner quelque chose à quelqu'un, expliquer quelque chose à quelqu'un
\trans \textit{\textcolor{brown}{\zh{解释一个道理、教一件事}}}
\end{exe}
\begin{exe}
\sn \textcolor{darkblue}{\textbf{\ipa{kwɤ˧pɤ˧ ɖɯ˧-kʰwɤ˥ | tʰi˧-lɑ˩-ɻ̍˩}}}
\trans comme ci-dessus: enseigner quelque chose à quelqu'un, expliquer quelque chose à quelqu'un
\trans \textit{\textcolor{brown}{\zh{同上:解释一个道理、教一件事}}}
\end{exe}
\begin{exe}
\sn \textcolor{darkblue}{\textbf{\ipa{[M23] kwɤ˧pɤ˧ lɑ˧˥}}}
\trans enseigner
\trans \textit{\textcolor{brown}{\zh{教、解释}}}
\end{exe}
 \mytextsc{clf}~: \textcolor{darkblue}{\textbf{\ipa{kʰwɤ˥}}} 

\paragraph{\hspace{-0.5cm} {\Large \textcolor{darkblue}{\textbf{\ipa{‑kwɤ˧tɕɯ˥}}}}} \hypertarget{‑kw7\string_Mts£M\string_T1}{}
\markboth{\textcolor{darkblue}{\textbf{\ipa{‑kwɤ˧tɕɯ˥}}}}{}
\textcolor{teal}{\mytextsc{conjonction}} \hspace{4pt} Ton~: H\#.

Comme; après; puisque.
\textcolor{brown}{\zh{因为,由于,既然。}}


\begin{exe}
\sn \textcolor{darkblue}{\textbf{\ipa{-kwɤ˧tɕɯ˥-lɑ˩}}}
\trans même sens
\trans \textit{\textcolor{brown}{\zh{同上}}}
\end{exe}
\begin{exe}
\sn \textcolor{darkblue}{\textbf{\ipa{ʈʂʰɯ˧ | go˩-kwɤ˩tɕɯ˥-lɑ˩, | hɑ˧ mɤ˧-dzɯ˥.}}}
\trans Comme il est malade, il ne mange pas.
\trans \textit{\textcolor{brown}{\zh{他病了,吃不下饭。}}}
\end{exe}
\begin{exe}
\sn \textcolor{darkblue}{\textbf{\ipa{[M18] ʈʂʰɯ˧ne˧-ʝi˥ | pi˧-kwɤ˩tɕɯ˩-lɑ˩, | wɤ˩˥ | lɑ˧hɑ˥ | ɖɯ˧-kʰwɤ˧ ʐwɤ˧˥.}}}
\trans ayant dit cela, il dit à nouveau autre chose (après avoir dit ça, il a dit ajouté autre chose!)
\trans \textit{\textcolor{brown}{\zh{他这样说完以后,又讲了些其它的。}}}
\end{exe}
\begin{exe}
\sn \textcolor{darkblue}{\textbf{\ipa{[M18] ʈʂʰɯ˧ | tʰi˧-dzi˩-kwɤ˩-tɕɯ˩, | ɖɯ˧-kʰwɤ˧ ʐwɤ˧-ɻ̍˥: | ``sɤ˧sɤ˧˥ | ʐwæ˧˥!"}}}
\trans après s’être assis/lorsqu’il fut assis, il dit une phrase: ``quel confort!"
\trans \textit{\textcolor{brown}{\zh{他坐下后,说了这么一句:“真舒服!”}}}
\end{exe}


\paragraph{\hspace{-0.5cm} {\Large \textcolor{darkblue}{\textbf{\ipa{kwɤ˩\textsubscript{a}}}}}} \hypertarget{kw7\string_Ba1}{}
\markboth{\textcolor{darkblue}{\textbf{\ipa{kwɤ˩\textsubscript{a}}}}}{}
\textcolor{teal}{\mytextsc{classificateur}} \hspace{4pt} Ton~: L\textsubscript{a}.

Classificateur des objets tressés, enfilés ou liés ensemble.
\textcolor{brown}{\zh{量词:串。}}


\begin{exe}
\sn \textcolor{darkblue}{\textbf{\ipa{kv̩˧ | ɖɯ˧-kwɤ˩}}}
\trans une tresse d'aïl
\trans \textit{\textcolor{brown}{\zh{一辫大蒜}}}
\end{exe}
\begin{exe}
\sn \textcolor{darkblue}{\textbf{\ipa{lɑ˧tsɯ˥ | ɖɯ˧-kwɤ˩}}}
\trans une ligature de piments
\trans \textit{\textcolor{brown}{\zh{一辫辣椒}}}
\end{exe}
\begin{exe}
\sn \textcolor{darkblue}{\textbf{\ipa{ʈʂʰɯ˧-kwɤ˥}}}
\trans \mytextsc{dem} \string_ (ton: H\# / H\$)
\trans \textit{\textcolor{brown}{\zh{\mytextsc{指示代词} \string_}}}
\end{exe}


\paragraph{\hspace{-0.5cm} {\Large \textcolor{darkblue}{\textbf{\ipa{kwɤ˩\textsubscript{a}}}} \textsubscript{1}}} \hypertarget{kw7\string_Ba1}{}
\markboth{\textcolor{darkblue}{\textbf{\ipa{kwɤ˩\textsubscript{a}}}} \textsubscript{1}}{}
\textcolor{teal}{\mytextsc{verbe}} \hspace{4pt} Ton~: L\textsubscript{a}.

Jeter.
\textcolor{brown}{\zh{扔掉。}}


\begin{exe}
\sn \textcolor{darkblue}{\textbf{\ipa{mv̩˩tɕo˧ kwɤ˩}}}
\trans jeter (détritus); littéralement: ``mettre en bas"
\trans \textit{\textcolor{brown}{\zh{扔掉(垃圾)}}}
\end{exe}
\begin{exe}
\sn \textcolor{darkblue}{\textbf{\ipa{tso˧\textasciitilde{}tso˧ kwɤ˥}}}
\trans jeter des choses
\trans \textit{\textcolor{brown}{\zh{扔东西}}}
\end{exe}


\paragraph{\hspace{-0.5cm} {\Large \textcolor{darkblue}{\textbf{\ipa{kwɤ˩\textsubscript{a}}}} \textsubscript{2}}} \hypertarget{kw7\string_Ba2}{}
\markboth{\textcolor{darkblue}{\textbf{\ipa{kwɤ˩\textsubscript{a}}}} \textsubscript{2}}{}
\textcolor{teal}{\mytextsc{verbe}} \hspace{4pt} Ton~: L\textsubscript{a}.

S'occuper de, gérer, superviser.
\textcolor{brown}{\zh{管(汉语借词)。}}


\begin{exe}
\sn \textcolor{darkblue}{\textbf{\ipa{ɖɯ˧-kʰwɤ˧ kwɤ˥}}}
\trans s'occuper un peu/s'occuper d'une partie (combinaison élicitée pour vérifier que le ton est L et non LM)
\trans \textit{\textcolor{brown}{\zh{管一些}}}
\end{exe}


\paragraph{\hspace{-0.5cm} {\Large \textcolor{darkblue}{\textbf{\ipa{kwɤ˩-tjɤ˧ljɤ\#˥}}}}} \hypertarget{kw7\string_B-tj7\string_Mlj7\#\string_T1}{}
\markboth{\textcolor{darkblue}{\textbf{\ipa{kwɤ˩-tjɤ˧ljɤ\#˥}}}}{}
\textcolor{teal}{\mytextsc{nom}} \hspace{4pt} Ton~: L-\#H.

Clochette s'accrochant autour du cou (ex.: clochette d'un cheval).
\textcolor{brown}{\zh{铃铛。}}


 \mytextsc{clf}~: \textcolor{darkblue}{\textbf{\ipa{ɭɯ˧}}} 

\paragraph{\hspace{-0.5cm} {\Large \textcolor{darkblue}{\textbf{\ipa{kwɤ˩to˥}}}}} \hypertarget{kw7\string_Bto\string_T1}{}
\markboth{\textcolor{darkblue}{\textbf{\ipa{kwɤ˩to˥}}}}{}
\textcolor{teal}{\mytextsc{nom}} \hspace{4pt} Ton~: LH.

Mandibule, mâchoire inférieure.
\textcolor{brown}{\zh{颌骨。}}


 \mytextsc{clf}~: \textcolor{darkblue}{\textbf{\ipa{ɭɯ˧}}} 

\newpage
\section*{\centering- \textcolor{darkblue}{\textbf{\ipa{kʰ}}} -}
\paragraph{\hspace{-0.5cm} {\Large \textcolor{darkblue}{\textbf{\ipa{kʰɤ˧ɕjɤ˧}}}}} \hypertarget{k\string_h7\string_Ms£j7\string_M1}{}
\markboth{\textcolor{darkblue}{\textbf{\ipa{kʰɤ˧ɕjɤ˧}}}}{}
\textcolor{teal}{\mytextsc{verbe}} \hspace{4pt} Ton~: .

Dépenser.
\textcolor{brown}{\zh{花掉。}}


\begin{exe}
\sn \textcolor{darkblue}{\textbf{\ipa{kʰɤ˧ɕjɤ˧-ze˩!}}}
\trans \mytextsc{pfv}
\trans \textit{\textcolor{brown}{\zh{花掉了!}}}
\end{exe}
\begin{exe}
\sn \textcolor{darkblue}{\textbf{\ipa{le˧-kʰɤ˧ɕjɤ˧ | le˧-se˩-kʰɯ˩}}}
\trans tout dépenser (argent)
\trans \textit{\textcolor{brown}{\zh{全部都花完}}}
\end{exe}
\begin{exe}
\sn \textcolor{darkblue}{\textbf{\ipa{mɤ˧-dʑo˧, | ɖwæ˩-mɤ˧-zo˧! | tʰi˧-ʂe˧ tʰi˧-kʰɤ˧ɕjɤ˧-tsæ˥-ɲi˩!}}}
\trans Si on n'a rien/qu'on est pauvre, il ne faut pas se faire de souci! On n'a qu'à aller gagner (littéralement ``chercher") puis dépenser! (Philosophie qu'enseignent les aînés: l'argent c'est quelque chose qui circule, pas quelque chose qui se thésaurise.)
\trans \textit{\textcolor{brown}{\zh{如果没有(钱),不用担心!要赚也要花!(这是长辈的一个教诲:钱赚了又花、花了又赚,不是来积累的。)}}}
\end{exe}


\paragraph{\hspace{-0.5cm} {\Large \textcolor{darkblue}{\textbf{\ipa{kʰɤ˧mi˥\$}}}}} \hypertarget{k\string_h7\string_Mmi\string_T\$1}{}
\markboth{\textcolor{darkblue}{\textbf{\ipa{kʰɤ˧mi˥\$}}}}{}
\textcolor{teal}{\mytextsc{nom}} \hspace{4pt} Ton~: \$H.

Grande hotte.
\textcolor{brown}{\zh{大背篓。}}


 \mytextsc{clf}~: \textcolor{darkblue}{\textbf{\ipa{kʰɤ˧˥}}} 

\paragraph{\hspace{-0.5cm} {\Large \textcolor{darkblue}{\textbf{\ipa{kʰɤ˧ʂɯ˧}}}}} \hypertarget{k\string_h7\string_Ms`M\string_M1}{}
\markboth{\textcolor{darkblue}{\textbf{\ipa{kʰɤ˧ʂɯ˧}}}}{}
\textcolor{teal}{\mytextsc{verbe}} \hspace{4pt} Ton~: M.

Commencer.
\textcolor{brown}{\zh{开始(汉语借词)。}}

 Emprunt~: chinois \textcolor{brown}{\zh{开始}}



\paragraph{\hspace{-0.5cm} {\Large \textcolor{darkblue}{\textbf{\ipa{kʰɤ˧zo˥\$}}}}} \hypertarget{k\string_h7\string_Mzo\string_T\$1}{}
\markboth{\textcolor{darkblue}{\textbf{\ipa{kʰɤ˧zo˥\$}}}}{}
\textcolor{teal}{\mytextsc{nom}} \hspace{4pt} Ton~: \$H.

Petite hotte.
\textcolor{brown}{\zh{小背篓。}}


 \mytextsc{clf}~: \textcolor{darkblue}{\textbf{\ipa{kʰɤ˧˥}}} 

\paragraph{\hspace{-0.5cm} {\Large \textcolor{darkblue}{\textbf{\ipa{kʰɤ˩njɤ˩\textasciitilde{}kʰɤ˧njɤ˧}}}}} \hypertarget{k\string_h7\string_Bnj7\string_B~k\string_h7\string_Mnj7\string_M1}{}
\markboth{\textcolor{darkblue}{\textbf{\ipa{kʰɤ˩njɤ˩\textasciitilde{}kʰɤ˧njɤ˧}}}}{}
\textcolor{teal}{\mytextsc{adjectif}} \hspace{4pt} Ton~: L-.

Souple (mouvement).
\textcolor{brown}{\zh{柔软(动作)。}}




\paragraph{\hspace{-0.5cm} {\Large \textcolor{darkblue}{\textbf{\ipa{kʰɤ˧˥}}} \textsubscript{1}}} \hypertarget{k\string_h7\string_M\string_T1}{}
\markboth{\textcolor{darkblue}{\textbf{\ipa{kʰɤ˧˥}}} \textsubscript{1}}{}
\textcolor{teal}{\mytextsc{verbe}} \hspace{4pt} Ton~: MH.

Éteindre le foyer.
\textcolor{brown}{\zh{灭(火)。}}


\textit{Voir~:} \hyperlink{hA\string_~\string_M\string_T1}{\textcolor{darkblue}{\textbf{\ipa{hɑ̃˧˥}}} \textsubscript{1}} 

\paragraph{\hspace{-0.5cm} {\Large \textcolor{darkblue}{\textbf{\ipa{kʰɤ˧˥}}} \textsubscript{2}}} \hypertarget{k\string_h7\string_M\string_T2}{}
\markboth{\textcolor{darkblue}{\textbf{\ipa{kʰɤ˧˥}}} \textsubscript{2}}{}
\textcolor{teal}{\mytextsc{nom}} \hspace{4pt} Ton~: MH.

Hotte.
\textcolor{brown}{\zh{背篓。}}


 \mytextsc{clf}~: \textcolor{darkblue}{\textbf{\ipa{kʰɤ˧˥}}} 

\paragraph{\hspace{-0.5cm} {\Large \textcolor{darkblue}{\textbf{\ipa{kʰɤ˧˥\textsubscript{a}}}}}} \hypertarget{k\string_h7\string_M\string_Ta1}{}
\markboth{\textcolor{darkblue}{\textbf{\ipa{kʰɤ˧˥\textsubscript{a}}}}}{}
\textcolor{teal}{\mytextsc{classificateur}} \hspace{4pt} Ton~: MH\textsubscript{a}.

Classificateur des cageots.
\textcolor{brown}{\zh{量词:筐。}}




\paragraph{\hspace{-0.5cm} {\Large \textcolor{darkblue}{\textbf{\ipa{kʰi˥}}}}} \hypertarget{k\string_hi\string_T1}{}
\markboth{\textcolor{darkblue}{\textbf{\ipa{kʰi˥}}}}{}
\textcolor{teal}{\mytextsc{verbe}} \hspace{4pt} Ton~: H.

Séparer, défaire (des fibres de lin: on sépare les fibres pour faire du fil).
\textcolor{brown}{\zh{拆开、分离(几根线)。}}


\begin{exe}
\sn \textcolor{darkblue}{\textbf{\ipa{sɑ˧ | le˧-kʰi˥}}}
\trans défaire des fibres de lin (pour fabriquer du fil)
\trans \textit{\textcolor{brown}{\zh{拆开粗麻(为了纺出细麻线)}}}
\end{exe}


\paragraph{\hspace{-0.5cm} {\Large \textcolor{darkblue}{\textbf{\ipa{kʰi˥}}} \textsubscript{1}}} \hypertarget{k\string_hi\string_T1}{}
\markboth{\textcolor{darkblue}{\textbf{\ipa{kʰi˥}}} \textsubscript{1}}{}
\textcolor{teal}{\mytextsc{nom}} \hspace{4pt} Ton~: \#H.

Porte.
\textcolor{brown}{\zh{门。}}


\begin{exe}
\sn \textcolor{darkblue}{\textbf{\ipa{kʰi˧-zo\#˥}}}
\trans petite porte
\trans \textit{\textcolor{brown}{\zh{小门}}}
\end{exe}
 \mytextsc{clf}~: \textcolor{darkblue}{\textbf{\ipa{̩pɤ˩}}} 

\paragraph{\hspace{-0.5cm} {\Large \textcolor{darkblue}{\textbf{\ipa{kʰi˥}}} \textsubscript{2}}} \hypertarget{k\string_hi\string_T2}{}
\markboth{\textcolor{darkblue}{\textbf{\ipa{kʰi˥}}} \textsubscript{2}}{}
\textcolor{teal}{\mytextsc{nom}} \hspace{4pt} Ton~: \#H.

Bord (monosyllabe).
\textcolor{brown}{\zh{边(单音节)。}}




\paragraph{\hspace{-0.5cm} {\Large \textcolor{darkblue}{\textbf{\ipa{kʰi˧bɤ\#˥}}}}} \hypertarget{k\string_hi\string_Mb7\#\string_T1}{}
\markboth{\textcolor{darkblue}{\textbf{\ipa{kʰi˧bɤ\#˥}}}}{}
\textcolor{teal}{\mytextsc{nom}} \hspace{4pt} Ton~: \#H.

Seuil.
\textcolor{brown}{\zh{门槛。}}


 \mytextsc{clf}~: \textcolor{darkblue}{\textbf{\ipa{ɭɯ˧}}} 

\paragraph{\hspace{-0.5cm} {\Large \textcolor{darkblue}{\textbf{\ipa{kʰi˧-bv̩˧lv̩˩}}}}} \hypertarget{k\string_hi\string_M-bv\string_=\string_Mlv\string_=\string_B1}{}
\markboth{\textcolor{darkblue}{\textbf{\ipa{kʰi˧-bv̩˧lv̩˩}}}}{}
\textcolor{teal}{\mytextsc{nom}} \hspace{4pt} Ton~: \mytextsc{L}\#.

Gonds (d'une porte).
\textcolor{brown}{\zh{(门的)合页。}}


 \mytextsc{clf}~: \textcolor{darkblue}{\textbf{\ipa{ɭɯ˧}}} \textit{Voir~:} \hyperlink{k\string_hi\string_M-q\string_hw7\string_B1}{\textcolor{darkblue}{\textbf{\ipa{kʰi˧-qʰwɤ˩}}}} 

\paragraph{\hspace{-0.5cm} {\Large \textcolor{darkblue}{\textbf{\ipa{-kʰi˧\textasciitilde{}kʰi˧}}}}} \hypertarget{-k\string_hi\string_M~k\string_hi\string_M1}{}
\markboth{\textcolor{darkblue}{\textbf{\ipa{-kʰi˧\textasciitilde{}kʰi˧}}}}{}
\textcolor{teal}{\mytextsc{postposition}} \hspace{4pt} Ton~: \#H.

Aux alentours de, au bord de, auprès de.
\textcolor{brown}{\zh{周围、左右、旁边。}}


\begin{exe}
\sn \textcolor{darkblue}{\textbf{\ipa{ʑi˧qʰwɤ˧-kʰi˧\textasciitilde{}kʰi˧}}}
\trans aux alentours de la maison
\trans \textit{\textcolor{brown}{\zh{房子周围}}}
\end{exe}
\begin{exe}
\sn \textcolor{darkblue}{\textbf{\ipa{[F5] njɤ˧-bv̩˧ | kʰi˧\textasciitilde{}kʰi˧}}}
\trans à côté de moi, autour de moi
\trans \textit{\textcolor{brown}{\zh{我的周围}}}
\end{exe}
\begin{exe}
\sn \textcolor{darkblue}{\textbf{\ipa{[M21] ʐɤ˩mi˩-kʰi˩\textasciitilde{}kʰi˩ se˩˥}}}
\trans marcher au bord de la route
\trans \textit{\textcolor{brown}{\zh{走在马路边}}}
\end{exe}


\paragraph{\hspace{-0.5cm} {\Large \textcolor{darkblue}{\textbf{\ipa{kʰi˧mi˧}}}}} \hypertarget{k\string_hi\string_Mmi\string_M1}{}
\markboth{\textcolor{darkblue}{\textbf{\ipa{kʰi˧mi˧}}}}{}
\textcolor{teal}{\mytextsc{nom}} \hspace{4pt} Ton~: M.

Grande porte (ex.: porte d'entrée d'une maison).
\textcolor{brown}{\zh{大门。}}


 \mytextsc{clf}~: \textcolor{darkblue}{\textbf{\ipa{pɤ˩}}} 

\paragraph{\hspace{-0.5cm} {\Large \textcolor{darkblue}{\textbf{\ipa{kʰi˧qʰv̩\#˥}}}}} \hypertarget{k\string_hi\string_Mq\string_hv\string_=\#\string_T1}{}
\markboth{\textcolor{darkblue}{\textbf{\ipa{kʰi˧qʰv̩\#˥}}}}{}
\textcolor{teal}{\mytextsc{nom}} \hspace{4pt} Ton~: \#H.

Porte.
\textcolor{brown}{\zh{门。}}


\begin{exe}
\sn \textcolor{darkblue}{\textbf{\ipa{kʰi˧qʰv˧ tʰv˧-ɲi˥}}}
\trans parvenir à la porte, atteindre la porte (contexte: retour d'un lointain périple)
\trans \textit{\textcolor{brown}{\zh{到达(家)门(情景:从远方回家,到达家门)}}}
\end{exe}
\begin{exe}
\sn \textcolor{darkblue}{\textbf{\ipa{ɑ˩ʁo˧ kʰi˧qʰv˧ tʰv˧}}}
\trans parvenir à la porte de la maison
\trans \textit{\textcolor{brown}{\zh{到达家门}}}
\end{exe}


\paragraph{\hspace{-0.5cm} {\Large \textcolor{darkblue}{\textbf{\ipa{kʰi˧-qʰwɤ˩}}}}} \hypertarget{k\string_hi\string_M-q\string_hw7\string_B1}{}
\markboth{\textcolor{darkblue}{\textbf{\ipa{kʰi˧-qʰwɤ˩}}}}{}
\textcolor{teal}{\mytextsc{nom}} \hspace{4pt} Ton~: L\#.

Gonds (d'une porte).
\textcolor{brown}{\zh{门的合页。}}


 \mytextsc{clf}~: \textcolor{darkblue}{\textbf{\ipa{ɭɯ˧}}} \textit{Syn~:} \hyperlink{k\string_hi\string_M-bv\string_=\string_Mlv\string_=\string_B1}{\textcolor{darkblue}{\textbf{\ipa{kʰi˧-bv̩˧lv̩˩}}}}. 

\paragraph{\hspace{-0.5cm} {\Large \textcolor{darkblue}{\textbf{\ipa{kʰi˧tɕʰɯ˩-mo˩}}}}} \hypertarget{k\string_hi\string_Mts£\string_hM\string_B-mo\string_B1}{}
\markboth{\textcolor{darkblue}{\textbf{\ipa{kʰi˧tɕʰɯ˩-mo˩}}}}{}
\textcolor{teal}{\mytextsc{nom}} \hspace{4pt} Ton~: L\#-.

Un champignon vénéneux.
\textcolor{brown}{\zh{一种有毒的菌子。}}


\begin{exe}
\sn \textcolor{darkblue}{\textbf{\ipa{ʈʂæ˧mo˧-kʰi˧tɕʰɯ˩-mo˩}}}
\trans même sens
\trans \textit{\textcolor{brown}{\zh{同上}}}
\end{exe}
\textit{Syn~:} \hyperlink{t`s`\{\string_Mmo\#\string_T1}{\textcolor{darkblue}{\textbf{\ipa{ʈʂæ˧mo\#˥}}}}. 

\paragraph{\hspace{-0.5cm} {\Large \textcolor{darkblue}{\textbf{\ipa{kʰi˧˥}}}}} \hypertarget{k\string_hi\string_M\string_T1}{}
\markboth{\textcolor{darkblue}{\textbf{\ipa{kʰi˧˥}}}}{}
\textcolor{teal}{\mytextsc{verbe}} \hspace{4pt} Ton~: MH.

Forme passée du verbe 'partir'.
\textcolor{brown}{\zh{走(过去式)。}}


\begin{exe}
\sn \textcolor{darkblue}{\textbf{\ipa{ʈʂʰɯ˧ | zo˩qo˧ kʰi˧?}}}
\trans Elle/il est parti où?
\trans \textit{\textcolor{brown}{\zh{他到哪里去了?}}}
\end{exe}
\begin{exe}
\sn \textcolor{darkblue}{\textbf{\ipa{[M23] no˧ | tsʰi˧ɲi˧ | ɑ˩pʰo˩˥ | ə˩-kʰi˩˥?}}}
\trans tu es allé faire un tour dehors, aujourd'hui?/tu es sorti, aujourd'hui? (Contexte: question posée par un consultant alors que je le raccompagne après une séance de travail vespérale)
\trans \textit{\textcolor{brown}{\zh{你今天出去了吗?}}}
\end{exe}


\paragraph{\hspace{-0.5cm} {\Large \textcolor{darkblue}{\textbf{\ipa{kʰo˥}}}}} \hypertarget{k\string_ho\string_T1}{}
\markboth{\textcolor{darkblue}{\textbf{\ipa{kʰo˥}}}}{}
\textcolor{teal}{\mytextsc{verbe}} \hspace{4pt} Ton~: H.

Étendre (un matelas), étaler (des fruits, des outils... partout par terre).
\textcolor{brown}{\zh{铺(床……)、铺得满地(果子、工具……)。}}


\begin{exe}
\sn \textcolor{darkblue}{\textbf{\ipa{kʰwæ˧ɻæ˧ kʰo˧}}}
\trans étendre une natte
\trans \textit{\textcolor{brown}{\zh{铺垫子}}}
\end{exe}


\paragraph{\hspace{-0.5cm} {\Large \textcolor{darkblue}{\textbf{\ipa{kʰo˧bɤ˧}}}}} \hypertarget{k\string_ho\string_Mb7\string_M1}{}
\markboth{\textcolor{darkblue}{\textbf{\ipa{kʰo˧bɤ˧}}}}{}
\textcolor{teal}{\mytextsc{nom}} \hspace{4pt} Ton~: M.

Foyer. Mot ancien, qui n'est utilisé que dans un registre soutenu; il désigne un espace de vie.
\textcolor{brown}{\zh{家(文言):母亲生活的空间:有家人,有火塘,有母亲在那里生活的那个空间。}}


\begin{exe}
\sn \textcolor{darkblue}{\textbf{\ipa{dʑi˧kʰi˧ le˧-gwɤ˩ | qo˩ tɑ˧-ze˥, | njɤ˧-ɕi˩ ə˩mɑ˩ kʰo˩bɤ˩ dʑɤ˩. |}}}
\trans xxxx traduction à affiner (chanson au sujet de la nostalgie du foyer)
\end{exe}
\begin{exe}
\sn \textcolor{darkblue}{\textbf{\ipa{dʑi˧kʰi˧ le˧-gwɤ˩ | qo˩ tɑ˧-ze˥, | njɤ˧-ɕi˩ ə˩mɑ˩ kʰo˩bɤ˩-qo˩. |}}}
\trans idem ci-dessus
\end{exe}
\begin{exe}
\sn \textcolor{darkblue}{\textbf{\ipa{dʑi˧kʰi˧ le˧-gwɤ˩ qo˩ tɑ˩-ze˩, | njɤ˧-ɕi˩ ə˩mɑ˩ kʰo˩bɤ˩ dʑɤ˩. |}}}
\trans idem
\end{exe}
 \mytextsc{clf}~: \textcolor{darkblue}{\textbf{\ipa{ɭɯ˧}}} 

\paragraph{\hspace{-0.5cm} {\Large \textcolor{darkblue}{\textbf{\ipa{kʰo˧lo˧}}}}} \hypertarget{k\string_ho\string_Mlo\string_M1}{}
\markboth{\textcolor{darkblue}{\textbf{\ipa{kʰo˧lo˧}}}}{}
\textcolor{teal}{\mytextsc{nom}} \hspace{4pt} Ton~: M.

Moulin à prière (aussi bien les très grands, fixés à des axes verticaux dans les monastères, que les plus petits, tenus à la main).
\textcolor{brown}{\zh{转经筒。}}Dialecte chinois local~: \zh{祈祷轱辘。}


 \mytextsc{clf}~: \textcolor{darkblue}{\textbf{\ipa{ɭɯ˧}}} 

\paragraph{\hspace{-0.5cm} {\Large \textcolor{darkblue}{\textbf{\ipa{kʰɯ˧di˧˥}}}}} \hypertarget{k\string_hM\string_Mdi\string_M\string_T1}{}
\markboth{\textcolor{darkblue}{\textbf{\ipa{kʰɯ˧di˧˥}}}}{}
\textcolor{teal}{\mytextsc{nom}} \hspace{4pt} Ton~: MH\#.

Récipient (terme générique).
\textcolor{brown}{\zh{容器。}}


 \mytextsc{clf}~: \textcolor{darkblue}{\textbf{\ipa{ɭɯ˧}}} 

\paragraph{\hspace{-0.5cm} {\Large \textcolor{darkblue}{\textbf{\ipa{kʰɯ˧dv̩\#˥}}}}} \hypertarget{k\string_hM\string_Mdv\string_=\#\string_T1}{}
\markboth{\textcolor{darkblue}{\textbf{\ipa{kʰɯ˧dv̩\#˥}}}}{}
\textcolor{teal}{\mytextsc{nom}} \hspace{4pt} Ton~: \#H.

Boiteux.
\textcolor{brown}{\zh{跛。}}


\begin{exe}
\sn \textcolor{darkblue}{\textbf{\ipa{kʰɯ˧dv̩˧-hĩ˧}}}
\trans boiteux
\trans \textit{\textcolor{brown}{\zh{跛}}}
\end{exe}
\begin{exe}
\sn \textcolor{darkblue}{\textbf{\ipa{kʰɯ˧dv̩˧-tsʰo˧qʰwɤ˧}}}
\trans démon boiteux
\trans \textit{\textcolor{brown}{\zh{跛鬼}}}
\end{exe}
 \mytextsc{clf}~: \textcolor{darkblue}{\textbf{\ipa{v̩˧}}} 

\paragraph{\hspace{-0.5cm} {\Large \textcolor{darkblue}{\textbf{\ipa{kʰɯ˧dʑɯ˧˥}}}}} \hypertarget{k\string_hM\string_Mdz£M\string_M\string_T1}{}
\markboth{\textcolor{darkblue}{\textbf{\ipa{kʰɯ˧dʑɯ˧˥}}}}{}
\textcolor{teal}{\mytextsc{nom}} \hspace{4pt} Ton~: MH\#.

Bande molletière.
\textcolor{brown}{\zh{裹腿。}}


 \mytextsc{clf}~: \textcolor{darkblue}{\textbf{\ipa{dzi˧}}} paire

\paragraph{\hspace{-0.5cm} {\Large \textcolor{darkblue}{\textbf{\ipa{kʰɯ˧pi˧}}}}} \hypertarget{k\string_hM\string_Mpi\string_M1}{}
\markboth{\textcolor{darkblue}{\textbf{\ipa{kʰɯ˧pi˧}}}}{}
\textcolor{teal}{\mytextsc{verbe}} \hspace{4pt} Ton~: .

Trébucher.
\textcolor{brown}{\zh{绊。}}


\begin{exe}
\sn \textcolor{darkblue}{\textbf{\ipa{njɤ˧ kʰɯ˧pi˧-ze˧!}}}
\trans j'ai trébuché!
\trans \textit{\textcolor{brown}{\zh{我绊了一跤!}}}
\end{exe}


\paragraph{\hspace{-0.5cm} {\Large \textcolor{darkblue}{\textbf{\ipa{kʰɯ˧pʰv̩˩}}}}} \hypertarget{k\string_hM\string_Mp\string_hv\string_=\string_B1}{}
\markboth{\textcolor{darkblue}{\textbf{\ipa{kʰɯ˧pʰv̩˩}}}}{}
\textcolor{teal}{\mytextsc{nom}} \hspace{4pt} Ton~: L\#.
\textit{Archaïque} [\textcolor{brown}{\zh{古语}}] 
Chinois.
\textcolor{brown}{\zh{汉族。}}


 \mytextsc{clf}~: \textcolor{darkblue}{\textbf{\ipa{v̩˧}}} 

\paragraph{\hspace{-0.5cm} {\Large \textcolor{darkblue}{\textbf{\ipa{kʰɯ˧tʰo˧˥}}}}} \hypertarget{k\string_hM\string_Mt\string_ho\string_M\string_T1}{}
\markboth{\textcolor{darkblue}{\textbf{\ipa{kʰɯ˧tʰo˧˥}}}}{}
\textcolor{teal}{\mytextsc{nom}} \hspace{4pt} Ton~: H\#.

Chaîne de fer, pour attacher les chevilles d'un criminel.
\textcolor{brown}{\zh{脚链。}}


\begin{exe}
\sn \textcolor{darkblue}{\textbf{\ipa{kʰɯ˧tʰo˧ kʰɯ˥}}}
\trans mettre les chaînes (aux pieds de quelqu'un)
\trans \textit{\textcolor{brown}{\zh{戴上脚链(在一个人的脚上)}}}
\end{exe}


\paragraph{\hspace{-0.5cm} {\Large \textcolor{darkblue}{\textbf{\ipa{kʰɯ˧tʰv̩\#˥}}}}} \hypertarget{k\string_hM\string_Mt\string_hv\string_=\#\string_T1}{}
\markboth{\textcolor{darkblue}{\textbf{\ipa{kʰɯ˧tʰv̩\#˥}}}}{}
\textcolor{teal}{\mytextsc{nom}} \hspace{4pt} Ton~: \#H.

Pédale du métier à tisser (pour inverser la position verticale des fils de trame entre 2 passages du volant).
\textcolor{brown}{\zh{织布机的脚蹬子=踏板。}}


 \mytextsc{clf}~: \textcolor{darkblue}{\textbf{\ipa{dze˩}}} 

\paragraph{\hspace{-0.5cm} {\Large \textcolor{darkblue}{\textbf{\ipa{kʰɯ˧tsɯ˧bæ˥}}}}} \hypertarget{k\string_hM\string_MtsM\string_Mb\{\string_T1}{}
\markboth{\textcolor{darkblue}{\textbf{\ipa{kʰɯ˧tsɯ˧bæ˥}}}}{}
\textcolor{teal}{\mytextsc{nom}} \hspace{4pt} Ton~: H\#.

Bande de tissu large d'une dizaine de centimères, utilisée autrefois pour serrer le pantalon, qui était très ample; c'était un élément fonctionnel mais également décoratif, préparé en belle étoffe; il provenait des régions tibétaines.
\textcolor{brown}{\zh{绑腿布:用来绑裤腿的一块缠布,也有装饰功能(从藏族地区传过来的)。}}




\paragraph{\hspace{-0.5cm} {\Large \textcolor{darkblue}{\textbf{\ipa{kʰɯ˧tsʰɤ˧˥}}}}} \hypertarget{k\string_hM\string_Mts\string_h7\string_M\string_T1}{}
\markboth{\textcolor{darkblue}{\textbf{\ipa{kʰɯ˧tsʰɤ˧˥}}}}{}
\textcolor{teal}{\mytextsc{nom}} \hspace{4pt} Ton~: MH\#.

Jambe.
\textcolor{brown}{\zh{腿,脚。}}


 \mytextsc{clf}~: \textcolor{darkblue}{\textbf{\ipa{pʰo˧˥}}} 

\paragraph{\hspace{-0.5cm} {\Large \textcolor{darkblue}{\textbf{\ipa{kʰɯ˧ʈʂæ˧˥}}}}} \hypertarget{k\string_hM\string_Mt`s`\{\string_M\string_T1}{}
\markboth{\textcolor{darkblue}{\textbf{\ipa{kʰɯ˧ʈʂæ˧˥}}}}{}
\textcolor{teal}{\mytextsc{nom}} \hspace{4pt} Ton~: MH\#.

Cheville.
\textcolor{brown}{\zh{踝关节。}}


 \mytextsc{clf}~: \textcolor{darkblue}{\textbf{\ipa{ʈʂæ˧˥}}} 

\paragraph{\hspace{-0.5cm} {\Large \textcolor{darkblue}{\textbf{\ipa{kʰɯ˧ʈʂɤ\#˥}}}}} \hypertarget{k\string_hM\string_Mt`s`7\#\string_T1}{}
\markboth{\textcolor{darkblue}{\textbf{\ipa{kʰɯ˧ʈʂɤ\#˥}}}}{}
\textcolor{teal}{\mytextsc{nom}} \hspace{4pt} Ton~: \#H.

Griffes de poulet.
\textcolor{brown}{\zh{鸡爪。}}


\begin{exe}
\sn \textcolor{darkblue}{\textbf{\ipa{kʰɯ˧ʈʂɤ˧ tʰv̩˧-ɭɯ\#˥}}}
\trans \mytextsc{n}+\mytextsc{dem}+\mytextsc{clf}
\trans \textit{\textcolor{brown}{\zh{这只鸡爪}}}
\end{exe}
\begin{exe}
\sn \textcolor{darkblue}{\textbf{\ipa{kʰɯ˧ʈʂɤ˧ tʰv̩˧-ʈv̩˥\#}}}
\trans \mytextsc{n}+\mytextsc{dem}+\mytextsc{clf}
\trans \textit{\textcolor{brown}{\zh{这只鸡爪}}}
\end{exe}
 \mytextsc{clf}~: \textcolor{darkblue}{\textbf{\ipa{ʈv̩˩ / ɭɯ˧}}} 

\paragraph{\hspace{-0.5cm} {\Large \textcolor{darkblue}{\textbf{\ipa{kʰɯ˧ʐɯ˥\$}}}}} \hypertarget{k\string_hM\string_Mz`M\string_T\$1}{}
\markboth{\textcolor{darkblue}{\textbf{\ipa{kʰɯ˧ʐɯ˥\$}}}}{}
\textcolor{teal}{\mytextsc{nom}} \hspace{4pt} Ton~: H\$.

Vin de riz (faiblement alcoolisé).
\textcolor{brown}{\zh{黄酒。}}


 \mytextsc{clf}~: \textcolor{darkblue}{\textbf{\ipa{qʰwɤ˧˥}}} bol

\paragraph{\hspace{-0.5cm} {\Large \textcolor{darkblue}{\textbf{\ipa{kʰɯ˩}}}}} \hypertarget{k\string_hM\string_B1}{}
\markboth{\textcolor{darkblue}{\textbf{\ipa{kʰɯ˩}}}}{}
\textcolor{teal}{\mytextsc{nom}} \hspace{4pt} Ton~: L.

Fil.
\textcolor{brown}{\zh{线。}}


 \mytextsc{clf}~: \textcolor{darkblue}{\textbf{\ipa{kʰɯ˩}}} objets longs

\paragraph{\hspace{-0.5cm} {\Large \textcolor{darkblue}{\textbf{\ipa{kʰɯ˩\textsubscript{b}}}}}} \hypertarget{k\string_hM\string_Bb1}{}
\markboth{\textcolor{darkblue}{\textbf{\ipa{kʰɯ˩\textsubscript{b}}}}}{}
\textcolor{teal}{\mytextsc{classificateur}} \hspace{4pt} Ton~: L\textsubscript{b}.

Brin (d'herbe, de fil, de ficelle…).
\textcolor{brown}{\zh{量词:线(一根、一条)。}}


\begin{exe}
\sn \textcolor{darkblue}{\textbf{\ipa{kʰɯ˧ | ɖɯ˧-kʰɯ˩}}}
\trans un brin de fil
\trans \textit{\textcolor{brown}{\zh{一根线}}}
\end{exe}
\begin{exe}
\sn \textcolor{darkblue}{\textbf{\ipa{zɯ˧ | ɖɯ˧-kʰɯ˩}}}
\trans un brin d'herbe
\trans \textit{\textcolor{brown}{\zh{一根草}}}
\end{exe}
\begin{exe}
\sn \textcolor{darkblue}{\textbf{\ipa{bæ˩ ɖɯ˥-kʰɯ˩}}}
\trans un brin de corde, un bout de corde
\trans \textit{\textcolor{brown}{\zh{一条绳子}}}
\end{exe}
\begin{exe}
\sn \textcolor{darkblue}{\textbf{\ipa{kʰɯ˧ | ʈʂʰɯ˧-kʰɯ˧˥}}}
\trans ce brin (note: schéma tonal irrégulier)
\trans \textit{\textcolor{brown}{\zh{这根线}}}
\end{exe}


\paragraph{\hspace{-0.5cm} {\Large \textcolor{darkblue}{\textbf{\ipa{kʰɯ˩pv̩˩}}}}} \hypertarget{k\string_hM\string_Bpv\string_=\string_B1}{}
\markboth{\textcolor{darkblue}{\textbf{\ipa{kʰɯ˩pv̩˩}}}}{}
\textcolor{teal}{\mytextsc{nom}} \hspace{4pt} Ton~: L.

Navette du métier à tisser; elle est actuellement confectionnée au plus simple, en prenant une tige de tournesol ou un bambou fin.
\textcolor{brown}{\zh{梭,梭子。}}


 \mytextsc{clf}~: \textcolor{darkblue}{\textbf{\ipa{ɭɯ˧}}} \textit{Voir~:} \hyperlink{pv\string_=\string_Mq\string_hw7\string_T1}{\textcolor{darkblue}{\textbf{\ipa{pv̩˧qʰwɤ˥}}}} 

\paragraph{\hspace{-0.5cm} {\Large \textcolor{darkblue}{\textbf{\ipa{kʰɯ˩ʈɯ˩}}}}} \hypertarget{k\string_hM\string_Bt`M\string_B1}{}
\markboth{\textcolor{darkblue}{\textbf{\ipa{kʰɯ˩ʈɯ˩}}}}{}
\textcolor{teal}{\mytextsc{nom}} \hspace{4pt} Ton~: L.

Racine.
\textcolor{brown}{\zh{根。}}


\begin{exe}
\sn \textcolor{darkblue}{\textbf{\ipa{si˧dzi˩-kʰɯ˩ʈɯ˩}}}
\trans racines d'arbre
\trans \textit{\textcolor{brown}{\zh{树根}}}
\end{exe}
 \mytextsc{clf}~: \textcolor{darkblue}{\textbf{\ipa{ʈv̩˩}}} 

\paragraph{\hspace{-0.5cm} {\Large \textcolor{darkblue}{\textbf{\ipa{kʰɯ˧˥}}} \textsubscript{1}}} \hypertarget{k\string_hM\string_M\string_T1}{}
\markboth{\textcolor{darkblue}{\textbf{\ipa{kʰɯ˧˥}}} \textsubscript{1}}{}
\textcolor{teal}{\mytextsc{verbe}} \hspace{4pt} Ton~: MH.
\ding{202} 
Mettre, mettre dans (ex.: mettre de la farine dans une casserole); libérer, lâcher (ex.: un poulet qu'on tenait par les pattes); semer en enfonçant les graines; ranger, remettre à sa place.
\textcolor{brown}{\zh{放,装(如:装进袋里),点种,收下。}}


\begin{exe}
\sn \textcolor{darkblue}{\textbf{\ipa{kʰɯ˩\textasciitilde{}kʰɯ˧˥}}}
\trans \mytextsc{red}
\trans \textit{\textcolor{brown}{\zh{\mytextsc{red}}}}
\end{exe}
\begin{exe}
\sn \textcolor{darkblue}{\textbf{\ipa{qwɤ˧-qo˧ | si˧ tʰi˧-kʰɯ˧˥}}}
\trans mettre/ajouter du bois dans le feu
\trans \textit{\textcolor{brown}{\zh{放木头在火中}}}
\end{exe}
\ding{203} 
Autoriser; a aussi valeur causative, mais ce causatif issu du verbe ``mettre" paraît avoir un sens plus proche de ``laisser": par exemple ``laisser sécher au soleil" plutôt que ``faire sécher au soleil".
\textcolor{brown}{\zh{让,\mytextsc{使动。}}}


\begin{exe}
\sn \textcolor{darkblue}{\textbf{\ipa{kʰv̩˩mi˩ zɯ˩\textasciitilde{}zɯ˩˥, | le˧-ɖæ˥-kʰɯ˩! | hĩ˧-zɯ˧\textasciitilde{}zɯ˥, | le˧-ʂæ˧-kʰɯ˥!}}}
\trans La vie des chiens s'en est trouvée écourtée, et celle des hommes allongée! (Résumé en quelques mots du récit ``Le chien échange sa longévité avec l'homme")
\trans \textit{\textcolor{brown}{\zh{够的寿命,变短了/使得变短!(而)人的寿命,变长了/使得变长!(《狗和人交换寿命》故事的一个提要)}}}
\end{exe}
\begin{exe}
\sn \textcolor{darkblue}{\textbf{\ipa{hwæ˧ kʰɯ˧ ə˥-bi˩? | - hwæ˧ kʰɯ˧-bi˥!}}}
\trans Tu es d'accord pour acheter? - Oui!
\trans \textit{\textcolor{brown}{\zh{(你)让买吗? - 让买!}}}
\end{exe}
\begin{exe}
\sn \textcolor{darkblue}{\textbf{\ipa{tɕʰi˧ kʰɯ˧ ə˥-bi˩?}}}
\trans Tu es d'accord pour vendre?
\trans \textit{\textcolor{brown}{\zh{(你)让卖吗?}}}
\end{exe}
\begin{exe}
\sn \textcolor{darkblue}{\textbf{\ipa{dzɯ˧ kʰɯ˩ ə˩-bi˩?}}}
\trans Tu es d'accord pour manger?
\trans \textit{\textcolor{brown}{\zh{(你)让吃吗?}}}
\end{exe}
\begin{exe}
\sn \textcolor{darkblue}{\textbf{\ipa{tɕi˩ kʰɯ˥ ə˩-bi˩?}}}
\trans Tu es d'accord pour écrire?
\trans \textit{\textcolor{brown}{\zh{(你)让写吗?}}}
\end{exe}
\begin{exe}
\sn \textcolor{darkblue}{\textbf{\ipa{ʈʰɯ˩ kʰɯ˩ ə˥-bi˩?}}}
\trans Tu es d'accord pour boire?
\trans \textit{\textcolor{brown}{\zh{(你)让喝吗?}}}
\end{exe}
\begin{exe}
\sn \textcolor{darkblue}{\textbf{\ipa{ʐv̩˧ kʰɯ˥ ə˩-bi˩?}}}
\trans Tu es d'accord pour coudre?
\trans \textit{\textcolor{brown}{\zh{(你)让缝吗?}}}
\end{exe}


\paragraph{\hspace{-0.5cm} {\Large \textcolor{darkblue}{\textbf{\ipa{kʰɯ˧˥}}} \textsubscript{2}}} \hypertarget{k\string_hM\string_M\string_T2}{}
\markboth{\textcolor{darkblue}{\textbf{\ipa{kʰɯ˧˥}}} \textsubscript{2}}{}
\textcolor{teal}{\mytextsc{verbe}} \hspace{4pt} Ton~: MH.

Lancer, jeter.
\textcolor{brown}{\zh{甩、扔(石头)。}}


\begin{exe}
\sn \textcolor{darkblue}{\textbf{\ipa{le˧-kʰɯ˧-ze˥}}}
\trans \mytextsc{accomp} \string_ \mytextsc{pfv}
\trans \textit{\textcolor{brown}{\zh{甩了}}}
\end{exe}
\begin{exe}
\sn \textcolor{darkblue}{\textbf{\ipa{lv̩˧mi˧ kʰɯ˧˥}}}
\trans jeter une pierre
\trans \textit{\textcolor{brown}{\zh{扔石头}}}
\end{exe}


\paragraph{\hspace{-0.5cm} {\Large \textcolor{darkblue}{\textbf{\ipa{kʰɯ˧˥}}} \textsubscript{3}}} \hypertarget{k\string_hM\string_M\string_T3}{}
\markboth{\textcolor{darkblue}{\textbf{\ipa{kʰɯ˧˥}}} \textsubscript{3}}{}
\textcolor{teal}{\mytextsc{verbe}} \hspace{4pt} Ton~: MH.

Porter (un bracelet).
\textcolor{brown}{\zh{戴(手镯)。}}


\begin{exe}
\sn \textcolor{darkblue}{\textbf{\ipa{le˧-kʰɯ˧-ze˥}}}
\trans \mytextsc{accomp} \string_ \mytextsc{pfv}
\trans \textit{\textcolor{brown}{\zh{戴了}}}
\end{exe}
\begin{exe}
\sn \textcolor{darkblue}{\textbf{\ipa{lo˩dʑo˧ kʰɯ˩}}}
\trans porter un bracelet
\trans \textit{\textcolor{brown}{\zh{戴手镯}}}
\end{exe}


\paragraph{\hspace{-0.5cm} {\Large \textcolor{darkblue}{\textbf{\ipa{kʰv̩˧˥}}}}} \hypertarget{k\string_hv\string_=\string_M\string_T1}{}
\markboth{\textcolor{darkblue}{\textbf{\ipa{kʰv̩˧˥}}}}{}
\textcolor{teal}{\mytextsc{nom}} \hspace{4pt} Ton~: MH.
\ding{202} 
Année, an.
\textcolor{brown}{\zh{年、岁。}}


\begin{exe}
\sn \textcolor{darkblue}{\textbf{\ipa{kʰv̩˧-mæ˥}}}
\trans fin de l'année
\trans \textit{\textcolor{brown}{\zh{年尾}}}
\end{exe}
\begin{exe}
\sn \textcolor{darkblue}{\textbf{\ipa{kʰv̩˧-mæ˥ ʂæ˩}}}
\trans année longue, de 13 mois; cela a lieu tous les 4 ans environ
\trans \textit{\textcolor{brown}{\zh{闰年(有13个月)}}}
\end{exe}
\begin{exe}
\sn \textcolor{darkblue}{\textbf{\ipa{kʰv̩˧-mæ˥ ɖæ˩}}}
\trans année normale, à douze mois
\trans \textit{\textcolor{brown}{\zh{正常的年份,普通年:一年十二个月}}}
\end{exe}
\ding{203} 
Signe astrologique.
\textcolor{brown}{\zh{生肖。}}


\begin{exe}
\sn \textcolor{darkblue}{\textbf{\ipa{no˧ | ə˧tso˧ kʰv̩˧ ɲi˥?}}}
\trans De quel signe es-tu?
\trans \textit{\textcolor{brown}{\zh{你是属什么的?}}}
\end{exe}


\paragraph{\hspace{-0.5cm} {\Large \textcolor{darkblue}{\textbf{\ipa{kʰv̩˥}}} \textsubscript{1}}} \hypertarget{k\string_hv\string_=\string_T1}{}
\markboth{\textcolor{darkblue}{\textbf{\ipa{kʰv̩˥}}} \textsubscript{1}}{}
\textcolor{teal}{\mytextsc{nom}} \hspace{4pt} Ton~: \#H.

Nid (monosyllabe).
\textcolor{brown}{\zh{(鸟)巢。}}


\begin{exe}
\sn \textcolor{darkblue}{\textbf{\ipa{kʰv̩˧ ʈʂʰɯ˧-ɭɯ\#˥}}}
\trans \mytextsc{n}+\mytextsc{dem}+\mytextsc{clf}
\trans \textit{\textcolor{brown}{\zh{这只鸟巢}}}
\end{exe}
 \mytextsc{clf}~: \textcolor{darkblue}{\textbf{\ipa{ɭɯ˧}}} 

\paragraph{\hspace{-0.5cm} {\Large \textcolor{darkblue}{\textbf{\ipa{kʰv̩˥}}} \textsubscript{2}}} \hypertarget{k\string_hv\string_=\string_T2}{}
\markboth{\textcolor{darkblue}{\textbf{\ipa{kʰv̩˥}}} \textsubscript{2}}{}
\textcolor{teal}{\mytextsc{verbe}} \hspace{4pt} Ton~: H.

Couper (ex.: de l'herbe) pour récolter.
\textcolor{brown}{\zh{割(草)。}}


\begin{exe}
\sn \textcolor{darkblue}{\textbf{\ipa{le˧-kʰv̩˥-ze˩}}}
\trans \mytextsc{accomp} \string_ \mytextsc{pfv}
\trans \textit{\textcolor{brown}{\zh{割了}}}
\end{exe}
\begin{exe}
\sn \textcolor{darkblue}{\textbf{\ipa{zɯ˧-kʰv̩˧}}}
\trans couper de l'herbe
\trans \textit{\textcolor{brown}{\zh{割草}}}
\end{exe}


\paragraph{\hspace{-0.5cm} {\Large \textcolor{darkblue}{\textbf{\ipa{kʰv̩˥}}} \textsubscript{3}}} \hypertarget{k\string_hv\string_=\string_T3}{}
\markboth{\textcolor{darkblue}{\textbf{\ipa{kʰv̩˥}}} \textsubscript{3}}{}
\textcolor{teal}{\mytextsc{nom}} \hspace{4pt} Ton~: \#H.

Chien (monosyllabe).
\textcolor{brown}{\zh{狗。}}


\begin{exe}
\sn \textcolor{darkblue}{\textbf{\ipa{kʰv̩˧-ʂe˧ dzɯ˧}}}
\trans manger de la viande de chien (pratique qui va droit à l'encontre de la culture na, dans laquelle le chien est considéré comme bienfaiteur de l'homme)
\trans \textit{\textcolor{brown}{\zh{吃狗肉}}}
\end{exe}
\begin{exe}
\sn \textcolor{darkblue}{\textbf{\ipa{kʰv̩˧-zɯ˧\textasciitilde{}zɯ˥}}}
\trans l'existence du chien, la vie du chien (qu'il a échangée avec l'homme, selon la légende)
\trans \textit{\textcolor{brown}{\zh{狗的生命(传说狗与人交换了生命)}}}
\end{exe}
\begin{exe}
\sn \textcolor{darkblue}{\textbf{\ipa{kʰv̩˧ tʰv̩˧-mi˥\#}}}
\trans \mytextsc{n}+\mytextsc{dem}+\mytextsc{clf}
\trans \textit{\textcolor{brown}{\zh{那条狗}}}
\end{exe}
\begin{exe}
\sn \textcolor{darkblue}{\textbf{\ipa{kʰv̩˧-gɤ˥ljɤ˩}}}
\trans chien errant
\trans \textit{\textcolor{brown}{\zh{流浪狗}}}
\end{exe}
 \mytextsc{clf}~: \textcolor{darkblue}{\textbf{\ipa{mi˩}}} \textcolor{darkblue}{\textbf{\ipa{v̩˧}}} \textcolor{darkblue}{\textbf{\ipa{jɤ˧˥}}} 

\paragraph{\hspace{-0.5cm} {\Large \textcolor{darkblue}{\textbf{\ipa{kʰv̩˥}}} \textsubscript{4}}} \hypertarget{k\string_hv\string_=\string_T4}{}
\markboth{\textcolor{darkblue}{\textbf{\ipa{kʰv̩˥}}} \textsubscript{4}}{}
\textcolor{teal}{\mytextsc{verbe}} \hspace{4pt} Ton~: H.

Voler.
\textcolor{brown}{\zh{偷。}}


\begin{exe}
\sn \textcolor{darkblue}{\textbf{\ipa{hĩ˧-bv̩˧ tso˧\textasciitilde{}tso˧ kʰv̩˧}}}
\trans voler les affaires de quelqu'un
\trans \textit{\textcolor{brown}{\zh{偷别人的东西}}}
\end{exe}


\paragraph{\hspace{-0.5cm} {\Large \textcolor{darkblue}{\textbf{\ipa{kʰv̩˧˥\textsubscript{a}}}}}} \hypertarget{k\string_hv\string_=\string_M\string_Ta1}{}
\markboth{\textcolor{darkblue}{\textbf{\ipa{kʰv̩˧˥\textsubscript{a}}}}}{}
\textcolor{teal}{\mytextsc{classificateur}} \hspace{4pt} Ton~: MH\textsubscript{a}.

Année.
\textcolor{brown}{\zh{量词:年、岁。}}


\begin{exe}
\sn \textcolor{darkblue}{\textbf{\ipa{ɖɯ˧-kʰv̩˧˥}}}
\trans une année
\trans \textit{\textcolor{brown}{\zh{一年}}}
\end{exe}


\paragraph{\hspace{-0.5cm} {\Large \textcolor{darkblue}{\textbf{\ipa{kʰv̩˧bv̩˧˥}}}}} \hypertarget{k\string_hv\string_=\string_Mbv\string_=\string_M\string_T1}{}
\markboth{\textcolor{darkblue}{\textbf{\ipa{kʰv̩˧bv̩˧˥}}}}{}
\textcolor{teal}{\mytextsc{nom}} \hspace{4pt} Ton~: MH\#.

Chenil.
\textcolor{brown}{\zh{狗窝。}}


 \mytextsc{clf}~: \textcolor{darkblue}{\textbf{\ipa{ɭɯ˧}}} 

\paragraph{\hspace{-0.5cm} {\Large \textcolor{darkblue}{\textbf{\ipa{kʰv̩˩-kʰɤ˩}}}}} \hypertarget{k\string_hv\string_=\string_B-k\string_h7\string_B1}{}
\markboth{\textcolor{darkblue}{\textbf{\ipa{kʰv̩˩-kʰɤ˩}}}}{}
\textcolor{teal}{\mytextsc{nom}} \hspace{4pt} Ton~: L.

Nid de poule, pondoir, endroit où la poule pond.
\textcolor{brown}{\zh{鸡窝。}}


 \mytextsc{clf}~: \textcolor{darkblue}{\textbf{\ipa{ɭɯ˧}}} 

\paragraph{\hspace{-0.5cm} {\Large \textcolor{darkblue}{\textbf{\ipa{kʰv̩˧kʰv̩˩}}}}} \hypertarget{k\string_hv\string_=\string_Mk\string_hv\string_=\string_B1}{}
\markboth{\textcolor{darkblue}{\textbf{\ipa{kʰv̩˧kʰv̩˩}}}}{}
\textcolor{teal}{\mytextsc{nom}} \hspace{4pt} Ton~: L\#.

Année du chien.
\textcolor{brown}{\zh{狗年。}}




\paragraph{\hspace{-0.5cm} {\Large \textcolor{darkblue}{\textbf{\ipa{kʰv̩˧kwæ˧}}}}} \hypertarget{k\string_hv\string_=\string_Mkw\{\string_M1}{}
\markboth{\textcolor{darkblue}{\textbf{\ipa{kʰv̩˧kwæ˧}}}}{}
\textcolor{teal}{\mytextsc{nom}} \hspace{4pt} Ton~: M.

Concombre amer.
\textcolor{brown}{\zh{苦瓜。}}

 Emprunt~: chinois \textcolor{brown}{\zh{苦瓜}}

 \mytextsc{clf}~: \textcolor{darkblue}{\textbf{\ipa{ɭɯ˧}}} 

\paragraph{\hspace{-0.5cm} {\Large \textcolor{darkblue}{\textbf{\ipa{kʰv̩˧mæ˧}}}}} \hypertarget{k\string_hv\string_=\string_Mm\{\string_M1}{}
\markboth{\textcolor{darkblue}{\textbf{\ipa{kʰv̩˧mæ˧}}}}{}
\textcolor{teal}{\mytextsc{nom}} \hspace{4pt} Ton~: M.

Voleur, bandit.
\textcolor{brown}{\zh{强盗。}}


\begin{exe}
\sn \textcolor{darkblue}{\textbf{\ipa{kʰv̩˧mæ˧ ʝi˧-hĩ˧-hĩ˧}}}
\trans personne qui est un bandit, bandit
\trans \textit{\textcolor{brown}{\zh{当强盗的人=强盗}}}
\end{exe}
\begin{exe}
\sn \textcolor{darkblue}{\textbf{\ipa{kʰv̩˧mæ˧-ni˩-zo˩! | hĩ˧ lɑ˩-ho˩!}}}
\trans Ca doit être un voleur! Il se pourrait qu'il frappe les gens!
\trans \textit{\textcolor{brown}{\zh{他像强盗似的!会打人的!}}}
\end{exe}
\begin{exe}
\sn \textcolor{darkblue}{\textbf{\ipa{kʰv̩˧mæ˧-ʑi˩}}}
\trans prison: littéralement ``maison des voleurs"
\trans \textit{\textcolor{brown}{\zh{监狱。直译:“贼家”}}}
\end{exe}
\begin{exe}
\sn \textcolor{darkblue}{\textbf{\ipa{kʰv̩˧mæ˧-ʝi˧-hĩ˧, | lo˧ʑi˥bv̩˩-qo˩ ʈæ˩!}}}
\trans les voleurs, on les met en prison!
\trans \textit{\textcolor{brown}{\zh{贼,被关在监狱!}}}
\end{exe}
\begin{exe}
\sn \textcolor{darkblue}{\textbf{\ipa{no˧ | kʰv̩˧mæ˧-pʰæ˧qʰwɤ˩-ne˩-ʝi˩-zo˩!}}}
\trans Toi, tu m'as une tête de voleur! (accusation lancée à quelqu'un qu'on pense être un voleur)
\trans \textit{\textcolor{brown}{\zh{你有一张贼脸!(控告一个人)}}}
\end{exe}
 \mytextsc{clf}~: \textcolor{darkblue}{\textbf{\ipa{v̩˧}}} 

\paragraph{\hspace{-0.5cm} {\Large \textcolor{darkblue}{\textbf{\ipa{kʰv̩˩mi˩}}}}} \hypertarget{k\string_hv\string_=\string_Bmi\string_B1}{}
\markboth{\textcolor{darkblue}{\textbf{\ipa{kʰv̩˩mi˩}}}}{}
\textcolor{teal}{\mytextsc{nom}} \hspace{4pt} Ton~: L.

Chien (sans spécifier le sexe).
\textcolor{brown}{\zh{狗。}}


\begin{exe}
\sn \textcolor{darkblue}{\textbf{\ipa{kʰv̩˩mi˩ ʈʂʰɯ˩-jɤ˧}}}
\trans \mytextsc{n}+\mytextsc{dem}+\mytextsc{clf}
\trans \textit{\textcolor{brown}{\zh{这条狗}}}
\end{exe}
\begin{exe}
\sn \textcolor{darkblue}{\textbf{\ipa{di˧qo˧-kʰv̩˩mi˩}}}
\trans les chiens de la plaine (qui à la différence des chiens des petits hameaux de montagne voient beaucoup de passage et sont moins susceptibles de mordre les inconnus de passage)
\trans \textit{\textcolor{brown}{\zh{平坝的狗}}}
\end{exe}
\begin{exe}
\sn \textcolor{darkblue}{\textbf{\ipa{kʰv̩˩mi˩-gɤ˥ljɤ˩}}}
\trans chien errant
\trans \textit{\textcolor{brown}{\zh{流浪的狗}}}
\end{exe}
 \mytextsc{clf}~: \textcolor{darkblue}{\textbf{\ipa{v̩˧, terme respectueux (le même que pour les humains)}}} \textcolor{darkblue}{\textbf{\ipa{on peut aussi dire: jɤ˧˥}}} 

\paragraph{\hspace{-0.5cm} {\Large \textcolor{darkblue}{\textbf{\ipa{kʰv̩˧mv̩˥}}}}} \hypertarget{k\string_hv\string_=\string_Mmv\string_=\string_T1}{}
\markboth{\textcolor{darkblue}{\textbf{\ipa{kʰv̩˧mv̩˥}}}}{}
\textcolor{teal}{\mytextsc{nom}} \hspace{4pt} Ton~: H\#.

Chienne, petit chiot femelle. Le terme est également employé comme nom provisoire pour les fillettes pendant leurs premiers mois, avant qu'on ne leur donne un vrai nom. Le vilain nom dont on l'affuble vise à éviter que le nourrisson ne soit repéré par de mauvais esprits. (Actuellement, l'état-civil nécessite qu'un nom soit donné dès la naissance; mais celui-ci ne commence à être employé dans les conversations familiales qu'une fois passés les premiers mois.).
\textcolor{brown}{\zh{小母狗(给刚出生的女孩起的名字,让鬼对她不感兴趣,不会来害小孩)。}}


 \mytextsc{clf}~: \textcolor{darkblue}{\textbf{\ipa{v̩˧}}} 

\paragraph{\hspace{-0.5cm} {\Large \textcolor{darkblue}{\textbf{\ipa{kʰv̩˧nɑ˥}}}}} \hypertarget{k\string_hv\string_=\string_MnA\string_T1}{}
\markboth{\textcolor{darkblue}{\textbf{\ipa{kʰv̩˧nɑ˥}}}}{}
\textcolor{teal}{\mytextsc{nom}} \hspace{4pt} Ton~: H\#.

Chien (registre de langage relevé).
\textcolor{brown}{\zh{狗。}}


 \mytextsc{clf}~: \textcolor{darkblue}{\textbf{\ipa{mi˩}}} 

\paragraph{\hspace{-0.5cm} {\Large \textcolor{darkblue}{\textbf{\ipa{kʰv̩˧pʰæ˧}}}}} \hypertarget{k\string_hv\string_=\string_Mp\string_h\{\string_M1}{}
\markboth{\textcolor{darkblue}{\textbf{\ipa{kʰv̩˧pʰæ˧}}}}{}
\textcolor{teal}{\mytextsc{nom}} \hspace{4pt} Ton~: M.

Âge.
\textcolor{brown}{\zh{年龄。}}


\begin{exe}
\sn \textcolor{darkblue}{\textbf{\ipa{kʰv̩˧pʰæ˧ tɕi˩}}}
\trans jeune
\trans \textit{\textcolor{brown}{\zh{年轻}}}
\end{exe}
\begin{exe}
\sn \textcolor{darkblue}{\textbf{\ipa{kʰv̩˧pʰæ˧ | tɕi˩-hĩ˩˥}}}
\trans jeune
\trans \textit{\textcolor{brown}{\zh{年轻的}}}
\end{exe}


\paragraph{\hspace{-0.5cm} {\Large \textcolor{darkblue}{\textbf{\ipa{kʰv̩˧-pʰo˥}}}}} \hypertarget{k\string_hv\string_=\string_M-p\string_ho\string_T1}{}
\markboth{\textcolor{darkblue}{\textbf{\ipa{kʰv̩˧-pʰo˥}}}}{}
\textcolor{teal}{\mytextsc{nom}} \hspace{4pt} Ton~: H\#.

Une demi-année.
\textcolor{brown}{\zh{半年。}}


\begin{exe}
\sn \textcolor{darkblue}{\textbf{\ipa{ɖɯ˧-kʰv̩˧-kʰv̩˥-pʰo˩}}}
\trans un an et demi
\trans \textit{\textcolor{brown}{\zh{一年半}}}
\end{exe}


\paragraph{\hspace{-0.5cm} {\Large \textcolor{darkblue}{\textbf{\ipa{kʰv̩˧pʰv̩\#˥}}}}} \hypertarget{k\string_hv\string_=\string_Mp\string_hv\string_=\#\string_T1}{}
\markboth{\textcolor{darkblue}{\textbf{\ipa{kʰv̩˧pʰv̩\#˥}}}}{}
\textcolor{teal}{\mytextsc{nom}} \hspace{4pt} Ton~: \#H.

Chien mâle (forme élicitée).
\textcolor{brown}{\zh{公狗。}}


\begin{exe}
\sn \textcolor{darkblue}{\textbf{\ipa{kʰv̩˧pʰv̩˧ ʈʂʰɯ˧-ɭɯ\#˥}}}
\trans \mytextsc{n}+\mytextsc{dem}+\mytextsc{clf}
\trans \textit{\textcolor{brown}{\zh{这只公狗}}}
\end{exe}
\begin{exe}
\sn \textcolor{darkblue}{\textbf{\ipa{kʰv̩˧pʰv̩˧ tʰv̩˧-mi˧˥}}}
\trans \mytextsc{n}+\mytextsc{dem}+\mytextsc{clf}
\trans \textit{\textcolor{brown}{\zh{这只公狗}}}
\end{exe}
\begin{exe}
\sn \textcolor{darkblue}{\textbf{\ipa{kʰv̩˧pʰv̩˧ tʰv̩˧-v̩\#˥}}}
\trans \mytextsc{n}+\mytextsc{dem}+\mytextsc{clf}
\trans \textit{\textcolor{brown}{\zh{这个公狗}}}
\end{exe}
 \mytextsc{clf}~: \textcolor{darkblue}{\textbf{\ipa{v̩˧ / mi˩ / ɭɯ˧}}} 

\paragraph{\hspace{-0.5cm} {\Large \textcolor{darkblue}{\textbf{\ipa{kʰv̩˧qʰwɤ˧˥}}}}} \hypertarget{k\string_hv\string_=\string_Mq\string_hw7\string_M\string_T1}{}
\markboth{\textcolor{darkblue}{\textbf{\ipa{kʰv̩˧qʰwɤ˧˥}}}}{}
\textcolor{teal}{\mytextsc{nom}} \hspace{4pt} Ton~: MH\#.

Mauvaise année, année de disette.
\textcolor{brown}{\zh{庄稼收成不好的(一)年。}}


\begin{exe}
\sn \textcolor{darkblue}{\textbf{\ipa{kʰv̩˧qʰwɤ˧ tʰv̩˧˥}}}
\trans une mauvaise année a lieu, une année de mauvaise récolte/de disette
\trans \textit{\textcolor{brown}{\zh{今年,收成不好。}}}
\end{exe}


\paragraph{\hspace{-0.5cm} {\Large \textcolor{darkblue}{\textbf{\ipa{kʰv̩˧ʂæ˧˥}}}}} \hypertarget{k\string_hv\string_=\string_Ms`\{\string_M\string_T1}{}
\markboth{\textcolor{darkblue}{\textbf{\ipa{kʰv̩˧ʂæ˧˥}}}}{}
\textcolor{teal}{\mytextsc{verbe}} \hspace{4pt} Ton~: MH.

Chasser; mener un chien de chasse.
\textcolor{brown}{\zh{打猎、赶走、驱逐。}}


\begin{exe}
\sn \textcolor{darkblue}{\textbf{\ipa{kʰv̩˧ʂæ˧ hɯ˧˥}}}
\trans (Elle/il) est parti(e) chasser
\trans \textit{\textcolor{brown}{\zh{狩猎去了}}}
\end{exe}


\paragraph{\hspace{-0.5cm} {\Large \textcolor{darkblue}{\textbf{\ipa{kʰv̩˧ʂɯ˥}}}}} \hypertarget{k\string_hv\string_=\string_Ms`M\string_T1}{}
\markboth{\textcolor{darkblue}{\textbf{\ipa{kʰv̩˧ʂɯ˥}}}}{}
\textcolor{teal}{\mytextsc{verbe}} \hspace{4pt} Ton~: .

Fêter le Nouvel An.
\textcolor{brown}{\zh{过年。}}




\paragraph{\hspace{-0.5cm} {\Large \textcolor{darkblue}{\textbf{\ipa{kʰv̩˧sɯ˧sɯ˩}}}}} \hypertarget{k\string_hv\string_=\string_MsM\string_MsM\string_B1}{}
\markboth{\textcolor{darkblue}{\textbf{\ipa{kʰv̩˧sɯ˧sɯ˩}}}}{}
\textcolor{teal}{\mytextsc{nom}} \hspace{4pt} Ton~: L\#.

Plante à fleurs, \textit{Flemingia strobilifera}, aussi appelée \textit{Moghania fruticulosa} (nom en chinois local: ``oreille de souris", du fait de la forme de la feuille).
\textcolor{brown}{\zh{球穗千斤拔、半灌木千斤拔、大苞千斤拔。}}Dialecte chinois local~: \zh{耗子耳朵。}


 \mytextsc{clf}~: \textcolor{darkblue}{\textbf{\ipa{kɤ˧˥}}} 

\paragraph{\hspace{-0.5cm} {\Large \textcolor{darkblue}{\textbf{\ipa{kʰv̩˧tɕʰi˥\$}}}}} \hypertarget{k\string_hv\string_=\string_Mts£\string_hi\string_T\$1}{}
\markboth{\textcolor{darkblue}{\textbf{\ipa{kʰv̩˧tɕʰi˥\$}}}}{}
\textcolor{teal}{\mytextsc{nom}} \hspace{4pt} Ton~: H\$.

Solution, méthode.
\textcolor{brown}{\zh{办法。}}


\begin{exe}
\sn \textcolor{darkblue}{\textbf{\ipa{kʰv̩˧tɕʰi˥ | mɤ˧-dʑo˧-ze˧! | ɻ̃˧-ɻ̍˧ tʰo˩!}}}
\trans Il n'y a plus rien à faire! C'est la catastrophe!
\trans \textit{\textcolor{brown}{\zh{没有办法了!糟糕了!}}}
\end{exe}


\paragraph{\hspace{-0.5cm} {\Large \textcolor{darkblue}{\textbf{\ipa{kʰv̩˧tsʰi˧-bo˥tsʰi˩}}}}} \hypertarget{k\string_hv\string_=\string_Mts\string_hi\string_M-bo\string_Tts\string_hi\string_B1}{}
\markboth{\textcolor{darkblue}{\textbf{\ipa{kʰv̩˧tsʰi˧-bo˥tsʰi˩}}}}{}
\textcolor{teal}{\mytextsc{nom}} \hspace{4pt} Ton~: \#H-.

Taupe.
\textcolor{brown}{\zh{鼹鼠。}}


 \mytextsc{clf}~: \textcolor{darkblue}{\textbf{\ipa{pʰo˧˥}}} \textcolor{darkblue}{\textbf{\ipa{v̩˧}}} 

\paragraph{\hspace{-0.5cm} {\Large \textcolor{darkblue}{\textbf{\ipa{kʰv̩˩tsɤ˩mi˥}}}}} \hypertarget{k\string_hv\string_=\string_Bts7\string_Bmi\string_T1}{}
\markboth{\textcolor{darkblue}{\textbf{\ipa{kʰv̩˩tsɤ˩mi˥}}}}{}
\textcolor{teal}{\mytextsc{nom}} \hspace{4pt} Ton~: L+H\#.

Chienne.
\textcolor{brown}{\zh{母狗。}}


 \mytextsc{clf}~: \textcolor{darkblue}{\textbf{\ipa{v̩˧}}} 

\paragraph{\hspace{-0.5cm} {\Large \textcolor{darkblue}{\textbf{\ipa{}}}}} \hypertarget{k\string_hv\string_=\string_Mzo\string_M | -bo\string_Bzo\string_M1}{}
\markboth{\textcolor{darkblue}{\textbf{\ipa{}}}}{}
\textcolor{teal}{\mytextsc{nom}} \hspace{4pt} Ton~: \mytextsc{LM}+\#H.

Chien-cochon, cochon-clébard. Terme employé comme nom provisoire pour les garçonnets pendant leurs premiers mois, avant qu'on ne leur donne un vrai nom. Le vilain nom dont on l'affuble vise à éviter que le nourrisson ne soit repéré par de mauvais esprits. (Actuellement, l'état-civil nécessite qu'un nom soit donné dès la naissance; mais celui-ci ne commence à être employé dans les conversations familiales qu'une fois passés les premiers mois.).
\textcolor{brown}{\zh{小畜生,直译:猪崽子、狗崽子。}}


 \mytextsc{clf}~: \textcolor{darkblue}{\textbf{\ipa{v̩˧}}} 

\paragraph{\hspace{-0.5cm} {\Large \textcolor{darkblue}{\textbf{\ipa{kʰv̩˧zo˥\$}}}}} \hypertarget{k\string_hv\string_=\string_Mzo\string_T\$1}{}
\markboth{\textcolor{darkblue}{\textbf{\ipa{kʰv̩˧zo˥\$}}}}{}
\textcolor{teal}{\mytextsc{nom}} \hspace{4pt} Ton~: H\$.

Nom de clan/famille étendue. Deux familles portent ce nom à Yongning.
\textcolor{brown}{\zh{一个姓。这个姓,永宁有两家。}}


\begin{exe}
\sn \textcolor{darkblue}{\textbf{\ipa{kʰv̩˧zo˧=ɻ̍˥\$}}}
\trans La famille \textcolor{darkblue}{\textbf{\ipa{/kʰv̩˧zo˥\$/}}}, les \textcolor{darkblue}{\textbf{\ipa{/kʰv̩˧zo˥\$/}}}
\trans \textit{\textcolor{brown}{\zh{\textcolor{darkblue}{\textbf{\ipa{/kʰv̩˧zo˥\$/}}}家族}}}
\end{exe}
\begin{exe}
\sn \textcolor{darkblue}{\textbf{\ipa{kʰv̩˧zo˥-tsʰɯ˩ɻ̍˩}}}
\trans nom d'une personne, comportant un nom de famille (\textcolor{darkblue}{\textbf{\ipa{/kʰv̩˧zo˥\$/}}}) et un prénom (\textcolor{darkblue}{\textbf{\ipa{/tsʰɯ˧ɻ\#˥/}}})
\trans \textit{\textcolor{brown}{\zh{一个人的名字:姓为\textcolor{darkblue}{\textbf{\ipa{/kʰv̩˧zo˥\$/}}},名为\textcolor{darkblue}{\textbf{\ipa{/tsʰɯ˧ɻ\#˥/}}}}}}
\end{exe}


\paragraph{\hspace{-0.5cm} {\Large \textcolor{darkblue}{\textbf{\ipa{kʰv̩˧zo\#˥}}}}} \hypertarget{k\string_hv\string_=\string_Mzo\#\string_T1}{}
\markboth{\textcolor{darkblue}{\textbf{\ipa{kʰv̩˧zo\#˥}}}}{}
\textcolor{teal}{\mytextsc{nom}} \hspace{4pt} Ton~: \#H.

Chien (mâle), chiot. Le terme est également employé comme nom provisoire pour les garçonnets pendant leurs premiers mois, avant qu'on ne leur donne un vrai nom. Le vilain nom dont on l'affuble vise à éviter que le nourrisson ne soit repéré par de mauvais esprits. (Actuellement, l'état-civil nécessite qu'un nom soit donné dès la naissance; mais celui-ci ne commence à être employé dans les conversations familiales qu'une fois passés les premiers mois.).
\textcolor{brown}{\zh{公狗(给刚出生的男孩子的名字,让鬼对他不感兴趣,不过来害小孩)。}}


\begin{exe}
\sn \textcolor{darkblue}{\textbf{\ipa{kʰv̩˧zo˧ ʈʂʰɯ˧-ɭɯ\#˥}}}
\trans \mytextsc{n}+\mytextsc{dem}+\mytextsc{clf}
\trans \textit{\textcolor{brown}{\zh{这只公狗}}}
\end{exe}
\begin{exe}
\sn \textcolor{darkblue}{\textbf{\ipa{kʰv̩˧zo˧ tʰv̩˧-mi˧˥}}}
\trans \mytextsc{n}+\mytextsc{dem}+\mytextsc{clf}
\trans \textit{\textcolor{brown}{\zh{这只公狗}}}
\end{exe}
\begin{exe}
\sn \textcolor{darkblue}{\textbf{\ipa{kʰv̩˧zo˧ tʰv̩˧-v̩\#˥}}}
\trans \mytextsc{n}+\mytextsc{dem}+\mytextsc{clf}
\trans \textit{\textcolor{brown}{\zh{这只公狗}}}
\end{exe}
\begin{exe}
\sn \textcolor{darkblue}{\textbf{\ipa{kʰv̩˧zo˥-kʰv̩˩mv̩˩}}}
\trans chien et chienne
\trans \textit{\textcolor{brown}{\zh{小狗与母狗}}}
\end{exe}
 \mytextsc{clf}~: \textcolor{darkblue}{\textbf{\ipa{v̩˧ / mi˩ / ɭɯ˧}}} 

\paragraph{\hspace{-0.5cm} {\Large \textcolor{darkblue}{\textbf{\ipa{kʰwæ˧hwæ˩}}}}} \hypertarget{k\string_hw\{\string_Mhw\{\string_B1}{}
\markboth{\textcolor{darkblue}{\textbf{\ipa{kʰwæ˧hwæ˩}}}}{}
\textcolor{teal}{\mytextsc{verbe}} \hspace{4pt} Ton~: .

Se draper de, endosser, mettre sur son dos.
\textcolor{brown}{\zh{披上。}}




\paragraph{\hspace{-0.5cm} {\Large \textcolor{darkblue}{\textbf{\ipa{kʰwæ˧ɻæ\#˥}}}}} \hypertarget{k\string_hw\{\string_Mr£`\{\#\string_T1}{}
\markboth{\textcolor{darkblue}{\textbf{\ipa{kʰwæ˧ɻæ\#˥}}}}{}
\textcolor{teal}{\mytextsc{nom}} \hspace{4pt} Ton~: \#H.

Feutre; par extension: natte, tapis (même en vannerie), coussin….
\textcolor{brown}{\zh{毡子。也用来指席子,垫子等。}}


\begin{exe}
\sn \textcolor{darkblue}{\textbf{\ipa{kʰwæ˧ɻæ˧ tʰi˧-kʰo˥}}}
\trans étendre la natte
\trans \textit{\textcolor{brown}{\zh{铺席子}}}
\end{exe}
 \mytextsc{clf}~: \textcolor{darkblue}{\textbf{\ipa{tsʰi˥}}} 

\paragraph{\hspace{-0.5cm} {\Large \textcolor{darkblue}{\textbf{\ipa{kʰwɤ˥\textsubscript{a}}}}}} \hypertarget{k\string_hw7\string_Ta1}{}
\markboth{\textcolor{darkblue}{\textbf{\ipa{kʰwɤ˥\textsubscript{a}}}}}{}
\textcolor{teal}{\mytextsc{classificateur}} \hspace{4pt} Ton~: H\textsubscript{a}.

Classificateur des morceaux/bouchées.
\textcolor{brown}{\zh{量词:块。一块肉、一口饭。}}


\begin{exe}
\sn \textcolor{darkblue}{\textbf{\ipa{ɖɯ˧-kʰwɤ˥\textasciitilde{}ɖɯ˩-kʰwɤ˩}}}
\trans par petites bouchées, par petits morceaux
\trans \textit{\textcolor{brown}{\zh{一块一块地}}}
\end{exe}
\begin{exe}
\sn \textcolor{darkblue}{\textbf{\ipa{kʰwɤ˧ | ɖɯ˧-ʂe˧-ɻ̍˩!}}}
\trans Décidez! / Il faut vous décider!
\trans \textit{\textcolor{brown}{\zh{你们得要做出决定!}}}
\end{exe}
\begin{exe}
\sn \textcolor{darkblue}{\textbf{\ipa{ɖɯ˧-kʰwɤ˧ so˧˥, | ɖɯ˧-kʰwɤ˥ fv̩˩!}}}
\trans Chaque mot appris représente une joie de plus! (Commentaire de la locutrice au sujet du travail de l'enquêteur. Tenant en main le manuscrit de ce dictionnaire, elle commente: cela représente un travail immense; l'important est que l'enquêteur y trouve de l'intérêt: chaque information nouvelle – chaque ``morceau" de langue – ajoutée au dictionnaire est une joie pour l'enquêteur.)
\trans \textit{\textcolor{brown}{\zh{学一点,就高兴一点!(评说语言调查工作:合作人看着本词典的初稿,说:这是一项很大的工程,关键的是调查者要有兴趣,欣赏每个新学的语言信息。)}}}
\end{exe}


\paragraph{\hspace{-0.5cm} {\Large \textcolor{darkblue}{\textbf{\ipa{kʰwɤ˧pʰv̩˧}}}}} \hypertarget{k\string_hw7\string_Mp\string_hv\string_=\string_M1}{}
\markboth{\textcolor{darkblue}{\textbf{\ipa{kʰwɤ˧pʰv̩˧}}}}{}
\textcolor{teal}{\mytextsc{nom}} \hspace{4pt} Ton~: M.

Pré: soit prairie de plaine, soit prairie d'altitude (alpage).
\textcolor{brown}{\zh{草坪、草地。}}




\paragraph{\hspace{-0.5cm} {\Large \textcolor{darkblue}{\textbf{\ipa{kʰwɤ˧pʰv̩˧-mo˧˥}}}}} \hypertarget{k\string_hw7\string_Mp\string_hv\string_=\string_M-mo\string_M\string_T1}{}
\markboth{\textcolor{darkblue}{\textbf{\ipa{kʰwɤ˧pʰv̩˧-mo˧˥}}}}{}
\textcolor{teal}{\mytextsc{nom}} \hspace{4pt} Ton~: MH\#.

Champignon des prés: une sorte de champignon comestible (pas encore identifiée): agaric champêtre ou rosé des prés, \textit{Agaricus campestris}?
\textcolor{brown}{\zh{可以吃的一种菌子:可能是四孢蘑菇。直译:“草坪菌”。}}




\newpage
\section*{\centering- \textcolor{darkblue}{\textbf{\ipa{l}}} \textcolor{darkblue}{\textbf{\ipa{ɭ}}} -}
\paragraph{\hspace{-0.5cm} {\Large \textcolor{darkblue}{\textbf{\ipa{‑lɑ˧}}} \textsubscript{1}}} \hypertarget{‑lA\string_M1}{}
\markboth{\textcolor{darkblue}{\textbf{\ipa{‑lɑ˧}}} \textsubscript{1}}{}
\textcolor{teal}{\mytextsc{adverbe}} \hspace{4pt} Ton~: 0.

Seulement.
\textcolor{brown}{\zh{只,才。}}


\begin{exe}
\sn \textcolor{darkblue}{\textbf{\ipa{ʈʂʰɯ˧-lɑ˩ ɲi˩-ze˩-mæ˩!}}}
\trans C'est tout ! / Voilà tout !
\trans \textit{\textcolor{brown}{\zh{就这些了! / 就这些而已! / 就这样!}}}
\end{exe}


\paragraph{\hspace{-0.5cm} {\Large \textcolor{darkblue}{\textbf{\ipa{‑lɑ˧}}} \textsubscript{2}}} \hypertarget{‑lA\string_M2}{}
\markboth{\textcolor{darkblue}{\textbf{\ipa{‑lɑ˧}}} \textsubscript{2}}{}
\textcolor{teal}{\mytextsc{adverbe}} \hspace{4pt} Ton~: 0.

Et, aussi.
\textcolor{brown}{\zh{和、与、跟。}}


\begin{exe}
\sn \textcolor{darkblue}{\textbf{\ipa{ɖɯ˧-kʰv̩˧-lɑ˥ | so˩-ɬi˩˥}}}
\trans un an et trois mois (contexte: on indique l'âge d'un petit enfant)
\trans \textit{\textcolor{brown}{\zh{一岁三个月}}}
\end{exe}
\begin{exe}
\sn \textcolor{darkblue}{\textbf{\ipa{ʈʂʰɯ˧-lɑ˧ | mɤ˧-bi˧, | njɤ˧-lɑ˧ mɤ˧-bi˧!}}}
\trans s'il n'y va pas, moi non plus!
\trans \textit{\textcolor{brown}{\zh{他不去(的话),我也不去!}}}
\end{exe}
\begin{exe}
\sn \textcolor{darkblue}{\textbf{\ipa{hĩ˧-lɑ˩ | dʑɤ˧˥, | mv̩˧di˧-lɑ˥ | dʑɤ˧˥! / hĩ˧-lɑ˩ | dʑɤ˧˥, | lv̩˧-lɑ˧ | dʑɤ˧˥!}}}
\trans les gens (y) sont bons, (et) la terre (y) est bonne (formule de recommandation pour la famille que va rejoindre une jeune femme lors de son mariage)
\trans \textit{\textcolor{brown}{\zh{人也好,田也好!(习语:将女孩嫁出去前,一家人打听对方家如何,推荐的人保证:“他们家,人也好,田也好!”)}}}
\end{exe}
\begin{exe}
\sn \textcolor{darkblue}{\textbf{\ipa{hĩ˧ F | dʑɤ˧˥, | mv̩˧di˧˥ F | dʑɤ˧˥!}}}
\trans même sens
\trans \textit{\textcolor{brown}{\zh{同上}}}
\end{exe}
\begin{exe}
\sn \textcolor{darkblue}{\textbf{\ipa{mɤ˧-lɑ˧ dʑɤ˧˥!}}}
\trans la graisse aussi (y) est bonne! (variation élicitée à partir des exemples qui précèdent)
\trans \textit{\textcolor{brown}{\zh{猪油也好!(按照上面例子的变体)}}}
\end{exe}
\begin{exe}
\sn \textcolor{darkblue}{\textbf{\ipa{qæ˩-lɑ˥ | dʑɤ˧˥!}}}
\trans l’huile aussi (y) est bonne! (variation à partir de l'exemple qui précède)
\trans \textit{\textcolor{brown}{\zh{油也好!}}}
\end{exe}
\begin{exe}
\sn \textcolor{darkblue}{\textbf{\ipa{ʈʂʰɯ˧-lɑ˧ | mɤ˧-bi˧, | njɤ˧ | mɤ˧-bi˧-ze˧! / ʈʰɯ˧ mɤ˧-bi˧-ze˧-dʑo˧, | njɤ˧-lɑ˧ | mɤ˧-bi˧-ze˧!}}}
\trans s'il n'y va pas, moi non plus!
\trans \textit{\textcolor{brown}{\zh{他如果不去,我也不去!}}}
\end{exe}


\paragraph{\hspace{-0.5cm} {\Large \textcolor{darkblue}{\textbf{\ipa{lɑ˧}}}}} \hypertarget{lA\string_M1}{}
\markboth{\textcolor{darkblue}{\textbf{\ipa{lɑ˧}}}}{}
\textcolor{teal}{\mytextsc{nom}} \hspace{4pt} Ton~: M.

Tigre.
\textcolor{brown}{\zh{老虎。}}


 \mytextsc{clf}~: \textcolor{darkblue}{\textbf{\ipa{pʰo˧˥}}} 

\paragraph{\hspace{-0.5cm} {\Large \textcolor{darkblue}{\textbf{\ipa{lɑ˧bi\#˥}}}}} \hypertarget{lA\string_Mbi\#\string_T1}{}
\markboth{\textcolor{darkblue}{\textbf{\ipa{lɑ˧bi\#˥}}}}{}
\textcolor{teal}{\mytextsc{nom}} \hspace{4pt} Ton~: \#H.

Escarpement, pente raide, terrain escarpé.
\textcolor{brown}{\zh{陡坡、土坡、斜坡。}}


\begin{exe}
\sn \textcolor{darkblue}{\textbf{\ipa{lɑ˧bi˧-tsɤ˧}}}
\trans raide, escarpé (littéralement 'comme un escarpement')
\trans \textit{\textcolor{brown}{\zh{‘像陡坡’,等于:很陡}}}
\end{exe}
\begin{exe}
\sn \textcolor{darkblue}{\textbf{\ipa{lɑ˧bi˧-tsɤ˧ | ʐwæ˩˥!}}}
\trans C'est très pentu! (Littéralement: 'Ca ressemble vraiment à une pente raide!')
\trans \textit{\textcolor{brown}{\zh{陡得很!}}}
\end{exe}


\paragraph{\hspace{-0.5cm} {\Large \textcolor{darkblue}{\textbf{\ipa{lɑ˧do\#˥}}}}} \hypertarget{lA\string_Mdo\#\string_T1}{}
\markboth{\textcolor{darkblue}{\textbf{\ipa{lɑ˧do\#˥}}}}{}
\textcolor{teal}{\mytextsc{nom}} \hspace{4pt} Ton~: \#H.

Palefrenier, caravanier (employé, pas chef de caravane).
\textcolor{brown}{\zh{马夫(参加马帮)。}}




\paragraph{\hspace{-0.5cm} {\Large \textcolor{darkblue}{\textbf{\ipa{lɑ˧hwɤ˩}}}}} \hypertarget{lA\string_Mhw7\string_B1}{}
\markboth{\textcolor{darkblue}{\textbf{\ipa{lɑ˧hwɤ˩}}}}{}
\textcolor{teal}{\mytextsc{nom}} \hspace{4pt} Ton~: L\#.

Village na hors de la plaine de Yongning, vers le Lac, non loin de \textcolor{darkblue}{\textbf{\ipa{/lɑ˧tʰɑ˧-di˧˥/}}}.
\textcolor{brown}{\zh{村落名。}}




\paragraph{\hspace{-0.5cm} {\Large \textcolor{darkblue}{\textbf{\ipa{lɑ˧kɤ˩}}}}} \hypertarget{lA\string_Mk7\string_B1}{}
\markboth{\textcolor{darkblue}{\textbf{\ipa{lɑ˧kɤ˩}}}}{}
\textcolor{teal}{\mytextsc{nom}} \hspace{4pt} Ton~: L\#.

Petite cruche, petit pot pour l'alcool; sert pour le conserver longtemps, pas seulement pour le verser.
\textcolor{brown}{\zh{小坛子,用来存酒。}}


 \mytextsc{clf}~: \textcolor{darkblue}{\textbf{\ipa{ɭɯ˧}}} 

\paragraph{\hspace{-0.5cm} {\Large \textcolor{darkblue}{\textbf{\ipa{lɑ˧kʰv̩˧˥}}}}} \hypertarget{lA\string_Mk\string_hv\string_=\string_M\string_T1}{}
\markboth{\textcolor{darkblue}{\textbf{\ipa{lɑ˧kʰv̩˧˥}}}}{}
\textcolor{teal}{\mytextsc{nom}} \hspace{4pt} Ton~: MH\#.

Année du Tigre.
\textcolor{brown}{\zh{虎年。}}




\paragraph{\hspace{-0.5cm} {\Large \textcolor{darkblue}{\textbf{\ipa{lɑ˧\textasciitilde{}lɑ˧}}}}} \hypertarget{lA\string_M~lA\string_M1}{}
\markboth{\textcolor{darkblue}{\textbf{\ipa{lɑ˧\textasciitilde{}lɑ˧}}}}{}
\textcolor{teal}{\mytextsc{adjectif}} \hspace{4pt} Ton~: M.

Ballant, flasque.
\textcolor{brown}{\zh{松弛。}}




\paragraph{\hspace{-0.5cm} {\Large \textcolor{darkblue}{\textbf{\ipa{lɑ˧\textasciitilde{}lɑ˧\textsubscript{b}}}}}} \hypertarget{lA\string_M~lA\string_Mb1}{}
\markboth{\textcolor{darkblue}{\textbf{\ipa{lɑ˧\textasciitilde{}lɑ˧\textsubscript{b}}}}}{}
\textcolor{teal}{\mytextsc{verbe}} \hspace{4pt} Ton~: M\textsubscript{b}.

Diluer (dans l’eau).
\textcolor{brown}{\zh{掺水。}}


\begin{exe}
\sn \textcolor{darkblue}{\textbf{\ipa{(dʑɯ˧-qo˧) le˧-lɑ˧\textasciitilde{}lɑ˧}}}
\trans diluer dans de l’eau
\trans \textit{\textcolor{brown}{\zh{掺水}}}
\end{exe}


\paragraph{\hspace{-0.5cm} {\Large \textcolor{darkblue}{\textbf{\ipa{lɑ˧lo˧-ʁwɤ˥}}}}} \hypertarget{lA\string_Mlo\string_M-Rw7\string_T1}{}
\markboth{\textcolor{darkblue}{\textbf{\ipa{lɑ˧lo˧-ʁwɤ˥}}}}{}
\textcolor{teal}{\mytextsc{nom}} \hspace{4pt} Ton~: H\#.

Un village de Yongning; prononciation chinoise: Laluowa.
\textcolor{brown}{\zh{拉洛瓦村(永宁的一个村落)。}}


\begin{exe}
\sn \textcolor{darkblue}{\textbf{\ipa{dʑɤ˩bv̩˧kɤ˧-sɑ˥ʁwɤ˩, | hi˩ʁwɤ˩-lo˥, | æ˩mi˧-ʁwɤ\#˥, | lɑ˧lo˧-ʁwɤ˥, | lɑ˧ŋwɤ˧, | bɤ˧tsʰo˧gv̩˥, | ə˧lɑ˧-ʁwɤ\#˥, | gæ˧ɻæ˩, | qʰæ˧tɕʰi˧, | tʰo˧ʈɯ\#˥}}}
\trans les dix villages comptant traditionnellement comme faisant partie de Yongning
\trans \textit{\textcolor{brown}{\zh{摩梭传统地理概念中,属于永宁的十个村落}}}
\end{exe}


\paragraph{\hspace{-0.5cm} {\Large \textcolor{darkblue}{\textbf{\ipa{lɑ˧ɬɑ˧˥}}}}} \hypertarget{lA\string_MKA\string_M\string_T1}{}
\markboth{\textcolor{darkblue}{\textbf{\ipa{lɑ˧ɬɑ˧˥}}}}{}
\textcolor{teal}{\mytextsc{conjonction}} \hspace{4pt} Ton~: MH\#.

À part, en dehors de.
\textcolor{brown}{\zh{这以外。}}


\begin{exe}
\sn \textcolor{darkblue}{\textbf{\ipa{tsɑ˧bɤ˧ mɤ˧-pʰv̩˧ɖɯ˧! | lɑ˧ɬɑ˧˥, | ə˧tso˧-mɤ˧-ɲi˩ | pʰv̩˩ɖɯ˩˥!}}}
\trans La farine n'est pas chère; à part ça, tout est cher! (Réflexion au sujet du coût de la vie dans la région aujourd'hui)
\trans \textit{\textcolor{brown}{\zh{面粉不贵。其它的呢,什么都贵!(题目:讲今日永宁食品物价)}}}
\end{exe}


\paragraph{\hspace{-0.5cm} {\Large \textcolor{darkblue}{\textbf{\ipa{lɑ˧ɬɑ˧˥}}} \textsubscript{1}}} \hypertarget{lA\string_MKA\string_M\string_T1}{}
\markboth{\textcolor{darkblue}{\textbf{\ipa{lɑ˧ɬɑ˧˥}}} \textsubscript{1}}{}
\textcolor{teal}{\mytextsc{pronom}} \hspace{4pt} Ton~: MH\#.

Autre, autres.
\textcolor{brown}{\zh{别的。}}


\begin{exe}
\sn \textcolor{darkblue}{\textbf{\ipa{lɑ˧ɬɑ˧˥ | ɖɯ˧-tɕi˥}}}
\trans quelques autres
\trans \textit{\textcolor{brown}{\zh{其它一些}}}
\end{exe}
\begin{exe}
\sn \textcolor{darkblue}{\textbf{\ipa{lɑ˧ɬɑ˧˥ | ʈʂʰɯ˧-tɕi˩}}}
\trans ces quelques autres, ceux qui restent
\trans \textit{\textcolor{brown}{\zh{其它的那些}}}
\end{exe}
\begin{exe}
\sn \textcolor{darkblue}{\textbf{\ipa{lɑ˧ɬɑ˧˥ | ɖɯ˧-ʁo˩ ɲi˩!}}}
\trans C'est autre chose! / Ca, c'est différent!
\trans \textit{\textcolor{brown}{\zh{是另一回事! / 是另一码事!}}}
\end{exe}
\textit{Voir~:} \hyperlink{lA\string_MKA\string_M\string_T2}{\textcolor{darkblue}{\textbf{\ipa{lɑ˧ɬɑ˧˥}}} \textsubscript{2}} 

\paragraph{\hspace{-0.5cm} {\Large \textcolor{darkblue}{\textbf{\ipa{lɑ˧ɬɑ˧˥}}} \textsubscript{2}}} \hypertarget{lA\string_MKA\string_M\string_T2}{}
\markboth{\textcolor{darkblue}{\textbf{\ipa{lɑ˧ɬɑ˧˥}}} \textsubscript{2}}{}
\textcolor{teal}{\mytextsc{adjectif}} \hspace{4pt} Ton~: MH\#.

Autre.
\textcolor{brown}{\zh{别的。}}


\begin{exe}
\sn \textcolor{darkblue}{\textbf{\ipa{lɑ˧ɬɑ˧ hĩ˥}}}
\trans les autres gens
\trans \textit{\textcolor{brown}{\zh{其它人}}}
\end{exe}
\begin{exe}
\sn \textcolor{darkblue}{\textbf{\ipa{ɖɯ˧-bæ˧ | le˧-se˩, | ɖɯ˧-bæ˧ ʝi˧! / ɖɯ˧-bæ˧ | le˧-se˩, | wɤ˩˥ | lɑ˧ɬɑ˧˥ | ɖɯ˧-bæ˧ ʝi˧! |}}}
\trans Quand on a fini une chose/une tâche, on en fait une autre / on passe à une autre!
\trans \textit{\textcolor{brown}{\zh{做完一件事情,就轮到另一个!}}}
\end{exe}
\textit{Voir~:} \hyperlink{lA\string_MKA\string_M\string_T1}{\textcolor{darkblue}{\textbf{\ipa{lɑ˧ɬɑ˧˥}}} \textsubscript{1}} 

\paragraph{\hspace{-0.5cm} {\Large \textcolor{darkblue}{\textbf{\ipa{lɑ˧mɑ˧}}}}} \hypertarget{lA\string_MmA\string_M1}{}
\markboth{\textcolor{darkblue}{\textbf{\ipa{lɑ˧mɑ˧}}}}{}
\textcolor{teal}{\mytextsc{nom}} \hspace{4pt} Ton~: M.

Lama.
\textcolor{brown}{\zh{喇嘛。}}

 Emprunt~: tibétain bla-ma

\begin{exe}
\sn \textcolor{darkblue}{\textbf{\ipa{hæ˧-lɑ˩mɑ˩}}}
\trans lama chinois
\trans \textit{\textcolor{brown}{\zh{汉族喇嘛}}}
\end{exe}
 \mytextsc{clf}~: \textcolor{darkblue}{\textbf{\ipa{v̩˧}}} 

\paragraph{\hspace{-0.5cm} {\Large \textcolor{darkblue}{\textbf{\ipa{lɑ˧mi\#˥}}}}} \hypertarget{lA\string_Mmi\#\string_T1}{}
\markboth{\textcolor{darkblue}{\textbf{\ipa{lɑ˧mi\#˥}}}}{}
\textcolor{teal}{\mytextsc{nom}} \hspace{4pt} Ton~: \#H.

Tigresse.
\textcolor{brown}{\zh{母老虎。}}


\begin{exe}
\sn \textcolor{darkblue}{\textbf{\ipa{lɑ˧mi˧ tʰv̩˧-mi˧˥ / lɑ˧mi˧ tʰv̩˧-mi˥\#}}}
\trans \mytextsc{n}+\mytextsc{dem}+\mytextsc{clf}
\trans \textit{\textcolor{brown}{\zh{那只老虎}}}
\end{exe}
 \mytextsc{clf}~: \textcolor{darkblue}{\textbf{\ipa{pʰo˧˥ / mi˩}}} 

\paragraph{\hspace{-0.5cm} {\Large \textcolor{darkblue}{\textbf{\ipa{lɑ˧ŋwɤ˧}}}}} \hypertarget{lA\string_MNw7\string_M1}{}
\markboth{\textcolor{darkblue}{\textbf{\ipa{lɑ˧ŋwɤ˧}}}}{}
\textcolor{teal}{\mytextsc{nom}} \hspace{4pt} Ton~: M.

Nom de montagne, sur le chemin de Yongning à Wujiao; est aussi le nom qui désigne les hameaux qui se trouvent sur cette montagne.
\textcolor{brown}{\zh{一座山的名字。}}


\begin{exe}
\sn \textcolor{darkblue}{\textbf{\ipa{dʑɤ˩bv̩˧kɤ˧-sɑ˥ʁwɤ˩, | hi˩ʁwɤ˩-lo˥, | æ˩mi˧-ʁwɤ\#˥, | lɑ˧lo˧-ʁwɤ˥, | lɑ˧ŋwɤ˧, | bɤ˧tsʰo˧gv̩˥, | ə˧lɑ˧-ʁwɤ\#˥, | gæ˧ɻæ˩, | qʰæ˧tɕʰi˧, | tʰo˧ʈɯ\#˥}}}
\trans les dix villages comptant traditionnellement comme faisant partie de Yongning
\trans \textit{\textcolor{brown}{\zh{摩梭传统地理概念中,属于永宁的十个村落}}}
\end{exe}


\paragraph{\hspace{-0.5cm} {\Large \textcolor{darkblue}{\textbf{\ipa{lɑ˧pʰv̩\#˥}}}}} \hypertarget{lA\string_Mp\string_hv\string_=\#\string_T1}{}
\markboth{\textcolor{darkblue}{\textbf{\ipa{lɑ˧pʰv̩\#˥}}}}{}
\textcolor{teal}{\mytextsc{nom}} \hspace{4pt} Ton~: \#H.

Tigre (mâle).
\textcolor{brown}{\zh{公老虎。}}


\begin{exe}
\sn \textcolor{darkblue}{\textbf{\ipa{lɑ˧pʰv̩˧ tʰv̩˧-ɭɯ\#˥}}}
\trans \mytextsc{n}+\mytextsc{dem}+\mytextsc{clf}
\trans \textit{\textcolor{brown}{\zh{那只老虎}}}
\end{exe}
 \mytextsc{clf}~: \textcolor{darkblue}{\textbf{\ipa{pʰo˧˥ / ɭɯ˧}}} 

\paragraph{\hspace{-0.5cm} {\Large \textcolor{darkblue}{\textbf{\ipa{lɑ˧tʰɑ˧-di˧˥}}}}} \hypertarget{lA\string_Mt\string_hA\string_M-di\string_M\string_T1}{}
\markboth{\textcolor{darkblue}{\textbf{\ipa{lɑ˧tʰɑ˧-di˧˥}}}}{}
\textcolor{teal}{\mytextsc{nom}} \hspace{4pt} Ton~: MH\#.

La région na qui entoure le lac Lugu: Zuosuo (actuel Luguhu Zhen), le village de Luoshui, et les autres localités du bord du Lac.
\textcolor{brown}{\zh{泸沽湖周边的摩梭地区,包括左所(今为泸沽湖镇)、洛水村等。}}


\begin{exe}
\sn \textcolor{darkblue}{\textbf{\ipa{ɬi˧ki˧, | ɲi˧se˩, | tɑ˧dzi˩, | mv̩˧qʰwæ˩, | lɑ˧tʰɑ˧-di˧˥}}}
\trans Villages dans l'ordre, après la plaine de Yongning, ne comptant pas comme faisant partie de Yongning. Le dernier, \textcolor{darkblue}{\textbf{\ipa{/lɑ˧tʰɑ˧-di˧˥/}}}, désigne toute la région na au-delà du quatrième village.
\trans \textit{\textcolor{brown}{\zh{从永宁往泸沽湖所经过的村落,依次是:里格、尼赛、大祖、木垮,然后到拉塔地(拉塔地指的是泸沽湖周边的摩梭地区,包括左所、洛水村等)}}}
\end{exe}


\paragraph{\hspace{-0.5cm} {\Large \textcolor{darkblue}{\textbf{\ipa{lɑ˧tʰɑ˧mi˥\$}}}}} \hypertarget{lA\string_Mt\string_hA\string_Mmi\string_T\$1}{}
\markboth{\textcolor{darkblue}{\textbf{\ipa{lɑ˧tʰɑ˧mi˥\$}}}}{}
\textcolor{teal}{\mytextsc{nom}} \hspace{4pt} Ton~: H\$.

Nom de clan/famille étendue. Cinq familles portent ce nom à Yongning. C'est l'un des trois premiers clans à s'être établis à proximité du monastère de Yongning, les deux autres étant \textcolor{darkblue}{\textbf{\ipa{/kɤ˧˥tʰɑ˩/}}} et \textcolor{darkblue}{\textbf{\ipa{/ə˧lɑ˧/}}}.
\textcolor{brown}{\zh{一个姓。这个姓,永宁有五个家。音译:拉他咪。}}


\begin{exe}
\sn \textcolor{darkblue}{\textbf{\ipa{lɑ˧tʰɑ˧mi˧=ɻ̍˥\$}}}
\trans le clan \textcolor{darkblue}{\textbf{\ipa{/lɑ˧tʰɑ˧mi˥\$/}}}, la famille \textcolor{darkblue}{\textbf{\ipa{/lɑ˧tʰɑ˧mi˥\$/}}}
\trans \textit{\textcolor{brown}{\zh{\textcolor{darkblue}{\textbf{\ipa{/lɑ˧tʰɑ˧mi˥\$/}}}家族}}}
\end{exe}


\paragraph{\hspace{-0.5cm} {\Large \textcolor{darkblue}{\textbf{\ipa{}}}}} \hypertarget{lA\string_Mt\string_hA\string_Mmi\string_T-t`\{\string_Ms`M\string_M-lA\string_Bmv\string_=\string_B1}{}
\markboth{\textcolor{darkblue}{\textbf{\ipa{}}}}{}
\textcolor{teal}{\mytextsc{nom}} \hspace{4pt} Ton~: H\#-M-L.

Nom propre (nom de famille et prénom) de la consultante de référence du présent travail (code locutrice: F4).
\textcolor{brown}{\zh{拉他咪•达石拉么:本著作的标准发音合作人。}}




\paragraph{\hspace{-0.5cm} {\Large \textcolor{darkblue}{\textbf{\ipa{lɑ˧zi˥}}}}} \hypertarget{lA\string_Mzi\string_T1}{}
\markboth{\textcolor{darkblue}{\textbf{\ipa{lɑ˧zi˥}}}}{}
\textcolor{teal}{\mytextsc{nom}} \hspace{4pt} Ton~: H\#.

Peintre (activité qui n'est pas réservée aux moines).
\textcolor{brown}{\zh{画家。}}


\begin{exe}
\sn \textcolor{darkblue}{\textbf{\ipa{ʈʂʰɯ˧-v̩˧, | lɑ˧zi˥ ɲi˩!}}}
\trans elle/il est peintre! / elle/il sait peindre!
\trans \textit{\textcolor{brown}{\zh{他是画家!}}}
\end{exe}


\paragraph{\hspace{-0.5cm} {\Large \textcolor{darkblue}{\textbf{\ipa{lɑ˧zo\#˥}}}}} \hypertarget{lA\string_Mzo\#\string_T1}{}
\markboth{\textcolor{darkblue}{\textbf{\ipa{lɑ˧zo\#˥}}}}{}
\textcolor{teal}{\mytextsc{nom}} \hspace{4pt} Ton~: \#H.

Petit tigre.
\textcolor{brown}{\zh{小老虎。}}


\begin{exe}
\sn \textcolor{darkblue}{\textbf{\ipa{lɑ˧zo˧ tʰv̩˧-ɭɯ\#˥}}}
\trans \mytextsc{n}+\mytextsc{dem}+\mytextsc{clf}
\trans \textit{\textcolor{brown}{\zh{那只小老虎}}}
\end{exe}
 \mytextsc{clf}~: \textcolor{darkblue}{\textbf{\ipa{ɭɯ˧}}} 

\paragraph{\hspace{-0.5cm} {\Large \textcolor{darkblue}{\textbf{\ipa{lɑ˩gv̩˧}}}}} \hypertarget{lA\string_Bgv\string_=\string_M1}{}
\markboth{\textcolor{darkblue}{\textbf{\ipa{lɑ˩gv̩˧}}}}{}
\textcolor{teal}{\mytextsc{adjectif}} \hspace{4pt} Ton~: LM.

Recourbé, tordu, courbe.
\textcolor{brown}{\zh{弯(树...)。}}


\begin{exe}
\sn \textcolor{darkblue}{\textbf{\ipa{si˧dzi˩ | lɑ˩-gv̩˧-ze˩}}}
\trans L'arbre est devenu courbé.
\trans \textit{\textcolor{brown}{\zh{树弯了。}}}
\end{exe}


\paragraph{\hspace{-0.5cm} {\Large \textcolor{darkblue}{\textbf{\ipa{lɑ˩gv̩˧-lɑ˩ɲi˩}}}}} \hypertarget{lA\string_Bgv\string_=\string_M-lA\string_BJi\string_B1}{}
\markboth{\textcolor{darkblue}{\textbf{\ipa{lɑ˩gv̩˧-lɑ˩ɲi˩}}}}{}
\textcolor{teal}{\mytextsc{adjectif}} \hspace{4pt} Ton~: LM-L.
\textit{De:} \textbf{lɑ˩gv̩˧} 
Tout tordu, tout recourbé.
\textcolor{brown}{\zh{弯(路,植物,人的四肢)。}}




\paragraph{\hspace{-0.5cm} {\Large \textcolor{darkblue}{\textbf{\ipa{lɑ˩jɤ˧-ɬi˧}}}}} \hypertarget{lA\string_Bj7\string_M-Ki\string_M1}{}
\markboth{\textcolor{darkblue}{\textbf{\ipa{lɑ˩jɤ˧-ɬi˧}}}}{}
\textcolor{teal}{\mytextsc{nom}} \hspace{4pt} Ton~: LM-.

Le douzième mois.
\textcolor{brown}{\zh{十二月。}}




\paragraph{\hspace{-0.5cm} {\Large \textcolor{darkblue}{\textbf{\ipa{lɑ˩\textasciitilde{}lɑ˧˥}}}}} \hypertarget{lA\string_B~lA\string_M\string_T1}{}
\markboth{\textcolor{darkblue}{\textbf{\ipa{lɑ˩\textasciitilde{}lɑ˧˥}}}}{}
\textcolor{teal}{\mytextsc{verbe}} \hspace{4pt} Ton~: MH.

Se disputer, se battre.
\textcolor{brown}{\zh{打架、吵架。}}


\begin{exe}
\sn \textcolor{darkblue}{\textbf{\ipa{lɑ˩lɑ˧-hĩ˥ | ʈʂʰɯ˧-tɕi˩}}}
\trans ces (gens) qui se disputent
\trans \textit{\textcolor{brown}{\zh{打架的这些(人)}}}
\end{exe}
\textit{Voir~:} \textcolor{darkblue}{\textbf{\ipa{lɑ˧˥}}} 

\paragraph{\hspace{-0.5cm} {\Large \textcolor{darkblue}{\textbf{\ipa{lɑ˩mɑ˩}}}}} \hypertarget{lA\string_BmA\string_B1}{}
\markboth{\textcolor{darkblue}{\textbf{\ipa{lɑ˩mɑ˩}}}}{}
\textcolor{teal}{\mytextsc{nom}} \hspace{4pt} Ton~: L.

Nom de clan/famille étendue. Quatre familles portent ce nom à Yongning.
\textcolor{brown}{\zh{一个姓。这个姓,永宁有四个家。}}


\begin{exe}
\sn \textcolor{darkblue}{\textbf{\ipa{lɑ˩mɑ˩=ɻ̍˥\$}}}
\trans le clan \textcolor{darkblue}{\textbf{\ipa{/lɑ˩mɑ˩/}}}, la famille \textcolor{darkblue}{\textbf{\ipa{/lɑ˩mɑ˩/}}}
\trans \textit{\textcolor{brown}{\zh{\textcolor{darkblue}{\textbf{\ipa{/lɑ˩mɑ˩/}}}家族}}}
\end{exe}
\begin{exe}
\sn \textcolor{darkblue}{\textbf{\ipa{lɑ˩mɑ˩-gv̩˥mɑ˩}}}
\trans nom d'une personne, comportant un nom de famille (\textcolor{darkblue}{\textbf{\ipa{/lɑ˩mɑ˩/}}}) et un prénom (\textcolor{darkblue}{\textbf{\ipa{/gv̩˧mɑ˧/}}})
\trans \textit{\textcolor{brown}{\zh{一个人的名字:姓为\textcolor{darkblue}{\textbf{\ipa{/lɑ˩mɑ˩/}}},名为\textcolor{darkblue}{\textbf{\ipa{/gv̩˧mɑ˧/}}}}}}
\end{exe}


\paragraph{\hspace{-0.5cm} {\Large \textcolor{darkblue}{\textbf{\ipa{lɑ˩tɑ˧}}}}} \hypertarget{lA\string_BtA\string_M1}{}
\markboth{\textcolor{darkblue}{\textbf{\ipa{lɑ˩tɑ˧}}}}{}
\textcolor{teal}{\mytextsc{adjectif}} \hspace{4pt} Ton~: LM.

De biais, de travers (ex.: porter son chapeau de travers).
\textcolor{brown}{\zh{歪,偏 (帽子戴得歪)。}}




\paragraph{\hspace{-0.5cm} {\Large \textcolor{darkblue}{\textbf{\ipa{-lɑ˩tɑ˩}}}}} \hypertarget{-lA\string_BtA\string_B1}{}
\markboth{\textcolor{darkblue}{\textbf{\ipa{-lɑ˩tɑ˩}}}}{}
\textcolor{teal}{\mytextsc{postposition}} \hspace{4pt} Ton~: L.

À proximité de.



\begin{exe}
\sn \textcolor{darkblue}{\textbf{\ipa{ɑ˩ʁo˧ | -lɑ˩tɑ˩˥}}}
\trans le périmètre de la maison, là où s'étend le domaine de la maison
\trans \textit{\textcolor{brown}{\zh{家的面积}}}
\end{exe}


\paragraph{\hspace{-0.5cm} {\Large \textcolor{darkblue}{\textbf{\ipa{lɑ˩ʈʂv̩˩}}}}} \hypertarget{lA\string_Bt`s`v\string_=\string_B1}{}
\markboth{\textcolor{darkblue}{\textbf{\ipa{lɑ˩ʈʂv̩˩}}}}{}
\textcolor{teal}{\mytextsc{nom}} \hspace{4pt} Ton~: L.

Bougie.
\textcolor{brown}{\zh{蜡烛。}}

 Emprunt~: chinois \textcolor{brown}{\zh{蜡烛}}

 \mytextsc{clf}~: \textcolor{darkblue}{\textbf{\ipa{kɤ˧˥}}} 

\paragraph{\hspace{-0.5cm} {\Large \textcolor{darkblue}{\textbf{\ipa{lɑ˧˥}}} \textsubscript{1}}} \hypertarget{lA\string_M\string_T1}{}
\markboth{\textcolor{darkblue}{\textbf{\ipa{lɑ˧˥}}} \textsubscript{1}}{}
\textcolor{teal}{\mytextsc{verbe}} \hspace{4pt} Ton~: MH.

Battre quelque chose, frapper quelque chose, enfoncer un clou, casser des cailloux; donner (une injonction…).
\textcolor{brown}{\zh{打(打人,钉钉子……)。}}


\begin{exe}
\sn \textcolor{darkblue}{\textbf{\ipa{hĩ˧ lɑ˩}}}
\trans frapper quelqu'un
\trans \textit{\textcolor{brown}{\zh{打人}}}
\end{exe}
\begin{exe}
\sn \textcolor{darkblue}{\textbf{\ipa{hɑ˧ lɑ˩}}}
\trans battre le grain
\trans \textit{\textcolor{brown}{\zh{打粮食}}}
\end{exe}
\begin{exe}
\sn \textcolor{darkblue}{\textbf{\ipa{nv̩˩ɭɯ˧ lɑ˧}}}
\trans battre les cosses de soja
\trans \textit{\textcolor{brown}{\zh{打大豆}}}
\end{exe}
\begin{exe}
\sn \textcolor{darkblue}{\textbf{\ipa{sɯ˩tʰi˩-po˥-ɳɯ˩ | lɑ˧˥}}}
\trans casser au moyen d'un couteau (du thé compressé en galettes ou en briques, à l'ancienne)
\trans \textit{\textcolor{brown}{\zh{用刀子来砍(沱茶、茶饼)}}}
\end{exe}
\begin{exe}
\sn \textcolor{darkblue}{\textbf{\ipa{ə˧ʝi˧-ʂɯ˥ʝi˩, | ɬi˧di˩-dʑo˩, | æ˧ lɑ˩-hĩ˩ F | dʑo˩˥! | ʂe˧ lɑ˧-hĩ˥ F | dʑo˩˥! | hæ̃˩ lɑ˩-hĩ˥ F | dʑo˩˥! | ŋv̩˩ lɑ˩-hĩ˥ F | dʑo˩˥!}}}
\trans Autrefois, à Yongning, il y avait des artisans qui travaillaient le cuivre! Il y avait des artisans qui travaillaient le fer! Il y avait des artisans qui travaillaient l'or! Il y avait des artisans qui travaillaient l'argent!
\trans \textit{\textcolor{brown}{\zh{过去,在永宁,有铜匠、铁匠、金匠、银匠。}}}
\end{exe}
\begin{exe}
\sn \textcolor{darkblue}{\textbf{\ipa{ə˧ʝi˧-ʂɯ˥ʝi˩, | ɬi˧di˩-dʑo˩, | æ˧ lɑ˩-hĩ˩ dʑo˩, | ʂe˧ lɑ˧-hĩ˥ dʑo˩, | hæ̃˩ lɑ˩-hĩ˥ dʑo˩, | ŋv̩˩ lɑ˩-hĩ˥ dʑo˩.}}}
\trans Autrefois, à Yongning, il y avait des artisans qui travaillaient le cuivre. Il y avait des artisans qui travaillaient le fer. Il y avait des artisans qui travaillaient l'or. Il y avait des artisans qui travaillaient l'argent.
\trans \textit{\textcolor{brown}{\zh{过去,在永宁,有铜匠、铁匠、金匠、银匠。}}}
\end{exe}
\textit{Voir~:} \hyperlink{lA\string_B~lA\string_M\string_T1}{\textcolor{darkblue}{\textbf{\ipa{lɑ˩~lɑ˧˥}}}} 

\paragraph{\hspace{-0.5cm} {\Large \textcolor{darkblue}{\textbf{\ipa{lɑ˧˥}}} \textsubscript{2}}} \hypertarget{lA\string_M\string_T2}{}
\markboth{\textcolor{darkblue}{\textbf{\ipa{lɑ˧˥}}} \textsubscript{2}}{}
\textcolor{teal}{\mytextsc{verbe}} \hspace{4pt} Ton~: MH.

Apparaître, y avoir (de la rosée).
\textcolor{brown}{\zh{有,结(露水)。}}


\begin{exe}
\sn \textcolor{darkblue}{\textbf{\ipa{ɖʐv̩˧ lɑ˧˥}}}
\trans Il y a de la rosée; de la rosée s'est formée
\trans \textit{\textcolor{brown}{\zh{结露水了。}}}
\end{exe}
\begin{exe}
\sn \textcolor{darkblue}{\textbf{\ipa{ɖʐv̩˧qʰɑ˧ lɑ˧˥}}}
\trans Il y a de la rosée; de la rosée s'est formée
\trans \textit{\textcolor{brown}{\zh{结露水了。}}}
\end{exe}


\paragraph{\hspace{-0.5cm} {\Large \textcolor{darkblue}{\textbf{\ipa{‑læ˧}}}}} \hypertarget{‑l\{\string_M1}{}
\markboth{\textcolor{darkblue}{\textbf{\ipa{‑læ˧}}}}{}
\textcolor{teal}{\mytextsc{suffixe}} \hspace{4pt} Ton~: M.

Topique, introduisant un élément nouveau, pas nécessairement en contraste avec ce qui précède. Gloses possibles: pour ce qui est de, en ce qui concerne, quant à.
\textcolor{brown}{\zh{\mytextsc{主题:……的话、关于……。}}}


\begin{exe}
\sn \textcolor{darkblue}{\textbf{\ipa{ɖɯ˩mɑ˧ | -læ˧…}}}
\trans pour ce qui est de ma petite-fille \textcolor{darkblue}{\textbf{\ipa{ɖɯ˩mɑ˧}}}, eh bien…
\trans \textit{\textcolor{brown}{\zh{关于独妈呢,……}}}
\end{exe}
\begin{exe}
\sn \textcolor{darkblue}{\textbf{\ipa{lɑ˩mv̩˩˥ | -læ˧...}}}
\trans pour ce qui est de \textcolor{darkblue}{\textbf{\ipa{lɑ˩mv̩˩˥}}} [nom propre], ...
\trans \textit{\textcolor{brown}{\zh{关于拉姆呢,……}}}
\end{exe}
\begin{exe}
\sn \textcolor{darkblue}{\textbf{\ipa{ti˧ɖo˥ | -læ˧…}}}
\trans pour ce qui est de \textcolor{darkblue}{\textbf{\ipa{ti˧ɖo˥}}} [nom propre], …
\trans \textit{\textcolor{brown}{\zh{关于\textcolor{darkblue}{\textbf{\ipa{ti˧ɖo˥}}}[人的名字]呢,……}}}
\end{exe}


\paragraph{\hspace{-0.5cm} {\Large \textcolor{darkblue}{\textbf{\ipa{læ˧dæ˧qæ˥}}}}} \hypertarget{l\{\string_Md\{\string_Mq\{\string_T1}{}
\markboth{\textcolor{darkblue}{\textbf{\ipa{læ˧dæ˧qæ˥}}}}{}
\textcolor{teal}{\mytextsc{nom}} \hspace{4pt} Ton~: H\#.

Aisselle.
\textcolor{brown}{\zh{腋下。}}Dialecte chinois local~: \zh{膈肢窝。}


 \mytextsc{clf}~: \textcolor{darkblue}{\textbf{\ipa{ɭɯ˧}}} 

\paragraph{\hspace{-0.5cm} {\Large \textcolor{darkblue}{\textbf{\ipa{læ˧ʁæ˥\$}}}}} \hypertarget{l\{\string_MR\{\string_T\$1}{}
\markboth{\textcolor{darkblue}{\textbf{\ipa{læ˧ʁæ˥\$}}}}{}
\textcolor{teal}{\mytextsc{nom}} \hspace{4pt} Ton~: H\$.

Corbeau.
\textcolor{brown}{\zh{乌鸦。}}


 \mytextsc{clf}~: \textcolor{darkblue}{\textbf{\ipa{mi˩}}} 

\paragraph{\hspace{-0.5cm} {\Large \textcolor{darkblue}{\textbf{\ipa{læ˧ʁæ˧mi˥\$}}}}} \hypertarget{l\{\string_MR\{\string_Mmi\string_T\$1}{}
\markboth{\textcolor{darkblue}{\textbf{\ipa{læ˧ʁæ˧mi˥\$}}}}{}
\textcolor{teal}{\mytextsc{nom}} \hspace{4pt} Ton~: H\$.

Corbeau femelle.
\textcolor{brown}{\zh{母乌鸦。}}


 \mytextsc{clf}~: \textcolor{darkblue}{\textbf{\ipa{mi˩}}} 

\paragraph{\hspace{-0.5cm} {\Large \textcolor{darkblue}{\textbf{\ipa{læ˧ʁæ˧-pʰv̩\#˥}}}}} \hypertarget{l\{\string_MR\{\string_M-p\string_hv\string_=\#\string_T1}{}
\markboth{\textcolor{darkblue}{\textbf{\ipa{læ˧ʁæ˧-pʰv̩\#˥}}}}{}
\textcolor{teal}{\mytextsc{nom}} \hspace{4pt} Ton~: \#H.

Corbeau mâle.
\textcolor{brown}{\zh{公乌鸦。}}


\begin{exe}
\sn \textcolor{darkblue}{\textbf{\ipa{læ˧ʁæ˧-pʰv̩˧ tʰv̩˧-mi˥\$}}}
\trans \mytextsc{n}+\mytextsc{dem}+\mytextsc{clf}
\trans \textit{\textcolor{brown}{\zh{那只公乌鸦}}}
\end{exe}
 \mytextsc{clf}~: \textcolor{darkblue}{\textbf{\ipa{mi˩}}} 

\paragraph{\hspace{-0.5cm} {\Large \textcolor{darkblue}{\textbf{\ipa{læ˧tsɯ˥}}}}} \hypertarget{l\{\string_MtsM\string_T1}{}
\markboth{\textcolor{darkblue}{\textbf{\ipa{læ˧tsɯ˥}}}}{}
\textcolor{teal}{\mytextsc{nom}} \hspace{4pt} Ton~: H\#.

Piment.
\textcolor{brown}{\zh{辣椒(汉语借词:辣子)。}}Dialecte chinois local~: \zh{辣子。}

 Emprunt~: chinois \textcolor{brown}{\zh{辣子}}

\begin{exe}
\sn \textcolor{darkblue}{\textbf{\ipa{læ˧tsɯ˥ hṽ˩\textasciitilde{}hṽ˩}}}
\trans frire des piments
\trans \textit{\textcolor{brown}{\zh{炒辣椒}}}
\end{exe}
 \mytextsc{clf}~: \textcolor{darkblue}{\textbf{\ipa{ɭɯ˧}}} 

\paragraph{\hspace{-0.5cm} {\Large \textcolor{darkblue}{\textbf{\ipa{}}}}} \hypertarget{le\string_M‑1}{}
\markboth{\textcolor{darkblue}{\textbf{\ipa{}}}}{}
\textcolor{teal}{\mytextsc{préfixe}} \hspace{4pt} Ton~: M/0.

\mytextsc{accomp}.
\textcolor{brown}{\zh{\mytextsc{实施。}}}




\paragraph{\hspace{-0.5cm} {\Large \textcolor{darkblue}{\textbf{\ipa{le˧-tɑ˧˥}}}}} \hypertarget{le\string_M-tA\string_M\string_T1}{}
\markboth{\textcolor{darkblue}{\textbf{\ipa{le˧-tɑ˧˥}}}}{}
\textcolor{teal}{\mytextsc{conjonction}} \hspace{4pt} Ton~: MH.

Jusqu'à; même.
\textcolor{brown}{\zh{到……为止、一直到……、连……。}}




\paragraph{\hspace{-0.5cm} {\Large \textcolor{darkblue}{\textbf{\ipa{le˧-wo˥}}}}} \hypertarget{le\string_M-wo\string_T1}{}
\markboth{\textcolor{darkblue}{\textbf{\ipa{le˧-wo˥}}}}{}
\textcolor{teal}{\mytextsc{adverbe}} \hspace{4pt} Ton~: H\#.

À nouveau, de nouveau.
\textcolor{brown}{\zh{再、又、重新。}}




\paragraph{\hspace{-0.5cm} {\Large \textcolor{darkblue}{\textbf{\ipa{le˧-wo˧}}}}} \hypertarget{le\string_M-wo\string_M1}{}
\markboth{\textcolor{darkblue}{\textbf{\ipa{le˧-wo˧}}}}{}
\textcolor{teal}{\mytextsc{adverbe}} \hspace{4pt} Ton~: M.

À nouveau.
\textcolor{brown}{\zh{又,……回去。}}


\begin{exe}
\sn \textcolor{darkblue}{\textbf{\ipa{le˧-wo˧ jo˧}}}
\trans revenir
\trans \textit{\textcolor{brown}{\zh{回}}}
\end{exe}
\begin{exe}
\sn \textcolor{darkblue}{\textbf{\ipa{le˧-wo˧ le˧-gv̩˩}}}
\trans recommencer
\trans \textit{\textcolor{brown}{\zh{从头开始}}}
\end{exe}
\begin{exe}
\sn \textcolor{darkblue}{\textbf{\ipa{le˧-wo˧ le˧-gv̩˧\textasciitilde{}gv̩˥}}}
\trans refaire, réinstaller, reconstruire
\trans \textit{\textcolor{brown}{\zh{重新做、重新建}}}
\end{exe}
\begin{exe}
\sn \textcolor{darkblue}{\textbf{\ipa{le˧-wo˧ le˥-tɕo˩ ʐwɤ˩}}}
\trans répéter sans arrêt
\trans \textit{\textcolor{brown}{\zh{重复讲说过的话}}}
\end{exe}


\paragraph{\hspace{-0.5cm} {\Large \textcolor{darkblue}{\textbf{\ipa{le˧-wo˧-tʰo˥-tɕo˩}}}}} \hypertarget{le\string_M-wo\string_M-t\string_ho\string_T-ts£o\string_B1}{}
\markboth{\textcolor{darkblue}{\textbf{\ipa{le˧-wo˧-tʰo˥-tɕo˩}}}}{}
\textcolor{teal}{\mytextsc{adverbe}} \hspace{4pt} Ton~: .

Ensuite; vient d'un sens littéral ``se retourner".
\textcolor{brown}{\zh{后来(原意:翻篇)。}}




\paragraph{\hspace{-0.5cm} {\Large \textcolor{darkblue}{\textbf{\ipa{le˩}}}}} \hypertarget{le\string_B1}{}
\markboth{\textcolor{darkblue}{\textbf{\ipa{le˩}}}}{}
\textcolor{teal}{\mytextsc{particule}} \textcolor{teal}{\mytextsc{de}} \textcolor{teal}{\mytextsc{discours}} \hspace{4pt} Ton~: L?.

Particule finale exclamative.
\textcolor{brown}{\zh{句尾助词:感叹。}}


\begin{exe}
\sn \textcolor{darkblue}{\textbf{\ipa{dʑɤ˩ le˥!}}}
\trans Bravo!
\trans \textit{\textcolor{brown}{\zh{好了!/ 太好了!}}}
\end{exe}


\paragraph{\hspace{-0.5cm} {\Large \textcolor{darkblue}{\textbf{\ipa{li˧\textsubscript{a}}}}}} \hypertarget{li\string_Ma1}{}
\markboth{\textcolor{darkblue}{\textbf{\ipa{li˧\textsubscript{a}}}}}{}
\textcolor{teal}{\mytextsc{verbe}} \hspace{4pt} Ton~: M\textsubscript{a}.
\ding{202} 
Regarder.
\textcolor{brown}{\zh{看。}}


\begin{exe}
\sn \textcolor{darkblue}{\textbf{\ipa{tʰi˧-li˧-dʑo˧}}}
\trans \mytextsc{dur} \string_ \mytextsc{prog}
\trans \textit{\textcolor{brown}{\zh{正在看}}}
\end{exe}
\begin{exe}
\sn \textcolor{darkblue}{\textbf{\ipa{tso˧\textasciitilde{}tso˧ li˩}}}
\trans regarder des choses
\trans \textit{\textcolor{brown}{\zh{看东西}}}
\end{exe}
\ding{203} 
S'occuper de.
\textcolor{brown}{\zh{管理。}}


\begin{exe}
\sn \textcolor{darkblue}{\textbf{\ipa{ɑ˩ʁo˧ li˧}}}
\trans s'occuper de la maison, veiller aux affaires de la maison; surveiller la maison
\trans \textit{\textcolor{brown}{\zh{管家、管家里的事情,看守家}}}
\end{exe}
\ding{204} 
Rendre visite à, aller voir (quelqu'un).
\textcolor{brown}{\zh{访问。}}


\begin{exe}
\sn \textcolor{darkblue}{\textbf{\ipa{pʰæ˧tɕi˥-zo˩-ɳɯ˩ | mv̩˩zo˩ li˥}}}
\trans Le jeune homme voit (=va voir) la jeune femme. (Euphémisme pour désigner les relations sexuelles.)
\trans \textit{\textcolor{brown}{\zh{小伙子去拜访年轻女人(委婉语,指性交)}}}
\end{exe}


\paragraph{\hspace{-0.5cm} {\Large \textcolor{darkblue}{\textbf{\ipa{li˧ʐv̩˩}}}}} \hypertarget{li\string_Mz`v\string_=\string_B1}{}
\markboth{\textcolor{darkblue}{\textbf{\ipa{li˧ʐv̩˩}}}}{}
\textcolor{teal}{\mytextsc{nom}} \hspace{4pt} Ton~: L\#.

Filet-mignon.
\textcolor{brown}{\zh{里脊肉。}}




\paragraph{\hspace{-0.5cm} {\Large \textcolor{darkblue}{\textbf{\ipa{li˩pi˥}}}}} \hypertarget{li\string_Bpi\string_T1}{}
\markboth{\textcolor{darkblue}{\textbf{\ipa{li˩pi˥}}}}{}
\textcolor{teal}{\mytextsc{nom}} \hspace{4pt} Ton~: LH.

Feuille de thé qui a trop infusé, qui est bonne à jeter.
\textcolor{brown}{\zh{已经泡了太久的茶叶。}}


 \mytextsc{clf}~: \textcolor{darkblue}{\textbf{\ipa{kʰwɤ˥}}} 

\paragraph{\hspace{-0.5cm} {\Large \textcolor{darkblue}{\textbf{\ipa{li˩pʰv̩˩}}}}} \hypertarget{li\string_Bp\string_hv\string_=\string_B1}{}
\markboth{\textcolor{darkblue}{\textbf{\ipa{li˩pʰv̩˩}}}}{}
\textcolor{teal}{\mytextsc{nom}} \hspace{4pt} Ton~: L.

Un lichen de montagne, \textit{Thamnolia vermicularis}, employé en décoction. Au souvenir de F4, ce lichen se cueillait au septième mois, seule période où il était abondant; on allait le cueillir en haute montagne.
\textcolor{brown}{\zh{雪茶。}}


\begin{exe}
\sn \textcolor{darkblue}{\textbf{\ipa{ŋwɤ˧hɑ̃˩-li˩pʰv˩}}}
\trans thé de lichen de la montagne \textcolor{darkblue}{\textbf{\ipa{ŋwɤ˧hɑ̃˩}}} (ce lichen y est abondant; c'est généralement là-bas qu'on le cueillait)
\trans \textit{\textcolor{brown}{\zh{\textcolor{darkblue}{\textbf{\ipa{ŋwɤ˧hɑ̃˩}}} 山的雪茶(说明:这种苔藓在那座山上多,七月份人家去采)}}}
\end{exe}


\paragraph{\hspace{-0.5cm} {\Large \textcolor{darkblue}{\textbf{\ipa{li˩˥}}}}} \hypertarget{li\string_B\string_T1}{}
\markboth{\textcolor{darkblue}{\textbf{\ipa{li˩˥}}}}{}
\textcolor{teal}{\mytextsc{nom}} \hspace{4pt} Ton~: LH.

Thé.
\textcolor{brown}{\zh{茶。}}


\begin{exe}
\sn \textcolor{darkblue}{\textbf{\ipa{li˩qʰɑ˩}}}
\trans 'thé amer': décoction de feuilles de pivoine blanche de Chine, que l'on buvait lorsqu'il n'y avait pas de thé à la maison; cela possédait des vertus médicinales.
\trans \textit{\textcolor{brown}{\zh{‘苦茶’:用白芍药来泡的饮料,没有茶的时候就喝这种‘苦茶’。它有医疗作用,帮助消化。}}}
\end{exe}
 \mytextsc{clf}~: \textcolor{darkblue}{\textbf{\ipa{qʰwɤ˧˥}}} bol

\paragraph{\hspace{-0.5cm} {\Large \textcolor{darkblue}{\textbf{\ipa{ljɤ˩\textsubscript{a}}}}}} \hypertarget{lj7\string_Ba1}{}
\markboth{\textcolor{darkblue}{\textbf{\ipa{ljɤ˩\textsubscript{a}}}}}{}
\textcolor{teal}{\mytextsc{classificateur}} \hspace{4pt} Ton~: L\textsubscript{a}.

Auto-classificateur des vies/destins.
\textcolor{brown}{\zh{量词:命、命运。}}


\begin{exe}
\sn \textcolor{darkblue}{\textbf{\ipa{ʈʂʰɯ˧-ljɤ˥}}}
\trans \mytextsc{dem} \string_ (tone: H\# / H\$)
\trans \textit{\textcolor{brown}{\zh{\mytextsc{指示代词} \string_}}}
\end{exe}


\paragraph{\hspace{-0.5cm} {\Large \textcolor{darkblue}{\textbf{\ipa{ljɤ˩mi˥}}}}} \hypertarget{lj7\string_Bmi\string_T1}{}
\markboth{\textcolor{darkblue}{\textbf{\ipa{ljɤ˩mi˥}}}}{}
\textcolor{teal}{\mytextsc{nom}} \hspace{4pt} Ton~: LH.

Poutre importante.
\textcolor{brown}{\zh{大梁。}}


 \mytextsc{clf}~: \textcolor{darkblue}{\textbf{\ipa{pʰæ˧˥}}} 

\paragraph{\hspace{-0.5cm} {\Large \textcolor{darkblue}{\textbf{\ipa{ljɤ˩mi˥-ʈæ˩qo˩}}}}} \hypertarget{lj7\string_Bmi\string_T-t`\{\string_Bqo\string_B1}{}
\markboth{\textcolor{darkblue}{\textbf{\ipa{ljɤ˩mi˥-ʈæ˩qo˩}}}}{}
\textcolor{teal}{\mytextsc{nom}} \hspace{4pt} Ton~: LH-.

Enjolivement sous une poutre; est perçu symboliquement comme 'l'oreille' de la poutre.
\textcolor{brown}{\zh{大梁的装饰:大梁的‘耳朵’。}}


 \mytextsc{clf}~: \textcolor{darkblue}{\textbf{\ipa{pʰæ˧˥}}} 

\paragraph{\hspace{-0.5cm} {\Large \textcolor{darkblue}{\textbf{\ipa{ljɤ˩ʂɯ˩}}}}} \hypertarget{lj7\string_Bs`M\string_B1}{}
\markboth{\textcolor{darkblue}{\textbf{\ipa{ljɤ˩ʂɯ˩}}}}{}
\textcolor{teal}{\mytextsc{nom}} \hspace{4pt} Ton~: L.

Céréales.
\textcolor{brown}{\zh{粮食(汉语借词)。}}

 Emprunt~: chinois \textcolor{brown}{\zh{粮食}}



\paragraph{\hspace{-0.5cm} {\Large \textcolor{darkblue}{\textbf{\ipa{ljɤ˩˥}}} \textsubscript{1}}} \hypertarget{lj7\string_B\string_T1}{}
\markboth{\textcolor{darkblue}{\textbf{\ipa{ljɤ˩˥}}} \textsubscript{1}}{}
\textcolor{teal}{\mytextsc{nom}} \hspace{4pt} Ton~: LH.

Poutre.
\textcolor{brown}{\zh{梁。}}


 \mytextsc{clf}~: \textcolor{darkblue}{\textbf{\ipa{pʰæ˧˥}}} 

\paragraph{\hspace{-0.5cm} {\Large \textcolor{darkblue}{\textbf{\ipa{ljɤ˩˥}}} \textsubscript{2}}} \hypertarget{lj7\string_B\string_T2}{}
\markboth{\textcolor{darkblue}{\textbf{\ipa{ljɤ˩˥}}} \textsubscript{2}}{}
\textcolor{teal}{\mytextsc{nom}} \hspace{4pt} Ton~: LM?LH?.

Sort, lot, existence, vie, destinée.
\textcolor{brown}{\zh{命、生命、命运。}}


\begin{exe}
\sn \textcolor{darkblue}{\textbf{\ipa{no˧ | ljɤ˩ ʈʂʰɯ˧-ljɤ˩-dʑo˩, | qʰæ˩˥ | ʐwæ˩˥!}}}
\trans Tu as une belle vie! Tu as la vie belle!
\trans \textit{\textcolor{brown}{\zh{你命好!}}}
\end{exe}
\begin{exe}
\sn \textcolor{darkblue}{\textbf{\ipa{hĩ˧-ljɤ˥}}}
\trans l'existence humaine
\trans \textit{\textcolor{brown}{\zh{人命、人类的命运}}}
\end{exe}
 \mytextsc{clf}~: \textcolor{darkblue}{\textbf{\ipa{ljɤ˩}}} 

\paragraph{\hspace{-0.5cm} {\Large \textcolor{darkblue}{\textbf{\ipa{lje˩fe˧}}}}} \hypertarget{lje\string_Bfe\string_M1}{}
\markboth{\textcolor{darkblue}{\textbf{\ipa{lje˩fe˧}}}}{}
\textcolor{teal}{\mytextsc{nom}} \hspace{4pt} Ton~: LM.

Liangfen: spécialité de Dali et des environs.
\textcolor{brown}{\zh{凉粉。}}

 Emprunt~: chinois \textcolor{brown}{\zh{凉粉}}



\paragraph{\hspace{-0.5cm} {\Large \textcolor{darkblue}{\textbf{\ipa{lo˧}}}}} \hypertarget{lo\string_M1}{}
\markboth{\textcolor{darkblue}{\textbf{\ipa{lo˧}}}}{}
\textcolor{teal}{\mytextsc{nom}} \hspace{4pt} Ton~: M.
\ding{202} 
Occupation, travail, tâche.
\textcolor{brown}{\zh{事情。}}


\begin{exe}
\sn \textcolor{darkblue}{\textbf{\ipa{lo˧ dʑo˧}}}
\trans avoir du travail, être occupé
\trans \textit{\textcolor{brown}{\zh{忙,有活要干}}}
\end{exe}
\begin{exe}
\sn \textcolor{darkblue}{\textbf{\ipa{njɤ˧ | lo˧ mɤ˧-dʑo˧.}}}
\trans Je ne suis pas occupé. / J'ai du temps libre. / Je suis disponible.
\trans \textit{\textcolor{brown}{\zh{我不忙。}}}
\end{exe}
\ding{203} 
Utilité.
\textcolor{brown}{\zh{用处。}}


\begin{exe}
\sn \textcolor{darkblue}{\textbf{\ipa{lo˧ mɤ˧-dʑo˧}}}
\trans C'est inutile / ça n'a aucune utilité. (Contexte: discussion au sujet du lierre, plante qui n'est utilisée ni comme fourrage, ni comme médicament, ni comme combustible, ni pour la confection de cordes ou autres outils ou objets)
\trans \textit{\textcolor{brown}{\zh{没有用!(情景:谈到常春藤,说它是没有用处的植物)}}}
\end{exe}
 \mytextsc{clf}~: \textcolor{darkblue}{\textbf{\ipa{lo˧}}} 

\paragraph{\hspace{-0.5cm} {\Large \textcolor{darkblue}{\textbf{\ipa{lo˧\textsubscript{b}}}}}} \hypertarget{lo\string_Mb1}{}
\markboth{\textcolor{darkblue}{\textbf{\ipa{lo˧\textsubscript{b}}}}}{}
\textcolor{teal}{\mytextsc{classificateur}} \hspace{4pt} Ton~: M\textsubscript{b}.

Auto-classificateur des travaux/occupations.
\textcolor{brown}{\zh{量词:事情(一件)、活(一个)。}}




\paragraph{\hspace{-0.5cm} {\Large \textcolor{darkblue}{\textbf{\ipa{lo˧bæ˧˥}}}}} \hypertarget{lo\string_Mb\{\string_M\string_T1}{}
\markboth{\textcolor{darkblue}{\textbf{\ipa{lo˧bæ˧˥}}}}{}
\textcolor{teal}{\mytextsc{nom}} \hspace{4pt} Ton~: MH\#.

Pont suspendu; pont de corde. La corde du pont suspendu aurait été faite d'écorces d'arbres, non de chanvre, car les cordes en chanvre se détériorent rapidement quand elles sont exposées à la pluie.
\textcolor{brown}{\zh{索桥,溜索。}}




\paragraph{\hspace{-0.5cm} {\Large \textcolor{darkblue}{\textbf{\ipa{lo˧bv̩˩-ʈʂʰɯ˩}}}}} \hypertarget{lo\string_Mbv\string_=\string_B-t`s`\string_hM\string_B1}{}
\markboth{\textcolor{darkblue}{\textbf{\ipa{lo˧bv̩˩-ʈʂʰɯ˩}}}}{}
\textcolor{teal}{\mytextsc{nom}} \hspace{4pt} Ton~: L\#-.

Éléphant.
\textcolor{brown}{\zh{象、大象。}}

 Emprunt~: tibétain

 \mytextsc{clf}~: \textcolor{darkblue}{\textbf{\ipa{pʰo˧˥}}} \textcolor{darkblue}{\textbf{\ipa{v̩˧}}} 

\paragraph{\hspace{-0.5cm} {\Large \textcolor{darkblue}{\textbf{\ipa{lo˧ɖʐɤ˩}}}}} \hypertarget{lo\string_Md`z`7\string_B1}{}
\markboth{\textcolor{darkblue}{\textbf{\ipa{lo˧ɖʐɤ˩}}}}{}
\textcolor{teal}{\mytextsc{nom}} \hspace{4pt} Ton~: L\#.

Serfouette, croc à trois dents: instrument à trois dents perpendiculaires au manche, pour ameublir la terre. Les modèles actuellement utilisés ont une tête en métal.
\textcolor{brown}{\zh{三齿耙。}}


\begin{exe}
\sn \textcolor{darkblue}{\textbf{\ipa{lo˧ɖʐɤ˩ ʈʂʰɯ˩-nɑ˩}}}
\trans \mytextsc{n}+\mytextsc{dem}+\mytextsc{clf}
\trans \textit{\textcolor{brown}{\zh{这把三齿耙}}}
\end{exe}
 \mytextsc{clf}~: \textcolor{darkblue}{\textbf{\ipa{nɑ˧}}} 

\paragraph{\hspace{-0.5cm} {\Large \textcolor{darkblue}{\textbf{\ipa{lo˧fv̩˧}}}}} \hypertarget{lo\string_Mfv\string_=\string_M1}{}
\markboth{\textcolor{darkblue}{\textbf{\ipa{lo˧fv̩˧}}}}{}
\textcolor{teal}{\mytextsc{adjectif}} \hspace{4pt} Ton~: .

Facile à faire.
\textcolor{brown}{\zh{容易,容易做。}}


\begin{exe}
\sn \textcolor{darkblue}{\textbf{\ipa{lo˧fv̩˧ | ʐwæ˩˥}}}
\trans très facile
\trans \textit{\textcolor{brown}{\zh{很容易}}}
\end{exe}


\paragraph{\hspace{-0.5cm} {\Large \textcolor{darkblue}{\textbf{\ipa{lo˧gv̩˩}}}}} \hypertarget{lo\string_Mgv\string_=\string_B1}{}
\markboth{\textcolor{darkblue}{\textbf{\ipa{lo˧gv̩˩}}}}{}
\textcolor{teal}{\mytextsc{nom}} \hspace{4pt} Ton~: L\#.

Ninglang; actuellement utilisé pour désigner un village na du comté de Ninglang, relativement proche du centre administratif.
\textcolor{brown}{\zh{宁蒗。}}


\begin{exe}
\sn \textcolor{darkblue}{\textbf{\ipa{lo˧gv̩˩-di˩mi˩}}}
\trans la plaine de Ninglang
\trans \textit{\textcolor{brown}{\zh{宁蒗坝子}}}
\end{exe}


\paragraph{\hspace{-0.5cm} {\Large \textcolor{darkblue}{\textbf{\ipa{lo˧hɑ˧}}}}} \hypertarget{lo\string_MhA\string_M1}{}
\markboth{\textcolor{darkblue}{\textbf{\ipa{lo˧hɑ˧}}}}{}
\textcolor{teal}{\mytextsc{adjectif}} \hspace{4pt} Ton~: .

Difficile, dur à faire.
\textcolor{brown}{\zh{难做。}}


\begin{exe}
\sn \textcolor{darkblue}{\textbf{\ipa{ʝi˧ lo˧hɑ˧}}}
\trans difficile à faire
\trans \textit{\textcolor{brown}{\zh{难做}}}
\end{exe}
\begin{exe}
\sn \textcolor{darkblue}{\textbf{\ipa{ʝi˧ lo˧hɑ˧ | ʐwæ˩˥}}}
\trans très difficile à faire
\trans \textit{\textcolor{brown}{\zh{非常难做}}}
\end{exe}


\paragraph{\hspace{-0.5cm} {\Large \textcolor{darkblue}{\textbf{\ipa{lo˧ʝi˧}}}}} \hypertarget{lo\string_Mj££i\string_M1}{}
\markboth{\textcolor{darkblue}{\textbf{\ipa{lo˧ʝi˧}}}}{}
\textcolor{teal}{\mytextsc{verbe}} \hspace{4pt} Ton~: .

Travailler.
\textcolor{brown}{\zh{劳动。}}


\textit{Voir~:} \textcolor{darkblue}{\textbf{\ipa{lo˧; ʝi˥1}}} 

\paragraph{\hspace{-0.5cm} {\Large \textcolor{darkblue}{\textbf{\ipa{lo˧ʝi˧-hĩ˧-hĩ\#˥}}}}} \hypertarget{lo\string_Mj££i\string_M-hi\string_~\string_M-hi\string_~\#\string_T1}{}
\markboth{\textcolor{darkblue}{\textbf{\ipa{lo˧ʝi˧-hĩ˧-hĩ\#˥}}}}{}
\textcolor{teal}{\mytextsc{nom}} \hspace{4pt} Ton~: \#H.

Travailleur (paysan, ouvrier…).
\textcolor{brown}{\zh{劳动人民,农民。}}


 \mytextsc{clf}~: \textcolor{darkblue}{\textbf{\ipa{v̩˧}}} 

\paragraph{\hspace{-0.5cm} {\Large \textcolor{darkblue}{\textbf{\ipa{lo˧lo˧}}}}} \hypertarget{lo\string_Mlo\string_M1}{}
\markboth{\textcolor{darkblue}{\textbf{\ipa{lo˧lo˧}}}}{}
\textcolor{teal}{\mytextsc{nom}} \hspace{4pt} Ton~: M.

Yi (groupe ethnique).
\textcolor{brown}{\zh{彝族。}}


 \mytextsc{clf}~: \textcolor{darkblue}{\textbf{\ipa{v̩˧}}} 

\paragraph{\hspace{-0.5cm} {\Large \textcolor{darkblue}{\textbf{\ipa{lo˧nɑ˥-bv̩˩}}}}} \hypertarget{lo\string_MnA\string_T-bv\string_=\string_B1}{}
\markboth{\textcolor{darkblue}{\textbf{\ipa{lo˧nɑ˥-bv̩˩}}}}{}
\textcolor{teal}{\mytextsc{nom}} \hspace{4pt} Ton~: .

Prison.
\textcolor{brown}{\zh{监狱。}}


\begin{exe}
\sn \textcolor{darkblue}{\textbf{\ipa{lo˧nɑ˥-bv̩˩-qo˩ ʈæ˩}}}
\trans enfermer en prison, mettre en prison
\trans \textit{\textcolor{brown}{\zh{放在监狱、关在监狱里}}}
\end{exe}
\textit{Voir~:} \hyperlink{ts£7\string_Mjo\string_B1}{\textcolor{darkblue}{\textbf{\ipa{tɕɤ˧jo˩}}}} 

\paragraph{\hspace{-0.5cm} {\Large \textcolor{darkblue}{\textbf{\ipa{lo˧ɲi˥}}}}} \hypertarget{lo\string_MJi\string_T1}{}
\markboth{\textcolor{darkblue}{\textbf{\ipa{lo˧ɲi˥}}}}{}
\textcolor{teal}{\mytextsc{nom}} \hspace{4pt} Ton~: H\#.

Doigt.
\textcolor{brown}{\zh{手指。}}


 \mytextsc{clf}~: \textcolor{darkblue}{\textbf{\ipa{ɭɯ˧}}} 

\paragraph{\hspace{-0.5cm} {\Large \textcolor{darkblue}{\textbf{\ipa{lo˧ɲi˥ | ɖɯ˧-ɭɯ˧}}}}} \hypertarget{lo\string_MJi\string_T | d`M\string_M-l\string_RM\string_M1}{}
\markboth{\textcolor{darkblue}{\textbf{\ipa{lo˧ɲi˥ | ɖɯ˧-ɭɯ˧}}}}{}
\textcolor{teal}{\mytextsc{nom}} \hspace{4pt} Ton~: H\# | M.

Index.
\textcolor{brown}{\zh{食指。}}




\paragraph{\hspace{-0.5cm} {\Large \textcolor{darkblue}{\textbf{\ipa{lo˧ɲi˥ | ɲi˧-ɭɯ˧}}}}} \hypertarget{lo\string_MJi\string_T | Ji\string_M-l\string_RM\string_M1}{}
\markboth{\textcolor{darkblue}{\textbf{\ipa{lo˧ɲi˥ | ɲi˧-ɭɯ˧}}}}{}
\textcolor{teal}{\mytextsc{nom}} \hspace{4pt} Ton~: H\# | M.

Majeur.
\textcolor{brown}{\zh{中指。}}




\paragraph{\hspace{-0.5cm} {\Large \textcolor{darkblue}{\textbf{\ipa{lo˧ɲi˥ | so˧-ɭɯ˧}}}}} \hypertarget{lo\string_MJi\string_T | so\string_M-l\string_RM\string_M1}{}
\markboth{\textcolor{darkblue}{\textbf{\ipa{lo˧ɲi˥ | so˧-ɭɯ˧}}}}{}
\textcolor{teal}{\mytextsc{nom}} \hspace{4pt} Ton~: .

Annulaire.
\textcolor{brown}{\zh{第四根手指。}}




\paragraph{\hspace{-0.5cm} {\Large \textcolor{darkblue}{\textbf{\ipa{lo˧ʂv̩˩}}}}} \hypertarget{lo\string_Ms`v\string_=\string_B1}{}
\markboth{\textcolor{darkblue}{\textbf{\ipa{lo˧ʂv̩˩}}}}{}
\textcolor{teal}{\mytextsc{nom}} \hspace{4pt} Ton~: L\#.

Luoshui: village du bord du Lac.
\textcolor{brown}{\zh{洛水村。}}




\paragraph{\hspace{-0.5cm} {\Large \textcolor{darkblue}{\textbf{\ipa{}}}}} \hypertarget{lo\string_Ms`v\string_=\string_B | -hi\string_B-nA\string_Mmi\string_M1}{}
\markboth{\textcolor{darkblue}{\textbf{\ipa{}}}}{}
\textcolor{teal}{\mytextsc{nom}} \hspace{4pt} Ton~: L\# | L-.

Lac Lugu.
\textcolor{brown}{\zh{泸沽湖。}}




\paragraph{\hspace{-0.5cm} {\Large \textcolor{darkblue}{\textbf{\ipa{lo˧tɑ˧-lo˧tɕi\#˥}}}}} \hypertarget{lo\string_MtA\string_M-lo\string_Mts£i\#\string_T1}{}
\markboth{\textcolor{darkblue}{\textbf{\ipa{lo˧tɑ˧-lo˧tɕi\#˥}}}}{}
\textcolor{teal}{\mytextsc{nom}} \hspace{4pt} Ton~: \#H.

Drapeau de prières.
\textcolor{brown}{\zh{经幡、风马旗(挂在山上)。}}

 Emprunt~: tibétain rlungrta

\begin{exe}
\sn \textcolor{darkblue}{\textbf{\ipa{lo˧tɑ˧-lo˧tɕi˧ | le˧-lɑ˧˥}}}
\trans imprimer un drapeau de prières; sens plus général: confectionner un drapeau de prières (chez soi, avant de se rendre sur le lieu où on l'installe)
\trans \textit{\textcolor{brown}{\zh{直译:印出一个经幡。也来指准备经幡的工作(到山上去挂之前)}}}
\end{exe}
 \mytextsc{clf}~: \textcolor{darkblue}{\textbf{\ipa{pɤ˥}}} \textcolor{darkblue}{\textbf{\ipa{pʰæ˧˥}}} 

\paragraph{\hspace{-0.5cm} {\Large \textcolor{darkblue}{\textbf{\ipa{lo˩}}} \textsubscript{1}}} \hypertarget{lo\string_B1}{}
\markboth{\textcolor{darkblue}{\textbf{\ipa{lo˩}}} \textsubscript{1}}{}
\textcolor{teal}{\mytextsc{verbe}} \hspace{4pt} Ton~: L.

Passer, franchir (un col).
\textcolor{brown}{\zh{过(垭口)。}}


\begin{exe}
\sn \textcolor{darkblue}{\textbf{\ipa{mv̩˩tɕo˧-lo˩}}}
\trans descendre (après avoir passé un col)
\trans \textit{\textcolor{brown}{\zh{往下过去(过了垭口以后)}}}
\end{exe}


\paragraph{\hspace{-0.5cm} {\Large \textcolor{darkblue}{\textbf{\ipa{lo˩}}} \textsubscript{2}}} \hypertarget{lo\string_B2}{}
\markboth{\textcolor{darkblue}{\textbf{\ipa{lo˩}}} \textsubscript{2}}{}
\textcolor{teal}{\mytextsc{nom}} \hspace{4pt} Ton~: L.

Vallée de montagne.
\textcolor{brown}{\zh{山谷。}}


\begin{exe}
\sn \textcolor{darkblue}{\textbf{\ipa{lo˧-qo˧}}}
\trans dans la vallée
\trans \textit{\textcolor{brown}{\zh{山谷里}}}
\end{exe}
 \mytextsc{clf}~: \textcolor{darkblue}{\textbf{\ipa{lo˩}}} 

\paragraph{\hspace{-0.5cm} {\Large \textcolor{darkblue}{\textbf{\ipa{lo˩\textsubscript{b}}}}}} \hypertarget{lo\string_Bb1}{}
\markboth{\textcolor{darkblue}{\textbf{\ipa{lo˩\textsubscript{b}}}}}{}
\textcolor{teal}{\mytextsc{classificateur}} \hspace{4pt} Ton~: L\textsubscript{b}.

Classificateur des vallées.
\textcolor{brown}{\zh{量词:谷。}}


\begin{exe}
\sn \textcolor{darkblue}{\textbf{\ipa{hĩ˧-ɻ̃˧ | ɖɯ˧-lo˩}}}
\trans tous les membres d'une grande famille: '[toute la population d']une vallée'
\trans \textit{\textcolor{brown}{\zh{住在一座山谷里的所有人(直译:‘一山谷的人’)}}}
\end{exe}
\begin{exe}
\sn \textcolor{darkblue}{\textbf{\ipa{si˧dzi˩ | ɖɯ˧-lo˩}}}
\trans une grande quantité d'arbres: 'une vallée [couverte] d'arbres'
\trans \textit{\textcolor{brown}{\zh{一山谷的树,一片森林(直译:‘一山谷的树’)}}}
\end{exe}


\paragraph{\hspace{-0.5cm} {\Large \textcolor{darkblue}{\textbf{\ipa{lo˩bɤ˩}}}}} \hypertarget{lo\string_Bb7\string_B1}{}
\markboth{\textcolor{darkblue}{\textbf{\ipa{lo˩bɤ˩}}}}{}
\textcolor{teal}{\mytextsc{nom}} \hspace{4pt} Ton~: L.

Paume.
\textcolor{brown}{\zh{手掌。}}


 \mytextsc{clf}~: \textcolor{darkblue}{\textbf{\ipa{ɭɯ˧}}} 

\paragraph{\hspace{-0.5cm} {\Large \textcolor{darkblue}{\textbf{\ipa{lo˩bv̩˧-ɭɯ˩}}}}} \hypertarget{lo\string_Bbv\string_=\string_M-l\string_RM\string_B1}{}
\markboth{\textcolor{darkblue}{\textbf{\ipa{lo˩bv̩˧-ɭɯ˩}}}}{}
\textcolor{teal}{\mytextsc{nom}} \hspace{4pt} Ton~: LM-L.

Partie saillante du coude, qd le bras est replié.
\textcolor{brown}{\zh{肘。}}


 \mytextsc{clf}~: \textcolor{darkblue}{\textbf{\ipa{ɭɯ˧}}} 

\paragraph{\hspace{-0.5cm} {\Large \textcolor{darkblue}{\textbf{\ipa{lo˩dv̩\#˥}}}}} \hypertarget{lo\string_Bdv\string_=\#\string_T1}{}
\markboth{\textcolor{darkblue}{\textbf{\ipa{lo˩dv̩\#˥}}}}{}
\textcolor{teal}{\mytextsc{nom}} \hspace{4pt} Ton~: LM+\#H.

Manchot.
\textcolor{brown}{\zh{独臂人:缺一只胳膊(手)的人。}}


 \mytextsc{clf}~: \textcolor{darkblue}{\textbf{\ipa{v̩˧}}} 

\paragraph{\hspace{-0.5cm} {\Large \textcolor{darkblue}{\textbf{\ipa{lo˩dzi˩}}}}} \hypertarget{lo\string_Bdzi\string_B1}{}
\markboth{\textcolor{darkblue}{\textbf{\ipa{lo˩dzi˩}}}}{}
\textcolor{teal}{\mytextsc{classificateur}} \hspace{4pt} Ton~: L.

Classificateur des poignées (à deux mains).
\textcolor{brown}{\zh{量词:捧(用两只手)。}}




\paragraph{\hspace{-0.5cm} {\Large \textcolor{darkblue}{\textbf{\ipa{lo˩dʑo˥}}}}} \hypertarget{lo\string_Bdz£o\string_T1}{}
\markboth{\textcolor{darkblue}{\textbf{\ipa{lo˩dʑo˥}}}}{}
\textcolor{teal}{\mytextsc{nom}} \hspace{4pt} Ton~: LH.

Bracelet.
\textcolor{brown}{\zh{手镯。}}


\begin{exe}
\sn \textcolor{darkblue}{\textbf{\ipa{ŋv̩˩-lo˩dʑo˧ (+ɲi˩)}}}
\trans bracelet en argent
\trans \textit{\textcolor{brown}{\zh{银手镯}}}
\end{exe}
\begin{exe}
\sn \textcolor{darkblue}{\textbf{\ipa{hæ̃˩-lo˩dʑo˥ (+ɲi˩)}}}
\trans bracelet en or
\trans \textit{\textcolor{brown}{\zh{金手镯}}}
\end{exe}
\begin{exe}
\sn \textcolor{darkblue}{\textbf{\ipa{jo˧-lo˥dʑo˩}}}
\trans bracelet en jade
\trans \textit{\textcolor{brown}{\zh{玉手镯}}}
\end{exe}
\begin{exe}
\sn \textcolor{darkblue}{\textbf{\ipa{lo˩dʑo˥ kʰɯ˩}}}
\trans mettre un bracelet
\trans \textit{\textcolor{brown}{\zh{戴上手镯}}}
\end{exe}
 \mytextsc{clf}~: \textcolor{darkblue}{\textbf{\ipa{pʰo˧˥}}} 

\paragraph{\hspace{-0.5cm} {\Large \textcolor{darkblue}{\textbf{\ipa{lo˩ɖɯ˧}}}}} \hypertarget{lo\string_Bd`M\string_M1}{}
\markboth{\textcolor{darkblue}{\textbf{\ipa{lo˩ɖɯ˧}}}}{}
\textcolor{teal}{\mytextsc{adjectif}} \hspace{4pt} Ton~: LM.

Généreux.
\textcolor{brown}{\zh{大方。}}




\paragraph{\hspace{-0.5cm} {\Large \textcolor{darkblue}{\textbf{\ipa{lo˩-gv̩˧dv̩˧}}}}} \hypertarget{lo\string_B-gv\string_=\string_Mdv\string_=\string_M1}{}
\markboth{\textcolor{darkblue}{\textbf{\ipa{lo˩-gv̩˧dv̩˧}}}}{}
\textcolor{teal}{\mytextsc{nom}} \hspace{4pt} Ton~: L-.

Dos de la main.
\textcolor{brown}{\zh{手背。}}


 \mytextsc{clf}~: \textcolor{darkblue}{\textbf{\ipa{kʰwɤ˥}}} 

\paragraph{\hspace{-0.5cm} {\Large \textcolor{darkblue}{\textbf{\ipa{lo˩jɤ˧}}}}} \hypertarget{lo\string_Bj7\string_M1}{}
\markboth{\textcolor{darkblue}{\textbf{\ipa{lo˩jɤ˧}}}}{}
\textcolor{teal}{\mytextsc{nom}} \hspace{4pt} Ton~: LM.

Pièce d'argent.
\textcolor{brown}{\zh{银元。}}


\begin{exe}
\sn \textcolor{darkblue}{\textbf{\ipa{lo˩jɤ˧ | ɖɯ˧-pʰæ˧˥}}}
\trans une pièce d'argent
\trans \textit{\textcolor{brown}{\zh{一块银元}}}
\end{exe}


\paragraph{\hspace{-0.5cm} {\Large \textcolor{darkblue}{\textbf{\ipa{lo˩ko˧}}}}} \hypertarget{lo\string_Bko\string_M1}{}
\markboth{\textcolor{darkblue}{\textbf{\ipa{lo˩ko˧}}}}{}
\textcolor{teal}{\mytextsc{nom}} \hspace{4pt} Ton~: LM.

Casserole, pour cuire les céréales, les légumes, les soupes... Elle était autrefois en cuivre.
\textcolor{brown}{\zh{煮饭或煮汤的锣锅。在过去,锣锅一般是铜做的。}}

 Emprunt~: chinois \textcolor{brown}{\zh{锣锅}}

\begin{exe}
\sn \textcolor{darkblue}{\textbf{\ipa{lo˩ko˧: | hɑ˧ tɕɤ˩-di˩! | æ˧-v̩˧, | dʑɯ˩-kʰɯ˩-di˩˥! | ʈʂʰɤ˧ho˥, | dʑɯ˩ tɕɯ˩-di˩˥! |}}}
\trans La casserole (\textcolor{darkblue}{\textbf{\ipa{/lo˩ko˧/}}}), ça sert à cuire la nourriture! La casserole de cuivre (\textcolor{darkblue}{\textbf{\ipa{/æ˧-v̩˧/}}}), ça sert à mettre de l'eau! La bouilloire (\textcolor{darkblue}{\textbf{\ipa{/ʈʂʰɤ˩ho˥/}}}), ça sert à faire bouillir l'eau! (Résumé des emplois des trois sortes de casseroles qui étaient en usage à Yongning vers le milieu du XXe siècle.)
\trans \textit{\textcolor{brown}{\zh{锣锅,是用来煮饭的!铜锅,是放水用的!水壶,是来煮水的!(描写永宁二十世纪中使用的三种锅)}}}
\end{exe}
 \mytextsc{clf}~: \textcolor{darkblue}{\textbf{\ipa{ɭɯ˧}}} 

\paragraph{\hspace{-0.5cm} {\Large \textcolor{darkblue}{\textbf{\ipa{lo˩mi˧}}}}} \hypertarget{lo\string_Bmi\string_M1}{}
\markboth{\textcolor{darkblue}{\textbf{\ipa{lo˩mi˧}}}}{}
\textcolor{teal}{\mytextsc{nom}} \hspace{4pt} Ton~: LM.

Pouce.
\textcolor{brown}{\zh{大拇指。}}


 \mytextsc{clf}~: \textcolor{darkblue}{\textbf{\ipa{ɭɯ˧}}} 

\paragraph{\hspace{-0.5cm} {\Large \textcolor{darkblue}{\textbf{\ipa{lo˩mi˧-qɑ˩}}}}} \hypertarget{lo\string_Bmi\string_M-qA\string_B1}{}
\markboth{\textcolor{darkblue}{\textbf{\ipa{lo˩mi˧-qɑ˩}}}}{}
\textcolor{teal}{\mytextsc{nom}} \hspace{4pt} Ton~: LM-L.

Espace entre le pouce et l'index.
\textcolor{brown}{\zh{虎口。}}


 \mytextsc{clf}~: \textcolor{darkblue}{\textbf{\ipa{ɭɯ˧}}} 

\paragraph{\hspace{-0.5cm} {\Large \textcolor{darkblue}{\textbf{\ipa{lo˩pv̩˧˥}}}}} \hypertarget{lo\string_Bpv\string_=\string_M\string_T1}{}
\markboth{\textcolor{darkblue}{\textbf{\ipa{lo˩pv̩˧˥}}}}{}
\textcolor{teal}{\mytextsc{nom}} \hspace{4pt} Ton~: LM+MH\#.

Anneau.
\textcolor{brown}{\zh{戒指。}}


\begin{exe}
\sn \textcolor{darkblue}{\textbf{\ipa{ŋv̩˩-lo˩pv̩˩}}}
\trans anneau en argent
\trans \textit{\textcolor{brown}{\zh{银戒指}}}
\end{exe}
\begin{exe}
\sn \textcolor{darkblue}{\textbf{\ipa{hæ̃˩-lo˩pv̩˩}}}
\trans anneau en or
\trans \textit{\textcolor{brown}{\zh{金戒指}}}
\end{exe}
 \mytextsc{clf}~: \textcolor{darkblue}{\textbf{\ipa{ɭɯ˧}}} 

\paragraph{\hspace{-0.5cm} {\Large \textcolor{darkblue}{\textbf{\ipa{lo˩qʰv̩˩}}}}} \hypertarget{lo\string_Bq\string_hv\string_=\string_B1}{}
\markboth{\textcolor{darkblue}{\textbf{\ipa{lo˩qʰv̩˩}}}}{}
\textcolor{teal}{\mytextsc{nom}} \hspace{4pt} Ton~: L.

Vallée, gorge, ravin.
\textcolor{brown}{\zh{山沟。}}


 \mytextsc{clf}~: \textcolor{darkblue}{\textbf{\ipa{lo˩}}} 

\paragraph{\hspace{-0.5cm} {\Large \textcolor{darkblue}{\textbf{\ipa{lo˩qʰwɤ˧}}}}} \hypertarget{lo\string_Bq\string_hw7\string_M1}{}
\markboth{\textcolor{darkblue}{\textbf{\ipa{lo˩qʰwɤ˧}}}}{}
\textcolor{teal}{\mytextsc{nom}} \hspace{4pt} Ton~: LM.
\ding{202} 
Bras.
\textcolor{brown}{\zh{胳膊。}}


\begin{exe}
\sn \textcolor{darkblue}{\textbf{\ipa{lo˩qʰwɤ˧ li˧}}}
\trans regarder le bras
\trans \textit{\textcolor{brown}{\zh{看胳膊}}}
\end{exe}
\ding{203} 
Main.
\textcolor{brown}{\zh{手。}}


\begin{exe}
\sn \textcolor{darkblue}{\textbf{\ipa{lo˩qʰwɤ˧ ʈʂʰæ˧}}}
\trans se laver les mains
\trans \textit{\textcolor{brown}{\zh{洗手}}}
\end{exe}
 \mytextsc{clf}~: \textcolor{darkblue}{\textbf{\ipa{pʰo˧˥}}} 

\paragraph{\hspace{-0.5cm} {\Large \textcolor{darkblue}{\textbf{\ipa{lo˩qʰwɤ˧-kʰɯ˧ʑi˧˥}}}}} \hypertarget{lo\string_Bq\string_hw7\string_M-k\string_hM\string_Mz£i\string_M\string_T1}{}
\markboth{\textcolor{darkblue}{\textbf{\ipa{lo˩qʰwɤ˧-kʰɯ˧ʑi˧˥}}}}{}
\textcolor{teal}{\mytextsc{nom}} \hspace{4pt} Ton~: LM+MH\#.

Gant.
\textcolor{brown}{\zh{手套。}}




\paragraph{\hspace{-0.5cm} {\Large \textcolor{darkblue}{\textbf{\ipa{lo˩ʁwæ\#˥}}}}} \hypertarget{lo\string_BRw\{\#\string_T1}{}
\markboth{\textcolor{darkblue}{\textbf{\ipa{lo˩ʁwæ\#˥}}}}{}
\textcolor{teal}{\mytextsc{nom}} \hspace{4pt} Ton~: LM+\#H.

Gaucher.
\textcolor{brown}{\zh{左撇子。}}




\paragraph{\hspace{-0.5cm} {\Large \textcolor{darkblue}{\textbf{\ipa{lo˩tʰo˧}}}}} \hypertarget{lo\string_Bt\string_ho\string_M1}{}
\markboth{\textcolor{darkblue}{\textbf{\ipa{lo˩tʰo˧}}}}{}
\textcolor{teal}{\mytextsc{nom}} \hspace{4pt} Ton~: LM.

Menottes: chaîne de fer pour attacher les mains d'un criminel.
\textcolor{brown}{\zh{手铐。}}


\begin{exe}
\sn \textcolor{darkblue}{\textbf{\ipa{lo˩tʰo˧ kʰɯ˧˥}}}
\trans passer les menottes à quelqu'un
\trans \textit{\textcolor{brown}{\zh{戴上手铐}}}
\end{exe}


\paragraph{\hspace{-0.5cm} {\Large \textcolor{darkblue}{\textbf{\ipa{lo˩tsʰɯ˥-sɑ˩}}}}} \hypertarget{lo\string_Bts\string_hM\string_T-sA\string_B1}{}
\markboth{\textcolor{darkblue}{\textbf{\ipa{lo˩tsʰɯ˥-sɑ˩}}}}{}
\textcolor{teal}{\mytextsc{nom}} \hspace{4pt} Ton~: LH-.

Viande des membres antérieurs.
\textcolor{brown}{\zh{牲畜前腿的肉。}}




\paragraph{\hspace{-0.5cm} {\Large \textcolor{darkblue}{\textbf{\ipa{lo˩ʈv̩˧}}}}} \hypertarget{lo\string_Bt`v\string_=\string_M1}{}
\markboth{\textcolor{darkblue}{\textbf{\ipa{lo˩ʈv̩˧}}}}{}
\textcolor{teal}{\mytextsc{nom}} \hspace{4pt} Ton~: LM.

Poing.
\textcolor{brown}{\zh{拳。}}


 \mytextsc{clf}~: \textcolor{darkblue}{\textbf{\ipa{ʈv̩˩}}} 

\paragraph{\hspace{-0.5cm} {\Large \textcolor{darkblue}{\textbf{\ipa{lo˩ʈʰɯ˧}}}}} \hypertarget{lo\string_Bt`\string_hM\string_M1}{}
\markboth{\textcolor{darkblue}{\textbf{\ipa{lo˩ʈʰɯ˧}}}}{}
\textcolor{teal}{\mytextsc{nom}} \hspace{4pt} Ton~: LM.

Coude.
\textcolor{brown}{\zh{肘。}}


 \mytextsc{clf}~: \textcolor{darkblue}{\textbf{\ipa{ʈv̩˩}}} 

\paragraph{\hspace{-0.5cm} {\Large \textcolor{darkblue}{\textbf{\ipa{lo˩ʈʂæ˧˥}}}}} \hypertarget{lo\string_Bt`s`\{\string_M\string_T1}{}
\markboth{\textcolor{darkblue}{\textbf{\ipa{lo˩ʈʂæ˧˥}}}}{}
\textcolor{teal}{\mytextsc{nom}} \hspace{4pt} Ton~: LM+MH\#.

Articulations du bras: le poignet, mais aussi le coude.
\textcolor{brown}{\zh{手臂的关节(手腕、肘弯)。}}


 \mytextsc{clf}~: \textcolor{darkblue}{\textbf{\ipa{ʈʂæ˧˥}}} 

\paragraph{\hspace{-0.5cm} {\Large \textcolor{darkblue}{\textbf{\ipa{lo˧˥}}} \textsubscript{1}}} \hypertarget{lo\string_M\string_T1}{}
\markboth{\textcolor{darkblue}{\textbf{\ipa{lo˧˥}}} \textsubscript{1}}{}
\textcolor{teal}{\mytextsc{adjectif}} \hspace{4pt} Ton~: MH.

Épais.
\textcolor{brown}{\zh{厚。}}


\begin{exe}
\sn \textcolor{darkblue}{\textbf{\ipa{ʈʂʰɯ˧ | lo˧-pæ˧-ɻæ˥-gv̩˩!}}}
\trans c'est très épais!
\trans \textit{\textcolor{brown}{\zh{很厚啊!}}}
\end{exe}


\paragraph{\hspace{-0.5cm} {\Large \textcolor{darkblue}{\textbf{\ipa{lo˧˥}}} \textsubscript{2}}} \hypertarget{lo\string_M\string_T2}{}
\markboth{\textcolor{darkblue}{\textbf{\ipa{lo˧˥}}} \textsubscript{2}}{}
\textcolor{teal}{\mytextsc{verbe}} \hspace{4pt} Ton~: MH.

Joindre les mains en signe de soumission.
\textcolor{brown}{\zh{拱手作揖。}}


\begin{exe}
\sn \textcolor{darkblue}{\textbf{\ipa{tsʰɤ˧tsʰɤ˧ lo˧˥}}}
\trans rendre hommage à, joindre les mains en signe de soumission/respect
\trans \textit{\textcolor{brown}{\zh{拱手作揖}}}
\end{exe}
\begin{exe}
\sn \textcolor{darkblue}{\textbf{\ipa{tsʰɤ˧tsʰɤ˧ | le˧-lo˧-ze˥}}}
\trans \mytextsc{accomp} \string_ \mytextsc{pfv}
\trans \textit{\textcolor{brown}{\zh{\mytextsc{accomp} \string_ \mytextsc{pfv}}}}
\end{exe}


\paragraph{\hspace{-0.5cm} {\Large \textcolor{darkblue}{\textbf{\ipa{*lo˩˧}}}}} \hypertarget{*lo\string_B\string_M1}{}
\markboth{\textcolor{darkblue}{\textbf{\ipa{*lo˩˧}}}}{}
\textcolor{teal}{\mytextsc{nom}} \hspace{4pt} Ton~: LM.

Pouce (forme reconstruite d'après le disyllabe).
\textcolor{brown}{\zh{大拇指(单音节,按照双音节词构拟出来的)。}}




\paragraph{\hspace{-0.5cm} {\Large \textcolor{darkblue}{\textbf{\ipa{lv̩˧}}} \textsubscript{1}}} \hypertarget{lv\string_=\string_M1}{}
\markboth{\textcolor{darkblue}{\textbf{\ipa{lv̩˧}}} \textsubscript{1}}{}
\textcolor{teal}{\mytextsc{nom}} \hspace{4pt} Ton~: M.

Champs.
\textcolor{brown}{\zh{田地。}}


 \mytextsc{clf}~: \textcolor{darkblue}{\textbf{\ipa{kɤ˧˥}}} 

\paragraph{\hspace{-0.5cm} {\Large \textcolor{darkblue}{\textbf{\ipa{lv̩˧}}} \textsubscript{2}}} \hypertarget{lv\string_=\string_M2}{}
\markboth{\textcolor{darkblue}{\textbf{\ipa{lv̩˧}}} \textsubscript{2}}{}
\textcolor{teal}{\mytextsc{nom}} \hspace{4pt} Ton~: M.

Grain (pour chevaux ou vaches), picotin.
\textcolor{brown}{\zh{喂给马或牛的粮食。}}


\begin{exe}
\sn \textcolor{darkblue}{\textbf{\ipa{ʐwæ˧-lv̩˧}}}
\trans grain pour pour cheval, picotin; même sens que: \textcolor{darkblue}{\textbf{\ipa{/ʐwæ˧-ɭɯ\#˥/}}}
\trans \textit{\textcolor{brown}{\zh{喂给马的粮食}}}
\end{exe}
\textit{Voir~:} \hyperlink{z`w\{\string_M-l\string_RM\#\string_T1}{\textcolor{darkblue}{\textbf{\ipa{ʐwæ˧-ɭɯ\#˥}}}} 

\paragraph{\hspace{-0.5cm} {\Large \textcolor{darkblue}{\textbf{\ipa{lv̩˧˥}}}}} \hypertarget{lv\string_=\string_M\string_T1}{}
\markboth{\textcolor{darkblue}{\textbf{\ipa{lv̩˧˥}}}}{}
\textcolor{teal}{\mytextsc{nom}} \hspace{4pt} Ton~: MH.

Larve.
\textcolor{brown}{\zh{蛆。}}




\paragraph{\hspace{-0.5cm} {\Large \textcolor{darkblue}{\textbf{\ipa{lv̩˧˥}}} \textsubscript{1}}} \hypertarget{lv\string_=\string_M\string_T1}{}
\markboth{\textcolor{darkblue}{\textbf{\ipa{lv̩˧˥}}} \textsubscript{1}}{}
\textcolor{teal}{\mytextsc{verbe}} \hspace{4pt} Ton~: MH.

Garder les animaux, mener paître les animaux.
\textcolor{brown}{\zh{放牧。}}


\begin{exe}
\sn \textcolor{darkblue}{\textbf{\ipa{go˩bo˥ lv̩˩}}}
\trans mener paître le bétail, garder le bétail
\trans \textit{\textcolor{brown}{\zh{放牧牲畜}}}
\end{exe}
\begin{exe}
\sn \textcolor{darkblue}{\textbf{\ipa{ʐwæ˧ lv̩˩}}}
\trans mener paître les chevaux
\trans \textit{\textcolor{brown}{\zh{放马}}}
\end{exe}
\begin{exe}
\sn \textcolor{darkblue}{\textbf{\ipa{ʝi˧ lv̩˩}}}
\trans mener paître les vaches
\trans \textit{\textcolor{brown}{\zh{放牛}}}
\end{exe}
\begin{exe}
\sn \textcolor{darkblue}{\textbf{\ipa{bo˩ lv̩˩˥}}}
\trans garder les cochons
\trans \textit{\textcolor{brown}{\zh{放猪}}}
\end{exe}
\begin{exe}
\sn \textcolor{darkblue}{\textbf{\ipa{tsʰɯ˧ lv̩˥}}}
\trans mener paître les chèvres
\trans \textit{\textcolor{brown}{\zh{放山羊}}}
\end{exe}
\begin{exe}
\sn \textcolor{darkblue}{\textbf{\ipa{ɖɯ˧-hɤ˧ mɤ˧-lv̩˩\textasciitilde{}lv̩˩}}}
\trans paresseux, qui ne s'occupe de rien
\trans \textit{\textcolor{brown}{\zh{懒,什么也不管}}}
\end{exe}


\paragraph{\hspace{-0.5cm} {\Large \textcolor{darkblue}{\textbf{\ipa{lv̩˧˥}}} \textsubscript{2}}} \hypertarget{lv\string_=\string_M\string_T2}{}
\markboth{\textcolor{darkblue}{\textbf{\ipa{lv̩˧˥}}} \textsubscript{2}}{}
\textcolor{teal}{\mytextsc{verbe}} \hspace{4pt} Ton~: MH.

S'enfuir.
\textcolor{brown}{\zh{逃跑,逃掉。}}




\paragraph{\hspace{-0.5cm} {\Large \textcolor{darkblue}{\textbf{\ipa{lv̩˥}}}}} \hypertarget{lv\string_=\string_T1}{}
\markboth{\textcolor{darkblue}{\textbf{\ipa{lv̩˥}}}}{}
\textcolor{teal}{\mytextsc{verbe}} \hspace{4pt} Ton~: H.

Enrouler (du fil); emballer.
\textcolor{brown}{\zh{缠(线……)、裹(毡子……)。}}


\begin{exe}
\sn \textcolor{darkblue}{\textbf{\ipa{le˧-qo˥-lv̩˩}}}
\trans enrouler
\trans \textit{\textcolor{brown}{\zh{裹起来}}}
\end{exe}
\begin{exe}
\sn \textcolor{darkblue}{\textbf{\ipa{kʰɯ˧ qo˧-lv̩˥}}}
\trans enrouler du fil
\trans \textit{\textcolor{brown}{\zh{缠线}}}
\end{exe}
\begin{exe}
\sn \textcolor{darkblue}{\textbf{\ipa{qo˧-lv̩˩}}}
\trans même sens: enrouler
\trans \textit{\textcolor{brown}{\zh{裹}}}
\end{exe}


\paragraph{\hspace{-0.5cm} {\Large \textcolor{darkblue}{\textbf{\ipa{lv̩˩\textsubscript{a}}}} \textsubscript{1}}} \hypertarget{lv\string_=\string_Ba1}{}
\markboth{\textcolor{darkblue}{\textbf{\ipa{lv̩˩\textsubscript{a}}}} \textsubscript{1}}{}
\textcolor{teal}{\mytextsc{verbe}} \hspace{4pt} Ton~: L\textsubscript{a}.

Aboyer.
\textcolor{brown}{\zh{狗吠。}}


\begin{exe}
\sn \textcolor{darkblue}{\textbf{\ipa{kʰv̩˩mi˩ lv̩˥ |}}}
\trans le chien aboie
\trans \textit{\textcolor{brown}{\zh{狗吠}}}
\end{exe}
\begin{exe}
\sn \textcolor{darkblue}{\textbf{\ipa{kʰv̩˩ lv̩˥-dʑo˩ |}}}
\trans le chien est en train d'aboyer
\trans \textit{\textcolor{brown}{\zh{狗在叫}}}
\end{exe}
\begin{exe}
\sn \textcolor{darkblue}{\textbf{\ipa{ɖɯ˧-lv̩˧\textasciitilde{}lv̩˥-ɻ̍˩}}}
\trans \mytextsc{délimitatif} \string_ \mytextsc{red} \mytextsc{inchoatif}
\trans \textit{\textcolor{brown}{\zh{叫一叫}}}
\end{exe}


\paragraph{\hspace{-0.5cm} {\Large \textcolor{darkblue}{\textbf{\ipa{lv̩˩\textsubscript{a}}}} \textsubscript{2}}} \hypertarget{lv\string_=\string_Ba2}{}
\markboth{\textcolor{darkblue}{\textbf{\ipa{lv̩˩\textsubscript{a}}}} \textsubscript{2}}{}
\textcolor{teal}{\mytextsc{verbe}} \hspace{4pt} Ton~: L\textsubscript{a}.

Enrouler (un tissu).
\textcolor{brown}{\zh{把布卷起来。}}


\begin{exe}
\sn \textcolor{darkblue}{\textbf{\ipa{le˧-qæ˥-lv̩˩}}}
\trans enrouler
\trans \textit{\textcolor{brown}{\zh{卷起来}}}
\end{exe}
\begin{exe}
\sn \textcolor{darkblue}{\textbf{\ipa{le˧-lv̩˧\textasciitilde{}lv̩˧}}}
\trans \mytextsc{accomp} \mytextsc{red}
\end{exe}
\begin{exe}
\sn \textcolor{darkblue}{\textbf{\ipa{tso˧\textasciitilde{}tso˧ lv̩˧\textasciitilde{}lv̩˧}}}
\trans enrouler des choses
\trans \textit{\textcolor{brown}{\zh{卷东西}}}
\end{exe}
\begin{exe}
\sn \textcolor{darkblue}{\textbf{\ipa{ɖɯ˧-kʰwɤ˧ lv̩˥}}}
\trans enrouler quelque chose
\trans \textit{\textcolor{brown}{\zh{卷一块(东西)}}}
\end{exe}


\paragraph{\hspace{-0.5cm} {\Large \textcolor{darkblue}{\textbf{\ipa{lv̩˩\textsubscript{a}}}} \textsubscript{3}}} \hypertarget{lv\string_=\string_Ba3}{}
\markboth{\textcolor{darkblue}{\textbf{\ipa{lv̩˩\textsubscript{a}}}} \textsubscript{3}}{}
\textcolor{teal}{\mytextsc{verbe}} \hspace{4pt} Ton~: L\textsubscript{a}.

Labourer.
\textcolor{brown}{\zh{耕种。}}


\begin{exe}
\sn \textcolor{darkblue}{\textbf{\ipa{le˧-lv̩˩-ze˩}}}
\trans \mytextsc{accomp} \string_ \mytextsc{pfv}
\trans \textit{\textcolor{brown}{\zh{耕种了}}}
\end{exe}
\begin{exe}
\sn \textcolor{darkblue}{\textbf{\ipa{ʝi˧-lv̩˧˥}}}
\trans labourer
\trans \textit{\textcolor{brown}{\zh{耕种}}}
\end{exe}
\begin{exe}
\sn \textcolor{darkblue}{\textbf{\ipa{dʑi˧mi˧ lv̩˧˥ / dʑi˧mi˧ lv̩˧-ze˥}}}
\trans labourer avec un buffle
\trans \textit{\textcolor{brown}{\zh{用水牛耕田}}}
\end{exe}
\begin{exe}
\sn \textcolor{darkblue}{\textbf{\ipa{ʝi˧ ɖɯ˧-lv̩˧\textasciitilde{}lv̩˥-ɻ̍˩}}}
\trans labourer un peu
\trans \textit{\textcolor{brown}{\zh{耕一耕}}}
\end{exe}


\paragraph{\hspace{-0.5cm} {\Large \textcolor{darkblue}{\textbf{\ipa{lv̩˩\textsubscript{a}}}} \textsubscript{4}}} \hypertarget{lv\string_=\string_Ba4}{}
\markboth{\textcolor{darkblue}{\textbf{\ipa{lv̩˩\textsubscript{a}}}} \textsubscript{4}}{}
\textcolor{teal}{\mytextsc{verbe}} \hspace{4pt} Ton~: L\textsubscript{a}.

Suffire.
\textcolor{brown}{\zh{足够。}}


\begin{exe}
\sn \textcolor{darkblue}{\textbf{\ipa{ə˩-lv̩˩˥? / ə˩-lv̩˩-ze˥?}}}
\trans est-ce que ça (te) suffit ?
\trans \textit{\textcolor{brown}{\zh{够了吗?}}}
\end{exe}


\paragraph{\hspace{-0.5cm} {\Large \textcolor{darkblue}{\textbf{\ipa{lv̩˧dʑɯ˥}}}}} \hypertarget{lv\string_=\string_Mdz£M\string_T1}{}
\markboth{\textcolor{darkblue}{\textbf{\ipa{lv̩˧dʑɯ˥}}}}{}
\textcolor{teal}{\mytextsc{nom}} \hspace{4pt} Ton~: H\#.

Éclats de pierre, débris de pierre, petits bouts de pierre (ne veut pas dire ``sable").
\textcolor{brown}{\zh{零碎的石头块。}}


 \mytextsc{clf}~: \textcolor{darkblue}{\textbf{\ipa{ʈʂwɤ˧}}} 

\paragraph{\hspace{-0.5cm} {\Large \textcolor{darkblue}{\textbf{\ipa{lv̩˩ʝi˧}}}}} \hypertarget{lv\string_=\string_Bj££i\string_M1}{}
\markboth{\textcolor{darkblue}{\textbf{\ipa{lv̩˩ʝi˧}}}}{}
\textcolor{teal}{\mytextsc{verbe}} \hspace{4pt} Ton~: LM.

Enregistrer.
\textcolor{brown}{\zh{录音(汉语借词)。}}

 Emprunt~: chinois \textcolor{brown}{\zh{录音}}

\begin{exe}
\sn \textcolor{darkblue}{\textbf{\ipa{hɑ˧ le˧-dzɯ˧-se˥, | lv̩˩ ʝi˧-bi˧ !}}}
\trans Quand (on) aura fini de manger, (on) fera un enregistrement !
\trans \textit{\textcolor{brown}{\zh{吃完饭,就录音吧! / 吃完饭就可以录音了!}}}
\end{exe}


\paragraph{\hspace{-0.5cm} {\Large \textcolor{darkblue}{\textbf{\ipa{lv̩˧mi˧}}}}} \hypertarget{lv\string_=\string_Mmi\string_M1}{}
\markboth{\textcolor{darkblue}{\textbf{\ipa{lv̩˧mi˧}}}}{}
\textcolor{teal}{\mytextsc{nom}} \hspace{4pt} Ton~: M.

Pierre.
\textcolor{brown}{\zh{石头。}}


\begin{exe}
\sn \textcolor{darkblue}{\textbf{\ipa{kʰv̩˧pʰæ˧tɕi˩, | lv̩˧mi˧ dzɯ˧-bi˧-ʁo˧-ho˩!}}}
\trans 'Quand on est jeune, on mangerait des pierres!' (Signification: quand on est jeune, on mange de tout, on a une digestion solide; tandis que quand on est vieux, on a facilement mal au ventre, dès qu'on mange quelque chose d'un peu indigeste, une nourriture ``trop dure".)
\trans \textit{\textcolor{brown}{\zh{‘年轻人,石头都能吃!’(意思:年轻人消化好,吃什么都行,而人变老就不那么容易消化了,要注意吃什么。)}}}
\end{exe}
 \mytextsc{clf}~: \textcolor{darkblue}{\textbf{\ipa{ɭɯ˧}}} 

\paragraph{\hspace{-0.5cm} {\Large \textcolor{darkblue}{\textbf{\ipa{lv̩˧mi˧-bo\#˥}}}}} \hypertarget{lv\string_=\string_Mmi\string_M-bo\#\string_T1}{}
\markboth{\textcolor{darkblue}{\textbf{\ipa{lv̩˧mi˧-bo\#˥}}}}{}
\textcolor{teal}{\mytextsc{nom}} \hspace{4pt} Ton~: \#H.

Mur en pierre.
\textcolor{brown}{\zh{石墙。}}


 \mytextsc{clf}~: \textcolor{darkblue}{\textbf{\ipa{ɭɯ˧}}} 

\paragraph{\hspace{-0.5cm} {\Large \textcolor{darkblue}{\textbf{\ipa{lv̩˧mi˧-dʑɯ˧dʑɯ˩}}}}} \hypertarget{lv\string_=\string_Mmi\string_M-dz£M\string_Mdz£M\string_B1}{}
\markboth{\textcolor{darkblue}{\textbf{\ipa{lv̩˧mi˧-dʑɯ˧dʑɯ˩}}}}{}
\textcolor{teal}{\mytextsc{nom}} \hspace{4pt} Ton~: \mytextsc{L}\#.

Éclats de pierre, débris de pierre, petits bouts de pierre (ne veut pas dire ``sable").
\textcolor{brown}{\zh{零碎的石块。}}


 \mytextsc{clf}~: \textcolor{darkblue}{\textbf{\ipa{kʰwɤ˥}}} 

\paragraph{\hspace{-0.5cm} {\Large \textcolor{darkblue}{\textbf{\ipa{lv̩˧pʰv̩˩}}} \textsubscript{1}}} \hypertarget{lv\string_=\string_Mp\string_hv\string_=\string_B1}{}
\markboth{\textcolor{darkblue}{\textbf{\ipa{lv̩˧pʰv̩˩}}} \textsubscript{1}}{}
\textcolor{teal}{\mytextsc{verbe}} \hspace{4pt} Ton~: .

Défricher.
\textcolor{brown}{\zh{开荒。}}


\begin{exe}
\sn \textcolor{darkblue}{\textbf{\ipa{lv̩˧pʰv̩˩-hɯ˩}}}
\trans (il/elle) est parti(e) défricher
\trans \textit{\textcolor{brown}{\zh{他开荒去了。}}}
\end{exe}
\begin{exe}
\sn \textcolor{darkblue}{\textbf{\ipa{lv̩˧pʰv̩˩-bi˩-tso˩-ɲi˩}}}
\trans Il va falloir aller défricher. / Il va falloir qu'on défriche de nouvelles terres.
\trans \textit{\textcolor{brown}{\zh{应该去开荒了。}}}
\end{exe}


\paragraph{\hspace{-0.5cm} {\Large \textcolor{darkblue}{\textbf{\ipa{lv̩˧pʰv̩˩}}} \textsubscript{2}}} \hypertarget{lv\string_=\string_Mp\string_hv\string_=\string_B2}{}
\markboth{\textcolor{darkblue}{\textbf{\ipa{lv̩˧pʰv̩˩}}} \textsubscript{2}}{}
\textcolor{teal}{\mytextsc{nom}} \hspace{4pt} Ton~: L\#.

Champs inondés.
\textcolor{brown}{\zh{水田。}}


 \mytextsc{clf}~: \textcolor{darkblue}{\textbf{\ipa{pʰv̩˩}}} 

\paragraph{\hspace{-0.5cm} {\Large \textcolor{darkblue}{\textbf{\ipa{lv̩˧qæ\#˥}}}}} \hypertarget{lv\string_=\string_Mq\{\#\string_T1}{}
\markboth{\textcolor{darkblue}{\textbf{\ipa{lv̩˧qæ\#˥}}}}{}
\textcolor{teal}{\mytextsc{nom}} \hspace{4pt} Ton~: \#˥.

Limite de propriété: limite entre les champs appartenant à des familles différentes. Elle est souvent matérialisée par une diguette.
\textcolor{brown}{\zh{地界:不同家庭田地之间的界限。}}




\paragraph{\hspace{-0.5cm} {\Large \textcolor{darkblue}{\textbf{\ipa{lv̩˧sɯ˥}}}}} \hypertarget{lv\string_=\string_MsM\string_T1}{}
\markboth{\textcolor{darkblue}{\textbf{\ipa{lv̩˧sɯ˥}}}}{}
\textcolor{teal}{\mytextsc{nom}} \hspace{4pt} Ton~: H\#.

Lisu (groupe ethnique).
\textcolor{brown}{\zh{傈僳族。}}


 \mytextsc{clf}~: \textcolor{darkblue}{\textbf{\ipa{v̩˧}}} 

\paragraph{\hspace{-0.5cm} {\Large \textcolor{darkblue}{\textbf{\ipa{lv̩˩tɕʰɯ˧}}}}} \hypertarget{lv\string_=\string_Bts£\string_hM\string_M1}{}
\markboth{\textcolor{darkblue}{\textbf{\ipa{lv̩˩tɕʰɯ˧}}}}{}
\textcolor{teal}{\mytextsc{nom}} \hspace{4pt} Ton~: LM.

Fengke: nom chinois ancien du village de Fengke, au bord du Yangtze.
\textcolor{brown}{\zh{六区,今奉科乡(汉语借词)。}}

 Emprunt~: chinois \textcolor{brown}{\zh{六区}}



\paragraph{\hspace{-0.5cm} {\Large \textcolor{darkblue}{\textbf{\ipa{lv̩˩tɕʰɯ˧-hĩ\#˥}}}}} \hypertarget{lv\string_=\string_Bts£\string_hM\string_M-hi\string_~\#\string_T1}{}
\markboth{\textcolor{darkblue}{\textbf{\ipa{lv̩˩tɕʰɯ˧-hĩ\#˥}}}}{}
\textcolor{teal}{\mytextsc{nom}} \hspace{4pt} Ton~: LM+\#H.

Gens de Fengke (Fv-kho).
\textcolor{brown}{\zh{奉科的人。}}

 Emprunt~: chinois \textcolor{brown}{\zh{六区}}



\paragraph{\hspace{-0.5cm} {\Large \textcolor{darkblue}{\textbf{\ipa{lv̩˧tsɯ˥}}}}} \hypertarget{lv\string_=\string_MtsM\string_T1}{}
\markboth{\textcolor{darkblue}{\textbf{\ipa{lv̩˧tsɯ˥}}}}{}
\textcolor{teal}{\mytextsc{nom}} \hspace{4pt} Ton~: H\#.

Four.
\textcolor{brown}{\zh{炉子(汉语借词)。}}

 Emprunt~: chinois \textcolor{brown}{\zh{炉子}}

 \mytextsc{clf}~: \textcolor{darkblue}{\textbf{\ipa{nɑ˧}}} 

\paragraph{\hspace{-0.5cm} {\Large \textcolor{darkblue}{\textbf{\ipa{lv̩˩\textasciitilde{}lv̩˧˥}}}}} \hypertarget{lv\string_=\string_B~lv\string_=\string_M\string_T1}{}
\markboth{\textcolor{darkblue}{\textbf{\ipa{lv̩˩\textasciitilde{}lv̩˧˥}}}}{}
\textcolor{teal}{\mytextsc{verbe}} \hspace{4pt} Ton~: MH.

Bouger, faire des mouvements.
\textcolor{brown}{\zh{动(虫、桌子、小孩子动)。}}


\begin{exe}
\sn \textcolor{darkblue}{\textbf{\ipa{lv̩˩\textasciitilde{}lv̩˧-ze˥}}}
\trans \mytextsc{pfv}
\trans \textit{\textcolor{brown}{\zh{动了}}}
\end{exe}
\begin{exe}
\sn \textcolor{darkblue}{\textbf{\ipa{tʰi˧-lv̩˩\textasciitilde{}lv̩˩(-ze˩)}}}
\trans \mytextsc{dur} \mytextsc{red}
\trans \textit{\textcolor{brown}{\zh{\mytextsc{dur} \mytextsc{red}}}}
\end{exe}
\begin{exe}
\sn \textcolor{darkblue}{\textbf{\ipa{tʰi˧-lv̩˩\textasciitilde{}lv̩˩ | se˧}}}
\trans marcher en se trémoussant, marcher de travers, marcher en se contorsionnant
\trans \textit{\textcolor{brown}{\zh{歪着走、扭着走、例如:残疾人走路有困难}}}
\end{exe}
\begin{exe}
\sn \textcolor{darkblue}{\textbf{\ipa{kʰɯ˧tsʰɤ˧ lv̩˥\textasciitilde{}lv̩˩}}}
\trans bouger la jambe, remuer la jambe
\trans \textit{\textcolor{brown}{\zh{活动一下(自己的)腿}}}
\end{exe}


\paragraph{\hspace{-0.5cm} {\Large \textcolor{darkblue}{\textbf{\ipa{lv˧bv˧}}}}} \hypertarget{lv\string_Mbv\string_M1}{}
\markboth{\textcolor{darkblue}{\textbf{\ipa{lv˧bv˧}}}}{}
\textcolor{teal}{\mytextsc{nom}} \hspace{4pt} Ton~: M.

Lit à légumes (dans le potager).
\textcolor{brown}{\zh{菜畦。}}


\begin{exe}
\sn \textcolor{darkblue}{\textbf{\ipa{v˩tsʰɤ˧-lv˧bv̩\#˥}}}
\trans même sens: lit à légumes (dans le potager)
\trans \textit{\textcolor{brown}{\zh{同上:菜畦}}}
\end{exe}
\begin{exe}
\sn \textcolor{darkblue}{\textbf{\ipa{qʰwæ˧ɭɯ˧-qo˧ | v˩tsʰɤ˧-lv˧bv˧ | le˧-gv˩, v˩tsʰɤ˧˥ | ɖɯ˧-jɤ˩ tʰi˩-pʰo˩}}}
\trans bâtir un lit à légumes dans le potager, et semer une rangée de légumes
\trans \textit{\textcolor{brown}{\zh{菜园里建菜畦,种一排菜}}}
\end{exe}
 \mytextsc{clf}~: \textcolor{darkblue}{\textbf{\ipa{kɤ˧˥}}} 

\paragraph{\hspace{-0.5cm} {\Large \textcolor{darkblue}{\textbf{\ipa{lwæ˩pʰv̩˩}}}}} \hypertarget{lw\{\string_Bp\string_hv\string_=\string_B1}{}
\markboth{\textcolor{darkblue}{\textbf{\ipa{lwæ˩pʰv̩˩}}}}{}
\textcolor{teal}{\mytextsc{nom}} \hspace{4pt} Ton~: L.

Cendres.
\textcolor{brown}{\zh{灰。}}


\begin{exe}
\sn \textcolor{darkblue}{\textbf{\ipa{[F5] lwæ˩pʰv̩˩-ni˥gv̩˩}}}
\trans de couleur grise; littéralement ``comme de la cendre"
\trans \textit{\textcolor{brown}{\zh{灰色的(直译:“像白灰”)}}}
\end{exe}


\paragraph{\hspace{-0.5cm} {\Large \textcolor{darkblue}{\textbf{\ipa{lwɤ˩˥}}}}} \hypertarget{lw7\string_B\string_T1}{}
\markboth{\textcolor{darkblue}{\textbf{\ipa{lwɤ˩˥}}}}{}
\textcolor{teal}{\mytextsc{nom}} \hspace{4pt} Ton~: LH.

Cendre (cendre végétale ou cendre de charbon); scories.
\textcolor{brown}{\zh{灰,灰烬(包括草木灰等等)。}}


\begin{exe}
\sn \textcolor{darkblue}{\textbf{\ipa{lwɤ˩-pʰæ˧di˩}}}
\trans comme de la cendre
\trans \textit{\textcolor{brown}{\zh{像灰烬,灰色}}}
\end{exe}
 \mytextsc{clf}~: \textcolor{darkblue}{\textbf{\ipa{ʈʂwɤ˧}}} 

\paragraph{\hspace{-0.5cm} {\Large \textcolor{darkblue}{\textbf{\ipa{ɭɯ˧\textsubscript{b}}}}}} \hypertarget{l\string_RM\string_Mb1}{}
\markboth{\textcolor{darkblue}{\textbf{\ipa{ɭɯ˧\textsubscript{b}}}}}{}
\textcolor{teal}{\mytextsc{classificateur}} \hspace{4pt} Ton~: M\textsubscript{b}.

Classificateur générique; à l'origine, classificateur pour les objets ronds, à l'emploi maintenant élargi.
\textcolor{brown}{\zh{最常用的量词,相当于汉语中‘个’的用法。本意是圆形颗粒。一粒(米……),一个(碗……),件(衣服……)。}}


\begin{exe}
\sn \textcolor{darkblue}{\textbf{\ipa{ɕi˧ ɖɯ˧-ɭɯ˧ |}}}
\trans un grain de riz
\trans \textit{\textcolor{brown}{\zh{一粒米}}}
\end{exe}
\begin{exe}
\sn \textcolor{darkblue}{\textbf{\ipa{hõ˧-ɭɯ˥}}}
\trans huit grains
\trans \textit{\textcolor{brown}{\zh{八粒}}}
\end{exe}
\begin{exe}
\sn \textcolor{darkblue}{\textbf{\ipa{ɖɯ˧-ɭɯ˧ hwæ˧-mɤ˧-ɖo˧! | le˧-qʰwæ˧-kv̩˥!}}}
\trans N'en achète pas un (unique)! Ca va se casser! (Explication: il faut acheter les objets par paires: 2, 4, 6, 8, 10…, pas en nombre impair, sinon cela porte malheur et les objets cassent, se perdent…)
\trans \textit{\textcolor{brown}{\zh{不要(只)买一个!会碎的!(东西要一对一对买:2、4、6、8、10……,单数不吉利,东西会碎的。)}}}
\end{exe}
\begin{exe}
\sn \textcolor{darkblue}{\textbf{\ipa{ʈʂʰɯ˧ | zo˧hṽ˥ | dʑɤ˩-ɭɯ˥ dʑo˩!}}}
\trans Elle a un bien bel enfant! (Contexte: lors d'une sortie, la consultante principale dit quelques politesses à une voisine qui se promenait avec un petit-fils attendrissant; elle me dit ensuite: ``Elle a un bien bel enfant!" Littéralement: ``elle, (d')enfant(s), (elle) en a un (de) bien!")
\trans \textit{\textcolor{brown}{\zh{她有个很漂亮的孩子!}}}
\end{exe}


\paragraph{\hspace{-0.5cm} {\Large \textcolor{darkblue}{\textbf{\ipa{ɭɯ˧˥\textsubscript{a}}}}}} \hypertarget{l\string_RM\string_M\string_Ta1}{}
\markboth{\textcolor{darkblue}{\textbf{\ipa{ɭɯ˧˥\textsubscript{a}}}}}{}
\textcolor{teal}{\mytextsc{classificateur}} \hspace{4pt} Ton~: MH\textsubscript{a}.

Classificateur des vêtements.
\textcolor{brown}{\zh{量词:衣服(一件)。}}


\begin{exe}
\sn \textcolor{darkblue}{\textbf{\ipa{ʈʰæ˧qʰwɤ˧ ɖɯ˧-ɭɯ˧˥}}}
\trans une jupe
\trans \textit{\textcolor{brown}{\zh{一件裙子}}}
\end{exe}
\begin{exe}
\sn \textcolor{darkblue}{\textbf{\ipa{bɑ˩lɑ˩˥ | ɖɯ˧-ɭɯ˧˥ |}}}
\trans un vêtement
\trans \textit{\textcolor{brown}{\zh{一件衣服}}}
\end{exe}
\begin{exe}
\sn \textcolor{darkblue}{\textbf{\ipa{*dʑi˧hṽ˥\$+ɖɯ˧-ɭɯ˧˥}}}
\trans Ce classificateur ne se combine pas avec /dʑi˧hṽ˥\$/, qui prend pour classificateur: /ɖɯ˧-dzi˩/.
\trans \textit{\textcolor{brown}{\zh{(这个量词不能与/dʑi˧hṽ˥\$/结合。)}}}
\end{exe}


\newpage
\section*{\centering- \textcolor{darkblue}{\textbf{\ipa{ɬ}}} -}
\paragraph{\hspace{-0.5cm} {\Large \textcolor{darkblue}{\textbf{\ipa{ɬɑ˧mv̩˥\$}}}}} \hypertarget{KA\string_Mmv\string_=\string_T\$1}{}
\markboth{\textcolor{darkblue}{\textbf{\ipa{ɬɑ˧mv̩˥\$}}}}{}
\textcolor{teal}{\mytextsc{nom}} \hspace{4pt} Ton~: H\$.

Prénom féminin.
\textcolor{brown}{\zh{女性名字。}}




\paragraph{\hspace{-0.5cm} {\Large \textcolor{darkblue}{\textbf{\ipa{ɬɑ˧pɤ˩}}}}} \hypertarget{KA\string_Mp7\string_B1}{}
\markboth{\textcolor{darkblue}{\textbf{\ipa{ɬɑ˧pɤ˩}}}}{}
\textcolor{teal}{\mytextsc{adverbe}} \hspace{4pt} Ton~: L\#.

Beaucoup.
\textcolor{brown}{\zh{多、使劲。}}


\begin{exe}
\sn \textcolor{darkblue}{\textbf{\ipa{ɬɑ˧pɤ˩ ʝi˩}}}
\trans en faire beaucoup
\trans \textit{\textcolor{brown}{\zh{使劲工作、使劲干}}}
\end{exe}
\begin{exe}
\sn \textcolor{darkblue}{\textbf{\ipa{ɬɑ˧pɤ˩ | ɖɯ˧-kʰwɤ˧ ʝi˧}}}
\trans en mettre un coup, beaucoup travailler, bien avancer dans son travail
\trans \textit{\textcolor{brown}{\zh{使劲工作一下}}}
\end{exe}
\begin{exe}
\sn \textcolor{darkblue}{\textbf{\ipa{ɬɑ˧pɤ˩ | ɖɯ˧-kʰwɤ˧ so˥}}}
\trans beaucoup étudier, faire un bon progrès dans l'étude
\trans \textit{\textcolor{brown}{\zh{努力学习一下}}}
\end{exe}


\paragraph{\hspace{-0.5cm} {\Large \textcolor{darkblue}{\textbf{\ipa{ɬɑ˧sɑ˧}}}}} \hypertarget{KA\string_MsA\string_M1}{}
\markboth{\textcolor{darkblue}{\textbf{\ipa{ɬɑ˧sɑ˧}}}}{}
\textcolor{teal}{\mytextsc{nom}} \hspace{4pt} Ton~: M.

Lhasa (capitale du Tibet).
\textcolor{brown}{\zh{拉萨。}}




\paragraph{\hspace{-0.5cm} {\Large \textcolor{darkblue}{\textbf{\ipa{ɬɑ˧tɑ˥}}}}} \hypertarget{KA\string_MtA\string_T1}{}
\markboth{\textcolor{darkblue}{\textbf{\ipa{ɬɑ˧tɑ˥}}}}{}
\textcolor{teal}{\mytextsc{nom}} \hspace{4pt} Ton~: H\#.
\textit{Archaïque} [\textcolor{brown}{\zh{古语}}] 
Gilet de cuir (mot sorti d'usage, n'apparaît que dans un proverbe).
\textcolor{brown}{\zh{皮革背心。}}


 \mytextsc{clf}~: \textcolor{darkblue}{\textbf{\ipa{ɭɯ˧}}} 

\paragraph{\hspace{-0.5cm} {\Large \textcolor{darkblue}{\textbf{\ipa{ɬɑ˧tsʰo\#˥}}}}} \hypertarget{KA\string_Mts\string_ho\#\string_T1}{}
\markboth{\textcolor{darkblue}{\textbf{\ipa{ɬɑ˧tsʰo\#˥}}}}{}
\textcolor{teal}{\mytextsc{nom}} \hspace{4pt} Ton~: \#H.

Prénom féminin.
\textcolor{brown}{\zh{女性名字。}}




\paragraph{\hspace{-0.5cm} {\Large \textcolor{darkblue}{\textbf{\ipa{ɬɑ˧˥}}}}} \hypertarget{KA\string_M\string_T1}{}
\markboth{\textcolor{darkblue}{\textbf{\ipa{ɬɑ˧˥}}}}{}
\textcolor{teal}{\mytextsc{adjectif}} \hspace{4pt} Ton~: MH.

Abondant, nombreux.
\textcolor{brown}{\zh{多、丰富、充分。}}


\begin{exe}
\sn \textcolor{darkblue}{\textbf{\ipa{dʑɤ˩-hĩ˩˥, | le˧-ɳɯ˥! | mɤ˧-dʑɤ˩-hĩ˩, | le˧-ɬɑ˧˥!}}}
\trans Les bons, il n'y en a guère; les médiocres, il y en a en quantité! (Contexte: au sujet des établissements universitaires entre lesquels les titulaires du baccalauréat chinois ont à choisir)
\trans \textit{\textcolor{brown}{\zh{好的,不多!不好的,就很多了!(情景:谈高中学生想入大学)}}}
\end{exe}


\paragraph{\hspace{-0.5cm} {\Large \textcolor{darkblue}{\textbf{\ipa{ɬi˥}}}}} \hypertarget{Ki\string_T1}{}
\markboth{\textcolor{darkblue}{\textbf{\ipa{ɬi˥}}}}{}
\textcolor{teal}{\mytextsc{verbe}} \hspace{4pt} Ton~: H.

Se reposer, se détendre.
\textcolor{brown}{\zh{休息,松懈。}}


\begin{exe}
\sn \textcolor{darkblue}{\textbf{\ipa{le˧-ɬi˥}}}
\trans \mytextsc{accomp} \string_
\trans \textit{\textcolor{brown}{\zh{\mytextsc{accomp} \string_}}}
\end{exe}


\paragraph{\hspace{-0.5cm} {\Large \textcolor{darkblue}{\textbf{\ipa{ɬi˧\textsubscript{b}}}}}} \hypertarget{Ki\string_Mb1}{}
\markboth{\textcolor{darkblue}{\textbf{\ipa{ɬi˧\textsubscript{b}}}}}{}
\textcolor{teal}{\mytextsc{classificateur}} \hspace{4pt} Ton~: M\textsubscript{b}.

Mois.
\textcolor{brown}{\zh{量词:月。}}




\paragraph{\hspace{-0.5cm} {\Large \textcolor{darkblue}{\textbf{\ipa{ɬi˧bo\#˥}}}}} \hypertarget{Ki\string_Mbo\#\string_T1}{}
\markboth{\textcolor{darkblue}{\textbf{\ipa{ɬi˧bo\#˥}}}}{}
\textcolor{teal}{\mytextsc{nom}} \hspace{4pt} Ton~: \#H.

Sourd, personne sourde.
\textcolor{brown}{\zh{聋子。}}


\begin{exe}
\sn \textcolor{darkblue}{\textbf{\ipa{ɬi˧bo˧-hĩ˧}}}
\trans personne sourde
\trans \textit{\textcolor{brown}{\zh{耳朵聋的人}}}
\end{exe}
 \mytextsc{clf}~: \textcolor{darkblue}{\textbf{\ipa{v̩˧}}} 

\paragraph{\hspace{-0.5cm} {\Large \textcolor{darkblue}{\textbf{\ipa{ɬi˧bv̩˧}}}}} \hypertarget{Ki\string_Mbv\string_=\string_M1}{}
\markboth{\textcolor{darkblue}{\textbf{\ipa{ɬi˧bv̩˧}}}}{}
\textcolor{teal}{\mytextsc{nom}} \hspace{4pt} Ton~: M.

Bai (groupe ethnique).
\textcolor{brown}{\zh{白族。}}


 \mytextsc{clf}~: \textcolor{darkblue}{\textbf{\ipa{v̩˧}}} 

\paragraph{\hspace{-0.5cm} {\Large \textcolor{darkblue}{\textbf{\ipa{}}}}} \hypertarget{Ki\string_Mbv\string_=\string_B | dz£7\string_Bts\string_hi\string_M-si\#\string_T1}{}
\markboth{\textcolor{darkblue}{\textbf{\ipa{}}}}{}
\textcolor{teal}{\mytextsc{nom}} \hspace{4pt} Ton~: L\# | LM+\#H.

Ailante, \textit{Ailanthus chinensis}; arbre très odorant.
\textcolor{brown}{\zh{香椿、香椿树。}}




\paragraph{\hspace{-0.5cm} {\Large \textcolor{darkblue}{\textbf{\ipa{ɬi˧bv̩˧-mi\#˥}}}}} \hypertarget{Ki\string_Mbv\string_=\string_M-mi\#\string_T1}{}
\markboth{\textcolor{darkblue}{\textbf{\ipa{ɬi˧bv̩˧-mi\#˥}}}}{}
\textcolor{teal}{\mytextsc{nom}} \hspace{4pt} Ton~: \#H.

Femme bai.
\textcolor{brown}{\zh{白族女人。}}


 \mytextsc{clf}~: \textcolor{darkblue}{\textbf{\ipa{v̩˧}}} 

\paragraph{\hspace{-0.5cm} {\Large \textcolor{darkblue}{\textbf{\ipa{ɬi˧bv̩˧-zo\#˥}}}}} \hypertarget{Ki\string_Mbv\string_=\string_M-zo\#\string_T1}{}
\markboth{\textcolor{darkblue}{\textbf{\ipa{ɬi˧bv̩˧-zo\#˥}}}}{}
\textcolor{teal}{\mytextsc{nom}} \hspace{4pt} Ton~: \#H.

Homme bai.
Dialecte chinois local~: \zh{白族男人。}


 \mytextsc{clf}~: \textcolor{darkblue}{\textbf{\ipa{v̩˧}}} 

\paragraph{\hspace{-0.5cm} {\Large \textcolor{darkblue}{\textbf{\ipa{ɬi˧di˩}}}}} \hypertarget{Ki\string_Mdi\string_B1}{}
\markboth{\textcolor{darkblue}{\textbf{\ipa{ɬi˧di˩}}}}{}
\textcolor{teal}{\mytextsc{nom}} \hspace{4pt} Ton~: L\#.

Yongning (nom de lieu).
\textcolor{brown}{\zh{永宁(地名)。}}




\paragraph{\hspace{-0.5cm} {\Large \textcolor{darkblue}{\textbf{\ipa{ɬi˧di˩-hĩ˩}}}}} \hypertarget{Ki\string_Mdi\string_B-hi\string_~\string_B1}{}
\markboth{\textcolor{darkblue}{\textbf{\ipa{ɬi˧di˩-hĩ˩}}}}{}
\textcolor{teal}{\mytextsc{nom}} \hspace{4pt} Ton~: L\#-.

Les gens de Yongning.
\textcolor{brown}{\zh{永宁人(纳人)。}}


 \mytextsc{clf}~: \textcolor{darkblue}{\textbf{\ipa{v̩˧}}} 

\paragraph{\hspace{-0.5cm} {\Large \textcolor{darkblue}{\textbf{\ipa{ɬi˧dʑɯ˩}}}}} \hypertarget{Ki\string_Mdz£M\string_B1}{}
\markboth{\textcolor{darkblue}{\textbf{\ipa{ɬi˧dʑɯ˩}}}}{}
\textcolor{teal}{\mytextsc{nom}} \hspace{4pt} Ton~: L\#.

La rivière qui traverse la plaine de Yongning.
\textcolor{brown}{\zh{永宁坝子的河流。}}


 \mytextsc{clf}~: \textcolor{darkblue}{\textbf{\ipa{kʰɯ˩}}} 

\paragraph{\hspace{-0.5cm} {\Large \textcolor{darkblue}{\textbf{\ipa{ɬi˧gv̩\#˥}}}}} \hypertarget{Ki\string_Mgv\string_=\#\string_T1}{}
\markboth{\textcolor{darkblue}{\textbf{\ipa{ɬi˧gv̩\#˥}}}}{}
\textcolor{teal}{\mytextsc{nom}} \hspace{4pt} Ton~: \#H.

Partie intermédiaire, milieu; au milieu.
\textcolor{brown}{\zh{中部,中间。}}


\begin{exe}
\sn \textcolor{darkblue}{\textbf{\ipa{ɬi˧gv̩˧ dzi˥}}}
\trans être assis au milieu
\trans \textit{\textcolor{brown}{\zh{坐在中间}}}
\end{exe}


\paragraph{\hspace{-0.5cm} {\Large \textcolor{darkblue}{\textbf{\ipa{ɬi˧hĩ\#˥}}} \textsubscript{1}}} \hypertarget{Ki\string_Mhi\string_~\#\string_T1}{}
\markboth{\textcolor{darkblue}{\textbf{\ipa{ɬi˧hĩ\#˥}}} \textsubscript{1}}{}
\textcolor{teal}{\mytextsc{nom}} \hspace{4pt} Ton~: \#H.

Homme en position intermédiaire dans la fratrie: ni aîné ni cadet; traduction littérale: ``personne du milieu".
\textcolor{brown}{\zh{兄弟里面夹中的男孩(上有哥哥下有弟弟的孩子)。}}




\paragraph{\hspace{-0.5cm} {\Large \textcolor{darkblue}{\textbf{\ipa{ɬi˧hĩ\#˥}}} \textsubscript{2}}} \hypertarget{Ki\string_Mhi\string_~\#\string_T2}{}
\markboth{\textcolor{darkblue}{\textbf{\ipa{ɬi˧hĩ\#˥}}} \textsubscript{2}}{}
\textcolor{teal}{\mytextsc{nom}} \hspace{4pt} Ton~: \#H.

Habitant de Yongning. Peut désigner les Prinmi qui habitent dans la plaine, aussi bien que les Na.
\textcolor{brown}{\zh{永宁的人。}}




\paragraph{\hspace{-0.5cm} {\Large \textcolor{darkblue}{\textbf{\ipa{ɬi˧ki\#˥}}}}} \hypertarget{Ki\string_Mki\#\string_T1}{}
\markboth{\textcolor{darkblue}{\textbf{\ipa{ɬi˧ki\#˥}}}}{}
\textcolor{teal}{\mytextsc{nom}} \hspace{4pt} Ton~: \#H.

Village na, hors de la plaine, proche du Lac.
\textcolor{brown}{\zh{泸沽湖附近的一个村落。}}


\begin{exe}
\sn \textcolor{darkblue}{\textbf{\ipa{ɬi˧ki˧, | ɲi˧se˩, | tɑ˧dzi˩, | mv̩˧qʰwæ˩, | lɑ˧tʰɑ˧-di˧˥}}}
\trans Villages dans l'ordre, après la plaine de Yongning, ne comptant pas comme faisant partie de Yongning. Le dernier, \textcolor{darkblue}{\textbf{\ipa{/lɑ˧tʰɑ˧-di˧˥/}}}, désigne toute la région na au-delà du quatrième village.
\trans \textit{\textcolor{brown}{\zh{永宁到泸沽湖所经过的村落,依次是:里格、尼赛、大祖、木垮,然后到拉塔地(拉塔地指的是泸沽湖周边的摩梭地区,包括左所、洛水村等)}}}
\end{exe}


\paragraph{\hspace{-0.5cm} {\Large \textcolor{darkblue}{\textbf{\ipa{ɬi˧ki˥}}}}} \hypertarget{Ki\string_Mki\string_T1}{}
\markboth{\textcolor{darkblue}{\textbf{\ipa{ɬi˧ki˥}}}}{}
\textcolor{teal}{\mytextsc{nom}} \hspace{4pt} Ton~: H\#.

Cérémonie pour les garçons atteignant 13 ans: littéralement ``porter/enfiler/mettre le pantalon". Après cette cérémonie, l'adolescent porte un pantalon, au lieu du vêtement unisexe des enfants. (Rituel parallèle avec fv{/ʈʰæ˩ ki˩˥/}, ``porter/enfiler/mettre la jupe", pour les jeunes filles.).
\textcolor{brown}{\zh{男性成年礼:直译“穿裤”。}}




\paragraph{\hspace{-0.5cm} {\Large \textcolor{darkblue}{\textbf{\ipa{ɬi˧mi˧}}} \textsubscript{1}}} \hypertarget{Ki\string_Mmi\string_M1}{}
\markboth{\textcolor{darkblue}{\textbf{\ipa{ɬi˧mi˧}}} \textsubscript{1}}{}
\textcolor{teal}{\mytextsc{nom}} \hspace{4pt} Ton~: M.
\ding{202} 
Lune (disyllabe).
\textcolor{brown}{\zh{月亮(双音节)。}}


\ding{203} 
Mois.
\textcolor{brown}{\zh{月(双音节)。}}


\begin{exe}
\sn \textcolor{darkblue}{\textbf{\ipa{ɬi˧mi˧ ɖɯ˧-gi˥}}}
\trans une quinzaine, la moitié d'un mois
\trans \textit{\textcolor{brown}{\zh{半个月}}}
\end{exe}
\begin{exe}
\sn \textcolor{darkblue}{\textbf{\ipa{ɬi˧mi˧ le˧-gv̩˩}}}
\trans le mois décroît; expression qui peut désigner la seconde période du mois
\trans \textit{\textcolor{brown}{\zh{下半月份}}}
\end{exe}
 \mytextsc{clf}~: \textcolor{darkblue}{\textbf{\ipa{ɭɯ˧}}} 

\paragraph{\hspace{-0.5cm} {\Large \textcolor{darkblue}{\textbf{\ipa{ɬi˧mi˧}}} \textsubscript{2}}} \hypertarget{Ki\string_Mmi\string_M2}{}
\markboth{\textcolor{darkblue}{\textbf{\ipa{ɬi˧mi˧}}} \textsubscript{2}}{}
\textcolor{teal}{\mytextsc{nom}} \hspace{4pt} Ton~: M.

Chevrotain femelle.
\textcolor{brown}{\zh{母獐子。}}


 \mytextsc{clf}~: \textcolor{darkblue}{\textbf{\ipa{v̩˧}}} 

\paragraph{\hspace{-0.5cm} {\Large \textcolor{darkblue}{\textbf{\ipa{ɬi˧mi˧dɑ˧dzɯ\#˥}}}}} \hypertarget{Ki\string_Mmi\string_MdA\string_MdzM\#\string_T1}{}
\markboth{\textcolor{darkblue}{\textbf{\ipa{ɬi˧mi˧dɑ˧dzɯ\#˥}}}}{}
\textcolor{teal}{\mytextsc{nom}} \hspace{4pt} Ton~: \#H.

Éclipse de lune.
\textcolor{brown}{\zh{月蚀。}}


\begin{exe}
\sn \textcolor{darkblue}{\textbf{\ipa{ɬi˧mi˧dɑ˧dzɯ˧ tʰv̩˧}}}
\trans il y a une éclipse de lune
\trans \textit{\textcolor{brown}{\zh{有月蚀}}}
\end{exe}
\begin{exe}
\sn \textcolor{darkblue}{\textbf{\ipa{ʈʂʰɯ˧ | ɬi˧mi˧dɑ˧dzɯ˧ ɲi˥!}}}
\trans c'est une éclipse de lune! (réponse à la question 'Qu'est-ce qui se passe?')
\trans \textit{\textcolor{brown}{\zh{这是月蚀!(一个人问:‘这是怎么回事?’,另一个回答:‘这是月蚀!’)}}}
\end{exe}
 \mytextsc{clf}~: \textcolor{darkblue}{\textbf{\ipa{ʂɯ˩}}} fois

\paragraph{\hspace{-0.5cm} {\Large \textcolor{darkblue}{\textbf{\ipa{ɬi˧ɳæ˩}}}}} \hypertarget{Ki\string_Mn`\{\string_B1}{}
\markboth{\textcolor{darkblue}{\textbf{\ipa{ɬi˧ɳæ˩}}}}{}
\textcolor{teal}{\mytextsc{nom}} \hspace{4pt} Ton~: L\#.

Menstrues.
\textcolor{brown}{\zh{月经。}}


\begin{exe}
\sn \textcolor{darkblue}{\textbf{\ipa{ʈʂʰɯ˧ | ɬi˧ɳæ˩-ze˩}}}
\trans Elle est en train d'avoir ses règles.
\trans \textit{\textcolor{brown}{\zh{她来了月经。}}}
\end{exe}
\begin{exe}
\sn \textcolor{darkblue}{\textbf{\ipa{ɬi˧ɳæ˩ go˩}}}
\trans avoir des menstrues douloureuses
\trans \textit{\textcolor{brown}{\zh{来了月经,疼}}}
\end{exe}
 \mytextsc{clf}~: \textcolor{darkblue}{\textbf{\ipa{ɬi˧}}} mois

\paragraph{\hspace{-0.5cm} {\Large \textcolor{darkblue}{\textbf{\ipa{ɬi˧pæ˥}}}}} \hypertarget{Ki\string_Mp\{\string_T1}{}
\markboth{\textcolor{darkblue}{\textbf{\ipa{ɬi˧pæ˥}}}}{}
\textcolor{teal}{\mytextsc{nom}} \hspace{4pt} Ton~: H\#.

Boucle d'oreille.
\textcolor{brown}{\zh{耳环。}}


\begin{exe}
\sn \textcolor{darkblue}{\textbf{\ipa{ŋv̩˩-ɬi˩pæ˥ (+ɲi˩)}}}
\trans boucle d'oreille en argent
\trans \textit{\textcolor{brown}{\zh{银耳环}}}
\end{exe}
\begin{exe}
\sn \textcolor{darkblue}{\textbf{\ipa{hæ̃˩-ɬi˩pæ˥ (+ɲi˩)}}}
\trans boucle d'oreille en or
\trans \textit{\textcolor{brown}{\zh{金耳环}}}
\end{exe}
 \mytextsc{clf}~: \textcolor{darkblue}{\textbf{\ipa{dze˩}}} paire

\paragraph{\hspace{-0.5cm} {\Large \textcolor{darkblue}{\textbf{\ipa{ɬi˧pi˩}}}}} \hypertarget{Ki\string_Mpi\string_B1}{}
\markboth{\textcolor{darkblue}{\textbf{\ipa{ɬi˧pi˩}}}}{}
\textcolor{teal}{\mytextsc{nom}} \hspace{4pt} Ton~: L\#.

Oreille.
\textcolor{brown}{\zh{耳朵。}}


 \mytextsc{clf}~: \textcolor{darkblue}{\textbf{\ipa{pʰo˧˥}}} 

\paragraph{\hspace{-0.5cm} {\Large \textcolor{darkblue}{\textbf{\ipa{ɬi˧pv̩˧lv̩˥}}}}} \hypertarget{Ki\string_Mpv\string_=\string_Mlv\string_=\string_T1}{}
\markboth{\textcolor{darkblue}{\textbf{\ipa{ɬi˧pv̩˧lv̩˥}}}}{}
\textcolor{teal}{\mytextsc{nom}} \hspace{4pt} Ton~: H\#.

Tumeur de l'oreille, excroissance pathologique de l'oreille.
\textcolor{brown}{\zh{耳朵瘤。}}


 \mytextsc{clf}~: \textcolor{darkblue}{\textbf{\ipa{ɭɯ˧}}} 

\paragraph{\hspace{-0.5cm} {\Large \textcolor{darkblue}{\textbf{\ipa{ɬi˧pʰv̩\#˥}}}}} \hypertarget{Ki\string_Mp\string_hv\string_=\#\string_T1}{}
\markboth{\textcolor{darkblue}{\textbf{\ipa{ɬi˧pʰv̩\#˥}}}}{}
\textcolor{teal}{\mytextsc{nom}} \hspace{4pt} Ton~: \#H.

Chevrotain mâle.
\textcolor{brown}{\zh{公獐子。}}


\begin{exe}
\sn \textcolor{darkblue}{\textbf{\ipa{ɬi˧pʰv̩˧ tʰv̩˧-mi˥\# / ɬi˧pʰv̩˧ tʰv̩˧-mi˧˥}}}
\trans \mytextsc{n}+\mytextsc{dem}+\mytextsc{clf}
\trans \textit{\textcolor{brown}{\zh{那只公獐子}}}
\end{exe}
 \mytextsc{clf}~: \textcolor{darkblue}{\textbf{\ipa{v̩˧}}} \textcolor{darkblue}{\textbf{\ipa{ɭɯ˧}}} \textcolor{darkblue}{\textbf{\ipa{mi˩}}} 

\paragraph{\hspace{-0.5cm} {\Large \textcolor{darkblue}{\textbf{\ipa{ɬi˧qʰæ\#˥}}}}} \hypertarget{Ki\string_Mq\string_h\{\#\string_T1}{}
\markboth{\textcolor{darkblue}{\textbf{\ipa{ɬi˧qʰæ\#˥}}}}{}
\textcolor{teal}{\mytextsc{nom}} \hspace{4pt} Ton~: \#H.

Cérumen.
\textcolor{brown}{\zh{耳垢。}}


 \mytextsc{clf}~: \textcolor{darkblue}{\textbf{\ipa{kʰwɤ˥}}} 

\paragraph{\hspace{-0.5cm} {\Large \textcolor{darkblue}{\textbf{\ipa{ɬi˧qʰv̩\#˥}}}}} \hypertarget{Ki\string_Mq\string_hv\string_=\#\string_T1}{}
\markboth{\textcolor{darkblue}{\textbf{\ipa{ɬi˧qʰv̩\#˥}}}}{}
\textcolor{teal}{\mytextsc{nom}} \hspace{4pt} Ton~: \#H.

Conduit auditif.
\textcolor{brown}{\zh{耳孔。}}


\begin{exe}
\sn \textcolor{darkblue}{\textbf{\ipa{ʈʂʰɯ˧ | ɬi˧qʰv̩˧ | ɖɯ˧-pi˧˥ | tʰɑ˩˥!}}}
\trans Elle a l'oreille fine! (Contexte: au sujet d'une petite fille de 2 ans qui se réveille aussitôt de sa sieste lorsqu'elle entend l'arrivée de visiteurs.)
\trans \textit{\textcolor{brown}{\zh{她耳朵很好使! / 她耳朵很尖!(情景:一有客人到家的声音,睡午觉的两岁女孩子立即醒来。)}}}
\end{exe}
 \mytextsc{clf}~: \textcolor{darkblue}{\textbf{\ipa{ɭɯ˧}}} 

\paragraph{\hspace{-0.5cm} {\Large \textcolor{darkblue}{\textbf{\ipa{ɬi˧ʈv̩˥}}}}} \hypertarget{Ki\string_Mt`v\string_=\string_T1}{}
\markboth{\textcolor{darkblue}{\textbf{\ipa{ɬi˧ʈv̩˥}}}}{}
\textcolor{teal}{\mytextsc{nom}} \hspace{4pt} Ton~: H\#.

Plantain (utilisé par les Na pour ses vertus médicinales; est abondant à Yongning).
\textcolor{brown}{\zh{车前草。}}


 \mytextsc{clf}~: \textcolor{darkblue}{\textbf{\ipa{po˧}}} 

\paragraph{\hspace{-0.5cm} {\Large \textcolor{darkblue}{\textbf{\ipa{ɬi˩}}} \textsubscript{1}}} \hypertarget{Ki\string_B1}{}
\markboth{\textcolor{darkblue}{\textbf{\ipa{ɬi˩}}} \textsubscript{1}}{}
\textcolor{teal}{\mytextsc{verbe}} \hspace{4pt} Ton~: L\textsubscript{a}.

Toiser: mesurer (une pièce de tissu…) à l'aune de la toise: distance entre les deux bras écartés.
\textcolor{brown}{\zh{量(一块布料……)有多长:有多少庹。}}


\begin{exe}
\sn \textcolor{darkblue}{\textbf{\ipa{ɬi˩-se˥ (-ze˩)}}}
\trans \string_ \mytextsc{achev} (\mytextsc{pfv})
\trans \textit{\textcolor{brown}{\zh{量完(了)}}}
\end{exe}


\paragraph{\hspace{-0.5cm} {\Large \textcolor{darkblue}{\textbf{\ipa{ɬi˩}}} \textsubscript{2}}} \hypertarget{Ki\string_B2}{}
\markboth{\textcolor{darkblue}{\textbf{\ipa{ɬi˩}}} \textsubscript{2}}{}
\textcolor{teal}{\mytextsc{nom}} \hspace{4pt} Ton~: L.

Chevrotain.
\textcolor{brown}{\zh{獐子。}}


 \mytextsc{clf}~: \textcolor{darkblue}{\textbf{\ipa{pʰo˧˥}}} \textcolor{darkblue}{\textbf{\ipa{mi˩}}} 

\paragraph{\hspace{-0.5cm} {\Large \textcolor{darkblue}{\textbf{\ipa{ɬi˩\textsubscript{b}}}}}} \hypertarget{Ki\string_Bb1}{}
\markboth{\textcolor{darkblue}{\textbf{\ipa{ɬi˩\textsubscript{b}}}}}{}
\textcolor{teal}{\mytextsc{classificateur}} \hspace{4pt} Ton~: L\textsubscript{b}.

Toise: envergure des bras =longueur des deux bras écartés. Cette unité correspond à environ 5 pieds chinois (1 mètre 78).
\textcolor{brown}{\zh{量词:庹。}}


\begin{exe}
\sn \textcolor{darkblue}{\textbf{\ipa{tsʰe˧-ɬi˧}}}
\trans 10 toises
\trans \textit{\textcolor{brown}{\zh{十庹}}}
\end{exe}


\paragraph{\hspace{-0.5cm} {\Large \textcolor{darkblue}{\textbf{\ipa{ɬi˩bi˩}}}}} \hypertarget{Ki\string_Bbi\string_B1}{}
\markboth{\textcolor{darkblue}{\textbf{\ipa{ɬi˩bi˩}}}}{}
\textcolor{teal}{\mytextsc{nom}} \hspace{4pt} Ton~: L.

Navet, gros radis.
\textcolor{brown}{\zh{萝卜。}}


 \mytextsc{clf}~: \textcolor{darkblue}{\textbf{\ipa{ɭɯ˧}}} 

\paragraph{\hspace{-0.5cm} {\Large \textcolor{darkblue}{\textbf{\ipa{ɬi˩qʰwɤ˩}}}}} \hypertarget{Ki\string_Bq\string_hw7\string_B1}{}
\markboth{\textcolor{darkblue}{\textbf{\ipa{ɬi˩qʰwɤ˩}}}}{}
\textcolor{teal}{\mytextsc{nom}} \hspace{4pt} Ton~: L.

Pantalon.
\textcolor{brown}{\zh{裤子。}}


 \mytextsc{clf}~: \textcolor{darkblue}{\textbf{\ipa{ɭɯ˧}}} 

\paragraph{\hspace{-0.5cm} {\Large \textcolor{darkblue}{\textbf{\ipa{ɬi˩ʁɑ˩}}}}} \hypertarget{Ki\string_BRA\string_B1}{}
\markboth{\textcolor{darkblue}{\textbf{\ipa{ɬi˩ʁɑ˩}}}}{}
\textcolor{teal}{\mytextsc{adjectif}} \hspace{4pt} Ton~: L.

Furieux, en rage (attitude d'une personne violente et présomptueuse).
\textcolor{brown}{\zh{大发雷霆。}}


\begin{exe}
\sn \textcolor{darkblue}{\textbf{\ipa{ɬi˩ʁɑ˩ ʝi˧}}}
\trans se livrer au courroux
\trans \textit{\textcolor{brown}{\zh{大发雷霆}}}
\end{exe}


\paragraph{\hspace{-0.5cm} {\Large \textcolor{darkblue}{\textbf{\ipa{ɬi˩ʈɯ˩mæ˥}}}}} \hypertarget{Ki\string_Bt`M\string_Bm\{\string_T1}{}
\markboth{\textcolor{darkblue}{\textbf{\ipa{ɬi˩ʈɯ˩mæ˥}}}}{}
\textcolor{teal}{\mytextsc{nom}} \hspace{4pt} Ton~: L+H\#.

Contrebas du foyer: place dans la salle principale entre le foyer et la porte (où les chiens sont tolérés en fin de repas; on leur y jette des os et autres débris de nourriture; dans la maison de F4, à la date de l’enquête, c’est un endroit où rien ne recouvre le sol cimenté.
\textcolor{brown}{\zh{主屋里面没有火铺的地方:没有木地板、小狗可以偶尔进来的地方(家人就给它扔骨头)。}}


\begin{exe}
\sn \textcolor{darkblue}{\textbf{\ipa{u˧=ɻ̍˩, | kʰv̩˩mi˩ ʈʂʰɯ˩-jɤ˥ | ɖɯ˧-njɤ˧-zo˥ | ɬi˩ʈɯ˩mæ˥ hĩ˩ dʑo˩.}}}
\trans Nous (=dans notre maison), ce chien, il se tient souvent assis en contrebas du foyer.
\trans \textit{\textcolor{brown}{\zh{咱们家这只狗经常呆在主屋火塘下面的地方。}}}
\end{exe}
 \mytextsc{clf}~: \textcolor{darkblue}{\textbf{\ipa{kʰwɤ˥}}} 

\paragraph{\hspace{-0.5cm} {\Large \textcolor{darkblue}{\textbf{\ipa{ɬi˩zo˩}}}}} \hypertarget{Ki\string_Bzo\string_B1}{}
\markboth{\textcolor{darkblue}{\textbf{\ipa{ɬi˩zo˩}}}}{}
\textcolor{teal}{\mytextsc{nom}} \hspace{4pt} Ton~: L.

Bébé chevrotain.
\textcolor{brown}{\zh{小獐子。}}




\paragraph{\hspace{-0.5cm} {\Large \textcolor{darkblue}{\textbf{\ipa{ɬi˧˥}}}}} \hypertarget{Ki\string_M\string_T1}{}
\markboth{\textcolor{darkblue}{\textbf{\ipa{ɬi˧˥}}}}{}
\textcolor{teal}{\mytextsc{verbe}} \hspace{4pt} Ton~: MH.

Faire sécher au soleil.
\textcolor{brown}{\zh{晒干。}}


\begin{exe}
\sn \textcolor{darkblue}{\textbf{\ipa{le˧-pv̩˧ tʰi˧-ɬi˧˥}}}
\trans exposer au soleil afin de faire sécher
\trans \textit{\textcolor{brown}{\zh{晒干}}}
\end{exe}


\paragraph{\hspace{-0.5cm} {\Large \textcolor{darkblue}{\textbf{\ipa{ɬo˥}}}}} \hypertarget{Ko\string_T1}{}
\markboth{\textcolor{darkblue}{\textbf{\ipa{ɬo˥}}}}{}
\textcolor{teal}{\mytextsc{nom}} \hspace{4pt} Ton~: \#H.

Côte.
\textcolor{brown}{\zh{肋骨。}}


\begin{exe}
\sn \textcolor{darkblue}{\textbf{\ipa{bo˩ɬo˧}}}
\trans côtes de porc
\trans \textit{\textcolor{brown}{\zh{猪肋骨}}}
\end{exe}
 \mytextsc{clf}~: \textcolor{darkblue}{\textbf{\ipa{ɭɯ˧}}} 

\paragraph{\hspace{-0.5cm} {\Large \textcolor{darkblue}{\textbf{\ipa{ɬo˧kʰv̩˧}}}}} \hypertarget{Ko\string_Mk\string_hv\string_=\string_M1}{}
\markboth{\textcolor{darkblue}{\textbf{\ipa{ɬo˧kʰv̩˧}}}}{}
\textcolor{teal}{\mytextsc{nom}} \hspace{4pt} Ton~: M.

Hanche.
\textcolor{brown}{\zh{胯。}}


 \mytextsc{clf}~: \textcolor{darkblue}{\textbf{\ipa{ɭɯ˧}}} \textit{Syn~:} \hyperlink{Ko\string_Bts\string_he\string_Bm\{\string_T1}{\textcolor{darkblue}{\textbf{\ipa{ɬo˩tsʰe˩mæ˥}}}}. 

\paragraph{\hspace{-0.5cm} {\Large \textcolor{darkblue}{\textbf{\ipa{ɬo˧pɤ˥}}}}} \hypertarget{Ko\string_Mp7\string_T1}{}
\markboth{\textcolor{darkblue}{\textbf{\ipa{ɬo˧pɤ˥}}}}{}
\textcolor{teal}{\mytextsc{nom}} \hspace{4pt} Ton~: H\#.

Ampoule.
\textcolor{brown}{\zh{水泡。}}


\begin{exe}
\sn \textcolor{darkblue}{\textbf{\ipa{ɬo˧pɤ˥ qʰwæ˩-ze˩!}}}
\trans (Il s'est/Tu t'es/Je me suis) fait une ampoule!
\trans \textit{\textcolor{brown}{\zh{起了水泡!}}}
\end{exe}
\begin{exe}
\sn \textcolor{darkblue}{\textbf{\ipa{ɬo˧pɤ˥ | ɖɯ˧-ɭɯ˧ | qʰwæ˧-ze˥!}}}
\trans (Il s'est/Tu t'es/Je me suis) fait une ampoule!
\trans \textit{\textcolor{brown}{\zh{起了一个水泡!}}}
\end{exe}
\begin{exe}
\sn \textcolor{darkblue}{\textbf{\ipa{ɬo˧pɤ˥ | ʁo˩-po˥-ɳɯ˩ | ʈʂe˩˥}}}
\trans percer une ampoule à l'aide d'une aiguille
\trans \textit{\textcolor{brown}{\zh{用针来扎水泡}}}
\end{exe}
 \mytextsc{clf}~: \textcolor{darkblue}{\textbf{\ipa{ɭɯ˧}}} 

\paragraph{\hspace{-0.5cm} {\Large \textcolor{darkblue}{\textbf{\ipa{ɬo˧pv̩˥}}}}} \hypertarget{Ko\string_Mpv\string_=\string_T1}{}
\markboth{\textcolor{darkblue}{\textbf{\ipa{ɬo˧pv̩˥}}}}{}
\textcolor{teal}{\mytextsc{nom}} \hspace{4pt} Ton~: H\#.

Prosternation, kow-tow (très probablement emprunt tibétain).
\textcolor{brown}{\zh{跪下磕头 (叩头)。}}


\begin{exe}
\sn \textcolor{darkblue}{\textbf{\ipa{ɬo˧pv̩˥ ti˩}}}
\trans se prosterner
\trans \textit{\textcolor{brown}{\zh{跪下磕头}}}
\end{exe}
\begin{exe}
\sn \textcolor{darkblue}{\textbf{\ipa{ɬo˧pv̩˥ | le˧-ti˩}}}
\trans se prosterner
\trans \textit{\textcolor{brown}{\zh{跪下磕头}}}
\end{exe}
\textit{Voir~:} \hyperlink{Ko\string_M\string_T1}{\textcolor{darkblue}{\textbf{\ipa{ɬo˧˥}}}} 

\paragraph{\hspace{-0.5cm} {\Large \textcolor{darkblue}{\textbf{\ipa{ɬo˧tɑ˧}}}}} \hypertarget{Ko\string_MtA\string_M1}{}
\markboth{\textcolor{darkblue}{\textbf{\ipa{ɬo˧tɑ˧}}}}{}
\textcolor{teal}{\mytextsc{préposition}} \hspace{4pt} Ton~: M.

À côté de, sur le côté de.
\textcolor{brown}{\zh{旁边。}}


\begin{exe}
\sn \textcolor{darkblue}{\textbf{\ipa{ɬo˧tɑ˧ ɻ̍˩}}}
\trans se tourner vers le côté, se tourner de côté
\trans \textit{\textcolor{brown}{\zh{向侧面转}}}
\end{exe}
\begin{exe}
\sn \textcolor{darkblue}{\textbf{\ipa{ʁo˧qʰwɤ˩ | ɬo˧tɑ˧ | go˩˥}}}
\trans avoir mal sur le côté de la tête, avoir les tempes qui bourdonnent (littéralement: 'avoir mal sur les côtés de la tête')
\trans \textit{\textcolor{brown}{\zh{头疼,太阳穴阵痛}}}
\end{exe}


\paragraph{\hspace{-0.5cm} {\Large \textcolor{darkblue}{\textbf{\ipa{ɬo˧tɑ˧-ɬo˩ɲi˩}}}}} \hypertarget{Ko\string_MtA\string_M-Ko\string_BJi\string_B1}{}
\markboth{\textcolor{darkblue}{\textbf{\ipa{ɬo˧tɑ˧-ɬo˩ɲi˩}}}}{}
\textcolor{teal}{\mytextsc{adverbe}} \hspace{4pt} Ton~: \mytextsc{L}.

À droite et à gauche, aux alentours; autour de.
\textcolor{brown}{\zh{周围,附近。}}




\paragraph{\hspace{-0.5cm} {\Large \textcolor{darkblue}{\textbf{\ipa{ɬo˩kɤ˩}}}}} \hypertarget{Ko\string_Bk7\string_B1}{}
\markboth{\textcolor{darkblue}{\textbf{\ipa{ɬo˩kɤ˩}}}}{}
\textcolor{teal}{\mytextsc{nom}} \hspace{4pt} Ton~: L.

Côte (partie du corps).
\textcolor{brown}{\zh{肋骨。}}


 \mytextsc{clf}~: \textcolor{darkblue}{\textbf{\ipa{kɤ˧˥}}} 

\paragraph{\hspace{-0.5cm} {\Large \textcolor{darkblue}{\textbf{\ipa{ɬo˩tsʰe˩mæ˥}}}}} \hypertarget{Ko\string_Bts\string_he\string_Bm\{\string_T1}{}
\markboth{\textcolor{darkblue}{\textbf{\ipa{ɬo˩tsʰe˩mæ˥}}}}{}
\textcolor{teal}{\mytextsc{nom}} \hspace{4pt} Ton~: L+H\#.
\textit{Archaïque} [\textcolor{brown}{\zh{古语}}] 
Hanche.
\textcolor{brown}{\zh{胯。}}


 \mytextsc{clf}~: \textcolor{darkblue}{\textbf{\ipa{ɭɯ˧}}} \textit{Syn~:} \hyperlink{Ko\string_Mk\string_hv\string_=\string_M1}{\textcolor{darkblue}{\textbf{\ipa{ɬo˧kʰv̩˧}}}}. 

\paragraph{\hspace{-0.5cm} {\Large \textcolor{darkblue}{\textbf{\ipa{ɬo˧˥}}}}} \hypertarget{Ko\string_M\string_T1}{}
\markboth{\textcolor{darkblue}{\textbf{\ipa{ɬo˧˥}}}}{}
\textcolor{teal}{\mytextsc{adjectif}} \hspace{4pt} Ton~: MH.

Profond (eau).
\textcolor{brown}{\zh{深(水深)。}}




\paragraph{\hspace{-0.5cm} {\Large \textcolor{darkblue}{\textbf{\ipa{ɬv̩˧˥}}}}} \hypertarget{Kv\string_=\string_M\string_T1}{}
\markboth{\textcolor{darkblue}{\textbf{\ipa{ɬv̩˧˥}}}}{}
\textcolor{teal}{\mytextsc{nom}} \hspace{4pt} Ton~: MH.
\ding{202} 
Cerveau, cervelle.
\textcolor{brown}{\zh{脑子、脑髓。}}


\ding{203} 
Moëlle (des os).
\textcolor{brown}{\zh{骨髓。}}


 \mytextsc{clf}~: \textcolor{darkblue}{\textbf{\ipa{ʈv̩˩}}} 

\paragraph{\hspace{-0.5cm} {\Large \textcolor{darkblue}{\textbf{\ipa{ɬv̩˩\textsubscript{a}}}} \textsubscript{1}}} \hypertarget{Kv\string_=\string_Ba1}{}
\markboth{\textcolor{darkblue}{\textbf{\ipa{ɬv̩˩\textsubscript{a}}}} \textsubscript{1}}{}
\textcolor{teal}{\mytextsc{verbe}} \hspace{4pt} Ton~: L\textsubscript{a}.

Garder dans la bouche, laisser fondre dans la bouche.
\textcolor{brown}{\zh{含在嘴里、在嘴巴里溶化。}}


\begin{exe}
\sn \textcolor{darkblue}{\textbf{\ipa{tso˧\textasciitilde{}tso˧ ɬv̩˥}}}
\trans mettre des choses dans sa bouche (contexte: un petit enfant qui ne fait pas encore la différence entre nourriture et choses non comestibles met des choses dans sa bouche)
\trans \textit{\textcolor{brown}{\zh{含在嘴里(情景:一个小孩把不能吃的东西含在嘴巴里)}}}
\end{exe}


\paragraph{\hspace{-0.5cm} {\Large \textcolor{darkblue}{\textbf{\ipa{ɬv̩˩\textsubscript{a}}}} \textsubscript{2}}} \hypertarget{Kv\string_=\string_Ba2}{}
\markboth{\textcolor{darkblue}{\textbf{\ipa{ɬv̩˩\textsubscript{a}}}} \textsubscript{2}}{}
\textcolor{teal}{\mytextsc{adjectif}} \hspace{4pt} Ton~: L\textsubscript{a}.

Chaud, tiède (agréablement tiède, pas froid).
\textcolor{brown}{\zh{温暖,暖和。}}


\begin{exe}
\sn \textcolor{darkblue}{\textbf{\ipa{dʑɤ˩˥ | ɬv̩˩˥}}}
\trans très tiède, bien tiède
\trans \textit{\textcolor{brown}{\zh{温暖}}}
\end{exe}
\begin{exe}
\sn \textcolor{darkblue}{\textbf{\ipa{ɖwæ˧˥ | ɬv̩˩˥}}}
\trans \mytextsc{intensif}.très: très tiède
\trans \textit{\textcolor{brown}{\zh{很暖和}}}
\end{exe}
\begin{exe}
\sn \textcolor{darkblue}{\textbf{\ipa{ɬv̩˩-hĩ˩˥}}}
\trans \mytextsc{rel}/\mytextsc{nmlz}
\trans \textit{\textcolor{brown}{\zh{温暖的}}}
\end{exe}


\paragraph{\hspace{-0.5cm} {\Large \textcolor{darkblue}{\textbf{\ipa{ɬv̩˩\textsubscript{a}}}} \textsubscript{3}}} \hypertarget{Kv\string_=\string_Ba3}{}
\markboth{\textcolor{darkblue}{\textbf{\ipa{ɬv̩˩\textsubscript{a}}}} \textsubscript{3}}{}
\textcolor{teal}{\mytextsc{verbe}} \hspace{4pt} Ton~: L\textsubscript{a}.
\textit{De:} \textbf{ɬv̩˩a 2} 
Réchauffer de la nourriture.
\textcolor{brown}{\zh{热饭。}}


\begin{exe}
\sn \textcolor{darkblue}{\textbf{\ipa{hɑ˧ ɬv̩˧˥}}}
\trans réchauffer du riz / de la nourriture
\trans \textit{\textcolor{brown}{\zh{热饭}}}
\end{exe}
\begin{exe}
\sn \textcolor{darkblue}{\textbf{\ipa{hɑ˧ | le˧-ɬv̩˩}}}
\trans réchauffer du riz / de la nourriture
\trans \textit{\textcolor{brown}{\zh{热饭}}}
\end{exe}
\begin{exe}
\sn \textcolor{darkblue}{\textbf{\ipa{hɑ˧ | ɖɯ˧-ɬv̩˧\textasciitilde{}ɬv̩˥-ɻ̍˩}}}
\trans réchauffer un peu la nourriture
\trans \textit{\textcolor{brown}{\zh{饭热一热}}}
\end{exe}


\paragraph{\hspace{-0.5cm} {\Large \textcolor{darkblue}{\textbf{\ipa{ɬv̩˧gv̩\#˥}}}}} \hypertarget{Kv\string_=\string_Mgv\string_=\#\string_T1}{}
\markboth{\textcolor{darkblue}{\textbf{\ipa{ɬv̩˧gv̩\#˥}}}}{}
\textcolor{teal}{\mytextsc{nom}} \hspace{4pt} Ton~: \#H.

Nourriture qu'on offre rituellement au défunt, sept jours après sa crémation.
\textcolor{brown}{\zh{火葬后第七天的送食物仪式。}}




\paragraph{\hspace{-0.5cm} {\Large \textcolor{darkblue}{\textbf{\ipa{ɬv̩˧mi˧mæ˧dv̩˧mi\#˥}}}}} \hypertarget{Kv\string_=\string_Mmi\string_Mm\{\string_Mdv\string_=\string_Mmi\#\string_T1}{}
\markboth{\textcolor{darkblue}{\textbf{\ipa{ɬv̩˧mi˧mæ˧dv̩˧mi\#˥}}}}{}
\textcolor{teal}{\mytextsc{nom}} \hspace{4pt} Ton~: \#H.

Mante religieuse.
\textcolor{brown}{\zh{螳螂。}}


\begin{exe}
\sn \textcolor{darkblue}{\textbf{\ipa{ɬv̩˧mi˧mæ˧dv̩˧mi˧ tʰv̩˧-mi˧˥ / ɬv̩˧mi˧mæ˧dv̩˧mi˧ tʰv̩˧-mi˥\#}}}
\trans \mytextsc{n}+\mytextsc{dem}+\mytextsc{clf}
\trans \textit{\textcolor{brown}{\zh{那只螳螂}}}
\end{exe}
 \mytextsc{clf}~: \textcolor{darkblue}{\textbf{\ipa{mi˩}}} 

\paragraph{\hspace{-0.5cm} {\Large \textcolor{darkblue}{\textbf{\ipa{ɬv̩˩pʰæ˩˥}}}}} \hypertarget{Kv\string_=\string_Bp\string_h\{\string_B\string_T1}{}
\markboth{\textcolor{darkblue}{\textbf{\ipa{ɬv̩˩pʰæ˩˥}}}}{}
\textcolor{teal}{\mytextsc{adjectif}} \hspace{4pt} Ton~: .

Bien tiède.
\textcolor{brown}{\zh{温暖,暖和。}}




\paragraph{\hspace{-0.5cm} {\Large \textcolor{darkblue}{\textbf{\ipa{ɬv̩˧ʁwɤ\#˥}}}}} \hypertarget{Kv\string_=\string_MRw7\#\string_T1}{}
\markboth{\textcolor{darkblue}{\textbf{\ipa{ɬv̩˧ʁwɤ\#˥}}}}{}
\textcolor{teal}{\mytextsc{nom}} \hspace{4pt} Ton~: \#H.

Village en aval de Qiansuo; leur langue serait relativement proche de celle de la vallée de Yongning.
\textcolor{brown}{\zh{村落名。}}




\newpage
\section*{\centering- \textcolor{darkblue}{\textbf{\ipa{m}}} -}
\paragraph{\hspace{-0.5cm} {\Large \textcolor{darkblue}{\textbf{\ipa{mɑ˧pʰv̩˧}}}}} \hypertarget{mA\string_Mp\string_hv\string_=\string_M1}{}
\markboth{\textcolor{darkblue}{\textbf{\ipa{mɑ˧pʰv̩˧}}}}{}
\textcolor{teal}{\mytextsc{nom}} \hspace{4pt} Ton~: M.

Beurre (pour la préparation du thé au beurre).
\textcolor{brown}{\zh{酥油。}}




\paragraph{\hspace{-0.5cm} {\Large \textcolor{darkblue}{\textbf{\ipa{mɑ˧tsɑ˥}}}}} \hypertarget{mA\string_MtsA\string_T1}{}
\markboth{\textcolor{darkblue}{\textbf{\ipa{mɑ˧tsɑ˥}}}}{}
\textcolor{teal}{\mytextsc{nom}} \hspace{4pt} Ton~: H\#.

Origine, cause (lointaine).
\textcolor{brown}{\zh{来历、发源地、深层原因/来源、来龙去脉、脉络。}}


\begin{exe}
\sn \textcolor{darkblue}{\textbf{\ipa{mɑ˧tsɑ˥ | ʈʂʰɯ˧-qo˧ le˧-tsʰɯ˩-ɲi˩! |}}}
\trans (Au sujet d'un événement) Ca vient de loin! / ça a une origine/ça n'arrive pas par hasard!
\trans \textit{\textcolor{brown}{\zh{这(件事情)出处很远! / 有它的来龙去脉(=不是突然一下子出现的)!}}}
\end{exe}
\begin{exe}
\sn \textcolor{darkblue}{\textbf{\ipa{mɑ˧tsɑ˥ ʈʂʰɯ˩-kʰwɤ˩ |}}}
\trans \mytextsc{n}+\mytextsc{dem}+\mytextsc{clf}: cette cause, cette origine
\trans \textit{\textcolor{brown}{\zh{这个来历}}}
\end{exe}
 \mytextsc{clf}~: \textcolor{darkblue}{\textbf{\ipa{kʰwɤ˥}}} 

\paragraph{\hspace{-0.5cm} {\Large \textcolor{darkblue}{\textbf{\ipa{mɑ˩dzɑ˩}}}}} \hypertarget{mA\string_BdzA\string_B1}{}
\markboth{\textcolor{darkblue}{\textbf{\ipa{mɑ˩dzɑ˩}}}}{}
\textcolor{teal}{\mytextsc{nom}} \hspace{4pt} Ton~: L.

Encre (solide).
\textcolor{brown}{\zh{墨。}}


 \mytextsc{clf}~: \textcolor{darkblue}{\textbf{\ipa{qʰwɤ˧˥}}} bol

\paragraph{\hspace{-0.5cm} {\Large \textcolor{darkblue}{\textbf{\ipa{mɑ˩ɳɯ\#˥}}}}} \hypertarget{mA\string_Bn`M\#\string_T1}{}
\markboth{\textcolor{darkblue}{\textbf{\ipa{mɑ˩ɳɯ\#˥}}}}{}
\textcolor{teal}{\mytextsc{nom}} \hspace{4pt} Ton~: LM+\#H.

Mur de mani: mur de pierre sèche et de sable, comportant des tablettes de pierre sur lesquelles est gravé une inscription: le plus souvent Om Mani Padme Hum. Un mur de mani doit être contourné dans le sens des aiguilles d'une montre: le sens de rotation de l'univers, selon la doctrine bouddhiste.
\textcolor{brown}{\zh{嘛呢堆。}}


 \mytextsc{clf}~: \textcolor{darkblue}{\textbf{\ipa{ɭɯ˧}}} \textit{Voir~:} \textcolor{darkblue}{\textbf{\ipa{mɑ˩ɳɯ˧-do˥bv˩, do˩bv̩\#˥}}} 

\paragraph{\hspace{-0.5cm} {\Large \textcolor{darkblue}{\textbf{\ipa{mɑ˩ɳɯ˧-do˥bv̩˩}}}}} \hypertarget{mA\string_Bn`M\string_M-do\string_Tbv\string_=\string_B1}{}
\markboth{\textcolor{darkblue}{\textbf{\ipa{mɑ˩ɳɯ˧-do˥bv̩˩}}}}{}
\textcolor{teal}{\mytextsc{nom}} \hspace{4pt} Ton~: LM+\#H-.

Mur de mani: mur de pierre sèche et de sable, comportant des tablettes de pierre sur lesquelles est gravé une inscription: le plus souvent Om Mani Padme Hum. Un mur de mani doit être contourné dans le sens des aiguilles d'une montre: le sens de rotation de l'univers, selon la doctrine bouddhiste.
\textcolor{brown}{\zh{嘛呢堆。}}


 \mytextsc{clf}~: \textcolor{darkblue}{\textbf{\ipa{ɭɯ˧}}} \textit{Voir~:} \textcolor{darkblue}{\textbf{\ipa{mɑ˩ɳɯ\#˥, do˩bv̩\#˥}}} 

\paragraph{\hspace{-0.5cm} {\Large \textcolor{darkblue}{\textbf{\ipa{‑mæ.mæ}}}}} \hypertarget{‑m\{.m\{1}{}
\markboth{\textcolor{darkblue}{\textbf{\ipa{‑mæ.mæ}}}}{}
\textcolor{teal}{\mytextsc{postposition}} \hspace{4pt} Ton~: .

À la fin de, vers la fin de.
\textcolor{brown}{\zh{……末/底。}}


\begin{exe}
\sn \textcolor{darkblue}{\textbf{\ipa{ɲi˧ɬi˧mi˧-mæ˧mæ˥, | qʰɑ˧dze˧ tv̩˧!}}}
\trans vers la fin du deuxième mois, on plante le maïs!
\trans \textit{\textcolor{brown}{\zh{二月底,就种玉米! / 玉米是在二月底种的!}}}
\end{exe}
\begin{exe}
\sn \textcolor{darkblue}{\textbf{\ipa{gv̩˩ɬi˩mi˩-mæ˩-mæ˥}}}
\trans vers la fin du neuvième mois
\trans \textit{\textcolor{brown}{\zh{九月底}}}
\end{exe}


\paragraph{\hspace{-0.5cm} {\Large \textcolor{darkblue}{\textbf{\ipa{mæ˧}}}}} \hypertarget{m\{\string_M1}{}
\markboth{\textcolor{darkblue}{\textbf{\ipa{mæ˧}}}}{}
\textcolor{teal}{\mytextsc{verbe}} \hspace{4pt} Ton~: M.

Parvenir à, réussir à.
\textcolor{brown}{\zh{……成、……成功。}}


\begin{exe}
\sn \textcolor{darkblue}{\textbf{\ipa{njɤ˧ ɖʐɤ˧˥ | tʰi˧-mɤ˧-mæ˧!}}}
\trans je ne parviens pas à attraper (ex.: un fruit sur une branche trop élevée)
\trans \textit{\textcolor{brown}{\zh{我够不着!(例如:够不着树枝上的果子)}}}
\end{exe}


\paragraph{\hspace{-0.5cm} {\Large \textcolor{darkblue}{\textbf{\ipa{mæ˧}}}}} \hypertarget{m\{\string_M1}{}
\markboth{\textcolor{darkblue}{\textbf{\ipa{mæ˧}}}}{}
\textcolor{teal}{\mytextsc{particule}} \textcolor{teal}{\mytextsc{de}} \textcolor{teal}{\mytextsc{discours}} \hspace{4pt} Ton~: M.

Particule finale exprimant l'évidence.
\textcolor{brown}{\zh{句尾助词,表示显然、理所当然:“……呗!”。}}


\begin{exe}
\sn \textcolor{darkblue}{\textbf{\ipa{[Healing.66] hu˧mi˧-ʈʂʰæ˧ɣɯ˧ | le˧-ʈʰɯ˩, | le˧-qʰwɤ˧-ze˧ mæ˧! |}}}
\trans On prend des médicaments pour l'estomac, et ça guérit!
\trans \textit{\textcolor{brown}{\zh{吃了胃药,就好了呗!}}}
\end{exe}


\paragraph{\hspace{-0.5cm} {\Large \textcolor{darkblue}{\textbf{\ipa{mæ˧}}} \textsubscript{1}}} \hypertarget{m\{\string_M1}{}
\markboth{\textcolor{darkblue}{\textbf{\ipa{mæ˧}}} \textsubscript{1}}{}
\textcolor{teal}{\mytextsc{verbe}} \hspace{4pt} Ton~: M.

Avoir le temps de, être libre.
\textcolor{brown}{\zh{(有)空。}}


\begin{exe}
\sn \textcolor{darkblue}{\textbf{\ipa{njɤ˧ | mɤ˧-mæ˧.}}}
\trans je suis occupé, je n'ai pas le temps
\trans \textit{\textcolor{brown}{\zh{我忙、我没有空}}}
\end{exe}
\begin{exe}
\sn \textcolor{darkblue}{\textbf{\ipa{njɤ˧ | mæ˧-mɤ˧-ho˩.}}}
\trans Je ne vais pas avoir le temps.
\trans \textit{\textcolor{brown}{\zh{我不会有时间。}}}
\end{exe}


\paragraph{\hspace{-0.5cm} {\Large \textcolor{darkblue}{\textbf{\ipa{mæ˧}}} \textsubscript{2}}} \hypertarget{m\{\string_M2}{}
\markboth{\textcolor{darkblue}{\textbf{\ipa{mæ˧}}} \textsubscript{2}}{}
\textcolor{teal}{\mytextsc{verbe}} \hspace{4pt} Ton~: M.

Parvenir à.
\textcolor{brown}{\zh{能够(做)。}}


\begin{exe}
\sn \textcolor{darkblue}{\textbf{\ipa{ɖɯ˩-hĩ˩ qʰɑ˥ mæ˩\textasciitilde{}mæ˩! | tɕi˩-hĩ˩ lə˥-mɤ˩-mæ˩! / ɖɯ˩-hĩ˩˥, | qʰɑ˧ mæ˥\textasciitilde{}mæ˩! | tɕi˩-hĩ˩˥, | le˧-mɤ˧-mæ˧!}}}
\trans ``Les adultes peuvent tout faire; les enfants, eux, n'y arrivent pas/n'en sont pas capables!" Sens: s'adresse à quelqu'un qui assigne des tâches aux enfants et adolescents: Laissez les enfants jouer! A chacun ses occupations: les adultes travaillent; les enfants, leur tâche, c'est de s'amuser entre eux, pas de travailler. Le travail des grands, c'est pas leur affaire!
\trans \textit{\textcolor{brown}{\zh{“大人管干活,小孩管玩耍!”这个谚语的意思是:不要让孩子干活,每个年龄有自己的事,孩子的事就是玩。成年人的活儿,不是他们的事!}}}
\end{exe}


\paragraph{\hspace{-0.5cm} {\Large \textcolor{darkblue}{\textbf{\ipa{mæ˧\textsubscript{a}}}}}} \hypertarget{m\{\string_Ma1}{}
\markboth{\textcolor{darkblue}{\textbf{\ipa{mæ˧\textsubscript{a}}}}}{}
\textcolor{teal}{\mytextsc{verbe}} \hspace{4pt} Ton~: M\textsubscript{a}.
\ding{202} 
Attraper (un objet en hauteur).
\textcolor{brown}{\zh{钩住(东西)。}}


\begin{exe}
\sn \textcolor{darkblue}{\textbf{\ipa{tʰi˧-mæ˧-ze˧}}}
\trans \mytextsc{dur} \string_ \mytextsc{pfv}
\trans \textit{\textcolor{brown}{\zh{钩住了}}}
\end{exe}
\ding{203} 
Rattraper, rejoindre (quelqu'un qui est plus avant sur un chemin/une route).
\textcolor{brown}{\zh{跟上。}}




\paragraph{\hspace{-0.5cm} {\Large \textcolor{darkblue}{\textbf{\ipa{mæ˧pæ˧}}}}} \hypertarget{m\{\string_Mp\{\string_M1}{}
\markboth{\textcolor{darkblue}{\textbf{\ipa{mæ˧pæ˧}}}}{}
\textcolor{teal}{\mytextsc{nom}} \hspace{4pt} Ton~: M.

Vannerie.
\textcolor{brown}{\zh{大筛子。}}


 \mytextsc{clf}~: \textcolor{darkblue}{\textbf{\ipa{nɑ˧}}} 

\paragraph{\hspace{-0.5cm} {\Large \textcolor{darkblue}{\textbf{\ipa{-mæ˧qo˩}}}}} \hypertarget{-m\{\string_Mqo\string_B1}{}
\markboth{\textcolor{darkblue}{\textbf{\ipa{-mæ˧qo˩}}}}{}
\textcolor{teal}{\mytextsc{postposition}} \hspace{4pt} Ton~: L\#.

Derrière, sous.
\textcolor{brown}{\zh{下面,后面。}}


\textit{Voir~:} \hyperlink{m\{\string_Mqo\string_B1}{\textcolor{darkblue}{\textbf{\ipa{mæ˧qo˩}}}} 

\paragraph{\hspace{-0.5cm} {\Large \textcolor{darkblue}{\textbf{\ipa{mæ˧qo˩}}}}} \hypertarget{m\{\string_Mqo\string_B1}{}
\markboth{\textcolor{darkblue}{\textbf{\ipa{mæ˧qo˩}}}}{}
\textcolor{teal}{\mytextsc{adverbe}} \hspace{4pt} Ton~: L\#.

En bas, au fond; à l'arrière, derrière.
\textcolor{brown}{\zh{在尽头、在极点,在下面、在后面。}}


\textit{Voir~:} \hyperlink{-m\{\string_Mqo\string_B1}{\textcolor{darkblue}{\textbf{\ipa{-mæ˧qo˩}}}} 

\paragraph{\hspace{-0.5cm} {\Large \textcolor{darkblue}{\textbf{\ipa{mæ˧qv̩˩}}}}} \hypertarget{m\{\string_Mqv\string_=\string_B1}{}
\markboth{\textcolor{darkblue}{\textbf{\ipa{mæ˧qv̩˩}}}}{}
\textcolor{teal}{\mytextsc{nom}} \hspace{4pt} Ton~: L\#.

Queue.
\textcolor{brown}{\zh{尾巴。}}


\begin{exe}
\sn \textcolor{darkblue}{\textbf{\ipa{ʝi˧-mæ˧qv̩˥}}}
\trans queue de la vache
\trans \textit{\textcolor{brown}{\zh{牛尾巴}}}
\end{exe}
 \mytextsc{clf}~: \textcolor{darkblue}{\textbf{\ipa{ɭɯ˧}}} 

\paragraph{\hspace{-0.5cm} {\Large \textcolor{darkblue}{\textbf{\ipa{mæ˧ɻæ˩}}}}} \hypertarget{m\{\string_Mr£`\{\string_B1}{}
\markboth{\textcolor{darkblue}{\textbf{\ipa{mæ˧ɻæ˩}}}}{}
\textcolor{teal}{\mytextsc{nom}} \hspace{4pt} Ton~: L\#.

Huile végétale.
\textcolor{brown}{\zh{植物油。}}




\paragraph{\hspace{-0.5cm} {\Large \textcolor{darkblue}{\textbf{\ipa{mæ˧ɻ̃\#˥}}}}} \hypertarget{m\{\string_Mr£`\string_~\#\string_T1}{}
\markboth{\textcolor{darkblue}{\textbf{\ipa{mæ˧ɻ̃\#˥}}}}{}
\textcolor{teal}{\mytextsc{nom}} \hspace{4pt} Ton~: \#H.

Coccyx.
\textcolor{brown}{\zh{尾椎骨。}}Dialecte chinois local~: \zh{尾结骨。}


 \mytextsc{clf}~: \textcolor{darkblue}{\textbf{\ipa{ɭɯ˧}}} 

\paragraph{\hspace{-0.5cm} {\Large \textcolor{darkblue}{\textbf{\ipa{mæ˩}}}}} \hypertarget{m\{\string_B1}{}
\markboth{\textcolor{darkblue}{\textbf{\ipa{mæ˩}}}}{}
\textcolor{teal}{\mytextsc{verbe}} \hspace{4pt} Ton~: L.

Irriguer (en faisant couler de l'eau dans de petites tranchées).
\textcolor{brown}{\zh{灌溉。}}


\begin{exe}
\sn \textcolor{darkblue}{\textbf{\ipa{dʑɯ˩ mæ˩˥}}}
\trans irriguer, arroser, mettre de l’eau
\trans \textit{\textcolor{brown}{\zh{浇灌}}}
\end{exe}
\begin{exe}
\sn \textcolor{darkblue}{\textbf{\ipa{dʑɯ˧ | le˧-mæ˩}}}
\trans \mytextsc{accomp}: irriguer, arroser
\trans \textit{\textcolor{brown}{\zh{浇灌}}}
\end{exe}


\paragraph{\hspace{-0.5cm} {\Large \textcolor{darkblue}{\textbf{\ipa{mæ˩\textsubscript{a}}}}}} \hypertarget{m\{\string_Ba1}{}
\markboth{\textcolor{darkblue}{\textbf{\ipa{mæ˩\textsubscript{a}}}}}{}
\textcolor{teal}{\mytextsc{verbe}} \hspace{4pt} Ton~: L\textsubscript{a}.

Viser; pointer, montrer du doigt.
\textcolor{brown}{\zh{瞄准,指。}}


\begin{exe}
\sn \textcolor{darkblue}{\textbf{\ipa{tʰi˧-mæ˩-ze˩, | qʰæ˧-bi˥-ze˩.}}}
\trans (Il) a visé, (il) va tirer.
\trans \textit{\textcolor{brown}{\zh{瞄准了,要开枪了。}}}
\end{exe}
\begin{exe}
\sn \textcolor{darkblue}{\textbf{\ipa{lo˧ɲi˥ mæ˩}}}
\trans montrer du doigt
\trans \textit{\textcolor{brown}{\zh{用手指出}}}
\end{exe}
\begin{exe}
\sn \textcolor{darkblue}{\textbf{\ipa{tso˧\textasciitilde{}tso˧ mæ˥}}}
\trans pointer des choses du doigt
\trans \textit{\textcolor{brown}{\zh{指东西}}}
\end{exe}


\paragraph{\hspace{-0.5cm} {\Large \textcolor{darkblue}{\textbf{\ipa{mæ˩\textsubscript{a}}}} \textsubscript{1}}} \hypertarget{m\{\string_Ba1}{}
\markboth{\textcolor{darkblue}{\textbf{\ipa{mæ˩\textsubscript{a}}}} \textsubscript{1}}{}
\textcolor{teal}{\mytextsc{classificateur}} \hspace{4pt} Ton~: L\textsubscript{a}.

Unité monétaire: un yuan.
\textcolor{brown}{\zh{量词:钱(一元)。}}


\begin{exe}
\sn \textcolor{darkblue}{\textbf{\ipa{ʈʂʰɯ˧-mæ˥}}}
\trans \mytextsc{dem} \string_ (tone: H\# / H\$)
\trans \textit{\textcolor{brown}{\zh{\mytextsc{指示代词} \string_}}}
\end{exe}


\paragraph{\hspace{-0.5cm} {\Large \textcolor{darkblue}{\textbf{\ipa{mæ˩\textsubscript{a}}}} \textsubscript{2}}} \hypertarget{m\{\string_Ba2}{}
\markboth{\textcolor{darkblue}{\textbf{\ipa{mæ˩\textsubscript{a}}}} \textsubscript{2}}{}
\textcolor{teal}{\mytextsc{classificateur}} \hspace{4pt} Ton~: L\textsubscript{a}.

10.000.
\textcolor{brown}{\zh{万(数词充当量词)。}}

 Emprunt~: chinois \textcolor{brown}{\zh{元}},MC*mjonH(Baxter2000)

\begin{exe}
\sn \textcolor{darkblue}{\textbf{\ipa{ɖɯ˧-mæ˩}}}
\trans 10.000
\trans \textit{\textcolor{brown}{\zh{一万}}}
\end{exe}
\begin{exe}
\sn \textcolor{darkblue}{\textbf{\ipa{tsʰe˩-tv̩˩ mæ˥}}}
\trans dix mille fois 10.000, soit cent millions
\trans \textit{\textcolor{brown}{\zh{十千万,等于一亿}}}
\end{exe}
\begin{exe}
\sn \textcolor{darkblue}{\textbf{\ipa{ɖɯ˧-ɕi˧ mæ˩}}}
\trans cent fois 10.000, soit un million
\trans \textit{\textcolor{brown}{\zh{一百万}}}
\end{exe}


\paragraph{\hspace{-0.5cm} {\Large \textcolor{darkblue}{\textbf{\ipa{mæ˩ɖʐo˥}}}}} \hypertarget{m\{\string_Bd`z`o\string_T1}{}
\markboth{\textcolor{darkblue}{\textbf{\ipa{mæ˩ɖʐo˥}}}}{}
\textcolor{teal}{\mytextsc{nom}} \hspace{4pt} Ton~: LH.

Fouet.
\textcolor{brown}{\zh{鞭子。}}


\begin{exe}
\sn \textcolor{darkblue}{\textbf{\ipa{ʐwæ˧-mæ˥ɖʐo˩}}}
\trans fouet de cheval
\trans \textit{\textcolor{brown}{\zh{马鞭}}}
\end{exe}
 \mytextsc{clf}~: \textcolor{darkblue}{\textbf{\ipa{kʰɯ˩}}} 

\paragraph{\hspace{-0.5cm} {\Large \textcolor{darkblue}{\textbf{\ipa{mæ˩ko˥}}}}} \hypertarget{m\{\string_Bko\string_T1}{}
\markboth{\textcolor{darkblue}{\textbf{\ipa{mæ˩ko˥}}}}{}
\textcolor{teal}{\mytextsc{nom}} \hspace{4pt} Ton~: LH.

Harnais.
\textcolor{brown}{\zh{挽具,后鞧。}}


\begin{exe}
\sn \textcolor{darkblue}{\textbf{\ipa{ʐwæ˧-mæ˥ko˩}}}
\trans harnais de cheval
\trans \textit{\textcolor{brown}{\zh{马后鞧}}}
\end{exe}
 \mytextsc{clf}~: \textcolor{darkblue}{\textbf{\ipa{ɭɯ˧}}} 

\paragraph{\hspace{-0.5cm} {\Large \textcolor{darkblue}{\textbf{\ipa{mæ˧˥}}}}} \hypertarget{m\{\string_M\string_T1}{}
\markboth{\textcolor{darkblue}{\textbf{\ipa{mæ˧˥}}}}{}
\textcolor{teal}{\mytextsc{verbe}} \hspace{4pt} Ton~: MH.
\ding{202} 
Fermer (la bouche).
\textcolor{brown}{\zh{闭(嘴)。}}


\begin{exe}
\sn \textcolor{darkblue}{\textbf{\ipa{ɲi˧to˧ | tʰi˧-mæ˧˥}}}
\trans fermer la bouche
\trans \textit{\textcolor{brown}{\zh{闭嘴}}}
\end{exe}
\begin{exe}
\sn \textcolor{darkblue}{\textbf{\ipa{mæ˩\textasciitilde{}mæ˧˥}}}
\trans \mytextsc{red}
\trans \textit{\textcolor{brown}{\zh{\mytextsc{重叠}}}}
\end{exe}
\begin{exe}
\sn \textcolor{darkblue}{\textbf{\ipa{ɲi˧to˧ | tʰi˧-mæ˩\textasciitilde{}mæ˩}}}
\trans fermer la bouche
\trans \textit{\textcolor{brown}{\zh{闭嘴}}}
\end{exe}
\ding{203} 
Pincer (les lèvres).
\textcolor{brown}{\zh{抿(嘴巴)。}}




\paragraph{\hspace{-0.5cm} {\Large \textcolor{darkblue}{\textbf{\ipa{}}}}} \hypertarget{m7\string_M‑1}{}
\markboth{\textcolor{darkblue}{\textbf{\ipa{}}}}{}
\textcolor{teal}{\mytextsc{préfixe}} \hspace{4pt} Ton~: M.

Negation.
\textcolor{brown}{\zh{否定:不,没。}}




\paragraph{\hspace{-0.5cm} {\Large \textcolor{darkblue}{\textbf{\ipa{mɤ˧-dɑ˩}}}}} \hypertarget{m7\string_M-dA\string_B1}{}
\markboth{\textcolor{darkblue}{\textbf{\ipa{mɤ˧-dɑ˩}}}}{}
\textcolor{teal}{\mytextsc{interjection}} \hspace{4pt} Ton~: \mytextsc{L}.

Hélas!
\textcolor{brown}{\zh{感叹词:唉呀!(自怨自艾的语气)。}}




\paragraph{\hspace{-0.5cm} {\Large \textcolor{darkblue}{\textbf{\ipa{mɤ˧-dɑ˩mi˩}}}}} \hypertarget{m7\string_M-dA\string_Bmi\string_B1}{}
\markboth{\textcolor{darkblue}{\textbf{\ipa{mɤ˧-dɑ˩mi˩}}}}{}
\textcolor{teal}{\mytextsc{interjection}} \hspace{4pt} Ton~: \mytextsc{L}.

Hélas!
\textcolor{brown}{\zh{感叹词:唉呀啊!(自怨自艾的语气)。}}




\paragraph{\hspace{-0.5cm} {\Large \textcolor{darkblue}{\textbf{\ipa{mɤ˧-dɑ˩qʰwɤ˩}}}}} \hypertarget{m7\string_M-dA\string_Bq\string_hw7\string_B1}{}
\markboth{\textcolor{darkblue}{\textbf{\ipa{mɤ˧-dɑ˩qʰwɤ˩}}}}{}
\textcolor{teal}{\mytextsc{interjection}} \hspace{4pt} Ton~: \mytextsc{L}.

Hélas!
\textcolor{brown}{\zh{感叹词:唉呀啊!(自怨自艾的语气)。}}




\paragraph{\hspace{-0.5cm} {\Large \textcolor{darkblue}{\textbf{\ipa{mɤ˧-ni˩\textasciitilde{}ni˩}}}}} \hypertarget{m7\string_M-ni\string_B~ni\string_B1}{}
\markboth{\textcolor{darkblue}{\textbf{\ipa{mɤ˧-ni˩\textasciitilde{}ni˩}}}}{}
\textcolor{teal}{\mytextsc{adjectif}} \hspace{4pt} Ton~: .
\ding{202} 
Différent, pas identique.
\textcolor{brown}{\zh{不一样、有区别。}}


\begin{exe}
\sn \textcolor{darkblue}{\textbf{\ipa{mɤ˧-ni˩\textasciitilde{}ni˩-hĩ˩-lɑ˩ ɲi˩-ze˩!}}}
\trans Ce n'est pas pareil!
\trans \textit{\textcolor{brown}{\zh{不是一样的!不是一回事!}}}
\end{exe}
\ding{203} 
Incroyable, extraordinaire, merveilleux.
\textcolor{brown}{\zh{难以相信、了不起、很特别。}}


\begin{exe}
\sn \textcolor{darkblue}{\textbf{\ipa{mɤ˧-ni˩\textasciitilde{}ni˩-hĩ˩-lɑ˩ ɲi˩-ze˩!}}}
\trans C'est absolument extraordinaire!
\trans \textit{\textcolor{brown}{\zh{这非常特别!}}}
\end{exe}
\begin{exe}
\sn \textcolor{darkblue}{\textbf{\ipa{ʈʂʰɯ˧ | mɤ˧-ni˩\textasciitilde{}ni˩-hĩ˩ | ɖɯ˧-v̩˧ ɲi˩!}}}
\trans C'est quelqu'un d'exceptionnel/d'extraordinaire!
\trans \textit{\textcolor{brown}{\zh{这是很利害的一个人!}}}
\end{exe}


\paragraph{\hspace{-0.5cm} {\Large \textcolor{darkblue}{\textbf{\ipa{mɤ˧ʈʂʰɤ˧}}}}} \hypertarget{m7\string_Mt`s`\string_h7\string_M1}{}
\markboth{\textcolor{darkblue}{\textbf{\ipa{mɤ˧ʈʂʰɤ˧}}}}{}
\textcolor{teal}{\mytextsc{nom}} \hspace{4pt} Ton~: M.

Charrette.
\textcolor{brown}{\zh{马车(汉语借词)。}}

 Emprunt~: chinois \textcolor{brown}{\zh{马车}}



\paragraph{\hspace{-0.5cm} {\Large \textcolor{darkblue}{\textbf{\ipa{mɤ˩}}}}} \hypertarget{m7\string_B1}{}
\markboth{\textcolor{darkblue}{\textbf{\ipa{mɤ˩}}}}{}
\textcolor{teal}{\mytextsc{nom}} \hspace{4pt} Ton~: L.

Huile animale, graisse.
\textcolor{brown}{\zh{动物油。}}


\begin{exe}
\sn \textcolor{darkblue}{\textbf{\ipa{njɤ˧ | mɤ˩ mɤ˩ dzɯ˩˥!}}}
\trans Je ne mange pas de graisse/de saindoux! (C'est là l'une des particularités de l'enquêteur)
\trans \textit{\textcolor{brown}{\zh{我不吃猪油!(这是调查者的特点之一)}}}
\end{exe}


\paragraph{\hspace{-0.5cm} {\Large \textcolor{darkblue}{\textbf{\ipa{mɤ˩\textsubscript{a}}}}}} \hypertarget{m7\string_Ba1}{}
\markboth{\textcolor{darkblue}{\textbf{\ipa{mɤ˩\textsubscript{a}}}}}{}
\textcolor{teal}{\mytextsc{classificateur}} \hspace{4pt} Ton~: L\textsubscript{a}.

Classificateur des petites quantités: quelques-uns, quelque peu de, un peu de.
\textcolor{brown}{\zh{量词:一些、一点。}}


\begin{exe}
\sn \textcolor{darkblue}{\textbf{\ipa{ɕi˧ɭɯ˧-ɻæ˩ | ɖɯ˧-mɤ˩}}}
\trans quelques graines de riz
\trans \textit{\textcolor{brown}{\zh{一些稻谷种子}}}
\end{exe}
\begin{exe}
\sn \textcolor{darkblue}{\textbf{\ipa{ɻæ˩˥ | ɖɯ˧-mɤ˩}}}
\trans quelques graines
\trans \textit{\textcolor{brown}{\zh{一些种子}}}
\end{exe}
\begin{exe}
\sn \textcolor{darkblue}{\textbf{\ipa{tsɑ˧bɤ˧ | ɖɯ˧-mɤ˩, | tsɑ˧bɤ˧ | ɲi˧-mɤ˩}}}
\trans un peu de farine; deux poignées/petites quantités de farine; etc.
\trans \textit{\textcolor{brown}{\zh{一小捧面粉、两小捧面粉……}}}
\end{exe}
\begin{exe}
\sn \textcolor{darkblue}{\textbf{\ipa{ʈʂʰɯ˧-mɤ˥}}}
\trans \mytextsc{dem} \string_ (tone: H\# / H\$)
\trans \textit{\textcolor{brown}{\zh{\mytextsc{指示代词} \string_}}}
\end{exe}


\paragraph{\hspace{-0.5cm} {\Large \textcolor{darkblue}{\textbf{\ipa{mɤ˩\textsubscript{b}}}}}} \hypertarget{m7\string_Bb1}{}
\markboth{\textcolor{darkblue}{\textbf{\ipa{mɤ˩\textsubscript{b}}}}}{}
\textcolor{teal}{\mytextsc{verbe}} \hspace{4pt} Ton~: L\textsubscript{b}.

Prendre dans la bouche un aliment en poudre.
\textcolor{brown}{\zh{将粉状的食品放在嘴里(如:干糌粑)。}}


\begin{exe}
\sn \textcolor{darkblue}{\textbf{\ipa{tsɑ˧bɤ˧ mɤ˩}}}
\trans manger du tsamba sec: on en prend une cuillère qu'on renverse dans sa bouche, et on laisse la farine s'imprégner de salive
\trans \textit{\textcolor{brown}{\zh{吃干糌粑}}}
\end{exe}
\begin{exe}
\sn \textcolor{darkblue}{\textbf{\ipa{tsɑ˧bɤ˧ | ɖɯ˧-mɤ˧\textasciitilde{}mɤ˩-ɻ̍˩}}}
\trans savourer un peu de tsamba
\trans \textit{\textcolor{brown}{\zh{品干糌粑、慢慢享受一点干糌粑}}}
\end{exe}


\paragraph{\hspace{-0.5cm} {\Large \textcolor{darkblue}{\textbf{\ipa{mɤ˩ɬi˩}}}}} \hypertarget{m7\string_BKi\string_B1}{}
\markboth{\textcolor{darkblue}{\textbf{\ipa{mɤ˩ɬi˩}}}}{}
\textcolor{teal}{\mytextsc{nom}} \hspace{4pt} Ton~: L.

Thé au beurre.
\textcolor{brown}{\zh{酥油茶。}}


 \mytextsc{clf}~: \textcolor{darkblue}{\textbf{\ipa{qʰwɤ˧˥}}} 

\paragraph{\hspace{-0.5cm} {\Large \textcolor{darkblue}{\textbf{\ipa{mɤ˩mv̩˩}}}}} \hypertarget{m7\string_Bmv\string_=\string_B1}{}
\markboth{\textcolor{darkblue}{\textbf{\ipa{mɤ˩mv̩˩}}}}{}
\textcolor{teal}{\mytextsc{nom}} \hspace{4pt} Ton~: L.

Porte-bougie: objet en cuivre dans lequel on verse de la paraffine fondue, ou de la graisse, et dans lequel on place une mèche; est utilisé dans les rituels.
\textcolor{brown}{\zh{烛台。}}


 \mytextsc{clf}~: \textcolor{darkblue}{\textbf{\ipa{qʰwɤ˧˥}}} 

\paragraph{\hspace{-0.5cm} {\Large \textcolor{darkblue}{\textbf{\ipa{mɤ˩tʰɑ˧}}}}} \hypertarget{m7\string_Bt\string_hA\string_M1}{}
\markboth{\textcolor{darkblue}{\textbf{\ipa{mɤ˩tʰɑ˧}}}}{}
\textcolor{teal}{\mytextsc{nom}} \hspace{4pt} Ton~: LM.

Confiserie au sésame.
\textcolor{brown}{\zh{麻糖(汉语借词)。}}

 Emprunt~: chinois \textcolor{brown}{\zh{麻糖}}



\paragraph{\hspace{-0.5cm} {\Large \textcolor{darkblue}{\textbf{\ipa{mi˧}}}}} \hypertarget{mi\string_M1}{}
\markboth{\textcolor{darkblue}{\textbf{\ipa{mi˧}}}}{}
\textcolor{teal}{\mytextsc{nom}} \hspace{4pt} Ton~: M.

Blessure, plaie.
\textcolor{brown}{\zh{伤口。}}


 \mytextsc{clf}~: \textcolor{darkblue}{\textbf{\ipa{kʰwɤ˥}}} 

\paragraph{\hspace{-0.5cm} {\Large \textcolor{darkblue}{\textbf{\ipa{mi˧kʰwɤ\#˥}}}}} \hypertarget{mi\string_Mk\string_hw7\#\string_T1}{}
\markboth{\textcolor{darkblue}{\textbf{\ipa{mi˧kʰwɤ\#˥}}}}{}
\textcolor{teal}{\mytextsc{nom}} \hspace{4pt} Ton~: \#H.
\ding{202} 
Blessure, plaie.
\textcolor{brown}{\zh{伤口。}}


\ding{203} 
Ulcère.
\textcolor{brown}{\zh{疮。}}


 \mytextsc{clf}~: \textcolor{darkblue}{\textbf{\ipa{kʰwɤ˥}}} 

\paragraph{\hspace{-0.5cm} {\Large \textcolor{darkblue}{\textbf{\ipa{mi˧ɬo\#˥}}}}} \hypertarget{mi\string_MKo\#\string_T1}{}
\markboth{\textcolor{darkblue}{\textbf{\ipa{mi˧ɬo\#˥}}}}{}
\textcolor{teal}{\mytextsc{nom}} \hspace{4pt} Ton~: \#H.

Prière.
\textcolor{brown}{\zh{祈祷。}}


\begin{exe}
\sn \textcolor{darkblue}{\textbf{\ipa{mi˧ɬo˧ lɑ˩}}}
\trans prier
\trans \textit{\textcolor{brown}{\zh{祈祷}}}
\end{exe}
 \mytextsc{clf}~: \textcolor{darkblue}{\textbf{\ipa{kʰwɤ˥}}} \textit{Voir~:} \hyperlink{Ko\string_M\string_T1}{\textcolor{darkblue}{\textbf{\ipa{ɬo˧˥}}}} 

\paragraph{\hspace{-0.5cm} {\Large \textcolor{darkblue}{\textbf{\ipa{mi˧mi˧}}}}} \hypertarget{mi\string_Mmi\string_M1}{}
\markboth{\textcolor{darkblue}{\textbf{\ipa{mi˧mi˧}}}}{}
\textcolor{teal}{\mytextsc{nom}} \hspace{4pt} Ton~: M.

Amande (d'un noyau).
\textcolor{brown}{\zh{核,仁。}}

 Emprunt~: chinois (dialectal)\textcolor{brown}{\zh{米米}}



\paragraph{\hspace{-0.5cm} {\Large \textcolor{darkblue}{\textbf{\ipa{mi˧pɤ\#˥}}}}} \hypertarget{mi\string_Mp7\#\string_T1}{}
\markboth{\textcolor{darkblue}{\textbf{\ipa{mi˧pɤ\#˥}}}}{}
\textcolor{teal}{\mytextsc{nom}} \hspace{4pt} Ton~: \#H.

Cicatrice.
\textcolor{brown}{\zh{疤。}}


 \mytextsc{clf}~: \textcolor{darkblue}{\textbf{\ipa{kʰwɤ˥}}} 

\paragraph{\hspace{-0.5cm} {\Large \textcolor{darkblue}{\textbf{\ipa{mi˧tʰv̩\#˥}}}}} \hypertarget{mi\string_Mt\string_hv\string_=\#\string_T1}{}
\markboth{\textcolor{darkblue}{\textbf{\ipa{mi˧tʰv̩\#˥}}}}{}
\textcolor{teal}{\mytextsc{nom}} \hspace{4pt} Ton~: \#H.

Bâton, canne pour marcher.
\textcolor{brown}{\zh{拐棍。}}


 \mytextsc{clf}~: \textcolor{darkblue}{\textbf{\ipa{kɤ˧˥}}} 

\paragraph{\hspace{-0.5cm} {\Large \textcolor{darkblue}{\textbf{\ipa{mi˩\textsubscript{a}}}}}} \hypertarget{mi\string_Ba1}{}
\markboth{\textcolor{darkblue}{\textbf{\ipa{mi˩\textsubscript{a}}}}}{}
\textcolor{teal}{\mytextsc{verbe}} \hspace{4pt} Ton~: L\textsubscript{a}.

Demander, quémander.
\textcolor{brown}{\zh{请求、要,讨饭。}}


\begin{exe}
\sn \textcolor{darkblue}{\textbf{\ipa{hɑ˧ mi˥}}}
\trans mendier (littéralement: 'demander à manger')
\trans \textit{\textcolor{brown}{\zh{讨饭}}}
\end{exe}
\begin{exe}
\sn \textcolor{darkblue}{\textbf{\ipa{hɑ˧ | ɖɯ˧-mi˧\textasciitilde{}mi˥-ɻ̍˩}}}
\trans mendier un peu
\trans \textit{\textcolor{brown}{\zh{讨点饭}}}
\end{exe}


\paragraph{\hspace{-0.5cm} {\Large \textcolor{darkblue}{\textbf{\ipa{mi˩\textsubscript{b}}}}}} \hypertarget{mi\string_Bb1}{}
\markboth{\textcolor{darkblue}{\textbf{\ipa{mi˩\textsubscript{b}}}}}{}
\textcolor{teal}{\mytextsc{classificateur}} \hspace{4pt} Ton~: L\textsubscript{b}.

Classificateur des petits animaux (poules…).
\textcolor{brown}{\zh{量词:小动物(一只)。}}


\begin{exe}
\sn \textcolor{darkblue}{\textbf{\ipa{ʈʂʰɯ˧-mi˧˥}}}
\trans cet animal
\trans \textit{\textcolor{brown}{\zh{这只}}}
\end{exe}


\paragraph{\hspace{-0.5cm} {\Large \textcolor{darkblue}{\textbf{\ipa{mi˩hwɑ˧}}}}} \hypertarget{mi\string_BhwA\string_M1}{}
\markboth{\textcolor{darkblue}{\textbf{\ipa{mi˩hwɑ˧}}}}{}
\textcolor{teal}{\mytextsc{nom}} \hspace{4pt} Ton~: LM.

Coton.
\textcolor{brown}{\zh{棉花(汉语借词)。}}

 Emprunt~: \textcolor{brown}{\zh{棉花}}

\begin{exe}
\sn \textcolor{darkblue}{\textbf{\ipa{mi˩hwɑ˧-bɑ˩lɑ˩}}}
\trans vêtement de coton
\trans \textit{\textcolor{brown}{\zh{棉布衣服}}}
\end{exe}


\paragraph{\hspace{-0.5cm} {\Large \textcolor{darkblue}{\textbf{\ipa{mi˩ɬi˩}}}}} \hypertarget{mi\string_BKi\string_B1}{}
\markboth{\textcolor{darkblue}{\textbf{\ipa{mi˩ɬi˩}}}}{}
\textcolor{teal}{\mytextsc{nom}} \hspace{4pt} Ton~: L.

Grand bambou.
\textcolor{brown}{\zh{大竹子。}}


\begin{exe}
\sn \textcolor{darkblue}{\textbf{\ipa{mi˩ɬi˩-bæ˩ʈʂo˥}}}
\trans balai en petites tiges de bambou
\trans \textit{\textcolor{brown}{\zh{竹扫帚}}}
\end{exe}
\begin{exe}
\sn \textcolor{darkblue}{\textbf{\ipa{mi˩ɬi˩-ʈʂæ˥do˩}}}
\trans seau en bambou pour porter de l'eau (sur le dos)
\trans \textit{\textcolor{brown}{\zh{竹桶,用来背水}}}
\end{exe}
 \mytextsc{clf}~: \textcolor{darkblue}{\textbf{\ipa{dzi˩}}} 

\paragraph{\hspace{-0.5cm} {\Large \textcolor{darkblue}{\textbf{\ipa{mi˩ɬi˩-ʁo˩bv̩˥}}}}} \hypertarget{mi\string_BKi\string_B-Ro\string_Bbv\string_=\string_T1}{}
\markboth{\textcolor{darkblue}{\textbf{\ipa{mi˩ɬi˩-ʁo˩bv̩˥}}}}{}
\textcolor{teal}{\mytextsc{nom}} \hspace{4pt} Ton~: L+H\#.

Pousse de bambou.
\textcolor{brown}{\zh{竹笋。}}


 \mytextsc{clf}~: \textcolor{darkblue}{\textbf{\ipa{kɤ˧˥}}} 

\paragraph{\hspace{-0.5cm} {\Large \textcolor{darkblue}{\textbf{\ipa{mi˩mo˩}}}}} \hypertarget{mi\string_Bmo\string_B1}{}
\markboth{\textcolor{darkblue}{\textbf{\ipa{mi˩mo˩}}}}{}
\textcolor{teal}{\mytextsc{nom}} \hspace{4pt} Ton~: L.

Petit crible.
\textcolor{brown}{\zh{小筛子。}}


 \mytextsc{clf}~: \textcolor{darkblue}{\textbf{\ipa{nɑ˧}}} 

\paragraph{\hspace{-0.5cm} {\Large \textcolor{darkblue}{\textbf{\ipa{mi˩pʰv̩˩}}}}} \hypertarget{mi\string_Bp\string_hv\string_=\string_B1}{}
\markboth{\textcolor{darkblue}{\textbf{\ipa{mi˩pʰv̩˩}}}}{}
\textcolor{teal}{\mytextsc{nom}} \hspace{4pt} Ton~: L.

Ortie.
\textcolor{brown}{\zh{荨麻。}}


 \mytextsc{clf}~: \textcolor{darkblue}{\textbf{\ipa{dzi˩}}} 

\paragraph{\hspace{-0.5cm} {\Large \textcolor{darkblue}{\textbf{\ipa{mi˩zɯ˩}}}}} \hypertarget{mi\string_BzM\string_B1}{}
\markboth{\textcolor{darkblue}{\textbf{\ipa{mi˩zɯ˩}}}}{}
\textcolor{teal}{\mytextsc{nom}} \hspace{4pt} Ton~: L.

Femme; aussi: nom du deuxième pilier de la maison (le pilier féminin).
\textcolor{brown}{\zh{女人。主屋的第二个柱子(代表女性的那个柱子)也是用这个名字。}}


 \mytextsc{clf}~: \textcolor{darkblue}{\textbf{\ipa{v̩˧}}} 

\paragraph{\hspace{-0.5cm} {\Large \textcolor{darkblue}{\textbf{\ipa{mi˧˥}}}}} \hypertarget{mi\string_M\string_T1}{}
\markboth{\textcolor{darkblue}{\textbf{\ipa{mi˧˥}}}}{}
\textcolor{teal}{\mytextsc{verbe}} \hspace{4pt} Ton~: MH.

Pousser.
\textcolor{brown}{\zh{推、拥挤。}}


\begin{exe}
\sn \textcolor{darkblue}{\textbf{\ipa{le˧-mi˧-ze˥}}}
\trans \mytextsc{accomp} \string_ \mytextsc{pfv}
\trans \textit{\textcolor{brown}{\zh{推开了}}}
\end{exe}
\begin{exe}
\sn \textcolor{darkblue}{\textbf{\ipa{le˧-mi˧˥}}}
\trans \mytextsc{accomp}
\trans \textit{\textcolor{brown}{\zh{推}}}
\end{exe}
\begin{exe}
\sn \textcolor{darkblue}{\textbf{\ipa{tʰi˧-mi˧˥}}}
\trans \mytextsc{dur}
\trans \textit{\textcolor{brown}{\zh{推}}}
\end{exe}
\begin{exe}
\sn \textcolor{darkblue}{\textbf{\ipa{tso˧\textasciitilde{}tso˧ mi˩}}}
\trans pousser quelque chose
\trans \textit{\textcolor{brown}{\zh{推开一个东西}}}
\end{exe}
\begin{exe}
\sn \textcolor{darkblue}{\textbf{\ipa{mi˩\textasciitilde{}mi˧˥}}}
\trans \mytextsc{red}
\trans \textit{\textcolor{brown}{\zh{\mytextsc{重叠:推、拥挤}}}}
\end{exe}
\begin{exe}
\sn \textcolor{darkblue}{\textbf{\ipa{mi˩\textasciitilde{}mi˧-ɻ̍˥}}}
\trans \mytextsc{red} \mytextsc{inchoatif}
\trans \textit{\textcolor{brown}{\zh{\mytextsc{重叠:推、拥挤}}}}
\end{exe}


\paragraph{\hspace{-0.5cm} {\Large \textcolor{darkblue}{\textbf{\ipa{-mi˩˧}}}}} \hypertarget{-mi\string_B\string_M1}{}
\markboth{\textcolor{darkblue}{\textbf{\ipa{-mi˩˧}}}}{}
\textcolor{teal}{\mytextsc{suffixe}} \hspace{4pt} Ton~: LM.
\ding{202} 
Suffixe féminin.
\textcolor{brown}{\zh{阴性后缀。}}


\ding{203} 
Suffixe augmentatif.
\textcolor{brown}{\zh{指大词。}}


 \mytextsc{clf}~: \textcolor{darkblue}{\textbf{\ipa{v̩˧}}} \textcolor{darkblue}{\textbf{\ipa{v̩˧}}} 

\paragraph{\hspace{-0.5cm} {\Large \textcolor{darkblue}{\textbf{\ipa{mi˩˧}}}}} \hypertarget{mi\string_B\string_M1}{}
\markboth{\textcolor{darkblue}{\textbf{\ipa{mi˩˧}}}}{}
\textcolor{teal}{\mytextsc{nom}} \hspace{4pt} Ton~: LM.

Femelle (animal femelle).
\textcolor{brown}{\zh{母的(动物)。}}


\begin{exe}
\sn \textcolor{darkblue}{\textbf{\ipa{ʈʂʰɯ˧, | mi˩˥! / ʈʂʰɯ˧, | mi˩ ɲi˥!}}}
\trans C'est une femelle!
\trans \textit{\textcolor{brown}{\zh{是母的!}}}
\end{exe}
 \mytextsc{clf}~: \textcolor{darkblue}{\textbf{\ipa{v̩˧}}} 

\paragraph{\hspace{-0.5cm} {\Large \textcolor{darkblue}{\textbf{\ipa{mje˧˥}}}}} \hypertarget{mje\string_M\string_T1}{}
\markboth{\textcolor{darkblue}{\textbf{\ipa{mje˧˥}}}}{}
\textcolor{teal}{\mytextsc{nom}} \hspace{4pt} Ton~: .

Nouilles, pâtes alimentaires.
\textcolor{brown}{\zh{面条。}}

 Emprunt~: chinois \textcolor{brown}{\zh{面}}

\begin{exe}
\sn \textcolor{darkblue}{\textbf{\ipa{mjæ˧˥ | dzɯ˧-bi˧! \textasciitilde{} mjæ˧ dzɯ˧-bi˧! \textasciitilde{} mjæ˧ dzɯ˥-bi˩!}}}
\trans On va manger des nouilles!
\trans \textit{\textcolor{brown}{\zh{吃面吧!}}}
\end{exe}
\begin{exe}
\sn \textcolor{darkblue}{\textbf{\ipa{mjæ˧˥ | ɖɯ˧-qʰwɤ˧ tɕɤ˥}}}
\trans faire cuire un bol de nouilles
\trans \textit{\textcolor{brown}{\zh{煮一碗面}}}
\end{exe}
\begin{exe}
\sn \textcolor{darkblue}{\textbf{\ipa{mjæ˧ hwæ˥-bi˩}}}
\trans (on) va acheter des nouilles
\trans \textit{\textcolor{brown}{\zh{要买面}}}
\end{exe}


\paragraph{\hspace{-0.5cm} {\Large \textcolor{darkblue}{\textbf{\ipa{mo˥\textsubscript{a}}}}}} \hypertarget{mo\string_Ta1}{}
\markboth{\textcolor{darkblue}{\textbf{\ipa{mo˥\textsubscript{a}}}}}{}
\textcolor{teal}{\mytextsc{classificateur}} \hspace{4pt} Ton~: H\textsubscript{a}.

Acre chinois: un sixième d'acre; 0,0667 hectare.
\textcolor{brown}{\zh{量词:地(一亩地)(汉语借词)。}}

 Emprunt~: chinois \textcolor{brown}{\zh{亩}}



\paragraph{\hspace{-0.5cm} {\Large \textcolor{darkblue}{\textbf{\ipa{mo˧}}}}} \hypertarget{mo\string_M1}{}
\markboth{\textcolor{darkblue}{\textbf{\ipa{mo˧}}}}{}
\textcolor{teal}{\mytextsc{adjectif}} \hspace{4pt} Ton~: .

Avide, glouton. S'applique à la nourriture, mais aussi à l'alcool, au tabac...
\textcolor{brown}{\zh{贪心不足,贪吃、贪喝酒、贪抽烟。}}


\begin{exe}
\sn \textcolor{darkblue}{\textbf{\ipa{hɑ˧ mo˧}}}
\trans avide de nourriture
\trans \textit{\textcolor{brown}{\zh{贪吃}}}
\end{exe}
\begin{exe}
\sn \textcolor{darkblue}{\textbf{\ipa{hɑ˧ mo˧ | ʐwæ˩˥}}}
\trans très avide de nourriture
\trans \textit{\textcolor{brown}{\zh{很贪吃}}}
\end{exe}
\begin{exe}
\sn \textcolor{darkblue}{\textbf{\ipa{mɤ˧-hɑ˧mo˧!}}}
\trans \mytextsc{neg}: pas avide
\trans \textit{\textcolor{brown}{\zh{不贪吃}}}
\end{exe}
\begin{exe}
\sn \textcolor{darkblue}{\textbf{\ipa{hɑ˧ | mɤ˧-mo˧!}}}
\trans \mytextsc{neg}: pas avide
\trans \textit{\textcolor{brown}{\zh{不贪吃}}}
\end{exe}


\paragraph{\hspace{-0.5cm} {\Large \textcolor{darkblue}{\textbf{\ipa{mo˧\textsubscript{a}}}}}} \hypertarget{mo\string_Ma1}{}
\markboth{\textcolor{darkblue}{\textbf{\ipa{mo˧\textsubscript{a}}}}}{}
\textcolor{teal}{\mytextsc{classificateur}} \hspace{4pt} Ton~: M\textsubscript{a}.

Classificateur des cadavres et tombeaux.
\textcolor{brown}{\zh{量词:尸体。}}




\paragraph{\hspace{-0.5cm} {\Large \textcolor{darkblue}{\textbf{\ipa{mo˧ɖʐv̩˥}}}}} \hypertarget{mo\string_Md`z`v\string_=\string_T1}{}
\markboth{\textcolor{darkblue}{\textbf{\ipa{mo˧ɖʐv̩˥}}}}{}
\textcolor{teal}{\mytextsc{nom}} \hspace{4pt} Ton~: H\#.

Morille: champignon comestible, particulièrement apprécié pour sa texture.
\textcolor{brown}{\zh{羊肚菌。}}Dialecte chinois local~: \zh{羊菌。}


\begin{exe}
\sn \textcolor{darkblue}{\textbf{\ipa{ʂɯ˧-ɬi˧mi˧, | mo˧ɖʐv̩˥!}}}
\trans Le septième mois, c'est la saison des morilles! (cette sorte de champignon) (Il pousse des paquets de champignons si compacts qu'on n'arrive même pas à les séparer.)
\trans \textit{\textcolor{brown}{\zh{七月份,是羊肚菌的季节!}}}
\end{exe}
\begin{exe}
\sn \textcolor{darkblue}{\textbf{\ipa{ʂɯ˧-ɬi˧mi˧ | mo˧ɖʐv̩˥-ne˩-ʝi˩-zo˩!}}}
\trans ``Il vous en vient comme des morilles au septième mois!" Commentaire humoristique: ce qu'on disait au sujet des gens qui avaient beaucoup d'enfants, qui avaient enfant après enfant: ``Ca prolifère comme les morilles au septième mois!"
\trans \textit{\textcolor{brown}{\zh{(你们家孩子)生得像七月份的羊肚菌一样!(来形容一家有很多孩子出生,一个又一个。在永宁地区,七月份羊肚菌很多。)}}}
\end{exe}


\paragraph{\hspace{-0.5cm} {\Large \textcolor{darkblue}{\textbf{\ipa{mo˧jo˩-mi˩}}}}} \hypertarget{mo\string_Mjo\string_B-mi\string_B1}{}
\markboth{\textcolor{darkblue}{\textbf{\ipa{mo˧jo˩-mi˩}}}}{}
\textcolor{teal}{\mytextsc{nom}} \hspace{4pt} Ton~: L\#-.

Chouette, hibou (toutes les espèces de \textit{bubo} et \textit{strix}).
\textcolor{brown}{\zh{猫头鹰。}}


 \mytextsc{clf}~: \textcolor{darkblue}{\textbf{\ipa{mi˩}}} 

\paragraph{\hspace{-0.5cm} {\Large \textcolor{darkblue}{\textbf{\ipa{mo˧jo˩mi˩-pʰv̩˩}}}}} \hypertarget{mo\string_Mjo\string_Bmi\string_B-p\string_hv\string_=\string_B1}{}
\markboth{\textcolor{darkblue}{\textbf{\ipa{mo˧jo˩mi˩-pʰv̩˩}}}}{}
\textcolor{teal}{\mytextsc{nom}} \hspace{4pt} Ton~: L\#-.

Hibou mâle.
\textcolor{brown}{\zh{公猫头鹰。}}


 \mytextsc{clf}~: \textcolor{darkblue}{\textbf{\ipa{mi˩}}} 

\paragraph{\hspace{-0.5cm} {\Large \textcolor{darkblue}{\textbf{\ipa{mo˧jo˩mi˩-zo˩}}}}} \hypertarget{mo\string_Mjo\string_Bmi\string_B-zo\string_B1}{}
\markboth{\textcolor{darkblue}{\textbf{\ipa{mo˧jo˩mi˩-zo˩}}}}{}
\textcolor{teal}{\mytextsc{nom}} \hspace{4pt} Ton~: L\#-.

Bébé hibou.
\textcolor{brown}{\zh{小的猫头鹰。}}


 \mytextsc{clf}~: \textcolor{darkblue}{\textbf{\ipa{ɭɯ˧}}} 

\paragraph{\hspace{-0.5cm} {\Large \textcolor{darkblue}{\textbf{\ipa{mo˧kɤ˩}}}}} \hypertarget{mo\string_Mk7\string_B1}{}
\markboth{\textcolor{darkblue}{\textbf{\ipa{mo˧kɤ˩}}}}{}
\textcolor{teal}{\mytextsc{nom}} \hspace{4pt} Ton~: L\#.

Azalée. Cette plante est perçue comme vénéneuse; on ne consomme pas les champignons qui poussent dans son voisinage.
\textcolor{brown}{\zh{杜鹃花、踯躅、山石榴、照山红、唐杜鹃。}}Dialecte chinois local~: \zh{杨花木。}


\begin{exe}
\sn \textcolor{darkblue}{\textbf{\ipa{mo˧kɤ˩-bæ˩bæ˩}}}
\trans fleurs d'azalée
\trans \textit{\textcolor{brown}{\zh{杜鹃花}}}
\end{exe}


\paragraph{\hspace{-0.5cm} {\Large \textcolor{darkblue}{\textbf{\ipa{mo˧ɬɑ˥}}}}} \hypertarget{mo\string_MKA\string_T1}{}
\markboth{\textcolor{darkblue}{\textbf{\ipa{mo˧ɬɑ˥}}}}{}
\textcolor{teal}{\mytextsc{nom}} \hspace{4pt} Ton~: H\#.

Médisance, calomnie.
\textcolor{brown}{\zh{诬蔑、坏话。}}


\begin{exe}
\sn \textcolor{darkblue}{\textbf{\ipa{mo˧ɬɑ˥ ʐwɤ˩}}}
\trans médire de quelqu'un, calomnier quelqu'un
\trans \textit{\textcolor{brown}{\zh{说人的坏话}}}
\end{exe}


\paragraph{\hspace{-0.5cm} {\Large \textcolor{darkblue}{\textbf{\ipa{mo˧mo˥}}}}} \hypertarget{mo\string_Mmo\string_T1}{}
\markboth{\textcolor{darkblue}{\textbf{\ipa{mo˧mo˥}}}}{}
\textcolor{teal}{\mytextsc{nom}} \hspace{4pt} Ton~: H\#.

Petits pains (pouvant contenir de la farine de maïs; mais surtout farine de blé) cuits à la vapeur.
\textcolor{brown}{\zh{馒头、包子。}}


 \mytextsc{clf}~: \textcolor{darkblue}{\textbf{\ipa{ɭɯ˧}}} 

\paragraph{\hspace{-0.5cm} {\Large \textcolor{darkblue}{\textbf{\ipa{mo˧nɑ˥}}} \textsubscript{1}}} \hypertarget{mo\string_MnA\string_T1}{}
\markboth{\textcolor{darkblue}{\textbf{\ipa{mo˧nɑ˥}}} \textsubscript{1}}{}
\textcolor{teal}{\mytextsc{nom}} \hspace{4pt} Ton~: H\#.

Médisance, ragot.
\textcolor{brown}{\zh{闲话。}}


\begin{exe}
\sn \textcolor{darkblue}{\textbf{\ipa{mo˧nɑ˥ ʐwɤ˩}}}
\trans ragoter, médire
\trans \textit{\textcolor{brown}{\zh{八卦、讲别人的坏话}}}
\end{exe}
\begin{exe}
\sn \textcolor{darkblue}{\textbf{\ipa{ʈʂʰɯ˧ | ɖɯ˧-ɲi˥ | mo˧nɑ˥ ʐwɤ˩-dʑo˩!}}}
\trans Il/elle ragote toute la journée!
\trans \textit{\textcolor{brown}{\zh{他一天到晚都在八卦!}}}
\end{exe}
\begin{exe}
\sn \textcolor{darkblue}{\textbf{\ipa{mo˧nɑ˥-ɕi˩mi˩}}}
\trans même sens: ragot, médisance
\trans \textit{\textcolor{brown}{\zh{同上:八卦、坏话}}}
\end{exe}
\begin{exe}
\sn \textcolor{darkblue}{\textbf{\ipa{mo˧nɑ˥-ɕi˩mi˩ ʐwɤ˩}}}
\trans ragoter, médire
\trans \textit{\textcolor{brown}{\zh{八卦、讲别人的坏话}}}
\end{exe}
\begin{exe}
\sn \textcolor{darkblue}{\textbf{\ipa{hĩ˧ | ʈʂʰɯ˧-v̩˧, | mo˧nɑ˥-ɕi˩mi˩ | ɖɯ˧-v̩˧ ɲi˩!}}}
\trans C'est un ragoteur, il est médisant
\trans \textit{\textcolor{brown}{\zh{他爱八卦、爱说别人坏话}}}
\end{exe}


\paragraph{\hspace{-0.5cm} {\Large \textcolor{darkblue}{\textbf{\ipa{mo˧nɑ˥}}} \textsubscript{2}}} \hypertarget{mo\string_MnA\string_T2}{}
\markboth{\textcolor{darkblue}{\textbf{\ipa{mo˧nɑ˥}}} \textsubscript{2}}{}
\textcolor{teal}{\mytextsc{nom}} \hspace{4pt} Ton~: H\#.

Paille hachée, utilisée dans la préparation des légumes en saumure.
\textcolor{brown}{\zh{剁成粉的秸杆。}}


\begin{exe}
\sn \textcolor{darkblue}{\textbf{\ipa{mv˩zɯ˩-mo˩nɑ˥}}}
\trans paille d'avoine hachée
\trans \textit{\textcolor{brown}{\zh{剁成粉的燕麦秸杆}}}
\end{exe}


\paragraph{\hspace{-0.5cm} {\Large \textcolor{darkblue}{\textbf{\ipa{mo˧qʰwɤ˥}}}}} \hypertarget{mo\string_Mq\string_hw7\string_T1}{}
\markboth{\textcolor{darkblue}{\textbf{\ipa{mo˧qʰwɤ˥}}}}{}
\textcolor{teal}{\mytextsc{nom}} \hspace{4pt} Ton~: H\#.

Navette en bois d'un pont de corde: la navette coulisse sur la corde; passager, cheval ou chargement y sont attachés.
\textcolor{brown}{\zh{溜索上往返移动的木头梭。}}


 \mytextsc{clf}~: \textcolor{darkblue}{\textbf{\ipa{ɭɯ˧}}} 

\paragraph{\hspace{-0.5cm} {\Large \textcolor{darkblue}{\textbf{\ipa{mo˧qʰwɤ˧˥}}}}} \hypertarget{mo\string_Mq\string_hw7\string_M\string_T1}{}
\markboth{\textcolor{darkblue}{\textbf{\ipa{mo˧qʰwɤ˧˥}}}}{}
\textcolor{teal}{\mytextsc{adjectif}} \hspace{4pt} Ton~: MH.

Qui a bon appétit, qui a un solide appétit; gourmand, vorace (peut être neutre, ou franchement négatif).
\textcolor{brown}{\zh{胃口好,或:贪吃。}}




\paragraph{\hspace{-0.5cm} {\Large \textcolor{darkblue}{\textbf{\ipa{mo˩}}}}} \hypertarget{mo\string_B1}{}
\markboth{\textcolor{darkblue}{\textbf{\ipa{mo˩}}}}{}
\textcolor{teal}{\mytextsc{particule}} \textcolor{teal}{\mytextsc{de}} \textcolor{teal}{\mytextsc{discours}} \hspace{4pt} Ton~: L.

Particule indiquant l'invitation à faire quelque chose.
\textcolor{brown}{\zh{句尾助词:请……。}}


\begin{exe}
\sn \textcolor{darkblue}{\textbf{\ipa{no˧ | ɖɯ˧-ʈʰɯ˩-ɻ̍˩ mo˩!}}}
\trans Bois donc un peu!
\trans \textit{\textcolor{brown}{\zh{请你喝一点!}}}
\end{exe}


\paragraph{\hspace{-0.5cm} {\Large \textcolor{darkblue}{\textbf{\ipa{mo˩\textsubscript{a}}}} \textsubscript{1}}} \hypertarget{mo\string_Ba1}{}
\markboth{\textcolor{darkblue}{\textbf{\ipa{mo˩\textsubscript{a}}}} \textsubscript{1}}{}
\textcolor{teal}{\mytextsc{adjectif}} \hspace{4pt} Ton~: L\textsubscript{a}.

Vieux, âgé.
\textcolor{brown}{\zh{年老。}}


\begin{exe}
\sn \textcolor{darkblue}{\textbf{\ipa{mo˩ hĩ˩˥}}}
\trans vieille personne
\trans \textit{\textcolor{brown}{\zh{老人}}}
\end{exe}
\begin{exe}
\sn \textcolor{darkblue}{\textbf{\ipa{si˧ mo˥}}}
\trans vieux bois, vieil arbre
\trans \textit{\textcolor{brown}{\zh{老树、老木头}}}
\end{exe}
\begin{exe}
\sn \textcolor{darkblue}{\textbf{\ipa{le˧-mo˩-ze˩}}}
\trans \mytextsc{accomp} \string_ \mytextsc{pfv}: (il/elle) a vieilli
\trans \textit{\textcolor{brown}{\zh{\mytextsc{accomp} \string_ \mytextsc{pfv}}}}
\end{exe}
\begin{exe}
\sn \textcolor{darkblue}{\textbf{\ipa{le˧-mo˩-hĩ˩}}}
\trans vieille personne, personne qui a vieilli
\trans \textit{\textcolor{brown}{\zh{老了的人}}}
\end{exe}
\begin{exe}
\sn \textcolor{darkblue}{\textbf{\ipa{hĩ˧ mo˥, | õ˧-di˧ fv̩˥! | ʐwæ˧ mo˥, | to˩ do˩ ɖwæ˥!}}}
\trans ``Les vieilles personnes aiment leur chez-eux; les vieux chevaux ont peur de grimper les pentes!" (Sens: avec l'âge, on devient moins entreprenant.)
\trans \textit{\textcolor{brown}{\zh{老人爱自家,老马怕山坡!(谚语,描写不爱到处跑的老年人)}}}
\end{exe}
\begin{exe}
\sn \textcolor{darkblue}{\textbf{\ipa{lv̩˧ mo˥ F | dʑɯ˧ | le˧-qv̩˩; | si˧ mo˥ F | le˧-dze˩-kv̩˩! | no˧ F | ə˧tse˧ | le˧-ʂɯ˧-mɤ˧-tʰɑ˧˥ | di˩!}}}
\trans Les vieilles pierres, le courant les emporte; le vieux bois, on le coupe! Alors pourquoi toi te voilà qui ne veux pas mourir! (Moquerie à l'égard d'une personne très âgée.)
\trans \textit{\textcolor{brown}{\zh{老石头要被河流冲走,老木头要被砍掉。你呢,怎么还不死? (嘲笑一个年龄很高的人。摩梭传统中,人的寿命是六十岁:过了七十岁的人,被认为是已经到达了命的尽头。)}}}
\end{exe}


\paragraph{\hspace{-0.5cm} {\Large \textcolor{darkblue}{\textbf{\ipa{mo˩\textsubscript{a}}}} \textsubscript{2}}} \hypertarget{mo\string_Ba2}{}
\markboth{\textcolor{darkblue}{\textbf{\ipa{mo˩\textsubscript{a}}}} \textsubscript{2}}{}
\textcolor{teal}{\mytextsc{verbe}} \hspace{4pt} Ton~: L\textsubscript{a}.

Mourir.
\textcolor{brown}{\zh{死、去世。}}


\begin{exe}
\sn \textcolor{darkblue}{\textbf{\ipa{mɤ˧-mo˩-sɯ˩!}}}
\trans (Il n'est) pas encore mort!
\trans \textit{\textcolor{brown}{\zh{还没死!}}}
\end{exe}
\begin{exe}
\sn \textcolor{darkblue}{\textbf{\ipa{si˧ mo˩}}}
\trans bois mort
\trans \textit{\textcolor{brown}{\zh{老干柴(直译:死了的木头)}}}
\end{exe}


\paragraph{\hspace{-0.5cm} {\Large \textcolor{darkblue}{\textbf{\ipa{mo˩kv̩\#˥}}}}} \hypertarget{mo\string_Bkv\string_=\#\string_T1}{}
\markboth{\textcolor{darkblue}{\textbf{\ipa{mo˩kv̩\#˥}}}}{}
\textcolor{teal}{\mytextsc{nom}} \hspace{4pt} Ton~: LM+\#H.

Sorte de champignon qui pousse sur les chênes (sur les troncs tombés, sur les arbres morts).
\textcolor{brown}{\zh{蘑菇:长在倒在地上的树(如青冈等树木)上的菌子(汉语借词)。}}

 Emprunt~: chinois \textcolor{brown}{\zh{蘑菇}}

\begin{exe}
\sn \textcolor{darkblue}{\textbf{\ipa{mo˩kv̩˥, | si˧dzi˩-mo˩!}}}
\trans \textcolor{darkblue}{\textbf{\ipa{/mo˩kv̩\#˥/}}}, ça désigne les champignons qui pousse sur les arbres! (littéralement: ``les champignons d'arbres", par opposition aux ``champignons de terre")
\trans \textit{\textcolor{brown}{\zh{\textcolor{darkblue}{\textbf{\ipa{/mo˩kv̩\#˥/}}},指的是长在(倒在地上的)树上的菌子!}}}
\end{exe}


\paragraph{\hspace{-0.5cm} {\Large \textcolor{darkblue}{\textbf{\ipa{mo˩ɻ\#˥}}}}} \hypertarget{mo\string_Br£`\#\string_T1}{}
\markboth{\textcolor{darkblue}{\textbf{\ipa{mo˩ɻ\#˥}}}}{}
\textcolor{teal}{\mytextsc{nom}} \hspace{4pt} Ton~: LM+\#H.

Champignon noir.
\textcolor{brown}{\zh{木耳(汉语借词)。}}

 Emprunt~: chinois \textcolor{brown}{\zh{木耳}}



\paragraph{\hspace{-0.5cm} {\Large \textcolor{darkblue}{\textbf{\ipa{mo˩zo\#˥}}}}} \hypertarget{mo\string_Bzo\#\string_T1}{}
\markboth{\textcolor{darkblue}{\textbf{\ipa{mo˩zo\#˥}}}}{}
\textcolor{teal}{\mytextsc{nom}} \hspace{4pt} Ton~: LM+\#H.

Militaire, soldat.
\textcolor{brown}{\zh{士兵。}}


\begin{exe}
\sn \textcolor{darkblue}{\textbf{\ipa{mo˩zo˧ ʝi˧-hɯ˧ ɲi˥!}}}
\trans Il est parti à l'armée! / Il s'est fait soldat!
\trans \textit{\textcolor{brown}{\zh{当兵去了!}}}
\end{exe}
 \mytextsc{clf}~: \textcolor{darkblue}{\textbf{\ipa{v̩˧}}} 

\paragraph{\hspace{-0.5cm} {\Large \textcolor{darkblue}{\textbf{\ipa{mo˧˥}}}}} \hypertarget{mo\string_M\string_T1}{}
\markboth{\textcolor{darkblue}{\textbf{\ipa{mo˧˥}}}}{}
\textcolor{teal}{\mytextsc{nom}} \hspace{4pt} Ton~: MH.

Champignon.
\textcolor{brown}{\zh{菌子、蘑菇。}}


 \mytextsc{clf}~: \textcolor{darkblue}{\textbf{\ipa{ɭɯ˧}}} \textcolor{darkblue}{\textbf{\ipa{mo˧˥}}} \textit{Voir~:} \hyperlink{mo\string_M\string_Ta1}{\textcolor{darkblue}{\textbf{\ipa{mo˧˥a}}}} 

\paragraph{\hspace{-0.5cm} {\Large \textcolor{darkblue}{\textbf{\ipa{mo˧˥\textsubscript{a}}}}}} \hypertarget{mo\string_M\string_Ta1}{}
\markboth{\textcolor{darkblue}{\textbf{\ipa{mo˧˥\textsubscript{a}}}}}{}
\textcolor{teal}{\mytextsc{classificateur}} \hspace{4pt} Ton~: MH\textsubscript{a}.

Auto-classificateur des champignons.
\textcolor{brown}{\zh{量词:蘑菇(一只)。}}


\textit{Voir~:} \hyperlink{mo\string_M\string_T1}{\textcolor{darkblue}{\textbf{\ipa{mo˧˥}}}} 

\paragraph{\hspace{-0.5cm} {\Large \textcolor{darkblue}{\textbf{\ipa{mv̩˩˥}}}}} \hypertarget{mv\string_=\string_B\string_T1}{}
\markboth{\textcolor{darkblue}{\textbf{\ipa{mv̩˩˥}}}}{}
\textcolor{teal}{\mytextsc{nom}} \hspace{4pt} Ton~: LH.

Fille.
\textcolor{brown}{\zh{女儿。}}


 \mytextsc{clf}~: \textcolor{darkblue}{\textbf{\ipa{v̩˧}}} 

\paragraph{\hspace{-0.5cm} {\Large \textcolor{darkblue}{\textbf{\ipa{mv̩˧}}}}} \hypertarget{mv\string_=\string_M1}{}
\markboth{\textcolor{darkblue}{\textbf{\ipa{mv̩˧}}}}{}
\textcolor{teal}{\mytextsc{nom}} \hspace{4pt} Ton~: M.

Nom (nom de famille ou prénom: nom donné à un individu).
\textcolor{brown}{\zh{姓名。}}


\begin{exe}
\sn \textcolor{darkblue}{\textbf{\ipa{ɑ˩ʁo˧-bv̩˧ | mv̩˧ (+ɲi˩)}}}
\trans c'est le nom de la famille / c'est mon nom de famille!
\trans \textit{\textcolor{brown}{\zh{这是家里的姓! / 这是我家的姓!}}}
\end{exe}
\begin{exe}
\sn \textcolor{darkblue}{\textbf{\ipa{njɤ˧ | mv̩˧ ɖɯ˧-kʰwɤ˥ | ʂe˧-zo˧-ho˩!}}}
\trans Il va falloir que j'aille chercher un nom (auprès des moines du monastère) (pour un enfant qui vient de naître)!
\trans \textit{\textcolor{brown}{\zh{我得去(向大寺里的和尚)求一个名字(给刚出生的孩子起名)}}}
\end{exe}


\paragraph{\hspace{-0.5cm} {\Large \textcolor{darkblue}{\textbf{\ipa{mv̩˧}}}}} \hypertarget{mv\string_=\string_M1}{}
\markboth{\textcolor{darkblue}{\textbf{\ipa{mv̩˧}}}}{}
\textcolor{teal}{\mytextsc{particule}} \textcolor{teal}{\mytextsc{de}} \textcolor{teal}{\mytextsc{discours}} \hspace{4pt} Ton~: M.

Particule finale affirmative.
\textcolor{brown}{\zh{句尾助词,表示肯定:“嘛”。}}




\paragraph{\hspace{-0.5cm} {\Large \textcolor{darkblue}{\textbf{\ipa{mv̩˥}}}}} \hypertarget{mv\string_=\string_T1}{}
\markboth{\textcolor{darkblue}{\textbf{\ipa{mv̩˥}}}}{}
\textcolor{teal}{\mytextsc{verbe}} \hspace{4pt} Ton~: H.
\ding{202} 
Entendre.
\textcolor{brown}{\zh{懂,听见。}}


\begin{exe}
\sn \textcolor{darkblue}{\textbf{\ipa{njɤ˧ | le˧-mv̩˥-ze˩}}}
\trans j'ai entendu
\trans \textit{\textcolor{brown}{\zh{我听见了}}}
\end{exe}
\ding{203} 
Comprendre.
\textcolor{brown}{\zh{懂。}}


\begin{exe}
\sn \textcolor{darkblue}{\textbf{\ipa{njɤ˧ | le˧-mv̩˥-ze˩}}}
\trans j'ai compris
\trans \textit{\textcolor{brown}{\zh{我懂了}}}
\end{exe}


\paragraph{\hspace{-0.5cm} {\Large \textcolor{darkblue}{\textbf{\ipa{mv̩˥}}} \textsubscript{1}}} \hypertarget{mv\string_=\string_T1}{}
\markboth{\textcolor{darkblue}{\textbf{\ipa{mv̩˥}}} \textsubscript{1}}{}
\textcolor{teal}{\mytextsc{nom}} \hspace{4pt} Ton~: \#H.

Ciel.
\textcolor{brown}{\zh{天。}}


\begin{exe}
\sn \textcolor{darkblue}{\textbf{\ipa{mv̩˧tʰv̩˧(-ze˩)}}}
\trans il fait clair, il fait grand jour, le ciel est clair
\trans \textit{\textcolor{brown}{\zh{天晴,天色亮}}}
\end{exe}
\begin{exe}
\sn \textcolor{darkblue}{\textbf{\ipa{hĩ˧-ɳɯ˩ mɤ˩-do˩, | mv̩˧-ɳɯ˩ | do˩˥!}}}
\trans ``Ce que les hommes ne voient pas, le ciel le voit!" (Sens: une bonne action n'est jamais perdue, et une mauvaise reçoit sa punition dans le monde d'en haut.)
\trans \textit{\textcolor{brown}{\zh{“人看不见的,老天能看见!”}}}
\end{exe}
 \mytextsc{clf}~: \textcolor{darkblue}{\textbf{\ipa{ɭɯ˧}}} 

\paragraph{\hspace{-0.5cm} {\Large \textcolor{darkblue}{\textbf{\ipa{mv̩˥}}} \textsubscript{2}}} \hypertarget{mv\string_=\string_T2}{}
\markboth{\textcolor{darkblue}{\textbf{\ipa{mv̩˥}}} \textsubscript{2}}{}
\textcolor{teal}{\mytextsc{nom}} \hspace{4pt} Ton~: \#H.

Feu.
\textcolor{brown}{\zh{火。}}


\begin{exe}
\sn \textcolor{darkblue}{\textbf{\ipa{mv̩˧ kʰɯ˩}}}
\trans allumer un feu, faire un feu
\trans \textit{\textcolor{brown}{\zh{点火}}}
\end{exe}
 \mytextsc{clf}~: \textcolor{darkblue}{\textbf{\ipa{æ̃˩}}} 

\paragraph{\hspace{-0.5cm} {\Large \textcolor{darkblue}{\textbf{\ipa{mv̩˩\textsubscript{a}}}} \textsubscript{1}}} \hypertarget{mv\string_=\string_Ba1}{}
\markboth{\textcolor{darkblue}{\textbf{\ipa{mv̩˩\textsubscript{a}}}} \textsubscript{1}}{}
\textcolor{teal}{\mytextsc{verbe}} \hspace{4pt} Ton~: L\textsubscript{a}.

Souffler (ex.: souffler sur le feu, attiser le feu; souffler dans un instrument à vent).
\textcolor{brown}{\zh{吹(灰,乐器)。}}


\begin{exe}
\sn \textcolor{darkblue}{\textbf{\ipa{mv̩˧\textasciitilde{}mv̩˥(-ze˩)}}}
\trans \mytextsc{red}
\trans \textit{\textcolor{brown}{\zh{\mytextsc{重叠:吹吹}}}}
\end{exe}
\begin{exe}
\sn \textcolor{darkblue}{\textbf{\ipa{ʝi˧qʰv̩˧ mv̩˥}}}
\trans souffler dans une corne
\trans \textit{\textcolor{brown}{\zh{吹号角}}}
\end{exe}


\paragraph{\hspace{-0.5cm} {\Large \textcolor{darkblue}{\textbf{\ipa{mv̩˩\textsubscript{a}}}} \textsubscript{2}}} \hypertarget{mv\string_=\string_Ba2}{}
\markboth{\textcolor{darkblue}{\textbf{\ipa{mv̩˩\textsubscript{a}}}} \textsubscript{2}}{}
\textcolor{teal}{\mytextsc{verbe}} \hspace{4pt} Ton~: L\textsubscript{a}.

Emporter (le courant emporte un nageur), balayer (une vague balaie une épave de bateau).
\textcolor{brown}{\zh{冲(走)。}}




\paragraph{\hspace{-0.5cm} {\Large \textcolor{darkblue}{\textbf{\ipa{mv̩˩\textsubscript{a}}}} \textsubscript{3}}} \hypertarget{mv\string_=\string_Ba3}{}
\markboth{\textcolor{darkblue}{\textbf{\ipa{mv̩˩\textsubscript{a}}}} \textsubscript{3}}{}
\textcolor{teal}{\mytextsc{adjectif}} \hspace{4pt} Ton~: L\textsubscript{a}.
\ding{202} 
Mûr (produit agricole).
\textcolor{brown}{\zh{熟、成熟(植物、水果)。}}


\begin{exe}
\sn \textcolor{darkblue}{\textbf{\ipa{mv̩˩-hĩ˩˥}}}
\trans \mytextsc{rel}
\trans \textit{\textcolor{brown}{\zh{熟的}}}
\end{exe}
\ding{203} 
Cuit (aliment).
\textcolor{brown}{\zh{熟(食物)。}}




\paragraph{\hspace{-0.5cm} {\Large \textcolor{darkblue}{\textbf{\ipa{mv̩˩\textsubscript{a}}}} \textsubscript{4}}} \hypertarget{mv\string_=\string_Ba4}{}
\markboth{\textcolor{darkblue}{\textbf{\ipa{mv̩˩\textsubscript{a}}}} \textsubscript{4}}{}
\textcolor{teal}{\mytextsc{verbe}} \hspace{4pt} Ton~: L\textsubscript{a}.

Brûler, se consumer (ex.: un corps sur le bûcher).
\textcolor{brown}{\zh{燃烧。}}




\paragraph{\hspace{-0.5cm} {\Large \textcolor{darkblue}{\textbf{\ipa{mv̩˧\textsubscript{a}}}}}} \hypertarget{mv\string_=\string_Ma1}{}
\markboth{\textcolor{darkblue}{\textbf{\ipa{mv̩˧\textsubscript{a}}}}}{}
\textcolor{teal}{\mytextsc{verbe}} \hspace{4pt} Ton~: M\textsubscript{a}.

Mettre, porter, enfiler, endosser (une chemise, une veste); se vêtir d'un habit.
\textcolor{brown}{\zh{穿(衣服、上衣)。}}


\begin{exe}
\sn \textcolor{darkblue}{\textbf{\ipa{bɑ˩lɑ˩ mv̩˥}}}
\trans mettre une chemise/veste
\trans \textit{\textcolor{brown}{\zh{穿衣服}}}
\end{exe}
\begin{exe}
\sn \textcolor{darkblue}{\textbf{\ipa{bɑ˩lɑ˩˥ | tʰi˧-mv̩˧}}}
\trans mettre une chemise/veste
\trans \textit{\textcolor{brown}{\zh{穿衣服}}}
\end{exe}
\begin{exe}
\sn \textcolor{darkblue}{\textbf{\ipa{dʑi˧hṽ˧ mv̩˩}}}
\trans enfiler un habit
\trans \textit{\textcolor{brown}{\zh{穿衣服}}}
\end{exe}
\textit{Voir~:} \hyperlink{ki\string_Ba1}{\textcolor{darkblue}{\textbf{\ipa{ki˩a}}}} 

\paragraph{\hspace{-0.5cm} {\Large \textcolor{darkblue}{\textbf{\ipa{mv̩˩-bæ˧mi˩}}}}} \hypertarget{mv\string_=\string_B-b\{\string_Mmi\string_B1}{}
\markboth{\textcolor{darkblue}{\textbf{\ipa{mv̩˩-bæ˧mi˩}}}}{}
\textcolor{teal}{\mytextsc{nom}} \hspace{4pt} Ton~: L-L\#.

Imbécile, idiote.
\textcolor{brown}{\zh{傻女人、笨女人。}}


 \mytextsc{clf}~: \textcolor{darkblue}{\textbf{\ipa{v̩˧}}} 

\paragraph{\hspace{-0.5cm} {\Large \textcolor{darkblue}{\textbf{\ipa{mv̩˧bɤ\#˥}}}}} \hypertarget{mv\string_=\string_Mb7\#\string_T1}{}
\markboth{\textcolor{darkblue}{\textbf{\ipa{mv̩˧bɤ\#˥}}}}{}
\textcolor{teal}{\mytextsc{nom}} \hspace{4pt} Ton~: \#H.

Plante du pied.
\textcolor{brown}{\zh{脚底。}}


 \mytextsc{clf}~: \textcolor{darkblue}{\textbf{\ipa{kʰwɤ˥}}} 

\paragraph{\hspace{-0.5cm} {\Large \textcolor{darkblue}{\textbf{\ipa{mv̩˧bv̩˧ʐv̩˥}}}}} \hypertarget{mv\string_=\string_Mbv\string_=\string_Mz`v\string_=\string_T1}{}
\markboth{\textcolor{darkblue}{\textbf{\ipa{mv̩˧bv̩˧ʐv̩˥}}}}{}
\textcolor{teal}{\mytextsc{nom}} \hspace{4pt} Ton~: H\#.

Dragon.
\textcolor{brown}{\zh{龙。}}


 \mytextsc{clf}~: \textcolor{darkblue}{\textbf{\ipa{mi˩}}} 

\paragraph{\hspace{-0.5cm} {\Large \textcolor{darkblue}{\textbf{\ipa{mv̩˧ɕi˥}}}}} \hypertarget{mv\string_=\string_Ms£i\string_T1}{}
\markboth{\textcolor{darkblue}{\textbf{\ipa{mv̩˧ɕi˥}}}}{}
\textcolor{teal}{\mytextsc{nom}} \hspace{4pt} Ton~: H\#.

Étincelle.
\textcolor{brown}{\zh{火花。}}


 \mytextsc{clf}~: \textcolor{darkblue}{\textbf{\ipa{æ̃˩}}} 

\paragraph{\hspace{-0.5cm} {\Large \textcolor{darkblue}{\textbf{\ipa{mv̩˧ɕi˥dʑɯ˩ʈʰɯ˩}}}}} \hypertarget{mv\string_=\string_Ms£i\string_Tdz£M\string_Bt`\string_hM\string_B1}{}
\markboth{\textcolor{darkblue}{\textbf{\ipa{mv̩˧ɕi˥dʑɯ˩ʈʰɯ˩}}}}{}
\textcolor{teal}{\mytextsc{nom}} \hspace{4pt} Ton~: H\#-L.

Arc-en-ciel.
\textcolor{brown}{\zh{彩虹。}}


 \mytextsc{clf}~: \textcolor{darkblue}{\textbf{\ipa{kʰɯ˩}}} on peut également dire simplement: mv̩˧ɕi˥

\paragraph{\hspace{-0.5cm} {\Large \textcolor{darkblue}{\textbf{\ipa{mv̩˩ɖæ˧}}}}} \hypertarget{mv\string_=\string_Bd`\{\string_M1}{}
\markboth{\textcolor{darkblue}{\textbf{\ipa{mv̩˩ɖæ˧}}}}{}
\textcolor{teal}{\mytextsc{nom}} \hspace{4pt} Ton~: LM.

Le bas du corps.
\textcolor{brown}{\zh{下半身。}}




\paragraph{\hspace{-0.5cm} {\Large \textcolor{darkblue}{\textbf{\ipa{mv̩˧di˧˥}}}}} \hypertarget{mv\string_=\string_Mdi\string_M\string_T1}{}
\markboth{\textcolor{darkblue}{\textbf{\ipa{mv̩˧di˧˥}}}}{}
\textcolor{teal}{\mytextsc{nom}} \hspace{4pt} Ton~: MH\#.
\ding{202} 
Champs (quel que soit ce qu'on y cultive).
\textcolor{brown}{\zh{田地。}}


\ding{203} 
La Terre, là où habitent les hommes (par opposition au ciel).
\textcolor{brown}{\zh{天下。}}


 \mytextsc{clf}~: \textcolor{darkblue}{\textbf{\ipa{kɤ˧˥}}} 

\paragraph{\hspace{-0.5cm} {\Large \textcolor{darkblue}{\textbf{\ipa{mv̩˩do˩}}}}} \hypertarget{mv\string_=\string_Bdo\string_B1}{}
\markboth{\textcolor{darkblue}{\textbf{\ipa{mv̩˩do˩}}}}{}
\textcolor{teal}{\mytextsc{verbe}} \hspace{4pt} Ton~: L.

Demander.
\textcolor{brown}{\zh{问。}}


\begin{exe}
\sn \textcolor{darkblue}{\textbf{\ipa{le˧-mv̩˩do˩}}}
\trans \mytextsc{accomp}
\trans \textit{\textcolor{brown}{\zh{\mytextsc{accomp}}}}
\end{exe}
\begin{exe}
\sn \textcolor{darkblue}{\textbf{\ipa{mv̩˩do˩-ze˥}}}
\trans \mytextsc{pfv}
\trans \textit{\textcolor{brown}{\zh{问了}}}
\end{exe}
\begin{exe}
\sn \textcolor{darkblue}{\textbf{\ipa{ə˧tso˧ mv̩˩do˩-bi˩? |}}}
\trans qu'est-ce que (tu) vas demander?
\trans \textit{\textcolor{brown}{\zh{要问什么呢?}}}
\end{exe}


\paragraph{\hspace{-0.5cm} {\Large \textcolor{darkblue}{\textbf{\ipa{mv̩˩ɖɯ˩}}}}} \hypertarget{mv\string_=\string_Bd`M\string_B1}{}
\markboth{\textcolor{darkblue}{\textbf{\ipa{mv̩˩ɖɯ˩}}}}{}
\textcolor{teal}{\mytextsc{nom}} \hspace{4pt} Ton~: L.

Fille aînée.
\textcolor{brown}{\zh{大女儿。}}


\begin{exe}
\sn \textcolor{darkblue}{\textbf{\ipa{zo˧ɖɯ˧-mv̩˥ɖɯ˩}}}
\trans fils aîné et fille aînée: les aînés
\trans \textit{\textcolor{brown}{\zh{大儿子与大女儿}}}
\end{exe}


\paragraph{\hspace{-0.5cm} {\Large \textcolor{darkblue}{\textbf{\ipa{mv̩˧dze˧}}}}} \hypertarget{mv\string_=\string_Mdze\string_M1}{}
\markboth{\textcolor{darkblue}{\textbf{\ipa{mv̩˧dze˧}}}}{}
\textcolor{teal}{\mytextsc{nom}} \hspace{4pt} Ton~: M.

Orge, \textit{Hordeum vulgare L}.
\textcolor{brown}{\zh{大麦。}}


 \mytextsc{clf}~: \textcolor{darkblue}{\textbf{\ipa{kɤ˧˥}}} 

\paragraph{\hspace{-0.5cm} {\Large \textcolor{darkblue}{\textbf{\ipa{mv̩˧dze˧-tɕʰi\#˥}}}}} \hypertarget{mv\string_=\string_Mdze\string_M-ts£\string_hi\#\string_T1}{}
\markboth{\textcolor{darkblue}{\textbf{\ipa{mv̩˧dze˧-tɕʰi\#˥}}}}{}
\textcolor{teal}{\mytextsc{nom}} \hspace{4pt} Ton~: \#H.

Barbe d'orge.
\textcolor{brown}{\zh{青稞芒。}}




\paragraph{\hspace{-0.5cm} {\Large \textcolor{darkblue}{\textbf{\ipa{mv̩˩dzɤ˧}}}}} \hypertarget{mv\string_=\string_Bdz7\string_M1}{}
\markboth{\textcolor{darkblue}{\textbf{\ipa{mv̩˩dzɤ˧}}}}{}
\textcolor{teal}{\mytextsc{nom}} \hspace{4pt} Ton~: LM.

Bas, partie inférieure (symboliquement: ``la queue").
\textcolor{brown}{\zh{下面部分。}}


\begin{exe}
\sn \textcolor{darkblue}{\textbf{\ipa{mv̩˩dzɤ˧ dzi˧˥}}}
\trans être assis au fond de la salle
\trans \textit{\textcolor{brown}{\zh{坐在(房间的)下面部分}}}
\end{exe}
\begin{exe}
\sn \textcolor{darkblue}{\textbf{\ipa{no˧ | mv̩˩dzɤ˧ dzi˧˥!}}}
\trans Assieds-toi en bas!
\trans \textit{\textcolor{brown}{\zh{你到下面去坐!}}}
\end{exe}


\paragraph{\hspace{-0.5cm} {\Large \textcolor{darkblue}{\textbf{\ipa{mv̩˧gɤ˥gɤ˩}}}}} \hypertarget{mv\string_=\string_Mg7\string_Tg7\string_B1}{}
\markboth{\textcolor{darkblue}{\textbf{\ipa{mv̩˧gɤ˥gɤ˩}}}}{}
\textcolor{teal}{\mytextsc{nom}} \hspace{4pt} Ton~: .

Les descendants, la descendance.
\textcolor{brown}{\zh{下一代、后裔、后人。}}




\paragraph{\hspace{-0.5cm} {\Large \textcolor{darkblue}{\textbf{\ipa{mv̩˧-gɤ˧lɑ˥}}}}} \hypertarget{mv\string_=\string_M-g7\string_MlA\string_T1}{}
\markboth{\textcolor{darkblue}{\textbf{\ipa{mv̩˧-gɤ˧lɑ˥}}}}{}
\textcolor{teal}{\mytextsc{nom}} \hspace{4pt} Ton~: H\#.

Esprit du ciel, Bodhisattva céleste.
\textcolor{brown}{\zh{天宫菩萨。}}


 \mytextsc{clf}~: \textcolor{darkblue}{\textbf{\ipa{v̩˧}}} 

\paragraph{\hspace{-0.5cm} {\Large \textcolor{darkblue}{\textbf{\ipa{mv̩˧-gv̩˧dv̩˧}}}}} \hypertarget{mv\string_=\string_M-gv\string_=\string_Mdv\string_=\string_M1}{}
\markboth{\textcolor{darkblue}{\textbf{\ipa{mv̩˧-gv̩˧dv̩˧}}}}{}
\textcolor{teal}{\mytextsc{nom}} \hspace{4pt} Ton~: M.

Partie supérieure du pied.
\textcolor{brown}{\zh{脚背。}}


 \mytextsc{clf}~: \textcolor{darkblue}{\textbf{\ipa{ɭɯ˧}}} 

\paragraph{\hspace{-0.5cm} {\Large \textcolor{darkblue}{\textbf{\ipa{mv̩˧gv̩˧-kʰv̩˩}}}}} \hypertarget{mv\string_=\string_Mgv\string_=\string_M-k\string_hv\string_=\string_B1}{}
\markboth{\textcolor{darkblue}{\textbf{\ipa{mv̩˧gv̩˧-kʰv̩˩}}}}{}
\textcolor{teal}{\mytextsc{nom}} \hspace{4pt} Ton~: L\#.

Année du dragon.
\textcolor{brown}{\zh{龙年。}}




\paragraph{\hspace{-0.5cm} {\Large \textcolor{darkblue}{\textbf{\ipa{mv̩˧gv̩\#˥}}}}} \hypertarget{mv\string_=\string_Mgv\string_=\#\string_T1}{}
\markboth{\textcolor{darkblue}{\textbf{\ipa{mv̩˧gv̩\#˥}}}}{}
\textcolor{teal}{\mytextsc{nom}} \hspace{4pt} Ton~: \#H.

Tonnerre.
\textcolor{brown}{\zh{雷、雷声。}}


\begin{exe}
\sn \textcolor{darkblue}{\textbf{\ipa{mv̩˧gv̩˧ | gv̩˧-ze˩}}}
\trans le tonnerre gronde
\trans \textit{\textcolor{brown}{\zh{打雷了}}}
\end{exe}
\begin{exe}
\sn \textcolor{darkblue}{\textbf{\ipa{mv̩˧gv̩˧ lɑ˩}}}
\trans il y a un coup de tonnerre
\trans \textit{\textcolor{brown}{\zh{打雷了}}}
\end{exe}
 \mytextsc{clf}~: \textcolor{darkblue}{\textbf{\ipa{ɭɯ˧}}} 

\paragraph{\hspace{-0.5cm} {\Large \textcolor{darkblue}{\textbf{\ipa{mv̩˩kʰv̩˧˥}}}}} \hypertarget{mv\string_=\string_Bk\string_hv\string_=\string_M\string_T1}{}
\markboth{\textcolor{darkblue}{\textbf{\ipa{mv̩˩kʰv̩˧˥}}}}{}
\textcolor{teal}{\mytextsc{nom}} \hspace{4pt} Ton~: LM+MH\#.

Soir, soirée (dès 17h, 18h, quand approche la tombée de la nuit).
\textcolor{brown}{\zh{晚上。}}




\paragraph{\hspace{-0.5cm} {\Large \textcolor{darkblue}{\textbf{\ipa{mv̩˧kʰv̩˧˥}}}}} \hypertarget{mv\string_=\string_Mk\string_hv\string_=\string_M\string_T1}{}
\markboth{\textcolor{darkblue}{\textbf{\ipa{mv̩˧kʰv̩˧˥}}}}{}
\textcolor{teal}{\mytextsc{nom}} \hspace{4pt} Ton~: MH\#.

Fumée.
\textcolor{brown}{\zh{烟。}}


\begin{exe}
\sn \textcolor{darkblue}{\textbf{\ipa{mv̩˧kʰv̩˧ lv̩˥}}}
\trans ça enfume tout le monde
\trans \textit{\textcolor{brown}{\zh{烟很多}}}
\end{exe}
 \mytextsc{clf}~: \textcolor{darkblue}{\textbf{\ipa{æ̃˩}}} 

\paragraph{\hspace{-0.5cm} {\Large \textcolor{darkblue}{\textbf{\ipa{mv̩˩ɬi˥}}}}} \hypertarget{mv\string_=\string_BKi\string_T1}{}
\markboth{\textcolor{darkblue}{\textbf{\ipa{mv̩˩ɬi˥}}}}{}
\textcolor{teal}{\mytextsc{nom}} \hspace{4pt} Ton~: LH.

Cadette, puinée (fille deuxième née); littéralement: ``fille du milieu".
\textcolor{brown}{\zh{二女儿。}}




\paragraph{\hspace{-0.5cm} {\Large \textcolor{darkblue}{\textbf{\ipa{mv̩˧ɭɯ˩}}}}} \hypertarget{mv\string_=\string_Ml\string_RM\string_B1}{}
\markboth{\textcolor{darkblue}{\textbf{\ipa{mv̩˧ɭɯ˩}}}}{}
\textcolor{teal}{\mytextsc{nom}} \hspace{4pt} Ton~: L\#.

Muli (localité dans le Sichuan, proche de Yongning).
\textcolor{brown}{\zh{木里。}}




\paragraph{\hspace{-0.5cm} {\Large \textcolor{darkblue}{\textbf{\ipa{mv̩˧mi˧}}}}} \hypertarget{mv\string_=\string_Mmi\string_M1}{}
\markboth{\textcolor{darkblue}{\textbf{\ipa{mv̩˧mi˧}}}}{}
\textcolor{teal}{\mytextsc{nom}} \hspace{4pt} Ton~: M.

Femme.
\textcolor{brown}{\zh{女人。}}


\begin{exe}
\sn \textcolor{darkblue}{\textbf{\ipa{mv̩˧mi˧ so˩tsʰi˩-kʰv̩˩, | qʰo˧mo˥ gi˩ le˩-ʈɤ˩! | ʝi˧=ɻæ˧ qʰv̩˧tsʰi˧-kʰv̩˩, | bɤ˧di˩ lɑ˩ hṽ˩ ɖʐæ˩!}}}
\trans ``A trente ans, la femme est déjà comme une vieille vache qu'il faut tirer pour qu'elle avance (=à trente ans, une femme, c'est déjà une vieille); à soixante, l'homme chevauche sur une peau [littéralement: des poils] de tigre au pays des Pumi!" (=pour l'homme, soixante ans c'est un âge qui permet encore les exploits) (Dicton au sujet de la façon dont vieillissent les deux sexes, au plan de l'attirance qu'ils exercent sur l'autre sexe; employé par une femme, peut véhiculer une nuance de moquerie à l'égard d'un homme âgé)
\trans \textit{\textcolor{brown}{\zh{“女人,到三十岁就算是得拉着的老牛。男人,到六十岁还能在普米山上骑老虎!”这个谚语讲男人与女人老化过程,特别描写相互吸引的程度:三十岁女人算是老了,六十岁男人还认为自己有伟大的威力。女人可以用这个谚语隐蔽地嘲弄一个老男人。}}}
\end{exe}
 \mytextsc{clf}~: \textcolor{darkblue}{\textbf{\ipa{v̩˧}}} 

\paragraph{\hspace{-0.5cm} {\Large \textcolor{darkblue}{\textbf{\ipa{mv̩˧-mv̩˥-di˩}}}}} \hypertarget{mv\string_=\string_M-mv\string_=\string_T-di\string_B1}{}
\markboth{\textcolor{darkblue}{\textbf{\ipa{mv̩˧-mv̩˥-di˩}}}}{}
\textcolor{teal}{\mytextsc{nom}} \hspace{4pt} Ton~: H\#-.

Soufflet.
\textcolor{brown}{\zh{风箱。}}


 \mytextsc{clf}~: \textcolor{darkblue}{\textbf{\ipa{ɭɯ˧}}} 

\paragraph{\hspace{-0.5cm} {\Large \textcolor{darkblue}{\textbf{\ipa{mv̩˧ɲi˧}}}}} \hypertarget{mv\string_=\string_MJi\string_M1}{}
\markboth{\textcolor{darkblue}{\textbf{\ipa{mv̩˧ɲi˧}}}}{}
\textcolor{teal}{\mytextsc{nom}} \hspace{4pt} Ton~: M.

Orteil.
\textcolor{brown}{\zh{脚趾。}}


 \mytextsc{clf}~: \textcolor{darkblue}{\textbf{\ipa{ɭɯ˧}}} 

\paragraph{\hspace{-0.5cm} {\Large \textcolor{darkblue}{\textbf{\ipa{mv̩˩pʰæ˧}}}}} \hypertarget{mv\string_=\string_Bp\string_h\{\string_M1}{}
\markboth{\textcolor{darkblue}{\textbf{\ipa{mv̩˩pʰæ˧}}}}{}
\textcolor{teal}{\mytextsc{nom}} \hspace{4pt} Ton~: LM.

Office, cuisine: pièce où on cuisine la pâtée des cochons, où on distille le vin, et où on prépare certains des plats pour les humains. Elle est située dans le même bâtiment que le foyer-salle à manger, à sa droite (vu depuis la cour).
\textcolor{brown}{\zh{备料房:煮猪食、煮酒的地方,有时候也在那边准备人的饭。}}


 \mytextsc{clf}~: \textcolor{darkblue}{\textbf{\ipa{ɭɯ˧}}} 

\paragraph{\hspace{-0.5cm} {\Large \textcolor{darkblue}{\textbf{\ipa{mv̩˩pʰæ˧˥}}}}} \hypertarget{mv\string_=\string_Bp\string_h\{\string_M\string_T1}{}
\markboth{\textcolor{darkblue}{\textbf{\ipa{mv̩˩pʰæ˧˥}}}}{}
\textcolor{teal}{\mytextsc{verbe}} \hspace{4pt} Ton~: .

Oublier.
\textcolor{brown}{\zh{忘记、落下。}}


\begin{exe}
\sn \textcolor{darkblue}{\textbf{\ipa{mv̩˩pʰæ˧-ze˥}}}
\trans (j'ai) oublié!
\trans \textit{\textcolor{brown}{\zh{忘记了}}}
\end{exe}
\begin{exe}
\sn \textcolor{darkblue}{\textbf{\ipa{le˧-mv̩˩pʰæ˩(-ze˩)}}}
\trans \mytextsc{accomp} \string_ \mytextsc{pfv}
\trans \textit{\textcolor{brown}{\zh{忘记了}}}
\end{exe}
\begin{exe}
\sn \textcolor{darkblue}{\textbf{\ipa{ə˧tso˧-lɑ˧ ʂv̩˧ɖv̩˧, | mɤ˧-do˩! | tso˧\textasciitilde{}tso˧ mv̩˥pʰæ˩!}}}
\trans Je ne sais pas à quoi il pense/je ne sais pas ce qu'il a dans la tête: il oublie les choses!/il oublie tout!
\trans \textit{\textcolor{brown}{\zh{不知道(他)在想什么呢!他(一直在)落东西!/ 他一直在丢三落四}}}
\end{exe}


\paragraph{\hspace{-0.5cm} {\Large \textcolor{darkblue}{\textbf{\ipa{mv̩˧qo˩}}}}} \hypertarget{mv\string_=\string_Mqo\string_B1}{}
\markboth{\textcolor{darkblue}{\textbf{\ipa{mv̩˧qo˩}}}}{}
\textcolor{teal}{\mytextsc{nom}} \hspace{4pt} Ton~: L\#.

Papaye.
\textcolor{brown}{\zh{木瓜。}}


\begin{exe}
\sn \textcolor{darkblue}{\textbf{\ipa{mv̩˧qo˩-dʑɯ˩}}}
\trans un liquide préparé à base de papaye, servant d'équivalent de vinaigre (le vinaigre a été introduit tardivement; il était acheté en pays chinois)
\trans \textit{\textcolor{brown}{\zh{用木瓜做的一种汁,用法类似于醋。过去,永宁没有醋,醋是从内地(汉族地区)买来的。}}}
\end{exe}
 \mytextsc{clf}~: \textcolor{darkblue}{\textbf{\ipa{ɭɯ˧}}} 

\paragraph{\hspace{-0.5cm} {\Large \textcolor{darkblue}{\textbf{\ipa{mv̩˧qʰwæ˩}}}}} \hypertarget{mv\string_=\string_Mq\string_hw\{\string_B1}{}
\markboth{\textcolor{darkblue}{\textbf{\ipa{mv̩˧qʰwæ˩}}}}{}
\textcolor{teal}{\mytextsc{nom}} \hspace{4pt} Ton~: L\#.

Village na hors de la plaine de Yongning, vers le Lac.
\textcolor{brown}{\zh{木垮:村落名。}}


\begin{exe}
\sn \textcolor{darkblue}{\textbf{\ipa{ɬi˧ki˧, | ɲi˧se˩, | tɑ˧dzi˩, | mv̩˧qʰwæ˩, | lɑ˧tʰɑ˧-di˧˥}}}
\trans Villages dans l'ordre, après la plaine de Yongning, ne comptant pas comme faisant partie de Yongning. Le dernier, \textcolor{darkblue}{\textbf{\ipa{/lɑ˧tʰɑ˧-di˧˥/}}}, désigne toute la région na au-delà du quatrième village.
\trans \textit{\textcolor{brown}{\zh{永宁到泸沽湖所经过的村落,依次是:里格、尼赛、大祖、木垮,然后到拉塔地(拉塔地指的是泸沽湖周边的摩梭地区,包括左所、洛水村等)}}}
\end{exe}


\paragraph{\hspace{-0.5cm} {\Large \textcolor{darkblue}{\textbf{\ipa{mv̩˧ʁo˥\$}}}}} \hypertarget{mv\string_=\string_MRo\string_T\$1}{}
\markboth{\textcolor{darkblue}{\textbf{\ipa{mv̩˧ʁo˥\$}}}}{}
\textcolor{teal}{\mytextsc{nom}} \hspace{4pt} Ton~: H\$.

Le ciel, les cieux.
\textcolor{brown}{\zh{天空。}}


 \mytextsc{clf}~: \textcolor{darkblue}{\textbf{\ipa{ɭɯ˧}}} 

\paragraph{\hspace{-0.5cm} {\Large \textcolor{darkblue}{\textbf{\ipa{mv̩˩ʁwɤ˧}}} \textsubscript{1}}} \hypertarget{mv\string_=\string_BRw7\string_M1}{}
\markboth{\textcolor{darkblue}{\textbf{\ipa{mv̩˩ʁwɤ˧}}} \textsubscript{1}}{}
\textcolor{teal}{\mytextsc{nom}} \hspace{4pt} Ton~: LM.

Cours inférieur, aval.
\textcolor{brown}{\zh{下游。}}




\paragraph{\hspace{-0.5cm} {\Large \textcolor{darkblue}{\textbf{\ipa{mv̩˩ʁwɤ˧}}} \textsubscript{2}}} \hypertarget{mv\string_=\string_BRw7\string_M2}{}
\markboth{\textcolor{darkblue}{\textbf{\ipa{mv̩˩ʁwɤ˧}}} \textsubscript{2}}{}
\textcolor{teal}{\mytextsc{nom}} \hspace{4pt} Ton~: LM.

``le village du bas": nom courant pour désigner un hameau d'un village, ou un village entier, par exemple le hameau du bas du village de Zhubo.
\textcolor{brown}{\zh{下村,比如者波下村(永宁的一个村落)。}}




\paragraph{\hspace{-0.5cm} {\Large \textcolor{darkblue}{\textbf{\ipa{mv̩˩si˧˥}}}}} \hypertarget{mv\string_=\string_Bsi\string_M\string_T1}{}
\markboth{\textcolor{darkblue}{\textbf{\ipa{mv̩˩si˧˥}}}}{}
\textcolor{teal}{\mytextsc{nom}} \hspace{4pt} Ton~: LM+MH\#.

Matin.
\textcolor{brown}{\zh{早晨。}}




\paragraph{\hspace{-0.5cm} {\Large \textcolor{darkblue}{\textbf{\ipa{mv̩˩si˧-njɤ˧˥}}}}} \hypertarget{mv\string_=\string_Bsi\string_M-nj7\string_M\string_T1}{}
\markboth{\textcolor{darkblue}{\textbf{\ipa{mv̩˩si˧-njɤ˧˥}}}}{}
\textcolor{teal}{\mytextsc{adverbe}} \hspace{4pt} Ton~: LM+MH\#.

Tôt le matin.
\textcolor{brown}{\zh{一大早。}}




\paragraph{\hspace{-0.5cm} {\Large \textcolor{darkblue}{\textbf{\ipa{mv̩˩tɑ\#˥}}}}} \hypertarget{mv\string_=\string_BtA\#\string_T1}{}
\markboth{\textcolor{darkblue}{\textbf{\ipa{mv̩˩tɑ\#˥}}}}{}
\textcolor{teal}{\mytextsc{verbe}} \hspace{4pt} Ton~: LM+\#H.

Louer, faire l'éloge de.
\textcolor{brown}{\zh{表扬。}}


\begin{exe}
\sn \textcolor{darkblue}{\textbf{\ipa{mv̩˩tɑ˧ ʝi˧}}}
\trans louer, faire l'éloge de
\trans \textit{\textcolor{brown}{\zh{表扬}}}
\end{exe}
\begin{exe}
\sn \textcolor{darkblue}{\textbf{\ipa{hĩ˧-ɳɯ˩ | mv̩˩tɑ˥ F | ʝi˧ le˧-hɯ˩-ze˩.}}}
\trans (Il a fait de bonnes choses, et) les gens l'ont loué/ ont chanté ses louanges
\trans \textit{\textcolor{brown}{\zh{(他做了好事情,于是)人家大大地表扬他了。}}}
\end{exe}


\paragraph{\hspace{-0.5cm} {\Large \textcolor{darkblue}{\textbf{\ipa{mv̩˧ʈʰæ\#˥}}}}} \hypertarget{mv\string_=\string_Mt`\string_h\{\#\string_T1}{}
\markboth{\textcolor{darkblue}{\textbf{\ipa{mv̩˧ʈʰæ\#˥}}}}{}
\textcolor{teal}{\mytextsc{adverbe}} \hspace{4pt} Ton~: \#H.

Dessous, en bas.
\textcolor{brown}{\zh{下面。}}


\begin{exe}
\sn \textcolor{darkblue}{\textbf{\ipa{ʈʂʰɯ˧ | mv̩˧ʈʰæ˧-lɑ˩ li˩! | gɤ˧bi˧ mɤ˧-li˩!}}}
\trans Il regarde tout le temps vers le bas! il ne regarde pas vers le haut! (au sujet d'une personne constamment assise à son bureau, et qui se plaint de mots de tête)
\trans \textit{\textcolor{brown}{\zh{他老低头是往下看,不往上看!(情景:有人经常脖子疼、头疼,阿妈提出,这应该跟工作姿势不对有关:那个人一直坐在办公桌前,低着头)}}}
\end{exe}


\paragraph{\hspace{-0.5cm} {\Large \textcolor{darkblue}{\textbf{\ipa{mv̩˩tɕi˥}}}}} \hypertarget{mv\string_=\string_Bts£i\string_T1}{}
\markboth{\textcolor{darkblue}{\textbf{\ipa{mv̩˩tɕi˥}}}}{}
\textcolor{teal}{\mytextsc{nom}} \hspace{4pt} Ton~: LH.

Benjamine, plus jeune fille.
\textcolor{brown}{\zh{最小的女儿。}}




\paragraph{\hspace{-0.5cm} {\Large \textcolor{darkblue}{\textbf{\ipa{mv̩˩tɕo˧}}}}} \hypertarget{mv\string_=\string_Bts£o\string_M1}{}
\markboth{\textcolor{darkblue}{\textbf{\ipa{mv̩˩tɕo˧}}}}{}
\textcolor{teal}{\mytextsc{adverbe}} \hspace{4pt} Ton~: LM.

Vers le bas.
\textcolor{brown}{\zh{往下。}}


\begin{exe}
\sn \textcolor{darkblue}{\textbf{\ipa{mv̩˩tɕo˧ kwɤ˩}}}
\trans jeter vers le bas
\trans \textit{\textcolor{brown}{\zh{往下扔}}}
\end{exe}
\begin{exe}
\sn \textcolor{darkblue}{\textbf{\ipa{mv̩˩tɕo˧ se˧!}}}
\trans Descends! (Ce qu'on dit au chien qui monte sur la partie haute de la cuisine, contrevenant à la règle)
\trans \textit{\textcolor{brown}{\zh{下去!(命令狗从主屋的地板下去:狗不准来上面)}}}
\end{exe}


\paragraph{\hspace{-0.5cm} {\Large \textcolor{darkblue}{\textbf{\ipa{mv̩˩tʰi˩}}}}} \hypertarget{mv\string_=\string_Bt\string_hi\string_B1}{}
\markboth{\textcolor{darkblue}{\textbf{\ipa{mv̩˩tʰi˩}}}}{}
\textcolor{teal}{\mytextsc{adjectif}} \hspace{4pt} Ton~: L.
\textit{De:} \textbf{mv̩˩˥ et tʰi˧} 
Intelligente (d'une femme).
\textcolor{brown}{\zh{聪明。}}


\begin{exe}
\sn \textcolor{darkblue}{\textbf{\ipa{ʈʂʰɯ˧ | mv̩˩tʰi˩˥ | ʐwæ˩˥!}}}
\trans elle est très intelligente!
\trans \textit{\textcolor{brown}{\zh{她很聪明!}}}
\end{exe}


\paragraph{\hspace{-0.5cm} {\Large \textcolor{darkblue}{\textbf{\ipa{mv̩˩ʈʂæ˧˥}}}}} \hypertarget{mv\string_=\string_Bt`s`\{\string_M\string_T1}{}
\markboth{\textcolor{darkblue}{\textbf{\ipa{mv̩˩ʈʂæ˧˥}}}}{}
\textcolor{teal}{\mytextsc{nom}} \hspace{4pt} Ton~: LM+MH\#.

Bas du corps, partie inférieure du corps.
\textcolor{brown}{\zh{下半(身)。}}




\paragraph{\hspace{-0.5cm} {\Large \textcolor{darkblue}{\textbf{\ipa{mv̩˧ʈʂæ˧˥}}}}} \hypertarget{mv\string_=\string_Mt`s`\{\string_M\string_T1}{}
\markboth{\textcolor{darkblue}{\textbf{\ipa{mv̩˧ʈʂæ˧˥}}}}{}
\textcolor{teal}{\mytextsc{verbe}} \hspace{4pt} Ton~: MH\#.

S'appeler, avoir... pour nom.
\textcolor{brown}{\zh{叫做、称作、名叫。}}


\begin{exe}
\sn \textcolor{darkblue}{\textbf{\ipa{(ʈʂʰɯ˧ | ) ə˧tso˧ mv̩˧ʈʂæ˧˥?}}}
\trans comment il s'appelle? Quel est son nom?
\trans \textit{\textcolor{brown}{\zh{他叫什么名字?}}}
\end{exe}
\begin{exe}
\sn \textcolor{darkblue}{\textbf{\ipa{njɤ˧ | ... mv̩˧ʈʂæ˧˥}}}
\trans Je m'appelle…
\trans \textit{\textcolor{brown}{\zh{我名字叫……}}}
\end{exe}


\paragraph{\hspace{-0.5cm} {\Large \textcolor{darkblue}{\textbf{\ipa{mv̩˧tsʰi\#˥}}}}} \hypertarget{mv\string_=\string_Mts\string_hi\#\string_T1}{}
\markboth{\textcolor{darkblue}{\textbf{\ipa{mv̩˧tsʰi\#˥}}}}{}
\textcolor{teal}{\mytextsc{nom}} \hspace{4pt} Ton~: \#H.

Saison sèche (hiver et printemps; du 9e mois au 2e mois du calendrier lunaire compris).
\textcolor{brown}{\zh{旱季(冬天与春天:农历九月到二月)。}}


\begin{exe}
\sn \textcolor{darkblue}{\textbf{\ipa{mv̩˧tsʰi˧-qo˩}}}
\trans durant la saison sèche
\trans \textit{\textcolor{brown}{\zh{旱季的时候}}}
\end{exe}


\paragraph{\hspace{-0.5cm} {\Large \textcolor{darkblue}{\textbf{\ipa{mv̩˩tsʰo˩}}}}} \hypertarget{mv\string_=\string_Bts\string_ho\string_B1}{}
\markboth{\textcolor{darkblue}{\textbf{\ipa{mv̩˩tsʰo˩}}}}{}
\textcolor{teal}{\mytextsc{nom}} \hspace{4pt} Ton~: L.

Bois gorgé de résine, pour faire partir le feu (pièces de la taille d'une bûche, qu'on débite en petits morceaux pour faire partir le feu).
\textcolor{brown}{\zh{含很多树脂的木头,用来引火。}}


 \mytextsc{clf}~: \textcolor{darkblue}{\textbf{\ipa{kɤ˧˥}}} 

\paragraph{\hspace{-0.5cm} {\Large \textcolor{darkblue}{\textbf{\ipa{mv̩˧ʈʂʰɤ˩}}}}} \hypertarget{mv\string_=\string_Mt`s`\string_h7\string_B1}{}
\markboth{\textcolor{darkblue}{\textbf{\ipa{mv̩˧ʈʂʰɤ˩}}}}{}
\textcolor{teal}{\mytextsc{nom}} \hspace{4pt} Ton~: L\#.

Menton.
\textcolor{brown}{\zh{下巴。}}


 \mytextsc{clf}~: \textcolor{darkblue}{\textbf{\ipa{kʰwɤ˥}}} 

\paragraph{\hspace{-0.5cm} {\Large \textcolor{darkblue}{\textbf{\ipa{mv̩˧ʈʂo˩-ti˩-bv̩˩}}}}} \hypertarget{mv\string_=\string_Mt`s`o\string_B-ti\string_B-bv\string_=\string_B1}{}
\markboth{\textcolor{darkblue}{\textbf{\ipa{mv̩˧ʈʂo˩-ti˩-bv̩˩}}}}{}
\textcolor{teal}{\mytextsc{nom}} \hspace{4pt} Ton~: .

Araignée d'eau, \textit{Gerridae}.
\textcolor{brown}{\zh{水黽。}}




\paragraph{\hspace{-0.5cm} {\Large \textcolor{darkblue}{\textbf{\ipa{mv̩˧tsɯ˧˥}}}}} \hypertarget{mv\string_=\string_MtsM\string_M\string_T1}{}
\markboth{\textcolor{darkblue}{\textbf{\ipa{mv̩˧tsɯ˧˥}}}}{}
\textcolor{teal}{\mytextsc{nom}} \hspace{4pt} Ton~: MH\#.

Barbe.
\textcolor{brown}{\zh{胡子。}}


\begin{exe}
\sn \textcolor{darkblue}{\textbf{\ipa{mv̩˧tsɯ˧ ʑi˥}}}
\trans avoir de la barbe
\trans \textit{\textcolor{brown}{\zh{有胡子}}}
\end{exe}
 \mytextsc{clf}~: \textcolor{darkblue}{\textbf{\ipa{kʰwɤ˥}}} 

\paragraph{\hspace{-0.5cm} {\Large \textcolor{darkblue}{\textbf{\ipa{mv̩˧ʈʂv̩˩}}}}} \hypertarget{mv\string_=\string_Mt`s`v\string_=\string_B1}{}
\markboth{\textcolor{darkblue}{\textbf{\ipa{mv̩˧ʈʂv̩˩}}}}{}
\textcolor{teal}{\mytextsc{nom}} \hspace{4pt} Ton~: L\#.

Mortier.
\textcolor{brown}{\zh{臼。}}


 \mytextsc{clf}~: \textcolor{darkblue}{\textbf{\ipa{nɑ˧}}} 

\paragraph{\hspace{-0.5cm} {\Large \textcolor{darkblue}{\textbf{\ipa{mv̩˧ʈʂv̩˥}}} \textsubscript{1}}} \hypertarget{mv\string_=\string_Mt`s`v\string_=\string_T1}{}
\markboth{\textcolor{darkblue}{\textbf{\ipa{mv̩˧ʈʂv̩˥}}} \textsubscript{1}}{}
\textcolor{teal}{\mytextsc{adjectif}} \hspace{4pt} Ton~: H\#.
\ding{202} 
Plissé, froissé.
\textcolor{brown}{\zh{皱(衣服)。}}


\ding{203} 
Ridé.
\textcolor{brown}{\zh{(脸)有皱纹。}}


\begin{exe}
\sn \textcolor{darkblue}{\textbf{\ipa{to˧kɤ˧ | mv̩˧ʈʂv̩˥ ze˩.}}}
\trans (Son) front s'est ridé / son front a pris des rides.
\trans \textit{\textcolor{brown}{\zh{他的前额有了皱纹。}}}
\end{exe}
\begin{exe}
\sn \textcolor{darkblue}{\textbf{\ipa{to˧kɤ˧ | le˧-mv̩˧ʈʂv̩˥}}}
\trans (Son) front est ridé.
\trans \textit{\textcolor{brown}{\zh{他的前额有皱纹。}}}
\end{exe}
\begin{exe}
\sn \textcolor{darkblue}{\textbf{\ipa{[F5] æ˩ʂe˩˥ | le˧-mv̩˧ʈʂv̩˥}}}
\trans La peau est ridée (littéralement: ``la chair est ridée")
\trans \textit{\textcolor{brown}{\zh{皮肤有皱纹(直译:“肉有皱纹”)}}}
\end{exe}
\ding{204} 
Fané.
\textcolor{brown}{\zh{谢(花谢了)。}}


\begin{exe}
\sn \textcolor{darkblue}{\textbf{\ipa{bæ˩bæ˩˥ | le˧-mv̩˧ʈʂv̩˥-ze˩}}}
\trans La fleur s'est fanée.
\trans \textit{\textcolor{brown}{\zh{花谢了。}}}
\end{exe}
\textit{Voir~:} \hyperlink{mv\string_=\string_Mt`s`v\string_=\string_T2}{\textcolor{darkblue}{\textbf{\ipa{mv̩˧ʈʂv̩˥}}} \textsubscript{2}} 

\paragraph{\hspace{-0.5cm} {\Large \textcolor{darkblue}{\textbf{\ipa{mv̩˧ʈʂv̩˥}}} \textsubscript{2}}} \hypertarget{mv\string_=\string_Mt`s`v\string_=\string_T2}{}
\markboth{\textcolor{darkblue}{\textbf{\ipa{mv̩˧ʈʂv̩˥}}} \textsubscript{2}}{}
\textcolor{teal}{\mytextsc{nom}} \hspace{4pt} Ton~: H\#.

Rides.
\textcolor{brown}{\zh{皱纹。}}


 \mytextsc{clf}~: \textcolor{darkblue}{\textbf{\ipa{kʰɯ˩}}} \textit{Voir~:} \hyperlink{mv\string_=\string_Mt`s`v\string_=\string_T1}{\textcolor{darkblue}{\textbf{\ipa{mv̩˧ʈʂv̩˥}}} \textsubscript{1}} 

\paragraph{\hspace{-0.5cm} {\Large \textcolor{darkblue}{\textbf{\ipa{mv̩˧ʈʂv̩˩-nv̩˩mi˩}}}}} \hypertarget{mv\string_=\string_Mt`s`v\string_=\string_B-nv\string_=\string_Bmi\string_B1}{}
\markboth{\textcolor{darkblue}{\textbf{\ipa{mv̩˧ʈʂv̩˩-nv̩˩mi˩}}}}{}
\textcolor{teal}{\mytextsc{nom}} \hspace{4pt} Ton~: L\#-.
\textit{De:} \textbf{mv̩˧ʈʂv̩˩ et nv̩˩mi˩} 
Petit pilon.
\textcolor{brown}{\zh{杵。}}


 \mytextsc{clf}~: \textcolor{darkblue}{\textbf{\ipa{nɑ˧}}} 

\paragraph{\hspace{-0.5cm} {\Large \textcolor{darkblue}{\textbf{\ipa{mv̩˧ʈʰɯ˧}}}}} \hypertarget{mv\string_=\string_Mt`\string_hM\string_M1}{}
\markboth{\textcolor{darkblue}{\textbf{\ipa{mv̩˧ʈʰɯ˧}}}}{}
\textcolor{teal}{\mytextsc{nom}} \hspace{4pt} Ton~: M.

Talon.
\textcolor{brown}{\zh{脚跟。}}


 \mytextsc{clf}~: \textcolor{darkblue}{\textbf{\ipa{kʰwɤ˥}}} 

\paragraph{\hspace{-0.5cm} {\Large \textcolor{darkblue}{\textbf{\ipa{mv̩˩tv̩˩}}}}} \hypertarget{mv\string_=\string_Btv\string_=\string_B1}{}
\markboth{\textcolor{darkblue}{\textbf{\ipa{mv̩˩tv̩˩}}}}{}
\textcolor{teal}{\mytextsc{nom}} \hspace{4pt} Ton~: L.

Fille unique.
\textcolor{brown}{\zh{独生女。}}


\begin{exe}
\sn \textcolor{darkblue}{\textbf{\ipa{mv̩˩tv̩˩˥ | ɖɯ˧-v̩˧-lɑ˧ dʑo˧˥!}}}
\trans (elle) n'a qu'une fille unique!
\trans \textit{\textcolor{brown}{\zh{她只有一个独生女!}}}
\end{exe}


\paragraph{\hspace{-0.5cm} {\Large \textcolor{darkblue}{\textbf{\ipa{mv̩˧tʰv̩˧˥}}}}} \hypertarget{mv\string_=\string_Mt\string_hv\string_=\string_M\string_T1}{}
\markboth{\textcolor{darkblue}{\textbf{\ipa{mv̩˧tʰv̩˧˥}}}}{}
\textcolor{teal}{\mytextsc{nom}} \hspace{4pt} Ton~: MH\#.

Torche.
\textcolor{brown}{\zh{火把。}}


 \mytextsc{clf}~: \textcolor{darkblue}{\textbf{\ipa{qɑ˩}}} 

\paragraph{\hspace{-0.5cm} {\Large \textcolor{darkblue}{\textbf{\ipa{mv̩˧ʐe˧˥}}} \textsubscript{1}}} \hypertarget{mv\string_=\string_Mz`e\string_M\string_T1}{}
\markboth{\textcolor{darkblue}{\textbf{\ipa{mv̩˧ʐe˧˥}}} \textsubscript{1}}{}
\textcolor{teal}{\mytextsc{nom}} \hspace{4pt} Ton~: MH\#.

Saison des pluies (été et automne: du 3e au 8e mois du calendrier lunaire).
\textcolor{brown}{\zh{雨季(夏天与秋天:三月份至八月份)。}}


\begin{exe}
\sn \textcolor{darkblue}{\textbf{\ipa{mv̩˧ʐe˧-qo˥}}}
\trans pendant la saison des pluies
\trans \textit{\textcolor{brown}{\zh{雨季的时候}}}
\end{exe}


\paragraph{\hspace{-0.5cm} {\Large \textcolor{darkblue}{\textbf{\ipa{mv̩˧ʐe˧-ʈʂʰæ˧ɣɯ\#˥}}}}} \hypertarget{mv\string_=\string_Mz`e\string_M-t`s`\string_h\{\string_MGM\#\string_T1}{}
\markboth{\textcolor{darkblue}{\textbf{\ipa{mv̩˧ʐe˧-ʈʂʰæ˧ɣɯ\#˥}}}}{}
\textcolor{teal}{\mytextsc{nom}} \hspace{4pt} Ton~: \#H.

Poudre à canon.
\textcolor{brown}{\zh{火药。}}


 \mytextsc{clf}~: \textcolor{darkblue}{\textbf{\ipa{po˩}}} 

\paragraph{\hspace{-0.5cm} {\Large \textcolor{darkblue}{\textbf{\ipa{mv̩˧ʐe\#˥}}} \textsubscript{2}}} \hypertarget{mv\string_=\string_Mz`e\#\string_T2}{}
\markboth{\textcolor{darkblue}{\textbf{\ipa{mv̩˧ʐe\#˥}}} \textsubscript{2}}{}
\textcolor{teal}{\mytextsc{nom}} \hspace{4pt} Ton~: \#H.

Arme à feu, fusil; arquebuse.
\textcolor{brown}{\zh{枪,明火枪。}}


 \mytextsc{clf}~: \textcolor{darkblue}{\textbf{\ipa{kʰɯ˩}}} 

\paragraph{\hspace{-0.5cm} {\Large \textcolor{darkblue}{\textbf{\ipa{mv̩˧ʑi˩}}}}} \hypertarget{mv\string_=\string_Mz£i\string_B1}{}
\markboth{\textcolor{darkblue}{\textbf{\ipa{mv̩˧ʑi˩}}}}{}
\textcolor{teal}{\mytextsc{nom}} \hspace{4pt} Ton~: L.

Nouvelle, ragot, information, histoire.
\textcolor{brown}{\zh{消息、闲话、八卦。}}


\begin{exe}
\sn \textcolor{darkblue}{\textbf{\ipa{mv̩˧ʑi˩ | ɖɯ˧-kʰwɤ˥}}}
\trans une nouvelle, un ragot, une information
\trans \textit{\textcolor{brown}{\zh{一个八卦}}}
\end{exe}
 \mytextsc{clf}~: \textcolor{darkblue}{\textbf{\ipa{kʰwɤ˥}}} 

\paragraph{\hspace{-0.5cm} {\Large \textcolor{darkblue}{\textbf{\ipa{mv̩˩zo˩}}}}} \hypertarget{mv\string_=\string_Bzo\string_B1}{}
\markboth{\textcolor{darkblue}{\textbf{\ipa{mv̩˩zo˩}}}}{}
\textcolor{teal}{\mytextsc{nom}} \hspace{4pt} Ton~: L.

Jeune fille.
\textcolor{brown}{\zh{姑娘。}}


\begin{exe}
\sn \textcolor{darkblue}{\textbf{\ipa{mv̩˩zo˩=ɻæ˧}}}
\trans les jeunes filles
\trans \textit{\textcolor{brown}{\zh{姑娘们}}}
\end{exe}
 \mytextsc{clf}~: \textcolor{darkblue}{\textbf{\ipa{ɭɯ˧}}} \textcolor{darkblue}{\textbf{\ipa{v̩˧}}} 

\paragraph{\hspace{-0.5cm} {\Large \textcolor{darkblue}{\textbf{\ipa{mv̩˩ʐɤ˩}}}}} \hypertarget{mv\string_=\string_Bz`7\string_B1}{}
\markboth{\textcolor{darkblue}{\textbf{\ipa{mv̩˩ʐɤ˩}}}}{}
\textcolor{teal}{\mytextsc{nom}} \hspace{4pt} Ton~: L.

Fille adoptive.
\textcolor{brown}{\zh{义女。}}




\paragraph{\hspace{-0.5cm} {\Large \textcolor{darkblue}{\textbf{\ipa{mv̩˩zo˩-ə˩mi˥}}}}} \hypertarget{mv\string_=\string_Bzo\string_B-@\string_Bmi\string_T1}{}
\markboth{\textcolor{darkblue}{\textbf{\ipa{mv̩˩zo˩-ə˩mi˥}}}}{}
\textcolor{teal}{\mytextsc{nom}} \hspace{4pt} Ton~: L+H\#.

(une) jeune fille et (sa) mère.
\textcolor{brown}{\zh{姑娘与母亲。}}




\paragraph{\hspace{-0.5cm} {\Large \textcolor{darkblue}{\textbf{\ipa{mv̩˩zɯ˩}}} \textsubscript{1}}} \hypertarget{mv\string_=\string_BzM\string_B1}{}
\markboth{\textcolor{darkblue}{\textbf{\ipa{mv̩˩zɯ˩}}} \textsubscript{1}}{}
\textcolor{teal}{\mytextsc{nom}} \hspace{4pt} Ton~: L.

Frères (aînés ou cadets).
\textcolor{brown}{\zh{兄弟(哥哥们与弟弟们)。}}


\begin{exe}
\sn \textcolor{darkblue}{\textbf{\ipa{ʈʂʰɯ˧ | nɑ˧dʑi˧-bv̩˧ | mv̩˩zɯ˩-ʝi˥-hĩ˩ ɲi˩!}}}
\trans il est frère de \textcolor{darkblue}{\textbf{\ipa{nɑ˧dʑi˧/!}}}
\trans \textit{\textcolor{brown}{\zh{他是\textcolor{darkblue}{\textbf{\ipa{nɑ˧dʑi˧/}}}的兄弟!}}}
\end{exe}
 \mytextsc{clf}~: \textcolor{darkblue}{\textbf{\ipa{v̩˧}}} 

\paragraph{\hspace{-0.5cm} {\Large \textcolor{darkblue}{\textbf{\ipa{mv̩˩zɯ˩}}} \textsubscript{2}}} \hypertarget{mv\string_=\string_BzM\string_B2}{}
\markboth{\textcolor{darkblue}{\textbf{\ipa{mv̩˩zɯ˩}}} \textsubscript{2}}{}
\textcolor{teal}{\mytextsc{nom}} \hspace{4pt} Ton~: L.

Avoine.
\textcolor{brown}{\zh{燕麦。}}


 \mytextsc{clf}~: \textcolor{darkblue}{\textbf{\ipa{kɤ˧˥}}} 

\paragraph{\hspace{-0.5cm} {\Large \textcolor{darkblue}{\textbf{\ipa{mv̩˩zɯ˩-ni˥mi˩}}}}} \hypertarget{mv\string_=\string_BzM\string_B-ni\string_Tmi\string_B1}{}
\markboth{\textcolor{darkblue}{\textbf{\ipa{mv̩˩zɯ˩-ni˥mi˩}}}}{}
\textcolor{teal}{\mytextsc{nom}} \hspace{4pt} Ton~: L+\#H-.

Frères et sœurs (tous les frères et sœurs; s'applique aussi aux cousins).
\textcolor{brown}{\zh{兄弟姐妹,堂兄弟姐妹。}}




\paragraph{\hspace{-0.5cm} {\Large \textcolor{darkblue}{\textbf{\ipa{}}}}} \hypertarget{mv\string_=\string_M‑1}{}
\markboth{\textcolor{darkblue}{\textbf{\ipa{}}}}{}
\textcolor{teal}{\mytextsc{préfixe}} \hspace{4pt} Ton~: M.

Aspect/mode: l'événement est imminent: sur le point de se produire.
\textcolor{brown}{\zh{即将、快要、马上会、立即。}}


\begin{exe}
\sn \textcolor{darkblue}{\textbf{\ipa{ʈʂʰɯ˧ | mv̩˧-dzɯ˧-kwɤ˩-tɕɯ˩!}}}
\trans Mange-le donc! / Finis donc ça! (Contexte: à table, quelqu'un ne finit pas son bol; sa mère ou grand-mère lui enjoint de finir, pour ne pas gaspiller de nourriture.)
\trans \textit{\textcolor{brown}{\zh{你吃完吧!}}}
\end{exe}
\begin{exe}
\sn \textcolor{darkblue}{\textbf{\ipa{tʰi˧-mv̩˧-dzɯ˧-kwɤ˩-tɕɯ˩!}}}
\trans Comme l'exemple précédent, avec le \mytextsc{duratif}
\trans \textit{\textcolor{brown}{\zh{同上}}}
\end{exe}
\begin{exe}
\sn \textcolor{darkblue}{\textbf{\ipa{[M18] ʈʂʰɯ˧ mv̩˧-ʂɯ˧ bi˩-ni˩gv̩˩! njɤ˧ | gv̩˩dʑɯ˩˥ | ʐwæ˩˥! |}}}
\trans Il/elle va mourir! Je suis au désespoir!
\trans \textit{\textcolor{brown}{\zh{他要死了!我很伤心!}}}
\end{exe}
\begin{exe}
\sn \textcolor{darkblue}{\textbf{\ipa{hĩ˧ ʈʂʰɯ˧-v̩˧ tʰv̩˧ mv̩˧-ʂɯ˧-kwɤ˧tɕɯ˥-lɑ˩...}}}
\trans du fait que cette personne va mourir très bientôt...
\trans \textit{\textcolor{brown}{\zh{因为这个人快要去世……}}}
\end{exe}
\begin{exe}
\sn \textcolor{darkblue}{\textbf{\ipa{mv̩˧-dzɯ˧-bi˩-ze˩!}}}
\trans [On] va manger tout de suite!
\trans \textit{\textcolor{brown}{\zh{马上要吃了!}}}
\end{exe}
\begin{exe}
\sn \textcolor{darkblue}{\textbf{\ipa{mv̩˧-hwæ˧}}}
\trans sur le point d'acheter
\trans \textit{\textcolor{brown}{\zh{即将买}}}
\end{exe}
\begin{exe}
\sn \textcolor{darkblue}{\textbf{\ipa{mv̩˧-tɕʰi˧}}}
\trans sur le point de vendre
\trans \textit{\textcolor{brown}{\zh{即将卖}}}
\end{exe}
\begin{exe}
\sn \textcolor{darkblue}{\textbf{\ipa{mv̩˧-dzɯ˧-kwɤ˧tɕɯ˥-lɑ˩...}}}
\trans puisqu'elle/il est sur le point de manger...
\trans \textit{\textcolor{brown}{\zh{因为马上要吃……}}}
\end{exe}
\begin{exe}
\sn \textcolor{darkblue}{\textbf{\ipa{mv̩˧-lɑ˩-kwɤ˩tɕɯ˩-lɑ˩...}}}
\trans puisqu'elle/il est sur le point de frapper...
\trans \textit{\textcolor{brown}{\zh{因为要打……}}}
\end{exe}


\paragraph{\hspace{-0.5cm} {\Large \textcolor{darkblue}{\textbf{\ipa{mv̩˧\textasciitilde{}mv̩\#˥}}}}} \hypertarget{mv\string_=\string_M~mv\string_=\#\string_T1}{}
\markboth{\textcolor{darkblue}{\textbf{\ipa{mv̩˧\textasciitilde{}mv̩\#˥}}}}{}
\textcolor{teal}{\mytextsc{adjectif}} \hspace{4pt} Ton~: .
\textit{De:} \textbf{mv̩˥} 
Clair (parole, événement…).
\textcolor{brown}{\zh{清楚(话、事情)。}}


\begin{exe}
\sn \textcolor{darkblue}{\textbf{\ipa{ʐwɤ˧ mv̩˧\textasciitilde{}mv̩˧}}}
\trans parler clairement
\trans \textit{\textcolor{brown}{\zh{讲清楚}}}
\end{exe}
\begin{exe}
\sn \textcolor{darkblue}{\textbf{\ipa{le˧-mv̩˧\textasciitilde{}mv̩˧-kʰɯ˩}}}
\trans éclaircir, tirer au clair, expliquer
\trans \textit{\textcolor{brown}{\zh{弄明白、讲清楚}}}
\end{exe}


\newpage
\section*{\centering- \textcolor{darkblue}{\textbf{\ipa{n}}} -}
\paragraph{\hspace{-0.5cm} {\Large \textcolor{darkblue}{\textbf{\ipa{nɑ˥}}}}} \hypertarget{nA\string_T1}{}
\markboth{\textcolor{darkblue}{\textbf{\ipa{nɑ˥}}}}{}
\textcolor{teal}{\mytextsc{adjectif}} \hspace{4pt} Ton~: H.

Grave, sérieux (ex.: une blessure).
\textcolor{brown}{\zh{严重,重要。}}


\begin{exe}
\sn \textcolor{darkblue}{\textbf{\ipa{mɤ˧-nɑ˥}}}
\trans bénin, pas grave, sans conséquence (ex.: une écorchure)
\trans \textit{\textcolor{brown}{\zh{不严重}}}
\end{exe}


\paragraph{\hspace{-0.5cm} {\Large \textcolor{darkblue}{\textbf{\ipa{nɑ˧\textsubscript{a}}}}}} \hypertarget{nA\string_Ma1}{}
\markboth{\textcolor{darkblue}{\textbf{\ipa{nɑ˧\textsubscript{a}}}}}{}
\textcolor{teal}{\mytextsc{classificateur}} \hspace{4pt} Ton~: M\textsubscript{a}.

Classificateur des outils.
\textcolor{brown}{\zh{量词:工具(一把)。}}


\begin{exe}
\sn \textcolor{darkblue}{\textbf{\ipa{ɖɯ˧-nɑ˧ dʑo˧}}}
\trans il y en a un; il y a un outil
\trans \textit{\textcolor{brown}{\zh{有一把(工具)}}}
\end{exe}


\paragraph{\hspace{-0.5cm} {\Large \textcolor{darkblue}{\textbf{\ipa{nɑ˧dʑi\#˥}}}}} \hypertarget{nA\string_Mdz£i\#\string_T1}{}
\markboth{\textcolor{darkblue}{\textbf{\ipa{nɑ˧dʑi\#˥}}}}{}
\textcolor{teal}{\mytextsc{nom}} \hspace{4pt} Ton~: \#H.

Prénom féminin.
\textcolor{brown}{\zh{女性名字。}}




\paragraph{\hspace{-0.5cm} {\Large \textcolor{darkblue}{\textbf{\ipa{nɑ˧mi\#˥}}}}} \hypertarget{nA\string_Mmi\#\string_T1}{}
\markboth{\textcolor{darkblue}{\textbf{\ipa{nɑ˧mi\#˥}}}}{}
\textcolor{teal}{\mytextsc{nom}} \hspace{4pt} Ton~: \#H.

Épuisement, misère, difficultés.
\textcolor{brown}{\zh{受累、劳累、辛苦、困难、艰难、艰苦。}}


\begin{exe}
\sn \textcolor{darkblue}{\textbf{\ipa{nɑ˧mi˧ tʰv̩˧!}}}
\trans Misère! / Des difficultés surviennent, on rencontre des difficultés; on est dans une période difficile
\trans \textit{\textcolor{brown}{\zh{现在是艰苦的时候! / 现在很贫困!}}}
\end{exe}
 \mytextsc{clf}~: \textcolor{darkblue}{\textbf{\ipa{kʰwɤ˥}}} 

\paragraph{\hspace{-0.5cm} {\Large \textcolor{darkblue}{\textbf{\ipa{nɑ˧\textasciitilde{}nɑ˥}}}}} \hypertarget{nA\string_M~nA\string_T1}{}
\markboth{\textcolor{darkblue}{\textbf{\ipa{nɑ˧\textasciitilde{}nɑ˥}}}}{}
\textcolor{teal}{\mytextsc{adverbe}} \hspace{4pt} Ton~: .
\textit{De:} \textbf{nɑ˩} 
En cachette, furtivement.
\textcolor{brown}{\zh{偷偷、暗暗、私自、暗地、暗里。}}


\begin{exe}
\sn \textcolor{darkblue}{\textbf{\ipa{nɑ˧nɑ˥ | le˧-li˧ le˧-do˧ |}}}
\trans apercevoir (quelqu'un) qu'on guettait en cachette
\trans \textit{\textcolor{brown}{\zh{偷偷地看见}}}
\end{exe}
\begin{exe}
\sn \textcolor{darkblue}{\textbf{\ipa{nɑ˧\textasciitilde{}nɑ˥ se˩ |}}}
\trans marcher furtivement
\trans \textit{\textcolor{brown}{\zh{贼头贼脑地走}}}
\end{exe}


\paragraph{\hspace{-0.5cm} {\Large \textcolor{darkblue}{\textbf{\ipa{nɑ˧ʈʂʰõ˧-õ˩di˩-pɤ˩}}}}} \hypertarget{nA\string_Mt`s`\string_ho\string_~\string_M-o\string_~\string_Bdi\string_B-p7\string_B1}{}
\markboth{\textcolor{darkblue}{\textbf{\ipa{nɑ˧ʈʂʰõ˧-õ˩di˩-pɤ˩}}}}{}
\textcolor{teal}{\mytextsc{nom}} \hspace{4pt} Ton~: .

Drapeau de prières attaché à des piliers de la maison.
\textcolor{brown}{\zh{经幡、风马旗(挂在家里的柱子上)。}}


 \mytextsc{clf}~: \textcolor{darkblue}{\textbf{\ipa{pʰæ˧˥}}} 

\paragraph{\hspace{-0.5cm} {\Large \textcolor{darkblue}{\textbf{\ipa{nɑ˩\textsubscript{b}}}}}} \hypertarget{nA\string_Bb1}{}
\markboth{\textcolor{darkblue}{\textbf{\ipa{nɑ˩\textsubscript{b}}}}}{}
\textcolor{teal}{\mytextsc{adjectif}} \hspace{4pt} Ton~: L\textsubscript{b}.

Noir, sombre.
\textcolor{brown}{\zh{黑,暗(颜色,天色)。}}


\begin{exe}
\sn \textcolor{darkblue}{\textbf{\ipa{nɑ˩-hĩ˥}}}
\trans \mytextsc{rel}
\trans \textit{\textcolor{brown}{\zh{黑的}}}
\end{exe}
\begin{exe}
\sn \textcolor{darkblue}{\textbf{\ipa{mɤ˧-nɑ˩}}}
\trans \mytextsc{neg}
\trans \textit{\textcolor{brown}{\zh{不黑}}}
\end{exe}


\paragraph{\hspace{-0.5cm} {\Large \textcolor{darkblue}{\textbf{\ipa{nɑ˩bɑ˧-ʁɑ˧ɭɯ\#˥}}}}} \hypertarget{nA\string_BbA\string_M-RA\string_Ml\string_RM\#\string_T1}{}
\markboth{\textcolor{darkblue}{\textbf{\ipa{nɑ˩bɑ˧-ʁɑ˧ɭɯ\#˥}}}}{}
\textcolor{teal}{\mytextsc{nom}} \hspace{4pt} Ton~: LM+\#H.

Nom d'une montagne de Yongning.
\textcolor{brown}{\zh{一座山的名字。}}




\paragraph{\hspace{-0.5cm} {\Large \textcolor{darkblue}{\textbf{\ipa{nɑ˩dzi˧}}}}} \hypertarget{nA\string_Bdzi\string_M1}{}
\markboth{\textcolor{darkblue}{\textbf{\ipa{nɑ˩dzi˧}}}}{}
\textcolor{teal}{\mytextsc{adverbe}} \hspace{4pt} Ton~: LM.

Sombre (au crépuscule, il se met à faire sombre).
\textcolor{brown}{\zh{暗(黄昏/暮的时候,天变暗)。}}


\begin{exe}
\sn \textcolor{darkblue}{\textbf{\ipa{nɑ˩dzi˧-ze˩!}}}
\trans le crépuscule est venu! / c'est le crépuscule!
\trans \textit{\textcolor{brown}{\zh{天变暗了! / 黄昏到了!}}}
\end{exe}
\begin{exe}
\sn \textcolor{darkblue}{\textbf{\ipa{nɑ˩dzi˧-ho˩-ze˩!}}}
\trans Il va faire sombre! La nuit va commencer à tomber!
\trans \textit{\textcolor{brown}{\zh{(天)要变暗了!}}}
\end{exe}


\paragraph{\hspace{-0.5cm} {\Large \textcolor{darkblue}{\textbf{\ipa{nɑ˩hĩ\#˥}}}}} \hypertarget{nA\string_Bhi\string_~\#\string_T1}{}
\markboth{\textcolor{darkblue}{\textbf{\ipa{nɑ˩hĩ\#˥}}}}{}
\textcolor{teal}{\mytextsc{nom}} \hspace{4pt} Ton~: LM+\#H.

Naxi (groupe ethnique).
\textcolor{brown}{\zh{纳西族。}}


\begin{exe}
\sn \textcolor{darkblue}{\textbf{\ipa{nɑ˩hĩ˧-mi˧ ɲi˥!}}}
\trans c'est une femme naxi!
\trans \textit{\textcolor{brown}{\zh{她是纳西族女人!}}}
\end{exe}
\begin{exe}
\sn \textcolor{darkblue}{\textbf{\ipa{nɑ˩hĩ˧-bɑ˧lɑ˥}}}
\trans vêtements naxi, costume naxi
\trans \textit{\textcolor{brown}{\zh{纳西族服装}}}
\end{exe}
\begin{exe}
\sn \textcolor{darkblue}{\textbf{\ipa{nɑ˩hĩ˧-ʐwɤ˧ so˥}}}
\trans apprendre la langue naxi
\trans \textit{\textcolor{brown}{\zh{学纳西语}}}
\end{exe}
\begin{exe}
\sn \textcolor{darkblue}{\textbf{\ipa{nɑ˩hĩ˧-tʰæ˧ɻæ˥}}}
\trans livres naxi
\trans \textit{\textcolor{brown}{\zh{纳西族的书}}}
\end{exe}
 \mytextsc{clf}~: \textcolor{darkblue}{\textbf{\ipa{v̩˧}}} 

\paragraph{\hspace{-0.5cm} {\Large \textcolor{darkblue}{\textbf{\ipa{nɑ˩kwɤ˧}}}}} \hypertarget{nA\string_Bkw7\string_M1}{}
\markboth{\textcolor{darkblue}{\textbf{\ipa{nɑ˩kwɤ˧}}}}{}
\textcolor{teal}{\mytextsc{nom}} \hspace{4pt} Ton~: LM.

Potiron.
\textcolor{brown}{\zh{南瓜。}}


 \mytextsc{clf}~: \textcolor{darkblue}{\textbf{\ipa{ɭɯ˧}}} 

\paragraph{\hspace{-0.5cm} {\Large \textcolor{darkblue}{\textbf{\ipa{nɑ˩mi\#˥}}}}} \hypertarget{nA\string_Bmi\#\string_T1}{}
\markboth{\textcolor{darkblue}{\textbf{\ipa{nɑ˩mi\#˥}}}}{}
\textcolor{teal}{\mytextsc{nom}} \hspace{4pt} Ton~: LM+\#H.

Une femme Na.
\textcolor{brown}{\zh{摩梭女人。}}




\paragraph{\hspace{-0.5cm} {\Large \textcolor{darkblue}{\textbf{\ipa{nɑ˩mv̩˥-nɑ˩dzi˩dzi˩}}}}} \hypertarget{nA\string_Bmv\string_=\string_T-nA\string_Bdzi\string_Bdzi\string_B1}{}
\markboth{\textcolor{darkblue}{\textbf{\ipa{nɑ˩mv̩˥-nɑ˩dzi˩dzi˩}}}}{}
\textcolor{teal}{\mytextsc{adjectif}} \hspace{4pt} Ton~: .

Tout sombre, tout noir (il fait nuit noire).
\textcolor{brown}{\zh{很暗(天变得很暗)。}}




\paragraph{\hspace{-0.5cm} {\Large \textcolor{darkblue}{\textbf{\ipa{nɑ˩pv̩˧-qʰwɤ˧}}}}} \hypertarget{nA\string_Bpv\string_=\string_M-q\string_hw7\string_M1}{}
\markboth{\textcolor{darkblue}{\textbf{\ipa{nɑ˩pv̩˧-qʰwɤ˧}}}}{}
\textcolor{teal}{\mytextsc{nom}} \hspace{4pt} Ton~: LM-.

Empereur (emprunt au mongole?).
\textcolor{brown}{\zh{皇帝。}}


\begin{exe}
\sn \textcolor{darkblue}{\textbf{\ipa{ʈʂʰɯ˧ | nɑ˩pʰv̩˧-qʰwɤ˧-ni˩gv̩˩!}}}
\trans Il vous prend des airs d'empereur! / Il se prend pour l'empereur! (Façon de se moquer d'un personnage qui veut en imposer à tous, qui se prend pour un grand chef.)
\trans \textit{\textcolor{brown}{\zh{他摆出做皇帝的样子! / 他以为他是皇帝吧!(嘲笑一个自以为是的人)}}}
\end{exe}


\paragraph{\hspace{-0.5cm} {\Large \textcolor{darkblue}{\textbf{\ipa{nɑ˩tsʰi˩}}}}} \hypertarget{nA\string_Bts\string_hi\string_B1}{}
\markboth{\textcolor{darkblue}{\textbf{\ipa{nɑ˩tsʰi˩}}}}{}
\textcolor{teal}{\mytextsc{nom}} \hspace{4pt} Ton~: L.

Nom d'une montagne de Yongning.
\textcolor{brown}{\zh{一座山的名字。}}


\begin{exe}
\sn \textcolor{darkblue}{\textbf{\ipa{kɤ˧mv̩˧˥, | æ˧ʂæ˧, | ŋwɤ˧hɑ̃˩, | ʂwæ˧gv̩\#˥, | nɑ˩tsʰi˩˥ | -tɕʰɤ˧pɤ˧mi\#˥, | qv̩˧ɻ̍˧-ʈʂʰɑ˧nɑ˥ |}}}
\trans Les six montagnes de Yongning qui portent un nom. Les autres sommets du voisinage n'ont pas une valeur symbolique comparable, et ne portent pas de nom communément utilisé.
\trans \textit{\textcolor{brown}{\zh{永宁地区有固定名字的六座山。其它的山,因为没有重要的象征意义,因此没有取名。}}}
\end{exe}


\paragraph{\hspace{-0.5cm} {\Large \textcolor{darkblue}{\textbf{\ipa{nɑ˩zo\#˥}}}}} \hypertarget{nA\string_Bzo\#\string_T1}{}
\markboth{\textcolor{darkblue}{\textbf{\ipa{nɑ˩zo\#˥}}}}{}
\textcolor{teal}{\mytextsc{nom}} \hspace{4pt} Ton~: LM+\#H.

Un homme Na.
\textcolor{brown}{\zh{摩梭男人。}}




\paragraph{\hspace{-0.5cm} {\Large \textcolor{darkblue}{\textbf{\ipa{nɑ˩-ʐwɤ˥}}}}} \hypertarget{nA\string_B-z`w7\string_T1}{}
\markboth{\textcolor{darkblue}{\textbf{\ipa{nɑ˩-ʐwɤ˥}}}}{}
\textcolor{teal}{\mytextsc{nom}} \hspace{4pt} Ton~: LH.

Langue na: endonyme de la langue na.
\textcolor{brown}{\zh{本语言:摩梭话(纳语)。}}




\paragraph{\hspace{-0.5cm} {\Large \textcolor{darkblue}{\textbf{\ipa{nɑ˧˥}}}}} \hypertarget{nA\string_M\string_T1}{}
\markboth{\textcolor{darkblue}{\textbf{\ipa{nɑ˧˥}}}}{}
\textcolor{teal}{\mytextsc{verbe}} \hspace{4pt} Ton~: MH.

Trembler.
\textcolor{brown}{\zh{发抖,颤抖。}}


\begin{exe}
\sn \textcolor{darkblue}{\textbf{\ipa{nɑ˩\textasciitilde{}nɑ˧-ze˥}}}
\trans \mytextsc{red} \mytextsc{pfv}
\trans \textit{\textcolor{brown}{\zh{发抖了}}}
\end{exe}
\begin{exe}
\sn \textcolor{darkblue}{\textbf{\ipa{le˧-nɑ˩\textasciitilde{}nɑ˩}}}
\trans \mytextsc{accomp} \mytextsc{red}
\trans \textit{\textcolor{brown}{\zh{\mytextsc{accomp} \mytextsc{red}}}}
\end{exe}
\begin{exe}
\sn \textcolor{darkblue}{\textbf{\ipa{lo˩qʰwɤ˥ | nɑ˩\textasciitilde{}nɑ˧˥}}}
\trans la main tremble
\trans \textit{\textcolor{brown}{\zh{手抖}}}
\end{exe}


\paragraph{\hspace{-0.5cm} {\Large \textcolor{darkblue}{\textbf{\ipa{nɑ˩˧}}}}} \hypertarget{nA\string_B\string_M1}{}
\markboth{\textcolor{darkblue}{\textbf{\ipa{nɑ˩˧}}}}{}
\textcolor{teal}{\mytextsc{nom}} \hspace{4pt} Ton~: LM.

Endonyme: les Na.
\textcolor{brown}{\zh{自称:摩梭族。}}


\begin{exe}
\sn \textcolor{darkblue}{\textbf{\ipa{nɑ˩-mv̩˧ nɑ˥-di˩ |}}}
\trans le territoire des Na
\trans \textit{\textcolor{brown}{\zh{摩梭人地区}}}
\end{exe}
\begin{exe}
\sn \textcolor{darkblue}{\textbf{\ipa{ə˧ʝi˧-ʂɯ˥ʝi˩, | nɑ˩zo˧-tɑ˥mv̩˩-ɳɯ˩ | dʑo˧-ɲi˥-tsɯ˩!}}}
\trans Autrefois, notre tradition, elle en parlait ! / Notre tradition, elle en parle! (Contexte: quand on fait référence à la coutume locale: ce qu'il est interdit de faire, ce qu'on est autorisé à faire…)
\trans \textit{\textcolor{brown}{\zh{过去,摩梭人的传统(里)有(关于这些问题的说法)嘛!}}}
\end{exe}
 \mytextsc{clf}~: \textcolor{darkblue}{\textbf{\ipa{v̩˧}}} 

\paragraph{\hspace{-0.5cm} {\Large \textcolor{darkblue}{\textbf{\ipa{næ˩-qʰæ˥ʈʂʰe˩}}}}} \hypertarget{n\{\string_B-q\string_h\{\string_Tt`s`\string_he\string_B1}{}
\markboth{\textcolor{darkblue}{\textbf{\ipa{næ˩-qʰæ˥ʈʂʰe˩}}}}{}
\textcolor{teal}{\mytextsc{adjectif}} \hspace{4pt} Ton~: .

Noir, sombre.
\textcolor{brown}{\zh{黑乎乎、很黑。}}




\paragraph{\hspace{-0.5cm} {\Large \textcolor{darkblue}{\textbf{\ipa{-ne}}}}} \hypertarget{-ne1}{}
\markboth{\textcolor{darkblue}{\textbf{\ipa{-ne}}}}{}
\textcolor{teal}{\mytextsc{suffixe}} \hspace{4pt} Ton~: .

Comme.
\textcolor{brown}{\zh{像。}}


\begin{exe}
\sn \textcolor{darkblue}{\textbf{\ipa{[M18] no˧-ɳɯ˧ hwæ˧-ne˧-ʝi˥ | zo˧!}}}
\trans c'est toi qui achètes!
\trans \textit{\textcolor{brown}{\zh{就像你来买一样!}}}
\end{exe}
\begin{exe}
\sn \textcolor{darkblue}{\textbf{\ipa{[M18] no˧-ɳɯ˧ lɑ˧˥ | -ne˧-ʝi˥ | zo˧!}}}
\trans c'est toi qui frappes!
\trans \textit{\textcolor{brown}{\zh{就像你来打一样!}}}
\end{exe}
\begin{exe}
\sn \textcolor{darkblue}{\textbf{\ipa{[M18] no˧-ɳɯ˧ pi˧-ne˧-ʝi˥ | zo˧!}}}
\trans c'est comme tu dis/c'est toi qui décides!
\trans \textit{\textcolor{brown}{\zh{就像你来说一样!}}}
\end{exe}
\begin{exe}
\sn \textcolor{darkblue}{\textbf{\ipa{ʈʂʰɯ˧ne˧-ʝi˥}}}
\trans ainsi, de la sorte
\trans \textit{\textcolor{brown}{\zh{这样}}}
\end{exe}
\begin{exe}
\sn \textcolor{darkblue}{\textbf{\ipa{tʰv̩˧-kʰv̩˩-ne˩-ʝi˩-zo˩}}}
\trans comme cette année-là, de la même façon que cette année-là
\trans \textit{\textcolor{brown}{\zh{像那年一样、就像那年}}}
\end{exe}
\begin{exe}
\sn \textcolor{darkblue}{\textbf{\ipa{tʰv̩˧-ɬi˧-ne˧-ʝi˥-zo˩}}}
\trans comme ce mois-là, de la même façon que ce mois-là
\trans \textit{\textcolor{brown}{\zh{像那个月一样、就像那个月}}}
\end{exe}
\begin{exe}
\sn \textcolor{darkblue}{\textbf{\ipa{tʰv̩˧-hɑ̃˩-ne˩-ʝi˩-zo˩}}}
\trans comme ce soir-là, de la même façon que ce soir-là
\trans \textit{\textcolor{brown}{\zh{像那天晚上一样、就像那个晚上}}}
\end{exe}
\begin{exe}
\sn \textcolor{darkblue}{\textbf{\ipa{tʰv̩˧-ʂɯ˥-ne˩-ʝi˩-zo˩}}}
\trans comme cette fois-là, de la même façon que cette fois-là
\trans \textit{\textcolor{brown}{\zh{像那次一样、就像那次}}}
\end{exe}
\begin{exe}
\sn \textcolor{darkblue}{\textbf{\ipa{tʰv̩˧ɲi˧-ne˧-ʝi˥-zo˩}}}
\trans comme ce jour-là, de la même façon que ce jour-là
\trans \textit{\textcolor{brown}{\zh{像那天一样、就像那天}}}
\end{exe}


\paragraph{\hspace{-0.5cm} {\Large \textcolor{darkblue}{\textbf{\ipa{ni˥}}}}} \hypertarget{ni\string_T1}{}
\markboth{\textcolor{darkblue}{\textbf{\ipa{ni˥}}}}{}
\textcolor{teal}{\mytextsc{nom}} \hspace{4pt} Ton~: \#H.

Amaranthe, \textit{Amaranthus}: minuscule graine qui n'est pas une céréale mais a une valeur nutritionnelle comparable aux céréales.
\textcolor{brown}{\zh{苋米。}}


 \mytextsc{clf}~: \textcolor{darkblue}{\textbf{\ipa{po˧}}} 

\paragraph{\hspace{-0.5cm} {\Large \textcolor{darkblue}{\textbf{\ipa{ni˧fv̩˥}}}}} \hypertarget{ni\string_Mfv\string_=\string_T1}{}
\markboth{\textcolor{darkblue}{\textbf{\ipa{ni˧fv̩˥}}}}{}
\textcolor{teal}{\mytextsc{nom}} \hspace{4pt} Ton~: H\#.

Très grand sac/très grande poche; en cuir, pour emballer les produits que l'on transportait sur de longues distances; s'emploie aussi pour désigner le sac de toile dans lequel on place le corps d'un défunt pendant son inhumation provisoire.
\textcolor{brown}{\zh{大包:用来包装物品的皮包(马帮用的),或者来装尸体的麻布包(为了在火葬前暂时存放尸体)。}}


\begin{exe}
\sn \textcolor{darkblue}{\textbf{\ipa{jɤ˧ŋɤ˧-ni˧fv̩˥}}}
\trans grand sac de Chengdu. Expression employée car les sacs de ce type provenaient généralement de la région de Chengdu.
\trans \textit{\textcolor{brown}{\zh{成都大包。(据说这类的包一般是成都地区生产的。)}}}
\end{exe}
 \mytextsc{clf}~: \textcolor{darkblue}{\textbf{\ipa{ɭɯ˧}}} 

\paragraph{\hspace{-0.5cm} {\Large \textcolor{darkblue}{\textbf{\ipa{‑ni˧gv̩˧˥}}}}} \hypertarget{‑ni\string_Mgv\string_=\string_M\string_T1}{}
\markboth{\textcolor{darkblue}{\textbf{\ipa{‑ni˧gv̩˧˥}}}}{}
\textcolor{teal}{\mytextsc{adverbe}} \hspace{4pt} Ton~: MH\#.

Comme (être comme, être semblable à).
\textcolor{brown}{\zh{如、像。}}


\begin{exe}
\sn \textcolor{darkblue}{\textbf{\ipa{zɯ˧hṽ˩-ni˩gv̩˩}}}
\trans comme de l’herbe, c'est-à-dire vert
\trans \textit{\textcolor{brown}{\zh{像草,等于绿色}}}
\end{exe}
\begin{exe}
\sn \textcolor{darkblue}{\textbf{\ipa{æ̃˧qæ˩-ni˩gv̩˩}}}
\trans bleu-vert; littéralement ``comme un perroquet", c'est-à-dire ``couleur perroquet"
\trans \textit{\textcolor{brown}{\zh{像鹦鹉,等于青色}}}
\end{exe}
\begin{exe}
\sn \textcolor{darkblue}{\textbf{\ipa{lwæ˩pʰv̩˩-ni˥gv̩˩}}}
\trans comme de la cendre = de couleur grise
\trans \textit{\textcolor{brown}{\zh{灰色}}}
\end{exe}
\begin{exe}
\sn \textcolor{darkblue}{\textbf{\ipa{sɯ˧pv̩˩-ni˩gv̩˩}}}
\trans comme une vessie, en forme de vessie
\trans \textit{\textcolor{brown}{\zh{像膀胱}}}
\end{exe}
\begin{exe}
\sn \textcolor{darkblue}{\textbf{\ipa{(nv̩˩mi˩˥ | ) ɖɯ˧-v̩˧-ni˩gv̩˩}}}
\trans comme un seul cœur, (leur) cœur est à l'unisson
\trans \textit{\textcolor{brown}{\zh{一条心,想得一致}}}
\end{exe}
\begin{exe}
\sn \textcolor{darkblue}{\textbf{\ipa{dzi˩bi˩-ni˩gv̩˩˥}}}
\trans s’habituer, s'accoutumer, prendre ses habitudes (dans un environnement)
\trans \textit{\textcolor{brown}{\zh{习惯(一个环境)}}}
\end{exe}
\begin{exe}
\sn \textcolor{darkblue}{\textbf{\ipa{[élicitation lors de la transcription de Agriculture.70] li˩ ʈʰɯ˩-bi˩-ni˩-gv̩˩˥}}}
\trans avoir l'habitude de boire du thé, être un buveur de thé
\trans \textit{\textcolor{brown}{\zh{习惯喝茶、有喝茶的习惯}}}
\end{exe}
\begin{exe}
\sn \textcolor{darkblue}{\textbf{\ipa{ʈʂʰɯ˧ | ʂɯ˧-bi˧-ni˩gv̩˩!}}}
\trans On dirait qu'il/elle va mourir!
\trans \textit{\textcolor{brown}{\zh{他好像要死了!}}}
\end{exe}


\paragraph{\hspace{-0.5cm} {\Large \textcolor{darkblue}{\textbf{\ipa{ni˧mi\#˥}}}}} \hypertarget{ni\string_Mmi\#\string_T1}{}
\markboth{\textcolor{darkblue}{\textbf{\ipa{ni˧mi\#˥}}}}{}
\textcolor{teal}{\mytextsc{nom}} \hspace{4pt} Ton~: \#H.

Soeurs (aînées ou cadettes).
\textcolor{brown}{\zh{姐妹。}}


\begin{exe}
\sn \textcolor{darkblue}{\textbf{\ipa{ʈʂʰɯ˧ | ʈæ˧ʂɯ˧-bv̩˧ | ni˧mi˧ ɲi˥.}}}
\trans Elle est soeur de \textcolor{darkblue}{\textbf{\ipa{/ʈæ˧ʂɯ˧/}}}.
\trans \textit{\textcolor{brown}{\zh{她是达石的姐姐(或妹妹)}}}
\end{exe}
 \mytextsc{clf}~: \textcolor{darkblue}{\textbf{\ipa{v̩˧}}} 

\paragraph{\hspace{-0.5cm} {\Large \textcolor{darkblue}{\textbf{\ipa{njæ˥-qv̩˩}}}}} \hypertarget{nj\{\string_T-qv\string_=\string_B1}{}
\markboth{\textcolor{darkblue}{\textbf{\ipa{njæ˥-qv̩˩}}}}{}
\textcolor{teal}{\mytextsc{verbe}} \hspace{4pt} Ton~: H\#-.

Détourner le regard, détourner la tête, se détourner.
\textcolor{brown}{\zh{看别的方向(蔑视态度)。}}


\begin{exe}
\sn \textcolor{darkblue}{\textbf{\ipa{hĩ˧ njæ˧qv̩˥}}}
\trans se détourner de quelqu'un, détourner la tête face à quelqu'un (que l'on méprise, déteste...)
\trans \textit{\textcolor{brown}{\zh{看别的方向,不直接看(蔑视态度)}}}
\end{exe}
\begin{exe}
\sn \textcolor{darkblue}{\textbf{\ipa{mɤ˧-njæ˥qv̩˩}}}
\trans \mytextsc{neg}: ne pas détourner le regard (face à quelqu'un)
\trans \textit{\textcolor{brown}{\zh{\mytextsc{neg}}}}
\end{exe}


\paragraph{\hspace{-0.5cm} {\Large \textcolor{darkblue}{\textbf{\ipa{njæ˧bæ˥}}}}} \hypertarget{nj\{\string_Mb\{\string_T1}{}
\markboth{\textcolor{darkblue}{\textbf{\ipa{njæ˧bæ˥}}}}{}
\textcolor{teal}{\mytextsc{nom}} \hspace{4pt} Ton~: H\#.

Larme.
\textcolor{brown}{\zh{眼泪。}}


 \mytextsc{clf}~: \textcolor{darkblue}{\textbf{\ipa{ʈʰɤ˥}}} 

\paragraph{\hspace{-0.5cm} {\Large \textcolor{darkblue}{\textbf{\ipa{njæ˧-sɯ˩kv̩˩}}}}} \hypertarget{nj\{\string_M-sM\string_Bkv\string_=\string_B1}{}
\markboth{\textcolor{darkblue}{\textbf{\ipa{njæ˧-sɯ˩kv̩˩}}}}{}
\textcolor{teal}{\mytextsc{pronom}} \hspace{4pt} Ton~: \mytextsc{L}.

1e personne du pluriel, exclusive.
\textcolor{brown}{\zh{我们。}}


\begin{exe}
\sn \textcolor{darkblue}{\textbf{\ipa{njæ˧-sɯ˩-kv̩˩ | mɤ˧-sɯ˥!}}}
\trans Nous ne savons pas! / Nous ne sommes pas au courant!
\trans \textit{\textcolor{brown}{\zh{我们不知道!}}}
\end{exe}
\begin{exe}
\sn \textcolor{darkblue}{\textbf{\ipa{ʈʂʰɯ˧-sɯ˩-kv̩˩, | njæ˧-sɯ˩-kv̩˩-bv̩˩ | mɤ˧-sɯ˥!}}}
\trans Ils ne nous comprennent pas! (Contexte: au sujet de gens venant d'autres régions, qui ne comprennent pas les us et coutumes locales)
\trans \textit{\textcolor{brown}{\zh{他们不了解我们!}}}
\end{exe}


\paragraph{\hspace{-0.5cm} {\Large \textcolor{darkblue}{\textbf{\ipa{njæ˧tsɯ˩}}}}} \hypertarget{nj\{\string_MtsM\string_B1}{}
\markboth{\textcolor{darkblue}{\textbf{\ipa{njæ˧tsɯ˩}}}}{}
\textcolor{teal}{\mytextsc{nom}} \hspace{4pt} Ton~: L\#.
\ding{202} 
Sourcil.
\textcolor{brown}{\zh{眉毛。}}


\begin{exe}
\sn \textcolor{darkblue}{\textbf{\ipa{njæ˧tsɯ˩-ɖæ˩}}}
\trans sourcil (formulation permettant de lever l'ambiguïté du terme, qui peut signifier 'sourcil' aussi bien que 'cil')
\trans \textit{\textcolor{brown}{\zh{眉毛}}}
\end{exe}
\begin{exe}
\sn \textcolor{darkblue}{\textbf{\ipa{njæ˧tsɯ˩ | mv̩˩tɕo˧ kʰɯ˧˥}}}
\trans froncer les sourcils (littéralement ``abaisser les sourcils")
\trans \textit{\textcolor{brown}{\zh{皱眉毛}}}
\end{exe}
\ding{203} 
Cil.
\textcolor{brown}{\zh{睫毛、眼睫毛、眼毛。}}


\begin{exe}
\sn \textcolor{darkblue}{\textbf{\ipa{njæ˧tsɯ˩-ʂæ˩}}}
\trans cil (formulation permettant de lever l'ambiguïté du terme, qui peut signifier ``sourcil" aussi bien que ``cil")
\end{exe}
 \mytextsc{clf}~: \textcolor{darkblue}{\textbf{\ipa{kʰwɤ˥}}} 

\paragraph{\hspace{-0.5cm} {\Large \textcolor{darkblue}{\textbf{\ipa{njæ˧=zɯ˩}}}}} \hypertarget{nj\{\string_M=zM\string_B1}{}
\markboth{\textcolor{darkblue}{\textbf{\ipa{njæ˧=zɯ˩}}}}{}
\textcolor{teal}{\mytextsc{pronom}} \hspace{4pt} Ton~: L\#.

Pronom de première personne duelle exclusive: nous deux (le locuteur et une autre personne qui n'est pas l'interlocuteur).
\textcolor{brown}{\zh{我们两个。}}


\begin{exe}
\sn \textcolor{darkblue}{\textbf{\ipa{ɑ˩ʁo˧(-hĩ˧) | njæ˧zɯ˩ ho˩-dʑo˩!}}}
\trans (On ne peut pas rester car) les gens de la famille nous attendent!
\trans \textit{\textcolor{brown}{\zh{(我们两个不能再呆在这里了,)家里在等我们!}}}
\end{exe}


\paragraph{\hspace{-0.5cm} {\Large \textcolor{darkblue}{\textbf{\ipa{njæ˩pʰv̩˧}}}}} \hypertarget{nj\{\string_Bp\string_hv\string_=\string_M1}{}
\markboth{\textcolor{darkblue}{\textbf{\ipa{njæ˩pʰv̩˧}}}}{}
\textcolor{teal}{\mytextsc{nom}} \hspace{4pt} Ton~: LM.

Blanc des yeux.
\textcolor{brown}{\zh{白眼球。}}


 \mytextsc{clf}~: \textcolor{darkblue}{\textbf{\ipa{ɭɯ˧}}} 

\paragraph{\hspace{-0.5cm} {\Large \textcolor{darkblue}{\textbf{\ipa{njæ˩qwæ˧˥}}}}} \hypertarget{nj\{\string_Bqw\{\string_M\string_T1}{}
\markboth{\textcolor{darkblue}{\textbf{\ipa{njæ˩qwæ˧˥}}}}{}
\textcolor{teal}{\mytextsc{adjectif}} \hspace{4pt} Ton~: LM+MH\#.

Aveugle.
\textcolor{brown}{\zh{眼睛瞎了。}}


\begin{exe}
\sn \textcolor{darkblue}{\textbf{\ipa{ʈʂʰɯ˧ | njæ˩qwæ˧-ze˥}}}
\trans Elle/il est devenu(e) aveugle.
\trans \textit{\textcolor{brown}{\zh{他眼睛瞎了。}}}
\end{exe}
\begin{exe}
\sn \textcolor{darkblue}{\textbf{\ipa{ʈʂʰɯ˧ | njæ˩qwæ˧ ɲi˥.}}}
\trans Elle/il est aveugle.
\trans \textit{\textcolor{brown}{\zh{他是瞎子。}}}
\end{exe}
\begin{exe}
\sn \textcolor{darkblue}{\textbf{\ipa{njæ˩qwæ˧-mi\#˥}}}
\trans femme aveugle
\trans \textit{\textcolor{brown}{\zh{眼睛瞎了的女人}}}
\end{exe}
\begin{exe}
\sn \textcolor{darkblue}{\textbf{\ipa{njæ˩qwæ˧-zo\#˥}}}
\trans homme aveugle
\trans \textit{\textcolor{brown}{\zh{眼睛瞎了的男人}}}
\end{exe}
\begin{exe}
\sn \textcolor{darkblue}{\textbf{\ipa{njæ˩qwæ˧-hĩ\#˥}}}
\trans personne aveugle
\trans \textit{\textcolor{brown}{\zh{瞎子}}}
\end{exe}
 \mytextsc{clf}~: \textcolor{darkblue}{\textbf{\ipa{v̩˧}}} 

\paragraph{\hspace{-0.5cm} {\Large \textcolor{darkblue}{\textbf{\ipa{njæ˩qʰæ\#˥}}}}} \hypertarget{nj\{\string_Bq\string_h\{\#\string_T1}{}
\markboth{\textcolor{darkblue}{\textbf{\ipa{njæ˩qʰæ\#˥}}}}{}
\textcolor{teal}{\mytextsc{nom}} \hspace{4pt} Ton~: LM+\#H.

Chassie.
\textcolor{brown}{\zh{眼屎。}}


 \mytextsc{clf}~: \textcolor{darkblue}{\textbf{\ipa{kʰwɤ˥}}} 

\paragraph{\hspace{-0.5cm} {\Large \textcolor{darkblue}{\textbf{\ipa{njɤ˧di˧˥}}}}} \hypertarget{nj7\string_Mdi\string_M\string_T1}{}
\markboth{\textcolor{darkblue}{\textbf{\ipa{njɤ˧di˧˥}}}}{}
\textcolor{teal}{\mytextsc{nom}} \hspace{4pt} Ton~: MH\#.

Colle.
\textcolor{brown}{\zh{胶。}}


 \mytextsc{clf}~: \textcolor{darkblue}{\textbf{\ipa{kʰwɤ˥}}} 

\paragraph{\hspace{-0.5cm} {\Large \textcolor{darkblue}{\textbf{\ipa{njɤ˧kv̩˩}}}}} \hypertarget{nj7\string_Mkv\string_=\string_B1}{}
\markboth{\textcolor{darkblue}{\textbf{\ipa{njɤ˧kv̩˩}}}}{}
\textcolor{teal}{\mytextsc{nom}} \hspace{4pt} Ton~: L\#.

Pommettes.
\textcolor{brown}{\zh{颧骨。}}


 \mytextsc{clf}~: \textcolor{darkblue}{\textbf{\ipa{ɭɯ˧}}} \textit{Voir~:} \hyperlink{kv\string_=\string_Bkv\string_=\string_B1}{\textcolor{darkblue}{\textbf{\ipa{kv̩˩kv̩˩}}}} 

\paragraph{\hspace{-0.5cm} {\Large \textcolor{darkblue}{\textbf{\ipa{njɤ˧kv̩˩-njɤ˩tsʰɤ˩}}}}} \hypertarget{nj7\string_Mkv\string_=\string_B-nj7\string_Bts\string_h7\string_B1}{}
\markboth{\textcolor{darkblue}{\textbf{\ipa{njɤ˧kv̩˩-njɤ˩tsʰɤ˩}}}}{}
\textcolor{teal}{\mytextsc{adjectif}} \hspace{4pt} Ton~: L\#-.

Belle; qui a un beau visage, qui a des traits gracieux.
\textcolor{brown}{\zh{美丽、面貌美。}}


\begin{exe}
\sn \textcolor{darkblue}{\textbf{\ipa{ə˧mi˧! | mv̩˩zo˩ ʈʂʰɯ˩-ɭɯ˥ | njɤ˧kv̩˩-njɤ˩tsʰɤ˩! | ɖwæ˧˥ | ə˧v̩˧˥!}}}
\trans Eh bien, cette jeune fille est vraiment belle! Très jolie!
\trans \textit{\textcolor{brown}{\zh{啊呀,这个少女真美丽!很漂亮!}}}
\end{exe}
\begin{exe}
\sn \textcolor{darkblue}{\textbf{\ipa{njɤ˧kv̩˩njɤ˩tsʰɤ˩ | ʐwæ˩˥}}}
\trans particulièrement belle
\trans \textit{\textcolor{brown}{\zh{非常美}}}
\end{exe}
\textit{Voir~:} \textcolor{darkblue}{\textbf{\ipa{njɤ˧kv̩˩, tsʰɤ˧˥a}}} 

\paragraph{\hspace{-0.5cm} {\Large \textcolor{darkblue}{\textbf{\ipa{njɤ˧lɑ˥}}}}} \hypertarget{nj7\string_MlA\string_T1}{}
\markboth{\textcolor{darkblue}{\textbf{\ipa{njɤ˧lɑ˥}}}}{}
\textcolor{teal}{\mytextsc{adverbe}} \hspace{4pt} Ton~: .

Tôt.
\textcolor{brown}{\zh{早、黎明的时候。}}


\begin{exe}
\sn \textcolor{darkblue}{\textbf{\ipa{njɤ˧lɑ˥ | gɤ˩-ʈi˧}}}
\trans se lever tôt
\trans \textit{\textcolor{brown}{\zh{起得早}}}
\end{exe}


\paragraph{\hspace{-0.5cm} {\Large \textcolor{darkblue}{\textbf{\ipa{njɤ˧le˧gv̩\#˥}}}}} \hypertarget{nj7\string_Mle\string_Mgv\string_=\#\string_T1}{}
\markboth{\textcolor{darkblue}{\textbf{\ipa{njɤ˧le˧gv̩\#˥}}}}{}
\textcolor{teal}{\mytextsc{nom}} \hspace{4pt} Ton~: \#H.

Journée, plein jour.
\textcolor{brown}{\zh{白天、大白天。}}


\begin{exe}
\sn \textcolor{darkblue}{\textbf{\ipa{ɲi˧mi˧-njɤ˩le˩gv̩˩}}}
\trans même sens
\trans \textit{\textcolor{brown}{\zh{白天}}}
\end{exe}


\paragraph{\hspace{-0.5cm} {\Large \textcolor{darkblue}{\textbf{\ipa{njɤ˧mv̩˥}}}}} \hypertarget{nj7\string_Mmv\string_=\string_T1}{}
\markboth{\textcolor{darkblue}{\textbf{\ipa{njɤ˧mv̩˥}}}}{}
\textcolor{teal}{\mytextsc{nom}} \hspace{4pt} Ton~: H\#.

Sorte de fourrage pour les cochons, \textit{Chenopodium album}. (Il y a en tout trois sortes de fourrage pour les cochons.).
\textcolor{brown}{\zh{灰条菜、灰灰菜:喂猪的牧草。}}Dialecte chinois local~: \zh{灰凋。}


 \mytextsc{clf}~: \textcolor{darkblue}{\textbf{\ipa{qɑ˩}}} 

\paragraph{\hspace{-0.5cm} {\Large \textcolor{darkblue}{\textbf{\ipa{njɤ˧mv̩˥-mi˩}}}}} \hypertarget{nj7\string_Mmv\string_=\string_T-mi\string_B1}{}
\markboth{\textcolor{darkblue}{\textbf{\ipa{njɤ˧mv̩˥-mi˩}}}}{}
\textcolor{teal}{\mytextsc{nom}} \hspace{4pt} Ton~: H\#-L.

Chameau.
\textcolor{brown}{\zh{骆驼。}}


\begin{exe}
\sn \textcolor{darkblue}{\textbf{\ipa{njɤ˧mv̩˥mi˩-zo˩}}}
\trans enfant du chameau, petit chameau
\trans \textit{\textcolor{brown}{\zh{小骆驼}}}
\end{exe}
\begin{exe}
\sn \textcolor{darkblue}{\textbf{\ipa{njɤ˧mv̩˥mi˩-pʰv̩˩}}}
\trans chameau mâle
\trans \textit{\textcolor{brown}{\zh{公骆驼}}}
\end{exe}
 \mytextsc{clf}~: \textcolor{darkblue}{\textbf{\ipa{mi˩}}} 

\paragraph{\hspace{-0.5cm} {\Large \textcolor{darkblue}{\textbf{\ipa{njɤ˧nɑ˩}}}}} \hypertarget{nj7\string_MnA\string_B1}{}
\markboth{\textcolor{darkblue}{\textbf{\ipa{njɤ˧nɑ˩}}}}{}
\textcolor{teal}{\mytextsc{nom}} \hspace{4pt} Ton~: L\#.

Prunelle.
\textcolor{brown}{\zh{眼珠。}}


 \mytextsc{clf}~: \textcolor{darkblue}{\textbf{\ipa{ɭɯ˧}}} 

\paragraph{\hspace{-0.5cm} {\Large \textcolor{darkblue}{\textbf{\ipa{njɤ˧ʈʂɤ˥}}}}} \hypertarget{nj7\string_Mt`s`7\string_T1}{}
\markboth{\textcolor{darkblue}{\textbf{\ipa{njɤ˧ʈʂɤ˥}}}}{}
\textcolor{teal}{\mytextsc{nom}} \hspace{4pt} Ton~: H\#.

Sornet, herbe à aiguilles, \textit{Bidens pilosa L.}: plante de la famille des Asteraceae, dont les graines noires, fines et allongées, de 5 à 10 mm, s'accrochent aux vêtements et aux poils d'animaux par deux piquants fins, situés à l'une de leurs extrémités.
\textcolor{brown}{\zh{鬼针草。}}




\paragraph{\hspace{-0.5cm} {\Large \textcolor{darkblue}{\textbf{\ipa{njɤ˩}}}}} \hypertarget{nj7\string_B1}{}
\markboth{\textcolor{darkblue}{\textbf{\ipa{njɤ˩}}}}{}
\textcolor{teal}{\mytextsc{pronom}} \hspace{4pt} Ton~: L.

Pronom de 1e personne du singulier.
\textcolor{brown}{\zh{我。}}


\begin{exe}
\sn \textcolor{darkblue}{\textbf{\ipa{njɤ˩ ɲi˩˥!}}}
\trans C'est moi! (Contexte: quelqu'un frappe à la porte, on demande qui c'est, et on reçoit pour réponse: ``C'est moi!")
\trans \textit{\textcolor{brown}{\zh{是我!(情景:一个人敲门,里面的人问是谁,人家回答:“是我!”)}}}
\end{exe}
\begin{exe}
\sn \textcolor{darkblue}{\textbf{\ipa{njɤ˧ no˧ lɑ˧˥}}}
\trans je te frappe
\trans \textit{\textcolor{brown}{\zh{我打你}}}
\end{exe}


\paragraph{\hspace{-0.5cm} {\Large \textcolor{darkblue}{\textbf{\ipa{njɤ˩\textsubscript{b}}}}}} \hypertarget{nj7\string_Bb1}{}
\markboth{\textcolor{darkblue}{\textbf{\ipa{njɤ˩\textsubscript{b}}}}}{}
\textcolor{teal}{\mytextsc{verbe}} \hspace{4pt} Ton~: L\textsubscript{b}.

Décortiquer le riz.
\textcolor{brown}{\zh{舂米。}}


\begin{exe}
\sn \textcolor{darkblue}{\textbf{\ipa{le˧-njɤ˩-ze˩}}}
\trans \mytextsc{accomp} \string_ \mytextsc{pfv}
\trans \textit{\textcolor{brown}{\zh{舂了}}}
\end{exe}
\begin{exe}
\sn \textcolor{darkblue}{\textbf{\ipa{hɑ˧ njɤ˧˥}}}
\trans décortiquer du riz
\trans \textit{\textcolor{brown}{\zh{舂米}}}
\end{exe}
\begin{exe}
\sn \textcolor{darkblue}{\textbf{\ipa{hɑ˧ | le˧-njɤ˩}}}
\trans décortiquer du riz
\trans \textit{\textcolor{brown}{\zh{舂米}}}
\end{exe}
\begin{exe}
\sn \textcolor{darkblue}{\textbf{\ipa{hɑ˧ | ɖɯ˧-njɤ˧\textasciitilde{}njɤ˩-ɻ̍˩}}}
\trans riz - \mytextsc{délimitatif} \string_ \mytextsc{red} \mytextsc{inchoatif} : décortiquer un peu le riz
\trans \textit{\textcolor{brown}{\zh{把米舂一舂}}}
\end{exe}


\paragraph{\hspace{-0.5cm} {\Large \textcolor{darkblue}{\textbf{\ipa{njɤ˩bi˥}}}}} \hypertarget{nj7\string_Bbi\string_T1}{}
\markboth{\textcolor{darkblue}{\textbf{\ipa{njɤ˩bi˥}}}}{}
\textcolor{teal}{\mytextsc{nom}} \hspace{4pt} Ton~: LH.

Paupière supérieure.
\textcolor{brown}{\zh{上眼皮。}}


 \mytextsc{clf}~: \textcolor{darkblue}{\textbf{\ipa{ɭɯ˧}}} 

\paragraph{\hspace{-0.5cm} {\Large \textcolor{darkblue}{\textbf{\ipa{njɤ˩-gɤ˧lɑ˩}}}}} \hypertarget{nj7\string_B-g7\string_MlA\string_B1}{}
\markboth{\textcolor{darkblue}{\textbf{\ipa{njɤ˩-gɤ˧lɑ˩}}}}{}
\textcolor{teal}{\mytextsc{nom}} \hspace{4pt} Ton~: L-L\#.

Prunelle.
\textcolor{brown}{\zh{眼珠。}}


 \mytextsc{clf}~: \textcolor{darkblue}{\textbf{\ipa{ɭɯ˧}}} 

\paragraph{\hspace{-0.5cm} {\Large \textcolor{darkblue}{\textbf{\ipa{njɤ˩kʰi\#˥}}}}} \hypertarget{nj7\string_Bk\string_hi\#\string_T1}{}
\markboth{\textcolor{darkblue}{\textbf{\ipa{njɤ˩kʰi\#˥}}}}{}
\textcolor{teal}{\mytextsc{nom}} \hspace{4pt} Ton~: LM+\#H.

Paupière inférieure.
\textcolor{brown}{\zh{下眼皮。}}


 \mytextsc{clf}~: \textcolor{darkblue}{\textbf{\ipa{kʰwɤ˥}}} 

\paragraph{\hspace{-0.5cm} {\Large \textcolor{darkblue}{\textbf{\ipa{njɤ˩ɭɯ˧}}}}} \hypertarget{nj7\string_Bl\string_RM\string_M1}{}
\markboth{\textcolor{darkblue}{\textbf{\ipa{njɤ˩ɭɯ˧}}}}{}
\textcolor{teal}{\mytextsc{nom}} \hspace{4pt} Ton~: LM.

Œil.
\textcolor{brown}{\zh{眼睛。}}


 \mytextsc{clf}~: \textcolor{darkblue}{\textbf{\ipa{ɭɯ˧}}} 

\paragraph{\hspace{-0.5cm} {\Large \textcolor{darkblue}{\textbf{\ipa{njɤ˩qʰwɤ˧˥}}}}} \hypertarget{nj7\string_Bq\string_hw7\string_M\string_T1}{}
\markboth{\textcolor{darkblue}{\textbf{\ipa{njɤ˩qʰwɤ˧˥}}}}{}
\textcolor{teal}{\mytextsc{nom}} \hspace{4pt} Ton~: LM+MH\#.

Orbite (de l'œil).
\textcolor{brown}{\zh{眼眶。}}


 \mytextsc{clf}~: \textcolor{darkblue}{\textbf{\ipa{ɭɯ˧}}} 

\paragraph{\hspace{-0.5cm} {\Large \textcolor{darkblue}{\textbf{\ipa{njɤ˩-tse˧\textasciitilde{}tse˩}}}}} \hypertarget{nj7\string_B-tse\string_M~tse\string_B1}{}
\markboth{\textcolor{darkblue}{\textbf{\ipa{njɤ˩-tse˧\textasciitilde{}tse˩}}}}{}
\textcolor{teal}{\mytextsc{nom}} \hspace{4pt} Ton~: L-L\#.

Prêle ramifiée, prêle rameuse, \textit{Equisetum ramosissimum Desf.} Herbe sauvage, utilisée en pharmacopée traditionnelle; sa tige est divisée en petits segments, et elle se brise à l'une de ces articulations si on l'arrache.
\textcolor{brown}{\zh{节节草。}}Dialecte chinois local~: \zh{节节高。}


 \mytextsc{clf}~: \textcolor{darkblue}{\textbf{\ipa{po˧}}} 

\paragraph{\hspace{-0.5cm} {\Large \textcolor{darkblue}{\textbf{\ipa{njɤ˩ʈʂv̩˧˥}}}}} \hypertarget{nj7\string_Bt`s`v\string_=\string_M\string_T1}{}
\markboth{\textcolor{darkblue}{\textbf{\ipa{njɤ˩ʈʂv̩˧˥}}}}{}
\textcolor{teal}{\mytextsc{nom}} \hspace{4pt} Ton~: LM+MH\#.

Loche (poisson).
\textcolor{brown}{\zh{泥鳅。}}


 \mytextsc{clf}~: \textcolor{darkblue}{\textbf{\ipa{mi˩}}} 

\paragraph{\hspace{-0.5cm} {\Large \textcolor{darkblue}{\textbf{\ipa{njɤ˧˥}}} \textsubscript{1}}} \hypertarget{nj7\string_M\string_T1}{}
\markboth{\textcolor{darkblue}{\textbf{\ipa{njɤ˧˥}}} \textsubscript{1}}{}
\textcolor{teal}{\mytextsc{verbe}} \hspace{4pt} Ton~: MH.

Coller (2 objets ensemble).
\textcolor{brown}{\zh{贴。}}


\begin{exe}
\sn \textcolor{darkblue}{\textbf{\ipa{le˧-njɤ˧-ze˥!}}}
\trans C'est recollé! / Ca y est, c'est collé!
\trans \textit{\textcolor{brown}{\zh{粘在一起了!}}}
\end{exe}
\begin{exe}
\sn \textcolor{darkblue}{\textbf{\ipa{tso˧\textasciitilde{}tso˧ le˧-ɖʐɤ˧, | le˧-njɤ˧˥!}}}
\trans (pour un livre, par ex.) Quand un truc est déchiré, on le recolle!
\trans \textit{\textcolor{brown}{\zh{东西撕破了,粘在一起(就好了)}}}
\end{exe}
\textit{Voir~:} \hyperlink{nj7\string_M\string_T2}{\textcolor{darkblue}{\textbf{\ipa{njɤ˧˥}}} \textsubscript{2}} 

\paragraph{\hspace{-0.5cm} {\Large \textcolor{darkblue}{\textbf{\ipa{njɤ˧˥}}} \textsubscript{2}}} \hypertarget{nj7\string_M\string_T2}{}
\markboth{\textcolor{darkblue}{\textbf{\ipa{njɤ˧˥}}} \textsubscript{2}}{}
\textcolor{teal}{\mytextsc{adjectif}} \hspace{4pt} Ton~: MH.

Poisseux, collant, visqueux (colle, résine...).
\textcolor{brown}{\zh{黏(胶,树脂)。}}


\textit{Voir~:} \hyperlink{nj7\string_M\string_T1}{\textcolor{darkblue}{\textbf{\ipa{njɤ˧˥}}} \textsubscript{1}} 

\paragraph{\hspace{-0.5cm} {\Large \textcolor{darkblue}{\textbf{\ipa{njɤ˧˥}}} \textsubscript{3}}} \hypertarget{nj7\string_M\string_T3}{}
\markboth{\textcolor{darkblue}{\textbf{\ipa{njɤ˧˥}}} \textsubscript{3}}{}
\textcolor{teal}{\mytextsc{adjectif}} \hspace{4pt} Ton~: MH.

Tôt.
\textcolor{brown}{\zh{早。}}




\paragraph{\hspace{-0.5cm} {\Large \textcolor{darkblue}{\textbf{\ipa{njɤ˩˥}}}}} \hypertarget{nj7\string_B\string_T1}{}
\markboth{\textcolor{darkblue}{\textbf{\ipa{njɤ˩˥}}}}{}
\textcolor{teal}{\mytextsc{nom}} \hspace{4pt} Ton~: LH.

Œil (monosyllabe).
\textcolor{brown}{\zh{眼睛(单音节)。}}


 \mytextsc{clf}~: \textcolor{darkblue}{\textbf{\ipa{ɭɯ˧}}} 

\paragraph{\hspace{-0.5cm} {\Large \textcolor{darkblue}{\textbf{\ipa{njo˥}}}}} \hypertarget{njo\string_T1}{}
\markboth{\textcolor{darkblue}{\textbf{\ipa{njo˥}}}}{}
\textcolor{teal}{\mytextsc{nom}} \hspace{4pt} Ton~: \#H.

Flûte à calebasse, hulusi: instrument à vent à anche libre.
\textcolor{brown}{\zh{葫芦丝、葫芦箫。}}


\begin{exe}
\sn \textcolor{darkblue}{\textbf{\ipa{njo˧ mv̩˥}}}
\trans jouer de la flûte à calebasse
\trans \textit{\textcolor{brown}{\zh{吹响葫芦丝}}}
\end{exe}


\paragraph{\hspace{-0.5cm} {\Large \textcolor{darkblue}{\textbf{\ipa{njo˧}}}}} \hypertarget{njo\string_M1}{}
\markboth{\textcolor{darkblue}{\textbf{\ipa{njo˧}}}}{}
\textcolor{teal}{\mytextsc{nom}} \hspace{4pt} Ton~: M.

Épi (de blé, d'orge, de riz...).
\textcolor{brown}{\zh{谷穗。}}


\begin{exe}
\sn \textcolor{darkblue}{\textbf{\ipa{hɑ˧-njo˩}}}
\trans épi de céréales
\trans \textit{\textcolor{brown}{\zh{谷穗}}}
\end{exe}
\begin{exe}
\sn \textcolor{darkblue}{\textbf{\ipa{mv̩˧dze˧-njo˧ (+ɲi˩)}}}
\trans épi d'orge
\trans \textit{\textcolor{brown}{\zh{大麦穗}}}
\end{exe}
\begin{exe}
\sn \textcolor{darkblue}{\textbf{\ipa{tsʰi˧zi˧-hɑ˧njo˥ (+ɲi˩)}}}
\trans épi d'orge d'altitude
\trans \textit{\textcolor{brown}{\zh{青稞穗}}}
\end{exe}


\paragraph{\hspace{-0.5cm} {\Large \textcolor{darkblue}{\textbf{\ipa{njo˧bi˧li˥}}}}} \hypertarget{njo\string_Mbi\string_Mli\string_T1}{}
\markboth{\textcolor{darkblue}{\textbf{\ipa{njo˧bi˧li˥}}}}{}
\textcolor{teal}{\mytextsc{nom}} \hspace{4pt} Ton~: H\#.

Lèvres.
\textcolor{brown}{\zh{嘴唇。}}


 \mytextsc{clf}~: \textcolor{darkblue}{\textbf{\ipa{ɭɯ˧}}} 

\paragraph{\hspace{-0.5cm} {\Large \textcolor{darkblue}{\textbf{\ipa{njo˩bi˥}}}}} \hypertarget{njo\string_Bbi\string_T1}{}
\markboth{\textcolor{darkblue}{\textbf{\ipa{njo˩bi˥}}}}{}
\textcolor{teal}{\mytextsc{nom}} \hspace{4pt} Ton~: LH.

Sein, mamelle.
\textcolor{brown}{\zh{乳房。}}


\begin{exe}
\sn \textcolor{darkblue}{\textbf{\ipa{ʝi˧-njo˥bi˩}}}
\trans mamelle de la vache
\trans \textit{\textcolor{brown}{\zh{牛的奶头}}}
\end{exe}
\begin{exe}
\sn \textcolor{darkblue}{\textbf{\ipa{[F5] njo˩bi˧-ʁo˧qʰwɤ˩}}}
\trans téton
\trans \textit{\textcolor{brown}{\zh{乳头}}}
\end{exe}
 \mytextsc{clf}~: \textcolor{darkblue}{\textbf{\ipa{ɭɯ˧}}} 

\paragraph{\hspace{-0.5cm} {\Large \textcolor{darkblue}{\textbf{\ipa{njo˩kæ˧tɕi˩˥}}}}} \hypertarget{njo\string_Bk\{\string_Mts£i\string_B\string_T1}{}
\markboth{\textcolor{darkblue}{\textbf{\ipa{njo˩kæ˧tɕi˩˥}}}}{}
\textcolor{teal}{\mytextsc{nom}} \hspace{4pt} Ton~: LM+LH.

Bolet, cèpe, \textit{Boletus edulis}; littéralement ``champignon-galette de sarrasin", du fait de sa texture.
\textcolor{brown}{\zh{牛肝菌(汉语借词)。}}

 Emprunt~: chinois \textcolor{brown}{\zh{牛肝菌}}

\textit{Voir~:} \hyperlink{j7\string_Mq\string_hA\string_M-p7\string_Tj7\string_B-mo\string_B1}{\textcolor{darkblue}{\textbf{\ipa{jɤ˧qʰɑ˧-pɤ˥jɤ˩-mo˩}}}} 

\paragraph{\hspace{-0.5cm} {\Large \textcolor{darkblue}{\textbf{\ipa{njo˩pɤ˩lv̩˥}}}}} \hypertarget{njo\string_Bp7\string_Blv\string_=\string_T1}{}
\markboth{\textcolor{darkblue}{\textbf{\ipa{njo˩pɤ˩lv̩˥}}}}{}
\textcolor{teal}{\mytextsc{nom}} \hspace{4pt} Ton~: L+H\#.

Pis de la vache.
\textcolor{brown}{\zh{牛的奶头。}}


 \mytextsc{clf}~: \textcolor{darkblue}{\textbf{\ipa{ɭɯ˧}}} 

\paragraph{\hspace{-0.5cm} {\Large \textcolor{darkblue}{\textbf{\ipa{njo˩˥}}}}} \hypertarget{njo\string_B\string_T1}{}
\markboth{\textcolor{darkblue}{\textbf{\ipa{njo˩˥}}}}{}
\textcolor{teal}{\mytextsc{nom}} \hspace{4pt} Ton~: LH.

Lait.
\textcolor{brown}{\zh{奶汁。}}


\begin{exe}
\sn \textcolor{darkblue}{\textbf{\ipa{njo˩ ki˧}}}
\trans donner le sein, donner la tétée, nourrir (un nourrisson); littéralement '``donner du lait"
\trans \textit{\textcolor{brown}{\zh{给(喂)奶}}}
\end{exe}
\begin{exe}
\sn \textcolor{darkblue}{\textbf{\ipa{njo˩ ʈʰɯ˩˥}}}
\trans boire du lait
\trans \textit{\textcolor{brown}{\zh{喝奶}}}
\end{exe}


\paragraph{\hspace{-0.5cm} {\Large \textcolor{darkblue}{\textbf{\ipa{no˧no˧}}}}} \hypertarget{no\string_Mno\string_M1}{}
\markboth{\textcolor{darkblue}{\textbf{\ipa{no˧no˧}}}}{}
\textcolor{teal}{\mytextsc{nom}} \hspace{4pt} Ton~: M.

Prénom féminin.
\textcolor{brown}{\zh{女性名字。}}




\paragraph{\hspace{-0.5cm} {\Large \textcolor{darkblue}{\textbf{\ipa{no˧no˧-ɖɯ˩mɑ˩}}}}} \hypertarget{no\string_Mno\string_M-d`M\string_BmA\string_B1}{}
\markboth{\textcolor{darkblue}{\textbf{\ipa{no˧no˧-ɖɯ˩mɑ˩}}}}{}
\textcolor{teal}{\mytextsc{nom}} \hspace{4pt} Ton~: \mytextsc{L}.

Prénom féminin.
\textcolor{brown}{\zh{女性名字。}}




\paragraph{\hspace{-0.5cm} {\Large \textcolor{darkblue}{\textbf{\ipa{no˧=ɻ̍˩}}}}} \hypertarget{no\string_M=r£`̍\string_B1}{}
\markboth{\textcolor{darkblue}{\textbf{\ipa{no˧=ɻ̍˩}}}}{}
\textcolor{teal}{\mytextsc{pronom}} \hspace{4pt} Ton~: L\#.

Deuxième personne du pluriel.
\textcolor{brown}{\zh{你们。}}


\textit{Voir~:} \hyperlink{no\string_M-sM\string_Bkv\string_=\string_B1}{\textcolor{darkblue}{\textbf{\ipa{no˧-sɯ˩kv̩˩}}}} 

\paragraph{\hspace{-0.5cm} {\Large \textcolor{darkblue}{\textbf{\ipa{no˧-sɯ˩kv̩˩}}}}} \hypertarget{no\string_M-sM\string_Bkv\string_=\string_B1}{}
\markboth{\textcolor{darkblue}{\textbf{\ipa{no˧-sɯ˩kv̩˩}}}}{}
\textcolor{teal}{\mytextsc{pronom}} \hspace{4pt} Ton~: \mytextsc{L}.

Pronom de deuxième personne du pluriel.
\textcolor{brown}{\zh{你们。}}


\begin{exe}
\sn \textcolor{darkblue}{\textbf{\ipa{no˧-sɯ˩-kv̩˩ | mɤ˧-sɯ˥!}}}
\trans vous ne savez pas!
\trans \textit{\textcolor{brown}{\zh{你们不知道!}}}
\end{exe}
\begin{exe}
\sn \textcolor{darkblue}{\textbf{\ipa{njɤ˧ | no˧-sɯ˩-kv̩˩ | mɤ˧-sɯ˥!}}}
\trans je ne vous connais pas!
\trans \textit{\textcolor{brown}{\zh{我不认识你们! / 我不熟悉你们!}}}
\end{exe}
\begin{exe}
\sn \textcolor{darkblue}{\textbf{\ipa{njɤ˧ | no˧-sɯ˩-kv̩˩ lɑ˩ / njɤ˧ | no˧-sɯ˩-kv̩˩ lɑ˩-bi˩}}}
\trans je vous frappe / je vais vous frapper
\trans \textit{\textcolor{brown}{\zh{我打你们 / 我要打你们}}}
\end{exe}
\begin{exe}
\sn \textcolor{darkblue}{\textbf{\ipa{no˧-sɯ˩-kv̩˩ lɑ˩-bi˩}}}
\trans vous allez frapper
\trans \textit{\textcolor{brown}{\zh{你们要打了}}}
\end{exe}
\begin{exe}
\sn \textcolor{darkblue}{\textbf{\ipa{no˧-sɯ˩-kv̩˩ | sɯ˧!}}}
\trans vous savez, vous êtes au courant
\trans \textit{\textcolor{brown}{\zh{你们知道}}}
\end{exe}
\begin{exe}
\sn \textcolor{darkblue}{\textbf{\ipa{no˧-sɯ˩-kv̩˩ | le˧-li˧}}}
\trans vous regardez
\trans \textit{\textcolor{brown}{\zh{你们在看}}}
\end{exe}
\begin{exe}
\sn \textcolor{darkblue}{\textbf{\ipa{no˧-sɯ˩-kv̩˩ | li˧-bi˧!}}}
\trans vous allez regarder
\trans \textit{\textcolor{brown}{\zh{你们去看吧!}}}
\end{exe}


\paragraph{\hspace{-0.5cm} {\Large \textcolor{darkblue}{\textbf{\ipa{no˧=zɯ˩}}}}} \hypertarget{no\string_M=zM\string_B1}{}
\markboth{\textcolor{darkblue}{\textbf{\ipa{no˧=zɯ˩}}}}{}
\textcolor{teal}{\mytextsc{pronom}} \hspace{4pt} Ton~: L\#.

Pronom personnel de deuxième personne duel: vous deux.
\textcolor{brown}{\zh{你们俩。}}


\textit{Voir~:} \hyperlink{no\string_BzM\string_M\string_T1}{\textcolor{darkblue}{\textbf{\ipa{no˩zɯ˧˥}}}} 

\paragraph{\hspace{-0.5cm} {\Large \textcolor{darkblue}{\textbf{\ipa{no˩}}}}} \hypertarget{no\string_B1}{}
\markboth{\textcolor{darkblue}{\textbf{\ipa{no˩}}}}{}
\textcolor{teal}{\mytextsc{pronom}} \hspace{4pt} Ton~: L.

Pronom de deuxième personne du singulier.
\textcolor{brown}{\zh{你。}}


\begin{exe}
\sn \textcolor{darkblue}{\textbf{\ipa{no˩ ɲi˩˥}}}
\trans c'est toi!
\trans \textit{\textcolor{brown}{\zh{是你!}}}
\end{exe}


\paragraph{\hspace{-0.5cm} {\Large \textcolor{darkblue}{\textbf{\ipa{no˩bv̩˧}}}}} \hypertarget{no\string_Bbv\string_=\string_M1}{}
\markboth{\textcolor{darkblue}{\textbf{\ipa{no˩bv̩˧}}}}{}
\textcolor{teal}{\mytextsc{nom}} \hspace{4pt} Ton~: LM.

Prénom masculin.
\textcolor{brown}{\zh{男性名字。}}




\paragraph{\hspace{-0.5cm} {\Large \textcolor{darkblue}{\textbf{\ipa{no˩qo˥}}}}} \hypertarget{no\string_Bqo\string_T1}{}
\markboth{\textcolor{darkblue}{\textbf{\ipa{no˩qo˥}}}}{}
\textcolor{teal}{\mytextsc{adverbe}} \hspace{4pt} Ton~: LH.

À proximité de, à côté de.
\textcolor{brown}{\zh{……附近。}}




\paragraph{\hspace{-0.5cm} {\Large \textcolor{darkblue}{\textbf{\ipa{no˩zɯ˧˥}}}}} \hypertarget{no\string_BzM\string_M\string_T1}{}
\markboth{\textcolor{darkblue}{\textbf{\ipa{no˩zɯ˧˥}}}}{}
\textcolor{teal}{\mytextsc{pronom}} \hspace{4pt} Ton~: LM+MH\#.

Pronom personnel de deuxième personne duel: vous deux.
\textcolor{brown}{\zh{你们俩。}}


\textit{Voir~:} \hyperlink{no\string_M=zM\string_B1}{\textcolor{darkblue}{\textbf{\ipa{no˧=zɯ˩}}}} 

\paragraph{\hspace{-0.5cm} {\Large \textcolor{darkblue}{\textbf{\ipa{}}}}} \hypertarget{no\string_Bxx1}{}
\markboth{\textcolor{darkblue}{\textbf{\ipa{}}}}{}
\textcolor{teal}{\mytextsc{verbe}} \hspace{4pt} Ton~: Lxx.

Mélanger, ajouter.
\textcolor{brown}{\zh{搀。}}


\begin{exe}
\sn \textcolor{darkblue}{\textbf{\ipa{ɳæ˧ | tsɑ˧bɤ˧-qo˧ tʰi˧-no˩}}}
\trans mettre du lait dans la farine grillée, mélanger la farine grillée avec du lait
\trans \textit{\textcolor{brown}{\zh{在糌粑里搀奶、糌粑里搀奶}}}
\end{exe}
\begin{exe}
\sn \textcolor{darkblue}{\textbf{\ipa{le˧-no˩}}}
\trans \mytextsc{accomp} \string_
\trans \textit{\textcolor{brown}{\zh{\mytextsc{accomp} \string_}}}
\end{exe}
\begin{exe}
\sn \textcolor{darkblue}{\textbf{\ipa{hɑ˧-qo˩ | tɕæ˧ɻæ˩ tʰi˩-no˩: hɑ˧-qo˩ | tɕæ˧ɻæ˩ tʰi˩-kʰɯ˩}}}
\trans paraphrase pour expliquer le sens du verbe: 'ajouter des légumes en saumure dans la nourriture, ça veut dire: mettre des légumes en saumure dans la nourriture.'
\trans \textit{\textcolor{brown}{\zh{关于这个动词的说明:‘饭里面搀泡菜,就是说:在饭里面放泡菜。’}}}
\end{exe}
\begin{exe}
\sn \textcolor{darkblue}{\textbf{\ipa{hɑ˧-qo˩ | tɕæ˧ɻæ˩ tʰi˩-no˩: tɕæ˧ɻæ˩-lɑ˩ | hɑ˧ | ɖɯ˧-tɕʰo˩ dzɯ˩}}}
\trans paraphrase pour expliquer le sens du verbe: 'ajouter des légumes en saumure dans la nourriture, ça veut dire: manger des légumes en saumure et de la nourriture (du riz) ensemble.'
\trans \textit{\textcolor{brown}{\zh{关于这个动词的说明:‘饭里面搀泡菜,就是说:把泡菜和饭一起吃。’}}}
\end{exe}


\paragraph{\hspace{-0.5cm} {\Large \textcolor{darkblue}{\textbf{\ipa{‑no˧˥}}}}} \hypertarget{‑no\string_M\string_T1}{}
\markboth{\textcolor{darkblue}{\textbf{\ipa{‑no˧˥}}}}{}
\textcolor{teal}{\mytextsc{particule}} \textcolor{teal}{\mytextsc{de}} \textcolor{teal}{\mytextsc{discours}} \hspace{4pt} Ton~: MH.

Topique contrastif. Gloses possibles: …, en revanche, ...; …, pour sa part, ...; quant à ….
\textcolor{brown}{\zh{\mytextsc{主题:对于、关于。}}}


\begin{exe}
\sn \textcolor{darkblue}{\textbf{\ipa{qæ˧do˧ | -no˧˥}}}
\trans en ce qui concerne le bois de construction, …
\trans \textit{\textcolor{brown}{\zh{关于木材,……}}}
\end{exe}
\begin{exe}
\sn \textcolor{darkblue}{\textbf{\ipa{ʑi˧qʰwɤ˧ | -no˧˥}}}
\trans en ce qui concerne la pièce principale, …
\trans \textit{\textcolor{brown}{\zh{关于主屋……}}}
\end{exe}


\paragraph{\hspace{-0.5cm} {\Large \textcolor{darkblue}{\textbf{\ipa{nv̩˥}}} \textsubscript{1}}} \hypertarget{nv\string_=\string_T1}{}
\markboth{\textcolor{darkblue}{\textbf{\ipa{nv̩˥}}} \textsubscript{1}}{}
\textcolor{teal}{\mytextsc{verbe}} \hspace{4pt} Ton~: H.

Suivre à la trace, pister.
\textcolor{brown}{\zh{追赶。}}


\begin{exe}
\sn \textcolor{darkblue}{\textbf{\ipa{le˧-nv̩˥}}}
\trans \mytextsc{accomp}
\trans \textit{\textcolor{brown}{\zh{\mytextsc{accomp}}}}
\end{exe}
\begin{exe}
\sn \textcolor{darkblue}{\textbf{\ipa{ʈʂɤ˩nv̩˩}}}
\trans suivre à la trace, pister
\trans \textit{\textcolor{brown}{\zh{追赶}}}
\end{exe}
\begin{exe}
\sn \textcolor{darkblue}{\textbf{\ipa{le˧-ʈʂɤ˩nv̩˩}}}
\trans suivre à la trace, pister
\trans \textit{\textcolor{brown}{\zh{追赶}}}
\end{exe}
\begin{exe}
\sn \textcolor{darkblue}{\textbf{\ipa{le˧-ʈʂɤ˩nv̩˩ | le˧-hɯ˩}}}
\trans Il est parti à la poursuite de / à la chasse de
\trans \textit{\textcolor{brown}{\zh{追赶去了}}}
\end{exe}


\paragraph{\hspace{-0.5cm} {\Large \textcolor{darkblue}{\textbf{\ipa{nv̩˥}}} \textsubscript{2}}} \hypertarget{nv\string_=\string_T2}{}
\markboth{\textcolor{darkblue}{\textbf{\ipa{nv̩˥}}} \textsubscript{2}}{}
\textcolor{teal}{\mytextsc{verbe}} \hspace{4pt} Ton~: H.

Enterrer.
\textcolor{brown}{\zh{埋。}}




\paragraph{\hspace{-0.5cm} {\Large \textcolor{darkblue}{\textbf{\ipa{nv̩˩dʑɯ˥}}}}} \hypertarget{nv\string_=\string_Bdz£M\string_T1}{}
\markboth{\textcolor{darkblue}{\textbf{\ipa{nv̩˩dʑɯ˥}}}}{}
\textcolor{teal}{\mytextsc{nom}} \hspace{4pt} Ton~: LH.

Tofu.
\textcolor{brown}{\zh{豆腐。}}


 \mytextsc{clf}~: \textcolor{darkblue}{\textbf{\ipa{v̩˥}}} casseroles

\paragraph{\hspace{-0.5cm} {\Large \textcolor{darkblue}{\textbf{\ipa{nv̩˩ho\#˥}}}}} \hypertarget{nv\string_=\string_Bho\#\string_T1}{}
\markboth{\textcolor{darkblue}{\textbf{\ipa{nv̩˩ho\#˥}}}}{}
\textcolor{teal}{\mytextsc{nom}} \hspace{4pt} Ton~: LM+\#H.

Tofu léger, longuement bouilli.
\textcolor{brown}{\zh{豆花。}}


 \mytextsc{clf}~: \textcolor{darkblue}{\textbf{\ipa{v̩˥}}} 

\paragraph{\hspace{-0.5cm} {\Large \textcolor{darkblue}{\textbf{\ipa{nv̩˧hṽ˩}}}}} \hypertarget{nv\string_=\string_Mhv\string_~\string_B1}{}
\markboth{\textcolor{darkblue}{\textbf{\ipa{nv̩˧hṽ˩}}}}{}
\textcolor{teal}{\mytextsc{nom}} \hspace{4pt} Ton~: L\#.

Haricot: terme générique.
\textcolor{brown}{\zh{豆子,四季豆,花腰豆。}}


 \mytextsc{clf}~: \textcolor{darkblue}{\textbf{\ipa{v̩˥}}} 

\paragraph{\hspace{-0.5cm} {\Large \textcolor{darkblue}{\textbf{\ipa{nv̩˧hṽ˩-bi˩bi˩}}}}} \hypertarget{nv\string_=\string_Mhv\string_~\string_B-bi\string_Bbi\string_B1}{}
\markboth{\textcolor{darkblue}{\textbf{\ipa{nv̩˧hṽ˩-bi˩bi˩}}}}{}
\textcolor{teal}{\mytextsc{nom}} \hspace{4pt} Ton~: L\#-.

Haricot vert; on consomme la cosse fraîche et la graine qu'il contient.
\textcolor{brown}{\zh{四季豆、玉豆、帶莢豌豆、菜豆、刀豆、豆角、敏豆仔、敏豆。}}


 \mytextsc{clf}~: \textcolor{darkblue}{\textbf{\ipa{kʰɤ˧˥}}} panier

\paragraph{\hspace{-0.5cm} {\Large \textcolor{darkblue}{\textbf{\ipa{nv̩˩ɭɯ˧}}}}} \hypertarget{nv\string_=\string_Bl\string_RM\string_M1}{}
\markboth{\textcolor{darkblue}{\textbf{\ipa{nv̩˩ɭɯ˧}}}}{}
\textcolor{teal}{\mytextsc{nom}} \hspace{4pt} Ton~: LM.

Soja.
\textcolor{brown}{\zh{黄豆。}}


 \mytextsc{clf}~: \textcolor{darkblue}{\textbf{\ipa{wɤ˩}}} une charge

\paragraph{\hspace{-0.5cm} {\Large \textcolor{darkblue}{\textbf{\ipa{nv̩˩mi˩}}}}} \hypertarget{nv\string_=\string_Bmi\string_B1}{}
\markboth{\textcolor{darkblue}{\textbf{\ipa{nv̩˩mi˩}}}}{}
\textcolor{teal}{\mytextsc{nom}} \hspace{4pt} Ton~: L.
\ding{202} 
Cœur.
\textcolor{brown}{\zh{心脏。}}


\begin{exe}
\sn \textcolor{darkblue}{\textbf{\ipa{hĩ˧ ʈʂʰɯ˧-v̩˧, | nv̩˩mi˩ tɕi˥! |}}}
\trans Celui-là, il manque de courage! (littéralement ``(a) un petit coeur", ``son coeur est petit")
\trans \textit{\textcolor{brown}{\zh{这个人,胆小!(直译:“心小”)}}}
\end{exe}
\begin{exe}
\sn \textcolor{darkblue}{\textbf{\ipa{nv̩˩mi˩˥ | ɖɯ˧-ɭɯ˧ tsɤ˧ |}}}
\trans être en sympathie, à l'unisson
\trans \textit{\textcolor{brown}{\zh{情投意合}}}
\end{exe}
\begin{exe}
\sn \textcolor{darkblue}{\textbf{\ipa{nv̩˩mi˩˥ | tʰi˧-tɕɯ˥ | so˩˥}}}
\trans enseigner patiemment/apprendre patiemment
\trans \textit{\textcolor{brown}{\zh{耐心地学习 / 耐心地教}}}
\end{exe}
\begin{exe}
\sn \textcolor{darkblue}{\textbf{\ipa{nv̩˩mi˩-qo˥ | tʰi˧-ʑi˥}}}
\trans se souvenir de, garder à l'esprit, avoir à l'esprit
\trans \textit{\textcolor{brown}{\zh{记住、记得(直译:‘心里存着’、‘心里有’)}}}
\end{exe}
\begin{exe}
\sn \textcolor{darkblue}{\textbf{\ipa{nv̩˩mi˩-qo˥ | tʰi˧-kʰɯ˧˥}}}
\trans faire l'effort de se souvenir, garder à l'esprit, garder en mémoire
\trans \textit{\textcolor{brown}{\zh{记住(直译:‘放在心里’)}}}
\end{exe}
\ding{203} 
État d'esprit.
\textcolor{brown}{\zh{心情。}}


\begin{exe}
\sn \textcolor{darkblue}{\textbf{\ipa{nv̩˩mi˩ dzɯ˩\textasciitilde{}dzɯ˩-ɻ̍˥}}}
\trans ne pas bien s'entendre; se chamailler sans répit; s'empoisonner mutuellement la vie, se faire la vie impossible, se bouffer le nez
\trans \textit{\textcolor{brown}{\zh{经常吵架、过不到一起去}}}
\end{exe}
 \mytextsc{clf}~: \textcolor{darkblue}{\textbf{\ipa{ɭɯ˧}}} 

\paragraph{\hspace{-0.5cm} {\Large \textcolor{darkblue}{\textbf{\ipa{nv̩˩mi˩-ɖɯ˩}}}}} \hypertarget{nv\string_=\string_Bmi\string_B-d`M\string_B1}{}
\markboth{\textcolor{darkblue}{\textbf{\ipa{nv̩˩mi˩-ɖɯ˩}}}}{}
\textcolor{teal}{\mytextsc{adjectif}} \hspace{4pt} Ton~: L.
\textit{De:} \textbf{nv̩˩mi˩ et ɖɯ˩a} 
Courageux, audacieux.
\textcolor{brown}{\zh{勇敢、有勇气的。}}


\begin{exe}
\sn \textcolor{darkblue}{\textbf{\ipa{ʈʂʰɯ˧ | nv̩˩mi˩˥ | ɖwæ˧˥ | ɖɯ˩˥! | hĩ˧ | mɤ˧-ɖwæ˥!}}}
\trans il est très courageux! il n'a peur de personne!
\trans \textit{\textcolor{brown}{\zh{他很勇敢!谁也不怕!}}}
\end{exe}
\begin{exe}
\sn \textcolor{darkblue}{\textbf{\ipa{pɤ˩mi˩˥ | nv̩˩mi˩ ɖɯ˩˥, | ʝi˧-ɳɯ˧ tʰv̩˧˥; | mi˩zɯ˩ nv̩˥mi˩ ɖɯ˩ (-dʑo˩), | hĩ˧-ɳɯ˧ lɑ˧˥!}}}
\trans ``La grenouille courageuse, elle se fait écraser par le bœuf; la femme courageuse, elle se fait frapper!" (Explication: les créatures qui ne sont pas les plus fortes doivent se garder d'être trop courageuses: la grenouille qui n'a peur de rien et s'aventure sur le grand chemin, elle se fait écraser; la femme qui se comporte avec une mâle assurance, elle finit par entrer dans des conflits où on en vient aux mains et où elle a finalement le dessous.)
\trans \textit{\textcolor{brown}{\zh{“勇敢的青蛙,被牛压死。勇敢的女人,被人家打!”(说明:青蛙太勇敢,上马路,就容易被压死,而女人太勇敢,容易跟别人发生矛盾,最后就打不过男人。)}}}
\end{exe}


\paragraph{\hspace{-0.5cm} {\Large \textcolor{darkblue}{\textbf{\ipa{nv̩˩mi˩-ki˧ki˩}}}}} \hypertarget{nv\string_=\string_Bmi\string_B-ki\string_Mki\string_B1}{}
\markboth{\textcolor{darkblue}{\textbf{\ipa{nv̩˩mi˩-ki˧ki˩}}}}{}
\textcolor{teal}{\mytextsc{adjectif}} \hspace{4pt} Ton~: .
\textit{De:} \textbf{nv̩˩mi˩ et ki˧a} 
En harmonie, à l'unisson.
\textcolor{brown}{\zh{心意相通。}}




\paragraph{\hspace{-0.5cm} {\Large \textcolor{darkblue}{\textbf{\ipa{nv̩˩mi˩-ʈʰi˩}}}}} \hypertarget{nv\string_=\string_Bmi\string_B-t`\string_hi\string_B1}{}
\markboth{\textcolor{darkblue}{\textbf{\ipa{nv̩˩mi˩-ʈʰi˩}}}}{}
\textcolor{teal}{\mytextsc{adjectif}} \hspace{4pt} Ton~: L.
\textit{De:} \textbf{nv̩˩mi˩ et ʈʰi˩a} 
Découragé, nostalgique, mélancolique.
\textcolor{brown}{\zh{累得没精神了。}}




\paragraph{\hspace{-0.5cm} {\Large \textcolor{darkblue}{\textbf{\ipa{nv̩˩pi˧}}}}} \hypertarget{nv\string_=\string_Bpi\string_M1}{}
\markboth{\textcolor{darkblue}{\textbf{\ipa{nv̩˩pi˧}}}}{}
\textcolor{teal}{\mytextsc{nom}} \hspace{4pt} Ton~: LM.

Tourteau de soja: le reste du soja, après qu'on en a tiré le lait de soja; sert de nourriture pour les porcs.
\textcolor{brown}{\zh{豆粕。}}




\paragraph{\hspace{-0.5cm} {\Large \textcolor{darkblue}{\textbf{\ipa{nv̩˧pɤ˩}}}}} \hypertarget{nv\string_=\string_Mp7\string_B1}{}
\markboth{\textcolor{darkblue}{\textbf{\ipa{nv̩˧pɤ˩}}}}{}
\textcolor{teal}{\mytextsc{nom}} \hspace{4pt} Ton~: L\#.

Fèves.
\textcolor{brown}{\zh{蚕豆。}}




\paragraph{\hspace{-0.5cm} {\Large \textcolor{darkblue}{\textbf{\ipa{nv̩˩tɕʰi\#˥}}}}} \hypertarget{nv\string_=\string_Bts£\string_hi\#\string_T1}{}
\markboth{\textcolor{darkblue}{\textbf{\ipa{nv̩˩tɕʰi\#˥}}}}{}
\textcolor{teal}{\mytextsc{nom}} \hspace{4pt} Ton~: LM+\#H.

Balle de légumineuse (fine, pour nourrir les bovidés).
\textcolor{brown}{\zh{豆类的细糠秕,来喂牛。}}


 \mytextsc{clf}~: \textcolor{darkblue}{\textbf{\ipa{kʰɤ˧˥}}} panier

\paragraph{\hspace{-0.5cm} {\Large \textcolor{darkblue}{\textbf{\ipa{nv̩˩tsɑ˧˥}}}}} \hypertarget{nv\string_=\string_BtsA\string_M\string_T1}{}
\markboth{\textcolor{darkblue}{\textbf{\ipa{nv̩˩tsɑ˧˥}}}}{}
\textcolor{teal}{\mytextsc{nom}} \hspace{4pt} Ton~: LM+MH\#.

Balle grossière de légumineuses.
\textcolor{brown}{\zh{粗的豆糠。}}


 \mytextsc{clf}~: \textcolor{darkblue}{\textbf{\ipa{mɤ˩, etc}}} 

\paragraph{\hspace{-0.5cm} {\Large \textcolor{darkblue}{\textbf{\ipa{nv̩˧tv̩˥}}}}} \hypertarget{nv\string_=\string_Mtv\string_=\string_T1}{}
\markboth{\textcolor{darkblue}{\textbf{\ipa{nv̩˧tv̩˥}}}}{}
\textcolor{teal}{\mytextsc{nom}} \hspace{4pt} Ton~: H\#.

Musette à grain, sac à grain: sac dans lequel on donnait à manger au cheval; le sac est pendu au cou du cheval.
\textcolor{brown}{\zh{(挂在马脖子下面的)饲料袋子、马粮袋子。}}


 \mytextsc{clf}~: \textcolor{darkblue}{\textbf{\ipa{ɭɯ˧}}} 

\paragraph{\hspace{-0.5cm} {\Large \textcolor{darkblue}{\textbf{\ipa{nv̩˩ze˧}}}}} \hypertarget{nv\string_=\string_Bze\string_M1}{}
\markboth{\textcolor{darkblue}{\textbf{\ipa{nv̩˩ze˧}}}}{}
\textcolor{teal}{\mytextsc{nom}} \hspace{4pt} Ton~: LM.

Pois chiche, \textit{Cicer arietinum}, de couleur noire, dont on prépare la spécialité de Dali: \textcolor{brown}{\zh{黑色凉粉}}.
\textcolor{brown}{\zh{鹰嘴豆、桃尔豆、鸡豆、鸡心豆。}}Dialecte chinois local~: \zh{鸡豆。}




\newpage
\section*{\centering- \textcolor{darkblue}{\textbf{\ipa{ɳ}}} -}
\paragraph{\hspace{-0.5cm} {\Large \textcolor{darkblue}{\textbf{\ipa{ɳæ˥}}}}} \hypertarget{n`\{\string_T1}{}
\markboth{\textcolor{darkblue}{\textbf{\ipa{ɳæ˥}}}}{}
\textcolor{teal}{\mytextsc{verbe}} \hspace{4pt} Ton~: H.

Se cacher.
\textcolor{brown}{\zh{躲藏。}}


\begin{exe}
\sn \textcolor{darkblue}{\textbf{\ipa{tʰi˧-ɳæ˥}}}
\trans \mytextsc{dur}
\trans \textit{\textcolor{brown}{\zh{\mytextsc{dur} \string_}}}
\end{exe}


\paragraph{\hspace{-0.5cm} {\Large \textcolor{darkblue}{\textbf{\ipa{ɳæ˧=ɻ̍˩}}}}} \hypertarget{n`\{\string_M=r£`̍\string_B1}{}
\markboth{\textcolor{darkblue}{\textbf{\ipa{ɳæ˧=ɻ̍˩}}}}{}
\textcolor{teal}{\mytextsc{pronom}} \hspace{4pt} Ton~: L\#.

Deuxième personne du pluriel: vous autres. Variante de \textcolor{darkblue}{\textbf{\ipa{/no˧=ɻ̍˩/}}}; la forme \textcolor{darkblue}{\textbf{\ipa{/no˧=ɻ̍˩/}}} est jugée plus correcte.
\textcolor{brown}{\zh{你们。这是\textcolor{darkblue}{\textbf{\ipa{/no˧=ɻ̍˩/}}}的一个变体。\textcolor{darkblue}{\textbf{\ipa{/no˧=ɻ̍˩/}}}被认为是更标准的。}}


\textit{Voir~:} \textcolor{darkblue}{\textbf{\ipa{ɳæ˩=ɻæ˧, no˧=ɻ̍˩}}} 

\paragraph{\hspace{-0.5cm} {\Large \textcolor{darkblue}{\textbf{\ipa{ɳæ˩=ɻæ˧}}}}} \hypertarget{n`\{\string_B=r£`\{\string_M1}{}
\markboth{\textcolor{darkblue}{\textbf{\ipa{ɳæ˩=ɻæ˧}}}}{}
\textcolor{teal}{\mytextsc{pronom}} \hspace{4pt} Ton~: LM.

Deuxième personne, pluriel associatif: vous autres.
\textcolor{brown}{\zh{你们家、你们家族。}}


\textit{Voir~:} \textcolor{darkblue}{\textbf{\ipa{ɳæ˧=ɻ̍˩, no˧=ɻ̍˩}}} 

\paragraph{\hspace{-0.5cm} {\Large \textcolor{darkblue}{\textbf{\ipa{ɳæ˧˥}}}}} \hypertarget{n`\{\string_M\string_T1}{}
\markboth{\textcolor{darkblue}{\textbf{\ipa{ɳæ˧˥}}}}{}
\textcolor{teal}{\mytextsc{verbe}} \hspace{4pt} Ton~: MH.
\ding{202} 
Aplatir; appuyer, peser sur; presser.
\textcolor{brown}{\zh{按(用手)、压扁、挤压。}}


\begin{exe}
\sn \textcolor{darkblue}{\textbf{\ipa{mv̩˩tɕo˧ ɳæ˧˥}}}
\trans appuyer vers le bas, peser sur
\trans \textit{\textcolor{brown}{\zh{往下按}}}
\end{exe}
\begin{exe}
\sn \textcolor{darkblue}{\textbf{\ipa{le˧-ɳæ˩\textasciitilde{}ɳæ˩}}}
\trans \mytextsc{accomp} \mytextsc{red}
\end{exe}
\ding{203} 
Opprimer, accabler, écraser de son autorité, en imposer par la violence.
\textcolor{brown}{\zh{压迫。}}


\begin{exe}
\sn \textcolor{darkblue}{\textbf{\ipa{hĩ˧ kʰv̩˧, | hĩ˧ ɳæ˩}}}
\trans voler et oppresser (description du comportement d'un despote)
\trans \textit{\textcolor{brown}{\zh{偷和迫(描述专制统治者的行为)}}}
\end{exe}


\paragraph{\hspace{-0.5cm} {\Large \textcolor{darkblue}{\textbf{\ipa{‑ɳɯ}}} \textsubscript{2}}} \hypertarget{‑n`M2}{}
\markboth{\textcolor{darkblue}{\textbf{\ipa{‑ɳɯ}}} \textsubscript{2}}{}
\textcolor{teal}{\mytextsc{conjonction}} \hspace{4pt} Ton~: .

Ou, ou alors.
\textcolor{brown}{\zh{或者、还是。}}




\paragraph{\hspace{-0.5cm} {\Large \textcolor{darkblue}{\textbf{\ipa{ɳɯ˥}}}}} \hypertarget{n`M\string_T1}{}
\markboth{\textcolor{darkblue}{\textbf{\ipa{ɳɯ˥}}}}{}
\textcolor{teal}{\mytextsc{adjectif}} \hspace{4pt} Ton~: H.

Peu, peu nombreux (dénombrable).
\textcolor{brown}{\zh{少。}}


\begin{exe}
\sn \textcolor{darkblue}{\textbf{\ipa{hĩ˧ ɳɯ˧}}}
\trans les gens sont peu nombreux
\trans \textit{\textcolor{brown}{\zh{好的,不多!不好的,就很多了!}}}
\end{exe}
\begin{exe}
\sn \textcolor{darkblue}{\textbf{\ipa{tso˧\textasciitilde{}tso˧ | ɳɯ˧-ze˩}}}
\trans il y a moins de choses, la quantité a diminué
\trans \textit{\textcolor{brown}{\zh{东西(变)少了}}}
\end{exe}
\begin{exe}
\sn \textcolor{darkblue}{\textbf{\ipa{dʑɤ˩-hĩ˩˥, | le˧-ɳɯ˥! | mɤ˧-dʑɤ˩-hĩ˩, | le˧-dʑɯ˧˥!}}}
\trans Les bons, il n'y en a guère; les médiocres, il y en a en quantité! (contexte: au sujet des établissements universitaires entre lesquels les lauréats du concours national d'entrée à l'université ont à choisir)
\trans \textit{\textcolor{brown}{\zh{好的少,不好的多!(关于大学:高考后,学生要报志愿)}}}
\end{exe}


\paragraph{\hspace{-0.5cm} {\Large \textcolor{darkblue}{\textbf{\ipa{‑ɳɯ˧}}} \textsubscript{1}}} \hypertarget{‑n`M\string_M1}{}
\markboth{\textcolor{darkblue}{\textbf{\ipa{‑ɳɯ˧}}} \textsubscript{1}}{}
\textcolor{teal}{\mytextsc{postposition}} \hspace{4pt} Ton~: M.

Ablatif, agent, et marqueur de topique.
\textcolor{brown}{\zh{离格,施动者,主题。接近汉语的‘由’。}}




\paragraph{\hspace{-0.5cm} {\Large \textcolor{darkblue}{\textbf{\ipa{ɳɯ˧ɕi˩}}}}} \hypertarget{n`M\string_Ms£i\string_B1}{}
\markboth{\textcolor{darkblue}{\textbf{\ipa{ɳɯ˧ɕi˩}}}}{}
\textcolor{teal}{\mytextsc{verbe}} \hspace{4pt} Ton~: .

Mignon, joli.
\textcolor{brown}{\zh{可爱。}}




\paragraph{\hspace{-0.5cm} {\Large \textcolor{darkblue}{\textbf{\ipa{ɳɯ˧go˧˥}}}}} \hypertarget{n`M\string_Mgo\string_M\string_T1}{}
\markboth{\textcolor{darkblue}{\textbf{\ipa{ɳɯ˧go˧˥}}}}{}
\textcolor{teal}{\mytextsc{adjectif}} \hspace{4pt} Ton~: .

Pitoyable, qui suscite la pitié.
\textcolor{brown}{\zh{可怜。}}


\begin{exe}
\sn \textcolor{darkblue}{\textbf{\ipa{no˧ | ʈʰɯ˧-ki˧ | ɖwæ˧˥ | ɳɯ˧go˧˥! / ...ɖwæ˧˥ | ɳɯ˧go˧ ʝi˥!}}}
\trans Tu la/le plains beaucoup! / Tu as vraiment pitié d'elle/de lui!
\trans \textit{\textcolor{brown}{\zh{你真的很可怜他。}}}
\end{exe}


\paragraph{\hspace{-0.5cm} {\Large \textcolor{darkblue}{\textbf{\ipa{ɳɯ˧˥}}}}} \hypertarget{n`M\string_M\string_T1}{}
\markboth{\textcolor{darkblue}{\textbf{\ipa{ɳɯ˧˥}}}}{}
\textcolor{teal}{\mytextsc{verbe}} \hspace{4pt} Ton~: MH.

Serrer.
\textcolor{brown}{\zh{拧。}}


\begin{exe}
\sn \textcolor{darkblue}{\textbf{\ipa{le˧-ɳɯ˧-ze˥}}}
\trans \mytextsc{accomp} \string_ \mytextsc{pfv}
\trans \textit{\textcolor{brown}{\zh{拧了}}}
\end{exe}
\begin{exe}
\sn \textcolor{darkblue}{\textbf{\ipa{ʁo˧qɑ˥ | ʈʰɯ˧-ɭɯ˧ | le˧-ɳɯ˧-qɑ˥-jo˩!}}}
\trans Serre donc ce couvercle! (celui d'un bocal en verre, utilisé comme verre)
\trans \textit{\textcolor{brown}{\zh{(你)拧一下盖子吧!}}}
\end{exe}


\paragraph{\hspace{-0.5cm} {\Large \textcolor{darkblue}{\textbf{\ipa{ɳv̩˩˧}}}}} \hypertarget{n`v\string_=\string_B\string_M1}{}
\markboth{\textcolor{darkblue}{\textbf{\ipa{ɳv̩˩˧}}}}{}
\textcolor{teal}{\mytextsc{nom}} \hspace{4pt} Ton~: LM.

Mite (insecte qui mange les vêments).
\textcolor{brown}{\zh{蛀虫。}}


 \mytextsc{clf}~: \textcolor{darkblue}{\textbf{\ipa{mi˩}}} 

\paragraph{\hspace{-0.5cm} {\Large \textcolor{darkblue}{\textbf{\ipa{ɳv̩˥}}}}} \hypertarget{n`v\string_=\string_T1}{}
\markboth{\textcolor{darkblue}{\textbf{\ipa{ɳv̩˥}}}}{}
\textcolor{teal}{\mytextsc{verbe}} \hspace{4pt} Ton~: H.
\ding{202} 
Sentir, renifler.
\textcolor{brown}{\zh{闻嗅。}}


\ding{203} 
Apprendre une nouvelle; être au courant de.
\textcolor{brown}{\zh{听到(消息)、风闻。}}


\begin{exe}
\sn \textcolor{darkblue}{\textbf{\ipa{mɤ˧-ɳv̩˥}}}
\trans \mytextsc{neg}: je ne suis pas au courant!
\trans \textit{\textcolor{brown}{\zh{(我)不知道这个消息!}}}
\end{exe}
\begin{exe}
\sn \textcolor{darkblue}{\textbf{\ipa{no˧ ə˧tso˧ ɳv̩˥?}}}
\trans Quelle nouvelle as-tu apprise?
\trans \textit{\textcolor{brown}{\zh{你听到了什么消息呢?}}}
\end{exe}


\newpage
\section*{\centering- \textcolor{darkblue}{\textbf{\ipa{ɲ}}} -}
\paragraph{\hspace{-0.5cm} {\Large \textcolor{darkblue}{\textbf{\ipa{ɲi˥}}} \textsubscript{1}}} \hypertarget{Ji\string_T1}{}
\markboth{\textcolor{darkblue}{\textbf{\ipa{ɲi˥}}} \textsubscript{1}}{}
\textcolor{teal}{\mytextsc{verbe}} \hspace{4pt} Ton~: H.

Écouter.
\textcolor{brown}{\zh{听。}}


\begin{exe}
\sn \textcolor{darkblue}{\textbf{\ipa{tʰi˧-ɲi˥}}}
\trans \mytextsc{dur}
\trans \textit{\textcolor{brown}{\zh{\mytextsc{dur}}}}
\end{exe}
\begin{exe}
\sn \textcolor{darkblue}{\textbf{\ipa{tso˧\textasciitilde{}tso˧ ɲi˧}}}
\trans écouter des choses
\trans \textit{\textcolor{brown}{\zh{听东西}}}
\end{exe}
\begin{exe}
\sn \textcolor{darkblue}{\textbf{\ipa{le˧-ɲi˥-ze˩}}}
\trans \mytextsc{accomp} \string_ \mytextsc{pfv}
\trans \textit{\textcolor{brown}{\zh{听了}}}
\end{exe}


\paragraph{\hspace{-0.5cm} {\Large \textcolor{darkblue}{\textbf{\ipa{ɲi˥}}} \textsubscript{2}}} \hypertarget{Ji\string_T2}{}
\markboth{\textcolor{darkblue}{\textbf{\ipa{ɲi˥}}} \textsubscript{2}}{}
\textcolor{teal}{\mytextsc{verbe}} \hspace{4pt} Ton~: H.

Emprunter (un objet).
\textcolor{brown}{\zh{向别人借。}}


\begin{exe}
\sn \textcolor{darkblue}{\textbf{\ipa{hĩ˧-ki˧ | tso˧\textasciitilde{}tso˧ ɲi˧ |}}}
\trans emprunter des choses à quelqu'un
\trans \textit{\textcolor{brown}{\zh{向别人借东西}}}
\end{exe}


\paragraph{\hspace{-0.5cm} {\Large \textcolor{darkblue}{\textbf{\ipa{ɲi˥}}} \textsubscript{3}}} \hypertarget{Ji\string_T3}{}
\markboth{\textcolor{darkblue}{\textbf{\ipa{ɲi˥}}} \textsubscript{3}}{}
\textcolor{teal}{\mytextsc{verbe}} \hspace{4pt} Ton~: H.

Échouer, perdre.
\textcolor{brown}{\zh{败、输。}}




\paragraph{\hspace{-0.5cm} {\Large \textcolor{darkblue}{\textbf{\ipa{ɲi˥\textsubscript{b}}}}}} \hypertarget{Ji\string_Tb1}{}
\markboth{\textcolor{darkblue}{\textbf{\ipa{ɲi˥\textsubscript{b}}}}}{}
\textcolor{teal}{\mytextsc{classificateur}} \hspace{4pt} Ton~: H\textsubscript{b}.

Un jour.
\textcolor{brown}{\zh{日、天。}}


\begin{exe}
\sn \textcolor{darkblue}{\textbf{\ipa{ɖɯ˧-ɲi˥}}}
\trans un jour
\trans \textit{\textcolor{brown}{\zh{一天}}}
\end{exe}


\paragraph{\hspace{-0.5cm} {\Large \textcolor{darkblue}{\textbf{\ipa{ɲi˧}}}}} \hypertarget{Ji\string_M1}{}
\markboth{\textcolor{darkblue}{\textbf{\ipa{ɲi˧}}}}{}
\textcolor{teal}{\mytextsc{adjectif}} \hspace{4pt} Ton~: M.

Rassasié, repu.
\textcolor{brown}{\zh{饱。}}


\begin{exe}
\sn \textcolor{darkblue}{\textbf{\ipa{le˧-ɲi˧-ze˧}}}
\trans \mytextsc{accomp} \string_ \mytextsc{pfv}
\trans \textit{\textcolor{brown}{\zh{饱了}}}
\end{exe}
\begin{exe}
\sn \textcolor{darkblue}{\textbf{\ipa{hɑ˧-ɲi˧(-ze˩)}}}
\trans (je) suis rassasié
\trans \textit{\textcolor{brown}{\zh{吃饱了。 / 吃饱饭了。}}}
\end{exe}
\begin{exe}
\sn \textcolor{darkblue}{\textbf{\ipa{njɤ˧ | le˧-ɲi˧-ze˧!}}}
\trans je suis rassasié
\trans \textit{\textcolor{brown}{\zh{我饱了。}}}
\end{exe}


\paragraph{\hspace{-0.5cm} {\Large \textcolor{darkblue}{\textbf{\ipa{ɲi˧\textsubscript{a}}}}}} \hypertarget{Ji\string_Ma1}{}
\markboth{\textcolor{darkblue}{\textbf{\ipa{ɲi˧\textsubscript{a}}}}}{}
\textcolor{teal}{\mytextsc{verbe}} \hspace{4pt} Ton~: M\textsubscript{a}.

Avoir besoin de, vouloir.
\textcolor{brown}{\zh{需要。}}


\begin{exe}
\sn \textcolor{darkblue}{\textbf{\ipa{no˧ | ə˩-ɲi˧? | mɤ˧-ɲi˧!}}}
\trans Tu en veux? - (Non,) je n'en veux pas/je n'en ai pas besoin!
\trans \textit{\textcolor{brown}{\zh{你要吗?- 不要!}}}
\end{exe}


\paragraph{\hspace{-0.5cm} {\Large \textcolor{darkblue}{\textbf{\ipa{ɲi˧dʑɯ˧}}}}} \hypertarget{Ji\string_Mdz£M\string_M1}{}
\markboth{\textcolor{darkblue}{\textbf{\ipa{ɲi˧dʑɯ˧}}}}{}
\textcolor{teal}{\mytextsc{nom}} \hspace{4pt} Ton~: H\#.

Pénis, organe sexuel masculin.
\textcolor{brown}{\zh{男生殖器。}}


 \mytextsc{clf}~: \textcolor{darkblue}{\textbf{\ipa{ɭɯ˧}}} 

\paragraph{\hspace{-0.5cm} {\Large \textcolor{darkblue}{\textbf{\ipa{ɲi˧gɤ\#˥}}}}} \hypertarget{Ji\string_Mg7\#\string_T1}{}
\markboth{\textcolor{darkblue}{\textbf{\ipa{ɲi˧gɤ\#˥}}}}{}
\textcolor{teal}{\mytextsc{nom}} \hspace{4pt} Ton~: \#H.

Nez.
\textcolor{brown}{\zh{鼻子。}}


 \mytextsc{clf}~: \textcolor{darkblue}{\textbf{\ipa{ɭɯ˧}}} 

\paragraph{\hspace{-0.5cm} {\Large \textcolor{darkblue}{\textbf{\ipa{ɲi˧gɤ˧-bæ˧˥}}}}} \hypertarget{Ji\string_Mg7\string_M-b\{\string_M\string_T1}{}
\markboth{\textcolor{darkblue}{\textbf{\ipa{ɲi˧gɤ˧-bæ˧˥}}}}{}
\textcolor{teal}{\mytextsc{nom}} \hspace{4pt} Ton~: MH\#.

Corde accrochée à l'anneau nasal, longe de vache; aussi utilisé par extension pour l'anneau nasal, pour lequel aucun terme propre n'existe.
\textcolor{brown}{\zh{牛鼻绳。也可以来指牛鼻圈。}}


 \mytextsc{clf}~: \textcolor{darkblue}{\textbf{\ipa{kʰɯ˩}}} 

\paragraph{\hspace{-0.5cm} {\Large \textcolor{darkblue}{\textbf{\ipa{ɲi˧gɤ˧-dʑɯ˧˥}}}}} \hypertarget{Ji\string_Mg7\string_M-dz£M\string_M\string_T1}{}
\markboth{\textcolor{darkblue}{\textbf{\ipa{ɲi˧gɤ˧-dʑɯ˧˥}}}}{}
\textcolor{teal}{\mytextsc{nom}} \hspace{4pt} Ton~: \mytextsc{MH}.

Mucus, morve.
\textcolor{brown}{\zh{鼻涕。}}




\paragraph{\hspace{-0.5cm} {\Large \textcolor{darkblue}{\textbf{\ipa{ɲi˧ɬi˧mi˧}}}}} \hypertarget{Ji\string_MKi\string_Mmi\string_M1}{}
\markboth{\textcolor{darkblue}{\textbf{\ipa{ɲi˧ɬi˧mi˧}}}}{}
\textcolor{teal}{\mytextsc{nom}} \hspace{4pt} Ton~: M.

Le deuxième mois.
\textcolor{brown}{\zh{二月。}}




\paragraph{\hspace{-0.5cm} {\Large \textcolor{darkblue}{\textbf{\ipa{ɲi˧mi˧}}}}} \hypertarget{Ji\string_Mmi\string_M1}{}
\markboth{\textcolor{darkblue}{\textbf{\ipa{ɲi˧mi˧}}}}{}
\textcolor{teal}{\mytextsc{nom}} \hspace{4pt} Ton~: M.
\ding{202} 
Soleil.
\textcolor{brown}{\zh{太阳。}}


\begin{exe}
\sn \textcolor{darkblue}{\textbf{\ipa{ɲi˧mi˧ tʰv̩˧}}}
\trans le soleil se lève
\trans \textit{\textcolor{brown}{\zh{太阳出来、日出}}}
\end{exe}
\ding{203} 
La journée; le temps.
\textcolor{brown}{\zh{日、时间。}}


 \mytextsc{clf}~: \textcolor{darkblue}{\textbf{\ipa{ɭɯ˧}}} 

\paragraph{\hspace{-0.5cm} {\Large \textcolor{darkblue}{\textbf{\ipa{ɲi˧mi˧dɑ˧dzɯ\#˥}}}}} \hypertarget{Ji\string_Mmi\string_MdA\string_MdzM\#\string_T1}{}
\markboth{\textcolor{darkblue}{\textbf{\ipa{ɲi˧mi˧dɑ˧dzɯ\#˥}}}}{}
\textcolor{teal}{\mytextsc{nom}} \hspace{4pt} Ton~: \#H.

Éclipse solaire.
\textcolor{brown}{\zh{日蚀。}}


\begin{exe}
\sn \textcolor{darkblue}{\textbf{\ipa{ɲi˧mi˧dɑ˧dzɯ˧ tʰv̩˧}}}
\trans il y a une éclipse de soleil
\trans \textit{\textcolor{brown}{\zh{有日蚀}}}
\end{exe}
\begin{exe}
\sn \textcolor{darkblue}{\textbf{\ipa{ɲi˧mi˧dɑ˧dzɯ˧ ɲi˥!}}}
\trans Oui, c'est bien une éclipse de soleil!
\trans \textit{\textcolor{brown}{\zh{是的,是日蚀!}}}
\end{exe}
 \mytextsc{clf}~: \textcolor{darkblue}{\textbf{\ipa{ʂɯ˩}}} 

\paragraph{\hspace{-0.5cm} {\Large \textcolor{darkblue}{\textbf{\ipa{ɲi˧mi˧-gv̩˩}}}}} \hypertarget{Ji\string_Mmi\string_M-gv\string_=\string_B1}{}
\markboth{\textcolor{darkblue}{\textbf{\ipa{ɲi˧mi˧-gv̩˩}}}}{}
\textcolor{teal}{\mytextsc{nom}} \hspace{4pt} Ton~: \mytextsc{L}.

Ouest; ``[la direction dans laquelle] le soleil se couche".
\textcolor{brown}{\zh{西方。}}


\begin{exe}
\sn \textcolor{darkblue}{\textbf{\ipa{ɲi˧mi˧gv̩˩-gi˩-dzɤ˩ se˩}}}
\trans marcher vers l'ouest
\trans \textit{\textcolor{brown}{\zh{往西边走}}}
\end{exe}


\paragraph{\hspace{-0.5cm} {\Large \textcolor{darkblue}{\textbf{\ipa{ɲi˧mi˧-kʰɯ˧ʂɯ˧}}}}} \hypertarget{Ji\string_Mmi\string_M-k\string_hM\string_Ms`M\string_M1}{}
\markboth{\textcolor{darkblue}{\textbf{\ipa{ɲi˧mi˧-kʰɯ˧ʂɯ˧}}}}{}
\textcolor{teal}{\mytextsc{nom}} \hspace{4pt} Ton~: M.

Rayons du soleil.
\textcolor{brown}{\zh{太阳的光线。}}


 \mytextsc{clf}~: \textcolor{darkblue}{\textbf{\ipa{kʰɯ˩}}} 

\paragraph{\hspace{-0.5cm} {\Large \textcolor{darkblue}{\textbf{\ipa{ɲi˧mi˧tʰv̩\#˥}}}}} \hypertarget{Ji\string_Mmi\string_Mt\string_hv\string_=\#\string_T1}{}
\markboth{\textcolor{darkblue}{\textbf{\ipa{ɲi˧mi˧tʰv̩\#˥}}}}{}
\textcolor{teal}{\mytextsc{nom}} \hspace{4pt} Ton~: \#H.

Est, orient.
\textcolor{brown}{\zh{东方。}}


\begin{exe}
\sn \textcolor{darkblue}{\textbf{\ipa{ɲi˧mi˧tʰv̩˧-gi˧}}}
\trans la direction de l'est
\trans \textit{\textcolor{brown}{\zh{东方方向}}}
\end{exe}
\begin{exe}
\sn \textcolor{darkblue}{\textbf{\ipa{ɲi˧mi˧tʰv̩˧-gi˧ | se˧}}}
\trans marcher vers l'est
\trans \textit{\textcolor{brown}{\zh{向东面走}}}
\end{exe}
\begin{exe}
\sn \textcolor{darkblue}{\textbf{\ipa{ɲi˧mi˧tʰv̩˧-gi˧ | dʑo˩˥}}}
\trans se trouver à l'est, habiter en Orient (contexte: la locutrice m'imagine, depuis l'Europe, pensant à elle, et disant: ``elle habite en Orient".
\trans \textit{\textcolor{brown}{\zh{住在东方(合作人想象我在欧洲,想着她说:‘她住在东方’。)}}}
\end{exe}


\paragraph{\hspace{-0.5cm} {\Large \textcolor{darkblue}{\textbf{\ipa{ɲi˧mi˧-ʈæ˧ʂɯ˧}}}}} \hypertarget{Ji\string_Mmi\string_M-t`\{\string_Ms`M\string_M1}{}
\markboth{\textcolor{darkblue}{\textbf{\ipa{ɲi˧mi˧-ʈæ˧ʂɯ˧}}}}{}
\textcolor{teal}{\mytextsc{nom}} \hspace{4pt} Ton~: M.

Tournesol.
\textcolor{brown}{\zh{葵花。}}


 \mytextsc{clf}~: \textcolor{darkblue}{\textbf{\ipa{dzi˩}}} 

\paragraph{\hspace{-0.5cm} {\Large \textcolor{darkblue}{\textbf{\ipa{ɲi˧nɑ˩}}}}} \hypertarget{Ji\string_MnA\string_B1}{}
\markboth{\textcolor{darkblue}{\textbf{\ipa{ɲi˧nɑ˩}}}}{}
\textcolor{teal}{\mytextsc{nom}} \hspace{4pt} Ton~: L\#.

Liane, rattan, vigne vierge, lierre.
\textcolor{brown}{\zh{藤子。}}




\paragraph{\hspace{-0.5cm} {\Large \textcolor{darkblue}{\textbf{\ipa{ɲi˧pʰv̩˩}}}}} \hypertarget{Ji\string_Mp\string_hv\string_=\string_B1}{}
\markboth{\textcolor{darkblue}{\textbf{\ipa{ɲi˧pʰv̩˩}}}}{}
\textcolor{teal}{\mytextsc{nom}} \hspace{4pt} Ton~: L\#.

Givre.
\textcolor{brown}{\zh{霜。}}


\begin{exe}
\sn \textcolor{darkblue}{\textbf{\ipa{ɲi˧pʰv̩˩ lɑ˩-ze˩}}}
\trans il y a du givre
\trans \textit{\textcolor{brown}{\zh{有霜}}}
\end{exe}


\paragraph{\hspace{-0.5cm} {\Large \textcolor{darkblue}{\textbf{\ipa{ɲi˧qʰv̩˧}}}}} \hypertarget{Ji\string_Mq\string_hv\string_=\string_M1}{}
\markboth{\textcolor{darkblue}{\textbf{\ipa{ɲi˧qʰv̩˧}}}}{}
\textcolor{teal}{\mytextsc{nom}} \hspace{4pt} Ton~: M.
\ding{202} 
Narine.
\textcolor{brown}{\zh{鼻孔。}}


\ding{203} 
Mucus, morve.
\textcolor{brown}{\zh{鼻涕。}}


 \mytextsc{clf}~: \textcolor{darkblue}{\textbf{\ipa{ɭɯ˧}}} 

\paragraph{\hspace{-0.5cm} {\Large \textcolor{darkblue}{\textbf{\ipa{ɲi˧se˩}}}}} \hypertarget{Ji\string_Mse\string_B1}{}
\markboth{\textcolor{darkblue}{\textbf{\ipa{ɲi˧se˩}}}}{}
\textcolor{teal}{\mytextsc{nom}} \hspace{4pt} Ton~: L\#.

Un village du bord du Lac.
\textcolor{brown}{\zh{小落水(村落名)。}}


\begin{exe}
\sn \textcolor{darkblue}{\textbf{\ipa{ɲi˧se˩, | nɑ˩-lɑ˧ ɲi˥!}}}
\trans Nhissei, c'est un village entièrement Na!
\trans \textit{\textcolor{brown}{\zh{小落水,是纯摩梭的一个村落!}}}
\end{exe}
\begin{exe}
\sn \textcolor{darkblue}{\textbf{\ipa{ɬi˧ki˧, | ɲi˧se˩, | tɑ˧dzi˩, | mv̩˧qʰwæ˩, | lɑ˧tʰɑ˧-di˧˥}}}
\trans Villages dans l'ordre, après la plaine de Yongning, ne comptant pas comme faisant partie de Yongning. Le dernier, \textcolor{darkblue}{\textbf{\ipa{/lɑ˧tʰɑ˧-di˧˥/}}}, désigne toute la région na au-delà du quatrième village.
\trans \textit{\textcolor{brown}{\zh{永宁到泸沽湖所经过的村落,依次是:里格、尼赛、大祖、木垮,然后到拉塔地(拉塔地指的是泸沽湖周边的摩梭地区,包括左所、洛水村等)}}}
\end{exe}


\paragraph{\hspace{-0.5cm} {\Large \textcolor{darkblue}{\textbf{\ipa{ɲi˧to˧}}}}} \hypertarget{Ji\string_Mto\string_M1}{}
\markboth{\textcolor{darkblue}{\textbf{\ipa{ɲi˧to˧}}}}{}
\textcolor{teal}{\mytextsc{nom}} \hspace{4pt} Ton~: M.

Bouche/pourtour de la bouche (autour des lèvre).
\textcolor{brown}{\zh{嘴巴,包括嘴周围的部位:颚等。}}


\begin{exe}
\sn \textcolor{darkblue}{\textbf{\ipa{ɲi˧to˧ ʈʂʰwæ˩}}}
\trans bavard (littéralement ``bouche rapide")
\trans \textit{\textcolor{brown}{\zh{多嘴、拉不断扯不断(直译:“嘴快”)}}}
\end{exe}
 \mytextsc{clf}~: \textcolor{darkblue}{\textbf{\ipa{kʰwɤ˥}}} 

\paragraph{\hspace{-0.5cm} {\Large \textcolor{darkblue}{\textbf{\ipa{ɲi˧tsi˧}}}}} \hypertarget{Ji\string_Mtsi\string_M1}{}
\markboth{\textcolor{darkblue}{\textbf{\ipa{ɲi˧tsi˧}}}}{}
\textcolor{teal}{\mytextsc{nombre}} \hspace{4pt} Ton~: .

20.
\textcolor{brown}{\zh{20。}}




\paragraph{\hspace{-0.5cm} {\Large \textcolor{darkblue}{\textbf{\ipa{ɲi˧tsi˧-ɖɯ˧˥}}}}} \hypertarget{Ji\string_Mtsi\string_M-d`M\string_M\string_T1}{}
\markboth{\textcolor{darkblue}{\textbf{\ipa{ɲi˧tsi˧-ɖɯ˧˥}}}}{}
\textcolor{teal}{\mytextsc{nombre}} \hspace{4pt} Ton~: \mytextsc{MH}\#.

21.
\textcolor{brown}{\zh{21。}}




\paragraph{\hspace{-0.5cm} {\Large \textcolor{darkblue}{\textbf{\ipa{ɲi˧tsi˧-gv̩˧}}}}} \hypertarget{Ji\string_Mtsi\string_M-gv\string_=\string_M1}{}
\markboth{\textcolor{darkblue}{\textbf{\ipa{ɲi˧tsi˧-gv̩˧}}}}{}
\textcolor{teal}{\mytextsc{nombre}} \hspace{4pt} Ton~: .

29.
\textcolor{brown}{\zh{29。}}




\paragraph{\hspace{-0.5cm} {\Large \textcolor{darkblue}{\textbf{\ipa{ɲi˧tsi˧-hõ˧˥}}}}} \hypertarget{Ji\string_Mtsi\string_M-ho\string_~\string_M\string_T1}{}
\markboth{\textcolor{darkblue}{\textbf{\ipa{ɲi˧tsi˧-hõ˧˥}}}}{}
\textcolor{teal}{\mytextsc{nombre}} \hspace{4pt} Ton~: \mytextsc{MH}\#.

28.
\textcolor{brown}{\zh{28。}}




\paragraph{\hspace{-0.5cm} {\Large \textcolor{darkblue}{\textbf{\ipa{ɲi˧tsi˧-ɲi˧˥}}}}} \hypertarget{Ji\string_Mtsi\string_M-Ji\string_M\string_T1}{}
\markboth{\textcolor{darkblue}{\textbf{\ipa{ɲi˧tsi˧-ɲi˧˥}}}}{}
\textcolor{teal}{\mytextsc{nombre}} \hspace{4pt} Ton~: \mytextsc{MH}\#.

22.
\textcolor{brown}{\zh{22。}}




\paragraph{\hspace{-0.5cm} {\Large \textcolor{darkblue}{\textbf{\ipa{ɲi˧tsi˧-ŋwɤ˧}}}}} \hypertarget{Ji\string_Mtsi\string_M-Nw7\string_M1}{}
\markboth{\textcolor{darkblue}{\textbf{\ipa{ɲi˧tsi˧-ŋwɤ˧}}}}{}
\textcolor{teal}{\mytextsc{nombre}} \hspace{4pt} Ton~: .

25.
\textcolor{brown}{\zh{25。}}




\paragraph{\hspace{-0.5cm} {\Large \textcolor{darkblue}{\textbf{\ipa{ɲi˧tsi˧-qʰv̩˧˥}}}}} \hypertarget{Ji\string_Mtsi\string_M-q\string_hv\string_=\string_M\string_T1}{}
\markboth{\textcolor{darkblue}{\textbf{\ipa{ɲi˧tsi˧-qʰv̩˧˥}}}}{}
\textcolor{teal}{\mytextsc{nombre}} \hspace{4pt} Ton~: \mytextsc{MH}\#.

26.
\textcolor{brown}{\zh{26。}}




\paragraph{\hspace{-0.5cm} {\Large \textcolor{darkblue}{\textbf{\ipa{ɲi˧tsi˧-so˩}}}}} \hypertarget{Ji\string_Mtsi\string_M-so\string_B1}{}
\markboth{\textcolor{darkblue}{\textbf{\ipa{ɲi˧tsi˧-so˩}}}}{}
\textcolor{teal}{\mytextsc{nombre}} \hspace{4pt} Ton~: \mytextsc{L}.

23.
\textcolor{brown}{\zh{23。}}




\paragraph{\hspace{-0.5cm} {\Large \textcolor{darkblue}{\textbf{\ipa{ɲi˧tsi˧-ʂɯ˧}}}}} \hypertarget{Ji\string_Mtsi\string_M-s`M\string_M1}{}
\markboth{\textcolor{darkblue}{\textbf{\ipa{ɲi˧tsi˧-ʂɯ˧}}}}{}
\textcolor{teal}{\mytextsc{nombre}} \hspace{4pt} Ton~: .

27.
\textcolor{brown}{\zh{27。}}




\paragraph{\hspace{-0.5cm} {\Large \textcolor{darkblue}{\textbf{\ipa{ɲi˧tsi˧-ʐv̩˧}}}}} \hypertarget{Ji\string_Mtsi\string_M-z`v\string_=\string_M1}{}
\markboth{\textcolor{darkblue}{\textbf{\ipa{ɲi˧tsi˧-ʐv̩˧}}}}{}
\textcolor{teal}{\mytextsc{nombre}} \hspace{4pt} Ton~: .

24.
\textcolor{brown}{\zh{24。}}




\paragraph{\hspace{-0.5cm} {\Large \textcolor{darkblue}{\textbf{\ipa{ɲi˧-ʈʂæ˧-ʑi˧˥}}}}} \hypertarget{Ji\string_M-t`s`\{\string_M-z£i\string_M\string_T1}{}
\markboth{\textcolor{darkblue}{\textbf{\ipa{ɲi˧-ʈʂæ˧-ʑi˧˥}}}}{}
\textcolor{teal}{\mytextsc{nom}} \hspace{4pt} Ton~: MH\#.

Bâtiment d'habitation; littéralement 'le bâtiment à deux étages', car c'est le seul qui ait des salles sur deux étages. Ce bâtiment se trouve face à l'entrée de la ferme.
\textcolor{brown}{\zh{二层房:农场里面的一栋楼,正对着农场大门。}}


\begin{exe}
\sn \textcolor{darkblue}{\textbf{\ipa{ɲi˧-ʈʂæ˧-ʑi˧-di˥}}}
\trans même sens
\trans \textit{\textcolor{brown}{\zh{同上}}}
\end{exe}
 \mytextsc{clf}~: \textcolor{darkblue}{\textbf{\ipa{ɭɯ˧}}} 

\paragraph{\hspace{-0.5cm} {\Large \textcolor{darkblue}{\textbf{\ipa{ɲi˧ze˧hæ̃˩ze˩}}}}} \hypertarget{Ji\string_Mze\string_Mh\{\string_~\string_Bze\string_B1}{}
\markboth{\textcolor{darkblue}{\textbf{\ipa{ɲi˧ze˧hæ̃˩ze˩}}}}{}
\textcolor{teal}{\mytextsc{nom}} \hspace{4pt} Ton~: \mytextsc{L}.

Hirondelle.
\textcolor{brown}{\zh{雨燕。}}


 \mytextsc{clf}~: \textcolor{darkblue}{\textbf{\ipa{mi˩}}} 

\paragraph{\hspace{-0.5cm} {\Large \textcolor{darkblue}{\textbf{\ipa{ɲi˧zo\#˥}}}}} \hypertarget{Ji\string_Mzo\#\string_T1}{}
\markboth{\textcolor{darkblue}{\textbf{\ipa{ɲi˧zo\#˥}}}}{}
\textcolor{teal}{\mytextsc{nom}} \hspace{4pt} Ton~: \#H.

Poisson.
\textcolor{brown}{\zh{鱼。}}


\begin{exe}
\sn \textcolor{darkblue}{\textbf{\ipa{ɲi˧zo˧ tʰv̩˧-mi˥\# / ɲi˧zo˧ tʰv̩˧-mi˧˥}}}
\trans \mytextsc{n}+\mytextsc{dem}+\mytextsc{clf}
\trans \textit{\textcolor{brown}{\zh{那条鱼}}}
\end{exe}
\begin{exe}
\sn \textcolor{darkblue}{\textbf{\ipa{ɲi˧zo˧-tɑ˧pv̩˥}}}
\trans poisson séché
\trans \textit{\textcolor{brown}{\zh{干鱼}}}
\end{exe}
 \mytextsc{clf}~: \textcolor{darkblue}{\textbf{\ipa{mi˩}}} 

\paragraph{\hspace{-0.5cm} {\Large \textcolor{darkblue}{\textbf{\ipa{ɲi˩}}}}} \hypertarget{Ji\string_B1}{}
\markboth{\textcolor{darkblue}{\textbf{\ipa{ɲi˩}}}}{}
\textcolor{teal}{\mytextsc{particule}} \textcolor{teal}{\mytextsc{de}} \textcolor{teal}{\mytextsc{discours}} \hspace{4pt} Ton~: L.

Particule indiquant la certitude; dérivée du verbe copule.
\textcolor{brown}{\zh{\mytextsc{肯定(°系词)。}}}




\paragraph{\hspace{-0.5cm} {\Large \textcolor{darkblue}{\textbf{\ipa{ɲi˩}}} \textsubscript{1}}} \hypertarget{Ji\string_B1}{}
\markboth{\textcolor{darkblue}{\textbf{\ipa{ɲi˩}}} \textsubscript{1}}{}
\textcolor{teal}{\mytextsc{verbe}} \hspace{4pt} Ton~: L\textsubscript{a}.

Serrer, tenir (ex.: tenir quelque chose serré sous le bras, serrer quelque chose entre les jambes).
\textcolor{brown}{\zh{夹、夹持。}}


\begin{exe}
\sn \textcolor{darkblue}{\textbf{\ipa{ɖɯ˧-ɲi˧\textasciitilde{}ɲi˥-ɻ̍˩}}}
\trans serrer un peu
\trans \textit{\textcolor{brown}{\zh{夹一点}}}
\end{exe}


\paragraph{\hspace{-0.5cm} {\Large \textcolor{darkblue}{\textbf{\ipa{ɲi˩}}} \textsubscript{2}}} \hypertarget{Ji\string_B2}{}
\markboth{\textcolor{darkblue}{\textbf{\ipa{ɲi˩}}} \textsubscript{2}}{}
\textcolor{teal}{\mytextsc{pronom}} \hspace{4pt} Ton~: L.

Pronom interrogatif: qui?
\textcolor{brown}{\zh{谁。}}


\begin{exe}
\sn \textcolor{darkblue}{\textbf{\ipa{ɲi˩ ɲi˧?}}}
\trans C'est qui?
\trans \textit{\textcolor{brown}{\zh{是谁?}}}
\end{exe}
\begin{exe}
\sn \textcolor{darkblue}{\textbf{\ipa{no˧ | ɲi˩ ɲi˧?}}}
\trans Qui êtes-vous?
\trans \textit{\textcolor{brown}{\zh{你是谁?}}}
\end{exe}
\begin{exe}
\sn \textcolor{darkblue}{\textbf{\ipa{ʈʂʰɯ˧ | ɲi˩ ɲi˧?}}}
\trans Qui est-ce?
\trans \textit{\textcolor{brown}{\zh{他是谁?}}}
\end{exe}
\begin{exe}
\sn \textcolor{darkblue}{\textbf{\ipa{no˧ | ɲi˩˥ | -ki˩ bi˩-pi˩, | ɖɯ˧-bæ˧ lɑ˧ ɲi˥!}}}
\trans Peu importe chez qui tu vas, c'est pareil partout!
\trans \textit{\textcolor{brown}{\zh{无论你去谁(家),都一样!}}}
\end{exe}
\begin{exe}
\sn \textcolor{darkblue}{\textbf{\ipa{no˧ | ɲi˩-ki˥ bi˩-pi˩, | ɖɯ˧-bæ˧ lɑ˧ ɲi˥!}}}
\trans Comme l'exemple précédent, avec une division en groupes tonals différente
\trans \textit{\textcolor{brown}{\zh{同上,声调段界不同}}}
\end{exe}


\paragraph{\hspace{-0.5cm} {\Large \textcolor{darkblue}{\textbf{\ipa{ɲi˩\textsubscript{a}}}} \textsubscript{1}}} \hypertarget{Ji\string_Ba1}{}
\markboth{\textcolor{darkblue}{\textbf{\ipa{ɲi˩\textsubscript{a}}}} \textsubscript{1}}{}
\textcolor{teal}{\mytextsc{verbe}} \hspace{4pt} Ton~: L\textsubscript{a}.

Tordre avec les doigts, enrouler, filer (pour fabriquer du fil de lin, pour tisser des vêtements).
\textcolor{brown}{\zh{捻,缠线。}}


\begin{exe}
\sn \textcolor{darkblue}{\textbf{\ipa{le˧-ɲi˩}}}
\trans \mytextsc{accomp}
\trans \textit{\textcolor{brown}{\zh{\mytextsc{accomp}}}}
\end{exe}
\begin{exe}
\sn \textcolor{darkblue}{\textbf{\ipa{sɑ˧ ɲi˥}}}
\trans tordre le chanvre/le lin (pour faire du fil)
\trans \textit{\textcolor{brown}{\zh{捻麻}}}
\end{exe}
\begin{exe}
\sn \textcolor{darkblue}{\textbf{\ipa{ɖɯ˧-ɲi˧\textasciitilde{}ɲi˥-ɻ̍˩}}}
\trans \mytextsc{délimitatif} \mytextsc{red} \mytextsc{inchoatif}
\trans \textit{\textcolor{brown}{\zh{捻一捻}}}
\end{exe}


\paragraph{\hspace{-0.5cm} {\Large \textcolor{darkblue}{\textbf{\ipa{ɲi˩\textsubscript{a}}}} \textsubscript{2}}} \hypertarget{Ji\string_Ba2}{}
\markboth{\textcolor{darkblue}{\textbf{\ipa{ɲi˩\textsubscript{a}}}} \textsubscript{2}}{}
\textcolor{teal}{\mytextsc{verbe}} \hspace{4pt} Ton~: L\textsubscript{a}.

S'abîmer, se casser; tomber en panne (ex.: appareil photo).
\textcolor{brown}{\zh{设备坏了。}}


\begin{exe}
\sn \textcolor{darkblue}{\textbf{\ipa{le˧-ɲi˩-ze˩}}}
\trans \mytextsc{accomp} \string_ \mytextsc{pfv}: c'est cassé!
\trans \textit{\textcolor{brown}{\zh{坏了!/破了!}}}
\end{exe}
\begin{exe}
\sn \textcolor{darkblue}{\textbf{\ipa{tso˧\textasciitilde{}tso˧ ɲi˥}}}
\trans casser des choses
\trans \textit{\textcolor{brown}{\zh{东西坏了}}}
\end{exe}


\paragraph{\hspace{-0.5cm} {\Large \textcolor{darkblue}{\textbf{\ipa{ɲi˩\textsubscript{a}}}} \textsubscript{3}}} \hypertarget{Ji\string_Ba3}{}
\markboth{\textcolor{darkblue}{\textbf{\ipa{ɲi˩\textsubscript{a}}}} \textsubscript{3}}{}
\textcolor{teal}{\mytextsc{verbe}} \hspace{4pt} Ton~: L\textsubscript{a}.

Verbe copule.
\textcolor{brown}{\zh{是\mytextsc{系词。}}}




\paragraph{\hspace{-0.5cm} {\Large \textcolor{darkblue}{\textbf{\ipa{ɲi˩bv̩˩}}}}} \hypertarget{Ji\string_Bbv\string_=\string_B1}{}
\markboth{\textcolor{darkblue}{\textbf{\ipa{ɲi˩bv̩˩}}}}{}
\textcolor{teal}{\mytextsc{nom}} \hspace{4pt} Ton~: L.

Criquet.
\textcolor{brown}{\zh{蟋蟀。}}


 \mytextsc{clf}~: \textcolor{darkblue}{\textbf{\ipa{mi˩}}} 

\paragraph{\hspace{-0.5cm} {\Large \textcolor{darkblue}{\textbf{\ipa{ɲi˩bv̩˩-ʂe˩sɑ˧}}}}} \hypertarget{Ji\string_Bbv\string_=\string_B-s`e\string_BsA\string_M1}{}
\markboth{\textcolor{darkblue}{\textbf{\ipa{ɲi˩bv̩˩-ʂe˩sɑ˧}}}}{}
\textcolor{teal}{\mytextsc{nom}} \hspace{4pt} Ton~: LM.

Libellule.
\textcolor{brown}{\zh{蜻蜓。}}


 \mytextsc{clf}~: \textcolor{darkblue}{\textbf{\ipa{mi˩}}} 

\paragraph{\hspace{-0.5cm} {\Large \textcolor{darkblue}{\textbf{\ipa{ɲi˩mɑ\#˥}}}}} \hypertarget{Ji\string_BmA\#\string_T1}{}
\markboth{\textcolor{darkblue}{\textbf{\ipa{ɲi˩mɑ\#˥}}}}{}
\textcolor{teal}{\mytextsc{nom}} \hspace{4pt} Ton~: LM+\#H.

Prénom masculin pour l'aîné des jumeaux (l'enfant né en premier).
\textcolor{brown}{\zh{男性名字,起给双胞胎中的老大。}}




\paragraph{\hspace{-0.5cm} {\Large \textcolor{darkblue}{\textbf{\ipa{ɲi˩pʰv̩˩}}}}} \hypertarget{Ji\string_Bp\string_hv\string_=\string_B1}{}
\markboth{\textcolor{darkblue}{\textbf{\ipa{ɲi˩pʰv̩˩}}}}{}
\textcolor{teal}{\mytextsc{nom}} \hspace{4pt} Ton~: L.

Une plante de montagne; la locutrice pense la reconnaître sur une photo de menthe aquatique, \textit{Mentha aquatica, Mentha hirsuta Huds.} mais ce n'est vraisemblablement pas la bonne identification.
\textcolor{brown}{\zh{一种植物。合作人看水薄荷的图片就觉得像这种植物,但很可能不是。李达珠等(2015:98)翻译为“野牡丹”但这好像也不准确。}}


\begin{exe}
\sn \textcolor{darkblue}{\textbf{\ipa{ɲi˩pʰv̩˩-bæ˥bæ˩}}}
\trans la fleur de cette plante
\trans \textit{\textcolor{brown}{\zh{这种植物的花}}}
\end{exe}


\paragraph{\hspace{-0.5cm} {\Large \textcolor{darkblue}{\textbf{\ipa{ɲi˩=ɻ̍˥}}}}} \hypertarget{Ji\string_B=r£`̍\string_T1}{}
\markboth{\textcolor{darkblue}{\textbf{\ipa{ɲi˩=ɻ̍˥}}}}{}
\textcolor{teal}{\mytextsc{pronom}} \hspace{4pt} Ton~: LM+H\#.

Pronom de 2e personne associatif: toi et les tiens.
\textcolor{brown}{\zh{第二人称,联想复数:你与周边的人(家人、家族、亲戚、朋友们……)。}}




\paragraph{\hspace{-0.5cm} {\Large \textcolor{darkblue}{\textbf{\ipa{ɲi˩tsɯ\#˥}}}}} \hypertarget{Ji\string_BtsM\#\string_T1}{}
\markboth{\textcolor{darkblue}{\textbf{\ipa{ɲi˩tsɯ\#˥}}}}{}
\textcolor{teal}{\mytextsc{nom}} \hspace{4pt} Ton~: LM+\#H.

Hmông (groupe ethnique).
\textcolor{brown}{\zh{苗族。}}


 \mytextsc{clf}~: \textcolor{darkblue}{\textbf{\ipa{v̩˧}}} 

\paragraph{\hspace{-0.5cm} {\Large \textcolor{darkblue}{\textbf{\ipa{ɲi˩ʈʂe˩}}}}} \hypertarget{Ji\string_Bt`s`e\string_B1}{}
\markboth{\textcolor{darkblue}{\textbf{\ipa{ɲi˩ʈʂe˩}}}}{}
\textcolor{teal}{\mytextsc{nom}} \hspace{4pt} Ton~: L.

Barre de porte: barre pour fermer la porte principale de la ferme.
\textcolor{brown}{\zh{门闩。}}


\begin{exe}
\sn \textcolor{darkblue}{\textbf{\ipa{ɲi˩ʈʂe˩ tʰi˥-kʰɯ˩, | tʰi˧-ʈæ˩!}}}
\trans On met la barre à la porte; on verrouille! / On met la barre à la porte, et c'est fermé!
\trans \textit{\textcolor{brown}{\zh{放门闩,(好好)锁(门)!}}}
\end{exe}


\paragraph{\hspace{-0.5cm} {\Large \textcolor{darkblue}{\textbf{\ipa{ɲi˧˥}}}}} \hypertarget{Ji\string_M\string_T1}{}
\markboth{\textcolor{darkblue}{\textbf{\ipa{ɲi˧˥}}}}{}
\textcolor{teal}{\mytextsc{nombre}} \hspace{4pt} Ton~: MH.

2.
\textcolor{brown}{\zh{2。}}




\newpage
\section*{\centering- \textcolor{darkblue}{\textbf{\ipa{ŋ}}} -}
\paragraph{\hspace{-0.5cm} {\Large \textcolor{darkblue}{\textbf{\ipa{ŋæ˧ʝi˩}}}}} \hypertarget{N\{\string_Mj££i\string_B1}{}
\markboth{\textcolor{darkblue}{\textbf{\ipa{ŋæ˧ʝi˩}}}}{}
\textcolor{teal}{\mytextsc{adjectif}} \hspace{4pt} Ton~: L\#.

À l'aise, dans le confort, dans l'abondance.
\textcolor{brown}{\zh{安逸(汉语借词)。}}

 Emprunt~: chinois \textcolor{brown}{\zh{安逸}}



\paragraph{\hspace{-0.5cm} {\Large \textcolor{darkblue}{\textbf{\ipa{ŋɤ˩ŋɤ˩}}}}} \hypertarget{N7\string_BN7\string_B1}{}
\markboth{\textcolor{darkblue}{\textbf{\ipa{ŋɤ˩ŋɤ˩}}}}{}
\textcolor{teal}{\mytextsc{nom}} \hspace{4pt} Ton~: L.

Palais.
\textcolor{brown}{\zh{上腭。}}


 \mytextsc{clf}~: \textcolor{darkblue}{\textbf{\ipa{kʰwɤ˥}}} 

\paragraph{\hspace{-0.5cm} {\Large \textcolor{darkblue}{\textbf{\ipa{ŋv̩˩}}}}} \hypertarget{Nv\string_=\string_B1}{}
\markboth{\textcolor{darkblue}{\textbf{\ipa{ŋv̩˩}}}}{}
\textcolor{teal}{\mytextsc{nom}} \hspace{4pt} Ton~: L.
\ding{202} 
Argent (métal).
\textcolor{brown}{\zh{银子。}}


\begin{exe}
\sn \textcolor{darkblue}{\textbf{\ipa{ŋv˧hæ̃˩/ or et ærgent càd ærgent, pætrimoine}}}
\trans argent, patrimoine, fortune; littéralement 'or et argent'
\trans \textit{\textcolor{brown}{\zh{金钱、钱财、财富。直译:‘银子与金子’}}}
\end{exe}
\ding{203} 
Argent (argent-papier et pièces de monnaie).
\textcolor{brown}{\zh{钱。}}




\paragraph{\hspace{-0.5cm} {\Large \textcolor{darkblue}{\textbf{\ipa{ŋv̩˩\textsubscript{a}}}}}} \hypertarget{Nv\string_=\string_Ba1}{}
\markboth{\textcolor{darkblue}{\textbf{\ipa{ŋv̩˩\textsubscript{a}}}}}{}
\textcolor{teal}{\mytextsc{verbe}} \hspace{4pt} Ton~: L\textsubscript{a}.

Pleurer.
\textcolor{brown}{\zh{哭。}}


\begin{exe}
\sn \textcolor{darkblue}{\textbf{\ipa{(tʰi˧-)ŋv̩˧\textasciitilde{}ŋv̩˥}}}
\trans \mytextsc{dur} \mytextsc{red}
\trans \textit{\textcolor{brown}{\zh{哭一场}}}
\end{exe}


\paragraph{\hspace{-0.5cm} {\Large \textcolor{darkblue}{\textbf{\ipa{ŋwæ˧qʰv̩˧}}}}} \hypertarget{Nw\{\string_Mq\string_hv\string_=\string_M1}{}
\markboth{\textcolor{darkblue}{\textbf{\ipa{ŋwæ˧qʰv̩˧}}}}{}
\textcolor{teal}{\mytextsc{nom}} \hspace{4pt} Ton~: M.

Four où on cuit les tuiles.
\textcolor{brown}{\zh{烧瓦的烤炉。}}


\begin{exe}
\sn \textcolor{darkblue}{\textbf{\ipa{ŋwæ˧qʰv̩˧ ʂɯ˧-ʑi˩}}}
\trans 'les sept familles du Four à tuiles': expression dont on désignait autrefois les gens du village de Alawa, du temps où il n'y avait là que sept familles
\trans \textit{\textcolor{brown}{\zh{‘瓦炉七家’:过去来指阿拉瓦村的人,当时那里只有七家住}}}
\end{exe}
\begin{exe}
\sn \textcolor{darkblue}{\textbf{\ipa{ə˧lɑ˧-ʁwɤ˧ | ŋwæ˧qʰv̩˧ | tsʰe˧ɲi˧ ʑi˩}}}
\trans ``les douze familles de Alawa": expression dont on désignait autrefois les gens du village de Alawa, du temps où le nombre de familles était passé de sept à douze par l'arrivée de nouveaux venus.
\trans \textit{\textcolor{brown}{\zh{‘阿拉瓦瓦炉十二家’:过去来指阿拉瓦村的人,当时那里住的人家,从七家已经增加到十二家}}}
\end{exe}
 \mytextsc{clf}~: \textcolor{darkblue}{\textbf{\ipa{ɭɯ˧}}} 

\paragraph{\hspace{-0.5cm} {\Large \textcolor{darkblue}{\textbf{\ipa{ŋwɤ˧}}}}} \hypertarget{Nw7\string_M1}{}
\markboth{\textcolor{darkblue}{\textbf{\ipa{ŋwɤ˧}}}}{}
\textcolor{teal}{\mytextsc{nombre}} \hspace{4pt} Ton~: M? H\#? (pas L).

5.
\textcolor{brown}{\zh{5。}}




\paragraph{\hspace{-0.5cm} {\Large \textcolor{darkblue}{\textbf{\ipa{ŋwɤ˧hɑ̃˩}}}}} \hypertarget{Nw7\string_MhA\string_~\string_B1}{}
\markboth{\textcolor{darkblue}{\textbf{\ipa{ŋwɤ˧hɑ̃˩}}}}{}
\textcolor{teal}{\mytextsc{nom}} \hspace{4pt} Ton~: L\#.

Nom d'une montagne au sud-ouest de Yongning.
\textcolor{brown}{\zh{位于永宁西南的一座山。}}


\begin{exe}
\sn \textcolor{darkblue}{\textbf{\ipa{kɤ˧mv̩˧˥, | æ˧ʂæ˧, | ŋwɤ˧hɑ̃˩, | ʂwæ˧gv̩\#˥, | nɑ˩tsʰi˩˥ | -tɕʰɤ˧pɤ˧mi\#˥, | qv̩˧ɻ̍˧-ʈʂʰɑ˧nɑ˥ |}}}
\trans Les six montagnes de Yongning qui portent un nom. Les autres sommets du voisinage n'ont pas une valeur symbolique comparable, et ne portent pas de nom communément utilisé.
\trans \textit{\textcolor{brown}{\zh{永宁地区有固定名字的六座山。其它的山,因为没有重要的象征意义,因此没有取名。}}}
\end{exe}


\paragraph{\hspace{-0.5cm} {\Large \textcolor{darkblue}{\textbf{\ipa{ŋwɤ˧pʰæ˧˥}}}}} \hypertarget{Nw7\string_Mp\string_h\{\string_M\string_T1}{}
\markboth{\textcolor{darkblue}{\textbf{\ipa{ŋwɤ˧pʰæ˧˥}}}}{}
\textcolor{teal}{\mytextsc{nom}} \hspace{4pt} Ton~: MH\#.

Tuile.
\textcolor{brown}{\zh{瓦(汉语借词)。}}

 Emprunt~: chinois \textcolor{brown}{\zh{瓦}}

 \mytextsc{clf}~: \textcolor{darkblue}{\textbf{\ipa{pʰæ˧˥}}} 

\paragraph{\hspace{-0.5cm} {\Large \textcolor{darkblue}{\textbf{\ipa{ŋwɤ˧qo˥}}}}} \hypertarget{Nw7\string_Mqo\string_T1}{}
\markboth{\textcolor{darkblue}{\textbf{\ipa{ŋwɤ˧qo˥}}}}{}
\textcolor{teal}{\mytextsc{nom}} \hspace{4pt} Ton~: H\#.

Genou.
\textcolor{brown}{\zh{膝盖。}}


 \mytextsc{clf}~: \textcolor{darkblue}{\textbf{\ipa{ɭɯ˧}}} \textit{Voir~:} \hyperlink{Nw7\string_BKv\string_=\string_M\string_T1}{\textcolor{darkblue}{\textbf{\ipa{ŋwɤ˩ɬv̩˧˥}}}} 

\paragraph{\hspace{-0.5cm} {\Large \textcolor{darkblue}{\textbf{\ipa{ŋwɤ˧tsʰi˩}}}}} \hypertarget{Nw7\string_Mts\string_hi\string_B1}{}
\markboth{\textcolor{darkblue}{\textbf{\ipa{ŋwɤ˧tsʰi˩}}}}{}
\textcolor{teal}{\mytextsc{nombre}} \hspace{4pt} Ton~: L\#.

50.
\textcolor{brown}{\zh{50。}}




\paragraph{\hspace{-0.5cm} {\Large \textcolor{darkblue}{\textbf{\ipa{ŋwɤ˩ɭɯ˧-tse˥pʰæ˩}}}}} \hypertarget{Nw7\string_Bl\string_RM\string_M-tse\string_Tp\string_h\{\string_B1}{}
\markboth{\textcolor{darkblue}{\textbf{\ipa{ŋwɤ˩ɭɯ˧-tse˥pʰæ˩}}}}{}
\textcolor{teal}{\mytextsc{nom}} \hspace{4pt} Ton~: LM+\#H-.

Os du genou.
\textcolor{brown}{\zh{膝盖骨。}}


 \mytextsc{clf}~: \textcolor{darkblue}{\textbf{\ipa{ɭɯ˧}}} 

\paragraph{\hspace{-0.5cm} {\Large \textcolor{darkblue}{\textbf{\ipa{ŋwɤ˩ɬi˩mi˩}}}}} \hypertarget{Nw7\string_BKi\string_Bmi\string_B1}{}
\markboth{\textcolor{darkblue}{\textbf{\ipa{ŋwɤ˩ɬi˩mi˩}}}}{}
\textcolor{teal}{\mytextsc{nom}} \hspace{4pt} Ton~: L.

5e mois.
\textcolor{brown}{\zh{五月。}}




\paragraph{\hspace{-0.5cm} {\Large \textcolor{darkblue}{\textbf{\ipa{ŋwɤ˩ɬv̩˧˥}}}}} \hypertarget{Nw7\string_BKv\string_=\string_M\string_T1}{}
\markboth{\textcolor{darkblue}{\textbf{\ipa{ŋwɤ˩ɬv̩˧˥}}}}{}
\textcolor{teal}{\mytextsc{nom}} \hspace{4pt} Ton~: LM+MH\#.

Genou, cartilages du genou, articulation du genou: littéralement ``moëlle du genou". L'expression insiste sur le caractère fragile de cette articulation.
\textcolor{brown}{\zh{膝盖(直译:“膝盖髓”)。这个说法强调膝盖的脆弱。}}


\begin{exe}
\sn \textcolor{darkblue}{\textbf{\ipa{[M23] ŋwɤ˩ɬv̩˧-ko˧lo˥ go˩.}}}
\trans avoir mal dans le genou
\trans \textit{\textcolor{brown}{\zh{膝盖疼。}}}
\end{exe}
 \mytextsc{clf}~: \textcolor{darkblue}{\textbf{\ipa{ɭɯ˧}}} \textit{Voir~:} \hyperlink{Nw7\string_Mqo\string_T1}{\textcolor{darkblue}{\textbf{\ipa{ŋwɤ˧qo˥}}}} 

\paragraph{\hspace{-0.5cm} {\Large \textcolor{darkblue}{\textbf{\ipa{ŋwɤ˧˥}}}}} \hypertarget{Nw7\string_M\string_T1}{}
\markboth{\textcolor{darkblue}{\textbf{\ipa{ŋwɤ˧˥}}}}{}
\textcolor{teal}{\mytextsc{verbe}} \hspace{4pt} Ton~: MH.

Percer, piquer.
\textcolor{brown}{\zh{刺痛。}}


\begin{exe}
\sn \textcolor{darkblue}{\textbf{\ipa{tɕʰi˧ ŋwɤ˩-ze˩}}}
\trans (Elle/il) a pris une écharde
\trans \textit{\textcolor{brown}{\zh{(他)被刺扎疼了。}}}
\end{exe}


\newpage
\section*{\centering- \textcolor{darkblue}{\textbf{\ipa{õ}}} -}
\paragraph{\hspace{-0.5cm} {\Large \textcolor{darkblue}{\textbf{\ipa{õ˧dɤ˧ɻ̍˧}}}}} \hypertarget{o\string_~\string_Md7\string_Mr£`̍\string_M1}{}
\markboth{\textcolor{darkblue}{\textbf{\ipa{õ˧dɤ˧ɻ̍˧}}}}{}
\textcolor{teal}{\mytextsc{nom}} \hspace{4pt} Ton~: M.

Fondement/fondamentalement.
\textcolor{brown}{\zh{根本。}}


\begin{exe}
\sn \textcolor{darkblue}{\textbf{\ipa{õ˧dɤ˧ɻ̍˧-ɳɯ˧, | hĩ˧ ʈʂʰɯ˧-v̩˧ | ʈʂʰɯ˧ne˧ gv̩˧˥ | -ɲi˩!}}}
\trans Voilà comment il se comporte en réalité/au fond! (Se dit de quelqu'un dont le comportement est irrespectueux des règles de savoir-vivre)
\trans \textit{\textcolor{brown}{\zh{他原来是这样做事情的! / 他原来这么不懂事!}}}
\end{exe}


\paragraph{\hspace{-0.5cm} {\Large \textcolor{darkblue}{\textbf{\ipa{õ˧tʰv̩˧ɲi˧}}}}} \hypertarget{o\string_~\string_Mt\string_hv\string_=\string_MJi\string_M1}{}
\markboth{\textcolor{darkblue}{\textbf{\ipa{õ˧tʰv̩˧ɲi˧}}}}{}
\textcolor{teal}{\mytextsc{nom}} \hspace{4pt} Ton~: .

Ce jour-là (il y a longtemps).
\textcolor{brown}{\zh{(早以前的)那天。}}




\paragraph{\hspace{-0.5cm} {\Large \textcolor{darkblue}{\textbf{\ipa{õ˧ʈʂwɤ˧}}}}} \hypertarget{o\string_~\string_Mt`s`w7\string_M1}{}
\markboth{\textcolor{darkblue}{\textbf{\ipa{õ˧ʈʂwɤ˧}}}}{}
\textcolor{teal}{\mytextsc{nom}} \hspace{4pt} Ton~: M.

Moustique.
\textcolor{brown}{\zh{蚊子。}}


\begin{exe}
\sn \textcolor{darkblue}{\textbf{\ipa{õ˧ʈʂwɤ˧ le˧-tʰv̩˧-ze˧!}}}
\trans voilà un moustique! / un moustique est entré (dans la pièce, sous la moustiquaire…)
\trans \textit{\textcolor{brown}{\zh{有一只蚊子!}}}
\end{exe}
\begin{exe}
\sn \textcolor{darkblue}{\textbf{\ipa{ʂɯ˧-ɬi˧mi˧, | õ˧ʈʂwɤ˧! |}}}
\trans Le septième mois, c'est un mois à moustiques!
\trans \textit{\textcolor{brown}{\zh{七月份,蚊子多! / 七月份,是蚊子多的一个月!}}}
\end{exe}
 \mytextsc{clf}~: \textcolor{darkblue}{\textbf{\ipa{mi˩}}} 

\paragraph{\hspace{-0.5cm} {\Large \textcolor{darkblue}{\textbf{\ipa{õ˧ʈʂwɤ˧-kv̩˧dʑɯ˧˥}}}}} \hypertarget{o\string_~\string_Mt`s`w7\string_M-kv\string_=\string_Mdz£M\string_M\string_T1}{}
\markboth{\textcolor{darkblue}{\textbf{\ipa{õ˧ʈʂwɤ˧-kv̩˧dʑɯ˧˥}}}}{}
\textcolor{teal}{\mytextsc{nom}} \hspace{4pt} Ton~: \mytextsc{MH}\#.

Moustiquaire.
\textcolor{brown}{\zh{蚊帐。}}


 \mytextsc{clf}~: \textcolor{darkblue}{\textbf{\ipa{ɭɯ˧}}} 

\paragraph{\hspace{-0.5cm} {\Large \textcolor{darkblue}{\textbf{\ipa{õ˧ʈʂʰɯ˧ne˧-ʝi˥}}}}} \hypertarget{o\string_~\string_Mt`s`\string_hM\string_Mne\string_M-j££i\string_T1}{}
\markboth{\textcolor{darkblue}{\textbf{\ipa{õ˧ʈʂʰɯ˧ne˧-ʝi˥}}}}{}
\textcolor{teal}{\mytextsc{adverbe}} \hspace{4pt} Ton~: MH\#.

De cette façon-là.
\textcolor{brown}{\zh{那样。}}




\paragraph{\hspace{-0.5cm} {\Large \textcolor{darkblue}{\textbf{\ipa{õ˩dv̩˧˥}}}}} \hypertarget{o\string_~\string_Bdv\string_=\string_M\string_T1}{}
\markboth{\textcolor{darkblue}{\textbf{\ipa{õ˩dv̩˧˥}}}}{}
\textcolor{teal}{\mytextsc{nom}} \hspace{4pt} Ton~: LM+MH\#.

Loup.
\textcolor{brown}{\zh{狼。}}


 \mytextsc{clf}~: \textcolor{darkblue}{\textbf{\ipa{mi˩}}} 

\paragraph{\hspace{-0.5cm} {\Large \textcolor{darkblue}{\textbf{\ipa{õ˩dv̩˧-kʰv̩˥mi˩}}}}} \hypertarget{o\string_~\string_Bdv\string_=\string_M-k\string_hv\string_=\string_Tmi\string_B1}{}
\markboth{\textcolor{darkblue}{\textbf{\ipa{õ˩dv̩˧-kʰv̩˥mi˩}}}}{}
\textcolor{teal}{\mytextsc{nom}} \hspace{4pt} Ton~: LM+MH\#-.

Chien-loup.
\textcolor{brown}{\zh{狼狗。}}


 \mytextsc{clf}~: \textcolor{darkblue}{\textbf{\ipa{mi˩}}} 

\paragraph{\hspace{-0.5cm} {\Large \textcolor{darkblue}{\textbf{\ipa{õ˩dv̩˧-mi˥}}}}} \hypertarget{o\string_~\string_Bdv\string_=\string_M-mi\string_T1}{}
\markboth{\textcolor{darkblue}{\textbf{\ipa{õ˩dv̩˧-mi˥}}}}{}
\textcolor{teal}{\mytextsc{nom}} \hspace{4pt} Ton~: LM+H\#.

Louve.
\textcolor{brown}{\zh{母狼。}}


 \mytextsc{clf}~: \textcolor{darkblue}{\textbf{\ipa{mi˩}}} 

\paragraph{\hspace{-0.5cm} {\Large \textcolor{darkblue}{\textbf{\ipa{õ˩dv̩˧-pʰv̩\#˥}}}}} \hypertarget{o\string_~\string_Bdv\string_=\string_M-p\string_hv\string_=\#\string_T1}{}
\markboth{\textcolor{darkblue}{\textbf{\ipa{õ˩dv̩˧-pʰv̩\#˥}}}}{}
\textcolor{teal}{\mytextsc{nom}} \hspace{4pt} Ton~: LM+\#H.

Loup mâle.
\textcolor{brown}{\zh{公狼。}}


 \mytextsc{clf}~: \textcolor{darkblue}{\textbf{\ipa{mi˩}}} 

\paragraph{\hspace{-0.5cm} {\Large \textcolor{darkblue}{\textbf{\ipa{õ˩dv̩˧-zo\#˥}}}}} \hypertarget{o\string_~\string_Bdv\string_=\string_M-zo\#\string_T1}{}
\markboth{\textcolor{darkblue}{\textbf{\ipa{õ˩dv̩˧-zo\#˥}}}}{}
\textcolor{teal}{\mytextsc{nom}} \hspace{4pt} Ton~: LM+\#H-.

Louveteau.
\textcolor{brown}{\zh{小狼。}}




\paragraph{\hspace{-0.5cm} {\Large \textcolor{darkblue}{\textbf{\ipa{õ˧˥}}}}} \hypertarget{o\string_~\string_M\string_T1}{}
\markboth{\textcolor{darkblue}{\textbf{\ipa{õ˧˥}}}}{}
\textcolor{teal}{\mytextsc{pronom}} \hspace{4pt} Ton~: MH.

Soi-même, propre.
\textcolor{brown}{\zh{自己。}}


\begin{exe}
\sn \textcolor{darkblue}{\textbf{\ipa{õ˧-ɑ˥ʁo˩}}}
\trans sa propre maison
\trans \textit{\textcolor{brown}{\zh{自己家}}}
\end{exe}
\begin{exe}
\sn \textcolor{darkblue}{\textbf{\ipa{õ˧-dʑɯ˥, õ˩ ʈʰɯ˩! |}}}
\trans Chacun boit sa propre boisson! (Contexte: un petit enfant s'empare du biberon d'un autre et s'apprête à boire; on l'en empêche.)
\trans \textit{\textcolor{brown}{\zh{自己喝自己的!(情景:一个婴儿抓另一个婴儿的奶瓶。)}}}
\end{exe}
\begin{exe}
\sn \textcolor{darkblue}{\textbf{\ipa{õ˧-ʂe˥, õ˩ ʈʰæ˩! |}}}
\trans chacun mange son propre morceau de viande! (Description des manières de table: dans le temps, on donnait un bout de viande à chacun et chacun mangeait son morceau, pas comme la coutume chinoise qui veut qu'on prélève bouchée par bouchée, avec ses baguettes, dans les bols/assiettes posés sur la table.)
\trans \textit{\textcolor{brown}{\zh{自己吃自己的(那块)肉!(关于饮食习惯:吃饭的时候,每人分得一块肉,自己吃完。当地人认为,汉族没有这种分吃的习惯。)}}}
\end{exe}
\begin{exe}
\sn \textcolor{darkblue}{\textbf{\ipa{õ˧-bv̩˥-õ˩ ʝi˩-ɳɯ˩ | sɯ˧-kv̩˩!}}}
\trans c'est en faisant soi-même qu'on apprend!
\trans \textit{\textcolor{brown}{\zh{自己做,就能学会!/ 要学会,就得自己熟练!}}}
\end{exe}
\begin{exe}
\sn \textcolor{darkblue}{\textbf{\ipa{õ˧-bv̩˥-õ˩ +N |}}}
\trans son propre N (soi-même+\mytextsc{poss}+soi-même)
\trans \textit{\textcolor{brown}{\zh{自己的(+名词)}}}
\end{exe}
\begin{exe}
\sn \textcolor{darkblue}{\textbf{\ipa{õ˧-bv̩˥-õ˩ ʐwæ˩}}}
\trans son propre cheval
\trans \textit{\textcolor{brown}{\zh{自己的马}}}
\end{exe}
\begin{exe}
\sn \textcolor{darkblue}{\textbf{\ipa{õ˧-bv̩˥-õ˩ ʝi˩}}}
\trans sa propre vache
\trans \textit{\textcolor{brown}{\zh{自己的牛}}}
\end{exe}
\begin{exe}
\sn \textcolor{darkblue}{\textbf{\ipa{õ˧-bv̩˥-õ˩ lv̩˩}}}
\trans son propre champ
\trans \textit{\textcolor{brown}{\zh{自己的田地}}}
\end{exe}
\begin{exe}
\sn \textcolor{darkblue}{\textbf{\ipa{õ˧-bv̩˥-õ˩ ɖʐe˩}}}
\trans son propre argent
\trans \textit{\textcolor{brown}{\zh{自己的钱}}}
\end{exe}
\begin{exe}
\sn \textcolor{darkblue}{\textbf{\ipa{õ˧mv̩˥-õ˩di˩}}}
\trans lieu de naissance, lieu d'origine
\trans \textit{\textcolor{brown}{\zh{出生的地方、老家、故乡}}}
\end{exe}
\begin{exe}
\sn \textcolor{darkblue}{\textbf{\ipa{hĩ˧-mv˥ hĩ˩-di˩ | qʰɑ˧-dʑɤ˥\textasciitilde{}dʑɤ˩, | õ˧-mv˥ õ˩-di˩ tsʰe˩ mɤ˩-gv˩!}}}
\trans Si belles soient les terres d'autrui, elles n'auront jamais la beauté de ses propres terres / de la terre natale !
\trans \textit{\textcolor{brown}{\zh{其他人的地方怎么好,也比不过自己的地方!}}}
\end{exe}
\begin{exe}
\sn \textcolor{darkblue}{\textbf{\ipa{õ˧-ə˧mv̩˥ / õ˧-ə˥mv̩˩ / õ˧-ə˧mv̩˧˥}}}
\trans son propre aîné (frère ou soeur)
\trans \textit{\textcolor{brown}{\zh{自家姐姐(或哥哥)}}}
\end{exe}
\begin{exe}
\sn \textcolor{darkblue}{\textbf{\ipa{õ˧-ə˧v̩˥ / õ˧-ə˥v̩˩}}}
\trans son propre oncle
\trans \textit{\textcolor{brown}{\zh{自家舅舅(母亲的兄弟)}}}
\end{exe}
\begin{exe}
\sn \textcolor{darkblue}{\textbf{\ipa{õ˧-ʐɤ˥mi˩, õ˩ ɲi˩! |}}}
\trans Chacun a son chemin! / Chacun vit sa vie! / A chacun sa destinée!
\trans \textit{\textcolor{brown}{\zh{自己的道路,就是自己!/ 每个人有自己的命运!}}}
\end{exe}


\newpage
\section*{\centering- \textcolor{darkblue}{\textbf{\ipa{p}}} -}
\paragraph{\hspace{-0.5cm} {\Large \textcolor{darkblue}{\textbf{\ipa{pɑ˧tɕɤ˧}}}}} \hypertarget{pA\string_Mts£7\string_M1}{}
\markboth{\textcolor{darkblue}{\textbf{\ipa{pɑ˧tɕɤ˧}}}}{}
\textcolor{teal}{\mytextsc{nom}} \hspace{4pt} Ton~: M.

Bananier plantain.
\textcolor{brown}{\zh{芭蕉(汉语借词)。}}

 Emprunt~: chinois \textcolor{brown}{\zh{芭蕉}}



\paragraph{\hspace{-0.5cm} {\Large \textcolor{darkblue}{\textbf{\ipa{pæ˥}}}}} \hypertarget{p\{\string_T1}{}
\markboth{\textcolor{darkblue}{\textbf{\ipa{pæ˥}}}}{}
\textcolor{teal}{\mytextsc{verbe}} \hspace{4pt} Ton~: H.

Déménager.
\textcolor{brown}{\zh{搬(家)。}}

 Emprunt~: chinois \textcolor{brown}{\zh{搬?}}



\paragraph{\hspace{-0.5cm} {\Large \textcolor{darkblue}{\textbf{\ipa{pæ˥\textsubscript{a}}}}}} \hypertarget{p\{\string_Ta1}{}
\markboth{\textcolor{darkblue}{\textbf{\ipa{pæ˥\textsubscript{a}}}}}{}
\textcolor{teal}{\mytextsc{classificateur}} \hspace{4pt} Ton~: H\textsubscript{a}.

Troupe (de chevaux, de soldats…).
\textcolor{brown}{\zh{量词:马、军人……(一队)。}}




\paragraph{\hspace{-0.5cm} {\Large \textcolor{darkblue}{\textbf{\ipa{pæ˧kʰwɤ\#˥}}}}} \hypertarget{p\{\string_Mk\string_hw7\#\string_T1}{}
\markboth{\textcolor{darkblue}{\textbf{\ipa{pæ˧kʰwɤ\#˥}}}}{}
\textcolor{teal}{\mytextsc{nom}} \hspace{4pt} Ton~: \#H.

Pièce d'argent de l'époque impériale.
\textcolor{brown}{\zh{民国之前的银币。}}


\begin{exe}
\sn \textcolor{darkblue}{\textbf{\ipa{ə˧mi˧! | pæ˧kʰwɤ˧ so˧-ɭɯ˥ ki˩-mæ˩!}}}
\trans Waouuu! [Il/elle] te donne trois pièces d'argent! (D'après le souvenir qu'en a la consultante principale, c'est le type de commentaire que faisaient autrefois les tantes ou oncles d'un enfant à qui on offrait une forte somme d'argent à l'occasion de son passage à l'âge adulte, à treize ans. Cela correspondrait aujourd'hui à la moitié d'un mois de salaire. Donner une seule pièce, c'est symboliquement inapproprié: on offre par paires. Donner deux pièces, c'est un beau cadeau, approprié et suffisant. Donner trois pièces, c'est un cadeau considérable, qui dépasse les attentes.)
\trans \textit{\textcolor{brown}{\zh{哇!(他)给三块银币!(在一个孩子成年时,亲戚会给银币。给一块,不合适,因为礼物不能只给一个,要给两个。给两块银币,是合适的,也是够的。给三块银币,超出期望,是大礼物了。按现在的标准/说法,三个银币等于半个月的工资左右。)}}}
\end{exe}
\begin{exe}
\sn \textcolor{darkblue}{\textbf{\ipa{pæ˧kʰwɤ˧ ɖɯ˧-ɭɯ˥\# ; pæ˧kʰwɤ˧ ɲi˧-ɭɯ˥\# ; pæ˧kʰwɤ˧ so˧-ɭɯ˥\#}}}
\trans une pièce d'argent; deux pièces d'argent; trois pièces d'argent
\trans \textit{\textcolor{brown}{\zh{一块银币,两块银币,三块银币}}}
\end{exe}
\begin{exe}
\sn \textcolor{darkblue}{\textbf{\ipa{pæ˧kʰwɤ˧ ɖɯ˧-ki˩tɑ˩}}}
\trans un sac de pièces d'argent, destiné à être caché/enterré
\trans \textit{\textcolor{brown}{\zh{一包银币(埋在地里,为了藏)}}}
\end{exe}
 \mytextsc{clf}~: \textcolor{darkblue}{\textbf{\ipa{ɭɯ˧}}} 

\paragraph{\hspace{-0.5cm} {\Large \textcolor{darkblue}{\textbf{\ipa{pæ˧li˩}}}}} \hypertarget{p\{\string_Mli\string_B1}{}
\markboth{\textcolor{darkblue}{\textbf{\ipa{pæ˧li˩}}}}{}
\textcolor{teal}{\mytextsc{nom}} \hspace{4pt} Ton~: L\#.

Châtaigne.
\textcolor{brown}{\zh{板栗。}}

 Emprunt~: chinois \textcolor{brown}{\zh{板栗}}

\begin{exe}
\sn \textcolor{darkblue}{\textbf{\ipa{pæ˧li˩-si˩dzi˩}}}
\trans châtaignier
\trans \textit{\textcolor{brown}{\zh{板栗树}}}
\end{exe}
\begin{exe}
\sn \textcolor{darkblue}{\textbf{\ipa{pæ˧li˩-dzi˩}}}
\trans châtaignier
\trans \textit{\textcolor{brown}{\zh{板栗树}}}
\end{exe}


\paragraph{\hspace{-0.5cm} {\Large \textcolor{darkblue}{\textbf{\ipa{pæ˧ɻæ˩-ʈʂʰo˩}}}}} \hypertarget{p\{\string_Mr£`\{\string_B-t`s`\string_ho\string_B1}{}
\markboth{\textcolor{darkblue}{\textbf{\ipa{pæ˧ɻæ˩-ʈʂʰo˩}}}}{}
\textcolor{teal}{\mytextsc{nom}} \hspace{4pt} Ton~: L\#-.

Hongqiao, village sur la route entre Ninglang et Yongning (principalement peuplé de Chinois Han).
\textcolor{brown}{\zh{红桥。}}


\begin{exe}
\sn \textcolor{darkblue}{\textbf{\ipa{no˧ | pæ˧ɻæ˩ʈʂʰo˩-hĩ˩-ni˩-zo˩!}}}
\trans ``Tu ressembles à quelqu'un de Hongqiao!" Insulte, pour dire de quelqu'un qu'il a un physique disgracieux. La géographie populaire na attribuait des traits grossiers aux gens de Hongqiao (localité que traversaient les caravanes): gros nez camus, en particulier.
\trans \textit{\textcolor{brown}{\zh{解放前用的侮辱语句:“你像红桥人!”=“你很丑!”摩梭民间文化中,红桥(马帮路过的一个乡)的人被认为难看,面貌不“眉清目秀”,比如有扁鼻子。}}}
\end{exe}


\paragraph{\hspace{-0.5cm} {\Large \textcolor{darkblue}{\textbf{\ipa{pæ˧sɯ˧}}}}} \hypertarget{p\{\string_MsM\string_M1}{}
\markboth{\textcolor{darkblue}{\textbf{\ipa{pæ˧sɯ˧}}}}{}
\textcolor{teal}{\mytextsc{nom}} \hspace{4pt} Ton~: M.

Rang (le plus bas) dans la hiérarchie des fonctionnaires féodaux.
\textcolor{brown}{\zh{把事(封建官员系统中的最低等级)(汉语借词)。}}

 Emprunt~: chinois \textcolor{brown}{\zh{把事}}



\paragraph{\hspace{-0.5cm} {\Large \textcolor{darkblue}{\textbf{\ipa{pæ˧te˩}}}}} \hypertarget{p\{\string_Mte\string_B1}{}
\markboth{\textcolor{darkblue}{\textbf{\ipa{pæ˧te˩}}}}{}
\textcolor{teal}{\mytextsc{nom}} \hspace{4pt} Ton~: L\#.

Banc, tabouret.
\textcolor{brown}{\zh{板凳。}}

 Emprunt~: chinois \textcolor{brown}{\zh{板凳}}

 \mytextsc{clf}~: \textcolor{darkblue}{\textbf{\ipa{ɭɯ˧}}} 

\paragraph{\hspace{-0.5cm} {\Large \textcolor{darkblue}{\textbf{\ipa{pæ˩\textsubscript{a}}}}}} \hypertarget{p\{\string_Ba1}{}
\markboth{\textcolor{darkblue}{\textbf{\ipa{pæ˩\textsubscript{a}}}}}{}
\textcolor{teal}{\mytextsc{verbe}} \hspace{4pt} Ton~: M\textsubscript{a}.

Mettre (la table), servir.
\textcolor{brown}{\zh{摆桌子、供应饭菜。}}

 Emprunt~: chinois \textcolor{brown}{\zh{摆?}}

\begin{exe}
\sn \textcolor{darkblue}{\textbf{\ipa{hɑ˧ tʰi˧-pæ˩ tsæ˩-ɲi˩-ze˩! | hɑ˧ dzɯ˧-bi˧-ze˩!}}}
\trans C'est servi! A table!
\trans \textit{\textcolor{brown}{\zh{饭摆好了!吃饭了!}}}
\end{exe}


\paragraph{\hspace{-0.5cm} {\Large \textcolor{darkblue}{\textbf{\ipa{pæ˩pʰæ˧˥}}} \textsubscript{1}}} \hypertarget{p\{\string_Bp\string_h\{\string_M\string_T1}{}
\markboth{\textcolor{darkblue}{\textbf{\ipa{pæ˩pʰæ˧˥}}} \textsubscript{1}}{}
\textcolor{teal}{\mytextsc{nom}} \hspace{4pt} Ton~: LM+MH\#.
\ding{202} 
Grosse planche de bois, épaisse d'une dizaine de centimètres, utilisée pour la charpente des maisons.
\textcolor{brown}{\zh{厚的木板、 木板子。}}


\ding{203} 
Herse en bois, qui consiste essentiellement en une grosse pièce de bois, sans dents, d'où l'emploi (par extension) du terme qui signifie ``planche".
\textcolor{brown}{\zh{耙。}}


 \mytextsc{clf}~: \textcolor{darkblue}{\textbf{\ipa{pʰæ˧˥}}} 

\paragraph{\hspace{-0.5cm} {\Large \textcolor{darkblue}{\textbf{\ipa{pæ˩pʰæ˧˥}}} \textsubscript{2}}} \hypertarget{p\{\string_Bp\string_h\{\string_M\string_T2}{}
\markboth{\textcolor{darkblue}{\textbf{\ipa{pæ˩pʰæ˧˥}}} \textsubscript{2}}{}
\textcolor{teal}{\mytextsc{nom}} \hspace{4pt} Ton~: LM+MH\#.

Prénom masculin.
\textcolor{brown}{\zh{男性名字。}}




\paragraph{\hspace{-0.5cm} {\Large \textcolor{darkblue}{\textbf{\ipa{pæ˧˥}}} \textsubscript{1}}} \hypertarget{p\{\string_M\string_T1}{}
\markboth{\textcolor{darkblue}{\textbf{\ipa{pæ˧˥}}} \textsubscript{1}}{}
\textcolor{teal}{\mytextsc{verbe}} \hspace{4pt} Ton~: MH.

Cultiver (une terre).
\textcolor{brown}{\zh{种(地)。}}




\paragraph{\hspace{-0.5cm} {\Large \textcolor{darkblue}{\textbf{\ipa{pæ˧˥}}} \textsubscript{2}}} \hypertarget{p\{\string_M\string_T2}{}
\markboth{\textcolor{darkblue}{\textbf{\ipa{pæ˧˥}}} \textsubscript{2}}{}
\textcolor{teal}{\mytextsc{verbe}} \hspace{4pt} Ton~: MH.

Dépasser, outrepasser; laisser passer (une occasion).
\textcolor{brown}{\zh{超过,错过。}}


\begin{exe}
\sn \textcolor{darkblue}{\textbf{\ipa{pæ˧˥ | -kʰɯ˩-pi˩, | mɤ˧-tsɤ˧! |}}}
\trans Ce n'est pas bien de laisser passer (un jour propice: pour la construction d'une maison, par exemple)!
\trans \textit{\textcolor{brown}{\zh{错过(一个吉日),不好!}}}
\end{exe}
\begin{exe}
\sn \textcolor{darkblue}{\textbf{\ipa{pæ˧˥ | -tʰɑ˧-kʰɯ˩}}}
\trans Il ne faut pas laisser passer/filer (une occasion/un moment propice)!
\trans \textit{\textcolor{brown}{\zh{不要错过(机会)!}}}
\end{exe}
\begin{exe}
\sn \textcolor{darkblue}{\textbf{\ipa{le˧-pæ˧-ze˥!}}}
\trans (On) a laissé filer (une occasion)/ c'est passé, c'est trop tard!
\trans \textit{\textcolor{brown}{\zh{错过了!}}}
\end{exe}


\paragraph{\hspace{-0.5cm} {\Large \textcolor{darkblue}{\textbf{\ipa{pæ˧˥hwɤ˧}}}}} \hypertarget{p\{\string_M\string_Thw7\string_M1}{}
\markboth{\textcolor{darkblue}{\textbf{\ipa{pæ˧˥hwɤ˧}}}}{}
\textcolor{teal}{\mytextsc{nom}} \hspace{4pt} Ton~: MH.M.

Solution, méthode (emprunt chinois ancien).
\textcolor{brown}{\zh{办法(早期汉语借词)。}}

 Emprunt~: chinois \textcolor{brown}{\zh{办法}}

\begin{exe}
\sn \textcolor{darkblue}{\textbf{\ipa{ʈʂʰɯ˧ | pæ˧˥hwɤ˧ | ɕjɤ˩ ɣɯ˧ (+ | ʐwæ˩˥)!}}}
\trans Il/elle excelle à trouver des solutions/ il a une solution à tout!
\trans \textit{\textcolor{brown}{\zh{他很会想办法的!}}}
\end{exe}
 \mytextsc{clf}~: \textcolor{darkblue}{\textbf{\ipa{kʰwɤ˥}}} 

\paragraph{\hspace{-0.5cm} {\Large \textcolor{darkblue}{\textbf{\ipa{pe˧ʂe˧}}}}} \hypertarget{pe\string_Ms`e\string_M1}{}
\markboth{\textcolor{darkblue}{\textbf{\ipa{pe˧ʂe˧}}}}{}
\textcolor{teal}{\mytextsc{adverbe}} \hspace{4pt} Ton~: M.

En soi.
\textcolor{brown}{\zh{本身(汉语借词)。}}

 Emprunt~: chinois \textcolor{brown}{\zh{本身}}



\paragraph{\hspace{-0.5cm} {\Large \textcolor{darkblue}{\textbf{\ipa{pɤ˥}}}}} \hypertarget{p7\string_T1}{}
\markboth{\textcolor{darkblue}{\textbf{\ipa{pɤ˥}}}}{}
\textcolor{teal}{\mytextsc{nom}} \hspace{4pt} Ton~: \#H.

Dessin, peinture.
\textcolor{brown}{\zh{画。}}


 \mytextsc{clf}~: \textcolor{darkblue}{\textbf{\ipa{pɤ˥}}} \textcolor{darkblue}{\textbf{\ipa{pʰæ˧˥}}} 

\paragraph{\hspace{-0.5cm} {\Large \textcolor{darkblue}{\textbf{\ipa{pɤ˥}}}}} \hypertarget{p7\string_T1}{}
\markboth{\textcolor{darkblue}{\textbf{\ipa{pɤ˥}}}}{}
\textcolor{teal}{\mytextsc{verbe}} \hspace{4pt} Ton~: H.

S'accroupir, se mettre en boule, se recroqueviller sur soi-même.
\textcolor{brown}{\zh{蜷曲、蜷缩。}}


\begin{exe}
\sn \textcolor{darkblue}{\textbf{\ipa{æ˩ ʈʂʰɯ˧-mi˥ | si˧dzi˩-ʈʰæ˩qo˩ | tʰi˧-pɤ˥-dʑo˩!}}}
\trans La poule est recroquevillée sous l'arbre/est accroupie sous l'arbre!
\trans \textit{\textcolor{brown}{\zh{那只鸡,在树下蜷缩着!}}}
\end{exe}
\begin{exe}
\sn \textcolor{darkblue}{\textbf{\ipa{ʈʂʰɯ˧-qo˧ ɖɯ˧-pɤ˥ ɕjɤ˩-ɻ̍˩!}}}
\trans Viens t'allonger par ici (pour te reposer)!
\trans \textit{\textcolor{brown}{\zh{过来这边躺一下!}}}
\end{exe}


\paragraph{\hspace{-0.5cm} {\Large \textcolor{darkblue}{\textbf{\ipa{pɤ˥\textsubscript{b}}}}}} \hypertarget{p7\string_Tb1}{}
\markboth{\textcolor{darkblue}{\textbf{\ipa{pɤ˥\textsubscript{b}}}}}{}
\textcolor{teal}{\mytextsc{classificateur}} \hspace{4pt} Ton~: H\textsubscript{b}.

Classificateur des images, peintures….
\textcolor{brown}{\zh{量词:雕像,如:佛像(一尊)。}}




\paragraph{\hspace{-0.5cm} {\Large \textcolor{darkblue}{\textbf{\ipa{pɤ˧\textsubscript{a}}}}}} \hypertarget{p7\string_Ma1}{}
\markboth{\textcolor{darkblue}{\textbf{\ipa{pɤ˧\textsubscript{a}}}}}{}
\textcolor{teal}{\mytextsc{verbe}} \hspace{4pt} Ton~: M\textsubscript{a}.

Porter sur son dos (le bois, …).
\textcolor{brown}{\zh{背(水、柴、孩子……)。}}


\begin{exe}
\sn \textcolor{darkblue}{\textbf{\ipa{pɤ˧\textasciitilde{}pɤ˧}}}
\trans \mytextsc{red}
\trans \textit{\textcolor{brown}{\zh{\mytextsc{重叠:背一背}}}}
\end{exe}
\begin{exe}
\sn \textcolor{darkblue}{\textbf{\ipa{tʰi˧-pɤ˥\textasciitilde{}pɤ˩}}}
\trans \mytextsc{dur} \mytextsc{red}
\trans \textit{\textcolor{brown}{\zh{背一背}}}
\end{exe}
\begin{exe}
\sn \textcolor{darkblue}{\textbf{\ipa{qʰæ˧ pɤ˧\textasciitilde{}pɤ˥}}}
\trans porter des engrais/ du fumier
\trans \textit{\textcolor{brown}{\zh{背肥料}}}
\end{exe}
\begin{exe}
\sn \textcolor{darkblue}{\textbf{\ipa{kʰɤ˧ pɤ˧\textasciitilde{}pɤ˥}}}
\trans porter un panier dorsal
\trans \textit{\textcolor{brown}{\zh{背背篓}}}
\end{exe}
\begin{exe}
\sn \textcolor{darkblue}{\textbf{\ipa{zɯ˧ pɤ˧\textasciitilde{}pɤ˥}}}
\trans porter de l'herbe
\trans \textit{\textcolor{brown}{\zh{背草}}}
\end{exe}
\begin{exe}
\sn \textcolor{darkblue}{\textbf{\ipa{tso˧\textasciitilde{}tso˧ pɤ˧\textasciitilde{}pɤ˥}}}
\trans porter des choses
\trans \textit{\textcolor{brown}{\zh{背东西}}}
\end{exe}
\begin{exe}
\sn \textcolor{darkblue}{\textbf{\ipa{*tso˧\textasciitilde{}tso˧ pɤ˩}}}
\trans porter des choses (l'expression est bien formée, mais pas usitée)
\trans \textit{\textcolor{brown}{\zh{背东西(语法上,这个短语没有问题,但发音合作人不那么说。)}}}
\end{exe}
\begin{exe}
\sn \textcolor{darkblue}{\textbf{\ipa{njɤ˧-ɳɯ˧ pɤ˧\textasciitilde{}pɤ˩ (+bi˩)!}}}
\trans c'est moi qui porte!
\trans \textit{\textcolor{brown}{\zh{我来背!}}}
\end{exe}
\begin{exe}
\sn \textcolor{darkblue}{\textbf{\ipa{dʑɯ˩ pɤ˩\textasciitilde{}pɤ˥}}}
\trans porter de l'eau
\trans \textit{\textcolor{brown}{\zh{背水}}}
\end{exe}
\begin{exe}
\sn \textcolor{darkblue}{\textbf{\ipa{zo˧mv̩˥ pɤ˩\textasciitilde{}pɤ˩}}}
\trans porter un enfant sur le dos
\trans \textit{\textcolor{brown}{\zh{背孩子}}}
\end{exe}


\paragraph{\hspace{-0.5cm} {\Large \textcolor{darkblue}{\textbf{\ipa{pɤ˧dʑɤ˩-di˩}}}}} \hypertarget{p7\string_Mdz£7\string_B-di\string_B1}{}
\markboth{\textcolor{darkblue}{\textbf{\ipa{pɤ˧dʑɤ˩-di˩}}}}{}
\textcolor{teal}{\mytextsc{nom}} \hspace{4pt} Ton~: L\#-.

Un village proche des Sources Chaudes.
\textcolor{brown}{\zh{温泉乡的一个村落。}}


\begin{exe}
\sn \textcolor{darkblue}{\textbf{\ipa{ə˧go˧-ʁwɤ˧, | ʁwɤ˧lɑ˩-bi˩, | bæ˧ʁwɤ˧, | tʰo˧tsʰe\#˥, | pi˧tsʰe˩-di˩, | pɤ˧dʑɤ˩-di˩, | ʁwɤ˧tv̩˧}}}
\trans Villages au sortir de la plaine de Yongning; les deux premiers comportent une population na; le troisième est un village na; les suivants sont essentiellement des villages pumi/prinmi.
\trans \textit{\textcolor{brown}{\zh{永宁背向泸沽湖方向经过的村落。前两个村落拥有相当大的摩梭人口比例,第三个村落是摩梭村,最后一个是普米村。}}}
\end{exe}


\paragraph{\hspace{-0.5cm} {\Large \textcolor{darkblue}{\textbf{\ipa{pɤ˧lɑ˩}}}}} \hypertarget{p7\string_MlA\string_B1}{}
\markboth{\textcolor{darkblue}{\textbf{\ipa{pɤ˧lɑ˩}}}}{}
\textcolor{teal}{\mytextsc{nom}} \hspace{4pt} Ton~: L\#.

Photo, photographie (néologisme).
\textcolor{brown}{\zh{相片,照片。}}


 \mytextsc{clf}~: \textcolor{darkblue}{\textbf{\ipa{pʰæ˧˥}}} 

\paragraph{\hspace{-0.5cm} {\Large \textcolor{darkblue}{\textbf{\ipa{pɤ˧ʁɑ˧}}}}} \hypertarget{p7\string_MRA\string_M1}{}
\markboth{\textcolor{darkblue}{\textbf{\ipa{pɤ˧ʁɑ˧}}}}{}
\textcolor{teal}{\mytextsc{classificateur}} \hspace{4pt} Ton~: M.

Un grand pas.
\textcolor{brown}{\zh{量词:一大步。}}


\begin{exe}
\sn \textcolor{darkblue}{\textbf{\ipa{ɖɯ˧-pɤ˧ʁɑ˧\textasciitilde{}ɖɯ˧-pɤ˧ʁɑ˧}}}
\trans à grands pas
\trans \textit{\textcolor{brown}{\zh{大步流星地}}}
\end{exe}
\begin{exe}
\sn \textcolor{darkblue}{\textbf{\ipa{ɲi˧-pɤ˧ʁɑ˧}}}
\trans deux grandes enjambées
\trans \textit{\textcolor{brown}{\zh{两大步}}}
\end{exe}


\paragraph{\hspace{-0.5cm} {\Large \textcolor{darkblue}{\textbf{\ipa{‑pɤ˧to˩}}}}} \hypertarget{‑p7\string_Mto\string_B1}{}
\markboth{\textcolor{darkblue}{\textbf{\ipa{‑pɤ˧to˩}}}}{}
\textcolor{teal}{\mytextsc{conjonction}} \hspace{4pt} Ton~: L\#.

Même.
\textcolor{brown}{\zh{连。}}


\begin{exe}
\sn \textcolor{darkblue}{\textbf{\ipa{ʈʂʰɯ˧ | li˩-pɤ˥to˩ | ʈʰɯ˩-ɲi˥!}}}
\trans Elle boit même du thé! (au sujet de l'alimentation d'un enfant d'un an)
\trans \textit{\textcolor{brown}{\zh{她连茶都喝!(关于一个一岁孩子的饮食习惯)}}}
\end{exe}
\begin{exe}
\sn \textcolor{darkblue}{\textbf{\ipa{ʈʂʰɯ˧ | pɤ˩jɤ˧-pɤ˥to˩ | dzɯ˩-ɲi˥!}}}
\trans Elle mange même du pain! (au sujet de l'alimentation d'un enfant d'un an)
\trans \textit{\textcolor{brown}{\zh{她连面包都吃!}}}
\end{exe}
\begin{exe}
\sn \textcolor{darkblue}{\textbf{\ipa{hæ˧, | kʰv̩˩mi˩-ʂe˩-pɤ˥to˩ dzɯ˩-kv̩˩!}}}
\trans les Chinois, ils mangent même du chien! (Note: l'un des interdits alimentaires na concerne la viande de chien, le chien étant un animal sacré dans la culture na.)
\trans \textit{\textcolor{brown}{\zh{汉族连狗肉都吃!(注:摩梭人不吃狗肉)}}}
\end{exe}
\begin{exe}
\sn \textcolor{darkblue}{\textbf{\ipa{hæ˧, | kʰv̩˩mi˩-ʂe˩˥ F dzɯ˩-kv̩˩!}}}
\trans même sens
\trans \textit{\textcolor{brown}{\zh{同上}}}
\end{exe}
\begin{exe}
\sn \textcolor{darkblue}{\textbf{\ipa{bo˩-pɤ˥to˩; lɑ˧-pɤ˧to˩; mv̩˩-pɤ˥to˩; ʐwæ˧-pɤ˧to˩; ɬi˧mi˧-pɤ˧to˩; ɲi˧mi˧-pɤ˧to˩; hwɤ˧li˧-pɤ˥-to˩; hwɤ˧mi˧-pɤ˥to˩; kʰv̩˩mi˩-pɤ˥-to˩; ʁo˧dzi˩-pɤ˩to˩; ʝi˩ʈʂæ˧-pɤ˥to˩; nɑ˩hĩ˧-pɤ˧to˩; bo˩mi˧-pɤ˧to˩; bo˩ɬɑ˧-pɤ˩to˩; ʁæ˧ʈv̩˥-pɤ˩to˩}}}
\trans en association avec des noms des diverses catégories tonales
\trans \textit{\textcolor{brown}{\zh{与不同声调类的名词结合}}}
\end{exe}


\paragraph{\hspace{-0.5cm} {\Large \textcolor{darkblue}{\textbf{\ipa{pɤ˧tv̩˥}}}}} \hypertarget{p7\string_Mtv\string_=\string_T1}{}
\markboth{\textcolor{darkblue}{\textbf{\ipa{pɤ˧tv̩˥}}}}{}
\textcolor{teal}{\mytextsc{nom}} \hspace{4pt} Ton~: H\#.

Panier de vannerie.
\textcolor{brown}{\zh{篮子、竹篮。}}


\begin{exe}
\sn \textcolor{darkblue}{\textbf{\ipa{ɖʐɯ˧ʂɯ˥-pɤ˩tv̩˩}}}
\trans panier (traditionnellement: en vannerie) dans lequel on range les baguettes
\trans \textit{\textcolor{brown}{\zh{筷子篮}}}
\end{exe}


\paragraph{\hspace{-0.5cm} {\Large \textcolor{darkblue}{\textbf{\ipa{pɤ˧tʰi˩}}}}} \hypertarget{p7\string_Mt\string_hi\string_B1}{}
\markboth{\textcolor{darkblue}{\textbf{\ipa{pɤ˧tʰi˩}}}}{}
\textcolor{teal}{\mytextsc{nom}} \hspace{4pt} Ton~: L\#.

Nom de clan/famille étendue. Deux familles portent ce nom à Yongning.
\textcolor{brown}{\zh{一个姓。这个姓,永宁有两家。}}


\begin{exe}
\sn \textcolor{darkblue}{\textbf{\ipa{pɤ˧tʰi˩=ɻ̍˩}}}
\trans le clan \textcolor{darkblue}{\textbf{\ipa{/pɤ˧tʰi˩/}}}, la famille \textcolor{darkblue}{\textbf{\ipa{/pɤ˧tʰi˩/}}}
\trans \textit{\textcolor{brown}{\zh{\textcolor{darkblue}{\textbf{\ipa{/pɤ˧tʰi˩/}}}家族}}}
\end{exe}


\paragraph{\hspace{-0.5cm} {\Large \textcolor{darkblue}{\textbf{\ipa{pɤ˧ʈʂʰwæ˩}}}}} \hypertarget{p7\string_Mt`s`\string_hw\{\string_B1}{}
\markboth{\textcolor{darkblue}{\textbf{\ipa{pɤ˧ʈʂʰwæ˩}}}}{}
\textcolor{teal}{\mytextsc{adjectif}} \hspace{4pt} Ton~: .

Écrasé, plat, aplati, raplapla, écrabouillé.
\textcolor{brown}{\zh{(鼻子)扁、被压扁、凹下去。}}


\begin{exe}
\sn \textcolor{darkblue}{\textbf{\ipa{ɲi˧gɤ˧ pɤ˧ʈʂʰwæ˥}}}
\trans nez aplati/camus; littéralement ``nez écrasé"
\trans \textit{\textcolor{brown}{\zh{扁鼻子}}}
\end{exe}
\begin{exe}
\sn \textcolor{darkblue}{\textbf{\ipa{le˧-pɤ˥ʈʂʰwæ˩-ze˩}}}
\trans \mytextsc{accomp} \string_ \mytextsc{pfv}
\trans \textit{\textcolor{brown}{\zh{被压扁}}}
\end{exe}
\begin{exe}
\sn \textcolor{darkblue}{\textbf{\ipa{ʈʂʰɯ˧-v̩˧ le˧-ʈʰi˩-zo˩, | dɤ˧-qo˧ | tʰi˧-pɤ˥ʈʂʰwæ˩-ze˩!}}}
\trans il s'est effondré là-bas, recru de fatigue!
\trans \textit{\textcolor{brown}{\zh{那个(人)筋疲力尽,躺倒在那边了!}}}
\end{exe}
\begin{exe}
\sn \textcolor{darkblue}{\textbf{\ipa{hæ˧mi˧-pɤ˧ʈʂʰwæ˧}}}
\trans Chinoise Han (au visage) raplapla (commentaire désobligeant à l'égard de dames au nez trop peu saillant selon les critères locaux)
\trans \textit{\textcolor{brown}{\zh{(脸、鼻子)扁的汉族女人(带偏见的称呼)}}}
\end{exe}


\paragraph{\hspace{-0.5cm} {\Large \textcolor{darkblue}{\textbf{\ipa{pɤ˩\textsubscript{a}}}}}} \hypertarget{p7\string_Ba1}{}
\markboth{\textcolor{darkblue}{\textbf{\ipa{pɤ˩\textsubscript{a}}}}}{}
\textcolor{teal}{\mytextsc{verbe}} \hspace{4pt} Ton~: L\textsubscript{a}.

Sortir, émerger, apparaître.
\textcolor{brown}{\zh{出现、出来、浮现。}}


\begin{exe}
\sn \textcolor{darkblue}{\textbf{\ipa{dʑɯ˩ pɤ˩˥}}}
\trans de l'eau sort
\trans \textit{\textcolor{brown}{\zh{涌出水来}}}
\end{exe}
\begin{exe}
\sn \textcolor{darkblue}{\textbf{\ipa{dʑɯ˧qʰv̩˧-qo˧ | dʑɯ˩ pɤ˩-ze˥}}}
\trans De l'eau apparaît à la source / de l'eau coule à la source
\trans \textit{\textcolor{brown}{\zh{水泉里面,涌出水来。}}}
\end{exe}
\begin{exe}
\sn \textcolor{darkblue}{\textbf{\ipa{tʰi˧-pɤ˩-dʑo˩}}}
\trans \mytextsc{dur} \string_ \mytextsc{prog}: ça sort, ça coule, ça émerge (ex.: de l'eau de source)
\trans \textit{\textcolor{brown}{\zh{正在涌出水来}}}
\end{exe}
\begin{exe}
\sn \textcolor{darkblue}{\textbf{\ipa{gɤ˩-pɤ˥}}}
\trans émerger, se lever: le soleil se lève
\trans \textit{\textcolor{brown}{\zh{出现、上来:太阳出来}}}
\end{exe}


\paragraph{\hspace{-0.5cm} {\Large \textcolor{darkblue}{\textbf{\ipa{pɤ˩\textsubscript{b}}}}}} \hypertarget{p7\string_Bb1}{}
\markboth{\textcolor{darkblue}{\textbf{\ipa{pɤ˩\textsubscript{b}}}}}{}
\textcolor{teal}{\mytextsc{classificateur}} \hspace{4pt} Ton~: L\textsubscript{b}.

Classificateur des éléments de menuiserie/charpente: échelles, portes….
\textcolor{brown}{\zh{量词:木工件,如梯子、门等等(一扇门,一把梯子)。}}




\paragraph{\hspace{-0.5cm} {\Large \textcolor{darkblue}{\textbf{\ipa{pɤ˩dʑɯ˩}}}}} \hypertarget{p7\string_Bdz£M\string_B1}{}
\markboth{\textcolor{darkblue}{\textbf{\ipa{pɤ˩dʑɯ˩}}}}{}
\textcolor{teal}{\mytextsc{nom}} \hspace{4pt} Ton~: L.

Eau de source.
\textcolor{brown}{\zh{泉水。}}




\paragraph{\hspace{-0.5cm} {\Large \textcolor{darkblue}{\textbf{\ipa{pɤ˩-ho˩\textasciitilde{}ho˥}}}}} \hypertarget{p7\string_B-ho\string_B~ho\string_T1}{}
\markboth{\textcolor{darkblue}{\textbf{\ipa{pɤ˩-ho˩\textasciitilde{}ho˥}}}}{}
\textcolor{teal}{\mytextsc{adjectif}} \hspace{4pt} Ton~: L+H\#.

Mou.
\textcolor{brown}{\zh{柔软。}}


\begin{exe}
\sn \textcolor{darkblue}{\textbf{\ipa{pɤ˩-ho˩\textasciitilde{}ho˥-gv̩˩}}}
\trans mou
\trans \textit{\textcolor{brown}{\zh{柔软}}}
\end{exe}
\begin{exe}
\sn \textcolor{darkblue}{\textbf{\ipa{ʁo˧qʰwɤ˩ | pɤ˩-ho˩\textasciitilde{}ho˥-gv̩˩-hĩ˩ | tʰv̩˧-kʰwɤ˥}}}
\trans l'endroit où la tête est toute molle =la fontanelle, chez les bébés
\trans \textit{\textcolor{brown}{\zh{头上软软的那块 =囟门}}}
\end{exe}


\paragraph{\hspace{-0.5cm} {\Large \textcolor{darkblue}{\textbf{\ipa{pɤ˩jɤ˧bv̩˥-di˩}}}}} \hypertarget{p7\string_Bj7\string_Mbv\string_=\string_T-di\string_B1}{}
\markboth{\textcolor{darkblue}{\textbf{\ipa{pɤ˩jɤ˧bv̩˥-di˩}}}}{}
\textcolor{teal}{\mytextsc{nom}} \hspace{4pt} Ton~: LM+\#H-.

Étuve pour cuire la pâte/le pain.
\textcolor{brown}{\zh{用来蒸面团(馒头等等)的蒸笼。}}


 \mytextsc{clf}~: \textcolor{darkblue}{\textbf{\ipa{ɭɯ˧}}} 

\paragraph{\hspace{-0.5cm} {\Large \textcolor{darkblue}{\textbf{\ipa{pɤ˩jɤ˧˥}}}}} \hypertarget{p7\string_Bj7\string_M\string_T1}{}
\markboth{\textcolor{darkblue}{\textbf{\ipa{pɤ˩jɤ˧˥}}}}{}
\textcolor{teal}{\mytextsc{nom}} \hspace{4pt} Ton~: LM+MH\#.
\ding{202} 
Pâte à pain (à cuire à la vapeur, pour obtenir des petits pains blancs).
\textcolor{brown}{\zh{做面包的面团(可以蒸成馒头)。}}


\ding{203} 
Galette.
\textcolor{brown}{\zh{饼。}}


\begin{exe}
\sn \textcolor{darkblue}{\textbf{\ipa{li˩-pɤ˥jɤ˩ | ɖɯ˧-ɭɯ˧}}}
\trans une galette de thé (feuilles de thé pressées en forme de galette)
\trans \textit{\textcolor{brown}{\zh{一块茶饼}}}
\end{exe}
\begin{exe}
\sn \textcolor{darkblue}{\textbf{\ipa{ɕi˧ʈʂʰwæ˧-pɤ˩jɤ˩}}}
\trans galette de riz
\trans \textit{\textcolor{brown}{\zh{米饼}}}
\end{exe}
\begin{exe}
\sn \textcolor{darkblue}{\textbf{\ipa{dze˧ɭɯ˧-pɤ˩jɤ˩}}}
\trans galette de froment, galette à la farine de blé
\trans \textit{\textcolor{brown}{\zh{小麦饼}}}
\end{exe}
\begin{exe}
\sn \textcolor{darkblue}{\textbf{\ipa{qʰɑ˧dze˧-pɤ˩jɤ˩}}}
\trans galette de maïs, galette à la farine de maïs
\trans \textit{\textcolor{brown}{\zh{玉米饼}}}
\end{exe}
\begin{exe}
\sn \textcolor{darkblue}{\textbf{\ipa{tsʰi˧zi˧-pɤ˥jɤ˩}}}
\trans galette à l'orge d'altitude
\trans \textit{\textcolor{brown}{\zh{青稞饼}}}
\end{exe}
\begin{exe}
\sn \textcolor{darkblue}{\textbf{\ipa{jɤ˧gɯ˩-pɤ˩jɤ˩}}}
\trans galette de sarrasin
\trans \textit{\textcolor{brown}{\zh{甜荞饼}}}
\end{exe}
\begin{exe}
\sn \textcolor{darkblue}{\textbf{\ipa{jɤ˧qʰɑ˧-pɤ˥jɤ˩}}}
\trans galette de sarrasin amer
\trans \textit{\textcolor{brown}{\zh{苦荞饼}}}
\end{exe}
 \mytextsc{clf}~: \textcolor{darkblue}{\textbf{\ipa{jɤ˧˥}}} \textcolor{darkblue}{\textbf{\ipa{jɤ˧˥}}} 

\paragraph{\hspace{-0.5cm} {\Large \textcolor{darkblue}{\textbf{\ipa{pɤ˩lv̩˩}}}}} \hypertarget{p7\string_Blv\string_=\string_B1}{}
\markboth{\textcolor{darkblue}{\textbf{\ipa{pɤ˩lv̩˩}}}}{}
\textcolor{teal}{\mytextsc{nom}} \hspace{4pt} Ton~: L.

Nuque.
\textcolor{brown}{\zh{项背 、项、脖颈儿。}}


 \mytextsc{clf}~: \textcolor{darkblue}{\textbf{\ipa{ɭɯ˧}}} 

\paragraph{\hspace{-0.5cm} {\Large \textcolor{darkblue}{\textbf{\ipa{pɤ˩lv̩˧}}}}} \hypertarget{p7\string_Blv\string_=\string_M1}{}
\markboth{\textcolor{darkblue}{\textbf{\ipa{pɤ˩lv̩˧}}}}{}
\textcolor{teal}{\mytextsc{nom}} \hspace{4pt} Ton~: LM.

Réserve, magasin: bâtiment à un seul étage, face au bâtiment principal (\textcolor{darkblue}{\textbf{\ipa{/ʑi˧mi˧/}}}), dans lequel on range les gros outils, tels que l'araire, et la viande séchée.
\textcolor{brown}{\zh{仓库:主屋对面的房子,只有一层。用来收藏大工具,例如犁,或者腊肉。}}




\paragraph{\hspace{-0.5cm} {\Large \textcolor{darkblue}{\textbf{\ipa{pɤ˩mi˩}}}}} \hypertarget{p7\string_Bmi\string_B1}{}
\markboth{\textcolor{darkblue}{\textbf{\ipa{pɤ˩mi˩}}}}{}
\textcolor{teal}{\mytextsc{nom}} \hspace{4pt} Ton~: L.

Grenouille.
\textcolor{brown}{\zh{青蛙。}}


\begin{exe}
\sn \textcolor{darkblue}{\textbf{\ipa{pɤ˩mi˩-pɤ˥pʰv̩˩}}}
\trans grenouille femelle et grenouille mâle
\trans \textit{\textcolor{brown}{\zh{母青蛙与公青蛙}}}
\end{exe}
\begin{exe}
\sn \textcolor{darkblue}{\textbf{\ipa{pɤ˩mi˩-ʝi˥pʰv̩˩}}}
\trans grosse grenouille (ou crapaud); animal très courant dans la plaine. C'est l'une des trois sortes de grenouilles que connaît la locutrice. Cet animal n'est pas consommé par les Naxi (ni par les Na, qui ne mangent aucune grenouille). La locutrice emploie ce terme pour \textit{Kaloula verrucosa} et \textit{Rana chaochiaoensis}.
\trans \textit{\textcolor{brown}{\zh{一种大青蛙,在永宁坝子很常见。这是发音合作人认识的三种蛙之一。纳西族人不吃这种动物(摩梭人不吃任何蛙类动物)。}}}
\end{exe}
\begin{exe}
\sn \textcolor{darkblue}{\textbf{\ipa{pɤ˩mi˩-ʝi˥pʰv̩˩-mi˩}}}
\trans même sens
\trans \textit{\textcolor{brown}{\zh{同上}}}
\end{exe}
\begin{exe}
\sn \textcolor{darkblue}{\textbf{\ipa{hæ̃˧ʂɯ˩-pɤ˩mi˩}}}
\trans Belle grenouille, de longue taille. Elle ne s'observe qu'en forêt, dans la montagne. C'est la deuxième des trois sortes de grenouilles que connaît la locutrice.
\trans \textit{\textcolor{brown}{\zh{一种很美的青蛙,身体很长。只出现在山上森林里。这是发音合作人认识的第二种青蛙。}}}
\end{exe}
\begin{exe}
\sn \textcolor{darkblue}{\textbf{\ipa{dʑɯ˩-pɤ˩mi˩˥}}}
\trans Grenouille ayant une petite tête et de grands yeux, qui passerait le plus clair de son temps dans l'eau. C'est la troisième des trois sortes de grenouilles que connaît la locutrice. Les Naxi la chassent, la dénichant sous les cailloux des ruisseaux, surtout au cinquième mois.
\trans \textit{\textcolor{brown}{\zh{一种青蛙,头小、眼睛大。这是发音合作人认识的第三种青蛙。纳西族吃这种青蛙。}}}
\end{exe}
\begin{exe}
\sn \textcolor{darkblue}{\textbf{\ipa{nɑ˩hĩ˥ | pɤ˧-ʂe˧ dzɯ˧; | pɤ˧-ɣɯ˧ | ɬɑ˧tɑ˥ mv̩˩! | pɤ˧-mæ˧, | bæ˧ʈʂo˥ ʝi˩!}}}
\trans ``Les Naxi mangent de la viande de grenouille; ils se vêtent de gilets en peau de grenouille; ils se font des balais avec la queue des grenouilles!"
\trans \textit{\textcolor{brown}{\zh{谚语:“纳西人吃青蛙,披青蛙皮衣,蛙尾巴当扫帚!”}}}
\end{exe}
 \mytextsc{clf}~: \textcolor{darkblue}{\textbf{\ipa{mi˩}}} 

\paragraph{\hspace{-0.5cm} {\Large \textcolor{darkblue}{\textbf{\ipa{pɤ˩pʰv̩˩}}}}} \hypertarget{p7\string_Bp\string_hv\string_=\string_B1}{}
\markboth{\textcolor{darkblue}{\textbf{\ipa{pɤ˩pʰv̩˩}}}}{}
\textcolor{teal}{\mytextsc{nom}} \hspace{4pt} Ton~: L.

Grenouille mâle.
\textcolor{brown}{\zh{公青蛙。}}


 \mytextsc{clf}~: \textcolor{darkblue}{\textbf{\ipa{mi˩}}} 

\paragraph{\hspace{-0.5cm} {\Large \textcolor{darkblue}{\textbf{\ipa{pɤ˩ti\#˥}}}}} \hypertarget{p7\string_Bti\#\string_T1}{}
\markboth{\textcolor{darkblue}{\textbf{\ipa{pɤ˩ti\#˥}}}}{}
\textcolor{teal}{\mytextsc{nom}} \hspace{4pt} Ton~: LM+\#H.

Tabouret, petit banc.
\textcolor{brown}{\zh{凳子。}}


 \mytextsc{clf}~: \textcolor{darkblue}{\textbf{\ipa{ɭɯ˧}}} 

\paragraph{\hspace{-0.5cm} {\Large \textcolor{darkblue}{\textbf{\ipa{pɤ˩tɕɯ˧-pɤ˥mi˩}}}}} \hypertarget{p7\string_Bts£M\string_M-p7\string_Tmi\string_B1}{}
\markboth{\textcolor{darkblue}{\textbf{\ipa{pɤ˩tɕɯ˧-pɤ˥mi˩}}}}{}
\textcolor{teal}{\mytextsc{nom}} \hspace{4pt} Ton~: LM+\#H-.

Têtard.
\textcolor{brown}{\zh{蝌蚪。}}


 \mytextsc{clf}~: \textcolor{darkblue}{\textbf{\ipa{mi˩}}} \textit{Voir~:} \textcolor{darkblue}{\textbf{\ipa{pɤ˩tɕɯ˧˥, pɤ˩tɕɯ˧-ʁo˧ɖɯ˧˥}}} 

\paragraph{\hspace{-0.5cm} {\Large \textcolor{darkblue}{\textbf{\ipa{pɤ˩tɕɯ˧-ʁo˧ɖɯ˧˥}}}}} \hypertarget{p7\string_Bts£M\string_M-Ro\string_Md`M\string_M\string_T1}{}
\markboth{\textcolor{darkblue}{\textbf{\ipa{pɤ˩tɕɯ˧-ʁo˧ɖɯ˧˥}}}}{}
\textcolor{teal}{\mytextsc{nom}} \hspace{4pt} Ton~: LM+MH\#.

Têtard.
\textcolor{brown}{\zh{蝌蚪。}}


 \mytextsc{clf}~: \textcolor{darkblue}{\textbf{\ipa{mi˩}}} \textit{Voir~:} \textcolor{darkblue}{\textbf{\ipa{pɤ˩tɕɯ˧˥, pɤ˩tɕɯ˧-pɤ˥mi˩}}} 

\paragraph{\hspace{-0.5cm} {\Large \textcolor{darkblue}{\textbf{\ipa{pɤ˩tɕɯ˧˥}}}}} \hypertarget{p7\string_Bts£M\string_M\string_T1}{}
\markboth{\textcolor{darkblue}{\textbf{\ipa{pɤ˩tɕɯ˧˥}}}}{}
\textcolor{teal}{\mytextsc{nom}} \hspace{4pt} Ton~: LM+MH\#.

Têtard.
\textcolor{brown}{\zh{蝌蚪。}}


 \mytextsc{clf}~: \textcolor{darkblue}{\textbf{\ipa{mi˩}}} \textit{Voir~:} \textcolor{darkblue}{\textbf{\ipa{pɤ˩tɕɯ˧-ʁo˧ɖɯ˧˥, pɤ˩tɕɯ˧-pɤ˥mi˩}}} 

\paragraph{\hspace{-0.5cm} {\Large \textcolor{darkblue}{\textbf{\ipa{pɤ˧˥}}}}} \hypertarget{p7\string_M\string_T1}{}
\markboth{\textcolor{darkblue}{\textbf{\ipa{pɤ˧˥}}}}{}
\textcolor{teal}{\mytextsc{verbe}} \hspace{4pt} Ton~: MH.

Passer la herse, aplanir (à l'aide d'une herse/instrument permettant de lisser le champ après labourage, afin qu'il soit prêt pour qu'on y repique le riz).
\textcolor{brown}{\zh{耙地。}}


\begin{exe}
\sn \textcolor{darkblue}{\textbf{\ipa{ʝi˧ pɤ˥}}}
\trans passer la herse
\trans \textit{\textcolor{brown}{\zh{耙地}}}
\end{exe}
\begin{exe}
\sn \textcolor{darkblue}{\textbf{\ipa{ɕi˧ tv̩˧-dʑo˧, | ʝi˧ le˧-pɤ˩!}}}
\trans Quand on plante du riz (=avant de planter le riz), il faut passer la herse!
\trans \textit{\textcolor{brown}{\zh{种稻谷,要(先)耙地!}}}
\end{exe}


\paragraph{\hspace{-0.5cm} {\Large \textcolor{darkblue}{\textbf{\ipa{pɤ˩˧ʐv̩˩}}}}} \hypertarget{p7\string_B\string_Mz`v\string_=\string_B1}{}
\markboth{\textcolor{darkblue}{\textbf{\ipa{pɤ˩˧ʐv̩˩}}}}{}
\textcolor{teal}{\mytextsc{nom}} \hspace{4pt} Ton~: LM-L.

Matelas.
\textcolor{brown}{\zh{褥子(汉语借词:被褥)。}}Dialecte chinois local~: \zh{被褥。}

 Emprunt~: chinois \textcolor{brown}{\zh{被褥}}

 \mytextsc{clf}~: \textcolor{darkblue}{\textbf{\ipa{tsʰi˥}}} 

\paragraph{\hspace{-0.5cm} {\Large \textcolor{darkblue}{\textbf{\ipa{pi˥}}}}} \hypertarget{pi\string_T1}{}
\markboth{\textcolor{darkblue}{\textbf{\ipa{pi˥}}}}{}
\textcolor{teal}{\mytextsc{verbe}} \hspace{4pt} Ton~: H.

Dire.
\textcolor{brown}{\zh{说。}}


\begin{exe}
\sn \textcolor{darkblue}{\textbf{\ipa{tʰɑ˧-pi˥!}}}
\trans Il ne faut pas (le) dire!
\trans \textit{\textcolor{brown}{\zh{别说!}}}
\end{exe}
\begin{exe}
\sn \textcolor{darkblue}{\textbf{\ipa{ə˧tso˧ pi˧?}}}
\trans Que dis-tu? (employé pour demander à quelqu'un de répéter)
\trans \textit{\textcolor{brown}{\zh{(你刚才)说什么?(请人家重新说一遍)}}}
\end{exe}
\begin{exe}
\sn \textcolor{darkblue}{\textbf{\ipa{ə˧tso˧ pi˧-ɲi˥?}}}
\trans Que dis-tu? (employé pour demander à quelqu'un de répéter)
\trans \textit{\textcolor{brown}{\zh{(你刚才)说什么?(请人家重新说一遍)}}}
\end{exe}


\paragraph{\hspace{-0.5cm} {\Large \textcolor{darkblue}{\textbf{\ipa{pi˧lv̩\#˥}}}}} \hypertarget{pi\string_Mlv\string_=\#\string_T1}{}
\markboth{\textcolor{darkblue}{\textbf{\ipa{pi˧lv̩\#˥}}}}{}
\textcolor{teal}{\mytextsc{nom}} \hspace{4pt} Ton~: \#H.

Déchet de la distillation: ce qui reste après la production de l'alcool; grain qu'on donne aux animaux.
\textcolor{brown}{\zh{酒糟:煮酒剩下的渣滓(一般给猪吃)。}}Dialecte chinois local~: \zh{酒糟。}


\begin{exe}
\sn \textcolor{darkblue}{\textbf{\ipa{pi˧lv̩˧, | hĩ˧ | dzɯ˧-mɤ˧-kv̩˩!}}}
\trans Les grains après distillation, ça ne se mange pas! / ce n'est pas propre à la consommation humaine!
\trans \textit{\textcolor{brown}{\zh{酒糟,人不能吃!}}}
\end{exe}


\paragraph{\hspace{-0.5cm} {\Large \textcolor{darkblue}{\textbf{\ipa{pi˧mɑ˧}}}}} \hypertarget{pi\string_MmA\string_M1}{}
\markboth{\textcolor{darkblue}{\textbf{\ipa{pi˧mɑ˧}}}}{}
\textcolor{teal}{\mytextsc{nom}} \hspace{4pt} Ton~: M.

Prénom unisexe: prénom utilisé pour les deux sexes.
\textcolor{brown}{\zh{男女通用名。}}




\paragraph{\hspace{-0.5cm} {\Large \textcolor{darkblue}{\textbf{\ipa{pi˧mɑ˧-ɬɑ˩mv̩˩}}}}} \hypertarget{pi\string_MmA\string_M-KA\string_Bmv\string_=\string_B1}{}
\markboth{\textcolor{darkblue}{\textbf{\ipa{pi˧mɑ˧-ɬɑ˩mv̩˩}}}}{}
\textcolor{teal}{\mytextsc{nom}} \hspace{4pt} Ton~: \mytextsc{L}.

Prénom féminin.
\textcolor{brown}{\zh{女性名字。}}




\paragraph{\hspace{-0.5cm} {\Large \textcolor{darkblue}{\textbf{\ipa{pi˧mɑ˧-ɬɑ˩tsʰo˩}}}}} \hypertarget{pi\string_MmA\string_M-KA\string_Bts\string_ho\string_B1}{}
\markboth{\textcolor{darkblue}{\textbf{\ipa{pi˧mɑ˧-ɬɑ˩tsʰo˩}}}}{}
\textcolor{teal}{\mytextsc{nom}} \hspace{4pt} Ton~: \mytextsc{L}.

Prénom féminin.
\textcolor{brown}{\zh{女性名字。}}




\paragraph{\hspace{-0.5cm} {\Large \textcolor{darkblue}{\textbf{\ipa{pi˧mv̩˥\$}}}}} \hypertarget{pi\string_Mmv\string_=\string_T\$1}{}
\markboth{\textcolor{darkblue}{\textbf{\ipa{pi˧mv̩˥\$}}}}{}
\textcolor{teal}{\mytextsc{nom}} \hspace{4pt} Ton~: H\$.

Dicton, parole du temps jadis, adage.
\textcolor{brown}{\zh{成语、俗语。}}




\paragraph{\hspace{-0.5cm} {\Large \textcolor{darkblue}{\textbf{\ipa{pi˧tsʰe˩-di˩}}}}} \hypertarget{pi\string_Mts\string_he\string_B-di\string_B1}{}
\markboth{\textcolor{darkblue}{\textbf{\ipa{pi˧tsʰe˩-di˩}}}}{}
\textcolor{teal}{\mytextsc{nom}} \hspace{4pt} Ton~: L\#-.

Un village proche des Sources Chaudes.
\textcolor{brown}{\zh{温泉乡的一个村落。}}


\begin{exe}
\sn \textcolor{darkblue}{\textbf{\ipa{ə˧go˧-ʁwɤ˧, | ʁwɤ˧lɑ˩-bi˩, | bæ˧ʁwɤ˧, | tʰo˧tsʰe\#˥, | pi˧tsʰe˩-di˩, | pɤ˧dʑɤ˩-di˩, | ʁwɤ˧tv̩˧}}}
\trans Villages au sortir de la plaine de Yongning; les deux premiers comportent une population na; le troisième est un village na; les suivants sont essentiellement des villages pumi/prinmi.
\trans \textit{\textcolor{brown}{\zh{永宁背向泸沽湖方向经过的村落。前两个村落拥有相当大的摩梭人口比例,第三个村落是摩梭村,最后一个是普米村。}}}
\end{exe}
\begin{exe}
\sn \textcolor{darkblue}{\textbf{\ipa{pi˧tsʰe˩: bɤ˩! |}}}
\trans \textcolor{darkblue}{\textbf{\ipa{/pi˧tsʰe˩/}}}, c'est un village pumi!
\trans \textit{\textcolor{brown}{\zh{fv:/pi˧tsʰe˩/是一个普米族村落!}}}
\end{exe}


\paragraph{\hspace{-0.5cm} {\Large \textcolor{darkblue}{\textbf{\ipa{pi˩ɻ̍˥}}}}} \hypertarget{pi\string_Br£`̍\string_T1}{}
\markboth{\textcolor{darkblue}{\textbf{\ipa{pi˩ɻ̍˥}}}}{}
\textcolor{teal}{\mytextsc{nom}} \hspace{4pt} Ton~: LH.

Double menton, bourrelet de chair sous le menton.
\textcolor{brown}{\zh{双下巴。}}


 \mytextsc{clf}~: \textcolor{darkblue}{\textbf{\ipa{ɭɯ˧}}} 

\paragraph{\hspace{-0.5cm} {\Large \textcolor{darkblue}{\textbf{\ipa{pi˩ti\#˥}}}}} \hypertarget{pi\string_Bti\#\string_T1}{}
\markboth{\textcolor{darkblue}{\textbf{\ipa{pi˩ti\#˥}}}}{}
\textcolor{teal}{\mytextsc{nom}} \hspace{4pt} Ton~: LM+\#H.

Pépite d'argent.
\textcolor{brown}{\zh{银块。}}


 \mytextsc{clf}~: \textcolor{darkblue}{\textbf{\ipa{ɭɯ˧}}} 

\paragraph{\hspace{-0.5cm} {\Large \textcolor{darkblue}{\textbf{\ipa{pi˧˥\textsubscript{a}}}}}} \hypertarget{pi\string_M\string_Ta1}{}
\markboth{\textcolor{darkblue}{\textbf{\ipa{pi˧˥\textsubscript{a}}}}}{}
\textcolor{teal}{\mytextsc{classificateur}} \hspace{4pt} Ton~: MH\textsubscript{a}.

Peu (indénombrable), un peu; souvent employé comme hypocoristique.
\textcolor{brown}{\zh{量词:少。}}


\begin{exe}
\sn \textcolor{darkblue}{\textbf{\ipa{ɖɯ˧-pi˧˥}}}
\trans un peu
\trans \textit{\textcolor{brown}{\zh{一点}}}
\end{exe}
\begin{exe}
\sn \textcolor{darkblue}{\textbf{\ipa{qʰæ˧-pi˩}}}
\trans un peu de crottin (ramassé comme engrais)
\trans \textit{\textcolor{brown}{\zh{一点粪肥}}}
\end{exe}
\begin{exe}
\sn \textcolor{darkblue}{\textbf{\ipa{ŋv̩˧-pi˧}}}
\trans un peu d’argent
\trans \textit{\textcolor{brown}{\zh{一点钱}}}
\end{exe}
\begin{exe}
\sn \textcolor{darkblue}{\textbf{\ipa{ŋv̩˧ | ɖɯ˧-pi˧˥}}}
\trans un peu d’argent
\trans \textit{\textcolor{brown}{\zh{一点钱}}}
\end{exe}
\begin{exe}
\sn \textcolor{darkblue}{\textbf{\ipa{lwɤ˧˥ | ɖɯ˧ pi˧˥}}}
\trans un peu de cendre; on ne peut dire: *lwɤ˧-pi˥, non plus que: *tsʰe˧-pi˩ (pour ``un peu de sel")
\trans \textit{\textcolor{brown}{\zh{一点灰}}}
\end{exe}
\begin{exe}
\sn \textcolor{darkblue}{\textbf{\ipa{ʈʂʰɯ˧ | ɖʐe˧ ɖɯ˧-pi˧ dʑo˧!}}}
\trans il a un peu d'argent!
\trans \textit{\textcolor{brown}{\zh{他有一些钱!}}}
\end{exe}


\paragraph{\hspace{-0.5cm} {\Large \textcolor{darkblue}{\textbf{\ipa{pi˩˥}}}}} \hypertarget{pi\string_B\string_T1}{}
\markboth{\textcolor{darkblue}{\textbf{\ipa{pi˩˥}}}}{}
\textcolor{teal}{\mytextsc{nom}} \hspace{4pt} Ton~: LH.

Pinceau (emprunt ancien).
\textcolor{brown}{\zh{笔。}}

 Emprunt~: chinois \textcolor{brown}{\zh{笔}}

\begin{exe}
\sn \textcolor{darkblue}{\textbf{\ipa{tʰæ˧ɻæ˩ tɕɯ˩-di˩, | pi˩˥!}}}
\trans Le truc pour écrire, ça s'appelle ``pinceau"!
\trans \textit{\textcolor{brown}{\zh{用来写字的那个东西,(叫做)“笔”!}}}
\end{exe}


\paragraph{\hspace{-0.5cm} {\Large \textcolor{darkblue}{\textbf{\ipa{pjɤ˥}}}}} \hypertarget{pj7\string_T1}{}
\markboth{\textcolor{darkblue}{\textbf{\ipa{pjɤ˥}}}}{}
\textcolor{teal}{\mytextsc{adjectif}} \hspace{4pt} Ton~: H.

Carré/anguleux (visage, pilier…).
\textcolor{brown}{\zh{方形的。}}


\begin{exe}
\sn \textcolor{darkblue}{\textbf{\ipa{tɑ˧-pjɤ˧\textasciitilde{}pjɤ˥ (-zo˩)}}}
\trans un peu anguleux/carré (terme péjoratif: objet / physique trop peu lisse pour être plaisant au regard ou au toucher)
\trans \textit{\textcolor{brown}{\zh{(脸、物品)太方,不圆滑}}}
\end{exe}


\paragraph{\hspace{-0.5cm} {\Large \textcolor{darkblue}{\textbf{\ipa{po˥}}}}} \hypertarget{po\string_T1}{}
\markboth{\textcolor{darkblue}{\textbf{\ipa{po˥}}}}{}
\textcolor{teal}{\mytextsc{verbe}} \hspace{4pt} Ton~: .

Emballer.
\textcolor{brown}{\zh{包(量词)(汉语借词)。}}

 Emprunt~: chinois \textcolor{brown}{\zh{包}}

\begin{exe}
\sn \textcolor{darkblue}{\textbf{\ipa{le˧-po˥}}}
\trans \mytextsc{accomp}
\trans \textit{\textcolor{brown}{\zh{\mytextsc{accomp}}}}
\end{exe}


\paragraph{\hspace{-0.5cm} {\Large \textcolor{darkblue}{\textbf{\ipa{po˧\textsubscript{a}}}}}} \hypertarget{po\string_Ma1}{}
\markboth{\textcolor{darkblue}{\textbf{\ipa{po˧\textsubscript{a}}}}}{}
\textcolor{teal}{\mytextsc{classificateur}} \hspace{4pt} Ton~: M\textsubscript{a}.

Classificateur des plantes à tiges (fleurs, poireaux…); aussi utilisé pour compter les types/catégories de vêtements.
\textcolor{brown}{\zh{量词:有根的植物,衣服(一棵,一件)。}}




\paragraph{\hspace{-0.5cm} {\Large \textcolor{darkblue}{\textbf{\ipa{po˧ɖʐɯ\#˥}}}}} \hypertarget{po\string_Md`z`M\#\string_T1}{}
\markboth{\textcolor{darkblue}{\textbf{\ipa{po˧ɖʐɯ\#˥}}}}{}
\textcolor{teal}{\mytextsc{nom}} \hspace{4pt} Ton~: \#H.

Artisan.
\textcolor{brown}{\zh{工匠。}}


\begin{exe}
\sn \textcolor{darkblue}{\textbf{\ipa{po˧ɖʐɯ˧ ʝi˧-hĩ˧-hĩ˧}}}
\trans personne qui travaille comme artisan
\trans \textit{\textcolor{brown}{\zh{当工匠的人}}}
\end{exe}
 \mytextsc{clf}~: \textcolor{darkblue}{\textbf{\ipa{v̩˧}}} 

\paragraph{\hspace{-0.5cm} {\Large \textcolor{darkblue}{\textbf{\ipa{po˧lo˧}}}}} \hypertarget{po\string_Mlo\string_M1}{}
\markboth{\textcolor{darkblue}{\textbf{\ipa{po˧lo˧}}}}{}
\textcolor{teal}{\mytextsc{nom}} \hspace{4pt} Ton~: M.

Bélier; bouc.
\textcolor{brown}{\zh{公绵羊。}}


\begin{exe}
\sn \textcolor{darkblue}{\textbf{\ipa{po˧lo˧ lɑ˧˥}}}
\trans frapper un bélier
\trans \textit{\textcolor{brown}{\zh{打公绵羊}}}
\end{exe}
 \mytextsc{clf}~: \textcolor{darkblue}{\textbf{\ipa{pʰo˧˥}}} 

\paragraph{\hspace{-0.5cm} {\Large \textcolor{darkblue}{\textbf{\ipa{po˧po˧}}}}} \hypertarget{po\string_Mpo\string_M1}{}
\markboth{\textcolor{darkblue}{\textbf{\ipa{po˧po˧}}}}{}
\textcolor{teal}{\mytextsc{nom}} \hspace{4pt} Ton~: M.

Ballon.
\textcolor{brown}{\zh{球。}}


\begin{exe}
\sn \textcolor{darkblue}{\textbf{\ipa{[F5] po˧po˧ lɑ˧˥}}}
\trans jouer au ballon
\trans \textit{\textcolor{brown}{\zh{打球}}}
\end{exe}
 \mytextsc{clf}~: \textcolor{darkblue}{\textbf{\ipa{ɭɯ˧}}} 

\paragraph{\hspace{-0.5cm} {\Large \textcolor{darkblue}{\textbf{\ipa{po˧po˧tsʰɤ˧˥}}}}} \hypertarget{po\string_Mpo\string_Mts\string_h7\string_M\string_T1}{}
\markboth{\textcolor{darkblue}{\textbf{\ipa{po˧po˧tsʰɤ˧˥}}}}{}
\textcolor{teal}{\mytextsc{nom}} \hspace{4pt} Ton~: MH\#.

Chou.
\textcolor{brown}{\zh{圆白菜。}}Dialecte chinois local~: \zh{包包菜。}

 Emprunt~: chinois \textcolor{brown}{\zh{包包菜}}

 \mytextsc{clf}~: \textcolor{darkblue}{\textbf{\ipa{ɭɯ˧}}} 

\paragraph{\hspace{-0.5cm} {\Large \textcolor{darkblue}{\textbf{\ipa{po˩\textsubscript{b}}}}}} \hypertarget{po\string_Bb1}{}
\markboth{\textcolor{darkblue}{\textbf{\ipa{po˩\textsubscript{b}}}}}{}
\textcolor{teal}{\mytextsc{classificateur}} \hspace{4pt} Ton~: L\textsubscript{b}.

Classificateur des paquets.
\textcolor{brown}{\zh{量词:包(汉语借词)。}}

 Emprunt~: chinois \textcolor{brown}{\zh{包}}



\paragraph{\hspace{-0.5cm} {\Large \textcolor{darkblue}{\textbf{\ipa{po˧˥}}}}} \hypertarget{po\string_M\string_T1}{}
\markboth{\textcolor{darkblue}{\textbf{\ipa{po˧˥}}}}{}
\textcolor{teal}{\mytextsc{verbe}} \hspace{4pt} Ton~: MH.
\ding{202} 
Amener, apporter; ramener, rapporter; faire cadeau de; envoyer (un message), transmettre; utiliser.
\textcolor{brown}{\zh{寄信、服送、带过来、拿、送。}}


\begin{exe}
\sn \textcolor{darkblue}{\textbf{\ipa{qʰwæ˧ po˧˥}}}
\trans amener une lettre/un message
\trans \textit{\textcolor{brown}{\zh{带来一封信/一个消息}}}
\end{exe}
\begin{exe}
\sn \textcolor{darkblue}{\textbf{\ipa{tso˧\textasciitilde{}tso˧ tʰi˧-po˧˥}}}
\trans amener quelque chose
\trans \textit{\textcolor{brown}{\zh{带来一个东西}}}
\end{exe}
\begin{exe}
\sn \textcolor{darkblue}{\textbf{\ipa{hĩ˧ ɖɯ˧-v̩˧ | tso˧\textasciitilde{}tso˧ ɖɯ˧-kʰwɤ˥ | tʰi˧-po˧˥}}}
\trans quelqu'un prend/amène quelque chose
\trans \textit{\textcolor{brown}{\zh{有人带东西过来}}}
\end{exe}
\begin{exe}
\sn \textcolor{darkblue}{\textbf{\ipa{ʈʂʰwæ˧˥ | po˧-jo˥!}}}
\trans amène vite!
\trans \textit{\textcolor{brown}{\zh{快拿过来吧!/ 快带过来吧!}}}
\end{exe}
\ding{203} 
Porter un enfant, c'est-à-dire être enceinte.
\textcolor{brown}{\zh{怀孕。}}


\begin{exe}
\sn \textcolor{darkblue}{\textbf{\ipa{ʈʂʰɯ˧ | zo˧mv̩˥ po˩.}}}
\trans Elle est enceinte.
\trans \textit{\textcolor{brown}{\zh{她怀孕了。}}}
\end{exe}
\begin{exe}
\sn \textcolor{darkblue}{\textbf{\ipa{zo˧ po˩ (+ze˩)}}}
\trans porter un enfant, être enceinte.
\trans \textit{\textcolor{brown}{\zh{怀孕}}}
\end{exe}


\paragraph{\hspace{-0.5cm} {\Large \textcolor{darkblue}{\textbf{\ipa{pv̩˩}}}}} \hypertarget{pv\string_=\string_B1}{}
\markboth{\textcolor{darkblue}{\textbf{\ipa{pv̩˩}}}}{}
\textcolor{teal}{\mytextsc{verbe}} \hspace{4pt} Ton~: L.

Passer, s'écouler: le temps passe, les jours passent.
\textcolor{brown}{\zh{过、过去(时间过去、日子过去)。}}


\begin{exe}
\sn \textcolor{darkblue}{\textbf{\ipa{ɲi˧mi˧ pv̩˩}}}
\trans le temps passe, la journée passe
\trans \textit{\textcolor{brown}{\zh{时间过去。直译:(一)天(慢慢)过(去)}}}
\end{exe}
\begin{exe}
\sn \textcolor{darkblue}{\textbf{\ipa{ɲi˧mi˧ | le˧-pv̩˩-ze˩}}}
\trans le temps a passé, la journée a passé
\trans \textit{\textcolor{brown}{\zh{时间过去了,(一)天过去了}}}
\end{exe}
\begin{exe}
\sn \textcolor{darkblue}{\textbf{\ipa{dʑɤ˩-dzɯ˧ qʰwɤ˧-dzɯ˥, | bi˧mi˧ ʂv̩˧-qʰwɤ˧-ɻ̍˥; | dʑɤ˩-ʐwɤ˥ qʰwɤ˩-ʐwɤ˩, | ɲi˧mi˧ ʂæ˧ pv̩˩-di˩!}}}
\trans Qu'on mange bien ou mal, on arrive à se remplir le ventre / Que la nourriture soit bonne ou mauvaise, peu importe au fond, tant qu'on a le ventre plein; qu'on dise des bonnes choses (=des éloges d'autrui) ou des mauvaises (=des ragots), on arrive à passer la journée / le jour se passe agréablement! (Proverbe qui fait l'éloge des vertus du bavardage et du commérage.)
\trans \textit{\textcolor{brown}{\zh{“吃好吃坏,(都)能填满肚子/(都)能吃饱!说好说坏,(都)能让一天(轻松)过去!”(这个谚语,说闲聊的好。)}}}
\end{exe}
\begin{exe}
\sn \textcolor{darkblue}{\textbf{\ipa{dʑɤ˩-dzɯ˧ qʰwɤ˧-dzɯ˥, | bi˧mi˧ ʂv̩˧˥; | dʑɤ˩-ʐwɤ˥ qʰwɤ˩-ʐwɤ˩, | ɲi˧mi˧ ʂæ˧-pv̩˩-di˩!}}}
\trans Variante du proverbe ci-dessus.
\trans \textit{\textcolor{brown}{\zh{上述谚语的变体}}}
\end{exe}


\paragraph{\hspace{-0.5cm} {\Large \textcolor{darkblue}{\textbf{\ipa{pv̩˧}}} \textsubscript{1}}} \hypertarget{pv\string_=\string_M1}{}
\markboth{\textcolor{darkblue}{\textbf{\ipa{pv̩˧}}} \textsubscript{1}}{}
\textcolor{teal}{\mytextsc{verbe}} \hspace{4pt} Ton~: M\textsubscript{c}.

Faire un sacrifice, faire un rituel, psalmodier.
\textcolor{brown}{\zh{祭。}}


\begin{exe}
\sn \textcolor{darkblue}{\textbf{\ipa{kʰv̩˧ pv̩˥}}}
\trans faire la veillée du Nouvel An (la veille de la nouvelle année), célébrer la veillée du Nouvel An
\trans \textit{\textcolor{brown}{\zh{做过年的祭礼}}}
\end{exe}
\begin{exe}
\sn \textcolor{darkblue}{\textbf{\ipa{tsʰi˧ɲi˧, | kʰv̩˧ pv̩˥-tso˩-ɲi˩!}}}
\trans ce soir, on va fêter le Nouvel An!
\trans \textit{\textcolor{brown}{\zh{今天就要过年了!}}}
\end{exe}


\paragraph{\hspace{-0.5cm} {\Large \textcolor{darkblue}{\textbf{\ipa{pv̩˧}}} \textsubscript{2}}} \hypertarget{pv\string_=\string_M2}{}
\markboth{\textcolor{darkblue}{\textbf{\ipa{pv̩˧}}} \textsubscript{2}}{}
\textcolor{teal}{\mytextsc{adjectif}} \hspace{4pt} Ton~: M.

Sec.
\textcolor{brown}{\zh{干燥。}}


\begin{exe}
\sn \textcolor{darkblue}{\textbf{\ipa{le˧-pv̩˧-ze˧}}}
\trans \mytextsc{accomp} \string_ \mytextsc{pfv}
\trans \textit{\textcolor{brown}{\zh{干了}}}
\end{exe}
\begin{exe}
\sn \textcolor{darkblue}{\textbf{\ipa{le˧-pv̩˧ le˧-ʐwæ˩-ze˩}}}
\trans ça a complètement séché/c'est entièrement sec/c'est parfaitement sec
\trans \textit{\textcolor{brown}{\zh{干透了}}}
\end{exe}
\begin{exe}
\sn \textcolor{darkblue}{\textbf{\ipa{pv̩˧-kæ˧-ɻæ˩-gv̩˩}}}
\trans tout sec
\trans \textit{\textcolor{brown}{\zh{全干、完全干}}}
\end{exe}


\paragraph{\hspace{-0.5cm} {\Large \textcolor{darkblue}{\textbf{\ipa{pv̩˧˥}}} \textsubscript{1}}} \hypertarget{pv\string_=\string_M\string_T1}{}
\markboth{\textcolor{darkblue}{\textbf{\ipa{pv̩˧˥}}} \textsubscript{1}}{}
\textcolor{teal}{\mytextsc{verbe}} \hspace{4pt} Ton~: MH.

Enlever, arracher (les mauvaises herbes); couper du fourrage pour les animaux domestiques.
\textcolor{brown}{\zh{拔、扯(草)。}}


\begin{exe}
\sn \textcolor{darkblue}{\textbf{\ipa{zɯ˧ pv̩˩}}}
\trans arracher les mauvaises herbes; couper du fourrage pour les animaux domestiques
\trans \textit{\textcolor{brown}{\zh{拔草}}}
\end{exe}
\begin{exe}
\sn \textcolor{darkblue}{\textbf{\ipa{zɯ˧ | le˧-pv̩˧˥}}}
\trans arracher les mauvaises herbes; couper du fourrage pour les animaux domestiques
\trans \textit{\textcolor{brown}{\zh{拔草}}}
\end{exe}


\paragraph{\hspace{-0.5cm} {\Large \textcolor{darkblue}{\textbf{\ipa{pv̩˧˥}}} \textsubscript{2}}} \hypertarget{pv\string_=\string_M\string_T2}{}
\markboth{\textcolor{darkblue}{\textbf{\ipa{pv̩˧˥}}} \textsubscript{2}}{}
\textcolor{teal}{\mytextsc{verbe}} \hspace{4pt} Ton~: MH.

Dégainer (une arme blanche), sortir du fourreau.
\textcolor{brown}{\zh{拉出(剑……)。}}


\begin{exe}
\sn \textcolor{darkblue}{\textbf{\ipa{ʁæ˧mi˧ | tʰi˧-pv̩˧˥}}}
\trans dégainer une épée
\trans \textit{\textcolor{brown}{\zh{拉出剑}}}
\end{exe}
\begin{exe}
\sn \textcolor{darkblue}{\textbf{\ipa{gæ˩-pv̩˧˥}}}
\trans dégainer, sortir une arme de son fourreau
\trans \textit{\textcolor{brown}{\zh{拉出(剑……)}}}
\end{exe}
\begin{exe}
\sn \textcolor{darkblue}{\textbf{\ipa{ʁæ˧mi˧ | gæ˩-pv̩˧˥}}}
\trans dégainer une épée
\trans \textit{\textcolor{brown}{\zh{拉出剑}}}
\end{exe}


\paragraph{\hspace{-0.5cm} {\Large \textcolor{darkblue}{\textbf{\ipa{pv̩˧˥}}} \textsubscript{3}}} \hypertarget{pv\string_=\string_M\string_T3}{}
\markboth{\textcolor{darkblue}{\textbf{\ipa{pv̩˧˥}}} \textsubscript{3}}{}
\textcolor{teal}{\mytextsc{classificateur}} \hspace{4pt} Ton~: MH\textsubscript{a}.

Classificateur des pas/enjambées; emprunt au chinois.
\textcolor{brown}{\zh{量词:步。}}

 Emprunt~: chinois \textcolor{brown}{\zh{步}}



\paragraph{\hspace{-0.5cm} {\Large \textcolor{darkblue}{\textbf{\ipa{pv̩˩\textsubscript{a}}}} \textsubscript{1}}} \hypertarget{pv\string_=\string_Ba1}{}
\markboth{\textcolor{darkblue}{\textbf{\ipa{pv̩˩\textsubscript{a}}}} \textsubscript{1}}{}
\textcolor{teal}{\mytextsc{verbe}} \hspace{4pt} Ton~: L\textsubscript{a}.

Raccompagner; escorter; mener, conduire (du bétail).
\textcolor{brown}{\zh{送行。}}


\begin{exe}
\sn \textcolor{darkblue}{\textbf{\ipa{hĩ˧bæ˧ pv̩˥}}}
\trans raccompagner un invité
\trans \textit{\textcolor{brown}{\zh{送客}}}
\end{exe}


\paragraph{\hspace{-0.5cm} {\Large \textcolor{darkblue}{\textbf{\ipa{pv̩˩\textsubscript{a}}}} \textsubscript{2}}} \hypertarget{pv\string_=\string_Ba2}{}
\markboth{\textcolor{darkblue}{\textbf{\ipa{pv̩˩\textsubscript{a}}}} \textsubscript{2}}{}
\textcolor{teal}{\mytextsc{verbe}} \hspace{4pt} Ton~: L\textsubscript{a}.

Autoriser (ex.: un mariage); demander (à quelqu'un de faire quelque chose), faire faire; commanditer, être commanditaire/investisseur (ex.: pour une caravane).
\textcolor{brown}{\zh{让,安排,投资,要求。}}


\begin{exe}
\sn \textcolor{darkblue}{\textbf{\ipa{sɯ˧pʰi˧-ɳɯ˧ | pv̩˩-kʰɯ˥-ɲi˩!}}}
\trans c'est le seigneur qui était le commanditaire!
\trans \textit{\textcolor{brown}{\zh{(马帮)是土司来投资的!}}}
\end{exe}
\begin{exe}
\sn \textcolor{darkblue}{\textbf{\ipa{ʈʂʰɯ˧ | ɖʐe˧ ʂe˧ pv̩˩-kʰɯ˩-tso˩-ɲi˩!}}}
\trans c'est elle/lui qui apporte le capital/qui commandite! (ex.: pour une caravane)
\trans \textit{\textcolor{brown}{\zh{是他来投资的!(如:马帮)}}}
\end{exe}
\begin{exe}
\sn \textcolor{darkblue}{\textbf{\ipa{hĩ˧-ɳɯ˩ | pv̩˩-mɤ˩-kʰɯ˥!}}}
\trans On n'est pas autorisé à y aller! (Contexte: discussion au sujet des difficultés pour l'enquêteur d'accès à une localité où sont parlées des langues naish, dans le Sichuan. La consultante résume: ``On n'est pas autorisé à y aller! / L'accès n'est pas autorisé!")
\trans \textit{\textcolor{brown}{\zh{人家不让去!}}}
\end{exe}


\paragraph{\hspace{-0.5cm} {\Large \textcolor{darkblue}{\textbf{\ipa{pv̩˩\textsubscript{a}}}} \textsubscript{3}}} \hypertarget{pv\string_=\string_Ba3}{}
\markboth{\textcolor{darkblue}{\textbf{\ipa{pv̩˩\textsubscript{a}}}} \textsubscript{3}}{}
\textcolor{teal}{\mytextsc{verbe}} \hspace{4pt} Ton~: L\textsubscript{a}.

Peigner.
\textcolor{brown}{\zh{梳。}}


\begin{exe}
\sn \textcolor{darkblue}{\textbf{\ipa{ʁo˧qʰwɤ˩ pv̩˩}}}
\trans se peigner
\trans \textit{\textcolor{brown}{\zh{梳头}}}
\end{exe}
\begin{exe}
\sn \textcolor{darkblue}{\textbf{\ipa{ʁo˧ pv̩˥}}}
\trans se peigner
\trans \textit{\textcolor{brown}{\zh{梳头}}}
\end{exe}


\paragraph{\hspace{-0.5cm} {\Large \textcolor{darkblue}{\textbf{\ipa{pv̩˩ɭɯ˥}}}}} \hypertarget{pv\string_=\string_Bl\string_RM\string_T1}{}
\markboth{\textcolor{darkblue}{\textbf{\ipa{pv̩˩ɭɯ˥}}}}{}
\textcolor{teal}{\mytextsc{nom}} \hspace{4pt} Ton~: LH.

Bouton (sur un vêtement).
\textcolor{brown}{\zh{扣子。}}


 \mytextsc{clf}~: \textcolor{darkblue}{\textbf{\ipa{ɭɯ˧}}} 

\paragraph{\hspace{-0.5cm} {\Large \textcolor{darkblue}{\textbf{\ipa{pv̩˧ɭɯ˧}}}}} \hypertarget{pv\string_=\string_Ml\string_RM\string_M1}{}
\markboth{\textcolor{darkblue}{\textbf{\ipa{pv̩˧ɭɯ˧}}}}{}
\textcolor{teal}{\mytextsc{verbe}} \hspace{4pt} Ton~: .

Rouler (une pierre roule).
\textcolor{brown}{\zh{滚动(石头滚动)。}}


\begin{exe}
\sn \textcolor{darkblue}{\textbf{\ipa{pv̩˧ɭɯ˧-ze˩}}}
\trans \mytextsc{pfv}
\trans \textit{\textcolor{brown}{\zh{滚动了}}}
\end{exe}
\begin{exe}
\sn \textcolor{darkblue}{\textbf{\ipa{le˧-pv̩˧-ɭɯ˧}}}
\trans \mytextsc{accomp}
\trans \textit{\textcolor{brown}{\zh{\mytextsc{accomp}}}}
\end{exe}


\paragraph{\hspace{-0.5cm} {\Large \textcolor{darkblue}{\textbf{\ipa{pv̩˧lv̩˧}}}}} \hypertarget{pv\string_=\string_Mlv\string_=\string_M1}{}
\markboth{\textcolor{darkblue}{\textbf{\ipa{pv̩˧lv̩˧}}}}{}
\textcolor{teal}{\mytextsc{nom}} \hspace{4pt} Ton~: M.

Champ sec/pluvial.
\textcolor{brown}{\zh{旱地。}}


 \mytextsc{clf}~: \textcolor{darkblue}{\textbf{\ipa{pʰv̩˩}}} 

\paragraph{\hspace{-0.5cm} {\Large \textcolor{darkblue}{\textbf{\ipa{pv̩˩mi˩}}}}} \hypertarget{pv\string_=\string_Bmi\string_B1}{}
\markboth{\textcolor{darkblue}{\textbf{\ipa{pv̩˩mi˩}}}}{}
\textcolor{teal}{\mytextsc{nom}} \hspace{4pt} Ton~: L.

Peigne grossier, à dents relativement écartées.
\textcolor{brown}{\zh{粗齿梳子。}}


 \mytextsc{clf}~: \textcolor{darkblue}{\textbf{\ipa{nɑ˧}}} 

\paragraph{\hspace{-0.5cm} {\Large \textcolor{darkblue}{\textbf{\ipa{pv̩˩pv̩˧}}}}} \hypertarget{pv\string_=\string_Bpv\string_=\string_M1}{}
\markboth{\textcolor{darkblue}{\textbf{\ipa{pv̩˩pv̩˧}}}}{}
\textcolor{teal}{\mytextsc{nom}} \hspace{4pt} Ton~: LM.

Poche.
\textcolor{brown}{\zh{衣兜。}}


\begin{exe}
\sn \textcolor{darkblue}{\textbf{\ipa{bɑ˩lɑ˩-pv̩˥pv̩˩}}}
\trans poche intérieure de chemise; on y serrait de petits objets: pièces de monnaie, tabac…
\trans \textit{\textcolor{brown}{\zh{上衣兜子}}}
\end{exe}
\textit{Syn~:} \hyperlink{tA\string_Bdv\string_=\string_M\string_T1}{\textcolor{darkblue}{\textbf{\ipa{tɑ˩dv̩˧˥}}}}. 

\paragraph{\hspace{-0.5cm} {\Large \textcolor{darkblue}{\textbf{\ipa{pv̩˧qʰwɤ˥}}}}} \hypertarget{pv\string_=\string_Mq\string_hw7\string_T1}{}
\markboth{\textcolor{darkblue}{\textbf{\ipa{pv̩˧qʰwɤ˥}}}}{}
\textcolor{teal}{\mytextsc{nom}} \hspace{4pt} Ton~: H\#.

Navette du métier à tisser: navette traditionnelle, en bois (n'est plus en usage actuellement, remplacée par une navette plus simple).
\textcolor{brown}{\zh{梭,梭子(传统的,木头做的)。}}


\begin{exe}
\sn \textcolor{darkblue}{\textbf{\ipa{ɣɯ˧dzo˩-bv̩˩ | pv̩˧qʰwɤ˥}}}
\trans la navette du métier à tisser
\trans \textit{\textcolor{brown}{\zh{织布机的梭子}}}
\end{exe}
 \mytextsc{clf}~: \textcolor{darkblue}{\textbf{\ipa{ɭɯ˧}}} \textit{Voir~:} \hyperlink{k\string_hM\string_Bpv\string_=\string_B1}{\textcolor{darkblue}{\textbf{\ipa{kʰɯ˩pv̩˩}}}} 

\paragraph{\hspace{-0.5cm} {\Large \textcolor{darkblue}{\textbf{\ipa{pv̩˧ɻ\#˥}}}}} \hypertarget{pv\string_=\string_Mr£`\#\string_T1}{}
\markboth{\textcolor{darkblue}{\textbf{\ipa{pv̩˧ɻ\#˥}}}}{}
\textcolor{teal}{\mytextsc{nom}} \hspace{4pt} Ton~: \#H.

Habit tibétain en laine (vêtement de grand prix).
\textcolor{brown}{\zh{氆氇。}}


 \mytextsc{clf}~: \textcolor{darkblue}{\textbf{\ipa{tsʰi˥}}} 

\paragraph{\hspace{-0.5cm} {\Large \textcolor{darkblue}{\textbf{\ipa{pv̩˧ʂɯ˩}}}}} \hypertarget{pv\string_=\string_Ms`M\string_B1}{}
\markboth{\textcolor{darkblue}{\textbf{\ipa{pv̩˧ʂɯ˩}}}}{}
\textcolor{teal}{\mytextsc{nom}} \hspace{4pt} Ton~: L\#.

Ambre.
\textcolor{brown}{\zh{琥珀。}}


 \mytextsc{clf}~: \textcolor{darkblue}{\textbf{\ipa{ɭɯ˧}}} 

\paragraph{\hspace{-0.5cm} {\Large \textcolor{darkblue}{\textbf{\ipa{pv̩˩tɑ˩}}}}} \hypertarget{pv\string_=\string_BtA\string_B1}{}
\markboth{\textcolor{darkblue}{\textbf{\ipa{pv̩˩tɑ˩}}}}{}
\textcolor{teal}{\mytextsc{nom}} \hspace{4pt} Ton~: L.

Seau.
\textcolor{brown}{\zh{桶。}}


 \mytextsc{clf}~: \textcolor{darkblue}{\textbf{\ipa{ɭɯ˧}}} 

\paragraph{\hspace{-0.5cm} {\Large \textcolor{darkblue}{\textbf{\ipa{pv̩˩tsɯ˧˥}}}}} \hypertarget{pv\string_=\string_BtsM\string_M\string_T1}{}
\markboth{\textcolor{darkblue}{\textbf{\ipa{pv̩˩tsɯ˧˥}}}}{}
\textcolor{teal}{\mytextsc{nom}} \hspace{4pt} Ton~: LM+MH\#.
\ding{202} 
Peigne fin (utilisé pour épouiller).
\textcolor{brown}{\zh{篦子(用来梳虱子)。}}


\ding{203} 
Fils de fer dans un cadre de bois: sorte de peigne dans lequel la trame est emprisonnée, et qui sert à tasser les fils à mesure que l'on tisse.
\textcolor{brown}{\zh{用来夯实布料的木头架子,里面有铁丝。}}


 \mytextsc{clf}~: \textcolor{darkblue}{\textbf{\ipa{nɑ˧}}} 

\paragraph{\hspace{-0.5cm} {\Large \textcolor{darkblue}{\textbf{\ipa{pv̩˧ʈʂɯ˩}}}}} \hypertarget{pv\string_=\string_Mt`s`M\string_B1}{}
\markboth{\textcolor{darkblue}{\textbf{\ipa{pv̩˧ʈʂɯ˩}}}}{}
\textcolor{teal}{\mytextsc{verbe}} \hspace{4pt} Ton~: L\#.

Presser, serrer.
\textcolor{brown}{\zh{挤、挤压。}}


\begin{exe}
\sn \textcolor{darkblue}{\textbf{\ipa{njɤ˧-ɳɯ˧ | pv̩˧ʈʂɯ˩-bi˩!}}}
\trans Je vais presser ça! / je m'occupe de serrer ça/presser ça!
\trans \textit{\textcolor{brown}{\zh{我来压吧!}}}
\end{exe}
\begin{exe}
\sn \textcolor{darkblue}{\textbf{\ipa{le˧-pv̩˥ʈʂɯ˩}}}
\trans \mytextsc{accomp}
\trans \textit{\textcolor{brown}{\zh{\mytextsc{accomp}}}}
\end{exe}


\paragraph{\hspace{-0.5cm} {\Large \textcolor{darkblue}{\textbf{\ipa{pv̩˩tsɯ˧-pv̩˥mi˩}}}}} \hypertarget{pv\string_=\string_BtsM\string_M-pv\string_=\string_Tmi\string_B1}{}
\markboth{\textcolor{darkblue}{\textbf{\ipa{pv̩˩tsɯ˧-pv̩˥mi˩}}}}{}
\textcolor{teal}{\mytextsc{nom}} \hspace{4pt} Ton~: LM+\#H-.

Peignes.
\textcolor{brown}{\zh{梳子(总称)。}}




\paragraph{\hspace{-0.5cm} {\Large \textcolor{darkblue}{\textbf{\ipa{pv̩˩ʈʰɯ˧}}}}} \hypertarget{pv\string_=\string_Bt`\string_hM\string_M1}{}
\markboth{\textcolor{darkblue}{\textbf{\ipa{pv̩˩ʈʰɯ˧}}}}{}
\textcolor{teal}{\mytextsc{nom}} \hspace{4pt} Ton~: LM.

Prénom féminin.
\textcolor{brown}{\zh{女性名字。}}




\paragraph{\hspace{-0.5cm} {\Large \textcolor{darkblue}{\textbf{\ipa{pv˧tsɤ\#˥}}}}} \hypertarget{pv\string_Mts7\#\string_T1}{}
\markboth{\textcolor{darkblue}{\textbf{\ipa{pv˧tsɤ\#˥}}}}{}
\textcolor{teal}{\mytextsc{nom}} \hspace{4pt} Ton~: \#H.

Mortaise.
\textcolor{brown}{\zh{榫眼。}}


\begin{exe}
\sn \textcolor{darkblue}{\textbf{\ipa{pv˧tsɤ˧ | ɖɯ˧-ɭɯ˧}}}
\trans une mortaise
\trans \textit{\textcolor{brown}{\zh{一个榫}}}
\end{exe}
 \mytextsc{clf}~: \textcolor{darkblue}{\textbf{\ipa{ɭɯ˧}}} 

\newpage
\section*{\centering- \textcolor{darkblue}{\textbf{\ipa{pʰ}}} -}
\paragraph{\hspace{-0.5cm} {\Large \textcolor{darkblue}{\textbf{\ipa{‑pʰæ˥di˩}}}}} \hypertarget{‑p\string_h\{\string_Tdi\string_B1}{}
\markboth{\textcolor{darkblue}{\textbf{\ipa{‑pʰæ˥di˩}}}}{}
\textcolor{teal}{\mytextsc{adverbe}} \hspace{4pt} Ton~: .

Semblable à; comme; comme si; on dirait que.
\textcolor{brown}{\zh{像、好像。}}


\begin{exe}
\sn \textcolor{darkblue}{\textbf{\ipa{mɤ˧-pʰæ˥di˩}}}
\trans différent, peu ressemblant; par exemple, rencontrant un enfant qui a beaucoup grandi en l'espace de quelques années, on peut faire la réflexion selon laquelle il/elle ne ressemble plus à ce qu'il était auparavant
\trans \textit{\textcolor{brown}{\zh{不像(比如:几年没见一个孩子,见的时候,觉得不像以前的样子了)}}}
\end{exe}


\paragraph{\hspace{-0.5cm} {\Large \textcolor{darkblue}{\textbf{\ipa{pʰæ˧\textsubscript{b}}}}}} \hypertarget{p\string_h\{\string_Mb1}{}
\markboth{\textcolor{darkblue}{\textbf{\ipa{pʰæ˧\textsubscript{b}}}}}{}
\textcolor{teal}{\mytextsc{verbe}} \hspace{4pt} Ton~: M\textsubscript{b}.
\ding{202} 
Attacher (un animal).
\textcolor{brown}{\zh{拴(牛……)。}}


\begin{exe}
\sn \textcolor{darkblue}{\textbf{\ipa{tʰi˧-pʰæ˧+ze˧}}}
\trans \mytextsc{dur} \string_ \mytextsc{pfv}
\trans \textit{\textcolor{brown}{\zh{\mytextsc{dur} \string_ \mytextsc{pfv}}}}
\end{exe}
\begin{exe}
\sn \textcolor{darkblue}{\textbf{\ipa{pʰæ˧\textasciitilde{}pʰæ˧}}}
\trans \mytextsc{red}
\trans \textit{\textcolor{brown}{\zh{\mytextsc{重叠}}}}
\end{exe}
\ding{203} 
Être lié, avoir des liens étroits: par exemple, les membres d'une famille ont des liens profonds.
\textcolor{brown}{\zh{有联系,有缘分,有深的关系。}}


\begin{exe}
\sn \textcolor{darkblue}{\textbf{\ipa{pʰæ˧\textasciitilde{}pʰæ˧=ɻæ˩ ɲi˩!}}}
\trans [Ils] sont unis, ils sont en couple! (au sujet de deux jeunes personnes)
\trans \textit{\textcolor{brown}{\zh{他们有联系了/他们成了一俩了!(关于两个年轻人)}}}
\end{exe}


\paragraph{\hspace{-0.5cm} {\Large \textcolor{darkblue}{\textbf{\ipa{pʰæ˧qʰwɤ˩}}}}} \hypertarget{p\string_h\{\string_Mq\string_hw7\string_B1}{}
\markboth{\textcolor{darkblue}{\textbf{\ipa{pʰæ˧qʰwɤ˩}}}}{}
\textcolor{teal}{\mytextsc{nom}} \hspace{4pt} Ton~: L\#.

Visage.
\textcolor{brown}{\zh{脸。}}


 \mytextsc{clf}~: \textcolor{darkblue}{\textbf{\ipa{ɭɯ˧}}} 

\paragraph{\hspace{-0.5cm} {\Large \textcolor{darkblue}{\textbf{\ipa{pʰæ˧ʂv̩˧-di˧˥}}}}} \hypertarget{p\string_h\{\string_Ms`v\string_=\string_M-di\string_M\string_T1}{}
\markboth{\textcolor{darkblue}{\textbf{\ipa{pʰæ˧ʂv̩˧-di˧˥}}}}{}
\textcolor{teal}{\mytextsc{nom}} \hspace{4pt} Ton~: .

Foulard (périphrase); autrefois, on utilisait un fichu, \textcolor{darkblue}{\textbf{\ipa{/qʰwæ˧ʈɯ˥/}}}.
\textcolor{brown}{\zh{围巾。}}




\paragraph{\hspace{-0.5cm} {\Large \textcolor{darkblue}{\textbf{\ipa{pʰæ˧tɕi˥}}}}} \hypertarget{p\string_h\{\string_Mts£i\string_T1}{}
\markboth{\textcolor{darkblue}{\textbf{\ipa{pʰæ˧tɕi˥}}}}{}
\textcolor{teal}{\mytextsc{nom}} \hspace{4pt} Ton~: H\#.
\ding{202} 
Jeune homme, petit gars.
\textcolor{brown}{\zh{小伙子、 青年男子。}}


\begin{exe}
\sn \textcolor{darkblue}{\textbf{\ipa{pʰæ˧tɕi˥-zo˩}}}
\trans jeune homme
\trans \textit{\textcolor{brown}{\zh{小伙子}}}
\end{exe}
\begin{exe}
\sn \textcolor{darkblue}{\textbf{\ipa{pʰæ˧tɕi˥=ɻæ˩}}}
\trans jeunes hommes; les jeunes hommes
\trans \textit{\textcolor{brown}{\zh{小伙子们}}}
\end{exe}
\ding{203} 
Nom du premier pilier (il y a deux grands piliers dans la maison traditionnelle), celui qui est le plus près de la porte: c'est le pilier masculin, ``le jeune homme", le second étant féminin, ``la jeune femme".
\textcolor{brown}{\zh{第一根柱子的名称(代表男人、男性的那根柱子)。}}


 \mytextsc{clf}~: \textcolor{darkblue}{\textbf{\ipa{v̩˧}}} \textcolor{darkblue}{\textbf{\ipa{v̩˧}}} 

\paragraph{\hspace{-0.5cm} {\Large \textcolor{darkblue}{\textbf{\ipa{pʰæ˧ʈʂʰæ˧lo\#˥}}}}} \hypertarget{p\string_h\{\string_Mt`s`\string_h\{\string_Mlo\#\string_T1}{}
\markboth{\textcolor{darkblue}{\textbf{\ipa{pʰæ˧ʈʂʰæ˧lo\#˥}}}}{}
\textcolor{teal}{\mytextsc{nom}} \hspace{4pt} Ton~: \#H.

Bassine pour se laver le visage; le même ustensile est utilisé pour servir les nourritures si un bol serait trop petit: pour servir le riz, les soupes...
\textcolor{brown}{\zh{脸盆,木盆。}}


 \mytextsc{clf}~: \textcolor{darkblue}{\textbf{\ipa{ɭɯ˧}}} 

\paragraph{\hspace{-0.5cm} {\Large \textcolor{darkblue}{\textbf{\ipa{pʰæ˧˥}}}}} \hypertarget{p\string_h\{\string_M\string_T1}{}
\markboth{\textcolor{darkblue}{\textbf{\ipa{pʰæ˧˥}}}}{}
\textcolor{teal}{\mytextsc{verbe}} \hspace{4pt} Ton~: MH.

Écarter, pousser, jouer des coudes.
\textcolor{brown}{\zh{推搡。}}


\begin{exe}
\sn \textcolor{darkblue}{\textbf{\ipa{ɖɯ˩-tɕo˧ pʰæ˧˥, | ʈʂʰɯ˧-tɕo˧ pʰæ˧˥}}}
\trans pousser par ici, pousser par là / jouer des coudes par ci, jouer des coudes par là (ex.: à la gare, quand il y a presse pour acheter un billet de train)
\trans \textit{\textcolor{brown}{\zh{东推西挤}}}
\end{exe}
\begin{exe}
\sn \textcolor{darkblue}{\textbf{\ipa{[Housebuilding2] ʈʂe˧ | le˧-pʰæ˩\textasciitilde{}pʰæ˩}}}
\trans rejeter la terre de droite et de gauche : une poule gratte la terre à la recherche de nourriture, et fait voler de la terre de droite et de gauche
\trans \textit{\textcolor{brown}{\zh{将土扔这里扔那里:一只鸡在抓地找吃的,让土飞这里飞那里}}}
\end{exe}


\paragraph{\hspace{-0.5cm} {\Large \textcolor{darkblue}{\textbf{\ipa{pʰæ˧˥\textsubscript{a}}}}}} \hypertarget{p\string_h\{\string_M\string_Ta1}{}
\markboth{\textcolor{darkblue}{\textbf{\ipa{pʰæ˧˥\textsubscript{a}}}}}{}
\textcolor{teal}{\mytextsc{classificateur}} \hspace{4pt} Ton~: MH\textsubscript{a}.

Classificateur des objets plats: feuilles de papier….
\textcolor{brown}{\zh{量词:平面的东西,如:纸(一张、一片)。}}




\paragraph{\hspace{-0.5cm} {\Large \textcolor{darkblue}{\textbf{\ipa{pʰe˧}}}}} \hypertarget{p\string_he\string_M1}{}
\markboth{\textcolor{darkblue}{\textbf{\ipa{pʰe˧}}}}{}
\textcolor{teal}{\mytextsc{interjection}} \hspace{4pt} Ton~: M.

Mais pas du tout! Mais non, enfin! Interjection par laquelle le locuteur signale qu'il reprend la main: que ses interlocuteurs lui paraissent être dans l'erreur, et qu'il va rectifier.
\textcolor{brown}{\zh{呸!(表示唾弃的感叹词)。}}




\paragraph{\hspace{-0.5cm} {\Large \textcolor{darkblue}{\textbf{\ipa{pʰɤ˧bɤ˧}}}}} \hypertarget{p\string_h7\string_Mb7\string_M1}{}
\markboth{\textcolor{darkblue}{\textbf{\ipa{pʰɤ˧bɤ˧}}}}{}
\textcolor{teal}{\mytextsc{nom}} \hspace{4pt} Ton~: M.

Cadeau (choses à manger ou boire; essentiellement: tabac, thé, bonbons, vin; on n'offre généralement pas de vêtements).
\textcolor{brown}{\zh{礼物。}}


\begin{exe}
\sn \textcolor{darkblue}{\textbf{\ipa{pʰɤ˧bɤ˧ po˧-tsʰɯ˧˥}}}
\trans amener des cadeaux
\trans \textit{\textcolor{brown}{\zh{带礼物}}}
\end{exe}
\begin{exe}
\sn \textcolor{darkblue}{\textbf{\ipa{ʈʂʰɯ˧ | ʈæ˧ʂɯ˧ ki˧-hĩ˧ pʰɤ˧bɤ˧ ŋi˩.}}}
\trans C'est un cadeau que m'a donné Dashi!
\trans \textit{\textcolor{brown}{\zh{这是达石给的礼物!}}}
\end{exe}
\begin{exe}
\sn \textcolor{darkblue}{\textbf{\ipa{ʈʂʰɯ˧ | ʈæ˧ʂɯ˧ tʰi˧-ki˧-hĩ˧ pʰɤ˧bɤ˧ ŋi˩.}}}
\trans C'est un cadeau que Dashi te fait! Voici un cadeau de la part de Dashi!
\trans \textit{\textcolor{brown}{\zh{这是达石送你的礼物!}}}
\end{exe}
 \mytextsc{clf}~: \textcolor{darkblue}{\textbf{\ipa{kʰwɤ˥}}} 

\paragraph{\hspace{-0.5cm} {\Large \textcolor{darkblue}{\textbf{\ipa{pʰɤ˧fv̩˩}}}}} \hypertarget{p\string_h7\string_Mfv\string_=\string_B1}{}
\markboth{\textcolor{darkblue}{\textbf{\ipa{pʰɤ˧fv̩˩}}}}{}
\textcolor{teal}{\mytextsc{nom}} \hspace{4pt} Ton~: L\#.

Théière.
\textcolor{brown}{\zh{茶壶。}}

 Emprunt~: chinois \textcolor{brown}{\zh{壶?}}



\paragraph{\hspace{-0.5cm} {\Large \textcolor{darkblue}{\textbf{\ipa{pʰɤ˧pʰv̩\#˥}}}}} \hypertarget{p\string_h7\string_Mp\string_hv\string_=\#\string_T1}{}
\markboth{\textcolor{darkblue}{\textbf{\ipa{pʰɤ˧pʰv̩\#˥}}}}{}
\textcolor{teal}{\mytextsc{nom}} \hspace{4pt} Ton~: \#H.

Chacal mâle.
\textcolor{brown}{\zh{公豺。}}


 \mytextsc{clf}~: \textcolor{darkblue}{\textbf{\ipa{mi˩}}} 

\paragraph{\hspace{-0.5cm} {\Large \textcolor{darkblue}{\textbf{\ipa{pʰɤ˩mi˩}}}}} \hypertarget{p\string_h7\string_Bmi\string_B1}{}
\markboth{\textcolor{darkblue}{\textbf{\ipa{pʰɤ˩mi˩}}}}{}
\textcolor{teal}{\mytextsc{nom}} \hspace{4pt} Ton~: L.

Femelle chacal.
\textcolor{brown}{\zh{母豺。}}


 \mytextsc{clf}~: \textcolor{darkblue}{\textbf{\ipa{mi˩}}} 

\paragraph{\hspace{-0.5cm} {\Large \textcolor{darkblue}{\textbf{\ipa{pʰɤ˩-so˩\textasciitilde{}so˥}}}}} \hypertarget{p\string_h7\string_B-so\string_B~so\string_T1}{}
\markboth{\textcolor{darkblue}{\textbf{\ipa{pʰɤ˩-so˩\textasciitilde{}so˥}}}}{}
\textcolor{teal}{\mytextsc{adjectif}} \hspace{4pt} Ton~: .

Meuble: la terre est meuble.
\textcolor{brown}{\zh{松(土)。}}


\begin{exe}
\sn \textcolor{darkblue}{\textbf{\ipa{ʈʂe˧ | pʰɤ˩-so˩\textasciitilde{}so˥-gv̩˩}}}
\trans la terre est meuble, la terre a été ameublie
\trans \textit{\textcolor{brown}{\zh{土是松的}}}
\end{exe}


\paragraph{\hspace{-0.5cm} {\Large \textcolor{darkblue}{\textbf{\ipa{pʰɤ˩zo˩}}}}} \hypertarget{p\string_h7\string_Bzo\string_B1}{}
\markboth{\textcolor{darkblue}{\textbf{\ipa{pʰɤ˩zo˩}}}}{}
\textcolor{teal}{\mytextsc{nom}} \hspace{4pt} Ton~: L.

Petit chacal.
\textcolor{brown}{\zh{豺崽子。}}


 \mytextsc{clf}~: \textcolor{darkblue}{\textbf{\ipa{ɭɯ˧}}} 

\paragraph{\hspace{-0.5cm} {\Large \textcolor{darkblue}{\textbf{\ipa{pʰɤ˩˧}}}}} \hypertarget{p\string_h7\string_B\string_M1}{}
\markboth{\textcolor{darkblue}{\textbf{\ipa{pʰɤ˩˧}}}}{}
\textcolor{teal}{\mytextsc{nom}} \hspace{4pt} Ton~: LM.

Hyène, chacal.
\textcolor{brown}{\zh{豺。}}


\begin{exe}
\sn \textcolor{darkblue}{\textbf{\ipa{pʰɤ˩ hwæ˧-ze˧}}}
\trans ...a acheté (une) hyène
\trans \textit{\textcolor{brown}{\zh{买了豺}}}
\end{exe}
\begin{exe}
\sn \textcolor{darkblue}{\textbf{\ipa{pʰɤ˩ dzɯ˧-ze˩}}}
\trans ...a mangé (une) hyène
\trans \textit{\textcolor{brown}{\zh{吃了豺}}}
\end{exe}
 \mytextsc{clf}~: \textcolor{darkblue}{\textbf{\ipa{mi˩}}} 

\paragraph{\hspace{-0.5cm} {\Large \textcolor{darkblue}{\textbf{\ipa{pʰi˧}}}}} \hypertarget{p\string_hi\string_M1}{}
\markboth{\textcolor{darkblue}{\textbf{\ipa{pʰi˧}}}}{}
\textcolor{teal}{\mytextsc{nom}} \hspace{4pt} Ton~: M.

Tissu de lin; anciennement le tissu dont étaient faits ts les vêtements (cf récit ``travaux").
\textcolor{brown}{\zh{麻布,亚麻布。}}


\begin{exe}
\sn \textcolor{darkblue}{\textbf{\ipa{pʰi˩ dɑ˩˥}}}
\trans tisser le lin, faire du tissu de lin
\trans \textit{\textcolor{brown}{\zh{织麻布}}}
\end{exe}
 \mytextsc{clf}~: \textcolor{darkblue}{\textbf{\ipa{kʰwɤ˥}}} 

\paragraph{\hspace{-0.5cm} {\Large \textcolor{darkblue}{\textbf{\ipa{pʰi˧\textsubscript{b}}}}}} \hypertarget{p\string_hi\string_Mb1}{}
\markboth{\textcolor{darkblue}{\textbf{\ipa{pʰi˧\textsubscript{b}}}}}{}
\textcolor{teal}{\mytextsc{verbe}} \hspace{4pt} Ton~: M\textsubscript{b}.

Vanner à l'aide d'un crible (vannerie ronde): on fait ``sauter" le grain dans un crible, et le vent emporte la balle.
\textcolor{brown}{\zh{簸。}}


\begin{exe}
\sn \textcolor{darkblue}{\textbf{\ipa{hɑ˧ pʰi˧}}}
\trans vanner du grain
\trans \textit{\textcolor{brown}{\zh{簸粮食}}}
\end{exe}
\begin{exe}
\sn \textcolor{darkblue}{\textbf{\ipa{pʰi˧\textasciitilde{}pʰi˧}}}
\trans \mytextsc{red}
\trans \textit{\textcolor{brown}{\zh{\mytextsc{重叠:簸一簸}}}}
\end{exe}
\begin{exe}
\sn \textcolor{darkblue}{\textbf{\ipa{le˧-pʰi˧(-ze˧)}}}
\trans \mytextsc{accomp} \string_ (\mytextsc{pfv})
\trans \textit{\textcolor{brown}{\zh{簸了}}}
\end{exe}


\paragraph{\hspace{-0.5cm} {\Large \textcolor{darkblue}{\textbf{\ipa{pʰi˧kʰv̩˧}}}}} \hypertarget{p\string_hi\string_Mk\string_hv\string_=\string_M1}{}
\markboth{\textcolor{darkblue}{\textbf{\ipa{pʰi˧kʰv̩˧}}}}{}
\textcolor{teal}{\mytextsc{nom}} \hspace{4pt} Ton~: M.

Pelle à poussière.
\textcolor{brown}{\zh{畚箕。}}


 \mytextsc{clf}~: \textcolor{darkblue}{\textbf{\ipa{nɑ˧}}} 

\paragraph{\hspace{-0.5cm} {\Large \textcolor{darkblue}{\textbf{\ipa{pʰi˧kʰv̩˧}}}}} \hypertarget{p\string_hi\string_Mk\string_hv\string_=\string_M1}{}
\markboth{\textcolor{darkblue}{\textbf{\ipa{pʰi˧kʰv̩˧}}}}{}
\textcolor{teal}{\mytextsc{nom}} \hspace{4pt} Ton~: M.

Coquillage.
\textcolor{brown}{\zh{贝壳。}}




\paragraph{\hspace{-0.5cm} {\Large \textcolor{darkblue}{\textbf{\ipa{pʰi˧li˩}}}}} \hypertarget{p\string_hi\string_Mli\string_B1}{}
\markboth{\textcolor{darkblue}{\textbf{\ipa{pʰi˧li˩}}}}{}
\textcolor{teal}{\mytextsc{nom}} \hspace{4pt} Ton~: L\#.

Papillon.
\textcolor{brown}{\zh{蝴蝶。}}


 \mytextsc{clf}~: \textcolor{darkblue}{\textbf{\ipa{mi˩}}} 

\paragraph{\hspace{-0.5cm} {\Large \textcolor{darkblue}{\textbf{\ipa{pʰi˧mo˩}}} \textsubscript{1}}} \hypertarget{p\string_hi\string_Mmo\string_B1}{}
\markboth{\textcolor{darkblue}{\textbf{\ipa{pʰi˧mo˩}}} \textsubscript{1}}{}
\textcolor{teal}{\mytextsc{nom}} \hspace{4pt} Ton~: L\#.

Vanneuse.
\textcolor{brown}{\zh{簸箕(用来簸粮食等)。}}


 \mytextsc{clf}~: \textcolor{darkblue}{\textbf{\ipa{nɑ˧}}} \textit{Voir~:} \hyperlink{p\string_hi\string_Mmo\string_B2}{\textcolor{darkblue}{\textbf{\ipa{pʰi˧mo˩}}} \textsubscript{2}} 

\paragraph{\hspace{-0.5cm} {\Large \textcolor{darkblue}{\textbf{\ipa{pʰi˧mo˩}}} \textsubscript{2}}} \hypertarget{p\string_hi\string_Mmo\string_B2}{}
\markboth{\textcolor{darkblue}{\textbf{\ipa{pʰi˧mo˩}}} \textsubscript{2}}{}
\textcolor{teal}{\mytextsc{nom}} \hspace{4pt} Ton~: L\#.

Piège pour attraper des oiseaux.
\textcolor{brown}{\zh{抓鸟的圈套。}}


\begin{exe}
\sn \textcolor{darkblue}{\textbf{\ipa{v̩˩dze˩ qo˥-di˩, | pʰi˧mo˩!}}}
\trans Ce dont on se sert pour attraper les oiseaux, on appelle ça ``piège pour oiseaux"!
\trans \textit{\textcolor{brown}{\zh{抓鸟的东西,(叫做)圈套!}}}
\end{exe}
\textit{Voir~:} \hyperlink{p\string_hi\string_Mmo\string_B1}{\textcolor{darkblue}{\textbf{\ipa{pʰi˧mo˩}}} \textsubscript{1}} 

\paragraph{\hspace{-0.5cm} {\Large \textcolor{darkblue}{\textbf{\ipa{pʰi˧tsʰo\#˥}}}}} \hypertarget{p\string_hi\string_Mts\string_ho\#\string_T1}{}
\markboth{\textcolor{darkblue}{\textbf{\ipa{pʰi˧tsʰo\#˥}}}}{}
\textcolor{teal}{\mytextsc{nom}} \hspace{4pt} Ton~: \#H.

Prénom masculin.
\textcolor{brown}{\zh{男性名字。}}




\paragraph{\hspace{-0.5cm} {\Large \textcolor{darkblue}{\textbf{\ipa{pʰi˧tsʰo˧-ɖɯ˩ɖʐɯ˩}}}}} \hypertarget{p\string_hi\string_Mts\string_ho\string_M-d`M\string_Bd`z`M\string_B1}{}
\markboth{\textcolor{darkblue}{\textbf{\ipa{pʰi˧tsʰo˧-ɖɯ˩ɖʐɯ˩}}}}{}
\textcolor{teal}{\mytextsc{nom}} \hspace{4pt} Ton~: \mytextsc{L}.

Prénom masculin.
\textcolor{brown}{\zh{男性名字。}}




\paragraph{\hspace{-0.5cm} {\Large \textcolor{darkblue}{\textbf{\ipa{pʰi˧ʈʂæ˧}}}}} \hypertarget{p\string_hi\string_Mt`s`\{\string_M1}{}
\markboth{\textcolor{darkblue}{\textbf{\ipa{pʰi˧ʈʂæ˧}}}}{}
\textcolor{teal}{\mytextsc{verbe}} \hspace{4pt} Ton~: M.

Se draper d'un feutre.
\textcolor{brown}{\zh{披毡(汉语借词)。}}

 Emprunt~: chinois \textcolor{brown}{\zh{披毡}}



\paragraph{\hspace{-0.5cm} {\Large \textcolor{darkblue}{\textbf{\ipa{pʰi˩}}}}} \hypertarget{p\string_hi\string_B1}{}
\markboth{\textcolor{darkblue}{\textbf{\ipa{pʰi˩}}}}{}
\textcolor{teal}{\mytextsc{adjectif}} \hspace{4pt} Ton~: L.

Plat.
\textcolor{brown}{\zh{平(汉语借词)。}}

 Emprunt~: chinois \textcolor{brown}{\zh{平}}



\paragraph{\hspace{-0.5cm} {\Large \textcolor{darkblue}{\textbf{\ipa{pʰi˩hæ˩}}}}} \hypertarget{p\string_hi\string_Bh\{\string_B1}{}
\markboth{\textcolor{darkblue}{\textbf{\ipa{pʰi˩hæ˩}}}}{}
\textcolor{teal}{\mytextsc{nom}} \hspace{4pt} Ton~: L.

Sandale.
\textcolor{brown}{\zh{凉鞋。汉语借词:第一个音节:皮,第二个音节:不明确,同\textcolor{darkblue}{\textbf{\ipa{/tɕæ˧hæ˩/}}}。}}

 Emprunt~: chinois \textcolor{brown}{\zh{皮(secondsyllable:notidentified}})

 \mytextsc{clf}~: \textcolor{darkblue}{\textbf{\ipa{dzi˧}}} paire

\paragraph{\hspace{-0.5cm} {\Large \textcolor{darkblue}{\textbf{\ipa{pʰi˩ko˧}}}}} \hypertarget{p\string_hi\string_Bko\string_M1}{}
\markboth{\textcolor{darkblue}{\textbf{\ipa{pʰi˩ko˧}}}}{}
\textcolor{teal}{\mytextsc{nom}} \hspace{4pt} Ton~: LM.

Pomme.
\textcolor{brown}{\zh{苹果。}}

 Emprunt~: chinois \textcolor{brown}{\zh{苹果}}

 \mytextsc{clf}~: \textcolor{darkblue}{\textbf{\ipa{ɭɯ˧}}} 

\paragraph{\hspace{-0.5cm} {\Large \textcolor{darkblue}{\textbf{\ipa{pʰi˩tʰɑ˩}}}}} \hypertarget{p\string_hi\string_Bt\string_hA\string_B1}{}
\markboth{\textcolor{darkblue}{\textbf{\ipa{pʰi˩tʰɑ˩}}}}{}
\textcolor{teal}{\mytextsc{nom}} \hspace{4pt} Ton~: L.

Amadou.
\textcolor{brown}{\zh{火草。}}


 \mytextsc{clf}~: \textcolor{darkblue}{\textbf{\ipa{po˧}}} 

\paragraph{\hspace{-0.5cm} {\Large \textcolor{darkblue}{\textbf{\ipa{pʰi˧˥}}}}} \hypertarget{p\string_hi\string_M\string_T1}{}
\markboth{\textcolor{darkblue}{\textbf{\ipa{pʰi˧˥}}}}{}
\textcolor{teal}{\mytextsc{verbe}} \hspace{4pt} Ton~: MH.

Vomir.
\textcolor{brown}{\zh{呕吐。}}


\begin{exe}
\sn \textcolor{darkblue}{\textbf{\ipa{le˧-pʰi˧-ze˥}}}
\trans \mytextsc{accomp} \string_ \mytextsc{pfv}
\trans \textit{\textcolor{brown}{\zh{呕吐了}}}
\end{exe}


\paragraph{\hspace{-0.5cm} {\Large \textcolor{darkblue}{\textbf{\ipa{pʰo˥}}}}} \hypertarget{p\string_ho\string_T1}{}
\markboth{\textcolor{darkblue}{\textbf{\ipa{pʰo˥}}}}{}
\textcolor{teal}{\mytextsc{verbe}} \hspace{4pt} Ton~: H.

Se draper de, endosser, mettre sur son dos. Le fait de porter un vêtement sur les épaules sans le boutonner était considéré comme mal élevé: seuls les voleurs gardent la veste ouverte, pour y fourrer subrepticement leur butin.
\textcolor{brown}{\zh{披上(不系扣子)。}}


\begin{exe}
\sn \textcolor{darkblue}{\textbf{\ipa{mɤ˧-pʰo˥}}}
\trans \mytextsc{neg}
\trans \textit{\textcolor{brown}{\zh{不披}}}
\end{exe}
\begin{exe}
\sn \textcolor{darkblue}{\textbf{\ipa{bɑ˩lɑ˩ qɑ˩-pʰo˩˥}}}
\trans endosser un habit, se mettre un habit sur les épaules (sans boutonner)
\trans \textit{\textcolor{brown}{\zh{披上衣服(不系扣子)}}}
\end{exe}


\paragraph{\hspace{-0.5cm} {\Large \textcolor{darkblue}{\textbf{\ipa{pʰo˧\textsubscript{b}}}}}} \hypertarget{p\string_ho\string_Mb1}{}
\markboth{\textcolor{darkblue}{\textbf{\ipa{pʰo˧\textsubscript{b}}}}}{}
\textcolor{teal}{\mytextsc{verbe}} \hspace{4pt} Ton~: M\textsubscript{b}.

Ouvrir (ex.: porte).
\textcolor{brown}{\zh{打开(例如:开门)。}}


\begin{exe}
\sn \textcolor{darkblue}{\textbf{\ipa{gɤ˩-pʰo˧ (-ze˧)}}}
\trans ouvrir
\trans \textit{\textcolor{brown}{\zh{打开}}}
\end{exe}
\begin{exe}
\sn \textcolor{darkblue}{\textbf{\ipa{kʰi˧ pʰo˧}}}
\trans ouvrir la porte
\trans \textit{\textcolor{brown}{\zh{开门}}}
\end{exe}
\begin{exe}
\sn \textcolor{darkblue}{\textbf{\ipa{kʰi˧mi˧ le˧-pʰo˧}}}
\trans ouvrir la porte
\trans \textit{\textcolor{brown}{\zh{开门}}}
\end{exe}
\begin{exe}
\sn \textcolor{darkblue}{\textbf{\ipa{tso˧\textasciitilde{}tso˧ pʰo˧}}}
\trans ouvrir quelque chose
\trans \textit{\textcolor{brown}{\zh{打开东西}}}
\end{exe}


\paragraph{\hspace{-0.5cm} {\Large \textcolor{darkblue}{\textbf{\ipa{pʰo˩\textsubscript{a}}}}}} \hypertarget{p\string_ho\string_Ba1}{}
\markboth{\textcolor{darkblue}{\textbf{\ipa{pʰo˩\textsubscript{a}}}}}{}
\textcolor{teal}{\mytextsc{verbe}} \hspace{4pt} Ton~: L\textsubscript{a}.

S'échapper, s'enfuir; détaler.
\textcolor{brown}{\zh{逃,逃跑,逃掉。}}


\begin{exe}
\sn \textcolor{darkblue}{\textbf{\ipa{le˧-pʰo˩-ze˩}}}
\trans \mytextsc{accomp} \string_ \mytextsc{pfv}
\trans \textit{\textcolor{brown}{\zh{逃跑了}}}
\end{exe}
\begin{exe}
\sn \textcolor{darkblue}{\textbf{\ipa{le˧-pʰo˩-hɯ˩-ze˩!}}}
\trans (Elle/il) s'est enfui(e)!
\trans \textit{\textcolor{brown}{\zh{(他)逃跑了!}}}
\end{exe}


\paragraph{\hspace{-0.5cm} {\Large \textcolor{darkblue}{\textbf{\ipa{pʰo˩lɑ˧˥}}}}} \hypertarget{p\string_ho\string_BlA\string_M\string_T1}{}
\markboth{\textcolor{darkblue}{\textbf{\ipa{pʰo˩lɑ˧˥}}}}{}
\textcolor{teal}{\mytextsc{verbe}} \hspace{4pt} Ton~: LM+MH\#.

Faire la guerre.
\textcolor{brown}{\zh{战争、打仗。}}


\begin{exe}
\sn \textcolor{darkblue}{\textbf{\ipa{mɤ˧-pʰo˩lɑ˩}}}
\trans \mytextsc{neg}
\trans \textit{\textcolor{brown}{\zh{不打仗}}}
\end{exe}
\begin{exe}
\sn \textcolor{darkblue}{\textbf{\ipa{pʰo˩lɑ˧˥ | ɖɯ˧-kʰv̩˧˥}}}
\trans une année de guerre
\trans \textit{\textcolor{brown}{\zh{打仗的一年}}}
\end{exe}


\paragraph{\hspace{-0.5cm} {\Large \textcolor{darkblue}{\textbf{\ipa{pʰo˧˥}}}}} \hypertarget{p\string_ho\string_M\string_T1}{}
\markboth{\textcolor{darkblue}{\textbf{\ipa{pʰo˧˥}}}}{}
\textcolor{teal}{\mytextsc{verbe}} \hspace{4pt} Ton~: MH.

Semer à la volée.
\textcolor{brown}{\zh{撒 (撒种子)、播(种子)。}}


\begin{exe}
\sn \textcolor{darkblue}{\textbf{\ipa{ɻæ˩ pʰo˧˥}}}
\trans semer des graines à la volée
\trans \textit{\textcolor{brown}{\zh{撒种子}}}
\end{exe}


\paragraph{\hspace{-0.5cm} {\Large \textcolor{darkblue}{\textbf{\ipa{pʰo˧˥\textsubscript{a}}}}}} \hypertarget{p\string_ho\string_M\string_Ta1}{}
\markboth{\textcolor{darkblue}{\textbf{\ipa{pʰo˧˥\textsubscript{a}}}}}{}
\textcolor{teal}{\mytextsc{classificateur}} \hspace{4pt} Ton~: MH\textsubscript{a}.

Classificateur des membres d'une paire. Par exemple: une chaussure, pas une paire. Ce classificateur est également employé pour le gros bétail: vaches, buffles….
\textcolor{brown}{\zh{量词:一对中的一只(例如一只鞋),一头大牲畜(牛……)。}}




\paragraph{\hspace{-0.5cm} {\Large \textcolor{darkblue}{\textbf{\ipa{pʰv̩˧}}} \textsubscript{1}}} \hypertarget{p\string_hv\string_=\string_M1}{}
\markboth{\textcolor{darkblue}{\textbf{\ipa{pʰv̩˧}}} \textsubscript{1}}{}
\textcolor{teal}{\mytextsc{nom}} \hspace{4pt} Ton~: M.

Mâle.
\textcolor{brown}{\zh{公的。}}


\begin{exe}
\sn \textcolor{darkblue}{\textbf{\ipa{ʈʂʰɯ˧, | pʰv̩˧ ɲi˩!}}}
\trans C'est un mâle!
\trans \textit{\textcolor{brown}{\zh{这(只动物)是公的!}}}
\end{exe}
\begin{exe}
\sn \textcolor{darkblue}{\textbf{\ipa{ʈʂʰɯ˧, | pʰv̩˧!}}}
\trans C'est un mâle!
\trans \textit{\textcolor{brown}{\zh{这(只动物)是公的!}}}
\end{exe}
 \mytextsc{clf}~: \textcolor{darkblue}{\textbf{\ipa{v̩˧}}} 

\paragraph{\hspace{-0.5cm} {\Large \textcolor{darkblue}{\textbf{\ipa{pʰv̩˧}}} \textsubscript{2}}} \hypertarget{p\string_hv\string_=\string_M2}{}
\markboth{\textcolor{darkblue}{\textbf{\ipa{pʰv̩˧}}} \textsubscript{2}}{}
\textcolor{teal}{\mytextsc{nom}} \hspace{4pt} Ton~: M.

Prix.
\textcolor{brown}{\zh{价格。}}




\paragraph{\hspace{-0.5cm} {\Large \textcolor{darkblue}{\textbf{\ipa{pʰv̩˧˥}}}}} \hypertarget{p\string_hv\string_=\string_M\string_T1}{}
\markboth{\textcolor{darkblue}{\textbf{\ipa{pʰv̩˧˥}}}}{}
\textcolor{teal}{\mytextsc{verbe}} \hspace{4pt} Ton~: MH.

Ôter, retirer (un vêtement).
\textcolor{brown}{\zh{脱(衣服)。}}




\paragraph{\hspace{-0.5cm} {\Large \textcolor{darkblue}{\textbf{\ipa{pʰv̩˧˥}}} \textsubscript{1}}} \hypertarget{p\string_hv\string_=\string_M\string_T1}{}
\markboth{\textcolor{darkblue}{\textbf{\ipa{pʰv̩˧˥}}} \textsubscript{1}}{}
\textcolor{teal}{\mytextsc{verbe}} \hspace{4pt} Ton~: MH.

Faire bouillir, faire cuire à l'eau (œuf, patates…).
\textcolor{brown}{\zh{煮(鸡蛋、洋芋……)。}}


\begin{exe}
\sn \textcolor{darkblue}{\textbf{\ipa{jɤ˩jo˥ F | pʰv̩˧˥! | æ˩ʁv̩˩˥ F | pʰv̩˧˥!}}}
\trans Les pommes de terre, ça se cuit à l'eau! Les œufs, ça se cuit à l'eau!
\trans \textit{\textcolor{brown}{\zh{洋芋,是(可以)煮的!鸡蛋,是(可以)煮的!}}}
\end{exe}
\begin{exe}
\sn \textcolor{darkblue}{\textbf{\ipa{æ˩ʁv̩˩ pʰv̩˥}}}
\trans cuire des œufs à l'eau, faire des oeufs durs
\trans \textit{\textcolor{brown}{\zh{煮鸡蛋}}}
\end{exe}
\begin{exe}
\sn \textcolor{darkblue}{\textbf{\ipa{jɤ˩jo˥ pʰv̩˩}}}
\trans cuire des pommes de terre à l'eau
\trans \textit{\textcolor{brown}{\zh{煮洋芋}}}
\end{exe}
\begin{exe}
\sn \textcolor{darkblue}{\textbf{\ipa{le˧-pʰv̩˧ | le˧-mv̩˩-ze˩!}}}
\trans C'est cuit (à l'eau)! Résultatif: ça a été suffisamment bouilli pour que ce soit maintenant cuit
\trans \textit{\textcolor{brown}{\zh{煮熟了!}}}
\end{exe}


\paragraph{\hspace{-0.5cm} {\Large \textcolor{darkblue}{\textbf{\ipa{pʰv̩˧˥}}} \textsubscript{2}}} \hypertarget{p\string_hv\string_=\string_M\string_T2}{}
\markboth{\textcolor{darkblue}{\textbf{\ipa{pʰv̩˧˥}}} \textsubscript{2}}{}
\textcolor{teal}{\mytextsc{verbe}} \hspace{4pt} Ton~: MH.

Verser; renverser; répandre; jeter.
\textcolor{brown}{\zh{倒(酒……),倒出来。}}


\begin{exe}
\sn \textcolor{darkblue}{\textbf{\ipa{ʐɯ˧ pʰv̩˧˥}}}
\trans verser du vin, servir du vin
\trans \textit{\textcolor{brown}{\zh{倒酒}}}
\end{exe}
\begin{exe}
\sn \textcolor{darkblue}{\textbf{\ipa{dʑɯ˩ pʰv̩˩˥}}}
\trans verser de l'eau
\trans \textit{\textcolor{brown}{\zh{倒水}}}
\end{exe}
\begin{exe}
\sn \textcolor{darkblue}{\textbf{\ipa{mv̩˩tɕo˧ pʰv̩˧˥}}}
\trans renverser, verser à terre, jeter à terre
\trans \textit{\textcolor{brown}{\zh{往外倒}}}
\end{exe}
\begin{exe}
\sn \textcolor{darkblue}{\textbf{\ipa{[F5] ɖæ˩˥ | mv̩˩tɕo˧ pʰv̩˥}}}
\trans jeter des ordures
\trans \textit{\textcolor{brown}{\zh{倒垃圾}}}
\end{exe}


\paragraph{\hspace{-0.5cm} {\Large \textcolor{darkblue}{\textbf{\ipa{pʰv̩˧˥}}} \textsubscript{3}}} \hypertarget{p\string_hv\string_=\string_M\string_T3}{}
\markboth{\textcolor{darkblue}{\textbf{\ipa{pʰv̩˧˥}}} \textsubscript{3}}{}
\textcolor{teal}{\mytextsc{verbe}} \hspace{4pt} Ton~: MH.

Retourner; se retourner (quelqu'un est allongé et se retourne).
\textcolor{brown}{\zh{翻身、翻来翻去。}}


\begin{exe}
\sn \textcolor{darkblue}{\textbf{\ipa{le˧-wo˧ tsɤ˥-pʰv̩˩ |}}}
\trans se retourner
\trans \textit{\textcolor{brown}{\zh{翻身}}}
\end{exe}
\begin{exe}
\sn \textcolor{darkblue}{\textbf{\ipa{ɖɯ˧-tɕo˥ tsɤ˩-pʰv̩˩, | ʈʂʰɯ˧-tɕo˥ tsɤ˩-pʰv̩˩}}}
\trans se retourner par-ci, se retourner par-là
\trans \textit{\textcolor{brown}{\zh{翻来翻去}}}
\end{exe}


\paragraph{\hspace{-0.5cm} {\Large \textcolor{darkblue}{\textbf{\ipa{pʰv̩˩\textsubscript{a}}}}}} \hypertarget{p\string_hv\string_=\string_Ba1}{}
\markboth{\textcolor{darkblue}{\textbf{\ipa{pʰv̩˩\textsubscript{a}}}}}{}
\textcolor{teal}{\mytextsc{adjectif}} \hspace{4pt} Ton~: L\textsubscript{a}.

Blanc (visage, habits, cheveux...).
\textcolor{brown}{\zh{白色(脸、衣服)。}}


\begin{exe}
\sn \textcolor{darkblue}{\textbf{\ipa{pʰv̩˩-hĩ˩˥}}}
\trans \mytextsc{rel}
\trans \textit{\textcolor{brown}{\zh{白的}}}
\end{exe}


\paragraph{\hspace{-0.5cm} {\Large \textcolor{darkblue}{\textbf{\ipa{pʰv̩˩\textsubscript{a}}}}}} \hypertarget{p\string_hv\string_=\string_Ba1}{}
\markboth{\textcolor{darkblue}{\textbf{\ipa{pʰv̩˩\textsubscript{a}}}}}{}
\textcolor{teal}{\mytextsc{classificateur}} \hspace{4pt} Ton~: L\textsubscript{b}.

Classificateur des parcelles de terre, des champs.
\textcolor{brown}{\zh{量词:田地(一块)。}}


\begin{exe}
\sn \textcolor{darkblue}{\textbf{\ipa{lv̩˧ | ɖɯ˧-pʰv̩˩}}}
\trans un champ; une parcelle
\trans \textit{\textcolor{brown}{\zh{一块田}}}
\end{exe}


\paragraph{\hspace{-0.5cm} {\Large \textcolor{darkblue}{\textbf{\ipa{pʰv̩˩\textsubscript{b}}}} \textsubscript{1}}} \hypertarget{p\string_hv\string_=\string_Bb1}{}
\markboth{\textcolor{darkblue}{\textbf{\ipa{pʰv̩˩\textsubscript{b}}}} \textsubscript{1}}{}
\textcolor{teal}{\mytextsc{verbe}} \hspace{4pt} Ton~: L\textsubscript{b}.

S'agiter.
\textcolor{brown}{\zh{摇动、翻滚。}}


\begin{exe}
\sn \textcolor{darkblue}{\textbf{\ipa{bo˩˥ | tʰi˧-pʰv̩˩-dʑo˩}}}
\trans le cochon s'agite dans son box
\trans \textit{\textcolor{brown}{\zh{猪在翻滚}}}
\end{exe}
\begin{exe}
\sn \textcolor{darkblue}{\textbf{\ipa{bo˩-ɳɯ˧ | pʰv̩˧\textasciitilde{}pʰv̩˩}}}
\trans même sens
\trans \textit{\textcolor{brown}{\zh{猪在翻滚}}}
\end{exe}


\paragraph{\hspace{-0.5cm} {\Large \textcolor{darkblue}{\textbf{\ipa{pʰv̩˩\textsubscript{b}}}} \textsubscript{2}}} \hypertarget{p\string_hv\string_=\string_Bb2}{}
\markboth{\textcolor{darkblue}{\textbf{\ipa{pʰv̩˩\textsubscript{b}}}} \textsubscript{2}}{}
\textcolor{teal}{\mytextsc{verbe}} \hspace{4pt} Ton~: L\textsubscript{b}.

Connaître une expansion, se répandre, s'étendre.
\textcolor{brown}{\zh{扩散、发展。}}


\begin{exe}
\sn \textcolor{darkblue}{\textbf{\ipa{zo˧mv̩˥ | tʰi˧-pʰv̩˩}}}
\trans les enfants se répandent, la famille essaime
\trans \textit{\textcolor{brown}{\zh{孩子们扩散(到新的地方)}}}
\end{exe}


\paragraph{\hspace{-0.5cm} {\Large \textcolor{darkblue}{\textbf{\ipa{pʰv̩˧ɖɯ˧˥}}}}} \hypertarget{p\string_hv\string_=\string_Md`M\string_M\string_T1}{}
\markboth{\textcolor{darkblue}{\textbf{\ipa{pʰv̩˧ɖɯ˧˥}}}}{}
\textcolor{teal}{\mytextsc{adjectif}} \hspace{4pt} Ton~: MH\#.
\textit{De:} \textbf{pʰv̩˧ 2 et ɖɯ˩a} 
Coûteux, cher.
\textcolor{brown}{\zh{贵。}}


\begin{exe}
\sn \textcolor{darkblue}{\textbf{\ipa{pʰv̩˧ɖɯ˧ ʝi˥}}}
\trans prêter attention à
\trans \textit{\textcolor{brown}{\zh{关心}}}
\end{exe}


\paragraph{\hspace{-0.5cm} {\Large \textcolor{darkblue}{\textbf{\ipa{pʰv̩˧dʑo˧-hĩ\#˥}}}}} \hypertarget{p\string_hv\string_=\string_Mdz£o\string_M-hi\string_~\#\string_T1}{}
\markboth{\textcolor{darkblue}{\textbf{\ipa{pʰv̩˧dʑo˧-hĩ\#˥}}}}{}
\textcolor{teal}{\mytextsc{nom}} \hspace{4pt} Ton~: \#H.

Personne de Labai, gens de Labai.
\textcolor{brown}{\zh{拉伯的人。}}




\paragraph{\hspace{-0.5cm} {\Large \textcolor{darkblue}{\textbf{\ipa{pʰv̩˧dʑo\#˥}}}}} \hypertarget{p\string_hv\string_=\string_Mdz£o\#\string_T1}{}
\markboth{\textcolor{darkblue}{\textbf{\ipa{pʰv̩˧dʑo\#˥}}}}{}
\textcolor{teal}{\mytextsc{nom}} \hspace{4pt} Ton~: \#H.

Village de Labai.
\textcolor{brown}{\zh{拉伯。}}


\begin{exe}
\sn \textcolor{darkblue}{\textbf{\ipa{pʰv̩˧dʑo˧ dzi˧˥}}}
\trans habiter à Labai
\trans \textit{\textcolor{brown}{\zh{在拉柏住}}}
\end{exe}


\paragraph{\hspace{-0.5cm} {\Large \textcolor{darkblue}{\textbf{\ipa{pʰv̩˧kɤ˧}}}}} \hypertarget{p\string_hv\string_=\string_Mk7\string_M1}{}
\markboth{\textcolor{darkblue}{\textbf{\ipa{pʰv̩˧kɤ˧}}}}{}
\textcolor{teal}{\mytextsc{nom}} \hspace{4pt} Ton~: M.

Couverture.
\textcolor{brown}{\zh{被子。}}


 \mytextsc{clf}~: \textcolor{darkblue}{\textbf{\ipa{ɭɯ˧}}} 

\paragraph{\hspace{-0.5cm} {\Large \textcolor{darkblue}{\textbf{\ipa{pʰv̩˧ɭɯ˧-ʈʰæ˧qʰwɤ˥}}}}} \hypertarget{p\string_hv\string_=\string_Ml\string_RM\string_M-t`\string_h\{\string_Mq\string_hw7\string_T1}{}
\markboth{\textcolor{darkblue}{\textbf{\ipa{pʰv̩˧ɭɯ˧-ʈʰæ˧qʰwɤ˥}}}}{}
\textcolor{teal}{\mytextsc{nom}} \hspace{4pt} Ton~: H\#.

Jupe de laine. (Ce n'est pas un vêtement courant à Yongning.).
\textcolor{brown}{\zh{羊毛裙子。}}


 \mytextsc{clf}~: \textcolor{darkblue}{\textbf{\ipa{ɭɯ˧}}} 

\paragraph{\hspace{-0.5cm} {\Large \textcolor{darkblue}{\textbf{\ipa{pʰv̩˧ɭɯ\#˥}}}}} \hypertarget{p\string_hv\string_=\string_Ml\string_RM\#\string_T1}{}
\markboth{\textcolor{darkblue}{\textbf{\ipa{pʰv̩˧ɭɯ\#˥}}}}{}
\textcolor{teal}{\mytextsc{nom}} \hspace{4pt} Ton~: \#H.

Tissu de laine tibétain.
\textcolor{brown}{\zh{氆氇。}}


 \mytextsc{clf}~: \textcolor{darkblue}{\textbf{\ipa{ɭɯ˧}}} 

\paragraph{\hspace{-0.5cm} {\Large \textcolor{darkblue}{\textbf{\ipa{pʰv̩˧ʂɯ˧}}}}} \hypertarget{p\string_hv\string_=\string_Ms`M\string_M1}{}
\markboth{\textcolor{darkblue}{\textbf{\ipa{pʰv̩˧ʂɯ˧}}}}{}
\textcolor{teal}{\mytextsc{nom}} \hspace{4pt} Ton~: M.

Crème de beauté; s'emploie aussi pour la crème solaire.
\textcolor{brown}{\zh{美容膏。也来指防晒霜。}}


\begin{exe}
\sn \textcolor{darkblue}{\textbf{\ipa{pʰv˧ʂɯ˧ jɤ˧˥}}}
\trans étaler de la crème solaire, mettre de la crème solaire
\trans \textit{\textcolor{brown}{\zh{抹美容膏,抹防晒霜}}}
\end{exe}
\begin{exe}
\sn \textcolor{darkblue}{\textbf{\ipa{pʰv˧ʂɯ˧ lɑ˧˥}}}
\trans étaler de la crème solaire, mettre de la crème solaire
\trans \textit{\textcolor{brown}{\zh{抹美容膏,抹防晒霜}}}
\end{exe}


\paragraph{\hspace{-0.5cm} {\Large \textcolor{darkblue}{\textbf{\ipa{pʰv̩˩-tɕæ˩ɻæ˥}}}}} \hypertarget{p\string_hv\string_=\string_B-ts£\{\string_Br£`\{\string_T1}{}
\markboth{\textcolor{darkblue}{\textbf{\ipa{pʰv̩˩-tɕæ˩ɻæ˥}}}}{}
\textcolor{teal}{\mytextsc{adjectif}} \hspace{4pt} Ton~: L+H\#.
\textit{De:} \textbf{pʰv̩˩a} 
Blanc (visage, habits, cheveux...).
\textcolor{brown}{\zh{很白(脸、衣服、头发)。}}


\begin{exe}
\sn \textcolor{darkblue}{\textbf{\ipa{pʰv̩˩tɕæ˩ɻæ˥-gv̩˩}}}
\trans tout blanc
\trans \textit{\textcolor{brown}{\zh{很白}}}
\end{exe}
\begin{exe}
\sn \textcolor{darkblue}{\textbf{\ipa{pʰv̩˩↑tɕæ˥ɻæ˩-gv̩˩}}}
\trans tout blanc
\trans \textit{\textcolor{brown}{\zh{很白}}}
\end{exe}
\begin{exe}
\sn \textcolor{darkblue}{\textbf{\ipa{pʰæ˧qʰwɤ˩ | pʰv̩˩tɕæ˩ɻæ˥-gv̩˩}}}
\trans le visage est très blanc
\trans \textit{\textcolor{brown}{\zh{脸很白}}}
\end{exe}


\paragraph{\hspace{-0.5cm} {\Large \textcolor{darkblue}{\textbf{\ipa{pʰv̩˥ʈʂʰe˩}}}}} \hypertarget{p\string_hv\string_=\string_Tt`s`\string_he\string_B1}{}
\markboth{\textcolor{darkblue}{\textbf{\ipa{pʰv̩˥ʈʂʰe˩}}}}{}
\textcolor{teal}{\mytextsc{verbe}} \hspace{4pt} Ton~: HL.

Distinguer, voir la différence (par exemple entre diverses espèces de champignons).
\textcolor{brown}{\zh{分开、区分、区别开来。}}


\begin{exe}
\sn \textcolor{darkblue}{\textbf{\ipa{le˧-pʰv̩˥ʈʂʰe˩}}}
\trans distinguer, voir la différence (par exemple entre diverses espèces de champignons)
\trans \textit{\textcolor{brown}{\zh{分开、区分、区别开来}}}
\end{exe}
\begin{exe}
\sn \textcolor{darkblue}{\textbf{\ipa{ɖɯ˧-pʰv̩˥ʈʂʰe˩=ɻ̍˩}}}
\trans \mytextsc{délimitatif} \string_ \mytextsc{inchoatif}
\trans \textit{\textcolor{brown}{\zh{试着区分}}}
\end{exe}
\begin{exe}
\sn \textcolor{darkblue}{\textbf{\ipa{mɤ˧-pʰv̩˥ʈʂʰe˩}}}
\trans ne pas distinguer, ne pas faire de différence, ne pas voir la différence (par exemple entre diverses espèces de champignons)
\trans \textit{\textcolor{brown}{\zh{不分开,不分,不区分}}}
\end{exe}


\paragraph{\hspace{-0.5cm} {\Large \textcolor{darkblue}{\textbf{\ipa{pʰv̩˧tv̩˥}}}}} \hypertarget{p\string_hv\string_=\string_Mtv\string_=\string_T1}{}
\markboth{\textcolor{darkblue}{\textbf{\ipa{pʰv̩˧tv̩˥}}}}{}
\textcolor{teal}{\mytextsc{nom}} \hspace{4pt} Ton~: H\#.

Buffle mâle.
\textcolor{brown}{\zh{公水牛。}}


\begin{exe}
\sn \textcolor{darkblue}{\textbf{\ipa{dʑi˧mi˧-pʰv̩˩tv̩˩}}}
\trans même sens: buffle mâle
\trans \textit{\textcolor{brown}{\zh{同上:公水牛}}}
\end{exe}
\begin{exe}
\sn \textcolor{darkblue}{\textbf{\ipa{dʑi˧mi˧ ʈʂʰɯ˧-pʰo˩ dʑo˩, | pʰv̩˧tv̩˥ ɲi˩!}}}
\trans ce buffle, c'est un mâle!
\trans \textit{\textcolor{brown}{\zh{这头水牛是公的/是公水牛!}}}
\end{exe}
 \mytextsc{clf}~: \textcolor{darkblue}{\textbf{\ipa{pʰo˧˥}}} 

\paragraph{\hspace{-0.5cm} {\Large \textcolor{darkblue}{\textbf{\ipa{pʰv̩˧ʐo˧˥}}}}} \hypertarget{p\string_hv\string_=\string_Mz`o\string_M\string_T1}{}
\markboth{\textcolor{darkblue}{\textbf{\ipa{pʰv̩˧ʐo˧˥}}}}{}
\textcolor{teal}{\mytextsc{adjectif}} \hspace{4pt} Ton~: MH\#.
\textit{De:} \textbf{pʰv̩˧ 2 et ʐo˩a} 
Bon marché.
\textcolor{brown}{\zh{便宜。}}


\textit{Voir~:} \textcolor{darkblue}{\textbf{\ipa{pʰv̩˧2; ʐo˩a2}}} 

\newpage
\section*{\centering- \textcolor{darkblue}{\textbf{\ipa{q}}} -}
\paragraph{\hspace{-0.5cm} {\Large \textcolor{darkblue}{\textbf{\ipa{qɑ˩\textsubscript{c}}}}}} \hypertarget{qA\string_Bc1}{}
\markboth{\textcolor{darkblue}{\textbf{\ipa{qɑ˩\textsubscript{c}}}}}{}
\textcolor{teal}{\mytextsc{classificateur}} \hspace{4pt} Ton~: L\textsubscript{c}.
\ding{202} 
Classificateur : une brassée (de bois coupé pour le feu, d'objets...).
\textcolor{brown}{\zh{量词:抱。}}


\begin{exe}
\sn \textcolor{darkblue}{\textbf{\ipa{ʈʂʰɯ˧-qɑ˥}}}
\trans cette brassée
\trans \textit{\textcolor{brown}{\zh{这一抱}}}
\end{exe}
\ding{203} 
Classificateur des bottes de céréales coupées, faites d'une dizaine de gerbes. Chaque gerbe est nouée à l'aide d'une tige, puis les gerbes sont liées ensemble avec de la ficelle. Une mule peut porter 4 bottes.
\textcolor{brown}{\zh{量词:粮食垛、干草垛。}}


\begin{exe}
\sn \textcolor{darkblue}{\textbf{\ipa{dze˧ɭɯ˧ ɖɯ˧-qɑ˩}}}
\trans une botte de blé (lors de la récolte)
\trans \textit{\textcolor{brown}{\zh{一垛小麦(收割时,将十束绑在一起成一垛)}}}
\end{exe}


\paragraph{\hspace{-0.5cm} {\Large \textcolor{darkblue}{\textbf{\ipa{qɑ˩\textsubscript{a}}}}}} \hypertarget{qA\string_Ba1}{}
\markboth{\textcolor{darkblue}{\textbf{\ipa{qɑ˩\textsubscript{a}}}}}{}
\textcolor{teal}{\mytextsc{verbe}} \hspace{4pt} Ton~: L\textsubscript{a}.
\ding{202} 
Couvrir: par exemple mettre un couvercle, ou couvrir un plat d'une coupelle pour éviter que les mouches ne s'y posent.
\textcolor{brown}{\zh{盖、覆盖。}}


\begin{exe}
\sn \textcolor{darkblue}{\textbf{\ipa{le˧-qɑ˩-ze˩}}}
\trans \mytextsc{accomp} \string_ \mytextsc{pfv}
\trans \textit{\textcolor{brown}{\zh{盖了}}}
\end{exe}
\begin{exe}
\sn \textcolor{darkblue}{\textbf{\ipa{tʰi˧-qɑ˩-ze˩}}}
\trans \mytextsc{dur} \string_ \mytextsc{pfv}
\trans \textit{\textcolor{brown}{\zh{\mytextsc{dur} \string_ \mytextsc{pfv}}}}
\end{exe}
\begin{exe}
\sn \textcolor{darkblue}{\textbf{\ipa{ɖɯ˧-kʰwɤ˥ | tʰi˧-qɑ˥}}}
\trans couvrir (un téléviseur) d'un morceau (de tissu) (pour le préserver de la poussière)
\trans \textit{\textcolor{brown}{\zh{用一块(布料)来盖(电视机,为了防灰)}}}
\end{exe}
\begin{exe}
\sn \textcolor{darkblue}{\textbf{\ipa{hæ̃˧qʰv̩˥ | tʰi˧-qɑ˩!}}}
\trans le soir, on recouvre (le téléviseur d'un tissu)! / (on) (le) met le soir (sur le téléviseur)/ (on) (en) recouvre (le téléviseur) le soir!
\trans \textit{\textcolor{brown}{\zh{晚上,(要)盖上! / 我们晚上盖电视机(为了防灰)!}}}
\end{exe}
\begin{exe}
\sn \textcolor{darkblue}{\textbf{\ipa{tso˧\textasciitilde{}tso˧ qɑ˥}}}
\trans recouvrir quelque chose
\trans \textit{\textcolor{brown}{\zh{覆盖东西}}}
\end{exe}
\ding{203} 
Voiler, bloquer (la lumière), cacher au regard.
\textcolor{brown}{\zh{遮(云遮月)、遮挡。}}




\paragraph{\hspace{-0.5cm} {\Large \textcolor{darkblue}{\textbf{\ipa{‑qɑ˧˥}}}}} \hypertarget{‑qA\string_M\string_T1}{}
\markboth{\textcolor{darkblue}{\textbf{\ipa{‑qɑ˧˥}}}}{}
\textcolor{teal}{\mytextsc{postposition}} \hspace{4pt} Ton~: MH.

À (datif); avec (comitatif).
\textcolor{brown}{\zh{给、对。}}




\paragraph{\hspace{-0.5cm} {\Large \textcolor{darkblue}{\textbf{\ipa{qɑ˧˥}}}}} \hypertarget{qA\string_M\string_T1}{}
\markboth{\textcolor{darkblue}{\textbf{\ipa{qɑ˧˥}}}}{}
\textcolor{teal}{\mytextsc{verbe}} \hspace{4pt} Ton~: MH.

Aider.
\textcolor{brown}{\zh{帮助。}}


\begin{exe}
\sn \textcolor{darkblue}{\textbf{\ipa{tʰi˧-qɑ˧˥}}}
\trans \mytextsc{dur}
\trans \textit{\textcolor{brown}{\zh{\mytextsc{dur}}}}
\end{exe}
\begin{exe}
\sn \textcolor{darkblue}{\textbf{\ipa{qɑ˩\textasciitilde{}qɑ˧˥}}}
\trans \mytextsc{red}
\trans \textit{\textcolor{brown}{\zh{\mytextsc{重叠:帮帮忙}}}}
\end{exe}
\begin{exe}
\sn \textcolor{darkblue}{\textbf{\ipa{hĩ˧ qɑ˩\textasciitilde{}qɑ˩}}}
\trans aider des gens; aller travailler chez autrui (par ex. pendant les récoltes)
\trans \textit{\textcolor{brown}{\zh{帮人,到别人家去工作(例如收庄稼的时候)}}}
\end{exe}
\begin{exe}
\sn \textcolor{darkblue}{\textbf{\ipa{njɤ˧ no˧ qɑ˧\textasciitilde{}qɑ˥}}}
\trans je t'aide
\trans \textit{\textcolor{brown}{\zh{我帮你}}}
\end{exe}


\paragraph{\hspace{-0.5cm} {\Large \textcolor{darkblue}{\textbf{\ipa{qæ˥}}} \textsubscript{1}}} \hypertarget{q\{\string_T1}{}
\markboth{\textcolor{darkblue}{\textbf{\ipa{qæ˥}}} \textsubscript{1}}{}
\textcolor{teal}{\mytextsc{verbe}} \hspace{4pt} Ton~: H.

Déplacer, transporter (ex.: transporter de la terre, après avoir creusé).
\textcolor{brown}{\zh{搬。}}Dialecte chinois local~: \zh{盘。}


\begin{exe}
\sn \textcolor{darkblue}{\textbf{\ipa{le˧-qæ˥}}}
\trans \mytextsc{accomp}
\trans \textit{\textcolor{brown}{\zh{\mytextsc{accomp}}}}
\end{exe}


\paragraph{\hspace{-0.5cm} {\Large \textcolor{darkblue}{\textbf{\ipa{qæ˥}}} \textsubscript{2}}} \hypertarget{q\{\string_T2}{}
\markboth{\textcolor{darkblue}{\textbf{\ipa{qæ˥}}} \textsubscript{2}}{}
\textcolor{teal}{\mytextsc{verbe}} \hspace{4pt} Ton~: H.

Changer.
\textcolor{brown}{\zh{换。}}


\begin{exe}
\sn \textcolor{darkblue}{\textbf{\ipa{le˧-qæ˥-ze˩}}}
\trans \mytextsc{accomp} \string_ \mytextsc{pfv}
\trans \textit{\textcolor{brown}{\zh{换了}}}
\end{exe}
\begin{exe}
\sn \textcolor{darkblue}{\textbf{\ipa{dʑi˧hṽ˧ qæ˧}}}
\trans changer de vêtements
\trans \textit{\textcolor{brown}{\zh{换衣服}}}
\end{exe}
\begin{exe}
\sn \textcolor{darkblue}{\textbf{\ipa{bɑ˩lɑ˩˥ | tʰi˧-qæ˥}}}
\trans changer de vêtements
\trans \textit{\textcolor{brown}{\zh{换衣服}}}
\end{exe}
\begin{exe}
\sn \textcolor{darkblue}{\textbf{\ipa{qæ˧\textasciitilde{}qæ˧}}}
\trans \mytextsc{red}: échanger (un objet contre un autre)
\trans \textit{\textcolor{brown}{\zh{\mytextsc{重叠:交换}}}}
\end{exe}
\begin{exe}
\sn \textcolor{darkblue}{\textbf{\ipa{qæ˧\textasciitilde{}qæ˧-ɻ̍˥}}}
\trans \mytextsc{red} \mytextsc{inchoatif}
\trans \textit{\textcolor{brown}{\zh{\mytextsc{red} \mytextsc{inceptive}}}}
\end{exe}
\begin{exe}
\sn \textcolor{darkblue}{\textbf{\ipa{qæ˧\textasciitilde{}qæ˧-ɻ̍˧-ze˥}}}
\trans \mytextsc{red} \mytextsc{inchoatif} \mytextsc{pfv}
\trans \textit{\textcolor{brown}{\zh{\mytextsc{red} \mytextsc{inceptive} \mytextsc{pfv}}}}
\end{exe}
\begin{exe}
\sn \textcolor{darkblue}{\textbf{\ipa{tso˧\textasciitilde{}tso˧ qæ˧\textasciitilde{}qæ˧}}}
\trans échanger des choses
\trans \textit{\textcolor{brown}{\zh{交换东西}}}
\end{exe}
\begin{exe}
\sn \textcolor{darkblue}{\textbf{\ipa{le˧-qæ˧\textasciitilde{}qæ˧(-ze˩)}}}
\trans \mytextsc{accomp} \mytextsc{red} (\mytextsc{pfv})
\trans \textit{\textcolor{brown}{\zh{\mytextsc{accomp} \mytextsc{red} (\mytextsc{pfv})}}}
\end{exe}


\paragraph{\hspace{-0.5cm} {\Large \textcolor{darkblue}{\textbf{\ipa{qæ˥}}} \textsubscript{3}}} \hypertarget{q\{\string_T3}{}
\markboth{\textcolor{darkblue}{\textbf{\ipa{qæ˥}}} \textsubscript{3}}{}
\textcolor{teal}{\mytextsc{verbe}} \hspace{4pt} Ton~: H.

Sculpter.
\textcolor{brown}{\zh{雕。}}


\begin{exe}
\sn \textcolor{darkblue}{\textbf{\ipa{le˧-qæ˥-ze˩}}}
\trans \mytextsc{accomp} \string_ \mytextsc{pfv}
\trans \textit{\textcolor{brown}{\zh{雕了}}}
\end{exe}
\begin{exe}
\sn \textcolor{darkblue}{\textbf{\ipa{bæ˩bæ˩ qæ˥}}}
\trans sculpter une fleur
\trans \textit{\textcolor{brown}{\zh{雕花}}}
\end{exe}


\paragraph{\hspace{-0.5cm} {\Large \textcolor{darkblue}{\textbf{\ipa{qæ˧do˧}}}}} \hypertarget{q\{\string_Mdo\string_M1}{}
\markboth{\textcolor{darkblue}{\textbf{\ipa{qæ˧do˧}}}}{}
\textcolor{teal}{\mytextsc{nom}} \hspace{4pt} Ton~: M.

Bois de charpente, tronc coupé.
\textcolor{brown}{\zh{木材、木料。}}


\begin{exe}
\sn \textcolor{darkblue}{\textbf{\ipa{ʑi˧mi˧-qæ˩do˩}}}
\trans bois de charpente utilisé pour le bâtiment principal
\trans \textit{\textcolor{brown}{\zh{建主房的木材}}}
\end{exe}
\begin{exe}
\sn \textcolor{darkblue}{\textbf{\ipa{ʑi˧qʰwɤ˧-qæ˧do\#˥}}}
\trans bois de charpente, bois pour la construction d'un bâtiment
\trans \textit{\textcolor{brown}{\zh{建房子的木材}}}
\end{exe}
 \mytextsc{clf}~: \textcolor{darkblue}{\textbf{\ipa{kɤ˧˥}}} \textit{Syn~:} \hyperlink{q\{\string_Mr£`̍\string_M1}{\textcolor{darkblue}{\textbf{\ipa{qæ˧ɻ̍˧}}}}. 

\paragraph{\hspace{-0.5cm} {\Large \textcolor{darkblue}{\textbf{\ipa{qæ˧dzɯ˩}}}}} \hypertarget{q\{\string_MdzM\string_B1}{}
\markboth{\textcolor{darkblue}{\textbf{\ipa{qæ˧dzɯ˩}}}}{}
\textcolor{teal}{\mytextsc{nom}} \hspace{4pt} Ton~: L\#.

Nom de clan/famille étendue. Deux familles portent ce nom à Yongning.
\textcolor{brown}{\zh{一个姓。这个姓,永宁有两家。}}


\begin{exe}
\sn \textcolor{darkblue}{\textbf{\ipa{qæ˧dzɯ˩-ɻ̍˩}}}
\trans le clan \textcolor{darkblue}{\textbf{\ipa{/qæ˧dzɯ˩/}}}, la famille \textcolor{darkblue}{\textbf{\ipa{/qæ˧dzɯ˩/}}}
\trans \textit{\textcolor{brown}{\zh{\textcolor{darkblue}{\textbf{\ipa{/qæ˧dzɯ˩/}}}家族}}}
\end{exe}
\begin{exe}
\sn \textcolor{darkblue}{\textbf{\ipa{qæ˧dzɯ˩ | -tsʰɯ˧ɻ̍˧}}}
\trans nom d'une personne, comportant un nom de famille (\textcolor{darkblue}{\textbf{\ipa{/qæ˧dzɯ˩/}}}) et un prénom (\textcolor{darkblue}{\textbf{\ipa{/tsʰɯ˧ɻ\#˥/}}})
\trans \textit{\textcolor{brown}{\zh{一个人的名字:姓为\textcolor{darkblue}{\textbf{\ipa{/qæ˧dzɯ˩/}}},名为\textcolor{darkblue}{\textbf{\ipa{/tsʰɯ˧ɻ\#˥/}}}}}}
\end{exe}


\paragraph{\hspace{-0.5cm} {\Large \textcolor{darkblue}{\textbf{\ipa{qæ˧ɻ̍˧}}}}} \hypertarget{q\{\string_Mr£`̍\string_M1}{}
\markboth{\textcolor{darkblue}{\textbf{\ipa{qæ˧ɻ̍˧}}}}{}
\textcolor{teal}{\mytextsc{nom}} \hspace{4pt} Ton~: M.

Bois de charpente, tronc coupé.
\textcolor{brown}{\zh{木材、木料。}}


 \mytextsc{clf}~: \textcolor{darkblue}{\textbf{\ipa{kɤ˧˥}}} \textit{Syn~:} \hyperlink{q\{\string_Mdo\string_M1}{\textcolor{darkblue}{\textbf{\ipa{qæ˧do˧}}}}. 

\paragraph{\hspace{-0.5cm} {\Large \textcolor{darkblue}{\textbf{\ipa{qæ˩\textsubscript{a}}}}}} \hypertarget{q\{\string_Ba1}{}
\markboth{\textcolor{darkblue}{\textbf{\ipa{qæ˩\textsubscript{a}}}}}{}
\textcolor{teal}{\mytextsc{verbe}} \hspace{4pt} Ton~: L\textsubscript{a}.

Cajoler un enfant.
\textcolor{brown}{\zh{哄(孩子)。}}


\begin{exe}
\sn \textcolor{darkblue}{\textbf{\ipa{zo˧ qæ˥}}}
\trans cajoler un enfant
\trans \textit{\textcolor{brown}{\zh{哄孩子}}}
\end{exe}
\begin{exe}
\sn \textcolor{darkblue}{\textbf{\ipa{le˧-qæ˧\textasciitilde{}qæ˥ | le˧-ʑi˧-kʰɯ˥}}}
\trans endormir (un enfant) en le cajolant
\trans \textit{\textcolor{brown}{\zh{哄睡着}}}
\end{exe}


\paragraph{\hspace{-0.5cm} {\Large \textcolor{darkblue}{\textbf{\ipa{qæ˩\textsubscript{b}}}}}} \hypertarget{q\{\string_Bb1}{}
\markboth{\textcolor{darkblue}{\textbf{\ipa{qæ˩\textsubscript{b}}}}}{}
\textcolor{teal}{\mytextsc{verbe}} \hspace{4pt} Ton~: L\textsubscript{b}.

Tromper.
\textcolor{brown}{\zh{欺骗。}}


\begin{exe}
\sn \textcolor{darkblue}{\textbf{\ipa{hĩ˧ qæ˥-kv̩˩}}}
\trans rusé, qui sait tromper son monde
\trans \textit{\textcolor{brown}{\zh{狡猾、很能骗人的}}}
\end{exe}
\begin{exe}
\sn \textcolor{darkblue}{\textbf{\ipa{hĩ˧ qæ˥ | ʐwæ˩˥}}}
\trans qui trompe magistralement son monde
\trans \textit{\textcolor{brown}{\zh{狡猾、很能骗人的}}}
\end{exe}
\begin{exe}
\sn \textcolor{darkblue}{\textbf{\ipa{hĩ˧ qæ˥ mɤ˩-ɖo˩!}}}
\trans il ne faut pas tromper (autrui)! (précepte inculqué à la locutrice par sa grand-mère)
\trans \textit{\textcolor{brown}{\zh{不要骗人!(这个信条,是发音合作人的祖母教的)}}}
\end{exe}
\begin{exe}
\sn \textcolor{darkblue}{\textbf{\ipa{qæ˩-mɤ˩-ɖo˩˥!}}}
\trans il ne faut pas tromper (autrui)! (précepte inculqué à la locutrice par sa grand-mère)
\trans \textit{\textcolor{brown}{\zh{不要骗人!(这个信条,是发音合作人的祖母教的)}}}
\end{exe}
\begin{exe}
\sn \textcolor{darkblue}{\textbf{\ipa{mɤ˧-qæ˩}}}
\trans \mytextsc{neg}
\trans \textit{\textcolor{brown}{\zh{不骗}}}
\end{exe}
\begin{exe}
\sn \textcolor{darkblue}{\textbf{\ipa{hĩ˧ qæ˥-tso˩\textasciitilde{}tso˩!}}}
\trans C'est de la camelote! / C'est un truc d'arnaqueurs! (au sujet d'une bobine de fil de mauvaise qualité, achetée dans le commerce)
\trans \textit{\textcolor{brown}{\zh{骗人的东西!(关于买来的一团线,质量不好)}}}
\end{exe}


\paragraph{\hspace{-0.5cm} {\Large \textcolor{darkblue}{\textbf{\ipa{qæ˩di˩}}}}} \hypertarget{q\{\string_Bdi\string_B1}{}
\markboth{\textcolor{darkblue}{\textbf{\ipa{qæ˩di˩}}}}{}
\textcolor{teal}{\mytextsc{verbe}} \hspace{4pt} Ton~: L.

Donner une chiquenaude.
\textcolor{brown}{\zh{弹(弹脸)。}}




\paragraph{\hspace{-0.5cm} {\Large \textcolor{darkblue}{\textbf{\ipa{qæ˧˥}}} \textsubscript{1}}} \hypertarget{q\{\string_M\string_T1}{}
\markboth{\textcolor{darkblue}{\textbf{\ipa{qæ˧˥}}} \textsubscript{1}}{}
\textcolor{teal}{\mytextsc{verbe}} \hspace{4pt} Ton~: MH.

Brûler quelque chose; par exemple: incinérer un corps.
\textcolor{brown}{\zh{燃烧,如:烧尸体(进行火葬时)。}}


\begin{exe}
\sn \textcolor{darkblue}{\textbf{\ipa{mv̩˧ qæ˩-ze˩}}}
\trans le feu est parti, ça brûle, ça flambe; un incendie est parti
\trans \textit{\textcolor{brown}{\zh{火烧着了 / 着火了}}}
\end{exe}
\begin{exe}
\sn \textcolor{darkblue}{\textbf{\ipa{mv̩˧ le˧-qæ˧˥ / mv̩˧ le˧-qæ˧-ze˥}}}
\trans ça brûle; il y a un incendie
\trans \textit{\textcolor{brown}{\zh{火在烧 / 着火了}}}
\end{exe}
\begin{exe}
\sn \textcolor{darkblue}{\textbf{\ipa{mv̩˧ qæ˥-ɻ̍˩}}}
\trans ça brûle; il y a un incendie
\trans \textit{\textcolor{brown}{\zh{火在烧 / 火烧着了}}}
\end{exe}
\begin{exe}
\sn \textcolor{darkblue}{\textbf{\ipa{mv̩˧ qæ˥-ɻ̍˩ kʰɯ˩}}}
\trans lancer un incendie, déclencher un incendie, mettre le feu (acte criminel)
\trans \textit{\textcolor{brown}{\zh{(有人)放火}}}
\end{exe}
\begin{exe}
\sn \textcolor{darkblue}{\textbf{\ipa{mv̩˧qæ˥-ɻ̍˩-hɯ˩}}}
\trans un incendie est parti
\trans \textit{\textcolor{brown}{\zh{(有人)放火了!}}}
\end{exe}


\paragraph{\hspace{-0.5cm} {\Large \textcolor{darkblue}{\textbf{\ipa{qæ˧˥}}} \textsubscript{2}}} \hypertarget{q\{\string_M\string_T2}{}
\markboth{\textcolor{darkblue}{\textbf{\ipa{qæ˧˥}}} \textsubscript{2}}{}
\textcolor{teal}{\mytextsc{verbe}} \hspace{4pt} Ton~: MH.

Souffrir, avoir mal.
\textcolor{brown}{\zh{疼。}}


\begin{exe}
\sn \textcolor{darkblue}{\textbf{\ipa{bi˧mi˧ qæ˧˥}}}
\trans avoir mal au ventre
\trans \textit{\textcolor{brown}{\zh{肚子疼}}}
\end{exe}
\begin{exe}
\sn \textcolor{darkblue}{\textbf{\ipa{ɬo˧kʰv̩˧ qæ˧˥}}}
\trans avoir mal à la hanche
\trans \textit{\textcolor{brown}{\zh{腰疼}}}
\end{exe}
\begin{exe}
\sn \textcolor{darkblue}{\textbf{\ipa{ʁo˧qʰwɤ˩ qæ˩}}}
\trans avoir mal à la tête
\trans \textit{\textcolor{brown}{\zh{头疼}}}
\end{exe}


\paragraph{\hspace{-0.5cm} {\Large \textcolor{darkblue}{\textbf{\ipa{qæ˩˥}}} \textsubscript{1}}} \hypertarget{q\{\string_B\string_T1}{}
\markboth{\textcolor{darkblue}{\textbf{\ipa{qæ˩˥}}} \textsubscript{1}}{}
\textcolor{teal}{\mytextsc{nom}} \hspace{4pt} Ton~: LH.

Huile (terme générique; huile de friture).
\textcolor{brown}{\zh{油,食用油。}}




\paragraph{\hspace{-0.5cm} {\Large \textcolor{darkblue}{\textbf{\ipa{qæ˩˥}}} \textsubscript{2}}} \hypertarget{q\{\string_B\string_T2}{}
\markboth{\textcolor{darkblue}{\textbf{\ipa{qæ˩˥}}} \textsubscript{2}}{}
\textcolor{teal}{\mytextsc{nom}} \hspace{4pt} Ton~: LH.

Colle.
\textcolor{brown}{\zh{胶。}}


 \mytextsc{clf}~: \textcolor{darkblue}{\textbf{\ipa{kʰwɤ˥}}} 

\paragraph{\hspace{-0.5cm} {\Large \textcolor{darkblue}{\textbf{\ipa{qi˧qi˧}}}}} \hypertarget{qi\string_Mqi\string_M1}{}
\markboth{\textcolor{darkblue}{\textbf{\ipa{qi˧qi˧}}}}{}
\textcolor{teal}{\mytextsc{adverbe}} \hspace{4pt} Ton~: M.

À l'origine.
\textcolor{brown}{\zh{原来、一开始。}}




\paragraph{\hspace{-0.5cm} {\Large \textcolor{darkblue}{\textbf{\ipa{qo˥}}} \textsubscript{1}}} \hypertarget{qo\string_T1}{}
\markboth{\textcolor{darkblue}{\textbf{\ipa{qo˥}}} \textsubscript{1}}{}
\textcolor{teal}{\mytextsc{verbe}} \hspace{4pt} Ton~: H.

S'agenouiller (les mains au sol).
\textcolor{brown}{\zh{跪下。}}




\paragraph{\hspace{-0.5cm} {\Large \textcolor{darkblue}{\textbf{\ipa{qo˥}}} \textsubscript{2}}} \hypertarget{qo\string_T2}{}
\markboth{\textcolor{darkblue}{\textbf{\ipa{qo˥}}} \textsubscript{2}}{}
\textcolor{teal}{\mytextsc{verbe}} \hspace{4pt} Ton~: H.

Aimer d'amour.
\textcolor{brown}{\zh{爱,关心。}}


\begin{exe}
\sn \textcolor{darkblue}{\textbf{\ipa{mɤ˧-qo˧}}}
\trans \mytextsc{neg}
\trans \textit{\textcolor{brown}{\zh{不爱}}}
\end{exe}
\begin{exe}
\sn \textcolor{darkblue}{\textbf{\ipa{zo˧mv̩˥zo˩ qo˩}}}
\trans aimer (ses) enfants
\trans \textit{\textcolor{brown}{\zh{爱孩子}}}
\end{exe}
\begin{exe}
\sn \textcolor{darkblue}{\textbf{\ipa{õ˧-hĩ˥ qo˩}}}
\trans aimer sa famille
\trans \textit{\textcolor{brown}{\zh{爱自己家人}}}
\end{exe}


\paragraph{\hspace{-0.5cm} {\Large \textcolor{darkblue}{\textbf{\ipa{-qo˧}}}}} \hypertarget{-qo\string_M1}{}
\markboth{\textcolor{darkblue}{\textbf{\ipa{-qo˧}}}}{}
\textcolor{teal}{\mytextsc{postposition}} \hspace{4pt} Ton~: M.

Dans.
\textcolor{brown}{\zh{里。}}


\textit{Voir~:} \hyperlink{-qo\string_Mlo\string_B1}{\textcolor{darkblue}{\textbf{\ipa{-qo˧lo˩}}}} 

\paragraph{\hspace{-0.5cm} {\Large \textcolor{darkblue}{\textbf{\ipa{qo˧lo˧ʂv̩˥}}}}} \hypertarget{qo\string_Mlo\string_Ms`v\string_=\string_T1}{}
\markboth{\textcolor{darkblue}{\textbf{\ipa{qo˧lo˧ʂv̩˥}}}}{}
\textcolor{teal}{\mytextsc{adjectif}} \hspace{4pt} Ton~: .

Obéissant, sage (enfant).
\textcolor{brown}{\zh{乖、听话。}}




\paragraph{\hspace{-0.5cm} {\Large \textcolor{darkblue}{\textbf{\ipa{-qo˧lo˩}}}}} \hypertarget{-qo\string_Mlo\string_B1}{}
\markboth{\textcolor{darkblue}{\textbf{\ipa{-qo˧lo˩}}}}{}
\textcolor{teal}{\mytextsc{postposition}} \hspace{4pt} Ton~: L\#.

Dans.
\textcolor{brown}{\zh{里面。}}


\begin{exe}
\sn \textcolor{darkblue}{\textbf{\ipa{ʈʂʰɯ˧ | ɑ˩ʁo˧-qo˧lo˩ dʑo˩}}}
\trans Il/elle est à la maison/dans la maison.
\trans \textit{\textcolor{brown}{\zh{他在家里。}}}
\end{exe}
\textit{Voir~:} \hyperlink{qo\string_Mlo\string_B1}{\textcolor{darkblue}{\textbf{\ipa{qo˧lo˩}}}} 

\paragraph{\hspace{-0.5cm} {\Large \textcolor{darkblue}{\textbf{\ipa{qo˧lo˩}}}}} \hypertarget{qo\string_Mlo\string_B1}{}
\markboth{\textcolor{darkblue}{\textbf{\ipa{qo˧lo˩}}}}{}
\textcolor{teal}{\mytextsc{adverbe}} \hspace{4pt} Ton~: L\#.

Dedans, à l'intérieur de, dans.
\textcolor{brown}{\zh{里面。}}


\textit{Voir~:} \hyperlink{-qo\string_Mlo\string_B1}{\textcolor{darkblue}{\textbf{\ipa{-qo˧lo˩}}}} 

\paragraph{\hspace{-0.5cm} {\Large \textcolor{darkblue}{\textbf{\ipa{qo˧pv̩˩}}}}} \hypertarget{qo\string_Mpv\string_=\string_B1}{}
\markboth{\textcolor{darkblue}{\textbf{\ipa{qo˧pv̩˩}}}}{}
\textcolor{teal}{\mytextsc{nom}} \hspace{4pt} Ton~: L\#.

Coucou.
\textcolor{brown}{\zh{布谷鸟。}}


\begin{exe}
\sn \textcolor{darkblue}{\textbf{\ipa{qo˧pv̩˩-ɻwæ˩ | ɖɯ˧-ɲi˥}}}
\trans Le Jour des Ancêtres, au 1er jour du 5e mois. Littéralement: ``le jour où chante le coucou".
\trans \textit{\textcolor{brown}{\zh{清明节。直译:“布谷鸟叫的那天”}}}
\end{exe}
 \mytextsc{clf}~: \textcolor{darkblue}{\textbf{\ipa{mi˩}}} 

\paragraph{\hspace{-0.5cm} {\Large \textcolor{darkblue}{\textbf{\ipa{qo˧pv̩˧}}}}} \hypertarget{qo\string_Mpv\string_=\string_M1}{}
\markboth{\textcolor{darkblue}{\textbf{\ipa{qo˧pv̩˧}}}}{}
\textcolor{teal}{\mytextsc{adjectif}} \hspace{4pt} Ton~: .

Assoiffé, ayant soif.
\textcolor{brown}{\zh{渴。}}


\begin{exe}
\sn \textcolor{darkblue}{\textbf{\ipa{qo˧pv̩˧-ze˧}}}
\trans \mytextsc{pfv}
\trans \textit{\textcolor{brown}{\zh{渴了}}}
\end{exe}
\begin{exe}
\sn \textcolor{darkblue}{\textbf{\ipa{qo˧pv̩˧ mɤ˧-tʰɑ˧-ze˥}}}
\trans avoir soif comme c'est pas possible, avoir terriblement soif
\trans \textit{\textcolor{brown}{\zh{渴得不行}}}
\end{exe}
\textit{Voir~:} \textcolor{darkblue}{\textbf{\ipa{qwɤ˩pi˩; pv̩˧2}}} 

\paragraph{\hspace{-0.5cm} {\Large \textcolor{darkblue}{\textbf{\ipa{qo˧pv̩˩-ʐwæ˩ɖʐæ˩}}}}} \hypertarget{qo\string_Mpv\string_=\string_B-z`w\{\string_Bd`z`\{\string_B1}{}
\markboth{\textcolor{darkblue}{\textbf{\ipa{qo˧pv̩˩-ʐwæ˩ɖʐæ˩}}}}{}
\textcolor{teal}{\mytextsc{nom}} \hspace{4pt} Ton~: LM-.

Geai, \textit{Garrulus glandarius sinensis}.
\textcolor{brown}{\zh{松鸦。}}




\paragraph{\hspace{-0.5cm} {\Large \textcolor{darkblue}{\textbf{\ipa{qo˧tv̩˩}}}}} \hypertarget{qo\string_Mtv\string_=\string_B1}{}
\markboth{\textcolor{darkblue}{\textbf{\ipa{qo˧tv̩˩}}}}{}
\textcolor{teal}{\mytextsc{nom}} \hspace{4pt} Ton~: L\#/LM.

Noyau (aussi pour: graines de tournesol; et pour: bobines de fil).
\textcolor{brown}{\zh{果核。}}


\begin{exe}
\sn \textcolor{darkblue}{\textbf{\ipa{dʑi˧ʁo˩-qo˩tv̩˩}}}
\trans noyau de pêche
\trans \textit{\textcolor{brown}{\zh{桃子果核}}}
\end{exe}
 \mytextsc{clf}~: \textcolor{darkblue}{\textbf{\ipa{ɭɯ˧}}} 

\paragraph{\hspace{-0.5cm} {\Large \textcolor{darkblue}{\textbf{\ipa{qo˩\textsubscript{a}}}}}} \hypertarget{qo\string_Ba1}{}
\markboth{\textcolor{darkblue}{\textbf{\ipa{qo˩\textsubscript{a}}}}}{}
\textcolor{teal}{\mytextsc{verbe}} \hspace{4pt} Ton~: L\textsubscript{a}.

Garder, serrer, ranger (de la nourriture dans un récipient à l'abri des mouches).
\textcolor{brown}{\zh{放、储存。}}




\paragraph{\hspace{-0.5cm} {\Large \textcolor{darkblue}{\textbf{\ipa{qo˩ho˧˥}}}}} \hypertarget{qo\string_Bho\string_M\string_T1}{}
\markboth{\textcolor{darkblue}{\textbf{\ipa{qo˩ho˧˥}}}}{}
\textcolor{teal}{\mytextsc{nom}} \hspace{4pt} Ton~: LM+MH\#.

Boîte en vannerie ronde, dans laquelle on place les cadeaux qu’on vient offrir; est formée de deux parties qui s’emboîtent; on la porte lorsqu'on se rend chez quelqu'un dans le cadre d'un événement social important. Cf récit F4. Une photo de cet objet est présente dans le rapport d'enquête de terrain publié en 1986 en 3 volumes (永宁纳西族……调查).
\textcolor{brown}{\zh{礼盒。}}


 \mytextsc{clf}~: \textcolor{darkblue}{\textbf{\ipa{ɭɯ˧}}} 

\paragraph{\hspace{-0.5cm} {\Large \textcolor{darkblue}{\textbf{\ipa{qo˩qɑ˩}}}}} \hypertarget{qo\string_BqA\string_B1}{}
\markboth{\textcolor{darkblue}{\textbf{\ipa{qo˩qɑ˩}}}}{}
\textcolor{teal}{\mytextsc{nom}} \hspace{4pt} Ton~: L.

Col (de montagne).
\textcolor{brown}{\zh{垭口。}}


 \mytextsc{clf}~: \textcolor{darkblue}{\textbf{\ipa{ɭɯ˧}}} 

\paragraph{\hspace{-0.5cm} {\Large \textcolor{darkblue}{\textbf{\ipa{qo˩tv̩˩-lv̩˥}}}}} \hypertarget{qo\string_Btv\string_=\string_B-lv\string_=\string_T1}{}
\markboth{\textcolor{darkblue}{\textbf{\ipa{qo˩tv̩˩-lv̩˥}}}}{}
\textcolor{teal}{\mytextsc{nom}} \hspace{4pt} Ton~: L+H\#.

Boule.
\textcolor{brown}{\zh{团。}}


\begin{exe}
\sn \textcolor{darkblue}{\textbf{\ipa{li˩-qo˩tv̩˥-lv̩˩}}}
\trans thé comprimé en boule
\trans \textit{\textcolor{brown}{\zh{沱茶}}}
\end{exe}
\begin{exe}
\sn \textcolor{darkblue}{\textbf{\ipa{li˩-qo˩tv̩˥-lv̩˩ | ɖɯ˧-qʰwɤ˧ tɕɤ˥}}}
\trans préparer un bol de thé avec du thé comprimé en boule
\trans \textit{\textcolor{brown}{\zh{煮一碗沱茶}}}
\end{exe}
 \mytextsc{clf}~: \textcolor{darkblue}{\textbf{\ipa{ɭɯ˧}}} 

\paragraph{\hspace{-0.5cm} {\Large \textcolor{darkblue}{\textbf{\ipa{qv̩˩˥}}}}} \hypertarget{qv\string_=\string_B\string_T1}{}
\markboth{\textcolor{darkblue}{\textbf{\ipa{qv̩˩˥}}}}{}
\textcolor{teal}{\mytextsc{nom}} \hspace{4pt} Ton~: LH.

Poignée, manche (d'une valise, d'une bouteille thermos, d'une louche...).
\textcolor{brown}{\zh{把手。}}


 \mytextsc{clf}~: \textcolor{darkblue}{\textbf{\ipa{kʰwɤ˥}}} 

\paragraph{\hspace{-0.5cm} {\Large \textcolor{darkblue}{\textbf{\ipa{qv̩˧˥}}}}} \hypertarget{qv\string_=\string_M\string_T1}{}
\markboth{\textcolor{darkblue}{\textbf{\ipa{qv̩˧˥}}}}{}
\textcolor{teal}{\mytextsc{verbe}} \hspace{4pt} Ton~: MH.

Faire peur, effrayer.
\textcolor{brown}{\zh{吓(吓唬)。}}


\begin{exe}
\sn \textcolor{darkblue}{\textbf{\ipa{hĩ˧ qv̩˩}}}
\trans faire peur aux gens
\trans \textit{\textcolor{brown}{\zh{吓人}}}
\end{exe}
\begin{exe}
\sn \textcolor{darkblue}{\textbf{\ipa{no˧ | hĩ˧ qv̩˩-zo˩! / ʈʂʰɯ˧-ɳɯ˧ | hĩ˧ qv̩˩-zo˩!}}}
\trans tu fais peur aux gens! Il fait peur aux gens!
\trans \textit{\textcolor{brown}{\zh{你吓人! / 他吓人!}}}
\end{exe}
\begin{exe}
\sn \textcolor{darkblue}{\textbf{\ipa{ʈʂʰɯ˧ | njæ˩ qv̩˩-tsʰɯ˩˥!}}}
\trans il fait peur!
\trans \textit{\textcolor{brown}{\zh{他吓人!}}}
\end{exe}
\begin{exe}
\sn \textcolor{darkblue}{\textbf{\ipa{njɤ˧ɳɯ˧ | ʈʂʰɯ˧ qv̩˩-bi˩!}}}
\trans Je vais lui faire peur!
\trans \textit{\textcolor{brown}{\zh{我要吓唬他一下!}}}
\end{exe}
\begin{exe}
\sn \textcolor{darkblue}{\textbf{\ipa{tʰɑ˧-qv̩˧˥!}}}
\trans \mytextsc{prohib}
\trans \textit{\textcolor{brown}{\zh{别吓唬(人家)!}}}
\end{exe}


\paragraph{\hspace{-0.5cm} {\Large \textcolor{darkblue}{\textbf{\ipa{qv̩˩\textsubscript{a}}}}}} \hypertarget{qv\string_=\string_Ba1}{}
\markboth{\textcolor{darkblue}{\textbf{\ipa{qv̩˩\textsubscript{a}}}}}{}
\textcolor{teal}{\mytextsc{verbe}} \hspace{4pt} Ton~: L\textsubscript{a}.

Faire rouler, emporter, charrier (un objet lourd: par exemple, le courant emporte des cailloux, les charriant au loin; le verbe ne peut s'employer pour des objets légers, par exemple des feuilles).
\textcolor{brown}{\zh{冲走。}}


\begin{exe}
\sn \textcolor{darkblue}{\textbf{\ipa{le˧-qv̩˩ | le˧-po˧-tsʰɯ˧˥}}}
\trans charrier, amener en faisant rouler: un torrent en crue charrie des cailloux jusque dans la plaine
\trans \textit{\textcolor{brown}{\zh{冲到某个地方}}}
\end{exe}
\begin{exe}
\sn \textcolor{darkblue}{\textbf{\ipa{lv̩˧mi˧ | ɬi˧dʑɯ˩-ɳɯ˩ | qv̩˩˥.}}}
\trans les pierres sont amenées par (le courant de) la rivière de Yongning
\trans \textit{\textcolor{brown}{\zh{石头被永宁河水冲(到坝子)}}}
\end{exe}
\begin{exe}
\sn \textcolor{darkblue}{\textbf{\ipa{dʑɯ˧-ɳɯ˧ | le˧-qv̩˩ | le˧-po˧-tsʰɯ˧-hĩ˥ | lv̩˧mi˧}}}
\trans pierres amenées par la rivière, pierres charriées (jusqu'ici) par la rivière
\trans \textit{\textcolor{brown}{\zh{水流冲下来的石头}}}
\end{exe}


\paragraph{\hspace{-0.5cm} {\Large \textcolor{darkblue}{\textbf{\ipa{qv̩˧dzi˩}}}}} \hypertarget{qv\string_=\string_Mdzi\string_B1}{}
\markboth{\textcolor{darkblue}{\textbf{\ipa{qv̩˧dzi˩}}}}{}
\textcolor{teal}{\mytextsc{nom}} \hspace{4pt} Ton~: L\#.

\textit{Pinus massoniana D.Don in Lamb.}, conifère de la famille des \textit{Pinaceae}. Ses pignes ne sont pas comestibles: les poissons les mangent, mais pour les hommes elles sont vénéneuses.
\textcolor{brown}{\zh{马尾松。}}Dialecte chinois local~: \zh{马松树。}


\begin{exe}
\sn \textcolor{darkblue}{\textbf{\ipa{qv̩˧dzi˩-lv̩˩\textasciitilde{}lv̩˩, | dzɯ˧ mɤ˧-ɖo˧!}}}
\trans Il ne faut pas manger les pignes du pin de Masson! (Elles sont vénéneuses.)
\trans \textit{\textcolor{brown}{\zh{马松树的果子,不要吃!(有毒)}}}
\end{exe}
 \mytextsc{clf}~: \textcolor{darkblue}{\textbf{\ipa{dzi˩, ʝi˧}}} 

\paragraph{\hspace{-0.5cm} {\Large \textcolor{darkblue}{\textbf{\ipa{qv̩˧ɻ\#˥}}}}} \hypertarget{qv\string_=\string_Mr£`\#\string_T1}{}
\markboth{\textcolor{darkblue}{\textbf{\ipa{qv̩˧ɻ\#˥}}}}{}
\textcolor{teal}{\mytextsc{nom}} \hspace{4pt} Ton~: \#H.

Une montagne de Yongning.
\textcolor{brown}{\zh{永宁的一座山。}}


\begin{exe}
\sn \textcolor{darkblue}{\textbf{\ipa{qv̩˧ɻ̍˧-ʁo˧-qʰwɤ˥}}}
\trans le sommet de la montagne \textcolor{darkblue}{\textbf{\ipa{/qv̩˧ɻ̍˧/}}}
\trans \textit{\textcolor{brown}{\zh{\textcolor{darkblue}{\textbf{\ipa{/qv̩˧ɻ̍˧/}}}山的山顶}}}
\end{exe}


\paragraph{\hspace{-0.5cm} {\Large \textcolor{darkblue}{\textbf{\ipa{qv̩˧tɕi˥}}}}} \hypertarget{qv\string_=\string_Mts£i\string_T1}{}
\markboth{\textcolor{darkblue}{\textbf{\ipa{qv̩˧tɕi˥}}}}{}
\textcolor{teal}{\mytextsc{nom}} \hspace{4pt} Ton~: H\#.

Crachat, mucus.
\textcolor{brown}{\zh{痰。}}




\paragraph{\hspace{-0.5cm} {\Large \textcolor{darkblue}{\textbf{\ipa{qv̩˧ʈʂæ˧˥}}}}} \hypertarget{qv\string_=\string_Mt`s`\{\string_M\string_T1}{}
\markboth{\textcolor{darkblue}{\textbf{\ipa{qv̩˧ʈʂæ˧˥}}}}{}
\textcolor{teal}{\mytextsc{nom}} \hspace{4pt} Ton~: MH\#.
\ding{202} 
Gorge.
\textcolor{brown}{\zh{喉咙。}}


\ding{203} 
Voix.
\textcolor{brown}{\zh{声音。}}


\begin{exe}
\sn \textcolor{darkblue}{\textbf{\ipa{ʈʂʰɯ˧ | qv̩˧ʈʂæ˧ dʑɤ˥!}}}
\trans Elle/il a une belle voix.
\trans \textit{\textcolor{brown}{\zh{他嗓子好。}}}
\end{exe}
\begin{exe}
\sn \textcolor{darkblue}{\textbf{\ipa{ʈʂʰɯ˧ | qv̩˧ʈʂæ˧˥ | ɖwæ˧˥ | dʑɤ˩˥!}}}
\trans Elle/il a une très belle voix.
\trans \textit{\textcolor{brown}{\zh{他嗓子很好。}}}
\end{exe}
 \mytextsc{clf}~: \textcolor{darkblue}{\textbf{\ipa{ɭɯ˧}}} 

\paragraph{\hspace{-0.5cm} {\Large \textcolor{darkblue}{\textbf{\ipa{qwɑ˧mæ\#˥}}}}} \hypertarget{qwA\string_Mm\{\#\string_T1}{}
\markboth{\textcolor{darkblue}{\textbf{\ipa{qwɑ˧mæ\#˥}}}}{}
\textcolor{teal}{\mytextsc{nom}} \hspace{4pt} Ton~: \#H.

Partie médiane du foyer: sur la partie surélevée, mais ``côté cuisine", pas la partie la plus noble de l'espace où on prend les repas.
\textcolor{brown}{\zh{主屋的中庭:在主屋上半部分与门之间的空间。}}


 \mytextsc{clf}~: \textcolor{darkblue}{\textbf{\ipa{kʰwɤ˥}}} 

\paragraph{\hspace{-0.5cm} {\Large \textcolor{darkblue}{\textbf{\ipa{qwæ˧}}}}} \hypertarget{qw\{\string_M1}{}
\markboth{\textcolor{darkblue}{\textbf{\ipa{qwæ˧}}}}{}
\textcolor{teal}{\mytextsc{nom}} \hspace{4pt} Ton~: M.

Sommier (de lit); banc large.
\textcolor{brown}{\zh{床垫子。}}


\begin{exe}
\sn \textcolor{darkblue}{\textbf{\ipa{qwæ˧mi\#˥}}}
\trans grand sommier
\trans \textit{\textcolor{brown}{\zh{大床垫子}}}
\end{exe}
 \mytextsc{clf}~: \textcolor{darkblue}{\textbf{\ipa{nɑ˧}}} 

\paragraph{\hspace{-0.5cm} {\Large \textcolor{darkblue}{\textbf{\ipa{qwæ˧lo˧˥}}}}} \hypertarget{qw\{\string_Mlo\string_M\string_T1}{}
\markboth{\textcolor{darkblue}{\textbf{\ipa{qwæ˧lo˧˥}}}}{}
\textcolor{teal}{\mytextsc{nom}} \hspace{4pt} Ton~: MH\#.

Petit passage, petit sentier.
\textcolor{brown}{\zh{过道、小道。}}


\begin{exe}
\sn \textcolor{darkblue}{\textbf{\ipa{qwæ˧lo˧-qo˥ | gɤ˩tɕo˧ le˧-jo˩}}}
\trans venir par le petit chemin/la sente (contexte: on demande à l'enquêteur si, pour se rendre de la maison de la consultante à son hameau natal, tout proche, il est passé par la rue principale de Yongning, ou a emprunté le petit chemin de derrière, parmi les champs)
\trans \textit{\textcolor{brown}{\zh{抄小道}}}
\end{exe}
 \mytextsc{clf}~: \textcolor{darkblue}{\textbf{\ipa{kʰɯ˩}}} 

\paragraph{\hspace{-0.5cm} {\Large \textcolor{darkblue}{\textbf{\ipa{qwæ˧ʁo\#˥}}}}} \hypertarget{qw\{\string_MRo\#\string_T1}{}
\markboth{\textcolor{darkblue}{\textbf{\ipa{qwæ˧ʁo\#˥}}}}{}
\textcolor{teal}{\mytextsc{nom}} \hspace{4pt} Ton~: \#H.

Banc de la pièce principale, proche du foyer, où s'asseyent les hôtes.
\textcolor{brown}{\zh{主屋里面的长凳:客人和老人坐的地方。}}


 \mytextsc{clf}~: \textcolor{darkblue}{\textbf{\ipa{ɭɯ˧}}} \textit{Voir~:} \hyperlink{qw\{\string_M\string_T3}{\textcolor{darkblue}{\textbf{\ipa{qwæ˧˥}}} \textsubscript{3}} 

\paragraph{\hspace{-0.5cm} {\Large \textcolor{darkblue}{\textbf{\ipa{qwæ˧ʂe\#˥}}}}} \hypertarget{qw\{\string_Ms`e\#\string_T1}{}
\markboth{\textcolor{darkblue}{\textbf{\ipa{qwæ˧ʂe\#˥}}}}{}
\textcolor{teal}{\mytextsc{nom}} \hspace{4pt} Ton~: \#H.

Punaise.
\textcolor{brown}{\zh{臭虫。}}


 \mytextsc{clf}~: \textcolor{darkblue}{\textbf{\ipa{mi˩}}} 

\paragraph{\hspace{-0.5cm} {\Large \textcolor{darkblue}{\textbf{\ipa{qwæ˧ʂe˧lɑ˧bv̩˥}}}}} \hypertarget{qw\{\string_Ms`e\string_MlA\string_Mbv\string_=\string_T1}{}
\markboth{\textcolor{darkblue}{\textbf{\ipa{qwæ˧ʂe˧lɑ˧bv̩˥}}}}{}
\textcolor{teal}{\mytextsc{nom}} \hspace{4pt} Ton~: H\#.

Sorte de ver.
\textcolor{brown}{\zh{一种蠕虫。}}


 \mytextsc{clf}~: \textcolor{darkblue}{\textbf{\ipa{mi˩}}} 

\paragraph{\hspace{-0.5cm} {\Large \textcolor{darkblue}{\textbf{\ipa{qwæ˧zo˧zo˩}}}}} \hypertarget{qw\{\string_Mzo\string_Mzo\string_B1}{}
\markboth{\textcolor{darkblue}{\textbf{\ipa{qwæ˧zo˧zo˩}}}}{}
\textcolor{teal}{\mytextsc{nom}} \hspace{4pt} Ton~: L\#.

Banc de la pièce principale, proche du foyer, où s'asseyent les hôtes.
\textcolor{brown}{\zh{主屋的长凳,离火塘近。这是客人的尊座。}}


 \mytextsc{clf}~: \textcolor{darkblue}{\textbf{\ipa{pɤ˩}}} 

\paragraph{\hspace{-0.5cm} {\Large \textcolor{darkblue}{\textbf{\ipa{qwæ˩ɖʐæ˩}}}}} \hypertarget{qw\{\string_Bd`z`\{\string_B1}{}
\markboth{\textcolor{darkblue}{\textbf{\ipa{qwæ˩ɖʐæ˩}}}}{}
\textcolor{teal}{\mytextsc{nom}} \hspace{4pt} Ton~: L.

Mâchoire; bouche.
\textcolor{brown}{\zh{颚、嘴、嘴巴、口。}}


\begin{exe}
\sn \textcolor{darkblue}{\textbf{\ipa{qwæ˩ɖʐæ˩-qo˥-ɳɯ˩ | ʈʰæ˧˥}}}
\trans mastiquer, ronger
\trans \textit{\textcolor{brown}{\zh{咬在嘴里}}}
\end{exe}
 \mytextsc{clf}~: \textcolor{darkblue}{\textbf{\ipa{ɭɯ˧}}} 

\paragraph{\hspace{-0.5cm} {\Large \textcolor{darkblue}{\textbf{\ipa{qwæ˩\textasciitilde{}qwæ˧˥}}}}} \hypertarget{qw\{\string_B~qw\{\string_M\string_T1}{}
\markboth{\textcolor{darkblue}{\textbf{\ipa{qwæ˩\textasciitilde{}qwæ˧˥}}}}{}
\textcolor{teal}{\mytextsc{verbe}} \hspace{4pt} Ton~: .

Se gratter; gratter, gratouiller.
\textcolor{brown}{\zh{抠痒。}}


\begin{exe}
\sn \textcolor{darkblue}{\textbf{\ipa{le˧-qwæ˧\textasciitilde{}qwæ˩-ze˩}}}
\trans \mytextsc{accomp} \string_ \mytextsc{red} \mytextsc{pfv}
\trans \textit{\textcolor{brown}{\zh{\mytextsc{accomp} \string_ \mytextsc{red} \mytextsc{pfv}}}}
\end{exe}


\paragraph{\hspace{-0.5cm} {\Large \textcolor{darkblue}{\textbf{\ipa{qwæ˩ʂv̩˧˥}}}}} \hypertarget{qw\{\string_Bs`v\string_=\string_M\string_T1}{}
\markboth{\textcolor{darkblue}{\textbf{\ipa{qwæ˩ʂv̩˧˥}}}}{}
\textcolor{teal}{\mytextsc{nom}} \hspace{4pt} Ton~: LM+MH\#.

Mors.
\textcolor{brown}{\zh{马嚼子。}}


\begin{exe}
\sn \textcolor{darkblue}{\textbf{\ipa{ʐwæ˧-qwæ˥ʂv̩˩}}}
\trans mors de cheval
\trans \textit{\textcolor{brown}{\zh{马嚼子}}}
\end{exe}
 \mytextsc{clf}~: \textcolor{darkblue}{\textbf{\ipa{nɑ˧}}} 

\paragraph{\hspace{-0.5cm} {\Large \textcolor{darkblue}{\textbf{\ipa{qwæ˧˥}}} \textsubscript{1}}} \hypertarget{qw\{\string_M\string_T1}{}
\markboth{\textcolor{darkblue}{\textbf{\ipa{qwæ˧˥}}} \textsubscript{1}}{}
\textcolor{teal}{\mytextsc{verbe}} \hspace{4pt} Ton~: MH.
\ding{202} 
Creuser, piocher (dans la terre meuble).
\textcolor{brown}{\zh{挖(土)。}}


\begin{exe}
\sn \textcolor{darkblue}{\textbf{\ipa{tv̩˧qʰv̩˧ qwæ˧˥}}}
\trans creuser un trou
\trans \textit{\textcolor{brown}{\zh{挖洞}}}
\end{exe}
\begin{exe}
\sn \textcolor{darkblue}{\textbf{\ipa{ʈʂe˧ qwæ˩}}}
\trans creuser la terre
\trans \textit{\textcolor{brown}{\zh{挖土}}}
\end{exe}
\begin{exe}
\sn \textcolor{darkblue}{\textbf{\ipa{qʰæ˧lo˧ qwæ˥}}}
\trans dégager une rigole
\trans \textit{\textcolor{brown}{\zh{挖水沟}}}
\end{exe}
\begin{exe}
\sn \textcolor{darkblue}{\textbf{\ipa{jɤ˩jo˥ qwæ˩}}}
\trans déterrer des pommes de terre, récolter des pommes de terre
\trans \textit{\textcolor{brown}{\zh{挖洋芋}}}
\end{exe}
\ding{203} 
Puiser (de l'eau).
\textcolor{brown}{\zh{舀(水)。}}


\begin{exe}
\sn \textcolor{darkblue}{\textbf{\ipa{dʑɯ˩ qwæ˩˥}}}
\trans puiser de l'eau
\trans \textit{\textcolor{brown}{\zh{舀水}}}
\end{exe}


\paragraph{\hspace{-0.5cm} {\Large \textcolor{darkblue}{\textbf{\ipa{qwæ˧˥}}} \textsubscript{2}}} \hypertarget{qw\{\string_M\string_T2}{}
\markboth{\textcolor{darkblue}{\textbf{\ipa{qwæ˧˥}}} \textsubscript{2}}{}
\textcolor{teal}{\mytextsc{verbe}} \hspace{4pt} Ton~: MH.

Graver.
\textcolor{brown}{\zh{雕刻。}}


\begin{exe}
\sn \textcolor{darkblue}{\textbf{\ipa{bæ˩bæ˩ qwæ˥}}}
\trans graver une fleur
\trans \textit{\textcolor{brown}{\zh{刻花}}}
\end{exe}
\begin{exe}
\sn \textcolor{darkblue}{\textbf{\ipa{qwæ˩\textasciitilde{}qwæ˧˥}}}
\trans \mytextsc{red}
\trans \textit{\textcolor{brown}{\zh{\mytextsc{重叠}}}}
\end{exe}
\begin{exe}
\sn \textcolor{darkblue}{\textbf{\ipa{bæ˩bæ˩ qwæ˥\textasciitilde{}qwæ˩}}}
\trans graver des fleurs
\trans \textit{\textcolor{brown}{\zh{刻一朵花}}}
\end{exe}


\paragraph{\hspace{-0.5cm} {\Large \textcolor{darkblue}{\textbf{\ipa{qwæ˧˥}}} \textsubscript{3}}} \hypertarget{qw\{\string_M\string_T3}{}
\markboth{\textcolor{darkblue}{\textbf{\ipa{qwæ˧˥}}} \textsubscript{3}}{}
\textcolor{teal}{\mytextsc{nom}} \hspace{4pt} Ton~: \#H.

Banc de la pièce principale, proche du foyer, où s'asseyent les hôtes.
\textcolor{brown}{\zh{主屋里面的长凳:客人和老人坐的地方。}}


 \mytextsc{clf}~: \textcolor{darkblue}{\textbf{\ipa{ɭɯ˧}}} \textit{Voir~:} \hyperlink{qw\{\string_MRo\#\string_T1}{\textcolor{darkblue}{\textbf{\ipa{qwæ˧ʁo\#˥}}}} 

\paragraph{\hspace{-0.5cm} {\Large \textcolor{darkblue}{\textbf{\ipa{qwæ˩˥}}}}} \hypertarget{qw\{\string_B\string_T1}{}
\markboth{\textcolor{darkblue}{\textbf{\ipa{qwæ˩˥}}}}{}
\textcolor{teal}{\mytextsc{nom}} \hspace{4pt} Ton~: LH.

Mâchoire (monosyllabe).
\textcolor{brown}{\zh{嘴巴(单音节)。}}


 \mytextsc{clf}~: \textcolor{darkblue}{\textbf{\ipa{ɭɯ˧}}} 

\paragraph{\hspace{-0.5cm} {\Large \textcolor{darkblue}{\textbf{\ipa{qwɤ˧}}}}} \hypertarget{qw7\string_M1}{}
\markboth{\textcolor{darkblue}{\textbf{\ipa{qwɤ˧}}}}{}
\textcolor{teal}{\mytextsc{nom}} \hspace{4pt} Ton~: M.

Foyer, âtre, lieu où on fait du feu dans la maison.
\textcolor{brown}{\zh{火塘。}}


\begin{exe}
\sn \textcolor{darkblue}{\textbf{\ipa{qwɤ˧, | mv̩˧ kʰɯ˩-di˩!}}}
\trans Le foyer, c'est là où on allume le feu!
\trans \textit{\textcolor{brown}{\zh{火塘,就是升火的地方!}}}
\end{exe}
 \mytextsc{clf}~: \textcolor{darkblue}{\textbf{\ipa{ɭɯ˧}}} 

\paragraph{\hspace{-0.5cm} {\Large \textcolor{darkblue}{\textbf{\ipa{qwɤ˧\textsubscript{a}}}}}} \hypertarget{qw7\string_Ma1}{}
\markboth{\textcolor{darkblue}{\textbf{\ipa{qwɤ˧\textsubscript{a}}}}}{}
\textcolor{teal}{\mytextsc{verbe}} \hspace{4pt} Ton~: M\textsubscript{a}.

Accuser.
\textcolor{brown}{\zh{告状。}}


\begin{exe}
\sn \textcolor{darkblue}{\textbf{\ipa{mɤ˧-qwɤ˧}}}
\trans \mytextsc{neg}
\trans \textit{\textcolor{brown}{\zh{不告状}}}
\end{exe}
\begin{exe}
\sn \textcolor{darkblue}{\textbf{\ipa{hĩ˧ qwɤ˩}}}
\trans dénoncer quelqu'un
\trans \textit{\textcolor{brown}{\zh{告一个人}}}
\end{exe}
\begin{exe}
\sn \textcolor{darkblue}{\textbf{\ipa{njɤ˧-ɳɯ˧ | qwɤ˧-bi˧!}}}
\trans je vais (te) dénoncer!
\trans \textit{\textcolor{brown}{\zh{我要告状!}}}
\end{exe}
\begin{exe}
\sn \textcolor{darkblue}{\textbf{\ipa{no˧ | le˧-qwɤ˧-hõ˧!}}}
\trans va (le) dénoncer!
\trans \textit{\textcolor{brown}{\zh{你去告状吧!}}}
\end{exe}
\begin{exe}
\sn \textcolor{darkblue}{\textbf{\ipa{qwɤ˧\textasciitilde{}qwɤ˩}}}
\trans \mytextsc{red}
\trans \textit{\textcolor{brown}{\zh{\mytextsc{重叠}}}}
\end{exe}


\paragraph{\hspace{-0.5cm} {\Large \textcolor{darkblue}{\textbf{\ipa{qwɤ˧ɭɯ\#˥}}}}} \hypertarget{qw7\string_Ml\string_RM\#\string_T1}{}
\markboth{\textcolor{darkblue}{\textbf{\ipa{qwɤ˧ɭɯ\#˥}}}}{}
\textcolor{teal}{\mytextsc{nom}} \hspace{4pt} Ton~: \#H.

Feu de camp: foyer bâti à l'extérieur, provisoirement, lorsqu'on campe sur la montagne.
\textcolor{brown}{\zh{营火、篝火。}}


\begin{exe}
\sn \textcolor{darkblue}{\textbf{\ipa{qwɤ˧ɭɯ˧-pʰɤ˧bɤ˥}}}
\trans les cadeaux offerts aux ancêtres: même lorsqu'il ne s'agit que d'un foyer provisoire, bâti pour une seule journée dans un campement en montagne, on pratique l'offrande d'un peu de nourriture
\trans \textit{\textcolor{brown}{\zh{敬给祖先的礼物:即使在山上升起篝火野餐,还是要像在家里一样,用餐前先敬给祖先一些饭。}}}
\end{exe}


\paragraph{\hspace{-0.5cm} {\Large \textcolor{darkblue}{\textbf{\ipa{qwɤ˩\textsubscript{a}}}}}} \hypertarget{qw7\string_Ba1}{}
\markboth{\textcolor{darkblue}{\textbf{\ipa{qwɤ˩\textsubscript{a}}}}}{}
\textcolor{teal}{\mytextsc{verbe}} \hspace{4pt} Ton~: L\textsubscript{a}.

Pousser, grandir.
\textcolor{brown}{\zh{生长、长。}}


\begin{exe}
\sn \textcolor{darkblue}{\textbf{\ipa{gɤ˩-qwɤ˥}}}
\trans grandir, pousser
\trans \textit{\textcolor{brown}{\zh{长大,生长}}}
\end{exe}
\begin{exe}
\sn \textcolor{darkblue}{\textbf{\ipa{ʈʂʰɯ˧ | gɤ˩-qwɤ˥-ze˩!}}}
\trans Il/elle a grandi! (Au sujet d'un enfant qu'on revoit après un certain temps)
\trans \textit{\textcolor{brown}{\zh{他长大了!(关于一个小孩)}}}
\end{exe}


\paragraph{\hspace{-0.5cm} {\Large \textcolor{darkblue}{\textbf{\ipa{qwɤ˩pi˩}}}}} \hypertarget{qw7\string_Bpi\string_B1}{}
\markboth{\textcolor{darkblue}{\textbf{\ipa{qwɤ˩pi˩}}}}{}
\textcolor{teal}{\mytextsc{nom}} \hspace{4pt} Ton~: L.

Bouche.
\textcolor{brown}{\zh{嘴巴。}}


\begin{exe}
\sn \textcolor{darkblue}{\textbf{\ipa{qwɤ˩pi˩-qo˩lo˥}}}
\trans à l'intérieur de la bouche
\trans \textit{\textcolor{brown}{\zh{嘴巴里}}}
\end{exe}
\begin{exe}
\sn \textcolor{darkblue}{\textbf{\ipa{[F5] ko˩pi˩-ko˩lo˧}}}
\trans dans la bouche, à l'intérieur de la bouche
\trans \textit{\textcolor{brown}{\zh{嘴巴里}}}
\end{exe}
 \mytextsc{clf}~: \textcolor{darkblue}{\textbf{\ipa{ɭɯ˧}}} 

\newpage
\section*{\centering- \textcolor{darkblue}{\textbf{\ipa{qʰ}}} -}
\paragraph{\hspace{-0.5cm} {\Large \textcolor{darkblue}{\textbf{\ipa{qʰɑ˥}}}}} \hypertarget{q\string_hA\string_T1}{}
\markboth{\textcolor{darkblue}{\textbf{\ipa{qʰɑ˥}}}}{}
\textcolor{teal}{\mytextsc{adjectif}} \hspace{4pt} Ton~: H.

Amer.
\textcolor{brown}{\zh{苦。}}




\paragraph{\hspace{-0.5cm} {\Large \textcolor{darkblue}{\textbf{\ipa{}}}}} \hypertarget{q\string_hA\string_M-1}{}
\markboth{\textcolor{darkblue}{\textbf{\ipa{}}}}{}
\textcolor{teal}{\mytextsc{adverbe}} \hspace{4pt} Ton~: .

Particulièrement, très.
\textcolor{brown}{\zh{多么、非常。}}


\begin{exe}
\sn \textcolor{darkblue}{\textbf{\ipa{qʰɑ˧-ɖɯ˧-hĩ˧}}}
\trans extrêmement gros
\trans \textit{\textcolor{brown}{\zh{非常大}}}
\end{exe}
\begin{exe}
\sn \textcolor{darkblue}{\textbf{\ipa{qʰɑ˧-ɖɯ˧-gv̩˧}}}
\trans particulièrement grand
\trans \textit{\textcolor{brown}{\zh{非常大}}}
\end{exe}
\begin{exe}
\sn \textcolor{darkblue}{\textbf{\ipa{qʰɑ˧-ʂwæ˧-gv̩˧}}}
\trans particulièrement grand, de très grande taille
\trans \textit{\textcolor{brown}{\zh{很高、非常高}}}
\end{exe}
\begin{exe}
\sn \textcolor{darkblue}{\textbf{\ipa{qʰɑ˧-ʂwæ˧-mi˧zo˥}}}
\trans très grand, de très haute taille
\trans \textit{\textcolor{brown}{\zh{很高}}}
\end{exe}
\begin{exe}
\sn \textcolor{darkblue}{\textbf{\ipa{qʰɑ˧-ɖɯ˧-mi˧zo˥}}}
\trans très gros, de très grande envergure
\trans \textit{\textcolor{brown}{\zh{很大}}}
\end{exe}
\textit{Voir~:} \hyperlink{q\string_hA\string_M1}{\textcolor{darkblue}{\textbf{\ipa{qʰɑ˧}}} \textsubscript{1}} 

\paragraph{\hspace{-0.5cm} {\Large \textcolor{darkblue}{\textbf{\ipa{qʰɑ˧}}} \textsubscript{1}}} \hypertarget{q\string_hA\string_M1}{}
\markboth{\textcolor{darkblue}{\textbf{\ipa{qʰɑ˧}}} \textsubscript{1}}{}
\textcolor{teal}{\mytextsc{pronom}} \hspace{4pt} Ton~: M.

Combien.
\textcolor{brown}{\zh{几、多少。}}


\begin{exe}
\sn \textcolor{darkblue}{\textbf{\ipa{hĩ˧ | qʰɑ˧-kv̩˧˥?}}}
\trans combien de gens?
\trans \textit{\textcolor{brown}{\zh{几个人?}}}
\end{exe}
\begin{exe}
\sn \textcolor{darkblue}{\textbf{\ipa{bæ˩bæ˩˥ | qʰɑ˧-bæ˩?}}}
\trans combien de fleurs?
\trans \textit{\textcolor{brown}{\zh{几朵花?}}}
\end{exe}
\begin{exe}
\sn \textcolor{darkblue}{\textbf{\ipa{qʰɑ˧-ʑi˩?}}}
\trans combien de familles?
\trans \textit{\textcolor{brown}{\zh{几家?}}}
\end{exe}
\begin{exe}
\sn \textcolor{darkblue}{\textbf{\ipa{hɑ˧ | qʰɑ˧-tɕʰi˩?}}}
\trans combien de repas?
\trans \textit{\textcolor{brown}{\zh{几顿饭?}}}
\end{exe}
\begin{exe}
\sn \textcolor{darkblue}{\textbf{\ipa{qʰɑ˧-ɲi˧?}}}
\trans combien de jours?
\trans \textit{\textcolor{brown}{\zh{几天?}}}
\end{exe}
\begin{exe}
\sn \textcolor{darkblue}{\textbf{\ipa{qʰɑ˧-kʰv̩˧ gv̩˧-ze˩?}}}
\trans quel âge avez-(vous)?
\trans \textit{\textcolor{brown}{\zh{几岁了?}}}
\end{exe}
\begin{exe}
\sn \textcolor{darkblue}{\textbf{\ipa{qʰɑ˧-kʰv̩˧˥?}}}
\trans combien d'années?
\trans \textit{\textcolor{brown}{\zh{几年?}}}
\end{exe}
\begin{exe}
\sn \textcolor{darkblue}{\textbf{\ipa{qʰɑ˧-kʰwɤ˧˥?}}}
\trans combien de morceaux?
\trans \textit{\textcolor{brown}{\zh{几块?}}}
\end{exe}
\begin{exe}
\sn \textcolor{darkblue}{\textbf{\ipa{qʰɑ˧-nɑ˧?}}}
\trans combien (d'outils...)?
\trans \textit{\textcolor{brown}{\zh{几把?}}}
\end{exe}
\begin{exe}
\sn \textcolor{darkblue}{\textbf{\ipa{sɯ˩tʰi˩˥ | qʰɑ˧-nɑ˧ dʑo˧?}}}
\trans Combien y a-t-il de couteaux?
\trans \textit{\textcolor{brown}{\zh{有几把刀?}}}
\end{exe}
\begin{exe}
\sn \textcolor{darkblue}{\textbf{\ipa{qʰɑ˧-kʰɯ˩}}}
\trans combien (d'objets longs)
\trans \textit{\textcolor{brown}{\zh{几条}}}
\end{exe}
\begin{exe}
\sn \textcolor{darkblue}{\textbf{\ipa{qʰɑ˧-kʰɯ˩ dʑo˩?}}}
\trans Combien y a-t-il (d'objets longs)?
\trans \textit{\textcolor{brown}{\zh{有几条?}}}
\end{exe}
\begin{exe}
\sn \textcolor{darkblue}{\textbf{\ipa{qʰɑ˧-mæ˩ dʑo˩?}}}
\trans Combien (tu) as d'argent?
\trans \textit{\textcolor{brown}{\zh{有几块(钱)?}}}
\end{exe}
\begin{exe}
\sn \textcolor{darkblue}{\textbf{\ipa{si˧dzi˩ | qʰɑ˧-dzi˩?}}}
\trans combien d'arbres?
\trans \textit{\textcolor{brown}{\zh{几棵树?}}}
\end{exe}
\begin{exe}
\sn \textcolor{darkblue}{\textbf{\ipa{si˧kɤ˧˥ | qʰɑ˧-kɤ˧˥?}}}
\trans combien de branches?
\trans \textit{\textcolor{brown}{\zh{几枝树枝?}}}
\end{exe}
\begin{exe}
\sn \textcolor{darkblue}{\textbf{\ipa{qʰɑ˧-kʰɤ˧˥?}}}
\trans combien de cageots?
\trans \textit{\textcolor{brown}{\zh{几筐?}}}
\end{exe}
\textit{Voir~:} \hyperlink{q\string_hA\string_M2}{\textcolor{darkblue}{\textbf{\ipa{qʰɑ˧}}} \textsubscript{2}} 

\paragraph{\hspace{-0.5cm} {\Large \textcolor{darkblue}{\textbf{\ipa{qʰɑ˧}}} \textsubscript{2}}} \hypertarget{q\string_hA\string_M2}{}
\markboth{\textcolor{darkblue}{\textbf{\ipa{qʰɑ˧}}} \textsubscript{2}}{}
\textcolor{teal}{\mytextsc{adverbe}} \hspace{4pt} Ton~: M.

Quelques, plusieurs.
\textcolor{brown}{\zh{几(如:十几个)。}}


\begin{exe}
\sn \textcolor{darkblue}{\textbf{\ipa{tsʰe˩-qʰɑ˩˥}}}
\trans dix et plus (entre dix et vingt)
\trans \textit{\textcolor{brown}{\zh{十几个、十来个}}}
\end{exe}
\begin{exe}
\sn \textcolor{darkblue}{\textbf{\ipa{tsʰe˩-qʰɑ˩-kv̩˩˥}}}
\trans dix et plus (entre dix et vingt)
\trans \textit{\textcolor{brown}{\zh{十几个、十来个}}}
\end{exe}
\textit{Voir~:} \hyperlink{q\string_hA\string_M1}{\textcolor{darkblue}{\textbf{\ipa{qʰɑ˧}}} \textsubscript{1}} 

\paragraph{\hspace{-0.5cm} {\Large \textcolor{darkblue}{\textbf{\ipa{qʰɑ˧dze˧}}}}} \hypertarget{q\string_hA\string_Mdze\string_M1}{}
\markboth{\textcolor{darkblue}{\textbf{\ipa{qʰɑ˧dze˧}}}}{}
\textcolor{teal}{\mytextsc{nom}} \hspace{4pt} Ton~: M.

Maïs.
\textcolor{brown}{\zh{玉米、包谷。}}


\begin{exe}
\sn \textcolor{darkblue}{\textbf{\ipa{qʰɑ˧dze˧-kʰɯ˩ʈɯ˩}}}
\trans racines des plants de maïs
\trans \textit{\textcolor{brown}{\zh{玉米的根}}}
\end{exe}
\begin{exe}
\sn \textcolor{darkblue}{\textbf{\ipa{qʰɑ˧dze˧ qʰæ˩}}}
\trans cueillir le maïs: arracher les épis de maïs
\trans \textit{\textcolor{brown}{\zh{采玉米:折断玉米棒子}}}
\end{exe}
\begin{exe}
\sn \textcolor{darkblue}{\textbf{\ipa{qʰɑ˧dze˧ ɖʐɤ˧˥}}}
\trans cueillir le maïs
\trans \textit{\textcolor{brown}{\zh{采玉米}}}
\end{exe}
\begin{exe}
\sn \textcolor{darkblue}{\textbf{\ipa{qʰɑ˧dze˧-tsɑ˩bɤ˩ | ɖɯ˧-mɤ˩}}}
\trans un peu de farine de maïs
\trans \textit{\textcolor{brown}{\zh{一点玉米粉}}}
\end{exe}
\begin{exe}
\sn \textcolor{darkblue}{\textbf{\ipa{qʰɑ˧dze˧-hɑ˧bɤ˥, | qʰɑ˧dze˧-hɑ˧ɭɯ\#˥, | qʰɑ˧dze˧-tsɑ˩bɤ˩}}}
\trans le maïs sous trois formes: épis de maïs; maïs en grains; farine de maïs
\trans \textit{\textcolor{brown}{\zh{玉米的三种形态:玉米棒子,玉米粒,玉米粉}}}
\end{exe}
 \mytextsc{clf}~: \textcolor{darkblue}{\textbf{\ipa{kɤ˧˥}}} 

\paragraph{\hspace{-0.5cm} {\Large \textcolor{darkblue}{\textbf{\ipa{qʰɑ˧dze˧-hwæ˩-di˩}}}}} \hypertarget{q\string_hA\string_Mdze\string_M-hw\{\string_B-di\string_B1}{}
\markboth{\textcolor{darkblue}{\textbf{\ipa{qʰɑ˧dze˧-hwæ˩-di˩}}}}{}
\textcolor{teal}{\mytextsc{nom}} \hspace{4pt} Ton~: \mytextsc{L}.

Poutrelle d'espalier à sécher le maïs: partie horizontale de la structure en bois. Périphrase: 'endroit où on accroche le maïs'.
\textcolor{brown}{\zh{粮架的横梁。}}


 \mytextsc{clf}~: \textcolor{darkblue}{\textbf{\ipa{kɤ˧˥}}} 

\paragraph{\hspace{-0.5cm} {\Large \textcolor{darkblue}{\textbf{\ipa{qʰɑ˧dze˧-lv̩˧}}}}} \hypertarget{q\string_hA\string_Mdze\string_M-lv\string_=\string_M1}{}
\markboth{\textcolor{darkblue}{\textbf{\ipa{qʰɑ˧dze˧-lv̩˧}}}}{}
\textcolor{teal}{\mytextsc{nom}} \hspace{4pt} Ton~: M.

Champ de maïs.
\textcolor{brown}{\zh{包谷田、玉米田。}}


 \mytextsc{clf}~: \textcolor{darkblue}{\textbf{\ipa{pʰv̩˩}}} 

\paragraph{\hspace{-0.5cm} {\Large \textcolor{darkblue}{\textbf{\ipa{qʰɑ˧tɑ˧}}}}} \hypertarget{q\string_hA\string_MtA\string_M1}{}
\markboth{\textcolor{darkblue}{\textbf{\ipa{qʰɑ˧tɑ˧}}}}{}
\textcolor{teal}{\mytextsc{pronom}} \hspace{4pt} Ton~: M.

Quand.
\textcolor{brown}{\zh{什么时候。}}


\begin{exe}
\sn \textcolor{darkblue}{\textbf{\ipa{qʰɑ˧tɑ˧ bi˧?}}}
\trans quand (y) vas(-tu)?
\trans \textit{\textcolor{brown}{\zh{你什么时候去?}}}
\end{exe}


\paragraph{\hspace{-0.5cm} {\Large \textcolor{darkblue}{\textbf{\ipa{qʰɑ˧-ʈʂʰɤ˧\textasciitilde{}ʈʂʰɤ˥}}}}} \hypertarget{q\string_hA\string_M-t`s`\string_h7\string_M~t`s`\string_h7\string_T1}{}
\markboth{\textcolor{darkblue}{\textbf{\ipa{qʰɑ˧-ʈʂʰɤ˧\textasciitilde{}ʈʂʰɤ˥}}}}{}
\textcolor{teal}{\mytextsc{adjectif}} \hspace{4pt} Ton~: .

Tout sec.
\textcolor{brown}{\zh{充分干燥。}}


\begin{exe}
\sn \textcolor{darkblue}{\textbf{\ipa{qʰɑ˧-ʈʂʰɤ˧\textasciitilde{}ʈʂʰɤ˥-gv̩˩}}}
\trans complètement sec
\trans \textit{\textcolor{brown}{\zh{充分干燥}}}
\end{exe}
\begin{exe}
\sn \textcolor{darkblue}{\textbf{\ipa{le˧-pv̩˧-zo˩, | qʰɑ˧ʈʂʰɤ˧ʈʂʰɤ˥-gv̩˩-ze˩!}}}
\trans c'est maintenant sec, et complètement sec!
\trans \textit{\textcolor{brown}{\zh{现在,干了:全干了!}}}
\end{exe}


\paragraph{\hspace{-0.5cm} {\Large \textcolor{darkblue}{\textbf{\ipa{qʰɑ˩jɤ˩}}}}} \hypertarget{q\string_hA\string_Bj7\string_B1}{}
\markboth{\textcolor{darkblue}{\textbf{\ipa{qʰɑ˩jɤ˩}}}}{}
\textcolor{teal}{\mytextsc{pronom}} \hspace{4pt} Ton~: L.

Combien.
\textcolor{brown}{\zh{多少。}}


\begin{exe}
\sn \textcolor{darkblue}{\textbf{\ipa{qʰɑ˩jɤ˩ tʰi˥-ki˩?}}}
\trans combien donner? / c'est combien?
\trans \textit{\textcolor{brown}{\zh{要给多少? = 多少钱?}}}
\end{exe}
\begin{exe}
\sn \textcolor{darkblue}{\textbf{\ipa{qʰɑ˩jɤ˩ ɲi˧?}}}
\trans combien (t'en) faut-il?
\trans \textit{\textcolor{brown}{\zh{要多少?}}}
\end{exe}
\begin{exe}
\sn \textcolor{darkblue}{\textbf{\ipa{ɖʐe˧ | qʰɑ˩jɤ˩ ɲi˧?}}}
\trans Combien d'argent ça coûte?
\trans \textit{\textcolor{brown}{\zh{要多少钱?}}}
\end{exe}


\paragraph{\hspace{-0.5cm} {\Large \textcolor{darkblue}{\textbf{\ipa{qʰɑ˩ne˩}}}}} \hypertarget{q\string_hA\string_Bne\string_B1}{}
\markboth{\textcolor{darkblue}{\textbf{\ipa{qʰɑ˩ne˩}}}}{}
\textcolor{teal}{\mytextsc{pronom}} \hspace{4pt} Ton~: L.

Pronom interrogatif: comment?
\textcolor{brown}{\zh{怎么样。}}


\begin{exe}
\sn \textcolor{darkblue}{\textbf{\ipa{qʰɑ˩ne˩ ʝi˥?}}}
\trans comment faire?
\trans \textit{\textcolor{brown}{\zh{怎么做?}}}
\end{exe}
\begin{exe}
\sn \textcolor{darkblue}{\textbf{\ipa{qʰɑ˩ne˩ ʝi˥-tso˩-ɲi˩?}}}
\trans comment faut-il faire?
\trans \textit{\textcolor{brown}{\zh{要怎么做?}}}
\end{exe}
\begin{exe}
\sn \textcolor{darkblue}{\textbf{\ipa{qʰɑ˩ne˩ gv̩˩˥?}}}
\trans comment faire?
\trans \textit{\textcolor{brown}{\zh{怎么做?}}}
\end{exe}
\begin{exe}
\sn \textcolor{darkblue}{\textbf{\ipa{qʰɑ˩ne˩ gv̩˩-ho˥-ze˩?}}}
\trans que s'est-il passé?
\trans \textit{\textcolor{brown}{\zh{怎么样了?发展到什么程度?}}}
\end{exe}


\paragraph{\hspace{-0.5cm} {\Large \textcolor{darkblue}{\textbf{\ipa{qʰæ˥}}}}} \hypertarget{q\string_h\{\string_T1}{}
\markboth{\textcolor{darkblue}{\textbf{\ipa{qʰæ˥}}}}{}
\textcolor{teal}{\mytextsc{nom}} \hspace{4pt} Ton~: \#H.
\ding{202} 
Excréments, fèces.
\textcolor{brown}{\zh{屎、垃圾、 肥料。}}


\begin{exe}
\sn \textcolor{darkblue}{\textbf{\ipa{qʰæ˧ ɖɯ˧-pɤ˧ ʂe˧˥}}}
\trans faire une crotte
\trans \textit{\textcolor{brown}{\zh{拉一泡屎}}}
\end{exe}
\begin{exe}
\sn \textcolor{darkblue}{\textbf{\ipa{qʰæ˧ kv̩˥}}}
\trans ramasser du crottin
\trans \textit{\textcolor{brown}{\zh{捡(马……)屎}}}
\end{exe}
\begin{exe}
\sn \textcolor{darkblue}{\textbf{\ipa{qʰæ˧-pi˩ kv̩˩}}}
\trans ramasser un peu de crottin
\trans \textit{\textcolor{brown}{\zh{捡一点(马……)屎}}}
\end{exe}
\begin{exe}
\sn \textcolor{darkblue}{\textbf{\ipa{qʰæ˧ ɖɯ˧-pi˧ kv̩˥}}}
\trans même sens: ramasser un peu de crottin
\trans \textit{\textcolor{brown}{\zh{同上:捡一点(马……)屎}}}
\end{exe}
\ding{203} 
Pet.
\textcolor{brown}{\zh{屁。}}


\begin{exe}
\sn \textcolor{darkblue}{\textbf{\ipa{qʰæ˧ kʰɯ˩}}}
\trans péter
\trans \textit{\textcolor{brown}{\zh{放屁}}}
\end{exe}
\begin{exe}
\sn \textcolor{darkblue}{\textbf{\ipa{qʰæ˧ | ɖɯ˧-pɤ˥ kʰɯ˩}}}
\trans faire un pet
\trans \textit{\textcolor{brown}{\zh{放一个屁}}}
\end{exe}
\ding{204} 
Ordure, détritus.
\textcolor{brown}{\zh{垃圾。}}


 \mytextsc{clf}~: \textcolor{darkblue}{\textbf{\ipa{pɤ˥}}} \textcolor{darkblue}{\textbf{\ipa{pɤ˥}}} 

\paragraph{\hspace{-0.5cm} {\Large \textcolor{darkblue}{\textbf{\ipa{qʰæ˥}}} \textsubscript{1}}} \hypertarget{q\string_h\{\string_T1}{}
\markboth{\textcolor{darkblue}{\textbf{\ipa{qʰæ˥}}} \textsubscript{1}}{}
\textcolor{teal}{\mytextsc{verbe}} \hspace{4pt} Ton~: H.

Ronger.
\textcolor{brown}{\zh{啃(啃骨头)。}}




\paragraph{\hspace{-0.5cm} {\Large \textcolor{darkblue}{\textbf{\ipa{qʰæ˥}}} \textsubscript{2}}} \hypertarget{q\string_h\{\string_T2}{}
\markboth{\textcolor{darkblue}{\textbf{\ipa{qʰæ˥}}} \textsubscript{2}}{}
\textcolor{teal}{\mytextsc{adjectif}} \hspace{4pt} Ton~: H.

Froide (eau).
\textcolor{brown}{\zh{冷(水)。}}


\begin{exe}
\sn \textcolor{darkblue}{\textbf{\ipa{dʑɯ˩qʰæ˩}}}
\trans eau froide
\trans \textit{\textcolor{brown}{\zh{凉水}}}
\end{exe}
\begin{exe}
\sn \textcolor{darkblue}{\textbf{\ipa{qʰæ˧-ɕjæ˧-gv̩˧}}}
\trans très froid
\trans \textit{\textcolor{brown}{\zh{冷得很}}}
\end{exe}


\paragraph{\hspace{-0.5cm} {\Large \textcolor{darkblue}{\textbf{\ipa{qʰæ˥}}} \textsubscript{3}}} \hypertarget{q\string_h\{\string_T3}{}
\markboth{\textcolor{darkblue}{\textbf{\ipa{qʰæ˥}}} \textsubscript{3}}{}
\textcolor{teal}{\mytextsc{nom}} \hspace{4pt} Ton~: \#H.

Canal, rigole.
\textcolor{brown}{\zh{水沟(单音节)。}}


 \mytextsc{clf}~: \textcolor{darkblue}{\textbf{\ipa{kʰɯ˩}}} objets longs

\paragraph{\hspace{-0.5cm} {\Large \textcolor{darkblue}{\textbf{\ipa{qʰæ˧\textsubscript{b}}}}}} \hypertarget{q\string_h\{\string_Mb1}{}
\markboth{\textcolor{darkblue}{\textbf{\ipa{qʰæ˧\textsubscript{b}}}}}{}
\textcolor{teal}{\mytextsc{verbe}} \hspace{4pt} Ton~: M\textsubscript{b}.

(se) briser, (se) casser (ex.: un bâton).
\textcolor{brown}{\zh{断,破(棍子,竹竿)。}}


\begin{exe}
\sn \textcolor{darkblue}{\textbf{\ipa{le˧-qʰæ˧-ze˧}}}
\trans \mytextsc{accomp} \string_ \mytextsc{pfv}
\trans \textit{\textcolor{brown}{\zh{断了}}}
\end{exe}
\begin{exe}
\sn \textcolor{darkblue}{\textbf{\ipa{si˧ qʰæ˩}}}
\trans briser du bois
\trans \textit{\textcolor{brown}{\zh{砸木头}}}
\end{exe}


\paragraph{\hspace{-0.5cm} {\Large \textcolor{darkblue}{\textbf{\ipa{qʰæ˧kʰwɤ\#˥}}}}} \hypertarget{q\string_h\{\string_Mk\string_hw7\#\string_T1}{}
\markboth{\textcolor{darkblue}{\textbf{\ipa{qʰæ˧kʰwɤ\#˥}}}}{}
\textcolor{teal}{\mytextsc{nom}} \hspace{4pt} Ton~: \#H.

Petit barrage pour bloquer un canal d'irrigation, fait de pierres et de terre. Pour irriguer, on l'ouvre à coups de houe.
\textcolor{brown}{\zh{小水坝,来堵塞田地里的小水渠。}}


\begin{exe}
\sn \textcolor{darkblue}{\textbf{\ipa{qʰæ˧kʰwɤ˧ ɖɯ˧-ɭɯ˧}}}
\trans un petit canal
\trans \textit{\textcolor{brown}{\zh{一个小水坝}}}
\end{exe}
 \mytextsc{clf}~: \textcolor{darkblue}{\textbf{\ipa{ɭɯ˧}}} 

\paragraph{\hspace{-0.5cm} {\Large \textcolor{darkblue}{\textbf{\ipa{qʰæ˧lo˧˥}}}}} \hypertarget{q\string_h\{\string_Mlo\string_M\string_T1}{}
\markboth{\textcolor{darkblue}{\textbf{\ipa{qʰæ˧lo˧˥}}}}{}
\textcolor{teal}{\mytextsc{nom}} \hspace{4pt} Ton~: MH\#.

Petite rigole, petit canal.
\textcolor{brown}{\zh{小水渠。}}


 \mytextsc{clf}~: \textcolor{darkblue}{\textbf{\ipa{kʰɯ˩}}} \textit{Voir~:} \hyperlink{q\string_h\{\string_Mzo\#\string_T1}{\textcolor{darkblue}{\textbf{\ipa{qʰæ˧zo\#˥}}}} 

\paragraph{\hspace{-0.5cm} {\Large \textcolor{darkblue}{\textbf{\ipa{qʰæ˧mi˧}}}}} \hypertarget{q\string_h\{\string_Mmi\string_M1}{}
\markboth{\textcolor{darkblue}{\textbf{\ipa{qʰæ˧mi˧}}}}{}
\textcolor{teal}{\mytextsc{nom}} \hspace{4pt} Ton~: M.

Grand canal.
\textcolor{brown}{\zh{大水渠。}}


 \mytextsc{clf}~: \textcolor{darkblue}{\textbf{\ipa{kʰɯ˩}}} objets longs

\paragraph{\hspace{-0.5cm} {\Large \textcolor{darkblue}{\textbf{\ipa{qʰæ˧mo˩}}}}} \hypertarget{q\string_h\{\string_Mmo\string_B1}{}
\markboth{\textcolor{darkblue}{\textbf{\ipa{qʰæ˧mo˩}}}}{}
\textcolor{teal}{\mytextsc{nom}} \hspace{4pt} Ton~: L\#.

Un champignon vénéneux.
\textcolor{brown}{\zh{有毒的一种菌子。}}




\paragraph{\hspace{-0.5cm} {\Large \textcolor{darkblue}{\textbf{\ipa{qʰæ˧tv̩˧}}}}} \hypertarget{q\string_h\{\string_Mtv\string_=\string_M1}{}
\markboth{\textcolor{darkblue}{\textbf{\ipa{qʰæ˧tv̩˧}}}}{}
\textcolor{teal}{\mytextsc{nom}} \hspace{4pt} Ton~: M.

Anus.
\textcolor{brown}{\zh{肛门。}}


 \mytextsc{clf}~: \textcolor{darkblue}{\textbf{\ipa{ɭɯ˧}}} 

\paragraph{\hspace{-0.5cm} {\Large \textcolor{darkblue}{\textbf{\ipa{qʰæ˧tɕʰi˧}}}}} \hypertarget{q\string_h\{\string_Mts£\string_hi\string_M1}{}
\markboth{\textcolor{darkblue}{\textbf{\ipa{qʰæ˧tɕʰi˧}}}}{}
\textcolor{teal}{\mytextsc{nom}} \hspace{4pt} Ton~: M.

Un village de Yongning; nom chinois: Kaiji.
\textcolor{brown}{\zh{开基(永宁的一个村落)。}}


\begin{exe}
\sn \textcolor{darkblue}{\textbf{\ipa{ʈʂʰɯ˧ | qʰæ˧tɕʰi˧-hĩ˧ ɲi˥!}}}
\trans c'est quelqu'un de Kaiji!
\trans \textit{\textcolor{brown}{\zh{他是开基村人!}}}
\end{exe}
\begin{exe}
\sn \textcolor{darkblue}{\textbf{\ipa{dʑɤ˩bv̩˧kɤ˧-sɑ˥ʁwɤ˩, | hi˩ʁwɤ˩-lo˥, | æ˩mi˧-ʁwɤ\#˥, | lɑ˧lo˧-ʁwɤ˥, | lɑ˧ŋwɤ˧, | bɤ˧tsʰo˧gv̩˥, | ə˧lɑ˧-ʁwɤ\#˥, | gæ˧ɻæ˩, | qʰæ˧tɕʰi˧, | tʰo˧ʈɯ\#˥}}}
\trans les dix villages comptant traditionnellement comme faisant partie de Yongning
\trans \textit{\textcolor{brown}{\zh{摩梭传统地理概念中,属于永宁的十个村落}}}
\end{exe}


\paragraph{\hspace{-0.5cm} {\Large \textcolor{darkblue}{\textbf{\ipa{qʰæ˧tɕʰi˧-ɬi˧dʑɯ˩}}}}} \hypertarget{q\string_h\{\string_Mts£\string_hi\string_M-Ki\string_Mdz£M\string_B1}{}
\markboth{\textcolor{darkblue}{\textbf{\ipa{qʰæ˧tɕʰi˧-ɬi˧dʑɯ˩}}}}{}
\textcolor{teal}{\mytextsc{nom}} \hspace{4pt} Ton~: \mytextsc{L}\#.

Nom de la rivière qui traverse la plaine de Yongning.
\textcolor{brown}{\zh{永宁坝的河流。}}




\paragraph{\hspace{-0.5cm} {\Large \textcolor{darkblue}{\textbf{\ipa{qʰæ˧ʈæ˧˥}}}}} \hypertarget{q\string_h\{\string_Mt`\{\string_M\string_T1}{}
\markboth{\textcolor{darkblue}{\textbf{\ipa{qʰæ˧ʈæ˧˥}}}}{}
\textcolor{teal}{\mytextsc{adjectif}} \hspace{4pt} Ton~: MH\#.
\textit{De:} \textbf{qʰæ˩a 1} 
Paisible, tranquille (personnalité, trait de caractère).
\textcolor{brown}{\zh{安静。}}


\begin{exe}
\sn \textcolor{darkblue}{\textbf{\ipa{qʰæ˧ʈæ˧˥ | tʰi˧-dzi˩}}}
\trans être assis tranquillement, être tranquille, avoir l'esprit libre
\trans \textit{\textcolor{brown}{\zh{安静地坐着}}}
\end{exe}
\begin{exe}
\sn \textcolor{darkblue}{\textbf{\ipa{qʰæ˧ʈæ˧˥ | tʰi˧-ʝi˧}}}
\trans travailler paisiblement
\trans \textit{\textcolor{brown}{\zh{安静地工作}}}
\end{exe}


\paragraph{\hspace{-0.5cm} {\Large \textcolor{darkblue}{\textbf{\ipa{qʰæ˧zo\#˥}}}}} \hypertarget{q\string_h\{\string_Mzo\#\string_T1}{}
\markboth{\textcolor{darkblue}{\textbf{\ipa{qʰæ˧zo\#˥}}}}{}
\textcolor{teal}{\mytextsc{nom}} \hspace{4pt} Ton~: \#H.

Petit canal.
\textcolor{brown}{\zh{小水渠。}}


 \mytextsc{clf}~: \textcolor{darkblue}{\textbf{\ipa{kʰɯ˩}}} objets longs\textit{Voir~:} \hyperlink{q\string_h\{\string_Mlo\string_M\string_T1}{\textcolor{darkblue}{\textbf{\ipa{qʰæ˧lo˧˥}}}} 

\paragraph{\hspace{-0.5cm} {\Large \textcolor{darkblue}{\textbf{\ipa{qʰæ˩}}}}} \hypertarget{q\string_h\{\string_B1}{}
\markboth{\textcolor{darkblue}{\textbf{\ipa{qʰæ˩}}}}{}
\textcolor{teal}{\mytextsc{verbe}} \hspace{4pt} Ton~: L\textsubscript{a}.

Casser; ex.: casser une branche, récolter un épi de maïs en le cassant du plant de maïs.
\textcolor{brown}{\zh{折断。}}


\begin{exe}
\sn \textcolor{darkblue}{\textbf{\ipa{qʰɑ˧dze˧ qʰæ˩}}}
\trans cueillir du maïs
\trans \textit{\textcolor{brown}{\zh{采玉米}}}
\end{exe}
\begin{exe}
\sn \textcolor{darkblue}{\textbf{\ipa{qʰɑ˧dze˧ | le˧-qʰæ˩-ze˩}}}
\trans le maïs est cueilli
\trans \textit{\textcolor{brown}{\zh{玉米收好了。}}}
\end{exe}
\begin{exe}
\sn \textcolor{darkblue}{\textbf{\ipa{qʰɑ˧dze˧ | ɖɯ˧-qʰæ˧\textasciitilde{}qʰæ˥-ɻ̍˩}}}
\trans cueillir un peu de maïs
\trans \textit{\textcolor{brown}{\zh{去采些玉米}}}
\end{exe}


\paragraph{\hspace{-0.5cm} {\Large \textcolor{darkblue}{\textbf{\ipa{qʰæ˩\textsubscript{a}}}} \textsubscript{1}}} \hypertarget{q\string_h\{\string_Ba1}{}
\markboth{\textcolor{darkblue}{\textbf{\ipa{qʰæ˩\textsubscript{a}}}} \textsubscript{1}}{}
\textcolor{teal}{\mytextsc{adjectif}} \hspace{4pt} Ton~: L\textsubscript{a}.

Faux, mensonger.
\textcolor{brown}{\zh{假。}}


\begin{exe}
\sn \textcolor{darkblue}{\textbf{\ipa{qʰæ˩-hĩ˩˥, | tʰɑ˧-ʐwɤ˩!}}}
\trans ne dis pas de mensonges!
\trans \textit{\textcolor{brown}{\zh{假话,不要说! =不要撒谎!}}}
\end{exe}
\begin{exe}
\sn \textcolor{darkblue}{\textbf{\ipa{qʰæ˧ ʐwɤ˧}}}
\trans dire des mensonges
\trans \textit{\textcolor{brown}{\zh{撒谎、说谎}}}
\end{exe}


\paragraph{\hspace{-0.5cm} {\Large \textcolor{darkblue}{\textbf{\ipa{qʰæ˩\textsubscript{a}}}} \textsubscript{2}}} \hypertarget{q\string_h\{\string_Ba2}{}
\markboth{\textcolor{darkblue}{\textbf{\ipa{qʰæ˩\textsubscript{a}}}} \textsubscript{2}}{}
\textcolor{teal}{\mytextsc{adjectif}} \hspace{4pt} Ton~: L\textsubscript{a}.
\ding{202} 
En paix, tranquille, paisible (époque); en bonne santé (corps).
\textcolor{brown}{\zh{平静、安静,安乐、(身体)健康。}}


\begin{exe}
\sn \textcolor{darkblue}{\textbf{\ipa{hĩ˧ | ə˩-qʰæ˩˥?}}}
\trans est-ce que ça va bien?/tu vas bien? (Formule équivalente du \textcolor{darkblue}{\textbf{\ipa{/ə˥-lɑ˧~lɑ˩/}}} de Lijiang 'Est-ce que vous êtes en bonne santé?', qui à Yongning évoque malencontreusement le rédupliqué \textcolor{darkblue}{\textbf{\ipa{/ə˩-lɑ˩~lɑ˧˥/}}} 'est-ce que [vous] vous disputez?')
\trans \textit{\textcolor{brown}{\zh{你好吗? / 一切好吗?}}}
\end{exe}
\begin{exe}
\sn \textcolor{darkblue}{\textbf{\ipa{njɤ˧ | mɤ˧-qʰæ˩.}}}
\trans je ne me sens pas bien.
\trans \textit{\textcolor{brown}{\zh{我不舒服。}}}
\end{exe}
\ding{203} 
Léger, peu fatigant (travail).
\textcolor{brown}{\zh{轻松。}}


\begin{exe}
\sn \textcolor{darkblue}{\textbf{\ipa{qʰæ˩-hĩ˩˥}}}
\trans \mytextsc{rel}
\trans \textit{\textcolor{brown}{\zh{轻松的}}}
\end{exe}


\paragraph{\hspace{-0.5cm} {\Large \textcolor{darkblue}{\textbf{\ipa{qʰæ˩bæ˩}}}}} \hypertarget{q\string_h\{\string_Bb\{\string_B1}{}
\markboth{\textcolor{darkblue}{\textbf{\ipa{qʰæ˩bæ˩}}}}{}
\textcolor{teal}{\mytextsc{nom}} \hspace{4pt} Ton~: L.

Cuillère de petite taille: pour le sel, le tsamba... Elle correspond aux cuillères à café et cuillères à soupe du paradigme européen.
\textcolor{brown}{\zh{调羹。}}


 \mytextsc{clf}~: \textcolor{darkblue}{\textbf{\ipa{nɑ˧}}} 

\paragraph{\hspace{-0.5cm} {\Large \textcolor{darkblue}{\textbf{\ipa{qʰæ˩ʈv̩˩ɻæ˥}}}}} \hypertarget{q\string_h\{\string_Bt`v\string_=\string_Br£`\{\string_T1}{}
\markboth{\textcolor{darkblue}{\textbf{\ipa{qʰæ˩ʈv̩˩ɻæ˥}}}}{}
\textcolor{teal}{\mytextsc{adjectif}} \hspace{4pt} Ton~: L+H\#.

Serein.
\textcolor{brown}{\zh{安宁。}}


\begin{exe}
\sn \textcolor{darkblue}{\textbf{\ipa{qʰæ˩ʈv̩˩ɻæ˥ | ɖɯ˧-dzi˩-ɻ̍˩}}}
\trans être assis tranquille, dans le calme
\trans \textit{\textcolor{brown}{\zh{安静地坐一会}}}
\end{exe}
\begin{exe}
\sn \textcolor{darkblue}{\textbf{\ipa{qʰæ˩ʈv̩˩ɻæ˥-gv̩˩}}}
\trans tranquillement
\trans \textit{\textcolor{brown}{\zh{安宁地}}}
\end{exe}


\paragraph{\hspace{-0.5cm} {\Large \textcolor{darkblue}{\textbf{\ipa{qʰæ˧˥}}} \textsubscript{1}}} \hypertarget{q\string_h\{\string_M\string_T1}{}
\markboth{\textcolor{darkblue}{\textbf{\ipa{qʰæ˧˥}}} \textsubscript{1}}{}
\textcolor{teal}{\mytextsc{verbe}} \hspace{4pt} Ton~: MH.

Paraître, se lever (lune, soleil).
\textcolor{brown}{\zh{出来(月亮,太阳)。}}


\begin{exe}
\sn \textcolor{darkblue}{\textbf{\ipa{tʰi˧-qʰæ˧-ze˥}}}
\trans \mytextsc{dur} \string_ \mytextsc{pfv}
\trans \textit{\textcolor{brown}{\zh{\mytextsc{dur} \string_ \mytextsc{pfv}}}}
\end{exe}


\paragraph{\hspace{-0.5cm} {\Large \textcolor{darkblue}{\textbf{\ipa{qʰæ˧˥}}} \textsubscript{2}}} \hypertarget{q\string_h\{\string_M\string_T2}{}
\markboth{\textcolor{darkblue}{\textbf{\ipa{qʰæ˧˥}}} \textsubscript{2}}{}
\textcolor{teal}{\mytextsc{verbe}} \hspace{4pt} Ton~: MH.

Démolir.
\textcolor{brown}{\zh{拆。}}


\begin{exe}
\sn \textcolor{darkblue}{\textbf{\ipa{ʑi˧qʰwɤ˧ qʰæ˧˥}}}
\trans démolir une maison
\trans \textit{\textcolor{brown}{\zh{拆房子}}}
\end{exe}


\paragraph{\hspace{-0.5cm} {\Large \textcolor{darkblue}{\textbf{\ipa{qʰæ˧˥}}} \textsubscript{3}}} \hypertarget{q\string_h\{\string_M\string_T3}{}
\markboth{\textcolor{darkblue}{\textbf{\ipa{qʰæ˧˥}}} \textsubscript{3}}{}
\textcolor{teal}{\mytextsc{verbe}} \hspace{4pt} Ton~: MH.

Partager, répartir.
\textcolor{brown}{\zh{分东西、(大家)平分东西。}}




\paragraph{\hspace{-0.5cm} {\Large \textcolor{darkblue}{\textbf{\ipa{qʰæ˧˥}}} \textsubscript{4}}} \hypertarget{q\string_h\{\string_M\string_T4}{}
\markboth{\textcolor{darkblue}{\textbf{\ipa{qʰæ˧˥}}} \textsubscript{4}}{}
\textcolor{teal}{\mytextsc{verbe}} \hspace{4pt} Ton~: MH.

Tirer (avec une arme à feu, une arbalète; aussi avec un arc: tirer une flèche).
\textcolor{brown}{\zh{开枪。}}


\begin{exe}
\sn \textcolor{darkblue}{\textbf{\ipa{le˧-qʰæ˧-ze˥}}}
\trans \mytextsc{accomp} \string_ \mytextsc{pfv}
\trans \textit{\textcolor{brown}{\zh{开枪了}}}
\end{exe}
\begin{exe}
\sn \textcolor{darkblue}{\textbf{\ipa{mv̩˧ʐe˧ qʰæ˩(-ze˩)}}}
\trans tirer avec une arme à feu
\trans \textit{\textcolor{brown}{\zh{开枪}}}
\end{exe}


\paragraph{\hspace{-0.5cm} {\Large \textcolor{darkblue}{\textbf{\ipa{qʰæ˧˥}}} \textsubscript{5}}} \hypertarget{q\string_h\{\string_M\string_T5}{}
\markboth{\textcolor{darkblue}{\textbf{\ipa{qʰæ˧˥}}} \textsubscript{5}}{}
\textcolor{teal}{\mytextsc{adjectif}} \hspace{4pt} Ton~: MH.

Heureux.
\textcolor{brown}{\zh{幸福,安逸,平安。}}


\begin{exe}
\sn \textcolor{darkblue}{\textbf{\ipa{le˧-qʰæ˧-ze˥}}}
\trans \mytextsc{accomp} \string_ \mytextsc{pfv}
\trans \textit{\textcolor{brown}{\zh{\mytextsc{accomp} \string_ \mytextsc{pfv}}}}
\end{exe}
\begin{exe}
\sn \textcolor{darkblue}{\textbf{\ipa{lo˧ qʰæ˩}}}
\trans travailler de façon tranquille, détendue, paisible
\trans \textit{\textcolor{brown}{\zh{轻松工作}}}
\end{exe}


\paragraph{\hspace{-0.5cm} {\Large \textcolor{darkblue}{\textbf{\ipa{qʰæ˧˥}}} \textsubscript{6}}} \hypertarget{q\string_h\{\string_M\string_T6}{}
\markboth{\textcolor{darkblue}{\textbf{\ipa{qʰæ˧˥}}} \textsubscript{6}}{}
\textcolor{teal}{\mytextsc{verbe}} \hspace{4pt} Ton~: MH.

Brûler, griller, noircir: la graisse chauffée à feu vif noircit, fume et donne de l'acroléine; le riz devient sec, très/trop cuit. La nourriture reste comestible.
\textcolor{brown}{\zh{糊、变黑(高温让油、食物变黑,变糊了)。}}


\begin{exe}
\sn \textcolor{darkblue}{\textbf{\ipa{le˧-qʰæ˧-ze˥}}}
\trans \mytextsc{accomp} \string_ \mytextsc{pfv}
\trans \textit{\textcolor{brown}{\zh{\mytextsc{accomp} \string_ \mytextsc{pfv}}}}
\end{exe}
\begin{exe}
\sn \textcolor{darkblue}{\textbf{\ipa{mɤ˧ | le˧-qʰæ˧-ze˥}}}
\trans L'huile a noirci / l'huile est parvenue à une très haute température.
\trans \textit{\textcolor{brown}{\zh{油焦了!}}}
\end{exe}
\begin{exe}
\sn \textcolor{darkblue}{\textbf{\ipa{hɑ˧ | le˧-qʰæ˧-ze˥}}}
\trans Le riz a brûlé. / Le riz est trop cuit.
\trans \textit{\textcolor{brown}{\zh{饭糊了。}}}
\end{exe}
\begin{exe}
\sn \textcolor{darkblue}{\textbf{\ipa{v̩˩tsʰɤ˩˥ | hṽ˧\textasciitilde{}hṽ˧ F | le˧-qʰæ˧-ze˥!}}}
\trans Les légumes, à force de frire, les voilà brûlés! / les voilà trop cuits!
\trans \textit{\textcolor{brown}{\zh{菜都炒糊了!}}}
\end{exe}
\begin{exe}
\sn \textcolor{darkblue}{\textbf{\ipa{ʂe˧ | hṽ˧\textasciitilde{}hṽ˧ F | le˧-qʰæ˧-ze˥!}}}
\trans La viande, à force de frire, la voilà brûlée! / la voilà trop cuite!
\trans \textit{\textcolor{brown}{\zh{肉都炒焦了!}}}
\end{exe}


\paragraph{\hspace{-0.5cm} {\Large \textcolor{darkblue}{\textbf{\ipa{qʰo˧\textsubscript{a}}}}}} \hypertarget{q\string_ho\string_Ma1}{}
\markboth{\textcolor{darkblue}{\textbf{\ipa{qʰo˧\textsubscript{a}}}}}{}
\textcolor{teal}{\mytextsc{verbe}} \hspace{4pt} Ton~: M\textsubscript{a}.

Empiler (par exemple des pierres).
\textcolor{brown}{\zh{堆起来。}}


\begin{exe}
\sn \textcolor{darkblue}{\textbf{\ipa{lv̩˧mi˧ tʰi˧-qʰo˧}}}
\trans empiler des pierres
\trans \textit{\textcolor{brown}{\zh{石头堆起来}}}
\end{exe}


\paragraph{\hspace{-0.5cm} {\Large \textcolor{darkblue}{\textbf{\ipa{qʰo˧lo˧}}}}} \hypertarget{q\string_ho\string_Mlo\string_M1}{}
\markboth{\textcolor{darkblue}{\textbf{\ipa{qʰo˧lo˧}}}}{}
\textcolor{teal}{\mytextsc{nom}} \hspace{4pt} Ton~: M.

Roue.
\textcolor{brown}{\zh{轮子。}}


 \mytextsc{clf}~: \textcolor{darkblue}{\textbf{\ipa{ɭɯ˧}}} 

\paragraph{\hspace{-0.5cm} {\Large \textcolor{darkblue}{\textbf{\ipa{qʰo˧mo˥}}}}} \hypertarget{q\string_ho\string_Mmo\string_T1}{}
\markboth{\textcolor{darkblue}{\textbf{\ipa{qʰo˧mo˥}}}}{}
\textcolor{teal}{\mytextsc{nom}} \hspace{4pt} Ton~: H\#.

Vieille vache (qui n'a plus de lait).
\textcolor{brown}{\zh{老牛(不产奶了)。}}


 \mytextsc{clf}~: \textcolor{darkblue}{\textbf{\ipa{pʰo˧˥}}} 

\paragraph{\hspace{-0.5cm} {\Large \textcolor{darkblue}{\textbf{\ipa{qʰo˩\textsubscript{b}}}}}} \hypertarget{q\string_ho\string_Bb1}{}
\markboth{\textcolor{darkblue}{\textbf{\ipa{qʰo˩\textsubscript{b}}}}}{}
\textcolor{teal}{\mytextsc{verbe}} \hspace{4pt} Ton~: L\textsubscript{b}.

Inviter.
\textcolor{brown}{\zh{邀请、请。}}


\begin{exe}
\sn \textcolor{darkblue}{\textbf{\ipa{hĩ˧ qʰo˧˥}}}
\trans inviter quelqu'un
\trans \textit{\textcolor{brown}{\zh{邀请人}}}
\end{exe}
\begin{exe}
\sn \textcolor{darkblue}{\textbf{\ipa{hĩ˧bæ˧ qʰo˧˥}}}
\trans inviter un hôte, convier un invité
\trans \textit{\textcolor{brown}{\zh{邀请客人}}}
\end{exe}
\begin{exe}
\sn \textcolor{darkblue}{\textbf{\ipa{hĩ˧bæ˧ | qʰo˧-zo˧-ho˥}}}
\trans Il va falloir inviter des hôtes!
\trans \textit{\textcolor{brown}{\zh{需要请一下客人!}}}
\end{exe}
\begin{exe}
\sn \textcolor{darkblue}{\textbf{\ipa{hĩ˧bæ˧ qʰo˧-di˧˥}}}
\trans Euphémisme pour désigner la mort-aux-rats. La croyance veut que si on expose clairement le projet, les rats vont se méfier et ne prendront pas cette nourriture empoisonnée.
\trans \textit{\textcolor{brown}{\zh{‘待客的东西’(老鼠药的委婉语。如果说出来要买老鼠药,老鼠会知道,就不会吃的。)}}}
\end{exe}
\begin{exe}
\sn \textcolor{darkblue}{\textbf{\ipa{ɖɯ˧-qʰo˥\textasciitilde{}qʰo˩-ɻ̍˩}}}
\trans \mytextsc{délimitative} \mytextsc{red} \mytextsc{inchoatif}
\trans \textit{\textcolor{brown}{\zh{\mytextsc{delimitative} \mytextsc{red} \mytextsc{inceptive:请一下}}}}
\end{exe}
\begin{exe}
\sn \textcolor{darkblue}{\textbf{\ipa{qʰo˩-mɤ˥-qʰo˩}}}
\trans inviter ou ne pas inviter
\trans \textit{\textcolor{brown}{\zh{请不请}}}
\end{exe}
\begin{exe}
\sn \textcolor{darkblue}{\textbf{\ipa{qʰo˩-mɤ˩-ho˥}}}
\trans ... ne vais pas inviter
\trans \textit{\textcolor{brown}{\zh{不请了 / 不要请了}}}
\end{exe}


\paragraph{\hspace{-0.5cm} {\Large \textcolor{darkblue}{\textbf{\ipa{qʰo˩dv̩˧˥}}}}} \hypertarget{q\string_ho\string_Bdv\string_=\string_M\string_T1}{}
\markboth{\textcolor{darkblue}{\textbf{\ipa{qʰo˩dv̩˧˥}}}}{}
\textcolor{teal}{\mytextsc{nom}} \hspace{4pt} Ton~: LM+MH\#.

Masse, marteau de grande taille; typiquement: masse en bois utilisée pour défoncer les grosses mottes après les labours.
\textcolor{brown}{\zh{大锤子。}}


\begin{exe}
\sn \textcolor{darkblue}{\textbf{\ipa{ʂe˩-qʰo˩dv̩˧˥}}}
\trans masse en fer
\trans \textit{\textcolor{brown}{\zh{铁锤子}}}
\end{exe}
 \mytextsc{clf}~: \textcolor{darkblue}{\textbf{\ipa{ɭɯ˧}}} 

\paragraph{\hspace{-0.5cm} {\Large \textcolor{darkblue}{\textbf{\ipa{qʰo˩mv̩˩}}}}} \hypertarget{q\string_ho\string_Bmv\string_=\string_B1}{}
\markboth{\textcolor{darkblue}{\textbf{\ipa{qʰo˩mv̩˩}}}}{}
\textcolor{teal}{\mytextsc{nom}} \hspace{4pt} Ton~: L.

Chapeau de paille.
\textcolor{brown}{\zh{斗笠。}}


 \mytextsc{clf}~: \textcolor{darkblue}{\textbf{\ipa{ɭɯ˧}}} 

\paragraph{\hspace{-0.5cm} {\Large \textcolor{darkblue}{\textbf{\ipa{qʰo˩tv̩˧˥}}}}} \hypertarget{q\string_ho\string_Btv\string_=\string_M\string_T1}{}
\markboth{\textcolor{darkblue}{\textbf{\ipa{qʰo˩tv̩˧˥}}}}{}
\textcolor{teal}{\mytextsc{nom}} \hspace{4pt} Ton~: LM+MH\#.

Souche.
\textcolor{brown}{\zh{树墩、树桩。}}


 \mytextsc{clf}~: \textcolor{darkblue}{\textbf{\ipa{ɭɯ˧}}} 

\paragraph{\hspace{-0.5cm} {\Large \textcolor{darkblue}{\textbf{\ipa{qʰo˧˥}}} \textsubscript{1}}} \hypertarget{q\string_ho\string_M\string_T1}{}
\markboth{\textcolor{darkblue}{\textbf{\ipa{qʰo˧˥}}} \textsubscript{1}}{}
\textcolor{teal}{\mytextsc{verbe}} \hspace{4pt} Ton~: MH.

Picorer.
\textcolor{brown}{\zh{啄。}}


\begin{exe}
\sn \textcolor{darkblue}{\textbf{\ipa{hɑ˧ qʰo˩(-ze˩)}}}
\trans picorer des céréales
\trans \textit{\textcolor{brown}{\zh{啄粮食}}}
\end{exe}
\begin{exe}
\sn \textcolor{darkblue}{\textbf{\ipa{hɑ˧ qʰo˥\textasciitilde{}qʰo˩ (-dʑo˩)}}}
\trans picorer des céréales
\trans \textit{\textcolor{brown}{\zh{啄粮食}}}
\end{exe}
\begin{exe}
\sn \textcolor{darkblue}{\textbf{\ipa{æ˩-ɳɯ˥ | hɑ˧ qʰo˩}}}
\trans la poule picore
\trans \textit{\textcolor{brown}{\zh{鸡在啄粮食}}}
\end{exe}


\paragraph{\hspace{-0.5cm} {\Large \textcolor{darkblue}{\textbf{\ipa{qʰo˧˥}}} \textsubscript{2}}} \hypertarget{q\string_ho\string_M\string_T2}{}
\markboth{\textcolor{darkblue}{\textbf{\ipa{qʰo˧˥}}} \textsubscript{2}}{}
\textcolor{teal}{\mytextsc{verbe}} \hspace{4pt} Ton~: MH.

Tuer; abattre un animal.
\textcolor{brown}{\zh{杀,宰牲畜。}}


\begin{exe}
\sn \textcolor{darkblue}{\textbf{\ipa{bo˩ qʰo˧˥ / bo˩ qʰo˧-ze˥}}}
\trans tuer le cochon
\trans \textit{\textcolor{brown}{\zh{杀猪}}}
\end{exe}
\begin{exe}
\sn \textcolor{darkblue}{\textbf{\ipa{bo˩˥ | le˧-qʰo˧-ze˥}}}
\trans le cochon a été abattu
\trans \textit{\textcolor{brown}{\zh{杀了猪}}}
\end{exe}
\begin{exe}
\sn \textcolor{darkblue}{\textbf{\ipa{æ˩ qʰo˧˥}}}
\trans tuer un poulet
\trans \textit{\textcolor{brown}{\zh{杀鸡}}}
\end{exe}
\begin{exe}
\sn \textcolor{darkblue}{\textbf{\ipa{ʝi˧ qʰo˩}}}
\trans tuer une vache
\trans \textit{\textcolor{brown}{\zh{杀牛}}}
\end{exe}


\paragraph{\hspace{-0.5cm} {\Large \textcolor{darkblue}{\textbf{\ipa{qʰv̩˧}}} \textsubscript{1}}} \hypertarget{q\string_hv\string_=\string_M1}{}
\markboth{\textcolor{darkblue}{\textbf{\ipa{qʰv̩˧}}} \textsubscript{1}}{}
\textcolor{teal}{\mytextsc{nom}} \hspace{4pt} Ton~: M.
\ding{202} 
Trou.
\textcolor{brown}{\zh{洞。}}


\ding{203} 
Terrier.
\textcolor{brown}{\zh{野兽的洞穴、野兽的窝。}}


\begin{exe}
\sn \textcolor{darkblue}{\textbf{\ipa{ɖɤ˧-qʰv̩˧}}}
\trans terrier de renard
\trans \textit{\textcolor{brown}{\zh{狐狸的窝}}}
\trans \textit{\textcolor{brown}{\zh{tone: M}}}
\end{exe}
\begin{exe}
\sn \textcolor{darkblue}{\textbf{\ipa{ʂwæ˧ qʰv̩˧}}}
\trans terrier de loutre
\trans \textit{\textcolor{brown}{\zh{水獭的窝}}}
\trans \textit{\textcolor{brown}{\zh{tone: M}}}
\end{exe}
 \mytextsc{clf}~: \textcolor{darkblue}{\textbf{\ipa{ɭɯ˧}}} 

\paragraph{\hspace{-0.5cm} {\Large \textcolor{darkblue}{\textbf{\ipa{qʰv̩˧}}} \textsubscript{2}}} \hypertarget{q\string_hv\string_=\string_M2}{}
\markboth{\textcolor{darkblue}{\textbf{\ipa{qʰv̩˧}}} \textsubscript{2}}{}
\textcolor{teal}{\mytextsc{nom}} \hspace{4pt} Ton~: M.

Corne.
\textcolor{brown}{\zh{犄角(锯下来的)。}}


\begin{exe}
\sn \textcolor{darkblue}{\textbf{\ipa{ʝi˧-qʰv̩\#˥}}}
\trans Corne de boeuf. La corne de boeuf était autrefois utilisée comme récipient pour boissons.
\trans \textit{\textcolor{brown}{\zh{牛角(过去,用牛角来当饮料容器)}}}
\end{exe}
\begin{exe}
\sn \textcolor{darkblue}{\textbf{\ipa{ʈʂʰæ˧-qʰv̩˥}}}
\trans corne de cerf
\trans \textit{\textcolor{brown}{\zh{鹿角}}}
\end{exe}
 \mytextsc{clf}~: \textcolor{darkblue}{\textbf{\ipa{ɭɯ˧}}} \textcolor{darkblue}{\textbf{\ipa{dze˩}}} paire

\paragraph{\hspace{-0.5cm} {\Large \textcolor{darkblue}{\textbf{\ipa{qʰv̩˧˥}}} \textsubscript{1}}} \hypertarget{q\string_hv\string_=\string_M\string_T1}{}
\markboth{\textcolor{darkblue}{\textbf{\ipa{qʰv̩˧˥}}} \textsubscript{1}}{}
\textcolor{teal}{\mytextsc{verbe}} \hspace{4pt} Ton~: MH.

Se recroqueviller.
\textcolor{brown}{\zh{蜷缩。}}


\begin{exe}
\sn \textcolor{darkblue}{\textbf{\ipa{ɲi˧-qʰv̩˧˥ | tʰi˧-dzi˩}}}
\trans être assis penché en avant, le torse penché vers les cuisses
\trans \textit{\textcolor{brown}{\zh{坐着身体缩成一团}}}
\end{exe}
\begin{exe}
\sn \textcolor{darkblue}{\textbf{\ipa{[M23] ɲi˧-qʰv̩˧-ʝi˥ | tʰi˧-dzi˩}}}
\trans être assis penché en avant, le torse penché vers les cuisses
\trans \textit{\textcolor{brown}{\zh{坐着身体缩成一团}}}
\end{exe}


\paragraph{\hspace{-0.5cm} {\Large \textcolor{darkblue}{\textbf{\ipa{qʰv̩˧˥}}} \textsubscript{2}}} \hypertarget{q\string_hv\string_=\string_M\string_T2}{}
\markboth{\textcolor{darkblue}{\textbf{\ipa{qʰv̩˧˥}}} \textsubscript{2}}{}
\textcolor{teal}{\mytextsc{nombre}} \hspace{4pt} Ton~: MH.

6.
\textcolor{brown}{\zh{6。}}




\paragraph{\hspace{-0.5cm} {\Large \textcolor{darkblue}{\textbf{\ipa{qʰv̩˥}}}}} \hypertarget{q\string_hv\string_=\string_T1}{}
\markboth{\textcolor{darkblue}{\textbf{\ipa{qʰv̩˥}}}}{}
\textcolor{teal}{\mytextsc{nom}} \hspace{4pt} Ton~: \#H.

Son, bruit.
\textcolor{brown}{\zh{声音。}}


\begin{exe}
\sn \textcolor{darkblue}{\textbf{\ipa{ʈʂʰɯ˧ | ə˧tso˧ qʰv̩˧ ɲi˥?}}}
\trans c'est quoi ce bruit?
\trans \textit{\textcolor{brown}{\zh{这是什么声音?}}}
\end{exe}
 \mytextsc{clf}~: \textcolor{darkblue}{\textbf{\ipa{kʰwɤ˥}}} 

\paragraph{\hspace{-0.5cm} {\Large \textcolor{darkblue}{\textbf{\ipa{qʰv̩˥\textsubscript{a}}}}}} \hypertarget{q\string_hv\string_=\string_Ta1}{}
\markboth{\textcolor{darkblue}{\textbf{\ipa{qʰv̩˥\textsubscript{a}}}}}{}
\textcolor{teal}{\mytextsc{classificateur}} \hspace{4pt} Ton~: H\textsubscript{a}.

Classificateur des hameaux.
\textcolor{brown}{\zh{量词:村落。}}


\begin{exe}
\sn \textcolor{darkblue}{\textbf{\ipa{ŋwɤ˧-qʰv̩˧, | tsʰe˧ɲi˧-ʑi˩}}}
\trans Cinq hameaux, douze familles! (Formule résumant la statistique du village de /ə˧lɑ˧-ʁwɤ\#˥/)
\trans \textit{\textcolor{brown}{\zh{五个村落,十二个家庭!(描写阿拉瓦村的情况)}}}
\end{exe}


\paragraph{\hspace{-0.5cm} {\Large \textcolor{darkblue}{\textbf{\ipa{qʰv̩˩ɖɯ˩}}}}} \hypertarget{q\string_hv\string_=\string_Bd`M\string_B1}{}
\markboth{\textcolor{darkblue}{\textbf{\ipa{qʰv̩˩ɖɯ˩}}}}{}
\textcolor{teal}{\mytextsc{nom}} \hspace{4pt} Ton~: L.

Attachement envers quelqu'un; observé seulement dans l'expression ``être attaché à, faire cas de, attacher du prix à".
\textcolor{brown}{\zh{关心。}}


\begin{exe}
\sn \textcolor{darkblue}{\textbf{\ipa{qʰv̩˩ɖɯ˩ pʰv̩˥}}}
\trans attacher de l'importance à, respecter, être attaché à (ex.: relation des enfants à leurs parents)
\trans \textit{\textcolor{brown}{\zh{关心(一个人),重视(如:孩子重视父母)}}}
\end{exe}


\paragraph{\hspace{-0.5cm} {\Large \textcolor{darkblue}{\textbf{\ipa{qʰv̩˩ɖʐæ˩}}}}} \hypertarget{q\string_hv\string_=\string_Bd`z`\{\string_B1}{}
\markboth{\textcolor{darkblue}{\textbf{\ipa{qʰv̩˩ɖʐæ˩}}}}{}
\textcolor{teal}{\mytextsc{nom}} \hspace{4pt} Ton~: L.

Cordelette, ficelle.
\textcolor{brown}{\zh{小绳子,细的绳子。}}


\begin{exe}
\sn \textcolor{darkblue}{\textbf{\ipa{qʰv̩˩ɖʐæ˩ ʈʂʰɯ˩-kʰɯ˥}}}
\trans \mytextsc{n}+\mytextsc{dem}+\mytextsc{clf}
\trans \textit{\textcolor{brown}{\zh{一条细的绳子}}}
\end{exe}
 \mytextsc{clf}~: \textcolor{darkblue}{\textbf{\ipa{kʰɯ˩}}} 

\paragraph{\hspace{-0.5cm} {\Large \textcolor{darkblue}{\textbf{\ipa{qʰv̩˧dʑɯ˥\$}}}}} \hypertarget{q\string_hv\string_=\string_Mdz£M\string_T\$1}{}
\markboth{\textcolor{darkblue}{\textbf{\ipa{qʰv̩˧dʑɯ˥\$}}}}{}
\textcolor{teal}{\mytextsc{nom}} \hspace{4pt} Ton~: H\$.

Cavité, trou (ex.: trou de souris; ou piège où on fait tomber les animaux sauvages).
\textcolor{brown}{\zh{窟窿。}}


\begin{exe}
\sn \textcolor{darkblue}{\textbf{\ipa{hwæ˧tsɯ˥-qʰv̩˩dʑi˩}}}
\trans trou de souris
\trans \textit{\textcolor{brown}{\zh{耗子洞}}}
\end{exe}
\begin{exe}
\sn \textcolor{darkblue}{\textbf{\ipa{qʰv̩˧dʑɯ˧ tsʰi˧ (-ze˩)}}}
\trans percer un trou
\trans \textit{\textcolor{brown}{\zh{挖一个洞}}}
\end{exe}
 \mytextsc{clf}~: \textcolor{darkblue}{\textbf{\ipa{ɭɯ˧}}} 

\paragraph{\hspace{-0.5cm} {\Large \textcolor{darkblue}{\textbf{\ipa{qʰv̩˧ɬi˧mi\#˥}}}}} \hypertarget{q\string_hv\string_=\string_MKi\string_Mmi\#\string_T1}{}
\markboth{\textcolor{darkblue}{\textbf{\ipa{qʰv̩˧ɬi˧mi\#˥}}}}{}
\textcolor{teal}{\mytextsc{nom}} \hspace{4pt} Ton~: \#H.

6e mois.
\textcolor{brown}{\zh{六月。}}




\paragraph{\hspace{-0.5cm} {\Large \textcolor{darkblue}{\textbf{\ipa{qʰv̩˩tsʰi˧˥}}}}} \hypertarget{q\string_hv\string_=\string_Bts\string_hi\string_M\string_T1}{}
\markboth{\textcolor{darkblue}{\textbf{\ipa{qʰv̩˩tsʰi˧˥}}}}{}
\textcolor{teal}{\mytextsc{nombre}} \hspace{4pt} Ton~: LM+MH\#.

60.
\textcolor{brown}{\zh{60。}}




\paragraph{\hspace{-0.5cm} {\Large \textcolor{darkblue}{\textbf{\ipa{qʰv̩˧tʰv˥\$}}}}} \hypertarget{q\string_hv\string_=\string_Mt\string_hv\string_T\$1}{}
\markboth{\textcolor{darkblue}{\textbf{\ipa{qʰv̩˧tʰv˥\$}}}}{}
\textcolor{teal}{\mytextsc{nom}} \hspace{4pt} Ton~: H\$.

Corne (de vache).
\textcolor{brown}{\zh{(牛)角。}}


\begin{exe}
\sn \textcolor{darkblue}{\textbf{\ipa{qʰv˧tʰv˥ | ɖɯ˧-ɭɯ˧}}}
\trans une corne
\trans \textit{\textcolor{brown}{\zh{一个角}}}
\end{exe}
\begin{exe}
\sn \textcolor{darkblue}{\textbf{\ipa{qʰv˧tʰv˧ ɲi˥}}}
\trans C'est une corne.
\trans \textit{\textcolor{brown}{\zh{是(牛)角。}}}
\end{exe}
 \mytextsc{clf}~: \textcolor{darkblue}{\textbf{\ipa{ɭɯ˧}}} \textit{Voir~:} \hyperlink{q\string_hv\string_=\string_Mt\string_hv\string_=\#\string_T1}{\textcolor{darkblue}{\textbf{\ipa{qʰv̩˧tʰv̩\#˥}}}} 

\paragraph{\hspace{-0.5cm} {\Large \textcolor{darkblue}{\textbf{\ipa{qʰv̩˧tʰv̩\#˥}}}}} \hypertarget{q\string_hv\string_=\string_Mt\string_hv\string_=\#\string_T1}{}
\markboth{\textcolor{darkblue}{\textbf{\ipa{qʰv̩˧tʰv̩\#˥}}}}{}
\textcolor{teal}{\mytextsc{classificateur}} \hspace{4pt} Ton~: \#H.

Classificateur: quantité de liquide (ou de poudre) que tient une corne de vache.
\textcolor{brown}{\zh{量词:一个牛角的容量。}}


\begin{exe}
\sn \textcolor{darkblue}{\textbf{\ipa{ɖɯ˧-qʰv̩˧tʰv̩\#˥, | ɲi˧-qʰv̩˧tʰv̩\#˥, | so˩-qʰv̩˩tʰv̩˩˥, | ʐv̩˧-qʰv̩˧tʰv̩\#˥, | ŋwɤ˧-qʰv̩˧tʰv̩\#˥, | qʰv̩˧-qʰv̩˧tʰv̩\#˥, | ʂɯ˧-qʰv̩˧tʰv̩\#˥, | hõ˧-qʰv̩˧tʰv̩\#˥, | gv̩˧-qʰv̩˧tʰv̩\#˥, | tsʰe˩-qʰv̩˩tʰv̩˩˥}}}
\trans association avec des numéraux, de 1 à 10
\trans \textit{\textcolor{brown}{\zh{与数词结合,一至十}}}
\end{exe}
\textit{Voir~:} \hyperlink{q\string_hv\string_=\string_Mt\string_hv\string_T\$1}{\textcolor{darkblue}{\textbf{\ipa{qʰv̩˧tʰv˥\$}}}} 

\paragraph{\hspace{-0.5cm} {\Large \textcolor{darkblue}{\textbf{\ipa{qʰv̩˩\textasciitilde{}qʰv̩˧˥}}}}} \hypertarget{q\string_hv\string_=\string_B~q\string_hv\string_=\string_M\string_T1}{}
\markboth{\textcolor{darkblue}{\textbf{\ipa{qʰv̩˩\textasciitilde{}qʰv̩˧˥}}}}{}
\textcolor{teal}{\mytextsc{verbe}} \hspace{4pt} Ton~: MH.

Plier (vêtements).
\textcolor{brown}{\zh{折叠、裹起来。}}


\begin{exe}
\sn \textcolor{darkblue}{\textbf{\ipa{qʰv̩˩\textasciitilde{}qʰv̩˧-ze˥}}}
\trans \mytextsc{pfv}
\trans \textit{\textcolor{brown}{\zh{折起来了}}}
\end{exe}
\begin{exe}
\sn \textcolor{darkblue}{\textbf{\ipa{le˧-qʰv̩˩\textasciitilde{}qʰv̩˩}}}
\trans \mytextsc{accomp}
\trans \textit{\textcolor{brown}{\zh{\mytextsc{accomp}}}}
\end{exe}


\paragraph{\hspace{-0.5cm} {\Large \textcolor{darkblue}{\textbf{\ipa{qʰwæ˧}}}}} \hypertarget{q\string_hw\{\string_M1}{}
\markboth{\textcolor{darkblue}{\textbf{\ipa{qʰwæ˧}}}}{}
\textcolor{teal}{\mytextsc{nom}} \hspace{4pt} Ton~: M.

Lettre, message, parole/récit.
\textcolor{brown}{\zh{信息,信。}}


\begin{exe}
\sn \textcolor{darkblue}{\textbf{\ipa{qʰwæ˧ po˧˥}}}
\trans envoyer une lettre
\trans \textit{\textcolor{brown}{\zh{带信息、传信息,传一封信}}}
\end{exe}
\begin{exe}
\sn \textcolor{darkblue}{\textbf{\ipa{qʰwæ˧ kʰwɤ˧˥}}}
\trans être en contact, être en correspondance; être en relation
\trans \textit{\textcolor{brown}{\zh{互相通信息、有联系(两个人互相通信息)}}}
\end{exe}
\begin{exe}
\sn \textcolor{darkblue}{\textbf{\ipa{dɑ˧pɤ˧-qʰwæ\#˥}}}
\trans les récits des prêtres \textcolor{darkblue}{\textbf{\ipa{/dɑ˧pɤ˧/}}}
\trans \textit{\textcolor{brown}{\zh{达巴的故事}}}
\end{exe}
 \mytextsc{clf}~: \textcolor{darkblue}{\textbf{\ipa{kʰwɤ˥}}} 

\paragraph{\hspace{-0.5cm} {\Large \textcolor{darkblue}{\textbf{\ipa{qʰwæ˧kʰwɤ\#˥}}}}} \hypertarget{q\string_hw\{\string_Mk\string_hw7\#\string_T1}{}
\markboth{\textcolor{darkblue}{\textbf{\ipa{qʰwæ˧kʰwɤ\#˥}}}}{}
\textcolor{teal}{\mytextsc{nom}} \hspace{4pt} Ton~: \#H.

Récit, racontar, ragot, histoire.
\textcolor{brown}{\zh{闲话、流言、蜚语、闲言碎语、八卦。}}


\begin{exe}
\sn \textcolor{darkblue}{\textbf{\ipa{ɖɯ˧-zɯ˧ qʰwæ˧kʰwɤ˧}}}
\trans raconter un petit racontar, rapporter un petit ragot
\trans \textit{\textcolor{brown}{\zh{讲一点八卦}}}
\end{exe}
 \mytextsc{clf}~: \textcolor{darkblue}{\textbf{\ipa{kʰwɤ˥}}} 

\paragraph{\hspace{-0.5cm} {\Large \textcolor{darkblue}{\textbf{\ipa{qʰwæ˧ɭɯ˧}}}}} \hypertarget{q\string_hw\{\string_Ml\string_RM\string_M1}{}
\markboth{\textcolor{darkblue}{\textbf{\ipa{qʰwæ˧ɭɯ˧}}}}{}
\textcolor{teal}{\mytextsc{nom}} \hspace{4pt} Ton~: M.

Potager.
\textcolor{brown}{\zh{菜园。}}


 \mytextsc{clf}~: \textcolor{darkblue}{\textbf{\ipa{kɤ˧˥}}} 

\paragraph{\hspace{-0.5cm} {\Large \textcolor{darkblue}{\textbf{\ipa{qʰwæ˧mi\#˥}}}}} \hypertarget{q\string_hw\{\string_Mmi\#\string_T1}{}
\markboth{\textcolor{darkblue}{\textbf{\ipa{qʰwæ˧mi\#˥}}}}{}
\textcolor{teal}{\mytextsc{nom}} \hspace{4pt} Ton~: \#H.

Message, information (d'où: lettre).
\textcolor{brown}{\zh{口信, 信息。}}


\begin{exe}
\sn \textcolor{darkblue}{\textbf{\ipa{qʰwæ˧mi˧ ʝi˧}}}
\trans porter un message
\trans \textit{\textcolor{brown}{\zh{带一个口信}}}
\end{exe}
 \mytextsc{clf}~: \textcolor{darkblue}{\textbf{\ipa{kʰwɤ˥}}} 

\paragraph{\hspace{-0.5cm} {\Large \textcolor{darkblue}{\textbf{\ipa{qʰwæ˧ʈɯ˥}}}}} \hypertarget{q\string_hw\{\string_Mt`M\string_T1}{}
\markboth{\textcolor{darkblue}{\textbf{\ipa{qʰwæ˧ʈɯ˥}}}}{}
\textcolor{teal}{\mytextsc{nom}} \hspace{4pt} Ton~: H\#.

Fichu (tissu qu'on porte sur la tête).
\textcolor{brown}{\zh{头帕。}}


 \mytextsc{clf}~: \textcolor{darkblue}{\textbf{\ipa{bɤ˧˥}}} 

\paragraph{\hspace{-0.5cm} {\Large \textcolor{darkblue}{\textbf{\ipa{qʰwæ˩}}}}} \hypertarget{q\string_hw\{\string_B1}{}
\markboth{\textcolor{darkblue}{\textbf{\ipa{qʰwæ˩}}}}{}
\textcolor{teal}{\mytextsc{nom}} \hspace{4pt} Ton~: L.

Haie, faite de bambou ou de broussailles épineuses.
\textcolor{brown}{\zh{篱笆。}}


 \mytextsc{clf}~: \textcolor{darkblue}{\textbf{\ipa{kɤ˧˥}}} 

\paragraph{\hspace{-0.5cm} {\Large \textcolor{darkblue}{\textbf{\ipa{qʰwæ˩\textsubscript{a}}}}}} \hypertarget{q\string_hw\{\string_Ba1}{}
\markboth{\textcolor{darkblue}{\textbf{\ipa{qʰwæ˩\textsubscript{a}}}}}{}
\textcolor{teal}{\mytextsc{verbe}} \hspace{4pt} Ton~: L\textsubscript{a}.

Bloquer.
\textcolor{brown}{\zh{挡住。}}




\paragraph{\hspace{-0.5cm} {\Large \textcolor{darkblue}{\textbf{\ipa{qʰwæ˩kɤ˩}}}}} \hypertarget{q\string_hw\{\string_Bk7\string_B1}{}
\markboth{\textcolor{darkblue}{\textbf{\ipa{qʰwæ˩kɤ˩}}}}{}
\textcolor{teal}{\mytextsc{nom}} \hspace{4pt} Ton~: L.

Une sorte d'arbuste d'environ 1 mètre 50 à 2 mètres de haut.
\textcolor{brown}{\zh{一种灌木,1.5至2米高,可以当篱笆用。}}


\begin{exe}
\sn \textcolor{darkblue}{\textbf{\ipa{qʰwæ˩kɤ˩-dzi˩˥}}}
\trans même sens
\trans \textit{\textcolor{brown}{\zh{同上}}}
\end{exe}
 \mytextsc{clf}~: \textcolor{darkblue}{\textbf{\ipa{dzi˩, ʝi˧}}} 

\paragraph{\hspace{-0.5cm} {\Large \textcolor{darkblue}{\textbf{\ipa{qʰwæ˧˥}}} \textsubscript{1}}} \hypertarget{q\string_hw\{\string_M\string_T1}{}
\markboth{\textcolor{darkblue}{\textbf{\ipa{qʰwæ˧˥}}} \textsubscript{1}}{}
\textcolor{teal}{\mytextsc{verbe}} \hspace{4pt} Ton~: MH.

Briser (verre, vaisselle…), faire éclater; casser (des noix).
\textcolor{brown}{\zh{弄碎。}}


\begin{exe}
\sn \textcolor{darkblue}{\textbf{\ipa{ʁo˧do˧ qʰwæ˧˥}}}
\trans casser des noix
\trans \textit{\textcolor{brown}{\zh{敲开坚果(在永宁,不用夹子:用锤子敲开)}}}
\end{exe}


\paragraph{\hspace{-0.5cm} {\Large \textcolor{darkblue}{\textbf{\ipa{qʰwæ˧˥}}} \textsubscript{2}}} \hypertarget{q\string_hw\{\string_M\string_T2}{}
\markboth{\textcolor{darkblue}{\textbf{\ipa{qʰwæ˧˥}}} \textsubscript{2}}{}
\textcolor{teal}{\mytextsc{verbe}} \hspace{4pt} Ton~: MH.

Gifler.
\textcolor{brown}{\zh{掴、打。}}


\begin{exe}
\sn \textcolor{darkblue}{\textbf{\ipa{le˧-qʰwæ˧-ze˥}}}
\trans \mytextsc{accomp} \string_ \mytextsc{pfv}
\trans \textit{\textcolor{brown}{\zh{掴了}}}
\end{exe}
\begin{exe}
\sn \textcolor{darkblue}{\textbf{\ipa{zɯ˧ɻ̍˧ qʰwæ˩}}}
\trans gifler
\trans \textit{\textcolor{brown}{\zh{打嘴巴}}}
\end{exe}
\begin{exe}
\sn \textcolor{darkblue}{\textbf{\ipa{zɯ˧ɻ̍˧ | ɖɯ˧-ɭɯ˧ | tʰi˧-qʰwæ˧-bi˥!}}}
\trans Je vais te gifler! / Je vais te flanquer une gifle! (A un enfant)
\trans \textit{\textcolor{brown}{\zh{我要打嘴巴了!(对孩子说)}}}
\end{exe}


\paragraph{\hspace{-0.5cm} {\Large \textcolor{darkblue}{\textbf{\ipa{qʰwæ˧˥\textsubscript{a}}}}}} \hypertarget{q\string_hw\{\string_M\string_Ta1}{}
\markboth{\textcolor{darkblue}{\textbf{\ipa{qʰwæ˧˥\textsubscript{a}}}}}{}
\textcolor{teal}{\mytextsc{classificateur}} \hspace{4pt} Ton~: MH\textsubscript{a}.

Classificateur des filaments de chanvre avant filage.
\textcolor{brown}{\zh{量词:丝,如纺之前的麻丝(一根)。}}




\paragraph{\hspace{-0.5cm} {\Large \textcolor{darkblue}{\textbf{\ipa{qʰwɤ˧}}}}} \hypertarget{q\string_hw7\string_M1}{}
\markboth{\textcolor{darkblue}{\textbf{\ipa{qʰwɤ˧}}}}{}
\textcolor{teal}{\mytextsc{nom}} \hspace{4pt} Ton~: M.

Traces, piste (d'un animal; lorsqu'on chasse, on suit la piste d'un animal, on le suit à la trace).
\textcolor{brown}{\zh{痕迹。}}


 \mytextsc{clf}~: \textcolor{darkblue}{\textbf{\ipa{pʰo˧˥}}} 

\paragraph{\hspace{-0.5cm} {\Large \textcolor{darkblue}{\textbf{\ipa{qʰwɤ˧\textsubscript{a}}}}}} \hypertarget{q\string_hw7\string_Ma1}{}
\markboth{\textcolor{darkblue}{\textbf{\ipa{qʰwɤ˧\textsubscript{a}}}}}{}
\textcolor{teal}{\mytextsc{verbe}} \hspace{4pt} Ton~: M\textsubscript{a}.

Se guérir (blessure, maladie); se rétablir (une fracture).
\textcolor{brown}{\zh{治好(骨折、病)。}}


\begin{exe}
\sn \textcolor{darkblue}{\textbf{\ipa{le˧-qʰwɤ˧-ɲi˥!}}}
\trans C'est guéri! / La fracture est rétablie!
\trans \textit{\textcolor{brown}{\zh{治好了!}}}
\end{exe}


\paragraph{\hspace{-0.5cm} {\Large \textcolor{darkblue}{\textbf{\ipa{qʰwɤ˧bi˩}}}}} \hypertarget{q\string_hw7\string_Mbi\string_B1}{}
\markboth{\textcolor{darkblue}{\textbf{\ipa{qʰwɤ˧bi˩}}}}{}
\textcolor{teal}{\mytextsc{nom}} \hspace{4pt} Ton~: L\#.
\ding{202} 
Sabot, patte.
\textcolor{brown}{\zh{马蹄、马的脚。}}


\begin{exe}
\sn \textcolor{darkblue}{\textbf{\ipa{ʐwæ˧-qʰwɤ˧bi˥\#}}}
\trans sabot de cheval
\trans \textit{\textcolor{brown}{\zh{马蹄、(马、狗……的)脚}}}
\end{exe}
\begin{exe}
\sn \textcolor{darkblue}{\textbf{\ipa{kʰv̩˩-qʰwɤ˩bi˥\#}}}
\trans patte de chien
\trans \textit{\textcolor{brown}{\zh{狗脚}}}
\end{exe}
\ding{203} 
Traces, piste (d'un animal).
\textcolor{brown}{\zh{动物脚的痕迹、行径。}}


 \mytextsc{clf}~: \textcolor{darkblue}{\textbf{\ipa{bi˩}}} \textcolor{darkblue}{\textbf{\ipa{tʰv̩˧˥}}} 

\paragraph{\hspace{-0.5cm} {\Large \textcolor{darkblue}{\textbf{\ipa{qʰwɤ˧mi˥\$}}}}} \hypertarget{q\string_hw7\string_Mmi\string_T\$1}{}
\markboth{\textcolor{darkblue}{\textbf{\ipa{qʰwɤ˧mi˥\$}}}}{}
\textcolor{teal}{\mytextsc{nom}} \hspace{4pt} Ton~: H\$.

Grand bol (autrefois, les bols étaient en bois).
\textcolor{brown}{\zh{大碗(以前碗是用木头做的)。}}


\textit{Voir~:} \hyperlink{q\string_hw7\string_Mp7\string_T\$1}{\textcolor{darkblue}{\textbf{\ipa{qʰwɤ˧pɤ˥\$}}}} 

\paragraph{\hspace{-0.5cm} {\Large \textcolor{darkblue}{\textbf{\ipa{qʰwɤ˧pɤ˥\$}}}}} \hypertarget{q\string_hw7\string_Mp7\string_T\$1}{}
\markboth{\textcolor{darkblue}{\textbf{\ipa{qʰwɤ˧pɤ˥\$}}}}{}
\textcolor{teal}{\mytextsc{nom}} \hspace{4pt} Ton~: H\$.

Grand bol.
\textcolor{brown}{\zh{大碗。}}


\textit{Voir~:} \hyperlink{q\string_hw7\string_Mmi\string_T\$1}{\textcolor{darkblue}{\textbf{\ipa{qʰwɤ˧mi˥\$}}}} 

\paragraph{\hspace{-0.5cm} {\Large \textcolor{darkblue}{\textbf{\ipa{qʰwɤ˧ʂe˩}}}}} \hypertarget{q\string_hw7\string_Ms`e\string_B1}{}
\markboth{\textcolor{darkblue}{\textbf{\ipa{qʰwɤ˧ʂe˩}}}}{}
\textcolor{teal}{\mytextsc{nom}} \hspace{4pt} Ton~: L\#.

Fer à cheval.
\textcolor{brown}{\zh{马蹄铁。}}


\begin{exe}
\sn \textcolor{darkblue}{\textbf{\ipa{ʐwæ˧-qʰwɤ˧ʂe˥ (+ɲi˩)}}}
\trans fer à cheval
\trans \textit{\textcolor{brown}{\zh{马蹄铁}}}
\end{exe}
 \mytextsc{clf}~: \textcolor{darkblue}{\textbf{\ipa{nɑ˧}}} \textcolor{darkblue}{\textbf{\ipa{pʰo˧˥}}} 

\paragraph{\hspace{-0.5cm} {\Large \textcolor{darkblue}{\textbf{\ipa{qʰwɤ˧to˩}}}}} \hypertarget{q\string_hw7\string_Mto\string_B1}{}
\markboth{\textcolor{darkblue}{\textbf{\ipa{qʰwɤ˧to˩}}}}{}
\textcolor{teal}{\mytextsc{nom}} \hspace{4pt} Ton~: L\#.

Épaule (extrémité de l'épaule, bout de l'épaule).
\textcolor{brown}{\zh{肩膀的末端。}}


\begin{exe}
\sn \textcolor{darkblue}{\textbf{\ipa{hĩ˧ ʈʂʰɯ˧-v̩˧, | qʰwɤ˧to˩ | ɖɯ˧-pi˧˥ | ʂwæ˧-hṽ˩-di˩!}}}
\trans Ce type, il a les épaules un peu de travers! / Il a une épaule plus haute que l'autre!
\trans \textit{\textcolor{brown}{\zh{这个人的肩膀不正,一高一低!}}}
\end{exe}
 \mytextsc{clf}~: \textcolor{darkblue}{\textbf{\ipa{ɭɯ˧}}} 

\paragraph{\hspace{-0.5cm} {\Large \textcolor{darkblue}{\textbf{\ipa{qʰwɤ˧tʰv̩\#˥}}}}} \hypertarget{q\string_hw7\string_Mt\string_hv\string_=\#\string_T1}{}
\markboth{\textcolor{darkblue}{\textbf{\ipa{qʰwɤ˧tʰv̩\#˥}}}}{}
\textcolor{teal}{\mytextsc{nom}} \hspace{4pt} Ton~: \#H.

Hotte en bambou pour porter de l'eau.
\textcolor{brown}{\zh{竹篓。}}


 \mytextsc{clf}~: \textcolor{darkblue}{\textbf{\ipa{ɭɯ˧}}} 

\paragraph{\hspace{-0.5cm} {\Large \textcolor{darkblue}{\textbf{\ipa{qʰwɤ˧tsʰi˩}}}}} \hypertarget{q\string_hw7\string_Mts\string_hi\string_B1}{}
\markboth{\textcolor{darkblue}{\textbf{\ipa{qʰwɤ˧tsʰi˩}}}}{}
\textcolor{teal}{\mytextsc{nom}} \hspace{4pt} Ton~: L\#.

Épaule.
\textcolor{brown}{\zh{肩膀。}}


\begin{exe}
\sn \textcolor{darkblue}{\textbf{\ipa{qʰwɤ˧tsʰi˩-ʁo˩ | hwæ˧pʰæ˩ | ɖɯ˧-nɑ˧-ʈʂʰɯ˧ gɤ˧˥}}}
\trans porter une houe à l'épaule
\trans \textit{\textcolor{brown}{\zh{肩上扛一把锄头}}}
\end{exe}
 \mytextsc{clf}~: \textcolor{darkblue}{\textbf{\ipa{ɭɯ˧}}} 

\paragraph{\hspace{-0.5cm} {\Large \textcolor{darkblue}{\textbf{\ipa{qʰwɤ˧zo˥\$}}}}} \hypertarget{q\string_hw7\string_Mzo\string_T\$1}{}
\markboth{\textcolor{darkblue}{\textbf{\ipa{qʰwɤ˧zo˥\$}}}}{}
\textcolor{teal}{\mytextsc{nom}} \hspace{4pt} Ton~: H\$.

Petit bol.
\textcolor{brown}{\zh{小碗。}}




\paragraph{\hspace{-0.5cm} {\Large \textcolor{darkblue}{\textbf{\ipa{qʰwɤ˩\textsubscript{a}}}} \textsubscript{1}}} \hypertarget{q\string_hw7\string_Ba1}{}
\markboth{\textcolor{darkblue}{\textbf{\ipa{qʰwɤ˩\textsubscript{a}}}} \textsubscript{1}}{}
\textcolor{teal}{\mytextsc{adjectif}} \hspace{4pt} Ton~: L\textsubscript{a}.

Intelligent.
\textcolor{brown}{\zh{聪明。}}


\begin{exe}
\sn \textcolor{darkblue}{\textbf{\ipa{qʰwɤ˩-hĩ˩˥}}}
\trans \mytextsc{rel}/\mytextsc{nmlz}
\trans \textit{\textcolor{brown}{\zh{聪明的}}}
\end{exe}
\begin{exe}
\sn \textcolor{darkblue}{\textbf{\ipa{qʰwɤ˩-le˥!}}}
\trans (il est/tu es) intelligent! (Commentaire lorsque quelqu'un dit ou fait quelque chose d'astucieux)
\trans \textit{\textcolor{brown}{\zh{很聪明! / 太聪明了!}}}
\end{exe}
\begin{exe}
\sn \textcolor{darkblue}{\textbf{\ipa{ɖwæ˧˥ | qʰwɤ˩˥!}}}
\trans \mytextsc{intensif}.très \string_: très intelligent
\trans \textit{\textcolor{brown}{\zh{很聪明!}}}
\end{exe}


\paragraph{\hspace{-0.5cm} {\Large \textcolor{darkblue}{\textbf{\ipa{qʰwɤ˩\textsubscript{a}}}} \textsubscript{2}}} \hypertarget{q\string_hw7\string_Ba2}{}
\markboth{\textcolor{darkblue}{\textbf{\ipa{qʰwɤ˩\textsubscript{a}}}} \textsubscript{2}}{}
\textcolor{teal}{\mytextsc{adjectif}} \hspace{4pt} Ton~: L\textsubscript{a}.

Mauvais.
\textcolor{brown}{\zh{坏。}}


\begin{exe}
\sn \textcolor{darkblue}{\textbf{\ipa{kʰv̩˧ | qʰwɤ˩-hĩ˩˥}}}
\trans une mauvaise année
\trans \textit{\textcolor{brown}{\zh{(收成)不好的一年}}}
\end{exe}
\begin{exe}
\sn \textcolor{darkblue}{\textbf{\ipa{kʰv̩˧ qʰwɤ˧˥}}}
\trans une mauvaise année
\trans \textit{\textcolor{brown}{\zh{(收成)不好的一年}}}
\end{exe}
\begin{exe}
\sn \textcolor{darkblue}{\textbf{\ipa{tsʰi˧-ʝi˧, | kʰv̩˧qʰwɤ˧ tʰv̩˧˥!}}}
\trans cette année, c'est une mauvaise année (les récoltes sont mauvaises)!
\trans \textit{\textcolor{brown}{\zh{今年,年景不好!(收成不好)}}}
\end{exe}
\begin{exe}
\sn \textcolor{darkblue}{\textbf{\ipa{ʈʂʰɯ˧ | nv̩˩mi˩˥ | ɖwæ˧˥ | qʰwɤ˩˥!}}}
\trans Il a l'âme bien noire!
\trans \textit{\textcolor{brown}{\zh{他心很坏!}}}
\end{exe}
\begin{exe}
\sn \textcolor{darkblue}{\textbf{\ipa{qʰwɤ˩-ʝi˩˥}}}
\trans faire des bêtises, faire du mal: abîmer des choses, faire des misères aux gens...
\trans \textit{\textcolor{brown}{\zh{干坏事:损坏东西,干扰人家……}}}
\end{exe}
\begin{exe}
\sn \textcolor{darkblue}{\textbf{\ipa{ʈʂʰɯ˧-ɳɯ˧ | njɤ˧-bv̩˧ tso˧\textasciitilde{}tso˧ | le˧-qʰwɤ˩-ʝi˩-ze˩!}}}
\trans Il a abîmé mes affaires!
\trans \textit{\textcolor{brown}{\zh{他弄坏了我的东西!}}}
\end{exe}
\begin{exe}
\sn \textcolor{darkblue}{\textbf{\ipa{hĩ˧ qʰwɤ˧-ʝi˥}}}
\trans embêter les gens, faire des misères aux gens
\trans \textit{\textcolor{brown}{\zh{干扰人家、麻烦人}}}
\end{exe}
\begin{exe}
\sn \textcolor{darkblue}{\textbf{\ipa{ʈʂʰɯ˧ | to˩to˧mi˥ hĩ˩ qʰwɤ˩-ʝi˩!}}}
\trans Il/elle fait exprès d'embêter les gens!
\trans \textit{\textcolor{brown}{\zh{他故意麻烦人!}}}
\end{exe}


\paragraph{\hspace{-0.5cm} {\Large \textcolor{darkblue}{\textbf{\ipa{qʰwɤ˩ɖɯ˩}}}}} \hypertarget{q\string_hw7\string_Bd`M\string_B1}{}
\markboth{\textcolor{darkblue}{\textbf{\ipa{qʰwɤ˩ɖɯ˩}}}}{}
\textcolor{teal}{\mytextsc{nom}} \hspace{4pt} Ton~: L.

Membres de la famille (étendue).
\textcolor{brown}{\zh{亲戚。}}


\begin{exe}
\sn \textcolor{darkblue}{\textbf{\ipa{ə˧zɯ˩ | qʰwɤ˩ɖɯ˩ ɲi˥}}}
\trans Tous deux, on est de la même famille
\trans \textit{\textcolor{brown}{\zh{咱们两个是一家人。}}}
\end{exe}
\begin{exe}
\sn \textcolor{darkblue}{\textbf{\ipa{qʰwɤ˩ɖɯ˩˥, | v̩˩dze˩˥}}}
\trans famille et amis, cercle familial élargi
\trans \textit{\textcolor{brown}{\zh{亲人:泛指亲戚与亲密朋友们}}}
\end{exe}
\begin{exe}
\sn \textcolor{darkblue}{\textbf{\ipa{qʰwɤ˩ɖɯ˩ to˥}}}
\trans établir des liens familiaux, unir deux familles (par un mariage)
\trans \textit{\textcolor{brown}{\zh{建立起两个家庭之间的联系(通过婚姻)}}}
\end{exe}
 \mytextsc{clf}~: \textcolor{darkblue}{\textbf{\ipa{v̩˧}}} 

\paragraph{\hspace{-0.5cm} {\Large \textcolor{darkblue}{\textbf{\ipa{qʰwɤ˧˥}}} \textsubscript{1}}} \hypertarget{q\string_hw7\string_M\string_T1}{}
\markboth{\textcolor{darkblue}{\textbf{\ipa{qʰwɤ˧˥}}} \textsubscript{1}}{}
\textcolor{teal}{\mytextsc{nom}} \hspace{4pt} Ton~: MH.

Bol.
\textcolor{brown}{\zh{碗。}}


 \mytextsc{clf}~: \textcolor{darkblue}{\textbf{\ipa{ɭɯ˧}}} 

\paragraph{\hspace{-0.5cm} {\Large \textcolor{darkblue}{\textbf{\ipa{qʰwɤ˧˥}}} \textsubscript{2}}} \hypertarget{q\string_hw7\string_M\string_T2}{}
\markboth{\textcolor{darkblue}{\textbf{\ipa{qʰwɤ˧˥}}} \textsubscript{2}}{}
\textcolor{teal}{\mytextsc{nom}} \hspace{4pt} Ton~: MH.

Histoire, récit.
\textcolor{brown}{\zh{故事。}}


\begin{exe}
\sn \textcolor{darkblue}{\textbf{\ipa{æ˧ʂæ˧-qʰwɤ˧˥}}}
\trans récit d'autrefois, conte
\trans \textit{\textcolor{brown}{\zh{老故事}}}
\end{exe}
 \mytextsc{clf}~: \textcolor{darkblue}{\textbf{\ipa{kʰwɤ˥}}} 

\paragraph{\hspace{-0.5cm} {\Large \textcolor{darkblue}{\textbf{\ipa{qʰwɤ˧˥\textsubscript{a}}}}}} \hypertarget{q\string_hw7\string_M\string_Ta1}{}
\markboth{\textcolor{darkblue}{\textbf{\ipa{qʰwɤ˧˥\textsubscript{a}}}}}{}
\textcolor{teal}{\mytextsc{classificateur}} \hspace{4pt} Ton~: MH\textsubscript{a}.

Classificateur des bols (utilisés comme quantité de mesure du non dénombrable).
\textcolor{brown}{\zh{量词:碗。}}




\newpage
\section*{\centering- \textcolor{darkblue}{\textbf{\ipa{ɻ}}} -}
\paragraph{\hspace{-0.5cm} {\Large \textcolor{darkblue}{\textbf{\ipa{ɻ̍˩}}}}} \hypertarget{r£`̍\string_B1}{}
\markboth{\textcolor{darkblue}{\textbf{\ipa{ɻ̍˩}}}}{}
\textcolor{teal}{\mytextsc{nom}} \hspace{4pt} Ton~: L.

Direction, côté.
\textcolor{brown}{\zh{面(一个四方形物品的四面)。}}


\begin{exe}
\sn \textcolor{darkblue}{\textbf{\ipa{[``Housebuilding"] ʐv̩˧-ɻ̍˥}}}
\trans les quatre directions, les quatre côtés (d'une maison)
\trans \textit{\textcolor{brown}{\zh{四面}}}
\end{exe}


\paragraph{\hspace{-0.5cm} {\Large \textcolor{darkblue}{\textbf{\ipa{ɻ̍˩\textsubscript{b}}}}}} \hypertarget{r£`̍\string_Bb1}{}
\markboth{\textcolor{darkblue}{\textbf{\ipa{ɻ̍˩\textsubscript{b}}}}}{}
\textcolor{teal}{\mytextsc{verbe}} \hspace{4pt} Ton~: L\textsubscript{b}.

Regarder, se tourner vers, faire face à.
\textcolor{brown}{\zh{对着。}}


\begin{exe}
\sn \textcolor{darkblue}{\textbf{\ipa{mɤ˧-ɻ̍˩}}}
\trans \mytextsc{neg}
\trans \textit{\textcolor{brown}{\zh{\mytextsc{neg}}}}
\end{exe}
\begin{exe}
\sn \textcolor{darkblue}{\textbf{\ipa{ɖɯ˧-ɻ̍˧\textasciitilde{}ɻ̍˩}}}
\trans \mytextsc{délimitatif} \string_ \mytextsc{red}
\trans \textit{\textcolor{brown}{\zh{\mytextsc{delimitative} \string_ \mytextsc{red}}}}
\end{exe}
\begin{exe}
\sn \textcolor{darkblue}{\textbf{\ipa{ze˩gi˧ ɻ̍˥?}}}
\trans dans quelle direction regarder?
\trans \textit{\textcolor{brown}{\zh{(我要)往哪边转?}}}
\end{exe}
\begin{exe}
\sn \textcolor{darkblue}{\textbf{\ipa{no˧ | ʈʂʰɯ˧tɕo˧ ɻ̍˩!}}}
\trans Tourne-toi par ici! / Regarde par ici!
\trans \textit{\textcolor{brown}{\zh{你往这里转/往这里看!}}}
\end{exe}
\begin{exe}
\sn \textcolor{darkblue}{\textbf{\ipa{gɤ˩-ɻ̍˥ mv̩˩-ɻ̍˩, | ə˧tso˧ li˧?}}}
\trans Qu’as-tu à regarder de toutes parts ? / Tu regardes de toutes parts, que cherches-tu ?
\trans \textit{\textcolor{brown}{\zh{你左转右转,(到底)在看什么?}}}
\end{exe}


\paragraph{\hspace{-0.5cm} {\Large \textcolor{darkblue}{\textbf{\ipa{ɻ̍˧bɤ˧}}}}} \hypertarget{r£`̍\string_Mb7\string_M1}{}
\markboth{\textcolor{darkblue}{\textbf{\ipa{ɻ̍˧bɤ˧}}}}{}
\textcolor{teal}{\mytextsc{nom}} \hspace{4pt} Ton~: M.

La vérité; le vrai et le faux; les faits authentiques.
\textcolor{brown}{\zh{实情,真理。}}


\begin{exe}
\sn \textcolor{darkblue}{\textbf{\ipa{njɤ˧-ɳɯ˧ | ɻ̍˧bɤ˧ | ʐwɤ˩-bi˩˥!}}}
\trans je vais dire toute la vérité/je vais faire la lumière!
\trans \textit{\textcolor{brown}{\zh{我要把实情说出来!}}}
\end{exe}
 \mytextsc{clf}~: \textcolor{darkblue}{\textbf{\ipa{kʰwɤ˥}}} 

\paragraph{\hspace{-0.5cm} {\Large \textcolor{darkblue}{\textbf{\ipa{ɻ̍˧qʰv̩˧}}}}} \hypertarget{r£`̍\string_Mq\string_hv\string_=\string_M1}{}
\markboth{\textcolor{darkblue}{\textbf{\ipa{ɻ̍˧qʰv̩˧}}}}{}
\textcolor{teal}{\mytextsc{nom}} \hspace{4pt} Ton~: M.

Source chaude.
\textcolor{brown}{\zh{温泉。}}


\begin{exe}
\sn \textcolor{darkblue}{\textbf{\ipa{ɻ̍˧qʰv̩˧-dʑɯ˩}}}
\trans eau de source chaude (non potable)
\trans \textit{\textcolor{brown}{\zh{温泉水(不可饮用)}}}
\end{exe}
 \mytextsc{clf}~: \textcolor{darkblue}{\textbf{\ipa{ɭɯ˧}}} 

\paragraph{\hspace{-0.5cm} {\Large \textcolor{darkblue}{\textbf{\ipa{ɻ̍˩ɻ̍˧-lo˩}}}}} \hypertarget{r£`̍\string_Br£`̍\string_M-lo\string_B1}{}
\markboth{\textcolor{darkblue}{\textbf{\ipa{ɻ̍˩ɻ̍˧-lo˩}}}}{}
\textcolor{teal}{\mytextsc{nom}} \hspace{4pt} Ton~: LM-L.

Le cheval qui marche en second (derrière le cheval de tête), dans une caravan.
\textcolor{brown}{\zh{马帮中的第二匹马。}}




\paragraph{\hspace{-0.5cm} {\Large \textcolor{darkblue}{\textbf{\ipa{ɻ̍˧tɑ˧}}}}} \hypertarget{r£`̍\string_MtA\string_M1}{}
\markboth{\textcolor{darkblue}{\textbf{\ipa{ɻ̍˧tɑ˧}}}}{}
\textcolor{teal}{\mytextsc{nom}} \hspace{4pt} Ton~: M.

Ganglions.
\textcolor{brown}{\zh{淋巴结。}}


 \mytextsc{clf}~: \textcolor{darkblue}{\textbf{\ipa{ɭɯ˧}}} 

\paragraph{\hspace{-0.5cm} {\Large \textcolor{darkblue}{\textbf{\ipa{ɻ̍˩ʈʂʰe˧-ɖɯ˩mɑ˩}}}}} \hypertarget{r£`̍\string_Bt`s`\string_he\string_M-d`M\string_BmA\string_B1}{}
\markboth{\textcolor{darkblue}{\textbf{\ipa{ɻ̍˩ʈʂʰe˧-ɖɯ˩mɑ˩}}}}{}
\textcolor{teal}{\mytextsc{nom}} \hspace{4pt} Ton~: LM-L.

Prénom féminin.
\textcolor{brown}{\zh{女性名字。}}




\paragraph{\hspace{-0.5cm} {\Large \textcolor{darkblue}{\textbf{\ipa{ɻ̍˩ʈʂʰe˧-tsʰɯ˩ɻ̍˩}}}}} \hypertarget{r£`̍\string_Bt`s`\string_he\string_M-ts\string_hM\string_Br£`̍\string_B1}{}
\markboth{\textcolor{darkblue}{\textbf{\ipa{ɻ̍˩ʈʂʰe˧-tsʰɯ˩ɻ̍˩}}}}{}
\textcolor{teal}{\mytextsc{nom}} \hspace{4pt} Ton~: LM-L.

Prénom masculin.
\textcolor{brown}{\zh{男性名字。}}




\paragraph{\hspace{-0.5cm} {\Large \textcolor{darkblue}{\textbf{\ipa{ɻ̍˩ʈʂʰe\#˥}}}}} \hypertarget{r£`̍\string_Bt`s`\string_he\#\string_T1}{}
\markboth{\textcolor{darkblue}{\textbf{\ipa{ɻ̍˩ʈʂʰe\#˥}}}}{}
\textcolor{teal}{\mytextsc{nom}} \hspace{4pt} Ton~: LM+\#H.

Prénom masculin.
\textcolor{brown}{\zh{男性名字。}}




\paragraph{\hspace{-0.5cm} {\Large \textcolor{darkblue}{\textbf{\ipa{‑ɻ̍˩}}}}} \hypertarget{‑r£`̍\string_B1}{}
\markboth{\textcolor{darkblue}{\textbf{\ipa{‑ɻ̍˩}}}}{}
\textcolor{teal}{\mytextsc{suffixe}} \hspace{4pt} Ton~: L.

\mytextsc{inchoatif}.
\textcolor{brown}{\zh{\mytextsc{发端。}}}




\paragraph{\hspace{-0.5cm} {\Large \textcolor{darkblue}{\textbf{\ipa{=ɻ̍˩}}}}} \hypertarget{=r£`̍\string_B1}{}
\markboth{\textcolor{darkblue}{\textbf{\ipa{=ɻ̍˩}}}}{}
\textcolor{teal}{\mytextsc{clitique}} \hspace{4pt} Ton~: L.

Pluriel associatif, ou pluriel d'accompagnement, couramment utilisé avec les termes de parenté, les noms de clans...
\textcolor{brown}{\zh{联想复数:一家人、一族人、一辈人……。}}


\begin{exe}
\sn \textcolor{darkblue}{\textbf{\ipa{ʈʂʰɯ˧-ʑi˧=ɻ̍˥}}}
\trans les gens de cette famille; cette maisonnée-ci
\trans \textit{\textcolor{brown}{\zh{这家的人}}}
\end{exe}


\paragraph{\hspace{-0.5cm} {\Large \textcolor{darkblue}{\textbf{\ipa{=ɻæ˩}}}}} \hypertarget{=r£`\{\string_B1}{}
\markboth{\textcolor{darkblue}{\textbf{\ipa{=ɻæ˩}}}}{}
\textcolor{teal}{\mytextsc{clitique}} \hspace{4pt} Ton~: L.

Pluriel.
\textcolor{brown}{\zh{多数。}}


\begin{exe}
\sn \textcolor{darkblue}{\textbf{\ipa{ʈʂʰɯ˧-ɻæ˥\$}}}
\trans ces choses-ci, cette sorte de choses
\trans \textit{\textcolor{brown}{\zh{这类的东西,……之类}}}
\end{exe}


\paragraph{\hspace{-0.5cm} {\Large \textcolor{darkblue}{\textbf{\ipa{ɻæ˩\textsubscript{a}}}}}} \hypertarget{r£`\{\string_Ba1}{}
\markboth{\textcolor{darkblue}{\textbf{\ipa{ɻæ˩\textsubscript{a}}}}}{}
\textcolor{teal}{\mytextsc{adjectif}} \hspace{4pt} Ton~: L\textsubscript{a}.

Plat, de forme plate, aplati.
\textcolor{brown}{\zh{瘪。}}


\begin{exe}
\sn \textcolor{darkblue}{\textbf{\ipa{ɻæ˩-hĩ˩˥}}}
\trans \mytextsc{rel}/\mytextsc{nmlz}
\trans \textit{\textcolor{brown}{\zh{瘪的}}}
\end{exe}
\begin{exe}
\sn \textcolor{darkblue}{\textbf{\ipa{ɻæ˩ti˩ɻæ˥ (-gv̩˩)}}}
\trans raplapla, ratatiné
\trans \textit{\textcolor{brown}{\zh{瘪瘪的}}}
\end{exe}


\paragraph{\hspace{-0.5cm} {\Large \textcolor{darkblue}{\textbf{\ipa{ɻæ˩˥}}} \textsubscript{1}}} \hypertarget{r£`\{\string_B\string_T1}{}
\markboth{\textcolor{darkblue}{\textbf{\ipa{ɻæ˩˥}}} \textsubscript{1}}{}
\textcolor{teal}{\mytextsc{nom}} \hspace{4pt} Ton~: LH.

Graine.
\textcolor{brown}{\zh{种子。}}


 \mytextsc{clf}~: \textcolor{darkblue}{\textbf{\ipa{ɭɯ˧}}} objets ronds

\paragraph{\hspace{-0.5cm} {\Large \textcolor{darkblue}{\textbf{\ipa{ɻæ˩˥}}} \textsubscript{2}}} \hypertarget{r£`\{\string_B\string_T2}{}
\markboth{\textcolor{darkblue}{\textbf{\ipa{ɻæ˩˥}}} \textsubscript{2}}{}
\textcolor{teal}{\mytextsc{nom}} \hspace{4pt} Ton~: LH.

Joug (le terme est le même pour un ou deux animaux).
\textcolor{brown}{\zh{牛轭(单行或双行)。}}


\begin{exe}
\sn \textcolor{darkblue}{\textbf{\ipa{ʝi˧-ɻæ˥}}}
\trans même sens que la forme monosyllabique: joug; littéralement 'joug de bœuf/buffle'
\trans \textit{\textcolor{brown}{\zh{牛轭}}}
\end{exe}
\begin{exe}
\sn \textcolor{darkblue}{\textbf{\ipa{ɻæ˩ ʈʂʰɯ˩-ɭɯ˥ / ɻæ˩ ʈʂʰɯ˩-ɭɯ˩˥}}}
\trans \mytextsc{n}+\mytextsc{dem}+\mytextsc{clf;} cette expression possède deux variantes tonales
\trans \textit{\textcolor{brown}{\zh{这个牛轭}}}
\end{exe}
 \mytextsc{clf}~: \textcolor{darkblue}{\textbf{\ipa{ɭɯ˧}}} 

\paragraph{\hspace{-0.5cm} {\Large \textcolor{darkblue}{\textbf{\ipa{ɻwæ˥}}}}} \hypertarget{r£`w\{\string_T1}{}
\markboth{\textcolor{darkblue}{\textbf{\ipa{ɻwæ˥}}}}{}
\textcolor{teal}{\mytextsc{verbe}} \hspace{4pt} Ton~: H.
\ding{202} 
Crier, hurler; miauler; braire; hennir; rugir (chat, bœuf, cochon, mouton, loup, lion).
\textcolor{brown}{\zh{喊、吼、叫(人、猫、牛、猪、羊、狼、驴、狮子、老虎、豺狼……)。}}


\begin{exe}
\sn \textcolor{darkblue}{\textbf{\ipa{mɤ˧-ɻwæ˥}}}
\trans \mytextsc{neg}
\trans \textit{\textcolor{brown}{\zh{不叫}}}
\end{exe}
\begin{exe}
\sn \textcolor{darkblue}{\textbf{\ipa{ɻwæ˧\textasciitilde{}ɻwæ˧}}}
\trans \mytextsc{red}
\trans \textit{\textcolor{brown}{\zh{\mytextsc{重叠}}}}
\end{exe}
\begin{exe}
\sn \textcolor{darkblue}{\textbf{\ipa{hĩ˧ ɻwæ˧-dʑo˩}}}
\trans Il y a quelqu'un qui est est en train d'appeler/de crier
\trans \textit{\textcolor{brown}{\zh{有人在叫。}}}
\end{exe}
\begin{exe}
\sn \textcolor{darkblue}{\textbf{\ipa{ɖɯ˧-ɻwæ˧-ɻ̍˥}}}
\trans appeler, lancer un appel
\trans \textit{\textcolor{brown}{\zh{叫一声}}}
\end{exe}
\begin{exe}
\sn \textcolor{darkblue}{\textbf{\ipa{hwɤ˧li˧ ɻwæ˥-dʑo˩}}}
\trans le chat miaule
\trans \textit{\textcolor{brown}{\zh{猫在叫}}}
\end{exe}
\begin{exe}
\sn \textcolor{darkblue}{\textbf{\ipa{æ̃˩ ɻwæ˥}}}
\trans la poule caquette
\trans \textit{\textcolor{brown}{\zh{鸡在叫}}}
\end{exe}
\begin{exe}
\sn \textcolor{darkblue}{\textbf{\ipa{ʐwæ˧pʰæ˧di˧˥ | tʰi˧-ɻwæ˥-dʑo˩}}}
\trans l'âne brait
\trans \textit{\textcolor{brown}{\zh{驴在叫}}}
\end{exe}
\begin{exe}
\sn \textcolor{darkblue}{\textbf{\ipa{[F5] ʐwæ˧ | tʰi˧-ɻwæ˥-dʑo˩}}}
\trans le cheval est en train de hennir
\trans \textit{\textcolor{brown}{\zh{马在嘶}}}
\end{exe}
\begin{exe}
\sn \textcolor{darkblue}{\textbf{\ipa{ʐwæ˧ ɻwæ˧-dʑo˩!}}}
\trans le cheval est en train de hennir
\trans \textit{\textcolor{brown}{\zh{马在嘶}}}
\end{exe}
\ding{203} 
Inviter, faire venir.
\textcolor{brown}{\zh{请、叫(来)。}}


\begin{exe}
\sn \textcolor{darkblue}{\textbf{\ipa{ɖɯ˧-ɻwæ˧-ɻ̍˥}}}
\trans \mytextsc{délimitatif} \string_ \mytextsc{inchoatif}
\trans \textit{\textcolor{brown}{\zh{请来一下}}}
\end{exe}
\begin{exe}
\sn \textcolor{darkblue}{\textbf{\ipa{tʰɑ˧-ɻwæ˥!}}}
\trans \mytextsc{prohib}
\trans \textit{\textcolor{brown}{\zh{不要请!}}}
\end{exe}
\begin{exe}
\sn \textcolor{darkblue}{\textbf{\ipa{ɻwæ˧-mɤ˧-bi˧!}}}
\trans (on) ne l'invite pas!
\trans \textit{\textcolor{brown}{\zh{不请他!}}}
\end{exe}


\paragraph{\hspace{-0.5cm} {\Large \textcolor{darkblue}{\textbf{\ipa{ɻwæ˥\textsubscript{b}}}}}} \hypertarget{r£`w\{\string_Tb1}{}
\markboth{\textcolor{darkblue}{\textbf{\ipa{ɻwæ˥\textsubscript{b}}}}}{}
\textcolor{teal}{\mytextsc{classificateur}} \hspace{4pt} Ton~: H\textsubscript{b}.

Classificateur des lieux, des endroits.
\textcolor{brown}{\zh{量词:地方(一处)。}}


\begin{exe}
\sn \textcolor{darkblue}{\textbf{\ipa{tʰv̩˧-ɻwæ˧-qo˥ | mɤ˧-tʰv̩˧-sɯ˩!}}}
\trans ...n'est jamais allé dans ces lieux-là
\trans \textit{\textcolor{brown}{\zh{还没到这些地方}}}
\end{exe}


\newpage
\section*{\centering- \textcolor{darkblue}{\textbf{\ipa{ɻ̃}}} -}
\paragraph{\hspace{-0.5cm} {\Large \textcolor{darkblue}{\textbf{\ipa{ɻ̃˥}}}}} \hypertarget{r£`\string_~\string_T1}{}
\markboth{\textcolor{darkblue}{\textbf{\ipa{ɻ̃˥}}}}{}
\textcolor{teal}{\mytextsc{nom}} \hspace{4pt} Ton~: \#H.

Os, ossement.
\textcolor{brown}{\zh{骨头。}}


 \mytextsc{clf}~: \textcolor{darkblue}{\textbf{\ipa{kɤ˧˥}}} 

\paragraph{\hspace{-0.5cm} {\Large \textcolor{darkblue}{\textbf{\ipa{ɻ̃˥}}}}} \hypertarget{r£`\string_~\string_T1}{}
\markboth{\textcolor{darkblue}{\textbf{\ipa{ɻ̃˥}}}}{}
\textcolor{teal}{\mytextsc{adjectif}} \hspace{4pt} Ton~: H.

Démuni, en mauvaise passe.
\textcolor{brown}{\zh{困难、贫穷。}}


\begin{exe}
\sn \textcolor{darkblue}{\textbf{\ipa{le˧-ɻ̃˥-ze˩!}}}
\trans (il) est en mauvaise passe!/il est à la rue!
\trans \textit{\textcolor{brown}{\zh{(他)真的很穷苦!}}}
\end{exe}
\begin{exe}
\sn \textcolor{darkblue}{\textbf{\ipa{le˧-ɻ̃˧-bi˧}}}
\trans \mytextsc{accomp} \string_ \mytextsc{fut}\string_imm
\trans \textit{\textcolor{brown}{\zh{\mytextsc{accomp} \string_ \mytextsc{fut}\string_imm}}}
\end{exe}
\begin{exe}
\sn \textcolor{darkblue}{\textbf{\ipa{mɤ˧-ɻ̃˥}}}
\trans \mytextsc{neg}
\trans \textit{\textcolor{brown}{\zh{\mytextsc{neg}}}}
\end{exe}
\begin{exe}
\sn \textcolor{darkblue}{\textbf{\ipa{le˧-ɻ̃˧-zo˥, | ɻ̃˧-lɑ˩ bi˩-mɤ˩-dʑɯ˩!}}}
\trans ``Pour démuni/mal nourri/affamé qu'on soit, on n'en est pas encore maigre au point d'avoir les os à découvert!" Jeu de mots sur 'démuni' et 'ossement', qui sont homophones. Le proverbe sert à relativiser le malheur ressenti par quelqu'un.
\trans \textit{\textcolor{brown}{\zh{很困难,也还没有到饿死的程度啊! / 再困难,也还没饿死!(直译:“再困难,也没有露出骨头!”这个成语,来安慰认为自己太可怜的人。)}}}
\end{exe}
\begin{exe}
\sn \textcolor{darkblue}{\textbf{\ipa{ɻ̃˧-ʐwɤ˧˥}}}
\trans se plaindre
\trans \textit{\textcolor{brown}{\zh{诉苦、抱怨}}}
\end{exe}
\begin{exe}
\sn \textcolor{darkblue}{\textbf{\ipa{ɻ̃˧-ʐwɤ˧ | dɑ˧-ʐwɤ˧-ɻ̍˥}}}
\trans raconter ses malheurs; se plaindre
\trans \textit{\textcolor{brown}{\zh{诉苦、讲自己的不幸}}}
\end{exe}
\begin{exe}
\sn \textcolor{darkblue}{\textbf{\ipa{ʈʂʰɯ˧ | mɑ˧dɑ˩-qʰwɤ˩, | ɻ̃˧-ʐwɤ˧ | dɑ˧-ʐwɤ˧-ɻ̍˥!}}}
\trans Il est malheureux; il passe son temps à se plaindre!
\trans \textit{\textcolor{brown}{\zh{他不幸福,一直在讲自己怎么可怜!}}}
\end{exe}


\paragraph{\hspace{-0.5cm} {\Large \textcolor{darkblue}{\textbf{\ipa{ɻ̃˧}}}}} \hypertarget{r£`\string_~\string_M1}{}
\markboth{\textcolor{darkblue}{\textbf{\ipa{ɻ̃˧}}}}{}
\textcolor{teal}{\mytextsc{nom}} \hspace{4pt} Ton~: M.

Clan: ensemble de familles.
\textcolor{brown}{\zh{家族。}}


\begin{exe}
\sn \textcolor{darkblue}{\textbf{\ipa{ɻ̃˧ ɖɯ˧-ɻ̃˧}}}
\trans un clan
\trans \textit{\textcolor{brown}{\zh{一个家族}}}
\end{exe}
 \mytextsc{clf}~: \textcolor{darkblue}{\textbf{\ipa{ɻ̃˧}}} 

\paragraph{\hspace{-0.5cm} {\Large \textcolor{darkblue}{\textbf{\ipa{ɻ̃˧\textsubscript{b}}}}}} \hypertarget{r£`\string_~\string_Mb1}{}
\markboth{\textcolor{darkblue}{\textbf{\ipa{ɻ̃˧\textsubscript{b}}}}}{}
\textcolor{teal}{\mytextsc{classificateur}} \hspace{4pt} Ton~: M\textsubscript{b}.

Classificateur des clans / groupes de familles: littéralement 'un os'. Echelon supérieur à celui de la 'communauté familiale' dans la terminologie de Fu Maoji (1983).
\textcolor{brown}{\zh{量词:家族。}}




\paragraph{\hspace{-0.5cm} {\Large \textcolor{darkblue}{\textbf{\ipa{ɻ̃˧hæ˩}}}}} \hypertarget{r£`\string_~\string_Mh\{\string_B1}{}
\markboth{\textcolor{darkblue}{\textbf{\ipa{ɻ̃˧hæ˩}}}}{}
\textcolor{teal}{\mytextsc{nom}} \hspace{4pt} Ton~: L\#.

Cartilage.
\textcolor{brown}{\zh{软骨。}}


 \mytextsc{clf}~: \textcolor{darkblue}{\textbf{\ipa{ɭɯ˧}}} \textcolor{darkblue}{\textbf{\ipa{kɤ˧˥}}} 

\paragraph{\hspace{-0.5cm} {\Large \textcolor{darkblue}{\textbf{\ipa{ɻ̃˧kɤ˩}}}}} \hypertarget{r£`\string_~\string_Mk7\string_B1}{}
\markboth{\textcolor{darkblue}{\textbf{\ipa{ɻ̃˧kɤ˩}}}}{}
\textcolor{teal}{\mytextsc{nom}} \hspace{4pt} Ton~: L\#.

Colonne vertébrale.
\textcolor{brown}{\zh{脊椎骨。}}


 \mytextsc{clf}~: \textcolor{darkblue}{\textbf{\ipa{kɤ˧˥}}} 

\paragraph{\hspace{-0.5cm} {\Large \textcolor{darkblue}{\textbf{\ipa{ɻ̃˧ko˩}}}}} \hypertarget{r£`\string_~\string_Mko\string_B1}{}
\markboth{\textcolor{darkblue}{\textbf{\ipa{ɻ̃˧ko˩}}}}{}
\textcolor{teal}{\mytextsc{nom}} \hspace{4pt} Ton~: L\#.

Tibia.
\textcolor{brown}{\zh{胫骨。}}


\begin{exe}
\sn \textcolor{darkblue}{\textbf{\ipa{hĩ˧-dzɑ˧ | ɖʐe˧ tʰɑ˧-ʝi˥, | ɻ̃˧ko˩ mi˩ tʰɑ˩-tʰv̩˩. |}}}
\trans ``Le pauvre ne doit pas emprunter d'argent; le tibia ne doit pas recevoir de blessure!" (Ce proverbe enseigne qu'il ne faut pas toucher les points les plus sensibles, les plus fragiles.)
\trans \textit{\textcolor{brown}{\zh{“穷人莫借钱,胫骨莫受伤!”}}}
\end{exe}
 \mytextsc{clf}~: \textcolor{darkblue}{\textbf{\ipa{kɤ˧˥}}} 

\paragraph{\hspace{-0.5cm} {\Large \textcolor{darkblue}{\textbf{\ipa{ɻ̃˧mi˧}}}}} \hypertarget{r£`\string_~\string_Mmi\string_M1}{}
\markboth{\textcolor{darkblue}{\textbf{\ipa{ɻ̃˧mi˧}}}}{}
\textcolor{teal}{\mytextsc{nom}} \hspace{4pt} Ton~: M.

Tronc.
\textcolor{brown}{\zh{树干。}}


\begin{exe}
\sn \textcolor{darkblue}{\textbf{\ipa{si˧dzi˩-ɻ̃˩mi˩}}}
\trans tronc d'arbre
\trans \textit{\textcolor{brown}{\zh{树干}}}
\end{exe}
 \mytextsc{clf}~: \textcolor{darkblue}{\textbf{\ipa{kɤ˧˥}}} 

\paragraph{\hspace{-0.5cm} {\Large \textcolor{darkblue}{\textbf{\ipa{ɻ̃˧ʈʂæ˩}}}}} \hypertarget{r£`\string_~\string_Mt`s`\{\string_B1}{}
\markboth{\textcolor{darkblue}{\textbf{\ipa{ɻ̃˧ʈʂæ˩}}}}{}
\textcolor{teal}{\mytextsc{nom}} \hspace{4pt} Ton~: L\#.

Articulations (de la jambe: la cheville, le genou…; du bras: le poignet, le coude…).
\textcolor{brown}{\zh{关节部位,关节。}}


 \mytextsc{clf}~: \textcolor{darkblue}{\textbf{\ipa{ʈʂæ˧˥}}} 

\paragraph{\hspace{-0.5cm} {\Large \textcolor{darkblue}{\textbf{\ipa{ɻ̃˧ʈʂwæ˩}}}}} \hypertarget{r£`\string_~\string_Mt`s`w\{\string_B1}{}
\markboth{\textcolor{darkblue}{\textbf{\ipa{ɻ̃˧ʈʂwæ˩}}}}{}
\textcolor{teal}{\mytextsc{nom}} \hspace{4pt} Ton~: L\#.

\textit{Toricellia angulata Oliv.}.
\textcolor{brown}{\zh{接骨丹。}}


\begin{exe}
\sn \textcolor{darkblue}{\textbf{\ipa{ɻ̃˧ʈʂwæ˩-si˩}}}
\trans même sens
\trans \textit{\textcolor{brown}{\zh{同上}}}
\end{exe}


\newpage
\section*{\centering- \textcolor{darkblue}{\textbf{\ipa{ʁ}}} -}
\paragraph{\hspace{-0.5cm} {\Large \textcolor{darkblue}{\textbf{\ipa{ʁɑ˥}}}}} \hypertarget{RA\string_T1}{}
\markboth{\textcolor{darkblue}{\textbf{\ipa{ʁɑ˥}}}}{}
\textcolor{teal}{\mytextsc{nom}} \hspace{4pt} Ton~: \#H.

Force.
\textcolor{brown}{\zh{力气。}}


\begin{exe}
\sn \textcolor{darkblue}{\textbf{\ipa{ʁɑ˧ ʑi˧}}}
\trans avoir de la force
\trans \textit{\textcolor{brown}{\zh{有力量}}}
\end{exe}
\begin{exe}
\sn \textcolor{darkblue}{\textbf{\ipa{no˧ɻ̍˩ | hĩ˧tɕʰi˧ ʁɑ˧ ʑi˧!}}}
\trans votre famille/lignée/tribu est puissante!
\trans \textit{\textcolor{brown}{\zh{你们家族很强大!}}}
\end{exe}
\begin{exe}
\sn \textcolor{darkblue}{\textbf{\ipa{ʁɑ˧ tʰv̩˧ (+ze˩)}}}
\trans faire des efforts, donner toutes ses forces, s'impliquer (dans une tâche)
\trans \textit{\textcolor{brown}{\zh{尽力}}}
\end{exe}


\paragraph{\hspace{-0.5cm} {\Large \textcolor{darkblue}{\textbf{\ipa{ʁɑ˥}}} \textsubscript{1}}} \hypertarget{RA\string_T1}{}
\markboth{\textcolor{darkblue}{\textbf{\ipa{ʁɑ˥}}} \textsubscript{1}}{}
\textcolor{teal}{\mytextsc{verbe}} \hspace{4pt} Ton~: H.

Convier, faire venir, inviter.
\textcolor{brown}{\zh{请。}}




\paragraph{\hspace{-0.5cm} {\Large \textcolor{darkblue}{\textbf{\ipa{ʁɑ˥}}} \textsubscript{2}}} \hypertarget{RA\string_T2}{}
\markboth{\textcolor{darkblue}{\textbf{\ipa{ʁɑ˥}}} \textsubscript{2}}{}
\textcolor{teal}{\mytextsc{verbe}} \hspace{4pt} Ton~: H.

Gagner.
\textcolor{brown}{\zh{赢。}}


\begin{exe}
\sn \textcolor{darkblue}{\textbf{\ipa{le˧-ʁɑ˥-ze˩}}}
\trans \mytextsc{accomp} \string_ \mytextsc{pfv}
\trans \textit{\textcolor{brown}{\zh{赢了}}}
\end{exe}


\paragraph{\hspace{-0.5cm} {\Large \textcolor{darkblue}{\textbf{\ipa{ʁɑ˥ʂe˩}}}}} \hypertarget{RA\string_Ts`e\string_B1}{}
\markboth{\textcolor{darkblue}{\textbf{\ipa{ʁɑ˥ʂe˩}}}}{}
\textcolor{teal}{\mytextsc{verbe}} \hspace{4pt} Ton~: .

Faire venir, faire appel à (un moine, un médecin…).
\textcolor{brown}{\zh{请来(和尚、医生……)。}}


\begin{exe}
\sn \textcolor{darkblue}{\textbf{\ipa{ʈæ˧bɤ˧ ʁɑ˧-ʂe˩}}}
\trans faire venir un moine, faire appel à un moine
\trans \textit{\textcolor{brown}{\zh{请和尚到家来}}}
\end{exe}
\begin{exe}
\sn \textcolor{darkblue}{\textbf{\ipa{ʈʂʰæ˧ɣɯ˧ ki˩-hĩ˩ | le˧-ʁɑ˥-ʂe˩ | le˧-po˧-jo˥ / ʈʂʰæ˧ɣɯ˧ ki˩-hĩ˩ | ɖɯ˧-ʁɑ˥-ʂe˩-ɻ̍˩ hõ˩}}}
\trans faire venir le médecin
\trans \textit{\textcolor{brown}{\zh{请医生来}}}
\end{exe}
\begin{exe}
\sn \textcolor{darkblue}{\textbf{\ipa{njɤ˧-ɳɯ˧ | no˩ ʁɑ˩-ʂe˥}}}
\trans je t'invite/je te convie/je te fais venir
\trans \textit{\textcolor{brown}{\zh{我请你来}}}
\end{exe}
\begin{exe}
\sn \textcolor{darkblue}{\textbf{\ipa{mɤ˧-ʁɑ˥-ʂe˩}}}
\trans \mytextsc{neg}: ne pas faire appel à
\trans \textit{\textcolor{brown}{\zh{\mytextsc{neg}}}}
\end{exe}


\paragraph{\hspace{-0.5cm} {\Large \textcolor{darkblue}{\textbf{\ipa{ʁɑ˧}}} \textsubscript{1}}} \hypertarget{RA\string_M1}{}
\markboth{\textcolor{darkblue}{\textbf{\ipa{ʁɑ˧}}} \textsubscript{1}}{}
\textcolor{teal}{\mytextsc{adjectif}} \hspace{4pt} Ton~: M.

Bon, fiable: objet de bonne qualité; travail de bonne tenue; personne ayant bon caractère.
\textcolor{brown}{\zh{好(质量好,品质好,脾气好)。}}


\begin{exe}
\sn \textcolor{darkblue}{\textbf{\ipa{mɤ˧-ʁɑ˧-hĩ˧ ʂe˧}}}
\trans de la mauvaise viande
\trans \textit{\textcolor{brown}{\zh{不好的肉(质量不好)}}}
\end{exe}
\begin{exe}
\sn \textcolor{darkblue}{\textbf{\ipa{mɤ˧-ʁɑ˧-hĩ˧ ʂe˧-kʰwɤ˧ ki˩}}}
\trans donner un morceau de mauvaise viande
\trans \textit{\textcolor{brown}{\zh{给一块不好的肉}}}
\end{exe}
\begin{exe}
\sn \textcolor{darkblue}{\textbf{\ipa{pʰi˩ko˧ | mɤ˧-ʁɑ˧-ze˧!}}}
\trans Les pommes ne sont plus bonnes! (Contexte: au mois de mars, les pommes de la récolte précédente ne sont plus bonnes, elles sont fripées ou pourries.)
\trans \textit{\textcolor{brown}{\zh{苹果不好了! / 苹果不新鲜了!(三月份,上一季收获的苹果已经不好吃的了,或者烂了,或者变酸)}}}
\end{exe}
\begin{exe}
\sn \textcolor{darkblue}{\textbf{\ipa{hĩ˧ ɖɯ˧-v̩˧ | ʁɑ˧-mɤ˧-ʑi˧-hĩ˧ ʐwɤ˧˥!}}}
\trans quelqu'un dit n'importe quoi
\trans \textit{\textcolor{brown}{\zh{有人在乱说话!}}}
\end{exe}


\paragraph{\hspace{-0.5cm} {\Large \textcolor{darkblue}{\textbf{\ipa{ʁɑ˧}}} \textsubscript{2}}} \hypertarget{RA\string_M2}{}
\markboth{\textcolor{darkblue}{\textbf{\ipa{ʁɑ˧}}} \textsubscript{2}}{}
\textcolor{teal}{\mytextsc{verbe}} \hspace{4pt} Ton~: M.
\textit{Archaïque} [\textcolor{brown}{\zh{古语}}] 
Présenter ses excuses, demander pardon.
\textcolor{brown}{\zh{道歉、请人家原谅。}}


\begin{exe}
\sn \textcolor{darkblue}{\textbf{\ipa{ʁɑ˧-ze˧!}}}
\trans Pardon! (adressé à une personne de haut rang)
\trans \textit{\textcolor{brown}{\zh{抱歉! / 请原谅!(对地位比自己高的人说)}}}
\end{exe}


\paragraph{\hspace{-0.5cm} {\Large \textcolor{darkblue}{\textbf{\ipa{ʁɑ˧\textsubscript{b}}}}}} \hypertarget{RA\string_Mb1}{}
\markboth{\textcolor{darkblue}{\textbf{\ipa{ʁɑ˧\textsubscript{b}}}}}{}
\textcolor{teal}{\mytextsc{verbe}} \hspace{4pt} Ton~: M\textsubscript{b}.

Enjamber (un ruisseau...).
\textcolor{brown}{\zh{跨(跨过小沟)。}}


\begin{exe}
\sn \textcolor{darkblue}{\textbf{\ipa{le˧-ʁɑ˧-ze˧}}}
\trans \mytextsc{accomp} \string_ \mytextsc{pfv}
\trans \textit{\textcolor{brown}{\zh{跨过了}}}
\end{exe}


\paragraph{\hspace{-0.5cm} {\Large \textcolor{darkblue}{\textbf{\ipa{ʁɑ˧dzi˧}}}}} \hypertarget{RA\string_Mdzi\string_M1}{}
\markboth{\textcolor{darkblue}{\textbf{\ipa{ʁɑ˧dzi˧}}}}{}
\textcolor{teal}{\mytextsc{nom}} \hspace{4pt} Ton~: M.

Peuplier.
\textcolor{brown}{\zh{杨树。}}


 \mytextsc{clf}~: \textcolor{darkblue}{\textbf{\ipa{dzi˩}}} 

\paragraph{\hspace{-0.5cm} {\Large \textcolor{darkblue}{\textbf{\ipa{ʁɑ˧ɭɯ\#˥}}}}} \hypertarget{RA\string_Ml\string_RM\#\string_T1}{}
\markboth{\textcolor{darkblue}{\textbf{\ipa{ʁɑ˧ɭɯ\#˥}}}}{}
\textcolor{teal}{\mytextsc{nom}} \hspace{4pt} Ton~: \#H.

Cairn: tas de pierres qui aide à repérer un sentier.
\textcolor{brown}{\zh{石堆。}}


\begin{exe}
\sn \textcolor{darkblue}{\textbf{\ipa{qo˩qɑ˩-ʁɑ˥ɭɯ˩}}}
\trans cairn situé à un col
\trans \textit{\textcolor{brown}{\zh{垭口石堆:垭口上的石堆}}}
\end{exe}
 \mytextsc{clf}~: \textcolor{darkblue}{\textbf{\ipa{ɭɯ˧}}} 

\paragraph{\hspace{-0.5cm} {\Large \textcolor{darkblue}{\textbf{\ipa{ʁɑ˧pv̩˧}}}}} \hypertarget{RA\string_Mpv\string_=\string_M1}{}
\markboth{\textcolor{darkblue}{\textbf{\ipa{ʁɑ˧pv̩˧}}}}{}
\textcolor{teal}{\mytextsc{nom}} \hspace{4pt} Ton~: M.

Poitrine.
\textcolor{brown}{\zh{胸脯、胸膛。}}


 \mytextsc{clf}~: \textcolor{darkblue}{\textbf{\ipa{ʈv̩˩}}} 

\paragraph{\hspace{-0.5cm} {\Large \textcolor{darkblue}{\textbf{\ipa{ʁɑ˧pv̩˧-ɻ̃\#˥}}}}} \hypertarget{RA\string_Mpv\string_=\string_M-r£`\string_~\#\string_T1}{}
\markboth{\textcolor{darkblue}{\textbf{\ipa{ʁɑ˧pv̩˧-ɻ̃\#˥}}}}{}
\textcolor{teal}{\mytextsc{nom}} \hspace{4pt} Ton~: \#H.

Clavicule.
\textcolor{brown}{\zh{锁骨。}}


 \mytextsc{clf}~: \textcolor{darkblue}{\textbf{\ipa{pʰæ˧˥}}} 

\paragraph{\hspace{-0.5cm} {\Large \textcolor{darkblue}{\textbf{\ipa{ʁɑ˧pʰv̩\#˥}}}}} \hypertarget{RA\string_Mp\string_hv\string_=\#\string_T1}{}
\markboth{\textcolor{darkblue}{\textbf{\ipa{ʁɑ˧pʰv̩\#˥}}}}{}
\textcolor{teal}{\mytextsc{nom}} \hspace{4pt} Ton~: \#H.

Salaire, littéralement ``prix du travail".
\textcolor{brown}{\zh{工资, 工钱。}}


 \mytextsc{clf}~: \textcolor{darkblue}{\textbf{\ipa{kʰwɤ˥}}} 

\paragraph{\hspace{-0.5cm} {\Large \textcolor{darkblue}{\textbf{\ipa{ʁɑ˧-ʐwɤ˧˥}}}}} \hypertarget{RA\string_M-z`w7\string_M\string_T1}{}
\markboth{\textcolor{darkblue}{\textbf{\ipa{ʁɑ˧-ʐwɤ˧˥}}}}{}
\textcolor{teal}{\mytextsc{verbe}} \hspace{4pt} Ton~: MH\#.

Provoquer/humilier.
\textcolor{brown}{\zh{欺负。}}


\begin{exe}
\sn \textcolor{darkblue}{\textbf{\ipa{ʈʂʰɯ˧-v̩˧ | hĩ˧-ki˧ ʁɑ˧-ʐwɤ˧-ʝi˥!}}}
\trans elle/il provoque/humilie quelqu'un d'autre
\trans \textit{\textcolor{brown}{\zh{他欺负人、他对人发脾气}}}
\end{exe}
\begin{exe}
\sn \textcolor{darkblue}{\textbf{\ipa{ʁɑ˧ ʐwɤ˧-ɻ̍˥}}}
\trans même sens
\trans \textit{\textcolor{brown}{\zh{同上}}}
\end{exe}
\begin{exe}
\sn \textcolor{darkblue}{\textbf{\ipa{no˧ | ʁɑ˧ ʐwɤ˧-tʰɑ˧-ɻ̍˥!}}}
\trans Ne provoque pas les gens!
\trans \textit{\textcolor{brown}{\zh{你不要欺负人!}}}
\end{exe}


\paragraph{\hspace{-0.5cm} {\Large \textcolor{darkblue}{\textbf{\ipa{ʁɑ˩mi˥}}}}} \hypertarget{RA\string_Bmi\string_T1}{}
\markboth{\textcolor{darkblue}{\textbf{\ipa{ʁɑ˩mi˥}}}}{}
\textcolor{teal}{\mytextsc{verbe}} \hspace{4pt} Ton~: LH.

Demander pardon; formule de requête, et de remerciement. Le spectre des significations rappelle l'étymologie de ``merci": de ``crier merci" (implorer la vie sauve) à un emploi comme formule de politesse courante.
\textcolor{brown}{\zh{道歉。}}


\begin{exe}
\sn \textcolor{darkblue}{\textbf{\ipa{ʁɑ˩mi˥-ze˩!}}}
\trans Merci!
\trans \textit{\textcolor{brown}{\zh{谢谢!}}}
\end{exe}


\paragraph{\hspace{-0.5cm} {\Large \textcolor{darkblue}{\textbf{\ipa{ʁɑ˩ʂɯ˧}}}}} \hypertarget{RA\string_Bs`M\string_M1}{}
\markboth{\textcolor{darkblue}{\textbf{\ipa{ʁɑ˩ʂɯ˧}}}}{}
\textcolor{teal}{\mytextsc{adverbe}} \hspace{4pt} Ton~: LM.

En fait, en réalité.
\textcolor{brown}{\zh{其实、事实上。}}




\paragraph{\hspace{-0.5cm} {\Large \textcolor{darkblue}{\textbf{\ipa{ʁæ˥}}}}} \hypertarget{R\{\string_T1}{}
\markboth{\textcolor{darkblue}{\textbf{\ipa{ʁæ˥}}}}{}
\textcolor{teal}{\mytextsc{nom}} \hspace{4pt} Ton~: \#H.

Cou (monosyllabe; moins usité que le disyllabe).
\textcolor{brown}{\zh{脖子(单音节)。}}


 \mytextsc{clf}~: \textcolor{darkblue}{\textbf{\ipa{ɭɯ˧}}} \textit{Voir~:} \hyperlink{R\{\string_MNv\string_=\string_T1}{\textcolor{darkblue}{\textbf{\ipa{ʁæ˧ŋv̩˥}}}} 

\paragraph{\hspace{-0.5cm} {\Large \textcolor{darkblue}{\textbf{\ipa{ʁæ˧}}}}} \hypertarget{R\{\string_M1}{}
\markboth{\textcolor{darkblue}{\textbf{\ipa{ʁæ˧}}}}{}
\textcolor{teal}{\mytextsc{adjectif}} \hspace{4pt} Ton~: M.

Riche.
\textcolor{brown}{\zh{富。}}




\paragraph{\hspace{-0.5cm} {\Large \textcolor{darkblue}{\textbf{\ipa{ʁæ˧bæ˧}}}}} \hypertarget{R\{\string_Mb\{\string_M1}{}
\markboth{\textcolor{darkblue}{\textbf{\ipa{ʁæ˧bæ˧}}}}{}
\textcolor{teal}{\mytextsc{nom}} \hspace{4pt} Ton~: M.

Assiette.
\textcolor{brown}{\zh{盘子。}}


 \mytextsc{clf}~: \textcolor{darkblue}{\textbf{\ipa{ɭɯ˧}}} 

\paragraph{\hspace{-0.5cm} {\Large \textcolor{darkblue}{\textbf{\ipa{ʁæ˧ɭɯ˥}}}}} \hypertarget{R\{\string_Ml\string_RM\string_T1}{}
\markboth{\textcolor{darkblue}{\textbf{\ipa{ʁæ˧ɭɯ˥}}}}{}
\textcolor{teal}{\mytextsc{nom}} \hspace{4pt} Ton~: H\#.

Carcan (était en bois); littéralement ``[objet dans lequel] on met le cou".
\textcolor{brown}{\zh{枷锁(木头做的)。}}


\begin{exe}
\sn \textcolor{darkblue}{\textbf{\ipa{ʁæ˧ɭɯ˥ | ɖɯ˧-ɭɯ˧ kʰɯ˧˥}}}
\trans mettre un joug (à quelqu'un)
\trans \textit{\textcolor{brown}{\zh{套上一个枷锁(在一个人的脖子上)}}}
\end{exe}
\begin{exe}
\sn \textcolor{darkblue}{\textbf{\ipa{ʁæ˧ɭɯ˥ kʰɯ˩}}}
\trans mettre le joug (à quelqu'un)
\trans \textit{\textcolor{brown}{\zh{套上枷锁(在一个人的脖子上)}}}
\end{exe}


\paragraph{\hspace{-0.5cm} {\Large \textcolor{darkblue}{\textbf{\ipa{ʁæ˧mi˧}}}}} \hypertarget{R\{\string_Mmi\string_M1}{}
\markboth{\textcolor{darkblue}{\textbf{\ipa{ʁæ˧mi˧}}}}{}
\textcolor{teal}{\mytextsc{nom}} \hspace{4pt} Ton~: M.

Épée.
\textcolor{brown}{\zh{剑。}}


 \mytextsc{clf}~: \textcolor{darkblue}{\textbf{\ipa{nɑ˧}}} 

\paragraph{\hspace{-0.5cm} {\Large \textcolor{darkblue}{\textbf{\ipa{ʁæ˧ŋv̩˥}}}}} \hypertarget{R\{\string_MNv\string_=\string_T1}{}
\markboth{\textcolor{darkblue}{\textbf{\ipa{ʁæ˧ŋv̩˥}}}}{}
\textcolor{teal}{\mytextsc{nom}} \hspace{4pt} Ton~: H\#.

Col (précieux, avec des fils d'argent).
\textcolor{brown}{\zh{银衣领。}}


 \mytextsc{clf}~: \textcolor{darkblue}{\textbf{\ipa{ɭɯ˧}}} 

\paragraph{\hspace{-0.5cm} {\Large \textcolor{darkblue}{\textbf{\ipa{ʁæ˧ɻ̍˥}}}}} \hypertarget{R\{\string_Mr£`̍\string_T1}{}
\markboth{\textcolor{darkblue}{\textbf{\ipa{ʁæ˧ɻ̍˥}}}}{}
\textcolor{teal}{\mytextsc{nom}} \hspace{4pt} Ton~: H\#.

Cou.
\textcolor{brown}{\zh{脖子。}}


 \mytextsc{clf}~: \textcolor{darkblue}{\textbf{\ipa{ɭɯ˧}}} \textit{Voir~:} \hyperlink{R\{\string_Mt`v\string_=\string_T1}{\textcolor{darkblue}{\textbf{\ipa{ʁæ˧ʈv̩˥}}}} 

\paragraph{\hspace{-0.5cm} {\Large \textcolor{darkblue}{\textbf{\ipa{ʁæ˧tɑ˩}}}}} \hypertarget{R\{\string_MtA\string_B1}{}
\markboth{\textcolor{darkblue}{\textbf{\ipa{ʁæ˧tɑ˩}}}}{}
\textcolor{teal}{\mytextsc{nom}} \hspace{4pt} Ton~: L\#.

Garrot: partie du corps de l'animal sur lequel repose le joug.
\textcolor{brown}{\zh{肩隆。}}


\begin{exe}
\sn \textcolor{darkblue}{\textbf{\ipa{ʝi˧-ʁæ˧tɑ˥}}}
\trans garrot de vache
\trans \textit{\textcolor{brown}{\zh{牛肩隆}}}
\end{exe}
\begin{exe}
\sn \textcolor{darkblue}{\textbf{\ipa{ʁæ˧tɑ˩ tʰv̩˩-ɭɯ˩}}}
\trans \mytextsc{n}+\mytextsc{dem}+\mytextsc{clf}
\trans \textit{\textcolor{brown}{\zh{这只肩隆}}}
\end{exe}
 \mytextsc{clf}~: \textcolor{darkblue}{\textbf{\ipa{ɭɯ˧}}} 

\paragraph{\hspace{-0.5cm} {\Large \textcolor{darkblue}{\textbf{\ipa{ʁæ˧ʈv̩˥}}}}} \hypertarget{R\{\string_Mt`v\string_=\string_T1}{}
\markboth{\textcolor{darkblue}{\textbf{\ipa{ʁæ˧ʈv̩˥}}}}{}
\textcolor{teal}{\mytextsc{nom}} \hspace{4pt} Ton~: H\#.

Cou.
\textcolor{brown}{\zh{脖子。}}


 \mytextsc{clf}~: \textcolor{darkblue}{\textbf{\ipa{ɭɯ˧}}} \textit{Voir~:} \hyperlink{R\{\string_Mr£`̍\string_T1}{\textcolor{darkblue}{\textbf{\ipa{ʁæ˧ɻ̍˥}}}} 

\paragraph{\hspace{-0.5cm} {\Large \textcolor{darkblue}{\textbf{\ipa{ʁæ˧zo\#˥}}}}} \hypertarget{R\{\string_Mzo\#\string_T1}{}
\markboth{\textcolor{darkblue}{\textbf{\ipa{ʁæ˧zo\#˥}}}}{}
\textcolor{teal}{\mytextsc{nom}} \hspace{4pt} Ton~: \#H.

Petite épée.
\textcolor{brown}{\zh{短剑。}}




\paragraph{\hspace{-0.5cm} {\Large \textcolor{darkblue}{\textbf{\ipa{ʁæ˧ʑi˧}}}}} \hypertarget{R\{\string_Mz£i\string_M1}{}
\markboth{\textcolor{darkblue}{\textbf{\ipa{ʁæ˧ʑi˧}}}}{}
\textcolor{teal}{\mytextsc{verbe}} \hspace{4pt} Ton~: M.

S'occuper de; se mêler de; prendre en considération.
\textcolor{brown}{\zh{考虑。}}


\begin{exe}
\sn \textcolor{darkblue}{\textbf{\ipa{hĩ˧ | qʰɑ˧-kv̩˧ dʑo˧˥ | mɤ˧-ʁæ˧ʑi˧, | njɤ˧-ɳɯ˧ qʰæ˧˥! |}}}
\trans Moi, j'aide, sans m'inquiéter de savoir combien il y a d'invités (littéralement ``de gens")! (contexte: F4 explique comment on se dévouait autrefois pour aider les amis, non membres de la famille, lors des grandes occasions, telles que les funérailles)
\trans \textit{\textcolor{brown}{\zh{无论有多少个人,我都会去帮助!(情景:合作人描写她在永宁有大事时怎么去帮其它家庭的忙,不考虑活多么累,只考虑怎么能给予帮助)}}}
\end{exe}
\begin{exe}
\sn \textcolor{darkblue}{\textbf{\ipa{no˧ | mɤ˧-ʁæ˧ʑi˧!}}}
\trans Fiche-moi la paix! / Laisse-moi tranquille! / Mêle-toi de tes affaires!
\trans \textit{\textcolor{brown}{\zh{别管我了! / 请让我安静! / 请不要打扰我了!}}}
\end{exe}


\paragraph{\hspace{-0.5cm} {\Large \textcolor{darkblue}{\textbf{\ipa{ʁæ˩\textsubscript{a}}}} \textsubscript{1}}} \hypertarget{R\{\string_Ba1}{}
\markboth{\textcolor{darkblue}{\textbf{\ipa{ʁæ˩\textsubscript{a}}}} \textsubscript{1}}{}
\textcolor{teal}{\mytextsc{verbe}} \hspace{4pt} Ton~: L\textsubscript{a}.

Se défaire, fondre, se dissoudre: une motte de terre plongée dans l'eau se défait.
\textcolor{brown}{\zh{散、散开,化,溶化(一块土在水里面散开)。}}


\begin{exe}
\sn \textcolor{darkblue}{\textbf{\ipa{le˧-ʁæ˩-ze˩}}}
\trans \mytextsc{accomp} \string_ \mytextsc{pfv}
\trans \textit{\textcolor{brown}{\zh{\mytextsc{accomp} \string_ \mytextsc{pfv}}}}
\end{exe}
\begin{exe}
\sn \textcolor{darkblue}{\textbf{\ipa{le˧-ʁæ˧\textasciitilde{}ʁæ˥ (-ze˩ / -bi˩)}}}
\trans \mytextsc{red}
\trans \textit{\textcolor{brown}{\zh{\mytextsc{重叠}}}}
\end{exe}
\begin{exe}
\sn \textcolor{darkblue}{\textbf{\ipa{ɖɯ˧-kʰwɤ˧ ʁæ˥}}}
\trans un morceau (de terre) se défait
\trans \textit{\textcolor{brown}{\zh{一块(土)散开}}}
\end{exe}
\begin{exe}
\sn \textcolor{darkblue}{\textbf{\ipa{ʈʂe˧ʈv̩˥ | le˧-ʁæ˩-ze˩}}}
\trans les mottes de terre se défond, se dissolvent (dans l'eau dont on inonde les champs après les labours)
\trans \textit{\textcolor{brown}{\zh{土块散开在了(耕田后灌溉,土块散在水里)}}}
\end{exe}


\paragraph{\hspace{-0.5cm} {\Large \textcolor{darkblue}{\textbf{\ipa{ʁæ˩\textsubscript{a}}}} \textsubscript{2}}} \hypertarget{R\{\string_Ba2}{}
\markboth{\textcolor{darkblue}{\textbf{\ipa{ʁæ˩\textsubscript{a}}}} \textsubscript{2}}{}
\textcolor{teal}{\mytextsc{adjectif}} \hspace{4pt} Ton~: L\textsubscript{a}.

Ivre, saoul.
\textcolor{brown}{\zh{醉。}}


\begin{exe}
\sn \textcolor{darkblue}{\textbf{\ipa{ʐɯ˧ ʁæ˩(-ze˩)}}}
\trans ivre d'alcool
\trans \textit{\textcolor{brown}{\zh{醉酒}}}
\end{exe}


\paragraph{\hspace{-0.5cm} {\Large \textcolor{darkblue}{\textbf{\ipa{ʁæ˩\textsubscript{a}}}} \textsubscript{3}}} \hypertarget{R\{\string_Ba3}{}
\markboth{\textcolor{darkblue}{\textbf{\ipa{ʁæ˩\textsubscript{a}}}} \textsubscript{3}}{}
\textcolor{teal}{\mytextsc{adjectif}} \hspace{4pt} Ton~: L\textsubscript{a}.

Approprié; propice, favorable.
\textcolor{brown}{\zh{合适,吉利。}}


\begin{exe}
\sn \textcolor{darkblue}{\textbf{\ipa{ʁæ˧ mɤ˧-ʑi˧}}}
\trans ce n'est pas propice/favorable
\trans \textit{\textcolor{brown}{\zh{不吉利、不合适}}}
\end{exe}
\begin{exe}
\sn \textcolor{darkblue}{\textbf{\ipa{ʁæ˧ mɤ˧-ʑi˧, | ʝi˧ mɤ˧-tʰɑ˩! / ʝi˧-mɤ˧-ɖo˧!}}}
\trans les circonstances ne sont pas propices / ce n'est pas une bonne idée, il ne faut pas le faire! (Mise en garde)
\trans \textit{\textcolor{brown}{\zh{不吉利 / 不合适(的事情),不能做!/ 不要做!(警告)}}}
\end{exe}
\begin{exe}
\sn \textcolor{darkblue}{\textbf{\ipa{ʁæ˧ mɤ˧-ʑi˧, | ʐwɤ˩ mɤ˩-tʰɑ˥! / ʁæ˧ mɤ˧-ʑi˧, | ʐwɤ˩ mɤ˩-ɖo˩˥!}}}
\trans Ca ne convient pas; on ne peut pas le dire / on ne doit pas le dire! / Il ne faut pas tenir de propos inappropriés! / Il faut faire attention à ce qu'on dit! (Mise en garde)
\trans \textit{\textcolor{brown}{\zh{不合适(的话),不能说! / 不合适(的话),不要说!(警告)}}}
\end{exe}
\begin{exe}
\sn \textcolor{darkblue}{\textbf{\ipa{ʁæ˧-mɤ˧-ʑi˧, | tɕi˩-mɤ˩-ɖo˩˥!}}}
\trans Il ne faut pas transcrire ceux qui sont pas bons! Il ne faut pas écrire n'importe quoi! (Contexte: cette phrase récapitule le principe qui préside au choix des récits à transcrire et traduire. J'expliquais de mon mieux mon souhait de choisir, parmi les récits enregistrés --relativement nombreux--, ceux qui sont les plus intéressants, et les plus réussis. Par cet énoncé, la locutrice apporte son assentiment, en même temps qu'elle indique qu'elle comprend l'idée: il faut transcrire les récits qui sont bons; il faut bien choisir, et écarter ceux qui ne seraient pas appropriés en quoi que ce soit. Par le même énoncé, la locutrice témoigne en outre de sa modestie: elle ne défend pas l'idée selon laquelle tous ses récits sont d'égale qualité, et accepte de bonne grâce l'idée que certains sont plus réussis que d'autres, ou plus adéquats pour le propos du linguiste.)
\trans \textit{\textcolor{brown}{\zh{乱七八糟的,不要记录! / 不好的,不要记录!(情景:选择一个故事来做记音翻译等。合作人提出,要考虑好记录哪些、选择好的资料,不能什么都记录。)}}}
\end{exe}
\begin{exe}
\sn \textcolor{darkblue}{\textbf{\ipa{ʈʂʰɯ˧ | lo˧ | ʁæ˧-mɤ˧-ʑi˧ ʝi˧!}}}
\trans Il ne fait pas attention dans son travail! il ne travaille pas avec soin! il fait n'importe quoi!
\trans \textit{\textcolor{brown}{\zh{他工作做得乱七八糟!}}}
\end{exe}


\paragraph{\hspace{-0.5cm} {\Large \textcolor{darkblue}{\textbf{\ipa{ʁæ˩\textsubscript{a}}}} \textsubscript{4}}} \hypertarget{R\{\string_Ba4}{}
\markboth{\textcolor{darkblue}{\textbf{\ipa{ʁæ˩\textsubscript{a}}}} \textsubscript{4}}{}
\textcolor{teal}{\mytextsc{adjectif}} \hspace{4pt} Ton~: L\textsubscript{a}.

Laid, vilain.
\textcolor{brown}{\zh{丑陋。}}


\textit{Voir~:} \hyperlink{d`z`v\string_=\string_Ba1}{\textcolor{darkblue}{\textbf{\ipa{ɖʐv̩˩a}}} \textsubscript{1}} 

\paragraph{\hspace{-0.5cm} {\Large \textcolor{darkblue}{\textbf{\ipa{ʁæ˩bæ˩}}}}} \hypertarget{R\{\string_Bb\{\string_B1}{}
\markboth{\textcolor{darkblue}{\textbf{\ipa{ʁæ˩bæ˩}}}}{}
\textcolor{teal}{\mytextsc{classificateur}} \hspace{4pt} Ton~: .

Assiettée, contenue d'une assiette.
\textcolor{brown}{\zh{量词:盘。}}


\begin{exe}
\sn \textcolor{darkblue}{\textbf{\ipa{ɖɯ˧-ʁæ˩bæ˩, | ɲi˧-ʁæ˩bæ˩, | so˩-ʁæ˩bæ˩˥, | ʐv̩˧-ʁæ˥bæ˩, | ŋwɤ˧-ʁæ˥bæ˩, | qʰv̩˧-ʁæ˥bæ˩, | ʂɯ˧-ʁæ˩bæ˩, | hõ˧-ʁæ˥bæ˩, | gv̩˧-ʁæ˥bæ˩, | tsʰe˩-ʁæ˩bæ˩˥}}}
\trans association avec des numéraux, de 1 à 10
\trans \textit{\textcolor{brown}{\zh{与数词结合,一至十}}}
\end{exe}


\paragraph{\hspace{-0.5cm} {\Large \textcolor{darkblue}{\textbf{\ipa{ʁæ˩ɖʐv̩˧}}}}} \hypertarget{R\{\string_Bd`z`v\string_=\string_M1}{}
\markboth{\textcolor{darkblue}{\textbf{\ipa{ʁæ˩ɖʐv̩˧}}}}{}
\textcolor{teal}{\mytextsc{verbe}} \hspace{4pt} Ton~: .

Laid, vilain.
\textcolor{brown}{\zh{丑陋。}}




\paragraph{\hspace{-0.5cm} {\Large \textcolor{darkblue}{\textbf{\ipa{ʁæ˧˥}}}}} \hypertarget{R\{\string_M\string_T1}{}
\markboth{\textcolor{darkblue}{\textbf{\ipa{ʁæ˧˥}}}}{}
\textcolor{teal}{\mytextsc{adjectif}} \hspace{4pt} Ton~: MH.

Écoeurant, dégoûtant, pas bon au goût (pas forcément à cause d'un excès de graisse: par exemple, selon les critères gastronomiques locaux, mes flocons d'avoine entrent dans cette catégorie).
\textcolor{brown}{\zh{不好吃,恶心。}}




\paragraph{\hspace{-0.5cm} {\Large \textcolor{darkblue}{\textbf{\ipa{ʁæ˩˥}}}}} \hypertarget{R\{\string_B\string_T1}{}
\markboth{\textcolor{darkblue}{\textbf{\ipa{ʁæ˩˥}}}}{}
\textcolor{teal}{\mytextsc{nom}} \hspace{4pt} Ton~: LH.

Sève, résine.
\textcolor{brown}{\zh{树液。}}


\begin{exe}
\sn \textcolor{darkblue}{\textbf{\ipa{tʰo˩ʁæ˩˥}}}
\trans même sens
\trans \textit{\textcolor{brown}{\zh{同上}}}
\end{exe}


\paragraph{\hspace{-0.5cm} {\Large \textcolor{darkblue}{\textbf{\ipa{‑ʁo}}}}} \hypertarget{‑Ro1}{}
\markboth{\textcolor{darkblue}{\textbf{\ipa{‑ʁo}}}}{}
\textcolor{teal}{\mytextsc{postposition}} \hspace{4pt} Ton~: .

Sur.
\textcolor{brown}{\zh{上面。}}




\paragraph{\hspace{-0.5cm} {\Large \textcolor{darkblue}{\textbf{\ipa{ʁo˥}}}}} \hypertarget{Ro\string_T1}{}
\markboth{\textcolor{darkblue}{\textbf{\ipa{ʁo˥}}}}{}
\textcolor{teal}{\mytextsc{nom}} \hspace{4pt} Ton~: \#H.
\ding{202} 
Tête (monosyllabique).
\textcolor{brown}{\zh{头(单音节)。}}


\ding{203} 
Début.
\textcolor{brown}{\zh{开头。}}


\begin{exe}
\sn \textcolor{darkblue}{\textbf{\ipa{ɬi˧-ʁo\#˥}}}
\trans le début du mois
\trans \textit{\textcolor{brown}{\zh{月初}}}
\end{exe}
\begin{exe}
\sn \textcolor{darkblue}{\textbf{\ipa{kʰv̩˧-ʁo˥\$}}}
\trans le début de l'année
\trans \textit{\textcolor{brown}{\zh{年初}}}
\end{exe}
\begin{exe}
\sn \textcolor{darkblue}{\textbf{\ipa{*ɲi˧-ʁo˩}}}
\trans *le début de la journée
\trans \textit{\textcolor{brown}{\zh{*天初}}}
\end{exe}
 \mytextsc{clf}~: \textcolor{darkblue}{\textbf{\ipa{ɭɯ˧}}} 

\paragraph{\hspace{-0.5cm} {\Large \textcolor{darkblue}{\textbf{\ipa{ʁo˥pv̩˩}}}}} \hypertarget{Ro\string_Tpv\string_=\string_B1}{}
\markboth{\textcolor{darkblue}{\textbf{\ipa{ʁo˥pv̩˩}}}}{}
\textcolor{teal}{\mytextsc{verbe}} \hspace{4pt} Ton~: .
\ding{202} 
Croiser, faire la rencontre de quelqu'un (typiquement: sur un sentier, rencontrer quelqu'un qui marche en sens inverse).
\textcolor{brown}{\zh{遇见(如:在路上遇见一个人)。}}


\begin{exe}
\sn \textcolor{darkblue}{\textbf{\ipa{tʰi˧-ʁo˥pv̩˩}}}
\trans \mytextsc{dur}
\trans \textit{\textcolor{brown}{\zh{\mytextsc{dur}}}}
\end{exe}
\ding{203} 
Intercepter, attaquer en route (des brigands attaquent une caravane).
\textcolor{brown}{\zh{阻截、在路上攻击。}}




\paragraph{\hspace{-0.5cm} {\Large \textcolor{darkblue}{\textbf{\ipa{ʁo˥-ʐv̩˩}}}}} \hypertarget{Ro\string_T-z`v\string_=\string_B1}{}
\markboth{\textcolor{darkblue}{\textbf{\ipa{ʁo˥-ʐv̩˩}}}}{}
\textcolor{teal}{\mytextsc{verbe}} \hspace{4pt} Ton~: .

Bénir et protéger.
\textcolor{brown}{\zh{保佑。}}


\begin{exe}
\sn \textcolor{darkblue}{\textbf{\ipa{mɤ˧-ʁo˥ʐv̩˩}}}
\trans \mytextsc{neg}
\trans \textit{\textcolor{brown}{\zh{\mytextsc{neg}}}}
\end{exe}
\begin{exe}
\sn \textcolor{darkblue}{\textbf{\ipa{gɤ˧lɑ˧ | ɖɯ˧-ʁo˥ʐv̩˩-ɻ̍˩!}}}
\trans Que les esprits [te/nous] bénissent!
\trans \textit{\textcolor{brown}{\zh{菩萨保佑!}}}
\end{exe}


\paragraph{\hspace{-0.5cm} {\Large \textcolor{darkblue}{\textbf{\ipa{ʁo˧}}} \textsubscript{1}}} \hypertarget{Ro\string_M1}{}
\markboth{\textcolor{darkblue}{\textbf{\ipa{ʁo˧}}} \textsubscript{1}}{}
\textcolor{teal}{\mytextsc{verbe}} \hspace{4pt} Ton~: M intrans.

Pondre.
\textcolor{brown}{\zh{下蛋。}}


\begin{exe}
\sn \textcolor{darkblue}{\textbf{\ipa{æ˩ ʁo˥}}}
\trans pondre des œufs
\trans \textit{\textcolor{brown}{\zh{下蛋}}}
\end{exe}
\begin{exe}
\sn \textcolor{darkblue}{\textbf{\ipa{æ˩mi˧ tʰi˧-ʁo˧-dʑo˧!}}}
\trans la poule est en train de pondre!
\trans \textit{\textcolor{brown}{\zh{母鸡在下蛋!}}}
\end{exe}
\begin{exe}
\sn \textcolor{darkblue}{\textbf{\ipa{æ˩mi˧ | æ˩ ʁo˧-ze˩!}}}
\trans la poule a pondu!
\trans \textit{\textcolor{brown}{\zh{母鸡下蛋了!}}}
\end{exe}


\paragraph{\hspace{-0.5cm} {\Large \textcolor{darkblue}{\textbf{\ipa{ʁo˧}}} \textsubscript{2}}} \hypertarget{Ro\string_M2}{}
\markboth{\textcolor{darkblue}{\textbf{\ipa{ʁo˧}}} \textsubscript{2}}{}
\textcolor{teal}{\mytextsc{verbe}} \hspace{4pt} Ton~: M intrans.

Arriver à, parvenir à.
\textcolor{brown}{\zh{能……、有能力做。}}


\begin{exe}
\sn \textcolor{darkblue}{\textbf{\ipa{njɤ˧ | tɕi˩-mɤ˩-ʁo˩˥!}}}
\trans je ne parviens pas à écrire/je ne sais pas écrire!
\trans \textit{\textcolor{brown}{\zh{我写不出来! / 我不会写!}}}
\end{exe}
\begin{exe}
\sn \textcolor{darkblue}{\textbf{\ipa{njɤ˧ | tɕi˩-ʁo˩˥!}}}
\trans je parviens à écrire/je sais écrire!
\trans \textit{\textcolor{brown}{\zh{我会写! / 我写得出来!}}}
\end{exe}


\paragraph{\hspace{-0.5cm} {\Large \textcolor{darkblue}{\textbf{\ipa{ʁo˧bv̩˧}}}}} \hypertarget{Ro\string_Mbv\string_=\string_M1}{}
\markboth{\textcolor{darkblue}{\textbf{\ipa{ʁo˧bv̩˧}}}}{}
\textcolor{teal}{\mytextsc{nom}} \hspace{4pt} Ton~: M.

Pousse d'arbre.
\textcolor{brown}{\zh{树的萌芽、新发出来的叶子。}}


\begin{exe}
\sn \textcolor{darkblue}{\textbf{\ipa{tʰo˧-ʁo˧bv˥}}}
\trans pousse de sapin
\trans \textit{\textcolor{brown}{\zh{小松树尖}}}
\end{exe}
\begin{exe}
\sn \textcolor{darkblue}{\textbf{\ipa{tʰo˩ʂv˩-ʁo˥bv˩}}}
\trans pousse de sapin; littéralement ``pousses d'aiguilles de sapin"
\trans \textit{\textcolor{brown}{\zh{小松树尖}}}
\end{exe}
 \mytextsc{clf}~: \textcolor{darkblue}{\textbf{\ipa{kʰwɤ˥}}} 

\paragraph{\hspace{-0.5cm} {\Large \textcolor{darkblue}{\textbf{\ipa{ʁo˧dɑ˧}}}}} \hypertarget{Ro\string_MdA\string_M1}{}
\markboth{\textcolor{darkblue}{\textbf{\ipa{ʁo˧dɑ˧}}}}{}
\textcolor{teal}{\mytextsc{adverbe}} \hspace{4pt} Ton~: M.

Devant, avant, auparavant.
\textcolor{brown}{\zh{前面,之前。}}


\begin{exe}
\sn \textcolor{darkblue}{\textbf{\ipa{ʂɯ˧-kʰv̩˧-ʁo˧dɑ˧}}}
\trans il y a sept ans
\trans \textit{\textcolor{brown}{\zh{七年前}}}
\end{exe}
\begin{exe}
\sn \textcolor{darkblue}{\textbf{\ipa{ʁo˧dɑ˧ ɖɯ˧-so˩ ɲi˩}}}
\trans ces derniers jours, les quelques jours passés, il y a quelques jours
\trans \textit{\textcolor{brown}{\zh{前几天}}}
\end{exe}


\paragraph{\hspace{-0.5cm} {\Large \textcolor{darkblue}{\textbf{\ipa{ʁo˧do˧}}} \textsubscript{1}}} \hypertarget{Ro\string_Mdo\string_M1}{}
\markboth{\textcolor{darkblue}{\textbf{\ipa{ʁo˧do˧}}} \textsubscript{1}}{}
\textcolor{teal}{\mytextsc{nom}} \hspace{4pt} Ton~: M.

Noix.
\textcolor{brown}{\zh{核桃。}}


\begin{exe}
\sn \textcolor{darkblue}{\textbf{\ipa{ʁo˧do˧ qʰwæ˧˥}}}
\trans casser des noix
\trans \textit{\textcolor{brown}{\zh{开核桃}}}
\end{exe}
\begin{exe}
\sn \textcolor{darkblue}{\textbf{\ipa{ʁo˧do˧ ʐwæ˧}}}
\trans peser des noix
\trans \textit{\textcolor{brown}{\zh{称核桃}}}
\end{exe}
 \mytextsc{clf}~: \textcolor{darkblue}{\textbf{\ipa{ɭɯ˧}}} 

\paragraph{\hspace{-0.5cm} {\Large \textcolor{darkblue}{\textbf{\ipa{ʁo˧do˧}}} \textsubscript{2}}} \hypertarget{Ro\string_Mdo\string_M2}{}
\markboth{\textcolor{darkblue}{\textbf{\ipa{ʁo˧do˧}}} \textsubscript{2}}{}
\textcolor{teal}{\mytextsc{nom}} \hspace{4pt} Ton~: M.

Intérêts.
\textcolor{brown}{\zh{利息。}}


 \mytextsc{clf}~: \textcolor{darkblue}{\textbf{\ipa{kʰwɤ˥}}} 

\paragraph{\hspace{-0.5cm} {\Large \textcolor{darkblue}{\textbf{\ipa{ʁo˧dzi˥}}}}} \hypertarget{Ro\string_Mdzi\string_T1}{}
\markboth{\textcolor{darkblue}{\textbf{\ipa{ʁo˧dzi˥}}}}{}
\textcolor{teal}{\mytextsc{verbe}} \hspace{4pt} Ton~: H\#.

Heurter.
\textcolor{brown}{\zh{碰撞。}}


\begin{exe}
\sn \textcolor{darkblue}{\textbf{\ipa{le˧-ʁo˧dzi˥}}}
\trans \mytextsc{accomp}
\trans \textit{\textcolor{brown}{\zh{\mytextsc{accomp}}}}
\end{exe}
\begin{exe}
\sn \textcolor{darkblue}{\textbf{\ipa{hĩ˧ | tʰi˧-ʁo˧dzi˥ tsʰɯ˩(-ze˩)}}}
\trans deux personnes se sont heurtées/sont entrées en collision
\trans \textit{\textcolor{brown}{\zh{人们(互相)碰撞}}}
\end{exe}


\paragraph{\hspace{-0.5cm} {\Large \textcolor{darkblue}{\textbf{\ipa{ʁo˧dzi˩}}}}} \hypertarget{Ro\string_Mdzi\string_B1}{}
\markboth{\textcolor{darkblue}{\textbf{\ipa{ʁo˧dzi˩}}}}{}
\textcolor{teal}{\mytextsc{nom}} \hspace{4pt} Ton~: L\#.

Tibétain.
\textcolor{brown}{\zh{藏族。}}


 \mytextsc{clf}~: \textcolor{darkblue}{\textbf{\ipa{v̩˧}}} 

\paragraph{\hspace{-0.5cm} {\Large \textcolor{darkblue}{\textbf{\ipa{ʁo˧dzi˩-di˩}}}}} \hypertarget{Ro\string_Mdzi\string_B-di\string_B1}{}
\markboth{\textcolor{darkblue}{\textbf{\ipa{ʁo˧dzi˩-di˩}}}}{}
\textcolor{teal}{\mytextsc{nom}} \hspace{4pt} Ton~: .

Nord (littéralement 'les contrées tibétaines').
\textcolor{brown}{\zh{北方(直译:‘藏族地区’)。}}




\paragraph{\hspace{-0.5cm} {\Large \textcolor{darkblue}{\textbf{\ipa{ʁo˧dzi˩-di˩}}}}} \hypertarget{Ro\string_Mdzi\string_B-di\string_B1}{}
\markboth{\textcolor{darkblue}{\textbf{\ipa{ʁo˧dzi˩-di˩}}}}{}
\textcolor{teal}{\mytextsc{nom}} \hspace{4pt} Ton~: L\#-.

Le Tibet (littéralement: 'la contrée des Tibétains').
\textcolor{brown}{\zh{西藏。}}




\paragraph{\hspace{-0.5cm} {\Large \textcolor{darkblue}{\textbf{\ipa{ʁo˧dzi˩-tʰæ˩ɻæ˩}}}}} \hypertarget{Ro\string_Mdzi\string_B-t\string_h\{\string_Br£`\{\string_B1}{}
\markboth{\textcolor{darkblue}{\textbf{\ipa{ʁo˧dzi˩-tʰæ˩ɻæ˩}}}}{}
\textcolor{teal}{\mytextsc{nom}} \hspace{4pt} Ton~: L\#-.

Drapeau, fanion (littéralement ``écritures tibétaines").
\textcolor{brown}{\zh{旗子。}}


 \mytextsc{clf}~: \textcolor{darkblue}{\textbf{\ipa{pʰæ˧˥}}} 

\paragraph{\hspace{-0.5cm} {\Large \textcolor{darkblue}{\textbf{\ipa{ʁo˧dʑɯ˧}}}}} \hypertarget{Ro\string_Mdz£M\string_M1}{}
\markboth{\textcolor{darkblue}{\textbf{\ipa{ʁo˧dʑɯ˧}}}}{}
\textcolor{teal}{\mytextsc{nom}} \hspace{4pt} Ton~: M.

Œillères.
\textcolor{brown}{\zh{马笼头。}}


\begin{exe}
\sn \textcolor{darkblue}{\textbf{\ipa{ʐwæ˧-ʁo˧dʑɯ˥ (ʈʂʰɯ˧ | ʐwæ˧-ʁo˧dʑɯ˥ ɲi˩)}}}
\trans œillères de cheval
\trans \textit{\textcolor{brown}{\zh{马笼头}}}
\end{exe}
 \mytextsc{clf}~: \textcolor{darkblue}{\textbf{\ipa{nɑ˧}}} \textcolor{darkblue}{\textbf{\ipa{pɤ˩}}} 

\paragraph{\hspace{-0.5cm} {\Large \textcolor{darkblue}{\textbf{\ipa{ʁo˧ɖɯ˧˥}}}}} \hypertarget{Ro\string_Md`M\string_M\string_T1}{}
\markboth{\textcolor{darkblue}{\textbf{\ipa{ʁo˧ɖɯ˧˥}}}}{}
\textcolor{teal}{\mytextsc{nom}} \hspace{4pt} Ton~: MH\#.

Têtard.
\textcolor{brown}{\zh{蝌蚪。}}




\paragraph{\hspace{-0.5cm} {\Large \textcolor{darkblue}{\textbf{\ipa{ʁo˧gv̩\#˥}}}}} \hypertarget{Ro\string_Mgv\string_=\#\string_T1}{}
\markboth{\textcolor{darkblue}{\textbf{\ipa{ʁo˧gv̩\#˥}}}}{}
\textcolor{teal}{\mytextsc{nom}} \hspace{4pt} Ton~: \#H.

Oreiller.
\textcolor{brown}{\zh{枕头。}}


 \mytextsc{clf}~: \textcolor{darkblue}{\textbf{\ipa{ɭɯ˧}}} 

\paragraph{\hspace{-0.5cm} {\Large \textcolor{darkblue}{\textbf{\ipa{ʁo˧hṽ˧˥}}}}} \hypertarget{Ro\string_Mhv\string_~\string_M\string_T1}{}
\markboth{\textcolor{darkblue}{\textbf{\ipa{ʁo˧hṽ˧˥}}}}{}
\textcolor{teal}{\mytextsc{nom}} \hspace{4pt} Ton~: MH\#.

Cheveux.
\textcolor{brown}{\zh{头发。}}


 \mytextsc{clf}~: \textcolor{darkblue}{\textbf{\ipa{kʰɯ˩}}} 

\paragraph{\hspace{-0.5cm} {\Large \textcolor{darkblue}{\textbf{\ipa{ʁo˧ʝi˧}}}}} \hypertarget{Ro\string_Mj££i\string_M1}{}
\markboth{\textcolor{darkblue}{\textbf{\ipa{ʁo˧ʝi˧}}}}{}
\textcolor{teal}{\mytextsc{adverbe}} \hspace{4pt} Ton~: M.

Dans deux ans.
\textcolor{brown}{\zh{后年。}}


\begin{exe}
\sn \textcolor{darkblue}{\textbf{\ipa{ʁo˧ʝi˧ ɖɯ˧-kʰv̩˧˥}}}
\trans l'année dans deux ans
\trans \textit{\textcolor{brown}{\zh{后年}}}
\end{exe}


\paragraph{\hspace{-0.5cm} {\Large \textcolor{darkblue}{\textbf{\ipa{ʁo˧kɤ˩}}}}} \hypertarget{Ro\string_Mk7\string_B1}{}
\markboth{\textcolor{darkblue}{\textbf{\ipa{ʁo˧kɤ˩}}}}{}
\textcolor{teal}{\mytextsc{nom}} \hspace{4pt} Ton~: L\#.

Coiffe en fils tressés; chez une jeune femme (qui a atteint ses 13 ans, mais n'a pas encore d'enfants), ce même objet est désigné comme \textcolor{darkblue}{\textbf{\ipa{/ʁo˧ni˥/}}}.
\textcolor{brown}{\zh{用来将长辫缠成盘头的黑色丝头饰(已经有孩子的女人戴的)。还没有孩子的青年女人,也戴这种头饰,但称作\textcolor{darkblue}{\textbf{\ipa{/ʁo˧ni˥/}}})。}}


 \mytextsc{clf}~: \textcolor{darkblue}{\textbf{\ipa{kɤ˧˥}}} 

\paragraph{\hspace{-0.5cm} {\Large \textcolor{darkblue}{\textbf{\ipa{ʁo˧lv̩˧}}}}} \hypertarget{Ro\string_Mlv\string_=\string_M1}{}
\markboth{\textcolor{darkblue}{\textbf{\ipa{ʁo˧lv̩˧}}}}{}
\textcolor{teal}{\mytextsc{verbe}} \hspace{4pt} Ton~: M.

Se perdre, perdre son chemin.
\textcolor{brown}{\zh{迷路。}}


\begin{exe}
\sn \textcolor{darkblue}{\textbf{\ipa{le˧-ʁo˧lv̩˧}}}
\trans \mytextsc{accomp}
\trans \textit{\textcolor{brown}{\zh{\mytextsc{accomp}}}}
\end{exe}


\paragraph{\hspace{-0.5cm} {\Large \textcolor{darkblue}{\textbf{\ipa{ʁo˧ɬi˥}}}}} \hypertarget{Ro\string_MKi\string_T1}{}
\markboth{\textcolor{darkblue}{\textbf{\ipa{ʁo˧ɬi˥}}}}{}
\textcolor{teal}{\mytextsc{nom}} \hspace{4pt} Ton~: H\#.

Grosse aiguille avec laquelle on coud le paquet de viande de cochon au salpêtre qui se conserve une décennie; contrairement à ce que dit M21, n'est pas utilisé pour coudre les peaux d'animaux (mouton, bœuf, yak…): pour cela, il faut utiliser un poinçon.
\textcolor{brown}{\zh{大粗针,用来缝琵琶肉。}}


\begin{exe}
\sn \textcolor{darkblue}{\textbf{\ipa{ʁo˧ɬi˥, | bo˩ʈʂʰæ˧ ʐv̩˩-di˩ ɲi˩.}}}
\trans La grosse aiguille, ça sert à coudre le cochon-conservé-entier (viande séchée ``pipa").
\trans \textit{\textcolor{brown}{\zh{大针,是用来缝琵琶肉的。}}}
\end{exe}
 \mytextsc{clf}~: \textcolor{darkblue}{\textbf{\ipa{ɭɯ˧}}} 

\paragraph{\hspace{-0.5cm} {\Large \textcolor{darkblue}{\textbf{\ipa{ʁo˧mæ˧}}}}} \hypertarget{Ro\string_Mm\{\string_M1}{}
\markboth{\textcolor{darkblue}{\textbf{\ipa{ʁo˧mæ˧}}}}{}
\textcolor{teal}{\mytextsc{verbe}} \hspace{4pt} Ton~: .

S'occuper de (personnes âgées, enfants, personnes ayant besoin d'aide).
\textcolor{brown}{\zh{照顾好、管好、关心(老人、孩子、需要帮助的人)。}}


\begin{exe}
\sn \textcolor{darkblue}{\textbf{\ipa{ʁo˧mæ˧-ze˩}}}
\trans \mytextsc{pfv}
\trans \textit{\textcolor{brown}{\zh{管好了}}}
\end{exe}
\begin{exe}
\sn \textcolor{darkblue}{\textbf{\ipa{hĩ˧ ʁo˧mæ˧}}}
\trans s'occuper des gens
\trans \textit{\textcolor{brown}{\zh{照顾人}}}
\end{exe}
\begin{exe}
\sn \textcolor{darkblue}{\textbf{\ipa{[F5] no˧ | njɤ˧ ʁo˧mæ˧˥!}}}
\trans Tu t'occupes de moi! / tu es aux petits soins pour moi! (Commentaire satisfait/élogieux)
\trans \textit{\textcolor{brown}{\zh{你很关心我啊!(老人满意地说)}}}
\end{exe}
\begin{exe}
\sn \textcolor{darkblue}{\textbf{\ipa{ʈʂʰɯ˧ | ɖwæ˧˥ | hĩ˧ ʁo˧mæ˧-kv̩˩!}}}
\trans Elle sait à merveille s'occuper des gens/prendre soin des gens! (Commentaire au sujet d'une maîtresse de maison)
\trans \textit{\textcolor{brown}{\zh{她很会管家!(对女主人的表扬)}}}
\end{exe}


\paragraph{\hspace{-0.5cm} {\Large \textcolor{darkblue}{\textbf{\ipa{ʁo˧mi˥\$}}}}} \hypertarget{Ro\string_Mmi\string_T\$1}{}
\markboth{\textcolor{darkblue}{\textbf{\ipa{ʁo˧mi˥\$}}}}{}
\textcolor{teal}{\mytextsc{nom}} \hspace{4pt} Ton~: H\$.

Grosse aiguille.
\textcolor{brown}{\zh{大针。}}


 \mytextsc{clf}~: \textcolor{darkblue}{\textbf{\ipa{ɭɯ˧}}} 

\paragraph{\hspace{-0.5cm} {\Large \textcolor{darkblue}{\textbf{\ipa{ʁo˧mi˧}}}}} \hypertarget{Ro\string_Mmi\string_M1}{}
\markboth{\textcolor{darkblue}{\textbf{\ipa{ʁo˧mi˧}}}}{}
\textcolor{teal}{\mytextsc{nom}} \hspace{4pt} Ton~: M.

Roi; haut dignitaire, grand mandarin; chef.
\textcolor{brown}{\zh{国王、大臣、头领。}}


\begin{exe}
\sn \textcolor{darkblue}{\textbf{\ipa{ʁo˧mi˧ ʝi˧-hĩ˧ hĩ˧}}}
\trans personne qui a fonction de dignitaire/chef
\trans \textit{\textcolor{brown}{\zh{当国王、土司、大臣、头领……的人}}}
\end{exe}
\begin{exe}
\sn \textcolor{darkblue}{\textbf{\ipa{kʰv̩˧mæ˧-ʁo˧mi˧}}}
\trans chef des brigands, capitaine d'une troupe de brigands
\trans \textit{\textcolor{brown}{\zh{土匪的头领}}}
\end{exe}
 \mytextsc{clf}~: \textcolor{darkblue}{\textbf{\ipa{v̩˧}}} 

\paragraph{\hspace{-0.5cm} {\Large \textcolor{darkblue}{\textbf{\ipa{ʁo˧ni˥}}}}} \hypertarget{Ro\string_Mni\string_T1}{}
\markboth{\textcolor{darkblue}{\textbf{\ipa{ʁo˧ni˥}}}}{}
\textcolor{teal}{\mytextsc{nom}} \hspace{4pt} Ton~: H\#.

Coiffe en fils tressés des jeunes femmes qui n'ont pas encore d'enfant. Les femmes qui ont des enfants portent également cette pièce de costume, mais elle est alors désignée comme \textcolor{darkblue}{\textbf{\ipa{/ʁo˧kɤ˩/}}}.
\textcolor{brown}{\zh{用来将长辫缠成盘头的黑色丝头饰(还没有孩子的青年女人戴的)。已经有孩子的女人,也戴这种头饰,但称作\textcolor{darkblue}{\textbf{\ipa{/ʁo˧kɤ˩/}}})。}}


 \mytextsc{clf}~: \textcolor{darkblue}{\textbf{\ipa{bo˩}}} 

\paragraph{\hspace{-0.5cm} {\Large \textcolor{darkblue}{\textbf{\ipa{ʁo˧pʰɤ˩-ʁo˩dv̩˩lv̩˩}}}}} \hypertarget{Ro\string_Mp\string_h7\string_B-Ro\string_Bdv\string_=\string_Blv\string_=\string_B1}{}
\markboth{\textcolor{darkblue}{\textbf{\ipa{ʁo˧pʰɤ˩-ʁo˩dv̩˩lv̩˩}}}}{}
\textcolor{teal}{\mytextsc{nom}} \hspace{4pt} Ton~: L\#.

\textit{Eugeron breviscapus} (une sorte de pâquerette).
\textcolor{brown}{\zh{短葶飞蓬。}}




\paragraph{\hspace{-0.5cm} {\Large \textcolor{darkblue}{\textbf{\ipa{ʁo˧qɑ˥}}}}} \hypertarget{Ro\string_MqA\string_T1}{}
\markboth{\textcolor{darkblue}{\textbf{\ipa{ʁo˧qɑ˥}}}}{}
\textcolor{teal}{\mytextsc{nom}} \hspace{4pt} Ton~: H\#.

Couvercle.
\textcolor{brown}{\zh{锅盖、盖子。}}


 \mytextsc{clf}~: \textcolor{darkblue}{\textbf{\ipa{ɭɯ˧}}} 

\paragraph{\hspace{-0.5cm} {\Large \textcolor{darkblue}{\textbf{\ipa{ʁo˧qʰwɤ˩}}}}} \hypertarget{Ro\string_Mq\string_hw7\string_B1}{}
\markboth{\textcolor{darkblue}{\textbf{\ipa{ʁo˧qʰwɤ˩}}}}{}
\textcolor{teal}{\mytextsc{nom}} \hspace{4pt} Ton~: L\#.
\ding{202} 
Tête.
\textcolor{brown}{\zh{头,上面部分。}}


\begin{exe}
\sn \textcolor{darkblue}{\textbf{\ipa{ʁo˧qʰwɤ˩ dzi˩}}}
\trans être assis à une place d'honneur
\trans \textit{\textcolor{brown}{\zh{坐在贵宾的位置上}}}
\end{exe}
\begin{exe}
\sn \textcolor{darkblue}{\textbf{\ipa{õ˧-ʁo˥qʰwɤ˩}}}
\trans sa propre tête
\trans \textit{\textcolor{brown}{\zh{自己的头}}}
\end{exe}
\begin{exe}
\sn \textcolor{darkblue}{\textbf{\ipa{õ˧-ʁo˥qʰwɤ˩ lɑ˩}}}
\trans se taper sur la tête (contexte: un enfant tape en rythme sur sa propre tête avec une baguette)
\trans \textit{\textcolor{brown}{\zh{打自己的头(情景:一个小孩用小棍子敲打自己的头)}}}
\end{exe}
\ding{203} 
Partie supérieure de.
\textcolor{brown}{\zh{上面部分。}}


 \mytextsc{clf}~: \textcolor{darkblue}{\textbf{\ipa{ɭɯ˧}}} 

\paragraph{\hspace{-0.5cm} {\Large \textcolor{darkblue}{\textbf{\ipa{ʁo˧so˩}}}}} \hypertarget{Ro\string_Mso\string_B1}{}
\markboth{\textcolor{darkblue}{\textbf{\ipa{ʁo˧so˩}}}}{}
\textcolor{teal}{\mytextsc{adverbe}} \hspace{4pt} Ton~: L\#.

Après-demain.
\textcolor{brown}{\zh{后天。}}


\begin{exe}
\sn \textcolor{darkblue}{\textbf{\ipa{ʁo˧so˩ | -ɖɯ˧ɲi˥}}}
\trans la journée d'après-demain
\trans \textit{\textcolor{brown}{\zh{后天}}}
\end{exe}


\paragraph{\hspace{-0.5cm} {\Large \textcolor{darkblue}{\textbf{\ipa{ʁo˧ʂv̩˧}}}}} \hypertarget{Ro\string_Ms`v\string_=\string_M1}{}
\markboth{\textcolor{darkblue}{\textbf{\ipa{ʁo˧ʂv̩˧}}}}{}
\textcolor{teal}{\mytextsc{verbe}} \hspace{4pt} Ton~: M.

Conduire, guider.
\textcolor{brown}{\zh{带头、带路。}}


\begin{exe}
\sn \textcolor{darkblue}{\textbf{\ipa{ʐɤ˩mi˩ ʁo˩ʂv̩˩˥}}}
\trans montrer le chemin
\trans \textit{\textcolor{brown}{\zh{带路}}}
\end{exe}
\begin{exe}
\sn \textcolor{darkblue}{\textbf{\ipa{ɖɯ˧-ʑi˩-ɳɯ˩ | ʁo˧ʂv̩˧}}}
\trans une famille montre l'exemple: par exemple, une famille commence à récolter le riz, et les autres suivent son exemple
\trans \textit{\textcolor{brown}{\zh{有一家带头:例如收庄稼时,一个家先开始收割,于是其它家庭也跟着开始收割。}}}
\end{exe}
\begin{exe}
\sn \textcolor{darkblue}{\textbf{\ipa{ʁo˧ʂv̩˧-ze˧}}}
\trans \mytextsc{pfv}
\trans \textit{\textcolor{brown}{\zh{带了路}}}
\end{exe}
\begin{exe}
\sn \textcolor{darkblue}{\textbf{\ipa{njɤ˧=ɻ̍˩-ɳɯ˩ | ʁo˧ʂv̩˧!}}}
\trans C'est nous qui lançons le mouvement!/ C'est nous qui donnons l'exemple aux autres! (explication: pour les travaux des champs, une maisonnée s'y attelait en premier, et les autres suivaient)
\trans \textit{\textcolor{brown}{\zh{是我们带头的!(其他家庭是跟着我们来的!)(情景:农业活动,如:收庄稼,是一个家庭先开始的,然后其他家庭也跟着来。)}}}
\end{exe}


\paragraph{\hspace{-0.5cm} {\Large \textcolor{darkblue}{\textbf{\ipa{-ʁo˧to˩}}}}} \hypertarget{-Ro\string_Mto\string_B1}{}
\markboth{\textcolor{darkblue}{\textbf{\ipa{-ʁo˧to˩}}}}{}
\textcolor{teal}{\mytextsc{postposition}} \hspace{4pt} Ton~: L\#.
\ding{202} 
Sur.
\textcolor{brown}{\zh{……之上。}}


\begin{exe}
\sn \textcolor{darkblue}{\textbf{\ipa{qo˩qɑ˩-ʁo˩to˥}}}
\trans en haut du col
\trans \textit{\textcolor{brown}{\zh{垭口上}}}
\end{exe}
\begin{exe}
\sn \textcolor{darkblue}{\textbf{\ipa{ʁo˧qʰwɤ˩-ʁo˩to˩}}}
\trans sur la tête, sur le sommet du crâne
\trans \textit{\textcolor{brown}{\zh{头上}}}
\end{exe}
\begin{exe}
\sn \textcolor{darkblue}{\textbf{\ipa{ʑi˧qʰwɤ˧-ʁo˧to˩}}}
\trans sur la maison; ex.: il y a un nid d’oiseaux sur la maison
\trans \textit{\textcolor{brown}{\zh{房子上面:例如:有鸟窝在房顶上}}}
\end{exe}
\ding{203} 
Pendant, au moment de.
\textcolor{brown}{\zh{……的时候。}}


\begin{exe}
\sn \textcolor{darkblue}{\textbf{\ipa{hɑ˧dzɯ˧-ʁo˧to˩, | ʈʂʰɯ˧-ɳɯ˧ | mɤ˧-fv̩˧-ʝi˧.}}}
\trans Au cours du repas, il se mit en colère/ devint triste.
\trans \textit{\textcolor{brown}{\zh{吃饭的时候,他不高兴了/生气了。}}}
\end{exe}
\ding{204} 
À l'endroit de, à l'égard de, en direction de.
\textcolor{brown}{\zh{向、往。}}


\ding{205} 
En comparaison de.
\textcolor{brown}{\zh{跟……相比。}}




\paragraph{\hspace{-0.5cm} {\Large \textcolor{darkblue}{\textbf{\ipa{-ʁo˧tʰo˩}}}}} \hypertarget{-Ro\string_Mt\string_ho\string_B1}{}
\markboth{\textcolor{darkblue}{\textbf{\ipa{-ʁo˧tʰo˩}}}}{}
\textcolor{teal}{\mytextsc{postposition}} \hspace{4pt} Ton~: L\#.

Derrière; depuis.
\textcolor{brown}{\zh{后面,自从。}}


\begin{exe}
\sn \textcolor{darkblue}{\textbf{\ipa{ʑi˧-tʰo˩}}}
\trans l'arrière de la maison (où il y a le potager)
\trans \textit{\textcolor{brown}{\zh{家后院(=菜园的地方)}}}
\end{exe}
\begin{exe}
\sn \textcolor{darkblue}{\textbf{\ipa{ʑi˧-ʁo˥tʰo˩}}}
\trans idem: derrière la maison, l'arrière de la maison
\trans \textit{\textcolor{brown}{\zh{同上:家后院}}}
\end{exe}
\begin{exe}
\sn \textcolor{darkblue}{\textbf{\ipa{ʑi˧qʰwɤ˧-ʁo˧tʰo˩}}}
\trans idem: derrière la maison, l'arrière de la maison
\trans \textit{\textcolor{brown}{\zh{同上:家后院}}}
\end{exe}


\paragraph{\hspace{-0.5cm} {\Large \textcolor{darkblue}{\textbf{\ipa{ʁo˧tɕʰɤ\#˥}}}}} \hypertarget{Ro\string_Mts£\string_h7\#\string_T1}{}
\markboth{\textcolor{darkblue}{\textbf{\ipa{ʁo˧tɕʰɤ\#˥}}}}{}
\textcolor{teal}{\mytextsc{adjectif}} \hspace{4pt} Ton~: \#H.

Pointu.
\textcolor{brown}{\zh{尖。}}


\begin{exe}
\sn \textcolor{darkblue}{\textbf{\ipa{[F5] ʁo˧tɕʰɤ˧\textasciitilde{}tɕʰɤ˧-gv̩˧}}}
\trans pointu
\trans \textit{\textcolor{brown}{\zh{尖}}}
\end{exe}


\paragraph{\hspace{-0.5cm} {\Large \textcolor{darkblue}{\textbf{\ipa{ʁo˧tsʰe˧ʁo\#˥}}}}} \hypertarget{Ro\string_Mts\string_he\string_MRo\#\string_T1}{}
\markboth{\textcolor{darkblue}{\textbf{\ipa{ʁo˧tsʰe˧ʁo\#˥}}}}{}
\textcolor{teal}{\mytextsc{nom}} \hspace{4pt} Ton~: \#H.

Sommet de, en haut de.
\textcolor{brown}{\zh{顶上,如:山顶。}}


\begin{exe}
\sn \textcolor{darkblue}{\textbf{\ipa{ʁwɤ˧-bv̩˧ | ʁo˧tsʰe˧ʁo˧}}}
\trans le sommet de la montagne
\trans \textit{\textcolor{brown}{\zh{山的顶,山顶}}}
\end{exe}
\begin{exe}
\sn \textcolor{darkblue}{\textbf{\ipa{ʁo˧qʰwɤ˩-ʁo˩tsʰe˩}}}
\trans le sommet de la tête
\trans \textit{\textcolor{brown}{\zh{头顶}}}
\end{exe}


\paragraph{\hspace{-0.5cm} {\Large \textcolor{darkblue}{\textbf{\ipa{ʁo˧ʈv̩˧ʈv̩˥}}}}} \hypertarget{Ro\string_Mt`v\string_=\string_Mt`v\string_=\string_T1}{}
\markboth{\textcolor{darkblue}{\textbf{\ipa{ʁo˧ʈv̩˧ʈv̩˥}}}}{}
\textcolor{teal}{\mytextsc{nom}} \hspace{4pt} Ton~: H\#.

Yi (groupe ethnique): terme péjoratif: ``les hirsutes", ``les ébouriffés".
\textcolor{brown}{\zh{彝族(带偏见的说法:“乱糟糟的头发”)。}}


 \mytextsc{clf}~: \textcolor{darkblue}{\textbf{\ipa{v̩˧}}} 

\paragraph{\hspace{-0.5cm} {\Large \textcolor{darkblue}{\textbf{\ipa{ʁo˧ʈʂe˩}}}}} \hypertarget{Ro\string_Mt`s`e\string_B1}{}
\markboth{\textcolor{darkblue}{\textbf{\ipa{ʁo˧ʈʂe˩}}}}{}
\textcolor{teal}{\mytextsc{nom}} \hspace{4pt} Ton~: L\#.

Teigne.
\textcolor{brown}{\zh{癣。}}


\begin{exe}
\sn \textcolor{darkblue}{\textbf{\ipa{[M23] ʁu˧ʈʂɯ˩ ɖwæ˧˥ tʰi˧ di˩!}}}
\trans (il) a vraiment la teigne à la tête!
\trans \textit{\textcolor{brown}{\zh{他长了很多癣!}}}
\end{exe}
 \mytextsc{clf}~: \textcolor{darkblue}{\textbf{\ipa{pʰæ˧˥}}} 

\paragraph{\hspace{-0.5cm} {\Large \textcolor{darkblue}{\textbf{\ipa{ʁo˧zo\#˥}}}}} \hypertarget{Ro\string_Mzo\#\string_T1}{}
\markboth{\textcolor{darkblue}{\textbf{\ipa{ʁo˧zo\#˥}}}}{}
\textcolor{teal}{\mytextsc{nom}} \hspace{4pt} Ton~: \#H.

Petite aiguille.
\textcolor{brown}{\zh{小针。}}


 \mytextsc{clf}~: \textcolor{darkblue}{\textbf{\ipa{ɭɯ˧}}} 

\paragraph{\hspace{-0.5cm} {\Large \textcolor{darkblue}{\textbf{\ipa{ʁo˧ʑi˧˥}}}}} \hypertarget{Ro\string_Mz£i\string_M\string_T1}{}
\markboth{\textcolor{darkblue}{\textbf{\ipa{ʁo˧ʑi˧˥}}}}{}
\textcolor{teal}{\mytextsc{adverbe}} \hspace{4pt} Ton~: MH\#.

À partir de.
\textcolor{brown}{\zh{从……开始。}}


\begin{exe}
\sn \textcolor{darkblue}{\textbf{\ipa{ɖɯ˧ɬi˧mi˧-ʁo˧ʑi˧˥}}}
\trans à partir du premier mois
\trans \textit{\textcolor{brown}{\zh{一月份开始}}}
\end{exe}
\begin{exe}
\sn \textcolor{darkblue}{\textbf{\ipa{tsʰi˧ɲi˧-ʁo˧ʑi˧˥}}}
\trans à partir d'aujourd'hui
\trans \textit{\textcolor{brown}{\zh{今天开始}}}
\end{exe}
\begin{exe}
\sn \textcolor{darkblue}{\textbf{\ipa{tsʰi˧ʝi˧ ɖɯ˧-kʰv̩˧˥-ʁo˧ʑi˧˥}}}
\trans à partir de cette année
\trans \textit{\textcolor{brown}{\zh{今年开始}}}
\end{exe}
\begin{exe}
\sn \textcolor{darkblue}{\textbf{\ipa{gv̩˩ɬi˩mi˩-ʁo˩ʑi˩˥}}}
\trans à partir du 9e mois
\trans \textit{\textcolor{brown}{\zh{九月份开始}}}
\end{exe}
\begin{exe}
\sn \textcolor{darkblue}{\textbf{\ipa{ʐe˧ʈæ˥ɬi˩-ʁo˩ʑi˩}}}
\trans à partir du 11e mois
\trans \textit{\textcolor{brown}{\zh{十一月份开始}}}
\end{exe}


\paragraph{\hspace{-0.5cm} {\Large \textcolor{darkblue}{\textbf{\ipa{ʁo˩}}}}} \hypertarget{Ro\string_B1}{}
\markboth{\textcolor{darkblue}{\textbf{\ipa{ʁo˩}}}}{}
\textcolor{teal}{\mytextsc{verbe}} \hspace{4pt} Ton~: L.

Tomber, sombrer (ex.: quelqu'un coule dans l'eau; un bateau sombre peu à peu dans le lac).
\textcolor{brown}{\zh{掉入、沉下去。}}


\begin{exe}
\sn \textcolor{darkblue}{\textbf{\ipa{mv̩˩tɕo˥ ʁo˩}}}
\trans s'enfoncer, être englouti (par l'eau…)
\trans \textit{\textcolor{brown}{\zh{掉入}}}
\end{exe}


\paragraph{\hspace{-0.5cm} {\Large \textcolor{darkblue}{\textbf{\ipa{ʁo˩\textsubscript{b}}}}}} \hypertarget{Ro\string_Bb1}{}
\markboth{\textcolor{darkblue}{\textbf{\ipa{ʁo˩\textsubscript{b}}}}}{}
\textcolor{teal}{\mytextsc{classificateur}} \hspace{4pt} Ton~: L\textsubscript{b}.

Classificateur des variétés/sortes de choses.
\textcolor{brown}{\zh{量词:种。}}


\begin{exe}
\sn \textcolor{darkblue}{\textbf{\ipa{ɖɯ˧-ʁo˩}}}
\trans une sorte (de vêtement, de nourriture...)
\trans \textit{\textcolor{brown}{\zh{一种(衣服、食物……)}}}
\end{exe}
\begin{exe}
\sn \textcolor{darkblue}{\textbf{\ipa{ʈʂʰɯ˧-ʁo˥}}}
\trans cette sorte (de vêtement, de nourriture...)
\trans \textit{\textcolor{brown}{\zh{这种(衣服、食物……)}}}
\end{exe}


\paragraph{\hspace{-0.5cm} {\Large \textcolor{darkblue}{\textbf{\ipa{ʁo˩\textsubscript{b}}}}}} \hypertarget{Ro\string_Bb1}{}
\markboth{\textcolor{darkblue}{\textbf{\ipa{ʁo˩\textsubscript{b}}}}}{}
\textcolor{teal}{\mytextsc{verbe}} \hspace{4pt} Ton~: L\textsubscript{b}.

Se former, apparaître (un cor se forme, un durillon se forme). Ce verbe n'a été observé qu'en association avec le mot 'durillon, cor'.
\textcolor{brown}{\zh{出现、形成(如:出了茧子)。}}


\begin{exe}
\sn \textcolor{darkblue}{\textbf{\ipa{sɯ˧ʈv̩˥ ʁo˩-ze˩! |}}}
\trans Un durillon s'est formé!
\trans \textit{\textcolor{brown}{\zh{磨出了茧子!}}}
\end{exe}
\begin{exe}
\sn \textcolor{darkblue}{\textbf{\ipa{ʁo˩-mɤ˩-ho˥}}}
\trans \string_ \mytextsc{neg} \mytextsc{désidératif}
\trans \textit{\textcolor{brown}{\zh{不会出(茧子)}}}
\end{exe}


\paragraph{\hspace{-0.5cm} {\Large \textcolor{darkblue}{\textbf{\ipa{ʁo˩di˥}}}}} \hypertarget{Ro\string_Bdi\string_T1}{}
\markboth{\textcolor{darkblue}{\textbf{\ipa{ʁo˩di˥}}}}{}
\textcolor{teal}{\mytextsc{nom}} \hspace{4pt} Ton~: LH.

Fou, aliéné.
\textcolor{brown}{\zh{疯子。}}


 \mytextsc{clf}~: \textcolor{darkblue}{\textbf{\ipa{v̩˧}}} 

\paragraph{\hspace{-0.5cm} {\Large \textcolor{darkblue}{\textbf{\ipa{ʁo˩ɖɯ˩so˧}}}}} \hypertarget{Ro\string_Bd`M\string_Bso\string_M1}{}
\markboth{\textcolor{darkblue}{\textbf{\ipa{ʁo˩ɖɯ˩so˧}}}}{}
\textcolor{teal}{\mytextsc{adverbe}} \hspace{4pt} Ton~: .

Dans trois jours.
\textcolor{brown}{\zh{大后天。}}




\paragraph{\hspace{-0.5cm} {\Large \textcolor{darkblue}{\textbf{\ipa{ʁo˩hi˩}}}}} \hypertarget{Ro\string_Bhi\string_B1}{}
\markboth{\textcolor{darkblue}{\textbf{\ipa{ʁo˩hi˩}}}}{}
\textcolor{teal}{\mytextsc{nom}} \hspace{4pt} Ton~: L.

Molaires et prémolaires.
\textcolor{brown}{\zh{臼齿+后臼齿。}}


 \mytextsc{clf}~: \textcolor{darkblue}{\textbf{\ipa{ɭɯ˧}}} 

\paragraph{\hspace{-0.5cm} {\Large \textcolor{darkblue}{\textbf{\ipa{ʁo˩kʰv̩˩}}}}} \hypertarget{Ro\string_Bk\string_hv\string_=\string_B1}{}
\markboth{\textcolor{darkblue}{\textbf{\ipa{ʁo˩kʰv̩˩}}}}{}
\textcolor{teal}{\mytextsc{nom}} \hspace{4pt} Ton~: L.

Arbre à épice, arbre à encens de petite taille, qui pousse en montagne, dans les espaces ombragés.
\textcolor{brown}{\zh{香木。}}Dialecte chinois local~: \zh{柏香。}


\begin{exe}
\sn \textcolor{darkblue}{\textbf{\ipa{ʁo˩kʰv̩˩-si˩}}}
\trans même sens
\trans \textit{\textcolor{brown}{\zh{同上}}}
\end{exe}
\textit{Voir~:} \hyperlink{ts7\string_Mdi\string_M1}{\textcolor{darkblue}{\textbf{\ipa{tsɤ˧di˧}}}} 

\paragraph{\hspace{-0.5cm} {\Large \textcolor{darkblue}{\textbf{\ipa{ʁo˧˥}}}}} \hypertarget{Ro\string_M\string_T1}{}
\markboth{\textcolor{darkblue}{\textbf{\ipa{ʁo˧˥}}}}{}
\textcolor{teal}{\mytextsc{nom}} \hspace{4pt} Ton~: MH.

Aiguille.
\textcolor{brown}{\zh{针。}}


 \mytextsc{clf}~: \textcolor{darkblue}{\textbf{\ipa{ɭɯ˧}}} \textit{Voir~:} \textcolor{darkblue}{\textbf{\ipa{ʈʂe˥}}} 

\paragraph{\hspace{-0.5cm} {\Large \textcolor{darkblue}{\textbf{\ipa{ʁv̩˧˥}}}}} \hypertarget{Rv\string_=\string_M\string_T1}{}
\markboth{\textcolor{darkblue}{\textbf{\ipa{ʁv̩˧˥}}}}{}
\textcolor{teal}{\mytextsc{nom}} \hspace{4pt} Ton~: MH.

Grue, Grus nigricollis Przew et autres espèces similaires. Il s'agit d'un oiseau migrateur.
\textcolor{brown}{\zh{黑颈鹤(候鸟)。}}


\begin{exe}
\sn \textcolor{darkblue}{\textbf{\ipa{ʁv̩˧nɑ˥mi˩}}}
\trans même sens: grue
\trans \textit{\textcolor{brown}{\zh{同上:黑颈鹤}}}
\end{exe}
\begin{exe}
\sn \textcolor{darkblue}{\textbf{\ipa{ʁv̩˧ dzɯ˥-ze˩}}}
\trans ...a mangé une grue
\trans \textit{\textcolor{brown}{\zh{吃了黑颈鹤}}}
\end{exe}
\begin{exe}
\sn \textcolor{darkblue}{\textbf{\ipa{ʁv̩˧ hwæ˥-ze˩}}}
\trans ...a acheté une grue
\trans \textit{\textcolor{brown}{\zh{买了黑颈鹤}}}
\end{exe}
 \mytextsc{clf}~: \textcolor{darkblue}{\textbf{\ipa{mi˩}}} 

\paragraph{\hspace{-0.5cm} {\Large \textcolor{darkblue}{\textbf{\ipa{ʁv̩˥}}}}} \hypertarget{Rv\string_=\string_T1}{}
\markboth{\textcolor{darkblue}{\textbf{\ipa{ʁv̩˥}}}}{}
\textcolor{teal}{\mytextsc{verbe}} \hspace{4pt} Ton~: H.

Avaler, déglutir.
\textcolor{brown}{\zh{吞,咽。}}


\begin{exe}
\sn \textcolor{darkblue}{\textbf{\ipa{le˧-ʁv̩˥}}}
\trans \mytextsc{accomp}
\trans \textit{\textcolor{brown}{\zh{\mytextsc{accomp}}}}
\end{exe}


\paragraph{\hspace{-0.5cm} {\Large \textcolor{darkblue}{\textbf{\ipa{ʁv̩˧-bv̩˥}}}}} \hypertarget{Rv\string_=\string_M-bv\string_=\string_T1}{}
\markboth{\textcolor{darkblue}{\textbf{\ipa{ʁv̩˧-bv̩˥}}}}{}
\textcolor{teal}{\mytextsc{nom}} \hspace{4pt} Ton~: .

Monnoyère, tablier des champs, \textit{Thlaspi arvense}; littéralement ``le légume de la grue".
\textcolor{brown}{\zh{菥蓂。}}


 \mytextsc{clf}~: \textcolor{darkblue}{\textbf{\ipa{po˧}}} 

\paragraph{\hspace{-0.5cm} {\Large \textcolor{darkblue}{\textbf{\ipa{ʁv̩˧mi˥\$}}}}} \hypertarget{Rv\string_=\string_Mmi\string_T\$1}{}
\markboth{\textcolor{darkblue}{\textbf{\ipa{ʁv̩˧mi˥\$}}}}{}
\textcolor{teal}{\mytextsc{nom}} \hspace{4pt} Ton~: H\$.

Grue femelle.
\textcolor{brown}{\zh{母鹤。}}


 \mytextsc{clf}~: \textcolor{darkblue}{\textbf{\ipa{mi˩}}} 

\paragraph{\hspace{-0.5cm} {\Large \textcolor{darkblue}{\textbf{\ipa{ʁv̩˧pʰv̩\#˥}}}}} \hypertarget{Rv\string_=\string_Mp\string_hv\string_=\#\string_T1}{}
\markboth{\textcolor{darkblue}{\textbf{\ipa{ʁv̩˧pʰv̩\#˥}}}}{}
\textcolor{teal}{\mytextsc{nom}} \hspace{4pt} Ton~: \#H.

Grue mâle.
\textcolor{brown}{\zh{公鹤。}}


\begin{exe}
\sn \textcolor{darkblue}{\textbf{\ipa{ʁv̩˧pʰv̩˧-ʁv̩˧mi\#˥}}}
\trans grue mâle et grue femelle
\trans \textit{\textcolor{brown}{\zh{公鹤与母鹤}}}
\end{exe}
 \mytextsc{clf}~: \textcolor{darkblue}{\textbf{\ipa{mi˩}}} 

\paragraph{\hspace{-0.5cm} {\Large \textcolor{darkblue}{\textbf{\ipa{ʁv̩˧zo\#˥}}}}} \hypertarget{Rv\string_=\string_Mzo\#\string_T1}{}
\markboth{\textcolor{darkblue}{\textbf{\ipa{ʁv̩˧zo\#˥}}}}{}
\textcolor{teal}{\mytextsc{nom}} \hspace{4pt} Ton~: \#H.

Enfant grue.
\textcolor{brown}{\zh{小鹤。}}


 \mytextsc{clf}~: \textcolor{darkblue}{\textbf{\ipa{ɭɯ˧}}} 

\paragraph{\hspace{-0.5cm} {\Large \textcolor{darkblue}{\textbf{\ipa{ʁwæ˥}}}}} \hypertarget{Rw\{\string_T1}{}
\markboth{\textcolor{darkblue}{\textbf{\ipa{ʁwæ˥}}}}{}
\textcolor{teal}{\mytextsc{nom}} \hspace{4pt} Ton~: \#H.

Gauche (monosyllabe).
\textcolor{brown}{\zh{左边(单音节)。}}




\paragraph{\hspace{-0.5cm} {\Large \textcolor{darkblue}{\textbf{\ipa{ʁwæ˧gi\#˥}}}}} \hypertarget{Rw\{\string_Mgi\#\string_T1}{}
\markboth{\textcolor{darkblue}{\textbf{\ipa{ʁwæ˧gi\#˥}}}}{}
\textcolor{teal}{\mytextsc{nom}} \hspace{4pt} Ton~: \#H.

Gauche, côté gauche.
\textcolor{brown}{\zh{左边。}}




\paragraph{\hspace{-0.5cm} {\Large \textcolor{darkblue}{\textbf{\ipa{ʁwæ˧gi˧dzɤ\#˥}}}}} \hypertarget{Rw\{\string_Mgi\string_Mdz7\#\string_T1}{}
\markboth{\textcolor{darkblue}{\textbf{\ipa{ʁwæ˧gi˧dzɤ\#˥}}}}{}
\textcolor{teal}{\mytextsc{nom}} \hspace{4pt} Ton~: \#H.

Gauche, côté gauche, direction de gauche.
\textcolor{brown}{\zh{左、左边。}}




\paragraph{\hspace{-0.5cm} {\Large \textcolor{darkblue}{\textbf{\ipa{ʁwæ˧lo˥}}}}} \hypertarget{Rw\{\string_Mlo\string_T1}{}
\markboth{\textcolor{darkblue}{\textbf{\ipa{ʁwæ˧lo˥}}}}{}
\textcolor{teal}{\mytextsc{nom}} \hspace{4pt} Ton~: H\#.

Gauche; direction de gauche.
\textcolor{brown}{\zh{左边,左手。}}




\paragraph{\hspace{-0.5cm} {\Large \textcolor{darkblue}{\textbf{\ipa{ʁwæ˧tsɯ˥}}}}} \hypertarget{Rw\{\string_MtsM\string_T1}{}
\markboth{\textcolor{darkblue}{\textbf{\ipa{ʁwæ˧tsɯ˥}}}}{}
\textcolor{teal}{\mytextsc{nom}} \hspace{4pt} Ton~: H\#.

Chaussettes.
\textcolor{brown}{\zh{袜子。}}

 Emprunt~: chinois \textcolor{brown}{\zh{袜子}}



\paragraph{\hspace{-0.5cm} {\Large \textcolor{darkblue}{\textbf{\ipa{ʁwæ˧ʈʂʰe˩}}}}} \hypertarget{Rw\{\string_Mt`s`\string_he\string_B1}{}
\markboth{\textcolor{darkblue}{\textbf{\ipa{ʁwæ˧ʈʂʰe˩}}}}{}
\textcolor{teal}{\mytextsc{verbe}} \hspace{4pt} Ton~: L\#.

Achever, mener à terme.
\textcolor{brown}{\zh{完成(汉语借词)。}}

 Emprunt~: chinois \textcolor{brown}{\zh{完成}}

\begin{exe}
\sn \textcolor{darkblue}{\textbf{\ipa{le˧-ʁwæ˧ʈʂʰe˩-ze˩!}}}
\trans C'est achevé!
\trans \textit{\textcolor{brown}{\zh{完成了!}}}
\end{exe}


\paragraph{\hspace{-0.5cm} {\Large \textcolor{darkblue}{\textbf{\ipa{ʁwɤ˧}}} \textsubscript{1}}} \hypertarget{Rw7\string_M1}{}
\markboth{\textcolor{darkblue}{\textbf{\ipa{ʁwɤ˧}}} \textsubscript{1}}{}
\textcolor{teal}{\mytextsc{nom}} \hspace{4pt} Ton~: M.

Montagne.
\textcolor{brown}{\zh{山。}}


\begin{exe}
\sn \textcolor{darkblue}{\textbf{\ipa{ʁo˧-ʂwæ˧}}}
\trans haute montagne
\trans \textit{\textcolor{brown}{\zh{高山}}}
\end{exe}
 \mytextsc{clf}~: \textcolor{darkblue}{\textbf{\ipa{ɭɯ˧}}} objets ronds

\paragraph{\hspace{-0.5cm} {\Large \textcolor{darkblue}{\textbf{\ipa{ʁwɤ˧}}} \textsubscript{2}}} \hypertarget{Rw7\string_M2}{}
\markboth{\textcolor{darkblue}{\textbf{\ipa{ʁwɤ˧}}} \textsubscript{2}}{}
\textcolor{teal}{\mytextsc{nom}} \hspace{4pt} Ton~: M.

Village, hameau.
\textcolor{brown}{\zh{村寨,村落。}}


\begin{exe}
\sn \textcolor{darkblue}{\textbf{\ipa{ʁwɤ˧-qo˧}}}
\trans dans le village
\trans \textit{\textcolor{brown}{\zh{村子里}}}
\end{exe}
\begin{exe}
\sn \textcolor{darkblue}{\textbf{\ipa{[M23] ɖɯ˧-ʁwɤ˧ mɤ˧-ɲi˩: | ʈʂʰɯ˧-ʁwɤ˧... | ʈʂʰɯ˧-ʁwɤ˧…}}}
\trans Ce n'est pas le même endroit (littéralement: ce n'est pas le même village): ici, c'est (le village de)…; là, c'est (le village de)…
\trans \textit{\textcolor{brown}{\zh{它们不属于一个村落:这边,是……村,而那边,是……村。}}}
\end{exe}
 \mytextsc{clf}~: \textcolor{darkblue}{\textbf{\ipa{ʁwɤ˧}}} 

\paragraph{\hspace{-0.5cm} {\Large \textcolor{darkblue}{\textbf{\ipa{ʁwɤ˧}}} \textsubscript{3}}} \hypertarget{Rw7\string_M3}{}
\markboth{\textcolor{darkblue}{\textbf{\ipa{ʁwɤ˧}}} \textsubscript{3}}{}
\textcolor{teal}{\mytextsc{nom}} \hspace{4pt} Ton~: M.

Argent (non pas le métal, mais la monnaie).
\textcolor{brown}{\zh{钱。}}


\begin{exe}
\sn \textcolor{darkblue}{\textbf{\ipa{ɖʐe˧-ʁwɤ˧}}}
\trans argent
\trans \textit{\textcolor{brown}{\zh{钱}}}
\end{exe}


\paragraph{\hspace{-0.5cm} {\Large \textcolor{darkblue}{\textbf{\ipa{ʁwɤ˧\textsubscript{a}}}}}} \hypertarget{Rw7\string_Ma1}{}
\markboth{\textcolor{darkblue}{\textbf{\ipa{ʁwɤ˧\textsubscript{a}}}}}{}
\textcolor{teal}{\mytextsc{classificateur}} \hspace{4pt} Ton~: M\textsubscript{a}.

Classificateur des tas / amoncellements (de céréales, de bois coupé...); littéralement: 'une montagne de'.
\textcolor{brown}{\zh{量词:堆(一堆粮食、一堆柴……)。}}




\paragraph{\hspace{-0.5cm} {\Large \textcolor{darkblue}{\textbf{\ipa{ʁwɤ˧\textsubscript{a}}}}}} \hypertarget{Rw7\string_Ma1}{}
\markboth{\textcolor{darkblue}{\textbf{\ipa{ʁwɤ˧\textsubscript{a}}}}}{}
\textcolor{teal}{\mytextsc{verbe}} \hspace{4pt} Ton~: M\textsubscript{a}.

Amasser, entasser.
\textcolor{brown}{\zh{堆 (例如:堆积泥土)。}}


\begin{exe}
\sn \textcolor{darkblue}{\textbf{\ipa{ɖɯ˧-ʁwɤ˧ tʰi˧-ʁwɤ˧}}}
\trans faire un tas, amasser en tas
\trans \textit{\textcolor{brown}{\zh{堆在一起}}}
\end{exe}
\begin{exe}
\sn \textcolor{darkblue}{\textbf{\ipa{ɖɯ˧-ʁwɤ˧ tʰi˧-tɕɯ˥}}}
\trans ranger en tas
\trans \textit{\textcolor{brown}{\zh{收拾成一堆}}}
\end{exe}
\begin{exe}
\sn \textcolor{darkblue}{\textbf{\ipa{tso˧\textasciitilde{}tso˧ | gɤ˩-ʁwɤ˥ lv̩˩}}}
\trans entasser des objets
\trans \textit{\textcolor{brown}{\zh{东西堆起来}}}
\end{exe}
\begin{exe}
\sn \textcolor{darkblue}{\textbf{\ipa{ɖɯ˧-ʁwɤ˥-lv̩˩}}}
\trans rassembler en un tas (ex.: des piments épars, des noix, des fruits…)
\trans \textit{\textcolor{brown}{\zh{收拾成一堆(如:有果子散在地上,把它们堆在一起)}}}
\end{exe}
\begin{exe}
\sn \textcolor{darkblue}{\textbf{\ipa{tso˧\textasciitilde{}tso˧ ʁwɤ˩}}}
\trans entasser des choses
\trans \textit{\textcolor{brown}{\zh{东西堆在一起}}}
\end{exe}


\paragraph{\hspace{-0.5cm} {\Large \textcolor{darkblue}{\textbf{\ipa{ʁwɤ˧lɑ˩-bi˩}}}}} \hypertarget{Rw7\string_MlA\string_B-bi\string_B1}{}
\markboth{\textcolor{darkblue}{\textbf{\ipa{ʁwɤ˧lɑ˩-bi˩}}}}{}
\textcolor{teal}{\mytextsc{nom}} \hspace{4pt} Ton~: L\#-.

Walabie, un village de la plaine de Yongning. Il est peuplé de Na et de Pumi.
\textcolor{brown}{\zh{瓦拉别(永宁的一个村落)。}}


\begin{exe}
\sn \textcolor{darkblue}{\textbf{\ipa{ə˧go˧-ʁwɤ˧, | ʁwɤ˧lɑ˩-bi˩, | bæ˧ʁwɤ˧, | tʰo˧tsʰe\#˥, | pi˧tsʰe˩-di˩, | pɤ˧dʑɤ˩-di˩, | ʁwɤ˧tv̩˧}}}
\trans Villages au sortir de la plaine de Yongning; les deux premiers comportent une population na; le troisième est un village na; les suivants sont essentiellement des villages pumi/prinmi.
\trans \textit{\textcolor{brown}{\zh{永宁背向泸沽湖方向经过的村落。前两个村落拥有相当大的摩梭人口比例,第三个村落是摩梭村,最后一个是普米村。}}}
\end{exe}


\paragraph{\hspace{-0.5cm} {\Large \textcolor{darkblue}{\textbf{\ipa{ʁwɤ˧qʰv̩˧}}}}} \hypertarget{Rw7\string_Mq\string_hv\string_=\string_M1}{}
\markboth{\textcolor{darkblue}{\textbf{\ipa{ʁwɤ˧qʰv̩˧}}}}{}
\textcolor{teal}{\mytextsc{nom}} \hspace{4pt} Ton~: M.

Grotte, caverne.
\textcolor{brown}{\zh{山洞。}}


 \mytextsc{clf}~: \textcolor{darkblue}{\textbf{\ipa{ɭɯ˧}}} \textit{Voir~:} \hyperlink{Rw7\string_Mq\string_hv\string_=\string_Mdz£M\#\string_T1}{\textcolor{darkblue}{\textbf{\ipa{ʁwɤ˧qʰv̩˧dʑɯ\#˥}}}} 

\paragraph{\hspace{-0.5cm} {\Large \textcolor{darkblue}{\textbf{\ipa{ʁwɤ˧qʰv̩˧dʑɯ\#˥}}}}} \hypertarget{Rw7\string_Mq\string_hv\string_=\string_Mdz£M\#\string_T1}{}
\markboth{\textcolor{darkblue}{\textbf{\ipa{ʁwɤ˧qʰv̩˧dʑɯ\#˥}}}}{}
\textcolor{teal}{\mytextsc{nom}} \hspace{4pt} Ton~: \#H.

Grotte, caverne (où il est facile d'entrer).
\textcolor{brown}{\zh{山洞。}}


 \mytextsc{clf}~: \textcolor{darkblue}{\textbf{\ipa{ɭɯ˧}}} \textit{Voir~:} \hyperlink{Rw7\string_Mq\string_hv\string_=\string_M1}{\textcolor{darkblue}{\textbf{\ipa{ʁwɤ˧qʰv̩˧}}}} 

\paragraph{\hspace{-0.5cm} {\Large \textcolor{darkblue}{\textbf{\ipa{ʁwɤ˧\textasciitilde{}ʁwɤ˥\textsubscript{a}}}}}} \hypertarget{Rw7\string_M~Rw7\string_Ta1}{}
\markboth{\textcolor{darkblue}{\textbf{\ipa{ʁwɤ˧\textasciitilde{}ʁwɤ˥\textsubscript{a}}}}}{}
\textcolor{teal}{\mytextsc{verbe}} \hspace{4pt} Ton~: L\textsubscript{a}.

Discuter, négocier.
\textcolor{brown}{\zh{商量。}}


\begin{exe}
\sn \textcolor{darkblue}{\textbf{\ipa{ɖɯ˧-ʁwɤ˧\textasciitilde{}ʁwɤ˥-ɻ̍˩}}}
\trans \mytextsc{délimitatif} \string_ \mytextsc{red} \mytextsc{inchoatif}
\trans \textit{\textcolor{brown}{\zh{商量商量}}}
\end{exe}


\paragraph{\hspace{-0.5cm} {\Large \textcolor{darkblue}{\textbf{\ipa{ʁwɤ˧tv̩˧}}}}} \hypertarget{Rw7\string_Mtv\string_=\string_M1}{}
\markboth{\textcolor{darkblue}{\textbf{\ipa{ʁwɤ˧tv̩˧}}}}{}
\textcolor{teal}{\mytextsc{nom}} \hspace{4pt} Ton~: M.

Un village proche de Wenquan.
\textcolor{brown}{\zh{温泉乡的一个村落。}}


\begin{exe}
\sn \textcolor{darkblue}{\textbf{\ipa{ʁwɤ˧tv̩˧-ʁwɤ˧}}}
\trans même sens
\trans \textit{\textcolor{brown}{\zh{同上}}}
\end{exe}
\begin{exe}
\sn \textcolor{darkblue}{\textbf{\ipa{ə˧go˧-ʁwɤ˧, | ʁwɤ˧lɑ˩-bi˩, | bæ˧ʁwɤ˧, | tʰo˧tsʰe\#˥, | pi˧tsʰe˩-di˩, | pɤ˧dʑɤ˩-di˩, | ʁwɤ˧tv̩˧}}}
\trans Villages au sortir de la plaine de Yongning; les deux premiers comportent une population na; le troisième est un village na; les suivants sont essentiellement des villages pumi/prinmi.
\trans \textit{\textcolor{brown}{\zh{永宁背向泸沽湖方向经过的村落。前两个村落拥有相当大的摩梭人口比例,第三个村落是摩梭村,最后一个是普米村。}}}
\end{exe}
\begin{exe}
\sn \textcolor{darkblue}{\textbf{\ipa{ʁwɤ˧tv̩˧: | bɤ˧!}}}
\trans \textcolor{darkblue}{\textbf{\ipa{/ʁwɤ˧tv̩˧/}}}, c'est un village pumi!
\trans \textit{\textcolor{brown}{\zh{fv:/ʁwɤ˧tv̩˧/是一个普米族村落!}}}
\end{exe}


\paragraph{\hspace{-0.5cm} {\Large \textcolor{darkblue}{\textbf{\ipa{ʁwɤ˧ʐv̩\#˥}}}}} \hypertarget{Rw7\string_Mz`v\string_=\#\string_T1}{}
\markboth{\textcolor{darkblue}{\textbf{\ipa{ʁwɤ˧ʐv̩\#˥}}}}{}
\textcolor{teal}{\mytextsc{nom}} \hspace{4pt} Ton~: \#H.

Qiansuo (localité perçue par F4 comme comportant beaucoup de Yi, et des Chinois/Han, en plus des Na, d'où des contacts linguistiques/emprunts/mélanges).
\textcolor{brown}{\zh{前所。}}


\begin{exe}
\sn \textcolor{darkblue}{\textbf{\ipa{ʁwɤ˧ʐv̩˧-lo˩mæ˩}}}
\trans même sens
\trans \textit{\textcolor{brown}{\zh{同上}}}
\end{exe}
\begin{exe}
\sn \textcolor{darkblue}{\textbf{\ipa{ʁwɤ˧ʐv̩˧, | jɤ˧qʰɑ˧ dʑɤ˥; | hwɤ˧li˧-hɑ˧ mɤ˧-dʑo˧˥!}}}
\trans dicton ancien: ``A Qiansuo, le sarrasin amer pousse à merveille; les chats n'y ont rien à manger!" Explication: les chats ne mangent pas de sarrasin. / le sarrasin est bon, le chat n'aura rien à manger! (le chat passe en miaulant; mais il n'y a rien pour lui!) yyyy
\trans \textit{\textcolor{brown}{\zh{俗语:“前所,苦荞(庄稼)很好。猫,没得吃!”(说明:猫不吃苦荞。)}}}
\end{exe}
\begin{exe}
\sn \textcolor{darkblue}{\textbf{\ipa{ʁwɤ˧ʐv˧, | jɤ˧qʰɑ˧ dʑɤ˥, | hwɤ˧li˧˥ | hɑ˧ mɤ˧-dʑo˧!}}}
\trans comme ci-dessus
\trans \textit{\textcolor{brown}{\zh{同上}}}
\end{exe}


\paragraph{\hspace{-0.5cm} {\Large \textcolor{darkblue}{\textbf{\ipa{*ʁwɤ˩\textsubscript{a}}}}}} \hypertarget{*Rw7\string_Ba1}{}
\markboth{\textcolor{darkblue}{\textbf{\ipa{*ʁwɤ˩\textsubscript{a}}}}}{}
\textcolor{teal}{\mytextsc{verbe}} \hspace{4pt} Ton~: L\textsubscript{a}.

Discuter, négocier (racine extraite de la forme rédupliquée).
\textcolor{brown}{\zh{商量(单音节)。}}




\paragraph{\hspace{-0.5cm} {\Large \textcolor{darkblue}{\textbf{\ipa{ʁwɤ˩ʁo˩}}}}} \hypertarget{Rw7\string_BRo\string_B1}{}
\markboth{\textcolor{darkblue}{\textbf{\ipa{ʁwɤ˩ʁo˩}}}}{}
\textcolor{teal}{\mytextsc{nom}} \hspace{4pt} Ton~: L.

Collines, versant des montagnes (pas forcément très escarpé, plutôt collines que très fortes pentes).
\textcolor{brown}{\zh{山坡。}}


\begin{exe}
\sn \textcolor{darkblue}{\textbf{\ipa{ʁwɤ˩ʁo˩ dʑɤ˥bv̩˩ ə˩bi˩?}}}
\trans tu viens te détendre sur la montagne?
\trans \textit{\textcolor{brown}{\zh{去山上玩,好吗?}}}
\end{exe}
 \mytextsc{clf}~: \textcolor{darkblue}{\textbf{\ipa{ʁwɤ˧}}} 

\newpage
\section*{\centering- \textcolor{darkblue}{\textbf{\ipa{s}}} -}
\paragraph{\hspace{-0.5cm} {\Large \textcolor{darkblue}{\textbf{\ipa{sɑ˥}}}}} \hypertarget{sA\string_T1}{}
\markboth{\textcolor{darkblue}{\textbf{\ipa{sɑ˥}}}}{}
\textcolor{teal}{\mytextsc{nom}} \hspace{4pt} Ton~: \#H.
\ding{202} 
Lin, \textit{Linum usitatissimum}, plante textile et oléagineuse.
\textcolor{brown}{\zh{亚麻。}}


\ding{203} 
Chanvre, \textit{Cannabis sativa}, plante textile.
\textcolor{brown}{\zh{火麻、胡麻。}}


 \mytextsc{clf}~: \textcolor{darkblue}{\textbf{\ipa{qʰwæ˧˥}}} \textcolor{darkblue}{\textbf{\ipa{qʰwæ˧˥}}} filaments avant le filage

\paragraph{\hspace{-0.5cm} {\Large \textcolor{darkblue}{\textbf{\ipa{sɑ˥}}}}} \hypertarget{sA\string_T1}{}
\markboth{\textcolor{darkblue}{\textbf{\ipa{sɑ˥}}}}{}
\textcolor{teal}{\mytextsc{classificateur}} \hspace{4pt} Ton~: H*.

Classificateur des choses/objets, utilisé seulement en tournure négative: 'quoi que ce soit'.
\textcolor{brown}{\zh{量词:样,如:‘一样都没有’。}}


\begin{exe}
\sn \textcolor{darkblue}{\textbf{\ipa{ɖɯ˧-sɑ˥ | mɤ˧-dʑo˧!}}}
\trans Il n’y a rien du tout [à manger]! (phrase polie qd on invite quelqu'un à manger: on prie le convive d’excuser la pauvreté des mets proposés)
\trans \textit{\textcolor{brown}{\zh{一样也没有! / 没什么东西!(请客时的礼貌、自我贬低说法:请客人原谅菜不够丰盛)}}}
\end{exe}
\textit{Voir~:} \hyperlink{so\string_T2}{\textcolor{darkblue}{\textbf{\ipa{so˥}}} \textsubscript{2}} 

\paragraph{\hspace{-0.5cm} {\Large \textcolor{darkblue}{\textbf{\ipa{sɑ˧bo\#˥}}}}} \hypertarget{sA\string_Mbo\#\string_T1}{}
\markboth{\textcolor{darkblue}{\textbf{\ipa{sɑ˧bo\#˥}}}}{}
\textcolor{teal}{\mytextsc{nom}} \hspace{4pt} Ton~: \#H.

Quenouille: instrument en bois pour enrouler le fil, pour filer le chanvre.
\textcolor{brown}{\zh{卷线杆、拉线棒。}}


\begin{exe}
\sn \textcolor{darkblue}{\textbf{\ipa{sɑ˧bo˧-di˧˥}}}
\trans même sens
\trans \textit{\textcolor{brown}{\zh{同上}}}
\end{exe}
 \mytextsc{clf}~: \textcolor{darkblue}{\textbf{\ipa{nɑ˧}}} 

\paragraph{\hspace{-0.5cm} {\Large \textcolor{darkblue}{\textbf{\ipa{sɑ˧pʰv̩˧˥}}}}} \hypertarget{sA\string_Mp\string_hv\string_=\string_M\string_T1}{}
\markboth{\textcolor{darkblue}{\textbf{\ipa{sɑ˧pʰv̩˧˥}}}}{}
\textcolor{teal}{\mytextsc{nom}} \hspace{4pt} Ton~: MH\#.

Fil de lin, \textit{Cannabis sativa}.
\textcolor{brown}{\zh{麻线。}}


\begin{exe}
\sn \textcolor{darkblue}{\textbf{\ipa{sɑ˧pʰv̩˧-sɑ˧jɤ˥}}}
\trans fil de lin
\trans \textit{\textcolor{brown}{\zh{麻线}}}
\end{exe}
 \mytextsc{clf}~: \textcolor{darkblue}{\textbf{\ipa{ɭɯ˧}}} 

\paragraph{\hspace{-0.5cm} {\Large \textcolor{darkblue}{\textbf{\ipa{sɑ˧tɕɯ˧}}}}} \hypertarget{sA\string_Mts£M\string_M1}{}
\markboth{\textcolor{darkblue}{\textbf{\ipa{sɑ˧tɕɯ˧}}}}{}
\textcolor{teal}{\mytextsc{nom}} \hspace{4pt} Ton~: M.

Organe sexuel féminin, vagin (mot tabou).
\textcolor{brown}{\zh{女生殖器。}}


 \mytextsc{clf}~: \textcolor{darkblue}{\textbf{\ipa{ɭɯ˧}}} 

\paragraph{\hspace{-0.5cm} {\Large \textcolor{darkblue}{\textbf{\ipa{sɑ˧tsʰv̩˩}}}}} \hypertarget{sA\string_Mts\string_hv\string_=\string_B1}{}
\markboth{\textcolor{darkblue}{\textbf{\ipa{sɑ˧tsʰv̩˩}}}}{}
\textcolor{teal}{\mytextsc{nom}} \hspace{4pt} Ton~: L\#.

Vinaigre.
\textcolor{brown}{\zh{酸醋(汉语借词)。}}

 Emprunt~: chinois \textcolor{brown}{\zh{酸醋}}

\textit{Voir~:} \hyperlink{ts\string_hv\string_=\string_B\string_T1}{\textcolor{darkblue}{\textbf{\ipa{tsʰv̩˩˥}}}} 

\paragraph{\hspace{-0.5cm} {\Large \textcolor{darkblue}{\textbf{\ipa{sɑ˩mi˩}}}}} \hypertarget{sA\string_Bmi\string_B1}{}
\markboth{\textcolor{darkblue}{\textbf{\ipa{sɑ˩mi˩}}}}{}
\textcolor{teal}{\mytextsc{nom}} \hspace{4pt} Ton~: L.

Chanvre indien, kanja, marijuana, \textit{Cannabis indica} (plante euphorisante/psychotrope, qui est également comestible: les Na en tiraient de l'huile).
\textcolor{brown}{\zh{大麻。}}


\begin{exe}
\sn \textcolor{darkblue}{\textbf{\ipa{sɑ˩mi˩-mæ˩ɻæ˥, | dzɯ˧-kv̩˩!}}}
\trans L'huile de lin, c'est comestible/ça se mange!
\trans \textit{\textcolor{brown}{\zh{大麻油,是可以吃的!}}}
\end{exe}
 \mytextsc{clf}~: \textcolor{darkblue}{\textbf{\ipa{kɤ˧˥}}} 

\paragraph{\hspace{-0.5cm} {\Large \textcolor{darkblue}{\textbf{\ipa{sɑ˩tsʰi˩}}} \textsubscript{1}}} \hypertarget{sA\string_Bts\string_hi\string_B1}{}
\markboth{\textcolor{darkblue}{\textbf{\ipa{sɑ˩tsʰi˩}}} \textsubscript{1}}{}
\textcolor{teal}{\mytextsc{nom}} \hspace{4pt} Ton~: L.

Rame.
\textcolor{brown}{\zh{桨。}}


 \mytextsc{clf}~: \textcolor{darkblue}{\textbf{\ipa{nɑ˧}}} \textit{Voir~:} \hyperlink{sA\string_Bts\string_hi\string_B2}{\textcolor{darkblue}{\textbf{\ipa{sɑ˩tsʰi˩}}} \textsubscript{2}} 

\paragraph{\hspace{-0.5cm} {\Large \textcolor{darkblue}{\textbf{\ipa{sɑ˩tsʰi˩}}} \textsubscript{2}}} \hypertarget{sA\string_Bts\string_hi\string_B2}{}
\markboth{\textcolor{darkblue}{\textbf{\ipa{sɑ˩tsʰi˩}}} \textsubscript{2}}{}
\textcolor{teal}{\mytextsc{nom}} \hspace{4pt} Ton~: L.

Pale en bois utilisée pour touiller la pâtée des cochons; ressemble à une rame: l'ustensile de cuisine est de taille nettement plus petite que la rame des bateaux, mais de forme similaire.
\textcolor{brown}{\zh{像桨的木头工具,来搅拌猪食。}}


\textit{Voir~:} \hyperlink{sA\string_Bts\string_hi\string_B1}{\textcolor{darkblue}{\textbf{\ipa{sɑ˩tsʰi˩}}} \textsubscript{1}} 

\paragraph{\hspace{-0.5cm} {\Large \textcolor{darkblue}{\textbf{\ipa{sɑ˧˥}}}}} \hypertarget{sA\string_M\string_T1}{}
\markboth{\textcolor{darkblue}{\textbf{\ipa{sɑ˧˥}}}}{}
\textcolor{teal}{\mytextsc{verbe}} \hspace{4pt} Ton~: MH.

Livrer (à destination).
\textcolor{brown}{\zh{运送(货到目的地)。}}


\begin{exe}
\sn \textcolor{darkblue}{\textbf{\ipa{le˧-sɑ˧-tʰi˥-ki˩}}}
\trans même sens: livrer (un objet, une marchandise)
\trans \textit{\textcolor{brown}{\zh{送(东西到人家里)}}}
\end{exe}


\paragraph{\hspace{-0.5cm} {\Large \textcolor{darkblue}{\textbf{\ipa{sɑ˧˥\textsubscript{a}}}}}} \hypertarget{sA\string_M\string_Ta1}{}
\markboth{\textcolor{darkblue}{\textbf{\ipa{sɑ˧˥\textsubscript{a}}}}}{}
\textcolor{teal}{\mytextsc{classificateur}} \hspace{4pt} Ton~: MH\textsubscript{a}.

Classificateur des pattes de cochon conservées.
\textcolor{brown}{\zh{量词:腊猪脚(烟熏腊猪蹄子)(一只)。}}


\begin{exe}
\sn \textcolor{darkblue}{\textbf{\ipa{ʂe˧sɑ˩ | ɖɯ˧-sɑ˧˥}}}
\trans une patte de cochon conservée (viande des membres du cochon, conservée --séchée-- avec l'os)
\trans \textit{\textcolor{brown}{\zh{一只腊猪脚}}}
\end{exe}


\paragraph{\hspace{-0.5cm} {\Large \textcolor{darkblue}{\textbf{\ipa{sæ˧tsʰɤ˩}}}}} \hypertarget{s\{\string_Mts\string_h7\string_B1}{}
\markboth{\textcolor{darkblue}{\textbf{\ipa{sæ˧tsʰɤ˩}}}}{}
\textcolor{teal}{\mytextsc{nom}} \hspace{4pt} Ton~: L\#.

Légumes en saumure.
\textcolor{brown}{\zh{酸菜(汉语借词)、泡菜。}}

 Emprunt~: chinois \textcolor{brown}{\zh{酸菜}}



\paragraph{\hspace{-0.5cm} {\Large \textcolor{darkblue}{\textbf{\ipa{‑se}}}}} \hypertarget{‑se1}{}
\markboth{\textcolor{darkblue}{\textbf{\ipa{‑se}}}}{}
\textcolor{teal}{\mytextsc{particule}} \textcolor{teal}{\mytextsc{de}} \textcolor{teal}{\mytextsc{discours}} \hspace{4pt} Ton~: .

Marqueur de topique qui délimite l'objet ou l'ensemble concerné, le faisant ressortir contre l'arrière-plan formé par les éléments proches de lui dans l'espace conceptuel. Plutôt que d'un contraste, il s'agit d'une appréhension de l'objet concerné dans son unité.
\textcolor{brown}{\zh{\mytextsc{主题。}}}




\paragraph{\hspace{-0.5cm} {\Large \textcolor{darkblue}{\textbf{\ipa{se˥}}}}} \hypertarget{se\string_T1}{}
\markboth{\textcolor{darkblue}{\textbf{\ipa{se˥}}}}{}
\textcolor{teal}{\mytextsc{verbe}} \hspace{4pt} Ton~: H.

Marcher.
\textcolor{brown}{\zh{走、走路。}}


\begin{exe}
\sn \textcolor{darkblue}{\textbf{\ipa{le˧-se˥-ze˩}}}
\trans \mytextsc{accomp} \string_ \mytextsc{pfv}
\trans \textit{\textcolor{brown}{\zh{走了}}}
\end{exe}
\begin{exe}
\sn \textcolor{darkblue}{\textbf{\ipa{se˧-ho˥-ze˩!}}}
\trans [Le bébé] va bientôt marcher / va bientôt savoir marcher!
\trans \textit{\textcolor{brown}{\zh{(婴儿)很快就学会走路了!}}}
\end{exe}
\begin{exe}
\sn \textcolor{darkblue}{\textbf{\ipa{ʐɤ˩mi˩-qo˥ | so˩-hɑ̃˩ se˩˥}}}
\trans passer trois nuits en route / faire un voyage qui va durer trois jours (au sujet d'un trajet de trois nuits de Lijiang à Hanoi: un train de nuit; un car de nuit le lendemain; et un second train de nuit le troisième jour)
\trans \textit{\textcolor{brown}{\zh{走在路上三天时间、走三天}}}
\end{exe}


\paragraph{\hspace{-0.5cm} {\Large \textcolor{darkblue}{\textbf{\ipa{se˧}}}}} \hypertarget{se\string_M1}{}
\markboth{\textcolor{darkblue}{\textbf{\ipa{se˧}}}}{}
\textcolor{teal}{\mytextsc{nom}} \hspace{4pt} Ton~: M.

\textit{Naemorhedus goral}. Le même terme est employé par la locutrice pour décrire des photos de \textit{Pseudois schaeferi}, sorte de bouquetin.
\textcolor{brown}{\zh{岩羊。}}


 \mytextsc{clf}~: \textcolor{darkblue}{\textbf{\ipa{pʰo˧˥}}} 

\paragraph{\hspace{-0.5cm} {\Large \textcolor{darkblue}{\textbf{\ipa{se˧gi\#˥}}}}} \hypertarget{se\string_Mgi\#\string_T1}{}
\markboth{\textcolor{darkblue}{\textbf{\ipa{se˧gi\#˥}}}}{}
\textcolor{teal}{\mytextsc{nom}} \hspace{4pt} Ton~: \#H.

Nom anciennement donné par les Tibétains à la montagne \textcolor{darkblue}{\textbf{\ipa{/kɤ˧mv̩˧˥/}}} (nom chinois: Gemu).
\textcolor{brown}{\zh{格姆山的藏语名字。}}


\begin{exe}
\sn \textcolor{darkblue}{\textbf{\ipa{se˧gi˧-kɤ˩mv̩˩}}}
\trans même sens
\trans \textit{\textcolor{brown}{\zh{同上}}}
\end{exe}


\paragraph{\hspace{-0.5cm} {\Large \textcolor{darkblue}{\textbf{\ipa{se˧kʰɯ˩}}}}} \hypertarget{se\string_Mk\string_hM\string_B1}{}
\markboth{\textcolor{darkblue}{\textbf{\ipa{se˧kʰɯ˩}}}}{}
\textcolor{teal}{\mytextsc{nom}} \hspace{4pt} Ton~: L\#.

Satin.
\textcolor{brown}{\zh{缎子。}}


\begin{exe}
\sn \textcolor{darkblue}{\textbf{\ipa{se˧kʰɯ˩-ʁo˩ni˩}}}
\trans coiffe en satin
\trans \textit{\textcolor{brown}{\zh{缎子发带}}}
\end{exe}
 \mytextsc{clf}~: \textcolor{darkblue}{\textbf{\ipa{kʰɯ˩}}} 

\paragraph{\hspace{-0.5cm} {\Large \textcolor{darkblue}{\textbf{\ipa{se˧mi\#˥}}}}} \hypertarget{se\string_Mmi\#\string_T1}{}
\markboth{\textcolor{darkblue}{\textbf{\ipa{se˧mi\#˥}}}}{}
\textcolor{teal}{\mytextsc{nom}} \hspace{4pt} Ton~: \#H.

\textit{Naemorhedus goral} femelle.
\textcolor{brown}{\zh{母岩羊。}}


 \mytextsc{clf}~: \textcolor{darkblue}{\textbf{\ipa{mi˩}}} 

\paragraph{\hspace{-0.5cm} {\Large \textcolor{darkblue}{\textbf{\ipa{se˧nɑ\#˥}}}}} \hypertarget{se\string_MnA\#\string_T1}{}
\markboth{\textcolor{darkblue}{\textbf{\ipa{se˧nɑ\#˥}}}}{}
\textcolor{teal}{\mytextsc{adjectif}} \hspace{4pt} Ton~: \#H.

Avare.
\textcolor{brown}{\zh{吝啬。}}


\begin{exe}
\sn \textcolor{darkblue}{\textbf{\ipa{ʈʂʰɯ˧ | se˧nɑ˧-hĩ˧ ɖɯ˧-v̩˧ ɲi˩!}}}
\trans C'est quelqu'un d'avare!
\trans \textit{\textcolor{brown}{\zh{他是一个吝啬的人!}}}
\end{exe}
\begin{exe}
\sn \textcolor{darkblue}{\textbf{\ipa{ʈʂʰɯ˧ | ə˧-se˧nɑ˧? - se˧nɑ˧ | ʐwæ˩˥!}}}
\trans Est-il avare? - Oui, très avare!
\trans \textit{\textcolor{brown}{\zh{他吝啬吗? - 非常吝啬!}}}
\end{exe}


\paragraph{\hspace{-0.5cm} {\Large \textcolor{darkblue}{\textbf{\ipa{se˧pʰɤ˧}}}}} \hypertarget{se\string_Mp\string_h7\string_M1}{}
\markboth{\textcolor{darkblue}{\textbf{\ipa{se˧pʰɤ˧}}}}{}
\textcolor{teal}{\mytextsc{nom}} \hspace{4pt} Ton~: M.

Complications.
\textcolor{brown}{\zh{大惊小怪,麻烦。}}


\begin{exe}
\sn \textcolor{darkblue}{\textbf{\ipa{se˧pʰɤ˧ ʝi˧}}}
\trans se faire toute une affaire de quelque chose, s’en faire au point de porter comme une pierre dans le cœur
\trans \textit{\textcolor{brown}{\zh{小事大作}}}
\end{exe}
 \mytextsc{clf}~: \textcolor{darkblue}{\textbf{\ipa{kʰwɤ˥}}} 

\paragraph{\hspace{-0.5cm} {\Large \textcolor{darkblue}{\textbf{\ipa{se˧pʰv̩\#˥}}}}} \hypertarget{se\string_Mp\string_hv\string_=\#\string_T1}{}
\markboth{\textcolor{darkblue}{\textbf{\ipa{se˧pʰv̩\#˥}}}}{}
\textcolor{teal}{\mytextsc{nom}} \hspace{4pt} Ton~: \#H.

\textit{Naemorhedus goral} mâle.
\textcolor{brown}{\zh{公岩羊。}}


 \mytextsc{clf}~: \textcolor{darkblue}{\textbf{\ipa{mi˩}}} 

\paragraph{\hspace{-0.5cm} {\Large \textcolor{darkblue}{\textbf{\ipa{se˧ʂɯ˩}}}}} \hypertarget{se\string_Ms`M\string_B1}{}
\markboth{\textcolor{darkblue}{\textbf{\ipa{se˧ʂɯ˩}}}}{}
\textcolor{teal}{\mytextsc{verbe}} \hspace{4pt} Ton~: L\#.

Gaspiller.
\textcolor{brown}{\zh{浪费。}}


\begin{exe}
\sn \textcolor{darkblue}{\textbf{\ipa{ɖwæ˧˥ | se˧ʂɯ˩!}}}
\trans C'est du gaspillage!
\trans \textit{\textcolor{brown}{\zh{很浪费! / 太浪费了!}}}
\end{exe}


\paragraph{\hspace{-0.5cm} {\Large \textcolor{darkblue}{\textbf{\ipa{se˧tʰo˩}}}}} \hypertarget{se\string_Mt\string_ho\string_B1}{}
\markboth{\textcolor{darkblue}{\textbf{\ipa{se˧tʰo˩}}}}{}
\textcolor{teal}{\mytextsc{nom}} \hspace{4pt} Ton~: L.

Tenon.
\textcolor{brown}{\zh{榫头(汉语借词)。}}

 Emprunt~: chinois \textcolor{brown}{\zh{榫头}}

\begin{exe}
\sn \textcolor{darkblue}{\textbf{\ipa{se˧tʰo˩ | ɖɯ˧-ɭɯ˧}}}
\trans un tenon
\trans \textit{\textcolor{brown}{\zh{一个榫头}}}
\end{exe}
 \mytextsc{clf}~: \textcolor{darkblue}{\textbf{\ipa{ɭɯ˧}}} 

\paragraph{\hspace{-0.5cm} {\Large \textcolor{darkblue}{\textbf{\ipa{se˧zo\#˥}}}}} \hypertarget{se\string_Mzo\#\string_T1}{}
\markboth{\textcolor{darkblue}{\textbf{\ipa{se˧zo\#˥}}}}{}
\textcolor{teal}{\mytextsc{nom}} \hspace{4pt} Ton~: \#H.

Petit de \textit{Naemorhedus goral}.
\textcolor{brown}{\zh{小岩羊。}}


 \mytextsc{clf}~: \textcolor{darkblue}{\textbf{\ipa{ɭɯ˧}}} 

\paragraph{\hspace{-0.5cm} {\Large \textcolor{darkblue}{\textbf{\ipa{se˧ʐɯ˩}}}}} \hypertarget{se\string_Mz`M\string_B1}{}
\markboth{\textcolor{darkblue}{\textbf{\ipa{se˧ʐɯ˩}}}}{}
\textcolor{teal}{\mytextsc{nom}} \hspace{4pt} Ton~: L\#.

Anniversaire.
\textcolor{brown}{\zh{生日(汉语借词)。}}

 Emprunt~: chinois \textcolor{brown}{\zh{生日}}

\begin{exe}
\sn \textcolor{darkblue}{\textbf{\ipa{se˧ʐɯ˩ ko˩}}}
\trans fêter un anniversaire
\trans \textit{\textcolor{brown}{\zh{过生日}}}
\end{exe}


\paragraph{\hspace{-0.5cm} {\Large \textcolor{darkblue}{\textbf{\ipa{‑se˩}}}}} \hypertarget{‑se\string_B1}{}
\markboth{\textcolor{darkblue}{\textbf{\ipa{‑se˩}}}}{}
\textcolor{teal}{\mytextsc{suffixe}} \hspace{4pt} Ton~: L.

Suffixe indiquant l'achèvement d'une action: l'action est conduite à son terme.
\textcolor{brown}{\zh{\mytextsc{完成。}}}


\begin{exe}
\sn \textcolor{darkblue}{\textbf{\ipa{se˩-ze˥!}}}
\trans C'est fini!
\trans \textit{\textcolor{brown}{\zh{完了!}}}
\end{exe}
\begin{exe}
\sn \textcolor{darkblue}{\textbf{\ipa{no˧ | tʰi˧-dzi˩-kʰɯ˩-se˩-dʑo˩, | dʑɯ˩-tsʰi˧ ɖɯ˧-qʰwɤ˧ pʰv̩˥ | tʰi˧-ki˧!}}}
\trans Après que tu te sois assis, je te verse un verre d’eau chaude.
\trans \textit{\textcolor{brown}{\zh{让(你)坐下以后,(我)给你倒一杯开水。}}}
\end{exe}


\paragraph{\hspace{-0.5cm} {\Large \textcolor{darkblue}{\textbf{\ipa{se˩\textsubscript{a}}}}}} \hypertarget{se\string_Ba1}{}
\markboth{\textcolor{darkblue}{\textbf{\ipa{se˩\textsubscript{a}}}}}{}
\textcolor{teal}{\mytextsc{verbe}} \hspace{4pt} Ton~: L\textsubscript{a}.

Achever.
\textcolor{brown}{\zh{完成。}}


\begin{exe}
\sn \textcolor{darkblue}{\textbf{\ipa{le˧-ʝi˥ | le˧-se˩-ze˩!}}}
\trans (je) l'ai fait, j'ai fini!
\trans \textit{\textcolor{brown}{\zh{做完了! / 完成了!}}}
\end{exe}
\begin{exe}
\sn \textcolor{darkblue}{\textbf{\ipa{le˧-se˧\textasciitilde{}se˥-ze˩!}}}
\trans C'est fini, c'est achevé!
\trans \textit{\textcolor{brown}{\zh{完成了!}}}
\end{exe}
\begin{exe}
\sn \textcolor{darkblue}{\textbf{\ipa{mɤ˧-se˩}}}
\trans Ce n'est pas fini!
\trans \textit{\textcolor{brown}{\zh{没有完!}}}
\end{exe}
\begin{exe}
\sn \textcolor{darkblue}{\textbf{\ipa{se˩˥ | -dʑo˩, | se˩-mɤ˩-tʰɑ˩˥! | dʑɤ˩˥ | -dʑo˩, | dʑɤ˩-kʰɯ˧ tʰɑ˥!}}}
\trans Parenthèse au sujet de la documentation linguistique: on ne peut pas en voir le bout (tout collecter de façon exhaustive); mais on peut réaliser de belles choses!
\trans \textit{\textcolor{brown}{\zh{(想做)完,但是没办法做完!不过最后还是可以做得很好!(情景:谈及收集语言的工作)}}}
\end{exe}


\paragraph{\hspace{-0.5cm} {\Large \textcolor{darkblue}{\textbf{\ipa{se˩di˩}}}}} \hypertarget{se\string_Bdi\string_B1}{}
\markboth{\textcolor{darkblue}{\textbf{\ipa{se˩di˩}}}}{}
\textcolor{teal}{\mytextsc{nom}} \hspace{4pt} Ton~: L.

Scie.
\textcolor{brown}{\zh{锯。}}


\begin{exe}
\sn \textcolor{darkblue}{\textbf{\ipa{se˩di˩˥ | ɖɯ˩-hĩ˩˥ | ɖɯ˧-nɑ˧}}}
\trans une grande scie
\trans \textit{\textcolor{brown}{\zh{一把大锯}}}
\end{exe}
\begin{exe}
\sn \textcolor{darkblue}{\textbf{\ipa{se˩di˩˥ | tɕi˩-hĩ˩˥ | ɖɯ˧-nɑ˧}}}
\trans une petite scie
\trans \textit{\textcolor{brown}{\zh{一把小锯}}}
\end{exe}
\begin{exe}
\sn \textcolor{darkblue}{\textbf{\ipa{se˩di˩˥ | ɬi˧-hĩ˧ | ɖɯ˧-nɑ˩}}}
\trans une scie de taille moyenne
\trans \textit{\textcolor{brown}{\zh{中间大小的锯子}}}
\end{exe}
 \mytextsc{clf}~: \textcolor{darkblue}{\textbf{\ipa{nɑ˧}}} 

\paragraph{\hspace{-0.5cm} {\Large \textcolor{darkblue}{\textbf{\ipa{se˩gwɤ˩mi˥}}}}} \hypertarget{se\string_Bgw7\string_Bmi\string_T1}{}
\markboth{\textcolor{darkblue}{\textbf{\ipa{se˩gwɤ˩mi˥}}}}{}
\textcolor{teal}{\mytextsc{nom}} \hspace{4pt} Ton~: L+H\#.

Vautour. Le terme n'est pas restreint aux vautours femelles; dans l'état actuel de la langue, il ne fournit pas d'indication de sexe.
\textcolor{brown}{\zh{雕(不仅来指母雕)。}}


\begin{exe}
\sn \textcolor{darkblue}{\textbf{\ipa{se˩gwɤ˩mi˥-pʰv̩˩}}}
\trans vautour mâle
\trans \textit{\textcolor{brown}{\zh{公雕}}}
\end{exe}
\begin{exe}
\sn \textcolor{darkblue}{\textbf{\ipa{se˩gwɤ˩mi˥-zo˩}}}
\trans petit vautour, bébé vautour
\trans \textit{\textcolor{brown}{\zh{小雕}}}
\end{exe}
\begin{exe}
\sn \textcolor{darkblue}{\textbf{\ipa{se˩gwɤ˩mi˥-ʈʂʰɯ˩, | mi˩ ɲi˥!}}}
\trans Ce vautour, c'est une femelle!
\trans \textit{\textcolor{brown}{\zh{这只雕是母的!}}}
\end{exe}
 \mytextsc{clf}~: \textcolor{darkblue}{\textbf{\ipa{mi˩}}} 

\paragraph{\hspace{-0.5cm} {\Large \textcolor{darkblue}{\textbf{\ipa{sɤ˥}}}}} \hypertarget{s7\string_T1}{}
\markboth{\textcolor{darkblue}{\textbf{\ipa{sɤ˥}}}}{}
\textcolor{teal}{\mytextsc{nom}} \hspace{4pt} Ton~: \#H.

Sang.
\textcolor{brown}{\zh{血。}}


 \mytextsc{clf}~: \textcolor{darkblue}{\textbf{\ipa{ʈʰɤ˥}}} 

\paragraph{\hspace{-0.5cm} {\Large \textcolor{darkblue}{\textbf{\ipa{sɤ˧ɭɯ˩}}}}} \hypertarget{s7\string_Ml\string_RM\string_B1}{}
\markboth{\textcolor{darkblue}{\textbf{\ipa{sɤ˧ɭɯ˩}}}}{}
\textcolor{teal}{\mytextsc{nom}} \hspace{4pt} Ton~: L\#.

Poire.
\textcolor{brown}{\zh{梨子。}}


 \mytextsc{clf}~: \textcolor{darkblue}{\textbf{\ipa{kʰɤ˧˥}}} \textcolor{darkblue}{\textbf{\ipa{ɭɯ˧}}} 

\paragraph{\hspace{-0.5cm} {\Large \textcolor{darkblue}{\textbf{\ipa{sɤ˧sɤ˧˥}}}}} \hypertarget{s7\string_Ms7\string_M\string_T1}{}
\markboth{\textcolor{darkblue}{\textbf{\ipa{sɤ˧sɤ˧˥}}}}{}
\textcolor{teal}{\mytextsc{adjectif}} \hspace{4pt} Ton~: MH\#.

Agréable, plaisant (circonstances).
\textcolor{brown}{\zh{舒畅。}}


\begin{exe}
\sn \textcolor{darkblue}{\textbf{\ipa{si˧dzi˩-ʈʰæ˩qo˩dzi˩, | sɤ˧sɤ˧˥ | ʐwæ˩˥!}}}
\trans assis sous cet arbre, c'est le bonheur!
\trans \textit{\textcolor{brown}{\zh{在树下坐着,感到很舒畅!}}}
\end{exe}
\begin{exe}
\sn \textcolor{darkblue}{\textbf{\ipa{ʈʂʰɯ˧-ɳɯ˧ | ɖɯ˧-ɖʐɯ˩ gwɤ˩-dʑo˩, | sɤ˧sɤ˧˥ | ʐwæ˩˥!}}}
\trans Il a chanté un moment; c'était vraiment plaisant!
\trans \textit{\textcolor{brown}{\zh{他唱了一会,真舒畅!}}}
\end{exe}


\paragraph{\hspace{-0.5cm} {\Large \textcolor{darkblue}{\textbf{\ipa{sɤ˧tʰo˧˥}}}}} \hypertarget{s7\string_Mt\string_ho\string_M\string_T1}{}
\markboth{\textcolor{darkblue}{\textbf{\ipa{sɤ˧tʰo˧˥}}}}{}
\textcolor{teal}{\mytextsc{nom}} \hspace{4pt} Ton~: MH\#.

Sorte de pin.
\textcolor{brown}{\zh{一种松树。}}Dialecte chinois local~: \zh{阔松。}


\begin{exe}
\sn \textcolor{darkblue}{\textbf{\ipa{sɤ˧tʰo˧-dzi˧˥}}}
\trans même sens (désigne une espèce de pin)
\trans \textit{\textcolor{brown}{\zh{同上}}}
\end{exe}
 \mytextsc{clf}~: \textcolor{darkblue}{\textbf{\ipa{dzi˩}}} 

\paragraph{\hspace{-0.5cm} {\Large \textcolor{darkblue}{\textbf{\ipa{sɤ˧tsi˥}}}}} \hypertarget{s7\string_Mtsi\string_T1}{}
\markboth{\textcolor{darkblue}{\textbf{\ipa{sɤ˧tsi˥}}}}{}
\textcolor{teal}{\mytextsc{nom}} \hspace{4pt} Ton~: H\#.

Veines.
\textcolor{brown}{\zh{血管。}}


 \mytextsc{clf}~: \textcolor{darkblue}{\textbf{\ipa{kʰɯ˩}}} 

\paragraph{\hspace{-0.5cm} {\Large \textcolor{darkblue}{\textbf{\ipa{sɤ˩˥}}}}} \hypertarget{s7\string_B\string_T1}{}
\markboth{\textcolor{darkblue}{\textbf{\ipa{sɤ˩˥}}}}{}
\textcolor{teal}{\mytextsc{nom}} \hspace{4pt} Ton~: LH.

Grain de beauté.
\textcolor{brown}{\zh{黑痣。}}


 \mytextsc{clf}~: \textcolor{darkblue}{\textbf{\ipa{ɭɯ˧}}} 

\paragraph{\hspace{-0.5cm} {\Large \textcolor{darkblue}{\textbf{\ipa{si˥}}}}} \hypertarget{si\string_T1}{}
\markboth{\textcolor{darkblue}{\textbf{\ipa{si˥}}}}{}
\textcolor{teal}{\mytextsc{nom}} \hspace{4pt} Ton~: \#H.

Bois.
\textcolor{brown}{\zh{木头。}}


\begin{exe}
\sn \textcolor{darkblue}{\textbf{\ipa{si˧-mo˩}}}
\trans bois mort
\trans \textit{\textcolor{brown}{\zh{枯木}}}
\end{exe}
 \mytextsc{clf}~: \textcolor{darkblue}{\textbf{\ipa{kɤ˧˥}}} 

\paragraph{\hspace{-0.5cm} {\Large \textcolor{darkblue}{\textbf{\ipa{si˧\textsubscript{a}}}}}} \hypertarget{si\string_Ma1}{}
\markboth{\textcolor{darkblue}{\textbf{\ipa{si˧\textsubscript{a}}}}}{}
\textcolor{teal}{\mytextsc{verbe}} \hspace{4pt} Ton~: M\textsubscript{a}.

Choisir.
\textcolor{brown}{\zh{挑选。}}


\begin{exe}
\sn \textcolor{darkblue}{\textbf{\ipa{le˧-si˧-ze˧}}}
\trans \mytextsc{accomp} \string_ \mytextsc{pfv}
\trans \textit{\textcolor{brown}{\zh{选了}}}
\end{exe}
\begin{exe}
\sn \textcolor{darkblue}{\textbf{\ipa{no˧ si˧-bi˧!}}}
\trans Tu choisis! / A toi le choix!
\trans \textit{\textcolor{brown}{\zh{你要选!}}}
\end{exe}
\begin{exe}
\sn \textcolor{darkblue}{\textbf{\ipa{njɤ˧-ɳɯ˧ si˧-bi˧!}}}
\trans C'est moi qui choisis!
\trans \textit{\textcolor{brown}{\zh{是我来选!}}}
\end{exe}
\begin{exe}
\sn \textcolor{darkblue}{\textbf{\ipa{le˧-si˥\textasciitilde{}si˩}}}
\trans \mytextsc{accomp} \string_ \mytextsc{red}
\trans \textit{\textcolor{brown}{\zh{\mytextsc{accomp} \string_ \mytextsc{red}}}}
\end{exe}
\begin{exe}
\sn \textcolor{darkblue}{\textbf{\ipa{tso˧\textasciitilde{}tso˧ si˩}}}
\trans choisir des choses
\trans \textit{\textcolor{brown}{\zh{选东西}}}
\end{exe}
\begin{exe}
\sn \textcolor{darkblue}{\textbf{\ipa{tso˧\textasciitilde{}tso˧ si˧\textasciitilde{}si˥}}}
\trans choisir des choses
\trans \textit{\textcolor{brown}{\zh{选选东西}}}
\end{exe}
\begin{exe}
\sn \textcolor{darkblue}{\textbf{\ipa{dʑɤ˩-hĩ˥ | si˧}}}
\trans choisir les plus beaux; en choisir de beaux (par exemple: sur la montagne, lorsqu'on choisit des arbres à abattre pour donner du bois du charpente)
\trans \textit{\textcolor{brown}{\zh{挑好的}}}
\end{exe}


\paragraph{\hspace{-0.5cm} {\Large \textcolor{darkblue}{\textbf{\ipa{si˧bv̩˧}}}}} \hypertarget{si\string_Mbv\string_=\string_M1}{}
\markboth{\textcolor{darkblue}{\textbf{\ipa{si˧bv̩˧}}}}{}
\textcolor{teal}{\mytextsc{nom}} \hspace{4pt} Ton~: M.

Démon (forme obtenue par élicitation; nettement moins courante que la forme féminine).
\textcolor{brown}{\zh{鬼。}}


 \mytextsc{clf}~: \textcolor{darkblue}{\textbf{\ipa{v̩˧}}} 

\paragraph{\hspace{-0.5cm} {\Large \textcolor{darkblue}{\textbf{\ipa{si˧bv̩˧-mi\#˥}}}}} \hypertarget{si\string_Mbv\string_=\string_M-mi\#\string_T1}{}
\markboth{\textcolor{darkblue}{\textbf{\ipa{si˧bv̩˧-mi\#˥}}}}{}
\textcolor{teal}{\mytextsc{nom}} \hspace{4pt} Ton~: \#H.

Démone.
\textcolor{brown}{\zh{妖精。}}


 \mytextsc{clf}~: \textcolor{darkblue}{\textbf{\ipa{v̩˧}}} 

\paragraph{\hspace{-0.5cm} {\Large \textcolor{darkblue}{\textbf{\ipa{si˧bv̩˧-zo\#˥}}}}} \hypertarget{si\string_Mbv\string_=\string_M-zo\#\string_T1}{}
\markboth{\textcolor{darkblue}{\textbf{\ipa{si˧bv̩˧-zo\#˥}}}}{}
\textcolor{teal}{\mytextsc{nom}} \hspace{4pt} Ton~: \#H.

Démon masculin (forme élicitée, sur la base de la forme féminine; est un mot qui existe, mais peu courant).
\textcolor{brown}{\zh{鬼。}}


 \mytextsc{clf}~: \textcolor{darkblue}{\textbf{\ipa{v̩˧}}} 

\paragraph{\hspace{-0.5cm} {\Large \textcolor{darkblue}{\textbf{\ipa{si˧ɕi˧˥}}}}} \hypertarget{si\string_Ms£i\string_M\string_T1}{}
\markboth{\textcolor{darkblue}{\textbf{\ipa{si˧ɕi˧˥}}}}{}
\textcolor{teal}{\mytextsc{nom}} \hspace{4pt} Ton~: MH\#.

Forêt (clairsemée).
\textcolor{brown}{\zh{森林。}}


\begin{exe}
\sn \textcolor{darkblue}{\textbf{\ipa{[F5] tʰo˧ɕi˧˥}}}
\trans forêt de pins
\trans \textit{\textcolor{brown}{\zh{松树森林}}}
\end{exe}
 \mytextsc{clf}~: \textcolor{darkblue}{\textbf{\ipa{pʰæ˧˥}}} 

\paragraph{\hspace{-0.5cm} {\Large \textcolor{darkblue}{\textbf{\ipa{si˧dzi˩}}}}} \hypertarget{si\string_Mdzi\string_B1}{}
\markboth{\textcolor{darkblue}{\textbf{\ipa{si˧dzi˩}}}}{}
\textcolor{teal}{\mytextsc{nom}} \hspace{4pt} Ton~: L\#.

Arbre.
\textcolor{brown}{\zh{树。}}


 \mytextsc{clf}~: \textcolor{darkblue}{\textbf{\ipa{dzi˩, ʝi˧}}} 

\paragraph{\hspace{-0.5cm} {\Large \textcolor{darkblue}{\textbf{\ipa{si˧dzi˩-mv̩˩tsɯ˩}}}}} \hypertarget{si\string_Mdzi\string_B-mv\string_=\string_BtsM\string_B1}{}
\markboth{\textcolor{darkblue}{\textbf{\ipa{si˧dzi˩-mv̩˩tsɯ˩}}}}{}
\textcolor{teal}{\mytextsc{nom}} \hspace{4pt} Ton~: \#L-L.

Radicelle, petites racine.
\textcolor{brown}{\zh{胚根。}}




\paragraph{\hspace{-0.5cm} {\Large \textcolor{darkblue}{\textbf{\ipa{si˧dʑɯ˥}}}}} \hypertarget{si\string_Mdz£M\string_T1}{}
\markboth{\textcolor{darkblue}{\textbf{\ipa{si˧dʑɯ˥}}}}{}
\textcolor{teal}{\mytextsc{nom}} \hspace{4pt} Ton~: H\#.

Petit bois, pour faire démarrer le feu; à Yongning, ce qu'on utilise: des morceaux de pin gorgés de résine, utilisés spécialement à cet effet.
\textcolor{brown}{\zh{火煤、火捻、火种、劈柴、引柴。}}


 \mytextsc{clf}~: \textcolor{darkblue}{\textbf{\ipa{kʰwɤ˥}}} 

\paragraph{\hspace{-0.5cm} {\Large \textcolor{darkblue}{\textbf{\ipa{si˧gɯ˧}}}}} \hypertarget{si\string_MgM\string_M1}{}
\markboth{\textcolor{darkblue}{\textbf{\ipa{si˧gɯ˧}}}}{}
\textcolor{teal}{\mytextsc{nom}} \hspace{4pt} Ton~: M.

Lion.
\textcolor{brown}{\zh{狮子。}}

 Emprunt~: tibétain senggeསེང་གེ

 \mytextsc{clf}~: \textcolor{darkblue}{\textbf{\ipa{mi˩}}} 

\paragraph{\hspace{-0.5cm} {\Large \textcolor{darkblue}{\textbf{\ipa{si˧gɯ˧-mi˩}}}}} \hypertarget{si\string_MgM\string_M-mi\string_B1}{}
\markboth{\textcolor{darkblue}{\textbf{\ipa{si˧gɯ˧-mi˩}}}}{}
\textcolor{teal}{\mytextsc{nom}} \hspace{4pt} Ton~: \mytextsc{L}.

Lionne.
\textcolor{brown}{\zh{母狮。}}


 \mytextsc{clf}~: \textcolor{darkblue}{\textbf{\ipa{mi˩}}} 

\paragraph{\hspace{-0.5cm} {\Large \textcolor{darkblue}{\textbf{\ipa{si˧gɯ˧-pʰv̩\#˥}}}}} \hypertarget{si\string_MgM\string_M-p\string_hv\string_=\#\string_T1}{}
\markboth{\textcolor{darkblue}{\textbf{\ipa{si˧gɯ˧-pʰv̩\#˥}}}}{}
\textcolor{teal}{\mytextsc{nom}} \hspace{4pt} Ton~: \#H.

Lion (mâle).
\textcolor{brown}{\zh{公狮子。}}


 \mytextsc{clf}~: \textcolor{darkblue}{\textbf{\ipa{mi˩}}} 

\paragraph{\hspace{-0.5cm} {\Large \textcolor{darkblue}{\textbf{\ipa{si˧gɯ˧-tsʰo\#˥}}}}} \hypertarget{si\string_MgM\string_M-ts\string_ho\#\string_T1}{}
\markboth{\textcolor{darkblue}{\textbf{\ipa{si˧gɯ˧-tsʰo\#˥}}}}{}
\textcolor{teal}{\mytextsc{nom}} \hspace{4pt} Ton~: \#H.

Danse du Lion: spectacle masqué, commandité par le seigneur féodal, qui participait lui-même à certaines des danses.
\textcolor{brown}{\zh{狮子舞:土司准备的礼仪性表演。土司也亲自参与舞蹈。}}




\paragraph{\hspace{-0.5cm} {\Large \textcolor{darkblue}{\textbf{\ipa{si˧gɯ˧-zo\#˥}}}}} \hypertarget{si\string_MgM\string_M-zo\#\string_T1}{}
\markboth{\textcolor{darkblue}{\textbf{\ipa{si˧gɯ˧-zo\#˥}}}}{}
\textcolor{teal}{\mytextsc{nom}} \hspace{4pt} Ton~: \#H.

Lionceau, petit lion.
\textcolor{brown}{\zh{小狮子。}}


 \mytextsc{clf}~: \textcolor{darkblue}{\textbf{\ipa{ɭɯ˧}}} \textcolor{darkblue}{\textbf{\ipa{mi˩}}} 

\paragraph{\hspace{-0.5cm} {\Large \textcolor{darkblue}{\textbf{\ipa{si˧kɤ˧˥}}}}} \hypertarget{si\string_Mk7\string_M\string_T1}{}
\markboth{\textcolor{darkblue}{\textbf{\ipa{si˧kɤ˧˥}}}}{}
\textcolor{teal}{\mytextsc{nom}} \hspace{4pt} Ton~: MH\#.

Branche; petite branche; bâton, gourdin, canne pour marcher.
\textcolor{brown}{\zh{树枝、小树枝,棍子。}}


 \mytextsc{clf}~: \textcolor{darkblue}{\textbf{\ipa{kɤ˧˥}}} 

\paragraph{\hspace{-0.5cm} {\Large \textcolor{darkblue}{\textbf{\ipa{si˧kwɤ˩}}}}} \hypertarget{si\string_Mkw7\string_B1}{}
\markboth{\textcolor{darkblue}{\textbf{\ipa{si˧kwɤ˩}}}}{}
\textcolor{teal}{\mytextsc{nom}} \hspace{4pt} Ton~: \#L.

Structure, charpente, gros œuvre en bois (d'une maison).
\textcolor{brown}{\zh{木头框架,如:房子的木头框架。}}


\begin{exe}
\sn \textcolor{darkblue}{\textbf{\ipa{ʑi˧mi˧-si˧kwɤ˩}}}
\trans la charpente d'une maison
\trans \textit{\textcolor{brown}{\zh{房子的木头框架}}}
\end{exe}
 \mytextsc{clf}~: \textcolor{darkblue}{\textbf{\ipa{kwɤ˩}}} 

\paragraph{\hspace{-0.5cm} {\Large \textcolor{darkblue}{\textbf{\ipa{si˧kʰɯ\#˥}}}}} \hypertarget{si\string_Mk\string_hM\#\string_T1}{}
\markboth{\textcolor{darkblue}{\textbf{\ipa{si˧kʰɯ\#˥}}}}{}
\textcolor{teal}{\mytextsc{nom}} \hspace{4pt} Ton~: H\#.

Yyyy.
\textcolor{brown}{\zh{色疙瘩。}}Dialecte chinois local~: \zh{根三香。}


\begin{exe}
\sn \textcolor{darkblue}{\textbf{\ipa{si˧kʰɯ˧-bæ˥bæ˩}}}
\trans fleur de...
\trans \textit{\textcolor{brown}{\zh{色疙瘩花}}}
\end{exe}
\textit{Voir~:} \textcolor{darkblue}{\textbf{\ipa{si˧kʰɯ˧ɭɯ˧bv̩˥}}} 

\paragraph{\hspace{-0.5cm} {\Large \textcolor{darkblue}{\textbf{\ipa{si˧kʰɯ˧-ɭɯ˧bv̩˥}}}}} \hypertarget{si\string_Mk\string_hM\string_M-l\string_RM\string_Mbv\string_=\string_T1}{}
\markboth{\textcolor{darkblue}{\textbf{\ipa{si˧kʰɯ˧-ɭɯ˧bv̩˥}}}}{}
\textcolor{teal}{\mytextsc{nom}} \hspace{4pt} Ton~: H\#.

Pivoine blanche de Chine, \textit{Paeonia lactiflora}.
\textcolor{brown}{\zh{白芍药。}}


 \mytextsc{clf}~: \textcolor{darkblue}{\textbf{\ipa{kɤ˧˥}}} \textit{Voir~:} \hyperlink{si\string_Mk\string_hM\#\string_T1}{\textcolor{darkblue}{\textbf{\ipa{si˧kʰɯ\#˥}}}} 

\paragraph{\hspace{-0.5cm} {\Large \textcolor{darkblue}{\textbf{\ipa{si˧nɑ˥}}}}} \hypertarget{si\string_MnA\string_T1}{}
\markboth{\textcolor{darkblue}{\textbf{\ipa{si˧nɑ˥}}}}{}
\textcolor{teal}{\mytextsc{nom}} \hspace{4pt} Ton~: H\#.

Forêt épaisse.
\textcolor{brown}{\zh{森林深处(难走路)。}}


 \mytextsc{clf}~: \textcolor{darkblue}{\textbf{\ipa{pʰæ˧˥}}} 

\paragraph{\hspace{-0.5cm} {\Large \textcolor{darkblue}{\textbf{\ipa{si˧qʰwæ˩}}}}} \hypertarget{si\string_Mq\string_hw\{\string_B1}{}
\markboth{\textcolor{darkblue}{\textbf{\ipa{si˧qʰwæ˩}}}}{}
\textcolor{teal}{\mytextsc{verbe}} \hspace{4pt} Ton~: .

Couper du bois, fendre du bois.
\textcolor{brown}{\zh{劈、剖。}}




\paragraph{\hspace{-0.5cm} {\Large \textcolor{darkblue}{\textbf{\ipa{si˧-ʁæ˧bæ˥}}}}} \hypertarget{si\string_M-R\{\string_Mb\{\string_T1}{}
\markboth{\textcolor{darkblue}{\textbf{\ipa{si˧-ʁæ˧bæ˥}}}}{}
\textcolor{teal}{\mytextsc{nom}} \hspace{4pt} Ton~: H\#.

Assiette en bois.
\textcolor{brown}{\zh{木盘子。}}


 \mytextsc{clf}~: \textcolor{darkblue}{\textbf{\ipa{ɭɯ˧}}} 

\paragraph{\hspace{-0.5cm} {\Large \textcolor{darkblue}{\textbf{\ipa{si˧ʁo\#˥}}}}} \hypertarget{si\string_MRo\#\string_T1}{}
\markboth{\textcolor{darkblue}{\textbf{\ipa{si˧ʁo\#˥}}}}{}
\textcolor{teal}{\mytextsc{nom}} \hspace{4pt} Ton~: \#H.

Arbre fruitier.
\textcolor{brown}{\zh{果树。}}


 \mytextsc{clf}~: \textcolor{darkblue}{\textbf{\ipa{dzi˩}}} 

\paragraph{\hspace{-0.5cm} {\Large \textcolor{darkblue}{\textbf{\ipa{si˧ʁo˧si˧ɭɯ\#˥}}}}} \hypertarget{si\string_MRo\string_Msi\string_Ml\string_RM\#\string_T1}{}
\markboth{\textcolor{darkblue}{\textbf{\ipa{si˧ʁo˧si˧ɭɯ\#˥}}}}{}
\textcolor{teal}{\mytextsc{nom}} \hspace{4pt} Ton~: \#H.

Fruit.
\textcolor{brown}{\zh{水果。}}


\begin{exe}
\sn \textcolor{darkblue}{\textbf{\ipa{si˧ʁo˧si˧ɭɯ˧ ɲi˥}}}
\trans \mytextsc{cop}
\trans \textit{\textcolor{brown}{\zh{是水果。}}}
\end{exe}
 \mytextsc{clf}~: \textcolor{darkblue}{\textbf{\ipa{ɭɯ˧}}} 

\paragraph{\hspace{-0.5cm} {\Large \textcolor{darkblue}{\textbf{\ipa{si˧-sæ˥qʰv̩˩}}}}} \hypertarget{si\string_M-s\{\string_Tq\string_hv\string_=\string_B1}{}
\markboth{\textcolor{darkblue}{\textbf{\ipa{si˧-sæ˥qʰv̩˩}}}}{}
\textcolor{teal}{\mytextsc{nom}} \hspace{4pt} Ton~: \#H.

Bouleau, \textit{Betula szechuanica (Betula Pendula var. szechuanica)}.
\textcolor{brown}{\zh{四川桦树,白桦树。}}




\paragraph{\hspace{-0.5cm} {\Large \textcolor{darkblue}{\textbf{\ipa{si˧tʰv̩\#˥}}}}} \hypertarget{si\string_Mt\string_hv\string_=\#\string_T1}{}
\markboth{\textcolor{darkblue}{\textbf{\ipa{si˧tʰv̩\#˥}}}}{}
\textcolor{teal}{\mytextsc{nom}} \hspace{4pt} Ton~: \#H.

Meuble-autel des ancêtres, dans la pièce principale, qui constitue le lieu symbolique où résident les ancêtres; on y met des bougies au Nouvel An.
\textcolor{brown}{\zh{供桌:主屋里面的一个家具,是祖先的象征性住所。}}


\begin{exe}
\sn \textcolor{darkblue}{\textbf{\ipa{ʑi˧dv̩˧-nv̩˩mi˩, | si˧tʰv̩˧!}}}
\trans le cœur de la maison, c'est le meuble-autel des ancêtres!
\trans \textit{\textcolor{brown}{\zh{屋子的中心,就是祖先的供桌!}}}
\end{exe}


\paragraph{\hspace{-0.5cm} {\Large \textcolor{darkblue}{\textbf{\ipa{si˩qʰɑ˩}}}}} \hypertarget{si\string_Bq\string_hA\string_B1}{}
\markboth{\textcolor{darkblue}{\textbf{\ipa{si˩qʰɑ˩}}}}{}
\textcolor{teal}{\mytextsc{nom}} \hspace{4pt} Ton~: L.

Abricotier du Japon (essence bien représentée à Yongning).
\textcolor{brown}{\zh{梅子。}}


\begin{exe}
\sn \textcolor{darkblue}{\textbf{\ipa{si˩qʰɑ˩-dʑɯ˩}}}
\trans un liquide préparé à base de prunelles d'abricotier du Japon, servant d'équivalent de vinaigre (le vinaigre a été introduit tardivement; il était acheté en pays chinois)
\trans \textit{\textcolor{brown}{\zh{用梅子做的一种汁,用法类似于醋。过去,永宁没有醋,醋是从内地买来的。}}}
\end{exe}


\paragraph{\hspace{-0.5cm} {\Large \textcolor{darkblue}{\textbf{\ipa{si˩tsʰɤ˩}}}}} \hypertarget{si\string_Bts\string_h7\string_B1}{}
\markboth{\textcolor{darkblue}{\textbf{\ipa{si˩tsʰɤ˩}}}}{}
\textcolor{teal}{\mytextsc{nom}} \hspace{4pt} Ton~: L.
\ding{202} 
Feuille.
\textcolor{brown}{\zh{叶子。}}


\begin{exe}
\sn \textcolor{darkblue}{\textbf{\ipa{si˧dzi˩-si˩tsʰɤ˩}}}
\trans feuilles d'arbre
\trans \textit{\textcolor{brown}{\zh{树叶}}}
\end{exe}
\ding{203} 
Crête (du coq, d'un oiseau).
\textcolor{brown}{\zh{鸡冠。}}


\begin{exe}
\sn \textcolor{darkblue}{\textbf{\ipa{æ̃˧ʂwæ˥-si˩tsʰɤ˩}}}
\trans crête de coq
\trans \textit{\textcolor{brown}{\zh{公鸡冠}}}
\end{exe}
 \mytextsc{clf}~: \textcolor{darkblue}{\textbf{\ipa{tsʰɤ˧˥}}} 

\paragraph{\hspace{-0.5cm} {\Large \textcolor{darkblue}{\textbf{\ipa{si˧˥}}} \textsubscript{1}}} \hypertarget{si\string_M\string_T1}{}
\markboth{\textcolor{darkblue}{\textbf{\ipa{si˧˥}}} \textsubscript{1}}{}
\textcolor{teal}{\mytextsc{verbe}} \hspace{4pt} Ton~: MH.

Raser (la barbe); gratter (la terre collée à un champignon).
\textcolor{brown}{\zh{剔,刮。}}


\begin{exe}
\sn \textcolor{darkblue}{\textbf{\ipa{mo˧ si˥}}}
\trans gratter des champignons, pour en retirer la terre, les aiguilles de pin... Cela se fait souvent à sec.
\trans \textit{\textcolor{brown}{\zh{刮菌子(刮掉污垢)}}}
\end{exe}
\begin{exe}
\sn \textcolor{darkblue}{\textbf{\ipa{mv̩˧tsɯ˧ si˥}}}
\trans raser la barbe
\trans \textit{\textcolor{brown}{\zh{刮胡子}}}
\end{exe}
\begin{exe}
\sn \textcolor{darkblue}{\textbf{\ipa{ʁo˧qʰwɤ˩ si˩}}}
\trans raser le crâne, raser la tête
\trans \textit{\textcolor{brown}{\zh{剃头}}}
\end{exe}
\begin{exe}
\sn \textcolor{darkblue}{\textbf{\ipa{ʁo˧qʰwɤ˩ si˩-di˩}}}
\trans Rasoir, objet utilisé pour raser le crâne, mais aussi pour raser la barbe. Dans la jeunesse de F4, il existait quelques rasoirs; chaque famille n'en possédait pas. On faisait venir une personne sachant manier l'instrument. Ce sont les moines et les vieilles personnes qui faisaient le plus fréquemment appel à ces services.
\trans \textit{\textcolor{brown}{\zh{理发刮刀}}}
\end{exe}


\paragraph{\hspace{-0.5cm} {\Large \textcolor{darkblue}{\textbf{\ipa{si˧˥}}} \textsubscript{2}}} \hypertarget{si\string_M\string_T2}{}
\markboth{\textcolor{darkblue}{\textbf{\ipa{si˧˥}}} \textsubscript{2}}{}
\textcolor{teal}{\mytextsc{verbe}} \hspace{4pt} Ton~: MH.

Assassiner, tuer (un homme).
\textcolor{brown}{\zh{杀(人)。}}


\begin{exe}
\sn \textcolor{darkblue}{\textbf{\ipa{hĩ˧ si˩}}}
\trans tuer quelqu'un, assassiner quelqu'un
\trans \textit{\textcolor{brown}{\zh{杀人}}}
\end{exe}


\paragraph{\hspace{-0.5cm} {\Large \textcolor{darkblue}{\textbf{\ipa{si˩˥}}}}} \hypertarget{si\string_B\string_T1}{}
\markboth{\textcolor{darkblue}{\textbf{\ipa{si˩˥}}}}{}
\textcolor{teal}{\mytextsc{nom}} \hspace{4pt} Ton~: LH.

Foie.
\textcolor{brown}{\zh{肝。}}


 \mytextsc{clf}~: \textcolor{darkblue}{\textbf{\ipa{ɭɯ˧}}} 

\paragraph{\hspace{-0.5cm} {\Large \textcolor{darkblue}{\textbf{\ipa{so˥}}} \textsubscript{1}}} \hypertarget{so\string_T1}{}
\markboth{\textcolor{darkblue}{\textbf{\ipa{so˥}}} \textsubscript{1}}{}
\textcolor{teal}{\mytextsc{nom}} \hspace{4pt} Ton~: \#H.

Offrande aux esprits: repas qu'on leur offre le matin; on y met du thé, du beurre, de la farine, et du miel (et éventuellement des fleurs: \textcolor{darkblue}{\textbf{\ipa{/so˧dze˧-bæ˩bæ˩/}}}); on le fait brûler sur un feu d'épines de pin.
\textcolor{brown}{\zh{早上献给神的食物(含茶、酥油、面、蜂蜜),扔进松针火里烧。}}


\begin{exe}
\sn \textcolor{darkblue}{\textbf{\ipa{so˧ dze˧ tʰi˧-qæ˩}}}
\trans faire brûler du miel en offrande
\trans \textit{\textcolor{brown}{\zh{烧蜂蜜献给神}}}
\end{exe}
\begin{exe}
\sn \textcolor{darkblue}{\textbf{\ipa{[M23] so˧ qæ˩}}}
\trans brûler une offrande; traditionnellement, du pin gorgé de résine.
\trans \textit{\textcolor{brown}{\zh{烧献给神(食物,……)}}}
\end{exe}


\paragraph{\hspace{-0.5cm} {\Large \textcolor{darkblue}{\textbf{\ipa{so˥}}} \textsubscript{2}}} \hypertarget{so\string_T2}{}
\markboth{\textcolor{darkblue}{\textbf{\ipa{so˥}}} \textsubscript{2}}{}
\textcolor{teal}{\mytextsc{classificateur}} \hspace{4pt} Ton~: H*.

Classificateur des choses/objets, utilisé seulement en tournure négative: 'quoi que ce soit'.
\textcolor{brown}{\zh{量词:样东西,如:‘一样东西都没有’。}}


\begin{exe}
\sn \textcolor{darkblue}{\textbf{\ipa{ɖɯ˧-so˥ | mɤ˧-dʑo˧!}}}
\trans Il n’y a rien du tout [à manger]! (phrase polie qd on invite quelqu'un à manger: on prie le convive d’excuser la pauvreté des mets proposés)
\trans \textit{\textcolor{brown}{\zh{一样也没有! / 没什么东西!(请客时的礼貌、自我贬低说法:请客人原谅菜不够丰盛)}}}
\end{exe}
\textit{Voir~:} \textcolor{darkblue}{\textbf{\ipa{sɑ˥}}} 

\paragraph{\hspace{-0.5cm} {\Large \textcolor{darkblue}{\textbf{\ipa{so˧\textsubscript{a}}}}}} \hypertarget{so\string_Ma1}{}
\markboth{\textcolor{darkblue}{\textbf{\ipa{so˧\textsubscript{a}}}}}{}
\textcolor{teal}{\mytextsc{classificateur}} \hspace{4pt} Ton~: M\textsubscript{a}.

Classificateur des matinées. Il existe trois expressions pour compter les journées: on peut dire: un jour; une matinée; ou une nuit.
\textcolor{brown}{\zh{量词:早晨(一个)。}}


\begin{exe}
\sn \textcolor{darkblue}{\textbf{\ipa{mv̩˩si˧-njɤ˧˥ | ɖɯ˧-so˧, | njɤ˧le˧gv̩˧ | ɖɯ˧-ɲi˥, | mv̩˧kʰv̩˥ | ɖɯ˧-hɑ̃˧˥!}}}
\trans Une matinée; une journée; [ou] une nuit! (Expression didactique résumant les trois façons de compter les jours: on peut compter les matinées, les journées, ou les soirées.)
\trans \textit{\textcolor{brown}{\zh{一个早晨,一个白天,(或者说)一个晚上!(这句话,总结数日子的三个方式:‘一天’,可以说成‘一个早晨’、‘一个白天’、或‘一个晚上’。)}}}
\end{exe}
\begin{exe}
\sn \textcolor{darkblue}{\textbf{\ipa{tʰv̩˧-so˩}}}
\trans ce matin-là
\trans \textit{\textcolor{brown}{\zh{那天早上}}}
\end{exe}


\paragraph{\hspace{-0.5cm} {\Large \textcolor{darkblue}{\textbf{\ipa{so˧dʑɯ\#˥}}}}} \hypertarget{so\string_Mdz£M\#\string_T1}{}
\markboth{\textcolor{darkblue}{\textbf{\ipa{so˧dʑɯ\#˥}}}}{}
\textcolor{teal}{\mytextsc{nom}} \hspace{4pt} Ton~: \#H.

Piège.
\textcolor{brown}{\zh{陷阱。}}


\begin{exe}
\sn \textcolor{darkblue}{\textbf{\ipa{so˧dʑɯ˧ | ɖɯ˧-ɭɯ˧ | qwæ˧˥}}}
\trans creuser une fosse, un piège
\trans \textit{\textcolor{brown}{\zh{挖一个陷阱}}}
\end{exe}


\paragraph{\hspace{-0.5cm} {\Large \textcolor{darkblue}{\textbf{\ipa{so˧hɑ̃˩}}}}} \hypertarget{so\string_MhA\string_~\string_B1}{}
\markboth{\textcolor{darkblue}{\textbf{\ipa{so˧hɑ̃˩}}}}{}
\textcolor{teal}{\mytextsc{adverbe}} \hspace{4pt} Ton~: L\#.

Demain soir.
\textcolor{brown}{\zh{明晚。}}


\begin{exe}
\sn \textcolor{darkblue}{\textbf{\ipa{so˧hɑ̃˩ | -ɖɯ˩hɑ̃˩˥}}}
\trans demain soir
\trans \textit{\textcolor{brown}{\zh{明天晚上}}}
\end{exe}


\paragraph{\hspace{-0.5cm} {\Large \textcolor{darkblue}{\textbf{\ipa{so˧hwɤ˩}}}}} \hypertarget{so\string_Mhw7\string_B1}{}
\markboth{\textcolor{darkblue}{\textbf{\ipa{so˧hwɤ˩}}}}{}
\textcolor{teal}{\mytextsc{adverbe}} \hspace{4pt} Ton~: L\#.

Ensuite; par la suite; à partir de maintenant, désormais.
\textcolor{brown}{\zh{后来、以后,从此以后。}}




\paragraph{\hspace{-0.5cm} {\Large \textcolor{darkblue}{\textbf{\ipa{so˧ʝi˥\$}}}}} \hypertarget{so\string_Mj££i\string_T\$1}{}
\markboth{\textcolor{darkblue}{\textbf{\ipa{so˧ʝi˥\$}}}}{}
\textcolor{teal}{\mytextsc{adverbe}} \hspace{4pt} Ton~: H\$.

L'année prochaine, l'an prochain.
\textcolor{brown}{\zh{明年。}}




\paragraph{\hspace{-0.5cm} {\Large \textcolor{darkblue}{\textbf{\ipa{so˧lo˧}}}}} \hypertarget{so\string_Mlo\string_M1}{}
\markboth{\textcolor{darkblue}{\textbf{\ipa{so˧lo˧}}}}{}
\textcolor{teal}{\mytextsc{nom}} \hspace{4pt} Ton~: M.

Influence; exemple (dans l'éducation de quelqu'un).
\textcolor{brown}{\zh{影响,榜样。}}


\begin{exe}
\sn \textcolor{darkblue}{\textbf{\ipa{so˧lo˧ dzɑ˧! | mɤ˧-dʑɤ˩-hĩ˩ | ɖɯ˧-ʑi˩ ɲi˩!}}}
\trans Il/elle exerce une mauvaise influence / il/elle donne un mauvais exemple! (Sa famille,) c'est une mauvaise famille!
\trans \textit{\textcolor{brown}{\zh{他(对周围的人)有一个不好的影响!(他的家庭)是个不好的家庭!}}}
\end{exe}
\begin{exe}
\sn \textcolor{darkblue}{\textbf{\ipa{so˧lo˧ mɤ˧-dʑɤ˩!}}}
\trans Même sens que l'exemple précédent: Son exemple n'est pas bon / son influence n'est pas bonne.
\trans \textit{\textcolor{brown}{\zh{同上:(他对别人的)影响不好。}}}
\end{exe}
\begin{exe}
\sn \textcolor{darkblue}{\textbf{\ipa{so˧lo˧ dʑɤ˩}}}
\trans bonne influence; bon exemple; bonne éducation
\trans \textit{\textcolor{brown}{\zh{好榜样、好例子、好教育}}}
\end{exe}
 \mytextsc{clf}~: \textcolor{darkblue}{\textbf{\ipa{kʰwɤ˥}}} 

\paragraph{\hspace{-0.5cm} {\Large \textcolor{darkblue}{\textbf{\ipa{so˧ɬi˧mi˧}}}}} \hypertarget{so\string_MKi\string_Mmi\string_M1}{}
\markboth{\textcolor{darkblue}{\textbf{\ipa{so˧ɬi˧mi˧}}}}{}
\textcolor{teal}{\mytextsc{nom}} \hspace{4pt} Ton~: M.

3e mois.
\textcolor{brown}{\zh{三月。}}




\paragraph{\hspace{-0.5cm} {\Large \textcolor{darkblue}{\textbf{\ipa{so˧ɲi˥}}}}} \hypertarget{so\string_MJi\string_T1}{}
\markboth{\textcolor{darkblue}{\textbf{\ipa{so˧ɲi˥}}}}{}
\textcolor{teal}{\mytextsc{adverbe}} \hspace{4pt} Ton~: .

Demain, le lendemain.
\textcolor{brown}{\zh{明天、第二天。}}




\paragraph{\hspace{-0.5cm} {\Large \textcolor{darkblue}{\textbf{\ipa{so˧tsʰi˥}}}}} \hypertarget{so\string_Mts\string_hi\string_T1}{}
\markboth{\textcolor{darkblue}{\textbf{\ipa{so˧tsʰi˥}}}}{}
\textcolor{teal}{\mytextsc{verbe}} \hspace{4pt} Ton~: .

Respirer.
\textcolor{brown}{\zh{呼吸。}}


\begin{exe}
\sn \textcolor{darkblue}{\textbf{\ipa{so˧tsʰi˥ | ʐwæ˩˥}}}
\trans respirer très vite, haleter
\trans \textit{\textcolor{brown}{\zh{喘气}}}
\end{exe}
\begin{exe}
\sn \textcolor{darkblue}{\textbf{\ipa{mɤ˧-so˧tsʰi˥}}}
\trans \mytextsc{neg}
\trans \textit{\textcolor{brown}{\zh{不喘气}}}
\end{exe}


\paragraph{\hspace{-0.5cm} {\Large \textcolor{darkblue}{\textbf{\ipa{so˧tsʰi˧}}}}} \hypertarget{so\string_Mts\string_hi\string_M1}{}
\markboth{\textcolor{darkblue}{\textbf{\ipa{so˧tsʰi˧}}}}{}
\textcolor{teal}{\mytextsc{nombre}} \hspace{4pt} Ton~: .

30.
\textcolor{brown}{\zh{30。}}




\paragraph{\hspace{-0.5cm} {\Large \textcolor{darkblue}{\textbf{\ipa{so˧tsʰɯ˧ɲi˧}}}}} \hypertarget{so\string_Mts\string_hM\string_MJi\string_M1}{}
\markboth{\textcolor{darkblue}{\textbf{\ipa{so˧tsʰɯ˧ɲi˧}}}}{}
\textcolor{teal}{\mytextsc{nom}} \hspace{4pt} Ton~: .

Le 30e jour.
\textcolor{brown}{\zh{三十号。}}




\paragraph{\hspace{-0.5cm} {\Large \textcolor{darkblue}{\textbf{\ipa{so˩}}}}} \hypertarget{so\string_B1}{}
\markboth{\textcolor{darkblue}{\textbf{\ipa{so˩}}}}{}
\textcolor{teal}{\mytextsc{nombre}} \hspace{4pt} Ton~: L.

3.
\textcolor{brown}{\zh{3。}}




\paragraph{\hspace{-0.5cm} {\Large \textcolor{darkblue}{\textbf{\ipa{so˩\textsubscript{a}}}} \textsubscript{1}}} \hypertarget{so\string_Ba1}{}
\markboth{\textcolor{darkblue}{\textbf{\ipa{so˩\textsubscript{a}}}} \textsubscript{1}}{}
\textcolor{teal}{\mytextsc{adjectif}} \hspace{4pt} Ton~: L\textsubscript{a}.

Agréable, bon (goût, odeur).
\textcolor{brown}{\zh{香(吃得香,气味香)。}}




\paragraph{\hspace{-0.5cm} {\Large \textcolor{darkblue}{\textbf{\ipa{so˩\textsubscript{a}}}} \textsubscript{2}}} \hypertarget{so\string_Ba2}{}
\markboth{\textcolor{darkblue}{\textbf{\ipa{so˩\textsubscript{a}}}} \textsubscript{2}}{}
\textcolor{teal}{\mytextsc{verbe}} \hspace{4pt} Ton~: L\textsubscript{a}.
\ding{202} 
Étudier.
\textcolor{brown}{\zh{学习。}}


\begin{exe}
\sn \textcolor{darkblue}{\textbf{\ipa{tʰæ˧ɻæ˩ so˩}}}
\trans étudier (des livres)
\trans \textit{\textcolor{brown}{\zh{读书、学习}}}
\end{exe}
\begin{exe}
\sn \textcolor{darkblue}{\textbf{\ipa{so˩ mɤ˩-se˥!}}}
\trans c'est sans fin! / tu n'as jamais fini d'étudier! (au sujet du travail du linguiste, étudier une langue: à la différence des travaux manuels, ce n'est jamais fini, on n'en voit jamais le bout)
\trans \textit{\textcolor{brown}{\zh{学不完!(关于语言学家的工作:做不完,不像做手工可以有一个明确的终点。)}}}
\end{exe}
\begin{exe}
\sn \textcolor{darkblue}{\textbf{\ipa{ɖɯ˧-so˧\textasciitilde{}so˥-ɻ̍˩}}}
\trans étudier un peu
\trans \textit{\textcolor{brown}{\zh{学一学}}}
\end{exe}
\ding{203} 
Imiter.
\textcolor{brown}{\zh{学一个人、模仿一个人。}}


\begin{exe}
\sn \textcolor{darkblue}{\textbf{\ipa{tʰv̩˧ tʰɑ˧-so˧˥!}}}
\trans Ne t'avise pas de suivre son exemple!/Ne va pas faire comme lui!/Ne va pas prendre exemple sur lui!
\trans \textit{\textcolor{brown}{\zh{别学他! / 别做得像他一样!}}}
\end{exe}
\ding{204} 
Enseigner.
\textcolor{brown}{\zh{教。}}


\begin{exe}
\sn \textcolor{darkblue}{\textbf{\ipa{tʰæ˧ɻæ˩ so˩}}}
\trans enseigner
\trans \textit{\textcolor{brown}{\zh{教书}}}
\end{exe}
\begin{exe}
\sn \textcolor{darkblue}{\textbf{\ipa{njɤ˧-ɳɯ˧ | no˧ so˧-bi˧!}}}
\trans Je vais t'enseigner/t'apprendre!
\end{exe}


\paragraph{\hspace{-0.5cm} {\Large \textcolor{darkblue}{\textbf{\ipa{so˩\textasciitilde{}so˧˥}}}}} \hypertarget{so\string_B~so\string_M\string_T1}{}
\markboth{\textcolor{darkblue}{\textbf{\ipa{so˩\textasciitilde{}so˧˥}}}}{}
\textcolor{teal}{\mytextsc{verbe}} \hspace{4pt} Ton~: MH.

Frotter dans ses mains.
\textcolor{brown}{\zh{揉在手里。}}


\begin{exe}
\sn \textcolor{darkblue}{\textbf{\ipa{le˧-so˩\textasciitilde{}so˩}}}
\trans \mytextsc{accomp} \string_ \mytextsc{red}
\trans \textit{\textcolor{brown}{\zh{揉来揉去}}}
\end{exe}


\paragraph{\hspace{-0.5cm} {\Large \textcolor{darkblue}{\textbf{\ipa{so˧˥}}}}} \hypertarget{so\string_M\string_T1}{}
\markboth{\textcolor{darkblue}{\textbf{\ipa{so˧˥}}}}{}
\textcolor{teal}{\mytextsc{nom}} \hspace{4pt} Ton~: MH.
\ding{202} 
Souffle.
\textcolor{brown}{\zh{(一口)气。}}


\ding{203} 
Vapeur.
\textcolor{brown}{\zh{蒸汽。}}


\begin{exe}
\sn \textcolor{darkblue}{\textbf{\ipa{so˧ tʰv̩˥-ze˩}}}
\trans il y a de la vapeur qui sort, ça fait de la vapeur
\trans \textit{\textcolor{brown}{\zh{热气冒出来了。}}}
\end{exe}
 \mytextsc{clf}~: \textcolor{darkblue}{\textbf{\ipa{kʰɯ˩}}} objets longs

\paragraph{\hspace{-0.5cm} {\Large \textcolor{darkblue}{\textbf{\ipa{sɯ˥}}} \textsubscript{1}}} \hypertarget{sM\string_T1}{}
\markboth{\textcolor{darkblue}{\textbf{\ipa{sɯ˥}}} \textsubscript{1}}{}
\textcolor{teal}{\mytextsc{verbe}} \hspace{4pt} Ton~: H.

Aiguiser.
\textcolor{brown}{\zh{磨(刀)。}}


\begin{exe}
\sn \textcolor{darkblue}{\textbf{\ipa{ɖɯ˧-sɯ˧\textasciitilde{}sɯ˧-ɻ̍˥}}}
\trans aiguiser un peu
\trans \textit{\textcolor{brown}{\zh{磨一磨}}}
\end{exe}
\begin{exe}
\sn \textcolor{darkblue}{\textbf{\ipa{sɯ˩tʰi˩ sɯ˩˥}}}
\trans aiguiser un couteau
\trans \textit{\textcolor{brown}{\zh{磨刀}}}
\end{exe}


\paragraph{\hspace{-0.5cm} {\Large \textcolor{darkblue}{\textbf{\ipa{sɯ˥}}} \textsubscript{2}}} \hypertarget{sM\string_T2}{}
\markboth{\textcolor{darkblue}{\textbf{\ipa{sɯ˥}}} \textsubscript{2}}{}
\textcolor{teal}{\mytextsc{verbe}} \hspace{4pt} Ton~: H.

Savoir.
\textcolor{brown}{\zh{知道。}}


\begin{exe}
\sn \textcolor{darkblue}{\textbf{\ipa{mɤ˧-sɯ˥}}}
\trans \mytextsc{neg}
\trans \textit{\textcolor{brown}{\zh{不知道}}}
\end{exe}


\paragraph{\hspace{-0.5cm} {\Large \textcolor{darkblue}{\textbf{\ipa{‑sɯ˧}}}}} \hypertarget{‑sM\string_M1}{}
\markboth{\textcolor{darkblue}{\textbf{\ipa{‑sɯ˧}}}}{}
\textcolor{teal}{\mytextsc{suffixe}} \hspace{4pt} Ton~: M.

D'abord; encore (dans la tournure: pas encore).
\textcolor{brown}{\zh{首先、先。}}


\begin{exe}
\sn \textcolor{darkblue}{\textbf{\ipa{njɤ˧ ʈʂʰɯ˧-sɯ˩ | dzɯ˧-bi˧!}}}
\trans je vais d'abord manger celui-ci!
\trans \textit{\textcolor{brown}{\zh{我要先吃这个!}}}
\end{exe}
\begin{exe}
\sn \textcolor{darkblue}{\textbf{\ipa{njɤ˧ | ʈʂʰɯ˧-sɯ˩ | li˧-bi˧!}}}
\trans je vais d'abord lire celui-ci! (au sujet de deux livres)
\trans \textit{\textcolor{brown}{\zh{我要先读这本!}}}
\end{exe}
\begin{exe}
\sn \textcolor{darkblue}{\textbf{\ipa{ʈʂʰɯ˧ sɯ˩ | hwæ˧-bi˧!}}}
\trans (je) vais d'abord acheter celui-ci!
\trans \textit{\textcolor{brown}{\zh{先买这个吧!}}}
\end{exe}
\begin{exe}
\sn \textcolor{darkblue}{\textbf{\ipa{ʈʂʰɯ˧ sɯ˩ | tɕʰi˧-bi˧!}}}
\trans (je) vais d'abord vendre celui-ci!
\trans \textit{\textcolor{brown}{\zh{先卖这个吧!}}}
\end{exe}
\begin{exe}
\sn \textcolor{darkblue}{\textbf{\ipa{ʈʂʰɯ˧ sɯ˩ | dzɯ˧-bi˧!}}}
\trans (je) vais d'abord manger celui-ci!
\trans \textit{\textcolor{brown}{\zh{先吃这个吧!}}}
\end{exe}
\begin{exe}
\sn \textcolor{darkblue}{\textbf{\ipa{ʈʂʰɯ˧ sɯ˩ | ʑi˩-bi˩˥}}}
\trans (je) vais d'abord prendre celui-ci!
\trans \textit{\textcolor{brown}{\zh{先拿这个吧!}}}
\end{exe}
\begin{exe}
\sn \textcolor{darkblue}{\textbf{\ipa{ʈʂʰɯ˧ sɯ˩ | ʈʰɯ˩-bi˩˥}}}
\trans (je) vais d'abord boire celui-ci!
\trans \textit{\textcolor{brown}{\zh{先喝这个吧!}}}
\end{exe}
\begin{exe}
\sn \textcolor{darkblue}{\textbf{\ipa{ʈʂʰɯ˧ sɯ˩ | lɑ˧-bi˥}}}
\trans (je) vais d'abord battre celui-ci!
\trans \textit{\textcolor{brown}{\zh{先打这个吧!}}}
\end{exe}
\begin{exe}
\sn \textcolor{darkblue}{\textbf{\ipa{tv̩˧tv̩˥ sɯ˩ | tʰi˧-tsʰi˥!}}}
\trans Mets d'abord ton chapeau! (injonction à un petit enfant, avant une sortie)
\trans \textit{\textcolor{brown}{\zh{你先戴上帽子!(情景:出门前,让孩子戴上帽子)}}}
\end{exe}
\begin{exe}
\sn \textcolor{darkblue}{\textbf{\ipa{no˧ | le˧-sɯ˧ gv̩˧\textasciitilde{}gv̩˥!}}}
\trans Commence par travailler tout seul! (Consigne de la locutrice quand j'arrive pour ma leçon du matin. Elle est occupée; et elle sait que j'ai de quoi m'occuper seul en l'attendant: toiletter des textes déjà transcrits, etc. Elle me dit ``Commence par travailler tout seul! / Commence par les tâches que tu peux faire tout seul!"
\trans \textit{\textcolor{brown}{\zh{你先自己工作(一会)吧!(情景:调查者早上到合作人的家,但她忙着,而她知道调查者有不同类型的工作要做,其中有一些可以自己做,比如重新核对记录过的长篇语料。她说:“你先忙自己的一会吧!”)}}}
\end{exe}


\paragraph{\hspace{-0.5cm} {\Large \textcolor{darkblue}{\textbf{\ipa{sɯ˧\textsubscript{a}}}}}} \hypertarget{sM\string_Ma1}{}
\markboth{\textcolor{darkblue}{\textbf{\ipa{sɯ˧\textsubscript{a}}}}}{}
\textcolor{teal}{\mytextsc{verbe}} \hspace{4pt} Ton~: M\textsubscript{a}.
\ding{202} 
Enfiler (des perles).
\textcolor{brown}{\zh{串(珠)。}}


\begin{exe}
\sn \textcolor{darkblue}{\textbf{\ipa{sɯ˧ɻ̍˧ sɯ˧}}}
\trans enfiler des perles
\trans \textit{\textcolor{brown}{\zh{串珠}}}
\end{exe}
\begin{exe}
\sn \textcolor{darkblue}{\textbf{\ipa{le˧-sɯ˧-se˩-ze˩}}}
\trans (j'ai) fini d'enfiler
\trans \textit{\textcolor{brown}{\zh{串完了!}}}
\end{exe}
\begin{exe}
\sn \textcolor{darkblue}{\textbf{\ipa{tso˧\textasciitilde{}tso˧ sɯ˩}}}
\trans enfiler des choses
\trans \textit{\textcolor{brown}{\zh{串东西}}}
\end{exe}
\ding{203} 
Enfiler (une jupe).
\textcolor{brown}{\zh{穿(裙子)。}}


\begin{exe}
\sn \textcolor{darkblue}{\textbf{\ipa{ʈʰæ˧qʰwɤ˧ sɯ˧}}}
\trans enfiler une robe
\trans \textit{\textcolor{brown}{\zh{穿裙子}}}
\end{exe}


\paragraph{\hspace{-0.5cm} {\Large \textcolor{darkblue}{\textbf{\ipa{sɯ˧gv̩\#˥}}}}} \hypertarget{sM\string_Mgv\string_=\#\string_T1}{}
\markboth{\textcolor{darkblue}{\textbf{\ipa{sɯ˧gv̩\#˥}}}}{}
\textcolor{teal}{\mytextsc{nom}} \hspace{4pt} Ton~: \#H.

Caisse, coffre; par extension: armoire.
\textcolor{brown}{\zh{箱子,柜子。}}


 \mytextsc{clf}~: \textcolor{darkblue}{\textbf{\ipa{ɭɯ˧}}} 

\paragraph{\hspace{-0.5cm} {\Large \textcolor{darkblue}{\textbf{\ipa{sɯ˧kʰɯ˩}}}}} \hypertarget{sM\string_Mk\string_hM\string_B1}{}
\markboth{\textcolor{darkblue}{\textbf{\ipa{sɯ˧kʰɯ˩}}}}{}
\textcolor{teal}{\mytextsc{nom}} \hspace{4pt} Ton~: L\#.

Rituel lors du décès d'une femme qui a quitté la maison de sa mère pour se marier.
\textcolor{brown}{\zh{斯克:嫁到外边的女人去世时进行的仪式。}}




\paragraph{\hspace{-0.5cm} {\Large \textcolor{darkblue}{\textbf{\ipa{sɯ˧ljɤ˧}}}}} \hypertarget{sM\string_Mlj7\string_M1}{}
\markboth{\textcolor{darkblue}{\textbf{\ipa{sɯ˧ljɤ˧}}}}{}
\textcolor{teal}{\mytextsc{nom}} \hspace{4pt} Ton~: M.

Plastique.
\textcolor{brown}{\zh{塑料(汉语借词)。}}

 Emprunt~: chinois \textcolor{brown}{\zh{塑料}}



\paragraph{\hspace{-0.5cm} {\Large \textcolor{darkblue}{\textbf{\ipa{sɯ˧ljɤ˧tʰo˧˥}}}}} \hypertarget{sM\string_Mlj7\string_Mt\string_ho\string_M\string_T1}{}
\markboth{\textcolor{darkblue}{\textbf{\ipa{sɯ˧ljɤ˧tʰo˧˥}}}}{}
\textcolor{teal}{\mytextsc{nom}} \hspace{4pt} Ton~: MH\#.

Container pour liquides, en matière plastique.
\textcolor{brown}{\zh{塑料桶(汉语借词)。}}


 \mytextsc{clf}~: \textcolor{darkblue}{\textbf{\ipa{ɭɯ˧}}} 

\paragraph{\hspace{-0.5cm} {\Large \textcolor{darkblue}{\textbf{\ipa{sɯ˧mɤ˩}}}}} \hypertarget{sM\string_Mm7\string_B1}{}
\markboth{\textcolor{darkblue}{\textbf{\ipa{sɯ˧mɤ˩}}}}{}
\textcolor{teal}{\mytextsc{nom}} \hspace{4pt} Ton~: L\#.

Shiso, \textit{Perilla frutescens}, akajiso: plante alimentaire, aromatique, médicinale et ornementale de la famille des Lamiacées. Ses petites graines noires ressemblent au sésame. Elle a été introduite à Yongning récemment (années 1980?) On se sert des graines pour confectionner des confiseries; et on en extrait de l'huile, qui se mange.
\textcolor{brown}{\zh{紫苏。}}


\begin{exe}
\sn \textcolor{darkblue}{\textbf{\ipa{sɯ˧mɤ˩, | ɬi˧di˩ | tv̩˧-kv̩˧˥!}}}
\trans Le shiso, on en cultive à Yongning!
\trans \textit{\textcolor{brown}{\zh{在永宁,有紫苏!/ 有人种紫苏!}}}
\end{exe}
\begin{exe}
\sn \textcolor{darkblue}{\textbf{\ipa{sɯ˧mɤ˩-dze˩}}}
\trans friandise contenant des graines de shiso
\trans \textit{\textcolor{brown}{\zh{含紫苏的糖果}}}
\end{exe}
\begin{exe}
\sn \textcolor{darkblue}{\textbf{\ipa{sɯ˧mɤ˩-mæ˩ɻæ˩}}}
\trans huile de shiso
\trans \textit{\textcolor{brown}{\zh{紫苏油}}}
\end{exe}


\paragraph{\hspace{-0.5cm} {\Large \textcolor{darkblue}{\textbf{\ipa{sɯ˧pv̩˩}}}}} \hypertarget{sM\string_Mpv\string_=\string_B1}{}
\markboth{\textcolor{darkblue}{\textbf{\ipa{sɯ˧pv̩˩}}}}{}
\textcolor{teal}{\mytextsc{nom}} \hspace{4pt} Ton~: L\#.

Vessie.
\textcolor{brown}{\zh{膀胱。}}


 \mytextsc{clf}~: \textcolor{darkblue}{\textbf{\ipa{kɤ˧˥}}} 

\paragraph{\hspace{-0.5cm} {\Large \textcolor{darkblue}{\textbf{\ipa{sɯ˧pv̩˩-ni˩gv̩˩}}}}} \hypertarget{sM\string_Mpv\string_=\string_B-ni\string_Bgv\string_=\string_B1}{}
\markboth{\textcolor{darkblue}{\textbf{\ipa{sɯ˧pv̩˩-ni˩gv̩˩}}}}{}
\textcolor{teal}{\mytextsc{adjectif}} \hspace{4pt} Ton~: L\#-.

Gonflé, galbé, rond: littéralement ``comme une vessie". Une vessie de porc que l'on gonfle prend une forme toute ronde.
\textcolor{brown}{\zh{膀肿。}}




\paragraph{\hspace{-0.5cm} {\Large \textcolor{darkblue}{\textbf{\ipa{sɯ˧pv̩˧-sɯ˥nɑ˩}}}}} \hypertarget{sM\string_Mpv\string_=\string_M-sM\string_TnA\string_B1}{}
\markboth{\textcolor{darkblue}{\textbf{\ipa{sɯ˧pv̩˧-sɯ˥nɑ˩}}}}{}
\textcolor{teal}{\mytextsc{nom}} \hspace{4pt} Ton~: \#H-.

Chenille.
\textcolor{brown}{\zh{毛虫。}}


 \mytextsc{clf}~: \textcolor{darkblue}{\textbf{\ipa{mi˩}}} 

\paragraph{\hspace{-0.5cm} {\Large \textcolor{darkblue}{\textbf{\ipa{sɯ˧pʰi˧}}}}} \hypertarget{sM\string_Mp\string_hi\string_M1}{}
\markboth{\textcolor{darkblue}{\textbf{\ipa{sɯ˧pʰi˧}}}}{}
\textcolor{teal}{\mytextsc{nom}} \hspace{4pt} Ton~: M.

Noble, seigneur, chef: la plus haute des 3 castes de la société ancienne.
\textcolor{brown}{\zh{贵族,土司,奴隶主,官。音译:“司沛”。}}


\begin{exe}
\sn \textcolor{darkblue}{\textbf{\ipa{ɖæ˩mi˧-sɯ˩pʰi˩}}}
\trans les nobles du monastère
\trans \textit{\textcolor{brown}{\zh{大寺贵族}}}
\end{exe}
\begin{exe}
\sn \textcolor{darkblue}{\textbf{\ipa{sɯ˧pʰi˧ hĩ˩}}}
\trans les sujets du seigneur, les gens du seigneur
\trans \textit{\textcolor{brown}{\zh{贵族的臣子、贵族手下的人}}}
\end{exe}
 \mytextsc{clf}~: \textcolor{darkblue}{\textbf{\ipa{v̩˧}}} 

\paragraph{\hspace{-0.5cm} {\Large \textcolor{darkblue}{\textbf{\ipa{sɯ˧pʰi˧-zo˧}}}}} \hypertarget{sM\string_Mp\string_hi\string_M-zo\string_M1}{}
\markboth{\textcolor{darkblue}{\textbf{\ipa{sɯ˧pʰi˧-zo˧}}}}{}
\textcolor{teal}{\mytextsc{nom}} \hspace{4pt} Ton~: H\#.

Jeune homme de la noblesse, fils de mandarin.
\textcolor{brown}{\zh{少爷。}}


 \mytextsc{clf}~: \textcolor{darkblue}{\textbf{\ipa{v̩˧}}} 

\paragraph{\hspace{-0.5cm} {\Large \textcolor{darkblue}{\textbf{\ipa{sɯ˧ɻ̍˧}}}}} \hypertarget{sM\string_Mr£`̍\string_M1}{}
\markboth{\textcolor{darkblue}{\textbf{\ipa{sɯ˧ɻ̍˧}}}}{}
\textcolor{teal}{\mytextsc{nom}} \hspace{4pt} Ton~: M.

Perle.
\textcolor{brown}{\zh{珠,珠子,珍珠。}}


 \mytextsc{clf}~: \textcolor{darkblue}{\textbf{\ipa{ɭɯ˧}}} 

\paragraph{\hspace{-0.5cm} {\Large \textcolor{darkblue}{\textbf{\ipa{sɯ˧ɻæ˧}}}}} \hypertarget{sM\string_Mr£`\{\string_M1}{}
\markboth{\textcolor{darkblue}{\textbf{\ipa{sɯ˧ɻæ˧}}}}{}
\textcolor{teal}{\mytextsc{nom}} \hspace{4pt} Ton~: M.

Table (à Yongning, vers le milieu du XXe siècle, elles étaient basses et carrées).
\textcolor{brown}{\zh{桌子。}}


 \mytextsc{clf}~: \textcolor{darkblue}{\textbf{\ipa{pɤ˩}}} \textit{Voir~:} \hyperlink{t`s`o\string_MtsM\string_T1}{\textcolor{darkblue}{\textbf{\ipa{ʈʂo˧tsɯ˥}}}} 

\paragraph{\hspace{-0.5cm} {\Large \textcolor{darkblue}{\textbf{\ipa{sɯ˧ɻ̃\#˥}}}}} \hypertarget{sM\string_Mr£`\string_~\#\string_T1}{}
\markboth{\textcolor{darkblue}{\textbf{\ipa{sɯ˧ɻ̃\#˥}}}}{}
\textcolor{teal}{\mytextsc{nom}} \hspace{4pt} Ton~: \#H.

Tronc.
\textcolor{brown}{\zh{树干。}}


\begin{exe}
\sn \textcolor{darkblue}{\textbf{\ipa{si˧dzi˩ tʰv̩˩-dzi˩, | sɯ˧ɻ̃˧ dʑɤ˥!}}}
\trans Cet arbre, il a un beau tronc! (c'est-à-dire qu'il est utilisable en menuiserie)
\trans \textit{\textcolor{brown}{\zh{这是棵好树!(可以用做木料)}}}
\end{exe}
 \mytextsc{clf}~: \textcolor{darkblue}{\textbf{\ipa{lo˩}}} 

\paragraph{\hspace{-0.5cm} {\Large \textcolor{darkblue}{\textbf{\ipa{sɯ˧ɻ̃˧mi\#˥}}}}} \hypertarget{sM\string_Mr£`\string_~\string_Mmi\#\string_T1}{}
\markboth{\textcolor{darkblue}{\textbf{\ipa{sɯ˧ɻ̃˧mi\#˥}}}}{}
\textcolor{teal}{\mytextsc{nom}} \hspace{4pt} Ton~: \#H.

Colonne vertébrale.
\textcolor{brown}{\zh{脊椎骨。}}


 \mytextsc{clf}~: \textcolor{darkblue}{\textbf{\ipa{dzi˩}}} cl des arbres

\paragraph{\hspace{-0.5cm} {\Large \textcolor{darkblue}{\textbf{\ipa{sɯ˧sɯ˩}}}}} \hypertarget{sM\string_MsM\string_B1}{}
\markboth{\textcolor{darkblue}{\textbf{\ipa{sɯ˧sɯ˩}}}}{}
\textcolor{teal}{\mytextsc{adjectif}} \hspace{4pt} Ton~: L\#.

Cru.
\textcolor{brown}{\zh{生(不熟)。}}


\begin{exe}
\sn \textcolor{darkblue}{\textbf{\ipa{ʂe˧ sɯ˧\textasciitilde{}sɯ˥}}}
\trans viande crue
\trans \textit{\textcolor{brown}{\zh{生肉}}}
\end{exe}
\begin{exe}
\sn \textcolor{darkblue}{\textbf{\ipa{ʈʂe˧ sɯ˧\textasciitilde{}sɯ˥}}}
\trans 'terre crue': terre qui n'a pas été préparée pour l'agriculture par l'ajout de fumier, etc
\trans \textit{\textcolor{brown}{\zh{‘生土’:没有经过加工(加肥料等等)的土,还不适合种农作物}}}
\end{exe}


\paragraph{\hspace{-0.5cm} {\Large \textcolor{darkblue}{\textbf{\ipa{sɯ˧tsɯ˧}}}}} \hypertarget{sM\string_MtsM\string_M1}{}
\markboth{\textcolor{darkblue}{\textbf{\ipa{sɯ˧tsɯ˧}}}}{}
\textcolor{teal}{\mytextsc{nom}} \hspace{4pt} Ton~: M.

Lion.
\textcolor{brown}{\zh{狮子(汉语借词)。}}

 Emprunt~: chinois \textcolor{brown}{\zh{狮子}}

 \mytextsc{clf}~: \textcolor{darkblue}{\textbf{\ipa{mi˩}}} 

\paragraph{\hspace{-0.5cm} {\Large \textcolor{darkblue}{\textbf{\ipa{sɯ˧tsɯ˩}}}}} \hypertarget{sM\string_MtsM\string_B1}{}
\markboth{\textcolor{darkblue}{\textbf{\ipa{sɯ˧tsɯ˩}}}}{}
\textcolor{teal}{\mytextsc{nom}} \hspace{4pt} Ton~: L\#.

Camphre (arbre).
\textcolor{brown}{\zh{樟。}}


\begin{exe}
\sn \textcolor{darkblue}{\textbf{\ipa{sɯ˧tsɯ˩-dzi˩}}}
\trans arbre à camphre
\trans \textit{\textcolor{brown}{\zh{樟树}}}
\end{exe}


\paragraph{\hspace{-0.5cm} {\Large \textcolor{darkblue}{\textbf{\ipa{sɯ˧ʈv̩˥}}}}} \hypertarget{sM\string_Mt`v\string_=\string_T1}{}
\markboth{\textcolor{darkblue}{\textbf{\ipa{sɯ˧ʈv̩˥}}}}{}
\textcolor{teal}{\mytextsc{nom}} \hspace{4pt} Ton~: H\#.

Cor, durillon.
\textcolor{brown}{\zh{茧子。}}


\begin{exe}
\sn \textcolor{darkblue}{\textbf{\ipa{hĩ˧ ʈʂʰɯ˧-v̩˧-bv̩˧ | mv̩˧ɲi˧, | sɯ˧ʈv̩˥ ʁo˩!}}}
\trans le pouce de cette personne a un cor/un durillon!
\trans \textit{\textcolor{brown}{\zh{这个人的拇指有茧子!}}}
\end{exe}
\begin{exe}
\sn \textcolor{darkblue}{\textbf{\ipa{sɯ˧ʈv̩˥ | mɤ˧-ʁo˩-ze˩!}}}
\trans (il n'y a/je n'ai) plus de durillon!
\trans \textit{\textcolor{brown}{\zh{没有茧子了!}}}
\end{exe}
\begin{exe}
\sn \textcolor{darkblue}{\textbf{\ipa{sɯ˧ʈv̩˥ ʁo˩-ze˩!}}}
\trans Un durillon s'est formé!
\trans \textit{\textcolor{brown}{\zh{磨出了茧子!}}}
\end{exe}
 \mytextsc{clf}~: \textcolor{darkblue}{\textbf{\ipa{ɭɯ˧}}} 

\paragraph{\hspace{-0.5cm} {\Large \textcolor{darkblue}{\textbf{\ipa{sɯ˧zɯ\#˥}}}}} \hypertarget{sM\string_MzM\#\string_T1}{}
\markboth{\textcolor{darkblue}{\textbf{\ipa{sɯ˧zɯ\#˥}}}}{}
\textcolor{teal}{\mytextsc{nom}} \hspace{4pt} Ton~: \#H.

Communauté familiale; échelon inférieur à ``os".
\textcolor{brown}{\zh{家族、支系。}}


\begin{exe}
\sn \textcolor{darkblue}{\textbf{\ipa{sɯ˧zɯ˧ ɖɯ˧-lo˩}}}
\trans une communauté familiale
\trans \textit{\textcolor{brown}{\zh{一个支系,一条线}}}
\end{exe}
\begin{exe}
\sn \textcolor{darkblue}{\textbf{\ipa{sɯ˧zɯ˧ ɖɯ˧-ʁwɤ˧}}}
\trans une communauté familiale
\trans \textit{\textcolor{brown}{\zh{一个支系,一条线}}}
\end{exe}
\begin{exe}
\sn \textcolor{darkblue}{\textbf{\ipa{sɯ˧zɯ˧ ə˩-dʑo˩?}}}
\trans est-ce qu'il a une grande famille/est-ce que sa famille est nombreuse? =est-ce que la mariée sera bien entourée, intégrée dans une grande famille? (Question que l'on pose lors des discussions préliminaires aux mariages: on s'inquiète des qualités de la maisonnée que la jeune femme va rejoindre)
\trans \textit{\textcolor{brown}{\zh{家族齐全吗?/ 家族,人多吗?(谈婚姻前的题目之一:男方家族人多不多。以人多为好。)}}}
\end{exe}
 \mytextsc{clf}~: \textcolor{darkblue}{\textbf{\ipa{lo˩, ʁwɤ˧}}} 

\paragraph{\hspace{-0.5cm} {\Large \textcolor{darkblue}{\textbf{\ipa{sɯ˩\textsubscript{a}}}}}} \hypertarget{sM\string_Ba1}{}
\markboth{\textcolor{darkblue}{\textbf{\ipa{sɯ˩\textsubscript{a}}}}}{}
\textcolor{teal}{\mytextsc{verbe}} \hspace{4pt} Ton~: L\textsubscript{a}.

Vivre, être vivant.
\textcolor{brown}{\zh{活。}}


\begin{exe}
\sn \textcolor{darkblue}{\textbf{\ipa{ʈʂʰɯ˧ tʰi˧-sɯ˩-dʑo˩!}}}
\trans Elle/il est vivant(e)!
\trans \textit{\textcolor{brown}{\zh{他活着!}}}
\end{exe}
\begin{exe}
\sn \textcolor{darkblue}{\textbf{\ipa{ʈʂʰɯ˧ | mɤ˧-ʂɯ˧! | tʰi˧-sɯ˩-dʑo˩!}}}
\trans ce n'est pas mort! c'est encore vivant! (au sujet d'une plante/d'un animal qui paraissait mort(e))
\trans \textit{\textcolor{brown}{\zh{它没死,还活着!(一个植物、动物)}}}
\end{exe}


\paragraph{\hspace{-0.5cm} {\Large \textcolor{darkblue}{\textbf{\ipa{sɯ˩pv̩˩}}}}} \hypertarget{sM\string_Bpv\string_=\string_B1}{}
\markboth{\textcolor{darkblue}{\textbf{\ipa{sɯ˩pv̩˩}}}}{}
\textcolor{teal}{\mytextsc{nom}} \hspace{4pt} Ton~: .

Cloque (par exemple, après qu'on se soit ébouillanté).
\textcolor{brown}{\zh{水泡(例如:开水烫了手,会形成水泡)。}}


\begin{exe}
\sn \textcolor{darkblue}{\textbf{\ipa{sɯ˩pv̩˩ qʰwæ˥-ze˩!}}}
\trans une cloque s'est formée!
\trans \textit{\textcolor{brown}{\zh{起了水泡!}}}
\end{exe}
 \mytextsc{clf}~: \textcolor{darkblue}{\textbf{\ipa{ɭɯ˧}}} 

\paragraph{\hspace{-0.5cm} {\Large \textcolor{darkblue}{\textbf{\ipa{sɯ˩ɻ̍˩}}}}} \hypertarget{sM\string_Br£`̍\string_B1}{}
\markboth{\textcolor{darkblue}{\textbf{\ipa{sɯ˩ɻ̍˩}}}}{}
\textcolor{teal}{\mytextsc{nom}} \hspace{4pt} Ton~: L.

Pierre à aiguiser, ``fusil".
\textcolor{brown}{\zh{磨刀石。}}


 \mytextsc{clf}~: \textcolor{darkblue}{\textbf{\ipa{ɭɯ˧}}} 

\paragraph{\hspace{-0.5cm} {\Large \textcolor{darkblue}{\textbf{\ipa{sɯ˩tʰi˩}}}}} \hypertarget{sM\string_Bt\string_hi\string_B1}{}
\markboth{\textcolor{darkblue}{\textbf{\ipa{sɯ˩tʰi˩}}}}{}
\textcolor{teal}{\mytextsc{nom}} \hspace{4pt} Ton~: L.

Couteau.
\textcolor{brown}{\zh{刀。}}


 \mytextsc{clf}~: \textcolor{darkblue}{\textbf{\ipa{nɑ˧}}} 

\paragraph{\hspace{-0.5cm} {\Large \textcolor{darkblue}{\textbf{\ipa{sɯ˩tʰi˩-kʰɯ˥ʑi˩}}}}} \hypertarget{sM\string_Bt\string_hi\string_B-k\string_hM\string_Tz£i\string_B1}{}
\markboth{\textcolor{darkblue}{\textbf{\ipa{sɯ˩tʰi˩-kʰɯ˥ʑi˩}}}}{}
\textcolor{teal}{\mytextsc{nom}} \hspace{4pt} Ton~: L+\#H-.

Fourreau du couteau, gaine du couteau.
\textcolor{brown}{\zh{刀鞘。}}


 \mytextsc{clf}~: \textcolor{darkblue}{\textbf{\ipa{ɭɯ˧}}} 

\newpage
\section*{\centering- \textcolor{darkblue}{\textbf{\ipa{ʂ}}} -}
\paragraph{\hspace{-0.5cm} {\Large \textcolor{darkblue}{\textbf{\ipa{ʂæ˥-ljɤ˩}}}}} \hypertarget{s`\{\string_T-lj7\string_B1}{}
\markboth{\textcolor{darkblue}{\textbf{\ipa{ʂæ˥-ljɤ˩}}}}{}
\textcolor{teal}{\mytextsc{verbe}} \hspace{4pt} Ton~: H-.

Discuter.
\textcolor{brown}{\zh{商量(汉语借词。}}

 Emprunt~: chinois \textcolor{brown}{\zh{商量}}



\paragraph{\hspace{-0.5cm} {\Large \textcolor{darkblue}{\textbf{\ipa{ʂæ˧}}}}} \hypertarget{s`\{\string_M1}{}
\markboth{\textcolor{darkblue}{\textbf{\ipa{ʂæ˧}}}}{}
\textcolor{teal}{\mytextsc{adjectif}} \hspace{4pt} Ton~: M.
\ding{202} 
Long.
\textcolor{brown}{\zh{长。}}


\begin{exe}
\sn \textcolor{darkblue}{\textbf{\ipa{qʰɑ˧-ʂæ˧-gv̩˧}}}
\trans très long
\trans \textit{\textcolor{brown}{\zh{非常长}}}
\end{exe}
\begin{exe}
\sn \textcolor{darkblue}{\textbf{\ipa{le˧-ʈɤ˧-le˧-ʂæ˧ (+kʰɯ˧˥)}}}
\trans allonger, étirer
\trans \textit{\textcolor{brown}{\zh{拉长}}}
\end{exe}
\ding{203} 
Lointain, distant, éloigné.
\textcolor{brown}{\zh{远。}}




\paragraph{\hspace{-0.5cm} {\Large \textcolor{darkblue}{\textbf{\ipa{ʂæ˧ɖæ\#˥}}}}} \hypertarget{s`\{\string_Md`\{\#\string_T1}{}
\markboth{\textcolor{darkblue}{\textbf{\ipa{ʂæ˧ɖæ\#˥}}}}{}
\textcolor{teal}{\mytextsc{nom}} \hspace{4pt} Ton~: \#H.

Différence de longueur.
\textcolor{brown}{\zh{长度区别。}}


\begin{exe}
\sn \textcolor{darkblue}{\textbf{\ipa{ʂæ˧ɖæ˧ di˥, | mɤ˧-dʑɤ˩!}}}
\trans S'il y a des différences de longueur, c'est vilain/ça ne convient pas! (Contexte: explication au sujet du choix d'arbres à abattre pour obtenir du bois de charpente.)
\trans \textit{\textcolor{brown}{\zh{如果长短不一,不好!/不行!(情景:解释砍树时如何选择合适的树)}}}
\end{exe}
\begin{exe}
\sn \textcolor{darkblue}{\textbf{\ipa{ʂæ˧ɖæ˧ | mɤ˧-di˩!}}}
\trans il n'y a pas de différences de longueur (=c'est très bien)! (Même contexte que ci-dessus: choix d'arbres à abattre pour obtenir du bois de charpente)
\trans \textit{\textcolor{brown}{\zh{没有长度区别,都一样齐!(等于是好的建房木料)(情景:同上)}}}
\end{exe}
 \mytextsc{clf}~: \textcolor{darkblue}{\textbf{\ipa{kʰwɤ˥}}} 

\paragraph{\hspace{-0.5cm} {\Large \textcolor{darkblue}{\textbf{\ipa{ʂæ˧-lo˩pv˩}}}}} \hypertarget{s`\{\string_M-lo\string_Bpv\string_B1}{}
\markboth{\textcolor{darkblue}{\textbf{\ipa{ʂæ˧-lo˩pv˩}}}}{}
\textcolor{teal}{\mytextsc{nom}} \hspace{4pt} Ton~: -L.

Yyyy.
\textcolor{brown}{\zh{山萝卜。}}

 Emprunt~: chinois \textcolor{brown}{\zh{山萝卜}}

\textit{Voir~:} \hyperlink{hw7\string_Mli\string_M-hw\{\string_Mq\string_h\{\#\string_T1}{\textcolor{darkblue}{\textbf{\ipa{hwɤ˧li˧-hwæ˧qʰæ\#˥}}}} 

\paragraph{\hspace{-0.5cm} {\Large \textcolor{darkblue}{\textbf{\ipa{ʂæ˧pʰi˧}}}}} \hypertarget{s`\{\string_Mp\string_hi\string_M1}{}
\markboth{\textcolor{darkblue}{\textbf{\ipa{ʂæ˧pʰi˧}}}}{}
\textcolor{teal}{\mytextsc{nom}} \hspace{4pt} Ton~: M.

Marchandise, objet qui peut se vendre au marché.
\textcolor{brown}{\zh{商品。}}

 Emprunt~: chinois \textcolor{brown}{\zh{商品}}



\paragraph{\hspace{-0.5cm} {\Large \textcolor{darkblue}{\textbf{\ipa{ʂæ˧ʁwɤ˩}}}}} \hypertarget{s`\{\string_MRw7\string_B1}{}
\markboth{\textcolor{darkblue}{\textbf{\ipa{ʂæ˧ʁwɤ˩}}}}{}
\textcolor{teal}{\mytextsc{nom}} \hspace{4pt} Ton~: L\#.

Shuhe: nom d'un village de la plaine de Lijiang (anciennement Longquan). Les terres de ce village étaient médiocres, et beaucoup de ses habitants se tournaient vers le commerce et voyageaient dans toute la région, d'où le fait que le nom de ce village soit connu à Yongning.
\textcolor{brown}{\zh{束河(旧称:龙泉):丽江坝子里的一个村落。由于束河商人多,经常有束河人到永宁等地,使得相当多的永宁人熟悉那个村落名。}}

 Emprunt~: naxi \textcolor{darkblue}{\textbf{\ipa{/sɑ˥wɤ˧/}}}



\paragraph{\hspace{-0.5cm} {\Large \textcolor{darkblue}{\textbf{\ipa{ʂæ˧tsɯ˧}}}}} \hypertarget{s`\{\string_MtsM\string_M1}{}
\markboth{\textcolor{darkblue}{\textbf{\ipa{ʂæ˧tsɯ˧}}}}{}
\textcolor{teal}{\mytextsc{nom}} \hspace{4pt} Ton~: M.

Caftan: vêtement que portaient les enfants avant leurs treize ans: robe ample (la même pour les filles et les garçons); anciennement, les hommes aussi portaient ce type de vêtement.
\textcolor{brown}{\zh{裋、卡夫坦长衣:成年前男女小孩均穿的裋,成年男人也穿。}}


 \mytextsc{clf}~: \textcolor{darkblue}{\textbf{\ipa{ɭɯ˧˥}}} 

\paragraph{\hspace{-0.5cm} {\Large \textcolor{darkblue}{\textbf{\ipa{ʂæ˩ɻ̃˩}}}}} \hypertarget{s`\{\string_Br£`\string_~\string_B1}{}
\markboth{\textcolor{darkblue}{\textbf{\ipa{ʂæ˩ɻ̃˩}}}}{}
\textcolor{teal}{\mytextsc{nom}} \hspace{4pt} Ton~: L.

Os, ossement.
\textcolor{brown}{\zh{骨头。}}


 \mytextsc{clf}~: \textcolor{darkblue}{\textbf{\ipa{kɤ˧˥}}} 

\paragraph{\hspace{-0.5cm} {\Large \textcolor{darkblue}{\textbf{\ipa{ʂæ˧˥}}} \textsubscript{1}}} \hypertarget{s`\{\string_M\string_T1}{}
\markboth{\textcolor{darkblue}{\textbf{\ipa{ʂæ˧˥}}} \textsubscript{1}}{}
\textcolor{teal}{\mytextsc{verbe}} \hspace{4pt} Ton~: MH.

Tenir un chien en laisse, mener un chien; mener, guider, amener (les vaches aux pâturages, etc).
\textcolor{brown}{\zh{牵(牵着牛)。}}


\begin{exe}
\sn \textcolor{darkblue}{\textbf{\ipa{kʰv̩˧ ʂæ˧˥}}}
\trans mener un chien; chasser
\trans \textit{\textcolor{brown}{\zh{遛狗,狩猎}}}
\end{exe}
\begin{exe}
\sn \textcolor{darkblue}{\textbf{\ipa{kʰv̩˧ʂæ˧ hɯ˧˥}}}
\trans parti chasser, parti à la chasse
\trans \textit{\textcolor{brown}{\zh{狩猎去了}}}
\end{exe}


\paragraph{\hspace{-0.5cm} {\Large \textcolor{darkblue}{\textbf{\ipa{ʂæ˧˥}}} \textsubscript{2}}} \hypertarget{s`\{\string_M\string_T2}{}
\markboth{\textcolor{darkblue}{\textbf{\ipa{ʂæ˧˥}}} \textsubscript{2}}{}
\textcolor{teal}{\mytextsc{verbe}} \hspace{4pt} Ton~: MH.
\ding{202} 
Attacher, nouer en bottes.
\textcolor{brown}{\zh{捆成一包。}}


\begin{exe}
\sn \textcolor{darkblue}{\textbf{\ipa{le˧-ʂæ˧˥}}}
\trans \mytextsc{accomp}
\trans \textit{\textcolor{brown}{\zh{\mytextsc{accomp}}}}
\end{exe}
\begin{exe}
\sn \textcolor{darkblue}{\textbf{\ipa{hɑ˧ ʂæ˩}}}
\trans nouer le riz coupé en bottes
\trans \textit{\textcolor{brown}{\zh{刚收割的稻子,捆成捆}}}
\end{exe}
\ding{203} 
Envelopper, emballer (monosyllabique).
\textcolor{brown}{\zh{包。}}


\begin{exe}
\sn \textcolor{darkblue}{\textbf{\ipa{ʂæ˩\textasciitilde{}ʂæ˧˥}}}
\trans \mytextsc{red}: emballer, envelopper
\trans \textit{\textcolor{brown}{\zh{\mytextsc{重叠:包一包}}}}
\end{exe}
\begin{exe}
\sn \textcolor{darkblue}{\textbf{\ipa{ʂæ˩\textasciitilde{}ʂæ˧-ze˥}}}
\trans \mytextsc{red} \mytextsc{accomp}
\trans \textit{\textcolor{brown}{\zh{\mytextsc{red} \mytextsc{accomp}}}}
\end{exe}
\begin{exe}
\sn \textcolor{darkblue}{\textbf{\ipa{tso˧\textasciitilde{}tso˧ ʂæ˥\textasciitilde{}ʂæ˩}}}
\trans emballer des choses
\trans \textit{\textcolor{brown}{\zh{包一包东西}}}
\end{exe}


\paragraph{\hspace{-0.5cm} {\Large \textcolor{darkblue}{\textbf{\ipa{ʂæ˧˥\textsubscript{a}}}}}} \hypertarget{s`\{\string_M\string_Ta1}{}
\markboth{\textcolor{darkblue}{\textbf{\ipa{ʂæ˧˥\textsubscript{a}}}}}{}
\textcolor{teal}{\mytextsc{classificateur}} \hspace{4pt} Ton~: MH\textsubscript{a}.

Classificateur des gerbes: ce qu'on coupe en un coup de faucille et attache d'un brin.
\textcolor{brown}{\zh{量词:捆。}}


\begin{exe}
\sn \textcolor{darkblue}{\textbf{\ipa{zɯ˧ | ɖɯ˧-ʂæ˧˥}}}
\trans une gerbe d'herbe (nouée ensemble par un brin)
\trans \textit{\textcolor{brown}{\zh{一捆草}}}
\end{exe}
\begin{exe}
\sn \textcolor{darkblue}{\textbf{\ipa{ɕi˧ɭɯ˧ | ɖɯ˧-ʂæ˧˥}}}
\trans une gerbe de riz (nouée par un brin)
\trans \textit{\textcolor{brown}{\zh{一捆稻谷}}}
\end{exe}


\paragraph{\hspace{-0.5cm} {\Large \textcolor{darkblue}{\textbf{\ipa{ʂe˥}}} \textsubscript{1}}} \hypertarget{s`e\string_T1}{}
\markboth{\textcolor{darkblue}{\textbf{\ipa{ʂe˥}}} \textsubscript{1}}{}
\textcolor{teal}{\mytextsc{nom}} \hspace{4pt} Ton~: \#H.

Viande, chair.
\textcolor{brown}{\zh{肉,肌肉。}}




\paragraph{\hspace{-0.5cm} {\Large \textcolor{darkblue}{\textbf{\ipa{ʂe˥}}} \textsubscript{2}}} \hypertarget{s`e\string_T2}{}
\markboth{\textcolor{darkblue}{\textbf{\ipa{ʂe˥}}} \textsubscript{2}}{}
\textcolor{teal}{\mytextsc{nom}} \hspace{4pt} Ton~: \#H.

Céréales pas encore mûres: céréales en herbe, dont on voit déjà l'épi mais dont l'épi ne s'est pas encore incliné sous le poids du grain.
\textcolor{brown}{\zh{未熟粮食。}}


\begin{exe}
\sn \textcolor{darkblue}{\textbf{\ipa{ʂe˧do˧˥}}}
\trans même sens
\trans \textit{\textcolor{brown}{\zh{未熟粮食}}}
\end{exe}


\paragraph{\hspace{-0.5cm} {\Large \textcolor{darkblue}{\textbf{\ipa{ʂe˧\textsubscript{a}}}}}} \hypertarget{s`e\string_Ma1}{}
\markboth{\textcolor{darkblue}{\textbf{\ipa{ʂe˧\textsubscript{a}}}}}{}
\textcolor{teal}{\mytextsc{verbe}} \hspace{4pt} Ton~: M\textsubscript{a}.

Chercher; se procurer.
\textcolor{brown}{\zh{寻找。}}


\begin{exe}
\sn \textcolor{darkblue}{\textbf{\ipa{le˧-ʂe˧ le˧-ɖɯ˧-ze˧!}}}
\trans (j'ai) cherché et (j'ai) trouvé!
\trans \textit{\textcolor{brown}{\zh{(我)找了……就找到了! / 找到了!}}}
\end{exe}
\begin{exe}
\sn \textcolor{darkblue}{\textbf{\ipa{hĩ˧ ɖɯ˧-v̩˧ ʂe˧}}}
\trans littéralement 'chercher quelqu'un'; sens: fréquenter quelqu'un du sexe opposé, rendre visite à quelqu'un du sexe opposé (généralement: se dit d'un homme)
\trans \textit{\textcolor{brown}{\zh{直译:‘找一个人’。实际含义:去访问异性的人(一般是男人去访问女人)}}}
\end{exe}
\begin{exe}
\sn \textcolor{darkblue}{\textbf{\ipa{hĩ˧ ʂe˩}}}
\trans prendre femme, épouser une femme
\trans \textit{\textcolor{brown}{\zh{娶媳妇}}}
\end{exe}
\begin{exe}
\sn \textcolor{darkblue}{\textbf{\ipa{tso˧\textasciitilde{}tso˧ ʂe˩}}}
\trans chercher quelque chose
\trans \textit{\textcolor{brown}{\zh{找东西}}}
\end{exe}
\begin{exe}
\sn \textcolor{darkblue}{\textbf{\ipa{lo˧ mɤ˧-dʑo˧, | lo˧ ʂe˧!}}}
\trans [Il/elle] se crée des complications / se donner du travail!
\trans \textit{\textcolor{brown}{\zh{没事找事!}}}
\end{exe}
\begin{exe}
\sn \textcolor{darkblue}{\textbf{\ipa{le˧-ʂe˧ tʰi˧-tɕɯ˥}}}
\trans préparer (des ingrédients pour une recette, ses bagages...)
\trans \textit{\textcolor{brown}{\zh{准备(做饭的材料、旅途用品……)}}}
\end{exe}


\paragraph{\hspace{-0.5cm} {\Large \textcolor{darkblue}{\textbf{\ipa{ʂe˧bæ˧}}}}} \hypertarget{s`e\string_Mb\{\string_M1}{}
\markboth{\textcolor{darkblue}{\textbf{\ipa{ʂe˧bæ˧}}}}{}
\textcolor{teal}{\mytextsc{nom}} \hspace{4pt} Ton~: M.

Collier; chaîne.
\textcolor{brown}{\zh{项圈、项链,锁链。}}


\begin{exe}
\sn \textcolor{darkblue}{\textbf{\ipa{ŋv̩˩-ʂe˩bæ˥}}}
\trans collier en argent
\trans \textit{\textcolor{brown}{\zh{银项链}}}
\end{exe}
\begin{exe}
\sn \textcolor{darkblue}{\textbf{\ipa{hæ̃˩-ʂe˩bæ˥}}}
\trans collier en or
\trans \textit{\textcolor{brown}{\zh{金项链}}}
\end{exe}
\begin{exe}
\sn \textcolor{darkblue}{\textbf{\ipa{ʂe˧mo˧ʂe˧bæ˧, | kʰv̩˩mi˩ pʰæ˩˥!}}}
\trans Le collier de fer, c'est pour attacher le chien!
\trans \textit{\textcolor{brown}{\zh{铁链,是来用拴狗的!}}}
\end{exe}
\begin{exe}
\sn \textcolor{darkblue}{\textbf{\ipa{kʰi˧-ʂe˧bæ˥, | ʂe˧mo˧ po˧-ɳɯ˧ | gv̩˩˥!}}}
\trans \textit{\textcolor{brown}{\zh{铁链,是来用拴狗的!}}}
\end{exe}
 \mytextsc{clf}~: \textcolor{darkblue}{\textbf{\ipa{kʰɯ˩}}} 

\paragraph{\hspace{-0.5cm} {\Large \textcolor{darkblue}{\textbf{\ipa{ʂe˧bv̩\#˥}}}}} \hypertarget{s`e\string_Mbv\string_=\#\string_T1}{}
\markboth{\textcolor{darkblue}{\textbf{\ipa{ʂe˧bv̩\#˥}}}}{}
\textcolor{teal}{\mytextsc{nom}} \hspace{4pt} Ton~: \#H.

Saucisse; viande séchée conservée dans les intestins.
\textcolor{brown}{\zh{香肠,把瘦肉装在肠子里。}}




\paragraph{\hspace{-0.5cm} {\Large \textcolor{darkblue}{\textbf{\ipa{ʂe˧di˩}}}}} \hypertarget{s`e\string_Mdi\string_B1}{}
\markboth{\textcolor{darkblue}{\textbf{\ipa{ʂe˧di˩}}}}{}
\textcolor{teal}{\mytextsc{adjectif}} \hspace{4pt} Ton~: L\#.

Gros.
\textcolor{brown}{\zh{胖。}}


\begin{exe}
\sn \textcolor{darkblue}{\textbf{\ipa{ʂe˧ di˩-ze˩!}}}
\trans (il/elle) a grossi!
\trans \textit{\textcolor{brown}{\zh{胖了!}}}
\end{exe}


\paragraph{\hspace{-0.5cm} {\Large \textcolor{darkblue}{\textbf{\ipa{ʂe˧dzo\#˥}}}}} \hypertarget{s`e\string_Mdzo\#\string_T1}{}
\markboth{\textcolor{darkblue}{\textbf{\ipa{ʂe˧dzo\#˥}}}}{}
\textcolor{teal}{\mytextsc{nom}} \hspace{4pt} Ton~: \#H.

Meuble de cuisine: structure en bois sur laquelle on fait la cuisine: on y pose la planche à découper, les ustensiles….
\textcolor{brown}{\zh{放案板的家具。}}


 \mytextsc{clf}~: \textcolor{darkblue}{\textbf{\ipa{pɤ˩}}} 

\paragraph{\hspace{-0.5cm} {\Large \textcolor{darkblue}{\textbf{\ipa{ʂe˧kʰɯ˧}}}}} \hypertarget{s`e\string_Mk\string_hM\string_M1}{}
\markboth{\textcolor{darkblue}{\textbf{\ipa{ʂe˧kʰɯ˧}}}}{}
\textcolor{teal}{\mytextsc{nom}} \hspace{4pt} Ton~: M.

Trépied de fer (dans le foyer, sur lequel on pose une casserole, une poële, une bouilloire…).
\textcolor{brown}{\zh{三脚架。}}


 \mytextsc{clf}~: \textcolor{darkblue}{\textbf{\ipa{nɑ˧}}} 

\paragraph{\hspace{-0.5cm} {\Large \textcolor{darkblue}{\textbf{\ipa{ʂe˧mi˧}}}}} \hypertarget{s`e\string_Mmi\string_M1}{}
\markboth{\textcolor{darkblue}{\textbf{\ipa{ʂe˧mi˧}}}}{}
\textcolor{teal}{\mytextsc{nom}} \hspace{4pt} Ton~: M.

Pou.
\textcolor{brown}{\zh{虱子。}}


 \mytextsc{clf}~: \textcolor{darkblue}{\textbf{\ipa{mi˩}}} 

\paragraph{\hspace{-0.5cm} {\Large \textcolor{darkblue}{\textbf{\ipa{ʂe˧mo˧}}}}} \hypertarget{s`e\string_Mmo\string_M1}{}
\markboth{\textcolor{darkblue}{\textbf{\ipa{ʂe˧mo˧}}}}{}
\textcolor{teal}{\mytextsc{nom}} \hspace{4pt} Ton~: M.

Fer (disyllabe).
\textcolor{brown}{\zh{铁(双音节)。}}


\textit{Voir~:} \hyperlink{s`e\string_B1}{\textcolor{darkblue}{\textbf{\ipa{ʂe˩}}}} 

\paragraph{\hspace{-0.5cm} {\Large \textcolor{darkblue}{\textbf{\ipa{ʂe˧nɑ˩}}}}} \hypertarget{s`e\string_MnA\string_B1}{}
\markboth{\textcolor{darkblue}{\textbf{\ipa{ʂe˧nɑ˩}}}}{}
\textcolor{teal}{\mytextsc{nom}} \hspace{4pt} Ton~: L\#.

Viande maigre.
\textcolor{brown}{\zh{瘦肉。}}




\paragraph{\hspace{-0.5cm} {\Large \textcolor{darkblue}{\textbf{\ipa{ʂe˧ɲi˩}}}}} \hypertarget{s`e\string_MJi\string_B1}{}
\markboth{\textcolor{darkblue}{\textbf{\ipa{ʂe˧ɲi˩}}}}{}
\textcolor{teal}{\mytextsc{nom}} \hspace{4pt} Ton~: L\#.

Pince à braises.
\textcolor{brown}{\zh{火钳。}}


 \mytextsc{clf}~: \textcolor{darkblue}{\textbf{\ipa{nɑ˧}}} 

\paragraph{\hspace{-0.5cm} {\Large \textcolor{darkblue}{\textbf{\ipa{ʂe˧pv̩˩}}}}} \hypertarget{s`e\string_Mpv\string_=\string_B1}{}
\markboth{\textcolor{darkblue}{\textbf{\ipa{ʂe˧pv̩˩}}}}{}
\textcolor{teal}{\mytextsc{nom}} \hspace{4pt} Ton~: L\#.

Viande séchée.
\textcolor{brown}{\zh{腊肉。}}




\paragraph{\hspace{-0.5cm} {\Large \textcolor{darkblue}{\textbf{\ipa{ʂe˧qʰv̩˧}}}}} \hypertarget{s`e\string_Mq\string_hv\string_=\string_M1}{}
\markboth{\textcolor{darkblue}{\textbf{\ipa{ʂe˧qʰv̩˧}}}}{}
\textcolor{teal}{\mytextsc{nom}} \hspace{4pt} Ton~: M.

Clou en fer.
\textcolor{brown}{\zh{铁钉,钉子。}}


\begin{exe}
\sn \textcolor{darkblue}{\textbf{\ipa{ʂe˧qʰv̩˧ lɑ˧˥}}}
\trans enfoncer un clou, planter un clou
\trans \textit{\textcolor{brown}{\zh{钉钉子}}}
\end{exe}
 \mytextsc{clf}~: \textcolor{darkblue}{\textbf{\ipa{ɭɯ˧}}} 

\paragraph{\hspace{-0.5cm} {\Large \textcolor{darkblue}{\textbf{\ipa{ʂe˧sɑ˩}}}}} \hypertarget{s`e\string_MsA\string_B1}{}
\markboth{\textcolor{darkblue}{\textbf{\ipa{ʂe˧sɑ˩}}}}{}
\textcolor{teal}{\mytextsc{nom}} \hspace{4pt} Ton~: L\#.

Viande des membres du cochon: les membres postérieurs aussi bien que les membres antérieurs. Le terme s'emploie pour la pièce de boucherie: de la viande conservée (séchée) avec l'os; mais le même terme peut également s'employer pour désigner les membres de la bête vivante.
\textcolor{brown}{\zh{猪腿肉。}}


 \mytextsc{clf}~: \textcolor{darkblue}{\textbf{\ipa{sɑ˧˥}}} 

\paragraph{\hspace{-0.5cm} {\Large \textcolor{darkblue}{\textbf{\ipa{ʂe˧-sɯ˧sɯ˥}}}}} \hypertarget{s`e\string_M-sM\string_MsM\string_T1}{}
\markboth{\textcolor{darkblue}{\textbf{\ipa{ʂe˧-sɯ˧sɯ˥}}}}{}
\textcolor{teal}{\mytextsc{nom}} \hspace{4pt} Ton~: H\#.

Viande crue.
\textcolor{brown}{\zh{生肉。}}




\paragraph{\hspace{-0.5cm} {\Large \textcolor{darkblue}{\textbf{\ipa{ʂe˧ʂe˧}}}}} \hypertarget{s`e\string_Ms`e\string_M1}{}
\markboth{\textcolor{darkblue}{\textbf{\ipa{ʂe˧ʂe˧}}}}{}
\textcolor{teal}{\mytextsc{verbe}} \hspace{4pt} Ton~: M.

Prendre froid, attraper un rhume, attraper froid.
\textcolor{brown}{\zh{着凉。}}


\begin{exe}
\sn \textcolor{darkblue}{\textbf{\ipa{ʂe˧ʂe˧-ze˩}}}
\trans \mytextsc{pfv}
\trans \textit{\textcolor{brown}{\zh{着凉了}}}
\end{exe}


\paragraph{\hspace{-0.5cm} {\Large \textcolor{darkblue}{\textbf{\ipa{ʂe˧ʈʂe˩}}}}} \hypertarget{s`e\string_Mt`s`e\string_B1}{}
\markboth{\textcolor{darkblue}{\textbf{\ipa{ʂe˧ʈʂe˩}}}}{}
\textcolor{teal}{\mytextsc{nom}} \hspace{4pt} Ton~: L\#.

Tissu de coton.
\textcolor{brown}{\zh{棉布,布料。}}


 \mytextsc{clf}~: \textcolor{darkblue}{\textbf{\ipa{pʰæ˧˥}}} \textcolor{darkblue}{\textbf{\ipa{kʰɤ˥}}} 

\paragraph{\hspace{-0.5cm} {\Large \textcolor{darkblue}{\textbf{\ipa{ʂe˧ʐe\#˥}}}}} \hypertarget{s`e\string_Mz`e\#\string_T1}{}
\markboth{\textcolor{darkblue}{\textbf{\ipa{ʂe˧ʐe\#˥}}}}{}
\textcolor{teal}{\mytextsc{nom}} \hspace{4pt} Ton~: \#H.

Viande de cochon préservée. Le terme recouvre diverses pièces de boucherie, dont le jambon.
\textcolor{brown}{\zh{腊肉,包括不同几类的腊肉,如火腿等。}}


 \mytextsc{clf}~: \textcolor{darkblue}{\textbf{\ipa{ʐe˥}}} 

\paragraph{\hspace{-0.5cm} {\Large \textcolor{darkblue}{\textbf{\ipa{ʂe˩}}}}} \hypertarget{s`e\string_B1}{}
\markboth{\textcolor{darkblue}{\textbf{\ipa{ʂe˩}}}}{}
\textcolor{teal}{\mytextsc{nom}} \hspace{4pt} Ton~: L.

Fer (monosyllabe).
\textcolor{brown}{\zh{铁(单音节)。}}




\paragraph{\hspace{-0.5cm} {\Large \textcolor{darkblue}{\textbf{\ipa{ʂe˩\textsubscript{b}}}}}} \hypertarget{s`e\string_Bb1}{}
\markboth{\textcolor{darkblue}{\textbf{\ipa{ʂe˩\textsubscript{b}}}}}{}
\textcolor{teal}{\mytextsc{verbe}} \hspace{4pt} Ton~: L\textsubscript{b}.

Uriner, pisser, faire pipi; déféquer.
\textcolor{brown}{\zh{小便,尿; 屙尿; 解溲; 拉(屎)。}}


\begin{exe}
\sn \textcolor{darkblue}{\textbf{\ipa{dʑi˧ ʂe˧˥}}}
\trans pisser
\trans \textit{\textcolor{brown}{\zh{屙尿}}}
\end{exe}
\begin{exe}
\sn \textcolor{darkblue}{\textbf{\ipa{qʰæ˧ ʂe˧˥}}}
\trans déféquer
\trans \textit{\textcolor{brown}{\zh{拉屎}}}
\end{exe}
\begin{exe}
\sn \textcolor{darkblue}{\textbf{\ipa{le˧-ʂe˩-ze˩}}}
\trans \mytextsc{accomp} \string_ \mytextsc{pfv}
\trans \textit{\textcolor{brown}{\zh{尿了}}}
\end{exe}
\begin{exe}
\sn \textcolor{darkblue}{\textbf{\ipa{ɖɯ˧-ʈʰɤ˧ ʂe˧˥}}}
\trans pisser une goutte
\trans \textit{\textcolor{brown}{\zh{尿一滴尿}}}
\end{exe}


\paragraph{\hspace{-0.5cm} {\Large \textcolor{darkblue}{\textbf{\ipa{ʂe˩lɑ˩}}}}} \hypertarget{s`e\string_BlA\string_B1}{}
\markboth{\textcolor{darkblue}{\textbf{\ipa{ʂe˩lɑ˩}}}}{}
\textcolor{teal}{\mytextsc{nom}} \hspace{4pt} Ton~: L.

Forger, battre le fer.
\textcolor{brown}{\zh{打铁。}}




\paragraph{\hspace{-0.5cm} {\Large \textcolor{darkblue}{\textbf{\ipa{ʂe˩-lɑ˩-hĩ˥}}}}} \hypertarget{s`e\string_B-lA\string_B-hi\string_~\string_T1}{}
\markboth{\textcolor{darkblue}{\textbf{\ipa{ʂe˩-lɑ˩-hĩ˥}}}}{}
\textcolor{teal}{\mytextsc{nom}} \hspace{4pt} Ton~: L+H\#.

Forgeron.
\textcolor{brown}{\zh{铁匠。}}


\begin{exe}
\sn \textcolor{darkblue}{\textbf{\ipa{ʂe˩lɑ˩-hĩ˥ hĩ˩}}}
\trans forgeron
\trans \textit{\textcolor{brown}{\zh{铁匠}}}
\end{exe}
 \mytextsc{clf}~: \textcolor{darkblue}{\textbf{\ipa{v̩˧}}} 

\paragraph{\hspace{-0.5cm} {\Large \textcolor{darkblue}{\textbf{\ipa{ʂe˩mɤ˩}}}}} \hypertarget{s`e\string_Bm7\string_B1}{}
\markboth{\textcolor{darkblue}{\textbf{\ipa{ʂe˩mɤ˩}}}}{}
\textcolor{teal}{\mytextsc{nom}} \hspace{4pt} Ton~: L.

Viande grasse.
\textcolor{brown}{\zh{肥肉。}}




\paragraph{\hspace{-0.5cm} {\Large \textcolor{darkblue}{\textbf{\ipa{ʂe˩-mo˧˥}}}}} \hypertarget{s`e\string_B-mo\string_M\string_T1}{}
\markboth{\textcolor{darkblue}{\textbf{\ipa{ʂe˩-mo˧˥}}}}{}
\textcolor{teal}{\mytextsc{nom}} \hspace{4pt} Ton~: LM+MH\#.

Champignon des pins, matsutake, \textit{Tricholoma matsutake}: un champignon comestible, rare et très apprécié.
\textcolor{brown}{\zh{松茸。}}




\paragraph{\hspace{-0.5cm} {\Large \textcolor{darkblue}{\textbf{\ipa{ʂe˩ʂv̩˩}}}}} \hypertarget{s`e\string_Bs`v\string_=\string_B1}{}
\markboth{\textcolor{darkblue}{\textbf{\ipa{ʂe˩ʂv̩˩}}}}{}
\textcolor{teal}{\mytextsc{nom}} \hspace{4pt} Ton~: L.

Lente, oeuf de pou.
\textcolor{brown}{\zh{虮子。}}


 \mytextsc{clf}~: \textcolor{darkblue}{\textbf{\ipa{ɭɯ˧}}} 

\paragraph{\hspace{-0.5cm} {\Large \textcolor{darkblue}{\textbf{\ipa{ʂɤ˧do˧˥}}}}} \hypertarget{s`7\string_Mdo\string_M\string_T1}{}
\markboth{\textcolor{darkblue}{\textbf{\ipa{ʂɤ˧do˧˥}}}}{}
\textcolor{teal}{\mytextsc{adjectif}} \hspace{4pt} Ton~: MH\#.
\ding{202} 
Honteux.
\textcolor{brown}{\zh{害羞。}}


\begin{exe}
\sn \textcolor{darkblue}{\textbf{\ipa{ʂɤ˧do˧ mɤ˧-sɯ˥!}}}
\trans (il/elle) est effronté(e), ne connaît pas la politesse, est malpoli
\trans \textit{\textcolor{brown}{\zh{不知羞耻!}}}
\end{exe}
\ding{203} 
Pudique; poli.
\textcolor{brown}{\zh{娴静、礼貌。}}


\begin{exe}
\sn \textcolor{darkblue}{\textbf{\ipa{ʈʂʰɯ˧ ʂɤ˧do˧-zo˥! / ʂɤ˧do˧ ʝi˥!}}}
\trans Cette personne a de la pudeur!
\trans \textit{\textcolor{brown}{\zh{他/她很娴静 / 很持重!}}}
\end{exe}


\paragraph{\hspace{-0.5cm} {\Large \textcolor{darkblue}{\textbf{\ipa{ʂɤ˧ɲi\#˥}}}}} \hypertarget{s`7\string_MJi\#\string_T1}{}
\markboth{\textcolor{darkblue}{\textbf{\ipa{ʂɤ˧ɲi\#˥}}}}{}
\textcolor{teal}{\mytextsc{nom}} \hspace{4pt} Ton~: \#H.

Conseil, avis.
\textcolor{brown}{\zh{建议、意见。}}


\begin{exe}
\sn \textcolor{darkblue}{\textbf{\ipa{ʂɤ˧ɲi˧ ʑi˧˥}}}
\trans demander un conseil / prendre le conseil (de quelqu'un)
\trans \textit{\textcolor{brown}{\zh{请求意见,求教}}}
\end{exe}
\begin{exe}
\sn \textcolor{darkblue}{\textbf{\ipa{no˧ | hĩ˧-ki˧ | ʂɤ˧ɲi˧ mɤ˧-ʑi˧-zo˥!}}}
\trans Tu n'as pas à prendre son conseil! / Tu n'as pas à prendre le conseil d'autrui [à ce sujet: à toi de décider]!
\trans \textit{\textcolor{brown}{\zh{你不要问人家的意见!}}}
\end{exe}
\begin{exe}
\sn \textcolor{darkblue}{\textbf{\ipa{ə˧tse˧ʝi˧-zo˥ | ʂɤ˧ɲi˧ ʑi˧-tso˧-ɲi˥?}}}
\trans Pourquoi donc lui demandes-tu conseil?
\trans \textit{\textcolor{brown}{\zh{你为什么要问(他的)意见!}}}
\end{exe}
 \mytextsc{clf}~: \textcolor{darkblue}{\textbf{\ipa{kʰwɤ˥}}} 

\paragraph{\hspace{-0.5cm} {\Large \textcolor{darkblue}{\textbf{\ipa{ʂɤ˩\textsubscript{a}}}}}} \hypertarget{s`7\string_Ba1}{}
\markboth{\textcolor{darkblue}{\textbf{\ipa{ʂɤ˩\textsubscript{a}}}}}{}
\textcolor{teal}{\mytextsc{verbe}} \hspace{4pt} Ton~: L\textsubscript{a}.

Déchirer (ex.: du papier).
\textcolor{brown}{\zh{撕(纸……)。}}


\begin{exe}
\sn \textcolor{darkblue}{\textbf{\ipa{tso˧\textasciitilde{}tso˧ ʂɤ˥}}}
\trans déchirer des choses
\trans \textit{\textcolor{brown}{\zh{撕东西}}}
\end{exe}
\begin{exe}
\sn \textcolor{darkblue}{\textbf{\ipa{tso˧\textasciitilde{}tso˧ ʂɤ˧\textasciitilde{}ʂɤ˥ (+ze˩)}}}
\trans déchirer des choses
\trans \textit{\textcolor{brown}{\zh{撕东西}}}
\end{exe}
\begin{exe}
\sn \textcolor{darkblue}{\textbf{\ipa{le˧-ʂɤ˧\textasciitilde{}ʂɤ˥+ze˩}}}
\trans \mytextsc{accomp} \string_ \mytextsc{red} \mytextsc{pfv}
\trans \textit{\textcolor{brown}{\zh{撕了}}}
\end{exe}


\paragraph{\hspace{-0.5cm} {\Large \textcolor{darkblue}{\textbf{\ipa{ʂɤ˩ŋɤ\#˥}}}}} \hypertarget{s`7\string_BN7\#\string_T1}{}
\markboth{\textcolor{darkblue}{\textbf{\ipa{ʂɤ˩ŋɤ\#˥}}}}{}
\textcolor{teal}{\mytextsc{nom}} \hspace{4pt} Ton~: LM+\#H.

Gong.
\textcolor{brown}{\zh{锣。}}


\begin{exe}
\sn \textcolor{darkblue}{\textbf{\ipa{ʂɤ˩ŋɤ˧ lɑ˩}}}
\trans jouer du gong
\trans \textit{\textcolor{brown}{\zh{打锣}}}
\end{exe}
 \mytextsc{clf}~: \textcolor{darkblue}{\textbf{\ipa{ɭɯ˧}}} 

\paragraph{\hspace{-0.5cm} {\Large \textcolor{darkblue}{\textbf{\ipa{ʂo˥}}}}} \hypertarget{s`o\string_T1}{}
\markboth{\textcolor{darkblue}{\textbf{\ipa{ʂo˥}}}}{}
\textcolor{teal}{\mytextsc{verbe}} \hspace{4pt} Ton~: H.

Récolter.
\textcolor{brown}{\zh{收割。}}


\begin{exe}
\sn \textcolor{darkblue}{\textbf{\ipa{le˧-ʂo˥-ze˩}}}
\trans \mytextsc{accomp} \string_ \mytextsc{pfv}
\trans \textit{\textcolor{brown}{\zh{收割了}}}
\end{exe}
\begin{exe}
\sn \textcolor{darkblue}{\textbf{\ipa{ɖɯ˧-kʰv̩˥ ɖɯ˧-ʂɯ˩ | gɤ˩-ʂo˥-ze˩!}}}
\trans on récolte (le riz) une fois par an!
\trans \textit{\textcolor{brown}{\zh{每年收一次稻谷!}}}
\end{exe}


\paragraph{\hspace{-0.5cm} {\Large \textcolor{darkblue}{\textbf{\ipa{ʂo˧}}}}} \hypertarget{s`o\string_M1}{}
\markboth{\textcolor{darkblue}{\textbf{\ipa{ʂo˧}}}}{}
\textcolor{teal}{\mytextsc{verbe}} \hspace{4pt} Ton~: M.

Rassembler, assembler, accumuler.
\textcolor{brown}{\zh{收集。}}


\begin{exe}
\sn \textcolor{darkblue}{\textbf{\ipa{le˧-ʂo˧\textasciitilde{}ʂo˧}}}
\trans \mytextsc{accomp} \string_ \mytextsc{red}
\trans \textit{\textcolor{brown}{\zh{\mytextsc{accomp} \string_ \mytextsc{red}}}}
\end{exe}
\begin{exe}
\sn \textcolor{darkblue}{\textbf{\ipa{ʂo˧\textasciitilde{}ʂo˧-zo˧-ho˩-ze˩}}}
\trans Il va falloir rassembler/assembler.
\trans \textit{\textcolor{brown}{\zh{该收集一些了。}}}
\end{exe}


\paragraph{\hspace{-0.5cm} {\Large \textcolor{darkblue}{\textbf{\ipa{ʂo˧}}}}} \hypertarget{s`o\string_M1}{}
\markboth{\textcolor{darkblue}{\textbf{\ipa{ʂo˧}}}}{}
\textcolor{teal}{\mytextsc{interjection}} \hspace{4pt} Ton~: M.

Interjection employée pour faire avancer les cochons, lorsqu'on les guide sur le chemin du pâturage: ``Zou! / Allez!".
\textcolor{brown}{\zh{赶猪用的叹词:走!走!。}}


\begin{exe}
\sn \textcolor{darkblue}{\textbf{\ipa{ʂo˧! / ʂo˧bɤ˩!}}}
\trans Interjection employée pour faire avancer les cochons, lorsqu'on les guide sur le chemin du pâturage: ``Zou! / Allez!"
\trans \textit{\textcolor{brown}{\zh{赶猪用的叹词}}}
\end{exe}


\paragraph{\hspace{-0.5cm} {\Large \textcolor{darkblue}{\textbf{\ipa{ʂo˧ʂo˧}}}}} \hypertarget{s`o\string_Ms`o\string_M1}{}
\markboth{\textcolor{darkblue}{\textbf{\ipa{ʂo˧ʂo˧}}}}{}
\textcolor{teal}{\mytextsc{verbe}} \hspace{4pt} Ton~: .

Préparer.
\textcolor{brown}{\zh{准备。}}


\begin{exe}
\sn \textcolor{darkblue}{\textbf{\ipa{le˧-ʂo˧\textasciitilde{}ʂo˧}}}
\trans \mytextsc{accomp}
\trans \textit{\textcolor{brown}{\zh{\mytextsc{accomp}}}}
\end{exe}


\paragraph{\hspace{-0.5cm} {\Large \textcolor{darkblue}{\textbf{\ipa{‑ʂo˧\textasciitilde{}ʂo˩}}}}} \hypertarget{‑s`o\string_M~s`o\string_B1}{}
\markboth{\textcolor{darkblue}{\textbf{\ipa{‑ʂo˧\textasciitilde{}ʂo˩}}}}{}
\textcolor{teal}{\mytextsc{suffixe}} \hspace{4pt} Ton~: .

Morphème d'adjectivation: signifie ``tout N-eux". Par exemple, ajouté à ``graisse", cela donne ``tout graisseux, plein de graisse".
\textcolor{brown}{\zh{形容词化:……乎乎。}}


\begin{exe}
\sn \textcolor{darkblue}{\textbf{\ipa{mɤ˩-ʂo˩\textasciitilde{}ʂo˥}}}
\trans plein de graisse
\trans \textit{\textcolor{brown}{\zh{油乎乎}}}
\end{exe}
\begin{exe}
\sn \textcolor{darkblue}{\textbf{\ipa{dze˧-ʂo˧\textasciitilde{}ʂo˥}}}
\trans plein de sucre
\trans \textit{\textcolor{brown}{\zh{甜乎乎}}}
\end{exe}
\begin{exe}
\sn \textcolor{darkblue}{\textbf{\ipa{si˧-ʂo˧\textasciitilde{}ʂo˥}}}
\trans plein de bois, tout plein de (beau) bois; ex.: d'une maison de bois nouvellement construite: jusque dans chaque recoin, c'est de bon bois, c'est ``pur bois".
\trans \textit{\textcolor{brown}{\zh{有很多木头,如:来形容一个新建的木头房子:面面都是新鲜木头,给人的感觉是“木头呼呼”}}}
\end{exe}
\begin{exe}
\sn \textcolor{darkblue}{\textbf{\ipa{ʂe˧-ʂo˧\textasciitilde{}ʂo˥}}}
\trans plein de viande: par exemple, un plat contient de la viande en abondance, pas juste quelques petits bouts noyés dans une masse de légumes
\trans \textit{\textcolor{brown}{\zh{肉乎乎:比如一个菜含有很多肉,而不是一点点肉淹在一大堆蔬菜里面。}}}
\end{exe}
\begin{exe}
\sn \textcolor{darkblue}{\textbf{\ipa{tɕi˧-ʂo˧\textasciitilde{}ʂo˥; ʁwɤ˧-ʂo˧\textasciitilde{}ʂo˥; qʰv̩˧-ʂo˧\textasciitilde{}ʂo˥}}}
\trans plein de nuages
\trans \textit{\textcolor{brown}{\zh{云乎乎}}}
\end{exe}
\begin{exe}
\sn \textcolor{darkblue}{\textbf{\ipa{ʁwɤ˧-ʂo˧\textasciitilde{}ʂo˥}}}
\trans plein de montagnes, entouré de montagnes de toutes parts
\trans \textit{\textcolor{brown}{\zh{“山乎乎”:到处都是山}}}
\end{exe}
\begin{exe}
\sn \textcolor{darkblue}{\textbf{\ipa{qʰv̩˧-ʂo˧\textasciitilde{}ʂo˥}}}
\trans plein de trous
\trans \textit{\textcolor{brown}{\zh{“洞乎乎”,有很多洞的}}}
\end{exe}
\begin{exe}
\sn \textcolor{darkblue}{\textbf{\ipa{ɬv̩˧-ʂo˧\textasciitilde{}ʂo˥}}}
\trans plein de cervelle
\trans \textit{\textcolor{brown}{\zh{脑髓乎乎、有很多脑髓(如:一道菜,里面都是脑髓)}}}
\end{exe}


\paragraph{\hspace{-0.5cm} {\Large \textcolor{darkblue}{\textbf{\ipa{ʂo˩\textsubscript{a}}}}}} \hypertarget{s`o\string_Ba1}{}
\markboth{\textcolor{darkblue}{\textbf{\ipa{ʂo˩\textsubscript{a}}}}}{}
\textcolor{teal}{\mytextsc{adjectif}} \hspace{4pt} Ton~: L\textsubscript{a}.

Propre (sens propre ou figuré); claire (eau).
\textcolor{brown}{\zh{干净、整洁,本质干净,清(水)。}}


\begin{exe}
\sn \textcolor{darkblue}{\textbf{\ipa{ʂo˩-hĩ˩˥}}}
\trans \mytextsc{nmlz}
\trans \textit{\textcolor{brown}{\zh{干净的}}}
\end{exe}
\begin{exe}
\sn \textcolor{darkblue}{\textbf{\ipa{mɤ˧-ʂo˩}}}
\trans sale, malpropre
\trans \textit{\textcolor{brown}{\zh{不干净、脏}}}
\end{exe}
\begin{exe}
\sn \textcolor{darkblue}{\textbf{\ipa{ʈʂʰɯ˧ | ʂo˩-hĩ˩ ɲi˥. |}}}
\trans C'est propre.
\trans \textit{\textcolor{brown}{\zh{这是干净的。}}}
\end{exe}
\begin{exe}
\sn \textcolor{darkblue}{\textbf{\ipa{dʑɯ˧ ʂo˧}}}
\trans de l'eau claire, de l'eau propre
\trans \textit{\textcolor{brown}{\zh{清水、干净的水}}}
\end{exe}


\paragraph{\hspace{-0.5cm} {\Large \textcolor{darkblue}{\textbf{\ipa{ʂo˩qæ˩}}}}} \hypertarget{s`o\string_Bq\{\string_B1}{}
\markboth{\textcolor{darkblue}{\textbf{\ipa{ʂo˩qæ˩}}}}{}
\textcolor{teal}{\mytextsc{adjectif}} \hspace{4pt} Ton~: L.

Tout propre.
\textcolor{brown}{\zh{很干净。}}


\begin{exe}
\sn \textcolor{darkblue}{\textbf{\ipa{ʂo˩qæ˩˥ | -gv̩˩}}}
\trans tout propre
\trans \textit{\textcolor{brown}{\zh{很干净}}}
\end{exe}
\begin{exe}
\sn \textcolor{darkblue}{\textbf{\ipa{ɑ˩ʁo˧ | le˧-gv̩˧\textasciitilde{}gv̩˥ | ʂo˩qæ˩˥ | -gv̩˩}}}
\trans ranger la maison, qu'elle soit bien propre
\trans \textit{\textcolor{brown}{\zh{家收拾得干干净净}}}
\end{exe}


\paragraph{\hspace{-0.5cm} {\Large \textcolor{darkblue}{\textbf{\ipa{ʂo˧˥}}} \textsubscript{1}}} \hypertarget{s`o\string_M\string_T1}{}
\markboth{\textcolor{darkblue}{\textbf{\ipa{ʂo˧˥}}} \textsubscript{1}}{}
\textcolor{teal}{\mytextsc{verbe}} \hspace{4pt} Ton~: MH.

Glisser.
\textcolor{brown}{\zh{滑,光滑(路……)。}}


\begin{exe}
\sn \textcolor{darkblue}{\textbf{\ipa{mv̩˩tɕo˧ ʂo˧˥}}}
\trans glisser vers le bas, glisser par terre
\trans \textit{\textcolor{brown}{\zh{滑下、滑倒}}}
\end{exe}
\begin{exe}
\sn \textcolor{darkblue}{\textbf{\ipa{ʈʂʰɯ˧ | le˧-ʂo˧˥, | tʰi˧-ʈwæ˧-ze˥}}}
\trans il a glissé et il est tombé
\trans \textit{\textcolor{brown}{\zh{他滑了一跤}}}
\end{exe}
\begin{exe}
\sn \textcolor{darkblue}{\textbf{\ipa{ʂo˩\textasciitilde{}ʂo˧˥}}}
\trans \mytextsc{red}
\trans \textit{\textcolor{brown}{\zh{\mytextsc{重叠}}}}
\end{exe}
\begin{exe}
\sn \textcolor{darkblue}{\textbf{\ipa{ɖæ˩ʂo˩˥ / ɖæ˩ʂo˩-ze˥}}}
\trans glisser; dévaler une pente en glissant
\trans \textit{\textcolor{brown}{\zh{往下滑}}}
\end{exe}
\begin{exe}
\sn \textcolor{darkblue}{\textbf{\ipa{no˧ | ɖæ˩ʂo˩\textasciitilde{}ɖæ˥ʂo˩! |}}}
\trans Tu es bien malhonnête! (Critique de quelqu'un qui n'est pas franc et direct, qui est faux jeton, qui n'a pas une bonne attitude, donnant une impression huileuse: qui s'esquive et se dérobe, comme un objet glissant qui se dérobe à la prise.)
\trans \textit{\textcolor{brown}{\zh{你真滑头!}}}
\end{exe}
\textit{Voir~:} \hyperlink{s`o\string_M\string_T2}{\textcolor{darkblue}{\textbf{\ipa{ʂo˧˥}}} \textsubscript{2}} 

\paragraph{\hspace{-0.5cm} {\Large \textcolor{darkblue}{\textbf{\ipa{ʂo˧˥}}} \textsubscript{2}}} \hypertarget{s`o\string_M\string_T2}{}
\markboth{\textcolor{darkblue}{\textbf{\ipa{ʂo˧˥}}} \textsubscript{2}}{}
\textcolor{teal}{\mytextsc{adjectif}} \hspace{4pt} Ton~: MH.

Lisse, glissant.
\textcolor{brown}{\zh{光滑(路……)。}}


\begin{exe}
\sn \textcolor{darkblue}{\textbf{\ipa{mɤ˩ ʂo˩-ʂo˥ |}}}
\trans huileux, tout poisseux de graisse
\trans \textit{\textcolor{brown}{\zh{油腻腻、滑腻}}}
\end{exe}
\begin{exe}
\sn \textcolor{darkblue}{\textbf{\ipa{ɲi˧to˧ ɖɯ˧-ɭɯ˧ | dze˧-ʂo˧\textasciitilde{}ʂo˥}}}
\trans toute (sa) bouche est/était pleine de sucre / toute poisseuse à force de sucre
\trans \textit{\textcolor{brown}{\zh{他嘴巴被糖粘得黏黏的}}}
\end{exe}
\textit{Voir~:} \hyperlink{s`o\string_M\string_T1}{\textcolor{darkblue}{\textbf{\ipa{ʂo˧˥}}} \textsubscript{1}} 

\paragraph{\hspace{-0.5cm} {\Large \textcolor{darkblue}{\textbf{\ipa{ʂɻ̍˧˥}}}}} \hypertarget{s`r£`̍\string_M\string_T1}{}
\markboth{\textcolor{darkblue}{\textbf{\ipa{ʂɻ̍˧˥}}}}{}
\textcolor{teal}{\mytextsc{adjectif}} \hspace{4pt} Ton~: MH.

Rempli, plein.
\textcolor{brown}{\zh{满。}}


\begin{exe}
\sn \textcolor{darkblue}{\textbf{\ipa{le˧-ʂɻ̍˧-ze˥}}}
\trans \mytextsc{accomp} \string_ \mytextsc{pfv}
\trans \textit{\textcolor{brown}{\zh{满了}}}
\end{exe}


\paragraph{\hspace{-0.5cm} {\Large \textcolor{darkblue}{\textbf{\ipa{ʂɯ˧}}}}} \hypertarget{s`M\string_M1}{}
\markboth{\textcolor{darkblue}{\textbf{\ipa{ʂɯ˧}}}}{}
\textcolor{teal}{\mytextsc{nombre}} \hspace{4pt} Ton~: M? H\#? (pas L).

7.
\textcolor{brown}{\zh{7。}}




\paragraph{\hspace{-0.5cm} {\Large \textcolor{darkblue}{\textbf{\ipa{ʂɯ˧\textsubscript{a}}}} \textsubscript{1}}} \hypertarget{s`M\string_Ma1}{}
\markboth{\textcolor{darkblue}{\textbf{\ipa{ʂɯ˧\textsubscript{a}}}} \textsubscript{1}}{}
\textcolor{teal}{\mytextsc{verbe}} \hspace{4pt} Ton~: M\textsubscript{a}.

Fuir, s'écouler, se répandre, se vider.
\textcolor{brown}{\zh{漏。}}


\begin{exe}
\sn \textcolor{darkblue}{\textbf{\ipa{tʰi˧-ʂɯ˥\textasciitilde{}ʂɯ˩(-ze˩)}}}
\trans Ca fuit / ça se vide!
\trans \textit{\textcolor{brown}{\zh{漏了!}}}
\end{exe}
\begin{exe}
\sn \textcolor{darkblue}{\textbf{\ipa{mɤ˧-ʂɯ˥\textasciitilde{}ʂɯ˩! | mɤ˧-ʑi˧!}}}
\trans Ca ne s'écoule pas, ça ne fuit pas! (ʑi˧: ``couler")
\trans \textit{\textcolor{brown}{\zh{没漏,没流出去!}}}
\end{exe}


\paragraph{\hspace{-0.5cm} {\Large \textcolor{darkblue}{\textbf{\ipa{ʂɯ˧\textsubscript{a}}}} \textsubscript{2}}} \hypertarget{s`M\string_Ma2}{}
\markboth{\textcolor{darkblue}{\textbf{\ipa{ʂɯ˧\textsubscript{a}}}} \textsubscript{2}}{}
\textcolor{teal}{\mytextsc{verbe}} \hspace{4pt} Ton~: M\textsubscript{a}.

Mourir, décéder.
\textcolor{brown}{\zh{死。}}


\begin{exe}
\sn \textcolor{darkblue}{\textbf{\ipa{le˧-ʂɯ˧-ho˩-ze˩}}}
\trans ça va mourir! (au sujet d'une plante ou d'un animal malade)
\trans \textit{\textcolor{brown}{\zh{快要死了!(病了的植物、动物)}}}
\end{exe}
\begin{exe}
\sn \textcolor{darkblue}{\textbf{\ipa{mɤ˧-ʂɯ˧-sɯ˩!}}}
\trans (Il/elle/ce) n'est pas encore mort!
\trans \textit{\textcolor{brown}{\zh{还没死!}}}
\end{exe}
\begin{exe}
\sn \textcolor{darkblue}{\textbf{\ipa{no˧ | le˧-ʂɯ˧-bi˧-tsæ˧-ɲi˧-ze˩!}}}
\trans Crève donc! / Crève, charogne! (imprécation/malédiction, qu'on lance sous le coup de la colère)
\trans \textit{\textcolor{brown}{\zh{你去死吧!}}}
\end{exe}


\paragraph{\hspace{-0.5cm} {\Large \textcolor{darkblue}{\textbf{\ipa{ʂɯ˧dʑi˧}}}}} \hypertarget{s`M\string_Mdz£i\string_M1}{}
\markboth{\textcolor{darkblue}{\textbf{\ipa{ʂɯ˧dʑi˧}}}}{}
\textcolor{teal}{\mytextsc{nom}} \hspace{4pt} Ton~: M.

Linceul, suaire, vêtement mortuaire (de: ``mourir" et ``habit").
\textcolor{brown}{\zh{寿衣。}}


\begin{exe}
\sn \textcolor{darkblue}{\textbf{\ipa{ʂɯ˧dʑi˧ ʐv̩˥}}}
\trans coudre les vêtements mortuaires, coudre le linceul
\trans \textit{\textcolor{brown}{\zh{缝寿衣}}}
\end{exe}


\paragraph{\hspace{-0.5cm} {\Large \textcolor{darkblue}{\textbf{\ipa{ʂɯ˧-ɬi˧mi˧}}}}} \hypertarget{s`M\string_M-Ki\string_Mmi\string_M1}{}
\markboth{\textcolor{darkblue}{\textbf{\ipa{ʂɯ˧-ɬi˧mi˧}}}}{}
\textcolor{teal}{\mytextsc{nom}} \hspace{4pt} Ton~: M.

7e mois.
\textcolor{brown}{\zh{七月。}}




\paragraph{\hspace{-0.5cm} {\Large \textcolor{darkblue}{\textbf{\ipa{ʂɯ˧ɲi˥}}}}} \hypertarget{s`M\string_MJi\string_T1}{}
\markboth{\textcolor{darkblue}{\textbf{\ipa{ʂɯ˧ɲi˥}}}}{}
\textcolor{teal}{\mytextsc{adverbe}} \hspace{4pt} Ton~: H\#.

Avant-hier.
\textcolor{brown}{\zh{前天。}}


\begin{exe}
\sn \textcolor{darkblue}{\textbf{\ipa{ʂɯ˧ɲi˥ | -ɖɯ˧ɲi˥}}}
\trans avant-hier
\trans \textit{\textcolor{brown}{\zh{前天}}}
\end{exe}


\paragraph{\hspace{-0.5cm} {\Large \textcolor{darkblue}{\textbf{\ipa{ʂɯ˧ʂɯ˧-dzi˩}}}}} \hypertarget{s`M\string_Ms`M\string_M-dzi\string_B1}{}
\markboth{\textcolor{darkblue}{\textbf{\ipa{ʂɯ˧ʂɯ˧-dzi˩}}}}{}
\textcolor{teal}{\mytextsc{nom}} \hspace{4pt} Ton~: -L.

Yyyy.
\textcolor{brown}{\zh{三颗针。}}


 \mytextsc{clf}~: \textcolor{darkblue}{\textbf{\ipa{dzi˩}}} 

\paragraph{\hspace{-0.5cm} {\Large \textcolor{darkblue}{\textbf{\ipa{ʂɯ˧tɤ˧ɻ\#˥}}}}} \hypertarget{s`M\string_Mt7\string_Mr£`\#\string_T1}{}
\markboth{\textcolor{darkblue}{\textbf{\ipa{ʂɯ˧tɤ˧ɻ\#˥}}}}{}
\textcolor{teal}{\mytextsc{adjectif}} \hspace{4pt} Ton~: \#H.

Lisse; par exemple: un pilier en bois, qui devient bien lisse par le travail du menuisier.
\textcolor{brown}{\zh{平滑。}}


\begin{exe}
\sn \textcolor{darkblue}{\textbf{\ipa{ʂɯ˧tɤ˧ɻ̍˧-zo˥}}}
\trans bien lisse
\trans \textit{\textcolor{brown}{\zh{很平滑}}}
\end{exe}
\begin{exe}
\sn \textcolor{darkblue}{\textbf{\ipa{ʂɯ˧tɤ˧ɻ̍˧ gv̩˧-ze˩}}}
\trans on l'a bien lissé, on l'a rendu bien lisse
\trans \textit{\textcolor{brown}{\zh{弄得平滑了}}}
\end{exe}


\paragraph{\hspace{-0.5cm} {\Large \textcolor{darkblue}{\textbf{\ipa{ʂɯ˧tsʰi˩}}}}} \hypertarget{s`M\string_Mts\string_hi\string_B1}{}
\markboth{\textcolor{darkblue}{\textbf{\ipa{ʂɯ˧tsʰi˩}}}}{}
\textcolor{teal}{\mytextsc{nombre}} \hspace{4pt} Ton~: L\#.

70.
\textcolor{brown}{\zh{70。}}




\paragraph{\hspace{-0.5cm} {\Large \textcolor{darkblue}{\textbf{\ipa{ʂɯ˩\textsubscript{b}}}}}} \hypertarget{s`M\string_Bb1}{}
\markboth{\textcolor{darkblue}{\textbf{\ipa{ʂɯ˩\textsubscript{b}}}}}{}
\textcolor{teal}{\mytextsc{classificateur}} \hspace{4pt} Ton~: L\textsubscript{b}.

Classificateur des fois (répétitions d'une action).
\textcolor{brown}{\zh{量词:次数。}}




\paragraph{\hspace{-0.5cm} {\Large \textcolor{darkblue}{\textbf{\ipa{ʂɯ˩ʝi\#˥}}}}} \hypertarget{s`M\string_Bj££i\#\string_T1}{}
\markboth{\textcolor{darkblue}{\textbf{\ipa{ʂɯ˩ʝi\#˥}}}}{}
\textcolor{teal}{\mytextsc{adverbe}} \hspace{4pt} Ton~: LM+\#H.

Il y a deux ans.
\textcolor{brown}{\zh{前年。}}


\begin{exe}
\sn \textcolor{darkblue}{\textbf{\ipa{ʂɯ˩ʝi˥ | ɖɯ˧-kʰv̩˧˥}}}
\trans il y a deux ans, l'année il y a deux ans
\trans \textit{\textcolor{brown}{\zh{前年}}}
\end{exe}


\paragraph{\hspace{-0.5cm} {\Large \textcolor{darkblue}{\textbf{\ipa{ʂɯ˩kwæ˩ɻæ˥}}}}} \hypertarget{s`M\string_Bkw\{\string_Br£`\{\string_T1}{}
\markboth{\textcolor{darkblue}{\textbf{\ipa{ʂɯ˩kwæ˩ɻæ˥}}}}{}
\textcolor{teal}{\mytextsc{adjectif}} \hspace{4pt} Ton~: L+H\#.

Jaune.
\textcolor{brown}{\zh{黄。}}


\begin{exe}
\sn \textcolor{darkblue}{\textbf{\ipa{ʂɯ˩kwæ˩ɻæ˥-hĩ˩ gv̩˩-ze˩}}}
\trans [le livre] a jauni!
\trans \textit{\textcolor{brown}{\zh{[书]变黄了!}}}
\end{exe}
\begin{exe}
\sn \textcolor{darkblue}{\textbf{\ipa{[F5] ʂɯ˩kwæ˩˥ | ʂɯ˩kwæ˩˥ | gv̩˩}}}
\trans tout jaune
\trans \textit{\textcolor{brown}{\zh{深黄}}}
\end{exe}


\paragraph{\hspace{-0.5cm} {\Large \textcolor{darkblue}{\textbf{\ipa{ʂɯ˩tsɯ˧}}}}} \hypertarget{s`M\string_BtsM\string_M1}{}
\markboth{\textcolor{darkblue}{\textbf{\ipa{ʂɯ˩tsɯ˧}}}}{}
\textcolor{teal}{\mytextsc{nom}} \hspace{4pt} Ton~: LM.

Kaki.
\textcolor{brown}{\zh{柿子(汉语借词)。}}

 Emprunt~: chinois \textcolor{brown}{\zh{柿子}}

\begin{exe}
\sn \textcolor{darkblue}{\textbf{\ipa{ʂɯ˩tsɯ˧ | ɖɯ˧-so˩-ɭɯ˩ hwæ˩-bi˩!}}}
\trans (Je) vais acheter quelques kakis!
\trans \textit{\textcolor{brown}{\zh{买一些柿子吧!}}}
\end{exe}


\paragraph{\hspace{-0.5cm} {\Large \textcolor{darkblue}{\textbf{\ipa{ʂɯ˩tsɯ˧}}}}} \hypertarget{s`M\string_BtsM\string_M1}{}
\markboth{\textcolor{darkblue}{\textbf{\ipa{ʂɯ˩tsɯ˧}}}}{}
\textcolor{teal}{\mytextsc{nom}} \hspace{4pt} Ton~: LM.

Pistolet.
\textcolor{brown}{\zh{手枪。}}


\begin{exe}
\sn \textcolor{darkblue}{\textbf{\ipa{ʂɯ˩tsɯ˧ | ɖɯ˧-nɑ˧ | tʰi˧-pɤ˥\textasciitilde{}pɤ˩}}}
\trans porter un pistolet
\trans \textit{\textcolor{brown}{\zh{带手枪}}}
\end{exe}


\paragraph{\hspace{-0.5cm} {\Large \textcolor{darkblue}{\textbf{\ipa{ʂɯ˧˥}}} \textsubscript{1}}} \hypertarget{s`M\string_M\string_T1}{}
\markboth{\textcolor{darkblue}{\textbf{\ipa{ʂɯ˧˥}}} \textsubscript{1}}{}
\textcolor{teal}{\mytextsc{verbe}} \hspace{4pt} Ton~: MH.

Éplucher, peler, décortiquer (avec un instrument).
\textcolor{brown}{\zh{削(用刀)。}}


\begin{exe}
\sn \textcolor{darkblue}{\textbf{\ipa{ɣɯ˩ ʂɯ˧˥ / ɣɯ˩ʂɯ˧ ze˥}}}
\trans éplucher la peau
\trans \textit{\textcolor{brown}{\zh{削皮}}}
\end{exe}
\begin{exe}
\sn \textcolor{darkblue}{\textbf{\ipa{ɣɯ˩kɯ˧ ʂɯ˥}}}
\trans éplucher la peau
\trans \textit{\textcolor{brown}{\zh{削皮}}}
\end{exe}
\begin{exe}
\sn \textcolor{darkblue}{\textbf{\ipa{jɤ˩jo˧ ɣɯ˥ʂɯ˩}}}
\trans peler des patates
\trans \textit{\textcolor{brown}{\zh{削洋芋皮}}}
\end{exe}
\begin{exe}
\sn \textcolor{darkblue}{\textbf{\ipa{[F5] tso˧tso˧ ɣɯ˥ʂɯ˩}}}
\trans éplucher des choses
\trans \textit{\textcolor{brown}{\zh{削东西}}}
\end{exe}


\paragraph{\hspace{-0.5cm} {\Large \textcolor{darkblue}{\textbf{\ipa{ʂɯ˧˥}}} \textsubscript{2}}} \hypertarget{s`M\string_M\string_T2}{}
\markboth{\textcolor{darkblue}{\textbf{\ipa{ʂɯ˧˥}}} \textsubscript{2}}{}
\textcolor{teal}{\mytextsc{adjectif}} \hspace{4pt} Ton~: MH.

Nouveau, neuf, frais.
\textcolor{brown}{\zh{新。}}


\begin{exe}
\sn \textcolor{darkblue}{\textbf{\ipa{ʂɯ˧-hĩ˧ ɲi˥!}}}
\trans c'est neuf!
\trans \textit{\textcolor{brown}{\zh{是新的!}}}
\end{exe}
\begin{exe}
\sn \textcolor{darkblue}{\textbf{\ipa{ʂe˧ ʂɯ˩}}}
\trans de la viande fraîche
\trans \textit{\textcolor{brown}{\zh{新鲜的肉}}}
\end{exe}


\paragraph{\hspace{-0.5cm} {\Large \textcolor{darkblue}{\textbf{\ipa{ʂv̩˧˥}}}}} \hypertarget{s`v\string_=\string_M\string_T1}{}
\markboth{\textcolor{darkblue}{\textbf{\ipa{ʂv̩˧˥}}}}{}
\textcolor{teal}{\mytextsc{verbe}} \hspace{4pt} Ton~: MH.

Tordre, essorer (vêtement).
\textcolor{brown}{\zh{拧(拧毛巾)。}}


\begin{exe}
\sn \textcolor{darkblue}{\textbf{\ipa{le˧-ʂv̩˧-ze˥}}}
\trans \mytextsc{accomp} \string_ \mytextsc{pfv}
\trans \textit{\textcolor{brown}{\zh{拧了}}}
\end{exe}
\begin{exe}
\sn \textcolor{darkblue}{\textbf{\ipa{dʑi˧hṽ˧ ʂv̩˩}}}
\trans essorer des vêtements
\trans \textit{\textcolor{brown}{\zh{拧衣服}}}
\end{exe}


\paragraph{\hspace{-0.5cm} {\Large \textcolor{darkblue}{\textbf{\ipa{ʂv̩˥}}}}} \hypertarget{s`v\string_=\string_T1}{}
\markboth{\textcolor{darkblue}{\textbf{\ipa{ʂv̩˥}}}}{}
\textcolor{teal}{\mytextsc{nom}} \hspace{4pt} Ton~: H.

Dé.
\textcolor{brown}{\zh{骰子。}}


\begin{exe}
\sn \textcolor{darkblue}{\textbf{\ipa{ʂv̩˧ | ʐv̩˩-ɭɯ˩˥}}}
\trans quatre dés (les dés allaient par quatre)
\trans \textit{\textcolor{brown}{\zh{四个骰子}}}
\end{exe}
 \mytextsc{clf}~: \textcolor{darkblue}{\textbf{\ipa{ɭɯ˧}}} 

\paragraph{\hspace{-0.5cm} {\Large \textcolor{darkblue}{\textbf{\ipa{ʂv̩˧\textsubscript{b}}}}}} \hypertarget{s`v\string_=\string_Mb1}{}
\markboth{\textcolor{darkblue}{\textbf{\ipa{ʂv̩˧\textsubscript{b}}}}}{}
\textcolor{teal}{\mytextsc{verbe}} \hspace{4pt} Ton~: M\textsubscript{b}.
\ding{202} 
S'occuper de, surveiller.
\textcolor{brown}{\zh{带(孩子……)。}}


\begin{exe}
\sn \textcolor{darkblue}{\textbf{\ipa{zo˧mv̩˥ | ɖɯ˧-ɭɯ˧ ʂv̩˧}}}
\trans garder un enfant, s'occuper d'un enfant
\trans \textit{\textcolor{brown}{\zh{带个孩子}}}
\end{exe}
\begin{exe}
\sn \textcolor{darkblue}{\textbf{\ipa{zo˧mv̩˥ ʂv̩˩}}}
\trans garder un enfant, s'occuper d'un enfant
\trans \textit{\textcolor{brown}{\zh{带孩子}}}
\end{exe}
\begin{exe}
\sn \textcolor{darkblue}{\textbf{\ipa{le˧-ʂv̩˧ tʰi˧-kʰɯ˧˥ | tʰæ˧ɻ̍˩ so˩}}}
\trans obliger à étudier (une mère oblige son enfant à faire ses devoirs)
\trans \textit{\textcolor{brown}{\zh{让他学习、要求他学习(家长管孩子,让他学习)}}}
\end{exe}
\ding{203} 
Mener, guider.
\textcolor{brown}{\zh{带(路)。}}


\begin{exe}
\sn \textcolor{darkblue}{\textbf{\ipa{ʈæ˧ʂɯ˧ | ɖʐv̩˧ ʂv̩˧-po˧-bi˧-ho˥!}}}
\trans Dashi va accompagner ses amis [=les emmener en excursion à Yongning]!
\trans \textit{\textcolor{brown}{\zh{达石要管朋友(带他们去永宁旅游)}}}
\end{exe}
\begin{exe}
\sn \textcolor{darkblue}{\textbf{\ipa{ʈæ˧ʂɯ˧ | ɖʐv̩˧ ʂv̩˧-bi˧-ho˩!}}}
\trans Dashi va accompagner ses amis [=les emmener en excursion à Yongning]!
\trans \textit{\textcolor{brown}{\zh{达石要管朋友(带他们去永宁旅游)}}}
\end{exe}


\paragraph{\hspace{-0.5cm} {\Large \textcolor{darkblue}{\textbf{\ipa{ʂv̩˧ɖv̩˧}}}}} \hypertarget{s`v\string_=\string_Md`v\string_=\string_M1}{}
\markboth{\textcolor{darkblue}{\textbf{\ipa{ʂv̩˧ɖv̩˧}}}}{}
\textcolor{teal}{\mytextsc{verbe}} \hspace{4pt} Ton~: M.
\ding{202} 
Penser, réfléchir.
\textcolor{brown}{\zh{想。}}


\begin{exe}
\sn \textcolor{darkblue}{\textbf{\ipa{ə˧tso˧ ʂv̩˧ɖv̩˧?}}}
\trans A quoi [tu] penses?
\trans \textit{\textcolor{brown}{\zh{在想什么?}}}
\end{exe}
\begin{exe}
\sn \textcolor{darkblue}{\textbf{\ipa{njɤ˧ | ɖɯ˧ bæ˧ ʂv̩˧dv̩˧}}}
\trans je pense à quelque chose
\trans \textit{\textcolor{brown}{\zh{我在想一件事情。}}}
\end{exe}
\begin{exe}
\sn \textcolor{darkblue}{\textbf{\ipa{ʂv̩˧ɖv̩˧ tʰv̩˧}}}
\trans comprendre; se souvenir
\trans \textit{\textcolor{brown}{\zh{明白,想起}}}
\end{exe}
\begin{exe}
\sn \textcolor{darkblue}{\textbf{\ipa{njɤ˧ | ʂv̩˧ɖv̩˧ tʰv̩˧}}}
\trans Je comprends.
\trans \textit{\textcolor{brown}{\zh{我明白。}}}
\end{exe}
\begin{exe}
\sn \textcolor{darkblue}{\textbf{\ipa{ʈʂʰɯ˧ | le˧-ʂv̩˧ɖv̩˧-le˧-tʰv̩˧-ze˧}}}
\trans Il a compris.
\trans \textit{\textcolor{brown}{\zh{他明白了。}}}
\end{exe}
\ding{203} 
Se souvenir.
\textcolor{brown}{\zh{想起、回忆。}}


\begin{exe}
\sn \textcolor{darkblue}{\textbf{\ipa{ʂv̩˧ɖv̩˧ tʰv̩˧}}}
\trans se souvenir
\trans \textit{\textcolor{brown}{\zh{想起}}}
\end{exe}
\begin{exe}
\sn \textcolor{darkblue}{\textbf{\ipa{njɤ˧ | ʂv̩˧ɖv̩˧ tʰv̩˧}}}
\trans je me souviens
\trans \textit{\textcolor{brown}{\zh{我想起}}}
\end{exe}
\begin{exe}
\sn \textcolor{darkblue}{\textbf{\ipa{ʈʂʰɯ˧ | le˧-ʂv̩˧ɖv̩˧-le˧-tʰv̩˧-ze˧}}}
\trans il se rappelle, ça lui revient
\trans \textit{\textcolor{brown}{\zh{他想起来了}}}
\end{exe}
\ding{204} 
Avoir la nostalgie de.
\textcolor{brown}{\zh{想念、感到悲哀。}}


\begin{exe}
\sn \textcolor{darkblue}{\textbf{\ipa{ʂv̩˧ɖv̩˧ tʰv̩˧ | ʐwæ˩˥}}}
\trans être plongé dans le chagrin
\trans \textit{\textcolor{brown}{\zh{特别想念}}}
\end{exe}
\begin{exe}
\sn \textcolor{darkblue}{\textbf{\ipa{njɤ˧ | no˩ ʂv̩˩ɖv̩˩˥}}}
\trans tu me manques!
\trans \textit{\textcolor{brown}{\zh{我想你!}}}
\end{exe}
\begin{exe}
\sn \textcolor{darkblue}{\textbf{\ipa{ʂv̩˧ɖv̩˧ qʰwɤ˧ tʰv̩˧˥}}}
\trans être nostalgique, avoir une crise de nostalgie
\trans \textit{\textcolor{brown}{\zh{想念}}}
\end{exe}
\begin{exe}
\sn \textcolor{darkblue}{\textbf{\ipa{[F5] ʂv̩˧ɖv̩˧ mɤ˧-zo˧}}}
\trans on n'a pas à se faire de souci, il n'y a pas lieu de se morfondre
\trans \textit{\textcolor{brown}{\zh{不用发愁}}}
\end{exe}


\paragraph{\hspace{-0.5cm} {\Large \textcolor{darkblue}{\textbf{\ipa{ʂv̩˩gv̩˩}}}}} \hypertarget{s`v\string_=\string_Bgv\string_=\string_B1}{}
\markboth{\textcolor{darkblue}{\textbf{\ipa{ʂv̩˩gv̩˩}}}}{}
\textcolor{teal}{\mytextsc{nom}} \hspace{4pt} Ton~: L.

Faucille.
\textcolor{brown}{\zh{镰刀。}}


 \mytextsc{clf}~: \textcolor{darkblue}{\textbf{\ipa{nɑ˧}}} 

\paragraph{\hspace{-0.5cm} {\Large \textcolor{darkblue}{\textbf{\ipa{ʂv̩˧kʰɯ˩}}}}} \hypertarget{s`v\string_=\string_Mk\string_hM\string_B1}{}
\markboth{\textcolor{darkblue}{\textbf{\ipa{ʂv̩˧kʰɯ˩}}}}{}
\textcolor{teal}{\mytextsc{verbe}} \hspace{4pt} Ton~: .

Parier.
\textcolor{brown}{\zh{赌博。}}


\begin{exe}
\sn \textcolor{darkblue}{\textbf{\ipa{ʂv̩˧kʰɯ˩ | -jɤ˩po˧}}}
\trans même sens
\trans \textit{\textcolor{brown}{\zh{同上}}}
\end{exe}


\paragraph{\hspace{-0.5cm} {\Large \textcolor{darkblue}{\textbf{\ipa{ʂv̩˩njɤ˥}}}}} \hypertarget{s`v\string_=\string_Bnj7\string_T1}{}
\markboth{\textcolor{darkblue}{\textbf{\ipa{ʂv̩˩njɤ˥}}}}{}
\textcolor{teal}{\mytextsc{nom}} \hspace{4pt} Ton~: LH.

Nœud (sur un arbre).
\textcolor{brown}{\zh{树瘤。}}


\begin{exe}
\sn \textcolor{darkblue}{\textbf{\ipa{ʂv̩˩njɤ˥ ɲi˩}}}
\trans \mytextsc{cop}
\trans \textit{\textcolor{brown}{\zh{是树瘤。}}}
\end{exe}
 \mytextsc{clf}~: \textcolor{darkblue}{\textbf{\ipa{ɭɯ˧}}} 

\paragraph{\hspace{-0.5cm} {\Large \textcolor{darkblue}{\textbf{\ipa{ʂv̩˧ʂv̩˧˥}}}}} \hypertarget{s`v\string_=\string_Ms`v\string_=\string_M\string_T1}{}
\markboth{\textcolor{darkblue}{\textbf{\ipa{ʂv̩˧ʂv̩˧˥}}}}{}
\textcolor{teal}{\mytextsc{nom}} \hspace{4pt} Ton~: MH\#.

Papier.
\textcolor{brown}{\zh{纸。}}


\begin{exe}
\sn \textcolor{darkblue}{\textbf{\ipa{ʂv̩˧ʂv̩˧˥ | ɖɯ˧-pʰæ˧˥}}}
\trans une feuille de papier
\trans \textit{\textcolor{brown}{\zh{一张纸}}}
\end{exe}
\begin{exe}
\sn \textcolor{darkblue}{\textbf{\ipa{[F5] ʂv̩˧ʂv̩˧ ɖɯ˧ pʰæ˧˥}}}
\trans une feuille de papier
\trans \textit{\textcolor{brown}{\zh{一张纸}}}
\end{exe}
 \mytextsc{clf}~: \textcolor{darkblue}{\textbf{\ipa{pʰæ˧˥}}} objets plats

\paragraph{\hspace{-0.5cm} {\Large \textcolor{darkblue}{\textbf{\ipa{‑ʂv̩˧ɖv̩˧}}}}} \hypertarget{‑s`v\string_=\string_Md`v\string_=\string_M1}{}
\markboth{\textcolor{darkblue}{\textbf{\ipa{‑ʂv̩˧ɖv̩˧}}}}{}
\textcolor{teal}{\mytextsc{suffixe}} \hspace{4pt} Ton~: .

Volitif: vouloir, souhaiter.
\textcolor{brown}{\zh{想、意志。}}




\paragraph{\hspace{-0.5cm} {\Large \textcolor{darkblue}{\textbf{\ipa{ʂwæ˧}}}}} \hypertarget{s`w\{\string_M1}{}
\markboth{\textcolor{darkblue}{\textbf{\ipa{ʂwæ˧}}}}{}
\textcolor{teal}{\mytextsc{nom}} \hspace{4pt} Ton~: M.

Loutre.
\textcolor{brown}{\zh{水獭。}}Dialecte chinois local~: \zh{水潭猫。}


\begin{exe}
\sn \textcolor{darkblue}{\textbf{\ipa{ʂwæ˧-ɣɯ˩}}}
\trans peau de loutre
\trans \textit{\textcolor{brown}{\zh{水獭皮}}}
\end{exe}


\paragraph{\hspace{-0.5cm} {\Large \textcolor{darkblue}{\textbf{\ipa{ʂwæ˧}}}}} \hypertarget{s`w\{\string_M1}{}
\markboth{\textcolor{darkblue}{\textbf{\ipa{ʂwæ˧}}}}{}
\textcolor{teal}{\mytextsc{adjectif}} \hspace{4pt} Ton~: M.

Haut, de haute taille, grand.
\textcolor{brown}{\zh{高。}}


\begin{exe}
\sn \textcolor{darkblue}{\textbf{\ipa{qʰɑ˧-ʂwæ˧-gv̩˧}}}
\trans très grand
\trans \textit{\textcolor{brown}{\zh{非常高}}}
\end{exe}
\begin{exe}
\sn \textcolor{darkblue}{\textbf{\ipa{ʈʂʰɯ˧ | ə˧pɤ˧ | -ʂwæ˩-gv̩˩˥!}}}
\trans Elle/il est très grand(e)!
\trans \textit{\textcolor{brown}{\zh{他非常高!}}}
\end{exe}
\begin{exe}
\sn \textcolor{darkblue}{\textbf{\ipa{gv̩˧mi˧ ʂwæ˧}}}
\trans de grande taille; littéralement '(qui a un) grand corps'
\trans \textit{\textcolor{brown}{\zh{高、身材高}}}
\end{exe}


\paragraph{\hspace{-0.5cm} {\Large \textcolor{darkblue}{\textbf{\ipa{ʂwæ˧\textsubscript{a}}}}}} \hypertarget{s`w\{\string_Ma1}{}
\markboth{\textcolor{darkblue}{\textbf{\ipa{ʂwæ˧\textsubscript{a}}}}}{}
\textcolor{teal}{\mytextsc{verbe}} \hspace{4pt} Ton~: M\textsubscript{a}.

Remuer (monosyllabe).
\textcolor{brown}{\zh{搅拌。}}


\begin{exe}
\sn \textcolor{darkblue}{\textbf{\ipa{le˧-ʂwæ˧}}}
\trans \mytextsc{accomp}
\trans \textit{\textcolor{brown}{\zh{\mytextsc{accomp}}}}
\end{exe}
\begin{exe}
\sn \textcolor{darkblue}{\textbf{\ipa{mɤ˧-ʂwæ˧}}}
\trans \mytextsc{neg}
\trans \textit{\textcolor{brown}{\zh{不搅拌}}}
\end{exe}
\begin{exe}
\sn \textcolor{darkblue}{\textbf{\ipa{tso˧\textasciitilde{}tso˧ ʂwæ˩}}}
\trans remuer des choses, touiller des choses
\trans \textit{\textcolor{brown}{\zh{搅拌东西}}}
\end{exe}


\paragraph{\hspace{-0.5cm} {\Large \textcolor{darkblue}{\textbf{\ipa{ʂwæ˧bæ˩}}}}} \hypertarget{s`w\{\string_Mb\{\string_B1}{}
\markboth{\textcolor{darkblue}{\textbf{\ipa{ʂwæ˧bæ˩}}}}{}
\textcolor{teal}{\mytextsc{nom}} \hspace{4pt} Ton~: L\#.

Camélia.
\textcolor{brown}{\zh{映山红。}}Dialecte chinois local~: \zh{山茶花。}


\begin{exe}
\sn \textcolor{darkblue}{\textbf{\ipa{ʂwæ˧bæ˩ bæ˩ |}}}
\trans Les camélias sont en fleurs.
\trans \textit{\textcolor{brown}{\zh{山茶花开了。}}}
\end{exe}
\begin{exe}
\sn \textcolor{darkblue}{\textbf{\ipa{so˧-ɬi˧mi˧, | ʂwæ˧bæ˩ bæ˩! |}}}
\trans Les camélias fleurissent au troisième mois!
\trans \textit{\textcolor{brown}{\zh{山茶花是在三月份开花的!}}}
\end{exe}
\begin{exe}
\sn \textcolor{darkblue}{\textbf{\ipa{ʂwæ˧bæ˩-si˩}}}
\trans camélia (arbre); littéralement ``arbre des fleurs de camélia"
\trans \textit{\textcolor{brown}{\zh{山茶树}}}
\end{exe}


\paragraph{\hspace{-0.5cm} {\Large \textcolor{darkblue}{\textbf{\ipa{ʂwæ˧gv̩\#˥}}}}} \hypertarget{s`w\{\string_Mgv\string_=\#\string_T1}{}
\markboth{\textcolor{darkblue}{\textbf{\ipa{ʂwæ˧gv̩\#˥}}}}{}
\textcolor{teal}{\mytextsc{nom}} \hspace{4pt} Ton~: \#H.

Une montagne au nord-ouest de Yongning.
\textcolor{brown}{\zh{加泽大山(位于永宁西北的一座山)。}}


\begin{exe}
\sn \textcolor{darkblue}{\textbf{\ipa{kɤ˧mv̩˧˥, | æ˧ʂæ˧, | ŋwɤ˧hɑ̃˩, | ʂwæ˧gv̩\#˥, | nɑ˩tsʰi˩˥ | -tɕʰɤ˧pɤ˧mi\#˥, | qv̩˧ɻ̍˧-ʈʂʰɑ˧nɑ˥ |}}}
\trans Les six montagnes de Yongning qui portent un nom. Les autres sommets du voisinage n'ont pas une valeur symbolique comparable, et ne portent pas de nom communément utilisé.
\trans \textit{\textcolor{brown}{\zh{永宁地区有固定名字的六座山。其它的山,因为没有重要的象征意义,因此没有取名。}}}
\end{exe}


\paragraph{\hspace{-0.5cm} {\Large \textcolor{darkblue}{\textbf{\ipa{ʂwæ˧ɻæ˧}}}}} \hypertarget{s`w\{\string_Mr£`\{\string_M1}{}
\markboth{\textcolor{darkblue}{\textbf{\ipa{ʂwæ˧ɻæ˧}}}}{}
\textcolor{teal}{\mytextsc{adjectif}} \hspace{4pt} Ton~: .

Être paralysé des jambes.
\textcolor{brown}{\zh{瘸腿。}}


\begin{exe}
\sn \textcolor{darkblue}{\textbf{\ipa{ʈʂʰɯ˧ | ʂwæ˧ɻæ˧-hĩ˧ ɖɯ˧-v̩˧ ɲi˩!}}}
\trans Il est paralysé des jambes! (=il ne peut que se traîner sur le sol)
\trans \textit{\textcolor{brown}{\zh{他腿瘸了!}}}
\end{exe}


\paragraph{\hspace{-0.5cm} {\Large \textcolor{darkblue}{\textbf{\ipa{ʂwæ˧si\#˥}}}}} \hypertarget{s`w\{\string_Msi\#\string_T1}{}
\markboth{\textcolor{darkblue}{\textbf{\ipa{ʂwæ˧si\#˥}}}}{}
\textcolor{teal}{\mytextsc{nom}} \hspace{4pt} Ton~: \#H.

Camélia (arbre).
\textcolor{brown}{\zh{山茶树。}}




\paragraph{\hspace{-0.5cm} {\Large \textcolor{darkblue}{\textbf{\ipa{ʂwæ˧tsɯ˥}}}}} \hypertarget{s`w\{\string_MtsM\string_T1}{}
\markboth{\textcolor{darkblue}{\textbf{\ipa{ʂwæ˧tsɯ˥}}}}{}
\textcolor{teal}{\mytextsc{nom}} \hspace{4pt} Ton~: H\#.

Brosse.
\textcolor{brown}{\zh{刷子。}}

 Emprunt~: chinois \textcolor{brown}{\zh{刷子}}

 \mytextsc{clf}~: \textcolor{darkblue}{\textbf{\ipa{nɑ˧}}} 

\paragraph{\hspace{-0.5cm} {\Large \textcolor{darkblue}{\textbf{\ipa{ʂwæ˩\textsubscript{a}}}}}} \hypertarget{s`w\{\string_Ba1}{}
\markboth{\textcolor{darkblue}{\textbf{\ipa{ʂwæ˩\textsubscript{a}}}}}{}
\textcolor{teal}{\mytextsc{verbe}} \hspace{4pt} Ton~: L\textsubscript{a}.

Fumer (aliment).
\textcolor{brown}{\zh{熏。}}


\begin{exe}
\sn \textcolor{darkblue}{\textbf{\ipa{ʂe˧ ʂwæ˥}}}
\trans fumer de la viande
\trans \textit{\textcolor{brown}{\zh{熏肉}}}
\end{exe}


\paragraph{\hspace{-0.5cm} {\Large \textcolor{darkblue}{\textbf{\ipa{ʂwæ˩gv̩˩}}}}} \hypertarget{s`w\{\string_Bgv\string_=\string_B1}{}
\markboth{\textcolor{darkblue}{\textbf{\ipa{ʂwæ˩gv̩˩}}}}{}
\textcolor{teal}{\mytextsc{nom}} \hspace{4pt} Ton~: L.

Buffet (où est rangée la vaisselle).
\textcolor{brown}{\zh{柜子。}}


 \mytextsc{clf}~: \textcolor{darkblue}{\textbf{\ipa{ɭɯ˧}}} 

\paragraph{\hspace{-0.5cm} {\Large \textcolor{darkblue}{\textbf{\ipa{ʂwæ˧˥}}}}} \hypertarget{s`w\{\string_M\string_T1}{}
\markboth{\textcolor{darkblue}{\textbf{\ipa{ʂwæ˧˥}}}}{}
\textcolor{teal}{\mytextsc{verbe}} \hspace{4pt} Ton~: MH.

Déféquer, faire caca.
\textcolor{brown}{\zh{拉(屎)。}}


\begin{exe}
\sn \textcolor{darkblue}{\textbf{\ipa{qʰæ˧ ʂwæ˩}}}
\trans déféquer
\trans \textit{\textcolor{brown}{\zh{拉屎}}}
\end{exe}


\paragraph{\hspace{-0.5cm} {\Large \textcolor{darkblue}{\textbf{\ipa{ʂwæ˩˥}}}}} \hypertarget{s`w\{\string_B\string_T1}{}
\markboth{\textcolor{darkblue}{\textbf{\ipa{ʂwæ˩˥}}}}{}
\textcolor{teal}{\mytextsc{nom}} \hspace{4pt} Ton~: LH.

Coin.
\textcolor{brown}{\zh{楔子。}}


\begin{exe}
\sn \textcolor{darkblue}{\textbf{\ipa{ʂwæ˩ lɑ˧˥ / ʂwæ˩ lɑ˧-ze˥}}}
\trans frapper un coin
\trans \textit{\textcolor{brown}{\zh{打一个楔子}}}
\end{exe}
\begin{exe}
\sn \textcolor{darkblue}{\textbf{\ipa{ʂwæ˩ hwæ˥-ze˩}}}
\trans ...a acheté un coin
\trans \textit{\textcolor{brown}{\zh{买了楔子}}}
\end{exe}
\begin{exe}
\sn \textcolor{darkblue}{\textbf{\ipa{ʂwæ˩ tʰv̩˩-ɭɯ˩˥ / ʂwæ˩ tʰv̩˩-ɭɯ˥}}}
\trans \string_ \mytextsc{dem}+\mytextsc{clf}.générique
\trans \textit{\textcolor{brown}{\zh{这个楔子}}}
\end{exe}
\begin{exe}
\sn \textcolor{darkblue}{\textbf{\ipa{ʂwæ˩ tʰv̩˩-kʰwɤ˩˥}}}
\trans \mytextsc{n}+\mytextsc{dem}+\mytextsc{clf}.morceaux
\trans \textit{\textcolor{brown}{\zh{这个楔子}}}
\end{exe}
\begin{exe}
\sn \textcolor{darkblue}{\textbf{\ipa{[F5] ʂwæ˩ kʰɯ˥}}}
\trans mettre un coin
\trans \textit{\textcolor{brown}{\zh{放一个楔子}}}
\end{exe}
 \mytextsc{clf}~: \textcolor{darkblue}{\textbf{\ipa{kʰwɤ˥ / ɭɯ˧}}} 

\paragraph{\hspace{-0.5cm} {\Large \textcolor{darkblue}{\textbf{\ipa{ʂwɤ˧ljɤ˧-kwɤ˩}}}}} \hypertarget{s`w7\string_Mlj7\string_M-kw7\string_B1}{}
\markboth{\textcolor{darkblue}{\textbf{\ipa{ʂwɤ˧ljɤ˧-kwɤ˩}}}}{}
\textcolor{teal}{\mytextsc{nom}} \hspace{4pt} Ton~: \mytextsc{L}.

Clochette qui se fixait sur le poitrail des chevaux.
\textcolor{brown}{\zh{挂在马胸前的铃铛。}}


 \mytextsc{clf}~: \textcolor{darkblue}{\textbf{\ipa{kwɤ˩}}} 

\newpage
\section*{\centering- \textcolor{darkblue}{\textbf{\ipa{t}}} -}
\paragraph{\hspace{-0.5cm} {\Large \textcolor{darkblue}{\textbf{\ipa{‑tɑ.kɤ}}}}} \hypertarget{‑tA.k71}{}
\markboth{\textcolor{darkblue}{\textbf{\ipa{‑tɑ.kɤ}}}}{}
\textcolor{teal}{\mytextsc{suffixe}} \hspace{4pt} Ton~: .

Seulement (dénombrable), seul.
\textcolor{brown}{\zh{只、才。}}


\begin{exe}
\sn \textcolor{darkblue}{\textbf{\ipa{õ˧-tɑ˧kɤ˥, | ʐwɤ˩ mɤ˩-ʁo˩˥!}}}
\trans ``Quand on est tout seul, on n'arrive pas à parler!" (Réflexion à l'écoute d'un enregistrement: le locuteur avait bien de la peine à déployer un récit avec pour seul public le linguiste-enquêteur.)
\trans \textit{\textcolor{brown}{\zh{自己一个人,说不出话来!/ 一个人说话,很难讲下去!(描述一个发音合作人在录音时的困难:如果只有半懂不懂的调查者在听,很难流利地讲话。)}}}
\end{exe}
\begin{exe}
\sn \textcolor{darkblue}{\textbf{\ipa{njɤ˩-tɑ˩kɤ˥}}}
\trans moi seul
\trans \textit{\textcolor{brown}{\zh{我一个人}}}
\end{exe}
\begin{exe}
\sn \textcolor{darkblue}{\textbf{\ipa{no˩-tɑ˩kɤ˥}}}
\trans toi seul
\trans \textit{\textcolor{brown}{\zh{你一个人}}}
\end{exe}
\begin{exe}
\sn \textcolor{darkblue}{\textbf{\ipa{ʈʂʰɯ˧-tɑ˧kɤ˥}}}
\trans lui seul
\trans \textit{\textcolor{brown}{\zh{他一个人}}}
\end{exe}
\begin{exe}
\sn \textcolor{darkblue}{\textbf{\ipa{ʈʂʰɯ˧ ə˧mi˧ | ə˧ɲi˧-tɑ˧kɤ˥ | le˧-ɬi˥!}}}
\trans sa mère n'a eu de congés que pour la journée d'hier! (contexte: on parle d'un petit enfant dont la mère est absente, au travail)
\trans \textit{\textcolor{brown}{\zh{她母亲昨天那天才有假期!(情景:谈一个孩子的母亲,说她前一天才能休息,而当天在上班。)}}}
\end{exe}
\begin{exe}
\sn \textcolor{darkblue}{\textbf{\ipa{ɑ˩ʁo˧, | hĩ˧ ɖɯ˧-v̩˧-tɑ˧kɤ˩ dʑo˩!}}}
\trans Il n'y a qu'une seule personne à la maison!
\trans \textit{\textcolor{brown}{\zh{只有一个人在家!}}}
\end{exe}


\paragraph{\hspace{-0.5cm} {\Large \textcolor{darkblue}{\textbf{\ipa{tɑ˥}}}}} \hypertarget{tA\string_T1}{}
\markboth{\textcolor{darkblue}{\textbf{\ipa{tɑ˥}}}}{}
\textcolor{teal}{\mytextsc{adjectif}} \hspace{4pt} Ton~: H.

Fiable.
\textcolor{brown}{\zh{可靠。}}


\begin{exe}
\sn \textcolor{darkblue}{\textbf{\ipa{le˧-tɑ˥ (| ʐwæ˩˥)}}}
\trans très fiable
\trans \textit{\textcolor{brown}{\zh{很靠谱}}}
\end{exe}
\begin{exe}
\sn \textcolor{darkblue}{\textbf{\ipa{le˧ mɤ˧-tɑ˥ (| ʐwæ˩˥)}}}
\trans pas fiable
\trans \textit{\textcolor{brown}{\zh{不靠谱}}}
\end{exe}
\begin{exe}
\sn \textcolor{darkblue}{\textbf{\ipa{no˧ | le˧-mɤ˧-tɑ˥-hĩ˩ ɖɯ˧-v̩˧ ɲi˩!}}}
\trans Tu es quelqu'un de pas fiable/pas responsable!
\trans \textit{\textcolor{brown}{\zh{你是不靠谱的人!}}}
\end{exe}


\paragraph{\hspace{-0.5cm} {\Large \textcolor{darkblue}{\textbf{\ipa{tɑ˥mo˩}}}}} \hypertarget{tA\string_Tmo\string_B1}{}
\markboth{\textcolor{darkblue}{\textbf{\ipa{tɑ˥mo˩}}}}{}
\textcolor{teal}{\mytextsc{verbe}} \hspace{4pt} Ton~: HL.

Faner.
\textcolor{brown}{\zh{萎、萎蔫。}}


\begin{exe}
\sn \textcolor{darkblue}{\textbf{\ipa{lə˧-tɑ˥mo˩-ze˩!}}}
\trans Ca a fané! (Exemple: une fleur coupée, une fleur abîmée par le vent ou par un soleil trop ardent.)
\trans \textit{\textcolor{brown}{\zh{萎蔫了!}}}
\end{exe}


\paragraph{\hspace{-0.5cm} {\Large \textcolor{darkblue}{\textbf{\ipa{tɑ˧\textsubscript{a}}}}}} \hypertarget{tA\string_Ma1}{}
\markboth{\textcolor{darkblue}{\textbf{\ipa{tɑ˧\textsubscript{a}}}}}{}
\textcolor{teal}{\mytextsc{verbe}} \hspace{4pt} Ton~: M\textsubscript{a}.

Chauffer au feu.
\textcolor{brown}{\zh{烘干。}}


\begin{exe}
\sn \textcolor{darkblue}{\textbf{\ipa{kwɤ˧-kʰɯ˧ tʰi˧-tɑ˧}}}
\trans réchauffer (la nourriture...) auprès du feu
\trans \textit{\textcolor{brown}{\zh{放在火炉旁边热一下(饭)}}}
\end{exe}


\paragraph{\hspace{-0.5cm} {\Large \textcolor{darkblue}{\textbf{\ipa{tɑ˧dzi˩}}}}} \hypertarget{tA\string_Mdzi\string_B1}{}
\markboth{\textcolor{darkblue}{\textbf{\ipa{tɑ˧dzi˩}}}}{}
\textcolor{teal}{\mytextsc{nom}} \hspace{4pt} Ton~: L\#.

Village na en contrebas de Nhissei, en contrehaut de Lataddi (\textcolor{darkblue}{\textbf{\ipa{/lɑ˧tʰɑ˧-di˧˥/}}}).
\textcolor{brown}{\zh{村落名。}}


\begin{exe}
\sn \textcolor{darkblue}{\textbf{\ipa{ɬi˧ki˧, | ɲi˧se˩, | tɑ˧dzi˩, | mv̩˧qʰwæ˩, | lɑ˧tʰɑ˧-di˧˥}}}
\trans Villages dans l'ordre, après la plaine de Yongning, ne comptant pas comme faisant partie de Yongning. Le dernier, \textcolor{darkblue}{\textbf{\ipa{/lɑ˧tʰɑ˧-di˧˥/}}}, désigne toute la région na au-delà du quatrième village.
\trans \textit{\textcolor{brown}{\zh{永宁到泸沽湖所经过的村落,依次是:里格、尼赛、大祖、木垮,然后到拉塔地(拉塔地指的是泸沽湖周边的摩梭地区,包括左所、洛水村等)}}}
\end{exe}


\paragraph{\hspace{-0.5cm} {\Large \textcolor{darkblue}{\textbf{\ipa{tɑ˧gɤ˩}}}}} \hypertarget{tA\string_Mg7\string_B1}{}
\markboth{\textcolor{darkblue}{\textbf{\ipa{tɑ˧gɤ˩}}}}{}
\textcolor{teal}{\mytextsc{adjectif}} \hspace{4pt} Ton~: L\#.

Maigre (personne maigre).
\textcolor{brown}{\zh{瘦弱、枯瘦。}}




\paragraph{\hspace{-0.5cm} {\Large \textcolor{darkblue}{\textbf{\ipa{tɑ˧ho˧}}}}} \hypertarget{tA\string_Mho\string_M1}{}
\markboth{\textcolor{darkblue}{\textbf{\ipa{tɑ˧ho˧}}}}{}
\textcolor{teal}{\mytextsc{adverbe}} \hspace{4pt} Ton~: M.

Ensemble.
\textcolor{brown}{\zh{一起。}}


\begin{exe}
\sn \textcolor{darkblue}{\textbf{\ipa{ɖɯ˧-ʁwɤ˧ tɑ˧ho˧ kʰi˧˥}}}
\trans Tout le village y est allé ensemble.
\trans \textit{\textcolor{brown}{\zh{全村一起去了。}}}
\end{exe}
\begin{exe}
\sn \textcolor{darkblue}{\textbf{\ipa{tɑ˧ho˧ ʝi˧}}}
\trans travailler ensemble
\trans \textit{\textcolor{brown}{\zh{一起工作}}}
\end{exe}
\begin{exe}
\sn \textcolor{darkblue}{\textbf{\ipa{tɑ˧ho˧ tsʰo˧}}}
\trans danser ensemble
\trans \textit{\textcolor{brown}{\zh{一起跳舞}}}
\end{exe}


\paragraph{\hspace{-0.5cm} {\Large \textcolor{darkblue}{\textbf{\ipa{tɑ˧ko˧}}}}} \hypertarget{tA\string_Mko\string_M1}{}
\markboth{\textcolor{darkblue}{\textbf{\ipa{tɑ˧ko˧}}}}{}
\textcolor{teal}{\mytextsc{verbe}} \hspace{4pt} Ton~: M.

Trouver du travail non qualifié (sur un chantier...), gagner de l'argent en faisant des petits boulots.
\textcolor{brown}{\zh{打工(汉语借词)。}}

 Emprunt~: chinois \textcolor{brown}{\zh{打工}}

\begin{exe}
\sn \textcolor{darkblue}{\textbf{\ipa{tɑ˧ko˧ hɯ˧-ze˩!}}}
\trans (Elle/il) est parti(e) gagner de l'argent en faisant des petits boulots!
\trans \textit{\textcolor{brown}{\zh{(他)打工去了!}}}
\end{exe}


\paragraph{\hspace{-0.5cm} {\Large \textcolor{darkblue}{\textbf{\ipa{tɑ˧ko˩}}}}} \hypertarget{tA\string_Mko\string_B1}{}
\markboth{\textcolor{darkblue}{\textbf{\ipa{tɑ˧ko˩}}}}{}
\textcolor{teal}{\mytextsc{verbe}} \hspace{4pt} Ton~: L\#.

Retarder.
\textcolor{brown}{\zh{耽误。}}


\begin{exe}
\sn \textcolor{darkblue}{\textbf{\ipa{hĩ˧ tɑ˧ko˥}}}
\trans retarder les gens
\trans \textit{\textcolor{brown}{\zh{耽误人家}}}
\end{exe}
\begin{exe}
\sn \textcolor{darkblue}{\textbf{\ipa{ʈʂʰɯ˧ hĩ˧ tɑ˧ko˥ | ʐwæ˩˥!}}}
\trans Il/elle retarde tout le monde!
\trans \textit{\textcolor{brown}{\zh{他耽误大家很多!}}}
\end{exe}


\paragraph{\hspace{-0.5cm} {\Large \textcolor{darkblue}{\textbf{\ipa{tɑ˧nɑ˩}}}}} \hypertarget{tA\string_MnA\string_B1}{}
\markboth{\textcolor{darkblue}{\textbf{\ipa{tɑ˧nɑ˩}}}}{}
\textcolor{teal}{\mytextsc{nom}} \hspace{4pt} Ton~: L\#.

Arbalète.
\textcolor{brown}{\zh{弩弓。}}


 \mytextsc{clf}~: \textcolor{darkblue}{\textbf{\ipa{pɤ˩}}} \textcolor{darkblue}{\textbf{\ipa{nɑ˧}}} 

\paragraph{\hspace{-0.5cm} {\Large \textcolor{darkblue}{\textbf{\ipa{tɑ˧pi˧}}}}} \hypertarget{tA\string_Mpi\string_M1}{}
\markboth{\textcolor{darkblue}{\textbf{\ipa{tɑ˧pi˧}}}}{}
\textcolor{teal}{\mytextsc{adjectif}} \hspace{4pt} Ton~: M.

Identique à, pareil à, semblable à, à l'exemple de.
\textcolor{brown}{\zh{如、像、像……那样。}}


\begin{exe}
\sn \textcolor{darkblue}{\textbf{\ipa{no˧-bi˧ tɑ˩pi˩, ...}}}
\trans selon ton exemple; comme toi; identique à toi
\trans \textit{\textcolor{brown}{\zh{像你}}}
\end{exe}
\begin{exe}
\sn \textcolor{darkblue}{\textbf{\ipa{njɤ˧-bi˧ tɑ˩pi˩…}}}
\trans comme moi; à mon exemple
\trans \textit{\textcolor{brown}{\zh{像我}}}
\end{exe}
\begin{exe}
\sn \textcolor{darkblue}{\textbf{\ipa{no˧=ɻ̍˩-bv̩˩, | njɤ˧=ɻ̍˩-bv̩˩, | tɑ˧pi˧!}}}
\trans Le mien et le tien, ils sont faits sur le même exemple =ils sont pareils! (au sujet de bâtiments, par exemple: les maisons d'un même village sont bâties sur le même modèle)
\trans \textit{\textcolor{brown}{\zh{你家的房子,我家的房子,都是一样的!(如:一个村子里的房子,都是按同一个模式建设的。)}}}
\end{exe}
\begin{exe}
\sn \textcolor{darkblue}{\textbf{\ipa{no˧-ɳɯ˧ gv̩˩, | njɤ˧-ɳɯ˧-gv̩˩, | tɑ˧pi˧!}}}
\trans que ce soit toi ou moi qui construise [une maison], c'est pareil! / c'est sur le même modèle!
\trans \textit{\textcolor{brown}{\zh{无论是谁来盖房,盖出来的都一样!}}}
\end{exe}
\begin{exe}
\sn \textcolor{darkblue}{\textbf{\ipa{ʈʂʰɯ˧-bi˩ | tɑ˧pi˧, | njɤ˧-ɳɯ˧ dɑ˧-bi˥-ze˩!}}}
\trans je vais construire [une maison] comme celle-là/ je vais imiter cette maison-là! / Je vais construire une maison qui sera pareille à la sienne!
\trans \textit{\textcolor{brown}{\zh{我要盖跟这一样的房子!}}}
\end{exe}


\paragraph{\hspace{-0.5cm} {\Large \textcolor{darkblue}{\textbf{\ipa{tɑ˧pi˧}}}}} \hypertarget{tA\string_Mpi\string_M1}{}
\markboth{\textcolor{darkblue}{\textbf{\ipa{tɑ˧pi˧}}}}{}
\textcolor{teal}{\mytextsc{verbe}} \hspace{4pt} Ton~: M.

Prendre pour exemple.
\textcolor{brown}{\zh{打比方(汉语借词:当地汉语方言‘打比’)。}}Dialecte chinois local~: \zh{打比。}

 Emprunt~: chinois \textcolor{brown}{\zh{打比}}

\begin{exe}
\sn \textcolor{darkblue}{\textbf{\ipa{tɑ˧pi˧-ze˩}}}
\trans \mytextsc{pfv}
\trans \textit{\textcolor{brown}{\zh{打比方}}}
\end{exe}


\paragraph{\hspace{-0.5cm} {\Large \textcolor{darkblue}{\textbf{\ipa{tɑ˧pv̩˩}}}}} \hypertarget{tA\string_Mpv\string_=\string_B1}{}
\markboth{\textcolor{darkblue}{\textbf{\ipa{tɑ˧pv̩˩}}}}{}
\textcolor{teal}{\mytextsc{adjectif}} \hspace{4pt} Ton~: L\#.

Séché (au soleil) (ex.: légumes).
\textcolor{brown}{\zh{晒干的(水果、蔬菜……)。}}


\begin{exe}
\sn \textcolor{darkblue}{\textbf{\ipa{v̩˩tsʰɤ˧-tɑ˧pv̩˥}}}
\trans légumes séchés au soleil
\trans \textit{\textcolor{brown}{\zh{晒干的蔬菜}}}
\end{exe}


\paragraph{\hspace{-0.5cm} {\Large \textcolor{darkblue}{\textbf{\ipa{tɑ˧pʰi˩}}}}} \hypertarget{tA\string_Mp\string_hi\string_B1}{}
\markboth{\textcolor{darkblue}{\textbf{\ipa{tɑ˧pʰi˩}}}}{}
\textcolor{teal}{\mytextsc{nom}} \hspace{4pt} Ton~: L\#.

Armoise de Chine, \textit{Artemisia argyi}.
\textcolor{brown}{\zh{艾、艾蒿。}}


 \mytextsc{clf}~: \textcolor{darkblue}{\textbf{\ipa{dzi˩}}} 

\paragraph{\hspace{-0.5cm} {\Large \textcolor{darkblue}{\textbf{\ipa{tɑ˧\textasciitilde{}tɑ˧}}} \textsubscript{1}}} \hypertarget{tA\string_M~tA\string_M1}{}
\markboth{\textcolor{darkblue}{\textbf{\ipa{tɑ˧\textasciitilde{}tɑ˧}}} \textsubscript{1}}{}
\textcolor{teal}{\mytextsc{adjectif}} \hspace{4pt} Ton~: M.

Sérieux, attentif, soigneux; exact, précis (une chaussure convient précisément à un pied; quelqu'un observe avec précision/exactitude).
\textcolor{brown}{\zh{严肃认真、细心、细致,(看得)清楚、清晰。}}


\begin{exe}
\sn \textcolor{darkblue}{\textbf{\ipa{mɤ˧-tɑ˧\textasciitilde{}tɑ˧}}}
\trans pas sérieux, négligé (au sujet d'un travail)
\trans \textit{\textcolor{brown}{\zh{邋遢、草率、潦草}}}
\end{exe}
\begin{exe}
\sn \textcolor{darkblue}{\textbf{\ipa{lo˧ ʝi˧ mɤ˧-tɑ˧\textasciitilde{}tɑ˧}}}
\trans travailler sans soin, de façon négligée
\trans \textit{\textcolor{brown}{\zh{工作草率}}}
\end{exe}
\begin{exe}
\sn \textcolor{darkblue}{\textbf{\ipa{hĩ˧ ʈʂʰɯ˧-v̩˧, | tɑ˧\textasciitilde{}tɑ˧!}}}
\trans lui, il est soigneux!
\trans \textit{\textcolor{brown}{\zh{他很认真!}}}
\end{exe}
\textit{Voir~:} \hyperlink{tA\string_M~tA\string_M2}{\textcolor{darkblue}{\textbf{\ipa{tɑ˧~tɑ˧}}} \textsubscript{2}} 

\paragraph{\hspace{-0.5cm} {\Large \textcolor{darkblue}{\textbf{\ipa{tɑ˧\textasciitilde{}tɑ˧}}} \textsubscript{2}}} \hypertarget{tA\string_M~tA\string_M2}{}
\markboth{\textcolor{darkblue}{\textbf{\ipa{tɑ˧\textasciitilde{}tɑ˧}}} \textsubscript{2}}{}
\textcolor{teal}{\mytextsc{adverbe}} \hspace{4pt} Ton~: M.

Précisément, justement, juste à point nommé.
\textcolor{brown}{\zh{刚(好)、正(好)。}}


\begin{exe}
\sn \textcolor{darkblue}{\textbf{\ipa{tɑ˧\textasciitilde{}tɑ˧ | ho˩˥! |}}}
\trans Ca convient exactement/ça convient précisément (ex.: au sujet d'une paire de chaussures qu'on vient de vous offrir)
\trans \textit{\textcolor{brown}{\zh{刚刚好!(如:一双鞋刚好合适)}}}
\end{exe}
\begin{exe}
\sn \textcolor{darkblue}{\textbf{\ipa{le˧-li˧ tɑ˧\textasciitilde{}tɑ˧}}}
\trans voir clairement
\trans \textit{\textcolor{brown}{\zh{看清楚}}}
\end{exe}
\textit{Voir~:} \hyperlink{tA\string_M~tA\string_M1}{\textcolor{darkblue}{\textbf{\ipa{tɑ˧~tɑ˧}}} \textsubscript{1}} 

\paragraph{\hspace{-0.5cm} {\Large \textcolor{darkblue}{\textbf{\ipa{tɑ˩\textsubscript{a}}}}}} \hypertarget{tA\string_Ba1}{}
\markboth{\textcolor{darkblue}{\textbf{\ipa{tɑ˩\textsubscript{a}}}}}{}
\textcolor{teal}{\mytextsc{classificateur}} \hspace{4pt} Ton~: L\textsubscript{a}.

Classificateur des fortes sommes d'argent.
\textcolor{brown}{\zh{量词:钱(一笔)。}}


\begin{exe}
\sn \textcolor{darkblue}{\textbf{\ipa{ɖʐe˧ | ɖɯ˧-tɑ˩}}}
\trans un paquet d'argent, une liasse de billets…
\trans \textit{\textcolor{brown}{\zh{一笔钱}}}
\end{exe}


\paragraph{\hspace{-0.5cm} {\Large \textcolor{darkblue}{\textbf{\ipa{tɑ˩dv̩˧˥}}}}} \hypertarget{tA\string_Bdv\string_=\string_M\string_T1}{}
\markboth{\textcolor{darkblue}{\textbf{\ipa{tɑ˩dv̩˧˥}}}}{}
\textcolor{teal}{\mytextsc{nom}} \hspace{4pt} Ton~: LM+MH\#.

Poche.
\textcolor{brown}{\zh{口袋、衣袋、兜子。}}


\begin{exe}
\sn \textcolor{darkblue}{\textbf{\ipa{njɤ˧ | tɑ˩dv̩˧-qo˥ | tsʰe˩mæ˩-tɑ˥kɤ˩-lɑ˩ dʑo˩!}}}
\trans Je n’ai que dix yuan en poche!
\trans \textit{\textcolor{brown}{\zh{我兜子里只有十元钱!}}}
\end{exe}
 \mytextsc{clf}~: \textcolor{darkblue}{\textbf{\ipa{ɭɯ˧}}} 

\paragraph{\hspace{-0.5cm} {\Large \textcolor{darkblue}{\textbf{\ipa{tɑ˩dʑɤ\#˥}}}}} \hypertarget{tA\string_Bdz£7\#\string_T1}{}
\markboth{\textcolor{darkblue}{\textbf{\ipa{tɑ˩dʑɤ\#˥}}}}{}
\textcolor{teal}{\mytextsc{nom}} \hspace{4pt} Ton~: LM+\#H.

Prénom masculin employé pour le second des jumeaux.
\textcolor{brown}{\zh{男性名字,双胞胎中老二的名字。}}




\paragraph{\hspace{-0.5cm} {\Large \textcolor{darkblue}{\textbf{\ipa{tɑ˩ɖʐo˧dzi˧˥}}}}} \hypertarget{tA\string_Bd`z`o\string_Mdzi\string_M\string_T1}{}
\markboth{\textcolor{darkblue}{\textbf{\ipa{tɑ˩ɖʐo˧dzi˧˥}}}}{}
\textcolor{teal}{\mytextsc{nom}} \hspace{4pt} Ton~: LM+MH\#.

Petit drapeau de prière.
\textcolor{brown}{\zh{小经幡。}}


 \mytextsc{clf}~: \textcolor{darkblue}{\textbf{\ipa{dzi˩}}} 

\paragraph{\hspace{-0.5cm} {\Large \textcolor{darkblue}{\textbf{\ipa{tɑ˩hwɤ˩}}}}} \hypertarget{tA\string_Bhw7\string_B1}{}
\markboth{\textcolor{darkblue}{\textbf{\ipa{tɑ˩hwɤ˩}}}}{}
\textcolor{teal}{\mytextsc{verbe}} \hspace{4pt} Ton~: L.




 Emprunt~: chinois \textcolor{brown}{\zh{打发?}}

\ding{202} 
Offrir des présents à des gens extérieurs à la famille.
\textcolor{brown}{\zh{送礼(给家里以外的人)。}}


\begin{exe}
\sn \textcolor{darkblue}{\textbf{\ipa{hĩ˧-ki˧ | ɖɯ˧-kʰwɤ˧ tɑ˥hwɤ˩-zo˩-ʝi˩!}}}
\trans Il va falloir faire un présent aux gens! / Ca va être l'occasion de faire un présent aux gens! (par exemple à l'occasion d'un rituel de passage à l'âge adulte)
\trans \textit{\textcolor{brown}{\zh{应该给人家送礼了!(例如,人家为孩子进行成年礼时,要送礼。)}}}
\end{exe}
\begin{exe}
\sn \textcolor{darkblue}{\textbf{\ipa{zo˧mv̩˥-ki˩, | tɑ˩hwɤ˩ mɤ˥-kv̩˩!}}}
\trans On ne fait pas de présents aux enfants! (Explication: le présent donné par une famille à une autre de façon ritualisée est offert aux aînés, pas aux enfants; c'est d'une nature différente des petits cadeaux qu'on peut leur faire au quotidien.)
\trans \textit{\textcolor{brown}{\zh{不会专门给孩子送(大)礼的!(说明:送礼,是送给家里的主人)}}}
\end{exe}
\begin{exe}
\sn \textcolor{darkblue}{\textbf{\ipa{ʐɯ˧ tɑ˩hwɤ˩}}}
\trans offrir du vin comme présent
\trans \textit{\textcolor{brown}{\zh{送酒(作为礼物)}}}
\end{exe}
\begin{exe}
\sn \textcolor{darkblue}{\textbf{\ipa{li˩ tɑ˥hwɤ˩}}}
\trans offrir du thé comme présent
\trans \textit{\textcolor{brown}{\zh{送茶(作为礼物)}}}
\end{exe}
\begin{exe}
\sn \textcolor{darkblue}{\textbf{\ipa{dze˧ tɑ˥hwɤ˩}}}
\trans offrir des sucreries/des bonbons comme présent
\trans \textit{\textcolor{brown}{\zh{送糖(作为礼物)}}}
\end{exe}
\ding{203} 
Fournir la dot: donner des biens à une jeune femme lorsqu'elle rejoint sa nouvelle famille lors du mariage. (Note: la dot est apportée à dos de cheval; elle est rangée dans deux caisses en bois: comme en d'autres circonstances, les présents doivent aller par deux.).
\textcolor{brown}{\zh{送陪嫁(嫁妆、陪奁)。}}


\begin{exe}
\sn \textcolor{darkblue}{\textbf{\ipa{ə˧tso˧ tɑ˩hwɤ˩-ʝi˩? | ə˧-sɯ˩kv̩˩ (-dʑo˩), | ɖɯ˧-li˧-ɻ̍˩-bi˩!}}}
\trans Qu'est-ce qu'ils ont donné comme dot? Regardons un peu! / Allons voir! (Ce que disent les villageois invités à un mariage; les biens offerts en dot sont alors exposés, de façon à ce que chacun puisse apprécier la générosité des parents.)
\trans \textit{\textcolor{brown}{\zh{给的是什么嫁妆?咱们去看一看吧!(结婚的时候,陪嫁展示在大家眼前,显示女方家的大方程度)}}}
\end{exe}
\begin{exe}
\sn \textcolor{darkblue}{\textbf{\ipa{ti˧tsɯ˥ | qʰɑ˧-ɭɯ˧ tɑ˩hwɤ˩? - ti˧tsɯ˥ | ɲi˧-ɭɯ˧ tɑ˩hwɤ˩!}}}
\trans Combien de caisses sont offertes en dot/ de combien de caisses la dot se compose-t-elle? - De deux caisses!
\trans \textit{\textcolor{brown}{\zh{陪嫁有几个木箱? - 陪嫁有两个(木箱)!}}}
\end{exe}


\paragraph{\hspace{-0.5cm} {\Large \textcolor{darkblue}{\textbf{\ipa{tɑ˩kɤ˧}}}}} \hypertarget{tA\string_Bk7\string_M1}{}
\markboth{\textcolor{darkblue}{\textbf{\ipa{tɑ˩kɤ˧}}}}{}
\textcolor{teal}{\mytextsc{verbe}} \hspace{4pt} Ton~: .

Taquiner.
\textcolor{brown}{\zh{逗弄(动作)。}}




\paragraph{\hspace{-0.5cm} {\Large \textcolor{darkblue}{\textbf{\ipa{tɑ˩li˥}}}}} \hypertarget{tA\string_Bli\string_T1}{}
\markboth{\textcolor{darkblue}{\textbf{\ipa{tɑ˩li˥}}}}{}
\textcolor{teal}{\mytextsc{nom}} \hspace{4pt} Ton~: LH.

Dali (nom de ville).
\textcolor{brown}{\zh{大理(汉语借词)。}}

 Emprunt~: chinois \textcolor{brown}{\zh{大理}}



\paragraph{\hspace{-0.5cm} {\Large \textcolor{darkblue}{\textbf{\ipa{tɑ˩mv̩˩}}}}} \hypertarget{tA\string_Bmv\string_=\string_B1}{}
\markboth{\textcolor{darkblue}{\textbf{\ipa{tɑ˩mv̩˩}}}}{}
\textcolor{teal}{\mytextsc{verbe}} \hspace{4pt} Ton~: L.

Proverbe.
\textcolor{brown}{\zh{谚语。}}


\begin{exe}
\sn \textcolor{darkblue}{\textbf{\ipa{æ˧ʂæ˧-tɑ˩mv̩˩}}}
\trans même sens: proverbe (littéralement: 'proverbe ancien')
\trans \textit{\textcolor{brown}{\zh{同上:谚语(直译:‘从前的老话’)}}}
\end{exe}
\begin{exe}
\sn \textcolor{darkblue}{\textbf{\ipa{[F5] æ˧ʂæ˧-tɑ˥mv̩˩}}}
\trans proverbe; histoire ancienne
\trans \textit{\textcolor{brown}{\zh{谚语、传统故事}}}
\end{exe}


\paragraph{\hspace{-0.5cm} {\Large \textcolor{darkblue}{\textbf{\ipa{tɑ˩so˩kʰo˥}}}}} \hypertarget{tA\string_Bso\string_Bk\string_ho\string_T1}{}
\markboth{\textcolor{darkblue}{\textbf{\ipa{tɑ˩so˩kʰo˥}}}}{}
\textcolor{teal}{\mytextsc{adverbe}} \hspace{4pt} Ton~: L+H\#.

En tailleur (posture assise).
\textcolor{brown}{\zh{盘腿(而坐)。}}


\begin{exe}
\sn \textcolor{darkblue}{\textbf{\ipa{tɑ˩so˩kʰo˥ | tʰi˧-dzi˩}}}
\trans être assis en tailleur (posture assise des moines, et posture également courante chez les gens du commun)
\trans \textit{\textcolor{brown}{\zh{打坐、盘腿而坐(和尚的坐姿)}}}
\end{exe}


\paragraph{\hspace{-0.5cm} {\Large \textcolor{darkblue}{\textbf{\ipa{tɑ˧˥}}}}} \hypertarget{tA\string_M\string_T1}{}
\markboth{\textcolor{darkblue}{\textbf{\ipa{tɑ˧˥}}}}{}
\textcolor{teal}{\mytextsc{verbe}} \hspace{4pt} Ton~: MH.

Reculer, se retirer.
\textcolor{brown}{\zh{退后。}}


\begin{exe}
\sn \textcolor{darkblue}{\textbf{\ipa{ʁo˧tʰo˩ tɑ˩}}}
\trans reculer, se retirer vers l'arrière
\trans \textit{\textcolor{brown}{\zh{往后退}}}
\end{exe}


\paragraph{\hspace{-0.5cm} {\Large \textcolor{darkblue}{\textbf{\ipa{tɑ˧˥\textsubscript{a}}}}}} \hypertarget{tA\string_M\string_Ta1}{}
\markboth{\textcolor{darkblue}{\textbf{\ipa{tɑ˧˥\textsubscript{a}}}}}{}
\textcolor{teal}{\mytextsc{classificateur}} \hspace{4pt} Ton~: MH\textsubscript{a}.

Entièrement, tout, tout le monde.
\textcolor{brown}{\zh{量词:全部、一切,大家。}}


\begin{exe}
\sn \textcolor{darkblue}{\textbf{\ipa{ɖɯ˧-tɑ˧˥}}}
\trans entièrement, tout, tout le monde
\trans \textit{\textcolor{brown}{\zh{全部、一切,大家}}}
\end{exe}
\begin{exe}
\sn \textcolor{darkblue}{\textbf{\ipa{ɖɯ˧-tɑ˧=ɻæ˥}}}
\trans entièrement, tout, tout le monde (même sens que ci-dessus, avec le morphème de pluriel)
\trans \textit{\textcolor{brown}{\zh{全部、一切,大家(同上,加上多数词素)}}}
\end{exe}


\paragraph{\hspace{-0.5cm} {\Large \textcolor{darkblue}{\textbf{\ipa{tɑ˩˥fv˩˥}}}}} \hypertarget{tA\string_B\string_Tfv\string_B\string_T1}{}
\markboth{\textcolor{darkblue}{\textbf{\ipa{tɑ˩˥fv˩˥}}}}{}
\textcolor{teal}{\mytextsc{nom}} \hspace{4pt} Ton~: LH.LH.

Excrément.
\textcolor{brown}{\zh{大粪(汉语借词)。}}

 Emprunt~: chinois \textcolor{brown}{\zh{大粪}}



\paragraph{\hspace{-0.5cm} {\Large \textcolor{darkblue}{\textbf{\ipa{tæ˧pv̩˩}}}}} \hypertarget{t\{\string_Mpv\string_=\string_B1}{}
\markboth{\textcolor{darkblue}{\textbf{\ipa{tæ˧pv̩˩}}}}{}
\textcolor{teal}{\mytextsc{adjectif}} \hspace{4pt} Ton~: L\#.

Maigre, sec (personne maigre, au corps sec).
\textcolor{brown}{\zh{瘦(人很瘦)。}}


\begin{exe}
\sn \textcolor{darkblue}{\textbf{\ipa{v˩tsʰɤ˧˥ | le˧-tæ˥pv˩ kʰɯ˩}}}
\trans faire sécher des légumes
\trans \textit{\textcolor{brown}{\zh{将蔬菜弄干(晒干)}}}
\end{exe}
\begin{exe}
\sn \textcolor{darkblue}{\textbf{\ipa{v˩tsʰɤ˧˥ | tæ˧pv˩ gv˩}}}
\trans faire sécher des légumes
\trans \textit{\textcolor{brown}{\zh{将蔬菜弄干(晒干)}}}
\end{exe}
\textit{Voir~:} \hyperlink{tA\string_Mg7\string_B1}{\textcolor{darkblue}{\textbf{\ipa{tɑ˧gɤ˩}}}} 

\paragraph{\hspace{-0.5cm} {\Large \textcolor{darkblue}{\textbf{\ipa{tæ˧ɻæ˩}}}}} \hypertarget{t\{\string_Mr£`\{\string_B1}{}
\markboth{\textcolor{darkblue}{\textbf{\ipa{tæ˧ɻæ˩}}}}{}
\textcolor{teal}{\mytextsc{nom}} \hspace{4pt} Ton~: L\#.

Pomme d'Adam; larynx, gorge, oesophage.
\textcolor{brown}{\zh{喉管、喉结。}}


 \mytextsc{clf}~: \textcolor{darkblue}{\textbf{\ipa{ɭɯ˧}}} 

\paragraph{\hspace{-0.5cm} {\Large \textcolor{darkblue}{\textbf{\ipa{ti˧}}}}} \hypertarget{ti\string_M1}{}
\markboth{\textcolor{darkblue}{\textbf{\ipa{ti˧}}}}{}
\textcolor{teal}{\mytextsc{verbe}} \hspace{4pt} Ton~: M.

Mûrir, devenir adulte (d'une personne).
\textcolor{brown}{\zh{成熟(人成熟)。}}


\begin{exe}
\sn \textcolor{darkblue}{\textbf{\ipa{hĩ˧ tʰv̩˧-v̩˧, | gɤ˩-ti˧-ze˧!}}}
\trans Cette personne a grandi / est devenue adulte / a mûri!
\trans \textit{\textcolor{brown}{\zh{这个人,成熟了! / 是大人了!}}}
\end{exe}
\begin{exe}
\sn \textcolor{darkblue}{\textbf{\ipa{hĩ˧ tʰv̩˧-v̩˧, | mɤ˧-ti˧-sɯ˩!}}}
\trans Cette personne n'est pas encore adulte!
\trans \textit{\textcolor{brown}{\zh{这个人,还不成熟!}}}
\end{exe}
\begin{exe}
\sn \textcolor{darkblue}{\textbf{\ipa{zo˩mv̩˧ | gɤ˩-ti˧, | lo˧ hɑ˧!}}}
\trans La croissance d'un enfant (jusqu'à l'âge adulte), c'est pas facile! (La difficulté est pour les parents, et aussi pour l'enfant)
\trans \textit{\textcolor{brown}{\zh{孩子长成熟(的过程),还是挺难的!}}}
\end{exe}


\paragraph{\hspace{-0.5cm} {\Large \textcolor{darkblue}{\textbf{\ipa{ti˧ɖo˥}}}}} \hypertarget{ti\string_Md`o\string_T1}{}
\markboth{\textcolor{darkblue}{\textbf{\ipa{ti˧ɖo˥}}}}{}
\textcolor{teal}{\mytextsc{nom}} \hspace{4pt} Ton~: H\#.

Prénom masculin.
\textcolor{brown}{\zh{男性名字。}}




\paragraph{\hspace{-0.5cm} {\Large \textcolor{darkblue}{\textbf{\ipa{ti˧pʰv̩\#˥}}}}} \hypertarget{ti\string_Mp\string_hv\string_=\#\string_T1}{}
\markboth{\textcolor{darkblue}{\textbf{\ipa{ti˧pʰv̩\#˥}}}}{}
\textcolor{teal}{\mytextsc{nom}} \hspace{4pt} Ton~: \#H.

Coupe de cuivre pour les offrandes; elle est évasée, et de la taille d'un petit gobelet à alcool.
\textcolor{brown}{\zh{铜杯盏,做仪式用的。}}


 \mytextsc{clf}~: \textcolor{darkblue}{\textbf{\ipa{fv̩˩}}} 

\paragraph{\hspace{-0.5cm} {\Large \textcolor{darkblue}{\textbf{\ipa{ti˧tsɯ˥}}}}} \hypertarget{ti\string_MtsM\string_T1}{}
\markboth{\textcolor{darkblue}{\textbf{\ipa{ti˧tsɯ˥}}}}{}
\textcolor{teal}{\mytextsc{nom}} \hspace{4pt} Ton~: H\#.

Boîte en vannerie (objet qui n'est plus en usage aujourd'hui).
\textcolor{brown}{\zh{竹箱。}}


 \mytextsc{clf}~: \textcolor{darkblue}{\textbf{\ipa{ɭɯ˧}}} 

\paragraph{\hspace{-0.5cm} {\Large \textcolor{darkblue}{\textbf{\ipa{ti˧ʈʂʰɯ˩}}}}} \hypertarget{ti\string_Mt`s`\string_hM\string_B1}{}
\markboth{\textcolor{darkblue}{\textbf{\ipa{ti˧ʈʂʰɯ˩}}}}{}
\textcolor{teal}{\mytextsc{nom}} \hspace{4pt} Ton~: L\#.

Marteau.
\textcolor{brown}{\zh{铁锤。}}


 \mytextsc{clf}~: \textcolor{darkblue}{\textbf{\ipa{ɭɯ˧}}} 

\paragraph{\hspace{-0.5cm} {\Large \textcolor{darkblue}{\textbf{\ipa{ti˩\textsubscript{a}}}}}} \hypertarget{ti\string_Ba1}{}
\markboth{\textcolor{darkblue}{\textbf{\ipa{ti˩\textsubscript{a}}}}}{}
\textcolor{teal}{\mytextsc{verbe}} \hspace{4pt} Ton~: L\textsubscript{a}.
\ding{202} 
Piler: réduire quelque chose en poudre dans un mortier, par des coups répétés; comprimer de la terre pour former un mur de terre.
\textcolor{brown}{\zh{捣(花椒、大蒜……)。}}


\begin{exe}
\sn \textcolor{darkblue}{\textbf{\ipa{læ˧tsɯ˥ ti˩}}}
\trans piler le piment, réduire le piment en poudre
\trans \textit{\textcolor{brown}{\zh{捣辣椒}}}
\end{exe}
\begin{exe}
\sn \textcolor{darkblue}{\textbf{\ipa{tsʰo˧ko˧ ti˩}}}
\trans piler la cardamome, réduire de cardamome en poudre
\trans \textit{\textcolor{brown}{\zh{捣草果}}}
\end{exe}
\begin{exe}
\sn \textcolor{darkblue}{\textbf{\ipa{dze˩ ti˥}}}
\trans piler le xanthoxyle, réduire le xanthoxyle en poudre
\trans \textit{\textcolor{brown}{\zh{捣花椒}}}
\end{exe}
\begin{exe}
\sn \textcolor{darkblue}{\textbf{\ipa{ʈʂo˩bo˩ ti˥}}}
\trans construire un mur en terre en comprimant la terre à coups de masse
\trans \textit{\textcolor{brown}{\zh{垒土墙}}}
\end{exe}
\ding{203} 
Donner une tape, tapoter, frapper quelqu'un légèrement.
\textcolor{brown}{\zh{拍打。}}


\begin{exe}
\sn \textcolor{darkblue}{\textbf{\ipa{hĩ˧ ti˥}}}
\trans donner une tape à quelqu'un
\trans \textit{\textcolor{brown}{\zh{拍打人}}}
\end{exe}
\begin{exe}
\sn \textcolor{darkblue}{\textbf{\ipa{hĩ˧ | ɖɯ˧-v̩˧ ti˩-ze˩}}}
\trans (Elle/il) a donné une tape à quelqu'un.
\trans \textit{\textcolor{brown}{\zh{(他)拍打了某人。}}}
\end{exe}


\paragraph{\hspace{-0.5cm} {\Large \textcolor{darkblue}{\textbf{\ipa{ti˩pʰo˩}}}}} \hypertarget{ti\string_Bp\string_ho\string_B1}{}
\markboth{\textcolor{darkblue}{\textbf{\ipa{ti˩pʰo˩}}}}{}
\textcolor{teal}{\mytextsc{nom}} \hspace{4pt} Ton~: L.

Plafond.
\textcolor{brown}{\zh{天花板。}}


 \mytextsc{clf}~: \textcolor{darkblue}{\textbf{\ipa{nɑ˧}}} 

\paragraph{\hspace{-0.5cm} {\Large \textcolor{darkblue}{\textbf{\ipa{ti˩tje˧}}}}} \hypertarget{ti\string_Btje\string_M1}{}
\markboth{\textcolor{darkblue}{\textbf{\ipa{ti˩tje˧}}}}{}
\textcolor{teal}{\mytextsc{verbe}} \hspace{4pt} Ton~: LM.

Traiter (quelqu'un).
\textcolor{brown}{\zh{对待(汉语借词)。}}

 Emprunt~: chinois \textcolor{brown}{\zh{对待}}



\paragraph{\hspace{-0.5cm} {\Large \textcolor{darkblue}{\textbf{\ipa{ti˧˥}}}}} \hypertarget{ti\string_M\string_T1}{}
\markboth{\textcolor{darkblue}{\textbf{\ipa{ti˧˥}}}}{}
\textcolor{teal}{\mytextsc{verbe}} \hspace{4pt} Ton~: MH.

Décider, fixer, arrêter.
\textcolor{brown}{\zh{决定(汉语借词:定)。}}

 Emprunt~: chinois \textcolor{brown}{\zh{定}}



\paragraph{\hspace{-0.5cm} {\Large \textcolor{darkblue}{\textbf{\ipa{ti˧˥\textsubscript{a}}}}}} \hypertarget{ti\string_M\string_Ta1}{}
\markboth{\textcolor{darkblue}{\textbf{\ipa{ti˧˥\textsubscript{a}}}}}{}
\textcolor{teal}{\mytextsc{classificateur}} \hspace{4pt} Ton~: MH\textsubscript{a}.

Couche (de poussière; de planches constituant un plancher; de tissu…).
\textcolor{brown}{\zh{量词:层(一层灰、一层木板……)。}}


\begin{exe}
\sn \textcolor{darkblue}{\textbf{\ipa{dʑɯ˩-nɑ˩mi˩˥ | gv̩˧-ti˩-qo˩ tʰv̩˩}}}
\trans se retrouver au plus profond de la forêt: littéralement ``dans la neuvième couche de forêt/d'alpage" (=au plus profond; on ne compte pas au-delà de la 9e ``couche"; ce décompte est métaphorique, il ne correspond pas à un décompte en étapes à pied, par exemple.
\trans \textit{\textcolor{brown}{\zh{到深山老林的最深处。直译:‘到深山老林的第九层’。这里的‘九’作为最高的数字,表示‘极深’的意思:不能说‘深山老林的第一层’、‘第二层’等。}}}
\end{exe}


\paragraph{\hspace{-0.5cm} {\Large \textcolor{darkblue}{\textbf{\ipa{tjɤ˧hwɑ˧˥}}}}} \hypertarget{tj7\string_MhwA\string_M\string_T1}{}
\markboth{\textcolor{darkblue}{\textbf{\ipa{tjɤ˧hwɑ˧˥}}}}{}
\textcolor{teal}{\mytextsc{nom}} \hspace{4pt} Ton~: MH\#.

Téléphone.
\textcolor{brown}{\zh{电话(汉语借词)。}}

 Emprunt~: chinois \textcolor{brown}{\zh{电话}}

 \mytextsc{clf}~: \textcolor{darkblue}{\textbf{\ipa{ɭɯ˧}}} 

\paragraph{\hspace{-0.5cm} {\Large \textcolor{darkblue}{\textbf{\ipa{tjɤ˧po˧}}}}} \hypertarget{tj7\string_Mpo\string_M1}{}
\markboth{\textcolor{darkblue}{\textbf{\ipa{tjɤ˧po˧}}}}{}
\textcolor{teal}{\mytextsc{nom}} \hspace{4pt} Ton~: M.

Fortin, forteresse.
\textcolor{brown}{\zh{碉堡(汉语借词)。}}

 Emprunt~: chinois \textcolor{brown}{\zh{碉堡}}

 \mytextsc{clf}~: \textcolor{darkblue}{\textbf{\ipa{ɭɯ˧}}} 

\paragraph{\hspace{-0.5cm} {\Large \textcolor{darkblue}{\textbf{\ipa{tjɤ˩˥ʂɯ˧}}}}} \hypertarget{tj7\string_B\string_Ts`M\string_M1}{}
\markboth{\textcolor{darkblue}{\textbf{\ipa{tjɤ˩˥ʂɯ˧}}}}{}
\textcolor{teal}{\mytextsc{nom}} \hspace{4pt} Ton~: LH.M.

Télévision.
\textcolor{brown}{\zh{电视(汉语借词)。}}

 Emprunt~: chinois \textcolor{brown}{\zh{电视}}

\begin{exe}
\sn \textcolor{darkblue}{\textbf{\ipa{tjɤ˩ʂɯ˧ li˥}}}
\trans regarder la télévision
\trans \textit{\textcolor{brown}{\zh{看电视}}}
\end{exe}
\begin{exe}
\sn \textcolor{darkblue}{\textbf{\ipa{tjɤ˩ʂɯ˧-qo˥}}}
\trans à la télévision
\trans \textit{\textcolor{brown}{\zh{电视上}}}
\end{exe}


\paragraph{\hspace{-0.5cm} {\Large \textcolor{darkblue}{\textbf{\ipa{to˥\textsubscript{a}}}}}} \hypertarget{to\string_Ta1}{}
\markboth{\textcolor{darkblue}{\textbf{\ipa{to˥\textsubscript{a}}}}}{}
\textcolor{teal}{\mytextsc{classificateur}} \hspace{4pt} Ton~: H\textsubscript{a}.

Une grande brassée, ce qu'on peut prendre dans les bras: par exemple lors de la récolte: une brassée de riz coupé.
\textcolor{brown}{\zh{量词:抱。}}


\begin{exe}
\sn \textcolor{darkblue}{\textbf{\ipa{qʰv̩˩ɖʐæ˩˥ | ɖɯ˧-to˥}}}
\trans toute une brassée de ficelle/cordelette (voir le récit TraderAndHisSon)
\trans \textit{\textcolor{brown}{\zh{一抱绳子}}}
\end{exe}


\paragraph{\hspace{-0.5cm} {\Large \textcolor{darkblue}{\textbf{\ipa{to˧bɤ\#˥}}}}} \hypertarget{to\string_Mb7\#\string_T1}{}
\markboth{\textcolor{darkblue}{\textbf{\ipa{to˧bɤ\#˥}}}}{}
\textcolor{teal}{\mytextsc{adjectif}} \hspace{4pt} Ton~: \#H.

Vide.
\textcolor{brown}{\zh{空。}}


\begin{exe}
\sn \textcolor{darkblue}{\textbf{\ipa{to˧bɤ˧-ze˩}}}
\trans il n'y a plus rien (ex.: un bol est complètement vidé)
\trans \textit{\textcolor{brown}{\zh{空了}}}
\end{exe}
\begin{exe}
\sn \textcolor{darkblue}{\textbf{\ipa{to˧bɤ˧ ɲi˥}}}
\trans \string_ \mytextsc{cop}: c'est vide
\trans \textit{\textcolor{brown}{\zh{是空的}}}
\end{exe}


\paragraph{\hspace{-0.5cm} {\Large \textcolor{darkblue}{\textbf{\ipa{to˧kɤ\#˥}}}}} \hypertarget{to\string_Mk7\#\string_T1}{}
\markboth{\textcolor{darkblue}{\textbf{\ipa{to˧kɤ\#˥}}}}{}
\textcolor{teal}{\mytextsc{nom}} \hspace{4pt} Ton~: \#H.
\ding{202} 
Front.
\textcolor{brown}{\zh{额头。}}


\ding{203} 
Chance, bonne fortune.
\textcolor{brown}{\zh{运气。}}


\begin{exe}
\sn \textcolor{darkblue}{\textbf{\ipa{to˧kɤ˧ dʑɤ˥}}}
\trans avoir de la chance; avoir un bon karma
\trans \textit{\textcolor{brown}{\zh{好运气,运气好}}}
\end{exe}
\begin{exe}
\sn \textcolor{darkblue}{\textbf{\ipa{njɤ˧ | tsʰi˧ʝi˧ | to˧kɤ˧ dʑjɤ˥ (+ | ʐwæ˩˥)}}}
\trans Cette année, j'ai de la chance!
\trans \textit{\textcolor{brown}{\zh{我今年运气好!}}}
\end{exe}
 \mytextsc{clf}~: \textcolor{darkblue}{\textbf{\ipa{ʈv̩˩}}} 

\paragraph{\hspace{-0.5cm} {\Large \textcolor{darkblue}{\textbf{\ipa{}}}}} \hypertarget{to\string_Mk7\string_Mq\string_h\{\string_Bdi\string_B | -b\{\string_Bb\{\string_B\string_T1}{}
\markboth{\textcolor{darkblue}{\textbf{\ipa{}}}}{}
\textcolor{teal}{\mytextsc{nom}} \hspace{4pt} Ton~: .

Plante à longs filaments.
\textcolor{brown}{\zh{永宁的一种植物。}}


 \mytextsc{clf}~: \textcolor{darkblue}{\textbf{\ipa{bæ˩}}} 

\paragraph{\hspace{-0.5cm} {\Large \textcolor{darkblue}{\textbf{\ipa{to˧qɑ˧}}}}} \hypertarget{to\string_MqA\string_M1}{}
\markboth{\textcolor{darkblue}{\textbf{\ipa{to˧qɑ˧}}}}{}
\textcolor{teal}{\mytextsc{nom}} \hspace{4pt} Ton~: M.

Chevreau.
\textcolor{brown}{\zh{羔羊。}}


 \mytextsc{clf}~: \textcolor{darkblue}{\textbf{\ipa{ɭɯ˧}}} 

\paragraph{\hspace{-0.5cm} {\Large \textcolor{darkblue}{\textbf{\ipa{to˧\textasciitilde{}to˧\textsubscript{b}}}}}} \hypertarget{to\string_M~to\string_Mb1}{}
\markboth{\textcolor{darkblue}{\textbf{\ipa{to˧\textasciitilde{}to˧\textsubscript{b}}}}}{}
\textcolor{teal}{\mytextsc{verbe}} \hspace{4pt} Ton~: M\textsubscript{b}.

Prendre un enfant dans ses bras.
\textcolor{brown}{\zh{抱小孩子、搂,互相拥抱。}}


\begin{exe}
\sn \textcolor{darkblue}{\textbf{\ipa{zo˧mv̩˥ to˩\textasciitilde{}to˩}}}
\trans porter un enfant dans ses bras
\trans \textit{\textcolor{brown}{\zh{抱小孩子}}}
\end{exe}


\paragraph{\hspace{-0.5cm} {\Large \textcolor{darkblue}{\textbf{\ipa{to˩\textsubscript{a}}}} \textsubscript{1}}} \hypertarget{to\string_Ba1}{}
\markboth{\textcolor{darkblue}{\textbf{\ipa{to˩\textsubscript{a}}}} \textsubscript{1}}{}
\textcolor{teal}{\mytextsc{verbe}} \hspace{4pt} Ton~: L\textsubscript{a}.

Lutter, faire de la lutte.
\textcolor{brown}{\zh{摔交。}}


\begin{exe}
\sn \textcolor{darkblue}{\textbf{\ipa{le˧-to˩-ze˩}}}
\trans \mytextsc{accomp} \string_ \mytextsc{pfv}
\trans \textit{\textcolor{brown}{\zh{\mytextsc{accomp} \string_ \mytextsc{pfv}}}}
\end{exe}
\begin{exe}
\sn \textcolor{darkblue}{\textbf{\ipa{ɖʐæ˧\textasciitilde{}ɖʐæ˧ to˩}}}
\trans lutter, faire de la lutte
\trans \textit{\textcolor{brown}{\zh{摔交}}}
\end{exe}


\paragraph{\hspace{-0.5cm} {\Large \textcolor{darkblue}{\textbf{\ipa{to˩\textsubscript{a}}}} \textsubscript{2}}} \hypertarget{to\string_Ba2}{}
\markboth{\textcolor{darkblue}{\textbf{\ipa{to˩\textsubscript{a}}}} \textsubscript{2}}{}
\textcolor{teal}{\mytextsc{verbe}} \hspace{4pt} Ton~: L\textsubscript{a}.

S'allonger.
\textcolor{brown}{\zh{躺下。}}


\begin{exe}
\sn \textcolor{darkblue}{\textbf{\ipa{tʰi˧-to˩-ɕjɤ˩}}}
\trans se reposer en position allongée, s'allonger pour prendre un peu de repos
\trans \textit{\textcolor{brown}{\zh{躺着休息}}}
\end{exe}


\paragraph{\hspace{-0.5cm} {\Large \textcolor{darkblue}{\textbf{\ipa{to˩\textsubscript{a}}}} \textsubscript{3}}} \hypertarget{to\string_Ba3}{}
\markboth{\textcolor{darkblue}{\textbf{\ipa{to˩\textsubscript{a}}}} \textsubscript{3}}{}
\textcolor{teal}{\mytextsc{verbe}} \hspace{4pt} Ton~: L\textsubscript{a}.

Être en relation, entretenir un lien de parenté.
\textcolor{brown}{\zh{有亲属关系。}}


\begin{exe}
\sn \textcolor{darkblue}{\textbf{\ipa{le˧-to˧\textasciitilde{}to˥}}}
\trans \mytextsc{accomp} \string_ \mytextsc{red}
\trans \textit{\textcolor{brown}{\zh{\mytextsc{accomp} \string_ \mytextsc{red}}}}
\end{exe}
\begin{exe}
\sn \textcolor{darkblue}{\textbf{\ipa{qʰwɤ˩ɖɯ˩˥ | le˧-to˩-ze˩}}}
\trans (nous) avons acquis un lien de parenté! (par adoption, mariage...)
\trans \textit{\textcolor{brown}{\zh{我们成了亲戚!(通过领养、婚姻……)}}}
\end{exe}


\paragraph{\hspace{-0.5cm} {\Large \textcolor{darkblue}{\textbf{\ipa{to˩bi\#˥}}}}} \hypertarget{to\string_Bbi\#\string_T1}{}
\markboth{\textcolor{darkblue}{\textbf{\ipa{to˩bi\#˥}}}}{}
\textcolor{teal}{\mytextsc{nom}} \hspace{4pt} Ton~: LM+\#H.

Bouteille.
\textcolor{brown}{\zh{瓶子。}}


 \mytextsc{clf}~: \textcolor{darkblue}{\textbf{\ipa{ɭɯ˧}}} 

\paragraph{\hspace{-0.5cm} {\Large \textcolor{darkblue}{\textbf{\ipa{to˩bi˩}}}}} \hypertarget{to\string_Bbi\string_B1}{}
\markboth{\textcolor{darkblue}{\textbf{\ipa{to˩bi˩}}}}{}
\textcolor{teal}{\mytextsc{classificateur}} \hspace{4pt} Ton~: L.

Auto-classificateur des bouteilles.
\textcolor{brown}{\zh{量词:瓶。}}


\begin{exe}
\sn \textcolor{darkblue}{\textbf{\ipa{ɖɯ˧-to˩bi˩, so˩-to˩bi˩˥, ʐv̩˧-to˥bi˩, qʰv̩˧-to˥bi˩, ʂɯ˧-to˩bi˩, gv̩˧-to˥bi˩, tsʰe˩-to˩bi˩˥}}}
\trans association avec des numéraux, de 1 à 10. Comportement tonal identique pour 1 et 2, 4 et 5, 6 et 8.
\trans \textit{\textcolor{brown}{\zh{与数词结合,一至十}}}
\end{exe}


\paragraph{\hspace{-0.5cm} {\Large \textcolor{darkblue}{\textbf{\ipa{to˩kʰv̩˩mi˥}}}}} \hypertarget{to\string_Bk\string_hv\string_=\string_Bmi\string_T1}{}
\markboth{\textcolor{darkblue}{\textbf{\ipa{to˩kʰv̩˩mi˥}}}}{}
\textcolor{teal}{\mytextsc{nom}} \hspace{4pt} Ton~: L+H\#.

Chien (animal mâle).
\textcolor{brown}{\zh{公狗。}}


 \mytextsc{clf}~: \textcolor{darkblue}{\textbf{\ipa{mi˩}}} \textcolor{darkblue}{\textbf{\ipa{pʰo˧˥}}} 

\paragraph{\hspace{-0.5cm} {\Large \textcolor{darkblue}{\textbf{\ipa{to˩mi˩}}} \textsubscript{1}}} \hypertarget{to\string_Bmi\string_B1}{}
\markboth{\textcolor{darkblue}{\textbf{\ipa{to˩mi˩}}} \textsubscript{1}}{}
\textcolor{teal}{\mytextsc{nom}} \hspace{4pt} Ton~: L.

Pilier.
\textcolor{brown}{\zh{柱子。}}


\begin{exe}
\sn \textcolor{darkblue}{\textbf{\ipa{hæ̃˧ʂɯ˩-to˩mi˩}}}
\trans les Piliers d'Or, les Précieux Piliers: appellation solennelle pour les deux piliers de la maison
\trans \textit{\textcolor{brown}{\zh{‘黄金柱’、‘宝贵柱’:对主屋两个柱子的庄严称呼}}}
\end{exe}
 \mytextsc{clf}~: \textcolor{darkblue}{\textbf{\ipa{ɭɯ˧}}} 

\paragraph{\hspace{-0.5cm} {\Large \textcolor{darkblue}{\textbf{\ipa{to˩mi˩}}} \textsubscript{2}}} \hypertarget{to\string_Bmi\string_B2}{}
\markboth{\textcolor{darkblue}{\textbf{\ipa{to˩mi˩}}} \textsubscript{2}}{}
\textcolor{teal}{\mytextsc{nom}} \hspace{4pt} Ton~: L.

Grande pente (mot attesté, pas combinaison artificielle).
\textcolor{brown}{\zh{大山坡。}}




\paragraph{\hspace{-0.5cm} {\Large \textcolor{darkblue}{\textbf{\ipa{to˩pi˩}}}}} \hypertarget{to\string_Bpi\string_B1}{}
\markboth{\textcolor{darkblue}{\textbf{\ipa{to˩pi˩}}}}{}
\textcolor{teal}{\mytextsc{classificateur}} \hspace{4pt} Ton~: L.

Fois, multiple de.
\textcolor{brown}{\zh{量词:倍(多几倍、少几倍等等)。}}


\begin{exe}
\sn \textcolor{darkblue}{\textbf{\ipa{ɖɯ˧-to˩pi˩, ɲi˧-to˩pi˩, so˩-to˩pi˩˥, ʐv̩˧-to˥pi˩, qʰv̩˧-to˥pi˩, ʂɯ˧-to˩pi˩, gv̩˧-to˥pi˩, tsʰe˩-to˩pi˩˥}}}
\trans association avec des numéraux, de 1 à 10. Comportement tonal identique pour 1 et 2, 4 et 5, 6 et 8.
\trans \textit{\textcolor{brown}{\zh{与数词结合,一至十}}}
\end{exe}


\paragraph{\hspace{-0.5cm} {\Large \textcolor{darkblue}{\textbf{\ipa{to˩pv̩˧}}}}} \hypertarget{to\string_Bpv\string_=\string_M1}{}
\markboth{\textcolor{darkblue}{\textbf{\ipa{to˩pv̩˧}}}}{}
\textcolor{teal}{\mytextsc{adverbe}} \hspace{4pt} Ton~: LM.

Au début, pour commencer.
\textcolor{brown}{\zh{最初。}}

 Emprunt~: tibétain?



\paragraph{\hspace{-0.5cm} {\Large \textcolor{darkblue}{\textbf{\ipa{to˩qo˩lv̩˥}}}}} \hypertarget{to\string_Bqo\string_Blv\string_=\string_T1}{}
\markboth{\textcolor{darkblue}{\textbf{\ipa{to˩qo˩lv̩˥}}}}{}
\textcolor{teal}{\mytextsc{adjectif}} \hspace{4pt} Ton~: L+H\#.

Rond.
\textcolor{brown}{\zh{圆形(球很圆)。}}


\begin{exe}
\sn \textcolor{darkblue}{\textbf{\ipa{to˩qo˩lv̩˥-gv̩˩}}}
\trans rond
\trans \textit{\textcolor{brown}{\zh{圆形}}}
\end{exe}


\paragraph{\hspace{-0.5cm} {\Large \textcolor{darkblue}{\textbf{\ipa{to˩qo˧˥}}}}} \hypertarget{to\string_Bqo\string_M\string_T1}{}
\markboth{\textcolor{darkblue}{\textbf{\ipa{to˩qo˧˥}}}}{}
\textcolor{teal}{\mytextsc{verbe}} \hspace{4pt} Ton~: L+MH\#.

Renverser, verser; à l'envers (ex: renverser le contenu d'une boîte sur la table).
\textcolor{brown}{\zh{倒过来、倒放倒置。}}


\begin{exe}
\sn \textcolor{darkblue}{\textbf{\ipa{to˩qo˧˥ | tɕɯ˧}}}
\trans mettre à l'envers, renverser
\trans \textit{\textcolor{brown}{\zh{倒过来放}}}
\end{exe}
\begin{exe}
\sn \textcolor{darkblue}{\textbf{\ipa{njɤ˧-ɳɯ˧ | to˩qo˧-bi˧!}}}
\trans je vais renverser (ce pot, cette assiette…)
\trans \textit{\textcolor{brown}{\zh{我要(将这个东西)倒过来放!}}}
\end{exe}
\begin{exe}
\sn \textcolor{darkblue}{\textbf{\ipa{to˩qo˧-ze˥}}}
\trans \mytextsc{pfv}
\trans \textit{\textcolor{brown}{\zh{倒过来了}}}
\end{exe}


\paragraph{\hspace{-0.5cm} {\Large \textcolor{darkblue}{\textbf{\ipa{to˩to˧mi˥}}}}} \hypertarget{to\string_Bto\string_Mmi\string_T1}{}
\markboth{\textcolor{darkblue}{\textbf{\ipa{to˩to˧mi˥}}}}{}
\textcolor{teal}{\mytextsc{adverbe}} \hspace{4pt} Ton~: LM+H\#.
\ding{202} 
Soigneusement, attentivement.
\textcolor{brown}{\zh{认真地。}}


\begin{exe}
\sn \textcolor{darkblue}{\textbf{\ipa{njɤ˧-ɳɯ˧ | to˩to˧ mi˥ | ʐwɤ˩-bi˩˥! |}}}
\trans je vais parler soigneusement/je vais bien expliquer!
\trans \textit{\textcolor{brown}{\zh{我要认真地讲!}}}
\end{exe}
\begin{exe}
\sn \textcolor{darkblue}{\textbf{\ipa{to˩to˧-mi˥ | so˩˥}}}
\trans étudier attentivement
\trans \textit{\textcolor{brown}{\zh{认真地学习}}}
\end{exe}
\ding{203} 
Volontairement, délibérément, de propos délibéré: quelqu'un fait exprès de faire quelque chose.
\textcolor{brown}{\zh{故意地。}}




\paragraph{\hspace{-0.5cm} {\Large \textcolor{darkblue}{\textbf{\ipa{to˩ʈɯ˩}}}}} \hypertarget{to\string_Bt`M\string_B1}{}
\markboth{\textcolor{darkblue}{\textbf{\ipa{to˩ʈɯ˩}}}}{}
\textcolor{teal}{\mytextsc{adjectif}} \hspace{4pt} Ton~: L.

Petit (d'un homme).
\textcolor{brown}{\zh{矮。}}


\begin{exe}
\sn \textcolor{darkblue}{\textbf{\ipa{to˩ʈɯ˩\textasciitilde{}ʈɯ˥}}}
\trans petit, de petite taille
\trans \textit{\textcolor{brown}{\zh{矮}}}
\end{exe}


\paragraph{\hspace{-0.5cm} {\Large \textcolor{darkblue}{\textbf{\ipa{to˩zo˩}}}}} \hypertarget{to\string_Bzo\string_B1}{}
\markboth{\textcolor{darkblue}{\textbf{\ipa{to˩zo˩}}}}{}
\textcolor{teal}{\mytextsc{nom}} \hspace{4pt} Ton~: L.

Petite pente (mot attesté, pas combinaison artificielle).
\textcolor{brown}{\zh{小山坡。}}




\paragraph{\hspace{-0.5cm} {\Large \textcolor{darkblue}{\textbf{\ipa{to˩˥}}}}} \hypertarget{to\string_B\string_T1}{}
\markboth{\textcolor{darkblue}{\textbf{\ipa{to˩˥}}}}{}
\textcolor{teal}{\mytextsc{nom}} \hspace{4pt} Ton~: LH.

Pente, versant escarpé de montagne/colline.
\textcolor{brown}{\zh{山坡,岗。}}


\begin{exe}
\sn \textcolor{darkblue}{\textbf{\ipa{to˩ do˩˥}}}
\trans grimper une pente
\trans \textit{\textcolor{brown}{\zh{爬山坡}}}
\end{exe}
\begin{exe}
\sn \textcolor{darkblue}{\textbf{\ipa{ʁwɤ˧-to˩}}}
\trans pente de montagne
\trans \textit{\textcolor{brown}{\zh{山坡}}}
\end{exe}
 \mytextsc{clf}~: \textcolor{darkblue}{\textbf{\ipa{ɭɯ˧}}} objets ronds

\paragraph{\hspace{-0.5cm} {\Large \textcolor{darkblue}{\textbf{\ipa{tv̩˧˥}}} \textsubscript{1}}} \hypertarget{tv\string_=\string_M\string_T1}{}
\markboth{\textcolor{darkblue}{\textbf{\ipa{tv̩˧˥}}} \textsubscript{1}}{}
\textcolor{teal}{\mytextsc{verbe}} \hspace{4pt} Ton~: MH.

Soutenir.
\textcolor{brown}{\zh{搀扶、撑住、稳住。}}




\paragraph{\hspace{-0.5cm} {\Large \textcolor{darkblue}{\textbf{\ipa{tv̩˧˥}}} \textsubscript{2}}} \hypertarget{tv\string_=\string_M\string_T2}{}
\markboth{\textcolor{darkblue}{\textbf{\ipa{tv̩˧˥}}} \textsubscript{2}}{}
\textcolor{teal}{\mytextsc{verbe}} \hspace{4pt} Ton~: MH.

Verser (un liquide) dans la bouche de quelqu'un, faire boire à quelqu'un.
\textcolor{brown}{\zh{喂,喂到嘴里。}}


\begin{exe}
\sn \textcolor{darkblue}{\textbf{\ipa{ʈʂʰæ˧ɣɯ˧ | tʰi˧-tv̩˧˥}}}
\trans verser un médicament dans la bouche de quelqu'un, faire boire un médicament à quelqu'un
\trans \textit{\textcolor{brown}{\zh{喂药}}}
\end{exe}


\paragraph{\hspace{-0.5cm} {\Large \textcolor{darkblue}{\textbf{\ipa{tv̩˧\textsubscript{a}}}}}} \hypertarget{tv\string_=\string_Ma1}{}
\markboth{\textcolor{darkblue}{\textbf{\ipa{tv̩˧\textsubscript{a}}}}}{}
\textcolor{teal}{\mytextsc{verbe}} \hspace{4pt} Ton~: M\textsubscript{a}.

Planter; aussi: repiquer (le riz).
\textcolor{brown}{\zh{耕种、插秧。}}


\begin{exe}
\sn \textcolor{darkblue}{\textbf{\ipa{ɕi˧ tv̩˧}}}
\trans repiquer le riz
\trans \textit{\textcolor{brown}{\zh{插秧}}}
\end{exe}
\begin{exe}
\sn \textcolor{darkblue}{\textbf{\ipa{le˧-tv̩˧-ze˧}}}
\trans \mytextsc{accomp} \string_ \mytextsc{pfv}
\end{exe}
\begin{exe}
\sn \textcolor{darkblue}{\textbf{\ipa{le˧-tv̩˥-tv̩˩-ze˩}}}
\trans \mytextsc{red}
\end{exe}


\paragraph{\hspace{-0.5cm} {\Large \textcolor{darkblue}{\textbf{\ipa{tv̩˧\textsubscript{a}}}} \textsubscript{1}}} \hypertarget{tv\string_=\string_Ma1}{}
\markboth{\textcolor{darkblue}{\textbf{\ipa{tv̩˧\textsubscript{a}}}} \textsubscript{1}}{}
\textcolor{teal}{\mytextsc{classificateur}} \hspace{4pt} Ton~: M\textsubscript{a}.

1.000.
\textcolor{brown}{\zh{千(数词充当量词)。}}


\begin{exe}
\sn \textcolor{darkblue}{\textbf{\ipa{ɖɯ˧-tv̩˧}}}
\trans mille
\trans \textit{\textcolor{brown}{\zh{一千}}}
\end{exe}
\begin{exe}
\sn \textcolor{darkblue}{\textbf{\ipa{ɖɯ˧-tv̩˧ tv̩˧}}}
\trans mille milliers = un million
\trans \textit{\textcolor{brown}{\zh{一千千,等于一百万}}}
\end{exe}
\begin{exe}
\sn \textcolor{darkblue}{\textbf{\ipa{tsʰe˩-tv̩˩ mæ˥}}}
\trans dix mille fois 10.000, soit cent millions
\trans \textit{\textcolor{brown}{\zh{十千万,等于一亿}}}
\end{exe}


\paragraph{\hspace{-0.5cm} {\Large \textcolor{darkblue}{\textbf{\ipa{tv̩˧\textsubscript{a}}}} \textsubscript{2}}} \hypertarget{tv\string_=\string_Ma2}{}
\markboth{\textcolor{darkblue}{\textbf{\ipa{tv̩˧\textsubscript{a}}}} \textsubscript{2}}{}
\textcolor{teal}{\mytextsc{classificateur}} \hspace{4pt} Ton~: M\textsubscript{a}.

Dixième d'unité monétaire.
\textcolor{brown}{\zh{量词:角(钱),一元的十分之一。}}




\paragraph{\hspace{-0.5cm} {\Large \textcolor{darkblue}{\textbf{\ipa{tv̩˧ɕi˩}}}}} \hypertarget{tv\string_=\string_Ms£i\string_B1}{}
\markboth{\textcolor{darkblue}{\textbf{\ipa{tv̩˧ɕi˩}}}}{}
\textcolor{teal}{\mytextsc{nom}} \hspace{4pt} Ton~: L\#.

Millepattes.
\textcolor{brown}{\zh{蜈蚣。}}


 \mytextsc{clf}~: \textcolor{darkblue}{\textbf{\ipa{mi˩}}} 

\paragraph{\hspace{-0.5cm} {\Large \textcolor{darkblue}{\textbf{\ipa{tv̩˩ɭɯ˧˥}}}}} \hypertarget{tv\string_=\string_Bl\string_RM\string_M\string_T1}{}
\markboth{\textcolor{darkblue}{\textbf{\ipa{tv̩˩ɭɯ˧˥}}}}{}
\textcolor{teal}{\mytextsc{nom}} \hspace{4pt} Ton~: LM+MH\#.

Hotte de grande qualité, dans laquelle on offrait des cadeaux; n'existe plus actuellement; était resserrée au milieu: de forme concave, pas convexe.
\textcolor{brown}{\zh{高级的背篓,过去用它放礼物。}}


 \mytextsc{clf}~: \textcolor{darkblue}{\textbf{\ipa{ɭɯ˧}}} 

\paragraph{\hspace{-0.5cm} {\Large \textcolor{darkblue}{\textbf{\ipa{tv̩˧po˩}}}}} \hypertarget{tv\string_=\string_Mpo\string_B1}{}
\markboth{\textcolor{darkblue}{\textbf{\ipa{tv̩˧po˩}}}}{}
\textcolor{teal}{\mytextsc{verbe}} \hspace{4pt} Ton~: L\#.

Parier, jouer à des jeux d'argent.
\textcolor{brown}{\zh{赌博(汉语借词)。}}

 Emprunt~: chinois \textcolor{brown}{\zh{赌博}}



\paragraph{\hspace{-0.5cm} {\Large \textcolor{darkblue}{\textbf{\ipa{tv̩˧qʰv̩˧}}}}} \hypertarget{tv\string_=\string_Mq\string_hv\string_=\string_M1}{}
\markboth{\textcolor{darkblue}{\textbf{\ipa{tv̩˧qʰv̩˧}}}}{}
\textcolor{teal}{\mytextsc{nom}} \hspace{4pt} Ton~: M.

Tombe provisoire, où on place le corps du défunt avant la crémation.
\textcolor{brown}{\zh{临时坟墓。}}




\paragraph{\hspace{-0.5cm} {\Large \textcolor{darkblue}{\textbf{\ipa{tv̩˧tsʰɯ˧}}}}} \hypertarget{tv\string_=\string_Mts\string_hM\string_M1}{}
\markboth{\textcolor{darkblue}{\textbf{\ipa{tv̩˧tsʰɯ˧}}}}{}
\textcolor{teal}{\mytextsc{nom}} \hspace{4pt} Ton~: M.
\ding{202} 
Temps.
\textcolor{brown}{\zh{时间。}}


\begin{exe}
\sn \textcolor{darkblue}{\textbf{\ipa{[F5] njɤ˧ | tv̩˧tsʰɯ˧ mɤ˧-dʑo˧.}}}
\trans Je n'ai pas le temps.
\trans \textit{\textcolor{brown}{\zh{我没时间。}}}
\end{exe}
\begin{exe}
\sn \textcolor{darkblue}{\textbf{\ipa{[F5] njɤ˧ | tv̩˧tsʰɯ˧ dʑo˧.}}}
\trans J'ai du temps libre. / J'ai le temps.
\trans \textit{\textcolor{brown}{\zh{我有时间。}}}
\end{exe}
\ding{203} 
Période de temps, heure.
\textcolor{brown}{\zh{时间段、小时。}}


\begin{exe}
\sn \textcolor{darkblue}{\textbf{\ipa{tv̩˧tsʰɯ˧ | ɖɯ˧-ɭɯ˧}}}
\trans une heure
\trans \textit{\textcolor{brown}{\zh{一个小时}}}
\end{exe}
\begin{exe}
\sn \textcolor{darkblue}{\textbf{\ipa{tv̩˧tsʰɯ˧ ɖɯ˧-ɭɯ˧ gv̩˧-ze˧!}}}
\trans Une heure a passé.
\trans \textit{\textcolor{brown}{\zh{一个小时过去了。}}}
\end{exe}
\begin{exe}
\sn \textcolor{darkblue}{\textbf{\ipa{tv̩˧tsʰɯ˧ ɖɯ˧-ɭɯ˧ le˧-hɯ˩-ze˩.}}}
\trans Une heure s'est écoulée.
\trans \textit{\textcolor{brown}{\zh{一个小时过去了。}}}
\end{exe}
\begin{exe}
\sn \textcolor{darkblue}{\textbf{\ipa{[F5] tv̩˧tsʰɯ˧ qʰɑ˧-ɭɯ˧?}}}
\trans Quelle heure est-il?
\trans \textit{\textcolor{brown}{\zh{几点了?}}}
\end{exe}
 \mytextsc{clf}~: \textcolor{darkblue}{\textbf{\ipa{ɭɯ˧}}} 

\paragraph{\hspace{-0.5cm} {\Large \textcolor{darkblue}{\textbf{\ipa{tv̩˧tsʰɯ˧li˧di˩}}}}} \hypertarget{tv\string_=\string_Mts\string_hM\string_Mli\string_Mdi\string_B1}{}
\markboth{\textcolor{darkblue}{\textbf{\ipa{tv̩˧tsʰɯ˧li˧di˩}}}}{}
\textcolor{teal}{\mytextsc{nom}} \hspace{4pt} Ton~: .

Horloge; montre. Littéralement: ``chose pour voir l'heure".
\textcolor{brown}{\zh{钟、手表。}}




\paragraph{\hspace{-0.5cm} {\Large \textcolor{darkblue}{\textbf{\ipa{tv̩˩tv̩˩}}}}} \hypertarget{tv\string_=\string_Btv\string_=\string_B1}{}
\markboth{\textcolor{darkblue}{\textbf{\ipa{tv̩˩tv̩˩}}}}{}
\textcolor{teal}{\mytextsc{adjectif}} \hspace{4pt} Ton~: L.
\ding{202} 
Droit, bien d'aplomb.
\textcolor{brown}{\zh{直,笔直的(如:站直)。}}


\ding{203} 
Droit, intègre, honnête.
\textcolor{brown}{\zh{耿直。}}




\paragraph{\hspace{-0.5cm} {\Large \textcolor{darkblue}{\textbf{\ipa{tv̩˧tv̩˥}}}}} \hypertarget{tv\string_=\string_Mtv\string_=\string_T1}{}
\markboth{\textcolor{darkblue}{\textbf{\ipa{tv̩˧tv̩˥}}}}{}
\textcolor{teal}{\mytextsc{nom}} \hspace{4pt} Ton~: H\#.

Chapeau.
\textcolor{brown}{\zh{帽子。}}


 \mytextsc{clf}~: \textcolor{darkblue}{\textbf{\ipa{ɭɯ˧}}} 

\newpage
\section*{\centering- \textcolor{darkblue}{\textbf{\ipa{tʰ}}} -}
\paragraph{\hspace{-0.5cm} {\Large \textcolor{darkblue}{\textbf{\ipa{}}}}} \hypertarget{t\string_hA\string_M‑1}{}
\markboth{\textcolor{darkblue}{\textbf{\ipa{}}}}{}
\textcolor{teal}{\mytextsc{préfixe}} \hspace{4pt} Ton~: M/0.

Prohibitif.
\textcolor{brown}{\zh{不要、别\mytextsc{禁止式。}}}


\begin{exe}
\sn \textcolor{darkblue}{\textbf{\ipa{tʰɑ˧-lɑ˩\textasciitilde{}lɑ˩-ze˩!}}}
\trans Arrêtez de vous disputer! / Ne vous disputez pas!
\trans \textit{\textcolor{brown}{\zh{别吵架了!}}}
\end{exe}
\begin{exe}
\sn \textcolor{darkblue}{\textbf{\ipa{tʰɑ˧-dzo˧\textasciitilde{}dzo˥!}}}
\trans Ne pas toucher! / Ne touchez pas!
\trans \textit{\textcolor{brown}{\zh{不要动来动去! / 不要碰来碰去!}}}
\end{exe}


\paragraph{\hspace{-0.5cm} {\Large \textcolor{darkblue}{\textbf{\ipa{tʰɑ˧-ni˥-ni˩-gv̩˩}}}}} \hypertarget{t\string_hA\string_M-ni\string_T-ni\string_B-gv\string_=\string_B1}{}
\markboth{\textcolor{darkblue}{\textbf{\ipa{tʰɑ˧-ni˥-ni˩-gv̩˩}}}}{}
\textcolor{teal}{\mytextsc{adverbe}} \hspace{4pt} Ton~: .

Comparable à, semblable à, similaire à.
\textcolor{brown}{\zh{类似,相近。}}


\textit{Voir~:} \textcolor{darkblue}{\textbf{\ipa{-ni˧gv̩˧˥}}} 

\paragraph{\hspace{-0.5cm} {\Large \textcolor{darkblue}{\textbf{\ipa{tʰɑ˧v̩˥}}}}} \hypertarget{t\string_hA\string_Mv\string_=\string_T1}{}
\markboth{\textcolor{darkblue}{\textbf{\ipa{tʰɑ˧v̩˥}}}}{}
\textcolor{teal}{\mytextsc{nom}} \hspace{4pt} Ton~: H\#.

Chambre des invités (emprunt au chinois local; sens en chinois standard: pièce centrale, salle de séjour). Il n'y a pas d'équivalent direct dans la maison na traditionnelle: c'est dans la resserre qu'on pouvait improviser une chambre supplémentaire.
\textcolor{brown}{\zh{堂屋(汉语借词),来指客房。}}


 \mytextsc{clf}~: \textcolor{darkblue}{\textbf{\ipa{ɭɯ˧}}} 

\paragraph{\hspace{-0.5cm} {\Large \textcolor{darkblue}{\textbf{\ipa{tʰɑ˩lo˧}}}}} \hypertarget{t\string_hA\string_Blo\string_M1}{}
\markboth{\textcolor{darkblue}{\textbf{\ipa{tʰɑ˩lo˧}}}}{}
\textcolor{teal}{\mytextsc{nom}} \hspace{4pt} Ton~: LM.

Prononciation par les Na de thar lam, nom anciennement donné par les Tibétains à la plaine de Yongning.
\textcolor{brown}{\zh{永宁的藏语名称。}}

 Emprunt~: tibétain tharlam

\begin{exe}
\sn \textcolor{darkblue}{\textbf{\ipa{tʰɑ˩lo˧-go˧bɤ˩}}}
\trans le temple de Thar lam =le temple de Yongning, tel que l'appellent les Tibétains
\trans \textit{\textcolor{brown}{\zh{永宁大寺}}}
\end{exe}
\begin{exe}
\sn \textcolor{darkblue}{\textbf{\ipa{tʰɑ˩lo˧ se˧gi˧ kɤ˩mv̩˩}}}
\trans la montagne Gemu de Yongning
\trans \textit{\textcolor{brown}{\zh{永宁格姆山}}}
\end{exe}


\paragraph{\hspace{-0.5cm} {\Large \textcolor{darkblue}{\textbf{\ipa{tʰɑ˩mi\#˥}}}}} \hypertarget{t\string_hA\string_Bmi\#\string_T1}{}
\markboth{\textcolor{darkblue}{\textbf{\ipa{tʰɑ˩mi\#˥}}}}{}
\textcolor{teal}{\mytextsc{nom}} \hspace{4pt} Ton~: LM+\#H.

Buffle femelle.
\textcolor{brown}{\zh{母水牛。}}


\begin{exe}
\sn \textcolor{darkblue}{\textbf{\ipa{dʑi˧mi˧-tʰɑ˩mi˩}}}
\trans même sens: buffle femelle
\trans \textit{\textcolor{brown}{\zh{母水牛}}}
\end{exe}
\begin{exe}
\sn \textcolor{darkblue}{\textbf{\ipa{dʑi˧mi˧ ʈʂʰɯ˧-pʰo˩ dʑo˩, | tʰɑ˩mi˧ ɲi˥!}}}
\trans ce buffle, c'est une femelle!
\trans \textit{\textcolor{brown}{\zh{这头水牛是母的!}}}
\end{exe}
 \mytextsc{clf}~: \textcolor{darkblue}{\textbf{\ipa{pʰo˧˥}}} 

\paragraph{\hspace{-0.5cm} {\Large \textcolor{darkblue}{\textbf{\ipa{tʰɑ˩pʰv̩\#˥}}}}} \hypertarget{t\string_hA\string_Bp\string_hv\string_=\#\string_T1}{}
\markboth{\textcolor{darkblue}{\textbf{\ipa{tʰɑ˩pʰv̩\#˥}}}}{}
\textcolor{teal}{\mytextsc{nom}} \hspace{4pt} Ton~: LM+\#H.

Buffle mâle.
\textcolor{brown}{\zh{公水牛。}}


 \mytextsc{clf}~: \textcolor{darkblue}{\textbf{\ipa{pʰo˧˥}}} 

\paragraph{\hspace{-0.5cm} {\Large \textcolor{darkblue}{\textbf{\ipa{tʰɑ˩tʰɑ˩}}}}} \hypertarget{t\string_hA\string_Bt\string_hA\string_B1}{}
\markboth{\textcolor{darkblue}{\textbf{\ipa{tʰɑ˩tʰɑ˩}}}}{}
\textcolor{teal}{\mytextsc{nom}} \hspace{4pt} Ton~: L.

Bon coin pour la cueillette de champignons, de plantes sauvages...
\textcolor{brown}{\zh{采野生植物如菌子等的好地方。}}


\begin{exe}
\sn \textcolor{darkblue}{\textbf{\ipa{tʰɑ˩tʰɑ˩˥ | ɖɯ˧-kʰwɤ˥}}}
\trans un bon coin (pour la cueillette)
\trans \textit{\textcolor{brown}{\zh{一个好地方}}}
\end{exe}
 \mytextsc{clf}~: \textcolor{darkblue}{\textbf{\ipa{kʰwɤ˥}}} 

\paragraph{\hspace{-0.5cm} {\Large \textcolor{darkblue}{\textbf{\ipa{tʰɑ˩zo\#˥}}}}} \hypertarget{t\string_hA\string_Bzo\#\string_T1}{}
\markboth{\textcolor{darkblue}{\textbf{\ipa{tʰɑ˩zo\#˥}}}}{}
\textcolor{teal}{\mytextsc{nom}} \hspace{4pt} Ton~: LM+\#H.

Petit buffle, enfant du buffle.
\textcolor{brown}{\zh{小水牛。}}


 \mytextsc{clf}~: \textcolor{darkblue}{\textbf{\ipa{mi˩}}} 

\paragraph{\hspace{-0.5cm} {\Large \textcolor{darkblue}{\textbf{\ipa{tʰɑ˩-ʐwæ˧mi˧}}}}} \hypertarget{t\string_hA\string_B-z`w\{\string_Mmi\string_M1}{}
\markboth{\textcolor{darkblue}{\textbf{\ipa{tʰɑ˩-ʐwæ˧mi˧}}}}{}
\textcolor{teal}{\mytextsc{nom}} \hspace{4pt} Ton~: L-.

Âne (mâle ou femelle).
\textcolor{brown}{\zh{驴子。}}


 \mytextsc{clf}~: \textcolor{darkblue}{\textbf{\ipa{pʰo˧˥}}} 

\paragraph{\hspace{-0.5cm} {\Large \textcolor{darkblue}{\textbf{\ipa{tʰɑ˧˥}}} \textsubscript{1}}} \hypertarget{t\string_hA\string_M\string_T1}{}
\markboth{\textcolor{darkblue}{\textbf{\ipa{tʰɑ˧˥}}} \textsubscript{1}}{}
\textcolor{teal}{\mytextsc{adjectif}} \hspace{4pt} Ton~: MH.

Aiguisé, qui coupe bien, affûté.
\textcolor{brown}{\zh{锋利。}}




\paragraph{\hspace{-0.5cm} {\Large \textcolor{darkblue}{\textbf{\ipa{tʰɑ˧˥}}} \textsubscript{2}}} \hypertarget{t\string_hA\string_M\string_T2}{}
\markboth{\textcolor{darkblue}{\textbf{\ipa{tʰɑ˧˥}}} \textsubscript{2}}{}
\textcolor{teal}{\mytextsc{verbe}} \hspace{4pt} Ton~: MH.

Être possible, être autorisé: \mytextsc{permissif}.
\textcolor{brown}{\zh{可以,允许。}}


\begin{exe}
\sn \textcolor{darkblue}{\textbf{\ipa{mɤ˧-tʰɑ˧˥ | mɤ˧-ʐv̩˩! | njɤ˧ | dzɯ˧-bi˧ni˧-mɤ˧-gv̩˧˥!}}}
\trans Ce n'est pas vraiment bon! Je n'aime pas en manger!
\trans \textit{\textcolor{brown}{\zh{不怎么好吃!我不喜欢吃!}}}
\end{exe}
\begin{exe}
\sn \textcolor{darkblue}{\textbf{\ipa{[Mushrooms] ə˩ljɤ˩hæ̃˩ʂɯ˥-mo˩-ʈʂʰɯ˩-dʑo˩, | hĩ˧ | mɤ˧-tʰɑ˧˥ | dv̩˩-mɤ˧-kv̩˧˥!}}}
\trans Le Champignon Doré, il n'est pas si vénéneux que ça!
\trans \textit{\textcolor{brown}{\zh{黄蜡伞不怎么会让人中毒!/ 毒性不太大!}}}
\end{exe}


\paragraph{\hspace{-0.5cm} {\Large \textcolor{darkblue}{\textbf{\ipa{tʰɑ˩˥}}}}} \hypertarget{t\string_hA\string_B\string_T1}{}
\markboth{\textcolor{darkblue}{\textbf{\ipa{tʰɑ˩˥}}}}{}
\textcolor{teal}{\mytextsc{nom}} \hspace{4pt} Ton~: LH.

Buffle; forme monosyllabique, qui n'est pas utilisée telle quelle, seulement dans des formes telles que \textcolor{brown}{\zh{/tʰɑ˩mi\#˥/}} 'buffle femelle'.
\textcolor{brown}{\zh{水牛。}}


 \mytextsc{clf}~: \textcolor{darkblue}{\textbf{\ipa{pʰo˧˥}}} 

\paragraph{\hspace{-0.5cm} {\Large \textcolor{darkblue}{\textbf{\ipa{tʰæ˧ɻæ˩}}}}} \hypertarget{t\string_h\{\string_Mr£`\{\string_B1}{}
\markboth{\textcolor{darkblue}{\textbf{\ipa{tʰæ˧ɻæ˩}}}}{}
\textcolor{teal}{\mytextsc{nom}} \hspace{4pt} Ton~: L\#.

Livre.
\textcolor{brown}{\zh{书。}}


 \mytextsc{clf}~: \textcolor{darkblue}{\textbf{\ipa{pɤ˩}}} 

\paragraph{\hspace{-0.5cm} {\Large \textcolor{darkblue}{\textbf{\ipa{tʰæ˩tsɯ˧}}}}} \hypertarget{t\string_h\{\string_BtsM\string_M1}{}
\markboth{\textcolor{darkblue}{\textbf{\ipa{tʰæ˩tsɯ˧}}}}{}
\textcolor{teal}{\mytextsc{nom}} \hspace{4pt} Ton~: LM.

Jarre.
\textcolor{brown}{\zh{坛子(汉语借词)。}}

 Emprunt~: chinois \textcolor{brown}{\zh{坛子}}



\paragraph{\hspace{-0.5cm} {\Large \textcolor{darkblue}{\textbf{\ipa{}}}}} \hypertarget{t\string_hi\string_M‑1}{}
\markboth{\textcolor{darkblue}{\textbf{\ipa{}}}}{}
\textcolor{teal}{\mytextsc{préfixe}} \hspace{4pt} Ton~: M/0.

Duratif (\mytextsc{dur}).
\textcolor{brown}{\zh{持续体。}}


\begin{exe}
\sn \textcolor{darkblue}{\textbf{\ipa{tʰi˧-dzɯ˥-dʑo˩!}}}
\trans (Elle) est en train de manger! / Elle mange! (Contexte: on constate avec joie qu'un enfant qui ne mangeait plus depuis deux jours est en train de ronger à belles dents un épi de maïs.)
\trans \textit{\textcolor{brown}{\zh{她在吃东西!}}}
\end{exe}
\begin{exe}
\sn \textcolor{darkblue}{\textbf{\ipa{tʰi˧-mɤ˧-ɲi˥}}}
\trans faute de quoi
\trans \textit{\textcolor{brown}{\zh{否则、要不然}}}
\end{exe}


\paragraph{\hspace{-0.5cm} {\Large \textcolor{darkblue}{\textbf{\ipa{tʰi˧}}}}} \hypertarget{t\string_hi\string_M1}{}
\markboth{\textcolor{darkblue}{\textbf{\ipa{tʰi˧}}}}{}
\textcolor{teal}{\mytextsc{adjectif}} \hspace{4pt} Ton~: M.

Compétent, habile.
\textcolor{brown}{\zh{能干。}}


\begin{exe}
\sn \textcolor{darkblue}{\textbf{\ipa{ɖwæ˧˥ | tʰi˧}}}
\trans \mytextsc{intensif}.très: très habile
\trans \textit{\textcolor{brown}{\zh{很能干}}}
\end{exe}
\begin{exe}
\sn \textcolor{darkblue}{\textbf{\ipa{mv̩˩tʰi˩ tʰv̩˩-v̩˩˥}}}
\trans cette femme intelligente
\trans \textit{\textcolor{brown}{\zh{那个聪明女人}}}
\end{exe}
\begin{exe}
\sn \textcolor{darkblue}{\textbf{\ipa{zo˧tʰi˧}}}
\trans homme intelligent
\trans \textit{\textcolor{brown}{\zh{聪明男人}}}
\end{exe}


\paragraph{\hspace{-0.5cm} {\Large \textcolor{darkblue}{\textbf{\ipa{tʰi˩\textsubscript{a}}}}}} \hypertarget{t\string_hi\string_Ba1}{}
\markboth{\textcolor{darkblue}{\textbf{\ipa{tʰi˩\textsubscript{a}}}}}{}
\textcolor{teal}{\mytextsc{verbe}} \hspace{4pt} Ton~: L\textsubscript{a}.

Raboter.
\textcolor{brown}{\zh{刨。}}


\begin{exe}
\sn \textcolor{darkblue}{\textbf{\ipa{tso˧\textasciitilde{}tso˧ tʰi˥(-ze˩)}}}
\trans raboter quelque chose
\trans \textit{\textcolor{brown}{\zh{刨东西}}}
\end{exe}
\begin{exe}
\sn \textcolor{darkblue}{\textbf{\ipa{le˧-tʰi˩-ze˩}}}
\trans \mytextsc{accomp} \string_ \mytextsc{pfv}
\trans \textit{\textcolor{brown}{\zh{刨了}}}
\end{exe}
\begin{exe}
\sn \textcolor{darkblue}{\textbf{\ipa{tso˧\textasciitilde{}tso˧ | le˧-tʰi˩(-ze˩)}}}
\trans raboter quelque chose
\trans \textit{\textcolor{brown}{\zh{刨东西}}}
\end{exe}
\begin{exe}
\sn \textcolor{darkblue}{\textbf{\ipa{pæ˩pʰæ˧ tʰi˥}}}
\trans raboter une planche
\trans \textit{\textcolor{brown}{\zh{刨木板}}}
\end{exe}


\paragraph{\hspace{-0.5cm} {\Large \textcolor{darkblue}{\textbf{\ipa{tʰi˩mi\#˥}}}}} \hypertarget{t\string_hi\string_Bmi\#\string_T1}{}
\markboth{\textcolor{darkblue}{\textbf{\ipa{tʰi˩mi\#˥}}}}{}
\textcolor{teal}{\mytextsc{nom}} \hspace{4pt} Ton~: LM+\#H.

Grand rabot.
\textcolor{brown}{\zh{大刨。}}


 \mytextsc{clf}~: \textcolor{darkblue}{\textbf{\ipa{nɑ˧}}} 

\paragraph{\hspace{-0.5cm} {\Large \textcolor{darkblue}{\textbf{\ipa{tʰi˩zo\#˥}}}}} \hypertarget{t\string_hi\string_Bzo\#\string_T1}{}
\markboth{\textcolor{darkblue}{\textbf{\ipa{tʰi˩zo\#˥}}}}{}
\textcolor{teal}{\mytextsc{nom}} \hspace{4pt} Ton~: LM+\#H.

Petit rabot.
\textcolor{brown}{\zh{小刨。}}


 \mytextsc{clf}~: \textcolor{darkblue}{\textbf{\ipa{nɑ˧}}} 

\paragraph{\hspace{-0.5cm} {\Large \textcolor{darkblue}{\textbf{\ipa{tʰi˩˥}}} \textsubscript{1}}} \hypertarget{t\string_hi\string_B\string_T1}{}
\markboth{\textcolor{darkblue}{\textbf{\ipa{tʰi˩˥}}} \textsubscript{1}}{}
\textcolor{teal}{\mytextsc{nom}} \hspace{4pt} Ton~: LH.

Rabot.
\textcolor{brown}{\zh{刨。}}


 \mytextsc{clf}~: \textcolor{darkblue}{\textbf{\ipa{nɑ˧}}} 

\paragraph{\hspace{-0.5cm} {\Large \textcolor{darkblue}{\textbf{\ipa{tʰi˩˥}}} \textsubscript{2}}} \hypertarget{t\string_hi\string_B\string_T2}{}
\markboth{\textcolor{darkblue}{\textbf{\ipa{tʰi˩˥}}} \textsubscript{2}}{}
\textcolor{teal}{\mytextsc{particule}} \textcolor{teal}{\mytextsc{de}} \textcolor{teal}{\mytextsc{discours}} \hspace{4pt} Ton~: LM? LH?.

Particule de discours: alors, donc, après.
\textcolor{brown}{\zh{然后。}}




\paragraph{\hspace{-0.5cm} {\Large \textcolor{darkblue}{\textbf{\ipa{tʰo˥\textsubscript{a}}}}}} \hypertarget{t\string_ho\string_Ta1}{}
\markboth{\textcolor{darkblue}{\textbf{\ipa{tʰo˥\textsubscript{a}}}}}{}
\textcolor{teal}{\mytextsc{classificateur}} \hspace{4pt} Ton~: H\textsubscript{a}.

Classificateur des solutions / issues heureuses.
\textcolor{brown}{\zh{量词:办法,解决的方法(一个)。}}


\begin{exe}
\sn \textcolor{darkblue}{\textbf{\ipa{ə˧tso˧ tʰo˧ dʑo˧-kv̩˩?}}}
\trans Qu'est-ce qu'on y peut? / Qu'est-ce qu'on peut y faire?
\trans \textit{\textcolor{brown}{\zh{有什么办法?}}}
\end{exe}


\paragraph{\hspace{-0.5cm} {\Large \textcolor{darkblue}{\textbf{\ipa{tʰo˥\textsubscript{a}}}}}} \hypertarget{t\string_ho\string_Ta1}{}
\markboth{\textcolor{darkblue}{\textbf{\ipa{tʰo˥\textsubscript{a}}}}}{}
\textcolor{teal}{\mytextsc{classificateur}} \hspace{4pt} Ton~: H\textsubscript{a}.

Classificateur des ensembles, des lots.
\textcolor{brown}{\zh{量词:套(汉语借词)。}}

 Emprunt~: chinois \textcolor{brown}{\zh{套}}



\paragraph{\hspace{-0.5cm} {\Large \textcolor{darkblue}{\textbf{\ipa{tʰo˧ɕi˩}}}}} \hypertarget{t\string_ho\string_Ms£i\string_B1}{}
\markboth{\textcolor{darkblue}{\textbf{\ipa{tʰo˧ɕi˩}}}}{}
\textcolor{teal}{\mytextsc{nom}} \hspace{4pt} Ton~: L\#.

Messager.
\textcolor{brown}{\zh{通信员(汉语借词)。}}

 Emprunt~: chinois \textcolor{brown}{\zh{通信}}

 \mytextsc{clf}~: \textcolor{darkblue}{\textbf{\ipa{v˧}}} 

\paragraph{\hspace{-0.5cm} {\Large \textcolor{darkblue}{\textbf{\ipa{tʰo˧ɕi˧˥}}}}} \hypertarget{t\string_ho\string_Ms£i\string_M\string_T1}{}
\markboth{\textcolor{darkblue}{\textbf{\ipa{tʰo˧ɕi˧˥}}}}{}
\textcolor{teal}{\mytextsc{nom}} \hspace{4pt} Ton~: MH\#.

Forêt de conifères.
\textcolor{brown}{\zh{松树林。}}


 \mytextsc{clf}~: \textcolor{darkblue}{\textbf{\ipa{pʰæ˧˥}}} 

\paragraph{\hspace{-0.5cm} {\Large \textcolor{darkblue}{\textbf{\ipa{tʰo˧dzi˩}}}}} \hypertarget{t\string_ho\string_Mdzi\string_B1}{}
\markboth{\textcolor{darkblue}{\textbf{\ipa{tʰo˧dzi˩}}}}{}
\textcolor{teal}{\mytextsc{nom}} \hspace{4pt} Ton~: L\#.

Pin.
\textcolor{brown}{\zh{松树。}}


 \mytextsc{clf}~: \textcolor{darkblue}{\textbf{\ipa{dzi˩}}} 

\paragraph{\hspace{-0.5cm} {\Large \textcolor{darkblue}{\textbf{\ipa{tʰo˧dzi˩-hwæ˩tsɯ˩}}}}} \hypertarget{t\string_ho\string_Mdzi\string_B-hw\{\string_BtsM\string_B1}{}
\markboth{\textcolor{darkblue}{\textbf{\ipa{tʰo˧dzi˩-hwæ˩tsɯ˩}}}}{}
\textcolor{teal}{\mytextsc{nom}} \hspace{4pt} Ton~: LM-.

Hérisson; littéralement ``souris des pins".
\textcolor{brown}{\zh{刺猬。}}




\paragraph{\hspace{-0.5cm} {\Large \textcolor{darkblue}{\textbf{\ipa{tʰo˧fv̩˧}}}}} \hypertarget{t\string_ho\string_Mfv\string_=\string_M1}{}
\markboth{\textcolor{darkblue}{\textbf{\ipa{tʰo˧fv̩˧}}}}{}
\textcolor{teal}{\mytextsc{nom}} \hspace{4pt} Ton~: M.

Bandit, maraudeur.
\textcolor{brown}{\zh{土匪(汉语借词)。}}

 Emprunt~: chinois \textcolor{brown}{\zh{土匪}}



\paragraph{\hspace{-0.5cm} {\Large \textcolor{darkblue}{\textbf{\ipa{tʰo˧lɑ˧tɕi˧}}}}} \hypertarget{t\string_ho\string_MlA\string_Mts£i\string_M1}{}
\markboth{\textcolor{darkblue}{\textbf{\ipa{tʰo˧lɑ˧tɕi˧}}}}{}
\textcolor{teal}{\mytextsc{nom}} \hspace{4pt} Ton~: M.

Tracteur.
\textcolor{brown}{\zh{拖拉机(汉语借词)。}}

 Emprunt~: chinois \textcolor{brown}{\zh{洋火}}

\begin{exe}
\sn \textcolor{darkblue}{\textbf{\ipa{bo˩mi˧-tʰo˧lɑ˧tɕi˧}}}
\trans 'tracteur-truie': petit tracteur (le premier modèle introduit à Yongning)
\trans \textit{\textcolor{brown}{\zh{‘母猪拖拉机’:小型拖拉机}}}
\end{exe}
 \mytextsc{clf}~: \textcolor{darkblue}{\textbf{\ipa{yyyy}}} 

\paragraph{\hspace{-0.5cm} {\Large \textcolor{darkblue}{\textbf{\ipa{tʰo˧li˧}}}}} \hypertarget{t\string_ho\string_Mli\string_M1}{}
\markboth{\textcolor{darkblue}{\textbf{\ipa{tʰo˧li˧}}}}{}
\textcolor{teal}{\mytextsc{nom}} \hspace{4pt} Ton~: M.

Lapin.
\textcolor{brown}{\zh{兔子。}}


 \mytextsc{clf}~: \textcolor{darkblue}{\textbf{\ipa{mi˩}}} 

\paragraph{\hspace{-0.5cm} {\Large \textcolor{darkblue}{\textbf{\ipa{tʰo˧li˧kʰv̩˧˥}}}}} \hypertarget{t\string_ho\string_Mli\string_Mk\string_hv\string_=\string_M\string_T1}{}
\markboth{\textcolor{darkblue}{\textbf{\ipa{tʰo˧li˧kʰv̩˧˥}}}}{}
\textcolor{teal}{\mytextsc{nom}} \hspace{4pt} Ton~: MH\#.

Année du Lapin.
\textcolor{brown}{\zh{兔年。}}




\paragraph{\hspace{-0.5cm} {\Large \textcolor{darkblue}{\textbf{\ipa{tʰo˧li˧-mi˩}}}}} \hypertarget{t\string_ho\string_Mli\string_M-mi\string_B1}{}
\markboth{\textcolor{darkblue}{\textbf{\ipa{tʰo˧li˧-mi˩}}}}{}
\textcolor{teal}{\mytextsc{nom}} \hspace{4pt} Ton~: \mytextsc{L}.

Lapin femelle.
\textcolor{brown}{\zh{母兔。}}


 \mytextsc{clf}~: \textcolor{darkblue}{\textbf{\ipa{mi˩}}} 

\paragraph{\hspace{-0.5cm} {\Large \textcolor{darkblue}{\textbf{\ipa{tʰo˧li˧-pʰv̩\#˥}}}}} \hypertarget{t\string_ho\string_Mli\string_M-p\string_hv\string_=\#\string_T1}{}
\markboth{\textcolor{darkblue}{\textbf{\ipa{tʰo˧li˧-pʰv̩\#˥}}}}{}
\textcolor{teal}{\mytextsc{nom}} \hspace{4pt} Ton~: \#H.

Lapin mâle.
\textcolor{brown}{\zh{公兔。}}


 \mytextsc{clf}~: \textcolor{darkblue}{\textbf{\ipa{mi˩}}} 

\paragraph{\hspace{-0.5cm} {\Large \textcolor{darkblue}{\textbf{\ipa{tʰo˧li˧-zo\#˥}}}}} \hypertarget{t\string_ho\string_Mli\string_M-zo\#\string_T1}{}
\markboth{\textcolor{darkblue}{\textbf{\ipa{tʰo˧li˧-zo\#˥}}}}{}
\textcolor{teal}{\mytextsc{nom}} \hspace{4pt} Ton~: \#H.

Petit lapin, bébé lapin.
\textcolor{brown}{\zh{小兔。}}


 \mytextsc{clf}~: \textcolor{darkblue}{\textbf{\ipa{ɭɯ˧}}} 

\paragraph{\hspace{-0.5cm} {\Large \textcolor{darkblue}{\textbf{\ipa{tʰo˧-mo˩}}}}} \hypertarget{t\string_ho\string_M-mo\string_B1}{}
\markboth{\textcolor{darkblue}{\textbf{\ipa{tʰo˧-mo˩}}}}{}
\textcolor{teal}{\mytextsc{nom}} \hspace{4pt} Ton~: L\#.

``champignon des sapins": champignon comestible, ainsi nommé parce qu'il pousse au pied des sapins.
\textcolor{brown}{\zh{“松树菌”:一种菌子。}}




\paragraph{\hspace{-0.5cm} {\Large \textcolor{darkblue}{\textbf{\ipa{tʰo˧ɻæ˥}}}}} \hypertarget{t\string_ho\string_Mr£`\{\string_T1}{}
\markboth{\textcolor{darkblue}{\textbf{\ipa{tʰo˧ɻæ˥}}}}{}
\textcolor{teal}{\mytextsc{nom}} \hspace{4pt} Ton~: H\#.

Pignon de pin (graine comestible).
\textcolor{brown}{\zh{松子。}}


 \mytextsc{clf}~: \textcolor{darkblue}{\textbf{\ipa{ʈʂwɤ˧}}} poignée

\paragraph{\hspace{-0.5cm} {\Large \textcolor{darkblue}{\textbf{\ipa{tʰo˧tɕo˧}}}}} \hypertarget{t\string_ho\string_Mts£o\string_M1}{}
\markboth{\textcolor{darkblue}{\textbf{\ipa{tʰo˧tɕo˧}}}}{}
\textcolor{teal}{\mytextsc{adverbe}} \hspace{4pt} Ton~: .

Vers l'arrière.
\textcolor{brown}{\zh{往后。}}


\begin{exe}
\sn \textcolor{darkblue}{\textbf{\ipa{tʰo˧tɕo˧ li˧}}}
\trans regarder derrière (soi)
\trans \textit{\textcolor{brown}{\zh{往后看}}}
\end{exe}
\begin{exe}
\sn \textcolor{darkblue}{\textbf{\ipa{tʰo˧tɕo˧ ɖɯ˧-li˧-ɻ̍˧}}}
\trans jeter un coup d'œil en arrière
\trans \textit{\textcolor{brown}{\zh{往后看一眼}}}
\end{exe}


\paragraph{\hspace{-0.5cm} {\Large \textcolor{darkblue}{\textbf{\ipa{tʰo˧tsʰe˧-ʁwɤ\#˥}}}}} \hypertarget{t\string_ho\string_Mts\string_he\string_M-Rw7\#\string_T1}{}
\markboth{\textcolor{darkblue}{\textbf{\ipa{tʰo˧tsʰe˧-ʁwɤ\#˥}}}}{}
\textcolor{teal}{\mytextsc{nom}} \hspace{4pt} Ton~: \#H.

Un village proche des Sources Chaudes.
\textcolor{brown}{\zh{温泉乡的一个村落。}}


\begin{exe}
\sn \textcolor{darkblue}{\textbf{\ipa{tʰo˧tsʰe\#˥}}}
\trans même sens
\trans \textit{\textcolor{brown}{\zh{同上}}}
\end{exe}
\begin{exe}
\sn \textcolor{darkblue}{\textbf{\ipa{ə˧go˧-ʁwɤ˧, | ʁwɤ˧lɑ˩-bi˩, | bæ˧ʁwɤ˧, | tʰo˧tsʰe\#˥, | pi˧tsʰe˩-di˩, | pɤ˧dʑɤ˩-di˩, | ʁwɤ˧tv̩˧}}}
\trans Villages au sortir de la plaine de Yongning; les deux premiers comportent une population na; le troisième est un village na; les suivants sont essentiellement des villages pumi/prinmi.
\trans \textit{\textcolor{brown}{\zh{永宁背向泸沽湖方向经过的村落。前两个村落拥有相当大的摩梭人口比例,第三个村落是摩梭村,最后一个是普米村。}}}
\end{exe}
\begin{exe}
\sn \textcolor{darkblue}{\textbf{\ipa{tʰo˧tsʰe˧: | bɤ˧!}}}
\trans \textcolor{darkblue}{\textbf{\ipa{/tʰo˧tsʰe˧/}}}, c'est un village pumi!
\trans \textit{\textcolor{brown}{\zh{fv:/tʰo˧tsʰe˧/是一个普米族村落!}}}
\end{exe}


\paragraph{\hspace{-0.5cm} {\Large \textcolor{darkblue}{\textbf{\ipa{tʰo˧ʈɯ\#˥}}}}} \hypertarget{t\string_ho\string_Mt`M\#\string_T1}{}
\markboth{\textcolor{darkblue}{\textbf{\ipa{tʰo˧ʈɯ\#˥}}}}{}
\textcolor{teal}{\mytextsc{nom}} \hspace{4pt} Ton~: \#H.

Un village de Yongning: Tuozhikaiji.
\textcolor{brown}{\zh{拖支开基村(永宁的一个村落)。}}


\begin{exe}
\sn \textcolor{darkblue}{\textbf{\ipa{dʑɤ˩bv̩˧kɤ˧-sɑ˥ʁwɤ˩, | hi˩ʁwɤ˩-lo˥, | æ˩mi˧-ʁwɤ\#˥, | lɑ˧lo˧-ʁwɤ˥, | lɑ˧ŋwɤ˧, | bɤ˧tsʰo˧gv̩˥, | ə˧lɑ˧-ʁwɤ\#˥, | gæ˧ɻæ˩, | qʰæ˧tɕʰi˧, | tʰo˧ʈɯ\#˥}}}
\trans les dix villages comptant traditionnellement comme faisant partie de Yongning
\trans \textit{\textcolor{brown}{\zh{摩梭传统地理概念中,属于永宁的十个村落}}}
\end{exe}


\paragraph{\hspace{-0.5cm} {\Large \textcolor{darkblue}{\textbf{\ipa{tʰo˧ʐv̩˥}}}}} \hypertarget{t\string_ho\string_Mz`v\string_=\string_T1}{}
\markboth{\textcolor{darkblue}{\textbf{\ipa{tʰo˧ʐv̩˥}}}}{}
\textcolor{teal}{\mytextsc{nom}} \hspace{4pt} Ton~: H\#.

Pigeon.
\textcolor{brown}{\zh{鸽子。}}


\begin{exe}
\sn \textcolor{darkblue}{\textbf{\ipa{tʰo˧ʐv̩˥-mi˩}}}
\trans pigeon femelle
\trans \textit{\textcolor{brown}{\zh{母鸽子}}}
\end{exe}
\begin{exe}
\sn \textcolor{darkblue}{\textbf{\ipa{tʰo˧ʐv̩˥-pʰv̩˩}}}
\trans pigeon mâle
\trans \textit{\textcolor{brown}{\zh{公鸽子}}}
\end{exe}
\begin{exe}
\sn \textcolor{darkblue}{\textbf{\ipa{tʰo˧ʐv̩˥-zo˩}}}
\trans petit pigeon
\trans \textit{\textcolor{brown}{\zh{小鸽子}}}
\end{exe}
 \mytextsc{clf}~: \textcolor{darkblue}{\textbf{\ipa{mi˩}}} 

\paragraph{\hspace{-0.5cm} {\Large \textcolor{darkblue}{\textbf{\ipa{tʰo˩\textsubscript{a}}}}}} \hypertarget{t\string_ho\string_Ba1}{}
\markboth{\textcolor{darkblue}{\textbf{\ipa{tʰo˩\textsubscript{a}}}}}{}
\textcolor{teal}{\mytextsc{verbe}} \hspace{4pt} Ton~: L\textsubscript{a}.

S’adosser à, s'appuyer.
\textcolor{brown}{\zh{靠。}}




\paragraph{\hspace{-0.5cm} {\Large \textcolor{darkblue}{\textbf{\ipa{tʰo˩lo˧}}}}} \hypertarget{t\string_ho\string_Blo\string_M1}{}
\markboth{\textcolor{darkblue}{\textbf{\ipa{tʰo˩lo˧}}}}{}
\textcolor{teal}{\mytextsc{nom}} \hspace{4pt} Ton~: LM.

Cheval de tête, dans une caravane.
\textcolor{brown}{\zh{头马:马帮里走在最前面的那匹马。}}




\paragraph{\hspace{-0.5cm} {\Large \textcolor{darkblue}{\textbf{\ipa{tʰo˩pʰv̩˧tɕʰɤ˧}}}}} \hypertarget{t\string_ho\string_Bp\string_hv\string_=\string_Mts£\string_h7\string_M1}{}
\markboth{\textcolor{darkblue}{\textbf{\ipa{tʰo˩pʰv̩˧tɕʰɤ˧}}}}{}
\textcolor{teal}{\mytextsc{nom}} \hspace{4pt} Ton~: LM.

Arme à feu, fusil; arquebuse.
\textcolor{brown}{\zh{枪,明火枪。}}

 Emprunt~: chinois?

 \mytextsc{clf}~: \textcolor{darkblue}{\textbf{\ipa{kʰɯ˩}}} 

\paragraph{\hspace{-0.5cm} {\Large \textcolor{darkblue}{\textbf{\ipa{tʰo˩ʁæ˩}}}}} \hypertarget{t\string_ho\string_BR\{\string_B1}{}
\markboth{\textcolor{darkblue}{\textbf{\ipa{tʰo˩ʁæ˩}}}}{}
\textcolor{teal}{\mytextsc{nom}} \hspace{4pt} Ton~: L.

Résine de pin.
\textcolor{brown}{\zh{松香。}}


 \mytextsc{clf}~: \textcolor{darkblue}{\textbf{\ipa{ʈʰɤ˥}}} 

\paragraph{\hspace{-0.5cm} {\Large \textcolor{darkblue}{\textbf{\ipa{tʰo˩ʂv̩˩}}}}} \hypertarget{t\string_ho\string_Bs`v\string_=\string_B1}{}
\markboth{\textcolor{darkblue}{\textbf{\ipa{tʰo˩ʂv̩˩}}}}{}
\textcolor{teal}{\mytextsc{nom}} \hspace{4pt} Ton~: L.

Aiguilles de pin.
\textcolor{brown}{\zh{树针。}}


 \mytextsc{clf}~: \textcolor{darkblue}{\textbf{\ipa{qɑ˩}}} botte

\paragraph{\hspace{-0.5cm} {\Large \textcolor{darkblue}{\textbf{\ipa{tʰo˩tɕi˧˥}}}}} \hypertarget{t\string_ho\string_Bts£i\string_M\string_T1}{}
\markboth{\textcolor{darkblue}{\textbf{\ipa{tʰo˩tɕi˧˥}}}}{}
\textcolor{teal}{\mytextsc{nom}} \hspace{4pt} Ton~: LM+MH\#.

Brique à l'ancienne: brique crue.
\textcolor{brown}{\zh{砖。}}


 \mytextsc{clf}~: \textcolor{darkblue}{\textbf{\ipa{ɭɯ˧}}} 

\paragraph{\hspace{-0.5cm} {\Large \textcolor{darkblue}{\textbf{\ipa{tʰv̩˧˥}}} \textsubscript{1}}} \hypertarget{t\string_hv\string_=\string_M\string_T1}{}
\markboth{\textcolor{darkblue}{\textbf{\ipa{tʰv̩˧˥}}} \textsubscript{1}}{}
\textcolor{teal}{\mytextsc{verbe}} \hspace{4pt} Ton~: MH.

Fouler du pied, marcher sur, écraser.
\textcolor{brown}{\zh{踩。}}


\begin{exe}
\sn \textcolor{darkblue}{\textbf{\ipa{tʰi˧-tʰo˩}}}
\trans \mytextsc{dur}
\trans \textit{\textcolor{brown}{\zh{\mytextsc{dur}}}}
\end{exe}
\begin{exe}
\sn \textcolor{darkblue}{\textbf{\ipa{tʰi˧-tʰo˩-ɻ̍˩}}}
\trans \mytextsc{dur} \string_ \mytextsc{inchoatif}
\trans \textit{\textcolor{brown}{\zh{\mytextsc{dur} \string_ \mytextsc{inceptive}}}}
\end{exe}
\begin{exe}
\sn \textcolor{darkblue}{\textbf{\ipa{ɖɯ˧-tʰo˩-ɻ̍˩}}}
\trans \mytextsc{délimitatif} \string_ \mytextsc{inchoatif}
\trans \textit{\textcolor{brown}{\zh{\mytextsc{delimitative} \string_ \mytextsc{inceptive}}}}
\end{exe}
\begin{exe}
\sn \textcolor{darkblue}{\textbf{\ipa{tʰv̩˩\textasciitilde{}tʰv̩˧˥}}}
\trans \mytextsc{red}
\trans \textit{\textcolor{brown}{\zh{\mytextsc{重叠}}}}
\end{exe}
\begin{exe}
\sn \textcolor{darkblue}{\textbf{\ipa{ɖɯ˧-tʰv̩˧ tʰi˥-tʰv̩˩}}}
\trans donner un coup de pied par terre, fouler le sol du pied
\trans \textit{\textcolor{brown}{\zh{踢一脚}}}
\end{exe}
\begin{exe}
\sn \textcolor{darkblue}{\textbf{\ipa{kʰɯ˧tsʰɤ˧ tʰv̩˥-tsʰɯ˩}}}
\trans donner un coup de pied par terre, fouler le sol du pied
\trans \textit{\textcolor{brown}{\zh{踢一脚}}}
\end{exe}
\begin{exe}
\sn \textcolor{darkblue}{\textbf{\ipa{kʰɯ˧tsʰɤ˧ tʰv̩˥\textasciitilde{}tʰv̩˩}}}
\trans donner un coup de pied par terre, fouler le sol du pied
\trans \textit{\textcolor{brown}{\zh{踢一脚}}}
\end{exe}
\begin{exe}
\sn \textcolor{darkblue}{\textbf{\ipa{kʰɯ˧tsʰɤ˧ tʰɑ˧-tʰv̩˧˥!}}}
\trans Ne donne pas de coup de pied!
\trans \textit{\textcolor{brown}{\zh{别踢!}}}
\end{exe}


\paragraph{\hspace{-0.5cm} {\Large \textcolor{darkblue}{\textbf{\ipa{tʰv̩˧˥}}} \textsubscript{2}}} \hypertarget{t\string_hv\string_=\string_M\string_T2}{}
\markboth{\textcolor{darkblue}{\textbf{\ipa{tʰv̩˧˥}}} \textsubscript{2}}{}
\textcolor{teal}{\mytextsc{verbe}} \hspace{4pt} Ton~: MH.

Se charger de, préparer, offrir (quelqu'un se charge d'offrir un repas aux gens du village; c'est lui qui paie, pas forcément qui fait la cuisine).
\textcolor{brown}{\zh{负担(某个活动的费用,如:请全村人吃饭)。}}




\paragraph{\hspace{-0.5cm} {\Large \textcolor{darkblue}{\textbf{\ipa{tʰv̩˥}}} \textsubscript{1}}} \hypertarget{t\string_hv\string_=\string_T1}{}
\markboth{\textcolor{darkblue}{\textbf{\ipa{tʰv̩˥}}} \textsubscript{1}}{}
\textcolor{teal}{\mytextsc{pronom}} \hspace{4pt} Ton~: \#H.

Démonstratif distal, qui forme un couple avec le démonstratif proximal.
\textcolor{brown}{\zh{那\mytextsc{指示}.远指。}}


\begin{exe}
\sn \textcolor{darkblue}{\textbf{\ipa{tʰv̩˧ ɲi˥!}}}
\trans c'est celui-là!
\trans \textit{\textcolor{brown}{\zh{是那个!}}}
\end{exe}
\begin{exe}
\sn \textcolor{darkblue}{\textbf{\ipa{tʰv̩˧-v̩\#˥}}}
\trans celui-là (\mytextsc{dem}.dist-\mytextsc{clf}.individu)
\trans \textit{\textcolor{brown}{\zh{那个}}}
\end{exe}
\textit{Voir~:} \hyperlink{t\string_hv\string_=\string_T2}{\textcolor{darkblue}{\textbf{\ipa{tʰv̩˥}}} \textsubscript{2}} \textit{Voir~:} \hyperlink{t\string_hv\string_=\string_T3}{\textcolor{darkblue}{\textbf{\ipa{tʰv̩˥}}} \textsubscript{3}} 

\paragraph{\hspace{-0.5cm} {\Large \textcolor{darkblue}{\textbf{\ipa{tʰv̩˥}}} \textsubscript{2}}} \hypertarget{t\string_hv\string_=\string_T2}{}
\markboth{\textcolor{darkblue}{\textbf{\ipa{tʰv̩˥}}} \textsubscript{2}}{}
\textcolor{teal}{\mytextsc{pronom}} \hspace{4pt} Ton~: \#H.

Pronom de troisième personne du singulier; provient du démonstratif distal.
\textcolor{brown}{\zh{他。}}


\begin{exe}
\sn \textcolor{darkblue}{\textbf{\ipa{tʰv̩˧=ɻ̍˩}}}
\trans sa famille, sa maisonnée, les siens
\trans \textit{\textcolor{brown}{\zh{他家、他家族、他的人}}}
\end{exe}
\textit{Voir~:} \hyperlink{t\string_hv\string_=\string_T1}{\textcolor{darkblue}{\textbf{\ipa{tʰv̩˥}}} \textsubscript{1}} \textit{Voir~:} \hyperlink{t\string_hv\string_=\string_T3}{\textcolor{darkblue}{\textbf{\ipa{tʰv̩˥}}} \textsubscript{3}} 

\paragraph{\hspace{-0.5cm} {\Large \textcolor{darkblue}{\textbf{\ipa{tʰv̩˥}}} \textsubscript{3}}} \hypertarget{t\string_hv\string_=\string_T3}{}
\markboth{\textcolor{darkblue}{\textbf{\ipa{tʰv̩˥}}} \textsubscript{3}}{}
\textcolor{teal}{\mytextsc{suffixe}} \hspace{4pt} Ton~: \#H.

Focalisateur; grammaticalisé à partir du démonstratif distal.
\textcolor{brown}{\zh{\mytextsc{主题(°指示}.远指)。}}


\textit{Voir~:} \hyperlink{t\string_hv\string_=\string_T1}{\textcolor{darkblue}{\textbf{\ipa{tʰv̩˥}}} \textsubscript{1}} \textit{Voir~:} \hyperlink{t\string_hv\string_=\string_T2}{\textcolor{darkblue}{\textbf{\ipa{tʰv̩˥}}} \textsubscript{2}} 

\paragraph{\hspace{-0.5cm} {\Large \textcolor{darkblue}{\textbf{\ipa{tʰv̩˧\textsubscript{a}}}}}} \hypertarget{t\string_hv\string_=\string_Ma1}{}
\markboth{\textcolor{darkblue}{\textbf{\ipa{tʰv̩˧\textsubscript{a}}}}}{}
\textcolor{teal}{\mytextsc{verbe}} \hspace{4pt} Ton~: M\textsubscript{a}.
\ding{202} 
Sortir.
\textcolor{brown}{\zh{出来。}}


\begin{exe}
\sn \textcolor{darkblue}{\textbf{\ipa{ɑ˩pʰo˩ tʰv̩˩˥}}}
\trans sortir, ex.: un animal sort de son terrier
\trans \textit{\textcolor{brown}{\zh{出来,如:动物从地洞里爬出来}}}
\end{exe}
\begin{exe}
\sn \textcolor{darkblue}{\textbf{\ipa{ɲi˧mi˧ tʰv̩˧}}}
\trans le soleil paraît
\trans \textit{\textcolor{brown}{\zh{太阳出来}}}
\end{exe}
\ding{203} 
Souffler (vent).
\textcolor{brown}{\zh{刮(风)。}}


\ding{204} 
Germer, bourgeonner, donner des bourgeons.
\textcolor{brown}{\zh{发芽、抽芽。}}


\begin{exe}
\sn \textcolor{darkblue}{\textbf{\ipa{si˧dzi˩ | ʁo˧bv̩˧ tʰv̩˧}}}
\trans l'arbre fait des bourgeons
\trans \textit{\textcolor{brown}{\zh{树抽芽}}}
\end{exe}
\ding{205} 
Apparaître, se faire: une blessure apparaît, on reçoit une blessure.
\textcolor{brown}{\zh{出现。}}


\begin{exe}
\sn \textcolor{darkblue}{\textbf{\ipa{mi˧ tʰv̩˧}}}
\trans se faire une blessure/avoir une blessure/se blesser
\trans \textit{\textcolor{brown}{\zh{受伤}}}
\end{exe}
\begin{exe}
\sn \textcolor{darkblue}{\textbf{\ipa{ɖɯ˧-v̩˧ mi˧ tʰv̩˧-ze˧!}}}
\trans quelqu'un s'est blessé!
\trans \textit{\textcolor{brown}{\zh{有人受伤了!}}}
\end{exe}
\ding{206} 
Créer, fonder; se trouver, se fabriquer.
\textcolor{brown}{\zh{建立、创造、制造出来。}}


\begin{exe}
\sn \textcolor{darkblue}{\textbf{\ipa{ʑi˩ tʰv̩˩}}}
\trans créer une nouvelle maison, fonder une nouvelle maison; traduit en chinois par \textcolor{brown}{\zh{分家}}, concept en fait assez différent dans la mesure où \textcolor{darkblue}{\textbf{\ipa{ʑi˩ tʰv̩˩}}} évoque un essaimage, plutôt qu'une séparation.
\trans \textit{\textcolor{brown}{\zh{分家、建立新家}}}
\end{exe}
\begin{exe}
\sn \textcolor{darkblue}{\textbf{\ipa{ʈʂʰɯ˧ | ʑi˩ tʰv̩˩-ze˥!}}}
\trans Il/elle a fondé sa propre maisonnée!
\trans \textit{\textcolor{brown}{\zh{他建了新家!}}}
\end{exe}
\begin{exe}
\sn \textcolor{darkblue}{\textbf{\ipa{ʈʂʰɯ˧ | ʑi˩ tʰv̩˩-bi˩˥!}}}
\trans Il/elle va fonder sa propre maisonnée!
\trans \textit{\textcolor{brown}{\zh{他要建个新家!}}}
\end{exe}
\begin{exe}
\sn \textcolor{darkblue}{\textbf{\ipa{lo˧ mɤ˧-dʑo˧, | lo˧ tʰv̩˧˥! / no˧ | lo˧ mɤ˧-dʑo˧, | lo˧ tʰv̩˧-ɲi˥!}}}
\trans Il n'a pas d'obligations, et pourtant il travaille! (Compliment à l'endroit d'un fonctionnaire qui pourrait se contenter de percevoir son salaire, mais qui se donne à lui-même des objectifs et des tâches à accomplir. La phrase peut également être employée de façon négative, pour critiquer quelqu'un qui déploie une activité inutile au lieu de se tenir tranquille.)
\trans \textit{\textcolor{brown}{\zh{自找麻烦!(这句,除贬义用法,还能用来表扬,如表扬一位当官的人努力去做好事,给自己找有意义的事情干。)}}}
\end{exe}


\paragraph{\hspace{-0.5cm} {\Large \textcolor{darkblue}{\textbf{\ipa{tʰv̩˧˥\textsubscript{a}}}}}} \hypertarget{t\string_hv\string_=\string_M\string_Ta1}{}
\markboth{\textcolor{darkblue}{\textbf{\ipa{tʰv̩˧˥\textsubscript{a}}}}}{}
\textcolor{teal}{\mytextsc{classificateur}} \hspace{4pt} Ton~: MH\textsubscript{a}.

Pas, enjambée.
\textcolor{brown}{\zh{量词:步。}}


\begin{exe}
\sn \textcolor{darkblue}{\textbf{\ipa{ɖɯ˧-tʰv̩˧\textasciitilde{}ɖɯ˥-tʰv̩˩}}}
\trans pas à pas
\trans \textit{\textcolor{brown}{\zh{一步一步}}}
\end{exe}
\begin{exe}
\sn \textcolor{darkblue}{\textbf{\ipa{ɖɯ˧-tʰv̩˧˥, | ɖɯ˧-tʰv̩˧˥}}}
\trans idem, détachant les deux parties; cette forme est plus proche d'une répétition que d'une réduplication
\trans \textit{\textcolor{brown}{\zh{一步又一步}}}
\end{exe}


\paragraph{\hspace{-0.5cm} {\Large \textcolor{darkblue}{\textbf{\ipa{tʰv̩˩\textsubscript{b}}}}}} \hypertarget{t\string_hv\string_=\string_Bb1}{}
\markboth{\textcolor{darkblue}{\textbf{\ipa{tʰv̩˩\textsubscript{b}}}}}{}
\textcolor{teal}{\mytextsc{classificateur}} \hspace{4pt} Ton~: L\textsubscript{b}.

Classificateur des dizaines. Autrefois, le terme servait à compter par ensembles de 8. Le terme a demeuré, mais son sens s'est déplacé vers le sens de 'dizaine', suivant la généralisation du système de numération à base dix.
\textcolor{brown}{\zh{量词:一套(有十个)。更早的意思是八个。}}


\begin{exe}
\sn \textcolor{darkblue}{\textbf{\ipa{qʰwɤ˩˥ | ɖɯ˧-tʰv̩˩}}}
\trans un lot de dix bols
\trans \textit{\textcolor{brown}{\zh{一套十个碗}}}
\end{exe}
\begin{exe}
\sn \textcolor{darkblue}{\textbf{\ipa{ɖʐɯ˧ʂɯ˥ | ɖɯ˧-tʰv̩˩}}}
\trans un paquet de dix (paires de) baguettes
\trans \textit{\textcolor{brown}{\zh{一套十(双)筷子}}}
\end{exe}


\paragraph{\hspace{-0.5cm} {\Large \textcolor{darkblue}{\textbf{\ipa{tʰv̩˧\textsubscript{b}}}}}} \hypertarget{t\string_hv\string_=\string_Mb1}{}
\markboth{\textcolor{darkblue}{\textbf{\ipa{tʰv̩˧\textsubscript{b}}}}}{}
\textcolor{teal}{\mytextsc{verbe}} \hspace{4pt} Ton~: M\textsubscript{b}.

Prêter (un objet).
\textcolor{brown}{\zh{借给人。}}


\begin{exe}
\sn \textcolor{darkblue}{\textbf{\ipa{tso˧\textasciitilde{}tso˧ tʰv̩˧}}}
\trans prêter quelque chose
\trans \textit{\textcolor{brown}{\zh{借东西(给人)}}}
\end{exe}


\paragraph{\hspace{-0.5cm} {\Large \textcolor{darkblue}{\textbf{\ipa{tʰv̩˧gi˧}}}}} \hypertarget{t\string_hv\string_=\string_Mgi\string_M1}{}
\markboth{\textcolor{darkblue}{\textbf{\ipa{tʰv̩˧gi˧}}}}{}
\textcolor{teal}{\mytextsc{adverbe}} \hspace{4pt} Ton~: .

Là-bas, de ce côté-là.
\textcolor{brown}{\zh{那边。}}




\paragraph{\hspace{-0.5cm} {\Large \textcolor{darkblue}{\textbf{\ipa{tʰv̩˧ne-ʝi˥}}}}} \hypertarget{t\string_hv\string_=\string_Mne-j££i\string_T1}{}
\markboth{\textcolor{darkblue}{\textbf{\ipa{tʰv̩˧ne-ʝi˥}}}}{}
\textcolor{teal}{\mytextsc{adverbe}} \hspace{4pt} Ton~: MH\#.

Ainsi, de cette façon (adverbe de manière), contenant le démonstratif distal.
\textcolor{brown}{\zh{那样。}}




\paragraph{\hspace{-0.5cm} {\Large \textcolor{darkblue}{\textbf{\ipa{tʰv̩˧ɲi\#˥}}}}} \hypertarget{t\string_hv\string_=\string_MJi\#\string_T1}{}
\markboth{\textcolor{darkblue}{\textbf{\ipa{tʰv̩˧ɲi\#˥}}}}{}
\textcolor{teal}{\mytextsc{adverbe}} \hspace{4pt} Ton~: \#H.

Ce jour-là (déictique lointain).
\textcolor{brown}{\zh{那天。}}




\paragraph{\hspace{-0.5cm} {\Large \textcolor{darkblue}{\textbf{\ipa{tʰv̩˧qo˧}}}}} \hypertarget{t\string_hv\string_=\string_Mqo\string_M1}{}
\markboth{\textcolor{darkblue}{\textbf{\ipa{tʰv̩˧qo˧}}}}{}
\textcolor{teal}{\mytextsc{pronom}} \hspace{4pt} Ton~: M.

Là-bas; cet endroit-là.
\textcolor{brown}{\zh{那里、那个地方。}}




\paragraph{\hspace{-0.5cm} {\Large \textcolor{darkblue}{\textbf{\ipa{tʰv̩˧-se˩-gɤ˩}}}}} \hypertarget{t\string_hv\string_=\string_M-se\string_B-g7\string_B1}{}
\markboth{\textcolor{darkblue}{\textbf{\ipa{tʰv̩˧-se˩-gɤ˩}}}}{}
\textcolor{teal}{\mytextsc{adverbe}} \hspace{4pt} Ton~: \mytextsc{L}.

Désormais, dorénavant.
\textcolor{brown}{\zh{今后、从此、此后。}}




\paragraph{\hspace{-0.5cm} {\Large \textcolor{darkblue}{\textbf{\ipa{tʰv̩˧-si˥}}}}} \hypertarget{t\string_hv\string_=\string_M-si\string_T1}{}
\markboth{\textcolor{darkblue}{\textbf{\ipa{tʰv̩˧-si˥}}}}{}
\textcolor{teal}{\mytextsc{adverbe}} \hspace{4pt} Ton~: H\#.

Nombreux.
\textcolor{brown}{\zh{多。}}


\begin{exe}
\sn \textcolor{darkblue}{\textbf{\ipa{mv̩˧ʁo˧=ɻ̍˥-dʑo˩, | ɻæ˩˥ | tʰv̩˧-si˥ | tʰv̩˩-jɤ˩ dʑo˩˥!}}}
\trans Les gens du Ciel, ils avaient des semences en abondance! (récit ``Seeds")
\trans \textit{\textcolor{brown}{\zh{天上的人,有许多许多种子!(故事:“种子”)}}}
\end{exe}


\paragraph{\hspace{-0.5cm} {\Large \textcolor{darkblue}{\textbf{\ipa{tʰv̩˧-sɯ˩kv̩˩}}}}} \hypertarget{t\string_hv\string_=\string_M-sM\string_Bkv\string_=\string_B1}{}
\markboth{\textcolor{darkblue}{\textbf{\ipa{tʰv̩˧-sɯ˩kv̩˩}}}}{}
\textcolor{teal}{\mytextsc{pronom}} \hspace{4pt} Ton~: \mytextsc{L}.

Pronom de troisième personne du pluriel.
\textcolor{brown}{\zh{他们。}}




\paragraph{\hspace{-0.5cm} {\Large \textcolor{darkblue}{\textbf{\ipa{‑tʰv̩˧}}} \textsubscript{1}}} \hypertarget{‑t\string_hv\string_=\string_M1}{}
\markboth{\textcolor{darkblue}{\textbf{\ipa{‑tʰv̩˧}}} \textsubscript{1}}{}
\textcolor{teal}{\mytextsc{postposition}} \hspace{4pt} Ton~: M.

Postposition temporelle: jusqu'à.
\textcolor{brown}{\zh{到……为止。}}




\paragraph{\hspace{-0.5cm} {\Large \textcolor{darkblue}{\textbf{\ipa{‑tʰv̩˧}}} \textsubscript{2}}} \hypertarget{‑t\string_hv\string_=\string_M2}{}
\markboth{\textcolor{darkblue}{\textbf{\ipa{‑tʰv̩˧}}} \textsubscript{2}}{}
\textcolor{teal}{\mytextsc{suffixe}} \hspace{4pt} Ton~: M.

Parvenir à, réussir à, réaliser avec succès; grammaticalisé à partir du verbe 'sortir'.
\textcolor{brown}{\zh{……成。}}


\begin{exe}
\sn \textcolor{darkblue}{\textbf{\ipa{lo˧ ʝi˧-mɤ˧-tʰv̩˧}}}
\trans ne pas parvenir à venir à bien d'une tâche; ex.: une personne est constamment dérangée et ne parvient pas à travailler de façon concentrée
\trans \textit{\textcolor{brown}{\zh{活做不出来、活做不成(比如:一个人经常被打扰,所以不能集中工作,没有效率,要做的事做不成)}}}
\end{exe}


\newpage
\section*{\centering- \textcolor{darkblue}{\textbf{\ipa{tɕ}}} -}
\paragraph{\hspace{-0.5cm} {\Large \textcolor{darkblue}{\textbf{\ipa{tɕæ˧hæ˩}}}}} \hypertarget{ts£\{\string_Mh\{\string_B1}{}
\markboth{\textcolor{darkblue}{\textbf{\ipa{tɕæ˧hæ˩}}}}{}
\textcolor{teal}{\mytextsc{nom}} \hspace{4pt} Ton~: L\#.

Caoutchouc.
\textcolor{brown}{\zh{橡胶(汉语借词。第二个音节:未确定。)。}}

 Emprunt~: chinois \textcolor{brown}{\zh{胶+?}}

\begin{exe}
\sn \textcolor{darkblue}{\textbf{\ipa{tɕæ˧hæ˩-dzɑ˩qʰwɤ˩}}}
\trans chaussures à semelle en gomme/en caoutchouc; baskets
\trans \textit{\textcolor{brown}{\zh{橡胶鞋、橡胶底鞋}}}
\end{exe}
 \mytextsc{clf}~: \textcolor{darkblue}{\textbf{\ipa{dzi˧}}} 

\paragraph{\hspace{-0.5cm} {\Large \textcolor{darkblue}{\textbf{\ipa{tɕæ˧pʰv̩˩}}}}} \hypertarget{ts£\{\string_Mp\string_hv\string_=\string_B1}{}
\markboth{\textcolor{darkblue}{\textbf{\ipa{tɕæ˧pʰv̩˩}}}}{}
\textcolor{teal}{\mytextsc{adjectif}} \hspace{4pt} Ton~: L\#.

Blanc (visage, habits, cheveux...).
\textcolor{brown}{\zh{白(脸、衣服)。}}


\begin{exe}
\sn \textcolor{darkblue}{\textbf{\ipa{tɕæ˧pʰv̩˩-bɑ˩lɑ˩}}}
\trans vêtement blanc
\trans \textit{\textcolor{brown}{\zh{白的衣服}}}
\end{exe}
\begin{exe}
\sn \textcolor{darkblue}{\textbf{\ipa{tɕæ˧pʰv̩˩-ʈæ˩qʰwɤ˩}}}
\trans robe blanche
\trans \textit{\textcolor{brown}{\zh{白色裙子}}}
\end{exe}


\paragraph{\hspace{-0.5cm} {\Large \textcolor{darkblue}{\textbf{\ipa{tɕæ˧ɻæ˩}}}}} \hypertarget{ts£\{\string_Mr£`\{\string_B1}{}
\markboth{\textcolor{darkblue}{\textbf{\ipa{tɕæ˧ɻæ˩}}}}{}
\textcolor{teal}{\mytextsc{nom}} \hspace{4pt} Ton~: L\#.

Légumes en saumure. On en mangeait une sorte chaque jour pendant la saison d'hiver: un jour navet en saumure, etc.
\textcolor{brown}{\zh{酸菜、泡菜。}}


\begin{exe}
\sn \textcolor{darkblue}{\textbf{\ipa{wo˩-tɕæ˩ɻæ˥}}}
\trans feuilles de navet conservées dans la saumure
\trans \textit{\textcolor{brown}{\zh{圆根叶子酸菜}}}
\end{exe}
\begin{exe}
\sn \textcolor{darkblue}{\textbf{\ipa{tsʰɑ˧-tɕæ˧ɻæ˥}}}
\end{exe}
\begin{exe}
\sn \textcolor{darkblue}{\textbf{\ipa{ɬi˩bi˩-tɕæ˩ɻæ˥}}}
\trans navet conservé dans la saumure
\trans \textit{\textcolor{brown}{\zh{圆根酸菜}}}
\end{exe}
\begin{exe}
\sn \textcolor{darkblue}{\textbf{\ipa{pɤ˧pɤ˧tsʰɯ˧-tɕæ˧ɻæ˥}}}
\trans chou chinois en saumure
\trans \textit{\textcolor{brown}{\zh{圆白菜酸菜}}}
\end{exe}


\paragraph{\hspace{-0.5cm} {\Large \textcolor{darkblue}{\textbf{\ipa{tɕɤ}}}}} \hypertarget{ts£71}{}
\markboth{\textcolor{darkblue}{\textbf{\ipa{tɕɤ}}}}{}
\textcolor{teal}{\mytextsc{interjection}} \hspace{4pt} Ton~: 0.

Interjection: tiens! eh!
\textcolor{brown}{\zh{感叹词:嘿!。}}




\paragraph{\hspace{-0.5cm} {\Large \textcolor{darkblue}{\textbf{\ipa{tɕɤ˥}}}}} \hypertarget{ts£7\string_T1}{}
\markboth{\textcolor{darkblue}{\textbf{\ipa{tɕɤ˥}}}}{}
\textcolor{teal}{\mytextsc{verbe}} \hspace{4pt} Ton~: H.

S'effacer (couleur).
\textcolor{brown}{\zh{褪色。}}


\begin{exe}
\sn \textcolor{darkblue}{\textbf{\ipa{le˧-tɕɤ˥-ze˩}}}
\trans \mytextsc{accomp} \string_ \mytextsc{pfv}
\trans \textit{\textcolor{brown}{\zh{褪色了}}}
\end{exe}


\paragraph{\hspace{-0.5cm} {\Large \textcolor{darkblue}{\textbf{\ipa{tɕɤ˧fv̩˩}}}}} \hypertarget{ts£7\string_Mfv\string_=\string_B1}{}
\markboth{\textcolor{darkblue}{\textbf{\ipa{tɕɤ˧fv̩˩}}}}{}
\textcolor{teal}{\mytextsc{nom}} \hspace{4pt} Ton~: L\#.

Container pour liquides; s'emploie pour désigner les containers en matière plastique.
\textcolor{brown}{\zh{塑料桶等存水用的容器。}}


 \mytextsc{clf}~: \textcolor{darkblue}{\textbf{\ipa{ɭɯ˧}}} 

\paragraph{\hspace{-0.5cm} {\Large \textcolor{darkblue}{\textbf{\ipa{tɕɤ˧ho˩pæ˧}}}}} \hypertarget{ts£7\string_Mho\string_Bp\{\string_M1}{}
\markboth{\textcolor{darkblue}{\textbf{\ipa{tɕɤ˧ho˩pæ˧}}}}{}
\textcolor{teal}{\mytextsc{nom}} \hspace{4pt} Ton~: MLM.

Contreplaqué, panneau en contreplaqué.
\textcolor{brown}{\zh{胶合板(汉语借词)。}}

 Emprunt~: chinois \textcolor{brown}{\zh{胶合板}}



\paragraph{\hspace{-0.5cm} {\Large \textcolor{darkblue}{\textbf{\ipa{tɕɤ˧jo˩}}}}} \hypertarget{ts£7\string_Mjo\string_B1}{}
\markboth{\textcolor{darkblue}{\textbf{\ipa{tɕɤ˧jo˩}}}}{}
\textcolor{teal}{\mytextsc{nom}} \hspace{4pt} Ton~: .

Prison.
\textcolor{brown}{\zh{监狱(汉语借词)。}}

 Emprunt~: chinois \textcolor{brown}{\zh{监狱}}

\begin{exe}
\sn \textcolor{darkblue}{\textbf{\ipa{tɕɤ˧jo˩-qo˩ | tʰi˧-ʈæ˩}}}
\trans enfermer en prison, mettre en prison
\trans \textit{\textcolor{brown}{\zh{关在监狱}}}
\end{exe}
\begin{exe}
\sn \textcolor{darkblue}{\textbf{\ipa{tɕɤ˧jo˩-qo˩ ʈæ˩-hɯ˩-ze˩!}}}
\trans (On l')a jeté en prison! / (On l')a mis en prison!
\trans \textit{\textcolor{brown}{\zh{被关在监狱!}}}
\end{exe}
\textit{Voir~:} \hyperlink{lo\string_MnA\string_T-bv\string_=\string_B1}{\textcolor{darkblue}{\textbf{\ipa{lo˧nɑ˥-bv̩˩}}}} 

\paragraph{\hspace{-0.5cm} {\Large \textcolor{darkblue}{\textbf{\ipa{tɕɤ˧qʰɑ\#˥}}}}} \hypertarget{ts£7\string_Mq\string_hA\#\string_T1}{}
\markboth{\textcolor{darkblue}{\textbf{\ipa{tɕɤ˧qʰɑ\#˥}}}}{}
\textcolor{teal}{\mytextsc{nom}} \hspace{4pt} Ton~: \#H.

Armoise, \textit{Artemisia vulgaris}.
\textcolor{brown}{\zh{蒿、青蒿。}}Dialecte chinois local~: \zh{蒿草、蒿枝。}


\begin{exe}
\sn \textcolor{darkblue}{\textbf{\ipa{tɕɤ˧qʰɑ˧-mo˩}}}
\trans un champignon comestible, nommé 'champignon de l'armoise' parce qu'il croît à proximité de l'armoise
\trans \textit{\textcolor{brown}{\zh{一种可以吃的菌子,长在蒿附近}}}
\end{exe}
\textit{Voir~:} \hyperlink{ho\string_Mt`s`M\string_M1}{\textcolor{darkblue}{\textbf{\ipa{ho˧ʈʂɯ˧}}}} 

\paragraph{\hspace{-0.5cm} {\Large \textcolor{darkblue}{\textbf{\ipa{tɕɤ˧tɑ˧}}}}} \hypertarget{ts£7\string_MtA\string_M1}{}
\markboth{\textcolor{darkblue}{\textbf{\ipa{tɕɤ˧tɑ˧}}}}{}
\textcolor{teal}{\mytextsc{nom}} \hspace{4pt} Ton~: M.

Joug.
\textcolor{brown}{\zh{牛轭(单行)(汉语借词)。}}Dialecte chinois local~: \zh{牛夹担、牛枷档、牛拴。}

 Emprunt~: chinois \textcolor{brown}{\zh{夹担}}

\begin{exe}
\sn \textcolor{darkblue}{\textbf{\ipa{tɕɤ˧tɑ˧ tʰv̩˧-ɭɯ˧}}}
\trans \mytextsc{n}+\mytextsc{dem}+\mytextsc{clf}
\trans \textit{\textcolor{brown}{\zh{这个牛轭}}}
\end{exe}
 \mytextsc{clf}~: \textcolor{darkblue}{\textbf{\ipa{ɭɯ˧}}} 

\paragraph{\hspace{-0.5cm} {\Large \textcolor{darkblue}{\textbf{\ipa{tɕɤ˧tɑ˧-bæ˩}}}}} \hypertarget{ts£7\string_MtA\string_M-b\{\string_B1}{}
\markboth{\textcolor{darkblue}{\textbf{\ipa{tɕɤ˧tɑ˧-bæ˩}}}}{}
\textcolor{teal}{\mytextsc{nom}} \hspace{4pt} Ton~: \mytextsc{L}.
\textit{De:} \textbf{tɕɤ˧tɑ˧ et bæ˩} 
Courroies entre le joug et l'araire.
\textcolor{brown}{\zh{牛皮绳 ,犁具连轭之绳。}}


\begin{exe}
\sn \textcolor{darkblue}{\textbf{\ipa{tɕɤ˧tɑ˧-bæ˩ tʰv̩˩-kʰɯ˩}}}
\trans \mytextsc{n}+\mytextsc{dem}+\mytextsc{clf}
\trans \textit{\textcolor{brown}{\zh{这条牛皮绳}}}
\end{exe}
 \mytextsc{clf}~: \textcolor{darkblue}{\textbf{\ipa{kʰɯ˩}}} 

\paragraph{\hspace{-0.5cm} {\Large \textcolor{darkblue}{\textbf{\ipa{tɕɤ˧\textasciitilde{}tɕɤ˧}}}}} \hypertarget{ts£7\string_M~ts£7\string_M1}{}
\markboth{\textcolor{darkblue}{\textbf{\ipa{tɕɤ˧\textasciitilde{}tɕɤ˧}}}}{}
\textcolor{teal}{\mytextsc{adverbe}} \hspace{4pt} Ton~: M.

Précisément, exactement (ex.: au moment précis où, juste au moment où).
\textcolor{brown}{\zh{将将(汉语借词)、刚刚。}}Dialecte chinois local~: \zh{将将。}

 Emprunt~: chinois \textcolor{brown}{\zh{将将}}



\paragraph{\hspace{-0.5cm} {\Large \textcolor{darkblue}{\textbf{\ipa{tɕɤ˩}}}}} \hypertarget{ts£7\string_B1}{}
\markboth{\textcolor{darkblue}{\textbf{\ipa{tɕɤ˩}}}}{}
\textcolor{teal}{\mytextsc{verbe}} \hspace{4pt} Ton~: .

Attacher (ex.: un joug sur une vache; des troncs…).
\textcolor{brown}{\zh{打结、系上。}}

 Emprunt~: chinois \textcolor{brown}{\zh{架?}}

\begin{exe}
\sn \textcolor{darkblue}{\textbf{\ipa{ʁæ˧ɻ̍˥ | tʰi˧-tɕɤ˩}}}
\trans fixer (un joug sur un buffle)
\trans \textit{\textcolor{brown}{\zh{系上牛轭}}}
\end{exe}


\paragraph{\hspace{-0.5cm} {\Large \textcolor{darkblue}{\textbf{\ipa{tɕɤ˩ho˩tsɯ˥}}}}} \hypertarget{ts£7\string_Bho\string_BtsM\string_T1}{}
\markboth{\textcolor{darkblue}{\textbf{\ipa{tɕɤ˩ho˩tsɯ˥}}}}{}
\textcolor{teal}{\mytextsc{nom}} \hspace{4pt} Ton~: L+H\#.

Escroc.
\textcolor{brown}{\zh{骗子。}}


\begin{exe}
\sn \textcolor{darkblue}{\textbf{\ipa{ʈʂʰɯ˧ | hĩ˧ ʈʂʰɯ˧-v̩˧ | tɕɤ˩ho˩tsɯ˥ ɲi˩.}}}
\trans Cet homme, c'est un escroc!
\trans \textit{\textcolor{brown}{\zh{这个人是骗子!}}}
\end{exe}
 \mytextsc{clf}~: \textcolor{darkblue}{\textbf{\ipa{v̩˧}}} 

\paragraph{\hspace{-0.5cm} {\Large \textcolor{darkblue}{\textbf{\ipa{tɕɤ˧˥}}}}} \hypertarget{ts£7\string_M\string_T1}{}
\markboth{\textcolor{darkblue}{\textbf{\ipa{tɕɤ˧˥}}}}{}
\textcolor{teal}{\mytextsc{verbe}} \hspace{4pt} Ton~: MH.

Bouillir; cuire en faisant bouillir; cuire dans une casserole.
\textcolor{brown}{\zh{煮。}}


\begin{exe}
\sn \textcolor{darkblue}{\textbf{\ipa{ʂe˧ tɕɤ˩}}}
\trans faire bouillir de la viande, faire cuire de la viande à l'eau
\trans \textit{\textcolor{brown}{\zh{煮肉}}}
\end{exe}
\begin{exe}
\sn \textcolor{darkblue}{\textbf{\ipa{bo˩-hɑ˧ tɕɤ˩}}}
\trans faire bouillir la pâtée des cochons
\trans \textit{\textcolor{brown}{\zh{煮猪食}}}
\end{exe}
\begin{exe}
\sn \textcolor{darkblue}{\textbf{\ipa{ho˧ tɕɤ˩}}}
\trans faire du ragoût
\trans \textit{\textcolor{brown}{\zh{煮粥}}}
\end{exe}
\begin{exe}
\sn \textcolor{darkblue}{\textbf{\ipa{dʑɯ˩ʁo˩˥, | mo˧-no˥, | mo˧ tɕɤ˥-hĩ˩ lɑ˩-ɲi˩-mæ˩! |}}}
\trans (Quand on se trouve sur) la montagne, les champignons, on les fait simplement cuire dans une casserole! (Littéralement: ``on se contente de les faire bouillir".) (On mettait simplement les champignons dans une casserole, sans eau, avec du sel et de la graisse; les champignons cuisaient alors dans leur propre eau.)
\trans \textit{\textcolor{brown}{\zh{在山上,菌子,就是简单煮一下而已!(放在锅里,加油、加盐。用菌子自身的水分)}}}
\end{exe}


\paragraph{\hspace{-0.5cm} {\Large \textcolor{darkblue}{\textbf{\ipa{tɕi˥}}}}} \hypertarget{ts£i\string_T1}{}
\markboth{\textcolor{darkblue}{\textbf{\ipa{tɕi˥}}}}{}
\textcolor{teal}{\mytextsc{verbe}} \hspace{4pt} Ton~: H.

Secouer (ex.: pour défroisser des vêtements après lavage; aussi: secouer la tête).
\textcolor{brown}{\zh{抖、抖动,摇动。}}


\begin{exe}
\sn \textcolor{darkblue}{\textbf{\ipa{le˧-tɕi˧\textasciitilde{}tɕi˧-ze˩}}}
\trans \mytextsc{accomp} \string_ \mytextsc{pfv}
\trans \textit{\textcolor{brown}{\zh{\mytextsc{accomp} \string_ \mytextsc{pfv}}}}
\end{exe}
\begin{exe}
\sn \textcolor{darkblue}{\textbf{\ipa{tʰi˧-tɕi˧\textasciitilde{}tɕi˧+ze˩}}}
\trans \mytextsc{dur} \string_ \mytextsc{pfv}
\trans \textit{\textcolor{brown}{\zh{\mytextsc{dur} \string_ \mytextsc{pfv}}}}
\end{exe}
\begin{exe}
\sn \textcolor{darkblue}{\textbf{\ipa{ʁo˧qʰwɤ˩ tɕi˩\textasciitilde{}tɕi˩}}}
\trans agiter la tête, secouer la tête
\trans \textit{\textcolor{brown}{\zh{摇头}}}
\end{exe}
\begin{exe}
\sn \textcolor{darkblue}{\textbf{\ipa{ɖɯ˧-tɕi˧\textasciitilde{}tɕi˧-ɻ̍˥}}}
\trans \mytextsc{délimitatif} \string_ \mytextsc{red} \mytextsc{inchoatif}
\trans \textit{\textcolor{brown}{\zh{摇一摇}}}
\end{exe}


\paragraph{\hspace{-0.5cm} {\Large \textcolor{darkblue}{\textbf{\ipa{tɕi˧}}} \textsubscript{1}}} \hypertarget{ts£i\string_M1}{}
\markboth{\textcolor{darkblue}{\textbf{\ipa{tɕi˧}}} \textsubscript{1}}{}
\textcolor{teal}{\mytextsc{adjectif}} \hspace{4pt} Ton~: M.
\ding{202} 
Acide.
\textcolor{brown}{\zh{酸。}}


\begin{exe}
\sn \textcolor{darkblue}{\textbf{\ipa{tɕʰɯ˩-hĩ˩˥}}}
\end{exe}
\begin{exe}
\sn \textcolor{darkblue}{\textbf{\ipa{[M18] tɕi˧-hĩ˧ pʰi˩}}}
\trans avoir des remontées acides
\trans \textit{\textcolor{brown}{\zh{吐酸水}}}
\end{exe}
\ding{203} 
Fermenté.
\textcolor{brown}{\zh{(通过发酵的)酸。}}




\paragraph{\hspace{-0.5cm} {\Large \textcolor{darkblue}{\textbf{\ipa{tɕi˧}}} \textsubscript{2}}} \hypertarget{ts£i\string_M2}{}
\markboth{\textcolor{darkblue}{\textbf{\ipa{tɕi˧}}} \textsubscript{2}}{}
\textcolor{teal}{\mytextsc{nom}} \hspace{4pt} Ton~: M.

Piège.
\textcolor{brown}{\zh{圈套。}}


\begin{exe}
\sn \textcolor{darkblue}{\textbf{\ipa{tɕi˧ kʰɯ˧˥}}}
\trans poser un piège
\trans \textit{\textcolor{brown}{\zh{设下圈套}}}
\end{exe}
 \mytextsc{clf}~: \textcolor{darkblue}{\textbf{\ipa{ɭɯ˧}}} 

\paragraph{\hspace{-0.5cm} {\Large \textcolor{darkblue}{\textbf{\ipa{tɕi˧\textsubscript{a}}}}}} \hypertarget{ts£i\string_Ma1}{}
\markboth{\textcolor{darkblue}{\textbf{\ipa{tɕi˧\textsubscript{a}}}}}{}
\textcolor{teal}{\mytextsc{classificateur}} \hspace{4pt} Ton~: M\textsubscript{a}.

Quelques-uns, certains, une partie.
\textcolor{brown}{\zh{量词:一些。}}


\begin{exe}
\sn \textcolor{darkblue}{\textbf{\ipa{ɖɯ˧-tɕi˧}}}
\trans quelques-uns, certains
\trans \textit{\textcolor{brown}{\zh{一些}}}
\end{exe}
\begin{exe}
\sn \textcolor{darkblue}{\textbf{\ipa{ʈʂʰɯ˧-tɕi˩}}}
\trans ceux-ci
\trans \textit{\textcolor{brown}{\zh{这些}}}
\end{exe}


\paragraph{\hspace{-0.5cm} {\Large \textcolor{darkblue}{\textbf{\ipa{tɕi˧do˩}}}}} \hypertarget{ts£i\string_Mdo\string_B1}{}
\markboth{\textcolor{darkblue}{\textbf{\ipa{tɕi˧do˩}}}}{}
\textcolor{teal}{\mytextsc{nom}} \hspace{4pt} Ton~: L\#.

Mandarine.
\textcolor{brown}{\zh{橘子。}}Dialecte chinois local~: \zh{黄果。}


 \mytextsc{clf}~: \textcolor{darkblue}{\textbf{\ipa{ɭɯ˧}}} 

\paragraph{\hspace{-0.5cm} {\Large \textcolor{darkblue}{\textbf{\ipa{tɕi˧-dʑɯ˩}}}}} \hypertarget{ts£i\string_M-dz£M\string_B1}{}
\markboth{\textcolor{darkblue}{\textbf{\ipa{tɕi˧-dʑɯ˩}}}}{}
\textcolor{teal}{\mytextsc{nom}} \hspace{4pt} Ton~: L\#.
\ding{202} 
Potion acide: une préparation à base de prunelles acides ou baies sauvages, utilisée pour faire vomir les personnes victimes d'un empoisonnement alimentaire (par exemple par des champignons vénéneux).
\textcolor{brown}{\zh{用梅子等野生果子做出来的一种药品(酸水),食物中毒的情况下给病人和这种酸水让他呕吐。}}


\ding{203} 
Vinaigre.
\textcolor{brown}{\zh{醋。}}




\paragraph{\hspace{-0.5cm} {\Large \textcolor{darkblue}{\textbf{\ipa{tɕi˧kwɤ˧}}}}} \hypertarget{ts£i\string_Mkw7\string_M1}{}
\markboth{\textcolor{darkblue}{\textbf{\ipa{tɕi˧kwɤ˧}}}}{}
\textcolor{teal}{\mytextsc{nom}} \hspace{4pt} Ton~: M.

Courge (inclut les courgettes).
\textcolor{brown}{\zh{瓜。}}


\begin{exe}
\sn \textcolor{darkblue}{\textbf{\ipa{tɕi˧kwɤ˧ bv̩˧-ɻ̍˧ (+ɲi˩)}}}
\trans petite courge
\trans \textit{\textcolor{brown}{\zh{小瓜}}}
\end{exe}
\begin{exe}
\sn \textcolor{darkblue}{\textbf{\ipa{tɕi˧kwɤ˧ kwɤ˧mo˩}}}
\trans grosse courge
\trans \textit{\textcolor{brown}{\zh{大瓜}}}
\end{exe}
 \mytextsc{clf}~: \textcolor{darkblue}{\textbf{\ipa{ɭɯ˧}}} 

\paragraph{\hspace{-0.5cm} {\Large \textcolor{darkblue}{\textbf{\ipa{tɕi˧sɯ˧pɤ˧}}}}} \hypertarget{ts£i\string_MsM\string_Mp7\string_M1}{}
\markboth{\textcolor{darkblue}{\textbf{\ipa{tɕi˧sɯ˧pɤ˧}}}}{}
\textcolor{teal}{\mytextsc{nom}} \hspace{4pt} Ton~: M.

Fromage au lait de yak. On commençait par écrémer le lait, puis on faisait bouillir, avec un additif pour le faire cailler; enfin la préparation se solidifiait. Cette préparation était utilisée dans l'alimentation: on en mettait dans les bouillies de céréales. Elle pouvait se conserver. Ce fromage, acide et dur, était recommandé aux personnes ayant des soucis digestifs (comme remède à la diarrhée), et aux personnes âgées.
\textcolor{brown}{\zh{牦牛奶酪。}}


\begin{exe}
\sn \textcolor{darkblue}{\textbf{\ipa{mv̩˧ɭɯ˩-pʰɤ˩bɤ˩, | tɕi˧sɯ˧pɤ˧!}}}
\trans Le cadeau (qu'on ramène de Muli), c'est le fromage de yak! / La spécialité de Muli, c'est le fromage de yak! (Autrefois, c'était un des cadeaux que les jeunes gens offraient aux jeunes filles au retour de leurs voyages.)
\trans \textit{\textcolor{brown}{\zh{木里的礼物:牦牛的奶酪! / 牦牛奶酪,是木里的特产!}}}
\end{exe}
\begin{exe}
\sn \textcolor{darkblue}{\textbf{\ipa{mv̩˧ɭɯ˩ pʰɤ˩bɤ˩, | tɕi˧sɯ˧pɤ˧! | ə˧ɖo˧ ʁo˧ ɖʐɯ˥\textasciitilde{}ɖʐɯ˩ ʝi˩-ze˩!}}}
\trans Le cadeau (qu'on ramène de Muli), c'est le fromage de yak! Ma bien-aimée va secouer la tête (lorsqu'elle goûtera à ce fromage, très acide)! (Paroles d'une chanson qu'on chantait en chemin, en imaginant le retour.)
\trans \textit{\textcolor{brown}{\zh{(从)木里(带回来)的礼物,就是牦牛奶酪!亲爱的(=收礼物的那个人),会摇头的!(吃、喝的时候会摇头,是因为牦牛奶奶酪比较酸)}}}
\end{exe}
\begin{exe}
\sn \textcolor{darkblue}{\textbf{\ipa{tɕi˧sɯ˧pɤ˧, | ɖɯ˧-tɑ˧˥ | gv̩˧-mɤ˧-kv̩˥! | ʝi˧-kʰv̩˥-lɑ˩ gv̩˩-kv̩˩!}}}
\trans Ce n'est pas tout le monde qui savait faire du fromage de yak! Il n'y a que certaines (personnes/familles) qui savaient le faire!
\trans \textit{\textcolor{brown}{\zh{不是每个人都会做牦牛奶酪!只有少数(人)才会做!}}}
\end{exe}
\begin{exe}
\sn \textcolor{darkblue}{\textbf{\ipa{tɕi˧sɯ˧pɤ˧-dʑɯ˩}}}
\trans eau dans laquelle on a dilué du fromage de yak; elle a des propriétés médicinales
\trans \textit{\textcolor{brown}{\zh{一种饮料:将牦牛奶酪溶化在水里}}}
\end{exe}
\begin{exe}
\sn \textcolor{darkblue}{\textbf{\ipa{tɕi˧sɯ˧pɤ˧ ʈʰɯ˩}}}
\trans boire de l'eau dans laquelle on a dilué du fromage de yak; littéralement: 'boire du fromage de yak'
\trans \textit{\textcolor{brown}{\zh{喝溶化在水里的牦牛奶酪(直译:喝牦牛奶酪)}}}
\end{exe}


\paragraph{\hspace{-0.5cm} {\Large \textcolor{darkblue}{\textbf{\ipa{tɕi˧tɕi˧ | læ˩sæ˧-dzi˩}}}}} \hypertarget{ts£i\string_Mts£i\string_M | l\{\string_Bs\{\string_M-dzi\string_B1}{}
\markboth{\textcolor{darkblue}{\textbf{\ipa{tɕi˧tɕi˧ | læ˩sæ˧-dzi˩}}}}{}
\textcolor{teal}{\mytextsc{nom}} \hspace{4pt} Ton~: M-LH-L.

Un arbre au bois très dur.
\textcolor{brown}{\zh{一种树,木质很硬。}}




\paragraph{\hspace{-0.5cm} {\Large \textcolor{darkblue}{\textbf{\ipa{tɕi˩\textsubscript{a}}}}}} \hypertarget{ts£i\string_Ba1}{}
\markboth{\textcolor{darkblue}{\textbf{\ipa{tɕi˩\textsubscript{a}}}}}{}
\textcolor{teal}{\mytextsc{adjectif}} \hspace{4pt} Ton~: L\textsubscript{a}.

Petit.
\textcolor{brown}{\zh{矮,低,小。}}


\begin{exe}
\sn \textcolor{darkblue}{\textbf{\ipa{tɕi˩-hĩ˩˥}}}
\trans (qui est) petit
\trans \textit{\textcolor{brown}{\zh{矮的}}}
\end{exe}
\begin{exe}
\sn \textcolor{darkblue}{\textbf{\ipa{gv̩˧mi˧ tɕi˩}}}
\trans de petite taille
\trans \textit{\textcolor{brown}{\zh{矮}}}
\end{exe}


\paragraph{\hspace{-0.5cm} {\Large \textcolor{darkblue}{\textbf{\ipa{tɕi˩nv̩˧˥}}}}} \hypertarget{ts£i\string_Bnv\string_=\string_M\string_T1}{}
\markboth{\textcolor{darkblue}{\textbf{\ipa{tɕi˩nv̩˧˥}}}}{}
\textcolor{teal}{\mytextsc{nom}} \hspace{4pt} Ton~: LM+MH\#.

Tapis de selle.
\textcolor{brown}{\zh{马鞍下面的毯子。}}


\begin{exe}
\sn \textcolor{darkblue}{\textbf{\ipa{ʐwæ˧-tɕi˥nv̩˩}}}
\trans tapis de selle de cheval
\trans \textit{\textcolor{brown}{\zh{马鞍毯子}}}
\end{exe}
 \mytextsc{clf}~: \textcolor{darkblue}{\textbf{\ipa{pɤ˩}}} 

\paragraph{\hspace{-0.5cm} {\Large \textcolor{darkblue}{\textbf{\ipa{tɕi˩qɑ˥}}}}} \hypertarget{ts£i\string_BqA\string_T1}{}
\markboth{\textcolor{darkblue}{\textbf{\ipa{tɕi˩qɑ˥}}}}{}
\textcolor{teal}{\mytextsc{nom}} \hspace{4pt} Ton~: LH.

Tapis.
\textcolor{brown}{\zh{毯子。}}


 \mytextsc{clf}~: \textcolor{darkblue}{\textbf{\ipa{ɭɯ˧}}} 

\paragraph{\hspace{-0.5cm} {\Large \textcolor{darkblue}{\textbf{\ipa{tɕi˧˥}}}}} \hypertarget{ts£i\string_M\string_T1}{}
\markboth{\textcolor{darkblue}{\textbf{\ipa{tɕi˧˥}}}}{}
\textcolor{teal}{\mytextsc{verbe}} \hspace{4pt} Ton~: .

Inviter (des hôtes) à boire du vin.
\textcolor{brown}{\zh{敬(酒)。}}


\begin{exe}
\sn \textcolor{darkblue}{\textbf{\ipa{ʐɯ˧ tɕi˧˥}}}
\trans inviter (des hôtes) à boire du vin
\trans \textit{\textcolor{brown}{\zh{敬酒}}}
\end{exe}


\paragraph{\hspace{-0.5cm} {\Large \textcolor{darkblue}{\textbf{\ipa{tɕi˩˥}}}}} \hypertarget{ts£i\string_B\string_T1}{}
\markboth{\textcolor{darkblue}{\textbf{\ipa{tɕi˩˥}}}}{}
\textcolor{teal}{\mytextsc{nom}} \hspace{4pt} Ton~: LH.

Selle.
\textcolor{brown}{\zh{马鞍。}}


\begin{exe}
\sn \textcolor{darkblue}{\textbf{\ipa{ʐwæ˧-tɕi˥}}}
\trans selle de cheval
\trans \textit{\textcolor{brown}{\zh{马鞍}}}
\end{exe}
 \mytextsc{clf}~: \textcolor{darkblue}{\textbf{\ipa{pɤ˩}}} 

\paragraph{\hspace{-0.5cm} {\Large \textcolor{darkblue}{\textbf{\ipa{tɕo˥}}}}} \hypertarget{ts£o\string_T1}{}
\markboth{\textcolor{darkblue}{\textbf{\ipa{tɕo˥}}}}{}
\textcolor{teal}{\mytextsc{nom}} \hspace{4pt} Ton~: H.

Sens, direction.
\textcolor{brown}{\zh{方向。}}


\begin{exe}
\sn \textcolor{darkblue}{\textbf{\ipa{ʈʂʰɯ˧-tɕo˧}}}
\trans dans cette direction-ci
\trans \textit{\textcolor{brown}{\zh{这个方向,向这里}}}
\end{exe}
\begin{exe}
\sn \textcolor{darkblue}{\textbf{\ipa{ɖɯ˧-tɕo˥}}}
\trans d'un côté, dans une direction
\trans \textit{\textcolor{brown}{\zh{一边}}}
\end{exe}
\begin{exe}
\sn \textcolor{darkblue}{\textbf{\ipa{gɤ˩-tɕo˧}}}
\trans vers le haut
\trans \textit{\textcolor{brown}{\zh{向上,往上}}}
\end{exe}
\begin{exe}
\sn \textcolor{darkblue}{\textbf{\ipa{dv̩˩tɕo˧}}}
\trans dans cette direction-là
\trans \textit{\textcolor{brown}{\zh{那边}}}
\end{exe}


\paragraph{\hspace{-0.5cm} {\Large \textcolor{darkblue}{\textbf{\ipa{tɕo˩ɕjo˧}}}}} \hypertarget{ts£o\string_Bs£jo\string_M1}{}
\markboth{\textcolor{darkblue}{\textbf{\ipa{tɕo˩ɕjo˧}}}}{}
\textcolor{teal}{\mytextsc{nom}} \hspace{4pt} Ton~: LM.

Sifflement.
\textcolor{brown}{\zh{口哨。}}


\begin{exe}
\sn \textcolor{darkblue}{\textbf{\ipa{tɕo˩ɕjo˧ | ɖɯ˧-ɖʐo˩ kʰɯ˩}}}
\trans siffler un air, siffler un coup
\trans \textit{\textcolor{brown}{\zh{吹口哨、吹一声口哨}}}
\end{exe}


\paragraph{\hspace{-0.5cm} {\Large \textcolor{darkblue}{\textbf{\ipa{tɕo˩mv̩˧}}}}} \hypertarget{ts£o\string_Bmv\string_=\string_M1}{}
\markboth{\textcolor{darkblue}{\textbf{\ipa{tɕo˩mv̩˧}}}}{}
\textcolor{teal}{\mytextsc{nom}} \hspace{4pt} Ton~: LM.

Femme de l'oncle maternel; constitué d'un emprunt chinois, \textcolor{brown}{\zh{舅}} 'oncle maternel', et d'un mot na: 'femme'.
\textcolor{brown}{\zh{舅妈(舅:汉语借词,妈:摩梭话“女人”)。}}

 Emprunt~: chinois \textcolor{brown}{\zh{舅}}

 \mytextsc{clf}~: \textcolor{darkblue}{\textbf{\ipa{v̩˧}}} 

\paragraph{\hspace{-0.5cm} {\Large \textcolor{darkblue}{\textbf{\ipa{‑tɕɯ}}}}} \hypertarget{‑ts£M1}{}
\markboth{\textcolor{darkblue}{\textbf{\ipa{‑tɕɯ}}}}{}
\textcolor{teal}{\mytextsc{suffixe}} \hspace{4pt} Ton~: .

Forme grammaticalisée de \textcolor{darkblue}{\textbf{\ipa{/tɕɯ˥/}}} 'mettre, placer'; elle exprime que l'action est finie, que son terme est maintenant dépassé, et qu'on peut passer à autre chose: de même que, une fois un objet posé à sa place, on peut tourner son attention vers un autre.
\textcolor{brown}{\zh{表示:已完成,可以轮到其它的了。}}


\begin{exe}
\sn \textcolor{darkblue}{\textbf{\ipa{[``Dog"] gɤ˩-ʈʂʰwæ˧-tɕɯ˧˥ / gɤ˩-ʈʂʰwæ˧-tɕɯ˧-zo˥}}}
\trans être éveillé, se réveiller
\trans \textit{\textcolor{brown}{\zh{醒过来、醒来}}}
\end{exe}
\begin{exe}
\sn \textcolor{darkblue}{\textbf{\ipa{ʈʂʰɯ˧-qo˧ dzi˩-tɕɯ˩-zo˩}}}
\trans s'asseoir ici
\trans \textit{\textcolor{brown}{\zh{在这里坐下来}}}
\end{exe}
\begin{exe}
\sn \textcolor{darkblue}{\textbf{\ipa{ʐɯ˧ ɖɯ˧-qʰwɤ˧ tʰi˧-mv̩˧-ʈʰɯ˧˥ | -tɕɯ˩-zo˩!}}}
\trans Bois donc ce bol de vin!
\trans \textit{\textcolor{brown}{\zh{把这碗酒喝了下去!}}}
\end{exe}


\paragraph{\hspace{-0.5cm} {\Large \textcolor{darkblue}{\textbf{\ipa{tɕɯ˥}}}}} \hypertarget{ts£M\string_T1}{}
\markboth{\textcolor{darkblue}{\textbf{\ipa{tɕɯ˥}}}}{}
\textcolor{teal}{\mytextsc{verbe}} \hspace{4pt} Ton~: H.
\ding{202} 
Poser, ranger, mettre, placer.
\textcolor{brown}{\zh{放置。}}


\begin{exe}
\sn \textcolor{darkblue}{\textbf{\ipa{tʰi˧-tɕɯ˥}}}
\trans \mytextsc{dur}
\trans \textit{\textcolor{brown}{\zh{\mytextsc{dur}}}}
\end{exe}
\begin{exe}
\sn \textcolor{darkblue}{\textbf{\ipa{[F5] ɖɯ˩hĩ˧ | ɖɯ˩˧ | tʰi˧-tɕɯ˥, | tɕi˩hĩ˧ | tɕi˩˧ | tʰi˧-tɕɯ˥}}}
\trans mettre les grands avec les grands, les petits avec les petits
\trans \textit{\textcolor{brown}{\zh{大小归类}}}
\end{exe}
\ding{203} 
Fixer, décider (ex.: les puissances suprêmes fixent la durée de la vie humaine).
\textcolor{brown}{\zh{决定、定下来。}}


\begin{exe}
\sn \textcolor{darkblue}{\textbf{\ipa{le˧-ʐwɤ˩ | tʰi˧-tɕɯ˥}}}
\trans fixer; arrêter; décider que
\trans \textit{\textcolor{brown}{\zh{说好、决定}}}
\end{exe}
\begin{exe}
\sn \textcolor{darkblue}{\textbf{\ipa{le˧-ʐwɤ˩ | tʰi˧-tɕɯ˧-ɲi˥-tsɯ˩!}}}
\trans C'est fixé/c'est décidé/c'est arrêté!
\trans \textit{\textcolor{brown}{\zh{说好了! / 决定好了!}}}
\end{exe}


\paragraph{\hspace{-0.5cm} {\Large \textcolor{darkblue}{\textbf{\ipa{tɕɯ˧}}}}} \hypertarget{ts£M\string_M1}{}
\markboth{\textcolor{darkblue}{\textbf{\ipa{tɕɯ˧}}}}{}
\textcolor{teal}{\mytextsc{nom}} \hspace{4pt} Ton~: M.

Nuage.
\textcolor{brown}{\zh{云。}}


\begin{exe}
\sn \textcolor{darkblue}{\textbf{\ipa{mv̩˧tɕɯ˥}}}
\trans il y a des nuages, le temps est nuageux
\trans \textit{\textcolor{brown}{\zh{天上多云}}}
\end{exe}
\begin{exe}
\sn \textcolor{darkblue}{\textbf{\ipa{mv̩˧ʁo˥, | tɕɯ˧!}}}
\trans le ciel est nuageux!
\trans \textit{\textcolor{brown}{\zh{天上有云!}}}
\end{exe}
\begin{exe}
\sn \textcolor{darkblue}{\textbf{\ipa{mv̩˧ʁo˥ tɕɯ˩ pʰv̩˩ |}}}
\trans le ciel est nuageux!
\trans \textit{\textcolor{brown}{\zh{天上有云!}}}
\end{exe}
\begin{exe}
\sn \textcolor{darkblue}{\textbf{\ipa{tɕɯ˧pʰv̩˩; tɕɯ˧ | pʰv̩˩tɕæ˩˥ | -gv̩˩}}}
\trans nuage blanc
\trans \textit{\textcolor{brown}{\zh{白云、白色的云}}}
\end{exe}
\begin{exe}
\sn \textcolor{darkblue}{\textbf{\ipa{mv̩˧nɑ˥-tɕɯ˩nɑ˩-ɻ̍˩!}}}
\trans il fait sombre/ le ciel est très nuageux!
\trans \textit{\textcolor{brown}{\zh{天很黑,有很多乌云}}}
\end{exe}
 \mytextsc{clf}~: \textcolor{darkblue}{\textbf{\ipa{kʰwɤ˥}}} morceaux

\paragraph{\hspace{-0.5cm} {\Large \textcolor{darkblue}{\textbf{\ipa{tɕɯ˧\textsubscript{b}}}}}} \hypertarget{ts£M\string_Mb1}{}
\markboth{\textcolor{darkblue}{\textbf{\ipa{tɕɯ˧\textsubscript{b}}}}}{}
\textcolor{teal}{\mytextsc{verbe}} \hspace{4pt} Ton~: M\textsubscript{b}.

Secouer (monosyllabe).
\textcolor{brown}{\zh{摇晃。}}


\begin{exe}
\sn \textcolor{darkblue}{\textbf{\ipa{tso˧\textasciitilde{}tso˧ tɕɯ˧}}}
\trans secouer des choses
\trans \textit{\textcolor{brown}{\zh{摇东西}}}
\end{exe}


\paragraph{\hspace{-0.5cm} {\Large \textcolor{darkblue}{\textbf{\ipa{tɕɯ˧ɭɯ˧}}}}} \hypertarget{ts£M\string_Ml\string_RM\string_M1}{}
\markboth{\textcolor{darkblue}{\textbf{\ipa{tɕɯ˧ɭɯ˧}}}}{}
\textcolor{teal}{\mytextsc{verbe}} \hspace{4pt} Ton~: M.

Enrouler, embobiner.
\textcolor{brown}{\zh{缠绕。}}


\begin{exe}
\sn \textcolor{darkblue}{\textbf{\ipa{njɤ˧-ɳɯ˧ | tɕɯ˧ɭɯ˧-bi˧!}}}
\trans Je me charge d'enrouler! / C'est moi qui vais enrouler!
\trans \textit{\textcolor{brown}{\zh{让我来缠吧!}}}
\end{exe}


\paragraph{\hspace{-0.5cm} {\Large \textcolor{darkblue}{\textbf{\ipa{tɕɯ˧mi˥\$}}}}} \hypertarget{ts£M\string_Mmi\string_T\$1}{}
\markboth{\textcolor{darkblue}{\textbf{\ipa{tɕɯ˧mi˥\$}}}}{}
\textcolor{teal}{\mytextsc{nom}} \hspace{4pt} Ton~: H\$.

Grande balance.
\textcolor{brown}{\zh{大称。}}




\paragraph{\hspace{-0.5cm} {\Large \textcolor{darkblue}{\textbf{\ipa{tɕɯ˧pv̩˧}}}}} \hypertarget{ts£M\string_Mpv\string_=\string_M1}{}
\markboth{\textcolor{darkblue}{\textbf{\ipa{tɕɯ˧pv̩˧}}}}{}
\textcolor{teal}{\mytextsc{adjectif}} \hspace{4pt} Ton~: .

À l'aise, peinard.
\textcolor{brown}{\zh{轻松快乐、舒畅。}}


\begin{exe}
\sn \textcolor{darkblue}{\textbf{\ipa{ʈʂʰɯ˧qo˧ | tɕɯ˧pv̩˧-ʂe˧\textasciitilde{}ʂe˧ | ɖɯ˧-dzi˩-zo˩-ho˩!}}}
\trans assieds-toi ici, bien peinard!
\trans \textit{\textcolor{brown}{\zh{在这边舒畅地坐一会吧!}}}
\end{exe}
\begin{exe}
\sn \textcolor{darkblue}{\textbf{\ipa{ʈʂʰɯ˧qo˧ | tɕɯ˧pv̩˧-ʂe˧\textasciitilde{}ʂe˧-zo˥ | ɖɯ˧-dzi˩-bi˩-ɻ̍˩!}}}
\trans asseyons-nous donc ici, bien peinards!
\trans \textit{\textcolor{brown}{\zh{在这边舒畅地坐一会吧!}}}
\end{exe}


\paragraph{\hspace{-0.5cm} {\Large \textcolor{darkblue}{\textbf{\ipa{tɕɯ˧sɯ˧˥}}}}} \hypertarget{ts£M\string_MsM\string_M\string_T1}{}
\markboth{\textcolor{darkblue}{\textbf{\ipa{tɕɯ˧sɯ˧˥}}}}{}
\textcolor{teal}{\mytextsc{nom}} \hspace{4pt} Ton~: MH\#.

Brume.
\textcolor{brown}{\zh{雾。}}


\begin{exe}
\sn \textcolor{darkblue}{\textbf{\ipa{tɕɯ˧sɯ˧mv̩˥}}}
\trans il y a de la brume
\trans \textit{\textcolor{brown}{\zh{有雾}}}
\end{exe}
 \mytextsc{clf}~: \textcolor{darkblue}{\textbf{\ipa{ti˧˥}}} une couche de

\paragraph{\hspace{-0.5cm} {\Large \textcolor{darkblue}{\textbf{\ipa{tɕɯ˧wɤ˧}}}}} \hypertarget{ts£M\string_Mw7\string_M1}{}
\markboth{\textcolor{darkblue}{\textbf{\ipa{tɕɯ˧wɤ˧}}}}{}
\textcolor{teal}{\mytextsc{verbe}} \hspace{4pt} Ton~: M.

Se réincarner.
\textcolor{brown}{\zh{转生、转世。}}


\begin{exe}
\sn \textcolor{darkblue}{\textbf{\ipa{le˧-tɕɯ˧wɤ˧-ho˥!}}}
\trans (La défunte / le défunt) va se réincarner!
\trans \textit{\textcolor{brown}{\zh{他要转生了!}}}
\end{exe}


\paragraph{\hspace{-0.5cm} {\Large \textcolor{darkblue}{\textbf{\ipa{tɕɯ˧zo˥\$}}}}} \hypertarget{ts£M\string_Mzo\string_T\$1}{}
\markboth{\textcolor{darkblue}{\textbf{\ipa{tɕɯ˧zo˥\$}}}}{}
\textcolor{teal}{\mytextsc{nom}} \hspace{4pt} Ton~: H\$.

Petite balance.
\textcolor{brown}{\zh{小称。}}




\paragraph{\hspace{-0.5cm} {\Large \textcolor{darkblue}{\textbf{\ipa{tɕɯ˩\textsubscript{a}}}}}} \hypertarget{ts£M\string_Ba1}{}
\markboth{\textcolor{darkblue}{\textbf{\ipa{tɕɯ˩\textsubscript{a}}}}}{}
\textcolor{teal}{\mytextsc{verbe}} \hspace{4pt} Ton~: L\textsubscript{a}.

Écrire.
\textcolor{brown}{\zh{写。}}


\begin{exe}
\sn \textcolor{darkblue}{\textbf{\ipa{le˧-tɕɯ˩-ze˩}}}
\trans \mytextsc{accomp}+\mytextsc{pfv}
\trans \textit{\textcolor{brown}{\zh{写了}}}
\end{exe}
\begin{exe}
\sn \textcolor{darkblue}{\textbf{\ipa{tʰæ˧ɻæ˩ tɕɯ˩}}}
\trans écrire quelque chose/écrire du texte/écrire un livre
\trans \textit{\textcolor{brown}{\zh{写、写书}}}
\end{exe}
\begin{exe}
\sn \textcolor{darkblue}{\textbf{\ipa{ɖɯ˧-kʰv̩˥ | tsʰe˧-ɲi˧ ɬi˧, | njɤ˧ | tsʰe˧-ɲi˧ bæ˧ tɕɯ˩-bi˩-ʂv̩˩ɖv̩˩!}}}
\trans Dans une année il y a douze mois; je voudrais transcrire douze histoires (au cours de l'année qui vient)! (contexte: en septembre 2011, Ama remarque que j'ai transcrit deux contes en deux mois; en m'offrant cet exemple, elle me souffle le projet de garder le rythme et de transcrire une histoire par mois soit douze pendant l'année qui vient)
\trans \textit{\textcolor{brown}{\zh{一年有十二个月,我就想(一年之内)记十二个故事!(情景:我两个月内完成了两个故事的记录工作。发音合作人举这个例句,鼓励我坚持这种速度,一年内再记十二个故事。)}}}
\end{exe}
\begin{exe}
\sn \textcolor{darkblue}{\textbf{\ipa{ɖɯ˧-tɕɯ˧\textasciitilde{}tɕɯ˥-ɻ̍˩}}}
\trans \mytextsc{délimitatif} \string_ \mytextsc{red} \mytextsc{inchoatif}
\trans \textit{\textcolor{brown}{\zh{\mytextsc{delimitative} \string_ \mytextsc{red} \mytextsc{inceptive}}}}
\end{exe}
\begin{exe}
\sn \textcolor{darkblue}{\textbf{\ipa{tɕɯ˩-di˩˥}}}
\trans pinceau; littéralement 'chose pour écrire'
\trans \textit{\textcolor{brown}{\zh{笔。直译:‘(用来)书写的(东西)’}}}
\end{exe}
\begin{exe}
\sn \textcolor{darkblue}{\textbf{\ipa{tʰæ˧ɻæ˩-tɕɯ˩-di˩}}}
\trans pinceau; littéralement 'chose pour écrire des livres'
\trans \textit{\textcolor{brown}{\zh{笔。直译:‘(用来)写书的(东西)’}}}
\end{exe}
\begin{exe}
\sn \textcolor{darkblue}{\textbf{\ipa{ʈʂʰɯ˧ | tʰi˧-tɕɯ˧\textasciitilde{}tɕɯ˥ dʑo˩}}}
\trans Elle/il est en train d'écrire
\trans \textit{\textcolor{brown}{\zh{他正在写写东西。}}}
\end{exe}


\paragraph{\hspace{-0.5cm} {\Large \textcolor{darkblue}{\textbf{\ipa{tɕɯ˩lv̩˩ho˥}}}}} \hypertarget{ts£M\string_Blv\string_=\string_Bho\string_T1}{}
\markboth{\textcolor{darkblue}{\textbf{\ipa{tɕɯ˩lv̩˩ho˥}}}}{}
\textcolor{teal}{\mytextsc{nom}} \hspace{4pt} Ton~: L+H\#.

Fronde (bande de tissu permettant de catapulter un objet).
\textcolor{brown}{\zh{弹弓。}}


 \mytextsc{clf}~: \textcolor{darkblue}{\textbf{\ipa{ɭɯ˧}}} 

\paragraph{\hspace{-0.5cm} {\Large \textcolor{darkblue}{\textbf{\ipa{tɕɯ˩ɭɯ˩}}}}} \hypertarget{ts£M\string_Bl\string_RM\string_B1}{}
\markboth{\textcolor{darkblue}{\textbf{\ipa{tɕɯ˩ɭɯ˩}}}}{}
\textcolor{teal}{\mytextsc{nom}} \hspace{4pt} Ton~: L.

Pie grièche du Tibet, \textit{Lanius tephronotus}.
\textcolor{brown}{\zh{伯劳鸟。}}


 \mytextsc{clf}~: \textcolor{darkblue}{\textbf{\ipa{mi˩}}} 

\paragraph{\hspace{-0.5cm} {\Large \textcolor{darkblue}{\textbf{\ipa{tɕɯ˩ɭɯ˩-qʰæ˥bæ˩}}}}} \hypertarget{ts£M\string_Bl\string_RM\string_B-q\string_h\{\string_Tb\{\string_B1}{}
\markboth{\textcolor{darkblue}{\textbf{\ipa{tɕɯ˩ɭɯ˩-qʰæ˥bæ˩}}}}{}
\textcolor{teal}{\mytextsc{nom}} \hspace{4pt} Ton~: L+\#H-.

Littéralement: ``cuillère de l'oiseau yyyy shrike"; mettre des renvois.
\textcolor{brown}{\zh{终石藤。}}


 \mytextsc{clf}~: \textcolor{darkblue}{\textbf{\ipa{dzi˩}}} 

\paragraph{\hspace{-0.5cm} {\Large \textcolor{darkblue}{\textbf{\ipa{tɕɯ˩mi˥}}}}} \hypertarget{ts£M\string_Bmi\string_T1}{}
\markboth{\textcolor{darkblue}{\textbf{\ipa{tɕɯ˩mi˥}}}}{}
\textcolor{teal}{\mytextsc{nom}} \hspace{4pt} Ton~: LH.

Passereau de la famille des Leiothrichidae: (\textit{Leucodioptron canorum}). Le nom ``hwamei" signifie «sourcils peints» en référence à la marque distinctive autour des yeux de l'oiseau.
\textcolor{brown}{\zh{画眉鸟。}}Dialecte chinois local~: \zh{画眉鸟。}


\begin{exe}
\sn \textcolor{darkblue}{\textbf{\ipa{tɕɯ˩mi˥ | ə˧mi˧ ɲi˩!}}}
\trans c'est une maman hwamei! (=une femelle)
\trans \textit{\textcolor{brown}{\zh{是一个画眉鸟妈妈!(=是母的画眉鸟)}}}
\end{exe}
\begin{exe}
\sn \textcolor{darkblue}{\textbf{\ipa{tɕɯ˩mi˥ | zo˧ ɲi˥!}}}
\trans c'est un petit hwamei! (=un enfant/bébé)
\trans \textit{\textcolor{brown}{\zh{是一个小画眉鸟!}}}
\end{exe}
\begin{exe}
\sn \textcolor{darkblue}{\textbf{\ipa{tɕɯ˩mi˥ | pʰv̩˧ ɲi˩!}}}
\trans C'est un hwamei mâle!
\trans \textit{\textcolor{brown}{\zh{是公的画眉鸟!}}}
\end{exe}
 \mytextsc{clf}~: \textcolor{darkblue}{\textbf{\ipa{mi˩}}} 

\paragraph{\hspace{-0.5cm} {\Large \textcolor{darkblue}{\textbf{\ipa{tɕɯ˧˥}}} \textsubscript{1}}} \hypertarget{ts£M\string_M\string_T1}{}
\markboth{\textcolor{darkblue}{\textbf{\ipa{tɕɯ˧˥}}} \textsubscript{1}}{}
\textcolor{teal}{\mytextsc{verbe}} \hspace{4pt} Ton~: MH.

Transporter à dos d'animaux.
\textcolor{brown}{\zh{驮运。}}


\begin{exe}
\sn \textcolor{darkblue}{\textbf{\ipa{ʐwæ˧ tɕɯ˩}}}
\trans faire du commerce par caravanes, transporter en caravane, organiser une caravane
\trans \textit{\textcolor{brown}{\zh{用马驮运、做马帮}}}
\end{exe}
\begin{exe}
\sn \textcolor{darkblue}{\textbf{\ipa{ʐwæ˧ʁo˧ tʰi˧-tɕɯ˧˥}}}
\trans transporter à dos de cheval
\trans \textit{\textcolor{brown}{\zh{用马驮运}}}
\end{exe}
\begin{exe}
\sn \textcolor{darkblue}{\textbf{\ipa{ʐwæ˧-tɕɯ˩-zo˩}}}
\trans caravanier, personne qui va avec les caravanes
\trans \textit{\textcolor{brown}{\zh{加入马帮的男人}}}
\end{exe}


\paragraph{\hspace{-0.5cm} {\Large \textcolor{darkblue}{\textbf{\ipa{tɕɯ˧˥}}} \textsubscript{2}}} \hypertarget{ts£M\string_M\string_T2}{}
\markboth{\textcolor{darkblue}{\textbf{\ipa{tɕɯ˧˥}}} \textsubscript{2}}{}
\textcolor{teal}{\mytextsc{nom}} \hspace{4pt} Ton~: MH.

Sangsue.
\textcolor{brown}{\zh{水蛭、蚂蟥。}}Dialecte chinois local~: \zh{蚂蟥。}


 \mytextsc{clf}~: \textcolor{darkblue}{\textbf{\ipa{mi˩}}} 

\paragraph{\hspace{-0.5cm} {\Large \textcolor{darkblue}{\textbf{\ipa{tɕɯ˧˥}}} \textsubscript{3}}} \hypertarget{ts£M\string_M\string_T3}{}
\markboth{\textcolor{darkblue}{\textbf{\ipa{tɕɯ˧˥}}} \textsubscript{3}}{}
\textcolor{teal}{\mytextsc{nom}} \hspace{4pt} Ton~: MH.

Guêpe.
\textcolor{brown}{\zh{马蜂 (黄蜂)。}}


\begin{exe}
\sn \textcolor{darkblue}{\textbf{\ipa{tɕɯ˧mi˥\$}}}
\trans guêpe femelle (élicité pour le propos de l'étude tonale)
\trans \textit{\textcolor{brown}{\zh{母蚂蜂(人工的词)}}}
\end{exe}
\begin{exe}
\sn \textcolor{darkblue}{\textbf{\ipa{tɕɯ˧pʰv̩\#˥}}}
\trans guêpe mâle (élicité pour le propos de l'étude tonale)
\trans \textit{\textcolor{brown}{\zh{公马蜂}}}
\end{exe}
\begin{exe}
\sn \textcolor{darkblue}{\textbf{\ipa{tɕɯ˧zo\#˥}}}
\trans bébé guêpe (élicité pour le propos de l'étude tonale)
\trans \textit{\textcolor{brown}{\zh{小马蜂}}}
\end{exe}
 \mytextsc{clf}~: \textcolor{darkblue}{\textbf{\ipa{mi˩}}} 

\paragraph{\hspace{-0.5cm} {\Large \textcolor{darkblue}{\textbf{\ipa{tɕɯ˧˥}}} \textsubscript{4}}} \hypertarget{ts£M\string_M\string_T4}{}
\markboth{\textcolor{darkblue}{\textbf{\ipa{tɕɯ˧˥}}} \textsubscript{4}}{}
\textcolor{teal}{\mytextsc{nom}} \hspace{4pt} Ton~: MH.

Balance.
\textcolor{brown}{\zh{称。}}


 \mytextsc{clf}~: \textcolor{darkblue}{\textbf{\ipa{nɑ˧}}} 

\paragraph{\hspace{-0.5cm} {\Large \textcolor{darkblue}{\textbf{\ipa{tɕɯ˧˥\textsubscript{a}}}} \textsubscript{1}}} \hypertarget{ts£M\string_M\string_Ta1}{}
\markboth{\textcolor{darkblue}{\textbf{\ipa{tɕɯ˧˥\textsubscript{a}}}} \textsubscript{1}}{}
\textcolor{teal}{\mytextsc{classificateur}} \hspace{4pt} Ton~: MH\textsubscript{a}.

Classificateur des charges sur une bête de somme.
\textcolor{brown}{\zh{量词:驮子(一匹)。}}




\paragraph{\hspace{-0.5cm} {\Large \textcolor{darkblue}{\textbf{\ipa{tɕɯ˧˥\textsubscript{a}}}} \textsubscript{2}}} \hypertarget{ts£M\string_M\string_Ta2}{}
\markboth{\textcolor{darkblue}{\textbf{\ipa{tɕɯ˧˥\textsubscript{a}}}} \textsubscript{2}}{}
\textcolor{teal}{\mytextsc{classificateur}} \hspace{4pt} Ton~: MH\textsubscript{a}.

Livre (aussi pour les liquides: pinte).
\textcolor{brown}{\zh{量词:斤(用于固体,也用于液体)(汉语借词)。}}

 Emprunt~: chinois \textcolor{brown}{\zh{斤}}

\begin{exe}
\sn \textcolor{darkblue}{\textbf{\ipa{ʐɯ˧ | ɖɯ˧-tɕɯ˧˥}}}
\trans une pinte de vin
\trans \textit{\textcolor{brown}{\zh{一斤酒}}}
\end{exe}


\paragraph{\hspace{-0.5cm} {\Large \textcolor{darkblue}{\textbf{\ipa{tɕɯ˩˥}}}}} \hypertarget{ts£M\string_B\string_T1}{}
\markboth{\textcolor{darkblue}{\textbf{\ipa{tɕɯ˩˥}}}}{}
\textcolor{teal}{\mytextsc{nom}} \hspace{4pt} Ton~: LH.

Salive.
\textcolor{brown}{\zh{口水、唾、唾沫、唾液。}}




\paragraph{\hspace{-0.5cm} {\Large \textcolor{darkblue}{\textbf{\ipa{tɕʰɤ˧ɲi˧-ne˧-ʝi˥}}}}} \hypertarget{ts£\string_h7\string_MJi\string_M-ne\string_M-j££i\string_T1}{}
\markboth{\textcolor{darkblue}{\textbf{\ipa{tɕʰɤ˧ɲi˧-ne˧-ʝi˥}}}}{}
\textcolor{teal}{\mytextsc{adverbe}} \hspace{4pt} Ton~: .

Tous les jours.
\textcolor{brown}{\zh{每天。}}




\paragraph{\hspace{-0.5cm} {\Large \textcolor{darkblue}{\textbf{\ipa{tɕʰɤ˧pɤ˧-mi\#˥}}}}} \hypertarget{ts£\string_h7\string_Mp7\string_M-mi\#\string_T1}{}
\markboth{\textcolor{darkblue}{\textbf{\ipa{tɕʰɤ˧pɤ˧-mi\#˥}}}}{}
\textcolor{teal}{\mytextsc{nom}} \hspace{4pt} Ton~: \#H.

Nom d'une source sacrée, située dans une grotte, sur la montagne \textcolor{darkblue}{\textbf{\ipa{/nɑ˩tsʰi˩/}}}.
\textcolor{brown}{\zh{一处神泉。}}


\begin{exe}
\sn \textcolor{darkblue}{\textbf{\ipa{nɑ˩tsʰi˩˥ | tɕʰɤ˧pɤ˧-mi\#˥}}}
\trans nom complet de la montagne où se trouve la source sacrée
\trans \textit{\textcolor{brown}{\zh{神泉所在山的全称}}}
\end{exe}


\paragraph{\hspace{-0.5cm} {\Large \textcolor{darkblue}{\textbf{\ipa{tɕʰɤ˧ʂo\#˥}}}}} \hypertarget{ts£\string_h7\string_Ms`o\#\string_T1}{}
\markboth{\textcolor{darkblue}{\textbf{\ipa{tɕʰɤ˧ʂo\#˥}}}}{}
\textcolor{teal}{\mytextsc{nom}} \hspace{4pt} Ton~: \#H.

Autel, lieu où on brûle de l'encens (dans la maison: l'autel principal est à l'étage, dans le bâtiment face à la porte de la ferme).
\textcolor{brown}{\zh{祭坛。}}


 \mytextsc{clf}~: \textcolor{darkblue}{\textbf{\ipa{nɑ˧}}} 

\paragraph{\hspace{-0.5cm} {\Large \textcolor{darkblue}{\textbf{\ipa{tɕʰɤ˧ti\#˥}}}}} \hypertarget{ts£\string_h7\string_Mti\#\string_T1}{}
\markboth{\textcolor{darkblue}{\textbf{\ipa{tɕʰɤ˧ti\#˥}}}}{}
\textcolor{teal}{\mytextsc{nom}} \hspace{4pt} Ton~: .

Chörten (reliquaire).
\textcolor{brown}{\zh{佛塔,嘛呢堆。}}




\paragraph{\hspace{-0.5cm} {\Large \textcolor{darkblue}{\textbf{\ipa{tɕʰɤ˧ti\#˥}}}}} \hypertarget{ts£\string_h7\string_Mti\#\string_T1}{}
\markboth{\textcolor{darkblue}{\textbf{\ipa{tɕʰɤ˧ti\#˥}}}}{}
\textcolor{teal}{\mytextsc{nom}} \hspace{4pt} Ton~: \#H.

Stupa, tour.
\textcolor{brown}{\zh{塔。}}


 \mytextsc{clf}~: \textcolor{darkblue}{\textbf{\ipa{ɭɯ˧}}} 

\paragraph{\hspace{-0.5cm} {\Large \textcolor{darkblue}{\textbf{\ipa{tɕʰɤ˧tɕo˩}}}}} \hypertarget{ts£\string_h7\string_Mts£o\string_B1}{}
\markboth{\textcolor{darkblue}{\textbf{\ipa{tɕʰɤ˧tɕo˩}}}}{}
\textcolor{teal}{\mytextsc{nom}} \hspace{4pt} Ton~: L\#.

Scie passe-partout: grande scie avec une poignée à chaque extrémité, maniée par deux bûcherons.
\textcolor{brown}{\zh{双人锯:以前用于把圆木截成板材的大的双人锯(汉语借词)。}}

 Emprunt~: chinois \textcolor{brown}{\zh{??}}



\paragraph{\hspace{-0.5cm} {\Large \textcolor{darkblue}{\textbf{\ipa{tɕʰɤ˧tɕʰɤ˧˥}}}}} \hypertarget{ts£\string_h7\string_Mts£\string_h7\string_M\string_T1}{}
\markboth{\textcolor{darkblue}{\textbf{\ipa{tɕʰɤ˧tɕʰɤ˧˥}}}}{}
\textcolor{teal}{\mytextsc{adverbe}} \hspace{4pt} Ton~: MH\#.

Entièrement, tout à fait, complètement.
\textcolor{brown}{\zh{彻底。}}




\paragraph{\hspace{-0.5cm} {\Large \textcolor{darkblue}{\textbf{\ipa{tɕʰɤ˩lv̩˩}}}}} \hypertarget{ts£\string_h7\string_Blv\string_=\string_B1}{}
\markboth{\textcolor{darkblue}{\textbf{\ipa{tɕʰɤ˩lv̩˩}}}}{}
\textcolor{teal}{\mytextsc{nom}} \hspace{4pt} Ton~: L.

Lis.
\textcolor{brown}{\zh{百合。}}


\begin{exe}
\sn \textcolor{darkblue}{\textbf{\ipa{tɕʰɤ˩lv̩˩-hṽ˩hṽ˩˥}}}
\trans lis cuits au wok
\trans \textit{\textcolor{brown}{\zh{炒百合}}}
\end{exe}
\begin{exe}
\sn \textcolor{darkblue}{\textbf{\ipa{tɕʰɤ˩lv̩˩˥, | kv̩˧-pʰæ˧di˥!}}}
\trans Le lis, ça ressemble à de l'ail!
\trans \textit{\textcolor{brown}{\zh{百合,像大蒜!}}}
\end{exe}
\begin{exe}
\sn \textcolor{darkblue}{\textbf{\ipa{tɕʰɤ˩lv̩˩˥, | dʑɯ˩-nɑ˩mi˩-ʁo˥ dʑɯ˩-nɑ˩mi˩-ʁo˥ di˩-kv̩˩!}}}
\trans Le lis, ça pousse dans la montagne/en haute montagne!
\trans \textit{\textcolor{brown}{\zh{百合长在高山上!}}}
\end{exe}
\begin{exe}
\sn \textcolor{darkblue}{\textbf{\ipa{tɕʰɤ˩lv̩˩˥, | dʑɯ˩-nɑ˩mi˩-ʁo˥ | di˩-kv̩˩˥! |}}}
\trans Le lis, ça pousse dans la montagne/en haute montagne!
\trans \textit{\textcolor{brown}{\zh{百合长在高山上!}}}
\end{exe}


\paragraph{\hspace{-0.5cm} {\Large \textcolor{darkblue}{\textbf{\ipa{tɕʰɤ˩ʈʂv̩˧}}}}} \hypertarget{ts£\string_h7\string_Bt`s`v\string_=\string_M1}{}
\markboth{\textcolor{darkblue}{\textbf{\ipa{tɕʰɤ˩ʈʂv̩˧}}}}{}
\textcolor{teal}{\mytextsc{nom}} \hspace{4pt} Ton~: LM.

Verre pour le vin ou autres liquides (sans anse); en verre ou autre matériau.
\textcolor{brown}{\zh{酒杯。}}


 \mytextsc{clf}~: \textcolor{darkblue}{\textbf{\ipa{ɭɯ˧}}} 

\paragraph{\hspace{-0.5cm} {\Large \textcolor{darkblue}{\textbf{\ipa{tɕʰɤ˩ʈʂv˧}}}}} \hypertarget{ts£\string_h7\string_Bt`s`v\string_M1}{}
\markboth{\textcolor{darkblue}{\textbf{\ipa{tɕʰɤ˩ʈʂv˧}}}}{}
\textcolor{teal}{\mytextsc{nom}} \hspace{4pt} Ton~: LM.

Verre, gobelet.
\textcolor{brown}{\zh{杯子。}}


\begin{exe}
\sn \textcolor{darkblue}{\textbf{\ipa{bo˧ʐæ˧-tɕʰɤ˩ʈʂv˩}}}
\trans gobelet à thé en verre
\trans \textit{\textcolor{brown}{\zh{玻璃茶杯}}}
\end{exe}


\paragraph{\hspace{-0.5cm} {\Large \textcolor{darkblue}{\textbf{\ipa{tɕʰɤ˧˥}}}}} \hypertarget{ts£\string_h7\string_M\string_T1}{}
\markboth{\textcolor{darkblue}{\textbf{\ipa{tɕʰɤ˧˥}}}}{}
\textcolor{teal}{\mytextsc{verbe}} \hspace{4pt} Ton~: MH.

Tromper.
\textcolor{brown}{\zh{欺骗。}}


\begin{exe}
\sn \textcolor{darkblue}{\textbf{\ipa{le˧-tɕʰɤ˧-ze˥}}}
\trans \mytextsc{accomp} \string_ \mytextsc{pfv}
\trans \textit{\textcolor{brown}{\zh{欺骗了}}}
\end{exe}
\begin{exe}
\sn \textcolor{darkblue}{\textbf{\ipa{hĩ˧ tɕʰɤ˧(-ze˩)}}}
\trans tromper les gens
\trans \textit{\textcolor{brown}{\zh{骗人}}}
\end{exe}
\begin{exe}
\sn \textcolor{darkblue}{\textbf{\ipa{no˧ | hĩ˧ tɕʰɤ˧!}}}
\trans vous trompez les gens!
\trans \textit{\textcolor{brown}{\zh{你骗人!}}}
\end{exe}
\begin{exe}
\sn \textcolor{darkblue}{\textbf{\ipa{(hĩ˧ |) no˩ tɕʰɤ˩˥!}}}
\trans les gens vous trompent!
\trans \textit{\textcolor{brown}{\zh{人家骗你!}}}
\end{exe}
\begin{exe}
\sn \textcolor{darkblue}{\textbf{\ipa{(no˧ |) njɤ˩ tɕʰɤ˩˥!}}}
\trans Vous me trompez!
\trans \textit{\textcolor{brown}{\zh{你骗我!}}}
\end{exe}


\paragraph{\hspace{-0.5cm} {\Large \textcolor{darkblue}{\textbf{\ipa{tɕʰi˥}}}}} \hypertarget{ts£\string_hi\string_T1}{}
\markboth{\textcolor{darkblue}{\textbf{\ipa{tɕʰi˥}}}}{}
\textcolor{teal}{\mytextsc{nom}} \hspace{4pt} Ton~: \#H.

Épine.
\textcolor{brown}{\zh{刺。}}


 \mytextsc{clf}~: \textcolor{darkblue}{\textbf{\ipa{kɤ˧˥}}} 

\paragraph{\hspace{-0.5cm} {\Large \textcolor{darkblue}{\textbf{\ipa{tɕʰi˧\textsubscript{b}}}} \textsubscript{1}}} \hypertarget{ts£\string_hi\string_Mb1}{}
\markboth{\textcolor{darkblue}{\textbf{\ipa{tɕʰi˧\textsubscript{b}}}} \textsubscript{1}}{}
\textcolor{teal}{\mytextsc{verbe}} \hspace{4pt} Ton~: M\textsubscript{b}.

Garder, surveiller (ex.: garder la maison).
\textcolor{brown}{\zh{守卫。}}


\begin{exe}
\sn \textcolor{darkblue}{\textbf{\ipa{ɑ˩ʁo˧ tɕʰi˧}}}
\trans garder la maison
\trans \textit{\textcolor{brown}{\zh{守护家}}}
\end{exe}
\begin{exe}
\sn \textcolor{darkblue}{\textbf{\ipa{ɑ˩ʁo˧ tʰi˧-tɕʰi˧-dʑo˧}}}
\trans en train de surveiller la maison
\trans \textit{\textcolor{brown}{\zh{守着家}}}
\end{exe}
\begin{exe}
\sn \textcolor{darkblue}{\textbf{\ipa{tso˧\textasciitilde{}tso˧ tɕʰi˧}}}
\trans surveiller des objets
\trans \textit{\textcolor{brown}{\zh{守着东西}}}
\end{exe}


\paragraph{\hspace{-0.5cm} {\Large \textcolor{darkblue}{\textbf{\ipa{tɕʰi˧\textsubscript{b}}}} \textsubscript{2}}} \hypertarget{ts£\string_hi\string_Mb2}{}
\markboth{\textcolor{darkblue}{\textbf{\ipa{tɕʰi˧\textsubscript{b}}}} \textsubscript{2}}{}
\textcolor{teal}{\mytextsc{verbe}} \hspace{4pt} Ton~: M\textsubscript{b}.

Vendre.
\textcolor{brown}{\zh{卖。}}




\paragraph{\hspace{-0.5cm} {\Large \textcolor{darkblue}{\textbf{\ipa{tɕʰi˧ɖv̩\#˥}}}}} \hypertarget{ts£\string_hi\string_Md`v\string_=\#\string_T1}{}
\markboth{\textcolor{darkblue}{\textbf{\ipa{tɕʰi˧ɖv̩\#˥}}}}{}
\textcolor{teal}{\mytextsc{nom}} \hspace{4pt} Ton~: \#H.

Prénom féminin.
\textcolor{brown}{\zh{女性名字。}}




\paragraph{\hspace{-0.5cm} {\Large \textcolor{darkblue}{\textbf{\ipa{tɕʰi˧nɑ˥}}}}} \hypertarget{ts£\string_hi\string_MnA\string_T1}{}
\markboth{\textcolor{darkblue}{\textbf{\ipa{tɕʰi˧nɑ˥}}}}{}
\textcolor{teal}{\mytextsc{nom}} \hspace{4pt} Ton~: H\#.

Prinsepia, \textit{Prinsepia utilis Royle}; végétal qui sert pour les haies, à grosses épines, petites fleurs jaunes, et tige verte vernissée. On tire de ses graines une huile de grand prix, utilisée dans des préparations alimentaires et comme cosmétique/huile de massage.
\textcolor{brown}{\zh{青刺果、青刺尖、阿娜斯果。}}


\begin{exe}
\sn \textcolor{darkblue}{\textbf{\ipa{tɕʰi˧nɑ˥-dzi˩}}}
\trans prinsepia (la plante)
\trans \textit{\textcolor{brown}{\zh{青刺尖}}}
\end{exe}
\begin{exe}
\sn \textcolor{darkblue}{\textbf{\ipa{tɕʰi˧nɑ˥-bæ˩bæ˩}}}
\trans fleur de prinsepia
\trans \textit{\textcolor{brown}{\zh{青刺果花}}}
\end{exe}


\paragraph{\hspace{-0.5cm} {\Large \textcolor{darkblue}{\textbf{\ipa{tɕʰi˧ʈʂʰɤ˥}}}}} \hypertarget{ts£\string_hi\string_Mt`s`\string_h7\string_T1}{}
\markboth{\textcolor{darkblue}{\textbf{\ipa{tɕʰi˧ʈʂʰɤ˥}}}}{}
\textcolor{teal}{\mytextsc{nom}} \hspace{4pt} Ton~: H\#.

Voiture, automobile.
\textcolor{brown}{\zh{汽车(汉语借词)。}}

 Emprunt~: chinois \textcolor{brown}{\zh{汽车}}

 \mytextsc{clf}~: \textcolor{darkblue}{\textbf{\ipa{nɑ˧}}} 

\paragraph{\hspace{-0.5cm} {\Large \textcolor{darkblue}{\textbf{\ipa{tɕʰi˩\textsubscript{b}}}}}} \hypertarget{ts£\string_hi\string_Bb1}{}
\markboth{\textcolor{darkblue}{\textbf{\ipa{tɕʰi˩\textsubscript{b}}}}}{}
\textcolor{teal}{\mytextsc{classificateur}} \hspace{4pt} Ton~: L\textsubscript{b}.

Classificateur des repas.
\textcolor{brown}{\zh{量词:饭(一顿)。}}


\begin{exe}
\sn \textcolor{darkblue}{\textbf{\ipa{ɖɯ˧-tɕʰi˩ dzɯ˩}}}
\trans prendre un repas
\trans \textit{\textcolor{brown}{\zh{吃一顿}}}
\end{exe}
\begin{exe}
\sn \textcolor{darkblue}{\textbf{\ipa{gv̩˧-tɕʰi˥}}}
\trans neuf repas
\trans \textit{\textcolor{brown}{\zh{就顿(饭)}}}
\end{exe}
\begin{exe}
\sn \textcolor{darkblue}{\textbf{\ipa{tɕʰi˩ tʰv̩˩˥}}}
\trans apporter de la nourriture, apporter un repas: lorsqu'on est invité à participer à des cérémonies funéraires, on apporte à manger, pour contribuer aux repas collectifs
\trans \textit{\textcolor{brown}{\zh{带饭,“出(一)顿(饭)”:被请参加守孝时,要给那家主人带上饭)}}}
\end{exe}
\begin{exe}
\sn \textcolor{darkblue}{\textbf{\ipa{tɕʰi˩tʰv̩˩-hĩ˥}}}
\trans la personne qui se charge du repas / qui nourrit tous les participants (lors d'un repas de veillée funéraire)
\trans \textit{\textcolor{brown}{\zh{给大家供饭的那个人(不一定是主人)}}}
\end{exe}


\paragraph{\hspace{-0.5cm} {\Large \textcolor{darkblue}{\textbf{\ipa{tɕʰi˩tsɯ˧}}}}} \hypertarget{ts£\string_hi\string_BtsM\string_M1}{}
\markboth{\textcolor{darkblue}{\textbf{\ipa{tɕʰi˩tsɯ˧}}}}{}
\textcolor{teal}{\mytextsc{nom}} \hspace{4pt} Ton~: LM.

Aubergine.
\textcolor{brown}{\zh{茄子。}}

 Emprunt~: chinois \textcolor{brown}{\zh{茄子}}



\paragraph{\hspace{-0.5cm} {\Large \textcolor{darkblue}{\textbf{\ipa{tɕʰo˩}}}}} \hypertarget{ts£\string_ho\string_B1}{}
\markboth{\textcolor{darkblue}{\textbf{\ipa{tɕʰo˩}}}}{}
\textcolor{teal}{\mytextsc{classificateur}} \hspace{4pt} Ton~: L *.

Ensemble.
\textcolor{brown}{\zh{量词:一起。}}


\begin{exe}
\sn \textcolor{darkblue}{\textbf{\ipa{ɖɯ˧-tɕʰo˩}}}
\trans ensemble
\trans \textit{\textcolor{brown}{\zh{一起}}}
\end{exe}
\begin{exe}
\sn \textcolor{darkblue}{\textbf{\ipa{le˧-tɕʰo˥\textasciitilde{}tɕʰo˩}}}
\trans même sens que ci-dessus: ensemble
\trans \textit{\textcolor{brown}{\zh{同上:一起}}}
\end{exe}


\paragraph{\hspace{-0.5cm} {\Large \textcolor{darkblue}{\textbf{\ipa{tɕʰo˩\textsubscript{a}}}}}} \hypertarget{ts£\string_ho\string_Ba1}{}
\markboth{\textcolor{darkblue}{\textbf{\ipa{tɕʰo˩\textsubscript{a}}}}}{}
\textcolor{teal}{\mytextsc{verbe}} \hspace{4pt} Ton~: L\textsubscript{a}.

Accompagner, suivre (quelqu'un lors d'un voyage, par exemple); aller avec.
\textcolor{brown}{\zh{陪伴、一起去、跟着。}}


\begin{exe}
\sn \textcolor{darkblue}{\textbf{\ipa{hĩ˧ tɕʰo˥}}}
\trans suivre quelqu'un
\trans \textit{\textcolor{brown}{\zh{陪伴某人}}}
\end{exe}
\begin{exe}
\sn \textcolor{darkblue}{\textbf{\ipa{ɖɯ˧-tɕʰo˩ tʰi˩-tɕʰo˩ |}}}
\trans aller ensemble, former un ensemble: par exemple, dans la pièce principale de la maison, le thangka au-dessus du foyer et les peintures sur le buffet-autel des ancêtres forment un tout, elles vont ensemble
\trans \textit{\textcolor{brown}{\zh{陪伴某人}}}
\end{exe}


\paragraph{\hspace{-0.5cm} {\Large \textcolor{darkblue}{\textbf{\ipa{tɕʰo˩mi\#˥}}}}} \hypertarget{ts£\string_ho\string_Bmi\#\string_T1}{}
\markboth{\textcolor{darkblue}{\textbf{\ipa{tɕʰo˩mi\#˥}}}}{}
\textcolor{teal}{\mytextsc{nom}} \hspace{4pt} Ton~: LM+\#H.

Grande louche.
\textcolor{brown}{\zh{大瓢。}}


 \mytextsc{clf}~: \textcolor{darkblue}{\textbf{\ipa{nɑ˧}}} 

\paragraph{\hspace{-0.5cm} {\Large \textcolor{darkblue}{\textbf{\ipa{tɕʰo˩qʰwɤ˧}}}}} \hypertarget{ts£\string_ho\string_Bq\string_hw7\string_M1}{}
\markboth{\textcolor{darkblue}{\textbf{\ipa{tɕʰo˩qʰwɤ˧}}}}{}
\textcolor{teal}{\mytextsc{nom}} \hspace{4pt} Ton~: LM.

Louche utilisée pour les aliments des animaux; à la date de l'enquête, c'était un objet en aluminium, tandis que celui utilisé pour puiser l'eau, que l'on peut porter à la bouche, est en étain.
\textcolor{brown}{\zh{用来煮猪食的勺子。}}


 \mytextsc{clf}~: \textcolor{darkblue}{\textbf{\ipa{nɑ˧}}} 

\paragraph{\hspace{-0.5cm} {\Large \textcolor{darkblue}{\textbf{\ipa{tɕʰo˩zo\#˥}}}}} \hypertarget{ts£\string_ho\string_Bzo\#\string_T1}{}
\markboth{\textcolor{darkblue}{\textbf{\ipa{tɕʰo˩zo\#˥}}}}{}
\textcolor{teal}{\mytextsc{nom}} \hspace{4pt} Ton~: LM+\#H.

Petite louche.
\textcolor{brown}{\zh{小瓢。}}


 \mytextsc{clf}~: \textcolor{darkblue}{\textbf{\ipa{nɑ˧}}} 

\paragraph{\hspace{-0.5cm} {\Large \textcolor{darkblue}{\textbf{\ipa{tɕʰo˧˥}}}}} \hypertarget{ts£\string_ho\string_M\string_T1}{}
\markboth{\textcolor{darkblue}{\textbf{\ipa{tɕʰo˧˥}}}}{}
\textcolor{teal}{\mytextsc{verbe}} \hspace{4pt} Ton~: MH.

Équarrir (une grosse pièce de bois de charpente).
\textcolor{brown}{\zh{(将木料)砍成方形。}}


\begin{exe}
\sn \textcolor{darkblue}{\textbf{\ipa{bi˩mi˩-ɳɯ˥ | tɕʰo˧˥}}}
\trans équarrir à la hache
\trans \textit{\textcolor{brown}{\zh{用斧头砍成方形}}}
\end{exe}


\paragraph{\hspace{-0.5cm} {\Large \textcolor{darkblue}{\textbf{\ipa{tɕʰo˩˧}}}}} \hypertarget{ts£\string_ho\string_B\string_M1}{}
\markboth{\textcolor{darkblue}{\textbf{\ipa{tɕʰo˩˧}}}}{}
\textcolor{teal}{\mytextsc{nom}} \hspace{4pt} Ton~: LM.

Louche (de grande taille: grosse louche pour puiser l'eau; tient plus d'un litre).
\textcolor{brown}{\zh{勺子、瓢。}}


 \mytextsc{clf}~: \textcolor{darkblue}{\textbf{\ipa{nɑ˧}}} 

\paragraph{\hspace{-0.5cm} {\Large \textcolor{darkblue}{\textbf{\ipa{tɕʰɯ˥}}}}} \hypertarget{ts£\string_hM\string_T1}{}
\markboth{\textcolor{darkblue}{\textbf{\ipa{tɕʰɯ˥}}}}{}
\textcolor{teal}{\mytextsc{verbe}} \hspace{4pt} Ton~: H.

Percer, transpercer.
\textcolor{brown}{\zh{穿刺、 刺破。}}


\begin{exe}
\sn \textcolor{darkblue}{\textbf{\ipa{ʝi˧ ʈʂʰɯ˧-pʰo˩, | ɲi˧ tɕʰi˧-ze˩!}}}
\trans Ce bœuf, on lui a percé le museau (pour y placer un anneau)!
\trans \textit{\textcolor{brown}{\zh{这头牛的鼻子被穿刺(为了安一个牛鼻圈)}}}
\end{exe}


\paragraph{\hspace{-0.5cm} {\Large \textcolor{darkblue}{\textbf{\ipa{tɕʰɯ˧\textsubscript{a}}}} \textsubscript{1}}} \hypertarget{ts£\string_hM\string_Ma1}{}
\markboth{\textcolor{darkblue}{\textbf{\ipa{tɕʰɯ˧\textsubscript{a}}}} \textsubscript{1}}{}
\textcolor{teal}{\mytextsc{verbe}} \hspace{4pt} Ton~: M\textsubscript{a}.

Lever (le bras…).
\textcolor{brown}{\zh{举、抬(胳膊)。}}


\begin{exe}
\sn \textcolor{darkblue}{\textbf{\ipa{lo˩qʰwɤ˥ | gɤ˩-tɕʰɯ˧}}}
\trans lever le bras
\trans \textit{\textcolor{brown}{\zh{举手、抬胳膊}}}
\end{exe}
\begin{exe}
\sn \textcolor{darkblue}{\textbf{\ipa{kʰɯ˧tsʰɤ˧˥ | gɤ˩-tɕʰɯ˧}}}
\trans lever la jambe
\trans \textit{\textcolor{brown}{\zh{抬脚}}}
\end{exe}
\begin{exe}
\sn \textcolor{darkblue}{\textbf{\ipa{gɤ˩-mɤ˧-tɕʰɯ˧}}}
\trans ne pas lever
\trans \textit{\textcolor{brown}{\zh{不抬起来}}}
\end{exe}


\paragraph{\hspace{-0.5cm} {\Large \textcolor{darkblue}{\textbf{\ipa{tɕʰɯ˧\textsubscript{a}}}} \textsubscript{2}}} \hypertarget{ts£\string_hM\string_Ma2}{}
\markboth{\textcolor{darkblue}{\textbf{\ipa{tɕʰɯ˧\textsubscript{a}}}} \textsubscript{2}}{}
\textcolor{teal}{\mytextsc{verbe}} \hspace{4pt} Ton~: M\textsubscript{a}.
\ding{202} 
Garder, faire la garde.
\textcolor{brown}{\zh{守护。}}


\ding{203} 
Veiller un défunt, lors d'une veillée funèbre.
\textcolor{brown}{\zh{居丧、守灵。}}


\begin{exe}
\sn \textcolor{darkblue}{\textbf{\ipa{hĩ˧ tɕʰɯ˧}}}
\trans même sens: veiller un défunt
\trans \textit{\textcolor{brown}{\zh{同上:守灵}}}
\end{exe}


\paragraph{\hspace{-0.5cm} {\Large \textcolor{darkblue}{\textbf{\ipa{tɕʰɯ˧bo˧˥}}}}} \hypertarget{ts£\string_hM\string_Mbo\string_M\string_T1}{}
\markboth{\textcolor{darkblue}{\textbf{\ipa{tɕʰɯ˧bo˧˥}}}}{}
\textcolor{teal}{\mytextsc{adjectif}} \hspace{4pt} Ton~: MH\#.

Frais.
\textcolor{brown}{\zh{凉快。}}




\paragraph{\hspace{-0.5cm} {\Large \textcolor{darkblue}{\textbf{\ipa{tɕʰɯ˧lo\#˥}}}}} \hypertarget{ts£\string_hM\string_Mlo\#\string_T1}{}
\markboth{\textcolor{darkblue}{\textbf{\ipa{tɕʰɯ˧lo\#˥}}}}{}
\textcolor{teal}{\mytextsc{nom}} \hspace{4pt} Ton~: \#H.

Grande assiette.
\textcolor{brown}{\zh{大盘子。}}


 \mytextsc{clf}~: \textcolor{darkblue}{\textbf{\ipa{ɭɯ˧}}} 

\paragraph{\hspace{-0.5cm} {\Large \textcolor{darkblue}{\textbf{\ipa{tɕʰɯ˧si˩-dʑɤ˩pv̩˩}}}}} \hypertarget{ts£\string_hM\string_Msi\string_B-dz£7\string_Bpv\string_=\string_B1}{}
\markboth{\textcolor{darkblue}{\textbf{\ipa{tɕʰɯ˧si˩-dʑɤ˩pv̩˩}}}}{}
\textcolor{teal}{\mytextsc{nom}} \hspace{4pt} Ton~: L\#-.

Monstre, revenant.
\textcolor{brown}{\zh{妖怪。}}


\begin{exe}
\sn \textcolor{darkblue}{\textbf{\ipa{no˧ | tɕʰɯ˧si˩-dʑɤ˩pv̩˩-ki˩ | le˧-hɯ˩-ɲi˩-ze˩!}}}
\trans Tu es parti rejoindre les monstres! (propos tenus à un revenant qu'on enjoint de ne plus revenir hanter les vivants)
\trans \textit{\textcolor{brown}{\zh{你已经到妖怪的世界那边(就恳求你不要回来了)!(对鬼说的话)}}}
\end{exe}


\paragraph{\hspace{-0.5cm} {\Large \textcolor{darkblue}{\textbf{\ipa{tɕʰɯ˧sɯ˥}}}}} \hypertarget{ts£\string_hM\string_MsM\string_T1}{}
\markboth{\textcolor{darkblue}{\textbf{\ipa{tɕʰɯ˧sɯ˥}}}}{}
\textcolor{teal}{\mytextsc{adjectif}} \hspace{4pt} Ton~: H\#.

Triste, dans l'affliction, plongé dans le chagrin.
\textcolor{brown}{\zh{悲哀、伤心。}}




\paragraph{\hspace{-0.5cm} {\Large \textcolor{darkblue}{\textbf{\ipa{tɕʰɯ˩\textsubscript{a}}}}}} \hypertarget{ts£\string_hM\string_Ba1}{}
\markboth{\textcolor{darkblue}{\textbf{\ipa{tɕʰɯ˩\textsubscript{a}}}}}{}
\textcolor{teal}{\mytextsc{adjectif}} \hspace{4pt} Ton~: L\textsubscript{a}.

Sucré.
\textcolor{brown}{\zh{甜。}}




\paragraph{\hspace{-0.5cm} {\Large \textcolor{darkblue}{\textbf{\ipa{tɕʰɯ˩di˩}}}}} \hypertarget{ts£\string_hM\string_Bdi\string_B1}{}
\markboth{\textcolor{darkblue}{\textbf{\ipa{tɕʰɯ˩di˩}}}}{}
\textcolor{teal}{\mytextsc{verbe}} \hspace{4pt} Ton~: L.

Chasser.
\textcolor{brown}{\zh{狩猎。}}


\textit{Voir~:} \textcolor{darkblue}{\textbf{\ipa{tɕʰɯ˩˥, di˧˥1}}} 

\paragraph{\hspace{-0.5cm} {\Large \textcolor{darkblue}{\textbf{\ipa{tɕʰɯ˩di˩kʰv̩˩}}}}} \hypertarget{ts£\string_hM\string_Bdi\string_Bk\string_hv\string_=\string_B1}{}
\markboth{\textcolor{darkblue}{\textbf{\ipa{tɕʰɯ˩di˩kʰv̩˩}}}}{}
\textcolor{teal}{\mytextsc{nom}} \hspace{4pt} Ton~: L.

Chien de chasse.
\textcolor{brown}{\zh{猎狗。}}


\begin{exe}
\sn \textcolor{darkblue}{\textbf{\ipa{tɕʰɯ˩di˩-kʰv̩˥mi˩}}}
\trans même sens
\trans \textit{\textcolor{brown}{\zh{猎狗}}}
\end{exe}
 \mytextsc{clf}~: \textcolor{darkblue}{\textbf{\ipa{mi˩}}} 

\paragraph{\hspace{-0.5cm} {\Large \textcolor{darkblue}{\textbf{\ipa{tɕʰɯ˩mi\#˥}}}}} \hypertarget{ts£\string_hM\string_Bmi\#\string_T1}{}
\markboth{\textcolor{darkblue}{\textbf{\ipa{tɕʰɯ˩mi\#˥}}}}{}
\textcolor{teal}{\mytextsc{nom}} \hspace{4pt} Ton~: LM+\#H / L.

Muntjac femelle.
\textcolor{brown}{\zh{母麂子。}}


 \mytextsc{clf}~: \textcolor{darkblue}{\textbf{\ipa{mi˩}}} 

\paragraph{\hspace{-0.5cm} {\Large \textcolor{darkblue}{\textbf{\ipa{tɕʰɯ˩pʰv̩\#˥}}}}} \hypertarget{ts£\string_hM\string_Bp\string_hv\string_=\#\string_T1}{}
\markboth{\textcolor{darkblue}{\textbf{\ipa{tɕʰɯ˩pʰv̩\#˥}}}}{}
\textcolor{teal}{\mytextsc{nom}} \hspace{4pt} Ton~: LM+\#H / L.

Muntjac mâle.
\textcolor{brown}{\zh{公麂子。}}


 \mytextsc{clf}~: \textcolor{darkblue}{\textbf{\ipa{mi˩}}} 

\paragraph{\hspace{-0.5cm} {\Large \textcolor{darkblue}{\textbf{\ipa{}}}}} \hypertarget{ts£\string_hM\string_B-Ro\string_B-ts£\string_hM\string_T!1}{}
\markboth{\textcolor{darkblue}{\textbf{\ipa{}}}}{}
\textcolor{teal}{\mytextsc{adverbe}} \hspace{4pt} Ton~: L+H\#.

A vos souhaits! (formule que l'on dit lorsque quelqu'un éternue).
\textcolor{brown}{\zh{旁边的人打嚏喷时说的祝愿话。}}




\paragraph{\hspace{-0.5cm} {\Large \textcolor{darkblue}{\textbf{\ipa{tɕʰɯ˩\textasciitilde{}tɕʰɯ˧˥}}}}} \hypertarget{ts£\string_hM\string_B~ts£\string_hM\string_M\string_T1}{}
\markboth{\textcolor{darkblue}{\textbf{\ipa{tɕʰɯ˩\textasciitilde{}tɕʰɯ˧˥}}}}{}
\textcolor{teal}{\mytextsc{verbe}} \hspace{4pt} Ton~: MH.

Sucer.
\textcolor{brown}{\zh{吸吮。}}


\begin{exe}
\sn \textcolor{darkblue}{\textbf{\ipa{lo˩mi˧ tɕʰi˩\textasciitilde{}tɕʰi˩}}}
\trans sucer son pouce
\trans \textit{\textcolor{brown}{\zh{吮拇指}}}
\end{exe}


\paragraph{\hspace{-0.5cm} {\Large \textcolor{darkblue}{\textbf{\ipa{tɕʰɯ˩zo\#˥}}}}} \hypertarget{ts£\string_hM\string_Bzo\#\string_T1}{}
\markboth{\textcolor{darkblue}{\textbf{\ipa{tɕʰɯ˩zo\#˥}}}}{}
\textcolor{teal}{\mytextsc{nom}} \hspace{4pt} Ton~: LM+\#H / L.

Petit muntjac.
\textcolor{brown}{\zh{麂子崽子。}}


 \mytextsc{clf}~: \textcolor{darkblue}{\textbf{\ipa{ɭɯ˧}}} 

\paragraph{\hspace{-0.5cm} {\Large \textcolor{darkblue}{\textbf{\ipa{tɕʰɯ˧˥}}}}} \hypertarget{ts£\string_hM\string_M\string_T1}{}
\markboth{\textcolor{darkblue}{\textbf{\ipa{tɕʰɯ˧˥}}}}{}
\textcolor{teal}{\mytextsc{nom}} \hspace{4pt} Ton~: MH.

Peinture, laque.
\textcolor{brown}{\zh{漆。}}

 Emprunt~: chinois \textcolor{brown}{\zh{漆}}

\begin{exe}
\sn \textcolor{darkblue}{\textbf{\ipa{tɕʰɯ˧ jɤ˥-zo˩-ho˩!}}}
\trans Il va falloir (re)peindre
\trans \textit{\textcolor{brown}{\zh{该刷漆了!}}}
\end{exe}


\paragraph{\hspace{-0.5cm} {\Large \textcolor{darkblue}{\textbf{\ipa{tɕʰɯ˧˥}}} \textsubscript{1}}} \hypertarget{ts£\string_hM\string_M\string_T1}{}
\markboth{\textcolor{darkblue}{\textbf{\ipa{tɕʰɯ˧˥}}} \textsubscript{1}}{}
\textcolor{teal}{\mytextsc{verbe}} \hspace{4pt} Ton~: MH.

Jeter, se débarrasser de (poubelles, détritus…); abandonner.
\textcolor{brown}{\zh{扔(垃圾)。}}


\begin{exe}
\sn \textcolor{darkblue}{\textbf{\ipa{ɖæ˩˥ | tʰi˧-tɕʰɯ˧˥}}}
\trans jeter des détritus
\trans \textit{\textcolor{brown}{\zh{扔垃圾}}}
\end{exe}


\paragraph{\hspace{-0.5cm} {\Large \textcolor{darkblue}{\textbf{\ipa{tɕʰɯ˧˥}}} \textsubscript{2}}} \hypertarget{ts£\string_hM\string_M\string_T2}{}
\markboth{\textcolor{darkblue}{\textbf{\ipa{tɕʰɯ˧˥}}} \textsubscript{2}}{}
\textcolor{teal}{\mytextsc{verbe}} \hspace{4pt} Ton~: MH.

Cracher.
\textcolor{brown}{\zh{吐(吐口水)。}}




\paragraph{\hspace{-0.5cm} {\Large \textcolor{darkblue}{\textbf{\ipa{tɕʰɯ˧˥}}} \textsubscript{3}}} \hypertarget{ts£\string_hM\string_M\string_T3}{}
\markboth{\textcolor{darkblue}{\textbf{\ipa{tɕʰɯ˧˥}}} \textsubscript{3}}{}
\textcolor{teal}{\mytextsc{verbe}} \hspace{4pt} Ton~: MH.

Perdre.
\textcolor{brown}{\zh{丢失、弄丢。}}


\begin{exe}
\sn \textcolor{darkblue}{\textbf{\ipa{le˧-tɕʰɯ˧-ze˥}}}
\trans \mytextsc{accomp} \string_ \mytextsc{pfv}
\trans \textit{\textcolor{brown}{\zh{丢了}}}
\end{exe}
\begin{exe}
\sn \textcolor{darkblue}{\textbf{\ipa{le˧-tɕʰɯ˧-hɯ˥-ze˩!}}}
\trans c'est perdu!
\trans \textit{\textcolor{brown}{\zh{丢掉了!}}}
\end{exe}


\paragraph{\hspace{-0.5cm} {\Large \textcolor{darkblue}{\textbf{\ipa{tɕʰɯ˧˥}}} \textsubscript{4}}} \hypertarget{ts£\string_hM\string_M\string_T4}{}
\markboth{\textcolor{darkblue}{\textbf{\ipa{tɕʰɯ˧˥}}} \textsubscript{4}}{}
\textcolor{teal}{\mytextsc{adjectif}} \hspace{4pt} Ton~: MH.

Inquiet, angoissé, tourmenté, oppressé.
\textcolor{brown}{\zh{担心。}}


\begin{exe}
\sn \textcolor{darkblue}{\textbf{\ipa{nv̩˩mi˩ tɕʰɯ˥}}}
\trans inquiet
\trans \textit{\textcolor{brown}{\zh{担心}}}
\end{exe}


\paragraph{\hspace{-0.5cm} {\Large \textcolor{darkblue}{\textbf{\ipa{tɕʰɯ˧˥}}} \textsubscript{5}}} \hypertarget{ts£\string_hM\string_M\string_T5}{}
\markboth{\textcolor{darkblue}{\textbf{\ipa{tɕʰɯ˧˥}}} \textsubscript{5}}{}
\textcolor{teal}{\mytextsc{adjectif}} \hspace{4pt} Ton~: MH.
\textit{Archaïque} [\textcolor{brown}{\zh{古语}}] 
Observé seulement en tournure négative: ne pas avoir (de quoi vivre); être démuni. On peut imaginer comme sens ancien ``bien doté, à l'aise".
\textcolor{brown}{\zh{舒服。}}


\begin{exe}
\sn \textcolor{darkblue}{\textbf{\ipa{mɤ˧-tɕʰɯ˧-bi˥ / mɤ˧-tɕʰɯ˧˥ |-bi˩}}}
\trans même si on est dans le besoin/quoi qu'on soit dans le besoin, ...
\trans \textit{\textcolor{brown}{\zh{虽然很贫穷,……}}}
\end{exe}
\begin{exe}
\sn \textcolor{darkblue}{\textbf{\ipa{mɤ˧-dʑo˧ mɤ˧-tɕʰɯ˧-ɻ̍˧-bi˥, | ɖwæ˩ mɤ˧-zo˧!}}}
\trans ``Même si on est sans rien, dans le besoin, il ne faut pas s'inquiéter!" (car le Ciel vient en aide aux gens qui font de leur mieux)
\trans \textit{\textcolor{brown}{\zh{虽然穷,莫担心!(因为菩萨会救好人)}}}
\end{exe}
\begin{exe}
\sn \textcolor{darkblue}{\textbf{\ipa{hĩ˧ ʈʂʰɯ˧-v̩˧-dʑo˩, | ɖwæ˧˥ | mɤ˧-tɕʰɯ˧˥!}}}
\trans Il est vraiment dans le besoin/nécessiteux!
\trans \textit{\textcolor{brown}{\zh{这个人,真的很穷!}}}
\end{exe}


\paragraph{\hspace{-0.5cm} {\Large \textcolor{darkblue}{\textbf{\ipa{tɕʰɯ˩˥}}}}} \hypertarget{ts£\string_hM\string_B\string_T1}{}
\markboth{\textcolor{darkblue}{\textbf{\ipa{tɕʰɯ˩˥}}}}{}
\textcolor{teal}{\mytextsc{nom}} \hspace{4pt} Ton~: LH.

Muntjac.
\textcolor{brown}{\zh{麂子。}}


\begin{exe}
\sn \textcolor{darkblue}{\textbf{\ipa{tɕʰɯ˩ hwæ˧-ze˩}}}
\trans ...a acheté un muntjac
\trans \textit{\textcolor{brown}{\zh{买麂子}}}
\end{exe}
\begin{exe}
\sn \textcolor{darkblue}{\textbf{\ipa{tɕʰɯ˩ dzɯ˩-ze˥}}}
\trans ...a mangé un muntjac
\trans \textit{\textcolor{brown}{\zh{吃了麂子}}}
\end{exe}
 \mytextsc{clf}~: \textcolor{darkblue}{\textbf{\ipa{pʰo˧˥}}} 

\newpage
\section*{\centering- \textcolor{darkblue}{\textbf{\ipa{ts}}} -}
\paragraph{\hspace{-0.5cm} {\Large \textcolor{darkblue}{\textbf{\ipa{tsɑ˧}}}}} \hypertarget{tsA\string_M1}{}
\markboth{\textcolor{darkblue}{\textbf{\ipa{tsɑ˧}}}}{}
\textcolor{teal}{\mytextsc{adjectif}} \hspace{4pt} Ton~: M.

Occupé, affairé, pressé.
\textcolor{brown}{\zh{忙。}}


\begin{exe}
\sn \textcolor{darkblue}{\textbf{\ipa{ɖwæ˧˥ | tsɑ˧}}}
\trans \mytextsc{intensif}.très: très occupé
\trans \textit{\textcolor{brown}{\zh{很忙}}}
\end{exe}
\begin{exe}
\sn \textcolor{darkblue}{\textbf{\ipa{tsɑ˧ | ʐwæ˩˥}}}
\trans extrêmement occupé
\trans \textit{\textcolor{brown}{\zh{非常忙}}}
\end{exe}


\paragraph{\hspace{-0.5cm} {\Large \textcolor{darkblue}{\textbf{\ipa{tsɑ˧bɤ˧}}}}} \hypertarget{tsA\string_Mb7\string_M1}{}
\markboth{\textcolor{darkblue}{\textbf{\ipa{tsɑ˧bɤ˧}}}}{}
\textcolor{teal}{\mytextsc{nom}} \hspace{4pt} Ton~: M.

Poudre; farine.
\textcolor{brown}{\zh{糌粑、面粉、粉、粉末。}}


\begin{exe}
\sn \textcolor{darkblue}{\textbf{\ipa{qʰɑ˧dze˧-tsɑ˩bɤ˩}}}
\trans farine de maïs
\trans \textit{\textcolor{brown}{\zh{玉米粉}}}
\end{exe}
\begin{exe}
\sn \textcolor{darkblue}{\textbf{\ipa{dze˧ɭɯ˧-tsɑ˩bɤ˩}}}
\trans farine de blé
\trans \textit{\textcolor{brown}{\zh{小麦面}}}
\end{exe}
\begin{exe}
\sn \textcolor{darkblue}{\textbf{\ipa{lv̩˧mi˧-tsɑ˩bɤ˩}}}
\trans poudre de pierre, pierre pulvérisée
\trans \textit{\textcolor{brown}{\zh{石头粉、被磨成粉的石头}}}
\end{exe}
\begin{exe}
\sn \textcolor{darkblue}{\textbf{\ipa{tsʰi˧zi˧-tsɑ˧bɤ˥}}}
\trans farine d'orge
\trans \textit{\textcolor{brown}{\zh{青稞面粉}}}
\end{exe}
\begin{exe}
\sn \textcolor{darkblue}{\textbf{\ipa{mv̩˩zɯ˩-tsɑ˩bɤ˥}}}
\trans farine d'avoine
\trans \textit{\textcolor{brown}{\zh{燕麦面粉}}}
\end{exe}
\begin{exe}
\sn \textcolor{darkblue}{\textbf{\ipa{jɤ˩jo˧-tsɑ˧bɤ˥}}}
\trans farine de pommes de terre
\trans \textit{\textcolor{brown}{\zh{洋芋面粉}}}
\end{exe}
\begin{exe}
\sn \textcolor{darkblue}{\textbf{\ipa{nv̩˩ɭɯ˧-tsɑ˩bɤ˩}}}
\trans farine de soja
\trans \textit{\textcolor{brown}{\zh{黄豆面粉}}}
\end{exe}
\begin{exe}
\sn \textcolor{darkblue}{\textbf{\ipa{læ˧tsɯ˥-tsɑ˩bɤ˩}}}
\trans piment en poudre
\trans \textit{\textcolor{brown}{\zh{辣椒粉}}}
\end{exe}
\begin{exe}
\sn \textcolor{darkblue}{\textbf{\ipa{ʈʂʰæ˧ɣɯ˧-tsɑ˧bɤ˥}}}
\trans Médicament en poudre. (Exemple: le désinfectant en poudre actuellement utilisé, de la marque \textcolor{brown}{\zh{云南白药}}.)
\trans \textit{\textcolor{brown}{\zh{药粉,粉状药品。如:“云南白药”消毒粉。}}}
\end{exe}
\begin{exe}
\sn \textcolor{darkblue}{\textbf{\ipa{ʂæ˩ɻ̃˩-tsɑ˩bɤ˥}}}
\trans poudre d'os
\trans \textit{\textcolor{brown}{\zh{骨头粉}}}
\end{exe}
\begin{exe}
\sn \textcolor{darkblue}{\textbf{\ipa{jɤ˧-tsɑ˧bɤ˧}}}
\trans tabac en poudre
\trans \textit{\textcolor{brown}{\zh{烟草粉}}}
\end{exe}
\begin{exe}
\sn \textcolor{darkblue}{\textbf{\ipa{jɤ˧ɻ̃˧-tsɑ˧bɤ˥}}}
\trans poudre de tabac, tabac en poudre
\trans \textit{\textcolor{brown}{\zh{烟草粉}}}
\end{exe}
\begin{exe}
\sn \textcolor{darkblue}{\textbf{\ipa{dze˩-tsɑ˩bɤ˥}}}
\trans xanthoxyle en poudre
\trans \textit{\textcolor{brown}{\zh{花椒粉}}}
\end{exe}
\begin{exe}
\sn \textcolor{darkblue}{\textbf{\ipa{dze˧-tsɑ˧bɤ˥}}}
\trans sucre en poudre
\trans \textit{\textcolor{brown}{\zh{砂糖}}}
\end{exe}
\begin{exe}
\sn \textcolor{darkblue}{\textbf{\ipa{tsɑ˧bɤ˧ mɤ˩}}}
\trans manger du tsamba sec
\end{exe}
\begin{exe}
\sn \textcolor{darkblue}{\textbf{\ipa{tsɑ˧bɤ˧ gv̩˩}}}
\trans préparer du tsamba/de la farine grillée
\trans \textit{\textcolor{brown}{\zh{炒糌粑,制作糌粑}}}
\end{exe}


\paragraph{\hspace{-0.5cm} {\Large \textcolor{darkblue}{\textbf{\ipa{tsɑ˧ʐo˩}}}}} \hypertarget{tsA\string_Mz`o\string_B1}{}
\markboth{\textcolor{darkblue}{\textbf{\ipa{tsɑ˧ʐo˩}}}}{}
\textcolor{teal}{\mytextsc{adjectif}} \hspace{4pt} Ton~: L\#.

Zélé, assidu.
\textcolor{brown}{\zh{勤快。}}




\paragraph{\hspace{-0.5cm} {\Large \textcolor{darkblue}{\textbf{\ipa{tsɑ˩}}}}} \hypertarget{tsA\string_B1}{}
\markboth{\textcolor{darkblue}{\textbf{\ipa{tsɑ˩}}}}{}
\textcolor{teal}{\mytextsc{verbe}} \hspace{4pt} Ton~: L.

Faire un clin d'oeil (discret signe d'intelligence).
\textcolor{brown}{\zh{眨眼。}}


\begin{exe}
\sn \textcolor{darkblue}{\textbf{\ipa{ʈʂʰɯ˧ | njɤ˩ɭɯ˧ tsɑ˩\textasciitilde{}tsɑ˩-dʑo˩!}}}
\trans \mytextsc{red}: Elle/il est en train de faire un clin d'oeil!
\trans \textit{\textcolor{brown}{\zh{\mytextsc{重叠:他在眨眨眼!}}}}
\end{exe}
\begin{exe}
\sn \textcolor{darkblue}{\textbf{\ipa{ʈʂʰɯ˧ | njɤ˩ɭɯ˧ tsɑ˩-dʑo˩!}}}
\trans Elle/il est en train de faire un clin d'oeil!
\trans \textit{\textcolor{brown}{\zh{他在眨眼!}}}
\end{exe}
\begin{exe}
\sn \textcolor{darkblue}{\textbf{\ipa{tsɑ˩\textasciitilde{}tsɑ˧˥}}}
\trans \mytextsc{red}
\trans \textit{\textcolor{brown}{\zh{\mytextsc{重叠}}}}
\end{exe}
\begin{exe}
\sn \textcolor{darkblue}{\textbf{\ipa{mɤ˧-tsɑ˩\textasciitilde{}tsɑ˩}}}
\trans \mytextsc{neg} \mytextsc{red}
\trans \textit{\textcolor{brown}{\zh{不眨眼}}}
\end{exe}
\textit{Voir~:} \hyperlink{tsM\string_Bp\string_h7\string_B1}{\textcolor{darkblue}{\textbf{\ipa{tsɯ˩pʰɤ˩}}}} 

\paragraph{\hspace{-0.5cm} {\Large \textcolor{darkblue}{\textbf{\ipa{tsɑ˩tɕi˩}}}}} \hypertarget{tsA\string_Bts£i\string_B1}{}
\markboth{\textcolor{darkblue}{\textbf{\ipa{tsɑ˩tɕi˩}}}}{}
\textcolor{teal}{\mytextsc{nom}} \hspace{4pt} Ton~: L.

Champignons divers, champignons variés.
\textcolor{brown}{\zh{杂菌(汉语借词)。}}

 Emprunt~: chinois \textcolor{brown}{\zh{杂菌}}

 \mytextsc{clf}~: \textcolor{darkblue}{\textbf{\ipa{ɭɯ˧}}} \textcolor{darkblue}{\textbf{\ipa{mo˧˥}}} 

\paragraph{\hspace{-0.5cm} {\Large \textcolor{darkblue}{\textbf{\ipa{tsɑ˧˥}}} \textsubscript{1}}} \hypertarget{tsA\string_M\string_T1}{}
\markboth{\textcolor{darkblue}{\textbf{\ipa{tsɑ˧˥}}} \textsubscript{1}}{}
\textcolor{teal}{\mytextsc{verbe}} \hspace{4pt} Ton~: MH.
\ding{202} 
Donner un coup de pied; briser (les mottes de terre, après le labour, avec une bêche, ou une masse en bois).
\textcolor{brown}{\zh{打碎(坷拉),踢(一脚)。}}


\begin{exe}
\sn \textcolor{darkblue}{\textbf{\ipa{le˧-tsɑ˧-ze˥}}}
\trans \mytextsc{accomp}+\mytextsc{pfv}
\trans \textit{\textcolor{brown}{\zh{\mytextsc{accomp}+\mytextsc{pfv}}}}
\end{exe}
\begin{exe}
\sn \textcolor{darkblue}{\textbf{\ipa{ʈʂe˧ tsɑ˩}}}
\trans briser les mottes de terre après le labour (avec un instrument manuel: houe, bêche)
\trans \textit{\textcolor{brown}{\zh{打碎土坷垃}}}
\end{exe}
\begin{exe}
\sn \textcolor{darkblue}{\textbf{\ipa{ɖɯ˧-tsɑ˧ tʰi˥-tsɑ˩}}}
\trans donner une succession de coups de pied
\trans \textit{\textcolor{brown}{\zh{踢了又踢}}}
\end{exe}
\ding{203} 
Ramer (=geste comparable à celui de briser les mottes: geste répétitif, exerçant la force coup après coup, sur l'eau, comme on le ferait sur des mottes de terre).
\textcolor{brown}{\zh{划(船)。}}


\begin{exe}
\sn \textcolor{darkblue}{\textbf{\ipa{tsɑ˧-hɯ˥-tsɑ˩-ɻ̍˩}}}
\trans ramer de façon soutenue
\trans \textit{\textcolor{brown}{\zh{用力地划船、一直划船}}}
\end{exe}


\paragraph{\hspace{-0.5cm} {\Large \textcolor{darkblue}{\textbf{\ipa{tsɑ˧˥}}} \textsubscript{2}}} \hypertarget{tsA\string_M\string_T2}{}
\markboth{\textcolor{darkblue}{\textbf{\ipa{tsɑ˧˥}}} \textsubscript{2}}{}
\textcolor{teal}{\mytextsc{verbe}} \hspace{4pt} Ton~: MH.

Déposer, poser.
\textcolor{brown}{\zh{放置、放下。}}


\begin{exe}
\sn \textcolor{darkblue}{\textbf{\ipa{mv̩˩tɕo˧ tsɑ˧˥}}}
\trans poser à terre
\trans \textit{\textcolor{brown}{\zh{放下、放在地上}}}
\end{exe}


\paragraph{\hspace{-0.5cm} {\Large \textcolor{darkblue}{\textbf{\ipa{‑tsæ˧}}}}} \hypertarget{‑ts\{\string_M1}{}
\markboth{\textcolor{darkblue}{\textbf{\ipa{‑tsæ˧}}}}{}
\textcolor{teal}{\mytextsc{suffixe}} \hspace{4pt} Ton~: M.

Causatif.
\textcolor{brown}{\zh{\mytextsc{使动:让。}}}


\begin{exe}
\sn \textcolor{darkblue}{\textbf{\ipa{tʰi˧-dzɯ˥-kʰɯ˩-tsæ˩-ɲi˩!}}}
\trans il faut l'obliger à manger/il faut la faire manger! (Commentaire d'un membre de la famille au sujet d'une petite fille qui refuse un repas)
\trans \textit{\textcolor{brown}{\zh{必须让她吃!(情景:一个小女孩拒绝吃饭,家人就说这句。)}}}
\end{exe}
\begin{exe}
\sn \textcolor{darkblue}{\textbf{\ipa{tʰi˧-ʐwɤ˩-kʰɯ˩-tsæ˩-ɲi˩!}}}
\trans il faut le faire parler/il faut l'obliger à parler! (Variante créée par analogie avec l'exemple précédent)
\trans \textit{\textcolor{brown}{\zh{必须让他说!(在以上例子的基础上编的句子)}}}
\end{exe}


\paragraph{\hspace{-0.5cm} {\Large \textcolor{darkblue}{\textbf{\ipa{tsæ˧qæ˥}}}}} \hypertarget{ts\{\string_Mq\{\string_T1}{}
\markboth{\textcolor{darkblue}{\textbf{\ipa{tsæ˧qæ˥}}}}{}
\textcolor{teal}{\mytextsc{nom}} \hspace{4pt} Ton~: H\#.
\ding{202} 
Crochet.
\textcolor{brown}{\zh{钩子。}}


\ding{203} 
Percuteur (de fusil).
\textcolor{brown}{\zh{撞针。}}


 \mytextsc{clf}~: \textcolor{darkblue}{\textbf{\ipa{nɑ˧}}} \textcolor{darkblue}{\textbf{\ipa{ɭɯ˧}}} 

\paragraph{\hspace{-0.5cm} {\Large \textcolor{darkblue}{\textbf{\ipa{tse˧bæ˥}}}}} \hypertarget{tse\string_Mb\{\string_T1}{}
\markboth{\textcolor{darkblue}{\textbf{\ipa{tse˧bæ˥}}}}{}
\textcolor{teal}{\mytextsc{nom}} \hspace{4pt} Ton~: H\#.

Amadou.
\textcolor{brown}{\zh{火绒。}}


 \mytextsc{clf}~: \textcolor{darkblue}{\textbf{\ipa{kʰɯ˩}}} 

\paragraph{\hspace{-0.5cm} {\Large \textcolor{darkblue}{\textbf{\ipa{tse˧bo\#˥}}}}} \hypertarget{tse\string_Mbo\#\string_T1}{}
\markboth{\textcolor{darkblue}{\textbf{\ipa{tse˧bo\#˥}}}}{}
\textcolor{teal}{\mytextsc{nom}} \hspace{4pt} Ton~: \#H.

Clochette (portée par le bétail: chevaux, parfois chiens).
\textcolor{brown}{\zh{铃铛。}}


 \mytextsc{clf}~: \textcolor{darkblue}{\textbf{\ipa{ɭɯ˧}}} 

\paragraph{\hspace{-0.5cm} {\Large \textcolor{darkblue}{\textbf{\ipa{tse˧di\#˥}}}}} \hypertarget{tse\string_Mdi\#\string_T1}{}
\markboth{\textcolor{darkblue}{\textbf{\ipa{tse˧di\#˥}}}}{}
\textcolor{teal}{\mytextsc{nom}} \hspace{4pt} Ton~: \#H.

Bois de santal arbre à épice, arbre à encens.
\textcolor{brown}{\zh{檀香木、檀香、檀木。}}


\begin{exe}
\sn \textcolor{darkblue}{\textbf{\ipa{tse˧di˧-si\#˥}}}
\trans même sens
\trans \textit{\textcolor{brown}{\zh{同上}}}
\end{exe}


\paragraph{\hspace{-0.5cm} {\Large \textcolor{darkblue}{\textbf{\ipa{tse˧kʰo˩}}}}} \hypertarget{tse\string_Mk\string_ho\string_B1}{}
\markboth{\textcolor{darkblue}{\textbf{\ipa{tse˧kʰo˩}}}}{}
\textcolor{teal}{\mytextsc{nom}} \hspace{4pt} Ton~: L\#.

Sanctuaire (petit sanctuaire sur la montagne; n'est pas habitable).
\textcolor{brown}{\zh{佛龛。}}


 \mytextsc{clf}~: \textcolor{darkblue}{\textbf{\ipa{ɭɯ˧}}} 

\paragraph{\hspace{-0.5cm} {\Large \textcolor{darkblue}{\textbf{\ipa{tse˧lv̩˥}}}}} \hypertarget{tse\string_Mlv\string_=\string_T1}{}
\markboth{\textcolor{darkblue}{\textbf{\ipa{tse˧lv̩˥}}}}{}
\textcolor{teal}{\mytextsc{nom}} \hspace{4pt} Ton~: H\#.

Silex.
\textcolor{brown}{\zh{燧石。}}


 \mytextsc{clf}~: \textcolor{darkblue}{\textbf{\ipa{ɭɯ˧}}} 

\paragraph{\hspace{-0.5cm} {\Large \textcolor{darkblue}{\textbf{\ipa{tse˧mi˥}}}}} \hypertarget{tse\string_Mmi\string_T1}{}
\markboth{\textcolor{darkblue}{\textbf{\ipa{tse˧mi˥}}}}{}
\textcolor{teal}{\mytextsc{nom}} \hspace{4pt} Ton~: H\#.

Briquet.
\textcolor{brown}{\zh{火镰。}}


 \mytextsc{clf}~: \textcolor{darkblue}{\textbf{\ipa{nɑ˧}}} 

\paragraph{\hspace{-0.5cm} {\Large \textcolor{darkblue}{\textbf{\ipa{tse˧mi˥-dʑɯ˩ʁo˩}}}}} \hypertarget{tse\string_Mmi\string_T-dz£M\string_BRo\string_B1}{}
\markboth{\textcolor{darkblue}{\textbf{\ipa{tse˧mi˥-dʑɯ˩ʁo˩}}}}{}
\textcolor{teal}{\mytextsc{nom}} \hspace{4pt} Ton~: H\#-L.

Le village de Wenquan (possède des sources chaudes).
\textcolor{brown}{\zh{温泉乡的主要村落。}}




\paragraph{\hspace{-0.5cm} {\Large \textcolor{darkblue}{\textbf{\ipa{tse˩\textsubscript{a}}}} \textsubscript{1}}} \hypertarget{tse\string_Ba1}{}
\markboth{\textcolor{darkblue}{\textbf{\ipa{tse˩\textsubscript{a}}}} \textsubscript{1}}{}
\textcolor{teal}{\mytextsc{verbe}} \hspace{4pt} Ton~: L\textsubscript{a}.

Suivre à la trace, poursuivre, pister.
\textcolor{brown}{\zh{追赶。}}


\begin{exe}
\sn \textcolor{darkblue}{\textbf{\ipa{hĩ˧ tse˥}}}
\trans suivre quelqu'un à la trace
\trans \textit{\textcolor{brown}{\zh{追赶某人}}}
\end{exe}


\paragraph{\hspace{-0.5cm} {\Large \textcolor{darkblue}{\textbf{\ipa{tse˩\textsubscript{a}}}} \textsubscript{2}}} \hypertarget{tse\string_Ba2}{}
\markboth{\textcolor{darkblue}{\textbf{\ipa{tse˩\textsubscript{a}}}} \textsubscript{2}}{}
\textcolor{teal}{\mytextsc{verbe}} \hspace{4pt} Ton~: L\textsubscript{a}.

Flotter.
\textcolor{brown}{\zh{漂浮 (浮在水上)。}}


\begin{exe}
\sn \textcolor{darkblue}{\textbf{\ipa{gɤ˩tse˧}}}
\trans même sens: flotter
\trans \textit{\textcolor{brown}{\zh{同上:漂浮 (浮在水上)}}}
\end{exe}
\begin{exe}
\sn \textcolor{darkblue}{\textbf{\ipa{ɖɯ˧-tse˧\textasciitilde{}tse˥-ɻ̍˩}}}
\trans \mytextsc{délimitatif} \string_ \mytextsc{red} \mytextsc{inchoatif}
\trans \textit{\textcolor{brown}{\zh{\mytextsc{delimitative} \string_ \mytextsc{red} \mytextsc{inceptive}}}}
\end{exe}
\begin{exe}
\sn \textcolor{darkblue}{\textbf{\ipa{dʑɯ˩ʁo˩˥ | tʰi˧-tse˩ (-dʑo˩)}}}
\trans faire flotter (dans un torrent), en montagne (ex.: des troncs qu'on ramène du lieu d'abattage jusqu'à la plaine)
\trans \textit{\textcolor{brown}{\zh{让木头漂到下游}}}
\end{exe}
\begin{exe}
\sn \textcolor{darkblue}{\textbf{\ipa{dʑɯ˩ʁo˩ tse˧}}}
\trans même sens: ramener de la montagne en faisant descendre la rivière
\trans \textit{\textcolor{brown}{\zh{同上:让木头漂到下游}}}
\end{exe}


\paragraph{\hspace{-0.5cm} {\Large \textcolor{darkblue}{\textbf{\ipa{tse˩\textsubscript{a}}}} \textsubscript{3}}} \hypertarget{tse\string_Ba3}{}
\markboth{\textcolor{darkblue}{\textbf{\ipa{tse˩\textsubscript{a}}}} \textsubscript{3}}{}
\textcolor{teal}{\mytextsc{verbe}} \hspace{4pt} Ton~: L\textsubscript{a}.

Fermer à clef.
\textcolor{brown}{\zh{锁门。}}


\begin{exe}
\sn \textcolor{darkblue}{\textbf{\ipa{kʰi˧ tse˥(-ze˩) / kʰi˧ tʰi˧-tse˩}}}
\trans verrouiller la porte
\trans \textit{\textcolor{brown}{\zh{锁门}}}
\end{exe}


\paragraph{\hspace{-0.5cm} {\Large \textcolor{darkblue}{\textbf{\ipa{tse˩pʰæ˧˥}}}}} \hypertarget{tse\string_Bp\string_h\{\string_M\string_T1}{}
\markboth{\textcolor{darkblue}{\textbf{\ipa{tse˩pʰæ˧˥}}}}{}
\textcolor{teal}{\mytextsc{nom}} \hspace{4pt} Ton~: LM+MH\#.

Pièces de l'époque impériale.
\textcolor{brown}{\zh{民国之前的货币。}}


\begin{exe}
\sn \textcolor{darkblue}{\textbf{\ipa{æ˧-tse˥pʰæ˩}}}
\trans pièce en bronze de l'époque impériale
\trans \textit{\textcolor{brown}{\zh{民国之前的铜币}}}
\end{exe}
 \mytextsc{clf}~: \textcolor{darkblue}{\textbf{\ipa{pʰæ˧˥}}} 

\paragraph{\hspace{-0.5cm} {\Large \textcolor{darkblue}{\textbf{\ipa{tse˩qwæ˧˥}}}}} \hypertarget{tse\string_Bqw\{\string_M\string_T1}{}
\markboth{\textcolor{darkblue}{\textbf{\ipa{tse˩qwæ˧˥}}}}{}
\textcolor{teal}{\mytextsc{nom}} \hspace{4pt} Ton~: LM+MH\#.

Clef.
\textcolor{brown}{\zh{钥匙。}}


 \mytextsc{clf}~: \textcolor{darkblue}{\textbf{\ipa{ɭɯ˧}}} 

\paragraph{\hspace{-0.5cm} {\Large \textcolor{darkblue}{\textbf{\ipa{tse˩tɑ˧˥}}}}} \hypertarget{tse\string_BtA\string_M\string_T1}{}
\markboth{\textcolor{darkblue}{\textbf{\ipa{tse˩tɑ˧˥}}}}{}
\textcolor{teal}{\mytextsc{nom}} \hspace{4pt} Ton~: LM+MH\#.

Ciseaux.
\textcolor{brown}{\zh{剪刀。}}


 \mytextsc{clf}~: \textcolor{darkblue}{\textbf{\ipa{nɑ˧}}} 

\paragraph{\hspace{-0.5cm} {\Large \textcolor{darkblue}{\textbf{\ipa{tse˩ʈʂʰv̩˩}}}}} \hypertarget{tse\string_Bt`s`\string_hv\string_=\string_B1}{}
\markboth{\textcolor{darkblue}{\textbf{\ipa{tse˩ʈʂʰv̩˩}}}}{}
\textcolor{teal}{\mytextsc{nom}} \hspace{4pt} Ton~: L.

Sac à puces: terme d'insulte pour un chien.
\textcolor{brown}{\zh{骂狗的话。}}


\textit{Voir~:} \hyperlink{tse\string_Bt`s`\string_hv\string_=\string_B-k\string_hv\string_=\string_T1}{\textcolor{darkblue}{\textbf{\ipa{tse˩ʈʂʰv̩˩-kʰv̩˥}}}} 

\paragraph{\hspace{-0.5cm} {\Large \textcolor{darkblue}{\textbf{\ipa{tse˩ʈʂʰv̩˩-kʰv̩˥}}}}} \hypertarget{tse\string_Bt`s`\string_hv\string_=\string_B-k\string_hv\string_=\string_T1}{}
\markboth{\textcolor{darkblue}{\textbf{\ipa{tse˩ʈʂʰv̩˩-kʰv̩˥}}}}{}
\textcolor{teal}{\mytextsc{nom}} \hspace{4pt} Ton~: L+H\#.

Sac à puces: terme d'insulte pour un chien.
\textcolor{brown}{\zh{骂狗的话。}}


\begin{exe}
\sn \textcolor{darkblue}{\textbf{\ipa{tse˩ʈʂʰv̩˩-kʰv̩˧ ! | mv̩˩tɕo˧ se˥ !}}}
\trans Descends, sac à puces! (Injonction adressée à un chien qui s'aventurait dans la partie haute de la salle à manger)
\trans \textit{\textcolor{brown}{\zh{你这坏狗,下去!}}}
\end{exe}
\textit{Voir~:} \hyperlink{tse\string_Bt`s`\string_hv\string_=\string_B1}{\textcolor{darkblue}{\textbf{\ipa{tse˩ʈʂʰv̩˩}}}} 

\paragraph{\hspace{-0.5cm} {\Large \textcolor{darkblue}{\textbf{\ipa{tse˩˥}}}}} \hypertarget{tse\string_B\string_T1}{}
\markboth{\textcolor{darkblue}{\textbf{\ipa{tse˩˥}}}}{}
\textcolor{teal}{\mytextsc{nom}} \hspace{4pt} Ton~: LH.

Serrure, verrou.
\textcolor{brown}{\zh{锁。}}


\begin{exe}
\sn \textcolor{darkblue}{\textbf{\ipa{æ˧tse˥}}}
\trans verrou en bronze
\trans \textit{\textcolor{brown}{\zh{铜锁}}}
\end{exe}
 \mytextsc{clf}~: \textcolor{darkblue}{\textbf{\ipa{nɑ˧}}} 

\paragraph{\hspace{-0.5cm} {\Large \textcolor{darkblue}{\textbf{\ipa{tsɤ˧}}} \textsubscript{1}}} \hypertarget{ts7\string_M1}{}
\markboth{\textcolor{darkblue}{\textbf{\ipa{tsɤ˧}}} \textsubscript{1}}{}
\textcolor{teal}{\mytextsc{verbe}} \hspace{4pt} Ton~: M intrans.

Se transformer, créer, devenir; être.
\textcolor{brown}{\zh{形成,变成。}}


\begin{exe}
\sn \textcolor{darkblue}{\textbf{\ipa{sɯ˧pv̩˧-sɯ˥nɑ˩-ʈʂʰɯ˩ | ə˧dzɤ˧\textasciitilde{}dzɤ˥-zo˩ | pʰi˧li˩ tsɤ˩-ɲi˩-kv̩˩-tsɯ˩ | -mv̩˩!}}}
\trans la chenille devient peu à peu papillon!
\trans \textit{\textcolor{brown}{\zh{毛虫能慢慢变成蝴蝶,不是吗?}}}
\end{exe}
\begin{exe}
\sn \textcolor{darkblue}{\textbf{\ipa{ɖɯ˧-bæ˧ mɤ˧-tsɤ˧}}}
\trans ce n'est pas la même chose, ce n'est pas pareil
\trans \textit{\textcolor{brown}{\zh{有区别、不一样}}}
\end{exe}


\paragraph{\hspace{-0.5cm} {\Large \textcolor{darkblue}{\textbf{\ipa{tsɤ˧}}} \textsubscript{2}}} \hypertarget{ts7\string_M2}{}
\markboth{\textcolor{darkblue}{\textbf{\ipa{tsɤ˧}}} \textsubscript{2}}{}
\textcolor{teal}{\mytextsc{adjectif}} \hspace{4pt} Ton~: M.

Correct, qui va bien.
\textcolor{brown}{\zh{对,合适,成。}}


\begin{exe}
\sn \textcolor{darkblue}{\textbf{\ipa{(le˧-)tsɤ˧-ze˧!}}}
\trans c'est bon, c'est arrangé!
\trans \textit{\textcolor{brown}{\zh{好了! / 弄好了! / 成!}}}
\end{exe}
\begin{exe}
\sn \textcolor{darkblue}{\textbf{\ipa{tsɤ˧-ʝi˧!}}}
\trans OK! C'est bon! (formule très courante, pour indiquer son acquiescement à une instruction reçue)
\trans \textit{\textcolor{brown}{\zh{行! / 好的!(表示同意或接受命令)}}}
\end{exe}
\begin{exe}
\sn \textcolor{darkblue}{\textbf{\ipa{tsɤ˧ ɲi˥!}}}
\trans c'est bon!
\trans \textit{\textcolor{brown}{\zh{好的!}}}
\end{exe}
\begin{exe}
\sn \textcolor{darkblue}{\textbf{\ipa{no˧ | mɤ˧-bi˧ mɤ˧-tsɤ˧!}}}
\trans Tu ne peux pas ne pas y aller!, littéralement ``que tu n'y ailles pas, ça ne va pas!"
\trans \textit{\textcolor{brown}{\zh{你如果不去,就不对! =你不能不去!}}}
\end{exe}
\begin{exe}
\sn \textcolor{darkblue}{\textbf{\ipa{ʈʂʰɯ˧ | ɖɯ˧-pi˧˥ | mɤ˧-tsɤ˧!}}}
\trans Lui, il est pas très net! / Y'a quelque chose qui va pas chez lui!
\trans \textit{\textcolor{brown}{\zh{他有一点不对劲吧!}}}
\end{exe}
\begin{exe}
\sn \textcolor{darkblue}{\textbf{\ipa{tsɤ˧ mɤ˧-ʝi˧-ze˧!}}}
\trans Ca ne va plus!
\trans \textit{\textcolor{brown}{\zh{不好了!/不行了!}}}
\end{exe}
\begin{exe}
\sn \textcolor{darkblue}{\textbf{\ipa{hĩ˧-ɳɯ˩ | le˧-so˩, | tsɤ˧!}}}
\trans Quand on t'apprend quelque chose, c'est une chance à saisir ! / Quand il se trouve quelqu'un qui est disposé à t'apprendre quelque chose, c'est une chance à saisir! / Si tu écoutes les bons conseils, tout ira bien! (Contexte: on évoque quelqu'un qui n'est pas enclin à écouter les bons conseils: qui se braque quand on lui fournit d'utiles conseils.)
\trans \textit{\textcolor{brown}{\zh{人家教,是好事! / 人家教,是要珍惜的! / 有人愿意教你,是件好事!}}}
\end{exe}
\begin{exe}
\sn \textcolor{darkblue}{\textbf{\ipa{hĩ˧-ɳɯ˩ | le˧-so˩, | tsɤ˧-kv˧˥!}}}
\trans même sens
\trans \textit{\textcolor{brown}{\zh{同上}}}
\end{exe}


\paragraph{\hspace{-0.5cm} {\Large \textcolor{darkblue}{\textbf{\ipa{tsɤ˧}}} \textsubscript{3}}} \hypertarget{ts7\string_M3}{}
\markboth{\textcolor{darkblue}{\textbf{\ipa{tsɤ˧}}} \textsubscript{3}}{}
\textcolor{teal}{\mytextsc{adjectif}} \hspace{4pt} Ton~: M.

Fine (poudre).
\textcolor{brown}{\zh{细(粉状)。}}


\begin{exe}
\sn \textcolor{darkblue}{\textbf{\ipa{tsɑ˧bɤ˧ tsɤ\#˥}}}
\trans farine fine
\trans \textit{\textcolor{brown}{\zh{细粮}}}
\end{exe}


\paragraph{\hspace{-0.5cm} {\Large \textcolor{darkblue}{\textbf{\ipa{tsɤ˧}}} \textsubscript{4}}} \hypertarget{ts7\string_M4}{}
\markboth{\textcolor{darkblue}{\textbf{\ipa{tsɤ˧}}} \textsubscript{4}}{}
\textcolor{teal}{\mytextsc{adjectif}} \hspace{4pt} Ton~: M.

Gourmand.
\textcolor{brown}{\zh{嘴馋。}}


\textit{Voir~:} \hyperlink{ts7\string_MRo\string_M-ts\string_hi\string_MRo\string_T1}{\textcolor{darkblue}{\textbf{\ipa{tsɤ˧ʁo˧-tsʰi˧ʁo˥}}}} 

\paragraph{\hspace{-0.5cm} {\Large \textcolor{darkblue}{\textbf{\ipa{tsɤ˧di˧}}}}} \hypertarget{ts7\string_Mdi\string_M1}{}
\markboth{\textcolor{darkblue}{\textbf{\ipa{tsɤ˧di˧}}}}{}
\textcolor{teal}{\mytextsc{nom}} \hspace{4pt} Ton~: M.

Arbre à épice, arbre à encens de grande taille.
\textcolor{brown}{\zh{香木。}}Dialecte chinois local~: \zh{柏香。}


\begin{exe}
\sn \textcolor{darkblue}{\textbf{\ipa{tsɤ˧di˧-dzi˩}}}
\trans même sens
\trans \textit{\textcolor{brown}{\zh{同上}}}
\end{exe}
\textit{Voir~:} \textcolor{darkblue}{\textbf{\ipa{ʁo˩kʰv˩}}} 

\paragraph{\hspace{-0.5cm} {\Large \textcolor{darkblue}{\textbf{\ipa{tsɤ˧ɖɯ˧}}}}} \hypertarget{ts7\string_Md`M\string_M1}{}
\markboth{\textcolor{darkblue}{\textbf{\ipa{tsɤ˧ɖɯ˧}}}}{}
\textcolor{teal}{\mytextsc{verbe}} \hspace{4pt} Ton~: .

Mettre bas (bovidé).
\textcolor{brown}{\zh{生崽子(牛类)。}}


\begin{exe}
\sn \textcolor{darkblue}{\textbf{\ipa{tsɤ˧ɖɯ˧-ze˩}}}
\trans \mytextsc{pfv}
\trans \textit{\textcolor{brown}{\zh{生崽子了}}}
\end{exe}
\begin{exe}
\sn \textcolor{darkblue}{\textbf{\ipa{(dʑi˧mi˧) tsɤ˧ɖɯ˧-ze˩}}}
\trans (le buffle) a enfanté.
\trans \textit{\textcolor{brown}{\zh{水牛生崽子了。}}}
\end{exe}


\paragraph{\hspace{-0.5cm} {\Large \textcolor{darkblue}{\textbf{\ipa{tsɤ˧ʁo˧-tsʰi˧ʁo˥}}}}} \hypertarget{ts7\string_MRo\string_M-ts\string_hi\string_MRo\string_T1}{}
\markboth{\textcolor{darkblue}{\textbf{\ipa{tsɤ˧ʁo˧-tsʰi˧ʁo˥}}}}{}
\textcolor{teal}{\mytextsc{adjectif}} \hspace{4pt} Ton~: H\#.

Gourmand.
\textcolor{brown}{\zh{馋。}}


\begin{exe}
\sn \textcolor{darkblue}{\textbf{\ipa{tsɤ˧ʁo˧-tsʰi˧ʁo˥ tsʰi˩}}}
\trans être gourmand
\trans \textit{\textcolor{brown}{\zh{馋}}}
\end{exe}
\textit{Voir~:} \hyperlink{ts7\string_M4}{\textcolor{darkblue}{\textbf{\ipa{tsɤ˧}}} \textsubscript{4}} 

\paragraph{\hspace{-0.5cm} {\Large \textcolor{darkblue}{\textbf{\ipa{tsɤ˧ʑi˧}}}}} \hypertarget{ts7\string_Mz£i\string_M1}{}
\markboth{\textcolor{darkblue}{\textbf{\ipa{tsɤ˧ʑi˧}}}}{}
\textcolor{teal}{\mytextsc{verbe}} \hspace{4pt} Ton~: .

Être grosse, être enceinte.
\textcolor{brown}{\zh{怀孕。}}




\paragraph{\hspace{-0.5cm} {\Large \textcolor{darkblue}{\textbf{\ipa{tsɤ˩pʰv̩˧-tsɤ˥li˩}}}}} \hypertarget{ts7\string_Bp\string_hv\string_=\string_M-ts7\string_Tli\string_B1}{}
\markboth{\textcolor{darkblue}{\textbf{\ipa{tsɤ˩pʰv̩˧-tsɤ˥li˩}}}}{}
\textcolor{teal}{\mytextsc{verbe}} \hspace{4pt} Ton~: .
\ding{202} 
(se) tourner en tous sens, (se) retourner dans tous les sens.
\textcolor{brown}{\zh{东翻西滚。}}


\begin{exe}
\sn \textcolor{darkblue}{\textbf{\ipa{ʈʂʰɯ˧ | tsɤ˩pʰv̩˧-tsɤ˥li˩-ɻ̍˩!}}}
\trans il/elle se retourne en tous sens! (ex.: un enfant qui se roule par terre)
\trans \textit{\textcolor{brown}{\zh{他在东翻西滚!}}}
\end{exe}
\ding{203} 
Être hypocrite, faux-jeton.
\textcolor{brown}{\zh{伪善。}}




\paragraph{\hspace{-0.5cm} {\Large \textcolor{darkblue}{\textbf{\ipa{tsi˥}}}}} \hypertarget{tsi\string_T1}{}
\markboth{\textcolor{darkblue}{\textbf{\ipa{tsi˥}}}}{}
\textcolor{teal}{\mytextsc{nom}} \hspace{4pt} Ton~: \#H.
\ding{202} 
Fissure, interstice.
\textcolor{brown}{\zh{裂缝、缝隙。}}


\begin{exe}
\sn \textcolor{darkblue}{\textbf{\ipa{tsi˧ qʰwæ˧-ze˥!}}}
\trans il y a une fissure qui s'est faite! / ça s'est fendu! to crack; to develop a chink/crack/fissure
\trans \textit{\textcolor{brown}{\zh{有了裂缝!}}}
\end{exe}
\begin{exe}
\sn \textcolor{darkblue}{\textbf{\ipa{tsi˧ hɯ˧-ze˧!}}}
\trans ça s'est fissuré!
\trans \textit{\textcolor{brown}{\zh{有了裂缝!}}}
\end{exe}
\ding{203} 
Couture.
\textcolor{brown}{\zh{针脚。}}


 \mytextsc{clf}~: \textcolor{darkblue}{\textbf{\ipa{pʰæ˧˥}}} 

\paragraph{\hspace{-0.5cm} {\Large \textcolor{darkblue}{\textbf{\ipa{tsi˧\textsubscript{a}}}}}} \hypertarget{tsi\string_Ma1}{}
\markboth{\textcolor{darkblue}{\textbf{\ipa{tsi˧\textsubscript{a}}}}}{}
\textcolor{teal}{\mytextsc{adjectif}} \hspace{4pt} Ton~: M\textsubscript{a}.

Piquant, pimenté.
\textcolor{brown}{\zh{辣。}}


\begin{exe}
\sn \textcolor{darkblue}{\textbf{\ipa{ʈʂʰɯ˧ tsi˧-hĩ˧ ɲi˥!}}}
\trans c'est piquant/c'est pimenté!
\trans \textit{\textcolor{brown}{\zh{这是辣的!}}}
\end{exe}


\paragraph{\hspace{-0.5cm} {\Large \textcolor{darkblue}{\textbf{\ipa{tsi˧\textsubscript{b}}}}}} \hypertarget{tsi\string_Mb1}{}
\markboth{\textcolor{darkblue}{\textbf{\ipa{tsi˧\textsubscript{b}}}}}{}
\textcolor{teal}{\mytextsc{verbe}} \hspace{4pt} Ton~: M\textsubscript{b}.

Fixer, installer, mettre en place (ex.: un pilier, dans une maison en construction).
\textcolor{brown}{\zh{安装。}}


\begin{exe}
\sn \textcolor{darkblue}{\textbf{\ipa{tso˧\textasciitilde{}tso˧ tsi˧}}}
\trans installer quelque chose
\trans \textit{\textcolor{brown}{\zh{安装东西}}}
\end{exe}
\begin{exe}
\sn \textcolor{darkblue}{\textbf{\ipa{tso˧\textasciitilde{}tso˧ | gɤ˩-tsi˧-ɻ̍˥}}}
\trans installer quelque chose, mettre quelque chose en place
\trans \textit{\textcolor{brown}{\zh{安装东西}}}
\end{exe}
\begin{exe}
\sn \textcolor{darkblue}{\textbf{\ipa{gɤ˩-tsi˧ tʰi˧-tɕɯ˥}}}
\trans (re)lever, (re)mettre d'aplomb
\trans \textit{\textcolor{brown}{\zh{立起来}}}
\end{exe}


\paragraph{\hspace{-0.5cm} {\Large \textcolor{darkblue}{\textbf{\ipa{tsi˧gi˥\$}}}}} \hypertarget{tsi\string_Mgi\string_T\$1}{}
\markboth{\textcolor{darkblue}{\textbf{\ipa{tsi˧gi˥\$}}}}{}
\textcolor{teal}{\mytextsc{nom}} \hspace{4pt} Ton~: H\$.

Fissure.
\textcolor{brown}{\zh{缝隙,例如:墙上的。}}


\begin{exe}
\sn \textcolor{darkblue}{\textbf{\ipa{tsi˧gi˥ | ɖɯ˧-kʰwɤ˥}}}
\trans une fissure
\trans \textit{\textcolor{brown}{\zh{一个缝隙}}}
\end{exe}
\begin{exe}
\sn \textcolor{darkblue}{\textbf{\ipa{tsi˧gi˥ | ɖɯ˧-kʰwɤ˧ tʰi˧-di˥}}}
\trans il y a une fissure
\trans \textit{\textcolor{brown}{\zh{有一个缝隙}}}
\end{exe}
 \mytextsc{clf}~: \textcolor{darkblue}{\textbf{\ipa{ɭɯ˧, kʰwɤ˥}}} 

\paragraph{\hspace{-0.5cm} {\Large \textcolor{darkblue}{\textbf{\ipa{tsi˩\textsubscript{a}}}}}} \hypertarget{tsi\string_Ba1}{}
\markboth{\textcolor{darkblue}{\textbf{\ipa{tsi˩\textsubscript{a}}}}}{}
\textcolor{teal}{\mytextsc{verbe}} \hspace{4pt} Ton~: L\textsubscript{a}.

Faire bouillir.
\textcolor{brown}{\zh{烧开。}}


\begin{exe}
\sn \textcolor{darkblue}{\textbf{\ipa{dʑɯ˧ | le˧-tsi˩-tʰv̩˩-ze˩!}}}
\trans L'eau bout!
\trans \textit{\textcolor{brown}{\zh{水开了!}}}
\end{exe}
\begin{exe}
\sn \textcolor{darkblue}{\textbf{\ipa{dʑɯ˩ tsi˩-tʰv̩˩-ze˥!}}}
\trans L'eau bout!
\trans \textit{\textcolor{brown}{\zh{水开了!}}}
\end{exe}
\begin{exe}
\sn \textcolor{darkblue}{\textbf{\ipa{mɤ˧-tsi˩-tʰv̩˩-sɯ˩!}}}
\trans Ca ne bout pas encore!
\trans \textit{\textcolor{brown}{\zh{还没有烧开!}}}
\end{exe}
\begin{exe}
\sn \textcolor{darkblue}{\textbf{\ipa{ɖɯ˧-tsi˩-tʰv̩˩-ɻ̍˩-kʰɯ˩}}}
\trans laisser bouillir un moment
\trans \textit{\textcolor{brown}{\zh{煮一会儿}}}
\end{exe}
\begin{exe}
\sn \textcolor{darkblue}{\textbf{\ipa{ɖɯ˧-tsi˧\textasciitilde{}tsi˥-ɻ̍˩ kʰɯ˩}}}
\trans faire bouillir un moment
\trans \textit{\textcolor{brown}{\zh{煮一会儿}}}
\end{exe}


\paragraph{\hspace{-0.5cm} {\Large \textcolor{darkblue}{\textbf{\ipa{tsi˩ɭɯ˩}}}}} \hypertarget{tsi\string_Bl\string_RM\string_B1}{}
\markboth{\textcolor{darkblue}{\textbf{\ipa{tsi˩ɭɯ˩}}}}{}
\textcolor{teal}{\mytextsc{nom}} \hspace{4pt} Ton~: L.

Petit oiseau de couleur bleue/verte.
\textcolor{brown}{\zh{一种小鸟。}}


 \mytextsc{clf}~: \textcolor{darkblue}{\textbf{\ipa{mi˩}}} 

\paragraph{\hspace{-0.5cm} {\Large \textcolor{darkblue}{\textbf{\ipa{}}}}} \hypertarget{tsi\string_Bl\string_RM\string_B-bv\string_=\string_M | -q\string_h\{\string_Bb\{\string_B\string_T1}{}
\markboth{\textcolor{darkblue}{\textbf{\ipa{}}}}{}
\textcolor{teal}{\mytextsc{nom}} \hspace{4pt} Ton~: .

Herbe rampante.
\textcolor{brown}{\zh{蔓草。}}




\paragraph{\hspace{-0.5cm} {\Large \textcolor{darkblue}{\textbf{\ipa{‑tso}}}}} \hypertarget{‑tso1}{}
\markboth{\textcolor{darkblue}{\textbf{\ipa{‑tso}}}}{}
\textcolor{teal}{\mytextsc{suffixe}} \hspace{4pt} Ton~: .

Nominalizer.
\textcolor{brown}{\zh{\mytextsc{名物化。}}}


\begin{exe}
\sn \textcolor{darkblue}{\textbf{\ipa{bæ˩-ʂo˧-tso˧ tʰi˧-tʰv̩˧-ho˩-ze˩!}}}
\trans la récolte va parvenir à terme/ les produits de la récolte seront bientôt mûrs!
\trans \textit{\textcolor{brown}{\zh{庄稼快要熟了!}}}
\end{exe}
\begin{exe}
\sn \textcolor{darkblue}{\textbf{\ipa{ʈʰɯ˩-tso˩˥}}}
\trans Chose liquide qui se boit; pas exactement identique avec \textcolor{darkblue}{\textbf{\ipa{/ʈʰɯ˩-di˩˥/}}} 'chose à boire, boisson'.
\trans \textit{\textcolor{brown}{\zh{喝的东西}}}
\end{exe}
\begin{exe}
\sn \textcolor{darkblue}{\textbf{\ipa{v̩˩dʑɯ˩˥, | ʈʰɯ˩-tso˩ ɲi˥!}}}
\trans la soupe, c'est liquide/ça se boit!
\trans \textit{\textcolor{brown}{\zh{汤,是喝的东西! / 汤是来喝的! / 汤,算是属于饮料类的!}}}
\end{exe}
\begin{exe}
\sn \textcolor{darkblue}{\textbf{\ipa{dʑɤ˩bv̩˧-tso˩}}}
\trans distraction, jeu, divertissement; le sens est distinct d'une autre forme nominalisée, \textcolor{darkblue}{\textbf{\ipa{/dʑɤ˩bv̩˧-di˩/}}} 'jouet' (objet)
\trans \textit{\textcolor{brown}{\zh{娱乐、游戏}}}
\end{exe}
\begin{exe}
\sn \textcolor{darkblue}{\textbf{\ipa{ʝi˧-tso˧ mɤ˧-ʂv̩˧ɖv̩˧, | dʑɤ˩bv̩˧-tso˩-lɑ˩ ʂv̩˩ɖv̩˩!}}}
\trans (il) ne pense pas à ses tâches/à ses devoirs/à ce qu'il doit faire, il ne pense qu'à ses distractions!
\trans \textit{\textcolor{brown}{\zh{不考虑任务,只想到娱乐! / 干正经事不想,只想玩!}}}
\end{exe}
\begin{exe}
\sn \textcolor{darkblue}{\textbf{\ipa{tv̩˧tso˧}}}
\trans chose que l'on plante. (La forme nominalisée \textcolor{darkblue}{\textbf{\ipa{/tv̩˧-di˩/}}} existe aussi mais il y a un conflit homophonique entre 'planter'+'terre', \textcolor{darkblue}{\textbf{\ipa{/tv̩˧-di˩/}}}, et le suffixe \textcolor{darkblue}{\textbf{\ipa{/-di/}}} 'nominalisateur', de sorte que pour dire 'chose qu'on plante', on ne dit pas \textcolor{darkblue}{\textbf{\ipa{/tv̩˧-di˩/}}}, mais \textcolor{darkblue}{\textbf{\ipa{/tv̩˧-tso˧/}}}.
\trans \textit{\textcolor{brown}{\zh{种的东西=农作物。存在有另外一种名物化:\textcolor{darkblue}{\textbf{\ipa{/tv̩˧-di˩/}}},可是那个与‘可以种的地、农业土地’\textcolor{darkblue}{\textbf{\ipa{/tv̩˧-di˩/}}}同音,于是用\textcolor{darkblue}{\textbf{\ipa{/tv̩˧-tso˧/}}}来表示‘农作物’而不用\textcolor{darkblue}{\textbf{\ipa{/tv̩˧-di˩/}}}。}}}
\end{exe}


\paragraph{\hspace{-0.5cm} {\Large \textcolor{darkblue}{\textbf{\ipa{‑tso˧}}}}} \hypertarget{‑tso\string_M1}{}
\markboth{\textcolor{darkblue}{\textbf{\ipa{‑tso˧}}}}{}
\textcolor{teal}{\mytextsc{suffixe}} \hspace{4pt} Ton~: M.

\mytextsc{volitif}.
\textcolor{brown}{\zh{\mytextsc{意志。}}}


\begin{exe}
\sn \textcolor{darkblue}{\textbf{\ipa{dʑɤ˩bv̩˥-tso˩-ɲi˩-mæ˩!}}}
\trans On joue, d'accord? / Allez, on va jouer!
\trans \textit{\textcolor{brown}{\zh{玩一玩吧!}}}
\end{exe}
\begin{exe}
\sn \textcolor{darkblue}{\textbf{\ipa{ə˧tso˧ tv̩˧-tso˧-ɲi˥ ?}}}
\trans Qu'est-ce que vous comptez planter?
\trans \textit{\textcolor{brown}{\zh{要种什么东西?}}}
\end{exe}
\begin{exe}
\sn \textcolor{darkblue}{\textbf{\ipa{ə˧tso˧ ʝi˧-bi˧-tso˧-ɲi˥?}}}
\trans Qu'est-ce que vous comptez faire maintenant?
\trans \textit{\textcolor{brown}{\zh{要做什么了?}}}
\end{exe}


\paragraph{\hspace{-0.5cm} {\Large \textcolor{darkblue}{\textbf{\ipa{tso˧kʰwɤ\#˥}}}}} \hypertarget{tso\string_Mk\string_hw7\#\string_T1}{}
\markboth{\textcolor{darkblue}{\textbf{\ipa{tso˧kʰwɤ\#˥}}}}{}
\textcolor{teal}{\mytextsc{nom}} \hspace{4pt} Ton~: \#H.

Sac (était fait en toile ou en cuir).
\textcolor{brown}{\zh{袋子。}}


 \mytextsc{clf}~: \textcolor{darkblue}{\textbf{\ipa{ɭɯ˧}}} 

\paragraph{\hspace{-0.5cm} {\Large \textcolor{darkblue}{\textbf{\ipa{tso˧lo˧-mv̩˥tso˩}}}}} \hypertarget{tso\string_Mlo\string_M-mv\string_=\string_Ttso\string_B1}{}
\markboth{\textcolor{darkblue}{\textbf{\ipa{tso˧lo˧-mv̩˥tso˩}}}}{}
\textcolor{teal}{\mytextsc{nom}} \hspace{4pt} Ton~: \#H-.

Outil; chose, objet, truc.
\textcolor{brown}{\zh{东西,工具。}}


 \mytextsc{clf}~: \textcolor{darkblue}{\textbf{\ipa{nɑ˧, ɭɯ˧}}} 

\paragraph{\hspace{-0.5cm} {\Large \textcolor{darkblue}{\textbf{\ipa{tso˧qwɤ˧}}}}} \hypertarget{tso\string_Mqw7\string_M1}{}
\markboth{\textcolor{darkblue}{\textbf{\ipa{tso˧qwɤ˧}}}}{}
\textcolor{teal}{\mytextsc{nom}} \hspace{4pt} Ton~: M.

Chambrette: partie de la pièce principale dans laquelle se trouve un couchage; on y place provisoirement les nouveaux-nés, et les défunts.
\textcolor{brown}{\zh{小床角:主屋里面的一个角落,有床垫子。用餐、招待客人的时候,会有人在上面坐。刚出生的婴儿也在此处睡觉。人去世后,尸体先放在那个地方。}}


 \mytextsc{clf}~: \textcolor{darkblue}{\textbf{\ipa{ɭɯ˧}}} 

\paragraph{\hspace{-0.5cm} {\Large \textcolor{darkblue}{\textbf{\ipa{tso˧tso\#˥}}}}} \hypertarget{tso\string_Mtso\#\string_T1}{}
\markboth{\textcolor{darkblue}{\textbf{\ipa{tso˧tso\#˥}}}}{}
\textcolor{teal}{\mytextsc{nom}} \hspace{4pt} Ton~: \#H.

Chose, truc, bidule, objet, machin.
\textcolor{brown}{\zh{东西。}}


\begin{exe}
\sn \textcolor{darkblue}{\textbf{\ipa{tso˧\textasciitilde{}tso˧-zo˧\textasciitilde{}zo˧-mv̩˧\textasciitilde{}mv̩˥}}}
\trans bidule
\trans \textit{\textcolor{brown}{\zh{各种东西、各种各样乱七八糟东西}}}
\end{exe}
\begin{exe}
\sn \textcolor{darkblue}{\textbf{\ipa{tso˧\textasciitilde{}tso˧ hwæ˩}}}
\trans acheter des choses
\trans \textit{\textcolor{brown}{\zh{买东西}}}
\end{exe}
\begin{exe}
\sn \textcolor{darkblue}{\textbf{\ipa{tso˧\textasciitilde{}tso˧ tɕʰi˧(-ze˩)}}}
\trans vendre des choses
\trans \textit{\textcolor{brown}{\zh{卖东西}}}
\end{exe}
\begin{exe}
\sn \textcolor{darkblue}{\textbf{\ipa{tso˧\textasciitilde{}tso˧ dzɯ˧(-ze˩)}}}
\trans manger des choses
\trans \textit{\textcolor{brown}{\zh{吃东西}}}
\end{exe}
\begin{exe}
\sn \textcolor{darkblue}{\textbf{\ipa{tso˧\textasciitilde{}tso˧ dze˥}}}
\trans couper des choses
\trans \textit{\textcolor{brown}{\zh{切东西}}}
\end{exe}
\begin{exe}
\sn \textcolor{darkblue}{\textbf{\ipa{tso˧\textasciitilde{}tso˧ ʈʰɯ˩}}}
\trans boire des choses
\trans \textit{\textcolor{brown}{\zh{喝东西}}}
\end{exe}
\begin{exe}
\sn \textcolor{darkblue}{\textbf{\ipa{tso˧\textasciitilde{}tso˧ lɑ˩}}}
\trans frapper des choses
\trans \textit{\textcolor{brown}{\zh{打东西}}}
\end{exe}
 \mytextsc{clf}~: \textcolor{darkblue}{\textbf{\ipa{nɑ˧, ɭɯ˧}}} 

\paragraph{\hspace{-0.5cm} {\Large \textcolor{darkblue}{\textbf{\ipa{tso˩\textsubscript{c}}}}}} \hypertarget{tso\string_Bc1}{}
\markboth{\textcolor{darkblue}{\textbf{\ipa{tso˩\textsubscript{c}}}}}{}
\textcolor{teal}{\mytextsc{classificateur}} \hspace{4pt} Ton~: L\textsubscript{c}.

Classificateur des pièces (dans la maison), des compartiments (dans un grenier).
\textcolor{brown}{\zh{量词:间(房间,分隔间,包间)。}}


\begin{exe}
\sn \textcolor{darkblue}{\textbf{\ipa{ʈʂʰɯ˧-tso˥}}}
\trans cette pièce
\trans \textit{\textcolor{brown}{\zh{这间}}}
\end{exe}


\paragraph{\hspace{-0.5cm} {\Large \textcolor{darkblue}{\textbf{\ipa{tso˩\textsubscript{a}}}}}} \hypertarget{tso\string_Ba1}{}
\markboth{\textcolor{darkblue}{\textbf{\ipa{tso˩\textsubscript{a}}}}}{}
\textcolor{teal}{\mytextsc{verbe}} \hspace{4pt} Ton~: L\textsubscript{a}.

Construire un mur, un pont….
\textcolor{brown}{\zh{砌(墙)、建(桥梁)。}}


\begin{exe}
\sn \textcolor{darkblue}{\textbf{\ipa{ɖʐɤ˩ tso˩}}}
\trans construire un escalier
\trans \textit{\textcolor{brown}{\zh{修一座楼梯}}}
\end{exe}
\begin{exe}
\sn \textcolor{darkblue}{\textbf{\ipa{dzo˩ tso˩}}}
\trans construire un pont
\trans \textit{\textcolor{brown}{\zh{修一座桥}}}
\end{exe}
\begin{exe}
\sn \textcolor{darkblue}{\textbf{\ipa{ɖɯ˧-tso˧\textasciitilde{}tso˥-ɻ̍˩}}}
\trans construire quelque chose
\trans \textit{\textcolor{brown}{\zh{修东西}}}
\end{exe}


\paragraph{\hspace{-0.5cm} {\Large \textcolor{darkblue}{\textbf{\ipa{tso˩qʰv̩˩ɻ̍˥}}}}} \hypertarget{tso\string_Bq\string_hv\string_=\string_Br£`̍\string_T1}{}
\markboth{\textcolor{darkblue}{\textbf{\ipa{tso˩qʰv̩˩ɻ̍˥}}}}{}
\textcolor{teal}{\mytextsc{nom}} \hspace{4pt} Ton~: H\#.

Porche, vestibule: espace situé entre la cour et la pièce principale, c'est-à-dire l'espace, protégé de la pluie par la toiture, où l'on parvient lorsqu'on passe le seuil en sortant de la pièce principale. Dans certaines maisons, ce porche est séparé de la cour par une cloison de bois, percée d'une porte à peu près au milieu de sa longueur.
\textcolor{brown}{\zh{玄关、门厅。}}


 \mytextsc{clf}~: \textcolor{darkblue}{\textbf{\ipa{kʰwɤ˥}}} 

\paragraph{\hspace{-0.5cm} {\Large \textcolor{darkblue}{\textbf{\ipa{tso˩\textasciitilde{}tso˧˥}}}}} \hypertarget{tso\string_B~tso\string_M\string_T1}{}
\markboth{\textcolor{darkblue}{\textbf{\ipa{tso˩\textasciitilde{}tso˧˥}}}}{}
\textcolor{teal}{\mytextsc{verbe}} \hspace{4pt} Ton~: LM+MH\#.

Touiller, faire la pâtée (du chien…).
\textcolor{brown}{\zh{拌好狗食。}}




\paragraph{\hspace{-0.5cm} {\Large \textcolor{darkblue}{\textbf{\ipa{tsɯ˥}}} \textsubscript{1}}} \hypertarget{tsM\string_T1}{}
\markboth{\textcolor{darkblue}{\textbf{\ipa{tsɯ˥}}} \textsubscript{1}}{}
\textcolor{teal}{\mytextsc{verbe}} \hspace{4pt} Ton~: H.
\ding{202} 
Attacher.
\textcolor{brown}{\zh{绑、捆、栓。}}


\begin{exe}
\sn \textcolor{darkblue}{\textbf{\ipa{dʑi˧mi˧ tʰi˧-tsɯ˥}}}
\trans attacher le buffle
\trans \textit{\textcolor{brown}{\zh{栓水牛}}}
\end{exe}
\begin{exe}
\sn \textcolor{darkblue}{\textbf{\ipa{tsɯ˧\textasciitilde{}tsɯ˧}}}
\trans \mytextsc{red}
\trans \textit{\textcolor{brown}{\zh{\mytextsc{重叠}}}}
\end{exe}
\begin{exe}
\sn \textcolor{darkblue}{\textbf{\ipa{le˧-tsɯ˧\textasciitilde{}tsɯ˧}}}
\trans \mytextsc{accomp} \mytextsc{red}
\trans \textit{\textcolor{brown}{\zh{\mytextsc{accomp} \mytextsc{red}}}}
\end{exe}
\begin{exe}
\sn \textcolor{darkblue}{\textbf{\ipa{tʰi˧-tsɯ˧\textasciitilde{}tsɯ˧}}}
\trans \mytextsc{dur} \mytextsc{red}
\trans \textit{\textcolor{brown}{\zh{\mytextsc{dur} \mytextsc{red}}}}
\end{exe}
\ding{203} 
Se pendre.
\textcolor{brown}{\zh{上吊自杀、缢。}}


\begin{exe}
\sn \textcolor{darkblue}{\textbf{\ipa{ʁæ˧tsɯ˧ le˧-ʂɯ˧ +ze˧}}}
\trans se pendre
\trans \textit{\textcolor{brown}{\zh{上吊自杀、缢}}}
\end{exe}


\paragraph{\hspace{-0.5cm} {\Large \textcolor{darkblue}{\textbf{\ipa{tsɯ˥}}} \textsubscript{2}}} \hypertarget{tsM\string_T2}{}
\markboth{\textcolor{darkblue}{\textbf{\ipa{tsɯ˥}}} \textsubscript{2}}{}
\textcolor{teal}{\mytextsc{verbe}} \hspace{4pt} Ton~: H.

Prendre avec une écumoire, récupérer dans l'eau.
\textcolor{brown}{\zh{打捞。}}




\paragraph{\hspace{-0.5cm} {\Large \textcolor{darkblue}{\textbf{\ipa{tsɯ˧}}}}} \hypertarget{tsM\string_M1}{}
\markboth{\textcolor{darkblue}{\textbf{\ipa{tsɯ˧}}}}{}
\textcolor{teal}{\mytextsc{nom}} \hspace{4pt} Ton~: M.

Lettre, caractère chinois.
\textcolor{brown}{\zh{字。}}

 Emprunt~: chinois \textcolor{brown}{\zh{字}}



\paragraph{\hspace{-0.5cm} {\Large \textcolor{darkblue}{\textbf{\ipa{tsɯ˩\textsubscript{a}}}}}} \hypertarget{tsM\string_Ba1}{}
\markboth{\textcolor{darkblue}{\textbf{\ipa{tsɯ˩\textsubscript{a}}}}}{}
\textcolor{teal}{\mytextsc{verbe}} \hspace{4pt} Ton~: L\textsubscript{a}.

Boucher/être bouché; obstruer (ex.: obstruer l'entrée d'un trou).
\textcolor{brown}{\zh{堵塞、塞住洞口。}}




\paragraph{\hspace{-0.5cm} {\Large \textcolor{darkblue}{\textbf{\ipa{tsɯ˩pʰɤ˩}}}}} \hypertarget{tsM\string_Bp\string_h7\string_B1}{}
\markboth{\textcolor{darkblue}{\textbf{\ipa{tsɯ˩pʰɤ˩}}}}{}
\textcolor{teal}{\mytextsc{verbe}} \hspace{4pt} Ton~: L.
\ding{202} 
Cligner des yeux.
\textcolor{brown}{\zh{眨眼。}}


\begin{exe}
\sn \textcolor{darkblue}{\textbf{\ipa{mɤ˧-tsɯ˩pʰɤ˩}}}
\trans \mytextsc{neg}
\trans \textit{\textcolor{brown}{\zh{不眨眼}}}
\end{exe}
\begin{exe}
\sn \textcolor{darkblue}{\textbf{\ipa{ɖɯ˧-tsɯ˧\textasciitilde{}tsɯ˥-ɻ̍˩}}}
\trans \mytextsc{délimitatif} \mytextsc{red} \mytextsc{inchoatif}
\trans \textit{\textcolor{brown}{\zh{\mytextsc{delimitative} \mytextsc{red} \mytextsc{inceptive}}}}
\end{exe}
\begin{exe}
\sn \textcolor{darkblue}{\textbf{\ipa{njɤ˩ɭɯ˧ tsɯ˩pʰɤ˩}}}
\trans cligner des yeux
\trans \textit{\textcolor{brown}{\zh{眨眼}}}
\end{exe}
\ding{203} 
Faire un clin d'oeil (discret signe d'intelligence).
\textcolor{brown}{\zh{眨眼。}}


\begin{exe}
\sn \textcolor{darkblue}{\textbf{\ipa{ʈʂʰɯ˧ | njɤ˩ɭɯ˧ tsɯ˩pʰɤ˩-dʑo˩!}}}
\trans Elle/il est en train de faire un clin d'oeil!
\trans \textit{\textcolor{brown}{\zh{他在眨眼!}}}
\end{exe}


\paragraph{\hspace{-0.5cm} {\Large \textcolor{darkblue}{\textbf{\ipa{tsɯ˧˥}}}}} \hypertarget{tsM\string_M\string_T1}{}
\markboth{\textcolor{darkblue}{\textbf{\ipa{tsɯ˧˥}}}}{}
\textcolor{teal}{\mytextsc{suffixe}} \hspace{4pt} Ton~: MH.

Particule d'évidentialité rapportée: elle indique une connaissance indirecte, par ouï-dire, et non par connaissance directe.
\textcolor{brown}{\zh{据说\mytextsc{传闻据素。}}}




\paragraph{\hspace{-0.5cm} {\Large \textcolor{darkblue}{\textbf{\ipa{tsɯ˧˥}}}}} \hypertarget{tsM\string_M\string_T1}{}
\markboth{\textcolor{darkblue}{\textbf{\ipa{tsɯ˧˥}}}}{}
\textcolor{teal}{\mytextsc{verbe}} \hspace{4pt} Ton~: MH.

Appeler, nommer, désigner.
\textcolor{brown}{\zh{叫、叫做。}}


\begin{exe}
\sn \textcolor{darkblue}{\textbf{\ipa{ʈæ˧ʂɯ˧-ɳɯ˧ | no˧-ki˥ | jɤ˩-ʐe˧ ɲi˩-tsɯ˩-mɤ˩-tsɯ˩!}}}
\trans \textcolor{darkblue}{\textbf{\ipa{/ʈæ˧ʂɯ˧/}}} t'appelle ``l'étranger" / il te traite d'étranger!
\trans \textit{\textcolor{brown}{\zh{达石把你叫作“老外”!}}}
\end{exe}


\paragraph{\hspace{-0.5cm} {\Large \textcolor{darkblue}{\textbf{\ipa{tsʰɑ˧bo\#˥}}}}} \hypertarget{ts\string_hA\string_Mbo\#\string_T1}{}
\markboth{\textcolor{darkblue}{\textbf{\ipa{tsʰɑ˧bo\#˥}}}}{}
\textcolor{teal}{\mytextsc{nom}} \hspace{4pt} Ton~: \#H.

Cuisinier.
\textcolor{brown}{\zh{厨师。}}


\begin{exe}
\sn \textcolor{darkblue}{\textbf{\ipa{tsʰɑ˧bo˧ lɑ˩}}}
\trans être cuisinier, s'engager comme cuisinier, faire le travail de cuisinier
\trans \textit{\textcolor{brown}{\zh{当厨师}}}
\end{exe}
\begin{exe}
\sn \textcolor{darkblue}{\textbf{\ipa{tsʰɑ˧bo˧ ʝi˧}}}
\trans être cuisinier, s'engager comme cuisinier, faire le travail de cuisinier
\trans \textit{\textcolor{brown}{\zh{当厨师}}}
\end{exe}


\paragraph{\hspace{-0.5cm} {\Large \textcolor{darkblue}{\textbf{\ipa{tsʰɑ˧kv̩˩}}}}} \hypertarget{ts\string_hA\string_Mkv\string_=\string_B1}{}
\markboth{\textcolor{darkblue}{\textbf{\ipa{tsʰɑ˧kv̩˩}}}}{}
\textcolor{teal}{\mytextsc{nom}} \hspace{4pt} Ton~: L\#.

Réserve, magasin.
\textcolor{brown}{\zh{仓库(汉语借词)。}}

 Emprunt~: chinois \textcolor{brown}{\zh{仓库}}



\paragraph{\hspace{-0.5cm} {\Large \textcolor{darkblue}{\textbf{\ipa{tsʰɑ˧tɕɤ˧˥}}}}} \hypertarget{ts\string_hA\string_Mts£7\string_M\string_T1}{}
\markboth{\textcolor{darkblue}{\textbf{\ipa{tsʰɑ˧tɕɤ˧˥}}}}{}
\textcolor{teal}{\mytextsc{nom}} \hspace{4pt} Ton~: MH.

Jeunes pousses, petites pousses qu'on récolte pour les manger.
\textcolor{brown}{\zh{青菜幼苗。}}




\paragraph{\hspace{-0.5cm} {\Large \textcolor{darkblue}{\textbf{\ipa{tsʰɑ˩pʰɑ˩lɑ˥}}}}} \hypertarget{ts\string_hA\string_Bp\string_hA\string_BlA\string_T1}{}
\markboth{\textcolor{darkblue}{\textbf{\ipa{tsʰɑ˩pʰɑ˩lɑ˥}}}}{}
\textcolor{teal}{\mytextsc{nom}} \hspace{4pt} Ton~: L+H\#.

Feuilles d'épi de maïs: les feuilles qui entourent l'épi de maïs.
\textcolor{brown}{\zh{苞谷叶(玉米穰子的叶子)。}}


\begin{exe}
\sn \textcolor{darkblue}{\textbf{\ipa{qʰɑ˧dze˧-tsʰɑ˩pʰɑ˩lɑ˩}}}
\trans même sens
\trans \textit{\textcolor{brown}{\zh{同上}}}
\end{exe}


\paragraph{\hspace{-0.5cm} {\Large \textcolor{darkblue}{\textbf{\ipa{tsʰæ˧pʰv˧˥}}}}} \hypertarget{ts\string_h\{\string_Mp\string_hv\string_M\string_T1}{}
\markboth{\textcolor{darkblue}{\textbf{\ipa{tsʰæ˧pʰv˧˥}}}}{}
\textcolor{teal}{\mytextsc{nom}} \hspace{4pt} Ton~: MH\#.

Chou chinois. Il s'agit d'un calque du chinois 'légume blanc', employant le nom chinois pour 'légume' associé à l'adjectif na pour 'blanc'.
\textcolor{brown}{\zh{白菜(借汉语‘白菜’的第二个音节来充当这个名词的第一个音节:按摩梭话句法,形容词在名词后面,跟汉语相反)。}}

 Emprunt~: chinois \textcolor{brown}{\zh{菜}}

 \mytextsc{clf}~: \textcolor{darkblue}{\textbf{\ipa{po˧}}} \textit{Voir~:} \hyperlink{v\string_=\string_Bts\string_h7\string_M-p\string_hv\string_=\string_T1}{\textcolor{darkblue}{\textbf{\ipa{v̩˩tsʰɤ˧-pʰv̩˥}}}} 

\paragraph{\hspace{-0.5cm} {\Large \textcolor{darkblue}{\textbf{\ipa{tsʰe\#˥}}}}} \hypertarget{ts\string_he\#\string_T1}{}
\markboth{\textcolor{darkblue}{\textbf{\ipa{tsʰe\#˥}}}}{}
\textcolor{teal}{\mytextsc{nom}} \hspace{4pt} Ton~: \#H.

Sel.
\textcolor{brown}{\zh{盐。}}




\paragraph{\hspace{-0.5cm} {\Large \textcolor{darkblue}{\textbf{\ipa{tsʰe˧}}}}} \hypertarget{ts\string_he\string_M1}{}
\markboth{\textcolor{darkblue}{\textbf{\ipa{tsʰe˧}}}}{}
\textcolor{teal}{\mytextsc{nombre}} \hspace{4pt} Ton~: L.

10.
\textcolor{brown}{\zh{10。}}




\paragraph{\hspace{-0.5cm} {\Large \textcolor{darkblue}{\textbf{\ipa{tsʰe˧do˧˥}}}}} \hypertarget{ts\string_he\string_Mdo\string_M\string_T1}{}
\markboth{\textcolor{darkblue}{\textbf{\ipa{tsʰe˧do˧˥}}}}{}
\textcolor{teal}{\mytextsc{adverbe}} \hspace{4pt} Ton~: MH\# | L.

Début du mois.
\textcolor{brown}{\zh{月初。}}


\begin{exe}
\sn \textcolor{darkblue}{\textbf{\ipa{tsʰe˧do˧-ɖɯ˧ɲi\#˥ / tsʰe˧do˧-ɖɯ˧ɲi˥}}}
\trans le 1er du mois
\trans \textit{\textcolor{brown}{\zh{初一}}}
\end{exe}
\begin{exe}
\sn \textcolor{darkblue}{\textbf{\ipa{tsʰe˧do˧-ɲi˧ɲi\#˥ / tsʰe˧do˧-ɲi˧ɲi˥}}}
\trans le deuxième jour du mois
\trans \textit{\textcolor{brown}{\zh{初二}}}
\end{exe}
\begin{exe}
\sn \textcolor{darkblue}{\textbf{\ipa{tsʰe˧do˧˥ | -so˩ɲi˩˥}}}
\trans le 3e du mois
\trans \textit{\textcolor{brown}{\zh{初三}}}
\end{exe}
\begin{exe}
\sn \textcolor{darkblue}{\textbf{\ipa{tsʰe˧do˧-ŋwɤ˥ɲi˩}}}
\trans le 5e jour du mois
\trans \textit{\textcolor{brown}{\zh{初五}}}
\end{exe}
\begin{exe}
\sn \textcolor{darkblue}{\textbf{\ipa{tsʰe˧do˧-hõ˥ɲi˩}}}
\trans le 8e jour du mois
\trans \textit{\textcolor{brown}{\zh{初八}}}
\end{exe}
\begin{exe}
\sn \textcolor{darkblue}{\textbf{\ipa{tsʰe˧do˧˥ | -tsʰe˩ɲi˩˥}}}
\trans le 10e jour du mois
\trans \textit{\textcolor{brown}{\zh{初十}}}
\end{exe}
\begin{exe}
\sn \textcolor{darkblue}{\textbf{\ipa{tsʰe˧do˧˥ | -tsʰe˩ɖɯ˩ɲi˩˥}}}
\trans le 11e jour du mois
\trans \textit{\textcolor{brown}{\zh{十一日}}}
\end{exe}


\paragraph{\hspace{-0.5cm} {\Large \textcolor{darkblue}{\textbf{\ipa{tsʰe˧ɖɯ˧}}}}} \hypertarget{ts\string_he\string_Md`M\string_M1}{}
\markboth{\textcolor{darkblue}{\textbf{\ipa{tsʰe˧ɖɯ˧}}}}{}
\textcolor{teal}{\mytextsc{nombre}} \hspace{4pt} Ton~: .

11.
\textcolor{brown}{\zh{11。}}




\paragraph{\hspace{-0.5cm} {\Large \textcolor{darkblue}{\textbf{\ipa{tsʰe˧hṽ˧˥}}}}} \hypertarget{ts\string_he\string_Mhv\string_~\string_M\string_T1}{}
\markboth{\textcolor{darkblue}{\textbf{\ipa{tsʰe˧hṽ˧˥}}}}{}
\textcolor{teal}{\mytextsc{nom}} \hspace{4pt} Ton~: MH\#.

Aspidistra.
\textcolor{brown}{\zh{万年青。}}


\begin{exe}
\sn \textcolor{darkblue}{\textbf{\ipa{tsʰe˧hṽ˧-dzi˧˥}}}
\trans arbre/plant d'aspidistra
\trans \textit{\textcolor{brown}{\zh{万年青树}}}
\end{exe}
\begin{exe}
\sn \textcolor{darkblue}{\textbf{\ipa{tsʰe˧hṽ˧-bæ˥bæ˩}}}
\trans fleurs d'aspidistra
\trans \textit{\textcolor{brown}{\zh{万年青花}}}
\end{exe}
 \mytextsc{clf}~: \textcolor{darkblue}{\textbf{\ipa{dzi˩}}} 

\paragraph{\hspace{-0.5cm} {\Large \textcolor{darkblue}{\textbf{\ipa{tsʰe˧jɤ˧mi˥}}}}} \hypertarget{ts\string_he\string_Mj7\string_Mmi\string_T1}{}
\markboth{\textcolor{darkblue}{\textbf{\ipa{tsʰe˧jɤ˧mi˥}}}}{}
\textcolor{teal}{\mytextsc{nom}} \hspace{4pt} Ton~: H\#.

Marécage.
\textcolor{brown}{\zh{沼泽。}}


\begin{exe}
\sn \textcolor{darkblue}{\textbf{\ipa{tsʰe˧jɤ˧mi˥-qo˩, | ʈʰæ˧tɕi˧ɭɯ˥ | tʰi˧-di˩!}}}
\trans sur les terres marécageuses, il ne pousse que des petites touffes d'herbe!
\trans \textit{\textcolor{brown}{\zh{沼泽里,只长野草!}}}
\end{exe}
 \mytextsc{clf}~: \textcolor{darkblue}{\textbf{\ipa{pʰæ˧˥}}} 

\paragraph{\hspace{-0.5cm} {\Large \textcolor{darkblue}{\textbf{\ipa{tsʰe˧kʰv̩˧}}}}} \hypertarget{ts\string_he\string_Mk\string_hv\string_=\string_M1}{}
\markboth{\textcolor{darkblue}{\textbf{\ipa{tsʰe˧kʰv̩˧}}}}{}
\textcolor{teal}{\mytextsc{verbe}} \hspace{4pt} Ton~: .

Avoir le hoquet.
\textcolor{brown}{\zh{打嗝。}}




\paragraph{\hspace{-0.5cm} {\Large \textcolor{darkblue}{\textbf{\ipa{tsʰe˧ɬi˧mi˧}}}}} \hypertarget{ts\string_he\string_MKi\string_Mmi\string_M1}{}
\markboth{\textcolor{darkblue}{\textbf{\ipa{tsʰe˧ɬi˧mi˧}}}}{}
\textcolor{teal}{\mytextsc{nom}} \hspace{4pt} Ton~: M.

10e mois.
\textcolor{brown}{\zh{十月。}}




\paragraph{\hspace{-0.5cm} {\Large \textcolor{darkblue}{\textbf{\ipa{tsʰe˧ɲi˧}}}}} \hypertarget{ts\string_he\string_MJi\string_M1}{}
\markboth{\textcolor{darkblue}{\textbf{\ipa{tsʰe˧ɲi˧}}}}{}
\textcolor{teal}{\mytextsc{nombre}} \hspace{4pt} Ton~: .

12.
\textcolor{brown}{\zh{12。}}




\paragraph{\hspace{-0.5cm} {\Large \textcolor{darkblue}{\textbf{\ipa{tsʰe˧qʰɑ˩}}}}} \hypertarget{ts\string_he\string_Mq\string_hA\string_B1}{}
\markboth{\textcolor{darkblue}{\textbf{\ipa{tsʰe˧qʰɑ˩}}}}{}
\textcolor{teal}{\mytextsc{adjectif}} \hspace{4pt} Ton~: L\#.

Trop salé.
\textcolor{brown}{\zh{太咸。}}


\textit{Voir~:} \hyperlink{ts\string_he\string_Mso\string_M\string_T1}{\textcolor{darkblue}{\textbf{\ipa{tsʰe˧so˧˥}}}} 

\paragraph{\hspace{-0.5cm} {\Large \textcolor{darkblue}{\textbf{\ipa{tsʰe˧so˧}}}}} \hypertarget{ts\string_he\string_Mso\string_M1}{}
\markboth{\textcolor{darkblue}{\textbf{\ipa{tsʰe˧so˧}}}}{}
\textcolor{teal}{\mytextsc{nombre}} \hspace{4pt} Ton~: .

13.
\textcolor{brown}{\zh{13。}}




\paragraph{\hspace{-0.5cm} {\Large \textcolor{darkblue}{\textbf{\ipa{tsʰe˧so˧˥}}}}} \hypertarget{ts\string_he\string_Mso\string_M\string_T1}{}
\markboth{\textcolor{darkblue}{\textbf{\ipa{tsʰe˧so˧˥}}}}{}
\textcolor{teal}{\mytextsc{adjectif}} \hspace{4pt} Ton~: MH\#.

Salé (agréablement: salé à point).
\textcolor{brown}{\zh{咸。}}


\textit{Voir~:} \hyperlink{ts\string_he\string_Mq\string_hA\string_B1}{\textcolor{darkblue}{\textbf{\ipa{tsʰe˧qʰɑ˩}}}} 

\paragraph{\hspace{-0.5cm} {\Large \textcolor{darkblue}{\textbf{\ipa{tsʰe˧ʂɯ˧}}}}} \hypertarget{ts\string_he\string_Ms`M\string_M1}{}
\markboth{\textcolor{darkblue}{\textbf{\ipa{tsʰe˧ʂɯ˧}}}}{}
\textcolor{teal}{\mytextsc{nombre}} \hspace{4pt} Ton~: .

17.
\textcolor{brown}{\zh{17。}}




\paragraph{\hspace{-0.5cm} {\Large \textcolor{darkblue}{\textbf{\ipa{tsʰe˧tʰv̩\#˥}}}}} \hypertarget{ts\string_he\string_Mt\string_hv\string_=\#\string_T1}{}
\markboth{\textcolor{darkblue}{\textbf{\ipa{tsʰe˧tʰv̩\#˥}}}}{}
\textcolor{teal}{\mytextsc{nom}} \hspace{4pt} Ton~: \#H.
\ding{202} 
Variole, petite vérole.
\textcolor{brown}{\zh{天花。}}


\begin{exe}
\sn \textcolor{darkblue}{\textbf{\ipa{tsʰe˧tʰv̩˧ | bæ˩bæ˩ bæ˥-ze˩}}}
\trans La variole s'est déclarée.
\trans \textit{\textcolor{brown}{\zh{天花/麻疹犯了。}}}
\end{exe}
\ding{203} 
Rougeole.
\textcolor{brown}{\zh{麻疹,疹子。}}


\begin{exe}
\sn \textcolor{darkblue}{\textbf{\ipa{gv̩˧-kʰv̩˩ mɤ˩-gv̩˩, | tsʰe˧ mɤ˧-tʰv̩˧, | hĩ˧ ʈʂɤ˧-mɤ˧-kv̩˩!}}}
\trans Si, avant l'âge de neuf ans, on ne contracte pas la rougeole, on ne peut pas devenir adulte! / Attraper la rougeole, ça fait partie du processus de croissance vers l'âge adulte!
\trans \textit{\textcolor{brown}{\zh{九岁前不得麻疹,不能成人! / 得麻疹,就是小孩生长过程中必须要的一件事情!}}}
\end{exe}
 \mytextsc{clf}~: \textcolor{darkblue}{\textbf{\ipa{ʂɯ˩}}} fois

\paragraph{\hspace{-0.5cm} {\Large \textcolor{darkblue}{\textbf{\ipa{tsʰe˧ʈʂæ˧}}}}} \hypertarget{ts\string_he\string_Mt`s`\{\string_M1}{}
\markboth{\textcolor{darkblue}{\textbf{\ipa{tsʰe˧ʈʂæ˧}}}}{}
\textcolor{teal}{\mytextsc{nom}} \hspace{4pt} Ton~: M.

Chef de village, petit officiel.
\textcolor{brown}{\zh{村长。}}

 Emprunt~: chinois \textcolor{brown}{\zh{村长}}



\paragraph{\hspace{-0.5cm} {\Large \textcolor{darkblue}{\textbf{\ipa{tsʰe˩\textsubscript{b}}}}}} \hypertarget{ts\string_he\string_Bb1}{}
\markboth{\textcolor{darkblue}{\textbf{\ipa{tsʰe˩\textsubscript{b}}}}}{}
\textcolor{teal}{\mytextsc{classificateur}} \hspace{4pt} Ton~: L\textsubscript{b}.

Pouce (cette unité de mesure n'était pas en usage chez les Na avant son introduction par emprunt au chinois).
\textcolor{brown}{\zh{量词:寸(汉语借词)。}}

 Emprunt~: chinois \textcolor{brown}{\zh{寸}}



\paragraph{\hspace{-0.5cm} {\Large \textcolor{darkblue}{\textbf{\ipa{tsʰe˩\textsubscript{b}}}}}} \hypertarget{ts\string_he\string_Bb1}{}
\markboth{\textcolor{darkblue}{\textbf{\ipa{tsʰe˩\textsubscript{b}}}}}{}
\textcolor{teal}{\mytextsc{classificateur}} \hspace{4pt} Ton~: L\textsubscript{b}.

Classificateur des nœuds dans une tresse.
\textcolor{brown}{\zh{量词:数辫子的节(一节)。}}




\paragraph{\hspace{-0.5cm} {\Large \textcolor{darkblue}{\textbf{\ipa{tsʰe˩gv̩˩}}}}} \hypertarget{ts\string_he\string_Bgv\string_=\string_B1}{}
\markboth{\textcolor{darkblue}{\textbf{\ipa{tsʰe˩gv̩˩}}}}{}
\textcolor{teal}{\mytextsc{nombre}} \hspace{4pt} Ton~: L.

19.
\textcolor{brown}{\zh{19。}}




\paragraph{\hspace{-0.5cm} {\Large \textcolor{darkblue}{\textbf{\ipa{tsʰe˩hõ˩}}}}} \hypertarget{ts\string_he\string_Bho\string_~\string_B1}{}
\markboth{\textcolor{darkblue}{\textbf{\ipa{tsʰe˩hõ˩}}}}{}
\textcolor{teal}{\mytextsc{nombre}} \hspace{4pt} Ton~: L.

18.
\textcolor{brown}{\zh{18。}}




\paragraph{\hspace{-0.5cm} {\Large \textcolor{darkblue}{\textbf{\ipa{tsʰe˩ŋwɤ˩}}}}} \hypertarget{ts\string_he\string_BNw7\string_B1}{}
\markboth{\textcolor{darkblue}{\textbf{\ipa{tsʰe˩ŋwɤ˩}}}}{}
\textcolor{teal}{\mytextsc{nombre}} \hspace{4pt} Ton~: L.

15.
\textcolor{brown}{\zh{15。}}




\paragraph{\hspace{-0.5cm} {\Large \textcolor{darkblue}{\textbf{\ipa{tsʰe˩ŋwɤ˩ɲi˩˥}}}}} \hypertarget{ts\string_he\string_BNw7\string_BJi\string_B\string_T1}{}
\markboth{\textcolor{darkblue}{\textbf{\ipa{tsʰe˩ŋwɤ˩ɲi˩˥}}}}{}
\textcolor{teal}{\mytextsc{nom}} \hspace{4pt} Ton~: .

Le 15e jour du mois.
\textcolor{brown}{\zh{十五号。}}




\paragraph{\hspace{-0.5cm} {\Large \textcolor{darkblue}{\textbf{\ipa{tsʰe˩qʰv̩˩}}}}} \hypertarget{ts\string_he\string_Bq\string_hv\string_=\string_B1}{}
\markboth{\textcolor{darkblue}{\textbf{\ipa{tsʰe˩qʰv̩˩}}}}{}
\textcolor{teal}{\mytextsc{nombre}} \hspace{4pt} Ton~: L.

16.
\textcolor{brown}{\zh{16。}}




\paragraph{\hspace{-0.5cm} {\Large \textcolor{darkblue}{\textbf{\ipa{tsʰe˩qʰv̩˩ɲi˩˥}}}}} \hypertarget{ts\string_he\string_Bq\string_hv\string_=\string_BJi\string_B\string_T1}{}
\markboth{\textcolor{darkblue}{\textbf{\ipa{tsʰe˩qʰv̩˩ɲi˩˥}}}}{}
\textcolor{teal}{\mytextsc{nom}} \hspace{4pt} Ton~: .

Le 16e jour du mois.
\textcolor{brown}{\zh{十六号。}}


\begin{exe}
\sn \textcolor{darkblue}{\textbf{\ipa{[F5] tsʰe˧do˧˥tsʰe˩kʰʶv̩˩ɲi˩˧}}}
\trans même sens
\trans \textit{\textcolor{brown}{\zh{同上}}}
\end{exe}


\paragraph{\hspace{-0.5cm} {\Large \textcolor{darkblue}{\textbf{\ipa{tsʰe˩ʐv̩˩}}}}} \hypertarget{ts\string_he\string_Bz`v\string_=\string_B1}{}
\markboth{\textcolor{darkblue}{\textbf{\ipa{tsʰe˩ʐv̩˩}}}}{}
\textcolor{teal}{\mytextsc{nombre}} \hspace{4pt} Ton~: L.

14.
\textcolor{brown}{\zh{14。}}




\paragraph{\hspace{-0.5cm} {\Large \textcolor{darkblue}{\textbf{\ipa{tsʰɤ˩\textsubscript{a}}}}}} \hypertarget{ts\string_h7\string_Ba1}{}
\markboth{\textcolor{darkblue}{\textbf{\ipa{tsʰɤ˩\textsubscript{a}}}}}{}
\textcolor{teal}{\mytextsc{verbe}} \hspace{4pt} Ton~: L\textsubscript{a}.

Tresser (les cheveux, fils).
\textcolor{brown}{\zh{编(头发,线)。}}


\begin{exe}
\sn \textcolor{darkblue}{\textbf{\ipa{ʁo˧qʰwɤ˩ tsʰɤ˩}}}
\trans tresser les cheveux, littéralement ``tresser la tête"
\trans \textit{\textcolor{brown}{\zh{编辫子}}}
\end{exe}
\begin{exe}
\sn \textcolor{darkblue}{\textbf{\ipa{hæ̃˧pɤ˧ le˧-tsʰɤ˩}}}
\trans faire une tresse
\trans \textit{\textcolor{brown}{\zh{梳一条辫子}}}
\end{exe}
\begin{exe}
\sn \textcolor{darkblue}{\textbf{\ipa{ɖɯ˧-tsʰɤ˧\textasciitilde{}tsʰɤ˥-ɻ̍˩}}}
\trans \mytextsc{délimitatif} \string_ \mytextsc{red} \mytextsc{inchoatif}
\trans \textit{\textcolor{brown}{\zh{\mytextsc{delimitative} \string_ \mytextsc{red} \mytextsc{inceptive}}}}
\end{exe}


\paragraph{\hspace{-0.5cm} {\Large \textcolor{darkblue}{\textbf{\ipa{tsʰɤ˧˥}}} \textsubscript{1}}} \hypertarget{ts\string_h7\string_M\string_T1}{}
\markboth{\textcolor{darkblue}{\textbf{\ipa{tsʰɤ˧˥}}} \textsubscript{1}}{}
\textcolor{teal}{\mytextsc{verbe}} \hspace{4pt} Ton~: MH.

Traire (vache, brebis).
\textcolor{brown}{\zh{挤奶。}}


\begin{exe}
\sn \textcolor{darkblue}{\textbf{\ipa{tso˧\textasciitilde{}tso˧ tsʰɤ˩ + ze˩}}}
\trans traire des choses
\trans \textit{\textcolor{brown}{\zh{挤出东西}}}
\end{exe}
\begin{exe}
\sn \textcolor{darkblue}{\textbf{\ipa{ʝi˧-bv̩˧ | ɳæ˧ tsʰɤ˩}}}
\trans traire (le lait de) la vache
\trans \textit{\textcolor{brown}{\zh{挤牛奶}}}
\end{exe}


\paragraph{\hspace{-0.5cm} {\Large \textcolor{darkblue}{\textbf{\ipa{tsʰɤ˧˥}}} \textsubscript{2}}} \hypertarget{ts\string_h7\string_M\string_T2}{}
\markboth{\textcolor{darkblue}{\textbf{\ipa{tsʰɤ˧˥}}} \textsubscript{2}}{}
\textcolor{teal}{\mytextsc{verbe}} \hspace{4pt} Ton~: MH.

Frotter, faire une friction (ex.: un tissu grossier frotte sur la peau, et l'irrite).
\textcolor{brown}{\zh{摩擦。}}




\paragraph{\hspace{-0.5cm} {\Large \textcolor{darkblue}{\textbf{\ipa{tsʰɤ˧˥}}} \textsubscript{3}}} \hypertarget{ts\string_h7\string_M\string_T3}{}
\markboth{\textcolor{darkblue}{\textbf{\ipa{tsʰɤ˧˥}}} \textsubscript{3}}{}
\textcolor{teal}{\mytextsc{verbe}} \hspace{4pt} Ton~: MH.

Attaquer, piller, s'en prendre à (des brigands attaquent un convoi).
\textcolor{brown}{\zh{抢。}}




\paragraph{\hspace{-0.5cm} {\Large \textcolor{darkblue}{\textbf{\ipa{tsʰɤ˧˥}}} \textsubscript{4}}} \hypertarget{ts\string_h7\string_M\string_T4}{}
\markboth{\textcolor{darkblue}{\textbf{\ipa{tsʰɤ˧˥}}} \textsubscript{4}}{}
\textcolor{teal}{\mytextsc{verbe}} \hspace{4pt} Ton~: MH.

Rendre.
\textcolor{brown}{\zh{还(东西)。}}




\paragraph{\hspace{-0.5cm} {\Large \textcolor{darkblue}{\textbf{\ipa{tsʰɤ˧˥\textsubscript{a}}}}}} \hypertarget{ts\string_h7\string_M\string_Ta1}{}
\markboth{\textcolor{darkblue}{\textbf{\ipa{tsʰɤ˧˥\textsubscript{a}}}}}{}
\textcolor{teal}{\mytextsc{classificateur}} \hspace{4pt} Ton~: MH\textsubscript{a}.

Classificateur des objets bosselés: feuilles, crêtes de coq, têtes d'ail (une tête d'ail se divise en gousses un peu comme on effeuille une branche ou une fleur).
\textcolor{brown}{\zh{量词:凸凹的物品,如:鸡冠(一顶)、叶子(一片)、蒜(一头)。}}


\begin{exe}
\sn \textcolor{darkblue}{\textbf{\ipa{bæ˩bæ˩˥ | ɖɯ˧-tsʰɤ˧˥}}}
\trans une fleur
\trans \textit{\textcolor{brown}{\zh{一朵花}}}
\end{exe}


\paragraph{\hspace{-0.5cm} {\Large \textcolor{darkblue}{\textbf{\ipa{tsʰi\#˥}}}}} \hypertarget{ts\string_hi\#\string_T1}{}
\markboth{\textcolor{darkblue}{\textbf{\ipa{tsʰi\#˥}}}}{}
\textcolor{teal}{\mytextsc{nom}} \hspace{4pt} Ton~: \#H.

Saison sèche (hiver et printemps; du 9e mois au 2e mois du calendrier lunaire compris).
\textcolor{brown}{\zh{旱季(冬天至春天:农历九月到来年二月)。}}




\paragraph{\hspace{-0.5cm} {\Large \textcolor{darkblue}{\textbf{\ipa{tsʰi˥\textsubscript{a}}}}}} \hypertarget{ts\string_hi\string_Ta1}{}
\markboth{\textcolor{darkblue}{\textbf{\ipa{tsʰi˥\textsubscript{a}}}}}{}
\textcolor{teal}{\mytextsc{classificateur}} \hspace{4pt} Ton~: H\textsubscript{a}.

Classificateur des peaux d'animaux, et des pièces de tissu.
\textcolor{brown}{\zh{量词:动物皮(一张),布料(一块)。}}


\begin{exe}
\sn \textcolor{darkblue}{\textbf{\ipa{ɖɯ˧-tsʰi˥}}}
\trans une peau
\trans \textit{\textcolor{brown}{\zh{一张动物皮}}}
\end{exe}
\begin{exe}
\sn \textcolor{darkblue}{\textbf{\ipa{ɖɯ˧-tsʰi˧ ɲi˥}}}
\trans c'est une peau
\trans \textit{\textcolor{brown}{\zh{这是一张(动物皮)}}}
\end{exe}


\paragraph{\hspace{-0.5cm} {\Large \textcolor{darkblue}{\textbf{\ipa{tsʰi˧}}} \textsubscript{1}}} \hypertarget{ts\string_hi\string_M1}{}
\markboth{\textcolor{darkblue}{\textbf{\ipa{tsʰi˧}}} \textsubscript{1}}{}
\textcolor{teal}{\mytextsc{adjectif}} \hspace{4pt} Ton~: M.

Chaud.
\textcolor{brown}{\zh{热,烫。}}


\begin{exe}
\sn \textcolor{darkblue}{\textbf{\ipa{tsʰi˧-zo˧ mɤ˧-tʰɑ˧˥!}}}
\trans il fait une chaleur insupportable!
\trans \textit{\textcolor{brown}{\zh{热得受不了!}}}
\end{exe}
\textit{Voir~:} \hyperlink{ts\string_hi\string_M2}{\textcolor{darkblue}{\textbf{\ipa{tsʰi˧}}} \textsubscript{2}} 

\paragraph{\hspace{-0.5cm} {\Large \textcolor{darkblue}{\textbf{\ipa{tsʰi˧}}} \textsubscript{2}}} \hypertarget{ts\string_hi\string_M2}{}
\markboth{\textcolor{darkblue}{\textbf{\ipa{tsʰi˧}}} \textsubscript{2}}{}
\textcolor{teal}{\mytextsc{adjectif}} \hspace{4pt} Ton~: M.

Brillant, lumineux, ardent.
\textcolor{brown}{\zh{明亮。}}


\begin{exe}
\sn \textcolor{darkblue}{\textbf{\ipa{ɲi˧mi˧ tsʰi˧}}}
\trans le soleil est très fort/le soleil tape dur
\trans \textit{\textcolor{brown}{\zh{太阳很晒}}}
\end{exe}
\begin{exe}
\sn \textcolor{darkblue}{\textbf{\ipa{ɬi˧mi˧ tsʰi˧}}}
\trans la lune brille, la lune luit, on y voit clair à la lumière de la lune
\trans \textit{\textcolor{brown}{\zh{月亮很亮、月光很明亮}}}
\end{exe}
\textit{Voir~:} \hyperlink{ts\string_hi\string_M1}{\textcolor{darkblue}{\textbf{\ipa{tsʰi˧}}} \textsubscript{1}} 

\paragraph{\hspace{-0.5cm} {\Large \textcolor{darkblue}{\textbf{\ipa{tsʰi˧\textsubscript{b}}}}}} \hypertarget{ts\string_hi\string_Mb1}{}
\markboth{\textcolor{darkblue}{\textbf{\ipa{tsʰi˧\textsubscript{b}}}}}{}
\textcolor{teal}{\mytextsc{verbe}} \hspace{4pt} Ton~: M\textsubscript{b}.

Porter (un chapeau).
\textcolor{brown}{\zh{戴帽子。}}


\begin{exe}
\sn \textcolor{darkblue}{\textbf{\ipa{tv̩˧tv̩˥ tsʰi˩}}}
\trans mettre un chapeau
\trans \textit{\textcolor{brown}{\zh{戴上帽子}}}
\end{exe}


\paragraph{\hspace{-0.5cm} {\Large \textcolor{darkblue}{\textbf{\ipa{tsʰi˧bv̩˩}}}}} \hypertarget{ts\string_hi\string_Mbv\string_=\string_B1}{}
\markboth{\textcolor{darkblue}{\textbf{\ipa{tsʰi˧bv̩˩}}}}{}
\textcolor{teal}{\mytextsc{adjectif}} \hspace{4pt} Ton~: L\#.

Étouffant.
\textcolor{brown}{\zh{闷热。}}




\paragraph{\hspace{-0.5cm} {\Large \textcolor{darkblue}{\textbf{\ipa{tsʰi˧ʝi\#˥}}}}} \hypertarget{ts\string_hi\string_Mj££i\#\string_T1}{}
\markboth{\textcolor{darkblue}{\textbf{\ipa{tsʰi˧ʝi\#˥}}}}{}
\textcolor{teal}{\mytextsc{adverbe}} \hspace{4pt} Ton~: \#H.

Cette année.
\textcolor{brown}{\zh{今年。}}


\begin{exe}
\sn \textcolor{darkblue}{\textbf{\ipa{tsʰi˧ʝi˧-se˥, | …}}}
\trans Jusqu'à cette année, ...
\trans \textit{\textcolor{brown}{\zh{到了今年,……}}}
\end{exe}


\paragraph{\hspace{-0.5cm} {\Large \textcolor{darkblue}{\textbf{\ipa{tsʰi˧ɲi\#˥}}}}} \hypertarget{ts\string_hi\string_MJi\#\string_T1}{}
\markboth{\textcolor{darkblue}{\textbf{\ipa{tsʰi˧ɲi\#˥}}}}{}
\textcolor{teal}{\mytextsc{adverbe}} \hspace{4pt} Ton~: \#H.

Aujourd'hui.
\textcolor{brown}{\zh{今天。}}


\begin{exe}
\sn \textcolor{darkblue}{\textbf{\ipa{tsʰi˧ɲi˧-ʁo˧dɑ˧}}}
\trans avant ce jour, avant aujourd'hui; précédemment
\trans \textit{\textcolor{brown}{\zh{今天之前}}}
\end{exe}


\paragraph{\hspace{-0.5cm} {\Large \textcolor{darkblue}{\textbf{\ipa{tsʰi˧qʰæ˧˥}}}}} \hypertarget{ts\string_hi\string_Mq\string_h\{\string_M\string_T1}{}
\markboth{\textcolor{darkblue}{\textbf{\ipa{tsʰi˧qʰæ˧˥}}}}{}
\textcolor{teal}{\mytextsc{adverbe}} \hspace{4pt} Ton~: MH.

En ce moment, actuellement, maintenant.
\textcolor{brown}{\zh{现在。}}




\paragraph{\hspace{-0.5cm} {\Large \textcolor{darkblue}{\textbf{\ipa{tsʰi˧si˩-dʑɤ˩pv̩˩}}}}} \hypertarget{ts\string_hi\string_Msi\string_B-dz£7\string_Bpv\string_=\string_B1}{}
\markboth{\textcolor{darkblue}{\textbf{\ipa{tsʰi˧si˩-dʑɤ˩pv̩˩}}}}{}
\textcolor{teal}{\mytextsc{nom}} \hspace{4pt} Ton~: L\#-.

Le monde des esprits, le monde des morts.
\textcolor{brown}{\zh{神灵的世界、死人的世界。}}




\paragraph{\hspace{-0.5cm} {\Large \textcolor{darkblue}{\textbf{\ipa{tsʰi˧ti\#˥}}}}} \hypertarget{ts\string_hi\string_Mti\#\string_T1}{}
\markboth{\textcolor{darkblue}{\textbf{\ipa{tsʰi˧ti\#˥}}}}{}
\textcolor{teal}{\mytextsc{nom}} \hspace{4pt} Ton~: \#H.

Prénom masculin.
\textcolor{brown}{\zh{男性名字。}}




\paragraph{\hspace{-0.5cm} {\Large \textcolor{darkblue}{\textbf{\ipa{tsʰi˧zi\#˥}}}}} \hypertarget{ts\string_hi\string_Mzi\#\string_T1}{}
\markboth{\textcolor{darkblue}{\textbf{\ipa{tsʰi˧zi\#˥}}}}{}
\textcolor{teal}{\mytextsc{nom}} \hspace{4pt} Ton~: \#H.

Orge d'altitude, \textit{Hordeum vulgare var. nudum Hook. f.}.
\textcolor{brown}{\zh{青稞。}}


\begin{exe}
\sn \textcolor{darkblue}{\textbf{\ipa{tsʰi˧zi˧ | nɑ˩-hĩ˩˥}}}
\trans orge noir
\trans \textit{\textcolor{brown}{\zh{黑青稞}}}
\end{exe}
\begin{exe}
\sn \textcolor{darkblue}{\textbf{\ipa{tsʰi˧zi˧ | pʰv̩˩-hĩ˩˥}}}
\trans orge blanc
\trans \textit{\textcolor{brown}{\zh{白青稞}}}
\end{exe}
 \mytextsc{clf}~: \textcolor{darkblue}{\textbf{\ipa{kɤ˧˥}}} 

\paragraph{\hspace{-0.5cm} {\Large \textcolor{darkblue}{\textbf{\ipa{tsʰi˧zi˧-ɻ̃\#˥}}}}} \hypertarget{ts\string_hi\string_Mzi\string_M-r£`\string_~\#\string_T1}{}
\markboth{\textcolor{darkblue}{\textbf{\ipa{tsʰi˧zi˧-ɻ̃\#˥}}}}{}
\textcolor{teal}{\mytextsc{nom}} \hspace{4pt} Ton~: \#H.

Paille d'orge.
\textcolor{brown}{\zh{青稞杆。}}


 \mytextsc{clf}~: \textcolor{darkblue}{\textbf{\ipa{kɤ˧˥}}} 

\paragraph{\hspace{-0.5cm} {\Large \textcolor{darkblue}{\textbf{\ipa{tsʰi˩\textsubscript{a}}}}}} \hypertarget{ts\string_hi\string_Ba1}{}
\markboth{\textcolor{darkblue}{\textbf{\ipa{tsʰi˩\textsubscript{a}}}}}{}
\textcolor{teal}{\mytextsc{adjectif}} \hspace{4pt} Ton~: L\textsubscript{a}.

Fin (objet).
\textcolor{brown}{\zh{细(树、体型细小)。}}


\begin{exe}
\sn \textcolor{darkblue}{\textbf{\ipa{tsʰi˩-hĩ˩˥}}}
\trans \mytextsc{nmlz}
\trans \textit{\textcolor{brown}{\zh{细的}}}
\end{exe}
\begin{exe}
\sn \textcolor{darkblue}{\textbf{\ipa{qʰɑ˧-tsʰi˧-gv̩˧}}}
\trans très fin
\trans \textit{\textcolor{brown}{\zh{非常细}}}
\end{exe}
\begin{exe}
\sn \textcolor{darkblue}{\textbf{\ipa{dʑɤ˧˥ | tsʰi˩-njæ˩˥ | -gv̩˩!}}}
\trans C'est vraiment fin!
\trans \textit{\textcolor{brown}{\zh{真细!}}}
\end{exe}


\paragraph{\hspace{-0.5cm} {\Large \textcolor{darkblue}{\textbf{\ipa{tsʰi˩mv̩˩tʰv̩˩}}}}} \hypertarget{ts\string_hi\string_Bmv\string_=\string_Bt\string_hv\string_=\string_B1}{}
\markboth{\textcolor{darkblue}{\textbf{\ipa{tsʰi˩mv̩˩tʰv̩˩}}}}{}
\textcolor{teal}{\mytextsc{nom}} \hspace{4pt} Ton~: L.

Démon qui danse.
\textcolor{brown}{\zh{跳着的鬼。}}




\paragraph{\hspace{-0.5cm} {\Large \textcolor{darkblue}{\textbf{\ipa{tsʰi˩tv̩˩}}}}} \hypertarget{ts\string_hi\string_Btv\string_=\string_B1}{}
\markboth{\textcolor{darkblue}{\textbf{\ipa{tsʰi˩tv̩˩}}}}{}
\textcolor{teal}{\mytextsc{nom}} \hspace{4pt} Ton~: L.

Moëlle.
\textcolor{brown}{\zh{骨髓。}}


 \mytextsc{clf}~: \textcolor{darkblue}{\textbf{\ipa{kʰwɤ˥}}} 

\paragraph{\hspace{-0.5cm} {\Large \textcolor{darkblue}{\textbf{\ipa{tsʰi˩tsʰi˩}}}}} \hypertarget{ts\string_hi\string_Bts\string_hi\string_B1}{}
\markboth{\textcolor{darkblue}{\textbf{\ipa{tsʰi˩tsʰi˩}}}}{}
\textcolor{teal}{\mytextsc{nom}} \hspace{4pt} Ton~: L.

Pois, petits pois.
\textcolor{brown}{\zh{豌豆。}}


 \mytextsc{clf}~: \textcolor{darkblue}{\textbf{\ipa{kɤ˧˥}}} 

\paragraph{\hspace{-0.5cm} {\Large \textcolor{darkblue}{\textbf{\ipa{tsʰi˧˥}}} \textsubscript{1}}} \hypertarget{ts\string_hi\string_M\string_T1}{}
\markboth{\textcolor{darkblue}{\textbf{\ipa{tsʰi˧˥}}} \textsubscript{1}}{}
\textcolor{teal}{\mytextsc{verbe}} \hspace{4pt} Ton~: MH.

Construire.
\textcolor{brown}{\zh{盖,建 (房子)。}}


\begin{exe}
\sn \textcolor{darkblue}{\textbf{\ipa{ʑi˧qʰwɤ˧ tsʰi˧˥}}}
\trans construire un bâtiment
\trans \textit{\textcolor{brown}{\zh{建 房子}}}
\end{exe}


\paragraph{\hspace{-0.5cm} {\Large \textcolor{darkblue}{\textbf{\ipa{tsʰi˧˥}}} \textsubscript{2}}} \hypertarget{ts\string_hi\string_M\string_T2}{}
\markboth{\textcolor{darkblue}{\textbf{\ipa{tsʰi˧˥}}} \textsubscript{2}}{}
\textcolor{teal}{\mytextsc{verbe}} \hspace{4pt} Ton~: MH.

Percer un trou, faire un trou (ex.: dans un tissu, un mur...).
\textcolor{brown}{\zh{穿一个洞。}}




\paragraph{\hspace{-0.5cm} {\Large \textcolor{darkblue}{\textbf{\ipa{tsʰi˧˥}}} \textsubscript{3}}} \hypertarget{ts\string_hi\string_M\string_T3}{}
\markboth{\textcolor{darkblue}{\textbf{\ipa{tsʰi˧˥}}} \textsubscript{3}}{}
\textcolor{teal}{\mytextsc{verbe}} \hspace{4pt} Ton~: MH.

Allumer (un feu).
\textcolor{brown}{\zh{点(火)。}}


\begin{exe}
\sn \textcolor{darkblue}{\textbf{\ipa{mv̩˧ tsʰi˧˥}}}
\trans allumer un feu
\trans \textit{\textcolor{brown}{\zh{点火}}}
\end{exe}
\begin{exe}
\sn \textcolor{darkblue}{\textbf{\ipa{njɤ˧-ɳɯ˧ | mv̩˧tsʰi˧-bi˥}}}
\trans je vais allumer le feu
\trans \textit{\textcolor{brown}{\zh{我要点个火}}}
\end{exe}
\begin{exe}
\sn \textcolor{darkblue}{\textbf{\ipa{mv̩˩tsʰo˩ tsʰi˧}}}
\trans mettre le feu à un bout de bois plein de sève (pour faire partir le feu)
\trans \textit{\textcolor{brown}{\zh{用含很多树脂的木头来引火}}}
\end{exe}


\paragraph{\hspace{-0.5cm} {\Large \textcolor{darkblue}{\textbf{\ipa{tsʰi˧˥}}} \textsubscript{4}}} \hypertarget{ts\string_hi\string_M\string_T4}{}
\markboth{\textcolor{darkblue}{\textbf{\ipa{tsʰi˧˥}}} \textsubscript{4}}{}
\textcolor{teal}{\mytextsc{verbe}} \hspace{4pt} Ton~: MH.

S'accroupir.
\textcolor{brown}{\zh{蹲。}}


\begin{exe}
\sn \textcolor{darkblue}{\textbf{\ipa{le˧-tsʰi˩\textasciitilde{}tsʰi˩ | tʰi˧-dzi˩}}}
\trans être accroupi (être assis avec les genoux regroupés sur la poitrine)
\trans \textit{\textcolor{brown}{\zh{盘腿坐}}}
\end{exe}
\begin{exe}
\sn \textcolor{darkblue}{\textbf{\ipa{gɤ˩-tsʰi˧\textasciitilde{}tsʰi˩ tʰi˧-dzi˩}}}
\trans même sens
\trans \textit{\textcolor{brown}{\zh{盘腿坐}}}
\end{exe}


\paragraph{\hspace{-0.5cm} {\Large \textcolor{darkblue}{\textbf{\ipa{tsʰi˧˥}}} \textsubscript{5}}} \hypertarget{ts\string_hi\string_M\string_T5}{}
\markboth{\textcolor{darkblue}{\textbf{\ipa{tsʰi˧˥}}} \textsubscript{5}}{}
\textcolor{teal}{\mytextsc{adjectif}} \hspace{4pt} Ton~: MH.
\textit{Archaïque} [\textcolor{brown}{\zh{古语}}] 
Malade, souffrant.
\textcolor{brown}{\zh{病。}}


\begin{exe}
\sn \textcolor{darkblue}{\textbf{\ipa{mɤ˧-go˩ mɤ˩-tsʰi˩-ɻ̍˩ |}}}
\trans être bien portant, ne pas être malade
\trans \textit{\textcolor{brown}{\zh{健康:不病、不痛}}}
\end{exe}


\paragraph{\hspace{-0.5cm} {\Large \textcolor{darkblue}{\textbf{\ipa{tsʰo˥}}}}} \hypertarget{ts\string_ho\string_T1}{}
\markboth{\textcolor{darkblue}{\textbf{\ipa{tsʰo˥}}}}{}
\textcolor{teal}{\mytextsc{adjectif}} \hspace{4pt} Ton~: H.

Complet, au grand complet.
\textcolor{brown}{\zh{齐全。}}


\begin{exe}
\sn \textcolor{darkblue}{\textbf{\ipa{ə˧tso˧-mɤ˧-ɲi˩, | tʰi˧-tsʰo˥-ze˩!}}}
\trans Tout y est! Tout est prêt! (Au sujet de préparatifs pour une fête, un repas...)
\trans \textit{\textcolor{brown}{\zh{什么都准备得很齐全!}}}
\end{exe}
\begin{exe}
\sn \textcolor{darkblue}{\textbf{\ipa{mɤ˧-tsʰo˧-sɯ˥! | wɤ˩˥ | ɲi˧-bæ˧ hwæ˧-zo˧-ho˩!}}}
\trans On n'y est pas encore tout à fait / ce n'est pas encore tout à fait prêt! Il reste deux ou trois trucs à acheter! (Contexte: on achève la décoration d'un appartement à la ville; aux compliments des visiteurs, l'heureux propriétaire répond: 'Ce n'est pas encore terminé!')
\trans \textit{\textcolor{brown}{\zh{还不算齐全! / 还没有装饰齐全!(情景:朋友们表扬新装修的丽江房子,主人谦虚回答:‘还不算齐全!’)}}}
\end{exe}
\begin{exe}
\sn \textcolor{darkblue}{\textbf{\ipa{tʰi˧-tsʰo˥-kʰɯ˩}}}
\trans \mytextsc{dur} \string_ \mytextsc{caus}: porter à son point d'achèvement, porter au grand complet
\trans \textit{\textcolor{brown}{\zh{\mytextsc{dur} \string_ \mytextsc{caus:完成、弄齐全}}}}
\end{exe}


\paragraph{\hspace{-0.5cm} {\Large \textcolor{darkblue}{\textbf{\ipa{tsʰo˧\textsubscript{b}}}}}} \hypertarget{ts\string_ho\string_Mb1}{}
\markboth{\textcolor{darkblue}{\textbf{\ipa{tsʰo˧\textsubscript{b}}}}}{}
\textcolor{teal}{\mytextsc{verbe}} \hspace{4pt} Ton~: M\textsubscript{b}.

Sauter.
\textcolor{brown}{\zh{跳。}}


\begin{exe}
\sn \textcolor{darkblue}{\textbf{\ipa{bæ˧ tsʰo˧}}}
\trans sauter à la corde
\trans \textit{\textcolor{brown}{\zh{跳绳}}}
\end{exe}
\begin{exe}
\sn \textcolor{darkblue}{\textbf{\ipa{tsʰo˧\textasciitilde{}tsʰo˧}}}
\trans forme rédupliquée: trépigner, sautiller ici et là
\trans \textit{\textcolor{brown}{\zh{\mytextsc{重叠}}}}
\end{exe}


\paragraph{\hspace{-0.5cm} {\Large \textcolor{darkblue}{\textbf{\ipa{tsʰo˧ɖɯ˩}}}}} \hypertarget{ts\string_ho\string_Md`M\string_B1}{}
\markboth{\textcolor{darkblue}{\textbf{\ipa{tsʰo˧ɖɯ˩}}}}{}
\textcolor{teal}{\mytextsc{nom}} \hspace{4pt} Ton~: L\#.

Danse en groupe: parfois dix personnes, parfois jusqu'à cent (un village entier).
\textcolor{brown}{\zh{集体舞。}}


\begin{exe}
\sn \textcolor{darkblue}{\textbf{\ipa{tsʰo˧ɖɯ˩ tsʰo˩}}}
\trans faire une grande danse collective
\trans \textit{\textcolor{brown}{\zh{跳一个集体舞}}}
\end{exe}


\paragraph{\hspace{-0.5cm} {\Large \textcolor{darkblue}{\textbf{\ipa{tsʰo˧ɖwæ\#˥}}}}} \hypertarget{ts\string_ho\string_Md`w\{\#\string_T1}{}
\markboth{\textcolor{darkblue}{\textbf{\ipa{tsʰo˧ɖwæ\#˥}}}}{}
\textcolor{teal}{\mytextsc{nom}} \hspace{4pt} Ton~: \#H.

Marche en pierre.
\textcolor{brown}{\zh{石头台阶。}}


 \mytextsc{clf}~: \textcolor{darkblue}{\textbf{\ipa{ɖwæ˥}}} 

\paragraph{\hspace{-0.5cm} {\Large \textcolor{darkblue}{\textbf{\ipa{tsʰo˧ko˧}}}}} \hypertarget{ts\string_ho\string_Mko\string_M1}{}
\markboth{\textcolor{darkblue}{\textbf{\ipa{tsʰo˧ko˧}}}}{}
\textcolor{teal}{\mytextsc{nom}} \hspace{4pt} Ton~: M.

Cardamome, \textit{Elletaria cardamomum}.
\textcolor{brown}{\zh{小豆蔻。}}Dialecte chinois local~: \zh{草果。}

 Emprunt~: \textcolor{brown}{\zh{草果}}

 \mytextsc{clf}~: \textcolor{darkblue}{\textbf{\ipa{ɭɯ˧}}} 

\paragraph{\hspace{-0.5cm} {\Large \textcolor{darkblue}{\textbf{\ipa{tsʰo˧pæ\#˥}}}}} \hypertarget{ts\string_ho\string_Mp\{\#\string_T1}{}
\markboth{\textcolor{darkblue}{\textbf{\ipa{tsʰo˧pæ\#˥}}}}{}
\textcolor{teal}{\mytextsc{nom}} \hspace{4pt} Ton~: \#H.

Chef de caravane.
\textcolor{brown}{\zh{马帮头领。}}

 Emprunt~: tibétain tshong.pa``merchant"



\paragraph{\hspace{-0.5cm} {\Large \textcolor{darkblue}{\textbf{\ipa{tsʰo˧pjɤ˧}}}}} \hypertarget{ts\string_ho\string_Mpj7\string_M1}{}
\markboth{\textcolor{darkblue}{\textbf{\ipa{tsʰo˧pjɤ˧}}}}{}
\textcolor{teal}{\mytextsc{nom}} \hspace{4pt} Ton~: M.

Savon. Sans doute mot emprunté à une langue de birmanie: cp. nung: tshɑ³¹ pi⁵⁵ iɔ⁵⁵ [Dai 1992]; achang de Luxi et Lianghe: tshɑu⁵⁵ pjɑu⁵⁵ [Dai 1985]; achang de Longchuan: tshau³¹ piau³¹ [Dai 1992]. Culturellement, il est plausible que le savon ait été introduit par le contact/commerce avec des groupes ethniques de Birmanie.
\textcolor{brown}{\zh{肥皂。}}


 \mytextsc{clf}~: \textcolor{darkblue}{\textbf{\ipa{ɭɯ˧}}} 

\paragraph{\hspace{-0.5cm} {\Large \textcolor{darkblue}{\textbf{\ipa{tsʰo˧qʰwɤ˧mi\#˥}}}}} \hypertarget{ts\string_ho\string_Mq\string_hw7\string_Mmi\#\string_T1}{}
\markboth{\textcolor{darkblue}{\textbf{\ipa{tsʰo˧qʰwɤ˧mi\#˥}}}}{}
\textcolor{teal}{\mytextsc{nom}} \hspace{4pt} Ton~: \#H.

Démon, fantôme.
\textcolor{brown}{\zh{鬼。}}


 \mytextsc{clf}~: \textcolor{darkblue}{\textbf{\ipa{v̩˧}}} \textit{Voir~:} \hyperlink{ts\string_ho\string_Mq\string_hw7\string_Mzo\#\string_T1}{\textcolor{darkblue}{\textbf{\ipa{tsʰo˧qʰwɤ˧zo\#˥}}}} 

\paragraph{\hspace{-0.5cm} {\Large \textcolor{darkblue}{\textbf{\ipa{tsʰo˧qʰwɤ˧mi˧-bæ˥bæ˩}}}}} \hypertarget{ts\string_ho\string_Mq\string_hw7\string_Mmi\string_M-b\{\string_Tb\{\string_B1}{}
\markboth{\textcolor{darkblue}{\textbf{\ipa{tsʰo˧qʰwɤ˧mi˧-bæ˥bæ˩}}}}{}
\textcolor{teal}{\mytextsc{nom}} \hspace{4pt} Ton~: \#H-.

Une fleur bleue, \textit{Delphinium grandiflorum}.
\textcolor{brown}{\zh{翠雀花。}}


\textit{Voir~:} \hyperlink{ts\string_ho\string_Mq\string_hw7\string_Mmi\#\string_T1}{\textcolor{darkblue}{\textbf{\ipa{tsʰo˧qʰwɤ˧mi\#˥}}}} 

\paragraph{\hspace{-0.5cm} {\Large \textcolor{darkblue}{\textbf{\ipa{tsʰo˧qʰwɤ˧zo\#˥}}}}} \hypertarget{ts\string_ho\string_Mq\string_hw7\string_Mzo\#\string_T1}{}
\markboth{\textcolor{darkblue}{\textbf{\ipa{tsʰo˧qʰwɤ˧zo\#˥}}}}{}
\textcolor{teal}{\mytextsc{nom}} \hspace{4pt} Ton~: \#H.

Démon, fantôme (forme moins courante que celle comportant un suffixe féminin).
\textcolor{brown}{\zh{鬼。}}


\textit{Voir~:} \hyperlink{ts\string_ho\string_Mq\string_hw7\string_Mmi\#\string_T1}{\textcolor{darkblue}{\textbf{\ipa{tsʰo˧qʰwɤ˧mi\#˥}}}} 

\paragraph{\hspace{-0.5cm} {\Large \textcolor{darkblue}{\textbf{\ipa{tsʰo˧qʰwɤ˩}}}}} \hypertarget{ts\string_ho\string_Mq\string_hw7\string_B1}{}
\markboth{\textcolor{darkblue}{\textbf{\ipa{tsʰo˧qʰwɤ˩}}}}{}
\textcolor{teal}{\mytextsc{adverbe}} \hspace{4pt} Ton~: L\#.

Ce soir.
\textcolor{brown}{\zh{今晚。}}


\begin{exe}
\sn \textcolor{darkblue}{\textbf{\ipa{tsʰo˧qʰwɤ˩ | mv̩˩kʰv̩˧˥}}}
\trans même sens: ce soir
\trans \textit{\textcolor{brown}{\zh{同上:今晚}}}
\end{exe}


\paragraph{\hspace{-0.5cm} {\Large \textcolor{darkblue}{\textbf{\ipa{tsʰo˧ʁo\#˥}}}}} \hypertarget{ts\string_ho\string_MRo\#\string_T1}{}
\markboth{\textcolor{darkblue}{\textbf{\ipa{tsʰo˧ʁo\#˥}}}}{}
\textcolor{teal}{\mytextsc{nom}} \hspace{4pt} Ton~: \#H.

Étable: bâtiment à l'entrée de la ferme, que l'on traverse en entrant dans la ferme. Construit en bois. Au rez-de-chaussée se trouvent les étables des porcs; à l'étage un grenier à foin.
\textcolor{brown}{\zh{牲畜圈:家门口的那栋楼,下为畜厩,上存饲料或另辟为房间。}}


 \mytextsc{clf}~: \textcolor{darkblue}{\textbf{\ipa{ɭɯ˧}}} 

\paragraph{\hspace{-0.5cm} {\Large \textcolor{darkblue}{\textbf{\ipa{tsʰo˧tsɯ˧}}}}} \hypertarget{ts\string_ho\string_MtsM\string_M1}{}
\markboth{\textcolor{darkblue}{\textbf{\ipa{tsʰo˧tsɯ˧}}}}{}
\textcolor{teal}{\mytextsc{nom}} \hspace{4pt} Ton~: M.

Poireau, oignon.
\textcolor{brown}{\zh{葱,韭葱。}}

 Emprunt~: chinois \textcolor{brown}{\zh{葱子}}

 \mytextsc{clf}~: \textcolor{darkblue}{\textbf{\ipa{po˧}}} 

\paragraph{\hspace{-0.5cm} {\Large \textcolor{darkblue}{\textbf{\ipa{tsʰo˩}}}}} \hypertarget{ts\string_ho\string_B1}{}
\markboth{\textcolor{darkblue}{\textbf{\ipa{tsʰo˩}}}}{}
\textcolor{teal}{\mytextsc{nom}} \hspace{4pt} Ton~: L.

Espèce humaine, êtres humains; terme ancien apparaissant dans certains proverbes.
\textcolor{brown}{\zh{人类。}}




\paragraph{\hspace{-0.5cm} {\Large \textcolor{darkblue}{\textbf{\ipa{tsʰo˩mo˩}}}}} \hypertarget{ts\string_ho\string_Bmo\string_B1}{}
\markboth{\textcolor{darkblue}{\textbf{\ipa{tsʰo˩mo˩}}}}{}
\textcolor{teal}{\mytextsc{nom}} \hspace{4pt} Ton~: L.

Vieil homme, vieillard.
\textcolor{brown}{\zh{老头。}}




\paragraph{\hspace{-0.5cm} {\Large \textcolor{darkblue}{\textbf{\ipa{tsʰo˩tsɯ˧}}}}} \hypertarget{ts\string_ho\string_BtsM\string_M1}{}
\markboth{\textcolor{darkblue}{\textbf{\ipa{tsʰo˩tsɯ˧}}}}{}
\textcolor{teal}{\mytextsc{nom}} \hspace{4pt} Ton~: LM.

Lime.
\textcolor{brown}{\zh{锉刀。}}

 Emprunt~: chinois \textcolor{brown}{\zh{锉子}}

 \mytextsc{clf}~: \textcolor{darkblue}{\textbf{\ipa{nɑ˧}}} 

\paragraph{\hspace{-0.5cm} {\Large \textcolor{darkblue}{\textbf{\ipa{tsʰo˧˥}}}}} \hypertarget{ts\string_ho\string_M\string_T1}{}
\markboth{\textcolor{darkblue}{\textbf{\ipa{tsʰo˧˥}}}}{}
\textcolor{teal}{\mytextsc{nom}} \hspace{4pt} Ton~: MH.

Respect, attention, estime.
\textcolor{brown}{\zh{重视、关心、恭敬。}}


\begin{exe}
\sn \textcolor{darkblue}{\textbf{\ipa{ʈʂʰɯ˧-ɳɯ˧ | njɤ˧-ki˧ | ɖwæ˧˥ | tsʰo˧ ʝi˥!}}}
\trans Il/elle me traite avec les plus grands égards / est aux petits soins pour moi!
\trans \textit{\textcolor{brown}{\zh{他很重视我 / 他对我很尊敬、很关心。}}}
\end{exe}


\paragraph{\hspace{-0.5cm} {\Large \textcolor{darkblue}{\textbf{\ipa{tsʰɯ˧hṽ˥\$}}}}} \hypertarget{ts\string_hM\string_Mhv\string_~\string_T\$1}{}
\markboth{\textcolor{darkblue}{\textbf{\ipa{tsʰɯ˧hṽ˥\$}}}}{}
\textcolor{teal}{\mytextsc{nom}} \hspace{4pt} Ton~: H\$.

Laine.
\textcolor{brown}{\zh{羊毛。}}


 \mytextsc{clf}~: \textcolor{darkblue}{\textbf{\ipa{kʰwɤ˥}}} 

\paragraph{\hspace{-0.5cm} {\Large \textcolor{darkblue}{\textbf{\ipa{tsʰɯ˧mi˥\$}}}}} \hypertarget{ts\string_hM\string_Mmi\string_T\$1}{}
\markboth{\textcolor{darkblue}{\textbf{\ipa{tsʰɯ˧mi˥\$}}}}{}
\textcolor{teal}{\mytextsc{nom}} \hspace{4pt} Ton~: H\$.

Chèvre.
\textcolor{brown}{\zh{母山羊。}}


\begin{exe}
\sn \textcolor{darkblue}{\textbf{\ipa{tsʰɯ˧mi˧-po˧lo˥}}}
\trans chèvre et bouc
\trans \textit{\textcolor{brown}{\zh{母山羊与公山羊}}}
\end{exe}
 \mytextsc{clf}~: \textcolor{darkblue}{\textbf{\ipa{pʰo˧˥}}} 

\paragraph{\hspace{-0.5cm} {\Large \textcolor{darkblue}{\textbf{\ipa{tsʰɯ˧mi˧-to˧qɑ˥\$}}}}} \hypertarget{ts\string_hM\string_Mmi\string_M-to\string_MqA\string_T\$1}{}
\markboth{\textcolor{darkblue}{\textbf{\ipa{tsʰɯ˧mi˧-to˧qɑ˥\$}}}}{}
\textcolor{teal}{\mytextsc{nom}} \hspace{4pt} Ton~: H\$.

Bouc; s'emploie aussi pour un chevreau (cabri), ou même pour toute l'espèce, y compris les chèvres.
\textcolor{brown}{\zh{公山羊(包括公山羊羔)(可以来指所有的山羊,包括母的和公的)。}}


 \mytextsc{clf}~: \textcolor{darkblue}{\textbf{\ipa{pʰo˧˥}}} 

\paragraph{\hspace{-0.5cm} {\Large \textcolor{darkblue}{\textbf{\ipa{tsʰɯ˧pʰv̩\#˥}}}}} \hypertarget{ts\string_hM\string_Mp\string_hv\string_=\#\string_T1}{}
\markboth{\textcolor{darkblue}{\textbf{\ipa{tsʰɯ˧pʰv̩\#˥}}}}{}
\textcolor{teal}{\mytextsc{nom}} \hspace{4pt} Ton~: \#H.

Bouc (terme élicité; plus courant: \textcolor{darkblue}{\textbf{\ipa{/po˧lo˧/}}}).
\textcolor{brown}{\zh{公山羊。}}


 \mytextsc{clf}~: \textcolor{darkblue}{\textbf{\ipa{pʰo˧˥}}} 

\paragraph{\hspace{-0.5cm} {\Large \textcolor{darkblue}{\textbf{\ipa{tsʰɯ˧ɻ̍˧-ɖɯ˩mɑ˩}}}}} \hypertarget{ts\string_hM\string_Mr£`̍\string_M-d`M\string_BmA\string_B1}{}
\markboth{\textcolor{darkblue}{\textbf{\ipa{tsʰɯ˧ɻ̍˧-ɖɯ˩mɑ˩}}}}{}
\textcolor{teal}{\mytextsc{nom}} \hspace{4pt} Ton~: \mytextsc{L}.

Prénom féminin.
\textcolor{brown}{\zh{女性名字。}}




\paragraph{\hspace{-0.5cm} {\Large \textcolor{darkblue}{\textbf{\ipa{tsʰɯ˧ɻ̍˧-ɬɑ˩mv̩˩}}}}} \hypertarget{ts\string_hM\string_Mr£`̍\string_M-KA\string_Bmv\string_=\string_B1}{}
\markboth{\textcolor{darkblue}{\textbf{\ipa{tsʰɯ˧ɻ̍˧-ɬɑ˩mv̩˩}}}}{}
\textcolor{teal}{\mytextsc{nom}} \hspace{4pt} Ton~: \mytextsc{L}.

Prénom féminin.
\textcolor{brown}{\zh{女性名字。}}




\paragraph{\hspace{-0.5cm} {\Large \textcolor{darkblue}{\textbf{\ipa{tsʰɯ˧ɻ̍˧-pʰi˩tsʰo˩}}}}} \hypertarget{ts\string_hM\string_Mr£`̍\string_M-p\string_hi\string_Bts\string_ho\string_B1}{}
\markboth{\textcolor{darkblue}{\textbf{\ipa{tsʰɯ˧ɻ̍˧-pʰi˩tsʰo˩}}}}{}
\textcolor{teal}{\mytextsc{nom}} \hspace{4pt} Ton~: \mytextsc{L}.

Prénom masculin.
\textcolor{brown}{\zh{男性名字。}}




\paragraph{\hspace{-0.5cm} {\Large \textcolor{darkblue}{\textbf{\ipa{tsʰɯ˧ɻ\#˥}}}}} \hypertarget{ts\string_hM\string_Mr£`\#\string_T1}{}
\markboth{\textcolor{darkblue}{\textbf{\ipa{tsʰɯ˧ɻ\#˥}}}}{}
\textcolor{teal}{\mytextsc{nom}} \hspace{4pt} Ton~: \#H.

Prénom unisexe: prénom utilisé pour les deux sexes.
\textcolor{brown}{\zh{男女通用名。}}




\paragraph{\hspace{-0.5cm} {\Large \textcolor{darkblue}{\textbf{\ipa{tsʰɯ˧ʂwæ˥}}}}} \hypertarget{ts\string_hM\string_Ms`w\{\string_T1}{}
\markboth{\textcolor{darkblue}{\textbf{\ipa{tsʰɯ˧ʂwæ˥}}}}{}
\textcolor{teal}{\mytextsc{nom}} \hspace{4pt} Ton~: H\#.

Bouc castré.
\textcolor{brown}{\zh{阉山羊。}}


 \mytextsc{clf}~: \textcolor{darkblue}{\textbf{\ipa{pʰo˧˥}}} 

\paragraph{\hspace{-0.5cm} {\Large \textcolor{darkblue}{\textbf{\ipa{tsʰɯ˧-to˧qɑ˥}}}}} \hypertarget{ts\string_hM\string_M-to\string_MqA\string_T1}{}
\markboth{\textcolor{darkblue}{\textbf{\ipa{tsʰɯ˧-to˧qɑ˥}}}}{}
\textcolor{teal}{\mytextsc{nom}} \hspace{4pt} Ton~: H\#.

Chevreau, cabri.
\textcolor{brown}{\zh{羔羊、羔子。}}


 \mytextsc{clf}~: \textcolor{darkblue}{\textbf{\ipa{pʰo˧˥}}} \textit{Voir~:} \hyperlink{ts\string_hM\string_Mzo\string_T\$1}{\textcolor{darkblue}{\textbf{\ipa{tsʰɯ˧zo˥\$}}}} 

\paragraph{\hspace{-0.5cm} {\Large \textcolor{darkblue}{\textbf{\ipa{tsʰɯ˧zo\#˥}}}}} \hypertarget{ts\string_hM\string_Mzo\#\string_T1}{}
\markboth{\textcolor{darkblue}{\textbf{\ipa{tsʰɯ˧zo\#˥}}}}{}
\textcolor{teal}{\mytextsc{nom}} \hspace{4pt} Ton~: \#H.

Chevrette.
\textcolor{brown}{\zh{母山羊羔。}}


 \mytextsc{clf}~: \textcolor{darkblue}{\textbf{\ipa{ɭɯ˧}}} 

\paragraph{\hspace{-0.5cm} {\Large \textcolor{darkblue}{\textbf{\ipa{tsʰɯ˧zo˥\$}}}}} \hypertarget{ts\string_hM\string_Mzo\string_T\$1}{}
\markboth{\textcolor{darkblue}{\textbf{\ipa{tsʰɯ˧zo˥\$}}}}{}
\textcolor{teal}{\mytextsc{nom}} \hspace{4pt} Ton~: H\$.

Chevreau, cabri.
\textcolor{brown}{\zh{山羊羔。}}


\begin{exe}
\sn \textcolor{darkblue}{\textbf{\ipa{tsʰɯ˧zo˧-to˧qɑ˥}}}
\trans chevrettes et chevreaux
\trans \textit{\textcolor{brown}{\zh{母山羊羔与公山羊羔}}}
\end{exe}
 \mytextsc{clf}~: \textcolor{darkblue}{\textbf{\ipa{ɭɯ˧}}} \textit{Voir~:} \hyperlink{ts\string_hM\string_M-to\string_MqA\string_T1}{\textcolor{darkblue}{\textbf{\ipa{tsʰɯ˧-to˧qɑ˥}}}} 

\paragraph{\hspace{-0.5cm} {\Large \textcolor{darkblue}{\textbf{\ipa{tsʰɯ˩\textsubscript{a}}}}}} \hypertarget{ts\string_hM\string_Ba1}{}
\markboth{\textcolor{darkblue}{\textbf{\ipa{tsʰɯ˩\textsubscript{a}}}}}{}
\textcolor{teal}{\mytextsc{verbe}} \hspace{4pt} Ton~: L\textsubscript{a}.

Venir (\mytextsc{pst}).
\textcolor{brown}{\zh{来(过去式)。}}


\begin{exe}
\sn \textcolor{darkblue}{\textbf{\ipa{le˧-gwɤ˩\textasciitilde{}gwɤ˩ | le˧-tsʰɯ˩-ze˩}}}
\trans revenir de promenade
\trans \textit{\textcolor{brown}{\zh{散步回来}}}
\end{exe}
\begin{exe}
\sn \textcolor{darkblue}{\textbf{\ipa{le˧-tsʰɯ˩-ze˩}}}
\trans être de retour
\trans \textit{\textcolor{brown}{\zh{回来了}}}
\end{exe}
\begin{exe}
\sn \textcolor{darkblue}{\textbf{\ipa{ɖɯ˧-ʝi˧-ɳɯ˧ tsʰɯ˧˥, | ɖɯ˧-ki˧ tʰv̩˧!}}}
\trans ``Venus de différents endroits, nous voici réunis en ce lieu!" L'expression est obscure pour qui ne l'a pas apprise (par exemple pour des locuteurs du bord du Lac): son sens littéral n'est pas particulièrement parlant: ``On vient d'un endroit; on arrive à un endroit!"
\trans \textit{\textcolor{brown}{\zh{“我们都来自不同的地方,但现在在一起了!”}}}
\end{exe}


\paragraph{\hspace{-0.5cm} {\Large \textcolor{darkblue}{\textbf{\ipa{tsʰɯ˩tsʰɯ˩ɻ̃˩}}}}} \hypertarget{ts\string_hM\string_Bts\string_hM\string_Br£`\string_~\string_B1}{}
\markboth{\textcolor{darkblue}{\textbf{\ipa{tsʰɯ˩tsʰɯ˩ɻ̃˩}}}}{}
\textcolor{teal}{\mytextsc{nom}} \hspace{4pt} Ton~: L.

Paille de petits pois.
\textcolor{brown}{\zh{豌豆干草。}}


\begin{exe}
\sn \textcolor{darkblue}{\textbf{\ipa{ʈʂʰɯ˧ | tsʰɯ˩tsʰɯ˩ɻ̃˩ ɲi˥.}}}
\trans C'est de la paille de haricots.
\trans \textit{\textcolor{brown}{\zh{这是豌豆干草。}}}
\end{exe}
 \mytextsc{clf}~: \textcolor{darkblue}{\textbf{\ipa{kɤ˧˥}}} 

\paragraph{\hspace{-0.5cm} {\Large \textcolor{darkblue}{\textbf{\ipa{tsʰɯ˧˥}}} \textsubscript{1}}} \hypertarget{ts\string_hM\string_M\string_T1}{}
\markboth{\textcolor{darkblue}{\textbf{\ipa{tsʰɯ˧˥}}} \textsubscript{1}}{}
\textcolor{teal}{\mytextsc{verbe}} \hspace{4pt} Ton~: MH.

Taillader (ex.: taillader un vêtement, le découper avec des ciseaux; n'est pas: tailler du tissu pour faire des vêtements).
\textcolor{brown}{\zh{剪成片。}}


\begin{exe}
\sn \textcolor{darkblue}{\textbf{\ipa{tʰɑ˧-tsʰɯ˧˥!}}}
\trans \mytextsc{prohib}
\trans \textit{\textcolor{brown}{\zh{\mytextsc{prohib}}}}
\end{exe}
\begin{exe}
\sn \textcolor{darkblue}{\textbf{\ipa{dʑi˧hṽ˧ tsʰɯ˩}}}
\trans couper des vêtements en morceaux
\trans \textit{\textcolor{brown}{\zh{把衣服剪成片}}}
\end{exe}


\paragraph{\hspace{-0.5cm} {\Large \textcolor{darkblue}{\textbf{\ipa{tsʰɯ˧˥}}} \textsubscript{2}}} \hypertarget{ts\string_hM\string_M\string_T2}{}
\markboth{\textcolor{darkblue}{\textbf{\ipa{tsʰɯ˧˥}}} \textsubscript{2}}{}
\textcolor{teal}{\mytextsc{nom}} \hspace{4pt} Ton~: MH.

Chèvre/bouc.
\textcolor{brown}{\zh{山羊。}}


 \mytextsc{clf}~: \textcolor{darkblue}{\textbf{\ipa{pʰo˧˥}}} 

\paragraph{\hspace{-0.5cm} {\Large \textcolor{darkblue}{\textbf{\ipa{tsʰv̩˩˥}}}}} \hypertarget{ts\string_hv\string_=\string_B\string_T1}{}
\markboth{\textcolor{darkblue}{\textbf{\ipa{tsʰv̩˩˥}}}}{}
\textcolor{teal}{\mytextsc{nom}} \hspace{4pt} Ton~: LH.

Vinaigre.
\textcolor{brown}{\zh{醋(汉语借词)。}}

 Emprunt~: chinois \textcolor{brown}{\zh{醋}}

\textit{Voir~:} \textcolor{darkblue}{\textbf{\ipa{sɑ˧tsʰv̩˩, tɕi˧-dʑɯ˩}}} 

\newpage
\section*{\centering- \textcolor{darkblue}{\textbf{\ipa{ʈ}}} -}
\paragraph{\hspace{-0.5cm} {\Large \textcolor{darkblue}{\textbf{\ipa{ʈæ˧bɤ˧}}}}} \hypertarget{t`\{\string_Mb7\string_M1}{}
\markboth{\textcolor{darkblue}{\textbf{\ipa{ʈæ˧bɤ˧}}}}{}
\textcolor{teal}{\mytextsc{nom}} \hspace{4pt} Ton~: M.

Moine, nonne.
\textcolor{brown}{\zh{和尚,尼姑。}}


\begin{exe}
\sn \textcolor{darkblue}{\textbf{\ipa{ʈæ˧bɤ˧ʈʂʰo˧}}}
\trans même sens
\trans \textit{\textcolor{brown}{\zh{同上}}}
\end{exe}
\begin{exe}
\sn \textcolor{darkblue}{\textbf{\ipa{ʈæ˧bɤ˧ ʝi˧-hĩ˧-hĩ˧}}}
\trans même sens
\trans \textit{\textcolor{brown}{\zh{当和尚的人}}}
\end{exe}
\begin{exe}
\sn \textcolor{darkblue}{\textbf{\ipa{hæ˧ʈæ˩bɤ˩}}}
\trans moine chinois
\trans \textit{\textcolor{brown}{\zh{汉人和尚}}}
\end{exe}
 \mytextsc{clf}~: \textcolor{darkblue}{\textbf{\ipa{v̩˧}}} 

\paragraph{\hspace{-0.5cm} {\Large \textcolor{darkblue}{\textbf{\ipa{ʈæ˧kwæ˧˥}}}}} \hypertarget{t`\{\string_Mkw\{\string_M\string_T1}{}
\markboth{\textcolor{darkblue}{\textbf{\ipa{ʈæ˧kwæ˧˥}}}}{}
\textcolor{teal}{\mytextsc{adjectif}} \hspace{4pt} Ton~: MH\#.

Prodigue, qui dépense tout.
\textcolor{brown}{\zh{爱浪费。}}


\begin{exe}
\sn \textcolor{darkblue}{\textbf{\ipa{ʈʂʰɯ˧ | ʈæ˧kwæ˧-hĩ˥ | ɖɯ˧-v̩˧ ɲi˩.}}}
\trans C'est un prodigue/quelqu'un qui dépense tout/qui mène la maison à la ruine.
\trans \textit{\textcolor{brown}{\zh{他是爱浪费的人。}}}
\end{exe}


\paragraph{\hspace{-0.5cm} {\Large \textcolor{darkblue}{\textbf{\ipa{ʈæ˧qo˧}}}}} \hypertarget{t`\{\string_Mqo\string_M1}{}
\markboth{\textcolor{darkblue}{\textbf{\ipa{ʈæ˧qo˧}}}}{}
\textcolor{teal}{\mytextsc{adverbe}} \hspace{4pt} Ton~: M.

Au fond de.
\textcolor{brown}{\zh{底下。}}


\begin{exe}
\sn \textcolor{darkblue}{\textbf{\ipa{hi˩nɑ˧mi˧-ʈæ˧qo˥}}}
\trans au fond du Lac
\trans \textit{\textcolor{brown}{\zh{在湖底下}}}
\end{exe}
\begin{exe}
\sn \textcolor{darkblue}{\textbf{\ipa{ʈæ˧qo˧ tɕɯ˧}}}
\trans mettre au fond de...
\trans \textit{\textcolor{brown}{\zh{放在底下}}}
\end{exe}
\begin{exe}
\sn \textcolor{darkblue}{\textbf{\ipa{hi˩nɑ˧mi˧, | ʈæ˧ mɤ˧-do˩; | hĩ˧-nv̩˥mi˩, | ɳv̩˧ mɤ˧-tʰɑ˩.}}}
\trans ``On ne voit pas le fond du lac; on ne connaît pas le cœur des hommes!" (Proverbe qui apparaît dans les chansons que se chantaient les jeunes gens se faisant la cour.)
\trans \textit{\textcolor{brown}{\zh{“人的心,湖底藏:看不清,摸不透!” 直译:“湖,(我们)看不到(它的)底下。人的心,是知道不了的!”(情歌里的一个谚语)}}}
\end{exe}


\paragraph{\hspace{-0.5cm} {\Large \textcolor{darkblue}{\textbf{\ipa{ʈæ˧ʂɯ˧}}}}} \hypertarget{t`\{\string_Ms`M\string_M1}{}
\markboth{\textcolor{darkblue}{\textbf{\ipa{ʈæ˧ʂɯ˧}}}}{}
\textcolor{teal}{\mytextsc{nom}} \hspace{4pt} Ton~: M.

Prénom masculin.
\textcolor{brown}{\zh{男性名字。}}




\paragraph{\hspace{-0.5cm} {\Large \textcolor{darkblue}{\textbf{\ipa{ʈæ˧ʂɯ˧-ɖɯ˩mɑ˩}}}}} \hypertarget{t`\{\string_Ms`M\string_M-d`M\string_BmA\string_B1}{}
\markboth{\textcolor{darkblue}{\textbf{\ipa{ʈæ˧ʂɯ˧-ɖɯ˩mɑ˩}}}}{}
\textcolor{teal}{\mytextsc{nom}} \hspace{4pt} Ton~: \mytextsc{L}.

Prénom féminin.
\textcolor{brown}{\zh{女性名字。}}




\paragraph{\hspace{-0.5cm} {\Large \textcolor{darkblue}{\textbf{\ipa{ʈæ˧ʂɯ˧-ɬɑ˩mv̩˩}}}}} \hypertarget{t`\{\string_Ms`M\string_M-KA\string_Bmv\string_=\string_B1}{}
\markboth{\textcolor{darkblue}{\textbf{\ipa{ʈæ˧ʂɯ˧-ɬɑ˩mv̩˩}}}}{}
\textcolor{teal}{\mytextsc{nom}} \hspace{4pt} Ton~: \mytextsc{L}.

Prénom féminin.
\textcolor{brown}{\zh{女性名字。}}




\paragraph{\hspace{-0.5cm} {\Large \textcolor{darkblue}{\textbf{\ipa{ʈæ˧ʂɯ˧-pæ˩pʰæ˩}}}}} \hypertarget{t`\{\string_Ms`M\string_M-p\{\string_Bp\string_h\{\string_B1}{}
\markboth{\textcolor{darkblue}{\textbf{\ipa{ʈæ˧ʂɯ˧-pæ˩pʰæ˩}}}}{}
\textcolor{teal}{\mytextsc{nom}} \hspace{4pt} Ton~: \mytextsc{L}.

Prénom masculin.
\textcolor{brown}{\zh{男性名字。}}




\paragraph{\hspace{-0.5cm} {\Large \textcolor{darkblue}{\textbf{\ipa{ʈæ˧ʂɯ˧-tsʰi˩ti˩}}}}} \hypertarget{t`\{\string_Ms`M\string_M-ts\string_hi\string_Bti\string_B1}{}
\markboth{\textcolor{darkblue}{\textbf{\ipa{ʈæ˧ʂɯ˧-tsʰi˩ti˩}}}}{}
\textcolor{teal}{\mytextsc{nom}} \hspace{4pt} Ton~: \mytextsc{L}.

Prénom masculin.
\textcolor{brown}{\zh{男性名字。}}




\paragraph{\hspace{-0.5cm} {\Large \textcolor{darkblue}{\textbf{\ipa{ʈæ˧ʂɯ˧-ʈæ˩ʈv̩˩}}}}} \hypertarget{t`\{\string_Ms`M\string_M-t`\{\string_Bt`v\string_=\string_B1}{}
\markboth{\textcolor{darkblue}{\textbf{\ipa{ʈæ˧ʂɯ˧-ʈæ˩ʈv̩˩}}}}{}
\textcolor{teal}{\mytextsc{nom}} \hspace{4pt} Ton~: \mytextsc{L}.

Prénom masculin.
\textcolor{brown}{\zh{男性名字。}}




\paragraph{\hspace{-0.5cm} {\Large \textcolor{darkblue}{\textbf{\ipa{ʈæ˩\textsubscript{a}}}}}} \hypertarget{t`\{\string_Ba1}{}
\markboth{\textcolor{darkblue}{\textbf{\ipa{ʈæ˩\textsubscript{a}}}}}{}
\textcolor{teal}{\mytextsc{verbe}} \hspace{4pt} Ton~: L\textsubscript{a}.
\ding{202} 
Fermer, refermer; enfermer (ex.: des moutons); aussi: fermer une route.
\textcolor{brown}{\zh{关(门、羊)。}}


\begin{exe}
\sn \textcolor{darkblue}{\textbf{\ipa{bv̩˩qo˩ ʈæ˥}}}
\trans fermer l'étable
\trans \textit{\textcolor{brown}{\zh{关牛圈}}}
\end{exe}
\begin{exe}
\sn \textcolor{darkblue}{\textbf{\ipa{tʰi˧-ʈæ˩}}}
\trans \mytextsc{dur}: fermer
\trans \textit{\textcolor{brown}{\zh{关门}}}
\end{exe}
\begin{exe}
\sn \textcolor{darkblue}{\textbf{\ipa{kʰi˧ ʈæ˥}}}
\trans fermer la porte
\trans \textit{\textcolor{brown}{\zh{关门}}}
\end{exe}
\ding{203} 
Nouer (un noeud).
\textcolor{brown}{\zh{扣(扣子)、系、结。}}




\paragraph{\hspace{-0.5cm} {\Large \textcolor{darkblue}{\textbf{\ipa{ʈæ˩ɖɯ˧}}}}} \hypertarget{t`\{\string_Bd`M\string_M1}{}
\markboth{\textcolor{darkblue}{\textbf{\ipa{ʈæ˩ɖɯ˧}}}}{}
\textcolor{teal}{\mytextsc{adjectif}} \hspace{4pt} Ton~: LM.

Satisfait, tranquille.
\textcolor{brown}{\zh{安乐。}}


\begin{exe}
\sn \textcolor{darkblue}{\textbf{\ipa{mɤ˧-ʈæ˩ɖɯ˩}}}
\trans mécontent, furieux
\trans \textit{\textcolor{brown}{\zh{不高兴、不安}}}
\end{exe}
\begin{exe}
\sn \textcolor{darkblue}{\textbf{\ipa{ə˧mɑ˧ | tsʰi˧-ɲi˧ | ʈæ˩ɖɯ˧ tʰi˧-dzi˩-dʑo˩!}}}
\trans Aujourd'hui, Ama est assise bien tranquille!
\trans \textit{\textcolor{brown}{\zh{今天,阿妈安乐地坐着。}}}
\end{exe}
\begin{exe}
\sn \textcolor{darkblue}{\textbf{\ipa{ʈʂʰɯ˧-ɳɯ˧ | njɤ˧-ki˧ | mɤ˧-ʈæ˩ɖɯ˩-hĩ˩ ʐwɤ˩!}}}
\trans il m'a dit des choses qui fâchent! (=il m'a vexé, il m'a dit des choses désobligeantes)
\trans \textit{\textcolor{brown}{\zh{他跟我说了一些让我不安的(事情)! / 他跟我说的,让我生气!}}}
\end{exe}


\paragraph{\hspace{-0.5cm} {\Large \textcolor{darkblue}{\textbf{\ipa{ʈæ˩tsʰo\#˥}}} \textsubscript{1}}} \hypertarget{t`\{\string_Bts\string_ho\#\string_T1}{}
\markboth{\textcolor{darkblue}{\textbf{\ipa{ʈæ˩tsʰo\#˥}}} \textsubscript{1}}{}
\textcolor{teal}{\mytextsc{nom}} \hspace{4pt} Ton~: LM+\#H.

Classe, groupe, ensemble (de prêtres).
\textcolor{brown}{\zh{班、小组。}}


\begin{exe}
\sn \textcolor{darkblue}{\textbf{\ipa{ʈæ˩tsʰo˧ | ɖɯ˧-ɭɯ˧}}}
\trans une classe, un groupe (de prêtres)
\trans \textit{\textcolor{brown}{\zh{一个小组、一帮(和尚)}}}
\end{exe}
 \mytextsc{clf}~: \textcolor{darkblue}{\textbf{\ipa{ɭɯ˧}}} \textit{Voir~:} \hyperlink{t`\{\string_Bts\string_ho\#\string_T2}{\textcolor{darkblue}{\textbf{\ipa{ʈæ˩tsʰo\#˥}}} \textsubscript{2}} 

\paragraph{\hspace{-0.5cm} {\Large \textcolor{darkblue}{\textbf{\ipa{ʈæ˩tsʰo\#˥}}} \textsubscript{2}}} \hypertarget{t`\{\string_Bts\string_ho\#\string_T2}{}
\markboth{\textcolor{darkblue}{\textbf{\ipa{ʈæ˩tsʰo\#˥}}} \textsubscript{2}}{}
\textcolor{teal}{\mytextsc{classificateur}} \hspace{4pt} Ton~: L.

Auto-classificateur des classes/groupes (de prêtres).
\textcolor{brown}{\zh{量词:和尚(一帮、一班)。}}


\begin{exe}
\sn \textcolor{darkblue}{\textbf{\ipa{ɖɯ˧-ʈæ˩tsʰo˩}}}
\trans un groupe (de prêtres)
\trans \textit{\textcolor{brown}{\zh{一班(和尚)}}}
\end{exe}
\textit{Voir~:} \hyperlink{t`\{\string_Bts\string_ho\#\string_T1}{\textcolor{darkblue}{\textbf{\ipa{ʈæ˩tsʰo\#˥}}} \textsubscript{1}} 

\paragraph{\hspace{-0.5cm} {\Large \textcolor{darkblue}{\textbf{\ipa{ʈæ˩ʈv̩\#˥}}}}} \hypertarget{t`\{\string_Bt`v\string_=\#\string_T1}{}
\markboth{\textcolor{darkblue}{\textbf{\ipa{ʈæ˩ʈv̩\#˥}}}}{}
\textcolor{teal}{\mytextsc{nom}} \hspace{4pt} Ton~: LM+\#H.

Prénom masculin.
\textcolor{brown}{\zh{男性名字。}}




\paragraph{\hspace{-0.5cm} {\Large \textcolor{darkblue}{\textbf{\ipa{ʈɤ˧\textsubscript{a}}}}}} \hypertarget{t`7\string_Ma1}{}
\markboth{\textcolor{darkblue}{\textbf{\ipa{ʈɤ˧\textsubscript{a}}}}}{}
\textcolor{teal}{\mytextsc{verbe}} \hspace{4pt} Ton~: M\textsubscript{a}.

Tirer.
\textcolor{brown}{\zh{拉、拽。}}


\begin{exe}
\sn \textcolor{darkblue}{\textbf{\ipa{tso˧\textasciitilde{}tso˧ ʈɤ˩(-ze˩)}}}
\trans tirer quelque chose
\trans \textit{\textcolor{brown}{\zh{拉拽东西}}}
\end{exe}
\begin{exe}
\sn \textcolor{darkblue}{\textbf{\ipa{mv̩˧ʐe˧ qʰæ˩ | le˧-wo˧-ʈɤ˥-di˩}}}
\trans périphrase pour désigner la gâchette d'un pistolet: ce qu'on tire vers soi pour faire feu
\trans \textit{\textcolor{brown}{\zh{扳机}}}
\end{exe}


\paragraph{\hspace{-0.5cm} {\Large \textcolor{darkblue}{\textbf{\ipa{ʈi˥\textsubscript{a}}}}}} \hypertarget{t`i\string_Ta1}{}
\markboth{\textcolor{darkblue}{\textbf{\ipa{ʈi˥\textsubscript{a}}}}}{}
\textcolor{teal}{\mytextsc{classificateur}} \hspace{4pt} Ton~: H\textsubscript{a}.

Empan: distance entre le pouce et l'index écartés. D'ordinaire, on n'emploie pas la distance entre pouce et majeur.
\textcolor{brown}{\zh{量词:拃(大拇指和食指之间的距离。一般不用大拇指和中指之间的距离。)。}}




\paragraph{\hspace{-0.5cm} {\Large \textcolor{darkblue}{\textbf{\ipa{ʈi˩\textsubscript{a}}}}}} \hypertarget{t`i\string_Ba1}{}
\markboth{\textcolor{darkblue}{\textbf{\ipa{ʈi˩\textsubscript{a}}}}}{}
\textcolor{teal}{\mytextsc{verbe}} \hspace{4pt} Ton~: L\textsubscript{a}.

Se lever.
\textcolor{brown}{\zh{起(如:起来,起床)。}}


\begin{exe}
\sn \textcolor{darkblue}{\textbf{\ipa{gɤ˩-ʈi˧}}}
\trans se lever
\trans \textit{\textcolor{brown}{\zh{起来}}}
\end{exe}
\begin{exe}
\sn \textcolor{darkblue}{\textbf{\ipa{ʑi˧ ʈi˥}}}
\trans se réveiller
\trans \textit{\textcolor{brown}{\zh{醒来}}}
\end{exe}
\begin{exe}
\sn \textcolor{darkblue}{\textbf{\ipa{ʑi˧ gɤ˧-ʈi˩}}}
\trans se réveiller
\trans \textit{\textcolor{brown}{\zh{醒来}}}
\end{exe}
\begin{exe}
\sn \textcolor{darkblue}{\textbf{\ipa{gɤ˩ mɤ˥-ʈi˩}}}
\trans ne pas se lever
\trans \textit{\textcolor{brown}{\zh{不起床}}}
\end{exe}
\begin{exe}
\sn \textcolor{darkblue}{\textbf{\ipa{mɤ˧-ʈi˩-sɯ˩!}}}
\trans (Il/elle) n'est pas encore levé(e)!
\trans \textit{\textcolor{brown}{\zh{还没起床!}}}
\end{exe}
\begin{exe}
\sn \textcolor{darkblue}{\textbf{\ipa{le˧-ʈi˩-ze˩!}}}
\trans (il/elle) s'est levé(e)!
\trans \textit{\textcolor{brown}{\zh{起床了!}}}
\end{exe}
\begin{exe}
\sn \textcolor{darkblue}{\textbf{\ipa{ɖɯ˧-ʈi˧\textasciitilde{}ʈi˥-ɻ̍˩}}}
\trans \mytextsc{délimitatif} \mytextsc{red} \mytextsc{inchoatif}
\trans \textit{\textcolor{brown}{\zh{起来一下}}}
\end{exe}


\paragraph{\hspace{-0.5cm} {\Large \textcolor{darkblue}{\textbf{\ipa{ʈɯ˧\textsubscript{a}}}}}} \hypertarget{t`M\string_Ma1}{}
\markboth{\textcolor{darkblue}{\textbf{\ipa{ʈɯ˧\textsubscript{a}}}}}{}
\textcolor{teal}{\mytextsc{verbe}} \hspace{4pt} Ton~: M\textsubscript{a}.

Mettre en place, installer à sa juste place.
\textcolor{brown}{\zh{安装、摆好。}}


\begin{exe}
\sn \textcolor{darkblue}{\textbf{\ipa{ʂe˧kʰɯ˧ tʰi˧-ʈɯ˧˥, | v̩˧ | tʰi˧-ʈɯ˧}}}
\trans mettre en place le trépied de fer dans le foyer, mettre en place la grande casserole (sur le trépied) (Contexte: description de la ``pendaison de crémaillère", dans une nouvelle maison)
\trans \textit{\textcolor{brown}{\zh{(建完新房后)安装三脚架、把锅摆好(在三脚架上)}}}
\end{exe}
\begin{exe}
\sn \textcolor{darkblue}{\textbf{\ipa{tsʰo˩-ɻ̃˩˥ | dʑɯ˩ mɤ˩-ʈɯ˩˥, | lɑ˧-ʂe˧ | kʰv̩˧ tʰɑ˩-ki˩!}}}
\trans ``Les ossements humains, on ne les met pas à l'eau! La chair du tigre, on ne la donne pas au chien!" Sens: on n'enterre pas les gens dans l'eau (à la différence de certaines coutumes tibétaines); on prenait soin de n'immerger dans l'eau, ni le corps, ni les cendres après la crémation.
\trans \textit{\textcolor{brown}{\zh{“人骨头,莫碰水!老虎肉,莫给狗!”(这个谚语,来强调摩梭与藏族的一些不同习惯:摩梭禁止让尸体或骨灰沾水。)}}}
\end{exe}


\paragraph{\hspace{-0.5cm} {\Large \textcolor{darkblue}{\textbf{\ipa{ʈɯ˧ʈʰæ\#˥}}}}} \hypertarget{t`M\string_Mt`\string_h\{\#\string_T1}{}
\markboth{\textcolor{darkblue}{\textbf{\ipa{ʈɯ˧ʈʰæ\#˥}}}}{}
\textcolor{teal}{\mytextsc{nom}} \hspace{4pt} Ton~: \#H.

Patrimoine.
\textcolor{brown}{\zh{家底、财产(贵重物品)。}}


\begin{exe}
\sn \textcolor{darkblue}{\textbf{\ipa{ɑ˩ʁo˧ ʈɯ˧ʈʰæ˧!}}}
\trans Prospérité à la famille!
\trans \textit{\textcolor{brown}{\zh{祝你们家发财!}}}
\end{exe}
\begin{exe}
\sn \textcolor{darkblue}{\textbf{\ipa{ʈʂʰɯ˧ | ʈɯ˧ʈʰæ˧ | ɖwæ˧˥ | dʑo˧-ʝi˧!}}}
\trans Il/elle est riche! / Sa famille est riche!
\trans \textit{\textcolor{brown}{\zh{他家底很好! / 他家有钱!}}}
\end{exe}
 \mytextsc{clf}~: \textcolor{darkblue}{\textbf{\ipa{kʰwɤ˥}}} 

\paragraph{\hspace{-0.5cm} {\Large \textcolor{darkblue}{\textbf{\ipa{ʈɯ˧˥}}}}} \hypertarget{t`M\string_M\string_T1}{}
\markboth{\textcolor{darkblue}{\textbf{\ipa{ʈɯ˧˥}}}}{}
\textcolor{teal}{\mytextsc{verbe}} \hspace{4pt} Ton~: MH.

Blanchir à l'eau bouillante: du lin pour préparer du fil pour le tissage, des légumes séchés avant de les utiliser pour la cuisine...
\textcolor{brown}{\zh{以滚水将蔬菜或亚麻灼过。}}


\begin{exe}
\sn \textcolor{darkblue}{\textbf{\ipa{tʰi˧-ʈɯ˧˥}}}
\trans \mytextsc{dur}
\trans \textit{\textcolor{brown}{\zh{\mytextsc{dur}}}}
\end{exe}
\begin{exe}
\sn \textcolor{darkblue}{\textbf{\ipa{dʑɯ˩tsʰi˩-qo˥ | tʰi˧-ʈɯ˧˥ / dʑɯ˩tsʰi˩-qo˥ | ʈɯ˧˥}}}
\trans blanchir à l'eau bouillante
\trans \textit{\textcolor{brown}{\zh{以滚水灼过}}}
\end{exe}
\begin{exe}
\sn \textcolor{darkblue}{\textbf{\ipa{dʑɯ˧-qo˧ | ʈɯ˧˥}}}
\trans blanchir à l'eau
\trans \textit{\textcolor{brown}{\zh{以水灼过}}}
\end{exe}
\begin{exe}
\sn \textcolor{darkblue}{\textbf{\ipa{v˩tsʰɤ˧ ʈɯ˥}}}
\trans blanchir des légumes
\trans \textit{\textcolor{brown}{\zh{灼蔬菜}}}
\end{exe}
\begin{exe}
\sn \textcolor{darkblue}{\textbf{\ipa{sɑ˧, | ʈɯ˧-kv˥!}}}
\trans Le chanvre, ça se blanchit! (Au cours de la préparation du chanvre pour en faire du fil, il faut le blanchir.)
\trans \textit{\textcolor{brown}{\zh{亚麻,要灼过!}}}
\end{exe}


\paragraph{\hspace{-0.5cm} {\Large \textcolor{darkblue}{\textbf{\ipa{ʈv̩˩}}}}} \hypertarget{t`v\string_=\string_B1}{}
\markboth{\textcolor{darkblue}{\textbf{\ipa{ʈv̩˩}}}}{}
\textcolor{teal}{\mytextsc{nom}} \hspace{4pt} Ton~: L.

Noeud.
\textcolor{brown}{\zh{死扣、死结。}}


\begin{exe}
\sn \textcolor{darkblue}{\textbf{\ipa{ɖɯ˧-ʈv̩˩}}}
\trans un noeud
\trans \textit{\textcolor{brown}{\zh{一个死结}}}
\end{exe}
\begin{exe}
\sn \textcolor{darkblue}{\textbf{\ipa{ɖɯ˧-ʈv̩˩ | tʰi˧-ʈv̩˩}}}
\trans faire un noeud
\trans \textit{\textcolor{brown}{\zh{打一个死结}}}
\end{exe}


\paragraph{\hspace{-0.5cm} {\Large \textcolor{darkblue}{\textbf{\ipa{ʈv̩˩\textsubscript{a}}}} \textsubscript{1}}} \hypertarget{t`v\string_=\string_Ba1}{}
\markboth{\textcolor{darkblue}{\textbf{\ipa{ʈv̩˩\textsubscript{a}}}} \textsubscript{1}}{}
\textcolor{teal}{\mytextsc{verbe}} \hspace{4pt} Ton~: L\textsubscript{a}.

Tresser (vannerie).
\textcolor{brown}{\zh{编(竹子)。}}


\begin{exe}
\sn \textcolor{darkblue}{\textbf{\ipa{qʰwɤ˧tʰv̩˧ ʈv̩˥}}}
\trans tresser une hotte dorsale (en bambou) pour porter l'eau
\trans \textit{\textcolor{brown}{\zh{编背水的背篓}}}
\end{exe}
\begin{exe}
\sn \textcolor{darkblue}{\textbf{\ipa{mi˩ɬi˩ ʈv̩˥}}}
\trans tresser du bambou
\trans \textit{\textcolor{brown}{\zh{编竹子}}}
\end{exe}
\begin{exe}
\sn \textcolor{darkblue}{\textbf{\ipa{tso˧\textasciitilde{}tso˧ ʈv̩˥}}}
\trans tresser des choses
\trans \textit{\textcolor{brown}{\zh{编东西}}}
\end{exe}


\paragraph{\hspace{-0.5cm} {\Large \textcolor{darkblue}{\textbf{\ipa{ʈv̩˩\textsubscript{a}}}} \textsubscript{2}}} \hypertarget{t`v\string_=\string_Ba2}{}
\markboth{\textcolor{darkblue}{\textbf{\ipa{ʈv̩˩\textsubscript{a}}}} \textsubscript{2}}{}
\textcolor{teal}{\mytextsc{verbe}} \hspace{4pt} Ton~: L\textsubscript{a}.

Lancer (une pierre sur quelqu'un).
\textcolor{brown}{\zh{掷(掷石头)。}}


\begin{exe}
\sn \textcolor{darkblue}{\textbf{\ipa{mɤ˧-ʈv̩˩}}}
\trans \mytextsc{neg}
\trans \textit{\textcolor{brown}{\zh{\mytextsc{neg}}}}
\end{exe}
\begin{exe}
\sn \textcolor{darkblue}{\textbf{\ipa{lv̩˧mi˧ ʈv̩˩}}}
\trans lancer une pierre
\trans \textit{\textcolor{brown}{\zh{掷石头}}}
\end{exe}
\begin{exe}
\sn \textcolor{darkblue}{\textbf{\ipa{tso˧\textasciitilde{}tso˧ ʈv̩˥}}}
\trans jeter des choses
\trans \textit{\textcolor{brown}{\zh{掷东西}}}
\end{exe}


\paragraph{\hspace{-0.5cm} {\Large \textcolor{darkblue}{\textbf{\ipa{ʈv̩˩\textsubscript{b}}}}}} \hypertarget{t`v\string_=\string_Bb1}{}
\markboth{\textcolor{darkblue}{\textbf{\ipa{ʈv̩˩\textsubscript{b}}}}}{}
\textcolor{teal}{\mytextsc{classificateur}} \hspace{4pt} Ton~: L\textsubscript{b}.

Classificateur des quartiers/pièces: morceaux de taille supérieure à une bouchée. Il peut s'agir d'un morceau de viande qu'on donne à un convive, et qui se mange en plusieurs bouchées, mais aussi d'un gros quartier de viande (plusieurs kilos).
\textcolor{brown}{\zh{量词:大块,如:一块肉,从一个人的份到几公斤的重量。}}




\paragraph{\hspace{-0.5cm} {\Large \textcolor{darkblue}{\textbf{\ipa{ʈv̩˩qʰv̩˩}}}}} \hypertarget{t`v\string_=\string_Bq\string_hv\string_=\string_B1}{}
\markboth{\textcolor{darkblue}{\textbf{\ipa{ʈv̩˩qʰv̩˩}}}}{}
\textcolor{teal}{\mytextsc{nom}} \hspace{4pt} Ton~: L.

Noeud coulant.
\textcolor{brown}{\zh{活扣。}}


\begin{exe}
\sn \textcolor{darkblue}{\textbf{\ipa{ʈv̩˩qʰv̩˩˥ | tʰi˧-ʈv̩˩}}}
\trans faire un noeud coulant
\trans \textit{\textcolor{brown}{\zh{打活扣}}}
\end{exe}
\begin{exe}
\sn \textcolor{darkblue}{\textbf{\ipa{ʈv̩˩qʰv̩˩˥ | ɖɯ˧-ɭɯ˧ | tʰi˧-ʈv̩˩}}}
\trans faire un noeud coulant
\trans \textit{\textcolor{brown}{\zh{打一个活扣}}}
\end{exe}


\paragraph{\hspace{-0.5cm} {\Large \textcolor{darkblue}{\textbf{\ipa{ʈwæ˩\textsubscript{a}}}}}} \hypertarget{t`w\{\string_Ba1}{}
\markboth{\textcolor{darkblue}{\textbf{\ipa{ʈwæ˩\textsubscript{a}}}}}{}
\textcolor{teal}{\mytextsc{verbe}} \hspace{4pt} Ton~: L\textsubscript{a}.

Geler, se figer.
\textcolor{brown}{\zh{冻。}}


\begin{exe}
\sn \textcolor{darkblue}{\textbf{\ipa{dʑɯ˩ ʈwæ˩˥}}}
\trans l'eau gèle, il gèle
\trans \textit{\textcolor{brown}{\zh{水冻成冰}}}
\end{exe}
\begin{exe}
\sn \textcolor{darkblue}{\textbf{\ipa{dʑɯ˩pʰæ˩ ʈwæ˧-ze˩}}}
\trans l'eau a gelé, de la glace s'est formée
\trans \textit{\textcolor{brown}{\zh{水冻成冰了。}}}
\end{exe}
\begin{exe}
\sn \textcolor{darkblue}{\textbf{\ipa{ɖɯ˧-ʈwæ˧\textasciitilde{}ʈwæ˥-ɻ̍˩ kʰɯ˩}}}
\trans faire geler, mettre à congeler
\trans \textit{\textcolor{brown}{\zh{冷冻,放在冷箱}}}
\end{exe}


\paragraph{\hspace{-0.5cm} {\Large \textcolor{darkblue}{\textbf{\ipa{ʈwæ˧˥}}}}} \hypertarget{t`w\{\string_M\string_T1}{}
\markboth{\textcolor{darkblue}{\textbf{\ipa{ʈwæ˧˥}}}}{}
\textcolor{teal}{\mytextsc{verbe}} \hspace{4pt} Ton~: MH.

Tomber (en glissant).
\textcolor{brown}{\zh{跌倒(路很滑)。}}




\paragraph{\hspace{-0.5cm} {\Large \textcolor{darkblue}{\textbf{\ipa{ʈwɤ˧\textsubscript{a}}}}}} \hypertarget{t`w7\string_Ma1}{}
\markboth{\textcolor{darkblue}{\textbf{\ipa{ʈwɤ˧\textsubscript{a}}}}}{}
\textcolor{teal}{\mytextsc{verbe}} \hspace{4pt} Ton~: M\textsubscript{a}.

Le coq chante/fait cocorico; un oiseau chante.
\textcolor{brown}{\zh{啼,鸡叫。}}


\begin{exe}
\sn \textcolor{darkblue}{\textbf{\ipa{æ̃˩ ʈwɤ˧ (+ze˧)}}}
\trans Le coq chante/fait cocorico.
\trans \textit{\textcolor{brown}{\zh{鸡叫。}}}
\end{exe}


\newpage
\section*{\centering- \textcolor{darkblue}{\textbf{\ipa{ʈʰ}}} -}
\paragraph{\hspace{-0.5cm} {\Large \textcolor{darkblue}{\textbf{\ipa{ʈʰæ˧-mɤ˧-ʝi\#˥}}}}} \hypertarget{t`\string_h\{\string_M-m7\string_M-j££i\#\string_T1}{}
\markboth{\textcolor{darkblue}{\textbf{\ipa{ʈʰæ˧-mɤ˧-ʝi\#˥}}}}{}
\textcolor{teal}{\mytextsc{adjectif}} \hspace{4pt} Ton~: \#H.

Désordonné.
\textcolor{brown}{\zh{乱。}}


\begin{exe}
\sn \textcolor{darkblue}{\textbf{\ipa{ɑ˩ʁo˧ | ʈʰæ˧-mɤ˧-ʝi˧! |}}}
\trans la maison est en grand désordre!
\trans \textit{\textcolor{brown}{\zh{家很乱!}}}
\end{exe}
\begin{exe}
\sn \textcolor{darkblue}{\textbf{\ipa{ʈʰæ˧-mɤ˧-ʝi˧ ɲi˥! |}}}
\trans C'est vraiment le désordre!
\trans \textit{\textcolor{brown}{\zh{真乱!}}}
\end{exe}


\paragraph{\hspace{-0.5cm} {\Large \textcolor{darkblue}{\textbf{\ipa{ʈʰæ˧mi˧-ɳɯ˩}}}}} \hypertarget{t`\string_h\{\string_Mmi\string_M-n`M\string_B1}{}
\markboth{\textcolor{darkblue}{\textbf{\ipa{ʈʰæ˧mi˧-ɳɯ˩}}}}{}
\textcolor{teal}{\mytextsc{adverbe}} \hspace{4pt} Ton~: L\#.

Vraiment, réellement.
\textcolor{brown}{\zh{真的。}}


\begin{exe}
\sn \textcolor{darkblue}{\textbf{\ipa{ʈʂʰɯ˧ | ʈʰæ˧mi˧-ɳɯ˩ | go˩˥!}}}
\trans Il/elle est vraiment malade!
\trans \textit{\textcolor{brown}{\zh{他真的病了!}}}
\end{exe}


\paragraph{\hspace{-0.5cm} {\Large \textcolor{darkblue}{\textbf{\ipa{-ʈʰæ˧qo˩}}}}} \hypertarget{-t`\string_h\{\string_Mqo\string_B1}{}
\markboth{\textcolor{darkblue}{\textbf{\ipa{-ʈʰæ˧qo˩}}}}{}
\textcolor{teal}{\mytextsc{postposition}} \hspace{4pt} Ton~: L\#.

Sous (sous le ciel; sous la tente); au pied (d'une montagne).
\textcolor{brown}{\zh{……之下、下面。}}




\paragraph{\hspace{-0.5cm} {\Large \textcolor{darkblue}{\textbf{\ipa{ʈʰæ˧qʰwɤ˧}}}}} \hypertarget{t`\string_h\{\string_Mq\string_hw7\string_M1}{}
\markboth{\textcolor{darkblue}{\textbf{\ipa{ʈʰæ˧qʰwɤ˧}}}}{}
\textcolor{teal}{\mytextsc{nom}} \hspace{4pt} Ton~: M.

Jupe.
\textcolor{brown}{\zh{裙子。}}


 \mytextsc{clf}~: \textcolor{darkblue}{\textbf{\ipa{ɭɯ˧˥}}} 

\paragraph{\hspace{-0.5cm} {\Large \textcolor{darkblue}{\textbf{\ipa{*ʈʰæ˩}}}}} \hypertarget{*t`\string_h\{\string_B1}{}
\markboth{\textcolor{darkblue}{\textbf{\ipa{*ʈʰæ˩}}}}{}
\textcolor{teal}{\mytextsc{nom}} \hspace{4pt} Ton~: L.

Jupe; monosyllabe extrait d'après le comportement dans l'expression figée /ʈʰæ˩ ki˩˥/ 'enfiler la jupe', avec verbe au ton La (nom du rituel de passage à l'âge adulte). Le monosyllabe n'est pas usité hors de cette expression. Par exemple, */ʈʰæ˩ ɲi˩˥/ 'c'est une jupe' est catégoriquement refusé par F4.
\textcolor{brown}{\zh{裙子(单音节)。}}


\begin{exe}
\sn \textcolor{darkblue}{\textbf{\ipa{ʈʰæ˧ | le˧-ki˩}}}
\trans enfiler la jupe (\mytextsc{accomp})
\trans \textit{\textcolor{brown}{\zh{穿上裙子}}}
\end{exe}
\begin{exe}
\sn \textcolor{darkblue}{\textbf{\ipa{ʈʰæ˩ ki˩˥}}}
\trans rituel de passage à l'âge adulte, pour les jeunes femmes: ``enfiler la jupe"
\trans \textit{\textcolor{brown}{\zh{女人成年的礼仪:“穿裙子”}}}
\end{exe}


\paragraph{\hspace{-0.5cm} {\Large \textcolor{darkblue}{\textbf{\ipa{ʈʰæ˩ki˩}}}}} \hypertarget{t`\string_h\{\string_Bki\string_B1}{}
\markboth{\textcolor{darkblue}{\textbf{\ipa{ʈʰæ˩ki˩}}}}{}
\textcolor{teal}{\mytextsc{verbe}} \hspace{4pt} Ton~: L.

Réaliser la cérémonie de passage à l'âge adulte des femmes.
\textcolor{brown}{\zh{举行女孩的成年礼。}}


\begin{exe}
\sn \textcolor{darkblue}{\textbf{\ipa{ʈʰæ˩ki˩-ze˥!}}}
\trans Elle est adulte maintenant! / La cérémonie de passage à l'âge adulte a été réalisée!
\trans \textit{\textcolor{brown}{\zh{穿裙了! / 行过穿裙礼了! / 她成年了!}}}
\end{exe}


\paragraph{\hspace{-0.5cm} {\Large \textcolor{darkblue}{\textbf{\ipa{ʈʰæ˩\textasciitilde{}ʈʰæ˧˥}}}}} \hypertarget{t`\string_h\{\string_B~t`\string_h\{\string_M\string_T1}{}
\markboth{\textcolor{darkblue}{\textbf{\ipa{ʈʰæ˩\textasciitilde{}ʈʰæ˧˥}}}}{}
\textcolor{teal}{\mytextsc{verbe}} \hspace{4pt} Ton~: MH.

Démanger.
\textcolor{brown}{\zh{痒。}}


\begin{exe}
\sn \textcolor{darkblue}{\textbf{\ipa{le˧-ʈʰæ˩\textasciitilde{}ʈʰæ˩-ze˩}}}
\trans \mytextsc{accomp} \mytextsc{red} \mytextsc{pfv}
\trans \textit{\textcolor{brown}{\zh{\mytextsc{accomp} \mytextsc{red} \mytextsc{pfv}}}}
\end{exe}


\paragraph{\hspace{-0.5cm} {\Large \textcolor{darkblue}{\textbf{\ipa{ʈʰæ˧˥}}}}} \hypertarget{t`\string_h\{\string_M\string_T1}{}
\markboth{\textcolor{darkblue}{\textbf{\ipa{ʈʰæ˧˥}}}}{}
\textcolor{teal}{\mytextsc{verbe}} \hspace{4pt} Ton~: MH.
\ding{202} 
Mordre (mordre à belles dents dans quelque chose); piquer (une abeille pique quelqu'un).
\textcolor{brown}{\zh{咬、叮。}}


\begin{exe}
\sn \textcolor{darkblue}{\textbf{\ipa{tso˧\textasciitilde{}tso˧ ʈʰæ˩(-ze˩)}}}
\trans mordre quelque chose
\trans \textit{\textcolor{brown}{\zh{咬东西}}}
\end{exe}
\begin{exe}
\sn \textcolor{darkblue}{\textbf{\ipa{hĩ˧ ʈʰæ˩}}}
\trans mordre quelqu'un (ex.: un chien mord un inconnu de passage)
\trans \textit{\textcolor{brown}{\zh{咬人}}}
\end{exe}
\ding{203} 
S'emboîter, s'ajuster (au sujet de pièces de charpenterie); emploi figuré de ``mordre": les pièces s'ajustent comme si elles mordaient les unes dans les autres.
\textcolor{brown}{\zh{对号、合适、相配:建房时,两块木材调剂地刚好合适,好像互相“咬紧”的样子。}}




\paragraph{\hspace{-0.5cm} {\Large \textcolor{darkblue}{\textbf{\ipa{ʈʰɤ˥\textsubscript{a}}}}}} \hypertarget{t`\string_h7\string_Ta1}{}
\markboth{\textcolor{darkblue}{\textbf{\ipa{ʈʰɤ˥\textsubscript{a}}}}}{}
\textcolor{teal}{\mytextsc{classificateur}} \hspace{4pt} Ton~: H\textsubscript{a}.

Goutte (une goutte de liquide).
\textcolor{brown}{\zh{量词:滴。}}




\paragraph{\hspace{-0.5cm} {\Large \textcolor{darkblue}{\textbf{\ipa{ʈʰɤ˧˥}}}}} \hypertarget{t`\string_h7\string_M\string_T1}{}
\markboth{\textcolor{darkblue}{\textbf{\ipa{ʈʰɤ˧˥}}}}{}
\textcolor{teal}{\mytextsc{verbe}} \hspace{4pt} Ton~: MH.

Goutter, dégouliner, couler goutte à goutte.
\textcolor{brown}{\zh{滴(水往下滴)。}}


\begin{exe}
\sn \textcolor{darkblue}{\textbf{\ipa{tʰi˧-ʈʰɤ˩\textasciitilde{}ʈʰɤ˩}}}
\trans \mytextsc{dur} \mytextsc{red}
\trans \textit{\textcolor{brown}{\zh{滴着滴着}}}
\end{exe}


\paragraph{\hspace{-0.5cm} {\Large \textcolor{darkblue}{\textbf{\ipa{ʈʰi˩\textsubscript{a}}}}}} \hypertarget{t`\string_hi\string_Ba1}{}
\markboth{\textcolor{darkblue}{\textbf{\ipa{ʈʰi˩\textsubscript{a}}}}}{}
\textcolor{teal}{\mytextsc{adjectif}} \hspace{4pt} Ton~: L\textsubscript{a}.

Fatigué.
\textcolor{brown}{\zh{累、疲倦、精疲力竭。}}


\begin{exe}
\sn \textcolor{darkblue}{\textbf{\ipa{le˧-ʈʰi˩-ze˩}}}
\trans \mytextsc{accomp} \string_ \mytextsc{pfv}
\trans \textit{\textcolor{brown}{\zh{累了}}}
\end{exe}
\begin{exe}
\sn \textcolor{darkblue}{\textbf{\ipa{njɤ˧ | ʈʰi˩˥!}}}
\trans je suis fatigué!
\trans \textit{\textcolor{brown}{\zh{我累了!}}}
\end{exe}
\begin{exe}
\sn \textcolor{darkblue}{\textbf{\ipa{njɤ˧ | ʈʰi˩-ze˥!}}}
\trans je suis fatigué!
\trans \textit{\textcolor{brown}{\zh{我累了!}}}
\end{exe}


\paragraph{\hspace{-0.5cm} {\Large \textcolor{darkblue}{\textbf{\ipa{ʈʰɯ˩\textsubscript{a}}}}}} \hypertarget{t`\string_hM\string_Ba1}{}
\markboth{\textcolor{darkblue}{\textbf{\ipa{ʈʰɯ˩\textsubscript{a}}}}}{}
\textcolor{teal}{\mytextsc{verbe}} \hspace{4pt} Ton~: L\textsubscript{a}.

Éternuer.
\textcolor{brown}{\zh{打喷嚏。}}


\begin{exe}
\sn \textcolor{darkblue}{\textbf{\ipa{ɖɯ˧-ʈʰɯ˧\textasciitilde{}ʈʰɯ˥}}}
\trans \mytextsc{inchoatif} \mytextsc{red}
\end{exe}


\paragraph{\hspace{-0.5cm} {\Large \textcolor{darkblue}{\textbf{\ipa{ʈʰɯ˩\textsubscript{b}}}}}} \hypertarget{t`\string_hM\string_Bb1}{}
\markboth{\textcolor{darkblue}{\textbf{\ipa{ʈʰɯ˩\textsubscript{b}}}}}{}
\textcolor{teal}{\mytextsc{verbe}} \hspace{4pt} Ton~: L\textsubscript{b}.

Boire.
\textcolor{brown}{\zh{喝。}}


\begin{exe}
\sn \textcolor{darkblue}{\textbf{\ipa{njɤ˧ | mɤ˧-ʈʰɯ˩}}}
\trans je ne bois pas
\trans \textit{\textcolor{brown}{\zh{我不喝}}}
\end{exe}
\begin{exe}
\sn \textcolor{darkblue}{\textbf{\ipa{ʈʰɯ˩-ze˥}}}
\trans \mytextsc{pfv}
\trans \textit{\textcolor{brown}{\zh{喝了}}}
\end{exe}
\begin{exe}
\sn \textcolor{darkblue}{\textbf{\ipa{le˧-ʈʰɯ˩-ze˩}}}
\trans \mytextsc{accomp} \string_ \mytextsc{pfv}
\trans \textit{\textcolor{brown}{\zh{\mytextsc{accomp} \string_ \mytextsc{pfv}}}}
\end{exe}
\begin{exe}
\sn \textcolor{darkblue}{\textbf{\ipa{ʐɯ˧ ʈʰɯ˩}}}
\trans boire du vin
\trans \textit{\textcolor{brown}{\zh{喝酒}}}
\end{exe}
\begin{exe}
\sn \textcolor{darkblue}{\textbf{\ipa{jɤ˧ ʈʰɯ˩}}}
\trans fumer (du tabac)
\trans \textit{\textcolor{brown}{\zh{抽烟}}}
\end{exe}
\begin{exe}
\sn \textcolor{darkblue}{\textbf{\ipa{dʑɯ˩qʰæ˩ ʈʰɯ˩˥}}}
\trans boire de l'eau froide
\trans \textit{\textcolor{brown}{\zh{喝凉水}}}
\end{exe}
\begin{exe}
\sn \textcolor{darkblue}{\textbf{\ipa{dʑɯ˩tsʰi˩ ʈʰɯ˩˥}}}
\trans boire de l'eau chaude
\trans \textit{\textcolor{brown}{\zh{喝热水}}}
\end{exe}
\begin{exe}
\sn \textcolor{darkblue}{\textbf{\ipa{li˩ ʈʰɯ˩}}}
\trans boire du thé
\trans \textit{\textcolor{brown}{\zh{喝茶}}}
\end{exe}
\begin{exe}
\sn \textcolor{darkblue}{\textbf{\ipa{v̩˩dʑɯ˩ ʈʰɯ˩˥}}}
\trans boire de la soupe
\trans \textit{\textcolor{brown}{\zh{喝汤}}}
\end{exe}
\begin{exe}
\sn \textcolor{darkblue}{\textbf{\ipa{dʑɯ˧ ʈʰɯ˧}}}
\trans boire de l'eau
\trans \textit{\textcolor{brown}{\zh{喝水}}}
\end{exe}
\begin{exe}
\sn \textcolor{darkblue}{\textbf{\ipa{njɤ˧ | dʑɯ˧ ʈʰɯ˧-ze˧}}}
\trans j'ai bu de l'eau
\trans \textit{\textcolor{brown}{\zh{我喝了水}}}
\end{exe}
\begin{exe}
\sn \textcolor{darkblue}{\textbf{\ipa{njɤ˧ | dʑɯ˧ ʈʰɯ˧-zo˧-ho˩}}}
\trans Il va falloir que je boive de l'eau.
\trans \textit{\textcolor{brown}{\zh{我应该喝水了。}}}
\end{exe}


\newpage
\section*{\centering- \textcolor{darkblue}{\textbf{\ipa{ʈʂ}}} -}
\paragraph{\hspace{-0.5cm} {\Large \textcolor{darkblue}{\textbf{\ipa{ʈʂɑ˧tɑ˥}}}}} \hypertarget{t`s`A\string_MtA\string_T1}{}
\markboth{\textcolor{darkblue}{\textbf{\ipa{ʈʂɑ˧tɑ˥}}}}{}
\textcolor{teal}{\mytextsc{nom}} \hspace{4pt} Ton~: H\#.

Signe.
\textcolor{brown}{\zh{记号。}}


\begin{exe}
\sn \textcolor{darkblue}{\textbf{\ipa{ʈʂɑ˧tɑ˥ ʝi˩}}}
\trans faire une marque, inscrire un signe
\trans \textit{\textcolor{brown}{\zh{写一个符号、画一个符号}}}
\end{exe}
\begin{exe}
\sn \textcolor{darkblue}{\textbf{\ipa{ʈʂɑ˧tɑ˥ tɕi˩}}}
\trans écrire des signes, faire des marques (pas pour un unique signe: se rapproche de l'écriture d'un message/texte)
\trans \textit{\textcolor{brown}{\zh{写符号、画符号}}}
\end{exe}
 \mytextsc{clf}~: \textcolor{darkblue}{\textbf{\ipa{kʰwɤ˥}}} 

\paragraph{\hspace{-0.5cm} {\Large \textcolor{darkblue}{\textbf{\ipa{ʈʂæ˧mo\#˥}}}}} \hypertarget{t`s`\{\string_Mmo\#\string_T1}{}
\markboth{\textcolor{darkblue}{\textbf{\ipa{ʈʂæ˧mo\#˥}}}}{}
\textcolor{teal}{\mytextsc{nom}} \hspace{4pt} Ton~: \#H.

Un champignon vénéneux.
\textcolor{brown}{\zh{一种有毒的菌子。}}


\begin{exe}
\sn \textcolor{darkblue}{\textbf{\ipa{ʈʂæ˧mo˧-kʰi˧tɕʰɯ˩-mo˩ / kʰi˧tɕʰɯ˩-mo˩}}}
\trans même sens
\trans \textit{\textcolor{brown}{\zh{同上}}}
\end{exe}
\textit{Syn~:} \hyperlink{k\string_hi\string_Mts£\string_hM\string_B-mo\string_B1}{\textcolor{darkblue}{\textbf{\ipa{kʰi˧tɕʰɯ˩-mo˩}}}}. 

\paragraph{\hspace{-0.5cm} {\Large \textcolor{darkblue}{\textbf{\ipa{ʈʂæ˧ʈʂɯ˧}}}}} \hypertarget{t`s`\{\string_Mt`s`M\string_M1}{}
\markboth{\textcolor{darkblue}{\textbf{\ipa{ʈʂæ˧ʈʂɯ˧}}}}{}
\textcolor{teal}{\mytextsc{adverbe}} \hspace{4pt} Ton~: M.

Véritablement, vraiment, pour de vrai.
\textcolor{brown}{\zh{确切、真的。}}




\paragraph{\hspace{-0.5cm} {\Large \textcolor{darkblue}{\textbf{\ipa{ʈʂæ˧wɤ˩}}}}} \hypertarget{t`s`\{\string_Mw7\string_B1}{}
\markboth{\textcolor{darkblue}{\textbf{\ipa{ʈʂæ˧wɤ˩}}}}{}
\textcolor{teal}{\mytextsc{nom}} \hspace{4pt} Ton~: L\#.

Serviteur.
\textcolor{brown}{\zh{仆人,佣人。}}


 \mytextsc{clf}~: \textcolor{darkblue}{\textbf{\ipa{v̩˧}}} 

\paragraph{\hspace{-0.5cm} {\Large \textcolor{darkblue}{\textbf{\ipa{ʈʂæ˩do\#˥}}}}} \hypertarget{t`s`\{\string_Bdo\#\string_T1}{}
\markboth{\textcolor{darkblue}{\textbf{\ipa{ʈʂæ˩do\#˥}}}}{}
\textcolor{teal}{\mytextsc{nom}} \hspace{4pt} Ton~: LM+\#H.

Récipient dans lequel on bat le thé au beurre (tube-baratte en bois); aussi: grande baratte pour baratter le beurre.
\textcolor{brown}{\zh{打酥油茶的罐、酥油茶搅拌器,黄油搅乳器。}}


 \mytextsc{clf}~: \textcolor{darkblue}{\textbf{\ipa{ɭɯ˧}}} 

\paragraph{\hspace{-0.5cm} {\Large \textcolor{darkblue}{\textbf{\ipa{ʈʂæ˧˥}}} \textsubscript{1}}} \hypertarget{t`s`\{\string_M\string_T1}{}
\markboth{\textcolor{darkblue}{\textbf{\ipa{ʈʂæ˧˥}}} \textsubscript{1}}{}
\textcolor{teal}{\mytextsc{verbe}} \hspace{4pt} Ton~: MH.

Voler, s'emparer de, extorquer, arracher.
\textcolor{brown}{\zh{抢劫、抢。}}


\begin{exe}
\sn \textcolor{darkblue}{\textbf{\ipa{le˧-ʈʂæ˧-ze˥}}}
\trans \mytextsc{accomp} \string_ \mytextsc{pfv}
\trans \textit{\textcolor{brown}{\zh{抢了}}}
\end{exe}
\begin{exe}
\sn \textcolor{darkblue}{\textbf{\ipa{tso˧\textasciitilde{}tso˧ ʈʂæ˩}}}
\trans voler des choses
\trans \textit{\textcolor{brown}{\zh{抢东西}}}
\end{exe}
\begin{exe}
\sn \textcolor{darkblue}{\textbf{\ipa{le˧-ʈʂæ˧-po˥-hɯ˩(-ze˩)}}}
\trans (il) a extorqué quelque chose et est parti avec
\trans \textit{\textcolor{brown}{\zh{把东西抢走了}}}
\end{exe}
\begin{exe}
\sn \textcolor{darkblue}{\textbf{\ipa{hĩ˧ ʈʂæ˩}}}
\trans voler les gens, extorquer des choses aux gens
\trans \textit{\textcolor{brown}{\zh{抢劫}}}
\end{exe}


\paragraph{\hspace{-0.5cm} {\Large \textcolor{darkblue}{\textbf{\ipa{ʈʂæ˧˥}}} \textsubscript{2}}} \hypertarget{t`s`\{\string_M\string_T2}{}
\markboth{\textcolor{darkblue}{\textbf{\ipa{ʈʂæ˧˥}}} \textsubscript{2}}{}
\textcolor{teal}{\mytextsc{verbe}} \hspace{4pt} Ton~: MH.

Envoyer qqun.
\textcolor{brown}{\zh{派人。}}


\begin{exe}
\sn \textcolor{darkblue}{\textbf{\ipa{ɖɯ˧-v̩˧ ʈʂæ˧˥}}}
\trans envoyer quelqu'un
\trans \textit{\textcolor{brown}{\zh{派一个人}}}
\end{exe}
\begin{exe}
\sn \textcolor{darkblue}{\textbf{\ipa{hĩ˧ ʈʂæ˩}}}
\trans idem
\trans \textit{\textcolor{brown}{\zh{同上}}}
\end{exe}


\paragraph{\hspace{-0.5cm} {\Large \textcolor{darkblue}{\textbf{\ipa{ʈʂæ˧˥}}} \textsubscript{3}}} \hypertarget{t`s`\{\string_M\string_T3}{}
\markboth{\textcolor{darkblue}{\textbf{\ipa{ʈʂæ˧˥}}} \textsubscript{3}}{}
\textcolor{teal}{\mytextsc{verbe}} \hspace{4pt} Ton~: MH.

Fixer, accrocher (ex.: coudre un bouton sur un vêtement; attacher la selle sur un cheval).
\textcolor{brown}{\zh{安上(如:缝扣子、安上马鞍)。}}


\begin{exe}
\sn \textcolor{darkblue}{\textbf{\ipa{pv̩˩ɭɯ˥ ʈʂæ˩}}}
\trans coudre un bouton (sur un vêtement)
\trans \textit{\textcolor{brown}{\zh{缝扣子}}}
\end{exe}
\begin{exe}
\sn \textcolor{darkblue}{\textbf{\ipa{ʐwæ˧tɕi˥ ʈʂæ˩}}}
\trans attacher la selle d'un cheval; seller un cheval
\trans \textit{\textcolor{brown}{\zh{备鞍}}}
\end{exe}
\begin{exe}
\sn \textcolor{darkblue}{\textbf{\ipa{ɖɯ˧-ɲi˥, | so˧-ʂɯ˧ ʈʂæ˧˥!}}}
\trans Au cours d'une journée, on selle trois fois (les chevaux, lorsqu'on est parti en caravane)!
\trans \textit{\textcolor{brown}{\zh{(走马帮,)一天备鞍三次!}}}
\end{exe}


\paragraph{\hspace{-0.5cm} {\Large \textcolor{darkblue}{\textbf{\ipa{ʈʂæ˧˥}}} \textsubscript{4}}} \hypertarget{t`s`\{\string_M\string_T4}{}
\markboth{\textcolor{darkblue}{\textbf{\ipa{ʈʂæ˧˥}}} \textsubscript{4}}{}
\textcolor{teal}{\mytextsc{nom}} \hspace{4pt} Ton~: MH.
\ding{202} 
Articulation.
\textcolor{brown}{\zh{关节。}}


\ding{203} 
Période, époque; segment de temps.
\textcolor{brown}{\zh{段(时间)、时代。}}


 \mytextsc{clf}~: \textcolor{darkblue}{\textbf{\ipa{ʈʂæ˧˥}}} 

\paragraph{\hspace{-0.5cm} {\Large \textcolor{darkblue}{\textbf{\ipa{ʈʂæ˧˥\textsubscript{a}}}}}} \hypertarget{t`s`\{\string_M\string_Ta1}{}
\markboth{\textcolor{darkblue}{\textbf{\ipa{ʈʂæ˧˥\textsubscript{a}}}}}{}
\textcolor{teal}{\mytextsc{classificateur}} \hspace{4pt} Ton~: MH\textsubscript{a}.

Classificateur des épis de maïs (mûrs).
\textcolor{brown}{\zh{量词.玉米(一棒)。}}


\begin{exe}
\sn \textcolor{darkblue}{\textbf{\ipa{qʰɑ˧dze˧ | ɖɯ˧-ʈʂæ˧˥}}}
\trans un épi de maïs
\trans \textit{\textcolor{brown}{\zh{一棒玉米}}}
\end{exe}
\begin{exe}
\sn \textcolor{darkblue}{\textbf{\ipa{qʰɑ˧dze˧ | ɖɯ˧-ʈʂæ˧ ɖʐɤ˥}}}
\trans arracher un épi de maïs, récolter un épi de maïs
\trans \textit{\textcolor{brown}{\zh{掰一棒玉米}}}
\end{exe}


\paragraph{\hspace{-0.5cm} {\Large \textcolor{darkblue}{\textbf{\ipa{ʈʂe˥}}} \textsubscript{1}}} \hypertarget{t`s`e\string_T1}{}
\markboth{\textcolor{darkblue}{\textbf{\ipa{ʈʂe˥}}} \textsubscript{1}}{}
\textcolor{teal}{\mytextsc{nom}} \hspace{4pt} Ton~: \#H.

Terre.
\textcolor{brown}{\zh{土壤。}}


\begin{exe}
\sn \textcolor{darkblue}{\textbf{\ipa{ʈʂe˧pv̩˩}}}
\trans terre sèche
\trans \textit{\textcolor{brown}{\zh{干土}}}
\end{exe}
\begin{exe}
\sn \textcolor{darkblue}{\textbf{\ipa{ʈʂe˧ sɯ˧\textasciitilde{}sɯ˥}}}
\trans 'terre crue': terre qui n'a pas été préparée pour l'agriculture par l'ajout de fumier, etc
\trans \textit{\textcolor{brown}{\zh{‘生土’:没有经过加工(加肥料等等)的土,还不适合种农作物}}}
\end{exe}


\paragraph{\hspace{-0.5cm} {\Large \textcolor{darkblue}{\textbf{\ipa{ʈʂe˥}}} \textsubscript{2}}} \hypertarget{t`s`e\string_T2}{}
\markboth{\textcolor{darkblue}{\textbf{\ipa{ʈʂe˥}}} \textsubscript{2}}{}
\textcolor{teal}{\mytextsc{nom}} \hspace{4pt} Ton~: \#H.

Aiguille.
\textcolor{brown}{\zh{针(汉语借词)。}}


 \mytextsc{clf}~: \textcolor{darkblue}{\textbf{\ipa{ɭɯ˧}}} \textit{Voir~:} \hyperlink{Ro\string_M\string_T1}{\textcolor{darkblue}{\textbf{\ipa{ʁo˧˥}}}} 

\paragraph{\hspace{-0.5cm} {\Large \textcolor{darkblue}{\textbf{\ipa{ʈʂe˧dɑ˥}}}}} \hypertarget{t`s`e\string_MdA\string_T1}{}
\markboth{\textcolor{darkblue}{\textbf{\ipa{ʈʂe˧dɑ˥}}}}{}
\textcolor{teal}{\mytextsc{nom}} \hspace{4pt} Ton~: H\#.

Cloison.
\textcolor{brown}{\zh{隔板。}}


 \mytextsc{clf}~: \textcolor{darkblue}{\textbf{\ipa{do˥}}} 

\paragraph{\hspace{-0.5cm} {\Large \textcolor{darkblue}{\textbf{\ipa{ʈʂe˧gi˥\$}}}}} \hypertarget{t`s`e\string_Mgi\string_T\$1}{}
\markboth{\textcolor{darkblue}{\textbf{\ipa{ʈʂe˧gi˥\$}}}}{}
\textcolor{teal}{\mytextsc{adverbe}} \hspace{4pt} Ton~: H\$.

Entre, au milieu de.
\textcolor{brown}{\zh{中间、之间、间。}}


\begin{exe}
\sn \textcolor{darkblue}{\textbf{\ipa{ə˧-sɯ˩kv̩˩-ʈʂe˩gi˩}}}
\trans entre nous, dans l'espace qui nous sépare
\trans \textit{\textcolor{brown}{\zh{在咱们之间(的空间)}}}
\end{exe}


\paragraph{\hspace{-0.5cm} {\Large \textcolor{darkblue}{\textbf{\ipa{ʈʂe˩\textsubscript{a}}}}}} \hypertarget{t`s`e\string_Ba1}{}
\markboth{\textcolor{darkblue}{\textbf{\ipa{ʈʂe˩\textsubscript{a}}}}}{}
\textcolor{teal}{\mytextsc{verbe}} \hspace{4pt} Ton~: L\textsubscript{a}.

Percer, transpercer (une écharde, un piquant de plante...).
\textcolor{brown}{\zh{刺痛。}}


\begin{exe}
\sn \textcolor{darkblue}{\textbf{\ipa{le˧-ʈʂe˩-ze˩}}}
\trans \mytextsc{accomp} \string_ \mytextsc{pfv}
\trans \textit{\textcolor{brown}{\zh{\mytextsc{accomp} \string_ \mytextsc{pfv}}}}
\end{exe}
\begin{exe}
\sn \textcolor{darkblue}{\textbf{\ipa{tɕʰi˧-ɳɯ˧ ʈʂe˩-ze˩}}}
\trans être piqué par une épine, se prendre une épine
\trans \textit{\textcolor{brown}{\zh{被刺所刺痛}}}
\end{exe}
\begin{exe}
\sn \textcolor{darkblue}{\textbf{\ipa{tso˧\textasciitilde{}tso˧ ʈʂe˥}}}
\trans percer quelque chose
\trans \textit{\textcolor{brown}{\zh{刺到一个东西}}}
\end{exe}


\paragraph{\hspace{-0.5cm} {\Large \textcolor{darkblue}{\textbf{\ipa{ʈʂe˩kʰɯ˩}}}}} \hypertarget{t`s`e\string_Bk\string_hM\string_B1}{}
\markboth{\textcolor{darkblue}{\textbf{\ipa{ʈʂe˩kʰɯ˩}}}}{}
\textcolor{teal}{\mytextsc{nom}} \hspace{4pt} Ton~: L.

Couture (d'un vêtement).
\textcolor{brown}{\zh{缝。}}


 \mytextsc{clf}~: \textcolor{darkblue}{\textbf{\ipa{ɭɯ˧}}} 

\paragraph{\hspace{-0.5cm} {\Large \textcolor{darkblue}{\textbf{\ipa{ʈʂe˩ʂwæ˧˥}}}}} \hypertarget{t`s`e\string_Bs`w\{\string_M\string_T1}{}
\markboth{\textcolor{darkblue}{\textbf{\ipa{ʈʂe˩ʂwæ˧˥}}}}{}
\textcolor{teal}{\mytextsc{nom}} \hspace{4pt} Ton~: LM+MH\#.

Gravier, sable grossier.
\textcolor{brown}{\zh{砾石。}}




\paragraph{\hspace{-0.5cm} {\Large \textcolor{darkblue}{\textbf{\ipa{ʈʂɤ˧\textsubscript{a}}}}}} \hypertarget{t`s`7\string_Ma1}{}
\markboth{\textcolor{darkblue}{\textbf{\ipa{ʈʂɤ˧\textsubscript{a}}}}}{}
\textcolor{teal}{\mytextsc{verbe}} \hspace{4pt} Ton~: M\textsubscript{a}.
\ding{202} 
Compter; calculer.
\textcolor{brown}{\zh{数、算。}}


\begin{exe}
\sn \textcolor{darkblue}{\textbf{\ipa{ʈʂɤ˧\textasciitilde{}ʈʂɤ˩}}}
\trans \mytextsc{red}
\trans \textit{\textcolor{brown}{\zh{\mytextsc{重叠:算一算}}}}
\end{exe}
\begin{exe}
\sn \textcolor{darkblue}{\textbf{\ipa{ɖɯ˧-ʈʂɤ˥\textasciitilde{}ʈʂɤ˩-ɻ̍˩}}}
\trans faire quelques calculs
\trans \textit{\textcolor{brown}{\zh{算一下}}}
\end{exe}
\begin{exe}
\sn \textcolor{darkblue}{\textbf{\ipa{tso˧\textasciitilde{}tso˧ ʈʂɤ˩}}}
\trans compter des choses
\trans \textit{\textcolor{brown}{\zh{数东西}}}
\end{exe}
\begin{exe}
\sn \textcolor{darkblue}{\textbf{\ipa{hĩ˧ ʈʂɤ˩}}}
\trans compter les gens
\trans \textit{\textcolor{brown}{\zh{数人}}}
\end{exe}
\begin{exe}
\sn \textcolor{darkblue}{\textbf{\ipa{bo˩ ʈʂɤ˧}}}
\trans compter les porcs
\trans \textit{\textcolor{brown}{\zh{数猪}}}
\end{exe}
\begin{exe}
\sn \textcolor{darkblue}{\textbf{\ipa{le˧-ʈʂɤ˧-ze˧}}}
\trans \mytextsc{accomp} \string_ \mytextsc{pfv}
\trans \textit{\textcolor{brown}{\zh{数了}}}
\end{exe}
\ding{203} 
Dire la bonne aventure, pratiquer la divination.
\textcolor{brown}{\zh{算命。}}


\begin{exe}
\sn \textcolor{darkblue}{\textbf{\ipa{le˧-ʈʂɤ˥\textasciitilde{}ʈʂɤ˩}}}
\trans dire la bonne aventure, pratiquer la divination
\trans \textit{\textcolor{brown}{\zh{算命}}}
\end{exe}
\begin{exe}
\sn \textcolor{darkblue}{\textbf{\ipa{ɖɯ˧-ʈʂɤ˥\textasciitilde{}ʈʂɤ˩-ɻ̍˩}}}
\trans dire la bonne aventure
\trans \textit{\textcolor{brown}{\zh{算一下命}}}
\end{exe}
\begin{exe}
\sn \textcolor{darkblue}{\textbf{\ipa{ɲi˧ŋwɤ˩hɑ̃˩tʰɑ˩ | ɖɯ˧-ɭɯ˧ | ʈʂɤ˧-bi˧!}}}
\trans On va chercher un jour propice! (pour un événement tel qu'un mariage ou la construction d'une maison)
\trans \textit{\textcolor{brown}{\zh{要掐算一下日子}}}
\end{exe}
\begin{exe}
\sn \textcolor{darkblue}{\textbf{\ipa{kɯ˧ ʈʂɤ˧, | hɑ̃˧ ʈʂɤ˧}}}
\trans chercher un jour propice pour un événement, tel que le début de la construction d'une maison; littéralement: ``compter les étoiles, compter les jours"
\trans \textit{\textcolor{brown}{\zh{掐算一下。直译:“算星星,算日子”。}}}
\end{exe}
\ding{204} 
Compter comme, être, avoir fonction de, avoir rôle de.
\textcolor{brown}{\zh{算是,当作。}}


\begin{exe}
\sn \textcolor{darkblue}{\textbf{\ipa{hĩ˧ ɖɯ˧-v̩˧ ʈʂɤ˧-ze˧!}}}
\trans (Elle/il) compte maintenant comme une grande personne! / C'est un(e) adulte, maintenant! (Ce qu'on dit d'un enfant qui atteint l'âge adulte: 13 ans.)
\trans \textit{\textcolor{brown}{\zh{变成大人了!(十三岁成年礼时常用的一句话)}}}
\end{exe}
\begin{exe}
\sn \textcolor{darkblue}{\textbf{\ipa{dʑɤ˩ ʈʂɤ˧}}}
\trans être très bien
\trans \textit{\textcolor{brown}{\zh{算是很好的}}}
\end{exe}
\begin{exe}
\sn \textcolor{darkblue}{\textbf{\ipa{ʈʂʰɯ˧ | õ˧-bv̩˥-õ˩ | dʑɤ˩ʈʂɤ˧ (+ | ʐwæ˩˥)}}}
\trans Il a une haute idée de lui-même! / Il est orgueilleux!
\trans \textit{\textcolor{brown}{\zh{他觉得自己很了不起!}}}
\end{exe}
\begin{exe}
\sn \textcolor{darkblue}{\textbf{\ipa{hɤ˩ ʈʂɤ˩˥}}}
\trans être habile / admirable / remarquable, être considéré comme habile, compter comme (quelqu'un d')habile
\trans \textit{\textcolor{brown}{\zh{算很了不起的,算很能干的}}}
\end{exe}
\begin{exe}
\sn \textcolor{darkblue}{\textbf{\ipa{ɖwæ˧˥ | hɤ˩ ʈʂɤ˩˥}}}
\trans même sens
\trans \textit{\textcolor{brown}{\zh{同上}}}
\end{exe}
\begin{exe}
\sn \textcolor{darkblue}{\textbf{\ipa{ʈʂʰɯ˧ | gi˧zɯ˧ ʈʂɤ˧-tso˧-ɲi˥.}}}
\trans C'est le petit frère / il a le statut de petit frère! (Commentaire qui rappelle le statut familial de la personne concernée.)
\trans \textit{\textcolor{brown}{\zh{他是做弟弟的!(强调该人的社会角色)}}}
\end{exe}
\begin{exe}
\sn \textcolor{darkblue}{\textbf{\ipa{ʈʂʰɯ˧ | gi˧zɯ˧ ʈʂɤ˧-ɲi˥!}}}
\trans C'est le petit frère / il a le statut de petit frère! (Commentaire qui rappelle le statut familial de la personne concernée.)
\trans \textit{\textcolor{brown}{\zh{他是做弟弟的!(强调该人的社会角色)}}}
\end{exe}
\begin{exe}
\sn \textcolor{darkblue}{\textbf{\ipa{ʈʂʰɯ˧ | bɤ˧ ʈʂɤ˧-tso˧-ɲi˥!}}}
\trans Il/elle est pumi! (Commentaire qui rappelle un élément de l'identité de la personne concernée.)
\trans \textit{\textcolor{brown}{\zh{他是普米族!(强调该人的民族)}}}
\end{exe}
\begin{exe}
\sn \textcolor{darkblue}{\textbf{\ipa{ʈʂʰɯ˧ | nɑ˩ ʈʂɤ˧-tso˧-ɲi˥!}}}
\trans Il/elle est na! (Commentaire qui rappelle un élément de l'identité de la personne concernée)
\trans \textit{\textcolor{brown}{\zh{他是摩梭人!(强调该人的民族身份)}}}
\end{exe}
\begin{exe}
\sn \textcolor{darkblue}{\textbf{\ipa{ʈʂʰɯ˧ | æ˧mv̩˩ ʈʂɤ˩-ɲi˩!}}}
\trans C'est elle la grande soeur! / C'est lui le grand frère! (Commentaire qui rappelle le statut familial de la personne concernée.)
\trans \textit{\textcolor{brown}{\zh{她是做姐姐的! / 他是做哥哥的!(强调该人的社会角色)}}}
\end{exe}
\begin{exe}
\sn \textcolor{darkblue}{\textbf{\ipa{ʈʂʰɯ˧ | gi˧zɯ˧-go˩mi˩ ʈʂɤ˩-ɲi˩!}}}
\trans Ils sont frère et sœur!
\trans \textit{\textcolor{brown}{\zh{他们是(兄)弟(姐)妹!}}}
\end{exe}
\begin{exe}
\sn \textcolor{darkblue}{\textbf{\ipa{ʈʂʰɯ˧ | æ˧mv̩˧-go˧mi˥ | ʈʂɤ˧-tso˧ mɤ˧-ɲi˥! | mɤ˧-ʈʂɤ˧!}}}
\trans Ils ne sont pas frères et sœurs! / Ils n'ont pas cette relation-là!
\trans \textit{\textcolor{brown}{\zh{他们不算是兄弟姐妹!}}}
\end{exe}
\begin{exe}
\sn \textcolor{darkblue}{\textbf{\ipa{ʐwæ˩ ʈʂɤ˩}}}
\trans remarquable, extraordinaire, exceptionnel
\trans \textit{\textcolor{brown}{\zh{了不起}}}
\end{exe}
\begin{exe}
\sn \textcolor{darkblue}{\textbf{\ipa{ʈʂʰɯ˧ | ə˧tso˧ ʐwæ˩ ʈʂɤ˩-tso˩ dʑo˩?}}}
\trans Qu'est-ce qu'elle/il a de si remarquable? / Quelles qualités exceptionnelles a-t-il/elle (que je doive prendre son conseil/suivre son avis)?
\trans \textit{\textcolor{brown}{\zh{他有什么了不起的?}}}
\end{exe}
\begin{exe}
\sn \textcolor{darkblue}{\textbf{\ipa{ʈʂʰɯ˧ | ʐwæ˩ ʈʂɤ˩˥ | ʐwæ˩˥!}}}
\trans C'est quelqu'un de vraiment remarquable/extraordinaire!
\trans \textit{\textcolor{brown}{\zh{他非常了不起!}}}
\end{exe}


\paragraph{\hspace{-0.5cm} {\Large \textcolor{darkblue}{\textbf{\ipa{ʈʂo˧kʰɯ˩}}}}} \hypertarget{t`s`o\string_Mk\string_hM\string_B1}{}
\markboth{\textcolor{darkblue}{\textbf{\ipa{ʈʂo˧kʰɯ˩}}}}{}
\textcolor{teal}{\mytextsc{nom}} \hspace{4pt} Ton~: L\#.

Rituel pour la mort d'une personne de sa famille.
\textcolor{brown}{\zh{忠克:亲人去世时举行的仪式。}}




\paragraph{\hspace{-0.5cm} {\Large \textcolor{darkblue}{\textbf{\ipa{ʈʂo˧ɭɯ\#˥}}}}} \hypertarget{t`s`o\string_Ml\string_RM\#\string_T1}{}
\markboth{\textcolor{darkblue}{\textbf{\ipa{ʈʂo˧ɭɯ\#˥}}}}{}
\textcolor{teal}{\mytextsc{nom}} \hspace{4pt} Ton~: \#H.

Moulin à main.
\textcolor{brown}{\zh{手推磨。}}


\begin{exe}
\sn \textcolor{darkblue}{\textbf{\ipa{ʈʂo˧ɭɯ˧-nv̩˥mi˩}}}
\trans l'axe du moulin (littéralement: son cœur)
\trans \textit{\textcolor{brown}{\zh{手推磨的轴心}}}
\end{exe}
 \mytextsc{clf}~: \textcolor{darkblue}{\textbf{\ipa{nɑ˧}}} 

\paragraph{\hspace{-0.5cm} {\Large \textcolor{darkblue}{\textbf{\ipa{ʈʂo˧ɭɯ˧ʈʂo˧˥}}}}} \hypertarget{t`s`o\string_Ml\string_RM\string_Mt`s`o\string_M\string_T1}{}
\markboth{\textcolor{darkblue}{\textbf{\ipa{ʈʂo˧ɭɯ˧ʈʂo˧˥}}}}{}
\textcolor{teal}{\mytextsc{nom}} \hspace{4pt} Ton~: MH\#.

Insecte aquatique.
\textcolor{brown}{\zh{一种水虫。}}


 \mytextsc{clf}~: \textcolor{darkblue}{\textbf{\ipa{mi˩}}} 

\paragraph{\hspace{-0.5cm} {\Large \textcolor{darkblue}{\textbf{\ipa{ʈʂo˧ʂɯ\#˥}}}}} \hypertarget{t`s`o\string_Ms`M\#\string_T1}{}
\markboth{\textcolor{darkblue}{\textbf{\ipa{ʈʂo˧ʂɯ\#˥}}}}{}
\textcolor{teal}{\mytextsc{nom}} \hspace{4pt} Ton~: \#H.

Village de Zhongshi.
\textcolor{brown}{\zh{忠实(永宁的一个村落)。}}


\begin{exe}
\sn \textcolor{darkblue}{\textbf{\ipa{ɖæ˩ʂɯ\#˥, | ʈʂo˧ʂɯ\#˥, | bɤ˩tɕʰɯ˩˥, | dɑ˧pʰo˥, | bɤ˧dzi˩, | dze˧bo˧}}}
\trans les six villages de la plaine de Yongning, dans l'ordre, qui prend comme point d'origine le village le plus proche du Lac
\trans \textit{\textcolor{brown}{\zh{永宁坝的六个村落,按传统排序:从距离泸沽湖最近的村落说起。}}}
\end{exe}


\paragraph{\hspace{-0.5cm} {\Large \textcolor{darkblue}{\textbf{\ipa{ʈʂo˧tsɯ˥}}}}} \hypertarget{t`s`o\string_MtsM\string_T1}{}
\markboth{\textcolor{darkblue}{\textbf{\ipa{ʈʂo˧tsɯ˥}}}}{}
\textcolor{teal}{\mytextsc{nom}} \hspace{4pt} Ton~: H\#.

Table.
\textcolor{brown}{\zh{桌子(汉语借词)。}}

 Emprunt~: chinois \textcolor{brown}{\zh{桌子}}

 \mytextsc{clf}~: \textcolor{darkblue}{\textbf{\ipa{pɤ˩}}} \textit{Voir~:} \hyperlink{sM\string_Mr£`\{\string_M1}{\textcolor{darkblue}{\textbf{\ipa{sɯ˧ɻæ˧}}}} 

\paragraph{\hspace{-0.5cm} {\Large \textcolor{darkblue}{\textbf{\ipa{ʈʂo˩}}}}} \hypertarget{t`s`o\string_B1}{}
\markboth{\textcolor{darkblue}{\textbf{\ipa{ʈʂo˩}}}}{}
\textcolor{teal}{\mytextsc{classificateur}} \hspace{4pt} Ton~: L.

Classificateur des repas.
\textcolor{brown}{\zh{量词:饭(一顿)。}}


\begin{exe}
\sn \textcolor{darkblue}{\textbf{\ipa{ɖɯ˧-ʈʂo˩ tʰi˩-pæ˩ |}}}
\trans servir un repas
\trans \textit{\textcolor{brown}{\zh{摆饭,摆饭桌}}}
\end{exe}
\begin{exe}
\sn \textcolor{darkblue}{\textbf{\ipa{ʐo˩˥, | njɤ˧ ɖɯ˧-ʈʂo˩ pæ˩-bi˩!}}}
\trans au déjeuner, je vais (vous) servir (tout le) repas!
\trans \textit{\textcolor{brown}{\zh{我来管午饭这一顿!}}}
\end{exe}
\begin{exe}
\sn \textcolor{darkblue}{\textbf{\ipa{hɑ˧ ɖɯ˧-ʈʂo˩}}}
\trans un repas
\trans \textit{\textcolor{brown}{\zh{一顿饭}}}
\end{exe}


\paragraph{\hspace{-0.5cm} {\Large \textcolor{darkblue}{\textbf{\ipa{ʈʂo˩bo˩}}}}} \hypertarget{t`s`o\string_Bbo\string_B1}{}
\markboth{\textcolor{darkblue}{\textbf{\ipa{ʈʂo˩bo˩}}}}{}
\textcolor{teal}{\mytextsc{nom}} \hspace{4pt} Ton~: L.

Mur en terre.
\textcolor{brown}{\zh{土墙。}}


\begin{exe}
\sn \textcolor{darkblue}{\textbf{\ipa{ʈʂo˩bo˩ ti˥}}}
\trans construire un mur en terre (en comprimant la terre)
\trans \textit{\textcolor{brown}{\zh{垒墙}}}
\end{exe}
 \mytextsc{clf}~: \textcolor{darkblue}{\textbf{\ipa{do˥}}} 

\paragraph{\hspace{-0.5cm} {\Large \textcolor{darkblue}{\textbf{\ipa{ʈʂo˩mv̩˩}}}}} \hypertarget{t`s`o\string_Bmv\string_=\string_B1}{}
\markboth{\textcolor{darkblue}{\textbf{\ipa{ʈʂo˩mv̩˩}}}}{}
\textcolor{teal}{\mytextsc{nom}} \hspace{4pt} Ton~: L.

Sable fin.
\textcolor{brown}{\zh{沙子。}}




\paragraph{\hspace{-0.5cm} {\Large \textcolor{darkblue}{\textbf{\ipa{ʈʂɻ̍˥}}}}} \hypertarget{t`s`r£`̍\string_T1}{}
\markboth{\textcolor{darkblue}{\textbf{\ipa{ʈʂɻ̍˥}}}}{}
\textcolor{teal}{\mytextsc{verbe}} \hspace{4pt} Ton~: H.

Tousser.
\textcolor{brown}{\zh{咳嗽。}}


\begin{exe}
\sn \textcolor{darkblue}{\textbf{\ipa{ʈʂʰɯ˧ | tʰi˧-ʈʂɻ̍˥-dʑo˩}}}
\trans il tousse
\trans \textit{\textcolor{brown}{\zh{他在咳嗽}}}
\end{exe}


\paragraph{\hspace{-0.5cm} {\Large \textcolor{darkblue}{\textbf{\ipa{ʈʂɯ˧}}}}} \hypertarget{t`s`M\string_M1}{}
\markboth{\textcolor{darkblue}{\textbf{\ipa{ʈʂɯ˧}}}}{}
\textcolor{teal}{\mytextsc{nom}} \hspace{4pt} Ton~: M.

Griffes (d'un animal); serres (d'un oiseau).
\textcolor{brown}{\zh{爪子。}}


 \mytextsc{clf}~: \textcolor{darkblue}{\textbf{\ipa{ɭɯ˧}}} \textit{Voir~:} \hyperlink{kv\string_=\string_Mt`s`M\string_M\string_T1}{\textcolor{darkblue}{\textbf{\ipa{kv̩˧ʈʂɯ˧˥}}}} 

\paragraph{\hspace{-0.5cm} {\Large \textcolor{darkblue}{\textbf{\ipa{ʈʂɯ˧dzi˩}}}}} \hypertarget{t`s`M\string_Mdzi\string_B1}{}
\markboth{\textcolor{darkblue}{\textbf{\ipa{ʈʂɯ˧dzi˩}}}}{}
\textcolor{teal}{\mytextsc{nom}} \hspace{4pt} Ton~: L\#.

Arbre à vernis.
\textcolor{brown}{\zh{漆树。}}




\paragraph{\hspace{-0.5cm} {\Large \textcolor{darkblue}{\textbf{\ipa{ʈʂɯ˧fv̩\#˥}}}}} \hypertarget{t`s`M\string_Mfv\string_=\#\string_T1}{}
\markboth{\textcolor{darkblue}{\textbf{\ipa{ʈʂɯ˧fv̩\#˥}}}}{}
\textcolor{teal}{\mytextsc{nom}} \hspace{4pt} Ton~: \#H.

Gouvernement local, pouvoir local.
\textcolor{brown}{\zh{(土)知府,如:永宁知府(汉语借词)。}}

 Emprunt~: chinois \textcolor{brown}{\zh{知府}}

\begin{exe}
\sn \textcolor{darkblue}{\textbf{\ipa{no˧ | ɬi˧di˩-ʈʂɯ˩fv̩˩-ni˩-zo˩!}}}
\trans Tu te prends pour le seigneur! (critique qu'on s'adressait aux gens qui se mêlaient de dicter leur conduite aux autres, comme s'ils étaient les maîtres des lieux)
\trans \textit{\textcolor{brown}{\zh{你像永宁土知府! / 你是永宁土知府吧!(批评独断的人、一手包办的人)}}}
\end{exe}
\begin{exe}
\sn \textcolor{darkblue}{\textbf{\ipa{no˧ | ʈʂɯ˧fv̩˧-mi˧-ni˧˥ | -zo˩!}}}
\trans Tu joues les princesses! (littéralement ``les femmes du pouvoir") Critique adressée à une femme qui prend de grands airs.
\trans \textit{\textcolor{brown}{\zh{你好像是永宁大公主! / 你好像是永宁知府女主人!(批评一个独断的女人)}}}
\end{exe}


\paragraph{\hspace{-0.5cm} {\Large \textcolor{darkblue}{\textbf{\ipa{ʈʂɯ˧mɤ˩}}}}} \hypertarget{t`s`M\string_Mm7\string_B1}{}
\markboth{\textcolor{darkblue}{\textbf{\ipa{ʈʂɯ˧mɤ˩}}}}{}
\textcolor{teal}{\mytextsc{nom}} \hspace{4pt} Ton~: L\#.

Sésame.
\textcolor{brown}{\zh{芝麻(汉语借词)。}}

 Emprunt~: chinois \textcolor{brown}{\zh{芝麻}}

\begin{exe}
\sn \textcolor{darkblue}{\textbf{\ipa{ʈʂɯ˧mɤ˩, | ɬi˧di˩ | mɤ˧-tʰv̩˧-ɲi˥!}}}
\trans Le sésame ne pousse pas à Yongning!
\trans \textit{\textcolor{brown}{\zh{永宁不产芝麻!}}}
\end{exe}


\paragraph{\hspace{-0.5cm} {\Large \textcolor{darkblue}{\textbf{\ipa{ʈʂɯ˧˥}}}}} \hypertarget{t`s`M\string_M\string_T1}{}
\markboth{\textcolor{darkblue}{\textbf{\ipa{ʈʂɯ˧˥}}}}{}
\textcolor{teal}{\mytextsc{verbe}} \hspace{4pt} Ton~: MH.

Tamiser.
\textcolor{brown}{\zh{筛。}}


\begin{exe}
\sn \textcolor{darkblue}{\textbf{\ipa{le˧-ʈʂɯ˧-ze˥}}}
\trans \mytextsc{accomp} \string_ \mytextsc{pfv}
\trans \textit{\textcolor{brown}{\zh{\mytextsc{accomp} \string_ \mytextsc{pfv}}}}
\end{exe}
\begin{exe}
\sn \textcolor{darkblue}{\textbf{\ipa{ɖɯ˧-ʈʂɯ˧-ɻ̍˥}}}
\trans \mytextsc{délimitatif} \string_ \mytextsc{inchoatif}
\trans \textit{\textcolor{brown}{\zh{\mytextsc{delimitative} \string_ \mytextsc{inceptive}}}}
\end{exe}


\paragraph{\hspace{-0.5cm} {\Large \textcolor{darkblue}{\textbf{\ipa{ʈʂv̩˩}}}}} \hypertarget{t`s`v\string_=\string_B1}{}
\markboth{\textcolor{darkblue}{\textbf{\ipa{ʈʂv̩˩}}}}{}
\textcolor{teal}{\mytextsc{adjectif}} \hspace{4pt} Ton~: L.

Paisible, aimable, pacifique, doux (signe astrologique); s'emploie au sujet des signes astrologiques: certains sont considérés comme 'paisibles', comme le Boeuf, le Lapin et la Chèvre, ce qui rend les personnes nées cette année-là appropriées pour certains rites/certaines tâches (ex.: lors du rite de passage à l'âge adulte), et au contraire non appropriées pour d'autres.
\textcolor{brown}{\zh{平和的(生肖)。}}


\begin{exe}
\sn \textcolor{darkblue}{\textbf{\ipa{kʰv̩˧ ʈʂv̩˧˥}}}
\trans signe pacifique, calme, non belliqueux
\trans \textit{\textcolor{brown}{\zh{平和的生肖,如牛、兔、羊}}}
\end{exe}


\paragraph{\hspace{-0.5cm} {\Large \textcolor{darkblue}{\textbf{\ipa{ʈʂv̩˩˥}}}}} \hypertarget{t`s`v\string_=\string_B\string_T1}{}
\markboth{\textcolor{darkblue}{\textbf{\ipa{ʈʂv̩˩˥}}}}{}
\textcolor{teal}{\mytextsc{nom}} \hspace{4pt} Ton~: LH.

Sueur (monosyllabe).
\textcolor{brown}{\zh{汗(单音节)。}}


\begin{exe}
\sn \textcolor{darkblue}{\textbf{\ipa{ʈʂv̩˧ bv̩˧nv̩˩}}}
\trans qui sent la sueur, malodorant
\trans \textit{\textcolor{brown}{\zh{有汗(臭)的味道}}}
\end{exe}


\paragraph{\hspace{-0.5cm} {\Large \textcolor{darkblue}{\textbf{\ipa{ʈʂv̩˩\textsubscript{a}}}} \textsubscript{1}}} \hypertarget{t`s`v\string_=\string_Ba1}{}
\markboth{\textcolor{darkblue}{\textbf{\ipa{ʈʂv̩˩\textsubscript{a}}}} \textsubscript{1}}{}
\textcolor{teal}{\mytextsc{verbe}} \hspace{4pt} Ton~: L\textsubscript{a}.

Contaminer, infecter.
\textcolor{brown}{\zh{传染。}}


\begin{exe}
\sn \textcolor{darkblue}{\textbf{\ipa{hĩ˧ ʈʂv̩˥-ho˩}}}
\trans (la maladie) va contaminer quelqu'un
\trans \textit{\textcolor{brown}{\zh{(病毒)会传染人的}}}
\end{exe}
\begin{exe}
\sn \textcolor{darkblue}{\textbf{\ipa{ʈʂv̩˧\textasciitilde{}ʈʂv̩˥}}}
\trans \mytextsc{red}
\trans \textit{\textcolor{brown}{\zh{\mytextsc{重叠}}}}
\end{exe}
\begin{exe}
\sn \textcolor{darkblue}{\textbf{\ipa{ʈʂv̩˧\textasciitilde{}ʈʂv̩˥-ɻ̍˩ ho˩}}}
\trans (la maladie) va contaminer (des gens)
\trans \textit{\textcolor{brown}{\zh{(病毒)会传染的。}}}
\end{exe}
\begin{exe}
\sn \textcolor{darkblue}{\textbf{\ipa{njɤ˧-ɳɯ˧ | no˧ ʈʂv̩˧-ʝi˥!}}}
\trans (Attention,) je vais te contaminer/te passer (mon rhume)!
\trans \textit{\textcolor{brown}{\zh{(要小心:)我会传染你的!}}}
\end{exe}


\paragraph{\hspace{-0.5cm} {\Large \textcolor{darkblue}{\textbf{\ipa{ʈʂv̩˩\textsubscript{a}}}} \textsubscript{2}}} \hypertarget{t`s`v\string_=\string_Ba2}{}
\markboth{\textcolor{darkblue}{\textbf{\ipa{ʈʂv̩˩\textsubscript{a}}}} \textsubscript{2}}{}
\textcolor{teal}{\mytextsc{verbe}} \hspace{4pt} Ton~: L\textsubscript{a}.

Allumer (une bougie).
\textcolor{brown}{\zh{点(蜡烛……)。}}




\paragraph{\hspace{-0.5cm} {\Large \textcolor{darkblue}{\textbf{\ipa{ʈʂv̩˧di˧˥}}}}} \hypertarget{t`s`v\string_=\string_Mdi\string_M\string_T1}{}
\markboth{\textcolor{darkblue}{\textbf{\ipa{ʈʂv̩˧di˧˥}}}}{}
\textcolor{teal}{\mytextsc{nom}} \hspace{4pt} Ton~: MH\#.

Village na hors de la plaine de Yongning, vers le Lac, non loin de \textcolor{darkblue}{\textbf{\ipa{/lɑ˧tʰɑ˧-di˧˥/}}}.
\textcolor{brown}{\zh{村落名。}}




\paragraph{\hspace{-0.5cm} {\Large \textcolor{darkblue}{\textbf{\ipa{ʈʂv̩˩dʑɯ˥}}}}} \hypertarget{t`s`v\string_=\string_Bdz£M\string_T1}{}
\markboth{\textcolor{darkblue}{\textbf{\ipa{ʈʂv̩˩dʑɯ˥}}}}{}
\textcolor{teal}{\mytextsc{nom}} \hspace{4pt} Ton~: LH.

Sueur.
\textcolor{brown}{\zh{汗。}}




\paragraph{\hspace{-0.5cm} {\Large \textcolor{darkblue}{\textbf{\ipa{ʈʂv̩˧pɤ˩}}}}} \hypertarget{t`s`v\string_=\string_Mp7\string_B1}{}
\markboth{\textcolor{darkblue}{\textbf{\ipa{ʈʂv̩˧pɤ˩}}}}{}
\textcolor{teal}{\mytextsc{nom}} \hspace{4pt} Ton~: L\#.

Planche à découper.
\textcolor{brown}{\zh{菜板、俎。}}


 \mytextsc{clf}~: \textcolor{darkblue}{\textbf{\ipa{nɑ˧}}} 

\paragraph{\hspace{-0.5cm} {\Large \textcolor{darkblue}{\textbf{\ipa{ʈʂv̩˧tɕɯ˥}}}}} \hypertarget{t`s`v\string_=\string_Mts£M\string_T1}{}
\markboth{\textcolor{darkblue}{\textbf{\ipa{ʈʂv̩˧tɕɯ˥}}}}{}
\textcolor{teal}{\mytextsc{nom}} \hspace{4pt} Ton~: H\#.

Crachat.
\textcolor{brown}{\zh{痰。}}




\paragraph{\hspace{-0.5cm} {\Large \textcolor{darkblue}{\textbf{\ipa{ʈʂwɑ˧\textasciitilde{}ʈʂwɑ˧-nɑ˧\textasciitilde{}nɑ\#˥}}}}} \hypertarget{t`s`wA\string_M~t`s`wA\string_M-nA\string_M~nA\#\string_T1}{}
\markboth{\textcolor{darkblue}{\textbf{\ipa{ʈʂwɑ˧\textasciitilde{}ʈʂwɑ˧-nɑ˧\textasciitilde{}nɑ\#˥}}}}{}
\textcolor{teal}{\mytextsc{adjectif}} \hspace{4pt} Ton~: \#H.

Divers, varié, désordonné.
\textcolor{brown}{\zh{杂、混杂。}}


\begin{exe}
\sn \textcolor{darkblue}{\textbf{\ipa{ʈʂwɑ˧\textasciitilde{}ʈʂwɑ˧-nɑ˧\textasciitilde{}nɑ˧-hĩ˥}}}
\trans \mytextsc{rel}/nmlz
\trans \textit{\textcolor{brown}{\zh{混杂的}}}
\end{exe}
\begin{exe}
\sn \textcolor{darkblue}{\textbf{\ipa{ʈʂwɑ˧\textasciitilde{}ʈʂwɑ˧-nɑ˧\textasciitilde{}nɑ˧-ɻ̍˥}}}
\trans désordonné
\trans \textit{\textcolor{brown}{\zh{混杂}}}
\end{exe}


\paragraph{\hspace{-0.5cm} {\Large \textcolor{darkblue}{\textbf{\ipa{ʈʂwæ˥\textsubscript{a}}}}}} \hypertarget{t`s`w\{\string_Ta1}{}
\markboth{\textcolor{darkblue}{\textbf{\ipa{ʈʂwæ˥\textsubscript{a}}}}}{}
\textcolor{teal}{\mytextsc{classificateur}} \hspace{4pt} Ton~: H\textsubscript{a}.

Classificateur des trajets.
\textcolor{brown}{\zh{量词:征途、路程、路途、征程,趟。}}


\begin{exe}
\sn \textcolor{darkblue}{\textbf{\ipa{ɖɯ˧-ɲi˥ | ɖɯ˧-ʈʂwæ˧ bi˧}}}
\trans (y) aller une fois par jour
\trans \textit{\textcolor{brown}{\zh{一天去一趟}}}
\end{exe}


\paragraph{\hspace{-0.5cm} {\Large \textcolor{darkblue}{\textbf{\ipa{ʈʂwæ˧tʰo˩}}}}} \hypertarget{t`s`w\{\string_Mt\string_ho\string_B1}{}
\markboth{\textcolor{darkblue}{\textbf{\ipa{ʈʂwæ˧tʰo˩}}}}{}
\textcolor{teal}{\mytextsc{nom}} \hspace{4pt} Ton~: L\#.

Brique.
\textcolor{brown}{\zh{砖头(汉语借词)。}}

 Emprunt~: chinois \textcolor{brown}{\zh{砖头}}

\textit{Voir~:} \hyperlink{t\string_ho\string_Bts£i\string_M\string_T1}{\textcolor{darkblue}{\textbf{\ipa{tʰo˩tɕi˧˥}}}} 

\paragraph{\hspace{-0.5cm} {\Large \textcolor{darkblue}{\textbf{\ipa{ʈʂwæ˧\textasciitilde{}ʈʂwæ˧}}}}} \hypertarget{t`s`w\{\string_M~t`s`w\{\string_M1}{}
\markboth{\textcolor{darkblue}{\textbf{\ipa{ʈʂwæ˧\textasciitilde{}ʈʂwæ˧}}}}{}
\textcolor{teal}{\mytextsc{verbe}} \hspace{4pt} Ton~: M.

Mélanger.
\textcolor{brown}{\zh{搅拌、使混合。}}




\paragraph{\hspace{-0.5cm} {\Large \textcolor{darkblue}{\textbf{\ipa{ʈʂwæ˩ho˧ɻ̍˧}}}}} \hypertarget{t`s`w\{\string_Bho\string_Mr£`̍\string_M1}{}
\markboth{\textcolor{darkblue}{\textbf{\ipa{ʈʂwæ˩ho˧ɻ̍˧}}}}{}
\textcolor{teal}{\mytextsc{nom}} \hspace{4pt} Ton~: LM.

Perceuse.
\textcolor{brown}{\zh{钻子。}}

 Emprunt~: chinois \textcolor{brown}{\zh{钻}}



\paragraph{\hspace{-0.5cm} {\Large \textcolor{darkblue}{\textbf{\ipa{ʈʂwæ˩\textasciitilde{}ʈʂwæ˧˥}}}}} \hypertarget{t`s`w\{\string_B~t`s`w\{\string_M\string_T1}{}
\markboth{\textcolor{darkblue}{\textbf{\ipa{ʈʂwæ˩\textasciitilde{}ʈʂwæ˧˥}}}}{}
\textcolor{teal}{\mytextsc{verbe}} \hspace{4pt} Ton~: MH.

Se tenir par la main, se tenir la main.
\textcolor{brown}{\zh{手拉手。}}


\begin{exe}
\sn \textcolor{darkblue}{\textbf{\ipa{le˧-ʈʂwæ˧\textasciitilde{}ʈʂwæ˧-ze˧!}}}
\trans \mytextsc{accomp} \mytextsc{red} \mytextsc{pfv}
\trans \textit{\textcolor{brown}{\zh{\mytextsc{accomp} \mytextsc{red} \mytextsc{pfv}}}}
\end{exe}
\begin{exe}
\sn \textcolor{darkblue}{\textbf{\ipa{ʈʂwæ˩\textasciitilde{}ʈʂwæ˧-ɻ̍˥}}}
\trans \mytextsc{red} \mytextsc{inchoatif;} même sens: se tenir par la main
\trans \textit{\textcolor{brown}{\zh{\mytextsc{red} \mytextsc{inceptive}}}}
\end{exe}


\paragraph{\hspace{-0.5cm} {\Large \textcolor{darkblue}{\textbf{\ipa{ʈʂwæ˧˥}}} \textsubscript{1}}} \hypertarget{t`s`w\{\string_M\string_T1}{}
\markboth{\textcolor{darkblue}{\textbf{\ipa{ʈʂwæ˧˥}}} \textsubscript{1}}{}
\textcolor{teal}{\mytextsc{verbe}} \hspace{4pt} Ton~: MH.




 Emprunt~: chinois \textcolor{brown}{\zh{装?}}

\ding{202} 
Installer.
\textcolor{brown}{\zh{安装。}}


\begin{exe}
\sn \textcolor{darkblue}{\textbf{\ipa{tjɤ˧hwɑ˧ ʈʂwæ˥}}}
\trans installer le téléphone (dans une demeure qui n'y était pas reliée précédemment)
\trans \textit{\textcolor{brown}{\zh{安装电话(座机)}}}
\end{exe}
\begin{exe}
\sn \textcolor{darkblue}{\textbf{\ipa{le˧-ʈʂwæ˧˥ le˧-tse˧-ze˧!}}}
\trans C'est bien installé!
\trans \textit{\textcolor{brown}{\zh{装好了!}}}
\end{exe}
\ding{203} 
Soigner, réparer (une dent).
\textcolor{brown}{\zh{补(牙)、修好(坏牙)。}}


\begin{exe}
\sn \textcolor{darkblue}{\textbf{\ipa{hi˧ ʈʂwæ˩}}}
\trans soigner une dent; littéralement ``remettre une dent"
\trans \textit{\textcolor{brown}{\zh{补牙、修好坏牙}}}
\end{exe}
\begin{exe}
\sn \textcolor{darkblue}{\textbf{\ipa{hi˧ | le˧-ʈʂwæ˧-ze˥!}}}
\trans La dent est soignée! / La dent est réparée!
\trans \textit{\textcolor{brown}{\zh{牙补好了!}}}
\end{exe}
\ding{204} 
Nouer (des fils).
\textcolor{brown}{\zh{结线。}}


\begin{exe}
\sn \textcolor{darkblue}{\textbf{\ipa{kʰɯ˩ ʈʂwæ˩˥}}}
\trans attacher des brins de fil ensemble, nouer des fils (par ex. lorsqu'on prépare le métier à tisser)
\trans \textit{\textcolor{brown}{\zh{结线}}}
\end{exe}


\paragraph{\hspace{-0.5cm} {\Large \textcolor{darkblue}{\textbf{\ipa{ʈʂwæ˧˥}}} \textsubscript{2}}} \hypertarget{t`s`w\{\string_M\string_T2}{}
\markboth{\textcolor{darkblue}{\textbf{\ipa{ʈʂwæ˧˥}}} \textsubscript{2}}{}
\textcolor{teal}{\mytextsc{verbe}} \hspace{4pt} Ton~: MH.

Savourer, déguster, siroter (nourriture ou boisson).
\textcolor{brown}{\zh{欣赏、品尝(饮食、音乐……)。}}


\begin{exe}
\sn \textcolor{darkblue}{\textbf{\ipa{no˧ | li˩ ʈʂwæ˧-ɻ̍˥! |}}}
\trans Veuillez prendre un peu de thé! (Invitation polie)
\trans \textit{\textcolor{brown}{\zh{请您品一点茶!(礼貌说法)}}}
\end{exe}
\begin{exe}
\sn \textcolor{darkblue}{\textbf{\ipa{ʐɯ˧ F | ʈʂwæ˧˥! | li˩˥ F | ʈʂwæ˧˥! hɑ˧ F | ʈʂwæ˧˥!}}}
\trans L'alcool, ça se savoure; le thé, ça se savoure! (Explication au sujet des emplois du verbe)
\trans \textit{\textcolor{brown}{\zh{酒,是可以品尝的!茶,是可以品尝的!饭,是可以品尝的!(关于‘品尝’这个动词的说明)}}}
\end{exe}
\begin{exe}
\sn \textcolor{darkblue}{\textbf{\ipa{hɑ˧ ʈʂwæ˩}}}
\trans savourer de la nourriture
\trans \textit{\textcolor{brown}{\zh{品尝食物}}}
\end{exe}
\begin{exe}
\sn \textcolor{darkblue}{\textbf{\ipa{li˩ ʈʂwæ˧˥}}}
\trans savourer du thé
\trans \textit{\textcolor{brown}{\zh{品茶}}}
\end{exe}
\begin{exe}
\sn \textcolor{darkblue}{\textbf{\ipa{ʐɯ˧ ʈʂwæ˧˥}}}
\trans savourer de l'alcool
\trans \textit{\textcolor{brown}{\zh{品酒}}}
\end{exe}
\begin{exe}
\sn \textcolor{darkblue}{\textbf{\ipa{ə˩kʰɯ˩ ʈʂwæ˥}}}
\trans savourer des feuilles de navet (formulation ironique mais tout à fait acceptable)
\trans \textit{\textcolor{brown}{\zh{尝尝圆根(玩笑话,因为圆根没有什么滋味)}}}
\end{exe}


\paragraph{\hspace{-0.5cm} {\Large \textcolor{darkblue}{\textbf{\ipa{ʈʂwɤ˧\textsubscript{a}}}}}} \hypertarget{t`s`w7\string_Ma1}{}
\markboth{\textcolor{darkblue}{\textbf{\ipa{ʈʂwɤ˧\textsubscript{a}}}}}{}
\textcolor{teal}{\mytextsc{classificateur}} \hspace{4pt} Ton~: M\textsubscript{a}.

Classificateur des poignées: ce que l'on peut prendre dans une seule main.
\textcolor{brown}{\zh{量词:捧。}}




\paragraph{\hspace{-0.5cm} {\Large \textcolor{darkblue}{\textbf{\ipa{ʈʂwɤ˧\textsubscript{a}}}}}} \hypertarget{t`s`w7\string_Ma1}{}
\markboth{\textcolor{darkblue}{\textbf{\ipa{ʈʂwɤ˧\textsubscript{a}}}}}{}
\textcolor{teal}{\mytextsc{verbe}} \hspace{4pt} Ton~: M\textsubscript{a}.

Griffer (ex.: un tigre griffe).
\textcolor{brown}{\zh{抓(用爪子抓)。}}


\begin{exe}
\sn \textcolor{darkblue}{\textbf{\ipa{tso˧\textasciitilde{}tso˧ ʈʂwɤ˩}}}
\trans griffer des objets
\trans \textit{\textcolor{brown}{\zh{抓东西}}}
\end{exe}


\paragraph{\hspace{-0.5cm} {\Large \textcolor{darkblue}{\textbf{\ipa{ʈʂwɤ˧\textasciitilde{}ʈʂwɤ˩}}}}} \hypertarget{t`s`w7\string_M~t`s`w7\string_B1}{}
\markboth{\textcolor{darkblue}{\textbf{\ipa{ʈʂwɤ˧\textasciitilde{}ʈʂwɤ˩}}}}{}
\textcolor{teal}{\mytextsc{verbe}} \hspace{4pt} Ton~: M.

Toucher.
\textcolor{brown}{\zh{触碰。}}


\begin{exe}
\sn \textcolor{darkblue}{\textbf{\ipa{ə˧tso˧ mɤ˧-ɲi˩ ʈʂwɤ˧\textasciitilde{}ʈʂwɤ˩!}}}
\trans (tu) touches vraiment à tout! (doux reproche adressé à un bébé qui se promène sur une table et s'empare de tout ce qui s'y trouve)
\trans \textit{\textcolor{brown}{\zh{你什么都碰,是吗!(小孩爬在桌子上,试着拿每个东西)}}}
\end{exe}


\paragraph{\hspace{-0.5cm} {\Large \textcolor{darkblue}{\textbf{\ipa{ʈʂʰɑ˧lɑ˧}}}}} \hypertarget{t`s`\string_hA\string_MlA\string_M1}{}
\markboth{\textcolor{darkblue}{\textbf{\ipa{ʈʂʰɑ˧lɑ˧}}}}{}
\textcolor{teal}{\mytextsc{verbe}} \hspace{4pt} Ton~: M.

Discuter, bavarder.
\textcolor{brown}{\zh{商量、交谈、谈天、聊天。}}


\begin{exe}
\sn \textcolor{darkblue}{\textbf{\ipa{hĩ˧-qɑ˩ ʈʂʰɑ˩lɑ˩}}}
\trans bavarder avec quelqu'un
\trans \textit{\textcolor{brown}{\zh{跟人聊天}}}
\end{exe}
\begin{exe}
\sn \textcolor{darkblue}{\textbf{\ipa{ɖɯ˧-kʰwɤ˧ ʈʂʰɑ˧lɑ˥}}}
\trans bavarder un peu, avoir une causerie avec
\trans \textit{\textcolor{brown}{\zh{聊聊天}}}
\end{exe}
\begin{exe}
\sn \textcolor{darkblue}{\textbf{\ipa{njɤ˧ | no˧-qɑ˧ ʈʂʰɑ˧lɑ˥}}}
\trans je te raconte, je te dis
\trans \textit{\textcolor{brown}{\zh{我给你讲、我跟你聊聊天}}}
\end{exe}


\paragraph{\hspace{-0.5cm} {\Large \textcolor{darkblue}{\textbf{\ipa{ʈʂʰɑ˧lɑ˧-mv̩˧lɑ˩}}}}} \hypertarget{t`s`\string_hA\string_MlA\string_M-mv\string_=\string_MlA\string_B1}{}
\markboth{\textcolor{darkblue}{\textbf{\ipa{ʈʂʰɑ˧lɑ˧-mv̩˧lɑ˩}}}}{}
\textcolor{teal}{\mytextsc{verbe}} \hspace{4pt} Ton~: M.

Discuter, bavarder.
\textcolor{brown}{\zh{商量、交谈、谈天、聊天。}}


\begin{exe}
\sn \textcolor{darkblue}{\textbf{\ipa{ʈʂʰɑ˧lɑ˧-mv̩˧lɑ˩-ɻ̍˩}}}
\trans bavarder un peu, avoir une causerie
\trans \textit{\textcolor{brown}{\zh{聊聊天}}}
\end{exe}


\paragraph{\hspace{-0.5cm} {\Large \textcolor{darkblue}{\textbf{\ipa{ʈʂʰɑ˧nɑ˥}}}}} \hypertarget{t`s`\string_hA\string_MnA\string_T1}{}
\markboth{\textcolor{darkblue}{\textbf{\ipa{ʈʂʰɑ˧nɑ˥}}}}{}
\textcolor{teal}{\mytextsc{nom}} \hspace{4pt} Ton~: H\#.

Nom d'une source sacrée, située au pied d'une falaise, sur la montagne \textcolor{darkblue}{\textbf{\ipa{/qv̩˧ɻ\#˥/;}}} on disait que son eau sortait du ventre de la montagne. Le récit DumbChildren raconte comment son eau était utilisée comme remède de fertilité.
\textcolor{brown}{\zh{一眼山泉的名字。}}


\begin{exe}
\sn \textcolor{darkblue}{\textbf{\ipa{qv̩˧ɻ̍˧-ʈʂʰɑ˧nɑ˥\#}}}
\trans nom complet de la montagne
\trans \textit{\textcolor{brown}{\zh{山的全名,包括水泉名}}}
\end{exe}
\begin{exe}
\sn \textcolor{darkblue}{\textbf{\ipa{kɤ˧mv̩˧˥, | æ˧ʂæ˧, | ŋwɤ˧hɑ̃˩, | ʂwæ˧gv̩\#˥, | nɑ˩tsʰi˩˥ | -tɕʰɤ˧pɤ˧mi\#˥, | qv̩˧ɻ̍˧-ʈʂʰɑ˧nɑ˥ |}}}
\trans Les six montagnes de Yongning qui portent un nom. Les autres sommets du voisinage n'ont pas une valeur symbolique comparable, et ne portent pas de nom communément utilisé.
\trans \textit{\textcolor{brown}{\zh{永宁地区有固定名字的六座山。其它的山,因为没有重要的象征意义,因此没有取名。}}}
\end{exe}


\paragraph{\hspace{-0.5cm} {\Large \textcolor{darkblue}{\textbf{\ipa{ʈʂʰæ˥}}}}} \hypertarget{t`s`\string_h\{\string_T1}{}
\markboth{\textcolor{darkblue}{\textbf{\ipa{ʈʂʰæ˥}}}}{}
\textcolor{teal}{\mytextsc{verbe}} \hspace{4pt} Ton~: H.

Laver (les habits, la vaisselle…), rincer (le riz…).
\textcolor{brown}{\zh{洗(洗衣服,洗澡……)。}}


\begin{exe}
\sn \textcolor{darkblue}{\textbf{\ipa{dʑi˧hṽ˧ ʈʂʰæ˧}}}
\trans laver des vêtements
\trans \textit{\textcolor{brown}{\zh{洗衣服}}}
\end{exe}
\begin{exe}
\sn \textcolor{darkblue}{\textbf{\ipa{bɑ˩lɑ˩ ʈʂʰæ˩˥}}}
\trans laver des chemises
\trans \textit{\textcolor{brown}{\zh{洗上衣}}}
\end{exe}
\begin{exe}
\sn \textcolor{darkblue}{\textbf{\ipa{ɬi˧qʰwɤ˩ ʈʂʰæ˩}}}
\trans laver des pantalons
\trans \textit{\textcolor{brown}{\zh{洗裤子}}}
\end{exe}
\begin{exe}
\sn \textcolor{darkblue}{\textbf{\ipa{gv̩˧mi˧ ʈʂʰæ˧}}}
\trans se laver, prendre un bain/une douche
\trans \textit{\textcolor{brown}{\zh{洗澡}}}
\end{exe}
\begin{exe}
\sn \textcolor{darkblue}{\textbf{\ipa{gv̩˧mi˧ ʈʂʰæ˧\textasciitilde{}ʈʂʰæ˧}}}
\trans se laver un coup
\trans \textit{\textcolor{brown}{\zh{洗一下身体}}}
\end{exe}
\begin{exe}
\sn \textcolor{darkblue}{\textbf{\ipa{hɑ˧ ʈʂʰæ˧}}}
\trans rincer une céréale (avant de la cuire)
\trans \textit{\textcolor{brown}{\zh{淘洗粮食}}}
\end{exe}
\begin{exe}
\sn \textcolor{darkblue}{\textbf{\ipa{ɕi˧ʈʂʰwæ˧ ʈʂʰæ˧(-ze˩)}}}
\trans rincer le riz (avant de le cuire)
\trans \textit{\textcolor{brown}{\zh{淘米}}}
\end{exe}


\paragraph{\hspace{-0.5cm} {\Large \textcolor{darkblue}{\textbf{\ipa{ʈʂʰæ˧ɣɯ\#˥}}}}} \hypertarget{t`s`\string_h\{\string_MGM\#\string_T1}{}
\markboth{\textcolor{darkblue}{\textbf{\ipa{ʈʂʰæ˧ɣɯ\#˥}}}}{}
\textcolor{teal}{\mytextsc{nom}} \hspace{4pt} Ton~: \#H.

Médicament.
\textcolor{brown}{\zh{药。}}


\begin{exe}
\sn \textcolor{darkblue}{\textbf{\ipa{ʈʂʰæ˧ɣɯ˧ ʈʰɯ˧˥}}}
\trans Prendre un médicament. Littéralement: ``boire un médicament" (collocation différente du chinois \textcolor{brown}{\zh{吃药}} ``manger un médicament")
\trans \textit{\textcolor{brown}{\zh{吃药(直译:“喝药”)}}}
\end{exe}
\begin{exe}
\sn \textcolor{darkblue}{\textbf{\ipa{ʈʂʰæ˧ɣɯ˧ lɑ˩}}}
\trans répandre des pesticides, traiter (un verger, un potager, un champ…)
\trans \textit{\textcolor{brown}{\zh{打农药}}}
\end{exe}


\paragraph{\hspace{-0.5cm} {\Large \textcolor{darkblue}{\textbf{\ipa{ʈʂʰæ˧ɣɯ˧-ki˩-hĩ˩-hĩ˩}}}}} \hypertarget{t`s`\string_h\{\string_MGM\string_M-ki\string_B-hi\string_~\string_B-hi\string_~\string_B1}{}
\markboth{\textcolor{darkblue}{\textbf{\ipa{ʈʂʰæ˧ɣɯ˧-ki˩-hĩ˩-hĩ˩}}}}{}
\textcolor{teal}{\mytextsc{nom}} \hspace{4pt} Ton~: \mytextsc{L}.

Médecin, docteur; littéralement: ``personne qui donne des médicaments".
\textcolor{brown}{\zh{医生。}}


 \mytextsc{clf}~: \textcolor{darkblue}{\textbf{\ipa{v̩˧}}} 

\paragraph{\hspace{-0.5cm} {\Large \textcolor{darkblue}{\textbf{\ipa{ʈʂʰæ˧mi˥\$}}}}} \hypertarget{t`s`\string_h\{\string_Mmi\string_T\$1}{}
\markboth{\textcolor{darkblue}{\textbf{\ipa{ʈʂʰæ˧mi˥\$}}}}{}
\textcolor{teal}{\mytextsc{nom}} \hspace{4pt} Ton~: H\$.

Biche.
\textcolor{brown}{\zh{母马鹿。}}


 \mytextsc{clf}~: \textcolor{darkblue}{\textbf{\ipa{pʰo˧˥}}} 

\paragraph{\hspace{-0.5cm} {\Large \textcolor{darkblue}{\textbf{\ipa{ʈʂʰæ˧nɑ˥}}}}} \hypertarget{t`s`\string_h\{\string_MnA\string_T1}{}
\markboth{\textcolor{darkblue}{\textbf{\ipa{ʈʂʰæ˧nɑ˥}}}}{}
\textcolor{teal}{\mytextsc{nom}} \hspace{4pt} Ton~: H\#.

Cerf noir: espèce légendaire, que seuls les esprits sont à même de chasser et abattre.
\textcolor{brown}{\zh{黑鹿。}}


 \mytextsc{clf}~: \textcolor{darkblue}{\textbf{\ipa{pʰo˧˥}}} 

\paragraph{\hspace{-0.5cm} {\Large \textcolor{darkblue}{\textbf{\ipa{ʈʂʰæ˧pʰv̩\#˥}}}}} \hypertarget{t`s`\string_h\{\string_Mp\string_hv\string_=\#\string_T1}{}
\markboth{\textcolor{darkblue}{\textbf{\ipa{ʈʂʰæ˧pʰv̩\#˥}}}}{}
\textcolor{teal}{\mytextsc{nom}} \hspace{4pt} Ton~: \#H.

Cerf (mâle).
\textcolor{brown}{\zh{公马鹿。}}


 \mytextsc{clf}~: \textcolor{darkblue}{\textbf{\ipa{pʰo˧˥}}} 

\paragraph{\hspace{-0.5cm} {\Large \textcolor{darkblue}{\textbf{\ipa{ʈʂʰæ˧qʰv̩˥\$}}}}} \hypertarget{t`s`\string_h\{\string_Mq\string_hv\string_=\string_T\$1}{}
\markboth{\textcolor{darkblue}{\textbf{\ipa{ʈʂʰæ˧qʰv̩˥\$}}}}{}
\textcolor{teal}{\mytextsc{nom}} \hspace{4pt} Ton~: H\$.

Bois d'un cerf (même mot pour les bois d'un jeune cerf, utilisés comme aphrodisiaque en médecine traditionnelle).
\textcolor{brown}{\zh{鹿角,鹿茸。}}


 \mytextsc{clf}~: \textcolor{darkblue}{\textbf{\ipa{ɭɯ˧}}} 

\paragraph{\hspace{-0.5cm} {\Large \textcolor{darkblue}{\textbf{\ipa{ʈʂʰæ˧sɯ˩}}}}} \hypertarget{t`s`\string_h\{\string_MsM\string_B1}{}
\markboth{\textcolor{darkblue}{\textbf{\ipa{ʈʂʰæ˧sɯ˩}}}}{}
\textcolor{teal}{\mytextsc{adverbe}} \hspace{4pt} Ton~: .

Dommage, regrettable.
\textcolor{brown}{\zh{太可惜了!}}


\begin{exe}
\sn \textcolor{darkblue}{\textbf{\ipa{ʈʂʰæ˧sɯ˩!}}}
\trans Dommage!
\trans \textit{\textcolor{brown}{\zh{可惜!}}}
\end{exe}


\paragraph{\hspace{-0.5cm} {\Large \textcolor{darkblue}{\textbf{\ipa{ʈʂʰæ˧\textasciitilde{}ʈʂʰæ˧}}}}} \hypertarget{t`s`\string_h\{\string_M~t`s`\string_h\{\string_M1}{}
\markboth{\textcolor{darkblue}{\textbf{\ipa{ʈʂʰæ˧\textasciitilde{}ʈʂʰæ˧}}}}{}
\textcolor{teal}{\mytextsc{adjectif}} \hspace{4pt} Ton~: M.

Solide, de bonne qualité, résistant (vêtement, outil, objet...).
\textcolor{brown}{\zh{结实、质量好,(东西)耐用,(人)可靠。}}




\paragraph{\hspace{-0.5cm} {\Large \textcolor{darkblue}{\textbf{\ipa{ʈʂʰæ˧zo\#˥}}}}} \hypertarget{t`s`\string_h\{\string_Mzo\#\string_T1}{}
\markboth{\textcolor{darkblue}{\textbf{\ipa{ʈʂʰæ˧zo\#˥}}}}{}
\textcolor{teal}{\mytextsc{nom}} \hspace{4pt} Ton~: \#H.

Faon.
\textcolor{brown}{\zh{小鹿。}}


 \mytextsc{clf}~: \textcolor{darkblue}{\textbf{\ipa{ɭɯ˧}}} 

\paragraph{\hspace{-0.5cm} {\Large \textcolor{darkblue}{\textbf{\ipa{ʈʂʰæ˧˥}}} \textsubscript{1}}} \hypertarget{t`s`\string_h\{\string_M\string_T1}{}
\markboth{\textcolor{darkblue}{\textbf{\ipa{ʈʂʰæ˧˥}}} \textsubscript{1}}{}
\textcolor{teal}{\mytextsc{nom}} \hspace{4pt} Ton~: MH.

Cerf, \textit{Cervus elaphus kansuensis}.
\textcolor{brown}{\zh{马鹿。}}


 \mytextsc{clf}~: \textcolor{darkblue}{\textbf{\ipa{pʰo˧˥}}} 

\paragraph{\hspace{-0.5cm} {\Large \textcolor{darkblue}{\textbf{\ipa{ʈʂʰæ˧˥}}} \textsubscript{2}}} \hypertarget{t`s`\string_h\{\string_M\string_T2}{}
\markboth{\textcolor{darkblue}{\textbf{\ipa{ʈʂʰæ˧˥}}} \textsubscript{2}}{}
\textcolor{teal}{\mytextsc{classificateur}} \hspace{4pt} Ton~: MH.

Classificateur des générations.
\textcolor{brown}{\zh{量词:代、世、辈、世代。}}




\paragraph{\hspace{-0.5cm} {\Large \textcolor{darkblue}{\textbf{\ipa{ʈʂʰe˧\textsubscript{b}}}}}} \hypertarget{t`s`\string_he\string_Mb1}{}
\markboth{\textcolor{darkblue}{\textbf{\ipa{ʈʂʰe˧\textsubscript{b}}}}}{}
\textcolor{teal}{\mytextsc{verbe}} \hspace{4pt} Ton~: M\textsubscript{b}.

Tendre, étendre (la main).
\textcolor{brown}{\zh{伸(伸手)。}}


\begin{exe}
\sn \textcolor{darkblue}{\textbf{\ipa{le˧-ʈʂʰe˧-ze˧}}}
\trans \mytextsc{accomp} \string_ \mytextsc{pfv}
\trans \textit{\textcolor{brown}{\zh{\mytextsc{accomp} \string_ \mytextsc{pfv}}}}
\end{exe}
\begin{exe}
\sn \textcolor{darkblue}{\textbf{\ipa{mv̩˩tɕo˧ ʈʂʰe˧}}}
\trans étendre vers le bas
\trans \textit{\textcolor{brown}{\zh{向下伸展}}}
\end{exe}
\begin{exe}
\sn \textcolor{darkblue}{\textbf{\ipa{lo˩qʰwɤ˧ | ə˩pʰo˩ ʈʂʰe˩˥}}}
\trans étendre la main à l'extérieur (par une fenêtre)
\trans \textit{\textcolor{brown}{\zh{手伸到外边}}}
\end{exe}
\begin{exe}
\sn \textcolor{darkblue}{\textbf{\ipa{tso˧\textasciitilde{}tso˧ ʈʂʰe˧}}}
\trans étendre quelque chose: par exemple, faire sortir un bâton par une fenêtre
\trans \textit{\textcolor{brown}{\zh{伸出一个东西,如:从车窗里伸出一个棍子}}}
\end{exe}


\paragraph{\hspace{-0.5cm} {\Large \textcolor{darkblue}{\textbf{\ipa{ʈʂʰe˧\textasciitilde{}ʈʂʰe˧}}}}} \hypertarget{t`s`\string_he\string_M~t`s`\string_he\string_M1}{}
\markboth{\textcolor{darkblue}{\textbf{\ipa{ʈʂʰe˧\textasciitilde{}ʈʂʰe˧}}}}{}
\textcolor{teal}{\mytextsc{classificateur}} \hspace{4pt} Ton~: M.
\textit{De:} \textbf{ʈʂʰe˧b} 
Classificateur pour les murs, et donc pour la largeur de toute une pièce: un buffet/placard occupe toute la largeur d'une pièce, par exemple.
\textcolor{brown}{\zh{量词:一面(墙)。}}




\paragraph{\hspace{-0.5cm} {\Large \textcolor{darkblue}{\textbf{\ipa{ʈʂʰe˩ko˧}}}}} \hypertarget{t`s`\string_he\string_Bko\string_M1}{}
\markboth{\textcolor{darkblue}{\textbf{\ipa{ʈʂʰe˩ko˧}}}}{}
\textcolor{teal}{\mytextsc{verbe}} \hspace{4pt} Ton~: LM.

Réussir.
\textcolor{brown}{\zh{成功(汉语借词)。}}

 Emprunt~: chinois \textcolor{brown}{\zh{成功}}



\paragraph{\hspace{-0.5cm} {\Large \textcolor{darkblue}{\textbf{\ipa{ʈʂʰɤ˧tsɯ˧}}}}} \hypertarget{t`s`\string_h7\string_MtsM\string_M1}{}
\markboth{\textcolor{darkblue}{\textbf{\ipa{ʈʂʰɤ˧tsɯ˧}}}}{}
\textcolor{teal}{\mytextsc{nom}} \hspace{4pt} Ton~: M.

Voiture, automobile, car.
\textcolor{brown}{\zh{车子(汉语借词)。}}

 Emprunt~: chinois \textcolor{brown}{\zh{车子}}



\paragraph{\hspace{-0.5cm} {\Large \textcolor{darkblue}{\textbf{\ipa{ʈʂʰɤ˧zo˥-ʈʂʰɤ˩mv̩˩}}}}} \hypertarget{t`s`\string_h7\string_Mzo\string_T-t`s`\string_h7\string_Bmv\string_=\string_B1}{}
\markboth{\textcolor{darkblue}{\textbf{\ipa{ʈʂʰɤ˧zo˥-ʈʂʰɤ˩mv̩˩}}}}{}
\textcolor{teal}{\mytextsc{nom}} \hspace{4pt} Ton~: H\#-L.

Enfant naturel.
\textcolor{brown}{\zh{私生子:没有名分的孩子、不明来路。}}


\begin{exe}
\sn \textcolor{darkblue}{\textbf{\ipa{ə˧dɑ˥ | ɲi˩-ɲi˥ | mɤ˧-sɯ˥ | ʈʂʰɯ˧-v̩˧, | ʈʂʰɤ˧zo˥-ʈʂʰɤ˩mv̩˩ mv̩˩ʈʂæ˩.}}}
\trans Celui qui ne sait pas qui est son père, on l'appelle ``enfant naturel".
\trans \textit{\textcolor{brown}{\zh{一个人不知道他父亲是谁,就称作“私生子”。}}}
\end{exe}
\begin{exe}
\sn \textcolor{darkblue}{\textbf{\ipa{ə˧ʝi˧-ʂɯ˥ʝi˩, | ʈʂʰɤ˧zo˥-ʈʂʰɤ˩mv̩˩ ʐɤ˩-hĩ˩-lɑ˩ ɲi˩!}}}
\trans Autrefois, les enfants naturels, on les élevait et voilà tout!/on les élevait tout simplement, sans faire de difficultés!
\trans \textit{\textcolor{brown}{\zh{过去,大家会公开把“私生子”养大,不会大惊小怪的!}}}
\end{exe}


\paragraph{\hspace{-0.5cm} {\Large \textcolor{darkblue}{\textbf{\ipa{ʈʂʰɤ˩\textsubscript{a}}}}}} \hypertarget{t`s`\string_h7\string_Ba1}{}
\markboth{\textcolor{darkblue}{\textbf{\ipa{ʈʂʰɤ˩\textsubscript{a}}}}}{}
\textcolor{teal}{\mytextsc{verbe}} \hspace{4pt} Ton~: L\textsubscript{a}.

Répartir, diviser.
\textcolor{brown}{\zh{分。}}


\begin{exe}
\sn \textcolor{darkblue}{\textbf{\ipa{ɖɯ˧-v̩˧ ɖɯ˧-kʰwɤ˥ | le˧-ʈʂʰɤ˧\textasciitilde{}ʈʂʰɤ˥}}}
\trans répartir un morceau par personne
\trans \textit{\textcolor{brown}{\zh{平分}}}
\end{exe}


\paragraph{\hspace{-0.5cm} {\Large \textcolor{darkblue}{\textbf{\ipa{ʈʂʰɤ˩ho˥}}}}} \hypertarget{t`s`\string_h7\string_Bho\string_T1}{}
\markboth{\textcolor{darkblue}{\textbf{\ipa{ʈʂʰɤ˩ho˥}}}}{}
\textcolor{teal}{\mytextsc{nom}} \hspace{4pt} Ton~: LH.

Bouilloire.
\textcolor{brown}{\zh{水壶(汉语借词:茶壶)。}}

 Emprunt~: \textcolor{brown}{\zh{茶壶}}

 \mytextsc{clf}~: \textcolor{darkblue}{\textbf{\ipa{ɭɯ˧}}} 

\paragraph{\hspace{-0.5cm} {\Large \textcolor{darkblue}{\textbf{\ipa{ʈʂʰɤ˩kɤ˧}}}}} \hypertarget{t`s`\string_h7\string_Bk7\string_M1}{}
\markboth{\textcolor{darkblue}{\textbf{\ipa{ʈʂʰɤ˩kɤ˧}}}}{}
\textcolor{teal}{\mytextsc{nom}} \hspace{4pt} Ton~: LM.

Gobelet (avec anse); en métal ou poterie.
\textcolor{brown}{\zh{缸子,杯子。}}


 \mytextsc{clf}~: \textcolor{darkblue}{\textbf{\ipa{ɭɯ˧}}} 

\paragraph{\hspace{-0.5cm} {\Large \textcolor{darkblue}{\textbf{\ipa{ʈʂʰɤ˩qo˧}}}}} \hypertarget{t`s`\string_h7\string_Bqo\string_M1}{}
\markboth{\textcolor{darkblue}{\textbf{\ipa{ʈʂʰɤ˩qo˧}}}}{}
\textcolor{teal}{\mytextsc{nom}} \hspace{4pt} Ton~: LM.

Attention, intérêt.
\textcolor{brown}{\zh{关注、关心。}}


\begin{exe}
\sn \textcolor{darkblue}{\textbf{\ipa{ʈʂʰɤ˩qo˧ kʰɯ˧˥}}}
\trans se soucier de, prêter attention à
\trans \textit{\textcolor{brown}{\zh{关心、关注}}}
\end{exe}
\begin{exe}
\sn \textcolor{darkblue}{\textbf{\ipa{ʈʂʰɤ˩qo˧ | ɖwæ˧˥ | tʰi˧-kʰɯ˧˥}}}
\trans prêter une grande attention à, être très attentif à (ex.: une grand-mère très attentive à l'alimentation d'un petit enfant)
\trans \textit{\textcolor{brown}{\zh{很关心、很关注}}}
\end{exe}
\begin{exe}
\sn \textcolor{darkblue}{\textbf{\ipa{ʈʂʰɤ˩qo˧ | mɤ˧-kʰɯ˧˥}}}
\trans être insensible à, ne pas prêter attention à
\trans \textit{\textcolor{brown}{\zh{不关心、不关注}}}
\end{exe}
 \mytextsc{clf}~: \textcolor{darkblue}{\textbf{\ipa{kʰwɤ˥}}} 

\paragraph{\hspace{-0.5cm} {\Large \textcolor{darkblue}{\textbf{\ipa{ʈʂʰɤ˩tɕʰɯ˩}}}}} \hypertarget{t`s`\string_h7\string_Bts£\string_hM\string_B1}{}
\markboth{\textcolor{darkblue}{\textbf{\ipa{ʈʂʰɤ˩tɕʰɯ˩}}}}{}
\textcolor{teal}{\mytextsc{adjectif}} \hspace{4pt} Ton~: L.

Très doué, très calé, possédant des qualités admirables (par ex.: personne très savante).
\textcolor{brown}{\zh{利害,值得崇拜。}}




\paragraph{\hspace{-0.5cm} {\Large \textcolor{darkblue}{\textbf{\ipa{ʈʂʰɤ˩\textasciitilde{}ʈʂʰɤ˧˥}}}}} \hypertarget{t`s`\string_h7\string_B~t`s`\string_h7\string_M\string_T1}{}
\markboth{\textcolor{darkblue}{\textbf{\ipa{ʈʂʰɤ˩\textasciitilde{}ʈʂʰɤ˧˥}}}}{}
\textcolor{teal}{\mytextsc{verbe}} \hspace{4pt} Ton~: MH.

Toucher.
\textcolor{brown}{\zh{抚摸。}}


\begin{exe}
\sn \textcolor{darkblue}{\textbf{\ipa{ʈʂʰɤ˩ʈʂʰɤ˧ mɤ˥-tʰɑ˩!}}}
\trans il ne faut pas toucher!
\trans \textit{\textcolor{brown}{\zh{禁止触碰!}}}
\end{exe}
\begin{exe}
\sn \textcolor{darkblue}{\textbf{\ipa{tʰɑ˧-ʈʂʰɤ˩ʈʂʰɤ˩!}}}
\trans ne touchez pas!
\trans \textit{\textcolor{brown}{\zh{别碰!}}}
\end{exe}
\begin{exe}
\sn \textcolor{darkblue}{\textbf{\ipa{tso˧\textasciitilde{}tso˧ ʈʂʰɤ˥ʈʂʰɤ˩}}}
\trans toucher quelque chose
\trans \textit{\textcolor{brown}{\zh{抚摸东西}}}
\end{exe}


\paragraph{\hspace{-0.5cm} {\Large \textcolor{darkblue}{\textbf{\ipa{ʈʂʰo˥}}}}} \hypertarget{t`s`\string_ho\string_T1}{}
\markboth{\textcolor{darkblue}{\textbf{\ipa{ʈʂʰo˥}}}}{}
\textcolor{teal}{\mytextsc{verbe}} \hspace{4pt} Ton~: H.

Prier (une divinité): réciter des prières, psalmodier des prières.
\textcolor{brown}{\zh{拜(神)。}}


\begin{exe}
\sn \textcolor{darkblue}{\textbf{\ipa{ʈʂʰo˧do˩ ʈʂʰo˩}}}
\trans prier l'esprit du foyer
\trans \textit{\textcolor{brown}{\zh{祭祀祖先}}}
\end{exe}
\begin{exe}
\sn \textcolor{darkblue}{\textbf{\ipa{hĩ˧-mo˥, | zo˩qo˧ ʂɯ˧, | zo˩qo˧-ɳɯ˧ ʈʂʰo˧-zo˧!}}}
\trans Les membres décédés de la famille, c'est à l'endroit où ils sont morts qu'on leur rend hommage!
\trans \textit{\textcolor{brown}{\zh{要在家人去世地点进行祭拜!}}}
\end{exe}


\paragraph{\hspace{-0.5cm} {\Large \textcolor{darkblue}{\textbf{\ipa{ʈʂʰo˧\textsubscript{b}}}}}} \hypertarget{t`s`\string_ho\string_Mb1}{}
\markboth{\textcolor{darkblue}{\textbf{\ipa{ʈʂʰo˧\textsubscript{b}}}}}{}
\textcolor{teal}{\mytextsc{verbe}} \hspace{4pt} Ton~: M\textsubscript{b}.

Lire à haute voix.
\textcolor{brown}{\zh{朗读。}}


\begin{exe}
\sn \textcolor{darkblue}{\textbf{\ipa{le˧-ʈʂʰo˧-ze˧}}}
\trans \mytextsc{accomp} \string_ \mytextsc{pfv}
\trans \textit{\textcolor{brown}{\zh{朗读了}}}
\end{exe}
\begin{exe}
\sn \textcolor{darkblue}{\textbf{\ipa{le˧-ʈʂʰo˧-le˧-se˩}}}
\trans (j'ai) fini de lire
\trans \textit{\textcolor{brown}{\zh{朗读完了。}}}
\end{exe}
\begin{exe}
\sn \textcolor{darkblue}{\textbf{\ipa{tʰæ˧ɻæ˩ ʈʂʰo˩}}}
\trans lire un livre
\trans \textit{\textcolor{brown}{\zh{朗读一本书}}}
\end{exe}
\begin{exe}
\sn \textcolor{darkblue}{\textbf{\ipa{ʈʂʰo˧\textasciitilde{}ʈʂʰo˧}}}
\trans \mytextsc{red}
\trans \textit{\textcolor{brown}{\zh{\mytextsc{重叠}}}}
\end{exe}


\paragraph{\hspace{-0.5cm} {\Large \textcolor{darkblue}{\textbf{\ipa{ʈʂʰo˧bɤ\#˥}}}}} \hypertarget{t`s`\string_ho\string_Mb7\#\string_T1}{}
\markboth{\textcolor{darkblue}{\textbf{\ipa{ʈʂʰo˧bɤ\#˥}}}}{}
\textcolor{teal}{\mytextsc{nom}} \hspace{4pt} Ton~: \#H.

Vêtement masculin, que les hommes portaient à partir de 13 ans: sorte de veste serrée à la ceinture, qu'on portait sur la chemise lors des grandes occasions: mariage, invitations….
\textcolor{brown}{\zh{男上衣。}}


 \mytextsc{clf}~: \textcolor{darkblue}{\textbf{\ipa{ɭɯ˧˥}}} 

\paragraph{\hspace{-0.5cm} {\Large \textcolor{darkblue}{\textbf{\ipa{ʈʂʰo˧bv̩˩}}}}} \hypertarget{t`s`\string_ho\string_Mbv\string_=\string_B1}{}
\markboth{\textcolor{darkblue}{\textbf{\ipa{ʈʂʰo˧bv̩˩}}}}{}
\textcolor{teal}{\mytextsc{nom}} \hspace{4pt} Ton~: L\#.

Acore odorant, jonc odorant, \textit{Acorus calamus}: plante herbacée aquatique, pérenne, rhizomateuse.
\textcolor{brown}{\zh{菖蒲。}}


 \mytextsc{clf}~: \textcolor{darkblue}{\textbf{\ipa{dzi˩}}} 

\paragraph{\hspace{-0.5cm} {\Large \textcolor{darkblue}{\textbf{\ipa{ʈʂʰo˧do˩}}}}} \hypertarget{t`s`\string_ho\string_Mdo\string_B1}{}
\markboth{\textcolor{darkblue}{\textbf{\ipa{ʈʂʰo˧do˩}}}}{}
\textcolor{teal}{\mytextsc{nom}} \hspace{4pt} Ton~: L\#.

Petite éminence juste à côté du foyer, en contrebas de l'autel où on offre des cadeaux aux ancêtres; c'est sur cette petite éminence qu'on dépose un peu de nourriture au début de chaque repas, en offrande aux ancêtres.
\textcolor{brown}{\zh{火塘上面祖先灵位。}}


 \mytextsc{clf}~: \textcolor{darkblue}{\textbf{\ipa{ɭɯ˧}}} 

\paragraph{\hspace{-0.5cm} {\Large \textcolor{darkblue}{\textbf{\ipa{ʈʂʰo˧lo\#˥}}}}} \hypertarget{t`s`\string_ho\string_Mlo\#\string_T1}{}
\markboth{\textcolor{darkblue}{\textbf{\ipa{ʈʂʰo˧lo\#˥}}}}{}
\textcolor{teal}{\mytextsc{nom}} \hspace{4pt} Ton~: \#H.

Grande poêle à fond plat, diamètre un peu supérieur à 50 cm, pour frire des aliments (galettes de pomme de terre, fèves).
\textcolor{brown}{\zh{平底大锅(直径大概半米),用来煎洋芋饼等等。}}


 \mytextsc{clf}~: \textcolor{darkblue}{\textbf{\ipa{ɭɯ˧}}} 

\paragraph{\hspace{-0.5cm} {\Large \textcolor{darkblue}{\textbf{\ipa{ʈʂʰɻ}}}}} \hypertarget{t`s`\string_hr£`1}{}
\markboth{\textcolor{darkblue}{\textbf{\ipa{ʈʂʰɻ}}}}{}
\textcolor{teal}{\mytextsc{idéophone}} \hspace{4pt} Ton~: .

Bruit de l'eau qui crépite au contact d'un métal rougi ou de bois incandescent: psshhhh!
\textcolor{brown}{\zh{形声词:噗嗤!(水浇在很热的金属上的声音)。}}




\paragraph{\hspace{-0.5cm} {\Large \textcolor{darkblue}{\textbf{\ipa{ʈʂʰɻ̍˧}}}}} \hypertarget{t`s`\string_hr£`̍\string_M1}{}
\markboth{\textcolor{darkblue}{\textbf{\ipa{ʈʂʰɻ̍˧}}}}{}
\textcolor{teal}{\mytextsc{nom}} \hspace{4pt} Ton~: M.

Soc de l'araire.
\textcolor{brown}{\zh{铧头,犁铧。}}


\begin{exe}
\sn \textcolor{darkblue}{\textbf{\ipa{ʈʂʰɻ̍˧ ʈʂʰɯ˧-ɭɯ˧}}}
\trans \mytextsc{n}+\mytextsc{dem}+\mytextsc{clf}
\trans \textit{\textcolor{brown}{\zh{这把铧头}}}
\end{exe}
 \mytextsc{clf}~: \textcolor{darkblue}{\textbf{\ipa{ɭɯ˧}}} 

\paragraph{\hspace{-0.5cm} {\Large \textcolor{darkblue}{\textbf{\ipa{ʈʂʰɻ̍˧˥}}} \textsubscript{1}}} \hypertarget{t`s`\string_hr£`̍\string_M\string_T1}{}
\markboth{\textcolor{darkblue}{\textbf{\ipa{ʈʂʰɻ̍˧˥}}} \textsubscript{1}}{}
\textcolor{teal}{\mytextsc{verbe}} \hspace{4pt} Ton~: MH.

Empoigner, prendre en main, saisir, tenir fermement (ex.: couteau); serrer, crisper (le poing).
\textcolor{brown}{\zh{握 (握刀把)。}}


\begin{exe}
\sn \textcolor{darkblue}{\textbf{\ipa{sɯ˩tʰi˩˥ | (ɖɯ˧)-nɑ˧ | tʰi˧-ʈʂʰɻ̍˧˥ (+dʑo˩)}}}
\trans empoigner un couteau
\trans \textit{\textcolor{brown}{\zh{手里握刀}}}
\end{exe}
\begin{exe}
\sn \textcolor{darkblue}{\textbf{\ipa{ʈʂʰɻ̍˧ mɤ˧-bi˧!}}}
\trans je ne veux pas empoigner/ pas question que j'empoigne (ce couteau,…)
\trans \textit{\textcolor{brown}{\zh{我不要拿(刀)!}}}
\end{exe}
\begin{exe}
\sn \textcolor{darkblue}{\textbf{\ipa{lo˩qʰwɤ˧ ʈʂʰɻ̍˩\textasciitilde{}ʈʂʰɻ̍˩ |}}}
\trans serrer le poing
\trans \textit{\textcolor{brown}{\zh{攥紧拳头}}}
\end{exe}


\paragraph{\hspace{-0.5cm} {\Large \textcolor{darkblue}{\textbf{\ipa{ʈʂʰɻ̍˧˥}}} \textsubscript{2}}} \hypertarget{t`s`\string_hr£`̍\string_M\string_T2}{}
\markboth{\textcolor{darkblue}{\textbf{\ipa{ʈʂʰɻ̍˧˥}}} \textsubscript{2}}{}
\textcolor{teal}{\mytextsc{nom}} \hspace{4pt} Ton~: MH.

Poumon.
\textcolor{brown}{\zh{肺。}}


 \mytextsc{clf}~: \textcolor{darkblue}{\textbf{\ipa{ɭɯ˧}}} 

\paragraph{\hspace{-0.5cm} {\Large \textcolor{darkblue}{\textbf{\ipa{ʈʂʰɻ̍˧˥\textsubscript{a}}}}}} \hypertarget{t`s`\string_hr£`̍\string_M\string_Ta1}{}
\markboth{\textcolor{darkblue}{\textbf{\ipa{ʈʂʰɻ̍˧˥\textsubscript{a}}}}}{}
\textcolor{teal}{\mytextsc{classificateur}} \hspace{4pt} Ton~: MH\textsubscript{a}.

Classificateur des boules/poignées: la quantité que l'on compacte en la serrant dans une main, par exemple une poignée de céréale cuite qu'on compresse en boule.
\textcolor{brown}{\zh{量词:团,掐。指的是一只手里能拿的量,压成团,如:手里拿煮熟的粮食,压成饭团。}}




\paragraph{\hspace{-0.5cm} {\Large \textcolor{darkblue}{\textbf{\ipa{-ʈʂʰɯ˥}}}}} \hypertarget{-t`s`\string_hM\string_T1}{}
\markboth{\textcolor{darkblue}{\textbf{\ipa{-ʈʂʰɯ˥}}}}{}
\textcolor{teal}{\mytextsc{suffixe}} \hspace{4pt} Ton~: \#H.

Focalisateur; grammaticalisé à partir du démonstratif proximal.
\textcolor{brown}{\zh{\mytextsc{主题(°指示}.近指)。}}


\textit{Voir~:} \hyperlink{t`s`\string_hM\string_T1}{\textcolor{darkblue}{\textbf{\ipa{ʈʂʰɯ˥}}} \textsubscript{1}} \textit{Voir~:} \hyperlink{t`s`\string_hM\string_T2}{\textcolor{darkblue}{\textbf{\ipa{ʈʂʰɯ˥}}} \textsubscript{2}} 

\paragraph{\hspace{-0.5cm} {\Large \textcolor{darkblue}{\textbf{\ipa{ʈʂʰɯ˥}}} \textsubscript{1}}} \hypertarget{t`s`\string_hM\string_T1}{}
\markboth{\textcolor{darkblue}{\textbf{\ipa{ʈʂʰɯ˥}}} \textsubscript{1}}{}
\textcolor{teal}{\mytextsc{pronom}} \hspace{4pt} Ton~: \#H.

Démonstratif proximal, qui forme un couple avec le démonstratif distal.
\textcolor{brown}{\zh{这\mytextsc{指示}.近指。}}


\begin{exe}
\sn \textcolor{darkblue}{\textbf{\ipa{ʈʂʰɯ˧ ɲi˥!}}}
\trans C'est ça!
\trans \textit{\textcolor{brown}{\zh{是这个! / 对的!}}}
\end{exe}
\begin{exe}
\sn \textcolor{darkblue}{\textbf{\ipa{ʈʂʰɯ˧-v̩\#˥}}}
\trans celui-ci (\mytextsc{dem}\string_prox-\mytextsc{clf}.individu)
\trans \textit{\textcolor{brown}{\zh{这个}}}
\end{exe}
\begin{exe}
\sn \textcolor{darkblue}{\textbf{\ipa{ʈʂʰɯ˧=ɻæ˥}}}
\trans \mytextsc{pl}: ces choses-là
\trans \textit{\textcolor{brown}{\zh{这些}}}
\end{exe}
\textit{Voir~:} \hyperlink{t`s`\string_hM\string_T2}{\textcolor{darkblue}{\textbf{\ipa{ʈʂʰɯ˥}}} \textsubscript{2}} \textit{Voir~:} \hyperlink{-t`s`\string_hM\string_T1}{\textcolor{darkblue}{\textbf{\ipa{-ʈʂʰɯ˥}}}} 

\paragraph{\hspace{-0.5cm} {\Large \textcolor{darkblue}{\textbf{\ipa{ʈʂʰɯ˥}}} \textsubscript{2}}} \hypertarget{t`s`\string_hM\string_T2}{}
\markboth{\textcolor{darkblue}{\textbf{\ipa{ʈʂʰɯ˥}}} \textsubscript{2}}{}
\textcolor{teal}{\mytextsc{pronom}} \hspace{4pt} Ton~: \#H.

Pronom de troisième personne du singulier.
\textcolor{brown}{\zh{他。}}


\begin{exe}
\sn \textcolor{darkblue}{\textbf{\ipa{ʈʂʰɯ˧ ɲi˥!}}}
\trans C'est elle/lui!
\trans \textit{\textcolor{brown}{\zh{是他!}}}
\end{exe}
\textit{Voir~:} \hyperlink{t`s`\string_hM\string_T1}{\textcolor{darkblue}{\textbf{\ipa{ʈʂʰɯ˥}}} \textsubscript{1}} \textit{Voir~:} \hyperlink{-t`s`\string_hM\string_T1}{\textcolor{darkblue}{\textbf{\ipa{-ʈʂʰɯ˥}}}} 

\paragraph{\hspace{-0.5cm} {\Large \textcolor{darkblue}{\textbf{\ipa{ʈʂʰɯ˧-gɤ˧}}}}} \hypertarget{t`s`\string_hM\string_M-g7\string_M1}{}
\markboth{\textcolor{darkblue}{\textbf{\ipa{ʈʂʰɯ˧-gɤ˧}}}}{}
\textcolor{teal}{\mytextsc{adverbe}} \hspace{4pt} Ton~: M.

Ici, à cet endroit-ci.
\textcolor{brown}{\zh{这里。}}


\textit{Voir~:} \hyperlink{t`s`\string_hM\string_Mgi\#\string_T1}{\textcolor{darkblue}{\textbf{\ipa{ʈʂʰɯ˧gi\#˥}}}} 

\paragraph{\hspace{-0.5cm} {\Large \textcolor{darkblue}{\textbf{\ipa{ʈʂʰɯ˧gi\#˥}}}}} \hypertarget{t`s`\string_hM\string_Mgi\#\string_T1}{}
\markboth{\textcolor{darkblue}{\textbf{\ipa{ʈʂʰɯ˧gi\#˥}}}}{}
\textcolor{teal}{\mytextsc{adverbe}} \hspace{4pt} Ton~: \#H.

Ici, à cet endroit-ci.
\textcolor{brown}{\zh{这边。}}


\textit{Voir~:} \hyperlink{t`s`\string_hM\string_M-g7\string_M1}{\textcolor{darkblue}{\textbf{\ipa{ʈʂʰɯ˧-gɤ˧}}}} 

\paragraph{\hspace{-0.5cm} {\Large \textcolor{darkblue}{\textbf{\ipa{ʈʂʰɯ˧ne˧-ʝi˥}}}}} \hypertarget{t`s`\string_hM\string_Mne\string_M-j££i\string_T1}{}
\markboth{\textcolor{darkblue}{\textbf{\ipa{ʈʂʰɯ˧ne˧-ʝi˥}}}}{}
\textcolor{teal}{\mytextsc{adverbe}} \hspace{4pt} Ton~: MH\#.

Ainsi, de cette façon (adverbe de manière).
\textcolor{brown}{\zh{这样,这么。}}


\begin{exe}
\sn \textcolor{darkblue}{\textbf{\ipa{ʈʂʰɯ˧ne˧-ʝi˥ | le˧-ʐwɤ˩!}}}
\trans c'est comme ça qu'on dit!
\trans \textit{\textcolor{brown}{\zh{是这样讲的!}}}
\end{exe}
\begin{exe}
\sn \textcolor{darkblue}{\textbf{\ipa{ʈʂʰɯ˧ne˧-ʝi˥ | le˧-pi˥!}}}
\trans c'est comme ça qu'on parle!
\trans \textit{\textcolor{brown}{\zh{是这样说的!}}}
\end{exe}
\begin{exe}
\sn \textcolor{darkblue}{\textbf{\ipa{ʈʂʰɯ˧ne˧-ʝi˥ | le˧-ʝi˥!}}}
\trans c'est comme ça qu'on fait!
\trans \textit{\textcolor{brown}{\zh{是这样做的!}}}
\end{exe}


\paragraph{\hspace{-0.5cm} {\Large \textcolor{darkblue}{\textbf{\ipa{ʈʂʰɯ˧qɑ˧}}}}} \hypertarget{t`s`\string_hM\string_MqA\string_M1}{}
\markboth{\textcolor{darkblue}{\textbf{\ipa{ʈʂʰɯ˧qɑ˧}}}}{}
\textcolor{teal}{\mytextsc{adverbe}} \hspace{4pt} Ton~: .

Ensemble.
\textcolor{brown}{\zh{一起。}}




\paragraph{\hspace{-0.5cm} {\Large \textcolor{darkblue}{\textbf{\ipa{ʈʂʰɯ˧-qo˧}}}}} \hypertarget{t`s`\string_hM\string_M-qo\string_M1}{}
\markboth{\textcolor{darkblue}{\textbf{\ipa{ʈʂʰɯ˧-qo˧}}}}{}
\textcolor{teal}{\mytextsc{adverbe}} \hspace{4pt} Ton~: M.

Ici, à cet endroit-ci.
\textcolor{brown}{\zh{这里。}}




\paragraph{\hspace{-0.5cm} {\Large \textcolor{darkblue}{\textbf{\ipa{ʈʂʰɯ˧-sɯ˩-kv̩˩}}}}} \hypertarget{t`s`\string_hM\string_M-sM\string_B-kv\string_=\string_B1}{}
\markboth{\textcolor{darkblue}{\textbf{\ipa{ʈʂʰɯ˧-sɯ˩-kv̩˩}}}}{}
\textcolor{teal}{\mytextsc{pronom}} \hspace{4pt} Ton~: \mytextsc{L}.

3e personne du pluriel.
\textcolor{brown}{\zh{他们。}}


\begin{exe}
\sn \textcolor{darkblue}{\textbf{\ipa{ʈʂʰɯ˧-sɯ˩-kv̩˩-bv̩˩ | mɤ˧-sɯ˥!}}}
\trans On ne connaît pas leurs histoires!
\trans \textit{\textcolor{brown}{\zh{我们不知道他们的事!}}}
\end{exe}


\paragraph{\hspace{-0.5cm} {\Large \textcolor{darkblue}{\textbf{\ipa{ʈʂʰɯ˧ʂo˧}}}}} \hypertarget{t`s`\string_hM\string_Ms`o\string_M1}{}
\markboth{\textcolor{darkblue}{\textbf{\ipa{ʈʂʰɯ˧ʂo˧}}}}{}
\textcolor{teal}{\mytextsc{adverbe}} \hspace{4pt} Ton~: .

Ce matin (à partir du petit matin).
\textcolor{brown}{\zh{今天早上。}}




\paragraph{\hspace{-0.5cm} {\Large \textcolor{darkblue}{\textbf{\ipa{ʈʂʰɯ˧tɕi˩}}}}} \hypertarget{t`s`\string_hM\string_Mts£i\string_B1}{}
\markboth{\textcolor{darkblue}{\textbf{\ipa{ʈʂʰɯ˧tɕi˩}}}}{}
\textcolor{teal}{\mytextsc{pronom}} \hspace{4pt} Ton~: L\#.

Pronom de troisième personne pluriel.
\textcolor{brown}{\zh{他们。}}




\paragraph{\hspace{-0.5cm} {\Large \textcolor{darkblue}{\textbf{\ipa{ʈʂʰɯ˧=zɯ˩}}}}} \hypertarget{t`s`\string_hM\string_M=zM\string_B1}{}
\markboth{\textcolor{darkblue}{\textbf{\ipa{ʈʂʰɯ˧=zɯ˩}}}}{}
\textcolor{teal}{\mytextsc{pronom}} \hspace{4pt} Ton~: L\#.

Pronom de troisième personne duel: eux deux.
\textcolor{brown}{\zh{他们两个。}}




\paragraph{\hspace{-0.5cm} {\Large \textcolor{darkblue}{\textbf{\ipa{ʈʂʰv̩˩}}} \textsubscript{1}}} \hypertarget{t`s`\string_hv\string_=\string_B1}{}
\markboth{\textcolor{darkblue}{\textbf{\ipa{ʈʂʰv̩˩}}} \textsubscript{1}}{}
\textcolor{teal}{\mytextsc{verbe}} \hspace{4pt} Ton~: MH.

Achever, terminer.
\textcolor{brown}{\zh{完成。}}


\begin{exe}
\sn \textcolor{darkblue}{\textbf{\ipa{le˧-ʈʂʰv̩˩-se˩}}}
\trans \mytextsc{accomp} \string_ \mytextsc{cmpl}
\trans \textit{\textcolor{brown}{\zh{完成了}}}
\end{exe}
\begin{exe}
\sn \textcolor{darkblue}{\textbf{\ipa{tsʰi˧-ɲi˧-bv̩˧ | lo˧ | le˧-ʈʂʰv̩˩! | le˧-se˩-ze˩!}}}
\trans Le travail d'aujourd'hui... on tourne la page! Il est fini!
\trans \textit{\textcolor{brown}{\zh{今天的工作完成了!就算完工了吧!}}}
\end{exe}


\paragraph{\hspace{-0.5cm} {\Large \textcolor{darkblue}{\textbf{\ipa{ʈʂʰv̩˩}}} \textsubscript{2}}} \hypertarget{t`s`\string_hv\string_=\string_B2}{}
\markboth{\textcolor{darkblue}{\textbf{\ipa{ʈʂʰv̩˩}}} \textsubscript{2}}{}
\textcolor{teal}{\mytextsc{verbe}} \hspace{4pt} Ton~: L.

Mettre à part.
\textcolor{brown}{\zh{除开。}}


\begin{exe}
\sn \textcolor{darkblue}{\textbf{\ipa{gɤ˩-ʈʂʰv̩˧, | mv̩˩-ʈʂʰv̩˧-tsæ˩-ɲi˩}}}
\trans laisser à part, distinguer, ne pas mettre ensemble, ne pas fourrer dans le même sac
\trans \textit{\textcolor{brown}{\zh{不算在里面、不算在一起}}}
\end{exe}
\begin{exe}
\sn \textcolor{darkblue}{\textbf{\ipa{no˧-bv̩˧ | gɤ˩-ʈʂʰv̩˧! | njɤ˧-bv̩˧, | mv̩˩-ʈʂʰv̩˧!}}}
\trans Ce qui est à toi est à toi; ce qui est à moi est à moi!
\trans \textit{\textcolor{brown}{\zh{你的算你的,我的算我的!}}}
\end{exe}


\paragraph{\hspace{-0.5cm} {\Large \textcolor{darkblue}{\textbf{\ipa{ʈʂʰv̩˧}}}}} \hypertarget{t`s`\string_hv\string_=\string_M1}{}
\markboth{\textcolor{darkblue}{\textbf{\ipa{ʈʂʰv̩˧}}}}{}
\textcolor{teal}{\mytextsc{nom}} \hspace{4pt} Ton~: M.

Repas du matin/ petit déjeuner.
\textcolor{brown}{\zh{早饭。}}


\begin{exe}
\sn \textcolor{darkblue}{\textbf{\ipa{ʈʂʰv̩˧ dzɯ˧(-ze˩)}}}
\trans prendre le petit déjeuner
\trans \textit{\textcolor{brown}{\zh{吃早饭}}}
\end{exe}
\begin{exe}
\sn \textcolor{darkblue}{\textbf{\ipa{bæ˧qʰæ˧ ʈʂʰv̩\#˥}}}
\trans le petit déjeuner qu'on prend au retour de la cérémonie de crémation: revenant du lieu où a eu lieu la crémation, les invités, en nombre relativement restreint, font une pause dans la maison du défunt, où on leur offre une collation avant qu'ils ne s'en retournent.
\trans \textit{\textcolor{brown}{\zh{丧礼早餐:参加火葬仪式的人留在去世的人家,一起吃一点早饭再回家。}}}
\end{exe}


\paragraph{\hspace{-0.5cm} {\Large \textcolor{darkblue}{\textbf{\ipa{ʈʂʰv̩˧˥}}}}} \hypertarget{t`s`\string_hv\string_=\string_M\string_T1}{}
\markboth{\textcolor{darkblue}{\textbf{\ipa{ʈʂʰv̩˧˥}}}}{}
\textcolor{teal}{\mytextsc{verbe}} \hspace{4pt} Ton~: MH.

Ajouter de l’eau, verser de l'eau.
\textcolor{brown}{\zh{掺和。}}


\begin{exe}
\sn \textcolor{darkblue}{\textbf{\ipa{le˧-ʈʂʰv̩˧-ze˥}}}
\trans \mytextsc{accomp} \string_ \mytextsc{pfv}
\trans \textit{\textcolor{brown}{\zh{\mytextsc{accomp} \string_ \mytextsc{pfv}}}}
\end{exe}
\begin{exe}
\sn \textcolor{darkblue}{\textbf{\ipa{dʑɯ˩ ʈʂʰv̩˩˥}}}
\trans ajouter de l'eau (dans une marmite, ...)
\trans \textit{\textcolor{brown}{\zh{加水(如:往锅里添加水)}}}
\end{exe}


\paragraph{\hspace{-0.5cm} {\Large \textcolor{darkblue}{\textbf{\ipa{ʈʂʰv̩˩\textsubscript{a}}}}}} \hypertarget{t`s`\string_hv\string_=\string_Ba1}{}
\markboth{\textcolor{darkblue}{\textbf{\ipa{ʈʂʰv̩˩\textsubscript{a}}}}}{}
\textcolor{teal}{\mytextsc{verbe}} \hspace{4pt} Ton~: L\textsubscript{a}.

Teindre.
\textcolor{brown}{\zh{染。}}


\begin{exe}
\sn \textcolor{darkblue}{\textbf{\ipa{mɤ˧-ʈʂʰv̩˩}}}
\trans \mytextsc{neg}
\trans \textit{\textcolor{brown}{\zh{\mytextsc{neg}}}}
\end{exe}
\begin{exe}
\sn \textcolor{darkblue}{\textbf{\ipa{ʈʂʰv̩˩ mɤ˩-bi˩˥!}}}
\trans \string_ \mytextsc{neg} \mytextsc{fut}\string_imm
\trans \textit{\textcolor{brown}{\zh{\string_ \mytextsc{neg} \mytextsc{fut}\string_imm}}}
\end{exe}
\begin{exe}
\sn \textcolor{darkblue}{\textbf{\ipa{tso˧\textasciitilde{}tso˧ ʈʂʰv̩˥}}}
\trans teindre des choses
\trans \textit{\textcolor{brown}{\zh{染东西}}}
\end{exe}


\paragraph{\hspace{-0.5cm} {\Large \textcolor{darkblue}{\textbf{\ipa{ʈʂʰv̩˧dʑɯ˧}}}}} \hypertarget{t`s`\string_hv\string_=\string_Mdz£M\string_M1}{}
\markboth{\textcolor{darkblue}{\textbf{\ipa{ʈʂʰv̩˧dʑɯ˧}}}}{}
\textcolor{teal}{\mytextsc{nom}} \hspace{4pt} Ton~: M.

Teinture.
\textcolor{brown}{\zh{染料。}}


\begin{exe}
\sn \textcolor{darkblue}{\textbf{\ipa{dʑi˧hṽ˧-ʈʂʰv̩˧dʑɯ˧}}}
\trans teinture pour vêtements
\trans \textit{\textcolor{brown}{\zh{衣服染料}}}
\end{exe}
\begin{exe}
\sn \textcolor{darkblue}{\textbf{\ipa{ʈʂʰv̩˧dʑɯ˧ | hṽ˩-hĩ˩˥}}}
\trans teinture rouge
\trans \textit{\textcolor{brown}{\zh{红色的染料}}}
\end{exe}
 \mytextsc{clf}~: \textcolor{darkblue}{\textbf{\ipa{kʰwɤ˥}}} 

\paragraph{\hspace{-0.5cm} {\Large \textcolor{darkblue}{\textbf{\ipa{ʈʂʰv̩˧mi˧}}}}} \hypertarget{t`s`\string_hv\string_=\string_Mmi\string_M1}{}
\markboth{\textcolor{darkblue}{\textbf{\ipa{ʈʂʰv̩˧mi˧}}}}{}
\textcolor{teal}{\mytextsc{nom}} \hspace{4pt} Ton~: M.

Épouse, femme.
\textcolor{brown}{\zh{太太、老婆、媳妇。}}


 \mytextsc{clf}~: \textcolor{darkblue}{\textbf{\ipa{v̩˧}}} 

\paragraph{\hspace{-0.5cm} {\Large \textcolor{darkblue}{\textbf{\ipa{ʈʂʰv̩˧ɻ̍˧qʰv̩\#˥}}}}} \hypertarget{t`s`\string_hv\string_=\string_Mr£`̍\string_Mq\string_hv\string_=\#\string_T1}{}
\markboth{\textcolor{darkblue}{\textbf{\ipa{ʈʂʰv̩˧ɻ̍˧qʰv̩\#˥}}}}{}
\textcolor{teal}{\mytextsc{nom}} \hspace{4pt} Ton~: \#H.

Fourmilière.
\textcolor{brown}{\zh{蚂蚁巢。}}


\begin{exe}
\sn \textcolor{darkblue}{\textbf{\ipa{ʈʂʰv̩˧ɻ̍˧qʰv̩˧ ɲi˥!}}}
\trans c'est une fourmilière!
\trans \textit{\textcolor{brown}{\zh{是蚂蚁巢!}}}
\end{exe}
 \mytextsc{clf}~: \textcolor{darkblue}{\textbf{\ipa{ɭɯ˧}}} \textit{Voir~:} \hyperlink{t`s`\string_hv\string_=\string_Mr£`̍\string_T\$1}{\textcolor{darkblue}{\textbf{\ipa{ʈʂʰv̩˧ɻ̍˥\$}}}} 

\paragraph{\hspace{-0.5cm} {\Large \textcolor{darkblue}{\textbf{\ipa{ʈʂʰv̩˧ɻ̍˥\$}}}}} \hypertarget{t`s`\string_hv\string_=\string_Mr£`̍\string_T\$1}{}
\markboth{\textcolor{darkblue}{\textbf{\ipa{ʈʂʰv̩˧ɻ̍˥\$}}}}{}
\textcolor{teal}{\mytextsc{nom}} \hspace{4pt} Ton~: H\$.

Fourmi.
\textcolor{brown}{\zh{蚂蚁。}}


\begin{exe}
\sn \textcolor{darkblue}{\textbf{\ipa{ʈʂʰv̩˧ɻ̍˧ | tɕi˩-hĩ˩˥}}}
\trans petite fourmi
\trans \textit{\textcolor{brown}{\zh{小蚂蚁}}}
\end{exe}
 \mytextsc{clf}~: \textcolor{darkblue}{\textbf{\ipa{mi˩}}} 

\paragraph{\hspace{-0.5cm} {\Large \textcolor{darkblue}{\textbf{\ipa{}}}}} \hypertarget{t`s`\string_hv\string_=\string_B~t`s`\string_hv\string_=\string_B\string_M1}{}
\markboth{\textcolor{darkblue}{\textbf{\ipa{}}}}{}
\textcolor{teal}{\mytextsc{verbe}} \hspace{4pt} Ton~: .

Frotter, essuyer.
\textcolor{brown}{\zh{擦。}}


\begin{exe}
\sn \textcolor{darkblue}{\textbf{\ipa{ʈʂʰv̩˩\textasciitilde{}ʈʂʰv̩˩-ze˧}}}
\trans \mytextsc{red} \mytextsc{pfv}
\trans \textit{\textcolor{brown}{\zh{\mytextsc{red} \mytextsc{pfv}}}}
\end{exe}
\begin{exe}
\sn \textcolor{darkblue}{\textbf{\ipa{le˧-ʈʂʰv̩˩\textasciitilde{}ʈʂʰv̩˩-ze˩}}}
\trans \mytextsc{accomp} \string_ \mytextsc{pfv}
\trans \textit{\textcolor{brown}{\zh{\mytextsc{accomp} \string_ \mytextsc{pfv}}}}
\end{exe}
\begin{exe}
\sn \textcolor{darkblue}{\textbf{\ipa{tso˧\textasciitilde{}tso˧ ʈʂʰv̩˥\textasciitilde{}ʈʂʰv̩˩}}}
\trans frotter des choses
\trans \textit{\textcolor{brown}{\zh{擦东西}}}
\end{exe}


\paragraph{\hspace{-0.5cm} {\Large \textcolor{darkblue}{\textbf{\ipa{ʈʂʰwæ˧\textsubscript{a}}}} \textsubscript{1}}} \hypertarget{t`s`\string_hw\{\string_Ma1}{}
\markboth{\textcolor{darkblue}{\textbf{\ipa{ʈʂʰwæ˧\textsubscript{a}}}} \textsubscript{1}}{}
\textcolor{teal}{\mytextsc{verbe}} \hspace{4pt} Ton~: M\textsubscript{a}.

Pourrir.
\textcolor{brown}{\zh{腐烂。}}


\begin{exe}
\sn \textcolor{darkblue}{\textbf{\ipa{ʈʂʰwæ˧-ze˧}}}
\trans \mytextsc{pfv}
\trans \textit{\textcolor{brown}{\zh{烂了}}}
\end{exe}
\begin{exe}
\sn \textcolor{darkblue}{\textbf{\ipa{le˧-ʈʂʰwæ˧-ze˧}}}
\trans \mytextsc{accomp} \string_ \mytextsc{pfv}
\trans \textit{\textcolor{brown}{\zh{\mytextsc{accomp} \string_ \mytextsc{pfv}}}}
\end{exe}
\begin{exe}
\sn \textcolor{darkblue}{\textbf{\ipa{hĩ˧-ɳɯ˩ | mɤ˧-dzɯ˥, | le˧-ʈʂʰwæ˧-ze˧! |}}}
\trans On a oublié de la manger, et maintenant c'est pourri! (au sujet d'une pastèque qui a traîné dans le garde-manger et est maintenant incomestible)
\trans \textit{\textcolor{brown}{\zh{没人吃,就烂了!(一个西瓜被忘记在橱柜里,就腐烂了)}}}
\end{exe}


\paragraph{\hspace{-0.5cm} {\Large \textcolor{darkblue}{\textbf{\ipa{ʈʂʰwæ˧\textsubscript{a}}}} \textsubscript{2}}} \hypertarget{t`s`\string_hw\{\string_Ma2}{}
\markboth{\textcolor{darkblue}{\textbf{\ipa{ʈʂʰwæ˧\textsubscript{a}}}} \textsubscript{2}}{}
\textcolor{teal}{\mytextsc{verbe}} \hspace{4pt} Ton~: M\textsubscript{a}.

Se réveiller.
\textcolor{brown}{\zh{醒来。}}


\begin{exe}
\sn \textcolor{darkblue}{\textbf{\ipa{le˧-ʈʂʰwæ˧-ze˧}}}
\trans \mytextsc{accomp} \string_ \mytextsc{pfv}
\trans \textit{\textcolor{brown}{\zh{\mytextsc{accomp} \string_ \mytextsc{pfv}}}}
\end{exe}
\begin{exe}
\sn \textcolor{darkblue}{\textbf{\ipa{gɤ˩ʈʂʰwæ˧}}}
\trans se réveiller
\trans \textit{\textcolor{brown}{\zh{醒来}}}
\end{exe}
\begin{exe}
\sn \textcolor{darkblue}{\textbf{\ipa{gɤ˩ʈʂʰwæ˧-ze˧!}}}
\trans (il) s'est réveillé!
\trans \textit{\textcolor{brown}{\zh{醒来了!}}}
\end{exe}


\paragraph{\hspace{-0.5cm} {\Large \textcolor{darkblue}{\textbf{\ipa{ʈʂʰwæ˧-bv̩˧nv̩\#˥}}}}} \hypertarget{t`s`\string_hw\{\string_M-bv\string_=\string_Mnv\string_=\#\string_T1}{}
\markboth{\textcolor{darkblue}{\textbf{\ipa{ʈʂʰwæ˧-bv̩˧nv̩\#˥}}}}{}
\textcolor{teal}{\mytextsc{adjectif}} \hspace{4pt} Ton~: \#H.

Puant, à l'odeur de pourriture.
\textcolor{brown}{\zh{食物变味,有臭味道了。}}


\begin{exe}
\sn \textcolor{darkblue}{\textbf{\ipa{ʈʂʰwæ˧-bv̩˧nv̩˧ ɲi˥!}}}
\trans Ca sent le pourri! / Ca pue la pourriture! / C'est vraiment malodorant!
\trans \textit{\textcolor{brown}{\zh{臭了、有臭味道了}}}
\end{exe}


\paragraph{\hspace{-0.5cm} {\Large \textcolor{darkblue}{\textbf{\ipa{ʈʂʰwæ˧kɯ˧}}}}} \hypertarget{t`s`\string_hw\{\string_MkM\string_M1}{}
\markboth{\textcolor{darkblue}{\textbf{\ipa{ʈʂʰwæ˧kɯ˧}}}}{}
\textcolor{teal}{\mytextsc{nom}} \hspace{4pt} Ton~: M.

Filet.
\textcolor{brown}{\zh{网。}}


\begin{exe}
\sn \textcolor{darkblue}{\textbf{\ipa{ʈʂʰwæ˧kɯ˧ tʰv̩˧-nɑ˩}}}
\trans \mytextsc{n}+\mytextsc{dem}+\mytextsc{clf}
\trans \textit{\textcolor{brown}{\zh{这个网}}}
\end{exe}
 \mytextsc{clf}~: \textcolor{darkblue}{\textbf{\ipa{nɑ˧}}} 

\paragraph{\hspace{-0.5cm} {\Large \textcolor{darkblue}{\textbf{\ipa{ʈʂʰwæ˧tsɯ˧}}}}} \hypertarget{t`s`\string_hw\{\string_MtsM\string_M1}{}
\markboth{\textcolor{darkblue}{\textbf{\ipa{ʈʂʰwæ˧tsɯ˧}}}}{}
\textcolor{teal}{\mytextsc{nom}} \hspace{4pt} Ton~: M.

Fenêtre.
\textcolor{brown}{\zh{窗户。}}Dialecte chinois local~: \zh{窗子。}

 Emprunt~: chinois \textcolor{brown}{\zh{窗子}}

 \mytextsc{clf}~: \textcolor{darkblue}{\textbf{\ipa{nɑ˧}}} 

\paragraph{\hspace{-0.5cm} {\Large \textcolor{darkblue}{\textbf{\ipa{ʈʂʰwæ˧ʈʂʰwæ˧}}}}} \hypertarget{t`s`\string_hw\{\string_Mt`s`\string_hw\{\string_M1}{}
\markboth{\textcolor{darkblue}{\textbf{\ipa{ʈʂʰwæ˧ʈʂʰwæ˧}}}}{}
\textcolor{teal}{\mytextsc{nom}} \hspace{4pt} Ton~: M.

Cymbales (mot emprunté au chinois).
\textcolor{brown}{\zh{钹。}}


 \mytextsc{clf}~: \textcolor{darkblue}{\textbf{\ipa{nɑ˧}}} 

\paragraph{\hspace{-0.5cm} {\Large \textcolor{darkblue}{\textbf{\ipa{ʈʂʰwæ˩\textsubscript{a}}}}}} \hypertarget{t`s`\string_hw\{\string_Ba1}{}
\markboth{\textcolor{darkblue}{\textbf{\ipa{ʈʂʰwæ˩\textsubscript{a}}}}}{}
\textcolor{teal}{\mytextsc{adjectif}} \hspace{4pt} Ton~: L\textsubscript{a}.

Rapide.
\textcolor{brown}{\zh{快(动作快,跑得快)。}}


\begin{exe}
\sn \textcolor{darkblue}{\textbf{\ipa{ʈʂʰwæ˩-hĩ˩˥}}}
\trans \mytextsc{rel}/\mytextsc{nmlz}
\trans \textit{\textcolor{brown}{\zh{快的}}}
\end{exe}
\begin{exe}
\sn \textcolor{darkblue}{\textbf{\ipa{ɲi˧to˧ ʈʂʰwæ˩}}}
\trans bavard, qui parle sans réfléchir suffisamment
\trans \textit{\textcolor{brown}{\zh{嘴快}}}
\end{exe}


\paragraph{\hspace{-0.5cm} {\Large \textcolor{darkblue}{\textbf{\ipa{ʈʂʰwæ˩tsʰɯ˩}}}}} \hypertarget{t`s`\string_hw\{\string_Bts\string_hM\string_B1}{}
\markboth{\textcolor{darkblue}{\textbf{\ipa{ʈʂʰwæ˩tsʰɯ˩}}}}{}
\textcolor{teal}{\mytextsc{verbe}} \hspace{4pt} Ton~: L.

Créer.
\textcolor{brown}{\zh{创造(汉语借词)。}}

 Emprunt~: chinois \textcolor{brown}{\zh{创造}}



\paragraph{\hspace{-0.5cm} {\Large \textcolor{darkblue}{\textbf{\ipa{ʈʂʰwæ˧˥}}} \textsubscript{1}}} \hypertarget{t`s`\string_hw\{\string_M\string_T1}{}
\markboth{\textcolor{darkblue}{\textbf{\ipa{ʈʂʰwæ˧˥}}} \textsubscript{1}}{}
\textcolor{teal}{\mytextsc{verbe}} \hspace{4pt} Ton~: MH.

Cacher: cacher un objet.
\textcolor{brown}{\zh{藏(东西)。}}




\paragraph{\hspace{-0.5cm} {\Large \textcolor{darkblue}{\textbf{\ipa{ʈʂʰwæ˧˥}}} \textsubscript{2}}} \hypertarget{t`s`\string_hw\{\string_M\string_T2}{}
\markboth{\textcolor{darkblue}{\textbf{\ipa{ʈʂʰwæ˧˥}}} \textsubscript{2}}{}
\textcolor{teal}{\mytextsc{verbe}} \hspace{4pt} Ton~: MH.

Enfoncer (un couteau) d'un geste brutal: poignarder, de haut en bas, avec force; insérer, planter, ficher (ex.: un couteau, une aiguille...).
\textcolor{brown}{\zh{插、戳。}}




\paragraph{\hspace{-0.5cm} {\Large \textcolor{darkblue}{\textbf{\ipa{ʈʂʰwæ˩˧}}}}} \hypertarget{t`s`\string_hw\{\string_B\string_M1}{}
\markboth{\textcolor{darkblue}{\textbf{\ipa{ʈʂʰwæ˩˧}}}}{}
\textcolor{teal}{\mytextsc{nom}} \hspace{4pt} Ton~: LM.

Bateau (emprunt chinois ancien); désormais, c'est le terme employé pour les grands bateaux, par ex. les barges pour passer le Yangtze; ce sont des Chinois et des Hmong qui auraient installé le bateau permettant de passer le Yangtze, d'où l'utilisation d'un mot chinois.
\textcolor{brown}{\zh{船(汉语借词)。}}

 Emprunt~: chinois \textcolor{brown}{\zh{船}}

 \mytextsc{clf}~: \textcolor{darkblue}{\textbf{\ipa{nɑ˧}}} 

\paragraph{\hspace{-0.5cm} {\Large \textcolor{darkblue}{\textbf{\ipa{ʈʂʰwɤ˧tsʰi˧˥}}}}} \hypertarget{t`s`\string_hw7\string_Mts\string_hi\string_M\string_T1}{}
\markboth{\textcolor{darkblue}{\textbf{\ipa{ʈʂʰwɤ˧tsʰi˧˥}}}}{}
\textcolor{teal}{\mytextsc{adjectif}} \hspace{4pt} Ton~: MH\#.

Étroit.
\textcolor{brown}{\zh{窄。}}




\paragraph{\hspace{-0.5cm} {\Large \textcolor{darkblue}{\textbf{\ipa{ʈʂʰwɤ˩}}}}} \hypertarget{t`s`\string_hw7\string_B1}{}
\markboth{\textcolor{darkblue}{\textbf{\ipa{ʈʂʰwɤ˩}}}}{}
\textcolor{teal}{\mytextsc{nom}} \hspace{4pt} Ton~: L.

Repas du soir, dîner.
\textcolor{brown}{\zh{晚饭。}}


\begin{exe}
\sn \textcolor{darkblue}{\textbf{\ipa{ʈʂʰwɤ˩ gv̩˩˥}}}
\trans cuisiner le dîner
\trans \textit{\textcolor{brown}{\zh{做晚饭}}}
\end{exe}
\begin{exe}
\sn \textcolor{darkblue}{\textbf{\ipa{ʈʂʰwɤ˩ tʰv̩˩˥}}}
\trans offrir à dîner, se charger du dîner (personne qui invite, pas nécessairement qui fait la cuisine elle-même)
\trans \textit{\textcolor{brown}{\zh{请吃晚饭,提供晚餐(不一定自己做:意思是提供原料)}}}
\end{exe}
\begin{exe}
\sn \textcolor{darkblue}{\textbf{\ipa{ʈʂʰwɤ˩ dzɯ˩˥}}}
\trans prendre le repas du soir
\trans \textit{\textcolor{brown}{\zh{吃晚饭}}}
\end{exe}


\newpage
\section*{\centering- \textcolor{darkblue}{\textbf{\ipa{u}}} -}
\paragraph{\hspace{-0.5cm} {\Large \textcolor{darkblue}{\textbf{\ipa{u˧}}}}} \hypertarget{u\string_M1}{}
\markboth{\textcolor{darkblue}{\textbf{\ipa{u˧}}}}{}
\textcolor{teal}{\mytextsc{pronom}} \hspace{4pt} Ton~: M.

Pronom de 1e personne, associatif: les miens. Cette racine n'apparaît qu'en combinaison avec un clitique pluriel ou associatif.
\textcolor{brown}{\zh{我家人、我家族。}}


\begin{exe}
\sn \textcolor{darkblue}{\textbf{\ipa{u˧=ɻ˩, ʈʂʰɯ˧=ɻ˩}}}
\trans mon clan, son clan : deux termes qui forment une opposition
\end{exe}
\begin{exe}
\sn \textcolor{darkblue}{\textbf{\ipa{u˧ɻ̍˩ | ə˧si˧}}}
\trans mon arrière-grand-mère
\trans \textit{\textcolor{brown}{\zh{我家祖母}}}
\end{exe}
\begin{exe}
\sn \textcolor{darkblue}{\textbf{\ipa{u˧=ɻæ˩, ʈʂʰɯ˧=ɻæ˩}}}
\trans Nous autres, eux : deux termes qui forment une opposition.
\end{exe}


\newpage
\section*{\centering- \textcolor{darkblue}{\textbf{\ipa{v}}} \textcolor{darkblue}{\textbf{\ipa{ṽ}}} -}
\paragraph{\hspace{-0.5cm} {\Large \textcolor{darkblue}{\textbf{\ipa{v̩˩}}}}} \hypertarget{v\string_=\string_B1}{}
\markboth{\textcolor{darkblue}{\textbf{\ipa{v̩˩}}}}{}
\textcolor{teal}{\mytextsc{verbe}} \hspace{4pt} Ton~: L\textsubscript{a}.

Embrasser, prendre dans ses bras (monosyllabe extrait de la forme \textcolor{darkblue}{\textbf{\ipa{/le˧-v̩˧~v̩˥/}}}).
\textcolor{brown}{\zh{搂(人的脖子)。}}


\begin{exe}
\sn \textcolor{darkblue}{\textbf{\ipa{ʁæ˧ | le˧-v̩˧\textasciitilde{}v̩˥}}}
\trans prendre le cou (de quelqu'un) dans son bras
\trans \textit{\textcolor{brown}{\zh{搂人的脖子}}}
\end{exe}
\begin{exe}
\sn \textcolor{darkblue}{\textbf{\ipa{ʁæ˧ | le˧-v̩˩}}}
\trans même sens
\trans \textit{\textcolor{brown}{\zh{同上}}}
\end{exe}
\begin{exe}
\sn \textcolor{darkblue}{\textbf{\ipa{ʁæ˧ v̩˥ se˩}}}
\trans marcher/se promener en tenant le cou de quelqu'un/enlacé avec quelqu'un
\trans \textit{\textcolor{brown}{\zh{互相搂着走}}}
\end{exe}


\paragraph{\hspace{-0.5cm} {\Large \textcolor{darkblue}{\textbf{\ipa{v̩˧}}}}} \hypertarget{v\string_=\string_M1}{}
\markboth{\textcolor{darkblue}{\textbf{\ipa{v̩˧}}}}{}
\textcolor{teal}{\mytextsc{classificateur}} \hspace{4pt} Ton~: M *.

Classificateur pour un individu humain; ne peut s'employer que pour le chiffre 'un', autrement dit soit au singulier, soit avec des nombres dont le dernier chiffre est 'un'.
\textcolor{brown}{\zh{量词:人(一个人)。只能用于单数。}}


\begin{exe}
\sn \textcolor{darkblue}{\textbf{\ipa{ɖɯ˧-v̩\#˥; ɖɯ˧-v̩˧ ɲi˥}}}
\trans 1 personne; c'est 1 personne (élicité pour vérifier le ton)
\trans \textit{\textcolor{brown}{\zh{一个人,是一个人(为了确认声调而问的短语)}}}
\end{exe}
\begin{exe}
\sn \textcolor{darkblue}{\textbf{\ipa{tsʰe˧ɖɯ˧-v̩˧}}}
\trans 11 personnes
\trans \textit{\textcolor{brown}{\zh{十一个人}}}
\end{exe}
\begin{exe}
\sn \textcolor{darkblue}{\textbf{\ipa{ɲi˧tsi˧ɖɯ˧-v̩˧}}}
\trans 21 personnes
\trans \textit{\textcolor{brown}{\zh{二十一个人}}}
\end{exe}
\begin{exe}
\sn \textcolor{darkblue}{\textbf{\ipa{so˧tsʰi˧ɖɯ˧-v̩˧}}}
\trans 31 personnes
\trans \textit{\textcolor{brown}{\zh{三十一个人}}}
\end{exe}
\begin{exe}
\sn \textcolor{darkblue}{\textbf{\ipa{ʐv̩˧tsʰi˩ɖɯ˩-v̩˩}}}
\trans 41 personnes
\trans \textit{\textcolor{brown}{\zh{四十一个人}}}
\end{exe}
\begin{exe}
\sn \textcolor{darkblue}{\textbf{\ipa{ŋwɤ˧tsʰi˩ɖɯ˩-v̩˩}}}
\trans 51 personnes
\trans \textit{\textcolor{brown}{\zh{五十一个人}}}
\end{exe}
\begin{exe}
\sn \textcolor{darkblue}{\textbf{\ipa{qʰv̩˧tsʰi˧ɖɯ˧-v̩˥}}}
\trans 61 personnes
\trans \textit{\textcolor{brown}{\zh{六十一个人}}}
\end{exe}
\begin{exe}
\sn \textcolor{darkblue}{\textbf{\ipa{ʂɯ˧tsʰi˩ɖɯ˩-v̩˩}}}
\trans 71 personnes
\trans \textit{\textcolor{brown}{\zh{七十一个人}}}
\end{exe}
\begin{exe}
\sn \textcolor{darkblue}{\textbf{\ipa{hõ˧tsʰi˩ɖɯ˩-v̩˩}}}
\trans 81 personnes
\trans \textit{\textcolor{brown}{\zh{八十一个人}}}
\end{exe}
\begin{exe}
\sn \textcolor{darkblue}{\textbf{\ipa{gv̩˧tsʰi˩ɖɯ˩-v̩˩}}}
\trans 91 personnes
\trans \textit{\textcolor{brown}{\zh{九十一个人}}}
\end{exe}


\paragraph{\hspace{-0.5cm} {\Large \textcolor{darkblue}{\textbf{\ipa{v̩˥}}}}} \hypertarget{v\string_=\string_T1}{}
\markboth{\textcolor{darkblue}{\textbf{\ipa{v̩˥}}}}{}
\textcolor{teal}{\mytextsc{nom}} \hspace{4pt} Ton~: \#H.

Casserole (terme générique).
\textcolor{brown}{\zh{锅。}}


 \mytextsc{clf}~: \textcolor{darkblue}{\textbf{\ipa{ɭɯ˧}}} 

\paragraph{\hspace{-0.5cm} {\Large \textcolor{darkblue}{\textbf{\ipa{v̩˥\textsubscript{b}}}}}} \hypertarget{v\string_=\string_Tb1}{}
\markboth{\textcolor{darkblue}{\textbf{\ipa{v̩˥\textsubscript{b}}}}}{}
\textcolor{teal}{\mytextsc{classificateur}} \hspace{4pt} Ton~: H\textsubscript{b}.

Auto-classificateur des casseroles; et classificateur des casserolées (utilisant la casserole comme mesure de quantité de nourriture, liquide ou solide).
\textcolor{brown}{\zh{量词:锅(一口),或锅的容量。}}




\paragraph{\hspace{-0.5cm} {\Large \textcolor{darkblue}{\textbf{\ipa{v̩˩dze˩}}}}} \hypertarget{v\string_=\string_Bdze\string_B1}{}
\markboth{\textcolor{darkblue}{\textbf{\ipa{v̩˩dze˩}}}}{}
\textcolor{teal}{\mytextsc{nom}} \hspace{4pt} Ton~: L.

Oiseau.
\textcolor{brown}{\zh{鸟。}}


\begin{exe}
\sn \textcolor{darkblue}{\textbf{\ipa{v̩˩dze˩-bi˥ | hṽ˧ ʑi˥}}}
\trans Sur l'oiseau, il y a des plumes.
\trans \textit{\textcolor{brown}{\zh{鸟(身)上有(羽)毛。}}}
\end{exe}
\begin{exe}
\sn \textcolor{darkblue}{\textbf{\ipa{v̩˩dze˩-mi˩}}}
\trans oiseau femelle
\trans \textit{\textcolor{brown}{\zh{母鸟}}}
\end{exe}
\begin{exe}
\sn \textcolor{darkblue}{\textbf{\ipa{v̩˩dze˩-pʰv̩˩}}}
\trans oiseau mâle
\trans \textit{\textcolor{brown}{\zh{公鸟}}}
\end{exe}
\begin{exe}
\sn \textcolor{darkblue}{\textbf{\ipa{v̩˩dze˩-zo˩}}}
\trans petit oiseau
\trans \textit{\textcolor{brown}{\zh{小鸟}}}
\end{exe}
 \mytextsc{clf}~: \textcolor{darkblue}{\textbf{\ipa{mi˩}}} 

\paragraph{\hspace{-0.5cm} {\Large \textcolor{darkblue}{\textbf{\ipa{v̩˩dze˩-kʰv̩˩}}}}} \hypertarget{v\string_=\string_Bdze\string_B-k\string_hv\string_=\string_B1}{}
\markboth{\textcolor{darkblue}{\textbf{\ipa{v̩˩dze˩-kʰv̩˩}}}}{}
\textcolor{teal}{\mytextsc{nom}} \hspace{4pt} Ton~: L.

Nid.
\textcolor{brown}{\zh{鸟窝,鸟巢。}}


\begin{exe}
\sn \textcolor{darkblue}{\textbf{\ipa{v̩˩dze˩kʰv̩˩ ɲi˥.}}}
\trans \string_ \mytextsc{cop}
\trans \textit{\textcolor{brown}{\zh{是鸟窝}}}
\end{exe}
 \mytextsc{clf}~: \textcolor{darkblue}{\textbf{\ipa{ɭɯ˧}}} 

\paragraph{\hspace{-0.5cm} {\Large \textcolor{darkblue}{\textbf{\ipa{v̩˧dʑo\#˥}}}}} \hypertarget{v\string_=\string_Mdz£o\#\string_T1}{}
\markboth{\textcolor{darkblue}{\textbf{\ipa{v̩˧dʑo\#˥}}}}{}
\textcolor{teal}{\mytextsc{nom}} \hspace{4pt} Ton~: \#H.

Wujiao (nom de village).
\textcolor{brown}{\zh{屋脚(村落名)。}}




\paragraph{\hspace{-0.5cm} {\Large \textcolor{darkblue}{\textbf{\ipa{v̩˩dʑɯ˩}}}}} \hypertarget{v\string_=\string_Bdz£M\string_B1}{}
\markboth{\textcolor{darkblue}{\textbf{\ipa{v̩˩dʑɯ˩}}}}{}
\textcolor{teal}{\mytextsc{nom}} \hspace{4pt} Ton~: L.

Soupe.
\textcolor{brown}{\zh{汤。}}


\begin{exe}
\sn \textcolor{darkblue}{\textbf{\ipa{æ˩ʂe˧-v̩˥dʑɯ˩}}}
\trans soupe de poulet
\trans \textit{\textcolor{brown}{\zh{鸡汤}}}
\end{exe}


\paragraph{\hspace{-0.5cm} {\Large \textcolor{darkblue}{\textbf{\ipa{v̩˧ko˧}}}}} \hypertarget{v\string_=\string_Mko\string_M1}{}
\markboth{\textcolor{darkblue}{\textbf{\ipa{v̩˧ko˧}}}}{}
\textcolor{teal}{\mytextsc{nom}} \hspace{4pt} Ton~: M.

Tortue.
\textcolor{brown}{\zh{乌龟(汉语借词)。}}




\paragraph{\hspace{-0.5cm} {\Large \textcolor{darkblue}{\textbf{\ipa{v̩˧lɑ˩-ʝi˩}}}}} \hypertarget{v\string_=\string_MlA\string_B-j££i\string_B1}{}
\markboth{\textcolor{darkblue}{\textbf{\ipa{v̩˧lɑ˩-ʝi˩}}}}{}
\textcolor{teal}{\mytextsc{verbe}} \hspace{4pt} Ton~: L\#-.

Faire du commerce.
\textcolor{brown}{\zh{做生意。}}




\paragraph{\hspace{-0.5cm} {\Large \textcolor{darkblue}{\textbf{\ipa{v̩˧lɑ˩-ʝi˩-hĩ˩-hĩ˩}}}}} \hypertarget{v\string_=\string_MlA\string_B-j££i\string_B-hi\string_~\string_B-hi\string_~\string_B1}{}
\markboth{\textcolor{darkblue}{\textbf{\ipa{v̩˧lɑ˩-ʝi˩-hĩ˩-hĩ˩}}}}{}
\textcolor{teal}{\mytextsc{nom}} \hspace{4pt} Ton~: L\#-.

Marchand.
\textcolor{brown}{\zh{商人。}}


\begin{exe}
\sn \textcolor{darkblue}{\textbf{\ipa{v̩˧lɑ˩-ʝi˩-hĩ˩}}}
\trans marchand
\trans \textit{\textcolor{brown}{\zh{商人}}}
\end{exe}
 \mytextsc{clf}~: \textcolor{darkblue}{\textbf{\ipa{v̩˧}}} 

\paragraph{\hspace{-0.5cm} {\Large \textcolor{darkblue}{\textbf{\ipa{v̩˧mi\#˥}}}}} \hypertarget{v\string_=\string_Mmi\#\string_T1}{}
\markboth{\textcolor{darkblue}{\textbf{\ipa{v̩˧mi\#˥}}}}{}
\textcolor{teal}{\mytextsc{nom}} \hspace{4pt} Ton~: \#H.

Grande casserole.
\textcolor{brown}{\zh{大锅。}}




\paragraph{\hspace{-0.5cm} {\Large \textcolor{darkblue}{\textbf{\ipa{v̩˩tsʰɤ˧˥}}}}} \hypertarget{v\string_=\string_Bts\string_h7\string_M\string_T1}{}
\markboth{\textcolor{darkblue}{\textbf{\ipa{v̩˩tsʰɤ˧˥}}}}{}
\textcolor{teal}{\mytextsc{nom}} \hspace{4pt} Ton~: LM+MH\#.

Légumes.
\textcolor{brown}{\zh{蔬菜。}}


\begin{exe}
\sn \textcolor{darkblue}{\textbf{\ipa{[Housebuilding2] v˩tsʰɤ˧-tsʰɑ˧nɑ˥}}}
\trans légumes frais. Littéralement 'légume de couleur sombre. L'expression ne renvoie pas à une espèce en particulier, mais désigne globalement les légumes verts, par opposition aux légumes conservés, qui perdaient de leur couleur au cours du processus de fermentation.
\trans \textit{\textcolor{brown}{\zh{新鲜蔬菜。直译:‘绿油油的青菜’。指的不是某种具体的青菜,而是任何新鲜蔬菜,分别于酸菜。制造酸菜的过程中,蔬菜(萝卜等等)褪色:失去原来的深色。}}}
\end{exe}
 \mytextsc{clf}~: \textcolor{darkblue}{\textbf{\ipa{po˧}}} 

\paragraph{\hspace{-0.5cm} {\Large \textcolor{darkblue}{\textbf{\ipa{v̩˩tsʰɤ˧-bv̩\#˥}}}}} \hypertarget{v\string_=\string_Bts\string_h7\string_M-bv\string_=\#\string_T1}{}
\markboth{\textcolor{darkblue}{\textbf{\ipa{v̩˩tsʰɤ˧-bv̩\#˥}}}}{}
\textcolor{teal}{\mytextsc{nom}} \hspace{4pt} Ton~: LM+\#H.

Coccinelle.
\textcolor{brown}{\zh{瓢虫。}}


 \mytextsc{clf}~: \textcolor{darkblue}{\textbf{\ipa{mi˩}}} 

\paragraph{\hspace{-0.5cm} {\Large \textcolor{darkblue}{\textbf{\ipa{v̩˩tsʰɤ˧-pʰv̩˥}}}}} \hypertarget{v\string_=\string_Bts\string_h7\string_M-p\string_hv\string_=\string_T1}{}
\markboth{\textcolor{darkblue}{\textbf{\ipa{v̩˩tsʰɤ˧-pʰv̩˥}}}}{}
\textcolor{teal}{\mytextsc{nom}} \hspace{4pt} Ton~: LM+H\#.

Chou chinois.
\textcolor{brown}{\zh{白菜。}}


 \mytextsc{clf}~: \textcolor{darkblue}{\textbf{\ipa{po˧}}} \textit{Voir~:} \hyperlink{ts\string_h\{\string_Mp\string_hv\string_M\string_T1}{\textcolor{darkblue}{\textbf{\ipa{tsʰæ˧pʰv˧˥}}}} 

\paragraph{\hspace{-0.5cm} {\Large \textcolor{darkblue}{\textbf{\ipa{v̩˩tsʰɤ˧-v̩˥ɲi˩}}}}} \hypertarget{v\string_=\string_Bts\string_h7\string_M-v\string_=\string_TJi\string_B1}{}
\markboth{\textcolor{darkblue}{\textbf{\ipa{v̩˩tsʰɤ˧-v̩˥ɲi˩}}}}{}
\textcolor{teal}{\mytextsc{nom}} \hspace{4pt} Ton~: LM+\#H-.

Légumes.
\textcolor{brown}{\zh{蔬菜。}}


 \mytextsc{clf}~: \textcolor{darkblue}{\textbf{\ipa{qɑ˩}}} botte

\paragraph{\hspace{-0.5cm} {\Large \textcolor{darkblue}{\textbf{\ipa{v̩˧zo\#˥}}}}} \hypertarget{v\string_=\string_Mzo\#\string_T1}{}
\markboth{\textcolor{darkblue}{\textbf{\ipa{v̩˧zo\#˥}}}}{}
\textcolor{teal}{\mytextsc{nom}} \hspace{4pt} Ton~: \#H.

Petite casserole.
\textcolor{brown}{\zh{小锅。}}




\paragraph{\hspace{-0.5cm} {\Large \textcolor{darkblue}{\textbf{\ipa{v̩˧\textasciitilde{}v̩˧\textsubscript{a}}}}}} \hypertarget{v\string_=\string_M~v\string_=\string_Ma1}{}
\markboth{\textcolor{darkblue}{\textbf{\ipa{v̩˧\textasciitilde{}v̩˧\textsubscript{a}}}}}{}
\textcolor{teal}{\mytextsc{verbe}} \hspace{4pt} Ton~: M\textsubscript{a}.

Mâcher.
\textcolor{brown}{\zh{嚼。}}


\begin{exe}
\sn \textcolor{darkblue}{\textbf{\ipa{le˧-v̩˧\textasciitilde{}v̩˧ +ze˩}}}
\trans \mytextsc{accomp}
\trans \textit{\textcolor{brown}{\zh{\mytextsc{accomp}}}}
\end{exe}
\begin{exe}
\sn \textcolor{darkblue}{\textbf{\ipa{le˧-wo˧ v̩˧\textasciitilde{}v̩˧}}}
\trans ruminer (la vache rumine)
\trans \textit{\textcolor{brown}{\zh{反刍}}}
\end{exe}


\newpage
\section*{\centering- \textcolor{darkblue}{\textbf{\ipa{w}}} \textcolor{darkblue}{\textbf{\ipa{wɑ}}} \textcolor{darkblue}{\textbf{\ipa{wæ}}} \textcolor{darkblue}{\textbf{\ipa{wɤ}}} \textcolor{darkblue}{\textbf{\ipa{wo}}} \textcolor{darkblue}{\textbf{\ipa{wɤ̃}}} -}
\paragraph{\hspace{-0.5cm} {\Large \textcolor{darkblue}{\textbf{\ipa{wɤ˧}}} \textsubscript{1}}} \hypertarget{w7\string_M1}{}
\markboth{\textcolor{darkblue}{\textbf{\ipa{wɤ˧}}} \textsubscript{1}}{}
\textcolor{teal}{\mytextsc{nom}} \hspace{4pt} Ton~: M.

Serf, esclave (la plus basse caste de la société ancienne).
\textcolor{brown}{\zh{奴隶,农奴。音译:“俄”。}}


 \mytextsc{clf}~: \textcolor{darkblue}{\textbf{\ipa{v̩˧}}} 

\paragraph{\hspace{-0.5cm} {\Large \textcolor{darkblue}{\textbf{\ipa{wɤ˧}}} \textsubscript{2}}} \hypertarget{w7\string_M2}{}
\markboth{\textcolor{darkblue}{\textbf{\ipa{wɤ˧}}} \textsubscript{2}}{}
\textcolor{teal}{\mytextsc{particule}} \textcolor{teal}{\mytextsc{de}} \textcolor{teal}{\mytextsc{discours}} \hspace{4pt} Ton~: M.

Particule finale exclamative, avec une nuance d'évidence.
\textcolor{brown}{\zh{句尾助词:吧、呗。}}




\paragraph{\hspace{-0.5cm} {\Large \textcolor{darkblue}{\textbf{\ipa{wɤ˩\textsubscript{a}}}}}} \hypertarget{w7\string_Ba1}{}
\markboth{\textcolor{darkblue}{\textbf{\ipa{wɤ˩\textsubscript{a}}}}}{}
\textcolor{teal}{\mytextsc{verbe}} \hspace{4pt} Ton~: L\textsubscript{a}.

Dépendre de, se reposer sur.
\textcolor{brown}{\zh{依赖。}}


\begin{exe}
\sn \textcolor{darkblue}{\textbf{\ipa{hĩ˧-bi˥ | wɤ˩-mɤ˩-bi˩˥!}}}
\trans Il ne faut pas dépendre des autres!
\trans \textit{\textcolor{brown}{\zh{不要依赖别人!}}}
\end{exe}
\begin{exe}
\sn \textcolor{darkblue}{\textbf{\ipa{hĩ˧-bi˥ | wɤ˩-v̩˩-tʰv̩˩˥!}}}
\trans On se trouve dépendre des autres!
\trans \textit{\textcolor{brown}{\zh{(无论如何)人都会依靠别人的!(意思是:人不能完全独立,人活在人间就会或多或少需要依靠别人。)}}}
\end{exe}


\paragraph{\hspace{-0.5cm} {\Large \textcolor{darkblue}{\textbf{\ipa{wɤ˩\textsubscript{b}}}}}} \hypertarget{w7\string_Bb1}{}
\markboth{\textcolor{darkblue}{\textbf{\ipa{wɤ˩\textsubscript{b}}}}}{}
\textcolor{teal}{\mytextsc{classificateur}} \hspace{4pt} Ton~: L\textsubscript{b}.

Classificateur des charges/fardeaux qu'une personne peut porter.
\textcolor{brown}{\zh{量词:担,负荷。}}


\begin{exe}
\sn \textcolor{darkblue}{\textbf{\ipa{ɖɯ˧-wɤ˩ pɤ˩\textasciitilde{}pɤ˩ |}}}
\trans porter une charge
\trans \textit{\textcolor{brown}{\zh{背一担}}}
\end{exe}
\begin{exe}
\sn \textcolor{darkblue}{\textbf{\ipa{ɖɯ˧-wɤ˩, | ɖɯ˧-wɤ˩ | le˧-kʰɯ˩\textasciitilde{}kʰɯ˩ | tʰi˧-tɕɯ˥ |}}}
\trans entasser les charges, l'une après l'autre
\trans \textit{\textcolor{brown}{\zh{将驮的大包堆起来}}}
\end{exe}


\paragraph{\hspace{-0.5cm} {\Large \textcolor{darkblue}{\textbf{\ipa{wɤ˩\textasciitilde{}wɤ˩}}}}} \hypertarget{w7\string_B~w7\string_B1}{}
\markboth{\textcolor{darkblue}{\textbf{\ipa{wɤ˩\textasciitilde{}wɤ˩}}}}{}
\textcolor{teal}{\mytextsc{verbe}} \hspace{4pt} Ton~: L.

Contourner.
\textcolor{brown}{\zh{绕过。}}


\begin{exe}
\sn \textcolor{darkblue}{\textbf{\ipa{le˧-wɤ˩-ze˩}}}
\trans \mytextsc{accomp} \string_ pfv
\trans \textit{\textcolor{brown}{\zh{绕了}}}
\end{exe}
\begin{exe}
\sn \textcolor{darkblue}{\textbf{\ipa{ɖɯ˧-wɤ˩\textasciitilde{}wɤ˩-ɻ̍˩}}}
\trans \mytextsc{délimitatif} \string_ \mytextsc{red} \mytextsc{inchoatif}
\trans \textit{\textcolor{brown}{\zh{绕一绕}}}
\end{exe}
\begin{exe}
\sn \textcolor{darkblue}{\textbf{\ipa{[PHONO] wɤ˩\textasciitilde{}wɤ˩ bi˩˥}}}
\trans \mytextsc{fut}\string_imm
\trans \textit{\textcolor{brown}{\zh{\mytextsc{imm}\string_fut}}}
\end{exe}
\begin{exe}
\sn \textcolor{darkblue}{\textbf{\ipa{[PHONO] wɤ˩\textasciitilde{}wɤ˩-ze˥}}}
\trans \mytextsc{pfv}
\trans \textit{\textcolor{brown}{\zh{绕了}}}
\end{exe}
\begin{exe}
\sn \textcolor{darkblue}{\textbf{\ipa{le˧-wɤ˩\textasciitilde{}wɤ˩ | le˧-se˥}}}
\trans contourner à pied
\trans \textit{\textcolor{brown}{\zh{走路绕过}}}
\end{exe}


\paragraph{\hspace{-0.5cm} {\Large \textcolor{darkblue}{\textbf{\ipa{wɤ˩˥}}}}} \hypertarget{w7\string_B\string_T1}{}
\markboth{\textcolor{darkblue}{\textbf{\ipa{wɤ˩˥}}}}{}
\textcolor{teal}{\mytextsc{adverbe}} \hspace{4pt} Ton~: LM? LH?.

À nouveau, encore; aussi.
\textcolor{brown}{\zh{又,再。}}


\begin{exe}
\sn \textcolor{darkblue}{\textbf{\ipa{wɤ˩˥ | ɖɯ˧-ʂɯ˩}}}
\trans une nouvelle fois, une fois de plus
\trans \textit{\textcolor{brown}{\zh{再一次、又一次}}}
\end{exe}


\paragraph{\hspace{-0.5cm} {\Large \textcolor{darkblue}{\textbf{\ipa{wo˥}}}}} \hypertarget{wo\string_T1}{}
\markboth{\textcolor{darkblue}{\textbf{\ipa{wo˥}}}}{}
\textcolor{teal}{\mytextsc{adjectif}} \hspace{4pt} Ton~: H.

Dur, solide, résistant.
\textcolor{brown}{\zh{硬,坚硬,结实。}}


\begin{exe}
\sn \textcolor{darkblue}{\textbf{\ipa{le˧-wo˥-ze˩}}}
\trans \mytextsc{accomp} \string_ \mytextsc{pfv}: ça a durci, c'est devenu dur
\trans \textit{\textcolor{brown}{\zh{硬了}}}
\end{exe}


\paragraph{\hspace{-0.5cm} {\Large \textcolor{darkblue}{\textbf{\ipa{wo˩\textsubscript{b}}}}}} \hypertarget{wo\string_Bb1}{}
\markboth{\textcolor{darkblue}{\textbf{\ipa{wo˩\textsubscript{b}}}}}{}
\textcolor{teal}{\mytextsc{classificateur}} \hspace{4pt} Ton~: L\textsubscript{b}.

Classificateur des paires de bœufs; attelage de bœufs pour tirer l'araire. A Yongning, l'attelage comporte deux bœufs, ou deux petits buffles, ou un seul buffle vigoureux.
\textcolor{brown}{\zh{量词:牛(一架)。}}


\begin{exe}
\sn \textcolor{darkblue}{\textbf{\ipa{dʑi˧mi˧ | ɲi˧-pʰo˧˥, | ɖɯ˧-wo˩!}}}
\trans Deux buffles, cela forme un attelage!
\trans \textit{\textcolor{brown}{\zh{两头水牛,等于一架!}}}
\end{exe}


\paragraph{\hspace{-0.5cm} {\Large \textcolor{darkblue}{\textbf{\ipa{wo˩kɤ\#˥}}}}} \hypertarget{wo\string_Bk7\#\string_T1}{}
\markboth{\textcolor{darkblue}{\textbf{\ipa{wo˩kɤ\#˥}}}}{}
\textcolor{teal}{\mytextsc{nom}} \hspace{4pt} Ton~: LM+\#H.

Balançoire.
\textcolor{brown}{\zh{秋千(鞦韆)。}}


\begin{exe}
\sn \textcolor{darkblue}{\textbf{\ipa{wo˩kɤ˧-tsɑ˧-di˧˥}}}
\trans même sens: balançoire
\trans \textit{\textcolor{brown}{\zh{同上:秋千}}}
\end{exe}
\begin{exe}
\sn \textcolor{darkblue}{\textbf{\ipa{wo˩kɤ˧ tsɑ˧˥}}}
\trans même sens: balançoire
\trans \textit{\textcolor{brown}{\zh{同上:秋千}}}
\end{exe}
 \mytextsc{clf}~: \textcolor{darkblue}{\textbf{\ipa{nɑ˧}}} 

\paragraph{\hspace{-0.5cm} {\Large \textcolor{darkblue}{\textbf{\ipa{wo˧˥}}}}} \hypertarget{wo\string_M\string_T1}{}
\markboth{\textcolor{darkblue}{\textbf{\ipa{wo˧˥}}}}{}
\textcolor{teal}{\mytextsc{verbe}} \hspace{4pt} Ton~: MH.

Se retourner (quelqu'un est assis et se retourne: mouvement du torse).
\textcolor{brown}{\zh{重新做、再来做。}}


\begin{exe}
\sn \textcolor{darkblue}{\textbf{\ipa{le˧-wo˧ ʐwɤ˧˥}}}
\trans répondre, donner une réponse
\trans \textit{\textcolor{brown}{\zh{回答}}}
\end{exe}
\begin{exe}
\sn \textcolor{darkblue}{\textbf{\ipa{le˧-wo˧-ɻ̍˥}}}
\trans se retourner
\trans \textit{\textcolor{brown}{\zh{转身}}}
\end{exe}
\begin{exe}
\sn \textcolor{darkblue}{\textbf{\ipa{le˧-wo˧ li˥}}}
\trans regarder derrière soi
\trans \textit{\textcolor{brown}{\zh{往后看}}}
\end{exe}
\begin{exe}
\sn \textcolor{darkblue}{\textbf{\ipa{lə-˧wo˧ tʰo˥-tɕo˩}}}
\trans se retourner
\trans \textit{\textcolor{brown}{\zh{转身}}}
\end{exe}
\begin{exe}
\sn \textcolor{darkblue}{\textbf{\ipa{le˧-wo˧-tɕo˥!}}}
\trans retourne-toi! (adressé à un bébé qui s'apprête à descendre d'un lit la tête la première)
\trans \textit{\textcolor{brown}{\zh{转身!(婴儿爬下床,头朝下。奶奶告诉她:要先转身)}}}
\end{exe}
\begin{exe}
\sn \textcolor{darkblue}{\textbf{\ipa{le˧-wo˧˥ | le˧-hɯ˩}}}
\trans ...est reparti
\trans \textit{\textcolor{brown}{\zh{回去了}}}
\end{exe}


\paragraph{\hspace{-0.5cm} {\Large \textcolor{darkblue}{\textbf{\ipa{wo˩˥}}}}} \hypertarget{wo\string_B\string_T1}{}
\markboth{\textcolor{darkblue}{\textbf{\ipa{wo˩˥}}}}{}
\textcolor{teal}{\mytextsc{nom}} \hspace{4pt} Ton~: LH.

Feuilles du navet.
\textcolor{brown}{\zh{圆根的叶子。}}


\begin{exe}
\sn \textcolor{darkblue}{\textbf{\ipa{wo˩bɤ˧˥}}}
\trans même sens: feuilles du navet
\trans \textit{\textcolor{brown}{\zh{同上:圆根叶子}}}
\end{exe}
\begin{exe}
\sn \textcolor{darkblue}{\textbf{\ipa{[Housebuilding2] wo˩-v˥tsʰɤ˩}}}
\trans même sens: feuilles du navet; littéralement 'légume-feuilles du navet'; l'expression souligne qu'il s'agit d'une variété de légume: d'un ingrédient pour la cuisine
\trans \textit{\textcolor{brown}{\zh{同上:圆根叶子}}}
\end{exe}
\begin{exe}
\sn \textcolor{darkblue}{\textbf{\ipa{[Housebuilding2] wo˩-tɕæ˩ɻæ˥}}}
\trans feuilles du navet conservées dans la saumure
\trans \textit{\textcolor{brown}{\zh{圆根叶子酸菜}}}
\end{exe}


\newpage
\section*{\centering- \textcolor{darkblue}{\textbf{\ipa{w̃}}} \textcolor{darkblue}{\textbf{\ipa{w̃æ}}} -}
\paragraph{\hspace{-0.5cm} {\Large \textcolor{darkblue}{\textbf{\ipa{w̃æ˧}}}}} \hypertarget{w\string_~\{\string_M1}{}
\markboth{\textcolor{darkblue}{\textbf{\ipa{w̃æ˧}}}}{}
\textcolor{teal}{\mytextsc{verbe}} \hspace{4pt} Ton~: M intrans.

Se gonfler, enfler (ventre).
\textcolor{brown}{\zh{肿,膨胀,(肚子)胀。}}


\begin{exe}
\sn \textcolor{darkblue}{\textbf{\ipa{ɻ̍˧tɑ˧ w̃æ˧ (-ze˧)}}}
\trans les ganglions sont enflés
\trans \textit{\textcolor{brown}{\zh{淋巴结肿了}}}
\end{exe}
\begin{exe}
\sn \textcolor{darkblue}{\textbf{\ipa{tso˧\textasciitilde{}tso˧ w̃æ˩}}}
\trans quelque chose a enflé
\trans \textit{\textcolor{brown}{\zh{东西膨胀了}}}
\end{exe}


\newpage
\section*{\centering- \textcolor{darkblue}{\textbf{\ipa{z}}} -}
\paragraph{\hspace{-0.5cm} {\Large \textcolor{darkblue}{\textbf{\ipa{zɑ˥}}}}} \hypertarget{zA\string_T1}{}
\markboth{\textcolor{darkblue}{\textbf{\ipa{zɑ˥}}}}{}
\textcolor{teal}{\mytextsc{adjectif}} \hspace{4pt} Ton~: H.

Limité à, restreint à (en tournure négative).
\textcolor{brown}{\zh{仅仅。}}


\begin{exe}
\sn \textcolor{darkblue}{\textbf{\ipa{ʁwɤ˧-qo˧-ɳɯ˧-lɑ˧ mɤ˧-zɑ˥ (…)}}}
\trans Il n'y avait pas que les gens du village (…) (récit ``Funeral")
\trans \textit{\textcolor{brown}{\zh{不仅有村子里的人}}}
\end{exe}


\paragraph{\hspace{-0.5cm} {\Large \textcolor{darkblue}{\textbf{\ipa{zɑ˧ɭɯ˧}}}}} \hypertarget{zA\string_Ml\string_RM\string_M1}{}
\markboth{\textcolor{darkblue}{\textbf{\ipa{zɑ˧ɭɯ˧}}}}{}
\textcolor{teal}{\mytextsc{nom}} \hspace{4pt} Ton~: M.

Porc castré.
\textcolor{brown}{\zh{阉猪。}}


 \mytextsc{clf}~: \textcolor{darkblue}{\textbf{\ipa{pʰo˧˥}}} \textcolor{darkblue}{\textbf{\ipa{v̩˧}}} 

\paragraph{\hspace{-0.5cm} {\Large \textcolor{darkblue}{\textbf{\ipa{zɑ˧zɑ˧}}}}} \hypertarget{zA\string_MzA\string_M1}{}
\markboth{\textcolor{darkblue}{\textbf{\ipa{zɑ˧zɑ˧}}}}{}
\textcolor{teal}{\mytextsc{adjectif}} \hspace{4pt} Ton~: M.

Attentif, soigneux.
\textcolor{brown}{\zh{细心、细致。}}




\paragraph{\hspace{-0.5cm} {\Large \textcolor{darkblue}{\textbf{\ipa{zɑ˩\textsubscript{a}}}}}} \hypertarget{zA\string_Ba1}{}
\markboth{\textcolor{darkblue}{\textbf{\ipa{zɑ˩\textsubscript{a}}}}}{}
\textcolor{teal}{\mytextsc{verbe}} \hspace{4pt} Ton~: L\textsubscript{a}.

Descendre (redescendre de la montagne).
\textcolor{brown}{\zh{下(山……)。}}


\begin{exe}
\sn \textcolor{darkblue}{\textbf{\ipa{ʁwɤ˩ zɑ˩˥}}}
\trans descendre de la montagne
\trans \textit{\textcolor{brown}{\zh{下山}}}
\end{exe}
\begin{exe}
\sn \textcolor{darkblue}{\textbf{\ipa{mɤ˧-zɑ˩-sɯ˩}}}
\trans ne pas descendre encore
\trans \textit{\textcolor{brown}{\zh{还没下来}}}
\end{exe}
\begin{exe}
\sn \textcolor{darkblue}{\textbf{\ipa{ɖɯ˧-zɑ˧\textasciitilde{}zɑ˥-ɻ̍˩}}}
\trans \mytextsc{délimitatif} \string_ \mytextsc{red} \mytextsc{inchoatif}
\trans \textit{\textcolor{brown}{\zh{下来一下}}}
\end{exe}


\paragraph{\hspace{-0.5cm} {\Large \textcolor{darkblue}{\textbf{\ipa{zɑ˩-bɑ˧lɑ˩}}}}} \hypertarget{zA\string_B-bA\string_MlA\string_B1}{}
\markboth{\textcolor{darkblue}{\textbf{\ipa{zɑ˩-bɑ˧lɑ˩}}}}{}
\textcolor{teal}{\mytextsc{nom}} \hspace{4pt} Ton~: L-L\#.
\ding{202} 
Peinture religieuse (thangka) sur bois, dans une niche sur le mur, au-dessus du foyer.
\textcolor{brown}{\zh{火塘旁边墙上的壁画(唐卡:内容来自藏传佛教)。}}


\ding{203} 
Divinité du feu, du foyer et de la maison.
\textcolor{brown}{\zh{火,火塘与家的神。}}




\paragraph{\hspace{-0.5cm} {\Large \textcolor{darkblue}{\textbf{\ipa{zɑ˩ɲi˥-ʂɤ˩}}}}} \hypertarget{zA\string_BJi\string_T-s`7\string_B1}{}
\markboth{\textcolor{darkblue}{\textbf{\ipa{zɑ˩ɲi˥-ʂɤ˩}}}}{}
\textcolor{teal}{\mytextsc{nom}} \hspace{4pt} Ton~: LH-.

Vampire; démon malfaisant de forme humaine (de la taille d'un humain de grande taille), qui ne mange pas de viande, et se nourrit de sang.
\textcolor{brown}{\zh{吸血鬼。}}




\paragraph{\hspace{-0.5cm} {\Large \textcolor{darkblue}{\textbf{\ipa{‑ze˧}}}}} \hypertarget{‑ze\string_M1}{}
\markboth{\textcolor{darkblue}{\textbf{\ipa{‑ze˧}}}}{}
\textcolor{teal}{\mytextsc{suffixe}} \hspace{4pt} Ton~: M.

Perfectif, \mytextsc{pfv}.
\textcolor{brown}{\zh{\mytextsc{整体体。}}}




\paragraph{\hspace{-0.5cm} {\Large \textcolor{darkblue}{\textbf{\ipa{ze˩}}}}} \hypertarget{ze\string_B1}{}
\markboth{\textcolor{darkblue}{\textbf{\ipa{ze˩}}}}{}
\textcolor{teal}{\mytextsc{pronom}} \hspace{4pt} Ton~: L.

Quel, lequel.
\textcolor{brown}{\zh{哪。}}




\paragraph{\hspace{-0.5cm} {\Large \textcolor{darkblue}{\textbf{\ipa{ze˩}}}}} \hypertarget{ze\string_B1}{}
\markboth{\textcolor{darkblue}{\textbf{\ipa{ze˩}}}}{}
\textcolor{teal}{\mytextsc{adjectif}} \hspace{4pt} Ton~: .
\textit{Archaïque} [\textcolor{brown}{\zh{古语}}] 
Pur.
\textcolor{brown}{\zh{纯洁。}}


\begin{exe}
\sn \textcolor{darkblue}{\textbf{\ipa{mɤ˧-ʂo˩ F | le˧-ʂo˩-kʰɯ˩! | mɤ˧-ze˩ F | le˧-ze˩-kʰɯ˩!}}}
\trans Que ce qui n'est pas propre, soit nettoyé / devienne propre! Que ce qui est impur soit purifié! (formule dite lors de rituels de purification: voir le récit Mountains)
\trans \textit{\textcolor{brown}{\zh{让不干净的变得干净!让不纯洁的变得纯洁!(仪式用语)}}}
\end{exe}


\paragraph{\hspace{-0.5cm} {\Large \textcolor{darkblue}{\textbf{\ipa{ze˩bæ˧}}}}} \hypertarget{ze\string_Bb\{\string_M1}{}
\markboth{\textcolor{darkblue}{\textbf{\ipa{ze˩bæ˧}}}}{}
\textcolor{teal}{\mytextsc{pronom}} \hspace{4pt} Ton~: LM.

Quelle sorte de, lequel.
\textcolor{brown}{\zh{哪,哪个 (哪个碗),哪一种。}}


\begin{exe}
\sn \textcolor{darkblue}{\textbf{\ipa{ze˩bæ˧ ɲi˥?}}}
\trans c'est lequel?/c'est de quelle sorte?
\trans \textit{\textcolor{brown}{\zh{是哪个?是哪一样?}}}
\end{exe}


\paragraph{\hspace{-0.5cm} {\Large \textcolor{darkblue}{\textbf{\ipa{ze˩bæ˩}}}}} \hypertarget{ze\string_Bb\{\string_B1}{}
\markboth{\textcolor{darkblue}{\textbf{\ipa{ze˩bæ˩}}}}{}
\textcolor{teal}{\mytextsc{nom}} \hspace{4pt} Ton~: L.

Éclair.
\textcolor{brown}{\zh{闪电、打闪电、霹雷。}}


\begin{exe}
\sn \textcolor{darkblue}{\textbf{\ipa{ze˩bæ˩-ze˥!}}}
\trans il y a eu un éclair!
\trans \textit{\textcolor{brown}{\zh{打闪电了!}}}
\end{exe}
\begin{exe}
\sn \textcolor{darkblue}{\textbf{\ipa{ze˩bæ˩˥ | -dʑo˩!}}}
\trans il y a des éclairs!
\trans \textit{\textcolor{brown}{\zh{打着闪电!}}}
\end{exe}
 \mytextsc{clf}~: \textcolor{darkblue}{\textbf{\ipa{bæ˩}}} 

\paragraph{\hspace{-0.5cm} {\Large \textcolor{darkblue}{\textbf{\ipa{ze˩gɤ˧}}}}} \hypertarget{ze\string_Bg7\string_M1}{}
\markboth{\textcolor{darkblue}{\textbf{\ipa{ze˩gɤ˧}}}}{}
\textcolor{teal}{\mytextsc{pronom}} \hspace{4pt} Ton~: LM.

Où, à quel endroit.
\textcolor{brown}{\zh{哪里,什么地方。}}




\paragraph{\hspace{-0.5cm} {\Large \textcolor{darkblue}{\textbf{\ipa{ze˩mi˩}}}}} \hypertarget{ze\string_Bmi\string_B1}{}
\markboth{\textcolor{darkblue}{\textbf{\ipa{ze˩mi˩}}}}{}
\textcolor{teal}{\mytextsc{nom}} \hspace{4pt} Ton~: L.

Nièce (enfant d'une soeur).
\textcolor{brown}{\zh{甥女(姐妹的女儿)。}}


 \mytextsc{clf}~: \textcolor{darkblue}{\textbf{\ipa{v̩˧}}} 

\paragraph{\hspace{-0.5cm} {\Large \textcolor{darkblue}{\textbf{\ipa{ze˩v̩˩}}}}} \hypertarget{ze\string_Bv\string_=\string_B1}{}
\markboth{\textcolor{darkblue}{\textbf{\ipa{ze˩v̩˩}}}}{}
\textcolor{teal}{\mytextsc{nom}} \hspace{4pt} Ton~: L.

Neveu (fils d'une sœur).
\textcolor{brown}{\zh{外甥(姐妹的儿子)。}}


 \mytextsc{clf}~: \textcolor{darkblue}{\textbf{\ipa{v̩˧}}} 

\paragraph{\hspace{-0.5cm} {\Large \textcolor{darkblue}{\textbf{\ipa{ze˩v̩˩-ze˧mi˩}}}}} \hypertarget{ze\string_Bv\string_=\string_B-ze\string_Mmi\string_B1}{}
\markboth{\textcolor{darkblue}{\textbf{\ipa{ze˩v̩˩-ze˧mi˩}}}}{}
\textcolor{teal}{\mytextsc{nom}} \hspace{4pt} Ton~: L-L\#.

Neveux et nièces (du côté des sœurs: enfants des sœurs).
\textcolor{brown}{\zh{外甥甥女(姐妹的儿女)。}}




\paragraph{\hspace{-0.5cm} {\Large \textcolor{darkblue}{\textbf{\ipa{‑zo}}}}} \hypertarget{‑zo1}{}
\markboth{\textcolor{darkblue}{\textbf{\ipa{‑zo}}}}{}
\textcolor{teal}{\mytextsc{suffixe}} \hspace{4pt} Ton~: .

Adverbialisateur, \mytextsc{advb}.
\textcolor{brown}{\zh{地\mytextsc{副词化。}}}


\begin{exe}
\sn \textcolor{darkblue}{\textbf{\ipa{gv̩˩dʑɯ˩-zo˥}}}
\trans tristement
\trans \textit{\textcolor{brown}{\zh{难过乎乎地}}}
\end{exe}
\begin{exe}
\sn \textcolor{darkblue}{\textbf{\ipa{ʂv̩˧ɖv̩˧-zo˩}}}
\trans pensivement
\trans \textit{\textcolor{brown}{\zh{沉思地}}}
\end{exe}
\begin{exe}
\sn \textcolor{darkblue}{\textbf{\ipa{ʐæ˧-zo˩}}}
\trans en souriant
\trans \textit{\textcolor{brown}{\zh{笑着地}}}
\end{exe}
\begin{exe}
\sn \textcolor{darkblue}{\textbf{\ipa{ɳɯ˧ɕi˩-zo˩}}}
\trans de façon mignonne
\trans \textit{\textcolor{brown}{\zh{可爱地}}}
\end{exe}
\begin{exe}
\sn \textcolor{darkblue}{\textbf{\ipa{ʈʂʰɯ˧ne˧-ʝi˥ | pi˧-zo˩, | njɤ˧ | mɤ˧-gɤ˩ | ʐwæ˩˥! (M18)}}}
\trans Qu’il parle ainsi, cela me contrarie terriblement!
\trans \textit{\textcolor{brown}{\zh{(他)这么说/这样的说法,让我非常不满意!}}}
\end{exe}
\begin{exe}
\sn \textcolor{darkblue}{\textbf{\ipa{ə˧v̩˧-zo˥!}}}
\trans C'est joli! C'est mignon tout plein! (Commentaire adressé à une fillette qui montre à la ronde des souliers tout neufs.)
\trans \textit{\textcolor{brown}{\zh{真漂亮!(情景:一个小姑娘让大家看她的一双新鞋子,奶奶就给予所需称赞)}}}
\end{exe}


\paragraph{\hspace{-0.5cm} {\Large \textcolor{darkblue}{\textbf{\ipa{zo˥}}}}} \hypertarget{zo\string_T1}{}
\markboth{\textcolor{darkblue}{\textbf{\ipa{zo˥}}}}{}
\textcolor{teal}{\mytextsc{nom}} \hspace{4pt} Ton~: \#H.
\ding{202} 
Fils.
\textcolor{brown}{\zh{儿子。}}


\begin{exe}
\sn \textcolor{darkblue}{\textbf{\ipa{zo˧ ɲi˥-kv̩˩}}}
\trans deux fils
\trans \textit{\textcolor{brown}{\zh{两个儿子}}}
\end{exe}
\ding{203} 
Homme, \textit{Vir}.
\textcolor{brown}{\zh{男人。}}


 \mytextsc{clf}~: \textcolor{darkblue}{\textbf{\ipa{v̩˧}}} 

\paragraph{\hspace{-0.5cm} {\Large \textcolor{darkblue}{\textbf{\ipa{‑zo˧}}}}} \hypertarget{‑zo\string_M1}{}
\markboth{\textcolor{darkblue}{\textbf{\ipa{‑zo˧}}}}{}
\textcolor{teal}{\mytextsc{suffixe}} \hspace{4pt} Ton~: M.

Obligatif.
\textcolor{brown}{\zh{应该、必须。}}




\paragraph{\hspace{-0.5cm} {\Large \textcolor{darkblue}{\textbf{\ipa{zo˧\textsubscript{a}}}}}} \hypertarget{zo\string_Ma1}{}
\markboth{\textcolor{darkblue}{\textbf{\ipa{zo˧\textsubscript{a}}}}}{}
\textcolor{teal}{\mytextsc{verbe}} \hspace{4pt} Ton~: M\textsubscript{a}.

Devoir.
\textcolor{brown}{\zh{要,应该。}}


\begin{exe}
\sn \textcolor{darkblue}{\textbf{\ipa{mɤ˧-zo˧ (-ze˧)! | tʰi˧-kwɤ˩-kʰɯ˩!}}}
\trans ce n'est pas la peine! laisse tomber!
\trans \textit{\textcolor{brown}{\zh{不用了!算了吧!}}}
\end{exe}
\begin{exe}
\sn \textcolor{darkblue}{\textbf{\ipa{ʈʂʰɯ˧ne˧-ʝi˥ | ʝi˧-zo˧-ho˥-ɲi˩!}}}
\trans Il faut faire comme ça!
\trans \textit{\textcolor{brown}{\zh{是应该这样做的!}}}
\end{exe}


\paragraph{\hspace{-0.5cm} {\Large \textcolor{darkblue}{\textbf{\ipa{zo˧bæ˩}}}}} \hypertarget{zo\string_Mb\{\string_B1}{}
\markboth{\textcolor{darkblue}{\textbf{\ipa{zo˧bæ˩}}}}{}
\textcolor{teal}{\mytextsc{nom}} \hspace{4pt} Ton~: L\#.

Imbécile, idiot.
\textcolor{brown}{\zh{笨人、傻瓜。}}


\begin{exe}
\sn \textcolor{darkblue}{\textbf{\ipa{mɤ˧-zo˧bæ˩!}}}
\trans (Non, tu n'es) pas idiot(e)! (Propos rassurant adressé à un interlocuteur accablé par ses propres maladresses.)
\trans \textit{\textcolor{brown}{\zh{(你)不是笨蛋!(情景:一个人批评自己是笨蛋,人家安慰他。)}}}
\end{exe}
\begin{exe}
\sn \textcolor{darkblue}{\textbf{\ipa{zo˧bæ˩-mv̩˩bæ˩}}}
\trans idiots, imbéciles (sans distinction de sexe)
\trans \textit{\textcolor{brown}{\zh{傻瓜们(不分男女)}}}
\end{exe}
 \mytextsc{clf}~: \textcolor{darkblue}{\textbf{\ipa{v̩˧}}} 

\paragraph{\hspace{-0.5cm} {\Large \textcolor{darkblue}{\textbf{\ipa{zo˧ɖɯ\#˥}}}}} \hypertarget{zo\string_Md`M\#\string_T1}{}
\markboth{\textcolor{darkblue}{\textbf{\ipa{zo˧ɖɯ\#˥}}}}{}
\textcolor{teal}{\mytextsc{nom}} \hspace{4pt} Ton~: \#H.

Fils aîné.
\textcolor{brown}{\zh{大儿子。}}


\begin{exe}
\sn \textcolor{darkblue}{\textbf{\ipa{zo˧ɖɯ˧-mv̩˥ɖɯ˩}}}
\trans fils et fille aînés
\trans \textit{\textcolor{brown}{\zh{大儿子与大女儿}}}
\end{exe}


\paragraph{\hspace{-0.5cm} {\Large \textcolor{darkblue}{\textbf{\ipa{zo˧hṽ˧-mv̩˥zo˩}}}}} \hypertarget{zo\string_Mhv\string_~\string_M-mv\string_=\string_Tzo\string_B1}{}
\markboth{\textcolor{darkblue}{\textbf{\ipa{zo˧hṽ˧-mv̩˥zo˩}}}}{}
\textcolor{teal}{\mytextsc{nom}} \hspace{4pt} Ton~: MH\#-.

Les descendants.
\textcolor{brown}{\zh{后代。}}


\begin{exe}
\sn \textcolor{darkblue}{\textbf{\ipa{zo˧hṽ˧mv̩˥zo˩=ɻæ˩}}}
\trans \string_ \mytextsc{associatif}
\trans \textit{\textcolor{brown}{\zh{\string_ \mytextsc{联想复数}}}}
\end{exe}


\paragraph{\hspace{-0.5cm} {\Large \textcolor{darkblue}{\textbf{\ipa{zo˧hṽ˧˥}}}}} \hypertarget{zo\string_Mhv\string_~\string_M\string_T1}{}
\markboth{\textcolor{darkblue}{\textbf{\ipa{zo˧hṽ˧˥}}}}{}
\textcolor{teal}{\mytextsc{nom}} \hspace{4pt} Ton~: MH\#.
\ding{202} 
Fils.
\textcolor{brown}{\zh{儿子。}}


\begin{exe}
\sn \textcolor{darkblue}{\textbf{\ipa{zo˧hṽ˧=ɻæ˥}}}
\trans les fils
\trans \textit{\textcolor{brown}{\zh{儿子们}}}
\end{exe}
\ding{203} 
Jeune homme, petit gars.
\textcolor{brown}{\zh{小伙子、 青年男子。}}


 \mytextsc{clf}~: \textcolor{darkblue}{\textbf{\ipa{v̩˧}}} 

\paragraph{\hspace{-0.5cm} {\Large \textcolor{darkblue}{\textbf{\ipa{zo˧mv̩˥}}}}} \hypertarget{zo\string_Mmv\string_=\string_T1}{}
\markboth{\textcolor{darkblue}{\textbf{\ipa{zo˧mv̩˥}}}}{}
\textcolor{teal}{\mytextsc{nom}} \hspace{4pt} Ton~: H\#.

Enfant.
\textcolor{brown}{\zh{孩子。}}


\begin{exe}
\sn \textcolor{darkblue}{\textbf{\ipa{zo˧mv̩˥ | æ˧mv̩˥tɕi˩-hĩ˩}}}
\trans nouveau-né, nourrisson
\trans \textit{\textcolor{brown}{\zh{新生婴儿}}}
\end{exe}
 \mytextsc{clf}~: \textcolor{darkblue}{\textbf{\ipa{ɭɯ˧}}} 

\paragraph{\hspace{-0.5cm} {\Large \textcolor{darkblue}{\textbf{\ipa{zo˧ʂv̩˧kʰv̩˥ɲi˩tʰv̩˩}}}}} \hypertarget{zo\string_Ms`v\string_=\string_Mk\string_hv\string_=\string_TJi\string_Bt\string_hv\string_=\string_B1}{}
\markboth{\textcolor{darkblue}{\textbf{\ipa{zo˧ʂv̩˧kʰv̩˥ɲi˩tʰv̩˩}}}}{}
\textcolor{teal}{\mytextsc{nom}} \hspace{4pt} Ton~: .

Orphelin.
\textcolor{brown}{\zh{孤儿。}}


\begin{exe}
\sn \textcolor{darkblue}{\textbf{\ipa{[F5] le˧-ko˧\textasciitilde{}ko˥ | po˧tsʰɯ˧ (+ tʰv̩˧-v̩˧ / zo˧mv̩˥)}}}
\trans il a été adopté; littéralement: ``il a été ramassé, celui-là/cet enfant"
\trans \textit{\textcolor{brown}{\zh{(这个孩子)是被领养的。}}}
\end{exe}


\paragraph{\hspace{-0.5cm} {\Large \textcolor{darkblue}{\textbf{\ipa{zo˧tv̩˧-mv̩˥tv̩˩}}}}} \hypertarget{zo\string_Mtv\string_=\string_M-mv\string_=\string_Ttv\string_=\string_B1}{}
\markboth{\textcolor{darkblue}{\textbf{\ipa{zo˧tv̩˧-mv̩˥tv̩˩}}}}{}
\textcolor{teal}{\mytextsc{nom}} \hspace{4pt} Ton~: \#H-.

Enfant unique (fils unique ou fille unique).
\textcolor{brown}{\zh{独生子(男女通用)。}}




\paragraph{\hspace{-0.5cm} {\Large \textcolor{darkblue}{\textbf{\ipa{zo˧tv̩\#˥}}}}} \hypertarget{zo\string_Mtv\string_=\#\string_T1}{}
\markboth{\textcolor{darkblue}{\textbf{\ipa{zo˧tv̩\#˥}}}}{}
\textcolor{teal}{\mytextsc{nom}} \hspace{4pt} Ton~: \#H.

Fils unique.
\textcolor{brown}{\zh{独生子,独生男孩。}}


\begin{exe}
\sn \textcolor{darkblue}{\textbf{\ipa{zo˧tv̩˧ ɖɯ˧-v̩˧-lɑ˧ dʑo˧˥!}}}
\trans (elle) n'a qu'un fils unique!
\trans \textit{\textcolor{brown}{\zh{(她)只有一个独生男孩子!}}}
\end{exe}
\begin{exe}
\sn \textcolor{darkblue}{\textbf{\ipa{ʂɯ˧-ɬi˧mi˧, | zo˧tv̩˧ ʐɤ˥-tʰɑ˩-se˩!}}}
\trans ``Au septième mois, un fils unique ne doit pas aller par les chemins!" (Le septième mois, au plus fort des grandes pluies, était considéré comme un mois défavorable pour voyager; les voyageurs risquaient d'y rester.)
\trans \textit{\textcolor{brown}{\zh{“七月份,独生子不要上路!”(七月份是大雨季,摩梭人认为七月份的路最不安全:有生命危险)}}}
\end{exe}


\paragraph{\hspace{-0.5cm} {\Large \textcolor{darkblue}{\textbf{\ipa{zo˧tʰi˧}}}}} \hypertarget{zo\string_Mt\string_hi\string_M1}{}
\markboth{\textcolor{darkblue}{\textbf{\ipa{zo˧tʰi˧}}}}{}
\textcolor{teal}{\mytextsc{nom}} \hspace{4pt} Ton~: M.

Personne intelligente.
\textcolor{brown}{\zh{聪明的人。}}


\begin{exe}
\sn \textcolor{darkblue}{\textbf{\ipa{zo˧tʰi˧ ɖɯ˧-v̩˧}}}
\trans une personne intelligente
\trans \textit{\textcolor{brown}{\zh{一个聪明的人}}}
\end{exe}


\paragraph{\hspace{-0.5cm} {\Large \textcolor{darkblue}{\textbf{\ipa{zo˧tʰi˧}}}}} \hypertarget{zo\string_Mt\string_hi\string_M1}{}
\markboth{\textcolor{darkblue}{\textbf{\ipa{zo˧tʰi˧}}}}{}
\textcolor{teal}{\mytextsc{adjectif}} \hspace{4pt} Ton~: M.

Intelligent.
\textcolor{brown}{\zh{聪明。}}


\begin{exe}
\sn \textcolor{darkblue}{\textbf{\ipa{ʈʂʰɯ˧ | zo˧tʰi˧ | ʐwæ˩˥!}}}
\trans il est très intelligent!
\trans \textit{\textcolor{brown}{\zh{他很聪明!}}}
\end{exe}
\begin{exe}
\sn \textcolor{darkblue}{\textbf{\ipa{ʈʂʰɯ˧ | mɤ˧-tʰi˧!}}}
\trans il n'est pas intelligent/il n'est pas bien malin! (on ne peut dire: /*mɤ˧-zo˧tʰi˧/)
\trans \textit{\textcolor{brown}{\zh{他不聪明!}}}
\end{exe}


\paragraph{\hspace{-0.5cm} {\Large \textcolor{darkblue}{\textbf{\ipa{zo˧tɕi˥}}}}} \hypertarget{zo\string_Mts£i\string_T1}{}
\markboth{\textcolor{darkblue}{\textbf{\ipa{zo˧tɕi˥}}}}{}
\textcolor{teal}{\mytextsc{nom}} \hspace{4pt} Ton~: H\#.

Fils dernier-né, benjamin.
\textcolor{brown}{\zh{最小的儿子。}}


\begin{exe}
\sn \textcolor{darkblue}{\textbf{\ipa{zo˧tɕi˥-mv̩˩tɕi˩}}}
\trans le benjamin et la benjamine: les plus jeunes enfants
\trans \textit{\textcolor{brown}{\zh{最小的儿子与女儿}}}
\end{exe}


\paragraph{\hspace{-0.5cm} {\Large \textcolor{darkblue}{\textbf{\ipa{zo˧zo˧-mv̩˧mv̩˥}}}}} \hypertarget{zo\string_Mzo\string_M-mv\string_=\string_Mmv\string_=\string_T1}{}
\markboth{\textcolor{darkblue}{\textbf{\ipa{zo˧zo˧-mv̩˧mv̩˥}}}}{}
\textcolor{teal}{\mytextsc{nom}} \hspace{4pt} Ton~: H\#.

Truc, bidule.
\textcolor{brown}{\zh{东西。}}


 \mytextsc{clf}~: \textcolor{darkblue}{\textbf{\ipa{kʰwɤ˥}}} 

\paragraph{\hspace{-0.5cm} {\Large \textcolor{darkblue}{\textbf{\ipa{zo˧ʐɤ\#˥}}}}} \hypertarget{zo\string_Mz`7\#\string_T1}{}
\markboth{\textcolor{darkblue}{\textbf{\ipa{zo˧ʐɤ\#˥}}}}{}
\textcolor{teal}{\mytextsc{nom}} \hspace{4pt} Ton~: \#H.

Fils adoptif.
\textcolor{brown}{\zh{义子。}}




\paragraph{\hspace{-0.5cm} {\Large \textcolor{darkblue}{\textbf{\ipa{zo˩bv̩˥li˩}}}}} \hypertarget{zo\string_Bbv\string_=\string_Tli\string_B1}{}
\markboth{\textcolor{darkblue}{\textbf{\ipa{zo˩bv̩˥li˩}}}}{}
\textcolor{teal}{\mytextsc{nom}} \hspace{4pt} Ton~: .

Univers.
\textcolor{brown}{\zh{宇宙。}}

 Emprunt~: tibétain? (Lidz2010:108)

\begin{exe}
\sn \textcolor{darkblue}{\textbf{\ipa{sɑ˧ | -zo˩bv̩˥-li˩}}}
\trans l'univers
\trans \textit{\textcolor{brown}{\zh{宇宙}}}
\end{exe}


\paragraph{\hspace{-0.5cm} {\Large \textcolor{darkblue}{\textbf{\ipa{zo˩no˧}}}}} \hypertarget{zo\string_Bno\string_M1}{}
\markboth{\textcolor{darkblue}{\textbf{\ipa{zo˩no˧}}}}{}
\textcolor{teal}{\mytextsc{adverbe}} \hspace{4pt} Ton~: LM.

Maintenant, actuellement: désigne le moment présent (heure de la journée), comme la période présente (époque contemporaine, par opposition à d'autres époques); également employé comme élément phatique, ``gap-filler": ``alors…"; ``eh bien…".
\textcolor{brown}{\zh{现在。}}


\begin{exe}
\sn \textcolor{darkblue}{\textbf{\ipa{zo˩no˥ | gɤ˩-ʈi˧!}}}
\trans Elle vient de se réveiller/de se lever! / Elle s'est réveillée à l'instant! (contexte: quelqu'un entre dans la maison, voit un petit enfant en train de jouer et constate: ``Elle est réveillée!" Sa grand-mère répond: ``Elle vient de se réveiller!")
\trans \textit{\textcolor{brown}{\zh{刚起床! / 刚才才起床!}}}
\end{exe}


\paragraph{\hspace{-0.5cm} {\Large \textcolor{darkblue}{\textbf{\ipa{zo˩qo˧}}}}} \hypertarget{zo\string_Bqo\string_M1}{}
\markboth{\textcolor{darkblue}{\textbf{\ipa{zo˩qo˧}}}}{}
\textcolor{teal}{\mytextsc{pronom}} \hspace{4pt} Ton~: LM.

Où.
\textcolor{brown}{\zh{哪里。}}


\begin{exe}
\sn \textcolor{darkblue}{\textbf{\ipa{no˧ | zo˩qo˧ bi˧?}}}
\trans Où tu vas?
\trans \textit{\textcolor{brown}{\zh{你去哪里?}}}
\end{exe}
\begin{exe}
\sn \textcolor{darkblue}{\textbf{\ipa{zo˩qo˧-ɳɯ˧ | tsʰɯ˩˥?}}}
\trans D'où (tu) viens?
\trans \textit{\textcolor{brown}{\zh{从哪里来?}}}
\end{exe}
\begin{exe}
\sn \textcolor{darkblue}{\textbf{\ipa{no˧ | hɑ˧ | zo˩qo˧ dzɯ˧-bi˧-pi˧, | ɖɯ˧-bæ˧ lɑ˧ ɲi˥!}}}
\trans Peu importe où tu vas manger, c'est partout pareil! (contexte: au sujet des restaurants récemment ouverts à Yongning, qui partagent les mêmes qualités et défauts dont des problèmes d'hygiène)
\trans \textit{\textcolor{brown}{\zh{无论你到哪里去吃,都一样!(情景:新开的饭馆)}}}
\end{exe}
\begin{exe}
\sn \textcolor{darkblue}{\textbf{\ipa{zo˩qo˧ tʰv̩˧?}}}
\trans Où tu es? (question typique quand on appelle quelqu'un sur son téléphone portable)
\trans \textit{\textcolor{brown}{\zh{你到哪里了?(打手机)}}}
\end{exe}


\paragraph{\hspace{-0.5cm} {\Large \textcolor{darkblue}{\textbf{\ipa{zɯ˥}}}}} \hypertarget{zM\string_T1}{}
\markboth{\textcolor{darkblue}{\textbf{\ipa{zɯ˥}}}}{}
\textcolor{teal}{\mytextsc{nom}} \hspace{4pt} Ton~: \#H.

Herbe.
\textcolor{brown}{\zh{草。}}


 \mytextsc{clf}~: \textcolor{darkblue}{\textbf{\ipa{kʰɤ˧˥}}} 

\paragraph{\hspace{-0.5cm} {\Large \textcolor{darkblue}{\textbf{\ipa{zɯ˧}}}}} \hypertarget{zM\string_M1}{}
\markboth{\textcolor{darkblue}{\textbf{\ipa{zɯ˧}}}}{}
\textcolor{teal}{\mytextsc{nom}} \hspace{4pt} Ton~: M.

Vie, existence.
\textcolor{brown}{\zh{生命。}}


\begin{exe}
\sn \textcolor{darkblue}{\textbf{\ipa{zɯ˧ʂæ\#˥}}}
\trans longue vie
\trans \textit{\textcolor{brown}{\zh{长命、长的人生}}}
\end{exe}
\begin{exe}
\sn \textcolor{darkblue}{\textbf{\ipa{zɯ˧ ʂæ˧ | hɑ̃˧-ʝi˧-kʰɯ˩!}}}
\trans Puisses-tu avoir longue vie! (bénédiction)
\trans \textit{\textcolor{brown}{\zh{祝你长寿!}}}
\end{exe}
\begin{exe}
\sn \textcolor{darkblue}{\textbf{\ipa{zɯ˧ɖæ\#˥}}}
\trans courte vie
\trans \textit{\textcolor{brown}{\zh{短命}}}
\end{exe}
\textit{Voir~:} \hyperlink{zM\string_Mb1}{\textcolor{darkblue}{\textbf{\ipa{zɯ˧b}}}} 

\paragraph{\hspace{-0.5cm} {\Large \textcolor{darkblue}{\textbf{\ipa{zɯ˧\textsubscript{b}}}}}} \hypertarget{zM\string_Mb1}{}
\markboth{\textcolor{darkblue}{\textbf{\ipa{zɯ˧\textsubscript{b}}}}}{}
\textcolor{teal}{\mytextsc{classificateur}} \hspace{4pt} Ton~: M\textsubscript{b}.

Auto-classificateur de la vie, de l'existence entière.
\textcolor{brown}{\zh{量词:辈子。}}


\begin{exe}
\sn \textcolor{darkblue}{\textbf{\ipa{ɖɯ˧-zɯ˧}}}
\trans toute la vie
\trans \textit{\textcolor{brown}{\zh{一辈子(的时间)}}}
\end{exe}
\textit{Voir~:} \hyperlink{zM\string_M1}{\textcolor{darkblue}{\textbf{\ipa{zɯ˧}}}} 

\paragraph{\hspace{-0.5cm} {\Large \textcolor{darkblue}{\textbf{\ipa{zɯ˧hṽ˩}}}}} \hypertarget{zM\string_Mhv\string_~\string_B1}{}
\markboth{\textcolor{darkblue}{\textbf{\ipa{zɯ˧hṽ˩}}}}{}
\textcolor{teal}{\mytextsc{adjectif}} \hspace{4pt} Ton~: L\#.

Vert.
\textcolor{brown}{\zh{绿(布料、线)。}}


\begin{exe}
\sn \textcolor{darkblue}{\textbf{\ipa{zɯ˧hṽ˩-ni˩gv̩˩}}}
\trans de couleur verte
\trans \textit{\textcolor{brown}{\zh{绿}}}
\end{exe}
\begin{exe}
\sn \textcolor{darkblue}{\textbf{\ipa{[F5] zɯ˧hṽ˩ | \textasciitilde{}zɯ˧hṽ˩-ni˩gv̩˩}}}
\trans tout vert
\trans \textit{\textcolor{brown}{\zh{全绿}}}
\end{exe}


\paragraph{\hspace{-0.5cm} {\Large \textcolor{darkblue}{\textbf{\ipa{zɯ˧pv̩˩}}}}} \hypertarget{zM\string_Mpv\string_=\string_B1}{}
\markboth{\textcolor{darkblue}{\textbf{\ipa{zɯ˧pv̩˩}}}}{}
\textcolor{teal}{\mytextsc{nom}} \hspace{4pt} Ton~: L\#.

Foin; s'emploie aussi parfois pour désigner la paille: dans la maison, on ne stocke que de la paille de riz, pas de foin; l'herbe cueillie verte puis séchée (foin) n'est pas entreposée, mais aussitôt donnée aux animaux.
\textcolor{brown}{\zh{干草。}}


 \mytextsc{clf}~: \textcolor{darkblue}{\textbf{\ipa{kʰɤ˧˥}}} 

\paragraph{\hspace{-0.5cm} {\Large \textcolor{darkblue}{\textbf{\ipa{zɯ˧-qʰɑ˧mi\#˥}}}}} \hypertarget{zM\string_M-q\string_hA\string_Mmi\#\string_T1}{}
\markboth{\textcolor{darkblue}{\textbf{\ipa{zɯ˧-qʰɑ˧mi\#˥}}}}{}
\textcolor{teal}{\mytextsc{nom}} \hspace{4pt} Ton~: \#H.

\textit{Eulaliopsis binata (Retz.) C. E. Hubb.}, herbe sauvage. L'herbe n'est pas prisée du bétail, non plus que ses racines, et n'est jamais consommée par les humains. Les racines de cette herbe sont utilisées dans les rituels: elles ont une odeur forte à la combustion.
\textcolor{brown}{\zh{蓑草、山草、山草根、龙须草、山茅草、羊草、拟金茅。}}Dialecte chinois local~: \zh{狗尾巴草。}


 \mytextsc{clf}~: \textcolor{darkblue}{\textbf{\ipa{po˧}}} 

\paragraph{\hspace{-0.5cm} {\Large \textcolor{darkblue}{\textbf{\ipa{zɯ˧ɻ\#˥}}}}} \hypertarget{zM\string_Mr£`\#\string_T1}{}
\markboth{\textcolor{darkblue}{\textbf{\ipa{zɯ˧ɻ\#˥}}}}{}
\textcolor{teal}{\mytextsc{nom}} \hspace{4pt} Ton~: \#H.

Joue (partie basse, en-dessous des pommettes; vers l'articulation des deux mâchoires).
\textcolor{brown}{\zh{腮、腮帮子。}}


\begin{exe}
\sn \textcolor{darkblue}{\textbf{\ipa{zɯ˧ɻ̍˧ qʰwæ˩}}}
\trans gifler
\trans \textit{\textcolor{brown}{\zh{掌掴、打嘴巴}}}
\end{exe}
 \mytextsc{clf}~: \textcolor{darkblue}{\textbf{\ipa{ɭɯ˧}}} 

\paragraph{\hspace{-0.5cm} {\Large \textcolor{darkblue}{\textbf{\ipa{zɯ˧\textasciitilde{}zɯ˧}}}}} \hypertarget{zM\string_M~zM\string_M1}{}
\markboth{\textcolor{darkblue}{\textbf{\ipa{zɯ˧\textasciitilde{}zɯ˧}}}}{}
\textcolor{teal}{\mytextsc{nom}} \hspace{4pt} Ton~: M.

Vie, existence.
\textcolor{brown}{\zh{生命。}}


\begin{exe}
\sn \textcolor{darkblue}{\textbf{\ipa{hĩ˧-zɯ˧\textasciitilde{}zɯ˥\$}}}
\trans la vie humaine
\trans \textit{\textcolor{brown}{\zh{人生}}}
\end{exe}
\begin{exe}
\sn \textcolor{darkblue}{\textbf{\ipa{hĩ˧ zɯ˧ | ʂæ˧ | ʐwæ˩˥}}}
\trans une très longue vie / la vie est longue
\trans \textit{\textcolor{brown}{\zh{很长的人生 / 人生很长}}}
\end{exe}
 \mytextsc{clf}~: \textcolor{darkblue}{\textbf{\ipa{ljɤ˩}}} 

\paragraph{\hspace{-0.5cm} {\Large \textcolor{darkblue}{\textbf{\ipa{zɯ˩\textasciitilde{}zɯ˩}}}}} \hypertarget{zM\string_B~zM\string_B1}{}
\markboth{\textcolor{darkblue}{\textbf{\ipa{zɯ˩\textasciitilde{}zɯ˩}}}}{}
\textcolor{teal}{\mytextsc{verbe}} \hspace{4pt} Ton~: L.

Être engourdi.
\textcolor{brown}{\zh{麻木。}}


\begin{exe}
\sn \textcolor{darkblue}{\textbf{\ipa{gv̩˧dv̩˧gv̩˧mi˧ | zɯ˩\textasciitilde{}zɯ˩˥}}}
\trans avoir le corps engourdi
\trans \textit{\textcolor{brown}{\zh{身体麻木、全身麻木}}}
\end{exe}
\begin{exe}
\sn \textcolor{darkblue}{\textbf{\ipa{gv̩˧mi˧ | zɯ˩\textasciitilde{}zɯ˩˥}}}
\trans avoir le corps engourdi
\trans \textit{\textcolor{brown}{\zh{身体麻木、全身麻木}}}
\end{exe}
\begin{exe}
\sn \textcolor{darkblue}{\textbf{\ipa{tʰi˧-zɯ˩\textasciitilde{}zɯ˩}}}
\trans \mytextsc{dur} \mytextsc{red}
\trans \textit{\textcolor{brown}{\zh{\mytextsc{dur} \mytextsc{red}}}}
\end{exe}


\newpage
\section*{\centering- \textcolor{darkblue}{\textbf{\ipa{ʐ}}} -}
\paragraph{\hspace{-0.5cm} {\Large \textcolor{darkblue}{\textbf{\ipa{ʐ}}}}} \hypertarget{z`1}{}
\markboth{\textcolor{darkblue}{\textbf{\ipa{ʐ}}}}{}
\textcolor{teal}{\mytextsc{idéophone}} \hspace{4pt} Ton~: 0.

Bruit de grondement des grosses charges qu'on traîne sur le sol, des moteurs de camions: Brrroum!
\textcolor{brown}{\zh{形声词:轰隆隆!(拉很重的物品在地板上的隆隆声,卡车的隆隆声)。}}




\paragraph{\hspace{-0.5cm} {\Large \textcolor{darkblue}{\textbf{\ipa{ʐæ˧}}}}} \hypertarget{z`\{\string_M1}{}
\markboth{\textcolor{darkblue}{\textbf{\ipa{ʐæ˧}}}}{}
\textcolor{teal}{\mytextsc{adjectif}} \hspace{4pt} Ton~: M.

Grand et fort, massif, baraqué.
\textcolor{brown}{\zh{高大。}}


\begin{exe}
\sn \textcolor{darkblue}{\textbf{\ipa{ʐæ˧-ni˩gv̩˩}}}
\trans grand et fort
\trans \textit{\textcolor{brown}{\zh{高大}}}
\end{exe}
\begin{exe}
\sn \textcolor{darkblue}{\textbf{\ipa{hĩ˧ | ʈʂʰɯ˧-v̩˧, | ʐæ˧-ni˩gv̩˩!}}}
\trans Elle/il est grand(e) et fort(e) / impressionnant(e)!
\trans \textit{\textcolor{brown}{\zh{这人很高大!}}}
\end{exe}
\begin{exe}
\sn \textcolor{darkblue}{\textbf{\ipa{ʐæ˧ni˩ | mɤ˧-gv̩˧}}}
\trans pas bien grand (en taille), pas bien impressionnant
\trans \textit{\textcolor{brown}{\zh{个子不高}}}
\end{exe}
\begin{exe}
\sn \textcolor{darkblue}{\textbf{\ipa{ʐæ˧ | ʐwæ˩˥}}}
\trans très grand et fort
\trans \textit{\textcolor{brown}{\zh{很高大}}}
\end{exe}


\paragraph{\hspace{-0.5cm} {\Large \textcolor{darkblue}{\textbf{\ipa{ʐæ˧\textsubscript{a}}}}}} \hypertarget{z`\{\string_Ma1}{}
\markboth{\textcolor{darkblue}{\textbf{\ipa{ʐæ˧\textsubscript{a}}}}}{}
\textcolor{teal}{\mytextsc{verbe}} \hspace{4pt} Ton~: M\textsubscript{a}.
\ding{202} 
Rire.
\textcolor{brown}{\zh{笑。}}


\begin{exe}
\sn \textcolor{darkblue}{\textbf{\ipa{zo˧hṽ˥ | hĩ˧ ʐæ˧\textasciitilde{}ʐæ˥-kʰɯ˩}}}
\trans les enfants taquinent les gens, les font rire
\trans \textit{\textcolor{brown}{\zh{孩子们把大家逗笑了。}}}
\end{exe}
\begin{exe}
\sn \textcolor{darkblue}{\textbf{\ipa{hĩ˧ | ʐæ˧\textasciitilde{}ʐæ˥ kʰɯ˩}}}
\trans faire rire les gens, amuser les gens, faire rire le public
\trans \textit{\textcolor{brown}{\zh{让大家笑一笑}}}
\end{exe}
\begin{exe}
\sn \textcolor{darkblue}{\textbf{\ipa{ʐæ˧\textasciitilde{}ʐæ˩-di˩}}}
\trans plaisanteries, blagues
\trans \textit{\textcolor{brown}{\zh{笑话,好笑的话}}}
\end{exe}
\ding{203} 
Être impertinent, déranger, se moquer du monde.
\textcolor{brown}{\zh{嘲笑别人、出言不逊。}}


\begin{exe}
\sn \textcolor{darkblue}{\textbf{\ipa{le˧-ʐæ˧-ze˧}}}
\trans \mytextsc{accomp} \string_ \mytextsc{pfv}
\trans \textit{\textcolor{brown}{\zh{出言不逊了}}}
\end{exe}
\begin{exe}
\sn \textcolor{darkblue}{\textbf{\ipa{le˧-ʐæ˥\textasciitilde{}ʐæ˩}}}
\trans \mytextsc{red}
\trans \textit{\textcolor{brown}{\zh{笑一笑(别人)}}}
\end{exe}
\begin{exe}
\sn \textcolor{darkblue}{\textbf{\ipa{hĩ˧ ʐæ˩}}}
\trans être impertinent avec les gens, déranger les gens
\trans \textit{\textcolor{brown}{\zh{嘲笑人家}}}
\end{exe}
\begin{exe}
\sn \textcolor{darkblue}{\textbf{\ipa{le˧-ʐæ˥\textasciitilde{}ʐæ˩-ze˩}}}
\trans \mytextsc{red} \mytextsc{pfv}
\trans \textit{\textcolor{brown}{\zh{嘲笑了}}}
\end{exe}


\paragraph{\hspace{-0.5cm} {\Large \textcolor{darkblue}{\textbf{\ipa{ʐæ˧v̩˩-tʰv̩˩}}}}} \hypertarget{z`\{\string_Mv\string_=\string_B-t\string_hv\string_=\string_B1}{}
\markboth{\textcolor{darkblue}{\textbf{\ipa{ʐæ˧v̩˩-tʰv̩˩}}}}{}
\textcolor{teal}{\mytextsc{verbe}} \hspace{4pt} Ton~: L\#-.

Blaguer, faire une blague, faire une plaisanterie.
\textcolor{brown}{\zh{开玩笑。}}


\begin{exe}
\sn \textcolor{darkblue}{\textbf{\ipa{ʐæ˧v̩˩-tʰv̩˩ | ʐwæ˩˥}}}
\trans plaisanter follement, rire beaucoup
\trans \textit{\textcolor{brown}{\zh{开很多玩笑、一直开玩笑}}}
\end{exe}
\begin{exe}
\sn \textcolor{darkblue}{\textbf{\ipa{ʐæ˧v̩˩-tʰv̩˩-hĩ˩ ʐwɤ˩}}}
\trans lancer une blague, dire une plaisanterie
\trans \textit{\textcolor{brown}{\zh{开个玩笑}}}
\end{exe}


\paragraph{\hspace{-0.5cm} {\Large \textcolor{darkblue}{\textbf{\ipa{ʐæ˩\textsubscript{b}}}}}} \hypertarget{z`\{\string_Bb1}{}
\markboth{\textcolor{darkblue}{\textbf{\ipa{ʐæ˩\textsubscript{b}}}}}{}
\textcolor{teal}{\mytextsc{verbe}} \hspace{4pt} Ton~: L\textsubscript{b}.

Mélanger, tourner (un mélange, une préparation).
\textcolor{brown}{\zh{搅拌合混。}}


\begin{exe}
\sn \textcolor{darkblue}{\textbf{\ipa{le˧-ʐæ˧\textasciitilde{}ʐæ˥}}}
\trans \mytextsc{accomp} \string_ \mytextsc{red}
\trans \textit{\textcolor{brown}{\zh{搅拌}}}
\end{exe}


\paragraph{\hspace{-0.5cm} {\Large \textcolor{darkblue}{\textbf{\ipa{ʐæ˩mi\#˥}}}}} \hypertarget{z`\{\string_Bmi\#\string_T1}{}
\markboth{\textcolor{darkblue}{\textbf{\ipa{ʐæ˩mi\#˥}}}}{}
\textcolor{teal}{\mytextsc{nom}} \hspace{4pt} Ton~: LM+\#H.

Léopard femelle.
\textcolor{brown}{\zh{母豹子。}}


\begin{exe}
\sn \textcolor{darkblue}{\textbf{\ipa{ʐæ˩mi˧-ʐæ˥zo˩}}}
\trans léopard femelle et léopard mâle
\trans \textit{\textcolor{brown}{\zh{母豹子与公豹子}}}
\end{exe}
 \mytextsc{clf}~: \textcolor{darkblue}{\textbf{\ipa{pʰo˧˥}}} 

\paragraph{\hspace{-0.5cm} {\Large \textcolor{darkblue}{\textbf{\ipa{ʐæ˩pʰv̩˧}}}}} \hypertarget{z`\{\string_Bp\string_hv\string_=\string_M1}{}
\markboth{\textcolor{darkblue}{\textbf{\ipa{ʐæ˩pʰv̩˧}}}}{}
\textcolor{teal}{\mytextsc{nom}} \hspace{4pt} Ton~: LM.

Léopard mâle.
\textcolor{brown}{\zh{公豹子。}}


\begin{exe}
\sn \textcolor{darkblue}{\textbf{\ipa{ʐæ˩pʰv̩˧-ʐæ˩mi˩}}}
\trans léopard mâle et léopard femelle
\trans \textit{\textcolor{brown}{\zh{公豹子与母豹子}}}
\end{exe}
 \mytextsc{clf}~: \textcolor{darkblue}{\textbf{\ipa{pʰo˧˥}}} 

\paragraph{\hspace{-0.5cm} {\Large \textcolor{darkblue}{\textbf{\ipa{ʐæ˩sɯ˩}}}}} \hypertarget{z`\{\string_BsM\string_B1}{}
\markboth{\textcolor{darkblue}{\textbf{\ipa{ʐæ˩sɯ˩}}}}{}
\textcolor{teal}{\mytextsc{nom}} \hspace{4pt} Ton~: L.

Feutre grossier, fait uniquement de laine de mouton, dont on se drape en extérieur pour se protéger du froid.
\textcolor{brown}{\zh{披毡。}}


 \mytextsc{clf}~: \textcolor{darkblue}{\textbf{\ipa{ɭɯ˧˥}}} 

\paragraph{\hspace{-0.5cm} {\Large \textcolor{darkblue}{\textbf{\ipa{ʐæ˩sɯ˩-kʰwæ˩ɻæ˧}}}}} \hypertarget{z`\{\string_BsM\string_B-k\string_hw\{\string_Br£`\{\string_M1}{}
\markboth{\textcolor{darkblue}{\textbf{\ipa{ʐæ˩sɯ˩-kʰwæ˩ɻæ˧}}}}{}
\textcolor{teal}{\mytextsc{nom}} \hspace{4pt} Ton~: .

Natte en feutre. Le terme désigne spécifiquement les nattes/matelas/tissus en feutre véritable, par opposition avec le sens étendu que peut avoir \textcolor{darkblue}{\textbf{\ipa{/kʰwæ˧ɻæ\#˥/}}}.
\textcolor{brown}{\zh{毡子(真正的毡子)做的垫子。}}


 \mytextsc{clf}~: \textcolor{darkblue}{\textbf{\ipa{tsʰi˥}}} 

\paragraph{\hspace{-0.5cm} {\Large \textcolor{darkblue}{\textbf{\ipa{ʐæ˩ʂæ˧}}}}} \hypertarget{z`\{\string_Bs`\{\string_M1}{}
\markboth{\textcolor{darkblue}{\textbf{\ipa{ʐæ˩ʂæ˧}}}}{}
\textcolor{teal}{\mytextsc{adjectif}} \hspace{4pt} Ton~: LM.

Loin, lointain.
\textcolor{brown}{\zh{远。}}


\begin{exe}
\sn \textcolor{darkblue}{\textbf{\ipa{no˧ | ʈʂʰɯ˧ | ə˩-ʐæ˥ʂæ˩? | dʑɤ˩kʰwɤ˧ ə˩-di˩? | - dʑɤ˩˥ | dʑɤ˩kʰwɤ˧ mɤ˧-di˥! | mɤ˧-ʐæ˩ʂæ˩!}}}
\trans tu es loin de lui? Y a-t-il de la distance entre vous? (=vous êtes proches/intimes, ou pas?) - Non, il n'y a guère de distance! Nous ne somme pas éloignés!
\trans \textit{\textcolor{brown}{\zh{你们很熟吗? - 不很熟!}}}
\end{exe}


\paragraph{\hspace{-0.5cm} {\Large \textcolor{darkblue}{\textbf{\ipa{ʐæ˩tsɯ˧˥}}}}} \hypertarget{z`\{\string_BtsM\string_M\string_T1}{}
\markboth{\textcolor{darkblue}{\textbf{\ipa{ʐæ˩tsɯ˧˥}}}}{}
\textcolor{teal}{\mytextsc{nom}} \hspace{4pt} Ton~: LM+MH\#.

Sentier, petit chemin.
\textcolor{brown}{\zh{小路、径道。}}


\begin{exe}
\sn \textcolor{darkblue}{\textbf{\ipa{ʐæ˩tsɯ˧-ʐɤ˥mi˩}}}
\trans chemin de traverse, raccourci
\trans \textit{\textcolor{brown}{\zh{径道}}}
\end{exe}
 \mytextsc{clf}~: \textcolor{darkblue}{\textbf{\ipa{kʰɯ˩}}} 

\paragraph{\hspace{-0.5cm} {\Large \textcolor{darkblue}{\textbf{\ipa{ʐæ˩zo\#˥}}}}} \hypertarget{z`\{\string_Bzo\#\string_T1}{}
\markboth{\textcolor{darkblue}{\textbf{\ipa{ʐæ˩zo\#˥}}}}{}
\textcolor{teal}{\mytextsc{nom}} \hspace{4pt} Ton~: LM+\#H.

Bébé léopard, petit léopard.
\textcolor{brown}{\zh{小豹子。}}


\begin{exe}
\sn \textcolor{darkblue}{\textbf{\ipa{ʐæ˩zo˧-ʐæ˥mi˩}}}
\trans petit léopard et léopard femelle
\trans \textit{\textcolor{brown}{\zh{小豹子与母豹子}}}
\end{exe}
 \mytextsc{clf}~: \textcolor{darkblue}{\textbf{\ipa{ɭɯ˧}}} 

\paragraph{\hspace{-0.5cm} {\Large \textcolor{darkblue}{\textbf{\ipa{ʐæ˩˥}}}}} \hypertarget{z`\{\string_B\string_T1}{}
\markboth{\textcolor{darkblue}{\textbf{\ipa{ʐæ˩˥}}}}{}
\textcolor{teal}{\mytextsc{nom}} \hspace{4pt} Ton~: LH.

Léopard, panthère (note: ces deux termes sont homonymes en français).
\textcolor{brown}{\zh{豹子。}}


\begin{exe}
\sn \textcolor{darkblue}{\textbf{\ipa{ʐæ˩ dzɯ˧-ze˩}}}
\trans ...a mangé (un/le) léopard
\trans \textit{\textcolor{brown}{\zh{吃了豹子}}}
\end{exe}
\begin{exe}
\sn \textcolor{darkblue}{\textbf{\ipa{ʐæ˩ hwæ˧-ze˩}}}
\trans ...a acheté (un/le) léopard
\trans \textit{\textcolor{brown}{\zh{买了豹子}}}
\end{exe}
 \mytextsc{clf}~: \textcolor{darkblue}{\textbf{\ipa{pʰo˧˥}}} 

\paragraph{\hspace{-0.5cm} {\Large \textcolor{darkblue}{\textbf{\ipa{ʐe˥}}}}} \hypertarget{z`e\string_T1}{}
\markboth{\textcolor{darkblue}{\textbf{\ipa{ʐe˥}}}}{}
\textcolor{teal}{\mytextsc{classificateur}} \hspace{4pt} Ton~: H\textsubscript{a}.

Classificateur des morceaux de viande conservée.
\textcolor{brown}{\zh{量词:熏肉(一块)。}}




\paragraph{\hspace{-0.5cm} {\Large \textcolor{darkblue}{\textbf{\ipa{ʐe˥}}} \textsubscript{1}}} \hypertarget{z`e\string_T1}{}
\markboth{\textcolor{darkblue}{\textbf{\ipa{ʐe˥}}} \textsubscript{1}}{}
\textcolor{teal}{\mytextsc{nom}} \hspace{4pt} Ton~: \#H.

Flèche.
\textcolor{brown}{\zh{箭。}}


\begin{exe}
\sn \textcolor{darkblue}{\textbf{\ipa{ʐe˧ɻ̃˧ | ɖɯ˧-kʰɯ˩}}}
\trans une flèche; désigne aussi, de façon métaphorique, une lignée/une famille
\trans \textit{\textcolor{brown}{\zh{一枝箭。也来指一个家庭}}}
\end{exe}
 \mytextsc{clf}~: \textcolor{darkblue}{\textbf{\ipa{kʰɯ˩}}} 

\paragraph{\hspace{-0.5cm} {\Large \textcolor{darkblue}{\textbf{\ipa{ʐe˥}}} \textsubscript{2}}} \hypertarget{z`e\string_T2}{}
\markboth{\textcolor{darkblue}{\textbf{\ipa{ʐe˥}}} \textsubscript{2}}{}
\textcolor{teal}{\mytextsc{nom}} \hspace{4pt} Ton~: \#H.

Saison des pluies (été et automne: du 3e au 8e mois du calendrier lunaire).
\textcolor{brown}{\zh{雨季(夏天至秋天:三月份至八月份)。}}




\paragraph{\hspace{-0.5cm} {\Large \textcolor{darkblue}{\textbf{\ipa{ʐe˧ʈæ˥-ɬi˩}}}}} \hypertarget{z`e\string_Mt`\{\string_T-Ki\string_B1}{}
\markboth{\textcolor{darkblue}{\textbf{\ipa{ʐe˧ʈæ˥-ɬi˩}}}}{}
\textcolor{teal}{\mytextsc{nom}} \hspace{4pt} Ton~: H\#-L.

11e mois.
\textcolor{brown}{\zh{十一月。}}




\paragraph{\hspace{-0.5cm} {\Large \textcolor{darkblue}{\textbf{\ipa{ʐe˧v̩\#˥}}}}} \hypertarget{z`e\string_Mv\string_=\#\string_T1}{}
\markboth{\textcolor{darkblue}{\textbf{\ipa{ʐe˧v̩\#˥}}}}{}
\textcolor{teal}{\mytextsc{nom}} \hspace{4pt} Ton~: \#H.

Taureau castré.
\textcolor{brown}{\zh{阉牛。}}


 \mytextsc{clf}~: \textcolor{darkblue}{\textbf{\ipa{pʰo˧˥}}} 

\paragraph{\hspace{-0.5cm} {\Large \textcolor{darkblue}{\textbf{\ipa{ʐe˧zo\#˥}}}}} \hypertarget{z`e\string_Mzo\#\string_T1}{}
\markboth{\textcolor{darkblue}{\textbf{\ipa{ʐe˧zo\#˥}}}}{}
\textcolor{teal}{\mytextsc{nom}} \hspace{4pt} Ton~: \#H.

Flèche.
\textcolor{brown}{\zh{箭。}}


\begin{exe}
\sn \textcolor{darkblue}{\textbf{\ipa{ʐe˧zo˧ | ɖɯ˧-kʰɯ˩}}}
\trans une flèche
\trans \textit{\textcolor{brown}{\zh{一枝箭}}}
\end{exe}


\paragraph{\hspace{-0.5cm} {\Large \textcolor{darkblue}{\textbf{\ipa{ʐe˩ʐe˧-bæ˩bæ˩}}}}} \hypertarget{z`e\string_Bz`e\string_M-b\{\string_Bb\{\string_B1}{}
\markboth{\textcolor{darkblue}{\textbf{\ipa{ʐe˩ʐe˧-bæ˩bæ˩}}}}{}
\textcolor{teal}{\mytextsc{nom}} \hspace{4pt} Ton~: LM-L.

Coton sauvage;; littéralement ``la fleur des Occidentaux".
\textcolor{brown}{\zh{野棉花(直译:‘洋人花’)。}}


\textit{Voir~:} \hyperlink{je\string_Bz`e\string_M1}{\textcolor{darkblue}{\textbf{\ipa{je˩ʐe˧}}}} 

\paragraph{\hspace{-0.5cm} {\Large \textcolor{darkblue}{\textbf{\ipa{ʐe˩ʐe˧-læ˧tsɯ˥}}}}} \hypertarget{z`e\string_Bz`e\string_M-l\{\string_MtsM\string_T1}{}
\markboth{\textcolor{darkblue}{\textbf{\ipa{ʐe˩ʐe˧-læ˧tsɯ˥}}}}{}
\textcolor{teal}{\mytextsc{nom}} \hspace{4pt} Ton~: LM-H\#.

Sorte de fourrage pour les cochons (il y en a trois en tout).
\textcolor{brown}{\zh{喂猪的牧草。}}




\paragraph{\hspace{-0.5cm} {\Large \textcolor{darkblue}{\textbf{\ipa{ʐɤ˧\textsubscript{b}}}}}} \hypertarget{z`7\string_Mb1}{}
\markboth{\textcolor{darkblue}{\textbf{\ipa{ʐɤ˧\textsubscript{b}}}}}{}
\textcolor{teal}{\mytextsc{verbe}} \hspace{4pt} Ton~: M\textsubscript{b}.

Élever (des enfants ou des animaux); s'occuper de (personnes âgées).
\textcolor{brown}{\zh{饲养(动物)、养(孩子)、管(老人)。}}


\begin{exe}
\sn \textcolor{darkblue}{\textbf{\ipa{bo˩ ʐɤ˧}}}
\trans élever des cochons
\trans \textit{\textcolor{brown}{\zh{养猪}}}
\end{exe}
\begin{exe}
\sn \textcolor{darkblue}{\textbf{\ipa{ʐwæ˧zo˧ ʐɤ˧}}}
\trans élever des poulains
\trans \textit{\textcolor{brown}{\zh{养小马}}}
\end{exe}


\paragraph{\hspace{-0.5cm} {\Large \textcolor{darkblue}{\textbf{\ipa{ʐɤ˩\textsubscript{c}}}}}} \hypertarget{z`7\string_Bc1}{}
\markboth{\textcolor{darkblue}{\textbf{\ipa{ʐɤ˩\textsubscript{c}}}}}{}
\textcolor{teal}{\mytextsc{classificateur}} \hspace{4pt} Ton~: L\textsubscript{c}.

Classificateur des motifs, tracés, lignes, dans les dessins, peintures et tissages.
\textcolor{brown}{\zh{量词:图案(画画或织布)(一个)。}}




\paragraph{\hspace{-0.5cm} {\Large \textcolor{darkblue}{\textbf{\ipa{ʐɤ˩\textsubscript{a}}}}}} \hypertarget{z`7\string_Ba1}{}
\markboth{\textcolor{darkblue}{\textbf{\ipa{ʐɤ˩\textsubscript{a}}}}}{}
\textcolor{teal}{\mytextsc{adjectif}} \hspace{4pt} Ton~: L\textsubscript{a}.

Propre.
\textcolor{brown}{\zh{干净。}}


\begin{exe}
\sn \textcolor{darkblue}{\textbf{\ipa{ʈʂʰɯ˧ | ʐɤ˩-hĩ˩ ɲi˥}}}
\trans c'est propre
\trans \textit{\textcolor{brown}{\zh{这是干净的}}}
\end{exe}
\begin{exe}
\sn \textcolor{darkblue}{\textbf{\ipa{mɤ˧-ʐɤ˩}}}
\trans crasseux, dégoûtant (vêtements, nourriture…)
\trans \textit{\textcolor{brown}{\zh{不干净}}}
\end{exe}


\paragraph{\hspace{-0.5cm} {\Large \textcolor{darkblue}{\textbf{\ipa{ʐɤ˩mi˩}}}}} \hypertarget{z`7\string_Bmi\string_B1}{}
\markboth{\textcolor{darkblue}{\textbf{\ipa{ʐɤ˩mi˩}}}}{}
\textcolor{teal}{\mytextsc{nom}} \hspace{4pt} Ton~: L.

Route.
\textcolor{brown}{\zh{路。}}


\begin{exe}
\sn \textcolor{darkblue}{\textbf{\ipa{hĩ˧ | ɖɯ˧-v̩˧\textasciitilde{}ɖɯ˧-v̩˧ | le˧-se˥, | ʐɤ˩mi˩ tsɤ˩˥!}}}
\trans Contexte: on va couper du bois en montagne, à un endroit où il n'y a pas de chemin. Les gens se succèdent, et cela finit par ouvrir un chemin/former une sorte de chemin
\trans \textit{\textcolor{brown}{\zh{路是人走出来的!}}}
\end{exe}
 \mytextsc{clf}~: \textcolor{darkblue}{\textbf{\ipa{kʰɯ˩}}} 

\paragraph{\hspace{-0.5cm} {\Large \textcolor{darkblue}{\textbf{\ipa{ʐɤ˩ni˩}}}}} \hypertarget{z`7\string_Bni\string_B1}{}
\markboth{\textcolor{darkblue}{\textbf{\ipa{ʐɤ˩ni˩}}}}{}
\textcolor{teal}{\mytextsc{adjectif}} \hspace{4pt} Ton~: L.

Proche.
\textcolor{brown}{\zh{近。}}




\paragraph{\hspace{-0.5cm} {\Large \textcolor{darkblue}{\textbf{\ipa{ʐɤ˩qo˩}}}}} \hypertarget{z`7\string_Bqo\string_B1}{}
\markboth{\textcolor{darkblue}{\textbf{\ipa{ʐɤ˩qo˩}}}}{}
\textcolor{teal}{\mytextsc{nom}} \hspace{4pt} Ton~: L.
\ding{202} 
Veau.
\textcolor{brown}{\zh{小牛。}}


\ding{203} 
Pianniu, pienniu, dzo, zopiok (mâle).
\textcolor{brown}{\zh{公犏牛。}}


 \mytextsc{clf}~: \textcolor{darkblue}{\textbf{\ipa{ɭɯ˧}}} 

\paragraph{\hspace{-0.5cm} {\Large \textcolor{darkblue}{\textbf{\ipa{ʐɤ˩ʐɤ˧˥}}}}} \hypertarget{z`7\string_Bz`7\string_M\string_T1}{}
\markboth{\textcolor{darkblue}{\textbf{\ipa{ʐɤ˩ʐɤ˧˥}}}}{}
\textcolor{teal}{\mytextsc{nom}} \hspace{4pt} Ton~: LM+MH\#.

Motif.
\textcolor{brown}{\zh{花纹、图案。}}


\begin{exe}
\sn \textcolor{darkblue}{\textbf{\ipa{ʐɤ˩ʐɤ˧ tʰi˧-di˥}}}
\trans qui a des motifs, des dessins (ex.: un tissu)
\trans \textit{\textcolor{brown}{\zh{有花纹}}}
\end{exe}
\begin{exe}
\sn \textcolor{darkblue}{\textbf{\ipa{[F5] bɑ˩lɑ˩˥ | ʈʰɯ˧-ɭɯ˥-bi˩ | ʐɤ˩ʐɤ˧ tʰi˧-di˥}}}
\trans sur ce vêtement il y a un motif
\trans \textit{\textcolor{brown}{\zh{这衣服上面有花纹。}}}
\end{exe}
 \mytextsc{clf}~: \textcolor{darkblue}{\textbf{\ipa{ʐɤ˩}}} 

\paragraph{\hspace{-0.5cm} {\Large \textcolor{darkblue}{\textbf{\ipa{ʐɤ˩˧}}}}} \hypertarget{z`7\string_B\string_M1}{}
\markboth{\textcolor{darkblue}{\textbf{\ipa{ʐɤ˩˧}}}}{}
\textcolor{teal}{\mytextsc{nom}} \hspace{4pt} Ton~: LM.

Route (monosyllabe).
\textcolor{brown}{\zh{路(单音节)。}}


\begin{exe}
\sn \textcolor{darkblue}{\textbf{\ipa{ʐɤ˩mi˩-qo˥}}}
\trans sur le chemin
\trans \textit{\textcolor{brown}{\zh{路上}}}
\end{exe}
\begin{exe}
\sn \textcolor{darkblue}{\textbf{\ipa{ʐɤ˩mi˩-qo˥, | hĩ˧ se˧! |}}}
\trans il y a des gens qui passent sur le chemin!
\trans \textit{\textcolor{brown}{\zh{路上有人走!}}}
\end{exe}
\begin{exe}
\sn \textcolor{darkblue}{\textbf{\ipa{ʐɤ˩ se˩-zo˩˥}}}
\trans voyageur, homme qui voyage; spécifiquement: personne partant faire du commerce en caravane
\trans \textit{\textcolor{brown}{\zh{旅人,特别指走马帮的商人}}}
\end{exe}
 \mytextsc{clf}~: \textcolor{darkblue}{\textbf{\ipa{kʰɯ˩}}} objets longs

\paragraph{\hspace{-0.5cm} {\Large \textcolor{darkblue}{\textbf{\ipa{ʐo˩}}}}} \hypertarget{z`o\string_B1}{}
\markboth{\textcolor{darkblue}{\textbf{\ipa{ʐo˩}}}}{}
\textcolor{teal}{\mytextsc{nom}} \hspace{4pt} Ton~: L.

Midi; repas de midi/déjeuner.
\textcolor{brown}{\zh{中午。}}


\begin{exe}
\sn \textcolor{darkblue}{\textbf{\ipa{ʐo˩ dzɯ˩˥}}}
\trans prendre son déjeuner
\trans \textit{\textcolor{brown}{\zh{吃午饭}}}
\end{exe}


\paragraph{\hspace{-0.5cm} {\Large \textcolor{darkblue}{\textbf{\ipa{ʐo˩\textsubscript{a}}}} \textsubscript{1}}} \hypertarget{z`o\string_Ba1}{}
\markboth{\textcolor{darkblue}{\textbf{\ipa{ʐo˩\textsubscript{a}}}} \textsubscript{1}}{}
\textcolor{teal}{\mytextsc{verbe}} \hspace{4pt} Ton~: L\textsubscript{a}.

Se balancer.
\textcolor{brown}{\zh{甩来甩去。}}


\begin{exe}
\sn \textcolor{darkblue}{\textbf{\ipa{ɖɯ˧-ʐo˩-ɻ̍˩}}}
\trans se balancer un peu
\trans \textit{\textcolor{brown}{\zh{甩来甩去}}}
\end{exe}
\begin{exe}
\sn \textcolor{darkblue}{\textbf{\ipa{ʐo˩\textasciitilde{}ʐo˧-ze˥}}}
\trans \mytextsc{red} \mytextsc{pfv}
\trans \textit{\textcolor{brown}{\zh{\mytextsc{red} \mytextsc{pfv}}}}
\end{exe}
\begin{exe}
\sn \textcolor{darkblue}{\textbf{\ipa{[PHONO] le˧-ʐo˩\textasciitilde{}ʐo˩}}}
\trans \mytextsc{accomp} \mytextsc{red}
\trans \textit{\textcolor{brown}{\zh{\mytextsc{accomp} \mytextsc{red}}}}
\end{exe}


\paragraph{\hspace{-0.5cm} {\Large \textcolor{darkblue}{\textbf{\ipa{ʐo˩\textsubscript{a}}}} \textsubscript{2}}} \hypertarget{z`o\string_Ba2}{}
\markboth{\textcolor{darkblue}{\textbf{\ipa{ʐo˩\textsubscript{a}}}} \textsubscript{2}}{}
\textcolor{teal}{\mytextsc{adjectif}} \hspace{4pt} Ton~: L\textsubscript{a}.

Léger.
\textcolor{brown}{\zh{轻。}}




\paragraph{\hspace{-0.5cm} {\Large \textcolor{darkblue}{\textbf{\ipa{ʐo˩dzɯ˩}}}}} \hypertarget{z`o\string_BdzM\string_B1}{}
\markboth{\textcolor{darkblue}{\textbf{\ipa{ʐo˩dzɯ˩}}}}{}
\textcolor{teal}{\mytextsc{verbe}} \hspace{4pt} Ton~: L.

Déjeuner, prendre le repas de midi.
\textcolor{brown}{\zh{吃午饭。}}


\begin{exe}
\sn \textcolor{darkblue}{\textbf{\ipa{ʐo˩ dzɯ˩˥}}}
\trans déjeuner (verbe), prendre le déjeuner
\trans \textit{\textcolor{brown}{\zh{吃午饭}}}
\end{exe}
\begin{exe}
\sn \textcolor{darkblue}{\textbf{\ipa{ʐo˩ dzɯ˩-se˥}}}
\trans l'après-midi
\trans \textit{\textcolor{brown}{\zh{下午}}}
\end{exe}


\paragraph{\hspace{-0.5cm} {\Large \textcolor{darkblue}{\textbf{\ipa{ʐo˩\textasciitilde{}ʐo˧˥}}}}} \hypertarget{z`o\string_B~z`o\string_M\string_T1}{}
\markboth{\textcolor{darkblue}{\textbf{\ipa{ʐo˩\textasciitilde{}ʐo˧˥}}}}{}
\textcolor{teal}{\mytextsc{verbe}} \hspace{4pt} Ton~: MH.

Se balancer.
\textcolor{brown}{\zh{摔、摇摆。}}




\paragraph{\hspace{-0.5cm} {\Large \textcolor{darkblue}{\textbf{\ipa{ʐɯ˥}}}}} \hypertarget{z`M\string_T1}{}
\markboth{\textcolor{darkblue}{\textbf{\ipa{ʐɯ˥}}}}{}
\textcolor{teal}{\mytextsc{adjectif}} \hspace{4pt} Ton~: H.

Lourd.
\textcolor{brown}{\zh{重。}}




\paragraph{\hspace{-0.5cm} {\Large \textcolor{darkblue}{\textbf{\ipa{ʐɯ˧}}}}} \hypertarget{z`M\string_M1}{}
\markboth{\textcolor{darkblue}{\textbf{\ipa{ʐɯ˧}}}}{}
\textcolor{teal}{\mytextsc{nom}} \hspace{4pt} Ton~: M.

Alcool fermenté, chang, vin.
\textcolor{brown}{\zh{酒。}}


\begin{exe}
\sn \textcolor{darkblue}{\textbf{\ipa{ʐɯ˧ pʰv̩˧˥}}}
\trans verser à boire
\trans \textit{\textcolor{brown}{\zh{斟酒}}}
\end{exe}


\paragraph{\hspace{-0.5cm} {\Large \textcolor{darkblue}{\textbf{\ipa{ʐɯ˧ɭɯ˧}}}}} \hypertarget{z`M\string_Ml\string_RM\string_M1}{}
\markboth{\textcolor{darkblue}{\textbf{\ipa{ʐɯ˧ɭɯ˧}}}}{}
\textcolor{teal}{\mytextsc{verbe}} \hspace{4pt} Ton~: M.

Tremblement de terre/la terre tremble.
\textcolor{brown}{\zh{地震。}}


\begin{exe}
\sn \textcolor{darkblue}{\textbf{\ipa{ʐɯ˧ɭɯ˧-ze˧!}}}
\trans Il y a un tremblement de terre! / La terre tremble!
\trans \textit{\textcolor{brown}{\zh{地震了!}}}
\end{exe}


\paragraph{\hspace{-0.5cm} {\Large \textcolor{darkblue}{\textbf{\ipa{ʐɯ˧nɑ˩}}}}} \hypertarget{z`M\string_MnA\string_B1}{}
\markboth{\textcolor{darkblue}{\textbf{\ipa{ʐɯ˧nɑ˩}}}}{}
\textcolor{teal}{\mytextsc{nom}} \hspace{4pt} Ton~: L\#.

Alcool fort; alcool de qualité supérieure.
\textcolor{brown}{\zh{醇酒,好酒。}}




\paragraph{\hspace{-0.5cm} {\Large \textcolor{darkblue}{\textbf{\ipa{ʐɯ˩dzi˥}}}}} \hypertarget{z`M\string_Bdzi\string_T1}{}
\markboth{\textcolor{darkblue}{\textbf{\ipa{ʐɯ˩dzi˥}}}}{}
\textcolor{teal}{\mytextsc{nom}} \hspace{4pt} Ton~: LH.

Cèdre.
\textcolor{brown}{\zh{杉树。}}


 \mytextsc{clf}~: \textcolor{darkblue}{\textbf{\ipa{dzi˩}}} 

\paragraph{\hspace{-0.5cm} {\Large \textcolor{darkblue}{\textbf{\ipa{ʐɯ˩gv̩˩}}}}} \hypertarget{z`M\string_Bgv\string_=\string_B1}{}
\markboth{\textcolor{darkblue}{\textbf{\ipa{ʐɯ˩gv̩˩}}}}{}
\textcolor{teal}{\mytextsc{nom}} \hspace{4pt} Ton~: L.

Canot, bateau (utilisé uniquement pour les barques circulant sur le Lac, pas pour les autres bateaux).
\textcolor{brown}{\zh{船。}}


\begin{exe}
\sn \textcolor{darkblue}{\textbf{\ipa{ʐɯ˩gv̩˩ dzi˩˥}}}
\trans être assis dans un bateau, être à bord d'un bateau
\trans \textit{\textcolor{brown}{\zh{坐船}}}
\end{exe}
 \mytextsc{clf}~: \textcolor{darkblue}{\textbf{\ipa{ɭɯ˧}}} \textcolor{darkblue}{\textbf{\ipa{nɑ˧}}} 

\paragraph{\hspace{-0.5cm} {\Large \textcolor{darkblue}{\textbf{\ipa{ʐɯ˩-mo˧˥}}}}} \hypertarget{z`M\string_B-mo\string_M\string_T1}{}
\markboth{\textcolor{darkblue}{\textbf{\ipa{ʐɯ˩-mo˧˥}}}}{}
\textcolor{teal}{\mytextsc{nom}} \hspace{4pt} Ton~: LM+MH\#.

``champignon des cèdres"; champignon comestible, de la même famille que le ``champignon des sapins", \textcolor{darkblue}{\textbf{\ipa{/tʰo˧-mo˩/}}}.
\textcolor{brown}{\zh{“杉树菌”:一种菌子。}}


\begin{exe}
\sn \textcolor{darkblue}{\textbf{\ipa{tʰo˧mo˩-ʐɯ˩mo˩}}}
\trans champignon des sapins et champignon des cèdres
\trans \textit{\textcolor{brown}{\zh{松树菌与杉树菌}}}
\end{exe}
\textit{Voir~:} \hyperlink{z`M\string_Bdzi\string_T1}{\textcolor{darkblue}{\textbf{\ipa{ʐɯ˩dzi˥}}}} 

\paragraph{\hspace{-0.5cm} {\Large \textcolor{darkblue}{\textbf{\ipa{ʐɯ˩tse˧}}}}} \hypertarget{z`M\string_Btse\string_M1}{}
\markboth{\textcolor{darkblue}{\textbf{\ipa{ʐɯ˩tse˧}}}}{}
\textcolor{teal}{\mytextsc{nom}} \hspace{4pt} Ton~: LM.

Esprit de la montagne.
\textcolor{brown}{\zh{山神。}}


 \mytextsc{clf}~: \textcolor{darkblue}{\textbf{\ipa{v̩˧}}} 

\paragraph{\hspace{-0.5cm} {\Large \textcolor{darkblue}{\textbf{\ipa{ʐɯ˩tse˧-mæ˧ʂæ˩}}}}} \hypertarget{z`M\string_Btse\string_M-m\{\string_Ms`\{\string_B1}{}
\markboth{\textcolor{darkblue}{\textbf{\ipa{ʐɯ˩tse˧-mæ˧ʂæ˩}}}}{}
\textcolor{teal}{\mytextsc{nom}} \hspace{4pt} Ton~: LM-L\#.

Faisan doré.
\textcolor{brown}{\zh{锦鸡。}}Dialecte chinois local~: \zh{山扎拉。}


\textit{Voir~:} \hyperlink{z`M\string_Btse\string_M1}{\textcolor{darkblue}{\textbf{\ipa{ʐɯ˩tse˧}}}} 

\paragraph{\hspace{-0.5cm} {\Large \textcolor{darkblue}{\textbf{\ipa{ʐɯ˩tsɯ˧}}}}} \hypertarget{z`M\string_BtsM\string_M1}{}
\markboth{\textcolor{darkblue}{\textbf{\ipa{ʐɯ˩tsɯ˧}}}}{}
\textcolor{teal}{\mytextsc{nom}} \hspace{4pt} Ton~: LM.

Jours, temps.
\textcolor{brown}{\zh{日子(汉语借词)。}}

 Emprunt~: chinois \textcolor{brown}{\zh{日子}}

\begin{exe}
\sn \textcolor{darkblue}{\textbf{\ipa{ʐɯ˩tsɯ˧ ʈʂɤ˧}}}
\trans rechercher une date propice (pour la construction d'une maison ou autre projet important)
\trans \textit{\textcolor{brown}{\zh{算日子(为了选择吉利的一天)}}}
\end{exe}


\paragraph{\hspace{-0.5cm} {\Large \textcolor{darkblue}{\textbf{\ipa{ʐɯ˩tsɯ˧mɤ˩ʈʂʰɤ˩}}}}} \hypertarget{z`M\string_BtsM\string_Mm7\string_Bt`s`\string_h7\string_B1}{}
\markboth{\textcolor{darkblue}{\textbf{\ipa{ʐɯ˩tsɯ˧mɤ˩ʈʂʰɤ˩}}}}{}
\textcolor{teal}{\mytextsc{nom}} \hspace{4pt} Ton~: LM-L.

Matelas.
\textcolor{brown}{\zh{褥子。}}


 \mytextsc{clf}~: \textcolor{darkblue}{\textbf{\ipa{tsʰi˥}}} 

\paragraph{\hspace{-0.5cm} {\Large \textcolor{darkblue}{\textbf{\ipa{ʐv̩˧}}}}} \hypertarget{z`v\string_=\string_M1}{}
\markboth{\textcolor{darkblue}{\textbf{\ipa{ʐv̩˧}}}}{}
\textcolor{teal}{\mytextsc{nombre}} \hspace{4pt} Ton~: M? H\#?.

4.
\textcolor{brown}{\zh{4。}}




\paragraph{\hspace{-0.5cm} {\Large \textcolor{darkblue}{\textbf{\ipa{ʐv̩˧˥}}}}} \hypertarget{z`v\string_=\string_M\string_T1}{}
\markboth{\textcolor{darkblue}{\textbf{\ipa{ʐv̩˧˥}}}}{}
\textcolor{teal}{\mytextsc{verbe}} \hspace{4pt} Ton~: MH.

Coudre.
\textcolor{brown}{\zh{缝。}}




\paragraph{\hspace{-0.5cm} {\Large \textcolor{darkblue}{\textbf{\ipa{ʐv̩˩\textsubscript{a}}}} \textsubscript{1}}} \hypertarget{z`v\string_=\string_Ba1}{}
\markboth{\textcolor{darkblue}{\textbf{\ipa{ʐv̩˩\textsubscript{a}}}} \textsubscript{1}}{}
\textcolor{teal}{\mytextsc{verbe}} \hspace{4pt} Ton~: L\textsubscript{a}.
\ding{202} 
Pétrir (la pâte), malaxer.
\textcolor{brown}{\zh{揉(面)。}}


\begin{exe}
\sn \textcolor{darkblue}{\textbf{\ipa{pɤ˩jɤ˧ ʐv̩˥}}}
\trans pétrir la pâte
\trans \textit{\textcolor{brown}{\zh{揉面}}}
\end{exe}
\begin{exe}
\sn \textcolor{darkblue}{\textbf{\ipa{ʐv̩˧\textasciitilde{}ʐv̩˥}}}
\trans \mytextsc{red}
\trans \textit{\textcolor{brown}{\zh{\mytextsc{重叠}}}}
\end{exe}
\begin{exe}
\sn \textcolor{darkblue}{\textbf{\ipa{ɖɯ˧-kʰwɤ˧ ʐv̩˥}}}
\trans pétrir un morceau
\trans \textit{\textcolor{brown}{\zh{揉一块(面团)}}}
\end{exe}
\ding{203} 
Froisser, plisser.
\textcolor{brown}{\zh{皱(衣服)。}}


\begin{exe}
\sn \textcolor{darkblue}{\textbf{\ipa{bɑ˩lɑ˩ ʐv̩˥(-ze˩)}}}
\trans les vêtements sont froissés, les vêtements ont été froissés
\trans \textit{\textcolor{brown}{\zh{衣服皱了}}}
\end{exe}
\begin{exe}
\sn \textcolor{darkblue}{\textbf{\ipa{ʐv̩˧\textasciitilde{}ʐv̩˥}}}
\trans \mytextsc{red}
\trans \textit{\textcolor{brown}{\zh{\mytextsc{重叠}}}}
\end{exe}
\begin{exe}
\sn \textcolor{darkblue}{\textbf{\ipa{le˧-ʐv̩˧\textasciitilde{}ʐv̩˥-ze˩}}}
\trans \mytextsc{accomp} \string_ \mytextsc{red} \mytextsc{pfv}
\trans \textit{\textcolor{brown}{\zh{\mytextsc{accomp} \string_ \mytextsc{red} \mytextsc{pfv}}}}
\end{exe}


\paragraph{\hspace{-0.5cm} {\Large \textcolor{darkblue}{\textbf{\ipa{ʐv̩˩\textsubscript{a}}}} \textsubscript{2}}} \hypertarget{z`v\string_=\string_Ba2}{}
\markboth{\textcolor{darkblue}{\textbf{\ipa{ʐv̩˩\textsubscript{a}}}} \textsubscript{2}}{}
\textcolor{teal}{\mytextsc{adjectif}} \hspace{4pt} Ton~: L\textsubscript{a}.

Bon (au goût).
\textcolor{brown}{\zh{好吃。}}




\paragraph{\hspace{-0.5cm} {\Large \textcolor{darkblue}{\textbf{\ipa{ʐv̩˧bæ˧}}}}} \hypertarget{z`v\string_=\string_Mb\{\string_M1}{}
\markboth{\textcolor{darkblue}{\textbf{\ipa{ʐv̩˧bæ˧}}}}{}
\textcolor{teal}{\mytextsc{nom}} \hspace{4pt} Ton~: M.

Serpent.
\textcolor{brown}{\zh{蛇。}}


\begin{exe}
\sn \textcolor{darkblue}{\textbf{\ipa{ʐv̩˧bæ˧ ɣɯ˩ pʰv̩˩}}}
\trans Le serpent mue
\trans \textit{\textcolor{brown}{\zh{蛇蜕皮}}}
\end{exe}
 \mytextsc{clf}~: \textcolor{darkblue}{\textbf{\ipa{mi˩}}} 

\paragraph{\hspace{-0.5cm} {\Large \textcolor{darkblue}{\textbf{\ipa{ʐv̩˧bæ˧-bv̩˧-hɑ\#˥}}}}} \hypertarget{z`v\string_=\string_Mb\{\string_M-bv\string_=\string_M-hA\#\string_T1}{}
\markboth{\textcolor{darkblue}{\textbf{\ipa{ʐv̩˧bæ˧-bv̩˧-hɑ\#˥}}}}{}
\textcolor{teal}{\mytextsc{nom}} \hspace{4pt} Ton~: \#H.

L'une des trois sortes de fourrage que l'on donne aux cochons; M18 propose comme étymologie populaire ``serpent en colère", du fait que cette plante a des feuilles grasses qui ressemblent à un serpent, et qui sont entortillées.
\textcolor{brown}{\zh{能喂给猪的三种草之一。}}


 \mytextsc{clf}~: \textcolor{darkblue}{\textbf{\ipa{qɑ˩}}} botte

\paragraph{\hspace{-0.5cm} {\Large \textcolor{darkblue}{\textbf{\ipa{ʐv̩˧bæ˧-mi˩}}}}} \hypertarget{z`v\string_=\string_Mb\{\string_M-mi\string_B1}{}
\markboth{\textcolor{darkblue}{\textbf{\ipa{ʐv̩˧bæ˧-mi˩}}}}{}
\textcolor{teal}{\mytextsc{nom}} \hspace{4pt} Ton~: \mytextsc{L}.

Serpent femelle.
\textcolor{brown}{\zh{母蛇。}}


 \mytextsc{clf}~: \textcolor{darkblue}{\textbf{\ipa{mi˩}}} 

\paragraph{\hspace{-0.5cm} {\Large \textcolor{darkblue}{\textbf{\ipa{ʐv̩˧bæ˧-pʰv̩\#˥}}}}} \hypertarget{z`v\string_=\string_Mb\{\string_M-p\string_hv\string_=\#\string_T1}{}
\markboth{\textcolor{darkblue}{\textbf{\ipa{ʐv̩˧bæ˧-pʰv̩\#˥}}}}{}
\textcolor{teal}{\mytextsc{nom}} \hspace{4pt} Ton~: \#H.

Serpent mâle.
\textcolor{brown}{\zh{公蛇。}}


 \mytextsc{clf}~: \textcolor{darkblue}{\textbf{\ipa{mi˩}}} 

\paragraph{\hspace{-0.5cm} {\Large \textcolor{darkblue}{\textbf{\ipa{ʐv̩˧bæ˧-zo\#˥}}}}} \hypertarget{z`v\string_=\string_Mb\{\string_M-zo\#\string_T1}{}
\markboth{\textcolor{darkblue}{\textbf{\ipa{ʐv̩˧bæ˧-zo\#˥}}}}{}
\textcolor{teal}{\mytextsc{nom}} \hspace{4pt} Ton~: \#H.

Petit serpent.
\textcolor{brown}{\zh{小蛇。}}


 \mytextsc{clf}~: \textcolor{darkblue}{\textbf{\ipa{ɭɯ˧}}} 

\paragraph{\hspace{-0.5cm} {\Large \textcolor{darkblue}{\textbf{\ipa{ʐv̩˧bæ˧-ʐv̩˧qʰɑ\#˥}}}}} \hypertarget{z`v\string_=\string_Mb\{\string_M-z`v\string_=\string_Mq\string_hA\#\string_T1}{}
\markboth{\textcolor{darkblue}{\textbf{\ipa{ʐv̩˧bæ˧-ʐv̩˧qʰɑ\#˥}}}}{}
\textcolor{teal}{\mytextsc{nom}} \hspace{4pt} Ton~: .

\textit{Arisaema consanguineum}.
\textcolor{brown}{\zh{天南星。}}




\paragraph{\hspace{-0.5cm} {\Large \textcolor{darkblue}{\textbf{\ipa{ʐv̩˧bɤ\#˥}}}}} \hypertarget{z`v\string_=\string_Mb7\#\string_T1}{}
\markboth{\textcolor{darkblue}{\textbf{\ipa{ʐv̩˧bɤ\#˥}}}}{}
\textcolor{teal}{\mytextsc{nom}} \hspace{4pt} Ton~: \#H.

Les Pumi des montagnes, du côté de Muli et Jiaze.
\textcolor{brown}{\zh{高山普米族(永宁以北地区:木里等)。}}


 \mytextsc{clf}~: \textcolor{darkblue}{\textbf{\ipa{v̩˧}}} 

\paragraph{\hspace{-0.5cm} {\Large \textcolor{darkblue}{\textbf{\ipa{ʐv̩˧di˧˥}}}}} \hypertarget{z`v\string_=\string_Mdi\string_M\string_T1}{}
\markboth{\textcolor{darkblue}{\textbf{\ipa{ʐv̩˧di˧˥}}}}{}
\textcolor{teal}{\mytextsc{nom}} \hspace{4pt} Ton~: MH\#.

Les rives du Yangtze; le climat y est chaud et humide. Ces régions sont perçus par les Na de Yongning comme peuplées de Pumi; ils imaginent que les habitants de Fengke et Labai seraient des descendants des Pumi. (Source: consultants F4, F5, M21.).
\textcolor{brown}{\zh{金沙江边的地方(气候热)。}}




\paragraph{\hspace{-0.5cm} {\Large \textcolor{darkblue}{\textbf{\ipa{ʐv̩˧dzi˩}}}}} \hypertarget{z`v\string_=\string_Mdzi\string_B1}{}
\markboth{\textcolor{darkblue}{\textbf{\ipa{ʐv̩˧dzi˩}}}}{}
\textcolor{teal}{\mytextsc{nom}} \hspace{4pt} Ton~: L\#.

Saule.
\textcolor{brown}{\zh{柳树,杨柳。}}


 \mytextsc{clf}~: \textcolor{darkblue}{\textbf{\ipa{dzi˩}}} 

\paragraph{\hspace{-0.5cm} {\Large \textcolor{darkblue}{\textbf{\ipa{ʐv̩˧hĩ\#˥}}}}} \hypertarget{z`v\string_=\string_Mhi\string_~\#\string_T1}{}
\markboth{\textcolor{darkblue}{\textbf{\ipa{ʐv̩˧hĩ\#˥}}}}{}
\textcolor{teal}{\mytextsc{nom}} \hspace{4pt} Ton~: \#H.

Désignation des Pumi.
\textcolor{brown}{\zh{普米族。}}


 \mytextsc{clf}~: \textcolor{darkblue}{\textbf{\ipa{v̩˧}}} 

\paragraph{\hspace{-0.5cm} {\Large \textcolor{darkblue}{\textbf{\ipa{ʐv̩˩-ɬi˩mi˩}}}}} \hypertarget{z`v\string_=\string_B-Ki\string_Bmi\string_B1}{}
\markboth{\textcolor{darkblue}{\textbf{\ipa{ʐv̩˩-ɬi˩mi˩}}}}{}
\textcolor{teal}{\mytextsc{nom}} \hspace{4pt} Ton~: L.

4e mois.
\textcolor{brown}{\zh{四月。}}




\paragraph{\hspace{-0.5cm} {\Large \textcolor{darkblue}{\textbf{\ipa{ʐv̩˩ɭɯ˥}}}}} \hypertarget{z`v\string_=\string_Bl\string_RM\string_T1}{}
\markboth{\textcolor{darkblue}{\textbf{\ipa{ʐv̩˩ɭɯ˥}}}}{}
\textcolor{teal}{\mytextsc{nom}} \hspace{4pt} Ton~: LH.

Poutre soutenant la toiture, posée horizontalement, dans le sens de la longueur du bâtiment. Sur elle reposent les poutrelles courtes posées inclinées dans le sens de la largeur du bâtiment, /hæ̃˧kʰɤ˧˥/.
\textcolor{brown}{\zh{支撑顶板的梁。}}


 \mytextsc{clf}~: \textcolor{darkblue}{\textbf{\ipa{ɭɯ˧}}} 

\paragraph{\hspace{-0.5cm} {\Large \textcolor{darkblue}{\textbf{\ipa{ʐv̩˩mi˩}}}}} \hypertarget{z`v\string_=\string_Bmi\string_B1}{}
\markboth{\textcolor{darkblue}{\textbf{\ipa{ʐv̩˩mi˩}}}}{}
\textcolor{teal}{\mytextsc{nom}} \hspace{4pt} Ton~: L.

Arc.
\textcolor{brown}{\zh{弓。}}


 \mytextsc{clf}~: \textcolor{darkblue}{\textbf{\ipa{nɑ˧}}} 

\paragraph{\hspace{-0.5cm} {\Large \textcolor{darkblue}{\textbf{\ipa{ʐv̩˧mi\#˥}}}}} \hypertarget{z`v\string_=\string_Mmi\#\string_T1}{}
\markboth{\textcolor{darkblue}{\textbf{\ipa{ʐv̩˧mi\#˥}}}}{}
\textcolor{teal}{\mytextsc{nom}} \hspace{4pt} Ton~: \#H.

Petite-fille.
\textcolor{brown}{\zh{孙女。}}


\begin{exe}
\sn \textcolor{darkblue}{\textbf{\ipa{njɤ˧ | ʐv̩˧mi˧ | ɖɯ˧-ɭɯ˧ dʑo˧}}}
\trans J'ai une petite-fille.
\trans \textit{\textcolor{brown}{\zh{我有一个孙女。}}}
\end{exe}
 \mytextsc{clf}~: \textcolor{darkblue}{\textbf{\ipa{ɭɯ˧}}} 

\paragraph{\hspace{-0.5cm} {\Large \textcolor{darkblue}{\textbf{\ipa{ʐv̩˧mv̩˧lɑ˧di˧˥}}}}} \hypertarget{z`v\string_=\string_Mmv\string_=\string_MlA\string_Mdi\string_M\string_T1}{}
\markboth{\textcolor{darkblue}{\textbf{\ipa{ʐv̩˧mv̩˧lɑ˧di˧˥}}}}{}
\textcolor{teal}{\mytextsc{nom}} \hspace{4pt} Ton~: MH\#.

Les territoires des Pumi, au bord du fleuve Yangtze. Le terme est connoté péjorativement: cette région est perçue comme périphérique et moins plaisante que la plaine de Yongning.
\textcolor{brown}{\zh{江边普米族地区(带偏见的说法)。}}


\begin{exe}
\sn \textcolor{darkblue}{\textbf{\ipa{ʐv̩˧mv̩˧lɑ˧di˧-hĩ˥}}}
\trans habitants des territoires pumi des bords du fleuve; personnes pumi
\trans \textit{\textcolor{brown}{\zh{普米族地区的人们}}}
\end{exe}
 \mytextsc{clf}~: \textcolor{darkblue}{\textbf{\ipa{v̩˧}}} 

\paragraph{\hspace{-0.5cm} {\Large \textcolor{darkblue}{\textbf{\ipa{ʐv̩˧-ɲi˧-ʁo˧tʰo˥}}}}} \hypertarget{z`v\string_=\string_M-Ji\string_M-Ro\string_Mt\string_ho\string_T1}{}
\markboth{\textcolor{darkblue}{\textbf{\ipa{ʐv̩˧-ɲi˧-ʁo˧tʰo˥}}}}{}
\textcolor{teal}{\mytextsc{adverbe}} \hspace{4pt} Ton~: H\#.

Dans quatre jours.
\textcolor{brown}{\zh{四天以后。}}




\paragraph{\hspace{-0.5cm} {\Large \textcolor{darkblue}{\textbf{\ipa{ʐv̩˧ɻ̍˥}}}}} \hypertarget{z`v\string_=\string_Mr£`̍\string_T1}{}
\markboth{\textcolor{darkblue}{\textbf{\ipa{ʐv̩˧ɻ̍˥}}}}{}
\textcolor{teal}{\mytextsc{adjectif}} \hspace{4pt} Ton~: H\#.

Carré.
\textcolor{brown}{\zh{正方形。}}


\begin{exe}
\sn \textcolor{darkblue}{\textbf{\ipa{ʐv̩˩-hĩ˩˥}}}
\trans \mytextsc{nmlz}
\trans \textit{\textcolor{brown}{\zh{方形的}}}
\end{exe}
\begin{exe}
\sn \textcolor{darkblue}{\textbf{\ipa{ʐv̩˧ɻ̍˥-gv̩˩}}}
\trans carré
\trans \textit{\textcolor{brown}{\zh{方形的}}}
\end{exe}


\paragraph{\hspace{-0.5cm} {\Large \textcolor{darkblue}{\textbf{\ipa{ʐv̩˧-tsʰi˩}}}}} \hypertarget{z`v\string_=\string_M-ts\string_hi\string_B1}{}
\markboth{\textcolor{darkblue}{\textbf{\ipa{ʐv̩˧-tsʰi˩}}}}{}
\textcolor{teal}{\mytextsc{nombre}} \hspace{4pt} Ton~: L\#.

40.
\textcolor{brown}{\zh{40。}}




\paragraph{\hspace{-0.5cm} {\Large \textcolor{darkblue}{\textbf{\ipa{ʐv̩˧v̩˥-ʐv̩˩mi˩}}}}} \hypertarget{z`v\string_=\string_Mv\string_=\string_T-z`v\string_=\string_Bmi\string_B1}{}
\markboth{\textcolor{darkblue}{\textbf{\ipa{ʐv̩˧v̩˥-ʐv̩˩mi˩}}}}{}
\textcolor{teal}{\mytextsc{nom}} \hspace{4pt} Ton~: H\#-.

Petits-enfants.
\textcolor{brown}{\zh{孙子孙女。}}




\paragraph{\hspace{-0.5cm} {\Large \textcolor{darkblue}{\textbf{\ipa{ʐv̩˧v̩\#˥}}}}} \hypertarget{z`v\string_=\string_Mv\string_=\#\string_T1}{}
\markboth{\textcolor{darkblue}{\textbf{\ipa{ʐv̩˧v̩\#˥}}}}{}
\textcolor{teal}{\mytextsc{nom}} \hspace{4pt} Ton~: \#H.

Petit-fils.
\textcolor{brown}{\zh{孙子。}}


\begin{exe}
\sn \textcolor{darkblue}{\textbf{\ipa{njɤ˧ | ʐv̩˧v̩˧ ɖɯ˧-ɭɯ˧ dʑo˧.}}}
\trans j'ai un petit-fils
\trans \textit{\textcolor{brown}{\zh{我有一个孙子。}}}
\end{exe}
 \mytextsc{clf}~: \textcolor{darkblue}{\textbf{\ipa{ɭɯ˧}}} objets ronds

\paragraph{\hspace{-0.5cm} {\Large \textcolor{darkblue}{\textbf{\ipa{ʐv̩˧-zo\#˥}}}}} \hypertarget{z`v\string_=\string_M-zo\#\string_T1}{}
\markboth{\textcolor{darkblue}{\textbf{\ipa{ʐv̩˧-zo\#˥}}}}{}
\textcolor{teal}{\mytextsc{nom}} \hspace{4pt} Ton~: \#H.

Petit arc.
\textcolor{brown}{\zh{小弓。}}


 \mytextsc{clf}~: \textcolor{darkblue}{\textbf{\ipa{nɑ˧}}} 

\paragraph{\hspace{-0.5cm} {\Large \textcolor{darkblue}{\textbf{\ipa{ʐwæ˥}}}}} \hypertarget{z`w\{\string_T1}{}
\markboth{\textcolor{darkblue}{\textbf{\ipa{ʐwæ˥}}}}{}
\textcolor{teal}{\mytextsc{nom}} \hspace{4pt} Ton~: \#H.

Cheval.
\textcolor{brown}{\zh{马。}}


\begin{exe}
\sn \textcolor{darkblue}{\textbf{\ipa{dʑɯ˩ʁo˩-ʐwæ˩}}}
\trans cheval sauvage, non domestiqué
\trans \textit{\textcolor{brown}{\zh{野马}}}
\end{exe}
\begin{exe}
\sn \textcolor{darkblue}{\textbf{\ipa{o-ho-ho! ʐwæ˧-ɳɯ˩ | dzɯ˧-po˧-hɯ˥-ze˩!}}}
\trans Houlàlà! Le cheval nous l'a mangé! (Contexte: on laisse dans la cour des céréales, ou du fourrage, et pendant qu'on a le dos tourné, le cheval chaparde cette nourriture.)
\trans \textit{\textcolor{brown}{\zh{啊呀嚒!马把饲料都吃光了!}}}
\end{exe}
 \mytextsc{clf}~: \textcolor{darkblue}{\textbf{\ipa{v̩˧}}} 

\paragraph{\hspace{-0.5cm} {\Large \textcolor{darkblue}{\textbf{\ipa{ʐwæ˧\textsubscript{a}}}}}} \hypertarget{z`w\{\string_Ma1}{}
\markboth{\textcolor{darkblue}{\textbf{\ipa{ʐwæ˧\textsubscript{a}}}}}{}
\textcolor{teal}{\mytextsc{verbe}} \hspace{4pt} Ton~: M\textsubscript{a}.

Peser (à l'aide d'une balance).
\textcolor{brown}{\zh{称。}}


\begin{exe}
\sn \textcolor{darkblue}{\textbf{\ipa{mɤ˧-ʐwæ˧}}}
\trans \mytextsc{neg}
\trans \textit{\textcolor{brown}{\zh{不称}}}
\end{exe}
\begin{exe}
\sn \textcolor{darkblue}{\textbf{\ipa{le˧-ʐwæ˧-ze˧}}}
\trans \mytextsc{accomp} \string_ \mytextsc{pfv}
\trans \textit{\textcolor{brown}{\zh{称了}}}
\end{exe}
\begin{exe}
\sn \textcolor{darkblue}{\textbf{\ipa{tso˧\textasciitilde{}tso˧ ʐwæ˩}}}
\trans peser des choses
\trans \textit{\textcolor{brown}{\zh{称东西}}}
\end{exe}
\begin{exe}
\sn \textcolor{darkblue}{\textbf{\ipa{ʁo˧do˧ ʐwæ˧}}}
\trans peser des noix
\trans \textit{\textcolor{brown}{\zh{称核桃}}}
\end{exe}


\paragraph{\hspace{-0.5cm} {\Large \textcolor{darkblue}{\textbf{\ipa{ʐwæ˧bv̩˧˥}}}}} \hypertarget{z`w\{\string_Mbv\string_=\string_M\string_T1}{}
\markboth{\textcolor{darkblue}{\textbf{\ipa{ʐwæ˧bv̩˧˥}}}}{}
\textcolor{teal}{\mytextsc{nom}} \hspace{4pt} Ton~: MH\#.

Enclos des chevaux.
\textcolor{brown}{\zh{马圈。}}


 \mytextsc{clf}~: \textcolor{darkblue}{\textbf{\ipa{ɭɯ˧}}} 

\paragraph{\hspace{-0.5cm} {\Large \textcolor{darkblue}{\textbf{\ipa{ʐwæ˧-hɑ\#˥}}}}} \hypertarget{z`w\{\string_M-hA\#\string_T1}{}
\markboth{\textcolor{darkblue}{\textbf{\ipa{ʐwæ˧-hɑ\#˥}}}}{}
\textcolor{teal}{\mytextsc{nom}} \hspace{4pt} Ton~: \#H.

Nourriture pour cheval.
\textcolor{brown}{\zh{马料、马饲料。}}




\paragraph{\hspace{-0.5cm} {\Large \textcolor{darkblue}{\textbf{\ipa{ʐwæ˧-kʰv̩˩}}}}} \hypertarget{z`w\{\string_M-k\string_hv\string_=\string_B1}{}
\markboth{\textcolor{darkblue}{\textbf{\ipa{ʐwæ˧-kʰv̩˩}}}}{}
\textcolor{teal}{\mytextsc{nom}} \hspace{4pt} Ton~: L\#.

Année du cheval.
\textcolor{brown}{\zh{马年。}}




\paragraph{\hspace{-0.5cm} {\Large \textcolor{darkblue}{\textbf{\ipa{ʐwæ˧-ɭɯ\#˥}}}}} \hypertarget{z`w\{\string_M-l\string_RM\#\string_T1}{}
\markboth{\textcolor{darkblue}{\textbf{\ipa{ʐwæ˧-ɭɯ\#˥}}}}{}
\textcolor{teal}{\mytextsc{nom}} \hspace{4pt} Ton~: \#H.

Nourriture pour cheval (céréales).
\textcolor{brown}{\zh{马料(粮食)。}}




\paragraph{\hspace{-0.5cm} {\Large \textcolor{darkblue}{\textbf{\ipa{ʐwæ˧pʰæ˧di˧˥}}}}} \hypertarget{z`w\{\string_Mp\string_h\{\string_Mdi\string_M\string_T1}{}
\markboth{\textcolor{darkblue}{\textbf{\ipa{ʐwæ˧pʰæ˧di˧˥}}}}{}
\textcolor{teal}{\mytextsc{nom}} \hspace{4pt} Ton~: MH\#.

Longe, objet pour accrocher le cheval.
\textcolor{brown}{\zh{拉马链子。}}


 \mytextsc{clf}~: \textcolor{darkblue}{\textbf{\ipa{ɭɯ˧}}} 

\paragraph{\hspace{-0.5cm} {\Large \textcolor{darkblue}{\textbf{\ipa{ʐwæ˧-qʰæ\#˥}}}}} \hypertarget{z`w\{\string_M-q\string_h\{\#\string_T1}{}
\markboth{\textcolor{darkblue}{\textbf{\ipa{ʐwæ˧-qʰæ\#˥}}}}{}
\textcolor{teal}{\mytextsc{nom}} \hspace{4pt} Ton~: \#H.

Crottin de cheval.
\textcolor{brown}{\zh{马粪。}}


 \mytextsc{clf}~: \textcolor{darkblue}{\textbf{\ipa{ʁwɤ˧}}} 

\paragraph{\hspace{-0.5cm} {\Large \textcolor{darkblue}{\textbf{\ipa{ʐwæ˧ʁo˩}}}}} \hypertarget{z`w\{\string_MRo\string_B1}{}
\markboth{\textcolor{darkblue}{\textbf{\ipa{ʐwæ˧ʁo˩}}}}{}
\textcolor{teal}{\mytextsc{nom}} \hspace{4pt} Ton~: L\#.

Cheval castré.
\textcolor{brown}{\zh{骟马。}}


 \mytextsc{clf}~: \textcolor{darkblue}{\textbf{\ipa{v̩˧}}} 

\paragraph{\hspace{-0.5cm} {\Large \textcolor{darkblue}{\textbf{\ipa{ʐwæ˧sɯ˩}}}}} \hypertarget{z`w\{\string_MsM\string_B1}{}
\markboth{\textcolor{darkblue}{\textbf{\ipa{ʐwæ˧sɯ˩}}}}{}
\textcolor{teal}{\mytextsc{nom}} \hspace{4pt} Ton~: L\#.

Étalon.
\textcolor{brown}{\zh{公马。}}


\begin{exe}
\sn \textcolor{darkblue}{\textbf{\ipa{ʐwæ˧sɯ˩-ʐwæ˩mi˩}}}
\trans étalon et jument
\trans \textit{\textcolor{brown}{\zh{公马与母马}}}
\end{exe}
\begin{exe}
\sn \textcolor{darkblue}{\textbf{\ipa{ʐwæ˧sɯ˩-ʐwæ˩zo˩}}}
\trans étalon et poulain
\trans \textit{\textcolor{brown}{\zh{公马与小马}}}
\end{exe}
 \mytextsc{clf}~: \textcolor{darkblue}{\textbf{\ipa{mi˩}}} 

\paragraph{\hspace{-0.5cm} {\Large \textcolor{darkblue}{\textbf{\ipa{ʐwæ˧zo\#˥}}}}} \hypertarget{z`w\{\string_Mzo\#\string_T1}{}
\markboth{\textcolor{darkblue}{\textbf{\ipa{ʐwæ˧zo\#˥}}}}{}
\textcolor{teal}{\mytextsc{nom}} \hspace{4pt} Ton~: \#H.

Poulain.
\textcolor{brown}{\zh{马驹子。}}


\begin{exe}
\sn \textcolor{darkblue}{\textbf{\ipa{ʐwæ˧zo˧-ʐwæ˥mi˩}}}
\trans poulain et jument
\trans \textit{\textcolor{brown}{\zh{马驹子与母马}}}
\end{exe}
 \mytextsc{clf}~: \textcolor{darkblue}{\textbf{\ipa{ɭɯ˧}}} 

\paragraph{\hspace{-0.5cm} {\Large \textcolor{darkblue}{\textbf{\ipa{ʐwæ˧-zɯ\#˥}}}}} \hypertarget{z`w\{\string_M-zM\#\string_T1}{}
\markboth{\textcolor{darkblue}{\textbf{\ipa{ʐwæ˧-zɯ\#˥}}}}{}
\textcolor{teal}{\mytextsc{nom}} \hspace{4pt} Ton~: \#H.

Fourrage pour le cheval, foin pour cheval.
\textcolor{brown}{\zh{喂马的草。}}




\paragraph{\hspace{-0.5cm} {\Large \textcolor{darkblue}{\textbf{\ipa{ʐwæ˧\textasciitilde{}ʐwæ˧}}}}} \hypertarget{z`w\{\string_M~z`w\{\string_M1}{}
\markboth{\textcolor{darkblue}{\textbf{\ipa{ʐwæ˧\textasciitilde{}ʐwæ˧}}}}{}
\textcolor{teal}{\mytextsc{verbe}} \hspace{4pt} Ton~: M.
\ding{202} 
Ranger (des objets).
\textcolor{brown}{\zh{收拾。}}


\begin{exe}
\sn \textcolor{darkblue}{\textbf{\ipa{tso˧\textasciitilde{}tso˧ ʐwæ˧\textasciitilde{}ʐwæ˧(-ze˩)}}}
\trans ranger des choses
\trans \textit{\textcolor{brown}{\zh{收拾东西}}}
\end{exe}
\begin{exe}
\sn \textcolor{darkblue}{\textbf{\ipa{le˧-ʐwæ˧\textasciitilde{}ʐwæ˧ ɖɯ˧-ʝi˧-tɕɯ˥}}}
\trans ranger des choses et les mettre à leur place, ranger des choses ensemble
\trans \textit{\textcolor{brown}{\zh{把东西收拾在一起}}}
\end{exe}
\ding{203} 
Rassembler (des gens).
\textcolor{brown}{\zh{聚集。}}




\paragraph{\hspace{-0.5cm} {\Large \textcolor{darkblue}{\textbf{\ipa{ʐwæ˩}}}}} \hypertarget{z`w\{\string_B1}{}
\markboth{\textcolor{darkblue}{\textbf{\ipa{ʐwæ˩}}}}{}
\textcolor{teal}{\mytextsc{adverbe}} \hspace{4pt} Ton~: L.

Extrêmement.
\textcolor{brown}{\zh{很、极。}}


\begin{exe}
\sn \textcolor{darkblue}{\textbf{\ipa{ʐwæ˩-ze˥!}}}
\trans Il y en a trop! / ça fait trop!
\trans \textit{\textcolor{brown}{\zh{太多了!}}}
\end{exe}


\paragraph{\hspace{-0.5cm} {\Large \textcolor{darkblue}{\textbf{\ipa{ʐwæ˩\textsubscript{a}}}} \textsubscript{1}}} \hypertarget{z`w\{\string_Ba1}{}
\markboth{\textcolor{darkblue}{\textbf{\ipa{ʐwæ˩\textsubscript{a}}}} \textsubscript{1}}{}
\textcolor{teal}{\mytextsc{verbe}} \hspace{4pt} Ton~: L\textsubscript{a}.

S'évanouir.
\textcolor{brown}{\zh{昏,昏厥。}}


\begin{exe}
\sn \textcolor{darkblue}{\textbf{\ipa{le˧-ʈʰi˩ | le˧-ʐwæ˩-ze˩}}}
\trans s'évanouir à force de fatigue, s'évanouir d'épuisement
\trans \textit{\textcolor{brown}{\zh{累得都昏倒了}}}
\end{exe}


\paragraph{\hspace{-0.5cm} {\Large \textcolor{darkblue}{\textbf{\ipa{ʐwæ˩\textsubscript{a}}}} \textsubscript{2}}} \hypertarget{z`w\{\string_Ba2}{}
\markboth{\textcolor{darkblue}{\textbf{\ipa{ʐwæ˩\textsubscript{a}}}} \textsubscript{2}}{}
\textcolor{teal}{\mytextsc{adjectif}} \hspace{4pt} Ton~: L\textsubscript{a}.

Habile, bon, capable.
\textcolor{brown}{\zh{好,能干。}}


\begin{exe}
\sn \textcolor{darkblue}{\textbf{\ipa{ʐwæ˩-hĩ˩˥}}}
\trans \mytextsc{nmlz}
\trans \textit{\textcolor{brown}{\zh{能干的}}}
\end{exe}
\begin{exe}
\sn \textcolor{darkblue}{\textbf{\ipa{ʈʂʰɯ˧ ɖwæ˧˥ | ʐwæ˩˥!}}}
\trans Il est très habile! / Il est formidable!
\trans \textit{\textcolor{brown}{\zh{他很能干!}}}
\end{exe}


\paragraph{\hspace{-0.5cm} {\Large \textcolor{darkblue}{\textbf{\ipa{ʐwæ˩mi˩}}}}} \hypertarget{z`w\{\string_Bmi\string_B1}{}
\markboth{\textcolor{darkblue}{\textbf{\ipa{ʐwæ˩mi˩}}}}{}
\textcolor{teal}{\mytextsc{nom}} \hspace{4pt} Ton~: L.

Jument; également employé pour une jeune jument: pouliche, ``bébé cheval" de sexe féminin.
\textcolor{brown}{\zh{母马。}}


\begin{exe}
\sn \textcolor{darkblue}{\textbf{\ipa{ʂe˩-ʐwæ˩mi˥}}}
\trans vélo; néologisme introduit par F4 d'après le taïwanais tiěmǎ \textcolor{brown}{\zh{铁马}} que j'essayais de traduire en na.
\trans \textit{\textcolor{brown}{\zh{自行车(“铁马”)}}}
\end{exe}
\begin{exe}
\sn \textcolor{darkblue}{\textbf{\ipa{ʐwæ˩mi˩-ʐwæ˩zo˩}}}
\trans jument et poulain
\trans \textit{\textcolor{brown}{\zh{母马与马驹子}}}
\end{exe}
 \mytextsc{clf}~: \textcolor{darkblue}{\textbf{\ipa{v̩˧}}} \textcolor{darkblue}{\textbf{\ipa{jɤ˧˥}}} 

\paragraph{\hspace{-0.5cm} {\Large \textcolor{darkblue}{\textbf{\ipa{ʐwæ˧˥}}}}} \hypertarget{z`w\{\string_M\string_T1}{}
\markboth{\textcolor{darkblue}{\textbf{\ipa{ʐwæ˧˥}}}}{}
\textcolor{teal}{\mytextsc{verbe}} \hspace{4pt} Ton~: MH.

Sarcler, biner.
\textcolor{brown}{\zh{薅锄、锄草。}}


\begin{exe}
\sn \textcolor{darkblue}{\textbf{\ipa{ʐwæ˩\textasciitilde{}ʐwæ˧˥}}}
\trans \mytextsc{red}
\trans \textit{\textcolor{brown}{\zh{\mytextsc{重叠}}}}
\end{exe}
\begin{exe}
\sn \textcolor{darkblue}{\textbf{\ipa{jɤ˩jo˥ ʐwæ˩}}}
\trans sarcler des pommes de terre
\trans \textit{\textcolor{brown}{\zh{洋芋地里锄草}}}
\end{exe}
\begin{exe}
\sn \textcolor{darkblue}{\textbf{\ipa{jɤ˩jo˧ ʐwæ˧\textasciitilde{}ʐwæ˥}}}
\trans sarcler des pommes de terre
\trans \textit{\textcolor{brown}{\zh{洋芋地里锄草}}}
\end{exe}
\begin{exe}
\sn \textcolor{darkblue}{\textbf{\ipa{qʰɑ˧dze˧ ʐwæ˧˥}}}
\trans sarcler du maïs
\trans \textit{\textcolor{brown}{\zh{苞谷地里锄草}}}
\end{exe}
\begin{exe}
\sn \textcolor{darkblue}{\textbf{\ipa{qʰɑ˧dze˧ ʐwæ˧\textasciitilde{}ʐwæ˥}}}
\trans sarcler du maïs
\trans \textit{\textcolor{brown}{\zh{苞谷地里锄草}}}
\end{exe}


\paragraph{\hspace{-0.5cm} {\Large \textcolor{darkblue}{\textbf{\ipa{ʐwɤ˧}}}}} \hypertarget{z`w7\string_M1}{}
\markboth{\textcolor{darkblue}{\textbf{\ipa{ʐwɤ˧}}}}{}
\textcolor{teal}{\mytextsc{adjectif}} \hspace{4pt} Ton~: M.

Qui a faim (forme monosyllabique). Se combine en disyllabe avec le mot ``nourriture".
\textcolor{brown}{\zh{饿。}}


\begin{exe}
\sn \textcolor{darkblue}{\textbf{\ipa{hɑ˧-ʐwɤ˩}}}
\trans avoir faim
\trans \textit{\textcolor{brown}{\zh{饿}}}
\end{exe}


\paragraph{\hspace{-0.5cm} {\Large \textcolor{darkblue}{\textbf{\ipa{ʐwɤ˧mv̩˧}}}}} \hypertarget{z`w7\string_Mmv\string_=\string_M1}{}
\markboth{\textcolor{darkblue}{\textbf{\ipa{ʐwɤ˧mv̩˧}}}}{}
\textcolor{teal}{\mytextsc{nom}} \hspace{4pt} Ton~: M.
\textit{De:} \textbf{ʐwɤ˩b} 
Formule toute faite, expression toute faite, expression idiomatique.
\textcolor{brown}{\zh{惯用语、习惯语、习语。}}


\begin{exe}
\sn \textcolor{darkblue}{\textbf{\ipa{ʐwɤ˧mv̩˧ dʑo˧-kv̩˧˥ !}}}
\trans C'est comme ça qu'on dit! / Il y a une expression comme ça!
\trans \textit{\textcolor{brown}{\zh{有这么一句老话! / 有这么一个说法!}}}
\end{exe}
\begin{exe}
\sn \textcolor{darkblue}{\textbf{\ipa{æ˧ʂæ˧-ʐwɤ˧mv̩˧ | ɖɯ˧-kʰwɤ˥}}}
\trans une expression toute faite du temps jadis, un dicton
\trans \textit{\textcolor{brown}{\zh{一句老话、一个传统的说法}}}
\end{exe}
 \mytextsc{clf}~: \textcolor{darkblue}{\textbf{\ipa{kʰwɤ˥}}} 

\paragraph{\hspace{-0.5cm} {\Large \textcolor{darkblue}{\textbf{\ipa{ʐwɤ˩\textsubscript{b}}}}}} \hypertarget{z`w7\string_Bb1}{}
\markboth{\textcolor{darkblue}{\textbf{\ipa{ʐwɤ˩\textsubscript{b}}}}}{}
\textcolor{teal}{\mytextsc{verbe}} \hspace{4pt} Ton~: L\textsubscript{b}.

Parler.
\textcolor{brown}{\zh{讲话。}}


\begin{exe}
\sn \textcolor{darkblue}{\textbf{\ipa{ʐwɤ˧\textasciitilde{}ʐwɤ˩ mɤ˩-hĩ˩}}}
\trans muet, personne muette, personne qui ne parle pas
\trans \textit{\textcolor{brown}{\zh{哑巴、不会讲话的人}}}
\end{exe}
\begin{exe}
\sn \textcolor{darkblue}{\textbf{\ipa{ʐwɤ˧\textasciitilde{}ʐwɤ˩ mɤ˩-hĩ˩, | ʈʂʰɯ˧-v̩˧!}}}
\trans Elle/il ne sait pas parler, elle/lui!
\trans \textit{\textcolor{brown}{\zh{不会讲话,这个人! / 这个人,不会讲话!}}}
\end{exe}
\begin{exe}
\sn \textcolor{darkblue}{\textbf{\ipa{le˧-ʐwɤ˩-ze˩}}}
\trans \mytextsc{accomp} \string_ \mytextsc{pfv}
\trans \textit{\textcolor{brown}{\zh{讲了}}}
\end{exe}
\begin{exe}
\sn \textcolor{darkblue}{\textbf{\ipa{no˧ | ə˧tso˧ ʐwɤ˩-ɲi˩?}}}
\trans Que dis-tu? / Qu'est-ce que tu veux dire?
\trans \textit{\textcolor{brown}{\zh{你说什么?}}}
\end{exe}
\begin{exe}
\sn \textcolor{darkblue}{\textbf{\ipa{ʐwɤ˧\textasciitilde{}ʐwɤ˩}}}
\trans \mytextsc{red}
\trans PHONO
\trans \textit{\textcolor{brown}{\zh{\mytextsc{重叠}}}}
\end{exe}
\begin{exe}
\sn \textcolor{darkblue}{\textbf{\ipa{le˧-ʐwɤ˩}}}
\trans répondre, donner une réponse
\trans \textit{\textcolor{brown}{\zh{回答}}}
\end{exe}
\begin{exe}
\sn \textcolor{darkblue}{\textbf{\ipa{le˧-wo˧ ʐwɤ˧˥}}}
\trans répondre, donner une réponse
\trans \textit{\textcolor{brown}{\zh{回答}}}
\end{exe}
\begin{exe}
\sn \textcolor{darkblue}{\textbf{\ipa{ʈʂʰɯ˧ | le˧-ʐwɤ˩-bi˩-dʑo˩...}}}
\trans A ce qu'elle/il dit...
\trans \textit{\textcolor{brown}{\zh{依照他的说法……}}}
\end{exe}
\begin{exe}
\sn \textcolor{darkblue}{\textbf{\ipa{hĩ˧-qɑ˧ ʐwɤ˧\textasciitilde{}ʐwɤ˥}}}
\trans parler aux gens
\trans \textit{\textcolor{brown}{\zh{对人家讲}}}
\end{exe}
\begin{exe}
\sn \textcolor{darkblue}{\textbf{\ipa{le˧-ʐwɤ˧\textasciitilde{}ʐwɤ˥-ze˩}}}
\trans \mytextsc{accomp} \mytextsc{red} \mytextsc{pfv}
\trans \textit{\textcolor{brown}{\zh{\mytextsc{accomp} \mytextsc{red} \mytextsc{pfv}}}}
\end{exe}


\newpage
\section*{\centering- \textcolor{darkblue}{\textbf{\ipa{ʑ}}} -}
\paragraph{\hspace{-0.5cm} {\Large \textcolor{darkblue}{\textbf{\ipa{ʑi˥}}}}} \hypertarget{z£i\string_T1}{}
\markboth{\textcolor{darkblue}{\textbf{\ipa{ʑi˥}}}}{}
\textcolor{teal}{\mytextsc{verbe}} \hspace{4pt} Ton~: H.

Être présent, y avoir; propriété du corps, de l'âme, d'un objet… Ex.: avoir de la force; avoir de la barbe; il y a une resserre dans la maison.
\textcolor{brown}{\zh{有,拥有(抽象:有力量,有勇气)。}}


\begin{exe}
\sn \textcolor{darkblue}{\textbf{\ipa{mɤ˧-ʑi˥}}}
\trans \mytextsc{neg}
\trans \textit{\textcolor{brown}{\zh{没有}}}
\end{exe}


\paragraph{\hspace{-0.5cm} {\Large \textcolor{darkblue}{\textbf{\ipa{ʑi˧\textsubscript{a}}}}}} \hypertarget{z£i\string_Ma1}{}
\markboth{\textcolor{darkblue}{\textbf{\ipa{ʑi˧\textsubscript{a}}}}}{}
\textcolor{teal}{\mytextsc{verbe}} \hspace{4pt} Ton~: M\textsubscript{a}.
\ding{202} 
Couler, avoir une fuite; s'écouler (fleuve).
\textcolor{brown}{\zh{漏(水)。}}


\begin{exe}
\sn \textcolor{darkblue}{\textbf{\ipa{mv̩˩tɕo˧ ʑi˧}}}
\trans fuir, avoir une fuite
\trans \textit{\textcolor{brown}{\zh{(水)往下漏}}}
\end{exe}
\ding{203} 
S'écouler, couler (rivière).
\textcolor{brown}{\zh{流(河水流着)。}}


\begin{exe}
\sn \textcolor{darkblue}{\textbf{\ipa{mv̩˩tɕo˧ ʑi˧}}}
\trans couler (rivière)
\trans \textit{\textcolor{brown}{\zh{(河)往下游流}}}
\end{exe}


\paragraph{\hspace{-0.5cm} {\Large \textcolor{darkblue}{\textbf{\ipa{ʑi˧dv̩˧}}}}} \hypertarget{z£i\string_Mdv\string_=\string_M1}{}
\markboth{\textcolor{darkblue}{\textbf{\ipa{ʑi˧dv̩˧}}}}{}
\textcolor{teal}{\mytextsc{nom}} \hspace{4pt} Ton~: M.
\ding{202} 
Maisonnée.
\textcolor{brown}{\zh{家。}}


\begin{exe}
\sn \textcolor{darkblue}{\textbf{\ipa{ɖɯ˧-ʑi˩dv̩˩}}}
\trans une maisonnée/ toute la maison
\trans \textit{\textcolor{brown}{\zh{一家人,包括所有成员}}}
\end{exe}
\begin{exe}
\sn \textcolor{darkblue}{\textbf{\ipa{ɑ˩ʁo˧-ʑi˧dv̩˧ ʝi˧}}}
\trans s'occuper de la maison, veiller au bon fonctionnement de la maison
\trans \textit{\textcolor{brown}{\zh{管家}}}
\end{exe}
\ding{203} 
L'ensemble de la ferme, comprenant plusieurs bâtiments, le bétail et les gens.
\textcolor{brown}{\zh{农舍,包括院子、人住的楼、动物住的楼等。}}


 \mytextsc{clf}~: \textcolor{darkblue}{\textbf{\ipa{ɭɯ˧}}} \textcolor{darkblue}{\textbf{\ipa{ɭɯ˧}}} 

\paragraph{\hspace{-0.5cm} {\Large \textcolor{darkblue}{\textbf{\ipa{ʑi˧dv̩˧ʝi˧-hĩ\#˥}}}}} \hypertarget{z£i\string_Mdv\string_=\string_Mj££i\string_M-hi\string_~\#\string_T1}{}
\markboth{\textcolor{darkblue}{\textbf{\ipa{ʑi˧dv̩˧ʝi˧-hĩ\#˥}}}}{}
\textcolor{teal}{\mytextsc{nom}} \hspace{4pt} Ton~: H\#.

La personne qui s'occupe de la maison, le maître/la maîtresse de céans.
\textcolor{brown}{\zh{一家之主、家长。}}




\paragraph{\hspace{-0.5cm} {\Large \textcolor{darkblue}{\textbf{\ipa{ʑi˧kv̩˧wo˧}}}}} \hypertarget{z£i\string_Mkv\string_=\string_Mwo\string_M1}{}
\markboth{\textcolor{darkblue}{\textbf{\ipa{ʑi˧kv̩˧wo˧}}}}{}
\textcolor{teal}{\mytextsc{nom}} \hspace{4pt} Ton~: M.

Toit.
\textcolor{brown}{\zh{房顶。}}


 \mytextsc{clf}~: \textcolor{darkblue}{\textbf{\ipa{tsʰi˩}}} 

\paragraph{\hspace{-0.5cm} {\Large \textcolor{darkblue}{\textbf{\ipa{ʑi˧mi˧}}}}} \hypertarget{z£i\string_Mmi\string_M1}{}
\markboth{\textcolor{darkblue}{\textbf{\ipa{ʑi˧mi˧}}}}{}
\textcolor{teal}{\mytextsc{nom}} \hspace{4pt} Ton~: M.
\ding{202} 
Le bâtiment de la maison où se trouve le foyer.
\textcolor{brown}{\zh{家里有火塘的那个房子(“祖母房”)。}}


\ding{203} 
L'ensemble de la maison; l'ensemble de la ferme.
\textcolor{brown}{\zh{整个家园。}}


 \mytextsc{clf}~: \textcolor{darkblue}{\textbf{\ipa{ɭɯ˧}}} 

\paragraph{\hspace{-0.5cm} {\Large \textcolor{darkblue}{\textbf{\ipa{ʑi˧mv̩˧˥}}}}} \hypertarget{z£i\string_Mmv\string_=\string_M\string_T1}{}
\markboth{\textcolor{darkblue}{\textbf{\ipa{ʑi˧mv̩˧˥}}}}{}
\textcolor{teal}{\mytextsc{nom}} \hspace{4pt} Ton~: MH\#.

Rêve.
\textcolor{brown}{\zh{梦。}}


\begin{exe}
\sn \textcolor{darkblue}{\textbf{\ipa{ʑi˧mv̩˧ qʰwɤ˧˥}}}
\trans faire un rêve
\trans \textit{\textcolor{brown}{\zh{做梦}}}
\end{exe}
\begin{exe}
\sn \textcolor{darkblue}{\textbf{\ipa{ʑi˧mv̩˧ sɯ˧}}}
\trans être somnambule; parler dans son sommeil
\trans \textit{\textcolor{brown}{\zh{梦游,梦呓}}}
\end{exe}
\begin{exe}
\sn \textcolor{darkblue}{\textbf{\ipa{njɤ˧ | ə˧hwɤ˧ | ʑi˧mv̩˥ | mɤ˧-dʑɤ˩!}}}
\trans J'ai fait un cauchemar hier!
\trans \textit{\textcolor{brown}{\zh{我昨天做了恶梦!}}}
\end{exe}
 \mytextsc{clf}~: \textcolor{darkblue}{\textbf{\ipa{kʰwɤ˥}}} 

\paragraph{\hspace{-0.5cm} {\Large \textcolor{darkblue}{\textbf{\ipa{ʑi˧ŋɤ˥}}}}} \hypertarget{z£i\string_MN7\string_T1}{}
\markboth{\textcolor{darkblue}{\textbf{\ipa{ʑi˧ŋɤ˥}}}}{}
\textcolor{teal}{\mytextsc{verbe}} \hspace{4pt} Ton~: H\#.

Somnoler.
\textcolor{brown}{\zh{打瞌睡。}}


\begin{exe}
\sn \textcolor{darkblue}{\textbf{\ipa{ʑi˧ŋɤ˥-ze˩}}}
\trans \mytextsc{pfv}
\trans \textit{\textcolor{brown}{\zh{\mytextsc{pfv}}}}
\end{exe}


\paragraph{\hspace{-0.5cm} {\Large \textcolor{darkblue}{\textbf{\ipa{ʑi˧ŋv̩˥}}}}} \hypertarget{z£i\string_MNv\string_=\string_T1}{}
\markboth{\textcolor{darkblue}{\textbf{\ipa{ʑi˧ŋv̩˥}}}}{}
\textcolor{teal}{\mytextsc{verbe}} \hspace{4pt} Ton~: H\#.

Dormir.
\textcolor{brown}{\zh{睡觉。}}


\begin{exe}
\sn \textcolor{darkblue}{\textbf{\ipa{le˧-ʑi˧ŋv̩˥}}}
\trans \mytextsc{accomp}
\trans \textit{\textcolor{brown}{\zh{\mytextsc{accomp}}}}
\end{exe}
\begin{exe}
\sn \textcolor{darkblue}{\textbf{\ipa{ʑi˧ŋv̩˥-ho˩}}}
\trans qui va s'endormir, qui est ensommeillé
\trans \textit{\textcolor{brown}{\zh{要睡了}}}
\end{exe}


\paragraph{\hspace{-0.5cm} {\Large \textcolor{darkblue}{\textbf{\ipa{ʑi˧qʰwɤ˧}}}}} \hypertarget{z£i\string_Mq\string_hw7\string_M1}{}
\markboth{\textcolor{darkblue}{\textbf{\ipa{ʑi˧qʰwɤ˧}}}}{}
\textcolor{teal}{\mytextsc{nom}} \hspace{4pt} Ton~: M.

Bâtiment, bâtiment d'habitation, pièce d'habitation; en naxi, a aussi le sens de ``maisonnée".
\textcolor{brown}{\zh{房屋。}}


\begin{exe}
\sn \textcolor{darkblue}{\textbf{\ipa{ʑi˧qʰwɤ˧ gv̩˩}}}
\trans bâtir une maison
\trans \textit{\textcolor{brown}{\zh{建房}}}
\end{exe}
\begin{exe}
\sn \textcolor{darkblue}{\textbf{\ipa{ʑi˧qʰwɤ˧-lɑ˧ do˥!}}}
\trans on ne voit que des maisons/des bâtiments! (commentaires au sujet de la ville de Lijiang, où on ne voit pas les champs, à la différence de la plaine de Yongning: campagne où il y avait peu de maisons et de grands espaces cultivés.
\trans \textit{\textcolor{brown}{\zh{只看到房子! / 能看见的只有房子!(合作者说,丽江市区都是房子,看不到田。这一点,不像永宁坝:二十世纪的永宁,只有一些小村落分散在一大片田地中。)}}}
\end{exe}
\begin{exe}
\sn \textcolor{darkblue}{\textbf{\ipa{ɕjo˩ɕjɤ˩-ʑi˩qʰwɤ˥}}}
\trans les bâtiments de l'école (du chinois \textcolor{brown}{\zh{学校}} ``école")
\trans \textit{\textcolor{brown}{\zh{学校的楼(‘学校’:汉语借词)}}}
\end{exe}
\begin{exe}
\sn \textcolor{darkblue}{\textbf{\ipa{ʑi˧qʰwɤ˧ ʈʂʰv̩˩}}}
\trans peindre une maison; littéralement: ``teindre une maison"
\trans \textit{\textcolor{brown}{\zh{给房子刷颜色(直译:‘染房’)}}}
\end{exe}
\begin{exe}
\sn \textcolor{darkblue}{\textbf{\ipa{ʑi˧qʰwɤ˧ tɕʰi˧-hĩ˧ kʰv̩˥mi˩}}}
\trans chien de garde, chien qui garde la maison
\trans \textit{\textcolor{brown}{\zh{看门狗}}}
\end{exe}
\begin{exe}
\sn \textcolor{darkblue}{\textbf{\ipa{ʑi˧qʰwɤ˧ tɕʰi˧-hĩ˧ kʰv̩˥}}}
\trans chien de garde, chien qui garde la maison
\trans \textit{\textcolor{brown}{\zh{看门狗}}}
\end{exe}
 \mytextsc{clf}~: \textcolor{darkblue}{\textbf{\ipa{ɭɯ˧}}} 

\paragraph{\hspace{-0.5cm} {\Large \textcolor{darkblue}{\textbf{\ipa{ʑi˧ʁæ˥\$}}}}} \hypertarget{z£i\string_MR\{\string_T\$1}{}
\markboth{\textcolor{darkblue}{\textbf{\ipa{ʑi˧ʁæ˥\$}}}}{}
\textcolor{teal}{\mytextsc{nom}} \hspace{4pt} Ton~: H\$.

L'espace situé derrière la maison: entre le bâtiment et les murs de la ferme.
\textcolor{brown}{\zh{房屋的上后方。}}


\begin{exe}
\sn \textcolor{darkblue}{\textbf{\ipa{ɲi˧ʈʂæ˧-ʑi˧-ʁo˧tʰo˥, | ʑi˧ʁæ˧ ɲi˥ mæ˩!}}}
\trans Derrière le bâtiment à deux étages, c'est 'ʑi˧ʁæ˥\$'! / Derrière le bâtiment à deux étages, il y a ce qu'on appelle 'l'espace derrière la maison'!
\trans \textit{\textcolor{brown}{\zh{两层楼房后面(这块地方)叫做“房屋的上后方”! / 房屋背后(这块地方)叫做“房屋的上后方”!}}}
\end{exe}
 \mytextsc{clf}~: \textcolor{darkblue}{\textbf{\ipa{kʰwɤ˥}}} 

\paragraph{\hspace{-0.5cm} {\Large \textcolor{darkblue}{\textbf{\ipa{ʑi˧ʁo˥\$}}}}} \hypertarget{z£i\string_MRo\string_T\$1}{}
\markboth{\textcolor{darkblue}{\textbf{\ipa{ʑi˧ʁo˥\$}}}}{}
\textcolor{teal}{\mytextsc{nom}} \hspace{4pt} Ton~: H\$.

Lit (le couchage entier).
\textcolor{brown}{\zh{床。}}


 \mytextsc{clf}~: \textcolor{darkblue}{\textbf{\ipa{ɭɯ˧˥}}} 

\paragraph{\hspace{-0.5cm} {\Large \textcolor{darkblue}{\textbf{\ipa{ʑi˩}}} \textsubscript{2}}} \hypertarget{z£i\string_B2}{}
\markboth{\textcolor{darkblue}{\textbf{\ipa{ʑi˩}}} \textsubscript{2}}{}
\textcolor{teal}{\mytextsc{classificateur}} \hspace{4pt} Ton~: L\textsubscript{2}.

Famille.
\textcolor{brown}{\zh{家庭(一户人)。}}


\begin{exe}
\sn \textcolor{darkblue}{\textbf{\ipa{hĩ˧ | ɖɯ˧-ʑi˩}}}
\trans une famille
\trans \textit{\textcolor{brown}{\zh{一家人}}}
\end{exe}
\begin{exe}
\sn \textcolor{darkblue}{\textbf{\ipa{ʈʂʰɯ˧-ʑi˥}}}
\trans cette famille-ci
\trans \textit{\textcolor{brown}{\zh{这家}}}
\end{exe}
\begin{exe}
\sn \textcolor{darkblue}{\textbf{\ipa{ŋwæ˧-qʰv̩˧, | tsʰe˧ɲi˧-ʑi˩}}}
\trans Cinq hameaux, douze familles! (Formule résumant la statistique du village de \textcolor{darkblue}{\textbf{\ipa{/ə˧lɑ˧-ʁwɤ\#˥/}}}.)
\trans \textit{\textcolor{brown}{\zh{五个村落,十二个家庭!(阿拉瓦村的人口简介)}}}
\end{exe}


\paragraph{\hspace{-0.5cm} {\Large \textcolor{darkblue}{\textbf{\ipa{ʑi˩\textsubscript{b}}}}}} \hypertarget{z£i\string_Bb1}{}
\markboth{\textcolor{darkblue}{\textbf{\ipa{ʑi˩\textsubscript{b}}}}}{}
\textcolor{teal}{\mytextsc{verbe}} \hspace{4pt} Ton~: L\textsubscript{b}.
\ding{202} 
Attraper, saisir, prendre (ex.: un animal récalcitrant).
\textcolor{brown}{\zh{拿,捉 (捉鸡)。}}


\begin{exe}
\sn \textcolor{darkblue}{\textbf{\ipa{æ˩ ʑi˧}}}
\trans attraper (un/le) poulet
\trans \textit{\textcolor{brown}{\zh{捉鸡}}}
\end{exe}
\begin{exe}
\sn \textcolor{darkblue}{\textbf{\ipa{æ˩˥ | le˧-ʑi˩}}}
\trans attraper (un/le) poulet
\trans \textit{\textcolor{brown}{\zh{捉鸡}}}
\end{exe}
\begin{exe}
\sn \textcolor{darkblue}{\textbf{\ipa{hĩ˧ ʑi˧˥}}}
\trans attraper quelqu'un
\trans \textit{\textcolor{brown}{\zh{抓人}}}
\end{exe}
\begin{exe}
\sn \textcolor{darkblue}{\textbf{\ipa{ʁæ˧ ʑi˧}}}
\trans passer le bras autour du cou de quelqu'un
\trans \textit{\textcolor{brown}{\zh{搂(用胳膊搂脖子)}}}
\end{exe}
\ding{203} 
Amener, prendre avec soi.
\textcolor{brown}{\zh{带、拿过来。}}


\begin{exe}
\sn \textcolor{darkblue}{\textbf{\ipa{tso˧\textasciitilde{}tso˧ ʑi˧˥}}}
\trans attraper quelque chose
\trans \textit{\textcolor{brown}{\zh{带东西过来}}}
\end{exe}


\paragraph{\hspace{-0.5cm} {\Large \textcolor{darkblue}{\textbf{\ipa{ʑi˩hṽ\#˥}}}}} \hypertarget{z£i\string_Bhv\string_~\#\string_T1}{}
\markboth{\textcolor{darkblue}{\textbf{\ipa{ʑi˩hṽ\#˥}}}}{}
\textcolor{teal}{\mytextsc{nom}} \hspace{4pt} Ton~: LM+\#H.

Poils corporels.
\textcolor{brown}{\zh{人身上的毛。}}


 \mytextsc{clf}~: \textcolor{darkblue}{\textbf{\ipa{kʰɯ˩}}} 

\paragraph{\hspace{-0.5cm} {\Large \textcolor{darkblue}{\textbf{\ipa{ʑi˩-kʰv̩˧˥}}}}} \hypertarget{z£i\string_B-k\string_hv\string_=\string_M\string_T1}{}
\markboth{\textcolor{darkblue}{\textbf{\ipa{ʑi˩-kʰv̩˧˥}}}}{}
\textcolor{teal}{\mytextsc{nom}} \hspace{4pt} Ton~: LM+MH\#.

Année du singe.
\textcolor{brown}{\zh{猴年。}}




\paragraph{\hspace{-0.5cm} {\Large \textcolor{darkblue}{\textbf{\ipa{ʑi˩mi\#˥}}}}} \hypertarget{z£i\string_Bmi\#\string_T1}{}
\markboth{\textcolor{darkblue}{\textbf{\ipa{ʑi˩mi\#˥}}}}{}
\textcolor{teal}{\mytextsc{nom}} \hspace{4pt} Ton~: LM+\#H.

Singe femelle.
\textcolor{brown}{\zh{母猴。}}


 \mytextsc{clf}~: \textcolor{darkblue}{\textbf{\ipa{mi˩}}} 

\paragraph{\hspace{-0.5cm} {\Large \textcolor{darkblue}{\textbf{\ipa{ʑi˩pʰv̩\#˥}}}}} \hypertarget{z£i\string_Bp\string_hv\string_=\#\string_T1}{}
\markboth{\textcolor{darkblue}{\textbf{\ipa{ʑi˩pʰv̩\#˥}}}}{}
\textcolor{teal}{\mytextsc{nom}} \hspace{4pt} Ton~: LM+\#H.

Singe mâle.
\textcolor{brown}{\zh{公猴。}}


 \mytextsc{clf}~: \textcolor{darkblue}{\textbf{\ipa{mi˩}}} 

\paragraph{\hspace{-0.5cm} {\Large \textcolor{darkblue}{\textbf{\ipa{ʑi˩zo\#˥}}}}} \hypertarget{z£i\string_Bzo\#\string_T1}{}
\markboth{\textcolor{darkblue}{\textbf{\ipa{ʑi˩zo\#˥}}}}{}
\textcolor{teal}{\mytextsc{nom}} \hspace{4pt} Ton~: LM+\#H.

Petit singe.
\textcolor{brown}{\zh{小猴子。}}


 \mytextsc{clf}~: \textcolor{darkblue}{\textbf{\ipa{ɭɯ˧}}} 

\paragraph{\hspace{-0.5cm} {\Large \textcolor{darkblue}{\textbf{\ipa{ʑi˧˥}}} \textsubscript{1}}} \hypertarget{z£i\string_M\string_T1}{}
\markboth{\textcolor{darkblue}{\textbf{\ipa{ʑi˧˥}}} \textsubscript{1}}{}
\textcolor{teal}{\mytextsc{verbe}} \hspace{4pt} Ton~: MH.

Dormir.
\textcolor{brown}{\zh{睡觉。}}


\begin{exe}
\sn \textcolor{darkblue}{\textbf{\ipa{le˧-ʑi˧˥}}}
\trans \mytextsc{accomp}
\trans \textit{\textcolor{brown}{\zh{\mytextsc{accomp}}}}
\end{exe}
\begin{exe}
\sn \textcolor{darkblue}{\textbf{\ipa{le˧-ʑi˧-ze˥}}}
\trans \mytextsc{accomp}  \mytextsc{pfv}
\trans \textit{\textcolor{brown}{\zh{\mytextsc{accomp}  \mytextsc{pfv}}}}
\end{exe}
\begin{exe}
\sn \textcolor{darkblue}{\textbf{\ipa{æ˩ ʑi˧-ze˥}}}
\trans la poule s'est endormie
\trans \textit{\textcolor{brown}{\zh{鸡睡觉了}}}
\end{exe}
\begin{exe}
\sn \textcolor{darkblue}{\textbf{\ipa{le˧-ʑi˧-bi˧-ze˩!}}}
\trans ( je) vais dormir!/(je) vais me coucher!
\trans \textit{\textcolor{brown}{\zh{要睡觉了!}}}
\end{exe}
\begin{exe}
\sn \textcolor{darkblue}{\textbf{\ipa{le˧-ʑi˧˥, | ʑi˧-mɤ˥-tʰɑ˩!}}}
\trans j'essaierais de dormir, que je n'y arriverais pas! / Dormir, il ne faut pas y penser/je n'y arriverais pas! (contexte: une personne âgée se plaint de maux de tête; on lui suggère d'aller se reposer/faire une sieste)
\trans \textit{\textcolor{brown}{\zh{想睡,但睡不了!}}}
\end{exe}
\begin{exe}
\sn \textcolor{darkblue}{\textbf{\ipa{pʰæ˧tɕi˥-zo˩-ɳɯ˩ | mv̩˩zo˩-qɑ˥ ʑi˩}}}
\trans Le jeune homme couche avec la jeune femme. (Formulation considérée comme vulgaire; ce thème est tabou.)
\trans \textit{\textcolor{brown}{\zh{小伙子跟年轻女人睡!(庸俗说法)}}}
\end{exe}


\paragraph{\hspace{-0.5cm} {\Large \textcolor{darkblue}{\textbf{\ipa{ʑi˩˥}}}}} \hypertarget{z£i\string_B\string_T1}{}
\markboth{\textcolor{darkblue}{\textbf{\ipa{ʑi˩˥}}}}{}
\textcolor{teal}{\mytextsc{nom}} \hspace{4pt} Ton~: LH.

Singe.
\textcolor{brown}{\zh{猴子。}}


\begin{exe}
\sn \textcolor{darkblue}{\textbf{\ipa{ʑi˩ dzɯ˧-ze˩}}}
\trans (sujet non spécifié: un tigre...) a mangé (le) singe
\trans \textit{\textcolor{brown}{\zh{吃了猴子}}}
\end{exe}
\begin{exe}
\sn \textcolor{darkblue}{\textbf{\ipa{ʑi˩ hwæ˧-ze˩}}}
\trans (sujet non spécifié: quelqu'un) a acheté (le) singe
\trans \textit{\textcolor{brown}{\zh{买了猴子}}}
\end{exe}
 \mytextsc{clf}~: \textcolor{darkblue}{\textbf{\ipa{mi˩}}} 

\paragraph{\hspace{-0.5cm} {\Large \textcolor{darkblue}{\textbf{\ipa{*ʑi˩˧}}}}} \hypertarget{*z£i\string_B\string_M1}{}
\markboth{\textcolor{darkblue}{\textbf{\ipa{*ʑi˩˧}}}}{}
\textcolor{teal}{\mytextsc{nom}} \hspace{4pt} Ton~: LM? LH?.

Maison, bâtiment d'habitation; monosyllabe extrait de l'expression /ʑi˩ tsʰi˩˥, | æ̃˩ tsʰi˧/ 'bâtir une demeure': le schéma tonal, avec un verbe au ton MH, peut provenir d'un nom au ton LM ou LH.
\textcolor{brown}{\zh{房屋。}}


\begin{exe}
\sn \textcolor{darkblue}{\textbf{\ipa{ʑi˩ tsʰi˧˥, | æ̃˩ tsʰi˥}}}
\trans Bâtir une demeure, bâtir un foyer (expression proverbiale)
\trans \textit{\textcolor{brown}{\zh{建房立家(固定词语)}}}
\end{exe}
\begin{exe}
\sn \textcolor{darkblue}{\textbf{\ipa{ʑi˩ tʰv̩˩˥}}}
\trans Fonder une nouvelle demeure: lorsque la famille est nombreuse, un de ses membres peut bâtir sa propre demeure
\trans \textit{\textcolor{brown}{\zh{分家,建立自己的新房屋比如:孩子多,一个孩子建自己的房子)}}}
\end{exe}
\begin{exe}
\sn \textcolor{darkblue}{\textbf{\ipa{ʑi˩ qʰæ˧˥}}}
\trans démolir une demeure
\trans \textit{\textcolor{brown}{\zh{拆房子(这个例子是调查者构造的,F4确定是可以说的。造这个例子的目的有两个:看单音节词根“家”能不能跟其它动词结合,也试着确定它的调类。)}}}
\end{exe}
\begin{exe}
\sn \textcolor{darkblue}{\textbf{\ipa{*ʑi˩ hwæ˧}}}
\trans *acheter une demeure; forme non correcte, proposée pour voir dans quelle mesure le monosyllabe peut se combiner avec divers verbes.
\trans \textit{\textcolor{brown}{\zh{*买房(这个例子是调查者构造的,F4确定是不可以说的。造这个例子的目的有两个:看单音节词根“家”能不能跟其它动词结合,也试着确定它的调类。)}}}
\end{exe}























\end{multicols}

\newpage
\section*{\centering Mots dont aucun équivalent n'a été trouvé}

Cette liste regroupe les mots dont aucun équivalent n'a été trouvé. Même s'il ne s'agit que d'informations négatives, elles éclairent les limites du vocabulaire na des consultants.

\begin{center}

\begin{longtable}{r|l}

blaireau & \textcolor{brown}{\zh{猪獾}} \\

lynx & \textcolor{brown}{\zh{猞猁}} \\

pangolin & \textcolor{brown}{\zh{穿山甲}} \\

lobe\string_de\string_l'oreille & \textcolor{brown}{\zh{耳垂}} \\

tempes & \textcolor{brown}{\zh{太阳穴}} \\

fromage\string_séché & \textcolor{brown}{\zh{乳扇}} \\

brosse\string_à\string_dent\string_en\string_poil\string_de\string_cochon & \textcolor{brown}{\zh{猪鬃毛牙刷}} \\

catapulte & \textcolor{brown}{\zh{抛石机}} \\

lance\string_pierre & \textcolor{brown}{\zh{绷弓子}} \\

Angelica & \textcolor{brown}{\zh{当归}} \\

Anisodus\string_tanguticus & \textcolor{brown}{\zh{山茛菪}} \\

Benincasa\string_hispida & \textcolor{brown}{\zh{冬瓜}} \\

céleri & \textcolor{brown}{\zh{芹菜}} \\

Ephedra\string_sinica & \textcolor{brown}{\zh{草麻黄}} \\

if & \textcolor{brown}{\zh{红豆杉}} \\

Ligularia\string_fischeria & \textcolor{brown}{\zh{山紫菀}} \\

Selaginella & \textcolor{brown}{\zh{卷柏}} \\

yéti & \textcolor{brown}{\zh{野人}} \\

jumeaux & \textcolor{brown}{\zh{双胞胎}} \\

boire\string_à\string_la\string_paille & \textcolor{brown}{\zh{用吸管喝}} \\

regretter & \textcolor{brown}{\zh{后悔}} \\

\end{longtable}\end{center}
\end{document}
