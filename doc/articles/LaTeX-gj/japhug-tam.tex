\documentclass[oldfontcommands,oneside,a4paper,11pt]{article} 
\usepackage{fontspec}
\usepackage{natbib}
\usepackage{booktabs}
\usepackage{xltxtra} 
\usepackage{longtable}
\usepackage{polyglossia} 
\usepackage[table]{xcolor}
\usepackage{gb4e} 
\usepackage{multicol}
\usepackage{graphicx}
\usepackage{float}
\usepackage{hyperref} 
\usepackage{lineno}
\hypersetup{bookmarks=false,bookmarksnumbered,bookmarksopenlevel=5,bookmarksdepth=5,xetex,colorlinks=true,linkcolor=blue,citecolor=blue}
\usepackage[all]{hypcap}
\usepackage{memhfixc}
\usepackage{lscape}

\bibpunct[: ]{(}{)}{,}{a}{}{,}

%\setmainfont[Mapping=tex-text,Numbers=OldStyle,Ligatures=Common]{Charis SIL} 
\newfontfamily\phon[Mapping=tex-text,Ligatures=Common,Scale=MatchLowercase,FakeSlant=0.3]{Charis SIL} 
\newcommand{\ipa}[1]{{\phon \mbox{#1}}} %API tjs en italique
\newcommand{\ipab}[1]{{\scriptsize \phon#1}} 

\newcommand{\grise}[1]{\cellcolor{lightgray}\textbf{#1}}
\newfontfamily\cn[Mapping=tex-text,Ligatures=Common,Scale=MatchUppercase]{MingLiU}%pour le chinois
\newcommand{\zh}[1]{{\cn #1}}
\newcommand{\refb}[1]{(\ref{#1})}


\XeTeXlinebreaklocale 'zh' %使用中文换行
\XeTeXlinebreakskip = 0pt plus 1pt %
 %CIRCG
 


\begin{document} 
\title{TAM in Japhug\footnote{
} }
\author{Guillaume Jacques}
\maketitle
\linenumbers

\section{Introduction}


 \section{Morphology}
\subsection{Directional prefixes} \label{sec:directional}
All finite verbal forms in Japhug (except the factual) and some nominalized forms have a directional prefix that contains information on TAM, transitivity and  (in the case of motion and concrete action verbs) the direction of the action.

With the exception of contracting verbs whose stem starts in \ipa{a--} and which present special alternations, Japhug intransitive verbs have three series of prefixes (A, B and D) and transitive ones four series, as shown in Table \refb{tab:directional}. The distribution of these four series will be explained in more detail in section \refb{sec:finite.TAM}.

\begin{table}[H]
\caption{Directional prefixes in Japhug Rgyalrong} \label{tab:directional}
\resizebox{\columnwidth}{!}{
\begin{tabular}{llllll}
\toprule
   &  	perfective  (A) &  	imperfective  (B)  &  	perfective 3$\rightarrow$3' (C)  &  	evidential  (D) \\  	
   \midrule
up   &  	\ipa{tɤ--}   &  	\ipa{tu--}   &  	\ipa{ta--}   &  	\ipa{to--}   \\  	
down   &  	\ipa{pɯ--}   &  	\ipa{pjɯ--}   &  	\ipa{pa--}   &  	\ipa{pjɤ--}   \\  	
upstream   &  	\ipa{lɤ--}   &  	\ipa{lu--}   &  	\ipa{la--}   &  	\ipa{lo--}   \\  	
downstream   &  	\ipa{tʰɯ--}   &  	\ipa{cʰɯ--}   &  	\ipa{tʰa--}   &  	\ipa{cʰɤ--}   \\  	
east   &  	\ipa{kɤ--}   &  	\ipa{ku--}   &  	\ipa{ka--}   &  	\ipa{ko--}   \\  	
west   &  	\ipa{nɯ--}   &  	\ipa{ɲɯ--}   &  	\ipa{na--}   &  	\ipa{ɲɤ--}   \\  	
no direction &\ipa{jɤ--}   &  	\ipa{ju--}   &  	\ipa{ja--}   &  	\ipa{jo--}   \\  	
\bottomrule
\end{tabular}}
\end{table}

Most verbs have one intrinsic direction which is lexically determined. For instance, the verb \ipa{sat} `kill' selects the direction `down' for all its forms: \textbf{perfective} \textsc{1sg$\rightarrow$3sg} \ipa{pɯ-sat-a}, \textbf{imperfective} \ipa{pjɯ-sat}, \textbf{perfective} \textsc{3sg$\rightarrow$3'} \ipa{pa-sat} and \textbf{evidential} \ipa{pjɤ-sat}. 

Some verbs may allow several directions with slightly different semantics. Thus, \ipa{ndza} `eat'   normally   selects the  `up' direction (\textbf{perfective} \textsc{3sg$\rightarrow$3'} \ipa{ta-ndza} `he ate it'), but when applied to carnivorous animals we also find the `downstream' direction. This can lead to further aspectual distinctions. For instance, the direction `downstream', when used with stative verbs, indicates a progressive development. 

Verbs of motion and some verbs of concrete action can be associated with all seven series of prefixes to indicate the direction of the motion. The  `no direction' series of prefixes only occurs with motion verbs. 

Only three verbs have defective paradigms and never occur with directional prefixes: the sensory existential copulas \ipa{ɣɤʑu} `exist' and \ipa{maŋe} `not exist' and the verb \ipa{kɤtɯpa} `speak' (see the paradigm of the latter in \citealt[1215]{jacques12incorp}).

\subsection{Contracting verbs}

\citet{jacques07passif}
The evidential and evidential imperfective forms are used with the circumfix \ipa{k--}...\ipa{--ci} in the case of verb forms whose stem begins in \ipa{a--} (including verbs with the progressive \ipa{asɯ--}).

\subsection{Stem alternation} \label{sec:stem}
The existence of stem alternations in Rgyalrong was first reported by \citet{jackson00puxi}, who proposes to distinguish three stems: the base stem (stem 1), the perfective stem (stem 2) and the non stem (stem 3). Some varieties of Zbu Rgyalrong appear to have an additional progressive stem distinct from stem 2 in the progressive form (\citealt[352]{jacques04these}).


In Kamnyu Japhug, only four verbs have a   perfective stem distinct from the base stem; the list is provided in Table \refb{tab:stem2}. 


 \begin{table} 
\caption{Stem 2 alternation in Japhug Rgyalrong} \label{tab:stem2} \centering
\begin{tabular}{llllll}
\toprule
Stem 1 & meaning &Stem 2 \\
\midrule
\ipa{ɕe}& to go (vi)&  \ipa{ari} \\
\ipa{sɯxɕe}& to sent (vt)  &\ipa{sɤɣri} \\
\ipa{ɣi}& to come (vi)  &\ipa{ɣe} \\
\ipa{ti}& to say (vt)  &\ipa{tɯt} \\
\bottomrule
\end{tabular}
\end{table}


Stem 3 on the other hand is fully productive. The rules of vowel alternation in Table \refb{tab:stem3} apply to all finite transitive verbs in the forms \textsc{1sg}$\rightarrow$3, \textsc{2sg}$\rightarrow$3 and \textsc{3sg}$\rightarrow$3'; stem 3 does not appear in verb forms with the inverse marker (see \citealt{gongxun14agreement}). \citet[351-7]{jacques04these} provides a historical analysis of these alternations, and shows that they result from the fusion of the verb stem with two suffixes. %Some Japhug dialects present \ipa{--u} $\rightarrow$ \ipa{--ɯm} and  \ipa{--i} $\rightarrow$ \ipa{--ɯm} alternations.  

 \begin{table} 
\caption{Stem 3 alternation in Japhug Rgyalrong} \label{tab:stem3} \centering
\begin{tabular}{llllll}
\toprule
Stem 1 & Stem 3 \\
\midrule
\ipa{--a} & \ipa{--e} \\
\ipa{--u} & \ipa{--e} \\
\ipa{--ɯ} & \ipa{--i} \\
\ipa{--o} & \ipa{--ɤm} \\
\bottomrule
\end{tabular}
\end{table}

Following the Leipzig glossing rules, we indicate stem 2 as [II] and stem 3 as [III] in the glosses in this paper.

\section{Simple TAM categories} \label{sec:finite.TAM}

There are nine basic finite TAM categories in Japhug, as represented in Table \refb{tab:finite.forms}. All finite forms except the factual require one and only one directional prefix. All forms can be correctly produced by combining the appropriate derivational prefixes and stems.\footnote{For the TAM categories requiring stem 3, it is restricted to  \textsc{1sg}$\rightarrow$3, \textsc{2sg}$\rightarrow$3 and \textsc{3sg}$\rightarrow$3' forms; all other forms take the base stem.  The person affixes  are not discussed here; for  more information on this topic, see \citet{jacques10inverse}.}

 terminology: \citet{tournadre14evidentiality},  \citet{tournadre08conjunct}, 
 
 
 In questions,  the speaker can anticipate the answer of the addressee
and use the form that he expect the addressee will use in responding to the question. (\citealt[244]{tournadre14evidentiality}).
 
 

\begin{table}
\caption{Finite verb categories in Japhug Rgyalrong} \label{tab:finite.forms} \centering
\begin{tabular}{lllllll}
\toprule
&	&	stem&	prefixes\\
\midrule
factual&	\textsc{fact} &	1 or 3&	no prefix\\
imperfective&	\textsc{ipfv} &	1 or 3&	B\\
perfective&	\textsc{pfv} &	2&	A or C\\
past imperfective&	\textsc{pst.ipfv} &	2&	\ipa{pɯ--}\\
inferential&	\textsc{evd} &	1&	D\\
inferential imperfective&	\textsc{evd.ipfv} &	1&	\ipa{pjɤ--}\\
sensory&	\textsc{sensory} &	1 or 3&	\ipa{ɲɯ--}\\
egophoric&	\textsc{egoph} &	1 or 3&	\ipa{ku--}\\
irrealis&	\textsc{irr} &	1 or 3&	\ipa{a--} + A\\
imperative&	\textsc{imp} &	1 or 3&	A\\
\bottomrule
\end{tabular}
\end{table}

 

\subsection{Factual}

The factual is the only TAM category in Japhug without a directional prefix. In the case of verbs without stem III alternation (i.e. transitive verbs with anon-alternating rhyme and intransitive verbs), it is realized as the bare stem. In spite of the absence of overt marking for most verbs, we still indicate `factual' in all glosses of finite verb forms. The gloss is indicated as a suffix (verb:\textsc{fact}), rather than as a prefix, to avoid confusion in the case of verb forms with several derivation prefixes.

The factual has three main functions when used in an independent clause without an auxiliary verb (in the later case see section \ref{sec:periphrastic.fact}). 

First,  in the case of stative verbs, whether adjectival stative verbs or existential verbs/copulas, the factual is used to describe facts considered to belong to everybody's common knowledge. Example \ref{ex:kumpGAtCW} illustrate five examples of the use of the factual in this way, including copulas and adjectives. 

\begin{exe}
\ex \label{ex:kumpGAtCW}
\gll
\ipa{tɕe} 	\ipa{kumpɣɤtɕɯ} 	\ipa{nɯnɯ} 	\ipa{pɣɤtɕɯ} 	\ipa{nɯ-rca,} 
 \ipa{kɯ-xtɕi} 	\ipa{ci} 	\ipa{zdoʁzdoʁ} 	\ipa{\textbf{ŋu}} 	\ipa{tɕe,}  \ipa{ɯʑo} 	\ipa{\textbf{xtɕi}} 	\ipa{ri} 	\ipa{wuma} 	\ipa{ʑo} 	\ipa{\textbf{ɕqraʁ}} \ipa{tɕe}  	\ipa{ɯ-mɲaʁ} 	\ipa{ɯ-rkɯ} 	\ipa{nɯnɯ} 	\ipa{ra} \ipa{kɯ-ɲaʁ} 	\ipa{kɯ} 	\ipa{tú-wɣ-fskɤr,} 	 	\ipa{nɯ} 	\ipa{ɯ-taʁ} 	\ipa{ri,} 	\ipa{χanɯni,} 	\ipa{kɯ-xtɕɯ\textasciitilde{}xtɕi} 	\ipa{kɯ-ɣɯrni} 	\ipa{kɯ-fse} 	\ipa{\textbf{tu},}  	\ipa{ɯ-xtɤpa} 	\ipa{nɯ} \ipa{ra,} 	\ipa{ɯ-rqopa} 	\ipa{pjɯ-ʑe} 	\ipa{tɕe,} 	\ipa{nɯ} \ipa{ra,}  \ipa{ɯ-jme} 	\ipa{mɯ-thɯ-nɯɬoʁ} 	\ipa{mɤɕtʂa} 	\ipa{nɯ} 	\ipa{\textbf{wɣrum}} \\
\textsc{lnk} sparrow \textsc{dem} bird \textsc{3pl}-among \textsc{nmlz}:S/A-be.small \textsc{indef} \textsc{ideo:stat}:small.and.cute \textbf{be}:\textsc{fact} \textsc{lnk}  \textsc{3sg} \textbf{be.small}:\textsc{fact} but really \textsc{emph} \textbf{be.smart}:\textsc{fact}  \textsc{lnk}  \textsc{3sg:poss}-eye \textsc{3sg:poss}-border \textsc{dem}  \textsc{pl}  \textsc{nmlz}:S/A-be.black \textsc{erg} \textsc{ipfv-inv}-surround \textsc{dem} \textsc{3sg}-on \textsc{loc} a.little \textsc{nmlz}:S/A-\textsc{emph}\textasciitilde{}be.small \textsc{nmlz}:S/A-be.red \textsc{nmlz}:S/A-be.like \textbf{exist}:\textsc{fact} \textsc{3sg:poss}-belly \textsc{dem}  \textsc{pl}  \textsc{3sg:poss}-throat \textsc{ipfv}-begin[III] \textsc{lnk}   \textsc{dem}  \textsc{pl}  \textsc{3sg:poss}-tail \textsc{neg-pfv-auto}-come.out until \textsc{dem} \textbf{be.white:}\textsc{fact} \\
\glt Among the birds, the sparrow is tiny and cute. Although it is small it is very intelligent. Its eyes are surrounded by black (feathers), and above that there are some red (dots). Its belly is white from the throat until the tail. (Sparrow, 2-7)
\end{exe}

 Dynamic verbs   are in the imperfective rather than the factual in such contexts with a generic (such as \ipa{tú-wɣ-fskɤr}  \textsc{ipfv-inv}-surround  it surrounds it' above)  or an expletive (\ipa{pjɯ-ʑe} \textsc{ipfv}-begin[III] `it begins' in the example) subject S/A. 
 
 The factual is however used with dynamic verbs in general knowledge sentences with overt or definite subjects, as in \ref{ex:mAndze}. It is also possible to use the imperfective with an auxiliary such as \ipa{ŋu} `be' or  \ipa{ŋgrɤl} `be usually the case" to express the same meaning (see section XXX).

 

\begin{exe}
\ex \label{ex:mAndze}
\gll
   	\ipa{ɯ-ku}  	\ipa{kɯ-mpɯ}  	\ipa{nɯ}  	\ipa{ɲɯ́-wɣ-pʰɯt}  	\ipa{tɕe,}  \ipa{nɯŋa}  	\ipa{ra}  	\ipa{kɯ}  	\ipa{ndza-nɯ,}  	\ipa{paʁ}  	\ipa{kɯ}  	\ipa{mɤ-ndze}  
(Vine, 20) \\
\textsc{3sg.poss}-head \textsc{nmlz}:S/A-be.soft \textsc{dem} \textsc{ipfv-inv}-pluck \textsc{lnk} cow \textsc{pl} \textsc{erg} eat:\textsc{fact-pl} pig \textsc{erg} \textsc{neg}-eat:\textsc{fact} \\
\glt  One plucks the (leaves) on the extremities, the soft ones, the cows eat it, the pigs don't. (vine, 20)
\end{exe}

 


With dynamic verbs, the factual can express an assertion or an intention in the first person in assertive sentences (example \ref{ex:kWsAthu.Cea}) or in the second in interrogatives (\ref{ex:tWGi}).

\begin{exe}
\ex \label{ex:kWsAthu.Cea}
\gll
\ipa{aʑo}  	\ipa{kɯ-sɤ-tʰu}  	\ipa{ɕe-a}  \\
\textsc{1sg} \textsc{nmlz}:S/A-\textsc{antipass:human}-ask go:\textsc{fact}-\textsc{1sg} \\
\glt I am going to ask (a girl in marriage). (Kunbzang, 5)
\end{exe}

\begin{exe}
\ex \label{ex:tWGi}
\gll
\ipa{mbarkhom} \ipa{tʰɤjtɕu} \ipa{tɯ-ɣi}?\\
Mbarkham when 2-come:\textsc{fact} \\
\glt When are you coming to Mbarkham? (conversation)
\end{exe}

It is also used for future events, when the speaker has reasonable reasons for assuming that they will take place as in \ref{ex:GWsata}. Verbs in the factual in this usage are often combined with auxiliaries such as the affirmative copula \ipa{ɕti} `be' (\ref{ex:Gi.Cti}) or sentence final particles such as \ipa{tʰaŋ} `maybe, probably' (\ref{ex:Gi.thaN}).

\begin{exe}
\ex \label{ex:GWsata}
\gll
\ipa{si-a}   \ipa{ɲɤ-sɯso,} \ipa{tʰa}  	\ipa{ɣɯ-sat-a}  \ipa{ɲɤ-sɯso} \\
die:\textsc{fact-1sg} \textsc{ifr}-think in.a.moment inv-kill:\textsc{fact-1sg} \textsc{ifr}-think \\
\glt He thought ``I will die", he thought `` It will kill me".
\end{exe}

\begin{exe}
\ex \label{ex:Gi.Cti}
\gll
 	\ipa{tʰa} 	\ipa{ɣi} 	\ipa{ɕti} 	\ipa{tɕe,} 	\ipa{sɲikuku} 	\ipa{ʑo} 	\ipa{ju-ɣi} 	\ipa{tɕe} 	\ipa{nɯ} 	\ipa{ntsɯ} 	\ipa{tu-ti} 	\ipa{ɲɯ-ɕti} 	\\
 in.a.moment come:\textsc{fact} be:\textsc{affirm:fact} \textsc{lnk} every.day \textsc{emph} \textsc{ipfv}-come \textsc{lnk} \textsc{dem} always \textsc{ipfv}-say \textsc{sens}-be:\textsc{affirm} \\
\glt It will come  soon, it comes everyday and each times says this. (Kubzang 185-186)
\end{exe}

\begin{exe}
\ex \label{ex:Gi.thaN}
\gll
 \ipa{a-mu} 	\ipa{tɕi-scawa} 	\ipa{ma} 	\ipa{jɯɣmɯr} 	\ipa{tɕe} 	\ipa{iɕqha} 	\ipa{kʰu} 	\ipa{ɣi} 	\ipa{tʰaŋ} 	\ipa{nɤ} 	\\
 \textsc{1sg.poss}-mother  \textsc{1du.poss}-poor.of because today.evening \textsc{lnk} the.aforementioned tiger \textsc{ipfv}-come probably \textsc{lnk} \\
\glt Mother, poor of us, today the tiger is probably coming (for us). (The tiger, 12)
 \end{exe}
 
 
 Unexplained cases:
 
The verb \ipa{sɯz}  `know' 
  
 \begin{exe}
\ex \label{ex:sWza}
\gll
 \ipa{ca}  	\ipa{ndɤre}  	\ipa{sɯz-a}  \\
 river.deer on.the.other.hand know:\textsc{fact-1sg} \\
\glt  River deers, on the other hand, I know about them. (River deer 24)
 \end{exe}
 
 
 Copulas / existential verbs.
 
 Despite  \ref{ex:kume}, generally
 
  \begin{exe}
\ex \label{ex:aRa.me}
\gll
 	\ipa{a-ʁa}  	\ipa{me}  \\
\textsc{1sg.poss}-free.time  not.exist :\textsc{fact} \\
\glt  I don't have time.
 \end{exe}
 
   \begin{exe}
\ex \label{ex:ata}
\gll
 \ipa{a-ɣe}  	\ipa{ɣɯ}  	\ipa{ɯ-pɤro}  	\ipa{ci}  	 qiche 	\ipa{kɯ-xtɕi}  	\ipa{ci}  	\ipa{to-χtɯ.}  	\ipa{tɕe}  	\ipa{andi}  	\ipa{ra}  	\ipa{a-ta.}  	\\
 \textsc{1sg.poss}-grandson \textsc{gen} \textsc{3sg.poss}-present \textsc{indef} car \textsc{nmlz}:S/A-be.small \textsc{indef} \textsc{ifr}-buy \textsc{lnk} west \textsc{pl} \textsc{pass}-put:\textsc{fact} \\
 \glt He bought a present, a small car for my grandson, it is there (at home). (Gongxun, 2-3)
  \end{exe}


\ipa{maʁ}  	\ipa{nɤ,}  	\ipa{aʑo}  	\ipa{tɤ-mɲo-t-a,}  	\ipa{kɯki}  	\ipa{kɯ}  	\ipa{tɕi}  	\ipa{mɤ-βze}  	\ipa{rca!}  


\ipa{ta-saʁjɤr-ndʑi} 	\ipa{tɕe} 	\ipa{ɕe-tɕi} 	\ipa{ɣe?} 

 
 \subsection{Sensory } \label{sec:sens}
The sensory evidential is built by combining stem III or stem I with the prefix \ipa{ɲɯ--} `towards east'.  For verbs whose intrinsic direction is `west', it is identical to the imperfective in the case of verbs. For instance \ipa{ɲɯ-ɤ<nɯ>ɣro} can either be analysed as \textsc{sens-<auto>}play or \textsc{ipfv-<auto>}play (both translatable as `he is playing').


The existential copulas \ipa{tu} `exist' and \ipa{me} `not exist' have suppletive sensory forms \ipa{ɣɤʑu} `exist'  and \ipa{maŋe} `not exist', that are not compatible with any directional prefix.\footnote{The form \ipa{ɲɯ-me} exists, but it is the imperfective, not the sensory form of \ipa{me} `not exist', and has a dynamic sense `disappear' (cf section \ref{sec:ipfv}). }

In affirmative sentences, the sensory is generally found with second or third person verb forms. First person is limited to  cases like \ref{ex:YWrAZia} when the speaker discovers something about himself.

Sentences \ref{ex:WYWpendZi} and \ref{ex:Wkupe} illustrate the difference of use of the sensory and egophoric forms in third person contexts. These questions expect answers in the sensory and egophoric forms respectively. Question \ref{ex:WYWpendZi} was asked when I phoned from my parents' home (when I came for the holidays). The sensory is used because I only seldom meet with my parents, and the expectation is that I had just realized whether or not they were well after having arrived at their place. 

Question \ref{ex:Wkupe} on the other hand, asked about my son, expects an answer in the egophoric because since I live with him in the same house, I always know whether he is fine or not.

The factual would not be appropriate in these contexts because neither involve a permanent state part of common knowledge.

\begin{exe}
\ex \label{ex:WYWpendZi}
\gll 
\ipa{nɤ-mu}  	\ipa{nɤ-wa}  	\ipa{ni}  	\ipa{ɯ-ɲɯ́-pe-ndʑi?}  \\
\textsc{2sg.poss}-mother \textsc{2sg.poss}-father \textsc{du} \textsc{qu-sens}-be.good-\textsc{du} \\
\glt Are your parents well? (2014.12 conversation, Chenzhen)
\end{exe}


\begin{exe}
\ex \label{ex:Wkupe}
\gll \ipa{nɤ-tɕɯ} \ipa{ɯ-kú-pe?}\\
\textsc{2sg.poss}-son \textsc{qu-egoph}-be.good\\
\glt Is your son well? (2014.08 conversation, Dpalcan)
\end{exe}

 
 As in other languages with sensory evidentials, this form can be used to in contexts where surprise is implied (hence the term `mirative', on which see \citealt{delancey97mirative} and \citealt{hill12mirativity}). For instance \ref{ex:YWrAZia} shows sensory used with first person, where the speaker, after having been revived, discover herself in an unfamiliar place. The sensory here marks not so much surprise \textit{per se} (the interjection \ipa{ama} expresses this meaning) but the discovery of a new situation.

\begin{exe}
\ex \label{ex:YWrAZia}
\gll 
\ipa{ama,}  	\ipa{kɯki}  	\ipa{aʑo}  	\ipa{ki}  	\ipa{ŋotɕu}  	\ipa{ɲɯ-rɤʑi-a}  	\ipa{ɲɯ-ŋu?}  \\
\textsc{intrj:surprise} \textsc{dem:prox} \textsc{1sg} \textsc{dem:prox} where \textsc{sens}-stay-\textsc{1sg} \textsc{sens}-be \\
\glt Where am I? (The three leaves, 105)
\end{exe}


The sensory is also used concerning information that is somehow part of common knowledge, but that the speaker has not had the opportunity to personally confirm. For instance, it is commonly used instead of the factual for describing facts about animals that do not live in Rgyalrong areas and that the speaker only knows through indirect channels. Compare for instance the forms of the stative verbs \ipa{sɤɣ-mu} `be terrifying' and \ipa{mpɕɤr} `be beautiful': they appear in the factual when referring to  spiders or flowers found in the area (\ref{ex:sAGmu} and \ref{ex:mpCAr}) and in the sensory when referring to lions and zebras, which the speaker has only seen in zoos or in the television  (\ref{ex:YWsAGmu} and \ref{ex:YWmpCAr}).

 
 
 \begin{exe}
\ex \label{ex:sAGmu}
\gll 
\ipa{ŋgoŋpu}  	\ipa{ɴɢoɕna}  	\ipa{kɤ-ti}  	\ipa{ci}  	\ipa{tu}  	\ipa{tɕe,}  	\ipa{nɯnɯ}  	\ipa{wxti}  	\ipa{nɯ}  	\ipa{stoʁ}  	\ipa{jamar}  	\ipa{tu.}  	\ipa{kú-wɣ-rtoʁ}  	\ipa{tɕe}  	\ipa{sɤɣ-mu.}  \\
disaster spider \textsc{nmlz}:P-say \textsc{indef} exist:\textsc{fact} \textsc{lnk} \textsc{dem} be.big:\textsc{fact} \textsc{dem} bean about exist:\textsc{fact} \textsc{ipfv-inv}-look.at \textsc{lnk} \textsc{deexp}-be.afraid:\textsc{fact}  \\
\glt There is one that is  called `disaster spider', it is big, like the size of a bean. It is terrifying to look at it. (spiders, 128)
\end{exe}

\begin{exe}
\ex \label{ex:YWsAGmu}
\gll 
\ipa{sɯŋgi}  	\ipa{nɯ}  	\ipa{ɲɯ-sɤɣ-mu.}  \\
tiger \textsc{dem} \textsc{sens-deexp}-be.afraid \\
\glt The tiger  is terrifying. (Tigers, 64)
\end{exe}


\begin{exe}
\ex \label{ex:mpCAr}
\gll
\ipa{nɯnɯ}  	\ipa{ɯ-mɯntoʁ}  	\ipa{nɯ}  	\ipa{mpɕɤr.}  \\
\textsc{dem} \textsc{3sg.poss}-flower \textsc{dem} be.beautiful:\textsc{fact} \\
\glt Its flower is beautiful (\ipa{qarɣɤpɤt}, 105)
\end{exe}


\begin{exe}
\ex \label{ex:YWmpCAr}
\gll 
<banma> 	\ipa{nɯ}  	\ipa{ɲɯ-mpɕɤr}  \\
zebra \textsc{dem} \textsc{sens}-be.beautiful \\
\glt The zebra is beautiful. (Zebra, 128)
\end{exe}

As in other languages of the area, the sensory form is used for endopathic sensations (pain, itch, cold etc)  relating to the speaker of affirmative sentences/addressee of interrogatives (\citealt{tournadre14evidentiality}), as in examples \ref{ex:YWmNAm} and \ref{ex:mWjmtshama}. Endopathic verbs may either be stative verbs or transitive tropative verbs (on which see \citealt{jacques13tropative}).

\begin{exe}
\ex \label{ex:YWmNAm}
\gll
\ipa{tʰam} 	\ipa{tɕe} 	\ipa{mɯ́j-cʰa-a,} 	\ipa{a-mi} 	\ipa{ɲɯ-mŋɤm.} \\
now \textsc{lnk} \textsc{neg:sens}-can-\textsc{1sg} \textsc{1sg.poss}-foot \textsc{sens}-hurt \\
\glt Now I can't, my foot hurts. (\ipa{ʑmbɯlɯm}, 24)
\end{exe}

\begin{exe}
\ex \label{ex:mWjmtshama}
\gll
\ipa{tɕe} 	\ipa{ɯ-qiɯ} 	\ipa{ɲɯ-mtsʰam-a,} 	\ipa{ɯ-qiɯ} 	\ipa{mɯ́j-mtsʰam-a} 	\ipa{qʰe,} 	\ipa{ɕe} 	\ipa{mɤ-ɕe} 	\ipa{maŋe.} \\
\textsc{lnk} \textsc{3sg.poss}-half \textsc{sens}-hear-\textsc{1sg}   \textsc{3sg.poss}-half \textsc{neg:sens}-hear-\textsc{1sg}  \textsc{lnk}  \textsc{bare.inf}:go \textsc{neg}-\textsc{bare.inf}:go \textsc{sens}:not.exist \\
\glt I hear half of it, and don't hear the rest, whether I go or not the result is the same. (conversation, 2014)
\end{exe}


Unlike in Lhasa Tibetan where the sensory \ipa{ɴdug} cannot be used for non-personal endopathic feelings, this possibility is available in Japhug. The speaker can use the factual combined with an auxiliary such as \ipa{ŋgrɤl}  `be usually the case' or the hypothetical \ipa{tʰaŋ} (\ref{ex:rAZa}), but the sensory is also possible, as in \ref{ex:tWCGa.YWmNAm} to \ref{Wmi.YWmNAm}.


\begin{exe}
\ex \label{ex:rAZa}
\gll
\ipa{ɯ-rni}  	\ipa{ɯ-stu}  	\ipa{nɯ}  	\ipa{rɤʑa}  	\ipa{tɕe}  	\ipa{tɕe}  	\ipa{nɤ-sɤɣ-dɯɣ}  	\ipa{ŋgrɤl}  	\ipa{loβ}  \\
\textsc{3sg.poss}-gum \textsc{3sg.poss}-place \textsc{dem} itch:\textsc{fact} \textsc{lnk} \textsc{lnk} \textsc{trop-deexp}-be.fed.up:\textsc{fact} be.usually.the.case:\textsc{fact} \textsc{sfp} \\
\glt His gums itch, and he can't bear it. (about a baby whose teeths are growing, conversation, 2014-10)
\end{exe}

In \ref{ex:tWCGa.YWmNAm}, the sensory is used in a generic sentence, when the speaker has experienced himself the feeling and recounts his experience while presenting it as a generic fact. 

\begin{exe}
\ex \label{ex:tWCGa.YWmNAm}
\gll
\ipa{kɯ-maqʰu}  	\ipa{qʰe}  	\ipa{tɯ-ɕɣa}  	\ipa{ɲɯ-mŋɤm}  \\
\textsc{nmlz}:S/A-be.after \textsc{lnk} \textsc{genr.poss}-tooth \textsc{sens}-hurt \\
\glt Afterwards tooths hurt. (toothache, 66)
\end{exe}

The following examples show that in Japhug the sensory is used for endopathic sensations even in the case of referents other than the speaker. In \ref{ex:YWNWGAtChom} and \ref{nWrqo.YWmNAm}, direct observation


\ref{Wmi.YWmNAm} and 
\begin{exe}
\ex \label{ex:YWNWGAtChom}
\gll
\ipa{ɯ-ndzɤtshi}  	\ipa{nɯ}  	\ipa{ɲɯ-nɯ-ɣɤ-tɕʰom}  	\ipa{tɕe,}  	\ipa{ɯ-xtu}  	\ipa{ɲɯ-mŋɤm}  	\ipa{tɕe,}  	\ipa{pjɯ-kɯ-si}  	\ipa{ɣɤʑu.}  \\
\textsc{3sg.poss}-food \textsc{dem} \textsc{sens-auto-caus}-be.too.much \textsc{lnk} \textsc{3sg.poss}-belly \textsc{sens}-hurt \textsc{lnk} \textsc{ipfv-nmlz}:S/A-die exist:\textsc{sens} \\
\glt (The monkey eats) too much food, its belly aches, and some die of it. (Monkeys, 56)
\end{exe}

\begin{exe}
\ex \label{nWrqo.YWmNAm}
\gll \ipa{nɯ-mci} 	\ipa{kɤ-rɤwum} 	\ipa{maka} 	\ipa{mɯ́j-cha-nɯ} 	\ipa{tɕe} 	\ipa{nɯ-mci} 	\ipa{tu-ɣɤrɯβrɯβ} 	\ipa{ʑo} 	\ipa{ɲɯ-ŋu.}  
\ipa{tɕe} 	\ipa{nɯ-rqo} 	\ipa{ɲɯ-mŋɤm} 	\ipa{rca,} \\
\textsc{3pl.poss}-saliva \textsc{inf}-collect at.all \textsc{neg:sens}-can-\textsc{pl} \textsc{lnk} \textsc{3pl.poss}-saliva \textsc{ipfv}-flow.continuously \textsc{emph} \textsc{sens}-be \textsc{lnk} \textsc{3pl.poss}-throat \textsc{sens}-hurt \textsc{top}  \\
\glt They cannot keep the saliva in their mouth, and it flows continuously. Their throat hurt. (Mouth-foot disease, 6)
\end{exe}


\begin{exe}
\ex \label{Wmi.YWmNAm}
\gll
\ipa{kɯɕnɤsqi} 	\ipa{thɯ-azɣɯt} 	\ipa{ri,} \ipa{tɕe} 	\ipa{pɤjkʰu} 	\ipa{ɯ-mi} 	\ipa{ɲɯ-mŋɤm} 	\ipa{tɕe} 	\ipa{ri,} 	\ipa{nɯ} 	\ipa{kɯnɤ} 	\ipa{kʰa} 	\ipa{tsʰitsuku} 	\ipa{ɲɯ-nɤme} 	\ipa{ɕti.} \\
seventy \textsc{pfv}-reach but \textsc{lnk} already \textsc{3sg.poss}-foot \textsc{sens}-hurt \textsc{lnk} but \textsc{dem} also house some.things \textsc{sens}-work[III] be:\textsc{affirm:fact} \\
\glt He is seventy, his foot hurts already, but even like that he does all sorts of work at home. (Relatives, 49-50)
\end{exe}


Perception or feelings relating to the speaker are not necessarily expressed with the sensory, as they can be construed as part of common knowledge, as in \ref{ex:mAmtshama}. The factual  \ipa{mɤ-mtsʰam-a} `I can't hear' is used in this example because the audition issues of the speaker are a permanent state (not recently discovered), and is known by everybody.

\begin{exe}
\ex \label{ex:mAmtshama}
\gll
\ipa{kukukuku kukukuku} 	\ipa{tu-ti.} 	\ipa{nɯ} 	\ipa{tu-ti} 	\ipa{ɲɯ-ŋu} 	\ipa{tɕe,} 	\ipa{ɯ-skɤt} 	\ipa{mɯ́j-wxti} 	\ipa{kʰi.} 	\ipa{a-pɯ-ŋu,} \ipa{mɤ-mtsʰam-a} 	\ipa{woma} \\
{ } \textsc{ipfv}-say \textsc{dem}  \textsc{ipfv}-say \textsc{sens}-be \textsc{lnk}  \textsc{3sg.poss}-voice \textsc{neg:sens}-be.big \textsc{hearsay} \textsc{irr-ipfv}-be \textsc{neg}-\textbf{hear}:\textsc{fact-1sg} \textsc{sfp} \\
\glt (The snowcock) calls `kukuku', it calls like that. It is said that its voice is not big. Maybe it is like that, I can't hear (well) anyway. (Snowcock, 30-32)
\end{exe}

With intransitive verbs, the sensory is not normally used in the first person in affirmative sentences, except in two cases. First, in the case of the discovery of a new situation relating to oneself (as in \ref{ex:YWrAZia} above). Second, when the speaker refers to himself through the eyes of the addressee, as in \ref{ex:mtChi.kWfse.rJAlpu} and \ref{ex:thWmqlaRa}.


\begin{exe}
\ex \label{ex:mtChi.kWfse.rJAlpu}
\gll
\ipa{aʑo}  	\ipa{tɕʰi}  	\ipa{kɯ-fse}  	\ipa{rɟɤlpu}  	\ipa{ɲɯ-ŋu-a?}  \\
\textsc{1sg} what \textsc{nmlz}:S/A-be.like king \textsc{sens}-be-\textsc{1sg} \\
\glt What kind of a king am I? (The ape  king, 23)
\end{exe}


\begin{exe}
\ex \label{ex:thWmqlaRa}
\gll
\ipa{a-zda}  	\ipa{tɤ-tɕɯ}  	\ipa{ɲɯ-ŋu}  	\ipa{tɕe,}  	\ipa{χsɤr}  	\ipa{qaɕpa}  	\ipa{ci}  	\ipa{ɣɤʑu}  	\ipa{tɕe,}  	\ipa{nɯ}  	\ipa{tʰa-mqlaʁ.} \ipa{aʑo}  	\ipa{tɕʰeme}  	\ipa{ɲɯ-ŋu-a}  	\ipa{tɕe,}  	\ipa{rŋɯl}  	\ipa{qaɕpa}  	\ipa{ci}  	\ipa{ɣɤʑu}  	\ipa{tɕe,}  	\ipa{nɯ}  	\ipa{tʰɯ-mqlaʁ-a.}  \\
\textsc{1sg.poss}-companion \textsc{indef.poss}-son \textsc{sens}-be  \textsc{lnk} gold frog \textsc{indef} exist:\textsc{sens} \textsc{lnk} \textsc{dem} \textsc{pfv}:3$\rightarrow$3'-swallow \textsc{1sg} girl \textsc{sens}-be-\textsc{1sg}  \textsc{lnk} silver frog \textsc{indef} exist:\textsc{sens} \textsc{lnk} \textsc{dem} \textsc{pfv}-swallow-\textsc{1sg} \\
\glt (As you can see) my companion (present at the moment) is a boy, there was a golden frog and he swallowed it, I am a girl, there was a silver frog and I swallowed it. (Nyima Wodzer, 144)
\end{exe}

In example \ref{ex:thWmqlaRa}, the speaker uses the sensory form to refer to himself and her companion, ie. the form the addressee would be expected to use. 


\subsection{Egophoric present } \label{sec:eggoph}
The egophoric present  is built by prefixing \ipa{ku--} `towards east' to the stem III or stem I of the verb. It is thus homophonous with the imperfective in the case of verbs whose intrinsic direction is `east'.  For instance the form \ipa{ku-rɤʑi} is ambiguous between \textsc{egoph}-remain and \textsc{ipfv}-remain (both could be translated as `he is there' or he is remaining there' depending on the context).

There are no specific egophoric   existentials verbs; they simply take the prefix \ipa{ku--} like regular verbs, as in \ref{ex:kume}.

\begin{exe}
\ex \label{ex:kume}
\gll 
\ipa{aʑo}  	\ipa{kɯre}  	\ipa{a-ʁa}  	\ipa{ku-me}  	\\
\textsc{1sg} here \textsc{1sg.poss}-free.time \textsc{egoph}-not.exist \\
\glt I don't have time here. (Rkangrgyal, 47)
\end{exe}

The egophoric  is rather rare in narratives, but very common in conversations. Like the factual, it is used to express intimate knowledge of an event or state on the part of the speaker, not resulting from guess or recent information mediated through the senses. It cannot express a general or gnomic state of affair, it is only used to refer to an ongoing state or action (as in \ref{ex:kutaRa}). With transitive verbs, it is most commonly used (though not exclusively) with the progressive \ipa{asɯ--} prefix.


\begin{exe}
\ex \label{ex:kutaRa}
\gll 
<kuabao> 	\ipa{ɯ-spa}  	\ipa{ci}  	\ipa{ku-taʁ-a}  \\
bag \textsc{3sg.poss}-material \textsc{indef} \textsc{egoph}-weave-\textsc{1sg} \\
\glt I am weaving a bag (conversation, 2014-10)
\end{exe}

It appears mainly  in second person form in questions (examples \ref{ex:WkutWscitnW} and \ref{ex:kutWnAme}) and in first person form in declarative sentences (as in the first clause in \ref{ex:kusciti}, the answer to the question in \ref{ex:WkutWscitnW}).  

\begin{exe}
\ex \label{ex:WkutWscitnW}
\gll  \ipa{`a-ʁi} 			\ipa{ɯ-kú-tɯ-scit-nɯ?}' 	\ipa{ra} 	\ipa{to-ti,} \\
\textsc{1sg.poss}-younger.sibling  \textsc{qu-egoph}-2-be.happy-\textsc{pl} \textsc{pl} \textsc{evd}-say \\
\glt She said: `Are you_p happy, my sister?' (The frog 2002, 121)
\end{exe}

\begin{exe}
\ex \label{ex:kutWnAme}
\gll \ipa{nɯtɕu}  \ipa{tɕʰi} \ipa{ku-tɯ-nɤme?}\\
what there \textsc{egoph}-2-work[III] \\
\glt What are you doing there? (The smart rabbit 2012, 8)
\end{exe}
 
 
\begin{exe}
\ex \label{ex:kusciti}
\gll
\ipa{tɕʰeme} 	\ipa{nɯ} 	\ipa{kɯ} 	\ipa{`wuma} 	\ipa{ʑo} 	\ipa{ku-scit-i,} \ipa{rɟɤlpu} 	\ipa{ri} 	\ipa{a-taʁ} 	\ipa{wuma} 	\ipa{ku-sna} \ipa{ʁjoʁ} 	\ipa{ra} 	\ipa{ri} 	\ipa{wuma} 	\ipa{ʑo} 	\ipa{ku-pe-nɯ'} \ipa{to-ti,} \\
girl \textsc{dem} \textsc{erg} really \textsc{emph} \textsc{egoph}-be.happy-\textsc{1pl}  roi also \textsc{1sg}-on really \textsc{egoph}-be.kind servant \textsc{pl} also really \textsc{emph}   \textsc{egoph}-be.good \textsc{ifr}-say \\
\glt The girl said `We are very happy, the king is very kind with me, the servants are very nice.'
(The frog 2002, 122-4)
\end{exe}

It is used also in either declarative (example \ref{ex:kusciti}) or  interrogative sentences (\ref{ex:Wkupe} and \ref{ex:WkudAn}) to refer to third persons, in the case of things belonging to or persons attached to the speaker (in declarative sentences) or the addressee (in the case of interrogatives). In example \ref{ex:kusciti}, the  use of present factual is motivated by the facts that (1) the speaker is affected by the state of the persons she refers to (2) these persons are member or her household (her husband (the king) and her servant).

 
 
 
 \begin{exe}
\ex \label{ex:WkudAn}
\gll \ipa{nɤ-kɤ-nɤma} 	\ipa{ɯ-kú-dɤn?}  \\
 \textsc{2sg.poss-nmlz:P}-work \textsc{qu-egoph}-be.many \\
\glt Do you have a lot of work? (2014.10 conversation, Chenzhen)
\end{exe}


 
As seen above, a question in the present factual expects an answer in the same form. However, answering in the periphrastic imperfective is also possible in the case of dynamic verbs, as in \ref{ex:pjWrAGrWa}, the answer to \ref{ex:kutWnAme}.\footnote{In another version of the same story by the same speaker however, the present factual is also found in the answer.}

 \begin{exe}
\ex \label{ex:pjWrAGrWa}
\gll
 \ipa{a-pi} 	\ipa{kɯrtsɤɣ} 	\ipa{ma-tɯ-ɤrju} 	\ipa{ma,} 	\ipa{maka} 	\ipa{aʑo} 	\ipa{a-xtu} 	\ipa{ɯ-tɯ-mŋɤm} 	\ipa{ɲɯ-sɤre} 	\ipa{ʑo} 	\ipa{tɕe,} 	\ipa{kukutɕu} 	\ipa{pjɯ-rɤɣrɯ-a} 	\ipa{ŋu} 	\\
 \textsc{1sg.poss}-elder.sibling leopard \textsc{neg:imp}-2-say because at.all \textsc{1sg} \textsc{1sg.poss}-belly \textsc{3sg-nmlz:degree}-hurt \textsc{sens}-be.funny \textsc{emph} \textsc{lnk} here \textsc{ipfv}-treat.with.heat-\textsc{1sg} be:\textsc{fact}  \\
\glt Brother leopard, don't talk, my belly hurts terribly, and I am treating it. (The smart rabbit 2012, 10-2)
 \end{exe}
 


\subsection{Imperfective} \label{sec:ipfv}
The imperfective form is mainly used in periphrastic TAM categories and in clause linking, and rarely appears 

use with \ipa{ɕɯŋgɯ} \citet{jacques14linking}


 Dynamic with stative verbs
 
\ipa{ʑɯrɯʑɤri}  	\ipa{tɕe}  	\ipa{chɯ-ɣɯrni}  	\ipa{ŋu}  	\ipa{tɕe,}  \ipa{thɯ-ɣɯrni}  	\ipa{tɕe}  	\ipa{tɕe}  	\ipa{chɯ-tɯt}  	\ipa{ŋu}  	\ipa{tɕe,}  




Habitual vs immediate
\ipa{tɯrme} 	\ipa{ra} 	\ipa{kɯ} 	\ipa{pjɯ-tɯ-mto} 	\ipa{ʑo} 	\ipa{sat-nɯ} 	\ipa{ɕti.} 
\ipa{ɕɤmɯɣdɯ} 	\ipa{tu-lɤt-nɯ} 	\ipa{qhe} 	\ipa{pjɯ-sat-nɯ} 	\ipa{ɕti} 

\subsection{Perfective}
sensory?

also factual

past transitive \ipa{--t} suffix 


\ipa{nɯ-tɯ-nɯ-jmɯt} 	\ipa{ndʐa} 	\ipa{ɕti} 	\ipa{ma} 	\ipa{tɕi-diandian} 	\ipa{ɯ-khokɯm} 	\ipa{nɯ-tɯ-ɣe.} 	\ipa{nɯ} 	\ipa{mɤɕɯŋgɯ} 


vs

\ipa{ɲɤ-nɯ-jmɯt-a}
\subsection{ɓast Inferential}

\subsection{Past imperfective}

\subsection{Irrealis}
\citet{jackson07irrealis}
\subsection{Imperative}

\section{Other TAM morphemes}
\subsection{Progressive}

Japhug has a progressive prefix \ipa{asɯ--} restricted to transitive verbs. It is not used on its own: it always appears in combination with one of the TAM categories described in section \ref{sec:directional}. Due to semantic mismatch, it does not appear with perfective and evidential forms.


Examples have only been found with the following five categories: past imperfective (example \ref{ex:pasWfCAtndZi}), evidential imperfective (\ref{ex:pjAkAsWtsxWBci}), sensory (\ref{ex:YAznAthWthu}), present (\ref{ex:kosWBzjoz}), factual (\ref{ex:asWndo}) and plain imperfective. 

Two of these TAM categories, namely past imperfective (prefixed in \ipa{pɯ--}) and evidential imperfective (\ipa{pjɤ--}), are not compatible on their own with most transitive verbs in Japhug, as suggested by \citet{lin11direction} (except for tropative verbs, see \citealt{jacques13tropative}). For all non-tropative transitive verbs, past imperfective and past evidential imperfective can only be build in combination with the progressive \ipa{asɯ--}. 

The progressive prefix is optional with the other four categories. Used with the sensory, plain imperfective and present, the progressive excludes habitual or generic interpretations of these tense. With the factual, it also excludes future interpretation.


The progressive \ipa{asɯ--} prefix presents four morphological peculiarities: allomorphy, combination with the evidential circumfix, loss of morphological transitivity and infixation of the inverse prefix.


\subsubsection{Allomorphy} \label{sec:prog.allomorphy}
Depending on the environment, the progressive prefix has six allomorphs: \ipa{asɯ--}, \ipa{az--}, \ipa{ɤsɯ--}, \ipa{ɤz--}, \ipa{osɯ--} and \ipa{oz--}. The allomorphs   \ipa{az--} / \ipa{ɤz--} / \ipa{oz--} occur when preceding a sonorant initial prefix (example \ref{ex:YAznAthWthu}). The allomorphs \ipa{asɯ--} and \ipa{az--} occur in word-initial position and following the past imperfective prefix \ipa{pɯ--} (examples \ref{ex:pasWfCAtndZi} and \ref{ex:asWndo}). The allomorphs \ipa{osɯ--} and \ipa{oz--} result from fusion with a preceding prefix whose main vowel in \ipa{u} (example \ref{ex:kosWBzjoz}).\footnote{In this example, despite what may transpire from the translation, \ipa{βzjoz} `study' is transitive; its P is the noun <chuzhong> `Junior High School'. It is a calque from Chinese \zh{读初中} \ipa{dú chūzhōng}.} The allomorphs \ipa{ɤsɯ--} and \ipa{ɤz--} are found in all other contexts. Note that verbs whose stem begins in \ipa{a--} present exactly the same vowel alternations (see \citealt{jacques07passif}; these verbs include passive, reflexive, some denominal verbs and a few others; all are intransitive verbs).


\begin{exe}
\ex \label{ex:pasWfCAtndZi}
\gll \ipa{pɯ-asɯ-fɕɤt-ndʑi} 	\ipa{nɯ} 	\ipa{ra,} 	\ipa{zlawawozɤr} 	\ipa{nɯ} 	\ipa{kɯ} 	\ipa{pjɤ-mtsʰɤm}\\
\textsc{pst.ipfv-prog}-tell-\textsc{du} \textsc{dem} \textsc{pl}  Zlaba.Wodzer \textsc{dem} \textsc{erg} \textsc{evd}-hear\\
\glt Zlaba Wodzer heard what they were saying. (Nyimawodzer1, 32)
\end{exe}

\begin{exe}
\ex \label{ex:kosWBzjoz}
\gll \ipa{akɯ} <xianzhong> \ipa{ri} <chuzhong> \ipa{ku-osɯ-βzjoz}. \\
east district.high.school \textsc{loc} high.school \textsc{pres-prog}-learn \\
\glt She is reading Junior High School at the District High school, east of here. (Relatives 363-4)
\end{exe}

 \begin{exe}
\ex \label{ex:YAznAthWthu}
\gll
\ipa{tɤrɣe}  	\ipa{ɯ-cʰɯ-z-ɣɯri}  	\ipa{ɲɯ-ɤz-nɤtʰɯtʰu}  	 \\
pearl \textsc{3sg-ipfv:downstream-nmlz:oblique}-thread.a.needle \textsc{sens-prog}-ask.everywhere \\
\glt He is asking everywhere about (where) the thing used to thread needle is. (Conversation \ipa{taʁrdo}, 72)
\end{exe}


\subsubsection{Evidential imperfective}

The evidential imperfective forms of verbs with the progressive prefix \ipa{asɯ--} follows the same pattern as verbs whose stems begins in \ipa{a--} (including passive and reflexive verbs with the \ipa{a--} prefix, see \citealt{jacques07passif}). In addition to the regular past evidential imperfective \ipa{pjɤ--} prefix, the circumfix \ipa{k--}...\ipa{--ci} is added. The first element \ipa{k--} of this circumfix occurs between the evidential imperfective prefix \ipa{pjɤ--} and the progressive prefix \ipa{ɤsɯ--}.\footnote{Note however that the suffixal element \ipa{--ci} is optional, though it is present most of the time.} This form is illustrated by examples \ref{ex:pjAkAsWtsxWBci} and \ref{ex:pjAkAsWNgaci}.

\begin{exe}
\ex \label{ex:pjAkAsWNgaci}
\gll
<lvguan>	\ipa{ɣɯ} 	\ipa{nɯ-ʁɲɤrpa} 	\ipa{nɯ} 	\ipa{kɯ} 	\ipa{tɯ-ŋga} 	\ipa{rca} 	\ipa{kɯ-mpɕɯ\textasciitilde{}mpɕɤr} 	\ipa{ʑo} 	\ipa{pjɤ-k-ɤsɯ-ŋga-ci} 	\ipa{tɕe,} 	\ipa{kɯm} 	\ipa{nɯ} 	\ipa{tɕu} 	\ipa{pjɤ-k-ɤmdzɯ-ci.} 	\\
hotel \textsc{gen} \textsc{3pl.poss}-manager \textsc{dem} \textsc{erg} \textsc{indef.poss}-clothes \textsc{emph} \textsc{nmlz:S/A-emph}\textasciitilde{}be.beautiful \textsc{emph} \textsc{evd.ipfv-evd-prog}-wear-\textsc{evd} \textsc{lnk} door \textsc{dem} \textsc{loc} \textsc{evd.ipfv-evd}-sit-\textsc{evd}
\\
\glt The hotel manager was wearing nice clothes and sitting near the door. (The thief and the landlord)
\end{exe}

\subsubsection{Transitivity}

Verb forms with the prefix \ipa{asɯ--} lack two of the obligatory transitive markers found in Japhug verbs, namely stem 3 alternation and past tense transitive \ipa{--t--} suffix. Only stem alternation is discussed here.

Japhug verbs exhibit stem alternation in non-past TAM categories (sensory, present, imperative, irrealis, imperfective and factual) in direct singular A forms (\textsc{1/2/3sg}$\rightarrow$3). Following \citet{jackson00puxi}, we refer to this stem as `stem 3' (stem 1 being the base stem, and stem 2 the past stem). The use of this stem is illustrated in example \ref{ex:YWndAm}, where the verb \ipa{ndo} `hold' in the imperfective form has stem 3 \ipa{ndɤm}.

\begin{exe}
\ex \label{ex:YWndAm}
\gll \ipa{kɤ-kɤ-sɯ-ɕke} 	\ipa{ɯ-mdoʁ} 	\ipa{kɯ-fse} 	\ipa{ɲɯ-ndɤm} 		\ipa{ŋu} \\
\textsc{pfv-nmlz:P-caus}-burn \textsc{3sg.poss}-colour \textsc{nmlz:S/A}-be.like \textsc{ipfv}-hold[III] be:\textsc{fact} \\
\glt  It has the colour of something that has been burnt. (\ipa{ɲɤβrɯɣ}, 14)
\end{exe}

When a verb in non-past TAM forms takes the \ipa{asɯ--}, stem alternation does not occur. Examples \ref{ex:asWndo} and \ref{ex:YAsWndo}, in factual and sensory form have the base stem \ipa{ndo} instead of stem 3 \ipa{ndɤm} as expected in forms without the progressive.


\begin{exe}
\ex \label{ex:asWndo}
\gll
\ipa{sɯjno} 	\ipa{ɯ-mdoʁ} 	\ipa{ʑo} 	\ipa{asɯ-ndo.} \\
grass \textsc{3sg.poss}-colour \textsc{emph} \textsc{prog}-hold:\textsc{fact} \\
\glt It has the colour of grass. (Caterpillar, 69)
\end{exe}


\begin{exe}
\ex \label{ex:YAsWndo}
\gll
\ipa{kɯki} 	\ipa{ɯ-mdoʁ} 	\ipa{tsa} 	\ipa{ɲɯ-ɤsɯ-ndo} \\
this  \textsc{3sg.poss}-colour  a.little \textsc{sens-prog}-hold \\
\glt It has a colour a bit like this one. (Slugs, 159)
\end{exe}

However, adding the progressive \ipa{asɯ--} has no effect on flagging: the A still receives ergative \ipa{kɯ} marking, as shown by example \ref{ex:pjAkAsWtsxWBci}.

\begin{exe}
\ex \label{ex:pjAkAsWtsxWBci}
\gll
\ipa{rgɤnmɯ}  	\ipa{nɯ}  	\ipa{kɯ}  	\ipa{li}  	\ipa{iɕqʰa}  	<yuwang>	\ipa{nɯ}  	\ipa{pjɤ-k-ɤsɯ-tʂɯβ-ci}  		\\
old.woman \textsc{dem} \textsc{erg} again the.aforementioned net \textsc{dem} \textsc{evd.ipfv-evd-prog}-sew-\textsc{evd} \\
\glt The old woman was sewing the nets as before. (The fisherman and his wife, 284)
\end{exe}


In the related Tshobdun language, the cognate prefix \ipa{ɐsɐ--} has a similar effect, and is labelled by \citet{jackson03caodeng} as `low transitivity progressive'.

\subsubsection{Infixation of the inverse prefix}
The inverse prefix \ipa{wɣ--}, whose morphosyntactic function is described in \citet{jacques10inverse}, appears in the same prefixal slot as the progressive \ipa{asɯ--} in the Japhug verbal template, and is actually infixed within this prefix, resulting in \ipa{ɤ́<wɣ>sɯ--} [\ipa{óɣsɯ}] or \ipa{ɤ́<wɣ>z--} [\ipa{óɣz}]. The form \ipa{pjɤ-k-ɤ́<wɣ>z-nɤjo-ci} `he was waiting for him' in \ref{ex:pjAkAwGznAjoci} is the only example of this combination in the corpus.


\begin{exe}
\ex \label{ex:pjAkAwGznAjoci}
\gll
\ipa{tɕe} 	\ipa{pjɤ-ɣi} 	\ipa{tɕe} 	\ipa{qala} 	\ipa{kɯ} 	\ipa{pjɤ-k-ɤ́<wɣ>z-nɤjo-ci} 	\ipa{tɕe,} \\
\textsc{lnk} \textsc{evd:down}-come \textsc{lnk} rabbit \textsc{erg} \textsc{evd.ipfv-evd<inv>}-wait-\textsc{evd} \textsc{lnk} \\
\glt (The leopard) came down, and the rabbit was waiting for him there. (The smart rabbit.2014, 60)
\end{exe}

It is however possible to elicitate other forms of this type, such as \ref{ex:YWtAwGsWzgroR}, without any constraint.

\begin{exe}
\ex  \label{ex:YWtAwGsWzgroR}
\gll  \ipa{ɲɯ-tɯ-ɤ́<wɣ>sɯ-zgroʁ}    \\
\textsc{sens-2-prog<inv>}-attach \\
\glt He is attaching you. (elicitation, Chen Zhen)
\end{exe}

\subsection{Conative}
\subsection{Hearsay}
\ipa{kʰi}

source vs access  \citet{tournadre14evidentiality},
 



  \ipa{pɣɤkhɯ}  	\ipa{nɯ}  	\ipa{kɯ}  	\ipa{qaɲi}  	\ipa{kɤ-sat}  	\ipa{wuma}  	\ipa{ʑo}  	\ipa{cha}  	\ipa{khi.}  

  
  
  
  \ipa{tɕeri}  	\ipa{ʁloŋbutɕhi}  	\ipa{nɯ}  	\ipa{ʁnɯtɯphɯ}  	\ipa{ɣɤʑu}  	\ipa{khi}  

  
    \section{Periphrastic TAM categories}
  
   (\ipa{ŋu} `be' and \ipa{maʁ} `not be').
   
  \subsection{Periphrastic imperfective} \label{sec:periphrastic.ipfv}
  \subsection{Periphrastic past imperfective } \label{sec:periphrastic.pst}
  
    \subsection{Periphrastic evidential imperfective } \label{sec:periphrastic.evd}
    
        \subsection{Periphrastic conative } \label{sec:periphrastic.fact}
        
  
  
\bibliographystyle{linquiry2}
\bibliography{bibliogj}
\end{document}