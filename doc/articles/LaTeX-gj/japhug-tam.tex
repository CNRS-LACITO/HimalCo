\documentclass[oldfontcommands,oneside,a4paper,11pt]{article} 
\usepackage{fontspec}
\usepackage{natbib}
\usepackage{booktabs}
\usepackage{xltxtra} 
\usepackage{longtable}
\usepackage{polyglossia} 
\usepackage[table]{xcolor}
\usepackage{gb4e} 
\usepackage{multicol}
\usepackage{graphicx}
\usepackage{float}
\usepackage{hyperref} 
\hypersetup{bookmarks=false,bookmarksnumbered,bookmarksopenlevel=5,bookmarksdepth=5,xetex,colorlinks=true,linkcolor=blue,citecolor=blue}
\usepackage[all]{hypcap}
\usepackage{memhfixc}
\usepackage{lscape}

\bibpunct[: ]{(}{)}{,}{a}{}{,}

%\setmainfont[Mapping=tex-text,Numbers=OldStyle,Ligatures=Common]{Charis SIL} 
\newfontfamily\phon[Mapping=tex-text,Ligatures=Common,Scale=MatchLowercase,FakeSlant=0.3]{Charis SIL} 
\newcommand{\ipa}[1]{{\phon \mbox{#1}}} %API tjs en italique
\newcommand{\ipab}[1]{{\scriptsize \phon#1}} 

\newcommand{\grise}[1]{\cellcolor{lightgray}\textbf{#1}}
\newfontfamily\cn[Mapping=tex-text,Ligatures=Common,Scale=MatchUppercase]{MingLiU}%pour le chinois
\newcommand{\zh}[1]{{\cn #1}}
\newcommand{\refb}[1]{(\ref{#1})}


\XeTeXlinebreaklocale 'zh' %使用中文换行
\XeTeXlinebreakskip = 0pt plus 1pt %
 %CIRCG
 


\begin{document} 
\title{TAM in Japhug\footnote{
} }
\author{Guillaume Jacques}
\maketitle
%\linenumbers

\section{Introduction}


 \section{Morphology}
\subsection{Directional prefixes} \label{sec:directional}
All finite verbal forms in Japhug (except the factual) and some nominalized forms have a directional prefix that contains information on TAM, transitivity and  (in the case of motion and concrete action verbs) the direction of the action.

With the exception of contracting verbs whose stem starts in \ipa{a--} and which present special alternations, Japhug intransitive verbs have three series of prefixes (A, B and D) and transitive ones four series, as shown in Table \refb{tab:directional}. The distribution of these four series will be explained in more detail in section \refb{sec:finite.TAM}.

\begin{table}[H]
\caption{Directional prefixes in Japhug Rgyalrong} \label{tab:directional}
\resizebox{\columnwidth}{!}{
\begin{tabular}{llllll}
\toprule
   &  	perfective  (A) &  	imperfective  (B)  &  	perfective 3$\rightarrow$3' (C)  &  	evidential  (D) \\  	
   \midrule
up   &  	\ipa{tɤ--}   &  	\ipa{tu--}   &  	\ipa{ta--}   &  	\ipa{to--}   \\  	
down   &  	\ipa{pɯ--}   &  	\ipa{pjɯ--}   &  	\ipa{pa--}   &  	\ipa{pjɤ--}   \\  	
upstream   &  	\ipa{lɤ--}   &  	\ipa{lu--}   &  	\ipa{la--}   &  	\ipa{lo--}   \\  	
downstream   &  	\ipa{tʰɯ--}   &  	\ipa{cʰɯ--}   &  	\ipa{tʰa--}   &  	\ipa{cʰɤ--}   \\  	
east   &  	\ipa{kɤ--}   &  	\ipa{ku--}   &  	\ipa{ka--}   &  	\ipa{ko--}   \\  	
west   &  	\ipa{nɯ--}   &  	\ipa{ɲɯ--}   &  	\ipa{na--}   &  	\ipa{ɲɤ--}   \\  	
no direction &\ipa{jɤ--}   &  	\ipa{ju--}   &  	\ipa{ja--}   &  	\ipa{jo--}   \\  	
\bottomrule
\end{tabular}}
\end{table}

Most verbs have one intrinsic direction which is lexically determined. For instance, the verb \ipa{sat} `kill' selects the direction `down' for all its forms: \textbf{perfective} \textsc{1sg$\rightarrow$3sg} \ipa{pɯ-sat-a}, \textbf{imperfective} \ipa{pjɯ-sat}, \textbf{perfective} \textsc{3sg$\rightarrow$3'} \ipa{pa-sat} and \textbf{evidential} \ipa{pjɤ-sat}. 

Some verbs may allow several directions with slightly different semantics. Thus, \ipa{ndza} `eat'   normally   selects the  `up' direction (\textbf{perfective} \textsc{3sg$\rightarrow$3'} \ipa{ta-ndza} `he ate it'), but when applied to carnivorous animals we also find the `downstream' direction. This can lead to further aspectual distinctions. For instance, the direction `downstream', when used with stative verbs, indicates a progressive development. 

Verbs of motion and some verbs of concrete action can be associated with all seven series of prefixes to indicate the direction of the motion. The  `no direction' series of prefixes only occurs with motion verbs. 

Only three verbs have defective paradigms and never occur with directional prefixes: the sensory existential copulas \ipa{ɣɤʑu} `exist' and \ipa{maŋe} `not exist' and the verb \ipa{kɤtɯpa} `speak' (see the paradigm of the latter in \citealt[1215]{jacques12incorp}).

\subsection{Contracting verbs}

\citet{jacques07passif}
The evidential and evidential imperfective forms are used with the circumfix \ipa{k--}...\ipa{--ci} in the case of verb forms whose stem begins in \ipa{a--} (including verbs with the progressive \ipa{asɯ--}).

\subsection{Stem alternation} \label{sec:stem}
The existence of stem alternations in Rgyalrong was first reported by \citet{jackson00puxi}, who proposes to distinguish three stems: the base stem (stem 1), the perfective stem (stem 2) and the non stem (stem 3). Some varieties of Zbu Rgyalrong appear to have an additional progressive stem distinct from stem 2 in the progressive form (\citealt[352]{jacques04these}).


In Kamnyu Japhug, only four verbs have a   perfective stem distinct from the base stem; the list is provided in Table \refb{tab:stem2}. 


 \begin{table} 
\caption{Stem 2 alternation in Japhug Rgyalrong} \label{tab:stem2} \centering
\begin{tabular}{llllll}
\toprule
Stem 1 & meaning &Stem 2 \\
\midrule
\ipa{ɕe}& to go (vi)&  \ipa{ari} \\
\ipa{sɯxɕe}& to sent (vt)  &\ipa{sɤɣri} \\
\ipa{ɣi}& to come (vi)  &\ipa{ɣe} \\
\ipa{ti}& to say (vt)  &\ipa{tɯt} \\
\bottomrule
\end{tabular}
\end{table}


Stem 3 on the other hand is fully productive. The rules of vowel alternation in Table \refb{tab:stem3} apply to all finite transitive verbs in the forms \textsc{1sg}$\rightarrow$3, \textsc{2sg}$\rightarrow$3 and \textsc{3sg}$\rightarrow$3'; stem 3 does not appear in verb forms with the inverse marker (see \citealt{gongxun14agreement}). \citet[351-7]{jacques04these} provides a historical analysis of these alternations, and shows that they result from the fusion of the verb stem with two suffixes. %Some Japhug dialects present \ipa{--u} $\rightarrow$ \ipa{--ɯm} and  \ipa{--i} $\rightarrow$ \ipa{--ɯm} alternations.  

 \begin{table} 
\caption{Stem 3 alternation in Japhug Rgyalrong} \label{tab:stem3} \centering
\begin{tabular}{llllll}
\toprule
Stem 1 & Stem 3 \\
\midrule
\ipa{--a} & \ipa{--e} \\
\ipa{--u} & \ipa{--e} \\
\ipa{--ɯ} & \ipa{--i} \\
\ipa{--o} & \ipa{--ɤm} \\
\bottomrule
\end{tabular}
\end{table}

Following the Leipzig glossing rules, we indicate stem 2 as [II] and stem 3 as [III] in the glosses in this paper.

\section{Simple TAM categories} \label{sec:finite.TAM}

There are nine basic finite TAM categories in Japhug, as represented in Table \refb{tab:finite.forms}. All finite forms except the factual require one and only one directional prefix. All forms can be correctly produced by combining the appropriate derivational prefixes and stems.\footnote{For the TAM categories requiring stem 3, it is restricted to  \textsc{1sg}$\rightarrow$3, \textsc{2sg}$\rightarrow$3 and \textsc{3sg}$\rightarrow$3' forms; all other forms take the base stem.  The person affixes  are not discussed here; for  more information on this topic, see \citet{jacques10inverse}.}

 terminology: \citet{tournadre14evidentiality},  \citet{tournadre08conjunct}, 
 

\begin{table}
\caption{Finite verb categories in Japhug Rgyalrong} \label{tab:finite.forms} \centering
\begin{tabular}{lllllll}
\toprule
&	&	stem&	prefixes\\
\midrule
factual&	\textsc{fact} &	1 or 3&	no prefix\\
imperfective&	\textsc{ipfv} &	1 or 3&	B\\
perfective&	\textsc{pfv} &	2&	A or C\\
past imperfective&	\textsc{pst.ipfv} &	2&	\ipa{pɯ--}\\
evidential&	\textsc{evd} &	1&	D\\
evidential imperfective&	\textsc{evd.ipfv} &	1&	\ipa{pjɤ--}\\
testimonial&	\textsc{testim} &	1 or 3&	\ipa{ɲɯ--}\\
present&	\textsc{pres} &	1 or 3&	\ipa{ku--}\\
irrealis&	\textsc{irr} &	1 or 3&	\ipa{a--} + A\\
imperative&	\textsc{imp} &	1 or 3&	A\\
\bottomrule
\end{tabular}
\end{table}

 

\subsection{Factual}

The factual is the only TAM category in Japhug without a directional prefix. In the case of verbs without stem III alternation (i.e. transitive verbs with anon-alternating rhyme and intransitive verbs), it is realized as the bare stem. In spite of the absence of overt marking for most verbs, we still indicate `factual' in all glosses of finite verb forms. The gloss is indicate as a suffix (verb:\textsc{fact}), rather than as a prefix, to avoid confusion in the case of verb forms with several derivation prefixes.

The factual has three main functions when used in an independent clause without an auxiliary verb (in the later case see section \ref{sec:periphrastic.fact}). 

First,  in the case of stative verbs, whether adjectival stative verbs or existential verbs/copulas, the factual is used to describe facts considered to belong to everybody's common knowledge. Example \ref{ex:kumpGAtCW} illustrate five examples of the use of the factual in this way, including copulas and adjectives. 

\begin{exe}
\ex \label{ex:kumpGAtCW}
\gll
\ipa{tɕe} 	\ipa{kumpɣɤtɕɯ} 	\ipa{nɯnɯ} 	\ipa{pɣɤtɕɯ} 	\ipa{nɯ-rca,} 
 \ipa{kɯ-xtɕi} 	\ipa{ci} 	\ipa{zdoʁzdoʁ} 	\ipa{\textbf{ŋu}} 	\ipa{tɕe,}  \ipa{ɯʑo} 	\ipa{\textbf{xtɕi}} 	\ipa{ri} 	\ipa{wuma} 	\ipa{ʑo} 	\ipa{\textbf{ɕqraʁ}} \ipa{tɕe}  	\ipa{ɯ-mɲaʁ} 	\ipa{ɯ-rkɯ} 	\ipa{nɯnɯ} 	\ipa{ra} \ipa{kɯ-ɲaʁ} 	\ipa{kɯ} 	\ipa{tú-wɣ-fskɤr,} 	 	\ipa{nɯ} 	\ipa{ɯ-taʁ} 	\ipa{ri,} 	\ipa{χanɯni,} 	\ipa{kɯ-xtɕɯ\textasciitilde{}xtɕi} 	\ipa{kɯ-ɣɯrni} 	\ipa{kɯ-fse} 	\ipa{\textbf{tu},}  	\ipa{ɯ-xtɤpa} 	\ipa{nɯ} \ipa{ra,} 	\ipa{ɯ-rqopa} 	\ipa{pjɯ-ʑe} 	\ipa{tɕe,} 	\ipa{nɯ} \ipa{ra,}  \ipa{ɯ-jme} 	\ipa{mɯ-thɯ-nɯɬoʁ} 	\ipa{mɤɕtʂa} 	\ipa{nɯ} 	\ipa{\textbf{wɣrum}} \\
\textsc{lnk} sparrow \textsc{dem} bird \textsc{3pl}-among \textsc{nmlz}:S/A-be.small \textsc{indef} \textsc{ideo:stat}:small.and.cute \textbf{be}:\textsc{fact} \textsc{lnk}  \textsc{3sg} \textbf{be.small}:\textsc{fact} but really \textsc{emph} \textbf{be.smart}:\textsc{fact}  \textsc{lnk}  \textsc{3sg:poss}-eye \textsc{3sg:poss}-border \textsc{dem}  \textsc{pl}  \textsc{nmlz}:S/A-be.black \textsc{erg} \textsc{ipfv-inv}-surround \textsc{dem} \textsc{3sg}-on \textsc{loc} a.little \textsc{nmlz}:S/A-\textsc{emph}\textasciitilde{}be.small \textsc{nmlz}:S/A-be.red \textsc{nmlz}:S/A-be.like \textbf{exist}:\textsc{fact} \textsc{3sg:poss}-belly \textsc{dem}  \textsc{pl}  \textsc{3sg:poss}-throat \textsc{ipfv}-begin[III] \textsc{lnk}   \textsc{dem}  \textsc{pl}  \textsc{3sg:poss}-tail \textsc{neg-pfv-auto}-come.out until \textsc{dem} \textbf{be.white:}\textsc{fact} \\
\glt Among the birds, the sparrow is tiny and cute. Although it is small it is very intelligent. Its eyes are surrounded by black (feathers), and above that there are some red (dots). Its belly is white from the throat until the tail. (Sparrow, 2-7)
\end{exe}

 Dynamic verbs (such as \ipa{tú-wɣ-fskɤr}  \textsc{ipfv-inv}-surround  it surrounds it' and \ipa{pjɯ-ʑe} \textsc{ipfv}-begin[III] `it begins' in the example) are often in the imperfective rather than the factual in such as context.

\begin{exe}
\ex \label{ex:mAmtshama}
\gll
\ipa{kukukuku kukukuku} 	\ipa{tu-ti.} 	\ipa{nɯ} 	\ipa{tu-ti} 	\ipa{ɲɯ-ŋu} 	\ipa{tɕe,} 	\ipa{ɯ-skɤt} 	\ipa{mɯ́j-wxti} 	\ipa{kʰi.} 	\ipa{a-pɯ-ŋu,} \ipa{\textbf{mɤ-mtsʰam-a}} 	\ipa{woma} \\
{ } \textsc{ipfv}-say \textsc{dem}  \textsc{ipfv}-say \textsc{testim}-be \textsc{lnk}  \textsc{3sg.poss}-voice \textsc{neg:testim}-be.big \textsc{hearsay} \textsc{irr-ipfv}-be \textsc{neg}-\textbf{hear}:\textsc{fact-1sg} \textsc{sfp} \\
\glt (The snowcock) calls `kukuku', it calls like that. It is said that its voice is not big. Maybe it is like that, I can't hear (well) anyway. (Snowcock, 30-32)
\end{exe}

\ipa{tɕe}  	\ipa{nɤkinɯ}  	\ipa{ɯ-ku}  	\ipa{kɯ-mpɯ}  	\ipa{nɯ}  	\ipa{ɲɯ́-wɣ-phɯt}  	\ipa{tɕe,}  \ipa{nɯŋa}  	\ipa{ra}  	\ipa{kɯ}  	\ipa{ndza-nɯ,}  	\ipa{paʁ}  	\ipa{kɯ}  	\ipa{mɤ-ndze}  
(Vine, 20)



Second, with dynamic verbs, the factual expresses an assertion or an intention (example \ref{ex:kWsAthu.Cea}. 

\begin{exe}
\ex \label{ex:kWsAthu.Cea}
\gll
\ipa{aʑo}  	\ipa{kɯ-sɤ-thu}  	\ipa{ɕe-a}  \\
\textsc{1sg} \textsc{nmlz}:S/A-\textsc{antipass:human}-ask go:\textsc{fact}-\textsc{1sg} \\
\glt I am going to ask (a girl in marriage). (Kunbzang, 5)
\end{exe}

 \subsection{Sensory }
The testimonial  is built by combining stem III or stem I with the prefix \ipa{ɲɯ--} `towards east'.  For verbs whose intrinsic direction is `west', it is identical to the imperfective in the case of verbs. For instance \ipa{ɲɯ-ɤ<nɯ>ɣro} can either be analysed as \textsc{testim-<auto>}play or \textsc{ipfv-<auto>}play (both translatable as `he is playing').

surprise:`mirative'

\citet{delancey97mirative}

\citet{hill12mirativity}


tɕe nɯ "ama, kɯki aʑo ki ŋotɕu ɲɯ-rɤʑi-a ɲɯ-ŋu?" to-ti

\subsection{Egophoric present }
The egophoric present  is built by prefixing \ipa{ku--} `towards east' to the stem III or stem I of the verb. It is thus homophonous with the imperfective in the case of verbs whose intrinsic direction is `east'.  For instance the form \ipa{ku-rɤʑi} is ambiguous between \textsc{prs}-remain and \textsc{ipfv}-remain (both could be translated as `he is there' or he is remaining there' depending on the context).

The present  is rather rare in narratives, but very common in conversations. Like the factual, it is used to express intimate knowledge of an event or state on the part of the speaker, not resulting from guess or recent information mediated through the senses. However,  it cannot express a general or gnomic state of affair, it is only used to refer to an ongoing state or action. It appears mainly in second person form in questions (examples \ref{ex:WkutWscitnW} and \ref{ex:kutWnAme}) and in first person form in declarative sentences (as in the first clause in \ref{ex:kusciti}, the answer to the question in \ref{ex:WkutWscitnW}).  

\begin{exe}
\ex \label{ex:WkutWscitnW}
\gll  \ipa{`a-ʁi} 			\ipa{ɯ-kú-tɯ-scit-nɯ?}' 	\ipa{ra} 	\ipa{to-ti,} \\
\textsc{1sg.poss}-younger.sibling  \textsc{interrog-prs-}2-be.happy-\textsc{pl} \textsc{pl} \textsc{evd}-say \\
\glt She said: `Are you_p happy, my sister?' (The frog 2002, 121)
\end{exe}

\begin{exe}
\ex \label{ex:kutWnAme}
\gll \ipa{nɯtɕu}  \ipa{tɕʰi} \ipa{ku-tɯ-nɤme?}\\
what there \textsc{prs}-2-work[III] \\
\glt What are you doing there? (The smart rabbit 2012, 8)
\end{exe}

\begin{exe}
\ex \label{ex:kusciti}
\gll
\ipa{tɕʰeme} 	\ipa{nɯ} 	\ipa{kɯ} 	\ipa{`wuma} 	\ipa{ʑo} 	\ipa{ku-scit-i,} \ipa{rɟɤlpu} 	\ipa{ri} 	\ipa{a-taʁ} 	\ipa{wuma} 	\ipa{ku-sna} \ipa{ʁjoʁ} 	\ipa{ra} 	\ipa{ri} 	\ipa{wuma} 	\ipa{ʑo} 	\ipa{ku-pe-nɯ'} \ipa{to-ti,} \\
girl \textsc{dem} \textsc{erg} really \textsc{emph} \textsc{prs}-be.happy-\textsc{1pl}  roi also \textsc{1sg}-on really \textsc{prs}-be.kind servant \textsc{pl} also really \textsc{emph} \textsc{emph} \textsc{prs}-be.good \\
\glt The girl said `We are very happy, the king is very kind with me, the servants are very nice.'
(The frog 2002, 122-4)
\end{exe}

It is used also in either declarative (example \ref{ex:kusciti}) or  interrogative sentences (\ref{ex:Wkupe} and \ref{ex:WkudAn}) to refer to third persons, in the case of things belonging to or persons attached to the speaker (in declarative sentences) or the addressee (in the case of interrogatives). In example \ref{ex:kusciti}, the  use of present factual is motivated by the facts that (1) the speaker is affected by the state of the persons she refers to (2) these persons are member or her household (her husband (the king) and her servant).

\begin{exe}
\ex \label{ex:Wkupe}
\gll \ipa{nɤ-tɕɯ} \ipa{ɯ-kú-pe?}\\
\textsc{2sg.poss}-son \textsc{interrog-prs}-be.good\\
\glt Is your son well? (2014.08 conversation, Dpalcan)
\end{exe}

\begin{exe}
\ex \label{ex:WkudAn}
\gll \ipa{nɤ-kɤ-nɤma} 	\ipa{ɯ-kú-dɤn?}  \\
 \textsc{2sg.poss-nmlz:P}-work \textsc{interrog-prs}-be.many \\
\glt Do you have a lot of work? (2014.10 conversation, Chenzhen)
\end{exe}
 
 
As seen above, a question in the present factual expects an answer in the same form. However, answering in the periphrastic imperfective is also possible in the case of dynamic verbs, as in \ref{ex:pjWrAGrWa}, the answer to \ref{ex:kutWnAme}.\footnote{In another version of the same story by the same speaker hwoever, the presetn factual is also found in the answer.}
 \begin{exe}
\ex \label{ex:pjWrAGrWa}
\gll
 \ipa{a-pi} 	\ipa{kɯrtsɤɣ} 	\ipa{ma-tɯ-ɤrju} 	\ipa{ma,} 	\ipa{maka} 	\ipa{aʑo} 	\ipa{a-xtu} 	\ipa{ɯ-tɯ-mŋɤm} 	\ipa{ɲɯ-sɤre} 	\ipa{ʑo} 	\ipa{tɕe,} 	\ipa{kukutɕu} 	\ipa{pjɯ-rɤɣrɯ-a} 	\ipa{ŋu} 	\\
 \textsc{1sg.poss}-elder.sibling leopard \textsc{neg:imp}-2-say because at.all \textsc{1sg} \textsc{1sg.poss}-belly \textsc{3sg-nmlz:degree}-hurt \textsc{testim}-be.funny \textsc{emph} \textsc{lnk} here \textsc{ipfv}-treat.with.heat-\textsc{1sg} be:\textsc{fact}  \\
\glt Brother leopard, don't talk, my belly hurts terribly, and I am treating it. (The smart rabbit 2012, 10-2)
 \end{exe}
 
 In present contexts, the present factual is exactly opposed to the testimonial, to express intimate knowledge vs new and not fully assimilated knowledge. 


\subsection{Imperfective}
The imperfective form is mainly used in periphrastic TAM categories and in clause linking, and rarely appears 

use with \ipa{ɕɯŋgɯ} \citet{jacques14linking}

\subsection{Perfective}

past transitive \ipa{--t} suffix 

permansive
\subsection{Evidential}

\subsection{Past imperfective}

\subsection{Irrealis}

\subsection{Imperative}

\section{Other TAM morphemes}
\subsection{Progressive}

Japhug has a progressive prefix \ipa{asɯ--} restricted to transitive verbs. It is not used on its own: it always appears in combination with one of the TAM categories described in section \ref{sec:directional}. Due to semantic mismatch, it does not appear with perfective and evidential forms.


Examples have only been found with the following five categories: past imperfective (example \ref{ex:pasWfCAtndZi}), evidential imperfective (\ref{ex:pjAkAsWtsxWBci}), testimonial (\ref{ex:YAznAthWthu}), present (\ref{ex:kosWBzjoz}), factual (\ref{ex:asWndo}) and plain imperfective. 

Two of these TAM categories, namely past imperfective (prefixed in \ipa{pɯ--}) and evidential imperfective (\ipa{pjɤ--}), are not compatible on their own with most transitive verbs in Japhug, as suggested by \citet{lin11direction} (except for tropative verbs, see \citealt{jacques13tropative}). For all non-tropative transitive verbs, past imperfective and past evidential imperfective can only be build in combination with the progressive \ipa{asɯ--}. 

The progressive prefix is optional with the other four categories. Used with the testimonial, plain imperfective and present, the progressive excludes habitual or generic interpretations of these tense. With the factual, it also excludes future interpretation.


The progressive \ipa{asɯ--} prefix presents four morphological peculiarities: allomorphy, combination with the evidential circumfix, loss of morphological transitivity and infixation of the inverse prefix.


\subsubsection{Allomorphy} \label{sec:prog.allomorphy}
Depending on the environment, the progressive prefix has six allomorphs: \ipa{asɯ--}, \ipa{az--}, \ipa{ɤsɯ--}, \ipa{ɤz--}, \ipa{osɯ--} and \ipa{oz--}. The allomorphs   \ipa{az--} / \ipa{ɤz--} / \ipa{oz--} occur when preceding a sonorant initial prefix (example \ref{ex:YAznAthWthu}). The allomorphs \ipa{asɯ--} and \ipa{az--} occur in word-initial position and following the past imperfective prefix \ipa{pɯ--} (examples \ref{ex:pasWfCAtndZi} and \ref{ex:asWndo}). The allomorphs \ipa{osɯ--} and \ipa{oz--} result from fusion with a preceding prefix whose main vowel in \ipa{u} (example \ref{ex:kosWBzjoz}).\footnote{In this example, despite what may transpire from the translation, \ipa{βzjoz} `study' is transitive; its P is the noun <chuzhong> `Junior High School'. It is a calque from Chinese \zh{读初中} \ipa{dú chūzhōng}.} The allomorphs \ipa{ɤsɯ--} and \ipa{ɤz--} are found in all other contexts. Note that verbs whose stem begins in \ipa{a--} present exactly the same vowel alternations (see \citealt{jacques07passif}; these verbs include passive, reflexive, some denominal verbs and a few others; all are intransitive verbs).


\begin{exe}
\ex \label{ex:pasWfCAtndZi}
\gll \ipa{pɯ-asɯ-fɕɤt-ndʑi} 	\ipa{nɯ} 	\ipa{ra,} 	\ipa{zlawawozɤr} 	\ipa{nɯ} 	\ipa{kɯ} 	\ipa{pjɤ-mtsʰɤm}\\
\textsc{pst.ipfv-prog}-tell-\textsc{du} \textsc{dem} \textsc{pl}  Zlaba.Wodzer \textsc{dem} \textsc{erg} \textsc{evd}-hear\\
\glt Zlaba Wodzer heard what they were saying. (Nyimawodzer1, 32)
\end{exe}

\begin{exe}
\ex \label{ex:kosWBzjoz}
\gll \ipa{akɯ} <xianzhong> \ipa{ri} <chuzhong> \ipa{ku-osɯ-βzjoz}. \\
east district.high.school \textsc{loc} high.school \textsc{pres-prog}-learn \\
\glt She is reading Junior High School at the District High school, east of here. (Relatives 363-4)
\end{exe}

 \begin{exe}
\ex \label{ex:YAznAthWthu}
\gll
\ipa{tɤrɣe}  	\ipa{ɯ-cʰɯ-z-ɣɯri}  	\ipa{ɲɯ-ɤz-nɤtʰɯtʰu}  	 \\
pearl \textsc{3sg-ipfv:downstream-nmlz:oblique}-thread.a.needle \textsc{testim-prog}-ask.everywhere \\
\glt He is asking everywhere about (where) the thing used to thread needle is. (Conversation \ipa{taʁrdo}, 72)
\end{exe}


\subsubsection{Evidential imperfective}

The evidential imperfective forms of verbs with the progressive prefix \ipa{asɯ--} follows the same pattern as verbs whose stems begins in \ipa{a--} (including passive and reflexive verbs with the \ipa{a--} prefix, see \citealt{jacques07passif}). In addition to the regular past evidential imperfective \ipa{pjɤ--} prefix, the circumfix \ipa{k--}...\ipa{--ci} is added. The first element \ipa{k--} of this circumfix occurs between the evidential imperfective prefix \ipa{pjɤ--} and the progressive prefix \ipa{ɤsɯ--}.\footnote{Note however that the suffixal element \ipa{--ci} is optional, though it is present most of the time.} This form is illustrated by examples \ref{ex:pjAkAsWtsxWBci} and \ref{ex:pjAkAsWNgaci}.

\begin{exe}
\ex \label{ex:pjAkAsWNgaci}
\gll
<lvguan>	\ipa{ɣɯ} 	\ipa{nɯ-ʁɲɤrpa} 	\ipa{nɯ} 	\ipa{kɯ} 	\ipa{tɯ-ŋga} 	\ipa{rca} 	\ipa{kɯ-mpɕɯ\textasciitilde{}mpɕɤr} 	\ipa{ʑo} 	\ipa{pjɤ-k-ɤsɯ-ŋga-ci} 	\ipa{tɕe,} 	\ipa{kɯm} 	\ipa{nɯ} 	\ipa{tɕu} 	\ipa{pjɤ-k-ɤmdzɯ-ci.} 	\\
hotel \textsc{gen} \textsc{3pl.poss}-manager \textsc{dem} \textsc{erg} \textsc{indef.poss}-clothes \textsc{emph} \textsc{nmlz:S/A-emph}\textasciitilde{}be.beautiful \textsc{emph} \textsc{evd.ipfv-evd-prog}-wear-\textsc{evd} \textsc{lnk} door \textsc{dem} \textsc{loc} \textsc{evd.ipfv-evd}-sit-\textsc{evd}
\\
\glt The hotel manager was wearing nice clothes and sitting near the door. (The thief and the landlord)
\end{exe}

\subsubsection{Transitivity}

Verb forms with the prefix \ipa{asɯ--} lack two of the obligatory transitive markers found in Japhug verbs, namely stem 3 alternation and past tense transitive \ipa{--t--} suffix. Only stem alternation is discussed here.

Japhug verbs exhibit stem alternation in non-past TAM categories (testimonial, present, imperative, irrealis, imperfective and factual) in direct singular A forms (\textsc{1/2/3sg}$\rightarrow$3). Following \citet{jackson00puxi}, we refer to this stem as `stem 3' (stem 1 being the base stem, and stem 2 the past stem). The use of this stem is illustrated in example \ref{ex:YWndAm}, where the verb \ipa{ndo} `hold' in the imperfective form has stem 3 \ipa{ndɤm}.

\begin{exe}
\ex \label{ex:YWndAm}
\gll \ipa{kɤ-kɤ-sɯ-ɕke} 	\ipa{ɯ-mdoʁ} 	\ipa{kɯ-fse} 	\ipa{ɲɯ-ndɤm} 		\ipa{ŋu} \\
\textsc{pfv-nmlz:P-caus}-burn \textsc{3sg.poss}-colour \textsc{nmlz:S/A}-be.like \textsc{ipfv}-hold[III] be:\textsc{fact} \\
\glt  It has the colour of something that has been burnt. (\ipa{ɲɤβrɯɣ}, 14)
\end{exe}

When a verb in non-past TAM forms takes the \ipa{asɯ--}, stem alternation does not occur. Examples \ref{ex:asWndo} and \ref{ex:YAsWndo}, in factual and testimonial form have the base stem \ipa{ndo} instead of stem 3 \ipa{ndɤm} as expected in forms without the progressive.


\begin{exe}
\ex \label{ex:asWndo}
\gll
\ipa{sɯjno} 	\ipa{ɯ-mdoʁ} 	\ipa{ʑo} 	\ipa{asɯ-ndo.} \\
grass \textsc{3sg.poss}-colour \textsc{emph} \textsc{prog}-hold:\textsc{fact} \\
\glt It has the colour of grass. (Caterpillar, 69)
\end{exe}


\begin{exe}
\ex \label{ex:YAsWndo}
\gll
\ipa{kɯki} 	\ipa{ɯ-mdoʁ} 	\ipa{tsa} 	\ipa{ɲɯ-ɤsɯ-ndo} \\
this  \textsc{3sg.poss}-colour  a.little \textsc{testim-prog}-hold \\
\glt It has a colour a bit like this one. (Slugs, 159)
\end{exe}

However, adding the progressive \ipa{asɯ--} has no effect on flagging: the A still receives ergative \ipa{kɯ} marking, as shown by example \ref{ex:pjAkAsWtsxWBci}.

\begin{exe}
\ex \label{ex:pjAkAsWtsxWBci}
\gll
\ipa{rgɤnmɯ}  	\ipa{nɯ}  	\ipa{kɯ}  	\ipa{li}  	\ipa{iɕqʰa}  	<yuwang>	\ipa{nɯ}  	\ipa{pjɤ-k-ɤsɯ-tʂɯβ-ci}  		\\
old.woman \textsc{dem} \textsc{erg} again the.aforementioned net \textsc{dem} \textsc{evd.ipfv-evd-prog}-sew-\textsc{evd} \\
\glt The old woman was sewing the nets as before. (The fisherman and his wife, 284)
\end{exe}


In the related Tshobdun language, the cognate prefix \ipa{ɐsɐ--} has a similar effect, and is labelled by \citet{jackson03caodeng} as `low transitivity progressive'.

\subsubsection{Infixation of the inverse prefix}
The inverse prefix \ipa{wɣ--}, whose morphosyntactic function is described in \citet{jacques10inverse}, appears in the same prefixal slot as the progressive \ipa{asɯ--} in the Japhug verbal template, and is actually infixed within this prefix, resulting in \ipa{ɤ́<wɣ>sɯ--} [\ipa{óɣsɯ}] or \ipa{ɤ́<wɣ>z--} [\ipa{óɣz}]. The form \ipa{pjɤ-k-ɤ́<wɣ>z-nɤjo-ci} `he was waiting for him' in \ref{ex:pjAkAwGznAjoci} is the only example of this combination in the corpus.


\begin{exe}
\ex \label{ex:pjAkAwGznAjoci}
\gll
\ipa{tɕe} 	\ipa{pjɤ-ɣi} 	\ipa{tɕe} 	\ipa{qala} 	\ipa{kɯ} 	\ipa{pjɤ-k-ɤ́<wɣ>z-nɤjo-ci} 	\ipa{tɕe,} \\
\textsc{lnk} \textsc{evd:down}-come \textsc{lnk} rabbit \textsc{erg} \textsc{evd.ipfv-evd<inv>}-wait-\textsc{evd} \textsc{lnk} \\
\glt (The leopard) came down, and the rabbit was waiting for him there. (The smart rabbit.2014, 60)
\end{exe}

It is however possible to elicitate other forms of this type, such as \ref{ex:YWtAwGsWzgroR}, without any constraint.

\begin{exe}
\ex  \label{ex:YWtAwGsWzgroR}
\gll  \ipa{ɲɯ-tɯ-ɤ́<wɣ>sɯ-zgroʁ}    \\
\textsc{testim-2-prog<inv>}-attach \\
\glt He is attaching you. (elicitation, Chen Zhen)
\end{exe}

\subsection{Conative}
\subsection{Hearsay}
\ipa{kʰi}

  tɕeri, pɣɤkhɯ nɯ kɯ qaɲi kɤ-sat wuma ʑo cha khi.
  
  
  tɕe nɯ ɯ-kɯ-sat stu kɯ-cha nɯ pɣɤkhɯ ɲɯ-ŋu khi.
  
  tɕeri ʁloŋbutɕhi nɯ ʁnɯtɯphɯ ɣɤʑu khi
  
    \section{Periphrastic TAM categories}
  
   (\ipa{ŋu} `be' and \ipa{maʁ} `not be').
   
  \subsection{Periphrastic imperfective} {sec:periphrastic.ipfv}
  \subsection{Periphrastic past imperfective } {sec:periphrastic.pst}
  
    \subsection{Periphrastic evidential imperfective } {sec:periphrastic.evd}
    
        \subsection{Periphrastic conative } \label{sec:periphrastic.fact}
        
        
  \section{Focus}
  
  
\bibliographystyle{linquiry2}
\bibliography{bibliogj}
\end{document}