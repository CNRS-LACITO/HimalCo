\documentclass[oldfontcommands,oneside,a4paper,11pt]{article} 
\usepackage{fontspec}
\usepackage{natbib}
\usepackage{booktabs}
\usepackage{xltxtra} 
\usepackage{polyglossia} 
\usepackage[table]{xcolor}
\usepackage{gb4e} 
\usepackage{multicol}
\usepackage{graphicx}
\usepackage{float}
\usepackage{lineno}
\usepackage{textcomp}
\usepackage{hyperref} 
\hypersetup{bookmarks=false,bookmarksnumbered,bookmarksopenlevel=5,bookmarksdepth=5,xetex,colorlinks=true,linkcolor=blue,citecolor=blue}
\usepackage[all]{hypcap}
\usepackage{memhfixc}
\usepackage{lscape}
 

%\setmainfont[Mapping=tex-text,Numbers=OldStyle,Ligatures=Common]{Charis SIL} 
\newfontfamily\phon[Mapping=tex-text,Ligatures=Common,Scale=MatchLowercase,FakeSlant=0.3]{Charis SIL} 
\newcommand{\ipa}[1]{{\phon#1}} %API tjs en italique
 
\newcommand{\grise}[1]{\cellcolor{lightgray}\textbf{#1}}
\newfontfamily\cn[Mapping=tex-text,Ligatures=Common,Scale=MatchUppercase]{MingLiU}%pour le chinois
\newcommand{\zh}[1]{{\cn #1}}
\newcommand{\topic}{\textsc{dem}}
\newcommand{\tete}{\textsuperscript{\textsc{head}}}
\newcommand{\rc}{\textsubscript{\textsc{rc}}}
\XeTeXlinebreaklocale 'zh' %使用中文换行
\XeTeXlinebreakskip = 0pt plus 1pt %
 %CIRCG
\newcommand{\ro}{$\Sigma$}
\newcommand{\siga}{$\Sigma_1$} 
\newcommand{\sigc}{$\Sigma_3$}   


\begin{document} 

\title{A counterexample to Watkins' law in Siouan}
\author{Guillaume Jacques}
\maketitle

\section{Introduction}
 
    \citealt{watkins62celtic}'s law 
    
 

%\citet{csd2006}

\citet{kim01venit}

Unfortunately, most of the work on comparative Siouan is unpublished (though available from the Siouan Archive, which is freely distributed to all interested scholars), so that this paper will first present the accepted knowledge on Siouan reconstruction, mainly based on the model in the Comparative Siouan Dictionary (\citealt{csd2006}).

\citet{eschenberg05omaha}
 

\section{Reflexes of simple *r, *R and *y in Dakotan and Dhegiha}
There is evidence in Mississippi valley Siouan languages of at least three distinct proto-phonemes *r, *R and *y before oral vowels. These phonemes merge as \ipa{r} in Chiwere and Winnebago, but are kept partially distinct in Dakotan and Dhegiha, according to the   correspondences in   Table \ref{tab:basic} (correspondences before nasal vowels are not indicated). 

These correspondences are valid word-initially or between vowels, except those involving the clusters *\ipa{ky--} and \ipa{wy--} (see below). In Kansa and Osage, the outcome of *\ipa{R} are palatalized as \ipa{ǰ} and \ipa{c} respectively before \ipa{e} (merging with *\ipa{ht}).

\begin{table}[h]
\caption{Basic correspondences of proto-Siouan *r, *R and *y in Dakotan and Dhegiha languages} \label{tab:basic} \centering
\begin{tabular}{l|ll|lllll}
\toprule
  & 	Lakhota  & 	Yankton/Sisseton & 	 Omaha  & 	 Kansa  & 	 Osage  & 	 Quapaw  &	\\	
  \midrule
\ipa{*y} & 	\ipa{čh} & 	\ipa{čh} & 	\ipa{ž} & 	\ipa{ž} & 	\ipa{ž} & 	\ipa{ž} & 	\\	
\ipa{*r} & 	\ipa{y} & 	\ipa{y} & 	\ipa{ð} & 	\ipa{y} & 	\ipa{ð} & 	\ipa{d} & 	\\	
\ipa{*R } & 	\ipa{l} & 	\ipa{d} & 	\ipa{n} & 	\ipa{d / ǰ} & 	\ipa{t / c} & 	\ipa{t} & 	\\	
\ipa{*r / i\_} & 	\ipa{čh} & 	\ipa{čh} & 	\ipa{ð} & 	\ipa{y} & 	\ipa{ð} & 	\ipa{d} & 	\\	
\midrule
\ipa{*kr} & 	\ipa{gl} & 	\ipa{hd} & 	\ipa{gð} & 	\ipa{l} & 	\ipa{l} & 	\ipa{kd} & 	\\		
\ipa{*xr} & 	\ipa{ȟl} & 	\ipa{hd} & 	\ipa{xð} & 	\ipa{xl} & 	? & 	? & 	\\		
\ipa{*wr} & 	\ipa{bl} & 	\ipa{md} & 	\ipa{bð} & 	\ipa{bl} & 	\ipa{br} & 	\ipa{bd} & 	\\		
\midrule
\ipa{*ky} & 	\ipa{čh / kč} & 	\ipa{čh / kč} & 	\ipa{gð} & 	\ipa{l} & 	\ipa{l} & 	\ipa{kd} & 	\\	
\ipa{*xy} & 	\ipa{ȟč} & 	\ipa{ȟč} & 	\ipa{xð} & 	\ipa{xl} & 	? & 	? & 	\\	
\ipa{*wy} & 	\ipa{čh / pč} & 	\ipa{čh / pč} & 	\ipa{bð} & 	? & 	\ipa{br} & 	? & 	\\	
\bottomrule
\end{tabular}
\end{table}

\begin{table}[h]
\caption{Examples of the correspondences of proto-Siouan *r, *R and *y in Dakotan and Dhegiha languages} \label{tab:basic.ex} \centering
 \resizebox{\columnwidth}{!}{
\begin{tabular}{ll|l|lll|ll}
\toprule
&proto-Siouan &Lakhota    & Omaha & Kansa & Osage   \\
	\midrule	
\ipa{*y} &	\ipa{*yo\_phe} &	\ipa{čho\_pȟÁ} &		\ipa{žúhe} &	\ipa{žophé} &	\ipa{žó\_pše} &		wade &	\\
&	\ipa{*yá•pE} &	\ipa{čhápa} &		\ipa{žábe} &	\ipa{žábe} &	\ipa{žápe} &		beaver &	\\
&	\ipa{*yá•že} &	\ipa{čhažé} &		\ipa{žáže} &	\ipa{žá•že} &	\ipa{žáže} &		name &	\\
&	\ipa{*yeká} &	\ipa{čhečá} &		\ipa{žegá} &	\ipa{žegá} &	\ipa{žeká} &		thigh, leg &	\\
&	\ipa{*yá•phe} &	\ipa{čha\_pȟÁ} &		\ipa{žáhe} &	\ipa{žáphe} &	\ipa{} &		stab &	\\
&	\ipa{*e/aya-ʔį} &	\ipa{ečhíŋ, ečháŋmi} &		\ipa{éžǫmį} &	\ipa{ažį́, ažamį} &	\ipa{ažį́, ážamį} &		think &	\\
	\midrule	
\ipa{*r-} &	\ipa{*ra-} &	\ipa{ya-} &		\ipa{ða-} &	\ipa{ya-} &	\ipa{ða-} &		by mouth &	\\	
&	\ipa{*ru-} &	\ipa{yu-} &		\ipa{ði-} &	\ipa{yü-} &	\ipa{ðu-} &		by hand &	\\	
&	\ipa{*xurá} &	\ipa{ȟuyá} &		\ipa{xiðá} &	\ipa{xüyá} &	\ipa{xúða} &		eagle &	\\	
&	\ipa{*rÉ} &	\ipa{yA´} &		\ipa{ðé} &	\ipa{yé} &	\ipa{(a)ðée} &		go &	\\	
	\midrule

 \ipa{*R} &	\ipa{*Ró•te} &	\ipa{loté} &		\ipa{núde} &	\ipa{dó•ǰe} &	\ipa{tóoce} &		throat &	\\
&	\ipa{*(i-)Rekší} &	\ipa{lekší} &		\ipa{inégi} &	\ipa{iǰégi} &	\ipa{įcéki} &		MB &	\\
&	\ipa{*Rá} &	\ipa{lá} &		\ipa{wana+} &	\ipa{da} &	\ipa{tá} &		ask &	\\
&	\ipa{*Ré•že} &	\ipa{léžA} &		\ipa{néže} &	\ipa{ǰéže} &	\ipa{céže} &		urine &	\\
&	\ipa{*Ré} &	\ipa{o\_lé} &		\ipa{u\_né} &	\ipa{o\_ǰé} &	\ipa{o\_cé} &		hunt, look for &	\\

\bottomrule
\end{tabular}}
\end{table}
Table \ref{tab:basic.ex} illustrates the first three correspondences; in this table and other Tables below, only Lakhota, Omaha, Kansa and Osage are presented. The cognates sets are from \citet{csd2006}, but the Lakhota, Osage  data were rechecked against     \citet{ullrich08} and \citet{quintero10osage} respectively and adapted to the spellings in these dictionaries (to ease data rechecking). The Omaha data was rechecked against the Dorsey text corpus (\citealt{dorsey1890cegiha}; the forms not found in the corpus are indicated by the symbol +, and are taken directly from the comparative dictionary).
 

In some possessed nouns (especially body parts), we find a special correspondence with \ipa{čh} in Dakotan against \ipa{ð} : \ipa{y} : \ipa{ð} : \ipa{d} in Dhegiha, as in Table \ref{tab:ir}. This correspondence is interpreted as the effect of paradigm levelling, due to a sound change *r-- $\rightarrow$ \ipa{čh} before the third person possessive prefix \ipa{i--}, even before nasal vowels (see \citealt{rankin05quapaw}, \citealt{csd2006}). It is never attested in verbs, even those with a locative \ipa {í--} prefix; for instance *\ipa{íre} regularly yields  \ipa{íðe} 'see, find' in Omaha and \ipa{iyé\_ya} ‘find' in Lakota (not *\ipa{ičhé}).

\begin{table}[h]
\caption{Proto-Siouan *r after the possessive *i-- prefix} \label{tab:ir} \centering
 \resizebox{\columnwidth}{!}{
\begin{tabular}{ll|l|lll|ll}
\toprule
&proto-Siouan &Lakhota  &   Omaha & Kansa & Osage   \\
	\midrule	
&	\ipa{*i-ré•ži} &	\ipa{čheží} &		\ipa{teðéze} &	\ipa{yéze} &	\ipa{ðéeze} &		tongue &	\\
&	\ipa{*i-rá•ɣu} &	\ipa{čhaǧú} &		\ipa{ðǫ́xį+} &	\ipa{yáxü} &	\ipa{ðǫɣu} &		lung &	\\
&	\ipa{*i-ráke} &	\ipa{čhaká} &		\ipa{} &	\ipa{ho yáge} &	\ipa{} &		palate / gills &	\\
&	\ipa{*i-rą́•h-ka} &	\ipa{čhaŋkȟáhu} &		\ipa{nǫ́kka} &	\ipa{ną́kka} &	\ipa{ną́hka} &		spine &	\\
\bottomrule
\end{tabular}}
\end{table}




Before nasal vowels, *r and *R merge as \ipa{n} in Dakotan and Dhegiha languages in nearly all contexts (see  \citealt{michaud-jacques12nasalite}), but this topic will not be discussed in the present paper, as it has no incidence on the issue at hand.

In addition, there is a residue of words showing irregular correspondences, such as   `mosquito' (Lakota \ipa{čhaphų́ka}, Omaha \ipa{náhǫga}, Kansa \ipa{yáphąįge}, Osage \ipa{yáphąįge}, Quapaw \ipa{daphąke} ), but these case are most likely Wanderwörter and do not reflect the correspondences of inherited words. Therefore, they are not taken into account here.

\section{Reflexes of clusters}
There were some clusters with *\ipa{r} or *\ipa{y} as second element in proto-Mississippi Valley Siouan (the contrast between  *\ipa{r} and *\ipa{R} was neutralized). Examples of *\ipa{wr--}, \ipa{kr--},  \ipa{sr--}, *\ipa{šr--} and *\ipa{xr--} are plentiful, as can be illustrated by the data in Table \ref{tab:kr} (many other additional examples could be added).


 \begin{table}[h]
\caption{Proto-Siouan *\ipa{kr--}, *\ipa{xr--} and *\ipa{wr--}} \label{tab:kr} \centering
 \resizebox{\columnwidth}{!}{
\begin{tabular}{ll|l|lll|ll}
\toprule
&proto-Siouan &Lakhota    & Omaha & Kansa & Osage     \\
	\midrule	
*\ipa{kr--}&	\ipa{*kré•pE} &	\ipa{glépA} &		\ipa{gðébe} &	\ipa{lébe} &	\ipa{} &		vomit &	\\
&	\ipa{*krézE} &	\ipa{glézA} &		\ipa{gðéže} &	\ipa{léze} &	\ipa{léze} &		spotted &	\\
&	\ipa{*krá} &	\ipa{gla-} &		\ipa{ðigða+} &	\ipa{layá} &	\ipa{} &		untie &	\\
 	\midrule	
*\ipa{xr--}&	\ipa{*xroke} &	\ipa{ȟlókA} &		\ipa{xðúge} &	\ipa{xlóge} &	\ipa{} &		hole, nostrils &	\\
&	\ipa{*xro} &	\ipa{ȟló} &		\ipa{xðúde} &	\ipa{} &	\ipa{} &		snort &	\\
	\midrule	
*\ipa{wr--}	&	\ipa{*awró} &	\ipa{abló} &		\ipa{} &	\ipa{abló} &	\ipa{ábro} &		shoulder &	\\
&	\ipa{*hąwre} &	\ipa{haŋblé} &		\ipa{hǫ́bðe} &	\ipa{hąblé} &	\ipa{hǫ́bre} &		dream &	\\
&	\ipa{wráska} &	\ipa{blaská} &		\ipa{bðáska} &	\ipa{bláska} &	\ipa{bráaska} &		flat &	\\
&	\ipa{*wréh-(ka)} &	\ipa{blečá-} &		\ipa{bðékka} &	\ipa{blékka} &	\ipa{bréhka} &		thin &	\\
	\bottomrule
\end{tabular}}
\end{table}

Examples of clusters with \ipa{y} as a second element are much more restricted, and   Table  \ref{tab:ky} contains all known examples (from \citet{koontz83rstem}  and \citet{csd2006}).

 \begin{table}[h]
\caption{Proto-Siouan *\ipa{ky--}, *\ipa{xy--} and *\ipa{wy--}} \label{tab:ky} \centering
 \resizebox{\columnwidth}{!}{
\begin{tabular}{ll|l|lll|ll}
\toprule
&proto-Siouan &Lakhota    & Omaha & Kansa & Osage     \\
	\midrule	
*\ipa{ky--}&	\ipa{kyé•wrą} &	\ipa{wikčémna} &		\ipa{gðéba} &	\ipa{léblą} &	\ipa{lébrą} &		ten &	\\
&	\ipa{kyetą́} &	\ipa{čhetáŋ} &		\ipa{gðedǫ́} &	\ipa{ledą́} &	\ipa{letǫ́} &		hawk &	\\
&	\ipa{*kyąšká} &	\ipa{čhaŋšká} &		\ipa{gðąšká+} &	\ipa{} &	\ipa{} &		 &	\\
&	\ipa{*rukyą} &	\ipa{i\_yúkčaŋ} &		\ipa{wa-ðígðǫ} &	\ipa{í\_yülą } &	\ipa{í\_ðilą} &		know &	\\
	\midrule	
*\ipa{xy--}	&	\ipa{*xyá} &	\ipa{ȟčá / waȟčá} &		\ipa{waxðá} &	\ipa{xlá} &	\ipa{} &		bloom, flower &	\\
	\midrule	
*\ipa{wy--}&	\ipa{*e-w-ye} &	\ipa{epčÁ} &		\ipa{ebðé} &	\ipa{} &	\ipa{ébre}  &	I think &	\\	
&	\ipa{*rąwyE} &	\ipa{napčÁ}  &	\ipa{} &	\ipa{} &	\ipa{wa-nǫ́bre} &		eat, swallow &	\\	
&	\ipa{*w-yi--} &	\ipa{čhi-}  &	\ipa{wi--} &	\ipa{} &	 	\ipa{} &		1$\rightarrow$ 2 prefix& 	\\	
\bottomrule
\end{tabular}}
\end{table}

In Dhegiha, clusters in *\ipa{Cy--} merge with their *	\ipa{Cr--} counterparts, except perhaps *	\ipa{wy--} if the 1$\rightarrow$ 2 prefix \ipa{wi--} is the regular outcome of \ipa{*w-yi--}, in which case *\ipa{wy--} perhaps present a different treatment word-initially.

In Lakhota, the cluster *\ipa{y--} element in *\ipa{Cy--} cluster changes to \ipa{č} word-internally, and the previous segment is preserved (and undergoes fortition, in the case of *\ipa{wy--}). Word-initially, *\ipa{wy--} and *\ipa{ky--} apparently merge with *\ipa{y} as 	\ipa{čh--}.\footnote{The 1$\rightarrow$ 2 prefix does not always appear word-initially, but we can safely assume that analogical pressure removed the expected \ipa{čh--} / \ipa{pč--} allomorphy.}
%wyi > chi- (wi- in O

 
%\citet{rankin02ofo}
\section{Verbal paradigms}
 
The paradigm of *\ipa{r--} initial verb stems (including verbs with the instrumental prefixes *\ipa{ra--} `by mouth' and *\ipa{ru--} `by hand') constitute a distinct conjugation class in MVS languages. The paradigms in these languages are provided in Table \ref{tab:go}. The first and second person forms result from vowel-less allomorphs *\ipa{w--} and *\ipa{y--} of the first person *\ipa{wa--} and second person  \ipa{ya--} prefixes.\footnote{The reconstruction of the second person prefix is a complex problem: Mississippi Valley Siouan languages point to *\ipa{ra--}, while Biloxi and Ofo suggest a reconstruction *\ipa{ya--} instead. This issue is in any case irrelevant to the present  topic, as we focus exclusively on MVS. } The *\ipa{y} become a fricative \ipa{*š}  in this context in all languages except Dakotan.

\begin{table}[h]
\caption{Paradigm of the verb *\ipa{re} `go' in Siouan languages} \label{tab:go}
 \resizebox{\columnwidth}{!}{
\begin{tabular}{ll|l|ll|llll|ll}
\toprule
 &	 &	Lakhota &	Winnebago &	Chiwere &	Omaha &	Osage &	Kansa &	Quapaw &	Ofo &	\\	
  \midrule
1sg.A &	\ipa{*wr-} &	\ipa{blÁ} &	\ipa{tée} &	\ipa{hajé} &	\ipa{bðe} &	\ipa{brée} &	\ipa{bne} &	\ipa{bde} &	\ipa{até˙kna} &	\\	
2sg.A &	\ipa{*šr-} <\ipa{*yr-}  &	\ipa{lÁ} &	\ipa{šeré} &	\ipa{slé} &	\ipa{šne} &	\ipa{šcée} &	\ipa{hne} &	\ipa{tte} &	\ipa{šté˙kna} &	\\	
base &	\ipa{*r-} &	\ipa{yÁ} &	\ipa{rée} &	\ipa{lé} &	\ipa{ðé} &	\ipa{ðée} &	\ipa{yé} &	\ipa{dé} &	\ipa{té˙kna} &	\\	
\bottomrule			

\end{tabular}}
\end{table} 
 
 In Omaha and other Dhegiha languages, the verb \ipa{ebðé}, \ipa{éðe} `think' follows the same conjugation, as shown in Table \ref{tab:think}. However, its Lakota cognate  \ipa{epčÁ} `I think', a defective verb attested only in the first singular, and belongs to the set of forms with *\ipa{wy--} in proto-Siouan.
 
 \begin{table}[h]
\caption{Paradigm of the verb *\ipa{e\_ye} `think' in Siouan languages}  \label{tab:think} \centering
\begin{tabular}{ll|l|ll|llll|ll}
\toprule
 &	 &	Lakhota &	 	Omaha &	Osage 	& Expected Omaha\\	
 \midrule
1sg.A &	*\ipa{e-w-ye} &	 \ipa{epčÁ} &   \ipa{ebðé} &\ipa{ébre} &\ipa{ebðe}\\
2sg.A &*\ipa{e-y-ye}&  &  \ipa{ešné} &? &*\ipa{eže} or *\ipa{eðe} ??? \grise{}& \\
base &	*\ipa{e-ye} &	  &\ipa{eðé} &\ipa{éðe} & *\ipa{eže}\grise{}\\
\bottomrule			
\end{tabular}
\end{table} 

 Thus, the Dhegiha paradigm for the verb \ipa{éðe} `to think' must have been partially renewed: while the first person should regularly be identical to that of a \ipa{r--} stem, according to the correspondences shown in Table \ref{tab:ky}, the the second person and third person should not. The third person form, if from *\ipa{e-ye}, should have been *\ipa{eže} instead of \ipa{eðé} in Dhegiha.

The only available explanation for this fact is analogical levelling. It is impossible to argue that \ipa{ð} is the regular outcome of *\ipa{y} in Dhegiha in intervocalic position, since clear examples of \ipa{ž} from *\ipa{y} are attested word-internally (see Table \ref{tab:basic}).

The verb *\ipa{e\_ye} `think' in Siouan was the only syncopating *\ipa{y--} initial stem, and had a unique conjugation. In Lakota it lost the second and third person forms (it is the only defective verb of this type), while in Dhegiha those forms were renewed on the basis of the first person singular.


\section{The directionality of analogy}

Mississippi Valley Siouan languages attest  a very rare type of analogy, from first person to third person, which as we saw before is almost non-attested in Indo-European languages.

An explanation for the preservation of the first person form \ipa{epčÁ} `I think' in Lakhota (and the loss of the rest of the paradigm) and the directionality of analogy in the paradigm of \ipa{ebðé}, \ipa{éðe} `think' in Dhegiha can however be provided by the relative frequency of the first person singular for verbs of cognition such as  think' and `know'.

 %(unfortunately, no example of  \ipa{epčÁ} `I think' could be found in \citealt{deloria32dakota} to confirm whether this use is attested).\citet{dorsey1890cegiha}
The Omaha corpus (\citealt{dorsey1890cegiha}) was conveniently retranscribed and corrected by Robert Rankin (in the Siouan Archive), and is therefore fully searchable. As shown in Table \ref{tab:cognition}, the most common cognition verbs meaning `think' or `know' in Omaha occur most often in the first person singular than in the third person by a considerable margin.


 \begin{table}[h]
\caption{Number of attestations of cognition verb forms in the Omaha corpus}  \label{tab:cognition} \centering
\begin{tabular}{l|ll|ll|lll}
\toprule
Person	&Think &	&	Think, suspect &	&	Know &	&	\\
 \midrule
1 & 	\ipa{ebðé(gǫ)}  & 	172 & 	\ipa{éžǫmį} & 	7 & 	\ipa{iðáppahǫ} & 	42 & 	\\
2 & 	\ipa{ehné(gǫ)}, \ipa{ené} or \ipa{ešné} & 	114 & 	 & 	0 & 	\ipa{íšpahǫ} & 	28 & 	\\
3 & 	\ipa{eðé(gǫ)} & 	22 & 	 & 	0 & 	\ipa{íbahǫ} & 	29 & 	\\
	\bottomrule			
\end{tabular}
\end{table} 

%nudoⁿhoⁿga akʰá tʼeóⁿtha-bázhi ai thóⁿshti, á-biamá

%óⁿkʰazhi, nudóⁿhoⁿga akʰá é wakʰa-bázhi ebthégoⁿ, á-bi-amá

In Omaha, as in many languages, the first person of the verb `to think'  can be used  in dialogue to present an assertion as one's point of view  as opposed to someone else's (as in  example   \ref{ex:ebdhegon}; sentence \ref{ex:no.ebdhegon} is repeated a few lines below  with the verb \ipa{ebðégǫ} `I think' added to it). 

% is particularly common in the  
 
\begin{exe}
 \ex \label{ex:ebdhegon} 
 \gll ``\ipa{ǫ́kʰaži,}  	\ipa{nudǫ́hǫga}  	\ipa{akʰá}  	\ipa{é}  	\ipa{wakʰa-báži}  	\ipa{ebðégǫ}"  	\ipa{á-bi-amá}\\
no leader \textsc{prox:sg} that mean-\textsc{neg} \textsc{1sg}:think say-\textsc{prox-evd}\\
%\end{exe}
%\begin{exe}
 \ex \label{ex:no.ebdhegon}
 \gll ``\ipa{ǫ́kʰaži,}  \ipa{ha}	 \ipa{nudǫ́hǫga}  	\ipa{akʰá}  	\ipa{é}  	\ipa{wakʰa-í}"   	\ipa{á-bi-amá}  \\
no \textsc{interj} leader \textsc{prox:sg} that mean-\textsc{prox}  say-\textsc{prox-evd} \\
\glt ``No, the leader did not mean that, I think," said the former. ``Yes, the leader did mean that," said the latter. (177:19-178.1)
\end{exe}

Alternatively, it is also used to introduce  sentences that is are necessarily strictly the speaker's point of view, but belong  to the commonly accepted knowledge of the culture, as in \ref{ex:ebdhegon2}.

\begin{exe}
 \ex \label{ex:ebdhegon2}
 \gll 
\ipa{nį́kašįga}  	\ipa{ðižúbaži}  	\ipa{kíǰi,}  	\ipa{égǫwéʼǫ}  	\ipa{góⁿðawáðe}  	\ipa{ebðégǫ.}  \\
man injure when do.this.way desirable \textsc{1sg}:think \\
\glt I think that when one man injures another, it is desirable to repay him. ( 438,  12)
  \end{exe}

First person singular forms such as \ipa{ebðégǫ}  `I think' can thus be used to represent the speaker's stance toward his own words, 
  
  
%níⁿkashiⁿga thizhúbazhi ki´ǰi, égoⁿwéʼoⁿ góⁿthawáthe ebthégoⁿ.

 
% \section{Verbs of cognition}
%This example does not disprove the validity of Watkins' law; rather, it can help to refine it. 



%generic forms are easily interpreted as being first person and vice versa. The semantic proximity between first person and generic in the case of these verbs is due to two opposite reasons. First, using a generic rather than a first person is a way for the speaker to avoid directly asserting his own point of a view. Second, a confusion between first person and generic can occur when the speaker equates what he thinks or believes with `common sense' or common knowledge.
%ehé, níⁿkashiⁿga thizhúbazhi ki´ǰi, égoⁿwéʼoⁿ góⁿthawáthe ebthégoⁿ.
%said (one). / Oho! / I said, / man / injures one / when, / to do so in return / desirable / I think. /
% I think that when one man injures another, it is desirable to repays him (438, line 12.)

%
%A example of semantic confusion between first person and generic with cognition verbs can be observed  in Japhug (Sino-Tibetan, \citealt[342]{jacques04these}, \citealt{jacques15generic}). In Japhug, the highly irregular negative generic form \ipa{mɤ-xsi} `one does not know' of the verb \ipa{sɯz} `know' is used in contexts where a first person singular (not first person plural) is  implied, and a generic interpreted should be excluded, as in examples \ref{ex:mAxsi} and \ref{ex:mAxsi2}.
%
%\begin{exe}
% \ex \label{ex:mAxsi}
% \gll
%\ipa{ma-tɯ-nɯqaɟy}  	\ipa{ma}  	\ipa{ɕu}  	\ipa{tɯ-ŋu}  	\ipa{mɤ-xsi}  	\ipa{ri} \ipa{kɯ-qarŋe}  	\ipa{thamtɕɤt}  	\ipa{nɯ}  	\ipa{rɟɤlpu}  	\ipa{ɣɯ}  	\ipa{χsɤrɲa}  	\ipa{ŋu,}  	
% \\
%\textsc{neg:imp}-2-fish because who 2-be:\textsc{fact} \textsc{neg-genr}:know but nmlz:S/A-be.yellow all \textsc{dem} king \textsc{gen} gold.fish be:\textsc{fact}\\
%\glt Don't fish; although I don't know who you are, the yellow ones are the King's gold fishes (ie, only the king can fish them, so whoever you are, you are not allowed). (Gesar, 368)
%\end{exe}
%\begin{exe}
% \ex \label{ex:mAxsi2}
% \gll
%\ipa{mɤ-xsi}  	\ipa{ko,}  	\ipa{nɯra}  	\ipa{ɲɤ-nɯ-jmɯt-a}  \\
%\textsc{neg-genr}:know \textsc{sfp} \textsc{dem:pl} \textsc{evd-auto}-forget-\textsc{1pl} \\
%\glt I don't know, I forgot those things. (Conversation, answer to the question: `How did you talk to your children when they were small babies?')
%\end{exe}
%
%Note that generic verbs forms are never used for first person singular with other verbs. Additionally, some languages have adverbs meaning `who knows' or `I don't know' (for instance Nepali \ipa{kunni}),  whose use is comparable to the Japhug generic form \ipa{mɤxsi}. 

\section{Conclusion}
Thus, in view of the Siouan data, Watkins's law can be reformulated as follows:

\begin{exe}
\ex 
\glt Analogy takes place from the most frequent form to the least frequent one (in discourse). 
\glt When the most frequent form is third person singular, it can be reinterpreted as a zero-marked form
\glt  When the most frequent form is  is first person, the other forms are remodelled by proportional analogy, by applying attested rules of correspondence between first person and other persons in a more regular paradigm.
\end{exe}
 
\bibliographystyle{linquiry2}
\bibliography{bibliogj}
\end{document}