 
\documentclass[oldfontcommands,oneside,a4paper,11pt]{article} 
\usepackage{fontspec}
\usepackage{natbib}
\usepackage{booktabs}
\usepackage{xltxtra} 
\usepackage{polyglossia} 
\usepackage[table]{xcolor}
\usepackage{tikz}
%\usetikzlibrary{shapes.geometric, arrows}
\usetikzlibrary{mindmap}
\usepackage{gb4e} 
\usepackage{multicol}
\usepackage{graphicx}
\usepackage{float}
\usepackage{hyperref} 
\hypersetup{bookmarks=false,bookmarksnumbered,bookmarksopenlevel=5,bookmarksdepth=5,xetex,colorlinks=true,linkcolor=blue,citecolor=blue}
\usepackage[all]{hypcap}
\usepackage{memhfixc}
\usepackage{lscape}
 \usepackage{lineno}
\bibpunct[: ]{(}{)}{,}{a}{}{,}
 
%\setmainfont[Mapping=tex-text,Numbers=OldStyle,Ligatures=Common]{Charis SIL} 
\newfontfamily\phon[Mapping=tex-text,Ligatures=Common,Scale=MatchLowercase,FakeSlant=0.3]{Charis SIL} 
\newcommand{\ipa}[1]{{\phon \mbox{#1}}} %API tjs en italique
 \newcommand{\ipab}[1]{{\phon \mbox{#1}}} %API tjs en italique
\newcommand{\grise}[1]{\cellcolor{lightgray}\textbf{#1}}
\newfontfamily\cn[Mapping=tex-text,Ligatures=Common,Scale=MatchUppercase]{MingLiU}%pour le chinois
\newcommand{\zh}[1]{{\cn #1}}

%\tikzstyle{process} = [rectangle, minimum width=3cm, minimum height=1cm, text centered, draw=black, fill=orange!30]
%\tikzstyle{decision} = [diamond, minimum width=3cm, minimum height=1cm, text centered, draw=black, fill=green!30]
%\tikzstyle{arrow} = [thick,->,>=stealth]

 \begin{document} 
 \title{ The spontaneous-autobenefactive prefix in Japhug Rgyalrong\footnote{Glosses follow the Leipzig glossing rules. Other abbreviations used here include: \textsc{auto} spontaneous-autobenefactive, \textsc{fact} factual/assumptive, \textsc{infr} inferential evidential, \textsc{hort} hortative, \textsc{inv} inverse, \textsc{pres} egorphoric present, \textsc{sens} sensory  evidential, \textsc{vert} vertitive. }}
 %Acknowledgements will be added after editorial decision.%Nathan W. Hill, Ken Mason, Alexis Michaud, Dmitry Nikolaev, Pavel Ozerov, Vladimir Plungian, Roland Pooth}
  
\author{Guillaume Jacques}
\maketitle
\linenumbers
\sloppy

\subsection{The Middle Domain}
%\begin{tikzpicture}[node distance=2cm] 
%
%\node (start) [startstop] {Start};
%\node (pro1) [process, below of=start] {Process 1};
%
%\end{tikzpicture}



\centering\begin{tikzpicture}[mindmap, grow cyclic, every node/.style=concept, concept color=orange!40, 
   level 1/.append style={level distance=5cm,sibling angle=90},
     %level 1 concept/.append style={level distance=130,sibling angle=30},
    level 2/.append style={level distance=3cm,sibling angle=30},]  
%  extra concept/.append style={color=blue!50,text=black}]
  \begin{scope}[mindmap, concept color=blue!30]
    \node [concept]  at (-2,6) {Reflexive \ipa{ʑɣɤ--}}
    ;
  \end{scope}
  
\begin{scope}[mindmap, concept color=purple!40]
    \node [concept] at (0,0) {Autobenefactive \ipa{nɯ--}} 
      child { node {Permansive}}
      child { node {Spontaneous}
      		child { node {Protasis of conditionals}}
      		child { node {Mistake}}
      		child { node {Against volition}}
		}
      child { node {Grooming / Possessed P}}
;
  \end{scope}
  
\begin{scope}[mindmap, concept color=orange!50]
    \node [concept] at (7,6) {Anticausative}
;
\end{scope}

\begin{scope}[mindmap, concept color=teal!50]
    \node [concept] at (5,11) {Passive \ipa{a--}}
      child [grow=-130]{ node {Reciprocal \ipa{a--} + \textsc{redp} }}
;
\end{scope}

\begin{scope}[mindmap, concept color=yellow!50]
    \node [concept] at (-3,-4) {Vertitive \ipa{nɯ--}}
;
\end{scope}


%\begin{scope}[mindmap, concept color=green!50]
%    \node [concept] at (9,15) {Antipassive     \ipa{sɤ--} / \ipa{rɤ--}} 
%      child [grow=-90] { node {Anti-experiencer}} 
%;
%\end{scope}

\end{tikzpicture}

 \end{document}
 