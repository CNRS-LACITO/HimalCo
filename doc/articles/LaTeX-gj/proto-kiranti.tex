\documentclass[oldfontcommands,oneside,a4paper,11pt]{article} 
\usepackage{fontspec}
\usepackage{natbib}
\usepackage{booktabs}
\usepackage{xltxtra} 
\usepackage{polyglossia} 
\usepackage[table]{xcolor}
\usepackage{gb4e} 
\usepackage{multicol}
\usepackage{graphicx}
\usepackage{float}
\usepackage{hyperref} 
\hypersetup{bookmarks=false,bookmarksnumbered,bookmarksopenlevel=5,bookmarksdepth=5,xetex,colorlinks=true,linkcolor=blue,citecolor=blue}
\usepackage[all]{hypcap}
\usepackage{memhfixc}
\usepackage{lscape}
 \usepackage{lineno}
\bibpunct[: ]{(}{)}{,}{a}{}{,}

%\setmainfont[Mapping=tex-text,Numbers=OldStyle,Ligatures=Common]{Charis SIL} 
\newfontfamily\phon[Mapping=tex-text,Ligatures=Common,Scale=MatchLowercase]{Charis SIL} 
\newcommand{\ipa}[1]{{\phon \mbox{#1}}} %API tjs en italique
\newcommand{\ipab}[1]{{\scriptsize \phon#1}} 

\newcommand{\grise}[1]{\cellcolor{lightgray}\textbf{#1}}
\newfontfamily\cn[Mapping=tex-text,Ligatures=Common,Scale=MatchUppercase]{MingLiU}%pour le chinois
\newcommand{\zh}[1]{{\cn #1}}

\newcommand{\sg}{\textsc{sg}}
\newcommand{\pl}{\textsc{pl}}
\newcommand{\ro}{$\Sigma$}
\newcommand{\ra}{$\Sigma_1$} 
\newcommand{\rc}{$\Sigma_3$}  
\newcommand{\dhatu}[2]{|\ipa{#1}| ``#2''}

\XeTeXlinebreakskip = 0pt plus 1pt %
 %CIRCG
 


\begin{document}

\title{Proto-Kiranti: a new reconstruction}
\author{Guillaume Jacques}
\maketitle

\section{Introduction}
\citet{starostin94kiranti}, \citet{michailovsky94stops}, \citet{opgenort05jero}, \citet{michailovsky10kiranti}

\citet{opgenort04wambule}
\citet{doornenbal09}
\citet{michailovsky02dico}
\citet{jacques15khaling}
\citet{jacques12khaling}
 
\section{Why verbs?}

There are three reasons why this reconstruction is based on verb roots, excluding nouns except to confirm an already established correspondence.

First,  many rhyme distinctions are neutralized in nouns, unlike verb roots, due to the lack of alternation. For instance, proto-Kiranti final *\ipa{-l} is preserved in Khaling and Wambule, but merges with *\ipa{-n} as \ipa{-n} in Bantawa and Limbu, as shown by the cognate set `village' *\ipa{dɛl}: Khaling \ipa{del}, Wambule \ipa{dyal}, Bantawa \ipa{ten}, Limbu \ipa{tɛn} `place'. However, in verb roots, final *\ipa{-l} is clearly distinct from *\ipa{-n} in both Limbu and Bantawa, as is demonstrated in section \ref{sec:rhymes}.

Second, the cross-linguistic tendency for verbs to be less borrowable than nouns (\citealt{wohlgemuth09verbal}) is confirmed in Kiranti languages, where verb roots are only very rarely borrowed. In Khaling, only two clear borrowings have been brought to light: \dhatu{jal}{strike} from Thulung \ipa{jalmu}  (id) and \dhatu{ɦel}{divert water} from Nepali \ipa{helnu} (id).\footnote{The only other possibility is  \dhatu{pil}{squeeze}, which could conceivably be borrowed from Nepali \ipa{pelnu} `press'; since the word appears however to be reconstructible to Proto-Kiranti, and presents some irregular alternations, the borrowing hypothesis is difficult.} 

Third, the great majority of nouns in Kiranti languages are polysyllabic and their internal morphological structure is still poorly understood. While most of these nouns are likely to be synchronically opaque compounds, this remains to be demonstrated on an item-by-item basis.  For instance, the Khaling trisyllabic noun \ipa{kokʦiŋgel} `snail' is not segmentable into elements such as \ipa{kok-}, \ipa{-ʦiŋ} or \ipa{-gel}, and not derivable from any know verb root by means of affixes.


\ipa{sēr}, Dumi ..., Wambule \ipa{syari}:the -i element in Wambule not reconstructible to PK, function unknown (maybe diminutive from \ipa{-si}?)

\section{Morphological alternations} \label{sec:alternations}
Before comparing languages between one another, it is crucial to have a correct understanding of all possible morphological alternations in Kiranti languages. Khaling is used here as a model, since it preserves all three series (unvoiced unaspirated, aspirated, voiced) of stops from proto-Kiranti.


\citet{jacques15derivational.khaling}

\dhatu{kɛnt}{make a hole} $\rightarrow$  \dhatu{ghɛn}{get a hole} $\rightarrow$  \dhatu{ghɛnt}{make a hole}
\section{Onsets} \label{sec:onsets}

\subsection{Stops and affricates}
Much work 
Previous work on proto-Kiranti (in particular \citet{starostin94kiranti}, \citet{michailovsky94stops} and \citet{opgenort05jero}) have already sorted out the correspondences of stops and affricates. Table (\ref{tab:stops}) presents the regular correspondences between the four languages under study. The reconstructions here follow \citet{michailovsky94stops} in reconstructing three series of stops. Unlike Michailovsky's (\citeyear{michailovsky10kiranti}) pessimistic assessment that `very  few etyma seem to support such a series', I found more than 25 cognate sets with non-ambiguous aspirated stops (see appendix). 

Unlike \citet{starostin94kiranti} and \citet{opgenort05jero}, I do not reconstruct a series of preglottalized unvoiced stops *\ipa{`p},  *\ipa{`t},  *\ipa{`ts},  *\ipa{`k} opposed to the plain unvoiced stops (see section \ref{sec:fourth}). 


\begin{table}[H]
\caption{Proto-Kiranti stops and their reflexes} \centering \label{tab:stops}
\begin{tabular}{llllll}
\toprule
Proto-Kiranti & Wambule & Khaling & Bantawa & Limbu \\
\midrule
\ipa{*p} & \ipa{p} & \ipa{p} & \ipa{b} & \ipa{pʰ}  \\
\ipa{*t} & \ipa{t} & \ipa{t} & \ipa{d} & \ipa{tʰ}  \\
\ipa{*ʦ} & \ipa{c} & \ipa{ʦ} & \ipa{cʰ} & \ipa{s}  \\
\ipa{*k} & \ipa{k} & \ipa{k} & \ipa{kʰ} & \ipa{kʰ}  \\
\midrule
\ipa{*kʷ} & \ipa{ɓ} & \ipa{k} & \ipa{k} & \ipa{kʰ}  \\
\midrule
\ipa{*pʰ} & \ipa{pʰ} & \ipa{pʰ} & \ipa{b} / \ipa{bʰ}  & \ipa{pʰ}  \\
\ipa{*tʰ} & \ipa{tʰ} & \ipa{tʰ} & \ipa{d} / \ipa{dʰ} & \ipa{tʰ}  \\
\ipa{*ʦʰ} & ? & \ipa{ʦʰ} & \ipa{cʰ} & \ipa{s}  \\
\ipa{*kʰ} & \ipa{kʰ} & \ipa{kʰ} & \ipa{kʰ} & \ipa{kʰ}  \\
\midrule
\ipa{*b} & \ipa{b} &\ipa{b} / \ipa{bʰ}  & \ipa{p} & \ipa{pʰ}  \\
\ipa{*d} & \ipa{d} & \ipa{d} / \ipa{dʰ}  & \ipa{t} & \ipa{tʰ}  \\
\ipa{*ʣ} & \ipa{j} & \ipa{ʣʰ} & \ipa{c} & \ipa{c}  \\
\ipa{*g} & \ipa{g} & \ipa{g} / \ipa{gʰ}  & \ipa{k} & \ipa{k}  \\
\bottomrule
\end{tabular}
\end{table}

These correspondences account for the great majority of comparanda between the four languages; exceptions are treated by assuming  morphological alternations of the type described in section (\ref{sec:alternations}).

\subsubsection{A fourth series?} \label{sec:fourth}
The plain unvoiced stops are reconstructed by Starostin and Opgenort  when the unvoiced unaspirated stops of Khaling and Wambule correspond to unvoiced unaspirated stops in Bantawa (as opposed to expected voiced stops) and to unaspirated stops in Limbu (as opposed to aspirated stops). Such examples are rare, though not unattested. 

I found three examples of Khaling \ipa{k} corresponding to Bantawa and/or Limbu \ipa{k}, and one example of Khaling \ipa{p} to Bantawa \ipa{p}, as indicated in Table (\ref{tab:kkk}), which would be candidates for plain *\ipa{p} and *\ipa{k} in the systems of Starostin and Opgenort. However, this reconstruction is problematic: as \citet[17]{opgenort05jero} himself notices, plain unaspirated stops are very rare, and there are barely any plain *\ipa{t} in his system (and none in verb roots), while *\ipa{`t} is very common.  


Moreover, other types of correspondences not accounted for by Table \ref{tab:stops} are also found in Kiranti., including Khaling  unaspirated to Bantawa aspirated, Khaling unaspirated to Bantawa aspirated, and Khaling voiced to Bantawa voiced. Reconstructing distinct series of stops to account for each of these correspondences is not useful, especially given the limited number of examples.

\begin{table}[H]
\caption{Irregular correspondences} \centering \label{tab:kkk}
\resizebox{\columnwidth}{!}{
\begin{tabular}{llllll}
\toprule
Type&   Khaling & Bantawa   \\
   \midrule
1&  \dhatu{kept}{sting}  & \dhatu{kept}{sting}  \\
  &\dhatu{keŋ}{cool down}  & \dhatu{keŋ}{be cold}  \\
 &\dhatu{kaŋt}{put over heat}  & \dhatu{kaŋt}{be heated (at the edge of the fire)}      \\
& \dhatu{pum}{hold in one's fist, make a fist}  & \dhatu{pumt}{hold tightly (in the fists)}     \\
 \midrule
 2&\dhatu{kopt}{cover} &   \dhatu{khapt}{thatch a roof} \\
 &\dhatu{ki}{argue} & \dhatu{khi}{quarrel} \\
 &\dhatu{kunt}{stretch}  & \dhatu{kʰɨnt}{stretch} \\
 &\dhatu{koŋt}{freeze} & \dhatu{kʰoŋt}{freeze}\\
 \midrule
3&\dhatu{kʰop}{gather}  & \dhatu{kapt}{put together} \\
&\dhatu{pʰuk}{get up} &  \dhatu{puk}{stand up, to rise} \\
 \midrule
4&\dhatu{bʰokt}{patch} &  \dhatu{bʰokt}{patch}  \\
&\dhatu{dʰuk}{bump into} &  \dhatu{dʰuŋs}{bump}  \\
\bottomrule
\end{tabular}}
\end{table}

A more likely solution is to invoke morphological alternations. As shown in section (\ref{sec:alternations}), Kiranti languages have alternations between voiced and unvoiced stops, the former being an intransitive verb and the latter its transitive counterpart, most probably a remnant of the anticausative derivation (on which see \citealt{jacques15spontaneous}), and also have a less well understood alternation between unvoiced unaspirated ans aspirated stops.

 Type 1 irregularities could be accounted for by supposing for Bantawa a development similar to that of Khaling \dhatu{ghɛn}{get a hole} and\dhatu{ghɛnt}{make a hole}. Bantawa \dhatu{kept}{sting} could be from a form *\ipa{gept} derived from proto-Kiranti *\ipa{kept} ``sting'' (directly reflected by Khaling \dhatu{kept}{sting}) in two steps, first anticausative voicing  *\ipa{kept} ``sting'' $\rightarrow$  *\ipa{gep} ``get a sting''  (which disappeared) and then applicative/causative *\ipa{-t} to *\ipa{gept} ``sting'', which regularly yields Bantawa \dhatu{kept}{sting}.
 
 For Bantawa \dhatu{keŋ}{be cold}, first *\dhatu{keŋ}{be cold} $\rightarrow$ *\dhatu{keŋt}{cool down (tr)} then anticausative *\dhatu{geŋ}{be cooled}, then by regular sound change Bantawa \dhatu{keŋ}{be cold}.

\subsubsection{\ipa{kʷ}}
\citet{opgenort04implosives}

\subsection{Clusters}
\begin{table}[H]
\caption{Proto-Kiranti clusters and their reflexes} \centering \label{tab:clusters}
\resizebox{\columnwidth}{!}{
\begin{tabular}{llllll}
\toprule
PK & Wambule & Khaling & Bantawa & Limbu \\
\midrule
\ipa{*plept} &  \dhatu{plyap}{fold over}& \dhatu{plept}{fold} &  x & x  \\
\ipa{*plept} &  \dhatu{pley}{refrain from}& \dhatu{plent}{postpone} &  x & x  \\
\ipa{*pram} &  x& \dhatu{prɛm}{scratch, claw} &  ?\dhatu{pramt}{xxxx} & x  \\
\ipa{*pʰrɑk} &  \dhatu{phrwaŋ}{untie}& \dhatu{phrok}{untie} & x &  \dhatu{phaːks}{xxxx}  \\
\ipa{*blat} &  x& \dhatu{blɛtt}{tell, explain} &  \dhatu{pemt}{press} &  x \\
\ipa{*blept} &  x& \dhatu{bhlept}{flatten to the ground} &  x &  \dhatu{paːt}{xxxx} \\
\ipa{*bl[u|i]m} &  \dhatu{blimt}{soak}   &\dhatu{blum}{be submerged}&  x & x  \\
\ipa{*blut} &  x& \dhatu{bhlitt}{boil} &  \dhatu{putt}{boil over} &  \dhatu{puːtt}{boil over} \\
\ipa{*brɑt} &  x& \dhatu{bhrot}{shout} &  \dhatu{pat}{cry out, shout} &  x \\
\midrule
\ipa{*khlum} &  x& \dhatu{khlum}{bury} &  \dhatu{kʰumt}{bury} &  x \\
\ipa{*khrap} &    \dhatu{khram}{cry, weep}& x &  \dhatu{kʰap}{cry} &  \dhatu{haːp}{weep} \\
\ipa{*gla[ŋ|k]} &    \dhatu{glak}{win}&  \dhatu{ghlaŋ}{win}  &  x& x \\
\ipa{*gla[ŋ|k]} &    \dhatu{glak}{win}&  \dhatu{ghlaŋ}{win}  &  x& x \\
\ipa{*glu[p|m]} &   x&  \dhatu{glumt}{brood}  &   \dhatu{kupt}{sit on eggs}& x \\
\ipa{*grikt} &  x&  \dhatu{ghrikt}{hold on, take}  &  \dhatu{kɨkt}{grab, take}& x \\
\bottomrule
\end{tabular}}
\end{table}

\dhatu{phlo}{help}, metathesis: Limbu \dhatu{phaˀr}{help}
 
\subsection{Other consonants}
\citet{driem90r}
\section{Vowels and rimes} \label{sec:rhymes}

 

 \section{Conclusion}




\bibliographystyle{unified}
\bibliography{bibliogj}
\end{document}