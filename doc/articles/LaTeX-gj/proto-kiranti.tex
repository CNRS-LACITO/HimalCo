\documentclass[oldfontcommands,oneside,a4paper,11pt]{article} 
\usepackage{fontspec}
\usepackage{natbib}
\usepackage{booktabs}
\usepackage{xltxtra} 
\usepackage{polyglossia} 
\usepackage[table]{xcolor}
\usepackage{gb4e} 
\usepackage{multicol}
\usepackage{graphicx}
\usepackage{float}
\usepackage{hyperref} 
\hypersetup{bookmarks=false,bookmarksnumbered,bookmarksopenlevel=5,bookmarksdepth=5,xetex,colorlinks=true,linkcolor=blue,citecolor=blue}
\usepackage[all]{hypcap}
\usepackage{memhfixc}
\usepackage{lscape}
\usepackage{amssymb}
 \usepackage{lineno}
\bibpunct[: ]{(}{)}{,}{a}{}{,}

%\setmainfont[Mapping=tex-text,Numbers=OldStyle,Ligatures=Common]{Charis SIL} 
\newfontfamily\phon[Mapping=tex-text,Ligatures=Common,Scale=MatchLowercase]{Charis SIL} 
\newcommand{\ipa}[1]{{\phon\mbox{#1}}} %API tjs en italique
\newcommand{\ipab}[1]{{\scriptsize \phon#1}} 

\newcommand{\grise}[1]{\cellcolor{lightgray}\textbf{#1}}
\newfontfamily\cn[Mapping=tex-text,Ligatures=Common,Scale=MatchUppercase]{MingLiU}%pour le chinois
\newcommand{\zh}[1]{{\cn #1}}

\newcommand{\sg}{\textsc{sg}}
\newcommand{\pl}{\textsc{pl}}
\newcommand{\ro}{$\Sigma$}
\newcommand{\ra}{$\Sigma_1$} 
\newcommand{\rc}{$\Sigma_3$}  
\newcommand{\dhatu}[2]{|\ipa{#1}| ``#2''}
\newcommand{\change}[2]{*\ipa{#1} $\rightarrow$ \ipa{#2}}


\XeTeXlinebreakskip = 0pt plus 1pt %
 %CIRCG
 


\begin{document}

\title{Proto-Kiranti: a new reconstruction}
\author{Guillaume Jacques}
\maketitle
\sloppy
\section{Introduction}
\citet{starostin94kiranti}, \citet{michailovsky94stops}, \citet{opgenort05jero}, \citet{michailovsky10kiranti}

\citet{opgenort04wambule}
\citet{doornenbal09}
\citet{michailovsky02dico}
\citet{jacques15khaling}
\citet{jacques12khaling}
 
\section{Why verbs?}

There are three reasons why this reconstruction is based on verb roots, excluding nouns except to confirm an already established correspondence.

First,  many rhyme distinctions are neutralized in nouns, unlike verb roots, due to the lack of alternation. For instance, proto-Kiranti final *\ipa{-l} is preserved in Khaling and Wambule, but merges with *\ipa{-n} as \ipa{-n} in Bantawa and Limbu, as shown by the cognate set `village' *\ipa{dɛl}: Khaling \ipa{del}, Wambule \ipa{dyal}, Bantawa \ipa{ten}, Limbu \ipa{tɛn} `place'. However, in verb roots, final *\ipa{-l} is clearly distinct from *\ipa{-n} in both Limbu and Bantawa, as is demonstrated in section \ref{sec:rhymes}.

Second, the cross-linguistic tendency for verbs to be less borrowable than nouns (\citealt{wohlgemuth09verbal}) is confirmed in Kiranti languages, where verb roots are only very rarely borrowed. In Khaling, only two clear borrowings have been brought to light: \dhatu{jal}{strike} from Thulung \ipa{jalmu}  (id) and \dhatu{ɦel}{divert water} from Nepali \ipa{helnu} (id).\footnote{The only other possibility is  \dhatu{pil}{squeeze}, which could conceivably be borrowed from Nepali \ipa{pelnu} `press'; since the word appears however to be reconstructible to Proto-Kiranti, and presents some irregular alternations, the borrowing hypothesis is difficult.} 

Third, the great majority of nouns in Kiranti languages are polysyllabic and their internal morphological structure is still poorly understood. While most of these nouns are likely to be synchronically opaque compounds, this remains to be demonstrated on an item-by-item basis.  For instance, the Khaling trisyllabic noun \ipa{kokʦiŋgel} `snail' is not segmentable into elements such as \ipa{kok-}, \ipa{-ʦiŋ} or \ipa{-gel}, and not derivable from any know verb root by means of affixes. Even in cases when a noun root is identifiable, in most cases the etyma have a slightly distinct morphological structure in different branches of Kiranti. For instance, the noun `louse' in Wambule \ipa{syari}, though partially related to Khaling \ipa{sēr} and Dumi \ipa{seːr} (id), has an unidentified \ipa{-i} element not found in the other languages: if the proto-form were *\ipa{seri}, the final \ipa{-i} ought to have been preserved in Dumi and a falling tone should have appeared in Khaling (\citealt{jacques16tonogenesis}).\footnote{This \ipa{-i} element might be a remnant of the diminutive \ipa{-si} suffix, but finding other examples of word-internal changes such as \change{-rs-}{-s-}. }

For verb roots by contrast, the wealth of synchronic alternations provide data to reconstruct lost contrasts in final consonants and a simpler vowel system (see in particular \citealt{michailovsky02dico, jacques12khaling, michailovsky12dumi}). This paper directly uses the roots forms reconstructed in the above sources as basic data for the comparative method. 

\section{Morphological alternations} \label{sec:alternations}
Before comparing languages between one another, it is crucial to have a correct understanding of all possible morphological alternations in Kiranti languages. 

\subsection{Applicative and causative} \label{sec:appl}
\citet{michailovsky85dental}

(\citealt{driem87} \citealt[xiii]{michailovsky02dico}) 

\begin{table}[H]
\caption{Cs-stems in Limbu} \centering \label{tab:Cs.limbu}
\begin{tabular}{lllll}
\toprule
Present Stem & Past Stem & Root form \\
\midrule
\ipa{-ŋ-} & \ipa{-ks-} & \ipa{-ks} \\
\ipa{-ŋ-} & \ipa{-ŋs-} & \ipa{-ŋs} \\
\ipa{-m-} & \ipa{-ps-} & \ipa{-ms} \\
\ipa{-m-} & \ipa{-ms-} & \ipa{-ms} \\
\bottomrule
\end{tabular}
\end{table} 

\subsection{Reflexive}\label{sec:refl}

\subsection{Anticausative} \label{sec:anticaus}
\citet{jacques15derivational.khaling}

on which see \citealt{jacques15spontaneous, jacques15causative}

\begin{table}[H]
\caption{Alternation between unvoiced and voiced aspirated verb roots in Khaling} \centering \label{tab:anticaus}
\begin{tabular}{lllll}
\toprule
Transitive & Intransitive&  \\
\midrule
\dhatu{tsɛm}{lose} & \dhatu{ʣʰɛm}{be lost} \\
\dhatu{tsɛp}{be able to do (sth)} & \dhatu{ʣʰɛp}{be possible} \\
\dhatu{kik}{tie} & \dhatu{gʰik}{hang oneself by accident} \\
\dhatu{pʰuk}{wake up} & \dhatu{bʰukt}{ferment (of alcohol)} \\
\midrule
\dhatu{kɛnt}{make a hole} & \dhatu{gʰɛn}{get a hole} \\
\dhatu{kukt}{bend} & \dhatu{gʰuk}{be bent}\\
\bottomrule
\end{tabular}
\end{table}

Double derivation 
\dhatu{kɛnt}{make a hole} $\rightarrow$  \dhatu{ghɛn}{get a hole} $\rightarrow$  \dhatu{ghɛnt}{make a hole}

\subsection{Other}
\dhatu{pil}{squeeze} $\rightarrow$ \dhatu{pʰil}{squeeze out}

\section{Onsets} \label{sec:onsets}

\subsection{Stops and affricates} \label{sec:stops}

Previous work on proto-Kiranti (in particular \citealt{starostin94kiranti}, \citealt{michailovsky94stops} and \citealt{opgenort05jero}) have already sorted out the correspondences of stops and affricates. Table (\ref{tab:stops}) presents the regular correspondences between the four languages under study. The reconstructions here follow \citet{michailovsky94stops} in reconstructing three series of stops. Unlike Michailovsky's (\citeyear{michailovsky10kiranti}) pessimistic assessment that `very  few etyma seem to support such a series', I found more than 25 cognate sets with non-ambiguous aspirated stops (see appendix). All proto-phonemes in Table (\ref{tab:stops}) except *\ipa{kw} and *\ipa{ʦʰ} are attested by a least five etymologies.

Unlike \citet{starostin94kiranti} and \citet{opgenort05jero}, I do not reconstruct a series of preglottalized unvoiced stops *\ipa{`p},  *\ipa{`t},  *\ipa{`ts},  *\ipa{`k} opposed to the plain unvoiced stops (see section \ref{sec:fourth}). 


\begin{table}[H]
\caption{Proto-Kiranti stops and their reflexes} \centering \label{tab:stops}
\begin{tabular}{llllll}
\toprule
Proto-Kiranti & Wambule & Khaling & Bantawa & Limbu \\
\midrule
\ipa{*p} & \ipa{p} & \ipa{p} & \ipa{b} & \ipa{pʰ}  \\
\ipa{*t} & \ipa{t} & \ipa{t} & \ipa{d} & \ipa{tʰ}  \\
\ipa{*ʦ} & \ipa{c} & \ipa{ʦ} & \ipa{cʰ} & \ipa{s}  \\
\ipa{*k} & \ipa{k} & \ipa{k} & \ipa{kʰ} & \ipa{kʰ}  \\
\midrule
\ipa{*kw} & \ipa{ɓ} & \ipa{k} & \ipa{g} & \ipa{kʰ}  \\
\midrule
\ipa{*pʰ} & \ipa{pʰ} & \ipa{pʰ} & \ipa{b} / \ipa{bʰ}  & \ipa{pʰ}  \\
\ipa{*tʰ} & \ipa{tʰ} & \ipa{tʰ} & \ipa{d} / \ipa{dʰ} & \ipa{tʰ}  \\
\ipa{*ʦʰ} & ? & \ipa{ʦʰ} & ? & \ipa{s}  \\
\ipa{*kʰ} & \ipa{kʰ} & \ipa{kʰ} & \ipa{kʰ} & \ipa{kʰ}  \\
\midrule
\ipa{*b} & \ipa{b} &\ipa{b} / \ipa{bʰ}  & \ipa{p} & \ipa{pʰ}  \\
\ipa{*d} & \ipa{d} & \ipa{d} / \ipa{dʰ}  & \ipa{t} & \ipa{tʰ}  \\
\ipa{*ʣ} & \ipa{ʣ} & \ipa{ʣʰ} & \ipa{c} & \ipa{c}  \\
\ipa{*g} & \ipa{g} & \ipa{g} / \ipa{gʰ}  & \ipa{k} & \ipa{k}  \\
\bottomrule
\end{tabular}
\end{table}

Proto-Kiranti *\ipa{d} corresponds to \ipa{ʣ} in Khaling before *\ipa{e} in \dhatu{ʣe}{say}, corresponding Wambule \dhatu{de}{say}.\footnote{There are two Khaling verb roots in \ipa{de-}, but none has known external cognates. }

These correspondences account for the great majority of comparanda between the four languages; exceptions are treated by assuming  morphological alternations of the type described in section (\ref{sec:alternations}).

The correspondences of affricates in Bantawa are not completely understood. In a few cases, Affricates in Khaling and Wambule correspond to \ipa{s} or dental stops in Bantawa. External comparanda (Japhug  \ipa{tɕur} `be sour' and Tibetan \ipa{ɴtɕʰams} `dance') suggest that Bantawa is innovative here. No attempt will be made to explain these correspondences for now.

\begin{table}[H]
\caption{Irregular correspondences of affricates in Bantawa} \centering \label{tab:affricates.bantawa}
\begin{tabular}{llllll}
\toprule
Khaling & Bantawa \\
\midrule
\dhatu{ʣʰur}{be sour} & \dhatu{sunt}{be sour}\\
\dhatu{ʦon}{jump} & \dhatu{tant}{jump}\\
\dhatu{ʦʰom}{dance} & \dhatu{tʰom}{dance}\\
\bottomrule
\end{tabular}
\end{table}

Affricates of other languages correspond to Wambule \ipa{s-} in a few cases indicated in Table \ref{tab:affricates.wambule}. Since \dhatu{sart}{soak with urine} also presents irregular vowel correspondence, these examples might  be borrowing from a language such as Yamphu or Limbu where *\ipa{ts-} changes to \ipa{s-}

\begin{table}[H]
\caption{Irregular correspondences of affricates in Wambule} \centering \label{tab:affricates.wambule}
\begin{tabular}{llllll}
\toprule
Khaling & Wambule \\
\midrule
\dhatu{ʦɛm}{lose} & \dhatu{samt}{lose}\\
\dhatu{ʦer}{urinate} & \dhatu{sart}{soak with urine} \\
\dhatu{ʦu}{be spicy} & \dhatu{su}{be hot, spicy}\\
\bottomrule
\end{tabular}
\end{table}

\subsubsection{A fourth series?} \label{sec:fourth}
The plain unvoiced stops are reconstructed by Starostin and Opgenort  when the unvoiced unaspirated stops of Khaling and Wambule correspond to unvoiced unaspirated stops in Bantawa (as opposed to expected voiced stops) and to unaspirated stops in Limbu (as opposed to aspirated stops). Such examples are rare, though not unattested. 

I found three examples of Khaling \ipa{k} corresponding to Bantawa and/or Limbu \ipa{k}, and one example of Khaling \ipa{p} to Bantawa \ipa{p}, as indicated in Table (\ref{tab:kkk}), which would be candidates for plain *\ipa{p} and *\ipa{k} in the systems of Starostin and Opgenort. However, this reconstruction is problematic: as \citet[17]{opgenort05jero} himself notices, plain unaspirated stops are very rare, and there are barely any plain *\ipa{t} in his system (and none in verb roots), while *\ipa{`t} is very common.  


Moreover, other types of correspondences not accounted for by Table \ref{tab:stops} are also found in Kiranti, including Khaling  unaspirated to Bantawa aspirated, Khaling unaspirated to Bantawa aspirated, and Khaling voiced to Bantawa voiced. Reconstructing distinct series of stops to account for each of these correspondences is not useful, especially given the limited number of examples.\footnote{Since Limbu merged the two unvoiced series, type 2 and 3 irregularities are not possible to detect. Only one type 2 irregularity is found between Khaling and Wambule, \dhatu{pit}{bring (horizontal plane)} to Wambule |\ipa{pʰit}| \textit{id}.}

\begin{table}[H]
\caption{Irregular correspondences} \centering \label{tab:kkk}
\resizebox{\columnwidth}{!}{
\begin{tabular}{llllll}
\toprule
Type&   Khaling & Bantawa   \\
   \midrule
1&  \dhatu{kept}{sting}  & \dhatu{kept}{sting}  \\
  &\dhatu{keŋ}{cool down}  & \dhatu{keŋ}{be cold}  \\
 &\dhatu{kaŋt}{put over heat}  & \dhatu{kaŋt}{be heated (at the edge of the fire)}      \\
& \dhatu{pum}{hold in one's fist, make a fist}  & \dhatu{pumt}{hold tightly (in the fists)}     \\
& \dhatu{ʦent}{teach}  & \dhatu{cint}{teach}  \\
 \midrule
 2 &\dhatu{ki}{argue} & \dhatu{khi}{quarrel} \\
 &\dhatu{kunt}{stretch}  & \dhatu{kʰɨnt}{stretch} \\
 &\dhatu{koŋt}{freeze} & \dhatu{kʰoŋt}{freeze}\\
 \midrule
3&\dhatu{kʰop}{gather}  & \dhatu{kapt}{put together} \\
&\dhatu{pʰuk}{get up} &  \dhatu{puk}{stand up, to rise} \\
 \midrule
4&\dhatu{bʰokt}{patch} &  \dhatu{bʰokt}{patch}  \\
&\dhatu{dʰuk}{bump into} &  \dhatu{dʰuŋs}{bump}  \\
\bottomrule
\end{tabular}}
\end{table}

A more likely solution is to invoke morphological alternations. As shown in section (\ref{sec:alternations}), Kiranti languages have alternations between voiced and unvoiced stops, the former being an intransitive verb and the latter its transitive counterpart, most probably a remnant of the anticausative derivation, and also have a less well understood alternation between unvoiced unaspirated ans aspirated stops.

 Type 1 irregularities could be accounted for by supposing for Bantawa a development similar to that of Khaling \dhatu{ghɛn}{get a hole} and \dhatu{ghɛnt}{make a hole}. Thus, Bantawa \dhatu{kept}{sting} could be from a form *\ipa{gept} derived from proto-Kiranti *\ipa{kept} ``sting'' (directly reflected by Khaling \dhatu{kept}{sting}) in two steps, first anticausative voicing  *\ipa{kept} ``sting'' $\rightarrow$  *\ipa{gep} ``get a sting''  (which disappeared) and then applicative/causative *\ipa{-t} to *\ipa{gept} ``sting'', which regularly yields Bantawa \dhatu{kept}{sting}.
 
 For Bantawa \dhatu{keŋ}{be cold}, first *\dhatu{keŋ}{be cold} $\rightarrow$ *\dhatu{keŋt}{cool down (tr)} then anticausative *\dhatu{geŋ}{be cooled}, then by regular sound change Bantawa \dhatu{keŋ}{be cold}.
 
 Alternatively, this correspondence could reflect loanwords either from a language with a Bantawa-type phonology into Khaling or the other way round.
 
Likewise, type 4  irregularities could be the inverse situation. \dhatu{bʰokt}{patch} could originate from a transitive root *\ipa{pokt} directly ancestral to Bantawa \dhatu{bʰokt}{patch}, to which anticausative derivation to *\ipa{bʰok} ``be patched" has been applied and then applicative/causative \ipa{-t} to the attested Khaling \dhatu{bʰhokt}{patch}. For the intransitive verb \dhatu{dʰuk}{bump into}, the hypothesis is slightly different. We can suppose a proto-Kiranti root *\ipa{tuk} ``bump into (tr)", which undergoes reflexive derivation (see section \ref{sec:bantawa}) in Bantawa to *\ipa{tuŋs} then \ipa{dʰuŋs} by regular sound change. Khaling \dhatu{dʰuk}{bump into} on the other hand is the anticausative of the root *\ipa{tuk} ``bump into (tr)", which did not leave a direct descendent in the sample of languages used in this paper.
 
 This double derivation hypothesis used to explain type  and irregularities specifically predicts the absence of transitive verbs with voiced initial in Khaling (resp. unvoiced initial in Bantawa) corresponding to transitive verbs with voiced initial in Bantawa (resp.unvoiced initial in Khaling) if the former has no \ipa{-t} complex coda. No such example has been found up to now.
 
 Irregularities of type 2 and 3 can potentially be accounted for by assuming derivations by aspirated of the type of Khaling \dhatu{pil}{squeeze} $\rightarrow$ \dhatu{pʰil}{squeeze out}, but since this derivation is poorly attested, I will not attempt at this stage a systematic scenario; the aspiration alternation of these roots is reconstructed directly to Proto-Kiranti. Alternatively, irregularity of type  2 could be cases of the initial *\ipa{kw} (reconstructed only when a Bantawa cognate is available).

\subsubsection{\ipa{kw}}
The correspondence reconstructed here as *\ipa{kw} following \citet{opgenort04implosives} is only attested in two verbs *\ipa{kwa} `eat (hard food)', attested in Khaling \dhatu{ka}{eat (hard food)} and Wambule \dhatu{ɓa}{eat by biting, bite} and  *\ipa{kwɑp} `cover' Khaling \dhatu{kopt}{cover}, Wambule \dhatu{ɓwap}{cover, snare}, Bantawa \dhatu{kʰapt}{to thatch a roof} and Limbu  \dhatu{kʰaps}{to put on (a cover or blanket)}. 
 

 This correspondence is otherwise well attested in monosyllabic nouns, in particular *\ipa{kwi} `yam, potato' (Khaling \ipa{ki}, Limbu \ipa{khe}).\footnote{The verb `cover' show that the Bantawa reflex o *\ipa{kw} is \ipa{kh}; thus, nouns such as \ipa{saki} `potato, yam' and \ipa{ogi} `sweet potato' must be borrowings.}

It is puzzling that no corresponding voiced or aspirated series is found, suggesting that *\ipa{kw} might originally have been a cluster rather than a labiovelar series. However, in this case the absence of other examples of *\ipa{Cw} clusters is also unexpected.

\subsection{Clusters} \label{sec:clusters}
The only initial clusters that are clearly reconstructible in Kiranti are those in labial or velar stops followed by *\ipa{l} or *\ipa{r}. Table \ref{tab:clusters} presents all available evidence in the four target languages.

\begin{table}[H]
\caption{Proto-Kiranti clusters and their reflexes} \centering \label{tab:clusters}
\resizebox{\columnwidth}{!}{
\begin{tabular}{llllll}
\toprule
PK & Wambule & Khaling & Bantawa & Limbu \\
\midrule
\ipa{*plept} &  \dhatu{pljap}{fold over}& \dhatu{plept}{fold} &  x & x  \\
\ipa{*plept} &  \dhatu{plej}{refrain from}& \dhatu{plent}{postpone} &  x & x  \\
\ipa{*pram} &  x& \dhatu{prɛm}{scratch, claw} &  ?\dhatu{pramt}{scratch, tear} & x  \\
\ipa{*pʰrɑk} &  \dhatu{pʰrwaŋ}{untie}& \dhatu{pʰrok}{untie} & x &  \dhatu{pʰaːks}{untie}  \\
\ipa{*blat} &  x& \dhatu{blɛtt}{tell, explain} &  \dhatu{paːt}{speak} &  x \\
\ipa{*blept} &  x& \dhatu{bʰlept}{flatten}   &  \dhatu{pemt}{press} &x \\
\ipa{*bl[u|i]m} &  \dhatu{blimt}{soak}   &\dhatu{blum}{be submerged}&  x & x  \\
\ipa{*blut} &  x& \dhatu{bʰlitt}{boil} &  \dhatu{putt}{boil over} &  \dhatu{puːtt}{boil over} \\
\ipa{*brɑt} &  x& \dhatu{bʰrot}{shout} &  \dhatu{pat}{cry out, shout} &  x \\
\midrule
\ipa{*klek} / {*kʰlekt}&  \dhatu{kʰljakt}{rub (body)} & \dhatu{klekt}{rub (with oil)} &  x&  x \\
\ipa{*kʰlum} &  x& \dhatu{kʰlum}{bury} &  \dhatu{kʰumt}{bury} &  x \\
\ipa{*kʰrap} &    \dhatu{kʰram}{cry, weep}& x &  \dhatu{kʰap}{cry} &  \dhatu{haːp}{weep} \\
\ipa{*gla[ŋ|k]} &    \dhatu{glak}{win}&  \dhatu{ghlaŋ}{win}  &  x& x \\
\ipa{*glu[p|m]} &   x&  \dhatu{glumt}{brood}  &   \dhatu{kupt}{sit on eggs}& x \\
\ipa{*grikt} &  x&  \dhatu{gʰrikt}{take}  &  \dhatu{kɨkt}{grab, take}& x \\
\bottomrule
\end{tabular}}
\end{table}

Clusters are well-preserved in Wambule and Khaling, and completely lost in Bantawa and Limbu. Wambule and Khaling generally agree the clusters in all roots in the corpus, except in two cases. First, the root represented by Khaling \dhatu{pʰekt}{flick away} is interesting in having two corresponding forms in Wambule: |\ipa{pʰjak}| without cluster and \dhatu{pʰrjak}{flick away}, suggesting the possibility of an \ipa{-r-} infix.\footnote{On the reconstruction of an *\ipa{-r-} infix in Sino-Tibetan, see \citet[111-120]{sagart99roc}.} Second, Khaling \dhatu{kat}{bite} corresponds to Wambule  \dhatu{krat}{bite with molar, gnaw}, here without alternating form.

There is one case where the initial clusters appears to have undergone metathesis to a coda, namely Khaling \dhatu{pʰlo}{help} corresponding to Limbu \dhatu{pʰaˀr}{help}; as shown in section (\ref{sec:codas}), Limbu \ipa{-ˀr} is the regular reflex of proto-Kiranti *\ipa{-l}. Given Bantawa \dhatu{pʰas}{help}, this root can be reconstructed as *\ipa{pʰlɑs} `help' in proto-Kiranti (see section \ref{sec:rhymes} for more details ion vowels and codas); Bantawa did not undergo metathesis like Limbu.
 
\subsection{Rhotics} 
As shown by  \citet{driem90r}, proto-Kiranti *\ipa{r} changes to \ipa{j} in Limbu. However, the correspondences of \ipa{r} and \ipa{j} across Kiranti languages is not as straightforward as might seem at first glance.

Khaling has both \ipa{r} and \ipa{j} initials, which both correspond to \ipa{j} (zero before \ipa{i} and \ipa{e}) in Limbu, as shown by Khaling \dhatu{ret}{laugh} and  \dhatu{jil}{make soft by squeezing} to Limbu \dhatu{et}{laugh} and \dhatu{iˀr}{rub} respectively.

In Bantawa, Khaling \ipa{r} mostly corresponds to \ipa{r}, except in the verb \dhatu{i}{laugh}. In Wambule, Khaling \ipa{r} corresponds to \ipa{r} except in case of the root ``to stand" (Khaling |\ipa{rep}|, Wambule  |\ipa{jam}|); also causative \dhatu{japs}{erect, make stand upright}). It is interesting to note that the applicative form of the same root presents a correspondence of \ipa{r} to \ipa{r}, namely Khaling \dhatu{rept}{respect (so's words)} to Wambule \dhatu{rjapt}{obey so, heed so's words}.\footnote{We see the same unpredictable semantic change in Khaling and Wambule ``stand for, by'' $\rightarrow$ ``respect''.} Rather than reconstructing a group *\ipa{rj} in proto-Kiranti or some other form, it is likely that one of the correspondence reflect an inherited layer, while the other is borrowed. For the time being I assume that \ipa{r} is the regular reflex of proto-Kiranti *\ipa{r} in Bantawa and Wambule.


 

\subsection{Other consonants}
The correspondences of all other initial consonants is relatively straightforward. I follow \citet{opgenort04implosives} in reconstructing a series of preglottalized nasals. The initial \ipa{*ʔŋ} is tentatively reconstructed on the basis of the comparison between Khaling \dhatu{ŋol}{mix} and Wambule \dhatu{ɓwalt}{mix}.

The only noteworthy sound change is the merger of *\ipa{ŋ} and \ipa{n} to *\ipa{n} in Limbu, and the partial shift of \change{ŋ-}{n-} before front vowel in Bantawa, as in \dhatu{nett}{disturb, irritate, annoy} corresponding to Khaling \dhatu{ŋet}{hurt}.

\begin{table}[H]
\caption{Proto-Kiranti nasals, lateral and sibilant intials} \centering \label{tab:nasals}
\begin{tabular}{llllll}
\toprule
Proto-Kiranti & Wambule & Khaling & Bantawa & Limbu \\
\midrule
\ipa{*m} & \ipa{m} & \ipa{m} & \ipa{m} & \ipa{m}  \\
\ipa{*ʔm} & \ipa{m} & \ipa{m} & \ipa{ɓ} & \ipa{m}  \\
\ipa{*n} & \ipa{n} & \ipa{n} & \ipa{n} & \ipa{n}  \\
\ipa{*ʔn} & \ipa{n} & \ipa{n} & \ipa{ɗ} & \ipa{n}  \\
\ipa{*ŋ} & \ipa{ŋ} & \ipa{ŋ} & \ipa{n} / \ipa{ŋ} & \ipa{n}  \\
\ipa{*ʔŋ} &  \ipa{ɓ}  & \ipa{ŋ} & x &x  \\
\ipa{*l} & \ipa{l} & \ipa{l} & \ipa{l} & \ipa{l} / \ipa{r} \\
\ipa{*s} & \ipa{s} & \ipa{s} & \ipa{s} & \ipa{s}  \\
\bottomrule
\end{tabular}
\end{table}

There are only two irregularities in the data at hand related to these initial consonants.

First, Wambule \dhatu{ŋwam}{smell} has \ipa{ŋ} where comparative evidence strongly suggest *\ipa{n} (Khaling |\ipa{nom}|, Bantawa  |\ipa{nam}| and outside of Kiranti Japhug \ipa{mnɤm} `smell (it)' and Tibetan \ipa{mnam} \textit{id.}). While it could be tempting to assume that the correspondence of Wambule \ipa{ŋ-} to \ipa{n-} in other languages is the reflex of the cluster *\ipa{mn-}, this hypothesis is better left aside until confirming data is found.

Second, Khaling \dhatu{sik}{string beads} and Wambule \dhatu{sikt}{thread a needle} appear to correspond to Limbu \dhatu{tiːks}{to thread, to string (beads)}  and Bantawa \dhatu{tɨk}{to thread, to string, to make a garland}, but the correspondence of \ipa{s-} to \ipa{t-} is unique. This correspondence might reflect a special cluster, but no attempt will be made to provide a reconstruction for the onset of this root.

\section{Vowels and rimes} \label{sec:rhymes}

\subsection{Bantawa} \label{sec:bantawa}
 
 For Limbu and Khaling, \citet{michailovsky02dico}, \citet{jacques12khaling} and \citet{jacques16si} explicitly provide root forms. For Wambule, in the case of alternating verbs (\citealt[255-263]{opgenort04wambule}), the first stem is taken as the root form, as apart from a few irregular verbs, the second stem can always be predicted from the first one.  
 
 For Bantawa, some more discussion is needed on how the roots were extracted from Doornenbal's work. Based on the correspondences given in \citet[129; 132]{doornenbal09}, the following final consonants are reconstructed (alternating forms separated by slashes are in complementary distribution):

\begin{table}[H]
\caption{Bantawa root codas} \centering \label{tab:bantawa.root}
\begin{tabular}{cccc}
\toprule
Stem-C & Stem-V & Reconstructed Coda \\
\midrule
\ipa{-k} & \ipa{-ʔ-} & \ipa{-k} \\
\ipa{-t} & \ipa{-t-} & \ipa{-t} \\
\ipa{-p} & \ipa{-ʔ-} /  \ipa{-w-} & \ipa{-p} \\
$\varnothing$ & $\varnothing$ / \ipa{-w-} / \ipa{-y-} & $\varnothing$ \\
\ipa{-ŋ} & \ipa{-ŋ-} & \ipa{-ŋ} \\
\ipa{-n} & \ipa{-l-} & \ipa{-l} \\
\ipa{-n} & \ipa{-y-} / \ipa{-n-}& \ipa{-n} \\
\ipa{-n} & \ipa{-r-} & \ipa{-r} \\
\ipa{-m} & \ipa{-m-} & \ipa{-m} \\
$\varnothing$ & \ipa{-s-} & \ipa{-s} \\
\midrule
\ipa{-k} & \ipa{-kt-} & \ipa{-kt} \\
\ipa{-t} & \ipa{-tt-} & \ipa{-tt} \\
\ipa{-p} & \ipa{-pt-} & \ipa{-pt} \\
\ipa{-ŋ} & \ipa{-ŋt-} & \ipa{-ŋt} \\
\ipa{-n} & \ipa{-nt-} & \ipa{-nt} \\
\ipa{-m} & \ipa{-mt-} & \ipa{-mt} \\
\midrule
\ipa{-ŋ} & \ipa{-ŋs-} & \ipa{-ŋs}  \\
\ipa{-n} & \ipa{-ns-} & \ipa{-ns} \\
\ipa{-m} & \ipa{-ms-} & \ipa{-ms}   \\
\bottomrule
\end{tabular}
\end{table}

The correspondences are relatively straightforward; simple stops undergo lenition when the verb root is followed by a vowel-initial suffix, and final dental sonorants merge to \ipa{-n} when followed by a consonant-initial suffix. We notice important gaps in the distribution of complex codas: unlike in Khaling, there are no groups such as \ipa{-rt} and \ipa{-lt}. Given the rule that \ipa{r} and \ipa{l} merge with \ipa{n} when followed by consonants, it can be surmised that *\ipa{-rt} and *\ipa{-lt} merged with \ipa{-nt}, an idea that is supported by comparative data, as is shown below.

For complex coda with \ipa{-s} as second element, only nasals are found as first element, an observation that can be historically explained The Cs-stems of Bantawa originate from the merger of transitive Cs stems (build in most cases by addition of the causative \ipa{-s} suffix, \citealt{michailovsky85dental}) and of reflexive C-si stems. 

We know from Limbu (\citealt[xiii]{michailovsky02dico}) that transitive Cs-stems, when the C is a stop consonant, show nasalization of the stop in the \textit{present} stem (which mainly appears before consonant-initial suffixes), as indicated in Table \ref{tab:Cs.limbu} in section \ref{sec:appl}. Hence, it is likely that Bantawa transitive Cs-stems underwent analogical levelling and that all \ipa{-ks} and \ipa{-ps} stems were converted to \ipa{-ŋs} and \ipa{-ms} stems.

As for intransitive Cs-stems, as shown in section \ref{sec:refl}, in the reflexive forms final stops are also nasalized in part of the paradigms. For instance, the verb \dhatu{ims}{sleep} in Bantawa is cognate to Khaling \dhatu{ʔip-si}{sleep}, the reflexive form of \dhatu{ʔipt}{put to sleep}.\footnote{The basic root *\ipa{ʔip} `sleep', cognate to Japhug \ipa{-ʑɯβ} `sleep' (<*\ipa{jip}), was lost already in proto-Kiranti and replaced by the reflexive*\ipa{ʔip-si} `put oneself to sleep' $\rightarrow$ `sleep'. This is one of the common innovations of Kiranti languages.} In Khaling this verb has nasalized forms in the non-past (except dual) and infinitive (\ipa{ʔʌ̂msiŋʌ} `I sleep', \ipa{ʔʌ̂msi} he sleeps'), and a non-nasal stop in the dual and first plural (\ipa{ʔipsiji} `you and I/they two sleep'). The alternating stem with nasalized final is more widespread than the non-nasalized ones, and here again, it is likely that levelling took place in Bantawa.

\subsection{Vowels} \label{sec:vowels}

While some Kiranti languages, in particular Khaling, present very complex vowel systems and bewildering vowel alternations, internal reconstruction allows to reduce this number to 5 (as in Khaling).\footnote{While a contrast between \ipa{ɛ} and \ipa{a} is kept in the representation of Khaling verb roots, it is clear that these two were originally in complementary distribution, which was only broken when \dhatu{jal}{strike} was borrowed from Thulung (\citealt[1110]{jacques12khaling}).} The only exception is Limbu, which present a contrast between long and short vowels, as well as a contrast between mid-low and mid-high vowels (\ipa{e} and \ipa{o} vs \ipa{ɛ} and \ipa{ɔ}).  

Khaling and Wambule have in several verbs \ipa{o} and \ipa{wa} corresponding to \ipa{a}in Bantawa and Limbu instead of \ipa{o}.\footnote{Except before \ipa{-r} and perhaps other coronal finals where Wambule has \ipa{a} and Khaling \ipa{a}, as shown by Wambule \dhatu{kʰart}{parch}	 	
and \dhatu{ʣar}{drip, drizzle}	corresponding to Khaling 	\dhatu{kʰor}{parch, fry} and \dhatu{ʣʰor}{leak}.} This correspondence is reconstructed as proto-Kiranti *\ipa{ɑ}.

Table \ref{tab:vowels} presents an account of the basic vowel correspondences between the four languages discussed in the present work.

\begin{table}[H]
\caption{Proto-Kiranti vowels} \centering \label{tab:vowels}
\begin{tabular}{llllll}
\toprule
Proto-Kiranti & Wambule & Khaling & Bantawa & Limbu \\
\midrule
\ipa{*a} & \ipa{a} & \ipa{a}  / \ipa{ɛ} & \ipa{a} & \ipa{a}  \\
\ipa{*ɑ} & \ipa{wa} & \ipa{o} & \ipa{a} & \ipa{a}  \\
\ipa{*ɔ} & \ipa{wa} & \ipa{o} & \ipa{o} & \ipa{ɔ}  \\
\ipa{*o} & \ipa{wa} & \ipa{o} & \ipa{o} & \ipa{o}  \\
\ipa{*ɛ} & \ipa{ja} & \ipa{e} & \ipa{e}  & \ipa{ɛ}  \\
\ipa{*e} & \ipa{ja} & \ipa{e} & \ipa{e}  / \ipa{ɨ} & \ipa{e}    \\
\ipa{*u} & \ipa{u} & \ipa{u} & \ipa{u} /  \ipa{ɨ} \  & \ipa{u} / \ipa{o} \\
\ipa{*i} & \ipa{i} & \ipa{i} & \ipa{i}  / \ipa{ɨ} & \ipa{i}  \\
\bottomrule
\end{tabular}
\end{table}
  
 The origin of Bantawa \ipa{ɨ} are *\ipa{i} (open-syllable stems, \ipa{-s}, velar final stems), *\ipa{e} (\ipa{-s} stems), and short *\ipa{u} (\ipa{-t} stems). There are two exceptions: \dhatu{es}{defecate} (no change to \ipa{ɨ}) and \dhatu{pʰɨpt}{to suck, to absorb} (change of *\ipa{i} to \ipa{ɨ} before \ipa{-p}).

Proto-Kiranti short *\ipa{-uk} yields Limbu \ipa{-ok}  in most examples (exceptions include   \dhatu{suks}{cough},  \dhatu{suk}{to wait in ambush for}, and \dhatu{luk}{be destroyed}).

\subsubsection{Irregularities between high vowels}
The correspondences between \ipa{u} and \ipa{i} in closed syllable present a few irregularities, collected in table \ref{tab:high.vowels}.  

 \begin{table}[H]
 \caption{Irregular correspondences between high vowels in closed syllables} \centering \label{tab:high.vowels}
 \resizebox{\columnwidth}{!}{
 \begin{tabular}{lllll}
 \toprule
 Wambule & Khaling & Bantawa & Limbu \\
\midrule
\dhatu{lum}{boil}	&	\dhatu{lum}{parboil}	&	\dhatu{limt}{parboil}	&		\\
\dhatu{ŋir}{roar}	&	\dhatu{ŋur}{roar}	&		&		\\
\dhatu{tikt}{support}	&	\dhatu{tikt}{support}	&&	\dhatu{thukt}{to support}			\\
&	\dhatu{bhlitt}{boil}	&		\dhatu{putt}{to boil over, steam}	&\dhatu{puːtt}{to boil over}	\\
\dhatu{blimt}{soak}	&	\dhatu{blum}{be submerged}	&		&		\\
\bottomrule
\end{tabular}}
\end{table}

 A phonological explanation here is unlikely. Rather, it should be noticed that in Khaling, in the intransitive and -Ct paradigms, high vowels are often neutralized to \ipa{ʌ} in stem alternations, and that the contexts where \ipa{i} and \ipa{u} can be distinguished are restricted to dual forms. In cases where it is Khaling that diverges from the other languages, it is likely that the other languages have a more conservative vocalism. As for the cases of `parboil' and `support' where Bantawa and Limbu are the divergent languages, since no productive vowel alternation involving high vowel is observable, no such explanation can be provided. 

\subsubsection{Irregularities between high and mid vowels}
In open syllables, the alternations between \ipa{u} and \ipa{o} on the one hand and \ipa{i} and \ipa{e/ɛ} on the other hand may be remnants of former apophony still present in Khaling in some intransitive verb forms.\footnote{In Khaling   some intransitive \ipa{-u} stem verbs (\dhatu{nu}{be good} and \dhatu{lu}{feel}) have unexplained long \ipa{ōː} instead of expected \ipa{ūː} in third and second singular past forms (\ipa{nōːtɛ}, \ipa{lōːtɛ}).} 

 \begin{table}[H]
 \caption{Irregular correspondences between high and mid vowels} \centering \label{tab:high.vowels}
 \resizebox{\columnwidth}{!}{
 \begin{tabular}{lllll}
 \toprule
 Wambule & Khaling & Bantawa & Limbu \\
\midrule
\dhatu{su}{be hot, spicy}	 &\dhatu{ʦu}{be spicy}	&	 		&	\dhatu{sos}{to be rich in taste}	\\	
&\dhatu{su}{itch}	&		\dhatu{sus}{itch}	&	\dhatu{sos}{itch}	\\	
&\dhatu{kʰutt}{comb}	&		\dhatu{kʰɨtt}{comb}	&	\dhatu{kʰott}{comb}	\\	
&\dhatu{ʦok}{to sharpen}	&			&	\dhatu{sukt }{to be sharp}	\\	
&\dhatu{rup}{cut into pieces}	&		\dhatu{rop}{to break into pieces}	&		\\	
\midrule
 \dhatu{pʰit}{bring}	&\dhatu{pi}{come}	&			&	\dhatu{pʰɛn}{come}	\\	
&\dhatu{pʰi}{be spoiled (of rice)}	&			&	\dhatu{pʰɛn}{be spoiled}	\\	
&\dhatu{ki}{fight, argue}	&			\dhatu{kʰi}{quarrel}	&	\dhatu{kʰe}{to quarrel}	\\	
&\dhatu{nekt}{cover}	&		\dhatu{nɨkt}{to cover, be buried}	&		\\	
\bottomrule
\end{tabular}}
\end{table}


\subsubsection{Irregularities between front and back vowels}
There are in addition a few irregularities involving correspondences between low vowels and mid-vowels, or back vowel and front vowels, as shown by Table \ref{tab:front.vowels}.

 \begin{table}[H]
 \caption{Irregular correspondences between  front and back vowels} \centering \label{tab:front.vowels}
 \resizebox{\columnwidth}{!}{
 \begin{tabular}{llllll}
 \toprule
 Wambule & Khaling & Bantawa & Limbu \\
\midrule
\dhatu{jwa}{come down}	&	\dhatu{je}{come down}	&	\dhatu{ji / ju}{come down}	&		&	\\
	&	\dhatu{mutt}{causative}	&	\dhatu{mett}{causative}	&	\dhatu{mɛtt}{to do, to do sth. to}	&	\\
\midrule
	&	\dhatu{ʔɛt}{say}	&		\dhatu{ett}{say}	&		&	\\
	&	\dhatu{tʰɛm}{lose (one's way)}	&	\dhatu{tʰem}{be lost}	&		&	\\
\dhatu{lwakt}{lick}	&	\dhatu{lak}{lick}	&	\dhatu{lek}{lick}	&	\dhatu{lak}{lick}	&	\\
\midrule
	&	\dhatu{ʦek}{be hard, be stingy}	&		&	\dhatu{sakt}{be hard}	&	\\
	&	\dhatu{gʰekt}{crack}	&		&	\dhatu{kak}{to crack}	&	\\
\dhatu{sart}{soak with urine}	&	\dhatu{ʦer}{urinate}	&	\dhatu{cʰens}{to urinate}	&	\dhatu{seˀr}{urinate on}	&	\\
 &	\dhatu{ɦak}{open}	&	\dhatu{hoŋs}{to open}	&		&	\\
\bottomrule
\end{tabular}}
\end{table}


The vocalic irregularities in the verbs ``come down'' and the causative auxiliary are probably traces of apophony. In Khaling \dhatu{je}{come down} is an irregular verb whose third singular past stem is \ipa{jāː-} instead of expected $\dagger$\ipa{jēː-}. The causative auxiliary is related synchronically within Khaling to the verb \dhatu{mu}{do}, also an irregular verb. While the vowel alternation patterns observed in Khaling do not readily offer an explanation for the correspondences, it may be surmised that more ablaut grades may have existed in proto-Kiranti for these two verbs, and were regularised differently in each of the modern languages.

The correspondence between Wambule \ipa{wa}, Khaling *\ipa{a} (\ipa{a} or \ipa{ɛ}, Bantawa \ipa{e} and Limbu \ipa{a} in the verb ``lick'' and the other ones is puzzling, and could be a distinct vowel in proto-Kiranti, if more examples can be found. 

\subsection{Simple codas} \label{sec:codas}

\begin{table}[H]
\caption{Proto-Kiranti simple codas} \centering \label{tab:codas}
\begin{tabular}{llllll}
\toprule
Proto-Kiranti & Wambule & Khaling & Bantawa & Limbu \\
\midrule
\ipa{*-k} & \ipa{-k} / vowel length& \ipa{-k} & \ipa{-k} & \ipa{-k}  \\
\ipa{*-t} & \ipa{-j} & \ipa{-t} & \ipa{-t} & \ipa{-t}  \\
\ipa{*-p} & \ipa{-p} & \ipa{-p} & \ipa{-p} & \ipa{-p}  \\
\midrule
\ipa{*-ŋ} & vowel length & \ipa{-ŋ} & \ipa{-ŋ} & \ipa{-ŋ}  \\
\ipa{*-n} & \ipa{-j} /  \ipa{-t} & \ipa{-t} & \ipa{-t} & \ipa{-t}  \\
\ipa{*-m} & \ipa{-m} & \ipa{-m} & \ipa{-m} & \ipa{-m}  \\
\midrule
\ipa{*-r} &  \ipa{-r}  & \ipa{-r} & \ipa{-r} & \ipa{-r}  \\
\ipa{*-l} & \ipa{-l} & \ipa{-l} & \ipa{-l} & \ipa{-ˀr}  \\
\ipa{*-s} &$\varnothing$ & $\varnothing$ & \ipa{-s} & \ipa{-s}  \\
\bottomrule
\end{tabular}
\end{table}

The correspondences of Table \ref{tab:codas} are relatively robust; exceptions involve correspondences between simple codas and complex codas, and are treated in section \ref{sec:Ct}. The only correspondence that deserves discussion is that between proto-Kiranti \ipa{-l} and Limbu \ipa{-ˀr}, as examples are very few, and only include the following:

\begin{itemize}
\item  Khaling \dhatu{kɛl}{cluster together, wear one's hair in a bun}, Limbu \dhatu{kʰaˀr}{to be matted (of hair)}  			 
\item  Khaling \dhatu{jil}{make soft by squeezing}, Limbu \dhatu{iˀr}{to rub, to scrub}  		 
\item   Khaling\dhatu{ʦʰelt}{be bright}, Limbu \dhatu{sɛˀr}{to turn white (of hair), to clear (of the milk of a cow that has just given birth)}  
\item  Khaling \dhatu{pʰɛl}{damage}, Limbu \dhatu{pʰɛˀr}{break (vi)}  (note the anticausative form \dhatu{pɛˀr}{break (vi)}  from *\ipa{bal}, without equivalent in other languages).
\end{itemize}

Nevertheless, there is no counterexample where Khaling \ipa{-l} would correspond to a single final other than \ipa{-ˀr} in Limbu. This shows that although Limbu has neutralized the contrast between final *\ipa{-l} and  *\ipa{-n} in nouns, this contrast is faithfully preserved in the verbal system.

\subsection{Complex codas} \label{sec:Ct}


%*rs nur (vm)	to be sprained, to be twisted	nuˀr	to be sprained, to be dislocated.
%ʔor	or	break(cob of the corn)	vt					ɔˀr 	break without separating (vi), ɔnt  vt to break off, to put aside (a part), to remove 
%seˀr	urinate on

 \section{Conclusion}




\bibliographystyle{unified}
\bibliography{bibliogj}
\end{document}