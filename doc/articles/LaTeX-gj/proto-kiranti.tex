\documentclass[oneside,a4paper,11pt]{article} 
\usepackage{fontspec}
\usepackage{natbib}
\usepackage{booktabs}
\usepackage{xltxtra} 
\usepackage{polyglossia} 
\usepackage[table]{xcolor}
\usepackage{tikz}
\usetikzlibrary{trees}
\usepackage{gb4e} 
\usepackage{multicol}
\usepackage{graphicx}
\usepackage{float}
\usepackage{hyperref} 
\hypersetup{bookmarksnumbered,bookmarksopenlevel=5,bookmarksdepth=5,colorlinks=true,linkcolor=blue,citecolor=blue}
\usepackage[all]{hypcap}
\usepackage{memhfixc}
\usepackage{lscape}
\usepackage{amssymb}
 
\bibpunct[: ]{(}{)}{,}{a}{}{,}

%\setmainfont[Mapping=tex-text,Numbers=OldStyle,Ligatures=Common]{Charis SIL} 
\newfontfamily\phon[Mapping=tex-text,Scale=MatchLowercase]{Charis SIL} 
\newcommand{\ipa}[1]{\textbf{{\phon\mbox{#1}}}} %API tjs en italique
%\newcommand{\ipab}[1]{{\scriptsize \phon#1}} 

\newcommand{\grise}[1]{\cellcolor{lightgray}\textbf{#1}}
\newfontfamily\cn[Mapping=tex-text,Ligatures=Common,Scale=MatchUppercase]{SimSun}%pour le chinois
\newcommand{\zh}[1]{{\cn #1}}

\newcommand{\sg}{\textsc{sg}}
\newcommand{\pl}{\textsc{pl}}
\newcommand{\ro}{$\Sigma$}
\newcommand{\ra}{$\Sigma_1$} 
\newcommand{\rc}{$\Sigma_3$}  
\newcommand{\dhatu}[2]{|\ipa{#1}| `#2'}
\newcommand{\dhat}[1]{|\ipa{#1}|}
\newcommand{\change}[2]{*\ipa{#1} $\rightarrow$ \ipa{#2}}
 

\XeTeXlinebreakskip = 0pt plus 1pt %
 %CIRCG
 


\begin{document}

\title{A reconstruction of Proto-Kiranti verb roots\footnote{I would like to thank Balthasar Bickel, Lukas Denk, Juho Pystynen, Marc Miyake, Martine Mazaudon, Boyd Michailovsky, Alexis Michaud, Amos Teo, Mikhail Zhivlov as well as two anonymous reviewers for useful comments on previous versions of this work. I am responsible for any remaining error. This research was funded by the HimalCo project(ANR-12-CORP-0006) and is related to the Labex EFL (PPC2 Evolutionary approaches to phonology: New goals and new methods (in diachrony and panchrony)}} 
\author{Guillaume Jacques}
\maketitle
\sloppy

\textbf{Abstract}: In Kiranti languages, rich alternations in verbal paradigms make internal reconstruction possible, and allow a better understanding of the vowels and codas of the proto-language than is possible for other parts of speech. This paper, using data from four representative languages (Wambule, Khaling, Bantawa, Limbu), proposes a new approach to Proto-Kiranti historical linguistics combining the comparative method and internal reconstruction, and taking morphological alternations and analogy into account. It presents a comprehensive account of the sound correspondences between the four target languages and reconstructs more than 280 proto-Kiranti verb roots.

\textbf{Keywords}: Kiranti, Khaling, Wambule, Limbu, Bantawa, Sound change, Analogy

\section{Introduction}
The reconstruction of Proto-Kiranti has already been the topic of much detailed research, in particular \citet{starostin94kiranti}, \citet{michailovsky94stops}, \citet{opgenort05jero} and \citet{michailovsky10kiranti}. However, while  initial consonants are relatively well understood, the reconstruction of the vocalic system and final consonants remains obscure.  

Recent work has improved our understanding of the internal reconstruction of verb stem alternations of various Kiranti languages (\citealt{michailovsky02dico, lahaussois11thulung, jacques12khaling, michailovsky12dumi}), and the time is ripe to investigate anew Proto-Kiranti historical phonology, focusing on \textit{verb stems}, whose vocalism and codas are better understood than nouns and words belonging to other parts of speech.

Thanks to the immense progress made in the documentation of Kiranti languages in the last ten years (see for instance \citealt{bickel07chintang, bickel07puma, schackow08puma, doornenbal09, lahaussois09, huysmans2011sampang, schackow15yakkha}), plentiful data is available on various Kiranti languages. 

Yet, at this stage of investigation, it is wiser to follow the models of \citet{bloomfield25central} and \citet{dempwolff34induktiver} and focus on a limited number of representative languages to establish a robust system of reconstruction, before embarking on a comprehensive investigation of all available data. Following Michailovsky's (\citeyear{michailovsky94stops}) classification of Kiranti into four subgroups, one language was chosen for each subgroup: Wambule (\citealt{opgenort04wambule}), Khaling (\citealt{jacques12khaling, jacques15khaling}), Bantawa (\citealt{doornenbal09}) and Limbu (\citealt{driem87, michailovsky02dico}). 

While excellent lexical sources exist on many other languages, in particular  Yamphu (\citealt{rutgers98yamphu}), Dumi  (\citealt{driem93dumi, rai11dumi}), Chintang  (\citealt{rai11chintang}) and  Yakkha  (\citealt{kongren07yakkha}), inclusion of these languages into the reconstruction is deferred to future research, and will actually be a useful test of the robustness of the model proposed in this paper.

The present paper contains four sections. First, I explain in detail why this reconstruction is based on verb roots rather than on the whole vocabulary of the target languages. Second, I provide a cursory account of the morphological alternations attested in Kiranti languages and how they affect the sound correspondences. Third, I describe the correspondences of initial consonants and clusters. Fourth, I present a reconstruction of the rhymes of verb roots in Proto-Kiranti.

In this work, verb roots, as opposed to conjugated forms, are indicated between bars: \dhatu{root}{meaning}. These roots are abstract forms, derived from the stem alternations by internal reconstruction, but to avoid confusion with reconstructions arrived at by the comparative method, this notation is used instead. 

\section{Why verbs?} \label{sec:why}

There are four reasons why this reconstruction is based on verb roots, excluding nouns except to confirm an already established correspondence.

First,  many rhyme distinctions are neutralized in nouns, unlike verb roots, due to the lack of alternation. For instance, proto-Kiranti final *\ipa{-l} is preserved in Khaling and Wambule, but merges with *\ipa{-n} as \ipa{-n} in Bantawa and Limbu, as shown by the cognate set *\ipa{dɛl} `village' (Khaling \ipa{del}, Wambule \ipa{djal}, Bantawa \ipa{ten}, Limbu \ipa{tɛn} `place'). However, in verb roots, final *\ipa{-l} is clearly distinct from *\ipa{-n} in both Limbu and Bantawa, as is demonstrated in section \ref{sec:codas}.

Second, the cross-linguistic tendency for verbs to be less borrowable than nouns (\citealt{wohlgemuth09verbal}) is confirmed in Kiranti languages, where verb roots are only very rarely borrowed. In Khaling, only two clear borrowings have been brought to light: \dhatu{jal}{strike} from Thulung \ipa{jalmu}  (id) and \dhatu{ɦel}{divert water} from Nepali \ipa{helnu} (id).\footnote{The only other possibility is  \dhatu{pil}{squeeze}, which could conceivably be borrowed from Nepali \ipa{pelnu} `press'; since the word appears however to be reconstructible to Proto-Kiranti, and presents some irregular alternations, the borrowing hypothesis is difficult to accept.} 

Third, there is some evidence that some sound changes may be specific to verb roots in Kiranti, so that verbs and other parts of speech need to be separately handled.\footnote{In languages of the ST family with complex tone alternation, such as Yongning Na (\citealt{michaud08na}), verbs also present unique alternations and tend to have a much richer morphology than the rest of the lexicon.} In particular, the lenition of the final \ipa{-t} and \ipa{-n} of verb roots to \ipa{r} when followed by a vowel-initial suffix in Limbu (as in \ipa{sɛr-uŋ}  `I kill' from \dhatu{sɛt}{kill}) does not appear to have taken place in words other than verbs. While there is strong evidence that alleged grammatically conditioned sound changes result from incorrect analyses (\citealt{hill14conditioned}), and while the absence of evidence for this lenition in nouns and pronouns in Limbu may have a purely phonological explanation, it is wise to postpone investigation of the phonology of polysyllabic nouns until that of verb roots is fully understood.

Fourth, the great majority of nouns in Kiranti languages are polysyllabic and their internal morphological structure is still poorly understood. This situation is reminiscent of Indo-European, where primary verbs are more easily reconstructible than nouns (see for instance \citealt{garnier10vocalisme}).

While polysyllabic nouns in Kiranti languages are likely to be synchronically opaque compounds which may contain inherited material, this remains to be demonstrated on an item-by-item basis.  For instance, the Khaling trisyllabic noun \ipa{kokʦiŋgel} `snail' is not segmentable into elements such as \ipa{kok-}, \ipa{-ʦiŋ} or \ipa{-gel}, and not derivable from any know verb root by means of affixes. Even in cases when a noun root is identifiable, in most cases the etyma have a slightly distinct morphological structure in different branches of Kiranti. For instance, the noun `louse' in Wambule \ipa{sjari}, though partially related to Khaling \ipa{sēr} and Dumi \ipa{seːr} (id), has an unidentified \ipa{-i} element not found in the other languages: if the proto-form were *\ipa{seri}, the final \ipa{-i} ought to have been preserved in Dumi and a falling tone should have appeared in Khaling ($\dagger$\ipa{sêr} would be expected, \citealt{jacques16tonogenesis}).\footnote{This \ipa{-i} element might be a remnant of the diminutive \ipa{-si} suffix, but to confirm this idea it is necessary to find other examples of word-internal changes such as \change{-rs-}{-s-}. }

For verb roots by contrast, the wealth of synchronic alternations provide data to reconstruct lost contrasts in final consonants and a simpler vowel system (see in particular \citealt{michailovsky02dico, jacques12khaling, michailovsky12dumi}). This paper directly uses the roots forms reconstructed in the above sources as basic data for the comparative method. It should be stressed that the present reconstruction neglects rare initial consonants or rhymes (such as *\ipa{ʔl}, see \citealt{opgenort04implosives}) that are not attested in verb roots, and therefore cannot be considered to be a complete account of Proto-Kiranti phonology.


\section{Morphological alternations} \label{sec:alternations}
Before comparing languages between one another, it is crucial to have a correct understanding of all possible morphological alternations in Kiranti languages. 

While sound changes are exceptionless, no paradigm, even the most conservative ones, is preserved without analogy, and therefore a correct understanding of all possible alternations is required to make sense of the variation observed across cognates in related languages.

This section focuses on the derivational morphology, and discusses applicative, causative, reflexive and anticausative derivations,  as well as some examples of denominal verbs.

\subsection{Applicative and causative} \label{sec:appl}
In the Sino-Tibetan family, Kiranti languages best preserve the applicative \ipa{-t} and causative \ipa{-s} suffixes, which can be exemplified by the triplet from Limbu (\citealt{michailovsky85dental}):

\begin{itemize}
\item  \dhatu{haːp}{weep} (intransitive, base form)
\item  \dhatu{haːpt}{mourn} (transitive, applicative)
\item  \dhatu{haːps}{cause to weep} (transitive, causative)
\end{itemize}

Traces of the \ipa{-t} applicative exist in all Kiranti languages, but the \ipa{-s} has been lost in Khaling due to the merger of \ipa{-s} stems with open syllable stems.  In Wambule the -s causative is rare, but examples do exist (\citealt[270-1]{opgenort04wambule}).

The semantics of the applicative suffix depends on the verb root. In some cases, the added argument is the stimulus, in other cases it can be experiencer or beneficiary (see \citealt{jacques15derivational.khaling}). In addition, the semantic distinction between causative and applicative not always as clear-cut as in the case of \dhatu{haːpt}{mourn} vs \dhatu{haːps}{cause to weep} . In particular, with motion verbs, applicative (\ref{ex:bring1}) and causative (\ref{ex:bring2}) derivations end up with the same semantic alternation, which may be the pivot form for reanalysis of the \ipa{-t} suffix as a causative in languages such as Khaling (\citealt{jacques15derivational.khaling}).
\begin{exe}
\ex \label{ex:bring1}
\glt `come with X' $\rightarrow$ `bring'
\ex \label{ex:bring2}
\glt `cause X to come' $\rightarrow$ `bring'
\end{exe}
 
In Limbu when the \ipa{-s} causative suffix is added to a final stop, it presents a synchronically unexplainable alternation between oral and nasal stops, shown in Table \ref{tab:Cs.limbu} (\citealt{driem87} \citealt[xiii]{michailovsky02dico}) .

\begin{table}[H]
\caption{Cs-stems in Limbu} \centering \label{tab:Cs.limbu}
\begin{tabular}{lllll}
\toprule
Present Stem & Past Stem & Root form \\
\midrule
\ipa{-ŋ-} & \ipa{-ks-} & \ipa{-ks} \\
\ipa{-ŋ-} & \ipa{-ŋs-} & \ipa{-ŋs} \\
\ipa{-m-} & \ipa{-ps-} & \ipa{-ps} \\
\ipa{-m-} & \ipa{-ms-} & \ipa{-ms} \\
\bottomrule
\end{tabular}
\end{table} 

In Bantawa, the nasalized allomorph has always been generalized, so that \ipa{-s} causative verbs only occurs with nasal codas. For instance, Bantawa \dhatu{ep}{stand} forms its causative as \dhatu{ems}{make stand, establish} (compare Limbu \dhatu{jɛp}{stand (vi)}  and \dhatu{jɛps}{stand sth up (vt)}).

The applicative appears to cause denasalization of nasal codas in some verbs in Limbu and Bantawa. A good example of this phenomenon is provided by the transitive verb \dhatu{makt}{see in a dream}	in Bantawa (same form in Limbu), which corresponds to the intransitive \dhatu{maŋ}{to dream} (cognate to Khaling \dhatu{moŋ}{dream}). There is evidence that the nasal coda is more ancient in this verb: in Limbu, although the simple intransitive verb is not attested, we find the noun \ipa{sɛpmaŋ} `dream' with a nasal, and cognates outside of Kiranti all have a nasal (Japhug \ipa{tɯ-jmŋo} < *\ipa{lmaˠŋ}, Chinese \zh{夢} *\ipa{C.məŋ-s} `dream'). The nasalization in applicative verb forms is not systematic; it may have been a regular morphological alternation at an early stage, which ceased to be operation later, so that \ipa{-ŋt} stem verbs were created afterwards in Limbu and Bantawa. It is thus only preserved in a few archaic applicative verbs.


Other examples of denasalization in applicative verbs include the following:

\begin{itemize}
\item Limbu \dhatu{juŋ}{sit} $\rightarrow $ \dhatu{jukt}{to sit on, to ride (e.g. a horse)} (cf Bantawa \dhatu{juŋ}{sit}, \dhatu{juŋs}{to put; to cause to sit})
\item Limbu \dhatu{haŋ}{send (sth)} $\rightarrow $ \dhatu{hakt}{send (sth) to} (cf Khaling \dhatu{ɦoŋ}{send})
\item Limbu \dhatu{tʰaŋ}{come from below} $\rightarrow $ \dhatu{tʰakt}{bring from below}
\item Bantawa \dhatu{tʰom}{dance} $\rightarrow $ \dhatu{tʰopt}{dance for someone}
\end{itemize}
 
\subsection{Deponent} \label{sec:deponent}
Not all verbs in \ipa{-t} are syntactically transitive. In both Limbu and Khaling, many impersonal verbs or verbs with experiencer (\citealt{michailovsky97deponent}) are only used in third singular, or in inverse forms.  

Examples of such verbs in Khaling include \dhatu{ʔopt}{rise (of the sun)},  \dhatu{dumt}{ripen} or \dhatu{ɦɛpt}{get stuck}. This \ipa{-t} suffix is probably unrelated to the \ipa{-t} applicative discussed in the previous section. 

\subsection{Reflexive} \label{sec:refl}
The reflexive / middle derivation in Kiranti is very productive in some languages (in Khaling, for instance, see \citealt{jacques16si}) and moribund in others (such as Wambule). 


Table \ref{tab:ip.vr} presents the paradigm of a common reflexive verb \dhatu{ʔip-si}{sleep}, derive from \dhatu{ʔipt}{put to sleep}. In the reflexive derivation, the complex finals Ct are always simplified, and one cannot distinguish the reflexive forms of transitive \ipa{-p} stems and \ipa{-pt} stems.

\begin{table}[h]
\label{tab:ip.vr} \centering 
\caption{Paradigm of \ipa{ʔʌ̂msinɛ}  `to sleep' in Khaling }
\begin{tabular}{l|l|l|l|l|l|l|l|l|l|l|l|l} 
\toprule
Form& Non-Past & Past & Imperative \\
\textsc{1s} & \ipa{ʔʌ̂m-si-ŋʌ}  \grise{}& \ipa{ʔʌ̂m-tʌsu} \grise{}&\\ 
\textsc{1di} &\ipa{ʔip-si-ji}  & \ipa{ʔip-sî-jti}  & \\
\textsc{1de} &\ipa{ʔip-si-ju}  & \ipa{ʔip-sî-jtu}  & \\ 
\textsc{1pi} &\ipa{ʔʌp-si-ki}  & \ipa{ʔʌp-si-ktiki} & \\ 
\textsc{1pe} &\ipa{ʔʌp-si-kʌ}  & \ipa{ʔʌp-si-ktʌkʌ} &  \\ 
\textsc{2s} & \ipa{ʔi-ʔʌ̂m-si}  \grise{}& \ipa{ʔi-ʔʌ̂m-tɛ-si} \grise{} & \ipa{ʔʌ̂m-sij-e}  \grise{}\\ 
\textsc{2d} & \ipa{ʔi-ʔip-si-ji}  & \ipa{ʔi-ʔipsî-jti}  & \ipa{ʔip-sî-jje}    \\
\textsc{2p} & \ipa{ʔiʔ-ʌ̂m-si-ni} \grise{} & \ipa{ʔi-ʔʌ̂m-tɛ-n-nu} \grise{} & \ipa{ʔʌ̂m-nu-je}  \grise{}\\ 
\textsc{3s} & \ipa{ʔʌ̂m-si}  \grise{}& \ipa{ʔʌ̂m-tɛ-si}  \grise{} &\\ 
\textsc{3d} & \ipa{ʔip-si-ji}  & \ipa{ʔip-sî-jti} &  \\ 
\textsc{3p} & \ipa{ʔʌ̂m-si-nu}\grise{}  & \ipa{ʔʌ̂m-tɛ-n-nu} \grise{}&\\ 
\bottomrule
\end{tabular}
\end{table}

We see that in Khaling, reflexive forms of verbs with a stop final root present an alternation between plain and nasalized codas (here indicated in grey). 

In most other Kiranti languages, the reflexive conjugation has been either completely (as in Wambule) or partially (as in Limbu) confused with the forms of the \ipa{-s} causative suffix paradigms by loss of the vowel \ipa{i}. For the verb `sleep', Limbu has an intransitive verb \dhatu{ʔips}{sleep} which does not follow the reflexive conjugation, and Bantawa has \dhatu{ʔims}{sleep}, with generalization of the nasalized allomorph. In Wambule, traces of the \ipa{-si} are very diffuse (\citealt[305]{opgenort04wambule}), and the main trace of the former reflexive derivation is the nasalization of the final stop.
 

\subsection{Anticausative} \label{sec:anticaus}
All Kiranti languages have intransitive / transitive pairs, in which the intransitive member has an initial going back to proto-Kiranti voiced initials, and the transitive one to unvoiced (aspirated or non-aspirated). Table \ref{tab:anticaus} presents examples of this alternation in Khaling (data from \citealt{jacques15derivational.khaling}).

\begin{table}[H]
\caption{Alternation between unvoiced and voiced aspirated verb roots in Khaling} \centering \label{tab:anticaus}
\begin{tabular}{lllll}
\toprule
Transitive & Intransitive&  \\
\midrule
\dhatu{tsɛm}{lose} & \dhatu{ʣʰɛm}{be lost} \\
\dhatu{tsɛp}{be able to do (sth)} & \dhatu{ʣʰɛp}{be possible} \\
\dhatu{kik}{tie} & \dhatu{gʰik}{hang oneself by accident} \\
\dhatu{pʰuk}{wake up} & \dhatu{bʰukt}{ferment (of alcohol)} \\
\dhatu{kʰlum}{bury} & \dhatu{gʰlum}{be deep} \\
\midrule
\dhatu{kɛnt}{make a hole} & \dhatu{gʰɛn}{get a hole} \\
\dhatu{kukt}{bend} & \dhatu{gʰuk}{be bent}\\
\bottomrule
\end{tabular}
\end{table}

As argued in \citet{jacques15derivational.khaling} this derivation originates not from the causative \ipa{s-} as has been repeatedly claimed, but from the anticausative derivation, as shown by the non-volitional / involuntary meaning of all intransitive verbs in these pairs (on the anticausative derivation in Sino-Tibetan, see  \citealt{jacques15spontaneous, jacques15causative, hill14voicing}). The base form is thus the transitive verb.\footnote{The hypothesis that the transitive verbs in these pairs are derived from the intransitive ones would be at pains to explains why transitive verbs can have both aspirated and unaspirated initial stops.}

It is possible to combine (intransitive) anticausative forms with the applicative \ipa{-t}, resulting in a double derivation whose outcome is transitive like the base verb. There is one example of such double derivation in Japhug, \dhatu{kɛnt}{make a hole} $\rightarrow$  \dhatu{gʰɛn}{get a hole} $\rightarrow$  \dhatu{gʰɛnt}{make a hole}. No clear synchronic semantic difference could be established between \dhatu{kɛnt}{make a hole} and \dhatu{gʰɛnt}{make a hole}. This type of derivation is common in Kiranti, and has obscured some of the correspondences between onsets across different languages (see section \ref{sec:fourth}).

\subsection{Isolated derivation} \label{sec:isolated}
Khaling has another isolated derivation between \dhatu{pil}{squeeze} and  \dhatu{pʰil}{squeeze out}, both transitive. The first verb must be the base form, as cognates are found elsewhere in Kiranti (for instance Bantawa \dhatu{bil}{squeeze (for juice)}) while the second has no known cognates. 


%%kɨma (kɨsa)	to be afraid	kis	to be afraid
%%kitma (kittu) vt (middle) to scare; to be afraid, scared
%%
%%%kinma (kintu) vt (middle) to frighten

In the absence of other examples, the origin and even the exact function of these alternations remains mysterious. It could be worthwhile however to look for comparable alternations in other Kiranti languages.

\subsection{Denominal verbs} \label{sec:denom}
Unlike Gyalrongic languages (\citealt{jacques14antipassive}), there are surprisingly few denominal verbs in Kiranti languages. Some denominal verbs are formed by zero derivations, but there is also meagre evidence for a denominal \ipa{-t}, better attested in West Himalayish languages such as Bunan (\citealt[426]{widmer14bunan}). Only the following examples have been found:
\begin{itemize}
\item Khaling \dhatu{ti}{lay an egg} (it) is derived from \ipa{ti} `egg' without overt derivational marker, but follows the vowel alternation paradigm of intransitive \ipa{-i} stem verbs. Bantawa \dhatu{dint}{lay eggs} and \dhatu{tʰiːnt}{lay an egg} derive from the corresponding nouns  \ipa{din} and \ipa{tʰiːn} `egg' respectively,\footnote{The \ipa{-n} coda in Bantawa and Limbu is probably suffixal.} but with a \ipa{-t} denominal suffix. Khaling on the one hand, and Bantawa and Limbu on the other hand, have thus separately built their verbs for `lay an egg'' from the same noun, but with a different morphological structure.
\item Wambule \dhatu{sja}{bear a fruit}, Bantawa  \dhatu{si}{bear fruit} and Khaling \dhatu{sit}{bear fruits} derive from the noun `fruit' (Khaling \ipa{si},  Bantawa \ipa{si}, Limbu \ipa{se}). Here Khaling differs from the other languages in having a denominal \ipa{-t}.
 \item The Limbu transitive verb \dhatu{kʰaːnt}{wound} derives with a denominal \ipa{-t} suffix from a noun not attested in Limbu, but found in Wambule \ipa{ɓari} `wound (n)' and Khaling \ipa{koɔ̄r} `wound (n)'
 \item Limbu \dhatu{sokt}{to aim, to point}, Khaling \dhatu{tsukt}{point (with a finger)}, Limbu \ipa{cok} `toe, finger' (which however comes from a proto-form *\ipa{dzʊk}, whereas the verb has an unvoiced initial).
 \item The Khaling \dhatu{kakt}{hoe} corresponds to the Japhug noun \ipa{qaʁ} `hoe' outside of Kiranti (perhaps also to Limbu \ipa{kaːŋ} `hoe', though the correspondence is difficult to explain): it is thus a fossilized \ipa{-t} denominal verb.
\end{itemize}  

\subsection{Summary of verb derivations in Kiranti}
The main derivations found in Kiranti languages discussed in the section can be summarized as follows:

\begin{itemize}
\item  Bi (base form, intransitive)
\item  Bt (base form, transitive)
\item  A (applicative form, t-transitive)
\item  C (causative form, s-transitive)
\item M (reflexive/middle form in -si/-s)
\item V (anticausative form with voicing of the onset)
\item D (impersonal, deponent, morphologically t-transitive), but only 3$\rightarrow$3 forms or inverse forms attested). This derivation is not observable within a single language, but can be deduced in case of intransitive verbs in a language corresponding to deponent t-transitive in another language, like  the correspondence of Bantawa \dhatu{tum}{ripen} to Khaling \dhatu{dumt}{ripen}.
\end{itemize}

Some derivations can be combined: A (and perhaps C) derivations can be applied to V forms (V+A), M derivations can be applied to A forms (A+M. The case of Khaling \dhatu{gʰɛnt}{make a hole} discussed in section \ref{sec:anticaus} is a clear examples of V+A, and it is likely that other examples of the same type exist in other Kiranti languages (see section \ref{sec:fourth}). 

Figure \ref{fig:kiranti.derivations} represents the tree of possible derivations; potential derivation pathways for which no clear example has been discovered yet are indicated by a dashed arrow. 

Note that in addition transitive roots can become intransitive by the effect of unmarked antipassive, which is especially common in Southern Kiranti (Bantawa and Puma; see for instance \citealt{bickel07puma}). This may be one of the origins of deponent t-transitive verbs.

   \begin{figure}[H]
   \caption{Verb derivations in Kiranti languages} \label{fig:kiranti.derivations}  
  \begin{tikzpicture}
  \node (Bi) at (-2,1) {Bi *\ipa{kʰraːp} (vi)};
  \node (Bt) at (4,1) {Bt *\ipa{kik} (vt)};
  \node (V) at (1,-1) {V *\ipa{gik} (vi)};
 \node (Ai) at (-1.5,-1) {A *\ipa{kʰraːp-t} (vt)};
   \node (AiM) at (-4,-4) {A+M *\ipa{kʰraːp(t)-si} (vi)};
  \node (Ci) at (-4.5,-1) {C *\ipa{kʰraːp-s} (vt)};
 \node (At) at (3,-2) {A *\ipa{kik-t} (vt)};
  \node (Ct) at (6,-2) {C *\ipa{kik-s} (vt)};
  \node (M) at (6,-1) {M *\ipa{kik-si} (vi)};
  \node (AtM) at (6,-4) {A+M *\ipa{kik(t)-si} (vi)};
  \node (VA) at (2,-4) {V+A *\ipa{gik-t} (vt)};
  \node (VC) at (0,-5) {V+C *\ipa{gik-s} (vt)};
  \node (VAM) at (4,-6) {V+A+M *\ipa{gik(t)-si} (vi)};
    
\tikzstyle{peutetre}=[->,dotted,very thick,>=latex]
\tikzstyle{sur}=[->,very thick,>=latex]
\draw[sur] (Bt)--(V);
\draw[sur] (Bi)--(Ai);
\draw[sur] (Bi)--(Ci);
\draw[sur] (Bt)--(At);
\draw[sur] (Bt)--(M);
\draw[sur] (Bt) to[bend right] (Ct);
\draw[sur] (Ai) to[bend left] (AiM);
\draw[sur] (At) to[bend right] (AtM);
\draw[sur] (V)--(VA);
\draw[peutetre] (VA)--(VAM);
\draw[peutetre] (At)--(V);
\draw[peutetre] (V) to[bend right] (VC);

\end{tikzpicture}
\end{figure}

\section{Semantics} \label{sec:semantic}
Cognate verbs between Kiranti languages not only have slightly different morphological structure, they may also present slightly divergent semantics. In many cases, the divergence is due to the way individual fieldworkers gloss verb for which no simple equivalent exist in English.

In some cases, semantic differences across languages are due to the fact that the meaning of a verb became restricted to a secondary meaning in one language. 

For instance, the Khaling verb \dhatu{glumt}{brood} is related to Limbu \dhatu{kupt}{sleep with}. The meaning `brood' also appears in Limbu with the complex predicate \dhat{ha kupt}: Khaling has thus restricted the meaning of this root to a particular derived sense.\footnote{Khaling \dhatu{glumt}{brood}and Limbu  \dhatu{kupt}{sleep with} are themselves applicatives of a root \ipa{*glum} `lie down', still directly attested by Wambule  \dhatu{glwam}{lie down} and Khaling \dhatu{glum}{be deep}.  }

Another such example is provided by the root *\ipa{ri}, attested by Khaling \dhatu{ri}{become dizzy} and Limbu \dhatu{iːr}{turn, go around}. While the dictionary glosses of these two verbs may appear to be completely different, they are both actually used in a similar noun-verb collocation meaning `dizzy' with the cognate nouns \ipa{-nûː} (from \ipa{*-nik}) and \ipa{-niŋwa} respectively, both meaning `mind', as in examples (\ref{ex:unu}) and (\ref{ex:kuningwa}).

\begin{exe}
\ex \label{ex:unu}
\gll \ipa{ʔu-nûː} \ipa{rūː-tɛ} \\
\textsc{3sg.poss}-mind be.dizzy-\textsc{2/3:pst} \\
\glt He got dizzy.
\end{exe}


\begin{exe}
\ex \label{ex:kuningwa}
\gll \ipa{ku-niŋwa} \ipa{iːr-ɛ} \\
\textsc{3sg.poss}-mind turn-\textsc{2/3:pst} \\
\glt He got dizzy.
\end{exe}



In other cases, the meaning of the proto-root is unclear. This is for instance the case of the root \ipa{*riŋ}, attested by Khaling \dhatu{riŋ}{praise}, Bantawa \dhatu{jɨŋ}{say, tell} and Limbu \dhatu{iŋ}{become famous}: it is not obvious here which languages innovated, and which semantic changes or restrictions (or extentions) took place, but such cases are a minority in the database.

There are clear semantic innovations common to all Kiranti languages. Two cases are discussed here.

First, although nearly all Kiranti languages have the applicative verb *\ipa{ʔipt} `put to sleep', no language has the simplex verb *\ipa{ʔip} `sleep', from a root widespread in the Trans-Himalayan family (for example cognate to Japhug \ipa{-ʑɯβ} `sleep' (<*\ipa{jip}). Apart from a few languages which have a different root for `sleep' (for instance Wambule), most Kiranti languages have a rzeflexive verb or an intransitive \ipa{-s} verb from the same root, as Khaling \dhatu{ʔip-si}{sleep} or Limbu \dhatu{ips}{sleep}. Kiranti languages here have replaced the simplex verb by the reflexive of the applicative, which is undistinguishable from a direct derivation from the bare root: *\ipa{ʔip-t-si} > *\ipa{ʔip-si}; this innovation must have occurred in proto-Kiranti.


Second,  the root *\ipa{tʊk} `bump into' (transitive)\footnote{This root is attested by Khaling \dhatu{dʰuk}{bump into} (from the anticausative  *\ipa{dʊk}), Bantawa  \dhatu{dʰuŋs}{bump} (from the reflexive *\ipa{tuk-si}) and Limbu \dhatu{tokt}{to bump against sth., to trip, to stumble}  (from the anticausative  *\ipa{dʊk}).} is apparently cognate with that of Japhug \ipa{tɯɣ} `touch, be in contact with' and \ipa{atɯɣ} `meet' and Tibetan \ipa{tʰug} `touch, meet', with a semantic extension also attested in Sanskrit with the root \ipa{ṛ} .
(\citealt[3]{pooth15eri}).

\section{Onsets} \label{sec:onsets}

\subsection{Stops and affricates} \label{sec:stops}

Previous work on proto-Kiranti (in particular \citealt{starostin94kiranti}, \citealt{michailovsky94stops} and \citealt{opgenort05jero}) have already sorted out the correspondences of stops and affricates. Table (\ref{tab:stops}) presents the regular correspondences between the four languages under study, and Table \ref{tab:stops.ex} presents examples for each of the proto-phonemes.\footnote{The conventions for symbols such as double chevron in the reconstructions are explained in the appendix (\ref{sec:appendix}).}


The reconstructions here follow \citet{michailovsky94stops} in reconstructing three series of stops.  All proto-phonemes in Table (\ref{tab:stops}) except *\ipa{ʔw} and *\ipa{ʦʰ} are attested by a least five etymologies.

Unlike \citet{starostin94kiranti} and \citet{opgenort05jero}, I do not reconstruct a series of preglottalized unvoiced stops *\ipa{`p},  *\ipa{`t},  *\ipa{`ts},  *\ipa{`k} opposed to the plain unvoiced stops (see section \ref{sec:fourth}). 


\begin{table}[h]
\caption{Proto-Kiranti stops and their reflexes} \centering \label{tab:stops}
\begin{tabular}{llllll}
\toprule
Proto-Kiranti & Wambule & Khaling & Bantawa & Limbu \\
\midrule
\ipa{*p} & \ipa{p} & \ipa{p} & \ipa{b} & \ipa{pʰ}  \\
\ipa{*t} & \ipa{t} & \ipa{t} & \ipa{d} & \ipa{tʰ}  \\
\ipa{*ʦ} & \ipa{c} & \ipa{ʦ} & \ipa{cʰ} & \ipa{s}  \\
\ipa{*k} & \ipa{k} & \ipa{k} & \ipa{kʰ} & \ipa{kʰ}  \\
\midrule
\ipa{*ʔw} & \ipa{ɓ} & \ipa{k} & \ipa{g} & \ipa{kʰ}  \\
\midrule
\ipa{*pʰ} & \ipa{pʰ} & \ipa{pʰ} & \ipa{b} / \ipa{bʰ}  & \ipa{pʰ}  \\
\ipa{*tʰ} & \ipa{tʰ} & \ipa{tʰ} & \ipa{d} / \ipa{dʰ} & \ipa{tʰ}  \\
\ipa{*ʦʰ} & ? & \ipa{ʦʰ} & \ipa{cʰ}  & \ipa{s}  \\
\ipa{*kʰ} & \ipa{kʰ} & \ipa{kʰ} & \ipa{kʰ} & \ipa{kʰ}  \\
\midrule
\ipa{*b} & \ipa{b} &\ipa{b} / \ipa{bʰ}  & \ipa{p} & \ipa{p}  \\
\ipa{*d} & \ipa{d} & \ipa{d} / \ipa{dʰ}  & \ipa{t} & \ipa{t}  \\
\ipa{*ʣ} & \ipa{ʣ} & \ipa{ʣʰ} & \ipa{c} & \ipa{c}  \\
\ipa{*g} & \ipa{g} & \ipa{g} / \ipa{gʰ}  & \ipa{k} & \ipa{k}  \\
\bottomrule
\end{tabular}
\end{table}
These correspondences account for the great majority of comparanda between the four languages; exceptions are treated by assuming  morphological alternations of the type described in section (\ref{sec:alternations}).

\begin{table}
\caption{Proto-Kiranti stops and affricates: representative examples} \centering \label{tab:stops.ex}
\begin{tabular}{lllllll}
\toprule
Proto-Kiranti & Meaning & Wambule & Khaling & Bantawa & Limbu\\
\midrule
\ipa{*pil > pil-t} &	milk &	&	\dhat{pil} &	\dhat{bʰil} &	\dhat{pʰiːnt} &	\\
\ipa{*pɔ >{}> pa} &	weave &	\dhat{pa} &	\dhat{pa} &	\dhat{ba} &	\dhat{pʰɔ} &	\\
\ipa{*pot > pot-t} &	tie up &	\dhat{pwat} &	\dhat{pott} &	&	&	\\
\midrule			
\ipa{*tuŋ} &	drink &	\dhat{tu:(s)} &	\dhat{tuŋ} &	\dhat{duŋ} &	\dhat{tʰuŋ} &	\\
\ipa{*tup > tup-t} &	beat &	\dhat{tupt} &	\dhat{tup} &	\ipa{-dʰup-} (n)&	&	\\
\ipa{*tan > tan-t} &	drop &	&	\dhat{tɛnt} &	\dhat{dant} &	\dhat{(mut) thaːnt} &	\\
\midrule				
\ipa{*tsor} &	pay &	&	\dhat{ʦor} &	\dhat{cʰor} &	&	\\
\ipa{*tsikt} &	know &	\dhat{cikt} &	\dhat{ʦikt} &	&	&	\\
\ipa{*tsek} &	pinch &	 &	\dhat{ʦek} &	\dhat{cʰɨk} &	\dhat{sekt} &	\\
\ipa{*tsept > tsep-s} &	taste &	&	\dhat{ʦept} &	\dhat{cʰems} &	&	\\
\midrule
\ipa{*kɔkt} &	chop &	\dhat{kwakt} &	\dhat{kokt} &	\dhat{kʰokt} &	\dhat{kʰɔkt} &	\\
\ipa{*k[e|i]} &	fight &	&	\dhat{ki} &	\dhat{kʰi} &	\dhat{kʰe} &	\\
\ipa{*k[e|i]p > kep-t } &	stick &	\dhat{kjap} &	\dhat{kept} &	\dhat{kʰept} &	\dhat{kʰipt} &	\\
\midrule				
\ipa{*pʰrok > pʰrok-si} &	untie &	\dhat{pʰrwaŋ} &	\dhat{pʰrok} &	&	\dhat{pʰaːks} &	\\
\ipa{*pʰek > pʰek-t} &	flick away &	\dhat{pʰjak} &	\dhat{pʰekt} &	&	 &	\\
\ipa{*pʰlɑs} &	help &	&	\dhat{pʰlo} &	\dhat{pʰas} &	\dhat{pʰaˀr} &	\\
\midrule				
\ipa{*tʰin > tʰin-t} &	wake &	&	\dhat{tʰint} &	\dhat{tʰint} &	&	\\
\ipa{*tʰar > tʰar-t} &	make stand &	\dhat{tʰar} &	&	\dhat{tʰant} &	&	\\
\ipa{*tʰʊk} &	spit at &	\dhat{tʰuk} &	&	\dhat{tʰukt} &	\dhat{tʰokt} &	\\
\midrule		
\ipa{*tsʰu[ŋ|k]-si} &	cough &	&	&	\dhat{cʰuŋs} &	\dhat{suks} &	\\
\ipa{*tsʰus} &	be late &	&	\dhat{ʦʰu} &	&	\dhat{sus} &	\\
\ipa{*tsʰem > tsʰem-t} &	lure &	&	\dhat{ʦʰem} &	\dhat{cʰemt} &	&	\\
\midrule
\ipa{*kʰ[u|o]l} &	transport &	&	\dhat{kʰol} &	\dhat{kʰol} &	\dhat{kʰuˀr} &	\\
\ipa{*kʰ[u|o]tt} &	comb &	&	\dhat{kʰutt} &	\dhat{kʰɨtt} &	\dhat{kʰott} &	\\
\ipa{*kʰa} &	be bitter &	\dhat{kʰa} &	\dhat{kʰɛ} &	&	&	\\
\midrule
\ipa{*bor} &	grow &	\dhat{bwar} &	\dhat{bʰor} &	\dhat{por} &	\dhat{por} &	\\
\ipa{*b[ɛː|i]r} &	fly &	&	\dhat{bʰer} &	\dhat{pir} &	\dhat{pɛːr} &	\\
\ipa{*bi} &	give &	&	\dhat{bi} &	\dhat{pɨ} &	\dhat{pi} &	\\
\midrule
\ipa{*dɔ >{}>  da} &	dig &	\dhat{dwa} &	\dhat{dʰA} &	\dhat{tu} &	\dhat{tɔ} &	\\
\ipa{*dɔŋ} &	be friends &	&	\dhat{dʰoŋ} &	\dhat{toŋ} &	\dhat{tɔŋ} &	\\
\ipa{*duk > dukt} &	hurt &	&	&	\dhat{tuk} &	\dhat{tuk} &	\\
\midrule				
\ipa{*dza} &	eat &	\dhat{dza} &	\dhat{dzA} &	\dhat{ca} &	\dhat{ca} &	\\
\ipa{*dzɑ:k} &	swim across &	\ipa{dzwak } (n) &	&	\dhat{cak} &	\dhat{caːk} &	\\
\ipa{*dzit} &	be wet &	\dhat{dzit}   &	\dhat{ʣʰit} &	&	&	\\
\midrule		
\ipa{*gɑŋ} &	warm &	&	\dhat{gʰoŋ} &	\dhat{kaŋt} &	\dhat{kaŋ} &	\\
\ipa{*gam} &	fit together &	&	\dhat{gʰɛm} &	\dhat{kams} &	\dhat{kam} &	\\
\ipa{*glumt} &	sleep with &	\dhat{glwam} &	\dhat{glumt} &	\dhat{kupt} &	\dhat{kupt} &	\\
\bottomrule
\end{tabular}
\end{table}

The shift from *\ipa{ʦʰ} to \ipa{s} in Limbu probably postdates contact with Tibetan, as we find the likely loanword Limbu \dhatu{sɔŋs}{sell} from \ipa{ɴtsʰoŋ, btsoŋs} `sell'.


The cognate set for `be sour', attested by Wambule |\ipa{dzur}|, Khaling |\ipa{dzʰur}|, Bantawa |\ipa{sunt}| and Limbu |\ipa{suːtt}| presents an unexplained correspondence of \ipa{s} in Bantawa and Limbu to a voiced affricate in Wambule and Khaling. Note that the rhyme of the Limbu cognate is also irregular.

Affricates of other languages correspond to Wambule \ipa{s-} in a few cases indicated in Table \ref{tab:affricates.wambule}. Since \dhatu{sart}{soak with urine} also presents an irregular vowel correspondence, these examples might  be borrowing from a Kiranti language where *\ipa{ts-} changes to \ipa{s-}.

\begin{table}[H]
\caption{Irregular correspondences of affricates in Wambule} \centering \label{tab:affricates.wambule}
\begin{tabular}{llllll}
\toprule
Khaling & Wambule \\
\midrule
\dhatu{ʦɛm}{lose} & \dhatu{samt}{lose}\\
\dhatu{ʦer}{urinate} & \dhatu{sart}{soak with urine} \\
\dhatu{ʦu}{be spicy} & \dhatu{su}{be hot, spicy}\\
\bottomrule
\end{tabular}
\end{table}

\subsubsection{Voicing polarity in Bantawa} \label{sec:polarity}
The polar correspondence between (labial and dental) voiced and unvoiced stops between Wambule and Khaling on the one hand, and Bantawa on the other hand, suggests the existence of a complex chain shift. All PK voiced stops and affricates become unvoiced in Bantawa, but PK unvoiced unaspirated obstruents  present divergent correspondences: *\ipa{p} and *\ipa{t} remain distinct and become voiced, while *\ipa{k} and *\ipa{ts} merge with *\ipa{kʰ} and *\ipa{tsʰ} and become aspirated.

The absence of merger between PK *\ipa{b} and *\ipa{d} on the one hand, and *\ipa{p} and *\ipa{t} on the other hand, implies that the unvoiced stops went through an intermediate stage distinct from both voiced and unvoiced stops during this sound shift. A possible solution (suggested to me by Marc Miyake, p.c.), involves the shift of voiced unaspirated stops to implosives, then to voiced stops, as indicated in Table \ref{tab:bantawa}.

The change of labial and dental stops to implosives is well attested in South-East Asia, in particular in Vietnamese (\citealt{ferlus82spirantisation}), and would account for the differential treatment of affricates and velars (which would not undergo implosivisation). 

 

\begin{table}[H]
\caption{Stop shift in Bantawa} \centering \label{tab:bantawa}
\begin{tabular}{lllll}
\toprule
Proto-Kiranti &  Pre-Bantawa &   Bantawa \\
\midrule
\ipa{*p} & *\ipa{ɓ} & \ipa{b} \\
\ipa{*t} & *\ipa{ɗ} & \ipa{d} \\
\ipa{*pʰ} & *\ipa{pʰ} &  \ipa{pʰ} \\
\ipa{*tʰ} & *\ipa{tʰ} & \ipa{tʰ} \\
\ipa{*b} & *\ipa{b} &  \ipa{p} \\
\ipa{*d} & *\ipa{d} & \ipa{t} \\
\bottomrule
\end{tabular}
\end{table}
 
 Split voicing is however attested in various families without an implosive stage (\citealt[49-51]{kuemmel07wandel}), so that alternative possibilities exist to account for this polarity.
 

\subsubsection{A fourth series?} \label{sec:fourth}
The plain unvoiced stops are reconstructed by Starostin and Opgenort  when the unvoiced unaspirated stops of Khaling and Wambule correspond to unvoiced unaspirated stops in Bantawa (as opposed to expected voiced stops) and to unaspirated stops in Limbu (as opposed to aspirated stops). Such examples are rare, though not unattested. 

I found a handful of candidates for plain *\ipa{p}, *\ipa{k} and \ipa{*ts} in the systems of Starostin and Opgenort (see Table \ref{tab:kkk}). However, this reconstruction is problematic: as \citet[17]{opgenort05jero} himself notices, plain unaspirated stops in his system are very rare, and there are barely any plain *\ipa{t} in his system (and none in verb roots), while *\ipa{`t} is very common.  

Moreover, other types of correspondences not accounted for by Table \ref{tab:stops} are also found in Kiranti, including Khaling  unaspirated to Bantawa aspirated, Khaling unaspirated to Bantawa unaspirated, and Khaling voiced to Bantawa voiced. Reconstructing distinct series of stops to account for each of these correspondences is not useful, especially given the limited number of examples. 

\begin{table}[h]
\caption{Irregular correspondences} \centering \label{tab:kkk}
\resizebox{\columnwidth}{!}{
\begin{tabular}{llllll}
\toprule
Type&   Khaling & Bantawa   \\
   \midrule
1&  \dhatu{kept}{sting}  & \dhatu{kept}{sting}  \\
  &\dhatu{keŋ}{cool down}  & \dhatu{keŋ}{be cold}  \\
% &\dhatu{kaŋt}{put over heat}  & \dhatu{kaŋt}{be heated (at the edge of the fire)}      \\
& \dhatu{pum}{hold in one's fist, make a fist}  & \dhatu{pumt}{hold tightly (in the fists)}     \\
 & \dhatu{prɛm}{scratch, claw} &  \dhatu{pamt}{scratch, tear}  \\
& \dhatu{ʦent}{teach}  & \dhatu{cint}{teach}  \\
 \midrule
2&\dhatu{kʰop}{gather}  & \dhatu{kapt}{put together} \\
&\dhatu{pʰuk}{get up} &  \dhatu{puk}{stand up, to rise} \\
 \midrule
3&\dhatu{bʰokt}{patch} &  \dhatu{bʰokt}{patch}  \\
&\dhatu{dʰuk}{bump into} &  \dhatu{dʰuŋs}{bump}  \\
\bottomrule
\end{tabular}}
\end{table}

A more likely solution is to invoke morphological alternations. As shown in section (\ref{sec:alternations}), Kiranti languages have alternations between voiced and unvoiced stops, the former being an intransitive verb and the latter its transitive counterpart, most probably a remnant of the anticausative derivation, and also have a less well understood alternation between unvoiced unaspirated and aspirated stops.

 Type 1 irregularities could be accounted for by positing  a development similar to that of Khaling \dhatu{kɛnt}{make a hole} and \dhatu{gʰɛnt}{make a hole} (cf section \ref{sec:anticaus}) in Bantawa. Thus, Bantawa \dhatu{kept}{sting} could be from a form *\ipa{gept} derived from proto-Kiranti *\ipa{kept} `sting'' (directly reflected by Khaling \dhatu{kept}{sting}) in two steps, first anticausative voicing  *\ipa{kept} `sting'' $\rightarrow$  *\ipa{gep} `get a sting''  (which disappeared) and then applicative/causative *\ipa{-t} to *\ipa{gept} `sting'', which regularly yields Bantawa \dhatu{kept}{sting}.\footnote{Proto-Kiranti *\ipa{kept} would yield Bantawa $\dagger$|\ipa{kʰept}|.}
 
Likewise, for Bantawa \dhatu{keŋ}{be cold}, we would have first *\dhatu{keŋ}{be cold} $\rightarrow$ *\dhatu{keŋt}{cool down (tr)} then anticausative *\dhatu{geŋ}{be cooled}, then by regular sound change Bantawa \dhatu{keŋ}{be cold}.
 
 Alternatively, this correspondence could reflect loanwords either from a language with a Bantawa-type phonology into Khaling or the other way round.
 
The same type of hypothesis can account for type 2 irregularities, except that in this case the base root  had an aspirated stop instead of a plain unvoiced one; we saw in section  (\ref{sec:anticaus}) that the anticausative derivation yields voiced obstruents from both unvoiced aspirated and unaspirated stops and affricates. Thus, from a proto-Kiranti root \ipa{*pʰʊk} `wake s.o. up' (transitive) directly ancestral to the Khaling labile verb \dhatu{pʰuk}{get up, wake s.o. up}, we have an anticausative \ipa{*buk} `get up', directly attested by Wambule \dhatu{buk}{get up}, Bantawa \dhatu{puk}{get up}, Limbu \dhatu{pok}{get up} and even Khaling \dhatu{bʰukt}{ferment} (see section \ref{sec:anticaus}) with a specialized meaning. In this particular case, the only real issue is the fact that the transitive root \ipa{*pʰʊk} `wake s.o. up' has become labile in Khaling, and superseded the anticausative in the meaning `get up, wake up (it)'.

For Bantawa  \dhatu{kapt}{put together}, I likewise posit *\ipa{kʰɑp} `gather (tr)' (ancestor of Khaling \dhatu{kʰop}{gather}) $\rightarrow$  *\ipa{gɑp} `get together' (unattested) $\rightarrow$  *\ipa{gɑp-t} `put together'.
 
Type 3  irregularities can be explained as the inverse situation. Khaling \dhatu{bʰokt}{patch} could originate from a transitive root *\ipa{pokt} directly ancestral to Bantawa \dhatu{bʰokt}{patch}, to which anticausative derivation to *\ipa{bok} `be patched' has been applied, followed by the applicative/causative \ipa{-t}, yielding the attested Khaling form \dhatu{bʰokt}{patch}. 

For the intransitive verb \dhatu{dʰuk}{bump into}, the hypothesis is slightly different. We can suppose a proto-Kiranti root *\ipa{tʊk} `bump into (tr)', which undergoes reflexive derivation (see section \ref{sec:bantawa}) in Bantawa to *\ipa{tuŋs} then \ipa{dʰuŋs} by regular sound change. Khaling \dhatu{dʰuk}{bump into} on the other hand is the anticausative of the root *\ipa{tʊk} `bump into (tr)', which did not leave a direct descendent in the sample of languages used in this paper.
 
 This double derivation hypothesis used to explain type  and irregularities specifically predicts the absence of transitive verbs with voiced initial in Khaling (resp. unvoiced initial in Bantawa) corresponding to transitive verbs with voiced initial in Bantawa (resp.unvoiced initial in Khaling) if the former has no \ipa{-t} complex coda. No such example has been found up to now.
 
The following irregularities cannot be accounted for by any of the the scenarios proposed above:

\begin{enumerate}
\item  Khaling \dhatu{pit}{come (horizontal plane)} / \dhatu{pit}{bring (horizontal plane)} vs Wambule \dhatu{pʰit}{bring (horizontal plane)}. The Limbu cognate \dhatu{pʰɛn}{come}	is ambiguous, since \ipa{pʰ} can come from either \ipa{*p} or \ipa{*pʰ}, but this verb also has an irregular rhyme. 
\item Khaling  \dhatu{tʰopt}{beat}, Bantawa \dhatu{dʰapt}{to wash clothes usually beating with a club}
\item Khaling  \dhatu{klekt}{plaster, rub (with oil)}, Wambule \dhatu{kʰljakt}{rub one's body}
 \end{enumerate}

\subsubsection{\ipa{ʔw}}
The correspondence reconstructed here as *\ipa{ʔw} following \citet{michailovsky94stops} (alternatively reconstructed as *\ipa{kw} by \citealt{opgenort04implosives}) is only attested in two verbs *\ipa{ʔwa} `eat (hard food)', attested in Khaling \dhatu{ka}{eat (hard food)} and Wambule \dhatu{ɓa}{eat by biting, bite} and  *\ipa{ʔwɑp} `cover', attested by  Khaling \dhatu{kopt}{cover}, Wambule \dhatu{ɓwap}{cover, snare}, Bantawa \dhatu{kʰapt}{to thatch a roof} and Limbu  \dhatu{kʰaps}{to put on (a cover or blanket)}. 
 
 This correspondence is otherwise well attested in monosyllabic nouns, in particular *\ipa{ʔwi} `yam, potato' (Khaling \ipa{ki}, Limbu \ipa{kʰe}).\footnote{The verb `cover' show that the Bantawa reflex of *\ipa{ʔw} is \ipa{kʰ}; thus, nouns such as \ipa{saki} `potato, yam' and \ipa{ogi} `sweet potato' must be borrowings.}


\subsection{Clusters} \label{sec:clusters}
The only initial clusters that are clearly reconstructible in Kiranti are those in labial or velar stops followed by *\ipa{l} or *\ipa{r}. Table \ref{tab:clusters} presents all available evidence in the four target languages.

\begin{table}[H]
\caption{Proto-Kiranti clusters and their reflexes} \centering \label{tab:clusters}
\resizebox{\columnwidth}{!}{
\begin{tabular}{llllll}
\toprule
PK & Wambule & Khaling & Bantawa & Limbu \\
\midrule
\ipa{*plept} &  \dhatu{pljap}{fold over}& \dhatu{plept}{fold} &  x & x  \\
\ipa{*plent} &  \dhatu{plej}{refrain from}& \dhatu{plent}{postpone} &  x & x  \\
\ipa{*pram} &  x& \dhatu{prɛm}{scratch, claw} &  \dhatu{pamt}{scratch, tear} & x  \\
\ipa{*prɑŋ} &  \dhatu{proŋ}{weave} &x&  \dhatu{bʰaŋt}{weave} & x  \\
\ipa{*pʰrɑːk} &  \dhatu{pʰrwaŋ}{untie}& \dhatu{pʰrok}{untie} & x &  \dhatu{pʰaːks}{untie}  \\
\ipa{*blaːt} &  x& \dhatu{blɛtt}{tell, explain} &  \dhatu{paːt}{speak} &  x \\
\ipa{*blept} &  x& \dhatu{bʰlept}{flatten}   &  \dhatu{pemt}{press} &x \\
\ipa{*blim} &  \dhatu{blimt}{soak}   &\dhatu{blum}{be submerged}&  x & x  \\
\ipa{*bluːt} &  x& \dhatu{bʰlitt}{boil} &  \dhatu{putt}{boil over} &  \dhatu{puːtt}{boil over} \\
\ipa{*brɑt} &  x& \dhatu{bʰrot}{shout} &  \dhatu{pat}{cry out, shout} &  x \\
\midrule
\ipa{*kruːk}  &   & \dhatu{kruk}{roar} &  x&  \dhatu{huːkt}{roar} \\
\ipa{*klek} / \ipa{*kʰlekt}&  \dhatu{kʰljakt}{rub (body)} & \dhatu{klekt}{rub (with oil)} &  x&  x \\
\ipa{*kʰlum} &  \dhatu{kʰlwamt}{put to sleep} & \dhatu{kʰlum}{bury} &  \dhatu{kʰumt}{bury} &  \dhatu{hum}{sink} \\
\ipa{*kʰraːp} &    \dhatu{kʰram}{cry, weep}& x &  \dhatu{kʰap}{cry} &  \dhatu{haːp}{weep} \\
\ipa{*kʰris} & x &  \dhatu{kʰri}{help to walk}&  x &  \dhatu{his}{cause to turn} \\
\ipa{*gla[ŋ|k]} &    \dhatu{glak}{win}&  \dhatu{gʰlaŋ}{win}  &  x& x \\
\ipa{*glu[p|m]} &   \dhatu{glwam}{lie down}&  \dhatu{glumt}{brood}  &   \dhatu{kupt}{sit on eggs}& \dhatu{kupt}{sleep with} \\
\ipa{*grikt} &  x&  \dhatu{gʰrikt}{take}  &  \dhatu{kɨkt}{grab, take}& x \\
\bottomrule
\end{tabular}}
\end{table}

Clusters are well-preserved in Wambule and Khaling, and completely lost in Bantawa and Limbu. In Limbu, voiceless velar + *\ipa{l} or *\ipa{r} clusters become \ipa{h}. Wambule and Khaling  agree on the clusters in all roots in the corpus, except in two cases. First, the root represented by Khaling \dhatu{pʰekt}{flick away} is interesting in having two corresponding forms in Wambule: |\ipa{pʰjak}| without cluster and \dhatu{pʰrjak}{flick away}, suggesting the possibility of an \ipa{-r-} infix.\footnote{On the reconstruction of an *\ipa{-r-} infix in Sino-Tibetan, see \citet[111-120]{sagart99roc}.} Second, Khaling \dhatu{kat}{bite} corresponds to Wambule  \dhatu{krat}{bite with molar, gnaw}, here without alternating form.

The correspondence between Khaling \dhatu{prɛm}{scratch, claw} and Bantawa  \dhatu{pamt}{scratch, tear} is irregular (Bantawa $\dagger$\ipa{bʰamt} would be expected). This correspondence belongs to the Type 1 irregularity of section (\ref{sec:fourth}) and can be explained by the same model.

There is one case where the initial clusters appears to have undergone metathesis to a coda, namely Khaling \dhatu{pʰlo}{help} corresponding to Limbu \dhatu{pʰaˀr}{help}; as shown in section (\ref{sec:codas}), Limbu \ipa{-ˀr} is the regular reflex of proto-Kiranti *\ipa{-l}. Given Bantawa \dhatu{pʰas}{help}, this root can be reconstructed as *\ipa{pʰlɑs} `help' in proto-Kiranti (see section \ref{sec:rhymes} for more details on vowels and codas); Bantawa did not undergo metathesis like Limbu.
 
\subsection{Rhotics} 
As shown by  \citet{driem90r}, proto-Kiranti *\ipa{r} changes to \ipa{j} in Limbu. However, the correspondences of \ipa{r} and \ipa{j} across Kiranti languages is not as straightforward as might seem at first glance.

Khaling has both \ipa{r} and \ipa{j} initials, which both correspond to \ipa{j} (zero before \ipa{i} and \ipa{e}) in Limbu, as shown by Khaling \dhatu{ret}{laugh} and  \dhatu{jil}{make soft by squeezing} to Limbu \dhatu{et}{laugh} and \dhatu{iˀr}{rub} respectively.

In Bantawa, Khaling \ipa{r} mostly corresponds to \ipa{r}, except in the verbs \dhatu{i}{laugh}, \dhatu{ep}{stand} and \dhatu{jɨŋ}{say, tell}. Since nearly all other examples have \ipa{r} before non-front vowels, I propose that proto-Kiranti *\ipa{r} corresponds to \ipa{r} in Bantawa before back vowels and zero before front vowels. Exceptions such as \dhatu{ript}{twist} (same form and meaning in Khaling and Bantawa) may be borrowings.

In Wambule, Khaling \ipa{r} corresponds to \ipa{r} except in case of the root `to stand' (Khaling |\ipa{rep}|, Wambule  |\ipa{jam}|); also causative \dhatu{japs}{erect, make stand upright}). It is interesting to note that the applicative form of the same root presents a correspondence of \ipa{r} to \ipa{r}, namely Khaling \dhatu{rept}{respect (so's words)} to Wambule \dhatu{rjapt}{obey so, heed so's words}.\footnote{We see the same unpredictable semantic change in Khaling and Wambule `stand for, by'' $\rightarrow$ `respect''.} One of the correspondences (most likely \ipa{r} to \ipa{j} before front vowel) reflects the inherited layer, while the other is borrowed.  

\subsection{Other consonants}
The correspondences of all other initial consonants is relatively straightforward. I follow \citet{opgenort04implosives} in reconstructing a series of preglottalized nasals. The initial \ipa{*ʔŋ} is tentatively reconstructed on the basis of the comparison between Khaling \dhatu{ŋol}{mix} and Wambule \dhatu{ɓwalt}{mix}.

The only noteworthy sound change is the merger of *\ipa{ŋ} and \ipa{n} to *\ipa{n} in Limbu, and the partial shift of \change{ŋ-}{n-} before front vowel in Bantawa, as in \dhatu{nett}{disturb, irritate, annoy} corresponding to Khaling \dhatu{ŋet}{hurt}.

\begin{table}[H]
\caption{Proto-Kiranti nasals, lateral and sibilant intials} \centering \label{tab:nasals}
\begin{tabular}{llllll}
\toprule
Proto-Kiranti & Wambule & Khaling & Bantawa & Limbu \\
\midrule
\ipa{*m} & \ipa{m} & \ipa{m} & \ipa{m} & \ipa{m}  \\
\ipa{*ʔm} & \ipa{ɓ} & \ipa{m} & \ipa{m} & \ipa{m}  \\
\ipa{*n} & \ipa{n} & \ipa{n} & \ipa{n} & \ipa{n}  \\
\ipa{*ʔn} & \ipa{ɗ} & \ipa{n} & \ipa{n} & \ipa{n}  \\
\ipa{*ŋ} & \ipa{ŋ} & \ipa{ŋ} & \ipa{n} / \ipa{ŋ} & \ipa{n}  \\
\ipa{*ʔŋ} &  \ipa{ɓ}  & \ipa{ŋ} & ? &?  \\
\ipa{*l} & \ipa{l} & \ipa{l} & \ipa{l} & \ipa{l} / \ipa{r} \\
\ipa{*w} & \ipa{w} & \ipa{w} & \ipa{w} & \ipa{w} \\
\ipa{*j} & \ipa{j} & \ipa{j} & \ipa{j} & \ipa{j} \\
\ipa{*s} & \ipa{s} & \ipa{s} & \ipa{s} & \ipa{s}  \\
\bottomrule
\end{tabular}
\end{table}

In Wambule, Bantawa and Limbu, the semi-vowels \ipa{w} and \ipa{j} are not distinct from the zero initial before back rounded and front vowels (except \ipa{ɛ} in Limbu) respectively. In Bantawa there is possibly a rule *\ipa{wap} > *\ipa{op} in the verb \dhatu{op}{to scoop, to get water out} (Khaling \dhatu{wɛp}{scoop}). *\ipa{wa} does not change to \ipa{o} in any other context.

There are only two irregularities in the data at hand related to these initial consonants.

First, Wambule \dhatu{ŋwam}{smell} has \ipa{ŋ} where comparative evidence strongly suggest *\ipa{n} (Khaling |\ipa{nom}|, Bantawa  |\ipa{nam}| and outside of Kiranti Japhug \ipa{mnɤm} `smell (it)' and Tibetan \ipa{mnam} \textit{id.}). While it could be tempting to assume that the correspondence of Wambule \ipa{ŋ-} to \ipa{n-} in other languages is the reflex of the cluster *\ipa{mn-}, this hypothesis is better left aside until confirming data is found.\footnote{Note that \ipa{m-} is not a prefix in this verb in Japhug and Tibetan (\citealt{hill14derivational, jacques14snom}).}

Second, Khaling \dhatu{sik}{string beads} and Wambule \dhatu{sikt}{thread a needle} appear to correspond to Limbu \dhatu{tiːks}{to thread, to string (beads)}  and Bantawa \dhatu{tɨk}{to thread, to string, to make a garland}, but the correspondence of \ipa{s-} to \ipa{t-} is unique. This correspondence might reflect a special cluster, but no attempt will be made to provide a reconstruction for the onset of this root.

\section{Vowels and rimes} \label{sec:rhymes}

\subsection{Bantawa} \label{sec:bantawa}
 
 For Limbu and Khaling, \citet{michailovsky02dico}, \citet{jacques12khaling} and \citet{jacques16si} explicitly provide root forms. For Wambule, in the case of alternating verbs (\citealt[255-263]{opgenort04wambule}), the first stem is taken as the root form, as apart from a few irregular verbs, the second stem can always be predicted from the first one.  
 
 For Bantawa, some more discussion is needed on how the roots were extracted from Doornenbal's work. Based on the correspondences given in \citet[129; 132]{doornenbal09}, the following final consonants are reconstructed (alternating forms separated by slashes are in complementary distribution):

\begin{table}[h]
\caption{Bantawa root codas} \centering \label{tab:bantawa.root}
\begin{tabular}{cccc}
\toprule
Stem-C & Stem-V & Reconstructed Coda \\
\midrule
\ipa{-k} & \ipa{-ʔ-} & \ipa{-k} \\
\ipa{-t} & \ipa{-r-} & \ipa{-t} \\
\ipa{-p} & \ipa{-ʔ-} /  \ipa{-w-} & \ipa{-p} \\
$\varnothing$ & $\varnothing$ / \ipa{-w-} / \ipa{-j-} & $\varnothing$ \\
\ipa{-ŋ} & \ipa{-ŋ-} & \ipa{-ŋ} \\
\ipa{-n} & \ipa{-l-} & \ipa{-l} \\
\ipa{-n} & \ipa{-j-} / \ipa{-r-} / $\varnothing$ & \ipa{-r} \\
\ipa{-m} & \ipa{-m-} & \ipa{-m} \\
$\varnothing$ & \ipa{-s-} & \ipa{-s} \\
\midrule
\ipa{-k} & \ipa{-kt-} & \ipa{-kt} \\
\ipa{-t} & \ipa{-tt-} & \ipa{-tt} \\
\ipa{-p} & \ipa{-pt-} & \ipa{-pt} \\
\ipa{-ŋ} & \ipa{-ŋt-} & \ipa{-ŋt} \\
\ipa{-n} & \ipa{-nt-} & \ipa{-nt} \\
\ipa{-m} & \ipa{-mt-} & \ipa{-mt} \\
\midrule
\ipa{-ŋ} & \ipa{-ŋs-} & \ipa{-ŋs}  \\
\ipa{-n} & \ipa{-ns-} & \ipa{-ns} \\
\ipa{-m} & \ipa{-ms-} & \ipa{-ms}   \\
\bottomrule
\end{tabular}
\end{table}

The correspondences are relatively straightforward; simple stops undergo lenition when the verb root is followed by a vowel-initial suffix, and final dental sonorants merge to \ipa{-n} when followed by a consonant-initial suffix. 

Note roots of intransitive verbs with \ipa{n} / \ipa{j} alternation are reconstructed with final *\ipa{-r}, although this coda does not appear in the paradigm, as in \ipa{pon-ma, poj-a} `grow' (root |\ipa{por}|) or \ipa{pin-ma, pi-a} `fly' (root |\ipa{pir}|).

We notice important gaps in the distribution of complex codas: unlike in Khaling, there are no groups such as \ipa{-rt} and \ipa{-lt}. It will be shown in section \ref{sec:Ct} that these groups have merged with other complex codas in Bantawa.

The Cs-stems of Bantawa originate from the merger of transitive Cs stems (build in most cases by addition of the causative \ipa{-s} suffix, \citealt{michailovsky85dental}) and of reflexive C-si stems. In Cs-stems, only nasals are found as first element of the coda. 

We know from Limbu (\citealt[xiii]{michailovsky02dico}) that transitive Cs-stems, when the C is a stop consonant, show nasalization of the stop in the \textit{present} stem (which mainly appears before consonant-initial suffixes), as indicated in Table \ref{tab:Cs.limbu} in section \ref{sec:appl}. Hence, it is likely that Bantawa transitive Cs-stems underwent analogical levelling and that all \ipa{-ks} and \ipa{-ps} stems were converted to \ipa{-ŋs} and \ipa{-ms} stems.

As for intransitive Cs-stems, as shown in section \ref{sec:refl}, in the reflexive forms final stops are also nasalized in part of the paradigms. For instance, the verb \dhatu{ims}{sleep} in Bantawa is cognate to Khaling \dhatu{ʔip-si}{sleep}, the reflexive form of \dhatu{ʔipt}{put to sleep} (See section \ref{sec:semantic} concerning this common Kiranti innovation). In Khaling this verb has nasalized forms in the non-past (except dual) and infinitive (\ipa{ʔʌ̂msiŋʌ} `I sleep', \ipa{ʔʌ̂msi} he sleeps'), and a non-nasal stop in the dual and first plural (\ipa{ʔipsiji} `you and I/they two sleep'). The alternating stem with nasalized final is more widespread than the non-nasalized ones, and here again, there is little doubt that levelling took place in Bantawa.

\subsection{Vowels} \label{sec:vowels}

While some Kiranti languages, in particular Khaling, present very complex vowel systems and bewildering vowel alternations, internal reconstruction allows to reduce this number to 5 (as in Khaling).\footnote{While a contrast between \ipa{ɛ} and \ipa{a} is kept in the representation of Khaling verb roots, it is clear that these two were originally in complementary distribution, which was only broken when \dhatu{jal}{strike} was borrowed from Thulung (\citealt[1110]{jacques12khaling}).} The only exception is Limbu, which present a contrast between long and short vowels, as well as a contrast between mid-low and mid-high vowels (\ipa{e} and \ipa{o} vs \ipa{ɛ} and \ipa{ɔ}).  

Khaling and Wambule have in several verbs \ipa{o} and \ipa{wa} corresponding to \ipa{a} in Bantawa and Limbu instead of \ipa{o}.\footnote{Except before \ipa{-r} and perhaps other coronal finals where Wambule has \ipa{a} and Khaling \ipa{a}, as shown by Wambule \dhatu{kʰart}{parch}	 	
and \dhatu{ʣar}{drip, drizzle}	corresponding to Khaling 	\dhatu{kʰor}{parch, fry} and \dhatu{ʣʰor}{leak}. The same restriction is found in the noun `wound' Khaling \ipa{koɔ̄r}, Wambule \ipa{ɓari} < *\ipa{ʔwɑr} (with an additional suffix in Wambule, as in the noun `louse' discussed in section \ref{sec:why}.} This correspondence is also found in nouns,\footnote{For instance *\ipa{ʔnɑm} `sun'  Khaling \ipa{noɔ̄m}, Wambule \ipa{ɗwam}, as opposed to *\ipa{lam} `path, trail' , Khaling \ipa{lɛ̄m}, Wambule \ipa{lam}.}and is reconstructed as proto-Kiranti *\ipa{ɑ}.

The \ipa{u} of other languages can correspond to either Limbu \ipa{o} or \ipa{u}, with a comparable number of examples before the codas \ipa{-s} and \ipa{-k}, as illustrated by Table \ref{tab:U}. To account for this, I tentatively postulate a proto-phoneme *\ipa{ʊ}, which yields \ipa{o} in Limbu and \ipa{u} elsewhere.\footnote{A contrast between \ipa{i} and \ipa{u} on the one hand, and \ipa{ɪ} and \ipa{ʊ} on the other hand, is found in Hayu, see \citet{michailovsky88}; it remains to be seen whether the contrast reconstructed here matches that of Hayu. } No clear evidence exists for a comparable *\ipa{ɪ} among front vowels.

 \begin{table}[h]
 \caption{A fifth degree of height?} \centering \label{tab:U}
 \resizebox{\columnwidth}{!}{
 \begin{tabular}{lllll}
 \toprule
 Wambule & Khaling & Bantawa  & Limbu\\
\midrule
\dhatu{su}{be hot, spicy}	 &\dhatu{ʦu}{be spicy}	&	 		&	\dhatu{sos}{to be rich in taste}	\\	
&\dhatu{su}{itch}	&		\dhatu{sus}{itch}	&	\dhatu{sos}{itch}	\\	
\dhatu{buk}{get up}&\dhatu{pʰuk}{get up}&		\dhatu{puk}{stand up}	&	\dhatu{pok}{get up}	\\	
 &\dhatu{dʰuk}{bump into}	&		\dhatu{dʰuŋs}{bump}	&	\dhatu{tokt}{to bump againt}	\\	
&\dhatu{tʰuŋt}{gore, stab}	&	 &	\dhatu{tʰoks}{butt, gore}	\\	
 &\dhatu{tsukt}{point (with a finger)}	&		 	&	\dhatu{sokt}{aim, point}	\\	
 \midrule
   &\dhatu{ʦʰu}{be late}	&	 \dhatu{lukt}{finish}&	\dhatu{sus}{be late}	\\	
 & &		\dhatu{sukt}{lie in wait}	&	\dhatu{suk}{wait in ambush for}	\\	
 & 	\ipa{tsʰūŋkʌl} `cough (n)' &		\dhatu{cʰuŋs}{cough}	&	\dhatu{suks}{cough}	\\	
\dhatu{dukt}{shake, intoxicate}& &		\dhatu{tuk}{hurt, be ill} &	\dhatu{tuk}{hurt, be ill}	\\	
 &\dhatu{luk}{be destroyed}	&	 \dhatu{lukt}{finish}&	\dhatu{luk}{be completed}	\\	
\bottomrule
\end{tabular}}
\end{table}
  
Table \ref{tab:vowels} presents an account of the basic vowel correspondences between the four languages discussed in the present work, and Table \ref{tab:vowels.ex} provides examples of cognate sets for each vowels. All roots lacking a Limbu cognate are ambiguous between *\ipa{o} and *\ipa{ɔ} on the one hand, and *\ipa{e} and *\ipa{ɛ} on the other hand.



\begin{table}[H]
\caption{Proto-Kiranti vowels} \centering \label{tab:vowels}
\begin{tabular}{llllll}
\toprule
Proto-Kiranti & Wambule & Khaling & Bantawa & Limbu \\
\midrule
\ipa{*a} & \ipa{a} & \ipa{a}  / \ipa{ɛ} & \ipa{a} & \ipa{a}  \\
\ipa{*ɑ} & \ipa{wa} & \ipa{o} & \ipa{a} & \ipa{a}  \\
\ipa{*ɔ} & \ipa{wa} & \ipa{o} & \ipa{o} & \ipa{ɔ}  \\
\ipa{*o} & \ipa{wa} & \ipa{o} & \ipa{o} & \ipa{o}  \\
\ipa{*ɛ} & \ipa{ja} /  \ipa{e} & \ipa{e} & \ipa{e}  & \ipa{ɛ}  \\
\ipa{*e} & \ipa{ja} /  \ipa{e}  & \ipa{e} & \ipa{e}  / \ipa{ɨ} / \ipa{i} & \ipa{e}    \\
\ipa{*ʊ} & \ipa{u} & \ipa{u} & \ipa{u}  /  \ipa{ɨ}  & \ipa{o} \\
\ipa{*u} & \ipa{u} / \ipa{wa} & \ipa{u} & \ipa{u} /  \ipa{ɨ} \  & \ipa{u} \\
\ipa{*i} & \ipa{i} & \ipa{i} & \ipa{i}  / \ipa{ɨ} & \ipa{i}  \\
\bottomrule
\end{tabular}
\end{table}

In Wambule, the mid-vowels \ipa{*e}, \ipa{*ɛ}, \ipa{*o} and \ipa{*ɔ} of proto-Kiranti are diphthongized to \ipa{ja} and \ipa{wa}, except before \ipa{-j} (from *\ipa{-n} where \ipa{*e} and \ipa{*ɛ} remain \ipa{e}. 

The reflexes of Proto-Kiranti *\ipa{ɛ}, *\ipa{e}, *\ipa{i} and *\ipa{ʊ} in Bantawa are not elucidated. The vowels *\ipa{ɛ} and *\ipa{e} correspond either to \ipa{e}, \ipa{i} or \ipa{ɨ}, and *\ipa{i} and *\ipa{ʊ} correspond to Bantawa \ipa{ɨ} (rather than \ipa{i} and \ipa{u}) in some examples, without clear conditioning. The vowel *\ipa{ɛ} corresponds to Bantawa \ipa{e} in all examples except `fly' *\ipa{bɛːr}, Bantawa \ipa{pir}.

While it is likely that these splits are related to the codas, no exceptionless generalization can be proposed to account for the correspondences at this stage.
 
 \begin{table}[H]
 \caption{Proto-Kiranti vowels: representative examples} \centering \label{tab:vowels.ex}
 \begin{tabular}{llllllll}
 \toprule
Proto-Kiranti & Meaning &Wambule & Khaling & Bantawa & Limbu \\
 \midrule
\ipa{*dza} &	eat &	\dhat{dza} &	\dhat{dzA} &	\dhat{ca} &	\dhat{ca} &	\\
\ipa{*gam} &	fit together &	&	\dhat{gʰɛm} &	\dhat{kams} &	\dhat{kam} &	\\
\ipa{*tan > tan-t} &	drop &	&	\dhat{tɛnt} &	\dhat{dant} &	\dhat{(mut) thaːnt} &	\\
\ipa{*gaŋ > gaŋ-s} &	agree &	&	\dhat{gʰaŋ} &	\dhat{kaŋs} &	&	\\
 \midrule						
\ipa{*dzɑːk} &	swim &	\ipa{dzwak } (n) &	&	\dhat{cak} &	\dhat{caːk} &	\\
\ipa{*nɑm > nɑm-s} &	smell &	\dhat{ŋwam} &	\dhat{nom} &	\dhat{nam} &	\dhat{nams} &	\\
\ipa{*dɑ[m|p]} &	fill &	\dhat{dwapt} &	\dhat{dom} &	\dhat{tap} &	&	\\
\ipa{*hɑŋ} &	send &	&	\dhat{ɦoŋ} &	\dhat{hant} &	\dhat{haŋ} &	\\
\midrule
\ipa{*ʔɔːk > ʔɔːk-t} &	crow &	\dhat{wak} &	\dhat{ʔok} &	\dhat{ok} &	\dhat{ɔːkt} &	\\
\ipa{*kɔkt} &	chop &	\dhat{kwakt} &	\dhat{kokt} &	\dhat{kʰokt} &	\dhat{kʰɔkt} &	\\
\ipa{*hɔl > hɔl-t} &	open &	\dhat{hwal(s)} &	\dhat{ɦol} &	&	\dhat{hɔnt} &	\\
\ipa{*hɔm > hɔm-t} &	swell &	&	\dhat{ɦom} &	\dhat{hom} &	\dhat{hɔmt} &	\\
\midrule
\ipa{*dont} &	straighten &	&	\dhat{dʰont} &	\dhat{tont} &	\dhat{tont} &	\\
\ipa{*joŋ > joŋ-t} &	melt &	&	\dhat{joŋ} &	&	\dhat{jont} &	\\
\ipa{*tokt > tok-s} &	dig, peck &	&	&	\dhat{dʰokt} &	\dhat{toks} &	\\
\ipa{*hopt} &	eat (soup) &	&	\dhat{ɦopt} &	\dhat{hopt} &	\dhat{hopt} &	\\
\midrule
\ipa{*tɛkt} &	block &	&	\dhat{tekt} &	\dhat{dʰekt} &	\dhat{tʰɛkt} &	\\
\ipa{*hɛk > hɛk-t} &	cut (grass) &	&	\dhat{ɦek} &	\dhat{hekt} &	\dhat{hɛk} &	\\
\ipa{*sɛl > sɛl-t} &	clean, separate &	\dhat{sjal} &	\dhat{sel} &	&	\dhat{sɛnt} &	\\
\ipa{*lɛmt} &	flatter, sweeten &	\dhat{ljamt} &	\dhat{lemt} &	\dhat{lemt} &	\dhat{lɛm} &	\\
\midrule
\ipa{*tsek} &	pinch &	&	\dhat{ʦek} &	\dhat{cʰɨk} &	\dhat{sekt} &	\\
\ipa{*ʔes} &	defecate &	&	\dhat{ʔe} &	\dhat{es} &	\dhat{es} &	\\
\ipa{*re[t]} &	laugh &	\dhat{rja} &	\dhat{ret} &	\dhat{i} &	\dhat{et} &	\\
\ipa{*tsent > dzen > dzent} &	teach &	\dhat{cejt} &	\dhat{ʦent} &	\dhat{cint} &	  &	\\
\midrule
\ipa{*tuŋ} &	drink &	\dhat{tu:(s)} &	\dhat{tuŋ} &	\dhat{duŋ} &	\dhat{tʰuŋ} &	\\
\ipa{*dum > dumt / dum-si} &	ripen &	&	\dhat{dumt} &	\dhat{tum} &	\dhat{tums} &	\\
\ipa{*huk > hukt} &	bark &	\dhat{huk} &	\dhat{ɦuk} &	\dhat{hukt} &	 &	\\
\ipa{*duk > dukt} &	hurt &	\dhat{dukt} & &	\dhat{tuk} &	\dhat{tuk} &	\\
\midrule
\ipa{*ʔip-si} &	sleep &	&	\dhat{ʔip-si} &	\dhat{ims} &	\dhat{ips} &	\\
\ipa{*riŋ} &	praise &	&	\dhat{riŋ} &	\dhat{jɨŋ} &	\dhat{iŋ} &	\\
\ipa{*ʔmimt} &	remember &	\dhat{ɓimt} &	\dhat{mimt} &	&	&	\\
\ipa{*dim > dimt} &	press down &	\dhat{dimt} &	&	\dhat{tim} &	&	\\
\bottomrule
 \end{tabular}
 \end{table}


\subsubsection{Irregularities between high vowels}
The correspondences between \ipa{u} and \ipa{i} in closed syllable present a few irregularities, collected in table \ref{tab:high.vowels}.  

 \begin{table}[H]
 \caption{Irregular correspondences between high vowels in closed syllables} \centering \label{tab:high.vowels}
 \resizebox{\columnwidth}{!}{
 \begin{tabular}{lllll}
 \toprule
 Wambule & Khaling & Bantawa & Limbu \\
\midrule
\dhatu{lum}{boil}	&	\dhatu{lum}{parboil}	&	\dhatu{limt}{parboil}	&		\\
\dhatu{ŋir}{roar}	&	\dhatu{ŋur}{roar}	&		&		\\
\dhatu{tikt}{support}	&	\dhatu{tikt}{support}	&&	\dhatu{tʰukt}{to support}			\\
&	\dhatu{bʰlitt}{boil}	&		\dhatu{putt}{to boil over, steam}	&\dhatu{puːtt}{to boil over}	\\
\dhatu{blimt}{soak}	&	\dhatu{blum}{be submerged}	&		&		\\
\bottomrule
\end{tabular}}
\end{table}

 A phonological explanation here is unlikely. Rather, it should be noticed that in Khaling, in the intransitive and -Ct paradigms, high vowels are often neutralized to \ipa{ʌ} in stem alternations, and that the contexts where \ipa{i} and \ipa{u} can be distinguished are restricted to dual forms. In cases where it is Khaling that diverges from the other languages, it is likely that the other languages have a more conservative vocalism. As for the cases of `parboil' and `support' where Bantawa and Limbu are the divergent languages, since no productive vowel alternation involving high vowel is observable, no such explanation can be provided. 

\subsubsection{Irregularities between high and mid vowels}
In open syllables, the alternations between \ipa{u} and \ipa{o} on the one hand and \ipa{i} and \ipa{e/ɛ} on the other hand may be remnants of former apophony. Note in particular that the Limbu verb \dhatu{kʰe}{to quarrel} presents an irregular stem form \ipa{kʰe} instead of expected \ipa{kʰejɛ}. A possible explanation is that \ipa{kʰe} is the regular outcome of *\ipa{kʰi-ɛ}, and that this fused form was reanalyzed as the root form, removing by analogical expected alternations between \ipa{i} and \ipa{e}.

For the verb `be sweet', note that in Limbu we find a vowel alternation between \dhatu{limt}{be sweet}	and \dhatu{lɛm}{coax}, the latter having the expected vowel (see Khaling \dhatu{lemt}{coax}).

 \begin{table}[H]
 \caption{Irregular correspondences between high and mid vowels} \centering \label{tab:mid.vowels}
 \resizebox{\columnwidth}{!}{
 \begin{tabular}{lllll}
 \toprule
 Wambule & Khaling & Bantawa  & Limbu\\
\midrule
&\dhatu{kʰol}{transport}		&	\dhatu{kʰol}{transport}	&		\dhatu{kʰuˀr}{deliver to a place}\\	
&\dhatu{rup}{cut into pieces}	&		\dhatu{rop}{to break into pieces}	&		\\	
\midrule
 &\dhatu{bʰer}{fly}	&\dhatu{pir}{fly}	&			&	\dhatu{pɛːr}{fly}	\\	
 \dhatu{pʰit}{bring}	&\dhatu{pi}{come}	&			&	\dhatu{pʰɛn}{come}	\\	
&\dhatu{pʰi}{be spoiled (of rice)}	&			&	\dhatu{pʰɛn}{be spoiled}	\\	
&\dhatu{pi}{fart}	&		\dhatu{bʰes}{fart}	&	\dhatu{pʰes}{fart}	\\	
&\dhatu{ki}{fight, argue}	&			\dhatu{kʰi}{quarrel}	&	\dhatu{kʰe}{to quarrel}	\\	
&\dhatu{kik}{tie}	&	 	&	\dhatu{kʰeks}{attach to}	\\	
\dhatu{kjap}{stick}	&\dhatu{kept}{paste, stick}		&  &		\dhatu{kʰipt}{adhere, stick}\\	
\dhatu{lem}{be sweet}	& \dhatu{lem}{be sweet}		&  \dhatu{lem}{be sweet}	&		\dhatu{limt}{be sweet}		 \\	
\bottomrule
\end{tabular}}
\end{table} 

\subsubsection{Irregularities between front and back vowels}
There are in addition a few irregularities involving correspondences between low vowels and mid-vowels, or back vowel and front vowels, as shown by Table \ref{tab:front.vowels}.

 \begin{table}[H]
 \caption{Irregular correspondences between  front and back vowels} \centering \label{tab:front.vowels}
 \resizebox{\columnwidth}{!}{
 \begin{tabular}{llllll}
 \toprule
 Wambule & Khaling & Bantawa  & Limbu\\
\midrule
\dhatu{dwa}{dig}	&	\dhatu{dʰa}{dig}	&		&	\dhatu{tɔ}{dig}	&	\\
 	&	\dhatu{pa}{braid}	&		&	\dhatu{pʰɔ}{braid}	&	\\
\dhatu{jwa}{come down}	&	\dhatu{je}{come down}	&	\dhatu{ji / ju}{come down}	&	\dhatu{ju / jɛ}{come down}	&	\\
\midrule
 \dhatu{kʰlwamt}{put to sleep} & \dhatu{kʰlum}{bury} &  \dhatu{kʰumt}{bury} &    \\
 \midrule
	&	\dhatu{tʰɛm}{lose (one's way)}	(<*\ipa{tʰam})&	\dhatu{tʰem}{be lost}	&		&	\\
\dhatu{lwakt}{lick}	&	\dhatu{lak}{lick}	&	\dhatu{lek}{lick}	&	\dhatu{lak}{lick}	&	\\
\dhatu{dwapt}{taste}	&	\dhatu{dɛpt}{taste}	&	 	&	\ 	&	\\
\midrule
	&	\dhatu{ʔemt}{heat}	&	\dhatu{ams}{heat, bake}	&	\dhatu{am}{be warmed}	 \\
	&	\dhatu{ʦek}{be hard, be stingy}	&		&	\dhatu{sakt}{be hard}	&	\\
	&	\dhatu{gʰekt}{crack}	&		&	\dhatu{kak}{to crack}	&	\\
\dhatu{sart}{soak with urine}	&	\dhatu{ʦer}{urinate}	&	\dhatu{cʰens}{to urinate}	&	\dhatu{seˀr}{urinate on}	&	\\
\bottomrule
\end{tabular}}
\end{table}

In the case of `dig'' and `braid', where Khaling as \ipa{-a} corresponding to vowels originating from *\ipa{-ɔ} in other languages, Khaling is clearly innovative. The paradigms of transitive \ipa{-o} and \ipa{-a} stem verbs have identifical forms in the infinitive, dual and first and second person plural, as shown by Table (data from \citealt[1124]{jacques12khaling}).

\begin{table}[H]
\caption{Transitive \ipa{-o} and \ipa{-a} stem verbs in Khaling} \centering \label{tab:a.o}
\begin{tabular}{lllll}
\toprule
Form &\dhatu{ʣa}{eat} & \dhatu{tʰo}{see} \\
\midrule
\textsc{inf} & \ipa{ʣɵ-nɛ} \grise{}& \ipa{tʰɵ-nɛ} \grise{}\\
\textsc{1sg.npst} & \ipa{ʣʌ-ŋʌ} & \ipa{tʰɵ-ŋʌ} \\
\textsc{1di.npst} & \ipa{ʣɵ-ji} \grise{}& \ipa{tʰɵ-ji} \grise{}\\
\textsc{1pi.npst} & \ipa{ʣɵ-ki} \grise{}& \ipa{tʰɵ-ki} \grise{}\\
\textsc{2sg.npst} & \ipa{ʔi-ʣɛ} & \ipa{ʔi-tʰɵ} \\
\textsc{2pl.npst} & \ipa{ʔi-ʣɵ-ni} \grise{}& \ipa{ʔi-tʰɵ-ni} \grise{}\\
\textsc{3sg.npst} & \ipa{ʣɛ} & \ipa{tʰɵ} \\
\bottomrule
\end{tabular}
\end{table}

Since the transitive  \ipa{-a} stem class includes some of the most common verbs in the Khaling language, it is no surprise that some \ipa{-o} stem verbs, which share a considerable part of their paradigm with them, could change inflectional class and become  \ipa{-a} stem verbs.

For the verb `come down', stem alternation between a front-vowel and a back rounded vowel (still attested in Bantawa and Limbu) must be reconstructed back to proto-Kiranti, but the exact reconstruction will require an exhaustive evaluation of all Kiranti languages.

The correspondence between Wambule \dhatu{kʰlwamt}{put to sleep} and Khaling \dhatu{kʰlum}{bury}  is a different matter. There is clear evidence in Wambule of a morphological alternation between \ipa{u} and \ipa{wa}, reflected in doublets such as \dhatu{tupt}{beat, strike} and \dhatu{twapt}{play an instrument}, corresponding to Khaling  \dhatu{tupt}{beat metal, play an instrument}. In all such cases the \ipa{wa}/\ipa{u} alternating doublets correspond to verbs with \ipa{u} in other languages. Thus, whatever the explanation for this Wambule-internal phenomenon, we can surmise that in the case of this verb the \ipa{u} allomorph has disappeared, but a form *\ipa{kʰlum} can unambiguously be reconstructed here.



As for the puzzling correspondences between where Wambule, Bantawa or Khaling have \ipa{a} corresponding to \ipa{e} in other languages, no good explanation is available, and it cannot be excluded that some of these comparison have to be eventually discarded.

\subsubsection{A Kiranti innovation?}
 There is some evidence that proto-Kiranti *\ipa{ɛ} before *\ipa{-l} and \ipa{-t} comes in part from a low vowel at an earlier stage. Table \ref{tab:fronting} presents evidence for this change using Tibetan or Japhug comparanda.\footnote{For Tibetan verbs, the past form with prefix \ipa{b-} is provided, as the present form has vowel alternation patterns irrelevant to the present topic (see \citealt{jacques12internal}).} All Kiranti languages share these correspondences, and this is a candidate for a common Kiranti phonological innovation. The fact that this correspondence is only attested with \ipa{*ɛ}, never with \ipa{*e}, is a confirmation that the Limbu split between \ipa{*ɛ} and \ipa{*e} should be taken into account for proto-Kiranti reconstruction, and cannot be dismissed as a language-particular innovation.

 
\begin{table}[H]
\caption{Vowel fronting in Kiranti} \centering \label{tab:fronting}
 \resizebox{\columnwidth}{!}{
\begin{tabular}{llll}
\toprule
Khaling & Limbu & Tibetan & Japhug \\
\midrule
\dhatu{set}{kill} & \dhatu{sɛt}{kill} & \ipa{bsad} & \ipa{sat} \\
\dhatu{ʦʰelt}{be bright}& \dhatu{sɛˀr}{to turn white} & \ipa{gsal} `clear'' &  \\
\dhatu{sel}{cut out the bad parts}& \dhatu{sɛnt}{separate} & \ipa{bsal} `wipe out'' &  \\
\dhatu{let}{release}& \dhatu{lɛt}{leave, abandon} &  & Japhug \ipa{lɤt} < *\ipa{lɐt} `throw, leave''  \\
\bottomrule
\end{tabular}}
\end{table}
 
 This hypothesis raises the question of the origin of proto-Kiranti \ipa{*-al} and \ipa{*-at}, as these rhymes should have disappeared if a sound change \ipa{*a} $\rightarrow$ \ipa{*ɛ} / \_\{-l,-t\} had taken place. Only one example of \ipa{*-al} is found (Proto-Kiranti \ipa{*kal}, Khaling \dhatu{kɛl}{cluster together}, Limbu \dhatu{kʰaˀr}{to be matted (of hair)}). Examples of \ipa{*-at} or \ipa{*-aːt} are more numerous:
 
\begin{itemize}
\item \ipa{*blaːt}, Khaling \dhatu{blɛtt}{tell, explain}, Limbu \dhatu{paːt}{speak}.
\item \ipa{*waːt}, Bantawa \dhatu{wat}{to put on (as of ornaments), to wear jewelry}, Limbu \dhatu{waːt}{wear (an ornament)}.
\item \ipa{*k(r)at}, Wambule \dhatu{krat}{gnaw}, Khaling \dhatu{kɛt}{bite}.
\item \ipa{*mat}, Wambule \dhatu{mat}{forget}, Bantawa \dhatu{mat}{forget}.
\item \ipa{*la[n|t]}, Khaling \dhatu{lɛn}{come out}, Bantawa \dhatu{lat}{take out}.
\end{itemize}
 
Of the above examples, the only verb with clear external cognates is \ipa{*waːt} `wear ornament', which is related among others to Gyalrongic *\ipa{ŋgwa} `wear' (Japhug \ipa{ŋga} `wear', \ipa{tɯ-ŋga} `clothes'), so that the \ipa{-t} is probably suffixal here. 

\subsection{Simple codas} \label{sec:codas}
A system of nine simple codas, similar to that of Old Tibetan (\citealt{hill10synchronic}), is reconstructed for proto-Kiranti, as indicated in Table \ref{tab:codas}. There does not appear to be evidence for reconstructing glottal stop verb stems to proto-Kiranti, though these exist in Limbu.

\begin{table}[H]
\caption{Proto-Kiranti simple codas} \centering \label{tab:codas}
\begin{tabular}{llllll}
\toprule
Proto-Kiranti & Wambule & Khaling & Bantawa & Limbu \\
\midrule
\ipa{*-k} & \ipa{-k} / vowel length& \ipa{-k} & \ipa{-k} & \ipa{-k}  \\
\ipa{*-t} & \ipa{-j} & \ipa{-t} & \ipa{-t} & \ipa{-t}  \\
\ipa{*-p} & \ipa{-p} & \ipa{-p} & \ipa{-p} & \ipa{-p}  \\
\midrule
\ipa{*-ŋ} & vowel length & \ipa{-ŋ} & \ipa{-ŋ} & \ipa{-ŋ}  \\
\ipa{*-n} & \ipa{-j} /  \ipa{-t} & \ipa{-t} & \ipa{-t} & \ipa{-t}  \\
\ipa{*-m} & \ipa{-m} & \ipa{-m} & \ipa{-m} & \ipa{-m}  \\
\midrule
\ipa{*-r} &  \ipa{-r}  & \ipa{-r} & \ipa{-r} (tr) / \ipa{-j} (itr) & \ipa{-r}  \\
\ipa{*-l} & \ipa{-l}  & \ipa{-l} & \ipa{-l} & \ipa{-ˀr}  \\
\ipa{*-s} &$\varnothing$ & $\varnothing$ & \ipa{-s} & \ipa{-s}  \\
\bottomrule
\end{tabular}
\end{table}

One correspondence that deserves discussion is that between proto-Kiranti \ipa{-l} and Limbu \ipa{-ˀr}, as examples are very few, and only include the following:

\begin{itemize}
\item  Khaling \dhatu{kɛl}{cluster together, wear one's hair in a bun}, Limbu \dhatu{kʰaˀr}{to be matted (of hair)}  			 
\item  Khaling \dhatu{jil}{make soft by squeezing}, Limbu \dhatu{iˀr}{to rub, to scrub}  		 
\item   Khaling \dhatu{ʦʰelt}{be bright}, Limbu \dhatu{sɛˀr}{to turn white (of hair), to clear (of the milk of a cow that has just given birth)}  
\item Khaling \dhatu{kʰol}{transport}, Bantawa	  \dhatu{kʰol}{transport}	 and	 Limbu \dhatu{kʰuˀr}{deliver to a place}	  (the vowel correspondence is irregular).
\end{itemize}

Nevertheless, there is no counterexample where Khaling \ipa{-l} would correspond to a single final\footnote{As is shown in section \ref{sec:Ct}, Khaling \ipa{-l} also corresponds to \ipa{-nt} in Limbu, in verbs where a \ipa{-t} suffix has been added.} other than \ipa{-ˀr} in Limbu. This shows that although Limbu has neutralized the contrast between final *\ipa{-l} and  *\ipa{-n} in nouns, this contrast is faithfully preserved in the verbal system.


Two Limbu verbs, \dhatu{nuːr}{be good} and \dhatu{iːr}{turn, go around} have a long vowel and final \ipa{-r} where other languages have open syllables (cf Khaling \dhatu{nu}{be good} and \dhatu{ri}{become dizzy} respectively). Limbu is almost certainly innovative here. We find a similar correspondence in the unexplained alternation between $\varnothing$ and \ipa{-r} in the Limbu irregular verb \dhatu{pi}{give}, whose 3$\rightarrow$3 form \ipa{pur-u} (with \textsc{3sg.P} \ipa{-u} suffix) has vowel alternation \ipa{i/u} and an additional \ipa{-r} (without vowel lengthening, however). It is possible that the verb `give' preserves a $\varnothing$ / \ipa{-r} alternation that was generalized to the \ipa{-r} allomorph in `be good' and `turn'. The historical origin of this \ipa{r} element is unclear; whether it comes from an ancient suffix or from an epenthetic consonant used to break hiatus has to be left to further research.


The correspondences of Table \ref{tab:codas} are robust; most exceptions involve correspondences between simple codas and complex codas, and are treated in section \ref{sec:Ct}. 

In addition to a few cases of open syllables corresponding to syllables in \ipa{-n} or \ipa{-r} in Limbu (Table \ref{tab:mid.vowels} and above), exceptions to the correspondences between simple codas are the following:

\begin{itemize}
\item Khaling and Bantawa \dhatu{lim}{sprout}	 (same form) vs Limbu	\dhatu{liŋ}{sprout}.
\item Khaling \dhatu{lɛn}{come out} (applicative 	\dhatu{lɛnt}{take out}) vs Bantawa  \dhatu{lat}{take out}.
\item Wambule \dhatu{gu}{pick up from the ground} vs Bantawa \dhatu{kup}{pick up from the ground}
\end{itemize}

The additional coda \ipa{-p-} in Bantawa \dhat{kup} may be due to analogical levelling, as the \textsc{3sg}$\rightarrow$3 from \ipa{kuu} is potentially ambiguous with an open syllable stem \ipa{ku}, due to the lenition of final \ipa{-p} in these paradigms.

\subsection{Complex codas} \label{sec:Ct}
Kiranti languages are among the very few languages that allow stem-final complex clusters in the Sino-Tibetan family. The second element of these clusters is either \ipa{-t} or \ipa{-s}. They are nearly in all cases morphologically complex, the \ipa{-t} element being either the applicative \ipa{-t} (section \ref{sec:appl}), the denominal \ipa{-t} (\ref{sec:denom}) or the \ipa{-t} of deponent verbs (\ref{sec:deponent}), and the \ipa{-s} element either the causative (section \ref{sec:appl}) or a trace of the reflexive (\ref{sec:refl}).

Complex verbs shared across the family are not common, and mainly restricted to the final group \ipa{-kt} and \ipa{-pt}. Table \ref{tab:Ct.coda.ex} presents some representative examples. It is obvious that the \ipa{-t} is morphological in some of these verbs (in the case of \ipa{*ʔipt} `put to sleep'), but in other cases there is no direct evidence for a morphological function of the \ipa{-t}.

Since complex codas in Cs have disappeared altogether from Wambule and Khaling, it is difficult to reconstruct such clusters to proto-Kiranti. At best, one can provide pre-Bantawa or pre-Limbu reconstruction with Cs clusters, intermediate between proto-Kiranti and the attested forms. 

\begin{table}[H]
\caption{Examples of proto-Kiranti Ct codas} \centering \label{tab:Ct.coda.ex}
 \resizebox{\columnwidth}{!}{
\begin{tabular}{llllll}
\toprule
PK & Wambule & Khaling & Bantawa & Limbu \\
\midrule
\ipa{*s[e|ɛ]kt}	&	\dhatu{sjakt}{clean}&	\dhatu{sekt}{clean}	&		\dhatu{sekt}{clean thoroughly}	& 	\\
\ipa{*tɛkt}	&	 &	\dhatu{tekt}{cork, fill up}	&		\dhatu{dʰekt}{block, close}	&\dhatu{tʰɛkt}{insert in} 	\\
\ipa{*ʔipt}	&& 	\dhatu{ʔipt}{put to sleep}	&		&\dhatu{ipt}{put to sleep}		\\
\ipa{*hɛpt}	&&	\dhatu{ɦept}{hug}	&		\dhatu{ɦept}{hug}	&\dhatu{hɛpt}{embrace}	\\
\ipa{*tsent}	&\dhatu{cejt}{teach}&\dhatu{ʦent}{teach}	&		\dhatu{cint}{teach}	&		\\	
\ipa{*tumt}	&&	\dhatu{tumt}{track}	&		\dhatu{dʰumt}{trace}	& 	\\
\bottomrule
\end{tabular}}
\end{table}
 
I reconstruct final clusters only in examples such as  those of Table \ref{tab:Ct.coda.ex}, where all languages are in agreement. For many examples, however, we find a non-suffixed verb in one language corresponding to a verb in \ipa{-t} or \ipa{-s} in another language, with sometimes nasalization of the final obstruents (on which see sections \ref{sec:alternations} and {sec:refl} above). Table \ref{tab:Ct.coda.irr} presents examples of such correspondences. As shown by these data, we find sometimes a nasal codas corresponding to stop codas in complex clusters in Bantawa and Limbu. Since there is clear evidence that some verb undergo coda denasalization in applicative forms (see \ref{sec:appl}), a nasal coda is reconstructed for proto-Kiranti in these cases.

\begin{table}[H]
\caption{Ct codas and simple final codas} \centering \label{tab:Ct.coda.irr}
 \resizebox{\columnwidth}{!}{
\begin{tabular}{llllll}
\toprule
PK & Wambule & Khaling & Bantawa & Limbu \\
\midrule
\ipa{*dɑm}	&\dhatu{dwapt}{draw (water)}&	\dhatu{dom}{collect (water)}&	\dhatu{tapt}{fill (vi)}	&		 	& 	\\
\ipa{*taŋt}	& &	\dhatu{daŋt}{congeal}&	\dhatu{takt}{congeal}	&		 	& 	\\
\ipa{*glumt}	& &	\dhatu{glumt}{brood}&	\dhatu{kupt}{brood}	&		 \dhatu{kupt}{sleep with}	& 	\\
\midrule
\ipa{*kam}	& &	\dhatu{kɛm}{chew}&	 	&		 \dhatu{kʰamt}{chew}	& 	\\
\ipa{*ʔŋ[o|ɑ]l}	&\dhatu{ɓwalt}{mix} &	\dhatu{ŋol}{chew}&	 	&		 	& 	\\
\ipa{*kʰɑr}	&\dhatu{kʰart}{fry} &	\dhatu{kʰor}{fry}&	 	&		 	& 	\\
\bottomrule
\end{tabular}}
\end{table}

Table \ref{tab:Ct.coda} summarizes the main correspondences for complex codas with \ipa{-t} as second element. While Wambule and Khaling preserve such codas well, Limbu and Bantawa merge them with \ipa{-nt}.

\begin{table}[H]
\caption{The correspondences of Proto-Kiranti Ct codas} \centering \label{tab:Ct.coda}
\begin{tabular}{llllll}
\toprule
Proto-Kiranti & Wambule & Khaling & Bantawa & Limbu \\
\midrule
\ipa{*-kt} & \ipa{-kt} / \ipa{ŋt} & \ipa{-kt} / \ipa{ŋt} & \ipa{-kt} / \ipa{ŋt} & \ipa{-kt} / \ipa{ŋt}  \\
\ipa{*-tt} & \ipa{-tt} & \ipa{-tt}  & \ipa{-tt} & \ipa{-tt} \\
\ipa{*-pt} & \ipa{-pt} / \ipa{mt} & \ipa{-pt} / \ipa{mt}& \ipa{-pt}/ \ipa{mt} & \ipa{-pt} / \ipa{mt} \\
\midrule
\ipa{*-ŋt} & ?& \ipa{-ŋt} & \ipa{-ŋt} & \ipa{-ŋt}  / \ipa{-nt} \\
\ipa{*-nt} & \ipa{-jt}  & \ipa{-nt} & \ipa{-nt} & \ipa{-nt}  \\
\ipa{*-mt} & \ipa{-mt} & \ipa{-mt} & \ipa{-mt} & \ipa{-mt}  \\
\midrule
\ipa{*-rt} &  \ipa{-rt}  & \ipa{-rt} & \ipa{-tt}  & \ipa{-nt}  \\
\ipa{*-lt} & \ipa{-lt} & \ipa{-lt} & \ipa{-tt} & \ipa{-nt}  \\
\bottomrule
\end{tabular}
\end{table}

Evidence that some \ipa{-nt} stems in Limbu come from \ipa{*-rt} or \ipa{*-lt} is provided in Table \ref{tab:Ct.coda.irr}, where verbs in \ipa{-nt} correspond to verbs or noun in \ipa{-r} or \ipa{-l} in other languages. Since  Limbu  lacks final clusters such as $\dagger$\ipa{-rt} or $\dagger$\ipa{-lt}, the hypothesis that all three final groups merge as \ipa{-nt} explains both the gap in the system of these languages and the examples in Table \ref{tab:Ct.coda.rt}. 

In Bantawa it appears that at least *\ipa{-lt} merges with *\ipa{-tt} into \ipa{-tt}, as shown by \dhatu{bitt}{milk} (note the non-applicative form of the same root \dhatu{bil}{squeeze (for juice)}).\footnote{This verb cannot be denominal from the noun *\ipa{bit} `cow'  (Khaling \ipa{bʌ̂j}, Bantawa \ipa{pit}), otherwise $\dagger$\ipa{pitt} would have been expected. In Bantawa we also find a root  \dhatu{pil}{squeeze} and its applicative \dhatu{pitt}{squeeze together}, which are likely to be loanwords from a Khaling-like language.}

Additional evidence from the sound change \ipa{*-lt-} and  \ipa{*-rt-} $\rightarrow$ \ipa{-tt-} in Bantawa comes from synchronic alternations: the applicatives of \ipa{-r} and \ipa{-l} verbs is in \ipa{-tt-}, as for instance:

\begin{itemize}
\item \dhatu{kol}{walk, move about} $\rightarrow$ \dhatu{kott}{accompany, make walk}
\item \dhatu{cʰor}{pay} $\rightarrow$ \dhatu{cʰott}{pay for someone else}
\end{itemize}

\begin{table}[H]
\caption{Examples of proto-Kiranti  \ipa{*-rt} and\ipa{*-lt}  codas} \centering \label{tab:Ct.coda.rt}
 \resizebox{\columnwidth}{!}{
\begin{tabular}{llllll}
\toprule
PK & Wambule & Khaling & Bantawa & Limbu \\
\midrule
\ipa{*sɛl}	&\dhatu{sjal}{clean}&\dhatu{sel}{cut out the bad parts}&& \dhatu{sɛnt}{separate, butcher}  < *\ipa{ɛlt}  \\
\ipa{*hɔl}	&\dhatu{hwal}{open} &\dhatu{ɦol}{open}& & \dhatu{hɔnt}{open}   < *\ipa{ɔlt} \\
\ipa{*piːl}	&   &\dhatu{pil}{squeeze} &\dhatu{bitt}{milk}& \dhatu{pʰiːnt}{milk} < *\ipa{-iːlt}  \\
\midrule
%\ipa{*[ʣ|s]ur}	& \dhatu{ʣur}{be sour} &\dhatu{ʣʰur}{be sour} & \dhatu{sunt}{be sour} < *\ipa{urt}&\\
\ipa{*kɑr}	&\ipa{ɓari} `wound (n)'&	\ipa{koɔ̄r} `wound (n)'&	 	&		\dhatu{kʰa:nt}{wound}	< *\ipa{ɑrt}& 	\\
\ipa{*ʔɔr}	&   &\dhatu{ʔor}{break (cob of corn)} && \dhatu{ɔnt}{break off} < *\ipa{-ɔrt}&\\
\bottomrule
\end{tabular}}
\end{table} 

There are in addition two particular cases in Limbu. Some \ipa{-ŋt} final clusters have changed to \ipa{-nt}, as shown by the examples in  Table \ref{tab:Ct.coda.Nt} where a \ipa{-nt} in Limbu corresponds to \ipa{-ŋ} or \ipa{-ŋt} in other languages. In Khaling, \ipa{-ŋt} paradigms have forms that are undistinguishable from \ipa{-nt} paradigms,  leading to some uncertainty in class assignment for some roots. For instance, the 1sg  \ipa{tēndu}  and 3sg \ipa{tēndʉ}  forms of the root \dhatu{teŋt}{insert, squeeze into} could be analyzed as either coming from a root |\ipa{teŋt}| or from a root |\ipa{tent}|, and there is hesitation among some speakers on whether the dual inclusive form is \ipa{teŋi} (expected for a root |\ipa{teŋt}|) or \ipa{tēːʦi} (the outcome of a root |\ipa{tent}|).

 While no such ambiguity exists in the Limbu paradigm, since stem alternations in Limbu are much simpler than in Khaling, it is possible that \ipa{-ŋt} stem had a comparable alternation at an earlier stage, which was simplified by analogy but left traces in the form of \ipa{-ŋt} verbs shifting conjugation class assignment to \ipa{-nt} stem, like the ongoing change occurring in Khaling.

 A second minor sound change is the case of Limbu \ipa{-ˀr} corresponding to \ipa{-r} instead of \ipa{-l} in Wambule or Khaling in some intransitive verbs. I propose to reconstruct here a group *\ipa{-rs} in pre-Limbu, the *\ipa{-s} being an allomorph of the reflexive suffix (see section \ref{sec:refl}). This final group is not reconstructed to proto-Kiranti.


\begin{table}[H]
\caption{Examples of proto-Kiranti  \ipa{*-ŋt} and *\ipa{-rs} codas in Limbu} \centering \label{tab:Ct.coda.Nt}
 \resizebox{\columnwidth}{!}{
\begin{tabular}{llllll}
\toprule
PK & Wambule & Khaling & Bantawa & Limbu \\
\ipa{*joŋ}	&&\dhatu{joŋ}{melt}	&& \dhatu{jont}{melt}	\\
 \ipa{*ruŋt}	&&\dhatu{ruŋt}{tremble, shiver}	&  \dhatu{ruŋs}{shake}, & \dhatu{juːnt}{rub, shake (of an earthquake)}	\\ 
\midrule 
 \ipa{*ʔɔr}	&   &\dhatu{ʔor}{break (cob of corn)} &  & \dhatu{ɔˀr}{break (vi)} <*\ipa{-ɔrs}    \\
\ipa{*nur}	&\dhatu{nur }{be sprained} && & \dhatu{nuˀr}{be sprained}  *\ipa{-urs}  \\
\bottomrule
\end{tabular}}
\end{table} 
 
Finally, in Khaling the verb \dhatu{ʔutt}{spoil (of grains, seeds)} has \ipa{-tt} where Limbu has \dhatu{unt}{to spoil (of seed), to be moist}. This isolated correspondence cannot be explained.


 

 \section{Conclusion}
The model of reconstruction proposed in the present paper can be expanded in several ways. 

The first priority is to compile a dictionary of verb roots taking into account all available data on Kiranti languages (included derived nouns), on the model of the \textit{Lexikon der Indogermanischen Verben} (\citealt{liv}).

Second, this research should be combined with the reconstruction of the person indexation paradigms, a task made difficult by the extensive analogical levelling that took place in each language (\citealt{jacques16tonogenesis}). Given the conservative character of Kiranti languages, especially as far as suffixal morphology is concerned (\citealt{driem93agreement, delancey10agreement, jacques12agreement}), a reliable Proto-Kiranti reconstruction can  have a far-reaching impact in Sino-Tibetan comparative linguistics.

Third, research on the etymology of nouns in each individual language should be undertaken, and language-specific patterns of noun formation need to be elucidated.


Fourth, the patterns of analogical levelling attested in Kiranti verb roots may be relevant to research on the general principles of analogical levelling (\citealt{hill07ausgleich, hill14conditioned,jacques16ebde}).


\section{Appendix} \label{sec:appendix}
The attached .ods file provides 288 reconstructed proto-Kiranti verb roots. In reconstructions, [a|b|c] indicates that either \textit{a}, \textit{b} or \textit{c} are possible reconstructions. The chevron > indicates a derivation process, either anticausative (voicing of the initial), applicative (\ipa{-t} suffix), causative (\ipa{-s} suffix) or reflexive (\ipa{-si} suffix). The double chevron {>}{>} indicates an analogical change of inflectional class.

The list does not include verbs only attested in one of the languages that have cognates outside of Kiranti (and thus should be reconstructed to Proto-Kiranti too) or in Kiranti languages other than the four target languages. These verbs are given here:

\begin{itemize}
\item Wambule \dhatu{li}{be heavy}: Japhug \ipa{rʑi}, Tibetan \ipa{ldʑid.po} `heavy'
\item Khaling \dhatu{kakt}{hoe}: Japhug \ipa{qaʁ} $\leftarrow$ *\ipa{kaq} `hoe (n)'. The Khaling verb is an ancient denominal in \ipa{-t}.
\item Khaling \dhatu{noŋt}{accuse} : Tibetan \ipa{noŋs} `make a mistake'. The applicative \ipa{-t} here has a tropative value (consider X to be wrong, cf \citealt{jacques13tropative}).
\item Khaling \dhatu{mit}{die}: Chinese \zh{滅} *\ipa{met} `destroy'. Note that Khaling attests a semantic change comparable to that of the Indo-European root *\ipa{mer} (`disappear' in Anatolian, `die' in non-Anatolian languages, \citealt[439-440]{liv}).
\item Khaling \dhatu{pʰiŋ}{send}: Tibetan \ipa{spriŋ} `send'.
\item Khaling \dhatu{pʰut}{take off}: Japhug \ipa{pʰɯt} `take out, remove'.
\item Khaling \dhatu{tʰokt}{understand}: Tibetan \ipa{rtogs} `know, understand'.
\item Khaling \dhatu{rakt}{make someone gag, choke on}: Japhug \ipa{raʁ} `get stuck'.
\item Bantawa \dhatu{jas}{tickle}: Japhug \ipa{rɤʑa} < *\ipa{rɐja} `itch', Tibetan \ipa{gja} `itch', Chinese \zh{癢} *\ipa{jaŋʔ} `itch'.
\item Bantawa \dhatu{cʰus}{be fat}: Japhug \ipa{tsʰu}, Tibetan \ipa{tsʰo}.
\item Bantawa \dhatu{tʰup}{sew}: Japhug \ipa{tʂɯβ} < *\ipa{trup} `sew', Tibetan \ipa{ɴdrub} `sew'.
\item Bantawa \dhatu{dus}{collect, gather}: Tibetan \ipa{ɴtʰu, btus} `gather'.
\item Bantawa \dhatu{bʰuŋs}{to destroy, break down}: Tibetan \ipa{ɴpʰuŋ, pʰuŋ} `bring disaster to, bring the downfall of'.
\item Bantawa \dhatu{mitt}{to think, remember}: Yakkha \dhatu{mit}{remember}
\item Limbu \dhatu{munt}{move}: Japhug \ipa{mɯnmu} `move'.
\item Limbu \dhatu{suːt}{finish}, \dhatu{cuːt}{be finished}: Chinese \zh{卒} *\ipa{tsut} `finish'.
\item Limbu \dhatu{paŋs}{send}: Tibetan \ipa{ɴpʰen, ɴpʰaŋs} `throw, shoot', Chinese \zh{放} *\ipa{paŋ-s} `release; let go'.
\item Limbu \dhatu{jaːkt}{to stay (esp. overnight), to remain behind}: Japhug \ipa{rʑaʁ} `stay one night'.
\item Limbu \dhatu{thuŋs}{be thick (of cloth)}: Tibetan \ipa{mtʰug.po} `thick'
\item Limbu \dhatu{lɛkt}{change}: Chinese \zh{易} *\ipa{lek} `change'.
\item Limbu \dhatu{wa}{be, exist}: Chinese \zh{有} *\ipa{ɢwɨʔ} `exist'.
\end{itemize}
 
\bibliographystyle{unified}
\bibliography{bibliogj}
\end{document}

%renma (rentu) vt to tell off, to rebuke, reŋt keep away from oneself, actively dislike