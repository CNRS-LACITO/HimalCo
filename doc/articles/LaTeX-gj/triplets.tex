

\documentclass[oldfontcommands,oneside,a4paper,11pt]{article} 
\usepackage{fontspec}
\usepackage{natbib}
\usepackage{booktabs}
\usepackage{xltxtra} 
\usepackage{polyglossia} 
\usepackage[table]{xcolor}
\usepackage{gb4e} 
\usepackage{multicol}
\usepackage{graphicx}
\usepackage{float}
\usepackage{hyperref} 
\hypersetup{bookmarks=false,bookmarksnumbered,bookmarksopenlevel=5,bookmarksdepth=5,xetex,colorlinks=true,linkcolor=blue,citecolor=blue}
\usepackage[all]{hypcap}
\usepackage{memhfixc}
\usepackage{lscape}
\bibpunct[: ]{(}{)}{,}{a}{}{,}
 
%\setmainfont[Mapping=tex-text,Numbers=OldStyle,Ligatures=Common]{Charis SIL} 
\newfontfamily\phon[Mapping=tex-text,Ligatures=Common,Scale=MatchLowercase,FakeSlant=0.3]{Charis SIL} 
\newcommand{\ipa}[1]{{\phon \mbox{#1}}} %API tjs en italique
 \newcommand{\ipab}[1]{{\phon \mbox{#1}}} %API tjs en italique
\newcommand{\grise}[1]{\cellcolor{lightgray}\textbf{#1}}
\newfontfamily\cn[Mapping=tex-text,Ligatures=Common,Scale=MatchUppercase]{MingLiU}%pour le chinois
\newcommand{\zh}[1]{{\cn #1}}

 

 \begin{document} 

All known examples of anticausative alternations  in Japhug are presented in Table \ref{tab:anticausative}.
 
 
\begin{table}[h]
\caption{Examples of anticausative in Japhug}\label{tab:anticausative}
\resizebox{\columnwidth}{!}{
\begin{tabular}{lllllllll} \toprule
basic verb  & &derived  verb &\\
\midrule
\ipa{ftʂi}  &	to melt (vt)	&		\ipa{ndʐi}  &	to melt (vi)		\\
\ipa{kio}  &	to cause to drop	&		\ipa{ŋgio}  &	to slip		\\
\ipa{kra}  &		to cause to fall&		\ipa{ŋgra}  &	to fall		\\
\ipa{plɯt}  &	to destroy	&		\ipa{mblɯt}  &	to be destroyed		\\
\ipa{prɤt}  &	to cut	&		\ipa{mbrɤt}  &		to be cut	\\
\ipa{pɣaʁ}  &	to turn over (vt)	&		\ipa{mbɣaʁ}  &		to turn over (vi)	\\
\ipa{qɤt}  &	to separate	&		\ipa{ɴɢɤt}  &	to be separated		\\
\ipa{qʰrɯt}  &	to completely scratch	&		\ipa{ɴɢrɯt}  &	to be completely scratched		\\
\ipa{qrɯ}  &	to cut, to tear, to break	&		\ipa{ɴɢrɯ}  &	to break (vi), be torn		\\
\ipa{tɕɤβ}  &	to burn (vt)	&		\ipa{ndʑɤβ}  &	to be burned		\\
\ipa{tʰɯ}  &	to pitch (tent),  	&		\ipa{ndɯ}  &	to appear (rainbow), 	\\
 &	 to build (road, bridge)	&		   &	  to be built (road, bridge)		\\
\ipa{χtɤr}  &	 to spill	&		\ipa{ʁndɤr}  &		to be spilled	\\
\ipa{tʂaβ}  &	to cause to roll	&		\ipa{ndʐaβ}  &	to roll (vi)		\\
\ipa{qraʁ}  &	to tear	&		\ipa{ɴɢraʁ}  &		to be torn	\\
\ipa{qia}  &	to tear	&		\ipa{ɴɢia}  &		to get loose  	\\
\ipa{qlɯt}  &	to break	&		\ipa{ɴɢlɯt}  &		to be broken	\\
\ipa{sɤpʰɤr}  &	to shake off, to wipe off	&		\ipa{mbɤr}  &	wiped off	 	\\
 \ipa{pri}  &	 to tear	&		\ipa{mbri}  &	to be torn	 	\\
  \ipa{xtʰom}  &	 to put horizontally	&		\ipa{ndom}  &	 	to be horizontal 	\\
  \ipa{tɕɣaʁ}  &	 to squeeze out 	&		\ipa{ndʑɣaʁ}  &	 to be squeezed out	 	\\ 
   \ipa{kɤɣ}  &	 to bend 	&		\ipa{ŋgɤɣ}  &	 to be bent	 	\\ 
   \ipa{qrɤz}  &	 to shave 	&		\ipa{ɴɢrɤz}  &	 	to break (of hair, dry leaves etc) 	\\ 
   \ipa{cʰɤβ}  &	 to flatten, to crush 	&		\ipa{ɲɟɤβ}  &	to be crushed, flattened 	 	\\ 
   \ipa{cɯ}  &	 to open 	&		\ipa{ɲɟɯ}  &	 to be opened	 	\\ 
      \ipa{pʰaʁ}  &	 to split 	&		\ipa{mbaʁ}  &	 to split, break	 	\\
 \bottomrule
\end{tabular}}
\end{table}


However, as pointed out by \citealt{hill14voicing}, alternations related to transitivity are not limited to the voicing alternation. In Tibetan,  there are not verb pairs, but verb triplets relating to transitivity, which Hill classifies in three classes A, B, C as in Table \ref{tab:triplets} (there are at least seven such triplets). A and C are invariable intransitive verbs, while B  are transitive controllable verbs, for which we indicate the present and past forms.

 \begin{table}[H] \label{tab:triplets}  
 \caption{Verb triplets in Tibetan}
 \resizebox{\columnwidth}{!}{
\begin{tabular}{lllllllllll}
\toprule
A & Meaning & B & &Meaning & C & Meaning \\
\midrule
\ipa{gaŋ} &`fill (intr)' &\ipa{ɴ-geŋ-s} & \ipa{b-kaŋ}  & `fill' & \ipa{kʰeŋ-s}& `be full’ \\
\ipa{gab} &`hide (intr)' &\ipa{ɴ-geb-s} & \ipa{b-kab}  & `cover' & \ipa{kʰeb-s} &`be covered over’ \\
\bottomrule
\end{tabular}}
\end{table}

The present forms of B type verbs have voiced obstruents, but this voicing alternation is unrelated to that observed in A type verbs (\citealt{jacques12internal}), and will not be discussed here. Rather, the unvoiced past tense form of B verbs represents the root form (with the regular past tense \ipa{b--} prefix). Type A verbs correspond to the Japhug  anticausative verbs, and thus derived from type B verbs by voicing of the consonant root initial. Type C verbs, however, have no equivalent in Japhug, and their origin has never been accounted for. 

A tentative explanation to account for type C verbs\footnote{Any attempt at explaining Tibetan verbal morphology is doomed to be speculative, as languages closely related to Tibetan have lost most of the   cognate verb morphology, and morphologically more complex languages such as Rgyalrong and  Kiranti  are only remotely related to Tibetan, and share very few common morphological elements with Tibetan.} is that verbs presenting the B/C alternation were originally labile and could be conjugated following both the transitive and the intransitive conjugations. They had a root with unvoiced consonants (which were voiced in the transitive consjugation in the present and future forms, but remained unvoiced in the intransitive conjugation). Type B verbs were formed from the transitive conjugation and type C from the intransitive one.
 

\bibliographystyle{unified}
\bibliography{bibliogj}

 \end{document}
 