\documentclass[oldfontcommands,oneside,a4paper,11pt]{article} 
\usepackage{fontspec}
\usepackage{natbib}
\usepackage{booktabs}
\usepackage{xltxtra} 
\usepackage{longtable}
\usepackage{polyglossia} 
\usepackage[table]{xcolor}
\usepackage{gb4e} 
\usepackage{multicol}
\usepackage{graphicx}
\usepackage{float}
\usepackage{hyperref} 
\hypersetup{bookmarks=false,bookmarksnumbered,bookmarksopenlevel=5,bookmarksdepth=5,xetex,colorlinks=true,linkcolor=blue,citecolor=blue}
\usepackage[all]{hypcap}
\usepackage{memhfixc}
\usepackage{lscape}

\bibpunct[: ]{(}{)}{,}{a}{}{,}

%\setmainfont[Mapping=tex-text,Numbers=OldStyle,Ligatures=Common]{Charis SIL} 
\newfontfamily\phon[Mapping=tex-text,Ligatures=Common,Scale=MatchLowercase,FakeSlant=0.3]{Charis SIL} 
\newcommand{\ipa}[1]{{\phon \mbox{#1}}} %API tjs en italique
\newcommand{\ipab}[1]{{\scriptsize \phon#1}} 

\newcommand{\grise}[1]{\cellcolor{lightgray}\textbf{#1}}
\newfontfamily\cn[Mapping=tex-text,Ligatures=Common,Scale=MatchUppercase]{MingLiU}%pour le chinois
\newcommand{\zh}[1]{{\cn #1}}
\newcommand{\refb}[1]{(\ref{#1})}


\XeTeXlinebreaklocale 'zh' %使用中文换行
\XeTeXlinebreakskip = 0pt plus 1pt %
 %CIRCG
 


\begin{document} 
\title{Verbal valency and Japhug / Tibetan language contact\footnote{
} }
\author{Guillaume Jacques}
\maketitle
%\linenumbers

\section{Transitivity and argumental structure in Tibetan}
Borrowings are mainly from an older stage of Tibetan, no from modern dialects, thus comparison with Classical Tibetan


volitionality

\section{Morphological transitivity in Japhug}

Japhug verbs have two conjugations, transitive and intransitive. The intransitive conjugation indexes the person and number (singular, dual, plural) of the S, while the transitive conjugation indexes the person and number of both A and P. The indexation of arguments on transitive verb follows a quasi-canonical direct-inverse system (see \citealt{jacques10inverse}, \citealt{jacques14inverse}). The person marking prefixes and suffixes of the intransitive conjugation can be combined with either direct marking (via stem alternation), inverse marking (the \ipa{wɣ-} prefix) or portmanteau prefixes (the local scenario markers \ipa{kɯ--} $2\rightarrow1$ and \ipa{ta--} $1\rightarrow2$).

Transitive verbs can be unambiguously distinguished from intransitive ones by three morphological criteria. First, some transitive verbs show stem alternation in the non-past \textsc{1sg}$\rightarrow$3, \textsc{2sg}$\rightarrow$3 and \textsc{3sg}$\rightarrow$3'; forms (including factual, imperfective, sensory, egophoric, imperative and irrealis). This stem alternation is regular for all transitive verbs in open syllable with a non-front vowel, and follows the rules   in Table \refb{tab:stem3}.

 \begin{table} 
\caption{Stem 3 alternation in Japhug Rgyalrong} \label{tab:stem3} \centering
\begin{tabular}{llllll}
\toprule
Stem 1 & Stem 3 \\
\midrule
\ipa{--a} & \ipa{--e} \\
\ipa{--u} & \ipa{--e} \\
\ipa{--ɯ} & \ipa{--i} \\
\ipa{--o} & \ipa{--ɤm} \\
\bottomrule
\end{tabular}
\end{table}

 Second, open syllable transitive verb stems take the past transitive prefix \ipa{--t--} in the past (inferential and perfective) \textsc{1/2sg}$\rightarrow$3 forms. 
 
 
 Third, in perfective 3$\rightarrow$3 direct forms (without the inverse \ipa{wɣ--} prefix), the perfective prefixes have a distinct form.

There is in addition a small class of labile verbs, which can be conjugated either transitively or intransitively (\citealt{jacques12demotion}). All examples exhibit agent-preserving  lability.
  
  
semi-transitive \ipa{aʁe} `get to eat' 
  
\begin{exe}
\ex
\gll
\ipa{pɣɤtɕɯ}  	\ipa{mɤ-kɯ-nɯmtɕi}  	\ipa{qajɯ}  	\ipa{mɤ-aʁe}  \\
bird \textsc{neg-nmlz}:S/A-get.up.early worm \textsc{neg}-get.to.eat:\textsc{fact} \\
\glt Birds which do not get up early do not get to eat worms. (proverb) 
\end{exe}  
  
  \section{Loan verbs}
  
  excluding denominal verbs (derived within Japhug)
  
  
\citealt[133-140]{jacques04these})  
  
\begin{table}
\caption{Correspondences between argument structure type in Tibetan loanverbs in Japhug}
\resizebox{\columnwidth}{!}{
\begin{tabular}{llllllll}
\toprule
Tibetan & & & Japhug &&&\\
\midrule
adjective & \ipa{brtan.po} &stable& intr, stat & \ipa{frtɤn} &be reliable& \\
 \textsc{abs} &  \ipa{ɴkʰruŋ} & be born &intr, \textsc{abs.S-agr} & \ipa{mkhroŋ} & id. \\
 \textsc{abs+loc} &  \ipa{rgʲug} &run &intr, \textsc{abs.S-agr+abs} or \textsc{abs.agr+loc} & \ipa{rɟɯɣ} & id. \\
\textsc{dat+abs} & \ipa{dga} & like & semi-trans, \textsc{abs.S-agr+abs}& \ipa{rga} & id. \\
\midrule
 \textsc{abs+loc} &  \ipa{bʲol} & go around &trans, \textsc{erg.A-agr+abs.P-agr}& \ipa{pjɤl} & go around \\
vol, \textsc{erg+abs} & \ipa{rŋo} & fry & trans, \textsc{erg.A-agr+abs.P-agr} & \ipa{rŋu} & id. \\
vol, \textsc{erg+abs} & \ipa{bgos} & organize & labile: \textsc{abs.S} or  \textsc{erg.A-agr+abs.P-agr}& \ipa{βgoz} & id. \\
vol, \textsc{erg+dat} & \ipa{mpʰʲa} & reproach & trans, \textsc{erg.A-agr+abs.P-agr}& \ipa{mpɕa} & id. \\
vol, \textsc{erg+abs} & \ipa{btɕad} & cut & trans, \textsc{erg.A-agr+abs.P-agr}, \textsc{erg.A-agr+inf} & \ipa{ftɕɤt} & abandon (bad habit) \\
\bottomrule
vol, \textsc{erg+dat+abs} & \ipa{bɕad} & tell & trans, \textsc{erg.A-agr+abs.P-agr+dat}& \ipa{fɕɤt} & id. \\
\bottomrule
\end{tabular}}
\end{table}  
  
  
  \subsection{\textsc{abs}}
  
  \subsection{\textsc{abs+loc}}  
  
    \subsection{\textsc{dat+abs}}
\ipa{rga}  + complement clause, inf
  
  \ipa{rtoʁ} \textsc{dat+abs} ==> trans?

\begin{exe}
\ex
\gll  
	\ipa{paʁ}  	\ipa{rcanɯ,}  	\ipa{ɯ-tɯpɯ}  	\ipa{roŋri}  	\ipa{kɯ}  	\ipa{ʑo}  	\ipa{pjɯ-χsu-nɯ}  	\ipa{ra}  \\
	pig \textsc{top} \textsc{3sg.poss}-household each \textsc{erg} \textsc{emph} \textsc{ipfv}-raise-\textsc{pl} have.to:\textsc{fact} \\
	\glt The pigs, every household have to raise (some). (Pigs, 4)
\end{exe}
  
\ipa{  thamakha kɤ-sko na-ftɕɤt}
  
  \ipa{ma-tɤ-kɯ-mpɕa-a}
  
  
  no secundative ditransitive verb among loanwords

  rɟɤlpu ɯ-ɕki tɕe nɯra pɯ-kɯ-fse nɯra pjɤ-fɕɤt.
recipient can be encoded as genitive rather than dative:  

  tɕendɤre shanluzuode nɯnɯ kɯ mɤ-kɯ-mbrɤt ʑo rɟɤlpu ɣɯ ɯ-χpi ntsɯ pjɯ-fɕɤt pjɤ-ŋu tɕe,
  \section{Complex predicates}  
  \ipa{thaʁ pɯ-tɕhot-a}
  
  \ipa{ɯ-ŋgu mɯ́j-thon}
  \section{Derivations}
additional morphology on loanverbs? (applicative, causative, passive, anticausative etc)  
  
%stative
%
%rgɤz, rgas
%ʁdɯɣ, gdug
%frtɤn, brtan
%mphrɤt, ‘phrod 
%ʁzɤβ, gzab bzabs gzab bzobs (tr>stat) / adj gzob.po %慎重
%
%intransitive
%mkhroŋ , ‘khrung ‘khrungs
%rɟɯɣ , rgyug rgyugs
%rlaʁ , brlag brlags
%ndʐoʁ , ‘drog ‘drogs
%ntshoʁ , ‘tshog ‘tshogs
%sci, skye skyes
%ŋgrɯ, ‘grub grub
%naχsos, gso gsos
%
%semi-transitive
%
%rga, dga
%
%transitive
%
%χtɤt,
%βzu, bzo bzos
%zgroʁ,sgrog bsgrogs bsgrog sgrogs
%rtoʁ, rtog brtags brtag rtogs / rtogs
%zwɤr, sbor sbar sbar sbor
%sprɤt, sprod sprad sprad sprod
%χsu, gso gsos
%mpɕa, ‘phya ‘phyas ‘phya ‘phyas
% 
%
%
%prs
%rtsi, rtsi brtsis brtsi rtsis
%rŋu, rngo brngos brngo rngos
%ndɯn, ‘don bton gdon thon
%zgrɯl, sgril bsgril bsgril sgril
%tʂɯl, sgril bsgril bsgril sgril
%rtsɯɣ, rtsig brtsigs brtsig rtsigs
%zgrɯβ, sgrub bsgrubs bsgrub sgrubs
%
%
%pst:
%ftɕɤt, gcod bcad gcad chod
%βgoz, bgod bgos bgo bgos
%ftsɯɣ, ‘dzugs btsugs gzugs tshugs
%ftɯl, ‘dul btul gdul thul
%βzɯr, ‘dzur bzur gzur zur
%fkot, ‘god bkod dgod khod
%fkaβ, ‘gebs bkab dgab khob
% pʰɯl,‘bul phul dbul phul
% pɕɯs, ‘byid phyis dbyi phyis
% 
%βzgɤr,
%βzjoz, sbyong sbyangs sbyang
%sbyongs
%
%βrɟaŋ,
%
%pst+fut
%βzɟɯr, sgyur bsgyur bsgyur sgyur
%fɕɤt, ‘chad bshad bshad shod
%fɕaʁ, ‘chags bshags bshag bshogs
%fsroŋ, srung bsrungs bsrung srungs
%fskɤr, skor bskor bskor skor
%fstɯn,stun bstun bstun stun
%
%
%fut
%βzdɯ, sdud bsdus bsdu sdus
%rkɤz, rko brkos brko rkos
%χtɤr, gtor
%
%pjɤl <intr ‘byol byol
% 

  \section{Conclusion}
  
\bibliographystyle{unified}
\bibliography{bibliogj}
\end{document}