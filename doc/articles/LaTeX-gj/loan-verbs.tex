\documentclass[oldfontcommands,oneside,a4paper,11pt]{article} 
\usepackage[T1]{fontenc}
\usepackage{fontspec}
\usepackage{natbib}
\usepackage{booktabs}
\usepackage{xltxtra} 
\usepackage{polyglossia} 
\usepackage[table]{xcolor}
\usepackage{gb4e} 
\usepackage{multicol}
\usepackage{graphicx}
\usepackage{float}
\usepackage{hyperref} 
\hypersetup{bookmarks=false,bookmarksnumbered,bookmarksopenlevel=5,bookmarksdepth=5,xetex,colorlinks=true,linkcolor=blue,citecolor=blue}
\usepackage[all]{hypcap}
\usepackage{memhfixc}
\usepackage{lscape}
\usepackage{lineno}

%\bibpunct[: ]{(}{)}{,}{a}{}{,}

%\setmainfont[Mapping=tex-text,Numbers=OldStyle,Ligatures=Common]{Charis SIL} 
\newfontfamily\phon[Mapping=tex-text,Ligatures=Common,Scale=MatchLowercase,FakeSlant=0.3]{Charis SIL} 
\newcommand{\ipa}[1]{{\phon \mbox{#1}}} %API tjs en italique
\newcommand{\ipab}[1]{{\scriptsize \phon#1}} 

\newcommand{\grise}[1]{\cellcolor{lightgray}\textbf{#1}}
\newfontfamily\cn[Mapping=tex-text,Ligatures=Common,Scale=MatchUppercase]{MingLiU}%pour le chinois
\newcommand{\zh}[1]{{\cn #1}}
\newcommand{\refb}[1]{(\ref{#1})}

\newcommand{\ro}{$\Sigma$}
\newcommand{\siga}{$\Sigma_1$} 
\newcommand{\sigc}{$\Sigma_3$}   

\XeTeXlinebreaklocale 'zh' %使用中文换行
\XeTeXlinebreakskip = 0pt plus 1pt %
 %CIRCG
 \DeclareTextCommandDefault{\nobreakspace}{\leavevmode\nobreak\ } 
\begin{document} 
\title{Verbal valency and Japhug / Tibetan language contact\footnote{I wish to thank Giorgio Arcodia, Eitan Grossmann, Lameen Souag, Alena Wiczlak-Makarevich as well as three anonymous reviewers for useful comments on previous version of this work. Glosses follow the Leipzig glossing rules. Other abbreviations used here include: \textsc{auto} spontaneous-autobenefactive, \textsc{cisloc} cislocative,  \textsc{deexp} deexperiencer, \textsc{fact} factual/assumptive, \textsc{genr} generic, \textsc{ifr} inferential evidential,  \textsc{inv} inverse, \textsc{lnk} linker, \textsc{recip} reciprocal, \textsc{sens} sensory  evidential, \textsc{transloc} translocative, \textsc{trop} tropative.  Old and Classical Tibetan are translitterated using Jacques' (\citeyear{acques12transcription}) system.  Chinese borrowings in Japhug are indicated in pinyin between chevrons. The examples are taken from a corpus that is progressively being made available on the Pangloss archive (\citealt{michailovsky14pangloss}). This research was funded by the HimalCo project (ANR-12-CORP-0006) and is related to the research strand LR-4.11 ‘‘Automatic Paradigm Generation and Language Description’’ of the Labex EFL (funded by the ANR/CGI) }} 
\author{Guillaume Jacques}
\maketitle
%\linenumbers

\textbf{Abstract}: This paper presents the case of a language with rich indexation and limited case marking (Japhug) massively borrowing verbs from a language without indexation but with case marking (an unattested Tibetic language close to the ancestor of Amdo Tibetan). It provides a comprehensive survey of the argument structure and transitivity categories of Japhug verbs of Tibetic origin in comparison with those of the corresponding verbs in Amdo Tibetan, the attested Tibetic language closest to the donor of loanwords into Japhug. This survey shows that verbs of Tibetic origin are fully integrated morphosyntactically into Japhug, and that with a few exceptions, the argument structure of the original verb is predictable from that of the Japhug verb.

\textbf{Keywords}: Transitivity, Loanwords, Indexation, Japhug, Tibetan


\section{Introduction}
The present paper deals with verb borrowings in Japhug, one of the Gyalrong languages spoken in Western Sichuan.\footnote{The Gyalrong languages belong to the Gyalrongic subgroup of Sino-Tibetan (see \citealt{jackson00puxi}). The Core Gyalrong group comprises Japhug, Tshobdun, Zbu and Situ. In addition to Core Gyalrong, Gyalrongic includes the Stau-Khroskyabs branch (\citealt{lai14person}). Japhug is spoken by about 10,000 people in the north of 'Barkhams county, Rngaba prefecture, Sichuan, China. } 

Until the mid-20^{th} century, Amdo Tibetan used to be the main lingua franca in the Gyalrong area, leaving hundreds of loanwords in the vocabulary of Japhug. Borrowing took place continuously from the Tibetan Imperial period (eighth century) to the present. The bulk of the Tibetic loanwords are from a unattested variety that was close to the ancestor of modern Amdo dialect, but which was phonologically more conservative than any Tibetic language spoken nowadays in Eastern Tibetan areas.\footnote{In particular, it preserves complex clusters (such as \ipa{βzɟɯr} `change' from a form corresponding to Old Tibetan \ipa{bsgʲur(d)}) and final consonants, including \ipa{-s}, which is lost in all Tibetic languages spoken in China.} Morphological, phonological and semantic criteria for distinguishing between loanwords and cognates on the one hand, and for sorting out the layers of borrowings, have been presented in detail in previous publications (\citealt[83-200]{jacques04these}).

A considerable number of these loanwords are verbs, and the aim of the present paper is to map the correspondences between argument structure of Amdo Tibetan verbs and that of their borrowed equivalents in Japhug, a task that has not been undertaken before. Loanwords from Situ and Tshobdun are not discussed in the present paper, as there are too few of them and because they are more difficult to distinguish from native verbs. Borrowing from Chinese started only after the mid-twentieth century; loanwords from earlier periods of Chinese were either mediated through a Tibetic language, or limited to a few isolated nouns.

This paper comprises five sections. First, I present minimal background information on grammatical relations in Japhug and Amdo Tibetan. Second, I survey criteria according to which Japhug verbs related to Tibetan are included in this study (cognates between Japhug and Amdo Tibetan on the one hand, and denominal verbs derived from nouns of Tibetan origin are excluded). Third, I list all attested correspondences of argument structure between Amdo Tibetan and Japhug. Fourth, I discuss the issue of complex predicates and incorporation among the verbs of Tibetic origin. Finally, I show how native Japhug derivations apply to verbs of Tibetic origin.

\section{Background information}
This section presents a simplified account of flagging and indexation in Japhug, and contains all the data necessary to follow the comparison between Japhug and Amdo Tibetan verb argument structure.

\subsection{Flagging in Japhug}
Japhug has an ergative case marking system. As illustrated by examples \refb{ex:abs} and \refb{ex:erg}, the S and P have no overt case marking, while the A takes the Ergative marker \ipa{kɯ}.\footnote{I use the symbols S, P and A in this paper in the Comrian sense, see \citet{haspelmath11SAPTR}.} This marker is obligatory with noun phrases and third person pronouns,\footnote{Unlike Tibetic languages, where it is optional with some transitive verb classes (\citealt{tournadre91}).} but optional in the case of first and second person pronouns.  The Ergative \ipa{kɯ} is also used to mark instruments,\footnote{Although both are marked with the Ergative, the morphosyntactic status of instruments differ from that of A both with regards to indexation and relativization.} but this is not relevant to the present study.

 \begin{exe}
\ex \label{ex:abs}
\gll
\ipa{rɟɤlpu}  	\ipa{nɯ}  	\ipa{mɯ-pjɤ-rɤʑit}  \\
king \textsc{dem} \textsc{neg-evd.ipfv}-be.there \\
 \glt `The king was not there.' (Nyima wodzer2003.1, 18)
\end{exe}

 \begin{exe}
\ex \label{ex:erg}
\gll 
\ipa{rɟɤlpu}  	\ipa{nɯ}  	\ipa{kɯ}  	\ipa{li}  	\ipa{ci}  	\ipa{ɯ-rʑaβ}  	\ipa{kɯ-ɕɤɣ}  	\ipa{ci}  	\ipa{ɲɤ-ɕar.}  	 \\
king \textsc{dem} \textsc{erg} again \textsc{indef} \textsc{3sg.poss}-wife \textsc{nmlz}:S/A-be.new \textsc{indef}  \textsc{ifr}-look.for \\
\glt `The king married a new wife.' (Snow-white, 15)
\end{exe}

It is significant that the Ergative/Instrumental marker \ipa{kɯ} is almost certainly borrowed from the Amdo Tibetan Ergative/Instrumental/Genitive marker \ipa{kə} / \ipa{ɣə} (on which see, for instance, \citealt[62]{haller04themchen}).

In addition, the argument structure of Japhug can take oblique arguments headed by various postpositions, such as the Genitive \ipa{ɣɯ} (borrowed from the allomorph \ipa{ɣə} of the Amdo Tibetan genitive/instrumental/ergative marker), the Comitative \ipa{cʰo}, the Locative marker \ipa{zɯ} or by relator nouns such as the Dative  \ipa{ɯ-pʰe} or \ipa{ɯ-ɕki} (see \citealt[272-274]{jacques14linking} on the distinction between postpositions and relator nouns in Japhug).

Most other Rgyalrongic languages apart from Japhug, including Situ, Zbu, Tshobdun and Khroskyabs (\citealt{jackson98morphology}, \citealt{lai13affixale}), have borrowed either the \ipa{kə} or the \ipa{ɣə} allomorph of the Amdo  Ergative/Genitive marker. Only Stau has an Ergative marker \ipa{-w} unrelated to the ergative of Tibetic languages (\citealt{jacques14rtau})

\subsection{Transitivity and indexation in Japhug}
Rgyalrong languages have polypersonal person and number indexation on the verb, and present a near-canonical direct-inverse system, which has been described in previous research (\citealt{delancey81direction} on Situ,  \citealt{jackson02rentongdengdi} on Tshobdun,  \citealt{jacques10inverse} on Japhug and \citealt{gongxun14agreement} on Zbu). 

Japhug verbs can index at most two arguments. Table \ref{tab:japhug.tr} shows the intransitive and transitive paradigms for the factual non-past form. A set of person affixes is shared by transitive and intransitive verbs. In the case of transitive verbs, these affixes are combined with either Direct marking (via stem alternation), Inverse marking (the \ipa{wɣ-} prefix) or portmanteau prefixes (the local scenario markers \ipa{kɯ-} $2\rightarrow1$ and \ipa{ta-} $1\rightarrow2$) in such a way that the person of both arguments (and number of at least one argument) is unambigously marked.

Intransitive verbs can be further divided into two classes. The first class comprises stative verbs (including adjectives)\footnote{In Japhug, adjectives can be formally distinguished from other stative verbs in that they allow the tropative derivation (see \citealt{jacques13tropative}).} and auxiliary verbs that do not allow a human S. These verbs form their infinitive in \ipa{kɯ-}. The second class includes all other intransitive verbs, whose infinitive form is prefixed in \ipa{kɤ-} like transitive verbs.

\begin{landscape}
 
\begin{table}[H]
\caption{Japhug transitive and intransitive paradigms}\label{tab:japhug.tr}
\resizebox{\columnwidth}{!}{
\begin{tabular}{l|l|l|l|l|l|l|l|l|l|l|}
\textsc{} & 	\textsc{1sg} & 	  \textsc{1du} & 	\textsc{1pl} & 	\textsc{2sg} & 	\textsc{2du} & 	\textsc{2pl} & 	\textsc{3sg} & 	\textsc{3du} & 	\textsc{3pl} & 	\textsc{3'} \\ 	
\hline
\textsc{1sg} & \multicolumn{3}{c|}{\grise{}} &	\ipa{} & 	\ipa{} & 	\ipa{} & 	\ipa{\sigc{}-a}   & 	 \ipa{\sigc{}-a-ndʑi} & 	 \ipa{\sigc{}-a-nɯ} & 	\grise{} \\	
\cline{8-10}
\textsc{1du} & 	\multicolumn{3}{c|}{\grise{}} &	\ipa{ta-\siga{}} & 	\ipa{ta-\siga{}-ndʑi} & 	\ipa{ta-\siga{}-nɯ} & 	\multicolumn{3}{c|}{ \ipa{\siga{}-tɕi}}  & 	\grise{} \\	
\cline{8-10}
\textsc{1pl} & 	\multicolumn{3}{c|}{\grise{}} & 	  & 	&  & 	\multicolumn{3}{c|}{ \ipa{\siga{}-ji}}  & 	\grise{} \\	
\cline{1-10}
\textsc{2sg} & 	\ipa{kɯ-\siga{}-a} & 	\ipa{} & 	\ipa{} & 	\multicolumn{3}{c|}{\grise{}}&	\multicolumn{3}{c|}{\ipa{tɯ-\sigc{}}} & 	\grise{} \\	
\cline{2-2}
\cline{8-10}
\textsc{2du} & 	\ipa{kɯ-\siga{}-a-ndʑi} & 	\ipa{kɯ-\siga{}-tɕi} & 	\ipa{kɯ-\siga{}-ji} & 	\multicolumn{3}{c|}{\grise{}} &	\multicolumn{3}{c|}{\ipa{tɯ-\siga{}-ndʑi}} & 	\grise{} \\	
\cline{2-2}
\cline{8-10}
\textsc{2pl} & 	\ipa{kɯ-\siga{}-a-nɯ} & 	\ipa{} & 	\ipa{} & 	\multicolumn{3}{c|}{\grise{}}&	\multicolumn{3}{c|}{\ipa{tɯ-\siga{}-nɯ}} & 	\grise{} \\	
\hline
\textsc{3sg} &  	\ipa{wɣɯ́-\siga{}-a} & 	\ipa{} & 	\ipa{} & 	\ipa{} & 	\ipa{} & 	\ipa{} & \multicolumn{3}{c|}{\grise{}} &	\ipa{\sigc{}} \\ 	
\cline{2-2}
\cline{11-11}
\textsc{3du} &  	\ipa{wɣɯ́-\siga{}-a-ndʑi} & 	 \ipa{wɣɯ́-\siga{}-tɕi} & 		\ipa{wɣɯ́-\siga{}-ji} & 	\ipa{tɯ́-wɣ-\siga{}} & 	\ipa{tɯ́-wɣ-\siga{}-ndʑi} & 	\ipa{tɯ́-wɣ-\siga{}-nɯ} & 	\multicolumn{3}{c|}{\grise{}} &	\ipa{\siga{}-ndʑi} \\ 
\cline{2-2}	
\cline{11-11}
\textsc{3pl} &  	\ipa{wɣɯ́-\siga{}-a-nɯ} & 	\ipa{} & 	\ipa{} & 	\ipa{} & 	\ipa{} & 	\ipa{} & \multicolumn{3}{c|}{\grise{}} &	\ipa{\siga{}-nɯ} \\ 	
\hline
\textsc{3'} & 	\multicolumn{6}{c|}{\grise{}} &	\ipa{wɣɯ́-\siga{}} & 	\ipa{wɣɯ́-\siga{}-ndʑi} & 	\ipa{wɣɯ́-\siga{}-nɯ} & 	\grise{} \\	
	\hline	\hline
\textsc{intr}&\ipa{\siga{}-a}&\ipa{\siga{}-tɕi}&\ipa{\siga{}-ji}&\ipa{tɯ-\siga{}}&\ipa{tɯ-\siga{}-ndʑi}&\ipa{tɯ-\siga{}-nɯ}&\ipa{\siga{}}&\ipa{\siga{}-ndʑi} &\ipa{\siga{}-nɯ}& 	\grise{} \\	
	\hline
\end{tabular}}
\end{table}

\end{landscape}
Transitive verbs can always be distinguished from intransitive ones by three morphological criteria. First, some transitive verbs show stem alternation in the non-past \textsc{1sg}$\rightarrow$3, \textsc{2sg}$\rightarrow$3 and \textsc{3sg}$\rightarrow$3'; forms (including factual, imperfective, sensory, egophoric, imperative and irrealis; see the forms with \sigc{} in Table \ref{tab:japhug.tr}). This stem alternation is regular for all transitive verbs in open syllables with a non-front vowel, and follows the rules indicated  in Table \refb{tab:stem3}. Following \citet{jackson00puxi}'s terminology, the base stem is referred to as \textit{Stem 1}, and the alternating stem as \textit{stem 3}.\footnote{Stem 2 is only found in the perfective form, and is irrelevant to the issues discussed in this paper.}  Second, open syllable transitive verb stems take the past transitive prefix \ipa{-t-} in the past (inferential and perfective) \textsc{1/2sg}$\rightarrow$3 forms.  Third, in perfective 3$\rightarrow$3 direct forms (without the inverse \ipa{wɣ-} prefix), the perfective prefixes have a distinct form.

 \begin{table}[H]
\caption{Stem 3 alternation in Japhug Rgyalrong} \label{tab:stem3} \centering
\begin{tabular}{llllll}
\toprule
Stem 1 & Stem 3 \\
\midrule
\ipa{-a} & \ipa{-e} \\
\ipa{-u} & \ipa{-e} \\
\ipa{-ɯ} & \ipa{-i} \\
\ipa{-o} & \ipa{-ɤm} \\
\bottomrule
\end{tabular}
\end{table}

There is in addition a small class of labile verbs, which can be conjugated either transitively or intransitively (\citealt{jacques12demotion}). All examples exhibit agent-preserving  lability (the S of the intransitive construction corresponds to the A of the transitive one).
  
As mentioned above, in addition to the core arguments S, A, and P, the argument structure of Japhug verb may contain obliques, that are never indexed on the verb.
  
A handful of intransitive verbs have an unmarked second argument, which is not indexed on the verb but is relativized in the same way as P arguments (\citealt{jacques14relatives}). I refer to such verbs as `semi-transitives' and their second argument as `semi-object'.  Example \refb{ex:maRe} illustrates the semi-transitive construction with the verb \ipa{aʁe} `get to eat'. Its semi-object \ipa{qajɯ} `worm' cannot be indexed on the verb, while its S \ipa{pɣɤtɕɯ}  	\ipa{mɤ-kɯ-nɯmtɕi}  `bird not getting up early', if in the plural or dual, will trigger agreement with the verb.
  
\begin{exe}
\ex \label{ex:maRe}
\gll
\ipa{pɣɤtɕɯ}  	\ipa{mɤ-kɯ-nɯmtɕi}  	\ipa{qajɯ}  	\ipa{mɤ-aʁe}  \\
bird \textsc{neg-nmlz}:S/A-get.up.early worm \textsc{neg}-get.to.eat:\textsc{fact} \\
\glt `Birds which do not get up early do not get to eat worms.' (proverb) 
\end{exe}  

Ditransitive verbs in Japhug can be either secundative or indirective (see \citealt{jacques14antipassive}), but only the second type is attested in loanwords from a Tibetic language. Indirective verbs treat the T morphosyntactically as the P of a monotransitive construction (in the absolutive form, indexed on the verb, and the R as an oblique marked with the Dative (not indexed on the verb).

In Japhug, non-overt arguments are interpreted as definite; excluding a handful of labile verbs, indefiniteness can only be expressed by means of voice derivations such as antipassive or passive (\citealt{jacques12demotion}), or by indefinite pronouns such as \ipa{tʰɯci} `something'.

\subsection{Argument structure in Amdo Tibetan and other Tibetic languages}
Tibetic languages differ from Japhug in that they have no person and number indexation on the verb; grammatical relations are marked exclusively by flagging.

The flagging system of Old Tibetan and modern languages is richer than that of Japhug,  with no less than ten distinct cases (\citealt{tournadre10cases, hill12bas}). As in Japhug, Tibetic languages have ergative alignment on noun phrases (\citealt{tournadre96erg}), with the A marked by the Ergative/Instrumental marker \ipa{-kʲis, -gʲis, -gis} (which becomes \ipa{-kə/-ɣə} in Amdo Tibetan and was borrowed as \ipa{kɯ} in Japhug) and the S/P is in the absolutive form. The other case relevant to the present paper is the Dative (\ipa{-la} in Old Tibetan, \ipa{-a} or vowel alternation in Amdo Tibetan).


In Old Tibetan, verbs had at most four distinct stems marking TAM distinctions, which are traditionally referred to as Present, Past, Future and Imperative stems.\footnote{The actual use of these stems is not discussed in this paper; the most detailed study on this topic is \citet{zeisler04}.} Amdo Tibetan preserves Present, Past and Imperative stems with some degree analogical levelling and regular phonetic changes.

Controllability, rather than transitivity, is the most important morphosyntactic feature of Amdo Tibetan verbs (\citealt[74-75, 136]{haller04themchen}). The absence of morphological transitivity, the isomorphy between ergative and instrumental, and pervasive P-preserving lability make it difficult to define transitivity in an unambiguous way in Tibetic languages (\citealt{tournadre96erg}).

In the following discussion, I use the classification of argument structure of Amdo Tibetan verbs as described in \citet{haller04themchen}; I use the term `transitive' to refer to verbs with two arguments, one of which is marked with the ergative (excluding instruments). 


\section{Loan verbs}
This section, building on the background information presented above, provides an explicit discussion of which Japhug verbs are included in this study. It comprises two subsections. First, I provide an overview of the method used to tease apart loanwords from cognates among related words between Japhug and Tibetic languages describe verb stem morphology in Tibetic loanwords into Japhug. Second, I discuss the issue of denominal verbs in Japhug, and why they are not taken into account in this study. 

\subsection{Loanwords vs cognates} \label{sec:layers}
Since Japhug and Tibetan are related languages, both belonging to the Sino-Tibetan family, it is not always trivial to distinguish between cognates and loanwords. 

\citet[83-200]{jacques04these} proposes a series of phonological, morphological and semantic criteria for distinguishing between cognates and  loanwords between Japhug and Tibetaic languages, and shows that at least five distinct layers of borrowings have to be posited. These criteria cannot be discussed in the present article due to considerations of space, but the basic principles are based on known innovations in both Japhug and Tibetic languages. 

Any Japhug word with a Tibetic comparandum presenting a Tibetic-proper phonological, morphological or semantic innovation (for instance, the palatalization of *l to \ipa{ʑ} in certain contexts, see \citealt{hill13laterals}) must be borrowed from a Tibetic language. Along the same line, any Japhug word with a Tibetic comparandum not presenting attested Japhug sound changes must be a borrowing (for instance, words where Japhug \ipa{-oz} corresponds to Old Tibetan \ipa{-os}, as proto-Japhug *\ipa{-os} becomes \ipa{-ɯz}, not \ipa{-oz}). A systematic application of rules of this type allows to determine with near-certainty which verbs in Japhug are borrowed from a Tibetic language and which are cognates inherited from their common ancestor. Only a handful of verbs can be analyzed both ways (\citealt[164]{jacques04these}), and these will not be included in the present study.

Cognates between the two languages are very few (less than 150), but on the other hand 850 borrowed words have been found in a dictionary comprising 6350 entries.


Of the four stem of Old Tibetan transitive controllable verbs, only the present and the past stems are widely attested in loanwords into Japhug (\citealt[138]{jacques04these}). Table  \ref{tab:four.stems} present examples of Japhug verbs based on the present or past stems of the original Old Tibetan verbs (in addition to one verb derived from a future stem). TAM affixes, such as the past and future \ipa{b-} prefix, the past and imperative \ipa{-s} suffix, and the present nasal prefix, are borrowed with the verb stem, and are not analyzable in Japhug.

\begin{table}[H]
\caption{Old Tibetan verb stems as attested in loanwords into Japhug} \label{tab:four.stems} \centering
\begin{tabular}{llllll}
\toprule
Stem&Tibetan && Japhug &\\
\midrule
Present & \ipa{ɴdon} & read (aloud) & \ipa{ndɯn} & read (aloud)&\\
Past  & \ipa{btsugs} & establish & \ipa{ftsɯɣ} &establish&\\
Future & \ipa{bsdu} & collect & \ipa{βzdɯ} & collect\\
\bottomrule
\end{tabular}
\end{table}

Since argument structure in Tibetic languages is independent of verb stem morphology, stem alternation is not taken into account in this study. In the case of transitive verbs, stems of verbs of Tibetic origin undergo the same regular alternations  as native verbs (Table \ref{tab:stem3}): verbs ending in \ipa{-a}, \ipa{-u}, \ipa{-ɯ}, \ipa{-o} change to \ipa{-e}, \ipa{-e}, \ipa{-i} and \ipa{-ɤm} respectively. For instance, the alternating stem form of \ipa{βzdɯ} is \ipa{βzdi}.

There is in addition one case of back-formation: the form \ipa{fkrɤm} `put in order' borrowed from the Tibetan past stem \ipa{bkram} `arrange, set' is reanalyzed by some speakers as an alternating Stem 3, and these speakers create an innovative Stem 1 form \ipa{fkro}.

 
\subsection{Denominal verbs}\label{sec:denominal}
Japhug has a very rich system of denominal prefixes (\citealt{jacques12incorp, jacques14antipassive}). Nouns of Tibetic or Chinese origin borrowed into Japhug can undergo these derivations and be turned into verbs, as shown by the examples in Table \ref{tab:denominal}.

\begin{table}[H]
\caption{Denominal verbs based on nouns borrowed from a Tibetic language} \label{tab:denominal}
\resizebox{\columnwidth}{!}{
\begin{tabular}{llllllll}
\toprule
\multicolumn{2}{c}{Tibetan}  &  \multicolumn{2}{c}{Japhug noun}&& verb&\\
\midrule
\ipa{ɕiŋ.tɕʰa} & timber & \ipa{ɕoŋtɕa}&timber & \ipa{ɣɯ-ɕoŋtɕa} & cut timber (vi) \\
\ipa{dgra} &enemy& \ipa{ʁgra}&enemy & \ipa{nɯ-ʁgra} & treat as an enemy (tr) \\
\ipa{tsʰos} &colour, paint, dye& \ipa{tsʰwi}&paint, dye & \ipa{sɯɣ-tsʰwi} & dye (tr) \\
\bottomrule
\end{tabular}}
\end{table}  

Denominal derivation is also used for what \citet{wohlgemuth09verbal} calls `indirect insertion': the borrowing of verbs with addition of denominal morphology. Tibetan and Chinese borrowed verbs differ in this regard: denominal prefixes are never added to Japhug verbs borrowed from verbs of Tibetic origin, while it is required for all verbs of Chinese origin, without a single exception.  

Even Chinese adjectives, which in that language constitute a distinct word class (\citealt{paul10adj}) have to be denominalized to be used predicatively or attributively. For instance, \zh{复杂} \ipa{fuza} `complex' and \zh{严} \ipa{yan} `strict' are borrowed as \ipa{rɯ-futsa} `be complex' or \ipa{nɯ-jen} `be strict' into Japhug, and have to be relativized with the participle prefix \ipa{kɯ-} to be used as attributes.

The differential treatment of verbs borrowed from Tibetic and Chinese is probably due to the time frame of their adoption into Japhug: Tibetic verbs were borrowed before the establishment of the PRC in 1949, and Chinese loanwords started after it. It may imply that borrowing by \textit{direct insertion} (\citealt[88]{wohlgemuth09verbal}) may not be productive anymore in Japhug.

Since denominal verbs do not always have a corresponding verb in any variety of Tibetan or Chinese, and since their argument structure is determined by the denominal prefix they receive, it is futile to attempt at establishing correspondences between the argument structure of denominal verbs of their closest equivalents (if any) in the donor languages. For this reason, only verbs directly borrowed, without denominalization, are considered in the following discussion, thus excluding all Chinese loanwords.
 
  \section{Attested correspondences} 
 In this section, I present examples of Japhug verbs borrowed from Tibetic languages, and compare their argument structure with that found in Amdo Tibetan (using Haller's \citeyear{haller04themchen} monograph as a source) or in Classical Tibetan if the information is not available for Amdo. Verbs of Tibetic origin in Japhug come from several distinct varieties, probably not all of which were ancestral to Amdo Tibetan. However, given the fact that Amdo Tibetan is the only Tibetic languages in Eastern Tibet with reliable and copious data on argument structure, it was hardly feasible to take other languages, even Old Tibetan, into account for this study.
 
 The constructions are presented from the least transitive ones (adjectives) to the most transitive ones (ditransitive verbs). 
   
  \subsection{Adjective : Stative intransitive}  
In Tibetic languages, most adjectives are originally built from verb stems by addition of the adjectivizing suffix \ipa{-po}. It is significant that not a single adjective has been borrowed into Japhug with the suffix \ipa{-po}. The only examples of words with the suffix \ipa{-po} in Tibetan loanwords are nouns of adjectival origin like \ipa{βdaχpu} `master, host, owner' from a form corresponding to Classical Tibetan \ipa{bdag.po} (same meaning).

All cases of adjectives suffixed in \ipa{-po} in Tibetic languages correspond to adjectival stative verbs in Japhug, without suffix, for instance \ipa{frtɤn} `reliable' borrowed from a form corresponding to Classical Tibetan \ipa{brtan.po} `stable' (Amdo \ipa{ɸtanpu} `verlässlich'). Although the non-adjectivized form is attested in Classical Tibetan for nearly all adjectives, it is not the case in the modern Tibetic languages. This implies that these adjective were borrowed from an extinct Tibetic variety in which all words corresponding to modern \ipa{-po} suffixed adjectives were still stative verbs and could be used predicatively without a copula.

  
\subsection{Non-controllable intransitive (\textsc{abs}) $\rightarrow$ Stative intransitive (\textsc{abs})}
The verbal roots from which adjectives in \ipa{-po} are derived are in certain cases still used in modern Tibetic languages in their dynamic form, and are non-controllable intransitive verbs with a unique argument in the absolutive.

For instance, Amdo Tibetan \ipa{rgi} `grow old' comes from the Old Tibetan past stem \ipa{rgas} of the verb \ipa{rga} `be old, become old' (following regular sound changes). It is not used however with a stative meaning `be old'.

\begin{exe}
\ex \label{ex:rgi}
\gll
\ipa{ʂtamɖʐən} 	\ipa{rgi-ɣokə}  	 \\
PN grow.old-\textsc{ipfv.nvol.evid} \\
\glt `Rtamgrin is growing old.' (\citealt[123, ex. 548]{haller04themchen})
\end{exe}

In Japhug, this verb was borrowed from the past stem \ipa{rgas} (`be.old:\textsc{pst}') as \ipa{rgɤz} `be old', but unlike Amdo Tibetan it can also be used as an adjectival stative verb with an infinitive in \ipa{kɯ-}. 

However, it is not necessary to assume that the stative meaning of \ipa{rgɤz} is an archaism, preserving the meaning of the verbal root before it become restricted to the dynamic use as in modern Amdo. In Japhug, all stative verbs acquire a dynamic interpretation when used in some TAM forms such as the imperfective, the perfective or the inferential as in example \refb{ex:thWrgAz}. Therefore, a dynamic verb of Tibetic origin expressing the acquisition of a quality (`become X') can be converted to an adjectival stative verb.

\begin{exe}
\ex \label{ex:thWrgAz}
\gll
\ipa{nɯnɯ}  	\ipa{ɯ-ʁrɯ}  	\ipa{nɯnɯ,}  	\ipa{tʰɯ-rgɤz}  	\ipa{tɕe}  	\ipa{tɕe}  	\ipa{nɯ}  	\ipa{ɯʑo}  	\ipa{pjɯ-ɤ<nɯ>tɤr}  	\ipa{ɲɯ-ŋu}  \\
\textsc{dem} \textsc{3sg.poss}-horn \textsc{dem} \textsc{pfv}-be.old \textsc{lnk}  \textsc{lnk} \textsc{dem} itself \textsc{ipfv}-<\textsc{auto}>fall \textsc{sens}-be \\
\glt `Its horn, when it becomes old, it falls off by itself.' (Deer, 35)
\end{exe}


\subsection{Non-controllable intransitive (abs) $\rightarrow$ Dynamic intransitive (abs)}  
Most non-controllable intransitive verbs of Tibetic origin are borrowed as dynamic intransitive verbs in Japhug, as for instance \ipa{sci} `be born' from a form corresponding to Classical Tibetan \ipa{skʲe} (same meaning) or \ipa{βzi} `be drunk' from \ipa{bzi} (same meaning). In both languages, the only argument of such verbs has no overt case marking.

There is however one critical difference between the two languages with this verb class. In Amdo Tibetan, it is possible in these cases to add an instrument without changing the verb form, as in \refb{ex:bzE}.

\begin{exe}
\ex \label{ex:bzE}
\gll
\ipa{ʂtamɖʐən} 	\ipa{tɕʰaŋ-ɣə} \ipa{bzə-tʰa} 	 \\
NP alcohol-\textsc{erg/instr} become.drunk-\textsc{pfv.nvol.evid} \\
\glt `Rtamgrin became drunk from the Chang.' (\citealt[112, ex. 443]{haller04themchen})
\end{exe}

In Japhug, on the other hand, it is not possible to add a concrete agent/instrument without causativizing the verb with the prefix \ipa{sɯ-} and turning it into a transitive verb, as in example \refb{ex:lowGsWBzi}.\footnote{In the case of transitive verbs, the causative prefix is also most commonly added to verbs that take an instrumental postpositional phrase, though in that case causativization is only optional.}

\begin{exe}
\ex \label{ex:lowGsWBzi}
\gll
\ipa{cʰa} 	\ipa{kɯ} 	\ipa{ló-wɣ-sɯ-βzi} \\ 
alcohol \textsc{erg} \textsc{ifr-inv-caus}-become.drunk \\
\glt `He became drunk from the Chang.' (elicited)
\end{exe}


\subsection{Motion verb (\textsc{abs+dat}) $\rightarrow$ semi-transitive (\textsc{abs+loc/abs})}  
Motion verbs are intransitive verbs in Amdo Tibetan (and, with a few exceptions, in Japhug too), but those with a definite goal (such as `go', as opposed to `walk') require a locative oblique argument. In Amdo Tibetan, as illustrated by \refb{ex:rgyug}, this argument is marked in the Dative.

\begin{exe}
\ex \label{ex:rgyug}
\gll
\ipa{ʂtamɖʐən} 	\ipa{jo} \ipa{rdʑəç-kokə} 	 \\
NP home:\textsc{dat} run-\textsc{ipfv.nvol.evid} \\
\glt `Rtamgrin ran home.' (\citealt[80, ex. 125]{haller04themchen})
\end{exe}

In Japhug, motion verbs of native or Tibetic origin alike mark the goal with various relator nouns (\ref{ex:rJWG1}), with the Dative (\ref{ex:rJWG2}), with a relator noun combined with the locative \ipa{zɯ} (\ref{ex:rJWG3}). Alternatively, the goal can also remain in the Absolutive form, and thus these verbs can be considered to be semi-transitive. 

\begin{exe}
\ex \label{ex:rJWG1}
\gll
\ipa{tɕendɤre}  	\ipa{kɯ-mɯrkɯ}  	\ipa{nɯnɯ}  	\ipa{jo-rɟɯɣ}  	\ipa{tɕe}  	\ipa{tɕendɤre,}  	\ipa{nɯ}  	\ipa{rɟara}  	\ipa{ɯ-ŋgɯ}  	\ipa{chɤ-rɟɯɣ}  	\ipa{ri}  \\
\textsc{lnk} \textsc{nmlz}:S/A-steal \textsc{dem} \textsc{ifr}-run \textsc{lnk} \textsc{lnk} \textsc{dem} \textsc{yard} 3sg-inside \textsc{ifr:downstream}-run \textsc{lnk} \\
\glt `The thief ran, he ran out to the yard.' (Bulaimei de yinyuejia, 105)
\end{exe}

\begin{exe}
\ex \label{ex:rJWG2}
\gll
\ipa{rɯdaʁ}  	\ipa{nɯnɯ}  	\ipa{ɯ-ɕki}  	\ipa{jo-rɟɯɣ}  	\ipa{ʑo}  	\ipa{tɕe}  	\ipa{ɣɯ-ndza}  	\ipa{pjɤ-ŋu}  \\
animal \textsc{dem} \textsc{3sg-dat} \textsc{ifr}-run \textsc{emph} \textsc{lnk} \textsc{inv}-eat:\textsc{fact} \textsc{ifr.ipfv}-be \\
\glt `The animal ran towards him, and was about to eat him.' (Yonggan de xiao caifeng, 196)
\end{exe}


\begin{exe}
\ex \label{ex:rJWG3}
\gll
\ipa{cɯβloʁ}  	\ipa{ɯ-rkɯ}  	\ipa{zɯ}  	\ipa{jɤ-rɟɯɣ-a}  	\ipa{tɕe}  \\
pond \textsc{3sg}-side \textsc{loc} \textsc{pfv}-run-\textsc{1sg} \textsc{lnk} \\
\glt `I ran to the side of the pond.' (The lotus flower)
\end{exe}

Amdo Tibetan makes a distinction between controllable and non-controllable motion verbs, for instance between \ipa{rdʑəç} $\leftarrow$ \ipa{rgʲug} `run' and \ipa{rda} $\leftarrow$ \ipa{rdal} `exceed' (\citealt[123, ex. 543]{haller04themchen}), which receive different set of TAM and evidential suffixes. In Japhug, no such distinction exists, and  the borrowed verbs \ipa{rɟɯɣ} `run' and \ipa{rdɤl} `overshoot' (from \ipa{rgʲug} and \ipa{rdal} respectively) do not belong to different verb classes.


Unlike some motion verbs such as \ipa{ɕe} `go' and \ipa{ɣi} `come', which can take a purposive complement (see \citealt{sun12complementation}, \citealt{jacques13harmonization}), \ipa{rɟɯɣ} cannot. It is not however a peculiarity of this verb, as some notion motion verbs such as \ipa{nɯqambɯmbjom} `fly' are also unable to take purposive complements.

The verb \ipa{rdɤl} `overshoot' also has a temporal meaning `exceed (time)' and can be used with an infinitival complement in Japhug, as in \refb{ex:kordala}.

\begin{exe}
\ex \label{ex:kordala}
\gll
\ipa{aʑo}  	\ipa{kɤ-nɯʑɯβ}  	\ipa{ko-rdal-a}  \\
\textsc{1sg} \textsc{inf}-sleep \textsc{ifr}-exceed-\textsc{1sg} \\
\glt `I slept too much.' (elicited)
\end{exe}

No such construction has been reported in available sources on Tibetic languages, and it might be a Japhug-internal innovation.

\subsection{Motion verb (\textsc{abs+dat}) $\rightarrow$ Transitive (\textsc{erg+abs})}  
The verb \ipa{bʲol} `give way, turn away, avoid' in Classical Tibetan is intransitive, and like other motion verbs has an argument in the absolutive and another one in the Dative or in the Ablative \ipa{-las} (see \citealt[268]{hill10dictionary}). It is not attested in spoken Amdo as far as we can tell. In Japhug however, the verb \ipa{pjɤl} `go around' borrowed from  \ipa{bʲol} was turned into a transitive verb whose A corresponds to the S of the Classical Tibetan verb and whose P corresponds to the goal. Example \refb{ex:tukWnWpjalanW} provides an example  showing this verb in the transitive construction, with both A and P unambiguously indexed on the verb form.


\begin{exe}
\ex \label{ex:tukWnWpjalanW}
\gll
\ipa{ɯ́-ɲɯ-tɯ-mbɣom-nɯ}  	\ipa{nɤ,}  	\ipa{tu-kɯ-nɯmɢla-a-nɯ,}  \ipa{mɯ́-ɲɯ-tɯ-mbɣom-nɯ}  	\ipa{nɤ}  	\ipa{tu-kɯ-nɯ-pjal-a-nɯ}  	\ipa{ma}  	\ipa{a-zrɯɣ}  	\ipa{a-ndʑrɯ}  	\ipa{ɯ-tɯ-dɤn} \\
\textsc{qu-ipfv}-2-be.in.a.hurry-\textsc{pl} \textsc{lnk} \textsc{ipfv}-2$\rightarrow$1-step.over-\textsc{1sg-pl} \textsc{qu:neg-ipfv}-2-be.in.a.hurry-\textsc{pl} \textsc{lnk} \textsc{ipfv}-2$\rightarrow$1-\textsc{auto}-go.around-\textsc{1sg-pl} \textsc{lnk}   \textsc{1sg.poss}-louse \textsc{1sg.poss}-nit \textsc{3sg-nmlz:degree}-be.many \\
\glt `If you are in a hurry, step over me; if you are not, go around me, I am full of lice and nits.' (Kunbzang, 15)
\end{exe}


\subsection{Experiencer verb (\textsc{abs+dat}) $\rightarrow$ Semi-transitive (\textsc{abs+abs})}
Some experiencer verbs in Amdo Tibetan encode the experiencer with the Absolutive and the stimulus with the Dative. This is the case of \ipa{rga} `like' in Amdo (Classical Tibetan \ipa{dga}), as illustrated by \refb{ex:rgaGokE}.

\begin{exe}
\ex \label{ex:rgaGokE}
\gll \ipa{ʂtamɖʐən}  \ipa{bdeʂtɕəl-a}  \ipa{rga-ɣokə} \\
NP NP-\textsc{dat} like-\textsc{ipfv.nvol.evid} \\
\glt `Rtamgrin likes Bdeskyid.' (\citealt[123, ex. 547]{haller04themchen})
\end{exe}

The Japhug verb \ipa{rga} `like' borrowed from Amdo Tibetan \ipa{rga} is a semi-transitive verb; the experiencer is the S, while the stimulus is a semi-object (unmarked oblique). In example \refb{ex:rgaa}, the S is first person singular,  with an overt pronoun and indexation on the verb, while the semi-object \ipa{ndɯχu} `baihe flower' has no marking and is not indexed on the verb.\footnote{It is not possible to add a plural or a dual suffix marker coreferent with the semi-object on the verb.}  

\begin{exe}
\ex \label{ex:rgaa}
\gll
\ipa{tɕe} 	\ipa{aʑo} 	\ipa{ndɤre} 	\ipa{ndɯχu} 	\ipa{rga-a} \\
\textsc{lnk} \textsc{1sg} \textsc{top} baihe like:\textsc{fact}-\textsc{1sg} \\
\glt `As for me, I like the baihe flower.'
(Baihe 129)
\end{exe}


In addition, Japhug \ipa{rga}  occurs with infinitival or finite complement clauses (see example \ref{ex:kAnArtoXpjAt}).

\begin{exe}
\ex \label{ex:kAnArtoXpjAt}
\gll  
\ipa{ma} 	\ipa{aʑo} 	\ipa{qajɯ} 	\ipa{nɯra} 	\ipa{kɤ-nɤrtoχpjɤt} 	\ipa{pɯ-rga-a} \\
\textsc{lnk} \textsc{1sg} insects \textsc{dem:pl} \textsc{inf}-observe \textsc{pst.ipfv}-like-\textsc{1sg} \\
\glt `I used to like to observe animals.' (Dragonfly, 15)
\end{exe}

\subsection{Controllable transitive verb (\textsc{erg+abs})  $\rightarrow$ Transitive (\textsc{erg+abs})}
Most controllable transitive verbs in Classical and Amdo Tibetan have an argument in the ergative and another one in the absolutive, as Amdo \ipa{ndon} (Classical Tibetan \ipa{ɴdon}, past \ipa{bton(d)}) `read, recite' in \refb{ex:ndonGokE}.

%\begin{exe}
%\ex
%\gll  
%	\ipa{paʁ}  	\ipa{rcanɯ,}  	\ipa{ɯ-tɯpɯ}  	\ipa{roŋri}  	\ipa{kɯ}  	\ipa{ʑo}  	\ipa{pjɯ-χsu-nɯ}  	\ipa{ra}  \\
%	pig \textsc{top} \textsc{3sg.poss}-household each \textsc{erg} \textsc{emph} \textsc{ipfv}-raise-\textsc{pl} have.to:\textsc{fact} \\
%	\glt `The pigs, every household have to raise (some).' (Pigs, 4)
%\end{exe}
%\citealt[84]{haller04themchen}  , ex 165

\begin{exe}
\ex \label{ex:ndonGokE}
\gll \ipa{ʂtamɖʐən-ɣə}  \ipa{χwetɕʰa}  \ipa{ndon-ɣokə} \\
NP-\textsc{erg} book read-\textsc{ipfv.nvol.evid} \\
\glt `Rtamgrin reads the book.' (\citealt[94, ex. 265]{haller04themchen})
\end{exe}


Japhug \ipa{ndɯn} `read aloud, recite', borrowed from the present stem of this verb, presents a similar construction, with the core arguments marked in the ergative and absolutive (both are indexed on the verb; in example \refb{ex:YAndWn} the indexation is not overt since both arguments are third singular).

\begin{exe}
\ex \label{ex:YAndWn}
\gll \ipa{ɯʑo} 	\ipa{kɯ} 	\ipa{masɤrɯrju} 	\ipa{kʰɤndɯn} 	\ipa{nɯ} 	\ipa{ɲɤ-ndɯn} 	\ipa{tɕe} \\
\textsc{3sg} \textsc{erg} in.secret sutra \textsc{dem} \textsc{ifr}-recite \textsc{lnk} \\
\glt `He recited the sutra in secret.' (Slob.dpon1, 172)
\end{exe}

Most verbs of Tibetic origin in Japhug belong to this category and have the same argument structure as \ipa{ndɯn}. 

The Amdo Tibetan verb \ipa{çtɕi} `cherish, love' (Classical \ipa{gtɕes}) also belongs to this category, as shown by example \refb{ex:gces}, but presents an irregular correspondence with Japhug.

\begin{exe}
\ex \label{ex:gces}
\gll \ipa{ʂtamɖʐən-ɣə}  \ipa{ɕaʑi} \ipa{çɕiɣə}  \ipa{çtɕi-ɣə} \\
NP-\textsc{erg} children very like-\textsc{ipfv.nvol.evid} \\
\glt `Rtamgrin likes children very much.' (\citealt[86, ex:186]{haller04themchen})
\end{exe}

Japhug \ipa{χtɕɤz}, unlike the form corresponding to Classical Tibetan \ipa{gtɕes} from which it was borrowed, is not a transitive verb, but an adjectival stative verb meaning `be cherished', as in \refb{ex:pjAXtCAz}. 

\begin{exe}
\ex \label{ex:pjAXtCAz}
\gll
\ipa{kɯ-mɤku} 	\ipa{ɯ-tɕɯ} 	\ipa{nɯnɯ} 	\ipa{wuma} 	\ipa{mɯ-pjɤ-χtɕɤz} \\
\textsc{nmlz}:S/A-be.before \textsc{3sg.poss}-son \textsc{dem} really \textsc{neg-ipfv.ifr}-be.cherished \\
\glt `The first child was not loved.' (Smanmi4, 2)
\end{exe}

The reason for this mismatch between Tibetic languages and Japhug is unclear; the likeliest explanation is that \ipa{gtɕes} was a labile verb in the Tibetic language from which it was borrowed into Japhug.


Another puzzling case is the Japhug verb \ipa{βgoz} `prepare, organize' (as in \refb{ex:tABgoza}, which is borrowed from the controllable transitive verb \ipa{ɴgod} (past \ipa{bgod}) `design, plan, arrange'. Its phonological shape is unexpected (borrowing from the past stem \ipa{bgod} should have yielded *\ipa{βgot} or *\ipa{βgɤt} depending on the layer), and unlike Old Tibetan, it is a labile verb, which can be conjugated as a transitive verb (example \ref{ex:tABgoza}) or as a semi-transitive verb (\ref{ex:pWBgoz}).\footnote{The form \ipa{βgoz} resembles the past stem \ipa{bgos} of the verb \ipa{bgot} `share', but the semantic difference excludes such comparison.} 



\begin{exe}
\ex \label{ex:tABgoza}
\gll
\ipa{tɯ-ŋga} 	\ipa{ɯ-spa} 	\ipa{ci} 	\ipa{tɤ-βgoz-a} \\
\textsc{indef.poss}-clothes \textsc{3sg.poss}-material \textsc{indef} \textsc{pfv}-prepare-\textsc{1sg} \\
\glt `I prepared the materials (threads and cloth) to (make) clothes.'
\end{exe}

In example \refb{ex:pWBgoz}, note that the argument corresponding to the A of the Old Tibetan verb \ipa{pʰama} `parents' does not take ergative marking, and that the Perfective prefix is in the intransitive form (otherwise \ipa{pa-} would have been expected instead of \ipa{pɯ-}).

\begin{exe}
\ex \label{ex:pWBgoz}
\gll \ipa{ndʑi-stɯnmɯ} 	\ipa{nɯ} 	\ipa{pʰama} 	\ipa{pɯ-βgoz} 	\ipa{pɯ-ŋu} \\
\textsc{3du.poss}-marriage \textsc{dem} parents \textsc{pfv}-organize \textsc{pst.ipfv}-be \\
\glt `Their marriage was arranged by their parents.'
\end{exe}


It is interesting that the intransitive use of this verb only appears with the noun \ipa{pʰama} `parents' of Tibetic origin, though this verb is not reported to have such a behaviour in any Tibetic language.



\subsection{Controllable transitive verb (\textsc{erg+abs})  $\rightarrow$ Transitive with complement clause (\textsc{erg+abs})}
The Classical Tibetan verb \ipa{gtɕod}, past \ipa{btɕad} `cut' (Amdo \ipa{çtɕol, ptɕal}) also belongs to the class of controllable transitive verbs (see \citealt[86, ex. 187]{haller04themchen}). It was borrowed into Japhug as a transitive verb \ipa{ftɕɤt}, but this verb has a much more restricted meaning `abstain (from a bad habit)',\footnote{This meaning is also attested in Tibetan.} as in example \refb{ex:naftCAt}. Unlike the original Tibetan verb, it cannot take a NP object, only a complement clause.

\begin{exe}
\ex \label{ex:naftCAt}
\gll
\ipa{tʰamakʰa} \ipa{kɤ-sko} \ipa{na-ftɕɤt}\\
tobacco \textsc{inf}-smoke \textsc{pfv}:3$\rightarrow$3'-abstain \\
\glt `He quit smoking.' (elicited)
\end{exe}

\subsection{Controllable transitive verb (\textsc{erg+dat})  $\rightarrow$ Transitive (\textsc{erg+abs}) }
A minority of controllable transitive verbs in Tibetic languages require an argument in the Dative instead of the absolutive (see \citealt[111]{haller04themchen} for a list of such verbs in Amdo Tibetan). Only one verb of this class has been borrowed into Japhug: \ipa{ɴpʰja} `blame', corresponding to Japhug \ipa{mpɕa} `blame, scold'. As shown by example \refb{ex:matAkWmpCaa}, this verb is a transitive verb, and the argument that receives the Dative case in the Tibetic original verb corresponds to the P of the verb \ipa{mpɕa}.

\begin{exe}
\ex \label{ex:matAkWmpCaa}
\gll
  \ipa{ma-tɤ-kɯ-mpɕa-a} \\
  \textsc{neg-imp}-2$\rightarrow$1-blame-\textsc{1sg} \\
\glt `Don't blame me!' (from a conversation)
\end{exe}
  
  \subsection{Ditransitive (\textsc{erg+abs+dat})  $\rightarrow$ Ditransitive (\textsc{erg+abs+dat})}
Ditransitive verbs in Tibetic languages use an indirective flagging pattern, with the recipient in the Dative and the theme in the Absolutive (like the P of a monotransitive verb). Example \refb{ex:fCakkokE} illustrates this construction with the verb \ipa{ɸɕal} `tell' (Classical \ipa{bɕad}).

\begin{exe}
\ex \label{ex:fCakkokE}
\gll \ipa{ʂtamɖʐən-ɣə}  \ipa{bdeʂtɕəl-a} \ipa{ʂkatɕʰa} \ipa{ɸɕak-kokə} \\
NP-\textsc{erg} NP-\textsc{dat} words tell-\textsc{ipfv.nvol.evid} \\
\glt `Rtamgrin talks to Bdeskyid.' (\citealt[87, ex. 195]{haller04themchen})
\end{exe}

In Japhug, although both secundative and indirective verbs are attested, all ditransitive verbs of Tibetic origin are of the indirective type, as shown by examples \refb{ex:pjAfCAt} and \refb{ex:pjWfCAt} with the verb \ipa{fɕɤt} `tell' from a form corresponding to Classical Tibetan \ipa{bɕad}. The recipient receives either Dative (\ref{ex:pjAfCAt}) or Genitive (\ref{ex:pjWfCAt}) marking (the Genitive conveys a benefactive meaning).

\begin{exe}
\ex \label{ex:pjAfCAt}
\gll
  \ipa{rɟɤlpu} 	\ipa{ɯ-ɕki} 	\ipa{tɕe} 	\ipa{nɯra} 	\ipa{pɯ-kɯ-fse} 	\ipa{nɯra} 	\ipa{pjɤ-fɕɤt.} \\
  king \textsc{3sg-dat} \textsc{lnk} \textsc{dem:pl} \textsc{pfv-nmlz}:S/A-be.like  \textsc{dem:pl} \textsc{ifr}-tell \\
  \glt `He told the king the things that had happened.' (Sanpian sheye, 140)
\end{exe}

\begin{exe}
\ex \label{ex:pjWfCAt}
\gll
  \ipa{tɕendɤre} 	<shanluzuode> 	\ipa{nɯnɯ} 	\ipa{kɯ} 	\ipa{mɤ-kɯ-mbrɤt} 	\ipa{ʑo} 	\ipa{rɟɤlpu} 	\ipa{ɣɯ} 	\ipa{ɯ-χpi} 	\ipa{ntsɯ} 	\ipa{pjɯ-fɕɤt} 	\ipa{pjɤ-ŋu} 	\ipa{tɕe,} \\
  \textsc{lnk} PN \textsc{dem} \textsc{erg} \textsc{neg-inf:n.hum}-\textsc{anticaus}:cut \textsc{emph} king \textsc{gen} \textsc{3sg.poss}-story always \textsc{ipfv}-tell \textsc{ipfv.ifr}-be \textsc{lnk} \\
\glt `Sheherazad told stories to the king without stop.' (Arabian Nights introduction, 57)
\end{exe}

  \subsection{Honorific verbs}  
All Tibetic languages have a more or less elaborate system of honorific verbs and nouns (see for instance \citealt{hill08moriendi} for a detailed case study in Old Tibetan). Honorific words and their non-honorific counterparts are generally suppletive, though there is some evidence for a non-productive honorific *\ipa{-j-} infix (\citealt{gong77}).

In Japhug, I only found three honorific verbs, all of which are of Tibetic origin (see Table \ref{tab:hon}). These verbs are rarely used in normal speech, and appear to be restricted to religious contexts.


\begin{table}[H]
\caption{Honorific verbs of Tibetic origin in Japhug} \centering \label{tab:hon}
\begin{tabular}{lllll}
\toprule
Classical Tibetan & Meaning & Japhug &Meaning &Morphosyntactic type \\
\midrule
\ipa{ɴkʰruŋ} & be born & \ipa{mkʰroŋ} &be born &dynamic intransitive \\
\ipa{pʰul} & offer, give  & \ipa{pʰɯl} &offer &ditransitive \\
\ipa{gɕegs} & go, pass away & \ipa{χɕaʁ} &pass away &dynamic intransitive \\
\bottomrule
\end{tabular}
\end{table}

In addition, in Japhug the plural (for second and third persons) can be used to express respect, as in example \refb{ex:chWkWtCatanW}. 

\begin{exe}
\ex \label{ex:chWkWtCatanW}
\gll \ipa{wortɕhiwojɤr}  	\ipa{ʑo}  	\ipa{chɯ-kɯ-tɕat-a-nɯ}  	\ipa{ɯ́-pe}  \\
please \textsc{emph} \textsc{ipfv:downstream}-2$\rightarrow$1-take.out-\textsc{1sg-pl} \textsc{qu}-be.good:\textsc{fact} \\
\glt `Please, could you(\textsc{sg:hon}) take me out of here?' (Gesar 141)
\end{exe}

Although there is no person or number indexation in any Tibetic language, the second person plural pronoun \ipa{kʰʲed} becomes used as a second person singular honorific after the Old Tibetan period (see \citealt[563-4]{hill10pronouns}). It cannot therefore be excluded that the honorific use of the plural in Japhug arose under the influence of Tibetic languages, first with the pronouns and then on verbal indexation.

\subsection{General Overview}
The ten main correspondences between Japhug and Tibetic verbs are listed in Table \ref{ex:correspondences}. Several of these categories are only attested by one example (\ipa{bʲol} / \ipa{pjɤl} `go around', \ipa{dga} /\ipa{rga} `like', \ipa{mpʰja} / \ipa{mpɕa} `blame', \ipa{btɕad} / \ipa{ftɕɤt} `quit (bad habit)'), but some categories (stative, dynamic intransitive and transitive) have scores of examples. XXXXXXX

If these minor categories are excluded, the correspondences between Japhug and Tibetic are fairly straightforward. Arguments in the Absolutive and Ergative in Tibetan nearly always corresponds to arguments in the Absolutive and Ergative respectively in Japhug. Arguments that are encoded by the Dative in Tibetic languages have distinct correspondences depending on whether the argument in question is the goal of a motion verb or a recipient.

Despite the rich variety of cases in Old and Classical Tibetan, no verb requiring an argument in any case other than Absolutive, Ergative or Dative is attested among the verbs borrowed into Japhug, so that it is unclear how these would be adapted. It is possible that an argument marked with the Comitative/Associative \ipa{-daŋ} in Classical Tibetan (\citealt{tournadre10cases, hill12bas}) would be marked with the Comitative \ipa{cʰo} in Japhug. The closest case to such a correspondence is the verb \ipa{naχtɕɯɣ} `be like', which includes a postpositional phrase in \ipa{cʰo} in its argument structure (see \citealt[273]{jacques14linking}). Although this verb is not directly borrowed from Tibetan (it is a denominal verb derived from the Tibetan numeral \ipa{gtɕig} `one'), its closest Tibetic equivalent \ipa{ɴdra} `be like' takes an argument with the comitative \ipa{-daŋ}. It is possible that the Japhug construction is a calque from the Tibetic one.


\begin{table}[H]
\caption{Correspondences between argument structure type between Amdo/Classical Tibetan verbs and loanverbs in Japhug} \label{ex:correspondences}
\resizebox{\columnwidth}{!}{
\begin{tabular}{llllllll}
\toprule

 \multicolumn{3}{c}{Classical Tibetan}  & \multicolumn{3}{c}{Japhug}&&\\
Flagging & Example & Meaning & Argument structure & Example & Meaning & \\
\midrule
adjective & \ipa{brtan.po} &stable&  stative, \textsc{abs} & \ipa{frtɤn} &be reliable& \\
 \textsc{abs} &  \ipa{rgas} & become old &stative, \textsc{abs} & \ipa{rgɤz} & be/become old \\
 \textsc{abs} &  \ipa{bzi} & be drunk &intransitive, \textsc{abs} & \ipa{βzi} & be drunk \\
 \textsc{abs+dat} &  \ipa{rgʲug} &run &intransitive, \textsc{abs}+goal  & \ipa{rɟɯɣ} & run \\
\textsc{dat+abs} & \ipa{dga} & like & semi-transitive, \textsc{abs+abs}& \ipa{rga} & like \\
\midrule
 \textsc{abs+dat/abl} &  \ipa{bʲol} & give way to, avoid &transitive, \textsc{erg+abs}& \ipa{pjɤl} & go around \\
\textsc{erg+abs} & \ipa{ɴdon} & read & transitive, \textsc{erg+abs} & \ipa{ndɯn} & read aloud \\

\textsc{erg+dat} & \ipa{mpʰʲa} & blame & transitive, \textsc{erg+abs}& \ipa{mpɕa} & blame \\
\textsc{erg+abs} & \ipa{btɕad} & cut & transitive, \textsc{erg+}complement & \ipa{ftɕɤt} & abandon (bad habit) \\
\bottomrule
\textsc{erg+abs+dat} & \ipa{bɕad} & tell & transitive, \textsc{erg+abs+dat}& \ipa{fɕɤt} & tell \\
\bottomrule
\end{tabular}}
\end{table}  


  \section{Complex predicates}  
In addition to the numerous verbs borrowed by direct insertion from Tibetic languages into Japhug, I also find some borrowed complex predicates.

There are two examples of borrowed noun-verb complex predicates, whose etymology is transparent in Tibetic languages but completely opaque in Japhug. The first is \ipa{ɯ-ŋgu} \ipa{tʰon} `have a good family situation', comprising an inalienably possessed noun \ipa{ɯ-ŋgu} and an intransitive verb \ipa{tʰon}, which cannot be used on their own and mean nothing in Japhug (see example \ref{ex:ndZiNgu.mAthon}). They come from the noun \ipa{ɴgo} `head' and the verb \ipa{tʰon} `come out' in Classical Tibetan respectively, where the expression \ipa{ɴgo} \ipa{tʰon} has the same meaning as \ipa{ɯ-ŋgu} \ipa{tʰon}.


\begin{exe}
\ex \label{ex:ndZiNgu.mAthon}
\gll
\ipa{ci}  	\ipa{tʰɯ-kɯ-rgɯ\textasciitilde{}rgɤz}  	\ipa{ɲɯ-ɕti}  	\ipa{tɕe,}  	\ipa{ci}  	\ipa{kɯ-xtɕɯ\textasciitilde{}xtɕi}  	\ipa{ɲɯ-ɕti}  	\ipa{tɕe,}  	\ipa{ndʑi-ŋgu}  	\ipa{mɤ-tʰon}  	\ipa{tɕe,}  	  \\
one \textsc{pfv-nmlz}:S/A-\textsc{emph}\textasciitilde{}be.old \textsc{sens}-be:\textsc{affirm} \textsc{lnk} one \textsc{pfv-nmlz}:S/A-\textsc{emph}\textasciitilde{}be.small \textsc{sens}-be:\textsc{affirm} \textsc{lnk} \textsc{3du.poss}-have.a.good.family.situation(1) \textsc{neg}-have.a.good.family.situation(2):\textsc{fact} \textsc{lnk} \\
\glt `One of them is an old man, the other one a small child, their situation is dire (they cannot take care of themselves without me).' (Nyimawodzer 2011.129-130)
\end{exe}  

The other example is \ipa{tʰaʁ} \ipa{tɕʰot} `make a decision', comprising the noun \ipa{tʰaʁ} and the transitive verb \ipa{tɕʰot}, borrowed from a form corresponding to Classical Tibetan  \ipa{tʰag tɕʰod} `decide', analyzable as \ipa{tʰag} `rope' and \ipa{tɕʰod} `be cut off'.

A more common type of borrowed complex predicates involves incorporation. As shown in \citet{jacques12incorp}, Japhug has a class of incorporating verbs derived from noun-verb action nominals by means of denominal derivations.\footnote{For instance, I find \ipa{ɣɯ-cʰɤtshi} `drink a lot of alcohol' derived from the compound \ipa{cʰɤtsʰi} `alcohol drinking' by the denominal prefix \ipa{ɣɯ-}. The action nominal \ipa{cʰɤtsʰi} itself is a compound from the noun \ipa{cʰa} `alcohol' and the transitive verb \ipa{tsʰi} `drink'.}

Some incorporating verbs comprise both a nominal root and a verbal root of Tibetan origin, such as \ipa{ɣɯ-rɟɯ-fsoʁ} `accumulate fortune' from the possessed noun \ipa{-rɟɯ} `fortune' and the transitive \ipa{fsoʁ} `accumulate', borrowed from a form corresponding to Classical Tibetan \ipa{rgʲu} and \ipa{bsogs} (same meanings) respectively. In this case, it is more likely that compounding occurred within Japhug rather than supposing that the compound was borrowed from a Tibetic language, since the verb \ipa{ɣɯ-rɟɯ-fsoʁ} is synchronically analyzable in Japhug.

Another case of an incorporating verb comprises a native noun root with a verb of Tibetic origin,\footnote{No example of an incorporating verb with a noun of Tibetic origin and a native verb has been found in Japhug.} for instance  \ipa{nɯ-pʰaʁɲɤl} `lie on the side' from the possessed noun \ipa{-pʰaʁ} `side, half' and an unattested verb *\ipa{ɲɤl} borrowed from a form corresponding to Classical Tibetan \ipa{ɲal} `lie down', or \ipa{nɯ-mbrɤrɟɯɣ} `take part in a horse race' from \ipa{mbro} `horse' (native noun) and \ipa{rɟɯɣ} `run' (borrowed from a form corresponding to Classical Tibetan \ipa{rgʲug} `run').\footnote{This latter incorporating verb is a calque from a form corresponding to Classical Tibetan \ipa{rta rgʲug} `horse race'.}
  
\section{Derivations}
In Japhug, several voice derivations, such as the Causative, the Antipassive, the Passive, the Tropative, Facilitative, Reciprocal and Deexperiencer are very productive, and can be applied to most of the verbs that are not morphosyntactically or semantically incompatible. 


The fact that nearly all derivations can be applied to verbs borrowed from Tibetan, is a testimony of their productivity in Japhug. Table \ref{ex:derivations.loanverbs} lists representative examples of voice derivations from verbs of Tibetic origin in Japhug.

The Anticausative on the other hand is limited to a closed set of verbs, and the presence of a Tibetic loanword among verbs undergoing this derivation is significant: it proves that the Anticausative was still productive until at least after contact with Tibetic languages began (\citealt{jacques12demotion}).\footnote{This fact is also significant for Sino-Tibetan comparative linguistics. Anticausative derivation is found in nearly all branches of the family, including Chinese (on which see \citealt{sagart12sprefix}), but the verb \ipa{ʁndɤr} `scatter (vi)' is the only reported example of an Anticausative derivation applied to a loanword, and thus the only evidence for the productivity of this derivation in the whole family. }


\begin{table}[H]
\caption{Derivations from verbs of Tibetic origin in Japhug}\centering \label{ex:derivations.loanverbs}
\resizebox{\columnwidth}{!}{
\begin{tabular}{lllllll}
\toprule
Derivation & Classical Tibetan & Base verb (Japhug) & Derived verb \\
\midrule
Anticausative & \ipa{gtor} `scatter' & \ipa{χtɤr} `scatter (vt)' &  \ipa{ʁndɤr} `scatter (vi)'\\
Applicative & \ipa{dga} `like' & \ipa{rga} `like (semi-trans.)' &  \ipa{nɯ-rga} `like (vt)'\\
Causative & \ipa{bɕad} `tell' & \ipa{fɕɤt} `tell (vt)' &  \ipa{sɯ-fɕɤt} `make so. tell (vt)'\\
Passive & \ipa{bzo} `make' & \ipa{βzu} `make, do (vt)' &  \ipa{a-βzu} `become (vi)'\\
Antipassive & \ipa{ɴdon} `read, recite' & \ipa{ndɯn} `read, recite (vt)' &  \ipa{rɤ-ndɯn} `recite sutras (vi)'\\
Tropative & \ipa{ldan} `having X' & \ipa{dɤn} `be many (vi)' &  \ipa{nɤ-dɤn} `consider to be many (vt)'\\
Deexperiencer & \ipa{skʲid} `happy' & \ipa{scit} `be happy (vi)' &  \ipa{sɤ-scit} `be nice (vi)'\\
Facilitative & \ipa{ɴtɕʰaŋ} `hold in the hand' & \ipa{ntɕʰoz} `use (vt)' &  \ipa{nɯɣɯ-ntɕʰoz} `be easy to use (vi)'\\
\bottomrule
\end{tabular}}
\end{table}

There are no restrictions on multiple derivations with verbs of Tibetic origin. For instance, the deexperiencer verb \ipa{sɤ-scit} `be nice, be in such a way that (people) are happy' (see Table \ref{ex:derivations.loanverbs}) can receive the Tropative \ipa{nɤ-} prefix to derive the verb \ipa{nɤ-sɤ-scit} \textsc{trop-deexp}-be.happy `to consider to be nice'. Similarly, the causative \ipa{ɣɤ-rlaʁ} `destroy' from \ipa{rlaʁ} `disappear' (from a form corresponding to Classical Tibetan \ipa{brlags} `destroy') can further undergo reciprocal derivation as  \ipa{a-ɣɤ-rlɯ\textasciitilde{}rlaʁ} \textsc{recip-caus-recip}\textasciitilde{}disappear `destroy each other'.
  
\section{Conclusion}
In this paper, I have documented the correspondences of argument structure between Japhug verbs of Tibetic origin and their equivalents in Amdo and Classical Tibetan. The main findings of this paper are threefold.

First, verbs borrowed from Tibetic languages in Japhug are borrowed mainly by \textit{direct insertion}, while those of Chinese origin are exclusively borrowed by \textit{indirect insertion}, by borrowing Chinese verbs as nouns and then applying to them denominal derivations (the most commonly attested process for verb borrowing crosslinguistically, see \citealt{wohlgemuth09verbal}). 

Second, correspondences of argument marking between Japhug and Tibetic languages, except for a few exceptional verbs, is relatively predictable: absolutive, Ergative-marked and Dative-marked arguments of verbs of Tibetic origin correspond to arguments with the same type of marking in Japhug. Since the ergative marker \ipa{kɯ} in Japhug is borrowed from Amdo Tibetan (and was borrowed at the latest three centuries ago, otherwise its expected phonological shape should be *\ipa{ciz} or *\ipa{kɯz}), the congruence of argument marking between the two languages in not unexpected; it is possible to speculate that at an earlier stage, Japhug and other Rgyalrong languages  had no core argument flagging, and that grammatical relations were exclusively expressed by verbal indexation.

Third, verbs of Tibetic origin, though they may present phonological peculiarities unattested in the native vocabulary, undergo the same morphological alternations (including stem alternation and derivations) as native verbs, and have been fully integrated in the language. Unlike what is found for example in the case of verbs of Greek origin in Coptic (\citealt{grossmann16contact}), Tibetic verbs in Japhug have not lead to the creation of a new transitivity category distinct from those attested in native verbs. 


\bibliographystyle{unified}
\bibliography{bibliogj}
\end{document}
