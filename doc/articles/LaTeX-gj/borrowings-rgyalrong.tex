

\documentclass[oldfontcommands,oneside,a4paper,11pt]{article} 
\usepackage{fontspec}
\usepackage{natbib}
\usepackage{booktabs}
\usepackage{xltxtra} 
\usepackage{longtable}
\usepackage{polyglossia} 
\usepackage[table]{xcolor}
\usepackage{gb4e} 
\usepackage{multicol}
\usepackage{graphicx}
\usepackage{float}
\usepackage{hyperref} 
\hypersetup{bookmarks=false,bookmarksnumbered,bookmarksopenlevel=5,bookmarksdepth=5,xetex,colorlinks=true,linkcolor=blue,citecolor=blue}
\usepackage[all]{hypcap}
\usepackage{memhfixc}
\usepackage{lscape}

\bibpunct[: ]{(}{)}{,}{a}{}{,}

%\setmainfont[Mapping=tex-text,Numbers=OldStyle,Ligatures=Common]{Charis SIL} 
\newfontfamily\phon[Mapping=tex-text,Ligatures=Common,Scale=MatchLowercase,FakeSlant=0.3]{Charis SIL} 
\newcommand{\ipa}[1]{{\phon \mbox{#1}}} %API tjs en italique
\newcommand{\ipab}[1]{{\scriptsize \phon#1}} 

\newcommand{\grise}[1]{\cellcolor{lightgray}\textbf{#1}}
\newfontfamily\cn[Mapping=tex-text,Ligatures=Common,Scale=MatchUppercase]{MingLiU}%pour le chinois
\newcommand{\zh}[1]{{\cn #1}}
\newcommand{\refb}[1]{(\ref{#1})}


\XeTeXlinebreaklocale 'zh' %使用中文换行
\XeTeXlinebreakskip = 0pt plus 1pt %
 %CIRCG
 


\begin{document} 
\title{On Intra-Rgyalrong language contact: the case of Tshobdun and Situ loanwords in Japhug }
%\author{Guillaume Jacques}
\maketitle


While borrowings from Tibetan in Rgyalrong languages are easily identifiable (\citealt{jacques04these}), borrowings between Rgyalrongic languages are more difficult to detect. 

In this paper, we propose a set of phonological criteria to distinguish borrowings from Tshobdun and Situ from inherited vocabulary.


The first and easiest method to detect borrowings is cross-Japhug dialect comparison. There are a few words in Kamnyu Japhug, one of the easternmost dialect of this language (in direct contact with Tshobdun speakers) that present unique correspondences with Western (`Xtokavian') dialects, and whose pronunciation is closer to that of the Tshobdun cognate. 

For instance, Kamnyu Japhug \ipa{qro} `ant' lost the final uvular still preserved in Tatshi \ipa{qroχ(ma)} `ant'; loss of final *\ipa{-ʁ} is a regular sound change in Tshobdun, where the cognate of this word is \ipa{qrɔ} `ant'. In this particular case, there is little doubt that Kamnyu Japhug has borrowed from Tshobdun.

Another case is when borrowing took place from pre-Tshobdun (before some sound changes), as in the case of Kamnyu Japhug \ipa{qaliaʁ} `eagle' (Tatshi Japhug \ipa{qarɟaʁ}`eagle'), which has the same onset as  Tshobdun \ipa{qália} `eagle', but represents a stage before *\ipa{--aʁ} $\Rightarrow$ \ipa{-a}.

Borrowing from Situ, given the higher status of that language in the traditional society, occurred in all varieties of Japhug, so that intra-Japhug comparison is of little help.

Failure of some sound changes to occur can only be interpreted in some cases as borrowings. For instance, \ipa{rkaŋ} `robust' was borrowed from the root *\ipa{rkaŋ} (Japhug \ipa{rko} `hard') from another Rgyalrong language (probably Situ) after the sound change *\ipa{-aŋ} $\Rightarrow$ \ipa{-o} took place, and before *\ipa{-aŋ} $\Rightarrow$ \ipa{-o} took place in Situ (this example incidentally demonstrates that although Japhug and Situ share the sound change *\ipa{-aŋ} $\Rightarrow$ \ipa{-o}, it did not occur at the same period and does not go back to their common ancestor).

In other cases, Situ-like sound changes in Japhug betray their foreign origin. This is the case of a few words which have velars instead of expected uvulars in Japhug, like \ipa{kʰɯtsa} `bowl' and \ipa{tɯ-ku} `head' (where Stau has \ipa{qʰəzə} and \ipa{ʁə} respectively, and uvular are found elsewhere, as pointed out by \citealt{gong15uvulars}).

%\ipa{ɕɤmɯɣdɯ} ?

  \bibliographystyle{unified}
\bibliography{bibliogj}
\end{document}