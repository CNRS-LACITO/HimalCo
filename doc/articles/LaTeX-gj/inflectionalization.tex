\documentclass[11pt]{article} 
\usepackage{fontspec}
\usepackage{natbib}
\usepackage{booktabs}
\usepackage{xltxtra} 
\usepackage{polyglossia} 
\usepackage[table]{xcolor}
\usepackage{tikz}
\usetikzlibrary{trees}
\usepackage{gb4e} 
\usepackage{multicol}
\usepackage{graphicx}
\usepackage{float}
\usepackage{hyperref} 
\hypersetup{bookmarks=false,bookmarksnumbered,bookmarksopenlevel=5,bookmarksdepth=5,xetex,colorlinks=true,linkcolor=blue,citecolor=blue}
\usepackage[all]{hypcap}
\usepackage{memhfixc}
\usepackage{lscape}
\usepackage{bbding}
 
%\setmainfont[Mapping=tex-text,Numbers=OldStyle,Ligatures=Common]{Charis SIL} 
\newfontfamily\phon[Mapping=tex-text,Ligatures=Common,Scale=MatchLowercase]{Charis SIL} 
\newcommand{\ipa}[1]{{\phon\textbf{#1}}} 
\newcommand{\ipap}[1]{{\phon#1}} 
\newcommand{\grise}[1]{\cellcolor{lightgray}\textbf{#1}}
\newfontfamily\cn[Mapping=tex-text,Ligatures=Common,Scale=MatchUppercase]{SimSun}%pour le chinois
\newcommand{\zh}[1]{{\cn #1}}
\newcommand{\Y}{\Checkmark} 
\newcommand{\N}{} 
\newcommand{\dhatu}[2]{|\ipa{#1}| `#2'}
\newcommand{\jpg}[2]{\ipa{#1} `#2'}  
\newcommand{\refb}[1]{(\ref{#1})}
\newcommand{\tld}{\textasciitilde{}}
\newfontfamily\mleccha[Mapping=tex-text,Ligatures=Common,Scale=MatchLowercase]{Galatia SIL}%pour le grec
\newcommand{\grec}[1]{{\mleccha #1}}


 \begin{document} 
\title{The analogical verbalization of predicative words}
% The examples are taken from a corpus that is progressively being made available on the Pangloss archive (\citealt{michailovsky14pangloss}, \url{http://lacito.vjf.cnrs.fr/pangloss/corpus/list\textunderscore rsc.php?lg=Japhug}).  }}
\author{Guillaume Jacques\\ CNRS-CRLAO-INALCO}
\maketitle

\section*{Introduction}



In languages with person indexation on the verb which are not omnipredicative (\citealt{launey94}), presence of person indexation is a robust criterion to tell apart verbs (including adjectival stative verbs) from other parts of speech. XXX


While derivation of adverbs from former imperatives is relatively common (Greek \grec{φέρε}, French voici XXX), the evolution following the opposite direction of the gramamticalization cline, while well known since at least \citet[414]{pott1859}, is much less common. XX


Adverbs or noun and participle used adverbially (in particular, but not exclusively, in oblique cases) XXX

\section{Adverbs of location, motion and manipulation}

\subsection{`here, take it'}
\citet[114]{viti15wandel}
\grec{τῆ}, \grec{τῆτε}

\citet[156]{miklosisch1883}
\ipa{na} `take it' \ipa{nata}, \ipa{nate}

%nuj
%bali
%jeli
%nikar

\subsection{`hither'}

\grec{δεῦρο}, \grec{δεῦτε}


%Lehman 103
The Gothic adverb \ipa{hiri} `hither' (translates Greek \grec{δεῦρο} `hither' or \grec{ἔρχου} `come', \citealt[185; H67]{lehmann86gothic}) is doubly anomalous. First, it is one of the very rare native words where \ipa{*i} (from proto-Germanic \ipa{*i} and \ipa{*e}) has not changed to <ai> [\ipap{ɛ}]. The most probable reason for this irregular is analogical influence from other locatives containing the element \ipa{hi-} such as \ipa{hidre} `hither' (translates \grec{ὧδε}) (\citealt{cercignani84hiri})

 \ipa{hirjats}, \ipa{hirjiþ}
 \citet[104]{braune53gotische}

\begin{exe}
\ex 
\gll \ipa{hirj-ats} \ipa{afar} \ipa{mis} \\
hither-\textsc{2du.imp} after \textsc{1sg:dat} \\
\glt `Come ye after me.' (\grec{δεῦτε ὀπίσω μου}, Mark 1:17)
\end{exe}

\begin{exe}
\ex 
\gll
\ipa{hirj-iþ} \ipa{usqim-am} \ipa{imma} \ipa{jah} \ipa{unsar} \ipa{wairþ-iþ} \ipa{þata} \ipa{arbi} \\
hither-\textsc{2du.imp} kill-\textsc{1pl:n.pst:ind} 3sg:dat and  ours:\textsc{nom:sg.n} become-\textsc{3sg:n.pst:ind} \textsc{dem:nom:sg:n} inheritance:\textsc{nom:sg:n} \\
\glt `Let us kill him, and the inheritance shall be ours.' (\grec{δεῦτε ἀποκτείνωμεν αὐτόν, καὶ ἡμῶν ἔσται ἡ κληρονομία.}, Mark 12:7)
\end{exe}

\citet[415]{osthoff1881morphologische}

\section{Adverbs of manner}

\grec{ἠμί}, \grec{ἦ}
A type of degrammaticalization which \citet[135]{norde09degrammaticalization} refers to as `degrammation'.

\section{Phatic expressions}
Japhug \ipa{sɤrma} `good night', an expression used to address someone leaving one's house in the evening, transparently derives from the oblique participle \ipa{sɤ-rma} `place where/time when one stays overnight' from the verb \ipa{rma} `to stay the night, live', as illustrated by example \refb{ex:Ckurma}.

\begin{exe}
\ex \label{ex:Ckurma}
\gll 
\ipa{tɕe}	\ipa{ɯ-sɤ-rma}	\ipa{nɯnɯ,}	\ipa{praʁ,}	\ipa{praʁpa}	\ipa{tɕe}	\ipa{ɕ-ku-rma}	\ipa{ɲɯ-ŋu} \\
\textsc{lnk} \textsc{3sg.poss-nmlz:oblique}-stay.the.night \textsc{dem} cliff cavern \textsc{lnk} \textsc{transloc-ipfv}-stay.the.night \textsc{sens}-be \\
\glt `The place where its stays during the night is the cliffs, it goes to spend the night in caverns under the cliffs.' (hist-20-xsar, 36)
\end{exe}

Yet, when addressing more than one person, the dual form  \ipa{sɤrma-ndʑi} and the plural  \ipa{sɤrma-nɯ} are used, with the same suffixes \ipa{-ndʑi} and \ipa{-nɯ} as found in verb paradigms (see \citealt{jacques17sketch}). These suffixes normally only appear on finite verbs. The forms \ipa{sɤrma}, \ipa{sɤrma-ndʑi} and \ipa{sɤrma-nɯ} could in principle be analyzed as a Factual Non-Past form (the only finite verb form without a directional prefix in Japhug), but there are three problems with this analysis. First, the meaning of the expression (`have a good night') is hardly compatible with the factual non-past; an Imperative or Irrealis form would be expected instead. Second, while there are several \ipa{sɤ-} verbal derivational prefix, none of them (de-experiencer, antipassive and causative, see \citealt{jacques14antipassive}) has a function compatible which could account for a derivation such as `spend the night' $\rightarrow$ `(have a) good night'.
Third, no other verb forms (including first or third person), finite or non-finite, are attested for \ipa{sɤrma}. 

A more promising approach to account for these verb forms is analogy with other phatic expressions involving finite verb forms. The most probable one is the verb \ipa{astu} `be straight', whose Imperative is used to mean `goodbye' (litterally `(walk) straight') as in \refb{ex:tAstundZi}.

\begin{exe}
\ex \label{ex:tAstundZi}
\gll \ipa{tɤ-ɤstu-ndʑi} \\
\textsc{imp}-be.straight-\textsc{du} \\
\glt `Goodbye' = `(walk) straight'
\end{exe}

The dual and plural forms of \ipa{sɤrma} `good night' can thus be explained as trivial four-part analogy as in \refb{ex:analogy.sArma}.

\begin{exe}
\ex \label{ex:analogy.sArma}
\gll
\ipa{tɤstu} `goodbye.\textsc{sg}' :: {\ipa{sɤrma} `good night.\textsc{sg}'} \\
\ipa{tɤstundʑi} `goodbye.\textsc{du}' :: {X $\rightarrow$ \ipa{sɤrmandʑi} `good night.\textsc{du}'} \\
\end{exe}

\section{Exclamative adverbs}
Japhug \ipa{dɯxpa} `poor of ...' is borrowed from Tibetan \ipa{sdug.pa} with the same meaning (spelled \ipa{sdug.ga} in \citealt{bodrgya}),\footnote{For an account of the alternation between \ipa{p/b} and \ipa{g} in the spelling, see \citet{hill11hb} and the references therein.} originally from a noun meaning `suffering'.

\begin{exe}
\ex 
\gll
\ipa{tɕiʑo} 	\ipa{ndɤ} 	\ipa{dɯxpa-tɕi} 	\ipa{ɣe,} 	\ipa{nɤ} 	\ipa{tɕendɤre,} 	\ipa{kɤ-ntɕha} 	\ipa{ɯ-spa} 	\ipa{ʑo} 	\ipa{ɕti-tɕi}  \\
\textsc{1du} on.the.other.hand pitiful-\textsc{1du} \textsc{sfp} \textsc{lnk} \textsc{lnk} \textsc{nmlz}:P-kill \textsc{3sg.poss}-material \textsc{emph} be.\textsc{assert:fact-1du} \\
\glt `We, on the other hand, poor of us! We are to be butchered.' (kandZislama2003.210)
\end{exe}

\section{Indirect pathways towards a finite verb}

Noun/participle (absolutive, instrumental, vocative)

demonstratives

\section*{Conclusion}

\bibliographystyle{unified}
\bibliography{bibliogj}

 \end{document}
 