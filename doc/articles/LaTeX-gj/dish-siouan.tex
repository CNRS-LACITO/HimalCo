\documentclass[oldfontcommands,twoside,a4paper,12pt]{article} 
\usepackage{fontspec}
\usepackage{natbib}
\usepackage{booktabs}
\usepackage{xltxtra} 
\usepackage{longtable}
\usepackage{tangutex2} 
%\usepackage{tangutex4} 
\usepackage{polyglossia} 
\usepackage[table]{xcolor}
\usepackage{color}
\usepackage{multirow}
\usepackage{gb4e} 
\usepackage{multicol}
\usepackage{graphicx}
\usepackage{float}
\usepackage{hyperref} 
\hypersetup{bookmarks=false,bookmarksnumbered,bookmarksopenlevel=5,bookmarksdepth=5,xetex,colorlinks=true,linkcolor=blue,citecolor=blue}
\usepackage{memhfixc}
\usepackage{lscape}
\usepackage[footnotesize,bf]{caption}


%%%%%%%%%%%%%%%%%%%%%%%%%%%%%%%
\setmainfont[Mapping=tex-text,Numbers=OldStyle,Ligatures=Common]{Charis SIL} 
\setsansfont[Mapping=tex-text,Ligatures=Common,Mapping=tex-text,Ligatures=Common,Scale=MatchLowercase]{Lucida Sans Unicode} 
 


\newfontfamily\phon[Mapping=tex-text,Ligatures=Common,Scale=MatchLowercase,FakeSlant=0.3]{Charis SIL} 
\newfontfamily\phondroit[Mapping=tex-text,Ligatures=Common,Scale=MatchLowercase]{Doulos SIL} 
\newfontfamily\greek[Mapping=tex-text,Ligatures=Common,Scale=MatchLowercase]{Doulos SIL} 
\newcommand{\ipa}[1]{{\phon\textbf{#1}}} 
\newcommand{\ipab}[1]{{\phon #1}}
\newcommand{\ipapl}[1]{{\phondroit #1}} 
\newcommand{\captionft}[1]{{\captionfont #1}} 
\newfontfamily\cn[Mapping=tex-text,Ligatures=Common,Scale=MatchUppercase]{MingLiU}%pour le chinois
\newcommand{\zh}[1]{{\cn #1}}
\newcommand{\tgf}[1]{\mo{#1}}
\newfontfamily\mleccha[Mapping=tex-text,Ligatures=Common,Scale=MatchLowercase]{Galatia SIL}%pour le grec
\newcommand{\grec}[1]{{\mleccha #1}}

 
\newcommand{\racine}[1]{\begin{math}\sqrt{#1}\end{math}} 
\newcommand{\grise}[1]{\cellcolor{lightgray}\textbf{#1}} 

\begin{document}
  \title{Disharmonization of affix ordering in Siouan}
\author{Guillaume Jacques}
\maketitle




 

  \section{Introduction}
%Since  \citet[93]{greenberg66}, various typological studies have revealed a cross-linguistic correlation between basic word order and morpheme ordering. 
 %\citet{vennemann74analogy}  \citet{lehmann73structural}  \citet[93]{greenberg66}, 
 
Languages with strict OV  order and mainly prefixing morphology are very rare among the world's languages. In the WALS' database, 6 six examples are found, as can be seen on Table \ref{tab:wals}, obtained by combining the data in chapter 26 (prefixing vs. suffixing inflectional morpholog, \citealt{dryer11chapter26}) with that of chapter  83 (OV vs. VO order, \citealt{dryer11ov}).

\begin{table}[H]

\caption{Correlation between OV order and prefixing vs. suffixing inflectional morphology (data from WALS); each row in the table adds up to 100\%} \label{tab:wals}
\resizebox{\columnwidth}{!}{
\begin{tabular}{llllllllll}
\toprule
 &	  	OV   &&	VO  &&	No &dominant order &Total \\
 \midrule
Little affixation  & 	35 & 	(25.0\%) & 	100 & 	(71.4\%) & 	5 & 	(3.6\%) & 	140 & 	(14.8\%) \\ 
Strongly suffixing  & 	269 & 	(68.6\%) & 	93 & 	(23.7\%) & 	30 & 	(7.7\%) & 	392 & 	(41.5\%) \\ 
Weakly suffixing  & 	70 & 	(57.9\%) & 	44 & 	(36.4\%) & 	7 & 	(5.8\%) & 	121 & 	(12.8\%) \\ 
Equal prefixing and suffixing  & 	49 & 	(34.5\%) & 	78 & 	(54.9\%) & 	15 & 	(10.6\%) & 	142 & 	(15\%) \\ 
Weakly prefixing  & 	23 & 	(25.0\%) & 	61 & 	(66.3\%) & 	8 & 	(8.7\%) & 	92 & 	(9.7\%) \\ 
Strongly prefixing  & 	6 & 	(10.3\%) & 	52 & 	(89.7\%) & 	 & 	(0.0\%) & 	58 & 	(6.1\%) \\ 
\midrule
 & 	452 & 	(47.8\%) & 	428 & 	(45.3\%) & 	65 & 	(6.9\%) & 	945 & 	(100\%) \\ 
\bottomrule
\end{tabular}}
\end{table}

Typologists have proposed several explanations to account for the rarity of the prefixing / OV type. In particular, a similar idea was put forth by \citet[227]{hawkins88prefixing}, who also added a second principle to account for the suffixing preference:
\begin{exe}
\ex \label{ex:hawkins}
\begin{xlist}[(ii)]
\exi{(a)} 
\glt \textbf{The Head Ordering Principle}: The affixal head of a word is ordered on the same side of its
subcategorized modifier(s) as P is ordered relative to NP within PP, and as V is ordered relative to a direct object NP.
\exi{(b)} 
\glt \textbf{Processing preference for suffixing}:
Lexical recognition precedes syntactic processing, so language users will prefer to process stems before affixes. Stem-affix order provides the most efficient structure for processing.
   \end{xlist}
\end{exe}

\citet{mithun03prefixes}

\citet{rice2000scope}

\citet{jacques13harmonization}

\section{Siouan: suffixing or prefixing?}
\citet{taylor96lakhota}

\citet[155]{reuse82enclitics}

(5) Ya'u kta he? Kte šni. 4
'Will you come? I will not.' (Bd:34)

\section{Instrumental prefixes}

\bibliographystyle{myenbib}
\bibliography{bibliogj}
\end{document}