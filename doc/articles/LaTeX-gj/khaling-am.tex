\documentclass[oneside,a4paper,11pt]{article} 
\usepackage{polyglossia}
%\usepackage{fontspec}
\usepackage{natbib}
\usepackage{booktabs}
\usepackage{xltxtra} 
\usepackage{longtable}
 \usepackage{geometry}
\usepackage[table]{xcolor}
\usepackage{color}
\usepackage{multirow}
\usepackage{gb4e} 
\usepackage{multicol}
\usepackage{graphicx}
\usepackage{float}
\usepackage{hyperref} 
\hypersetup{colorlinks=true,linkcolor=blue,citecolor=blue}
\usepackage{amssymb} 
\usepackage{memhfixc}
\usepackage{lscape}
%\usepackage[footnotesize,bf]{caption}
\usepackage{lineno}

%%%%%%%%%%%%%%%%%%%%%%%%%%%%%%%
%\setmainfont[Mapping=tex-text,Numbers=OldStyle,Ligatures=Common]{Charis SIL} 
\newfontfamily\phon[Mapping=tex-text,Ligatures=Common,Scale=MatchLowercase]{Charis SIL} 
 \newcommand{\ipa}[1]{{\phon#1}} 
\newcommand{\ipab}[1]{{\phon #1}}
\newcommand{\ipapl}[1]{{\phondroit #1}} 
\newcommand{\captionft}[1]{{\captionfont #1}} 
\newfontfamily\cn[Mapping=tex-text,Scale=MatchUppercase]{MingLiU}%pour le chinois
\newcommand{\zh}[1]{{\cn #1}}
\bibpunct[: ]{(}{)}{,}{a}{}{,}
 
\newcommand{\dhatu}[2]{|\ipa{#1}| `#2'}
\newcommand{\grise}[1]{\cellcolor{lightgray}\textbf{#1}} 
 \newcommand{\tld}{\textasciitilde}

\begin{document}



\title{Associated motion in Khaling } 
\author{Guillaume Jacques\\Aimée Lahaussois}
\maketitle
\linenumbers
\section{Introduction}

\section{Bipartite verb paradigms in Khaling}
Associated motion is expressed in Khaling by the second verbal stem (henceforth V_2) of bipartite verb constructions. These constructions are well attested in all Kiranti languages (see in particular \citealt[118-132]{driem87}, \citealt[199-214]{driem93dumi},  \citealt[137-194]{rutgers98yamphu}, \citealt{bickel07chintang}, \citealt[170-172]{doornenbal09} and \citealt[283-328]{schackow15yakkha}). The description of bipartite verb paradigms is thus a prerequisite to any discussion of AM in Khaling, and Kiranti languages in general.

Previous work on Khaling verbal morphology (\citealt{jacques12khaling, jacques16si}) has focused on simple verbs, and though Khaling bipartite verbs have been discussed in a comparative context (\citealt{jacques18bipartite}), the complete paradigms have never been presented. The focus of the present paper being AM, the intricacies of verbal stem alternations cannot be dealt here in a systematic way, and only the minimal necessary information will be presented.

\subsection{V1 stem alternations}

Table \ref{tab:kurledu} illustrates the paradigm of the circumambulative V_2 \ipa{-le-} (see \ref{sec:v2.le}) with two verbs in closed syllable roots,\footnote{The abstract roots are indicated between bars |, following the conventions in \citet{jacques12khaling}.} the intransitive verb \dhatu{ŋok}{cry} (`go around crying') and the transitive \dhatu{kur}{carry on the back} (`carry around'). Only part of the paradigm are given: the dual and plural exclusive, the second and third person dual,  and for transitive verbs the inverse configurations (3$\rightarrow$1/2) and the local forms (1$\rightarrow$2, 2$\rightarrow$1) are not given here, as their stem forms are derivable from those presented in this table (see \citealt{jacques12khaling} for a complete account).

%ʦ ʣ ɵ ː

\begin{table}[H]
\caption{Paradigm of bipartite verbs (with the circumambulative \ipa{-le-} as V2)} \label{tab:kurledu}  \centering
\begin{tabular}{lllllll}
\toprule
&\multicolumn{2}{c}{\textsc{n.pst}} &  \multicolumn{2}{c}{\textsc{pst}} \\
&Simple verb &  Bipartite verb &  Simple verb&  Bipartite verb \\
\midrule
\textsc{1s} &  \ipa{ŋôŋ-ŋʌ} & \ipa{ŋôŋ-\textbf{le}-ŋʌ} & \ipa{ŋɵk-ʌtʌ} & \ipa{ŋɵk-\textbf{les}-tʌ} \\
\textsc{1di} &  \ipa{ŋɵk-i} & \ipa{ŋɵk-\textbf{leʦ}-i} & \ipa{ŋɵk-iti} & \ipa{ŋɵk-\textbf{les}-ti} \\
\textsc{1pi} &  \ipa{ŋok-ki} & \ipa{ŋok-\textbf{le}-ki} & \ipa{ŋok-tiki} & \ipa{ŋok-\textbf{le}-ktiki} \\
\textsc{2s} &  \ipa{ʔi-ŋôː} & \ipa{ʔi-ŋôː-\textbf{le}} & \ipa{ʔi-ŋɵk-tɛ} & \ipa{ʔi-ŋɵk-\textbf{les}-tɛ} \\
\textsc{2p} &  \ipa{ʔi-ŋôː-ni} & \ipa{ʔi-ŋô-n-\textbf{le}-ni} & \ipa{ʔi-ŋɵk-tɛnu} & \ipa{ʔi-ŋɵk-\textbf{les}-tɛnu}  \\
\textsc{3s} &  \ipa{ŋôː} & \ipa{ŋôː-\textbf{le}} & \ipa{ŋɵk-tɛ} & \ipa{ŋɵk-\textbf{les}-tɛ} \\
\textsc{3p} &  \ipa{ŋôː-nu} & \ipa{ŋô-n-\textbf{le}-nu} & \ipa{ŋɵk-tɛnu} & \ipa{ŋɵk-\textbf{les}-tɛnu} \\
\midrule
\textsc{1s$\rightarrow$3} &  \ipa{kur-u} & \ipa{kûr-\textbf{led}-u} & \ipa{kur-utʌ} & \ipa{kûr-\textbf{le}-tʌ} \\
\textsc{1di$\rightarrow$3} &  \ipa{kʉr-i} & \ipa{kʉ̂r-\textbf{leʦ}-i} & \ipa{kʉr-iti} & \ipa{kʉ̂r-\textbf{les}-ti} \\
\textsc{1pi$\rightarrow$3} &  \ipa{kʌ̄r-ki} & \ipa{kʌ̄r-\textbf{le}-ki} & \ipa{kʌ̄r-tiki} & \ipa{kʌ̄r-\textbf{le}-ktiki} \\
\textsc{2s$\rightarrow$3} &  \ipa{ʔi-kʉ̄ːr-ʉ} & \ipa{ʔi-kʉ̂r-\textbf{led}-ʉ} & \ipa{ʔi-kʉ̂r-tɛ} & \ipa{ʔi-kʉ̂r-\textbf{le}-tɛ} \\
\textsc{2p$\rightarrow$3} &  \ipa{ʔi-kʌ̄r-ni} & \ipa{ʔi-kʌ̄r-\textbf{le}-ni} & \ipa{ʔi-kʉr-tɛnu} & \ipa{ʔi-kʉr-\textbf{les}-tɛnu} \\
\textsc{3s$\rightarrow$3} &  \ipa{kʉ̄ːr-ʉ} & \ipa{kʉ̂r-\textbf{led}-ʉ} & \ipa{kʉ̂r-tɛ} & \ipa{kʉ̂r-\textbf{le}-tɛ} \\
\textsc{3p$\rightarrow$3} &  \ipa{kʉ̂r-nu} & \ipa{kʉ̂r-\textbf{let}-nu} & \ipa{kʉ̂r-tɛnu} & \ipa{kʉ̂r-\textbf{le}-tɛnu} \\
\bottomrule
\end{tabular}
\end{table}
  
  
The form of the V_1 can be predicted from the corresponding simple verb by the four following rules:

\begin{enumerate}
\item If the verb stem contains a final consonant, and is followed by a suffix beginning with a  consonant (as in \ipa{kʌ̄r-ki} `we bring it'), the V_1 stem is identical to that of the simple verb (\ipa{kʌ̄r-\textbf{le}-ki} `we bring it around').
\item If the verb stem contains a final consonant, and is followed by a suffix beginning with a  vowel, the coda is resyllabified with that suffix. For instance, the form \ipa{kur-u} `I bring it' is syllabified as \ipa{ku/ru}. In such cases, if the final consonant is an obstruent (as in \ipa{ŋɵk-i} `you and I cry'), the V_1 stem is identical to that of the simple verb (\ipa{ŋɵk-\textbf{lets}-i} `you and I go around crying'), but if it is a sonorant, a final tone results (as in \ipa{kur-u} $\rightarrow$   \ipa{kûr-\textbf{led}-u}).\footnote{Since the stress is rather on the V_2, the tonal contrast on the V_1 is difficult to perceive.} This reduction is presumably due to a sound law that applies elsewhere in Khaling, and constitutes one of the origins of the tonal contrast in this language (see \citealt{jacques16tonogenesis}).
\item If the verb stem ends in a vowel (\ipa{ʔi-ŋôː} `you cry'), the stem of the V_1 is identical to that of the simple verb if there are no indexation suffixes (\ipa{ʔi-ŋôː-\textbf{le}} `you go around crying'), but if an indexation suffix is present, its vowel is dropped, and the consonant of the suffix becomes the coda of the V_1 stem, with falling tone (\ipa{ʔi-ŋôː-ni}   $\rightarrow$ \ipa{ʔi-ŋô-n-\textbf{le}-ni}).
\item In the case of Ct stem verbs (not in Table \ref{tab:kurledu}), the \ipa{-t} is always removed. For instance, in the paradigm of the bipartite verb comprising \dhatu{gʰurt}{run with, drive} and the V_2 \ipa{-kʰoŋt-} `bring up' (see \ref{sec:v2.khoN}), the stem of \ipa{gʰʌ̄rd-u} `I run with it, I drive it' becomes \ipa{gʰʌ̄r-} in \ipa{gʰʌ̄r-kʰoŋd-u} `I bring it up running' (a form like $\dagger$\ipa{gʰʌ̄rtkʰoŋdu} would violate the phonotactic structure of the Khaling language).
\end{enumerate}

The above rules predict all the V_1 forms of bipartite verbs with all V_2, except the benefactive V_2 \ipa{-sʌ-}.\footnote{With this V_2, the strong stem (only found in the \textsc{1pi} in the paradigms in Table \ref{tab:kurledu}) is found in all forms other that the dual and \textsc{2p.pst$\rightarrow$3}. For instance, the bipartite verb combining \dhatu{kur}{carry} and the benefactive has the \textsc{1s$\rightarrow$3.n.pst} form \ipa{kʌ̄r-sʌt-u} `I carry it for him', with the strong stem \ipa{kʌ̄r-}, unlike the corresponding simplex verb  \ipa{kur-u}  I carry it'. This exception can be explained historically. Khaling has a \ipa{-t-} causative/applicative suffix, one of whose function is benefactive, but this suffix is not productive anymore (see \citealt{jacques15derivational.khaling}). Transitive verbs with this suffix follow a distinct conjugation (described in \citealt[1119-1122]{jacques12khaling}), with the strong stem  in all slots of the paradigms except dual and \textsc{2pl$\rightarrow$3.pst}. The verb \dhatu{kur}{carry} is among the few verbs possessing a benefactive \ipa{-t-} suffix form \dhatu{kurt}{carry for someone}. The V_1 stem form \ipa{kʌ̄r-} in \ipa{kʌ̄r-sʌt-u} `I carry it for him' is better analyzable as deriving from \ipa{kʌ̄rd-u} `I bring it for him' (the \textsc{1sg$\rightarrow$3} of \dhatu{kurt}{carry for someone}), with application of the rule 4 (loss of \ipa{-t/d-} in \ipa{-Ct-} stem verbs). This explanation is however only historical, as the same stem distribution is observed even for verbs that do not synchronically have a benefactive \ipa{-t-} suffix form.}

\subsection{V2 stem alternations}
Table \ref{tab:kurledu} illustrates the stem alternations of the circumambulative V_2 \ipa{-le-}, which presents three distinct stems \ipa{-le-}, \ipa{-leʦ-} and \ipa{-les-} in the intransitive paradigm, and two additional ones \ipa{-let-} and \ipa{-led-} in the transitive one.

No simple verb shows exactly the same alternation pattern; rather,  \ipa{-le-} combines stems from the paradigm of open syllable \ipa{-e-} verbs and close syllable \ipa{-et-} verbs, as shown in Table \ref{tab:le.dze.ret}. Forms with long vowels in the transitive verb paradigms occur with short vowel in the V_2 paradigm.

\begin{table}[H] 
\caption{Comparison of the stem alternations of the V2 \ipa{|-le-|}  with that of several simple verbs} \label{tab:le.dze.ret} \centering 
\begin{tabular}{lllll} 
\toprule 
&& \multicolumn{2}{c}{\textsc{intr}} & \multicolumn{1}{c}{\textsc{tr}} \\ 
&\ipa{|-le-|}&\dhatu{ʣe}{speak} & \dhatu{ret}{laugh} & \dhatu{set}{kill}\\ 
\midrule
\textsc{1s.n.pst} &  \ipa{X-{le}-ŋʌ} & \ipa{ʣe-ŋʌ} & \ipa{rêj-ŋʌ} \grise{}&\grise{} \\ 
\textsc{1di.n.pst} &  \ipa{X-{leʦ}-i} & \ipa{ʣe-ji} \grise{}& \ipa{reʦ-i} &\grise{} \\ 
\textsc{1s.pst} & \ipa{X-{les}-tʌ} & \ipa{ʣe-ŋʌtʌ} \grise{}&  \ipa{res-tʌ} &\grise{} \\ 
\hline 
\textsc{1s$\rightarrow$3n.pst} &  \ipa{X-{led}-u} &\grise{} & \grise{} & \ipa{sed-u} \\ 
\textsc{3s$\rightarrow$3n.pst} &    \ipa{X-{led}-ʉ} &\grise{} & \grise{} & \ipa{sēːd-u} \\ 
\textsc{3p$\rightarrow$3n.pst} &    \ipa{X-{let}-nu} &\grise{} & \grise{} & \ipa{sêːt-nu} \\ 
\bottomrule 
\end{tabular}
\end{table}

Not all V_2 follow the same alternations as \ipa{-le-}; four patterns are attested, as shown in Table (\ref{tab:le.khAt.pi}) with the V_2 \ipa{-kʰʌ- / -kʰɵʦ- / kʰʌt} `do X and go; do X completely' (§ \ref{sec:v2.khot}), \ipa{-pi- / -pid-} `do X and come'  (§ \ref{sec:v2.pi}) and \ipa{-sʌ- / -sɵ-} `finish'.

In classes I and II, the basic stem alternation is between open syllable stem (\ipa{-le-}, \ipa{-kʰʌ-}) and closed syllable stems (with a final dental obstruent).  The open syllable stems occur in all non past intransitive forms (except dual), past intransitive first plural, non-past transitive first and second plural, and all past transitive (except first and second dual and second plural). Classes I and II differ from each other in that in 123\textsc{sg}$\rightarrow$3 forms, the stem final dental is \ipa{-d-} in class I (\ipa{X-led-u}) and \ipa{-t-} in class II (\ipa{X-kʰʌt-u}). In addition, some class II  V_2 have \ipa{ʌ \tld{} ɵ \tld{} a} vowel alternation (on which § XXX).

Class III includes V_2 whose stem is open syllable (\dhatu{-pi-}{come (same level)} as in Table \ref{tab:le.khAt.pi} and § \ref{sec:v2.pi}) or closed syllable (\dhatu{-kʰoŋ-}{come up}, § \ref{sec:v2.khoN}). Unlike Class I and II V_2, the intransitive forms of Class III V_2 do not have an extra dental obstruent in the intransitive paradigm. In the transitive paradigm on the other hand, the extra dental obstruent has the same distribution and form as in Class I verbs.

\begin{table}[H] 
\caption{V2 stem alternation patterns} \label{tab:le.khAt.pi} \centering 
\begin{tabular}{lllll} 
\toprule 
&   I  &   II & III &  IV \\
\midrule
\textsc{1s.n.pst} &  \ipa{X-le-ŋʌ} & \ipa{X-kʰʌ-ŋʌ} &  \ipa{X-pi-ŋʌ}&  \ipa{X-sʌ-ŋʌ} \\
\textsc{1di.n.pst} &  \ipa{X-leʦ-i} & \ipa{X-kʰɵʦ-i} & \ipa{X-pi-ji} &  \ipa{X-sɵ-ji}   \\
%\textsc{1s.pst} & \ipa{X-les-tʌ} & \ipa{X-kʰɵs-tʌ} & \ipa{X-pi-ŋʌtʌ} &\ipa{X-sɵ-ŋʌtʌ}   \\
\hline 
\textsc{1s$\rightarrow$3n.pst} &  \ipa{X-led-u} &  \ipa{X-kʰʌt-u} &   \ipa{X-pid-u} & \ipa{X-sʌ-ŋʌ} \\
\textsc{3p$\rightarrow$3n.pst} &    \ipa{X-let-nu} &    \ipa{X-kʰat-nu} & \ipa{X-pit-nu} & \ipa{X-sɛ-nu} \\
\bottomrule 
\end{tabular}
\end{table}

In the case of V_2 transitive paradigms, the extra obstruent clearly originates from the applicative / causative \ipa{-t-} suffix. In classes I and III , forms like the \textsc{1s$\rightarrow$3n.pst} \ipa{X-led-u}  or  \ipa{X-pid-u} directly correspond to the applicative forms of open syllable verbs like \dhatu{pit}{bring} (\textsc{1s$\rightarrow$3n.pst}  \ipa{pid-u} `I bring it') from \dhatu{pi}{come (same level)}. Class II forms like \textsc{1s$\rightarrow$3n.pst}  \ipa{X-kʰʌt-u} with an \ipa{-t-} correspond to the applicative forms of verb roots in \ipa{-t-} such as \dhatu{kʰott}{take away} from \dhatu{kʰot}{go} (the \textsc{1s$\rightarrow$3n.pst} of verbs of \ipa{CVtt} verbs have two alternative forms in free variation, the analogical and regular form \ipa{kʰoɔtt-u} `I take it away' and another form with \ipa{-t-}, probably more conservative, \ipa{kʰot-u} `I take it away').

\subsection{List of V2}
There are 17 known V_2 in Khaling (see Table \ref{tab:khyal.v2}), most of which have cognates in Dumi. Among these V_2, seven have associated motion uses, which are presented in more detail in the following section. Other include aspectivizers, benefactives and a few verbs roots with barely grammaticalized meaning (such as  \ipa{-se- / -seʦ-} `kill by doing X' from \dhatu{set}{kill}).

\begin{table}[H]
\caption{Inventory of V2 in Khaling} \label{tab:khyal.v2} \centering 
\begin{tabular}{llllll}
\toprule 
Origin & Stems & Class & Meaning \\%& Example \\
\midrule
?&   \ipa{-le- / -leʦ- / -led-} & I &  Circumambulative   \\
\midrule
\dhatu{set}{kill} &   \ipa{-se- / -seʦ-} & I &  `kill by'   \\
\midrule
\dhatu{mit}{die} &   \ipa{-me- / -meʦ-} & I &  Detrimental   \\
\midrule
\dhatu{tɛn}{fall} / &  \ipa{-tɛ- / -tɛʦ- / -tɛnd-} & I &  Subsequent cislocative   \\
\dhatu{tɛnt}{drop} &\\
\midrule
 ? &  \ipa{-pɛ- / -pɛʦ- / -pɛd-} & I &  Prior translocative \\
 \midrule
 ?&   \ipa{-de- / -deʦ- / -det-} & I &  Progressive   \\
 \midrule 
\dhatu{sa}{pass} / &  \ipa{-sʌ- / -sɵʦ- / -sʌt-} & II &  Benefactive \\
\midrule
\dhatu{kʰot}{go} / &  \ipa{-kʰʌ- / -kʰɵʦ- / -kʰʌt-} & II &  Subsequent translocative / \\
\dhatu{kʰott}{take away} & &&Completed action & \\
\midrule
\dhatu{pi}{come (same level)} / &  \ipa{-pi- / -pi- / -pid-} & III &  cislocative \\
\dhatu{pit}{bring (same level)} & &&  & \\
\midrule
\dhatu{ɦo}{come} / &  \ipa{-ɦɵ- / -ɦɵ- / -ɦod-} & III &  Subsequent cislocative \\
\dhatu{ɦot}{bring} & &&  & \\
\midrule
\dhatu{kʰoŋ}{come (up)} / &  \ipa{-kʰoŋ- / -kʰɵŋ- / -kʰond-} & III &  Subsequent cislocative \\
\dhatu{kʰoŋt}{bring (up)} & &&  & \\
%\midrule
%\dhatu{je}{come (down)} / &  \ipa{-je- / -je- / -jed-} & III &  Subsequent Cislocative \\
%\dhatu{jet}{bring (down)} & &&  & \\
\midrule
 ? &  \ipa{-tʰer- / -tʰer- / -tʰerd-} & III &  Habitual   \\
\midrule
 ? &   \ipa{-sʌ- / -sɵ- / -sʌ-} & IV &  Terminative   \\
 \midrule
 \dhatu{ʣa}{eat} / &  \ipa{-ʣʌ- / -ʣɵ- / -ʣʌ-} & IV &  Continuative   \\
 \midrule
  \dhatu{ta}{put} / &  \ipa{-tʌ- / -tɵ- / -tʌ-} & IV &  Benefactive, reliquitive   \\
  \midrule
\dhatu{dok}{be enough} &   \ipa{-do- / -dɵk-} & IV &  Lexicalized (in \dhatu{ʔip-dok}{fall asleep})  \\
\bottomrule 
\end{tabular}
\end{table}


\section{Associated motion V2}

\subsection{Circumambulative \dhatu{-le(t)-}{go around doing X}} \label{sec:v2.le}
The Class I V_2 \dhatu{-le(t)-}{go around doing X}describes a concurrent motion event distributed over different places. The action of the main verb can occurs either continuously, or iteratively as illustrated by (\ref{ex:thile}) and (\ref{ex:ngonlenya}). 

\begin{exe}
\ex \label{ex:thile}
 \gll  \ipa{tsi-ʔɛ} \ipa{seç-pɛ} \ipa{ɦʌs} \ipa{kʰɛbi} \ipa{mɛː} \ipa{tʰi-le} \\
 alcohol-\textsc{erg} kill-\textsc{nmlz}:S/A man where there stumble-\textsc{go.around.doing}:\textsc{n.pst.3sg} \\
 \glt `The drunk person falls around everywhere.' 
\end{exe}

\begin{exe}
\ex \label{ex:ngonlenya}
 \gll \ipa{lɛsbɛ-ʦɵ} \ipa{ŋô-n-le-nɛ} \ipa{mʌ-dʰēr-ki} \\
 boy-child cry-\textsc{inf}-\textsc{go.around.doing}-\textsc{inf} \textsc{neg}:\textsc{inv}-be.suitable-\textsc{1pi} \\
\glt `It is not suitable for boys to go around crying.' 
\end{exe}

With the motion verb \dhatu{kʰot}{go}, it forms a lexicalized bipartite verb \ipa{kʰot-le} meaning `be already gone' where it has a not-AM value. Apart from this isolated example, the V_2 \ipa{-le-} always implies a distinct motion distinct event. Its Dumi cognate, the \ipa{-li- / -lɨt-} `frolicsome aspectivizer' (\citealt[209-210]{driem93dumi}), has a similar function.

\subsection{Subsequent translocative \dhatu{-kʰo(t)-}{do X and go}} \label{sec:v2.khot}
The Class II V_2 \ipa{-kʰot-} has two distinct but related meanings. Its first use does not belong to AM: it expresses that the action is completed to the extent that a related participant completely disappears. That participant can be the S (the flea in \ref{ex:oapkhatya}), the O (the tree in \ref{ex:bhoapkhatyu}) or the A (in \ref{ex:wetkhatya}). With verbs of motion such as \dhatu{ʔopt}{jump} or \dhatu{wett}{cross over} as in (\ref{ex:oapkhatya}) and (\ref{ex:wetkhatya}), the implication is that as a result of this motion, the S/A subject leaves and is not visible any more. With non-motion verbs such as \dhatu{bʰopt}{cover with an extra layer}  as in (\ref{ex:bhoapkhatyu}), or \dhatu{tuŋ}{drink} as (\ref{ex:tyungkhatya}), the V_2  \ipa{-kʰot-} implies that the object has been hidden or used up. 

This V_2 appears in many lexicalized and non-compositional bipartite verbs, in particular with V_1 that are not attested independently, for instance \dhatu{ʔak-khot-}{be astonished, be displeased} (a deponent verb whose \textsc{3$\rightarrow$1sg.n.pst} form is \ipa{ʔi-ʔâ-ŋ-kʰʌ-ŋʌ} \textsc{inv/2}-be.displeased-\textsc{1sg}-\textsc{away}-\textsc{1sg} `I am displeased'). 

\begin{exe}
\ex \label{ex:oapkhatya}
 \gll   \ipa{ʦʌkʌpɛ} \ipa{ʔoɔp-kʰʌ-tɛ}. \\
 flea jump-\textsc{away}-\textsc{2/3:pst} \\
 \glt `The flea jumped away.'
\end{exe}

\begin{exe}
\ex \label{ex:wetkhatya}
 \gll  \ipa{ʔɛ̂n} \ipa{ni} \ipa{ʔʌ̄m-ʔɛ} \ipa{ʔudʰʌ̂m=ŋʌ} \ipa{wet-kʰʌ-tɛ}. \\
 now \textsc{top} \textsc{3sg}-\textsc{erg} ridge=\textsc{emph} cross-\textsc{away}-\textsc{pst} \\
\glt `He now crossed over the ridge.'
\end{exe}

\begin{exe}
\ex \label{ex:bhoapkhatyu}
 \gll   \ipa{kʌrela-ʔɛ} \ipa{sʌ̄ŋ} \ipa{bʰoɔp-kʰʌt-ʉ}. \\
vine-\textsc{erg} tree cover-\textsc{away:tr}-\textsc{3sg.A.n.pst} \\
\glt `The vine grows over the tree (it's already invisible, but the vine is still growing)'
\end{exe}

The second meaning of the V_2 \ipa{-kʰot-} is less commonly used that the previous one; it designates subsequent translocative motion `do X and leave/go'. Examples with the AM meaning generally allow an non-AM aspectual interpretation as in (\ref{ex:tyungkhatya}) with the verb \dhatu{tuŋ}{drink} and (\ref{ex:plepkhatya}) with \dhatu{plept}{hold under arm}.

\begin{exe}
\ex \label{ex:tyungkhatya}
 \gll   \ipa{tʉ̂ŋ-kʰʌ-tɛ} \\
 drink-\textsc{away/do\&go}-\textsc{2/3:pst} \\
 \glt `He drank it up' or `He drank it and left.'
 \end{exe}

 \begin{exe}
\ex \label{ex:plepkhatya}
 \gll  \ipa{ʔuŋʌ} \ipa{ʣʰola} \ipa{plep-kʰʌt-ʌ}. \\
 \textsc{1sg:erg} bag hold.under.arm-\textsc{do\&go:tr}-\textsc{1sg}:\textsc{pst} \\
 \glt `I held the bag under my arm (and left).'
 \end{exe}  

In Dumi, \citet[210-212]{driem93dumi} mentions a cognate `itive aspectivizer' V_2 also transparently from the verb `go', but no examples of AM use are given.
 
\subsection{Prior translocative \dhatu{-pɛ-}{go and do}} \label{sec:v2.pE}
The Class I V_2 \dhatu{-pɛ(t)-}{go and do X} is a prior translocative AM marker, which can be used with both transitive (as in \ref{ex:7yopyatya} and \ref{ex:ithyopyatya}) and intransitive (\ref{ex:tserpyasta}) verbs -- the motion event strictly concerns the S/A subject.  

 \begin{exe}
\ex \label{ex:7yopyatya}
 \gll  \ipa{sirise-ʔɛ} \ipa{ʣɛ} \ipa{kɵ-pɛ} \ipa{bʰîr} \ipa{ʔɵp-pɛ-tɛ} \\
 p.n.-\textsc{erg} grain eat-\textsc{nmlz}:S/A  deer shoot-\textsc{go\&do}-2/\textsc{3.pst} \\
 \glt  `Sirise went and shot the deer eating the grains.'
 \end{exe}  
 
  \begin{exe}
\ex \label{ex:ithyopyatya}
 \gll   \ipa{kʰɛbi} \ipa{go} \ipa{ku} \ipa{ʔi-tʰɵ-pɛ-tɛ}  \\
 where foc water 2/\textsc{inv}-see-\textsc{go\&do}-2/3.\textsc{pst} \\
\glt `Where did you go and find this water?'
 \end{exe}  
 
  \begin{exe}
\ex \label{ex:tserpyasta}
 \gll  \ipa{tsêr-pɛs-tʌ} \\
 urinate-\textsc{go\&do}-\textsc{1sg.pst} \\
 \glt `I went and urinated.'
  \end{exe}  
 
The AM marker \dhatu{-pɛ(t)-}{go and do X} contrasts with the more commonly used purposive motion verb construction with the verb \dhatu{kʰot}{go}, as illustrated by examples (\ref{ex:ryappyadyu}) and (\ref{ex:ryapbi}), both involving the transitive verb \dhatu{rɛp}{hit with a stick}. 

 
 \begin{exe}
\ex \label{ex:ryappyadyu}
 \gll   \ipa{ʔʌ̄m-ʔɛ} \ipa{bʰrɛ̂m} \ipa{rɛp-pɛd-ʉ} \\
 \textsc{3sg}-\textsc{erg} buckwheat beat-\textsc{go\&do:tr}-\textsc{3sg.A.n.pst} \\
\glt  `He goes and beats the buckwheat (and eventually returns)'
  \end{exe}  
  
\begin{exe}
\ex \label{ex:ryapbi}
 \gll   \ipa{ʔʌ̄m} \ipa{bʰrɛ̂m} \ipa{rɛp-bi} \ipa{kʰoɔ̂j}   \\
 \textsc{3sg} buckwheat beat-\textsc{loc} go:\textsc{3sg.n.pst} \\
 \glt `He goes to beat the buckwheat.'
    \end{exe}  
   
The differences between the two sentences are not limited to morphology, but also concern syntax and semantics. In  (\ref{ex:ryapbi}), the common subject of the intransitive motion verb \dhatu{kʰot}{go} and the of the transitive verb \dhatu{rɛp}{hit with a stick} in the purposive clause receives absolutive case, the case selected by the motion verb. By contrast, in (\ref{ex:ryappyadyu}), the common subject receives ergative case following the transitive V_1 of the bipartite verb \ipa{rɛp-pɛ-} `go and beat'.

The V_2 \ipa{-pɛ(t)-} can be used with motion verbs such as \dhatu{ɦo}{come}; in this usage it expresses not AM, but accomplishment Aktionsart: the bipartite verb \ipa{ɦo-pɛ(t)-} means `reach, arrive' (see \ref{ex:yostyastya} below).
    
\subsection{Subsequent cislocative \dhatu{-ɦo-}{do and come}} \label{sec:v2.ho}
The Class III V_2 \dhatu{-ɦo(t)-}{do X and come} is grammaticalized from \dhatu{ɦo}{come}, one of the four cislocative motion verbs in Japhug, alongside \dhatu{kʰoŋ}{come up}, \dhatu{pi}{come (same level)} and \dhatu{je}{come down}. As is shown in the following sections, these verbs (except for \dhatu{je}{come down}) also have AM V_2 counterparts in Japhug. The verb \dhatu{ɦo}{come} and its corresponding V_2  \dhatu{-ɦo(t)-}{do X and come} differ from these three verbs in that it does not specifies vertical dimension of the cislocative motion.

Use of \dhatu{-ɦo(t)-}{do X and come} as an AM marker is less common than the other cislocative  \dhatu{-kʰoŋ-}{do X and come up} and \dhatu{-pi(t)-}{do X and come (same level)}. Example (\ref{ex:maseihyonya}) show such a usage.

\begin{exe}
\ex \label{ex:maseihyonya}
 \gll  <pʰilim> \ipa{mʌ-sēj-ɦɵ-nɛ} \ipa{ni} \ipa{nûː} \ipa{mu-gʰāŋ} \\
 film \textsc{neg}-see-\textsc{do\&come} \textsc{top} mind \textsc{neg}-agree:\textsc{3sg.n.pst} \\
\glt `He is unhappy (to come home) without having seen a film.'
\end{exe}

The V_2 \ipa{-ɦo(t)-} also occurs in lexicalized bipartite verbs such as \ipa{ʦɛm-ɦo-} `forget' (whose V_1 is \dhatu{ʦɛm}{lose}).

\subsection{Subsequent cislocative \dhatu{-kʰoŋ-}{do and come up}} \label{sec:v2.khoN}
The Class III V_2 \dhatu{-kʰoŋ-}{do X and come up}, which clearly originates from \dhatu{kʰoŋ}{come up} expresses subsequent cislocative motion, as is shown by examples (\ref{ex:nyaaskhyongata}) and (\ref{ex:hungkhondu}) respectively with an intransitive, and a transitive V_1. Like the motion verb  \dhatu{kʰoŋ}{come up}, the V_2  \dhatu{-kʰoŋ-}{do X and come up} implies a vertical motion from a lower place to a higher one.

Note that in (\ref{ex:hungkhondu}) the verb form \ipa{ɦûŋ-kʰond-u}, although the V_2 has a transitive form (it resembles the \textsc{1sg$\rightarrow$3.n.pst} \ipa{kʰōndu}{I bring it up} from the applicative/causative \dhatu{kʰoŋt}{bring up} of \dhatu{kʰoŋ}{come up}), there is no manipulative meaning in the bipartite verbs, which means `wait and come', not `wait and bring'. The morphological transitivity harmony between the V_1 and the V_2 thus does not affect the semantics of the latter.

\begin{exe}
\ex \label{ex:nyaaskhyongata}
 \gll  \ipa{sʉ̂rnɛj-bi} \ipa{ʔûŋ} \ipa{ŋɛ̄ːs-kʰɵŋ-ʌtʌ} \\
 p.n.-\textsc{loc} \textsc{1sg} sit-\textsc{do\&come.up}-\textsc{1sg.S/P.pst} \\
\glt  `I sat at Surnei and then came up.'
\end{exe}

\begin{exe}  
\ex \label{ex:hungkhondu}
 \gll  \ipa{ʔīn} \ipa{kʰɵs-tʰer-e} \ipa{uŋʌ} \ipa{ʔʌ-dʌrʌm} \ipa{ɦûŋ-kʰond-u} \\
 2sg go-\textsc{habit}-\textsc{imp}:\textsc{2sg} \textsc{1sg}.\textsc{erg} \textsc{1sg}.\textsc{poss}-friend wait-\textsc{do\&come.up}-\textsc{1sg.A.n.pst} \\
\glt `You keep going, I will wait for my friends and come up then.'
\end{exe}

Example (\ref{ex:gyokkhongpya}) is an interesting case of AM applied to an existential verb. It is difficult to render into English without backgrounding the meaning of the V_1; the participial relative \ipa{lɛ̂l-kʌ} \ipa{gɵ-k-kʰoŋ-pɛ} literally means `which exists after having come (up) from before'.

\begin{exe}
\ex \label{ex:gyokkhongpya}
 \gll <asik> \ipa{ʔi-mʌç-kî-m} \ipa{ʦʌ̄j} \ipa{ʔik-po} \ipa{mɛ} \ipa{lɛ̂l-kʌ} \ipa{gɵ-k-kʰoŋ-pɛ} \\
blessing \textsc{inv}-do-\textsc{1pi}-\textsc{nmlz} \textsc{top} \textsc{1pi}-\textsc{gen} \textsc{dem} before-\textsc{abl} exist:\textsc{inan}-\textsc{nmlz}-\textsc{do\&come.up}-\textsc{nmlz}:S/A \\
\glt   `He blesses us, and this (tradition) of ours comes up from earlier times.'
\end{exe}

 

\subsection{Cislocative \dhatu{-pi(d)-}{come (same level)}} \label{sec:v2.pi}
The type III V_2 \dhatu{-pi(d)-}{come (same level)} expresses a cislocative motion on the same vertical level, which can be either prior, concurrent or subsequent, though the latter isthe most common, as shown by example (\ref{ex:ngyaipinya}). With transitive verbs, the V_2 \dhatu{-pi(d)-}{come (same level)} can be interpreted as a subsequent manipulative AM marker `do X and bring', as in  (\ref{ex:lompidu}).

\begin{exe}
\ex \label{ex:ngyaipinya}
 \gll \ipa{jakâ-m}	\ipa{mɛ-jô-ŋ}	\ipa{ʦʌutara-bi}	\ipa{ŋɛ̄j-pi-nɛ}	\ipa{mʌtt-ʉ}	\ipa{ɦolʌ} \\
 there-\textsc{rel} \textsc{dem-loc.same.level}-\textsc{emph} resting.place-\textsc{lo} sit-\textsc{do\&come}-\textsc{inf} have.to-\textsc{3sg$\rightarrow$3} maybe \\
 \glt `We should go sit on that resting platform, and then come back.'
 \end{exe}
 
\begin{exe}
\ex \label{ex:lompidu}
 \gll \ipa{tsʰɛ̂l}	\ipa{bɛ̄j}	\ipa{mʌ̂n-nɛ}	\ipa{taktibuŋ-po}	\ipa{pūŋ}	\ipa{tʰūːnɛm-bi}	\ipa{lom-pid-u} \\
 guest wear \textsc{caus}-\textsc{inf} rhododendron-\textsc{gen} flower forest-loc look.for-\textsc{bring.same.level}-\textsc{1sg$\rightarrow$3:n.pst} \\
\glt `I search for a rhododendron flower in the forest and bring it for my guest to wear.'
\end{exe}

\subsection{Subsequent cislocative \dhatu{-tɛ(nt)-}{do and bring down}} \label{sec:v2.tya}
The Class III V_2 \dhatu{-tɛ(nt)-}{come down} from the verbs \dhatu{tɛn}{fall} and \dhatu{tɛnt}{drop, sow} is mainly used in combination with motion verbs to express a downward motion path as in (\ref{ex:yostyastya}) and (\ref{ex:lyaastyastyanu}) with the verbs \dhatu{ʔot}{return} and \dhatu{lɛn}{come out} which are neutral as to the verticality dimension.

\begin{exe}
\ex \label{ex:yostyastya}
 \gll \ipa{mʌnʌ}	\ipa{mɛ}	\ipa{mi}	\ipa{mu-dʰʌ̂m-wɛ}	\ipa{kekuwa}	\ipa{mɛ-tʉ}	\ipa{pɛri-tʉ}	\ipa{ɦɵ̂-n-pɛ-nɛ=ŋʌ}	\ipa{mu-ʦɛ̂ːp-wɛ}	\ipa{mʌnʌ}	\ipa{dʰɛpɛ̂ŋ}	\ipa{ʔɵs-tɛs-tɛ}	 \\
 then \textsc{dem} fire \textsc{neg}-blow-\textsc{2/3.sg:pst:neg} kite \textsc{dem}-\textsc{loc}:\textsc{up} heaven-\textsc{loc}:\textsc{up} come-\textsc{inf}-\textsc{do\&go}-\textsc{inf}=\textsc{emph} \textsc{neg}-be.able-\textsc{2/3sg:pst} then immediately go.back-\textsc{down}-2/3.\textsc{sg}:\textsc{pst} \\
   \glt `She did not blow that fire (strongly enough) and the kite bird was unable to reach the heavens and it returned down immediately.'
 \end{exe}
 
 \begin{exe}
\ex \label{ex:lyaastyastyanu}
 \gll  \ipa{mɛ}	\ipa{ʔu-dʰêm-bi}	\ipa{lɛːs-tɛs-tɛ-nu}	\ipa{ʔe}  \\
 \textsc{dem} \textsc{3sg.poss}-flat.land-\textsc{loc} go.out-\textsc{down}-2/3:\textsc{pst}-\textsc{pl} \textsc{hearsay} \\
 \glt `They came out down from that flat place.'
  \end{exe}
  
When used with a transitive (or labile) non-motion verbs, \dhatu{-tɛ(nt)-}{down} can havets a manipulative AM meaning `bring down' as in (\ref{ex:khiptyanta}), but the non-manipulative meaning `come down' is also attested as in (\ref{ex:ryaptyandyu1}) and (\ref{ex:ryaptyandyu2}).
 
\begin{exe}
\ex \label{ex:khiptyanta}
 \gll \ipa{tukkâ-m}	\ipa{ʦulo-tʉ}	\ipa{kʰir}	\ipa{kʰip-tɛn-tʌ} \\
  up.there.\textsc{distal}-\textsc{rel} hearth-\textsc{loc}:\textsc{up} porridge cook-\textsc{bring.down}-\textsc{1sg}.\textsc{pst} \\
  \glt `I cooked porridge on the stove up there (on the roof) and brought it down.'
 \end{exe}
 
The temporal relationship of the motion event expressed by  \dhatu{-tɛ(nt)-}{(bring) down} is not fixed, and can be interpreted as prior or subsequent to the action of the V_1 depending on pragmatics. From instance, in example   (\ref{ex:ryaptyandyu1}) the motion event follows he action of the V_1 as buckwheat fields are at a higher altitude than the house and such a sentence makes sense pragmatically, whereas in   (\ref{ex:ryaptyandyu2}),  since rice fields have lower altitude than the household, interpreting it as subsequent motion would not make sense, and it is therefore understood as a prior motion (followed by return home after the action of the V_1 has been completed).

 \begin{exe}
\ex \label{ex:ryaptyandyu1}
  \gll   \ipa{siriseʔ-ɛ} \ipa{bʰrɛ̂m}  \ipa{rɛp-tɛnd-ʉ} \\
   p.n.-\textsc{erg} buckwheat beat-\textsc{come.down}-\textsc{3sg$\rightarrow$3.n.pst} \\
\glt `Sirise beats the buckwheat and comes down.'
 \end{exe}

 \begin{exe}
\ex \label{ex:ryaptyandyu2}
 \gll   \ipa{dzirise-ʔɛ} \ipa{rɵ̂:} \ipa{rɛp-tɛnd-ʉ} \\
 p.n.-\textsc{erg} rice beat-\textsc{come.down}-\textsc{3sg$\rightarrow$3.n.pst} \\
\glt `Jirise comes down, beats the rice (and returns).'
 \end{exe}
 
Some speakers accept an additional AM V_2 based on \dhatu{je}{come down} meaning `do X and come down', but since most speakers don't accept it and it is not found in texts, it may have been built by analogy to  \dhatu{-kʰoŋ-}{do X and come up} and \dhatu{-pi-}{come (same level)} during elicitation sessions and is therefore not included in this work.
 
\section{Morphosyntactic parameters of AM in Khaling}

\subsection{Non-AM meanings} \label{sec:non.am}
\subsection{Temporal relation} \label{sec:temporal}
\subsection{Deixis} \label{sec:deixis}
\subsection{Path (vertical dimension)} \label{sec:vertical}
\subsection{Argument of motion} \label{sec:argument}
\subsection{Mono- vs pluriactionality} \label{sec:pluriactionality}

Unlike other languages with AM such as Japhug (\citealt[202-203]{jacques13harmonization}) in Khaling negation does not necessarily have scope over both the motion event and the action of the V_1. In example (\ref{ex:mahunkhonya}) (compare with \ref{ex:hungkhondu} § \ref{sec:v2.khoN}), only the action of the V_1 is negated, not the motion event.  


\begin{exe}
\ex \label{ex:mahunkhonya}
 \gll
\ipa{ʔuŋʌ} \ipa{ʔʌ̄m} \ipa{mʌ-ɦû-n-kʰōː-nɛ-ʔɛ} \ipa{ʔu-nûː} \ipa{ŋes-tɛ} \\
\textsc{1sg}.\textsc{erg} \textsc{3sg} \textsc{neg}-wait-\textsc{inf}-\textsc{do\&come.up} \textsc{3sg}.\textsc{poss}-mind hurt-2/3:\textsc{pst} \\
\glt `I did not wait for him before going up and he was sad. (I was already gone up by the time he arrived at the waiting place)'
\end{exe}

\section{Conclusion}

\bibliographystyle{unified}
\bibliography{bibliogj}
\end{document}