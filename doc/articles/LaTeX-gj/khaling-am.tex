\documentclass[oneside,a4paper,11pt]{article} 
\usepackage{polyglossia}
%\usepackage{fontspec}
\usepackage{natbib}
\usepackage{booktabs}
\usepackage{xltxtra} 
\usepackage{longtable}
 \usepackage{geometry}
\usepackage[table]{xcolor}
\usepackage{color}
\usepackage{multirow}
\usepackage{gb4e} 
\usepackage{multicol}
\usepackage{graphicx}
\usepackage{float}
\usepackage{hyperref} 
\hypersetup{colorlinks=true,linkcolor=blue,citecolor=blue}
\usepackage{amssymb} 
\usepackage{memhfixc}
\usepackage{lscape}
%\usepackage[footnotesize,bf]{caption}
\usepackage{lineno}

%%%%%%%%%%%%%%%%%%%%%%%%%%%%%%%
%\setmainfont[Mapping=tex-text,Numbers=OldStyle,Ligatures=Common]{Charis SIL} 
\newfontfamily\phon[Mapping=tex-text,Ligatures=Common,Scale=MatchLowercase]{Charis SIL} 
 \newcommand{\ipa}[1]{{\phon#1}} 
\newcommand{\ipab}[1]{{\phon #1}}
\newcommand{\ipapl}[1]{{\phondroit #1}} 
\newcommand{\captionft}[1]{{\captionfont #1}} 
\newfontfamily\cn[Mapping=tex-text,Scale=MatchUppercase]{MingLiU}%pour le chinois
\newcommand{\zh}[1]{{\cn #1}}
\bibpunct[: ]{(}{)}{,}{a}{}{,}
 
\newcommand{\dhatu}[2]{|\ipa{#1}| `#2'}
\newcommand{\grise}[1]{\cellcolor{lightgray}\textbf{#1}} 
 \newcommand{\tld}{\textasciitilde}

\begin{document}



\title{Associated motion in Khaling } 
\author{Guillaume Jacques\\Aimée Lahaussois}
\maketitle
\linenumbers
\section{Introduction}
\section{Bipartite verb paradigms in Khaling}
Associated motion is expressed in Khaling by the second verbal stem (henceforth V_2) of bipartite verb constructions. These constructions are well attested in all Kiranti languages (see in particular \citealt[118-132]{driem87}, \citealt[199-214]{driem93dumi},  \citealt[137-194]{rutgers98yamphu}, \citealt{bickel07chintang}, \citealt[170-172]{doornenbal09} and \citealt[283-328]{schackow15yakkha}). The description of bipartite verb paradigms is thus a prerequisite to any discussion of AM in Khaling, and Kiranti languages in general.

Previous work on Khaling verbal morphology (\citealt{jacques12khaling, jacques16si}) has focused on simple verbs, and though Khaling bipartite verbs have been discussed in a comparative context (\citealt{jacques18bipartite}), the complete paradigms have never been presented. The focus of the present paper being AM, the intricacies of verbal stem alternations cannot be dealt here in a systematic way, and only the minimal necessary information will be presented.

\subsection{V1 stem alternations}

Table \ref{tab:kurledu} illustrates the paradigm of the circumambulative V_2 \ipa{-le-} (see \ref{sec:v2.le}) with two verbs in closed syllable roots,\footnote{The abstract roots are indicated between bars |, following the conventions in \citet{jacques12khaling}.} the intransitive verb \dhatu{ŋok}{cry} (`go around crying') and the transitive \dhatu{kur}{carry on the back} (`carry around'). Only part of the paradigm are given: the dual and plural exclusive, the second and third person dual,  and for transitive verbs the inverse configurations (3$\rightarrow$1/2) and the local forms (1$\rightarrow$2, 2$\rightarrow$1) are not given here, as their stem forms are derivable from those presented in this table (see \citealt{jacques12khaling} for a complete account).

%ʦ ʣ ɵ ː

\begin{table}[H]
\caption{Paradigm of bipartite verbs (with the circumambulative \ipa{-le-} as V2)} \label{tab:kurledu}  \centering
\begin{tabular}{lllllll}
\toprule
&\multicolumn{2}{c}{\textsc{n.pst}} &  \multicolumn{2}{c}{\textsc{pst}} \\
&Simple verb &  Bipartite verb &  Simple verb&  Bipartite verb \\
\midrule
\textsc{1s} &  \ipa{ŋôŋ-ŋʌ} & \ipa{ŋôŋ-\textbf{le}-ŋʌ} & \ipa{ŋɵk-ʌtʌ} & \ipa{ŋɵk-\textbf{les}-tʌ} \\
\textsc{1di} &  \ipa{ŋɵk-i} & \ipa{ŋɵk-\textbf{leʦ}-i} & \ipa{ŋɵk-iti} & \ipa{ŋɵk-\textbf{les}-ti} \\
\textsc{1pi} &  \ipa{ŋok-ki} & \ipa{ŋok-\textbf{le}-ki} & \ipa{ŋok-tiki} & \ipa{ŋok-\textbf{le}-ktiki} \\
\textsc{2s} &  \ipa{ʔi-ŋôː} & \ipa{ʔi-ŋôː-\textbf{le}} & \ipa{ʔi-ŋɵk-tɛ} & \ipa{ʔi-ŋɵk-\textbf{les}-tɛ} \\
\textsc{2p} &  \ipa{ʔi-ŋôː-ni} & \ipa{ʔi-ŋô-n-\textbf{le}-ni} & \ipa{ʔi-ŋɵk-tɛnu} & \ipa{ʔi-ŋɵk-\textbf{les}-tɛnu}  \\
\textsc{3s} &  \ipa{ŋôː} & \ipa{ŋôː-\textbf{le}} & \ipa{ŋɵk-tɛ} & \ipa{ŋɵk-\textbf{les}-tɛ} \\
\textsc{3p} &  \ipa{ŋôː-nu} & \ipa{ŋô-n-\textbf{le}-nu} & \ipa{ŋɵk-tɛnu} & \ipa{ŋɵk-\textbf{les}-tɛnu} \\
\midrule
\textsc{1s$\rightarrow$3} &  \ipa{kur-u} & \ipa{kûr-\textbf{led}-u} & \ipa{kur-utʌ} & \ipa{kûr-\textbf{le}-tʌ} \\
\textsc{1di$\rightarrow$3} &  \ipa{kʉr-i} & \ipa{kʉ̂r-\textbf{leʦ}-i} & \ipa{kʉr-iti} & \ipa{kʉ̂r-\textbf{les}-ti} \\
\textsc{1pi$\rightarrow$3} &  \ipa{kʌ̄r-ki} & \ipa{kʌ̄r-\textbf{le}-ki} & \ipa{kʌ̄r-tiki} & \ipa{kʌ̄r-\textbf{le}-ktiki} \\
\textsc{2s$\rightarrow$3} &  \ipa{ʔi-kʉ̄ːr-ʉ} & \ipa{ʔi-kʉ̂r-\textbf{led}-ʉ} & \ipa{ʔi-kʉ̂r-tɛ} & \ipa{ʔi-kʉ̂r-\textbf{le}-tɛ} \\
\textsc{2p$\rightarrow$3} &  \ipa{ʔi-kʌ̄r-ni} & \ipa{ʔi-kʌ̄r-\textbf{le}-ni} & \ipa{ʔi-kʉr-tɛnu} & \ipa{ʔi-kʉr-\textbf{les}-tɛnu} \\
\textsc{3s$\rightarrow$3} &  \ipa{kʉ̄ːr-ʉ} & \ipa{kʉ̂r-\textbf{led}-ʉ} & \ipa{kʉ̂r-tɛ} & \ipa{kʉ̂r-\textbf{le}-tɛ} \\
\textsc{3p$\rightarrow$3} &  \ipa{kʉ̂r-nu} & \ipa{kʉ̂r-\textbf{let}-nu} & \ipa{kʉ̂r-tɛnu} & \ipa{kʉ̂r-\textbf{le}-tɛnu} \\
\bottomrule
\end{tabular}
\end{table}
  
  
The form of the V_1 can be predicted from the corresponding simple verb by the four following rules:

\begin{enumerate}
\item If the verb stem contains a final consonant, and is followed by a suffix beginning with a  consonant (as in \ipa{kʌ̄r-ki} `we bring it'), the V_1 stem is identical to that of the simple verb (\ipa{kʌ̄r-\textbf{le}-ki} `we bring it around').
\item If the verb stem contains a final consonant, and is followed by a suffix beginning with a  vowel, the coda is resyllabified with that suffix. For instance, the form \ipa{kur-u} `I bring it' is syllabified as \ipa{ku/ru}. In such cases, if the final consonant is an obstruent (as in \ipa{ŋɵk-i} `you and I cry'), the V_1 stem is identical to that of the simple verb (\ipa{ŋɵk-\textbf{lets}-i} `you and I go around crying'), but if it is a sonorant, a final tone results (as in \ipa{kur-u} $\rightarrow$   \ipa{kûr-\textbf{led}-u}).\footnote{Since the stress is rather on the V_2, the tonal contrast on the V_1 is difficult to perceive.} This reduction is presumably due to a sound law that applies elsewhere in Khaling, and constitutes one of the origins of the tonal contrast in this language (see \citealt{jacques16tonogenesis}).
\item If the verb stem ends in a vowel (\ipa{ʔi-ŋôː} `you cry'), the stem of the V_1 is identical to that of the simple verb if there are no indexation suffixes (\ipa{ʔi-ŋôː-\textbf{le}} `you go around crying'), but if an indexation suffix is present, its vowel is dropped, and the consonant of the suffix becomes the coda of the V_1 stem, with falling tone (\ipa{ʔi-ŋôː-ni}   $\rightarrow$ \ipa{ʔi-ŋô-n-\textbf{le}-ni}).
\item In the case of Ct stem verbs (not in Table \ref{tab:kurledu}), the \ipa{-t} is always removed. For instance, in the paradigm of the bipartite verb comprising \dhatu{gʰurt}{run with, drive} and the V_2 \ipa{-kʰoŋt-} `bring up' (see \ref{sec:v2.khoN}), the stem of \ipa{gʰʌ̄rd-u} `I run with it, I drive it' becomes \ipa{gʰʌ̄r-} in \ipa{gʰʌ̄r-kʰoŋd-u} `I bring it up running' (a form like $\dagger$\ipa{gʰʌ̄rtkʰoŋdu} would violate the phonotactic structure of the Khaling language).
\end{enumerate}

The above rules predict all the V_1 forms of bipartite verbs with all V_2, except the benefactive V_2 \ipa{-sʌ-}.\footnote{With this V_2, the strong stem (only found in the \textsc{1pi} in the paradigms in Table \ref{tab:kurledu}) is found in all forms other that the dual and \textsc{2p.pst$\rightarrow$3}. For instance, the bipartite verb combining \dhatu{kur}{carry} and the benefactive has the \textsc{1s$\rightarrow$3.n.pst} form \ipa{kʌ̄r-sʌt-u} `I carry it for him', with the strong stem \ipa{kʌ̄r-}, unlike the corresponding simplex verb  \ipa{kur-u}  I carry it'. This exception can be explained historically. Khaling has a \ipa{-t-} causative/applicative suffix, one of whose function is benefactive, but this suffix is not productive anymore (see \citealt{jacques15derivational.khaling}). Transitive verbs with this suffix follow a distinct conjugation (described in \citealt[1119-1122]{jacques12khaling}), with the strong stem  in all slots of the paradigms except dual and \textsc{2pl$\rightarrow$3.pst}. The verb \dhatu{kur}{carry} is among the few verbs possessing a benefactive \ipa{-t-} suffix form \dhatu{kurt}{carry for someone}. The V_1 stem form \ipa{kʌ̄r-} in \ipa{kʌ̄r-sʌt-u} `I carry it for him' is better analyzable as deriving from \ipa{kʌ̄rd-u} `I bring it for him' (the \textsc{1sg$\rightarrow$3} of \dhatu{kurt}{carry for someone}), with application of the rule 4 (loss of \ipa{-t/d-} in \ipa{-Ct-} stem verbs). This explanation is however only historical, as the same stem distribution is observed even for verbs that do not synchronically have a benefactive \ipa{-t-} suffix form.}

\subsection{V2 stem alternations}
Table \ref{tab:kurledu} illustrates the stem alternations of the circumambulative V_2 \ipa{-le-}, which presents three distinct stems \ipa{-le-}, \ipa{-leʦ-} and \ipa{-les-} in the intransitive paradigm, and two additional ones \ipa{-let-} and \ipa{-led-} in the transitive one.

No simple verb shows exactly the same alternation pattern; rather,  \ipa{-le-} combines stems from the paradigm of open syllable \ipa{-e-} verbs and close syllable \ipa{-et-} verbs, as shown in Table \ref{tab:le.dze.ret}. Forms with long vowels in the transitive verb paradigms occur with short vowel in the V_2 paradigm.

\begin{table}[H] 
\caption{Comparison of the stem alternations of the V2 \ipa{|-le-|}  with that of several simple verbs} \label{tab:le.dze.ret} \centering 
\begin{tabular}{lllll} 
\toprule 
&& \multicolumn{2}{c}{\textsc{intr}} & \multicolumn{1}{c}{\textsc{tr}} \\ 
&\ipa{|-le-|}&\dhatu{ʣe}{speak} & \dhatu{ret}{laugh} & \dhatu{set}{kill}\\ 
\midrule
\textsc{1s.n.pst} &  \ipa{X-{le}-ŋʌ} & \ipa{ʣe-ŋʌ} & \ipa{rêj-ŋʌ} \grise{}&\grise{} \\ 
\textsc{1di.n.pst} &  \ipa{X-{leʦ}-i} & \ipa{ʣe-ji} \grise{}& \ipa{reʦ-i} &\grise{} \\ 
\textsc{1s.pst} & \ipa{X-{les}-tʌ} & \ipa{ʣe-ŋʌtʌ} \grise{}&  \ipa{res-tʌ} &\grise{} \\ 
\hline 
\textsc{1s$\rightarrow$3n.pst} &  \ipa{X-{led}-u} &\grise{} & \grise{} & \ipa{sed-u} \\ 
\textsc{3s$\rightarrow$3n.pst} &    \ipa{X-{led}-ʉ} &\grise{} & \grise{} & \ipa{sēːd-u} \\ 
\textsc{3p$\rightarrow$3n.pst} &    \ipa{X-{let}-nu} &\grise{} & \grise{} & \ipa{sêːt-nu} \\ 
\bottomrule 
\end{tabular}
\end{table}

Not all V_2 follow the same alternations as \ipa{-le-}; four patterns are attested, as shown in Table (\ref{tab:le.khAt.pi}) with the V_2 \ipa{-kʰʌ- / -kʰɵʦ- / kʰʌt} `do X and go; do X completely' (§ \ref{sec:v2.khot}), \ipa{-pi- / -pid-} `do X and come'  (§ \ref{sec:v2.pi}) and \ipa{-sʌ- / -sɵ-} `finish'.

In classes I and II, the basic stem alternation is between open syllable stem (\ipa{-le-}, \ipa{-kʰʌ-}) and closed syllable stems (with a final dental obstruent).  The open syllable stems occur in all non past intransitive forms (except dual), past intransitive first plural, non-past transitive first and second plural, and all past transitive (except first and second dual and second plural). Classes I and II differ from each other in that in 123\textsc{sg}$\rightarrow$3 forms, the stem final dental is \ipa{-d-} in class I (\ipa{X-led-u}) and \ipa{-t-} in class II (\ipa{X-kʰʌt-u}). In addition, some class II  V_2 have \ipa{ʌ \tld{} ɵ \tld{} a} vowel alternation (on which § XXX).

Class III includes V_2 whose stem is open syllable (\dhatu{-pi-}{come (same level)} as in Table \ref{tab:le.khAt.pi} and § \ref{sec:v2.pi}) or closed syllable (\dhatu{-kʰoŋ-}{come up}, § \ref{sec:v2.khoN}). Unlike Class I and II V_2, the intransitive forms of Class III V_2 do not have an extra dental obstruent in the intransitive paradigm. In the transitive paradigm on the other hand, the extra dental obstruent has the same distribution and form as in Class I verbs.

\begin{table}[H] 
\caption{V2 stem alternation patterns} \label{tab:le.khAt.pi} \centering 
\begin{tabular}{lllll} 
\toprule 
&   I  &   II & III &  IV \\
\midrule
\textsc{1s.n.pst} &  \ipa{X-le-ŋʌ} & \ipa{X-kʰʌ-ŋʌ} &  \ipa{X-pi-ŋʌ}&  \ipa{X-sʌ-ŋʌ} \\
\textsc{1di.n.pst} &  \ipa{X-leʦ-i} & \ipa{X-kʰɵʦ-i} & \ipa{X-pi-ji} &  \ipa{X-sɵ-ji}   \\
%\textsc{1s.pst} & \ipa{X-les-tʌ} & \ipa{X-kʰɵs-tʌ} & \ipa{X-pi-ŋʌtʌ} &\ipa{X-sɵ-ŋʌtʌ}   \\
\hline 
\textsc{1s$\rightarrow$3n.pst} &  \ipa{X-led-u} &  \ipa{X-kʰʌt-u} &   \ipa{X-pid-u} & \ipa{X-sʌ-ŋʌ} \\
\textsc{3p$\rightarrow$3n.pst} &    \ipa{X-let-nu} &    \ipa{X-kʰat-nu} & \ipa{X-pit-nu} & \ipa{X-sɛ-nu} \\
\bottomrule 
\end{tabular}
\end{table}

In the case of V_2 transitive paradigms, the extra obstruent clearly originates from the applicative / causative \ipa{-t-} suffix. In classes I and III , forms like the \textsc{1s$\rightarrow$3n.pst} \ipa{X-led-u}  or  \ipa{X-pid-u} directly correspond to the applicative forms of open syllable verbs like \dhatu{pit}{bring} (\textsc{1s$\rightarrow$3n.pst}  \ipa{pid-u} `I bring it') from \dhatu{pi}{come (same level)}. Class II forms like \textsc{1s$\rightarrow$3n.pst}  \ipa{X-kʰʌt-u} with an \ipa{-t-} correspond to the applicative forms of verb roots in \ipa{-t-} such as \dhatu{kʰott}{take away} from \dhatu{kʰot}{go} (the \textsc{1s$\rightarrow$3n.pst} of verbs of \ipa{CVtt} verbs have two alternative forms in free variation, the analogical and regular form \ipa{kʰoɔtt-u} `I take it away' and another form with \ipa{-t-}, probably more conservative, \ipa{kʰot-u} `I take it away').

\subsection{List of V2}
There are 17 known V_2 in Khaling (see Table \ref{tab:khyal.v2}), must of which have cognates in Dumi. Among these V_2, seven have associated motion uses, which are presented in more detail in the following section. Other include aspectivizers, benefactives and a few verbs roots with barely grammaticalized meaning (such as  \ipa{-se- / -seʦ-} `kill by doing X' from \dhatu{set}{kill}).

\begin{table}[H]
\caption{Inventory of V2 in Khaling} \label{tab:khyal.v2} \centering 
\begin{tabular}{llllll}
\toprule 
Origin & Stems & Class & Meaning \\%& Example \\
\midrule
?&   \ipa{-le- / -leʦ- / -led-} & I &  Circumambulative   \\
\midrule
\dhatu{set}{kill} &   \ipa{-se- / -seʦ-} & I &  `kill by'   \\
\midrule
\dhatu{mit}{die} &   \ipa{-me- / -meʦ-} & I &  Detrimental   \\
\midrule
\dhatu{tɛn}{fall} / &  \ipa{-tɛ- / -tɛʦ- / -tɛnd-} & I &  Subsequent motion   \\
\dhatu{tɛnt}{drop} &\\
\midrule
 ? &  \ipa{-pɛ- / -pɛʦ- / -pɛd-} & I &  Prior translocative \\
 \midrule
 ?&   \ipa{-de- / -deʦ- / -det-} & I &  Progressive   \\
 \midrule 
\dhatu{sa}{pass} / &  \ipa{-sʌ- / -sɵʦ- / -sʌt-} & II &  Benefactive \\
\midrule
\dhatu{kʰot}{go} / &  \ipa{-kʰʌ- / -kʰɵʦ- / -kʰʌt-} & II &  Subsequent translocative / \\
\dhatu{kʰott}{take away} & &&Completed action & \\
\midrule
\dhatu{pi}{come (same level)} / &  \ipa{-pi- / -pi- / -pid-} & III &  Cislocative \\
\dhatu{pit}{bring (same level)} & &&  & \\
\midrule
\dhatu{ɦo}{come} / &  \ipa{-ɦɵ- / -ɦɵ- / -ɦod-} & III &  Subsequent cislocative \\
\dhatu{ɦot}{bring} & &&  & \\
\midrule
\dhatu{kʰoŋ}{come (up)} / &  \ipa{-kʰoŋ- / -kʰɵŋ- / -kʰond-} & III &  Subsequent cislocative \\
\dhatu{kʰoŋt}{bring (up)} & &&  & \\
\midrule
\dhatu{je}{come (down)} / &  \ipa{-je- / -je- / -jed-} & III &  Subsequent Cislocative \\
\dhatu{jet}{bring (down)} & &&  & \\
\midrule
 ? &  \ipa{-tʰer- / -tʰer- / -tʰerd-} & III &  Habitual   \\
\midrule
 ? &   \ipa{-sʌ- / -sɵ- / -sʌ-} & IV &  Terminative   \\
 \midrule
 \dhatu{ʣa}{eat} / &  \ipa{-ʣʌ- / -ʣɵ- / -ʣʌ-} & IV &  Continuative   \\
 \midrule
  \dhatu{ta}{put} / &  \ipa{-tʌ- / -tɵ- / -tʌ-} & IV &  Benefactive, reliquitive   \\
  \midrule
\dhatu{dok}{be enough} &   \ipa{-do- / -dɵk-} & IV &  Lexicalized (in \dhatu{ʔip-dok}{fall asleep})  \\
\bottomrule 
\end{tabular}
\end{table}


\section{Associated motion V2}

\subsection{Circumambulative \dhatu{-le(t)-}{X around}} \label{sec:v2.le}
The circumambulative \dhatu{-le(t)-}{X around} describes a concurrent motion event distributed over different places, as illustrated by (\ref{ex:thile}) and (\ref{ex:ngonlenya}).

\begin{exe}
\ex \label{ex:thile}
 \gll  \ipa{tsi-ʔɛ} \ipa{seç-pɛ} \ipa{ɦʌs} \ipa{kʰɛbi} \ipa{mɛː} \ipa{tʰi-le} \\
 alcohol-\textsc{erg} kill-\textsc{nmlz}:S/A man where there stumble-\textsc{go.around.doing}:\textsc{n.pst.3sg} \\
 \glt `The drunk person falls around everywhere.' 
\end{exe}

\begin{exe}
\ex \label{ex:ngonlenya}
 \gll \ipa{lɛsbɛ-ʦɵ} \ipa{ŋô-n-le-nɛ} \ipa{mʌ-dʰēr-ki} \\
 boy-child cry-\textsc{inf}-\textsc{go.around.doing}-\textsc{inf} \textsc{neg}:\textsc{inv}-be.suitable-\textsc{1pi} \\
\glt `It is not suitable for boys to go around crying.' 
\end{exe}



It is the only V_2 in Khaling to be a dedicated AM marker, without additional functions. Its Dumi cognate, the \ipa{-li- / -lɨt-} `frolicsome aspectivizer' (\citealt[209-210]{driem93dumi}), has a similar function.


\subsection{Subsequent translocative \dhatu{-kʰo(t)-}{go}} \label{sec:v2.khot}
\subsection{Prior translocative \dhatu{-pɛ-}{go}} \label{sec:v2.pE}
\subsection{Cislocative \dhatu{-pi-}{come (same level)}} \label{sec:v2.pi}
\subsection{Subsequent cislocative \dhatu{-ɦo-}{come}} \label{sec:v2.ho}
\subsection{Subsequent cislocative \dhatu{-kʰoŋ-}{come up}} \label{sec:v2.khoN}
\subsection{Subsequent cislocative \dhatu{-je-}{come down}} \label{sec:v2.je}

\section{Morphosyntactic parameters of AM in Khaling}
\subsection{Non-AM meanings}
\subsection{Temporal relation}
\subsection{Deixis}
\subsection{Path (vertical dimension)}
\subsection{Argument of motion}
\subsection{Mono- vs pluriactionality}

\section{Conclusion}

\bibliographystyle{unified}
\bibliography{bibliogj}
\end{document}