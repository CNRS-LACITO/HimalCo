\documentclass{article} 
\usepackage{fontspec}
\usepackage{natbib}
\usepackage{booktabs}
\usepackage{xltxtra} 
\usepackage{polyglossia} 
\usepackage{gb4e} 
\usepackage{multicol}
\usepackage{multirow}
\usepackage{graphicx}
\usepackage{float}
\usepackage{hyperref} 
\hypersetup{bookmarks=false,bookmarksnumbered,bookmarksopenlevel=5,bookmarksdepth=5,xetex,colorlinks=true,linkcolor=blue,citecolor=blue}
\usepackage[all]{hypcap}
\usepackage{memhfixc}
\usepackage{lscape}


%\setmainfont[Mapping=tex-text,Numbers=OldStyle,Ligatures=Common]{Charis SIL} 
\newfontfamily\phon[Mapping=tex-text,Ligatures=Common,Scale=MatchLowercase]{Charis SIL} 
\newcommand{\ipa}[1]{{\phon\textbf{#1}}} 
\newcommand{\grise}[1]{\cellcolor{lightgray}\textbf{#1}}
\newcommand{\bleu}[1]{\cellcolor{blue!20}\textbf{#1}}
\newfontfamily\cn[Mapping=tex-text,Ligatures=Common,Scale=MatchUppercase]{SimSun}%pour le chinois
\newcommand{\zh}[1]{{\cn #1}}
\newcommand{\Y}{\Checkmark} 
\newcommand{\N}{} 
\newcommand{\dhatu}[2]{|\ipa{#1}| `#2'}
\newcommand{\jpg}[2]{\ipa{#1} `#2'}  
\newcommand{\refb}[1]{(\ref{#1})}
\newcommand{\tld}{\textasciitilde{}}
\newfontfamily\mleccha[Mapping=tex-text,Ligatures=Common,Scale=MatchLowercase]{Galatia SIL}%pour le grec
\newcommand{\grec}[1]{{\mleccha #1}}

 \begin{document} 
\title{Sanskrit \ipa{hṛd-} `coeur'}
\author{Guillaume Jacques\\ CNRS-CRLAO-INALCO}
\maketitle

%\paragraph{§1}
\section*{Introduction}
Le nom du `coeur' a un indo-iranien un reflet phonétiquement irrégulier. Le sanskrit \ipa{hṛd-} n. et l'avestique \ipa{zərəd-} n. (et leurs variantes thématiques \ipa{hṛdaya-} n. et \ipa{zərədaiia-} n.) remontent en effet à un proto-indo-iranien \ipa{*j́ʰṛd-} au lieu du $\dagger$\ipa{cṛd-} que l'on attendrait étant donné que l'initiale de cet étymon provient de \ipa{*ḱ} dans toutes les autres langues de la famille (voir le dossier complet dans \citealt[417-423]{wodtko08NIL}). On attendrait en sanskrit $\dagger$\ipa{śṛd-}, une forme qui d'ailleurs est indirectement attestée par le nom \ipa{śraddhā-} f. `fidélité'  avec un thème 2 secondaire.

\section{Contamination} \label{sec:hira}
La seule explication claire présentée dans la littérature, d'après \citet[420]{wodtko08NIL}, est celle de \citet{szemerenyi70heart} qui propose une contamination avec la racine nominale \ipa{*ĝʰerd-} ou \ipa{*ĝʰerH-} qui se retrouve dans le grec \grec{χορδή} < \ipa{*ĝʰord-eh_2}`entrailles', le sanskrit \ipa{hirā́-} f. < \ipa{*ĝʰrH-eh_2} `veine' et le hittite \ipa{karāt-} `entrailles' *[\ipa{krʔāt}] (\citealt[208]{schrijver91laryngeals}, \citealt[446]{kloekhorst08edhil}).

Notons au passage que l'alternance entre \ipa{*d} et \ipa{*H} dans ces étymons pourrait s'expliquer si l'on accepte l'effet Kortlandt, selon lequel \ipa{*d} $\rightarrow$ \ipa{*h_1} précédant ou suivant  une autre consonne (\citealt{kortlandt83numerals, garnier14kortlandt}). Il faudrait ainsi postuler une alternance ancienne entre un degré zéro \ipa{*ĝʰṛd-} et  \ipa{*ĝʰerh_1-} (< \ipa{*ĝʰerd-} par effet Kortland), le grec ayant généralisé le premier et le sanskrit et le hittite le second. On pourrait ainsi affiner la reconstruction de l'étymon `veine' en indo-iranien comme \ipa{*ĝʰrh_1-eh_2} avec une laryngale 1.

L'hypothèse de Szemerényi est possible -- les cas de contamination irrégulière entre noms appartenant à la même classe sémantique sont attestées, mais elle n'est pas fermement supportée par la phraséologie; les réflexes de \ipa{*ĝʰer[d|H]-} et \ipa{k̂erd-} ne sont pas communément associés dans les textes, même tardifs.

\section{Assimilation syntagmatique}
Alternativement à la suggestion de Szemerényi, je propose  contamination syntagmatique due à une collocation entre le nom `cœur' et une racine verbale comprenant la séquence \ipa{*j́ʰr} en indo-iranien. 

Les assimilations syntagmatiques sont un phénomène communément cité des travaux sur l'analogie, qui mentionnent en particulier l'exemple français \ipa{au fur et à mesure} qui vient d'un plus ancient \ipa{au feur et à mesure} par assimilation de la dernière syllabe (\citealt[16]{andersen80morpho}, \citealt[64]{fertig13analogy}). %\citet[49-50]{wackernagel26syntax}

Un exemple plus proche du cas qui nous intéresse est offert par le latin, où le nom \ipa{lingua} `langue' avait en vieux latin une initial \ipa{d-}; il est généralment admis que le passage irrégulier à \ipa{l-} est dû à la contamination avec le verbe \ipa{lingere} `lécher' (\citealt[304]{hock91principles}). Il semble possible de suggérer que cette contamination a pu avoir lieu dans une collocation telle que \ipa{*cum dinguā lingēre} qui aurait donné \ipa{cum linguā lingēre} par assimilation syntagmatique, puis généralisée à l'ensemble de la langue.
 
On trouve en indo-iranien trois racines ayant pu apparaître dans des collocations faisant intervenir le nom `cœur', soit des verbes psychologiques comportant la séquence \ipa{*j́ʰr} en proto-indo-iranien. Il s'agit des racines ayant donné les verbes sanskrits \ipa{háryati} `se réjouir de', \ipa{hṛṣyati} `se réjouir, frissonner, se hérisser' et \ipa{hṛṇīte} `être fâché', qui seront évalués un à un.

\subsection{\ipa{háryati}} \label{sec:haryati}

Le sanskrit \ipa{háryati} `se réjouir de' est habituellement reconstruit comme \ipa{ĝʰér-ye-}, d'une racine \ipa{*ĝʰer-} `Gefallen finden, begehren' (\citealt[176]{liv}). Il n'existe pas de collocation combinant ce verbe et sa famille avec le nom \ipa{hṛd-}, mais l'exemple (\ref{ex:haryato}) illustre un vers avec double allitération \ipa{i ṣ / h r / h ṛ / i ṣ} faisant intervenir ces deux racines.\footnote{Remarquons au passage que l'interprétation de ce vers fort obscur doit tenir en compte de l'ambiguïté du verbe \ipa{íyakṣati}, désidératif à la fois de \ipa{yaj-} `sacrifier' et de \ipa{naś-} `obtenir' (\ipa{íyakṣati} < \ipa{*h_2i-h_2ṇḱ-se/o-} et de celle de la racine \ipa{iṣ-}, qui résulte de la confusion de \ipa{*h_2eys-} `suchen' avec \ipa{*h_1eysh_2-} `antreiben'. }

\begin{exe}
\ex \label{ex:haryato}
\glt \ipa{údīraya pitárā jārá ā́ bhágamíyakṣati haryató hṛttá iṣyati}
\gll \ipa{úd.īraya}	\ipa{pitárā}	\ipa{jārás}	\ipa{ā́}	\ipa{bhágam}	\ipa{íyakṣati}	\ipa{haryatás}	\ipa{hṛttás}	\ipa{iṣyati} \\
stimuler:\textsc{imp:sg} père:\textsc{acc:du} amant:\textsc{nom:sg} \textsc{prep} fortune:\textsc{acc:sg:n} vouloir.obtenir:\textsc{present:ind:3sg}  désiré:\textsc{nom:sg} coeur:\textsc{abl:sg} aspirer:\textsc{present:ind:3sg} \\
\glt Promeus tes deux parents (Ciel et Terre), comme l'amant (promeut) le bonheur! (Agni, ce dieu) désirable, cherche à obtenir (un bien pour l'Homme), il suscite (l'inspiration) du fond du cœur. (RV X,11.06; Traduction Renou, EVP XIV, 1965, p. 8)
\end{exe}

Il semblerait possible de proposer l'existence d'une collocation proto-indo-iranienne du type \ipa{*ćṛd-í j́ʰáryati} `il se réjouit dans son cœur', d'où la contamination de la consonne initiale \ipa{*j́ʰṛd-í j́ʰáryati} dans nom `cœur' qui aurait été généralisée ensuite partout.

Toutefois, \citet[104]{cheung07dictionary}, sur la base de formes chorasmiennes, reconstruit une vélaire en proto-iranien (\ipa{*gar} `rejoice'), qui devrait être projetée en indo-européen, et amener à réviser la reconstruction comme \ipa{*gʰer-}. Les formes iraniennes en \ipa{z} citées pour démontrer la palatale dans cette racine sont en effet douteuses; le nom avestique \ipa{zara-} est de sens incertain, et l'ossète \ipa{uzæld-} < \ipa{*awa-zarya} `to nurse, tend, to coax' est rapproché par  \citet[470]{cheung07dictionary} de racine \ipa{*zarH} `to bewail the deceased' (de \ipa{*ĝar-} `tönen, rufen', \citealt[161]{liv}). La reconstruction de Cheung invalide donc l'hypothèse proposée ci-dessus.

\subsection{\ipa{hṛṣyati}} \label{sec:hrsyati}
Le sanskrit \ipa{hṛṣyati} `se réjouir, frissonner, se hérisser' et l'avestique \ipa{zarš-} (\citealt[471]{cheung07dictionary}) proviennent de la racine \ipa{*ĝʰers-} `sich sträuben, erstarren' (\citealt[178]{liv}) à palatale. 

Le LIV (\citealt[198]{liv}), suivant \citet[808]{mayrhofer92ewa}  suggère la confusion avec une autre racine \ipa{*g(ʷ)ʰers-} `sich erfreuen', mais cette idée est contestée par \citet[471]{cheung07dictionary}, qui explique les formes sans palatales telles que le parthe \ipa{gš-}  `to rejoice' comme due à la dépalatalisation de \ipa{*ĝʰṛs-} au degré zéro (\ipa{*ĝʰṛs-} $\rightarrow$ \ipa{*gʰṛs-}).



%07,121.033c	durhṛdām apraharṣāya suhṛdāṃ harṣaṇāya ca
%03,023.038c	śaṅkhaṃ pradhmāpya harṣeṇa suhṛdaḥ paryaharṣayam
%05,007.030c	vṛtaḥ pratiyayau hṛṣṭaḥ suhṛdaḥ saṃpraharṣayan
%12,039.019c	suhṛdāṃ harṣajananaḥ puṇyaḥ śrutisukhāvahaḥ

Si aucune collocation entre le nom \ipa{hṛd-} et ses dérivés et la racine \ipa{hṛṣ-} n'est attestée en védique, on trouve des cas de ce type en sanskrit classique. Par exemple, (\ref{ex:hrdayaharsinis}) illustre le composé \ipa{mano-hṛdaya-harṣin-} `réjouissant les esprits et les cœurs', formé de \ipa{manas-} n. `esprit' de \ipa{hṛdaya-} n. `cœur' (un dérivé thématique de \ipa{hṛd-}, qui a cognat avestique exact \ipa{zərədaiia-}) et du nom \ipa{harṣin-} `réjouissant' dérivé de la racine \ipa{hṛṣ-}. 

\begin{exe}
\ex \label{ex:hrdayaharsinis}
\glt sādhu sādhviti sarvatra niśceruḥ stutisaṃhitāḥ
\glt vācaḥ puṇyāḥ kīrtimatāṃ manohṛdayaharṣiṇīḥ
\gll 
\ipa{sādhu} \ipa{sādhu} \ipa{iti} \ipa{sarvatra} \ipa{niścerur} \ipa{stuti-saṃhitās} 
\ipa{vācas} \ipa{puṇyās} \ipa{kīrtimatāṃ} \ipa{mano-hṛdaya-harṣiṇīs}\\
excellent excellent \textsc{quot} partout apparaître:\textsc{parfait:actif:3pl} louange-pourvu.de:\textsc{fem:pl:nom} parole:\textsc{fem:pl:nom}
pur:\textsc{fem:pl:nom} glorieux:\textsc{masc:pl:gen} esprit-cœur-réjouissant:\textsc{fem:pl:nom} \\ 
\glt `Excellent,--Excellent,--were the delightful words everywhere bruited about, coupled with eulogistic hymns about those famous men. And in consequence of this the minds and hearts of every one there were attracted towards them.' (Mbh, 06,041.102)
\end{exe}

Même si un composé de ce type ne saurait être ancien, il pourrait refléter la trace d'une collocation \ipa{*j́ʰṛdayam j́ʰaršayati} `réjouir le cœur' de date indo-iranienne, où l'assimilation syntagmatique aurait pu avoir eu lieu.

%02,023.006a	svasti vācyārhato viprān prayāhi bharatarṣabha
%02,023.006c	durhṛdām apraharṣāya suhṛdāṃ nandanāya ca
%Depart, bull among Bh´aratas, with the blessings of worthy priests,
%in pursuit of happiness for our friends and misery for our foes!

\subsection{\ipa{hṛṇīte}} \label{sec:hrnite}
La racine du sanskrit \ipa{hṛṇīte} `être fâché' (\ipa{*ĝʰe[r|l]H}, \citealt[178]{liv}), si elle manque de cognat externes, avait en proto-indo-iranien une palatale assurée par les formes iraniennes (voir \citealt[469]{cheung07dictionary} \ipa{*zarH} ‘to hurt, wound, anger’). 

On trouve en avestique à l'actif (présent de classe 9 comme le sanskrit) avec un sens causatif, comme dans la forme de 1pl optatif actif \ipa{zaranaēmā} `que nous (ne) provoquions (pas) ta colère' (Y 28.9),\footnote{On attendrait $\dagger$\ipa{zərənaēmā} au degré zéro, voir \citet[178]{liv}, note 3.  } ou préverbée en \ipa{ā-}. La différence entre avestique et sanskrit suggère ici une opposition de diathèse dans la proto-langue, l'actif ayant un sens causatif `provoquer la colère' et le médio-passif `se fâcher'. Parmi les formes iraniennes, le sens non causatif se retrouve en parthe \ipa{zr-} `to become angry'.


Je propose donc une collocation proto-indo-iranienne au médio-passif \ipa{*j́ʰṛd-í j́ʰr<n>H-tai} `il est fâché dans son cœur'. Elle proviendrait d'un plus ancien \ipa{*ćṛd-í j́ʰr<n>H-tai} par contamination de la séquence \ipa{*j́ʰr} de la racine verbale à un stade proto-indo-iranien, et aurait affecté tous les dérivés de la racine \ipa{*ćṛd} `cœur' excepté l'étymon `fidélité' mentionné ci-dessus, dont la parenté étymologique n'était plus ressentie à ce stade.

Si aucune collocation combinant les réflexes du nom \ipa{*k̂erd-} `cœur' à la racine de \ipa{hṛṇīte} `être fâché' n'est attestée en indo-iranien, il convient de noter que  le nom \ipa{*k̂erd-} `cœur' apparaît ailleurs dans la famille dans  la formation de collocation et même de verbes signifiant `être fâché'. Ainsi, le verbe hittite  \ipa{kardimiye/a-} `be angry' comporte un radical complexe dont le premier élément est universellement accepté comme apparenté à \ipa{ker / kard(i)-} `cœur', le réflexe de \ipa{*k̂erd-} (\citealt[456-7]{kloekhorst08edhil} suggère un verbe à incorporation, dont le deuxième élément serait \ipa{imiye/a} `mélanger'). Il ne s'agit d'ailleurs en aucun cas d'une spécificité indo-européenne : on trouve par exemple en lakhota (siouan) le verbe  \ipa{čhaŋzé} `be angry' dont le premier élément provient du nom \ipa{čhaŋté} `heart' (\citealt{ullrich08}).


\section*{Conclusion}



\bibliographystyle{unified}
\bibliography{bibliogj}
\end{document}
 