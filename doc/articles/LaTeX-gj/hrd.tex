\documentclass{article} 
\usepackage{fontspec}
\usepackage{natbib}
\usepackage{booktabs}
\usepackage{xltxtra} 
\usepackage{polyglossia} 
\setdefaultlanguage{french}
\usepackage{gb4e} 
\usepackage{multicol}
\usepackage{multirow}
\usepackage{graphicx}
\usepackage{float}
\usepackage{hyperref} 
\hypersetup{bookmarks=false,bookmarksnumbered,bookmarksopenlevel=5,bookmarksdepth=5,xetex,colorlinks=true,linkcolor=blue,citecolor=blue}
\usepackage[all]{hypcap}
\usepackage{memhfixc}
\usepackage{lscape}


%\setmainfont[Mapping=tex-text,Numbers=OldStyle,Ligatures=Common]{Charis SIL} 
\newfontfamily\phon[Mapping=tex-text,Ligatures=Common,Scale=MatchLowercase]{Charis SIL} 
\newcommand{\ipa}[1]{{\phon\textit{#1}}} 
\newcommand{\grise}[1]{\cellcolor{lightgray}\textbf{#1}}
\newcommand{\bleu}[1]{\cellcolor{blue!20}\textbf{#1}}
\newfontfamily\cn[Mapping=tex-text,Ligatures=Common,Scale=MatchUppercase]{SimSun}%pour le chinois
\newcommand{\zh}[1]{{\cn #1}}
\newcommand{\Y}{\Checkmark} 
\newcommand{\N}{} 
\newcommand{\dhatu}[2]{|\ipa{#1}| `#2'}
\newcommand{\jpg}[2]{\ipa{#1} `#2'}  
\newcommand{\refb}[1]{(\ref{#1})}
\newcommand{\tld}{\textasciitilde{}}
\newfontfamily\mleccha[Mapping=tex-text,Ligatures=Common,Scale=MatchLowercase]{Galatia SIL}%pour le grec
\newcommand{\grec}[1]{{\mleccha #1}}
%https://www.academia.edu/s/39e54582db/sanskrit-h%E1%B9%9Bd-coeur

\begin{document} 
\title{Sanskrit \ipa{hṛd-} `coeur'}
\author{Guillaume Jacques\\ CNRS-CRLAO-INALCO}
\maketitle
%Merci: Agnes Korn, Romain Garnier, Rémy Viredaz, Eugen Hill, Sergio Neri, Tomasz Majtczak,Marc Bavant, Piotr Gąsiorowski, Mikhail Zhivlov, Siva Kalyan
%\paragraph{§1}
\section*{Introduction}
Le nom du `cœur' a en indo-iranien un reflet phonétiquement irrégulier. Le sanskrit \ipa{hṛd-} n. et l'avestique \ipa{zərəd-} n. (et leurs variantes thématiques \ipa{hṛdaya-} n. et \ipa{zərədaiia-} n.) remontent en effet à un proto-indo-iranien \ipa{*j́ʰṛd-} au lieu du $\dagger$\ipa{ćṛd-} que l'on attendrait étant donné que l'initiale de cet étymon provient de \ipa{*ḱ} dans toutes les autres langues de la famille (voir le dossier complet dans \citealt[417-423]{wodtko08NIL}). On attendrait en sanskrit $\dagger$\ipa{śṛd-}, forme qui est d'ailleurs  indirectement attestée par le nom \ipa{śraddhā-} f. `fidélité' et la collocation \ipa{śrad}+\textsc{dha} construite avec le datif, illustrée par l'exemple (\ref{ex:zrad}) (< \ipa{k̂red-dʰh_1-eh_2}, cf latin \ipa{crēdere} et vieil irlandais \ipa{cretim} `croire') \footnote{En avestique en revanche, l'adjectif cognat \ipa{zrazdā-} `gläubig, vertrauend' a le même développement irrégulier que le nom \ipa{zərəd-} `coeur' (\citealt[663]{mayrhofer92ewa}), même si l'on peut s'interroger sur la possibilité qu'il s'agisse d'une assimilation avec le \ipa{-z-} épenthétique provenant du groupe \ipa{*-d-dʰ-}. }

\begin{exe}
\ex \label{ex:zrad}
\gll \ipa{śrád} \ipa{índrasya} \ipa{dhattana} \ipa{vīryā̀ya}\\
avoir.foi.en(1) Indra:\textsc{gen.sg} avoir.foi.en(2):\textsc{imp:pl} héroisme:\textsc{dat.sg}\\
\glt `Ayez foi en l'héroisme d'Indra.' (RV 1.103.05)
\end{exe}
 
 \section{Contamination}
La seule explication claire présentée dans la littérature, d'après \citet[420]{wodtko08NIL}, est celle de \citet{szemerenyi70heart} qui propose une contamination avec la racine nominale \ipa{*ĝʰerd-} ou \ipa{*ĝʰerH-} qui se retrouve dans le grec \grec{χορδή} < \ipa{*ĝʰord-eh_2}`entrailles', le sanskrit \ipa{hirā́-} f. < \ipa{*ĝʰrH-eh_2} `Ader' et le hittite \ipa{karāt-} `entrailles' *[\ipa{krʔāt}] (\citealt[208]{schrijver91laryngeals}, \citealt[446]{kloekhorst08edhil}).

Notons au passage que l'alternance entre \ipa{*d} et \ipa{*H} dans ces étymons pourrait s'expliquer si l'on accepte l'effet Kortlandt, selon lequel \ipa{*d} $\rightarrow$ \ipa{*h_1} précédant ou une autre consonne (\citealt{kortlandt83numerals, garnier14kortlandt}). Il faudrait ainsi postuler une alternance ancienne entre un degré zéro \ipa{*ĝʰṛd-} et  \ipa{*ĝʰerh_1-} (< \ipa{*ĝʰerd-} par effet Kortlandt), le grec ayant généralisé le premier et le sanskrit et le hittite le second. On pourrait ainsi affiner la reconstruction de l'étymon `veine' en indo-iranien comme \ipa{*ĝʰrh_1-eh_2} avec une laryngale 1.

L'hypothèse de Szemerényi est possible -- les cas de contamination irrégulière entre noms appartenant à la même classe sémantique sont attestés, mais elle n'est pas fermement supportée par la phraséologie; les réflexes de \ipa{*ĝʰer[d|H]-} et \ipa{*k̂erd-} ne sont pas communément associés dans les textes, même tardifs.

Alternativement à la suggestion de Szemerényi, je propose une contamination due à une collocation entre le nom `cœur' et une racine verbale comprenant la séquence \ipa{*j́ʰr} en indo-iranien. 
%ljazyk "tongue" (with l from lizat' "to lick") instead (cf. Max Vasmer's Russisches Etymologisches Wörterbuch III, 485 s.v. jazyk).

Cette contamination pourrait être simplement due à une proximité sémantique (comme la forme russe dialectale \ipa{лязык} `langue' sous l'influence du verbe \ipa{лизать} `lécher', voir \citealt[III, 485]{vasmer55russisches}), mais il est possible également la possibilité d'une assimilation syntagmatique, contamination entre mots apparaissant dans la même collocation, le français \ipa{au fur et à mesure} qui vient d'un plus ancient \ipa{au feur et à mesure} par assimilation de la dernière syllabe (\citealt[16]{andersen80morpho}, \citealt[64]{fertig13analogy}). %\citet[49-50]{wackernagel26syntax}

On trouve en indo-iranien trois racines ayant pu apparaître dans des collocations faisant intervenir le nom `cœur', soit des verbes psychologiques comportant la séquence \ipa{*j́ʰr} en proto-indo-iranien. Il s'agit des racines ayant donné les verbes sanskrits \ipa{háryati} `se réjouir de', \ipa{hṛṣyati} `se réjouir, frissonner, se hérisser' et \ipa{hṛṇīte} `être fâché', qui seront évalués un à un.

\section{\ipa{háryati}} \label{sec:haryati}
Le sanskrit \ipa{háryati} `se réjouir de' est habituellement reconstruit comme \ipa{*ĝʰér-ye-}, d'une racine \ipa{*ĝʰer-} `Gefallen finden, begehren' (\citealt[176]{liv}). Il n'existe pas de collocation usuelle combinant ce verbe et sa famille avec le nom \ipa{hṛd-}. Le seule exemple du RV où ces deux mots apparaissent (\ref{ex:haryato}) est un vers avec double allitération \ipa{\textbf{I}yak\textbf{Ṣ}ati \textbf{H}a\textbf{R}yató \textbf{HṚ}ttá \textbf{IṢ}yati} où elles n'appartiennent pas dans le même syntagme.\footnote{Remarquons au passage que l'interprétation de ce vers fort obscur doit tenir en compte de l'ambiguïté du verbe \ipa{íyakṣati}, désidératif à la fois de \ipa{yaj-} `sacrifier' et de \ipa{naś-} `obtenir' (\ipa{íyakṣati} < \ipa{*h_2i-h_2ṇḱ-se/o-} et de celle de la racine \ipa{iṣ-}, qui résulte de la confusion de \ipa{*h_2eys-} `suchen' avec \ipa{*h_1eysh_2-} `antreiben'. } L'hypothèse d'une collocation proto-indo-iranienne du type \ipa{*ćṛd-í j́ʰáryati} `il se réjouit dans son cœur', d'où la contamination de la consonne initiale \ipa{*j́ʰṛd-í j́ʰáryati} dans nom `cœur' n'a donc pas de support textuel.

\begin{exe}
\ex \label{ex:haryato}
\glt \ipa{údīraya pitárā jārá ā́ bhágamíyakṣati haryató hṛttá iṣyati}
\gll \ipa{úd.īraya}	\ipa{pitárā}	\ipa{jārás}	\ipa{ā́}	\ipa{bhágam}	\ipa{íyakṣati}	\ipa{haryatás}	\ipa{hṛttás}	\ipa{iṣyati} \\
stimuler:\textsc{imp:sg} père:\textsc{acc:du} amant:\textsc{nom:sg} \textsc{prep} fortune:\textsc{acc:sg:n} vouloir.obtenir:\textsc{present:ind:3sg}  désiré:\textsc{nom:sg} coeur:\textsc{abl:sg} aspirer:\textsc{present:ind:3sg} \\
\glt Promeus tes deux parents (Ciel et Terre), comme l'amant (promeut) le bonheur! (Agni, ce dieu) désirable, cherche à obtenir (un bien pour l'Homme), il suscite (l'inspiration) du fond du cœur. (RV X,11.06; Traduction Renou, EVP XIV, 1965, p. 8)
\end{exe}

\citet[104]{cheung07dictionary}, sur la base de formes chorasmiennes, reconstruit une vélaire en proto-iranien (\ipa{*gar} `rejoice'), qui devrait être projetée en indo-européen, et amener à réviser la reconstruction comme \ipa{*gʰer-}. Les formes iraniennes en \ipa{z} citées pour démontrer la palatale dans cette racine sont en effet douteuses; le nom avestique \ipa{zara-} est de sens incertain, et l'ossète \ipa{uzæld-} < \ipa{*awa-zarya} `to nurse, tend, to coax' est rapproché par  \citet[470]{cheung07dictionary} de racine \ipa{*zarH} `to bewail the deceased' (de \ipa{*ĝar-} `tönen, rufen', \citealt[161]{liv}). La reconstruction de Cheung, si elle est acceptée, invalide donc l'hypothèse proposée ci-dessus.

\section{\ipa{hṛṣyati}} \label{sec:hrsyati}
Le sanskrit \ipa{hṛṣyati} `se réjouir, frissonner, se hérisser' et l'avestique \ipa{zarš-} (\citealt[471]{cheung07dictionary}) proviennent de la racine \ipa{*ĝʰers-} `sich sträuben, erstarren' (\citealt[178]{liv}) à palatale. 

Le LIV (\citealt[198]{liv}), suivant \citet[808]{mayrhofer92ewa}  suggère la confusion avec une autre racine \ipa{*g(ʷ)ʰers-} `sich erfreuen', mais cette idée est contestée par \citet[471]{cheung07dictionary}, qui explique les formes sans palatales telles que le parthe \ipa{gš-} `to rejoice' comme due à la dépalatalisation de \ipa{*ĝʰṛs-} au degré zéro (\ipa{*ĝʰṛs-} $\rightarrow$ \ipa{*gʰṛs-}) par la loi de Weise. En védique on trouve des formes non-palatalisées \ipa{ghṛ́ṣu-} et \ipa{ghṛ́ṣvi-}, épithètes s'appliquant aux marouts, dont il n'est pas clair si elles dérivent du sens de `se réjouir' (\ipa{*g(ʷ)ʰers-}) ou de celui de `frissonner, se hérisser' (\ipa{*ĝʰers-}) par la même loi de Weise (\ipa{ghṛ́ṣu-} $\leftarrow$ \ipa{*gʰṛsu-} $\leftarrow$ \ipa{*ĝʰṛsu-}): comme le montre (\ref{ex:ghrsvayah}), on ne peut absolument exclure l'une ou l'autre des interprétations.

\begin{exe}
\ex \label{ex:ghrsvayah}
\glt
\ipa{mádanti} \ipa{vīrā́} \ipa{vidátheṣu} \ipa{ghṛ́ṣvayaḥ}
\gll \ipa{mádanti} \ipa{vīrā́s} \ipa{vidátheṣu} \ipa{ghṛ́ṣvayas} \\
se.réjouir/s'enivrer:\textsc{present.act.3pl} héros:\textsc{nom.pl} congrégation.rituelle:\textsc{loc.pl} frissonant/se.réjouissant:\textsc{nom.pl} \\
\glt The heroes reach exhilaration, eager at the rites. (Jamison)
\glt \ipa{(эти) мужи опьяняются на жертвенных собраниях, дрожа (от нетерпения) (Jelizarenkova)} (RV.1.85.1)
\end{exe}
%07,121.033c	durhṛdām apraharṣāya suhṛdāṃ harṣaṇāya ca 
%03,023.038c	śaṅkhaṃ pradhmāpya harṣeṇa suhṛdaḥ paryaharṣayam 
%05,007.030c	vṛtaḥ pratiyayau hṛṣṭaḥ suhṛdaḥ saṃpraharṣayan 
%12,039.019c	suhṛdāṃ harṣajananaḥ puṇyaḥ śrutisukhāvahaḥ 

Si aucune collocation entre le nom \ipa{hṛd-} et ses dérivés et la racine \ipa{hṛṣ-} n'est attestée en védique, on trouve des cas de ce type en sanskrit classique. Par exemple, (\ref{ex:hrdayaharsinis}) illustre le composé \ipa{mano-hṛdaya-harṣin-} `réjouissant les esprits et les cœurs', formé de \ipa{manas-} n. `esprit' de \ipa{hṛdaya-} n. `cœur' (un dérivé thématique de \ipa{hṛd-}, qui a pour cognat avestique exact \ipa{zərədaiia-}) et du nom \ipa{harṣin-} `réjouissant' dérivé de la racine \ipa{hṛṣ-}. 

\begin{exe} 
\ex \label{ex:hrdayaharsinis} 
\glt sādhu sādhviti sarvatra niśceruḥ stutisaṃhitāḥ 
\glt vācaḥ puṇyāḥ kīrtimatāṃ manohṛdayaharṣiṇīḥ 
\gll 
\ipa{sādhu} \ipa{sādhu} \ipa{iti} \ipa{sarvatra} \ipa{niścerur} \ipa{stuti-saṃhitās} 
\ipa{vācas} \ipa{puṇyās} \ipa{kīrtimatāṃ} \ipa{mano-hṛdaya-harṣiṇīs}\\
excellent excellent \textsc{quot} partout apparaître:\textsc{parfait:actif:3pl} louange-pourvu.de:\textsc{fem:pl:nom} parole:\textsc{fem:pl:nom}
pur:\textsc{fem:pl:nom} glorieux:\textsc{masc:pl:gen} esprit-cœur-réjouissant:\textsc{fem:pl:nom} \\ 
\glt `Les mots agréables `excellent, excellent', accompagnés de louanges à ces  hommes glorieux s'élevèrent de partout, réjouissant leurs esprits et leurs cœurs.' (Mbh, 06,041.102)
\end{exe}
%Excellent,--Excellent,--were the delightful words everywhere bruited about, coupled with eulogistic hymns about those famous men. And in consequence of this the minds and hearts of every one there were attracted towards them

Même si un composé de ce type ne saurait être ancien, il pourrait refléter la trace d'une collocation \ipa{*j́ʰṛdayam j́ʰaršayati} `réjouir le cœur' de date indo-iranienne, où l'assimilation syntagmatique aurait pu avoir eu lieu.

%02,023.006a	svasti vācyārhato viprān prayāhi bharatarṣabha
%02,023.006c	durhṛdām apraharṣāya suhṛdāṃ nandanāya ca
%Depart, bull among Bh´aratas, with the blessings of worthy priests,
%in pursuit of happiness for our friends and misery for our foes!

\section{\ipa{hṛṇīte}} \label{sec:hrnite}
La racine du sanskrit \ipa{hṛṇīte} `être fâché' (\ipa{*ĝʰe[r|l]H}, \citealt[178]{liv}), si elle manque de cognats externes, avait en proto-indo-iranien une palatale assurée par les formes iraniennes (voir \citealt[469]{cheung07dictionary} \ipa{*zarH} ‘to hurt, wound, anger’). 

On trouve en avestique à l'actif (présent de classe 9 comme le sanskrit) avec un sens causatif, comme dans la forme de 1pl optatif actif \ipa{zaranaēmā} `que nous (ne) provoquions (pas) ta colère' (Y 28.9),\footnote{On attendrait $\dagger$\ipa{zərənaēmā} au degré zéro, voir \citet[178]{liv}, note 3.  } ou préverbée en \ipa{ā-}. La différence entre avestique et sanskrit suggère ici une opposition de diathèse dans la proto-langue, l'actif ayant un sens causatif `provoquer la colère' et le médio-passif `se fâcher'. Parmi les formes iraniennes, le sens non causatif se retrouve en parthe \ipa{zr-} `to become angry'.

Je propose donc une collocation proto-indo-iranienne au médio-passif \ipa{*j́ʰṛd-í j́ʰr<n>H-tai} `il est fâché dans son cœur' (il est concevable que le nom `cœur' ait été à un cas autre que le locatif).  Elle proviendrait d'un plus ancien \ipa{*ćṛd-í j́ʰr<n>H-tai} par contamination de la séquence \ipa{*j́ʰr} de la racine verbale à un stade proto-indo-iranien, et aurait affecté tous les dérivés de la racine \ipa{*ćṛd} `cœur' excepté l'étymon `fidélité' mentionné ci-dessus, dont la parenté étymologique n'était plus ressentie à ce stade.

Si aucune collocation combinant les réflexes du nom \ipa{*k̂erd-} `cœur' à la racine de \ipa{hṛṇīte} `être fâché' n'est attestée en indo-iranien, il convient de noter que  le nom \ipa{*k̂erd-} `cœur' apparaît ailleurs dans la famille dans  la formation de collocation et même de verbes signifiant `être fâché'. Ainsi, le verbe hittite  \ipa{kardimiye/a-} `be angry' comporte un radical complexe dont le premier élément est universellement accepté comme apparenté à \ipa{ker / kard(i)-} `cœur', le réflexe de \ipa{*k̂erd-} (\citealt[456-7]{kloekhorst08edhil} suggère un verbe à incorporation, dont le deuxième élément serait \ipa{imiye/a} `mélanger'). Il ne s'agit d'ailleurs en aucun cas d'une spécificité indo-européenne : on trouve par exemple en lakhota (siouan) le verbe  \ipa{čhaŋzé} `be angry' dont le premier élément provient du nom \ipa{čhaŋté} `heart' (\citealt{ullrich08}).

\section*{Conclusion}
Ce travail a proposé des hypothèses alternatives à la solution de Szemerényi concernant l'irrégularité de la consonne initiale du nom `cœur' en indo-iranien. Aucune de ces hypothèses ne peut être considérée comme définitive ou irréfutable. 

Toutefois, ces hypothèses ne sont pas nécessairement incompatibles les unes avec les autres: si plusieurs collocations combinant le nom `cœur' avec des verbes en \ipa{*j́ʰ+r} ont existé en proto-indo-iranien (`réjouir le cœur', `se fâcher dans son cœur'), la pression d'une assimilation syntagmatique a pu être d'autant plus forte, et peut contribuer à expliquer pourquoi seule cette branche de la famille a connu une assimilation dans ce nom, contrairement par exemple au nom pour `langue', où l'assimilation avec le verbe `lécher' a eu lieu indépendamment dans plusieurs branches.


\bibliographystyle{unified}
\bibliography{bibliogj}
\end{document}
 