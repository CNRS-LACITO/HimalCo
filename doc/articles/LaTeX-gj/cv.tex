\documentclass[oldfontcommands,oneside,a4paper,11pt]{article} 
\usepackage{fontspec}
\usepackage{natbib}
\usepackage{booktabs}
\usepackage{xltxtra} 
\usepackage{polyglossia} 
%\setmainlanguage{french}
% \usepackage{geometry}
% \geometry{
% a4paper,
% total={210mm,297mm},
% left=10mm,
% right=10mm,
% top=15mm,
% bottom=15mm,
% }
\usepackage{bibentry}
\usepackage[table]{xcolor}
\usepackage{gb4e} 
\usepackage{graphicx}
\usepackage{float}
\usepackage{hyperref} 
\hypersetup{bookmarks=false,bookmarksnumbered,bookmarksopenlevel=5,bookmarksdepth=5,xetex,colorlinks=true,linkcolor=blue,citecolor=blue}
\usepackage[all]{hypcap}
\usepackage{memhfixc}

\bibpunct[: ]{(}{)}{,}{a}{}{,}
%%%%%%%%%quelques options de style%%%%%%%%
%\setsecheadstyle{\SingleSpacing\LARGE\scshape\raggedright\MakeLowercase}
%\setsubsecheadstyle{\SingleSpacing\Large\itshape\raggedright}
%\setsubsubsecheadstyle{\SingleSpacing\itshape\raggedright}
%\chapterstyle{veelo}
%\setsecnumdepth{subsubsection}
%%%%%%%%%%%%%%%%%%%%%%%%%%%%%%%
\setmainfont[Mapping=tex-text,Numbers=OldStyle,Ligatures=Common]{Charis SIL} %ici on définit la police par défaut du texte

\newfontfamily\phon[Mapping=tex-text,Ligatures=Common,Scale=MatchLowercase,FakeSlant=0.3]{Charis SIL} 
\newcommand{\ipa}[1]{{\phon #1}} %API tjs en italique
 \newcommand{\ipapl}[1]{{\phon #1}} %API tjs en italique
\newcommand{\grise}[1]{\cellcolor{lightgray}\textbf{#1}}
\newfontfamily\cn{SimSun}%pour le chinois
\newcommand{\zh}[1]{{\cn #1}}

\newcommand{\jg}[1]{\ipa{#1}\index{Japhug #1}}
\newcommand{\wav}[1]{#1.wav}
\newcommand{\tgz}[1]{\mo{#1} \tg{#1}}
\newcommand{\langue}[2]{#2}
\newcommand{\lingua}[3]{#2 (\zh{#3})}
 \XeTeXlinebreaklocale 'zh' %使用中文换行
\XeTeXlinebreakskip = 0pt plus 1pt %
\begin{document}
 \sloppy
%\section*{{\Huge \lingua{Curriculum Vitae}{Curriculum Vitae}{简历}}}
%\begin{itemize}
%\item \lingua{Guillaume \textsc{Jacques}}{Guillaume \textsc{Jacques}}{向柏霖}
%\item \lingua{Né le 16 septembre 1979 à Paris}{Born 16^{th} of September 1979 in Paris}{1979年9月16日出生于法国巴黎}
%\item \lingua{Marrié, un enfant.}{Married, one son.}{已婚;一个儿子}
%\end{itemize}


\title{ \lingua{Curriculum Vitae}{Curriculum Vitae}{简历}}
\author{\lingua{Guillaume \textsc{Jacques}}{Guillaume \textsc{Jacques}}{向柏霖}}
\maketitle

\sloppy
\section*{\lingua{Centres d'intérêt}{Research topics}{研究简介} }
\begin{itemize}
\item \lingua{Documentation de langues en danger}{Documentation of endangered languages}{描述濒危语言} 
\item \lingua{Linguistique historique}{Historical linguistics}{历史语言学} 
\item \lingua{Typologie morphosyntaxique}{Typological morposyntax}{形态句法的类型学研究} 
\item \lingua{Linguistique panchronique}{Panchronic linguistics}{泛时语言学}
\item \lingua{Phylogénie linguistics}{Linguistic phylogeny}{语言的谱系分类}
\end{itemize}

\section*{\lingua{Cursus universitaire}{Education}{学历}}
\begin{itemize}
\item 2014: \lingua{Habilitation à diriger les recherches}{Habilitation à diriger les recherches}{博士生导师}: \textit{Esquisse de phonologie et de morphologie historique du tangoute}, INALCO  (\lingua{directeur}{advisor}{导师}: Laurent Sagart)
\item 2004: \lingua{Doctorat de linguistique}{PhD in Linguistics}{语言学博士学位}: \textit{Phonologie et morphologie du japhug (rGyalrong)} Université Paris 7 (\lingua{directeur}{advisor}{导师}: Marie-Claude Paris)
\end{itemize}

\section*{\lingua{Situation professionnelle}{Work experience}{经历}}
\begin{itemize}
\item 2015-\lingua{présent}{present}{迄今}: \lingua{Directeur de recherches}{Senior researcher}{研究员} (DR2), \lingua{CNRS}{CNRS}{法国国立科学中心}, Centre de recherches linguistiques sur l'Asie Orientale (UMR 8563)
\item 2009-2015: \lingua{Chargé de recherche au CNRS}{Research associate}{副研究员} (CR1), \lingua{CNRS}{CNRS}{法国国立科学中心}, Centre de recherches linguistiques sur l'Asie Orientale (UMR 8563)
\item 2010: Visiting scholar, Research Center for Linguistic Typology, LaTrobe University, Melbourne.
\item 2005-2009: \lingua{Maître de conférences}{Associate Professor}{讲师}, \lingua{Université Paris Descartes}{Université Paris Descartes}{巴黎第五大学}, département de sciences du langage.
\item 2004-2005: \lingua{Postdoctorant}{Postdoctoral researcher}{博士后}: \lingua{Fondation Chiang Ching-Kuo}{Chiang Ching-Kuo foundation}{蒋经国基金会}
\end{itemize}


\section*{\lingua{Prix}{Awards}{荣誉}}  
\begin{itemize}
\item 2015: \lingua{Médaille de bronze du CNRS (section 34)}{CNRS Bronze medal}{法国国立科学研究中心的铜奖}.
\end{itemize}

\section*{\lingua{Activités éditoriales}{Editorial activities}{编辑}}
\begin{itemize}
\item 2013-\lingua{présent}{present}{迄今}: \lingua{Rédacteur en chef}{Editor}{主编}, \textit{Cahiers de Linguistique -- Asie orientale} (Brill)
\item 2015-\lingua{présent}{present}{迄今}: \lingua{Area editor pour la linguistique historique}{Area editor for historical linguistics}{历史语言学副主编},  \textit{Linguistic Vanguard} (Mouton de Gruyter)
\item 2014-\lingua{présent}{present}{迄今}: \lingua{membre du comité éditorial}{editorial committee member}{编委},  \textit{Linguistics of the Tibeto-Burman Area}  (Benjamins)
\item 2008-\lingua{présent}{present}{迄今}: \lingua{membre du comité éditorial}{editorial committee member}{编委},  \textit{Diachronica} (Benjamins)
\item \lingua{Relecteur pour les revues suivantes (en plus des précédentes)}{Reviewer for the following journals}{为下列期刊审稿}: \textit{Lingua}, \textit{Linguistic Typology}, \textit{Linguistics}, \textit{Studies in Language}, \textit{Folia Linguistica}, \textit{Journal of the International Phonetic Alphabet},  \textit{Language and Linguistics} (\zh{语言暨语言学}), \textit{Transactions of the Philological Society}, \textit{Journal of Chinese Linguistics}, \textit{SKY journal of linguistics}, \textit{Langages}, \textit{Yuyanxue luncong} (\zh{语言学论丛}), 	\textit{Himalayan Linguistics}, \textit{Bulletin de l’école Française d’Extrême-Orient}, 	\textit{International Journal of Diachronic Linguistics and Linguistic Reconstruction}, \textit{Writing Systems Research}, 	\textit{Journal of the Southeast Asian Linguistics Society}, \textit{Journal of Historical Linguistics}.
\end{itemize}

\subsection*{\lingua{Encadrement d'étudiants}{Doctoral supervision}{博导}}
\begin{itemize}
\item Lai Yunfan \zh{赖云帆} 2017. \textit{Grammaire du khroskyabs de Wobzi}, Université Paris III (en co-direction avec Pollet Samvelian)
\item Gao Yang \zh{高杨} 2015. \textit{Description de la langue muya}, EHESS.
\item 2013-\lingua{présent}{present}{迄今}: Gong Xun \zh{龚勋}, Normale supérieure-INALCO, \textit{Etude descriptive et historique de la langue zbu}.
\item 2016-\lingua{présent}{present}{迄今}: Zhang Shuya  \zh{张舒娅}, INALCO,  \textit{Le système de temps-aspect-mode-évidentialité du dialecte de Brag-dbar du situ: Description, implications typologique et historique}. 
\item 2016-\lingua{présent}{present}{迄今}: Julie Marsault, Paris III, \textit{Les modifieurs de valence verbale en omaha}. (en co-direction avec Pollet Samvelian)
\end{itemize}

\subsection*{\lingua{Projets de recherche}{Research projects}{研究计划}}
\begin{itemize}
%\item 2010-\lingua{présent}{present}{迄今}: Responsable de l'opération LR4.11-\textit{Automatic paradigm generation and language description} dans le cadre du Labex \textit{Empirical Foundations of Linguistics}.(\url{http://www.labex-efl.org/?q=fr/recherche/axe6})
\item 2017-\lingua{présent}{present}{迄今}: \lingua{Participant}{Participant}{计划参与者}, ERC Starting Grant 2016 (715618–CALC, 2017-2022) ``Computer-Assisted Language Comparison. Reconciling classical and computational approaches in historical linguistics." (\lingua{porteur du projet: Johann-Mattis List}{headed by Johann-Mattis List}{计划主持人:Johann-Mattis List})
\item  2013-2016: \lingua{Porteur}{Project head}{计划主持人}, \textbf{HimalCo}, \langue{projet ANR}{ANR project} (\url{http://himalco.hypotheses.org/})
\item 2008-2012: \lingua{Participant}{Participant}{计划参与者} \textbf{PASQi}, \langue{projet ANR}{ANR project} (\lingua{vec Katia Chirkova (porteur) et Alexis Michaud}{headed by Katia Chirkova}{计划主持人:齐卡佳})

\end{itemize}
\subsection*{\lingua{Conférences organisées}{Organisation of conferences}{主办的会议}}
\begin{itemize}
\item Journées de linguistique d'Asie orientale (juin 2008 et juin 2011, EHESS)
\item Workshop on Ergative Markers (09 novembre 2009, Villejuif)
\item 3rd Workshop on Sino-Tibetan Languages of Sichuan (septembre 2013, EHESS)
\item Kiranti Workshop (1-2 décembre 2016, Paris VII).
\end{itemize}

\newcommand{\deuxsemaines}{\lingua{2 semaines}{2 weeks}{两个星期}}
\newcommand{\unmois}{\lingua{1 mois}{1 month}{一个月}}
\newcommand{\deuxmois}{\lingua{2 mois}{2 months}{两个月}}
\newcommand{\troismois}{\lingua{3 mois et demi}{3.5 months}{三个半月}}
\subsection*{\lingua{Séjours de terrain}{Fieldwork}{田野调查}}
\begin{itemize}
\item   \lingua{Japhug}{Japhug}{茶堡嘉绒语} (\lingua{Chine, Sichuan, Mbarkhams}{China, Sichuan, Mbarkhams}{四川省马尔康县}): 2002 (\deuxmois{}), 2003 (\troismois{}), 2005 (\deuxmois{}), 2010 (\unmois{}), 2012 (\unmois{}), 2014 (\unmois{}), 2015 (\unmois{}), 2016 (\unmois{})
\item  \lingua{Zbu}{Zbu}{日部嘉绒语} (\lingua{Chine, Sichuan, Mbarkhams}{China, Sichuan, Mbarkhams}{四川省马尔康县}): 2003 (\unmois{})
\item  \lingua{Situ}{Situ}{四土嘉绒语}(\lingua{Chine, Sichuan, Mbarkhams}{China, Sichuan, Mbarkhams}{四川省马尔康县}): 2012 (\unmois{}), 2014 (\unmois{})
\item  \lingua{Pumi}{Pumi}{普米语} (\lingua{Chine, Sichuan, Muli et Yunnan, Yongning}{China, Sichuan, Muli and Yunnan, Yongning}{四川省木里县以及云南省宁蒗县}): 2008 (\unmois{}), 2009 (\unmois{})
\item  \lingua{Tibétain de Tchoné}{Chone Tibetan}{卓尼藏语} (Chine, Gansu; recherche effectuée à Chengdu): 2010 (\unmois{})
\item   \lingua{Chang naga}{Chang naga}{江纳尕语} (\lingua{Inde, Shillong}{India, Shillong}{印度东北}): 2007 (\unmois{})
\item  \lingua{Khaling}{Khaling}{卡陵语} (\lingua{Népal, Kathmandu et Solukhumbu}{Nepal, Kathmandu and Solukhumbu}{尼泊尔}): 2012 (\unmois{}), 2013 (\unmois{}), 2015 (\deuxsemaines{})
%\item 2013-4: étude du stau avec un informateur à Paris.
  \end{itemize}
  
\section*{\lingua{Corpus de données en ligne}{Online data corpora}{网上语料库}}
\begin{itemize}
\item \lingua{Textes japhug}{Japhug text corpus}{茶堡嘉绒语长篇语料库}: \url{http://lacito.vjf.cnrs.fr/pangloss/corpus/list\_rsc.php?lg=Japhug}
\item \lingua{Textes pumi}{Pumi text corpus}{普米语长篇语料库}: \url{http://lacito.vjf.cnrs.fr/pangloss/corpus/list_rsc.php?lg=Prinmi}
\item \lingua{Dictionnaire japhug}{Japhug dictionary}{茶堡嘉绒语词典}: \url{http://lacito.vjf.cnrs.fr/pangloss/dictionaries/ViewOneCharacter.php?sortorder=sort_order.xml&alphabet=*&dict=japhug&lang1=eng&lang2=fra&langn=cmn&char=a}
\item \lingua{Dictionnaire des verbes khaling}{Khaling verb dictionary}{卡陵语动词词典}: \url{http://lacito.vjf.cnrs.fr/pangloss/dictionaries/ViewOneCharacter.php?sortorder=sort_order.xml&alphabet=ipa&dict=khaling&lang1=eng&lang2=*&langn=*&char=\%CA\%94}
%\item \textbf{Japhug}: 60 heures de textes (depuis 2002), dont 45 transcrites et 8 traduites (sans compter les phrases et les listes de mots). 
%\item \textbf{Khaling}: 2 heures de textes transcrits et traduits (depuis 2011). Dictionnaire de verbes (653 racines, 148 pages), avec conjugueur automatique.
%\item \textbf{Stau}: 40 minutes de textes transcrits et traduits (depuis 2012). Dictionnaire de 1007 entrées (89 pages).
\end{itemize}
  
\section*{\lingua{Enseignement}{Teaching activities}{授课课程}}
\begin{itemize}
\item 2014-\lingua{présent}{present}{迄今}: Chargé de cours (INALCO): typologie et description morphosyntaxique.
\item 2014 école d'été du LACITO, Roscoff (\url{http://lacito.vjf.cnrs.fr/colloque/methodes/index\_en.htm})
%\item 2011 école d'été de l'INALCO, Agay
\item 2011 Chargé de cours (Paris 3): linguistique historique.
%\item 2010 école d'été de l'INALCO, Porquerolles
\item  2006 Ecole d'été de linguistique chinoise, Université de Leiden.
\item 2005-2009 Maître de conférences (Université Paris Descartes): linguistique diachronique, phonologie, syntaxe, langages formels et automates, phonétique.
\item 2001-2005 Allocataire moniteur au département de Linguistique (Paris 7): linguistique française, linguistique chinoise, phonétique.
\end{itemize}
 
  
%\section*{Invitations à l'étranger}
%\begin{itemize}
%\item   2010 (janvier-juin) Visiting scholar, Research Centre for Linguistic Typology, LaTrobe University, Melbourne, Australie.
% \item   \textbf{séminaires invités}:  Université de Genève (2003),  Université d'Oxford (2009), Université de Berne (2012),  Université de Zürich (2013).
% \item \textbf{conférences invitées}: Université Fudan, Shanghai (2005), Academia Sinica, Taiwan (2008, 2009 et 2010), School of Oriental and African Studies, Londres (2011, 2013, 2015), Musée d'Ethnologie, Osaka (2009), Université de Washington à Seattle (2013),  Université de Cambridge (2014), Université de Mayence (2015),  Université Pompeu Fabra, Barcelone (2017).
%  \end{itemize}
%    \bibliographystyle{Linquiry2}
%  \nobibliography{bibliogj.bib}
%\section*{Publications}
%  \subsection*{Ouvrages}
%  \begin{enumerate}

% \item  \bibentry{jacques14esquisse}
%\item \bibentry{jacques10gesar}
%   \item \bibentry{jacques08}
%  \item  \bibentry{jacques07textes}
%     
%\end{enumerate}
%    \subsection*{Articles}
%  Articles publiés (sans inclure les compte-rendus), selon le classement ERIH.
%    \begin{enumerate}
%\item A (17): \textit{Linguistic Inquiry} (1), \textit{Lingua} (3), \textit{Studies in Language} (1), \textit{Linguistic Typology} (1), \textit{Diachronica} (2), \textit{Anthropological Linguistics} (1), \textit{Bulletin of the School of Oriental and African Studies} (3),  \textit{Cahiers de linguistique d’Asie Orientale} (3), \textit{Etudes mongoles et sibériennes, centrasiatiques et tibétaines} (1), \textit{Journal of Chinese Linguistics} (1)
%
%\item B (13): \textit{Folia Linguistica Historica} (1), \textit{Transactions of the Philological Society} (1), \textit{Journal of the American Oriental Society} (1), \textit{Linguistics of the Tibeto-Burman Area} (3), \textit{Minzu yuwen} (3), \textit{Amerindia} (1), \textit{Faits de langues} (1), \textit{La linguistique} (1), \textit{Central Asiatic Journal} (1)
%\item C (1): \textit{Language and Linguistics Compass} (1) 
%\item Autres (8):       \textit{Language and Linguistics} (5; Q2 dans Scimago),   \textit{Studia Etymologica Cracoviensia} (1), \textit{Revue d'études tibétaines} (2)
%\item Participations à des volumes collectifs: Mouton de Gruyter (1), Brill (3)
%\end{enumerate}
%
%    \subsection*{Articles représentatifs}
%  %  Les  publications les plus importantes sont indiquées en gras.
%%  \small
%      \begin{enumerate}
%\item {\bibentry{jacques14antipassive}}
%\item {\bibentry{jacques14auditory}}
%\item {\bibentry{antonov14need}}
%%\item \bibentry{jacques14inverse}  
%\item {\bibentry{japhug14ideophones}}
%%\item  \bibentry{jacques13tropative}
%\item {\bibentry{jacques13harmonization}}
%%   \item   \bibentry{jacques13arapaho}   
%\item{ \bibentry{jacques12incorp}}
%%\item \bibentry{jacques12agreement}  
%%\item \bibentry{jacques12internal}  
%%  \item  \bibentry{rg-gj12yod}
%\item  {\bibentry{michaud-jacques12nasalite}}
% \item  {\bibentry{jacques12khaling}}
%%\item \bibentry{jacques11pumi.tone} 
%\item {\bibentry{jacques11lingua} }
%%\item \bibentry{jacques11tangut.verb} 
%%\item \bibentry{jacques11ngwemi} 
%\item {\bibentry{jacques.michaud11naish}} 
%\item {\bibentry{jacques10inverse}}
%%\item \bibentry{jacques10refl}
%% \item   \bibentry{jacques09tangutverb}   
%% \item  \bibentry{jacques07passif}
% 
% 
%  \end{enumerate}
%%      \subsection*{Autres publications}
%%      \begin{enumerate}
%%       
%%    \item   \bibentry{jacques14snom}   
%%    \item   \bibentry{jacques13vama}   
%%  \item   \bibentry{jacques13yod}   
%% 
%%       \item   \bibentry{jacques12bear}   
%%\item \bibentry{jacques12demotion}
%%     \item   \bibentry{jacques12transcription}  
%%              \item   \bibentry{jacques11kinship}    
%% \item   \bibentry{jacques10imperial}   
%% \item \bibentry{jacques10zos} 
%%    \item  \bibentry{jacques10kitan}  
%%       \item  \bibentry{michaud10bonin}   
%%      \item   \bibentry{jacques09zz}   
%% \item     \bibentry{jacques09e}   
%% \item \bibentry{jacques09wazur}
%%\item  \bibentry{jacques08debther} 
%% \item  \bibentry{jacques07redupl} 
%%  \item  \bibentry{jacques07chang} 
%%    \item  \bibentry{jacques03s.houzhui}  
%%  \item  \bibentry{jacques03dissimilation}  
%%   \item  \bibentry{jacques00ywij}  
% 
%  \end{enumerate}
%\bibliographystyle{Linquiry2}
%\bibliography{bibliogj}
\end{document}