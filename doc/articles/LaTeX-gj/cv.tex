\documentclass[oldfontcommands,oneside,a4paper,11pt]{article} 
\usepackage{fontspec}
\usepackage{natbib}
\usepackage{booktabs}
\usepackage{xltxtra} 
\usepackage{polyglossia} 
%\setmainlanguage{french}
 \usepackage{geometry}
 \geometry{
 a4paper,
 total={210mm,297mm},
 left=10mm,
 right=10mm,
 top=15mm,
 bottom=15mm,
 }
\usepackage{bibentry}
\usepackage[table]{xcolor}
\usepackage{gb4e} 
\usepackage{graphicx}
\usepackage{float}
\usepackage{hyperref} 
\hypersetup{bookmarks=false,bookmarksnumbered,bookmarksopenlevel=5,bookmarksdepth=5,xetex,colorlinks=true,linkcolor=blue,citecolor=blue}
\usepackage[all]{hypcap}
\usepackage{memhfixc}

\bibpunct[: ]{(}{)}{,}{a}{}{,}
%%%%%%%%%quelques options de style%%%%%%%%
%\setsecheadstyle{\SingleSpacing\LARGE\scshape\raggedright\MakeLowercase}
%\setsubsecheadstyle{\SingleSpacing\Large\itshape\raggedright}
%\setsubsubsecheadstyle{\SingleSpacing\itshape\raggedright}
%\chapterstyle{veelo}
%\setsecnumdepth{subsubsection}
%%%%%%%%%%%%%%%%%%%%%%%%%%%%%%%
\setmainfont[Mapping=tex-text,Numbers=OldStyle,Ligatures=Common]{Charis SIL} %ici on définit la police par défaut du texte

\newfontfamily\phon[Mapping=tex-text,Ligatures=Common,Scale=MatchLowercase,FakeSlant=0.3]{Charis SIL} 
\newcommand{\ipa}[1]{{\phon #1}} %API tjs en italique
 \newcommand{\ipapl}[1]{{\phon #1}} %API tjs en italique
\newcommand{\grise}[1]{\cellcolor{lightgray}\textbf{#1}}
\newfontfamily\cn[Mapping=tex-text,Ligatures=Common,Scale=MatchUppercase]{MingLiU}%pour le chinois
\newcommand{\zh}[1]{{\cn #1}}

\newcommand{\jg}[1]{\ipa{#1}\index{Japhug #1}}
\newcommand{\wav}[1]{#1.wav}
\newcommand{\tgz}[1]{\mo{#1} \tg{#1}}


 
\begin{document}
 \sloppy
\section*{{\LARGE Curriculum Vitae}}
Guillaume Jacques

%  \title{Curriculum Vitae}
% 
%\author{Guillaume Jacques}
%\maketitle
Né le 16 septembre 1979 à Paris, marié, un fils.
\sloppy
\section*{Compétences et centres d’intérêt}
\begin{itemize}
\item Documentation de langues en danger (japhug, khaling, stau)
\item Linguistique historique (sino-tibétain, indo-européen, sémitique, algonquien, sioux), 
\item Typologie morphosyntaxique (incorporation, voix et valence, systèmes d'indexation complexes, pivots syntaxiques)
\item Linguistique panchronique (principes généraux des changements phonétiques, de la grammaticalisation et des changements analogiques)
\item Phylogénie des langues.
\end{itemize}
%\begin{itemize}
%\item  langue maternelle: français
%\textbf{Langues}:  \textbf{C}: français (maternel), anglais, chinois; 
%   \textbf{B}: russe, japhug; 
%  \textbf{B (écrit)}: allemand, néerlandais;
%  \textbf{A (écrit)}: sanskrit, tangoute, tibétain classique, breton;  \textbf{A (parlé)}: khaling, stau
%  %\end{itemize}
%  
%    \textbf{Informatique}: \LaTeX, PERL
\section*{Situation professionnelle}
\begin{itemize}
\item 2015 Directeur de recherches au CNRS (DR2), Centre de recherches linguistiques sur l'Asie Orientale (UMR 8563)
\item 2014 Habilitation à diriger les recherches: \textit{Esquisse de phonologie et de morphologie historique du tangoute}, INALCO  (directeur: Laurent Sagart; jury: Alexander Vovin, Konstantin Pozdniakov, Boyd Michailovsky, Nathan W. Hill, Gilles Authier)
\item Depuis 2009 Chargé de recherche au CNRS (CR1), Centre de recherches linguistiques sur l'Asie Orientale (UMR 8563)
\item 2005-2009 Maître de conférences, Université Paris Descartes, département de sciences du langage.
\item 2004-2005 Postdoctorant, Fondation Chiang Ching-Kuo.
\item 2001-2004 Doctorat de linguistique: \textit{Phonologie et morphologie du japhug (rGyalrong)} Université Paris 7 (directeur: Marie-Claude Paris; jury: Laurent Sagart, Boyd Michailovsky, Jackson T.-S. Sun, Nicolas Tournadre)
\end{itemize}
 
 
  
  
 
\section*{Activités éditoriales}
\begin{itemize}
\item Rédacteur en chef de la revue \textit{Cahiers de Linguistique -- Asie orientale} (Brill, depuis 2013) et futur responsable de la linguistique historique à partir de 2015 pour la revue \textit{Linguistic Vanguard} (Mouton de Gruyter)
\item membre du comité éditorial de \textit{Diachronica} (Benjamins, depuis 2008) et de \textit{Linguistics of the Tibeto-Burman Area}  (Benjamins, depuis 2014).
\item Relecteur pour les revues suivantes (en plus des précédentes): \textit{Lingua}, \textit{Linguistic Typology}, \textit{Studies in Language}, \textit{Folia Linguistica}, \textit{Journal of the International Phonetic Alphabet},  \textit{Language and Linguistics}, \textit{Transactions of the Philological Society}, \textit{Journal of Chinese Linguistics}, \textit{SKY journal of linguistics}, \textit{Langages}, \textit{Yuyanxue luncong} \zh{语言学论丛}.
\end{itemize}

\section*{Enseignement}
\begin{itemize}
\item 2014-7 Chargé de cours (INALCO): typologie et description morphosyntaxique.
\item 2014 école d'été du LACITO, Roscoff (\url{http://lacito.vjf.cnrs.fr/colloque/methodes/index\_en.htm})
%\item 2011 école d'été de l'INALCO, Agay
\item 2011 Chargé de cours (Paris 3): linguistique historique.
%\item 2010 école d'été de l'INALCO, Porquerolles
\item  2006 Ecole d'été de linguistique chinoise, Université de Leiden.
\item 2005-2009 Maître de conférences (Université Paris Descartes): linguistique diachronique, phonologie, syntaxe, langages formels et automates, phonétique.
\item 2001-2005 Allocataire moniteur au département de Linguistique (Paris 7): linguistique française, linguistique chinoise, phonétique.
\end{itemize}


\subsection*{Encadrement d'étudiants}
\begin{itemize}
\item Gao Yang, EHESS, \textit{Description de la langue muya} (Sino-tibétain, birmo-qianguique, en co-direction avec Laurent Sagart), thèse soutenue en décembre 2015.
\item 2013-présent, doctorat de Gong Xun, Normale supérieure-INALCO, \textit{Etude descriptive et historique de la langue zbu}.
\item 2013-présent, doctorat de Lai Yunfan, Paris III, \textit{Grammaire du Kroskyabs} (en co-direction avec Pollet Samvelian)
\item 2016-présent,  doctorat de Zhang Shuya, INALCO,  \textit{Le système de temps-aspect-mode-évidentialité du dialecte de Brag-dbar du situ: Description, implications typologique et historique}.
\item 2016-présent,  doctorat de Julie Marsault, Paris III, \textit{Les modifieurs de valence verbale en omaha}.
\end{itemize}
\subsection*{Animation de la recherche}
\begin{itemize}
\item 2009-présent: Directeur de l'équipe de recherche \textit{Linguistique descriptive des langues d’Asie, phonologie, morphosyntaxe et comparatisme} au sein du CRLAO.
\item 2008-2012: Projet ANR \textbf{PASQi} (avec Katia Chirkova (porteur) et Alexis Michaud) 
\item  2013-2016: Projet ANR Corpus \textbf{HimalCo} (porteur du projet; avec Alexis Michaud, Aimée Lahaussois et Séverine Guillaume) (\url{http://himalco.hypotheses.org/})
\item 2010-présent: Responsable de l'opération PPC2-\textit{Evolutionary approaches to phonology} et de l'opération LR4.11-\textit{Automatic paradigm generation and language description} dans le cadre du Labex \textit{Empirical Foundations of Linguistics}.(\url{http://www.labex-efl.org/?q=fr/recherche/axe6})
\item conférences organisées: Journées de linguistique d'Asie orientale (juin 2008 et juin 2011, EHESS),  Workshop on Ergative Markers (09 novembre 2009, Villejuif), 3rd Workshop on Sino-Tibetan Languages of Sichuan (septembre 2013, EHESS), Kiranti Workshop (1-2 décembre 2016, Paris VII).
\end{itemize}
\subsection*{Séjours de terrain} 
\begin{itemize}
\item   2002 (2 mois), 2003 (3 mois et demi), 2005 (2 mois), 2010 (1 mois), 2012 (1 mois), 2014 (1 mois), 2015 (1 mois), 2016 (1 mois): \textbf{japhug} (Chine, Sichuan, Mbarkhams)
\item  2003 (1 mois): \textbf{zbu} (Chine, Sichuan, Mbarkhams)
\item  2012 (1 mois), 2014 (1 mois): \textbf{situ} (Chine, Sichuan, Mbarkhams)
\item  2008 (1 mois), 2009 (1 mois): \textbf{pumi} (Chine, Yunnan, Yongning et Sichuan, Muli)
\item  2010 (1 mois): \textbf{tibétain de Tchoné} (Chine, Sichuan)
\item  2007 (1 mois): \textbf{chang naga} (Inde, Shillong)
\item  2012 (1 mois), 2013 (1 mois), 2015 (2 semaines): \textbf{khaling} (Népal, Kathmandu et Solukhumbu)
\item 2013-4: étude du stau avec un informateur à Paris.
  \end{itemize}
  
\section*{Corpus de données en ligne}
\begin{itemize}
\item Textes japhug: \url{http://lacito.vjf.cnrs.fr/pangloss/corpus/list_rsc.php?lg=Japhug}
\item Textes prinmi: \url{http://lacito.vjf.cnrs.fr/pangloss/corpus/list_rsc.php?lg=Prinmi}
\item Dictionnaire japhug, version en ligne: \url{http://lacito.vjf.cnrs.fr/pangloss/dictionaries/ViewOneCharacter.php?sortorder=sort_order.xml&alphabet=*&dict=japhug&lang1=eng&lang2=fra&langn=cmn&char=a}
\item Dictionnaire khaling, version en ligne: \url{http://lacito.vjf.cnrs.fr/pangloss/dictionaries/ViewOneCharacter.php?sortorder=sort_order.xml&alphabet=ipa&dict=khaling&lang1=eng&lang2=*&langn=*&char=\%CA\%94}
%\item \textbf{Japhug}: 60 heures de textes (depuis 2002), dont 45 transcrites et 8 traduites (sans compter les phrases et les listes de mots). 
%\item \textbf{Khaling}: 2 heures de textes transcrits et traduits (depuis 2011). Dictionnaire de verbes (653 racines, 148 pages), avec conjugueur automatique.
%\item \textbf{Stau}: 40 minutes de textes transcrits et traduits (depuis 2012). Dictionnaire de 1007 entrées (89 pages).
\end{itemize}
  
 
  
\section*{Invitations à l'étranger}
\begin{itemize}
\item   2010 (janvier-juin) Visiting scholar, Research Centre for Linguistic Typology, LaTrobe University, Melbourne, Australie.
 \item   \textbf{séminaires invités}:  Université de Genève (2003),  Université d'Oxford (2009), Université de Berne (2012),  Université de Zürich (2013).
 \item \textbf{conférences invitées}: Université Fudan, Shanghai (2005), Academia Sinica, Taiwan (2008, 2009 et 2010), School of Oriental and African Studies, Londres (2011, 2013, 2015), Musée d'Ethnologie, Osaka (2009), Université de Washington à Seattle (2013),  Université de Cambridge (2014), Université de Mayence (2015),  Université Pompeu Fabra, Barcelone (2017).
  \end{itemize}
    \bibliographystyle{Linquiry2}
  \nobibliography{bibliogj.bib}
%\section*{Publications}
%  \subsection*{Ouvrages}
%  \begin{enumerate}

% \item  \bibentry{jacques14esquisse}
%\item \bibentry{jacques10gesar}
%   \item \bibentry{jacques08}
%  \item  \bibentry{jacques07textes}
%     
%\end{enumerate}
%    \subsection*{Articles}
%  Articles publiés (sans inclure les compte-rendus), selon le classement ERIH.
%    \begin{enumerate}
%\item A (17): \textit{Linguistic Inquiry} (1), \textit{Lingua} (3), \textit{Studies in Language} (1), \textit{Linguistic Typology} (1), \textit{Diachronica} (2), \textit{Anthropological Linguistics} (1), \textit{Bulletin of the School of Oriental and African Studies} (3),  \textit{Cahiers de linguistique d’Asie Orientale} (3), \textit{Etudes mongoles et sibériennes, centrasiatiques et tibétaines} (1), \textit{Journal of Chinese Linguistics} (1)
%
%\item B (13): \textit{Folia Linguistica Historica} (1), \textit{Transactions of the Philological Society} (1), \textit{Journal of the American Oriental Society} (1), \textit{Linguistics of the Tibeto-Burman Area} (3), \textit{Minzu yuwen} (3), \textit{Amerindia} (1), \textit{Faits de langues} (1), \textit{La linguistique} (1), \textit{Central Asiatic Journal} (1)
%\item C (1): \textit{Language and Linguistics Compass} (1) 
%\item Autres (8):       \textit{Language and Linguistics} (5; Q2 dans Scimago),   \textit{Studia Etymologica Cracoviensia} (1), \textit{Revue d'études tibétaines} (2)
%\item Participations à des volumes collectifs: Mouton de Gruyter (1), Brill (3)
%\end{enumerate}
%
%    \subsection*{Articles représentatifs}
%  %  Les  publications les plus importantes sont indiquées en gras.
%%  \small
%      \begin{enumerate}
%\item {\bibentry{jacques14antipassive}}
%\item {\bibentry{jacques14auditory}}
%\item {\bibentry{antonov14need}}
%%\item \bibentry{jacques14inverse}  
%\item {\bibentry{japhug14ideophones}}
%%\item  \bibentry{jacques13tropative}
%\item {\bibentry{jacques13harmonization}}
%%   \item   \bibentry{jacques13arapaho}   
%\item{ \bibentry{jacques12incorp}}
%%\item \bibentry{jacques12agreement}  
%%\item \bibentry{jacques12internal}  
%%  \item  \bibentry{rg-gj12yod}
%\item  {\bibentry{michaud-jacques12nasalite}}
% \item  {\bibentry{jacques12khaling}}
%%\item \bibentry{jacques11pumi.tone} 
%\item {\bibentry{jacques11lingua} }
%%\item \bibentry{jacques11tangut.verb} 
%%\item \bibentry{jacques11ngwemi} 
%\item {\bibentry{jacques.michaud11naish}} 
%\item {\bibentry{jacques10inverse}}
%%\item \bibentry{jacques10refl}
%% \item   \bibentry{jacques09tangutverb}   
%% \item  \bibentry{jacques07passif}
% 
% 
%  \end{enumerate}
%%      \subsection*{Autres publications}
%%      \begin{enumerate}
%%       
%%    \item   \bibentry{jacques14snom}   
%%    \item   \bibentry{jacques13vama}   
%%  \item   \bibentry{jacques13yod}   
%% 
%%       \item   \bibentry{jacques12bear}   
%%\item \bibentry{jacques12demotion}
%%     \item   \bibentry{jacques12transcription}  
%%              \item   \bibentry{jacques11kinship}    
%% \item   \bibentry{jacques10imperial}   
%% \item \bibentry{jacques10zos} 
%%    \item  \bibentry{jacques10kitan}  
%%       \item  \bibentry{michaud10bonin}   
%%      \item   \bibentry{jacques09zz}   
%% \item     \bibentry{jacques09e}   
%% \item \bibentry{jacques09wazur}
%%\item  \bibentry{jacques08debther} 
%% \item  \bibentry{jacques07redupl} 
%%  \item  \bibentry{jacques07chang} 
%%    \item  \bibentry{jacques03s.houzhui}  
%%  \item  \bibentry{jacques03dissimilation}  
%%   \item  \bibentry{jacques00ywij}  
% 
%  \end{enumerate}
%\bibliographystyle{Linquiry2}
%\bibliography{bibliogj}
\end{document}