\documentclass[oldfontcommands,twoside,a4paper,12pt]{article} 
\usepackage{fontspec}
\usepackage{natbib}
\usepackage{booktabs}
\usepackage{xltxtra} 
\usepackage{polyglossia} 
 \usepackage{geometry}
% \geometry{
% a4paper,
% total={210mm,297mm},
% left=10mm,
% right=10mm,
% top=15mm,
% bottom=15mm,
% }
\usepackage[table]{xcolor}
\usepackage{color}
\usepackage{multirow}
\usepackage{gb4e} 
\usepackage{multicol}
\usepackage{graphicx}
\usepackage{float}
\usepackage{hyperref} 
\hypersetup{bookmarks=false,bookmarksnumbered,bookmarksopenlevel=5,bookmarksdepth=5,xetex,colorlinks=true,linkcolor=blue,citecolor=blue}
\usepackage{memhfixc}
\usepackage{lscape}
\usepackage[footnotesize,bf]{caption}
 

%%%%%%%%%%%%%%%%%%%%%%%%%%%%%%%
\setmainfont[Mapping=tex-text,Numbers=OldStyle,Ligatures=Common]{Charis SIL} 
\setsansfont[Mapping=tex-text,Ligatures=Common,Mapping=tex-text,Ligatures=Common,Scale=MatchLowercase]{Lucida Sans Unicode} 
 


\newfontfamily\phon[Mapping=tex-text,Ligatures=Common,Scale=MatchLowercase]{Charis SIL} 
\newfontfamily\phondroit[Mapping=tex-text,Ligatures=Common,Scale=MatchLowercase]{Doulos SIL} 
\newcommand{\ipa}[1]{{\phon\textit{#1}}} 
\newcommand{\ipab}[1]{{\phon #1}}
\newcommand{\ipapl}[1]{{\phondroit #1}} 
\newcommand{\captionft}[1]{{\captionfont #1}} 
\newfontfamily\cn[Mapping=tex-text,Ligatures=Common,Scale=MatchUppercase]{SimSun}%pour le chinois
\newcommand{\zh}[1]{{\cn #1}}
\newcommand{\tgf}[1]{{\large\mo{#1}}}
\newcommand{\rc}{}
\newcommand{\tete}{}
\newcommand{\topic}{\textsc{top}}
\newcommand{\racine}[1]{\begin{math}\sqrt{#1}\end{math}} 
\newcommand{\grise}[1]{\cellcolor{lightgray}\textbf{#1}} 
\newcommand{\tinynb}[1]{\tiny#1}

\newcommand{\ro}{$\Sigma$}
\newcommand{\siga}{$\Sigma_1$} 
\newcommand{\sigc}{$\Sigma_3$}   
\begin{document}
\title{Examen de typologie, 6 décembre 2017}
\maketitle
\section*{Exercice 1 (budugh)}

 \begin{enumerate}
  \item Déterminez, à partir de ces phrases en budugh, l'alignement (du marquage) de cette langue.
  \item Pourquoi \ipa{rij(ir)} à l'exemple (2) admet aussi bien la forme \ipa{rij} que \ipa{rijir} ?
  \item A quoi pourraient être dues les variations du vocalisme des formes verbales ?
  \item Quelle est l'information précieuse que nous apporte l'exemple (4) ?
  \item Glosez (en segmentant au maximum) les exemples.
  \item Traduisez : ``C’est le garçon qui a fait sécher la robe. La fille s’est endormie.''

\end{enumerate}

\vspace{0.5cm}

\begin{exe}
\ex
\gll \ipa{Gada’r} \ipa{rijovon} \ipa{``Buluşa} \ipa{ sö’ür!''} \ipa{yıpaci.}\\
%{2\sg-\erg} {1\sg(\abs)} {catch.\masc.\perf-\pret=2\sg}
\\
\glt Le garçon a dit à la fille « Fais sécher la robe ! ».
\end{exe}

\begin{exe}
\ex
\gll \ipa{Rij(ir)} \ipa{gadoxun} \ipa{ösül} \ipa{gadovon} \ipa{``Buluşa} \ipa{so’oci.} \ipa{Buluşa} \ipa{ ösül!''} \ipa{yıpaci.}\\
%{2\sg-\erg} {1\sg(\abs)} {catch.\masc.\perf-\pret=2\sg}
\\
\glt La fille s’est tournée vers le garçon (et) a dit au garçon « La robe est sèche. Tourne la robe. »
\end{exe}

\begin{exe}
\ex
\gll \ipa{Xhın} \ipa{sa’aci.} \ipa{Gada} \ipa{exirci.}\\
%{2\sg-\erg} {1\sg(\abs)} {catch.\masc.\perf-\pret=2\sg}
\\
\glt L’herbe a séché. Le garçon s’est endormi.
\end{exe}

\begin{exe}
\ex
\gll \ipa{Xhın} \ipa{rijir}/\ipa{gada'r} \ipa{se’ir} \ipa{esilci.}\\
%{2\sg-\erg} {1\sg(\abs)} {catch.\masc.\perf-\pret=2\sg}
\\
\glt C’est la fille/le garçon qui a fait sécher (et) retourné l’herbe.
\end{exe} 


%\newpage
\section*{Exercice 2 (garifuna)}
%3ms.st

\begin{enumerate}
\item Décrivez le système d'indexation dans cette langue en précisant son alignement.
\item Traduisez les phrases suivantes: 
\begin{enumerate}
\item `Je vais la tuer.'
\item `Tu seras gentil.'
\item `Je vais t'épouser.'
\end{enumerate}
\end{enumerate}

\begin{multicols}{2}
 \begin{exe}
\ex 
\glt \ipa{bwídubei}  \ipa{ównli}
 \glt  `Le chien (m) va être gentil.'
\end{exe}

%3fs.st
 \begin{exe}
\ex 
\glt \ipa{bímebon}  \ipa{fáluma}
 \glt  `La noix de coco (f) va être sucrée.'
\end{exe}

 \begin{exe}
\ex 
\glt \ipa{nídiba}  
 \glt  `J'irai.'
\end{exe}

 \begin{exe}
\ex 
\glt \ipa{tídiba}  
 \glt  `Elle ira.'
\end{exe}

 \begin{exe}
\ex 
\glt \ipa{narísiduba}  
 \glt  `Je m'enrichirai.'
\end{exe}

 \begin{exe}
\ex 
\glt \ipa{barísiduba}  
 \glt  `Tu t'enrichiras.'
\end{exe}

 \begin{exe}
\ex 
\glt \ipa{larísiduba}  
 \glt  `Il s'enrichira.'
\end{exe}

 \begin{exe}
\ex 
\glt \ipa{lónweba}  
 \glt  `Il mourra.'
\end{exe}

%%2p.act
% \begin{exe}
%\ex 
%\glt \ipa{harúmuguba}  
% \glt  `Vous allez dormir.'
%\end{exe}
 
%3ms>3ms 
  \begin{exe}
\ex 
\glt \ipa{láfarubei}    
 \glt  `Il va le tuer'.
\end{exe}

  \begin{exe}
\ex 
\glt \ipa{lamáriedubana}    
 \glt  `Il va m'épouser.'
\end{exe}

  \begin{exe}
\ex 
\glt \ipa{lamáriedubabu}    
 \glt  `Il va t'épouser.'
\end{exe}

  \begin{exe}
\ex 
\glt \ipa{tamáriedubei}    
 \glt  `Elle va l'épouser.'
\end{exe}

  \begin{exe}
\ex 
\glt \ipa{naníbirubei}    
 \glt  `Je le bénirai.'
\end{exe}

%1s>3fs
 \begin{exe}
\ex 
\glt \ipa{nátubon}    
 \glt  `Je vais la boire (la bière, f).'
\end{exe}

%1p>3fs
% \begin{exe}
%\ex 
%\glt \ipa{waséfurubon} \ipa{gurúyara}    
% \glt  `Nous allons sauver le canoé (f).'
%\end{exe}
\end{multicols}
\end{document}

