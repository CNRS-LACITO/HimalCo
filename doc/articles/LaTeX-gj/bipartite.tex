\documentclass[11pt]{article} 
\usepackage{fontspec}
\usepackage{natbib}
\usepackage{booktabs}
\usepackage{xltxtra} 
\usepackage{polyglossia} 
\usepackage[table]{xcolor}
\usepackage{tikz}
\usetikzlibrary{trees}
\usepackage{gb4e} 
\usepackage{multicol}
\usepackage{graphicx}
\usepackage{float}
\usepackage{hyperref} 
\hypersetup{bookmarks=false,bookmarksnumbered,bookmarksopenlevel=5,bookmarksdepth=5,xetex,colorlinks=true,linkcolor=blue,citecolor=blue}
\usepackage[all]{hypcap}
\usepackage{memhfixc}
\usepackage{lscape}
\usepackage{bbding}
 
%\setmainfont[Mapping=tex-text,Numbers=OldStyle,Ligatures=Common]{Charis SIL} 
\newfontfamily\phon[Mapping=tex-text,Ligatures=Common,Scale=MatchLowercase]{Charis SIL} 
\newcommand{\ipa}[1]{{\phon\textbf{#1}}} 
\newcommand{\grise}[1]{\cellcolor{lightgray}\textbf{#1}}
\newfontfamily\cn[Mapping=tex-text,Ligatures=Common,Scale=MatchUppercase]{SimSun}%pour le chinois
\newcommand{\zh}[1]{{\cn #1}}
\newcommand{\Y}{\Checkmark} 
\newcommand{\N}{} 
\newcommand{\dhatu}[2]{|\ipa{#1}| `#2'}
\newcommand{\jpg}[2]{\ipa{#1} `#2'}  
\newcommand{\refb}[1]{(\ref{#1})}
\newcommand{\tld}{\textasciitilde{}}

 \begin{document} 
\title{Bipartite verbs in Japhug and other Trans-Himalayan languages\footnote{Acknowledgements to be added after editorial decision.  The examples are taken from a corpus that is progressively being made available on the Pangloss archive (\citealt{michailovsky14pangloss}, 
 \url{http://lacito.vjf.cnrs.fr/pangloss/corpus/list\textunderscore rsc.php?lg=Japhug}).  }}
%\author{Guillaume Jacques\\ CNRS-CRLAO-INALCO}
\maketitle

%Scott DeLancey, Olivier Bonami, Nathan W. Hill, Randy LaPolla, Willem de Reuse, Alexey Vinyar, Nathan Straub, Yoram Meroz

\textbf{Abstract}: This paper presents an overview of bipartite verbs in Japhug and describes related constructions in that language, including compound and serial verbs. Several hypotheses are proposed to account for the genesis of these constructions and the historical relationship between them. Typological comparisons with other languages of the family, including Kiranti and Rawang, are offered to illustrate the specificities of the Japhug constructions.

\textbf{Keywords}: Bipartite verbs, Complex predicates Serial verb constructions, Person indexation, Japhug, Khaling, Rawang, Bantawa, Kiranti

\sloppy

 \section*{Introduction}
Bipartite verbs, though common in some areas of the world (\citealt{delancey96bipartite}), are relatively rare in Eurasia. In the Trans-Himalayan family, bipartite verbs are found in Kiranti and Gyalrongic, and present another uncommon typological characteristic, 
multiple argument indexation (\citealt{denk15multiple}). While in Kiranti many studies have documented bipartite verbs (see in particular \citealt{driem87}, \citealt{rutgers98yamphu}, \citealt{bickel07chintang}, \citealt{doornenbal09}, \citealt{schackow15yakkha}) this type of construction is still poorly investigated in Gyalrongic.

In this paper, the term `bipartite verbs' exclusively refers to lexical units comprising two verb stems, both of which can be (at least partially) conjugated. Incorporating verbs (on which \citealt{jacques12incorp}) and noun+verb collocations are not discussed in this work.

This paper comprises five sections. First, it provides a description of bipartite verbs and of their four conjugation types in Japhug. Second, it describes compound verbs and how they differ from bipartite verbs. Third, it discusses serial verb constructions and how they relate to bipartite verbs. Fourth, it presents an overview of the different scenarios that could explain the genesis of bipartite verbs in Japhug. Fifth, it shows how the model in section 4 can be used to better interpret the prehistory of bipartite verbs in Kiranti languages.

\section{Bipartite verbs in Japhug} \label{sec:japhug.bipart}

Bipartite verbs in Japhug and other Gyalrong languages are a small class (only ten such verbs have been discovered up to now). 
The most common bipartite verb, \jpg{stu=mbat}{try hard, do one's best (\zh{努力})} can be conjugated in four distinct ways. Table \ref{tab:four} illustrates these four options with the imperative second dual form `try hard (the two of you)'.


\begin{table}[H]
\caption{Four degrees of integration} \centering \label{tab:four}
\begin{tabular}{lllllll}
\toprule
Type & Example & V_1 suffix & V_2 prefix \\
\midrule
A (quasi-SVC) & \ipa{tɤ-stu-ndʑi} \ipa{tɤ-mbat-ndʑi} &\Y &\Y \\
 &\textsc{imp}-V_1-\textsc{du}  \textsc{imp}-V_2-\textsc{du} \\
B (right-dominant) & \ipa{tɤ-stu=tɤ-mbat-ndʑi} &\N  &\Y \\
 &\textsc{imp}-V_1-\textsc{imp}-V_2-\textsc{du} \\
C (left-dominant) & \ipa{tɤ-stu-ndʑi=mbat-ndʑi} &\Y  &\N \\
 &\textsc{imp}-V_1-\textsc{du}-V_2-\textsc{du} \\
D (quasi-compound)& \ipa{tɤ-stu-mbat-ndʑi} &\N  &\N \\
 &\textsc{imp}-V_1-V_2-\textsc{du} \\
\bottomrule
\end{tabular}
\end{table}

In type A forms, the two verb stems are not phonologically integrated, and each of them takes both prefixes and suffixes. This form is a type of lexicalized serial verb construction (cf section \ref{sec:serial}). Note that  no word  can be inserted between \ipa{stu} and \ipa{mbat} in the corpus, unlike other examples of SVC in Japhug. With \jpg{fse=raŋ}{happen so many things}, it is possible to repeat the subject, as in example \refb{ex:fse.rang}.

\begin{exe}
\ex \label{ex:fse.rang}
\gll 
\ipa{nɯra} \ipa{pɯ-fse}, \ipa{nɯra} \ipa{pɯ-raŋ} \\
\textsc{dem:pl} \textsc{pst.ipfv}-be.like \textsc{dem:pl} \textsc{pst.ipfv}-last.a.long.time \\ 
\glt `All these things happened.' (many attestations of this sentence, occurs in traditional stories typically to avoid repeating sentences when a characters tells another character what has happened previously)
\end{exe}

In type B and C forms, the two conjugated verb forms merge phonologically, and either the suffixe(s) of the first verb (in the case of the right-dominant B type, where the V_1 has a reduced form and the V_2 the full form) or the prefixe(s) of the second one (left-dominant) are removed. Prefixal or suffixal chains are either preserved or completely removed; it is not possible to have intermediate forms. For instance, the type B form in example \refb{ex:atAtWstu} cannot be changed to something like $\dagger$\ipa{a-tɤ-tɯ-stu=tɯ-mbat-ndʑi} with only the second person \ipa{tɯ-} prefix without the irrealis prefixes.

\begin{exe}
\ex \label{ex:atAtWstu}
\gll \ipa{a-tɤ-tɯ-stu=a-tɤ-tɯ-mbat-ndʑi} \\
\textsc{irr-pfv-2}-try.hard(1)=\textsc{irr-pfv-2}-try.hard(2)-\textsc{du} \\
\glt `(while I am gone),  may the two of you do your best.' (Smanmi0, 56)
\end{exe}

In type D forms, the two verbs come to share the same prefixal and suffixal chain, with no intervening affix between the two verb stems.

Table \ref{tab:bipartite} summarizes all bipartite verbs discovered up to now in Japhug. Note the important proportion of Tibetan loanwords (\ipa{=raŋ}, \ipa{zdɯɣ=sŋɤl}, \ipa{ntsʰɤβ=}, \ipa{rga=}, respectively from \ipa{riŋ}, \ipa{sdug.bsŋal}, \ipa{ⁿtsʰab}, \ipa{dga}).

The stative bipartite verbs \ipa{zdɯɣ=sŋɤl}, \ipa{rga=le} and \ipa{rga=χi} are only attested in A type  non-finite forms. 

It is difficult to apply the rich array of voice derivations available in Japhug to bipartite verbs. The only unproblematic derivation I could elicit is the tropative\footnote{Tropative is a voice deriving a transitive verb meaning `find, consider X to be Y' from a stative verb Y, see \citet{jacques13tropative}.} verb \jpg{nɤ-stu-mbat}{consider that X tries hard}, which is derived from the type D form.

\begin{table}[h]
\caption{Bipartite verbs in Japhug} \label{tab:bipartite} \centering
\begin{tabular}{lllllllllll}
\toprule
Compound verb& Meaning	 & 	A & 	B & 	C & 	D & \\
\midrule
\ipa{stu=mbat} & 	try hard & 	\Y & 	\Y & 	\Y & 	\Y & 	\\	
\ipa{mu=cɯɣ} & 	be  terrified  & 	\Y & 	 & 	 & 	 & 	\\	
\ipa{χɕu=rnaʁ} & 	thank a lot & 	\Y & 	\Y & 	 & 	 & 	\\	
\ipa{ntsʰɤβ=rlu} & 	in a hurry & 	 & 	\Y & 	 & 	\Y & 	\\	
\ipa{fse=raŋ} & 	happen so many things & 	\Y & 	 & 	 & 	 & 	\\	
\ipa{kʰrɯ=jɤβ} & 	be extremely dry & 	\Y & 	 & 	 & 	\Y & 	\\	
\ipa{zdɯɣ=sŋɤl} & 	suffer extremely & 	\Y? & 	 & 	 & 	\Y & 	\\	
\ipa{rga=le} & 	be extremely happy & 	\Y? & 	\Y & 	 & 	 & 	\\	
\ipa{rga=χi} & 	be extremely happy & 	\Y? & 	\Y & 	 & 	 & 	\\	
\midrule
\ipa{spa=rka tu/me} & 	be guilty/innocent & 	\Y & 	 & 	 & 	\Y & 	\\	
\bottomrule
\end{tabular}
\end{table}

There is only one bipartite transitive verb, \ipa{spa=rka}, which is actually even tripartite, since its always occur with an existential verb -- it means `be guilty' when used with the positive existential verb \jpg{tu}{exist} and  `be innocent' with the negative one \jpg{me}{not exist} (which cannot take person/number indexation in this construction).\footnote{The etymology of the elements \ipa{spa=} and \ipa{=rka} still eludes the author of this paper.} Although morphologically transitive (see for instance the presence of stem III \ipa{a} $\rightarrow$ \ipa{e} alternation in \textsc{sg}$\rightarrow$3 non-past forms, in examples \ref{ex:mAspea} to \ref{ex:mAtWspe2}), this verb has an expletive object which cannot be overt. Examples \refb{ex:mAspea}  and \refb{ex:mAspanW.mArkAnW} illustrate the A form (the second one while the existential verb in nominalized form, in a relative clause), and \refb{ex:mAtWsparkandZi} and \refb{ex:mAtWspe2} the D form. 

\begin{exe}
\ex \label{ex:mAspea}
\gll \ipa{mɤ-spe-a} \ipa{mɤ-rke-a} \ipa{me} \\
\textsc{neg}-be.innocent(1)[III]:\textsc{fact}-\textsc{1sg} \textsc{neg}-be.innocent(2)[III]:\textsc{fact}-\textsc{1sg} not.exist:\textsc{fact} \\
\glt `I am innocent.'
\end{exe} 

\begin{exe}
\ex \label{ex:mAspanW.mArkAnW}
\gll
\ipa{mkʰɤrmaŋ} 	[\ipa{mɤ-spa-nɯ} 	\ipa{mɤ-rka-nɯ}] 	\ipa{kɯ-me} 	\ipa{nɯ} 	\ipa{ʑɯmkʰɤm} 	\ipa{pjɤ-sɯ-sat} 	\ipa{pjɤ-ra} 	\ipa{rcanɯ,} 	\ipa{tɯrme} 	\ipa{ra} 	\ipa{kɯ} 	\ipa{wuma} 	\ipa{ʑo} 	\ipa{pjɤ-qʰa-nɯ} \\
people \textsc{neg}-be.innocent(1):\textsc{fact-pl} \textsc{neg}-be.innocent(2):\textsc{fact-pl} \textsc{nmlz}:S/A-not.exist \textsc{dem} a.lot \textsc{ifr-caus}-kill \textsc{ifr.ipfv}-have.to \textsc{unexpected} man \textsc{pl} \textsc{erg} really \textsc{emph} \textsc{ifr}-hate-\textsc{pl} \\
 \glt `As he had to have many innocent people killed, people hated him.' (hist140514 xiee de shewang, 82)
\end{exe} 


\begin{exe}
\ex \label{ex:mAtWspe1}
\gll \ipa{mɤ-tɯ-spe} \ipa{mɤ-tɯ-rke} \ipa{me} \\
\textsc{neg}-2-be.innocent(1)[III]:\textsc{fact} \textsc{neg}-2-be.innocent(2)[III]:\textsc{fact} not.exist:\textsc{fact} \\
\glt `You_{sg} are innocent.'
\end{exe} 


\begin{exe}
\ex \label{ex:mAtWsparkandZi}
\gll \ipa{mɤ-tɯ-spa=rka-ndʑi} \ipa{me} \\
\textsc{neg}-2-be.innocent(1)-be.innocent(2):\textsc{fact}-\textsc{du} not.exist:\textsc{fact} \\
\glt `You_{du} are innocent.'
\end{exe} 

Example \refb{ex:mAtWspe2}, where both verb roots \ipa{spa} and \ipa{rka} are in stem III form (\ipa{spe} and \ipa{rke} respectively) reveals two facts about type D forms. First, stem III alternation (marking singular A and third person object in non-past TAM forms), although historically partially suffixal in origin (see \citealt[357]{jacques04these}), is synchronically distinct from the suffixal chain. Second, although type D forms could appear at first glance to be simple compound verbs, the presence of stem alternation on both roots (instead of having $\dagger$\ipa{mɤ-tɯ-spa-rke} with alternation on the second root only) shows that these two roots are still morphologically active.

\begin{exe}
\ex \label{ex:mAtWspe2}
\gll \ipa{mɤ-tɯ-spe=rke} \ipa{me} \\
\textsc{neg}-2-be.innocent(1)[III]-be.innocent(2)[III]:\textsc{fact} not.exist:\textsc{fact} \\
\glt `You_{sg} are innocent.'
\end{exe} 

\section{Compound verbs in Japhug} \label{sec:compounds}
In addition to bipartite verbs, we find in Japhug another class of verbs combining two verb roots into one lexeme: compound verbs.

Unlike bipartite verbs,  compound verbs do not allow multiple indexation (such as inflectional morphemes intervening between the two roots) and multiple stem alternation (as in example \ref{ex:mAtWspe2} above) and the first member of the compound undergoes \textit{status constructus} vowel change (see  \citealt[1215]{jacques12incorp}, \ipa{a/u} $\rightarrow$ \ipa{ɤ} in examples 1 and 4 below). An additional derivational prefix is often present (\ipa{a-} and \ipa{rɤ-/a-} in examples 1, 2 and 3 below).

\begin{enumerate}
\item \jpg{pa}{do} + \jpg{mbat}{be easy} $\Rightarrow$ \jpg{a-pɤ-mbat}{easy to do}
\item \jpg{χtɯ}{buy} + \jpg{ntsɣe}{sell} $\Rightarrow$ \jpg{ra-χtɯ-tsɣe}{do commerce (\zh{做买卖})} 
\item \jpg{joʁ}{raise} + \jpg{βzɯr}{move}   $\Rightarrow$  \jpg{rɤ-joʁ-βzɯr}{tidy up}
\item \jpg{ɕe}{go} + \jpg{ɣi}{come} $\Rightarrow$ \jpg{nɤ-ɕej-ɣi}{come and go, circulate} 
\item \jpg{ngu}{feed} + \jpg{jtsʰi}{give to drink} $\Rightarrow$ \jpg{ngɤ-jtsʰi}{feed}
\end{enumerate}

The first four examples are actually historically denominal verbs, derived from action noun compounds made of two verb roots, in a way similar to incorporating verbs, which originate from the denominal derivation of noun-verb compounds (\citealt{jacques12incorp}). The noun compound \jpg{joʁ-βzɯr}{tidying up} from which \jpg{rɤ-joʁ-βzɯr}{tidy up} is derived is still attested, and can be used in a light verb construction:

\begin{exe}
\ex 
 \gll \ipa{joʁβzɯr} \ipa{tɤ-βzu-t-a} \\
 tidying.up \textsc{pfv}-do-\textsc{pst:tr-1sg} \\
 \glt `I did some tidying up.'
\end{exe}

The other action nominals *\ipa{ɕejɣi} `coming and going', *\ipa{pɤ-mbat} `action that is easy to do' and ?\ipa{χtɯ-tsɣe} `commerce' have not been recorded, but it is unproblematic to assume that they used to exist and that the corresponding verbs derive from them.

Note the antipassive-like value of the derivation in examples 2 and 3, where the compound verb derived from two transitive verbs is intransitive. Note that the denominal prefix \ipa{rɤ-} is the same from which the antipassive derivation was grammaticalized (see  \citealt{jacques14antipassive}). 

The verbs \jpg{ɕe}{go} and \jpg{ɣi}{come} are among a handful of irregular verbs in Japhug which preserve stem II alternation in the perfective (see for instance \citealt[267]{jacques14linking}; in other Gyalrongic languages stem II is much more widespread, see in particular \citealt{jackson00sidaba, jackson04showu, linyj03tense, zhang16bragdbar, lai17khroskyabs}): their stem II are \ipa{ari} and \ipa{ɣe} respectively. The compound verb \jpg{nɤɕejɣi}{come and go, circulate} based from these motion verbs however, does not have stem II alternation (its perfective form is marked exclusively by orientation prefixes), although one could have expected a form such as $\dagger$\ipa{nɤriɣe}.

Example 5 \jpg{ngɤ-jtsʰi}{feed} differs from the preceding one by having no denominal prefix. Such compound verbs are a minority (only four are attested, see Table \ref{tab:compound.verbs}), but all start in a cluster whose first element is a nasal. It is thus possible that a reduced form of the denominal prefix (as in \jpg{ngo}{be sick} from denominal \ipa{n(ɯ)}+\jpg{--ŋgo}{disease}) has been absorbed by the verb stem.\footnote{For examples of such a development in Khroskyabs, see \citet{jacques12incorp} and \citealt{lai13affixale}. } Nevertheless, since some of these compound verbs do not have visible \textit{status constructus} alternation (since for instance \ipa{ɯ} does not alternate), some of them are not synchronically distinguishable from type D bipartite verbs.

\begin{table}[h]
\caption{Compound verbs in Japhug} \label{tab:compound.verbs} 
 \resizebox{\columnwidth}{!}{
\begin{tabular}{lllllllll}
\toprule
Compound verb &&$V_1$ & & $V_2$ \\
\midrule
\ipa{andʑɤmstu} (vi) &	well-ironed &	\ipa{ndʑɤm} (vi) & be warm &	\ipa{astu} (vi)&	be straight &	\\	
\ipa{argɤle} (vi) &	be extremely happy &	\ipa{rga} (vi) &	be happy&	\ipa{le} &	-- &	\\	
\ipa{apɤmbat} (vi) &	be easy to do &	\ipa{pa} (vt)&	do &	\ipa{mbat} (vi)&	be easy &	\\	
\ipa{arɟumtɕɤr} (vi) &	having uneven wideness &	\ipa{rɟum} (vi)&	wide &	\ipa{tɕɤr} (vi)&	narrow &	\\	
\ipa{nɤrtoχpjɤt} (vt) &	observe &	\ipa{rtoʁ} (vt)&	look, watch &	\ipa{χpjɤt} (vt)&	observe &	\\	
\ipa{nɤscɤlɤt}  (vt)&	take to somewhere &	\ipa{sco} (vt) &	see off &	\ipa{lɤt}  (vt) &	get so. back home &	\\	
&and back home \\
\ipa{nɤtsɯmɣɯt}  (vt) &	take away and  &	\ipa{tsɯm} (vt) &	take away &	\ipa{ɣɯt} (vt)&	bring &	\\	
&bring back\\
\ipa{nɯndzɤmbɣom}  (vi)&	in a hurry to eat &	\ipa{ndza} (vt)&	eat &	\ipa{mbɣom} (vi)&	be in a hurry &	\\	
\ipa{nɯndzɤqɤr}  (vt) &	not let eat together &	\ipa{ndza} (vt)&	eat &	\ipa{qɤr} (vt)&	choose &	\\	
\ipa{nɯrkorlɯt}  (vi) &	be obstinate &	\ipa{rko} (vi)&	be hard &	\ipa{arlɯt} (vi)&	be many &	\\	
\ipa{nɯrŋgɯmbri}   (vi)&	make noise in the bed &	\ipa{rŋgɯ} (vi)&	lie down &	\ipa{mbri} (vi)&	make noise &	\\	
\ipa{rɤjoʁβzɯr}   (vt)&	tidy up &	\ipa{joʁ} (vt)&	raise &	\ipa{βzɯr} (vt)&	move &	\\	
\midrule
\ipa{mpɯmnu}   (vi)&	soft and smooth &	\ipa{mpɯ} (vi)&	be soft &	\ipa{mnu} (vi)&be	smooth &	\\	
\ipa{ngɤjtsʰi}   (vt)&	feed &	\ipa{ngu} (vt)&	feed &	\ipa{jtsʰi} (vt)&	give to drink &	\\	
\ipa{mbijtsʰi}   (vt)&	go to eat and drink &	\ipa{mbi} (vt)&	give &	\ipa{jtsʰi} (vt)&	give to drink &	\\	
\ipa{mtsɯrɕpaʁ}   (vi)&	be hungry and thirsty &	\ipa{mtsɯr} (vi)&	be hungry &	\ipa{ɕpaʁ} (vi)&	be thirsty &	\\	
\bottomrule
\end{tabular}}
\end{table}

Most compound verbs in Japhug are \textit{dvandva}-like (`do $V_1$ and $V_2$')  but in a few examples, like \jpg{a-pɤ-mbat}{easy to do} or \jpg{nɯ-ndzɤ-mbɣom}{be in a hurry to eat}, the semantic relationship between $V_1$  and $V_2$ is the same as that of a main verb and its complement (`$V_2$ to $V_1$').

The verb \jpg{a-rgɤ-le}{be extremely happy} is related to the bipartite verb \ipa{rga=le} mentioned in Table \ref{tab:bipartite}, showing that bipartite verbs too can be converted into compound verb verbs by means of denominal derivation.

\section{Serial verb constructions in Japhug} \label{sec:serial}
The A form of bipartite verbs is formally similar to serial verb constructions (\citealt{sun12complementation}, \citealt{jacques13harmonization}, \citealt{jacques16complementation}). In these constructions, both verbs share the same arguments, transitivity, TAM, polarity and associated motion markers. Unlike serial constructions in other languages, various linkers and emphatic markers are possible between the two verbs (examples \ref{ex:totChW2} and \ref{ex:kuWGstuanW}).

Serial verb constructions mainly occur with deideophonic verbs (example \ref{ex:totChW2}) and action deixis verbs (transitive \ref{ex:kuWGstuanW} or intransitive \ref{ex:ki.fsea}).

\begin{exe}
\ex \label{ex:totChW2}
\gll 	\ipa{srɯnmɯ} 	\ipa{nɯ} 	\ipa{to-nɯdrɯβ} 	\ipa{ʑo} 	 	\ipa{to-tɕʰɯ} \\
 râkshasî \textsc{dem}  \textsc{ifr}-repeatedly.gore  \textsc{emph}  \textsc{ifr}-gore \\
 \glt `(The rhinoceros) gored the râkshasî repeatedly and killed her.' 
\end{exe}	

\begin{exe}
\ex \label{ex:kuWGstuanW}
\gll 	
 \ipa{aʑo} 	\ipa{kɯki} 	\ipa{ntsɯ} 	\ipa{kú-wɣ-stu-a-nɯ} 	\ipa{tɕe,} 	\ipa{kú-wɣ-znɯkʰrɯm-a-nɯ} \\
 \textsc{1sg} \textsc{dem:prox} always \textsc{ipfv-inv}-do.like-\textsc{1sg-pl} \textsc{lnk} \textsc{ipfv-inv}-punish-\textsc{1sg-pl} \\
 \glt `They punished me like this.' (Gesar, 278)
\end{exe}	

\begin{exe}
\ex \label{ex:ki.fsea}
\gll \ipa{aʑo} 	\ipa{nɯ} 	\ipa{sŋiɕɤr} 	\ipa{ʑo} 	\ipa{kutɕu} 	\ipa{ki} 	\ipa{fse-a} 	\ipa{ndzur-a} 	\ipa{ntsɯ} 	\ipa{ɲɯ-ra} 	\ipa{tɕe,} \\
\textsc{1sg} \textsc{dem} night.and.day \textsc{emph} here \textsc{dem:prox} be.like:\textsc{fact-1sg} stand:\textsc{fact-1sg} always \textsc{sens}-have.to like \\
\glt `I have to stand like this night and day.' (The divination, 2002, 44)
\end{exe}

It is very likely that bipartite verbs originate from serial verb constructions,\footnote{Note that for instance associated motion prefixes are likely to have originated in a serial verb construction too (see \citealt{jacques13harmonization}).} especially since in the case of \jpg{mu=cɯɣ}{be  terrified}, \jpg{ntsʰɤβ=rlu}{in a hurry}, \jpg{kʰrɯ=jɤβ}{be extremely dry}, \jpg{rga=le}{be extremely happy} and \jpg{rga=χi}{be extremely happy}, the V_2 is clearly ideophonic-like. 

The only other possible construction from which bipartite verb could have originated are finite complements, as in example \refb{ex:tundzxi}, where the verb \ipa{tu-ndʐi} `it melts' is in a finite form inside the complement clause (in S function) of the verb 	\ipa{ɲɯ-cʰa} `it can'.

\begin{exe}
\ex \label{ex:tundzxi}
\gll 
\ipa{tɤjpa} 	\ipa{kɯ-xtɕɯ\tld{}xtɕi} 	\ipa{ka-lɤt} 	\ipa{ri,} 	\ipa{mɯ́j-ʁdɯɣ,} 	\ipa{pɤjkʰu} 	\ipa{tu-ndʐi} 	\ipa{ɲɯ-cʰa} \\
snow \textsc{inf:stat-emph}\tld{}be.small \textsc{pfv}:3$\rightarrow$3'-throw but \textsc{neg:sens}-be.serious still \textsc{ipfv}-melt \textsc{sens}-can \\
\glt `There was a little snow, but it doesn't matter, it can still melt.' (conversation, 2015/12/17)
\end{exe}

However, none of V_2 of bipartite verbs in Japhug are complement taking verbs, so that such hypothesis is highly unlikely. 

\section{The genesis of bipartite verbs} \label{sec:genesis}
Bipartite verbs in Japhug are a marginal class, still barely grammaticalized from a serial verb construction, and may thus provide a useful model to understand how bipartite verbs in other languages came into being. 

Figure \ref{fig:bipartite.pathways} presents a summary of all pathways suggested in this paper to account for the origin of bipartite verbs.

Most bipartite verbs originate from lexicalized serial verb constructions. The origin of these serial constructions is an issue in itself which is not discussed in the present paper, but either parataxis or finite complements (see example \ref{ex:tundzxi} above) are conceivable sources.


   \begin{figure}[H]
   \caption{Possible pathways leading to the creation of bipartite verbs} \label{fig:bipartite.pathways}  
  \begin{tikzpicture}
  \node (X1) at (-6,6) {Parataxis};
  \node (X2) at (4,6) {Finite complement+Matrix verb};
  \node (Y) at (-5,4) {\textbf{Serial verb construction}};
  \node (A) at (-4,2) {\textbf{Bipartite verb (type A)}};
  \node (A2) at (4,4) {Action noun compound (V+V)};
  \node (B) at (-5,-0.5) {\textbf{Bipartite verb (type B)}};
  \node (C) at (0,-0.5) {Bipartite verb (type C)};    
  \node (D) at (-1,-4) {\textbf{Bipartite verb (type D)}};    
  \node (E) at (4,-0.5) {\textbf{Compound verb}};        
\tikzstyle{peutetre}=[->,dotted,very thick,>=latex]
\tikzstyle{sur}=[->,very thick,>=latex]
\draw[sur] (Y)--(A);
\draw[sur] (A) to node[left] {\textit{\footnotesize Loss of V_1 suffixes}} (B);
\draw[sur] (A) to node[right] {\textit{\footnotesize Loss of V_2 prefixes}} (C);
\draw[sur] (B) to node[left] {\textit{\footnotesize Loss of V_2 prefixes}} (D);
\draw[sur] (C) to node[right] {\textit{\footnotesize Loss of V_1 suffixes}} (D);
\draw[sur] (A2) to node[right] {\textit{\footnotesize Denominal derivation}} (E);
\draw[peutetre] (X1)--(Y);
\draw[peutetre] (X2)--(Y);
\draw[peutetre] (X1)--(A2);
\draw[peutetre] (X2) to (A2);
\draw[peutetre] (E) to[bend left] node[right] {\textit{\footnotesize Absorption of denominal prefix}} (D) ;
\end{tikzpicture}
\end{figure}

In type A bipartite verbs, the two verbs, though still phonologically two distinct words, are more closely attached to one another than in common serial verb constructions: insertion of linkers is difficult, and the relative order of the two verbs cannot be changed.

Type B and C are intermediate stages from the quasi-serial verb construction (type A) to the quasi-verb compounds (type D). Rather than directly fusing into a phonological word, the two verbs first share either their suffixes or their prefixes; note that the first option is by far the most common. Type C is only attested with the verb \jpg{stu=mbat}{try hard, do one's best}.

Type D verbs are almost identical to compound verbs; the only differences are (a)  absence of \textit{status constructus} vowel alternation on the V_1 (b) presence of stem alternation on the V_1 (c) absence of denominal derivation prefix. Since (a) and (c) are also found on some compound verbs, and since (b) only applies to transitive verbs, some verbs can actually be synchronically analyzed as either bipartite or compound verb.

Thus, in a language whose bipartite verbs are more ancient, and whose grammaticalization process is less transparent than in Japhug, several competing pathways should be postulated and taken as possibilities.

\section{Bipartite verbs in Kiranti and beyond} \label{sec:kiranti.bipart}
Unlike Gyalrongic languages, where bipartite verbs are a poorly documented and marginal category, Kiranti languages are well-known for having a class of complex predicates combining a lexical verb with an auxiliary, which share some of their person indexation and TAM markers. 

In Kiranti, the V_2 derive from auxiliaries which are sometimes still attested as independent verbs in the languages, and are used to express voice (in particular applicative or causative), aspect or associated motion (see \citealt[283-328]{schackow15yakkha} for a detailed overview of bipartite verbs in Yakkha).

In this section, I discuss of the commonalities and differences between bipartite verbs in Kiranti (taking Khaling and Bantawa as representative examples of the typological variety found in Kiranti) and those of Japhug, and show that some typological properties of Kiranti languages, such as multiple exponence and variability in affix ordering, make sense by supposing that they originate from a serial verb construction and have gone through a stage close to the type C bipartite verbs in Japhug.

 Then, I present arguments showing that the TAM *\ipa{-t-} suffix found in most Kiranti languages was originally grammaticalized from a V_2, and that with time, bipartite verbs may end up being reanalyzed as a single verb when either the V_2 (as in this case) or the V_1 becomes fully grammaticalized as an affix.

\subsection{Bipartite verbs and prefix ordering}
In some Kiranti languages, such as Khaling, bipartite verbs are fully integrated into a single phonological word.  Example (\ref{ex:ityodzyoyi}) from Khaling shows the verb root \dhatu{tA}{put}\footnote{See \citet{jacques12khaling} for an account of stem alternations in Khaling.} combined with the auxiliary \dhatu{-dzA}{keep on} grammaticalized from the verb \dhatu{dzA}{eat}. The verb and the auxiliary share the second person prefix \ipa{ʔi-}, but the suffixes remain distinct: the auxiliary receives the full form of the suffix \ipa{-ji}, while the lexical verb has a reduced form \ipa{-j-}: dual number is thus redundantly indexed two times in this form.

\begin{exe}
\ex \label{ex:ityodzyoyi}
\gll \ipa{ʔi-tɵ-j-dzɵ-ji}\\
2/\textsc{inv}-put-\textsc{du.incl}-keep.on-\textsc{du.incl} \\
\glt You_{du} keep on putting it.
\end{exe}

This form is intermediate between types C and D: both verb stems share the same prefixal slot, but the first verb has a phonologically reduced suffixal slot, which cannot be a distinct syllable.

In other Kiranti languages such as Bantawa, bipartite verbs are phonologically less closely integrated. As in Khaling, prefixes only appear once in the bipartite verb; some suffixes are obligatorily redundantly replicated in bipartite verb forms, such as the 1/2 plural \ipa{-in}, while other ones such as the exclusive \ipa{-ka} are not, as illustrated by \refb{ex:kharin}. 

\begin{exe}
\ex \label{ex:timannin}
\gll
\ipa{tɨ-man-nin} \ipa{kʰan-nin} \\
2-lose-\textsc{1ns$\leftrightarrow$2} send.away-\textsc{1ns$\leftrightarrow$2} \\
\glt ‘You_s have forgotten us_{pe}’ (\citealt[168]{doornenbal09})
\end{exe}

 \begin{exe}
\ex  \label{ex:kharin}
\gll \ipa{kʰar-in} \ipa{lont-in-ka} \\
go-\textsc{1/2pl} come.out-\textsc{1/2pl-excl} \\
\glt `We shall rise again.' (\citealt[254]{doornenbal09})
\end{exe}

We see that bipartite verbs in Bantawa differs from Japhug type C in that only \textit{part} of the suffixal chain is replicated. Interestingly, while we find several type-C-like bipartite verbs in Kiranti, examples of type-B are not attested as far as I have been able to ascertain from the available data.

The Japhug model may also help to historically interpret the origin of \textit{variable prefix ordering}  in Kiranti languages such as Chintang (\citealt{bickel07chintang}) and Bantawa (\citealt[170-172]{doornenbal09}), as illustrated by the following examples :

\begin{exe}
\ex \label{ex:ukosagonde}
\gll \ipa{u-kos-a-gond-e} \\
\textsc{3ns.S/A}-walk-\textsc{pst-amb-pst} \\
\ex \label{ex:kosaugonde}
\gll \ipa{kos-a-u-gond-e} \\
walk-\textsc{pst-3ns.S/A-amb-pst} \\
\glt ‘They walked around.’ (\citealt[51]{bickel07chintang})
\end{exe}

Such affix variability has no equivalent in Japhug (and is not found everywhere even in Kiranti; for instance Khaling has a fixed verbal template). To explain the variable placement of the \ipa{u-} person prefix, I propose that Chintang has undergone  fusion from a serial verb construction such as \refb{ex:pchintang};\footnote{This is a schematic representation of the structure of the proto-form; a real reconstruction is not yet possible at this stage of our knowledge of Kiranti historical morphology; in particular, I make no attempt to explain the origin of the \ipa{a/e} alternation of the \textsc{pst} suffix. } at that stage, what became later the ambulative suffix was still a motion verb, inflected like a normal intransitive verb.

\begin{exe}
\ex \label{ex:pchintang}
\gll *\ipa{u-kos-e} \ipa{u-gond-e} \\
\textsc{3ns.S/A}-walk-\textsc{pst} \textsc{3ns.S/A}-walk.around-\textsc{pst} \\
\end{exe}


The integration of the two verb forms into a bipartite verb, following the general pattern in Kiranti, results in only one of the prefixes being kept in the fused form. In example \ref{ex:ukosagonde}, it is the prefix of the V_2 that is lost, as in the type C conjugation found in Japhug. However, Chintang (and Bantawa), unlike Japhug, had the option of preserving the prefix of the V_2 (trapped between the two verb stems) rather than that of the V_1, resulting in form \ref{ex:kosaugonde}; I will refer to such forms as C'.

Note that it is not possible to account for C' forms as being grammaticalized from the fusion of a non-finite verb form with a fully conjugated light verb, since the V_1 still keeps part of the TAM suffixes and person indexes.

%Given the intricate complexity of variable prefix ordering in Chintang and Bantawa, this hypothesis needs to be tested against the whole corpus of verb forms in these languages, but this research goes beyond the topic of the present paper.


\subsection{The past tense prefix} \label{sec:pst}
A TAM suffix originating from a form *\ipa{-tV-} is found in various Kiranti languages, including including Western Kiranti (Thulung, \citealt{lahaussois03}), North-East Kiranti (Khaling, Dumi and Koyi, cf \citealt{driem93dumi, lahaussois09, jacques12khaling}), Southern Kiranti (Bantawa, \citealt{doornenbal09}) and Eastern Kiranti (Belhare, \citealt[551]{bickel03belhare}).

It marks past tense in most languages, but is used for the non-past in Dumi, Koyi and Belhare, a puzzling fact that has remained unexplained.

In most Kiranti languages, in particular Dumi and Khaling, the \ipa{-t-} suffix has a fixed place in the verbal template. 

However, in Bantawa we observe that the second person prefix has variable placement and can occur either before the verb stem, or between the verb stem and the past tense suffix (\citealt[171]{doornenbal09}).

\begin{exe}
\ex \label{ex:timankhatda}
\gll 
\ipa{tɨ-man-kʰat-da} \\
2-\textsc{neg}-go-\textsc{pst} \\
\ex
\gll 
\ipa{man-tɨ-kʰat-da} \\
\textsc{neg}-2-go-\textsc{pst} \\
\ex 
\gll 
\ipa{man-kʰat-tɨ-da} \\
\textsc{neg}-go-2-\textsc{pst} \\
\glt `You did not go.'
\end{exe}

In view of the previous discussion, this suggest that the past suffix originates from a V_2 in a serial construction, and that a type A proto-form such as *\ipa{tə-kʰat=tə-ta} `He went'\footnote{Bantawa \ipa{d} comes from proto-Kiranti *\ipa{t} in stressed syllables (\citealt{michailovsky94stops}); this sound law does not apply to prefixes (\citealt{jacques12agreement}).} either changed to type C *\ipa{tə-kʰat-ta} or to type C' *\ipa{kʰat-tə-ta}, from which in turn the attested Bantawa verb forms directly originate. Indeed, \citet[165;272]{doornenbal09} treats the past tense suffix as derived from the verb \dhatu{da}{effect, do, finalize}.

Even in Khaling and Dumi, where affix order is more rigid, we find  an additional clue that the TAM suffix comes from a V_2: person indexation is redundantly indicated after the the \ipa{-t-} suffix, as illustrated by the following examples:

\begin{exe}
\ex \label{ex:dzungta}
\gll \ipa{dzû-ŋ-t-ʌ} \\
eat:\textsc{1sg}$\rightarrow$3-\textsc{1sg-pst-1sg} \\
\glt `I ate it'
\ex \label{ex:dzyoiti}
\gll \ipa{dzɵ̂-j-t-i} \\
eat-\textsc{du-pst-du} \\
\glt `We_{di}/They_d ate it'
\ex \label{ex:dzyoktiki}
\gll \ipa{dzɵ-k-t-iki} \\
eat-\textsc{1p-pst-1pi} \\
\glt `We_{pi} ate it'
\ex \label{ex:dzyutya}
\gll \ipa{dzʉ-t-ɛ} \\
eat:\textsc{2/3sg$\rightarrow$3}-\textsc{pst-2/3} \\
\glt `He ate it'
\end{exe}

The reduced person suffix found in \refb{ex:dzyoiti} and \refb{ex:dzyoktiki} between the verb stem and the past tense suffix is exactly the same as that found in complex predicates (compare with \refb{ex:ityodzyoyi} above). 

The hypothesis that the TAM suffix *\ipa{-t-} did originate from a V_2 is further supported by the existence of a verb root which could be the source of this suffix:the verb *\ipa{ta}, whose original meaning must have been `put' (Khaling \dhatu{tA}{put}, Limbu \dhatu{tha}{put away})\footnote{Cognates of this root are found elsewhere in the Trans-Himalayan family, e.g. Japhug \ipa{ta} `put'.}.  In Khaling, the forms of the person indexation suffix  following \ipa{-t-} partially resemble, but are not identical to, the non-past forms of the verb \dhatu{ta}{put}, as shown in Table \ref{tab:tANA}; the first person past suffix \ipa{-tʌ} is identical to the stem of the \textsc{1sg$\rightarrow$3}  \ipa{tʌ-ŋʌ} `I put it' without the suffix \ipa{-ŋʌ}, and the third/second person singular suffix \ipa{-tɛ} is identical to the stem of the and \textsc{1sg$\rightarrow$3} \textsc{3sg$\rightarrow$3}  \ipa{tɛ} `he puts it'. The form \ipa{-tʌ} may actually be the regular phonetic Khaling outcome of proto-Kiranti \ipa{*-taŋ} in unstressed syllables (compare \ipa{bhâ:tʌ} `shoulder' with Bantawa \ipa{bʰaktaŋ} from \ipa{*pakdaŋ}).\footnote{The Khaling form is probably borrowed from a South Kiranti language, as we would expect $\dagger$\ipa{pâ:dʌ} in Khaling. This borrowing would have taken place after the Lautverschiebung (*\ipa{b}/\ipa{d} $\rightarrow$ \ipa{p}/\ipa{t}; *\ipa{p}/\ipa{t} $\rightarrow$ \ipa{bʰ}/\ipa{dʰ}, see \citealt{michailovsky94stops} and \citealt{jacques17pkiranti}) in the donor language, but before the change \ipa{*-taŋ} > \ipa{-tʌ}.}


 The first person dual and plural forms cannot be straightforwardly explained at this stage; one possibility is that \ipa{-ti} is the phonetic outcome of proto-Kiranti *\ipa{ta-si} (the probably original form of the \textsc{1di}, see \citealt[58]{jacques16tonogenesis}), while the \textsc{1di} of  put' \ipa{tɵ-ji} was remade by analogy with forms such as the infinitive \ipa{tɵ-nɛ}.

\begin{table}[H]
\caption{Non-Past paradigm of the verb \dhatu{ta}{put} in Khaling and corresponding suffixes in the Past paradigms} \centering \label{tab:tANA}
\begin{tabular}{cccc}
\toprule
S/A & \dhatu{ta}{put} & Past Suffix + Person Indexation Suffix \\
\midrule
\textsc{1sg} & \ipa{tʌ-ŋʌ} &  \ipa{-tʌ}  \\
\textsc{1di} & \ipa{tɵ-ji} &  \ipa{-ti}  \\
\textsc{1pi} & \ipa{tɵ-ki} & \ipa{-tiki}  \\
\textsc{3sg} & \ipa{ʔi-tɛ} & \ipa{-tɛ}  \\
\textsc{3sg} & \ipa{tɛ} & \ipa{-tɛ}  \\
\bottomrule
\end{tabular}
\end{table}


The development from a verb `put' to completive aspect is amply attested (\citealt[58]{bybee94TAM}, \citealt[248]{heine-kuteva02}) and could account from the grammaticalization of this verb as a past tense marker. Another possibility,\footnote{This idea was suggested to me by XXX. %Scott Delancey.
} is that `put' was first grammaticalized as a perfect marker, from which it became non-past in some languages (Dumi and Koyi, see \citealt{driem93dumi}, \citealt{lahaussois09}) and past in other ones, while the unmarked verb form became specialized for the opposite category (past vs non-past respectively).

Given the wide distribution of this suffix, appearing in all identified subgroups of Kiranti, its absence in some languages (such as Limbu, \citealt{driem87}) is better explained by the hypothesis that they lost it rather than never developed it, and it is thus a potential candidate for a common Kiranti innovation.

\subsection{Other languages} \label{sec:rawang.bipart}
In Trans-Himalayan, bipartite verbs are not restricted to Gyalrongic and Kiranti, but data is still lacking for a full overview of the extent of this phenomenon in the family.

Dulong/Rawang languages share with Gyalrongic and Kiranti many typological features, including polypersonal indexation and transitivity marking (\citealt{lapolla11transitivity}), a rich voice system (\citealt{lapolla01valency}) including antipassive derivations (\citealt{jacques18antipass}). Unsurprisingly, bipartite verbs similar to those found in the two previously studied branches are also attested in this group.

\citet{straub17compounds}, building on data from \citet{lapolla15rawang} and the Rawang bible (Shànshér Kàru), shows that several bipartite verb constructions are attested in Rawang. This type of construction is not as widespread as in Kiranti, and is limited to a restricted number of lexical entries, as in Japhug; possible construction types depend on each particular verb. Some verbs allow type B (\ref{ex:medWngmemannWng}) and D (\ref{ex:medWngmannWng}) patterns (from the verb \ipa{dɯŋ=man} `be truthful'; some informants also marginally accept type A (\ref{ex:medWngnWngmemannWng}).

\begin{exe}
\ex \label{ex:medWngmemannWng}
 \gll \ipa{m-è-dɯŋ} \ipa{m-è-man-nɯ̀ŋ} \\
\textsc{neg-2/inv}-be.reliable	\textsc{neg-2/inv}-be.prophetic-\textsc{2pl} \\
\glt `You are untruthful.'
\end{exe}

\begin{exe}
\ex \label{ex:medWngmannWng}
 \gll \ipa{m-è-dɯŋ-man-nɯ̀ŋ} \\
\textsc{neg-2/inv}-be.reliable-be.prophetic-\textsc{2pl} \\
\glt `You are untruthful.'
\end{exe}

\begin{exe}
\ex \label{ex:medWngnWngmemannWng}
 \gll \ipa{m-è-dɯŋ-nɯ̀ŋ} \ipa{m-è-man-nɯ̀ŋ} \\
\textsc{neg-2/inv}-be.reliable-\textsc{2pl}	\textsc{neg-2/inv}-be.prophetic-\textsc{2pl} \\
\glt `You are untruthful.'
\end{exe}

From the discussion in section \ref{sec:genesis} above, the A type (\ref{ex:medWngnWngmemannWng}) should be the older construction from which type B (\ref{ex:medWngmemannWng}) and eventually type D (\ref{ex:medWngmannWng}) are derived, as in Japhug.

Other verbs such as  \ipa{ʃàn=ʃér} 'be holy' only appear in type B form (\ref{ex:shan.sher}); Straub notes that informant refuse alternative type forms corresponding to type A, C and D patterns. This again is not unexpected in view of the verb-specific restrictions found in Table \ref{tab:bipartite} above in Japhug.

\begin{exe}
\ex \label{ex:shan.sher}
 \gll \ipa{d-è-ʃàn}	 \ipa{d-è-ʃér-ʃì-nɯ̀ŋ}. \\
\textsc{caus-2/inv}-be.pure	\textsc{caus-2/inv}-be.clean-\textsc{refl}-\textsc{2pl} \\
\glt `Purify yourselves.'
\end{exe}

Example (\ref{ex:shan.sher}) also illustrates a very conspicuous difference between Rawang bipartite verbs and those of Gyalrongic and Kiranti: the causative derivation in \ipa{d-} can be applied to the verb complex, and unlike Japhug where the tropative \ipa{nɤ-} derivation imposes type D form, in Rawang the causative is repeated on both verbal elements and compatible with B form. This difference is probably linked to the place of the causative \ipa{d-}  in the verbal template of Rawang, as it lies further from the verb stem than the person indexation prefix \ipa{e-} (marking second person and/or inverse configuration), while person indexation prefixes in Japhug are always further from the verb stem than derivation prefixes.

%m-è-ʃàn	m-è-ʃér=(é)!
%NEG-N1-pure	NEG-N1-clean=EXCL
%
%questionable:
%m-è-ʃàn-ʃér=(é)!
%NEG-N1-pure-clean=EXCL

 
%A
%*d-è-ʃàn-ʃì-nɯ̀ŋ	d-è-ʃér-ʃì-nɯ̀ŋ
%CAUS-N1-pure-R/M-2PL	CAUS-N1-clean-R/M-2PL
%
%D*d-è-ʃàn-ʃér-ʃì-nɯ̀ŋ
%CAUS-N1-pure-R/M-2P}
\section*{Conclusion}
In Japhug, the nascent bipartite verbs are historically transparent, and derive from serial verb constructions rather than from compound verbs, though the final outcome of bipartite verbs (type D) is formally very close to a compound verb.

In Kiranti, bipartite verbs are older, though they also derive from serial verb constructions, a grammaticalization process which left traces such as multiple exponence (redundant person indexation) and free prefix ordering. The TAM suffix \ipa{-t-}, found in all branches of Kiranti, is argued to have been grammaticalized from a V_2, and thus that that all verb forms with this suffix were originally bipartite verbs. 

Bipartite verbs are a fragile phenomenon, which can come into existence only when specific conditions are met (languages with rich morphology \textsc{and} serial verb constructions), but which can easily disappear, either by being progressively absorbed into a class of compound verbs as in Japhug, or when either the V_2 (as in Kiranti) or the V_1 is reanalyzed as an affix.

The important typological differences between Japhug, Kiranti and Rawang, both in terms of preferred bipartite pattern (A, B, C or D) and compatibility with voice derivations, show that this phenomenon on the border of syntax and morphology requires a fined-grained typology that can only be built on the basis of rich fieldwork data and openly accessible text corpora.

\bibliographystyle{unified}
\bibliography{bibliogj}

 \end{document}
 