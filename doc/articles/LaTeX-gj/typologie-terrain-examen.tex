\documentclass{article} 
\usepackage{fontspec}
\usepackage{natbib}
\usepackage{booktabs}
\usepackage{xltxtra} 
\usepackage{polyglossia} 
 \usepackage{geometry}
 \geometry{
 a4paper,
 total={210mm,297mm},
 left=10mm,
 right=10mm,
 top=15mm,
 bottom=15mm,
 }
\usepackage[table]{xcolor}
\usepackage{color}
\usepackage{multirow}
\usepackage{gb4e} 
\usepackage{multicol}
\usepackage{graphicx}
\usepackage{float}
\usepackage{hyperref} 
\hypersetup{bookmarks=false,bookmarksnumbered,bookmarksopenlevel=5,bookmarksdepth=5,xetex,colorlinks=true,linkcolor=blue,citecolor=blue}
\usepackage{memhfixc}
\usepackage{lscape}
\usepackage[footnotesize,bf]{caption}
\usepackage{multicol}
 
 
\setmainfont[Mapping=tex-text,Numbers=OldStyle,Ligatures=Common]{Charis SIL} 
\setsansfont[Mapping=tex-text,Ligatures=Common,Mapping=tex-text,Ligatures=Common,Scale=MatchLowercase]{Lucida Sans Unicode} 
 


\newfontfamily\phon[Mapping=tex-text,Ligatures=Common,Scale=MatchLowercase]{Charis SIL} 
\newcommand{\ipa}[1]{{\phon\textbf{#1}}} 

\begin{document}
\title{Typologie : partiel du 28 mars 2018}
\date{}
\maketitle
 

\section{API}
\subsection{Donner une définition des symboles API suivants}
\begin{enumerate}
\item \ipa{ɢ}
\item \ipa{ɸ}
\item \ipa{ɳ}
\item \ipa{ɤ}
\item \ipa{ħ}
\end{enumerate}
\subsection{Donner les symboles API correspondant aux définitions suivantes}
\begin{enumerate}
\item fricative vélaire sonore:
\item occlusive rétroflexe voisée:
\item voyelle centrale (entre antérieure et postérieure) fermée arrondie:
\item approximante latérale palatale:
\item implosive bilabiale:
\end{enumerate}

\section{Transcription d'enregistrements}
\begin{enumerate}
\item ours: %311 ʁri
\item cerf: %313 rtse
\item biche: %313 zɮə
\item moineau: %336 ɣəzəzə
\item corbeau: %339 qali
\item serpent: %347 mpri
\item miel: %367 rbəɣdʐõ
\item noyau: %379 rqʰərqʰa
\item pierre à feu: %536 mirtɕɑ
\item vingt: %816 ɣnəsqʰa
\end{enumerate}
\end{document}

