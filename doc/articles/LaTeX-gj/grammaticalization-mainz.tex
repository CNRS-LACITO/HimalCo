\documentclass[oldfontcommands,oneside,a4paper,11pt]{article} 
\usepackage{fontspec}
\usepackage{natbib}
\usepackage{booktabs}
\usepackage{xltxtra} 
\usepackage{longtable}
\usepackage{polyglossia} 
%\usepackage[table]{xcolor}
\usepackage{gb4e} 
\usepackage{multicol}
\usepackage{graphicx}
\usepackage{float}
\usepackage{lineno}
\usepackage{textcomp}
\usepackage{hyperref} 
\hypersetup{bookmarks=false,bookmarksnumbered,bookmarksopenlevel=5,bookmarksdepth=5,xetex,colorlinks=true,linkcolor=blue,citecolor=blue}
\usepackage[all]{hypcap}
\usepackage{memhfixc}
\usepackage{lscape}
 


\newfontfamily\phon[Mapping=tex-text,Ligatures=Common,Scale=MatchLowercase,FakeSlant=0.3]{Charis SIL} 
\newcommand{\ipa}[1]{{\phon #1}} %API tjs en italique
 
\newcommand{\grise}[1]{\cellcolor{lightgray}\textbf{#1}}
\newfontfamily\cn[Mapping=tex-text,Ligatures=Common,Scale=MatchUppercase]{MingLiU}%pour le chinois
\newcommand{\zh}[1]{{\cn #1}}
\newcommand{\topic}{\textsc{dem}}
\newcommand{\tete}{\textsuperscript{\textsc{head}}}
\newcommand{\rc}{\textsubscript{\textsc{rc}}}
\XeTeXlinebreaklocale 'zh' %使用中文换行
\XeTeXlinebreakskip = 0pt plus 1pt %
 %CIRCG
 


\begin{document} 

\title{Gramamticalization in Japhug}
\author{Guillaume Jacques}
\maketitle
\linenumbers
\section{Garden-variety grammaticalization pathways}


TAM markers from directional prefixes (\citealt{lin11direction})

Reflexive from third person (\citealt{jacques10refl})

Associated motion from motion verb (\citealt{jacques13harmonization}); however, grammaticalized as prefixes, not suffixes

Converbs from nominalized verbs (\citealt{jacques14linking})

\section{Voice markers from denominative prefixes}

\citealt{jacques13tropative}
\citealt{jacques12incorp}
\citealt{jacques14antipassive}



\section{Progressive}
\ipa{asɯ--} from passive \ipa{a--} + causative \ipa{sɯ--}

infixation of inverse 

\ipa{tɕe} 	\ipa{pjɤ-ɣi} 	\ipa{tɕe} 	\ipa{qala} 	\ipa{kɯ} 	\ipa{pjɤ-kɤ́<wɣ>z-nɤjo-ci} 	\ipa{tɕe,} 


loss of transitivity marking

\section{Comitative adverbs}
Japhug has a very rich denominal morphology, and also has a productive comitative adverb derivation with the prefix \ipa{kɤɣɯ--} or \ipa{kɤ́--} and reduplicated noun stem, as in \ipa{tɤ-rtaʁ} `branch' $\rightarrow$ \ipa{kɤɣɯrtɯrtaʁ} `with branches'. These adverbs probably originate historically from the S-participle of denominal propriety verbs with the prefix \ipa{aɣɯ--} (\ipa{tɤ-rtaʁ} `branch' $\rightarrow$ \ipa{aɣɯrtɯrtaʁ} `have many branches').

\bibliographystyle{Linquiry2}
\bibliography{bibliogj}
\end{document}