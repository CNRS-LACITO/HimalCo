\documentclass[oldfontcommands,oneside,a4paper,11pt]{article} 
\usepackage{fontspec}
\usepackage{natbib}
\usepackage{booktabs}
\usepackage{xltxtra} 
\usepackage{longtable}
\usepackage{polyglossia} 
%\usepackage[table]{xcolor}
\usepackage{gb4e} 
\usepackage{multicol}
\usepackage{graphicx}
\usepackage{float}
\usepackage{lineno}
\usepackage{textcomp}
\usepackage{hyperref} 
\hypersetup{bookmarks=false,bookmarksnumbered,bookmarksopenlevel=5,bookmarksdepth=5,xetex,colorlinks=true,linkcolor=blue,citecolor=blue}
\usepackage[all]{hypcap}
\usepackage{memhfixc}
\usepackage{lscape}
 


\newfontfamily\phon[Mapping=tex-text,Ligatures=Common,Scale=MatchLowercase,FakeSlant=0.3]{Charis SIL} 
\newcommand{\ipa}[1]{{\phon #1}} %API tjs en italique
 
\newcommand{\grise}[1]{\cellcolor{lightgray}\textbf{#1}}
\newfontfamily\cn[Mapping=tex-text,Ligatures=Common,Scale=MatchUppercase]{MingLiU}%pour le chinois
\newcommand{\zh}[1]{{\cn #1}}
\newcommand{\topic}{\textsc{dem}}
\newcommand{\tete}{\textsuperscript{\textsc{head}}}
\newcommand{\rc}{\textsubscript{\textsc{rc}}}
\XeTeXlinebreaklocale 'zh' %使用中文换行
\XeTeXlinebreakskip = 0pt plus 1pt %
 %CIRCG
 


\begin{document} 

\title{Gramamticalization in Japhug}
\author{Guillaume Jacques}
\maketitle
\linenumbers
\section{Garden-variety grammaticalization pathways}


TAM markers from directional prefixes (\citealt{lin11direction})

Reflexive from third person (\citealt{jacques10refl})

Associated motion from motion verb (\citealt{jacques13harmonization}); however, grammaticalized as prefixes, not suffixes

Converbs from nominalized verbs (\citealt{jacques14linking})

\section{Voice markers from denominative prefixes}

\citealt{jacques13tropative}
\citealt{jacques12incorp}
\citealt{jacques14antipassive}



\section{Progressive}
Japhug has a progressive prefix \ipa{asɯ--} restricted to transitive verbs. It is not used on its own, but is combined with any of the following five TAM categories: past imperfective, evidential imperfective, testimonial, factual and plain imperfective. Depending on the environment, it has six allomorphs: \ipa{asɯ--}, \ipa{az--}, \ipa{ɤsɯ--}, \ipa{ɤz--}, \ipa{osɯ--} and \ipa{oz--}. The allomorphs with \ipa{z--} occur when preceding a sonorant initial prefix. The allomorphs in \ipa{a} occur in word-initial position and following the past imperfective prefix \ipa{pɯ--}. The allomorphs in \ipa{o} result from fusion with a preceding prefix whose main vowel in \ipa{u}. The allomorphs in \ipa{ɤ} are found in all other contexts.

Verb forms with the prefix \ipa{asɯ--} lack two of the obligatory transitive markers found in Japhug verbs, namely stem 3 alternation and past tense transitive \ipa{--t--} suffix. Only stem alternation is discussed here.


Japhug verbs exhibit stem alternation in non-past TAM categories (testimonial \ipa{ɲɯ--}, present \ipa{ku--}, imperative, irrealis, imperfective and factual) in direct singular A forms (\textsc{1/2/3sg}$\rightarrow$3). Following \citet{jackson00puxi}, we refer to this stem as `stem 3' (stem 1 being the base stem, and stem 2 the past stem). The use of this stem is illustrated in example \ref{ex:YWndAm}, where the verb \ipa{ndo} `hold' in the imperfective form has stem 3 \ipa{ndɤm}.

\begin{exe}
\ex \label{ex:YWndAm}
\gll \ipa{kɤ-kɤ-sɯ-ɕke} 	\ipa{ɯ-mdoʁ} 	\ipa{kɯ-fse} 	\ipa{ɲɯ-ndɤm} 		\ipa{ŋu} \\
\textsc{pfv-nmlz:P-caus}-burn \textsc{3sg.poss}-colour \textsc{nmlz:S/A}-be.like \textsc{ipfv}-hold[III] be:\textsc{fact} \\
\glt  (\ipa{ɲɤβrɯɣ}, 14)
\end{exe}

When a verb in non-past TAM forms takes the \ipa{asɯ--}, stem alternation does not occur. Examples \ref{ex:asWndo} and \ref{ex:YAsWndo}, in factual and testimonial form have the base stem \ipa{ndo} instead of stem 3 \ipa{ndɤm} as expected in forms without the progressive.


\begin{exe}
\ex \label{ex:asWndo}
\gll
\ipa{sɯjno} 	\ipa{ɯ-mdoʁ} 	\ipa{ʑo} 	\ipa{asɯ-ndo.} \\
grass \textsc{3sg.poss}-colour \textsc{emph} \textsc{prog}-hold:\textsc{fact} \\
\glt It has the colour of grass. (Caterpillar, 69)
\end{exe}


\begin{exe}
\ex \label{ex:YAsWndo}
\gll
\ipa{kɯki} 	\ipa{ɯ-mdoʁ} 	\ipa{tsa} 	\ipa{ɲɯ-ɤsɯ-ndo} \\
this  \textsc{3sg.poss}-colour  a.little \textsc{testim-prog}-hold \\
\glt It has a colour a bit like this one. (Slugs, 159)
\end{exe}




\ipa{pɯ-asɯ-fɕɤt-ndʑi} 	\ipa{nɯ} 	\ipa{ra,} 	\ipa{zlawawozɤr} 	\ipa{nɯ} 	\ipa{kɯ} 	\ipa{pjɤ-mtshɤm} 
(Nyimawodzer1, 32)


In the related Tshobdun language, the cognate prefix \ipa{ɐsɐ--} is labelled by \citet{jackson03caodeng} as `low transitivity progressive'.

\ipa{asɯ--} from passive \ipa{a--} + causative \ipa{sɯ--}

infixation of inverse 

\ipa{tɕe} 	\ipa{pjɤ-ɣi} 	\ipa{tɕe} 	\ipa{qala} 	\ipa{kɯ} 	\ipa{pjɤ-kɤ́<wɣ>z-nɤjo-ci} 	\ipa{tɕe,} 


loss of transitivity marking

\section{Comitative adverbs}
Japhug has a very rich denominal morphology, and also has a productive comitative adverb derivation with the prefix \ipa{kɤɣɯ--} or \ipa{kɤ́--} and reduplicated noun stem, as in \ipa{tɤ-rtaʁ} `branch' $\rightarrow$ \ipa{kɤɣɯrtɯrtaʁ} `with branches'. These adverbs probably originate historically from the S-participle of denominal propriety verbs with the prefix \ipa{aɣɯ--} (\ipa{tɤ-rtaʁ} `branch' $\rightarrow$ \ipa{aɣɯrtɯrtaʁ} `have many branches').

\bibliographystyle{Linquiry2}
\bibliography{bibliogj}
\end{document}