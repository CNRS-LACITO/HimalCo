\documentclass[oldfontcommands,oneside,a4paper,11pt]{article} 
\usepackage{fontspec}
\usepackage{natbib}
\usepackage{booktabs}
\usepackage{xltxtra} 
\usepackage{longtable}
\usepackage{polyglossia} 
%\usepackage[table]{xcolor}
\usepackage{gb4e} 
\usepackage{multicol}
\usepackage{graphicx}
\usepackage{float}
\usepackage{lineno}
\usepackage{textcomp}
\usepackage{hyperref} 
\hypersetup{bookmarks=false,bookmarksnumbered,bookmarksopenlevel=5,bookmarksdepth=5,xetex,colorlinks=true,linkcolor=blue,citecolor=blue}
\usepackage[all]{hypcap}
\usepackage{memhfixc}
\usepackage{lscape}
 


\newfontfamily\phon[Mapping=tex-text,Ligatures=Common,Scale=MatchLowercase,FakeSlant=0.3]{Charis SIL} 
\newcommand{\ipa}[1]{{\phon #1}} %API tjs en italique
 
\newcommand{\grise}[1]{\cellcolor{lightgray}\textbf{#1}}
\newfontfamily\cn[Mapping=tex-text,Ligatures=Common,Scale=MatchUppercase]{MingLiU}%pour le chinois
\newcommand{\zh}[1]{{\cn #1}}
\newcommand{\topic}{\textsc{dem}}
\newcommand{\tete}{\textsuperscript{\textsc{head}}}
\newcommand{\rc}{\textsubscript{\textsc{rc}}}
\XeTeXlinebreaklocale 'zh' %使用中文换行
\XeTeXlinebreakskip = 0pt plus 1pt %
 %CIRCG
 


\begin{document} 

\title{Grammaticalization in Japhug}
\author{Guillaume Jacques}
\maketitle
\linenumbers
\section{Garden-variety grammaticalization pathways}

\subsection{Directional prefixes and TAM marking} \label{sec:directional}
TAM markers from directional prefixes (\citealt{lin11direction})

\subsection{Reflexive}
Reflexive from third person (\citealt{jacques10refl})

\subsection{Associated motion}
Associated motion from motion verb (\citealt{jacques13harmonization}); however, grammaticalized as prefixes, not suffixes

\subsection{Converbs}
Converbs from nominalized verbs (\citealt{jacques14linking})

\section{Voice markers from denominative prefixes} \label{sec:voice}

\citealt{jacques13tropative}
\citealt{jacques12incorp}
\citealt{jacques14antipassive}
\citealt{jacques15causative}

\begin{exe}
\ex
\glt \textsc{causative} $\Rightarrow$ \textsc{abilitative}
\end{exe}


 
%Any attempt at etymologizing the progressive prefix \ipa{asɯ--} must take into account its morphological peculiarities, in particular the infixation of the inverse prefix.
%
%That the inverse prefix \ipa{wɣ--}is infixed within \ipa{asɯ--} strongly suggests that the progressive is composite in origin, and should be divided into two elements \ipa{a--} and \ipa{sɯ--}. There are several prefixes with these shapes in Japhug.
%
%There are three prefixes with the shape \ipa{sɯ--} in Japhug:   causative,   abilitative and (instrumental / causative) denominal. As seen in  \ref{sec:voice} (see also \citealt{jacques15causative}), the following pathways of grammaticalization can be postulated:
%
%\begin{exe}
%\ex
%\glt \textsc{causative/instrumental denominal} + \textsc{bare infinitive} $\Rightarrow$ \textsc{causative} 
%\glt \textsc{causative} $\Rightarrow$ \textsc{abilitative}
%\end{exe}
%
%As for prefixes with the shape \ipa{a--}, if we exclude the first person possessive prefix  \ipa{a--}, we find the passive  / reflexive \ipa{a--} and the stative denominal \ipa{a--}. Here again, the voice prefix historically derives from the denominal one by the following pathway:
%
%\begin{exe}
%\ex
%\glt \textsc{stative denominal} + \textsc{bare infinitive} $\Rightarrow$ \textsc{passive} 
%\end{exe}
%
%Hence, there are several potential scenarios to explain the composite origin of the prefix \ipa{asɯ--}. First, it could originate from an amalgamation of voice  prefixes (first causative, since it is closer to the verb stem, then passive). Second, it could be derived from a combination of instrumental and stative denominal prefixes. 
%
%\subsubsection{Causative+passive}
%In Japhug, the combination of causativer and passive is only possible in the reversed order \ipa{sɯ-ɤ--}, and its semantics is completely different. Only three examples are found:    \ipa{sɤmbi} ``to require something from someone'' ,  \ipa{sɤβzu} ``to prepare, to make ready to use', 
%\ipa{sɤpa} ``transform (tr.)',   \ipa{sɤjtsʰi} ``to ask for something to drink''
%
% %\ipa{sɤmbi} ``to require something from someone'' is a causative form derived from the passive \ipa{a-mbi} ``to be given'' of the verb \ipa{mbi} ``to give''. Etymologically, the verb means ``to cause someone to give to oneself''. 
%
%However, the causative \ipa{sɯ--} can be combined with another intransitivizing prefix, the reflexive \ipa{ʑɣɤ--}
%
%\begin{exe}
%\ex
%\gll \ipa{ɯ-jmŋo} 	\ipa{ɯ-ŋgɯ} 	\ipa{ko-ʑɣɤ-sɯ-ntɕʰɤr} 	\ipa{tɕe} \\
%\textsc{3sg.poss}-dream \textsc{3sg}-in \textsc{evd-refl-caus}-appear \textsc{lnk} \\
%\glt He made himself appear in her dream. (Slobdpon2.97)
%\end{exe}

\section{Comitative adverbs}
Japhug has a very rich denominal morphology, and also has a productive comitative adverb derivation with the prefix \ipa{kɤɣɯ--} or \ipa{kɤ́--} and reduplicated noun stem, as in \ipa{tɤ-rtaʁ} `branch' $\rightarrow$ \ipa{kɤɣɯrtɯrtaʁ} `with branches'. These adverbs probably originate historically from the S-participle of denominal propriety verbs with the prefix \ipa{aɣɯ--} (\ipa{tɤ-rtaʁ} `branch' $\rightarrow$ \ipa{aɣɯrtɯrtaʁ} `have many branches').

\bibliographystyle{Linquiry2}
\bibliography{bibliogj}
\end{document}