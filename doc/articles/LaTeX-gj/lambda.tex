\documentclass[oldfontcommands,twoside,a4paper,12pt]{article} 
\usepackage{xunicode}%packages de base pour utiliser xetex
\usepackage{fontspec}
\usepackage{natbib}
\usepackage{booktabs}
\usepackage{xltxtra} 
\usepackage{longtable}
\usepackage{polyglossia} 
\usepackage[table]{xcolor}
\usepackage{color}
\usepackage{multirow}
\usepackage{gb4e} 
\usepackage{graphicx}
\usepackage{float}
\usepackage{memhfixc}
\usepackage{lscape}
\usepackage[footnotesize,bf]{caption}


%%%%%%%%%%%%%%%%%%%%%%%%%%%%%%%
\setmainfont[Mapping=tex-text,Numbers=OldStyle,Ligatures=Common]{Charis SIL} 
\setsansfont[Mapping=tex-text,Ligatures=Common,Mapping=tex-text,Ligatures=Common,Scale=MatchLowercase]{Lucida Sans Unicode} 
 


\newfontfamily\phon[Mapping=tex-text,Ligatures=Common,Scale=MatchLowercase,FakeSlant=0.3]{Charis SIL} 
\newfontfamily\phondroit[Mapping=tex-text,Ligatures=Common,Scale=MatchLowercase]{Doulos SIL} 
\newcommand{\ipa}[1]{{\phon #1}} 
 
 \newfontfamily\cn[Mapping=tex-text,Ligatures=Common,Scale=MatchUppercase]{MingLiU}%pour le chinois
\newcommand{\zh}[1]{{\cn #1}}
 



\begin{document}

  \title{Lateral fricatives in Panchronic perspective}

\maketitle

\section{Lateral fricatives in the world's languages}
Northern America
\section{The attested origins of the lateral fricatives}

min

central tai 

mongolian

Athabaskan
\subsection{Tibetan}
Zhongu (\citet[787-8]{sun03zhongu}): origin of lh as *hl in pre-Tibetan, Baima lɦ

\citet[96]{skalbzang02}

\begin{table}[H]
\caption{Sdedge Tibetan}\label{tab:sdedge} \centering
\begin{tabular}{lllllllll} \toprule
Old Tibetan & Sdedge&Meaning \\
\midrule 
 rluŋ & \ipa{lũ45} & wind\\
klad.pa & \ipa{lɛ45pa53} & \\
\midrule 
lhogs & \ipa{ɬoʔ53} & read (imperative) \\
bslaŋs & \ipa{ɬõ45} & take up (past) \\
 sla & \ipa{ɬa53} & \\
 \midrule 
sna & \ipa{n̥a53} & nose\\
sman & \ipa{m̥ẽ45} & medecine\\
 \midrule 
 so & \textit{sʰo53} & tooth\\
 gso & \textit{so53} & raise\\
\bottomrule
\end{tabular}
\end{table}

ɬ from lh and sl (confusion)

other clusters rl kl yield l

parallel to sn sm to aspirated nasals







\section{The attested historical outcomes of lateral fricatives}
\section{The palatal laterals}
Romance languages > similar to lateral fricatives in their origins and outcomes

\section{Debated reconstructions}
\subsection{Proto-Algonquian}
\citet{bloomfield25central}
\citet{bloomfield46proto}
\citet{goddard79comparative}
\cite{picard94}

\subsection{Turkic}


\bibliographystyle{plainnat}
\bibliography{bibliogj}

\end{document}