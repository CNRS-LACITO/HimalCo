\documentclass[oldfontcommands,oneside,a4paper,11pt]{article} 
\usepackage{fontspec}
\usepackage{natbib}
\usepackage{booktabs}
\usepackage{xltxtra} 
\usepackage{polyglossia} 
\setdefaultlanguage{french} 
\usepackage[table]{xcolor}
\usepackage{gb4e} 
\usepackage{multicol}
\usepackage{graphicx}
\usepackage{float}
\usepackage{hyperref} 
\hypersetup{bookmarks=false,bookmarksnumbered,bookmarksopenlevel=5,bookmarksdepth=5,xetex,colorlinks=true,linkcolor=blue,citecolor=blue}
\usepackage[all]{hypcap}
\usepackage{memhfixc}
\usepackage{lscape}

\bibpunct[: ]{(}{)}{,}{a}{}{,}

%\setmainfont[Mapping=tex-text,Numbers=OldStyle,Ligatures=Common]{Charis SIL} 
\newfontfamily\phon[Mapping=tex-text,Ligatures=Common,Scale=MatchLowercase]{Charis SIL} 
\newcommand{\ipa}[1]{{\phon\textbf{\mbox{#1}}}} %API tjs en italique
\newcommand{\ipab}[1]{{\scriptsize \phon#1}} 

\newcommand{\grise}[1]{\cellcolor{lightgray}\textbf{#1}}
\newfontfamily\cn[Mapping=tex-text,Ligatures=Common,Scale=MatchUppercase]{SimSun}
\newcommand{\zh}[1]{{\cn #1}}
\newcommand{\refb}[1]{(\ref{#1})}

\newcommand{\ra}{$\Sigma_1$} 
\newcommand{\rc}{$\Sigma_3$} 
\newcommand{\ro}{$\Sigma$} 

\XeTeXlinebreaklocale 'zh' %使用中文换行
\XeTeXlinebreakskip = 0pt plus 1pt %
 %CIRCG
 
\sloppy

\begin{document} 
\title{Index of Japhug grammatical topics}
\author{Guillaume Jacques}
\maketitle

I am working on a reference grammar of Japhug, but until it is complete, the following list presents the grammatical topics discussed in various articles and papers, either published or formally accepted for publication. Forthcoming articles (not included here) discuss complementation, similative constructions, the progressive prefix, evidentiality and indirect speech.

\begin{itemize}
\item Abilitative: \citet{jacques15causative}
\item Action nominals: \citet[7]{jacques14antipassive}
\item Adjectives: \citet[275]{jacques16complementation}, \citet[627]{jacques17sketch}
\item Alternative concessive conditionals: \citet[298]{jacques14linking}
\item Andative: \citet[200-6]{jacques13harmonization}
\item Anticausative: \citet[213-4]{jacques12demotion}, \citet{jacques15spontaneous}
\item Antipassive: \citet[215-6]{jacques12demotion}, \citet{jacques14antipassive}
\begin{itemize}
\item of ditransitive verbs : \citet[13-4]{jacques14antipassive}
\item irregular:  \citet[18-20]{jacques14antipassive}
\end{itemize}
\item Applicative: \citet{jacques13tropative}
\item Associated motion: \citet[200-6]{jacques13harmonization}, \citet[623]{jacques17sketch}
\item Autobenefactive: \citet{jacques15spontaneous}
\item Bare infinitive: \citet[9]{jacques14antipassive}
\item Calling/Chasing sounds:   \citet[283-4]{japhug14ideophones}
\item Case marking: \citet{jacques17sketch}, \citet{jacques16comparative}, \citet[260-261]{jacques16complementation}
\item Causal linking:  \citet[303-6]{jacques14linking}, \citet{jacques16comparative}
\item Causative: \citet{jacques15causative}, \citet[273-275]{jacques16complementation}
\item Comitative: \citet[272-4]{jacques14linking}
\item Comparative: \citet{jacques16comparative}
\item Complementation: \citet{jacques08},  \citet{jacques17sketch}, \citet{jacques16complementation}
\begin{itemize}
\item  Raising: \citet{jacques15causative}, \citet[260]{jacques16complementation}
\end{itemize}
\item Complex predicates: \citet{jacques12incorp}, \citet{jacques16complementation}
\item Conditionals:  \citet[296-300]{jacques14linking}
\item Conjunctions:  \citet[276-7]{jacques14linking}
\item Contrast:  \citet[315-8]{jacques14linking}, \citet{jacques16comparative}
\item Converb:  \citet[269-272;307-8;321-2]{jacques14linking}
\item Counterfactuals:  \citet[301-2]{jacques14linking}
\item Deexperiencer: \citet[216-7]{jacques12demotion}, \citet{jacques14antipassive}
\item Degree: \citet{jacques16comparative}
\item Deideophonic verbs: \citet[278-282]{japhug14ideophones}
\item Demonstrative: \citet[627]{jacques17sketch}
\item Denominal verbs:  \citet{jacques12incorp}, \citet{jacques14antipassive}
\item Direct/Inverse marking: \citet{jacques10inverse}
\item Directional prefixes:  \citet[267-8]{jacques14linking}
\item Disjunction:  \citet[318-9]{jacques14linking}
\item Distributive: \citet{jacques16comparative}
\item Egophoric: \citet[617-620]{jacques17sketch}
\item Ergative:   \citet{jacques16comparative}
\begin{itemize}
\item Basic marking: \citet[131-2]{jacques10inverse} 
\item In complement clauses: \citet[260-261]{jacques16complementation}
\item Long distance: \citet[278]{jacques14linking}
\item Syntactic pivot:   \citet[208]{jacques12demotion}, \citet[256-257]{jacques16complementation}
\end{itemize}
\item Essive: \citet[225]{jacques16complementation}
\item Evidentiality: \citet[380-390]{jacques04these}, \citet{jacques08}, \citet[617-620]{jacques17sketch}
\item Estimative: \citet{jacques13tropative}
\item Generic person:   \citet[204-8]{jacques12demotion}, \citet{jacques17generic}
\item Grammatical relations: \citet{jacques16relatives}
\item Ideophones: \citet{japhug14ideophones}
\begin{itemize}
\item in relatives: \citet[275]{japhug14ideophones}
\end{itemize}
\item Inalienable possession: \citet[4]{jacques14antipassive}, \citet{jacques17generic}
\item Incorporation: \citet{jacques12incorp}
\item Indefinite possessor: \citet[1212]{jacques12incorp}, \citet[4]{jacques14antipassive}, \citet{jacques17generic}
\item Indexation: \citet{jacques10inverse}, \citet[85]{jacques12agreement}
\item Infinitive: \citet[227-228]{jacques16complementation}
\item Instrumental: \citet{jacques16comparative}
\item Interjections: \citet[283]{japhug14ideophones}
\item Irregular verbs: \citet[91]{jacques12agreement}, \citet[1215]{jacques12incorp}
\item Iterative coincidence: \citet[296]{jacques14linking}
\item Kinship system: \citet{jacques11kinship}
\item Lability: \citet[216-9]{jacques12demotion}, \citet[626]{jacques17sketch}
\item Linkers:  \citet[276-7]{jacques14linking}
\item Loanwords: \citet[83-199]{jacques04these}
\item Manner linking:  \citet[320-5]{jacques14linking}
\item Middle:  \citet{jacques12demotion}, \citet{jacques15spontaneous}
\item Modality: \citet[261-265]{jacques16complementation}, \citet[615-621]{jacques17sketch}
\item Motion verbs: \citet[201-6]{jacques13harmonization}, \citet[267]{jacques16complementation}
\item Negative existential verbs: \citet[270-273]{jacques16complementation}
\item Nominalization: \citet[5-7]{jacques14antipassive}, \citet{jacques16relatives}, \citet{jacques16complementation}
\item Noun phrase: \citet[627]{jacques17sketch}
\item Numerals: \citet[4]{jacques14antipassive}
\item Onomatopoiea:   \citet[282]{japhug14ideophones}
\item Parataxis:  \citet[312;315]{jacques14linking}
\item Participles: \citet[5-6]{jacques14antipassive}, \citet{jacques16relatives}, \citet{jacques16complementation}
\item Passive: \citet{jacques07passif}, \citet[208-13]{jacques12demotion}
\item Permansive: \citet{jacques15spontaneous}
\item Phasal verbs: \citet[265]{jacques16complementation}
\item Phonology: \citet[12-82]{jacques04these}, \citet{jacques08}, \citet{jacques17ipa}
\item Portmanteau: \citet[136-7]{jacques10inverse}, \citet{jacques17generic}
\item Possessive prefixes: \citet[4]{jacques14antipassive}
\item Possible consequence:  \citet[308-311]{jacques14linking}
\item Postpositions:  \citet[272-4]{jacques14linking}
\item Prefixes: \citet[196-199]{jacques13harmonization}
\item Purposive:  \citet[306-8]{jacques14linking}
\item Reciprocal: \citet{jacques07passif}, \citet[212]{jacques12demotion}
\item Reduplication: \citet{jacques04these}, \citet{jacques07redupl}
\item Reflexive: \citet{jacques10refl}, \citet[85]{jacques12agreement}
\item Relator nouns:  \citet[274-6]{jacques14linking}
\item Relativization: \citet{jacques08}, \citet{jacques16relatives},\citet{jacques17sketch}
\begin{itemize}
\item Instrument: \citet{jacques16comparative}
\item Comitative: \citet[272-4]{jacques14linking}
\end{itemize}
\item Reported Speech: \citet[241-244]{jacques16complementation}
\item Scalar concessive conditionals: \citet[300]{jacques14linking}
\item Serial Verb Constructions: \citet[253-255]{jacques16complementation}
\item Spontaneous-autobenefactive: \citet{jacques15spontaneous}
\item Stem alternation:  \citet[267]{jacques14linking}, \citet[615]{jacques17sketch}
\item Syntactic pivots: \citet{jacques16relatives}, \citet{jacques16complementation}
\item Synthetic compounds: \citet[1220-3]{jacques12incorp}
\item Template (verbal): \citet[196-199]{jacques13harmonization}
\item Tense-Aspect-Modality:  \citet[265-9]{jacques14linking}, \citet{jacques17sketch}
\item Tail-head linkage:  \citet[279-280]{jacques14linking}
\item Tense: \citet[615-621]{jacques17sketch}
\item Temporal subordinate clause:  \citet[281-295]{jacques14linking}
\item Transitivity: \citet[9-10]{jacques14antipassive}, \citet[625-626]{jacques17sketch}
\item Tropative: \citet{jacques13tropative}
\item Venitive: \citet[200-6]{jacques13harmonization}, \citet{jacques17sketch}
\item Vertitive:  \citet{jacques15spontaneous}
\item Verb fronting:  \citet[280]{jacques14linking}
\end{itemize}

\bibliographystyle{unified}
\bibliography{bibliogj}
\end{document}