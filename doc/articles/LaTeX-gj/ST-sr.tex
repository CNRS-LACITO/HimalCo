\documentclass[oldfontcommands,oneside,a4paper,11pt]{article} 
\usepackage{fontspec}
\usepackage{natbib}
\usepackage{booktabs}
\usepackage{xltxtra} 
\usepackage{longtable}
\usepackage{tangutex2} 
\usepackage{tangutex4} 
\usepackage{polyglossia} 
\usepackage[table]{xcolor}
\usepackage{gb4e} 
\usepackage{multicol}
\usepackage{graphicx}
\usepackage{float}
\usepackage{hyperref} 
\hypersetup{bookmarks=false,bookmarksnumbered,bookmarksopenlevel=5,bookmarksdepth=5,xetex,colorlinks=true,linkcolor=blue,citecolor=blue}
\usepackage[all]{hypcap}
\usepackage{memhfixc}
\usepackage{lscape}
\bibpunct[: ]{(}{)}{,}{a}{}{,}
%%%%%%%%%quelques options de style%%%%%%%%
%\setsecheadstyle{\SingleSpacing\LARGE\scshape\raggedright\MakeLowercase}
%\setsubsecheadstyle{\SingleSpacing\Large\itshape\raggedright}
%\setsubsubsecheadstyle{\SingleSpacing\itshape\raggedright}
%\chapterstyle{veelo}
%\setsecnumdepth{subsubsection}
%%%%%%%%%%%%%%%%%%%%%%%%%%%%%%%
%\setmainfont[Mapping=tex-text,Numbers=OldStyle,Ligatures=Common]{Charis SIL} %ici on définit la police par défaut du texte
\newfontfamily\phon[Mapping=tex-text,Ligatures=Common,Scale=MatchLowercase,FakeSlant=0.3]{Charis SIL} 
\newcommand{\ipa}[1]{{\phon #1}} %API tjs en italique
 
\newcommand{\grise}[1]{\cellcolor{lightgray}\textbf{#1}}
\newfontfamily\cn[Mapping=tex-text,Ligatures=Common,Scale=MatchUppercase]{MingLiU}%pour le chinois
\newcommand{\zh}[1]{{\cn #1}}

\newcommand{\jg}[1]{\ipa{#1}\index{Japhug #1}}
\newcommand{\wav}[1]{#1.wav}
\newcommand{\tgz}[1]{\mo{#1} \tg{#1}}
\newcommand{\ra}{$\Sigma$} 
\XeTeXlinebreaklocale "zh" %使用中文换行
\XeTeXlinebreakskip = 0pt plus 1pt %
 %CIRCG
 
\begin{document} 


\title{On the cluster *sr-- in Sino-Tibetan\footnote{Old Chinese follows \citealt{bs14oc}, Tibetan is transcribed according to \citet{jacques12transcription}.}}
\author{Guillaume Jacques}
\maketitle

\section{Introduction}

The Middle Chinese \zh{生} shēng initial consonant \ipa{ʂ}, which originates from Old Chinese *\ipa{sr--} in all modern systems of reconstruction, is attested in a words of Sino-Tibetan origin, and corresponds to onsets either preserving a fricative+/r/cluster or originating from one. Thus, *sr appears to be  one of the few consonant clusters uncontroversially reconstructible to proto-Sino-Tibetan. 

In this paper, we first discuss previously proposed etymologies, and present the known correspondences of *sr in languages other than Chinese. Second, we present a new etymology and discuss its significance for the conditioning of the sound laws in individual languages.

\section{Previous comparisons} \label{sec:previous}
Only three Chinese words with initial *\ipa{sr-} correspond to forms that are widespread in the rest of the family and can be solidly reconstructed with initial *\ipa{sr--} clusters.  They were first proposed by \citet{benedict72}.

The first such etymon is the word for `louse' \zh{蝨} *\ipa{srik} $\rightarrow$ \ipa{ʂit}, which can be compared to Tibetan \ipa{ɕig}, Burmese \ipa{hrac}, Japhug \ipa{zrɯɣ} and Limbu \ipa{siʔ}. All words in this cognate set share the same meaning, and there is little doubt that they are related.

The second comparison is Chinese \zh{色} *\ipa{srək} $\rightarrow$ \ipa{ʂik} `colour, sex, shame', which is compared to Tibetan \ipa{ɴtɕʰags}, \ipa{bɕags} `confess', Burmese \ipa{hrak} `shame', Japhug \ipa{tɤ-zraʁ} `shame'. This comparison is however less convincing from the point of view of semantics, and in the case of Tibetan, philology suggests that the meaning `confess' is secondary, and evolved from `declare', the meaning attested in its oldests attestation, the bilingual Sino-Tibetan treaty incription (example \ref{ex:bshags}, \citealt[40,80]{licoblin87}), where 	\ipa{bɕags} corresponds to Chinese \zh{稽告} `make known, explain, declare' 


\begin{exe}
\ex \label{ex:bshags}
\gll
\ipa{ɴdï-ltar} 	\ipa{bod} 	\ipa{rgʲa} 	\ipa{gɲïs} 	\ipa{kʲï} 	\ipa{rdʑe} \ipa{blon-gʲis} 	\ipa{kʰa.tɕïg} 	\ipa{bɕags} 	\ipa{mnaɦ} 	\ipa{bor-te} \\
this-like Tibet China two \textsc{gen} sovereign minister-\textsc{erg} together \textsc{pst}:declare oath \textsc{pst}:throw-\textsc{conv} \\
\glt Thus the sovereigns and the ministers of both Tibet and China together declared and swore an oath. (Sino-Tibetan Treaty, West face, l. 71-72)
\end{exe}

The third one is \zh{生} *\ipa{N-sreŋ} $\rightarrow$ \ipa{ʂæŋ} `live, alive', corresponding to Burmese \ipa{hraŋ²} `alive' and other comparanda (see STEDT \#71). This root has no cognate in Tibetan or Rgyalrong languages.

Other comparisons of Old Chinese \ipa{*sr--} have been proposed by \citet{coblin86handlist} in particular, but they are restricted to Tibetan comparanda, and involve words with the onset \ipa{sr--} in Tibetan. Nearly all such comparisons can be shown to be invalid for various reasons.\footnote{The comparison of Chinese \zh{率} *\ipa{s-rut} $\rightarrow$ \ipa{ʂwit}`rule' to Tibetan \ipa{srid} `government' is problematic for several reasons. The vowel correspondence is not a match (\citealt{gong95st}), and the Chinese verb is obviously related to \zh{律} *\ipa{rut} $\rightarrow$ \ipa{lwit} `law, rule': the \ipa{s--} is here denominal. On the etymology of \zh{律} *\ipa{rut}, see \citet{sagart14lv}.

The only other comparison, \zh{產} *\ipa{s-ŋrˁarʔ} $\rightarrow$ \ipa{ʂɛn^x} `produce' to Tibetan \ipa{srel} `bring up', which appears possible on the basis of Middle Chinese, is to be ruled out once Old Chinese reconstruction is taken into account.
}
The only promising such correspondence is \zh{甥} *\ipa{srˁeŋ} $\rightarrow$ \ipa{ʂæŋ} `sister's son' with Tibetan \ipa{sriŋ.mo} `sister'. 



The double correspondence *\ipa{sr--} to \ipa{sr--} or \ipa{ɕ--} in Tibetan suggests that two proto-onsets must be reconstructed here: *\ipa{sə-r--} with a reduced vowel yielding \ipa{sr--}, while the actual cluster *\ipa{sr--} changes to \ipa{ɕ--}, perhaps through a stage *[\ipa{ʂ}].



\section{A new example of proto-Trans-Himalayan *\ipa{sr}}


The word for `root'. In Rgyalrong, it is attested in Japhug as the possessed noun \ipa{tɤ-zrɤm} `root' (Situ \ipa{--srám}, Zbu \ipa{rzám} from proto-Rgyalrong *\ipa{srɐm}, see \citealt[243]{jacques04these}). This possessed noun does not derive from any verb (see \citealt[3-7]{jacques14antipassive} on the derivation of possessed nouns).

Japhug has a variant \ipa{--srɤm} which refers to the meaning `root' in a more abstract sense of `family lineage', as illustrated by the following example:

\begin{exe}
\ex \label{ex:sram}
\gll
\ipa{nɯnɯ} 	\ipa{ɕɯ-kɤ-ru} 	\ipa{mɯ\textasciitilde{}mɤ-pɯ-tɯ-cha} 	\ipa{ŋu} 	\ipa{nɤ,} 	\ipa{li} 	\ipa{nɤ-srɤm} 	\ipa{nɤ-sroʁ} 	\ipa{ma} 	\ipa{me} \\
\textsc{dem} \textsc{transloc-inf}-bring \textsc{cond\textasciitilde{}neg-ipfv}-2-can \textsc{fact}:be again \textsc{2sg.poss}-root \textsc{2sg.poss}-life apart.from \textsc{fact}:not.exist\\
\glt If you cannot bring it here, again, there is only your family and your life (for you to lose).
\end{exe}

This restricted meaning in a context involving a king and his subjects suggests that \ipa{--srɤm} in Japhug is not inherited: it is borrowed from Situ Rgyalrong, which was the language of the local chieftain (Tusi). Note that most borrowings from Tibetan (such as \ipa{--sroʁ} `life' from \ipa{srog} have \ipa{sr--} in Japhug corresponding to Tibetan \ipa{sr--}.

Apart from this example, voicing of \textit{s--} in Japhug in this cluster and metathesis in Zbu occurs everywhere.

In Kiranti, we find a root *sam attested by Khaling \ipa{sɛ̄m}, Yakkha  \ipa{sam}, Kulung \ipa{sam} (\citealt{kongren07yakkha}, \citealt{tolsma06kulung}). The simple initial *s-- in Kiranti can correspond to Japhug \ipa{zr--} `louse' (Japhug \ipa{zrɯɣ} vs Kulung \ipa{si}).

A search in STEDT reveals no similar form in any other Sino-Tibetan language. However, this word is phonologically comparable with Chinese \zh{參} \ipa{shēn} (Middle Chinese \ipa{ʂim}, Old Chinese *\ipa{srəm}). The character  \zh{參} has several readings, but Middle Chinese \ipa{ʂim} is associated with two meanings: one of the 28 constellations, and rhizomous medicinal plants such as Ginseng (still called in modern Chinese \zh{人參}\ipa{rénshēn}). The earliest attestation of the use of \zh{參} \ipa{ʂim} for a medicinal plant goes back to the Western Han dynasty, and some scholars have argued for an earlier date (for instance \citealt{xu11shen}, \citealt{sun92renshen}). Old Chinese *srəm is a perfect match for proto-Rgyalrongic *\ipa{srɐm} and proto-Kiranti *\ipa{sam} (as shown by \citealt{gong95st}, Old Chinese *\ipa{ə}regularly corresponds to \ipa{a} in Tibetan and other languages). \citet{bs14oc} reconstruct *\ipa{srum} here, but no evidence either from loanwords or phonetic series rule out the reconstruction *\ipa{srəm}, which is the one adopted by other scholars (\citealt{schuessler06}).


\section{Loss of *-r-?}
In addition to the correspondences seen in section \ref{sec:previous}, comparisons where Chinese *\ipa{sr} corresponds to \ipa{s} in other languages have been proposed (in particular by \citealt{coblin86handlist}). Most of these examples either represent more complex correspondences (\zh{髟} *\ipa{sˁram} $\rightarrow$ \ipa{ʂæm} `hair' corresponds to \ipa{s} in some languages, and to an affricate in others, as in Burmese \ipa{chaṃ³} `hair') or are spurious.\footnote{The comparison of Chinese \zh{雙} *\ipa{sˁroŋ} $\rightarrow$ \ipa{ʂæwŋ} `pair' to Tibetan \ipa{zuŋ} `pair' is impossible as Tibetan \ipa{z} originates from pre-Tibetan *\ipa{dz}, see \citet{hill14dz}.

Chinese \zh{沙} *\ipa{sˁraj} $\rightarrow$ \ipa{ʂæ} `sand' has been compared with Tibetan \ipa{sa} `place', but the coda *\ipa{--j} normally corresponds to Tibetan \ipa{--r} or \ipa{--l}, and for this reason this comparison must be rejected. }

The only uncontroversible example of the correspondence *\ipa{sr} : \ipa{s} is  \zh{殺} *\ipa{srat} $\rightarrow$ \ipa{ʂɛt} `kill', corresponding to Tibetan \ipa{gsod, bsad} `kill', Japhug \ipa{sat} `kill' etc.

%Burmese \ipa{sok} `drink', though the meaning 


The only attempt to explain the double correspondence of Chinese *\ipa{sr--} to other languages is \citet[25]{handel02r}. According to Handel, original PST *\ipa{sr} changed to s in non-Chinese languages (`Tibeto-Burman') before non-front vowels. This phonological solution has the merit of simplicity, and, if true, provides a common phonological innovation to all languages besides Chinese (the only one that has been explicitly proposed in print apart from the merger of *\ipa{a} and *\ipa{ə}, on which see \citealt{gong95st}, \citealt{handel08st}).

However, examples such as \zh{色} *\ipa{srək} `colour, shame' or \zh{參} *\ipa{srəm} `rhizome' refute Handel's theory, as they show that the conditioning factor that he proposed is not valid.

Since, as proposed by \citet{sagart99roc}, an infix *\ipa{-r-} is reconstructible in Old Chinese, an alternative explanation is to consider Chinese here to be innovative: it only preserved the infixed form of the verb \zh{殺} *\ipa{s<r>at}. Therefore, there is no need to look for a phonological conditioning of this correspondence.

\section{Conclusion}

The contribution of this paper is twofold. First, it provides a critical overview of previously proposed etymologies involving the onset *\ipa{sr--} in Old Chinese, and shows which etymologies are possible and which should be discarded, on the basis of philological and comparative data.

Second, it shows a new example of proto-Sino-Tibetan *\ipa{sr--}, and in particular the second comparison including Kiranti languages. It confirms that proto-Sino-Tibetan *\ipa{sr--} is simplified to *\ipa{s--} in proto-Kiranti.


\bibliographystyle{linquiry2}
\bibliography{bibliogj}
\end{document}
