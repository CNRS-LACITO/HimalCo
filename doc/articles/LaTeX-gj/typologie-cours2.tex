\documentclass[oldfontcommands,twoside,a4paper,12pt]{article} 
\usepackage{fontspec}
\usepackage{natbib}
\usepackage{booktabs}
\usepackage{xltxtra} 
\usepackage{longtable}
\usepackage{tangutex2} 
%\usepackage{tangutex4} 
\usepackage{polyglossia} 
\usepackage[table]{xcolor}
\usepackage{color}
\usepackage{multirow}
\usepackage{gb4e} 
\usepackage{multicol}
\usepackage{graphicx}
\usepackage{float}
\usepackage{hyperref} 
\hypersetup{bookmarks=false,bookmarksnumbered,bookmarksopenlevel=5,bookmarksdepth=5,xetex,colorlinks=true,linkcolor=blue,citecolor=blue}
\usepackage{memhfixc}
\usepackage{lscape}
\usepackage[footnotesize,bf]{caption}


%%%%%%%%%%%%%%%%%%%%%%%%%%%%%%%
\setmainfont[Mapping=tex-text,Numbers=OldStyle,Ligatures=Common]{Charis SIL} 
\setsansfont[Mapping=tex-text,Ligatures=Common,Mapping=tex-text,Ligatures=Common,Scale=MatchLowercase]{Lucida Sans Unicode} 
 


\newfontfamily\phon[Mapping=tex-text,Ligatures=Common,Scale=MatchLowercase,FakeSlant=0.3]{Charis SIL} 
\newfontfamily\phondroit[Mapping=tex-text,Ligatures=Common,Scale=MatchLowercase]{Doulos SIL} 
\newcommand{\ipa}[1]{{\phon\textbf{#1}}} 
\newcommand{\ipab}[1]{{\phon #1}}
\newcommand{\ipapl}[1]{{\phondroit #1}} 
\newcommand{\captionft}[1]{{\captionfont #1}} 
\newfontfamily\cn[Mapping=tex-text,Ligatures=Common,Scale=MatchUppercase]{MingLiU}%pour le chinois
\newcommand{\zh}[1]{{\cn #1}}
\newcommand{\tgf}[1]{{\large\mo{#1}}}

\newcommand{\acc}{\textsc{acc}}
\newcommand{\antierg}{\textsc{antierg}}
\newcommand{\allat}{\textsc{all}}
\newcommand{\aor}{\textsc{aor}}
\newcommand{\assert}{\textsc{assert}}
\newcommand{\auto}{\textsc{auto}}
\newcommand{\caus}{\textsc{caus}}
\newcommand{\classif}{\textsc{class}}
\newcommand{\concessif}{\textsc{concsf}}
\newcommand{\comit}{\textsc{comit}}
\newcommand{\conj}{\textsc{conj}}
\newcommand{\const}{\textsc{const}}
\newcommand{\conv}{\textsc{conv}}
\newcommand{\cop}{\textsc{cop}}
\newcommand{\dat}{\textsc{dat}}
\newcommand{\dem}{\textsc{dem}}
\newcommand{\detm}{\textsc{det}}
\newcommand{\dir}{\textsc{dir1}}
\newcommand{\du}{\textsc{du}}
\newcommand{\duposs}{\textsc{du.poss}}
\newcommand{\dur}{\textsc{dur}}
\newcommand{\erg}{\textsc{erg}}
\newcommand{\evd}{\textsc{evd}}
\newcommand{\fut}{\textsc{fut}}
\newcommand{\gen}{\textsc{gen}}
\newcommand{\hypot}{\textsc{hyp}}
\newcommand{\ideo}{\textsc{ideo}}
\newcommand{\imp}{\textsc{imp}}
\newcommand{\impf}{\textsc{ipfv}}
\newcommand{\instr}{\textsc{instr}}
\newcommand{\intens}{\textsc{intens}}
\newcommand{\intrg}{\textsc{intrg}}
\newcommand{\inv}{\textsc{inv}}
\newcommand{\irreel}{\textsc{irr}}
\newcommand{\loc}{\textsc{loc}}
\newcommand{\med}{\textsc{med}}
\newcommand{\negat}{\textsc{neg}}
\newcommand{\neu}{\textsc{neu}}
\newcommand{\nmlz}{\textsc{nmlz}}
\newcommand{\nonps}{\textsc{n.pst}}
\newcommand{\opt}{\textsc{dir2}}
\newcommand{\perf}{\textsc{pfv}}
\newcommand{\pl}{\textsc{pl}}
\newcommand{\plposs}{\textsc{pl.poss}}
\newcommand{\poss}{\textsc{poss}}
\newcommand{\pot}{\textsc{pot}}
\newcommand{\prohib}{\textsc{prohib}}
\newcommand{\pst}{\textsc{pst}}
\newcommand{\recip}{\textsc{recip}}
\newcommand{\redp}{\textsc{redp}}
\newcommand{\refl}{\textsc{refl}}
\newcommand{\sg}{\textsc{sg}}
\newcommand{\sgposs}{\textsc{sg.poss}}
\newcommand{\stat}{\textsc{stat}}
\newcommand{\topic}{\textsc{top}}
\newcommand{\volit}{\textsc{vol}}

\newcommand{\racine}[1]{\begin{math}\sqrt{#1}\end{math}} 
\newcommand{\grise}[1]{\cellcolor{lightgray}\textbf{#1}} 
\newcommand{\tinynb}[1]{\tiny#1}
\begin{document}

\title{Alignement et indexation}
\author{Guillaume Jacques\\Anton Antonov}
\maketitle
 

%\section{Indexation accusative}
%
%\begin{exe}
%\ex 
%\glll  \ipa{bāl-o} \ipa{bhar-aty}  \ipa{udak-aṃ} \\
% \ipa{bāl-as} \ipa{bhar-ati} \ipa{udak-am} \\
%enfant-\textsc{nom:sg:m} porter-\textsc{3sg:prs:ind} eau-\textsc{nom/acc:n}\\
%\glt L'enfant porte de l'eau.
%\end{exe}
%
%\begin{exe}
%\ex 
%\glll  \ipa{bāl-o}  \ipa{bhram-ati} \\
%\ipa{bāl-as}  \ipa{bhram-ati} \\
%enfant-\textsc{nom:sg:m} rôder-\textsc{3sg:prs:ind}   \\
%\glt L'enfant rôde.
%\end{exe}
%
%\begin{exe}
%\ex 
%\glll  \ipa{bāl-ā}  \ipa{bhar-anty} \ipa{udak-aṃ} \\
%\ipa{bāl-ās} \ipa{bhar-anti} \ipa{udak-am} \\
% enfant-\textsc{nom:pl:m} porter-\textsc{3pl:prs:ind} eau-\textsc{nom/acc:n}\\
%\glt Les enfants portent de l'eau.
%\end{exe}
%
%\begin{exe}
%\ex 
%\glll  \ipa{bālā} \ipa{bhram-anti}  \\
%\ipa{bāl-ās} \ipa{bhram-anti}  \\
%enfant-\textsc{nom:pl:m}  rôder-\textsc{3pl:prs:ind}  \\
%\glt Les enfants rôdent.
%\end{exe}
 
 \section{Indexation ergative}
%\begin{exe}\ex
%\gll \ipa{Hau} 	\ipa{neska} 	\ipa{polit-a} 	\ipa{da} \\
%{\textsc{dem}} {fille} {joli-\textsc{det}} {être:\textsc{prs}:\textsc{3sg}}\\
%\trans  C'est une jolie fille. 
%\end{exe}
%
%\begin{exe}\ex
%\gll  \ipa{Sorgin-ak} 	\ipa{neska} 	\ipa{polit-a} 	\ipa{du}  \\
%{sorcière-\textsc{det.erg}} {fille} {joli-\textsc{det}} {avoir:\textsc{prs}:\textsc{3sg$\rightarrow$3sg}}\\
%\trans La sorcière a une jolie fille 
%\end{exe}



\begin{exe}
\ex 
\label{ex:1a}
\gll (\ipa{Ni}) \ipa{Bilbo-tik} \ipa{na-tor}\\
(\textsc{1sg.abs}) {Bilbao-\textsc{elat}} {\textsc{1sg}:S/P-come}\\ %marginpar{1}
\trans Je viens de Bilbao.
\end{exe}

\begin{exe}
\ex 
\label{ex:1a}
\gll (\ipa{Kepa}) \ipa{Bilbo-tik} \ipa{da-tor}\\
(Kepa.\textsc{abs}) {Bilbao-\textsc{elat}} {\textsc{3sg}:S/P-venir}\\ %marginpar{3}
\trans (Kepa) Il/Elle vient de Bilbao.
\end{exe}


\begin{exe}
\ex 
\label{ex:1a}
\gll (\ipa{Kepa-k}) (\ipa{ni}) \ipa{Bilbo-tik} \ipa{na-rama}\\
(Kepa-\textsc{erg}) {\textsc{1sg.abs}} {Bilbao-\textsc{elat}} {\textsc{1sg}:S/P-ramener}\\ %marginpar{3>1}
\trans (Kepa) Il/elle me ramène de Bilbao.
\end{exe}

\begin{exe}
\ex 
\label{ex:1a}
\gll (\ipa{Ni-k}) (\ipa{Kepa}) \ipa{Bilbo-tik} \ipa{da-rama-t}\\
({\textsc{1sg-erg}}) (Kepa.\textsc{abs}) {Bilbao-\textsc{elat}} {\textsc{3sg}:S/P-ramener-\textsc{1sg:A}}\\ %marginpar{3>1}
\trans Je le/la ramène (Kepa) de Bilbao.
\end{exe}


\begin{table}[H]
\caption{The present paradigm of \ipa{eraman} `bring' and \ipa{etorri} `arrive'}
\centering
 \resizebox{\columnwidth}{!}{
\begin{tabular}{*7{l}}  
 \toprule %\cline{1-7}
   &  1\sg  &  1\pl  & 2\sg & 2\pl & 3 \sg & 3 \pl  \\  
\toprule 1\sg  &   \multicolumn{2}{c}{\cellcolor{lightgray}} & \ipa{ha-rama-t} & \ipa{za-rama-tza/zte-t} & \ipa{da-rama-t} & \ipa{da-rama-tza-t}\\  
1\pl  &  \multicolumn{2}{c}{\cellcolor{lightgray}}  & \ipa{ha-rama-gu} & \ipa{za-rama-tza/zte-gu} & \ipa{da-rama-gu} & \ipa{da-rama-tza-gu} \\  
2\sg &\ipa{na-rama-k/n} & \ipa{ga-rama-tza-k/n} & \multicolumn{2}{c}{\cellcolor{lightgray}}  & \ipa{da-rama-k/n} & \ipa{da-rama-tza-k/n}\\ 
2\pl &\ipa{na-rama-zu(e)} & \ipa{ga-rama-tza-zu(e)} & \multicolumn{2}{c}{\cellcolor{lightgray}}  & \ipa{da-rama-zu(e)} & \ipa{da-rama-tza-zu(e)}\\ 
3\sg &  \ipa{na-rama} & \ipa{ga-rama-tza} & \ipa{ha-rama} & \ipa{za-rama-tza/zte}  & \ipa{da-rama} & \ipa{da-rama-tza}\\
3\pl &  \ipa{na-rama-te} & \ipa{ga-rama-tza-te} & \ipa{ha-rama-te} & \ipa{za-rama-tza/zte-te}  & \ipa{da-rama-te} & \ipa{da-rama-tza-te}\\\\ 
\midrule
\textsc{intr} &  \ipa{na-tor} & \ipa{ga-to-z} & \ipa{ha-tor} & \ipa{za-to-z(te)}  & \ipa{da-tor} & \ipa{da-to-z}\\
\bottomrule
\end{tabular}}
\end{table}
\section{Intransitivité scindée}
 

\begin{exe}
\ex 
\gll \ipa{Wakíŋyaŋ} \ipa{ni-kté-pi} \\
    oiseau-tonnerre  2:P-tuer-\textsc{pl} \\
\glt Les oiseaux-tonnerre t'ont tué.
\end{exe}


\begin{exe}
\ex 
\gll    \ipa{mitĥ-óyate}  \ipa{kiŋ}  \ipa{oyás'iŋ} \ipa{wičhá-ya-kte-pi}  \\
\textsc{1sg.poss}-peuple \textsc{def} tout \textsc{coll:anim}:S/P-2:A-tuer-\textsc{pl} \\
\glt Vous avez tué tout mon peuple.
\end{exe}

\begin{table}[H]
\caption{Les verbes \ipa{kté} `tuer', \ipa{hí} `venir' et \ipa{káŋ} `être vieux' en lakhota} \centering
\begin{tabular}{l|lllllllllllll}
\toprule
&	\textsc{1sg }  &	 	\textsc{2sg }  &	 \textsc{3sg }    &	\\
	\midrule
\textsc{1sg }  &	\grise{} &	\ipa{čhi-kté} &		\ipa{wa-kté} &	 &	\\
\textsc{2sg }  &	\ipa{ma-yá-kte} &	\grise{} &		\ipa{ya-kté} &		\\
\textsc{3sg }  &	\ipa{ma-kté} &		\ipa{ni-kté} &		\ipa{kté} &		\\
	\midrule
actif & \ipa{wa-hí} &\ipa{ya-hí} & \ipa{hí} & \\
		\midrule
statif &  \ipa{ma-káŋ} &\ipa{ni-káŋ} & \ipa{káŋ} & \\
	\bottomrule
\end{tabular}
\end{table}

\begin{table}[H]
\resizebox{\columnwidth}{!}{
\begin{tabular}{l|lllllllllllll}
\toprule
&	\textsc{1sg }  &	\textsc{1pl }  &	\textsc{2sg }  &	\textsc{2pl }  &	\textsc{3sg }  &	\textsc{3pl}  &	\\
	\midrule
\textsc{1sg }  &	\grise{} &	\grise{} &	\ipa{čhi-kté} &	\ipa{čhi-kté-pi} &	\ipa{wa-kté} &	\ipa{wičhá-wa-kte} &	\\
\textsc{1pl }  &	\grise{} &	\grise{} &	\ipa{uŋ-ní-kte-pi}  &	\ipa{uŋ-ní-kte-pi} &	\ipa{uŋ-kté-pi} &	\ipa{wičhá-uŋ-kte-pi} &	\\
\textsc{2sg }  &	\ipa{ma-yá-kte} &	\ipa{uŋ-yá-kte-pi} &	\grise{} &	\grise{} &	\ipa{ya-kté} &	\ipa{wičhá-ya-kte} &	\\
\textsc{2pl }  &	\ipa{ma-yá-kte-pi} &	\ipa{uŋ-yá-kte-pi} &	\grise{} &	\grise{} &	\ipa{ya-kté-pi} &	\ipa{wičhá-ya-kte-pi} &	\\
\textsc{3sg }  &	\ipa{ma-kté} &	\ipa{uŋ-kté-pi} &	\ipa{ni-kté} &	\ipa{ni-kté-pi} &	\ipa{kté} &	\ipa{wičhá-kte} &	\\
\textsc{3pl}  &	\ipa{ma-kté-pi} &	\ipa{uŋ-kté-pi} &	\ipa{ni-kte-pi} &	\ipa{ni-kté-pi} &	\ipa{kté-pi} &	\ipa{wičhá-kte-pi} &	\\
		\midrule
	actif & \ipa{wa-hí} & \ipa{uŋ-hí-pi} &\ipa{ya-hí} &\ipa{ya-hí-pi} & \ipa{hí} &\ipa{hí-pi} & \\
		\midrule
		statif &  \ipa{ma-káŋ} &\ipa{uŋ-káŋ-pi} &\ipa{ni-káŋ} &\ipa{ni-káŋ-pi} & \ipa{káŋ} &\ipa{káŋ-pi} / \ipa{wičhá-kaŋ} \\
	\bottomrule
\end{tabular}}
\end{table}

\begin{exe}
\ex 
\gll    \ipa{íŋ<ma>skokeča} \\
<\textsc{1sg:P}>être.aussi.grand  \\
\glt Je suis aussi grand que lui.
\gll    \ipa{íŋ<ni-ma>skokeča} \\
<\textsc{2sg:P-1sg:P}>être.aussi.grand  \\
\glt Je suis aussi grand que toi
\end{exe}



\end{document}