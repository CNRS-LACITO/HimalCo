\documentclass[oldfontcommands,twoside,a4paper,12pt]{article} 
\usepackage{fontspec}
\usepackage{natbib}
\usepackage{booktabs}
\usepackage{xltxtra} 
\usepackage{longtable}
\usepackage{tangutex2} 
%\usepackage{tangutex4} 
\usepackage{polyglossia} 
\usepackage[table]{xcolor}
\usepackage{color}
\usepackage{multirow}
\usepackage{gb4e} 
\usepackage{multicol}
\usepackage{graphicx}
\usepackage{float}
\usepackage{hyperref} 
\hypersetup{bookmarks=false,bookmarksnumbered,bookmarksopenlevel=5,bookmarksdepth=5,xetex,colorlinks=true,linkcolor=blue,citecolor=blue}
\usepackage{memhfixc}
\usepackage{lscape}
\usepackage[footnotesize,bf]{caption}


%%%%%%%%%%%%%%%%%%%%%%%%%%%%%%%
\setmainfont[Mapping=tex-text,Numbers=OldStyle,Ligatures=Common]{Charis SIL} 
\setsansfont[Mapping=tex-text,Ligatures=Common,Mapping=tex-text,Ligatures=Common,Scale=MatchLowercase]{Lucida Sans Unicode} 
 


\newfontfamily\phon[Mapping=tex-text,Ligatures=Common,Scale=MatchLowercase,FakeSlant=0.3]{Charis SIL} 
\newfontfamily\phondroit[Mapping=tex-text,Ligatures=Common,Scale=MatchLowercase]{Doulos SIL} 
\newcommand{\ipa}[1]{{\phon\textbf{#1}}} 
\newcommand{\ipab}[1]{{\phon #1}}
\newcommand{\ipapl}[1]{{\phondroit #1}} 
\newcommand{\captionft}[1]{{\captionfont #1}} 
\newfontfamily\cn[Mapping=tex-text,Ligatures=Common,Scale=MatchUppercase]{MingLiU}%pour le chinois
\newcommand{\zh}[1]{{\cn #1}}
\newcommand{\tgf}[1]{{\large\mo{#1}}}

\newcommand{\acc}{\textsc{acc}}
\newcommand{\antierg}{\textsc{antierg}}
\newcommand{\allat}{\textsc{all}}
\newcommand{\aor}{\textsc{aor}}
\newcommand{\assert}{\textsc{assert}}
\newcommand{\auto}{\textsc{auto}}
\newcommand{\caus}{\textsc{caus}}
\newcommand{\classif}{\textsc{class}}
\newcommand{\concessif}{\textsc{concsf}}
\newcommand{\comit}{\textsc{comit}}
\newcommand{\conj}{\textsc{conj}}
\newcommand{\const}{\textsc{const}}
\newcommand{\conv}{\textsc{conv}}
\newcommand{\cop}{\textsc{cop}}
\newcommand{\dat}{\textsc{dat}}
\newcommand{\dem}{\textsc{dem}}
\newcommand{\detm}{\textsc{det}}
\newcommand{\dir}{\textsc{dir1}}
\newcommand{\du}{\textsc{du}}
\newcommand{\duposs}{\textsc{du.poss}}
\newcommand{\dur}{\textsc{dur}}
\newcommand{\erg}{\textsc{erg}}
\newcommand{\evd}{\textsc{evd}}
\newcommand{\fut}{\textsc{fut}}
\newcommand{\gen}{\textsc{gen}}
\newcommand{\hypot}{\textsc{hyp}}
\newcommand{\ideo}{\textsc{ideo}}
\newcommand{\imp}{\textsc{imp}}
\newcommand{\impf}{\textsc{ipfv}}
\newcommand{\instr}{\textsc{instr}}
\newcommand{\intens}{\textsc{intens}}
\newcommand{\intrg}{\textsc{intrg}}
\newcommand{\inv}{\textsc{inv}}
\newcommand{\irreel}{\textsc{irr}}
\newcommand{\loc}{\textsc{loc}}
\newcommand{\med}{\textsc{med}}
\newcommand{\negat}{\textsc{neg}}
\newcommand{\neu}{\textsc{neu}}
\newcommand{\nmlz}{\textsc{nmlz}}
\newcommand{\nonps}{\textsc{n.pst}}
\newcommand{\opt}{\textsc{dir2}}
\newcommand{\perf}{\textsc{pfv}}
\newcommand{\pl}{\textsc{pl}}
\newcommand{\plposs}{\textsc{pl.poss}}
\newcommand{\poss}{\textsc{poss}}
\newcommand{\pot}{\textsc{pot}}
\newcommand{\prohib}{\textsc{prohib}}
\newcommand{\pst}{\textsc{pst}}
\newcommand{\recip}{\textsc{recip}}
\newcommand{\redp}{\textsc{redp}}
\newcommand{\refl}{\textsc{refl}}
\newcommand{\sg}{\textsc{sg}}
\newcommand{\sgposs}{\textsc{sg.poss}}
\newcommand{\stat}{\textsc{stat}}
\newcommand{\topic}{\textsc{top}}
\newcommand{\volit}{\textsc{vol}}

\newcommand{\racine}[1]{\begin{math}\sqrt{#1}\end{math}} 
\newcommand{\grise}[1]{\cellcolor{lightgray}\textbf{#1}} 
\newcommand{\tinynb}[1]{\tiny#1}
\begin{document}

\title{Alignement et indexation}
\author{Guillaume Jacques\\Anton Antonov}
\maketitle
 

\section{Indexation accusative}

\begin{exe}
\ex 
\glll  \ipa{bāl-o} \ipa{bhar-aty}  \ipa{udak-aṃ} \\
 \ipa{bāl-as} \ipa{bhar-ati} \ipa{udak-am} \\
enfant-\textsc{nom:sg:m} porter-\textsc{3sg:prs:ind} eau-\textsc{nom/acc:n}\\
\glt L'enfant porte de l'eau.
\end{exe}

\begin{exe}
\ex 
\glll  \ipa{bāl-o}  \ipa{bhram-ati} \\
\ipa{bāl-as}  \ipa{bhram-ati} \\
enfant-\textsc{nom:sg:m} roder-\textsc{3sg:prs:ind}   \\
\glt L'enfant rode.
\end{exe}

\begin{exe}
\ex 
\glll  \ipa{bāl-ā}  \ipa{bhar-anty} \ipa{udak-aṃ} \\
\ipa{bāl-ās} \ipa{bhar-anti} \ipa{udak-am} \\
 enfant-\textsc{nom:pl:m} porter-\textsc{3pl:prs:ind} eau-\textsc{nom/acc:n}\\
\glt Les enfants portent de l'eau.
\end{exe}

\begin{exe}
\ex 
\glll  \ipa{bālā} \ipa{bhram-anti}  \\
\ipa{bāl-ās} \ipa{bhram-anti}  \\
enfant-\textsc{nom:pl:m}  roder-\textsc{3pl:prs:ind}  \\
\glt Les enfants rodent.
\end{exe}
 
 \section{Intransitivité ergative}

\section{Intransitivité scindée}


\begin{exe}
\ex 
\gll   \ipa{wašícu} \ipa{kiŋ} \ipa{matĥó} \ipa{waŋ} \ipa{kté-pi} \\
blancs \textsc{def} ours \textsc{indef} tuer-\textsc{pl} \\
\glt Les blancs ont tué un ours.
\end{exe}

\begin{exe}
\ex 
\gll    \ipa{matĥó} \ipa{kiŋ} \ipa{wašícu} \ipa{kiŋ}  \ipa{wičhá-kte} \\
ours \textsc{def} blanc \textsc{def} \textsc{pl:anim:O}-tuer  \\
\glt L'ours a tué des blancs.
\end{exe}


\begin{table}[H]

\begin{tabular}{l|lllllllllllll}
\toprule
&	\textsc{1sg }  &	 	\textsc{2sg }  &	 \textsc{3sg }    &	\\
	\midrule
\textsc{1sg }  &	\grise{} &	\ipa{ni-wa-kte} &		\ipa{wa-kte} &	 &	\\
\textsc{2sg }  &	\ipa{ma-ya-kte} &	\grise{} &		\ipa{ya-kte} &		\\
\textsc{3sg }  &	\ipa{ma-kte} &		\ipa{ni-kte} &		\ipa{kte} &		\\
	\bottomrule
\end{tabular}
\end{table}

\begin{table}[H]
\resizebox{\columnwidth}{!}{
\begin{tabular}{l|lllllllllllll}
\toprule
&	\textsc{1sg }  &	\textsc{1pl }  &	\textsc{2sg }  &	\textsc{2pl }  &	\textsc{3sg }  &	\textsc{3pl}  &	\\
	\midrule
\textsc{1sg }  &	\grise{} &	\grise{} &	\ipa{ni-wa-kte} &	\ipa{ni-wa-kte-pi} &	\ipa{wa-kte} &	\ipa{wicha-wa-kte} &	\\
\textsc{1pl }  &	\grise{} &	\grise{} &	\ipa{uŋ-ni-kte-pi} &	\ipa{uŋ-ni-kte-pi} &	\ipa{uŋ-kte-pi} &	\ipa{uŋ-wicha-kte-pi} &	\\
\textsc{2sg }  &	\ipa{ma-ya-kte} &	\ipa{uŋ-ya-kte-pi} &	\grise{} &	\grise{} &	\ipa{ya-kte} &	\ipa{wicha-ya-kte} &	\\
\textsc{2pl }  &	\ipa{ma-ya-kte-pi} &	\ipa{uŋ-ya-kte-pi} &	\grise{} &	\grise{} &	\ipa{ya-kte-pi} &	\ipa{wicha-ya-kte-pi} &	\\
\textsc{3sg }  &	\ipa{ma-kte} &	\ipa{uŋ-kte-pi} &	\ipa{ni-kte} &	\ipa{ni-kte-pi} &	\ipa{kte} &	\ipa{wicha-kte} &	\\
\textsc{3pl}  &	\ipa{ma-kte-pi} &	\ipa{uŋ-kte-pi} &	\ipa{ni-kte-pi} &	\ipa{ni-kte-pi} &	\ipa{kte-pi} &	\ipa{wicha-kte-pi} &	\\
	\bottomrule
\end{tabular}}
\end{table}

\section{Alignement secondatif}
 
 
 
 
 
\section{Alignement indirectif}
 

\bibliographystyle{Linquiry2}
\bibliography{bibliogj}
\end{document}