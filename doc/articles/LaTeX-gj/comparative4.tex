% sɯŋgɯ pɤjka nɯ kɯ ɯ-jwaʁ chondɤre, nɤkinɯ, ɯ-jwaʁ nɯra xtɕi.
%locus du comparatif
%tɕethi tɕe, tɯ-ci ɯ-ŋgɯ ɕ-pɯ-ru tɕe, ɕu kɯ ɲɯ-mpɕɤr-tɕi
%445

%grammaticalization pathway:
%kɤ-ʑɣɤsɤzɣɯt wuma ʑo pjɤ-nɤɴqa ri kɯ-maqhu tɕe, 
%ɯ-tɯ-nɯrɟɯrŋom ɯ-xɕɤt kɯ jo-ʑɣɤsɤzɣɯt.

\documentclass[oldfontcommands,oneside,a4paper,11pt]{article} 
\usepackage{fontspec}
\usepackage{natbib}
\usepackage{booktabs}
\usepackage{xltxtra} 
\usepackage{polyglossia} 
\usepackage[table]{xcolor}

\usepackage{multicol}
\usepackage{graphicx}
\usepackage{lineno}
\usepackage{float}
\usepackage{hyperref} 
\hypersetup{bookmarks=false,bookmarksnumbered,bookmarksopenlevel=5,bookmarksdepth=5,xetex,colorlinks=true,linkcolor=blue,citecolor=blue}
\usepackage[all]{hypcap}
\usepackage{memhfixc}
\usepackage{lscape}
\usepackage{tikz}
%
\usetikzlibrary{trees}
\usepackage{gb4e} 
\bibpunct[: ]{(}{)}{,}{a}{}{,}
 
%\setmainfont[Mapping=tex-text,Numbers=OldStyle,Ligatures=Common]{Charis SIL}  
\newfontfamily\phon[Mapping=tex-text,Ligatures=Common,Scale=MatchLowercase,FakeSlant=0.3]{Charis SIL} 
\newcommand{\ipa}[1]{{\phon #1}} %API tjs en italique
 
\newcommand{\grise}[1]{\cellcolor{lightgray}\textbf{#1}}
\newfontfamily\cn[Mapping=tex-text,Ligatures=Common,Scale=MatchUppercase]{MingLiU}%pour le chinois
\newcommand{\zh}[1]{{\cn #1}}
   



\begin{document} 
%\linenumbers

\title{From ergative to index of comparison: multiple reanalyses and polyfunctionality\footnote{The glosses follow the Leipzig glossing rules. Other abbreviations used here are:   \textsc{assert} assertive, \textsc{auto} autobenefactive / spontaneous,  \textsc{dem} demonstrative, \textsc{dist} distal, \textsc{emph} emphatic, \textsc{indef} indefinite, \textsc{inv} inverse,  \textsc{lnk} linker, \textsc{pfv} perfective, \textsc{poss} possessor, \textsc{fact} factual,  \textsc{sens} sensory.} }
%Giorgio Arcodia, Lauren Gawne, Alexis Michaud, Amos Teo, Nicolas Tournadre, Fernando Zúñiga, +2 anonymous

\author{Guillaume JACQUES}
\maketitle
   \sloppy
\textbf{Abstract}: This paper describes the uses of the marker  \ipa{kɯ} in Japhug, which presents many distinct functions, including ergative, instrumental, distributive, causal linker, manner linker, emphatic adversative, interrogative sentence final particle and index of the comparee in the comparative construction. A series of grammaticalization pathways, some of which have never been documented before, are proposed to account for the polyfunctionality of this marker.

\textbf{Keywords}: comparative; ergative; instrumental; clause linking; adversative; distributive; Japhug
 
% Acknowledgements: Anton Antonov,  Enrique Palancar, Mathieu Ségui,
 \section{Introduction}
  
 
 
In Japhug,\footnote{Japhug is a language spoken in Mbarkham county, Rngaba prefecture, Sichuan province China, by under 10000 speakers. Together with Situ, Zbu and Tshobdun, it belongs to the Rgyalrong subgroup of the Sino-Tibetan family (see \citealt{jackson00sidaba}), and is known for its polysynthetic morphology (see \citealt{jacques13harmonization} and  \citealt{jacques14antipassive}) and its elaborate consonant clusters. A small grammar  (\citealt{jacques08zh}) and a text collection (\citealt{jacques10gesar}) are available.  This research is based on oral narratives and conversations collected by the author (the total corpus comprises more than 50 hours of transcribed recordings), which is being made progressively available on the Pangloss Archive (\citealt{michailovsky14pangloss}). } clitic markers of the form \ipa{kɯ} appear in a wide array of different constructions. The most common one is the ergative marker, but we also find homophonous markers with syntactic functions as varied as distributive, causal linker  and  index of the comparative construction. 

Syncretisms between agent or ergative markers  on the one hand and various spatial cases on the other hand are common typologically (\citealt{agent02palancar}). However, the apparent polyfunctionality of the marker \ipa{kɯ} in Japhug, if indeed one marker has to be posited, is without equivalents. It is unclear at first glance to what extent these various uses are related synchronically and diachronically, and whether one or several distinct markers have to be described. 
 
 In order to avoid circular reasoning and to clearly  separate the exposition of the data from the hypotheses, the synchronic and diachronic aspects of the topic are treated separately in this paper, which is divided into three main sections.

The first part of the paper is devoted to the synchronic description of the various functions of the marker \ipa{kɯ} besides its use as an ergative / instrumental marker, including its function in a distributive construction, in manner and cause clause linking, in the degree construction and in the comparative construction. 


The second part discusses the etymology of the marker \ipa{kɯ}. A detailed overview of all markers and grammatical morphemes that are phonologically similar to \ipa{kɯ} is provided, including external comparisons with other Sino-Tibetan languages.  The marker is demonstrated to be borrowed from the Amdo Tibetan ergative \ipa{kə/ɣə}.

  The third part of the paper, building on the two previous sections, presents a series of historical scenarios showing the links between each single function of the marker \ipa{kɯ}, in particular the tortuous path from ergative / instrumental to index of comparison.


\section{Synchronic functions of the marker \ipa{kɯ}} \label{sec:synchronic}
The marker \ipa{kɯ} has five distinct functions described in detail in the present section. Besides its use as an Ergative / Instrumental marker, \ipa{kɯ} is found in distributive, clause linking, degree and comparative constructions. 

The use of \ipa{kɯ} in the comparative construction stands out in being typologically unusual.  While many languages, including Tibetan, have Ergative / Instrumental markers in comparative constructions, they are typically used to mark the \textit{standard} of comparison. In Japhug, as we show below, it is rather the \textit{comparee} which is unexpectedly marked with \ipa{kɯ}; a historical explanation for this fact is presented in section \ref{sec:historical}.

\subsection{Ergative / Instrumental}
Japhug has a very clear   distinction between transitive and intransitive verbs, which is reflected in both case marking and verbal morphology.\footnote{An account of argument indexation and morphological transitivity marking in Japhug goes beyond the cope of this paper; see \citet{jacques10inverse} for a detailed description.}

Japhug has ergative alignment on noun phrases; S and P are unmarked (examples \ref{ex:abs} and \ref{ex:erg} respectively) , while the A of transitive verbs receives the clitic \ipa{kɯ} (example \ref{ex:erg}). This clitic is obligatory with nouns and third person pronouns, but optional for first and second person pronouns, since  the verb agreement morphology distinguishes between agent and patient in a non-ambiguous way.\footnote{In all examples containing the marker \ipa{kɯ} in this paper, we indicate with square brackets the constituent over which it has syntactic scope.}

\begin{exe}
\ex \label{ex:abs}
\gll \ipa{tɤ-tɕɯ}  	\ipa{nɯ}  	 	\ipa{jo-ɕe}   \\
\textsc{indef.poss}-boy \textsc{dem}   \textsc{ifr}-go \\
\glt `The boy went (there).'
\end{exe}

\begin{exe}
\ex \label{ex:erg}
\gll [\ipa{tɤ-tɕɯ}  	\ipa{nɯ}]  	\ipa{\textbf{kɯ}} 	\ipa{χsɤr}  	\ipa{qaɕpa}  	\ipa{nɯ}  	\ipa{cʰɤ-mqlaʁ}   \\
\textsc{indef.poss}-boy \textsc{dem} \textsc{erg} gold frog \textsc{dem} \textsc{ifr}-swallow \\
\glt `The boy swallowed the golden frog.' (Nyima Wodzer.1, 131)
\end{exe}


Additionally, as in many languages, the ergative is also used as an instrumental marker (\citealt[32]{agent02palancar}). When an  instrumental phrase with 	\ipa{kɯ}  appears in the sentence, the verb is generally marked with the causative prefix \ipa{sɯ}-- / \ipa{z}--. It is possible  to have sentences with two noun phrases marked \ipa{kɯ} (one A and one instrument) as in \ref{ex:instr3}, but such examples are extremely rare in our corpus.

 \begin{exe}
\ex \label{ex:instr}
\gll \ipa{kɯɕɯŋgɯ}   	\ipa{tɕe,}   	[\ipa{rɤɣo}]   	\ipa{\textbf{kɯ}}   	\ipa{tɯrju}   	\ipa{tu-sɯ-βzu-nɯ}   	\ipa{pjɤ-ŋu}   	\ipa{tɕe,}     \\
before \textsc{lnk} song \textsc{erg/instr} word \textsc{ipfv}-\textsc{caus}-do-\textsc{pl} \textsc{ipfv:ifr}-be \textsc{lnk}\\
\glt  `In old times, people used to speak by song.' (Gesar, 37)
\end{exe}
 \begin{exe}
\ex \label{ex:instr2}
\gll [\ipa{nɯ-mtsioʁ}   	\ipa{nɯ}]   	\ipa{\textbf{kɯ}}   	\ipa{lu-sɯ-lɣa-nɯ}   	\ipa{qʰe,}   	\ipa{tɤtsoʁ}   	\ipa{lu-nɯ-tɕɤt-nɯ}   	\ipa{ɲɯ-ŋgrɤl}   	\ipa{kʰi}        \\
\textsc{3pl.poss}-beak \textsc{dem} \textsc{erg/instr} \textsc{ipfv-caus}-dig-\textsc{pl} \textsc{lnk} potentilla.anserina \textsc{ipfv-auto}-take.out-\textsc{pl} \textsc{ipfv}-be.usually.the.case \textsc{hearsay} \\
\glt  `(The wild geese) dig (the ground) with their beaks, and they take out Potentilla anserina roots.' (Wild Geese 28)
\end{exe}

\begin{exe}
\ex \label{ex:instr3}
\gll
 \ipa{sɯŋgi} 	\ipa{kɯ} 	\ipa{ɯ-jaʁ} 	\ipa{kɯ} 	\ipa{rcanɯ,} 	\ipa{ɯ-rŋa} 	\ipa{ra} 	\ipa{tɯ-mɯrʁɯz} 	\ipa{ʑo} 	\ipa{pjɤ-sɯ-lɤt} 	\ipa{tɕe} \\
 lion \textsc{erg} \textsc{3sg.poss}-hand \textsc{erg/instr} \textsc{top} \textsc{3sg.poss}-face \textsc{pl} \textsc{nmlz:action}-scratch \textsc{emph} \textsc{ifr-caus}-throw \textsc{lnk}\\
\glt The lion scratched its (own) face with its paw. (The lion and the mosquito, 23)
\end{exe}
Causative marking on the verb with an instrument is not compulsory, and one can find the two constructions with or without the causative marker side by side in the same narrative:

\begin{exe} 
\ex \label{ex:instr3}
\gll   [\ipa{qartsʰaz}   	\ipa{ɯ-ndʐi}]   	\ipa{\textbf{kɯ}}   	\ipa{cʰɯ-βzu-nɯ}   	\ipa{tɕe,}   	\ipa{nɯ}   	\ipa{stu}   	\ipa{kɯ-ʑru.}     \\
 deer \textsc{3.sg.poss}-hide \textsc{erg/instr} \textsc{ipfv}-do-\textsc{pl} \textsc{lnk} \textsc{dem} most \textsc{nmlz:S/A}-strong \\
 \glt   `They make (shoes) with deer hide, it is the most resistant (type of skin).' (Shoes, 48)
\end{exe} 

 \begin{exe} 
\ex \label{ex:instr4}
\gll   [\ipa{qartsʰaz}   	\ipa{ɯ-ndʐi}   	\ipa{ʁɟa}]   	\ipa{\textbf{kɯ}}   	\ipa{ʑo}   	\ipa{tʰɯ-kɤ-sɯ-βzu}\\
 deer \textsc{3.sg.poss}-hide entirely \textsc{erg/instr} \textsc{emph} \textsc{pfv-nmlz:P-caus}-do\\
 \glt   `(It is) entirely made of deer hide.' (Shoes, 53)
\end{exe} 


While the instrument and the agent are both marked by \ipa{kɯ}, their syntactic status is different, as shown by   relativization.

The agent is relativized in prenominal or head-internal relative clauses with the main verb in the S/A-participle (marked by the prefix  \ipa{kɯ--})   with an  additional possessive prefix coreferent with the P, as \ipa{ɯ--} in \ipa{ɯ-kɯ-nɯmbrɤpɯ} `the one who rides it' in \ref{ex:WkWnWmbrApW}.

\begin{exe}
   \ex  \label{ex:WkWnWmbrApW}
\gll [[\ipa{tɤpɤtso}  	\ipa{ci}]  	\ipa{kɯ}  	<yangma> 	\ipa{ɯ-kɯ-nɯmbrɤpɯ}]   	\ipa{ci}  	\ipa{jɤ-ɣe}  \\
boy \textsc{indef} \textsc{erg} bicycle \textsc{3sg-nmlz:A}-ride \textsc{indef} \textsc{pfv}-come[II] \\
\glt `A boy who was riding a bicycle arrived.' (Pear story, Chenzhen, 5)
\end{exe}

On the other hand, the instrument is relativized in prenominal relative clauses with the verb in the oblique participle (with the prefix \ipa{sɤ}--). The  causative prefix \ipa{sɯ}-- is also removed in this form, as illustrated by example \ref{ex:sAxtCAr}.  

 \begin{exe}
  \ex  \label{ex:sAxtCAr}  
  \gll [\ipa{nɯ-mtʰɤɣ}  	\ipa{sɤ-xtɕɤr}]  	\ipa{xɕɤfsa}  	\ipa{ma}  	\ipa{pjɤ-me}  \\
\textsc{3pl.poss}-waist \textsc{nmlz:oblique}-tie thread apart.from \textsc{ifr.ipfv}-not.exist \\
\glt `They only had threads to tie their waists (the only things that they could use to tie their waist were threads).' (Milaraspa translation)
   \end{exe} 
   
 
 The marker \ipa{kɯ} can be used with   abstract nouns to indicate cause, as in \ref{ex:sWmWzdWG.kW}, where \ipa{sɯmɯzdɯɣ}  `worry' is a noun borrowed from Tibetan \ipa{sems.sdug} (same meaning). 

\begin{exe}
\ex \label{ex:sWmWzdWG.kW}
\gll 
[\ipa{sɯmɯzdɯɣ}]  	\ipa{kɯ}  	\ipa{ɕɤr}  	\ipa{ɯ-ʑɯβ}  	\ipa{mucin}  	\ipa{mɯ-pɯ-ɣe}  	\ipa{ɲɯ-ŋu.}  	 \\
worry  \textsc{erg/instr} night \textsc{3sg.poss}-sleep at.all \textsc{neg-pfv}-come[II] \textsc{sens}-be \\
\glt `As she was worried, she could not get any sleep for the whole night.' (Slobdpon, 174)
\end{exe}


  It also appears in a clausal construction with the possessed noun `its strength' \ipa{ɯ-xɕɤt} borrowed from Tibetan \ipa{ɕed} `strength'.\footnote{This borrowed is specifically from Amdo, as it reflect an intermediate stage in the sound change from [\ipa{ɕet}] to [\ipa{xɕit}] as in most Amdo varieties.}

 

\begin{exe}
\ex \label{ex:WxCAt.kW}
\gll
[\ipa{tɯ-mŋɤm} 	\ipa{ɯ-xɕɤt}] 	\ipa{kɯ} 	\ipa{aʑo} 	\ipa{nɯ} 	\ipa{a-ku} 	\ipa{ɕɤrɯ} 	\ipa{pjɤ-ɣɤtsɯr} 	\ipa{ɲɯ-ŋu} 	\ipa{nɯ-sɯso-t-a.} 	\\
\textsc{nmlz:action}-hurt \textsc{3sg.poss}-strength \textsc{erg/instr} \textsc{1sg} \textsc{dem} \textsc{1sg.poss}-head bone \textsc{ifr}-have.a.crack \textsc{sens}-be \textsc{pfv}-think-\textsc{pst-1sg} \\
\glt `Because of the pain, I felt as though my skull had cracked.' (Headache, 77)
  \end{exe}

  \begin{exe}
\ex \label{ex:WxCAt.kW2}
\gll
  [\ipa{tɤ-zdɯɣ} 	\ipa{ɯ-xɕɤt}] 	\ipa{kɯ} 	\ipa{pjɯ-si} 	\ipa{ɕti,} \\
  \textsc{indef.poss}-toil  \textsc{3sg.poss}-strength \textsc{erg/instr} \textsc{ipfv}-die be:\textsc{affirm}:\textsc{fact} \\
 \glt `(The bee) dies of exhaustion.' (Bee, 40)
  \end{exe}
  
In this construction, we either find  an abstract noun as in \ref{ex:WxCAt.kW2}  or an action nominal in \ipa{tɯ--} (as in \ref{ex:WxCAt.kW}). 

Another causal construction in \ipa{kɯ} involves the noun \ipa{tʰɯrʑi} `mercy' (from Tibetan \ipa{tʰugs.rdʑe}), as in examples \ref{ex:WthWrZi.kW}.

  \begin{exe}
\ex \label{ex:WthWrZi.kW}
\gll
\ipa{tɕe}  	[\ipa{tɯrpa}  	\ipa{ɯ-tʰɯrʑi}]  	\ipa{kɯ}  	\ipa{nɯ}-<shenghuo>  	\ipa{nɯ} \ipa{ra}  	\ipa{wuma}  	\ipa{ʑo}  	\ipa{pjɤ-pe}  	\ipa{ɲɯ-ŋu.}  \\
\textsc{lnk} axe \textsc{3sg.poss}-mercy \textsc{erg/instr} \textsc{3pl.poss}-life \textsc{dem} \textsc{pl} really \textsc{emph} \textsc{ifr.ipfv}-be.good \textsc{sens}-be \\
\glt `Thanks to the axes, their life was very good.' (The little village, 35)
  \end{exe}
  
  

 Outside of its use as agent, instrumental or causal marker, \ipa{kɯ} is also selected by the transitive verb \ipa{kʰɤt} ``to do several times, a long time", where an abstract noun or an action nominalization with the ergative indicates the action realized.
 \begin{exe} 
\ex \label{ex:instr5}
\gll   [\ipa{tɯ-qioʁ}]   	\ipa{\textbf{kɯ}}   	\ipa{tó-wɣ-sɯ-kʰɤt}   	\ipa{ʑo}   	\ipa{tɕe,}       \\
  \textsc{nmlz:action}-vomit \textsc{erg} \textsc{ifr-inv-caus}-do.a.long.time \textsc{emph} \textsc{lnk}\\
 \glt  `(The medicine) made him vomit a long time.' (Gesar, 266)
\end{exe}  
 
 
 % tɕe  tɯʑo kɯ tɯʑo  tukɯnɯʑɣɤβri ra kɤti ɲɯŋu
\subsection{Distributive} \label{sec:distributive}
The marker \ipa{kɯ} can have   a distributive meaning (`for one X', `per') when used with a classifier designating a quantity. It occurs  in constructions with intransitive verbs where no agent or instrument is present, but exclusively to express the price of the quantity designated, as in  \ref{ex:tWtWrpa.kW1} and \ref{ex:tWtWrpa.kW2}.

 \begin{exe} 
\ex \label{ex:tWtWrpa.kW1}
\gll  
[\ipa{tɯ-tɯrpa}] 	\ipa{\textbf{kɯ}} 	\ipa{sqi} 	\ipa{jamar} 	\ipa{ɲɯ-ra.} 	\\
one-pound \textsc{erg} ten about \textsc{sens}-have.to \\
\glt `You need ten (yuans) per pound (of Angelica).' (Angelica, 22)
\end{exe}  

 \begin{exe} 
\ex \label{ex:tWtWrpa.kW2}
\gll  
[\ipa{tɯ-tɯrpa}]  	\ipa{\textbf{kɯ}}  	\ipa{ɣurʑa}  	\ipa{ɯ-ro,}  	\ipa{ɯ-pʰɯ}  	\ipa{ɲɯ-ɣi.}  \\
one-pound \textsc{erg} hundred \textsc{3sg.poss}-more \textsc{3sg.poss}-price \textsc{sens}-come \\
\glt `It costs more than one hundred (yuans) per pound.' (Matsutake, 5)
\end{exe}  

  \begin{exe} 
\ex \label{ex:tWtCha.kW}
\gll  
[\ipa{tɯ-xtsa} 	\ipa{\textbf{tɯ-tɕʰa}}] 	\ipa{kɯ} 	\ipa{ɣurʑa} 	\ipa{ɯ-ro} 	\ipa{ɲɯ-ra} \\
\textsc{indef.poss}-shoe one-pair \textsc{erg} hundred \textsc{3sg.poss}-more \textsc{sens}-have.to \\
 \glt  `It costs more than one hundred for a pair of shoes.' (elicited)
\end{exe}  
The construction with \ipa{kɯ} cannot be used with classifiers expressing duration, as in \ref{ex:tWsla.tCe} (replacing the linker \ipa{tɕe} by \ipa{kɯ} here would be agrammatical).
 
 \begin{exe} 
\ex \label{ex:tWsla.tCe}
\gll  
  \ipa{tɯ-sla} \ipa{tɕe} \ipa{ɯ-ŋgra} \ipa{ɣurʑa} \ipa{ɯ-ro} \ipa{ɣɤʑu} \\
  one-month \textsc{lnk} 3sg.poss-salary hundred \textsc{3sg.poss}-more exist:\textsc{sens} \\
  \glt `His salary is more than one hundred a month.'
  \end{exe}  
 
  
 \subsection{Clause linking} \label{sec:linking}
 The marker \ipa{kɯ} appears in manner clause linking, with the verb of the subordinate clause in either infinitival or finite form (see \citealt{jacques14linking} for additional examples).
 
 \subsubsection{Infinitival manner linking} \label{sec:manner}
We find in Japhug infinitival clauses in clause linking and complementation constructions.

The infinitive of the verb is either \ipa{kɤ--} for (dynamic verbs) or \ipa{kɯ--} (for adjectives, copulas, some modal auxiliaries and intransitive dynamic verbs that are incompatible with an animate S). These prefixes happen to be homophonous with, and probably historically derived from, the P-participle \ipa{kɤ--} and the S/A-participle \ipa{kɯ--} prefixes. Aside from their functional differences, infinitives clearly differ from participles in that dynamic intransitive verbs have an infinitive in \ipa{kɤ--}, but no P-participle \ipa{kɤ--}. For instance the intransitive \ipa{ŋke} `walk' has no P-participle, only a S/A-participle \ipa{kɯ-ŋke} `the one who walks', but its infinitive is \ipa{kɤ-ŋke} and can be used as a manner subordinate clause as in \ref{ex:kANke}.

\begin{exe}
\ex \label{ex:kANke}
\gll
\ipa{kɤ-ŋke} \ipa{jo-ɕe} \\
\textsc{inf}-walk \textsc{ifr}-go \\
\glt She went on foot. (Bailingniao, 215)
\end{exe} 

Bare infinitival subordinate clauses with dynamic verbs as in \ref{ex:kANke} are rare, and almost only attested in our corpus by motion verbs as in \ref{ex:kANke} or by verb of cognition such as `think' or `know'.

 Infinitival subordinate clauses are more commonly used with the marker \ipa{kɯ}, to indicate the circumstance or background situation in which the action described by the verb of the main clause took place, as in \ref{ex:mAkApa.kW}. 

\begin{exe}
\ex \label{ex:mAkApa.kW}
\gll
\ipa{tɕe}   	\ipa{ɯ-ŋgɯ}   	\ipa{nɯ} \ipa{tɕu}   	\ipa{paʁndza}   	\ipa{ɲɤ-raʁ}   	\ipa{tɕe,}   	\ipa{tɕendɤre}   	[<dian>   	<guan>   	\ipa{mɤ-kɤ-βzu}] 	\ipa{\textbf{kɯ}}   	[\ipa{mɤ-kɤ-pa}]   	\ipa{\textbf{kɯ}}   	\ipa{ɯ-jaʁ}   	\ipa{lo-tsɯm}   \\
\textsc{lnk} \textsc{3sg}-inside \textsc{dem} \textsc{loc} pig.fodder \textsc{ifr}-be.stuck \textsc{lnk}
\textsc{lnk} electricity turn.off \textsc{neg-inf}-make \textsc{erg}  \textsc{neg-inf}-close \textsc{erg}  \textsc{3sg.poss}-hand \textsc{ifr:upstream}-take \\
\glt `Some pig fodder got stuck inside (the machine) he put his hand into it without turning it off,' (Relatives, 372-3)
\end{exe} 

 While  examples such as \ref{ex:nWCe.kAsWso} could lead to believe that the infinitival clause marked with \ipa{kɯ}  expresses the cause and the main clause the result, it is clear that this construction does not entail a unidirectional causal relationship between the main and the subordinate clause. In \ref{ex:mAkApa.kW} the events expressed by the main and the subordinate clauses are not causally linked to one another, and in  \ref{ex:mWYWkWYJWr} it is the subordinate clause which expresses the purpose of the action of the main clause (the causality is reversed in comparison with \ref{ex:nWCe.kAsWso}).

  \begin{exe}
\ex \label{ex:nWCe.kAsWso}
\gll 
[\ipa{nɯɕe}  	\ipa{kɤ-sɯso}]  	\ipa{kɯ,}  	\ipa{ɯ-mbro}  	\ipa{nɯnɯ}  	\ipa{taqaβ}  	\ipa{cʰɤ-z-nɯtɕʰaʁ-nɯ,}  	\ipa{ɯ-kʰɯna}  	\ipa{nɯ}  	\ipa{rkorsa}  	\ipa{ɯ-pa}  	\ipa{lo-ja-nɯ}  \\
go.back:\textsc{fact} \textsc{inf}-think \textsc{erg} \textsc{3sg.poss}-horse \textsc{dem} needle \textsc{ifr-caus}-eat-\textsc{pl}   \textsc{3sg.poss}-dog \textsc{dem} toilet \textsc{3sg.poss}-down \textsc{ifr}-pen-\textsc{pl} \\
\glt `Thinking that he (was about to) go back, they fed his horse with needles and penned his dog in the toilets.' (Gesar 250-1)
 \end{exe} 
 
\begin{exe}
\ex \label{ex:mWYWkWYJWr}
\gll
\ipa{tɯ-xtsa}   	\ipa{nɯnɯ}   	[\ipa{ɯ-ʁzɯɣ}   	\ipa{mɯ-ɲɯ-kɯ-ɲɟɯr}]   	\ipa{\textbf{kɯ}}   	\ipa{ɲɯ-z-rɤsta-nɯ}   \\
\textsc{indef.poss}-shoe \textsc{dem} shape \textsc{neg-ipfv-inf:non.hum-anticaus}:change \textsc{erg} \textsc{ipfv-caus}-be.fixed \\
\glt `They wedge the shoes (with a shoe tree)  in such a way that their shape does not change.' (\ipa{komar}, 109)
\end{exe}

\subsubsection{Finite manner linking}
In the finite manner clause linking, the subordinate clause is likewise marked by \ipa{kɯ}, but its verb is in finite form. The  finite  main clause adds further information on the state or situation described by the subordinate clause, as in \ref{ex:mWjfse.kW}. 
 

 \begin{exe}
\ex \label{ex:mWjfse.kW}
\gll
\ipa{ri} 	\ipa{ɯ-jwaʁ} 	\ipa{nɯnɯ} 	[\ipa{kɯmaʁ} 	\ipa{ɕɤɣ} 	\ipa{nɯ} \ipa{ra} 	\ipa{mɯ́j-fse}] 	\ipa{\textbf{kɯ}} 	\ipa{ɲɯ-ɤrʁɯrʁu} 	\ipa{ʑo} 	\ipa{qʰe} 	\ipa{ɲɯ-ɤndɯndo} 	\ipa{ʑo.} \\
\textsc{lnk} \textsc{3sg.poss}-leaf \textsc{dem} other juniper \textsc{dem} \textsc{pl} \textsc{neg:sens}-be.like \textsc{erg?} \textsc{sens}-be.wrinkled \textsc{emph} \textsc{lnk}  \textsc{sens}-be.clustered.together \textsc{emph} \\
\glt `Its leaves differ from other junipers in that they are wrinkled and clustered together.' (Ephedra, 71)
 \end{exe}
 
Semantically,  the infinitival and the finite manner clause linkings are quite distinct. In the former, the subordinate clause presents background information (whether a circumstance, a cause or even a purpose) on the main clause. In the latter, on the other hand, the subordinate clause preceding \ipa{kɯ} indicates the main event / state of affair, and the main clause represents an additional characterization of this event.

 

 \subsubsection{Emphatic adversative} \label{sec:advers}
Combined with the copula \ipa{maʁ} `not be' in finite form or with a verb negative form, the marker \ipa{kɯ} is also used to express adversative meaning between the subordinate clause the main clause (example \ref{ex:YWmaR.kW}).
 
  \begin{exe} 
 \ex \label{ex:YWmaR.kW}
\gll [\ipa{tɯrgi} 	\ipa{kɯ-fse} 	\ipa{ɯ-stu} 	\ipa{tu-ɕe} 	\ipa{ɲɯ-maʁ}] 	\ipa{kɯ,} \ipa{aʁɤndɯndɤt} 	\ipa{ɯ-rtaʁ} 	\ipa{ɲɯ-ɬoʁ} 	\ipa{ɲɯ-ŋu} 	\ipa{tɕe,} 
\\
fir \textsc{inf:stat}-be.like \textsc{3sg.poss}-straight \textsc{ipfv:up}-go \textsc{sens}-not.be \textsc{erg}? everywhere \textsc{3sg.poss}-branch \textsc{sens}-come.out \textsc{sens}-be \textsc{lnk} \\ 
\glt `It does not grow straight like the fir, on the contrary, its branches spread out in all directions.'  
(maldo, 54)
 \end{exe}  
 
 This construction is used to focus on the contrast between the (negated) event/situation described in the subordinate clause, and that of the main clause, and corresponds to the English phrase `on the contrary'.
 
\subsection{Degree construction} \label{sec:degree}
 
 
Japhug has two degree constructions built by nominalizing a verb (generally an adjective)  with the action nominalization \ipa{tɯ--} prefix and a possessive prefix coreferent with the referent presenting the property described by the nominalized verb. 

This degree nominal (like \ipa{ɯ-tɯ-tɕur} `its sourness' in \ref{ex:YWsWxtCur}) is the S in both constructions.

First, the degree nominal can be  combined with an adjective expressing degree such as \ipa{saχaʁ} `be extremely ...', \ipa{sɤre} `be funny, be extremely ...', \ipa{tɕʰom} `be excessive', in the \textit{monoclausal nominalized degree} construction, exemplified by example \ref{ex:YWsWxtCur}.

Second, it can be associated  with one or several  clause(s) containing a simile describing the degree of the property, in the \textit{multiclausal nominalized degree} construction. In multiclausal nominalized degree constructions, the nominalized verb has to be combined with the marker \ipa{kɯ}, as in   example \ref{ex:YWsWxtCur2} (the sentence following  \ref{ex:YWsWxtCur} in the same text).

%ʁlaŋsaŋtɕhin	ɯ-tɯ-ɣɤχsrɯ	kɯ	saŋtɕɤn-mbrɯɣmu	χsɯ-sŋi	χsɤ-rʑaʁ	chɤ-ru
%226


\begin{exe}
\ex \label{ex:YWsWxtCur}
\gll 
\ipa{mtɕʰi}  	\ipa{ɯ-mat}  	\ipa{rca}  	\ipa{ɯ-tɯ-tɕur}  	\ipa{saχaʁ.}  	   \\
sea.buckthorn \textsc{3sg.poss}-fruit \textsc{dem} \textsc{3sg-nmlz:degree}-be.sour be.extremely:\textsc{fact}   \\
\glt `The fruit of the sea-buckthorn is very sour,' (Sea-buckthorn, 65)
\end{exe}

\begin{exe}
\ex \label{ex:YWsWxtCur2}
\gll 
 	[\ipa{ɯ-tɯ-tɕur}]  	\ipa{\textbf{kɯ}}  	[\ipa{tɯ-kɯr}  	\ipa{ɯ-ŋgɯ}  	\ipa{lú-wɣ-rku}  	\ipa{qʰe}  	\ipa{maka}  	\ipa{ɲɯ-sɯ-ɤmɯzɣɯt}  	\ipa{qʰe,}  	\ipa{tɯ-pʰoŋbu}  	\ipa{ra}  	\ipa{kɯnɤ}  	\ipa{ɲɯ-sɯx-tɕur}  	\ipa{kɯ-fse}  	\ipa{ɕti}]  \\
  \textsc{3sg-nmlz:degree}-be.sour \textsc{erg?} \textsc{indef:poss}-mouth \textsc{3sg}-inside \textsc{ipfv:upstream-inv}-put.in \textsc{lnk} at.all \textsc{ipfv-caus}-be.evenly.distributed \textsc{lnk} \textsc{indef:poss}-body \textsc{pl} also \textsc{ipfv-caus}-be.sour \textsc{nmlz:S/A}-be.like be:\textsc{affirm}:\textsc{fact} \\
\glt `(The fruit of the sea-buckthorn) is so sour that when one puts it in one's mouth, it makes it completely (sour), and it is as if one's (whole) body became sour.' (Sea-buckthorn, 66)
\end{exe}

The nominalized verb and the marker \ipa{kɯ} of multiclausal  nominalized degree constructions form a constituent and can be right dislocated together, as in \ref{ex:WtWmbjom}.

 \begin{exe}
\ex \label{ex:WtWmbjom}
\gll 
\ipa{tɯ-ci}  	\ipa{ɯ-ɣmbɤj}  	\ipa{nɯ}  	\ipa{tɕu,}  	  	\ipa{tɤ-rtsa}  	\ipa{kɯ-xtɕɯ-xtɕi}  	\ipa{nɯ}  	\ipa{χanɯni}  	\ipa{ju-ɕe}  	\ipa{ɲɯ-ŋu}  	\ipa{ma}  	\ipa{nɯ}  	\ipa{ma}  	\ipa{kɯ-saχsɤl}  	\ipa{maŋe,}  	[\ipa{ɯ-tɯ-mbjom}]  	\ipa{\textbf{kɯ}.}  \\
\textsc{indef.poss}-water \textsc{3sg.poss}-side \textsc{dem} \textsc{loc} \textsc{indef.poss}-wave \textsc{nmlz:S/A-redp}-be.small \textsc{dem} a.little \textsc{ipfv}-go \textsc{sens}-be apart.from \textsc{dem} apart.from   \textsc{nmlz:S/A}-be.clear not.exist:\textsc{sens} \textsc{3sg-nmlz:degree}-be.quick \textsc{erg?} \\
\glt `(When it dives into the water), it is so quick that one can only see little ripples near the shore.'
 (Kingfisher, 54)
\end{exe}

Unlike other nominalizing prefixes such as the \ipa{kɯ--} (S/A participle) or \ipa{sɤ--} (oblique participle), the degree action nominal in \ipa{tɯ--} cannot bear any TAM markers (whether prefixes or stem alternation). However, TAM is not neutralized in this construction: it is marked on the following verb; compare \ipa{saχaʁ} `it is extremely X' in \ref{ex:YWsWxtCur} with its past imperfective form \ipa{pɯ-saχaʁ}.
in \ref{ex:ndZtAmWmi}.

 \begin{exe}
\ex \label{ex:ndZtAmWmi}
\gll 
 \ipa{tɕendɤre}  	\ipa{ndʑi-tɯ-ɤmɯmi}  	\ipa{ndʑi-tɯ-scit}  	\ipa{pɯ-saχaʁ}  	\ipa{ʑo}  	\ipa{ɲɯ-ŋu}  \\
 \textsc{lnk} \textsc{3du-nmlz:degree}-be.in.good.terms \textsc{3du-nmlz:degree}-be.happy \textsc{pst.ipfv}-be.extremely \textsc{emph} \textsc{sens}-be \\
 \glt `They were very happy together.' (Lobzang, 13)
\end{exe}

%This construction is not restricted to degree nominals in \ipa{tɯ--}, it also attested with bare action nominals (see \citealt[7-9]{jacques14antipassive}) as in \ref{ex:tAmtsWr.kW}.
%    \begin{exe}
%  \ex  \label{ex:tAmtsWr.kW}  
%  \gll [\ipa{tɤ-mtsɯr}]  	\ipa{kɯ}  	\ipa{mɯ́j-cʰa-a}  \\
%  \textsc{indef.poss}-be.hungry \textsc{erg} \textsc{neg:sens}-can-\textsc{1sg} \\
%\glt `I am starving.'  (A calque from Chinese: \zh{我饿得不行了}; Crow and Raven, 53) 
%   \end{exe} 
  
 The marker \ipa{kɯ} is not restricted to multiclausal nominalized degree construction, it also occurs optionally with the monoclausal one, as in \ref{ex:kW.pWsaXaR}, though this use is very rare.
 
      \begin{exe}
  \ex  \label{ex:kW.pWsaXaR}  
  \gll 
   [\ipa{tɤ-ɣɲat}  	\ipa{tɤ-mtsɯr}]  	\ipa{kɯ}  	\ipa{pɯ-saχaʁ}  	\ipa{ʑo}  	\ipa{ɲɯ-ŋu}  \\
      \textsc{indef.poss}-be.tired     \textsc{indef.poss}-be.hungry \textsc{erg?} \textsc{pst.ipfv}-be.extremely \textsc{emph} \textsc{sens}-be \\
      \glt `He was extremely tired and hungry.' (Lobzang, 66)
   \end{exe} 

Optionally, in the monoclausal degree construction,  the infinitive of the verb `say' \ipa{kɤ-ti} can be inserted between the degree nominal and the marker \ipa{kɯ} as in \ref{ex:kAti.kW}.

    \begin{exe}
  \ex  \label{ex:kAti.kW}  
  \gll
[\ipa{tɯtsɣe}  	\ipa{kɤ-βzu}  	\ipa{ɯ-tɯ-cʰa}  	\ipa{kɤ-ti}]  	\ipa{kɯ}  	\ipa{pɯ-saχaʁ}  	\ipa{ʑo}  	\ipa{ɲɯ-ŋu.}  \\
commerce \textsc{inf}-make \textsc{3sg-nmlz:degree}-can \textsc{inf}-say \textsc{erg?} \textsc{pst.ipfv}-be.extremely \textsc{emph} \textsc{sens}-be \\
\glt `He was extremely proficient in commerce.' (Slopdpon, 2)
   \end{exe} 
 
\subsection{Comparative construction} \label{sec:comparative}
%No comparative adjectives XXX

A clitic \ipa{kɯ} formally identical to the ergative also appears in the main Japhug comparative construction, which can be illustrated by   examples \ref{ex:comp1} and \ref{ex:comparative.complete} (example \ref{ex:simple} illustrates a non-comparative sentence with an adjectival predicate).\footnote{It is possible to define a category of adjectives in Japhug as a subclass of stative intransitive verbs on the basis of morphology: adjectives are the stative verbs that allow the tropative derivation in \ipa{nɤ--} (\citealt{jacques13tropative}).}  

\begin{exe}
\ex \label{ex:simple}
\gll  \ipa{pʰu}   	\ipa{nɯ}   \ipa{mpɕɤr}     \\
  male \textsc{dem}    be.beautiful:\textsc{fact} \\
\glt `The male (pheasant) is beautiful.' (Pheasant, 64)
\end{exe}

\begin{exe}
\ex \label{ex:comp1}
\gll  \ipa{ɯ-ʁi}   	\ipa{sɤz}   	[\ipa{ɯ-pi}   	\ipa{nɯ}]   	\ipa{\textbf{kɯ}}   	\ipa{mpɕɤr}     \\
\textsc{3sg.poss}-younger.sibling \textsc{comparative} \textsc{3sg.poss}-elder.sibling \textsc{dem} \textsc{erg?}  be.beautiful:\textsc{fact} \\
\glt `The elder one is more beautiful than the young one.' (elicited)
\end{exe}
 
 
\begin{exe}
\ex \label{ex:comparative.complete}
\gll \ipa{jɯfɕɯr}   	\ipa{sɤz }   	[\ipa{jɯsŋi}]   	\ipa{kɯ}   	\ipa{ɲɯ-mpja}   \\
yesterday \textsc{comparative} today \textsc{erg} \textsc{sens}-warm \\
\glt `Today is warmer than yesterday.' (elicited)
\end{exe}

The terminological framework used in this section is mainly based on \citet{dixon08comparative} and \citet{stassen11comparative}. The following English sentence   illustrates their  terminology:

\begin{exe}
\ex \label{ex:comp.eng}
\gll  John is more intelligent than Paul \\
\textsc{comparee} { } \textsc{index} \textsc{parameter} \textsc{mark} \textsc{standard}  \\
\end{exe}

Comparative constructions involve two participants that are not equal. The \textsc{comparee} is  the entity that is being compared, while the other one, the \textsc{standard}, is the entity against which the comparee is compared. The \textsc{parameter} indicates the property in terms of which the comparison is carried out. It is generally an adjective, more rarely an active verb. The \textsc{index} is an element indicating the degree of the parameter, and the \textsc{mark} an element (case marker or otherwise) appearing on the standard to distinguish it from the comparee. All languages that have monoclausal comparative constructions have comparees, standards and parameters, but indexes and marks may or may not be present depending on the language.

 

The Japhug comparative construction comprises three syntactic constituents, corresponding to the standard, the comparee and the parameter. It is possible to have partial constructions with either only the comparee  (as in \ref{ex:comp2}), or only the standard (as in \ref{ex:comp3}). In both cases, the elided element is definite and anaphorically linked to a previously mentioned referent.

 \begin{exe}
\ex \label{ex:comp2}
\gll 
[\ipa{co}  	\ipa{ɣɯ}  	\ipa{nɯnɯ}]  	\ipa{\textbf{kɯ}}  	\ipa{mɤku}  	\ipa{ma}  	[\ipa{nɯnɯ} \ipa{tɕu}]  	\ipa{\textbf{kɯ}}  	\ipa{ɲɯ-mpja}  \\
valley \textsc{gen} \textsc{dem} \textsc{erg?} be.early:\textsc{fact} because \textsc{dem} \textsc{loc} \textsc{erg?} \textsc{sens}-be.warm \\
\glt `The one in the valley (grows) earlier, because it is warmer there.' (Rhododendron, 64)
\end{exe}

\begin{exe}
\ex \label{ex:comp3}
\gll  \ipa{qɤjdo}  	\ipa{sɤznɤ}  	\ipa{wxti.}     \\
pigeon \textsc{comparative} be.big:\textsc{fact} \\
\glt `(It) is bigger than a pigeon.' (Hawk, 7)
\end{exe}


The standard  bears a comparative marker when overt, either \ipa{sɤz} as in \ref{ex:comp1} and \ref{ex:comp3} or its variants \ipa{sɤznɤ}, \ipa{staʁ} and \ipa{staʁnɤ}. The comparative marker can only be dropped when a mark such as \ipa{stʰɯci} `that much, as much' is present (as in \ref{ex:sthWci.mA}).
 
 When both the standard and the comparee are overt, the standard can occur either before (as in  \ref{ex:comp1}) or after (\ref{ex:comp4}) the comparee.  
 
\begin{exe}
\ex \label{ex:comp4}
\gll 
\ipa{tɕeri}  	\ipa{ɯ-rʑaβ}  	\ipa{aʑo}  	\ipa{sɤz}  	\ipa{wxti,}  \\
but \textsc{3sg.poss}-wife \textsc{1sg} \textsc{comp} be.big:\textsc{fact} \\
\glt `But his wife is older than me.' (Relatives, 198)
\end{exe}


The marker \ipa{kɯ} on the comparee is optional, except when the standard is not overt. When the comparee has a genitival modifier, four constructions are attested. First,  the marker \ipa{kɯ} appears after the whole noun phrase (as in example \ref{ex:GW.WjwaR.kW}). 

\begin{exe}
\ex \label{ex:GW.WjwaR.kW}
\gll 
[\ipa{tɤru}  	\ipa{ɣɯ}  	\ipa{ɯ-jwaʁ}]  	\ipa{kɯ}  	\ipa{ɲɯ-jndʐɤz}  \\
Pyracantha \textsc{gen} \textsc{3sg.poss}-leaf \textsc{erg?} \textsc{sens}-be.big \\
\glt `The leaves of the Pyracantha are bigger.' (Pyracantha, 130)
 \end{exe}
 
Second,  the comparee can be elided, leaving only the modifier with the genitive marker \ipa{ɣɯ} (\ref{ex:GW.kW}).  

\begin{exe}
\ex \label{ex:GW.kW}
\gll 
\ipa{ɯ-ru}  	\ipa{tsa}  	\ipa{fse}  	\ipa{ri,}  	[\ipa{qrose}  	\ipa{ɣɯ}]  	\ipa{kɯ}  	\ipa{mbro.}  \\
\textsc{3sg.poss}-trunk a.little be.like:\textsc{fact} but plant.sp. \textsc{gen} \textsc{erg?} be.high:\textsc{fact} \\
\glt `Its  trunk is a little similar, but the one of the \ipa{qrose} grows higher.' (\ipa{qrose}, 210)
\end{exe}

Third, even the genitive marker can be elided, leaving only the genitival modifier without mark (\ref{ex:qaliaR.nW.kW}).

\begin{exe}
\ex \label{ex:qaliaR.nW.kW}
\gll 
\ipa{tɕeri}  	\ipa{ndʑi-mdoʁ}  	\ipa{ra}  	\ipa{kɯnɤ,}  	[\ipa{qaliaʁ}  	\ipa{nɯ}]  	\ipa{kɯ}  	\ipa{ɲɯ-ɲaʁ}  \\
but \textsc{3du.poss}-beak \textsc{pl} also eagle \textsc{dem} \textsc{erg?} \textsc{sens}-be.black \\
\glt `As for their beaks, that of the eagle is blacker.' (Kite, 37)
\end{exe}

Fourth, in the case when   the standard and the comparee share the same head noun and differ only by their genitival modifier, the postpositional phrase comprising the modifier with \ipa{kɯ} can be left dislocated as in example (\ref{ex:kW.WphW}), where \ipa{ɯ-pʰɯ} `its price' is the head noun of both the standard and the comparee, and \ipa{kɯ} is placed after \ipa{nɯnɯ}, the  modifier of   \ipa{ɯ-pʰɯ}.  This is crucial evidence that \ipa{kɯ} does not form a constituent with the verb.

 \begin{exe}
\ex \label{ex:kW.WphW}
\gll 
[\ipa{nɯnɯ}]  	\ipa{kɯ}  	\ipa{ɯ-pʰɯ}  	\ipa{ɲɯ-wxti,}  \\
\textsc{dem} \textsc{erg?} \textsc{3sg.poss}-price \textsc{sens}-be.big \\
\glt `This one is more expensive.' (the price of this one is bigger) (Fern, 175)
\end{exe}

The comparee marked by \ipa{kɯ} differs from both the A and the instrument in the way it is relativized. The instrument is relativized by using the oblique participle in \ipa{sɤ--} (see \ref{sec:erg}), the A by the S/A-participle in \ipa{kɯ--} with a possessive prefix coreferent with the P. The comparee is relativized in the same way as any S, in a head-internal relative with the relativized verb in the S/A participle  \ipa{kɯ--} as in \ref{ex:pjWkAm}, without an additional possessive prefix on the participle: a form such as *\ipa{ɯ-kɯ-wxti} would not be correct in \ref{ex:pjWkAm}.

 
 
\begin{exe}
\ex \label{ex:pjWkAm}
\gll
[\ipa{ɯʑo}  	\ipa{sɤz}  	\ipa{kɯ-wxti}]  	\ipa{rɯdaʁ}  	\ipa{ra}  	\ipa{kɯnɤ}  	\ipa{pjɯ-kɤm}  	\ipa{ɕti}  \\
it \textsc{comp} \textsc{nmlz}:S-big animal \textsc{pl} also \textsc{ipfv}-prevail[III] be:\textsc{assert}:\textsc{fact} \\
\glt `It also prevails over animals that are bigger than itself.' (The lion, 23)
  \end{exe}

 
The marker   \ipa{kɯ} is not a typical  \textsc{index}. It is not translated by speakers as meaning `more', and does not form a constituent with the verb. Examples like \ref{ex:kW.WphW} rather show that it forms a constituent with either the comparee or with a constituent within the noun phrase corresponding to the comparee, and that it is not necessarily adjacent to the verb.

  
The index \ipa{kɯ} is not restricted to the comparative constructions seen above. It also occurs in  tropative constructions (see \citealt{jacques13tropative}) with the verb \ipa{sɯpa} `consider' and an infinitival complement (\ref{ex:kWkWmWm}), with   tropative verbs (\ref{ex:nAmWm}) or with experiencer verbs such as \ipa{rga} `like' that include  a stimulus in their argumental structure (\ref{ex:kWrga}).

\begin{exe}
\ex \label{ex:kWkWmWm}
\gll  \ipa{tɕe}   [\ipa{tʰoŋ-raʁ}   	\ipa{nɯ}]   	\ipa{kɯ}   	\ipa{kɯ-mɯm}   	\ipa{tu-sɯpa-nɯ}   	\ipa{ŋu}   \\
\textsc{lnk} bucket.alcohol \textsc{dem} \textsc{erg} \textsc{inf:stat}-be.tasty \textsc{ipfv}-consider-\textsc{pl} be:\textsc{fact} \\
\glt `They consider  bucket alcohol to be tastier (than the pan-alcohol).' (Distilled alcohol, 15)
\end{exe}

\begin{exe}
\ex \label{ex:nAmWm}
\gll  [\ipa{tʰoŋraʁ} 	\ipa{nɯ}] 	\ipa{kɯ} 	\ipa{ɲɯ-nɤ-mɯm-nɯ} \\
 bucket.alcohol \textsc{dem} \textsc{erg}  \textsc{sens-trop}-be.tasty-\textsc{pl} \\
 \glt `They consider  bucket alcohol to be tastier (than the pan-alcohol).'  (elicited).
\end{exe}

\begin{exe}
\ex \label{ex:kWrga}
\gll \ipa{tsuku}   	\ipa{tɕe}   	 [[\ipa{tɯŋguraʁ}]   	\ipa{kɯ}   	\ipa{kɯ-rga}]   	\ipa{ɣɤʑu-nɯ,}   		\ipa{tsuku}   	\ipa{tɕe}   	[[\ipa{tʰoŋraʁ}]   	\ipa{kɯ}   	\ipa{kɯ-rga}]   	\ipa{ɣɤʑu-nɯ}   \\
some \textsc{lnk} pan.alcohol \textsc{erg?} \textsc{nmlz:S/A}-like exist\textsc{:sensory}-\textsc{pl} some \textsc{lnk} pan.alcohol \textsc{erg?} \textsc{nmlz:S/A}-like exist\textsc{:sensory}-\textsc{pl} \\
\glt `There are people who like more pan alcohol, and there are who like more bucket alcohol.' (Distilled alcohol, 17-18)
\end{exe}


%tɕeri ndʑimdoʁ ra kɯnɤ, qaliaʁ nɯ kɯ ɲɯɲaʁ
%
%aʁɤndɯndɤt tu-ɬoʁ ɕti. ri ɕkrɤz chondɤre, tɯrgi ɯ-ŋgɯ kɯ dɤn.
%tshAYCAnW 28
 
 
 
 \section{The origin of the markers \ipa{kɯ}}
Before investigating the possible historical relationships between the five main functions of the marker \ipa{kɯ},  it is necessary to evaluate whether the five types of \ipa{kɯ} are go back to one or more etymological sources. These could include grammaticalization from Japhug-internal sources,  inheritance from a proto-Sino-Tibetan  marker with a relatable function, or borrowing from another language. The evidence points to a single etymological origin: borrowing from Amdo Tibetan.

\subsection{Japhug-internal etymology}
Table \ref{tab:origins} presents attested grammaticalizations pathways leading to markers with grammatical functions overlapping with those of \ipa{kɯ} studied in the previous section. Since no exact typological parallels  to the use of \ipa{kɯ} in the degree and the comparative constructions are known to me, they are not included in the table; in the case of the marker on the comparee, a development from a genuine \textsc{index} of comparison like `more'  could be considered.

\begin{table}[H]
\caption{Attested historical origins of functions expressed with \ipa{kɯ}} \centering \label{tab:origins}
\resizebox{\columnwidth}{!}{
\begin{tabular}{llll}
\toprule
Function & Origin & Reference \\
  \midrule
 Ergative / instrumental & spatial case & \citet{agent02palancar} \\
 & discourse markers & \citet{gaby10thaayorre.ergative} \\
 & verb `to do' in converbal form & \citet{jacques14ergative} \\
Distributive & spatial case `though, along'& \citealt[213]{wartburg58few8} \\
  Causal linker & back; `here'; locative; `matter' &\citet{heine-kuteva02} \\
  &place, purposive, `say' & \\
 Manner linker & comitative, instrumental & \citet{heine-kuteva02} \\
\bottomrule
\end{tabular}}
\end{table}

This list of pathways provide a framework to evaluate the possibilities of language-internal derivation. In the following, the possibilities of grammaticalization from independent words or affixes with a phonological shape similar to \ipa{kɯ} are evaluated for each of the five functions described above.
 
\subsubsection{Locative `in the east'}
The bound lexeme \ipa{---kɯ}  `east' is found in locative nouns and adverbs such as \ipa{akɯ} and  \ipa{tɕɤkɯ} `in the east'\footnote{The elements \ipa{a--} and \ipa{tɕɤ--} are prefixes used to build locative adverbs, see \citealt[162]{linxr93jiarong}; there are cognates of  \ipa{akɯ} in all Rgyalrong languages, for instance Situ \ipa{akû} `east', \citealt[29]{lin02dimension}.}. Although grammaticalization of ergative, distributive, and causal  markers  from spatial cases has been documented, it is very unlikely that any of the functions of \ipa{kɯ} described in section  \ref{sec:synchronic} can be shown to derive from this element. 

Before being grammaticalized as ergative or causal marker, the marker \ipa{--kɯ} would have had to become the general locative marker. This cannot have been the case, since Japhug shares with other Rgyalrong languages the locative marker \ipa{zɯ} (Situ \ipa{--s}, cf \citealt[330-336]{linxr93jiarong}) and since there is no evidence in any dialect of a locative *\ipa{kɯ} or *\ipa{ku}.

Moreover, the  locative markers \ipa{akɯ} and  \ipa{tɕɤkɯ} are exclusively prenominal in Japhug, as illustrated by example \ref{ex:tCAkW}, and it is unclear how they could have become postpositions.


\begin{exe}
\ex \label{ex:tCAkW}
\gll 
\ipa{tɕɤkɯ}  	\ipa{co}  	\ipa{ci}  	\ipa{zɯ}  	\ipa{kha-ɴqra}  	\ipa{ci}  	\ipa{tɤ-khɯ}  	\ipa{kɯ-nɯ-ɬoʁ}  	\ipa{ci}  	\ipa{pjɤ-tu}  \\
east valley \textsc{indef} \textsc{loc} house-shabby \textsc{indef} \textsc{indef.poss}-smoke \textsc{nmlz:S/A-auto}-come.out  \textsc{indef}  \textsc{ipfv.ifr}-exist \\
\glt In a valley in the east, there was a shabby house from which smoke was rising.  (Nyima wodzer02, 63)
\end{exe}

\subsubsection{Proximal demonstrative}
The demonstrative \ipa{ki} `this' takes the form \ipa{kɯ--} as the first element of compounds, in particular \ipa{kɯ-sthɯci} `this much',  and the reduplicated form \ipa{kɯki} `this'.  Since \ipa{sthɯci} `as much' and \ipa{kɯ-sthɯci} `this much' do appear in equative and comparative constructions, it raises the question whether the use of \ipa{kɯ} on the comparee NP described above might not derive from the first element of \ipa{kɯ-sthɯci} by compound breaking.

The adverb \ipa{stʰɯci} `as much' and the forms derived from it can be used in a comparative construction as in \ref{ex:sthWci.mWj} to mark the standard, together with or without the comparative \ipa{sɤz}, which is optional in this case. However, since  \ipa{kɯ-sthɯci} `this much' is always contiguous with the standard NP, as in \ref{ex:kWsthWci}, never with the comparee NP, it is impossible that \ipa{kɯ} in the construction described in section \ref{sec:comparative} derives from this element.
 
 
  \begin{exe}
\ex \label{ex:sthWci.mWj}
\gll 
 \ipa{tɕʰɯkɤɣar}  	\ipa{ʑmbri}  	\ipa{nɯ}  	\ipa{sɤz}  	\ipa{stʰɯci}  	\ipa{mɯ́j-rɲɟi.}  \\
beach willow \textsc{dem} \textsc{comp} as.much \textsc{neg:sens}-be.long \\
\glt `It is not a long as the beach willow.' (Willow, 21)
  \end{exe}

  \begin{exe}
\ex \label{ex:kWsthWci}
\gll 
 \ipa{ɯ-jme}  	\ipa{nɯra}  	\ipa{koŋla}  	<kongque> 	\ipa{nɯ}  	\ipa{kɯ-sthɯci}  	\ipa{nɯ}  	\ipa{mɯ́j-zri}  \\
\textsc{ 3sg.poss}-tail \textsc{dem:pl} really peacock \textsc{dem} this-much \textsc{dem} \textsc{neg:sens}-be.long \\
\glt Its tail is not as long as that of the real peacock. (hist-24-qro, 75)
  \end{exe}
 
\subsubsection{Interrogative}
The sentence final particle \ipa{kɯ} is used in rhetorical or  introspective questions as in  (\ref{ex:pWwGsat}).  
 
  \begin{exe} 
 \ex \label{ex:pWwGsat}
\gll 
`\ipa{pɯ́-wɣ-sat} 	\ipa{ɯ-mɤ-ɕti} 	\ipa{\textbf{kɯ}?}' 	\ipa{tu-ʁjit} 	\ipa{pjɤ-ŋu,} \\
\textsc{pfv-inv}-kill \textsc{qu-neg}-be:\textsc{affirm} \textsc{qu} \textsc{ipfv}-think \textsc{ifr:ipfv}-be \\
\glt `She was thinking: `Have they killed him?'' (The frog, 91)
 \end{exe}  
 
 A related sentence final particle \ipa{kɯma} has the same functions. In  \ref{ex:kWma1}, its use implies that   the speaker is not sure whether there is any tree left to talk about.
 
   \begin{exe} 
 \ex \label{ex:kWma1}
\gll 
 \ipa{tɕendɤre} 	\ipa{mɤʑɯ} 	\ipa{si,} 	\ipa{mɤʑɯ} 	\ipa{tɕʰi} 	\ipa{mɯ-pɯ-fɕɤt-tɕi} 	\ipa{\textbf{kɯma}?} \\
\textsc{lnk} again tree again what \textsc{neg-pfv}-tell-\textsc{1du} \textsc{qu} \\
\glt `Which trees, which ones haven't we yet talked about?'  (\ipa{qandzɤjo}, 62)
  \end{exe} 

This particle, probably cognate with Tangut \ipa{kjɨ} (\citealt{jacques11tangut.verb}; see also \citealt{sunhk95yiwen} for possible cognates in other languages), can appear in reported speech with cognitive verbs such as \ipa{sɯso} `think'.

    \begin{exe} 
 \ex \label{ex:kW.YWsWsama}
\gll \ipa{a-kɤ-nɯtsʰɤβ-nɯ} 	\ipa{tɕe} 	\ipa{a-tɤ-tɕʰɯ-nɯ} 	\ipa{tɕe,} 	\ipa{a-pɯ-sat-nɯ} 	\ipa{\textbf{kɯ}} 	\ipa{ɲɯ-sɯsam-a} 	\ipa{ri}  \ipa{nɯ} \ipa{ra} 	\ipa{mɯ́j-stu-nɯ} 	\ipa{ri} \\
\textsc{irr-pfv}-do.together-\textsc{pl} \textsc{lnk} \textsc{irr-pfv}-gore-\textsc{pl} \textsc{lnk } \textsc{irr-pfv}-kill-\textsc{pl} \textsc{qu} \textsc{ipfv}-think[III]-\textsc{1sg} but \textsc{dem} \textsc{pl} \textsc{neg:sens}-do.this.way-\textsc{pl}  but \\
\glt `I think that if they would attack the leopard together and gore it, they could kill it, but they don't do that, rather,' (Wild yak, 66)
 \end{exe} 
 
 From such a construction, it could be reanalyzed as a complementizer; a grammaticalization path towards causal or manner linker would not be unthinkable. 
  
  
  
\subsubsection{Verbal affixes}
Japhug, like other Rgyalrongic languages, has a mainly prefixing typology, with very few suffixes (\citealt{jacques13harmonization}).  There are strong phonological constraints on the possible shapes of the prefixes: with the exception of a few directional prefixes of recent origin, prefixes may not contain (i)  stops other than voiceless unaspirated stops (ii) consonant clusters (iii) vowels other than \ipa{a}, \ipa{ɤ} and \ipa{ɯ}  (\citealt{jacques14antipassive}).  

 As a consequence of these constraints, many homophonous prefixes with shapes such as consonant+\ipa{ɯ--}  are found in the language; for instance, no less than six  unrelated prefixes have the shape \ipa{nɯ--}.There are four prefixes with the \ipa{kɯ--}, as indicated in Table \ref{tab:homophones.kW}.  All \ipa{kɯ--} prefixes can be shown to derive historically from the function of S/A participle (see \citealt{jacques15generic}).  

\begin{table}[H]
\caption{Grammatical morphemes with the form \ipa{kɯ}} \centering \label{tab:homophones.kW}
\begin{tabular}{llll}
\toprule
Function & Example \\
  \midrule
  S/A participle & \ipa{sat} `kill $\Rightarrow$ \begin{tabular}{l}\ipa{ɯ-\textbf{kɯ}-sat}\\ \textsc{3sg.poss-nmlz}:S/A-kill \\`the one killing him' \end{tabular}\\
  2$\rightarrow$1 portmanteau & \ipa{sat} `kill' $\Rightarrow$ \begin{tabular}{l}\ipa{a-mɤ-pɯ-\textbf{kɯ}-sat-a}  \\ \textsc{irr-neg-pfv-}2$\rightarrow$1-kill-\textsc{1sg} \\ `don't kill me' \end{tabular} \\
S/P generic &  \ipa{ngo} `be sick' $\Rightarrow$ \begin{tabular}{l}\ipa{tu-\textbf{kɯ}-ɕɯ-ngo}  \\ \textsc{ipfv-genr}:S/P-\textsc{caus}-be.sick \\ `it makes people  sick' \end{tabular} \\
evidential  & \ipa{ŋu} `be'  $\Rightarrow$ \begin{tabular}{l}\ipa{ɯ-mɤ-\textbf{kɯ}-ŋu-ci}  \\ \textsc{qu-neg-evd}-be-\textsc{evd}  \\ `maybe it is' \end{tabular} \\
\bottomrule
\end{tabular}
\end{table}

A change from prefix to enclitic through a ditropic clitic stage is not completely unheard of (see for instance  \citet{himmelmann14asymmetries} and the references therein). Were the functions of the prefixes in Table \ref{tab:homophones.kW} related in any way to those of the \ipa{kɯ} markers described in the previous section, a historical scenario linking them would not be out of the question. 

 There are two reasons why none of the five \ipa{kɯ} described in section \ref{sec:synchronic} can derive from the \ipa{kɯ--} verbal prefixes. First, from a morphological point of view, the \ipa{kɯ--} prefix occurs close to the verb root and generally takes one or more affixes to its left. It is thus rarely contiguous with the preceding noun, and shows no sign of evolution toward a ditropic clitic. Second, from the point of view of morphosyntax, if the use of \ipa{kɯ} as a linker (section \ref{sec:linking}) were indeed derived from the nominalization \ipa{kɯ--} prefix, this would imply that the verb in the clause \textit{following} \ipa{kɯ--} -- the main clause -- would be originally nominalized: the original construction implied by such a hypothesis would be typologically highly unusual.

 
 

\subsection{Sino-Tibetan etymology}
Various authors have proposed the existence of genitive or ergative markers with the form **\ipa{kV} for proto-Sino-Tibetan/proto-Tibeto-Burman (see for instance \citealt[95-6]{benedict72} or \citealt{delancey84case}), which could seem to be a potential source for Japhug \ipa{kɯ}. However, a careful examination of the data has revealed that the comparisons on which there proposals were made are rather shaky (\citealt{lapolla95ergative}).

The Tibetan ergative / instrumental \ipa{gʲis, kʲis, gis, --s} and the genitive \ipa{gʲi, kʲi, gi, --i} do indeed resemble Japhug \ipa{kɯ}, and it is tempting to suppose that the two markers are cognate. However, Japhug regularly preserves final *\ipa{--s} as \ipa{--z} (sporadically as \ipa{--t}) in the inherited vocabulary and even in the oldest layers of borrowings, for instance \ipa{ʁnɯz} `two'  vs. Tibetan \ipa{gɲis} `two' (inherited) and \ipa{saŋrɟɤz} `Buddha' (Tibetan \ipa{saŋs.rgʲas} `Buddha', borrowed).\footnote{See \citet[83-200]{jacques04these} for a detailed account of the criteria used to distinguish cognates from borrowings in Japhug.}


\subsection{Language contact} \label{sec:contact}
Of all potential hypotheses to explain the origin of the \ipa{kɯ} markers explored in the previous section, the only viable one, reanalysis from the interrogative sentence final particle to a causal / manner linker, only accounts for one of the five attested functions of \ipa{kɯ}. A more satisfactory hypothesis, involving borrowing from Tibetan, is presented here.\footnote{I am indebted to Jackson Sun for the idea that Japhug \ipa{kɯ} is borrowed from Amdo Tibetan (p.c., 2002). }

The Japhug ergative \ipa{kɯ} and genitive \ipa{ɣɯ} markers, while distinct from their Old and Classical Tibetan counterparts, do resemble Amdo Tibetan forms -- the language that used to be the main \textit{lingua franca} of the area before the mid-twentieth century.

In Amdo Tibetan, the ergative / instrumental and genitive clitics are only distinguished in pronouns. For all other forms there is syncretism, and the genitive/ergative is realized as as \ipa{ɣə}, \ipa{kə} or fronting vowel alternation depending on the stem form of the last word of the preceding NP (\citealt[62]{haller04themchen}). 

Interestingly, dialects Japhug do not all agree on the forms of the ergative and the genitive postpositions. While the Kamnyu dialect documented in the present paper has ergative \ipa{kɯ} vs genitive \ipa{ɣɯ}, the Datshang dialect has the opposite forms, ergative \ipa{ɣə} and genitive \ipa{kə} (data from \citealt[63-4]{lin11direction}). The \ipa{k} : \ipa{ɣ } and \ipa{ɣ} : \ipa{k} correspondences are not attested in any other lexical item between these dialects, and this exceptional correspondence can hardly be explained as anything else than borrowing having taken place \textit{after} proto-Japhug,  Japhug dialects attributing  different functions to the \ipa{kə} and \ipa{ɣə} allomorphs of the Tibetan marker.\footnote{Most other Rgyalrongic languages have borrowed the ergative from Tibetan, including Zbu \ipa{kə} (\citealt{gongxun14agreement}), Khroskyabs \ipa{ɣə} (\citealt[36]{lai13affixale}), Cogtse Situ \ipa{kə}  (\citealt[336]{linxr93jiarong}). Stau however has a distinct marker \ipa{--w} (\citealt{jacques14rtau}) potentially cognate to the Tangut instrumental \ipa{ŋwu}.}
 
The borrowing hypothesis is all the more meaningful that the ergative / instrumental marker in Tibetan is also used in causal and manner clause linking constructions. 

In all described varieties of Tibetan, from the Classical language to all modern dialects, the ergative can be used to mark the causal subordinate clause, either on its own or with nouns such as \ipa{dbaŋ} `power' or \ipa{ɕed} `strength'. In Amdo Tibetan, for instance, the ergative \ipa{ɣə} appears in examples such as \ref{ex:erg.caus.amdo} (\citealt[271-272]{vbrugmo03maqu}; for similar examples in Lhasa and classical Tibetan see \citealt[129]{tournadre96erg}).
 \begin{exe} 
\ex \label{ex:erg.caus.amdo}
\gll  \ipa{kʰokjaŋwi} 	\ipa{rtsatʰaŋ-na} 	\ipa{wdatrdot-nə} 	\ipa{ɣə} 	\ipa{rtsatʰaŋ} 	\ipa{ʰdandʐa-ni} 	\ipa{kʰokrdʑa} 	\ipa{jaŋmo} 	\ipa{jɔkʰə}  \\
boundless steppe-\textsc{loc} live-\textsc{nmlz} \textsc{erg} steppe like-\textsc{nmlz:gen} heart broad have \\
\glt  `Because he lives on the boundless steppe, his mind is broad like the steppe.'
\end{exe}  
 
It is thus probable that the whole construction with \ipa{ɯ-xɕɤt kɯ} was borrowed from Tibetan together with the use of \ipa{kɯ} as a simple ergative / instrumental. The use of  \ipa{kɯ} in Japhug as a clausal linker is restricted outside of the  \ipa{ɯ-xɕɤt kɯ} construction; it is essentially limited to  abstract nouns.\footnote{Note however that in the closely related language Tshobdun, the cognate ergative marker \ipa{kə} does appear with causal subordinate clauses, even finite ones (see \citealt[479]{sun12complementation}).}

The ergative / instrumental in Tibetan languages can also mark manner (see in  \citealt[128]{tournadre96erg} and \citet{tournadre10cases} on Lhasa and Classical Tibetan). In Amdo Tibetan, the ergative / instrumental \ipa{ɣə} is well attested as a gerund marker (example \ref{ex:erg.gerund.amdo} from \citealt[162; 167]{haller04themchen}), a use relatively close to that of Japhug \ipa{kɯ} in manner subordinate clauses (section \ref{sec:manner}).

 \begin{exe} 
\ex \label{ex:erg.gerund.amdo}
\gll   \ipa{ta} 	\ipa{ɲiɣa} 	\ipa{ɸtsi-ɣə} 	\ipa{ɸtsi-ɣə} 	\ipa{ta} 	\ipa{tʰaŋ-a} 	\ipa{rdom-sʰuŋ}  \\
now \textsc{3du} play:\textsc{pst}-\textsc{erg} play:\textsc{pst}-\textsc{erg} now steppe-\textsc{loc} roam-\textsc{pfv} \\
\glt  `They roamed the steppe, playing (around).'
\end{exe}  

The hypothesis that \ipa{kɯ} in its ergative / instrumental and clause linking functions is borrowed from Tibetan is the only one that accounts for the discrepancy of form between Japhug  dialects and which explains two out of the five functions of \ipa{kɯ}.

It should be noted that the Amdo Tibetan  ergative does appear in the comparative construction, but predictably on the standard rather than on the comparee, as in \ref{ex:erg.comp.amdo}  (\citealt[239]{vbrugmo03maqu})

 \begin{exe} 
\ex \label{ex:erg.comp.amdo}
\gll 
\ipa{totshək-kə}  	\ipa{tɕʰartɕʰə}  	\ipa{lohdɑ-ɣə}  	\ipa{maŋ-ⁿɡə.}  \\
this.year-\textsc{gen} rain last.year-\textsc{erg} be.many-\textsc{sens} \\
 \glt   There was more rain this year than last year.
\end{exe}  

The uses of Japhug \ipa{kɯ} in the distributive, comparative and degree construction have not equivalent in Amdo Tibetan and have to be explained as internal developments since there are no alternative possible sources from which they could be derived.

\section{Typological and historical perspectives} \label{sec:historical}

\subsection{Isomorphism between ergative and index of comparison}
The most unexpected isomorphism between the various markers having the form \ipa{kɯ} in Japhug is that between the ergative / instrumental on the one hand and the comparee (not the standard) on the other hand.   

The optional clitic \ipa{kɯ} on the comparee  clearly forms a constituent with the comparee NP (or its genitival modifier in a few limited cases), not with the verb. The comparee NP, though a S from the point of view of agreement with the predicate and from that of relativization, optionally receives  the same flagging as the A.

 
While many comparative constructions in the world's language do treat the comparee in the same way as the A (types B, C and E in \citealt[789]{dixon08comparative}'s survey, `exceed comparative' in  \citealt{stassen11comparative}), in all these constructions the standard has the same status as the P.  The comparative construction in Japhug should be classifed differently. 

Since the standard NP is marked by an oblique case (\ipa{sɤz} or \ipa{staʁ}) specific to this construction, and since the parameter of comparison is marked by a morphologically intransitive predicate, the Japhug comparative construction belongs to  \citet{stassen11comparative}'s `particle comparative' type and to \citet[789]{dixon08comparative}'s type A2.  While this type is not attested in combination with  ergative flagging in WALS (cf Table \ref{tab:stassen}, obtained by combining \citealt{stassen11comparative} with \citealt{comrie11case}), this may be due to the assignment of particular languages to the \textit{locative} rather than \textit{particle}  comparative types, and might not reflect a real gap in the data.

On the other hand, no case of a marker   on the comparee NP isomorphic with the ergative or instrumental, as \ipa{kɯ} in Japhug, has been documented in   previous surveys of comparative constructions.

\begin{table}[h]
\caption{Combination of chapters 98 (Alignment of Case Marking of Full Nouns) and 121 (Comparative constructions) of the WALS} \label{tab:stassen}
\resizebox{\columnwidth}{!}{
\begin{tabular}{lllllll}
\toprule
   &	 Neutral   &	 Nominative    &	 Nominative -   &	 Ergative -    &	 Tripartite   &	 Active-   \\   
   &&- accusative&accusative  &absolutive & &inactive\\
   && (standard) &(marked nominative)&&\\
   \midrule
 Locational   &	7  &	13  &	2  &	5  &	2  &	1  \\
Exceed   &	10  &	  &	1  &	  &	  &	  \\
Conjoined   &	10  &	3  &	  &	2  &	  &	  \\
Particle   &	3  &	8  &	  &	  &	  &	1  \\
\bottomrule
\end{tabular}}
\end{table}
 
Isomorphism between ergative / instrumental and the marker of the \textit{standard NP}, rather than the comparee NP, is expected given the well attested grammaticalization pathways \ref{ex:abl2A} and \ref{ex:abl2comp} (\citealt[29]{heine-kuteva02}).
 


\begin{exe}
\ex \label{ex:abl2A}
\glt \textsc{ablative} $\rightarrow$ \textsc{agent} 
\ex \label{ex:abl2comp}
\glt \textsc{ablative} $\rightarrow$ \textsc{comparative}
\end{exe}

Since   locational cases like ablative can change both into comparative markers (on the standard) and  into ergative markers, if both grammaticalizations occurs in the same language, isomorphism between ergative and comparative is a logical consequence. This situation is attested in Amdo Tibetan, a language with which Japhug is in contact with, as can be seen in examples \ref{ex:erg.comp.amdo} above and \ref{ex:vbris} from \citet[255]{skalbzang02dialectes}.\footnote{Even in Classical Tibetan, the comparative \ipa{--bas} (on which see \citealt{tournadre10cases} and \citealt{hill12bas}), while distinct from the ergative, contains an \ipa{--s}  element that originates from the same source as the allomorph \ipa{--s} of the ergative marker.}

\begin{exe}
\ex \label{ex:vbris}
\gll \ipa{ndzomi}  	\ipa{oma}  	\ipa{ndʐi}  	\ipa{maŋ}  	\ipa{ret}  \\
female.hybrid.yak:\textsc{gen} milk female.yak:\textsc{erg} be.numerous \textsc{copula} \\
\glt `The female hybrid yak has more milk that the female yak.'
\end{exe}

Using the same marker on the comparee NP and the A on the other hand is a typological oddity, whose explanation can only be sought for by proposing a historical account of the grammaticalization of the markers \ipa{kɯ} in all the constructions where they are attested.

A first theoretical possibility to explain this isomorphism\footnote{This idea was suggested to me by Amso Teo, to which I am most grateful.} would involve two parallel pathways. First, the evolution of a contrastive focus marker to an ergative marker following model proposed by \citealt{gaby10thaayorre.ergative}. Second, the constructionalization of the focus marker on the comparee: in comparative constructions, the comparee is more often the focus than the standard. This evolution could be paraphrased an a reanalysis of a surface form meaning `in comparison with X, \textbf{it is Y who is}  Z' (with a focus marker on Y) to  `(in comparison with X), Y is more  Z' (with the focus marker reanalysed as a comparee marker). 
If these two paths were to occur in the same language, an isomorphism similar to the one observe in Japhug could come into being.

However, this hypothesis cannot be valid in the case of Japhug for two reasons. First, it is clear that the ergative marker \ipa{kɯ}, being borrowed from Tibetan, has never been a focus marker. Second,  there is already a focus marker in Japhug expressing unexpectedness, which can appear on the comparee as in \ref{ex:rcanW}.

\begin{exe}
\ex \label{ex:rcanW}
\gll   \ipa{ɯ-ŋgɯm} 	\ipa{rcanɯ} 	\ipa{ɯʑo} 	\ipa{sɤz} 	\ipa{ɲɯ-wxti.} 	 \\
\textsc{3sg.poss}-egg \textsc{unexpected} \textsc{3sg} \textsc{comp} \textsc{sens}-be.big \\
\glt Its eggs are bigger than it. (of the ants, hist-26-qro, 9)
\end{exe}

Since this focus marker is attested in other Rgyalrong languages (for instance, in Zbu, Gong Xun p.c.) and is most probably inherited from proto-Rgyalrong, its presence would have blocked the pathway from focus marker to mark on the comparee anyway. 

In the following section, we explore a different solution to explain the ergative / comparee marker isomorphism.
 
\subsection{Historical pathways}


While the structure of the Japhug comparative construction, and in particular the homophony between the comparee marker and the ergative, does not appear to have clear typological parallels elsewhere, most of the uses of the \ipa{kɯ} marker in Japhug described in the previous sections can be argued to be derived from the basic ergative-instrumental function. 

In this section, a series of diachronic pathways leading from one construction to the other are postulated. In the absence of ancient written evidence from Japhug and the other Rgyalrong languages, part of the following developments are necessarily hypothetical, and in some cases several competing explanations are provided.

Only grammatical changes caused by the reanalysis of an existing construction in an ambiguous context are proposed, and examples of potentially ambiguous sentences (pivot constructions) taken from our Japhug corpus are provided in each case. The semantic changes hypothesized in this section  have either attested parallels in other language families or are straightforward if paraphrased in English. All the intermediate stages of the chain of reanalysis proposed here are actually attested synchronically at least in specific contexts in our Japhug corpus.

The ergative, instrumental, causal linker and manner functions of \ipa{kɯ} in Japhug, which are already present in Amdo Tibetan, do not need a separate grammaticalization hypothesis: it is safe to assume that Japhug borrowed the marker with all these additional functions from the donor language. Only the uses of \ipa{kɯ} in the distributive, degree and comparative constructions require specific explanations.

 \subsubsection{\textsc{instrumental} $\rightarrow$ \textsc{distributive} }
  
Syncretism between agent marker and distributive is well-attested in Romance languages. For instance, in French, the preposition  \ipa{par}  is used to mark the instrument, the (optional) agent in passive constructions and also occurs with a distributive meaning. Yet, it is unlikely that the distributive meaning of this preposition originates from the instrumental function in the case of Romance; rather, it comes from its  spatial and temporal use `through, along' (see \citealt[213]{wartburg58few8}). 
Thus, the Romance evidence does not support the existence grammaticalization path \textsc{instrumental} $\rightarrow$ \textsc{distributive}.



In the case of Japhug, the distributive use of \ipa{kɯ}, as we saw in section \ref{sec:distributive}, is restricted to a very specific context: classifiers expressing a quantity, to indicate the price of a product per unit. This highly restricted function, illustrated by example \ref{ex:tWtWrpa.kW1.plus} (reproduced from \ref{ex:tWtWrpa.kW1}), is in itself a clue to the possible origin of this construction.


 \begin{exe} 
\ex \label{ex:tWtWrpa.kW1.plus}
\gll  
[\ipa{tɯ-tɯrpa}] 	\ipa{\textbf{kɯ}} 	\ipa{sqi} 	\ipa{jamar} 	\ipa{ɲɯ-ra.} 	\\
one-pound \textsc{erg?} ten about \textsc{sens}-have.to \\
\glt `You need ten (yuans) per pound (of Angelica).' (Angelica, 22)
\end{exe}  

The transitive verb \ipa{sɤndu} `exchange' can be used with adjuncts in \ipa{kɯ} containing a classifier, as in \ref{ex:YWwGsAndu}, to express the price of a product per unit of quantity. In this example, the postpositional phrase [\ipa{tɯ-tɯrpa} 	\ipa{kɯ}] is clearly instrumental: `with one pound, one can exchange a hundred yuans'.

  \begin{exe}
\ex \label{ex:YWwGsAndu}
\gll 
[\ipa{tɯ-tɯrpa}]  	\ipa{kɯ}  	\ipa{ɣurʑa}  	\ipa{jamar}  	\ipa{ɲɯ́-wɣ-sɤndu}  	\ipa{ɲɯ-kʰɯ}  \\
one-pound \textsc{erg/instr} hundred about \textsc{ipfv-inv}-exchange \textsc{sens}-be.possible \\
\glt One can exchange (sell) one pound for a hundred (yuans). (elicited)
\end{exe}

This use of \ipa{kɯ} with the verb \ipa{sɤndu} `exchange' is also found in examples such as \ref{ex:YAsAndu} without a classifier.

  \begin{exe}
\ex \label{ex:YAsAndu}
\gll 
<yezi> 	\ipa{nɯ}  	\ipa{kɯ}  	\ipa{rŋɯl}  	\ipa{ɲɤ-sɤndu}  \\
coconut \textsc{dem} \textsc{erg} money \textsc{ifr}-exchange  \\
\glt He exchanged the coconuts for money. (Alibaba, 276)
\end{exe}

The meaning of sentence \ref{ex:YWwGsAndu} is slightly different from \ref{ex:tWtWrpa.kW1.plus}: the former is  said by someone selling the product in question, while the latter is said by the buyer. However, the obvious parallelism between the two constructions suggests that the construction in \ref{ex:YWwGsAndu} is the pivot construction in which  \ipa{kɯ} could be reanalyzed as a restricted distributive marker when used with certain types of classifiers. After reanalysis, this type of postpositional phrases could be generalized to sentences with a meaning close to the construction in \ref{ex:YWwGsAndu}, but without the verb  \ipa{sɤndu} `exchange'.


Thus, in the case of Japhug, unlike Romance, we do have evidence for the path \textsc{instrumental} $\rightarrow$ \textsc{distributive}.
  
 
  

\subsubsection{\textsc{cause} $\rightarrow$ \textsc{multiclausal degree} construction} \label{sec:cause2degree}

Multiclausal degree constructions  present an intrinsic ambiguity between the attested degree interpretation (`so X that Y') and a potential causal interpretation (`because of X, Y').  For instance, the sentence \ref{ex:WtWrga.kW} would also make sense  with a causal interpretation (`She forgot it because of her being (so) happy').



 \begin{exe} 
 \ex \label{ex:WtWrga.kW}
\gll 
\ipa{tɤ-mu}  	\ipa{nɯ}  	\ipa{kɯ,}  	  	[\ipa{ɯ-tɯ-rga}]  	\ipa{kɯ}  	\ipa{ɲɤ-nɯ-jmɯt}  	\ipa{qʰe,}  \\
\textsc{indef.poss}-mother \textsc{dem} \textsc{erg} \textsc{3sg-nmlz:degree}-be.happy \textsc{erg?} \textsc{ifr-auto}-forget  \textsc{lnk} \\
\glt `The old woman was so happy that she forgot (how to do).' (The frog, 261)
 \end{exe} 
 
 

% \begin{exe} 
% \ex \label{ex:tWYJAt.kW}
%\gll 
%\ipa{tɯ-ɲɟɤt}  	\ipa{kɯ}  	\ipa{tɯ-ci}  	\ipa{ɯ-ŋgɯ}  	\ipa{pjɤ-mtsaʁ}  	\ipa{qʰe}  	\ipa{pjɤ-si.}  \\
%\textsc{nmlz:action}-regret \textsc{erg} \textsc{indef.poss}-water \textsc{3sg}-inside \textsc{ifr:down}-jump \textsc{lnk} \textsc{ifr}-die \\
%\glt Out of regret, he jumped into the river and died. (The raven 28.08.12, 177)
% \end{exe} 
% 
 

Although the two meanings would appear to be entirely unrelated, the derivation from causal to degree is straightforward: for a property to be the cause of an event or a situation, this property must reach a sufficiently high degree to trigger a change of state or an action. Thus,  the causal construction necessary entails high degree, and evolution from the former to the latter is simply a restriction of the semantics of the construction.

Therefore, the historical origin of the multiclausal   degree construction in \ipa{kɯ}   (section \ref{sec:degree})   can be hypothesized to derive  from  the causal use of \ipa{kɯ}   with abstract nouns.  

 
 
 \subsubsection{ \textsc{infinitival manner} $\rightarrow$  \textsc{monoclausal degree}} \label{sec:manner2adj}
The  \ipa{kɯ} clitic marker, while obligatory in the multiclausal nominalized degree construction, is very rare in the monoclausal one. Sentence \ref{ex:kW.pWsaXaR2}  (reproduced from \ref{ex:kW.pWsaXaR}) is the only such example in the whole corpus, but similar examples can be elicited.

      \begin{exe}
  \ex  \label{ex:kW.pWsaXaR2}  
  \gll 
   [\ipa{tɤ-ɣɲat}  	\ipa{tɤ-mtsɯr}]  	\ipa{kɯ}  	\ipa{pɯ-saχaʁ}  	\ipa{ʑo}  	\ipa{ɲɯ-ŋu}  \\
      \textsc{indef.poss}-be.tired     \textsc{indef.poss}-be.hungry \textsc{erg?} \textsc{pst.ipfv}-be.extremely \textsc{emph} \textsc{sens}-be \\
      \glt `He was extremely tired and hungry.' (Lobzang, 66)
   \end{exe} 

One possible explanation to account for this is to suppose generalization from the multiclausal nominalized construction. Yet, the fact that the monoclausal nominalized degree construction commonly occurs with the infinitive of the verb `say' \ipa{kɤ-ti} in combination with   \ipa{kɯ} as in \ref{ex:kAti.kW2}  suggests otherwise.
      \begin{exe}
  \ex  \label{ex:kAti.kW2}  
  \gll 
\ipa{tɤ-rʑaβ} 	\ipa{ra} 	 	[\ipa{nɯ-tɯ-rga} 	\ipa{kɤ-ti}] 	\ipa{kɯ} 	\ipa{pɯ-saχaʁ} 	\ipa{ɲɯ-ŋu} 	\ipa{tɕe} \\
\textsc{indef.poss}-wife \textsc{pl} \textsc{3pl-nmlz:degree}-be.glad \textsc{inf}-say \textsc{erg?} \textsc{pst.impf}-be.extremely \textsc{sens}-be \textsc{lnk} \\
\glt `The wives were extremely glad.' (The brides, 7)
   \end{exe} 
   
Example \ref{ex:kAti.kW2}  from the point of view of syntactic structure is an example of \ipa{kɯ} in manner subordinate clauses (section \ref{sec:manner}). As for its syntactic function, the form \ipa{kɤ-ti kɯ} is a topicalizer, akin to English  `talking of ...', and example \ref{ex:kAti.kW2}  could be literally glossed as `Talking of the gladness of the wives, it was extreme'. 
 
Thus, the marker \ipa{kɯ} in example \ref{ex:kW.pWsaXaR2}  is more likely to reflect a topicalized construction such as that in \ref{ex:kAti.kW2}  with elision of the infinitive \ipa{kɤ-ti}. It is perfectly grammatical to add \ipa{kɤ-ti} before \ipa{kɯ} in sentence \ref{ex:kW.pWsaXaR2}. 

Thus, the presence of the marker \ipa{kɯ} in monoclausal degree constructions is unrelated to that in multiclausal degree constructions, and derives from the use of \ipa{kɯ} in manner subordinate clauses. 


 \subsubsection{\textsc{finite manner} $\rightarrow$ \textsc{adversative} or \textsc{cause} $\rightarrow$ \textsc{adversative} }
 
The adversative use of \ipa{kɯ}, while treated separately  in section \ref{sec:advers}, does not fundamentally  differ from the finite manner linking. Rather, it represents one of its several possible interpretations. Thus, a sentence such as  \ref{ex:tuGAzjAGlAG} can be construed either with adversative meaning (`it cannot stay in place like other boulders; rather, it rocks around continuously') or without it.
 
 \begin{exe}
\ex \label{ex:tuGAzjAGlAG}
\gll  
    \ipa{mɤʑɯ}  	\ipa{rŋgɯ}  	\ipa{ci}  	\ipa{ɣɤʑu}  	\ipa{ri,}  	[\ipa{ɯ-zda}  	\ipa{ra}  	\ipa{kɯ-fse}  	\ipa{ku-rɤʑi}  	\ipa{mɯ́j-kʰɯ}]  	\ipa{kɯ}  	\ipa{tu-ɣɤzjɤɣlɤɣ}  	\ipa{nɤ}  	\ipa{tu-ɣɤzjɤɣlɤɣ}  	\ipa{ɲɯ-ra}  	\ipa{tɕe,}  \\
    again boulder \textsc{indef} exist:\textsc{sens} but \textsc{3sg.poss}-companion \textsc{pl} \textsc{inf:stat}-be.like \textsc{ipfv}-stay \textsc{neg:sens}-can \textsc{erg}? \textsc{ipfv}-rock.around \textsc{lnk} \textsc{ipfv}-rock.around  \textsc{sens}-have.to \textsc{lnk}     \\
   \glt `Then, there is this boulder, it cannot stay in place like other boulders as it rocks around continuously.' (Divination  56)
 \end{exe}
 
Direct evolution from causative to adversative without intermediate stage as a manner linking has been documented in the history of Italian for instance (the linker \ipa{però}, see  \citealt{mauri12adversative}). Thus, 
the adversative \ipa{kɯ} in Japhug can either be derived from the finite manner clause linking or from the causal use of  \ipa{kɯ}, though the second possibility is less likely since \ipa{kɯ} in the causal construction is only used with abstract nouns or nominalized verbs.
 
 
 
  \subsubsection{\textsc{adversative} $\rightarrow$ \textsc{index} of comparison}
  As in most previous constructions, several hypotheses can be entertained to account for the origin of \ipa{kɯ} as an index on the comparee as in example \ref{ex:kW.YWjpum}  (section \ref{sec:comparative}).
  
        \begin{exe}
  \ex  \label{ex:kW.YWjpum}  
  \gll 
  [\ipa{tɯ-ɣli} 	\ipa{kɯ-dɤn} 	\ipa{ɯ-stu} \ipa{qandʐe} 	\ipa{nɯ}] 	\ipa{kɯ} 	\ipa{ɲɯ-jpum.} \\
  \textsc{indef.poss}-excrement \textsc{nmlz}:S/A-be.many \textsc{3sg.poss}-place earthworm \textsc{dem} \textsc{erg?} \textsc{sens}-be.thick   \\
  \glt `Earthworms (that are) in places rich in manure are thicker (than the other ones).'
   (Earthworm, 125)
   \end{exe} 

  First, it could be hypothesized that \ipa{kɯ} here derives from the topicalizer \ipa{kɤ-ti kɯ}  as in the monoclausal degree construction (section \ref{sec:manner2adj}). This hypothesis is very unlikely however in view of the fact that, unlike in the degree construction, the clitic \ipa{kɯ} in the comparative construction cannot be replaced by \ipa{kɤ-ti kɯ} and there is no evidence that it was ever  possible in any Rgyalrong language.
  
  
Second, an alternative possibility is a derivation from the \ipa{kɯ} in adversative constructions (section \ref{sec:advers}). The derivation is straightforward, but requires three steps.

The first stage is attested in modern Japhug: an adversative construction with adjectives in both  clauses, with the first adjective negated and the adverb \ipa{stʰɯci} `as much' in the first clause as in \ref{ex:mAYaR.kW} and \ref{ex:mWYArtWm.kW}.  This construction is a variant of \citet{stassen11comparative}'s `conjoined comparative'.

        \begin{exe}
  \ex  \label{ex:mAYaR.kW}  
  \gll 
  \ipa{mtsʰalɤɲaʁ} 	\ipa{nɯ} 	\ipa{ɯ-mdoʁ} 	\ipa{ɲaʁ} 	\ipa{tsa} 	\ipa{mtsʰalɤɣrum} 	\ipa{nɯnɯ} 	\ipa{tɕe,} 	[\ipa{nɯ} \ipa{stʰɯci} 	\ipa{mɤ-ɲaʁ}] 	\ipa{kɯ}   	\ipa{aqarŋɯrŋe} 	\ipa{kɯ-fse} 	\\
  black.nettle \textsc{dem} \textsc{3sg.poss}-colour be.black:\textsc{fact} a.little   white.nettle \textsc{dem} \textsc{lnk} \textsc{dem} as.much neg-be.black:\textsc{fact}  \textsc{erg?} be.yellowish:\textsc{fact} \textsc{inf:stat}-be.like \\
  \glt `The colour of the black nettle is black, and the white nettle, it is not as  black but rather yellowish.'   (Nettle 21-23)
        \end{exe}
        
           \begin{exe}
  \ex  \label{ex:mWYArtWm.kW}  
  \gll      
\ipa{ɯ-jwaʁ} 	\ipa{nɯ} \ipa{ra} 	\ipa{iʑora} 	\ipa{ji-paχɕi} 	[\ipa{stʰɯci} 	\ipa{mɯ-ɲɯ-ɤrtɯm}] 	\ipa{kɯ} 	\ipa{ɲɯ-rɲɟi} 	\ipa{tsa.} \\
\textsc{3sg.poss}-leaf \textsc{dem} \textsc{pl} \textsc{1pl} \textsc{1pl.poss}-apple so.much \textsc{neg-sens}-be.round \textsc{erg?} \textsc{sens}-be.long little \\
\glt `Its leaves  are not as round as those of the apple of our (country), but are rather a little long.'
(Apple, 49)
          \end{exe} 
 
%The comparative in \ipa{stʰɯci} `as much' can be removed, as in \ref{ex:mWYArtWm.kW2}.  
% 
%            \begin{exe}
%  \ex  \label{ex:mWYArtWm.kW2}  
%  \gll      
%\ipa{ɯ-jwaʁ} 	\ipa{nɯ} \ipa{ra} 	 	\ipa{mɯ-ɲɯ-ɤrtɯm} 	\ipa{kɯ} 	\ipa{ɲɯ-rɲɟi} 	\ipa{tsa.} \\
%\textsc{3sg.poss}-leaf \textsc{dem} \textsc{pl}  \textsc{neg-sens}-be.round \textsc{erg?} \textsc{sens}-be.long little \\
%\glt Its leaves are not round, but rather a little long.
%(Adapted from \ref{ex:mWYArtWm.kW})
%          \end{exe} 
          
          In the case of a pair of adjectives in polar, or quasi-polar, opposition as `be round' and  `be long' in \ref{ex:mWYArtWm.kW}, the information conveyed by them is redundant, and suppressing one of them does not entail loss of much information.   
          
          The surface form of an adversative construction derived from \ref{ex:mWYArtWm.kW}  with elision  of the first clause would be  \ref{ex:mWYArtWm.kW3}  and its expected meaning would be * `Its leaves are  rather a little long'.   This meaning is not found, and \ref{ex:mWYArtWm.kW3}   is instead a comparative construction whose meaning is `Its leaves are a little longer.' 
            
                      \begin{exe}
  \ex  \label{ex:mWYArtWm.kW3}  
  \gll      
[\ipa{ɯ-jwaʁ} 	\ipa{nɯ}   \ipa{ra}] 	\ipa{kɯ} 	\ipa{ɲɯ-rɲɟi} 	\ipa{tsa.} \\
\textsc{3sg.poss}-leaf \textsc{dem} \textsc{pl}   \textsc{erg?} \textsc{sens}-be.long little \\
\glt *`Its leaves are rather a little long'  $\rightarrow$  `Its leaves are a little longer.'
(Adapted from \ref{ex:mWYArtWm.kW})
          \end{exe} 
  
  
  Yet, there is much semantic overlap   between the expected (adversative) meaning and the attested comparative meaning, and \ref{ex:mWYArtWm.kW3}   represents the ambiguous structure in which the use of \ipa{kɯ} in comparative constructions was introduced by reanalysis from an adversative. After deletion, \ipa{kɯ} is reinterpreted as having syntactic scope over the constituent directly preceding it, i.e. the noun phrase corresponding to the S of the verb in the next clause.
  
  From there, after reanalysis of the marker \ipa{kɯ}, it could be generalized to the complete comparative construction with overt standard NP marked with the comparative marker, and could even be introduced in tropative constructions with a transitive predicate. The complete pathway from adversative to comparee NP index can be summarized as follows:
  
\begin{enumerate} 
\item 
\begin{enumerate} 
\item S [not.long] \ipa{kɯ} long 
\item `S is rather long than not long'
\end{enumerate} 
\item Elision of the verb phrase in the adversative construction  and reanalysis of the S as being under the syntactic scope of \ipa{kɯ}
\begin{enumerate} 
\item{} [S] \ipa{kɯ} long  
\item *`S is rather long' (not attested)
\end{enumerate} 
\item Semantic change to a comparative construction (without overt standard of comparison)
\begin{enumerate} 
\item{} [S] \ipa{kɯ} long  
\item *`S is longer'  
\end{enumerate} 
\item Introduction of the standard (S_2)
\begin{enumerate} 
\item{} S_2 \ipa{sɤz} [S] \ipa{kɯ} long  
\item  `S is longer than S_2'
\end{enumerate} 
\item Generalization to tropative construction
\begin{enumerate} 
\item{}  [S] \ipa{kɯ} long  consider  
\item  `consider S to be be longer'
\end{enumerate} 
 
\end{enumerate}
  
%its leaves are not as round as those of the apple, but rather long
%its leaves are not round, but rather long
%its leaves are rather long  
% its leaves are longer
 


  \section{Conclusion}
Despite the extreme polyfunctionality of the    clitic \ipa{kɯ} in Japhug Rgyalrong, a series of historical scenarios can be proposed to account for most of its uses (all except the sentence final particle / complementizer). The grammaticalization pathways  proposed in this paper are summarized in Figure \ref{fig:scenario}. In particular, the evolution from ergative / instrumental to index on comparee NP involves no less than four steps: \textsc{instrumental} $\rightarrow$ \textsc{infinitival manner} linker $\rightarrow$  \textsc{finite manner} / \textsc{adversative}  linker      $\rightarrow$  \textsc{index} on the comparee NP.

   \begin{figure}[H]
   \caption{Pathways of development from Ergative / Instrumental to the other functions of \ipa{kɯ}} \label{fig:scenario}  
  \begin{tikzpicture}
  \node (A) at (4,1) {\textbf{Ergative}/Instrumental};
   \node (B) at (-2,-1) {Distributive};
    \node (C) at (4,-1) {Cause};
        \node (D) at (-2,-3)  {Multiclausal degree};
    \node (E) at (4,-3) {Infinitival Manner}; 
       \node (F) at (1.5,-5)  {Finite Manner};%\\ Adversative};
        \node (G) at (6,-5) {Quotative topic}; 
             \node (H) at (-1,-7)  {\textbf{Comparee NP}};
             \node (I) at (8,-7) {Monoclausal degree}; 
  %  \node (J) at (-0.5,-9) {\textbf{Comparee NP}};
    
    
\tikzstyle{peutetre}=[->,dotted,very thick,>=latex]
\tikzstyle{sur}=[->,very thick,>=latex]
\draw[peutetre] (A)--(B);
\draw[sur] (A)--(C);
\draw[sur] (C)--(D);
\draw[peutetre] (C)--(E);
\draw[peutetre] (A) to[bend left] (E);
\draw[sur] (E)--(F);
\draw[sur] (E)--(G);
\draw[sur] (F)--(H);
\draw[sur] (G)--(I);
%\draw[sur] (H)--(J);
\draw[peutetre] (C) to[bend right] (F);

\end{tikzpicture}
\end{figure}

This case study also documents cases of ergative case borrowing that have taken place independently in several languages (Japhug, Zbu, Situ and Khroskyabs) from Amdo Tibetan. 

In the absence of more detailed descriptions, it is difficult to ascertain whether the additional functions innovated in Japhug, those found in the distributive, degree and comparative constructions, are also attested in the other Rgyalrongic languages which borrowed their ergative marker from Amdo Tibetan.


  
\bibliographystyle{unified}
\bibliography{bibliogj}
\end{document}
