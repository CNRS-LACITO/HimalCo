\documentclass{article} 
\usepackage{fontspec}
\usepackage{natbib}
\usepackage{booktabs}
\usepackage{xltxtra} 
\usepackage{polyglossia} 
  \usepackage{geometry}
 \geometry{
 a4paper,
 total={210mm,297mm},
 left=1in,
 right=1in,
 top=20mm,
 bottom=20mm,
 }
\usepackage[table]{xcolor}
\usepackage{gb4e} 
\usepackage{multicol}
\usepackage{multirow}
\usepackage{graphicx}
\usepackage{float}
\usepackage{hyperref} 
\hypersetup{bookmarks=false,bookmarksnumbered,bookmarksopenlevel=5,bookmarksdepth=5,xetex,colorlinks=true,linkcolor=blue,citecolor=blue}
\usepackage[all]{hypcap}
\usepackage{memhfixc}
\usepackage{lscape}
\usepackage{amssymb}

%\setmainfont[Mapping=tex-text,Numbers=OldStyle,Ligatures=Common]{Charis SIL} 
\newfontfamily\phon[Mapping=tex-text,Ligatures=Common,Scale=MatchLowercase]{Charis SIL} 
\newcommand{\ipa}[1]{{\phon\textbf{#1}}} 
\newcommand{\grise}[1]{\cellcolor{lightgray}\textbf{#1}}
\newcommand{\bleu}[1]{\cellcolor{blue!20}\textbf{#1}}
\newfontfamily\cn[Mapping=tex-text,Ligatures=Common,Scale=MatchUppercase]{SimSun}%pour le chinois
\newcommand{\zh}[1]{{\cn #1}}
\newcommand{\Y}{\Checkmark} 
\newcommand{\N}{} 
\newcommand{\dhatu}[2]{|\ipa{#1}| `#2'}
\newcommand{\jpg}[2]{\ipa{#1} `#2'}  
\newcommand{\refb}[1]{(\ref{#1})}
\newcommand{\tld}{\textasciitilde{}}
\newcommand{\ro}{$\Sigma$}
\newcommand{\siga}{$\Sigma_1$} 
\newcommand{\sigc}{$\Sigma_3$}   

 \begin{document} 
\title{The status of the 1sg in Rgyalrongic and Kiranti
languages}
\author{Guillaume Jacques\\ CNRS-CRLAO-INALCO}
\maketitle


Jacques, Guillaume 2017. The status of the 1sg in Rgyalrongic and Kiranti
languages, paper presented at the \textit{Highest Argument Agreement meeting}, Universitat Pompeu Fabra, Barcelona, 2\textsuperscript{nd} February.
\sloppy
\section*{Introduction}
Several criteria have been used in the typological literature to argue for person hierarchies in indexation systems (\citealt[27]{zuniga06}):

\begin{enumerate}
\item \textbf{Direct/inverse marking}. For instance, the presence of an inverse marker in 2$\rightarrow$1 and 3$\rightarrow$1/2 configurations in Gyalrong languages (and not in 1$\rightarrow$2) has been used to argue for a hierarchy 1 > 2 > 3 (\citealt{delancey81direction, jackson02rentongdengdi}).
\item \textbf{Slot accessibility}. In the Independent Order of most Algonquian languages, there is only one prefixal slot for person; in local configurations (2$\rightarrow$1 and 1$\rightarrow$2) and in inclusive forms, the verb takes the second person prefix, a fact that has been interpreted as reflecting a hierarchy 2 > 1 > 3.
\item \textbf{Access to syntactic functions}. While some languages (like Gyalrong or Algonquian) have person indexation markers that are used for all core arguments (S, A and P), it is common to have markers that are restricted to a subset of syntactic functions.
\end{enumerate}
When all these criteria are taken together, conflicting hierarchies may result (see in particular \citealt[85-6]{zuniga06} on Plains Cree). This suggests that they actually reflect distinct morphosyntactic phenomena, and that it may be misleading to build a single indexability hierarchy for a particular language.

In this paper, I discuss two topics related to the indexability hierarchies, which suggest a special status for the \textsc{1sg}: the presence of portmanteau affixes (a sub-case of criterion 3) and the question of slot accessibilitzation (an extension of criterion 2), on the basis of data from Gyalrongic and Kiranti, and discuss their relative value for establishing an indexibility hierarchy.

\section{Portmanteau forms}

\subsection{Khaling}
 
As shown in Tables \ref{tab:nonloc} and \ref{tab:loc} (data from \citealt[1102]{jacques12khaling}), in Khaling \textsc{1/2n.sg}$\rightarrow$2/3 and 3$\rightarrow$1/2  forms are identical to their intransitive counterparts (except for the presence of inverse/2 \ipa{ʔi-} prefix in \textsc{3sg}$\rightarrow$1 forms), while \textsc{1/2sg/3}$\rightarrow$2/3 direct forms have a specific stem and suffix. The \textsc{1sg$\rightarrow$3} has a portmanteau \ipa{-u} suffix (homophonous with the \textsc{1de}) and a unique stem form \ipa{lob-} with short, vowel, voicing of the final consonant, and backing effect (\ipa{o} instead of \ipa{ɵ}) not found elsewhere in the paradigm, and the \textsc{1sg$\rightarrow$2sg} has a portmanteau \ipa{-nɛ} suffix.

\begin{table}[H]
\caption{Intransitive non-past forms in Khaling compared to transitive direct and inverse configurations} \label{tab:nonloc} \centering
\begin{tabular}{llllllllllllll}
 \toprule
 &	X$\rightarrow$\textsc{3sg} &	\textsc{2/3sg}$\rightarrow$X &	\textsc{intr} &	\\	
 \midrule
\textsc{1s} &	\ipa{lob-u} \bleu{}&	\ipa{ʔi-loɔ̂m-ŋʌ} &	\ipa{soɔ̂m-ŋʌ} &	\\	
\textsc{1di} &	\ipa{lɵp-i} &	\ipa{ʔi-lɵp-i} &	\ipa{sɵp-i} &	\\	
\textsc{1de} &	\ipa{lɵp-u} &	\ipa{ʔi-lɵp-u} &	\ipa{sɵp-u} &	\\	
\textsc{1pi} &	\ipa{loɔp-ki} &	\ipa{ʔi-loɔp-ki} &	\ipa{soɔp-ki} &	\\	
\textsc{1pe} &	\ipa{loɔp-kʌ} &	\ipa{ʔi-loɔp-kʌ} &	\ipa{soɔp-kʌ} &	\\	
\textsc{2s} &	\ipa{ʔi-lɵ̄ːb-ʉ} \bleu{}&	\ipa{ʔi-loɔp} &	\ipa{ʔi-soɔp} &	\\	
\textsc{2d} &	\ipa{ʔi-lɵp-i} &	\ipa{ʔi-lɵp-i} &	\ipa{ʔi-sɵp-i} &	\\	
\textsc{2p} &	\ipa{ʔi-loɔ̂m-ni} &	\ipa{ʔi-loɔ̂m-ni} &	\ipa{ʔi-soɔ̂m-ni} &	\\	
\textsc{3s} &	\ipa{lɵ̄ːb-ʉ} \bleu{}&	 \grise{} &	\ipa{soɔp} &	\\	
\textsc{3d} &	\ipa{lɵ̂ːb-su} \bleu{}&	\grise{} &	\ipa{sɵp-i} &	\\	
\textsc{3p} &	\ipa{lɵ̂ːb-nu} \bleu{}&	\grise{} &	\ipa{soɔ̂m-nu} &	\\	
\bottomrule
\end{tabular}
\end{table}

\begin{table}[H]
\caption{1$\rightarrow$2 configuration} \label{tab:loc} \centering
\begin{tabular}{llllllllllllll}
 \toprule
  &	\textsc{2sg} &	\textsc{2du} &	\textsc{2pl} &	\\
   \midrule
\textsc{1sg} &	\ipa{loɔ̂m-nɛ} \bleu{}&	\ipa{loɔ̂m-su} \bleu{}&	\ipa{loɔ̂m-nu} \bleu{}&	\\
\textsc{1du/pl.excl }&	\ipa{ʔi-loɔp} &	\ipa{ʔi-lɵp-i} &	\ipa{ʔi-loɔ̂m-ni} &	\\
\textsc{intr} &	\ipa{ʔi-soɔp} &	\ipa{ʔi-sɵp-i} &	\ipa{ʔi-soɔ̂m-ni} &	\\
 \bottomrule
\end{tabular}
\end{table}

Comparative Kiranti comparison shows that these portmanteau forms are inherited, as shown in Table \ref{tab:kiranti} (data from \citealt{driem87}, \citealt{doornenbal09},  \citealt{bickel07puma}, \citealt{schackow15yakkha}). 


\begin{table}[H]
\caption{Khaling portmanteau suffixes in comparative light} \label{tab:kiranti} \centering
\begin{tabular}{lllllll}
\toprule
Configuration &  Khaling & Bantawa/Puma & Yakkha& Limbu \\
\midrule
\textsc{1sg$\rightarrow$3} & \ipa{-u} $\leftarrow$ \ipa{*-uŋ} & \ipa{-uŋ} & \ipa{-uŋ} &\ipa{-uŋ} \\
\textsc{1sg$\rightarrow$2sg} &  \ipa{-nɛ} $\leftarrow$ \ipa{*-na} &  \ipa{-na} &  \ipa{-nen} & \ipa{-nɛ} & \\
 \bottomrule
\end{tabular}
\end{table}

The suffix \ipa{-u} comes from \ipa{*-uŋ} due to a sound change,\footnote{Proto-Kiranti \ipa{*-uŋ} remains \ipa{-uŋ} in word-final stressed syllables (as in \ipa{lūŋ} `stone' and the \textsc{2sg} and \textsc{3sg} of intransitive verbs with stem in \ipa{-uŋ}), but changes to \ipa{ūː} with compensatory lengthening in stressed syllables when followed by a consonant and to \ipa{-u} in unstressed word-final syllables. } a complex suffix which can be analyzed as comprising third person P \ipa{*-u} and 1sg A \ipa{*-ŋ}. The suffix \ipa{-nɛ} comes from \ipa{*-na}, a form identical to that found in West and South Kiranti languages (including Puma and Bantawa).


\subsection{Plains Cree} \label{sec:cree}
The Plains Cree conjunct order paradigm presents  a pattern reminiscent of Khaling: 1/\textsc{2sg$\leftrightarrow$3} configurations have portmanteau forms, while 1/\textsc{2pl$\leftrightarrow$3} ones are built by combining the direct \ipa{-â-} or the inverse \ipa{-iko-} suffixes with the same suffixes as the intransitive paradigm.

\begin{table}[H]
\caption{Plains Cree Conjunct Order indicative paradigms.  (\citealt{wolfart96sketch})}
\label{tab:cree.conj} \centering
\begin{tabular}{llllllllll}
  \bottomrule
  &	X$\rightarrow$\textsc{3sg} &	\textsc{3sg}$\rightarrow$X &	\textsc{intr} &	\\	
  \midrule
  1s &	\ipa{-ak} \bleu{} &	\ipa{-it} \bleu{}&	\ipa{-yân} &	\\
1pi &	\ipa{-â-yahk} &	\ipa{-iko-yahk} &	\ipa{-yahk} &	\\
1pe &	\ipa{-â-yâhk} &	\ipa{-iko-yâhk} &	\ipa{-yâhk} &	\\
2s &	\ipa{-at} \bleu{}&	\ipa{-isk} \bleu{}&	\ipa{-yan} &	\\
2p &	\ipa{-â-yêk} &	\ipa{-iko-yêk} &	\ipa{-yêk} &	\\
3s &	\ipa{-(im)-â-t} &	\ipa{-iko-t} &	\ipa{-t} &	\\
3p &	\ipa{-(im)-â-cik} &	\ipa{-iko-cik} &	\ipa{-cik} &	\\
    \bottomrule
\end{tabular}
\end{table}

However, that  the pattern found in Table   \ref{tab:cree.conj} is recent. In documents from the 19\textsuperscript{th} century, the corresponding forms were nearly all unsegmentable (see table \ref{tab:creedia.conj}).

\begin{table}[H]
\caption{19\textsuperscript{th} century Plains Cree Conjunct Order indicative paradigms (based on \citealt{dahlstrom89change})}
\label{tab:creedia.conj} \centering
\begin{tabular}{llllllllll}
  \bottomrule
  &	X$\rightarrow$\textsc{3sg} &	\textsc{3sg}$\rightarrow$X &	\textsc{intr} &	\\	
  \midrule
\textsc{1s} &	\ipa{-ak} &	\ipa{-it} &	\ipa{-yân} &	\\
\textsc{1pi} &	\ipa{-ahk} &	\ipa{-itahk} &	\ipa{-yahk} &	\\
\textsc{1pe} &	\ipa{-akiht} &	\ipa{-iyamiht} &	\ipa{-yâhk} &	\\
\textsc{2s} &	\ipa{-at} &	\ipa{-isk} &	\ipa{-yan} &	\\
\textsc{2p} &	\ipa{-êk} &	\ipa{-itêk} &	\ipa{-yêk} &	\\
\textsc{3s} &	\ipa{-(im)-â-t} &	\ipa{-iko-t} &	\ipa{-t} &	\\
\textsc{3p} &	\ipa{-(im)-â-cik} &	\ipa{-iko-cik} &	\ipa{-cik} &	\\
  \bottomrule
\end{tabular}
\end{table}

The modern Conjunct Order paradigm results from analogical levelling: the portmanteau \textsc{1/2pl}$\rightarrow$3 forms have been replaced by a combination of the direct \ipa{-â-} suffix with the suffixes \ipa{-yahk}, \ipa{-yâhk}  and \ipa{-yêhk} of the intransitive animate paradigm, and the 3$\rightarrow$\textsc{1/2pl} forms by the combination of the inverse  \ipa{-iko-} suffix with the same intransitive endings. Only the \textsc{1sg} and \textsc{2sg} forms have been untouched by analogy.

Similar developments,  independent from the Plains Cree case, occurred in other Algonquian languages (Ojibwe, Micmac, and more prominently, Arapaho). 
\subsection{Analogy: the Algonquian model} \label{sec:analogy}
\citet{jacques17directionality} propose the following four generalizations regarding the directionality of analogy in direct/inverse systems based on Algonquian data which include the Plain Cree example given above:

\begin{enumerate}
\item Analogy operates from 3'$\rightarrow$3 to 3$\rightarrow$SAP forms and from 3$\rightarrow$3' to all SAP$\rightarrow$3 forms. This is a particular case of   Watkins's law (\citealt{watkins62celtic}): Analogy starts out from the third person and extends to the other forms through a reanalysis of the third person ending as part of the verb stem.
\item Analogy can apply from SAP$\rightarrow$3 forms to 3$\rightarrow$SAP and  local ones (as shown by the reshaping of 3$\rightarrow$1\textsc{pe} and 3$\rightarrow$2\textsc{pl} in Nishnaabemwin).
\item Analogy first applies to plural SAP forms before influencing singular SAP forms, both in the case of 3$\rightarrow$SAP and SAP$\rightarrow$3 paradigms. There is no evidence of a hierarchy between third singular and third plural, as we saw that the \textsc{3pl$\rightarrow$1sg} resisted analogy in Arapaho while its singular counterpart \textsc{3sg$\rightarrow$1sg} was remade.
\item Analogy first applies  to 3$\rightarrow$SAP forms before affecting SAP$\rightarrow$3 forms. There appears  to be no hierarchy between 3$\rightarrow$SA and local forms as to their sensitivity to analogy.
\end{enumerate}

According to these generations, the  direct singular configurations \textsc{1sg$\rightarrow$3} and \textsc{2sg$\rightarrow$3} are precisely the ones where analogy is expected to take place last, and thus to better preserve portmanteau forms. These generalisations are assumed to be explainable to the relative frequency of the forms in discourse: the less frequent forms are more easily replaced by analogy that the most frequent ones.

The pattern found in Khaling \ref{tab:nonloc} can thus be accounted for by assuming that the \textsc{1sg$\rightarrow$3} and \textsc{2sg$\rightarrow$3} forms are unsegmentable not because of a special cognitive status of the \textsc{1sg} and the \textsc{2sg}, but simply because these two forms escape analogy by virtue of being more common. 

The presence of portmanteau forms thus, while highly interesting for historical linguistics, is a poor criterion to argue for a special status of the \textsc{1sg}, or any other person.

\section{Slot accessibilization}
A striking particularity of \textsc{1sg} forms in the transitive paradigms of Gyalrong and Kiranti languages is the fact that nearly all configurations including the \textsc{1sg}  allow a second suffix coreferent with the number of the other argument of the verb, while forms other than the \textsc{1sg} 
almost never allow to specify the number of more than one argument.

Tables \ref{tab:japhug.tr} and \ref{tab:khaling2} respectively present the non-past transitive paradigms  in Japhug (Gyalrongic, \citealt{jacques10inverse}) and Khaling (Kiranti, \citealt[1102]{jacques12khaling}). Other Gyalrong languages have paradigms that are nearly identical to that of Japhug (see \citealt{delancey81direction}, \citealt{jackson02rentongdengdi}, \citealt{gongxun14agreement}, \citealt{zhang16bragdbar}); Kiranti languages have greater diversity in person indexation, but Dumi is very close to Khaling (\citealt{driem93agreement}).

\subsection{Japhug}

In Japhug, \textsc{1sg}$\rightarrow$3, 3$\rightarrow$\textsc{1sg} and 2$\rightarrow$\textsc{1sg} all take a first person \ipa{-a} suffix. The  local 1$\rightarrow$2 form is the only configuration including a first person which does not bear a suffix coreferent with the first person. The three configurations with an \ipa{-a} suffix are precisely those that allow another number suffix (dual \ipa{-ndʑi} or plural \ipa{-nɯ}) coreferent with the other argument, whether A or P, second or third person. No other form in the paradigm allows the indexation of the number of more than one argument.

We could imagine two possible ways of accounting for this constraint without any reference to the indexability hierarchy.

First, this constraint could be seen as a prohibition against two identical suffixes in the same verb form. Thus, since second and third person number markers are identical, the only way to express a form such as \textsc{2du$\rightarrow$3du} would be $\dagger$\siga{}-\ipa{ndʑi-ndʑi} with two times the same suffix. While such an OCP-like constraint certainly makes sense, it cannot account for the absence of second number marker in verb forms suffixed with the \textsc{1du} \ipa{-tɕi} or the \textsc{1pl} \ipa{-ji}.

Second, it could be argued to be a question of phonology: all person indexation suffixes apart from \ipa{-a} have high vowels \ipa{i} or \ipa{ɯ}, and one could suppose that the ban on double suffixation in this paradigm is due to a constraint against two unstressed suffixes with high vowels.\footnote{Stress in Japhug is by default on the last syllable of the word, except in the case of a few stress-attracting prefixes, and of some verbal suffixes with are unstressed and even devoiced.}

However, this hypothesis is contradicted by the fact that other verbal paradigms in Japhug do contain verb forms with two suffixes in high vowels, as in example (\ref{ex:tokAlWlAtndZici}).\footnote{The circumfix \ipa{k-...-ci}, here glossed as \textsc{evd}, occurs in this variety of Japhug in the inferential forms of verb whose stem begins in \ipa{a/ɤ-}, as well as in hypothetical mode. }

\begin{exe}
\ex \label{ex:tokAlWlAtndZici}
\gll \ipa{to-k-ɤlɯlɤt-ndʑi-ci} \\
\textsc{ifr-evd}-fight-\textsc{du-evd} \\
\glt `They fought each other.'
\end{exe}

Given the fact that the combination of two unstressed suffixes in high vowel are possible in Japhug, there is no reason which a form such as $\dagger$\ipa{kɯ-}\siga{}-\ipa{tɕi-ndʑi} (intended for \textsc{2du$\rightarrow$1sg}) would be impossible.

It thus seems that the presence of the \textsc{1sg} \ipa{-a} literally opens a slot for the additional number suffix, and that this unique ability cannot be explained away by pure morphological or phonological reasons. 

\subsection{Khaling}
Khaling has more configurations in its paradigm which specify the number of both arguments than Japhug:
\begin{itemize}
\item  \textsc{1sg}$\rightarrow$3
\item 3$\rightarrow$\textsc{1sg} 
\item 2$\rightarrow$\textsc{1sg}
\item \textsc{1sg}$\rightarrow$2
\item \textsc{2sg}$\rightarrow$3
\end{itemize}

The first three are shared with Japhug (and are precisely the ones with double suffixation in Khaling too), and all except the last one involve the \textsc{1sg}. Arguments against a phonological or morphological account of this distribution can also be proposed:

\begin{itemize}
\item While Khaling does have a constraint against suffix stacking for some configurations (such as $\dagger$-\ipa{ʉ-su/nu}), the strategy adopted by the language is to remove the first suffix and keeping the second one (thus \textsc{1sg$\rightarrow$2} \ipa{-nɛ}, 3P \ipa{-ʉ} are deleted when followed by \ipa{-su} and \ipa{-nu}).
\item The \textsc{1sg$\rightarrow$3} suffix \ipa{-u} is homophonous with the \textsc{1de}, the two forms being only distinguished by the verb stem (not for all conjugation types). There is no phonological or morphological reason why additional number marking should be  possible for the former and not for the latter.
\end{itemize}

\subsection{Conclusion}
In both Khaling and Japhug, the \textsc{1sg} suffixes, but not any other one, open a second suffixal slot marking the number of the other argument; no phonological or purely morphological arguments have been found to account for this distribution. Given  the transparency of this phenomenon (compared for instance with the question of portmanteau markers, where sound changes have obscured morpheme boundaries), it is extremely unlikely to be old, and reflects an independent development in the two branches. Khaling and Japhug are thus two independent witnesses of a peculiar status for the \textsc{1sg}.


\bibliographystyle{unified}
\bibliography{bibliogj}

\begin{landscape}
\begin{table}[H]
\caption{Japhug transitive and intransitive paradigms}\label{tab:japhug.tr}
\resizebox{\columnwidth}{!}{
\begin{tabular}{l|l|l|l|l|l|l|l|l|l|l|}
\textsc{} & 	\textsc{1sg} & 	  \textsc{1du} & 	\textsc{1pl} & 	\textsc{2sg} & 	\textsc{2du} & 	\textsc{2pl} & 	\textsc{3sg} & 	\textsc{3du} & 	\textsc{3pl} & 	\textsc{3'} \\ 	
\hline
\textsc{1sg} & \multicolumn{3}{c|}{\grise{}} &	\ipa{} & 	\ipa{} & 	\ipa{} & 	\ipa{\sigc{}-a}   & 	 \ipa{\sigc{}-a-ndʑi} & 	 \ipa{\sigc{}-a-nɯ} & 	\grise{} \\	
\cline{8-10}
\textsc{1du} & 	\multicolumn{3}{c|}{\grise{}} &	\ipa{ta-\siga{}} & 	\ipa{ta-\siga{}-ndʑi} & 	\ipa{ta-\siga{}-nɯ} & 	\multicolumn{3}{c|}{ \ipa{\siga{}-tɕi}}  & 	\grise{} \\	
\cline{8-10}
\textsc{1pl} & 	\multicolumn{3}{c|}{\grise{}} & 	  & 	&  & 	\multicolumn{3}{c|}{ \ipa{\siga{}-ji}}  & 	\grise{} \\	
\hline
\textsc{2sg} & 	\ipa{kɯ-\siga{}-a} & 	\ipa{} & 	\ipa{} & 	\multicolumn{3}{c|}{\grise{}}&	\multicolumn{3}{c|}{\ipa{tɯ-\sigc{}}} & 	\grise{} \\	
\cline{2-2}
\cline{8-10}
\textsc{2du} & 	\ipa{kɯ-\siga{}-a-ndʑi} & 	\ipa{kɯ-\siga{}-tɕi} & 	\ipa{kɯ-\siga{}-ji} & 	\multicolumn{3}{c|}{\grise{}} &	\multicolumn{3}{c|}{\ipa{tɯ-\siga{}-ndʑi}} & 	\grise{} \\	
\cline{2-2}
\cline{8-10}
\textsc{2pl} & 	\ipa{kɯ-\siga{}-a-nɯ} & 	\ipa{} & 	\ipa{} & 	\multicolumn{3}{c|}{\grise{}}&	\multicolumn{3}{c|}{\ipa{tɯ-\siga{}-nɯ}} & 	\grise{} \\	
\hline
\textsc{3sg} &  	\ipa{wɣɯ́-\siga{}-a} & 	\ipa{} & 	\ipa{} & 	\ipa{} & 	\ipa{} & 	\ipa{} & \multicolumn{3}{c|}{\grise{}} &	\ipa{\sigc{}} \\ 	
\cline{2-2}
\cline{11-11}
\textsc{3du} &  	\ipa{wɣɯ́-\siga{}-a-ndʑi} & 	 \ipa{wɣɯ́-\siga{}-tɕi} & 		\ipa{wɣɯ́-\siga{}-ji} & 	\ipa{tɯ́-wɣ-\siga{}} & 	\ipa{tɯ́-wɣ-\siga{}-ndʑi} & 	\ipa{tɯ́-wɣ-\siga{}-nɯ} & 	\multicolumn{3}{c|}{\grise{}} &	\ipa{\siga{}-ndʑi} \\ 
\cline{2-2}	
\cline{11-11}
\textsc{3pl} &  	\ipa{wɣɯ́-\siga{}-a-nɯ} & 	\ipa{} & 	\ipa{} & 	\ipa{} & 	\ipa{} & 	\ipa{} & \multicolumn{3}{c|}{\grise{}} &	\ipa{\siga{}-nɯ} \\ 	
\hline
\textsc{3'} & 	\multicolumn{6}{c|}{\grise{}} &	\ipa{wɣɯ́-\siga{}} & 	\ipa{wɣɯ́-\siga{}-ndʑi} & 	\ipa{wɣɯ́-\siga{}-nɯ} & 	\grise{} \\	
	\hline	\hline
\textsc{intr}&\ipa{\siga{}-a}&\ipa{\siga{}-tɕi}&\ipa{\siga{}-ji}&\ipa{tɯ-\siga{}}&\ipa{tɯ-\siga{}-ndʑi}&\ipa{tɯ-\siga{}-nɯ}&\ipa{\siga{}}&\ipa{\siga{}-ndʑi} &\ipa{\siga{}-nɯ}& 	\grise{} \\	
	\hline
\end{tabular}}
\end{table}

\begin{table}[H]
\caption{Khaling transitive paradigm} \label{tab:khaling2}
\resizebox{\columnwidth}{!}{
\begin{tabular}{l|l|l|l|l|l|l|l|l|l|l|l|l|}
	\hline
 &	\textsc{1s} &	\textsc{1di} &	\textsc{1de} &	\textsc{1pi} &	\textsc{1pe} &	\textsc{2s} &	\textsc{2d} &	\textsc{2p} &	\textsc{3s} &	\textsc{3d} &	\textsc{3p} 	\\
	\hline 
\textsc{1s} & \multicolumn{5}{c|}{\grise{}} &	\ipa{loɔ̂m-nɛ} &	\ipa{loɔ̂m-su} &	\ipa{loɔ̂m-nu} &	\ipa{lob-u} &	\ipa{lob-u-su}  &	\ipa{lob-u-nu}  	\\
\cline{7-12}
\textsc{1di} &	 \multicolumn{8}{c|}{\grise{}}  &	\multicolumn{3}{c|}{\ipa{lɵp-i}}   	\\
\cline{7-12}
\textsc{1de} &	 \multicolumn{5}{c|}{\grise{}} 	 &	\ipa{ʔi-loɔp} &	\ipa{ʔi-lɵp-i} &	\ipa{ʔi-loɔ̂m-ni} &	\multicolumn{3}{c|}{\ipa{lɵp-u}}  	\\
\cline{7-12}
\textsc{1pi} &	 \multicolumn{8}{c|}{\grise{}}   &	\multicolumn{3}{c|}{\ipa{loɔp-ki}}  	\\
\cline{7-12}
\textsc{1pe} &	\multicolumn{5}{c|}{\grise{}} 	 &	\ipa{ʔi-loɔp} &	\ipa{ʔi-lɵp-i} &	\ipa{ʔi-loɔ̂m-ni} &	\multicolumn{3}{c|}{\ipa{loɔp-kʌ}}  	\\
\cline{2-2}
\cline{4-4}
\cline{6-12}
\textsc{2s} &	\ipa{ʔi-loɔ̂m-ŋʌ} &	 \grise{}&	 &	\grise{}& & \multicolumn{3}{c|}{\grise{}}  &	\ipa{ʔi-lɵ̄ːb-ʉ} &	\ipa{ʔi-lɵ̂ːp-su} &	\ipa{ʔi-lɵ̂ːp-nu} 	\\
\cline{2-2}
\cline{10-12}
\textsc{2d} &	\ipa{ʔi-loɔ̂m-ŋʌ-su}  &	\grise{}&	 &	\grise{}& &\multicolumn{3}{c|}{\grise{}}  &	\multicolumn{3}{c|}{\ipa{ʔi-lɵp-i}}   	\\
\cline{2-2}
\cline{10-12}
\textsc{2p} &	\ipa{ʔi-loɔ̂m-ŋʌ-nu}  &	\grise{}& & \grise{}& &	\multicolumn{3}{c|}{\grise{}}  &	\multicolumn{3}{c|}{\ipa{ʔi-loɔ̂m-ni}}  	\\
\cline{2-3}
\cline{5-5}
\cline{7-12}
\textsc{3s} &	\ipa{ʔi-loɔ̂m-ŋʌ} &	\ipa{} &	\ipa{ʔi-lɵp-u} &	\ipa{} &	\ipa{ʔi-loɔp-kʌ} &	 & & &	\ipa{lɵ̄ːb-ʉ} &\multirow{2}{*}{\ipa{lɵ̂ːp-su}} &	\multirow{3}{*}{\ipa{lɵ̂ːb-nu} } 	\\
\cline{2-2}
\cline{10-10}
3d &	\ipa{ʔi-loɔ̂m-ŋʌ-su} &	\ipa{ʔi-lɵp-i} &	\ipa{} &	\ipa{ʔi-loɔp-ki} &	\ipa{} &	\ipa{ʔi-loɔp} &	\ipa{ʔi-lɵp-i} &	\ipa{ʔi-loɔ̂m-ni}  &	\multicolumn{1}{c}{}	 &	 & 	\\
\cline{2-2}
\cline{10-11}
\textsc{3p} &	\ipa{ʔi-loɔ̂m-ŋʌ-nu} &	\ipa{} &	\ipa{} &	\ipa{} &	\ipa{} &	\ipa{} &	\ipa{} &	\ipa{} &	\multicolumn{2}{c}{} 	& 	\\
	\hline
\end{tabular}}
\end{table}

\end{landscape}


 \end{document}
 