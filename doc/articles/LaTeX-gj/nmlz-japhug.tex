\documentclass[oldfontcommands,oneside,a4paper,11pt]{article} 
\usepackage{fontspec}
\usepackage{natbib}
\usepackage{booktabs}
\usepackage{xltxtra} 
\usepackage{longtable}
\usepackage{polyglossia} 
\usepackage[table]{xcolor}
\usepackage{gb4e} 
\usepackage{multicol}
\usepackage{graphicx}
\usepackage{float}
\usepackage{lineno}
\usepackage{textcomp}
\usepackage{hyperref} 
\hypersetup{bookmarks=false,bookmarksnumbered,bookmarksopenlevel=5,bookmarksdepth=5,xetex,colorlinks=true,linkcolor=blue,citecolor=blue}
\usepackage[all]{hypcap}
\usepackage{memhfixc}
\usepackage{lscape}
 

%\setmainfont[Mapping=tex-text,Numbers=OldStyle,Ligatures=Common]{Charis SIL} 
\newfontfamily\phon[Mapping=tex-text,Ligatures=Common,Scale=MatchLowercase,FakeSlant=0.3]{Charis SIL} 
\newcommand{\ipa}[1]{{\phon #1}} %API tjs en italique
 
\newcommand{\grise}[1]{\cellcolor{lightgray}\textbf{#1}}
\newcommand{\bleute}[1]{\cellcolor{green}\textbf{#1}}
\newcommand{\rouge}[1]{\cellcolor{red}\textbf{#1}}
\newfontfamily\cn[Mapping=tex-text,Ligatures=Common,Scale=MatchUppercase]{MingLiU}%pour le chinois
\newcommand{\zh}[1]{{\cn #1}}
\newcommand{\topic}{\textsc{dem}}
\newcommand{\tete}{\textsuperscript{\textsc{head}}}
\newcommand{\rc}{\textsubscript{\textsc{rc}}}
\XeTeXlinebreaklocale 'zh' %使用中文换行
\XeTeXlinebreakskip = 0pt plus 1pt %
 %CIRCG
 


\begin{document} 
\title{Nominalization in Japhug in historical perspective}
%\author{Guillaume Jacques}
\maketitle
\linenumbers
 
\section{Introduction}


 

\section{Nominalized verbal forms}
 
 
 
  \subsection{Finite vs non-finite verbs} \label{sec:finite.verbs}
Japhug and other Rgyalrong languages have a very strict distinction between finite and non-finite forms, as described in \citet{jacques14linking}. Finite forms present  polypersonal indexation paradigms combining prefixes and suffixes, while non-finite forms either don't mark person, or only take possessive prefixes like nouns.

The Rgyalrong languages are conspicuous in the Sino-Tibetan language family in lacking a construction of the ‘Standard Sino-Tibetan Nominalization’ type (\citealt{bickel99nmlz}), whereby a attributive/genitive marker is also used as a nominalizer and a relativizer. We do  find relative clauses or complement clauses with a finite verb (this type of construction has been referred to as `clausal nominalization' in the literature, for instance \citealt{genetti08nmlz}), but these clauses do not receive any nominalizer. Example \ref{ex:nWwGmbi} illustrates this type of construction: its main verb \ipa{nɯ́-wɣ-mbi} is in a finite form (with the inverse prefix \ipa{wɣ--}). 

 \begin{exe}
\ex \label{ex:nWwGmbi}
\gll
[\ipa{tɤ-wɯ} 	\ipa{kɯ} 	\ipa{ʑmbrɯ} 	\ipa{nɯ́-wɣ-mbi}] 	\ipa{nɯ} 	 	\ipa{cʰɤ-lɤt} \\
\textsc{indef.poss}-grandfather \textsc{erg} boat \textsc{pfv-inv}-give \textsc{dem} \textsc{ifr}-throw \\
\glt He took the boat that the old man had given him. (140430 jin e, 245)
\end{exe}

Demonstratives such as \ipa{nɯ} `that' or \ipa{ki} `this' often occur  on the right side of relative clauses of this type, but are not obligatory in this position, as in \ref{ex:jAkWsWCGaza}, where no demonstrative occurs between the finite prenominal relative \ipa{jɤ-kɯ-sɯ-ɕɣaz-a} `you told me to take back' and the head noun \ipa{mbalɤpɯ} `calf'

 \begin{exe}
\ex \label{ex:jAkWsWCGaza}
\gll
[\ipa{jɤ-kɯ-sɯ-ɕɣaz-a}] 	\ipa{mbalɤpɯ} 	\ipa{nɯ} 	\ipa{pa-mto} \\
\textsc{pfv}-2$\rightarrow$1-\textsc{caus}-take.back-\textsc{1sg} calf \textsc{dem} \textsc{pfv}:3$\rightarrow$3'-see \\
\glt She saw the calf that you told me to take back. (140512 fushang he yaomo, 132)
\end{exe}

Although the finite verbs in these constructions do present some morphosyntactic features distinguishing them from finite forms of main clauses, they lie outside of the scope of this paper and will be  treated elsewhere; in the following, only constructions involving non-finite verbs are discussed.

 \subsection{Participles} \label{sec:participles}
Three of  the nominalization prefixes found in Japhug,  \ipa{kɯ--}, \ipa{kɤ}-- and \ipa{sɤ}-- are used to build \textit{participles}, which unlike other nominalized forms preserve the argument structure of the verb and can take both argument and adjuncts with the same marking as in independent clauses.

The \ipa{kɯ--} S/A participial prefix appears with both intransitive and transitive verbs, but in the latter case a possessive prefix  coreferent with the patient is added (see \ref{ex:see}). This nominalized form can be used as one of the tests to determine whether a particular verb is transitive or intransitive.  

 \begin{exe}
\ex
\gll \ipa{kɯ-nɯʑɯβ}    \\
  \textsc{nmlz}:S/A-sleep \\
 \glt  `The one who sleeps'
 
\ex \label{ex:see}
\gll \ipa{ɯ-kɯ-mto}    \\
  \textsc{3sg}-\textsc{nmlz}:S/A-see \\
 \glt  `The one who sees him.'
 

\ex \label{ex:see2}
\gll \ipa{kɤ-mto}    \\
   \textsc{nmlz}:P-see \\
 \glt  `The one that is seen.'
 \end{exe}
 
  The patient participial prefix \ipa{kɤ--} can appear with an optional possessive prefix coreferent to the agent as in \ref{ex:see3}.
  
  \begin{exe}
\ex \label{ex:see3}
\gll \ipa{a-kɤ-mto}    \\
   \textsc{1sg-nmlz}:P-kill \\
 \glt  `The one that I kill.'
 \end{exe}

The \ipa{sɤ}--prefix (and its allomorphs \ipa{sɤz}-- and \ipa{z}--) is used for oblique participle, which among its various uses is used to build relatives whose relativized element is not a core argument, but either the recipient (for indirective verb), a comitative, instrument, place or time adjunct. It receives a possessive prefix  which can be coreferent with either S, A or P.

   \begin{exe}
\ex \label{ex:come}
\gll \ipa{ɯ-sɤ-ɣi}    \\
   \textsc{3sg-nmlz}:S-come \\
 \glt  `The place/moment where/when it comes.'
 \end{exe}
 
 
 
 \subsubsection{The template of participial forms}
 
Participial forms cannot receive person marking, inverse \ipa{wɣ}--, direct \ipa{a}--, irrealis \ipa{a}-- or evidential directional prefixes, but are compatible with associated motion prefixes \ipa{ɣɯ}-- and \ipa{ɕɯ}--, negative prefixes and perfective and imperfective directional prefixes.\footnote{The template of finite verb forms is presented in \citet{jacques13harmonization}.} When a nominalized form has a negative, TAM or associated motion prefix, the possessive prefix of A-nominalization and oblique nominalization is optional. Examples \ref{ex:makWndza} to \ref{ex:thongthar} illustrate nominalized forms without possessive prefix.
 

    \begin{exe}
\ex \label{ex:makWndza}
\gll
[\ipa{ɯ-zda}  	\ipa{ra}  	\ipa{cʰɯ-kɯ-ndza}]  	\ipa{ci,}  	\ipa{ɕa}  	\ipa{ma}  	\ipa{mɤ-kɯ-ndza}  	\ipa{ɲɯ-ŋu.}  	 \\
\textsc{3sg.poss}-companion  \textsc{pl} \textsc{ipfv-nmlz}:S/A-eat \textsc{indef} meat apart.from \textsc{neg-nmlz}:S/A-eat \textsc{testim}-be \\
\glt (The dhole) is (an animal) that eats other animals, that only eats meat. (Dhole, 2-3)
 \end{exe}
     \begin{exe}
\ex \label{ex:kill4}
\gll
\ipa{qɤjtʂʰa}  	\ipa{nɯ}  	[\ipa{pɯ-kɤ-sat}]  	\ipa{kɯnɤ}  	\ipa{kɤ-mto}  	\ipa{mɯ-pɯ-rɲo-t-a.}  \\
vulture \topic{} \textsc{pfv-nmlz:P}-kill  also \textsc{inf}-see \textsc{neg-pfv}-experience-\textsc{pst:tr-1sg} \\
\glt I have never seen a vulture, even a dead (killed) one. (Vulture 54)
 \end{exe}
  
  
 \begin{exe}
\ex \label{ex:thongthar}
\gll [\ipa{qandʑi}   	\ipa{cʰɯ-sɤ-ɣnda}]   	\ipa{nɯ}   	\ipa{tʰoŋtʰɤr}   	  	\ipa{ɲɯ-rmi}    \\
bullet \textsc{ipf}-\textsc{nmlz:oblique}-ram   \textsc{dem} ramrod \textsc{testim}-call \\
 \glt What is used to ram a bullet (into the muzzle of the gun) is called a ramrod. (Arquebus)
 \end{exe}

It is possible to combine several prefixes before the nominalization prefix; the limit is three prefixes, as in example \ref{ex:WGWjAkWqru}.

 \begin{exe}
\ex \label{ex:WGWjAkWqru}
\gll
  	\ipa{ɯ-ɣɯ-jɤ-kɯ-qru}  	\ipa{tɤ-tɕɯ}  	   \\
  \textsc{3sg-cisloc-pfv-nmlz:}S/A-meet \textsc{indef.poss}-boy   \\
\glt The boy  who had come to look for her (The three sisters 231)
 \end{exe}
 
The ordering of the inflexional prefixes in  Japhug is shown in Table \ref{tab:template.nmlz}; derivational prefixes are not represented here - they are all conflated within   `enlarged stem'.



\begin{table}[H]
\caption{The template of nominalized verbal forms in Japhug} \centering \label{tab:template.nmlz}
\resizebox{\columnwidth}{!}{
\begin{tabular}{lllllll}
\toprule
-5 & -4&-3 &-2&-1\\
possessive & negative&associated   & TAM & nominalization &enlarged  \\
prefix & prefix &motion prefix  &directional&&stem\\
\bottomrule
\end{tabular}}
\end{table}

The non-past verb stem (Stem III) never appears in nominalized forms. On the other hand, the perfective stem (Stem II) is obligatory in perfective nominalized verbs as in \ref{ex:jAkWGe} (compare with the imperfective nominalization in example \ref{ex:jukWGi}).

 \begin{exe}
\ex \label{ex:jAkWGe}
\gll
  	\ipa{jɤ-kɯ-ɣe}	   \\
  \textsc{pfv-nmlz:}S/A-come[II]   \\
\glt The one who came.
\ex \label{ex:jukWGi}
\gll
  	\ipa{ju-kɯ-ɣi}	   \\
  \textsc{ipfv-nmlz:}S/A-come   \\
\glt The one who is coming.
 \end{exe}
 
The oblique nominalizer is only compatible with imperfective TAM prefixes, not with perfective ones. 
 
There are some constraints on  the prefixal slots. Possessive and TAM prefixes are compatible for oblique nominalization and for A as in \ref{ex:WtusAGi} and \ref{ex:WtukWrACi}.

 \begin{exe}
\ex \label{ex:WtusAGi}
\gll
\ipa{tɯ-ci}  	\ipa{ɯ-tu-sɤ-ɣi}  \\
\textsc{indef.poss}-water \textsc{3sg-ipfv-nmlz}:come \\
\glt  The place where water comes up (Alcohol jug, 18)
 \end{exe}
 \begin{exe}
\ex \label{ex:WtukWrACi}
\gll 
\ipa{ɕombri}  	\ipa{ɯ-tu-kɯ-rɤɕi}  	\ipa{ra}  	\ipa{kɯ}  \\
chain \textsc{3sg-ipfv-nmlz:A}-pull \textsc{pl} \textsc{erg} \\
\glt Those who were pulling the chain (The fox, 80)
 \end{exe}

However, with relativization of P, TAM prefixes and personal prefixes are not compatible with each other. It is thus possible to say \ipa{pɯ-kɤ-mto} \textsc{pfv-nmlz:P}-\textit{see} `which was seen' or \ipa{a-kɤ-mto} \textsc{1sg-nmlz:P}-\textit{see} `which I see' but not to combine the two in a form such as *\ipa{a-pɯ-kɤ-mto}. No such constraint is found with negative and associated motion prefixes. 

\subsection{Infinitive}
Like other Rgyalrong languages such as Tshobdun (\citealt[476]{sun12complementation}), Japhug verbs have two infinitival prefixes \ipa{kɯ--} or \ipa{kɤ--}, homophonous either with the S/A-- or the P--participle. The \ipa{kɯ--} infinitive occurs with stative verbs or verbs that do not allow a human argument (in particular, some modal auxiliaries), and the \ipa{kɤ--} infinitive with all dynamic verbs. Infinitive forms are used in three context. First, they are the usual citation form of verbs (though not for all speakers, who may prefer a finite form). Second, they occur in some complement clauses (section \ref{sec:inf1}). Third, they are used in manner and clausal subordinate clauses (section \ref{sec:linking}; see also \citealt{jacques14linking}).


Given their formal resemblance with participles, it is legitimate to wonder whether participle and infinitives really need to be distinguished. Apart from the constructions in which they occur, and the fact that they cannot take possessive or directional prefixes, the clearest argument for this distinction is the fact that intransitive dynamic verbs (except semi-transitive verbs, on which see \ref{sec:kA.rel}) lack a P-participle, but do have \ipa{kɤ--} infinitives, as in example \ref{ex:kAsi}. 

\begin{exe}
\ex \label{ex:kAsi}
\gll
\ipa{kɤ-si} \ipa{mɯ-pjɤ-ra} \ipa{rcanɯ} \\
\textsc{inf}-die \textsc{neg-ifr.ipfv}-have.to \textsc{unexpected} \\
\glt He did not have to die. (140512, fushang he yaomo, 185)
\end{exe} 


\subsection{Fossilized nominal forms}
Alongside the participial \ipa{kɯ--} prefix, we find residual traces of vowel-less allomorphs of this prefix in irregular nominal forms (\citealt[5]{jacques14antipassive}), as shown in Table \ref{tab:irr.nmlz}. These nouns can only be considered to be nominalizations from a historical point of view. Synchronically, they have become opaque and their relationship with the base verb has been obscured by independent semantic innovations. Interestingly, some of the nouns in this list have cognates in Khroskyabs and Stau (\citealt{lai13affixale}).

\begin{table}[H]
\caption{Irregular nominalizations in \ipa{ɣ}-- and \ipa{x}--} \label{tab:irr.nmlz} \centering
\begin{tabular}{llll}
\toprule
 noun & meaning &base verb & meaning\\
\midrule
\ipa{\textbf{ɣ}ndʑɤβ} & disastrous fire & \ipa{ndʑɤβ} & burn \\
\ipa{--\textbf{ɣ}ɲaʁ}   &disaster& \ipa{ɲaʁ} & be black \\
\ipa{--\textbf{ɣ}ɲɟɯ}   & orifice & \ipa{ɲɟɯ} & be opened \\
\ipa{--\textbf{x}so}   &  empty thing &\ipa{so} & be empty \\
\bottomrule
\end{tabular}
\end{table}

In addition, we find some limited evidence for a \ipa{--s} nominalizing suffix in Japhug and Situ (see \citealt{jacques03s.houzhui}) in fossilized forms such as the possessed noun \ipa{--ʁjiz} `desire' from the verb \ipa{ʁjit} `think, plan'.

\subsection{Action nominal}
There are two types of action nominals in Japhug, the \ipa{tɯ--} action nominal and the bare infinitive.

Nominals build with the nominalization prefix \ipa{tɯ--} differ from the participles in that the argument structure is neutralized (\citealt[6-7]{jacques14antipassive}). The resulting nominals are used either as action nouns (\ipa{rɟaʁ} `dance (vi)' $\rightarrow$ \ipa{tɯ-rɟaʁ} `dance (n)'), in the degree construction (section \ref{sec:degree}) or in some complement clauses (section \ref{sec:inf2}).

The bare infinitive is built by combining the bare stem of the verb with a possessive prefix coreferent with the one core argument (\citealt[6-7]{jacques14antipassive}). These infinitives mainly occur in specific complement clauses (\ref{sec:inf2}), but are also attested as action nominals  in a few examples indicated in Table \ref{tab:nmlz-inalienably}.


\begin{table}[H]
\caption{Bare nominalizations (inalienably possessed nouns)} \label{tab:nmlz-inalienably} \centering
\begin{tabular}{llllllllll}
\toprule
base verb & meaning & noun stem & indefinite  & meaning\\
&&&possessor\\
\midrule
\ipa{fkaβ} & cover& \ipa{--fkaβ} & \ipa{tɤ-fkaβ} & lid\\
\ipa{ɕpʰɤt} & patch& \ipa{--ɕpʰɤt} & \ipa{tɤ-ɕpʰɤt} & patch (n.) \\
\ipa{sɯso} & think & \ipa{--sɯso} & \ipa{tɯ-sɯso} & thought \\
\bottomrule
\end{tabular}
\end{table}



\subsection{Synthetic action nominal}
Another way to build a noun out of a verb is by compounding a noun and a verb into a nominal, as the examples in Table \ref{tab:noun.verb}. If the first element of the compound is an inalienably possessed noun, it loses its possessive prefix. The first element of the compound also undergoes the \textit{status constructus} vowel alternation (see \citealt{jacques12incorp}).



\begin{table}[H] \centering
\caption{Examples of nominal noun-verb compounds}\label{tab:noun.verb}
\resizebox{\columnwidth}{!}{
\begin{tabular}{lllllllll} \toprule
Element 1  & Element 2  & compound noun  \\
\midrule
\ipa{tɯ-rcu} ``leather & \ipa{ŋga} ``wear'' (vt)& \ipa{rcɤ-mbe-ŋga}  ``beggar (the one  \\
 jacket''& \ipa{mbe} ``old''&who wears old jackets''\\
\ipa{tɯ-ku} ``head'' & \ipa{tɕʰɯ} ``gore'' (vt) & \ipa{kɤ-tɕʰɯ} ``head-butt''\\
\ipa{pʰoŋ} ``bottle'' & \ipa{sti} ``fill, block'' (vt) & \ipa{pʰoŋ-sti} ``bottle stopper'' \\
\ipa{si} ``wood'' & \ipa{tɕʰaʁ} ``diminish'' (vi) & \ipa{sɯ-tɕʰaʁ} ``shrinking (of wood)'' \\
 \bottomrule
\end{tabular}}
\end{table}

This type of compounds can further undergo denominal derivation, giving rise to an incorporating construction  (\citealt{jacques12incorp, lai13affixale}).


%{Zero-derivation or inverse derivation?}

\section{Relativization} \label{sec:relative}
Most relatives in Japhug are participial relatives, although correlatives and finite relatives (examples \ref{ex:nWwGmbi} and \ref{ex:jAkWsWCGaza} above) are also attested. Finite relatives are highly restricted.\footnote{They can only be used when the relativized element is the P-argument, the object of a semi-transitive verb, the T-argument of a ditransitive verb,  the R-argument of secundative ditransitive verbs and the goal of motion / manipulation verbs. } All argument and adjunct accessible to relativization in Japhug can be relativized with a participial relative.


\subsection{\ipa{kɯ--} participle} \label{sec:kW.rel}

\subsubsection{S}
The only argument of an intransitive verb, whether stative or dynamic, can only be relativized by means of a  \ipa{kɯ}-participle. The head noun, when overt, normally occurs   before the nominalized verb (examples \ref{ex:tchi} and \ref{ex:kAkAmdzW}). 

 
 \begin{exe}
   \ex   \label{ex:tchi}
 \gll  	\ipa{ɯ-ɣmbɤj}  	\ipa{zɯ}  	[\ipa{tɕʰi}\tete{}  	\ipa{tu-kɯ-ndɯ}]\rc{}  	\ipa{ci}  	\ipa{pɯ-tu}  	\ipa{ɲɯ-ŋu}  		\\
\textsc{3sg.poss}-side \textsc{loc} ladder \textsc{ipfv-nmlz:S}-be.built  \textsc{indef} \textsc{pst.ipfv}-exist \textsc{testim}-be  \\
 \glt    There was a ladder which was leaning on the side (of the tower). (Slopdpon, 55)
   \end{exe} 

\begin{exe}
\ex \label{ex:kAkAmdzW}
\gll
[\textbf{\ipa{tɯrme}}\tete{}  	\ipa{kɤ-kɯ-ɤmdzɯ}]  	\ipa{ɣɯ}  	\ipa{tɯ-ku}  	\ipa{ɯ-fsu}  	\ipa{jamar}  	\ipa{ra}  \\
man \textsc{pfv-nmlz:S}-sit \textsc{gen} \textsc{indef.poss}-head \textsc{3sg.poss}-as about need:\textsc{fact} \\
\glt It has to be  about as tall as the head of a man who sat down. (posti, 4)
  \end{exe}
  
 Adjectives in Japhug are a sub-class of stative verbs.\footnote{One can nevertheless unambiguously distinguish adjectives from other stative verbs (such as copulas, existential verbs and some modal auxiliaries) in that the former can be receive the tropative \ipa{nɤ--} (\citealt{jacques13tropative}), while the latter  cannot.} Noun phrases containing a head noun and an attributive adjective should be analyzed as relative clauses with relativized S-argument. Degree adverbs appear either between the noun and the verb (as in \ref{ex:mazw}) or, less commonly, before the entire  relative (as in \ref{ex:camtsho}).


 \begin{exe}
   \ex   \label{ex:mazw}
 \gll 
\ipa{nɯnɯ}  	\ipa{li}  	[\textbf{\ipa{smɤn}}\tete{}   	\ipa{mɤʑɯ}  	\ipa{kɯ-pe}]\rc{}  	\ipa{ɲɯ-ŋu.}  \\
\textsc{dem} again medicine not.only \textsc{nmlz:S}-good \textsc{testim}-be \\
 \glt    This is an even better medicine. (Bear, 85)
   \end{exe} 
 
  \begin{exe}
   \ex   \label{ex:camtsho}
 \gll  \ipa{cɤmtsʰo}  	\ipa{ndɤre}  	\ipa{wuma}  	\ipa{ʑo}  	[\textbf{\ipa{smɤn}}\tete{}   	\ipa{kɯ-pe}]\rc{} 	\ipa{ŋu}  \\
 musk \textsc{lnk} very \textsc{emph}  medicine \textsc{nmlz:S}-good be:\textsc{fact}  \\
 \glt    Musk is a very good medicine. (Musk, 59)
   \end{exe}  

Various modifiers can be inserted between the head noun and the verb as in example \ref{ex:kWNgWrtsAG}.

\begin{exe}
   \ex  \label{ex:kWNgWrtsAG}
\gll
[\textbf{\ipa{tɤjpa}}\tete{}  	\ipa{kɯŋgɯ-rtsɤɣ} 	\ipa{kɯ-jaʁ}]\rc{} 	\ipa{ko-sɯ-lɤt} 	\\
snow nine-stairs \textsc{nmlz:S}-thick \textsc{evd-caus}-throw \\
\glt He caused a snowfall that was nine flights of stairs thick. (Gesar, 149)
   \end{exe} 

Ideophones, whose normal position in the sentence is to the  left of the verb, are commonly right-dislocated in Japhug, and this also occurs with relative clauses with relativized S, as in example \ref{ex:phoRphoR}.\footnote{In order to avoid misunderstanding concerning the interpretation of example \ref{ex:phoRphoR}, it is necessary to point out that the verb \ipa{rɤloʁ} `make a nest' (a denominal verb derived from the possessed noun \ipa{--loʁ} `(its) nest') is intransitive, and the relativized element in this clause is \ipa{pɣɤtɕɯ}  `bird', not   \ipa{ɯ-loʁ}  its nest'. }

  \begin{exe}
   \ex   \label{ex:phoRphoR}
 \gll [\ipa{pɣɤtɕɯ}\tete{}   	\ipa{kɤ-kɯ-nɯ-rɤloʁ}]\rc{}  	\ipa{pʰoʁpʰoʁ}  	\ipa{nɯ}  	\ipa{ɣɯ}  	\ipa{ɯ-loʁ}  	\ipa{nɯ-ŋgɯ}  	\ipa{nɯ}  	\ipa{ra,}  	\ipa{ɯʑo}  	\ipa{ɕ-tu-ndze}  \\
 bird \textsc{pfv-nmlz:S/A-auto}-make.a.nest \textsc{ideo:stative:}well \textsc{dem} \textsc{gen} \textsc{3sg.poss}-nest \textsc{3pl.poss}-inside C \textsc{pl} he \textsc{cisloc-ipfv}-eat[III] \\
 \glt  He goes into the nests of birds that have made nice nests, and eats them. (The buzzard, 3)
   \end{exe}  

  
  Since the general word order in Rgyalrong language is verb-final, it is possible to consider the relative clauses seen above as internally-headed relatives, rather than simply post-nominal relatives; we will see that this analysis better accounts for the relativization of A and O arguments, and thus allows for a more uniform treatment of all cases.

Other adjuncts can be extracted from the relative. Thus, the group \ipa{ki jamar} `like this', while it normally appears within the relative (as in \ref{ex:kWzri}), can also be extraposed to its right (example \ref{ex:ki.jamar1}).

\begin{exe}
   \ex  \label{ex:kWzri}
   \gll
\ipa{ɯ-ndzɣi}   	\ipa{rcanɯ}   	[\ipa{ki}   	\ipa{jamar}   	\ipa{kɯ-rɲɟi}]   	\ipa{ɲɯ-ɕti}    \\
\textsc{3sg.poss}-tusk \topic{}   \textsc{dem.prox} about \textsc{nmlz:S}-long \textsc{testim}-be:\textsc{assert} \\
\glt Its tusks are long like this. (Elephant, 17)
\end{exe}

While the literal meaning of \ref{ex:ki.jamar1} would appear to be `there are thick ones that are like this' (this is also a possible interpretation, with a different syntactic structure), the context makes it clear that it must be translated as indicated below.

\begin{exe}
   \ex  \label{ex:ki.jamar1}
   \gll
[$\emptyset_j$ \ipa{kɯ-jpɯ\textasciitilde{}jpum}]\rc{}   	[\ipa{ki}   	\ipa{jamar}]_j   	\ipa{ɣɤʑu.}   \\
   { } \textsc{nmlz:S-emph}\textasciitilde{}thick \textsc{dem.prox} about exist:\textsc{sensory} \\
\glt There are (yak horns) that are thick like this. (Wild yak, 25)
\end{exe}


  
In the case of semi-transitive verbs (see section \ref{sec:trans}), the S  can be relativized with \ipa{kɯ}-- (the latter is discussed in section \ref{sec:other}). We find relative clauses with overt S (example \ref{ex:kWrga}) or overt semi-object (as \ref{ex:paR} )  in pre-verbal position, but no example with both overt nouns.

 \begin{exe}
   \ex   \label{ex:kWrga}
 \gll  [\textbf{\ipa{tɯrme}}\tete{}  	\ipa{kɯ-rga,}]\rc{}  	[\ipa{wuma}  	\ipa{ʑo}  	\ipa{kɤ-ndza}  	\ipa{kɯ-rga}]\rc{}  	\ipa{ɣɤʑu.} \\
person \textsc{nmlz:S}-like very \textsc{emph} \textsc{inf}-eat  \textsc{nmlz:S}-like exist:\textsc{sensory} \\
 \glt  There are persons who like it, who like to eat it. (βlamajmɤɣ, 60)
   \end{exe} 

 \begin{exe}
   \ex   \label{ex:paR}  
\gll  [\ipa{paʁ}\tete{}  	\ipa{mɤ-kɯ-ɤro}]\rc{}  	\ipa{maka}  	\ipa{me}  	\\
pig \textsc{neg-nmlz:P}-own  at.all not.exist:\textsc{fact} \\
 \glt  There is not anybody who does not have a pig. (Pigs, 3)
   \end{exe} 
   
The adjunct of verbs of this type is relativized with the prefix \ipa{kɤ--} as the P of a transitive verb (cf section \ref{sec:other}).

Totalitative reduplication of the S is common, but only with non-finite verbs, as in \ref{ex:kakwnandza}.


\begin{exe}
   \ex  \label{ex:kakwnandza}
\gll [<quanxian>  	\ipa{tɕe}  	\ipa{kɯ\textasciitilde{}kɤ-kɯ-nɤndza}]\rc{}  	\ipa{nɯ}  	\ipa{ɲɤ-ɣɤme.}      	\\
the.whole.county \textsc{lnk}  \textsc{total\textasciitilde{}pfv-nmlz:S}-have.leprosy \topic{} \textsc{evd}-destroy \\
 \glt  (This doctor) cured (suppressed) all lepers in the whole county. (leprosy 72)
   \end{exe} 

  
Prenominal S-relativization is relatively uncommon in our corpus. One case when this happens is to   allow stacking of relatives, which is impossible with head-internal clauses. Thus, one can find examples with a head-internal relative preceded by a prenominal one as in \ref{ex:mANidpon}.

\begin{exe}
   \ex  \label{ex:mANidpon}
\gll [[\ipa{mɤŋi}  	\ipa{kɤ-kɯ-ɣe}]\rc{}  	\ipa{χpɯn}\tete{}  	\ipa{tʰɯ-kɯ-rgɤz}]\rc{}  	\ipa{ci}  	\ipa{pjɤ-tu}  	\ipa{tɕe,}    	\\
Mangi \textsc{pfv:east-nmlz:S}-come[II] monk \textsc{pfv-nmlz:S}-old \textsc{indef} \textsc{evd.ipfv}-exist   \textsc{lnk} \\
 \glt  There was an old monk who had come from Mangi. (The prank, 19)
   \end{exe} 


Prenominal S-relativization is also possible with long relatives containing several adjuncts, as illustrated by example \ref{ex:pri1} vs \ref{ex:pri2}.

%Line 62: ɯ-xso kukɯtɕu pɕoʁ, ku-kɯ-nɯ-rɤʑi qɤjdo ci tu.

\begin{exe}
   \ex  \label{ex:pri1}
\gll [\ipa{ndzɤpri}\tete{}  	\ipa{nɯ}  	\ipa{koʁmɯz}  	\ipa{ɯ-ɣɲɟɯ}  	\ipa{ɯ-ŋgɯ}  	\ipa{tu-kɯ-nɯ-ɬoʁ}]\rc{}  	\ipa{ci}  	\ipa{pjɤ-mto-ndʑi.}  \\
\textbf{bear} \textsc{dem} just.before \textsc{3sg.poss}-hole \textsc{3sg.poss}-inside \textsc{ipfv:up-nmlz:S-auto-}come.out \textsc{indef} \textsc{evd}-see-\textsc{du}\\
   \ex  \label{ex:pri2}
\gll  [\ipa{nɯ}  	\ipa{koʁmɯz}  	\ipa{ɯ-ɣɲɟɯ}  	\ipa{ɯ-ŋgɯ}  	\ipa{tu-kɯ-nɯ-ɬoʁ}]\rc{}  	\ipa{ndzɤpri}\tete{} 	\ipa{ci}  	\ipa{pjɤ-mton-dʑi.}   \\
 \textsc{dem} just.before \textsc{3sg.poss}-hole \textsc{3sg.poss}-inside \textsc{ipfv:up-nmlz:S-auto-}come.out \textbf{bear} \textsc{indef} \textsc{evd}-see-\textsc{du}\\
\glt They saw  a bear which was just coming out of his lair. (semi-elicitation based on the Aesop story, the two friends and the bear)
\end{exe}
\subsubsection{A}
Like the S, the A can only be relativized with a \ipa{kɯ--}participle. Two main constructions are found,  prenominal relatives (as in \ref{ex:cnat}, \ref{ex:wkwtshi}  and \ref{ex:wnwkwnwBde}) or internally-headed relatives (as in \ref{ex:WkWnWmbrApW}).

When the A argument is relativized, the nominalized verb has two prefixes: the same marker \ipa{kɯ}-- as in S-relativisation, and a possessive prefix coreferent with the patient, as in examples \ref{ex:wkwtshi} and \ref{ex:wnwkwnwBde}.


\begin{exe}
   \ex  \label{ex:wkwtshi}
\gll [\ipa{tɯ-nɯ}  	\ipa{ɯ-kɯ-tsʰi}]\rc{}  	\ipa{tɤpɤtso}\tete{}  	\ipa{ɣɯ}  	\ipa{ɯ-kɯ-mŋɤm}  	\ipa{ɲɯ-ŋu}  \\
\textsc{indef.poss}-breast \textsc{3sg-nmlz:A}-drink child \textsc{gen} \textsc{3sg.poss-nmlz:S}-be.painful \textsc{testim}-be \\
\glt It is a disease of children who drink milk from the breast. (Children diarrhea, 3)
\end{exe}

\begin{exe}
   \ex  \label{ex:wnwkwnwBde}
\gll  
 [\ipa{iɕqʰa}  	\ipa{tɯrpa}  	\ipa{ɯ-nɯ-kɯ-nɯ-βde}]\rc{}  	\ipa{tɯrme}\tete{}  	\ipa{nɯ}  	\ipa{ra}  	\ipa{tɯrpa}  	\ipa{ɯ-kɯ-ɕar}  	\ipa{jo-ɣi-nɯ}  \\
 the.aforementioned axe \textsc{3sg-pfv-nmlz:A-auto}-throw people \topic{} \textsc{pl} axe \textsc{3sg-nmlz:A}-search \textsc{evd}-come-\textsc{pl} \\
\glt The people who had lost the axe came to look for it. (semi-elicitation based on the Aesop story, the travelers and the axe)
\end{exe}

However, when the nominalized verb has additional suffixes, such as TAM or negation markers, the possessive prefix coreferent with the P is only optionally present, and is generally elided as in \ref{ex:cnat}.

\begin{exe}
   \ex  \label{ex:cnat}
\gll [\ipa{ɕnat}  	\ipa{tu-kɯ-rɤɕi}]\rc{}  	\ipa{ndʑu}\tete{}  	\ipa{nɯnɯ,}   	\ipa{ɕnat-ndʑu}  	\ipa{rmi,}  \\
weft \textsc{ipfv-nmlz:A}-pull stick \topic{}  weft-stick be.called:\textsc{fact} \\
\glt The stick that  pulls the weft is called the `weft-stick'. (colored belts 64)
\end{exe}
 

All three examples \ref{ex:cnat},  \ref{ex:wkwtshi} and \ref{ex:wnwkwnwBde} show that prenominal relatives with relativized A can be restrictive relatives.

Examples such as \ref{ex:WkWnWmbrApW} with overt A marked with the ergative are extremely rare in our corpus; prenominal relatives are by far the preferred strategy for relativization of A.
\begin{exe}
   \ex  \label{ex:WkWnWmbrApW}
\gll [[\ipa{tɤpɤtso}  	\ipa{ci}  	\ipa{kɯ}]\tete{}  	<yangma> 	\ipa{ɯ-kɯ-nɯmbrɤpɯ}]\rc{}  	\ipa{ci}  	\ipa{jɤ-ɣe}  \\
boy \textsc{indef} \textsc{erg} bicycle \textsc{3sg-nmlz:A}-ride \textsc{indef} \textsc{pfv}-come[II] \\
\glt A boy who was riding a bicycle arrived. (Pear story, Chenzhen, 5)
\end{exe}


Totalitative reduplication alone cannot be used in the case of A-relativization, and it must be applied to a verb form already nominalized with the prefix \ipa{kɯ}--, as in example \ref{ex:qartshaz}.

\begin{exe}
   \ex \label{ex:qartshaz}
   \gll \ipa{qartsʰaz}  	\ipa{mu}  	\ipa{nɯra,}  	[\ipa{pɯ\textasciitilde{}pɯ-kɯ-mtsʰɤm}]\rc{}  	\ipa{nɯ}  	\ipa{ɯ-rkɯ}  	\ipa{nɯtɕu}  	\ipa{tu-owɯwum-nɯ}  	\ipa{ŋu}  \\
   deer female \textsc{dem:pl} \textsc{total\textasciitilde{}pfv-nmlz:A}-hear \textsc{dem} \textsc{3sg.poss}-side there \textsc{ipfv}-gather-\textsc{pl} be:\textsc{fact} \\
\glt The female deer, all those who hear it (the male's call) gather around it. (Deer, 133)
\end{exe}

\subsubsection{Possessor}
When possessors are relativized, the possessed noun remains \textit{in situ} and the verb are nominalized with the prefix \ipa{kɯ}--. A resumptive possessive prefix on the possessed noun is obligatory whether the possessor is overt (as in \ref{ex:WRrWkWtu}) or not (\ref{ex:lrWba}).


      \begin{exe}
   \ex \label{ex:WRrWkWtu}
 \gll 
\ipa{akɯ}   	\ipa{zɯ}   	[\ipa{qapri}\tete{}   	\ipa{ci}   	\ipa{ɯ}\tete{}-\ipa{kɤχcɤl}  	\ipa{ɯ}\tete{}-\ipa{ʁrɯ}   	\ipa{kɯ-tu}]\rc{}   	\ipa{ci}   	\ipa{ɣɤʑu}   	\ipa{tɕe,}   \\
east \textsc{loc} snake \textsc{indef} \textsc{3sg.poss}-middle.of.the.head  \textsc{3sg.poss}-horn \textsc{nmlz:S}-exist \textsc{indef} exist:\textsc{sensory}  \textsc{lnk} \\
\glt In the east, there is a snake with a horn in the middle of his head.  (The divination, 43)
\end{exe}
 
       \begin{exe}
   \ex \label{ex:lrWba}
 \gll 
[\ipa{iɕqʰa}   	 \ipa{nɯ}\tete{}-\ipa{me}   	\ipa{lʁɯba}   	\ipa{kɯ-ŋu}]   	\ipa{ra}   	\ipa{ɣɯ}   	\ipa{nɯ-kʰɤru}   	\ipa{lɤ-nɯ-ɬoʁ,}   \\
the.aforementioned \textsc{3pl.poss}-daughter mute \textsc{nmlz:S}-be \textsc{pl} \textsc{gen} \textsc{3pl.poss}-kitchen.door \textsc{pfv:upstream-auto}-come.out \\
\glt  As he entered the  door of the kitchen of those whose daughter was mute.  (The divination2, 55)
\end{exe}
 
When the possessor is   first or second person,  the resumptive possessive prefixes  are not neutralized to third person (see example \ref{ex:kWtshoz}).
        \begin{exe}
   \ex \label{ex:kWtshoz}
 \gll 
[\ipa{nɤ}\tete{}-\ipa{mu}   	\ipa{nɤ}\tete{}-\ipa{wa}   	\ipa{kɯ-tsʰoz}]\rc{}   	\ipa{tɯ-ŋu,}   	\ipa{aʑo}   	[\ipa{a}\tete{}-\ipa{mu}   	\ipa{kɯ-me}]\rc{}   	\ipa{ŋu-a}   	\ipa{tɕe}    \\
\textsc{2sg.poss}-mother \textsc{2sg.poss}-father \textsc{nmlz:S}-complete 2-be:{fact} I \textsc{1sg.poss}-mother \textsc{nmlz:S}-not.exist be:{fact}-\textsc{1sg} \textsc{lnk} \\
\glt You are someone whose father and mother are all there, I am someone without a mother. (Nyima wodzer, 12)
\end{exe}

Ideophones are commonly extracted outside of the relative clause as in example \ref{ex:takwgrum}, as shown by the presence of the determiner \ipa{ci} `a' just after the relative.

     \begin{exe}
   \ex \label{ex:takwgrum}
 \gll \ipa{praʁkʰaŋ}   	\ipa{zɯ}   	[\ipa{tɤ-mu}\tete{}   	\ipa{ci}   	\ipa{ɯ}\tete{}-\ipa{ku}   	\ipa{tɤ-kɯ-wɣrum}]\rc{}   	 	\ipa{ci} \ipa{zɯŋzɯŋ}    	\ipa{pjɤ-rɤʑi}   	\ipa{tɕe,}        \\
cave \textsc{loc} \textsc{indef.poss}-mother \textsc{indef} \textsc{3sg.poss}-head \textsc{pfv-nmlz:}S-white \textsc{indef} \textsc{ideo}:II:completely.white \textsc{evd.ipfv}-remain \textsc{lnk}  \\
\glt In the cave, there was an old lady whose hair was completely white. (The prince, 68)
\end{exe}

 
Only non-finite head-internal relatives have been observed for relativization of the possessor.

\subsection{\ipa{kɤ--} participle} \label{sec:kA.rel}

\subsubsection{P}
The P argument can be relativized by nominalizing the verb with the prefix \ipa{kɤ}--. Such non-finite relatives are most commonly internally-headed, as illustrated by examples \ref{ex:ragdwt}, \ref{ex:nandzwt}, \ref{ex:sazgwr} and \ref{ex:qaR}. 


     \begin{exe}
   \ex \label{ex:ragdwt}
\gll 
[\ipa{nɯŋa}  	\ipa{ɯ-ndʐi}\tete{}  	\ipa{tʰɯ-kɤ-rɤɣdɯt,}]\rc{}  	\ipa{tʰɯ-kɤ-tʂɯβ}  	\ipa{nɯ}  	\ipa{ɯ-ŋgɯ}  	\ipa{nɯ}  	\ipa{tɕu}  	\ipa{ko-ɕe}  \\
cow \textsc{3sg.poss-}skin \textsc{pfv-nmlz:P}-skin \textsc{pfv-nmlz:P}-sew \textsc{dem} \textsc{3sg.poss-}inside \textsc{dem} \textsc{loc} \textsc{evd:east}-go \\
  \glt  He went into the cow hide that had been  skinned and sewed.    (The flood2, 32)
   \end{exe}  

     \begin{exe}
   \ex \label{ex:nandzwt}
\gll   [\ipa{tɯrme}\tete{}  	\ipa{mɤ-kɤ-nɯfse}]\rc{}  	\ipa{jɤ-ɣe}  	\ipa{tɕe}  	\ipa{tu-nɯ-ɤndzɯt}\\
person \textsc{neg-nmzl:P}-know \textsc{pfv}-come[II] \textsc{lnk} \textsc{ipfv-appl}-bark\\
  \glt  When an unknown person  comes, it barks at him.  (The dogs, 9)
   \end{exe}  

     \begin{exe}
   \ex \label{ex:sazgwr}
\gll [\ipa{cʰɤmdɤru}\tete{}  	\ipa{tɤ-kɤ-sɯ-ɤzgɯr}]\rc{}  	\ipa{nɯ}  	\ipa{ɲɤ-sɯ-ɤstu-nɯ}  	\ipa{qʰe,}  	\ipa{tɕe}  	\ipa{to-mna}  \\
drinking.straw \textsc{pfv-nmlz:P-caus}-bent \topic{} \textsc{evd-caus}-straight-\textsc{pl} coord coord \textsc{evd}-recover \\
\glt He put straight the straw that  had been bent, and (her son) recovered. (Gesar 315)
   \end{exe}  
   

There are two types of  participial relatives in \ipa{kɤ}-- in Japhug. The first type, illustrated by examples  \ref{ex:ragdwt}, \ref{ex:nandzwt}, \ref{ex:sazgwr} and \ref{ex:qaR}, have a TAM marker but no possessive prefix. Such sentences cannot be used when the agent is  SAP, and imply a indefinite agent. They cannot be used with an overt agent marked with the ergative, but there are several sentences such as \ref{ex:qaR} with an instrument marked in the ergative placed before the head noun.\footnote{ \citet{jacksonlin07} analyse constructions of this type in Tshobdun as   containing the passive derivational prefix in combination with the S/A nominalizer \ipa{kə}--. This analysis is possible historically for Japhug too, as the combination of the S/A nominalizing prefix \ipa{kɯ}-- with the passive \ipa{a}-- regularly yields \ipa{kɤ}-- (see \citealt{jacques07passif} and \citealt{jacques12demotion}). However, it is unclear whether this analysis is still valid synchronically; we leave this question for further research.}

\begin{exe}
   \ex \label{ex:qaR}
\gll  \ipa{tɕe}  	\ipa{stɤmku}  	\ipa{ɯ-ndo}  	\ipa{nɯ} \ipa{ra,}  	[\ipa{rŋɯl}  	\ipa{kɯ} \ipa{qaʁ}\tete{}  	\ipa{tʰɯ-kɤ-sɯ-βzu}]\rc{} 	\ipa{nɯ} \ipa{ra}  	\ipa{ko-sɤʑɯrja-nɯ.}   \\
\textsc{lnk} plain \textsc{3sg.poss}-side \textsc{dem} \textsc{pl} 
silver \textsc{erg} ploughshare \textsc{pfv-nmlz:P-caus}-make  \textsc{dem} \textsc{pl} \textsc{evd}-put.in.order-\textsc{pl}\\
  \glt On the side of the plain, they placed in order the ploughshares that had been made with silver. (The raven4.97)
   \end{exe}  


The second type of verb nominalized with \ipa{kɤ}-- has no TAM directional prefix, but requires a possessive prefix coreferent with the A. Relatives with this type of nominalized verb mainly occur in prenominal relatives or headless relatives, as \ref{ex:tajmag} and \ref{ex:khu} respectively.

     \begin{exe}
   \ex \label{ex:tajmag}
   \gll
[\ipa{aʑo}  	\ipa{a-mɤ-kɤ-sɯz}]\rc{}  	\ipa{tɤjmɤɣ}\tete{}  	\ipa{nɯ}  	\ipa{kɤ-ndza}  	\ipa{mɤ-naz-a}  \\
\textsc{1sg} \textsc{1sg-neg-nmlz:P}-know mushroom \textsc{dem} \textsc{inf}-eat \textsc{neg}-dare-\textsc{1sg} \\
\glt I do not dare to eat the mushrooms that I do not know. (mbrɤʑɯm,103)
\end{exe}
 

  
Unlike in Tshobdun (\citealt[10]{jacksonlin07}), in Japhug non-finite relatives with possessive prefixes are not restricted to generic state of affairs, but can refer to particular situations as in \ref{ex:khu}.


     \begin{exe}
   \ex \label{ex:khu}
   \gll  \ipa{lɤ-fsoʁ}  	\ipa{ɯ-jɯja}  	\ipa{nɯ}  	\ipa{pjɯ-ru}  	\ipa{tɕe}  	[\ipa{ɯ-kɤ-nɯmbrɤpɯ}]\rc{}  	\ipa{nɯ}  	\ipa{kʰu}  	\ipa{pɯ-ɕti}  	\ipa{ɲɯ-ŋu,}  \\
\textsc{pfv}-be.clear    \textsc{3sg}-along  \textsc{dem} \textsc{ipfv:down}-look \textsc{lnk} \textsc{3sg-nmlz:P}-ride \topic{} tiger \textsc{pst.ipfv}-be.\textsc{assert}  \textsc{testim}-be \\
\glt As the day was breaking, looking down, he (progressively realized that) what he was riding was a tiger. (Tiger, 20)
\end{exe}

However, these relatives cannot be used for perfective relativization. To express a meaning such as `the thing that I have  seen', the only possibility in Japhug is to use a finite relative.


\subsubsection{R}
\subsubsection{Semi-object}

\subsection{\ipa{sɤ--} participle} \label{sec:sA.rel}

\section{Complementation} \label{sec:complement}
\citet{jacques15causative}
\citet{sun12complementation}

\subsection{Infinitival clauses} \label{sec:inf1}

\subsection{Bare infinitival clauses} \label{sec:inf2}


\begin{exe}
\ex \label{ex:bare.inf}
\gll \ipa{nɤʑo} 	\ipa{kɯ-fse} 	\ipa{a-ŋkʰor} 	\ipa{nɯ} 	\ipa{ɯ-mto} 	\ipa{mɯ-pɯ-rɲo-t-a} \\
you \textsc{nmlz:stative}-be.like \textsc{1sg.poss}-subject \textsc{top} \textsc{3sg}-\textsc{bare.inf:}see \textsc{neg-pfv}-experience-\textsc{pst:tr-1sg} \\
\glt  `I never saw anyone like you among my subjects.' (Smanmi metog koshana1.157)
\end{exe}


\begin{exe}
\ex \label{ex:bare.inf.noun}
\gll \ipa{ndʑi-mi}   	\ipa{ɯ-tsʰoʁ}   	\ipa{ɯ-tsʰɯɣa}   	\ipa{nɯra}   	\ipa{wuma}   	\ipa{ʑo}   	\ipa{naχtɕɯɣ-ndʑi.}   \\
\textsc{3du.poss}-foot \textsc{3sg}-\textsc{bare.inf:}attach.to \textsc{3sg.poss}-form \textsc{top:pl} very \textsc{emph}  \textsc{npst}:similar-\textsc{du}  \\
\glt `The way their feet (of fleas and crickets) touch the ground is very similar.' (the cricket 17)
\end{exe}


\subsection{Agent participle} 

\section{Degree} \label{sec:degree}
\section{Manner and purpose} \label{sec:linking}
\citet{jacques14linking}
 nmlz > converb
 
\section{Historical perspectives} \label{sec:historical}
 \citet{konnerth09nmlz} 
 
 
\section{Conclusion}

\bibliographystyle{unified}
\bibliography{bibliogj}
\end{document}