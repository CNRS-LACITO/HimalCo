\documentclass[oldfontcommands,oneside,a4paper,11pt]{article} 
\usepackage{fontspec}
\usepackage{natbib}
\usepackage{booktabs}
\usepackage{xltxtra} 
\usepackage{polyglossia} 
\usepackage[table]{xcolor}
\usepackage{gb4e} 
%\usepackage{multicol}
\usepackage{hyperref} 
\hypersetup{bookmarks=false,bookmarksnumbered,bookmarksopenlevel=5,bookmarksdepth=5,xetex,colorlinks=true,linkcolor=blue,citecolor=blue}
\usepackage[all]{hypcap}
\usepackage{memhfixc}

 
%\setmainfont[Mapping=tex-text,Numbers=OldStyle,Ligatures=Common]{Charis SIL} 
\newfontfamily\phon[Mapping=tex-text,Ligatures=Common,Scale=MatchLowercase]{Charis SIL} 
\newcommand{\ipa}[1]{\textbf{{\phon\mbox{#1}}}} %API tjs en italique
 \newcommand{\ipab}[1]{{\phon \mbox{#1}}} %API tjs en italique
\newcommand{\grise}[1]{\cellcolor{lightgray}\textbf{#1}}
\newfontfamily\cn[Mapping=tex-text,Ligatures=Common,Scale=MatchUppercase]{SimSun}%pour le chinois
\newcommand{\zh}[1]{{\cn #1}}

 

 \begin{document} 
 \title{Associated motion in Manchu in typological perspective }
\author{José Andrés Alonso de la Fuente \\ Guillaume Jacques}
\maketitle 

\section*{Introduction}
The category of `Associated Motion' (henceforth AM), first discovered in Pama-Nyungan languages (\citealt{koch84associated.motion}, \citealt{wilkins91associated.motion}), is progressively becoming a topic of typological research in other areas, in particular South America (see \citealt{guillaume08cavinena}, \citealt{guillaume16am}), and but also Eurasia (for instance, Sino-Tibetan, \citealt{jacques13harmonization}). Following \citet[12]{guillaume16am}, AM markers can be defined as `grammatical morpheme[s] that [are] associated with the verb
and that [have] among [their] possible functions the coding of translational
motion.'

Most languages with AM can also express the association of translational motion with an action (`go to X, go and X, go after X etc')  by means of motion verb constructions with purposive construction (henceforth MVC).\footnote{Exceptions include Algonquian languages such as Ojibwe (\citealt[729-733]{valentine01grammar}), where only AM markers can be used to express meanings such as `go/come to X', and motion verbs apparently cannot take purposive complements.} This raises the question of the semantic value of the contrast between AM and MVC in languages where both constructions are possible. This important question, however, as pointed out by \citet[10]{guillaume16am}, has not yet been the focus of much research in the typological literature.

This paper deals with the system of AM in the Manchu language, and focuses on the precise semantic value of the AM suffixes, in particular their differences with the corresponding MVC and their interaction with the causative. 

This paper comprises three sections. First, we present an overview of previous studies of AM, MVC and valency-increasing morphemes in the literature. Second, we provide a detailed description of AM in Manchu on the basis of a philological investigation of authentic early Manchu texts from the XVII-XVIII centuries (excluding translated documents). Third, we investigate how the commonalities and differences in the use of AM in Manchu and other languages such as Japhug can be account for by differences in grammaticalization pathways.

\section{Associated motion}
In this section, we present three typological questions that deserve to be treated when describing any AM system: the parameters of the AM system, the semantic difference between AM markers and their MVC equivalent, and the interaction between AM and derivational morphemes such as the cuasative.

\subsection{Parameters}
In languages with complex AM systems such as Tacanan and Pama-Nyungan, \citet[8]{guillaume16am} identifies three main parameters parameters: \textsc{path} (cislocative, translocative, circumambulative etc),  \textsc{temporal relation} (prior, concurrent, subsequent motion) and moving argument (S/A vs O). 

In languages with simpler AM systems comprising only two morphemes as in Manchu, there is little variety regarding these parameters: only cislocative vs translocative AM markers of prior motion of the (S/A) subject are attested.

\subsection{Associated motion vs motion verb construction}
As mentioned above, few previous publications have addressed the issue of the contrast between AM and MVC. This topic is easier to study in languages with simpler AM systems, in the case when there is a one-to-one correspondence between AM and MVC.

One of the few languages where this question has been studied is Japhug (Rgyalrongic, Trans-Himalayan). In Japhug, the semantic difference between AM and MVC is most obvious in the aorist (\citealt[203]{jacques13harmonization}). In the MVC, a motion verb in the aorist implies that the motion has been completed, but does not specify whether the action referred to in the purposive complement clause has taken place or not. This can be shown by the fact that the action described by the purposive complement can be negated, as in examples (\ref{ex:mvc.jpg}) and  (\ref{ex:mvc.jpg2}).

 \begin{exe}
\ex \label{ex:mvc.jpg}
\gll  \ipa{laχtɕʰa} \ipa{ɯ-kɯ-χtɯ} \ipa{jɤ-ari-a}  \ipa{ri} \ipa{tɤ-χtɯ-t-a} \ipa{maka} \ipa{me} \\
thing \textsc{3sg}-\textsc{nmlz}:S/A-buy \textsc{aor}-go[II]-\textsc{1sg} but \textsc{aor}-buy-\textsc{pst-1sg} at.all \textsc{n.pst}:not.have\\
\glt  I went to buy things \textbf{but did not buy anything}.
\ex \label{ex:mvc.jpg2}
\gll  \ipa{kɯ-nɯ-rŋgɯ}  	\ipa{jɤ-ari-a}  	\ipa{ri}  	\ipa{kɤ-nɯ-rŋgɯ}  	\ipa{mɯ-pɯ-ŋgrɯ}  	\\  
\textsc{nmlz}:S/A-\textsc{auto}-lie.down 	\textsc{aor}-go[II]-\textsc{1sg} but	\textsc{inf}--\textsc{auto}-lie.down 	\textsc{neg-pst.ipfv}-succeed \\
\glt  I went to sleep \textbf{but could not sleep}.
  \end{exe}
  
On the other hand, AM prefixes imply that \textbf{both} the motion and the subsequent action have both been completed, and it is therefore nonsensical to negate the action described by the verb, as in examples (\ref{ex:am.jpg}) and (\ref{ex:am.jpg2}).

 \begin{exe}
\ex \label{ex:am.jpg}
\gll  *\ipa{laχtɕʰa} \ipa{ɕ-tɤ-χtɯ-t-a}  \ipa{ri} \ipa{tɤ-χtɯ-t-a} \ipa{maka} \ipa{me} \\
thing \textsc{transloc}-\textsc{aor}-buy-\textsc{pst-1sg} but \textsc{aor}-buy-\textsc{pst-1sg}  at.all \textsc{n.pst}:not.have\\
\glt  (intended meaning: sentence (\ref{ex:mvc.jpg}) 
\ex \label{ex:am.jpg2}
\gll  *\ipa{ɕ-pɯ-nɯ-rŋgɯ-a}  	\ipa{ri}  	\ipa{kɤ-nɯ-rŋgɯ}  	\ipa{mɯ-pɯ-ŋgrɯ}  \\
\textsc{transloc-aor-auto}-lie.down-\textsc{1sg} but	\textsc{inf-auto}-lie.down 	\textsc{neg-pst.ipfv}-succeed \\
\glt  (intended meaning: sentence (\ref{ex:mvc.jpg2}) 
\end{exe}  

This finding, tested on the basis of elicited examples, is consistent with natural data from narratives and conversations (see \citealt{jacques13harmonization} for a detailed account of both constructions).
  
We will show in section \ref{sec:manchu} that a similar contrast is observed in Manchu, suggesting that this semantic difference may be crosslinguistically widespread.
  
\subsection{AM and causative}
All AM systems described up to now display accusative alignment, and only a handful of languages have object AM markers. 

However, in the case of causative constructions, the motion argument can at least in some languages be either the causee or the causer. In a languages such as Japhug with highly templatic morphology, the semantic scope of the causative is ambiguous (\citealt[182]{jacques15causative}). the causative can either only apply to the action expressed by the verb root (as in  \ref{ex:am.jpg.caus}, `go and cause to X')  or apply to the motion (as in \ref{ex:am.jpg.caus2}, `cause to go and X').


  \begin{exe}
\ex \label{ex:am.jpg.caus}
\gll
\ipa{mpʰrɯmɯ} 	\ipa{ɕ-pɯ-sɯ-re} 	\ipa{tɕe,} 	\ipa{ɕ-tɤ-tʰe} 	\ipa{ra} \\
divination \textsc{transloc-imp-caus}-look[III] \textsc{lnk} transloc-imp-ask[III] have.to:\textsc{fact} \\
\glt  `Go and have him make a divination and ask him about it.' (divination03 7)
  \end{exe} 

  \begin{exe}
\ex \label{ex:am.jpg.caus2}
\gll
\ipa{tɕe} 	\ipa{kupa} 	\ipa{cʰu} 	\ipa{nɯra} 	\ipa{atʰi} 	\ipa{pɕoʁ} 	\ipa{nɯra,} 	\ipa{ɯ-pɕi} 	\ipa{nɯra} 	\ipa{kɯ} 	\ipa{kɯreri} 	\ipa{ɣɯ-cʰɯ-sɯ-χtɯ-nɯ} 	\ipa{ŋu.}  \\
\textsc{lnk} Chinese \textsc{loc} \textsc{dem:pl} downstream direction \textsc{dem:pl} \textsc{3sg}-outside  \textsc{dem:pl}  \textsc{erg} here \textsc{cisloc-ipfv:downstream-caus}-buy-\textsc{pl} be\textsc{:fact} \\
\glt `People from the Chinese areas, people from outside send people to come here to buy (matsutake and sell them in the areas downstream).' (hist-20-grWBgrWB 58)
  \end{exe} 
  
  In section \ref{sec:manchu}, we will explain how Manchu and Japhug differ in this regard.
  
\section{AM in Classical Manchu} \label{sec:manchu}
%UG=  \citealt{nikolaeva01udihe}

Associated motion in Manchu is expressed by two suffixes: \ipa{-nji-} (cislocative or venitive) and \ipa{-na-/-ne-} (translocative or andative). These suffixes are generally considered to be grammaticalized from the corresponding motion verb constructions \ipa{-me gene-} and \ipa{-me ji-} with the verb of the purposive complement marked with the imperfective converb suffix \ipa{-me} (a more detailed account of the origin of these suffixes is provided in section \ref{sec:grammaticalization}). Examples (\ref{ex:go.to}) and (\ref{ex:come.to}) illustrate both associate motion and their corresponding motion verb construction.

\begin{exe}
\ex \label{ex:go.to}
\begin{xlist}
\exi{(a)} \label{ex:alanambi}
\gll \ipa{ala-na-mbi} \\
tell-\textsc{transloc-pre} \\
\exi{(b)} \label{ex:alame.genembi}
\gll \ipa{ala-me gene-mbi} \\
tell-\textsc{ipfv.conv} go-\textsc{pre} \\
\glt `go to tell'
\end{xlist}
\ex \label{ex:come.to}
\begin{xlist}
\exi{(a)} \label{ex:alanjimbi}
\gll \ipa{ala-nji-mbi} \\
tell-\textsc{cisloc-pre} \\
\exi{(b)}  \label{ex:alame.jimbi}
\gll \ipa{ala-me ji-mbi} \\
tell-\textsc{ipfv.conv} come-\textsc{pre} \\
\glt `come to tell'
\end{xlist}
\end{exe}


Cislocative and translocative are usually described alongside the so-called missive \ipa{-nggi-} ‘to send to V’ (from \ipa{unggi-} ‘to send, dispatch’), e.g. \ipa{ala-nggi-} ‘to send to say’, etc. We defer a detailed discussion of this suffix to future research, as it is far less frequent than the cislocative and translocative.\footnote{It is likely that the same conclusions drawn in this paper apply to the missive too (i.e., \ipa{ala-nggi-} and \ipa{ala-me unggi-} are not always synonymous).}

It is traditionally held that predicates with associated motion suffixes and motion verb constructions are synonymous, as suggested by the translations of examples (\ref{ex:go.to}) and (\ref{ex:come.to}). Such a characterization can be found in both grammars and handbooks, from the pioneering work of \citet[163-165 §§133-134]{zaxarov10manchu} to more recent works such as \citet[34-35]{pashkov63manchu}, \citet[173-174]{avrorin00manchu},  \citet[19]{li00manchu} or \citet[239-240]{gorelova02manchu}.\footnote{Chinese and Japanese grammars agree on this description as well as early European grammars. On a curious note, \citet[51]{harlez84mandchou} describes \ipa{-na-/-ne-} as inchoative (“Inchoatif”), e.g. \ipa{taci-} ‘apprendre’ $\rightarrow$ \ipa{taci-ne-} ‘aller apprendre’, and \ipa{-nji-} as illative (“Illatif”), e.g. ara-ji-me ‘venir écrire’. This seems to be De Harlez’s interpretation, based on Gabelentz’s data and examples (\citeyear[51]{gabelentz32mandchou}, it is not entirely clear why Gabelentz, and after him De Harlez, has \ipa{-ji-} instead of \ipa{-nji-}) and followed by later grammarians (e.g., \citet[367]{peeters40manjurische}). \citet[53]{haenisch61mandschu}, however, calls \ipa{-na-} “Illativ” and \ipa{-nji-} “Allativ” (cf.\citet[401b]{li00manchu} \ipa{-na-} “allative verbal suffic”, but \ipa{-nji-} is just called “verbal suffix” on p. 402a).}

Textual analysis\footnote{Our corpus includes three narrative texts, two of them from the 17th century or earlier (\textit{Manju-i yargiyan kooli}, \citealt{shunjuu92yargiyan} and \textit{Beye-i cooha bade yabuha babe ejehe bithe}, \citealt{cosmo06dzengseo}), the other text is of oral origin and later date, though it is believed that it continues a long tradition (\ipa{Nišan samani bithe}, \citealt{nowak77nisan} and \citealt{jaxontov93nisan}). We also include one text from the beginning of the 18th century (\ipa{Lakcaha jecen de takûraha babe ejehe bithe}, \citealt{shunjuu64tulishen}). These texts are not translations, but original native Manchu literary compositions. Modern editions allow us to confront the \textit{communis opinio} regarding the translation of associated motion formations with the idea suggested in the present contribution. For \citet{shunjuu92yargiyan} we provide page of the Manchu text in brackets and lines, for the remaining texts “page with Manchu text [Manchu manuscript page] / English, Japanese or Russian translation”.} shows that, generally speaking, motion verb constructions occur less frequently than associated motion formations. For instance, in \citet{cosmo06dzengseo} motion verb constructions appear 20 times (\ipa{-me gene-} 5x, \ipa{-me ji-} 15x), whereas associated motion suffixes appear 56 times (\ipa{-na-} 27x, \ipa{-nji-} 29x), a proportion comparable to what has been observed in Japhug narratives (\citealt[209]{jacques13harmonization}).\footnote{In other texts, such as \citealt{jaxontov93nisan} however, we find a almost equal number occurrence of the two constructions:  motion verb constructions appear 52 times (-me gene- 33x, -me ji- 29x) and associated motion suffixes 50 times (-na- 26x, -nji- 24x).}


\subsubsection{Lexicalization}
Numerous verbal forms in \ipa{-na-/-ne-} and \ipa{-nji-} show lexicalization, as illustrated by the examples in Table \ref{tab:lexicalization}.

\begin{table}[h]
\caption{Lexicalization of AM verbs in Manchu} \label{tab:lexicalization} \centering
\begin{tabular}{lllll}
\toprule
Base verb & AM form  \\
\midrule
\ipa{ebu-} ‘to dismount, get off,  & \ipa{ebu-nji-} ‘to come to dismount, \\
get down’& to descend (from a deity)’\\
 \ipa{wasi-} ‘to go down’ & \ipa{wasi-nji-} ‘to come down; to descend’\\
 \ipa{hafu-} ‘to communicate’ & \ipa{hafu-nji-} ‘to come (straight) through’ \\
 & \ipa{hafu-na-} ‘to connect with’\\
 \bottomrule
\end{tabular}
\end{table}

\begin{exe}
\ex 
\gll \ipa{baldubayan} 	\ipa{tere} 	\ipa{hehe} 	\ipa{be} 	\ipa{cincile-me} 	\ipa{tuwa-ci} 	\ipa{orin} 	\ipa{se-i} 	\ipa{šurdeme} 	\ipa{o-hobi.} 	\ipa{banji-ha} 	\ipa{arbun}/ 	\ipa{yargiyan} 	\ipa{i} 	\ipa{pan an gung} 	\ipa{ni} 	\ipa{sargan} 	\ipa{jui} 	\ipa{jalan} 	\ipa{de} 	\ipa{ebunji-he} 	\ipa{se-ci} 	\ipa{o-mbi.} \\
{Baldu Bayan} that woman \textsc{acc} scrutinize-\textsc{ipfv.conv} look-\textsc{cond} twenty year-\textsc{gen} \textsc{aux-pst.fin} around born-\textsc{pst} form real \textsc{gen} {Pan An gung} \textsc{gen} female son generation \textsc{loc} descend-\textsc{pst} say-\textsc{cond} \textsc{aux-aor} \\
\glt Baldu Bayan looked scrutinizing that woman. She was around twenty years old. It could be said that, (judging by her) appearance, she was a true-born daughter of gung Pan An. (\citealt[71,6a-6b/100]{jaxontov93nisan})
\end{exe}

These examples show that Manchu AM is a derivation, unlike AM in languages such as Japhug, where it is clearly inflectional morphology, with no cases of lexicalized AM markers.\footnote{The only potential case of lexicalization AM marker in Japhug is the verb `fetch', which is peculiar in requiring the presence of a AM prefix (\citealt[210]{jacques13harmonization}). However, we do not find any example of an AM prefix modifying the verb's meaning in an unpredictable way.}

\subsubsection{Difference between AM and MCV}

\subsubsection{Non-AM meanings}

\section{The grammaticalization of AM} \label{sec:grammaticalization}
serial verb construction vs purposive complement

inflectional vs derivational morphology

which is the most grammaticalized?

\section{Conclusion}

\bibliographystyle{unified}
\bibliography{bibliogj}

 \end{document}
 