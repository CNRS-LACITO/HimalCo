\documentclass[oldfontcommands,oneside,a4paper,11pt]{article} 
\usepackage{fontspec}
\usepackage{natbib}
\usepackage{booktabs}
\usepackage{xltxtra} 
\usepackage{polyglossia} 
\usepackage[table]{xcolor}
\usepackage{gb4e} 
%\usepackage{multicol}
\usepackage{hyperref} 
\hypersetup{bookmarks=false,bookmarksnumbered,bookmarksopenlevel=5,bookmarksdepth=5,xetex,colorlinks=true,linkcolor=blue,citecolor=blue}
\usepackage[all]{hypcap}
\usepackage{memhfixc}

 
%\setmainfont[Mapping=tex-text,Numbers=OldStyle,Ligatures=Common]{Charis SIL} 
\newfontfamily\phon[Mapping=tex-text,Ligatures=Common,Scale=MatchLowercase]{Charis SIL} 
\newcommand{\ipa}[1]{\textbf{{\phon\mbox{#1}}}} %API tjs en italique
 \newcommand{\ipab}[1]{{\phon \mbox{#1}}} %API tjs en italique
\newcommand{\grise}[1]{\cellcolor{lightgray}\textbf{#1}}
\newfontfamily\cn[Mapping=tex-text,Ligatures=Common,Scale=MatchUppercase]{SimSun}%pour le chinois
\newcommand{\zh}[1]{{\cn #1}}

 

 \begin{document} 
 \title{Associated motion in Manchu in typological perspective }
\author{José Andrés Alonso de la Fuente \\ Guillaume Jacques}
\maketitle 

\section*{Introduction}
The category of `Associated Motion' (henceforth AM), first discovered in Pama-Nyungan languages (\citealt{koch84associated.motion}, \citealt{wilkins91associated.motion}), is progressively becoming a topic of typological research in other areas, in particular South America (see \citealt{guillaume08cavinena}, \citealt{guillaume16am}), and but also Eurasia (for instance, Sino-Tibetan, \citealt{jacques13harmonization}). Following \citet[12]{guillaume16am}, AM markers can be defined as `grammatical morpheme[s] that [are] associated with the verb
and that [have] among [their] possible functions the coding of translational
motion.'

Most languages with AM can also express the association of translational motion with an action (`go to X, go and X, go after X etc')  by means of motion verb constructions with purposive construction (henceforth MVC).\footnote{Exceptions include Algonquian languages such as Ojibwe (\citealt[729-733]{valentine01grammar}), where only AM markers can be used to express meanings such as `go/come to X', and motion verbs apparently cannot take purposive complements.} This raises the question of the semantic value of the contrast between AM and MVC in languages where both constructions are possible. This important question, however, as pointed out by \citet[10]{guillaume16am}, has not yet been the focus of much research in the typological literature.

This paper deals with the system of AM in the Manchu language, and focuses on the precise semantic value of the AM suffixes, in particular their differences with the corresponding MVC and their interaction with the causative. 

This paper comprises three sections. First, we present an overview of previous studies of AM, MVC and valency-increasing morphemes in the literature. Second, we provide a detailed description of AM in Manchu on the basis of a philological investigation of authentic early Manchu texts from the XVII-XVIII centuries (excluding translated documents). Third, we investigate how the commonalities and differences in the use of AM in Manchu and other languages such as Japhug can be account for by differences in grammaticalization pathways.

\section{Associated motion}
In this section, we present three typological questions that deserve to be treated when describing any AM system: the parameters of the AM system, the semantic difference between AM markers and their MVC equivalent, and the interaction between AM and derivational morphemes such as the cuasative.

\subsection{Parameters}
In languages with complex AM systems such as Tacanan and Pama-Nyungan, \citet[8]{guillaume16am} identifies three main parameters parameters: \textsc{path} (cislocative, translocative, circumambulative etc),  \textsc{temporal relation} (prior, concurrent, subsequent motion) and moving argument (S/A vs O). 

In languages with simpler AM systems comprising only two morphemes as in Manchu, there is little variety regarding these parameters: only cislocative vs translocative AM markers of prior motion of the (S/A) subject are attested.

\subsection{Associated motion vs motion verb construction}
As mentioned above, few previous publications have addressed the issue of the contrast between AM and MVC. This topic is easier to study in languages with simpler AM systems, in the case when there is a one-to-one correspondence between AM and MVC.

A language like Manchu with only two AM morphemes where this question has been studied is Japhug (Rgyalrongic, Trans-Himalayan). In Japhug, the semantic difference between AM and MVC is most obvious in the aorist (\citealt[203]{jacques13harmonization}). In the case of the MVC, a motion verb in the aorist implies that the motion has been completed, but does not specify whether the action referred to in the purposive complement clause has taken place or not. In  this case, it is possible to negate the complement, as in examples (\ref{ex:mvc.jpg}) and  (\ref{ex:mvc.jpg2}).

 \begin{exe}
\ex \label{ex:mvc.jpg}
\gll  \ipa{laχtɕʰa} \ipa{ɯ-kɯ-χtɯ} \ipa{jɤ-ari-a}  \ipa{ri} \ipa{tɤ-χtɯ-t-a} \ipa{maka} \ipa{me} \\
thing \textsc{3sg}-\textsc{nmlz}:S/A-buy \textsc{aor}-go[II]-\textsc{1sg} but \textsc{aor}-buy-\textsc{pst-1sg} at.all \textsc{n.pst}:not.have\\
\glt  I went to buy things \textbf{but did not buy anything}.
\ex \label{ex:mvc.jpg2}
\gll  \ipa{kɯ-nɯ-rŋgɯ}  	\ipa{jɤ-ari-a}  	\ipa{ri}  	\ipa{kɤ-nɯ-rŋgɯ}  	\ipa{mɯ-pɯ-ŋgrɯ}  	\\  
\textsc{nmlz}:S/A-\textsc{auto}-lie.down 	\textsc{aor}-go[II]-\textsc{1sg} but	\textsc{inf}--\textsc{auto}-lie.down 	\textsc{neg-pst.ipfv}-succeed \\
\glt  I went to sleep \textbf{but could not sleep}.
  \end{exe}
  
On the other hand, the using the AM prefixes imply that \textbf{both} the motion and the subsequent action have been completed, and it is therefore nonsensical to negated the verb, as in examples (\ref{ex:am.jpg}) and (\ref{ex:am.jpg2})

 \begin{exe}
\ex \label{ex:am.jpg}
\gll  *\ipa{laχtɕʰa} \ipa{ɕ-tɤ-χtɯ-t-a}  \ipa{ri} \ipa{tɤ-χtɯ-t-a} \ipa{maka} \ipa{me} \\
thing \textsc{transloc}-\textsc{aor}-buy-\textsc{pst-1sg} but \textsc{aor}-buy-\textsc{pst-1sg}  at.all \textsc{n.pst}:not.have\\
\glt  (intended meaning: sentence (\ref{ex:mvc.jpg}) 
\ex \label{ex:am.jpg2}
\gll  *\ipa{ɕ-pɯ-nɯ-rŋgɯ-a}  	\ipa{ri}  	\ipa{kɤ-nɯ-rŋgɯ}  	\ipa{mɯ-pɯ-ŋgrɯ}  \\
\textsc{transloc-aor-auto}-lie.down-\textsc{1sg} but	\textsc{inf-auto}-lie.down 	\textsc{neg-pst.ipfv}-succeed \\
\glt  (intended meaning: sentence (\ref{ex:mvc.jpg2}) 
  \end{exe}  
  
  We will show in section \ref{sec:manchu} that a similar contrast is observed in Manchu, suggesting that this semantic difference may be crosslinguistically widespread.
  
\subsection{AM and causative}
All AM systems described up to now display accusative alignment, and only a handful of languages have object AM markers. 

However, in the case of causative constructions, the motion argument can at least in some languages be either the causee or the causer. In a languages such as Japhug with highly templatic morphology, the semantic scope of the causative is ambiguous (\citealt[182]{jacques15causative}). the causative can either only apply to the action expressed by the verb root (as in  \ref{ex:am.jpg.caus}, `go and cause to X')  or apply to the motion (as in \ref{ex:am.jpg.caus2}, `cause to go and X').


  \begin{exe}
\ex \label{ex:am.jpg.caus}
\gll
\ipa{mpʰrɯmɯ} 	\ipa{ɕ-pɯ-sɯ-re} 	\ipa{tɕe,} 	\ipa{ɕ-tɤ-tʰe} 	\ipa{ra} \\
divination \textsc{transloc-imp-caus}-look[III] \textsc{lnk} transloc-imp-ask[III] have.to:\textsc{fact} \\
\glt  `Go and have him make a divination and ask him about it.' (divination03 7)
  \end{exe} 

  \begin{exe}
\ex \label{ex:am.jpg.caus2}
\gll
\ipa{tɕe} 	\ipa{kupa} 	\ipa{cʰu} 	\ipa{nɯra} 	\ipa{atʰi} 	\ipa{pɕoʁ} 	\ipa{nɯra,} 	\ipa{ɯ-pɕi} 	\ipa{nɯra} 	\ipa{kɯ} 	\ipa{kɯreri} 	\ipa{ɣɯ-cʰɯ-sɯ-χtɯ-nɯ} 	\ipa{ŋu.}  \\
\textsc{lnk} Chinese \textsc{loc} \textsc{dem:pl} downstream direction \textsc{dem:pl} \textsc{3sg}-outside  \textsc{dem:pl}  \textsc{erg} here \textsc{cisloc-ipfv:downstream-caus}-buy-\textsc{pl} be\textsc{:fact} \\
\glt `People from the Chinese areas, people from outside send people to come here to buy (matsutake and sell them in the areas downstream).' (hist-20-grWBgrWB 58)
  \end{exe} 
  
  In section \ref{sec:manchu}, we will explain how Manchu and Japhug differ in this regard.
  
\section{AM in Classical Manchu} \label{sec:manchu}
%MYS \citealt{shunjuu92yargiyan}
%
%NSB-1 \citealt{nowak77nisan}
%
%NSB-2 \citealt{jaxontov93nisan}
%
%Soldier = \citealt{cosmo06dzengseo}
%
%Tulishen = \citealt{shunjuu64tulishen}
%
%UG=  \citealt{nikolaeva01udihe}
%
%\citet{gabelentz32mandchou}
%\citet{haenisch61mandschu}

Associated motion in Manchu is expressed by two suffixes: \ipa{-na-} (cislocative or venitive) and \ipa{-nji-} (translocative or andative).

\section{The grammaticalization of AM} \label{sec:grammaticalization}
serial verb construction vs purposive complement

inflectional vs derivational morphology

which is the most grammaticalized?

\section{Conclusion}

\bibliographystyle{unified}
\bibliography{bibliogj}

 \end{document}
 