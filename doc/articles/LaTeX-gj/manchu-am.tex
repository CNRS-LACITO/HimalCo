\documentclass{article} 
\usepackage{fontspec}
\usepackage{natbib}
\usepackage{booktabs}
\usepackage{xltxtra} 
\usepackage{polyglossia} 
\usepackage[table]{xcolor}
\usepackage{gb4e} 
%\usepackage{multicol}
\usepackage{hyperref} 
\hypersetup{bookmarks=false,bookmarksnumbered,bookmarksopenlevel=5,bookmarksdepth=5,xetex,colorlinks=true,linkcolor=blue,citecolor=blue}
\usepackage[all]{hypcap}
\usepackage{memhfixc}

 
%\setmainfont[Mapping=tex-text,Numbers=OldStyle,Ligatures=Common]{Charis SIL} 
\newfontfamily\phon[Mapping=tex-text,Ligatures=Common,Scale=MatchLowercase]{Charis SIL} 
\newcommand{\ipa}[1]{\textbf{{\phon\mbox{#1}}}} %API tjs en italique
 \newcommand{\ipab}[1]{{\phon \mbox{#1}}} %API tjs en italique
\newcommand{\grise}[1]{\cellcolor{lightgray}\textbf{#1}}
\newfontfamily\cn[Mapping=tex-text,Ligatures=Common,Scale=MatchUppercase]{SimSun}%pour le chinois
\newcommand{\zh}[1]{{\cn#1}}

 
\begin{document} 
\title{Associated motion in Manchu in typological perspective }
\author{José Andrés Alonso de la Fuente \\ Guillaume Jacques}
\maketitle 
%acknow: Daniel Ross, Sven Osterkamp, Brigitte Palendorff
\textbf{Abstract}: The present paper presents a detailed description of the Associated Motion system of Classical Manchu, on the basis of original texts from the XVII-XVIII^{th} centuries. It shows that despite superficial similarities, Classical Manchu differs in many ways from previously described simple AM systems only comprising translocative vs cislocative makers, such as that of Japhug. This paper provides a basic framework for further research on the typology of simple AM systems.

\textbf{Keywords}: Associated motion, Tungusic, Manchu, Motion verbs, Andative, Venitive, Translocative, Cislocative, Grammaticalization

\section*{Introduction}
The category of `Associated Motion' (henceforth AM), first discovered in Pama-Nyungan languages (\citealt{koch84associated.motion}, \citealt{wilkins91associated.motion}), is progressively becoming a topic of typological research in other areas, in particular South America (see \citealt{guillaume08cavinena}, \citealt{guillaume16am}), and but also Eurasia (for instance, Sino-Tibetan, \citealt{jacques13harmonization}). Following \citet[12]{guillaume16am}, AM markers can be defined as `grammatical morpheme[s] that [are] associated with the verb
and that [have] among [their] possible functions the coding of translational
motion.'

Most languages with AM can also express the association of translational motion with an action (`go to X, go and X, go after X etc')  by means of motion verb constructions with purposive construction (henceforth MVC).\footnote{Exceptions include Algonquian languages such as Ojibwe (\citealt[729-733]{valentine01grammar}), where only AM markers can be used to express meanings such as `go/come to X', and motion verbs apparently cannot take purposive complements.} This raises the question of the semantic value of the contrast between AM and MVC in languages where both constructions are possible. This important question, however, as pointed out by \citet[10]{guillaume16am}, has not yet been the focus of much research in the typological literature.

This paper deals with the system of AM in the Manchu language, and focuses on the precise semantic value of the AM suffixes, in particular their differences with the corresponding MVC and their interaction with the causative. 

This paper comprises three sections. First, we present an overview of previous studies of AM, MVC and valency-increasing morphemes in the literature. Second, we provide a detailed description of AM in Manchu on the basis of a philological investigation of authentic early Manchu texts from the XVII-XVIII centuries (excluding translated documents). Third, we investigate how the commonalities and differences in the use of AM in Manchu and other languages such as Japhug can be account for by differences in grammaticalization pathways.

\section{Associated motion}
In this section, we present three typological questions that deserve to be treated when describing any AM system: the parameters of the AM system, the semantic difference between AM markers and their MVC equivalent, and the interaction between AM and derivational morphemes such as the causative.

\subsection{Parameters}
In languages with complex AM systems such as Tacanan and Pama-Nyungan, \citet[8]{guillaume16am} identifies three main parameters parameters: \textsc{path} (cislocative, translocative, circumambulative etc),  \textsc{temporal relation} (prior, concurrent, subsequent motion) and moving argument (S/A vs O). 

In languages with simpler AM systems comprising only two morphemes as in Manchu, there is little variety regarding these parameters: only cislocative vs translocative AM markers of prior motion of the (S/A) subject are attested.

\subsection{Associated motion vs motion verb construction} \label{sec:japhug.am.mvc}
As mentioned above, few previous publications have addressed the issue of the contrast between AM and MVC. This topic is easier to study in languages with simpler AM systems, in the case when there is a one-to-one correspondence between AM and MVC.

One of the few languages where this question has been studied is Japhug (Rgyalrongic, Trans-Himalayan). In Japhug, the semantic difference between AM and MVC is most obvious in the aorist (\citealt[203]{jacques13harmonization}). In the MVC, a motion verb in the aorist implies that the motion has been completed, but does not specify whether the action referred to in the purposive complement clause has taken place or not. This can be shown by the fact that the action described by the purposive complement can be negated, as in examples (\ref{ex:mvc.jpg}) and  (\ref{ex:mvc.jpg2}).

 \begin{exe}
\ex \label{ex:mvc.jpg}
\gll  \ipa{laχtɕʰa} \ipa{ɯ-kɯ-χtɯ} \ipa{jɤ-ari-a}  \ipa{ri} \ipa{tɤ-χtɯ-t-a} \ipa{maka} \ipa{me} \\
thing \textsc{3sg}-\textsc{nmlz}:S/A-buy \textsc{aor}-go[II]-\textsc{1sg} but \textsc{aor}-buy-\textsc{pst-1sg} at.all \textsc{n.pst}:not.have\\
\glt  I went to buy things \textbf{but did not buy anything}.
\ex \label{ex:mvc.jpg2}
\gll  \ipa{kɯ-nɯ-rŋgɯ}  	\ipa{jɤ-ari-a}  	\ipa{ri}  	\ipa{kɤ-nɯ-rŋgɯ}  	\ipa{mɯ-pɯ-ŋgrɯ}  	\\  
\textsc{nmlz}:S/A-\textsc{auto}-lie.down 	\textsc{aor}-go[II]-\textsc{1sg} but	\textsc{inf}--\textsc{auto}-lie.down 	\textsc{neg-pst.ipfv}-succeed \\
\glt  I went to sleep \textbf{but could not sleep}.
  \end{exe}
  
On the other hand, AM prefixes imply that \textbf{both} the motion and the subsequent action have both been completed, and it is therefore nonsensical to negate the action described by the verb, as in examples (\ref{ex:am.jpg}) and (\ref{ex:am.jpg2}).

 \begin{exe}
\ex \label{ex:am.jpg}
\gll  *\ipa{laχtɕʰa} \ipa{ɕ-tɤ-χtɯ-t-a}  \ipa{ri} \ipa{tɤ-χtɯ-t-a} \ipa{maka} \ipa{me} \\
thing \textsc{transloc}-\textsc{aor}-buy-\textsc{pst-1sg} but \textsc{aor}-buy-\textsc{pst-1sg}  at.all \textsc{n.pst}:not.have\\
\glt  (intended meaning: sentence (\ref{ex:mvc.jpg}) 
\ex \label{ex:am.jpg2}
\gll  *\ipa{ɕ-pɯ-nɯ-rŋgɯ-a}  	\ipa{ri}  	\ipa{kɤ-nɯ-rŋgɯ}  	\ipa{mɯ-pɯ-ŋgrɯ}  \\
\textsc{transloc-aor-auto}-lie.down-\textsc{1sg} but	\textsc{inf-auto}-lie.down 	\textsc{neg-pst.ipfv}-succeed \\
\glt  (intended meaning: sentence (\ref{ex:mvc.jpg2}) 
\end{exe}  

This finding, tested on the basis of elicited examples, is consistent with natural data from narratives and conversations (see \citealt{jacques13harmonization} for a detailed account of both constructions).
  
We will show in section \ref{sec:manchu} that a similar contrast is observed in Manchu, suggesting that this semantic difference may be crosslinguistically widespread.

  
\subsection{AM and causative} \label{sec:japhug.am.caus}
All AM systems described up to now display accusative alignment, and only a handful of languages have object AM markers. 

However, in the case of causative constructions, the motion argument can at least in some languages be either the causee or the causer. In a languages such as Japhug with highly templatic morphology, the semantic scope of the causative is ambiguous (\citealt[182]{jacques15causative}). the causative can either only apply to the action expressed by the verb root (as in  \ref{ex:am.jpg.caus}, `go and cause to X')  or apply to the motion (as in \ref{ex:am.jpg.caus2}, `cause to go and X').


  \begin{exe}
\ex \label{ex:am.jpg.caus}
\gll
\ipa{mpʰrɯmɯ} 	\ipa{ɕ-pɯ-sɯ-re} 	\ipa{tɕe,} 	\ipa{ɕ-tɤ-tʰe} 	\ipa{ra} \\
divination \textsc{transloc-imp-caus}-look[III] \textsc{lnk} transloc-imp-ask[III] have.to:\textsc{fact} \\
\glt  `Go and have him make a divination and ask him about it.' (divination03 7)
  \end{exe} 

  \begin{exe}
\ex \label{ex:am.jpg.caus2}
\gll
\ipa{tɕe} 	\ipa{kupa} 	\ipa{cʰu} 	\ipa{nɯra} 	\ipa{atʰi} 	\ipa{pɕoʁ} 	\ipa{nɯra,} 	\ipa{ɯ-pɕi} 	\ipa{nɯra} 	\ipa{kɯ} 	\ipa{kɯreri} 	\ipa{ɣɯ-cʰɯ-sɯ-χtɯ-nɯ} 	\ipa{ŋu.}  \\
\textsc{lnk} Chinese \textsc{loc} \textsc{dem:pl} downstream direction \textsc{dem:pl} \textsc{3sg}-outside  \textsc{dem:pl}  \textsc{erg} here \textsc{cisloc-ipfv:downstream-caus}-buy-\textsc{pl} be\textsc{:fact} \\
\glt `People from the Chinese areas, people from outside send people to come here to buy (matsutake and sell them in the areas downstream).' (hist-20-grWBgrWB 58)
  \end{exe} 
  
  In section \ref{sec:manchu.caus}, we will explain how Manchu and Japhug differ in this regard.
  
\section{AM in Classical Manchu} \label{sec:manchu}
%UG=  \citealt{nikolaeva01udihe}

Associated motion in Manchu is expressed by two suffixes: \ipa{-nji-} (cislocative or venitive) and \ipa{-na-/-ne-/-no} (translocative or andative). These suffixes are generally considered to be grammaticalized from the corresponding motion verb constructions \ipa{-me gene-} and \ipa{-me ji-} with the verb of the purposive complement marked with the imperfective converb suffix \ipa{-me} (a more detailed account of the origin of these suffixes is provided in section \ref{sec:grammaticalization}). Examples (\ref{ex:go.to}) and (\ref{ex:come.to}) illustrate both associate motion and their corresponding motion verb construction.

\begin{exe}
\ex \label{ex:go.to}
\begin{xlist}
\exi{(a)} \label{ex:alanambi}
\gll \ipa{ala-na-mbi} \\
tell-\textsc{transloc-pre} \\
\exi{(b)} \label{ex:alame.genembi}
\gll \ipa{ala-me gene-mbi} \\
tell-\textsc{ipfv.conv} go-\textsc{pre} \\
\glt `go to tell'
\end{xlist}
\ex \label{ex:come.to}
\begin{xlist}
\exi{(a)} \label{ex:alanjimbi}
\gll \ipa{ala-nji-mbi} \\
tell-\textsc{cisloc-pre} \\
\exi{(b)}  \label{ex:alame.jimbi}
\gll \ipa{ala-me ji-mbi} \\
tell-\textsc{ipfv.conv} come-\textsc{pre} \\
\glt `come to tell'
\end{xlist}
\end{exe}


Cislocative and translocative are usually described alongside the so-called missive \ipa{-nggi-} ‘to send to V’ (from \ipa{unggi-} ‘to send, dispatch’), e.g. \ipa{ala-nggi-} ‘to send to say’, etc. We defer a detailed discussion of this suffix to future research, as it is far less frequent than the cislocative and translocative.\footnote{It is likely that the same conclusions drawn in this paper apply to the missive too (i.e., \ipa{ala-nggi-} and \ipa{ala-me unggi-} are not always synonymous).}

It is traditionally held that predicates with associated motion suffixes and motion verb constructions are synonymous, as suggested by the translations of examples (\ref{ex:go.to}) and (\ref{ex:come.to}). Such a characterization can be found in both grammars and handbooks, from the pioneering work of \citet[163-165 §§133-134]{zaxarov10manchu} to more recent works such as \citet[34-35]{pashkov63manchu}, \citet[173-174]{avrorin00manchu},  \citet[19]{li00manchu} or \citet[239-240]{gorelova02manchu}.\footnote{Chinese and Japanese grammars agree on this description as well as early European grammars. On a curious note, \citet[51]{harlez84mandchou} describes \ipa{-na-/-ne-/-no} as inchoative (“Inchoatif”), e.g. \ipa{taci-} ‘apprendre’ $\rightarrow$ \ipa{taci-ne-} ‘aller apprendre’, and \ipa{-nji-} as illative (“Illatif”), e.g. ara-ji-me ‘venir écrire’. This seems to be De Harlez’s interpretation, based on Gabelentz’s data and examples (\citeyear[51]{gabelentz32mandchou}, it is not entirely clear why Gabelentz, and after him De Harlez, has \ipa{-ji-} instead of \ipa{-nji-}) and followed by later grammarians (e.g., \citet[367]{peeters40manjurische}). \citet[53]{haenisch61mandschu}, however, calls \ipa{-na-/-ne-/-no} “Illativ” and \ipa{-nji-} “Allativ” (cf.\citet[401b]{li00manchu} \ipa{-na-/-ne-/-no} “allative verbal suffic”, but \ipa{-nji-} is just called “verbal suffix” on p. 402a).}

Textual analysis\footnote{Our corpus includes three narrative texts, two of them from the 17th century or earlier (\textit{Manju-i yargiyan kooli}, \citealt{shunjuu92yargiyan} and \textit{Beye-i cooha bade yabuha babe ejehe bithe}, \citealt{cosmo06dzengseo}), the other text is of oral origin and later date, though it is believed that it continues a long tradition (\ipa{Nišan samani bithe}, \citealt{nowak77nisan} and \citealt{jaxontov93nisan}). We also include one text from the beginning of the 18th century (\ipa{Lakcaha jecen de takûraha babe ejehe bithe}, \citealt{shunjuu64tulishen}). These texts are not translations, but original native Manchu literary compositions. Modern editions allow us to confront the \textit{communis opinio} regarding the translation of associated motion formations with the idea suggested in the present contribution. For \citet{shunjuu92yargiyan} we provide page of the Manchu text in brackets and lines, for the remaining texts “page with Manchu text [Manchu manuscript page] / English, Japanese or Russian translation”.} shows that, generally speaking, motion verb constructions occur less frequently than associated motion formations. For instance, in \citet{cosmo06dzengseo} motion verb constructions appear 20 times (\ipa{-me gene-} 5x, \ipa{-me ji-} 15x), whereas associated motion suffixes appear 56 times (\ipa{-na-/-ne-/-no} 27x, \ipa{-nji-} 29x), a proportion comparable to what has been observed in Japhug narratives (\citealt[209]{jacques13harmonization}).\footnote{In other texts, such as \citealt{jaxontov93nisan} however, we find a almost equal number occurrence of the two constructions:  motion verb constructions appear 52 times (\ipa{-me gene-} 33x, \ipa{-me ji-} 29x) and associated motion suffixes 50 times (\ipa{-na-/-ne-/-no} 26x, \ipa{-nji-} 24x).}



\subsection{Difference between AM and MCV} \label{sec:mvc.manchu}
As we saw in section \ref{sec:japhug.am.mvc}, an easily detectable difference between AM and MVC is observed in languages such as Japhug. In past perfective forms of the MVC ‘X went to do’, the motional event (‘go’) is understood as completed, but the completion of the main event (‘do’) is left unspecified. The corresponding form with AM markers however, implies that both motional and main events are completed (‘X went and did’). 


There is some evidence that this semantic difference is also found in Manchu. While elicitation is not possible since Classical Manchu is an extinct language,\footnote{The only surviving descendant of seventeenth century Manchu, Sibe, also has AM, but this would require a separate paper. Moreover, \citet[178]{zikmundova13sibe} argues that, in reference to -\ipa{na-}, \ipa{-nji-} and several other derivational suffixes, they `[...] are fixed in particular verbal forms and are not recognized by native speakers with the exception of those educated in Manchu grammar' suggesting that it may be difficult to elicit minimal pairs as in languages such as Japhug, where AM prefixes are not lexicalized. } its extensive text corpus makes it possible to some extent to ascertain fine semantic differences between common constructions.

In the following, we present a series of examples of verbs with AM in past forms (with either the past \ipa{-ha/-he/-ho} or the perfective converb \ipa{-fi}) showing that both the motion event and the action of the main verb have taken place. In (\ref{ex:maktanaha}) the motion event is backgrounded and that the translator did not include it in their renditions of the passage.

\begin{exe}
\ex  \label{ex:maktanaha}
\gll
\ipa{se-he} 	\ipa{manggi} 	\ipa{holkonde} 	\ipa{emu} 	\ipa{amba} 	\ipa{daimin} 	\ipa{wasinji-fi} 	\ipa{nisan} 	\ipa{saman} 	\ipa{i} 	\ipa{eigen} 	\ipa{be} 	\ipa{/} 	\ipa{šoforo-me} 	\ipa{fengdu} 	\ipa{hecen} 	\ipa{de} 	\ipa{makta-na-ha} 	\ipa{yargiyan} 	\ipa{i} 	\ipa{tumen} 	\ipa{jalan} 	\ipa{de} 	\ipa{niyalma-i} 	\ipa{/} 	\ipa{beye} 	\ipa{banji-bu-rakû} 	\ipa{o-bu-ha.} \\
say-\textsc{pst} after in.an.instant one big eagle descend-\textsc{pfv.conv} Nisan shaman \textsc{gen} husband \textsc{acc} { } seize.in.the.claws-\textsc{ipfv.conv} Fengtu walled.city \textsc{loc} throw-\textsc{cisloc-pst} real \textsc{gen} 10.000 generation \textsc{loc} man-\textsc{gen} { } body born-\textsc{caus/pass-neg} \textsc{aux-caus-pst} \\
\glt `After having said (this), one big eagle descended, seized in his claws the husband of Nishan shaman and \textbf{threw} (him): “You will not be able to reincarnate in a human body for the next good 10000 years!”. (\citealt[85,20b/119]{jaxontov93nisan})
\end{exe}

In examples such as (\ref{ex:tuwanafi}), the context makes it clear that both motion event and main action have been completed, even if the translation does not make it explicit.

\begin{exe}
\ex \label{ex:tuwanafi}
\gll
\ipa{orin} 	\ipa{juwe} 	\ipa{de} 	\ipa{ki} 	\ipa{hiyan} 	\ipa{be} 	\ipa{dule-me} 	\ipa{.} 	\ipa{jakûnju} 	\ipa{sunja} 	\ipa{ba} 	\ipa{yabu-fi} 	\ipa{.} 	\ipa{jeo} 	\ipa{gurun-i} 	\ipa{cenghiyang} 	\ipa{big’an-i} 	\ipa{eifu} 	\ipa{be} 	\ipa{tuwa-na-fi} 	\ipa{.} 	\ipa{miyoo-i} 	\ipa{boigon-i} 	\ipa{oren} 	\ipa{de} 	\ipa{hengkile-fi} 	\ipa{.} 	\ipa{ing} 	\ipa{ili-ha} \\
twenty two \textsc{dat} Qi county \textsc{acc} pass-\textsc{ipfv.conv} { } eighty five li  go-\textsc{pfv.conv} { } Zhou country-\textsc{gen} chengxiang bigan-\textsc{gen} grave \textsc{acc} see-\textsc{transloc-ipfv.conv} { } temple-\textsc{gen} household earthen.statue \textsc{dat} kneel-\textsc{pfv.conv} { } camp stand-\textsc{pst} \\
\glt `On the twenty-second we passed Qi county, covering 85 li, then went to see the tomb of Chengxiang Bigan of the kingdom of Zhou, knelt in front of the earthen statue of the temple, and set up our camp.' (\citealt[83/103]{cosmo06dzengseo})
\end{exe}

By contrast, the MVC in perfective form as in (\ref{ex:efuleme}) is used when only the motion event has been completed (it is obvious from the context in this example that the soldiers were not successful).

\begin{exe}
\ex \label{ex:efuleme}
\gll
 \ipa{boo} 	\ipa{be} 	\ipa{efule-me} 	\ipa{gene-fi} 	\ipa{.} 	\ipa{hoton-i} 	\ipa{hûlha} 	\ipa{de} 	\ipa{ilan} 	\ipa{tanggû} 	\ipa{funce-me} 	\ipa{meite-bu-he} \\
 house \textsc{acc} destroy-\textsc{ipfv.conv} go-\textsc{pfv.conv} { } city-\textsc{gen} rebel \textsc{dat} three hundred be.over-\textsc{ipfv.conv} cut.off-\textsc{pass-pst} \\
 \glt ‘When they went to pull down the houses [to get the timber], over 300 of them were cut off [and captured] by the rebels inside the city’ (\citealt[69/97]{cosmo06dzengseo})
\end{exe}

The semantic contrast between AM and MVC is best illustrated by the following minimal pair with the causative verb \ipa{ili-bu-} `build' (`cause to stand') in cislocative AM form (\ref{ex:ilibunjiha}) vs MVC (\ref{ex:ilibume}). In (\ref{ex:ilibunjiha}), the focus is on the completion on the lexical verb, while the motion event is secondary and not even taken into account in the translation as in the above examples, while in (\ref{ex:ilibume}) it is clear that only the motion event has taken place, and that the action described by the purposive complement has not even started; this contrast is identical to the one described in Japhug.

\begin{exe}
\ex \label{ex:ilibunjiha}
\gll
\ipa{tere-ci} 	\ipa{wan} 	\ipa{lii} 	\ipa{han} 	\ipa{ba-be} 	\ipa{duri-me} 	\ipa{jase-i} 	\ipa{tule} 	\ipa{ududu} 	\ipa{ba-de} 	\ipa{wehe-i} 	\ipa{bithe} 	\ipa{ili-bu-nji-ha} \\
that-\textsc{abl} Wan Li khan place-\textsc{acc} seize-\textsc{ipfv.conv} frontier-\textsc{gen} outside several place-\textsc{loc} stone-\textsc{gen} book stand-\textsc{caus-cisloc-pst} \\
\glt ‘After Wan Li khan’s place was seized, (they) established frontier lines and \textbf{made inscriptions be erected} on several places.’ (\citealt[129-130;46]{shunjuu92yargiyan})
\end{exe}

\begin{exe}
\ex \label{ex:ilibume}
\gll
\ipa{ba} 	\ipa{ambula} 	\ipa{ehe} 	\ipa{.} 	\ipa{afa-ci} 	\ipa{ojo-rakû} 	\ipa{o-fi} 	\ipa{.} 	\ipa{uthai} 	\ipa{cooha} 	\ipa{be} 	\ipa{goci-ka} 	\ipa{:} 	\ipa{ging} 	\ipa{hecen} 	\ipa{ci} 	\ipa{giyamun} 	\ipa{ili-bu-me} 	\ipa{ji-he} 	\ipa{icihiyara} 	\ipa{hafan} 	\ipa{walda} 	\ipa{se} 	\ipa{isi-nji-fi} 	\ipa{.} 	\ipa{dergi-ci} 	\ipa{.} 	\ipa{hafan} 	\ipa{cooha-i} 	\ipa{sain} 	\ipa{be} 	\ipa{fonji-ha} \\
place great evil { } attack-\textsc{cond} \textsc{aux-neg} become-\textsc{pfv.conv} { } therefore soldier \textsc{acc} withdraw-\textsc{pst} { } capital walled.city \textsc{abl} relay.station stand-\textsc{caus-ipfv.conv} come-\textsc{pst} departmental director Walda others reach-\textsc{cisloc-pfv.conv} { } emperor-\textsc{abl} { } official soldier-\textsc{gen} well-being \textsc{acc} ask-\textsc{pst} \\
\glt `That place was a really wicked one, and could not be attacked. Therefore [the general] withdrew the army. Departmental Director Walda and others, who \textbf{had come from the capital to build} relay stations, arrived with an imperial dispatch, in which the emperor inquired about the well-being of officers and soldiers.' (\citealt[94/62]{cosmo06dzengseo})
\end{exe}

Table \ref{tab:counts.dzengsheo} presents text counts of AM and MVC in perfective form (with either the past \ipa{-ha/-he/-ho} or the perfective converb \ipa{-fi}) in \citet{cosmo06dzengseo}, indicating whether only the motion event is completed, or whether both motion event and action of the main verb are completed (the data is provided in the supplementary files). This table shows that the semantic contrast observed between (\ref{ex:ilibunjiha}) and (\ref{ex:ilibume}) is not anecdotal: MVC, rather than AM,  is consistently used in perfective forms when only the motion event event is completed, while AM is more commonly used when both events are completed.


\begin{table}[h]
\caption{Number of attestations of AM and MVC in perfective form in \citet{cosmo06dzengseo} } \centering \label{tab:counts.dzengsheo}
\begin{tabular}{llllllllll}
\toprule
 & 	\ipa{-me gene-} & 	\ipa{-me ji-} & 	\ipa{-na/-ne/-no} & 	\ipa{-nji} & 	\\
 \midrule
both events completed & 	0 & 	1 & 	2 & 	2 & 	\\
ambiguous & 	0 & 	2 & 	1 & 	0 & 	\\
only motion event completed & 	1 & 	4 & 	0 & 	0 & 	\\
no motion event & 	0 & 	0 & 	3 & 	1 & 	\\
motion verb & 	3 & 	2 & 	19 & 	26 & 	\\
\bottomrule
\end{tabular}
\end{table}


Previous scholarship on AM and MCV in Manchu (\citealt{hayata95yuku}, \citealt{kubo97come}) focuses on their motional component (physical orientation, speaker’s and listener’s relative position, etc.) and the semantic difference between the two construction has only been discussed by Hayata, who claimed that AM forms mean ‘to go/come in order to V’ (\zh{……に行く} and \zh{……に来る}), while MVC mean 'do X and go/come' (\zh{……て行く} / \zh{……て来る}), i.e. almost the exact opposite of the contrast proposed in this paper. The data in our corpus, in particular the text counts in Table \ref{tab:counts.dzengsheo}, flatly contradict Hayata's proposal.

While these data show the existence of a semantic contrast between AM and MVC in Manchu similar to that found in Japhug, this contrast is a strong tendency rather than an absolute principle. It is possible to find in Manchu (unlike in Japhug), examples of AM verbs in past form where only the motion event is completed can be found in the complete corpus (not restricted to \citet{cosmo06dzengseo}). For instance,  in (\ref{ex:huulhanjiha}), both the AM \ipa{hûlha-nji-ha} and the corresponding MVC \ipa{hûlha-me} \ipa{ji-he} are synonymous and clearly mean `came to steal (the bull)' rather than `came and stole (the bull)' (the thief was caught before he could steal the animal).

\begin{exe}
\ex \label{ex:huulhanjiha}
\gll
\ipa{taidzu} 	\ipa{sure} 	\ipa{beile} 	\ipa{jortai} 	\ipa{hendu-me} 	\ipa{ere} 	\ipa{hûlha} 	\ipa{ainci} 	\ipa{mini} 	\ipa{ihan} 	\ipa{hûlha-me} 	\ipa{ji-hebi} 	\ipa{=dere} 	\ipa{seme} 	\ipa{hendu-he} 	\ipa{manggi,} 	\ipa{tere} 	\ipa{hûlha} 	\ipa{jabu-me,} 	\ipa{bi} 	\ipa{ihan} 	\ipa{hûlha-nji-ha} 	\ipa{mujangga} 	\ipa{se-re} 	\ipa{jakade,} 	\ipa{loohan} 	\ipa{hendu-me,} 	\ipa{ere} 	\ipa{hûlha} 	\ipa{holto-me} 	\ipa{hendu-mbi,} 	\ipa{cohome} 	\ipa{simbe} 	\ipa{wa-me} 	\ipa{ji-hebi=kai,} 	\ipa{ere} 	\ipa{be} 	\ipa{wa-ki} 	\ipa{=dere} 	\ipa{se-ci,} 	\ipa{taidzu} 	\ipa{sure} 	\ipa{beile} 	\ipa{oji-rakû} 	\ipa{hendu-me,} 	\ipa{ihan} 	\ipa{hûlha-me} 	\ipa{ji-he} 	\ipa{mujangga} 	\ipa{oci} 	\ipa{sinda-fi} 	\ipa{unggi} 	\ipa{seme,} 	\ipa{tere} 	\ipa{hûlha} 	\ipa{be} 	\ipa{sinda-fi} 	\ipa{unggi-he.} \\
Taidzu wise ruler pretending say-\textsc{ipfv.conv} this thief perhaps \textsc{1sg:gen} bull steal-\textsc{ipfv.conv} come-\textsc{pst.fin} probably \textsc{quot} say-\textsc{pst} after that thief say-\textsc{ipfv.conv} me bull steal-\textsc{cisloc-pst} truly say-\textsc{aor} after Luohan answer-\textsc{ipfv.conv} this thief lie-\textsc{ipfv.conv} say-\textsc{ipfv.conv} especially you.\textsc{acc} kill-\textsc{ipfv.conv} come-\textsc{pst.fin}=\textsc{emph} this \textsc{acc} kill-\textsc{opt} likely say-\textsc{cond} Taidzu wise ruler \textsc{aux-neg} said-\textsc{ipfv.conv} bull steal-\textsc{ipfv.conv} come-\textsc{pst} truly if let.go-\textsc{pfv.conv} dispatch:\textsc{imp} \textsc{quot} that rebel \textsc{acc} let.go-\textsc{pfv.conv} dispatch-\textsc{pst} \\
\glt  `The Taidzu wise ruler said in pretence: “It is most likely that this thief has come to steal my bull”. The thief answered: “I’ve really come to steal the bull”. Then Luohan said: “this thief is lying. In fact he has specially come in order to kill you, let us kill him!’. The Taidzu wise ruler refused and said “if you have really come to steal the bull, I will set you free”, (and) he set the thief free.' (\citealt[35;171-172]{shunjuu92yargiyan})
\end{exe}


It can thus be concluded that the semantic difference between AM and MVC observed in Japhug in section \ref{sec:japhug.am.mvc} is not absolutely valid for Manchu, and that an examination of this issue in other languages with AM might reveal examples similar to (\ref{ex:huulhanjiha}).

\subsection{Combination of AM and MVC}
Unlike many languages with AM such a Japhug, it is possible to combine both AM and MVC in the same construction, as in (\ref{ex:belheneme}), (\ref{ex:acaname.geneki}) and (\ref{ex:alaname.gene}). Such examples are rare (and have never been mentioned in previous scholarship on motion verbs in Manchu). 

With a past tense suffix as in (\ref{ex:belheneme}), it can be used in contexts where only the motion event is completed, like simple MVC. 

It is premature at this stage to speculate on the precise semantics of this construction, though it can be surmised that it is a way to express emphasis on the motion event.

\begin{exe}
\ex \label{ex:belheneme}
\gll 
\ipa{ahalji} 	\ipa{se} 	\ipa{je} 	\ipa{se-fi} 	\ipa{jabu-mbi-me} 	{\ipa{teisu} \ipa{teisu}} 	\ipa{belhe-ne-me} 	\ipa{gene-he} \\
Ahalji \textsc{pl} yes.sir say-\textsc{pfv.conv} answer-\textsc{pre-ipfv.conv} one.by.one prepare-\textsc{transloc-ipfv.conv} go-\textsc{pst} \\
\glt ‘Ahalji and the herdsmen [lit. ‘and the others’] responded “Yes, sir!” and each went to make preparations.’ (\citealt[14/45]{nowak77nisan})
\end{exe} 

%\begin{exe}
%\ex 
%\gll 
%\ipa{geren} 	\ipa{gemu} 	\ipa{je} 	\ipa{se-me} 	\ipa{jabu-fi.} 	\ipa{meni} 	\ipa{meni} 	\ipa{fakca-me} 	\ipa{belhe-ne-me} 	\ipa{gene-he} \\
%everyone all yes.sir say-\textsc{ipfv.conv} answer--\textsc{pfv.conv} each each leave-\textsc{ipfv.conv} prepare-\textsc{transloc-ipfv.conv} go-\textsc{pst} \\
%‘Al1 answered “Yes sir!” and each left to make preparations.’ (\citealt[15/46]{nowak77nisan})
%\end{exe} 


\begin{exe}
\ex \label{ex:acaname.geneki}
\gll
\ipa{genggiyen} 	\ipa{han} 	\ipa{ini} 	\ipa{beye} 	\ipa{aca-na-me} 	\ipa{gene-ki} \\
bright han \textsc{3sg:gen} body meet-\textsc{transloc-ipfv.conv} go-\textsc{opt} \\
\glt ‘(the Taidzu) bright han himself wished to go to meet (him).’ (\citealt[243, 121]{shunjuu92yargiyan})
\end{exe}

\begin{exe}
\ex \label{ex:alaname.gene}
\gll
\ipa{`meni} 	\ipa{cooha} 	\ipa{suweni} 	\ipa{jaisai} 	\ipa{cooha} 	\ipa{be} 	\ipa{gida-fi} 	\ipa{jaisai} 	\ipa{beye} 	\ipa{uheri} 	\ipa{ninggun} 	\ipa{beile,} 	\ipa{emu} 	\ipa{tanggû} 	\ipa{susai} 	\ipa{funceme} 	\ipa{niyalma} 	\ipa{be} 	\ipa{weihun} 	\ipa{jafa-ha} 	\ipa{medege} 	\ipa{be} 	\ipa{ala-na-me} 	\ipa{gene'} 	\ipa{seme} 	\ipa{sinda-fi} 	\ipa{unggi-fi,} 	\ipa{tere-ci} 	\ipa{amba} 	\ipa{cooha} 	\ipa{bedere-me} 	\ipa{ji-he.} \\
our army your Zhaisai army \textsc{acc} crush-\textsc{pfv.conv} Zhaisai himself altogether six ruler one 100 fifty over man \textsc{acc} alive seize news \textsc{acc} report-\textsc{transloc-ipfv.conv} go:\textsc{IMP} \textsc{quot} set.out-\textsc{pfv.conv} dispatch-\textsc{pfv.conv} that-\textsc{abl} big army withdraw-\textsc{ipfv.conv} come-\textsc{pst} \\
\glt ‘(He) said: “Our army has crushed yours and Zhaisai’s army. Zhaisai himself along with six rulers and over 150 men have been captured alive, go to spread the news!”. (They) liberated (the man), and after that the imperial army withdrawn.’ (\citealt[227, 20-21]{shunjuu92yargiyan})
\end{exe}

\subsection{Non-AM meanings}
As pointed out by works on the typology of AM such as \citet[3]{guillaume16am}, it is common for AM to have additional meanings not involving a motion event distinct from that of the main verb. Manchu is such a language: the suffixes \ipa{-na/-ne/-no} and \ipa{-nji} do not exclusively express AM meanings. 

Some motion verbs such as \ipa{isi-} `‘to reach, arrive, approach, come up' are neutral as to the cislocative / translocative contrast, and the suffixes \ipa{-na/-ne/-no} and \ipa{-nji} are very commonly added to such verbs to specify motion towards or from the point of reference, without implying an additional motion event (see \ref{ex:isinafi}).
%\ipa{isi-na-} vs \ipa{isi-nji-}

\begin{exe}
\ex \label{ex:isinafi}
\gll
\ipa{se-he} 	\ipa{manggi.} 	\ipa{bahaljin} 	\ipa{morin} 	\ipa{/} 	\ipa{yalu-fi} 	\ipa{juwan} 	\ipa{niyalma} 	\ipa{be} 	\ipa{gai-fi} 	\ipa{juleri} 	\ipa{feksi-me} 	\ipa{umai} 	\ipa{goidahakû} 	\ipa{lolo} 	\ipa{/} 	\ipa{gašan} 	\ipa{de} 	\ipa{isi-nji-fi} 	\ipa{boo-i} 	\ipa{duka} 	\ipa{be} 	\ipa{isi-na-fi} 	\ipa{boo-de} 	\ipa{dosi-fi} 	\ipa{mafa} 	\ipa{mama} 	\ipa{de} 	\ipa{/} 	\ipa{aca-fi} \\
say-\textsc{pst} after Bahaljin horse { } ride-\textsc{pfv.conv} 10 man acc take-\textsc{pfv.conv} front gallop-\textsc{ipfv.conv} not.at.all before.long Lolo { } town \textsc{loc} reach-\textsc{cisloc-pfv.conv} house-\textsc{gen} door \textsc{acc} reach-\textsc{transloc-pfv.conv} house-\textsc{loc} enter-\textsc{pfv.conv} old.man old.woman \textsc{loc} { } meet-\textsc{pfv.conv} \\
\glt `After having said (this), Bahaljin mounted his horse, took (with him) ten men, and at a gallop arrived in no time to the entrance to Lolo town. He went to the door, entered.' (\citealt[68;3a/96]{jaxontov93nisan})
\end{exe}

In addition, we find some rare examples where  \ipa{-na/-ne/-no} and \ipa{-nji} are used in contexts where no motion event took place, and where the cislocative/ translocative orientation is more abstract. 

For instance, in (\ref{ex:boolanjiha}), context implies that the Department Magistrate wrote reports (and had them sent), not that he came to write reports or even came with the reports that he himself wrote. The cislocative \ipa{-nji} is rather used here to indicate transmission of information towards the deixis center, ie the group to which the author of the text belongs (for other examples of the use of cislocative markers to mark first person recipients,  see \citealt{jacques14inverse}).

\begin{exe}
\ex \label{ex:boolanjiha}
\gll
\ipa{ice} 	\ipa{ninggun} 	\ipa{de} 	\ipa{ganduhai} 	\ipa{buyarame} 	\ipa{janggi-sa} 	\ipa{morin-i} 	\ipa{ton} 	\ipa{be} 	\ipa{ciralame} 	\ipa{baica-ha} 	\ipa{:} 	{\ipa{bin} 	\ipa{jeo-i}} 	\ipa{jy} 	\ipa{jeo} 	\ipa{hafan} 	\ipa{ma} 	{\ipa{ciyang} 	\ipa{yen}} 	\ipa{be} 	\ipa{ubaša-ha} 	\ipa{seme} 	\ipa{nurhûme} 	\ipa{boola-nji-ha} 	\\
first six \textsc{loc} Ganduhai lower official-\textsc{pl} horse-\textsc{gen} number \textsc{acc} thoroughly inspect-\textsc{pst} { } Binzhou-\textsc{gen} district department official Ma Chengying \textsc{acc} revolt-\textsc{pst} \textsc{quot} repeatedly report-\textsc{cisloc-pst} \\
\glt `On the sixth Ganduhai and his lower officers thoroughly checked the number of the horses. The Department Magistrate of the Binzhou district reported repeatedly that Ma Chengyin had rebelled.' (\citealt[88/47]{cosmo06dzengseo})
\end{exe}

Sibe went further than Classical Manchu in this regard. As pointed out by \citet[156]{zikmundova13sibe}, MVC are partially grammaticalized in some contexts and have lost motional semantics, as show by example \ref{sec:sibe}, where the verb \ipa{ji-} `come' has become an aspectual marker.

\begin{exe}
\ex \label{sec:sibe}
\gll
\ipa{bi} 	\ipa{tǝňi} 	\ipa{bodǝ-m} 	\ipa{ji-ɣǝi} \\
\textsc{1sg} now think-\textsc{ipfv.conv} come-\textsc{pst} \\
\glt ‘I have just got it.'
\end{exe}


\subsection{Lexicalization} \label{sec:lexicalization}
Numerous verbal forms in \ipa{-na-/-ne-} and \ipa{-nji-} show lexicalization, as illustrated by the examples in Table \ref{tab:lexicalization}.

\begin{table}[h]
\caption{Lexicalization of AM verbs in Manchu} \label{tab:lexicalization} \centering
\begin{tabular}{lllll}
\toprule
Base verb & AM form  \\
\midrule
\ipa{ebu-} ‘to dismount, get off,  & \ipa{ebu-nji-} ‘to come to dismount, \\
get down’& to descend (from a deity)’\\
 \ipa{wasi-} ‘to go down’ & \ipa{wasi-nji-} ‘to come down; to descend’\\
 \ipa{hafu-} ‘to communicate’ & \ipa{hafu-nji-} ‘to come (straight) through’ \\
 & \ipa{hafu-na-} ‘to connect with’\\
 \ipa{aca-} `meet' &  \ipa{aca-na-} `go to meet; suit, fit' \\
 \bottomrule
\end{tabular}
\end{table}



\begin{exe}
\ex 
\gll \ipa{baldubayan} 	\ipa{tere} 	\ipa{hehe} 	\ipa{be} 	\ipa{cincile-me} 	\ipa{tuwa-ci} 	\ipa{orin} 	\ipa{se-i} 	\ipa{šurdeme} 	\ipa{o-hobi.} 	\ipa{banji-ha} 	\ipa{arbun}/ 	\ipa{yargiyan} 	\ipa{i} 	\ipa{pan an gung} 	\ipa{ni} 	\ipa{sargan} 	\ipa{jui} 	\ipa{jalan} 	\ipa{de} 	\ipa{ebunji-he} 	\ipa{se-ci} 	\ipa{o-mbi.} \\
{Baldu Bayan} that woman \textsc{acc} scrutinize-\textsc{ipfv.conv} look-\textsc{cond} twenty year-\textsc{gen} around \textsc{aux-pst.fin}  born-\textsc{pst} form real \textsc{gen} {Pan An gung} \textsc{gen} female son generation \textsc{loc} descend-\textsc{pst} say-\textsc{cond} \textsc{aux-aor} \\
\glt `Baldu Bayan looked scrutinizing that woman. She was around twenty years old. It could be said that, (judging by her) appearance, she was a true-born daughter of gung Pan An.' (\citealt[71,6a-6b/100]{jaxontov93nisan})
\end{exe}


%\begin{exe}
%\ex 
%\gll
%\ipa{se-he} 	\ipa{manggi} 	\ipa{enduri} 	\ipa{sargan} 	\ipa{jui} 	\ipa{wasinji-fi} 	\ipa{imcan} 	\ipa{be} 	\ipa{bira-de} 	\ipa{sinda-fi} 	\ipa{nisan} 	//	\ipa{saman} 	\ipa{be} 	\ipa{doo-bu-ha.} \\
%say-\textsc{pst} after spirit female son descend-\textsc{pfv.conv} shaman.drum \textsc{acc} river-\textsc{loc} put-\textsc{pfv.conv} Nisan { } shaman \textsc{acc} cross-\textsc{caus-pst} \\
%\glt `After having said (this), the spirit of the girl descended, put the shaman-drum in the river and help Nisan shaman to cross it.' (\citealt[80,15a-15b/111]{jaxontov93nisan})
%\end{exe}

%\begin{exe}
%\ex 
%\gll
%{\ipa{giyoo} 	\ipa{sui} 	\ipa{jeo}} 	\ipa{.} 	{\ipa{yûn} 	\ipa{nan}} 	\ipa{.} 	{\ipa{gui} 	\ipa{jeo}} 	\ipa{.} 	\ipa{sycuwan-i} 	\ipa{jugûn} 	\ipa{hafu-nji-ha} 	\ipa{oyonggo} 	\ipa{hoton} 	\ipa{seme} 	\ipa{dobori} 	\ipa{meni} 	\ipa{uju} 	\ipa{meyen-i} 	\ipa{cooha} 	\ipa{be} 	\ipa{unggi-he} 	\\
%Jiaoshui { } Yunnan { } Guizhou { } Sichuan road communicate-\textsc{pst} important city.wall because night \textsc{1sg}:\textsc{gen} head section-\textsc{gen} soldier \textsc{acc} dispatch-\textsc{pst} \\
%\glt `Because Jiaoshui is an important city on the routes to (lit. ‘that communicates straight with’) Yunnan, Guizhou, and Sichuan, [the Field Marshal] ordered my first squadron’s soldiers to go there at night.' (\citealt[95/65]{cosmo06dzengseo})
%\end{exe}

\begin{exe}
\ex 
\gll
\ipa{juwe} 	\ipa{ilan} 	\ipa{ba-ci} 	\ipa{šeri} 	\ipa{eye-me} 	\ipa{tuci-fi} 	\ipa{ajige} 	\ipa{omo} 	\ipa{banji-na-habi} \\
two three place-\textsc{abl} spring flow-\textsc{ipfv.conv} go.out-\textsc{pfv.conv} small pond be.born-\textsc{transloc-pst.fin} \\
‘from two or three places springs flowed and a small pond was formed.’ (\citealt[68;14b]{shunjuu64tulishen})
\end{exe}

Some verbs with an AM marker such as \ipa{acana-} `suit' (example \ref{ex:acanarakuu}) have lost any trace of motional meaning.

\begin{exe}
\ex \label{ex:acanarakuu}
\gll
\ipa{niyalma} 	\ipa{takûra-fi} 	\ipa{bene-bu-fi,} 	\ipa{meni} 	\ipa{gurun} 	\ipa{i} 	\ipa{doro} 	\ipa{de } 	\ipa{acana-rakû} \\
man delegate-\textsc{pfv.conv} send-\textsc{caus}-\textsc{pfv.conv} \textsc{1pl:gen} country \textsc{gen} rite \textsc{dat} suit-\textsc{neg} \\
\glt  'One person was sent in delegation, (but) he does not comply with our laws and customs' (\citealt[XXX]{shunjuu64tulishen})
\end{exe} 


These examples show that Manchu AM is a derivation, unlike AM in languages such as Japhug, where it is clearly inflectional morphology, with no cases of lexicalized AM markers.\footnote{The only potential case of lexicalization AM marker in Japhug is the verb `fetch', which is peculiar in requiring the presence of a AM prefix (\citealt[210]{jacques13harmonization}). However, we do not find any example of an AM prefix modifying the verb's meaning in an unpredictable way.}



\subsection{AM and causative} \label{sec:manchu.caus}

Unlike Japhug where causative and AM prefixes occupy fixed slots in the verbal template, in Manchu causative suffixes can either precede or follow the AM suffixes. 

In the verb \ipa{te-bu-ne-bu-} `cause to do garrison duty' we even find the causative \ipa{-bu} occurring two times in the same verb form before and after the translocative suffix \ipa{-ne}. As mentioned in the previous section, AM suffixes are derivational suffixes (like the causative), and the semantics of derived forms is not always predictable from the base form. This complex verb illustrates this phenomenon: \ipa{te-} `sit, reside' $\rightarrow$ \ipa{te-bu} `set out, plant, pour, put into, install (as an official)' $\rightarrow$ \ipa{te-bu-ne} `do garrison duty'
$\rightarrow$ \ipa{te-bu-ne-bu} `cause to do garrison duty'.

The semantic scope of the AM and the causative affixes, unlike in Japhug, depends on their relative position in the suffixal chain. When the AM suffix is closer the verb stem than the causative suffix, causation applies to both the action of the verb and the motion, as in (\ref{ex:okdonjibuha}).

\begin{exe}
\ex \label{ex:okdonjibuha}
\gll 
\ipa{susai}	\ipa{ila.ci}	\ipa{aniya}	\ipa{duin}	\ipa{biya-i}	\ipa{ice}	\ipa{sunja}	\ipa{de,}	\ipa{Turgût}	\ipa{Gurun}	\ipa{i}	\ipa{Ayuki}	\ipa{Han}	\ipa{ini}	\ipa{harangga}	\ipa{Taiji}	\ipa{Weijeng}	\ipa{se-be}	\ipa{takûra-fi,}	\ipa{Oros}	\ipa{Gurun}	\ipa{i}	\ipa{Saratofu}	\ipa{de}	\ipa{isi-tala}	\ipa{okdo-nji-bu-ha} \\
fifty third year four month-\textsc{gen} new five \textsc{loc} Turgût nation \textsc{gen} Ayuki Khan his subordinate Taiji Weijeng \textsc{pl-acc} delegate-\textsc{pfv.conv} Russian nation \textsc{gen} Saratov \textsc{loc} reach-\textsc{tr.conv} meet-\textsc{cisloc-caus-pst} \\
\glt ‘in the fifth day of the fourth month of the the 53rd year, Ayuki Khan of the Turgût nation sent his subordinate Taiji Weijeng and others in delegation. Until we arrived,
they waited on us (lit. `he sent them to meet us halfway’) in Saratov, [a city] of the Russian nation.’ (\citealt[175;88b-89a]{shunjuu64tulishen})
\end{exe}

On the other hand, when the causative is closer to the verb stem, causation only applies to the action of the verb stem, as in the verb \ipa{ili-bu-nji-} `come to erect (=cause to cause to stand)' cited in example (\ref{ex:ilibunjiha}) above.



\section{The grammaticalization of AM in Manchu and Tungusic} \label{sec:grammaticalization}
%serial verb construction vs purposive complement
%inflectional vs derivational morphology
%which is the most grammaticalized?
Since AM is mainly attested in languages families whose history is poorly attested, the study of the diachronic origins of AM is still a domain in infancy.

A crosslinguistically common source of AM markers are motion verbs such as `come' and `go' (\citealt[70;155]{heine-kuteva02}). \citet{jacques14inverse} also suggest that AM is an intermediate stage in the grammaticalization pathway from a motion verb `come' to inverse marker, as attested in several languages:

\begin{exe}
\ex\label{ex:cisloc}
\glt  \textsc{come} $\rightarrow$   \textsc{cislocative}_{\textit{associated motion}} $\rightarrow$  \textsc{cislocative} _{\textit{directional}} $\rightarrow$  \textsc{inverse} (1/2 person patient)
\end{exe}

\citet{konnerth15cisloc} argues that a grammaticalization pathway occurred in the opposite direction in Karbi, namely that the second person pronoun became an AM marker. This hypothesis, which contradicts the unidirectionality of the pathway in (\ref{ex:cisloc}), needs to be confirmed by further study on the comparative phonology and morphology of the Karbi languages and its neighbors. If confirmed, it would provide a possible origin of AM markers distinct from motion verbs.

The grammaticalization from motion verb to AM marker results from the coalescence of the motion  verb and the action verb in a construction where both occur. In some languages such as Japhug, it can be shown that the ancestor of the AM construction was not a MVC (in which the action verb is in a purposive complement), but rather a serial verb construction in which both verbs were in finite form and shared the same TAM and person markers (see \citealt{jacques13harmonization}). In other languages, including Manchu (and some cited by \citealt[70;155]{heine-kuteva02}), AM derives from the corresponding MVC.

In Manchu, the grammaticalization of \ipa{-nji} and \ipa{-na/-ne/-no} must be discussed separately.

\subsection{The cislocative \ipa{-nji} }
The cislocative \ipa{-nji} suffix is a Manchu innovation, found in Classical Manchu and modern varieties, including Sibe, but absent from other Tungusic languages and even Jurchen, though in the latter this may be due to the scarcity of documents left in that language.

All references works on Manchu agree that \ipa{X-nji} results from the coalescence of the imperfective converb \ipa{X-me} with the verb \ipa{ji-} `come'. Apocope of the vowel \ipa{-e-} widely attested in Manchu in (\citealt[18-19]{hattori56manchu}, \citet[43-44]{gorelova02manchu}, and assimilation of the *\ipa{m} to the following \ipa{j} is found for instance in \ipa{ganji} `all, completely' (compare Oroch \ipa{gaamji} `all').\footnote{The internal cluster \ipa{-mj-} does occur in Manchu, but results from loss of unstressed vowels that occurred after the sound change *\ipa{-mj-} $\rightarrow$ \ipa{-nj-}. }

%as in \ipa{gamji}, but only at morpheme boundary (here \ipa{-ji} is the \textit{nomen agentis} suffix) 

There are however fossilized AM verbs in Manchu combining \ipa{ji-} `come' with another verb root without any nasal element between them, as for instance \ipa{gaji-} `bring' from a verb root *\ipa{ga-} not attested on its own but found in the corresponding translocative form \ipa{gana-} `fetch, to go to get' and in the verbs \ipa{gama-} `take' and  \ipa{gai-} `take away'. 


\subsection{The translocative \ipa{-na/-ne/-no} }
Unlike \ipa{-nji}, the Manchu translocative \ipa{-na/-ne/-no} is a suffix of proto-Tungusic pedigree, attested in all other Tungusic languages, where it however has both cislocative and translocative functions, as shown by the following examples from Udihe:

\begin{exe}
\ex 
\gll \ipa{weisi-ne-je-fie} \\
rescue-\textsc{dir-subj-1pl.in} \\
\glt ‘Let us go to rescue (him).’ (\citealt[47]{nikolaeva01udihe})
\end{exe}

\begin{exe}
\ex 
\gll \ipa{jekpu-ne-y} \ipa{jeu}. \\
eat-\textsc{dir-imp.2sg} food \\
\glt ‘Come to eat some food!’ (\citealt[122]{nikolaeva01udihe})
\end{exe}

There is no comparable MVC in this language with a verb in imperfective converbial form; rather, coordinated finite verb forms are used, as in (\ref{ngenejefi}).

\begin{exe}
\ex \label{ngenejefi}
\gll 
\ipa{Wee-tigi} 	\ipa{ŋene-je-fi} 	\ipa{seutigi} 	\ipa{diga-na-ja-fi} \\
forest-\textsc{lat} go-\textsc{subj-1pl.in} nut eat--\textsc{dir-subj-1pl.in} \\
\glt ‘Let us go to the forest and eat some nuts.’ (\citealt[121]{nikolaeva01udihe})
\end{exe}

While the MVC construction combining the action verb in imperfective converbial form with the motion verb only exists in Manchu, there is evidence that the construction *\ipa{X-ma ŋänä-} from which Manchu \ipa{X-me gene-} `go to X' did exist in proto-Tungusic, and is indirectly reflected outside of Manchu as a future suffix *\ipa{-ŋaa} (\citealt[64++]{fuente11tungusic}).


Supposing a pathway such as \ipa{-na/-ne/-no} $\leftarrow$ \ipa{-me gene-} would thus be anachronistic. Besides, since *\ipa{-ma ŋänä-} yields \ipa{-ŋaa} in languages other than Manchu, it is not possible to suppose that the AM suffix *\ipa{-nä-} was grammaticalized from *\ipa{-ma ŋänä-} in proto-Tungusic.

One could suppose that proto-Tungusic *\ipa{-nä-} could be grammaticalized from the verb *\ipa{ŋänä-} `go', possibly through a coordinated construction rather than a MVC with purposive complement. Yet, this hypothesis is also problematic in that proto-Tungusic *\ipa{ŋ}, even in clusters, is always preserved in Northern Tungusic (\citealt[241-4]{cincius49fonetika}), and would in no way yield the reflexes of *\ipa{-nä} in every language.

An alternative approach is preferable. Rather than analyzing \ipa{-nä} as being grammaticalized from the verb *\ipa{ŋänä-} `go', we propose that the verb *\ipa{ŋänä-} `go' is a frozen translocative form containing the suffix \ipa{-nä} attached to a root \ipa{ŋä-}. The root *\ipa{ŋä-} is not otherwise attested, the fossilization of this suffix having occurred already in proto-Tungusic. In this view, the AM suffix *\ipa{-nä} has no recoverable etymology, and must be reconstructed as it is back to proto-Tungusic.

\section{Conclusion}
The AM system of Classical Manchu superficially resembles that of the Japhug language (previously described in \citealt{jacques13harmonization}) in only having a pair of markers (translocative \ipa{-na/-ne/-no} vs cislocative \ipa{-nji}). The present study revealed some commonalities, but also important differences between the two languages. 

First, Manchu differs from Japhug in the nature of the semantic contrast between AM suffixes and the corresponding motion verb constructions. In Japhug, a verb with AM marker in the perfective necessarily implies that both the motion event and the action expressed by the verb stem have been completed. In Manchu, while this constraint is generally respected, it is only a strong tendency, as counterexamples do exist (see example \ref{ex:huulhanjiha} in section \ref{sec:mvc.manchu}).

Second, while in Japhug the semantic scope of causative and AM affixes is independent of their relative position, in Manchu the relative position of the affixes has semantic scope effects (section \ref{sec:manchu.caus}). This difference can be attributed to the fact that the verb morphology of the former is mostly templatic, while that of the latter is exclusively of the layered type (on the distinction between templatic vs layered morphology, see \citealt{bickel07inflectional}).

Third, in Manchu AM suffixes are derivational morphemes, and   verbs derived with them have multiple unpredictable semantic peculiarities.

Fourth, in Japhug the two prefixes have been grammaticalized recently at the same time (in proto-Gyalrong) from a serial verb construction; in Manchu, the cislocative was grammaticalized from a motion verb construction in recent times (it is not found in any other Tungusic language), while the translocative goes back to proto-Tungusic and its ultimate origin cannot be determined.


This study thus contributes not only to Manchu synchronic grammar and Tungusic historical morphology, but also to the typology of AM systems. It shows in particular that much work is needed even in the study of `simple' AM systems comprising only two members. Despite their apparent similarities, these systems are widely different in the details, and further research on this topic would require testing the same parameters on other languages. It remains to be determine to what extent the type represented by Manchu, or that represented by Japhug, is the commonest cross-linguistically, and how the synchronic properties of these systems are relatable to their lexical source (serial verb construction, purposive motion verb construction or other).

\bibliographystyle{unified}
\bibliography{bibliogj}

 \end{document}
 