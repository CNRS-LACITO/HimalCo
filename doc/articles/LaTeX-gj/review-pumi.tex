\documentclass[oldfontcommands,oneside,a4paper,11pt]{article} 
\usepackage{fontspec}
\usepackage{natbib}
\usepackage{booktabs}
\usepackage{xltxtra} 
\usepackage{polyglossia} 
\usepackage[table]{xcolor}
\usepackage{multirow}
\usepackage{gb4e} 
\usepackage{graphicx}
\usepackage{float}
\usepackage{lscape}
\usepackage{hyperref} 
\hypersetup{bookmarks=false,bookmarksnumbered,bookmarksopenlevel=5,bookmarksdepth=5,xetex,colorlinks=true,linkcolor=blue,citecolor=blue}
\usepackage[all]{hypcap}
\usepackage{memhfixc}
 
\setmainfont[Mapping=tex-text,Numbers=OldStyle,Ligatures=Common]{Charis SIL} %ici on définit la police par défaut du texte


\newfontfamily\phon[Mapping=tex-text,Ligatures=Common,Scale=MatchLowercase,FakeSlant=0.3]{Charis SIL} 
\newfontfamily\phondroit[Mapping=tex-text,Ligatures=Common,Scale=MatchLowercase]{Charis SIL} 
\newcommand{\ipa}[1]{{\phon #1}} %API tjs en italique
\newcommand{\ipac}[1]{{\tiny #1}}
\newcommand{\ipapl}[1]{{\phondroit #1}} 
\newfontfamily\cn[Mapping=tex-text,Ligatures=Common,Scale=MatchUppercase]{MingLiU}%pour le chinois
\newcommand{\zh}[1]{{\cn #1}}
\newfontfamily\mleccha[Mapping=tex-text,Ligatures=Common,Scale=MatchLowercase]{Galatia SIL}%pour le grec
\newcommand{\grec}[1]{{\mleccha #1}}
\newcommand{\tgz}[1]{#1 \mo{#1} \tg{#1}}
\newcommand{\indextg}[1]{\index{\tge{#1}@\tgz{#1}}}
\newcommand{\tgb}[1]{\tgz{#1}\indextg{#1}}
\newcommand{\tgc}[1]{\tg{#1} \#1\indextg{#1}}
\newcommand{\tgd}[1]{\tge{#1}\indextg{#1}}
\newcommand{\tgf}[1]{\mo{#1}\indextg{#1}}
\newcommand{\petit}[1]{\tiny#1}
\newcommand{\sig}{\begin{math}\Sigma\end{math}}
\newcommand{\phone}{\begin{math}\Phi\end{math}}
\newcommand{\ra}{$\Sigma_1$} 
\newcommand{\rc}{$\Sigma_3$} 
\newcommand{\grise}[1]{\cellcolor{lightgray}\textbf{#1}}


\begin{document}
%\OnehalfSpacing
\title{Review of `A grammar of Pumi' by Picus Ding}
\author{Guillaume Jacques}
\maketitle

\sloppy

The year 2014 saw the completion of two important monographs on Pumi: the book under review (\citealt{ding14grammar}) and \citet{daudey14grammar}'s dissertation, defended at LaTrobe University. These two typologically-informed and corpus-based grammars represent a welcome improvement over previous descriptions of Pumi (\citealt{fual98pumi} and   \citealt{lusz01pumi}). While the scope of these two grammars is similar, they are by no means redundant, since they describe quite distinct varieties of Northern Pumi, and are based on different text corpora.


The present book is partially based on the author's PhD dissertation (\citealt{ding98phd} -- incidentally also defended at LaTrobe University under D. Bradley). It represents however a considerable improvement over Ding's previous work on several accounts. 

First, the corpus has been considerably enlarged, in particular by data collected in a series of field trips in 2004, 2005 and 2006.

Second, the analysis of tones and tonal alternations (chapter 3)  considerably gained in sophistication and readability, and fully integrates recent research on Pumi tonology  (\citealt{matisoff97pumi},  \citealt{ding01pitch},  \citealt{ding03sketch}, \citealt{ding06tonal}, \citealt{ding07perception}, \citealt{jacques11pumi.tone}).

Third, the description of syntax (chapters 9 to 12) has been thoroughly rewritten and enriched with new data.  

allows more detailed dialectology of Pumi.
    
 In comparison with other dialects, slighly more conservative than the Wadu variety, which does not preserve any clusters (\citealt[20-1]{daudey14grammar})
 
 
However, while other dialects, such as Shuiluo, and Wadu, have a very productive <w> non-egophoric infix (\citealt{jacques11pumi.tone}, \citealt[338]{daudey14grammar}, \citealt[80]{daudey14volition}), in the Xinyingpan variety this infixs appears to be fossilized and only retained in a handful of verbs (p.121).
 
 

p9, dialectological classification, main isoglosses:
loss of clusters
fate of s+stop clusters as stops or as fricatives
could have added the reflexes of sth stʃh, cf    \citet{jacques11lingua}




My only reservation about this work is the near complete absence of any reference to the Tibetan cultural heritage of Pmui people, and the deep influence of Tibetan on the language. This is not simply a matter of detail, as not properly recognizing Tibetan loanwords leads to incorrect translations. For instance,  p184 the noun \ipa{sõ^Hdʒe^H} is translated as `Shakyamuni'. However, it is clearly borrowed from Tibetan \ipa{saŋs.rgʲas} `Buddha' -- although Shakyamuni is a buddha, other deities such as the Bonpo  \textit{Gshen.rab Mi.bo} are also called \ipa{saŋs.rgʲas} in Tibetan. This problem is particularly clear in the sample text on the wealth god Dzam.bha.lha (appendix 1), where Tibetan wordsq are left untranslated, for instance \ipa{dʒja^Lkɑ^F} from \ipa{rgʲa.gar} `India' is simply rendered as `Jjiaga people' in English).


It is hoped that the author will make his corpus of glossed texts available to the linguistic community, and that he will continue to do research on Pumi, for instance studying in more detail information structure  using he methodology he applied to Cantonese in his recent article (\citealt{ding14cantonese}).


\bibliographystyle{unified}
\bibliography{bibliogj}
 
\end{document}
