\documentclass[oldfontcommands,oneside,a4paper,11pt]{article} 
\usepackage{fontspec}
\usepackage{natbib}
\usepackage{booktabs}
\usepackage{xltxtra} 
\usepackage{polyglossia} 
\usepackage[table]{xcolor}
\usepackage{multirow}
\usepackage{gb4e} 
\usepackage{graphicx}
\usepackage{float}
\usepackage{lscape}
\usepackage{hyperref} 
\hypersetup{bookmarks=false,bookmarksnumbered,bookmarksopenlevel=5,bookmarksdepth=5,xetex,colorlinks=true,linkcolor=blue,citecolor=blue}
\usepackage[all]{hypcap}
\usepackage{memhfixc}
 
\setmainfont[Mapping=tex-text,Numbers=OldStyle,Ligatures=Common]{Charis SIL} %ici on définit la police par défaut du texte


\newfontfamily\phon[Mapping=tex-text,Ligatures=Common,Scale=MatchLowercase,FakeSlant=0.3]{Charis SIL} 
\newfontfamily\phondroit[Mapping=tex-text,Ligatures=Common,Scale=MatchLowercase]{Charis SIL} 
\newcommand{\ipa}[1]{{\phon #1}} %API tjs en italique
\newcommand{\ipac}[1]{{\tiny #1}}
\newcommand{\ipapl}[1]{{\phondroit #1}} 
\newfontfamily\cn[Mapping=tex-text,Ligatures=Common,Scale=MatchUppercase]{MingLiU}%pour le chinois
\newcommand{\zh}[1]{{\cn #1}}
\newfontfamily\mleccha[Mapping=tex-text,Ligatures=Common,Scale=MatchLowercase]{Galatia SIL}%pour le grec
\newcommand{\grec}[1]{{\mleccha #1}}
\newcommand{\tgz}[1]{#1 \mo{#1} \tg{#1}}
\newcommand{\indextg}[1]{\index{\tge{#1}@\tgz{#1}}}
\newcommand{\tgb}[1]{\tgz{#1}\indextg{#1}}
\newcommand{\tgc}[1]{\tg{#1} \#1\indextg{#1}}
\newcommand{\tgd}[1]{\tge{#1}\indextg{#1}}
\newcommand{\tgf}[1]{\mo{#1}\indextg{#1}}
\newcommand{\petit}[1]{\tiny#1}
\newcommand{\sig}{\begin{math}\Sigma\end{math}}
\newcommand{\phone}{\begin{math}\Phi\end{math}}
\newcommand{\ra}{$\Sigma_1$} 
\newcommand{\rc}{$\Sigma_3$} 
\newcommand{\grise}[1]{\cellcolor{lightgray}\textbf{#1}}


\begin{document}
%\OnehalfSpacing
\title{Review of `A grammar of Pumi' by Picus Ding}
\author{Guillaume Jacques}
\maketitle

\sloppy


\section{Introduction}
The year 2014 saw the completion of two important monographs on Pumi: the book under review (\citealt{ding14grammar}) and \citet{daudey14grammar}'s dissertation, defended at LaTrobe University. These two typologically-informed and corpus-based grammars represent a welcome improvement over previous descriptions of Pumi (\citealt{fual98pumi} and   \citealt{lusz01pumi}) which were essentially based on elicited sentences. While the scope of these two grammars is similar, they are by no means redundant, since they describe quite distinct varieties of Northern Pumi, and are based on different text corpora.

The book under review is partially based on the author's PhD dissertation (\citealt{ding98phd} -- incidentally also defended at LaTrobe University under D. Bradley). It represents however a considerable improvement over Ding's previous work on several accounts. 

First, the corpus has been considerably enlarged, in particular by data collected in a series of field trips in 2004, 2005 and 2006.

Second, the analysis of tones and tonal alternations (chapter 3)  considerably gained in sophistication and readability, and fully integrates recent research on Pumi tonology  (\citealt{matisoff97pumi},  \citealt{ding01pitch},  \citealt{ding03sketch}, \citealt{ding06tonal}, \citealt{ding07perception}, \citealt{jacques11pumi.tone}).

Third, the description of syntax (chapters 9 to 12) has been thoroughly rewritten and enriched with new data.  

Since another review of Ding's book already presents a general overview of each chapter in detail (\citealt{daudey15review}), the present work focuses on three topics: the contribution of this work to Pumi dialectology and to linguistic typology, as well as a few issues concerning the treatment of Tibetan loanwords.


\section{Dialectology}
Although this grammar is primarily the synchronic description of one variety of Pumi,  the dialectological context is fully integrated. The book starts (pp.8-9) with a brief overview of dialect classification based on phonological innovations, and discusses data from dialects other than Niuwozi (either from the author's own fieldwork or published sources) whenever appropriate. 

The features used in the author's classification are well-chosen; another important sound change that could have been included is the fate of proto-Pumi s+stop/affricate clusters as stops/affricates or as fricatives. As shown in \citet{jacques11lingua}, the proto-Pumi cluster *\ipa{stɕʰ--} for instance becomes an aspirated fricative \ipa{ɕʰ--} in Shuiluo vs \ipa{tɕʰ--} in Mudiqing. Relatively few dialects belong to the Shuiluo-type; in \citet{lusz01pumi}'s data, only Taoba and (for some items) Ludian dialects present fricatives corresponding to Proto-Pumi s+stop/affricate clusters (see items 166 and 582 for instance).

The Niuwozi dialect described in this grammar presents a combination of innovative and archaic features. 
 
From the point of view of phonology, the Niuwozi dialect is generally more conservative than the Wadu variety, which does not preserve any clusters (\citealt[20-1]{daudey14grammar}). It sides with dialects such as Mudiqing in its treatment of Proto-Pumi *s+stop/affricates clusters. The author describes several ongoing sound changes (pp. 45-8), such as the debuccalisation of voiceless nasals (\ipapl{n̥} > [\ipapl{h̃}]) with spread of nasality onto the following vowel (\citealt{michaud-jacques12nasalite} provide a typological overview of sound changes of this type, using among others data from Na languages in contact with Pumi) and the debuccalisation of the voiced velar obstruents to [\ipapl{ʔ}], which have already been brought to completion in other varieties.
 
On the other hand, the verbal morphology of the Niuwozi dialect appears to be slightly more innovative than other dialects. While the Shuiluo and  Wadu dialects have a very productive <w> non-egophoric infix (\citealt{jacques11pumi.tone}, \citealt[338]{daudey14grammar}, \citealt[80]{daudey14volition}), in the Niuwozi variety this infix appears to be fossilized and only retained in a handful of verbs as non-productive stem alternation (p.121). 
 
 

\section{Typology}
The rich and original data in this grammar make it a contribution to tonology and  morphosyntactic typology. Pumi is especially interesting in having an Tibetan-like egophoric system in combination with remnants of a system of polypersonal agreement.\footnote{In particular, it is possible that the non-egophoric <w> infix originates from an inverse marker cognate with the one found in Rgyalrong languages, such as the \ipa{wə--} prefix in Zbu (\citealt{gongxun14agreement}); note that in Khroskyabs and Stau, the inverse prefix has been generalized to all transitive non-local forms and reanalyzed as a type of transitive marker (\citealt{lai14person, jacques14rtau}).} It is fortunate that this unusual system could be adequately described and thus contribute to the typology of evidentials and their complex relation to person marking.

While some important references on evidentiality appeared too late for the author to  use (\citealt{tournadre14evidentiality} in particular), it is unfortunate, since the term `mirative' appears in this grammar (p.211-2), that the recent debate about this term in \textit{Linguistic Typology} is not discussed (\citealt{hill12mirativity}, \citealt{delancey12still}, \citealt{aikhenvald12mirativity}); the unfortunate neologism `surprisive' (p.214) could have been avoided. 

Another domain of morphosyntax where Pumi grammar is particularly interesting is information structure, and the author's rich corpus data allow a detailed and valuable description of topic and focus in this language (pp. 155-161 and pp. 297-317). Typologists interested in this domain of research can now include Pumi to their field of investigation.


\section{Tibetology}
My only reservation about Ding's work is the near complete absence of any reference to the Tibetan cultural heritage of Pumi people, and the deep influence of Tibetan on the language. This is not simply a matter of detail, as not properly recognizing Tibetan loanwords leads to incorrect translations. For instance,  on p.184 the noun \ipa{sõ^Hdʒe^H} is translated as `Shakyamuni'. However, it is clearly borrowed from Tibetan \ipa{saŋs.rgʲas} `Buddha' -- although Shakyamuni is a buddha, other deities such as the Bonpo  \textit{Gshen.rab Mi.bo} are also called \ipa{saŋs.rgʲas} in Tibetan. 


This problem is particularly clear in the sample text on the wealth god \textit{Dzam.bha.lha} (appendix 1), where Tibetan words are left untranslated. For instance \ipa{dʒja^Lkɑ^F} from \ipa{rgʲa.gar} `India' is simply rendered as `Jjiaga people' in English).

\section{Conclusion}
Picus Ding's grammar of Pumi is an important event in Sino-Tibetan linguistics, and will remain a lasting reference on the study of this language.

It is hoped that the author will make his corpus of glossed texts available to the linguistic community,\footnote{The Pangloss archive (\citealt{michailovsky14pangloss}) already contains a small Pumi text corpus collected by the author of this review, that could be enriched with additional data. } and that he will continue to do research on Pumi, for instance studying in more detail information structure  using the methodology he applied to Cantonese in his recent article (\citealt{ding14cantonese}).


\bibliographystyle{unified}
\bibliography{bibliogj}
 
\end{document}
