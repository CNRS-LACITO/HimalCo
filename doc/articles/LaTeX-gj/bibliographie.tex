\documentclass[oldfontcommands,oneside,a4paper,11pt]{article} 
\usepackage{fontspec}
\usepackage{natbib}
\usepackage{booktabs}
\usepackage{xltxtra} 
\usepackage{polyglossia} 
%\setdefaultlanguage{french} 
\usepackage[table]{xcolor}
\usepackage{gb4e} 
\usepackage{graphicx}
\usepackage{float}
\usepackage{hyperref} 
\hypersetup{bookmarks=false,bookmarksnumbered,bookmarksopenlevel=5,bookmarksdepth=5,xetex,colorlinks=true,linkcolor=blue,citecolor=blue}
\usepackage[all]{hypcap}
\usepackage{memhfixc}

\bibpunct[: ]{(}{)}{,}{a}{}{,}
 
\setmainfont[Mapping=tex-text,Numbers=OldStyle,Ligatures=Common]{Charis SIL}  

\newfontfamily\phon[Mapping=tex-text,Ligatures=Common,Scale=MatchLowercase,FakeSlant=0.3]{Charis SIL} 
\newcommand{\ipa}[1]{{\phon #1}} %API tjs en italique
 \newcommand{\ipapl}[1]{{\phon #1}} %API tjs en italique
\newcommand{\grise}[1]{\cellcolor{lightgray}\textbf{#1}}
\newfontfamily\cn[Mapping=tex-text,Ligatures=Common,Scale=MatchUppercase]{MingLiU}%pour le chinois
\newcommand{\zh}[1]{{\cn #1}}

\newcommand{\jg}[1]{\ipa{#1}\index{Japhug #1}}
\newcommand{\wav}[1]{#1.wav}
\newcommand{\tgz}[1]{\mo{#1} \tg{#1}}
\newcommand{\langue}[2]{#2}

 
\begin{document}


  \title{\langue{List of publications}{Liste de publications}}
 
\author{Guillaume Jacques, CNRS-INALCO-EHESS (CRLAO)}
\maketitle
\sloppy
 
 \tableofcontents
  
 \section{\langue{By journals}{Articles classés par revue}}
% 43 articles
%\langue{This list only includes published research articles, excluding book reviews and articles in press.}{Cette liste inclut seulement les articles de recherche déjà publiés, à l'exclusion des compte-rendus de livres et des articles acceptés sous presse.}
 \begin{enumerate}
 \item Amérindia (1): \citet{jacques12bear}
 \item Anthropological linguistics (1): \citet{japhug14ideophones}
 \item Bulletin of Chinese Linguistics (1): \citet{jacques15derivational.khaling}
 \item Bulletin of the School of Asian and Oriental Studies (3):  \citet{jacques10refl}, \citet{rg-gj12yod}, \citet{jacques13yod}
 \item Cahiers de linguistique d'Asie orientale (4): \citet{jacques00ywij},  \citet{jacques03dissimilation},   \citet{jacques07chang},  \citet{michaud10bonin}
 \item Central Asiatic Journal (2):  \citet{jacques10imperial}, \citet{jacques14ergative}
  \item Diachronica (3): \citet{jacques.michaud11naish}, \citet{michaud-jacques12nasalite}, \citet{jacques15comparative}
  \item Etudes Mongoles \& Sibériennes, Centrasiatiques \& Tibétaines (1):  \citet{jacques09e}
  \item Faits de langue (1): \citet{jacques07redupl}
 \item Folia Linguistica Historica (1): \citet{jacques13arapaho}, \citet{jacques15causative}
 \item International Journal of American Linguistics (1): \citet{jacques16ebde}
 \item Himalayan Linguistics (2): \citet{jacques10zos},  \citet{jacques14rtau}
 \item Historische Sprachforschung (1): \citet{jacques15cochon}
 \item Journal of Chinese Linguistics (3):   \citet{jacques11tangut.verb}, \citet{jacques15sr}, \citet{jacques16relatives}
 \item Journal of the American Oriental Society (1): \citet{jacques11ngwemi}
   \item Language and Linguistics (6):  \citet{jacques07passif}, \citet{jacques09tangutverb}, \citet{jacques10inverse},     \citet{jacques11pumi.tone}, \citet{jacques12agreement},    \citet{jacques12khaling}.  
   \item  Language and Linguistics Compass (1): \citet{jacques14inverse}   
 \item Lingua (3):  \citet{jacques11lingua}, \citet{jacques12incorp}, \citet{jacques14antipassive}
 \item Linguistic Inquiry (1): \citet{antonov14need}
 \item Linguistics of the Tibeto-Burman Area (3): \citet{jacques09wazur}, \citet{jacques13tropative}, \citet{jacques14linking}.
 \item Linguistic Typology (1): \citet{jacques13harmonization}
 \item Minzu yuwen \zh{民族语文} (3): \citet{jacques03s.houzhui}, \citet{jacques04redupl}, \citet{jacques08weiyu}
 \item Revue d'études tibétaines (3): \citet{jacques07naksatram},  \citet{jacques08debther},   \citet{jacques10ndr}
 \item Studia Etymologica Cracoviensia (1):  \citet{jacques13vama}
  \item Studies in Language (1): \citet{jacques14auditory}
  \item Transactions of the Philological Society (1):  \citet{jacques12internal}
  \item Turkic Languages (1): \citet{antonov12kumush}
 \end{enumerate} 
 
\section{\langue{By languages}{Publications classées par langue étudiée}}
\begin{itemize}
\item \langue{Sino-Tibetan}{Sino-tibétain}
\begin{enumerate}
\item \langue{General}{Général}: \citet{jacques03s.houzhui}, \citet{jacques06morpho}, \citet{jacques07chang}, \citet{antonov12kumush}, \citet{jacques12agreement},    \citet{michaud-jacques12nasalite}.     
\item Japhug:  \citet{jacques04redupl},     \citet{jacques04these},   \citet{jacques07passif},  \citet{jacques07redupl}, \citet{jacques08},  \citet{jacques10gesar}, \citet{jacques10refl},  \citet{jacques10inverse},  \citet{jacques12incorp},   \citet{jacques12demotion},  \citet{jacques13harmonization},  \citet{jacques13tropative}, \citet{jacques14antipassive}, \citet{japhug14ideophones}, \citet{jacques14inverse}, \citet{jacques14linking}, \citet{jacques15comparative}, \citet{jacques15causative}, \citet{jacques15japhug}, \citet{jacques16relatives}.
\item Khaling: \citet{jacques12khaling},  \citet{jacques13derivational.khaling}, \citet{jacques14auditory}, \citet{jacques15derivational.khaling}, \citet{jacques15khaling}.
\item Stau: \citet{antonov14rtau}, \citet{jacques14rtau}.
\item \langue{Tibetan}{Tibétain}:  \citet{jacques01dg}, \citet{jacques04thimphu}, \citet{jacques07naksatram},      \citet{jacques08debther},  \citet{jacques09wazur}, \citet{jacques09e},  \citet{jacques10zos},   \citet{jacques10ndr},  \citet{jacques11lingua},  \citet{jacques12internal},  \citet{jacques12transcription}, \citet{jacques13yod}, \citet{jacques14snom}, \citet{jacques14cone}.
\item Pumi:  \citet{michaud10bonin}, \citet{jacques11pumi.tone}, \citet{jacques11lingua}, 
\item \langue{Tangut}{Tangoute}: \citet{jacques06comparaison},  \citet{jacques07textes}, \citet{jacques08weiyu}, \citet{jacques08alternations},   \citet{jacques09tangutverb},  \citet{jacques10imperial},  \citet{jacques11tangut.verb}, \citet{jacques11ngwemi}, \citet{jacques11kinship},  \citet{jacques14esquisse}, \citet{jacques14ergative}.
\item Naish: \citet{jacques.michaud11naish}.
\item  Zhang-zhung: \citet{jacques09zz}.
\item \langue{Chinese}{Chinois}:  \citet{jacques00ywij},  \citet{jacques03dissimilation}, \citet{jacques15sr},   \citet{jacques2015genetic}, \citet{jacques2015traditional}.
\end{enumerate}
\item \langue{Indo-European}{Indo-européen}
\begin{enumerate}
\item Sanskrit: \citet{jacques13vama}.
\item \langue{Celtic}{Celtique}: \citet{michaud-jacques12nasalite}, \citet{jacques15cochon}
\end{enumerate}
\item \langue{Siouan}{Sioux}
\begin{enumerate}
\item \langue{General}{Général}: \citet{jacques12bear},      \citet{michaud-jacques12nasalite}.  
\item Omaha: \citet{jacques16ebde}.
\end{enumerate}
\item \langue{Semitic}{Sémitique}
\begin{enumerate}
\item \langue{Hebrew}{Hébreu}: \citet{rg-gj12yod}.
\end{enumerate}
\item \langue{Algonquian}{Algonquien}
\begin{enumerate}
\item \langue{General}{Général}: \citet{jacques12bear}, \citet{jacques14inverse}.
\item Arapaho: \citet{jacques13arapaho}.
\end{enumerate}
\item  \langue{Turkic}{Turcique}
\begin{enumerate}
\item  \langue{General}{Général}: \citet{antonov12kumush}.
\end{enumerate}
\end{itemize} 
 
\section{\langue{By topics}{Publications classées par domaine de recherche}}
\begin{itemize}

\item  Documentation
\begin{enumerate}
\item  \langue{Text collections}{Collections de textes}: \citet{jacques10gesar}.
\item \langue{General descriptions}{Descriptions générales}: \citet{jacques04these}, \citet{jacques08}.
\item \langue{Dictionaries}{Dictionnaires}: \citet{jacques15japhug}, \citet{jacques15khaling}
\end{enumerate}


\item \langue{Phonology}{Phonologie}
\begin{enumerate}
\item  \langue{Panchronic Phonology}{Phonologie panchronique}:  \citet{jacques11lingua}, \citet{michaud-jacques12nasalite},     \citet{jacques13arapaho}.
\item   \langue{Tonology}{Tonologie}: \citet{jacques11pumi.tone}.
\item \langue{Morphophonology}{Morphophonologie}: \citet{jacques12khaling}.
\item \langue{Reduplication}{Réduplication}:  \citet{jacques04redupl},  \citet{jacques07redupl}.
\end{enumerate}

\item \langue{Morphosyntax}{Morphosyntaxe}
\begin{enumerate}
\item  Incorporation: \citet{jacques11tangut.verb}, \citet{jacques12incorp}.
\item  \langue{Voice}{Voix}:  \citet{jacques07passif}, \citet{jacques10refl}, \citet{jacques12demotion}, \citet{jacques13derivational.khaling}, \citet{jacques13tropative}, \citet{jacques14antipassive}, \citet{jacques15derivational.khaling}, \citet{jacques15causative}.
\item \langue{Hierarchical agreement}{Indexation hiérarchique}:  \citet{jacques10inverse},     \citet{jacques12khaling},   \citet{antonov14rtau}, \citet{jacques14inverse}, \citet{jacques14rtau}.
\item \langue{Evidentiality}{Evidentialité}: \citet{jacques14auditory}.
\item \langue{Word-order typology}{Typologie de l'ordre des mots}: \citet{jacques13harmonization}.
\item \langue{Implicational universals}{Universels implicationnels}: \citet{antonov14need}.
\item \langue{Ideophones}{Idéophones}: \citet{japhug14ideophones}.
\item \langue{Relativization}{Relatives}: \citet{jacques16relatives}
\item \langue{Comparative clauses}{Comparatives}: \citet{jacques15comparative}
\end{enumerate}

\item \langue{Formal linguistics}{Linguistique formelle}
\begin{enumerate}
\item \citet{walther14inv.canon}.
\item \citet{walther14compactness}.
\end{enumerate}

\item \langue{Historical linguistics}{Linguistique historique}
\begin{enumerate}
\item \langue{Sound laws}{Lois phonétiques}:  \citet{jacques00ywij}, \citet{jacques01dg}, \citet{jacques03dissimilation},   \citet{jacques04thimphu}, \citet{jacques06comparaison}, \citet{jacques09wazur}, \citet{jacques09e}, \citet{michaud10bonin}, \citet{jacques10ndr}, \citet{jacques.michaud11naish},   \citet{rg-gj12yod}, \citet{jacques13arapaho}, \citet{jacques13yod}, \citet{jacques14snom},   \citet{jacques14esquisse}, \citet{jacques14cone}, \citet{jacques16ebde}.
\item  \langue{Etymology}{Etymologie}:    \citet{jacques07naksatram}, \citet{jacques08debther}, \citet{jacques09zz}, \citet{jacques10imperial}, \citet{jacques11ngwemi}, \citet{jacques12bear},  \citet{jacques13vama},   \citet{jacques14esquisse}, \citet{jacques15sr}, \citet{jacques15cochon}.
\item \langue{Historical morphology}{Morphologie historique}:  \citet{jacques03s.houzhui}, \citet{jacques06morpho}, \citet{jacques07chang},    \citet{jacques09tangutverb}, \citet{jacques10zos}, \citet{jacques11tangut.verb}, \citet{jacques12agreement}, \citet{jacques12internal}, \citet{jacques14antipassive},   \citet{jacques14esquisse}, \citet{jacques14cone}, \citet{jacques15derivational.khaling}, \citet{jacques15causative}, \citet{jacques16ebde}.
\item \langue{Grammaticalization:}{Grammaticalisation} \citet{jacques12internal},\citet{jacques13harmonization}, \citet{jacques14antipassive}, \citet{jacques14ergative}, \citet{jacques15comparative}
\item \langue{Language contact:}{Contact de langues} \citet{antonov12kumush},   \citet{jacques12bear}.
\item \langue{Genetic linguistics}{Linguistique génétique}:  \citet{jacques07chang}, \citet{jacques12agreement}, \citet{jacques2015genetic}.
\end{enumerate}
\item \langue{Ethnolinguistics}{Ethnolinguistique}
\begin{enumerate}
\item \langue{Kinship}{Parenté}: \citet{jacques11kinship}. 

\end{enumerate}
  
  
\end{itemize}


\bibliographystyle{Linquiry2}
\bibliography{bibliogj}
\end{document}