\documentclass[oneside,a4paper,11pt]{article} 
\usepackage{fontspec}
\usepackage[numbers]{natbib}
\usepackage{booktabs}
\usepackage{xltxtra} 
\usepackage{polyglossia} 
%\setdefaultlanguage{french} 
\usepackage[table]{xcolor}
\usepackage{gb4e} 
\usepackage{graphicx}
\usepackage{float}
\usepackage{hyperref} 
\hypersetup{bookmarks=false,bookmarksnumbered,bookmarksopenlevel=5,bookmarksdepth=5,xetex,colorlinks=true,linkcolor=blue,citecolor=blue}
\usepackage[all]{hypcap}
\usepackage{memhfixc}

\setmainfont[Mapping=tex-text,Numbers=OldStyle,Ligatures=Common]{Charis SIL}  

\newfontfamily\phon[Mapping=tex-text,Ligatures=Common,Scale=MatchLowercase]{Charis SIL} 
\newcommand{\ipa}[1]{{\phon #1}} %API tjs en italique
 \newcommand{\ipapl}[1]{{\phon #1}} %API tjs en italique
\newcommand{\grise}[1]{\cellcolor{lightgray}\textbf{#1}}
\newfontfamily\cn[Mapping=tex-text,Ligatures=Common,Scale=MatchUppercase]{SimSun}%pour le chinois
\newcommand{\zh}[1]{{\cn #1}}

\newcommand{\jg}[1]{\ipa{#1}\index{Japhug #1}}
\newcommand{\wav}[1]{#1.wav}
\newcommand{\tgz}[1]{\mo{#1} \tg{#1}}
\newcommand{\langue}[2]{#1}
\newcommand{\lingua}[3]{#2}%{#1 (\zh{#3})}
 
\begin{document}


  \title{\lingua{List of publications}{Liste de publications}{论文著作}}
 
\author{Guillaume Jacques, CNRS-INALCO-EHESS (CRLAO)}
\maketitle
\sloppy
 
 \tableofcontents
  
 \section{\lingua{By journals}{Articles classés par revue}{发表的刊物}}
% 43 articles
%\langue{This list only includes published research articles, excluding book reviews and articles in press.}{Cette liste inclut seulement les articles de recherche déjà publiés, à l'exclusion des compte-rendus de livres et des articles acceptés sous presse.}
 \begin{enumerate}
 \item Amérindia (1): \cite{jacques12bear}.
 \item Anthropological linguistics (1): \cite{japhug14ideophones}.
 \item Bulletin of Chinese Linguistics (2): \cite{jacques15derivational.khaling}, \cite{jacques16ssuffixes}.
 \item Bulletin of the School of Asian and Oriental Studies (3):  \cite{jacques10refl}, \cite{rg-gj12yod}, \cite{jacques13yod}.
 \item Cahiers de linguistique d'Asie orientale (4): \cite{jacques00ywij},  \cite{jacques03dissimilation},   \cite{jacques07chang},  \cite{michaud10bonin}.
 \item Central Asiatic Journal (2):  \cite{jacques10imperial}, \cite{jacques14ergative}.
  \item Diachronica (3): \cite{jacques.michaud11naish}, \cite{michaud-jacques12nasalite}, \cite{jacques16comparative}.
  \item Etudes Mongoles \& Sibériennes, Centrasiatiques \& Tibétaines (1):  \cite{jacques09e}.
  \item Faits de langue (2): \cite{jacques07redupl}, \cite{jacques16polysynthetique}.
 \item Folia Linguistica Historica (2): \cite{jacques13arapaho}, \cite{jacques15causative}.
 \item International Journal of American Linguistics (1): \cite{jacques16ebde}.
 \item Himalayan Linguistics (2): \cite{jacques10zos},  \cite{jacques14rtau}.
 \item Historische Sprachforschung (1): \cite{jacques15cochon}.
 \item Journal of Chinese Linguistics (4):   \cite{jacques11tangut.verb}, \cite{jacques15sr}, \cite{jacques16relatives}, \cite{jacques17buyang}.
 \item Journal of Ethnobiology and Ethnomedicine (1): \cite{Kang2016}.
 \item Journal of the American Oriental Society (1): \cite{jacques11ngwemi}.
 \item Journal of the International Phonetic Association (1): \cite{jacques18ipa}.
   \item Language and Linguistics (6):  \cite{jacques07passif}, \cite{jacques09tangutverb}, \cite{jacques10inverse},     \cite{jacques11pumi.tone}, \cite{jacques12agreement},    \cite{jacques12khaling}.  
   \item  Language and Linguistics Compass (1): \cite{jacques14inverse}.   
 \item Lingua (3):  \cite{jacques11lingua}, \cite{jacques12incorp}, \cite{jacques14antipassive}.
 \item Linguistic Inquiry (1): \cite{antonov14need}.
 \item Linguistic Vanguard (1): \cite{jacques16th}.
 \item Linguistics of the Tibeto-Burman Area (6): \cite{jacques09wazur}, \cite{jacques13tropative}, \cite{jacques14linking}, \cite{jacques15spontaneous}, \cite{jacques16si}, \cite{jacques16complementation}.
 \item Linguistic Typology (1): \cite{jacques13harmonization}.
 \item Minzu yuwen \zh{民族语文} (3): \cite{jacques03s.houzhui}, \cite{jacques04redupl}, \cite{jacques08weiyu}.
 \item Revue d'études tibétaines (3): \cite{jacques07naksatram},  \cite{jacques08debther},   \cite{jacques10ndr}.
 \item Studia Etymologica Cracoviensia (1):  \cite{jacques13vama}.
  \item Studies in Language (1): \cite{jacques14auditory}.
  \item Transactions of the Philological Society (1):  \cite{jacques12internal}.
  \item Turkic Languages (1): \cite{antonov12kumush}.
\item *Wékʷos (3): \cite{jacques14honey}, \cite{jacques16camara}, \cite{jacques17hrd}.
 \end{enumerate} 
 
\section{\lingua{By languages}{Publications classées par langue étudiée}{研究的语言}}
\begin{itemize}
\item \lingua{Sino-Tibetan}{Sino-tibétain}{汉藏语系}
\begin{enumerate}
\item \lingua{General}{Général}{总论}: \cite{jacques03s.houzhui}, \cite{jacques06morpho}, \cite{jacques07chang}, \cite{antonov12kumush}, \cite{jacques12agreement},    \cite{michaud-jacques12nasalite}, \cite{jacques16th}, \cite{jacques16polysynthetique}, \cite{jacques17rgy}.     
\item \lingua{Japhug}{Japhug}{茶堡嘉绒语}:  \cite{jacques04redupl},     \cite{jacques04these},   \cite{jacques07passif},  \cite{jacques07redupl}, \cite{jacques08},  \cite{jacques10gesar}, \cite{jacques10refl},  \cite{jacques10inverse},  \cite{jacques12incorp},   \cite{jacques12demotion},  \cite{jacques13harmonization},  \cite{jacques13tropative}, \cite{jacques14antipassive}, \cite{japhug14ideophones}, \cite{jacques14inverse}, \cite{jacques14linking}, \cite{jacques16comparative}, \cite{jacques15causative}, \cite{jacques15japhug}, \cite{jacques16relatives}, \cite{jacques16complementation}, \cite{jacques17sketch}, \cite{jacques17num}, \cite{jacques17comitative}, \cite{jacques17volitional}, \cite{jacques18ipa}, \cite{jacques18generic}, \cite{jacques18bipartite}, \cite{jacques19contact}, \cite{jacques19egophoric}.
\item \lingua{Khaling}{Khaling}{卡陵语}: \cite{jacques12khaling},  \cite{jacques13derivational.khaling}, \cite{jacques14auditory}, \cite{jacques15derivational.khaling}, \cite{jacques15khaling},  \cite{jacques16si}, \cite{jacques16tonogenesis}.
\item \lingua{Stau}{Stau}{道孚语}: \cite{antonov14rtau}, \cite{jacques14rtau}, \cite{jacques17stau}.
\item \lingua{Tibetan}{Tibétain}{藏语}:  \cite{jacques01dg}, \cite{jacques04thimphu}, \cite{jacques07naksatram},      \cite{jacques08debther},  \cite{jacques09wazur}, \cite{jacques09e},  \cite{jacques10zos},   \cite{jacques10ndr},  \cite{jacques11lingua},  \cite{jacques12internal},  \cite{jacques12transcription}, \cite{jacques13yod}, \cite{jacques14snom}, \cite{jacques14cone}, \cite{jacques19contact}.
\item \lingua{Pumi}{Pumi}{普米语}:  \cite{michaud10bonin}, \cite{jacques11pumi.tone}, \cite{jacques11lingua}.
\item \lingua{Tangut}{Tangoute}{西夏语}: \cite{jacques06comparaison},  \cite{jacques07textes}, \cite{jacques08weiyu}, \cite{jacques08alternations},   \cite{jacques09tangutverb},  \cite{jacques10imperial},  \cite{jacques11tangut.verb}, \cite{jacques11ngwemi}, \cite{jacques11kinship},  \cite{jacques14esquisse}, \cite{jacques14ergative}, \cite{jacques16th}.
\item \lingua{Naish}{Naish}{纳语支}: \cite{jacques.michaud11naish}.
\item  \lingua{Zhang-zhung}{Zhang-zhung}{象雄语}: \cite{jacques09zz}.
\item \lingua{Chinese}{Chinois}{汉语}:  \cite{jacques00ywij},  \cite{jacques03dissimilation},   \cite{jacques14honey},\cite{jacques15sr},\cite{jacques17genetic}, \cite{jacques17traditional}, \cite{jacques17buyang}, \cite{jacques16ssuffixes}.
\end{enumerate}
\item \lingua{Indo-European}{Indo-européen}{印欧语系}
\begin{enumerate}
\item \lingua{Sanskrit}{Sanskrit}{梵文}: \cite{jacques13vama}, \cite{jacques16camara}, \cite{jacques17hrd}.
\item \lingua{Celtic}{Celtique}{凯尔特语族}: \cite{michaud-jacques12nasalite}, \cite{jacques15cochon}.
\item \lingua{Tokharian}{Tocharien}{吐火罗语}: \cite{jacques14honey}.
\end{enumerate}
\item \lingua{Siouan}{Sioux}{苏语系}
\begin{enumerate}
\item \lingua{General}{Général}{总论}: \cite{jacques12bear}, \cite{michaud-jacques12nasalite}.  
\item \lingua{Omaha}{Omaha}{奥马哈语}: \cite{jacques16ebde}.
\end{enumerate}
\item \lingua{Semitic}{Sémitique}{闪语族}
\begin{enumerate}
\item \lingua{Hebrew}{Hébreu}{希伯来语}: \cite{rg-gj12yod}.
\end{enumerate}
\item \lingua{Algonquian}{Algonquien}{阿尔冈昆语族}
\begin{enumerate}
\item \lingua{General}{Général}{总论}:\cite{jacques12bear}, \cite{jacques14inverse}, \cite{jacques18directionality}.
\item \lingua{Arapaho}{Arapaho}{阿拉帕霍语}: \cite{jacques13arapaho}, \cite{jacques16phono.arapaho}.
\end{enumerate}
\item \lingua{Tungusic}{Toungouse}{通古斯语系}
\begin{enumerate}
\item \lingua{Manchu}{Mandchou}{满语}: \cite{fuente19am}.
\end{enumerate}
\item  \lingua{Turkic}{Turcique}{突厥语系}
\begin{enumerate}
\item \lingua{General}{Général}{总论}: \cite{antonov12kumush}.
\end{enumerate}
\item \lingua{Kra-Dai}{Kra-Dai}{侗台语系}
\begin{enumerate}
\item \lingua{Buyang}{Buyang}{布央语}: \cite{jacques17buyang}.
\end{enumerate}
\end{itemize}  

\section{\lingua{By topics}{Publications classées par domaine de recherche}{研究领域}}
\begin{itemize}
\item  \lingua{Linguistics}{Linguistique}{语言学}
\begin{itemize}
\item  \lingua{Documentation}{Documentation}{记录语言学}
\begin{enumerate}
\item  \lingua{Text collections}{Collections de textes}{语料库}: \cite{jacques10gesar}.
\item \lingua{General descriptions}{Descriptions générales}{语法概论以及语法参考书}: \cite{jacques04these}, \cite{jacques08}, \cite{jacques17sketch}, \cite{jacques17stau}.
\item \lingua{Dictionaries}{Dictionnaires}{词典}: \cite{jacques15japhug}, \cite{jacques15khaling}.
\end{enumerate}

\item \lingua{Phonology}{Phonologie}{音系学}
\begin{enumerate}
\item  \lingua{Phonemic inventory}{Inventaire phonémique}{音位系统}: \cite{jacques04these},  \cite{jacques12khaling}, \cite{jacques14cone}, \cite{jacques18ipa}.
\item  \lingua{Panchronic Phonology}{Phonologie panchronique}{泛时音系学}:  \cite{jacques11lingua}, \cite{michaud-jacques12nasalite},     \cite{jacques13arapaho}, \cite{jacques16phono.arapaho}.
\item   \lingua{Tonology}{Tonologie}{声调}: \cite{jacques11pumi.tone}, \cite{jacques16tonogenesis}.
\item \lingua{Morphophonology}{Morphophonologie}{形态音系学}: \cite{jacques12khaling}.
\item \lingua{Reduplication}{Réduplication}{重叠式}:  \cite{jacques04redupl},  \cite{jacques07redupl}.
\end{enumerate}

\item \lingua{Morphosyntax}{Morphosyntaxe}{形态句法}
\begin{enumerate}
\item \lingua{Associated motion}{Mouvement associé}{关联位移}:
\cite{jacques13harmonization}, \cite{fuente19am}
\item  \lingua{Incorporation}{Incorporation}{名词并入}: \cite{jacques11tangut.verb}, \cite{jacques12demotion}, \cite{jacques12incorp}.
\item  \lingua{Voice}{Voix}{语态}:  \cite{jacques07passif}, \cite{jacques10refl}, \cite{jacques12demotion}, \cite{jacques13derivational.khaling}, \cite{jacques13tropative}, \cite{jacques14antipassive}, \cite{jacques15derivational.khaling}, \cite{jacques15causative},  \cite{jacques15spontaneous},  \cite{jacques16si}, \cite{jacques18generic}, \cite{jacques17volitional}.
\item \lingua{Hierarchical indexation}{Indexation hiérarchique}{人称范畴}:  \cite{jacques10inverse},     \cite{jacques12khaling},   \cite{antonov14rtau}, \cite{jacques14inverse}, \cite{jacques14rtau}, \cite{jacques16th}, \cite{jacques17stau}, \cite{jacques18generic}.
\item \lingua{Evidentiality}{Evidentialité}{示证范畴}: \cite{jacques14auditory}, \cite{jacques18nonpropositional}, \cite{jacques19egophoric}.
\item \lingua{Word-order typology}{Typologie de l'ordre des mots}{语序}: \cite{jacques13harmonization}.
\item \lingua{Implicational universals}{Universels implicationnels}{蕴涵共性}: \cite{antonov14need}.
\item \lingua{Ideophones}{Idéophones}{状貌词}: \cite{japhug14ideophones}, \cite{jacques18ipa}.
\item \lingua{Relativization}{Relatives}{关系句}: \cite{jacques16relatives}.
\item \lingua{Comparative clauses}{Comparatives}{差比句}: \cite{jacques16comparative}.
\item \lingua{Complementation}{Complétives}{补语}: \cite{jacques16complementation}.
\item \lingua{Hybrid indirect speech}{Discours indirect hybride}{混合间接引语}: \cite{jacques16complementation}, \cite{jacques17stau}.
\item \lingua{Noun phrase}{Syntagmes nominaux}{名词组}: \cite{jacques17num}, \cite{jacques17sketch}, \cite{jacques17comitative}.
\end{enumerate}

\item \lingua{Formal linguistics}{Linguistique formelle}{形式语言学}
\begin{enumerate}
\item \lingua{Implemented morphology}{Morphologie implémentée}{有计算机程序实现的形态学研究}: \cite{walther14inv.canon}, \cite{walther14compactness}.
\end{enumerate}

\item \lingua{Historical linguistics}{Linguistique historique}{历史语言学}
\begin{enumerate}
\item \lingua{Sound laws}{Lois phonétiques}{语音规则}:  \cite{jacques00ywij}, \cite{jacques01dg}, \cite{jacques03dissimilation},   \cite{jacques04thimphu}, \cite{jacques06comparaison}, \cite{jacques09wazur}, \cite{jacques09e}, \cite{michaud10bonin}, \cite{jacques10ndr}, \cite{jacques.michaud11naish},   \cite{rg-gj12yod}, \cite{jacques13arapaho}, \cite{jacques13yod}, \cite{jacques14snom},   \cite{jacques14esquisse}, \cite{jacques14cone}, \cite{jacques16ebde}, \cite{jacques16tonogenesis}, \cite{jacques17pkiranti}.
\item  \lingua{Etymology}{Etymologie}{词源研究}:    \cite{jacques07naksatram}, \cite{jacques08debther}, \cite{jacques09zz}, \cite{jacques10imperial}, \cite{jacques11ngwemi}, \cite{jacques12bear},  \cite{jacques13vama},   \cite{jacques14esquisse}, \cite{jacques14honey}, \cite{jacques15sr}, \cite{jacques15cochon}, \cite{jacques17stau}, \cite{jacques17pkiranti}.
\item \lingua{Historical morphology}{Morphologie historique}{历史形态学}:  \cite{jacques03s.houzhui}, \cite{jacques06morpho}, \cite{jacques07chang},    \cite{jacques09tangutverb}, \cite{jacques10zos}, \cite{jacques11tangut.verb}, \cite{jacques12agreement}, \cite{jacques12internal}, \cite{jacques14antipassive},   \cite{jacques14esquisse}, \cite{jacques14cone}, \cite{jacques15derivational.khaling}, \cite{jacques15causative},  \cite{jacques15spontaneous}, \cite{jacques16ebde}, \cite{jacques16tonogenesis}, \cite{jacques16th}, \cite{jacques17stau}, \cite{jacques17num}, \cite{jacques17pkiranti}, \cite{jacques18directionality}.
\item \lingua{Grammaticalization}{Grammaticalisation}{语法化}: \cite{jacques12internal},\cite{jacques13harmonization}, \cite{jacques14antipassive}, \cite{jacques14ergative}, \cite{jacques16comparative},  \cite{jacques18generic}, \cite{jacques16ssuffixes}.
\item \lingua{Language contact}{Contact de langues}{语言接触}: \cite{antonov12kumush},   \cite{jacques12bear}, \cite{jacques14honey}, \cite{jacques19contact}.
\item \lingua{Genetic linguistics}{Linguistique génétique}{语言的亲属关系}: \cite{jacques07chang}, \cite{jacques12agreement}, \cite{jacques17genetic}, \cite{jacques17stau}.
\end{enumerate}
\item \lingua{Ethnolinguistics}{Ethnolinguistique}{人类文化语言学}
\begin{enumerate}
\item \lingua{Kinship}{Parenté}{亲属制度}: \cite{jacques11kinship}. 
\end{enumerate}
  \end{itemize}
  \item  \lingua{Interdisciplinary}{Interdisciplinaire}{跨学科}
\begin{itemize}
  \item \lingua{Ethnobotany}{Ethnobotanique}{民族植物学} :  \cite{Kang2016}
\end{itemize}
\end{itemize}
\bibliographystyle{unified}
\bibliography{bibliogj}
\end{document}