\documentclass[oneside,a4paper,11pt]{article} 
\usepackage{fontspec}
\usepackage[numbers]{natbib}
\usepackage{booktabs}
\usepackage{xltxtra} 
\usepackage{polyglossia} 
%\setdefaultlanguage{french} 
\usepackage[table]{xcolor}
\usepackage{gb4e} 
\usepackage{graphicx}
\usepackage{float}
\usepackage{hyperref} 
\hypersetup{bookmarks=false,bookmarksnumbered,bookmarksopenlevel=5,bookmarksdepth=5,xetex,colorlinks=true,linkcolor=blue,citecolor=blue}
\usepackage[all]{hypcap}
\usepackage{memhfixc}

\setmainfont[Mapping=tex-text,Numbers=OldStyle,Ligatures=Common]{Charis SIL}  

\newfontfamily\phon[Mapping=tex-text,Ligatures=Common,Scale=MatchLowercase,FakeSlant=0.3]{Charis SIL} 
\newcommand{\ipa}[1]{{\phon #1}} %API tjs en italique
 \newcommand{\ipapl}[1]{{\phon #1}} %API tjs en italique
\newcommand{\grise}[1]{\cellcolor{lightgray}\textbf{#1}}
\newfontfamily\cn[Mapping=tex-text,Ligatures=Common,Scale=MatchUppercase]{SimSun}%pour le chinois
\newcommand{\zh}[1]{{\cn #1}}

\newcommand{\jg}[1]{\ipa{#1}\index{Japhug #1}}
\newcommand{\wav}[1]{#1.wav}
\newcommand{\tgz}[1]{\mo{#1} \tg{#1}}
\newcommand{\langue}[2]{#1}

 
\begin{document}


  \title{\langue{List of publications}{Liste de publications}}
 
\author{Guillaume Jacques, CNRS-INALCO-EHESS (CRLAO)}
\maketitle
\sloppy
 
 \tableofcontents
  
 \section{\langue{By journals}{Articles classés par revue}}
% 43 articles
%\langue{This list only includes published research articles, excluding book reviews and articles in press.}{Cette liste inclut seulement les articles de recherche déjà publiés, à l'exclusion des compte-rendus de livres et des articles acceptés sous presse.}
 \begin{enumerate}
 \item Amérindia (1): \cite{jacques12bear}.
 \item Anthropological linguistics (1): \cite{japhug14ideophones}.
 \item Bulletin of Chinese Linguistics (2): \cite{jacques15derivational.khaling}, \cite{jacques16ssuffixes}.
 \item Bulletin of the School of Asian and Oriental Studies (3):  \cite{jacques10refl}, \cite{rg-gj12yod}, \cite{jacques13yod}.
 \item Cahiers de linguistique d'Asie orientale (4): \cite{jacques00ywij},  \cite{jacques03dissimilation},   \cite{jacques07chang},  \cite{michaud10bonin}.
 \item Central Asiatic Journal (2):  \cite{jacques10imperial}, \cite{jacques14ergative}.
  \item Diachronica (3): \cite{jacques.michaud11naish}, \cite{michaud-jacques12nasalite}, \cite{jacques16comparative}.
  \item Etudes Mongoles \& Sibériennes, Centrasiatiques \& Tibétaines (1):  \cite{jacques09e}.
  \item Faits de langue (1): \cite{jacques07redupl}.
 \item Folia Linguistica Historica (2): \cite{jacques13arapaho}, \cite{jacques15causative}.
 \item International Journal of American Linguistics (1): \cite{jacques16ebde}.
 \item Himalayan Linguistics (2): \cite{jacques10zos},  \cite{jacques14rtau}.
 \item Historische Sprachforschung (1): \cite{jacques15cochon}.
 \item Journal of Chinese Linguistics (4):   \cite{jacques11tangut.verb}, \cite{jacques15sr}, \cite{jacques16relatives}, \cite{jacques17buyang}.
 \item Journal of Ethnobiology and Ethnomedicine (1): \cite{Kang2016}.
 \item Journal of the American Oriental Society (1): \cite{jacques11ngwemi}.
 \item Journal of the International Phonetic Association (1): \cite{jacques17ipa}.
   \item Language and Linguistics (6):  \cite{jacques07passif}, \cite{jacques09tangutverb}, \cite{jacques10inverse},     \cite{jacques11pumi.tone}, \cite{jacques12agreement},    \cite{jacques12khaling}.  
   \item  Language and Linguistics Compass (1): \cite{jacques14inverse}.   
 \item Lingua (3):  \cite{jacques11lingua}, \cite{jacques12incorp}, \cite{jacques14antipassive}.
 \item Linguistic Inquiry (1): \cite{antonov14need}.
 \item Linguistic Vanguard (1): \cite{jacques16th}.
 \item Linguistics of the Tibeto-Burman Area (6): \cite{jacques09wazur}, \cite{jacques13tropative}, \cite{jacques14linking}, \cite{jacques15spontaneous}, \cite{jacques16si}, \cite{jacques16complementation}.
 \item Linguistic Typology (1): \cite{jacques13harmonization}.
 \item Minzu yuwen \zh{民族语文} (3): \cite{jacques03s.houzhui}, \cite{jacques04redupl}, \cite{jacques08weiyu}.
 \item Revue d'études tibétaines (3): \cite{jacques07naksatram},  \cite{jacques08debther},   \cite{jacques10ndr}.
 \item Studia Etymologica Cracoviensia (1):  \cite{jacques13vama}.
  \item Studies in Language (1): \cite{jacques14auditory}.
  \item Transactions of the Philological Society (1):  \cite{jacques12internal}.
  \item Turkic Languages (1): \cite{antonov12kumush}.
\item *Wékʷos (2): \cite{jacques14honey}, \cite{jacques16camara}.
 \end{enumerate} 
 
\section{\langue{By languages}{Publications classées par langue étudiée}}
\begin{itemize}
\item \langue{Sino-Tibetan}{Sino-tibétain}
\begin{enumerate}
\item \langue{General}{Général}: \cite{jacques03s.houzhui}, \cite{jacques06morpho}, \cite{jacques07chang}, \cite{antonov12kumush}, \cite{jacques12agreement},    \cite{michaud-jacques12nasalite}, \cite{jacques16th}, \cite{jacques17rgy}.     
\item Japhug:  \cite{jacques04redupl},     \cite{jacques04these},   \cite{jacques07passif},  \cite{jacques07redupl}, \cite{jacques08},  \cite{jacques10gesar}, \cite{jacques10refl},  \cite{jacques10inverse},  \cite{jacques12incorp},   \cite{jacques12demotion},  \cite{jacques13harmonization},  \cite{jacques13tropative}, \cite{jacques14antipassive}, \cite{japhug14ideophones}, \cite{jacques14inverse}, \cite{jacques14linking}, \cite{jacques16comparative}, \cite{jacques15causative}, \cite{jacques15japhug}, \cite{jacques16relatives}, \cite{jacques16complementation}, \cite{jacques17sketch}, \cite{jacques17num}, \cite{jacques17generic}, \cite{jacques17ipa}.
\item Khaling: \cite{jacques12khaling},  \cite{jacques13derivational.khaling}, \cite{jacques14auditory}, \cite{jacques15derivational.khaling}, \cite{jacques15khaling},  \cite{jacques16si}, \cite{jacques16tonogenesis}.
\item Stau: \cite{antonov14rtau}, \cite{jacques14rtau}, \cite{jacques17stau}.
\item \langue{Tibetan}{Tibétain}:  \cite{jacques01dg}, \cite{jacques04thimphu}, \cite{jacques07naksatram},      \cite{jacques08debther},  \cite{jacques09wazur}, \cite{jacques09e},  \cite{jacques10zos},   \cite{jacques10ndr},  \cite{jacques11lingua},  \cite{jacques12internal},  \cite{jacques12transcription}, \cite{jacques13yod}, \cite{jacques14snom}, \cite{jacques14cone}.
\item Pumi:  \cite{michaud10bonin}, \cite{jacques11pumi.tone}, \cite{jacques11lingua}.
\item \langue{Tangut}{Tangoute}: \cite{jacques06comparaison},  \cite{jacques07textes}, \cite{jacques08weiyu}, \cite{jacques08alternations},   \cite{jacques09tangutverb},  \cite{jacques10imperial},  \cite{jacques11tangut.verb}, \cite{jacques11ngwemi}, \cite{jacques11kinship},  \cite{jacques14esquisse}, \cite{jacques14ergative}, \cite{jacques16th}.
\item Naish: \cite{jacques.michaud11naish}.
\item  Zhang-zhung: \cite{jacques09zz}.
\item \langue{Chinese}{Chinois}:  \cite{jacques00ywij},  \cite{jacques03dissimilation},   \cite{jacques14honey},\cite{jacques15sr},\cite{jacques17genetic}, \cite{jacques17traditional}, \cite{jacques17buyang}, \cite{jacques16ssuffixes}.
\end{enumerate}
\item \langue{Indo-European}{Indo-européen}
\begin{enumerate}
\item Sanskrit: \cite{jacques13vama}, \cite{jacques16camara}.
\item \langue{Celtic}{Celtique}: \cite{michaud-jacques12nasalite}, \cite{jacques15cochon}.
\item \langue{Tokharian}{Tocharien}: \cite{jacques14honey}.
\end{enumerate}
\item \langue{Siouan}{Sioux}
\begin{enumerate}
\item \langue{General}{Général}: \cite{jacques12bear},      \cite{michaud-jacques12nasalite}.  
\item Omaha: \cite{jacques16ebde}.
\end{enumerate}
\item \langue{Semitic}{Sémitique}
\begin{enumerate}
\item \langue{Hebrew}{Hébreu}: \cite{rg-gj12yod}.
\end{enumerate}
\item \langue{Algonquian}{Algonquien}
\begin{enumerate}
\item \langue{General}{Général}: \cite{jacques12bear}, \cite{jacques14inverse}, \cite{jacques17directionality}.
\item Arapaho: \cite{jacques13arapaho}.
\end{enumerate}
\item  \langue{Turkic}{Turcique}
\begin{enumerate}
\item  \langue{General}{Général}: \cite{antonov12kumush}.
\end{enumerate}
\item Kra-Dai
\begin{enumerate}
\item Buyang: \cite{jacques17buyang}.
\end{enumerate}
\end{itemize} 
 
\section{\langue{By topics}{Publications classées par domaine de recherche}}
\begin{itemize}
\item  \langue{Linguistics}{Linguistique} 
\begin{itemize}
\item  Documentation
\begin{enumerate}
\item  \langue{Text collections}{Collections de textes}: \cite{jacques10gesar}.
\item \langue{General descriptions}{Descriptions générales}: \cite{jacques04these}, \cite{jacques08}, \cite{jacques17sketch}, \cite{jacques17stau}.
\item \langue{Dictionaries}{Dictionnaires}: \cite{jacques15japhug}, \cite{jacques15khaling}.
\end{enumerate}


\item \langue{Phonology}{Phonologie}
\begin{enumerate}
\item  \langue{Phonemic inventory}{Inventaire phonémique}: \cite{jacques04these},  \cite{jacques12khaling}, \cite{jacques14cone}, \cite{jacques17ipa}.
\item  \langue{Panchronic Phonology}{Phonologie panchronique}:  \cite{jacques11lingua}, \cite{michaud-jacques12nasalite},     \cite{jacques13arapaho}.
\item   \langue{Tonology}{Tonologie}: \cite{jacques11pumi.tone}, \cite{jacques16tonogenesis}.
\item \langue{Morphophonology}{Morphophonologie}: \cite{jacques12khaling}.
\item \langue{Reduplication}{Réduplication}:  \cite{jacques04redupl},  \cite{jacques07redupl}.
\end{enumerate}

\item \langue{Morphosyntax}{Morphosyntaxe}
\begin{enumerate}
\item  Incorporation: \cite{jacques11tangut.verb}, \cite{jacques12demotion}, \cite{jacques12incorp}.
\item  \langue{Voice}{Voix}:  \cite{jacques07passif}, \cite{jacques10refl}, \cite{jacques12demotion}, \cite{jacques13derivational.khaling}, \cite{jacques13tropative}, \cite{jacques14antipassive}, \cite{jacques15derivational.khaling}, \cite{jacques15causative},  \cite{jacques15spontaneous},  \cite{jacques16si}, \cite{jacques17generic}.
\item \langue{Hierarchical agreement}{Indexation hiérarchique}:  \cite{jacques10inverse},     \cite{jacques12khaling},   \cite{antonov14rtau}, \cite{jacques14inverse}, \cite{jacques14rtau}, \cite{jacques16th}, \cite{jacques17stau}.
\item \langue{Evidentiality}{Evidentialité}: \cite{jacques14auditory}.
\item \langue{Word-order typology}{Typologie de l'ordre des mots}: \cite{jacques13harmonization}.
\item \langue{Implicational universals}{Universels implicationnels}: \cite{antonov14need}.
\item \langue{Ideophones}{Idéophones}: \cite{japhug14ideophones}, \cite{jacques17ipa}.
\item \langue{Relativization}{Relatives}: \cite{jacques16relatives}.
\item \langue{Comparative clauses}{Comparatives}: \cite{jacques16comparative}.
\item \langue{Complementation}{Complétives}: \cite{jacques16complementation}.
\item \langue{Hybrid indirect speech}{Discours indirect hybride}: \cite{jacques16complementation}, \cite{jacques17stau}.
\item \langue{Noun phrase}{Syntagmes nominaux}: \cite{jacques17num}, \cite{jacques17sketch}.
\end{enumerate}

\item \langue{Formal linguistics}{Linguistique formelle}
\begin{enumerate}
\item \langue{Implemented morphology}{Morphologie implémentée}: \cite{walther14inv.canon}, \cite{walther14compactness}.
\end{enumerate}

\item \langue{Historical linguistics}{Linguistique historique}
\begin{enumerate}
\item \langue{Sound laws}{Lois phonétiques}:  \cite{jacques00ywij}, \cite{jacques01dg}, \cite{jacques03dissimilation},   \cite{jacques04thimphu}, \cite{jacques06comparaison}, \cite{jacques09wazur}, \cite{jacques09e}, \cite{michaud10bonin}, \cite{jacques10ndr}, \cite{jacques.michaud11naish},   \cite{rg-gj12yod}, \cite{jacques13arapaho}, \cite{jacques13yod}, \cite{jacques14snom},   \cite{jacques14esquisse}, \cite{jacques14cone}, \cite{jacques16ebde}, \cite{jacques16tonogenesis}, \cite{jacques17pkiranti}.
\item  \langue{Etymology}{Etymologie}:    \cite{jacques07naksatram}, \cite{jacques08debther}, \cite{jacques09zz}, \cite{jacques10imperial}, \cite{jacques11ngwemi}, \cite{jacques12bear},  \cite{jacques13vama},   \cite{jacques14esquisse}, \cite{jacques14honey}, \cite{jacques15sr}, \cite{jacques15cochon}, \cite{jacques17stau}, \cite{jacques17pkiranti}.
\item \langue{Historical morphology}{Morphologie historique}:  \cite{jacques03s.houzhui}, \cite{jacques06morpho}, \cite{jacques07chang},    \cite{jacques09tangutverb}, \cite{jacques10zos}, \cite{jacques11tangut.verb}, \cite{jacques12agreement}, \cite{jacques12internal}, \cite{jacques14antipassive},   \cite{jacques14esquisse}, \cite{jacques14cone}, \cite{jacques15derivational.khaling}, \cite{jacques15causative},  \cite{jacques15spontaneous}, \cite{jacques16ebde}, \cite{jacques16tonogenesis}, \cite{jacques16th}, \cite{jacques17stau}, \cite{jacques17num}, \cite{jacques17pkiranti}.
\item \langue{Grammaticalization:}{Grammaticalisation} \cite{jacques12internal},\cite{jacques13harmonization}, \cite{jacques14antipassive}, \cite{jacques14ergative}, \cite{jacques16comparative},  \cite{jacques17generic}, \cite{jacques16ssuffixes}.
\item \langue{Language contact:}{Contact de langues} \cite{antonov12kumush},   \cite{jacques12bear}, \cite{jacques14honey}.
\item \langue{Genetic linguistics}{Linguistique génétique}:  \cite{jacques07chang}, \cite{jacques12agreement}, \cite{jacques17genetic}, \cite{jacques17stau}.
\end{enumerate}
\item \langue{Ethnolinguistics}{Ethnolinguistique}
\begin{enumerate}
\item \langue{Kinship}{Parenté}: \cite{jacques11kinship}. 
\end{enumerate}
  \end{itemize}
  \item  \langue{Interdisciplinary}{Interdisciplinaire} 
\begin{itemize}
  \item \langue{Ethnobotany}{Ethnobotanique} :  \cite{Kang2016}
  \end{itemize}
\end{itemize}
\bibliographystyle{unified}
\bibliography{bibliogj}
\end{document}