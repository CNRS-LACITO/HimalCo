\documentclass[oldfontcommands,oneside,a4paper,11pt]{article} 
\usepackage{xunicode}%packages de base pour utiliser xetex
\usepackage{fontspec}
\usepackage{natbib}
\usepackage{booktabs}
\usepackage{xltxtra} 
\usepackage{longtable}
\usepackage{polyglossia} 
\usepackage[table]{xcolor}
\usepackage{gb4e} 
\usepackage{graphicx}
 \usepackage{setspace}
\usepackage{float}
\usepackage{hyperref} 
\hypersetup{bookmarks=false,bookmarksnumbered,bookmarksopenlevel=5,bookmarksdepth=5,xetex,colorlinks=true,linkcolor=blue,citecolor=blue}
\usepackage[all]{hypcap}
\usepackage{memhfixc}

\bibpunct[: ]{(}{)}{,}{a}{}{,}


\setmainfont[Mapping=tex-text,Numbers=OldStyle,Ligatures=Common]{Charis SIL} 
\setsansfont[Mapping=tex-text,Ligatures=Common,Mapping=tex-text,Ligatures=Common,Scale=MatchLowercase]{Lucida Sans Unicode} 
\SetSymbolFont{letters}{normal}{\encodingdefault}{\rmdefault}{m}{rm}
\setmathrm{Charis SIL}
\newcommand{\racine}[1]{\begin{math}\sqrt{#1}\end{math}} 
\newcommand{\forme}[1]{\textit{#1}} 
\newfontfamily\mleccha[Mapping=tex-text,Ligatures=Common,Scale=MatchLowercase]{Galatia SIL}%pour le grec
\newcommand{\grec}[1]{{\mleccha #1}}

\renewcommand \thesection {\arabic{section}.}
\renewcommand \thesubsection {\arabic{section}.\arabic{subsection}.}

\begin{document}
\noindent \textbf{\Large A neglected phonetic law: the assimilation of pretonic yod to a following coronal in North-West Semitic}\footnote{We wish to thank Nathan Hill, Benjamin Sass  as well as two anonymous reviewers for insightful comments and corrections.}


\noindent Romain Garnier, Université de Limoges 

\noindent and Guillaume Jacques, CNRS (Paris)

\noindent \textbf{Abstract}: 
\begin{sloppypar}
This paper shows the existence of a pretonic assimilation of *y to a following coronal consonant (including *y from proto-Semitic *y and *w) in North-West Semitic languages.  This rule, which has been obscured by analogy in each of the North-West Semitic languages, explains three independent sets of facts: the formation of irregular maqtal-s in Hebrew, Phoenician and Aramaic; the irregular conjugations of several verbs in Hebrew; the plural formation of the irregular noun ``house'' in Hebrew and Aramaic.  This proposal also solves the long-standing problem of the etymology of the verb ``to give'' in North-West Semitic languages (NTN in Hebrew vs. YTN in Phoenician).

\noindent  \textbf{Keywords}: 

Gemination, Assimilation, Coronal consonant, Hebrew, Phoenician, Aramaic
  
\pagebreak
\section{Introduction}


In Hebrew and other North-West Semitic languages, we observe clear traces of y (either from proto-Semitic *w- or *y) assimilating to a following consonant in a way similar to n, as previously noted by  \citet{huehnergard06}. In the present paper, we will study all available examples of y-assimilation in Hebrew, Phoenician and Aramaic, and propose the probable phonetic conditioning and time frame of this phonetic rule, which is no longer productive in any attested language.  

We will start this investigation by looking at several maqtal nouns from I-y roots which demonstrate  this assimilation. 

Second, we will study a series of I-y Hebrew verbs which not only have y-assimilation in derived nouns, but also in some imperfective forms. We will show that the Hebrew verbal root \racine{NTN} ‘to give’ is an innovation, and originally going back to a form *\racine{YTN} still attested in Phoenician: it was renewed on the basis the paradigms of I-n verbal roots. Finally, we will provide examples of that the same y-assimilation took place in Aramaic with the verbs ``to know'' \racine{YDʕ} < *\racine{wdʕ}, ``to sit'' \racine{YTB} < *\racine{wθb} and ``to blossom'' \racine{Yʕʔ} < *\racine{wɬ'ʕ}.\footnote{We represent the reconstruction of proto-Semitic consonants in IPA reconstruction: the consonant corresponding to Arabic ḍ is reconstructed as an ejective lateral alveolar fricative *ɬ', that corresponding to Arabic ẓ as an ejective interdental fricative *θ' and that corresponding to Hebrew ś as a lateral alveolar fricative *ɬ.}

	Third, we will show that the y-assimilation rule can be used to explain the irregular plural of \forme{bayit̠} ``house'' in Hebrew and Aramaic. This  example will also provide critical evidence to assess the exact conditioning factors for the hypothesized sound change.
	

\section{y-assimilation in maqtal deverbal nouns} \label{maqtal}

Maqtal deverbal nouns of I-y roots are normally formed according to the following pattern: \racine{yC_2C_3} > môC_2āC_3. For instance,  the root \racine{YŠB} < *\racine{wθb} ``to sit'' yields the regular maqtal \forme{môšāb̠} ``seat, above'' .


	This noun formation reflects the proto-Semitic *w initial before it changed to y- in Hebrew. The original form of this maqtal was *ma-wθab-(u). The *ma- prefix prevented initial *w from becoming y- as in perfect forms such as \forme{yāšab̠} < *waθaba ``he sat down'', and *aw monophthongized into long ô in Hebrew, hence *ma-wθab > *mawšab > *mōšab > \forme{môšāb̠}.
	
	
	Nevertheless, a few maqtal nouns from I-y verbal roots do not have this expected {môC_2āC_3} configuration, in particular \forme{maddāʕ} ``knowledge'' from \racine{YDʕ} ``to know'' and \forme{massāḏ} ``foundation'' from \racine{YSD} ``to establish''. Several other examples will be treated in the following section, but these two are non-controversial, as the corresponding verb roots have no traces of assimilation.
	
	
	Alongside these irregular deverbal nouns, the regular maqtal-s of these I-y roots are also attested: \forme{môsād̠} ``foundation'' and \forme{môd̠āʕ} ``parent''. While \forme{maddāʕ} is a relatively common noun, \forme{massāḏ} is considerably rarer than its regular equivalent \forme{môsād̠}. 

% meaning differences between massāḏ and môsād̠ ??
\begin{exe}
\ex \label{fondation}
	 û-mim-massaḏ ʕaḏ-haṭ-ṭǝp̄āḥōṯ  \\
 ``even from the foundation unto the coping'' (I Kings 7:9)
\end{exe}

The only way to explain these forms is to assume a phonetic change  *mayC_2aC_3 > *{maC_2C_2aC_3}   identical to the one present in I-n roots *manC_2aC_3 > *maC_2C_2aC_3. Alternatively, the change could have been *mawC_2aC_3 > *maC_2C_2aC_3, with the assimilation of w as proposed by \citet{huehnergard06}, but we will show in section 4 that some data cannot be accounted for by that hypothesis.

\section{y-assimilation in Hebrew and Aramaic verbal  conjugation} \label{yodtsade}
Evidence for this y-assimilation rule  is not limited to a few maqtal-s. Clear traces are also found in the conjugation of six I-yṣ verbs and one I-yS_2 verb:
%( reference to murakoa): 
\racine{YṢB} ``to take one’s stand”, \racine{YṢG} ``to set”, \racine{YṢʕ} ``to lay, spread”, \racine{YṢQ} ``to pour”, \racine{YṢR} ``to knead”, \racine{YṢT} ``to lighten” and \racine{YSR} ``to chastise''. The most common I-yṣ verb, however, \racine{YṢʔ} ``to go out, to depart”, shows no such assimilation in Hebrew. \citet[185]{jouon06} posit alternating I-n roots to account for these assimilations. However, comparative evidence does not support this hypothesis. 


	In this section, we will present attested forms of each of these seven verbs to illustrate y-assimilation. These verbs will be divided into three groups: first, \racine{YṢT} and \racine{YṢG}, which have no external cognates; second, \racine{YṢQ} and  and \racine{YSR}, which have cognates among North-West Semitic languages; third, \racine{YṢʕ}, \racine{YṢR} and \racine{YṢB}, which have cognates outside North-West Semitic, and whose initial I-y comes from proto-Semitic *w-. These data are well known from Hebrew grammars, but it is nevertheless important to set out the facts clearly, as we will see concerning the root \racine{YṢB}.


Finally, we will show that the Hebrew root \racine{NTN} belongs in fact to the group of verbs presented in this section: it comes from an earlier *\racine{ytn}, a form still attested in Phoenician.

\subsection{YṢT and YṢG } \label{yst}

The roots \racine{YṢT} and \racine{YṢG} have no known cognate outside of Hebrew, so  we have no way of knowing whether their initial I-y comes from proto-Semitic *y or *w. 
 

  \racine{YṢT} ``to lighten, to burn, to catch fire'' is attested in three forms: qal (for instance the 3sg. fem. waw-impf. \forme{wattiṣṣat̠}), nip̠ʕal (3pl. masc. perf. \forme{niṣṣət̠ū}) and hip̠ʕîl (2pl. masc. impf. \forme{taṣṣît̠ū}). The expected forms of a regular I-y verb, such as hip̠ʕîl, *hôṣît̠ or *hēṣît̠, are not attested.


   \racine{YṢG}  ``to set'' has hip̠ʕîl (3pl. masc. waw-impf. \forme{wayyaṣṣîg̠û}) and hop̠ʕal (3sg; masc. impf. \forme{yuṣṣāg̠}) forms. The regular forms *hôṣîg̠/*hēṣig̠ are not attested.
\begin{exe}
\ex \label{placer}
	 wayyiqəḥû pəlištîm ʔet̠ ʔărôn hāʔĕlōhîm wayyāb̠îʔû ʔōt̠ô b̠ēyt̠ Dāg̠ôn wayyaṣṣîg̠û ʔōt̠ô ʔēṣel Dāg̠ôn  \\
``When the Philistines took the ark of God, they brought it into the house of Dagon, and set it by Dagon.'' (I Samuel 5:2)
\end{exe}
 

\subsection{YṢQ and YSR } \label{ysq}
The roots  \racine{YṢQ} and \racine{YSR} are attested in other North-West Semitic languages (Phoenician and Ugaritic), but since these languages share the innovation *w- > *y-, we have no way of knowing whether these roots were *w-initial or *y-initial in the proto-language.
 

  \racine{YṢQ}  ``to pour” has a  Ugaritic cognate <YṢQ>. This root is attested in qal, hip̠ʕîl and hop̠ʕal, but unlike the previous roots, it has both y-assimilating and regular forms. In the qal, we have both  the imperfective form \forme{ʔeṣṣōq} with assimilation (see example \ref{verser}) and the regular waw-imperfective \forme{wayyīṣeq} without assimilation (example \ref{verser2}). 

\begin{exe}
\ex \label{verser}
	 ʔeṣṣōq rûḥî ʕal zarʕek̠ā   \\
I will pour my spirit upon thy seed and my blessing upon thine offspring.  (Isaiah 44:3)
\end{exe}

\begin{exe}
\ex \label{verser2}
	 wayyīṣeq dam=hammakkāh ʔel ḥēyq hārāk̠eb̠ \\
And the blood  ran out  of the wound  into  the midst  of the chariot. (I Kings 22:35)
\end{exe}
In the hip̠ʕîl, we find the waw-imperfective \forme{wayyaṣṣīqû} with assimilation of yôd̠, but the infinitive \forme{môṣāqet̠} (II Kings, 4:5) shows no assimilation. Finally, in the hop̠ʕal, only regular forms are found: perfective \forme{hûṣaq}, imperfective \forme{yûṣaq}.


  \racine{YSR} ``to chastise'' has a cognate  D-stem form in Ugaritic <YWSRNN>, with geminated initial I-w (\citealt[459, n. 9]{huehnergard06}). In Hebrew, it shows gemination in some qal forms such as \forme{ʔessŏrēm} “I will chastise them” (Hosea 10:10). It is the only II-s verb to do so.

 

\subsection{YṢʕ, YṢB and YṢR } \label{ysb}
The roots \racine{YṢʕ} ``to spread'' and \racine{YṢB} ``to take one's stand'' both have Arabic cognates, respectively \forme{waḍaʕa} ``he laid down'' and \forme{waṣaba} ``he was firm'', from proto-Semitic \racine{wɬ'ʕ} and \racine{ws'b}. In these two roots, the assimilating yôd̠ comes from an older *w (\citealt[460]{huehnergard06}). As for \racine{YṢR} ``to form'', comparative evidence is ambiguous.
 
   \racine{YṢʕ} ``to spread'' is only attested in hip̠ʕîl (3sg. masc. impf. \forme{yaṣṣî^aʕ}) and in hop̠ʕal (3sg. masc. impf. \forme{yuṣṣaʕ}). Only forms with y-assimilation are found. This root has a maqtal deverbal noun \forme{maṣṣāʕ} ``couch, bed” which belongs to the same category as the two examples presented in example \ref{maqtal}.

 \racine{YṢB} ``to station oneself, take one's stand''\footnote{Hebrew \racine{YṢB} is not to be compared with the root \racine{NṢB} ``to erect'' (reflected by Arabic \forme{naṣaba, yanṣubu} ``he set up, he erected''), whence Ugaritic\racine{NṢB} ``to erect'' (<SKN> ``a stele''), Hebrew nip̠ʕal 3sg. masc. perf. \forme{niṣṣaḇ} <*na-NṢáB-a) and \forme{maṣṣēḇāh} ``stele'' (= phoen. <MṢBT>, neo-Pun. <MNṢBT>), pointing to *ma-NṢiB-atu- (\citealt[128]{krahmalov01phoenician}).} is attested only in the hit̠paʕēl (3sg. masc. impf. \forme{yit̠yaṣṣēb̠}). There is no evidence of y-assimilation in the verbal conjugation of this verb, since I-C is always prevocalic in the paradigm of the hit̠paʕēl. However, this verb has a derived maqtal \forme{maṣṣāḇ} ``place, military post'', whose exact meaning can be illustrated by the following example:\footnote{In the sentence following this passage (I Samuel 14:12) \forme{ʔanǝšēy ham-maṣṣāḇâ} ``the men of the garrison'', the word \forme{maṣṣāḇ} is likely to have been a glotta, being mistaken for a proper name in the Septuaginta, which renders \forme{ʔanǝšēy ham-maṣṣāḇâ} by \grec{οἱ ἄνδρες Μεσσαβ} ``the men of Messab'' (the Vulgate correctly reads \textit{uirī dē statiōne} ``the men of the garrison''). }
\begin{exe}
\ex \label{place}
	 wa-yyiggālû šǝnēyhem ʔel maṣṣaḇ pəlištîm \\
and both of them (Jonathan and his armour-bearer) appeared to [the men] of the garrison of the Philistines (I Samuel 14:11)
\end{exe}

 \racine{YṢR} ``knead, make (as a potter)'' has cognates in Ugaritic and Phoenician: the qāṭil of the root (written <YṢR>) is attested in the sense of ``potter'' in these two languages. The corresponding Akkadian cognate \forme{eṣērum} would suggest a I-y root, but other languages such as Eblaite reflect I-w (\citealt[459, n. 8]{huehnergard06}).

This root mainly has forms without assimilation, such as nip̠ʕal \forme{nōṣar} and  hop̠ʕal \forme{yûṣar} and qal waw-imperfective 3sg. masc. \forme{wayyīṣer}.

Forms showing y-assimilation are only found in the qal imperfective with suffixed pronouns, such as \forme{ʔeṣṣārək̠ā}:
\begin{exe}
\ex \label{fingere}
	 bə-ṭerem ʔeṣṣārək̠ā (ʔṢWRK) b̠abbeṭen yəd̠aʕtîk̠ā \\
``Before I formed thee in the belly I knew thee.'' (Jeremiah 1:5)
\end{exe}

%
In example \ref{fingere}, the  <W> in the spelling <ʔṢWRK> (for expected <ʔṢR-K>) probably transcribes the stem vowel, suggesting perhaps an alternative pronunciation */ʔeṣṣôrək̠ā/.
 

The data from these seven verbs are summarized in the following table. Verb forms without assimilations are indicated between brackets.



\begin{table}[H]

\begin{tabular}{lllllll} \toprule
Root&	Meaning&	qal&	nip̠ʕal&	hip̠ʕîl&	hop̠ʕal&	yit̠paʕēl\\
\midrule
\racine{YṢG}&	to set&	&	&	wayyaṣṣîg̠û&	yuṣṣāg̠&	\\
\racine{YṢR}&	to knead&	(wayyīṣer)&	(nōṣar)&	&	(yūṣar)&	\\
&	&	ʔeṣṣārək̠ā&	&	&	&	\\
\racine{YṢT}&	to lighten&	wattiṣṣat̠&	niṣṣət̠ū&	taṣṣît̠û&	&	\\
\racine{YṢʕ}&	to lay&	&	&	yaṣṣî^aʕ&	yuṣṣaʕ&	\\
\racine{YṢB}&	to station&	&	&	&	&	(yit̠yaṣṣēb̠)\\
\racine{YṢQ}&	to pour&	(wayyīṣeq)&	&	wayyaṣṣîqû&	(yûṣaq)&	\\
&	&	ʔeṣṣōq&	\\
\racine{YSR} & to chastise & ʔessŏrēm &	&		(ʔayəsîrēm)& \\
\bottomrule
\end{tabular}
\caption{y-assimilating verbs in Hebrew} \label{tab:yassimilatingverbs}
\end{table}

Most forms without assimilation are  analogical, as were maqtal-s of the form {môC_2āC_3} discussed in the previous section. However, we will show   in section 4 that the waw-imperfective \forme{wayyīṣer} and  \forme{wayyīṣeq} are most probably inherited forms, and that the absence of assimilation here is due to a constraint on the application of the rule.

\subsection{NTN ``to give''} \label{ntn}
Hebrew \racine{NTN} seems at first glance to be entirely distinct from the seven verbs presented in this section. However, strong evidence suggests that this verb was y-assimilating at some stage of proto-Hebrew. 

The corresponding Phoenician cognate is \racine{YTN}. If we suppose that Phoenician preserved the proto-Cananean form while Hebrew innovated, it becomes possible to account for this irregular correspondence I-y::I-n. Attested Phoenician forms of the verb \racine{YTN} are summarized in Table \ref{tab:donner}.\footnote{The abbreviations CIL and CIS respectively stand for \textit{Corpus Inscriptionum Latinarum} and \textit{Corpus Inscriptionum Semitarum} (Pars prima Inscriptiones Phœnicias continens). }

In Hebrew, the form \racine{YTN} is reflected in one personal name \forme{Yaṯnīʔēl} (1 Chronicles 26:2) and the place name \forme{Yiṯnān} (Josuah 15:23). These names either represent preservation from an earlier stage of Hebrew or borrowing from a Cananean language preserving the older root.


\begin{table}

  \smallskip
  \begin{minipage}{12 cm}
  	\scriptsize
\begin{tabular}{lllll} \toprule
Class&	Testimonia&	Phœn. restitution&	Can. etymon&	Heb. parallel\\
\midrule
maqtal&	ma-ta-an-ba-ʕ-al\footnote{Annals of Assarhaddon 5^6^0 (Reign : 680—669 \textsc{bc}). Data apud \citet[89]{friedrich51punisch}.} &	*mattōn ``gift''&	*ma-WTáN-u-&	PN \forme{mattān}\footnote{Name of a priest of Baal (II Kings 11:18).}\\
&	Neo-Pun. m‘t‘b‘l  &	CS *mattan-baʕal&	``gift''&	 \forme{mattan-yāhû} \footnote{ Levite name (Chronicles I, 25^4). Compare with \forme{Nǝṯanyāhû} ``YHWH has given''), another Levite name (II Chronicles 17:8). Note also the shortened by-forms \forme{Mattanyāh} ``gift of YHWH'' (II King 24:17) and \forme{Nǝṯanyāh} (II Kings  25:23).} \\


miqtal\footnote{As pointed out by an anonymous reviewer, we do not find any gemination in the Latin transcriptions of miqtal; this fact is unexplained.}&	Mitun,\footnote{CIL 8, 27527.} Metun\footnote{CIL 8, 20492.}  &	*mittōn ``gift''&	*mi-WTáN-u-&	Ø\\
&	Metunilim\footnote{CIL 8, 12322. Properly ``given by the gods''.} &	Juxt. *mittōn+ilīm&	``gift''&	\\



maqtil&	\grec{Ματτήν},\footnote{Herodotus, VII, 9. Personal name of a Tyrian leading a ship among the Persian fleet (\grec{Τύριος Ματτὴν Εἰρώμου} ``Mattḗn the Tyrian, son of ʔaḥīrōm''). Note the accent on the final syllable.} &	*mattḗn ``gift''&	*ma-WTíN-u-&	Ø\\
&	ma-ti-nu-ba-ʕ-li\footnote{Annals of Salmanazar 2:93 (Reign : 727—722 \textsc{bc}). Compare with the seventh century PN *Mattanbaʕl.} &	Juxt. *mattinu+baʕli&	``gift''&	\\



miqtil&	mi-e-te-en-na\footnote{Annals of Tiglath-pileser III 67:66 (he conquered Phoenicia from 743 to 738).} &	*mittḗn ``gift''&	*mi-WTíN-u-&	Ø\\
&	\grec{Μεττηνος}\footnote{Flavius Josephus, Contra Apionem 1, 124. King of Tyre, son of \grec{Βαλεζωρος} (*Baʕl 'azōr ``Baal helped me''). His reign was from 850 to 821 \textsc{bc}. In the ninth century, his name was perhaps still something like *Mittínu rather than *Mittḗn.} &	&	``gift''&	\\


muqtal&	\grec{Μυττυνος},\footnote{Judge (i.e. suffet) of Tyre (Jos., Ap. 1, 157).} \grec{Μοττονης}\footnote{\citealt[585, 86]{dittenberger1915}.}  &	*muttōn ``given''&	*mu-WTáN-u-&	\\
&	MUT(H)UN\footnote{CIL 8, 8714. Compare Mutto (Just. 184). Note also the Punic PN MUT(H)UNBAL (CIL 8. 68, 16726)  and MUTHUNILIM ``god(s)-given'' (CIL 8, 23904), reflected by the Latin PN \forme{Ādeōdatus} (son of St. Augustine, who died at 19).  \citet[85]{segert76punic} explains this form as a  maqtūl *ma-WTūN-u-} &	&	``given'' &	Ø\\

&	*\grec{Μιλκιιαθων}&	*Milk(u) yatōn&	*X YáTaN-a&	PN \forme{yô-nāṯān}\\
Deus dedit&	(Rhod. gen. sg. &	``the god Milk &	(Proto-Phœn. &	``YHWH has given''\\
&	mi-li-ki-ya-to-no-se)\footnote{CIS 1, 10.2.  See also \citet[66a, 78c, 132b and 193b]{friedrich51punisch}.} &	has given''&	*YaTáN)&	\\


&	&	&	&	\forme{nǝṯan-ʔēl}\footnote{From Proto-Hebr. *natana-ʔil(u) ``the (bull-)god ’ilu has given''.} \\
dedit Deus&	Ø&	Ø&	*YaTaN-a X&	\forme{nǝṯan-yāhû}\footnote{From Proto-Hebr. *natana-YHWH ``YHWH has given''.} \\
&	&	&	&	\forme{nǝṯan-meleḵ}\footnote{Maybe reflecting *proto-Hebr. *natana-Milk(u) ``the god Milku has given'', with a Massoretic trivialization of the second part of the compound, no longer understood as a theophoric PN.} \\
\bottomrule
  \end{tabular}\par
   \vspace{-0.75\skip\footins}
   \renewcommand{\footnoterule}{}
    \end{minipage}
      \caption{Nominal forms of the verb ``to give'' in Phoenician and Hebrew}\label{tab:donner}
\end{table}
 \normalsize
Outside of Hebrew and Phoenician, this root is also attested in Ugaritic as \racine{YTN}, a fact that confirms the antiquity of I-y in this root.  The spelling <YTT> for the first person singular perfective can only be interpreted as *yatattu according to \citet[I:69]{bordreuil04}, a form deriving from earlier *yatan-tu.



An alternative hypothesis is mentioned by \citet[469-1, fn. 57]{huehnergard06}, according to which Ugaritic and Phoenician innovated the y-initial form. In this theory,  imperative \forme{tēn} < *tin served as the pivot form: for both I-y and I-n, the first radical disappears in the imperative (\forme{gaš} from \racine{NGŠ} ``to get closer'' vs. \forme{šēb} from \racine{YŠB} ``to sit down''). This hypothesis, however, would imply that the innovation occurred independently in Ugaritic and Phoenician, and is at odds with the fact that traces of the form \racine{YTN} can be found in Hebrew. The Akkadian form \forme{nadānum}, though probably cognate to Hebrew \racine{NTN}, presents an unexplainable second radical II-d which cannot in any way correspond to Hebrew and Phoenician II-t. Besides, Assyrian \forme{tadānum} (\citealt[603]{huehnergard97}) has no initial n--. It seems  that this root underwent major refection in Akkadian dialects: analogical change from I-w to I-t is well attested in Akkadian (\citealt[464]{huehnergard06}). The Akkadian form cannot be used a proof that the I-n in Hebrew is original. We suggest a reconstruction *\racine{wtn} for this root in proto-Semitic: it would account for all the data except the II-d in Akkadian.

Finally, since assimilation of the first radical consonant in I-y verbs is much rarer than in I-n verbs, where it is fully regular, analogy can only have taken place from I-y to I-n, not the other way round.

\subsection{y-assimilation in Aramaic verbal conjugation} \label{aramaic}
The assimilation of y- before coronals is not a phenomenon limited to Hebrew; other North-West Semitic languages show traces of it. Unfortunately, for Phoenician and Ugaritic, the absence of vocalization and gemination in the writing system make it impossible to determine with confidence whether or not such a phonetic change took place. However, in the case of Biblical Aramaic and Syriac, we are fortunate to have fully adequate writing systems.


 In Aramaic, three verbs show traces of y-assimilation: \racine{YDʕ} ``to know'', \racine{YTB} ``to sit'' (<*\racine{wθb}) and \racine{Yʕʔ} ``to bloom'' (<*\racine{wɬ'ʔ}. The conjugation of the first two verbs is well documented in all grammars of Biblical Aramaic (see for instance \citealt[73]{rosenthal88arameen}).\footnote{The verb ``to be able'' \racine{YKL} is often cited with these two verbs, as gemination is found in the imperfective \forme{yikkul}. However, gemination in this verb has a different origin, see \citet[471]{huehnergard06}.} \racine{YTB} has the imperfective form \forme{yittib̠}, which presents a clear case of y-assimilation: 
 
 \begin{exe}
\ex \label{asseoir}	
\glt \forme{yittib̠} <*yaθθib < *yayθib < *yawθib-u 
\end{exe}
The case of  \racine{YDʕ} ``to know'' is slightly more complex, since its imperfective (3sg. fem.)  is \forme{tindaʕ}, instead of expected *tiddaʕ if y-assimilation had occurred. We propose here that the geminated *d was dissimilated to a cluster *nd, a phonetic rule that has left many other traces in Aramaic (\citealt[83]{davidson1848}):
 \begin{exe}
\ex \label{savoir}	 \forme{*tindaʕ} <*tandaʕ < *taddaʕ <* taydaʕ 
\end{exe}
	The root \racine{Yʕʔ} ``to bloom'' presents an even more complex evolution. Targum Aramaic <YNʕY> \forme{yinʕēʔ} is the imperfective 3sg. masc. of the verb \forme{yǝʕaʔ} meaning ``to bloom''. It is found in the Onkelos Targum, where it glosses Hebrew \racine{PRḤ} ``to grow sprouts'' or \racine{ṢWṢ} ``bloom'' (\citealt[583]{jastrow1903targum}). The perfective form \forme{yəʕaʔ} goes back to a Common-Semitic protoform *waɬ'aʔ-a ``he went out'' (Ge'ez \forme{waḍaʔa}, Hebrew \forme{yāṣāʔ}). The meaning ``to grow sprouts'' is found in Akkadian (w)aṣûm (<*waɬ'āʔ-u-) ``to go out, to grow, to bloom''. 

	The imperfective form <YNʕY> \forme{yinʕēʔ} is extremely irregular; dictionaries set a distinct root \racine{Nʕʔ} alternating with \racine{Yʕʔ}. We propose a different solution, which involves y-assimilation like the two previous verbs: *yawɬ'iʔ-u > *yayɬ'iʔ > *yaɬ'ɬ'iʔ
\begin{exe}
\ex \label{bourgeonner1}	*yaɬ'ɬ'iʔ < *yayɬ'iʔ < *yawɬ'iʔ-u
\end{exe}

Assimilation took place before the regular Aramaic change *ɬ' > ʕ, when the place of articulation of this consonant was still coronal. After this assimilation, a dissimilation occurred, exactly as with \racine{YDʕ} ``to know''.
\begin{exe}
\ex \label{bourgeonner2}	*yanḍiʔ < *yaḍḍiʔ <  *yaɬ'ɬ'iʔ 
\end{exe}
 This dissimilation took place at an intermediate stage of change, when the consonant coming from proto-Semitic *ɬ' was still a coronal, but had become voiced: *ɬ' changed to ʕ through a voiced pharyngealized stop transcribed here as *ḍ (its exact pronunciation is difficult to ascertain). Then, the regular vowel changes applied, yielding the attested form \forme{yinʕēʔ} < *yanḍiʔ.


\subsection{Concluding remarks}

The Hebrew, Phoenician and Aramaic data reviewed in this section have shown that the cases of gemination in various verbal forms of I-y verbs is better explained as being due to  assimilation of y-- to the following consonant following the rule *VyCV > *VCCV. These data cannot decide whether assimilation took place before or after the change *w-- > *y--, so that they  would be compatible with Huehnergard's hypothesis that * VwCV > *VCCV (where C stands for a dental consonant).


In cases where cognate I-n and I-y roots are attested (such as Hebrew \racine{NTN}, Phoenician \racine{YTN}), the I-n form must be the analogical one, as gemination resulting from assimilation is regular in I-n verbs, whereas it is only residual in I-y verbs. 



\section{Bayit} \label{bayit}
The noun for ``house'' in Semitic (Hebrew \forme{báyiṯ}, Arabic \forme{baytu^n}, etc.) is notorious for its irregular paradigm, which has never been satisfactorily explained. However, we will show that the rule of assimilation illustrated by  verbal alternations in the previous sections can account for the Hebrew and Aramaic  data.

In Hebrew, the plural of \forme{báyiṯ} shows unexplained gemination \forme{bāttîm} (\citealt[294]{jouon06}). The same gemination is found in Aramaic dialects. In Biblical Aramaic, the attested plural is \forme{battê-ḵôn} < *battáy-kum (Daniel 2_5), and in Syriac, the singular and plural forms of this noun are \forme{bayt-ā} and \forme{battē} respectively. 

The singular form goes back to *báytu in proto-North West Semitic, hence Hebrew \forme{báyiṯ} in pausa with vowel fracture, but status constructus \forme{bêṯ}=, 1sg possessive \forme{bêṯ-î} from proto-Semitic *báyt-i-ya with monophthongization (--i-- being the Genitive case suffix, and --ya the 1sg possessive suffix).

The plural must be reconstructed as *batt-ū-ma in the Nominative and as *batt-ī-ma in the oblique cases, with status constructus *battáy= (Hebrew  \forme{bāttê-ḵem}, Biblical Aramaic \forme{battê-ḵôn} ``your^p houses''). 

\citet[294, fn. 4]{jouon06} suggests that Aramaic \forme{batt--} is due to the intervocalic syncope of --y--: Common Semitic *bayat- > proto-Cananean *bahat- > proto-Aramaic **baht- with compensatory gemination, but this ad hoc theory requires one to suppose a special phonetic rule which applied only to this word. Besides, it would not account in any way for the Hebrew form, and it is highly unlikely that Hebrew \forme{bāttîm} could be a borrowing from Aramaic.




The rule of assimilation presented in the previous section offers a simpler explanation: the geminate in the plural of this noun is due to the assimilation of *y to the following consonant:
\begin{exe} 
\ex *bayt-áy- > *batt-áy- (status constructus plural, Hebrew  \forme{battê}--) \\ \label{bayit.pl}
  *bayt-ī́ma > *batt-īma (status absolutus plural, Hebrew \forme{bāttîm}).
\end{exe}
This noun, however, allows us   further to refine the conditioning of the y-assimilation rule, as no gemination is found in  the singular:
\begin{exe} 
\ex *báyt-  (status constructus singular, Hebrew  \forme{bêṯ}--) \\ \label{bayit.sg}
 *báytu (status absolutus singular, Hebrew \forme{báyiṯ}). 
\end{exe}
The main difference between examples \ref{bayit.pl} and \ref{bayit.sg} is that in the former, the stressed syllable follows the postulated *-yt- cluster, while in the latter, the stressed syllable precedes it. This shows that y-assimilation only occurs in pretonic position (*-VyTV́- > *-VTTV-). 

No other CayC- noun shows the same alternation in any North-West Semitic language; however, this is probably due to the fact that less common nouns underwent analogy and the original geminated plural was replaced by a plural following a more regular pattern. As pointed out by an anonymous reviewer, the expected regular plural of \forme{báyiṯ} should be a broken plural *bayatīm > *bəyāṯîm.\footnote{Plurals built on the binyan QaTaL are very widespread in North-West Semitic, as in Hebrew \forme{meleḵ} < *málk-u- ``king'' vs. \forme{mǝlāḵîm} < *malak-īm ``kings''.} This is actually the form attested in Ugaritic.\footnote{In Ugaritic, the singular BT *bêtu comes from the same proto-form *báyt-u-  as Hebrew \forme{báyiṯ}, but the plural BHT-M ``the houses'' is not directly comparable to \forme{bāttîm}.
In BHT-M ``the houses'', the spelling --H-- probably represents a hiatus. \citet[34-5]{sivan01ugaritic} cites an alternative spelling BWT-M, and it is most likely that both BHT-M and BWT-M stand for a plural form *ba.at-ūma. This form would reflect an innovative broken plural *ba(y)at-u ``houses''. This broken plural, which originally probably had a collective meaning ``a group of houses'' or maybe ``the rooms (of the house)'',  would have superseded the original geminated plural *batt-ū-ma. }

This pattern is found with some other CayC nouns, such as \forme{ḥáyil}, plural \forme{ḥăyālîm} ``strength, army'’. However, we  also find simple plurals of the type *CayC-īm, such as \forme{zayt̠} ``olive'',  plural  \forme{zēyt̠îm} < *zaytīm ``olive trees'' (as in the place-name  \forme{har hazzēyt̠îm} ``Mount of Olives'').



The irregular plural of \forme{báyiṯ} constitutes important evidence for the rule of y-assimilation : it proves that this rule cannot have taken place before the change *w > *y, otherwise \forme{báyiṯ} would not have undergone assimilation, since the --y-- in this noun  goes back to proto-Semitic. Besides, it proves that the assimilation rule was conditioned by supra-segmental factors.

With this rule in mind, we are now in a position to explain the forms \forme{wayyī́ṣer} from  \racine{YṢR} ``to make'' and  \forme{wayyī́ṣeq} from \racine{YṢQ}  ``to pour” in section \ref{ysq} that show no assimilation of y--. The expected forms if y-assimilation had occurred in all VyCV contexts would have been *wayyíṣṣer and *wayyíṣṣeq on the model of I-n roots.

In these two waw-imperfectives, the stress falls on the personal prefix:
\begin{exe} 
\ex \forme{wayyī́ṣer}  < *wa-yá-yṣir \\
  \forme{wayyī́ṣeq} < *wa-yá-yṣiq 
\end{exe}
The absence of gemination here is expected given the accentual conditioning of y-assimilation: since the stressed syllable precedes the *--yC-- cluster, no assimilation takes place here as in example \ref{bayit.sg} above. 

By contrast, imperfective forms without waw have the stress on the radical, and undergo assimilation:
\begin{exe} 
\ex \forme{ʔeṣṣōq} < *ʔa-yṣúq
\end{exe}
The rule of y-assimilation can therefore not only explain various irregular paradigms, but also sheds some light on the reconstruction of the proto-North-West Semitic accentual system.

\section{Conclusion}
 This article has shown the existence of a rule involving the assimilation of y-- to a following consonant in North-West Semitic and set out its precise phonetic conditioning. Its clearest traces are found in verbal flexional and derivational morphology, but evidence is also found  in the peculiar flexion of the irregular noun ``house''. 
 
 The data presented here show that *y (either from proto-Semitic *w or *y) assimilates in pretonic position to a following coronal consonant, including proto-Semitic *t, *θ, *s, *d as well as the emphatic (or ejective) *s', *ɬ', *θ'. No traces of assimilation with other coronals such as *z, *n, *ð, *ɬ, *ʃ, *l and *t' have been found, but this may reflect a gap in our data rather than an original constraint on this phonetic rule, given the limited number of examples which have resisted analogy. Among the verbs preserving the y-assimilation rule, the important proportion of roots with Ṣ as a second root consonant in Hebrew probably reflects the fact that this consonant results from the merger of three proto-Semitic consonants *s', *ɬ' and *θ'.
 
The effect of this rule has been largely levelled by analogy in most North-West Semitic languages, and traces can only be detected in old derivations or irregular paradigms.

	 \citet{huehnergard06} has already  proposed  explaining the maqtal formations and some of the irregular verbs discussed in this paper by the assimilation of the first radical consonant. However, he argues for a much earlier time frame than we do: according to him, it goes back to proto-Semitic, and the assimilation of w- to a following t- in  Akkadian and Arabic (\citealt[I:177]{brockelmann}) would be traces of this rule.  In  our hypothesis,  the y-assimilation rule postdates the change *w > *y, and assimilation of *w to *t in proto-Semitic is an unrelated phenomenon.
	 
	 The hypothesis laid out in the present article has two advantages over Huehnergard's. First, in Arabic and Akkadian, assimilation only occurs before t, whereas in North-West Semitic, as we have seen, it occurs with most coronal consonants; Huehnergard argues that assimilation of w- to all dental consonants (not just to t-) is of proto-Semitic date, but it seems highly unlikely that no trace of this rule on dental consonants other than t-- would have been preserved in Arabic and Akkadian.
	 
	 Second, Huehnergard's hypothesis cannot account for the plural form of \forme{báyiṯ}, which would have to be analysed as an entirely unrelated fact.
	 
	
 \end{sloppypar}
\bibliographystyle{myenbib}
\bibliography{bibliogj}

\end{document}