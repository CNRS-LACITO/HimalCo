\documentclass[oneside,a4paper,11pt]{article} 
\usepackage{fontspec}
\usepackage{natbib}
\usepackage{booktabs}
\usepackage{xltxtra} 
\usepackage{polyglossia} 
\usepackage[table]{xcolor}
\usepackage{gb4e} 
\usepackage{multicol}
\usepackage{graphicx}
\usepackage{float}
\usepackage{hyperref} 
\hypersetup{bookmarksnumbered,bookmarksopenlevel=5,bookmarksdepth=5,colorlinks=true,linkcolor=blue,citecolor=blue}
\usepackage[all]{hypcap}
\usepackage{memhfixc}
\usepackage{lscape}

\setmainfont[Mapping=tex-text,Numbers=OldStyle,Ligatures=Common]{Charis SIL} 
\newfontfamily\phon[Mapping=tex-text,Ligatures=Common,Scale=MatchLowercase]{Charis SIL} 
\newcommand{\ipa}[1]{{\phon\mbox{\textbf{#1}}}}
\newcommand{\ipab}[1]{{\scriptsize \phon#1}} 

\newcommand{\grise}[1]{\cellcolor{lightgray}\textbf{#1}}
\newfontfamily\cn[Mapping=tex-text,Ligatures=Common,Scale=MatchUppercase]{SimSun}%pour le chinois
\newcommand{\zh}[1]{{\cn #1}}
\newcommand{\refb}[1]{(\ref{#1})}
\newcommand{\factual}[1]{\textsc{:fact}}
\newcommand{\rdp}{\textasciitilde{}}

\XeTeXlinebreaklocale 'zh' %使用中文换行
\XeTeXlinebreakskip = 0pt plus 1pt %
 %CIRCG
 \newcommand{\bleu}[1]{{\color{blue}#1}}
\newcommand{\rouge}[1]{{\color{red}#1}} 
\newcommand{\ch}[3]{\zh{#1} \ipa{#2} `#3'} 
\newcommand{\change}[2]{*\ipa{#1} $\rightarrow$ \ipa{#2}}
\newcommand{\deux}[1]{/#1/}
\newcommand{\trois}[1]{/#1/}

\newcommand{\tib}[1]{\cellcolor{lightgray}\textbf{#1}}
\newcommand{\idph}[1]{\cellcolor{gray}\textbf{#1}}



\begin{document} 
\title{On the reconstruction of medial *-r- in Old Chinese}
\author{Guillaume Jacques}
\maketitle
 
\sloppy
\section{Introduction}
All systems of Old Chinese reconstructions used  presently, from that of \citealt{li71oc} to \citet{schuessler09minimal} and \citet{bs14oc} all follow \citet{yakhontov61sochetaniya} in reconstructing [consonant+liquid] clusters in words whose Middle Chinese rhymes belong to second division or chongniu third division (as well as all other third division words after Middle Chinese retroflex initials). Yakhontov originally proposed to reconstruct a medial *\ipa{-l-} in these words, but later scholars have agreed that this medial consonant was more probably *\ipa{-r-}.

This reconstruction has attracted little controversy in recent years. One of the rare attempts at challenging the received model,  \citet{handel02r}, proposed that rhotic metathesis \change{rC-}{*Cr-}  took place at some stage of Old Chinese, on the basis of comparative evidence with Tibetan, but this work has until now receive little follow-up.

In the present paper, I argue that, while medial *\ipa{-r-} do certainly constitute one of the origins of  second division and chongniu third division, it is not necessarily the only origin of these Middle Chinese categories, and that mechanically projecting *\ipa{-r-} for all syllables of this type result in a highly unrealistic reconstruction.

In the first section, I present a series of arguments, based on typology and Sino-Tibetan comparison, against generalized reconstruction of *\ipa{-r-}.

Then, I show that reconstructing *\ipa{r-} preinitials, and possibly clusters not containing rhotic, can account for the development of rhymes and initials in Middle Chinese without supposing metathesis.

\section{What is wrong with medial *\ipa{-r-}?} \label{sec:medial}
Despite its widespread acceptance, the hypothesis that all words with second division and/or retroflex initials in Middle Chinese (henceforth R-type words), and part of chongniu 3 and rounded rhyme words, had a medial *\ipa{-r-} presents serious problems from the point of view of both Trans-Himalayan comparison and linguistic typology. The main problems involved with this reconstruction is the large excess of reconstructed *\ipa{-r-} in Old Chinese in comparison with more conservative languages of the Trans-Himalayan that preserve the clusters, the presence of a *\ipa{lr-} cluster and of dental + *\ipa{r}, and the rather poor correspondences with \ipa{-r-} clusters in other languages.

\subsection{Excess of *-r- in Old Chinese reconstructions}
The proportion of R-type syllables in Middle Chinese can be estimated from Table \ref{tab:gy}, obtained from the electronic version of the Guangyun. Listed as R-type syllables are second division, as well as all syllables with the initial consonants \zh{知} \ipa{ʈ}, \zh{徹} \ipa{ʈʰ},  \zh{澄} \ipa{ɖ},  \zh{娘} \ipa{ɳ},  \zh{莊} \ipa{tʂ},  \zh{初} \ipa{tʂʰ},  \zh{崇} \ipa{dʐ} and \zh{生} \ipa{ʂ} in third division. This total number strongly underestimates syllables with potential medial *\ipa{-r-} in Old Chinese. Many chongniu 3 syllables are also to be reconstructed with *\ipa{-r-} (\citealt[217-8]{bs14oc}), but the presence of the medial is unambiguous only with *\ipa{e}.  In addition, *\ipa{-r-} medial is undetectable in words with rounded rhymes such as \zh{侯} \ipa{-uw} (\citealt[362]{starostin89}, \citealt[501]{baxter92}).

\begin{table}
\caption{Proportion of R-type syllables in the Guangyun} \label{tab:gy} \centering
\begin{tabular}{lllll}
\toprule
Type& Nb of characters & Nb of distinct syllables \\
\midrule
Second division	& 3126	 & 619 \\
%Chongniu 3	& 1997	& 420 \\
Retroflex initials & 1592	&335 \\%930	& 187 \\
\midrule
%All &	6053	&1226\\
Total& 4718	&1094 \\
out of &	25300	&3883 \\
	&19\%	&28\% \\
	\bottomrule
\end{tabular}
\end{table}

If all chongniu 3 syllables are included in the count, the total number is 6053 syllables (about 24\%; still excluding syllables with rounded initials). Therefore, it is safe to assume that in all modern systems of Old Chinese reconstruction, the total number of syllable reconstructed with medial *\ipa{-r-} is between 19\% and a figure over 24\%.

%知	136	30 
%徹	150	33
%澄	190	31
%娘	64	13
%莊	101	18
%初	77	20
%崇	51	20
%生	161	22
% en comptant les chongniu:
%知	 279 56
%徹	 256 60
%澄	 373 62
%娘	 126 29
%莊	 127 27
%初	 107 31
%崇	 85 29
%生	 239 41

This figure can be compared with a count of words reconstructed medial *\ipa{-r-} in an actual reconstruction system of Old Chinese for which a searchable database in available. In the list of reconstructions provided online as a companion to \citet{bs14oc}, 1070 words out of 4974 are reconstructed with a non-ambiguous medial *\ipa{-r-} (21.5 \%), indeed within the range estimated an the basis of Middle Chinese alone.

\subsection{Clusters with r in Trans-Himalayan languages other than Chinese}
In order to better evaluate the plausibility of the system of clusters with medial *\ipa{-r-} reconstructed for Old Chinese, it is advisable to observe the distribute of \ipa{r} clusters in  with cluster-rich languages of the Trans-Himalayan family is necessary. In this work, I focus on two languages for which electronic dictionaries are available, Japhug and Classical Tibetan.


In Japhug, a Rgyalrongic language, we find 411 clusters, including 64 clusters with \ipa{r} as final element (Tables \ref{tab:Cr} and \ref{tab:CCr}),\footnote{In these clusters \ipa{r} may be either medial or initial consonant. Reduplication patterns can discriminate between the two (\citealt{jacques07redupl}).} and 47 clusters with \ipa{r} as a non-final element (Table \ref{tab:rC}): 26\% of all clusters contain an \ipa{r}. 


\begin{table}[H]
\caption{List of Cr- consonant clusters in Japhug} \label{tab:Cr} \centering
\begin{tabular}{llll}
\toprule
Cluster &Example &Meaning \\
\midrule
   \deux{br} \idph{}  	&   \ipa{brɯbrɯz}   	&   having pimples \\
  \deux{cʰr}\idph{}   	&   \ipa{cʰrɤβcʰrɤβ}   	&   messy and dirty \\
  \deux{cr} \idph{}  	&   \ipa{crɯɣcrɯɣ}   	&   in a mess  \\
  \deux{ɕr}   	&   \ipa{ɕri}   	&   it leaks \\
  \deux{dr} \idph{}  	&   \ipa{droŋdroŋ}   	&   big and dirty \\
  \deux{gr}   	&   \ipa{grɯβgrɯβ}   	&  matsutake  \\
  \deux{ɣr}   	&   \ipa{ɣro}   	&   he suffocates \\
  \deux{jr}   	&   \ipa{ɯ-jroʁ}   	&  its furrow  \\
  \deux{ɟr} \idph{}  	&   \ipa{ɟrɯɣɟrɯɣ}   	&   gurgling \\
  \deux{kʰr}   	&   \ipa{kʰro}   	&  much  \\
  \deux{kr}   	&   \ipa{krɤɣ}   	&   he cuts/mows it \\
  \deux{mbr}   	&   \ipa{mbrɤt}   	&  it breaks  \\
  \deux{ndr}   	&   \ipa{qomndroŋ}   	&   wild geese \\
  \deux{ndzr}   	&   \ipa{ndzri}   	&   he wrings it \\
  \deux{ɴɢr}   	&   \ipa{ɴɢraʁ}   	&  it is torn  \\
  \deux{ŋgr}   	&   \ipa{ŋgrɤl}   	&  it is usually the case  \\
  \deux{pʰr}   	&   \ipa{kʰɤpʰrɯ}   	&   spraying water with the mouth \\
  \deux{qr}   	&   \ipa{qro}   	&   pigeon \\
  \deux{ʁr}   	&   \ipa{ʁrɯlu}   	&   without horns \\
  \deux{sr}   	&   \ipa{ɯ-srɯβ}   	&  its interstice  \\
  \deux{tɕr} \idph{}  	&   \ipa{tɕrɯɣnɤtɕrɯɣ}   	&   crunching \\
  \deux{tsr}   	&   \ipa{tsri}   	&   it is salty \\
  \deux{wr}   	&   \ipa{βraʁ}   	&  he attaches it  \\
  \deux{zr}   	&   \ipa{zrɯ}   	&  sunny side of the mountain  \\
  \deux{ʑr}   	&   \ipa{ʑru}   	&  it is strong  \\
  \bottomrule
  \end{tabular}  
\end{table}  
  
  
\begin{table}[H]
\caption{List of CCr- consonant clusters in Japhug} \label{tab:CCr} \centering
\begin{tabular}{llll}
\toprule
Cluster &Example &Meaning \\
\midrule
  \trois{xpr}    	&   \ipa{ta-ɣɤxpra}    	&  he sent him  \\
 \deux{pr}   	&   \ipa{pri}   	&   bear \\
 \trois{ɕkr}    	&   \ipa{ɕkrɤz}    	&  oak  \\
 \trois{ɕpr}    	&   \ipa{aɕprɯm}    	&   it is badly sewed \\
 \trois{ɕqr}    	&   \ipa{ɕqraʁ}    	&   he is intelligent  \\
 \trois{ɕtr}  \idph{}  	&   \ipa{ɕtraŋɕtraŋ}    	&   long and soft \\
 \trois{jkr}    	&   \ipa{jkrɯt}    	&   it will solidify  \\
 \trois{jtsr}    	&   \ipa{jtsraβ}    	&  he delays his departure  \\
 \trois{mgr}    	&   \ipa{mgrɯn}    	&   he receives him \\
 \trois{mkʰr}    	&   \ipa{mkʰroŋ}    	&  he will be reincarnated  \\
 \trois{mpʰr}    	&   \ipa{mpʰrɯmɯ}    	&   divination \\
 \trois{mtsr}    	&   \ipa{mɯmtsrɯɣ}    	&  he drinks it with a straw  \\
 \trois{nbr}    	&   \ipa{nbraʁ}    	&  he hoes it  \\
 \trois{ɴqr}    	&   \ipa{ɯ-ɴqra}    	&   shabby \\
 \trois{ɲcr} \idph{}   	&   \ipa{ɲcɯɲcri}    	&  thin, diluted  \\
 \trois{ŋkʰr}    	&   \ipa{ŋkʰrɯli}    	&  screw  \\
 \trois{ʁgr} \tib{}    	&   \ipa{ʁgra}    	&   enemy \\
 \trois{ʁɟr}  \idph{}  	&   \ipa{ʁɟɯʁɟri}    	&   fat and soft \\
 \trois{ʁmbr}    	&   \ipa{taʁmbra}    	&   crying and shouting \\
 \trois{ʁzr}    	&   \ipa{aʁzrɤwɤlu}    	&   dishevelled \\
 \trois{scr}  \idph{}  	&   \ipa{scraʁscraʁ}    	&   very small \\
 \trois{skʰr}    	&   \ipa{tɯ-skʰrɯ}    	&   body  \\
 \trois{skr}    	&   \ipa{skraskra}    	&   impolite  \\
 \trois{spr}    	&   \ipa{sprɯskɯ}    	&    reincarnated \\
 \trois{sqr}    	&   \ipa{sɤsqra}     	&  limit \\
 \trois{sthr}  \idph{}   	&   \ipa{stʰrɯβ}    	&   dangling (of snot)  \\
 \trois{wɣr}    	&  \ipa{wɣrum}    	&  it is white \\
 \trois{wkr}   \tib{}  	&   \ipa{fkrɯz}    	&  he is greedy  \\
 \trois{wsr}    	&  \ipa{wsraŋ}    	&  he protects it \\
 \trois{zbr}  \tib{}   	&   \ipa{zbrilu}    	&   year of the snake  \\
 \trois{zgr} \tib{}   	&   \ipa{zgrawa}      	&  leather sack \\
 \trois{zɟr} \idph{}   	&   \ipa{zɟraŋzɟraŋ}    	&   soft and bloated \\
 \trois{zmbr}    	&   \ipa{sɤzmbri}    	&   he makes him angry \\
 \trois{ʑdr}  \idph{}  	&   \ipa{ʑdraŋʑdraŋ}    	&    long and soft \\
 \trois{ʑgr}    	&   \ipa{ʑgrɯ}    	&  certainly  \\
 \trois{ʑmbr}    	&   \ipa{ʑmbri}    	&   willow \\
 \trois{ʑŋgr}    	&   \ipa{ʑŋgri}    	&  star  \\
 \trois{ʑɴɢr}  	& \ipa{ʑɴɢro}  	&Jew's harp \\
 \trois{χcr} \idph{}   	&   \ipa{χcɯχcri}    	&   thin, diluted  \\
 \trois{χpr}    	&   \ipa{tɕʰɯχpri}    	&   newt \\
 \trois{χsr}    	&   \ipa{ɣɤχsrɯ}    	&   handsome  \\
  \bottomrule
  \end{tabular}  
\end{table}  

\begin{table}[H]
\caption{List of consonant clusters with \ipa{r} as non-final element in Japhug} \label{tab:rC} \centering
\begin{tabular}{lllll}
\toprule
Cluster &Example &Meaning \\
\midrule
\deux{rp}  	&	 \ipa{tɯ-rpa}  	&	 axe 	\\
 \deux{rpʰ} \idph{}  	&	 \ipa{rpʰɤβrpʰɤβ}  	&	 flapping wings 	\\
 \deux{rmb}  	&	 \ipa{armbat}  	&	near  	\\
 \deux{rm}  	&	 \ipa{rmɤβja}  	&	peacock  	\\
 \deux{rt}  	&	 \ipa{rtalu}  	&	horse year  	\\
 \deux{rtʰ}  	&	 \ipa{ɯ-pɤrtʰɤβ}  	&	middle  	\\
 \deux{rd}  	&	 \ipa{rdɤstaʁ}  	&	 stone 	\\
 \deux{rnd}  	&	 \ipa{rnde}  	&	 he finds it 	\\
 \deux{rn}  	&	 \ipa{rnaʁ}  	&	it is deep  	\\
 \deux{rts}  	&	 \ipa{rtsot}  	&	vengeance  	\\
 \deux{rtsʰ}  	&	 \ipa{rtshom}  	&	 it has a crack (bucket) 	\\
 \deux{rdz} \idph{}  	&	 \ipa{rdzardza}  	&	 insolent 	\\
 \deux{rndz}  	&	 \ipa{rndzɤkɤŋe}  	&	 shade of the mountain 	\\
 \deux{rs} \idph{}  	&	 \ipa{rsɯβrsɯβ}  	&	 hairy 	\\
 \deux{rz}  	&	 \ipa{tɯ-rzɯɣ}  	&	 one section 	\\
 \deux{rtɕ}  	&	 \ipa{nɯrtɕe}  	&	he teases him  	\\
 \deux{rtɕʰ}  	&	 \ipa{rtɕʰɯʁjɯ}  	&	caterpillar  	\\
 \deux{rndʑ}  	&	 \ipa{cɯrndʑi}  	&	sand  	\\
 \deux{rɕ}  	&	 \ipa{rɕɯβrɕɯβ}  	&	rough 	\\
 \deux{rʑ}  	&	 \ipa{tɤ-rʑaβ}  	&	wife  	\\
 \deux{rc}  	&	 \ipa{tɤ-rcoʁ}  	&	 mud 	\\
 \deux{rcʰ}  	&	 \ipa{ɯ-rcʰarcʰɤβ}  	&	 interstice 	\\
 \deux{rɟ}  	&	 \ipa{rɟaʁ}  	&	he dances  	\\
 \deux{rɲɟ}  	&	 \ipa{rɲɟaʁlo}  	&	 bolt 	\\
 \deux{rɲ}  	&	 \ipa{rɲaŋ}  	&	 its is ancient 	\\
 \deux{rk}  	&	 \ipa{rko}  	&	 it is hard 	\\
 \deux{rkʰ}  	&	 \ipa{tɤ-rkʰom}  	&	 feather rachis 	\\
 \deux{rg}  	&	 \ipa{rga}  	&	 he likes it 	\\
 \deux{rŋg}  	&	 \ipa{rŋgɤm}  	&	hard piece  	\\
 \deux{rŋ}  	&	 \ipa{tɯ-rŋa}  	&	 face 	\\
 \deux{rq}  	&	 \ipa{rqoʁ}  	&	he hugs him  	\\
 \deux{rqʰ}  	&	 \ipa{tɤ-rqʰu}  	&	bark, skin 	\\
 \deux{rɴɢ}  	&	 \ipa{ɕɯrɴɢo}  	&	Anisodus tanguticus  	\\
 \deux{rl} 	& \ipa{rlaʁ} 	&it disappears \\
\deux{rw}\tib{} 	&\ipa{rwa} 	&yak felt tent	\\
\deux{rj} 	&\ipa{tɯ-rju} 	&word \\	
 \deux{rɣ}  	&  \ipa{tɯ-rɣi}  	&  seed	\\
\deux{rʁ}	&\ipa{rʁe}	& it puts it through\\	
	\trois{frt}  \tib{} &	\ipa{frtɤn}  &	he is trustworthy\\
	\trois{βrɟ}  \tib{} &	\ipa{βrɟaŋ}  &he stretches it (skin)	\\
	\trois{rɴɢl}  	& \ipa{arɴɢlɯm}  	&it is concave \\
 \trois{rmbɣ}  	&\ipa{tɤ-rmbɣo}  	&drum	\\
 \trois{rpɣ}  	&\ipa{rpɣo}  	&up on the mountain	\\
\trois{rpj} 		&\ipa{rpjɯ} 		&it is spoiled (milk)\\
\trois{rmbj} 		&\ipa{tɤ-rmbja} 		&flash of lightning\\
\trois{rtsj} 		&\ipa{rtsjaʁ} 		&it is steep (road)\\
\trois{rqʰj} 		&\ipa{ɯ-rqʰioʁ} 		&its notch\\
\trois{rɴɢj} 		&\ipa{arɤrɴɢioʁ} 		&having a notch\\
\bottomrule
\end{tabular}
\end{table}

Despite this richness in clusters, which is in part due to the prevalence of \ipa{r} clusters in ideophones, we only find 472 words with medial \ipa{-r} in their last syllable\footnote{Since Japhug is a strongly prefixing language (\citealt{jacques13harmonization}), and given the fact that prefixes have a restricted phonological inventory, the last syllable of the word is nearly always the one with the richest clusters, except for a few compound nouns.} out of 6829 (7\%) in the Japhug dictionary (\citealt{jacques15japhug}), and 645 words with preinitial \ipa{r-} (9\%), in total 1117 (16\%).


Clusters in Classical Tibetan reveal a similar distribution. In \citet{bodrgya}, out of 53922 words, there are 4640 words with (C)Cr- clusters (9\%), and 4516 words with (C)rC(C) clusters (8\%), in total 9156 words (17\%). 
%Cr- 4640 %\%[^aeiou ]+r[aeiou ]
%rC- 4516 %\%[^aeiou ]*r[^aeiou ]
%53922

The proportion of words with medial \ipa{r} in Japhug and Tibetan is thus a third of the postulated amount of medial \ipa{r} in Baxter and Sagart's (\citeyear{bs14oc}) reconstruction, a figure which, as we have seen, would differ little in other reconstruction systems. Even putting together words with medial and preinitial \ipa{r}, we only reach 16\% or 17\%, less than the minimal proportion of words in medial \ipa{r} in all systems (19\%, see Table \ref{tab:gy} above). 

While Gyalrongic and Tibetic languages cannot serve as a direct model for Old Chinese, they reveal that in the present models of Old Chinese reconstruction medial *\ipa{-r-} are considerably more common that in languages where clusters are directly observable without any reconstruction.

\subsection{Laterals}
Various authors, including \citet{schuessler09minimal} and \citet{bs14oc}, reconstruct clusters such as *\ipa{lr} and *\ipa{l̥r} to account for the presence of the initials \zh{澄} \ipa{ɖ} and \zh{徹} \ipa{ʈʰ} respectively. This view is not universally accepted: \citet[217-225]{starostin89} in particular interpret these as lateral affricates, a solution which however raises problems with Chinese-internal evidence (\citealt[36-40]{sagart99roc}).


Reconstructing *\ipa{lr} and *\ipa{l̥r} in Old Chinese makes the system typologically highly unusual. In the Trans-Himalayan family, even in cluster-rich languages such as Japhug and Old Tibetan, only \ipa{rl} is attested, never $\dagger$\ipa{lr}.

Outside of the Trans-Himalayan family, clusters such as \ipa{lr} are very rare, and have even been argued not to exist (\citealt[78]{baroni14invariant}). While some examples of surface phonetic \ipa{lr} can be found, for instance in the Mande language Wan (see the data in \citealt{nikitina12logophoric}), no example of \ipa{l̥r} is known to me.


\subsection{Dental + r}
Clusters with dental stops and affricates followed by *\ipa{-r-}, while extremely common in reconstruction of Old Chinese, are unusual in the rest of the Trans-Himalayan family.

The only languages in the family to have both clusters such as \ipa{tsr} and \ipa{tr} in Trans-Himalayan are Gyalrongic languages. In Japhug, examples of these clusters include \ipa{jtsraβ} `delay departure' and \ipa{adrɤt} `in disorder' respectively, contrasting with retroflex affricates such as \ipa{tʂ}.

Yet, there is evidence from morphological alternations that some retroflex affricates in Japhug originate from dental stop + \ipa{r} clusters. The clearest examples are the numeral sixteen \ipa{sqa-p-rɤɣ} alternating with \ipa{kɯ-tʂɤɣ} `six' and the noun \ipa{ftɕɤru} `path to walk through a field during summer', a compound from \ipa{ftɕar} `summer' and \ipa{tʂu} `path, road'. Both of these examples implies that the proto-Gyalrong cluster *\ipa{tr} became \ipa{tʂ} in Japhug, an idea which requires that dental + r clusters in Japhug are secondary. Since most examples of dental+\ipa{r} clusters in Japhug are from ideophones, and since ideophones have a well-known tendency to present combination of phonemes not observed in the normal vocabulary (\citealt{japhug14ideophones}), it is likely that these clusters are ancient.

As for dental affricates+\ipa{r} clusters, while found in non-ideophonic vocabulary and showing no sign of being recent, are not attested in any word with cognates outside of Gyalrongic. 


Thus, the reconstruction of clusters with dental stops and dental affricates followed by *\ipa{-r-}in Old Chinese is an oddity without real parallel in the family.
%subsection{Final *r}
%*CrVr
%
%> final *\ipa{l}

\subsection{Comparative evidence}
\ipa{mɯrkɯ} `steal', \ipa{rku} \ch{寇}{khuwH}{rob} 
\ipa{tɯ-rpa} `axe', \ch{斧}{pjuX}{axe}

\section{Medial or preinitial *r?}
Comparison \citet{coblin86handlist}
Medial:
\citet{gong95st}
\citet{jacques15sr}

\footnote{The Kamnyu Japhug form \ipa{tɯ-zgrɯ} is probably a contamination of  pre-Kamnyu *\ipa{tɯ-ɣrɯ} `elbow' with the verb \ipa{azgrɯ} `be crooked'. Hence, the form \ipa{tɯ-ɣru} `elbow' from the Sarndzu dialect is given instead.}

\begin{tabular}{llllll}
\toprule
Japhug & Tibetan & Chinese \\
\midrule
\ipa{qʰrɯt} `scrape' &&\ch{刮}{kwæt}{scrape} \\
\ipa{tɯ-ɣru} `elbow' &\ipa{gru.mo} `elbow' &\ch{肘}{ʈjuwX}{elbow}\\
\ipa{pri} `bear'&& \ch{羆}{bje}{bear} \\
\ipa{tɯ-ndzrɯ} `nail, claws' &&\ch{爪}{tʂæwX}{claw}\\
\ipa{mbri} `call, cry'&& \ch{鳴}{mjæŋ}{cry}\\
\ipa{zrɯɣ} `louse'&\ipa{ɕig} `louse'& \ch{蝨}{ʂit}{louse} \\
\ipa{tɤ-zrɤm} `root' &&\ch{參}{ʂim}{rhyzome}\\
\ipa{tɤ-zraʁ}  `shame' && \ch{色}{ʂik}{colour, appearance}\\
\bottomrule
\end{tabular}



For ice, *\ipa{lpaˠm}`ice'
\ipa{rko}`hard' \ch{硬}{ŋŋH}{hard}
%doubtful comparisons not included
%&\ipa{rdal} `spread flat'& \ch{展}{ʈjenX}{extend}\\
\begin{tabular}{lllllll}
\toprule
Japhug & Tibetan & Chinese \\
\midrule
\ipa{tɤjpɣom}  &&\ch{冰}{piŋ}{ice} \\
\ipa{ɯ-rqʰu} `shell, crust' &&\ch{殼}{khæwk}{shell, crust}\\
\ipa{tɤ-rme} `hair' &&\ch{眉}{mij}{eyebrows}\\
\ipa{tɯ-rŋa} `face'&& \ch{颜}{ŋæn}{face} \\
\midrule
&\ipa{rŋam.pa} `awe-inspiring'& \ch{嚴}{ŋjæm}{majestic}\\
&\ipa{rdul} `dust'& \ch{嚴}{ŋjæm}{majestic}\\
&\ipa{rduŋ} `hit'& \ch{撞}{ɖæwŋH}{strike}\\
&\ipa{rdzas} `thing'& \ch{事}{ɖʐiH}{affair}\\
&\ipa{rtod} `tether (animal)'& \ch{綴}{ʈjwet}{tie}\\
\bottomrule
\end{tabular}
%2 , *rtuk (*trok) > té ak’ ‘beat,strike’ rdug ‘to strike against
%- *rtjungx (*trjong÷) > tjwong é ‘mound, peak’
%rdung ‘a small mound, hillock’

%rtjanx (*trjen÷11) > tjé anŸ ‘roll over, unfold’

No evidence:


\begin{tabular}{llllll}
\toprule
Japhug & Tibetan & Chinese \\
\midrule
\ipa{sat} `kill' &\ipa{gsod, bsad} `kill'& \ch{殺}{ʂæt}{kill} \\
\ipa{tɯ-rpaʁ} `shoulder'& \ipa{pʰrag} `shoulder'& \ch{膊}{phak}{arm} \\
\ipa{tɯ-rna} `ear'& \ipa{rna}`ear' & \ch{耳}{ɲiX}{ear} \\
\ipa{tɯ-rtsɤɣ} `articulation, joint'& \ipa{tsʰigs} `joint'& \ch{節}{tset}{joint} \\
\ipa{tɤjmɤɣ} `mushroom' &&\ch{帽}{mawH}{hat} \\
\ipa{tɤ-jme} `tail' &&\ch{尾}{mjɨjX}{tail} \\
\ipa{tɤ-rmi} `name'& \ipa{miŋ} `name'& \ch{名}{mjieŋ}{name} \\ 
\ipa{rŋu}`parch'& \ipa{rŋo} `parch'& \ch{熬}{ŋaw}{fry, roast} \\
\ipa{tɯ-rnoʁ}`brain' && \ch{腦}{nawX}{brain} \\
&\ipa{rtul} `blunt' & \ch{鈍}{dwonH}{blunt} \\
\bottomrule
\end{tabular}


Borrowings and Wanderwörter:

\ipa{mbro} `horse' \ch{馬}{mæX}{horse}

\ipa{qambrɯ} `male yak' \ch{犛}{li, mæw}{horse} < *\ipa{rə, mrˁə}

\subsection{The Tangut-Rmaic model}
 
 Mawo Rma \ipa{rmu} `corpse', Yonghe \ipa{muʳ} (\citealt[41]{sims14yonghe})

\ipa{βuʳ} `drum'
hair (on body) \ipa{hwə̃ʳ}
\subsection{Presyllables in Viet-Muong}

\subsection{The *r infix}

 \citet{sagart99roc}


%
%rdul
%rduŋ


 

 

 

\section{Conclusion}


\bibliographystyle{unified}
\bibliography{bibliogj}

\end{document}