\documentclass[oldfontcommands,oneside,a4paper,11pt]{article} 
\usepackage{fontspec}
\usepackage{natbib}
\usepackage{booktabs}
\usepackage{xltxtra} 
\usepackage{polyglossia} 
\usepackage[table]{xcolor}
\usepackage{gb4e} 
\usepackage{multicol}
\usepackage{graphicx}
\usepackage{float}
\usepackage{hyperref} 
\hypersetup{bookmarksnumbered,bookmarksopenlevel=5,bookmarksdepth=5,colorlinks=true,linkcolor=blue,citecolor=blue}
\usepackage[all]{hypcap}
\usepackage{memhfixc}
\usepackage{lscape}

\bibpunct[: ]{(}{)}{,}{a}{}{,}

\setmainfont[Mapping=tex-text,Numbers=OldStyle,Ligatures=Common]{Charis SIL} 
\newfontfamily\phon[Mapping=tex-text,Ligatures=Common,Scale=MatchLowercase]{Charis SIL} 
\newcommand{\ipa}[1]{{\phon \mbox{#1}}} %API tjs en italique
\newcommand{\ipab}[1]{{\scriptsize \phon#1}} 

\newcommand{\grise}[1]{\cellcolor{lightgray}\textbf{#1}}
\newfontfamily\cn[Mapping=tex-text,Ligatures=Common,Scale=MatchUppercase]{SimSun}%pour le chinois
\newcommand{\zh}[1]{{\cn #1}}
\newcommand{\refb}[1]{(\ref{#1})}
\newcommand{\factual}[1]{\textsc{:fact}}
\newcommand{\rdp}{\textasciitilde{}}

\XeTeXlinebreaklocale 'zh' %使用中文换行
\XeTeXlinebreakskip = 0pt plus 1pt %
 %CIRCG
 \newcommand{\bleu}[1]{{\color{blue}#1}}
\newcommand{\rouge}[1]{{\color{red}#1}} 
\newcommand{\ch}[3]{\zh{#1} \ipa{#2} `#3'} 
\newcommand{\change}[2]{*\ipa{#1} $\rightarrow$ \ipa{#2}}
\begin{document} 
\title{On the reconstruction of medial *-r- in Old Chinese}
\author{Guillaume Jacques}
\maketitle
 

\section{Introduction}
All systems of Old Chinese reconstructions used  presently, from that of \citealt{li71oc} to \citet{schuessler09minimal} and \citet{bs14oc} all follow \citet{yakhontov61sochetaniya} in reconstructing [consonant+liquid] clusters in words whose Middle Chinese rhymes belong to second division or chongniu third division (as well as all other third division words after Middle Chinese retroflex initials). Yakhontov originally proposed to reconstruct a medial *\ipa{-l-} in these words, but later scholars have agreed that this medial consonant was more probably *\ipa{-r-}.

This reconstruction has attracted little controversy in recent years. One of the rare attempts at challenging the received model,  \citet{handel02r}, proposed that rhotic metathesis \change{rC-}{*Cr-}  took place at some stage of Old Chinese, on the basis of comparative evidence with Tibetan, but this work has until now receive little follow-up.

In the present paper, I argue that, while medial *\ipa{-r-} do certainly constitute one of the origins of  second division and chongniu third division, it is not necessarily the only origin of these Middle Chinese categories, and that mechanically projecting *\ipa{-r-} for all syllables of this type result in a highly unrealistic reconstruction.

In the first section, I present a series of arguments, based on typology and Sino-Tibetan comparison, against generalized reconstruction of *\ipa{-r-}.

Then, I show that reconstructing *\ipa{r-} preinitials, and possibly clusters not containing rhotic, can account for the development of rhymes and initials in Middle Chinese without supposing metathesis.

\section{What is wrong with medial *\ipa{-r-}?} \label{sec:medial}
Despite its widespread acceptance, the hypothesis that all word with second division and/or retroflex initials in Middle Chinese (henceforth R-type words), and part of chongniu 3 and rounded rhyme words, had a medial *\ipa{-r-} presents serious problems from the point of view of both Sino-Tibetan comparison and linguistic typology. The main problems involved with this reconstruction is the large excess of reconstructed *\ipa{-r-} in Old Chinese in comparison with more conservative languages of the Sino-Tibetan that preserve the clusters, the presence of a *\ipa{lr-} cluster and of dental + *\ipa{r}, and the rather poor correspondences with \ipa{-r-} clusters in other languages.

\subsection{Excess of *-r- in Old Chinese reconstructions}
The proportion of R-type syllables in Middle Chinese can be estimated from Table \ref{tab:gy}, obtained from the electronic version of the Guangyun. Listed as R-type syllables are second division, as well as all syllables with the initial consonants \zh{知} \ipa{ʈ}, \zh{徹} \ipa{ʈʰ},  \zh{澄} \ipa{ɖ},  \zh{娘} \ipa{ɳ},  \zh{莊} \ipa{tʂ},  \zh{初} \ipa{tʂʰ},  \zh{崇} \ipa{dʐ} and \zh{生} \ipa{ʂ} in third division. This total number strongly underestimates syllables with potential medial *\ipa{-r-} in Old Chinese. Many chongniu 3 syllables are also to be reconstructed with *\ipa{-r-} (\citealt[217-8]{bs14oc]), but the presence of the medial is unambiguous only with *\ipa{e}.  In addition, *\ipa{-r-} medial is undetectable in words with rounded rhymes such as \zh{侯} \ipa{-uw} (\citealt[362]{starostin89}, \citealt[501]{baxter92}).

\begin{table}
\caption{Proportion of R-type syllables in the Guangyun} \label{tab:gy} \centering
\begin{tabular}{lllll}
\toprule
Type& Nb of characters & Nb of distinct syllables \\
\midrule
Second division	& 3126	 & 619 \\
%Chongniu 3	& 1997	& 420 \\
Retroflex initials & 1592	&335 \\%930	& 187 \\
\midrule
%All &	6053	&1226\\
Total& 4718	&1094 \\
out of &	25300	&3883 \\
	&19\%	&28\% \\
	\bottomrule
\end{tabular}
\end{table}

If all chongniu 3 syllables are included in the count, the total number is 6053 syllables (about 24\%; still excluding syllables with rounded initials). Therefore, it is safe to assume that in all modern systems of Old Chinese reconstruction, the total number of syllable reconstructed with medial *\ipa{-r-} is between 19\% and 24\%.

%知	136	30 
%徹	150	33
%澄	190	31
%娘	64	13
%莊	101	18
%初	77	20
%崇	51	20
%生	161	22
% en comptant les chongniu:
%知	 279 56
%徹	 256 60
%澄	 373 62
%娘	 126 29
%莊	 127 27
%初	 107 31
%崇	 85 29
%生	 239 41

This figure can be compared with a count of words reconstructed medial *\ipa{-r-} in an actual reconstruction system of Old Chinese for which a searchable database in available. In the list of reconstructions provided online as a companion to \citet{bs14oc}, 1070 words out of 4974 are reconstructed with a non-ambiguous medial *\ipa{-r-} (21.5 \%), indeed within the range estimated an the basis of Middle Chinese alone.

This figure can be compared with the actual proportion of clusters with medial or preinitial \ipa{r} in other languages of the Trans-Himalayan family.

The languages richest in clusters containing \ipa{r} in Sino-Tibetan are without doubt Gyalrongic languages, and to a lesser extent Tibetan.

In Japhug, a Rgyalrongic languages whose cluster inventory has been investigated in detail, we find 411 clusters, including 59 clusters with \ipa{r} as final element, and 47 clusters with \ipa{r} as a non-final element: 26\% of clusters contain an \ipa{r}. Japhug is in addition particularly notable in having clusters with dental affricates of tsr dr

%rC 34+3
%Cr 26
%rCC 10
%CCr 33

\citet{jacques15japhug}, we find 472 words with medial \ipa{-r} out of 6829 (7\%), and 645 words with preinitial \ipa{r-} (9\%), in total 1117 (16\%).

\subsection{Laterals}
To account for the presence of retroflex stops in 
*lr

\citet[217-225]{starostin89}, \citet[36-40]{sagart99roc}


\citet[78]{baroni14invariant}

\citet{nikitina12logophoric}

\subsection{Dental + r}

%\subsection{Final *r}
%*CrVr
%
%> final *\ipa{l}

\subsection{Comparative evidence}

\section{Medial or preinitial *r?}

\subsection{The Tangut-Rmaic model}
 
 Mawo Rma \ipa{rmu} `corpse', Yonghe \ipa{muʳ} (\citealt[41]{sims14yonghe})

\ipa{βuʳ} `drum'
hair (on body) \ipa{hwə̃ʳ}
\subsection{Presyllables in Viet-Muong}

\subsection{The *r infix}

 \citet{sagart99roc}

%\subsection{Grave stops}
%
%
%
%\ipa{tɤjpɣom} $\leftarrow$ *\ipa{lpaˠm}`ice' \ch{冰}{piŋ}{ice} *\ipa{rpəm}
%
%\ipa{ɯ-rqʰu} `shell, crust' \ch{殼}{khæwk}{shell, crust}
%
%\ipa{rka} `\zh{水渠、灌溉渠}' \ch{渠}{gju}{canal} *rga (borrowing)
%
%
%medial r:
%
%\ipa{pri} `bear' \ch{羆}{bje}{bear} *braj
%
%\ipa{qhrɯt} `scrape' \ch{刮}{kwæt}{scrape} *krot
%
%
%\ipa{ɴkʰruŋ} `be born' \ch{降}{kæwŋ^h}{descend} *kˁruŋs
%Evidence of *r?
%
%\ipa{mɯrkɯ} `steal', \ipa{rku} \ch{寇}{khuwH}{rob} 
%
%rmug \ch{霧}{mjuH}{fog}
%
%
%no *r:
%
%\ipa{tɯ-rpaʁ} `shoulder', \ipa{phrag} (metathesis) \ch{膊}{phak}{arm}
%
%\ipa{tɯ-rpa} `axe', \ch{斧}{pjuX}{axe}
%
%\ipa{tɯ-rna} `ear', \ipa{rna}, \ch{耳}{ɲiX}{ear}
%
%\ipa{tɯ-rtsɤɣ} `articulation, joint', \ipa{tshigs}, \ch{節}{tset}{joint}
%
%\ipa{tɤjmɤɣ} `mushroom' \ch{帽}{mawH}{hat}
%
%\ipa{tɯ-jmŋo} `dream' \ch{夢}{mjuwŋH}{dream}
%
%\subsection{Dental stops and affricates}
%\ipa{tɯ-zgrɯ} / \ipa{tɯ-ɣru} `elbow' (contamination with \ipa{azgrɯ})  \ch{肘}{ʈjuwX}{elbow}
%
%rdul
%rduŋ
%
%\ipa{rdzas} `thing' \ch{事}{dʐiH}{affair} (borrowing)
%
%
%\ipa{tɯ-ndzrɯ} `nail, claws' \ch{爪}{tʂæwX}{claw}
%\subsection{Nasals}
%Expected *\ipa{mr} $\rightarrow$ \ipa{br-} nowhere attested
%
%
%no *r:
%
%\ipa{tɤ-jme} `tail' \ch{尾}{mjɨjX}{tail}
%
%\ipa{tɤ-rmi} `name', \ipa{miŋ} \ch{名}{mjieŋ}{cry} 
%
%\ipa{rŋo}`parch', \ipa{rŋo}, \ch{熬}{ŋaw}{fry, roast}
%
%\ipa{tɯ-rnoʁ}`brain', \ch{腦}{nawX}{brain}
%
%Nr:
%
%\ipa{mbri} `call, cry' \ch{鳴}{mjæŋ}{cry}
%
%
%rN:
%
%\ipa{tɤ-rme} `hair' \ch{眉}{mij}{eyebrows}
%
%\ipa{tɯ-rŋa} `face' \ch{颜}{ŋæn}{face} *\ipa{rŋan} not *\ipa{ŋran}
%
%\ipa{rŋam.pa} `awe-inspiring' \ch{嚴}{ŋjæm}{majestic} *\ipa{rŋam} \citet{coblin86handlist}
%
%
%Borrowings and Wanderwörter:
%
%\ipa{mbro} `horse' \ch{馬}{mæX}{horse}
%
%\ipa{qambrɯ} `male yak' \ch{犛}{li, mæw}{horse} < *\ipa{rə, mrˁə}


 

 

 

\section{Conclusion}


\bibliographystyle{unified}
\bibliography{bibliogj}

\end{document}