%\citet[34-36]{manczak63analogique}
\documentclass[oneside,a4paper,11pt]{article} 
\usepackage{fontspec}
\usepackage{natbib}
\usepackage{booktabs}
\usepackage{xltxtra} 
\usepackage{polyglossia} 
\usepackage[table]{xcolor}
\usepackage{gb4e} 
\usepackage{multicol}
\usepackage{graphicx}
\usepackage{float}
\usepackage{lineno}
\usepackage{textcomp}
\usepackage{hyperref} 
\hypersetup{bookmarks=false,bookmarksnumbered,bookmarksopenlevel=5,bookmarksdepth=5,xetex,colorlinks=true,linkcolor=blue,citecolor=blue}
\usepackage[all]{hypcap}
\usepackage{memhfixc}
\usepackage{lscape}
 \usepackage{setspace}
\makeatletter
 

%\setmainfont[Mapping=tex-text,Numbers=OldStyle,Ligatures=Common]{Charis SIL} 
\newfontfamily\phon[Mapping=tex-text,Ligatures=Common,Scale=MatchLowercase,FakeSlant=0.3]{Charis SIL} 
\newcommand{\ipa}[1]{{\phon#1}} %API tjs en italique
 
\newcommand{\grise}[1]{\cellcolor{lightgray}\textbf{#1}}
\newfontfamily\cn[Mapping=tex-text,Ligatures=Common,Scale=MatchUppercase]{MingLiU}%pour le chinois
\newcommand{\zh}[1]{{\cn #1}}
\newcommand{\topic}{\textsc{dem}}
\newcommand{\tete}{\textsuperscript{\textsc{head}}}
\newcommand{\rc}{\textsubscript{\textsc{rc}}}
\XeTeXlinebreaklocale 'zh' %使用中文换行
\XeTeXlinebreakskip = 0pt plus 1pt %
 %CIRCG
\newcommand{\ro}{$\Sigma$}
\newcommand{\siga}{$\Sigma_1$} 
\newcommand{\sigc}{$\Sigma_3$}   
\newfontfamily\mleccha[Mapping=tex-text,Ligatures=Common,Scale=MatchLowercase]{Galatia SIL}%pour le grec
\newcommand{\grec}[1]{{\mleccha #1}}
\linenumbers

\begin{document} 

\title{On the directionality of analogy in a Dhegiha paradigm\footnote{Glosses follow the Leipzig Glossing rules, except for \textsc{prox} proximate (on which see \citealt{eschenberg05omaha}). The underscore \_ in a verb form indicates where the personal prefixes are inserted, in the case of verbs with discontinuous stems. In Lakota, capital \ipa{A} indicates an ablauting final vowel, which is realized as \ipa{a}, \ipa{e} or \ipa{iŋ} depending on the present of particular enclitics. Acknowledgements will be added after editorial decision. This article is dedicated to the memory of Bob Rankin; although I never met him, xxxx}}
 %Acknowledgements: Anton Antonov and two anonymous reviewers
\author{Guillaume Jacques}
\maketitle
% \onehalfspacing
\textbf{Abstract}: This paper documents a case of analogy from the first person to the third in the Dhegiha languages. It discusses the significance of this example for historical linguistics in general, and proposes that higher frequency in discourse of the first person form in the case of cognition verbs explains why analogy took place in the opposite direction in the case of this verb.

\textbf{Keywords}: Siouan, Lakota, Dhegiha, Omaha, Osage, defective verb, analogy, verb of cognition, sound laws
 
\section{Introduction}
%\raggedright
%\parindent=2em

In the Neogrammarian model, while exceptionless sound laws are fully deterministic, the same cannot be said of analogy, by definition irregular and unpredictable. Yet, it is possible to some extent to constrain  the space of hypotheses involving analogy, and research on the general principles of analogy is of utmost importance for historical linguistics.

The most successful generalization involving analogical change is without doubt the observation that in personal paradigms, the third person singular is nearly always the form on which levelling is based. One particular instance of this type of levelling has been referred to as  \citet{watkins62celtic}'s law, according to which the third person can be reanalyzed as being zero marked, and the resulting stem is generalized to the whole paradigm by addition of the regular affixes. 

The directionality of analogy from third person to first or second is also verified even when the third person singular is not reanalyzed as being zero-marked. For instance, in the perfect paradigm in Vedic, there was a  \ipa{ā} / \ipa{a} alternation between first and third person singular, due to the effect of the Brugmann/Kuryłowicz' law (\citealt{kurylowicz27indoiranien}, \citealt[205]{fortson10intro}).\footnote{Proto-Indo-european *\ipa{o} becomes \ipa{ā} in Sanskrit in open syllable, and \ipa{a} in closed syllables including in contexts such as *\ipa{oCH}, where H is a laryngeal.} In later stages of the language, the first person stem in \ipa{a} tends to be be replaced by a stem in \ipa{ā} whose source can only be the third singular (\citealt[283-9]{whitney24}, see Table \ref{tab:skt}). Here the third person is not reanalyzed as zero-marked (since there is a third person suffix \ipa{--a}), but the long vowel stem is generalized to the first person.

\begin{table}
\caption{Analogy from third singular to first singular in the paradigm of \ipa{KṚ} `do' in Sanskrit} \label{tab:skt} \centering
\begin{tabular}{llll}
\toprule
Person & PIE & Original paradigm & Analogical paradigm \\
\midrule
\textsc{1sg} & *\ipa{kʷe-kʷór-h_2e} & \ipa{cakár-a} & \ipa{cakā́r-a} \\
\textsc{2sg} & *\ipa{kʷe-kʷór-th_2e} & \ipa{cakár-tha} & id. \\
\textsc{3sg} & *\ipa{kʷe-kʷór-e} & \ipa{cakā́r-a} & id. \\
\bottomrule
\end{tabular}
\end{table}

 The validity of the 3 $\rightarrow$ SAP directionality in analogical levelling has been tested in various language families, and is apparent even in direct / inverse systems, where analogy operates from third person to SAP forms and affects plural forms before affecting singular ones (see \citealt{goddard65arapaho} on Arapaho, \citealt{dahlstrom89change} on plains Cree and \citealt{jacques15directionality} for a general overview of these phenomena in Algonquian).
   

Counterexamples, where a first or second person form resists analogy better than the third person singular form, do occur when a phonological alternation is levelled across the board. For instance in Attic Greek,  where  the labiovelars become dentals before front vowels and labials in most other contexts (\citealt[253]{fortson10intro}), most verbal and nominal paradigms should present alternations between dental and labial stops in words whose last stem consonant was a labiovelar. A labial consonant would have been expected before endings in non-front vowels (for instance in the  first person singular \ipa{-ō} of the present thematic paradigm), and  dentals before front vowels (for instance the third person singular \ipa{-ei}). Yet, no  verb shows such an alternation, as the labiovelars have been generalized everywhere:  both the first person singular \grec{νήφω} \ipa{nḗpʰō}  $\leftarrow $ *\ipa{ne-egʰʷ-ō} `I am sober' and the third person singular \grec{νήφει} \ipa{nḗpʰei}  $\leftarrow $ *\ipa{ne-egʰʷ-ei} `he is sober' have a labial \ipa{pʰ} from *\ipa{gʰʷ}, though \ipa{tʰ} would have been expected in the third person. 

Yet, this does not count as analogical levelling from the first person to the third, as the third person plural also has a non-front vowel \grec{νήφουσι} \ipa{nḗpʰousi}  $\leftarrow $ *\ipa{ne-egʰʷ-onti} `they are sober', and some tenses, such as the future (\grec{νήψει} \ipa{nḗpsei} $\leftarrow $ *\ipa{ne-egʰʷ-s-ei} `he will be sober'), have consonant-initial endings where the labial is expected in all forms by applying the regular sound laws.

Genuine examples of analogical levelling where the third person is renewed on the basis of the first person are extremely rare, and any such case deserves to be reported and carefully documented.\footnote{Cases where the second person is renewed on the basis of the first person singular may be less rare, see for instance the case of some Turkish dialects (\citealt{saygin02reanalysis}). The relative sensitivity of the first and the second person to analogy is not discussed in this paper.}  Yet, Siouan languages offer one uncontroversial example  of analogy from the first person to the second and third,  involving the verb `to think'. This paper presents a detailed account of this analogical change and provides possible explanations as to why specifically this verb underwent analogy in this direction. Most of the work on comparative Siouan is unpublished (though available from the Siouan Archive),\footnote{This archive is maintained by the Hidatsa specialist John Boyle and  is freely distributed to all interested scholars.} so that this paper will first present the accepted knowledge on Siouan reconstruction, mainly based on the model in the Comparative Siouan Dictionary (\citealt{rankin15csd}). 

 The first section presents the correspondences and reconstructions for the  consonants *\ipa{y}, *\ipa{r} and *\ipa{R}, which are relevant for understanding the case of analogy discussed in this paper. The following section discusses the main clusters containing these consonants. Then, we describe the verb paradigms of *\ipa{r} stem verbs, and show that the paradigm of the verb  \ipa{ebðé} `I think' in Omaha, which originally belonged to a different conjugation type, was remodelled following the *\ipa{r} stem verb conjugation on the basis of the first person: the third  person is renewed, while the first person is inherited. Finally, we discuss a possible explanation for the directionality of analogy in the case of this verb.
 

\section{Reflexes of simple *r, *R and *y in Dakotan and Dhegiha}
There is evidence in Mississippi valley Siouan languages of at least three distinct proto-phonemes *\ipa{r}, *\ipa{R} and *\ipa{y} before oral vowels. These phonemes merge as \ipa{r} in Chiwere and Winnebago, but are kept partially distinct in Dakotan and Dhegiha, according to the   correspondences in   Table \ref{tab:basic} (correspondences before nasal vowels are not indicated). 

These correspondences are valid word-initially or between vowels, except those involving the clusters *\ipa{ky--} and *\ipa{wy--} (see below). In Kansa and Osage, the outcomes of *\ipa{R} are palatalized as \ipa{ǰ} and \ipa{c} respectively before \ipa{e} (merging with *\ipa{ht}).

\begin{table}[h]
\caption{Basic correspondences of proto-Siouan *r, *R and *y in Dakotan and Dhegiha languages} \label{tab:basic} \centering
\begin{tabular}{l|ll|lllll}
\toprule
  & 	Lakhota  & 	Yankton/Sisseton & 	 Omaha  & 	 Kansa  & 	 Osage  & 	 Quapaw  &	\\	
  \midrule
\ipa{*y} & 	\ipa{čh} & 	\ipa{čh} & 	\ipa{ž} & 	\ipa{ž} & 	\ipa{ž} & 	\ipa{ž} & 	\\	
\ipa{*r} & 	\ipa{y} & 	\ipa{y} & 	\ipa{ð} & 	\ipa{y} & 	\ipa{ð} & 	\ipa{d} & 	\\	
\ipa{*R } & 	\ipa{l} & 	\ipa{d} & 	\ipa{n} & 	\ipa{d / ǰ} & 	\ipa{t / c} & 	\ipa{t} & 	\\	
\ipa{*r / i\_} & 	\ipa{čh} & 	\ipa{čh} & 	\ipa{ð} & 	\ipa{y} & 	\ipa{ð} & 	\ipa{d} & 	\\	
\midrule
\ipa{*kr} & 	\ipa{gl} & 	\ipa{hd} & 	\ipa{gð} & 	\ipa{l} & 	\ipa{l} & 	\ipa{kd} & 	\\		
\ipa{*xr} & 	\ipa{ȟl} & 	\ipa{hd} & 	\ipa{xð} & 	\ipa{xl} & 	? & 	? & 	\\		
\ipa{*wr} & 	\ipa{bl} & 	\ipa{md} & 	\ipa{bð} & 	\ipa{bl} & 	\ipa{br} & 	\ipa{bd} & 	\\		
\midrule
\ipa{*ky} & 	\ipa{čh / kč} & 	\ipa{čh / kč} & 	\ipa{gð} & 	\ipa{l} & 	\ipa{l} & 	\ipa{kd} & 	\\	
\ipa{*xy} & 	\ipa{ȟč} & 	\ipa{ȟč} & 	\ipa{xð} & 	\ipa{xl} & 	? & 	? & 	\\	
\ipa{*wy} & 	\ipa{čh / pč} & 	\ipa{čh / pč} & 	\ipa{bð} & 	? & 	\ipa{br} & 	? & 	\\	
\bottomrule
\end{tabular}
\end{table}

\begin{table}[h]
\caption{Examples of the correspondences of proto-Siouan *r, *R and *y in Dakotan and Dhegiha languages} \label{tab:basic.ex} \centering
 \resizebox{\columnwidth}{!}{
\begin{tabular}{ll|l|lll|ll}
\toprule
&proto-Siouan &Lakhota    & Omaha & Kansa & Osage   \\
	\midrule	
\ipa{*y} &	\ipa{*yo\_phe} &	\ipa{čho\_pȟÁ} &		\ipa{žúhe} &	\ipa{žophé} &	\ipa{žó\_pše} &		wade &	\\
&	\ipa{*yá•pE} &	\ipa{čhápa} &		\ipa{žábe} &	\ipa{žábe} &	\ipa{žápe} &		beaver &	\\
&	\ipa{*yá•že} &	\ipa{čhažé} &		\ipa{žáže} &	\ipa{žá•že} &	\ipa{žáže} &		name &	\\
&	\ipa{*yeká} &	\ipa{čhečá} &		\ipa{žegá} &	\ipa{žegá} &	\ipa{žeká} &		thigh, leg &	\\
&	\ipa{*yá•phe} &	\ipa{čha\_pȟÁ} &		\ipa{žáhe} &	\ipa{žáphe} &	\ipa{} &		stab &	\\
&	\ipa{*e/aya-ʔį} &	\ipa{ečhíŋ, ečháŋmi} &		\ipa{éžǫmį} &	\ipa{ažį́, ažamį} &	\ipa{ažį́, ážamį} &		think &	\\
	\midrule	
\ipa{*r-} &	\ipa{*ra-} &	\ipa{ya-} &		\ipa{ða-} &	\ipa{ya-} &	\ipa{ða-} &		by mouth &	\\	
&	\ipa{*ru-} &	\ipa{yu-} &		\ipa{ði-} &	\ipa{yü-} &	\ipa{ðu-} &		by hand &	\\	
&	\ipa{*xurá} &	\ipa{ȟuyá} &		\ipa{xiðá} &	\ipa{xüyá} &	\ipa{xúða} &		eagle &	\\	
&	\ipa{*rÉ} &	\ipa{yA´} &		\ipa{ðé} &	\ipa{yé} &	\ipa{(a)ðée} &		go &	\\	
	\midrule

 \ipa{*R} &	\ipa{*Ró•te} &	\ipa{loté} &		\ipa{núde} &	\ipa{dó•ǰe} &	\ipa{tóoce} &		throat &	\\
&	\ipa{*(i-)Rekší} &	\ipa{lekší} &		\ipa{inégi} &	\ipa{iǰégi} &	\ipa{įcéki} &		uncle (MB) &	\\
&	\ipa{*Rá} &	\ipa{lá} &		\ipa{wana+} &	\ipa{da} &	\ipa{tá} &		ask &	\\
&	\ipa{*Ré•že} &	\ipa{léžA} &		\ipa{néže} &	\ipa{ǰéže} &	\ipa{céže} &		urine &	\\
&	\ipa{*Ré} &	\ipa{o\_lé} &		\ipa{u\_né} &	\ipa{o\_ǰé} &	\ipa{o\_cé} &		hunt, look for &	\\

\bottomrule
\end{tabular}}
\end{table}
Table \ref{tab:basic.ex} illustrates the first three correspondences; in this table and other Tables below, only Lakhota, Omaha, Kansa and Osage are presented. The cognate sets are from \citet{rankin15csd}, but the Lakhota and Osage  data were rechecked against \citet{ullrich08} and \citet{quintero10osage}, respectively, and adapted to the spellings in these dictionaries. As for the Omaha data, forms not found in the Dorsey text corpus (\citealt{dorsey1890cegiha}) are indicated by the symbol +, and are taken directly from the comparative dictionary.
 

In some possessed nouns (especially body parts), we find a special correspondence with \ipa{čh} in Dakotan against \ipa{ð} : \ipa{y} : \ipa{ð} : \ipa{d} in Dhegiha, as in Table \ref{tab:ir}. This correspondence is interpreted as the effect of paradigm levelling, due to a sound change *\ipa{r}-- $\rightarrow$ \ipa{čh} after the third person possessive prefix \ipa{i--}, even before nasal vowels (see \citealt{rankin05quapaw}, \citealt{rankin15csd}). It is never attested in verbs, even those with a locative \ipa {í--} prefix; for instance *\ipa{íre} regularly yields  \ipa{íðe} 'see, find' in Omaha and \ipa{iyé\_ya} ‘find' in Lakota (not *\ipa{ičhé}).

\begin{table}[h]
\caption{Proto-Siouan *r after the possessive *i-- prefix} \label{tab:ir} \centering
 \resizebox{\columnwidth}{!}{
\begin{tabular}{ll|l|lll|ll}
\toprule
&proto-Siouan &Lakhota  &   Omaha & Kansa & Osage   \\
	\midrule	
&	\ipa{*i-ré•ži} &	\ipa{čheží} &		\ipa{teðéze} &	\ipa{yéze} &	\ipa{ðéeze} &		tongue &	\\
&	\ipa{*i-rá•ɣu} &	\ipa{čhaǧú} &		\ipa{ðǫ́xį+} &	\ipa{yáxü} &	\ipa{ðǫɣu} &		lung &	\\
&	\ipa{*i-ráke} &	\ipa{čhaká} &		\ipa{} &	\ipa{ho yáge} &	\ipa{} &		palate / gills &	\\
&	\ipa{*i-rą́•h-ka} &	\ipa{čhaŋkȟáhu} &		\ipa{nǫ́kka} &	\ipa{ną́kka} &	\ipa{ną́hka} &		spine &	\\
\bottomrule
\end{tabular}}
\end{table}

Before nasal vowels, *\ipa{r} and *\ipa{R} merge as \ipa{n} in Dakotan and Dhegiha languages in nearly all contexts (see  \citealt{michaud-jacques12nasalite}), but this issue will not be discussed in the present paper, as it has no relevance to the issue at hand.

In addition, there is a residue of words showing irregular correspondences, such as   `mosquito' (Lakota \ipa{čhaphų́ka}, Omaha \ipa{náhǫga}, Kansa \ipa{yáphąįge}, Osage \ipa{yáphąįge}, Quapaw \ipa{daphąke}), but these cases are most likely Wanderwörter and do not reflect the correspondences of inherited words. Therefore, they are not taken into account here.

\section{Reflexes of clusters}
There were some clusters with *\ipa{r} or *\ipa{y} as second element in proto-Mississippi Valley Siouan.\footnote{The contrast between  *\ipa{r} and *\ipa{R} was neutralized in clusters; the reflexes of the liquid in clusters and more similar to those of *\ipa{R}, but we follow here the transcription generally adopted in the Comparative Siouan Dictionary.} Examples of *\ipa{wr--}, \ipa{kr--},  \ipa{sr--}, *\ipa{šr--} and *\ipa{xr--} are plentiful, as can be illustrated by the data in Table \ref{tab:kr} (many other additional examples could be added).


 \begin{table}[h]
\caption{Proto-Siouan *\ipa{kr--}, *\ipa{xr--} and *\ipa{wr--}} \label{tab:kr} \centering
 \resizebox{\columnwidth}{!}{
\begin{tabular}{ll|l|lll|ll}
\toprule
&proto-Siouan &Lakhota    & Omaha & Kansa & Osage     \\
	\midrule	
*\ipa{kr--}&	\ipa{*kré•pE} &	\ipa{glépA} &		\ipa{gðébe} &	\ipa{lébe} &	\ipa{} &		vomit &	\\
&	\ipa{*krézE} &	\ipa{glézA} &		\ipa{gðéže} &	\ipa{léze} &	\ipa{léze} &		spotted &	\\
&	\ipa{*krá} &	\ipa{gla-} &		\ipa{ðigða+} &	\ipa{layá} &	\ipa{} &		untie &	\\
 	\midrule	
*\ipa{xr--}&	\ipa{*xroke} &	\ipa{ȟlókA} &		\ipa{xðúge} &	\ipa{xlóge} &	\ipa{} &		hole, nostrils &	\\
&	\ipa{*xro} &	\ipa{ȟló} &		\ipa{xðúde} &	\ipa{} &	\ipa{} &		snort &	\\
	\midrule	
*\ipa{wr--}	&	\ipa{*awró} &	\ipa{abló} &		\ipa{} &	\ipa{abló} &	\ipa{ábro} &		shoulder &	\\
&	\ipa{*hąwre} &	\ipa{haŋblé} &		\ipa{hǫ́bðe} &	\ipa{hąblé} &	\ipa{hǫ́bre} &		dream &	\\
&	\ipa{*wráska} &	\ipa{blaská} &		\ipa{bðáska} &	\ipa{bláska} &	\ipa{bráaska} &		flat &	\\
&	\ipa{*wréh-(ka)} &	\ipa{blečá-} &		\ipa{bðékka} &	\ipa{blékka} &	\ipa{bréhka} &		thin &	\\
	\bottomrule
\end{tabular}}
\end{table}

Examples of clusters with \ipa{y} as a second element are much more restricted, and   Table  \ref{tab:ky} contains all known examples (from \citealt{koontz83rstem}  and \citealt{rankin15csd}).

 \begin{table}[h]
\caption{Proto-Siouan *\ipa{ky--}, *\ipa{xy--} and *\ipa{wy--}} \label{tab:ky} \centering
 \resizebox{\columnwidth}{!}{
\begin{tabular}{ll|l|lll|ll}
\toprule
&proto-Siouan &Lakhota    & Omaha & Kansa & Osage     \\
	\midrule	
*\ipa{ky--}&	\ipa{*kyé•wrą} &	\ipa{wikčémna} &		\ipa{gðéba} &	\ipa{léblą} &	\ipa{lébrą} &		ten &	\\
&	\ipa{*kyetą́} &	\ipa{čhetáŋ} &		\ipa{gðedǫ́} &	\ipa{ledą́} &	\ipa{letǫ́} &		hawk &	\\
&	\ipa{*kyąšká} &	\ipa{čhaŋšká} &		\ipa{gðąšká+} &	\ipa{} &	\ipa{} &	raptor	 &	\\
&	\ipa{*rukyą} &	\ipa{i\_yúkčaŋ} &		\ipa{wa-ðígðǫ} &	\ipa{í\_yülą } &	\ipa{í\_ðilą} &		know &	\\
	\midrule	
*\ipa{xy--}	&	\ipa{*xyá} &	\ipa{ȟčá / waȟčá} &		\ipa{waxðá} &	\ipa{xlá} &	\ipa{} &		bloom, flower &	\\
	\midrule	
*\ipa{wy--}&	\ipa{*e-w-ye} &	\ipa{epčÁ} &		\ipa{ebðé} &	\ipa{} &	\ipa{ébre}  &	I think &	\\	
&	\ipa{*rąwyE} &	\ipa{napčÁ}  &	\ipa{} &	\ipa{} &	\ipa{wa-nǫ́bre} &		eat, swallow &	\\	
&	\ipa{*w-yi--} &	\ipa{čhi-}  &	\ipa{wi--} &	\ipa{} &	 	\ipa{} &		1$\rightarrow$2 prefix& 	\\	
\bottomrule
\end{tabular}}
\end{table}

In Dhegiha, clusters in *\ipa{Cy--} merge with their *\ipa{Cr--} counterparts.\footnote{The 1$\rightarrow$2 portmanteau prefix presents a unique correspondence; given the well-known tendency of local forms to be irregular and display usual developments (\citealt{heath98skewing}), no attempt at explaining these forms will be undertaken here. }

In Lakhota, the cluster *\ipa{y--} element in *\ipa{Cy--} cluster changes to \ipa{č} word-internally, and the previous segment is preserved (and undergoes fortition, in the case of *\ipa{wy--}). Word-initially, *\ipa{wy--} and *\ipa{ky--} apparently merge with *\ipa{y} as 	\ipa{čh--}.\footnote{The 1$\rightarrow$ 2 prefix does not always appear word-initially, but we can safely assume that analogical pressure would have removed the expected \ipa{čh--} / \ipa{pč--} allomorphy.}
%wyi > chi- (wi- in O

 
%\citet{rankin02ofo}
\section{Verbal paradigms}
 
The paradigm of *\ipa{r--} initial verb stems (including verbs with the instrumental prefixes *\ipa{ra--} `by mouth' and *\ipa{ru--} `by hand') constitute a distinct conjugation class in Mississippi Valley Siouan languages. The paradigms in these languages are provided in Table \ref{tab:go} (see \citealt{koontz83rstem, koontz90syncopating}). 

The first and second person forms result from the first singular *\ipa{w--} and second person *\ipa{y--} prefixes respectively which occur in the so-called `syncopating paradigms'.\footnote{The non-syncopating  regular active and stative paradigms, whose correspondences are quite complex between Siouan languages, will not be discussed here, as they are not relevant to the topic at hand.} In all languages except Lakota, the *\ipa{y--} prefix becomes \ipa{š} in this paradigm by obstruentization.\footnote{The absence of \ipa{š} in Lakota might be secondary; in Omaha, second person forms of  *\ipa{r--} stem verbs have variants in \ipa{hn} and \ipa{n} instead of  \ipa{šn} in the text corpus.}. That *\ipa{y--}, rather than *\ipa{š--}, should be reconstructed here is shown by other syncopating paradigms such as that of the glottal stop stems; for instance, the second person form of \ipa{ʔǫ́} `be, do' in all Dhegiha languages is \ipa{žǫ́}  `you do, you are', which can only be from *\ipa{y-ã} $\leftarrow$ *\ipa{y-ʔã}. 

%of the first person *\ipa{wa--} and second person  \ipa{ya--} prefixes.\footnote{The reconstruction of the second person prefix is a complex problem: Mississippi Valley Siouan languages point to *\ipa{ra--}, while Biloxi and Ofo suggest a reconstruction *\ipa{ya--} instead. This issue is in any case irrelevant to the present  topic, as we focus exclusively on MVS. }


\begin{table}[h]
\caption{Paradigm of the verb *\ipa{re} `go' in Siouan languages} \label{tab:go}
 \resizebox{\columnwidth}{!}{
\begin{tabular}{ll|l|ll|llll|ll}
\toprule
 &	 &	Lakhota &	Winnebago &	Chiwere &	Omaha &	Osage &	Kansa &	Quapaw &	Ofo &	\\	
  \midrule
1sg.A &	\ipa{*wre•} &	\ipa{blÁ} &	\ipa{tée} &	\ipa{hajé} &	\ipa{bðe} &	\ipa{brée} &	\ipa{bne} &	\ipa{bde} &	\ipa{até•kna} &	\\	
2sg.A &	\ipa{*šre•} <\ipa{*yre•}  &	\ipa{lÁ} &	\ipa{šeré} &	\ipa{slé} &	\ipa{šne} &	\ipa{šcée} &	\ipa{hne} &	\ipa{tte} &	\ipa{šté•kna} &	\\	
base &	\ipa{*re•} &	\ipa{yÁ} &	\ipa{rée} &	\ipa{lé} &	\ipa{ðé} &	\ipa{aðée} &	\ipa{yé} &	\ipa{dé} &	\ipa{té•kna} &	\\	
\bottomrule			

\end{tabular}}
\end{table} 
 
 In Omaha and other Dhegiha languages, the verb \ipa{ebðé}, \ipa{éðe} `think' follows the same conjugation as *\ipa{r} stem verbs, as shown in Table \ref{tab:think}.\footnote{This verb has a discontinuous stem  \ipa{e\_ðe}. The element \ipa{e--} is probably originally a demonstrative related to Omaha \ipa{é}  that'.} However, its Lakota cognate  \ipa{epčÁ} `I think', a defective verb attested only in the first singular, belongs to the set of forms with *\ipa{wy--} in proto-Siouan.  
 
 There is little doubt that these two verbs are cognate, and that the first person form  should be reconstructed with  *\ipa{wy--}, a cluster also attested in the comparison between Lakota \ipa{napčÁ}  `swallow'	and Osage \ipa{wa-nǫ́bre} `eat (it)'.\footnote{The \ipa{wa--} prefix in Osage is the antipassive; the antipassive form \ipa{wa-nápčA} `swallow' also exists in Lakhota). This word is apparently not attested in other Dhegiha languages.} Not only are these verbs phonologically compatible, they occur in the same constructions;  Omaha  \ipa{ebðé}, \ipa{éðe} `think' and Lakhota \ipa{epčÁ}  can only be used with a complement clause (in Omaha, the verb \ipa{síðe} `think of' is used when the P is not a complement clause), which can appear before (example \ref{ex:ebdhegon}) or after (\ref{ex:ebdhegon2})  the verb (see also \citealt{ullrich08}). This morphosyntactic property should also be reconstructed back to proto-Mississippi Valley Siouan.

\begin{exe}
 \ex \label{ex:ebdhegon} 
 \gll ``\ipa{ǫ́kʰaži,}  	\ipa{nudǫ́hǫga}  	\ipa{akʰá}  	\ipa{é}  	\ipa{wakʰa-báži}  	\ipa{ebðégǫ}"  	\ipa{á-bi-amá}\\
no leader \textsc{prox:sg} that mean-\textsc{neg} \textsc{1sg}:think say-\textsc{prox-evd}\\
%\end{exe}
%\begin{exe}
% \ex \label{ex:no.ebdhegon}
% \gll ``\ipa{ǫ́kʰaži,}  \ipa{ha}	 \ipa{nudǫ́hǫga}  	\ipa{akʰá}  	\ipa{é}  	\ipa{wakʰa-í}"   	\ipa{á-bi-amá}  \\
%no \textsc{interj} leader \textsc{prox:sg} that mean-\textsc{prox}  say-\textsc{prox-evd} \\
\glt ``No, the leader did not mean that, I think," said the former. (177:19-178.1)
\end{exe}



\begin{exe}
 \ex \label{ex:ebdhegon2}
 \gll 
\ipa{nį́kašįga}  	\ipa{ðižúbaži}  	\ipa{kíǰi,}  	\ipa{égǫwéʼǫ}  	\ipa{gǫðawáðe}  	\ipa{ebðégǫ.}  \\
man injure when do.this.way desirable \textsc{1sg}:think \\
\glt I think that when one man injures another, it is desirable to repay him. ( 438,  12)
  \end{exe}

  
 
 Since the cluster \ipa{pč} is extremely rare in Lakota, the dearth of cognates is not surprising. Moreover, the parallelism with the group *\ipa{ky--}, which merges with *\ipa{kr--} in Dhegiha but remains distinct as \ipa{kč} in Lakota, confirms that *\ipa{wy--} is indeed the correct reconstruction for the correspondence \ipa{pč} in Lakhota to \ipa{bð} in Omaha.
 
 \begin{table}[h]
\caption{Paradigm of the verb *\ipa{e\_ye} `think' in Siouan languages}  \label{tab:think} \centering
\begin{tabular}{ll|l|ll|llll|ll}
\toprule
 &	 &	Lakhota &	 	Omaha &	Osage 	& Expected Omaha\\	
 \midrule
1sg.A &	*\ipa{e-w-ye} &	 \ipa{epčÁ} &   \ipa{ebðé} &\ipa{ébre} &\ipa{ebðe}\\
2sg.A &*\ipa{e-y-ye}&  &  \ipa{ešné} &? &???& \\
base &	*\ipa{e-ye} &	  &\ipa{eðé} &\ipa{éðe} & *\ipa{eže}\grise{}\\
\bottomrule			
\end{tabular}
\end{table} 

 Thus, in the Dhegiha paradigm for the verb \ipa{ebðé} `I think'     the first person should  be identical to that of an \ipa{r--} stem, since *\ipa{wy} and *\ipa{wr} merge as \ipa{bð} in Omaha and \ipa{br} in Osage according to the correspondences shown in Table \ref{tab:ky}. However,  the third person should not have the same reflexes in Dhegiha:  *\ipa{e-ye} should have yielded *\ipa{eže} instead of the attested form \ipa{eðé} in Dhegiha. The third person \ipa{eðé}  `he thinks' is the expected  form of an *\ipa{r} stem verb, not that of a *\ipa{y} stem verb.
 
 
 For the second person, it is unclear what the outcome of *\ipa{e-y-ye} or *\ipa{e-š-ye} would have been; while  *\ipa{eže}  is the most likely outcome, we cannot exclude the possibility that the second person form  \ipa{ešné} is the regular phonetic reflex from  *\ipa{e-y-ye} (since *\ipa{y} otherwise merges with *\ipa{r} when occurring as second element of clusters), in the absence of other examples of *\ipa{--yy--} in Siouan.

As was briefly suggested in \citet{rankin15csd}, the only available explanation for the irregular \ipa{ð} in the form \ipa{eðé} `he thinks' is analogical levelling based on the model of *\ipa{r} stem paradigms, taking the first person as the pivot form. It is impossible to argue that \ipa{ð} is the regular outcome of *\ipa{y} in Dhegiha in intervocalic position, since clear examples of \ipa{ž} from *\ipa{y} are attested word-internally (see Table \ref{tab:basic}). It is thus a plain example of four-part analogy (\citealt[167-175]{hock91principles}, see Table \ref{tab:four}).

 \begin{table}[h]
\caption{Four-part analogy}  \label{tab:four} \centering
\begin{tabular}{l|l|ll}
\toprule
person & `go' & `think' \\
 \midrule
\textsc{1sg} &\ipa{bðé} & \ipa{ebðé} \\
\textsc{3sg}&\ipa{ðé} & *\ipa{ežé} $\rightarrow$ \ipa{eðé} \\
\bottomrule			
\end{tabular}
\end{table} 


The verb *\ipa{e\_ye} `think' in Siouan was the only syncopating *\ipa{y--} initial stem, and had a unique conjugation. In Lakota it lost the second and third person forms (it is the only defective verb of this type), while in Dhegiha the third and perhaps the second person forms were renewed on the basis of the first person singular.


\section{The directionality of analogy} \label{sec:directionality}

Mississippi Valley Siouan languages attest  a very rare type of analogy, from first person to third person. In Omaha the third person plural  and TAM morphology are marked by suffixes, so that no other form of the paradigm could have served as a basis for the \ipa{ð} in \ipa{éðe} `he thinks'.

An explanation for the preservation of the first person form \ipa{epčÁ} `I think' in Lakhota (and the loss of the rest of the paradigm) and the directionality of analogy in the paradigm of \ipa{ebðé}, \ipa{éðe} `think' in Dhegiha can however be provided by the relative frequency of the first person singular for verbs of cognition such as   `think' and `know'.

 %(unfortunately, no example of  \ipa{epčÁ} `I think' could be found in \citealt{deloria32dakota} to confirm whether this use is attested).\citet{dorsey1890cegiha}
The Omaha corpus (\citealt{dorsey1890cegiha}) was conveniently retranscribed and corrected by Robert Rankin (\citealt{rankin08dhegiha}), and is therefore fully searchable. As shown in Table \ref{tab:cognition}, the most common  verbs of cognition meaning `think' or `know' in Omaha occur more often in the first person singular than in the third person by a considerable margin. This strong tendency is verified in other languages too, as Google counts can easily show for any major language.%; Table \ref{tab:cognition2} presents the token counts of the singular forms of the verb `to think' in French, English and Chinese (in Google).

  \begin{table}[h]
\caption{Number of attestations of verb forms in the Omaha corpus (`think', `suspect' and `know')}  \label{tab:cognition} \centering
\begin{tabular}{l|ll|ll|lll}
\toprule
Person	&think &	&	think, suspect &	&	know &	&	\\
 \midrule
1 & 	\ipa{ebðé(gǫ)}  & 	172 & 	\ipa{éžǫmį} & 	7 & 	\ipa{iðáppahǫ} & 	42 & 	\\
2 & 	\ipa{ehné(gǫ)},  & 	114 & 	 & 	0 & 	\ipa{íšpahǫ} & 	28 & 	\\
& \ipa{ené(gǫ)} or \ipa{ešné(gǫ)} &&&&\\
3 & 	\ipa{eðé(gǫ)} & 	22 & 	 & 	0 & 	\ipa{íbahǫ} & 	29 & 	\\
	\bottomrule			
\end{tabular}
\end{table} 


% \begin{table}[h]
%\caption{Paradigm of the verb `to think' in several languages (in millions of tokens)}  \label{tab:cognition2} \centering
%\begin{tabular}{l|ll|ll|lll}
%\toprule
%Person	&French & & English && Chinese && \\
% \midrule
%1 & \ipa{je pense} & 96.9 & \ipa{I think}  & 1290 & \zh{我想} &40.9 \\
%2 & \ipa{tu penses} & 5.6 & \ipa{you think}  &777& \zh{你想} &23.1 \\
%3 &  \ipa{il/elle pense} & 4.4 & \ipa{He/she thinks}  &61.7& \zh{他/她想} &14.1 \\
%	\bottomrule			
%\end{tabular}
%\end{table} 

The higher frequency of the first person singular form for verbs of cognition offers a plausible explanation for the both the anomalous direction of analogy (from first singular to third singular) and the development of defective verbs only attested in the first singular. It has been observed that `frequency leads to memory strength and fast lexical access, so that frequent items are less susceptible to analogical levelling' (\citealt[276]{haspelmath10morpho}). Therefore, it is not altogether surprising that different frequency patterns lead to different directions of analogical levelling.

As far as I know, the only documented case of analogical levelling where the third person singular is remade on the basis of the first outside of Siouan is found in English. Some dialects of  English appear to have generalized the first singular form \ipa{am} of the verb `to be' (\citealt[163]{dillard75black}; it is in particular the dialect in which Elliot Blaine Henderson's poems were written).\footnote{I am indebted to two IJAL anonymous reviewers who both pointed out this example independently of each other.} It is unclear whether the frequency hypothesis proposed here to account for the paradigm of the verb `to think' Dhegiha could also be applied to this dialect of English -- this topic has to be left to further research.


\section{Conclusion}
This paper documents one of the very rare uncontroversial example of analogy from first person singular to third person, a counterexample to the generalization that third person forms are less susceptible to undergo analogical levelling than other forms in verbal paradigms.
 
This work also shows that generalizations about the direction of analogy may not be valid for all verbs (or any particular part of speech), but that some specific semantic classes (such as for instance cognition verbs) can display exceptions due to frequency effects. 
 
\bibliographystyle{unified}
\bibliography{bibliogj}

 \tableofcontents
\end{document}