\documentclass[oldfontcommands,oneside,a4paper,11pt]{article} 
\usepackage{fontspec}
\usepackage{natbib}
\usepackage{booktabs}
\usepackage{xltxtra} 
\usepackage{longtable}
\usepackage{polyglossia} 
\usepackage[table]{xcolor}
\usepackage{gb4e} 
\usepackage{multicol}
\usepackage{graphicx}
\usepackage{float}
\usepackage{hyperref} 
\hypersetup{bookmarks=false,bookmarksnumbered,bookmarksopenlevel=5,bookmarksdepth=5,xetex,colorlinks=true,linkcolor=blue,citecolor=blue}
\usepackage[all]{hypcap}
\usepackage{memhfixc}
\usepackage{lscape}
 \usepackage{lineno}
\bibpunct[: ]{(}{)}{,}{a}{}{,}

\setmainfont[Mapping=tex-text,Numbers=OldStyle,Ligatures=Common]{Charis SIL} 
\newfontfamily\phon[Mapping=tex-text,Ligatures=Common,Scale=MatchLowercase,FakeSlant=0.3]{Charis SIL} 
\newcommand{\ipa}[1]{{\phon \mbox{#1}}} %API tjs en italique
\newcommand{\ipab}[1]{{\scriptsize \phon#1}} 

\newcommand{\grise}[1]{\cellcolor{lightgray}\textbf{#1}}
\newfontfamily\cn[Mapping=tex-text,Ligatures=Common,Scale=MatchUppercase]{MingLiU}%pour le chinois
\newcommand{\zh}[1]{{\cn #1}}

\newcommand{\sg}{\textsc{sg}}
\newcommand{\pl}{\textsc{pl}}
\newcommand{\ro}{$\Sigma$}
\newcommand{\ra}{$\Sigma_1$} 
\newcommand{\rc}{$\Sigma_3$}  


\XeTeXlinebreakskip = 0pt plus 1pt %
 %CIRCG
 


\begin{document} 
\linenumbers
\title{Generic marking in Japhug and other Rgyalrong languages }
\author{Guillaume Jacques}
\maketitle
%\linenumbers
%Direct-inverse and generic marking in Rgyalrong languages
%Abstract: Rgyalrong languages use a set of two prefixes for marking generic arguments; one is the inverse prefix (otherwise used for 3>2, 3>1, 3'>3 and in some languages 2>1 scenarios) and another one is a prefix phonologically identical to the accusative subject (S/A) nominalization prefix. Yet, the function of these obviously related morphemes is different across the four Rgyalrong languages. In Japhug, the inverse is used for A generic argument, while the other generic prefix is used for S/P; the system thus displays  ergative alignment. 
%The use of the inverse for A generic argument suggests that the empathy hierarchy in Japhug can be described as: (1) SAP > 3 animate > 3 inanimate > generic human In the related Zbu and Tshobdun languages, the functions of the inverse and of the other generic prefixes are reversed: the inverse is restricted to P generic arguments, while the other prefixis used for S and A; the system thus shows accusative alignment. 
%In this paper, I explain how the Rgyalrong system of generic person marking came into being from two entirely distinct constructions. The nominalization S/A prefix was reanalyzed as a generic marker following the reanalysis of subordinate nominalized clauses as main clauses. The inverse 'obviative> proximate' 3'>3 form was reanalyzed as a generic P form. Thus, Tshobdun and Zbu are more conservative than Japhug in this regards and their generic marking represents the proto-Rgyalrongic stage; Japhug underwent remodeling from accusative to ergative alignment in its encoding of generic person, leading to the anomalous hierarchy in (1).
 
\section{Introduction}
Rgyalrong languages\footnote{Rgyalrong languages are spoken in some areas of the Rngaba Tibetan autonomous district, Sichuan, China. There are at least four distinct languages: Situ, Japhug, Tshobdun and Zbu, each of which comprises many distinct dialects.} are unique in the Sino-Tibetan family in having a quasi-canonical direct / inverse system (\citealt{delancey81direction}, \citealt{jackson02rentongdengdi}, \citealt{jacques10inverse}, \citealt{gongxun14agreement}, \citealt{jacques14inverse}). 

In Rgyalrong languages, the inverse is obligatory in 3$\rightarrow$SAP and prohibited in SAP$\rightarrow$3 scenarios (cf Table \ref{tab:japhug.tr} in section \ref{sec:japhug}). It is also found in  2$\rightarrow$1 forms in all Rgyalrong languages except Japhug) and is optional in 3$\rightarrow$3 forms, depending on various parameters discussed  in section \ref{sec:obv.jpg}.\footnote{Closely related languages such as Khroskyabs have a prefix cognate to the Rgyalrong inverse, but it has been generalized in all transitive non-local 3$\rightarrow$3 forms (\citealt{lai13affixale}).}

In addition, the inverse prefix is also used to mark generic person in Japhug and Tshobdun  (\citealt{jacques12demotion},\citealt{sun14generic}). However, these two languages differ as to which syntactic function is targeted by the inverse: in Japhug the inverse is used to mark generic A, while in Tshobdun it marks generic P. 

  This paper is divided in three sections. First, I present an account of the use of the inverse prefix in Japhug, and describe the system of generic person marking in this language. Second, I  discuss data from other Rgyalrong languages, especially Situ and Tshobdun. Third, I compare the available data on Rgyalrong languages and propose   a historical scenario to explain how the distribution of the inverse marker came be to be the way it is in Japhug and Tshobdun.
 
 
\section{Japhug} \label{sec:japhug}
Table \ref{tab:japhug.tr} (from  \citealt{jacques10inverse}) presents the Japhug non-past transitive and intransitive paradigms. The symbol \ro{} represent the verb stem. In Japhug, intransitive verbs do not have stem alternations in the non-past, while some transitive verbs have two stems (called here stem 1 and stem 3 following \citealt{jackson00sidaba}'s terminology; stem 2 is the perfective stem and will not be discussed here, as it is irrelevant to the question of person marking). Stem 3 is restricted to \textsc{sg}$\rightarrow$3 forms; all other direct forms are identical to the corresponding forms of the intransitive paradigm.


\begin{table}[H]
\caption{Japhug transitive and intransitive paradigms}\label{tab:japhug.tr}
\resizebox{\columnwidth}{!}{
\begin{tabular}{l|l|l|l|l|l|l|l|l|l|l|}
\textsc{} & 	\textsc{1sg} & 	  \textsc{1du} & 	\textsc{1pl} & 	\textsc{2sg} & 	\textsc{2du} & 	\textsc{2pl} & 	\textsc{3sg} & 	\textsc{3du} & 	\textsc{3pl} & 	\textsc{3'} \\ 	
\hline
\textsc{1sg} & \multicolumn{3}{c|}{\grise{}} &	\ipa{} & 	\ipa{} & 	\ipa{} & 	\ipa{\rc{}-a}   & 	 \ipa{\rc{}-a-ndʑi} & 	 \ipa{\rc{}-a-nɯ} & 	\grise{} \\	
\cline{8-10}
\textsc{1du} & 	\multicolumn{3}{c|}{\grise{}} &	\ipa{ta-\ra{}} & 	\ipa{ta-\ra{}-ndʑi} & 	\ipa{ta-\ra{}-nɯ} & 	\multicolumn{3}{c|}{ \ipa{\ra{}-tɕi}}  & 	\grise{} \\	
\cline{8-10}
\textsc{1pl} & 	\multicolumn{3}{c|}{\grise{}} & 	  & 	&  & 	\multicolumn{3}{c|}{ \ipa{\ra{}-ji}}  & 	\grise{} \\	
\cline{1-10}
\textsc{2sg} & 	\ipa{kɯ-\ra{}-a} & 	\ipa{} & 	\ipa{} & 	\multicolumn{3}{c|}{\grise{}}&	\multicolumn{3}{c|}{\ipa{tɯ-\rc{}}} & 	\grise{} \\	
\cline{2-2}
\cline{8-10}
\textsc{2du} & 	\ipa{kɯ-\ra{}-a-ndʑi} & 	\ipa{kɯ-\ra{}-tɕi} & 	\ipa{kɯ-\ra{}-ji} & 	\multicolumn{3}{c|}{\grise{}} &	\multicolumn{3}{c|}{\ipa{tɯ-\ra{}-ndʑi}} & 	\grise{} \\	
\cline{2-2}
\cline{8-10}
\textsc{2pl} & 	\ipa{kɯ-\ra{}-a-nɯ} & 	\ipa{} & 	\ipa{} & 	\multicolumn{3}{c|}{\grise{}}&	\multicolumn{3}{c|}{\ipa{tɯ-\ra{}-nɯ}} & 	\grise{} \\	
\hline
\textsc{3sg} &  	\ipa{wɣɯ́-\ra{}-a} & 	\ipa{} & 	\ipa{} & 	\ipa{} & 	\ipa{} & 	\ipa{} & \multicolumn{3}{c|}{\grise{}} &	\ipa{\rc{}} \\ 	
\cline{2-2}
\cline{11-11}
\textsc{3du} &  	\ipa{wɣɯ́-\ra{}-a-ndʑi} & 	 \ipa{wɣɯ́-\ra{}-tɕi} & 		\ipa{wɣɯ́-\ra{}-ji} & 	\ipa{tɯ́-wɣ-\ra{}} & 	\ipa{tɯ́-wɣ-\ra{}-ndʑi} & 	\ipa{tɯ́-wɣ-\ra{}-nɯ} & 	\multicolumn{3}{c|}{\grise{}} &	\ipa{\ra{}-ndʑi} \\ 
\cline{2-2}	
\cline{11-11}
\textsc{3pl} &  	\ipa{wɣɯ́-\ra{}-a-nɯ} & 	\ipa{} & 	\ipa{} & 	\ipa{} & 	\ipa{} & 	\ipa{} & \multicolumn{3}{c|}{\grise{}} &	\ipa{\ra{}-nɯ} \\ 	
\hline
\textsc{3'} & 	\multicolumn{6}{c|}{\grise{}} &	\ipa{wɣɯ́-\ra{}} & 	\ipa{wɣɯ́-\ra{}-ndʑi} & 	\ipa{wɣɯ́-\ra{}-nɯ} & 	\grise{} \\	
	\hline	\hline
\textsc{intr}&\ipa{\ra{}-a}&\ipa{\ra{}-tɕi}&\ipa{\ra{}-ji}&\ipa{tɯ-\ra{}}&\ipa{tɯ-\ra{}-ndʑi}&\ipa{tɯ-\ra{}-nɯ}&\ipa{\ra{}}&\ipa{\ra{}-ndʑi} &\ipa{\ra{}-nɯ}& 	\grise{} \\	
	\hline
\end{tabular}}
\end{table}


The distribution of the inverse prefix in local and mixed scenarios has been discussed in previous publications (\citealt{jacques10inverse} and \citealt{jacques14inverse}) and is not treated in the present paper. In the following sections, I focus exclusively on the use of the inverse in non-local scenarios, including its function as generic person marker for the A.

\subsection{Inverse in non-local scenarios} \label{sec:obv.jpg}
In Japhug, the presence of the inverse marking in  3$\rightarrow$3 scenarios is determined by semantic and pragmatic factors. 

The only case where the inverse is required in non-local scenarios is in sentences with an inanimate agent acting upon an animate patient, be it an animal (as in \ref{ex:YAwGsWGYARnW}) or a human (as in \ref{ex:taR.nA.taR}). In both examples suppressing the inverse would result in a non-grammatical sentence.  
 

 \begin{exe}
\ex \label{ex:YAwGsWGYARnW} 
\gll
yancong	\ipa{ɯ-ŋgɯ} 	\ipa{ɲɯ-ɲaʁ} 	\ipa{rcanɯ,} 	\ipa{tɕe} 	\ipa{kumpɣɤtɕɯ} 	\ipa{ra} 	\ipa{ɲɤ́-wɣ-sɯɣ-ɲaʁ-nɯ} 	\ipa{ʑo,} 	\ipa{nɯ-kɯ-ɤkhra} 	\ipa{ra} 	\ipa{mɯ-ɲɤ-χsɤl} 	\ipa{ʑo} 	\ipa{tɕendɤre} 	\ipa{ʑɯrɯʑɤri} 	\ipa{qale} 	\ipa{tu-βze,} 	\ipa{tɯmɯ} 	\ipa{kɯ} 	\ipa{pjɯ́-wɣ-χtɕi-nɯ} 	\ipa{tɕe} 	\ipa{ɲɯ-me} 	\ipa{ɲɯ-ŋu} \\
chimney \textsc{3sg}-inside \textsc{ipfv}-be.black \textsc{top} \textsc{lnk} sparrow \textsc{pl} \textsc{evd-inv-caus}-be.black-\textsc{pl} \textsc{emph} \textsc{3pl.poss-nmlz}:S-be.colourful \textsc{pl} \textsc{neg-evd}-be.clear \textsc{emph} \textsc{lnk} progressively wind \textsc{ipfv}-make[III] rain \textsc{erg} \textsc{ipfv-inv}-wash-\textsc{pl} \textsc{lnk} \textsc{ipfv}-not.exist \textsc{testim}-be \\
\glt As it is black inside the chimney, the sparrows were completely blackened by it, the patterns and colours (on their feathers) were not visible any more, but progressively, the wind and the rain washed them and (the soot on their feather) disappeared.
 (Sparrows, 74-76)
\end{exe}

 \begin{exe}
\ex \label{ex:taR.nA.taR} 
\gll
\ipa{taʁ}   	\ipa{nɤ}   	\ipa{taʁ,}   	\ipa{taʁ}   	\ipa{nɤ}   	\ipa{taʁ}   	\ipa{tó-wɣ-tsɯm}   \\
up \textsc{lnk} up up \textsc{lnk} up \textsc{evd:up-inv-}take.away \\
\glt He was (progressively)  carried up (by the water). (The flood3, 21)
\end{exe}

Conversely, the inverse cannot be used in sentence a non-generic animate agent acting upon an inanimate.

When both agent and patient are animate, both inverse or direct forms can be used. Even in the case of a non-human agent acting upon a human patient, the inverse is not required, as is shown by example \ref{ex:thaCkWt}.

 
\begin{exe}
\ex \label{ex:thaCkWt}
\gll  	\ipa{ndʑi-sɤtɕha} 	\ipa{nɯnɯ} 	\ipa{ɣɯ} 	\ipa{jil} 	\ipa{nɯnɯ} 	\ipa{rcanɯ} 	\ipa{khu} 	\ipa{kɯ} 	\ipa{lonba} 	\ipa{ʑo} 	\ipa{tha-ɕkɯt} 	\ipa{ɲɯ-ŋu}  \\
\textsc{3du.poss}-place \textsc{top} \textsc{gen} villager \textsc{top} \textsc{top:unexp} tiger \textsc{erg} all \textsc{emph} \textsc{pfv:dir:3}$\rightarrow$3-eat \textsc{testim}-be \\
 \glt All the villagers in their land had been eaten by a tiger.  (The tiger, 5)
\end{exe}


When both arguments are inanimates, the direct form is generally used, but I found one exceptional example with the inverse in the Japhug corpus (\ref{ex:YWwGzmaqhu}, describing the effect of big trees on smaller trees).

\begin{exe}
\ex \label{ex:YWwGzmaqhu}
\gll
 \ipa{ɯ-rkɯ} 	\ipa{nɯ} 	\ipa{tɕu,} 	\ipa{si} 	\ipa{kɯ-wxti} 	\ipa{a-pɯ-tu} 	\ipa{tɕe} 	\ipa{ɲɯ́-wɣ-z-maqhu} 	\ipa{qhe} 	\ipa{ɯʑo} 	\ipa{tu-mbro} 	\ipa{mɯ́j-cha.} 	\\
\textsc{3du.poss}-side \textsc{top} \textsc{loc} tree \textsc{nmlz}:S-be.big \textsc{irr-ipfv}-exist \textsc{lnk} \textsc{ipfv-inv-caus}-be.after \textsc{lnk} \textsc{3sg} \textsc{ipfv}-be.big \textsc{neg:testim}-can \\
\glt If there is a big tree_j next to it_i, it_j delays its_i growth and it_i cannot grow very big. (laŋlaŋ, 242)
\end{exe}


 Algonquian languages (as well as Mayan, see \citealt{aissen97obviation}) are known to have a constraint whereby the inverse configuration is required when the possessee acts upon the possessor. No such phenomenon is found in Japhug, as shown by   example \ref{ex:YAznWqatWkWr}.
\begin{exe}
\ex \label{ex:YAznWqatWkWr}
\gll   \ipa{ji-βdaʁmu} 	\ipa{kɯki,} 	\ipa{ɯ-pi} 	\ipa{kɯ} 	\ipa{ɣɯ-ja-sɯɣe} 	\ipa{tɕe,} 	\ipa{ɲɤ-znɯqatɯkɯr} 	\ipa{ma} \\
\textsc{1pl.poss}-lady \textsc{dem:prox} \textsc{3sg.poss}-elder.sibling \textsc{erg} \textsc{cisloc-pfv:dir:3}$\rightarrow$3-invite \textsc{lnk} \textsc{evd}-give.bad.advice because \\
\glt Our lady, her elder sister invited her and gave her bad advice. (The frog, 147)
\end{exe}

 
When both arguments are non-generic animate, the use of the inverse is determined by the relative saliency of the agent and the patient in the narrative in question. For instance, in \ref{ex:towgtsWm}, two referents are found: the main character (a boy, the last survivor of the flood) and a secondary character (a girl coming from heaven). When the boy is agent and the girl patient, we find the direct form (\ipa{ko-rqoʁ}  `he hug her' without inverse), but in the reversed situation the inverse is required (\ipa{tɤ́-wɣ-tsɯm} `she took him away').\footnote{The absence of inverse in the other verbs in the example is straightforward: the first verb 	\ipa{to-nɯ-ŋga}  `she wore it' has an inanimate patient and the last one \ipa{to-nɯqambɯmbjom}  `she flew up' is intransitive. }

 \begin{exe}
\ex \label{ex:towgtsWm} 
\gll
\ipa{tɕendɤre} 	\ipa{qro} 	\ipa{nɯnɯ} 	\ipa{ɯ-ŋga} 	\ipa{nɯ} 	\ipa{to-nɯ-ŋga} 	\ipa{qhe,}  \ipa{tɕe} 	\ipa{nɯ} 	\ipa{ɯ-mke} 	\ipa{ko-rqoʁ} 	\ipa{qhe,} \ipa{tɕendɤre} 	\ipa{tɤ́-wɣ-tsɯm} 	\ipa{to-nɯqambɯmbjom} 	\ipa{qhe} \\
\textsc{lnk} pigeon \textsc{dem} \textsc{3sg.poss}-clothes \textsc{top} \textsc{evd-auto}-wear \textsc{lnk} \textsc{lnk} \textsc{dem} \textsc{3sg.poss}-neck \textsc{evd}-hug \textsc{lnk} \textsc{lnk} \textsc{pfv-inv}-take.away \textsc{evd:up}-fly \textsc{lnk} \\
\glt She wore the pigeon skin, he hug her, and she took him and flew away (with him). (Flood, 2012.2, 69-70)
\end{exe}

The inverse prefix \ipa{--wɣ--} is not a switch reference marker: it is common to find several verbs in a row with inverse marking sharing the same A and P (if it were switch reference, only the first should be specially marked).

In comparison with Algonquian and Kutenai (\citealt{dryer94inverse}), the inverse is relatively rare in Japhug narratives.

\subsection{Generic marking}  \label{sec:genr.jpg}

In Japhug, non-overt arguments are generally interpreted as definite. In order to express indefinite referents, four strategies are available. 

First, arguments can be demoted by means of argument demoting voice derivations (passive, anticausative,  antipassive and incorporation, see \citealt{jacques12incorp, jacques14antipassive}). Second, a few dozen of transitive verbs present agent-preserving lability, and the semantic patient is interpreted as indefinite whenever verbs in this class are conjugated intransitively (\citealt{jacques12demotion}).  Third, indefinite pronouns such as \ipa{tʰɯci} or generic nouns in plural like \ipa{tɯrme} (with plural marking on the verb) can be used to indicate indefinite referents. Fourth,  specific generic markers \ipa{kɯ--} and \ipa{wɣ--} express generic human arguments, and are the focus of the present section. 

%The generic human verb forms are especially common in procedural texts.


In regular verbs, the generic marker \ipa{kɯ--} is used to designate generic human S or P, as shown by examples \ref{ex:pannWri} and \ref{ex:tukWCWngo} respectively. Verb forms marked with the prefix \ipa{kɯ--} are  finite: unlike nominalized verb forms, they are compatible with all TAM categories. Yet, transitive verb forms with the prefix \ipa{kɯ--} cannot bear any person or number marker referring to the A, which is always third person and definite.


\begin{exe}
\ex  \label{ex:pannWri}
\gll
\ipa{tɕe}  	[\ipa{tɯ-sɯm}  	\textbf{\ipa{pɯ-a<nɯ>ri}}]  	\ipa{nɤ}  	\ipa{ju-kɯ-ɕe,}  	[\textbf{\ipa{mɯ-pɯ-a<nɯ>ri}}]  	\ipa{nɤ}  	\ipa{ju-kɯ-ɕe}  	\ipa{pɯ-ra}  \\
\textsc{lnk} \textsc{indef.poss}-mind  \textsc{pfv-<auto>}go[II] \textsc{lnk} \textsc{ipfv-genr}:S/P-go \textsc{neg-pfv-<auto>}go[II] \textsc{lnk} \textsc{ipfv-genr}:S/P-go \textsc{pst.ipfv}-have.to \\
\glt Whether one liked it or not, one had to go. (Relatives, 212)
\end{exe}


\begin{exe}
\ex \label{ex:tukWCWngo}
\gll  \ipa{tɕe} 	\ipa{ʁja} 	\ipa{nɯnɯ} 	\ipa{tɯ-qʰoχpa} 	\ipa{a-mɤ-tʰɯ-ɕe} 	\ipa{ra} 	\ipa{ma} 	\ipa{tu-kɯ-ɕɯ-ngo} 	\ipa{ɲɯ-ɕti} \\
\textsc{lnk} rust \textsc{top} \textsc{indef.poss}-inner.organ \textsc{irr-neg-pfv:downstream}-go \textsc{fact}:have.to \textsc{ipfv-genr:S/P-caus}-be.sick  \textsc{testim}-be:\textsc{assert} \\
\glt Rust should not go into one's organs, otherwise it would cause one to get sick. (Iron, 86)
\end{exe}

In sentences with verbs in the generic form, the generic  human referent is either non-overt or expressed by the noun \ipa{tɯrme} `people, man'. Examples \ref{ex:kukWnWfse} and \ref{ex:tuwGndza.sna} respectively illustrate this noun used to  refer to P and A generic human respectively; in the second case, it   compulsorily receives  ergative flagging \ipa{kɯ} like any   noun phrase.

\begin{exe}
\ex \label{ex:kukWnWfse}
\gll
\ipa{tɕe}  	\ipa{li}  	\ipa{nɯ}  	\ipa{tɯrme}  	\ipa{kɯnɤ}  	\ipa{ku-kɯ-nɯfse}  	\ipa{ɲɯ-ŋu,}\\
\textsc{lnk} again \textsc{dem} people also \textsc{ipfv-genr:S/P}-recognize \textsc{testim}-be\\
\glt  (The monkey) recognizes people. (monkey, 17)
\end{exe}

\begin{exe}
\ex \label{ex:tuwGndza.sna}
\gll
\ipa{tɯrme}  	\ipa{kɯ}  	\ipa{tú-wɣ-ndza}  	\ipa{mɤ-sna.}   \\
people \textsc{erg} \textsc{ipfv-inv}-eat \textsc{neg-fact}:be.fine \\
\glt It is not edible by people. (khɯrtshɤz, 39)
\end{exe}



Two transitive verbs have irregular generic forms. First, the verb \ipa{ti} `say' has generic A form \ipa{kɯ-ti} instead of expected *\ipa{ɣɯ-ti}. Second, the verb \ipa{sɯz} `know' has a negative generic A \ipa{mɤ-xsi} `one does not know' with a reduced allomorph \ipa{x--} of the \ipa{kɯ--} prefix (on this type of phonological reduction see \citealt{jacques14antipassive}).

Generic arguments are not only indexed by verbal morphology; there is also a generic pronoun \ipa{tɯʑo} `one' and a generic possessor prefix \ipa{tɯ--}.

\begin{exe}
\ex \label{ex:YWwGnWCar}
\gll
\ipa{nɯ} 	\ipa{tɯʑo} 	\ipa{kɯ} 	\ipa{tɯ-χti} 	\ipa{ɲɯ́-wɣ-nɯ-ɕar} 	\ipa{kɯ-maʁ} 	\ipa{kɯ,} 	\ipa{tɯ-phama} 	\ipa{ra} 	\ipa{kɯ} 	\ipa{tɯ-χti} 	\ipa{ɲɯ-ɕar-nɯ} 	\\
\textsc{dem} \textsc{genr} \textsc{erg} \textsc{genr:poss}-spouse \textsc{ipfv-inv-autoben}-search \textsc{inf:stat}-not.be \textsc{erg} \textsc{genr:poss}-parent \textsc{pl} \textsc{erg} \textsc{genr:poss}-spouse \textsc{ipfv}-search-\textsc{pl} \\
\glt One would not search one's spouse, one's parents would search one's spouse. (Relatives, 210)
\end{exe}


\begin{exe}
\ex \label{ex:YWwGnWCar}
\gll
\ipa{tɯʑo-sti}  	\ipa{a-mɤ-nɯ-kɯ-ɤtɯɣ}  	\ipa{ɲɯ-ra}  	\ipa{ma}  	\ipa{ɲɯ-sɤɣ-mu.}  \\
\textsc{genr}-alone \textsc{irr-neg-auto-genr:S/P}-meet \textsc{testim}-have.to because \textsc{testim-deexp}-be.afraid \\
\glt One should not meet (a bear) alone, it is frightening. (Bear, 98)
\end{exe}


Non-possessed nouns can receive the prefix \ipa{tɯ--} when they have a generic human possessor, as in example \ref{ex:tWlaXtCha}. 

\begin{exe}
\ex \label{ex:tWlaXtCha}
\gll 
\ipa{tɕe}  	\ipa{aʁɤndɯndɤt}  	\ipa{ʑo}  	\ipa{ku-zo}  	\ipa{qhe}  	\ipa{ɯ-qe}  	\ipa{ku-lɤt}  	\ipa{qhe}	\ipa{wuma}  	\ipa{ʑo}  	\ipa{tɯ-kha}  	\ipa{cho}  	\ipa{tɯ-laχtɕha}  	\ipa{ra}  	\ipa{sɯ-ɴqhi.}  \\
\textsc{lnk} everywhere \textsc{emph} \textsc{ipfv}-land \textsc{lnk} \textsc{3sg.poss}-feces \textsc{ipfv}-throw \textsc{lnk} really \textsc{emph} \textsc{indef.poss}-house \textsc{comit} \textsc{indef.poss}-thing \textsc{pl} \textsc{fact:caus}-be.dirty \\
\glt (Flies) land everywhere, shit on it and make ones' houses and things dirty. (Flies, 59)
\end{exe}




\begin{exe}
\ex
\gll
\ipa{nɯ} 	\ipa{kɯ-fse} 	\ipa{tɕe} 	\ipa{tɯʑo} 	\ipa{tɯ-rɟit} 	\ipa{kɯnɤ} 	\ipa{ʑa} 	\ipa{mɤ-sci} 	\ipa{tu-ti-nɯ} \\
\textsc{dem} \textsc{nmlz}:S-be.like \textsc{lnk} \textsc{indef} \textsc{indef.poss}-child also early \textsc{neg-fact}:be.born \textsc{ipfv}-say-\textsc{pl} \\
\glt People say that in this way, one's child will be born late. (Deer, 111)
\end{exe}



Possessed nouns differ from non-possessed nouns in that they obligatorily take either a indefinite  (\ipa{tɯ--} or \ipa{tɤ--})  or a definite possessive prefix (\ipa{a--} `my', \ipa{nɤ--} `your' etc; see \citealt{jacques14antipassive}). In Japhug, there is only one series of definite possessive prefixes; the distribution of the two indefinite possessive prefixes \ipa{tɯ--} and \ipa{tɤ--} is lexically determined (for instance, \ipa{--jaʁ} `hand' has the indefinite form \ipa{tɯ-jaʁ} `a hand' while \ipa{--tɕɯ} `son' has the form \ipa{tɤ-tɕɯ} `a boy'). Some possessed nouns only allow definite possessive prefixes.

While the indefinite possessive \ipa{tɯ--} and the generic possessive \ipa{tɯ--} are phonetically identical and functionally very close, they are nevertheless distinct. The contrast between the two is only visible in the case of possessed nouns with an indefinite possive form in \ipa{tɯ--}, as both forms are possible, compare for instance \ipa{a-rpɯ} `my uncle (mother's brother)', \ipa{tɤ-rpɯ} `an uncle' with \ipa{tɯ-rpɯ} `one's uncle', in examples such as: 


 



%Only one referent per sentence may be marked 
%
%
%over tɯrme 
%\citet{jacques10inverse}
%
%tɯrme tɯ-fsu ɕoŋtaʁ tu-mbro mɤ-cha.
% tɯrme ɯ-fsu jamar tu-βze cha.
% tɯrme ɣɯ tɯ-ɕa ɯ-mdoʁ tsa asɯ-ndo kɯ-fse
 
 

\section{Other Rgyalrong languages}
\subsection{Situ}
\citet{linxr93jiarong}
\citet[163]{lin09phd}

Third person plural

\citet[101-102]{jacques12agreement}
\subsection{Tshobdun}
\citet{sun14generic}
\section{A historical account}


\bibliographystyle{linquiry2}
\bibliography{bibliogj}
\end{document}