
\documentclass[oldfontcommands,oneside,a4paper,11pt]{article} 
\usepackage{fontspec}
\usepackage{natbib}
\usepackage{booktabs}
\usepackage{xltxtra} 
\usepackage{longtable}
\usepackage{polyglossia} 
\usepackage[table]{xcolor}
\usepackage{gb4e} 
\usepackage{multicol}
\usepackage{graphicx}
\usepackage{float}
\usepackage{hyperref} 
\hypersetup{bookmarks=false,bookmarksnumbered,bookmarksopenlevel=5,bookmarksdepth=5,xetex,colorlinks=true,linkcolor=blue,citecolor=blue}
\usepackage[all]{hypcap}
\usepackage{memhfixc}
\usepackage{lscape}
 \usepackage{lineno}
\bibpunct[: ]{(}{)}{,}{a}{}{,}

\setmainfont[Mapping=tex-text,Numbers=OldStyle,Ligatures=Common]{Charis SIL} 
\newfontfamily\phon[Mapping=tex-text,Ligatures=Common,Scale=MatchLowercase,FakeSlant=0.3]{Charis SIL} 
\newcommand{\ipa}[1]{{\phon \mbox{#1}}} %API tjs en italique
\newcommand{\ipab}[1]{{\scriptsize \phon#1}} 

\newcommand{\grise}[1]{\cellcolor{lightgray}\textbf{#1}}
\newfontfamily\cn[Mapping=tex-text,Ligatures=Common,Scale=MatchUppercase]{MingLiU}%pour le chinois
\newcommand{\zh}[1]{{\cn #1}}



\XeTeXlinebreaklocale 'zh' %使用中文换行
\XeTeXlinebreakskip = 0pt plus 1pt %
 %CIRCG
 


\begin{document} 
\linenumbers
\title{Generic marking in Japhug and other Rgyalrong languages }
\author{Guillaume Jacques}
\maketitle
%\linenumbers

 
\section{Introduction}
Rgyalrong languages\footnote{Rgyalrong languages are spoken in some areas of the Rngaba Tibetan autonomous district, Sichuan, China. There are at least four disntinct languages: Situ, Japhug, Tshobdun and Zbu, each of which comprises many distinct dialects.} are unique in the Sino-Tibetan family in having a quasi-canonical direct / inverse system (\citealt{delancey81direction}, \citealt{jackson02rentongdengdi}, \citealt{jacques10inverse}, \citealt{gongxun14agreement}, \citealt{jacques14inverse}). 

In addition to... the inverse marker, in Japhug and Tshobdun, is also used as a generic person marker for some specific configurations (\citealt{jacques12demotion},\citealt{sun14generic})

 Not found in related languages such as khroskyabs \citet{lai13affixale}
 
 
\section{Japhug}
\citet{jacques12demotion} \citet{jacques10inverse}
\section{Situ}
\citet{linxr93jiarong}
\citet[163]{lin09phd}

Third person plural

\citet[101-102]{jacques12agreement}
\section{Tshobdun and Zbu}
\citet{sun14generic}
\section{A historical account}


\bibliographystyle{linquiry2}
\bibliography{bibliogj}
\end{document}