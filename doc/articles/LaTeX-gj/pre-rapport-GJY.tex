\documentclass[oldfontcommands,oneside,a4paper,11pt]{article} 
\usepackage{fontspec}
\usepackage{natbib}
\usepackage{booktabs}
\usepackage{xltxtra} 
\usepackage{polyglossia} 
\setdefaultlanguage{french} 
\usepackage[table]{xcolor}
\usepackage{gb4e} 
\usepackage{multicol}
\usepackage{graphicx}
\usepackage{float}
\usepackage{hyperref} 
\hypersetup{bookmarks=false,bookmarksnumbered,bookmarksopenlevel=5,bookmarksdepth=5,xetex,colorlinks=true,linkcolor=blue,citecolor=blue}
\usepackage[all]{hypcap}
\usepackage{memhfixc}
\usepackage{lscape}

\bibpunct[: ]{(}{)}{,}{a}{}{,}

%\setmainfont[Mapping=tex-text,Numbers=OldStyle,Ligatures=Common]{Charis SIL} 
\newfontfamily\phon[Mapping=tex-text,Ligatures=Common,Scale=MatchLowercase,FakeSlant=0.3]{Charis SIL} 
\newcommand{\ipa}[1]{{\phon \mbox{#1}}} %API tjs en italique
\newcommand{\ipab}[1]{{\scriptsize \phon#1}} 

\newcommand{\grise}[1]{\cellcolor{lightgray}\textbf{#1}}
\newfontfamily\cn[Mapping=tex-text,Ligatures=Common,Scale=MatchUppercase]{MingLiU}%pour le chinois
\newcommand{\zh}[1]{{\cn #1}}
\newcommand{\refb}[1]{(\ref{#1})}

\newcommand{\ra}{$\Sigma_1$} 
\newcommand{\rc}{$\Sigma_3$} 
\newcommand{\ro}{$\Sigma$} 

\XeTeXlinebreaklocale 'zh' %使用中文换行
\XeTeXlinebreakskip = 0pt plus 1pt %
 %CIRCG
 
\sloppy

\begin{document} 
\title{Pré-rapport de la thèse présentée par Mlle Gao Jiayin \textit{Interdepence between Tones, Segments and Phonation types in Shanghai Chinese} en vue de l'obtention du doctorat }
\author{Guillaume Jacques, chargé de recherches HDR, CNRS\\rgyalrongskad@gmail.com}
\maketitle

La thèse de Gao Jiayin "Interdepence between Tones, Segments and Phonation types in Shanghai Chinese" présente des contributions à de nombreux domaines de la linguistique. 

Cette thèse est avant tout une étude de phonétique expérimentale, comprenant des analyses acoustiques, des tests de perceptions et des mesures physiologiques (électro-glottographe et mouvement des lèvres) démontrant la maîtrise technique de l'auteur.  L'auteur a effectué des mesures sur des sujets de différentes classes d'âge et de sexe différents, ce qui lui permet de valoriser ces riches données phonétiques en les replaçant dans plusieurs perspectives pertinentes, en particulier  celle de la phonologie panchronique et de la sociolinguistique labovienne. Ce rapport, écrit par un non-phonéticien, s'intéresse en priorité aux contributions de cette thèse à la phonologie et à la sociolinguistique.  

Le sujet traité par la thèse, celui de l'interaction entre tons et voix soufflée en chinois shangaïen, est intéressant pour les phonologues (synchroniciens et diachroniciens) à plus d'un titre. 
La voix soufflée est un trait redondant avec les tons, au point qu'il n'est pas trivial de déterminer quels traits phonologiques distinctifs seraient pertinents pour décrire un tel système. 
La phonologie historique du chinois, domaine hermétique et difficile d'accès sur lequel l'auteur démontre une connaissance détaillée, permet d'aborder cette question dans une perspective diachronique, et en particulier de répondre à deux questions d'intérêt fondamental pour la phonologie (pp72 et 267-272):
(1) Les traits phonétiques redondants sont-ils les effets de la coarticulation, ou sont-ils des traces de changements diachroniques?
(2) Comment les traits redondants et distinctifs évoluent-ils dans la langue actuelle?

L'auteur apporte des réponses détaillées et originales à ces deux questions sur la base de ses données expérimentales. Il était généralement admis par les linguistes spécialistes de phonologie panchronique (Mazaudon 2010) que la relation entre voix soufflée et registre tonal pouvait s'expliquer par un chemin d'évolution phonétique unidirectionnel (3):

(3) voisement > voix soufflée > registre tonal bas

Or, l'auteur montre (pp. 98-99, 106, 259) que les si les occlusives dans une syllabe ayant un ton de série basse (T3) en début de mot sont généralement réalisée sans voisement, il n'en est pas de même des fricatives, qui sont le plus souvent voisées. Or, le voisement des fricatives dans ce contexte ne semble pas être dû à la préservation du voisement du chinois moyen. En effet, le voisement des fricatives à l'initiale de mot est plus fréquent dans la génération de locuteurs de 20-30 ans que celle des 60-80 ans. Les données expérimentales démontrent ainsi que le voisement des fricatives a été restauré en initiale de mots, ce qui va à l'encontre du chemin (3) jusqu'ici considéré comme strictement unidirectionnel. 

L'auteur apporte une hypothèse pour expliquer ce paradoxe. La consonne prononcée [v] en shangaïais a deux origines distinctes en chinois moyen (b et m palatalisés), et ces deux origines étaient encore distinguées au dix-neuvième siècle comme /fɦ/ et /v/, et la confusion aurait eu lieu au profit de /v/. Ce début d'explication, pour être considéré comme une démonstration, demandera des travaux additionnels sur d'autres langues wu, recherches qui dépassent le cadre de cette thèse.

Une autre contribution importante de ce travail est de montrer (p261-2) que la voix soufflée est perdue dans les syllabes à sonantes initiales, et donc que la disparition de la voix soufflée s'opère dans les syllabes à initiales sonantes avant celles à obstruentes. Cette question est d'autant plus intéressante qu'une description récente du dialecte de Chongming \zh{崇明} (il semble d'ailleurs que c'est la région d'origine de certains sujets de cette thèse) présente une voix soufflée avec certaines consonnes initiales sonantes, et une préglottalisation avec d'autres (Zhang Huiying \zh{张惠英} 2009. \zh{崇明方言研究} Chongming fangyan yanjiu. Beijing: Zhongguo shehui kexue chubanshe). Dans un travail futur, il pourrait être utile de confronter les données du shangaïais standard avec celles de ce dialecte et d'expliquer pourquoi la voix soufflée a survécu plus longtemps en shangaïais avec les obstruentes.

Ce travail recèle d'autres remarques intéressantes pour des questions variées de phonologie (par exemple, p264, la découverte que les obstruentes voisées intervocaliques ont un effet dépresseur sur la syllabe précédente seulement dans le cas où le ton est haut).

Cette thèse confirme la tendance générale observée en sociolinguistique (Labov 2001) selon laquelle les femmes d'âge moyen sont à la pointe des innovations phonologiques. Néanmoins, l'auteur (pp 273-4) montre clairement que le mécanisme social de l'innovation linguistique dans le cas du shangaïais présente des particularités distinctes des deux types de changements (``from above" and ``from below") décrits par Labov sur la base des dialectes de l'anglais.

Il ne fait aucun doute de cette thèse est une contribution importante à la phonétique, la phonologie et la sociolinguistique, et mérite tout à fait d'être soutenue.

 
\bibliographystyle{unified}
\bibliography{bibliogj}
\end{document}