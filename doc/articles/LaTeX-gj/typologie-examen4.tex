\documentclass[oldfontcommands,twoside,a4paper,12pt]{article} 
\usepackage{fontspec}
\usepackage{natbib}
\usepackage{booktabs}
\usepackage{xltxtra} 
\usepackage{polyglossia} 
 \usepackage{geometry}
 \geometry{
 a4paper,
 total={210mm,297mm},
 left=10mm,
 right=10mm,
 top=15mm,
 bottom=15mm,
 }
\usepackage[table]{xcolor}
\usepackage{color}
\usepackage{multirow}
\usepackage{gb4e} 
\usepackage{multicol}
\usepackage{graphicx}
\usepackage{float}
\usepackage{hyperref} 
\hypersetup{bookmarks=false,bookmarksnumbered,bookmarksopenlevel=5,bookmarksdepth=5,xetex,colorlinks=true,linkcolor=blue,citecolor=blue}
\usepackage{memhfixc}
\usepackage{lscape}
\usepackage[footnotesize,bf]{caption}
\usepackage{multicol}
 
 
\setmainfont[Mapping=tex-text,Numbers=OldStyle,Ligatures=Common]{Charis SIL} 
\setsansfont[Mapping=tex-text,Ligatures=Common,Mapping=tex-text,Ligatures=Common,Scale=MatchLowercase]{Lucida Sans Unicode} 
 


\newfontfamily\phon[Mapping=tex-text,Ligatures=Common,Scale=MatchLowercase]{Charis SIL} 
\newcommand{\ipa}[1]{{\phon\textbf{#1}}} 

\begin{document}
\title{Typologie : partiel du 18 novembre 2016}
\date{}
\maketitle
 

\section*{Exercice 1 (Omaha)}

\begin{enumerate}
\item Décrivez l'indexation dans cette langue et son alignement (tableau obligatoire) sur la base des exemples \ref{ex:aboN} à \ref{ex:dhii}.
\item Pouvez-vous prévoir, à partir des données à votre disposition, une/plusieurs forme(s) ambiguê(s) dans le paradigme transitif et pourquoi?
\item Comment analyser les formes \ref{ex:naoNkade} à \ref{ex:noNpa}? Permettent-elles d'affiner la description du système d'indexation dans cette langue?
\item Traduisez en omaha:
\end{enumerate}
\begin{multicols}{3}
\begin{itemize}
\item `Tu nous le donnes'
\item `J'ai peur de toi'
\item `Tu as peur de moi'
\item `Il a peur de moi'
\end{itemize}
\end{multicols}
\begin{multicols}{3}
\begin{exe}
\ex \label{ex:aboN}
\glt abõ 
\glt Je crie.
\end{exe} 

\begin{exe}
\ex 
\glt ðabõ 
\glt tu cries.
\end{exe} 
 
 \begin{exe}
\ex 
\glt bõ 
\glt il crie.
\end{exe} 
 
  \begin{exe}
\ex 
\glt õbõ 
\glt nous crions.
\end{exe} 

  \begin{exe}
\ex 
\glt õgðõði
\glt je suis fou.
\end{exe} 

  \begin{exe}
\ex 
\glt ðigðõði
\glt tu es fou.
\end{exe} 

  \begin{exe}
\ex 
\glt gðõði
\glt il est fou.
\end{exe} 

  \begin{exe}
\ex 
\glt wagðõði
\glt nous sommes fou.
\end{exe} 

\begin{exe}
\ex 
\glt aʔi
\glt Je le lui donne.
\end{exe} 

\begin{exe}
\ex 
\glt ðaʔi
\glt tu le lui donnes.
\end{exe} 
 
 \begin{exe}
\ex 
\glt ʔi
\glt il le lui donne.
\end{exe} 

 \begin{exe}
\ex 
\glt wiʔi
\glt je te le donne.
\end{exe} 

 \begin{exe}
\ex 
\glt õðaʔi
\glt tu me le donnes.
\end{exe} 

 \begin{exe}
\ex 
\glt õʔi
\glt il me le donne.
\end{exe} 

 \begin{exe}
\ex \label{ex:dhii}
\glt ðiʔi
\glt il te le donne.
\end{exe} 

 \begin{exe}
\ex \label{ex:naoNkade}
\glt naõkade
\glt j'ai faim.
\end{exe} 


 \begin{exe}
\ex 
\glt naðikade
\glt tu as faim.
\end{exe} 


 \begin{exe}
\ex 
\glt nakada
\glt il a faim.
\end{exe} 

 \begin{exe}
\ex 
\glt nawakada
\glt nous avons faim.
\end{exe} 

 \begin{exe}
\ex 
\glt nõape
\glt j'ai peur de lui.
\end{exe} 

 \begin{exe}
\ex 
\glt nõpa
\glt il a peur de lui.
\end{exe} 

 \begin{exe}
\ex 
\glt nõðipa
\glt il a peur de toi.
\end{exe} 


 \begin{exe}
\ex \label{ex:noNpa}
\glt õnõpa
\glt nous avons peur de lui.
\end{exe} 

\end{multicols}
\end{document}

