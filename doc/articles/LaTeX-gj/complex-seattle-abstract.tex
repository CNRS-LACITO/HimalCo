\documentclass[oldfontcommands,oneside,a4paper,11pt]{article} 
\usepackage{fontspec}
\usepackage{natbib}
\usepackage{booktabs}
\usepackage{xltxtra} 
\usepackage{polyglossia} 
\usepackage[table]{xcolor}
\usepackage{gb4e} 
\usepackage{multicol}
\usepackage{graphicx}
\usepackage{float}
\usepackage{hyperref} 
\hypersetup{bookmarks=false,bookmarksnumbered,bookmarksopenlevel=5,bookmarksdepth=5,xetex,colorlinks=true,linkcolor=blue,citecolor=blue}
\usepackage[all]{hypcap}
\usepackage{memhfixc}
\usepackage{lscape}
 \usepackage{bbding}
 
%\setmainfont[Mapping=tex-text,Numbers=OldStyle,Ligatures=Common]{Charis SIL} 
\newfontfamily\phon[Mapping=tex-text,Ligatures=Common,Scale=MatchLowercase]{Charis SIL} 
\newcommand{\ipa}[1]{{\phon\textbf{#1}}} 
\newcommand{\grise}[1]{\cellcolor{lightgray}\textbf{#1}}
\newfontfamily\cn[Mapping=tex-text,Ligatures=Common,Scale=MatchUppercase]{SimSun}%pour le chinois
\newcommand{\zh}[1]{{\cn #1}}
\newcommand{\Y}{\Checkmark} 
\newcommand{\N}{} 
\newcommand{\dhatu}[2]{|\ipa{#1}| `#2'}
\newcommand{\refb}[1]{(\ref{#1})}

 \begin{document} 
\title{The life cycle of multiple indexation and bipartite verbs in Sino-Tibetan}
\author{Guillaume Jacques, CNRS-CRLAO-INALCO}
\maketitle

\textbf{Fourth Workshop on Sino-Tibetan Languages of Southwest China}

\textbf{University of Washington, Seattle}

\section*{Abstract}
Bipartite verbs, though common in some areas of the world (\citealt{delancey96bipartite}), are relatively rare in Eurasia. In the Trans-Himalayan/Sino-Tibetan family, bipartite verbs are found in Kiranti and Gyalrongic, and present another uncommon typological characteristic, 
multiple argument indexation (\citealt{denk15multiple}).

Bipartite verbs in Japhug and other Gyalrong languages are a small class (only ten such verbs have been discovered up to now). The most common bipartite verb, \ipa{-stu -mbat} `try hard, do one's best' can be conjugated in four distinct ways, as shown in Table \ref{tab:four}.


\begin{table}[h]
\caption{Four degrees of integration} \centering \label{tab:four}
\begin{tabular}{lllllll}
\toprule
Type & Example & V_1 suffix & V_2 prefix \\
\midrule
A& \ipa{tɤ-stu-ndʑi} \ipa{tɤ-mbat-ndʑi} &\Y &\Y \\
 &\textsc{imp}-V_1-\textsc{du}  \textsc{imp}-V_2-\textsc{du} \\
B& \ipa{tɤ-stu-tɤ-mbat-ndʑi} &\N  &\Y \\
 &\textsc{imp}-V_1-\textsc{imp}-V_2-\textsc{du} \\
C& \ipa{tɤ-stu-ndʑi-mbat-ndʑi} &\Y  &\N \\
 &\textsc{imp}-V_1-\textsc{du}-V_2-\textsc{du} \\
B& \ipa{tɤ-stu-mbat-ndʑi} &\N  &\N \\
 &\textsc{imp}-V_1-V_2-\textsc{du} \\
\bottomrule
\end{tabular}
\end{table}

In this paper, I will discuss the morphological pecularities of Japhug bipartite verbs, their historical relationship to serial verb constructions (on which see \citealt{sun12complementation}, \citealt{jacques13harmonization}), how they shed light on the origin of bipartite verbs in Kiranti and help understanding phenomena such as variability in affix ordering (\citealt{bickel07chintang},  \citealt[170-172]{doornenbal09}), and how they can improve our understanding of the diachronic origin of bipartite verbs in general (see \citealt{grossmann17suffix}).

\bibliographystyle{unified}
\bibliography{bibliogj}

 \end{document}
 