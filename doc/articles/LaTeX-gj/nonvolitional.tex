\documentclass[oneside,a4paper,11pt]{article} 
\usepackage{fontspec}
\usepackage{natbib}
\usepackage{booktabs}
\usepackage{xltxtra} 
\usepackage{polyglossia} 
\usepackage[table]{xcolor}
\usepackage{gb4e} 
\usepackage{multicol}
\usepackage{graphicx}
\usepackage{float}
\usepackage{hyperref} 
\hypersetup{bookmarksnumbered,bookmarksopenlevel=5,bookmarksdepth=5,colorlinks=true,linkcolor=blue,citecolor=blue}
\usepackage[all]{hypcap}
\usepackage{memhfixc}
\usepackage{lscape}
\usepackage{amssymb}
 
\bibpunct[: ]{(}{)}{,}{a}{}{,}

%\setmainfont[Mapping=tex-text,Numbers=OldStyle,Ligatures=Common]{Charis SIL} 
\newfontfamily\phon[Mapping=tex-text,Scale=MatchLowercase]{Charis SIL} 
\newcommand{\ipa}[1]{\textbf{{\phon\mbox{#1}}}} %API tjs en italique
%\newcommand{\ipab}[1]{{\scriptsize \phon#1}} 

\newcommand{\grise}[1]{\cellcolor{lightgray}\textbf{#1}}
\newfontfamily\cn[Mapping=tex-text,Ligatures=Common,Scale=MatchUppercase]{SimSun}%pour le chinois
\newcommand{\zh}[1]{{\cn #1}}

 

\XeTeXlinebreakskip = 0pt plus 1pt %
 %CIRCG
 
\begin{document}

\title{A note on volitional and non-volitional prefixes in Gyalrong languages}
\author{Guillaume Jacques}
\maketitle

\section*{Introduction}
Volitionality is an important morphosyntactic parameter in languages of the Tibetosphere (see for instance Amdo Tibetan, \citealt[960-3]{sun93evidentiality}), and Gyalrongic languages are no exception. In a recent publication, \citet[505-6]{prins16kyomkyo} posits the existence of a non-volitional \ipa{m-} prefix in the Situ Gyalrong dialect of Kyomkyo. This is a novel proposal; previous overviews of the verbal  morphology of Gyalrongic languages make no mention of such a prefix (\citealt{jackson03caodeng}, \citealt{jackson06paisheng}, \citealt{jacques12demotion}, \citealt{lai13affixale} \citealt{jackson14morpho},  \citealt{jacques17sketch}).

Given the crucial importance of Gyalrongic languages for comparative Trans-Himalayan / Sino-Tibetan and even Old Chinese reconstruction (\citealt{gong17xingtaixue, gong17clusters}) and linguistic typology (\citealt{jacques13harmonization, jacques14antipassive}), Prins's proposal deserves to be properly discussed and  evaluated.

In this paper, I provide alternative analyses for Prins' examples of non-volitional \ipa{m-} prefix: some are argued to be better analyzed as examples of anticausative prenasalization, others as denominal derivation. I show however that volitionality is indeed marked by derivational morphology in Japhug and other Gyalrongic languages; the autobenefactive-spontaneous prefix (\citealt{jacques16japhug}) does have non-volitional overtones, and a volitional \ipa{mɯ-} prefix appears to be attested in one example (\ipa{mɯ-nmu} `move').

\section{Anticausative prenasalization}
As primary examples for the non-volitional \ipa{m-} prefix, Prins provides the examples in Table \ref{tab:nonvol}.\footnote{The transcription is the same as in the source, but the infinitive \ipa{ka-} prefix has been remove to increase readability.}

Table \ref{tab:nonvol2} presents corresponding data in the Cogtse dialect of Situ, using Huang and Sun's (\citeyear{huangsun02}) dictionary.\footnote{The transcription is slightly adapted, tones 55 and 53 on the last syllable of a verb form are converted to the falling tone \ipa{v̂}, and tone 33 to the level tone \ipa{v̄}, in accordance with more recent accounts of Situ tonology (\citealt{jackson05yingao}, \citealt{linyj12tone}). Infinitive prefix are also removed.}

\begin{table}[h]
\caption{Examples of \ipa{m-} non-volitional prefix in Situ Gyalrong according to \citet[505]{prins16kyomkyo}. } \label{tab:nonvol} \centering
\begin{tabular}{lllllllll}
\toprule
Volitional & Non-Volitional \\
\midrule
\ipa{nəscar} `frighten' & \ipa{ʒder} `fear'  \\
\ipa{tɽap} `push down, cause to fall' & \ipa{ndɽaʔp} `tumble, stumble'  \\
\ipa{pʰət} `pull down, throw' & \ipa{mbət} `fall'  \\
\ipa{pʰek} `split in two' (vt)& \ipa{mbek} `split in two' (vi)  \\
\ipa{pjoʔt} `fill up' & \ipa{məjot} `full'  \\
\bottomrule
\end{tabular}
\end{table}

\begin{table}[h]
\caption{Corresponding verbs in the Cogtse dialect of Situ (data from \citealt{huangsun02})} \label{tab:nonvol2} \centering
\begin{tabular}{lllllllll}
\toprule
&&Page \\
\midrule 
 \ipa{nɑstɕɑ̂r}  `be startled' (\zh{受惊}) &  \ipa{ʑdɑ̂r}  `fear' (\zh{害怕}) &  469;178  \\
\ipa{tʂɑ̄p}   `cause to roll down' (\zh{滚动、翻滚}) &  \ipa{ndʐɑ̄p} `roll down' (\zh{滚动})   &  172  \\
 \ipa{pʰôt} `fell (tree)' (\zh{砍伐}) & \ipa{mbôt} `collapse' (\zh{塌}) & 271; 489 \\
 \ipa{pʰâk} `split (wood)' (\zh{劈}) & \ipa{mbâk} `split' (vi) (\zh{裂开}) & 380; 309 \\
\ipa{pjōt} `fill' (\zh{装满}) & \ipa{mɲōt} `full' (\zh{满}) & 330; 670 \\
\bottomrule
\end{tabular}
\end{table}

 The first pair of verbs, \ipa{nəscar} `frighten' vs \ipa{ʒder} `fear', should be discarded from this list, as it does not display any evidence for a \ipa{m-} prefix, and since these verbs are actually historically unrelated.\footnote{The first one is actually a denominal verb; its Japhug cognate \ipa{nɤscɤr} `be startled' derives from the noun \ipa{tɤscɤr} `fear'. No known morphological process in Gyalrongic languages could lead to an alternation between \ipa{sc} and \ipa{ʑd}.}

The remaining fours pairs, rather than being analyzed as examples of a non-volitional \ipa{m-} prefix, actually illustrate \textsc{anticausative prenasalization}, a morphological process well attested throughout the Trans-Himalayan family (see \citealt{sagart03prenasalized} on Old Chinese for instance). In Gyalrongic languages, this alternation has been described in detail in several publications (\citealt{lai13affixale}, \citealt{jacques15spontaneous}, \citealt{jacques15causative}). 

This alternation, which derives a transitive verb into an intransitive one, turns the initial unvoiced aspirated and unaspirated stops and affricates into corresponding prenasalized one.\footnote{There are three independent pieces of evidence showing that the directionality of this derivation is from transitive to intransitive, and not the other way round. First, one can derive the intransitive forms from the transitive ones, but not the other way round, since aspirated and unaspirated obstruents correspond both to prenasalized ones. Second, in Japhug the borrowed Tibetan verb \ipa{χtɤr} `spill' has an intransitive counterpart \ipa{ʁndɤr} `be spilled' which underwent this prenasalizing alternation (there are no verbs in any variety of Tibetan which could yield the form \ipa{ʁndɤr}, as the cluster \ipa{ʁnd-} does not conform to the phonotactics of Tibetan clusters). Third, the notion that in verb pairs of this type, the transitive verb derives from the intransitive one by means of an causative \ipa{s-} prefix is incompatible with the fact that in Gyalrongic, the causative is still productive and can in some cases be added even to anticausative verbs, as \ipa{ftʂi} `melt' (vt) $\rightarrow$ \ipa{ndʐi} `melt' (vi, anticausative) $\rightarrow$ \ipa{sɯɣ-ndʐi} `melt' (vt, anticausative+causative). See \citealt{jacques15spontaneous}, \citealt{jacques15causative} and \citealt{gong17xingtaixue} for a more detailed discussion. } In this derivation, the nasal element is always homorganic with the following voiced stops, and there is agreement among Gyalrongologists that prenasalized voiced stops should actually be analyzed as unitary phonemes (see \citealt{jackson03caodeng}, \citealt[8-9]{jacques08}).

The fact that this derivation should not be analyzed as \ipa{m-} prefixation, but as an alternation between unvoiced obstruents and prenasalized one is shown by the fact that we observe \ipa{m-} in the intransitive verbs only when their transitive counterparts have a labial initial consonant, as shown by the examples from \citet[193-4]{linxr93jiarong} in Table \ref{tab:anticaus} (converted to the transcription system of \citealt{huangsun02}), and by Prins' own example \ipa{tɽap} $\rightarrow$ \ipa{ndɽaʔp} (not $\dagger$\ipa{mdɽaʔp}).


\begin{table}[h]
\caption{Additional examples of anticausative derivations in the Cogtse dialect of Situ} \label{tab:anticaus} \centering
\begin{tabular}{lllllllll}
\toprule
Base verb &Anticausative &Page \\
\midrule 
\ipa{tɕʰôp} `shatter' (vt) (\zh{使破成碎块}) & \ipa{ndʑōp} `shatter' (vi) (\zh{破碎}) & 388-9 \\
\ipa{tɕōp} `burn' (vt) (\zh{烧}) & \ipa{ndʑōp} `burn' (vi) (\zh{烧着}) & 444-5 \\
\ipa{klɑ̂k} `erase' (\zh{抹掉}) & \ipa{ŋglɑ̂k} `drop' (vi) (\zh{脱落}) & 347;514 \\
\ipa{krɑ̂k} `demolish (house)' (vt) (\zh{破坏}) & \ipa{ŋgrɑ̂k} `be demolished' (vi) (\zh{破败}) & 388 \\
\bottomrule
\end{tabular}
\end{table}

The last example \ipa{pjoʔt} `fill up' $\rightarrow$ \ipa{məjot} `full'  deserves a more detailed discussion. 

First, notice that some dialects of Situ  have the initial cluster \ipa{mɲ-} for the anticausative verb derived from   \ipa{pjōt} `fill' (Cogtse \ipa{mɲōt}  `full'), while others, like Kyomkyo, have instead a prefixal element \ipa{mə-} followed by \ipa{j-} (it is also the case in Brag-dbar where we find \ipa{məjōt} `plein', see \citealt[77]{zhang16bragdbar}). Since dialects like Cogtse have simple clusters \ipa{mC-} corresponding to either \ipa{məC-} or \ipa{mC-} in Kyomkyo and Bragdbar (see Table \ref{tab:mEC}).\footnote{Bragdbar data (from \citealt{zhang16bragdbar}) are used here due to greater accessibility, but agree in all cases with Kyomkyo (when the data is available) as regards to the presence or absence of a schwa. } It is therefore more likely to assume that Kyomkyo and Bradg-dbar are more archaic in preserving presyllables, while dialects like Cogtse have lost the vowel and created secondary clusters.

Second, the expected intransitive form of \ipa{pjōt} `fill' should be $\dagger$\ipa{mbjōt}, not attested \ipa{məjōt}. Since the cluster \ipa{mbj-} exists in Kyomkyo, there are no phonotactic reasons why $\dagger$\ipa{mbjōt} would be avoided. 

\begin{table}[H]
\caption{The contrast between \ipa{mC-} and \ipa{məC-} in Bragdbar} \label{tab:mEC} \centering
\begin{tabular}{lllll}
\toprule
Cogtse & Bragdbar   \\
\midrule
\ipa{mtō} `see' (\zh{看见}) & \ipa{mətō}   \\
\ipa{mdə̄} `arrive' (\zh{到达}) & \ipa{məndə̂}    \\
\ipa{mniɛ̄} `few' (\zh{少}) & \ipa{mənē}    \\
\midrule
\ipa{mtsɑ̂k} `jump' (\zh{跳}) & \ipa{mtsāk}    \\
\ipa{tɑ-mtsū} `button' (\zh{纽扣}) & \ipa{ta-mtsû}    \\
\ipa{tə-mɲɑ̄k} `eye' (\zh{眼睛}) & \ipa{tə-mɲāk}    \\
\bottomrule
\end{tabular}
\end{table}

In order to account for the pair \ipa{pjōt} `fill up' $\rightarrow$ \ipa{məjōt} `full', I propose that Situ \ipa{pj-} in fact originates from \ipa{*pəj-} with a presyllable in proto-Gyalrong. This idea follows \citet[263;275;331]{jacques04these}, where the correspondence of Situ \ipa{pr-} to Japhug \ipa{pr-} (Situ \ipa{prə̂} `tear', Japhug \ipa{pri}) is reconstructed as \ipa{*pər-} with a presyllable while that of Situ \ipa{pr-} to Japhug \ipa{βr-} (Situ \ipa{prāk} `attach', Japhug \ipa{βraʁ}) is reconstructed with a real cluster \ipa{*pr-} in the common ancestor of Japhug and Situ. Since the Japhug cognate \ipa{pjɤt} `fill (a sausage)' also has \ipa{pj-} here, the reconstruction \ipa{pəj-} is possible for this word. The form \ipa{*pəjōt} would then regularly yield Situ  \ipa{pjōt} and Japhug  \ipa{pjɤt}.

The expected anticausative from a proto-form \ipa{*pəjōt} would be \ipa{*mbəjōt}. I posit this intermediate stage in proto-Situ. However, given the strong constraints on presyllables in Gyalrong languages (see \citealt[1220]{jacques12incorp},  \citealt[92]{jacques12agreement}), a presyllable such as \ipa{*mbə-} was not viable in the synchronic system of Situ, and was converted to the closest available presyllable type, \ipa{*mə-}, yielding the form \ipa{məjōt} which is attested in Brag-dbar and Kyomkyo.

In Cogtse, the vowel of the presyllable was regularly lost (as in the examples in Table \ref{tab:mEC}), yielding a cluster \ipa{*mj-} which was automatically converted to \ipa{mɲ-} (the cluster $\dagger$\ipa{mj-} is not allowed by the phonotactic of Cogtse as an onset).

The analysis of the verb pairs in Table \ref{tab:nonvol} as anticausative prenasalization rather than non-volitional \ipa{m-} prefixation is not simply a matter of morphological segmentation. While it is true that anticausative verbs lack an external agent, it is not correct to claim that the actions described by anticausative verbs is necessarily non-volitional. In Japhug for instance, the anticausative \ipa{mbɣaʁ} `turn over' (vi) from \ipa{pɣaʁ} `turn over' (vt) can be used in volitional and controllable actions, like tossing over in one's bed (see \ref{ex:kombGaR}) or animals rolling on the ground (see \ref{ex:mbGaRlaR} with  the derived verb \ipa{nɤmbɣalaʁ} `turn over in all directionals').\footnote{In the absence of a freely available corpus of Situ texts, it is not possible to look for similar examples in Situ dialects.}

 \begin{exe}
\ex \label{ex:kombGaR}
\gll \ipa{khri} 	\ipa{ɯ-taʁ} 	\ipa{nɯtɕu} 	\ipa{ko-mbɣaʁ} 	\ipa{nɤ} 	\ipa{ɲɤ-mbɣaʁ} 	\ipa{tɕe}  \\
bed \textsc{3sg}-on \textsc{dem:loc} \textsc{ifr:east-anticaus}:turn.over \textsc{lnk} \textsc{ifr:west-anticaus}:turn.over \textsc{lnk} \\
\glt `He turned and tossed in his bed.' (hist140430_yufu_he_tade_qizi, 237)
\end{exe}

 \begin{exe}
\ex \label{ex:mbGaRlaR}
\gll \ipa{rtamtɕhoʁ} 	\ipa{rɯmbuɕhi} 	\ipa{kɯ-nɯci} 	\ipa{pɯ-ɣe} 	\ipa{ɲɯ-ŋu.} \ipa{tɕe} 	\ipa{nɯ} 	\ipa{χsɯ-ɣjɤn} 	\ipa{nɯ-nɤmbɣaʁlaʁ} 	\ipa{ɲɯ-ŋu.}  \\
 Rta.mchog Rin.po.che \textsc{nmlz}:S/A-drink.water \textsc{pfv:down}-come[II]  \textsc{sens}-be \textsc{lnk} \textsc{dem} three-times \textsc{pfv}-turn.over.in.all.directions  \textsc{sens}-be \\
\glt  (The horse) Rtamchog Rinpoche came down (from heaven) to drink water, and rolled three times (on the beach).' (slobdpon2, 105)
 \end{exe}

It is true that most anticausative verbs (such as `burn', `split', `shatter' in tables \ref{tab:nonvol2} and \ref{tab:anticaus}) are only used to describe non-volitional events, this is a matter of pragmatics rather than an intrinsic morphosyntactic constraint on anticausative verbs. The label `non-volitional' is thus misleading for this category of verbs.

\section{Denominal prefixes}
Another series of examples of non-volitional / spontaneous \ipa{mə-} proposed by \citet[506]{prins16kyomkyo}  are the following:

\begin{itemize}
\item \ipa{pʰət} `throw' (vt); \ipa{təmpʰat} `vomit' (n); \ipa{məmpʰət} `vomit' (vi, non-volitional)
\item \ipa{təskruʔ} `body' $\rightarrow$ \ipa{məskruʔ} `(be) pregnant'
\item \ipa{zdək} `sad' $\rightarrow$ \ipa{məzdəkpe} `pitiful'
\end{itemize}

In the first example, it is necessary to remove \ipa{pʰət} `throw' (vt), which is unrelated to the other forms, as shown by the fact that its main vowel differs from that of the verb `vomit' in all other known varieties in Gyalrong languages (Cogtse \ipa{pʰôt} vs \ipa{məmpʰāt}, Japhug \ipa{pʰɯt} vs \ipa{mɯjpʰɤt}).

In the last example,  what we have is a denominal prefix \ipa{mə-} deriving a verb from a \ipa{məzdəkpe} derives from a noun corresponding to Tibetan \ipa{sdug.pa} `suffering', not directly from the base verb  \ipa{zdək} `sad' (also borrowed from Tibetan \ipa{sdug}, one of whose meanings is `sad').

All three examples are thus denominal derivations, not non-volitional action marker. That the \ipa{mə-/ma-} denominal prefix in Situ\footnote{On the vocalism of the prefix, see the discussion on denominal in \citealt{jackson98morphology} and \citealt{linxr93jiarong} for instance.} is unrelated to non-volitionality can be illustrated by examples such as \ipa{tɑ-ʂpɑ̄k}  shoulder'  (\zh{肩膀}) $\rightarrow$ \ipa{mɑʂpɑ̄k} `carry on the shoulder' (\zh{扛}).

%\citet[371, 511]{huangsun02} \ipa{məmphāt} 
%\citet[371]{huangsun02} \ipa{təmphāt}

\section{Autobenefactive-spontaneous}
While the case for a non-volitional \ipa{m(ə)-} prefix in Situ is rather weak, as we have seen, the spontaneous-autobenefactive prefix (Situ \ipa{-nə-}, Japhug \ipa{-nɯ-}, Khroskyabs \ipa{-n-}, see  \citet{jacques16japhug} and  \citet[158-160]{lai13affixale}) does have overtones of involuntary action in some specific contexts, as in \ref{ex:pjAnWClWG}. 

\begin{exe}
\ex \label{ex:pjAnWClWG}
\glt \ipa{popo} 	\ipa{pjɤ-nɯ-ɕlɯɣ} 	\ipa{tɕe,} 	\ipa{pjɤ-ɴɢrɯ.}  \\
earthenware \textsc{ifr:down-auto}-drop \textsc{lnk} \textsc{ifr-anticaus}:break \\
\glt `She dropped the earthenware by mistake, and it broke.' (Gesar, 327)
\end{exe}

 However, given the many distinct meanings of this prefix (autobenefactive,  permansive, spontaneous see \citet{jacques16japhug} for a detailed overview), it would not be appropriate to label this prefix as `non-volitional', since this is a special usage deriving from the `spontaneous' meaning.

\section{A volitional \ipa{mɯ-}prefix in Japhug}
We find however in Japhug an isolated example of a prefix \ipa{mɯ-} that might be related to volitionality, though not in the way proposed by Prins. Alongside the intransitive verb \ipa{mɯnmu} `move' which is attested in all Gyalrong languages (see Costse Situ \ipa{mənmō}, Kyomkyo \ipa{mənmu}), we find another verb \ipa{nmu} `move' which can only be used with earthquakes (see example \ref{ex:YAnmu} from \citealt{jacques16japhug}).

\begin{exe}
\ex \label{ex:YAnmu}
\gll
\ipa{waɟɯ} \ipa{ɲɤ-nmu} \\
earthquake \textsc{ifr}-move \\
\glt `There was an earthquake.'
\end{exe}

Since (contra \citealt[506]{prins16kyomkyo}), Japhug \ipa{mɯnmu}  move' and its Situ cognates are  volitional verbs (as shown by example  \ref{ex:YWmWnmu}, which describes the motion of an insect trapped in a spider's web), while \ipa{nmu} `move' is not, it seems possible to describe this prefix as deriving a volitional verb from a non-volitional one.

\begin{exe}
\ex \label{ex:YWmWnmu}
\gll
\ipa{tɕe}	\ipa{ɲɯ-mɯnmu}	\ipa{nɤ}	\ipa{ɲɯ-mɯnmu}	\ipa{tɕe,}	\ipa{nɯ}	\ipa{ɯ-ŋgɯ}	\ipa{ɴɢoɕna}	\ipa{kɯ-rɤʑi}	\ipa{nɯ}	\ipa{kɯ}	\ipa{pjɯ-mtsʰɤm}	\ipa{tɕe,} \\
\textsc{lnk} \textsc{ipfv}-move \textsc{lnk}  \textsc{ipfv}-move \textsc{lnk} \textsc{dem} \textsc{3sg.poss}-inside spider \textsc{nmlz}:S/A-remain \textsc{dem} \textsc{erg} \textsc{ipfv}-feel \textsc{lnk} \\
\glt `It moves and moves, and the spider that sits in the (web) feels it.' (hist-26-mYaRmtsaR, 63-4)
\end{exe}

In the absence of any more example in Japhug and other Gyalrongic languages, it is difficult to interpret this pair of verbs. In particular, whether this \ipa{mɯ-} prefix is relatable to the volitional *\ipa{m-} prefix reconstructed by \citet[55]{bs14oc} must be deferred to future research.

\bibliographystyle{unified}
\bibliography{bibliogj}
\end{document}
