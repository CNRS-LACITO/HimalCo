\documentclass[oneside,a4paper,11pt]{article} 
\usepackage{fontspec}
\usepackage{natbib}
\usepackage{booktabs}
\usepackage{xltxtra} 
\usepackage{polyglossia} 
\usepackage[table]{xcolor}
\usepackage{gb4e} 
\usepackage{multicol}
\usepackage{graphicx}
\usepackage{float}
\usepackage{hyperref} 
\hypersetup{bookmarksnumbered,bookmarksopenlevel=5,bookmarksdepth=5,colorlinks=true,linkcolor=blue,citecolor=blue}
\usepackage[all]{hypcap}
\usepackage{memhfixc}
\usepackage{lscape}
\usepackage{amssymb}
 
\bibpunct[: ]{(}{)}{,}{a}{}{,}

%\setmainfont[Mapping=tex-text,Numbers=OldStyle,Ligatures=Common]{Charis SIL} 
\newfontfamily\phon[Mapping=tex-text,Scale=MatchLowercase]{Charis SIL} 
\newcommand{\ipa}[1]{\textbf{{\phon\mbox{#1}}}} %API tjs en italique
%\newcommand{\ipab}[1]{{\scriptsize \phon#1}} 

\newcommand{\grise}[1]{\cellcolor{lightgray}\textbf{#1}}
\newfontfamily\cn[Mapping=tex-text,Ligatures=Common,Scale=MatchUppercase]{SimSun}%pour le chinois
\newcommand{\zh}[1]{{\cn #1}}

 

\XeTeXlinebreakskip = 0pt plus 1pt %
 %CIRCG
 


\begin{document}

\title{Is there a non-volitional \ipa{mə-} prefix in Gyalrong languages}
\author{Guillaume Jacques}
\maketitle

\section*{Introduction}

Given the crucial importance of 
\citet{gong17xingtaixue}


\subsection{Anticausative prenazalisation}
\citet{sagart03prenasalized}
\citet{jacques15spontaneous, jacques15causative}


\citet[330]{huangsun02} \ipa{mɲōt} `full'

\citet[670]{huangsun02} \ipa{pjōt} `fill'

\subsection{Denominal prefixes}

\subsection{A volitional \ipa{mɯ-}prefix in Japhug}
 \citet{jacques16japhug}

\begin{exe}
\ex 
\gll
\ipa{waɟɯ} \ipa{ɲɤ-nmu} \\
earthquake \textsc{ifr}-move \\
\glt `There was an earthquake.'
\end{exe}

\begin{exe}
\ex 
\gll
\ipa{tɕe}	\ipa{ɲɯ-mɯnmu}	\ipa{nɤ}	\ipa{ɲɯ-mɯnmu}	\ipa{tɕe,}	\ipa{nɯ}	\ipa{ɯ-ŋgɯ}	\ipa{ɴɢoɕna}	\ipa{kɯ-rɤʑi}	\ipa{nɯ}	\ipa{kɯ}	\ipa{pjɯ-mtsʰɤm}	\ipa{tɕe,} \\
\textsc{lnk} \textsc{ipfv}-move \textsc{lnk}  \textsc{ipfv}-move \textsc{lnk} \textsc{dem} \textsc{3sg.poss}-inside spider \textsc{nmlz}:S/A-remain \textsc{dem} \textsc{erg} \textsc{ipfv}-feel \textsc{lnk} \\
\glt `It moves and moves, and the spider that sits in the (web) feels it.' (hist-26-mYaRmtsaR, 63-4)
\end{exe}

Whether this \ipa{mɯ-} prefix is relatable to the volitional *\ipa{m-} prefix reconstructed by \citet[55]{bs14oc}...

\section*{Conclusion}

\bibliographystyle{unified}
\bibliography{bibliogj}
\end{document}
