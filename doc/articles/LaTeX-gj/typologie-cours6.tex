\documentclass[oldfontcommands,twoside,a4paper,12pt]{article} 
\usepackage{fontspec}
\usepackage{natbib}
\usepackage{booktabs}
\usepackage{xltxtra} 
\usepackage{polyglossia} 
 \usepackage{geometry}
 \geometry{
 a4paper,
 total={210mm,297mm},
 left=10mm,
 right=10mm,
 top=15mm,
 bottom=15mm,
 }
\usepackage[table]{xcolor}
\usepackage{color}
\usepackage{multirow}
\usepackage{gb4e} 
\usepackage{multicol}
\usepackage{graphicx}
\usepackage{float}
\usepackage{hyperref} 
\hypersetup{bookmarks=false,bookmarksnumbered,bookmarksopenlevel=5,bookmarksdepth=5,xetex,colorlinks=true,linkcolor=blue,citecolor=blue}
\usepackage{memhfixc}
\usepackage{lscape}
\usepackage[footnotesize,bf]{caption}
 

%%%%%%%%%%%%%%%%%%%%%%%%%%%%%%%
\setmainfont[Mapping=tex-text,Numbers=OldStyle,Ligatures=Common]{Charis SIL} 
\setsansfont[Mapping=tex-text,Ligatures=Common,Mapping=tex-text,Ligatures=Common,Scale=MatchLowercase]{Lucida Sans Unicode} 
 


\newfontfamily\phon[Mapping=tex-text,Ligatures=Common,Scale=MatchLowercase,FakeSlant=0.3]{Charis SIL} 
\newfontfamily\phondroit[Mapping=tex-text,Ligatures=Common,Scale=MatchLowercase]{Doulos SIL} 
\newcommand{\ipa}[1]{{\phon\textbf{#1}}} 
\newcommand{\ipab}[1]{{\phon #1}}
\newcommand{\ipapl}[1]{{\phondroit #1}} 
\newcommand{\captionft}[1]{{\captionfont #1}} 
\newfontfamily\cn[Mapping=tex-text,Ligatures=Common,Scale=MatchUppercase]{MingLiU}%pour le chinois
\newcommand{\zh}[1]{{\cn #1}}
\newcommand{\tgf}[1]{{\large\mo{#1}}}



\newcommand{\racine}[1]{\begin{math}\sqrt{#1}\end{math}} 
\newcommand{\grise}[1]{\cellcolor{lightgray}\textbf{#1}} 
\newcommand{\tinynb}[1]{\tiny#1}

\newcommand{\ro}{$\Sigma$}
\newcommand{\siga}{$\Sigma_1$} 
\newcommand{\sigc}{$\Sigma_3$}   
\begin{document}

\begin{exe}
\ex 
\gll tɤ-rʑaβ nɯ kɯ  tɯ-ŋga pjɤ-ɕphɤt \\ 
\textsc{indef.poss}-épouse \textsc{dem} \textsc{erg} \textsc{indef.poss}-habit \textsc{ifr}-rapiécer \\
\glt La femme a rapiécé l'habit.
\end{exe} 

\begin{exe}
\ex 
\gll tɤ-pɤtso nɯ kɯ tɤscoz pjɤ-rɤt \\ 
\textsc{indef.poss}-enfant \textsc{dem} \textsc{erg} lettre \textsc{ifr}-écrit \\
\glt L'enfant a écrit une lettre.
\end{exe} 

\begin{exe}
\ex 
\gll tɤscoz a-rɤt \\ 
lettre ?-écrire:\textsc{fact} \\
\glt La lettre est écrite.
\end{exe}

 \begin{exe}
\ex 
\gll tɤ-rʑaβ  nɯ pjɤ-rɤ-ɕphɤt \\
\textsc{indef.poss}-épouse \textsc{dem} \textsc{ifr}-?-rapiécer \\
\glt La femme a rapiécé/rapiécait (des vêtements).
\end{exe} 

\begin{exe}
\ex 
\gll tɤ-pɤtso nɯ  pjɤ-rɤ-rɤt \\ 
\textsc{indef.poss}-enfant \textsc{dem}  \textsc{ifr}-?-écrit \\
\glt L'enfant écrivait.
\end{exe} 

% \begin{exe}
%\ex 
%\gll tɯ-ŋga  a-ɕphɤt \\
%\textsc{indef.poss}-habit ?-rapiécer:\textsc{fact} \\
%\glt Le vêtement est rapiécé.
%\end{exe} 

\begin{exe}
\ex 
\gll tɤ-rʑaβ nɯ kɯ taqaβ to-ndo tɕe tɯ-ŋga ko-(sɯ)-ɕphɤt \\
\textsc{indef.poss}-épouse \textsc{dem} \textsc{erg} aiguille \textsc{ifr}-prendre \textsc{conj} \textsc{indef.poss}-habit \textsc{ifr}-?-rapiécer \\
\glt La femme a rapiécé l'habit avec une aiguille.
\end{exe} 

\begin{exe}
\ex 
\gll ɯʑo kɯ ɯ-rʑaβ (kɯ) tɯ-ŋga pjɤ-sɯ-ɕphɤt \\
\textsc{3sg} \textsc{erg} \textsc{3sg.poss}-épouse  \textsc{erg} \textsc{indef.poss}-habit \textsc{ifr}-?-rapiécer \\
\glt Il a fait rapiécer l'habit par sa femme.
\end{exe} 

\begin{exe}
\ex sloχpɯn kɯ slama (kɯ) tɤscoz pjɤ-sɯ-rɤt \\
enseignant \textsc{erg} étudiant \textsc{erg} lettre \textsc{ifr}-?-écrire \\
\glt L'enseignant a demandé à l'élève décrire une lettre.
\end{exe}

\begin{exe}
\ex
\gll 
tɕaχpa nɯni kɯ alibaba ɣɯ ɯ-kɯm nɯ tɕu fenbi kɯ kɯ-ɤrtɯm ci to-sɯ-rɤt-ndʑi \\
brigand \textsc{dem:duel} \textsc{erg} (nom.propre) \textsc{gen} \textsc{3sg.poss}-porte \textsc{dem} \textsc{loc} craie \textsc{erg} \textsc{nmls}:S/A-être;rond un \textsc{ifr}-?-écrire-\textsc{du} \\
\glt Les deux brigands dessinèrent un rond sur la porte de (la maison) d'Ali Baba avec une craie.
\end{exe}

\begin{exe}
\ex 
\gll Dai.Song ɣɯ ɯ-ɕaχpu nɯ kɯ pjɤ́-wɣ-z-rɤ-rɤt \\
(nom.propre) \textsc{gen} \textsc{3sg.poss}-ami \textsc{dem} \textsc{erg}  \textsc{ifr}-\textsc{inv}-?-?-écrire \\
\glt Un ami de Dai Song lui demanda de dessiner quelquechose.
\end{exe} 

\begin{exe}
\ex 
\gll ɯʑo ɲɯ-mu \\
\textsc{3sg} \textsc{sens}-avoir.peur \\
\glt Il a peur.
\end{exe}

\begin{exe}
\ex 
\gll ɯʑo kɯ sɯŋgi ɲɯ-nɯɣ-me \\
\textsc{3sg} \textsc{erg} lion  \textsc{sens}-?-avoir.peur(thème.III) \\
\glt Il a peur du lion.
\end{exe}

\begin{exe}
\ex 
\gll sɯŋgi nɯ ɲɯ-sɤɣ-mu. \\
lion \textsc{dem} sens-?-avoir.peur \\
\glt Le lion est terrifiant.
\end{exe}

\begin{exe}
\ex 
\gll  sɯŋgi kɯ tɤ-pɤtso ra ɲɯ-ɕɯɣ-me ŋgrɤl \\
lion \textsc{erg} \textsc{indef.poss}-enfant \textsc{pl} \textsc{sens}-?-avoir.peur(thème.III) être.le.cas.habituellement \\
\glt Le lion fait peur aux enfants.
\end{exe}


\end{document}

