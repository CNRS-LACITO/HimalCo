% sɯŋgɯ pɤjka nɯ kɯ ɯ-jwaʁ chondɤre, nɤkinɯ, ɯ-jwaʁ nɯra xtɕi.
%locus du comparatif

\documentclass[oldfontcommands,oneside,a4paper,11pt]{article} 
\usepackage{fontspec}
\usepackage{natbib}
\usepackage{booktabs}
\usepackage{xltxtra} 
\usepackage{longtable}
\usepackage{polyglossia} 
\usepackage[table]{xcolor}

\usepackage{multicol}
\usepackage{graphicx}
\usepackage{lineno}
\usepackage{float}
\usepackage{hyperref} 
\hypersetup{bookmarks=false,bookmarksnumbered,bookmarksopenlevel=5,bookmarksdepth=5,xetex,colorlinks=true,linkcolor=blue,citecolor=blue}
\usepackage[all]{hypcap}
\usepackage{memhfixc}
\usepackage{lscape}
\usepackage{tikz}
%
\usetikzlibrary{trees}
\usepackage{gb4e} 
\bibpunct[: ]{(}{)}{,}{a}{}{,}
%%%%%%%%%quelques options de style%%%%%%%%
%\setsecheadstyle{\SingleSpacing\LARGE\scshape\raggedright\MakeLowercase}
%\setsubsecheadstyle{\SingleSpacing\Large\itshape\raggedright}
%\setsubsubsecheadstyle{\SingleSpacing\itshape\raggedright}
%\chapterstyle{veelo}
%\setsecnumdepth{subsubsection}
%%%%%%%%%%%%%%%%%%%%%%%%%%%%%%%
\setmainfont[Mapping=tex-text,Numbers=OldStyle,Ligatures=Common]{Charis SIL} %ici on définit la police par défaut du texte
%\renewcommand \thesection {\arabic{section}.}
%\renewcommand \thesubsection {\arabic{section}.\arabic{subsection}.}
%\renewcommand \thesubsubsection {\arabic{section}.\arabic{subsection}.\arabic{subsubsection}.}
\newfontfamily\phon[Mapping=tex-text,Ligatures=Common,Scale=MatchLowercase,FakeSlant=0.3]{Charis SIL} 
\newcommand{\ipa}[1]{{\phon #1}} %API tjs en italique
 
\newcommand{\grise}[1]{\cellcolor{lightgray}\textbf{#1}}
\newfontfamily\cn[Mapping=tex-text,Ligatures=Common,Scale=MatchUppercase]{MingLiU}%pour le chinois
\newcommand{\zh}[1]{{\cn #1}}
   

\XeTeXlinebreaklocale "zh" %使用中文换行
\XeTeXlinebreakskip = 0pt plus 1pt %


\begin{document} 
\linenumbers

\title{From ergative to index of comparison: multiple reanalyses and polyfunctionality\footnote{The glosses follow the Leipzig glossing rules. Other abbreviations used here are:   \textsc{auto} autobenefactive / spontaneous, \textsc{lnk} coordinator, \textsc{dem} demonstrative, \textsc{dist} distal, \textsc{emph} emphatic, \textsc{indef} indefinite, \textsc{inv} inverse,  \textsc{pfv} perfective, \textsc{poss} possessor, \textsc{fact} factual,  \textsc{testim} testimonial.} }

\author{Guillaume JACQUES}
\maketitle
%The origins of voice markers has attracted a considerable body of literature, and much work has focused on determining the lexical items that can become voice markers and the pathways of grammaticalization from one voice to another. 
 
 \section{Introduction}
 
 
In Japhug Rgyalrong, clitic markers of the form \ipa{kɯ} appear in a wide array of different constructions. The most common one is the ergative marker, but we also find homophonous markers as a distributive, a causal linker, a complementizer,    as the index of the comparee in the comparative construction  and as a linker in the degree construction.

Syncretisms between agent or ergative markers  on the one hand and various spatial cases on the other hand are common typologically (\citealt{agent02palancar}). However, the apparent polyfunctionality of the marker \ipa{kɯ} in Japhug, if indeed one marker has to be posited, is without equivalents in is unclear to what extent these various uses are related synchronically and diachronically, and whether one or several distinct markers have to be described.
 
 In this paper, we provide a description of all the constructions where a clitic \ipa{kɯ} is attested, from the ergative to the degree construction. Then, we propose a series of historical scenarios showing that most of  attested uses of this marker are diachronically related, and explaining how they  came into being.

 A series of previously unknown pathways of grammaticalization are proposed here for the first time on the basis of Japhug Rgyalrong data.

\section{Ergative} \label{sec:erg}
Japhug has a very clear   distinction between transitive and intransitive verbs, which is reflected in both case marking and verbal morphology.

Japhug has ergative alignment on nouns; absolutive is unmarked (examples \ref{ex:abs} and \ref{ex:erg}) , while the A of transitive verb receives the clitic \ipa{kɯ} (example \ref{ex:erg}). This clitic is obligatory with nouns and third person pronouns, but optional for first and second person pronouns, since anyway the verb agreement morphology distinguishes between agent and patient in a non-ambiguous way.

\begin{exe}
\ex \label{ex:abs}
\gll \ipa{tɤ-tɕɯ}  	\ipa{nɯ}  	 	\ipa{jo-ɕe}   \\
\textsc{indef.poss}-boy \textsc{dem}   \textsc{evd}-go \\
\glt `The boy went (there).'
\end{exe}

\begin{exe}
\ex \label{ex:erg}
\gll \ipa{tɤ-tɕɯ}  	\ipa{nɯ}  	\ipa{\textbf{kɯ}} 	\ipa{χsɤr}  	\ipa{qaɕpa}  	\ipa{nɯ}  	\ipa{chɤ-mqlaʁ}   \\
\textsc{indef.poss}-boy \textsc{dem} \textsc{erg} gold frog \textsc{dem} \textsc{evd}-swallow \\
\glt `The boy swallowed the golden frog.' (Nyima Wodzer.1, 131)
\end{exe}


Additionally, as in many languages, the ergative is also used as an instrumental marker (\citealt[32]{agent02palancar}). When the instrumental in the ergative appears in the sentence, the verb is generally marked with the causative prefix \ipa{sɯ}-- / \ipa{z}--. 

\begin{exe}
\ex \label{ex:instr}
\gll \ipa{kɯɕɯŋgɯ}   	\ipa{tɕe,}   	\ipa{rɤɣo}   	\ipa{\textbf{kɯ}}   	\ipa{tɯrju}   	\ipa{tu-sɯ-βzu-nɯ}   	\ipa{pjɤ-ŋu}   	\ipa{tɕe,}     \\
before \textsc{coord} song \textsc{erg} word \textsc{ipfv}-\textsc{caus}-do-\textsc{pl} \textsc{ipfv:evd}-be \textsc{lnk}\\
\glt  `In old times, people used to speak by song.' (Gesar, 37)
\end{exe}
 \begin{exe}
\ex \label{ex:instr2}
\gll \ipa{nɯ-mtsioʁ}   	\ipa{nɯ}   	\ipa{\textbf{kɯ}}   	\ipa{lu-sɯ-lɣa-nɯ}   	\ipa{qhe,}   	\ipa{tɤtsoʁ}   	\ipa{lu-nɯ-tɕɤt-nɯ}   	\ipa{ɲɯ-ŋgrɤl}   	\ipa{khi}        \\
\textsc{3pl.poss}-beak \textsc{dem} \textsc{erg} \textsc{ipfv-caus}-dig-\textsc{pl} \textsc{coord} potentilla.anserina \textsc{ipfv-auto}-take.out-\textsc{pl} \textsc{ipfv}-be.usually.the.case \textsc{hearsay} \\
\glt  `(The wild geese) dig (the ground) with their beaks, and they take out Potentilla anserina roots.' (Wild Geese 28)
\end{exe}



Causative marking on the verb with an instrument is not compulsory, and one can find the two constructions with or without the causative marker side by side in the same narrative:

\begin{exe} 
\ex \label{ex:instr3}
\gll   \ipa{qartsʰaz}   	\ipa{ɯ-ndʐi}   	\ipa{\textbf{kɯ}}   	\ipa{cʰɯ-βzu-nɯ}   	\ipa{tɕe,}   	\ipa{nɯ}   	\ipa{stu}   	\ipa{kɯ-ʑru.}     \\
 deer \textsc{3.sg.poss}-hide \textsc{erg} \textsc{ipfv}-do-\textsc{pl} \textsc{coord} \textsc{dem} most \textsc{nmlz:S/A}-strong \\
 \glt   `They make (shoes) with deer hide, it is the most resistant (type of skin).' (Shoes, 48)
\end{exe} 

 \begin{exe} 
\ex \label{ex:instr4}
\gll   \ipa{qartshaz}   	\ipa{ɯ-ndʐi}   	\ipa{ʁɟa}   	\ipa{\textbf{kɯ}}   	\ipa{ʑo}   	\ipa{thɯ-kɤ-sɯ-βzu}       \\
 deer \textsc{3.sg.poss}-hide entirely \textsc{erg} \textsc{emph} \textsc{aor-nmlz:O-caus}-do \\
 \glt   `(It is) entirely made of deer hide.' (Shoes, 53)
\end{exe} 


While the instrument and the agent are both marked by \ipa{kɯ}, their syntactic status is different, as show by their relativized forms. 

The agent is relativized by using the A participle, formed by prefixing the \ipa{kɯ--} nominalizing prefix preceded by a possessive prefix coreferent with the P to the verb stem, as in \ref{ex:WkWnWmbrApW}.
\begin{exe}
   \ex  \label{ex:WkWnWmbrApW}
\gll [\ipa{tɤpɤtso}  	\ipa{ci}  	\ipa{kɯ}  	<yangma> 	\ipa{ɯ-kɯ-nɯmbrɤpɯ}]  	\ipa{ci}  	\ipa{jɤ-ɣe}  \\
boy \textsc{indef} \textsc{erg} bicycle \textsc{3sg-nmlz:A}-ride \textsc{indef} \textsc{pfv}-come[II] \\
\glt `A boy who was riding a bicycle arrived.' (Pear story, Chenzhen, 5)
\end{exe}

On the other hand, the instrument is relativized by the oblique participle in \ipa{sɤ}-- and removing the   causative prefix \ipa{sɯ}-- as in \ref{ex:sAxtCAr}.  

 \begin{exe}
  \ex  \label{ex:sAxtCAr}  
  \gll [\ipa{nɯ-mthɤɣ}  	\ipa{sɤ-xtɕɤr}]  	\ipa{xɕɤfsa}  	\ipa{ma}  	\ipa{pjɤ-me}  \\
\textsc{3pl.poss}-waist \textsc{nmlz:oblique}-tie thread apart.from \textsc{evd.ipfv}-not.exist \\
\glt `They only had threads to tie their waists (the only things that they could use to tie their waist were threads).' (Milaraspa translation)
   \end{exe} 
   
 
 

 Outside of its used as agent or instrumental marker, \ipa{kɯ} is also selected by the transitive verb \ipa{khɤt} ``to do several times, a long time", where an abstract noun or an action nominalization with the ergative indicates the action realized.
 \begin{exe} 
\ex \label{ex:instr5}
\gll   \ipa{tɯ-qioʁ}   	\ipa{\textbf{kɯ}}   	\ipa{tó-wɣ-sɯ-khɤt}   	\ipa{ʑo}   	\ipa{tɕe,}       \\
  \textsc{nmlz:action}-vomit \textsc{erg} \textsc{evd-inv-caus}-do.a.long.time \textsc{emph} \textsc{coord}\\
 \glt  `(The medicine) made him vomit a long time.' (Gesar, 266)
\end{exe}  
 
 
 % tɕe  tɯʑo kɯ tɯʑo  tukɯnɯʑɣɤβri ra kɤti ɲɯŋu
\section{Distributive} \label{sec:distributive}
The marker \ipa{kɯ} can have   a distributive meaning (`for one X', `per') when used with a with a classifier designating a quantity. It occurs  in constructions with intransitive verbs where no agent or instrument is present, but exclusively to express the price of the quantity designated, as in  \ref{ex:tWtWrpa.kW1} and \ref{ex:tWtWrpa.kW2}.

 \begin{exe} 
\ex \label{ex:tWtWrpa.kW1}
\gll  
\ipa{tɯ-tɯrpa} 	\ipa{\textbf{kɯ}} 	\ipa{sqi} 	\ipa{jamar} 	\ipa{ɲɯ-ra.} 	\\
one-pound \textsc{erg} ten about \textsc{testim}-have.to \\
\glt `You need ten (yuans) per pound (of Angelica).' (Angelica, 22)
\end{exe}  

 \begin{exe} 
\ex \label{ex:tWtWrpa.kW2}
\gll  
\ipa{tɯ-tɯrpa}  	\ipa{\textbf{kɯ}}  	\ipa{ɣurʑa}  	\ipa{ɯ-ro,}  	\ipa{ɯ-phɯ}  	\ipa{ɲɯ-ɣi.}  \\
one-pound \textsc{erg} hundred \textsc{3sg.poss}-more \textsc{testim}-come \\
\glt `It costs more than one hundred (yuans) per pound.' (Matsutake, 5)
\end{exe}  
  \begin{exe} 
\ex \label{ex:tWtCha.kW}
\gll  
\ipa{tɯ-xtsa} 	\ipa{\textbf{tɯ-tɕʰa}} 	\ipa{kɯ} 	\ipa{ɣurʑa} 	\ipa{ɯ-ro} 	\ipa{ɲɯ-ra} \\
\textsc{indef.poss}-shoe one-pair \textsc{erg} hundred \textsc{3sg.poss}-more \textsc{testim}-have.to \\
 \glt  `It costs more than one hundred for a pair of shoes.' (elicited)
\end{exe}  
The construction with \ipa{kɯ} cannot be used with classifiers expressing duration, as in \ref{ex:tWsla.tCe} (replacing the linker \ipa{tɕe} by \ipa{kɯ} here would be agrammatical).
 
 \begin{exe} 
\ex \label{ex:tWsla.tCe}
\gll  
  \ipa{tɯ-sla} \ipa{tɕe} \ipa{ɯ-ŋgra} \ipa{ɣurʑa} \ipa{ɯ-ro} \ipa{ɣɤʑu} \\
  one-month \textsc{lnk} 3sg.poss-salary hundred \textsc{3sg.poss}-more exist:\textsc{sens} \\
  \glt `His salary is more than one hundred a month.'
  \end{exe}  
 \section{Clause linking} \label{sec:linking}
 The marker \ipa{kɯ} appears in two unrelated types of clause linking, it can either be combined with an infinitival clause to express a circumstance, or with the negative copula to convey an adversative meaning.
 
 \subsection{Infinitival manner linking} \label{sec:manner}
The marker \ipa{kɯ}  appears on an infinitival subordinate clause  to indicate the circumstance or manner in which the action described by the verb of the main clause took place, as in \ref{ex:mAkApa.kW}. 

\begin{exe}
\ex \label{ex:mAkApa.kW}
\gll
\ipa{tɕe}   	\ipa{ɯ-ŋgɯ}   	\ipa{nɯ} \ipa{tɕu}   	\ipa{paʁndza}   	\ipa{ɲɤ-raʁ}   	\ipa{tɕe,}   	\ipa{tɕendɤre}   	<dian>   	<guan>   	\ipa{mɤ-kɤ-βzu} 	\ipa{\textbf{kɯ}}   	\ipa{mɤ-kɤ-pa}   	\ipa{\textbf{kɯ}}   	\ipa{ɯ-jaʁ}   	\ipa{lo-tsɯm}   \\
\textsc{lnk} \textsc{3sg}-inside \textsc{top} \textsc{loc} pig.fodder \textsc{evd}-be.stuck \textsc{lnk}
\textsc{lnk} electricity turn.off \textsc{neg-inf}-make \textsc{erg}  \textsc{neg-inf}-close \textsc{erg}  \textsc{3sg.poss}-hand \textsc{evd:upstream}-take.away \\
\glt `Some pig fodder got stuck inside (the machine) he put his hand into it without turning it off,' (Relatives, 372-3)
\end{exe} 

 While  examples such as \ref{ex:nWCe.kAsWso} could lead to believe that the infinitival clause marked with \ipa{kɯ}  expresses  cause and the main clause the result, it is clear that this construction does entail a unidirectional causal relationship between the main and the subordinate clause. In \ref{ex:mAkApa.kW} the events expressed by the main and the subordinate clauses are not causally linked to one another, and in  \ref{ex:mWYWkWYJWr} it is the subordinate clause expresses the purpose of the action of the main clause (the causality is reversed in comparison with \ref{ex:nWCe.kAsWso}).

  \begin{exe}
\ex \label{ex:nWCe.kAsWso}
\gll 
\ipa{nɯɕe}  	\ipa{kɤ-sɯso}  	\ipa{kɯ,}  	\ipa{ɯ-mbro}  	\ipa{nɯnɯ}  	\ipa{taqaβ}  	\ipa{cʰɤ-z-nɯtɕʰaʁ-nɯ,}  	\ipa{ɯ-kʰɯna}  	\ipa{nɯ}  	\ipa{rkorsa}  	\ipa{ɯ-pa}  	\ipa{lo-ja-nɯ}  \\
\textsc{fact}:go.back \textsc{inf}-think \textsc{erg} \textsc{3sg.poss}-horse \textsc{top} needle \textsc{evd-caus}-eat-\textsc{pl}   \textsc{3sg.poss}-dog \textsc{top} toilet \textsc{3sg.poss}-down \textsc{evd}-pen-\textsc{pl} \\
\glt `Thinking that he (was about to) go back, they fed his horse with needles and penned his dog in the toilets.' (Gesar 250-1)
 \end{exe} 
 
 

\begin{exe}
\ex \label{ex:mWYWkWYJWr}
\gll
\ipa{tɯ-xtsa}   	\ipa{nɯnɯ}   	\ipa{ɯ-ʁzɯɣ}   	\ipa{mɯ-ɲɯ-kɯ-ɲɟɯr}   	\ipa{\textbf{kɯ}}   	\ipa{ɲɯ-z-rɤsta-nɯ}   \\
\textsc{indef.poss}-shoe \textsc{top} shape \textsc{neg-ipfv-inf:non.hum-anticaus}:change \textsc{erg} \textsc{ipfv-caus}-be.fixed \\
\glt `They wedge the shoes (with a shoe tree)  in such a way that their shape does not change.' (komar, 109)
\end{exe}

\subsection{Finitive manner linking}
In the finite manner clause linking, the subordinate clause is likewise marked by \ipa{kɯ}, but its verb is in finite form. The  finite  main clause adds further information on the state or situation described by the subordinate clause, as in \ref{ex:mWjfse.kW}. 
 

 \begin{exe}
\ex \label{ex:mWjfse.kW}
\gll
\ipa{ri} 	\ipa{ɯ-jwaʁ} 	\ipa{nɯnɯ} 	\ipa{kɯmaʁ} 	\ipa{ɕɤɣ} 	\ipa{nɯ} \ipa{ra} 	\ipa{mɯ́j-fse} 	\ipa{\textbf{kɯ}} 	\ipa{ɲɯ-ɤrʁɯrʁu} 	\ipa{ʑo} 	\ipa{qʰe} 	\ipa{ɲɯ-ɤndɯndo} 	\ipa{ʑo.} \\
\textsc{lnk} \textsc{3sg.poss}-leaf \textsc{dem} other juniper \textsc{top} \textsc{pl} \textsc{neg:const}-be.like \textsc{erg?} \textsc{testim}-be.wrinkled \textsc{emph} \textsc{lnk}  \textsc{testim}-be.clustered.together \textsc{emph} \\
\glt `Its leaves differ from other junipers in that they are wrinkled and clustered together.' (Ephedra, 71)
 \end{exe}
 
Semantically,  the infinitival and the finite manner clause linkings are quite distinct. In the former, the subordinate clause presents background informations (whether a circumstance, a cause or even a purpose) on the main clause. In the latter, on the other hand, the subordinate clause preceding \ipa{kɯ} indicates the main event / state of affair, and the main clause represents an additional characterization of this event.


\subsection{Cause} \label{sec:cause}

The marker \ref{ex:sWmWzdWG.kW} can be used with an abstract noun to indicate cause, as in \ref{ex:sWmWzdWG.kW}.

\begin{exe}
\ex \label{ex:sWmWzdWG.kW}
\gll 
\ipa{sɯmɯzdɯɣ}  	\ipa{kɯ}  	\ipa{ɕɤr}  	\ipa{ɯ-ʑɯβ}  	\ipa{mucin}  	\ipa{mɯ-pɯ-ɣe}  	\ipa{ɲɯ-ŋu.}  	 \\
worry \textsc{erg} night \textsc{3sg.poss}-sleep at.all \textsc{neg-pfv}-come[II] \textsc{testim}-be \\
\glt `As she was worried, she could not get any sleep for the whole night.' (Slobdpon, 174)
\end{exe}


  It also appear in a clausal construction with the possessed noun `its strength' \ipa{ɯ-xɕɤt} borrowed from Tibetan \ipa{ɕed} `strength'.\footnote{This borrowed is specifically from Amdo, as it reflect an intermediate stage in the sound change from [ɕet] to [xɕit] as in most Amdo varieties.}

 

\begin{exe}
\ex \label{ex:WxCAt.kW}
\gll
\ipa{tɯ-mŋɤm} 	\ipa{ɯ-xɕɤt} 	\ipa{kɯ} 	\ipa{aʑo} 	\ipa{nɯ} 	\ipa{a-ku} 	\ipa{ɕɤrɯ} 	\ipa{pjɤ-ɣɤtsɯr} 	\ipa{ɲɯ-ŋu} 	\ipa{nɯ-sɯso-t-a.} 	\\
\textsc{nmlz:action}-hurt \textsc{3sg.poss}-strength \textsc{erg} \textsc{1sg} \textsc{top} \textsc{1sg.poss}-head bone \textsc{evd}-have.a.crack \textsc{testim}-be \textsc{pfv}-think-\textsc{pst-1sg} \\
\glt `Because to the pain, I felt as though my skull had cracked.' (Headache, 77)
  \end{exe}

  \begin{exe}
\ex \label{ex:WxCAt.kW2}
\gll
  \ipa{tɤ-zdɯɣ} 	\ipa{ɯ-xɕɤt} 	\ipa{kɯ} 	\ipa{pjɯ-si} 	\ipa{ɕti,} \\
  \textsc{indef.poss}-toil  \textsc{3sg.poss}-strength \textsc{erg} \textsc{ipfv}-die \textsc{fact}:be:\textsc{affirm} \\
 \glt `(The bee) dies of exhaustion.' (Bee, 40)
  \end{exe}
  
In this construction, we either find a nominalized verb form (in \ipa{tɯ--} or with the infinitive \ipa{kɯ--} / \ipa{kɤ--}) as in \ref{ex:WxCAt.kW} or an abstract noun as in \ref{ex:WxCAt.kW2} in the causal subordinate clause.

Another causal construction in \ipa{kɯ} involves the noun \ipa{tʰɯrʑi} `mercy' (from Tibetan \ipa{tʰugs.rdʑe}), as in examples \ref{ex:WthWrZi.kW}.

  \begin{exe}
\ex \label{ex:WthWrZi.kW}
\gll
\ipa{tɕe}  	\ipa{tɯrpa}  	\ipa{ɯ-tʰɯrʑi}  	\ipa{kɯ}  	\ipa{nɯ}-<shenghuo>  	\ipa{nɯ} \ipa{ra}  	\ipa{wuma}  	\ipa{ʑo}  	\ipa{pjɤ-pe}  	\ipa{ɲɯ-ŋu.}  \\
\textsc{lnk} axe \textsc{3sg.poss}-mercy \textsc{erg} \textsc{3pl.poss}-life \textsc{top} \textsc{pl} really \textsc{emph} \textsc{evd.ipfv}-be.good \textsc{testim}-be \\
\glt `Thanks to the axe, their life was very good.' (The little village, 35)
  \end{exe}
  
  
 \subsection{Emphatic adversative} \label{sec:advers}
Combined with the copula \ipa{maʁ} `not be' in finite form, the marker \ipa{kɯ} is also used to express adversative meaning between the subordinate clause the main clause (example \ref{ex:YWmaR.kW}).
 
  \begin{exe} 
 \ex \label{ex:YWmaR.kW}
\gll \ipa{tɯrgi} 	\ipa{kɯ-fse} 	\ipa{ɯ-stu} 	\ipa{tu-ɕe} 	\ipa{ɲɯ-maʁ} 	\ipa{kɯ,} \ipa{aʁɤndɯndɤt} 	\ipa{ɯ-rtaʁ} 	\ipa{ɲɯ-ɬoʁ} 	\ipa{ɲɯ-ŋu} 	\ipa{tɕe,} 
\\
fir \textsc{inf:stat}-be.like \textsc{3sg.poss}-straight \textsc{ipfv:up}-go \textsc{testim}-not.be \textsc{erg}? everywhere \textsc{3sg.poss}-branch const-come.out \textsc{testim}-be \textsc{lnk} \\ 
\glt `It does not grow straight like the fir, on the contrary, its branches spread out in all directions.'  
(maldo, 54)
 \end{exe}  
 
 This construction is used to focus on the contrast between the (negated) event/situation described in the subordinate clause, and that of the main clause, and corresponds to the English phrase `on the contrary'.
 
 \section{Sentence final particle} \label{sec:compl}
 
 
The sentence final particle \ipa{kɯ} is used in rhetorical questions (\ref{ex:kAkWnaXtCAn}) or in introspective questions (\ref{ex:pWwGsat}).
 \begin{exe} 
 \ex \label{ex:kAkWnaXtCAn}
\gll 
`\ipa{tɯ-ci} 	\ipa{kɤ-kɯ-naχtɕɤn} 	\ipa{ɯ-ŋgɯ} 	\ipa{ɕ-pjɯ́-wɣ-βde} 	\ipa{ra}' 	\ipa{ɲɯ-ti-nɯ} 	\ipa{kɯ-fse} 	\ipa{tɕe,} 	\ipa{ɯ-ɲɯ́-ŋu} 	\ipa{\textbf{kɯ}?} \\
\textsc{indef.poss}-water \textsc{pfv-nmlz}:S/A-be.mad \textsc{3sg}-inside \textsc{transloc-ipfv-inv}-throw \textsc{fact}:have.to \textsc{testim}-say-\textsc{pl} \textsc{inf:stat}-be.like \textsc{lnk} \textsc{qu-const}-be \textsc{qu} \\
\glt `(The Râkshasas are saying)  `Let's throw him in the river of madness', aren't they?' (Smanmi Metog Koshana 2011, 101)
\end{exe} 
 
  \begin{exe} 
 \ex \label{ex:pWwGsat}
\gll 
`\ipa{pɯ́-wɣ-sat} 	\ipa{ɯ-mɤ-ɕti} 	\ipa{\textbf{kɯ}?}' 	\ipa{tu-ʁjit} 	\ipa{pjɤ-ŋu,} \\
\textsc{pfv-inv}-kill \textsc{qu-neg}-be:\textsc{affirm} \textsc{qu} \textsc{ipfv}-think \textsc{evd:ipfv}-be \\
\glt `She was thinking: `Have they killed him?'' (The frog, 91)
 \end{exe}  
 
 A related sentence final particle \ipa{kɯma} has the same functions. In  \ref{ex:kWma1}, its use implies that   the speaker is not sure whether there is any tree left to talk about.
   \begin{exe} 
 \ex \label{ex:kWma1}
\gll 
 \ipa{tɕendɤre} 	\ipa{mɤʑɯ} 	\ipa{si,} 	\ipa{mɤʑɯ} 	\ipa{tɕhi} 	\ipa{mɯ-pɯ-fɕɤt-tɕi} 	\ipa{\textbf{kɯma}?} \\
\textsc{lnk} again tree again what \textsc{neg-pfv}-tell-\textsc{1du} \textsc{qu} \\
\glt `Which trees, which ones haven't we yet talked about?'  (qandzɤjo, 62)
  \end{exe} 
  
  The particle \ipa{kɯma} can be combined with the polar question particle \ipa{kɯ} to express a hypothetical quasi-rhetorical question as in \ref{ex:kWma2} and \ref{ex:kWma3}.
     \begin{exe} 
 \ex \label{ex:kWma2}
\gll 
 <Yunnan> \ipa{nɯ} 	\ipa{tɕu} 	\ipa{tu} 	\ipa{maʁ} 	\ipa{ɕi} 	\ipa{\textbf{kɯma}} \\
 Yunnan \textsc{top} \textsc{loc} \textsc{n.pst:}exit \textsc{fact}:not.be \textsc{polar:qu} \textsc{qu} \\
\glt  `Maybe in Yunnan there are some (elephants).' (Elephant, 9)
   \end{exe}
   
        \begin{exe} 
 \ex \label{ex:kWma3}
\gll 
 \ipa{nɤʑo} 	\ipa{ɯ-tɯ-sɯz} 	\ipa{ɕi} 	\ipa{\textbf{kɯma}?}  \\
 \textsc{2sg} \textsc{qu-2-n.pst}:know \textsc{polar:qu} \textsc{qu} \\
\glt `You wouldn't know, would you?' (ʁzɤr, 34)
    \end{exe}
    
    
    The resulting interrogative sentences can appear as complements of verbs of speech and thought, as in \ref{ex:kW.nWsWsota} or \ref{ex:kW.YWsWsama}. ɪn this context, the particle \ipa{kɯ} can alternatively be analyzed as a complementizer.
    
    \begin{exe} 
 \ex \label{ex:kW.nWsWsota}
\gll 
    \ipa{jɯfɕɯr} 	\ipa{mbɤxɕɯβ} 	\ipa{nɯ} 	\ipa{pɯ-fɕat-a} 	\ipa{ɕi} 	\ipa{\textbf{kɯ}} 	\ipa{nɯ-sɯso-t-a} \\
    yesterday plant.sp. \textsc{top} \textsc{pfv}-tell-\textsc{1sg} \textsc{polar:qu} \textsc{qu} \textsc{pfv}-think-\textsc{pst-1sg} \\
    \glt `I thought I had told about the   \ipa{mbɤxɕɯβ} yesterday.' (mbɤxɕɯβ, 102)
 \end{exe}  
 

    \begin{exe} 
 \ex \label{ex:kW.YWsWsama}
\gll \ipa{a-kɤ-nɯtshɤβ-nɯ} 	\ipa{tɕe} 	\ipa{a-tɤ-tɕhɯ-nɯ} 	\ipa{tɕe,} 	\ipa{a-pɯ-sat-nɯ} 	\ipa{\textbf{kɯ}} 	\ipa{ɲɯ-sɯsam-a} 	\ipa{ri}  \ipa{nɯ} \ipa{ra} 	\ipa{mɯ́j-stu-nɯ} 	\ipa{ri} \\
\textsc{irr-pfv}-do.together-\textsc{pl} \textsc{lnk} \textsc{irr-pfv}-gore-\textsc{pl} \textsc{lnk } \textsc{irr-pfv}-kill-\textsc{pl} \textsc{qu} \textsc{ipfv}-think[III]-\textsc{1sg} but \textsc{dem} \textsc{pl} \textsc{neg:const}-do.this.way-\textsc{pl}  but \\
\glt `I think that if they would attack the leopard together and gore it, they could kill it, but they don't do that, rather,' (Wild yak, 66)
 \end{exe} 
 
 
Although all examples of this construction in the corpus occur with the verb \ipa{sɯso} `think', it is possible to elicitate it with other thought or speech verbs, as in \ref{ex:hesitate}.


 \begin{exe} 
 \ex \label{ex:hesitate}
\gll \ipa{chɯ-ɕe-a}   	\ipa{ɕi}   	\ipa{ma-thɯ-ɕe-a}   	\ipa{\textbf{kɯ}}   	\ipa{ku-nɯsɯmɯʁɲɯz-a}          \\
\textsc{ipfv:downstream}-go-\textsc{1sg} \textsc{qu} \textsc{neg}-\textsc{imp}-go-\textsc{1sg} \textsc{erg} \textsc{present}-hesitate-\textsc{1sg}   \\
 \glt  `I hesitate whether to go or not.' (elicited)
\end{exe}  

 

\section{Comparative construction} \label{sec:comparative}
%No comparative adjectives XXX

A clitic \ipa{kɯ} formally identical to the ergative also appears in the main Japhug comparative construction, which can be illustrated by   examples \ref{ex:comp1} and \ref{ex:comparative.complete}.

\begin{exe}
\ex \label{ex:comp1}
\gll  \ipa{ɯ-ʁi}   	\ipa{sɤz}   	\ipa{ɯ-pi}   	\ipa{nɯ}   	\ipa{\textbf{kɯ}}   	\ipa{mpɕɤr}     \\
\textsc{3sg.poss}-younger.sibling \textsc{comparative} \textsc{3sg.poss}-elder.sibling \textsc{dem} \textsc{erg?}  \textsc{n.pst:}be.beautiful \\
\glt `The elder one is more beautiful than the young one.' (elicited)
\end{exe}
 
 
\begin{exe}
\ex \label{ex:comparative.complete}
\gll \ipa{jɯfɕɯr}   	\ipa{sɤz }   	\ipa{jɯsŋi}   	\ipa{kɯ}   	\ipa{ɲɯ-mpja}   \\
yesterday \textsc{comparative} today \textsc{erg} \textsc{testim}-warm \\
\glt `Today is warmer than yesterday.' (elicited)
\end{exe}

The terminological framework used in this section is mainly based on \citet{dixon08comparative} and \citet{stassen11comparative}. The following English sentence will illustrate the  terminology:

\begin{exe}
\ex \label{ex:comp.eng}
\gll  John is more intelligent than Paul \\
\textsc{comparee} { } \textsc{index} \textsc{parameter} \textsc{mark} \textsc{standard}  \\
\end{exe}

Comparative constructions involve two participants which are not equal. The \textsc{comparee} is  the object of comparison, while the other one, the \textsc{standard}, is the entity against which it is compared. The \textsc{parameter} indicating the property in terms of which the comparee and the standard are compared, is generally an adjective or a stative verb (more rarely an active verb). The \textsc{index} is an adverb or a verb indicating the degree of the parameter, and the \textsc{mark} an element (case marker or otherwise) appearing on the standard to distinguish it from the comparee. All languages that have monoclausal comparative constructions have comparees, standards and parameters, but indexes and marks may or may not be present depending on the language.

 

The Japhug comparative construction comprises three syntactic constituents, corresponding to the standard, the comparee and the parameter. It is possible to have partial constructions with either only the standard (as in \ref{ex:comp2}), or only the comparee (as in \ref{ex:comp3}).

 \begin{exe}
\ex \label{ex:comp2}
\gll 
\ipa{co}  	\ipa{ɣɯ}  	\ipa{nɯnɯ}  	\ipa{\textbf{kɯ}}  	\ipa{mɤku}  	\ipa{ma}  	\ipa{nɯnɯ} \ipa{tɕu}  	\ipa{\textbf{kɯ}}  	\ipa{ɲɯ-mpja}  \\
valley \textsc{gen} \textsc{dem} \textsc{erg?} \textsc{fact}:be.early because \textsc{dem} \textsc{loc} \textsc{erg?} \textsc{testim}-be.warm \\
\glt `The one in the valley (grows) earlier, because it is warmer there.' (Rhododendron, 64)
\end{exe}

\begin{exe}
\ex \label{ex:comp3}
\gll  \ipa{qɤjdo}  	\ipa{sɤznɤ}  	\ipa{wxti.}     \\
pigeon \textsc{comparative} \textsc{n.pst:}be.big \\
\glt `(It) is bigger than a pigeon.' (Hawk, 7)
\end{exe}


The comparee almost always bears a comparative marker, either \ipa{sɤz} as in \ref{ex:comp1} and \ref{ex:comp3} or its variants \ipa{sɤznɤ}, \ipa{staʁ} and \ipa{staʁnɤ}. It can only be dropped when a mark such as \ipa{sthɯci} `that much, as much' is present (as in \ref{ex:sthWci.mA}).

 When both the standard and the comparee are overt, the standard can occur either after (as in  \ref{ex:comp1}) or before (\ref{ex:comp3}) the comparee.  
 
\begin{exe}
\ex \label{ex:comp3}
\gll 
\ipa{tɕeri}  	\ipa{ɯ-rʑaβ}  	\ipa{aʑo}  	\ipa{sɤz}  	\ipa{wxti,}  \\
but \textsc{3sg.poss}-wife \textsc{1sg} \textsc{comp} \textsc{n.pst:}be.big \\
\glt `But his wife is older than me.' (Relatives, 198)
\end{exe}


The marker \ipa{kɯ} on the standard is always optional, even when the comparee is not overt. When the comparee has a genitival modifier, four constructions are attested. First,  the marker \ipa{kɯ} appears after the whole noun phrase (\ref{ex:GW.WjwaR.kW}). 

\begin{exe}
\ex \label{ex:GW.WjwaR.kW}
\gll 
\ipa{tɤru}  	\ipa{ɣɯ}  	\ipa{ɯ-jwaʁ}  	\ipa{kɯ}  	\ipa{ɲɯ-jndʐɤz}  \\
Pyracantha \textsc{gen} \textsc{3sg.poss}-leaf \textsc{erg?} \textsc{testim}-be.big \\
\glt `The leaves of the Pyracantha are bigger.' (Pyracantha, 130)
 \end{exe}
Second,  the comparee can be elided, leaving only the modifier with the genitive marker \ipa{ɣɯ} (\ref{ex:GW.kW}).  
\begin{exe}
\ex \label{ex:GW.kW}
\gll 
\ipa{ɯ-ru}  	\ipa{tsa}  	\ipa{fse}  	\ipa{ri,}  	\ipa{qrose}  	\ipa{ɣɯ}  	\ipa{kɯ}  	\ipa{mbro.}  \\
\textsc{3sg.poss}-trunk a.little \textsc{fact}:be.like but plant.sp. \textsc{gen} \textsc{erg?} \textsc{fact}:be.high \\
\glt `Its  trunk is a little similar, but the one of the \ipa{qrose} grows higher.' (qrose, 210)
\end{exe}

Third, even the genitive marker can be elided, leaving only the genitial modifier without mark (\ref{ex:qaliaR.nW.kW}).
\begin{exe}
\ex \label{ex:qaliaR.nW.kW}
\gll 
\ipa{tɕeri}  	\ipa{ndʑi-mdoʁ}  	\ipa{ra}  	\ipa{kɯnɤ,}  	\ipa{qaliaʁ}  	\ipa{nɯ}  	\ipa{kɯ}  	\ipa{ɲɯ-ɲaʁ}  \\
but \textsc{3du.poss}-beak \textsc{pl} also, eagle \textsc{top} \textsc{erg?} \textsc{testim}-be.black \\
\glt `As for their beaks, that of the eagle is blacker.' (Kite, 37)
\end{exe}
Fourth, in the case when   the standard and the comparee share the same head noun and differ only by their genitival modifier, \ipa{kɯ} can appear between the modifier and the head noun. In example (\ref{ex:kW.WphW}), for instance \ipa{ɯ-phɯ} `its price' is the head noun of both the standard and the comparee, and \ipa{kɯ} is placed after \ipa{nɯnɯ}, the  modifier of   \ipa{ɯ-phɯ}. Note the absence of genitive marking here.

 \begin{exe}
\ex \label{ex:kW.WphW}
\gll 
\ipa{nɯnɯ}  	\ipa{kɯ}  	\ipa{ɯ-phɯ}  	\ipa{ɲɯ-wxti,}  \\
\textsc{dem} \textsc{erg?} \textsc{3sg.poss}-price \textsc{testim}-be.big \\
\glt `This one is more expensive.' (the price of this one is higher) (Fern, 175)
\end{exe}

The standard marked by \ipa{kɯ} differs from both the A and the instrument in the way it is relativized. While the A is relativized by  nominalizing the verb of the subordinate clause with  \textsc{possessive}+\ipa{kɯ--}, and the instrument by the oblique nominalization prefix \ipa{sɤ--} (see \ref{sec:erg}), the standard is simply relativized by the S nominalization prefix  \ipa{kɯ--} as in   \ref{ex:pjWkAm}. It differs from A nominalization by  the absence of any possessive prefix on the verb.

 
 
\begin{exe}
\ex \label{ex:pjWkAm}
\gll
[\ipa{ɯʑo}  	\ipa{sɤz}  	\ipa{kɯ-wxti}]  	\ipa{rɯdaʁ}  	\ipa{ra}  	\ipa{kɯnɤ}  	\ipa{pjɯ-kɤm}  	\ipa{ɕti}  \\
it \textsc{comp} \textsc{nmlz}:S-big animal \textsc{pl} also \textsc{ipfv}-prevail \textsc{n.pst:be}:\textsc{assert} \\
\glt `It also prevails over animals that are bigger than itself.' (The lion, 23)
  \end{exe}


 
The marker   \ipa{kɯ} is not a typical  \textsc{index}. It is translated by speakers as meaning `more', and does not form a constituent with the verb. Examples like \ref{ex:kW.WphW} rather show that it forms a constituent with either the comparee or with a constituent within the noun phrase corresponding to the comparee, and that it is not necessarily adjacent to the verb.

In a competing comparative construction, we find a different type of index   \ipa{sthɯci} `as much', used with a verb in negative form as in \ref{ex:sthWci.mA} and \ref{ex:sthWci.mWj}. In this construction, the comparative \ipa{sɤz} is optional (as in \ref{ex:sthWci.mA}).
 
 
 \begin{exe}
\ex \label{ex:sthWci.mA}
\gll 
\ipa{ɯ-jme}  	\ipa{nɯ}  	\ipa{βʑɯ}  	\ipa{ɣɯ}  	\ipa{sthɯci}  	\ipa{mɤ-rɲɟi.}  \\
\textsc{3sg.poss}-tail \textsc{top} mouse \textsc{gen} as.much \textsc{neg}-be.long \\
\glt `Its tail is not as long as that of the mouse.'  (Mouse, 216)
 \end{exe}
 
  \begin{exe}
\ex \label{ex:sthWci.mWj}
\gll 
 \ipa{tɕhɯkɤɣar}  	\ipa{ʑmbri}  	\ipa{nɯ}  	\ipa{sɤz}  	\ipa{sthɯci}  	\ipa{mɯ́j-rɲɟi.}  \\
beach willow \textsc{top} \textsc{comp} as.much \textsc{neg:const}-be.long \\
\glt `It is not a long as the beach willow.' (Willow, 21)
  \end{exe}
  
Japhug has thus two (mutually exclusive) types of indexes: a case marker index \ipa{kɯ} index on the comparee noun phrase or an adverbial modifier \ipa{sthɯci} `as much' on the verb. These indexes can in turn be combined with degree modifiers such as \ipa{kɯ-xtɕɯ-xtɕi} `a little', \ipa{dɯxpa} `barely' or \ipa{khro} `much' (see \ref{ex:kW.kWxtCWxtCi} and \ref{ex:sthWci.kWxtCWxtCi}).
  
    \begin{exe}
\ex \label{ex:kW.kWxtCWxtCi}
\gll 
  \ipa{ɯ-rna}  	\ipa{ɯ-ntsi}  	\ipa{nɯ}  	\ipa{kɯ}  	\ipa{kɯ-xtɕɯ-xtɕi}  	\ipa{ɲɯ-mna}  	\ipa{ri,}  	\ipa{nɯ}  	\ipa{kɯnɤ}  	\ipa{khro}  	\ipa{mɯ́j-pe}  \\
  \textsc{3sg.poss}-ear \textsc{3sg.poss}-one.of.a.pair \textsc{top} \textsc{erg} \textsc{nmlz:S/A-redp}-be.small \textsc{testim}-be.well but \textsc{dem} also much \textsc{neg:const}-be.good \\
  \glt `One of his ears is a bit better (than the other), but even this one is not very good.' (Relatives, 75)
  \end{exe}
 
     \begin{exe}
\ex \label{ex:sthWci.kWxtCWxtCi}
\gll 
  \ipa{kumpɣɤtɕɯ}  	\ipa{ɯ-phoŋbu}  	\ipa{jamar}  	\ipa{ɣɤʑu}  	\ipa{ri,}  	\ipa{nɯ}  	\ipa{sthɯci}  	\ipa{kɯ-xtɕɯ-xtɕi}  	\ipa{mɯ́j-wxti.}  \\
  chicken \textsc{3sg.poss}-body about exist:\textsc{sens} but \textsc{dem} as.much \textsc{nmlz:S/A-redp}-be.small \textsc{neg:const}-be.big \\
\glt `Its body about like that of a chicken, but a little not as big as that.' (tasa.pɣɤtɕɯ, 85)
  \end{exe}

 
The index \ipa{kɯ} is not restricted to the comparative constructions seen above. It also occurs in  tropative constructions (see \citealt{jacques13tropative}) with the verb \ipa{sɯpa} `consider' an an infinitival complement (\ref{ex:kWkWmWm}), with the tropative derivation (\ref{ex:nAmWm}) on with experiencer verbs such as \ipa{rga} `like' that include  a stimulus in their argumental structure (\ref{ex:kWrga}).

\begin{exe}
\ex \label{ex:kWkWmWm}
\gll  \ipa{tɕe}   	\ipa{thoŋ-raʁ}   	\ipa{nɯ}   	\ipa{kɯ}   	\ipa{kɯ-mɯm}   	\ipa{tu-sɯpa-nɯ}   	\ipa{ŋu}   \\
\textsc{coord} bucket.alcohol \textsc{dem} \textsc{erg} \textsc{inf:stat}-be.tasty \textsc{ipfv}-consider-\textsc{pl} \textsc{fact}:be \\
\glt `They consider  bucket alcohol to be tastier (than the pan-alcohol).' (Distilled alcohol, 15)
\end{exe}

\begin{exe}
\ex \label{ex:nAmWm}
\gll  \ipa{thoŋraʁ} 	\ipa{nɯ} 	\ipa{kɯ} 	\ipa{ɲɯ-nɤ-mɯm-nɯ} \\
 bucket.alcohol \textsc{dem} \textsc{erg}  \textsc{const-trop}-be.tasty-\textsc{pl} \\
 \glt `They consider  bucket alcohol to be tastier (than the pan-alcohol).'  (elicited).
\end{exe}

\begin{exe}
\ex \label{ex:kWrga}
\gll \ipa{tsuku}   	\ipa{tɕe}   	 [\ipa{tɯŋguraʁ}   	\ipa{kɯ}   	\ipa{kɯ-rga}]   	\ipa{ɣɤʑu-nɯ,}   		\ipa{tsuku}   	\ipa{tɕe}   	[\ipa{thoŋraʁ}   	\ipa{kɯ}   	\ipa{kɯ-rga}]   	\ipa{ɣɤʑu-nɯ}   \\
some \textsc{coord} pan.alcohol \textsc{erg?} \textsc{nmlz:S/A}-like exist\textsc{:sensory}-\textsc{pl} some \textsc{coord} pan.alcohol \textsc{erg?} \textsc{nmlz:S/A}-like exist\textsc{:sensory}-\textsc{pl} \\
\glt `There are people who like more pan alcohol, and there are who like more bucket alcohol.' (Distilled alcohol, 17-18)
\end{exe}


%tɕeri ndʑimdoʁ ra kɯnɤ, qaliaʁ nɯ kɯ ɲɯɲaʁ
%
%aʁɤndɯndɤt tu-ɬoʁ ɕti. ri ɕkrɤz chondɤre, tɯrgi ɯ-ŋgɯ kɯ dɤn.
%tshAYCAnW 28
 
\section{Degree construction} \label{sec:degree}
 
 
Japhug has two degree constructions build by nominalizing a verb (generally an adjective)\footnote{It is possible to define a category of adjectives in Japhug as a subclass of stative intransitive verbs: the verbs that allow the tropative derivation \ipa{nɤ--} (\citealt{jacques13tropative}).}  with the action nominalization \ipa{tɯ--} prefix and a possessive prefix coreferent with the referent presenting the property described by the nominalized verb. 

This degree nominal (like \ipa{ɯ-tɯ-tɕur} `its sourness' in \ref{ex:YWsWxtCur}) is the S in both constructions,  both of which are illustrated by example \ref{ex:YWsWxtCur}. 

First, the degree nominal can be  combined with an adjective expressing a degree such as \ipa{saχaʁ} `be extremely ...', \ipa{sɤre} `be funny, be extremely ...', \ipa{tɕhom} `be excessive', in the \textit{monoclausal nominalized degree} construction, exemplified by the first sentence in \ref{ex:YWsWxtCur}.

Second, it can be associated  with one or several  clause(s) containing a simile describing the degree of the property, in the \textit{multiclausal nominalized degree} construction. In multiclausal nominalized degree constructions, the nominalized verb has to be combined with the marker \ipa{kɯ}, as in the second sentence of example \ref{ex:YWsWxtCur}.

%ʁlaŋsaŋtɕhin	ɯ-tɯ-ɣɤχsrɯ	kɯ	saŋtɕɤn-mbrɯɣmu	χsɯ-sŋi	χsɤ-rʑaʁ	chɤ-ru
%226



\begin{exe}
\ex \label{ex:YWsWxtCur}
\gll 
\ipa{mtɕhi}  	\ipa{ɯ-mat}  	\ipa{rca}  	\ipa{ɯ-tɯ-tɕur}  	\ipa{saχaʁ.}  	\ipa{ɯ-tɯ-tɕur}  	\ipa{\textbf{kɯ}}  	[\ipa{tɯ-kɯr}  	\ipa{ɯ-ŋgɯ}  	\ipa{lú-wɣ-rku}  	\ipa{qhe}  	\ipa{maka}  	\ipa{ɲɯ-sɯ-ɤmɯzɣɯt}  	\ipa{qhe,}  	\ipa{tɯ-phoŋbu}  	\ipa{ra}  	\ipa{kɯnɤ}  	\ipa{ɲɯ-sɯx-tɕur}  	\ipa{kɯ-fse}  	\ipa{ɕti}]  \\
sea.buckthorn \textsc{3sg.poss}-fruit \textsc{top} \textsc{3sg-nmlz:degree}-be.sour n.pst:be.extremely \textsc{3sg-nmlz:degree}-be.sour \textsc{erg?} \textsc{indef:poss}-mouth \textsc{3sg}-inside \textsc{ipfv:upstream-inv}-put.in \textsc{lnk} at.all \textsc{ipfv-caus}-be.evenly.distributed \textsc{lnk} \textsc{indef:poss}-body \textsc{pl} also \textsc{ipfv-caus}-be.sour \textsc{nmlz:S/A}-be.like \textsc{fact}:be:\textsc{affirm} \\
\glt `The fruit of the sea-buckthorn is very sour, so sour that when one puts it in one's mouth, it makes it completely (sour), and it is as if one's (whole) body became sour.' (Sea-buckthorn, 66)
\end{exe}

The nominalized verb and the marker \ipa{kɯ} of multiclausal  nominalized degree constructions form a constituent and can be right dislocated together, as in \ref{ex:WtWmbjom}.

 \begin{exe}
\ex \label{ex:WtWmbjom}
\gll 
\ipa{tɯ-ci}  	\ipa{ɯ-ɣmbɤj}  	\ipa{nɯ}  	\ipa{tɕu,}  	  	\ipa{tɤ-rtsa}  	\ipa{kɯ-xtɕɯ-xtɕi}  	\ipa{nɯ}  	\ipa{χanɯni}  	\ipa{ju-ɕe}  	\ipa{ɲɯ-ŋu}  	\ipa{ma}  	\ipa{nɯ}  	\ipa{ma}  	\ipa{kɯ-saχsɤl}  	\ipa{maŋe,}  	\ipa{ɯ-tɯ-mbjom}  	\ipa{\textbf{kɯ}.}  \\
\textsc{indef.poss}-water \textsc{3sg.poss}-side \textsc{top} \textsc{loc} \textsc{indef.poss}-wave \textsc{nmlz:S/A-redp}-be.small \textsc{top} a.little \textsc{ipfv}-go \textsc{testim}-be apart.from \textsc{dem} apart.from   \textsc{nmlz:S/A}-be.clear not.exist:\textsc{sens} \textsc{3sg-nmlz:degree}-be.quick \textsc{erg?} \\
\glt `(When it dives into the water), it is so quick that one can only see little ripples near the shore.'
 (Kingfisher, 54)
\end{exe}

Unlike other nominalizing prefixes such as the \ipa{kɯ--} (S/A participle) or \ipa{sɤ--} (oblique participle), the degree action nominal in \ipa{tɯ--} cannot bear any TAM markers (whether prefixes or stem alternation). However, TAM is not neutralized in this construction: it is marked on the following verb; compare \ipa{saχaʁ} `is extremely X' in \ref{ex:YWsWxtCur} with its past imperfective form \ipa{pɯ-saχaʁ}.
in \ref{ex:ndZtAmWmi}.

 \begin{exe}
\ex \label{ex:ndZtAmWmi}
\gll 
 \ipa{tɕendɤre}  	\ipa{ndʑi-tɯ-ɤmɯmi}  	\ipa{ndʑi-tɯ-scit}  	\ipa{pɯ-saχaʁ}  	\ipa{ʑo}  	\ipa{ɲɯ-ŋu}  \\
 \textsc{lnk} \textsc{3du-nmlz:degree}-be.in.good.terms \textsc{3du-nmlz:degree}-be.happy \textsc{pst.ipfv}-be.extremely \textsc{emph} \textsc{testim}-be \\
 \glt `They were very happy together.' (Lobzang, 13)
\end{exe}

This construction is not restricted to degree nominals in \ipa{tɯ--}, it also attested with bare action nominals (see \citealt[7-9]{jacques14antipassive}) as in \ref{ex:tAmtsWr.kW}.
    \begin{exe}
  \ex  \label{ex:tAmtsWr.kW}  
  \gll \ipa{tɤ-mtsɯr}  	\ipa{kɯ}  	\ipa{mɯ́j-cha-a}  \\
  \textsc{indef.poss}-be.hungry \textsc{erg} \textsc{neg:const}-can-\textsc{1sg} \\
\glt `I am starving.'  (A calque from Chinese: \zh{我饿得不行了}; Crow and Raven, 53) 
   \end{exe} 
  
 The marker \ipa{kɯ} is not restricted to multiclausal nominalized degree construction, it also occurs optionally with the monoclausal one, as in \ref{ex:kW.pWsaXaR}, though this use is very rare.
 
      \begin{exe}
  \ex  \label{ex:kW.pWsaXaR}  
  \gll 
   \ipa{tɤ-ɣɲat}  	\ipa{tɤ-mtsɯr}  	\ipa{kɯ}  	\ipa{pɯ-saχaʁ}  	\ipa{ʑo}  	\ipa{ɲɯ-ŋu}  \\
      \textsc{indef.poss}-be.tired     \textsc{indef.poss}-be.hungry \textsc{erg?} \textsc{pst.ipfv}-be.extremely \textsc{emph} \textsc{testim}-be \\
      \glt `He was extremely tired and hungry.' (Lobzang, 66)
   \end{exe} 

Optionally, the infinitive of the verb `say' \ipa{kɤ-ti} can be inserted between the degree nominal and the marker \ipa{kɯ} as in \ref{ex:kAti.kW}.

    \begin{exe}
  \ex  \label{ex:kAti.kW}  
  \gll
\ipa{tɯtsɣe}  	\ipa{kɤ-βzu}  	\ipa{ɯ-tɯ-cha}  	\ipa{kɤ-ti}  	\ipa{kɯ}  	\ipa{pɯ-saχaʁ}  	\ipa{ʑo}  	\ipa{ɲɯ-ŋu.}  \\
commerce \textsc{inf}-make \textsc{3sg-nmlz:degree}-can \textsc{inf}-say \textsc{erg?} \textsc{pst.ipfv}-be.extremely \textsc{emph} \textsc{testim}-be \\
\glt `He was extremely proficient in commerce.' (Slopdpon, 2)
   \end{exe} 
%	113	ɕ-pjɤ-nɤʁáʁ-ndʑi	tɕe,	ɯ-tɯ-sɤscit	kɯ	lu	χsɯ-xpa	pjɤ-tsú-ndʑi	ri,
% 
%A	4	tɤ-mu	kɤ-tsa	ci	pɯ-tú-ndʑi,	ndʑi-tɯ-ŋgɯ	kɯ	pɯ-ɣɤdí-ndʑi,

 
%
%ɯjɯ, ɯsɤndo tɯjaʁ kɯ kurɤspra kɯkhɯ jamar mɤɕtʂa chɯ́wɣβʑoʁ
%要把把子削成可以握住的那么细27
%
%tasasqari
%
%mu nɯ ndɤre, nɯ sthɯci mɯ́j-fse kɯ qhjiqhji ʑo ɲɯ-ap tɕe ɲɯ-pɣi.
%scuz>_tɤkhe pɣɤtɕɯ
 
 
 
 
\section{Typological perspectives}

\subsection{Isomorphism between ergative and index of comparison}
The most unexpected isomorphy between the various markers having the form \ipa{kɯ} in Japhug is that between the ergative / instrumental on the one hand and the comparee on the other hand.   

The optional clitic \ipa{kɯ} on the comparee  clearly forms a constituent with the comparee NP (or its genitival modifier in a few limited cases), not with the verb. The comparee NP, though a S from the point of view of agreement with the predicate and from that of relativization, optionally receives  the same flagging as the A.

 
While many comparative constructions in the world's language do treat the comparee in the same way as the A (types B, C and E in \citealt[789]{dixon08comparative}'s survey, `exceed comparative' in  \citealt{stassen11comparative}), in all these constructions the standard has the same status as the O.  The comparative construction in Japhug should be classifed differently. 

Since the standard NP is marked by an oblique case (\ipa{sɤz} or \ipa{staʁ}) specific to this construction, and since the parameter of comparison is marked by an intransitive predicate, the Japhug comparative construction belongs to  \citet{stassen11comparative}'s `particle comparative' type and to \citealt[789]{dixon08comparative}'s type A2.  While this type is not attested in combination with  ergative flagging in WALS (cf Table \ref{tab:stassen}, obtained by combining \citealt{stassen11comparative} with \citealt{comrie11case}), this may be due to the assignment of particular languages to the \textit{locative} rather than \textit{particle}  comparative types, and does not reflect a real gap in the data.

On the other hand, no case of a marker   on the comparee NP isomorphic with the ergative or instrumental, as \ipa{kɯ} in Japhug, has been documented in the previous survey of comparative constructions.

\begin{table}[h]
\caption{Combination of chapters 98 (Alignment of Case Marking of Full Nouns) and 121 (Comparative constructions) of the WALS} \label{tab:stassen}
\resizebox{\columnwidth}{!}{
\begin{tabular}{lllllll}
\toprule
   &	 Neutral   &	 Nominative    &	 Nominative -   &	 Ergative -    &	 Tripartite   &	 Active-   \\   
   &&- accusative&accusative  &absolutive & &inactive\\
   && (standard) &(marked nominative)&&\\
   \midrule
 Locational   &	7  &	13  &	2  &	5  &	2  &	1  \\
Exceed   &	10  &	  &	1  &	  &	  &	  \\
Conjoined   &	10  &	3  &	  &	2  &	  &	  \\
Particle   &	3  &	8  &	  &	  &	  &	1  \\
\bottomrule
\end{tabular}}
\end{table}
 
Isomorphy between ergative / instrumental and the standard NP, rather than the comparee NP, is expected given the well attested grammaticalization pathways \ref{ex:abl2A} and \ref{ex:abl2comp} (\citealt[29]{heine-kuteva02}).
 


\begin{exe}
\ex \label{ex:abl2A}
\glt \textsc{ablative} $\rightarrow$ \textsc{agent} 
\ex \label{ex:abl2comp}
\glt \textsc{ablative} $\rightarrow$ \textsc{comparative}
\end{exe}

Since an locational case like ablative can be derived both as a comparative marker (on the standard) and  an ergative, if both grammaticalizations occurs in the same language, isomorphy between ergative and comparative will occur. This situation is attested in Amdo Tibetan, a language with which Japhug is in contact with, as can be seen in example from \citet[255]{skalbzang02dialectes}.

\begin{exe}
\ex \label{ex:vbris}
\gll \ipa{ndzomi}  	\ipa{oma}  	\ipa{ndʐi}  	\ipa{maŋ}  	\ipa{ret}  \\
female.hybrid.yak:\textsc{gen} milk female.yak:\textsc{erg} be.numerous \textsc{copula} \\
\glt `The female hybrid yak has more milk that the female yak.'
\end{exe}

Using the same marker on the comparee NP and the A on the other hand is a typological oddity, whose explanation can only be sought for by proposing a historical account of the grammaticalization of the markers \ipa{kɯ} in all the constructions where they are attested.
 

 
\subsection{Historical pathways}


While the structure of the Japhug comparative construction, and in particular the homophony between the index and the ergative, does not appear to have clear typological parallel elsewhere, most of the uses of the \ipa{kɯ} marker in Japhug described in the previous sections can be argued to be derived from the basic ergative-instrumental function. 

In this section, a series of diachronic pathways leading from one construction to the other are postulated. In the absence of ancient written evidence from japhug and the other Rgyalrong data, part of the following developments are necessarily hypothestical, and in some cases several competing explanations are provided.


Only grammatical changes caused by the reanalysis of an existing construction in an ambiguous context are proposed, and examples of potentially ambiguous sentences (pivot constructions) taken from our Japhug corpus are provided in each case. The semantic changes hypothesized in this section  have either attested parallels in other language families or are straightforward if paraphrased in English. All the intermediate stages of the chain of reanalysis proposed here are actually attested synchronically at least in specific contexts in our Japhug corpus.


 \subsubsection{Origin}
The point of origin postulated here for most of the uses of \ipa{kɯ} is the Ergative and Instrumental. 

The most probable etymological hypothesis for this marker is a borrowing from Amdo Tibetan, the dominant prestige  language of the area before 1949. In Amdo the Ergative and Genitive are only distinguished in pronouns, for all other forms there is syncretism, and the Genitive/Ergative is realized as as \ipa{ɣə}, \ipa{kə} or fronting vowel alternation depending on the stem form of the last word of the preceding NP (\citealt[62]{haller04themchen}). The ergative is also used for instrumental.

The \ipa{ɣə} allomorph of the Amdo Genitive/Ergative was borrowed as the Japhug Genitive \ipa{ɣɯ}, while the \ipa{kə} allomorph was borrowed  as the ergative \ipa{kɯ}. 

Since none of the functions of Japhug \ipa{kɯ} are found in the Ergative/Instrumental \ipa{kə} / \ipa{ɣə} in Amdo Tibetan, these all represent Japhug-internal developments, and can only derive from, or be unrelated to, the Ergative/Instrumental; the directionality cannot be the opposite.



 \subsubsection{\textsc{instrumental} $\rightarrow$ \textsc{distributive} }
  
Syncretism between agent marker and distributive is well-attested in Romance languages. For instance, in French, the preposition  \ipa{par}  is used to mark the instrument, the (optional) agent in passive constructions and also occurs with a distributive meaning. Yet, it is unlikely that the distributive meaning of this preposition originates from the instrumental function; rather, it comes from its  spatial and temporal use `through, along' (see \citealt[213]{wartburg58few8}). 
Thus, the Romance evidence does not support the existence grammaticalization path \textsc{instrumental} $\rightarrow$ \textsc{distributive}.

In Japhug, the distributive meaning in the examples quoted in section \ref{sec:distributive} is due in part to the use of classifiers, which intrinsically allow a distributive interpretation. This interpretation is also found with the ergative / instrumental \ipa{kɯ} as in \ref{ex:tWrme.tWrdoR}.
\begin{exe}
\ex \label{ex:tWrme.tWrdoR}
\gll 
\ipa{tɯrme}  	\ipa{tɯ-rdoʁ}  	\ipa{kɯ}  	\ipa{chɤmdɤru}  	\ipa{tɯ-ldʑa}  	\ipa{tu-nɯ-ndɤm}  \\
man one-\textsc{cl} \textsc{erg}  straw one-\textsc{cl}:long.objects \textsc{ipfv-auto}-take[III] \\
\glt `Each person (of them) had a straw (for drinking wine).' (Chang, 37)
\end{exe}
 
 The construction in section \ref{sec:distributive} is due to the partial reinterpretation of the phrase classifier + ergative / instrumental with a verb meaning `to exchange' as a distributive form, later generalized to other contexts.%recheck
 

 \subsubsection{\textsc{instrumental} $\rightarrow$ \textsc{cause}} \label{sec:instr2cause}

In all described varieties of Tibetan, from the Classical Language to modern dialects, the ergative can be used to mark the causal subordinate clause, either on its own or with nouns such as \ipa{dbaŋ} `power' or \ipa{ɕed} `strength'. In Amdo Tibetan, for instance, the ergative \ipa{ɣə} appears in examples such as \ref{ex:erg.caus.amdo} (\citealt[271-272]{vbrugmo03maqu}; for similar examples in Lhasa and classical Tibetan see \citealt[129]{tounadre96erg}).
 \begin{exe} 
\ex \label{ex:erg.caus.amdo}
\gll  \ipa{kʰokjaŋwi} 	\ipa{rtsatʰaŋ-na} 	\ipa{wdatrdot-nə} 	\ipa{ɣə} 	\ipa{rtsathaŋ} 	\ipa{hdandʐa-ni} 	\ipa{kʰokrdʑa} 	\ipa{jaŋmo} 	\ipa{jɔkʰə}  \\
boundless steppe-\textsc{loc} live-\textsc{nmlz} \textsc{erg} steppe like-\textsc{nmlz:gen} heart broad have \\
\glt  `Because he lives on the boundless steppe, his mind is broad like the steppe.'
\end{exe}  
 
It is thus probable that the whole construction with \ipa{ɯ-xɕɤt kɯ} was borrowed from Tibetan together with the use of \ipa{kɯ} as a simple ergative / instrumental.

The use of  \ipa{kɯ} in Japhug as a clausal linker is restricted outside of the  \ipa{ɯ-xɕɤt kɯ} construction; it is essentially limited to use with an abstract noun.\footnote{Note however that in the closely related language Tshobdun, the cognate ergative marker \ipa{kə} does appear with causal subordinate clauses, even finite ones (see \citealt[479]{sun12complementation}).}
 
 \subsubsection{\textsc{cause} $\rightarrow$ \textsc{infinitival manner}}
Two hypotheses can be proposed to account for the  use of  \ipa{kɯ} to express manner or circumstance.

First,   the ergative / instrumental in Tibetan languages can also mark manner (see in  \citealt[128]{tounadre96erg} on Lhasa and Classicial Tibetan). In Amdo Tibetan, the ergative / instrumental \ipa{ɣə} is well attested as a gerund marker (example \ref{ex:erg.gerund.amdo} from \citealt[162; 167]{haller04themchen}), a use relatively close to that of Japhug \ipa{kɯ} in manner subordinate clauses (section \ref{sec:manner}).

 \begin{exe} 
\ex \label{ex:erg.gerund.amdo}
\gll   \ipa{ta} 	\ipa{ɲiɣa} 	\ipa{ɸtsi-ɣə} 	\ipa{ɸtsi-ɣə} 	\ipa{ta} 	\ipa{tʰaŋ-a} 	\ipa{rdom-sʰuŋ}  \\
now \textsc{3du} play:\textsc{pst}-\textsc{erg} play:\textsc{pst}-\textsc{erg} now steppe-\textsc{loc} roam-\textsc{pfv} \\
\glt  `They roamed the steppe, playing (around).'
\end{exe}  

It is possible that, like the causal use and the ergative / instrumental, this function was directly borrowed from Amdo.

Alternatively, it could be a Japhug-internal development from the causal use of \ipa{kɯ}  (\ref{sec:instr2cause}) or even directly from its use as an instrument marker, following the pathway \ref{ex:instr2manner} proposed by \citet[180]{heine-kuteva02}:
   \begin{exe}
\ex \label{ex:instr2manner}
\glt \textsc{instrument} $\rightarrow$ \textsc{manner} 
\end{exe}

 

\subsubsection{\textsc{cause} $\rightarrow$ \textsc{multiclausal degree} construction} \label{sec:cause2degree}

Multiclausal degree constructions  present an intrinsic ambiguity between the attested degree interpretation (`so X that Y') and a potential causal interpretation (`because of X, Y').  For instance, the sentence \ref{ex:WtWrga.kW} would also make sense  with a causal interpretation (`She forgot it because of her being (so) happy').



 \begin{exe} 
 \ex \label{ex:WtWrga.kW}
\gll 
\ipa{tɤ-mu}  	\ipa{nɯ}  	\ipa{kɯ,}  	  	\ipa{ɯ-tɯ-rga}  	\ipa{kɯ}  	\ipa{ɲɤ-nɯ-jmɯt}  	\ipa{qhe,}  \\
\textsc{indef.poss}-mother \textsc{top} \textsc{erg} \textsc{3sg-nmlz:degree}-be.happy \textsc{erg?} \textsc{evd-auto}-forget  \textsc{lnk} \\
\glt `The old woman was so happy that she forgot (how to do).' (The frog, 261)
 \end{exe} 
 
 

% \begin{exe} 
% \ex \label{ex:tWYJAt.kW}
%\gll 
%\ipa{tɯ-ɲɟɤt}  	\ipa{kɯ}  	\ipa{tɯ-ci}  	\ipa{ɯ-ŋgɯ}  	\ipa{pjɤ-mtsaʁ}  	\ipa{qhe}  	\ipa{pjɤ-si.}  \\
%\textsc{nmlz:action}-regret \textsc{erg} \textsc{indef.poss}-water \textsc{3sg}-inside \textsc{evd:down}-jump \textsc{lnk} \textsc{evd}-die \\
%\glt Out of regret, he jumped into the river and died. (The raven 28.08.12, 177)
% \end{exe} 
% 
 

Although the two meanings would appear to be entirely unrelated, the derivation from causal to degree is straightforward: for a property to be the cause of an event or a situation, this property must reach a sufficiently high degree to trigger a change of state or an action. Thus,  the causal construction necessary entails high degree, and evolution from the former to the latter is simply a restriction of the semantics of the construction.

Therefore, the historical origin of the multiclausal   degree construction in \ipa{kɯ}   (section \ref{sec:degree})   can be hypothesized to derive  from  the causal use of \ipa{kɯ}   with abstract nouns (attested in Japhug, see section \ref{sec:cause}). While derivation from the nominalized manner linking to the clausal degree construction would appear to be an alternative possibility for the origin of the multiclause degree construction, it is unlikely in that only infinitives in \ipa{kɤ--} or \ipa{kɯ--} are found in nominalized manner linking.  

 
  \subsubsection{\textsc{infinitival manner} $\rightarrow$ \textsc{finite manner}}

The verb of the subordinate clause in the infinitival manner linking appears in the infinitive, which is either \ipa{kɤ--} for (dynamic verbs) or \ipa{kɯ--} (for adjectives, copulas, some modal auxiliaries and intransitive dynamic verbs that are incompatible with an animate S). These prefixes happen to be homophonous with, and probably historically derived from, the P-nominalization \ipa{kɤ--} and the S/A nominalization \ipa{kɯ--} prefixes.

Thus, a infinitival clause preceding \ipa{kɯ} can easily be reinterpreted as a headless relative; this ambiguity is synchronically attested in our Japhug corpus, where a surface form such as \ipa{kɯ-wɣrum} in \ref{ex:nmlz.inf.ambiguous} can be either analyzed as the stative infinitive of \ipa{wɣrum} `be white' or as its S/A nominalized form `the white one'.
 
 \begin{exe}
\ex \label{ex:nmlz.inf.ambiguous}
\gll \ipa{nɯnɯ}  	\ipa{kɯ-ɲaʁ}  	\ipa{ʁɟa}  	\ipa{tu,}  	\ipa{kɯ-wɣrum}  	\ipa{ʁɟa}  	\ipa{tu,}   \ipa{qhe}  	[_{relative}[_{manner}\ipa{kɯ-ɲaʁ}  	\ipa{qhe}  	\ipa{kɯ-wɣrum}  	\ipa{kɯ}]_{manner}   	\ipa{kɯ-ɤscaʁa}]_{relative}  	\ipa{tu,}  \\
\textsc{dem} \textsc{nmlz}:S/A-be.black completely \textsc{fact}:exist
\textsc{nmlz}:S/A-be.white completely \textsc{fact}:exist
\textsc{lnk}   \textsc{inf:stat}-be.black \textsc{lnk} \textsc{inf:stat}-be.white \textsc{erg?} \textsc{nmlz}:S/A-be.piebald  \textsc{fact}:exist \\
\glt `Some are completely black, some are completely black, and some are black and white so that they are piebald.' (Piebald,  219)
\end{exe}

In sentence \ref{ex:nmlz.inf.ambiguous}, the infinitival manner subordinate clause is embedded within a headless nominalized relative clause. The nominalized main verb of the relative \ipa{kɯ-ɤscaʁa} `the one that is piebald' (which corresponds to the main clause of the manner linking) receives a prefix \ipa{kɯ--} that is in surface the same as the stative infinitive \ipa{kɯ--} of the verbs in the manner subordinate clause.

From a sentence such as \ref{ex:nmlz.inf.ambiguous}, if the verbs of the manner clause are reanalyzed by the speakers as S/A nominalization, it can then be de-relativized and turned into a construction with finite verbs in both the main and the subordinate clause.. 

Reanalysis could only take place in the case of adjectives, copulas or other verbs with a \ipa{kɯ--} infinitive; in the case of verbs with \ipa{kɤ--} infinitive, since S/A nominalization differs from the infinitive, reanalysis in the subordinate infinitival clause would not be possible. 

Then, it became possible to extend this construction to all verbs, including dynamic ones such as \ipa{rɤʑi} `stay' (whose infinitive is \ipa{kɤ-rɤʑi} `to stay') as in \ref{ex:tWcipaR}.

  \begin{exe}
\ex \label{ex:tWcipaR}
\gll 
\ipa{nɤ-stu}  	\ipa{mɤ-tɯ-rɤʑi}  	\ipa{kɯ}  	\ipa{tɯcipaʁ}  	\ipa{ʑo}  	\ipa{ɲɯ-tɯ-fse}  \\
\textsc{2sg.poss}-place \textsc{neg-2-n.pst}-stay \textsc{erg?} water.beetle \textsc{emph} \textsc{testim}-2-be.like \\
\glt `You don't sit straight, you are like a water beetle.'  (Water beetle, 26)
 \end{exe}
 
 
 \subsubsection{ \textsc{infinitival manner} $\rightarrow$  \textsc{monoclausal degree}} \label{sec:manner2adj}
The  \ipa{kɯ} clitic marker, while obligatory in the multiclausal nominalized degree construction, is very rare in the monoclausal one. Sentence \ref{ex:kW.pWsaXaR2}  (reproduced from \ref{ex:kW.pWsaXaR}) is the only such example in the whole corpus, but similar examples can be elicited.

      \begin{exe}
  \ex  \label{ex:kW.pWsaXaR2}  
  \gll 
   \ipa{tɤ-ɣɲat}  	\ipa{tɤ-mtsɯr}  	\ipa{kɯ}  	\ipa{pɯ-saχaʁ}  	\ipa{ʑo}  	\ipa{ɲɯ-ŋu}  \\
      \textsc{indef.poss}-be.tired     \textsc{indef.poss}-be.hungry \textsc{erg?} \textsc{pst.ipfv}-be.extremely \textsc{emph} \textsc{testim}-be \\
      \glt `He was extremely tired and hungry.' (Lobzang, 66)
   \end{exe} 

One possible explanation to account for this is to suppose generalization from the multiclausal nominalized construction. Yet, the fact that the monoclausal nominalized degree construction commonly occurs with the infinitive of the verb `say' \ipa{kɤ-ti} in combination with   \ipa{kɯ} as in \ref{ex:kAti.kW2}  suggests otherwise.
      \begin{exe}
  \ex  \label{ex:kAti.kW2}  
  \gll 
\ipa{tɤ-rʑaβ} 	\ipa{ra} 	 	\ipa{nɯ-tɯ-rga} 	\ipa{kɤ-ti} 	\ipa{kɯ} 	\ipa{pɯ-saχaʁ} 	\ipa{ɲɯ-ŋu} 	\ipa{tɕe} \\
\textsc{indef.poss}-wife \textsc{pl} \textsc{3pl-nmlz:degree}-be.glad \textsc{inf}-say \textsc{erg?} \textsc{pst.impf}-be.extremely \textsc{testim}-be \textsc{lnk} \\
\glt `The wives were extremely glad.' (The brides, 7)
   \end{exe} 
   
Example \ref{ex:kAti.kW2}  from the point of view of syntactic structure is an example of \ipa{kɯ} in manner subordinate clauses (section \ref{sec:manner}). As for its syntactic function, the form \ipa{kɤ-ti kɯ} is a topicalizer, akin to English  `talking of ...', and example \ref{ex:kAti.kW2}  could be literally glossed as `Talking of the gladness of the wives, it was extreme'. 
 
Thus, the marker \ipa{kɯ} in example \ref{ex:kW.pWsaXaR2}  is more likely to reflect a topicalized construction such as that in \ref{ex:kAti.kW2}  with elision of the infinitive \ipa{kɤ-ti}. It is perfectly grammatical to add \ipa{kɤ-ti} before \ipa{kɯ} in sentence \ref{ex:kW.pWsaXaR2}. 

Thus, the presence of the marker \ipa{kɯ} in monoclausal degree constructions is unrelated to that in multiclausal degree constructions, and derives from the use of \ipa{kɯ} in manner subordinate clauses. 


 \subsubsection{\textsc{finite manner} $\rightarrow$ \textsc{adversative} or \textsc{cause} $\rightarrow$ \textsc{adversative} }
 
The adversative use of \ipa{kɯ}, while treated separately  in section \ref{sec:advers}, does not fundamentally  differ from the finite manner linking. Rather, it represents one of the possible interpretations of the manner linking. Thus, a sentence such as  \ref{ex:tuGAzjAGlAG} can be construed either with adversative meaning (`it cannot stay in place like other boulders; rather, it rocks around continuously') or without it.
 
 \begin{exe}
\ex \label{ex:tuGAzjAGlAG}
\gll  
    \ipa{mɤʑɯ}  	\ipa{rŋgɯ}  	\ipa{ci}  	\ipa{ɣɤʑu}  	\ipa{ri,}  	\ipa{ɯ-zda}  	\ipa{ra}  	\ipa{kɯ-fse}  	\ipa{ku-rɤʑi}  	\ipa{mɯ́j-khɯ}  	\ipa{kɯ}  	\ipa{tu-ɣɤzjɤɣlɤɣ}  	\ipa{nɤ}  	\ipa{tu-ɣɤzjɤɣlɤɣ}  	\ipa{ɲɯ-ra}  	\ipa{tɕe,}  \\
    again boulder \textsc{indef} exist:\textsc{sens} but \textsc{3sg.poss}-companion \textsc{pl} \textsc{inf:stat}-be.like \textsc{ipfv}-stay \textsc{neg:const}-can \textsc{ipfv}-rock.around \textsc{lnk} \textsc{ipfv}-rock.around  \textsc{testim}-have.to \textsc{lnk}     \\
   \glt `Then, there is this boulder, it cannot stay in place like other boulders as it rocks around continuously.' (Divination  56)
 \end{exe}
 
 While a direct evolution from causative to adversative without intermediate stage as a manner linking has been documented in the history of Italian for instance (the linker \ipa{però}, see  \citealt{mauri12adversative}), the simplest explanation for the development of adversative \ipa{kɯ} in Japhug is a specialization of the use of the finite manner clause linking.
 
 
 
  \subsubsection{\textsc{adversative} $\rightarrow$ \textsc{index} of comparison}
  As in most previous constructions, several hypotheses can be entertained to account for the origin of \ipa{kɯ} as a index on the comparee as in example \ref{ex:kW.YWjpum}  (section \ref{sec:comparative}).
  
        \begin{exe}
  \ex  \label{ex:kW.YWjpum}  
  \gll 
  \ipa{tɯ-ɣli} 	\ipa{kɯ-dɤn} 	\ipa{ɯ-stu} 	\ipa{qandʐe} 	\ipa{nɯ} 	\ipa{kɯ} 	\ipa{ɲɯ-jpum.} \\
  \textsc{indef.poss}-excrement \textsc{nmlz}:S/A-be.many \textsc{3sg.poss}-place earthworm \textsc{top} \textsc{erg?} \textsc{testim}-be.thick   \\
  \glt `Earthworms (that are) in places rich in manure are thicker (than the other ones).'
   (Earthworm, 125)
   \end{exe} 

  First, it could be hypothesized that \ipa{kɯ} here derives from the topicalizer \ipa{kɤ-ti kɯ}  as the in the monoclausal degree construction (section \ref{sec:manner2adj}). This hypothesis is very unlikely however in view of the fact that, unlike in the degree construction, the clitic \ipa{kɯ} in the comparative construction cannot be replaced by \ipa{kɤ-ti kɯ} and there is no evidence that it was ever  possible in any Rgyalrong language.
  
  
Second, an alternative possibility is a derivation from the \ipa{kɯ} in adversative constructions (section \ref{sec:advers}. The derivation is straightforward, but requires three steps.

The first stage is attested in modern Japhug: an adversative construction with   adjectives in both  clauses, with the first adjective   negated and a comparative using the adverb \ipa{sthɯci} `as much' in the first clause as in \ref{ex:mAYaR.kW} and \ref{ex:mWYArtWm.kW}.  This construction is a variant of \citet{stassen11comparative}'s `conjoined comparative'.
        \begin{exe}
  \ex  \label{ex:mAYaR.kW}  
  \gll 
  \ipa{mtshalɤɲaʁ} 	\ipa{nɯ} 	\ipa{ɯ-mdoʁ} 	\ipa{ɲaʁ} 	\ipa{tsa} 	\ipa{mtshalɤɣrum} 	\ipa{nɯnɯ} 	\ipa{tɕe,} 	\ipa{nɯ} \ipa{sthɯci} 	\ipa{mɤ-ɲaʁ} 	\ipa{kɯ}   	\ipa{aqarŋɯrŋe} 	\ipa{kɯ-fse} 	\\
  black.nettle \textsc{top} \textsc{3sg.poss}-colour \textsc{fact}:be.black a.little   white.nettle \textsc{top} \textsc{lnk} \textsc{dem} as.much neg-\textsc{fact}:be.black  \textsc{erg?} \textsc{fact}:be.yellowish \textsc{inf:stat}-be. like \\
  \glt `The colour of the black nettle is black, and the white nettle, it not as   black but rather yellowish.'   (Nettle 21-23)
        \end{exe}
        
           \begin{exe}
  \ex  \label{ex:mWYArtWm.kW}  
  \gll      
\ipa{ɯ-jwaʁ} 	\ipa{nɯ} \ipa{ra} 	\ipa{iʑora} 	\ipa{ji-paχɕi} 	\ipa{sthɯci} 	\ipa{mɯ-ɲɯ-ɤrtɯm} 	\ipa{kɯ} 	\ipa{ɲɯ-rɲɟi} 	\ipa{tsa.} \\
\textsc{3sg.poss}-leaf \textsc{top} \textsc{pl} \textsc{1pl} \textsc{1pl.poss}-apple \textsc{neg-const}-be.round \textsc{erg?} \textsc{testim}-be.long little \\
\glt `Its leaves  are not as round as those of the apple of our (country), but are rather a little long.'
(Apple, 49)
          \end{exe} 
 
%The comparative in \ipa{sthɯci} `as much' can be removed, as in \ref{ex:mWYArtWm.kW2}.  
% 
%            \begin{exe}
%  \ex  \label{ex:mWYArtWm.kW2}  
%  \gll      
%\ipa{ɯ-jwaʁ} 	\ipa{nɯ} \ipa{ra} 	 	\ipa{mɯ-ɲɯ-ɤrtɯm} 	\ipa{kɯ} 	\ipa{ɲɯ-rɲɟi} 	\ipa{tsa.} \\
%\textsc{3sg.poss}-leaf \textsc{top} \textsc{pl}  \textsc{neg-const}-be.round \textsc{erg?} \textsc{testim}-be.long little \\
%\glt Its leaves are not round, but rather a little long.
%(Adapted from \ref{ex:mWYArtWm.kW})
%          \end{exe} 
          
          In the case of a pair of adjectives in polar, or quasi-polar, opposition as `be round' and  `be long' in \ref{ex:mWYArtWm.kW2, the information conveyed by them is redundant, and suppressing one of them does not entail loss of much information.  The surface form of an adversative construction derived from \ref{ex:mWYArtWm.kW}  with elision  of the first adjective would be    \ref{ex:mWYArtWm.kW3}  and its expected meaning would be * `Its leaves are  rather a little long'.
          
          This meaning is not found, and \ref{ex:mWYArtWm.kW3}   is instead a comparative construction whose meaning is `Its leaves are a little longer.' 
            
                      \begin{exe}
  \ex  \label{ex:mWYArtWm.kW3}  
  \gll      
\ipa{ɯ-jwaʁ} 	\ipa{nɯ}   \ipa{ra} 	\ipa{kɯ} 	\ipa{ɲɯ-rɲɟi} 	\ipa{tsa.} \\
\textsc{3sg.poss}-leaf \textsc{top} \textsc{pl}   \textsc{erg?} \textsc{testim}-be.long little \\
\glt *`Its leaves are rather a little long'  $\rightarrow$  `Its leaves are a little longer.'
(Adapted from \ref{ex:mWYArtWm.kW})
          \end{exe} 
  
  
  Yet, there is much semantic overlap   between the expected (adversative) meaning and the attested comparative meaning, and \ref{ex:mWYArtWm.kW3}   represents the ambiguous structure in which the use of \ipa{kɯ} in comparative constructions was introduced by reanalysis from an adversative.
  
  From there, it could be generalized to the complete comparative construction with overt standard NP marked with the comparative marker. The complete pathway from adversative to comparee NP index can be summarized in \ref{ex:adv2comp}.
  
  \begin{exe}
\ex \label{ex:adv2comp}
\glt rather long than non-long $\rightarrow$ rather long $\rightarrow$   longer $\rightarrow$ longer than
\end{exe}
  
%its leaves are not as round as those of the apple, but rather long
%its leaves are not round, but rather long
%its leaves are rather long  
% its leaves are longer
 


  \section{Conclusion}
Despite the extremely polyfunctionality of the    clitic \ipa{kɯ} in Japhug Rgyalrong, a series of historical scenarios can be proposed to account for most of its uses (all expcept the sentence final particle / complementizer). The grammaticalization pathways  proposed in this paper are summarized in Figure \ref{fig:scenario}. In particular, the evolution from ergative / instrumental to index on comparee NP involves no less than four steps: \textsc{instrumental} $\rightarrow$ \textsc{infinitival manner} linker $\rightarrow$  \textsc{finite manner} linker $\rightarrow$  \textsc{adversative} linker $\rightarrow$  \textsc{index} on the comparee NP.

   \begin{figure}
   \caption{Scenarios of development from Ergative / Instrumental to the other functions of \ipa{kɯ}} \label{fig:scenario}  
  \begin{tikzpicture}
  \node (A) at (4,1) {\textbf{Ergative}/Instrumental};
   \node (B) at (-2,-1) {Distributive};
    \node (C) at (4,-1) {Cause};
        \node (D) at (-2,-3)  {Multiclausal degree};
    \node (E) at (4,-3) {Infinitival Manner}; 
       \node (F) at (2.5,-5)  {Finite Manner};
        \node (G) at (6,-5) {Quotative topic}; 
             \node (H) at (0.5,-7)  {Adversative};
             \node (I) at (8,-7) {Monoclausal degree}; 
    \node (J) at (-1.5,-9) {\textbf{Comparee NP}};
    
    
\tikzstyle{peutetre}=[->,dotted,very thick,>=latex]
\tikzstyle{sur}=[->,very thick,>=latex]
\draw[peutetre] (A)--(B);
\draw[sur] (A)--(C);
\draw[sur] (C)--(D);
\draw[peutetre] (C)--(E);
\draw[peutetre] (A) to[bend left] (E);
\draw[sur] (E)--(F);
\draw[sur] (E)--(G);
\draw[sur] (F)--(H);
\draw[sur] (G)--(I);
\draw[sur] (H)--(J);
\draw[peutetre] (C) to[bend right] (H);

\end{tikzpicture}
\end{figure}

Although all Rgyalrong languages appear to share similar ergative / instrumental markers,\footnote{We find \ipa{kə} in Tshobdun (\citealt[129-131]{jackson98morphology}),  \ipa{kə} in Situ (\citealt[336]{linxr93jiarong}) and  \ipa{kə} in Zbu  (\citealt{gongxun12}).} all these markers are recently borrowed from Amdo Tibetan, and it is unlikely that borrowing took place in the common ancestors of all these languages. In the absence of more detailed descriptions of other Rgyalrong languages, it is difficult to ascertain to what extent the ergative marker also present a comparable polyfunctionality in these languages.


  
\bibliographystyle{Linquiry2}
\bibliography{bibliogj}
\end{document}
