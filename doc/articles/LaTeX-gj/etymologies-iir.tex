\documentclass[11pt]{article} 
\usepackage{fontspec}
\usepackage{natbib}
\usepackage{booktabs}
\usepackage{xltxtra} 
\usepackage{polyglossia} 
\setdefaultlanguage{french}
\usepackage{gb4e} 
\usepackage{multicol}
\usepackage{graphicx}
\usepackage{float}
\usepackage{hyperref} 
\hypersetup{bookmarks=false,bookmarksnumbered,bookmarksopenlevel=5,bookmarksdepth=5,xetex,colorlinks=true,linkcolor=blue,citecolor=blue}
\usepackage[all]{hypcap}
\usepackage{memhfixc}
 
%\setmainfont[Mapping=tex-text,Numbers=OldStyle,Ligatures=Common]{Charis SIL} 
\newfontfamily\phon[Mapping=tex-text,Ligatures=Common,Scale=MatchLowercase]{Charis SIL} 
\newcommand{\ipa}[1]{{\phon\textit{#1}}} 
\newcommand{\forme}[1]{{\phon#1}} 
\newcommand{\grise}[1]{\cellcolor{lightgray}\textbf{#1}}
\newfontfamily\cn[Mapping=tex-text,Ligatures=Common,Scale=MatchUppercase]{SimSun}%pour le chinois
\newcommand{\zh}[1]{{\cn #1}}
\newcommand{\Y}{\Checkmark} 
\newcommand{\N}{} 
\newcommand{\dhatu}[2]{|\ipa{#1}| `#2'}
\newcommand{\jpg}[2]{\ipa{#1} `#2'}  
\newcommand{\refb}[1]{(\ref{#1})}
\newcommand{\tld}{\textasciitilde{}}
\newfontfamily\mleccha[Mapping=tex-text,Ligatures=Common,Scale=MatchLowercase]{Galatia SIL}%pour le grec
\newcommand{\grec}[1]{{\mleccha #1}}


 \begin{document} 
\title{Quelques étymologies indo-iraniennes}
\author{Guillaume Jacques\\ CNRS-CRLAO-INALCO}
\maketitle

\sloppy
\section{Skt \ipa{áka-}, Av.  \ipa{aka-}}
Le sanskrit \ipa{áka-} n.  `Leid, Schmerz' et son cognat avestique, l'adjective \ipa{aka-} `schlecht, übel, böse', sont isolés en indo-européen et aucune étymologie satisfaisante n'a jusqu'ici été proposée  (\citealt[I:39]{mayrhofer92ewa}). Ils connaissent un seul dérivé, le nom \ipa{*akti-} attesté par l'av.  \ipa{axti-} m. `Leiden' et le sanskrit \ipa{śīrṣ-aktí-} `Kopfschmerz'.

\citet{lubotsky2001indo} considère ces mots comme faisant partie d'une couche d'emprunts à une langue non-indo-européenne peut-être liée au BMAC.

Le ressemblance superficielle avec le mot sino-tibétain représenté par le chinois \zh{惡} \ipa{'ak} $\leftarrow$ \ipa{*ʔˁak} `bad' (\citealt{bs14oc}) et la racine verbal kirantie \ipa{*ʔak} `feel pain, be displeased' (\citealt{jacques17pkiranti}), qui aurait pu avoir suggéré un contact entre les deux familles, motive une investigation de l'étymologie de ces mots.

L'adjectif proto-indo-iranien  \ipa{*aká-} qu'impliquent les formes sanskrites et avestique (avec recul d'accent \ipa{áka-} en sanskrit dû à la substantivation) peut néanmoins s'expliquer par un adjectif départiculatif \ipa{*ṇ-kó-} de la négation \ipa{*ne}, voulant dire à l'origine `vaurien', d'où `mauvais'. Notons que le degré zéro de la négation est attendu en composition.

On trouve de nombreux parallèles sémantiques à une telle dérivation, aussi bien en indo-européen (le latin \ipa{nēquam} `vaurien') qu'ailleurs (japhug \ipa{kɯmaʁ} `ennui, mauvaise chose', d'où le dénominal \ipa{nɯkɯmaʁ} `commettre une erreur', venant du participe sujet de \ipa{maʁ} `ne pas être', \citealt{jacques16japhug}).  

Après le passage du \ipa{*ṇ} voyelle à \ipa{*a}, l'adjectif a été ré-interprété comme un dérivé thématique d'une racine \ipa{*ak}, et il a été possible d'en tirer un dérivé en \ipa{-ti}, \ipa{*akti-}, peut surprenant en vue de la productivité considérable de ce suffixe.

\section{Skt \ipa{gandhá-}, Av.  \ipa{gaiṇti-}}
Le gothique \ipa{gund} n. `gangrène', un hapax (\ref{ex:gund}) apparenté au norwégien dialectal \ipa{gund} `skin crust', au vieil anglais \ipa{gund} m. `Eiter, eiterndes Geschwür', remonte du proto-germanique *\ipa{gunda-}, forme qui n'a pas jusqu'ici reçu d'étymologie satisfaisante (\citealt[195]{kroonen13dict}, \citealt[163, G116]{lehmann86gothic}).


\begin{exe}
\ex \label{ex:gund}
\glt \ipa{jah waurd ize swe gund wuliþ}
\glt \grec{καὶ ὁ λόγος αὐτῶν ὡς γάγγραινα νομὴν ἕξει}
\glt `Et leur parole se répandra comme la gangrène.' (Timothée II 2:17)
\end{exe}


Le vieil haut allemand \ipa{gund} m. `Eiter, eiterndes Geschwür, Vereiterung, Fäulnis' provient d'un \ipa{*gunþa-} (\citealt[128]{schaffner01vernersche}) qui montre que le \ipa{*d} du gothique et du vieil anglais est issu d'un voisement Verner et non d'un \ipa{*dʰ}, et invalide donc l'étymologie de \citet{holthausen1887etymologien} qui rapproche ce mot du grec \grec{κανθύλη} `Geschwür' et celle de \citet[268]{trautmann05etym} qui suggère une relation avec l'avestique \ipa{gunda-} `lump of
dough'.\footnote{A propos de ce mot et de la famille des mots qui y sont apparentés en iranien, voir \citet{rossi15kund}.}

La forme germanique est toutefois candidate à une comparaison avec des formes indo-iraniennes qui sont elles-mêmes restées sans explication: le sanskrit \ipa{gandha-} m. `odeur' et les formes nominales et verbales sous la racine \ipa{*gant} en iranien (\citealt[103-104]{cheung07dictionary}), en particulier le vieux perse \ipa{gasta-} `répugnant, désagréable' et l'avestique \ipa{gaiṇti-} f. `puanteur', un terme daévique comme l'illustre l'exemple (\ref{ex:gantish}), (qui s'oppose à \ipa{baoiδi-} f. `odeur' $\leftarrow$ \ipa{*bʰówdʰ-i-} voir \citealt[493]{bartholomae1904altiranisches}).


\begin{exe}
\ex \label{ex:gantish}
\glt 
\ipa{aēšō} \ipa{zī} \ipa{asti} \ipa{daēuuanąm} \ipa{rapakō} \ipa{vīspəm} \ipa{ā} \ipa{ahmāt̰} \ipa{yat̰} \ipa{aēša} \ipa{gaiṇtiš} \ipa{upaŋhacaiti}
\glt `Diese [Leichenstätte] bildet ja eine Stütze der Daeva's, solang der Gestank (davon) noch wahrnehmbar ist.' (Vidēvdād 7.56, \citealt[1741]{bartholomae1904altiranisches}, \citealt[363]{wolff1910avesta})\footnote{Noter la locution \ipa{vīspəm} \ipa{ā} \ipa{ahmāt̰} \ipa{yat̰} `bis zu dem Zeitpunkt ...', `solange als' (\citealt[1466]{bartholomae1904altiranisches}). Le nom \ipa{rapaka-} `eine Stütze bildend für..., im Dienst stehend von...' dérive de la racine \ipa{rap-} ‘to help, assist’ (\citealt[314]{cheung07dictionary}) qui ne connaît pas de cognats extra-iraniens. Cette racine isolée pourrait être secondaire, formée de la même façon que le sanskrit \ipa{rapś-} `strotzen' (\citealt[II:559]{mayrhofer92ewa}) %587 
qui provient du nom \ipa{virapśá-} m. `Ueberfluss, Fülle' (d'un composé \ipa{*vīra-pśv-a-} `Männer und Vieh besitzend') par réanalyse du \ipa{vi-} comme un préverbe. Pour l'iranien un composé thématisé \ipa{*vīrapa-} `protecteur des hommes' ayant \ipa{*vīra°} comme premier élément et \ipa{*paH-} `to protect, guard' comme second membre (\citealt[288]{cheung07dictionary}; suggestion de Romain Garnier), parallèle au sanskrit classique \ipa{nṛpa-} `roi' aurait pu être réinterprété en \ipa{*vi-rap-a-} (la différence de longueur peut s'expliquer soit parce que le préverbe \ipa{*vi-} ayant des variantes à voyelles longues devant les racines à laryngales initiales, soit par une variante métrique à voyelle courte de \ipa{vīra-}, comme la forme \ipa{virāṣā́ṭ} `qui soumet les hommes', RV I.35.6), d'où la racine \ipa{*rap-} aurait pu être extraite. Cette hypothèse serait renforcée si une forme verbale ou nominale de cette racine pourvue du préfixe \ipa{*vi-} pouvait être documentée dans les langues iraniennes anciennes ou modernes.
}
\end{exe}

Dans les langues moyennes-iraniennes et modernes, on trouve des formes verbales liées à cette racine, en particulier l'ossète \ipa{iǧændun} `\forme{пачкать, марать, осквернять}' (\citealt[542]{abaev58vol1}), le sogdien \ipa{’’ɣ’ynt-} ‘to defile’ et le pashto \ipa{ɣandəl} ‘to dislike'.


Si le vieux perse \ipa{gasta-} et les formes ossète, sogdienne et pashto citées ci-dessus ont développé un sens abstrait (voir \ref{ex:gasta}),\footnote{Pour de nombreux parallèles d'une telle évolution sémantique, dont le latin \ipa{ōdī} `haïr', voir \citet[55-59]{garnier10vocalisme}
} les formes des autres langues indo-iraniennes désignent une perception olfactive. 

\begin{exe}
\ex \label{ex:gasta}
\glt \ipa{m-r-t-i-y-a: h-y-a: a-u-r-m-z-d-a-h-a: f-r-m-a-n-a: h-u-v-t-i-y: g-s-t-a: m-a: θ-d-y:}
\glt \ipa{martiyā} \ipa{hayā} \ipa{Auramazdāhā} \ipa{framānā} \ipa{hawtay} \ipa{gastā} \ipa{mā} \ipa{θadaya}
\glt `Mann! Das Gebot Ahuramazdās, das erscheine dir nicht übel.' (DNa.56-58 \citealt[137]{kent53op}, \citealt[104]{schmitt09altpersichen})

\end{exe}

Une telle comparaison soulève trois questions qu'il est nécessaire de discuter: 

\begin{enumerate}
\item Le problème de la structure d'une  racine ayant la forme \ipa{*g(ʷ)ʰent-} (la co-occurrence d'une voisée aspirée et d'une sourde)
\item L'incompatibilité phonétique entre le sanskrit \ipa{gandha-} (qui implique une proto-forme \ipa{*g(ʷʰ)ondʰo-}) et les formes de l'iranien (qui requièrent une racine de la forme \ipa{*g(ʷʰ)ent-}), ainsi que la différence de sens en sanskrit et iranien.
\item L'évolution sémantique et phonétique précise ayant donné la forme germanique \ipa{*gunda-}.
\end{enumerate}

Certaines formes iraniennes dont l'avestique  \ipa{gaiṇti-} impliquent une racine verbale \ipa{*gant} provenant virtuellement d'un \ipa{*g(ʰʷ)ent-} `avoir une mauvaise odeur'.\footnote{Notons que \citet[542]{abaev58vol1} pose un \ipa{*vi-gand-} pour rendre compte de \ipa{iǧændun} `\forme{осквернять}'; toutefois, en ossète les \ipa{*t} de l'iranien deviennent \ipa{d} après sonante (\citealt[97]{huebschmann1877ossetischen}, comme dans \ipa{fændag} < \ipa{*pantâka-} ``chemin", cf avestique \ipa{paṇtā̊, paθō}). Une proto-forme \ipa{*vi-gant-} est donc tout aussi bien appropriée pour rendre compte du verbe ossète. } Néanmoins, pour pouvoir relier les formes iraniennes au germanique \ipa{*gunda-} et même au sanskrit \ipa{gandha-}, comme nous le verrons plus bas, il est nécessaire de supposer une voisée aspirée en proto-indo-iranien. Or, une racine verbale \ipa{*g(ʷ)ʰent-} contreviendrait à la règle de co-occurrence des voisées aspirées et des sourdes (\citealt{meillet1912zrazda}; voir aussi  \citealt{vaan99terdh}) et ne peut pas être indo-européenne -- poser une telle racine serait anachronique. Pour rendre compte des formes attestées, il est donc nécessaire de la considérer comme secondaire. Dans ce travail, nous supposons l'existence d'un nom \ipa{*g(ʷ)ʰont- \textasciitilde *g(ʷ)ʰṇt-} `puanteur' non directement attesté, mais qui permet de rendre compte de toutes les formes indo-iraniennes (y compris verbales, comme nous l'expliquons plus bas) et de leur relation avec le germanique.

Une étymologie possible pour ce nom serait un dérivé  archaïque en \ipa{*-t-} de la racine \ipa{*gʷʰen} `schlagen' (\citealt[218]{liv}, type \ipa{*per} $\rightarrow$ \ipa{pṛt-} `combat'), morphologiquement du même type que le nom de la `nuit'. Selon \citet{schindler67nekuz}, cette classe de noms avait originellement une apophonie acrostatique, nominatif \ipa{*gʷʰónt-s}, génitif \ipa{*gʷʰént-s}. Toutefois, les formes attestées (degré zéro et degré \ipa{*o}) suggèrent qu'il a très tôt acquis une flexion similaire à celle du nom `dent' (nom. \ipa{*h_1dónt-s}, acc. \ipa{*h_1dónt-ṃ}, gen.\ipa{*h_1dṇt-és}), avec degré zéro aux formes faibles, notamment au locatif pluriel d'où `blessure purulente', puis `puanteur' en proto-indo-iranien, un changement sémantique parallèle à celui proposé pour la racine \ipa{*puh_1} `puer' par \citet{garnier16secondary}.
 
L'indo-iranien \ipa{*gʰánti-} f. `puanteur', ancêtre direct de l'avestique \ipa{gaiṇti-} f., peut s'expliquer comme la régularisation à la flexion des thèmes en \ipa{-i} d'un ancien collectif hystérocinétique \ipa{*gʷʰón-t-(ō|ē)y- \textasciitilde *gʷʰṇ-t-y-és} `puanteurs, endroit puant' de  \ipa{*gʷʰón-t-} `puanteur' (voir \citealt{oettinger95kollektiv} et \citealt{ garnier13ghosti} pour d'autres exemples de cette dérivation).

Le participe passé au sens évolué \ipa{gasta-} `répugnant' du vieux perse et le verbe ossète \ipa{iǧændun} impliquent que la racine iranienne \ipa{*gant-} devait avoir des formes verbales, même si celles-ci ne sont pas attestées en avestique et en vieux perse. Comme la racine \ipa{*gant-}, selon la discussion ci-dessus, est secondaire, des formes verbales finies n'ont pu en être tirées que par formation inverse. 

Or, comme il est admis que le passif aoriste de l'indo-iranien tient son origine précisément dans les thèmes en \ipa{-i} au degré \ipa{*o} (\citealt[15]{kummel96stativ}), un passif aoriste \ipa{*agʰánti} `il/ça sens mauvais', d'où `c'est répugnant', aurait pu servir de forme pivot, de laquelle on a ensuite pu tirer d'autres formes verbales, dont le participe passé et un présent de classe X \ipa{*gantaya-} au sens  tropatif\footnote{Pour une étude des dérivations tropatives et leur relation avec le causatif dans une perspective typologique, voir \citet{jacques13tropative}.} `trouver répugnant' ou causatif `souiller' qui est à la base de l'ossète \ipa{iǧændun} `\forme{осквернять}' et du sogdien \ipa{’’ɣ’ynt-} ‘to defile'(avec les préverbes \ipa{*vi-} et \ipa{*ā-} respectivement).


Pour expliquer la forme sanskrite, il n'est pas nécessaire de supposer l'existence d'une racine verbale secondaire comme en iranien. Dans la synchronie du sanskrit, le nom-racine venant de \ipa{*g(ʷ)ʰont- \textasciitilde *g(ʷ)ʰnt-} `puanteur' aurait présenté dans son paradigme certaines formes telles que le locatif pluriel \ipa{*ghatsú} $\leftarrow$ \ipa{*g(ʷ)ʰṇt-sú} `dans la puanteur' (type \ipa{pṛt-sú}) où l'opposition entre les quatre modes d'articulation des occlusives est neutralisé, et qui se prêtaient donc à une ré-interprétation comme provenant d'une racine plus normale du point de vue de la phonotactique du sanskrit \ipa{*gandh} par hyper-Grassmannisation (par analogie avec des nom-racines tels que \ipa{°búdh-} en tant que partie finale de composé nominal, qui donne au nominatif \ipa{°bhut} et au locatif pluriel \ipa{°bhutsú}).\footnote{Comme la racine \ipa{*g(ʷ)ʰent} n'est attestée que par des formations nominales en sanskrit, il serait controuvé de rechercher la forme pivot dans la conjugaison d'une racine verbale (un futur $\dagger$\ipa{ghantsya-ti} ou un désidératif $\dagger$\ipa{jighantsa-ti}). }

D'une telle forme-pivot, on a ensuite refait un ancien $\dagger$\ipa{ghánti-} `puanteur' en \ipa{*gándhi-}, forme qui est attestée en second membre de composé (\ipa{su-gándhi-} `ayant une bonne odeur').

Notons qu'en iranien, du fait de la confusion des voisées aspirées et des voisées simples, une telle réinterprétation n'aurait pas pu avoir lieu.

L'avestique \ipa{duž-gaiṇti-} `qui a une mauvaise odeur' suggère de lui-même une explication à la différence sémantique entre sanskrit et iranien: le préfixe \ipa{duž-}, originellement utilisé pour intensifier le sens (`forte puanteur'), a dans le cadre du sanskrit été réinterprété comme exprimant un sens péjoratif (\ipa{durgandhi-} `ayant une mauvaise odeur'), et donc pourvu d'une doublet en \ipa{su-} (\ipa{sugandhi-} `ayant une bonne odeur').\footnote{A titre d'illustration de l'extrême productivité de cette formation, notons que même le nom de personne Duryodhana a  une dénomination alternative euphémistique \textit{Suyodhana} dans le Mahâbhârata.} Dans la synchronie du sanskrit, la pseudo-racine \ipa{gandh-} a donc été associée au sens neutre d'odeur, et un dérivé thématique \ipa{gándha-} `odeur' en a été tiré.\footnote{Notons malgré l'absence de relation étymologique entre ce nom \ipa{gandha-} `odeur' et celui des créatures mythologiques \ipa{gandharva-} (et de son correspondant avestique phonétiquement  irrégulier \ipa{gaṇdərəβa-}, voir \citealt{lubotsky2001indo}), l'équivalent tibétain de ce nom \ipa{dri.za} `mangeur d'odeur' suggère que les traducteurs de ce terme ont peut-être imaginé que \ipa{gandharva-} était une forme tronquée tirée d'un \ipa{*gandha-bharva-}.}


En germanique, \ipa{*gʷʰon-t-} `ulcère purulent' aurait été thématique en \ipa{*gʷʰṇt-ó-}, donnant \ipa{*gunda-} par voisement Verner (\citealt[128]{schaffner01vernersche}) et loi de Seebold. Du point de vue sémantique le germanique serait donc plus archaïque que l'indo-iranien.



\bibliographystyle{unified}
\bibliography{bibliogj}

 \end{document}
 