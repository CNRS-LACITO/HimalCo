\documentclass[oldfontcommands,oneside,a4paper,11pt]{article} 
\usepackage{fontspec}
\usepackage{natbib}
\usepackage{booktabs}
\usepackage{xltxtra} 
\usepackage{polyglossia} 
\usepackage[table]{xcolor}
\usepackage{gb4e} 
\usepackage{multicol}
\usepackage{graphicx}
\usepackage{float}
\usepackage{hyperref} 
\hypersetup{bookmarks=false,bookmarksnumbered,bookmarksopenlevel=5,bookmarksdepth=5,xetex,colorlinks=true,linkcolor=blue,citecolor=blue}%\usepackage[all]{hypcap}
\usepackage{memhfixc}
\usepackage{lscape}
\bibpunct[: ]{(}{)}{,}{a}{}{,}

%\setmainfont[Mapping=tex-text,Numbers=OldStyle,Ligatures=Common]{Times New Roman} 
\newfontfamily\phon[Mapping=tex-text,Ligatures=Common,Scale=MatchLowercase,FakeSlant=0.3]{Charis SIL} 
\newcommand{\ipa}[1]{{\phon \mbox{#1}}} %API tjs en italique
\newcommand{\ipab}[1]{{\scriptsize \phon#1}} 

\newcommand{\grise}[1]{\cellcolor{lightgray}\textbf{#1}}
\newfontfamily\cn[Mapping=tex-text,Ligatures=Common,Scale=MatchUppercase]{MingLiU}%pour le chinois
\newcommand{\zh}[1]{{\cn #1}}

\newcommand{\sg}{\textsc{sg}}
\newcommand{\pl}{\textsc{pl}}
\newcommand{\ro}{$\Sigma$}
\newcommand{\ra}{$\Sigma_1$} 
\newcommand{\rc}{$\Sigma_3$}  


\newcommand{\jg}[1]{\ipa{#1}\index{Japhug #1}}
\newcommand{\wav}[1]{}%#1.wav}

\newcommand{\acc}{\textsc{acc}}
 \newcommand{\acaus}{\textsc{acaus}}
 \newcommand{\advers}{\textsc{advers}}
\newcommand{\apass}{\textsc{apass}}
\newcommand{\appl}{\textsc{appl}}
\newcommand{\allat}{\textsc{all}}
\newcommand{\pfv}{\textsc{pfv}}
\newcommand{\assert}{\textsc{assert}}
\newcommand{\auto}{\textsc{autoben}}
\newcommand{\caus}{\textsc{caus}}
\newcommand{\cl}{\textsc{cl}}
\newcommand{\cisl}{\textsc{cisl}}
\newcommand{\classif}{\textsc{class}}
\newcommand{\concsv}{\textsc{concsv}}
\newcommand{\comit}{\textsc{comit}}
\newcommand{\compl}{\textsc{compl}} %complementizer
\newcommand{\comptv}{\textsc{comptv}} %comparative
\newcommand{\cond}{\textsc{cond}}
\newcommand{\conj}{\textsc{conj}}
\newcommand{\lnk}{\textsc{lnk}}
\newcommand{\conv}{\textsc{conv}}
\newcommand{\cop}{\textsc{cop}}
\newcommand{\dat}{\textsc{dat}}
\newcommand{\dem}{\textsc{dem}}
\newcommand{\degr}{\textsc{degr}}
\newcommand{\deexp}{\textsc{deexp}}
\newcommand{\dist}{\textsc{dist}}
\newcommand{\du}{\textsc{du}}
\newcommand{\duposs}{\textsc{du.poss}}
\newcommand{\dur}{\textsc{dur}}
\newcommand{\erg}{\textsc{erg}}
\newcommand{\emphat}{\textsc{emph}}
\newcommand{\evd}{\textsc{ifr}}
\newcommand{\fut}{\textsc{fut}}
\newcommand{\gen}{\textsc{gen}}
\newcommand{\genr}{\textsc{genr}}
\newcommand{\hort}{\textsc{hort}}
\newcommand{\hypot}{\textsc{hyp}}
\newcommand{\ideo}{\textsc{ideo}}
\newcommand{\imp}{\textsc{imp}}
\newcommand{\indef}{\textsc{indef}}
\newcommand{\inftv}{\textsc{inf}}
\newcommand{\instr}{\textsc{instr}}
\newcommand{\intens}{\textsc{intens}}
\newcommand{\intrg}{\textsc{intrg}}
\newcommand{\inv}{\textsc{inv}}
\newcommand{\ipf}{\textsc{ipf}}
\newcommand{\irr}{\textsc{irr}}
\newcommand{\loc}{\textsc{loc}}
\newcommand{\med}{\textsc{med}}
\newcommand{\mir}{\textsc{mir}}
\newcommand{\negat}{\textsc{neg}}
\newcommand{\neu}{\textsc{neu}}
\newcommand{\nmlz}{\textsc{nmlz}}
\newcommand{\fact}{\textsc{fact}}
\newcommand{\plposs}{\textsc{pl.poss}}
\newcommand{\pass}{\textsc{pass}}
\newcommand{\poss}{\textsc{poss}}
\newcommand{\pot}{\textsc{pot}}
\newcommand{\pres}{\textsc{pres}}
\newcommand{\prohib}{\textsc{prohib}}
\newcommand{\prox}{\textsc{prox}}
\newcommand{\pst}{\textsc{pst}}
\newcommand{\qu}{\textsc{qu}}
\newcommand{\recip}{\textsc{recip}}
\newcommand{\redp}{\textsc{redp}}
\newcommand{\refl}{\textsc{refl}}
\newcommand{\sens}{\textsc{sens}}
\newcommand{\sgposs}{\textsc{sg.poss}}
\newcommand{\stat}{\textsc{stat}}
\newcommand{\topic}{\textsc{top}}
\newcommand{\volit}{\textsc{vol}}
\newcommand{\transloc}{\textsc{transl}}
\newcommand{\cisloc}{\textsc{cisl}}
\newcommand{\quind}{\textsc{qu.ind}} %revoir glose
\newcommand{\trop}{\textsc{trop}} 
 \newcommand{\abil}{\textsc{abil}}  
 \newcommand{\facil}{\textsc{facil}}  
 

\XeTeXlinebreakskip = 0pt plus 1pt 
 


\begin{document} 
\title{The origin of the causative prefix in Rgyalrong languages and its implication for proto-Sino-Tibetan reconstruction\footnote{
The glosses follow the Leipzig glossing rules. Other abbreviations used here are: \textsc{auto}  autobenefactive-spontaneous, \textsc{anticaus} anticausative, \textsc{antipass} antipassive, \textsc{appl} applicative, \textsc{dem} demonstrative,  \textsc{dist} distal, \textsc{emph} emphatic, \textsc{fact} factual, \textsc{genr} generic, \textsc{ifr} inferential, \textsc{indef} indefinite, \textsc{inv} inverse,  \textsc{lnk} linker, \textsc{pfv} perfective, \textsc{poss} possessor,  \textsc{sens} sensory. I would like to thank Nathan W. Hill, Laurent Sagart and three anonymous reviewers for valuable comments and suggestions on a previous version of this article. The examples are taken from a corpus that is progressively being made available on the Pangloss archive (\citealt{michailovsky14pangloss}). This research was funded by the HimalCo project (ANR-12-CORP-0006) and is related to the research strand LR-4.11 ‘‘Automatic Paradigm Generation and Language Description’’ of the Labex EFL (funded by the ANR/CGI). } }
\author{Guillaume Jacques}
\maketitle
\sloppy

\textbf{Abstract}: This paper presents the first detailed description of the two causative derivations in Japhug Rgyalrong based on a corpus of spontaneous speech, and proposes two new pathways of grammaticalization: causative from denominal derivation and abilitative from causative. 

Then , it evaluates the implication of these grammaticalization hypotheses for the reconstruction of Sino-Tibetan morphology as a whole.

\textbf{Keywords}: Causative, Denominal verbs, Japhug, Rgyalrongic, Grammaticalization, Abilitative, Sino-Tibetan

\section{Introduction}

Ever since \citet{conrady1896}'s seminal work on comparative Sino-Tibetan, the hypothesis that a causative *s-- prefix can be reconstructed for the  common ancestor of Chinese, Tibetan and all related languages has been widely accepted and can be said to be the only fact about proto-Sino-Tibetan morphosyntax that is completely uncontroversial (\citealt{wolfenden29outlines, benedict72, matisoff03}, \citealt{lapolla03}).  Yet, despite this consensus, the history of the causative prefix in the ST family is by no means trivial. 

In many Sino-Tibetan languages, including Old Chinese, the  language with oldest attestations, the causative prefix can only be indirectly detected through internal reconstruction. In  phonologically more conservative languages such as Old Tibetan, the causative prefix is directly attested, but restricted to a limited number of examples and completely lexicalized. Rgyalrong is one of the few language groups in Sino-Tibetan where the causative is still productive (ie can be applied to recent loanwords).\footnote{There may be other languages in which this prefix is productive, such as Jinghpo or Dulong (see \citealt[71-8]{dai90yufa} and \citealt[101-2]{sunhk82dulong}), but additional data is necessary, especially concerning the treatment of loanwords.} Thus, a detailed description of the morphosyntax of the causative in Rgyalrong languages is necessary before any further comparative research.

In this paper, we first provide an overview of the morphology and morphosyntax of the causative prefixes \ipa{sɯ--} and \ipa{ɣɤ--} in Japhug Rgyalrong, one of the Rgyalrong languages, and discuss the historical relationship of the causative prefix with homophonous prefixes, in particular the abilitative and the denominal instrumental \ipa{sɯ--}. Then, we show on the basis of comparative evidence that the likeliest explanation for the homophony between the causative and the denominal prefixes in Rgyalrong languages is due to the fact that the former was grammaticalized from the latter.
Finally, we present comparative evidence from Chinese, Tibetan and Kiranti showing that the lexicalized remnants of the causative and denominal prefixes in these languages are compatible with the grammaticalization hypothesis presented in section 3.


\section{The causative in Rgyalrongic languages}


Rgyalrongic languages are particularly crucial for understanding the morphosyntax of the causative in Sino-Tibetan as a whole, as they constitute the only subgroup of the family where the causative is still fully productive and can be applied to recent loanwords, at least in the four core Rgyalrong languages Situ, Japhug, Tshobdun and Zbu. It is thus of considerable importance, before any diachronic research, to describe the synchronic system of these conservative languages in detail, and then evaluate how much of this system has been preserved elsewhere. Since previous description of Rgyalrong languages only briefly discuss the morphosyntax of the causative constructions (see \citealt{jackson06paisheng}, \citealt{jacques08zh}), it is necessary at this stage to provide the first in depth description of the use of the causative prefixes in a Rgyalrong language.

In this paper, we focus on the Japhug language, but similar causative prefixes are found in Tshobdun (\citealt{jackson06paisheng, jackson14morpho}) and in Khroskyabs (\citealt{lai13affixale,lai14caus}), and the facts described for Japhug are  valid for Rgyalrongic as a whole.

This section comprises four subsections. First, we describe the morphology and uses of the causative prefixes \ipa{sɯ--} and \ipa{ɣɤ--} respectively, as well as the semantic distinctions between the two. Then, we treat the homophonous abilitative and denominal \ipa{sɯ--} and discuss their potential historical relationships with the causative.


\subsection{The causative prefix  \ipa{sɯ--}}

Unlike most Sino-Tibetan languages, Rgyalrong languages have a complex templatic prefixal morphology (\citealt{jacques13harmonization, jackson14morpho}) and numerous valency increasing and decreasing prefixes, including reflexive (\citealt{jacques10refl}), passive, anticausative (\citealt{jacques12demotion}), applicative (\citealt{jacques13tropative}), antipassive (\citealt{jacques14antipassive}) as well as nominal incorporation (\citealt{jacques12incorp}).

Of all the derivational verbal morphological processes in Japhug, the causative \ipa{sɯ-/z-/sɯɣ-} is the most commonly used, the most productive and the morphophonologically most complex affixal element, a fact that betrays its antiquity in comparison with other affixes, most of which can be shown to be recent developments (see in particular \citealt{jacques14antipassive}).

\subsubsection{Morphophonology} \label{subsub:caus:morphophon}
Although not as complex as the causative in Stodsde (\citealt{jackson07shangzhai}) or in Khroskyabs (\citealt{lai14caus}), the Japhug causative   \ipa{sɯ--} prefix presents  considerable allomorphy, and numerous irregular forms. It has four regular allomorphs \ipa{sɯ-}, \ipa{sɯɣ-}, \ipa{z-} and \ipa{sɤ--} depending on the following element.

The \ipa{z-} allomorph appears before all derivational prefixes (or unanalysable prefixal elements synchronically belonging to the verb root) with sonorant initial (beginning in \ipa{r-}, \ipa{n-}, \ipa{ɣ-} or \ipa{m-}). Table \ref{tab:causative.z} illustrates some examples of this allomorph.

\begin{table}[h]
\caption{Examples of the \ipa{z}- allomorph of the causative prefix}\label{tab:causative.z} \centering
\begin{tabular}{lllllllll} \toprule
nature of the prefix & base  & &derived  \\
\midrule
non-analysable &  \ipa{mɯnmu} &move& \ipa{z-mɯnmu} &cause to move\\
  &  \ipa{nɯna} &rest& \ipa{z-nɯna} &stop\\
denominal &  \ipa{nɤma} &work& \ipa{z-nɤma} &cause to work\\
antipassive \ipa{rɤ-} &  \ipa{rɤrɤt} &write& \ipa{z-rɤrɤt} &cause to write\\

\bottomrule
\end{tabular}
\end{table}

%causative \ipa{ɣɤ-} &  \ipa{ɣɤme} &destroy, kill& \ipa{z-ɣɤme} &cause s.o. to kill\\

The distribution of the \ipa{sɯ-} and \ipa{sɯɣ-} allomorphs depends on both phonology and morphology. The   latter allomorph occurs when the base verb is intransitive, has no  prefixal element, has no initial cluster and no velar or uvular initial consonant, and the former appears in all other cases.

\begin{table}[h]
\caption{The \ipa{sɯ}- and  \ipa{sɯɣ-} allomorphs of the causative prefix}\label{tab:causative.sW} \centering
\begin{tabular}{lllllllll} \toprule
 transitivity & base & & derived & \\
 \midrule
 intr. & \ipa{ɕe} & go & \ipa{sɯɣ-ɕe} & send \\
  tr. & \ipa{ɕɯm} & brood & \ipa{sɯ-ɕɯm} & cause to brood \\
  intr. & \ipa{ndzur} & stand & \ipa{sɯɣ-ndzur} & cause to stand up \\
  tr. & \ipa{ndza} & eat & \ipa{sɯ-ndza} & cause to eat \\ 
    intr. & \ipa{tso} & understand & \ipa{sɯɣ-tso} & cause to understand \\
  tr. & \ipa{tsɯm} & take away & \ipa{sɯ-tsɯm} & cause to take away, send \\ 
 \bottomrule
\end{tabular}
\end{table}
 
A fourth predictable allomorph of \ipa{sɯ-} appears with   verbs whose stem begins in \ipa{a--}, where \ipa{sɯ-}  and the \ipa{a-} passive prefix or intransitive thematic element merge as \ipa{sɤ-}, as in Table \ref{tab:causative.sA}.

\begin{table}[h]
\caption{The \ipa{sɤ}-   allomorph  of the causative prefix}\label{tab:causative.sA} \centering
\begin{tabular}{lllllllll} \toprule
  base & & derived & \\
 \midrule
 \ipa{ambrɤqɤt} & be different & \ipa{sɤmbrɤqɤt} & distinguish \\
\ipa{andɯja} & gather & \ipa{sɤndɯja} & cause to gather \\
 \ipa{amɲɤm} & be homogeneous & \ipa{sɤmɲɤm} & homogenize \\
 \bottomrule
\end{tabular}
\end{table}

The causative has four additional irregular allomorphs: \ipa{ɕɯ-}, \ipa{ɕɯɣ-}, \ipa{ɕ-}, \ipa{ʑ-}, \ipa{s-} and \ipa{j-}. All known examples are presented in Table \ref{tab:causative.irregular}.

\begin{table}[h]
\caption{The irregular allomorphs of the causative prefix}\label{tab:causative.irregular} \centering
\resizebox{\columnwidth}{!}{
\begin{tabular}{lllllllll} \toprule
  base & & derived & \\
 \midrule
\jg{fka}  &  be full (after eating) &  \jg{ɕɯ-fka}  &  cause to be full  \\ 
\jg{fkaβ}  &  cover &  \jg{ɕɯ-fkaβ}  &  cover with  something \\ 
\jg{mbɣom}  &  be in a hurry &  \jg{ɕɯ-mbɣom}  & cause to be in a hurry \\ 
\jg{mnɤm}  &  have a smell &  \jg{ɕɯ-mnɤm}  & cause to have a smell \\ 
\jg{mŋɤm}  &   hurt  (of a body part) &  \jg{ɕɯ-mŋɤm}  & hurt (somebody)   \\ 
\jg{ntaβ}  & be stable  &  \jg{ɕɯ-ntaβ}  & put      \\ 
\jg{ngo}  & sick   &  \jg{ɕɯ-ngo}  & cause to become sick  \\ 
\jg{nŋo}  & lose   &  \jg{ɕɯ-nŋo}  & win   \\ 
\jg{ɴqoʁ}  & be hung  &  \jg{ɕɯ-ɴqoʁ}  & hang   \\ 
\jg{rŋo}  & borrow  &  \jg{ɕɯ-rŋo}  & lend     \\ 
\jg{(tɯ-mbrɯ) ŋgɯ}  & be / become angry  &  \jg{(tɯ-mbrɯ) ɕɯ-ŋgɯ}  & anger someone     \\ 
\jg{rŋgɯ}  & lie down  &  \jg{ɕɯ-rŋgɯ}  & cause to lie down, \\
&&&ferment (alcohol)   \\ 
\jg{rga}  & be glad  &  \jg{ɕɯ-rga}  &  please somebody  \\ 
\midrule
\jg{mu}  & be afraid  &  \jg{ɕɯɣ-mu}  & frighten   \\ 
\midrule
\jg{pʰɣo}  &  flee  &  \jg{ɕpʰɣo}  &  take away     \\ 
\jg{lɯɣ}  &  get loose, get free  &  \jg{ɕlɯɣ}  &  drop   \\ 
\midrule
\jg{ɴqoʁ}  & be hung&  \jg{ʑɴɢoʁ}  & hang on a hook  \\ 
\jg{ŋga}  & wear  &  \jg{ʑŋga}  & help someone to wear  \\ 
\jg{mbri}  & cry    &  \jg{ʑmbri}  & play (an instrument) \\ 
\midrule
\jg{tsʰi}  & drink   &  \jg{jtsʰi}  & give to drink  \\ 
\midrule
\jg{qanɯ}  & dark   &  \jg{sqanɯ}  & put in darkness  \\ 
 \bottomrule
\end{tabular}}
\end{table}
Some of the verbs in Table \ref{tab:causative.irregular}, such as \jg{tsʰi} ``to drink'', \jg{pʰɣo} ``to flee'',  \jg{lɯɣ} ``to get free'' and \jg{rga}  ``to be glad, to like'', can appear with the regular causative \jg{sɯ-}. In the first cases the meaning of the derived verbs are slightly different:
\begin{enumerate}
\item \jg{sɯ-tsʰi} means ``make s.o. drink'' rather than ``give s.t. to drink to s.o.''
\item \jg{sɯ-pʰɣo} ``make someone escape'' instead of ``take away''
\item \jg{sɯɣ-lɯɣ} ``cause to get free'' instead of ``drop''
\end{enumerate}
We observe that the regular causatives also have a regular semantic derivation from the basic verbs. The irregular causatives of these verbs can be used with the additional regular causative (\jg{sɯ-ɕlɯɣ} ``cause to drop'').   We can infer from these two facts that some of the irregular causatives (not including those in \ipa{ɕɯ-}) are no longer synchronically causatives or the original verb, as their meaning has begun to evolve independently.

The presence of a Tibetan loanword \jg{rga} ``to be glad, to like'' (from Tibetan \ipa{dga}) in this list shows that these allomorphs were still productive relatively recently. 

The original distribution of the allomorphs in alveolo-palatals \ipa{ɕɯ-}, \ipa{ɕɯɣ-}, \ipa{ɕ-} and \ipa{ʑ-} is unclear. They are only marginally restricted by the place of articulation of the initial consonant (they occur with labial, dental, velar and uvular - all except alveolo-palatal and palatal consonants), and occur with both simple initials and complex clusters. 

\ipa{ɕɯɣ-}, like \ipa{sɯɣ-},  probably occurred with intransitive verbs without initial cluster, possibly with labial initials. We will see that a similar \ipa{-ɣ-} intrusive element appears with the applicative and the tropative prefixes.

\ipa{s-} seems to occur with polysyllabic verb stems whose first element begins with a voiceless stop.

It is interesting to note that \jg{ɴqoʁ} has two distinct irregular causatives with different meanings; \jg{ʑɴɢoʁ} is not   a causative any more from a synchronic point of view, since it can appear with a causative prefix \jg{sɯ-} (meaning ``hang with something'', see \ref{subsub:causation}). The causative \jg{jtsʰi} of \jg{tsʰi} is only found in the Kamnyu dialect of Japhug. In most Japhug dialects, where the verb ``to drink'' is \jg{tʰi}, its irregular causative is \jg{ɕtʰi}. The Kamnyu form results from a dissimilation *\ipa{ɕtʰi} > *\ipa{ɕtsʰi} > \ipa{jtsʰi}. 


A few causative verbs not derived from intransitives in \ipa{a}-- have the allomorph \ipa{sɤ}--, in particular \ipa{sɤpe} ``do well'' (from \ipa{pe} ``be good'') and \ipa{sɤrmi} ``give a name'' (from \ipa{rmi} ``be named''). The latter could however perhaps be analysed as a denominal verb from \ipa{tɤ-rmi} ``name''.

The allomorphy of the causative prefix in Rgyalrongic languages raises a more general methodological question: should we assume that Rgyalrongic languages are innovative and that the observed allomorphy is due to relatively recent sound changes, or should this allomorphy be reconstructed back to proto-Rgyalrongic or even earlier? Given the present status of proto-Rgyalrongic reconstruction, a discussion of this issue has to be deferred to future research but we should keep in mind, when analyzing data from languages where no allomorphy is detectable that the absence of allomorphy may be due to levelling, rather than absence of innovative sound change.

\subsubsection{Syntactic constructions} \label{subsub:causation}
The causative prefix \ipa{sɯ-} is the most common morphosyntactic device to express causation in Japhug, though not the only one.\footnote{The causative verbs \ipa{sɯ-βzu} `cause to do' and \ipa{ɣɤ-kʰɯ} `cause to be possible' can be used with verb complements to express causation; this construction is not discussed in the present paper. }

When the causative is applied to an intransitive verb, the S becomes the O of the derived verb (\citealt[45]{dixon00causative}) . The A of the derived verb corresponds to the causer  or the stimulus of the causation.  The following example of the verb \ipa{sɯ-ɤʁdɤt} [\ipa{saʁdɤt}] ``to cause to slip'', causative of \ipa{aʁdɤt} ``to slip'', illustrates this principle:
\begin{exe}
\ex
\gll \ipa{tɯ-qe} 	\ipa{kɯ} 	\ipa{pɯ́-wɣ-sɯ-ɤʁdɤt} 	\ipa{nɯ} 	\ipa{pɯ-atɤr} 	\ipa{ɲɯ-ŋu} \\
\textsc{indef.poss}-dung \erg{} \pfv{}-\inv{}-\caus{}-slip \dem{} \pfv{}-fall \ipf{}-be \\
 \glt The dung caused him to slip and he fell down. (The Demon, 51)
\end{exe} 


With transitive verbs, a different situation is observed. The causative derivation adds an argument, the causer, which becomes the A of the causative verb. Since Japhug verbs cannot encode more than two arguments in their morphology, one of the arguments of the base verb must be demoted to leave place for the causer.  Four types of derivations are observed.

First, the causee (the A of the original verb) becomes the O of the derived verb, while the original O is demoted. Most causatives are formed this way:\footnote{3$\rightarrow$\textsc{1sg} and 2$\rightarrow$\textsc{1sg} forms are chosen because these are the only ones where the number and person of both arguments are indexed on the verb.}
 \begin{exe}
\ex
\gll  \ipa{aʑo} 	\ipa{ɯ-ʁɟo} 	\ipa{ɲɯ́-wɣ-jtsʰi-a-ndʑi} 	\ipa{pɯ-ɕti}  \\
  I 3\sg{}:\poss{}-diluted.wine \ipf{}-\inv{}-\caus{}:drink-1\sg{}-\du{} \pst{}.\ipf{}-be.\emphat{} \\
  \glt {They_d} gave me diluted wine to drink. (Kunbzang, 71)
\end{exe} 
 
 \begin{exe}
\ex
\gll  \ipa{nɤ-tɤɲi} 	\ipa{taʁ} 	\ipa{kɤ-rɤt} 	\ipa{nɯ} 	\ipa{ɯβrɤ-kɯ-z-nɤmɲo-a-nɯ} \\
  2\sg{}-staff on \nmlz{}:O-write \topic{} \qu{}-2$\rightarrow$1-\textsc{caus}-observe-1\sg{}-\pl{} \\
  \glt Could you show me what is written on your staff? (The prince, 61)
\end{exe} 

 \begin{exe}
\ex
\gll \ipa{nɤ-pi} 	\ipa{ni} 	\ipa{kɯ} 	\ipa{nɯ} 	\ipa{lɤ́-wɣ-sɯ-tɕat-a-ndʑi} 	\ipa{ɕti}  \\
2\sg{}.\poss{}-elder.sibling \du{} \erg{} \dem{} \textsc{pfv}-\inv{}-\caus{}-take.out-1\sg{}-\du{} be.\emphat{}:\fact{} \\
  \glt Your two elder sisters forced me to spit it out. (The three sisters, 254)
\end{exe} 

Second, we find ambiguous causative forms for some verbs, where either the agent or the patient of the original verb is preserved: in other words, the O of the causative verb can either correspond to the original A of the original O. These two types of derivation would respectively belong to types iv. and v. in \citet[48]{dixon00causative}'s typology): shift of either the A or the O of the original verb to non-core status. Note that in Japhug the non-core status of these arguments is indicated by the absence of agreement on the verb, not by any overt marking on the noun phrases. 
 
As an example, the causative of \ipa{qur} ``to help''  \ipa{sɯ-qur} has two meanings:

\begin{exe} 
\ex \label{ex:caus:help.2>3>1}
\gll   \ipa{tɤ-kɯ-sɯ-qur-a-ndʑi}  \\
 \pfv{}-2$\rightarrow$1-\caus{}-help-1\sg{}-\du{}  \\
 \glt  You_d caused me to help him. OR You_d caused him to help me. \wav{8_tAkWsWqura}
\end{exe} 

Similarly the causative of \ipa{mto} ```to see'' \ipa{sɯ-mto} means either ``to cause X to be seen'' or ``to show to X (to cause X to see)'':

\begin{exe} 
\ex \label{ex:caus:show.2>3>1}
\gll  \ipa{kɯm} 	\ipa{pɯ-a-pa} 	\ipa{ɕti} 	\ipa{ri,} 	\ipa{kɯm} 	\ipa{lɤ-tɯ-cɯ-t} 	\ipa{tɕe,} 	\ipa{tɯrme} 	\ipa{ra} 	\ipa{kɯ} 	\ipa{pɯ-kɯ-sɯ-mto-a}  \\
door \pst{}.\ipf{}-\pass{}-close be.\emphat{}:\fact{} but door \pfv{}-2-open-\pst{} \lnk{} people \pl{} \erg{}  \pfv{}-2$\rightarrow$1-\caus{}-see-1\sg{}  \\
 \glt The door was closed, but you opened it, you caused me to be seen by the people. (elicited, Chen Zhen 2011) \wav{8_pWkWsWmtoa} 
\end{exe} 


\begin{exe}
\ex
\gll \ipa{kɯki} 	\ipa{laχtɕʰa} 	\ipa{ki} 	\ipa{wuma} 	\ipa{ʑo} 	\ipa{nɯ-ɕar-a} 	\ipa{ri,} 	\ipa{aʑo} 	\ipa{mɯ-pɯ-mto-t-a} 	\ipa{ri,} 	\ipa{nɤʑo} 	\ipa{kɯ} 	\ipa{pɯ-kɯ-sɯ-mto-a}   \\
this thing this very \emphat{} \pfv{}-search-1\sg{} but I \negat{}-\pfv{}-\caus{}-see-\pst{}-1\sg{} but you \erg{}  \pfv{}-2$\rightarrow$1-\caus{}-see-1\sg{} \\
 \glt I looked for this thing for a long time, but could not find it, but you showed it to me. (elicited, Chen Zhen 2011)
\end{exe} 

\begin{exe}
\ex
\gll   \ipa{koŋla} 	\ipa{tɤjpa} 	\ipa{pjɯ-kɯ-sɯ-mto-j} 	\ipa{ɯ-jɤɣ?}   \\
real snow \ipf{}-2$\rightarrow$1-\caus{}-see-1\pl{} \intrg{}-be.possible:\fact{}  \\
 \glt Can you show us real snow? (The snow08.25)
\end{exe} 


Here is an example of O-preservation from a narrative:
\begin{exe} 
\ex \label{ex:caus:beaten}
\gll  \ipa{nɯnɯ} 	\ipa{ni} 	\ipa{pjɤ́-wɣ-sɯ-ʁndɯ-ndʑi} 	\ipa{tɕe}  \\
\dem{} \du{} \evd{}-\inv{}-\caus{}-beat-\du{} \lnk{} \\
\glt They_p had the two of them beaten (by people). (not to be understood as: ``They made two people beat them'', Fox 126)
\end{exe} 
Preservation of the original O   instead of the A occurs in verbs with  human patients, when the patient is higher than the agent on the empathy hierarchy (in example \ref{ex:caus:show.2>3>1}, first person > third person indefinite), or, when all arguments are third person, when the O of the original verb is more topical than the A (example \ref{ex:caus:beaten}).


Third, the causative appears in sentences with an overt instrument in the ergative case (example \ref{ex:kuWGsWrkhe}). This specific use of the \ipa{sɯ--} prefix will be referred to as `instrumental causative'.

\begin{exe}
\ex \label{ex:kuWGsWrkhe}
\gll  \ipa{ɯ-χto} 	\ipa{nɯ} 	\ipa{mbrɯtɕɯ} 	\ipa{kɯ} 	\ipa{kú-wɣ-sɯ-rkʰe}  \\
3\sg{}.\poss{}-slit \topic{} knife \erg{}  \ipf{}-\inv{}-\caus{}-carve \\
 \glt  The slit is carved with a knife. (Colored belts 13)
\end{exe} 

At least some irregular causatives are also used in this way:

\begin{exe}
\ex
\gll \ipa{kʰɤlɤβ} 	\ipa{kɯ} 	\ipa{tɯtʰɯ} 	\ipa{pɯ-ɕɯ-fkaβ-a}  \\
cover \erg{} pan \pfv{}-\caus{}-cover-1\sg{} \\
 \glt  I covered the pan with a cover.
\end{exe} 

The instrumental causative prefix \ipa{sɯ--} can be added to a causative verb, resulting in two \ipa{sɯ--} prefixes in the same form, as in \ref{ex:pjWGsWsWspoR}. This is the only case of recursive application of a prefix in Japhug. 

\begin{exe}
\ex \label{ex:pjWGsWsWspoR}
\gll \ipa{nɯnɯ}  	\ipa{kɯ}  	\ipa{pjɯ́-wɣ-sɯ-sɯ-spoʁ}  	\ipa{ɲɯ-ŋu}  \\
\textsc{dem} \textsc{erg} \textsc{ipfv-inv-caus-caus}-have.a.hole \textsc{sens}-be \\
\glt One makes a hole (into it) with this. (Plough, 8)
 \end{exe} 
 
 In the sentences above, using the verbs \jg{kú-wɣ-rkhe} and \jg{pɯ-fkaβ-a} without the causative with the overt instrument is not  ungrammatical; the instrumental causative is optional when the base verb is transitive. However, it is obligatory when the base verb is intransitive; using the non-causative form \ipa{lo-βzi} \textsc{ifr}-become.drunk in example \ref{ex:lowGsWBzi} would result in an ungrammatical sentence.
 
 
\begin{exe}
\ex \label{ex:lowGsWBzi}
\gll
\ipa{cʰa} 	\ipa{kɯ} 	\ipa{ló-wɣ-sɯ-βzi} \\ 
alcohol \textsc{erg} \textsc{ifr-inv-caus}-become.drunk \\
\glt He became drunk from the Chang. (elicited)
\end{exe}

 
 Animates can occur as instruments in some rare cases:
 
\begin{exe}
\ex
\gll  \ipa{βʑar} 	\ipa{nɯnɯ} 	\ipa{kɯ,} 	\ipa{nɯnɯ,} 	\ipa{pɣa} 	\ipa{kɯ-xtɕi} 	\ipa{nɯ} 	\ipa{ra} 	\ipa{ʁɟa} 	\ipa{ʑo} 	\ipa{tu-ndze} 	\ipa{ʁɟa} 	\ipa{ʑo} 	\ipa{nɯnɯ} 	\ipa{kɯ} 	\ipa{ɯ-xtu} 	\ipa{cʰɯ-nɯ-sɯ-χse} 	\ipa{ɲɯ-ŋu} 	\ipa{kʰi}  \\
 buzzard \dem{} \erg{} \dem{} bird \nmlz{}:S-small \topic{} \pl{} entire \emphat{} \ipf{}-eat[III] entire \emphat{} \dem{} \erg{} 3\sg{}.\poss{}-belly \ipf{}-\auto{}-\caus{}-feed[III] \ipf{}-be \textsc{hearsay} \\
\glt  The buzzard always eats small birds, and always nourishes himself with them, it is said. (The buzzard, 3)
\end{exe} 

Although both instruments and A's are marked with the ergative, they differs from each other by their relativization patterns: instruments are relativized with the oblique participle \ipa{sɤ--} instead of the S/A participle \ipa{kɯ--} (see \ref{ex:sAxtCAr}).  

% \begin{exe}
%  \ex   \label{ex:kuwGsWxtCAr}  
%\gll \ipa{ɯnɯnɯ}  	\ipa{ri}  	\ipa{qase}  	\ipa{kɯ}  	\ipa{kú-wɣ-sɯ-xtɕɤr}  \\
%\textsc{dem} \textsc{loc} leather.rope \textsc{erg} \textsc{ipfv-inv-caus}-tie \\
%\glt There, one ties it with a leather rope. (Plough, 97)
%   \end{exe} 



 \begin{exe}
  \ex  \label{ex:sAxtCAr}  
  \gll [\ipa{nɯ-mtʰɤɣ}  	\ipa{sɤ-xtɕɤr}]  	\ipa{xɕɤfsa}  	\ipa{ma}  	\ipa{pjɤ-me}  \\
\textsc{3pl.poss}-waist \textsc{nmlz:oblique}-tie thread apart.from \textsc{evd.ipfv}-not.exist \\
\glt They only had threads to tie their waists (the only things that they could use to tie their waists were threads). (Milaraspa translation)
   \end{exe} 

Fourth, the causative also occurs in very special syntactic constructions involving stative verbs. First, the causative form of a stative verb can occur with the infinitive of an action verb as its complement, expressing the manner of the action:
\begin{exe}
\ex
\gll 
 \ipa{kɤ-ɣndʑɯr} 	\ipa{cʰɤ-sɯ-ɤmɲɤm}  \\
 \inftv{}-grind \evd{}-\caus{}-homogeneous \\
 \glt  He ground it smooth.
\end{exe} 

In this construction, we observe raising of the directional prefix of the complement verb (in the example above, for instance, the intrinsic directional prefix of \jg{ɣndʑɯr} ``to grind'' is \ipa{tʰɯ- / cʰɤ-} ``downstream''). Both the causative verb and the complement transitive verb share the same A and O.

Second, the causativized stative verb occurs as the first element of a serial verb construction, expressing again the manner or circumstances of the second verb:

\begin{exe}
\ex
\gll \ipa{a-tʂʰa} 	\ipa{ci} 	\ipa{pɯ-z-mɤke} 	\ipa{pɯ-rke} \\
1\sg{}.\poss{}-tea a.little \imp{}-\caus{}-be.before[III] \imp{}-put.in[III] \\
\glt  Serve me some tea first.
\end{exe} 
As in all such constructions, both verbs share the verb TAM and person features.

 
The causee (the original A) can be marked with the ergative, as seen in the examples above. When the causee is an instrument, ergative marking is obligatory, and one can find sentences with two ergatives, though these are rarely attested in stories:
\begin{exe}
\ex
\gll \ipa{nɤ-pi} 	\ipa{ni} 	\ipa{kɯ} 	\ipa{scoʁ} 	\ipa{kɯ} 	\ipa{tú-wɣ-sɯ-ʁndɯ-a-ndʑi} 	\ipa{pɯ-ɕti} 	\ipa{tɕe,} 	\ipa{nɤʑo} 	\ipa{kɯ́nɤ} 	\ipa{nɯ} 	\ipa{tɤ-ste} 	\ipa{jɤɣ}  \\
2\du{}.\poss{}-elder.sibling \du{} \erg{} ladle \erg{} \ipf{}-\inv{}-\caus{}-hit-1\sg{}-\du{} \pst{}.\ipf{}-be.\emphat{} \lnk{} you also \dem{} \imp{}-do.this.way[III] be.possible:\fact{}  \\
\glt   Your two sisters hit me with a ladle, you can do the same. (Sentence retold by Chen Zhen from the story ``The three sisters'') \wav{8_scoRkW}
\end{exe} 


However, we do find  causees without ergative. First, topicalized ones (with a pause after the topicalizer):
\begin{exe}
\ex
\gll  \ipa{pʰuɲi} 	\ipa{nɯ}, 	\ipa{tʂaqʰu} 	\ipa{rŋgɯ} 	\ipa{taʁ} 	\ipa{a-ʑ-lɤ-sɯ-rpe} 	\ipa{ra}  \\
 broom.shrub \topic{} side.of.the.road rock on \irr{}-\transloc{}-\pfv{}:upstream-\caus{}-bump.into[III] have.to:\fact{}   \\
 \glt   With the broom shrub, you will have to touch the rock on the road. (Smanmi2.62)
\end{exe} 
 

 
Second, when the causee is human, the ergative rarely appears (though it is not ungrammatical): 
 
 \begin{exe}
\ex
\gll \ipa{tɕʰeme} 	\ipa{nɯ} 	\ipa{kɯjŋu} 	\ipa{kɯ-wxtɯ-wxti} 	\ipa{ʑo} 	\ipa{pa-sɯ-ta-ndʑi} \\
girl \topic{} oath \nmlz{}:\stat{}-\intens{}-big \emphat{} \pfv{}:3$\rightarrow$3'-\caus{}-put-\du{} \\
 \glt   They forced the girl to take a great oath. (Fox, 141)
\end{exe} 
  
  
When the original verb is intransitive, the causee is not marked with ergative (this does not apply to cases of instrumental causative).
 
 \begin{exe}
\ex
\gll \ipa{tɤ-se} 	\ipa{mtsʰu} 	\ipa{tú-wɣ-sɯ-mtsʰɤt} \\
\textsc{indef.poss}-blood lake \ipf{}-\inv{}-\caus{}-full \\
 \glt   Let's fill the lake with blood. (Smanmi2.95)
\end{exe} 

The stative verb \ipa{mtsʰɤt} ``be full'' can appear with both the container and the containee without case marking (the container is the real S, while the containee is an adjunct). Adding causative marking on the verb does not promote the containee to core argument status.

\subsubsection{Compatibilities} \label{subsub:caus1.compatibility}
The  causative \jg{-sɯ-} is highly productive and can appear with various other derivational prefixes, including the  reflexive \jg{-ʑɣɤ-}, the causative \jg{ɣɤ-}, the passive \jg{a-}, the autobenefactive-spontaneous \jg{-nɯ-} and all other derivational prefixes. 


%tropative, facilitative

The reflexive  \jg{-ʑɣɤ-} is the only derivational prefix that occurs before the causative. 	Only the order  \jg{-ʑɣɤ-sɯ-} is attested, and the reverse order is unintelligible to Japhug speakers.

 The combination of these two prefixes could potentially have two interpretations: either X cause Y to do to X (scope of the reflexive over the causative), or X cause to Y to do to Y (scope of the causative over the reflexive). However, only the first interpretation is possible, as shown by the examples:
%\citet[214]{bickel07inflectional}
\begin{exe}
\ex 
\gll \ipa{pɯ-ʑɣɤ-sɯ-sat}  \\
  \pfv{}-\refl{}-\caus{}-kill \\
\glt   He_i caused (him, them) to kill himself_i
\end{exe}
	\begin{exe}
\ex 
\gll \ipa{ɯʑo} 	\ipa{mɯ-to-rɯndzaŋspa} 	\ipa{tɕe} 	\ipa{pjɤ-ʑɣɤ-sɯ-mto} \\
 he \negat{}-\evd{}-careful \lnk{} \evd{}-\refl{}-\caus{}-see \\
\glt   He wasn't careful enough and got himself seen. \wav{8_ZGAsWmto}
\end{exe}

The two sentences above cannot be understood as  ``He_i caused (him, them)_j to kill (him,them)selve(s)_j'' or ``he_i caused him_j to see himself_j''.


The causative commonly appears with the autobenefactive-spontaneous \ipa{nɯ-}:


 \begin{exe}
\ex 
\gll \ipa{ɯ-sci} 	\ipa{iɕqʰa} 	\ipa{ɯ-sroʁ} 	\ipa{nɯ} 	\ipa{ɯ-kɤ-kɯ-ri} 	\ipa{nɯnɯ} \ipa{ʑ-la-nɯ-sɯ-ɣe-nɯ} \\
3\sg{}.\poss{}-replacement the.aforementioned 3\sg{}.\poss{}-life \topic{} 3\sg{}-\pfv{}-\nmlz{}:A-save \dem{} \transloc{}-\pfv{}:3$\rightarrow$3-\auto{}-\caus{}-come-\pl{} \\
\glt In his_j place, they (send people) to invite him_i to come, he_i who saved her life.    (the demon, 162)
\end{exe}

 The causative also appears in combination with the passive as \ipa{sɤ}-- < \ipa{sɯ-ɤ}--, but only in a limited number of verbs:
 
 
  \textbf{\ipa{sɤmbi}} ``to require something from someone'' is a causative form derived from the passive \ipa{a-mbi} ``to be given'' of the verb \ipa{mbi} ``to give''. Etymologically, the verb means ``to cause someone to give to oneself''. 
  
  
  
    \textbf{\ipa{sɤjtsʰi}} ``to ask for something to drink'' derives from the irregular lexicalized causative \ipa{jtsʰi} ``to give to drink''. The etymological causative prefixes \ipa{j}-- being fossilized and not analysed synchronically as such in modern Japhug, this form is not a counterexample to the verbal template. As \ipa{mbi} ``to give'', \ipa{jtsʰi} has the recipient coded as the O:
      \begin{exe}
\ex
\gll   \ipa{a-wɯ} 	\ipa{tɯ-ci} 	\ipa{ɲɯ-kɯ-jtsʰi-tɕi} 	\ipa{ɯ́-jɤɣ}  \\
1\sg{}.\poss{}-grandfather \textsc{indef.poss}-water \ipf{}-2$\rightarrow$1-give.to.drink-1\du{} \qu{}-could:\fact{} \\
 \glt Grandfather, could you give us water to drink? (Nima Vodzer 72)
\end{exe}   
 
  
  \textbf{\ipa{sɤβzu}} ``to prepare, to make ready to use'' derives from \ipa{a-βzu}, a verb whose meaning in modern Japhug is ``to grow'', but which originally was the passive of \ipa{βzu} ``to make''. \ipa{sɤβzu}  is therefore etymologically ``to cause to be made''.   The reflexive prefix \ipa{ʑɣɤ}-- can further be added to form the verb \ipa{ʑɣɤ-sɯ-ɤβzu} ``to transform oneself into".

  \textbf{\ipa{sɤpa}} ``transform (tr.)'' is the causative of \ipa{apa} ``become'', itself the passive of the verb \ipa{pa}, which means ``close (the door)'' in modern Japhug (among other meaning) but used to be the  regularly verb ``to do'' in Rgyalrongic languages. \ipa{sɤpa} is always transitive.
  
  This verb can in turn be combined with the reflexive \ipa{ʑɣɤ}-- to form \ipa{ʑɣɤ-sɯ-ɤ-pa} ``to transform oneself into'':
  
  \begin{exe}
\ex
\gll \ipa{ɯ-tɕɯ} 	\ipa{nɯ} 	\ipa{ɕkɤrɯ} 	\ipa{na-sɯ-ɤpa,} 	\ipa{ɯʑo} 	\ipa{xtɯt} 	\ipa{nɯ-ʑɣɤ-sɯ-ɤpa} 	\ipa{ɲɯ-ŋu} \\
3\sg{}.\poss{}-son \topic{} serow \pfv{}.3$\rightarrow$3-\caus{}-become she wild.cat \pfv{}-\refl{}-\caus{}-become \ipf{}-be\\
 \glt She changed her son into a serow, and herself into a wild cat.  (Lobzang 54)
\end{exe} 


It is quite clear that the combination of the causative with the passive is not productive in Japhug, and that the analysis proposed above is only true from a diachronic perspective. 

The causative \ipa{sɯ-} is also compatible with the causative \ipa{ɣɤ-}, though such examples are unusual, and appear limited to the use of  the causative to mark the instrument, as in \ref{ex:double.caus}.
 \begin{exe}
\ex  \label{ex:double.caus}
\gll \ipa{smɤnba} 	\ipa{kɯ} 	\ipa{smɤn} 	\ipa{ɲo-kʰo} 	\ipa{tɕe,} 	\ipa{ɯ-kɯ-mŋɤm} 	\ipa{to-z-ɣɤ-mna}    \\
  doctor \erg{} medicine \evd{}-give \lnk{} 3\sg{}-\nmlz{}:\stat{}-hurt \evd{}-\caus{}-\caus{}-be.cured \\
\glt The doctor gave him a medicine and cured him with it. \wav{8_zGAmna}
\end{exe}
 
%In very rare cases, it is possible to combine a causative \ipa{sɯ--} with a causative instrumental, as in 


\subsubsection{The semantics of the causative } \label{subsub:caus.semantics}
In his cross-linguistic overview of causatives, \citet[62-68]{dixon00causative} proposes nine parameters to study the semantic specificities of causative constructions. The first two, state vs. action and transitivity, are treated in the morphology, and will not concern us here. 

Of the seven remaining parameters, three (Control, Volition, Affectedness) relate to the causee, and four (Directness, Intention, Naturalness, Involvement) to the causer; in this section, we will regroup them into four groups by combining directness, naturalness and involvement, as these three parameters are most often intertwined in our examples.

 
 We will show that the \ipa{sɯ-} causative in Japhug has a wide range of uses, and can appear independently of these parameters. Is it compatible with either obligation, authorisation, accompaniment or various modes of causation. 
 
   \textbf{Control}. The prefix \ipa{sɯ-} occurs both with actions on which the causee has control, but also with action on which he/it has no control, either because it is inanimate, or because the action itself is not controllable:
 \begin{exe}
\ex 
\gll  \ipa{ɯrɟɤnpanma} 	\ipa{kɯ} 	\ipa{tɯɣ} 	\ipa{pjɤ-lɤt,} 	\ipa{tɕendɤre} 	\ipa{nɯ-wa} 	\ipa{ko-z-nɤndza,}  \\
Padmasambhava \erg{} poison \evd{}-use \lnk{} 3\pl{}.\poss{}-father \evd{}-\caus{}-have.leprosy \\
 \glt Padmasambhava used a poison, and caused their father to contract leprosy.    (Gesar 15)
\end{exe}

 \textbf{Volition}. The causative \ipa{sɯ-} appears regardless of whether causee acts willingly (`let, ask') or unwillingly (`make, force').
 
 This first example shows that the causative can be used when doing someone a favour:
  \begin{exe}
\ex 
\gll   \ipa{a-mu} 	\ipa{ndʑu} 	\ipa{cinɤ}	\ipa{ʑo} 	\ipa{a-mɤ-nɯ-tɯ-sɯ-qlɯt-nɯ} \\
  1\sg{}.\poss{}-mother chopsticks even \emphat{} \irr{}-\negat{}-\pfv{}-2-\caus{}-break-\pl{} \\
 \glt    Please make sure that my mother does not even need to break chopsticks (go out to break twigs from the trees to make chopsticks; this idiomatic expression means ``take care of her every need''). (Slobdpon, 220)
\end{exe}

 It can also be used when one asks someone to do something:
 \begin{exe}
\ex 
\gll  \ipa{βlama} 	\ipa{kɯ-wxti} 	\ipa{ʑo} 	\ipa{ɲɤ-sqɤr-nɯ} 	\ipa{tɕe,} 	\ipa{tɤ-rpi} 	\ipa{kɯ-wxtɯ-wxti} 	\ipa{ʑo} 	\ipa{ɲɤ-sɯ-βzu-nɯ}  \\
lama \nmlz{}:\stat{} \emphat{} \evd{}-ask.to.do-\pl{} \lnk{}  \textsc{indef.poss}-sutra \nmlz{}:\stat{}-\intens{}-big \emphat{} \evd{}-\caus{}-do-\pl{} \\
 \glt   They employed a great lama and asked him to recite a major sutra. (Rkang-rgyal, 19-20)
\end{exe}

Finally, it can also express coercion, with adverbs such as \ipa{tɤrkoz} 	or \ipa{mɤkɯftsʰi} ``forcefully'':
 \begin{exe}
\ex 
\gll \ipa{kɤ-ndza} 	\ipa{a-ʁjiz} 	\ipa{mɯ́j-ɣi} 	\ipa{ri} 	\ipa{ɯʑo} 	\ipa{kɯ} 	\ipa{tɤrkoz} 	\ipa{tʰɯ́-wɣ-sɯ-ndza-a} \\
\nmlz{}:O-eat 1\sg{}-will \negat{}:\textsc{sens}-come but he \erg{} forcefully \pfv{}-\inv{}-\caus{}-eat-1\sg{} \\
 \glt   I did not want to eat it, but he forced me to. (Chen Zhen, 2005)
\end{exe}

 
   \textbf{Intention}. The causative prefix can appear with unintentional actions:
 \begin{exe}
\ex 
\gll 
\ipa{tɯ-ŋga} 	\ipa{ɲɤ-sɯ-ɤrŋi-t-a} \\
	 \textsc{indef.poss}-clothes \evd{}-\caus{}-blue-\pst{}-1\sg{} \\
 \glt   I caused the clothes to become blue (unintentionally, by washing them the wrong way; el., Chen Zhen)
\end{exe}
 
% \begin{exe}
%\ex 
%\gll 
% 	\ipa{tɯtʰɯ} 	\ipa{kɯ} 	\ipa{a-jaʁ} 	\ipa{ʂaʁ} 	\ipa{kɤ-nɯ-sɯ-ta-t-a} \\
% 	pan \erg{} 1\sg{}.\poss{}-hand burn \pfv{}-\auto{}-\caus{}-put-\pst{}-1\sg{} \\
% 	\glt   I burned my hands with the pan.
%\end{exe} > ko-nɯ-sɯ
 	

   \textbf{Directness, naturalness and involvement}.  The causative prefix \ipa{sɯ-} can express various degrees of involvement on the part of the causer, as exemplified by the following example:
 
  \begin{exe}
\ex 
\gll \ipa{ɯ-mbro} 	\ipa{kɯ} 	\ipa{qapri} 	\ipa{tɯ-rdoʁ} 	\ipa{nɯ} 	\ipa{pjɤ-z-rɤtɕaʁ} 	\ipa{tɕe,} 	\ipa{tɤte} 	\ipa{kɯ-wɣrum} 	\ipa{nɯ} 	\ipa{lo-sɯ-qioʁ} \\
3\sg{}.\poss{}-horse \erg{} snake one-piece \pl{} \evd{}-\caus{}-trample \lnk{} that.is \nmlz{}:\stat{}-white \topic{} \evd{}\caus{}-vomit \\
 \glt  (Nyima Wodzer) had his horse trample one of the snakes, and caused it to throw up the white one (snake). (Nyima Wodzer,30)
\end{exe}
In the first  clause, the causee of the verb verb \ipa{pjɤ-z-rɤtɕaʁ} 	``he caused him to trample'' is the horse, while in the second one, the causee is the snake that was trampled; the causer (the character Nyima Wodzer) in the second case only acts indirectly (through the action of his horse).
 
 
 
 
 The causative is also used to express authorisation, where the causer's involvement is even more indirect, and only amounts to an absence of action:

 \begin{exe}
\ex 
\gll \ipa{ku-kɯ-z-rɤʑi-a-nɯ} 	\ipa{ɲɯ-ntsʰi} \\
\ipf{}-2$\rightarrow$1-\caus{}-stay-1\sg{}-\pl{} \ipf{}-have.better \\
 \glt  Could you let me stay? (The raven, 68)
\end{exe}

Finally, it can even  appear in situations where the ``causer'' merely omits to act upon a naturally occurring event:

  \begin{exe}
\ex 
\gll \ipa{tɤ-mtʰɯm} 	\ipa{ɲɤ-z-ɣɤdi-t-a}  \\
 \textsc{indef.poss}-meat \evd{}-\caus{}-be.smelly-\pst{}-1\sg{} \\
 \glt  I let the meat spoil.
\end{exe}

Some examples of \ipa{sɯ-/z-} have a  semantics which is more reminiscent of the tropative (\citealt{jacques13tropative}): \jg{znɤja} ``consider to be a shame'', \jg{sɯpa} ``regard as'' and \jg{znɤkɤro} ``consider to be acceptable''.

The intransitive verb \jg{nɤja} means ``to be a shame, to be a pity''.
  \begin{exe}
\ex
\gll \ipa{iɕqʰa} 	\ipa{laχtɕʰa} 	\ipa{pjɤ-ɴɢrɯ,} 	\ipa{pɯ-nɤja} \\
the.aforementioned thing \evd{}-\acaus{}:break \pfv{}-be.a.shame \\
  \glt  That thing broke, what a shame!
   \end{exe}

   The transitive \jg{z-nɤja}, rather than meaning ``to cause to be a shame'' as expected regularly, rather means ``to regret, be reluctant'' (Chinese \zh{不舍得}), in other words ``to consider something to be a pity'':
   
     \begin{exe}
\ex 
\gll \ipa{wuma} 	\ipa{ʑo} 	\ipa{pɯ-znɤja-t-a}  \\
very \emphat{} \pfv{}-regret-\pst{}-1\sg{} \\
  \glt  I regretted it very much. (a lost cellphone cover, Dpalcan, conversation, 2010)
   \end{exe}
Another verb having unpredictable semantics with the prefix \ipa{sɯ-} is the transitive verb \ipa{sɯ-pa} ``to consider, to regard as''. The original verb   is \ipa{pa} ``to close'', etymologically ``to do'':

     \begin{exe}
\ex 
\gll \ipa{tɤkʰe-pɣɤtɕɯ} 	\ipa{nɯ} 	\ipa{ɯʑo} 	\ipa{pɣɤtɕɯ} 	\ipa{nɯ} 	\ipa{kɯ-kʰe} 	\ipa{tu-sɯpa-nɯ} \\
stupid-bird \topic{} he bird \topic{} \nmlz{}:\stat{}-stupid \ipf{}-consider-\pl{} \\
 \glt The \ipa{tɤkʰe-pɣɤtɕɯ} is considered to be a stupid bird. (the buzzard, 13)
   \end{exe}
   
   \subsubsection{Scope ambiguity}
   
  
The causative presents scope ambiguity with several other prefixes, in particular negation and associated motion.

The negative prefix can either have scope over the base verb (cause not to do = hinder) or over the causative (not cause to do). This may be an effect of the rigid  verbal template, as the relative order of the negation and the causative are strictly fixed.

Examples with negation  in the sense of ``hinder'', ``cause not to do'') are quite common:


  \begin{exe}
\ex
\gll \ipa{a-ʑɯβ} \ipa{mɯ́j-sɯ-ɣe-nɯ} \\
	1\sg{}-sleep \negat{}:\sens{}-\caus{}-come-\pl{} \\
    \glt They don't let me sleep. NOT ``They do not cause me to sleep'' (Dpalcan, 2010, elicitation)
  \end{exe} 
  
  \begin{exe}
\ex
\gll  \ipa{ɯ-tɯ-ɣɤcraŋlaŋ} 	\ipa{kɯ} 	\ipa{koŋla} 	\ipa{mɯ́j-kɯ-z-rɤ-βzjoz} 	\ipa{ʑo} \\
3\sg{}-\nmlz{}:\degr{}-make.noise \erg{} really \negat{}:\sens{}-\genr{}:S/O-\caus{}-\apass{}-learn \emphat{} \\
  \glt  They make so much noise that they do not let people study at all. (Dpalcan, 2010, elicitation)
  \end{exe}   
  
   \begin{exe}
\ex
\gll   \ipa{nɯtɕu} 	\ipa{ku-je} 	\ipa{tɕe} 	\ipa{tɯ-ci} 	\ipa{tɯ-mɯm} 	\ipa{mɯ-pjɤ-sɯ-tsʰi} 	\ipa{ɲɯ-ŋgrɤl} \\
there \ipf{}-keep.in.enclosure[III] \lnk{}  \textsc{indef.poss}-water one-swallow \negat{}-\evd{}-\caus{}-drink \ipf{}-be.usually.the.case \\
\glt (The male deer) prevent (the female) to leave the place, and do not let them drink even a swallow of water. (dictionary entry on ``deer'', 2005)
  \end{exe} 
  
\begin{exe}
\ex
\gll  \ipa{aʑo} 	\ipa{ɲo-nɯ-jmɯt-a} 	\ipa{tɕe,} 	\ipa{rɟɤlpu} 	\ipa{ɯ-ɕki} 	\ipa{pɣɤtɕɯ} 	\ipa{kɯ} 	\ipa{mɯ-tɤ-sɯ-tɯt-a}   \\
 I \textsc{ifr}-\auto{}-forget-1\sg{} \lnk{} king 3\sg{}-\dat{} bird \erg{} \negat{}-\pfv{}-\caus{}-say[II]-1\sg{} \\
  
  \glt  I forgot it, so  on my account  the bird did not convey its message to the king (literally: I did not let the bird tell the king)\footnote{This example is adapted from a traditional story; the speaker here is the shepherd Askyabs \ipa{kɯlɤɣ acɤβ}, whom a bird (in fact a reincarnated queen) asks to deliver a message to the king. The bird itself does not go to see the king directly.  } \wav{8_mWtAsWtWta2}
  \end{exe} 

  However, the alternative interpretation, with the scope of the negation on the causation is also possible:

  \begin{exe}
\ex
\gll \ipa{aʑo} 	\ipa{ɕ-tɤ-nɯ-tɯt-a} 	\ipa{ma} 	\ipa{tɯrme} 	\ipa{mɯ-tɤ-sɯ-tɯt-a} \\
I \transloc{}-\pfv{}-\auto{}-say[II]-1\sg{} apart.from people \negat{}-\pfv{}-\caus{}-1\sg{} \\
  \glt    I went to convey (the message) myself,  and I did not make anyone else convey it. \wav{8_mWtAsWtWta}
  \end{exe} 
 
A scope ambiguity is also observed with the associated motion prefixes \ipa{ɕɯ--} `go and' and  \ipa{ɣɯ--} `come and' (on which see \citealt{jacques13harmonization}). In Japhug, associated motion prefixes normally have accusative alignment: the referent undergoing the motion is either the A or the S of the verb (depending on its transitivity), never the O. However, in the case of verbs with a causative \ipa{sɯ--}, the referent in motion can be either the  the causer (as in \ref{ex:assoc.motion1}, where the causative is used to indicated the presence of an instrument), the causee (as in \ref{ex:assoc.motion2}) or both.


  \begin{exe}
\ex \label{ex:assoc.motion1}
\gll
\ipa{wortɕʰi} 	\ipa{ʑo,} 	\ipa{kɯki} 	\ipa{jɤ-tsɯm} 	\ipa{tɕe,} 	\ipa{tʰɯci} 	\ipa{ftɕaka} 	\ipa{kɯra} 	\ipa{tsuku} 	\ipa{ɕ-tɤ-sɯ-χti} \\
please \textsc{emph} this \textsc{imp}-take.away \textsc{lnk} something thing these some \textsc{transloc-imp-caus}-buy[III] \\
\glt Please, take this and go to buy something with it. (The raven4 72)
  \end{exe} 


  \begin{exe}
\ex \label{ex:assoc.motion2}
\gll
\ipa{tɕe} 	\ipa{kupa} 	\ipa{cʰu} 	\ipa{nɯra} 	\ipa{atʰi} 	\ipa{pɕoʁ} 	\ipa{nɯra,} 	\ipa{ɯ-pɕi} 	\ipa{nɯra} 	\ipa{kɯ} 	\ipa{kɯreri} 	\ipa{ɣɯ-cʰɯ-sɯ-χtɯ-nɯ} 	\ipa{ŋu.}  \\
\textsc{lnk} Chinese \textsc{loc} \textsc{dem:pl} downstream direction \textsc{dem:pl} \textsc{3sg}-outside  \textsc{dem:pl}  \textsc{erg} here \textsc{cisloc-ipfv:downstream-caus}-buy-\textsc{pl} be\textsc{:fact} \\
\glt People from the Chinese areas, people from outside send people to come here to buy (matsutake and sell them in the areas downstream). (hist-20grWBgrWB 58)
  \end{exe} 



 \subsubsection{The causative \ipa{sɯ-} with stative verbs} \label{subsub:caus.sW.stative}
 Although the prefix \ipa{ɣɤ-}, rather than \ipa{sɯ-}, is used with most stative verbs, some stative verbs only appear with \ipa{sɯ-}. The following non-exhaustive list illustrates some examples:
 \begin{table}[H]
\caption{Examples of the \ipa{sɯ}- causative with stative verbs }\label{tab:causative.sW.stative} \centering
\begin{tabular}{lllllll} \toprule
  base  & &causative  \\
\midrule
   \ipa{arŋi} & blue & \ipa{sɯ-ɤrŋi} \\
 \ipa{wɣrum} & white & \ipa{sɯ-wɣrum} \\
 \ipa{ɲaʁ} & black & \ipa{sɯɣ-ɲaʁ} \\
  \ipa{ɣɯrni} & red & \ipa{z-ɣɯrni} \\
    \ipa{mɤrtsaβ} & spicy & \ipa{z-mɤrtsaβ} \\
       \ipa{mŋɤm} & be painful & \ipa{ɕɯ-mŋɤm} \\
\bottomrule
\end{tabular}
\end{table}
   
    Stative verbs with a prefixal element (\ipa{mɤ-}, \ipa{rɤ-}, \ipa{ɣɯ-} etc), always appear with \ipa{z-}, never with \ipa{ɣɤ-} (except some examples with the prefixal element \ipa{a--}). This constraint explains for instance why the causative of \ipa{mɤrtsaβ}  ``spicy'' is  in \ipa{z-} rather than \ipa{ɣɤ-}, while almost all other stative verbs denoting feelings or taste have a causative in \ipa{ɣɤ-}, for instance \ipa{tɕur}  ``sour'' > \ipa{ɣɤ-tɕur}  ``make   sour'',   \ipa{tsri}  ``salty'' > \ipa{ɣɤ-tsri}  ``make   salty'' etc.  
  
Color stative verbs and stative verbs related to disease and pain (\ipa{ngo} ``sick'', \ipa{mŋɤm} etc) also do  form their causative with \ipa{sɯ-} and its variants rather than with \ipa{ɣɤ-}, as seen in the table above.
  
  Very few stative verbs have been found which are compatible with both \ipa{ɣɤ-} and \ipa{sɯ-}; the semantic contrast between the two prefixes is treated in \ref{subsub:caus-g:semantics}.
 
 

\subsection{The causative prefix \ipa{ɣɤ--}} \label{sub:caus2}

The causative \ipa{ɣɤ-} has a much more restricted usage than \ipa{sɯ-} treated in the previous section. The prefix \ipa{ɣɤ-}  appears with most stative verbs, though as we have seen in \ref{subsub:caus.sW.stative},  some stative verbs also appear with \ipa{sɯ-}; the semantic differences between the two prefixes for stative verbs is treated in \ref{subsub:caus-g:semantics}. Cognates of this prefix are found in other Rgyalrongic languages (in Tshobdun and in Khroskyabs, see \citealt{jackson14morpho} and \citealt{lai13affixale}) and in Tangut (\citealt[253-4]{jacques14esquisse}).


Unlike \ipa{sɯ-}, \ipa{ɣɤ-} presents no allomorphy. With verbs having the intransitive determiner \ipa{a-}, this syllable is absorbed by the prefix. For instance, the causative of \jg{artɯm} ``round'' is \jg{ɣɤ-rtɯm} ``to coil (threads)''.

\subsubsection{Syntactic constructions} \label{subsub:caus-g:syntax}
Unlike \ipa{sɯ-}, \ipa{ɣɤ-} only appears with stative intransitive verbs. The added argument, the causer, is always the A, while the original S becomes the O.

\begin{exe}
\ex 
\gll \ipa{ɯ-mke}  	\ipa{cʰo-ɣɤ-rɲɟi} \\
3\sg{}-neck \evd{}-\caus{}-long \\
 \glt  He stretched his nec.k (elicited, Dpalcan 2010)
   \end{exe}
 
 \begin{exe}
\ex 
\gll   \ipa{ɯ-pʰɯ} \ipa{ɲɯ-wxti} \ipa{tɕe}, \ipa{nɯ} \ipa{ra} \ipa{tʰamtɕɤt} \ipa{ma-tɤ-tɯ-ɣɤ-wxti}\\
 3\sg{}.\poss{}-price \sens{}-big \lnk{} \dem{} \pl{} all \negat{}-\imp{}-2-\caus{}-big\\
 \glt  It is expensive, don't make it that expensive. (Bargaining, 11) 
   \end{exe}
 
 
Like the prefix \ipa{sɯ-} (see \ref{subsub:causation}), causative verbs with \ipa{ɣɤ-} are used with a complement in \ipa{kɤ-} infinitive to express the manner in which the action takes place:

\begin{exe}
\ex 
\gll \ipa{paʁndza}  	\ipa{kɤ-rɤkrɯ}  	\ipa{pa-ɣɤ-ndɯβ}  \\
pig.food \inftv{}-cut \pfv{}:3$\rightarrow$3-\caus{}-fine \\
 \glt  He chopped the pig food very fine. (elicited, Dpalcan 2010)
   \end{exe}
   
 \begin{exe}
\ex 
\gll \ipa{kɯm}  	\ipa{tɤ-ɣɤ-βdi-t-a}  	\ipa{tɕe,}  	\ipa{kɤ-cɯ}  	\ipa{tɤ-ɣɤ-kʰɯ-t-a}  	  \\
door \evd{}-\caus{}-good-\pst{}-1\sg{} \lnk{} \inftv{}-open \evd{}-\caus{}-be.possible-\pst{}-1\sg{} \\
 \glt   I repaired the door, so that it can be opened. (literally: I made the door openable, el. Chen Zhen, 2011)
   \end{exe}

    
As with the construction involving the prefix \ipa{sɯ-}, we observe that the directional prefix of the complement verb (in the infinitive) is raised to the causativized stative verb: \ipa{pɯ-} ``down'' and  \ipa{tɤ-} ``down'' are respectively the intrinsic directional prefixes of \ipa{rɤkrɯ} ``cut'' and  \ipa{cɯ} ``open'' (this verb also occurs with \ipa{kɤ-} ``towards east''). 
 
In this construction, the scope of the negation is normally on the causativized stative verb, not on the whole action:
  \begin{exe}
\ex 
 \gll \ipa{kɤ-rɤt} \ipa{mɯ-pjɤ-tɯ-ɣɤ-βdi-t} \\
\inftv{}-write \negat{}-\evd{}-2-\caus{}-good-\pst{} \\
 \glt    You did not write it well (``you wrote it badly'', not in the sense ``you did not write it at all''), (el. Chen Zhen 2011)
   \end{exe}  
   \begin{exe}
\ex 
 \gll  	 \ipa{kʰɤdaʁ}  	\ipa{tɤ-sɯfsaŋ}  	\ipa{tɕe}  	\ipa{tʰɯ-mqlaʁ}  	\ipa{ma}  	\ipa{kɤ-sci}  	\ipa{mɯ-nɯ-tɯ-ɣɤ-kʰɯ-t}  \\
Khatag \imp{}-fumigate \lnk{} \imp{}-swallow otherwise \inftv{}-be.born \negat{}-\pfv{}-\caus{}-be.possible-\pst{} \\
 \glt   Fumigate a \textit{khatag} and swallow it, otherwise you would make my birth impossible. (\textsc{not}: ``you did not make my birth possible''). (Gesar, 61)
   \end{exe}  
   
   
   The raising of the directional prefix from the complement verb to the causativizer stative verb can remain even when the complement verb is elided. For instance, with a verb such as \ipa{ɣɤ-xtɯt} ``shorten'', one can distinguish between:
   
   \begin{exe}
\ex 
 \gll  	 \ipa{nɯ-ɣɤ-xtɯt-a} / \ipa{tɤ-ɣɤ-xtɯt-a}   	   \\
 \pfv{}-\caus{}-short-1\sg{} / \pfv{}-\caus{}-short-1\sg{} \\
 \glt    I made it shorter. (elicited, Chen Zhen)
   \end{exe}     
   
 The first form means ``shorten by cutting (clothes)'', as the implicit complement verb is \ipa{qrɯ} ``cut'', whose intrinsic directional prefix (in the meaning ``to cut clothes'') is \ipa{nɯ}--.  \ipa{nɯ-ɣɤ-xtɯt-a} is actually a short form for:
 
   \begin{exe}
\ex 
 \gll  	\ipa{tɯ-ŋga} \ipa{kɤ-qrɯ} \ipa{nɯ-ɣɤ-xtɯt-a}  	   \\
 \textsc{indef.poss}-clothes \inftv{}-cut \pfv{}-\caus{}-short-1\sg      \\
 \glt    I made the clothes shorter. (el., Chen Zhen)
   \end{exe}
 \ipa{tɤ-ɣɤ-xtɯt-a} , with the prefix \ipa{tɤ}-- ``up'' instead means that the clothes were made shorter by rolling sleeves up, without cutting the cloth.
 
 
\subsubsection{Compatibilities} \label{subsub:caus2:compat}
The prefix \ipa{ɣɤ-} , is incompatible with other valency-increasing prefixes such as the tropative  \ipa{nɤ-} and the applicative  \ipa{nɯ-}. However, both the reflexive  \ipa{ʑɣɤ-}   and causative  \ipa{sɯ-} can appear before it:
 \begin{exe}
\ex 
\gll \ipa{nɤʑo}  	\ipa{tɤ-muj}  	\ipa{stʰɯci}  	\ipa{a-tɤ-tɯ-ʑɣɤ-ɣɤ-ʑo,}  	\ipa{nɤ-mbro}  	\ipa{qale}  	\ipa{stʰɯci}  	\ipa{a-nɯ-ʑɣɤ-ɣɤ-mbjom}  	  \\
you  \textsc{indef.poss}-feather  so.much \irr{}-\pfv{}-2-\refl-\caus{}-light, 2\sg{}.\poss{}-horse wind so.much \irr{}-\pfv{}-\refl-\caus{}-fast \\
 \glt   May you be as light as a feather, and your horse as swift as the wind. (Smanmi Metog Koshana, 62)
   \end{exe}
Example  \ref{ex:double.caus} above (\ref{subsub:caus1.compatibility}) illustrates a verb with both causative prefixes.



It also with the reciprocal \ipa{a}--+\textsc{reduplication} in forms such as \ipa{rlaʁ}    ``to disappear'' > \ipa{ɣɤ-rlaʁ} ``to lose, to cause to be destroyed'' > \ipa{a-ɣɤ-rlɯ-rlaʁ} ``to destroy each other''. The reverse order is however not possible.
 

\subsubsection{Semantics} \label{subsub:caus-g:semantics}
\citet{jackson06paisheng, jackson14morpho}, with regard to the causative prefixes \ipa{sə}-- and \ipa{wɐ}-- in Tshobdun, has shown that in the case of some stative verbs, the former indicates an increase of degree, while the latter expresses a change of state. This contrast appears to have been lost in Japhug (at least in the variety under study).
Some labile verbs, such as \ipa{mto} (which means `see' as a transitive verb and `have the ability to see' as an intransitive stative verb') have distinct causative forms depending on the base meaning: \ipa{sɯ-mto} `cause to see, show' is based on the transitive \ipa{mto}, while \ipa{ɣɤ-mto} `cause (a blind person) to recover sight' is based on the stative \ipa{mto}.

\subsection{Abilitative}
The abilitative \ipa{sɯ--} is homophonous with the causative, and does   present two allomorphs \ipa{sɯ--} and \ipa{z--} with the same distribution as for the causative.\footnote{Only examples with transitive verbs have been found; thus there are no allomorphs \ipa{sɯɣ--} and \ipa{sɤ--}, that can only be found on intransitive verbs.} The abilitative expresses that the S/A is physically able to realize the action described by the verb. It is completely homophonous with the causative; for instance \ipa{sɯ-ndza} means both `cause/force to eat' and `be able to eat'. Although it appears to be quite productive (it can be applied to most transitive voluntary verbs), it is quite rare in the corpus, and most commonly appears in negative forms (see \ref{ex:mWpjAznACqa}).

 \begin{exe}
\ex \label{ex:mWpjAznACqa}
\gll
\ipa{tɕeri} 	\ipa{tɤ-mu} 	\ipa{nɯ} 	\ipa{kɯ} 	\ipa{maka} 	\ipa{mɯ-pjɤ-z-nɤɕqa,} \\
but \textsc{indef.poss}-mother \textsc{dem} \textsc{erg} at.all \textsc{neg-ifr-abilitative}-bear \\
\glt But the old woman was not able to resist (couldn't help) (and told them). (The three sisters 2014, 596)
   \end{exe}

The only case of irregular abilitative is the verb \ipa{spʰɯt} `be able to cut through (of a cutting instrument)' (see example \ref{ex:mWjsphWt}) derived from \ipa{pʰɯt} `cut, pick, pluck', with the   allomorph \ipa{s--} and with more restricted semantics.

 \begin{exe}
\ex \label{ex:mWjsphWt}
\gll
\ipa{tsɯntu} 	\ipa{kɯ} 	\ipa{ɯ-ndzrɯ} 	\ipa{mɯ́j-s-pʰɯt} 	\ipa{ma} 	\ipa{ɯ-tɯ-rko} 	\ipa{ɯ-tɯ-jaʁ} 	\ipa{ɲɯ-sɤre} 	\ipa{ʑo} \\
scissors \textsc{erg} \textsc{3sg.poss}-nail \textsc{neg:sens-abilitative}-cut because \textsc{3sg-nmlz:degree}-hard \textsc{3sg-nmlz:degree}-thick \textsc{sens}-be.extremely \textsc{emph} \\
\glt Scissors cannot cut through her nails, as they are very hard and thick. (notes 2012.8.6)
   \end{exe}
   
The transitive verb \ipa{spa} `be able to (through learning)' is a lexicalized abilitative that has cognates outside of the Rgyalrong group, in particular in Tangut (\citealt[255-6]{jacques14esquisse}), showing that the abilitative must be reconstructed for a larger group of languages than simply Rgyalrong.

\subsection{Denominal derivation}
In addition to the causative and abilitative \ipa{sɯ--}, denominal verbs in \ipa{sɯ/z-/sɯɣ-/sɤ--} can be either transitive or intransitive, and belong to four semantic categories: property, position, use of an instrument or body part, and causative, as presented in table \ref{tab:denom.sa.ex}.

When a verb is derived from a possessed noun, the indefinite possessor prefixes \ipa{tɤ--} / \ipa{tɯ--} or the other possessive prefixes are not preserved, and the prefix is directly added to the nominal root  (\citealt{jacques14antipassive}). In Table \ref{tab:denom.sa.ex}, possessed nouns are indicated with the indefinite possessive prefix between brackets.
 
 \begin{table}[H] \label{tab:denom.sa.ex} \centering
 \caption{Examples of denominal verbs in \ipa{sɯ}-- and \ipa{sɤ}-- in Japhug}
 \resizebox{\columnwidth}{!}{
\begin{tabular}{llllll}
\toprule
Category &Transitivity&  Derived verb& Meaning &Base noun  & Meaning \\
\midrule
property & intr. & \ipa{sɤ-ndɤɣ} & to be poisonous & \ipa{(tɤ)-ndɤɣ} & poison \\
&intr. & \ipa{sɤ-mbrɯ} & to be angry & \ipa{(tɤ)-mbrɯ} & anger \\
position & intr. &\ipa{sɯ-ndzɯpe} & to sit (in a special way) & \ipa{ndzɯpe} & way of sitting \\
 instrument & intr. &  \ipa{sɯ-ʁejlu} & be left-handed &\ipa{ʁejlu} & left hand \\
  instrument &tr. &\ipa{sɤ-kʰɯ} & to smoke& \ipa{(tɤ)-kʰɯ} & smoke\\
  instrument &tr.& \ipa{sɯ-fsaŋ} & to perform  & \ipa{fsaŋ} & fumigation \\
&&&ritual fumigation\\
  instrument & tr. &\ipa{sɯɣ-tsʰaʁ} & to sieve & \ipa{tsʰaʁ} & sieve\\
    instrument & tr. &\ipa{sɯɣ-tsʰwi} & to dye & \ipa{tsʰwi} & colour, paint\\
causative&tr.& \ipa{sɯ-ɕtʂi}& to cause to sweat & \ipa{(tɯ)-ɕtʂi} & sweat\\
causative&tr. & \ipa{sɤ-rmi} & to give a name & \ipa{(tɤ)-rmi} & name\\
\bottomrule
\end{tabular}}
\end{table}

 Table \ref{tab:denom.sa.ex} shows that the instrumental   denominal use of \ipa{sɯ--} is fully productive, as it can be applied to Tibetan loanwords (respectively \ipa{bsaŋ} `fumigation', \ipa{tsʰag} `sieve' and \ipa{tsʰos} `paint').
 
 

\section{Historical derivation}
On the basis of the data in the previous section, we show that two grammaticalization pathways can be proposed in Japhug. First, the causative \ipa{sɯ--} was derived from the instrumental / causative denominal \ipa{sɯ--}, following a more general path of grammaticalization well-attested in Rgyalrong languages. Second, the abilitative is grammaticalized from the causative.

\subsection{From Denominal to Causative}
Previous research (\citealt{jacques14antipassive}) has shown that several valency-changing prefixes in Japhug (and in all Rgyalrong languages) are historically derived from denominal prefixes through a two-stage process. First, the base verb is nominalized to a bare infinitive, a nominal form comprising the verb root prefixed  either by an indefinite possessor prefix \ipa{tɯ--/tɤ--} or by a possessive prefix coreferent with the O in the case of transitive verbs. This nominalization neutralizes the transitivity of the verb. Then a denominal verb is created from this bare infinitive with a transitivity value different from the base verb. 

Three voice prefixes in Japhug have been shown to originate from such a grammaticalization process: the antipassive \ipa{rɤ--}, the deexperiencer \ipa{sɤ--} (on this prefix, see \citealt{jacques12demotion}) and the applicative \ipa{nɯ--}, respectively from the intransitive denominal \ipa{rɯ--/rɤ--}, the (stative verb) property denominal \ipa{sɤ--} (seen above) and the transitive denominal \ipa{nɯ--}.

This grammaticalization pathway, which is not restricted to Rgyalrong languages but also attested in language families such as Mande (\citealt{creissels12antip}) and Eskaleut (\citealt{fortescue96halftrans}),  can be summarized as follows:


\begin{exe}
\ex \label{ex:pathway}
\glt \textsc{action nominalization} of transitive verb + \textsc{intransitive denominal derivation} $\Rightarrow$ \textsc{antipassive}
\ex
\glt \textsc{action nominalization} of intransitive verb + \textsc{transitive denominal derivation} $\Rightarrow$ \textsc{applicative} / \textsc{causative}
\end{exe}


The same mechanism can explain the causative as being derived from the denominal \ipa{sɯ--}, which changes a noun X into a transitive verb meaning either `use X' (instrumental denominal) or `cause (people) to have X' (causative denominal). In addition to the phonological and semantic similarity between the causative and instrumental/causative denominal prefixes, we should note the fact that both share the same allomorphy \ipa{sɯ-- / z--/ sɯɣ--}, the  same double meaning (instrumental and causative proper) which makes it extremely unlikely that both prefixes could be unrelated.

The opposite direction of derivation (from causative to denominal)  is highly unlikely for two reasons. 

First, in the case of the antipassive \ipa{rɤ--}, there is strong evidence of the directionality of derivation from denominal to antipassive, as we find several examples of verbs whose nominal form and antipassive form share a common semantic or morphological innovation  (\citealt{jacques14antipassive}). For instance, \ipa{pɣaʁ} `turn over, plough' has an antipassive \ipa{rɤ-pɣaʁ} meaning specifically `reclaim land (plough for the first time)', with the same irregular restricted meaning as the derived noun \ipa{tɯ-pɣaʁ} `land reclamation'. While such examples have not been found in the case of the causative, the parallelism with other voice markers suggest that this direction of derivation is indeed the most likely. The denominal to causative derivation is almost a synchronic process and does not involve any reconstruction (all intermediate steps of the grammaticalization pathways are independently attested). 

Second, while the   extension of a  causative marker to  instrumental denominal function could be conceivable in an omnipredicative language where nouns  are predicative in their own right, this seems impossible in a language with a very strong noun / verb distinction like Japhug,\footnote{Verb and nouns have very different morphological properties in Japhug, and there is no zero derivation from noun to verb.} unless one explains the development of the denominal as  a backformation from the causative following the pathways in \ref{ex:pathway} in the opposite direction.

Thus, of the three logically possible historical relationships between the causative and denominal \ipa{sɯ--} (unrelated, causative to denominal, denominal to causative), only the last one is a likely explanation.


\subsection{From Causative to Abilitative}

Two distinct hypotheses can be put forward to explain the origin of the abilitative \ipa{sɯ--}: directly from the denominal \ipa{sɯ--} or indirectly from the causative \ipa{sɯ--} after its creation from the denominal. 

Although formally possible, the hypothesis that the abilitative derives from the denominal \ipa{sɯ--} is not possible on semantic grounds: there are no denominal verbs derived from a noun X whose meaning is `be able to do X'.

The derivation from the causative is also difficult at first glance, as abilitative and causative share little semantic commonality. Yet, there are cases where both an interpretation in terms of causative and one in terms of abilitative is possible and would have a very similar meaning, differing only in perspective. Example \ref{ex:mAkAsWrqoR} is an example of abilitative with the verb \ipa{sɯ-rqoʁ} `be able to hug'.


 \begin{exe}
\ex \label{ex:mAkAsWrqoR}
\gll
\ipa{tɯrme} 	\ipa{laʁnɯlaχsɯm} 	\ipa{kɯnɤ} 	\ipa{mɤ-kɤ-sɯ-rqoʁ} 	\ipa{kɯ-fse} 	\ipa{kɯ-jpum} 	\ipa{ɲɯ-βze} 	\ipa{cʰa} \\
people two.or.three also \textsc{neg-inf-abilitative}-hug \textsc{inf:stat}-be.like \textsc{nmlz:S}-be.thick \textsc{ipfv}-do[III] can\textsc{:fact} \\
\glt  (The Fir) can grow so thick that two or three people cannot hug (its trunk). (Fir, 6)
   \end{exe}

However, it is also possible to construe the meaning in a different way: `The fir can grow so thick that it prevents even two or three people from hugging (its trunk)', with a causative interpretation. This interpretation is possible due to the ambiguity of the scope of the negation of the causative, which generates the preventive meaning `prevent, hinder' in negative form, from which a modal meaning `not able to' can be derived, if the causee, rather than the causer, is reanalyzed as the real A of the \ipa{sɯ--} prefixed verb. This kind of reanalysis is particularly easy in Japhug as all of the arguments, whether S, A, O, causer or causee can be elided.

Examples such as \ref{ex:mAkAsWrqoR} therefore constitute the pivot construction whose reanalysis has allowed the abilitative to be created out of the causative. This hypothesis accounts for the fact that nearly all examples of abilitative in natural speech are found in negative verb forms. We can propose the previously unknown pathway of grammaticalization:

{\small
\begin{exe}
\ex \label{ex:pathway2}
\glt \textsc{negative}   + \textsc{causative} $\Rightarrow$ \textsc{preventive} $\Rightarrow$ \textsc{negative} +  \textsc{abilitative} 
\glt \textbf{not cause} to X $\Rightarrow$ \textbf{prevent} from X $\Rightarrow$ be \textbf{unable} to X (by removing the causer and promoting the causee to A status)
\end{exe}
}


The grammaticalization of the abilitative, as seen above, must predate he common ancestor of Rgyalrong and Tangut, but it is unclear whether the languages which have no trace of the abilitative have lost all traces of it or never have grammaticalized it.

\section{Comparative evidence}

 While from a Japhug-internal point of view the derivation from denominal \ipa{sɯ--} to causative seems straightforward, this hypothesis raises an important problem:  the sibilant causative is one of the very few morphological element that appears to be ubiquitous in the Sino-Tibetan family. Indeed, even highly innovative languages such as Chinese and Lolo-Burmese appear to present  traces of this prefix. In languages other than Rgyalrongic, the semantic of the causative cannot be studied in comparable detail as it has become completely lexicalized and is not any more the productive mechanism to express causation. In particular, the instrumental use of the causative attested in Japhug does not appear widespread outside of Rgyalrongic.
 
 In this section, we present data from Tibetan and Chinese showing that data from these languages do not contradict the above hypothesis.

 
\subsection{Tibetan}

Of all ancient Sino-Tibetan languages, Tibetan is the only one which directly preserves the causative in a form that does not require a reconstruction.\footnote{Tibetan in this paper is transcribed according to \citet{jacques12transcription}'s conventions.} There is clear evidence of both the causative and the denominal \ipa{s--} prefixes.\footnote{The causative has an allomorph \ipa{z--} before \ipa{l--}.}  Examples of the causative are plentiful. \citet[210-8]{zhang09cizu} counts 107 causative pairs in Tibetan, such as \ipa{ɴgul} `move (it)' and \ipa{sgul} `move (tr)'. Although some of the pairs collected by Zhang Jichuan must be explained differently (in particular the ones that involve \ipa{s--} / \ipa{z--} alternations such as \ipa{sub} `rub off', \ipa{zub} `be rubbed off'), there are about a hundred of good examples of causatives in \ipa{s--} in Tibetan.

Clear examples of the denominal \ipa{s--} are rarer and are generally less transparent semantically (for instance \ipa{ŋag} `word' $\Rightarrow$ \ipa{bsŋags} `extoll'), suggesting that productivity was lost before that of the causative.\footnote{We also find one example of a denominal stative property verb \ipa{sɲan} `pleasant (speech), melodious' from \ipa{ɲan} `hear'. }

Yet, there is evidence also in Tibetan that the causative derives from the denominal.  In almost all of these pairs, the intransitive counterpart has a prenasalized prefix \ipa{ɴ--} which is not usually commented on by comparativists. There is a frequent \ipa{ɴ--} present tense prefix appearing in intransitive verbs in Tibetan, but in the causative pairs, the  \ipa{ɴ--} in the intransitive forms is in most cases not a tense marker: it is retained through the whole paradigm, and appears in both present and past stems.

 In some cases, we find a cognate noun that does not have the  \ipa{ɴ--} prefixal element, as in \ipa{grib.ma} `shade, shadow'  and \ipa{grib} `defilement, stain' versus \ipa{ɴgrib} `diminish, fail, be obstructed, be obscured' and  \ipa{sgrib} `cover'. Rather than assuming, as is generally done, a direct derivation from intransitive  \ipa{ɴgrib}  to the transitive controllable \ipa{sgrib}, it is better for both semantic and morphological reasons to suppose that both verbs derive from the base noun, whose original meaning was `shadow' (hence the secondary evolution to `defilement, stain'), by addition of the denominal intransitive \ipa{ɴ--} and denominal causative \ipa{s--} prefixes:
 
\begin{exe}
\ex \label{exgrib}
\glt  \ipa{grib(-ma)} `shade, shadow'  $\Rightarrow$   \ipa{ɴ-grib} `fail, be obscured'
\glt 
 \ipa{grib(-ma)} `shade, shadow'  $\Rightarrow$   \ipa{s-grib} `cover'
\end{exe}

The intransitive \ipa{ɴgrib} has retained all the meanings of the original noun, and developed some more meanings, while the transitive \ipa{sgrib} `cover' has only retained the base meaning of the noun.  

Though in most cases no corresponding noun is found, it is likely that other examples of  `causative \ipa{s--}' should be in fact historically analyzed as denominal prefixes. This question is deferred to further research, which will require corpus study of Old Tibetan texts.


\subsection{Old Chinese}
Already in antiquity, Chinese languages were phonologically and morphologically much more innovative than modern languages such as Rgyalrong or Kiranti. The remnants of former morphological alternations directly attested in modern varieties of Chinese are few and ambiguous, and can only be accessed through reconstructions. 

Not all authors agree about how to interpret and reconstruct the traces of morphological alternations found in Chinese. In particular, an important debate concerns verbs pairs presenting a voicing alternation correlated with transitivity in Middle Chinese as presented in Table \ref{tab:mc.voicing}.

\begin{table}[h]
\caption{Examples of voicing alternation in Middle Chinese}\label{tab:mc.voicing} \centering
\begin{tabular}{lllllllll} \toprule
Intransitive &Meaning & Transitive&Meaning \\
\midrule
 \zh{現} \ipa{ɣen³}  & appear &  \zh{見} \ipa{ken³}  & see \\
 \zh{敗} \ipa{bæj³}  & be defeated &  \zh{敗} \ipa{pæj³}  & defeat \\
  \zh{别} \ipa{bjet}  & be different, leave &  \zh{别} \ipa{pjet}  & separate \\
    \zh{折} \ipa{dʑet}  & break, bend (it) &  \zh{折} \ipa{tɕet}  & break, bend (tr) \\
\bottomrule
\end{tabular}
\end{table}
Some scholars believe that this type of voicing alternation\footnote{In the case of velars, there was no non-palatalized \ipa{g} in Middle Chinese, so that the alternation  \ipa{ɣ} / \ipa{k} is expected, and all specialists of Old Chinese reconstruction agree that Middle Chinese \ipa{ɣ}-- originates from *\ipa{g}--, see \citet{baxter92}.} is a trace of the cognate of the causative *\ipa{s--} prefix, and that the transitive verbs derive from the intransitive ones, the *\ipa{s-} prefix having a devoicing effect on the initial consonant (\citealt{mei12caus} is a representative example of this line of thought). 

However, it is clear that this view is a misconception. In all phonologically conservative languages where the causative is preserved as a distinct segment, we also find traces of a distinct and historically unrelated voice alternation: anticausative prenasalization. Table \ref{tab:anticausative} shows some examples of verb pairs in Japhug (see \citealt{jacques12demotion, jacques15spontaneous} for more details on the semantics of this derivation and for more examples). The anticausative prenasalization changes transitive verbs to intransitive with voicing of the initial stop or affricates (there are no examples of this alternation with verbs having sonorant or fricative initials). The directionality (from transitive to intransitive) is proven by two pieces of evidence.

First, the Tibetan loanword \ipa{χtɤr} `spill' (Tibetan \ipa{gtor}), whose intransitive counterpart  \ipa{ʁndɤr} `be spilled' has no equivalent in Tibetan. Moreover the phonotactics of the cluster fricative+prenasalized voiced stop is incompatible with the phonotactic structure of Tibetic languages. Hence, this verb can only have been created within Japhug from its transitive counterpart \ipa{χtɤr} `scatter' after this latter had been borrowed.

Second, there are transitive verbs with aspirated or unaspirated obstruents, but this contrast is neutralized in the corresponding anticausative verbs.


 \begin{table}[h]
\caption{Examples of the anticausative alternation in Japhug}\label{tab:anticausative} \centering
\resizebox{\columnwidth}{!}{
\begin{tabular}{lllllllll} \toprule
Base   verb &Meaning &Derived verb &Meaning \\
transitive & &intransitive\\
\midrule
   \ipa{χtɤr} & spill & \ipa{ʁndɤr} & be spilled \\
      \ipa{prɤt} & break & \ipa{mbrɤt} & break, cut (it) \\
   \ipa{tɕɣaʁ} & squeeze out & \ipa{ndʑɣaʁ} & be squeezed out (spontaneously) \\
   \ipa{pɣaʁ} & turn over &    \ipa{mbɣaʁ} & roll about (it) \\
     \ipa{xtʰom}  &	 to put horizontally	&		\ipa{ndom}  &	 	to be horizontal 	\\
   \ipa{cʰɤβ}  &	 to flatten, to crush 	&		\ipa{ɲɟɤβ}  &	to be crushed, flattened 	 	\\ 
   \ipa{cɯ}  &	 to open 	&		\ipa{ɲɟɯ}  &	 to be opened	 	\\ 
\bottomrule
\end{tabular}}
\end{table}


Similar pairs can be found in languages such as Tibetan (\citealt{jacques12internal, hill14voicing}), Tangut (\citealt{gong88alternations}, \citealt[245-8]{jacques14esquisse}) or Jinghpo (\citealt[78]{dai90yufa}), which preserve the causative prefix as a distinct segment (\ipa{s--} in Tibetan and \ipa{ɕə--} or \ipa{tɕə--} in Jinghpo) or as a suprasegmental feature unrelated with voicing (\citealt{gong99jinyuanyin}, \citealt[250-1]{jacques14esquisse}). Since causative and anticausative derivations are clearly distinct in Tibetan, Rgyalrong, Tangut and Jinghpo, it is not possible that the verb pairs in Chinese such as those presented in Table \ref{tab:mc.voicing} can be explained as being traces of a causative prefix *\ipa{s--} (see also \citealt{lapolla03}).\footnote{It is possible however, that in highly eroded languages like Lolo-Burmese, ancient *S+voiced obstruent clusters have become unvoiced, as specialists of these languages generally believe (\citealt{bradley79, gerner07caus}), so that distinguishing between anticausative and causative pairs is not straightforward on Lolo-Burmese internal grounds. Some verb pairs found in Lolo-Burmese (for instance Burmese \ipa{prat} vs \ipa{phrat} `break') are also attested in Japhug (\ipa{mbrɤt} `break (it)' vs \ipa{prɤt} `break, cut (tr)'), showing that at least part of these voicing or aspiration alternations originate from anticausative prenasalization, not from causative *\ipa{s--}.  } It is more likely to assume, following  \citet{sagart12sprefix}, that the voicing alternation in Chinese is cognate to the anticausative derivation.

While \citet{sagart12sprefix} deny that verb pairs such as those in Table \ref{tab:mc.voicing} are traces of the causative *\ipa{s--} prefix, they still reconstruct a causative *\ipa{s--} to account for different alternations (Table \ref{tab:caus.oc}).

   \begin{table}[h]
\caption{Examples of the causative *\ipa{s--} in Old Chinese according to \citet{sagart12sprefix}}\label{tab:caus.oc} \centering
\begin{tabular}{llllll}
\toprule
  &Middle Chinese  &Old Chinese &Meaning \\
  \midrule
  \zh{視}& \ipa{dʑij²} &$\leftarrow$ *\ipa{gijʔ}  & see \\
\zh{示}& \ipa{ʑij³} &$\leftarrow$ *\ipa{s-gijʔ-s}  & show \\
\zh{食}& \ipa{ʑik} &$\leftarrow$ *\ipa{mə-lək}  & eat \\
\zh{飼}& \ipa{zi³} &$\leftarrow$ *\ipa{s-m-lək-s}  & feed \\
\bottomrule
\end{tabular}
\end{table}

It should be noted however that in nearly all the verb pairs provided by \citealt{sagart99roc} and \citet{sagart12sprefix}, the causative is actually indicated by a circumfix *\ipa{s-- --s}, with the verb prefixed and suffixed by *\ipa{s}. The reconstruction of a causative *\ipa{s--} prefix is thus by no means straightforward. While all authors agree on the existence of examples, there are barely any example of a verb pair that is agreed on by all scholars.

 By contrast, we do find examples of denominal *\ipa{s--} that are accepted by all specialists, in the case of pairs between Middle Chinese \ipa{l} (from Old Chinese *\ipa{r--}) and \ipa{ʂ-} (from Old Chinese *\ipa{sr--}). Table \ref{tab:denom.oc} includes two of the most convincing pairs. Other pairs have been proposed, but their acceptance crucially depends on one's particular Old Chinese reconstruction system, and a detailed discussion goes beyond the scope of the present paper.  
  
  \begin{table}[h]
\caption{Examples of the denominal *\ipa{s--} in Old Chinese}\label{tab:denom.oc} \centering
\begin{tabular}{llllll}
\toprule 
  &Middle Chinese  &Old Chinese &Meaning \\
  \midrule
\zh{吏}& \ipa{li³} &$\leftarrow$ *\ipa{rəʔ-s}  & officer \\
\zh{使}& \ipa{ʂi²} &$\leftarrow$ *\ipa{s-rəʔ} &send  \\
\zh{率}& \ipa{lwit} &$\leftarrow$ *\ipa{rut} & norm, standard \\
\zh{率} &\ipa{ʂwit} &$\leftarrow$ *\ipa{s-rut} & follow, go along  \\
\bottomrule
\end{tabular}
\end{table}
 

 In conclusion, the only uncontroversial fact about Old Chinese morphology is the existence of a denominal *\ipa{s--} prefix, whose exact semantics is unclear due to the dearth of examples. It is possible that a causative *\ipa{s--} prefix can be reconstructed, but the evidence is less clear and allows differing interpretations.
  
\section{Conclusion}

This paper   provides the first detailed description of the two causative derivations in Japhug Rgyalrong, and in addition  proposes two new pathways of grammaticalization.

 First, it shows that the \ipa{sɯ--} causative in Japhug Rgyalrong is derived from the denominal instrumental  / causative denominal derivation (X $\Rightarrow$ `use X' or `cause to have X') through a two-step process already attested for antipassive and applicative derivations (\citealt{jacques14antipassive}). Second, it   suggests that the abilitative \ipa{sɯ--} prefix evolved from the causative through reanalysis of the causee as the agent in negative forms, following pathway \ref{ex:pathway3}.
 
{\small
\begin{exe}
\ex \label{ex:pathway3}
\glt \textsc{negative}   + \textsc{causative} $\Rightarrow$ \textsc{preventive} $\Rightarrow$ \textsc{negative} +  \textsc{abilitative} 
\end{exe}
}

The first pathway has considerable implications for this family as a whole: both denominal and causative sibilant prefixes are found across the Sino-Tibetan family. The hypothesis proposed here implies either that only the denominal derivation is reconstructible to proto-Sino-Tibetan (and that the causative has been innovated independently several times from the denominal prefix throughout the family) or that the grammaticalization took place in proto-Sino-Tibetan times.

We have shown that  some apparent examples of causative \ipa{s--} in Tibetan are better analyzed as denominal verbs, suggesting that the reanalysis from denominal to causative was still ongoing in Old Tibetan times. In the case of Chinese, the dearth of convincing examples of causative *\ipa{s--} possibly implies that it has never developed a real causative prefix.

\bibliographystyle{unified}
\bibliography{bibliogj}
\end{document}