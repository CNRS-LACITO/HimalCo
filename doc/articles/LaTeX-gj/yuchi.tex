
\documentclass[oneside,a4paper,11pt]{article} 
\usepackage{fontspec}
\usepackage{natbib}
\usepackage{booktabs}
\usepackage{xltxtra} 
\usepackage{polyglossia} 
\usepackage[table]{xcolor}
\usepackage{gb4e} 
\usepackage{multicol}
\usepackage{graphicx}
\usepackage{float}
\usepackage{textcomp}
\usepackage{hyperref} 
\hypersetup{bookmarks=false,bookmarksnumbered,bookmarksopenlevel=5,bookmarksdepth=5,xetex,colorlinks=true,linkcolor=blue,citecolor=blue}
\usepackage[all]{hypcap}
\usepackage{memhfixc}
\usepackage{lscape}

%\setmainfont[Mapping=tex-text,Numbers=OldStyle,Ligatures=Common]{Charis SIL} 
\newfontfamily\phon[Mapping=tex-text,Ligatures=Common,Scale=MatchLowercase,FakeSlant=0.3]{Charis SIL} 
\newcommand{\ipa}[1]{{\phon#1}} %API tjs en italique
 
\newcommand{\grise}[1]{\cellcolor{lightgray}\textbf{#1}}
\newfontfamily\cn[Mapping=tex-text,Ligatures=Common,Scale=MatchUppercase]{MingLiU}%pour le chinois
\newcommand{\zh}[1]{{\cn #1}}
\newcommand{\topic}{\textsc{dem}}
\newcommand{\tete}{\textsuperscript{\textsc{head}}}
\newcommand{\rc}{\textsubscript{\textsc{rc}}}
\XeTeXlinebreaklocale 'zh' %使用中文换行
\XeTeXlinebreakskip = 0pt plus 1pt %
 %CIRCG
\newcommand{\ro}{$\Sigma$}
\newcommand{\siga}{$\Sigma_1$} 
\newcommand{\sigc}{$\Sigma_3$}   
\newfontfamily\mleccha[Mapping=tex-text,Ligatures=Common,Scale=MatchLowercase]{Galatia SIL}%pour le grec
\newcommand{\grec}[1]{{\mleccha #1}}


\begin{document} 
\title{Yuchi and Siouan}
\author{Guillaume Jacques}
\maketitle

Comparative Siouan Dictionary (\citealt{rankin15csd}). 

Table adapted from \citet{ranking98yuchi}
  
  \citet[325]{wagner38yuchi}
  \begin{table}[H]
  \caption{Comparison of personal prefixes in proto-Siouan, Catawba and Yuchi}
\begin{tabular}{llllllllll}
\toprule
&Proto-Siouan4  & 	Catawba  & 	Yuchi  & 	  & 	  \\ 
  & 	  && 	set I  & 	set II   \\ 
  \midrule
1s.A & 	\ipa{wa-}  & 	\ipa{dV-}  & 	<di> \ipa{di-}  & 	<do> \ipa{do-}  \\ 
2s.A  & 	\ipa{*ya-/ra-}  & 	\ipa{ya-}  & 	<nɛ> \ipa{nɛ̃-}  & <yo>	\ipa{jo-}  \\ 
\midrule
1s.P  & 	\ipa{wį-}  & 	\ipa{ni-}  & 	<tsɛ> \ipa{tsɛ-}  & 	<dzio> \ipa{dzio-}  \\ 
2s.P  & 	\ipa{*rį-/yį-}  & 	\ipa{yV-}  & <nɛndzɛ>	\ipa{nɛ̃dzɛ-}  & <nɛndzio>	\ipa{nɛ̃dzio-}  \\ 
\midrule
1i  & 	\ipa{ʔų(k)-}  & 	\ipa{ha-}  & 	<ɔ̨>\ipa{ʔɔ̃-}  & <ɔndzɛ>	\ipa{ʔɔ̃dzɛ-}  \\ 
1e  & 	\ipa{rų-}  & 	\ipa{nǫ-}  & <nɔ̨>	\ipa{nɔ̃-}  & 	 <nɔndzɛ> \ipa{nɔ̃dzɛ-}  \\ 
\bottomrule
\end{tabular}
\end{table}
\ipa{di}
\ipa{tsɛ-}

<ɔ̨dí>

<nɔ̨dí>


i-- hi-- instrumental


go--
wi--

 \citet{wagner38yuchi}
\citet{wagner31tales}
\citet{ullrich08}
\citet{linn01euchee}
\citet{taylor76motion}
\citet{rankin95ablaut}
\citet{ranking98yuchi}
\citet{csd2006}
\citet{ballard75yuchi}
\citet{koontz83syncopating}
\citet{chafe76macro}
\citet{elmendorf64}

\citet[334-5]{wagner38yuchi}
 documents thirteen irregular verbs, which include some of the most convincing Yuchi-Siouan cognates
\begin{enumerate}
\item   \ipa{á-go-ˀę} ‘to think’ (1sg \ipa{-ts’ę}, 2sg \ipa{-ˀyǫ}, 3sg \ipa{-ˀę}) / Siouan *\ipa{ˀį} ‘id.’, 
\item \ipa{go-ła} ‘to go’ (1sg \ipa{-da}, 2sg \ipa{-ca}, 3sg \ipa{-ła}) / Siouan *\ipa{re}˙‘id.’, 
\item \ipa{go-ˀǫ́} ‘to be here’ (1sg \ipa{-tʰǫ}, 2sg \ipa{-yǫ}, 3sg \ipa{-ˀǫ}) / Siouan *\ipa{ˀų} ‘id.’, 
\item \ipa{nɛhɛ́ˀę-go-łi} ‘to arrive’ (1sg \ipa{-dzi}, 2sg \ipa{-ši}, 3sg \ipa{-łi}) / Siouan *\ipa{rhi} ‘id’. 
\end{enumerate}  

Three other irregular verbs have potential Siouan cognates: \ipa{kɛ́ˀę-go-ła} ‘to do’ (1sg \ipa{-ša}, 2sg \ipa{-ša}, 3sg \ipa{-ła}) / Siouan causative *(r)e- (but this idea is problematic as *-r- is generally considered to be epenthetic in this suffix), \ipa{k’alágo-ła} ‘to eat’ (1sg \ipa{-da}, 2sg \ipa{-ša}, 3sg \ipa{-ła}) / Siouan *\ipa{rú˙te}, and Yuchi go-twáˀ ‘to kill’ (1sg -tswaˀ, 2sg –tšwaˀ, 3sg -twáˀ) / Siouan *kité ‘to kill’, derived from *tˀe˙ ‘to die’. 

Besides, some regular Yuchi verbs correspond to Siouan athematic verbs: \ipa{go-kˀǫ́} ‘to make’ (Wagner 1938:364), Siouan *\ipa{ˀų} ‘to use’ (glottal stop stem) and \ipa{–xpʰǫ} ‘yell’ (p.328) / Siouan *\ipa{pąhé} ‘to call’ (p-stem in Dhegiha).   
%  Ngandi Ritharrngu
%  Borrowing of irregular morpho: \citet{harvey11lexical} \citet{heath78arnhem}
  
 to dance	dicTi`	_ʃtʰi					*wa_čhí < *ri•h-ší	to dance
to cut open 	dop'a'	p'aˀ					pa-	by cutting


\begin{table}
\begin{tabular}{llllllllll}
	\ipa{} &	to miss & 	\ipa{di`kiɬɔ~} &	\ipa{_sna (woʂna)} &	328 & 	198 & 	\\
	\ipa{} &	to crush & 	\ipa{dokasa`} &	\ipa{_ksA} &	329 & 	 & 	\\
	\ipa{cPi} &	wet & 	\ipa{ʃpʰi} &	\ipa{spayA} &	344 & 	 & 	\\
	\ipa{di} &	yellow & 	\ipa{di} &	\ipa{zi} &	344 & 	 & 	\\
	\ipa{isPi`} &	black & 	\ipa{ispʰí} &	\ipa{sápa} &	345 & 	 & 	\\
	\ipa{cpa} &	blackberry & 	\ipa{ʃpa} &	\ipa{} &	 & 	 & 	\\
	\ipa{Ka`x.Ka} &	white & 	\ipa{kʰakʰá} &	\ipa{ská} &	345 & 	 & 	\\
	\ipa{nɔ~`wɛ} &	two & 	\ipa{nõwɛ} &	\ipa{} &	346 & 	 & 	\\
	\ipa{ya} &	tree & 	\ipa{ja} &	\ipa{chaN} &	317 & 	 & 	\\
	\ipa{wi'i} &	blood & 	\ipa{wiʔi} &	\ipa{*(wa-)ʔí•(-re); O wamį́ < *waWį́; L *wé H *í•re} &	 & 	4 & 	\\
	\ipa{wɛ`nt'e} &	woman & 	\ipa{wẽt'e} &	\ipa{*wĩ} &	 & 	54 & 	\\
	\ipa{to} &	potato & 	\ipa{to} &	\ipa{*Ro, *wRo} &	316 & 	 & 	\\
	\ipa{co} &	body & 	\ipa{ʃo} &	\ipa{*yo} &	316 & 	 & 	\\
	\ipa{Ti} &	name & 	\ipa{tʰi} &	\ipa{*RatE} &	316 & 	 & 	\\
	\ipa{Ta} &	face & 	\ipa{tʰa} &	\ipa{*iitE} &	316 & 	 & 	\\
	\ipa{wɛdi} &	buffalo & 	\ipa{wɛdi} &	\ipa{*pte} &	 & 	160 & 	\\
	\ipa{dito`} &	head & 	\ipa{to} &	\ipa{*rą•tų́} &	338 & 	 & 	\\
	\ipa{dac'i`} &	mouth & 	\ipa{da-} &	\ipa{*ra-} &	309 & 	 & 	\\
	\ipa{Tɛti`} &	root & 	\ipa{tʰɛtí} &	\ipa{hute} &	318 & 	 & 	\\
	\ipa{TaPi`} &	salt & 	\ipa{tʰapʰí} &	\ipa{pha} &	318 & 	 & 	\\
	\ipa{hɔ~dju`b'a} &	ear & 	\ipa{dʒúba} &	\ipa{*rą́•tpa} &	 & 	44 & 	\\


\end{tabular}
\end{table}

\bibliographystyle{unified}
\bibliography{bibliogj}

\end{document}