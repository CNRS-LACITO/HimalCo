\documentclass[oneside,a4paper,11pt]{article} 
\usepackage{fontspec}
\usepackage{natbib}
\usepackage{booktabs}
\usepackage{xltxtra} 
\usepackage{polyglossia} 
\usepackage[table]{xcolor}
\usepackage{gb4e} 
\usepackage{multicol}
\usepackage{graphicx}
\usepackage{float}
\usepackage{textcomp}
\usepackage{hyperref} 
\hypersetup{bookmarks=false,bookmarksnumbered,bookmarksopenlevel=5,bookmarksdepth=5,xetex,colorlinks=true,linkcolor=blue,citecolor=blue}
\usepackage[all]{hypcap}
\usepackage{memhfixc}
\usepackage{lscape}

%\setmainfont[Mapping=tex-text,Numbers=OldStyle,Ligatures=Common]{Charis SIL} 
\newfontfamily\phon[Mapping=tex-text,Ligatures=Common,Scale=MatchLowercase,FakeSlant=0.3]{Charis SIL} 
\newcommand{\ipa}[1]{{\phon#1}} %API tjs en italique
\newcommand{\yuchi}[1]{}  %{<{\phon#1}>} 
\newcommand{\grise}[1]{\cellcolor{lightgray}\textbf{#1}}
\newfontfamily\cn[Mapping=tex-text,Ligatures=Common,Scale=MatchUppercase]{MingLiU}%pour le chinois
\newcommand{\zh}[1]{{\cn #1}}
\newcommand{\topic}{\textsc{dem}}
\newcommand{\tete}{\textsuperscript{\textsc{head}}}
\newcommand{\rc}{\textsubscript{\textsc{rc}}}
\XeTeXlinebreaklocale 'zh' %使用中文换行
\XeTeXlinebreakskip = 0pt plus 1pt %
 %CIRCG
\newcommand{\ro}{$\Sigma$}
\newcommand{\siga}{$\Sigma_1$} 
\newcommand{\sigc}{$\Sigma_3$}   
%\newfontfamily\mleccha[Mapping=tex-text,Ligatures=Common,Scale=MatchLowercase]{Galatia SIL}%pour le grec



\begin{document} 
\title{Yuchi and Siouan}
\author{Guillaume Jacques}
\maketitle

\section{Introduction}
Comparative Siouan Dictionary (\citealt{rankin15csd}). 

\section{Cognate verbs}

While related words between Siouan and Yuchi are few, it is significant that the clearest comparisons involve irregular verbs, which are unlikely to be recently borrowed or derived from ideophones. \citet[334-5]{wagner38yuchi} documents thirteen irregular verbs, among which seven have Siouan potential cognates. The following list presents the infinitive stem / third person stem of these Yuchi verbs. Despite its brevity, it shows recurrent correspondences between Yuchi \ipa{ł} and \ipa{ˀ} and PS *\ipa{r} and *\ipa{ˀ}. The list includes a comparison with an instrumental prefix in Siouan, which are known to correspond to verbs in Catawba (\citealt{rankin98proto}). Note that the verbs reconstructed with *\ipa{e} in PS in this list have \ipa{a}/\ipa{e} ablaut, so that the comparison with Yuchi is \ipa{a} is unproblematic.

\begin{enumerate}
\item   \ipa{-ˀę} ‘to think’,  PS *\ipa{ˀį} ‘id.’, 
\item \ipa{-ła} ‘to go’,  PS *\ipa{re}˙‘id.’, 
\item \ipa{-ˀǫ́} ‘to be here’,  PS *\ipa{ˀų} ‘id.’, 
\item \ipa{-łi} ‘to arrive’,  PS *\ipa{rhi} ‘id’. 
\item \ipa{-ła} ‘to eat’,  PS *\ipa{ra-}˙‘with the mouth (instrumental prefix)’.
\item \ipa{-ła} ‘to do’, PS *\ipa{(r)e-} `causative'.
\item \ipa{-tʰɛdɛ} ‘to hit, to beat’,  PS *\ipa{hkú•te} `shoot' or *\ipa{kité} `kill'
\end{enumerate}  
The last two comparisons are not completely straightforward. The *\ipa{r} in the causative suffix *\ipa{(r)e} in Siouan is perhaps of epenthetic origin, as it does not undergo the same alternations as regular *\ipa{r-} initial verbs. As for \ipa{-tʰɛdɛ} ‘to hit, to beat', the comparison is between the first syllable of the Yuchi verb and the last syllable of the Siouan dissyllabic verbs.

Other comparisons include Yuchi regular verbs corresponding to Siouan athematic verbs, such as:
\begin{enumerate}
\item   \ipa{-kˀǫ́} ‘to make’ (\citealt[364]{wagner38yuchi}), PS *\ipa{ˀų} ‘to use’ (glottal stop stem)
\item \ipa{-pǫ} ‘call, yell’ (\citealt[320, 328]{wagner38yuchi}, PS *\ipa{pąhé} ‘to call’ (p-stem in Dhegiha, \citealt{rankin05quapaw}).   
\end{enumerate}  

In addition to \ipa{-ła} ‘to eat’, other Yuchi verbs may correspond to Siouan instrumental prefix, in particular  \ipa{-pʰa} `chop' (\citealt[329]{wagner38yuchi}, \citealt[176]{crawford73yuchi}), PS *\ipa{wa-} `by cutting (instrumental prefix)'.

Even if uncertain cases (Yuchi `do', `hit', `chop') cited above are discarded, there is little doubt that Yuchi and Siouan share verbal etyma belonging to the core vocabulary. The comparisons cited above do not exhaust all verbs shared between Yuchi and Siouan. Verb pairs that include neither a Yuchi irregular verb or an syncopating Siouan verb include for instance \ipa{-ʃtʰi} `dance' (\citealt[354]{wagner38yuchi}), PS *\ipa{*ri•h-ší} `dance' (Lakota \ipa{wa\_čhí}).

A complete investigation of all related words between Yuchi and Siouan is not possible until more lexical material becomes available on Yuchi, but many more comparanda are likely to be discovered when the sound laws progressively become better understood.

\section{A Comparison of Siouan, Catawba and Yuchi paradigms}
The best evidence for a genetic relationship between Siouan-Catawba and Yuchi comes however not from the lexicon, but rather from personal indexation morphology.

\subsection{Siouan}
Siouan languages have active/stative person indexation, and thus have two series of person indexation prefixes, respectively for A/S_A and P/S_P arguments. Only first singular and second person prefixes are discussed here; the only other uncontroversial person indexation prefix, the dual inclusive \ipa{*ʔũ(k)-}, present straightforward correspondences.

The first singular and second prefixes present unexplained irregular correspondences across Siouan languages. 

For the \textsc{1sg} active, most languages have forms going back to \ipa{*wa-}, but Hochank/Chiwere and Dhegiha (represented by Omaha) have forms whose direct ancestral forms were *\ipa{ha-} and *\ipa{a-} respectively. 

The initial consonant of second person forms goes back to *\ipa{y} in some languages (Biloxi, Ofo), to *\ipa{r} in others (Lakota, Dhegiha) and is ambiguous between \ipa{*r} and \ipa{*y} in the rest of the languages (Crow, Mandan, Winnebago).

Accounting these irregularities is a complex topic that may have to do with inter-paradigmatic analogical leveling within each language, and will have to be treated in another paper.

\begin{table}[H]
\caption{Person indexation prefixes in representative Siouan languages} \label{tab:siouan}
\begin{tabular}{llllllllllllllll}
\toprule
 & 	Proto-Siouan & 	Crow & 	Lakhota & 	Hochank & 	Omaha & 	Biloxi & 	Ofo & 	\\
 \midrule
1sg.A & 	\ipa{*wa-/*a-/*ha-} & 	\ipa{baa-} & 	\ipa{wa-} & 	\ipa{ha-} & 	\ipa{a-} & 	\ipa{/} & 	\ipa{ba-} & 	\\
2.A & 	\ipa{*ra-/ya-} & 	\ipa{daa-} & 	\ipa{ya-} & 	\ipa{ra-} & 	\ipa{ða-} & 	\ipa{ya-} & 	\ipa{ča- } & 	\\
1sg.P & 	\ipa{*wĩ-/*wã-} & 	\ipa{bii-} & 	\ipa{mi-/
ma-} & 	\ipa{hį-} & 	\ipa{ą-} & 	\ipa{/} & 	\ipa{yi-} & 	\\
2.P & 	\ipa{*rį-/yį-} & 	\ipa{dii-} & 	\ipa{ni-} & 	\ipa{nį-} & 	\ipa{ði-} & 	\ipa{yi-} & 	\ipa{či-} & 	\\
\bottomrule
\end{tabular}
\end{table}

Some Siouan languages, in particular Mississipi Valley Siouan (Lakota, Dhegiha, Hochank/Chiwere) present other active paradigms for verb stems beginning in glottal stop, *\ipa{h}, *\ipa{r}, plain stops, *\ipa{w} and *\ipa{y}. These paradigms are best preserved in Dhegiha languages except the y-stem paradigm, which is confused with the r-stem paradigm in Dhegiha but partially kept distinct from it in Lakota (\citealt{jacques16ebde}). The w-stem paradigm is probably derived from the glottal stop stem paradigm (see \citealt[496]{rankin05quapaw}, citing a personal communication by J. Koontz).

In these paradigms (known as `syncopating paradigms' in the Siouanist literature), the first and second person prefixes lack a vowel, and interact with the verb stem. By far the best attested of the syncopating paradigms is the r-stem paradigm, illustrated here by the verb `to go' (Table \ref{tab:go}). It is productive in Dhegiha and Dakotan.

\begin{table}[H]
\caption{R-stem paradigms} \label{tab:go}
 \resizebox{\columnwidth}{!}{
\begin{tabular}{ll|l|ll|llll|ll}
\toprule
 &	 &	Lakhota &	Hochank &	Chiwere &	Omaha &	Osage &	Kansa &	Quapaw &	Ofo &	\\	
  \midrule
1sg.A &	\ipa{*wre•} &	\ipa{blÁ} &	\ipa{tée} &	\ipa{hajé} &	\ipa{bðe} &	\ipa{brée} &	\ipa{bne} &	\ipa{bde} &	\ipa{até•kna} &	\\	
2.A &	\ipa{*šre•} &	\ipa{lÁ} &	\ipa{šeré} &	\ipa{slé} &	\ipa{šne} &	\ipa{šcée} &	\ipa{hne} &	\ipa{tte} &	\ipa{šté•kna} &	\\	
base &	\ipa{*re•} &	\ipa{yÁ} &	\ipa{rée} &	\ipa{lé} &	\ipa{ðé} &	\ipa{aðée} &	\ipa{yé} &	\ipa{dé} &	\ipa{té•kna} &	\\	
\bottomrule			
\end{tabular}}
\end{table} 

The p-stem paradigm is attested only in Hochank/Chiwere and Dhegiha, but relatively common as all verbs with the instrumental prefix *\ipa{pa-} `by pushing' present these alternations (Table {tab:stops}; Hochank \ipa{wapóx} ‘poke a hole’) and Omaha \ipa{baxí} ‘shove, push, nudge’).

\begin{table}[H]
\caption{Stop stem paradigms} \label{tab:stops}\centering
\begin{tabular}{llllllll}
\toprule
 & 	proto-Siouan & 	Winnebago & 	Omaha & \\
 \midrule
1sg.A & 	\ipa{*hpa-} & 	\ipa{paapóx} & 	\ipa{ppáxi} & \\
2.A & 	\ipa{*špa-} & 	\ipa{šawapóx} & 	\ipa{špáxi} & \\
base & 	\ipa{*pa-} & 	\ipa{wapóx} & 	\ipa{baxí} & \\
\bottomrule
\end{tabular}
\end{table}

Other syncopating paradigms are more restricted, and only attested with a handful of verbs. This is the case in particular of the glottal stop paradigm (Table \ref{tab:glottal}, illustrating Omaha \ipa{ˀǫ́} `be, use, wear' and its cognates). 

\begin{table}[H]
\caption{Glottal stop paradigms} \label{tab:glottal} \centering
\begin{tabular}{llllllll}
\toprule
 & 	 & 	Lakota & 	Hochank & 	Omaha & \\
 \midrule
\textsc{1sg.A} & 	\ipa{*w-ų}   & 	\ipa{múŋ} & 	\ipa{haˀų́} & 	\ipa{mǫ} & \\
2.A & 	\ipa{*y-ˀų} or 	\ipa{*y-ų}  & 	\ipa{núŋ} & 	\ipa{šˀų́ų} & 	\ipa{žǫ} &\\ 
base & 	\ipa{*ˀų} & 	\ipa{'úŋ} & 	\ipa{ˀų́ų} & 	\ipa{ˀǫ́} & \\
\bottomrule
\end{tabular}
\end{table}

In the syncopating paradigms, the first person singular appears either as \ipa{*w}, \ipa{*p} or as preaspiration, while the second person appears either as *\ipa{y-}(Dhegiha \ipa{ž}) or as \ipa{š}, which \citet{rankin15csd} analyze as an obstruentized allomorph of *\ipa{y-} before obstruent.


The glottal stop paradigm (Table \ref{tab:glottal}) deserves a more detailed discussion here, as it will become relevant for Yuchi-Siouan comparison in section (\ref{sec:comparison}).

The forms of the three language do not match perfectly, in particular the second person form.

The Omaha forms \ipa{mǫ}, \ipa{žǫ} and \ipa{ˀǫ́} originate from \ipa{*w-ũ}, \ipa{*y-ũ} and \ipa{*ʔũ} respectively:\footnote{The nasalization of \ipa{*w} before nasal vowel is regular, see \citet{michaud-jacques12nasalite}.}: the clusters \ipa{*wʔ-} and \ipa{*yʔ} are simplified as \ipa{*w-} and \ipa{*y} respectively.

In the Lakota paradigm \ipa{úŋ}, the first person singular \ipa{múŋ} and the third person \ipa{úŋ} go back to the same forms as their Omaha counterparts, but the second person \ipa{núŋ} looks as if originating from a proto-form \ipa{*r-ũ}. The expected outcome of \ipa{*y-ũ} in Lakota would have been $\dagger$\ipa{čhúŋ}. This paradigm is reminiscent of the stative paradigm of some verbs in \ipa{i}in Lakota, where the first and second person synchronically look like they are marked by \ipa{m-} and \ipa{n-} prefixes respectively. I propose that the second person \ipa{núŋ} is made by four-part analogy on the model of such stative verbs (see Table \ref{tab:analogy.glottal}).

\begin{table}[H]
\caption{Four-part analogy in the glottal stop stem paradigm in Lakota} \label{tab:analogy.glottal} \centering
\begin{tabular}{lllllllll}
\toprule
Verb & sleep & use (expected) & use (analogical) \\
\midrule
\textsc{1sg} &\ipa{m-ištíŋme} &\ipa{m-úŋ} &\ipa{m-úŋ} &\\
2 & 	\ipa{n-ištíŋme}&$\dagger$\ipa{čhúŋ} $\Rightarrow$ &\ipa{n-úŋ} &  \\ 
base & \ipa{ištíŋme}	 &\ipa{úŋ} &\ipa{úŋ} & 	\\
\bottomrule
\end{tabular}
\end{table}

The Hochank paradigm has its first singular with \ipa{ha-} taken over from the regular paradigm. The second person \ipa{šˀų́ų}, unlike Omaha, presents the same obstruentization rule of \ipa{*y-} as before oral stops (Table \ref{tab:stops}).

At this stage, it should be stressed that the Dhegiha form \ipa{žǫ} `you are, you do' is synchronically more opaque than its Hochank counterpart \ipa{šˀų́ų} is. The glottal stop stem paradigm (and the w-stem paradigm, which is derived from it, see \citealt[496]{rankin05quapaw}) is the only syncopating paradigm where the second person form marked with \ipa{ž-}: in all other cases, we find the allomorph \ipa{š}. Thus, whileit is possible to explain the form \ipa{šˀų́ų} in Hochank as resulting from analogy with other syncopating paradigms, the same explanation is not available for Dhegiha \ipa{žǫ}. Therefore, it is justified to reconstruct PS \ipa{*y-ų} (the form suggested by Dhegiha \ipa{žǫ}) rather than \ipa{*y-ˀų} here. This implies that at the PS stage, the glottal stop was already lost before the prefixes \ipa{*w-} and \ipa{*y-}, and that itwas analogically restored in Hochank.

\subsection{Yuchi}
Yuchi, like Siouan, is an active-stative language with 


\subsection{Comparison} \label{sec:comparison}
Table adapted from \citet{ranking98yuchi} and \citet[325]{wagner38yuchi}
  \begin{table}[H]
  \caption{Comparison of personal prefixes in proto-Siouan, Catawba and Yuchi} \centering \label{tab:comparison}
\begin{tabular}{llllllllll}
\toprule
&Proto-Siouan  & 	Catawba  & 	Yuchi  & 	  & 	  \\ 
  & 	  && 	set I  & 	set II   \\ 
  \midrule
1s.A & 	\ipa{*wa-}  & 	\ipa{dV-}  & 	\yuchi{di} \ipa{di-}  & \yuchi{do} \ipa{do-} \\ 
2s.A  & 	\ipa{*ya-/ra-}  & 	\ipa{ya-}  & 	\yuchi{nɛ} \ipa{nɛ̃-}  & \yuchi{yo}	\ipa{yo-}  \\ 
\midrule
1s.P  & 	\ipa{*wį-}  & 	\ipa{ni-}  & 	\yuchi{tsɛ} \ipa{tsʰɛ-}  & 	\yuchi{dzio} \ipa{dzio-}  \\ 
2s.P  & 	\ipa{*rį-/yį-}  & 	\ipa{yV-}  & \yuchi{nɛndzɛ}	\ipa{nɛ̃dzɛ-}  & \yuchi{nɛndzio}	\ipa{nɛ̃dzio-}  \\ 
\midrule
\textsc{1incl}  & 	\ipa{*ʔų(k)-}  & 	\ipa{ha-}  & 	\yuchi{ɔ̨}\ipa{ʔɔ̃-}  & \yuchi{ɔndzɛ}	\ipa{ʔɔ̃dzɛ-}  \\ 
%\textsc{1excl}  & 	\ipa{*rų-}  & 	\ipa{nǫ-}  & \yuchi{nɔ̨}	\ipa{nɔ̃-}  & 	 \yuchi{nɔndzɛ} \ipa{nɔ̃dzɛ-}  \\ 
\bottomrule
\end{tabular}
\end{table}



\section{Conclusion}
%\ipa{di}
%\ipa{tsɛ-}
%
%<ɔ̨dí>
%
%<nɔ̨dí>
%
%
%i-- hi-- instrumental
%
%
%go--
%wi--
%
% \citet{wagner38yuchi}
%\citet{wagner31tales}
%\citet{ullrich08}
%\citet{linn01euchee}
%\citet{taylor76motion}
%\citet{rankin95ablaut}
%\citet{ranking98yuchi}
%\citet{csd2006}
%\citet{ballard75yuchi}
%\citet{koontz83syncopating}
%\citet{chafe76macro}
%\citet{elmendorf64}
%
%\begin{enumerate}
%\item   \ipa{á-go-ˀę} ‘to think’ (1sg \ipa{-ts’ę}, 2sg \ipa{-ˀyǫ}, 3sg \ipa{-ˀę}) / Siouan *\ipa{ˀį} ‘id.’, 
%\item \ipa{go-ła} ‘to go’ (1sg \ipa{-da}, 2sg \ipa{-ca}, 3sg \ipa{-ła}) / Siouan *\ipa{re}˙‘id.’, 
%\item \ipa{go-ˀǫ́} ‘to be here’ (1sg \ipa{-tʰǫ}, 2sg \ipa{-yǫ}, 3sg \ipa{-ˀǫ}) / Siouan *\ipa{ˀų} ‘id.’, 
%\item \ipa{nɛhɛ́ˀę-go-łi} ‘to arrive’ (1sg \ipa{-dzi}, 2sg \ipa{-ši}, 3sg \ipa{-łi}) / Siouan *\ipa{rhi} ‘id’. 
%\end{enumerate}  
%
%Three other irregular verbs have potential Siouan cognates: \ipa{kɛ́ˀę-go-ła} ‘to do’ (1sg \ipa{-ša}, 2sg \ipa{-ša}, 3sg \ipa{-ła}) / Siouan causative *(r)e- (but this idea is problematic as *-r- is generally considered to be epenthetic in this suffix), \ipa{k’alágo-ła} ‘to eat’ (1sg \ipa{-da}, 2sg \ipa{-ša}, 3sg \ipa{-ła}) / Siouan *\ipa{rú˙te}, and Yuchi go-tʰwáˀ ‘to kill’ (1sg -tsʰwaˀ, 2sg –tšʰwaˀ, 3sg -tʰwáˀ) / Siouan *kité ‘to kill’, derived from *tˀe˙ ‘to die’. 
%



%\begin{table}
%\caption{XXXX}
%\begin{tabular}{llllllllll}
%	\ipa{} &	to miss & 	\ipa{di`kiɬɔ~} &	\ipa{_sna (woʂna)} &	328 & 	198 & 	\\
%	\ipa{} &	to crush & 	\ipa{dokasa`} &	\ipa{_ksA} &	329 & 	 & 	\\
%	\ipa{cPi} &	wet & 	\ipa{ʃpi} &	\ipa{spayA} &	344 & 	 & 	\\
%	\ipa{di} &	yellow & 	\ipa{di} &	\ipa{zi} &	344 & 	 & 	\\
%	\ipa{isPi`} &	black & 	\ipa{ispí} &	\ipa{sápa} &	345 & 	 & 	\\
%	\ipa{cpa} &	blackberry & 	\ipa{ʃpa} &	\ipa{} &	 & 	 & 	\\
%	\ipa{Ka`x.Ka} &	white & 	\ipa{kaká} &	\ipa{ská} &	345 & 	 & 	\\
%	\ipa{nɔ~`wɛ} &	two & 	\ipa{nõwɛ} &	\ipa{} &	346 & 	 & 	\\
%	\ipa{ya} &	tree & 	\ipa{ja} &	\ipa{chaN} &	317 & 	 & 	\\
%	\ipa{wi'i} &	blood & 	\ipa{wiʔi} &	\ipa{*(wa-)ʔí•(-re); O wamį́ < *waWį́; L *wé H *í•re} &	 & 	4 & 	\\
%	\ipa{wɛ`nt'e} &	woman & 	\ipa{wẽt'e} &	\ipa{*wĩ} &	 & 	54 & 	\\
%	\ipa{to} &	potato & 	\ipa{tʰo} &	\ipa{*Ro, *wRo} &	316 & 	 & 	\\
%	\ipa{co} &	body & 	\ipa{ʃo} &	\ipa{*yo} &	316 & 	 & 	\\
%	\ipa{Ti} &	name & 	\ipa{ti} &	\ipa{*RatE} &	316 & 	 & 	\\
%	\ipa{Ta} &	face & 	\ipa{ta} &	\ipa{*iitE} &	316 & 	 & 	\\
%	\ipa{wɛdi} &	buffalo & 	\ipa{wɛdi} &	\ipa{*pte} &	 & 	160 & 	\\
%	\ipa{dito`} &	head & 	\ipa{tʰo} &	\ipa{*rą•tų́} &	338 & 	 & 	\\
%	\ipa{dac'i`} &	mouth & 	\ipa{da-} &	\ipa{*ra-} &	309 & 	 & 	\\
%	\ipa{Tɛti`} &	root & 	\ipa{tɛtʰí} &	\ipa{hute} &	318 & 	 & 	\\
%	\ipa{TaPi`} &	salt & 	\ipa{tapí} &	\ipa{pha} &	318 & 	 & 	\\
%	\ipa{hɔ~dju`b'a} &	ear & 	\ipa{dʒúba} &	\ipa{*rą́•tpa} &	 & 	44 & 	\\
%\end{tabular}
%\end{table}

\bibliographystyle{unified}
\bibliography{bibliogj}

\end{document}