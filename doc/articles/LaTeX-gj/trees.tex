\documentclass[oneside,a4paper,11pt]{article} 
\usepackage{fontspec}
\usepackage{natbib}
\usepackage{booktabs}
\usepackage{xltxtra} 
\usepackage{polyglossia} 
\usepackage[table]{xcolor}
\usepackage{gb4e} 
%\usepackage{multicol}
\usepackage{hyperref} 
\hypersetup{bookmarks=false,bookmarksnumbered,bookmarksopenlevel=5,bookmarksdepth=5,xetex,colorlinks=true,linkcolor=blue,citecolor=blue}
\usepackage[all]{hypcap}
\usepackage{memhfixc}

 
%\setmainfont[Mapping=tex-text,Numbers=OldStyle,Ligatures=Common]{Charis SIL} 
\newfontfamily\phon[Mapping=tex-text,Ligatures=Common,Scale=MatchLowercase]{Charis SIL} 
\newcommand{\ipa}[1]{\textbf{{\phon\mbox{#1}}}} %API tjs en italique
 \newcommand{\ipab}[1]{{\phon \mbox{#1}}} %API tjs en italique
\newcommand{\grise}[1]{\cellcolor{lightgray}\textbf{#1}}
\newfontfamily\cn[Mapping=tex-text,Ligatures=Common,Scale=MatchUppercase]{SimSun}%pour le chinois
\newcommand{\zh}[1]{{\cn#1}}

 
\begin{document} 
\section{The Advantages of the Tree Model}
\section{Saving the Trees from the Critics}
\subsection{Not all Trees are Easy to Spot}
\subsection{Diffusability of Features}


\subsection{Limitations of Sound Correspondences to Identify Lexical Innovations}
In order to identify inherited lexical innovations and distinguish them from recent borrowings, \citet[176-8]{francois15tree} uses a fairly uncontroversial criterion: etyma whose reflexes follow regular sound correspondences are considered to be inherited. Thus, lexical innovations which 

François' approach however neglects an important factor: while regular sound correspondences is a necessary condition for analyzing forms in related languages as cognates, i.e. originating from the same etymon in their common ancestor,\footnote{Note however that cognacy is a more complex concept that is usually believed (\citealt{list16cognacy}), and that even forms originating from exactly the same etymon in the proto-language may present irregular correspondences due to analogy. } it is not a \textbf{sufficient} condition due to the existence of \textbf{undetectable borrowings} and \textbf{nativized loanwords}.  


\subsubsection{Undetectable borrowings}
There are two cases when sound changes are not informative enough to allow the researcher to discriminate between inherited word and borrowing.

First, when a form contains phonemes that remained unchanged, or nearly unchanged, from the proto-language in all daughter languages (because no sound change, or only trivial changes, affected them), there is no way to know whether it was inherited from the proto-language or whether it was borrowed at a later stage. 

Examples of this type are plentiful; we provide here three typical examples: `aluminum' in Tibetan languages, `pig' in some Algonquian languages, `tobacco' in XXX,

Amdo Tibetan \ipa{hajaŋ} `aluminum' and Lhasa \ipa{hájã} `aluminum' look like they regularly originate from a Common Tibetan form *\ipa{ha.jaŋ}.\footnote{In Amdo Tibetan, Common Tibetan \ipa{h-}, \ipa{j-}, \ipa{-a} and \ipa{-aŋ} remain unchanged (\citealt{gong16amdo}). In Lhasa Tibetan, two sound changes relevant to this form occurred: a phonological high tone developped with the initial \ipa{h-}, and \ipa{-aŋ} became nasalized \ipa{ã}.} This is of course impossible for obvious historical reasons, as aluminum came into use in Tibetan areas in the twentieth century, at a time when Amdo Tibetan and Lhasa Tibetan were already completely unintelligible. This word is generally explained (Gong Xun, p.c.) as an abbreviated form of \ipa{ha.tɕaŋ jaŋ.po} `very light', but this etymology is not transparent to native speakers of either Amdo or Lhasa Tibetan. This word has been coined only once,\footnote{We are not aware of a detailed historical research on the history of this particular word, but in any case it matters little for our demonstration whether it was first coined in Central Tibetan or in Amdo.} and was then borrowed into other Tibetan languages\footnote{In some Tibetan languages such as Cone \ipa{hæ̀jãː}, \citet[306]{jacques14cone}, there is clear evidence that the word is borrowed from Amdo Tibetan and is not native (otherwise $\dagger$\ipa{hájaː} would have been expected). } and neighboring minority languages under Tibetan influence (as for instance Japhug \ipa{χajaŋ} `aluminum').

In this case a phonetic borrowing from Amdo \ipa{hajaŋ} could only yield Lhasa Lhasa \ipa{hájã}, since \ipa{h-} only occurs in high tone in Lhasa, and since final \ipa{-ŋ} has been transphonologized as vowel nasality.\footnote{Likewise, in the case of borrowing from Lhasa into Amdo, the rhyme \ipa{-aŋ} would be the only reasonable match for Lhasa \ipa{-ã}.} 

Several Algonquian languages, share a word for `pig' (Fox \ipa{koohkooša}, Miami \ipa{koohkooša} and Cree \ipa{kôhkôs}) ultimately of Dutch origin (\citealt{goddard74dutch}, \citealt{costa13borrowing}). \citet[266]{hockett57k} pointed out that these forms must be considered to be loanwords `because of the clearly post-Columbian meaning; but if we did not have the extralinguistic information the agreement in shape (apart from M[enominee]) would lead us to reconstruct a [Proto-Central-Algonquian] prototype.'\footnote{The forms from these three languages could be regularly derived from Proto-Algonquian *\ipa{koohkooša}, a reconstruction identical to the attested Fox and Miami forms.}


%Ojibwe gookoosh
Second, when an identical form shared by several related languages contains phonemes that are the reflexes of phonemes that underwent exactly the same series of sound changes from the proto-language to all daughter languages where it is attested,  

From Biblical Hebrew \ipa{hêkāl} `palace' and Arabic \ipa{haykal} `palace', it would be possible to reconstruct a Proto-Semitic etymon *\ipa{haykal(u)}; it is however well-known that these words originate from Sumerian \ipa{é.gal} `palace', probably through Akkadian \ipa{ekallum}, and that borrowing from Akkadian took place at a time when the ancestors of Hebrew and Arabic respectively were already distinct languages.

\subsubsection{Nativization of loanwords}

\citet{aikio06nativization}

%analyzable compounds?
\subsubsection{Implications for the interpretation of François' data}
In his database of 474 innovations, \citet[177]{francois15tree} counts 236 lexical replacements and 116 lexically-specific sound changes (in other words, lexical replacements where the innovative form is related to the archaic form by some unpredictable phonetic change). A group of languages are considered to share these innovations whenever they present reflexes of these etyma following regular correspondences.

Thus, out of 352 out of 474 innovations (74\%) are potential cases of either undetectable or nativized borrowings. XXX


\bibliographystyle{unified}
\bibliography{bibliogj}

 \end{document}
 