\documentclass[oneside,a4paper,11pt]{article} 
\usepackage{fontspec}
\usepackage{natbib}
\usepackage{booktabs}
\usepackage{xltxtra} 
\usepackage{polyglossia} 
\usepackage[table]{xcolor}
\usepackage{multicol} 
\usepackage{gb4e} 
%\usepackage{multicol}
\usepackage{hyperref} 
\hypersetup{bookmarks=false,bookmarksnumbered,bookmarksopenlevel=5,bookmarksdepth=5,xetex,colorlinks=true,linkcolor=blue,citecolor=blue}
\usepackage[all]{hypcap}
\usepackage{memhfixc}

 
%\setmainfont[Mapping=tex-text,Numbers=OldStyle,Ligatures=Common]{Charis SIL} 
\newfontfamily\phon[Mapping=tex-text,Ligatures=Common,Scale=MatchLowercase]{Charis SIL} 
\newcommand{\ipa}[1]{\textbf{{\phon\mbox{#1}}}} %API tjs en italique
 \newcommand{\ipab}[1]{{\phon \mbox{#1}}} %API tjs en italique
\newcommand{\grise}[1]{\cellcolor{lightgray}\textbf{#1}}
\newfontfamily\cn[Mapping=tex-text,Ligatures=Common,Scale=MatchUppercase]{SimSun}%pour le chinois
\newcommand{\zh}[1]{{\cn#1}}

 
\begin{document} 
\section{The Advantages of the Tree Model}
\section{Saving the Trees from the Critics}
\subsection{Not all Trees are Easy to Spot}
\subsection{Diffusability of Features}
%Not all innovations are equally informative; certain types of innovations are easily diffused even across unrelated languages, or are likely to occur several times independently.
%
%Yet, attempting to weigh innovations is a difficult enterprise, and would involve subjective judgments, which is why \citet[177]{francois15tree} decided to give an equal status to all of his 474 innovations.


%Affix borrowing \citet{seifart15borrowing}, directionality (\citealt{norde09degrammaticalization})
%
%Borrowing of irregular morphology: English take, give

\subsection{Variation in the proto-language}
Languages are never completely uniform, and fieldwork linguists working on unwritten languages commonly notice that even siblings may present significant differences (for instance \citealt[29-30]{genetti07grammar}). 

While some innovations can spread quickly to the entire community (or at least to all members of a specific age-group), in other cases it is possible for two competing forms (innovative vs archaic) to remain used in the same  speech community for a considerable period of time. This is observed in particular with sporadic changes, such as irregular metatheses / dissimilation / assimilation, or item-specific analogy.

When language differentiation occurs while forms are still competing, daughter languages can inherit the competing forms; then the innovative form may eventually prevail or disappear in a non-predictable way in each daughter language. If such situation occurs, the distribution of the innovation will not be relatable to the phylogenetic tree.

The distribution of r-metathesis in the word `frost' in Germanic languages is a case in point. As shown in Table \ref{tab:frost} (data from \citealt[157]{kroonen13dict}, \citealt{richthofen1840woerterbuch, bosworth1898dictionary, stratmann1891dictionary}), the form with metathesis \ipa{forst} was in competition with the regular form \ipa{frost} in Old and Middle English and in Old Frisian, but only the latter is found in modern English and Frisian languages. Only Dutch consistently has the form with metathesis from the earliest texts (13^{th} century, \citealt[112]{vaan15dutch}) to modern dialects.

\begin{table}[h]
\caption{The distribution of r-metathesis in reflexes of the etymon `frost' in Germanic languages } \centering \label{tab:frost}
\resizebox{\columnwidth}{!}{
\begin{tabular}{llllllll}
\toprule
\multicolumn{6}{c}{Proto-Germanic \ipa{*frusta-} m./n.}  \\
\midrule
Old Norse & \ipa{frost} n. & &&Swedish & \ipa{frost} n. & \\
Old Saxon & \ipa{frost} m. & \\
Old English & \ipa{frost}, \ipa{forst} m.\grise{} &Middle E. &\ipa{frost}, \ipa{forst} \grise{} & Modern E.& \ipa{frost} \\
&&Old F. & \ipa{frost}, \ipa{forst} m.\grise{} & West Frisian &\ipa{froast}\\
&&&\grise{}&East Frisian & \ipa{fröst} \\
&&Middle D. & \ipa{vorst} m.\grise{}&Modern Dutch & \ipa{vorst} c.\grise{}& \\
&&&&German & \ipa{Frost} m.& \\
\bottomrule
\end{tabular}}
\end{table}

Whether r-metathesis in this word is an innovation of the common ancestor of Dutch, English and Frisian, or was diffused through contact after these languages became unintelligible, these data show that competition between \ipa{forst} and \ipa{frost} took centuries in English, and that the archaic form ultimately prevailed, removing any trace that the form with r-metathesis ever existed.

Thus, the distribution of a particular innovation in languages without written records can be misleading: one needs to distinguish between the onset of the innovation, its maximal spread (which would include here Dutch and some English and Frisian speakers) and its eventual fate (restricted to Dutch in this case). 

Thus, while \citet[178]{francois15tree} puts much value on \textit{lexically-specific sound changes}, arguing that they are ``strongly indicative of genealogy, because they are unlikely to diffuse across separate languages'', the example of the etymon `frost' in Germanic shows that the observable distribution of a sporadic innovation in the present may not be identical to its spread in the past. Failure to distinguish between the two will naturally introduce non-tree-like patterns in the data.


\subsection{Limitations of Sound Correspondences to Identify Lexical Innovations}
In order to identify inherited lexical innovations and distinguish them from recent borrowings, \citet[176-8]{francois15tree} uses a fairly uncontroversial criterion: etyma whose reflexes follow regular sound correspondences are considered to be inherited. Thus, whenever a common proto-form can be postulated for a particular set of words across several languages (which can thus be derived from this proto-form by the mechanical application of regular sound changes), it is considered in this model to be part of the inherited vocabulary, and can be used, if applicable, as a common innovation.

François' approach however neglects an important factor: while regular sound correspondences is a necessary condition for analyzing forms in related languages as cognates, i.e. originating from the same etymon in their common ancestor,\footnote{Note however that cognacy is a more complex concept that is usually believed (\citealt{list16cognacy}), and that even forms originating from exactly the same etymon in the proto-language may present irregular correspondences due to analogy. } it is not a \textbf{sufficient} condition due to the existence of \textbf{undetectable borrowings} and \textbf{nativized loanwords}.  


\subsubsection{Undetectable borrowings}
Sound changes are not always informative enough to allow the researcher to discriminate between inherited word and borrowing. When a form contains phonemes that remained unchanged, or nearly unchanged, from the proto-language in all daughter languages (because no sound change, or only trivial changes, affected them), there is no way to know whether it was inherited from the proto-language or whether it was borrowed at a later stage. 

This type of situation is by no means exceptional, and can be found in various language families. We present here three examples of borrowings undetectable by phonology alone: `aluminum' in Tibetan languages, `pig' in some Algonquian languages, and `palace' in Semitic. 

Amdo Tibetan \ipa{hajaŋ} `aluminum' and Lhasa \ipa{hájã} `aluminum' look like they regularly originate from a Common Tibetan form *\ipa{ha.jaŋ}.\footnote{In Amdo Tibetan, Common Tibetan \ipa{h-}, \ipa{j-}, \ipa{-a} and \ipa{-aŋ} remain unchanged (\citealt{gong16amdo}). In Lhasa Tibetan, two sound changes relevant to this form occurred: a phonological high tone developped with the initial \ipa{h-}, and \ipa{-aŋ} became nasalized \ipa{ã}.} This is of course impossible for obvious historical reasons, as aluminum came into use in Tibetan areas in the twentieth century, at a time when Amdo Tibetan and Lhasa Tibetan were already completely unintelligible. This word is generally explained (Gong Xun, p.c.) as an abbreviated form of \ipa{ha.tɕaŋ jaŋ.po} `very light', but this etymology is not transparent to native speakers of either Amdo or Lhasa Tibetan. This word has been coined only once,\footnote{We are not aware of a detailed historical research on the history of this particular word, but in any case it matters little for our demonstration whether it was first coined in Central Tibetan or in Amdo.} and was then borrowed into other Tibetan languages\footnote{In some Tibetan languages such as Cone \ipa{hæ̀jãː}, \citet[306]{jacques14cone}, there is clear evidence that the word is borrowed from Amdo Tibetan and is not native (otherwise $\dagger$\ipa{hæ̀jaː} would have been expected). } and neighboring minority languages under Tibetan influence (as for instance Japhug \ipa{χajaŋ} `aluminum').

In this case a phonetic borrowing from Amdo \ipa{hajaŋ} could only yield Lhasa Lhasa \ipa{hájã}, since \ipa{h-} only occurs in high tone in Lhasa, and since final \ipa{-ŋ} has been transphonologized as vowel nasality.\footnote{Likewise, in the case of borrowing from Lhasa into Amdo, the rhyme \ipa{-aŋ} would be the only reasonable match for Lhasa \ipa{-ã}.} 

Several Algonquian languages, share a word for `pig' (Fox \ipa{koohkooša}, Miami \ipa{koohkooša} and Cree \ipa{kôhkôs}) ultimately of Dutch origin (\citealt{goddard74dutch}, \citealt{costa13borrowing}). \citet[266]{hockett57k} pointed out that these forms must be considered to be loanwords `because of the clearly post-Columbian meaning; but if we did not have the extralinguistic information the agreement in shape (apart from M[enominee]) would lead us to reconstruct a [Proto-Central-Algonquian] prototype.' The forms from these three languages could be regularly derived from Proto-Algonquian *\ipa{koohkooša}, a reconstruction identical to the attested Fox and Miami forms.
%Ojibwe gookoosh

Semitic languages abound in common vocabulary which presents the same correspondences as inherited vocabulary, but which was diffused after the breakup of the family. For instance, from Biblical Hebrew \ipa{hêkāl} `palace' and Arabic \ipa{haykal} `palace', it would be possible to reconstruct a Proto-Semitic etymon *\ipa{haykal(u)}; it is however well-known that these words originate from Sumerian \ipa{é.gal} `palace', probably through Akkadian \ipa{ekallum}, and that borrowing from Akkadian took place at a time when the ancestors of Hebrew and Arabic respectively were already distinct languages.

Undetectable borrowings is also a pervasive phenomenon in Pama-Nyungan, where with a few exception such as the Arandic and Paman groups, most languages present too few phonological innovations to allow easy discrimination for loanwords from cognates (\citealt[46]{koch04method}).

The same situation can be observed even if latter sound changes apply to both borrowings and inherited words. Whenever borrowing takes place after the break-up of two languages, but before any diagnostic sound change occurred in either the donor or the receiver language, phonology alone is not a sufficient criterion to distinguish between inherited words and loanwords. 


A classical case is that of Persian borrowings in Armenian. As \citet[16-17]{huebschmann97armenische} put it, `in isolated cases, the Iranian and the genuine Armenian forms match each other phonetically, and the question whether borrowing [or common inheritance] has to be assumed must be decided from a non-linguistic point of view.'\footnote{`In einzelnen Fällen kann allerdings das persische und echt armenische Wort sich lautlich decken und die Frage, ob Entlehnung anzunehmen ist oder nicht, muss dann nach andern als sprachlichen Gesichtspunkten entschieden werden.'} Table \ref{tab:armenian} presents a non-exhaustive list of such words.

\begin{table}[h]
\caption{Armenian words which cannot be conclusively demonstrated to be either borrowings from Iranian or inherited words from a phonetic point of view    } \centering \label{tab:armenian}
\begin{tabular}{lllllll}
\toprule 
Armenian & Meaning & Indo-Iranian & Reference \\
\midrule 
\ipa{naw}& boat & Skt. \ipa{nau-} & \citet[16-17;201]{huebschmann97armenische},\\
&&& \citet[466;715]{martirosyan10etymological} \\
\midrule 
\ipa{mēg}& mist & Skt. \ipa{megha-},  & \citet[474]{huebschmann97armenische},\\
&&Avestan \ipa{maēɣa-}& \citet[466;715]{martirosyan10etymological} \\
\midrule 
\ipa{mēz}& urine & Skt. \ipa{meha-} & \citet[474]{huebschmann97armenische},\\
&&& \citet[466;715]{martirosyan10etymological} \\
\midrule 
\ipa{sar}& head & Skt. \ipa{śiras-} & \citet[236;489]{huebschmann97armenische},\\
&&Y.Avestan \ipa{sarah-}& \citet[571]{martirosyan10etymological} \\
\midrule 
\ipa{ayrem}& burn & Skt. \ipa{edh-} & \citet[418]{huebschmann97armenische},\\
&&& \citet[145]{martzloff16geri} \\
\bottomrule
\end{tabular}
\end{table}
The Armenian case shows that undetectable loans are not restricted to cases like those studied above, when a particular word only contains segments which have not been affected by sound changes from the proto-language to all its daughter languages. Undetectable loans are possible when a particular word is borrowed before any sound change which could affect its phonetic material occurred in either the giver or recipient language, even if numerous sound changes occurred \textit{after} borrowing took place. It is possible that post-borrowing sound changes may even remove phonetic clues which could have allowed to distinguish between loanwords and inherited words.

 We have shown clear evidence that undetectable borrowings can occur even when two language varieties are mutually unintelligible. Neglecting the distinction between inherited words and undetectable borrowings, as in François' model, amounts to losing crucial historical information.

\subsubsection{Nativization of loanwords}
In the previous section, we have discussed cases when borrowing took place before diagnostic sound changes, thus making it impossible to effectively use sound changes to distinguish between loanwords and inherited words. There is however evidence that even when diagnostic sound changes exist, they may not always be an absolutely reliable criterion.

When a particular language contains a sizeable layer of borrowings from another language, bilingual speakers can develop a intuition of the phonological correspondences between the two languages, and apply these correspondences to newly borrowed words, a phenomenon known as loan nativization.

The best documented case of loan nativization is that between Saami and Finnish (the following discussion is based on \citealt{aikio06nativization}). Finnish and Saami are only remotely related within the Finno-Ugric branch of Uralic, but  Saami has borrowed a considerable quantity of vocabulary from Finnish, some at a stage before most characteristic sound changes had taken place, other more recently. Table \ref{tab:native} presents examples of cognates between Finnish and Saami illustrating some recurrent vowel and consonant correspondences.

\begin{table}[h]
\caption{Examples of of sound correspondences in inherited words between Finnish and Saami (data from \citealt[27]{aikio06nativization})} \centering \label{tab:native}
\begin{tabular}{lllll}
\toprule
Finnish & Saami & Proto-Finno-Ugric & Meaning \\
\midrule
\ipa{käsi} & \ipa{giehta} & *\ipa{käti} & `hand' \\
\ipa{nimi} & \ipa{namma} & *\ipa{nimi} & `name' \\
\ipa{kala} & \ipa{guolli} & *\ipa{kala} & `fish' \\
\ipa{muna} & \ipa{monni} & *\ipa{muna} & `egg' \\
\bottomrule
\end{tabular}
\end{table}

The correspondence of final \ipa{-a} to \ipa{-i} and final \ipa{-i} to \ipa{-a} in disyllabic words found in the native vocabulary, as illustrated by the data in Table \ref{tab:native}, is also observed in Saami words borrowed from Finnish, including recent borrowings, such as \ipa{mearka} from \ipa{merkki} `sign, mark' and \ipa{báhppa} from \ipa{pappi} `priest' (from Russian \ipa{поп}, itself of Greek origin), even though the sound change from proto-Uralic to Saami leading to the correspondence \ipa{-a} : \ipa{-i} had already taken place at the time of contact. These correspondences are pervasive even in the most recent borrowings, to the extent that according to \citet[36]{aikio06nativization} `examples of phonetically unmarked substitutions of the type F[innish] \ipa{-i} > Saa[mi] \ipa{-i} and F[innish] \ipa{-a} > Saa[mi] \ipa{-a} are practically nonexistent, young borrowings included.'

In cases such as \ipa{báhppa} `priest', the vowel correspondence in the first syllable \ipa{á} : \ipa{a} betrays its origin as a loanword, as the expected correspondence for a native word would be \ipa{uo} : \ipa{a} as in the word `fish' in Table \ref{tab:native} (\citealt[35]{aikio06nativization} notes that this correspondence is never found in borrowed words).

However, there are cases where recent loanwords from Finnish in Saami present correspondences indistinguishable from those of the inherited lexicon, as \ipa{barta} `cabin' from Finnish \ipa{pirtti}, itself from dialectal Russian \ipa{перть} `a type of cabin', which show the same \ipa{CiCi} : \ipa{CaCa} vowel correspondence as the word `name' in table \ref{tab:native}. Here again, the foreign origin of this word is a clear indication that \ipa{barta} `cabin' cannot have undergone the series of regular sound changes leading from proto-Finno-Ugric *\ipa{CiCi} to Saami \ipa{CaCa}, and that instead the common vowel correspondence \ipa{CiCi} : \ipa{CaCa} was applied to Finnish \ipa{pirtti}.

 
Loan nativization can also occur between genetically unrelated language. A clear example is provided by the case of Basque and Spanish (\citealt[53-54]{trask00chronology}, \citealt[21-3]{aikio06nativization}). 

A recurrent correspondence between Spanish and Basque is word-final \ipa{-ón} to \ipa{-oi}. Early Romance *\ipa{-one} (from Latin \ipa{-onem}) yields Spanish \ipa{-ón}. In Early Romance borrowings into Basque, however, this ending undergoes the regular loss of intervocalic *\ipa{-n-} (a Basque-internal sound change), and yields *\ipa{-one} $\rightarrow$ *\ipa{-oe} $\rightarrow$ \ipa{-oi}. An example of this correspondence is provided by Spanish \ipa{razón} and Basque \ipa{arrazoi} `reason' both from Early Romance *\ipa{ratsone} (from the Latin accusative form $\leftarrow$ \ipa{ratiōnem}).

This common correspondence have however been recently applied to recent borrowings from Spanish such as \ipa{kamioi} `truck' and \ipa{abioi} `plane' (from \ipa{camión} and \ipa{avión}). This adaptation has no phonetical motivation, since word-final \ipa{-on} is attested in Basque, and can only be accounted for as overapplication of the \ipa{-oi} : \ipa{-ón} correspondence.

%analyzable compounds?


\subsubsection{Implications for the interpretation of François' data}
In his database of 474 innovations, \citet[177]{francois15tree} counts 236 lexical replacements and 116 lexically-specific sound changes (in other words, lexical replacements where the innovative form is related to the archaic form by some unpredictable phonetic change). A group of languages are considered to share these innovations whenever they present reflexes of these etyma following regular correspondences.

Thus, 352 out of 474 innovations (74\%) are potential cases of either undetectable or nativized borrowings. XXX


\bibliographystyle{unified}
\bibliography{bibliogj}

 \end{document}
 