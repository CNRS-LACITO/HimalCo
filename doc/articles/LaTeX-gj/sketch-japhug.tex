\documentclass[oldfontcommands,oneside,a4paper,11pt]{article} 
\usepackage{fontspec}
\usepackage{natbib}
\usepackage{booktabs}
\usepackage{xltxtra} 
\usepackage{longtable}
\usepackage{polyglossia} 
%\usepackage[table]{xcolor}
\usepackage{gb4e} 
\usepackage{multicol}
\usepackage{graphicx}
\usepackage{float}
\usepackage{lineno}
\usepackage{textcomp}
\usepackage{hyperref} 
\hypersetup{bookmarks=false,bookmarksnumbered,bookmarksopenlevel=5,bookmarksdepth=5,xetex,colorlinks=true,linkcolor=blue,citecolor=blue}
\usepackage[all]{hypcap}
\usepackage{memhfixc}
\usepackage{lscape}
 

%\setmainfont[Mapping=tex-text,Numbers=OldStyle,Ligatures=Common]{Charis SIL} 
\newfontfamily\phon[Mapping=tex-text,Ligatures=Common,Scale=MatchLowercase,FakeSlant=0.3]{Charis SIL} 
\newcommand{\ipa}[1]{{\phon #1}} %API tjs en italique
 
\newcommand{\grise}[1]{\cellcolor{lightgray}\textbf{#1}}
\newfontfamily\cn[Mapping=tex-text,Ligatures=Common,Scale=MatchUppercase]{MingLiU}%pour le chinois
\newcommand{\zh}[1]{{\cn #1}}
\newcommand{\topic}{\textsc{dem}}
\newcommand{\tete}{\textsuperscript{\textsc{head}}}
\newcommand{\rc}{\textsubscript{\textsc{rc}}}
\XeTeXlinebreaklocale 'zh' %使用中文换行
\XeTeXlinebreakskip = 0pt plus 1pt %
 %CIRCG
 


\begin{document} 

\title{A sketch of Japhug}
\author{Guillaume Jacques}
\maketitle

\section{Introduction}

\section{Phonology}

\section{Verbal morphology}


\subsection{TAM and directional prefixes}
\subsection{Person marking}
\subsection{Derivation}
\subsection{Nominalization and other non-finite forms}

\section{The noun phrase}

\section{Ideophones}

\section{Relativization}

\section{Complementation}

\section{Clause linking}

\citet{jacques12demotion}
\citet{jacques14linking}
\citet{jacques14antipassive}
\citet{jacques13harmonization}
\citet{jacques13tropative}
\citet{japhug14ideophones}
\citet{jacques10inverse}
\citet{jacques08}
\citet{jacques07redupl}
\citet{jacques04these}

\bibliographystyle{linquiry2}
\bibliography{bibliogj}
\end{document}