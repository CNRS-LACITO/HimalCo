\documentclass[oldfontcommands,oneside,a4paper,11pt]{article} 
\usepackage{fontspec}
\usepackage{natbib}
\usepackage{booktabs}
\usepackage{xltxtra} 
\usepackage{polyglossia} 
\usepackage[table]{xcolor}
\usepackage{gb4e} 
\usepackage{multicol}
\usepackage{graphicx}
\usepackage{float}
\usepackage{textcomp}
%\usepackage{hyperref} 
%\hypersetup{bookmarks=false,bookmarksnumbered,bookmarksopenlevel=5,bookmarksdepth=5,xetex,colorlinks=true,linkcolor=blue,citecolor=blue}
%\usepackage[all]{hypcap}
\usepackage{memhfixc}
\usepackage{lscape}
 

%\setmainfont[Mapping=tex-text,Numbers=OldStyle,Ligatures=Common]{Times New Roman} 
\newfontfamily\phon[Mapping=tex-text,Ligatures=Common,Scale=MatchLowercase]{Charis SIL} 
\newcommand{\ipa}[1]{{\phon#1}} %API tjs en italique
 
\newcommand{\grise}[1]{\cellcolor{lightgray}\textbf{#1}}
\newfontfamily\cn[Mapping=tex-text,Ligatures=Common,Scale=MatchUppercase]{MingLiU}%pour le chinois
\newcommand{\zh}[1]{{\cn#1}}
\newcommand{\topic}{\textsc{dem}}
\newcommand{\tete}{\textsuperscript{\textsc{head}}}
\newcommand{\rc}{\textsubscript{\textsc{rc}}}
\XeTeXlinebreaklocale 'zh' %使用中文换行
\XeTeXlinebreakskip = 0pt plus 1pt %
 %CIRCG
\newcommand{\ro}{$\Sigma$}
\newcommand{\siga}{$\Sigma_1$} 
\newcommand{\sigc}{$\Sigma_3$}   
\newcommand{\refb}[1]{(\ref{#1})}
\newcommand{\factual}[1]{\textsc{:fact}}

\renewcommand\thetable{31.\arabic{table}}

\begin{document} 

\title{A sketch of Japhug\footnote{I wish to thank Randy LaPolla, Laurent Sagart, Graham Thurgood, Nicolas Tournadre and Theo Yeh for useful comments on this chapter. Glosses follow the Leipzig rules, to which the following are added: \textsc{cisloc} cislocative, \textsc{fact} factual, \textsc{genr} generic, \textsc{egoph} egophoric, \textsc{emph} emphatic, \textsc{ifr} inferential, \textsc{inv} inverse, \textsc{lnk} linker, \textsc{sens} sensory, \textsc{total} totalitative, \textsc{transloc} translocative. The examples are taken from a corpus that is progressively being made available on the Pangloss archive (\citealt{michailovsky14pangloss}). This research was funded by the HimalCo project (ANR-12-CORP-0006) and is related to the research strand LR-4.11 ‘‘Automatic Paradigm Generation and Language Description’’ of the Labex EFL (funded by the ANR/CGI).  }}
\author{Guillaume Jacques}
\maketitle

%\textbf{Abstract}: This chapter is a grammatical sketch of the Kamnyu variety of the Japhug language spoken in Mbarkham county, Rngaba prefecture, Sichuan Province, China. It focuses on verbal and nominal morphology, especially argument indexation, voice and tense-aspect-modality marking. The chapter also briefly describes relative clauses, complement clauses and comparative constructions.

\section{Introduction}
Japhug (\ipa{kɯrɯskɤt}) is one of the four Rgyalrong languages. It is spoken in Gdong-brgyad (\ipa{ʁdɯrɟɤt}), Gsar-rdzong (\ipa{sarndzu}) and Da-tshang (\ipa{tatsʰi}) areas in Mbarkhams (\ipa{mbarkʰom}) county, Sichuan province, China, by around 10000 speakers, and presents some dialectal diversity, especially in the Gdong-brgyad area.

Documentation on this language includes various articles, a short grammar with glossary (\citealt{jacques08}), a text collection (\citealt{jacques10gesar}) and a dictionary (\citealt{jacques15japhug}).

Like all Rgyalrongic languages, Japhug has undergone considerable lexical influence from Tibetan (\citealt{jacques04these}), but this influence is less conspicuous in the domain of morphosyntax.



\section{Phonology}
Japhug has complex onsets with 50 consonant phonemes and at least 404 consonant clusters (308 biconsonantal and 96 triconsonantal clusters), of which 58 are only attested in ideophones (\citealt{japhug14ideophones}) and 30 in Tibetan loanwords. Among the most unusual clusters in Japhug, we find palatal stop+\ipa{ɣ} (\ipa{lɟɣaʁ} `hang on sth'), clusters with \ipa{j--} as first element (\ipa{jpum} `thick') and fricative + prenasalized stop (\ipa{ʑmbrɯ} `boat')


Only twelve consonants appear in coda position, and no clusters are allowed:  \ipa{--p}, \ipa{--β}, \ipa{--m}, \ipa{--t}, \ipa{--z}, \ipa{--n}, \ipa{--l}, \ipa{--r}, \ipa{--j}, \ipa{--ɣ}, \ipa{--ŋ}, \ipa{--ʁ}. The stop \ipa{--p} is only attested in a few ideophones, and the only final stop in non-ideophonic vocabulary is \ipa{--t}. Final voiced segments, whether sonorant or obstruents, are devoiced before a pause but voicing surfaces in sandhi.

Japhug differs from all other Rgyalrong languages in that it has lost tonal contrasts. The only suprasegmental feature is the word stress, which is always word final except in the case of a few stress-attracting prefixes (the inverse \ipa{wɣɯ́--}, the negative sensory evidential \ipa{mɯ́j--}, and the comitative \ipa{kɤ́--}). Verb suffixes are always unstressed (and can be realized with voiceless vowels), and stress in verb forms are either on the stress-attracting prefixes or on the last stem syllable.

Japhug dialects vary as to their exact number of vowels. The Kamnyu dialect has eight vowel phonemes \ipa{a},  \ipa{e},  \ipa{i},  \ipa{o},  \ipa{u},   \ipa{ɤ},  \ipa{ɯ} and \ipa{y}, the latter only occurring in the word \ipa{qaɟy} `fish' and verbs derived from it and in Chinese loanwords. 

Partial reduplication in Japhug provides important insight into syllabic structure (\citealt{jacques07redupl}). When partial reduplication is applied to a syllable, the rhyme of the replicated syllable is changed to \ipa{ɯ} in the replicant.
 
Some clusters are affected by partial reduplication:   when the last consonant  of a cluster is one of the non-nasal sonorants (\ipa{r}, \ipa{l}, \ipa{j}, \ipa{w}, \ipa{ɣ} or \ipa{ʁ}), and the preceding consonant is neither a sonorant nor a sibilant fricative, the sonorant is deleted, as in example \refb{ex:medial.r}. 
 
 \begin{exe}
\ex \label{ex:medial.r}
\glt \ipa{praʁ} `cut, break' $\rightarrow$ \ipa{pɯ-praʁ}
\end{exe}

When the penultimate consonant of the cluster is a sonorant and the last consonant is one of \{\ipa{r}, \ipa{l}, \ipa{ɣ}, \ipa{ʁ}\}, this last consonant  is not deleted. 

 \begin{exe}
\ex  \label{ex:initial.r}
\glt \ipa{βraʁ} `attach' $\rightarrow$ \ipa{βrɯ-βraʁ}
\end{exe}
 
The contrast between \refb{ex:medial.r} and \refb{ex:initial.r} shows that the \ipa{r} in these cluster does not have the same phonological status: in the first case, it forms a phonological constituent with the rhyme.



\section{Verbal morphology}

\subsection{Verb stems}

Japhug verbs present stem alternations determined by TAM and person/number marking. 

Verbs have at most three distinct stems, which we designate as stem 1, stem 2 and stem 3 following \citet{jackson00sidaba}. Stem 1 is the default stem. Stem 2 appears only in the perfective and past imperfective forms. It is very residual in Japhug, attested only in a limited number of verbs indicated in Table \refb{tab:stem2}.


 \begin{table} 
\caption{Stem 2 alternations in Japhug Rgyalrong} \label{tab:stem2} \centering
\begin{tabular}{llllll}
\toprule
Stem 1 & meaning &Stem 2 \\
\midrule
\ipa{ɕe}& to go (vi)&  \ipa{ari} \\
\ipa{sɯxɕe}& to send (vt)  &\ipa{sɤɣri} \\
\ipa{ɣi}& to come (vi)  &\ipa{ɣe} \\
\ipa{ti}& to say (vt)  &\ipa{tɯt} \\
\bottomrule
\end{tabular}
\end{table}

Stem 3 in Japhug on the other hand is fully productive, and always appears in the forms \textsc{1sg}$\rightarrow$3, \textsc{2sg}$\rightarrow$3 and \textsc{3sg}$\rightarrow$3' of non-past TAM categories (factual, imperfective, sensory, egophoric present) for transitive verbs. Stem 3 does not appear in verb forms with the inverse marker (see below). Intransitive verbs lack stem 3 alternation.

Table \refb{tab:stem3} presents all Stem 3 alternations attested in various Japhug dialects. The vowel alternation applies to the last vowel (in open syllables only) of the verb stem. In the case of verb stems ending in  \ipa{--u} and  \ipa{--ɯ}, some Japhug dialects allow two possible alternations, some verbs displaying vowel fronting and other addition of the \ipa{--m} suffixal element (as in Datshang, see \citealt{linluo03}), while other dialects have generalized vowel fronting.

 \begin{table} 
\caption{Stem 3 alternations in Japhug Rgyalrong} \label{tab:stem3} \centering
\begin{tabular}{llllll}
\toprule
Stem 1 & Stem 3& type \\
\midrule
 \ipa{--a} & \ipa{--e} & vowel fronting\\
 \ipa{--u} & \ipa{--e} & vowel fronting\\
\ipa{--u} & \ipa{--ɤm} & \ipa{--m} \\
\ipa{--ɯ} & \ipa{--i} & vowel fronting\\
\ipa{--ɯ} & \ipa{--ɯm} & \ipa{--m}\\
\ipa{--o} & \ipa{--ɤm} & \ipa{--m} \\
\bottomrule
\end{tabular}
\end{table}




\subsection{Orientational prefixes}
All finite verb forms except the \textsc{factual} have orientational prefixes belonging to one of the four series in Table \refb{tab:orientational} (some Japhug dialects have \ipa{co--} instead of \ipa{pjɤ--} in series D for the `down' prefix, see \citealt{linluo03}). The non-periphrastic TAM categories are indicated in Table \refb{tab:finite.forms}. Some of these categories require one particular orientational prefix (sensory, egophoric present, past imperfective, evidential imperfective). The other TAM categories only require a prefix of either set A, B, C or D, but the direction depends on the verb.

Most verbs have one intrinsic direction which is lexically determined, and used in the perfective, imperfective, evidential, irrealis and imperative. For instance, the verb \ipa{ndza} `eat' selects the `up' series of orientational prefixes for all its forms: \textbf{perfective} \textsc{1sg$\rightarrow$3sg} \ipa{tɤ-ndza-t-a} `I ate it', \textbf{imperfective} (with stem 3) \ipa{tu-ndze} `He eats it', \textbf{perfective} \textsc{3sg$\rightarrow$3'} \ipa{ta-ndza} `He ate it' and \textbf{evidential} \ipa{to-ndza} `He ate it'. 

In addition to the orientational prefixes, two additional TAM markers appear in specific forms. First, the past \ipa{--t} suffix (\ipa{--z} in some dialects of Japhug)in the \textsc{1sg}$\rightarrow$3 and \textsc{2sg}$\rightarrow$3 forms of transitive verbs whose last stem syllable is an open syllable as \ipa{tɤ-ndza-t-a} \textsc{pfv}-eat-\textsc{pst-1sg} `I ate it'. Second, the evidential circumfix \ipa{kɯ--...-ci} is required whenever set D orientational prefixes are used with a verb whose stem begins in \ipa{a--}, including verb with the progressive prefix \ipa{asɯ--} (see below). For instance, the evidential form of the verb \ipa{anbaʁ} `hide' is \ipa{ko-k-ɤnbaʁ-ci} \textsc{ifr-evd}-hide-\textsc{evd} `he hid'.

\begin{table}[H]
\caption{Orientational prefixes in Japhug Rgyalrong} \label{tab:orientational}
\resizebox{\columnwidth}{!}{
\begin{tabular}{llllll}
\toprule
   &  	perfective  (A) &  	imperfective  (B)  &  	perfective 3$\rightarrow$3' (C)  &  	evidential  (D) \\  	
   \midrule
up   &  	\ipa{tɤ--}   &  	\ipa{tu--}   &  	\ipa{ta--}   &  	\ipa{to--}   \\  	
down   &  	\ipa{pɯ--}   &  	\ipa{pjɯ--}   &  	\ipa{pa--}   &  	\ipa{pjɤ--}   \\  	
upstream   &  	\ipa{lɤ--}   &  	\ipa{lu--}   &  	\ipa{la--}   &  	\ipa{lo--}   \\  	
downstream   &  	\ipa{tʰɯ--}   &  	\ipa{cʰɯ--}   &  	\ipa{tʰa--}   &  	\ipa{cʰɤ--}   \\  	
east   &  	\ipa{kɤ--}   &  	\ipa{ku--}   &  	\ipa{ka--}   &  	\ipa{ko--}   \\  	
west   &  	\ipa{nɯ--}   &  	\ipa{ɲɯ--}   &  	\ipa{na--}   &  	\ipa{ɲɤ--}   \\  	
no direction &\ipa{jɤ--}   &  	\ipa{ju--}   &  	\ipa{ja--}   &  	\ipa{jo--}   \\  	
\bottomrule
\end{tabular}}
\end{table}


Verbs of motion and some verbs of concrete action can be associated with all seven series of prefixes to indicate the direction of the motion. The  `no direction' series of prefixes only occurs with motion verbs. 

\begin{table}
\caption{Finite verb categories in Japhug Rgyalrong} \label{tab:finite.forms} \centering
\begin{tabular}{lllllll}
\toprule
&	&	stem&	prefixes\\
\midrule
factual&	\textsc{fact} &	1 or 3&	no prefix\\
imperfective&	\textsc{ipfv} &	1 or 3&	B\\
perfective&	\textsc{pfv} &	2&	A or C\\
past imperfective&	\textsc{pst.ipfv} &	2&	\ipa{pɯ--}\\
inferential&	\textsc{ifr} &	1&	D\\
inferential imperfective&	\textsc{ifr.ipfv} &	1&	\ipa{pjɤ--}\\
sensory&	\textsc{sens} &	1 or 3&	\ipa{ɲɯ--}\\
egophoric present&	\textsc{pres} &	1 or 3&	\ipa{ku--}\\
irrealis&	\textsc{irr} &	1 or 3&	\ipa{a--} + A\\
imperative&	\textsc{imp} &	1 or 3&	A\\
\bottomrule
\end{tabular}
\end{table}

The simple TAM categories in Table \refb{tab:finite.forms} can be combined with other TAM markers such as the conative \ipa{ji--} and the progressive \ipa{asɯ--}/\ipa{az--}. The prefix \ipa{asɯ--} can only be used with transitive verbs. Verbs prefixed with \ipa{asɯ--} lose some transitivity features, such as stem alternation and past \ipa{-t-} suffix, but can still take inverse and 1$\rightarrow$2 / 2$\rightarrow$1 markers. The inverse prefix is \textit{infixed} within the progressive prefix, as in the form \ipa{pjɤ-k-ɤ́<wɣ>z-nɤjo-ci} \textsc{ipfv.ifr-evd-prog<inv>}-wait.for-\textsc{evd} `he was waiting for him'.

In addition, periphrastic TAM categories, combining simple verb forms with the copula \ipa{ŋu} `be', \ipa{maʁ} `not be' or \ipa{ɕti} `be (affirmative)' are also attested. The most common periphrastic tenses are the periphrastic past imperfective and periphrastic inferential imperfective, built by combining the imperfective form of the verb with the past imperfective \ipa{pɯ-ŋu} and inferential imperfective \ipa{pjɤ-ŋu} of the copula. Most dynamic verbs cannot be used in the simple past and evidential imperfective in Japhug (see \citealt{lin11direction}) and periphrastic tenses are required as in example \refb{ex:evd.ipfv.periphrastic}: directly combining the prefix \ipa{pjɤ--} with the stem of the verb `eat' results  in an incorrect form. The past and inferential imperfective of dynamic verbs is attested only in counterfactual and in combination with the progressive \ipa{asɯ--} (see \citealt[297-301]{jacques14linking}).\footnote{The presence of the circumfix \ipa{k--...-ci} in example \refb{ex:evd.ipfv.prog} is explained in the previous section.}

\begin{exe}
\ex \label{ex:evd.ipfv.periphrastic}
\gll
\ipa{tu-ndze} \ipa{pjɤ-ŋu} \\
 \textsc{ipfv}-eat[III] \textsc{ipfv.ifr}-be \\
\glt He was eating it. / He used to eat it.
\ex \label{ex:evd.ipfv.prog}
\gll
 \ipa{pjɤ-k-ɤsɯ-ndza-ci} \\
 \textsc{ipfv.ifr-evd-prog}-eat-\textsc{evd} \\
\glt He was eating it.
\end{exe}



\subsection{Person marking}
Japhug person marking is encoded by a series of prefixes and suffixes, in addition to stem 1 / Stem 3 alternations in the case of transitive verbs.

Table \refb{tab:japhug.tr} presents the regular transitive and intransitive verbal paradigms in Japhug in the \textsc{factual}. The symbols \siga{} and \sigc{} represent Stem 1 and Stem 3 respectively. The affixes found in the intransitive paradigm (the second person \ipa{tɯ--} and the suffixes) also appear in the transitive paradigm, in addition to portmanteau prefixes (\ipa{kɯ--} and \ipa{ta--} for the local scenarios 2$\rightarrow$1 and 1$\rightarrow$2) and the inverse prefix \ipa{wɣɯ--} which is obligatory in 3$\rightarrow$1 and 3$\rightarrow$2 mixed scenario and cannot occur in local and 1$\rightarrow$3 and 2$\rightarrow$3 scenarios. When both arguments are third person, the use of the inverse is determined by semantic and pragmatic factors (see \citealt{jacques10inverse}).
\begin{landscape}


\begin{table}[H]
\caption{Japhug transitive and intransitive paradigms}\label{tab:japhug.tr}
\resizebox{\columnwidth}{!}{
\begin{tabular}{l|l|l|l|l|l|l|l|l|l|l|}
\textsc{} & 	\textsc{1sg} & 	  \textsc{1du} & 	\textsc{1pl} & 	\textsc{2sg} & 	\textsc{2du} & 	\textsc{2pl} & 	\textsc{3sg} & 	\textsc{3du} & 	\textsc{3pl} & 	\textsc{3'} \\ 	
\hline
\textsc{1sg} & \multicolumn{3}{c|}{\grise{}} &	\ipa{} & 	\ipa{} & 	\ipa{} & 	\ipa{\sigc{}-a}   & 	 \ipa{\sigc{}-a-ndʑi} & 	 \ipa{\sigc{}-a-nɯ} & 	\grise{} \\	
\cline{8-10}
\textsc{1du} & 	\multicolumn{3}{c|}{\grise{}} &	\ipa{ta-\siga{}} & 	\ipa{ta-\siga{}-ndʑi} & 	\ipa{ta-\siga{}-nɯ} & 	\multicolumn{3}{c|}{ \ipa{\siga{}-tɕi}}  & 	\grise{} \\	
\cline{8-10}
\textsc{1pl} & 	\multicolumn{3}{c|}{\grise{}} & 	  & 	&  & 	\multicolumn{3}{c|}{ \ipa{\siga{}-ji}}  & 	\grise{} \\	
\cline{1-10}
\textsc{2sg} & 	\ipa{kɯ-\siga{}-a} & 	\ipa{} & 	\ipa{} & 	\multicolumn{3}{c|}{\grise{}}&	\multicolumn{3}{c|}{\ipa{tɯ-\sigc{}}} & 	\grise{} \\	
\cline{2-2}
\cline{8-10}
\textsc{2du} & 	\ipa{kɯ-\siga{}-a-ndʑi} & 	\ipa{kɯ-\siga{}-tɕi} & 	\ipa{kɯ-\siga{}-ji} & 	\multicolumn{3}{c|}{\grise{}} &	\multicolumn{3}{c|}{\ipa{tɯ-\siga{}-ndʑi}} & 	\grise{} \\	
\cline{2-2}
\cline{8-10}
\textsc{2pl} & 	\ipa{kɯ-\siga{}-a-nɯ} & 	\ipa{} & 	\ipa{} & 	\multicolumn{3}{c|}{\grise{}}&	\multicolumn{3}{c|}{\ipa{tɯ-\siga{}-nɯ}} & 	\grise{} \\	
\hline
\textsc{3sg} &  	\ipa{wɣɯ́-\siga{}-a} & 	\ipa{} & 	\ipa{} & 	\ipa{} & 	\ipa{} & 	\ipa{} & \multicolumn{3}{c|}{\grise{}} &	\ipa{\sigc{}} \\ 	
\cline{2-2}
\cline{11-11}
\textsc{3du} &  	\ipa{wɣɯ́-\siga{}-a-ndʑi} & 	 \ipa{wɣɯ́-\siga{}-tɕi} & 		\ipa{wɣɯ́-\siga{}-ji} & 	\ipa{tɯ́-wɣ-\siga{}} & 	\ipa{tɯ́-wɣ-\siga{}-ndʑi} & 	\ipa{tɯ́-wɣ-\siga{}-nɯ} & 	\multicolumn{3}{c|}{\grise{}} &	\ipa{\siga{}-ndʑi} \\ 
\cline{2-2}	
\cline{11-11}
\textsc{3pl} &  	\ipa{wɣɯ́-\siga{}-a-nɯ} & 	\ipa{} & 	\ipa{} & 	\ipa{} & 	\ipa{} & 	\ipa{} & \multicolumn{3}{c|}{\grise{}} &	\ipa{\siga{}-nɯ} \\ 	
\hline
\textsc{3'} & 	\multicolumn{6}{c|}{\grise{}} &	\ipa{wɣɯ́-\siga{}} & 	\ipa{wɣɯ́-\siga{}-ndʑi} & 	\ipa{wɣɯ́-\siga{}-nɯ} & 	\grise{} \\	
	\hline	\hline
\textsc{intr}&\ipa{\siga{}-a}&\ipa{\siga{}-tɕi}&\ipa{\siga{}-ji}&\ipa{tɯ-\siga{}}&\ipa{tɯ-\siga{}-ndʑi}&\ipa{tɯ-\siga{}-nɯ}&\ipa{\siga{}}&\ipa{\siga{}-ndʑi} &\ipa{\siga{}-nɯ}& 	\grise{} \\	
	\hline
\end{tabular}}
\end{table}
\end{landscape}
The sensory existential copulas \ipa{ɣɤʑu} `exist' and \ipa{maŋe} `not exist' are extremely irregular. They cannot be used with orientational prefixes, unlike nearly all other verbs, cannot be nominalized, and have infixed forms for the second person \ipa{ɣɤtɤʑu} `you are there' and \ipa{mataŋe} `you are not there'.\footnote{In \ipa{mataŋe}  the element \ipa{ma--} is related to the negative prefixes \ipa{mɤ--}  and \ipa{mɯ--}, but cannot be analyzed as such synchronically in this form, as there is no independent stem \ipa{--ŋe}.}


\subsection{Evidentiality}
Japhug has a very rich evidential system,  which is typologically very similar to that of Tibetic languages.\footnote{Other Rgyalrong languages have similar systems, see in particular \citet{linyj03tense} on Situ.} For reasons of space, we only discuss the evidential contrasts in the present contexts in this chapter (in addition, Japhug has distinct markers for past perfective and past imperfective evidential categories).

As in Tibetan, the evidential forms present a rule of anticipation (see \citealt{tournadre08conjunct} and \citealt{tournadre14evidentiality}), whereby the speaker uses in questions the forms that he expects his addressee will employ in his answer. For this reason, I use the notation `1/2' to describe forms appearing with the first person in affirmative sentences and with the second person in interrogative sentences. 

In present tense contexts, we find three evidential categories: the factual (using the prefixless verb form in stem III, without any auxiliary verb), the sensory evidential (marked by the prefix \ipa{ɲɯ--} or by suppletion in the case of existential verbs)  and the egophoric present (\ipa{ku--}).

The factual is used in the present to describe facts considered by the speaker to belong to commonly accepted knowledge, as in \refb{ex:mAndze} or present (and future) situations about which the speaker is fairly certain (example \ref{ex:ata}). The factual can be used with all persons without strong constraints.

\begin{exe}
\ex \label{ex:mAndze}
\gll
   	\ipa{ɯ-ku}  	\ipa{kɯ-mpɯ}  	\ipa{nɯ}  	\ipa{ɲɯ́-wɣ-pʰɯt}  	\ipa{tɕe,}  \ipa{nɯŋa}  	\ipa{ra}  	\ipa{kɯ}  	\ipa{ndza-nɯ,}  	\ipa{paʁ}  	\ipa{kɯ}  	\ipa{mɤ-ndze}    \\
\textsc{3sg.poss}-head \textsc{nmlz}:S/A-be.soft \textsc{dem} \textsc{ipfv-inv}-pluck \textsc{lnk} cow \textsc{pl} \textsc{erg} eat:\textsc{fact-pl} pig \textsc{erg} \textsc{neg}-eat\factual{} \\
\glt  One plucks the (leaves) on the extremities, the soft ones, the cows eat it, the pigs don't. (06 qaZmbri, 20)
\end{exe}


\begin{exe}
\ex \label{ex:ata}
\gll
 \ipa{a-ɣe}  	\ipa{ɣɯ}  	\ipa{ɯ-pɤro}  	\ipa{ci}  	 <qiche> 	\ipa{kɯ-xtɕi}  	\ipa{ci}  	\ipa{to-χtɯ.}  	\ipa{tɕe}  	\ipa{andi}  	\ipa{ra}  	\ipa{a-ta.}  	\\
 \textsc{1sg.poss}-grandson \textsc{gen} \textsc{3sg.poss}-present \textsc{indef} car \textsc{nmlz}:S/A-be.small \textsc{indef} \textsc{ifr}-buy \textsc{lnk} west \textsc{pl} \textsc{pass}-put\factual{} \\
 \glt He bought a present, a small car for my grandson, it is there (at home). (14 gongxun, 2-3)
  \end{exe}

The egophoric \ipa{ku--} is only used for 1/2 arguments (\ref{ex:kutaRa}), or third persons possessed by 1/2 (\ref{ex:WkudAn}). It is used to express intimate knowledge of an event or state on the part of the 1/2, not resulting from guess or recent information mediated through the senses. It cannot express a general or gnomic state of affair, it is only used to refer to an ongoing state or action. It is virtually absent from procedural texts and narratives (expect in conversations quoted in the stories), but very common in conversations.


\begin{exe}
\ex \label{ex:kutaRa}
\gll 
<kuabao> 	\ipa{ɯ-spa}  	\ipa{ci}  	\ipa{ku-taʁ-a}  \\
bag \textsc{3sg.poss}-material \textsc{indef} \textsc{egoph}-weave-\textsc{1sg} \\
\glt I am weaving a bag (conversation, 2014-10)
\end{exe}

 
 \begin{exe}
\ex \label{ex:WkudAn}
\gll \ipa{nɤ-kɤ-nɤma} 	\ipa{ɯ-kú-dɤn?}  \\
 \textsc{2sg.poss-nmlz:P}-work \textsc{qu-egoph}-be.many \\
\glt Do you have a lot of work (right now)? (2014.10 conversation, Chenzhen)
\end{exe}

The sensory evidential\footnote{This refers to the category that has been variously referred to as `constatif', `mirative' or `testimonial' in the literature (see \citealt{hill12mirativity}). I here adopt \citet{tournadre14evidentiality}'s term.} is primarily used when the speaker's knowledge of a situation has been obtained through his senses (whether vision, audition, touch etc) and is not yet fully assimilated, as in \ref{ex:aZWB.mWjGi}.

\begin{exe}
\ex \label{ex:aZWB.mWjGi} 
\gll 
\ipa{nɤʑo}  	\ipa{nɤ-skɤt}  	\ipa{kɯ-fse}  	\ipa{ʑo}  	\ipa{kɯ-sna}  	\ipa{nɯ}  	\ipa{pɯ-mtsʰam-a}  	\ipa{ndɤre,}  	\ipa{aʑo}  	\ipa{tɕe}  	\ipa{mɤʑɯ}  	\ipa{a-ʑɯβ}  	\ipa{mɯ́j-ɣi}  	\ipa{ma}  	\ipa{ɲɯ-mpɕɤr}   \\
\textsc{2sg} \textsc{2sg.poss}-voice \textsc{nmlz}:S/A-be.like \textsc{emph} \textsc{nmlz}:S/A-be.nice \textsc{dem} \textsc{pfv}-hear-\textsc{1sg} \textsc{lnk}  \textsc{1sg} \textsc{lnk} even.more  \textsc{2sg.poss}-slumber \textsc{neg:sens}-come because \textsc{sens}-be.beautiful \\
\glt After hearing you wonderful voice, I am feeling even less sleepy, as it is (so) beautiful. (140427 zhameng he maotouying, 22)
\end{exe}

Unlike the egophoric, the sensory evidential is not incompatible with general statements. In these contexts using the sensory rather than the factual expresses the speaker's lack of authority and confidence, for instance when describing animals that do not exist in Rgyalrong areas (as in \ref{ex:YWmpCAr}).

 
\begin{exe}
\ex \label{ex:YWmpCAr}
\gll 
<banma> 	\ipa{nɯ}  	\ipa{ɲɯ-mpɕɤr}  \\
zebra \textsc{dem} \textsc{sens}-be.beautiful \\
\glt The zebra is (a) beautiful (animal). (20 RmbroN, 128)
\end{exe}

The sensory is rarely used with the first person in affirmative sentences, except in the case of endopathic predicates (expressing pain, hunger, cold, etc) and some cognitive verbs (such as \ipa{sɯso} `think'). The sensory is also usable with endopathic predicates with third person referents (Japhug differs in this regard from Tibetan, see \citealt[244]{tournadre14evidentiality}).

In addition to evidential markers encoded in verbal morphology, the hearsay sentence final particle \ipa{kʰi} is used to indicate that the speaker's knowledge is based on a second-hand source. This particle most commonly appears with the sensory evidential, but examples with the factual also exist.

\subsection{Derivation}
Japhug a a very rich system of derivations, including voice markers, denominal and deideophonic prefixes. Only the former will be discussed here. The system of voice derivation in Japhug is very similar to that described in Tshobdun (\citealt{jackson14morpho}).

\subsubsection{Argument demotion}
In Japhug, non-overt arguments in a sentence are always construed as being definite. To express indefinite agent or patient, several strategies are possible, including indefinite pronouns such as \ipa{tʰɯci} `something', generic person and argument-demoting voice markers.

Japhug has a wide array of argument-demoting prefixes, including passive \ipa{a--}, anticausative, antipassive \ipa{sɤ--} and \ipa{rɤ--} and deexperiencer \ipa{sɤ--} (see \citealt{jacques12demotion}, and also \citealt{jackson14morpho} on cognate markers in Tshobdun). The antipassive prefixes have been grammaticalized from denominal prefixes (\citealt{jacques14antipassive}).

The  \ipa{a--} prefix is an agentless passive, that converts a transitive verb into an intransitive one whose S corresponds to the P of the base verb. The original A cannot be expressed, but is semantically recoverable, as in \refb{ex:passive} with the light verb construction \ipa{sɤcɯ --lɤt} `lock'. 

\begin{exe}
\ex \label{ex:passive}
\gll 
\ipa{ɯ-ŋgɯ} 	\ipa{lɤ-ɣi} 	\ipa{jɤɣ} 	\ipa{ma} 	\ipa{sɤcɯ} 	\ipa{mɤ-a-lɤt} \\
\textsc{3sg}-inside \textsc{imp:upstream}-come  be.possible\factual{} because key \textsc{neg-pass}-throw\factual{} \\
\glt Come in, (the door) is not locked.
\end{exe}

The anticausative is distinct from the agentless passive in that the action is construed as having occurred spontaneously without any agent. Only 24 pairs of verbs with anticausative derivation have been found in Japhug (see some examples in Table \refb{tab:anticausative}). Note the presence of the verb \ipa{χtɤr} `spill' borrowed from Tibetan \ipa{gtor} (same meaning): this verb shows that the direction of derivation is from transitive to intransitive, not the other way round, since the corresponding intransitive verb \ipa{ʁndɤr} `be spilled' cannot be a Tibetan borrowing.


\begin{table}[H]
\caption{Examples of anticausative in Japhug}\label{tab:anticausative} \centering
\begin{tabular}{lllllllll} \toprule
basic verb  & &derived  verb &\\
\midrule
\ipa{ftʂi}  &	to melt (vt)	&		\ipa{ndʐi}  &	to melt (vi)		\\
\ipa{kio}  &	to cause to drop	&		\ipa{ŋgio}  &	to slip		\\
\ipa{prɤt}  &	to break	&		\ipa{mbrɤt}  &		to be broken	\\
\ipa{χtɤr}  &	 to spill	&		\ipa{ʁndɤr}  &		to be spilled	\\
\ipa{tʂaβ}  &	to cause to roll	&		\ipa{ndʐaβ}  &	to roll (vi)		\\
   \ipa{cɯ}  &	 to open 	&		\ipa{ɲɟɯ}  &	 to be opened	 	\\ 
 \bottomrule
\end{tabular}
\end{table}

The semantic difference with the passive can be illustrated with \ipa{pjɤ-mbrɤt} \textsc{ifr-anticaus}:break `it broke (of a rope)' vs \ipa{pjɤ-k-ɤ-prɤt-ci} \textsc{ifr-evd-pass-break}-\textsc{evd}`(someone) broke it'.

There are two antipassive prefixes \ipa{rɤ--} and \ipa{sɤ--}, which convert a transitive verb into an intransitive one whose S corresponds to the A of the base verb. The prefix \ipa{rɤ--} is used in two cases: when the P of the base verb designates a non-human (\ipa{rɤt} `write (tr)' $\rightarrow$ \ipa{rɤ-rɤt} `write (intr)', \ipa{ɕar} `search, look for (tr)' $\rightarrow$ \ipa{rɤ-ɕar} `look for things (intr)') or in the case of ditransitive verbs, even when the P corresponds to the (human) recipient (\ipa{mbi} `give X to' $\rightarrow$ \ipa{rɤmbi} `give X to someone'). The antipassive \ipa{sɤ--} derives either a dynamic verb whose demoted argument is necessarily human (\ipa{sɤ-ɕar} `look for people (intr)') or a stative verb, whose demoted P can be either human or non-human (\ipa{sat} `kill' $\rightarrow$ \ipa{sɤ-sat} `have a killing power, be deadly'). 

A historically related derivation is the deexperiencer \ipa{sɤ--}, which derives a stative verb from any stative verb, changing an experiencer S into the stimulus (\ipa{ŋgio} `slip' $\rightarrow$ \ipa{sɤ-ŋgio} `be slippery').

\subsubsection{Incorporation}
Japhug has an incorporation-like construction in which noun-verb nominal compounds are turned into verbs by means of a denominal prefix (\citealt{jacques12incorp}). For instance, from the noun \ipa{cɯ} `stone' and the verb\ipa{pʰɯt} `pluck, take out' one can derive an action nominal    \ipa{cɯpʰɯt} `clearing the stones', which can in turn be made into an incorporating verb by denominal derivation  \ipa{ɣɯ-cɯpʰɯt } `take out stones (out of the field)'.

\begin{exe}   
\ex
\begin{xlist}[(ii)]
\exi{(i)} 
\gll     \ipa{cɯ-pʰɯt} \ipa{nɯ-βzu-t-a}  \\
  stone-clearing \textsc{pfv}-do-\textsc{pst}-\textsc{1sg} \\
\exi{(ii)} 
\gll     \ipa{nɯ-ɣɯ-cɯ-pʰɯt-a}  \\
  \textsc{pfv-denominal}-stone-take.out-\textsc{1sg} \\
\exi{(iii)} 
\gll     \ipa{cɯ} \ipa{nɯ-pʰɯt-a}  \\
  stone \textsc{pfv}-take.out-\textsc{1sg} \\
  \end{xlist}
 \glt   I cleared the stones (from the field). 
\end{exe}   

The three constructions above have a slightly different meaning: the compound action nominal with light verb construction (i) implies that the action took considerable time, while the incorporating construction (ii) cannot be used if the action refers to specific stones that have been previously mentioned in discourse, contrary  to (iii).

\subsubsection{Causative}
There are two productive causative prefixes in Japhug, \ipa{sɯ--} and \ipa{ɣɤ--} (\citealt{jacques15causative}).

The causative   \ipa{sɯ--} prefix is extremely productive, and can be added to nearly any verb, including recent borrowings from Tibetan and Chinese. It presents  considerable allomorphy, and numerous irregular forms. It has four regular allomorphs \ipa{sɯ-}, \ipa{sɯɣ-}, \ipa{sɯx-} and \ipa{z-} depending on the following element. The \ipa{z-} allomorph appears before all prefixes with sonorant initial (such as the denominal \ipa{nɯ--}, \ipa{rɯ--}, \ipa{ɣɤ--} prefixes, the antipassive \ipa{rɤ--} etc). The allomorphs \ipa{sɯɣ-}/\ipa{sɯx-}  only appear  with intransitive verbs stems with an onset containing neither a velar or a consonant cluster. The \ipa{sɯ--} allomorph appears in all other cases, and is considered here as the default form.


The causative   \ipa{sɯ--}  expresses very broad causative meanings, including indirect causative (including by one's inaction), permissive (`allow to') and is also (optionally) used with  instruments marked in the ergative as in \refb{ex:caus.instrument}.


\begin{exe}
\ex \label{ex:caus.instrument}
\gll  \ipa{ɯ-χto} 	\ipa{nɯ} 	\ipa{mbrɯtɕɯ} 	\ipa{kɯ} 	\ipa{kú-wɣ-sɯ-rkʰe}  \\
\textsc{3sg.poss}-slit \textsc{dem} knife \textsc{erg}  \textsc{ipfv-inv-cause}-carve \\
 \glt  The slit is carved with a knife. (Colored belts 13)
\end{exe} 

Causative verbs with irregular allomorphs such as \ipa{jtsʰi} `give to drink'  from \ipa{tsʰi} `drink' generally have unpredictable semantics. The regular form is always possible at least to express an action with an instrument (\ipa{sɯ-tsʰi} `drink with').

When the causative \ipa{sɯ--} is applied to a transitive verb, whenever the causee (the S/A of the original verb) or the P (of the original verb) is first or second person, it is always encoded on the verb, resulting in ambiguous forms. As an example, the causative of \ipa{qur} ``to help''  \ipa{sɯ-qur} can have two meanings in specific cases (\ref{ex:caus:show.2>3>1}).

\begin{exe} 
\ex \label{ex:caus:show.2>3>1}
\gll   \ipa{tɤ-kɯ-sɯ-qur-a-ndʑi}  \\
 \textsc{pfv-2$\rightarrow$1-caus}-help-\textsc{1sg-du}  \\
 \glt  You_d caused me to help him. OR You_d caused him to help me. 
\end{exe} 

The causative \ipa{ɣɤ--} is more restricted, and can only occur with some stative verbs, such \ipa{wxti} `be big' $\rightarrow$ \ipa{ɣɤwxti} `make bigger'. Although \citet{jackson06paisheng, jackson14morpho} reports a semantic contrast between the cognate prefixes \ipa{sə}-- and \ipa{wɐ}-- in Tshobdun, this contrast appears to have been lost in the variety of Japhug under study.

Some labile verbs, such as \ipa{mto} (which means `see' as a transitive verb and `have the ability to see' as an intransitive stative verb') have distinct causative forms depending on the base meaning: \ipa{sɯ-mto} `cause to see, show' is based on the transitive \ipa{mto}, while \ipa{ɣɤ-mto} `cause (a blind person) to recover sight' is based on the stative \ipa{mto}.


\subsubsection{Applicative and tropative}
In addition to the causative, Japhug has two valency-increasing voice derivation, the applicative \ipa{nɯ--} / \ipa{nɯɣ--} and the tropative \ipa{nɤ--} /\ipa{nɤɣ--} (\citealt{jacques13tropative}).

The applicative converts an intransitive verb into a transitive one whose A corresponds to the S of the base verb and whose P can be either a stimulus or a recipient (for instance \ipa{mu} `be afraid' $\rightarrow$ \ipa{nɯɣ-mu} `be afraid of'). 

The tropative is an extremely productive derivation that can be applied to all stative verbs describing a quality (i.e., it is a criterion for defining an adjective subclass among stative verbs), deriving a transitive whose P corresponds to the S of the base verb (like a causative) but whose A is an experiencer, not an agent. The tropative verb has the meaning `to find / consider to be', thus \ipa{xtɕi} `be small' $\rightarrow$ \ipa{nɤ-xtɕi} `to find small'.

\subsubsection{Other voice markers}
In addition to the voice markers described above, Japhug also has a reflexive \ipa{ʑɣɤ--} distinct from the reciprocal (expressed by combining the prefix \ipa{a--} with a partially reduplicated verb stem).

We also find three verbal derivations with modal meanings: the abilitative \ipa{sɯ--/z--} (expressing the meaning `be able to X', \ipa{nɤɕqa} `bear' $\rightarrow$ \ipa{z-nɤɕqa} `be able to bear)', the stative facilitative \ipa{ɣɤ--} (\ipa{wxti} `be big' $\rightarrow$ \ipa{ɣɤ-wxti} `grow big easily') and the passive facilitative \ipa{nɯɣɯ--} (\ipa{ntɕʰoz} `use' $\rightarrow$ \ipa{nɯɣɯ-ntɕʰoz} `be easy to use').
   
\subsection{Autobenefactive-spontaneous}

The autobenefactive-spontaneous \ipa{nɯ--} prefix expresses a wide range of meanings. First, it is used to express that the S/A is affected by the action; it appears in particular with transitive verbs when the P is a body part of or belongs to the S/A (\ref{ex:pWnWXtCi}).

\begin{exe}
\ex \label{ex:pWnWXtCi}
\gll 
\ipa{nɤ-ku} 	\ipa{pɯ-nɯ-χtɕi} \\
\textsc{2sg.poss}-head \textsc{imp-auto}-wash \\
\glt Wash your head.
\end{exe}

The prefix \ipa{nɯ--} can also be used to express involuntary actions, as in \refb{ex:panWClWG}.

\begin{exe}
\ex \label{ex:panWClWG}
\gll \ipa{ɯ-qom} 	\ipa{ci} 	\ipa{pa-nɯ-ɕlɯɣ} 	\ipa{ɲɯ-ŋu,} \\
\textsc{3sg.poss}-tear \textsc{indef} \textsc{pfv:3$\rightarrow$3'-auto}-drop \textsc{sens}-be \\
\glt She shed a tear (unvoluntarily).(Kunbzang 228)
\end{exe}


The prefix \ipa{nɯ--} is also used to express an action occurring by itself, without exterior agent/causer, even if the action is voluntary and controllable, as in \refb{ex:pjWnWmtsaRa}.

\begin{exe}
\ex \label{ex:pjWnWmtsaRa}
\gll 
\ipa{aʑo} 	\ipa{pjɯ-kɯ-ɣɤrat-a-nɯ} 	\ipa{mɤ-ra} 	\ipa{ma} 	\ipa{aʑo} 	\ipa{pjɯ-nɯ-mtsaʁ-a} 	\ipa{jɤɣ} \\
\textsc{1sg} \textsc{ipfv:down}-2$\rightarrow$1-throw-\textsc{1sg-pl} \textsc{neg-}need\factual{} because \textsc{1sg} \textsc{neg-ipfv:down-auto}-jump-\textsc{1sg} be.possible\factual{} \\
\glt You don't need to throw me in there, I will jump of my own free will. (Nyima Wodzer2002, 124)
\end{exe}

Finally, \ipa{nɯ--} also has a permansive meaning (translatable as `still'), as \refb{ex:pjAnWGAwu}.


\begin{exe}
\ex \label{ex:pjAnWGAwu}
\gll
\ipa{tɕʰeme} 	\ipa{nɯ} 	\ipa{ɲɤ-nɯkʰɤda} 	\ipa{ri,} 	\ipa{mɯ-pjɤ-pʰɤn,} 	\ipa{tɕʰeme} 	\ipa{nɯ} 	\ipa{pjɤ-nɯ-ɣɤwu} 	\ipa{ɕti,} \\
girl \textsc{dem} \textsc{infr}-convince \textsc{lnk} \textsc{neg-infr}-be.efficient girl \textsc{dem} \textsc{ipfv.ifr-auto}-cry  be.\textsc{affirm:fact} \\
\glt She (tried to) comfort the girl, but it was for nothing, the girl was still crying. (Bean and linen, 48)
\end{exe} 

The autobenefactive-spontaneous prefix has no influence on the verb transitivity.

\subsection{Associated motion}
Japhug has two associated motion prefixes \ipa{ɕɯ--} and \ipa{ɣɯ--} expressing a motion taking place before the action expressed by the verb to which the prefixes are attached 
(\citealt{jacques13harmonization}). The translocative \ipa{ɕɯ--} and the cislocative \ipa{ɣɯ--} respectively express the meaning `go to ..., go and ...' or `come to..., come and ...', as in example \refb{ex:cisloc}.


\begin{exe}
\ex \label{ex:cisloc}
\gll
\ipa{tɤrɤku} 	\ipa{kɯ-fse} 	\ipa{ɣɯ-tu-ndze} 	\ipa{nɯra} 	\ipa{mɤ-ŋgrɤl.} \\
crops \textsc{nmlz}:S-be.like \textsc{cisloc-ipfv}-eat[III] \textsc{dem:pl} \textsc{neg-}be.usually.the.case\factual{} \\
\glt It does not come to eat crops and things like that.
\end{exe}

These prefixes are semantically close to the motion verb construction with \ipa{ɕe} `go' and \ipa{ɣi} `come' and a complement with the verb in the S/A nominalization form. A crucial difference, however, can be found in the perfective:  motion verb constructions such as \refb{ex:motion.verb} imply that the motion event has taken place, but do not specify whether the action of the complement clause has taken place or not. Using the associated motion form \ipa{ɣɯ-tɤ-tɯ-nɤma-t} instead would mean `What have you done? (since you came)' .

\begin{exe}
\ex \label{ex:motion.verb}
\gll
\ipa{tɕʰi} 	\ipa{ɯ-kɯ-nɤma} 	\ipa{jɤ-tɯ-ɣe?} \\
what \textsc{3sg-nmlz:}S/A-do \textsc{pfv-2}-come[II] \\
\glt What have you come to do?
\end{exe}
%retrouver l'exemple


With causative verbs, the associated motion can refer either to the motion of the causer or that of the causee as in example \refb{ex:assoc.motion2}.

  \begin{exe}
\ex \label{ex:assoc.motion2}
\gll
\ipa{tɕe} 	\ipa{kupa} 	\ipa{chu} 	\ipa{nɯra} 	\ipa{athi} 	\ipa{pɕoʁ} 	\ipa{nɯra,} 	\ipa{ɯ-pɕi} 	\ipa{nɯra} 	\ipa{kɯ} 	\ipa{kɯreri} 	\ipa{ɣɯ-chɯ-sɯ-χtɯ-nɯ} 	\ipa{ŋu.}  \\
\textsc{lnk} Chinese \textsc{loc} \textsc{dem:pl} downstream direction \textsc{dem:pl} \textsc{3sg}-outside  \textsc{dem:pl}  \textsc{erg} here \textsc{cisloc-ipfv:downstream-caus}-buy-\textsc{pl} be\factual{} \\
\glt People from the Chinese areas, people from outside send people to come here to buy (matsutake and sell them in the areas downstream). (20 grWBgrWB 58)
  \end{exe} 


\subsection{Nominalization and other non-finite forms}

Japhug has a rich set of non-finite verb forms: participles, action nominalizer, infinitives and converbs. These forms have in common the facts that (i) they can only mark person by means of possessive prefixes like nouns, cannot take the inverse prefix, and cannot encode two arguments in the case of transitive verbs; (ii) they cannot use stem 3; (iii) they cannot serve as the predicate of a main clause.

There are three participles in Japhug: the S/A-participle \ipa{kɯ--}, the P-participle \ipa{kɤ--} and the oblique participle \ipa{sɤ--}. In the case of transitive verbs, the S/A-participle takes a possessive prefix (see Table \ref{tab:pronoun}) coreferent with the P (\ipa{ndza} `eat' $\rightarrow$ \ipa{ɯ_i-kɯ-ndza} `the one who eats it_i'), while the P-participle optionally takes a possessive prefix coreferent with the A (\ipa{ɯ_i-kɤ-ndza} `what he/it_i eats'). The oblique participle \ipa{sɤ--} can refer to  the instrument, the place, time or recipient of an action (see the section on relative clauses).

The participles can be used with negation and orientational prefixes (only set A and B); participles with perfective orientational prefixes take stem 2 if they have a distinct stem.

In addition, Japhug verbs have infinitives in \ipa{kɤ--} (for dynamic verbs allowing an animate argument) or \ipa{kɯ--} (for intransitive stative verbs or verbs only used with inanimate arguments). While the infinitives are superficially similar to the S/A- and P-participles, their morphology and uses are different. They cannot take any possessive prefix, and in the case of intransitive dynamic verbs, the forms in \ipa{kɤ--} cannot be interpreted a P-participles since such verbs only have an S argument and thus only a participle in \ipa{kɯ--}. Moreover, in some complements clauses, transitive verbs have a bare infinitive made of the bare stem 1 with a possessive prefix coreferent with the P (\citealt{jacques14antipassive}).

Finally, we find three converbs which are used in various clause linking constructions (see \citealt{jacques14linking}). First, the perfective converb is built by combining a set B orientational prefix (normally restricted to imperfective tenses!),  the prefix \ipa{--tɯ--} and the verb stem 1, as in \ipa{mto} `see' $\rightarrow$ \ipa{pjɯ-tɯ-mto} `as soon as X saw Y'; it cannot receive any person marking. Second, the gerundive is made of the prefix \ipa{sɤ--} and a reduplicated verb stem, as in \ipa{mu} `be afraid' $\rightarrow$ \ipa{sɤ-mɯ\textasciitilde{}mu} `while being afraid'. Third, the purposive converb comprises a possessive prefix coreferent with any core argument (A, P or S), a negative prefix (optional), a set B orientational prefix, the prefix \ipa{sɤ--} and a partially reduplicated verb stem 1, as in \ipa{jmɯt} `forget' $\rightarrow$ \ipa{ɯ-mɤ-ɲɯ-sɤ-jmɯ\textasciitilde{}jmɯt} `in order not to forget'.

\subsection{Generic person} \label{sec:generic}
Apart from argument-demoting voice prefixes, another way of expressing indefinite arguments in Japhug is generic person marking. It is available only for generic human referents, not for inanimates or animals.

This type of person marking follows an ergative alignment: generic S and P arguments are marked with the prefix \ipa{kɯ--} (see examples \ref{ex:pWkWNu} and \ref{ex:kukWnWfse}) while generic A argument is marked with the inverse prefix \ipa{wɣ--} (\ref{ex:tuwGndza.sna}; the verb \ipa{ti} `` say' is irregular in that its generic A form \ipa{tu-kɯ-ti} takes the \ipa{kɯ--} prefix, see example \ref{ex:tWrpW}).  Verbs with generic person markers cannot take any other person or number markers.


\begin{exe}
\ex \label{ex:pWkWNu}
\gll
\ipa{tɕeri} 	\ipa{tɤ-pɤtso} 	\ipa{pɯ-kɯ-ŋu} 	\ipa{tɕe,} 	\ipa{nɯ} 	\ipa{kɤ-ndza} 	\ipa{wuma} 	\ipa{ʑo} 	\ipa{pɯ-kɯ-rga.} \\
but \textsc{indef.poss-}child \textsc{pst.ipfv-genr}:S/P-be \textsc{lnk} \textsc{dem} \textsc{inf}-eat really \textsc{emph} \textsc{pst.ipfv-genr}:S/P-like \\
\glt When (we) were children, (we) liked it a lot. (12 ndZiNgri, 135)
\end{exe}


\begin{exe}
\ex \label{ex:kukWnWfse}
\gll
\ipa{tɕe}  	\ipa{li}  	\ipa{nɯ}  	\ipa{tɯrme}  	\ipa{kɯnɤ}  	\ipa{ku-kɯ-nɯfse}  	\ipa{ɲɯ-ŋu,}\\
\textsc{lnk} again \textsc{dem} people also \textsc{ipfv-genr:S/P}-recognize \textsc{sens}-be\\
\glt  (The monkey) recognizes people. (19 GzW2, 17)
\end{exe}

\begin{exe}
\ex \label{ex:tuwGndza.sna}
\gll
\ipa{tɯrme}  	\ipa{kɯ}  	\ipa{tú-wɣ-ndza}  	\ipa{mɤ-sna.}   \\
people \textsc{erg} \textsc{ipfv-inv}-eat \textsc{neg-}be.fine\factual{} \\
\glt People cannot eat it. (11 paRzwamWntoR, 39)
\end{exe}

\subsection{Transitivity}
The single most important feature in Japhug verbal morphology is transitivity. As seen above, transitive verbs differ from intransitive verbs in several ways: the S/A-participle can take a possessive prefix coreferent with the P, the perfective third person forms take set C as orientational prefixes (or the inverse prefix), and verbs whose base stem is in an open syllable have the past \ipa{-t-} suffix in \textsc{1sg$\rightarrow$3} and \textsc{2sg$\rightarrow$3} in past forms and stem 3 alternation in non-past forms in some cases.

There is never any ambiguity as to whether a particular verb is morphologically  transitive or intransitive, but some verbs, such as \ipa{taʁ} `weave' or \ipa{mɯrkɯ} `steal' are labile and can be conjugated either transitively or intransitively (\citealt{jacques12demotion}). All labile verbs in Japhug are A-preserving; there are no examples of P-preserving lability.

There is a class of semi-transitive verbs, such as \ipa{rga} `like' or \ipa{aro} `have, own' that are morphologically intransitive but present some syntactic transitivity. While their finite conjugation is identical to that of intransitive verbs, and while their S does not take ergative marking, they do have P-participles in \ipa{kɤ--} and thus have an object-like argument that can be relativized exactly in the same way as the P of a transitive verb. This argument however, even if first or second person, cannot be marked in the verb morphology.

Ditransitive verbs can be either secundative (indexing the recipient like the P of a monotransitive verb, like \ipa{mbi} `give'), indirective (indexing the theme like the P of a monotransitive verb, and treating the recipient as in oblique not indexed in verb morphology, as \ipa{tʰu} `ask') or both (causative verbs deriving from transitive base verbs) from the point of view of person marking on the verb. No more than two arguments can be encoded by verbal morphology. The noun phrases corresponding to recipients of indirective verbs receive dative flagging by means of the relator nouns \ipa{ɯ-ɕki} or \ipa{ɯ-pʰe}, but cannot be marked on the verb, even if first or second person.

The relativization patterns however slightly differ from argument indexation on the verb.
In the case of secundative verbs, both the R and the T of secundative verbs are relativizable in the same way as the P of a monotransitive verb, even though only the R is indexed on the verb. As for indirective verbs, the R is relativized by means of the oblique participle (see the section on relativization).

 

\section{Nominal morphology}
Japhug nouns can be divided into three sub-classes: simple nouns, inalienably possessed nouns and nouns with numeral prefixes (classifiers). Only the first two classes are discussed in this sketch. 

The same set of possessive prefixes (see Table \ref{tab:pronoun}) is used for all nouns, but inalienably possessed nouns cannot be used on their own without one of these prefixes. When there is no definite possessor, the indefinite possessive prefixes \ipa{tɤ--} or \ipa{tɯ--} are used. It is the citation form of inalienably possessed nouns (\ipa{tɤ-lu} `milk', \ipa{tɯ-ŋga} `clothes', \ipa{tɤ-rpɯ} `uncle', \ipa{tɯ-ci} `water'). The choice of the prefix \ipa{tɤ--} vs \ipa{tɯ--} is lexically determined.  When a specific possessor is present, the indefinite prefix is replaced by the appropriate possessive prefix (\ipa{ɯ-lu} `her/its milk (from her nipple)', \ipa{a-ŋga} `my clothes', \ipa{nɤ-rpɯ} `your uncle', \ipa{ɯ-ci} `its juice'). It is possible to turn an inalienably possessed noun into an alienably possessed one by prefixing a definite possessive prefix to the indefinite one (\ipa{ɯ-tɤ-lu} `his milk (to drink)', \ipa{ɯ-tɯ-ci} `its water (of irrigated water, to a plant)'). Simple nouns cannot take indefinite possessive prefixes.


\begin{table}[H] \centering
\caption{Pronouns and possessive prefixes }\label{tab:pronoun}
\begin{tabular}{lllllllll} 
\toprule
 Free pronoun & Prefix & Person\\
\midrule
 \ipa{aʑo},    \ipa{aj} &	\ipa{a--}  &		1\textsc{sg} \\
\ipa{nɤʑo},  \ipa{nɤj} &	\ipa{nɤ--}  &			2\textsc{sg}\\
\ipa{ɯʑo}  &	\ipa{ɯ--}  &			3\textsc{sg}\\
\midrule
\ipa{tɕiʑo}  &	\ipa{tɕi--}  &			1\textsc{du} \\
\ipa{ndʑiʑo}  &	\ipa{ndʑi--}  &		2\textsc{du} \\	
\ipa{ʑɤni}  &	\ipa{ndʑi--}  &		3\textsc{du} \\	
\midrule
\ipa{iʑo}, \ipa{iʑora},   \ipa{iʑɤra}   &	\ipa{i--}  &			1\textsc{pl} \\
\ipa{nɯʑo}, \ipa{nɯʑora},   \ipa{nɯʑɤra}  &	\ipa{nɯ--}  &			2\textsc{pl} \\
\ipa{ʑara}  &	\ipa{nɯ--}  &			3\textsc{pl} \\
\midrule
&  \ipa{tɯ--},  \ipa{tɤ--} & indefinite \\
\ipa{tɯʑo} & \ipa{tɯ--}   &  generic\\
\bottomrule
\end{tabular}
\end{table}

The indefinite possessive prefixes should not be confused with the generic possessive prefix \ipa{tɯ--}, which can be added to any noun, and which is coreferent with the argument marked with generic marking on the verb, as in example \refb{ex:tWrpW}. Note also that inalienably possessed nouns that select the indefinite possessive prefix \ipa{tɤ--} have \ipa{tɯ--} instead when the possessor is generic (\ipa{tɤ-rpɯ} `an uncle' vs \ipa{tɯ-rpɯ} `one's uncle').

\begin{exe}
\ex \label{ex:tWrpW}
\gll
 \ipa{tɯ-rpɯ} 	\ipa{ɯ-rɟit} 	\ipa{ɯ-ɕki} 	\ipa{tɕe} 	\ipa{tɕe} 	``\ipa{a-rpɯ} \ipa{a-ɬaʁ}" 	\ipa{tu-kɯ-ti} 	\ipa{ŋu.} \\
\textsc{genr.poss}-uncle \textsc{3sg.poss}-offspring \textsc{3sg-dat} \textsc{lnk} \textsc{lnk} \textsc{1sg.poss}-uncle \textsc{1sg.poss}-aunt \textsc{ipfv-genr}-say  be\factual{} \\
\glt One has to say ``my maternal uncle, my maternal aunt" to one's maternal uncle's sons and daughters. (140425 kWmdza01, 69)
\end{exe}

Some possessed nouns have restrictions on the interpretation of the possessor. Thus, the possessive prefix of the noun \ipa{tɤ-pɤro} `present' can only refer to the person giving the present, never the recipient: \ipa{a-pɤro} \textsc{1sg.poss}-\textit{present} can only mean `my present (to him, to you, etc)' not `the present (you, he gave) me'.

Japhug has a very rich denominal morphology, and also has a productive comitative adverb derivation with the prefix \ipa{kɤɣɯ--} or \ipa{kɤ́--} and reduplicated noun stem, as in \ipa{tɤ-rtaʁ} `branch' $\rightarrow$ \ipa{kɤɣɯrtɯrtaʁ} `with branches'. These adverbs probably originate historically from the S-participle of denominal propriety verbs with the prefix \ipa{aɣɯ--} (\ipa{tɤ-rtaʁ} `branch' $\rightarrow$ \ipa{aɣɯrtɯrtaʁ} `have many branches').

\section{The noun phrase} 
Noun phrases in Japhug follow the general template given in \refb{ex:noun.template}, as illustrated by examples \refb{ex:atCW.XsWm} and \refb{ex:tCheme.XsWm}.

\begin{exe}
\ex \label{ex:noun.template}
\glt \textsc{dem-noun^{modifier}-noun^{head}-adj-num-dem}
\end{exe}

The demonstratives \ipa{ki} `this' and \ipa{nɯ} `that' can either appear in the beginning or the end of the noun phrase, or be repeated at both ends.

\begin{exe}
\ex \label{ex:atCW.XsWm}
\gll
[\ipa{ki} 	\ipa{a-tɕɯ} 	\ipa{χsɯm} 	\ipa{ki}]_{\textsc{np}}  \\
this \textsc{1sg.poss}-son three this \\
\glt These three sons of mine.
\end{exe}

Adjectival stative verbs can serve as noun modifier only in the S/A-participle (in \ipa{kɯ--}), and the [noun+adjective] constituent should be analyzed as a head-internal relative clause.

\begin{exe}
\ex \label{ex:tCheme.XsWm}
\gll
[[\ipa{tɕʰeme}^{head} 	\ipa{kɯ-mpɕɯ\textasciitilde{}mpɕɤr}]_{\textsc{rc}} 	\ipa{ʑo} 	\ipa{χsɯm}]_{\textsc{np}} 	\ipa{ɲɯ-nɯ-ɬoʁ-nɯ}  \\
girl \textsc{nmlz:S/A-emph}\textasciitilde{}be.beautiful \textsc{emph} three \textsc{ipfv-auto}-come.out-\textsc{pl} \\
\glt Three beautiful girls come out of it (each day).
\end{exe}

Nouns modifying other nouns normally appear before them as in \refb{ex:GW.nWrJAlpu}. The head noun has an obligatory possessive prefix coreferent with the modifier noun, and a genitive postposition \ipa{ɣɯ} can be optionally added.

\begin{exe}
\ex \label{ex:GW.nWrJAlpu}
\gll
\ipa{aʑo}  	[\ipa{ndzaʁlaŋ}  	\ipa{ɣɯ}  	\ipa{nɯ-rɟɤlpu}]_{\textsc{np}}  	\ipa{ŋu-a}  	\ipa{tɕe,}  
 \\
\textsc{1sg} Jambudvîpa \textsc{gen} \textsc{3pl.poss}-king  be\factual{}-\textsc{1sg} \textsc{lnk} \\
\glt I am the king of the people of Jambudvîpa. (2011-4-smanmi, 243)
\end{exe}

There is a  restricted subclass of possessed nouns referring to old or ragged objects such as \ipa{ɯ-do} `old (of animals)', \ipa{ɯ-mbe} `old (of clothes)', \ipa{ɯ-ɴqra} `shabby' which are strictly postnominal. Although in translation they would correspond to adjectives, they are not noun modifiers syntactically -- rather, it is a structure comparable to the marginal construction in English exemplified by forms such as `an old rag of a gown'.


\begin{exe}
\ex \label{ex:Wmbe}
\gll
[\ipa{tɯ-rcu} 	\ipa{ɯ-mbe} 	\ipa{ci}]_{\textsc{np}}	\ipa{to-ŋga} \\
\textsc{indef.poss}-leather.jacket \textsc{3sg.poss}-old \textsc{indef} \textsc{ifr}-wear \\
\glt He wore an old leather jacket.
\end{exe}

Japhug is an exclusively postpositional language. Common postpositions include the ergative / instrumental \ipa{kɯ}, the comitative \ipa{cʰo} `with' and the locatives \ipa{zɯ} and \ipa{tɕu}. The locative postpositions have a vague locative meaning, and can indicate either a fixed position within or on the surface of an entity, or motion into or from it (they differ in that \ipa{zɯ} cannot follow a demonstrative). Relator nouns are also similar to postpositions, but are morphologically possessed nouns with an obligatory possessive prefix, as with the dative \ipa{ɯ-ɕki} or \ipa{ɯ-pʰe} (usage depends on  idiolects) or locative relator nouns such as \ipa{ɯ-ŋgɯ} `inside' or \ipa{ɯ-taʁ} `on', and can always be optionally followed by the locative postposition \ipa{zɯ}.

\section{Simple clauses} 

Japhug has strict verb-final word order. Only a few elements can appear after a verb: sentence final particles, the adverb \ipa{ntsɯ} `always', ideophones and afterthought constituents.

Transitive sentences tend to have at least one covert argument (which is then always interpreted as being definite), but when all arguments are overt the canonical word order is A-P-verb (though P-A order is also attested, depending of information structure), the A being obligatorily marked with the ergative \ipa{kɯ} unless it is a first or second person pronoun (see \ref{ex:WkaGW}).

\begin{exe}
\ex \label{ex:WkaGW}
\gll
\ipa{rɟɤlpu} 	\ipa{nɯ} 	\ipa{kɯ} 	\ipa{pɣɤtɕɯ} 	\ipa{nɯ} 	\ipa{ɯ-kaɣɯ} 	\ipa{ɯ-ŋgɯ} 	\ipa{nɯ} 	\ipa{tɕu} 	\ipa{pjɤ-nɯ-rku.} \\
king \textsc{dem} erg bird \textsc{dem} \textsc{3sg.poss}-silver.necklace \textsc{3sg}-inside \textsc{dem} \textsc{loc} \textsc{ifr:down-auto}-put.in \\
\glt The king put the bird in his necklace. (2002qaCpa, 202)
\end{exe}

Right-dislocated elements (afterthoughts) keep their postpositions or relator nouns, as in \refb{ex:tungo.Nu}.

\begin{exe}
\ex \label{ex:tungo.Nu}
\gll
\ipa{sla} 	\ipa{tu-ngo} 	\ipa{ŋu} 	\ipa{tu-ti-nɯ} 	\ipa{ŋu,} 	\ipa{kɯrɯ} 	\ipa{ra} 	\ipa{kɯ} \\
moon \textsc{ipfv}-be.sick  be\factual{} \textsc{ipfv}-say-\textsc{pl}  be\factual{}  Tibetan \textsc{pl} \textsc{erg} \\
\glt They say that the moon is sick, the Tibetans. (29 mWBZi, 149)
\end{exe}

\section{Comparative}

Japhug comparative constructions are illustrated by example \refb{ex:comp1}: the standard is marked by the comparative postpositions \ipa{sɤz} or \ipa{staʁ}, while the comparee can optionally be followed by a marker \ipa{kɯ} homophonous with the ergative / instrumental (\citealt{jacques16comparative}).

\begin{exe}
\ex \label{ex:comp1}
\gll  \ipa{ɯ-ʁi}   	\ipa{sɤz}   	\ipa{ɯ-pi}   	\ipa{nɯ}   	\ipa{\textbf{kɯ}}   	\ipa{mpɕɤr}     \\
\textsc{3sg.poss}-younger.sibling \textsc{comparative} \textsc{3sg.poss}-elder.sibling \textsc{dem} \textsc{erg?}   be.beautiful\factual{} \\
\glt The elder sister is more beautiful than the younger sister. (elicited)
\end{exe}

There is a degree construction in which adjectival stative verbs are nominalized with the prefix \ipa{tɯ--} and take a possessive prefix coreferent with the S, followed either by a finite verb indicating degree (most commonly \ipa{saχaʁ} `be extremely ...') or by a clause containing a simile describing the degree of the property.


\begin{exe}
\ex \label{ex:YWsWxtCur}
\gll 
\ipa{mtɕʰi}  	\ipa{ɯ-mat}  	\ipa{rca}  	\ipa{ɯ-tɯ-tɕur}  	\ipa{saχaʁ.}  	\ipa{ɯ-tɯ-tɕur}  	\ipa{\textbf{kɯ}}  	[\ipa{tɯ-kɯr}  	\ipa{ɯ-ŋgɯ}  	\ipa{lú-wɣ-rku}  	\ipa{qʰe}  	\ipa{maka}  	\ipa{ɲɯ-sɯ-ɤmɯzɣɯt}  	\ipa{qhe,}  	\ipa{tɯ-pʰoŋbu}  	\ipa{ra}  	\ipa{kɯnɤ}  	\ipa{ɲɯ-sɯx-tɕur}  	\ipa{kɯ-fse}  	\ipa{ɕti}]  \\
sea.buckthorn \textsc{3sg.poss}-fruit \textsc{top} \textsc{3sg-nmlz:degree}-be.sour  be.extremely\factual{} \textsc{3sg-nmlz:degree}-be.sour \textsc{erg?} \textsc{indef:poss}-mouth \textsc{3sg}-inside \textsc{ipfv:upstream-inv}-put.in \textsc{lnk} at.all \textsc{ipfv-caus}-be.evenly.distributed \textsc{lnk} \textsc{indef:poss}-body \textsc{pl} also \textsc{ipfv-caus}-be.sour \textsc{nmlz:S/A}-be.like  be.\textsc{affirmative}\factual{} \\
\glt `The fruit of the sea-buckthorn is very sour, so sour that when one puts it in one's mouth, it makes it completely (sour), and it is as if one's (whole) body became sour.' (09 mi, 66)
\end{exe}

Superlative can be expressed by the adverb \ipa{stu} `most', but the most common construction is a construction combining the negative existential copula with a nominalized predicate and an equative adjunct in \ipa{kɯ-fse} `like ...' as in \refb{ex:kWtu.me}.

\begin{exe}
\ex \label{ex:kWtu.me}
\gll 
\ipa{kʰɯna}  	\ipa{kɯ-fse}  	\ipa{ʑo} 	\ipa{ɯ-tsʰuxtoʁ}  	\ipa{kɯ-tu}  	\ipa{me}  	\ipa{kʰi} \\  
dog \textsc{nmlz}:S/A-be.like  \textsc{emph} \textsc{3sg.poss}-loyalty \textsc{nmlz}:S/A-exist  not.exist\factual{} \textsc{hearsay} \\
\glt The dog is the most loyal animal (there is no animal whose loyalty is like that of a dog) (05-khWna, 5)
\end{exe}




\section{Relativization}
Japhug relative clauses can be classified according to two main criteria: finiteness of the verb in the clause and  place / presence of head noun.

Both head-internal (see examples \ref{ex:WkWnWmbrApW}) and prenominal relatives (\ref{ex:tajmag}, \ref{ex:inverse.relative}) are found in Japhug. Head-internal relatives are possible for the relativization of core arguments and possessors, while prenominal relatives are required for all oblique arguments and adjuncts. Prenominal relatives are also possible for core arguments, except S in the case of stative verbs. In this section, the head nouns are indicated in bold.

Non-finite relative clauses have a verb in one of the three participles. The S/A-participle in \ipa{kɯ--} is used for relativizing the S, the A or the possessor of an argument, as in \refb{ex:WkWnWmbrApW} . Note that the presence of ergative marking on the A shows that the noun phrase \ipa{tɤ-pɤtso} \ipa{ci} `a boy' belongs to this head-internal relative, as the main verb \ipa{jɤ-ɣe} `he came' is intransitive.  

\begin{exe}
   \ex  \label{ex:WkWnWmbrApW}
\gll [\ipa{\textbf{tɤ-pɤtso}}  	\ipa{ci}  	\ipa{kɯ}  	<yangma> 	\ipa{ɯ-kɯ-nɯmbrɤpɯ}]_{\textsc{rc}} 	\ipa{ci}  	\ipa{jɤ-ɣe}  \\
\textsc{indef.poss}-child \textsc{indef} \textsc{erg} bicycle \textsc{3sg-nmlz:A}-ride \textsc{indef} \textsc{pfv}-come[II] \\
\glt A boy who was riding a bicycle arrived. (Pear story, Chenzhen, 5)
\end{exe}

The P-participle in \ipa{kɤ--} is used for  P (\ref{ex:tajmag}), for the object of semi-transitive verbs, and either the T or the R of secundative ditransitive verbs.

\begin{exe}
   \ex \label{ex:tajmag}
   \gll
[\ipa{aʑo}  	\ipa{a-mɤ-kɤ-sɯz}]_{\textsc{rc}}  	\ipa{\textbf{tɤjmɤɣ}}  	\ipa{nɯ}  	\ipa{kɤ-ndza}  	\ipa{mɤ-naz-a}\\
\textsc{1sg} \textsc{1sg-neg-nmlz:P}-know mushroom \textsc{dem} \textsc{inf}-eat \textsc{neg}-dare:\textsc{fact}-\textsc{1sg}\\
\glt I do not dare to eat mushrooms that I do not know. (23 mbrAZim, 103)
\end{exe}

The oblique participle can occur in relatives whose relativized element is a time or place adjunct, an instrument (marked with the ergative / instrumental \ipa{kɯ} in the original sentence),  the recipient of an indirective verb (\ref{ex:WsAfCAt}).

\begin{exe}
\ex \label{ex:WsAfCAt}
\gll
[\ipa{ɯ-sɤ-fɕɤt}]\rc{} 
\ipa{pjɤ-me} 	\ipa{qʰe} 	\ipa{tɕe} 	\ipa{tɤ-pɤtso} 	\ipa{ɯ-ɕki} 	\ipa{nɯ} 	\ipa{tɕu} 	\ipa{nɯra} 	\ipa{tɕʰi} 	\ipa{pɯ-kɯ-fse} 	\ipa{nɯra} 	\ipa{pjɤ-fɕɤt.} \\
\textsc{3sg-nmlz:oblique}-tell \textsc{ipfv.ifr}-not.exist \textsc{lnk} \textsc{lnk} \textsc{indef.poss}-child \textsc{3sg-dat} \textsc{dem} \textsc{loc} \textsc{dem:pl} what \textsc{pst-nmlz:S}-be.like  \textsc{dem:pl} \textsc{ifr}-tell \\
\glt She had no one (else) to tell it to, so she told the boy everything that had happened. (140515 congming de wusui xiaohai, 77)
\end{exe} 

Some verbs require oblique arguments marked with the comitative postposition \ipa{cʰo}, and these arguments are also relativized by a participle in \ipa{sɤ--} (example \ref{ex:WsAmWmi}, a headless relative)
\begin{exe}
   \ex \label{ex:WsAmWmi}
 \gll 
\ipa{tɕe}   	[\ipa{ɯʑo}   	\ipa{ɯ-sɤ-ɤmɯmi}]\rc{}   	\ipa{nɯ}   	\ipa{dɤn}   	\ipa{ma}   	\ipa{ca}   	\ipa{kɯ-fse}   	\ipa{qaʑo}   	\ipa{kɯ-fse,}   	\ipa{tsʰɤt}   	\ipa{kɯ-fse,}   	 \ipa{ɯʑo}   	\ipa{cʰo}   	\ipa{kɯ-naχtɕɯɣ}   	\ipa{sɯjno,}   	\ipa{xɕaj}   	\ipa{ma}   	\ipa{mɤ-kɯ-ndza}   	\ipa{nɯ} \ipa{ra}   	\ipa{cʰo}   	\ipa{nɯ}   	\ipa{amɯmi-nɯ}   	\ipa{tɕe,}   \\
\textsc{lnk} it \textsc{3sg-nmlz:oblique}-be.in.good.terms \topic{} be.many:\textsc{fact} because musk.deer \textsc{nmlz:S}-be.like sheep \textsc{nmlz:S}-be.like goat  \textsc{nmlz:S}-be.like it with  \textsc{nmlz:S}-be.identical herbs grass apart.from \textsc{neg-nmlz:A}-eat \textsc{dem} \textsc{pl} with \textsc{dem} be.in.good.term:\textsc{fact}-\textsc{pl} \textsc{lnk} \\
\glt The (animals) that are in good terms with the rabbit are many, it is in good terms with those that only eat grass, like musk deer, sheep or goats. (04 qala1, 33-4)
\end{exe}


 
Finite relative clauses are limited to the relativization of P, the object of semi-transitive verbs, the theme of indirect verbs (\ref{ex:tutianw}), the recipient or the theme of secundative verbs and also locative adjuncts in the case of motion/manipulation verbs (\ipa{ɕe} `go' or \ipa{tsɯm} `take away', as in example \ref{ex:jowGtsWmnW}).

     \begin{exe}
   \ex \label{ex:tutianw}
 \gll \ipa{nɯ}  	[\ipa{\textbf{qajɯ}}  	\ipa{kɯ-ɲaʁ}  	\ipa{tu-ti-a}]_{\textsc{rc}}  	\ipa{nɯ}  	\ipa{nɯ}  	\ipa{kɯ-fse}  	\ipa{ɲɯ-βze}  	\ipa{ɲɯ-ŋu}  \\
\textsc{dem} worm \textsc{nmlz:S}-black \textsc{ipfv}-say-\textsc{1sg} \textsc{dem}  \textsc{dem} \textsc{nmlz:S}-be.like \textsc{ipfv}-grow \textsc{sens}-be \\
\glt The black worm that I was talking about grows like that. (28 kWpAz, 30)
\end{exe}

     \begin{exe}
   \ex \label{ex:jowGtsWmnW}
 \gll
[\ipa{\textbf{kʰa}}  	\ipa{jɤ́-wɣ-tsɯm-nɯ}]_{\textsc{rc}}  	\ipa{nɯnɯ,}  	\ipa{lonba}  	[\ipa{ɕom}  	\ipa{kɯ}  	\ipa{nɯ-kɤ-sɯ-βzu}]_{\textsc{rc}}  	\ipa{\textbf{kʰa}}  	\ipa{pjɤ-ŋu}  \\
house \textsc{pfv-inv}-take.away-\textsc{pl} \textsc{dem} all iron \textsc{erg} \textsc{pfv-nmlz:P-caus}-make house \textsc{ipfv.ifr}-be \\
\glt The house to which he had taken them, it was a house made of iron. (140505 liuhaohan zoubian tianxia, 148)
\end{exe}

While inverse marking does not occur in non-finite relative clauses, it is possible in finite ones as in \refb{ex:jowGtsWmnW} and \refb{ex:inverse.relative}. Inverse marking has no effect on the accessibility to relativization of the arguments (in \ref{ex:inverse.relative}, the ditransitive verb \ipa{sɯxɕɤt} is secundative, and the relativized element is the T). In particular, it does not make any argument or adjunct other than the P or the T (in particular, the A) accessible to relativization with a finite relative clause.

     \begin{exe}
   \ex \label{ex:inverse.relative}
 \gll
\ipa{tɤ-tɕɯ}  	\ipa{nɯ}  	\ipa{kɯ,}  	[\ipa{saŋrɟɤz}  	\ipa{ra}  	\ipa{kɯ}  	\ipa{pɯ́-wɣ-sɯxɕɤt}]_{\textsc{rc}}  	\ipa{ɣɯ}  	\ipa{\textbf{kʰɤndɯn}}  	\ipa{nɯ}  	\ipa{ɯ-rʑaβ}  	\ipa{χsɯm}  	\ipa{nɯ}  	\ipa{pjɤ-sɯxɕɤt}  \\
\textsc{indef.poss}-boy \textsc{dem} \textsc{erg} Buddha \textsc{pl} \textsc{erg} \textsc{pfv-inv}-teach   \textsc{gen} sûtra \textsc{dem} \textsc{3sg.poss}-wife three \textsc{dem} \textsc{ifr}-teach \\
\glt The boy taught his three wives the sûtra that the Buddhas had taught him. (sloXpWn, 354)
\end{exe}


Finite relative clauses present some syntactic features distinguishing them from independent clauses (\citealt{jacques16relatives}). First, the possessive prefixes in the sentence referring to a core argument can be neutralized to the indefinite possessive prefix. In \refb{ex:tApAro} for instance, the first singular possessive form \ipa{a-pɤro} `my present (to him)' is the only possible form in the corresponding main clause, but the indefinite possessive form \ipa{tɤ-pɤro} `a present' can be used instead in the relative clause.

		\begin{exe}
\ex \label{ex:tApAro}
\gll
	[\ipa{tɤ-pɤro}  	\ipa{nɯ-mbi-t-a}]_{\textsc{rc}}  	\ipa{\textbf{tɤ-rɟit}}  	\ipa{nɯ}  	\ipa{a-tɕɯ}  	\ipa{ŋu}   \\
	\textsc{indef.poss}-present \textsc{pfv}-give-\textsc{pst:tr-1sg} 	\textsc{indef.poss}-child \topic{} \textsc{1sg.poss}-son be\factual{} \\
\glt The child to whom I gave a present is my son. (Elicited)
 	  \end{exe} 
 	  
Second, the verb of a finite relative clause can undergo totalitative reduplication (of the first syllabe of the word), a morphological process normally restricted to nominalized verb forms that have a quantifying meaning `all' (in reference to the relativized element, in this case normally the P).

  \begin{exe}
\ex \label{ex:pWpaGWt}
\gll
\ipa{tɕe}  	[\ipa{nɯ} \ipa{ra}  	\ipa{\textbf{tɤrɤkusna}} 	\ipa{nɯ}  	\ipa{pɯ\textasciitilde{}pa-ɣɯt}]_{\textsc{rc}}  	\ipa{nɯ}  	\ipa{lo-ji-ndʑi}  \\
\textsc{lnk} \textsc{dem} \textsc{pl} good.crops \topic{} \textsc{total\textasciitilde{}pfv:3$\rightarrow$3:down}-bring \topic{} \textsc{ifr}-plant-\textsc{du} \\
\glt They^{du} planted all the crops that she had brought (from heaven). (07 deluge, 111)
\end{exe}

Third, evidential marking is neutralized: the inferential cannot be used in relative clauses, only the perfective.
 
\section{Complementation}
In Japhug, complement clauses, like relative clauses, can be either finite or non-finite. Non-finite complement clauses include four types. Infinitive clauses, with the verb prefixed in \ipa{kɤ--}, are found with a wide range of verbs, including modal verbs (\ref{ex:erg.rga}), some phasal verbs, and causative verbs. 


 \begin{exe}
\ex \label{ex:erg.rga}
\gll
[\ipa{paʁ}  	\ipa{ra}  	\ipa{kɯ}  	\ipa{kɤ-ndza}]  	\ipa{wuma}  	\ipa{ʑo}  	\ipa{rga-nɯ}  \\
pig \textsc{pl} \textsc{erg} \textsc{inf}-eat very \textsc{emph}  like\factual{}-\textsc{pl} \\
 \glt Pigs like to eat it. (12 ndZiNgri, 149)
\end{exe}

Note that \ipa{paʁ} \ipa{ra} \ipa{kɯ} `pigs' is marked with ergative \ipa{kɯ} and thus belongs to the complement clause (whose main verb \ipa{ndza} is transitive) and is not an argument of the main clause (whose main verb \ipa{rga} is intransitive, and cannot take arguments in \ipa{kɯ}).


S/A-participial clauses are used for the purposive complements of motion verbs and the complements of the verbs \ipa{ʑɣɤpa} and \ipa{nɯɕpɯz} `pretend' and of the phasal verb \ipa{rɤŋgat} `prepare for, be about to', as in \refb{ex:akWRndW}.


 \begin{exe}
\ex \label{ex:akWRndW}
\gll
\ipa{nɤʑo}  	\ipa{a-kɯ-ʁndɯ}  	\ipa{tu-tɯ-rɤŋgat}  	\ipa{tɕe,}  	\ipa{nɤ-pʰɯŋgɯ}  	\ipa{rdɤstaʁ}  	\ipa{ɲɯ-tɯ-rke}  	\ipa{ɲɯ-ŋu}  \\
\textsc{2sg} \textsc{1sg-nmlz}:S/A-hit \textsc{ipfv}-2-prepare \textsc{lnk} \textsc{2sg.poss}-folds.of.clothes stone \textsc{ipfv}-2-put.in[III] \textsc{sens}-be \\
\glt You have put a stone in your clothes, and are about to hit me. (2002qaCpa, 171)
\end{exe}

A few verbs, such as \ipa{ʑa} `start' and \ipa{rɲo} `experience', take bare infinitive complements (\ref{ex:tuZanW}) when the complement verb is transitive. Unlike the two previous types of complement clauses, bare infinitival clauses appear to be unattested in Tshobdun and other Rgyalrong languages (\citealt{sun12complementation}).

 \begin{exe}
\ex \label{ex:tuZanW}
\gll
\ipa{tɕe}  	\ipa{pɤjkʰu}  	\ipa{pjɯ-si}   	\ipa{ɕɯŋgɯ}  	\ipa{ʑo}  	\ipa{ɯ-ɕa}  	\ipa{ɯ-ndza}  	\ipa{tu-ʑa-nɯ}  	\ipa{ɕti.}  \\
\textsc{lnk} yet \textsc{ipfv}-die before \textsc{emph} \textsc{3sg.poss}-meat \textsc{3sg-bare.inf}:eat \textsc{ipfv}-start-\textsc{pl}  be.\textsc{affirmative}\factual{} \\
 \glt They start eating it before has died. (20 sWNgi, 47)
\end{exe}

Causative verbs derived from adjectival stative verbs can be used with a bare infinitive complement, the complement indicating the action and the causative verb the manner or degree in which the action is performed, as in \refb{ex:koGAtChom} and in \refb{ex:atAtWGABdi}, the latter with the complex predicate \ipa{ɯ-pɯ} \ipa{--pa} `preserve, keep, take care of' in the complement clause.

 \begin{exe}
\ex \label{ex:koGAtChom}
\gll
[\ipa{cʰa} \ipa{ɯ-tsʰi}] \ipa{ko-ɣɤ-tɕʰom} \\
alcohol \textsc{3sg.poss}-\textsc{bare.inf}:drink \textsc{ifr-caus}-be.too.much \\
\glt He drank too much alcohol.
\end{exe}
 \begin{exe}
\ex \label{ex:atAtWGABdi}
\gll
[\ipa{ɯ-pɯ}  	\ipa{ɯ-pa}]  	\ipa{a-tɤ-tɯ-ɣɤ-βdi}  	\ipa{ma}  \\
\textsc{3sg.poss}-CP:keep \textsc{3sg.poss}-\textsc{bare.inf}:CP:keep \textsc{irr-pfv-2-caus}-be.well \textsc{lnk} \\
\glt You will have to keep it well.
\end{exe}

Note that the causative verb takes the orientational prefix normally selected by the complement verb: \ipa{tsʰi} `drink' and \ipa{ɯ-pɯ} \ipa{--pa} `preserve'  respectively select the `east' \ipa{ko-- kɤ--} and `up' \ipa{to-- tɤ--} orientational prefixes, respectively.

No verb \textit{requires} a bare infinitive complement. The phasal verb \ipa{ʑa} `start' can also be used with an action nominal in \ipa{tɯ--} (see \citealt[6-9]{jacques14antipassive}), and \ipa{rɲo} `experience' and the causative verbs can also take a \ipa{kɤ--}infinitival complement.

Finite complement clauses appear with some speech, thought and modal verbs, but most verbs taking finite complements also accept infinitival complements.

Complement clauses do not necessarily occur as P-argument or adjuncts; some are used in S position, as noun complements, or in topical position as in \refb{ex:tWkhAftsWG}.


 \begin{exe}
\ex \label{ex:tWkhAftsWG}
\gll
[\ipa{taqaβ} 	\ipa{ci} 	\ipa{cʰɯ́-wɣ-lɤt}] 	\ipa{nɯ} 	\ipa{tɯ-kʰɤftsɯɣ} 	\ipa{tu-kɯ-ti} 	\ipa{ŋu} \\
needle once \textsc{ipfv-inv}-throw \textsc{dem} one-stitch \textsc{ipfv-genr}-say be\factual{} \\
\glt Passing the needle once (in the fabric) is called `a stitch'. (12 kAtsxWb, 21)
\end{exe}

Finite complement clauses, like relative clauses, can appear with inverse marking in Japhug, as shown by example \refb{ex:tWkhAftsWG} (used here in its value as a generic A marker, see section \ref{sec:generic}).

%
%Although Japhug has several converbs (in addition to the infinitives, which can be used converbially), the frequency of these converbs in texts is relatively low, and there is in all cases a corresponding finite clause linking construction with a similar meaning (\citealt{jacques14linking}). Japhug disallows sequences of converbs, which is unusual among the languages of the Tibetan area. 
%
%
%Some constructions require a subordinate clause in a particular TAM category, regardless of the tense of the main clause. For instance, temporal subordinate clauses with the postposition \ipa{ɕɯŋgɯ} `before' are always in the imperfective, even if the main clause is in the perfective or evidential as in \refb{ex:CWNgW}.
%
%\begin{exe}
%\ex \label{ex:CWNgW}
%\gll
%\ipa{pjɯ-ɣi-a}  	\ipa{ɕɯŋgɯ}  	\ipa{tɕe}  	\ipa{jo-nɯɕe}  \\
%\textsc{ipfv:down}-come-\textsc{1sg} before \textsc{lnk} \textsc{ifr}-go.back \\
% \glt He went back before I came (home).
%\end{exe}
%
%There is a wide range of conditional constructions in Japhug, each of which has dedicated morphological marking. The verb of the protasis of real conditionals is marked either by the interrogative prefix or by reduplication of the first syllable (or first morpheme if shorter than a syllable), as illustrated by \refb{ex:CWCkAtshita} (iterative coincidence conditional), \refb{ex:mWmAkWtsWma} (implicative conditional).
%
%
%\begin{exe}
%   \ex \label{ex:CWCkAtshita}
%   \gll
%[\ipa{cʰa}   	\textbf{\ipa{ɕɯ\textasciitilde{}ɕ-kɤ-tsʰi-t-a}}]   	(\ipa{ʑo})   	\ipa{lu-βzi-a}   	\ipa{ŋu}   \\
%alcohol \textsc{cond\textasciitilde{}transloc-pfv}-drink-\textsc{pst:tr-1sg} \textsc{emph} \textsc{ipfv}-be.drunk-\textsc{1sg} be\factual{} \\
%\glt Each time I drink alcohol, I get intoxicated. (elicited)
%\end{exe}
%
%\begin{exe}
%\ex  \label{ex:mWmAkWtsWma}
%\gll
%[\textbf{\ipa{mɯ\textasciitilde{}mɤ-kɯ-tsɯm-a-nɯ}}]  	\ipa{nɤ}  	\ipa{mɤ-kʰam-a}\\
%\textsc{cond\textasciitilde{}neg}-2$\rightarrow$1-take.away\factual{}-\textsc{1sg-pl} \textsc{lnk} neg-give[III]\factual{}-\textsc{1sg}\\
%\glt Unless you take me (with you), I won't give it to you. (flood1, 62)
%\end{exe}
%
%%Real conditional with the verb in the protasis in the irrealis are also attested, as in \refb{ex:anWYatnW}.
%%
%%\begin{exe}
%%\ex  \label{ex:anWYatnW}
%%\gll
%%[\textbf{\ipa{a-nɯ-ɲat-nɯ}}]  	\ipa{tɕe}  	\ipa{tɯ-tɕʰa}  	\ipa{nɤ}  	\ipa{tɯ-tɕʰa}  	\ipa{nɯ,}  	<dianxian>  	\ipa{ɯ-taʁ,}  	\ipa{qʰe}  	\ipa{sɯku}  	\ipa{ɯ-taʁ}  	\ipa{nɯ} \ipa{tɕu}  	\ipa{tu-nɯna-nɯ}  	\ipa{tɕe,}  \\
%%\textsc{irr-pfv}-be.tired-\textsc{pl} \textsc{lnk} one-pair \textsc{lnk} one-pair \textsc{dem} electric.wire \textsc{3sg.poss}-on \textsc{lnk} treetop \textsc{3sg.poss}-on \textsc{dem} \textsc{loc} \textsc{ipfv}-rest-\textsc{pl} \textsc{lnk} \\
%%\glt If/Whenever  (the swallows) are tired, they rest in pairs on electric wires or on trees. (Swallows 55)
%%\end{exe}
%
%Alternative and scalar concessive conditionals have the protasis in the past imperfective with the autobenefactive/ spontaneous prefix, as in \refb{ex:pWnnWNu.pWnnWmaR.mAxsi}.
%
%\begin{exe}
%\ex  \label{ex:pWnnWNu.pWnnWmaR.mAxsi}
%\gll
% [[\ipa{nɯ} \ipa{ra}  	 \ipa{pɯ-nnɯ-ŋu}]  	 [\ipa{pɯ-nnɯ-maʁ}]]   	\ipa{mɤxsi}  \ipa{ri}\\
%\textsc{dem} \textsc{pl} \textsc{pst.ipfv-auto}-be \textsc{pst.ipfv-auto}-not.be \textsc{genr:A:neg}:know \textsc{lnk} \\
%\glt I don't know whether this is true or not, (\ipa{kʰɯli}, 60)
%\end{exe}
%
%There are several counterfactual constructions, one of which having a verb in simple irrealis or periphrastic irrealis (with a verb in the imperfective followed by the copula \ipa{a-pɯ-ŋu} in the irrealis) in the protasis and the simple past imperfective in the apodosis, even for dynamic verbs as in \refb{ex:tundzea.apWNu}.
%
%\begin{exe}
%   \ex \label{ex:tundzea.apWNu}
%   \gll
%[\ipa{smɤn}   	\ipa{ʑa} \ipa{tsa}   	\ipa{tu-ndze-a}   	\textbf{\ipa{a-pɯ-ŋu}}]   	\ipa{tɕe}   	\ipa{mɯ-pɯ-ngo-a}   \\
%medicine early  a.little \textsc{ipfv}-eat[III]-\textsc{1sg} \textsc{irr-ipfv}-be \textsc{lnk} \textsc{neg-pst.ipfv}-be.sick-\textsc{1sg} \\
%\glt If I had taken my medicine earlier, I would not have gotten sick. (elicited)
%\end{exe}


\bibliographystyle{unified}
\bibliography{bibliogj}
\end{document}