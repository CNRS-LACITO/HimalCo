\documentclass[oldfontcommands,oneside,a4paper,11pt]{article} 
\usepackage{fontspec}
\usepackage{natbib}
\usepackage{booktabs}
\usepackage{xltxtra} 
\usepackage{polyglossia} 
\usepackage[table]{xcolor}
\usepackage{gb4e} 
\usepackage{multicol}
\usepackage{graphicx}
\usepackage{float}
\usepackage{lineno}
\usepackage{textcomp}
\usepackage{hyperref} 
\hypersetup{bookmarks=false,bookmarksnumbered,bookmarksopenlevel=5,bookmarksdepth=5,xetex,colorlinks=true,linkcolor=blue,citecolor=blue}
\usepackage[all]{hypcap}
\usepackage{memhfixc}
\usepackage{lscape}
 

%\setmainfont[Mapping=tex-text,Numbers=OldStyle,Ligatures=Common]{Charis SIL} 
\newfontfamily\phon[Mapping=tex-text,Ligatures=Common,Scale=MatchLowercase,FakeSlant=0.3]{Charis SIL} 
\newcommand{\ipa}[1]{{\phon #1}} %API tjs en italique
 
\newcommand{\grise}[1]{\cellcolor{lightgray}\textbf{#1}}
\newfontfamily\cn[Mapping=tex-text,Ligatures=Common,Scale=MatchUppercase]{MingLiU}%pour le chinois
\newcommand{\zh}[1]{{\cn #1}}
\newcommand{\topic}{\textsc{dem}}
\newcommand{\tete}{\textsuperscript{\textsc{head}}}
\newcommand{\rc}{\textsubscript{\textsc{rc}}}
\XeTeXlinebreaklocale 'zh' %使用中文换行
\XeTeXlinebreakskip = 0pt plus 1pt %
 %CIRCG
\newcommand{\ro}{$\Sigma$}
\newcommand{\siga}{$\Sigma_1$} 
\newcommand{\sigc}{$\Sigma_3$}   


\begin{document} 

\title{A sketch of Japhug}
\author{Guillaume Jacques}
\maketitle

\section{Introduction}

\section{Phonology}
\citet{jacques07redupl}

\citet{japhug14ideophones}

\section{Verbal morphology}

\subsection{Verb stems}

Japhug verbs present stem alternations determined by TAM and person/number marking. 

Verbs have at most three distinct stems, which we designate as stem 1, stem 2 and stem 3 following \citet{jackson00sidaba}. Stem 1 is the default stem. Stem 2 appears only in the perfective and past imperfective forms. It is very residual in Japhug, attested only in a limited number of verbs indicated in Table \ref{tab:stem2}.


 \begin{table} 
\caption{Stem 2 alternation in Japhug Rgyalrong} \label{tab:stem2} \centering
\begin{tabular}{llllll}
\toprule
Stem 1 & meaning &Stem 2 \\
\midrule
\ipa{ɕe}& to go (vi)&  \ipa{ari} \\
\ipa{sɯxɕe}& to sent (vt)  &\ipa{sɤɣri} \\
\ipa{ɣi}& to come (vi)  &\ipa{ɣe} \\
\ipa{ti}& to say (vt)  &\ipa{tɯt} \\
\bottomrule
\end{tabular}
\end{table}

Stem 3 in Japhug on the other hand is fully productive, and always appear in the forms \textsc{1sg}$\rightarrow$3, \textsc{2sg}$\rightarrow$3 and \textsc{3sg}$\rightarrow$3' of non-past TAM categories (factual, imperfective, testimonial, present) for transitive verbs. Stem 3 does not appear in verb forms with the inverse marker (see below). Intransitive verbs lack stem 3 alternation.

Table \ref{tab:stem3} presents all Stem 3 alternations attested in various Japhug dialects. The vowel alternation applies to the last vowel (in open syllables only) of the verb stem. In the case of verb stems ending in  \ipa{--u} and  \ipa{--ɯ}, some Japhug dialects allow two possible alternations, some verbs displaying vowel fronting and other addition of the \ipa{--m} suffixal element (as in Datshang, see \citealt{linluo03}), while other dialects have generalized vowel fronting.

 \begin{table} 
\caption{Stem 3 alternation in Japhug Rgyalrong} \label{tab:stem3} \centering
\begin{tabular}{llllll}
\toprule
Stem 1 & Stem 3& type \\
\midrule
 \ipa{--a} & \ipa{--e} & vowel fronting\\
 \ipa{--u} & \ipa{--e} & vowel fronting\\
\ipa{--u} & \ipa{--ɤm} & \ipa{--m} \\
\ipa{--ɯ} & \ipa{--i} & vowel fronting\\
\ipa{--ɯ} & \ipa{--ɯm} & \ipa{--m}\\
\ipa{--o} & \ipa{--ɤm} & \ipa{--m} \\
\bottomrule
\end{tabular}
\end{table}




\subsection{Directional prefixes}
All finite verb forms except the \textsc{factual} have directional prefixes belonging to one of the four series in Table \ref{tab:directional} (some Japhug dialects have a slightly different paradigm, see \citealt{linluo03}). The non-periphrastic TAM categories are indicated in Table \ref{tab:finite.forms}. Some of these categories require a particular (testimonial, present, past imperfective) are marked by one particular directional prefix. The other TAM category only require a prefix of either set A, B, C or D, but the direction depends on the verb.

Most verbs have one intrinsic direction which is lexically determined, and used in the perfective, imperfective, evidential, irrealis and imperative. For instance, the verb \ipa{ndza} `eat' selects the direction `up' for all its forms: \textbf{perfective} \textsc{1sg$\rightarrow$3sg} \ipa{tɤ-ndza-t-a} `I ate it', \textbf{imperfective} (with stem 3) \ipa{tu-ndze} `He eats it', \textbf{perfective} \textsc{3sg$\rightarrow$3'} \ipa{ta-ndza} `He ate it' and \textbf{evidential} \ipa{to-ndza} `He ate it'. 

In addition to the directional prefixes, two additional TAM markers appear in specific forms. First, the past \ipa{--t} suffix in the \textsc{1sg}$\rightarrow$3 and \textsc{2sg}$\rightarrow$3 forms of transitive verbs whose last stem syllable is an open syllable as \ipa{tɤ-ndza-t-a} \textsc{pfv}-eat-\textsc{pst-1sg} `I ate it'. Second, the evidential circumfix \ipa{kɯ--...-ci} is required whenever set D directional prefixes are used with a verb whose stem begins in \ipa{a--}, including verb with the progressive prefix \ipa{asɯ--} (see below). For instance, the evidential form of the verb \ipa{anbaʁ} `hide' is \ipa{ko-kɯ-ɤnbaʁ-ci} \textsc{evd-evd}-hide-\textsc{evd} `he hid'.

\begin{table}[H]
\caption{Directional prefixes in Japhug Rgyalrong} \label{tab:directional}
\resizebox{\columnwidth}{!}{
\begin{tabular}{llllll}
\toprule
   &  	perfective  (A) &  	imperfective  (B)  &  	perfective 3$\rightarrow$3' (C)  &  	evidential  (D) \\  	
   \midrule
up   &  	\ipa{tɤ--}   &  	\ipa{tu--}   &  	\ipa{ta--}   &  	\ipa{to--}   \\  	
down   &  	\ipa{pɯ--}   &  	\ipa{pjɯ--}   &  	\ipa{pa--}   &  	\ipa{pjɤ--}   \\  	
upstream   &  	\ipa{lɤ--}   &  	\ipa{lu--}   &  	\ipa{la--}   &  	\ipa{lo--}   \\  	
downstream   &  	\ipa{tʰɯ--}   &  	\ipa{cʰɯ--}   &  	\ipa{tʰa--}   &  	\ipa{cʰɤ--}   \\  	
east   &  	\ipa{kɤ--}   &  	\ipa{ku--}   &  	\ipa{ka--}   &  	\ipa{ko--}   \\  	
west   &  	\ipa{nɯ--}   &  	\ipa{ɲɯ--}   &  	\ipa{na--}   &  	\ipa{ɲɤ--}   \\  	
no direction &\ipa{jɤ--}   &  	\ipa{ju--}   &  	\ipa{ja--}   &  	\ipa{jo--}   \\  	
\bottomrule
\end{tabular}}
\end{table}


Verbs of motion and some verbs of concrete action can be associated with all seven series of prefixes to indicate the direction of the motion. The  `no direction' series of prefixes only occurs with motion verbs. 

\begin{table}
\caption{Finite verb categories in Japhug Rgyalrong} \label{tab:finite.forms} \centering
\begin{tabular}{lllllll}
\toprule
&	&	stem&	prefixes\\
\midrule
factual&	\textsc{fact} &	1 or 3&	no prefix\\
imperfective&	\textsc{ipfv} &	1 or 3&	B\\
perfective&	\textsc{pfv} &	2&	A or C\\
past imperfective&	\textsc{pst.ipfv} &	2&	\ipa{pɯ--}\\
evidential&	\textsc{evd} &	1&	D\\
evidential imperfective&	\textsc{evd.ipfv} &	1&	\ipa{pjɤ--}\\
testimonial&	\textsc{testim} &	1 or 3&	\ipa{ɲɯ--}\\
present&	\textsc{pres} &	1 or 3&	\ipa{ku--}\\
irrealis&	\textsc{irr} &	1 or 3&	\ipa{a--} + A\\
imperative&	\textsc{imp} &	1 or 3&	A\\
\bottomrule
\end{tabular}
\end{table}

The simple TAM categories in Table \ref{tab:finite.forms} can be combined with other TAM markers such as the conative \ipa{ji--} and the progressive \ipa{asɯ--}. 

In addition, periphrastic TAM categories, combining simple verb forms with the copula \ipa{ŋu} `be', \ipa{maʁ} `not be' or \ipa{ɕti} `be (affirmative)' are also attested. The most common periphrastic tenses are the periphrastic past imperfective and periphrastic evidential imperfective, built by combining the imperfective form of the verb with the past imperfective \ipa{pɯ-ŋu} and evidential imperfective \ipa{pjɤ-ŋu} of the copula. Most dynamic verbs cannot be used in the simple past and evidential imperfective in Japhug (see \citealt{lin11direction}) and periphrastic tenses are required as in example \ref{ex:evd.ipfv.periphrastic}: directly combining the prefix \ipa{pjɤ--} with the stem of the verb `eat' results  in an incorrect form. The past and evidential imperfective of dynamic verbs is attested only in counterfactual and in combination with the progressive \ipa{asɯ--}.\footnote{The present of the circumfix \ipa{kɯ--...-ci} in example \ref{ex:evd.ipfv.prog} is explained in the previous section.}

\begin{exe}
\ex \label{ex:evd.ipfv.periphrastic}
\gll
\ipa{tu-ndze} \ipa{pjɤ-ŋu} \\
 \textsc{ipfv}-eat[III] \textsc{evd.ipfv}-be \\
\glt He was eating it. / He used to eat it.
\ex \label{ex:evd.ipfv.prog}
\gll
 \ipa{pjɤ-kɯ-ɤsɯ-ndza-ci} \\
 \textsc{evd.ipfv-evd-prog}-eat-\textsc{evd} \\
\glt He was eating it.
\end{exe}



\subsection{Person marking}
Japhug person marking is encoded by a series of prefixes and suffixes, in addition to stem 1 / Stem3 alternations in the case of transitive verbs.

Table \ref{tab:japhug.tr} presents the regular transitive and intransitive verbal paradigms in Japhug in the \textsc{factual}. The symbols \siga{} and \sigc{} represent Stem 1 and Stem 3 respectively. The affixes found in the intransitive paradigm (the second person \ipa{tɯ--} and the suffixes) also appear in the transitive paradigm, in addition to portmanteau prefixes (\ipa{kɯ--} and \ipa{ta--} for the local scenarios 2$\rightarrow$1 and 1$\rightarrow$2) and the inverse prefix \ipa{wɣɯ--} which is obligatory in 3$\rightarrow$1 and 3$\rightarrow$2 mixed scenario and cannot occur in local and 1$\rightarrow$3 and 2$\rightarrow$3 scenarios. When both arguments are third person, the use of the inverse is determined by semantic and pragmatic factors (see \citealt{jacques10inverse}).

\begin{table}[H]
\caption{Japhug transitive and intransitive paradigms}\label{tab:japhug.tr}
\resizebox{\columnwidth}{!}{
\begin{tabular}{l|l|l|l|l|l|l|l|l|l|l|}
\textsc{} & 	\textsc{1sg} & 	  \textsc{1du} & 	\textsc{1pl} & 	\textsc{2sg} & 	\textsc{2du} & 	\textsc{2pl} & 	\textsc{3sg} & 	\textsc{3du} & 	\textsc{3pl} & 	\textsc{3'} \\ 	
\hline
\textsc{1sg} & \multicolumn{3}{c|}{\grise{}} &	\ipa{} & 	\ipa{} & 	\ipa{} & 	\ipa{\sigc{}-a}   & 	 \ipa{\sigc{}-a-ndʑi} & 	 \ipa{\sigc{}-a-nɯ} & 	\grise{} \\	
\cline{8-10}
\textsc{1du} & 	\multicolumn{3}{c|}{\grise{}} &	\ipa{ta-\siga{}} & 	\ipa{ta-\siga{}-ndʑi} & 	\ipa{ta-\siga{}-nɯ} & 	\multicolumn{3}{c|}{ \ipa{\siga{}-tɕi}}  & 	\grise{} \\	
\cline{8-10}
\textsc{1pl} & 	\multicolumn{3}{c|}{\grise{}} & 	  & 	&  & 	\multicolumn{3}{c|}{ \ipa{\siga{}-ji}}  & 	\grise{} \\	
\cline{1-10}
\textsc{2sg} & 	\ipa{kɯ-\siga{}-a} & 	\ipa{} & 	\ipa{} & 	\multicolumn{3}{c|}{\grise{}}&	\multicolumn{3}{c|}{\ipa{tɯ-\sigc{}}} & 	\grise{} \\	
\cline{2-2}
\cline{8-10}
\textsc{2du} & 	\ipa{kɯ-\siga{}-a-ndʑi} & 	\ipa{kɯ-\siga{}-tɕi} & 	\ipa{kɯ-\siga{}-ji} & 	\multicolumn{3}{c|}{\grise{}} &	\multicolumn{3}{c|}{\ipa{tɯ-\siga{}-ndʑi}} & 	\grise{} \\	
\cline{2-2}
\cline{8-10}
\textsc{2pl} & 	\ipa{kɯ-\siga{}-a-nɯ} & 	\ipa{} & 	\ipa{} & 	\multicolumn{3}{c|}{\grise{}}&	\multicolumn{3}{c|}{\ipa{tɯ-\siga{}-nɯ}} & 	\grise{} \\	
\hline
\textsc{3sg} &  	\ipa{wɣɯ́-\siga{}-a} & 	\ipa{} & 	\ipa{} & 	\ipa{} & 	\ipa{} & 	\ipa{} & \multicolumn{3}{c|}{\grise{}} &	\ipa{\sigc{}} \\ 	
\cline{2-2}
\cline{11-11}
\textsc{3du} &  	\ipa{wɣɯ́-\siga{}-a-ndʑi} & 	 \ipa{wɣɯ́-\siga{}-tɕi} & 		\ipa{wɣɯ́-\siga{}-ji} & 	\ipa{tɯ́-wɣ-\siga{}} & 	\ipa{tɯ́-wɣ-\siga{}-ndʑi} & 	\ipa{tɯ́-wɣ-\siga{}-nɯ} & 	\multicolumn{3}{c|}{\grise{}} &	\ipa{\siga{}-ndʑi} \\ 
\cline{2-2}	
\cline{11-11}
\textsc{3pl} &  	\ipa{wɣɯ́-\siga{}-a-nɯ} & 	\ipa{} & 	\ipa{} & 	\ipa{} & 	\ipa{} & 	\ipa{} & \multicolumn{3}{c|}{\grise{}} &	\ipa{\siga{}-nɯ} \\ 	
\hline
\textsc{3'} & 	\multicolumn{6}{c|}{\grise{}} &	\ipa{wɣɯ́-\siga{}} & 	\ipa{wɣɯ́-\siga{}-ndʑi} & 	\ipa{wɣɯ́-\siga{}-nɯ} & 	\grise{} \\	
	\hline	\hline
\textsc{intr}&\ipa{\siga{}-a}&\ipa{\siga{}-tɕi}&\ipa{\siga{}-ji}&\ipa{tɯ-\siga{}}&\ipa{tɯ-\siga{}-ndʑi}&\ipa{tɯ-\siga{}-nɯ}&\ipa{\siga{}}&\ipa{\siga{}-ndʑi} &\ipa{\siga{}-nɯ}& 	\grise{} \\	
	\hline
\end{tabular}}
\end{table}

The sensory existential copulas \ipa{ɣɤʑu} `exist' and \ipa{maŋe} `not exist' are extremely irregular. They cannot be used with directional prefixes, unlike nearly all other verbs, cannot be nominalized, and have infixed forms for the second person \ipa{ɣɤtɤʑu} `you are there' and \ipa{mataŋe} `you are not there'.

\subsection{Derivation}
Japhug a a very rich system of derivations, including voice markers, denominal and deideophonic prefixes. Only the former will be discussed here.

\subsubsection{Argument demotion}
In Japhug, non-overt arguments in a sentence are always construed as being definite. To express indefinite agent or patient, several strategies are possible, including indefinite pronouns such as \ipa{tʰɯci} `something', generic person or argument-demoting voice markers.

Japhug has a wide array of argument-demoting prefixes, including passive \ipa{a--}, anticausative, antipassive \ipa{sɤ--} and \ipa{rɤ--} and deexperiencer \ipa{sɤ--} (see \citealt{jacques12demotion}). The antipassive prefixes have been grammaticalized from denominal prefixes (\citealt{jacques14antipassive}).

The  \ipa{a--} prefix is an agentless passive, that converts a transitive verb into an intransitive one whose S corresponds to the P of the base verb. The original A cannot be expressed, but is semantically recoverable, as in \ref{ex:passive} with the light verb construction \ipa{sɤcɯ lɤt}`lock'. 

\begin{exe}
\ex \label{ex:passive}
\gll 
\ipa{ɯ-ŋgɯ} 	\ipa{lɤ-ɣi} 	\ipa{jɤɣ} 	\ipa{ma} 	\ipa{sɤcɯ} 	\ipa{mɤ-a-lɤt} \\
\textsc{3sg}-inside \textsc{imp:upstream}-come \textsc{fact}:be.possible because key \textsc{neg-fact:pass}-throw \\
\glt Come in, (the door) is not locked.
\end{exe}

The anticausative is distinct from the causative in that the action is construed as having occurred spontaneously without any agent. Only 24 pairs of verbs with anticausative derivation have been found in Japhug (see some examples in Table \ref{tab:anticausative}). Note the presence of the verb \ipa{χtɤr} `spill' borrowed from Tibetan \ipa{gtor}(same meaning): this verb shows that the direction of derivation is from transitive to intransitive, not the other way round, since the corresponding intransitive verb \ipa{ʁndɤr} `be spilled' cannot be a Tibetan borrowing.


\begin{table}[H]
\caption{Examples of anticausative in Japhug}\label{tab:anticausative}
\begin{tabular}{lllllllll} \toprule
basic verb  & &derived  verb &\\
\midrule
\ipa{ftʂi}  &	to melt (vt)	&		\ipa{ndʐi}  &	to melt (vi)		\\
\ipa{kio}  &	to cause to drop	&		\ipa{ŋgio}  &	to slip		\\
\ipa{prɤt}  &	to break	&		\ipa{mbrɤt}  &		to be broken	\\
\ipa{χtɤr}  &	 to spill	&		\ipa{ʁndɤr}  &		to be spilled	\\
\ipa{tʂaβ}  &	to cause to roll	&		\ipa{ndʐaβ}  &	to roll (vi)		\\
   \ipa{cɯ}  &	 to open 	&		\ipa{ɲɟɯ}  &	 to be opened	 	\\ 
 \bottomrule
\end{tabular}
\end{table}

The semantic difference with the passive can be illustrated with \ipa{pjɤ-mbrɤt} \textsc{evd-anticaus}:break `it broke (of a rope)' vs \ipa{pjɤ-kɯ-ɤ-prɤt-ci} \textsc{evd-evd-pass-break}-\textsc{evd}`(someone) broke it'.

There are two antipassive prefixes \ipa{rɤ--} and \ipa{sɤ--}, which convert a transitive verb into an intransitive one whose S corresponds to the A of the base verb. The prefix \ipa{rɤ--} is used in two cases: when the P of the base verb designates a non-human (\ipa{rɤt} `write (tr)' $\rightarrow$ \ipa{rɤ-rɤt} `write (it)', \ipa{ɕar} `search, look for (tr)' $\rightarrow$ \ipa{rɤ-ɕar} `look for things (it)') or in the case of ditransitive verbs, even when the P corresponds to the (human)recipient (\ipa{mbi} `give X to' $\rightarrow$ \ipa{rɤmbi} `give X to someone'). The antipassive \ipa{sɤ--} derives either a dynamic verb whose demoted argument is necessarily human (\ipa{sɤ-ɕar} `look for people (it)') or a stative verb, whose demoted P can be either human or non-human (\ipa{sat} `kill' $\rightarrow$ \ipa{sɤ-sat}`have a killing power, be deadly'). 

A historically related derivation is the deexperiencer \ipa{sɤ--}, which derives a stative verb from any stative verb, changing an experiencer S into the stimulus (\ipa{ɲɟio} `slip' $\rightarrow$ \ipa{sɤ-ŋgio} `be slippery').

\subsubsection{Incorporation}
Japhug has an incorporation-like construction in which noun-verb nominal compounds are turned into verbs by means of a denominal prefix (\citealt{jacques12incorp}). For instance, from the noun \ipa{cɯ} `stone' and the verb\ipa{pʰɯt} `pluck, take out' one can derive a action nominal    \ipa{cɯpʰɯt} `clearing the stones', which can in turn be made into a incorporating verb by denominal derivation  \ipa{ɣɯ-cɯpʰɯt } `take out stones (out of the field)'.

\begin{exe}   
\ex
\begin{xlist}[(ii)]
\exi{(i)} 
\gll     \ipa{cɯ-pʰɯt} \ipa{nɯ-βzu-t-a}  \\
  stone-clearing \textsc{pfv}-do-\textsc{pst}-\textsc{1sg} \\
\exi{(ii)} 
\gll     \ipa{nɯ-ɣɯ-cɯ-pʰɯt-a}  \\
  \textsc{pfv-denominal}-stone-take.out-\textsc{1sg} \\
\exi{(iii)} 
\gll     \ipa{cɯ} \ipa{nɯ-pʰɯt-a}  \\
  stone \textsc{pfv}-take.out-\textsc{1sg} \\
  \end{xlist}
 \glt   I cleared the stones (from the field). 
\end{exe}   

The three constructions above have a slightly different meaning: the compound action nominal with light verb construction (i) implies that the action took considerable time, while the incorporating construction (ii) cannot be used if the action refers to specific stones that have been previously mentioned in discourse.

\subsubsection{Causative}
There are two causative prefixes in Japhug, \ipa{sɯ--} and \ipa{ɣɤ--}. 

The causative   \ipa{sɯ--} prefix is extremely productive, and can be added to nearly any verb, including recent borrowings from Tibetan and Chinese. It presents  considerable allomorphy, and numerous irregular forms. It has three regular allomorphs \ipa{sɯ-}, \ipa{sɯɣ-} and \ipa{z-} depending on the following element. The \ipa{z-} allomorph appears before all prefixes in sonorant initial. It expresses very broad causative meanings, including indirect causative (including by one's inaction), permissive (`allow to') and is also (optionally) used with  instruments marked in the ergative as in \ref{ex:caus.instrument}.


\begin{exe}
\ex \label{ex:caus.instrument}
\gll  \ipa{ɯ-χto} 	\ipa{nɯ} 	\ipa{mbrɯtɕɯ} 	\ipa{kɯ} 	\ipa{kú-wɣ-sɯ-rkhe}  \\
\textsc{3sg.poss}-slit \textsc{dem} knife \textsc{erg}  \textsc{ipfv-inv-cause}-carve \\
 \glt  The slit is carved with a knife. (Colored belts 13)
\end{exe} 

Causative verbs with irregular allomorphs such as \ipa{jtsʰi} `give to drink'  from \ipa{tsʰi} `drink' generally have unpredictable semantics. The regular form is always possible at least to express an action with an instrument (\ipa{sɯ-tsʰi} `drink with').

When the causative \ipa{sɯ--} is applied to a transitive verb, the causee can be either treated as the P in the verbal morphology or not encoded at all, resulting in ambiguous forms. As an example, the causative of \ipa{qur} ``to help''  \ipa{sɯ-qur} can have two meanings in specific cases (\ref{ex:caus:show.2>3>1}).

\begin{exe} 
\ex \label{ex:caus:show.2>3>1}
\gll   \ipa{tɤ-kɯ-sɯ-qur-a-ndʑi}  \\
 \textsc{pfv-2$\rightarrow$1-caus}-see-\textsc{1sg-du}  \\
 \glt  You_d caused me to help him. OR You_d caused him to help me. 
\end{exe} 

The causative \ipa{ɣɤ--} is more restricted, and can only occur with some stative verbs. There is a slight semantic contrast between \ipa{sɯ--} and \ipa{ɣɤ--} in Japhug: in the case of some stative verbs, the former indicates a change of state, while the latter expresses a increase of degree (see also \citealt{jackson06paisheng} about Tshobdun). Example \ref{ex:ngolo-tAjko}  illustrates the use of \ipa{sɯ--} to express a change of state. Using \ipa{pjɯ-ɣɤ-tɕur} (\textsc{ipfv-caus}-be.sour) with the prefix \ipa{ɣɤ--} instead would mean `make it sourer' (it would imply that the leaves were already sour in the first place).

 \begin{exe}
\ex \label{ex:ngolo-tAjko}
\gll
\ipa{tɕe} 	\ipa{tɤjko} 	\ipa{mɯ-tɤ-tɕur} 	\ipa{tɕe,} 	\ipa{ɴɢolo} 	\ipa{ɯ-mat} 	\ipa{nɯ} 	\ipa{ɲɯ́-wɣ-phɯt} 	\ipa{tɕe,} 	\ipa{tɕe} 	\ipa{tɤrca} 	\ipa{pjɯ́-wɣ-ɣɤla} 	\ipa{tɕe,} 	\ipa{tɕe} 	\ipa{tɤjko} 	\ipa{pjɯ-sɯɣ-tɕur} 	\ipa{cha.} \\ 
\textsc{lnk} pickled.turnip.leaves \textsc{neg-pfv}-be.sour \textsc{lnk} tree.sp \textsc{3sg.poss}-fruit \textsc{dem} ipfv-inv-pluck \textsc{lnk} \textsc{lnk} together \textsc{ipfv-inv}-soak \textsc{lnk} \textsc{lnk}  pickled.turnip.leaves \textsc{ipfv-caus}-be.sour \textsc{fact}:can \\
 \glt When the pickled turnip leaves are not sour, one picks \ipa{ɴɢolo} fruit and soak it with them, and it can make the turnip leaves sour.   (\ipa{ɴɢolo} 27)
   \end{exe}

\subsubsection{Applicative and tropative}
In addition to the causative, Japhug has two valency-increasing voice derivation, the applicative \ipa{nɯ--} / \ipa{nɯɣ--} and the tropative \ipa{nɤ--} /\ipa{nɤɣ--} (\citealt{jacques13tropative}).

The applicative converts an intransitive verb into a transitive one whose A corresponds to the S of the base verb and whose P can be either a stimulus or a recipient (for instance \ipa{mu} `be afraid' $\rightarrow$\ipa{nɯɣmu} `be afraid of'). 

The tropative is an extremely productive derivation that can be applied to all stative verbs describing a quality (ie, it is a criterion for defining an adjective subclass among stative verbs), deriving a transitive whose P corresponds to the S of the base verb (like a causative) but whose A is an experiencer, not a agent. The tropative verb has the meaning `to find / consider to be', thus \ipa{xtɕi} `be small' $\rightarrow$ `to find small'.

\subsubsection{Other voice markers}
In addition to the voice markers described above, Japhug also has a reflexive \ipa{ʑɣɤ--} distinct from the reciprocal (expressed by combining the prefix \ipa{a--} with a partially reduplicated verb stem).

We also find three verbal derivations with modal meanings: the abilitative \ipa{sɯ--/z--} (expressing the meaning `be able to X', \ipa{nɤɕqa} `bear' $\rightarrow$ \ipa{z-nɤɕqa} `be able to bear), the stative facilitative \ipa{ɣɤ--} (\ipa{wxti} `be big' $\rightarrow$ \ipa{ɣɤwxti} `grow big easily') and the passive facilitative \ipa{nɯɣɯ--} (\ipa{ntɕʰoz} `use' $\rightarrow$\ipa{nɯɣɯntɕʰoz} `easy to use').
   
\subsection{Autobenefactive-spontaneous}

The autobenefactive-spontaneous \ipa{nɯ--} prefix expresses a wide range a meanings. First, it is used to express that the S/A is affected by the action; it appears in particular with transitive verbs when the P is a body part of or belongs to the S/A (\ref{ex:pWnWXtCi}).

\begin{exe}
\ex \label{ex:pWnWXtCi}
\gll 
\ipa{nɤ-ku} 	\ipa{pɯ-nɯ-χtɕi} \\
\textsc{2sg.poss}-head \textsc{imp-auto}-wash \\
\glt Wash your head.
\end{exe}

The prefix \ipa{nɯ--} can also be used to express involuntary actions, as in \ref{ex:panWClWG}.

\begin{exe}
\ex \label{ex:panWClWG}
\gll \ipa{ɯ-qom} 	\ipa{ci} 	\ipa{pa-nɯ-ɕlɯɣ} 	\ipa{ɲɯ-ŋu,} \\
\textsc{3sg.poss}-tear \textsc{indef} \textsc{pfv:3$\rightarrow$3'-auto}-drop \textsc{testim}-be \\
\glt She shed a tear (unvoluntarily).(Kunbzang 228)
\end{exe}


The prefix \ipa{nɯ--} is also used to express an action occurring by itself, without exterior agent/causer, even if the action is voluntary and controllable, as in \ref{ex:pjWnWmtsaRa}.

\begin{exe}
\ex \label{ex:pjWnWmtsaRa}
\gll 
\ipa{aʑo} 	\ipa{pjɯ-kɯ-ɣɤrat-a-nɯ} 	\ipa{mɤ-ra} 	\ipa{ma} 	\ipa{aʑo} 	\ipa{pjɯ-nɯ-mtsaʁ-a} 	\ipa{jɤɣ} \\
\textsc{1sg} \textsc{ipfv:down}-2$\rightarrow$1-throw-\textsc{1sg-pl} \textsc{neg-fact}:need because \textsc{1sg} \textsc{neg-ipfv:down-auto}-jump-\textsc{1sg} \textsc{fact}:be.possible \\
\glt You don't need to throw me in there, I will jump of my own free will.
\end{exe}

The autobenefactive-spontaneous prefix has no influence on the verb transitivity.

\subsection{Associated motion}
Japhug has two associated motion prefixes \ipa{ɕɯ--} and \ipa{ɣɯ--} expressing a motion taking place before the action expressed by the verb to which the prefixes are attached 
(\citealt{jacques13harmonization}). The translocative \ipa{ɕɯ--} and the cislocative \ipa{ɣɯ--} respectively express the meaning `go to ..., go and ...' or `come to..., come and ...', as in example \ref{ex:cisloc}.


\begin{exe}
\ex \label{ex:cisloc}
\gll
\ipa{tɤrɤku} 	\ipa{kɯ-fse} 	\ipa{ɣɯ-tu-ndze} 	\ipa{nɯra} 	\ipa{mɤ-ŋgrɤl.} \\
crops \textsc{nmlz}:S-be.like \textsc{cisloc-ipfv}-eat[III] \textsc{dem:pl} \textsc{neg-fact}:be.usually.the.case \\
\glt It does not come to eat crops and things like that.
\end{exe}

These prefixes are semantically close to the motion verb construction with \ipa{ɕe} `go' and \ipa{ɣi} `come' and a complement with the verb in the S/A nominalization form. A crucial difference, however, can be found detected in the perfective:  motion verb constructions such as \ref{ex:motion.verb} imply that the motion event has taken place, but do not specify whether the action of the complement clause has taken place or not. Using the associated motion form \ipa{ɣɯ-tɤ-tɯ-nɤma-t} instead would mean `What have you done ? (after coming here)'.

\begin{exe}
\ex \label{ex:motion.verb}
\gll
\ipa{tɕʰi} 	\ipa{ɯ-kɯ-nɤma} 	\ipa{jɤ-tɯ-ɣe?} \\
what \textsc{3sg-nmlz:}S/A-do \textsc{pfv-2}-come[II] \\
\glt What have you come to do?
\end{exe}
%retrouver l'exemple


With causative verbs, the associated motion can refer either to the motion of the causer or that of the causee as in example \ref{ex:assoc.motion2}.

  \begin{exe}
\ex \label{ex:assoc.motion2}
\gll
\ipa{tɕe} 	\ipa{kupa} 	\ipa{chu} 	\ipa{nɯra} 	\ipa{athi} 	\ipa{pɕoʁ} 	\ipa{nɯra,} 	\ipa{ɯ-pɕi} 	\ipa{nɯra} 	\ipa{kɯ} 	\ipa{kɯreri} 	\ipa{ɣɯ-chɯ-sɯ-χtɯ-nɯ} 	\ipa{ŋu.}  \\
\textsc{lnk} Chinese \textsc{loc} \textsc{dem:pl} downstream direction \textsc{dem:pl} \textsc{3sg}-outside  \textsc{dem:pl}  \textsc{erg} here \textsc{cisloc-ipfv:downstream-caus}-buy-\textsc{pl} \textsc{fact}:be \\
\glt People from the Chinese areas, people from outside send people to come here to buy (matsutake and sell them in the areas downstream). (Matsutake 58)
  \end{exe} 


\subsection{Nominalization and other non-finite forms}

Japhug has a rich set of non-finite verb forms: participles, action nominalizer, infinitives and converbs. These forms have in common the facts that (i) they can only mark person by means of possessive prefixes like nouns, cannot take the inverse prefix, and cannot encode two arguments in the case of transitive verbs (ii) they cannot use stem 3 (iii) they cannot serve as the predicate of a main clause.

There are three participles in Japhug: the S/A-participle \ipa{kɯ--}, the P-participle \ipa{kɤ--} and the oblique participle \ipa{sɤ--}. In the case of transitive verbs, the S/A-participle takes a possessive prefix (see Table \ref{tab:pronoun}) coreferent with the P (\ipa{ndza} `eat' $\rightarrow$ \ipa{ɯ_i-kɯ-ndza} `the one who eats it_i'), while the P-participle optionally takes a possessive prefix coreferent with the A (\ipa{ɯ_i-kɤ-ndza} `what he/it_i eats'). The oblique participle \ipa{sɤ--} can refer to  the instrument, the place, time or recipient of an action (see the section on relatives).

The participles can be used with negation and directional prefixes (only set A and B); participles with perfective directional prefixes take stem 2 if they have a distinct stem.

In addition, Japhug verbs have infinitives in \ipa{kɤ--} (for dynamic verbs allowing an animate argument) or \ipa{kɯ--} (for intransitive stative verbs or verbs only used with inanimate arguments). While the infinitives are superficially similar to the S/A- and P-participles, their morphology and uses are different. they cannot take any possessive prefix, and in the case of intransitive dynamic verbs, the forms in \ipa{kɤ--}

bare infinitive (\citealt{jacques14antipassive}).


\citet{jacques14linking}

perfective \ipa{pjɯ-tɯ-mto}

\ipa{sɤ-mɯ\textasciitilde{}mu}

\ipa{ɯ-mɤ-ɲɯ-sɤ-jmɯ\textasciitilde{}jmɯt}

\subsection{Generic person}
Apart from argument-demoting voice prefixes, another way of expressing indefinite arguments in Japhug is generic person marking. It is available only for generic human referents, not for inanimates or animals.

This type of person marking follows an ergative alignment: generic S and P arguments are marked with the prefix \ipa{kɯ--} (see examples \ref{ex:pWkWNu} and \ref{ex:kukWnWfse}) while generic A argument is marked with the inverse prefix \ipa{wɣ--} (\ref{ex:tuwGndza.sna}; the verb \ipa{ti} `` say' is irregular in that its generic A form takes the \ipa{kɯ--} prefix).  Verbs with generic person markers cannot take any other person or number markers.


\begin{exe}
\ex \label{ex:pWkWNu}
\gll
\ipa{tɕeri} 	\ipa{tɤpɤtso} 	\ipa{pɯ-kɯ-ŋu} 	\ipa{tɕe,} 	\ipa{nɯ} 	\ipa{kɤ-ndza} 	\ipa{wuma} 	\ipa{ʑo} 	\ipa{pɯ-kɯ-rga.} \\
but child \textsc{pst.ipfv-genr}:S/P-be \textsc{lnk} \textsc{dem} \textsc{inf}-eat really \textsc{emph} \textsc{pst.ipfv-genr}:S/P-like \\
\glt When (we) were children, (we) liked it a lot. (Sambucus, 135)
\end{exe}


\begin{exe}
\ex \label{ex:kukWnWfse}
\gll
\ipa{tɕe}  	\ipa{li}  	\ipa{nɯ}  	\ipa{tɯrme}  	\ipa{kɯnɤ}  	\ipa{ku-kɯ-nɯfse}  	\ipa{ɲɯ-ŋu,}\\
\textsc{lnk} again \textsc{dem} people also \textsc{ipfv-genr:S/P}-recognize \textsc{testim}-be\\
\glt  (The monkey) recognizes people. (monkey, 17)
\end{exe}

\begin{exe}
\ex \label{ex:tuwGndza.sna}
\gll
\ipa{tɯrme}  	\ipa{kɯ}  	\ipa{tú-wɣ-ndza}  	\ipa{mɤ-sna.}   \\
people \textsc{erg} \textsc{ipfv-inv}-eat \textsc{neg-fact}:be.fine \\
\glt It is not edible by people. (\ipa{khɯrtshɤz}, 39)
\end{exe}

\subsection{Transitivity}
The single most important feature in Japhug verbal morphology is transitivity. As seen above, transitive verbs differ from intransitive verbs in several ways: the S/A-participle can take a possessive prefix coreferent with the P, the perfective third person forms take set C directional prefixes (or the inverse prefix), and verbs whose base stem is in open syllable have the past \ipa{-t-} suffix in \textsc{1sg$\rightarrow$3} \textsc{2sg$\rightarrow$3} and stem 3 alternation in some cases.

There is never any ambiguity as to whether a particular verb is morphologically  transitive or intransitive, but some verbs, such as \ipa{taʁ} `weave' or \ipa{mɯrkɯ} `steal' are labile and can be conjugated either transitively or intransitively (\citealt{jacques12demotion}). All labile verbs in Japhug are A-preserving; there are no examples of P-preserving lability.

There is a class of semi-transitive verbs, such as \ipa{rga} `like' or \ipa{aro} `have, own' that are morphologically intransitive but present some syntactic transitivity. While their finite conjugation is identical to that of intransitive verbs, and while their S does not take ergative marking, they do have P-participle in \ipa{kɤ--} and thus have a object-like argument that can be relativized exactly in the same way as the P of a transitive verb. This argument however, even if SAP, cannot be marked in the verb morphology.

Ditransitive can be either secundative (with $R=P$, like \ipa{mbi} `give'), indirective ($T=P$, as \ipa{tʰu} `ask') or both (causative verbs deriving from transitive base verbs) from the point of view of person marking on the verb. No more than two arguments can be encoded by verbal morphology. The noun phrases corresponding to recipients of indirective verbs receive dative flagging \ipa{ɯ-ɕki} or \ipa{ɯ-pʰe}, but cannot be marked on the verb, even if first or second person.

The relativization patterns however slightly differ from argument indexation on the verb.
In the case of secundative verbs, both the R and the T of secundative verbs are relativizable like the P, even though only the R is indexed on the verb. As for indirective verbs, the R is relativized by means of the oblique participle (see the section on relativization).

\section{Nominal morphology}
Japhug nouns can be divided in three sub-classes: simple nouns, inalienably possessed nouns and counted nouns (classifiers). Only the first two classes are discussed in this sketch. 

The same set of possessive prefixes (see Table \ref{tab:pronoun}) is used for all nouns, but inalienably possessed nouns cannot be used on their own without one of these prefixes. When there is no definite possessor, the indefinite possessive prefixes \ipa{tɤ--} or \ipa{tɯ--} are used. It is the citation form of inalienably possessed nouns (\ipa{tɤ-lu} `milk', \ipa{tɯ-ŋga} `clothes', \ipa{tɤ-rpɯ} `uncle', \ipa{tɯ-ci} `water'). The choice of the prefix \ipa{tɤ--} vs \ipa{tɯ--} is lexically determined.  When a specific possessor is present, the indefinite prefix is replaced by the appropriate possessive prefix (\ipa{ɯ-lu} `her/its milk (from her nipple)', \ipa{a-ŋga} `my clothes', \ipa{nɤ-rpɯ} `your uncle', \ipa{ɯ-ci} `its juice'). It is possible to turn an inalienably possessor noun into an alienably possessed one by prefixing definite possessive prefix to the indefinite one (\ipa{ɯ-tɤ-lu} `his milk (to drink)', \ipa{ɯ-tɯ-ci} `its water (of irrigated water, to a plant)'). Simple nouns cannot take indefinite possessive prefixes.


\begin{table}[H] \centering
\caption{Pronouns and possessive prefixes }\label{tab:pronoun}
\begin{tabular}{lllllllll} 
\toprule
 Free pronoun & Prefix & Person\\
\midrule
 \ipa{aʑo},    \ipa{ɤj} &	\ipa{a--}  &		1\textsc{sg} \\
\ipa{nɤʑo},  \ipa{nɤj} &	\ipa{nɤ--}  &			2\textsc{sg}\\
\ipa{ɯʑo}  &	\ipa{ɯ--}  &			3\textsc{sg}\\
\midrule
\ipa{tɕiʑo}  &	\ipa{tɕi--}  &			1\textsc{du} \\
\ipa{ndʑiʑo}  &	\ipa{ndʑi--}  &		2\textsc{du} \\	
\ipa{ʑɤni}  &	\ipa{ndʑi--}  &		3\textsc{du} \\	
\midrule
\ipa{iʑo}, \ipa{iʑora},   \ipa{iʑɤra}   &	\ipa{i--}  &			1\textsc{pl} \\
\ipa{nɯʑo}, \ipa{nɯʑora},   \ipa{nɯʑɤra}  &	\ipa{nɯ--}  &			2\textsc{pl} \\
\ipa{ʑara}  &	\ipa{nɯ--}  &			3\textsc{pl} \\
\midrule
&  \ipa{tɯ--},  \ipa{tɤ--} & indefinite \\
\ipa{tɯʑo} & \ipa{tɯ--}   &  generic\\
\bottomrule
\end{tabular}
\end{table}

The indefinite possessive prefixes should not be confused with the generic possessive prefix \ipa{tɯ--}, which can be added to any noun, and which is coereferent with the argument marked with generic marking on the verb, as in example \ref{ex:tWrpW}. Note also that inalienably possessed nouns that select the indefinite possessive prefix \ipa{tɤ--} have \ipa{tɯ--} instead when the possessor is generic (\ipa{tɤ-rpɯ} `an uncle' vs \ipa{tɯ-rpɯ} `one's uncle').

\begin{exe}
\ex \label{ex:tWrpW}
\gll
 \ipa{tɯ-rpɯ} 	\ipa{ɯ-rɟit} 	\ipa{ɯ-ɕki} 	\ipa{tɕe} 	\ipa{tɕe} 	``\ipa{a-rpɯ} \ipa{a-ɬaʁ}" 	\ipa{tu-kɯ-ti} 	\ipa{ŋu.} \\
\textsc{genr.poss}-uncle \textsc{3sg.poss}-offspring 3sg-dat lnk \textsc{lnk} \textsc{1sg.poss}-uncle \textsc{1sg.poss}-aunt \textsc{ipfv-genr}-say \textsc{fact}:be \\
\glt One has to say ``my maternal uncle, my maternal aunt" to one's maternal uncle's sons and daughters. (Kinship, 69)
\end{exe}

Some possessed nouns have restrictions on the interpretation of the possessor. Thus, the possessive prefix of the noun \ipa{tɤ-pɤro} `present' can only refer to the person giving the present, never the recipient: \ipa{a-pɤro} \textsc{1sg.poss}-\textit{present} can only mean `my present (to him, to you etc)' not `the present (you, he gave) me'.


\ipa{kɤɣɯrtɯrtaʁ}

\section{The noun phrase} 
demonstratives, quantifiers, postpositions


tɕheme χsɯm nɯ wuma pjɤmpɕɤrnɯ
\section{Simple clauses} 

\section{Relativization}

\section{Complementation}

\section{Clause linking}



\citet{jacques08}

\citet{jacques04these}
\citet{jacques14linking}

\bibliographystyle{linquiry2}
\bibliography{bibliogj}
\end{document}